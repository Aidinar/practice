\documentclass[10pt]{book}
\usepackage[utf8]{inputenc}

\usepackage{latexsym,amssymb,amsfonts,amsmath,amsxtra,indentfirst,shapepar,%fleqn,%
picinpar,shadow,floatflt,enumerate,multicol,colortbl,moreverb,cite,ipi}

\usepackage{rotating}
\usepackage{mathrsfs}
\usepackage[noend]{algorithmic}
\usepackage{ulem}
\usepackage{graphicx}
%\usepackage{algorithm2e}
\usepackage[linesnumbered,boxed,ruled]{algorithm2e}
%\usepackage{xypic}
\usepackage{oldgerm}
\usepackage{epic}
\usepackage{eepic}


\SetAlgorithmName{Algorithm}{алгоритм}{Список алгоритмов}

%из Дюковой

\newcommand{\algKeyword}[1]{{\bf #1}}
\newcommand{\Proc}[1]{\text{\tt #1}}
\def\CALL{\algKeyword{call}~}

\newenvironment{AlgProcedure}[1]
{
    \small
    \medskip
    %    \hrule
    \medskip
    \algKeyword{PROCEDURE} #1
    \begin{algorithmic}[1]}
    {\end{algorithmic}
    %    \hrule
    \bigskip
}

\def\CALL{\algKeyword{call}~}

%конец для Дюковой

%\RequirePackage[ruled]{algorithm}


\input{epsf}

%\nofiles

%\includeonly{avtor} %+pdf+
%\includeonly{obchak,avtor}
%\includeonly{pred}                 %+
%\includeonly{podgot-rus-site,podgot-eng-site}  
%\includeonly{ocherk} 
%\includeonly{nekrol} 
%\includeonly{ipi-ind} 
%\includeonly{index12}
%\includeonly{toc-rus, toc-en}
%\includeonly{toc-rus}
%\includeonly{toc-en} 


%\includeonly{sam+gaidam}                              %+ %1pdf
%\includeonly{razumchik}                               %2без рис+pdf
%\includeonly{gorsh+korolev}                           %3+pdf
%\includeonly{agalarov}                                %4+pdf
%\includeonly{ushakov}                                 %5+pdf
%\includeonly{grusho}                                  %6без рис +pdf 
%\includeonly{basok}                                   %7без рис+pdf
%\includeonly{frenkel}                                 %8+алгоритмы???? +pdf
%\includeonly{strijov}                                 %9+pdf
%\includeonly{logachev}                                %10+pdf
%\includeonly{kor+gor+zei}                             %11+без рисpdf
%\includeonly{korolev+dor}                             %12+без рисpdf
%\includeonly{kudr}                                    %13без рис+pdf
%\includeonly{gonch+zatsman}                           %14без рис+pdf
%\includeonly{gorshenin}                               %15+pdf

%\includeonly{nekrol}             %+


%\includeonly{obchak}
%\includeonly{rekl}
%\includeonly{rekl-1}
%\includeonly{reshal}  %
%\includeonly{cover3}

\usepackage{acad}
%\usepackage{courier}
\usepackage{decor}
\usepackage{newton}
\usepackage{pragmatica}
\usepackage{zapfchan}
\usepackage{petrotex}
\usepackage{bm}                     % полужирные греческие буквы
\usepackage{upgreek}                % прямые греческие буквы
\usepackage{eufrak}
\usepackage{verbatim}

\renewcommand{\bottomfraction}{0.99}
\renewcommand{\topfraction}{0.99}
\renewcommand{\textfraction}{0.01}

\setcounter{secnumdepth}{1} %здесь - 3 + chapter = 4

\arraycolsep=1.5pt

%\usepackage[pdftex]{graphicx}

%\usepackage{oz}

%NEW COMMANDS


\renewcommand*{\hm}[1]{#1\nobreak\discretionary{}%
            {\hbox{$\mathsurround=0pt #1$}}{}} %% Дублирует знаки операций
                               %при переносе в формуле (перед знаком, который
                               %надо продублировать ставится команда \hm)

%\newcommand{\endproof}{\hfill$\Box$}
\renewcommand{\r}{\mathbb{R}}
%\newcommand{\I}{{\rm I\hspace{-0.7mm}I}}
%\newcommand{\Ikl}{{\tt{1}}\hspace*{-1.44mm}\mathtt{1}}
\newcommand{\Ik}{\mbox{{\small \tt {1}}\hspace{-1.3mm}{\tt 1}}}
\newcommand{\argmin}{\mathop{\mathrm{arg}\,\mathrm{min}}}
\newcommand{\argmax}{\mathop{\mathrm{arg}\,\mathrm{max}}}
%\newcommand{\capr}{\mathop{\cap\,}}
%\newcommand{\cupr}{\mathop{\cup\,}}
%\def\argmin{\mathop{arg\,min}}

\def\vrp{\varphi}
\def\prt{\partial}
\def\mm{{\sf M}}
\def\modnop#1{\mathop{#1}\limits_{n}}
\def\eam{\mathbin{{\mathop{=}\limits^{\mathrm{def}}}}}
\def\dey#1#2{#1 (#2)}
\def\deyc#1#2{#1 \cdot  #2}
\def\ra#1{\;\mathop{\to}\limits^{#1}\;}
\def\raz#1{\;\mathop{\longrightarrow}\limits^{\!\!\!#1}\;}
\def\ral#1{\;\mathop{\longrightarrow}\limits^{#1}\;}

\newcommand{\Nor}{\mathcal{N}}
\newcommand{\T}{\mathbb{T}}
\newcommand{\Z}{\mathbb{Z}}



\newcommand{\il}[2]{\int\limits_{#1}^{#2}}%интеграл с пределами #1 и #2

\def\sm2{\mathop {\sum\limits^{n^\Theta}\sum\limits^{n^\Theta}}}
\def\sss{\sum\limits}
\def\tr{,\,\ldots\,,\,}
\def\rk{\right]}
\def\lk{\left[}
\def\rf{\right\}}
\def\lf{\left\{}
\def\lv{\,\left\vert}
\def\rv{\right\vert\,}
\def\iii{\int\limits}
\def\iin{\int\limits_{-\infty}^\infty}
\def\rrv{\right\vert}


\def\ee{{\cal E}}
\def\ww{{\cal W}}
\def\yy{{\cal Y}}
\def\vv{{\cal V}}

\newcommand{\R}{\mathbb R}
\newcommand{\E}{\mathbb E}
\newcommand{\N}{\mathbb N}

\renewcommand{\P}{\mathbb{P}}

\newcommand{\h}{{\bf H}}
\newcommand{\p}{{\sf P}}  % вероятность

\newcommand{\e}{{\sf E}}  % мат. ожидание
\newcommand{\D}{{\sf D}}  % дисперсия
\newcommand{\eps}{\varepsilon}
\newcommand{\vp}{{\mathbf p}}
\newcommand{\vz}{{\mathbf z}}
\newcommand{\vx}{{\mathbf x}}
\newcommand{\vf}{{\mathbf f}}
\newcommand{\F}{{\mathcal F}}
\def\ap{{\mathrm{ЭР}}}
\newcommand{\ud}{\Delta_n} %uniform ditance
\newcommand{\nud}{\Delta_n(x)}
%\renewcommand{\Re}{\mathrm{Re}\,}

\newcommand{\abs}[1]{\left\vert#1\right\vert}

\newcommand{\norm}[1]{\left\Vert#1\right\Vert}
\def\da{(\Delta_t,A)}

\newcommand{\corr}{\mathrm{corr}}

\newcommand{\cov}{\mathrm{cov}}
\newcommand{\Expect}{\mathbb{E}}

\def\w{\omega}
\def\W{\Omega}

\def\inh{\int\limits_{nh}^{(n+1)h}}

\def\sumin{\sum_{i=1}^N}


\def\bxt{(Y,t)}
\def\xt{(y,t)}

\def\ovth{{\fr{\tau-nh}{h}}}
\def\ov{\overline}
\def\tm{\tilde m}
\def\tl{\tilde\lambda}
\def\tB{\widetilde B}
\def\tb{\tilde b}
\def\ld{\ldots}
\def\cd{\cdots}


\DeclareMathOperator{\sign}{sign}

%\newcommand{\gr}{{\geqslant}}


\newcommand{\g}{\mbox{\textit{g}}}

\renewcommand{\la}{\lambda}
\newcommand{\si}{\sigma}
\newcommand{\alp}{\alpha}

\newcommand{\pto}{\stackrel{P}{\longrightarrow}} % сходимость по веpоятности

\newcommand{\eqd}{\stackrel{\mathrm{d}}{=}} % равенство по pаспpеделению
\newcommand{\eqdelta}{\stackrel{\triangle}{=}} % равенство по pаспpеделению

\def\be#1{\begin{equation}\label{#1}}
\def\ee{\end{equation}}
\def\re#1{(\ref{#1})}

\def\bn{\begin{enumerate}}
\def\en{\end{enumerate}}
\def\bi{\begin{itemize}}
\def\ei{\end{itemize}}
%\def\i{\item}

%\newcommand{\kp}{\kappa}
%\def\Q{{\cal Q}} \def\H{{\cal H}}
%\newcommand{\bet}{\beta_{2+\delta}}


%\newtheorem{definition}{Определение}
%\renewcommand{\thedefinition}{\arabic{definition}.}
%END NEW COMMANDS

%\renewcommand{\baselinestretch}{1.2}

%\pagestyle{myheadings}

\setlength{\textwidth}{167mm}      % 122mm
\setlength{\textheight}{658pt}
%\setlength{\textheight}{635.6pt}
\setlength{\columnsep}{4.5mm}

\setcounter{secnumdepth}{4}

%\addtolength{\headheight}{2pt}
%\addtolength{\headsep}{-2mm}

\addtolength{\topmargin}{-7mm}  % for printing


%\hoffset=-30mm  % From Yap
\hoffset=-23mm  % From Acrobat

%\voffset=0mm % From Yap
\voffset=-5mm   % From Acrobat

%\addtolength{\evensidemargin}{-2.5mm} % for printing
%\addtolength{\oddsidemargin}{2.5mm}  % for printing

\addtolength{\evensidemargin}{-12mm} % for printing
\addtolength{\oddsidemargin}{8mm}  % for printing

%\renewcommand{\thefootnote}{\fnsymbol{footnote}}
%\renewcommand{\thefootnote}{\arabic{footnote}}
\renewcommand{\figurename}{\protect\bf Рис.}
\renewcommand{\tablename}{\protect\bf Таблица}

\newcommand{\Caption}[1]{\caption{\protect\small %\baselineskip=2.5ex
#1}}

\renewcommand{\thefigure}{\arabic{figure}}
\renewcommand{\thetable}{\arabic{table}}
\renewcommand{\theequation}{\arabic{equation}}
\renewcommand{\thesection}{\arabic{section}}

\renewcommand{\contentsname}{СОДЕРЖАНИЕ}
\newcommand{\fr}[2]{\displaystyle\frac{\displaystyle #1\mathstrut}{\displaystyle #2\mathstrut}}

%\renewcommand{\thefootnote}{\fnsymbol{footnote}}
%\newcommand{\g}{\mbox{\textit{g}}}

%\newcommand{\Caption}[1]{\caption{\protect\small\baselineskip=2ex #1}}
\newcounter{razdel}
\setcounter{razdel}{0}


\newcommand{\titel}[4]{%
\

\vspace*{5pt}

\ifodd\therazdel {\raggedright\noindent\Large\textrm\textbf
 \lineskip .75em
  \baselineskip=3.2ex #1 \par}
\vskip 1em {\noindent\large\textrm\textbf #2 \par}
\addcontentsline{toc}{subsection}{{\textrm\textbf #1}\protect\newline #2}
\def\rightheadline{\underline{\noindent\hbox to \textwidth{\hfill\small\textrm{#4}
%\hfill \large\bf\thepage
}}}
\def\leftheadline{\underline{\noindent\parbox{\textwidth}{
%\raggedleft\large\bf\thepage \hfill
\small\textit{#3}\hfill}}}
\def\leftfootline{\small{\textbf{\thepage}
\hfill ИНФОРМАТИКА И ЕЁ ПРИМЕНЕНИЯ\ \ \ том~12\ \ \ выпуск 4\ \ \ 2018}
}%
 \def\rightfootline{\small{ИНФОРМАТИКА И ЕЁ ПРИМЕНЕНИЯ\ \ \ том~12\ \ \ выпуск~4\ \ \ 2018
\hfill \textbf{\thepage}}}
\vskip 2em \setcounter{figure}{0}
\setcounter{table}{0}
\setcounter{equation}{0}
\setcounter{section}{0}
\setcounter{subsection}{0}
\setcounter{subsubsection}{0}
\setcounter{footnote}{0}
\setcounter{razdel}{0}
%\end{flushleft}
\else {
 \raggedright\noindent\Large\textrm\textbf
 \lineskip .75em
\baselineskip=3.2ex #1 \par} \vskip 1em
%\begin{flushleft}
{\noindent\large\textrm\textbf #2 \par}
\addcontentsline{toc}{subsection}{{\textrm\textbf #1}\protect\newline #2}
\def\rightheadline{\underline{\noindent\hbox to \textwidth{\hfill\small\textrm{#4}
%\hfill \large\bf\thepage
}}}
\def\leftheadline{\underline{\noindent\parbox{\textwidth}{%\raggedleft\large\bf\thepage \hfill
\small\textit{#3}\hfill}}}
\def\leftfootline{\small{\textbf{\thepage}
\hfill ИНФОРМАТИКА И ЕЁ ПРИМЕНЕНИЯ\ \ \ том~12\ \ \ выпуск~4\ \ \ 2018}
}%
 \def\rightfootline{\small{ИНФОРМАТИКА И ЕЁ ПРИМЕНЕНИЯ\ \ \ том~12\ \ \ выпуск~4\ \ \ 2018
\hfill \textbf{\thepage}}} \vskip 2em \setcounter{figure}{0}
\setcounter{table}{0} \setcounter{equation}{0} \setcounter{section}{0}
\setcounter{subsection}{0} \setcounter{subsubsection}{0}
\setcounter{footnote}{0}
%\end{flushleft}
\fi}

\newcommand{\titelr}[2]{%
\

\vspace*{5pt}

\ifodd\therazdel {\raggedright\noindent%\Large\textrm\textbf
 \lineskip .75em
  \baselineskip=3.2ex #1 \par}
\vskip 1em {\noindent\normalsize\textrm\textbf #2 \par}
\else {
 \raggedright\noindent\Large\textrm\textbf
 \lineskip .75em
\baselineskip=3.2ex #1 \par} \vskip 1em
%\begin{flushleft}
{\noindent\large\textrm\textbf #2 \par
%\noindent\normalsize\textrm\textbf #2 \par
} \fi}

\newcommand{\titele}[5]{%
\

%\vspace*{5pt}

\ifodd\therazdel {\raggedright\noindent\large
\textrm\textbf
 \lineskip .75em
%  \baselineskip=3.2ex
#1 \par}
\vskip .5em {\noindent\large\textrm\textbf #2 \par}
\vskip .5em
 {\noindent\textrm #3 \par}
\addcontentsline{toc}{subsection}{{\textrm\textbf #1}\protect\newline #2}
\def\rightheadline{\underline{\noindent\hbox to \textwidth{\hfill\small\textrm{#4}
%\hfill \large\bf\thepage
}}}
\def\leftheadline{\underline{\noindent\parbox{\textwidth}{
%\raggedleft\large\bf\thepage \hfill
\small\textrm{#5}\hfill}}}
\def\leftfootline{\small{\textbf{\thepage}
\hfill ИНФОРМАТИКА И ЕЁ ПРИМЕНЕНИЯ\ \ \ том~12\ \ \ выпуск~4\ \ \ 2018}
}%
 \def\rightfootline{\small{ИНФОРМАТИКА И ЕЁ ПРИМЕНЕНИЯ\ \ \ том~12\ \ \ выпуск~4\ \ \ 2018
\hfill \textbf{\thepage}}} \vskip 1em \setcounter{figure}{0}
\setcounter{table}{0} \setcounter{equation}{0} \setcounter{section}{0}
\setcounter{subsection}{0} \setcounter{subsubsection}{0}
\setcounter{footnote}{0} \setcounter{razdel}{0}
%\end{flushleft}
\else {
 \raggedright\noindent\large
 \textrm\textbf
 \lineskip .75em
%\baselineskip=3.2ex
#1 \par} \vskip .5em
%\begin{flushleft}
{\noindent\large\textrm\textbf #2 \par} \vskip .5em
 {\noindent\textrm #3 \par}
\addcontentsline{toc}{subsection}{{\textrm\textbf #1}\protect\newline #2}
\def\rightheadline{\underline{\noindent\hbox to \textwidth{\hfill\small\textrm{#4}
%\hfill \large\bf\thepage
}}}
\def\leftheadline{\underline{\noindent\parbox{\textwidth}{%\raggedleft\large\bf\thepage \hfill
\small\textrm{#5}\hfill}}}
\def\leftfootline{\small{\textbf{\thepage}
\hfill ИНФОРМАТИКА И ЕЁ ПРИМЕНЕНИЯ\ \ \ том~12\ \ \ выпуск~4\ \ \ 2018}
}%
 \def\rightfootline{\small{ИНФОРМАТИКА И ЕЁ ПРИМЕНЕНИЯ\ \ \ том~12\ \ \ выпуск~4\ \ \ 2018
\hfill \textbf{\thepage}}} \vskip 1em \setcounter{figure}{0}
\setcounter{table}{0} \setcounter{equation}{0} \setcounter{section}{0}
\setcounter{subsection}{0} \setcounter{subsubsection}{0}
\setcounter{footnote}{0}
%\end{flushleft}
\fi}

\def\Abst#1{
\begin{center}\small\nwt
\parbox{150mm}{%\baselineskip=2.5ex
\textbf{Аннотация:}\ \
%\hspace*{\parindent}
#1}
\end{center}}
\def\Abste#1{
\begin{center}\small\nwt
\parbox{150mm}{%\baselineskip=2.5ex
\textbf{Abstract:}\ \
%\hspace*{\parindent}
#1}
\end{center}}

\def\DOI#1{
\begin{center}\small\nwt
\parbox{150mm}{%\baselineskip=2.5ex
\textbf{DOI:}\ \
%\hspace*{\parindent}
#1}
\end{center}}

\def\Abstend#1{
\begin{center}\small\nwt
\parbox{150mm}{%\baselineskip=2.5ex
%\hspace*{\parindent}
#1}
\end{center}}


\def\KW#1{
\begin{center}\small\nwt
\parbox{150mm}{%\baselineskip=2.5ex
\textbf{Ключевые слова:}\ \ #1}
\end{center}}

\def\KWE#1{
\begin{center}\small\nwt
\parbox{150mm}{%\baselineskip=2.5ex
\textbf{Keywords:}\ \ #1}
\end{center}}


\def\KWN#1{
%\begin{center}
%\small
%\parbox{150mm}\end{center}
}

\newcommand{\Avtors}[1]{%\smallskip
%\vspace*{.5pt}
\hangindent=23pt\noindent
%\nwt
{\bfseries#1}\
}


\renewcommand{\thesubsection}{\thesection.\arabic{subsection}\hspace*{-5pt}}
\renewcommand{\thesubsubsection}{\thesubsection\hspace*{5pt}.\arabic{subsubsection}\hspace*{-3pt}}

\newcommand{\Ack}{\section*{\protect\rmfamily Acknowledgments}\noindent}
\newcommand{\Contr}{\section*{\protect\rmfamily Contributors}\noindent}
\newcommand{\Contrl}{\section*{\protect\rmfamily Contributor}\noindent}

\makeindex


\begin{document}
\Rus

\nwt
%\ptb


%\renewcommand{\contentsname}{\protect\Large\bf Содержание}

\setcounter{tocdepth}{2}

%\tableofcontents

\renewcommand{\bibname}{\protect\rmfamily Литература}
  \def\Au#1{{\it #1}}
    \def\Aue#1{{#1}}

%\newcommand{\No}{№}
  \newcommand{\tg}{\,\mathrm{tg}\,}
    \newcommand{\ctg}{\,\mathrm{ctg}\,}
  \newcommand{\arctg}{\,\mathrm{arctg}\,}

\def\forallb{\mathop{\forall}}
\def\cupb{\mathop{\cup}}
\def\existsb{\mathop{\exists}}


\newpage
\addtocounter{razdel}{1}
%\def\razd{РЕГУЛИРУЕМЫЙ ЭЛЕКТРОПРИВОД ДЛЯ ЭЛЕКТРОЭНЕРГЕТИКИ}


\setcounter{page}{2}

%   { %\Large  
   { %\baselineskip=16.6pt
   
   \vspace*{-48pt}
   \begin{center}\LARGE
   \textit{Предисловие}
   \end{center}
   
   %\vspace*{2.5mm}
   
   \vspace*{25mm}
   
   \thispagestyle{empty}
   
   { %\small 

    
Вниманию читателей журнала <<Информатика и её применения>> предлагается 
очередной тематический выпуск <<Вероятностно-статистические методы и 
задачи информатики и информационных технологий>>. Предыдущие тематические 
выпуски журнала по данному направлению вышли в 2008~г.\ (т.~2, вып.~2), 
в 2009~г.\ (т.~3, вып.~3) и в 2010~г.\ (т.~4, вып.~2). 

Статьи, собранные в данном журнале, посвящены разработке новых вероятностно-статистических 
методов, ориентированных на применение к решению конкретных задач информатики и информационных 
технологий, а также~--- в ряде случаев~--- и других прикладных задач. Проблематика, охватываемая 
публикуемыми работами, развивается в рамках научного сотрудничества между Институтом проблем 
информатики Российской академии наук (ИПИ РАН) и Факультетом вычислительной математики и 
кибернетики Московского государственного университета им.\ М.\,В.~Ломоносова в ходе работ 
над совместными научными проектами (в том числе в рамках функционирования 
Научно-образовательного центра <<Вероятностно-статистические методы анализа рисков>>). 
Многие из авторов статей, включенных в данный номер журнала, являются активными участниками 
традиционного международного семинара по проблемам устойчивости стохастических моделей, 
руководимого В.\,М.~Золотаревым и В.\,Ю.~Королевым; регулярные сессии этого семинара 
проводятся под эгидой МГУ и ИПИ РАН (в 2011~г.\ указанный семинар проводится в октябре 
в Калининградской области РФ). 

Наряду с представителями ИПИ РАН и МГУ в число авторов данного выпуска журнала входят 
ученые из Научно-исследовательского института системных исследований РАН, Института 
проблем технологии микроэлектроники и особочистых материалов РАН, Института 
прикладных математических исследований Карельского НЦ РАН, Московского 
авиационного института, Вологодского государственного педагогического университета, 
НИИММ им.\ Н.\,Г.~Чеботарева, Казанского государственного университета, Дебреценского 
университета (Венгрия).

Несколько статей выпуска посвящено разработке и применению стохастических методов и 
информационных технологий для решения различных прикладных задач. В~работе В.\,Г.~Ушакова 
и О.\,В.~Шестакова рассмотрена задача определения вероятностных характеристик случайных 
функций по распределениям интегральных преобразований, возникающих в задачах эмиссионной 
томографии. В~статье Д.\,О.~Яковенко и М.\,А.~Целищева рассмотрены некоторые вопросы 
математической теории риска и предложен новый подход к диверсификации инвестиционных 
портфелей. Работа И.\,А.~Кудрявцевой и А.\,В.~Пантелеева посвящена построению и 
исследованию математической модели, описывающей динамику сильноионизованной плазмы. 
В~статье П.\,П.~Кольцова изучается качество работы ряда алгоритмов сегментации изображений. 
Статья А.\,Н.~Чупрунова и И.~Фазекаша посвящена вероятностному анализу числа без\-оши\-бочных 
блоков при помехоустойчивом кодировании; получены усиленные законы больших чисел для указанных 
величин.

В данном выпуске традиционно присутствует тематика, весьма активно разрабатываемая в течение 
многих лет специалистами ИПИ РАН и МГУ,~--- методы моделирования и управления для 
информационно-телекоммуникационных и вычислительных систем, в частности методы 
теории массового обслуживания. В~статье А.\,И.~Зейфмана с соавторами рассматриваются 
модели обслуживания, описываемые марковскими цепями с непрерывным временем в случае 
наличия катастроф. В~работе М.\,М.~Лери и И.\,А.~Чеплюковой рассматриваются случайные 
графы Интернет-типа, т.\,е.\ графы, степени вершин которых имеют степенные распределения; 
такие задачи находят применение при исследовании глобальных сетей передачи данных. 
Работа Р.\,В.~Разумчика посвящена исследованию систем массового обслуживания специального 
вида~--- с отрицательными заявками и хранением вытесненных заявок.

Ряд статей посвящен развитию перспективных теоретических 
вероятностно-статистических методов, которые находят широкое применение в различных 
задачах информатики и информационных технологий. В~работе В.\,Е.~Бенинга, А.\,К.~Горшенина 
и В.\,Ю.~Королева рассмотрена задача статистической проверки гипотез о числе компонент 
смеси вероятностных распределений, приводится конструкция асимптотически наиболее мощного 
критерия. Результаты этой работы найдут применение в ряде прикладных задач, использующих 
математическую модель смеси вероятностных распределений (в информатике, моделировании 
финансовых рынков, физике турбулентной плазмы и~т.\,д.). В~статье В.\,Ю.~Королева, 
И.\,Г.~Шевцовой и С.\,Я.~Шоргина строится новая, улучшенная оценка точности нормальной 
аппроксимации для пуассоновских случайных сумм; как известно, указанные случайные суммы 
широко используются в качестве моделей многих реальных объектов, в том числе в информатике, 
физике и других прикладных областях. Работа В.\,Г.~Ушакова и Н.\,Г.~Ушакова посвящена 
исследованию ядерной оценки плотности распределения; эти результаты могут применяться, 
в част\-ности, при анализе трафика в телекоммуникационных системах. Серьезные приложения 
в статистике могут получить результаты работы О.\,В.~Шестакова, в которой доказаны оценки 
скорости сходимости распределения выборочного абсолютного медианного отклонения к нормальному 
закону. 

\smallskip

Редакционная коллегия журнала выражает надежду, что данный тематический  выпуск 
будет интересен специалистам в области теории вероятностей и математической статистики 
и их применения к решению задач информатики и информационных технологий.
     
     %\vfill 
     \vspace*{20mm}
     \noindent
     Заместитель главного редактора журнала <<Информатика и её 
применения>>,\\
     директор ИПИ РАН, академик  \hfill
     \textit{И.\,А.~Соколов}\\
     
     \noindent
     Редактор-составитель тематического выпуска,\\
     профессор кафедры математической статистики факультета\\
      вычислительной математики и кибернетики МГУ им.\ М.\,В.~Ломоносова,\\
     ведущий научный сотрудник ИПИ РАН,\\ 
доктор физико-математических наук \hfill
      \textit{В.\,Ю.~Королев}
     
     } }
     }

\def\stat{gaid+sam}

\def\tit{ПРИМЕНЕНИЕ МОДЕЛЕЙ СЛУЧАЙНОГО БЛУЖДАНИЯ ПРИ~МОДЕЛИРОВАНИИ 
ПЕРЕМЕЩЕНИЯ УСТРОЙСТВ В~БЕСПРОВОДНОЙ СЕТИ$^*$}

\def\titkol{Применение моделей случайного блуждания при моделировании 
перемещения устройств в~беспроводной сети}

\def\aut{К.\,Е.~Самуйлов$^1$, Ю.\,В.~Гайдамака$^2$, С.\,Я.~Шоргин$^3$}

\def\autkol{К.\,Е.~Самуйлов, Ю.\,В.~Гайдамака, С.\,Я.~Шоргин}

\titel{\tit}{\aut}{\autkol}{\titkol}

\index{Самуйлов К.\,Е.}
\index{Гайдамака Ю.\,В.}
\index{Шоргин С.\,Я.}
\index{Samouylov K.\,E.}
\index{Gaidamaka Yu.\,V.} 
\index{Shorgin S.\,Ya.}




{\renewcommand{\thefootnote}{\fnsymbol{footnote}} \footnotetext[1]
{Исследование выполнено при финансовой поддержке Российского научного фонда (проект 16-11-10227).}}


\renewcommand{\thefootnote}{\arabic{footnote}}
\footnotetext[1]{Российский университет дружбы народов; Федеральный исследовательский центр <<Информатика 
и~управление>> Российской академии наук, 
\mbox{samouylov\_ke@rudn.university}}
\footnotetext[2]{Российский университет дружбы народов; Федеральный исследовательский центр <<Информатика 
и~управление>> Российской академии наук, \mbox{gaydamaka\_yuv@rudn.university}}
\footnotetext[3]{Институт проблем информатики Федерального исследовательского центра <<Информатика 
и~управление>> Российской академии наук, \mbox{ssorgin@ipiran.ru}}

%\vspace*{8pt}

   
  
  \Abst{Выполнен обзор моделей случайного блуж\-да\-ния объектов, применяемых при 
моделировании передвижения при\-емо\-пе\-ре\-да\-ющих устройств пользователей 
беспроводной сети пятого поколения (5G). Рассмотрены модели мо\-биль\-ности, характерные для 
имитационного моделирования поведения пользователей беспроводной 
самоорганизующейся сети. Обсуждаются особенности различных моделей 
индивидуального движения объектов, а~так\-же моделей движения групп объектов с~точки 
зрения применения к~анализу интерференции в~беспроводных сетях. Цель статьи~--- 
предложить ряд моделей мо\-биль\-ности для принятия обоснованного решения при выборе 
модели случайного блуж\-да\-ния для анализа качества предо\-став\-ле\-ния услуг 
в~беспроводных сетях. В~качестве иллюстрации применения разработанного авторами 
комплекса аналитических и~имитационных моделей проведен анализ отношения 
сигнал/\-ин\-тер\-фе\-рен\-ция, определяющего качество пред\-остав\-ле\-ния 
услуг в~сетях пятого 
поколения, для сценария случайного блуж\-да\-ния мобильных абонентов 
в~тор\-го\-во-раз\-вле\-ка\-тель\-ном цент\-ре при использовании 
модели случайного блуж\-да\-ния с~остановками Random Waypoint.}
  
  \KW{модель случайного блуждания; модель мобильности; отношение  
сиг\-нал/ин\-тер\-фе\-рен\-ция; отношение сиг\-нал/шум}

\DOI{10.14357/19922264180401}
  
\vspace*{5pt}


\vskip 10pt plus 9pt minus 6pt

\thispagestyle{headings}

\begin{multicols}{2}

\label{st\stat}

\section{Введение}

  В беспроводных сетях 5G интерференция служит одним из существенных 
источников помех, вли\-я\-ющим на показатели качества функционирования 
сети, к~которым относятся пиковые ско\-рости передачи данных между 
устройствами, задержка\linebreak начала передачи данных, отношение 
сигнал/\-ин\-тер\-фе\-рен\-ция (signal to interference ratio, SIR), энер\-го\-сбе\-ре\-же\-ние, 
эффективность использования частот\-но\-го спектра и~др.~[1]. При анализе 
интерференции\linebreak следует учитывать особенности современных беспроводных 
сетей, которые при использовании в~них технологии прямого взаимодействия 
оконечных устройств (device-to-device, D2D) образуют самоорганизующиеся 
сети (mobile ad hoc network, MANET) с~перемещающимися в~зоне покрытия 
узлами.\linebreak
 Относительно небольшие расстояния между подвижными узлами, 
соответствующими при\-емо\-пе\-ре\-да\-ющим устройствам, делают необходимым 
при анализе интерференции между соседними источниками сигнала в~таких 
сетях учитывать траектории перемещения узлов, которые фактически 
определяют динамику показателя SIR в~радиоканале между приемником 
и~передатчиком. 

В~работах~[2, 3] перемещение беспроводных устройств 
моделировалось с~по\-мощью ки\-не\-ти\-ческого уравнения  
Фок\-ке\-ра--Планка~[4], регулирование па\-ра\-мет\-ров (снос, диффузия) 
которого позволяет исследовать различные типы движения большого чис\-ла 
объектов, не строя индивидуальную траекторию перемещения каждого 
объекта. 

Однако такой подход не применим при анализе интерференции 
в~задачах, где необходимо принимать во внимание особенности 
предоставления услуг передачи данных в~сети, например учитывать 
препятствия в~зоне перемещения устройств, наличие нескольких сред 
распространения сигнала и~другие ограничения. Для решения таких задач 
необходимо детальное моделирование траектории движения каждого 
беспроводного устройства с~по\-мощью аналитических~[5--7] 
и~имитационных~[8, 9] моделей. 




В~статье проведен обзор моделей 
мо\-биль\-ности объектов, применяемых при моделировании перемещения 
устройств в~беспроводных самоорганизующихся сетях. В~разд.~2 
об\-суж\-да\-ют\-ся особенности различных моделей индивидуального движения 
объектов, а~также моделей движения\linebreak
 групп объектов с~точки зрения 
применения к~анализу интерференции в~беспроводных сетях. В~качест-\linebreak ве 
иллюстрации в~разд.~3 на примере прикладной\linebreak задачи анализа движения 
мобильных абонентов в~тор\-го\-во-раз\-вле\-ка\-тель\-ном центре с~по\-мощью 
разработанного авторами комплекса аналитических и~имитационных 
моделей на наборе исходных данных, близ\-ких к~реальным, проведен расчет 
показателя SIR, определяющего качество пред\-остав\-ле\-ния услуг в~сетях 
пятого поколения.

\vspace*{-8pt}

\section{Модели случайного блуждания}

\vspace*{-2pt}

  В настоящее время для имитационного моделирования передвижения 
беспроводных устройств в~сетях MANET традиционно используются как 
модели мо\-биль\-ности объектов, основанные на сборе и~анализе статистики 
движения абонентов в~реальных беспроводных сетях (traces), так 
и~синтетические (synthetic) модели~[8--10]. Первые построены на основе 
обработки данных от большого чис\-ла узлов сети, со\-бран\-ных в~течение 
длительного периода наблюдения, поэтому обеспечивают достоверное 
моделирование, однако их применение возможно лишь для анализа уже 
су\-ще\-ст\-ву\-ющих сетей. Поскольку сети~5G в~полной мере еще не 
реализованы, востребованными оказались синтетические модели, с~помощью 
которых мож\-но реалистично воспроизводить поведение абонентов 
беспроводной сети, регулируя правила изменения ско\-рости и~на\-прав\-ле\-ния 
движения мобильных узлов. Например, мобильные узлы не долж\-ны иметь 
прямую траекторию движения и~постоянную ско\-рость в~течение всего 
времени моделирования, потому что в~реальных сетях движение абонентов 
имеет более слож\-ный характер. Как правило, для моделирования движения 
абонентов беспроводной сети\linebreak
 используются как модели мо\-биль\-ности, 
опи\-сы\-ва\-ющие независимое друг от друга движение абонен\-тов беспроводной 
сети, модели так на\-зы\-ва\-емой <<индивидуальной>> мо\-биль\-ности (entity 
mobility models), так и~модели <<групповой>> мо\-биль\-ности (group mobility 
models), в~которых, например, движение группы основано на траектории 
логического цент\-ра~\cite{9-sg, 10-sg, 11-sg}.
  
  Примерами моделей индивидуальной мобильности, использующихся для 
описания перемещения независимых объектов, являются модель RW 
(Random Walk), для которой в~каж\-дой точке на\-прав\-ле\-ние и~ско\-рость 
движения разыгрываются случайным образом (рис.~1,\,\textit{а}), и~ее 
расширение~--- модель RWP (Random WayPoint), в~которой пред\-усмот\-ре\-но 
время остановки в~каж\-дой точке перед продолжением движения 
(рис.~1,\,\textit{б}). 

Модели RW и~RWP чаще других используются для 
моделирования движения мобильных устройств в~самоорганизующейся 
беспроводной сети. Особенностью обеих моделей является отсутствие 
<<памяти>>~--- текущие значения па\-ра\-мет\-ров модели (на\-прав\-ле\-ние, 
ско\-рость, длительность остановки) не зависят от значений этих па\-ра\-мет\-ров 
на прошлом шаге, что приводит к~генерации траекторий с~внезапными 
остановками и~резкими поворотами. При небольших значениях ско\-рости 
в~модели RW движение объектов становится броуновским; следовательно, 
эту модель мож\-но рекомендовать для анализа интерференции в~статической 
сети. 

Интересным развитием модели RWP является ее вероятностная версия, 
в~которой следующая позиция мобильного узла определяется в~соответствии 
с~заданными вероятностями. Недостатком модели RWP является замеченная 
осо\-бен\-ность~--- при достаточно длительном периоде работы имитационной 
модели плот\-ность объектов, в~начале моделирования распределенных 
равномерно, по краям об\-ласти моделирования становится заметно ниже, чем 
в~цент\-ре. Поскольку рас\-сто\-яние до передатчика является клю\-че\-вым 
фактором, оказывающим вли\-яние на мощ\-ность фик\-си\-ру\-емо\-го на приемнике 
сигнала, безосновательное увеличение чис\-ла ближайших интерферирующих 
передатчиков при оценке\linebreak
 показателя SIR исказит вывод о~качестве 
со\-еди\-не\-ния в~мо\-де\-ли\-ру\-емой сети. Однако при моделировании некоторых 
сценариев поведения пользователей, например осмот\-ра музея в~соответствии\linebreak 
с~пред\-ла\-га\-емой схемой знакомства с~экспозицией, модель случайного 
блуж\-да\-ния RWP за счет своей гиб\-кости создает реалистичные траектории 
движения объектов. Кроме того, описанный эффект скопления объектов 
в~цент\-ре об\-ласти моделирования практически исчезает для случая дол\-гих 
остановок даже при высоких значениях па\-ра\-мет\-ра ско\-рости~\cite{10-sg}.


  Указанного для модели RWP недостатка лишены модель RD (Random 
Direction), для которой на\-прав\-ле\-ние и~ско\-рость движения меняются при 
достижении объектом границы об\-ласти моделирования (рис.~1,\,\textit{в}), 
а~так\-же модель движения Гаус\-са--Мар\-ко\-ва (Gauss--Markov, GM), 
которая позволяет получить плав\-ную траекторию движения объекта 
(рис.~1,\,\textit{г}). В~\cite{10-sg} описан метод, который для модели GM 
принудительно меняет на\-прав\-ле\-ние движения объекта при при\-бли\-же\-нии 
к~границе об\-ласти моделирования, что позволяет избежать нежелатель-\linebreak\vspace*{-12pt}

\pagebreak

\end{multicols}

\begin{figure*} %fig1
\vspace*{1pt}
 \begin{center}
 \mbox{%
 \epsfxsize=163.242mm 
 \epsfbox{gai-1.eps}
 }
 \end{center}
\vspace*{-11pt}
\Caption{Примеры траекторий перемещения объекта при случайном блуж\-да\-нии: 
(\textit{а})~модель RW; (\textit{б})~модель RWP; (\textit{в})~модель RD; 
(\textit{г})~модель GM}
\vspace*{-4pt}
\end{figure*}

\begin{multicols}{2} 

\noindent
ных
эффектов <<прилипания>> объекта к~краю об\-ласти. 

Характерная для всех 
упомянутых выше моделей проб\-ле\-ма <<краевого эффекта>> при 
приближении объекта к~границе об\-ласти отсутствует в~модели Boundless 
Simulation Area (BSA), об\-ласть моделирования которой пред\-став\-ля\-ет собою 
тор. В~этой модели текущие значения на\-прав\-ле\-ния и~ско\-рости движения 
зависят от значений этих па\-ра\-мет\-ров на прошлом шаге, что создает 
реалистичную траекторию движения объекта. Однако при моделировании 
беспроводной сети с~по\-мощью модели BSA не избежать искажения 
динамики показателя SIR, поскольку движущееся по по\-верх\-ности тора 
беспроводное устройство регулярно становится соседом ка\-ко\-го-ли\-бо 
неподвижного беспроводного устройства. 

Еще одной моделью 
индивидуальной мо\-биль\-ности является так на\-зы\-ва\-емое <<вероятностное 
блуж\-да\-ние по сетке>> (Probabilistic Grid, PG)~--- модель в~дискретном 
времени, со\-глас\-но которой на каждом временн$\acute{\mbox{о}}$м такте объект делает шаг 
единичной длины, а~выбор одного из четырех на\-прав\-ле\-ний задается 
вероятностной мат\-ри\-цей~\cite{10-sg}. Благодаря простоте реализации эта 
модель так\-же широко применяется при моделировании движения, однако 
задание вероятностной мат\-ри\-цы для конкретного сценария поведения 
пользователей пред\-став\-ля\-ет определенную труд\-ность.
  
  К моделям групповой мо\-биль\-ности~\cite{10-sg, 11-sg} относятся модель 
ECRM (Exponential Correlated Random Mobility), основанная на 
экспоненциальной за\-ви\-си\-мости ско\-рости движения объектов; модель CM 
(Column Mobility), в~которой моделируется движение объектов, вы\-стро\-ен\-ных 
в~линию; модель перемещения кочевников NCM (Nomadic Community 
Mobility); модель преследования (Pursue Mobility),\linebreak
 в~которой группа следует 
за лидером, пе\-ре\-дви\-гающимся по заданной траектории, а~также на\-и\-более 
общая модель групповой мо\-биль\-ности\linebreak
 с~опорной точ\-кой RPGM (Reference 
Point Group Mobility), в~которой пред\-усмот\-ре\-но случайное движение группы 
с~одновременным случайным перемещением каждого отдельного объекта 
внут\-ри группы (рис.~2). Недостатком модели ECRM, которая позволяет 
описать практически все виды групповой мо\-биль\-ности, является 
существенная слож\-ность подбора па\-ра\-мет\-ров модели.

\begin{figure*} %fig2
\vspace*{1pt}
 \begin{center}
 \mbox{%
 \epsfxsize=158.36mm 
 \epsfbox{gai-2.eps}
 }
 \end{center}
\vspace*{-11pt}
\Caption{Пример траекторий для модели групповой мобильности с~опорной точ\-кой 
RPGM для трех объектов: (\textit{а})~перемещение опорной точ\-ки; 
(\textit{б})~траектории перемещения объектов}
\vspace*{-4pt}
\end{figure*}

  Перечисленные модели движения традиционно используются при 
исследовании про\-из\-во\-ди\-тель\-ности различных сетевых протоколов, 
при\-ме\-ня\-емых в~са\-мо\-ор\-га\-низу\-ющих\-ся беспроводных сетях, при этом 
сравнение проводится по таким показателям, как переданные полезная 
и~служебная на\-груз\-ка, джиттер, межконцевая за\-держ\-ка, за\-тра\-ты на 
маршрутизацию~\cite{11-sg}. При выборе модели мо\-биль\-ности с~целью 
исследования интерференции важно учитывать сценарий поведения 
пользователей. Наиболее универсальными моделями индивидуальной 
мо\-биль\-ности являются модель Random Waypoint и~модель  
Гаус\-са--Мар\-ко\-ва, настройка па\-ра\-мет\-ров которых позволяет гиб\-ко 
под\-стро\-ить\-ся под большинство сценариев. Для воспроизведения 
перемещения группы пользователей беспроводной сети рекомендуется 
использовать модель групповой мо\-биль\-ности с~опорной точ\-кой Reference 
Point Group Mobility, которая при со\-от\-вет\-ст\-ву\-ющих значениях па\-ра\-мет\-ров 
позволяет реализовать модели Column, Nomadic Community и~Pursue. 

\section{Пример анализа интерференции при~случайном~блуждании} 
  
  Одной из основных характеристик качества канала в~беспроводных сетях 
связи служит отношение уров\-ня сигнала к~уров\-ню интерференции и~шума 
(ОСШ, \textit{англ}.\ Signal to Interference and Noise Ratio, SINR), которое 
определяется отношением мощ\-ности принимаемого сигнала от 
соответствующего передатчика к~суммарной мощ\-ности шума 
и~при\-ни\-ма\-емо\-го сигнала от интерферирующих передатчиков~\cite{12-sg}. 
При этом мощ\-ности фик\-си\-ру\-емо\-го на приемнике сигнала как от целевого, 
так и~от каждого из интерферирующих передатчиков определяются 
с~соответствии с~классической моделью распространения сигнала, а~именно: 
прямо пропорционально базовой мощ\-ности сигнала передатчика и~обратно 
пропорционально рас\-сто\-янию меж\-ду передатчиком и~приемником 
в~некоторой по\-сто\-ян\-ной степени, зависящей от среды рас\-про\-стра\-не\-ния 
сигнала. Как и~в~\cite{12-sg, 13-sg}, для оценки отношения SIR далее 
используется формула 
$$
\mathrm{SIR}= \fr{r_0^{-\gamma_0}}{\sum\nolimits^N_{n=1} 
d_n^{-\gamma_n}}\,,
$$
 где $r_0$~--- рас\-сто\-яние между приемником 
и~передатчиком в~ис\-сле\-ду\-емой целевой паре; $d_n$~--- рассто\-яние между 
приемником целевой пары и~передатчиком $n$-й ин\-тер\-фе\-ри\-ру\-ющей пары; 
$\gamma$~--- коэффициент распространения сигнала, ха\-рак\-те\-ри\-зу\-ющий 
среду передачи (от~2 в~условиях прямой ви\-ди\-мости до~6 в~худшем случае, 
при котором возможна связь). Расчет проведен в~предположении о~рав\-ных 
из\-лу\-ча\-емых мощностях и~коэффициентах усиления приемной и~пе\-ре\-да\-ющей 
антенн для всех устройств. Для моделирования препятствий для 
про\-хож\-де\-ния сигнала использовались различные значения коэффициентов 
рас\-про\-стра\-не\-ния сигнала~$\gamma_n$, $n\hm= 0,1,\ldots, N$~\cite{13-sg}.
  
  На рис.~3 для одной из моделей индивидуальной мо\-биль\-ности, модели 
случайного блуж\-да\-ния RWP, приведена кривая, по\-ка\-зы\-ва\-ющая изменение 
показателя SIR в~течение~500~с. Для на\-гляд\-ности вы\-бран случай блуж\-да\-ния 
по сетке четырех мобильных \mbox{устройств}~--- целевой пары  
при\-ем\-ник-пе\-ре\-дат\-чик и~двух интерферирующих передатчиков, 
работающих на близ\-ких час\-тотах.

\setcounter{figure}{3}
\begin{figure*}[b] %fig4
\vspace*{1pt}
 \begin{center}
 \mbox{%
 \epsfxsize=162.925mm 
 \epsfbox{gai-4.eps}
 }
 \end{center}
\vspace*{-11pt}
 \Caption{Траектории и~взаимное расположение устройств: 
(\textit{а})~минимальное значение SIR (402-я секунда); (\textit{б})~максимальное 
значение SIR (449-я секунда)}
 \end{figure*}
  
  
     
  Моделировалось целенаправленное движение, когда пользователи, 
носители мобильных\linebreak\vspace*{-12pt}

{ \begin{center}  %fig3
 \vspace*{0.5pt}
  \mbox{%
 \epsfxsize=79mm 
 \epsfbox{gai-3.eps}
 }


\vspace*{6pt}


\noindent
{{\figurename~3}\ \ \small{Динамика SIR}}
\end{center}
}

\vspace*{6pt}

\addtocounter{figure}{1}

\noindent
 устройств, перемещались по кратчайшему пути между 
заранее выбранными точ\-ка\-ми своих маршрутов с~по\-сто\-ян\-ной 
ско\-ростью~1~м/с независимо друг от друга в~квад\-ра\-те $500\times500$~м. 
Такой сценарий характерен, например, для последовательного посещения 
магазинов тор\-го\-во-раз\-вле\-ка\-тель\-но\-го цент\-ра по заранее 
намеченному маршруту. Траектории устройств показаны на рис.~4, где 
перемещение целевого приемника, на котором оценивалось отношение 
сигнал/\-ин\-тер\-фе\-рен\-ция, показано сплош\-ной линией, перемещение 
передатчиков~--- пунктирными линиями, при этом целевому передатчику 
соответствует траектория с~самым длинным размером штриха. 
  
  На рис.~4 точки, отмеченные крестиками на соответствующих 
траекториях, позволяют судить о~взаимном расположении устройств. Для 
на\-гляд\-ности целевые передатчик и~приемник соединены показывающим 
на\-прав\-ле\-ние передачи в~радиоканале вектором, модуль которого равен 
рас\-сто\-янию в~целевой паре. Так, на рис.~4,\,\textit{а}, который соответствует 
402-й секунде моделирования, один из интерферирующих передатчиков 
расположен значительно ближе к~приемнику, чем целевой передатчик, а~на 
рис.~4,\,\textit{б} отражена обратная ситуация, когда на 449-й секунде 
расстояние в~целевой паре становится минимальным. Соответствующие 
локальные экстремумы показателя SIR в~указанные моменты отражены на 
рис.~3.
  
  
  Предложенный метод расчета SIR, основанный на моделировании 
траекторий движения устройств, поз\-во\-ля\-ет оценивать качество 
предо\-став\-ле\-ния услуг в~сети при заданных для каждой услуги требованиях 
к~минимальному допустимому значению этого показателя.

\section{Заключение}

  Проведенный в~работе обзор моделей случайного блуж\-да\-ния, традиционно 
применяемых для моделирования перемещения мобильных узлов 
в~беспроводных са\-мо\-ор\-га\-ни\-зу\-ющих\-ся сетях, поз\-во\-ля\-ет при выборе модели 
для проведения эксперимента учесть особенности каж\-дой модели, 
существенные с~точ\-ки зрения сценария поведения пользователей. 
Универсальной модели мо\-биль\-ности, способной воспроизвести поведение 
пользователя при любом сценарии, не существует, поэтому анализ 
интерференции рекомендуется проводить, применяя несколько моделей 
движения объектов. Также задачей дальнейших исследований может стать 
разработка новой комбинированной модели мо\-биль\-ности для 
воспроизведения перемещения пользователей беспроводной 
самоорганизующейся сети, сочетающей подход модели Gauss--Markov на 
границе об\-ласти моделирования и~принцип перемещения объектов модели 
Random Waypoint внутри об\-ласти, таким образом сохраняя преимущества 
и~компенсируя недостатки каж\-дой из этих моделей.

\bigskip

Авторы выражают благодарность магистрам кафедры прикладной 
информатики и~тео\-рии вероятностей РУДН А.~Жданкову и~О.~Крупко за 
подготовку иллюстраций к~статье по разработанному комплексу 
аналитических и~имитационных моделей.
  
{\small\frenchspacing
 {%\baselineskip=10.8pt
 \addcontentsline{toc}{section}{References}
 \begin{thebibliography}{99}
\bibitem{1-sg}
\Au{Andrews J.\,G., Buzzi~S., Choi~W., Hanly~S.\,V., Lozano~A., Soong~A.\,C., 
Zhang~J.\,C.} What will 5G be?~// IEEE J.~Sel. Area. Comm., 2014. 
Vol.~32. No.\,6. P.~1065--1082. doi: 10.1109/JSAC.2014.2328098.
\bibitem{2-sg}
\Au{Orlov Yu.\,N., Fedorov~S.\,L., Samuylov~A.\,K., Gaidamaka~Yu.\,V., 
Molchanov~D.\,A.} Simulation of devices mobility to estimate wireless channel quality 
metrics in 5G networks~// AIP Conf. Proc., 2017. Vol.~1863.  
P.~090005-1--090005-3. doi: 10.1063/1.4992270.
\bibitem{3-sg}
\Au{Гайдамака Ю.\,В., Орлов Ю.\,Н., Молчанов~Д.\,А., Самуйлов~А.\,К.} 
Моделирование отношения сиг\-нал/ин\-тер\-фе\-рен\-ция в~мобильной сети со 
случайным блужданием взаимодействующих устройств~// Информатика и~её 
применения, 2017. Т.~11. Вып.~2. С.~50--58. doi: 10.14357/19922264170206.
\bibitem{4-sg}
\Au{Risken~H., Frank T.} The Fokker--Planck equation: Methods of solution and 
applications.~--- Springer ser. in synergetics.~--- Berlin--Heidelberg: Springer-Verlag, 1996. 
Vol.~18. 486~p. doi: 10.1007/978-3-642-61544-3.

\bibitem{5-sg}
\Au{Тоффоли Т., Марголус~Н.} Машины клеточных автоматов~/ Пер. с~англ.~--- М.: 
Мир, 1991. 283~с. (\Au{Toffoli~T., Margolus~N.} Cellular automata machines.~--- 
The MIT Press, 1987. 276~p.)
\bibitem{6-sg}
\Au{Grewal M.\,S., Andrews A.\,P.} Kalman filtering: Theory and practice
using MATLAB.~--- 2nd ed.~--- John Wiley \& Sons, Inc., 2001. 410~p.
\bibitem{7-sg}
\Au{Семушин И.\,В., Цыганов~А.\,В., Цыганова~Ю.\,В., Голубков~А.\,В., 
Винокуров~С.\,Д.} Моделирование и~оцени\-ва\-ние траектории движущегося объекта~// 
Вестник\linebreak  ЮУрГУ. Сер. Матем. моделирование и~программирование, 2017. Т.~10. №\,3. 
С.~108--119. doi: 10.14529/ mmp170309.
\bibitem{8-sg}
The VINT Project (Virtual InterNetwork Testbed). The Network Simulator~--- ns-2. {\sf 
http://www.isi.edu/ nsnam/ns}.
\bibitem{9-sg}
Wolfram Demonstrations Project. Constrained Random Walk.
 {\sf http://demonstrations.wolfram.com}.
\bibitem{10-sg}
\Au{Camp T., Boleng J., Davies~V.} A~survey of mobility models for ad hoc network 
research~// Wirel. Commun. Mob. Com., 2002. No.\,2. P.~483--502. doi: 
10.1002/wcm.72.
\bibitem{11-sg}
\Au{Talwar G., Narang~H., Pandey~K., Singhal~P.} Analysis of different mobility models 
for ad hoc on-demand distance vector routing protocol and dynamic source routing 
protocol.~--- Lecture notes in electrical engineering ser.~--- New York, NY, USA: Springer, 
2013. Vol.~131. P.~579--588. doi: 10.1007/978-1-4614-6154-8\_57. 
\bibitem{12-sg}
\Au{Отт Г.} Методы подавления шумов и~помех в~электронных системах~/
Пер. с~англ.~--- М.: 
Мир, 1979. 318~с.
(\Au{Ott~G.} {Noise reduction techniques 
in electronic systems}.~--- 
New York, NY, USA: Wiley, 1976. 312~p.)
\bibitem{13-sg}
\Au{Гайдамака Ю.\,В., Андреев~С.\,Д., Сопин~Э.\,С., Самуйлов~К.\,Е., Шоргин~С.\,Я.} 
Анализ характеристик интерференции в~модели взаимодействия устройств с~учетом 
среды распространения сигнала~// Информатика и~её применения, 2016. Т.~10. 
Вып.~4. С.~2--10. doi: 10.14357/19922264160401.

 \end{thebibliography}

 }
 }

\end{multicols}

\vspace*{-3pt}

\hfill{\small\textit{Поступила в~редакцию 11.09.18}}

\vspace*{8pt}

%\pagebreak

%\newpage

%\vspace*{-28pt}

\hrule

\vspace*{2pt}

\hrule

%\vspace*{-2pt}

\def\tit{MODELING MOVEMENT OF DEVICES IN~A~WIRELESS NETWORK BY~RANDOM WALK MODELS}

\def\titkol{Modeling the movement of devices in a~wireless network by random 
walk models}

\def\aut{K.\,E.~Samouylov$^{1,2}$, Yu.\,V.~Gaidamaka$^{1,2}$, 
and~S.\,Ya.~Shorgin$^3$}

\def\autkol{K.\,E.~Samouylov, Yu.\,V.~Gaidamaka, 
and~S.\,Ya.~Shorgin}

\titel{\tit}{\aut}{\autkol}{\titkol}

\vspace*{-11pt}


\noindent
$^1$Peoples' Friendship University of Russia (RUDN University), 6~Miklukho-Maklaya Str., 
Moscow 117198, Russian\linebreak
$\hphantom{^1}$Federation

\noindent
$^2$Federal Research Center ``Computer Science and Control'' of the Russian Academy of 
Sciences, 44-2~Vavilov\linebreak
$\hphantom{^1}$Str., Moscow 119333, Russian Federation

\noindent
$^3$Institute of Informatics Problems, 
Federal Research Center ``Computer Science and Control'' of the Russian\linebreak
$\hphantom{^1}$Academy of Sciences, 44-2~Vavilov Str., Moscow 119333, Russian Federation


\def\leftfootline{\small{\textbf{\thepage}
\hfill INFORMATIKA I EE PRIMENENIYA~--- INFORMATICS AND
APPLICATIONS\ \ \ 2018\ \ \ volume~12\ \ \ issue\ 4}
}%
 \def\rightfootline{\small{INFORMATIKA I EE PRIMENENIYA~---
INFORMATICS AND APPLICATIONS\ \ \ 2018\ \ \ volume~12\ \ \ issue\ 4
\hfill \textbf{\thepage}}}

\vspace*{6pt}


\Abste{The authors overview mobility models which are applicable 
for simulation of  movement of users' devices in a~fifth generation (5G)
wireless network. Mobility patterns that are typical for simulating the 
behavior of users of a~wireless ad hoc network are considered. The features 
of the models are discussed, both for individual motion of objects and 
for motion of groups of objects, from the point of view of appliance to the 
analysis of interference in wireless networks. The purpose of the paper is 
to propose a~number of mobility models for making an informed decision when 
choosing a~model for evaluating the quality of service in 5G wireless networks. 
The authors present\linebreak\vspace*{-12pt}}

\Abstend{simulation results that illustrate the method of 
estimation of the key performance quality parameter, i.\,e., 
signal to interference ratio. For illustration, the developed complex of 
analytical and simulation models is used for simulation of movement of shopping moll 
customers with the help of the grid random walk mobility model.}

\KWE{entity mobility model; group mobility model; ad hoc network simulation; 
signal to interference and noise ratio; SINR}
  

  
\DOI{10.14357/19922264180401}

\vspace*{-16pt}

\Ack
\noindent
The reported study was supported by the Russian Science Foundation, 
research project  No.\,16-11-10227.



%\vspace*{-2pt}

  \begin{multicols}{2}

\renewcommand{\bibname}{\protect\rmfamily References}
%\renewcommand{\bibname}{\large\protect\rm References}

{\small\frenchspacing
 {%\baselineskip=10.8pt
 \addcontentsline{toc}{section}{References}
 \begin{thebibliography}{99}
\bibitem{1-sg-1}
\Aue{Andrews, J.\,G., S.~Buzzi, W.~Choi, S.\,V.~Hanly, A.~Lozano, A.\,C.~Soong., and 
J.\,C.~Zhang.} 2014. What will 5G be? \textit{IEEE J.~Sel. Area. Comm.} 
 32(6):1065--1082. doi: 10.1109/ JSAC.2014.2328098.
\bibitem{2-sg-1}
\Aue{Orlov, Yu.\,N., S.\,L.~Fedorov, A.\,K.~Sa\-muylov, Yu.\,V.~Gai\-da\-ma\-ka, and 
D.\,A.~Molchanov.} 2016. Simulation of devices mobility to estimate wireless channel 
quality metrics in 5G networks. \textit{AIP Conf. Proc.}  
1863:090005-1--090005-3. 2017. doi: 
10.1063/1.4992270.
\bibitem{3-sg-1}
\Aue{Gaidamaka, Yu.\,V., Yu.\,N.~Orlov, D.\,A.~Molchanov, and A.\,K.~Samuylov.} 
2017. Modelirovanie otnosheniya signal/\linebreak interferentsiya v~mobil'noy seti so sluchaynym 
bluzhdaniem vzaimodeystvuyushchikh ustroystv [Modeling the signal--interference ratio in 
a~mobile network with moving devices]. \textit{Informatika i~ee Primeneniya~--- Inform. 
Appl.} 11(2):50--58. doi: 10.14357/19922264170206.
\bibitem{4-sg-1}
\Aue{Risken, H., and T.~Frank.} 1996. 
\textit{The Fokker--Planck equation: Methods of solution 
and applications}. Springer ser. in synergetics.  Berlin--Heidelberg: 
Springer-Verlag.  Vol.~8.
 486~p. doi: 10.1007/978-3-642-61544-3.
\bibitem{5-sg-1}
\Aue{Toffoli, T., and N.~Margolus.} 1987. \textit{Cellular automata machines}. 
The MIT Press. 276~p.
\bibitem{6-sg-1}
\Aue{Grewal, M.\,S., and A.\,P.~Andrews.} 2001. \textit{Kalman filtering: 
Theory and practice using MATLAB}. 2nd ed.  John Wiley \& Sons, Inc. 410~p.

\bibitem{7-sg-1}
\Aue{Semushin, I.\,V., A.\,V.~Tsyganov, Yu.\,V.~Tsyganova, A.\,V.~Golubkov, and 
S.\,D.~Vinokurov.} 2017. Modelirovanie i~otsenivanie traektorii dvizhushchegosya ob''ekta 
[Modelling and estimation of a~moving object trajectory]. 
\textit{South Ural State University Bulletin. Ser. Mathematical Modelling, 
Programming \& Computer Software} 10(3):108--119. doi: 10.14529/mmp170309.
\bibitem{8-sg-1}
The VINT Project (Virtual InterNetwork Testbed). The Network Simulator ns-2. Available 
at: {\sf http://www.isi. edu/nsnam/ns/} (accessed September~10, 2018).
\bibitem{9-sg-1}
The Wolfram Demonstrations Project. Constrained Random Walk. Available at: {\sf 
http://demonstrations.wolfram.\linebreak com} (accessed September~10, 2018).
\bibitem{10-sg-1}
\Aue{Camp,~T., J.~Boleng, and V.~Davies.} 2002. A~survey of mobility models for ad hoc 
network research. \textit{Wirel. Commun. Com.} (2):483--502.
\bibitem{11-sg-1}
\Aue{Talwar, G., H.~Narang, K.~Pandey, and P.~Singhal.} 2013. Analysis of different 
mobility models for ad hoc on-demand distance vector routing protocol and dynamic 
source routing protocol. {Lecture notes in electrical engineering ser.} New York, NY: 
Springer. 131:579--588. doi: 10.1007/978-1-4614-6154-8\_57. 
\bibitem{12-sg-1}
\Aue{Ott, G.} 1976. \textit{Noise reduction techniques 
in electronic systems}. 
New York, NY: Wiley. 312~p.
\bibitem{13-sg-1}
\Aue{Gaidamaka, Yu.\,V., S.\,D.~Andreev, E.\,S.~Sopin, K.\,E.~Sa\-mouylov, and 
S.\,Ya.~Shorgin.} 2016. Analiz kharakteristik interferentsii v~modeli vzaimodeystviya 
ustroystv s~uche\-tom sredy rasprostraneniya signala [Analysis of the characteristics of the 
interference in the model of interaction between devices taking into account the signal 
propagation environment]. \textit{Informatika i~ee Primeneniya~--- Inform. Appl.}  
10(4):2--10. doi:10.14357/19922264160401.
\end{thebibliography}

 }
 }

\end{multicols}

\vspace*{-6pt}

\hfill{\small\textit{Received September 11, 2018}}

%\pagebreak

\vspace*{-18pt}

\Contr

\noindent
\textbf{Samouylov Konstantin E.} (b.\ 1955)~--- Doctor of Science in technology, professor, Head of Department, Director 
of Institute of Applied Mathematics and Telecommunications, Peoples' Friendship University of Russia (RUDN University), 
6~Miklukho-Maklaya Str., Moscow 117198, Russian Federation; senior scientist, Federal Research Center 
``Computer Science and Control'' of the Russian Academy of Sciences, 44-2~Vavilov Str., Moscow 119333, Russian Federation, 
\mbox{samuylov\_ke@rudn.university}

\vspace*{1pt}

\noindent
\textbf{Gaidamaka Yuliya V.} (b.\ 1971)~--- Doctor of Science in physics and 
mathematics, professor, Peoples' Friendship University of Russia (RUDN 
University), 6~Miklukho-Maklaya Str., Moscow 117198, Russian Federation; 
senior scientist, Federal Research Center ``Computer Science and Control'' of the 
Russian Academy of Sciences, 44-2~Vavilov Str., Moscow 119333, Russian 
Federation, \mbox{gaydamaka\_yuv@rudn.university}

\vspace*{1pt}

\noindent
\textbf{Shorgin Sergey Ya.} (b.\ 1952)~--- Doctor of Science in physics and 
mathematics, professor, principal scientist, Institute of Informatics Problems, 
Federal Research Center ``Computer Science and Control'' of the Russian 
Academy of Sciences, 44-2~Vavilov Str., Moscow 119333, Russian Federation; 
\mbox{sshorgin@ipiran.ru}
\label{end\stat}

\renewcommand{\bibname}{\protect\rm Литература}        %1
%\renewcommand{\figurename}{\protect\bf Figure}
\renewcommand{\tablename}{\protect\bf Table}

\def\stat{razum}


\def\tit{COMPARISON OF TWO ACTIVE QUEUE MANAGEMENT SCHEMES THROUGH THE~$M/D/1/N$ 
QUEUE}

\def\titkol{Comparison of two active queue management schemes through the $M/D/1/N$ 
queue}

\def\autkol{M.\,G.~Konovalov and R.\,V.~Razumchik}

\def\aut{M.\,G.~Konovalov$^1$ and R.\,V.~Razumchik$^2$}

\titel{\tit}{\aut}{\autkol}{\titkol}

%{\renewcommand{\thefootnote}{\fnsymbol{footnote}}
%\footnotetext[1] {The 
%research of Yuri Kabanov was done under partial financial support   of the grant 
%of  RSF No.\,14-49-00079.}}

\renewcommand{\thefootnote}{\arabic{footnote}}
\footnotetext[1]{Institute of Informatics Problems, Federal Research Center ``Computer Science and Control'' of the Russian Academy of Sciences,
44/2~Vavilov Str., Moscow 119333, Russian Federation, \mbox{mkonovalov@ipiran.ru}}
\footnotetext[2]{Institute of Informatics Problems, Federal Research Center 
``Computer Science and Control'' of the Russian Academy of Sciences,
44/2~Vavilov Str., Moscow 119333, Russian Federation; 
Peoples' Friendship University of Russia (RUDN University),
6~Miklukho-Maklaya Str., Moscow 117198, Russian 
Federation; 
\mbox{rrazumchik@ipiran.ru} %\mbox{razumchik\_rv@rudn.ru
}


\index{Konovalov M.\,G.}
\index{Razumchik R.\,V.}
\index{Коновалов М.\,Г.}
\index{Разумчик Р.\,В.}

\def\leftfootline{\small{\textbf{\thepage}
\hfill INFORMATIKA I EE PRIMENENIYA~--- INFORMATICS AND
APPLICATIONS\ \ \ 2018\ \ \ volume~12\ \ \ issue\ 4}
}%
 \def\rightfootline{\small{INFORMATIKA I EE PRIMENENIYA~---
INFORMATICS AND APPLICATIONS\ \ \ 2018\ \ \ volume~12\ \ \ issue\ 4
\hfill \textbf{\thepage}}}



\Abste{The paper focuses on giving the first in the literature numerical evidence
that the stationary performance characteristics of single-server queues
with the general renovation mechanism may be as good as of single-server queues
with the RED-type active queue management mechanisms
(AQM). Comparison is made in the queueing
theory context: the basic model is the $M/D/1/N$ queue. 
The characteristics reported are: the loss ratio, average system size, and 
average number of consecutive losses along 
with the standard deviations. Numerical results are based on the 
well-known facts and some new analytic results, presented in the paper.}

\KWE{queueing system; active queue management; RED; renovation}

\DOI{10.14357/19922264180402}


\vspace*{1pt}


\vskip 12pt plus 9pt minus 6pt

      \thispagestyle{myheadings}

      \begin{multicols}{2}

                  \label{st\stat}


\section{Introduction}

\noindent
A large number of AQM mechanisms have been developed
up to nowadays and quite a~lot of efforts have been devoted to the studies of
their efficiency. 
These mechanisms may be applicable in different contexts but historically, 
they are more often related to communication networks
in the context of mitigation of congestion and congestion avoidance.
This problem, as highlighted in the latest RFC~7567~\cite{RFC7567},
still does not have a~satisfying solution. 
An AQM mechanism is an advanced rejection discipline, 
when an arriving customer (packet, job, etc.) is lost randomly with a~probability 
that may depend on the (current, past, average, etc.) system state or performance.
The most popular class of AQM mechanisms seems to be the Random Early Detection (RED) and
its ramifications like GRED (Gentle RED), REM 
(Random Exponential Marking),
etc.\ (a~recent survey on the AQM can be found in~\cite{Adams}).
The goals of AQM are usually diverse and conflicting: 
prevent queues from growing too long, maintain high server (processor) utilization
and low variance of the queue size, ensure fairness among competing flows, 
and others. These are discussed in detail in~\cite{RFC7567} in the context
of communications network but most of the goals are applicable in other contexts as well
(buffer-bloat problems in data-center, etc.).

Besides simulation, analytic performance evaluation of systems with AQM is quite often
carried out in the queueing theory context (see,
for example,~\cite{Bonald,Chyd,Chyd2,oleg,hao,konnew} and references therein). 
Usually, the system with an AQM mechanism is modeled as a~queueing system or network
and then its performance characteristics are studied using known analytic techniques. 
Throughout the paper, we stay within the queueing theory context.

In the series of recent papers~\cite{Kreinin,Zaryadov2010,zarN1,zarN2,Zaryadov2009},
the authors have proposed the new type of AQM mechanism which they call 
\textit{renovation}. 
Roughly speaking, renovation implies that each customer, 
having received service, may remove some additional work from the system
(i.\,e., may renovate it). We will make this definition more precise in the next 
section 
but for now, note 
that queue management \textit{after service completions} is what makes the renovation
 different 
from the most known AQM schemes\footnote[3]{Indeed, renovation and most of the known AQM 
mechanisms
are conceptually different. One of the main goals of AQM mechanisms is to prevent 
queue from growing too large
leaving space for potential new arrivals.
In systems with renovation, the queue can become full (meaning that fewer customers are 
lost)
but after a~service completion, several customers may be removed from it. In this way, 
the content of the queue
can be preserved at a~certain average level but the loss pattern becomes intricate.}, 
in which the decisions are made \textit{upon arrivals}.  
To our best knowledge, there are no studies, 
which tell whether the performance of the systems with renovation is
better/same/worse than that of the same systems but with the implemented AQM mechanisms.
Thus, there is a~lack of bridge between available theoretical results for renovation and 
its practical perspective.

The scope of this paper is to give the first in the literature numerical evidence 
that the stationary performance of 
single-server queueing systems with the implemented renovation mechanism can
be as good as of 
the same single-server queues but the well-known packed dropping procedures like RED.
The emphasis is primarily on the reporting of this finding, complemented 
with some new insights into 
queueing systems with renovation. The relation to other 
AQM mechanisms like CoDel~\cite{RFC8289} is not discussed here. 
Moreover, in the numerical experiments presented here, 
we did not use any benchmarks to generate the traffic profiles 
but used the theoretical distributions instead.

The main stationary performance characteristics reported are: the loss ratio, 
the average number in the system
(average system size), and the average number of consecutive losses along with 
their standard deviations.
After introducing the renovation mechanism and the analytic setting,
in which  renovation mechanism is compared with RED, we give the new analytic results 
for computing system size moments and the loss ratio under the renovation mechanism.
The results presented in the numerical section are based on the analytic results. 
Monte-Carlo simulation is used only for the average (and standard deviation) 
number of consecutive losses in the system with renovation. 


\section{Settings and the Model}

\noindent
We follow the queueing theoretic approach and as the basic model, we use $M/D/1/N$ queue,
i.\,e., queue of finite capacity~$N$ fed at rate~$\lambda$ by a~Poisson flow of customers,
which are served on a~first-come-first-served basis  by a~single server with constant
service time $d>0$.
We assume that the system is in the steady state.
When an arriving customer sees that the queue is full,
it is lost. If no other type of losses occur in the system,
we say that the Tail Drop mechanism is implemented in it.

If an arriving customer is lost with probability~$d_n$
where~$n$ is the total number of customers 
it sees in the system on arrival, then we say that an AQM mechanism 
is implemented in the system. Various dropping functions can be obtained
by specifying the values of~$d_n$
(see, for example, RED dropping function in~\cite[Example 1]{Chyd}).
Important notice should be made here. In practice, RED-type mechanisms 
may use moving averages of the queue size instead of its instantaneous value. 
Thus, the way~$d_n$ introduced above is a~simplified way of thinking.
Yet, this trade-off is important because it allows to keep the mathematical 
models of RED-type AQM tractable.
Luckily, as noticed in~\cite[Section II.C]{Bonald}, such approximation may not
lead to significant bias, when the weight of the moving average scheme is small
(which is claimed to be the case sometimes in practice).

The renovation mechanism, which is implemented in a~system with Tail Drop,
works as follows. Define $N+1$ numbers, say, $q_i\ge 0$,
$0 \le i \le N$, satisfying \mbox{$\sum\nolimits_{i=0}^N q_i=1$}.
If upon service completion there are $i$, $1 \le i \le N$, customers
waiting in the queue, then the served customer leaves the system and
\begin{itemize}
\item with probability $q_0+Q_i$ nothing else happens, where 
$Q_i=q_i+q_{i+1}+\dots+ q_N$; and
\item with probability $q_j$, $0<j<i$, exactly $j$ customers
from the queue leave the system and those customers 
are chosen successively \textit{starting from the head of the queue}.
\end{itemize}
%\noindent
The served customer, which sees the empty queue, leaves the system.
Thus, after the renovation (if it happened), the system never becomes empty.
%what is appealing from the practical point of view. 

In the numerical section, we rank the systems with RED 
and renovation according to the stationary loss ratio, average system size,  
and average number of consecutive losses along with their standard deviations. 
The system with the Tail Drop is the standard $M/D/1/N$ queue, 
for which all these performance characteristics follow
from the classical results in queueing theory (see, for example,~\cite{Riordan1962}).
Analytic results for the systems of $M/G/1/N$ type with relatively 
arbitrary dropping functions are given in~\cite{Chyd}.
Yet, for the system with renovation, we need to derive these 
performance characteristics anew, since 
the available results in~\cite{Zaryadov2010,Zaryadov2009} 
are not valid for the renovation mechanism introduced above.
We briefly sketch the derivations in the next
section and omit most of the details
since they are based on the methodology, 
developed in~\cite{Zaryadov2010,Zaryadov2009},
and reviewed in~\cite{arxivRK}.

%Note that the above mentioned performance characteristics 
%do not depend on the order in which the customers
%are removed from the system; yet in the derivations we assume that the customers
%are chosen successively \textit{starting from the head of the queue}.

%The analytic results and parameters' values for 
%RED and REM are due to \cite{Chyd}.

\section{Performance Characteristics}

\noindent
Consider the $M/D/1/N$ queue with the renovation mechanism
introduced above. Since a~customer is served for constant service time
$d$, then for the cumulative distribution function $B(x)$ 
of its service time, one has: 
$$
B(x)=
\begin{cases}
0 & \mbox{if } x \le d\,;\\
1 & \mbox{if } x>d\,.
\end{cases}
$$
Let $N(t)$ be the total number of customers %\footnote{We assume that the system 
%starts empty, i.\,e., $N(0)=0$.} 
in the system at instant $t$ 
and $E(t)$ be the elapsed service time of the customer in server
(if there is one). 
In order to compute the stationary system size moments, 
one needs to know the stationary distribution:
\begin{equation*}
%\label{pn}
P_n=\lim\limits_{t \rightarrow \infty} \mathbf{P}\{ N(t)=n \},\enskip  0 \le n \le N+1\,.
\end{equation*} 
For the computation of the loss ratio,
due to the \mbox{PASTA} (Poisson Arrivals See Time Averages) 
property, it is sufficient to know
 the stationary probability densities
$p_n(x)=P'_n(x)$ where
\begin{multline*}
%\label{pnx}
P_n(x)=\lim\limits_{t \rightarrow \infty} \mathbf{P}\{ N(t)=n, E(t)<x \}, \\ 
1\le n \le N, \ x \in [0,d]\,.
\end{multline*}
Since we are dealing with the finite-capacity queue 
and work conserving service discipline, the
introduced stationary distributions exist.  
The probabilities~$P_n$ and 
the densities~$p_n(x)$ can be computed as follows. 
Let~$t_n$ denotes the $n$th service completion epoch 
and $N_n=N(t_n+0)$ denotes the total number of customers in the system. 
Clearly, $\{ N_n, \ n \ge 1\}$ is the finite-state Markov chain.
The entries of the transition probability matrix $\mathbb{P}=(p_{ij})$
of this chain have the form:
$$
p_{0j}=p_{1j}=
\begin{cases}
\beta_0, & \hspace*{-20mm}j=0;\\
\beta_j Q_j + \displaystyle\sum\limits_{k=j}^N \beta_k q_{k-j} +  B_N q_{N-j}, &\\
&\hspace*{-20mm} 1 \le j \le N-1\,;\\
(q_0 + q_N) B_{N-1}, & \hspace*{-20mm}j=N\,;
\end{cases}
$$
\begin{multline*}
\!\!p_{ij}=
\begin{cases}
0, & \hspace*{-38mm}j=0;\\
\sum\limits_{k=j-1}^{N-1} \beta_k q_{k-j+1} +  B_{N-1} q_{N-j}, & \\
& \hspace*{-38mm}1 \le j \le i-2\,;\\
\beta_{j-i+1} Q_j + \displaystyle\!\sum\limits_{k=j-1}^{N-1}\!\! \beta_k q_{k-j+1} + 
 B_{N-1} q_{N-j}, &\\
 &\hspace*{-38mm} i-1 \le j \le N-1;\\
(q_{0} +  q_{N})B_{N-i} , &\hspace*{-38mm} j=N\,,
\end{cases}
\\
 2 \le i \le N\,.
\end{multline*}
Here, $B_0=1-\beta_0$; $B_k=B_{k-1}-\beta_k$; and 
$\beta_k=[{(\lambda d)^k / k!}]e^{-\lambda d}$.
The matrix $\mathbb{P}$ does not have any special structure. 
So, the stationary distribution $\{P^+_n, \ 0 \le n\linebreak \le N\}$
may be found in a~straightforward manner by solving the system of linear algebraic 
equations 
$$
{\vec P}^+={\vec P}^+ \mathbb{P};\enskip 
{\vec P}^+ {\vec 1} =1
$$ 
where ${\vec P}^+= (P^+_0,\dots,P^+_N)$ and $\vec 1$ is the vector of ones. 
{\looseness=1

}

Once the probabilities $P^+_n$ are found,
the stationary distribution \mbox{$\{P_n, \ 0 \le n \le N+1\}$} 
is computed from the relation\footnote{This follows
from the well-known results for the Markov regenerative processes (see, for 
example,~\cite[Theorem 9.19]{kulk}).} 
$$
P_n=\sum\limits_{i=0}^N P^+_i \fr{f_{in}}{f^*}
$$
where $f_{in}$ is the average time during which there were $n$ customers in the system
provided that the system started with~$i$ customers in it; 
and~$f^*$ is the mean time between transitions of the embedded
Markov chain $\{ N_n, \ n \ge 1\}$.
{\looseness=1

}


Finally,the stationary probability densities $p_n(x)\linebreak =P'_n(x)$
can be computed using the fact that the relation for~$p_n(x)$ 
coincides with the relation for $p_n(x)$ in 
the standard $M/D/1/N$ queue.
Thus, $p_n(x)$ are given by (see, for example,~\cite[p.~72]{Riordan1962})

\noindent
\begin{multline}
\label{eq3}
p_n(x)=e^{-\lambda x} \left (1-B(x) \right ) 
\sum\limits_{k=0}^{n-1} p_{n-k}(0) 
\fr{(\lambda x)^k}{k!}\,, \\
1 \le n \le N\,,  \ x \in [0,d]\,.
\end{multline}
Even though~(\ref{eq3}) holds,
due to the presence of renovation, the boundary conditions $p_{n}(0)$ 
for the considered queue do not coincide 
with boundary conditions $p_{n}(0)$ for the standard $M/D/1/N$ queue.
By integration~(\ref{eq3}) from~0 to~$d$, one gets 
the following relation between~$p_{n}(0)$ and $P_n=\int\nolimits_0^d p_n(x) dx$: 
\begin{multline}
\label{eq3nn}
p_n(0)
= \fr{1}{B_0} \left (\lambda P_n- \sum\limits_{k=1}^{n-1} B_k p_{n-k}(0)
\right )\,, \\ 
1 \le n \le N\,.
\end{multline}
Since the stationary distribution \mbox{$\{P_n, \ 0 \le n \le N+1\}$} 
is already known, the values of $p_n(0)$ are computed recursively from~(\ref{eq3nn}).
The closed-form expressions for
 the average and the standard deviation of the system size are, in the most cases,
 not available and thus, they can be computed,
respectively, by $\sum\nolimits_{n=0}^{N+1} nP_n$ and  
$\sqrt{\sum\nolimits_{n=0}^{N+1} n^2P_n-(\sum_{n=0}^{N+1} nP_n)^2}$.

The computation of the loss ratio~$\pi$, i.\,e., the probability that the arriving customer is lost, 
is more involved. This is due to the fact that the accepted customer
may be lost either after the first service completion or the second, etc.
and the chance to be lost varies, depending on the number of
new customers that have arrived between successive service completions.
 
Let us introduce two quantities:
\begin{enumerate}[(1)]
\item $\gamma_{ij}$, $1 \le i \le N$, $j \ge 0$,~--- probability that the arriving customer
finds~$i$~customers in the system and until the next
service completion, exactly $j$ new customers arrive 
at the system; and
\item
$r_{ij}$, $0\le j \le N-1$, $0 \le i \le N-j-1$,~--- probability that the 
tagged customer waiting in the queue
\textit{will not} be served (i.\,e., will be lost), if there are~$j$~customers
 in front of it in the queue (excluding the one in server)
and~$i$ behind.
\end{enumerate}

Given that $\gamma_{ij}$ and $r_{ij}$ are known, the loss ratio~$\pi$  
can be computed as
\begin{multline*}
\pi =
P_{N+1}
 + 
\sum\limits_{i=1}^{N} \left [
\sum\limits_{j=0}^{N-i}
\gamma_{ij} \left ( \sum\limits_{k=0}^{i-2} q_k r_{j,i-2-k}
 +{}\right.\right.\\
\left. {}+ \sum\limits_{k=i}^{i+j-1} q_k  + Q_{j+i} r_{j,i-2} \right )
 + {}
\end{multline*}

\noindent
\begin{multline*}
{}+ \sum\limits_{j=N-i+1}^{\infty} \gamma_{ij} \left (
\sum\limits_{k=0}^{i-2} q_k r_{N-i,i-2-k}  + {}\right.\\
\left.\left.{}+\sum\limits_{k=i}^{N-1}
q_k  + Q_{N} r_{N-i,i-2} \right )
\right ]\,.
%\label{ploss2}
\end{multline*}

Due to the PASTA property of Poisson arrivals,
the expression for $\gamma_{ij}$ follows 
from the law of total probability:
\begin{multline*}
\gamma_{ij}
= \int\limits_0^d p_{i}(x) \fr{(\lambda (d- x))^j }{j!}\, e^{-\lambda (d-x)}\, dx\,,
\\
 1 \le i \le N\,, \ j \ge 0\,.
\end{multline*}
Again, by applying the law of total probability,
one gets the relations for the recursive computation of~$r_{ij}$, 
$0\le j \le N-1$, $0 \le i \le N-j-1$:
\begin{align*}
r_{i0}&= \sum\limits_{m=0}^{N-i-1}
\beta_m 
\sum\limits_{k=1}^{m+i} q_k +
B_{N-i-1} \sum\limits_{k=1}^{N-1} q_k\,;\\
r_{ij}&= \sum\limits_{m=i}^{N-1-j} \beta_{m-i} \left (
\vphantom{\sum\limits_{k=j+1}^{m+j} q_k + Q_{j+m+1} r_{m,j-1}}
\sum\limits_{k=0}^{j-1} q_k r_{m,j-1-k} +{}\right.\\
&\left.{}+
\sum\limits_{k=j+1}^{m+j} q_k + Q_{j+m+1} r_{m,j-1}
\right ) +{}
\\
&{}+ B_{N-j-i-1}
\left ( \vphantom{\sum\limits_{k=j+1}^{m+j} q_k + Q_{j+m+1} r_{m,j-1}}
\sum\limits_{k=0}^{j-1} q_k r_{N-j-1,j-1-k} +{}\right.\\
&\hspace*{16mm}\left.{}+\sum\limits_{k=j+1}^{N-1} q_k +Q_{N} r_{N-j-1,j-1}
\right ).
\end{align*}
The expressions above can be further simplified\footnote{There
are no principal difficulties in generalizing
the results for the $\mathrm{BMAP}/G/1/N$ queue.
Yet, this would obscure the goal of the paper and thus, 
we remain with the simple model.} by computing 
the integrals explicitly, but we do not dwell on it here.
For small and moderate values of~$d$, $N$, and~$\lambda$,
they can be directly used for numerical implementation.
In the numerical section, precisely these expressions are used to calculate
the loss ratio. The expressions for the average and the standard
deviation of consecutive losses are much harder to derive
and we leave it for a~separate study. The values of these 
two parameters were taken from the Monte-Carlo simulation. 

\section{Numerical Example}

\noindent
As the reference point, we have chosen the numerical results in~\cite{Chyd}
which are based on the analytic expressions and which show the 
performance characteristics 
of the $M/D/1/N$ queue with four different AQM mechanisms. 
Since RED scheme is one of the best among the four,
our goal here is to rank the $M/D/1/N$ queue with RED from~\cite{Chyd}
and the $M/D/1/N$ queue with renovation. Comparison is made 
according to the stationary loss ratio, average system size,  
and average number of consecutive losses along with 
their standard deviations.

The initial conditions are: 
the maximum queue size is $N=9$ and the service time is $d=1$. 
Thus, the offered load is $\rho=\lambda d$. 
The RED dropping function is given by (see~\cite[Eq. (59)]{Chyd}):
\begin{equation}
\label{df}
d_n=
\begin{cases}
0\,, & n\le 3\,;\\
0.11917n - 0.35752\,, & 4 \le n \le 9\,;\\
1\,, & n=10.
\end{cases}
\end{equation}
The performance of the $M/D/1/N$ queue with this RED dropping function 
is given in~\cite[Tables 1, 3, and~4]{Chyd}.
In order to find out whether there exists a~renovation 
mechanism under which the $M/D/1/N$ queue can perform at least as good as
under RED, one needs to perform exhaustive search over
the possible values of the renovation parameters $\{q_i, \ 0 \le i \le N\}$.
Since we are unaware of any analytic way of choosing these values,
adaptive search algorithms for partially observable
Markov decision processes from~\cite{kono1} were used instead.
Meta-heuristics (like particle swarm optimization), which are also applicable here,
were not used.

In Tables~1 and~2, one can see the numerical results for the four different
cases of the offered load\footnote{For the sake of reproducibility 
of the results 
presented in the paper, we also report the obtained values of the renovation 
probabilities: for $\rho=0.5$,
${\vec q}=(0.2544,0.0037,0.0053,0.0065,0.0122,0.0352,0.1108,0.1898,0.2129,0.1691)$;
for $\rho=1$, $q_0=0.0551$, $q_6=0.051$, $q_7=0.7166$, $q_8=0.0917$, and
$q_9=0.0856$;
for $\rho=2$, $q_0=0.1078$, $q_1=0.6374$, $q_4=0.0042$, $q_6=0.0084$, and
$q_9=0.2422$; and
for $\rho=3$, $q_1=0.4608$, and $q_2=0.5392$.}~$\rho$: $\rho=0.5$~--- underloaded system;
$\rho=1$~--- critically loaded system;
and $\rho=2$ and~3~--- overloaded system. The values displayed are the 
values from the numerical experiments rounded to three decimal digits.

%\noindent and in each case compute the stationary 
%loss ratio, average system content  
%and average number of consecutive losses along with
%their standard deviations. 


\begin{table*}\small
\begin{center}
\parbox{400pt}{\Caption{Performance characteristics of the $M/D/1/9$ system with the RED 
mechanism~(\ref{df}) and the $M/D/1/9$ the renovation mechanism (ren.)\
under the offered load $\rho=0.5$ and $\rho=1$}
}
%\label{my-label}
\vspace*{2ex}

\tabcolsep=8pt
\begin{tabular}{cc|c|c|c||c|c|c|}
\cline{3-8}
                                        &  & \multicolumn{3}{c||}{$\rho=0.5$} & \multicolumn{3}{c|}{$\rho=1$} \\ \cline{3-8} 
                                        &  &   Tail Drop      & RED     &    ren.   &    Tail Drop   &      RED &    ren.   \\ \hline
\multicolumn{2}{|c|}{loss ratio}    &   0    &   0.002        &   0.002  &    0.051   &    0.091        &   0.104    \\ \hline
\multicolumn{1}{|c|}{\textbf{system}} & average &   0.750    &   0.741        &    0.744    &    5.064   &      3.000       &   2.999  \\ \cline{2-8} 
\multicolumn{1}{|c|}{\textbf{size}} & standard deviation &   0.946    &    0.920      &    0.935      &    2.897   &   1.887       &   2.091      \\ \hline
\multicolumn{1}{|c|}{\textbf{consecutive}} &average  &  1.152     &  1.053        &   1.800     &    1.359   &    1.239       &  6.876         \\ \cline{2-8} 
\multicolumn{1}{|c|}{\textbf{losses}} & standard deviation &  0.403     &   0.240       &    1.260     &    0.647   &       0.561        &   0.852     \\ \hline
\end{tabular}
\end{center}
\vspace*{-6pt}
\end{table*}




\begin{table*}\small %tabl2
\begin{center}
\parbox{400pt}{\Caption{Performance characteristics of the $M/D/1/9$ system with the RED 
mechanism~(\ref{df}) and the $M/D/1/9$ the renovation mechanism (ren.)\
under the offered load $\rho=2$ and $\rho=3$}
}
%\label{my-label}
\vspace*{2ex}

\tabcolsep=8pt
\begin{tabular}{cc|c|c|c||c|c|c|}
\cline{3-8}
                                        &  & \multicolumn{3}{c||}{$\rho=2$} & \multicolumn{3}{c|}{$\rho=3$} \\ \cline{3-8} 
                                        &  &   Tail Drop      & RED     &    ren.   &    Tail Drop   &      RED &    ren.   \\ \hline
\multicolumn{2}{|c|}{loss ratio}    &   0.500    &    0.500        &   0.502     &    0.667   &    0.667       &     0.667      \\ \hline
\multicolumn{1}{|c|}{\textbf{system}} & average &   9.372    &    7.146       &   7.142      &  9.646     &      8.390        &    7.114  \\ \cline{2-8} 
\multicolumn{1}{|c|}{\textbf{size}} & standard deviation &   0.744    &     1.436       &  2.387      &    0.523   &     1.090        &    2.246  \\ \hline
\multicolumn{1}{|c|}{\textbf{consecutive}} &average  &  1.884     &   1.996         &  1.592       &     2.542  &     2.876        &   2.141    \\ \cline{2-8} 
\multicolumn{1}{|c|}{\textbf{losses}} & standard deviation &   1.069    &    1.366        &  1.100      &    1.454   &       2.064       &  1.236        \\ \hline
\end{tabular}
\end{center}
\end{table*}


Data is the tables show that with respect to the loss ratio,
renovation can perform as good as RED in the wide range of the offered load~$\rho$.
The only exception is the case $\rho=1$: here, renovation can keep
only the average system size at the same level as RED; other four 
performance characteristics are worse. 

If we fix the loss ratio, then the renovation mechanism
can guarantee at least the same value of the average system size as guaranteed by RED.
It is worth noticing that as the offered load increases, the average system size 
under the renovation mechanism becomes smaller than the average system size under RED.
Yet, renovation keeps the queue less stable than RED:
the standard deviation of system size is always smaller for RED.

If we fix the loss ratio and the average system size,
then the renovation mechanism spreads out the losses 
worse than RED when the system is underloaded and 
better than RED when it is overloaded.
This can be seen from the values of the averages and
standard deviations in the last two rows of Tables~1 and~2.

\vspace*{-6pt}

\section{Concluding Remarks}

\noindent
Even though the idea behind the renovation-type AQMs is completely 
different from the idea behind RED-type AQMs, 
renovation-type AQMs may allow one to achieve in some cases at least 
the same system performance level as guaranteed by RED-type AQMs. 
Although the comparison presented here is based only on a~single RED dropping 
function~(\ref{df}), 
our numerical experiments show that the results remain 
qualitatively the same for RED-type AQMs with other dropping functions.
Being defined by~$N$~parameters, the renovation mechanism is very flexible
and this constitutes its strength and weakness.
By varying the values of the renovation probabilities~$q_i$,
it is possible to carry out conditional optimization,
but good searching procedures are required here.

Implementation of the renovation as a~packet dropping mechanism
requires \textit{a~priori} tuning and/or operational configuration of its parameters.
Thus, whether it is appropriate to use renovation as a~packet dropping mechanism 
or not in practice heavily depends on the use case.
Although the tuning of the renovation parameters~$q_i$ can be made on the 
fly during operation, with respect to the recommendations of the RFC~7567~\cite{RFC7567},
renovation mechanism is not the proper choice for the network congestion control 
unless simple recommendations on how to set up the renovation parameters are given.
We believe this can be done based on more deep and insightful numerical experiments.

There remain a~large number of unresolved issues 
related to the renovation mechanism 
(e.\,g., can renovation ensure fairness among competing flows?
may the average queue size instead of its instantaneous value
increase the efficiency of the renovation mechanism?)
and this motivates its further analysis. 
Furthermore, evaluation of the renovation mechanism with parameters 
adapted to a~realistic router/switch use case
and/or evaluation which includes TCP feedback loops 
of several flows remains an open issue as well.

\vspace*{-6pt}


\Ack
  \noindent
   The reported study was partially funded by the Russian Foundation for 
Basic Research according to the research project No.\,18-07-00692.

The authors would like to thank the anonymous referees for their valuable comments 
which helped to improve the paper.
  
 \renewcommand{\bibname}{\protect\rmfamily References}

%\vspace*{-6pt}

\vspace*{-6pt}

{\small\frenchspacing
{\baselineskip=10.65pt
\begin{thebibliography}{99}
\bibitem{RFC7567} %1
\Aue{Baker, F., and G.~Fairhurst.} 2015.
IETF recommendations regarding active queue management.
Available at: {\sf https://tools.ietf.org/html/7567} (accessed October~4, 2018).



\bibitem{Adams} %2
\Aue{Adams, R.} 2013. 
Active queue management: A~survey. \textit{IEEE Commun. Surv.
Tut.} 15(3):1425--1476.

\bibitem{Bonald} %3
\Aue{Bonald, T., M.~May, and J.\,C.~Bolot.}
2000. Analytic evaluation of RED performance. 
\textit{IEEE Conference on Computer Communications Proceedings}
3:1415--1424.

\bibitem{hao} %4
\Aue{Hao, W., and Y.~Wei.} 2005.
An extended $GI^X/M/1/N$ queueing
model for evaluating the performance of AQM algorithms
with aggregate traffic.
\textit{Networking and mobile computing}.
Eds.\ Xicheng Lu and Wei Zhao.
{Lecture notes in computer science ser.} Springer. 3619:395--404.

\bibitem{Chyd} %5
\Aue{Chydzi$\acute{\mbox{n}}$nski, A., and L.~Chr$\acute{\mbox{o}}$st.} 
2011. Analysis of AQM queues with queue size based packet
dropping. \textit{Int. J.~Appl. Math. Comp.} 21(3):567--577.

\bibitem{Chyd2} %6
\Aue{Chydzi$\acute{\mbox{n}}$nski, A., and P.~Mrozowski.} 2016. 
Queues with dropping functions and general arrival
processes. \textit{PLoS ONE} 11(3):e0150702. Available at: 
{\sf https://\linebreak journals.plos.org/plosone/article?id=10.1371/journal.\linebreak pone.0150702} 
(accessed October~4, 2018).

\bibitem{oleg} %7
\Aue{Tikhonenko, O., and W.~Kempa.} 2016. Performance evaluation of 
an $M/G/n$-type queue
with bounded capacity and packet dropping. \textit{Int. J.~Appl.
Math. Comp.} 26(4):841--854.




\bibitem{konnew} %8
\Aue{Konovalov, M.\,G., and R.\,V.~Razumchik.} 2018. 
Numerical analysis of improved access restriction algorithms in a~$GI/G/1/N$
system. \textit{J.~Commun. Technol. El.} 63(6):616--625. 


\bibitem{Kreinin} %9
\Aue{Kreinin, A.\,Y.} 1997.
Queueing systems with renovation. 
\textit{J.~Appl. Math. Stochastic Analysis} 10(4):431--441.


\bibitem{zarN2} %10
\Aue{Zaryadov, I.\,S.} 2009. Queueing systems with general renovation.
\textit{Conference (International)
on Ultra Modern Telecommunications Proceedings}. 1--4.

\bibitem{Zaryadov2009} %11
\Aue{Zaryadov, I.\,S., and A.\,V.~Pechinkin.} 2009.
Stationary time characteristics of the ${GI/M/n/\infty}$
system with some variants of the generalized renovation discipline. \textit{Automat.
Rem. Contr.} 70(12):2085--2097.

\bibitem{Zaryadov2010} %12
\Aue{Zaryadov, I.\,S.}
2010. The ${GI/M/n/\infty}$ queuing system with generalized renovation.
\textit{Automat. Rem. Contr.} 71(4):663--671.

\bibitem{zarN1} %13
\Aue{Korolkova, A., and I.~Zaryadov.} 2010.
The mathematical model of the traffic transfer process with a~rate adjustable by {RED}.
\textit{Conference (International) on Ultra Modern Telecommunications Proceedings}. 
1046--1050.

\bibitem{RFC8289} %14
\Aue{Nichols, K., V.~Jacobson, A.~McGregor, and J.~Iyengar.} 2018.
Controlled delay active queue management.
Available at: {\sf https://datatracker.ietf.org/doc/rfc8289} (accessed October~4, 2018).

\bibitem{Riordan1962} %15
\Aue{Riordan, J.} 1962. \textit{Stochastic service systems}. 
SIAM ser. in applied mathematics. New York, NY: Wiley. 139~p.

\bibitem{arxivRK}  %16
\Aue{Konovalov, M.,  and R.~Razumchik.} 2017.
Queueing systems with renovation vs.\ queues with RED. Supplementary material. 
\textit{ArXiv e-prints}. Available at: {\sf https://arxiv.\linebreak org/abs/1709.01477/}
(accessed October~4, 2018).

\bibitem{kulk} %17
\Aue{Kulkarni, V.\,G.} 2016. \textit{Modeling and analysis of stochastic systems}. 
3rd ed. Chapman \&~Hall/CRC texts in statistical science ser.
Chapman \&~Hall/CRC. 606~p.

\bibitem{kono1} %18
\Aue{Konovalov, M.\,G.} 2007.
\textit{Metody adaptivnoy obrabotki informatsii i~ikh prilozheniya}
[Methods of adaptive information processing and their applications]. 
Moscow: Institute of Informatics Problems of RAS. 212~p.
\end{thebibliography} } }

\end{multicols}

\vspace*{-6pt}

\hfill{\small\textit{Received October 9, 2018}}

\vspace*{-18pt}
  

 \Contr

\noindent
\textbf{Konovalov Mikhail G.} (b.\ 1950)~--- 
Doctor of Science in technology, principal scientist, Institute of Informatics
Problems, Federal Research Center ``Computer Science and Control'' 
of the Russian Academy of Sciences, 44-2~Vavilov Str., Moscow 119333, 
Russian Federation; \mbox{mkonovalov@ipiran.ru}

\vspace*{3pt}


\noindent
\textbf{Razumchik Rostislav V.} (b.\ 1984)~--- 
Candidate of Science (PhD) in physics and mathematics, leading scientist,
Institute of Informatics Problems, Federal Research Center ``Computer 
Science and Control'' of the Russian
Academy of Sciences, 44-2~Vavilov Str., Moscow 119333, Russian Federation; 
associate professor, Peoples'
Friendship University of Russia (RUDN University), 
6~Miklukho-Maklaya Str., Moscow 117198, Russian
Federation; \mbox{rrazumchik@ipiran.ru} %; \mbox{razumchik\_rv@rudn.ru}

\vspace*{6pt}

\hrule

\vspace*{2pt}

\hrule

\vspace*{-2pt}

%\newpage

%\vspace*{-24pt}

\def\tit{СРАВНЕНИЕ ДВУХ МЕХАНИЗМОВ АКТИВНОГО УПРАВЛЕНИЯ ОЧЕРЕДЬЮ В~СИСТЕМЕ $M/D/1/N$$^*$}

\def\titkol{Сравнение двух механизмов активного управления очередью в~системе $M/D/1/N$}

\def\aut{М.\,Г.~Коновалов$^1$, Р.\,В.~Разумчик$^{1,2}$}

\def\autkol{М.\,Г.~Коновалов, Р.\,В.~Разумчик}

{\renewcommand{\thefootnote}{\fnsymbol{footnote}} \footnotetext[1]
{Исследование выполнено при частичной финансовой поддержке РФФИ (проект 18-07-00692).}}



\titel{\tit}{\aut}{\autkol}{\titkol}

\vspace*{-11pt}

\noindent
$^1$Институт проблем информатики Федерального исследовательского 
центра <<Информатика и управление>>\linebreak
$\hphantom{^1}$Российской академии наук

\noindent
$^2$Российский университет дружбы народов 

\vspace*{5pt}

\def\leftfootline{\small{\textbf{\thepage}
\hfill ИНФОРМАТИКА И ЕЁ ПРИМЕНЕНИЯ\ \ \ том\ 12\ \ \ выпуск\ 4\ \ \ 2018}
}%
 \def\rightfootline{\small{ИНФОРМАТИКА И ЕЁ ПРИМЕНЕНИЯ\ \ \ том\ 12\ \ \ выпуск\ 4\ \ \ 2018
\hfill \textbf{\thepage}}}

\vspace*{-3pt}


\Abst{Представлены некоторые результаты численных экспериментов, подтверждающие
следующее обстоятельство: параметры механизма обобщенного обновления
могут быть подобраны таким образом,\linebreak\vspace*{-12pt}}

\Abstend{что уровень производительности
однолинейных сис\-тем массового обслуживания с обобщенным обновлением
может быть не ниже уровня производительности
систем с RED-по\-доб\-ны\-ми механизмами активного управ\-ле\-ния очередями.
Механизмы сравниваются на примере сис\-те\-мы $M/D/1/N$
по стационарным значениям сле\-ду\-ющих характеристик:
вероятность потери заявки, среднее число заявок в сис\-те\-ме,
среднее чис\-ло последовательных потерь заявок 
и~их средние квадратические отклонения.
Расчеты основаны на известных фактах,
а~также на ряде новых аналитических результатов для систем
с~обобщенным обновлением, полученных в данной работе.}

\KW{система массового обслуживания; 
алгоритмы активного управления очередями; обобщенное обновление}

\DOI{10.14357/19922264180402}



%\vspace*{-3pt}


 \begin{multicols}{2}

\renewcommand{\bibname}{\protect\rmfamily Литература}
%\renewcommand{\bibname}{\large\protect\rm References}

{\small\frenchspacing
{%\baselineskip=10.8pt
\begin{thebibliography}{99}
%\vspace*{-3pt}

\bibitem{RFC7567-1} %1
\Au{Baker F., Fairhurst~G.}
IETF recommendations regarding active queue management, 2015.
{\sf https://tools.\linebreak ietf.org/html/7567}.



\bibitem{Adams-1} %2
\Au{Adams R.}
Active queue management: A~survey~// 
{IEEE Commun. Surv. Tut.}, 2013. Vol.~15. No.\,3. P.~1425--1476.

\bibitem{Bonald-1} %3
\Au{Bonald T., May M., Bolot~J.\,C.} Analytic evaluation of RED performance~//
{IEEE Conference on Computer Communications Proceedings}, 2000. 
Vol.~3. P.~1415--1424.

\bibitem{hao-1} %4
\Au{Hao W., Wei~Y.}
An extended $GI^X/M/1/N$ queueing
model for evaluating the performance of AQM algorithms
with aggregate traffic~// Networking and mobile computing~/
Eds. Xicheng Lu and Wei Zhao.~---
Lecture notes in computer science ser.~--- Springer, 2005. Vol.~3619. P.~395--404.

\bibitem{Chyd-1} %5
\Au{Chydzi$\acute{\mbox{n}}$ski A., Chr$\acute{\mbox{o}}$st~L.} 
Analysis of AQM queues with queue size based packet
dropping~// Int. J.~Appl. Math. Comp., 2011. Vol.~21. No.\,3. P.~567--577.

\bibitem{Chyd2-1} %6
\Au{Chydzi$\acute{\mbox{n}}$ski A.,  Mrozowski~P.}
 Queues with dropping functions and general arrival
processes~// PLoS ONE, 2016. Vol.~11. No.\,3. 
{\sf https://journals.plos.org/plosone/\linebreak article?id=10.1371/journal.pone.0150702}.

\bibitem{oleg-1} %7
\Au{Tikhonenko O., Kempa~W.} Performance evaluation of an $M/G/n$-type queue
with bounded capacity and packet dropping~// {Int. J.~Appl.
Math. Comp.}, 2016. Vol.~26. No.~4. P.~841--854.



\bibitem{konnew-1} %8
\Au{Konovalov M.\,G., Razumchik~R.\,V.}
Numerical analysis of improved access restriction algorithms in a~$GI/G/1/N$
system // {J.~Commun. Technol. El.}, 2018. Vol.~63. No.\,6. P.~616--625.

\bibitem{Kreinin-1} %9
\Au{Kreinin A.\,Y.}
Queueing systems with renovation //
{J.~Appl. Math. Stochastic Analysis}, 1997. Vol.~10. No.~4. P.~431--441.

\bibitem{zarN2-1} %10
\Au{Zaryadov~I.\,S.} Queueing systems with general renovation~//
{Conference (International) on Ultra Modern Telecommunications Proceedings}, 2009.
P.~1--4.

\bibitem{Zaryadov2009-1} %11
\Au{Зарядов И.\,С.,  Печинкин~А.\,В.}
Стационарные временные характеристики системы ${GI/M/n/\infty}$
с~некоторыми вариантами дисциплины обобщенного об\-нов\-ле\-ния~//
{Автоматика и~телемеханика}, 2009. Вып.~12. С.~161--174.

\bibitem{Zaryadov2010-1} %12
\Au{Зарядов И.\,С.} 
Система массового обслуживания $GI/M/n/\infty$ с~обобщенным об\-нов\-ле\-ни\-ем~//
{Автоматика и~телемеханика}, 2010. Вып.~4. С.~130--139.

\bibitem{zarN1-1} %13
\Au{Korolkova A., Zaryadov~I.}
The mathematical model of the traffic transfer process with a~rate adjustable by {RED}~//
{Conference (International) on Ultra Modern Telecommunications Proceedings}, 2010.
P.~1046--1050.

\bibitem{RFC8289-1} %14
\Au{Nichols K., Jacobson V., McGregor A., Iyengar J.}
Controlled delay active queue management, 2018.
{\sf https:// datatracker.ietf.org/doc/rfc8289}.


\bibitem{Riordan1962-1} %15
\Au{Riordan J.} {Stochastic service systems}.~--- 
SIAM ser. in applied mathematics.~--- New York, NY, USA: Wiley, 1962. 139~p.

\bibitem{arxivRK-1} %16
\Au{Konovalov M.,  Razumchik~R.}
Queueing systems with renovation vs.\ queues with RED. Supplementary material~//
{ArXiv e-prints}, 2017. {\sf https://arxiv.\linebreak org/abs/1709.01477/}.

\bibitem{kulk-1} %17
\Au{Kulkarni V.\,G.} Modeling and analysis of stochastic systems. 3rd ed.~--- 
Chapman \&~Hall/CRC texts in statistical science ser.~---
Chapman \& Hall/CRC, 2016. 606~p.

\bibitem{kono1-1}
\Au{Коновалов М.\,Г.} 
{Методы адаптивной обработки информации и~их приложения.}~--- 
М.: ИПИ РАН, 2007. 212~с.
\end{thebibliography}
} }

\end{multicols}

 \label{end\stat}

 \vspace*{-3pt}

\hfill{\small\textit{Поступила в~редакцию  09.10.2018}}


%\renewcommand{\bibname}{\protect\rm Литература}
%\renewcommand{\figurename}{\protect\bf Рис.}
\renewcommand{\tablename}{\protect\bf Таблица}  %2
\def\stat{gorsh+korolev}

\def\tit{ОПРЕДЕЛЕНИЕ ЭКСТРЕМАЛЬНОСТИ ОБЪЕМОВ ОСАДКОВ НА~ОСНОВЕ МОДИФИЦИРОВАННОГО 
МЕТОДА ПРЕВЫШЕНИЯ ПОРОГОВОГО ЗНАЧЕНИЯ$^*$}

\def\titkol{Определение экстремальности объемов осадков на~основе %модифицированного 
метода превышения порогового значения}

\def\aut{А.\,К.~Горшенин$^1$, В.\,Ю.~Королев$^2$}

\def\autkol{А.\,К.~Горшенин, В.\,Ю.~Королев}

\titel{\tit}{\aut}{\autkol}{\titkol}

\index{Горшенин А.\,К.}
\index{Королев В.\,Ю.}
\index{Gorshenin A.\,K.}
\index{Korolev V.\,Yu.}




{\renewcommand{\thefootnote}{\fnsymbol{footnote}} \footnotetext[1]
{Работа выполнена при поддержке РФФИ (проект 17-07-00851) 
и~Стипендии Президента Российской Федерации молодым ученым и~аспирантам 
(СП-538.2018.5).}}


\renewcommand{\thefootnote}{\arabic{footnote}}
\footnotetext[1]{Институт проблем информатики Федерального исследовательского
центра <<Информатика и~управ\-ле\-ние>> Российской академии наук; факультет
вычислительной математики и~кибернетики Московского государственного университета 
им.\ М.\,В.~Ломоносова,
\mbox{agorshenin@frccsc.ru}}
\footnotetext[2]{Факультет вычислительной математики и~кибернетики
Московского государственного университета им.\ М.\,В.~Ломоносова;
Институт проб\-лем информатики Федерального исследовательского
центра <<Информатика и~управ\-ле\-ние>> Российской академии наук,
\mbox{vkorolev@cs.msu.ru}}


\vspace*{-8pt}

\Abst{Задача корректного определения того, какие наблюдения следует признавать 
экстремальными, чрезвычайно важна при изучении метеорологических явлений. 
Предложены восходящий и~нисходящий методы определения порогового (экстремального) 
уровня на основе использования теорем Реньи для редеющих потоков и~результатов 
Пи\-канд\-са\,--\,Бал\-ке\-мы\,--\,Де Хаана. На примере данных за~60~лет 
наблюдений в~Потсдаме и~Элисте продемонстрировано, что восходящий метод 
показывает отличные результаты для суточных объемов осадков, однако для 
дождливых периодов необходимо использовать нисходящий метод. 
Приведено сравнение результатов подобного непараметрического подхода с~параметрическим 
критерием, который был предложен авторами в~предшествующих работах.}

\KW{осадки; дождливые периоды; экстремальные наблюдения; пороговые значения; 
теорема Реньи; теорема Пи\-канд\-са\,--\,Бал\-ке\-мы\,--\,Де Хаана; 
проверка статистических гипотез; анализ данных}

\DOI{10.14357/19922264180403}
  
\vspace*{-4pt}


\vskip 10pt plus 9pt minus 6pt

\thispagestyle{headings}

\begin{multicols}{2}

\label{st\stat}

\section{Введение}

Оценки закономерностей и~тенденций в~наблюдениях аномально
экстремальных метеорологических явлений важны для понимания
процесса изменения климата. Однако известно, что различные методы
оценки экстремальных осадков в~применении к~разным моделям дают
существенно отличающиеся результаты. Чаще всего решение основано
на выборе порогового значения, определяемого как квантиль 
некоторого распределения~\cite{Groisman1999}. Однако изменения 
в~максимальных значениях объемов осадков (и,~соответственно, большее число 
превышений порога) не всегда
ведут к~качественному изменению того, какие именно события
действительно должны быть признаны экстремальными. В~частности,
увеличение доли осадков подобного рода может быть связано 
с~изменениями суммарного объема или с~увеличением интенсивности
осадков в~сочетании с~уменьшением числа дождливых дней~\cite{Zolina2008}.

Для решения задачи определения аномальных наблюдений в~климатологических
 задачах часто используется подход теории экстремальных значений, 
 называемый \verb"Peaks over Threshold" (\verb"PoT"; пики, превышающие 
 порог)~\cite{Leadbetter1991}. В~частности, известно использование подобного 
 подхода в~задачах, связанных со штормовыми волнами~\cite{Mendez2006}, 
 дневными температурами~\cite{Kysely2010}, 
 осадками~\cite{Begueria2006,Begueria2011,Roth2012}. 
 Для выявления кри\-ти\-че\-ских/экстре\-маль\-ных наблюдений в~информационных 
 потоках различной природы авторами данной статьи ранее была предложена 
 методология поиска порогового значения на основе модификации \verb"PoT"-ме\-то\-да 
 с~использованием результатов теоремы Реньи для редеющих потоков~\cite{Gorshenin2016a}. 
 В~настоящей работе подобный подход будет развит для определения аномально 
 экстремальных значений объемов осадков, в~том числе выпавших в~течение так 
 называемого дождливого периода (подряд идущих дней, в~которые наблюдались осадки). 
 Кроме того, приведено сравнение результатов указанного непараметрического 
 подхода с~параметрическим критерием, который был ранее предложен авторами 
 в~статье~\cite{Gorshenin2018as} для решения подобного класса задач.
 
 \begin{figure*}[b] %fig1
\vspace*{9pt}
 \begin{center}
 \mbox{%
 \epsfxsize=161.412mm 
 \epsfbox{gok-1.eps}
 }
 \end{center}
\vspace*{-6pt}
\Caption{Пороговый уровень, полученный для дневных объемов восходящим методом, 
Потсдам: (\textit{а})~распределение интервалов между превышениями порога;
(\textit{б})~распределение превышений порога; (\textit{в})~суточные объемы;
\textit{1}~--- гистограмма; \textit{2}~--- экспоненциальное распределение 
с~параметром~0,002;
\textit{3}~--- обобщенное распределение Парето с~параметрами~0,226,
9,196 и~30,2;
\textit{4}~--- данные; \textit{5}~--- порог~30,2~мм}\label{FigPotsdamPOT}
\end{figure*}

\vspace*{-18pt}

\section{Использование PoT-методологии для~суточных~объемов~осадков}
\label{DailyData}

\vspace*{-1pt}

Воспользуемся подходом на основе двух фундаментальных результатов~--- 
тео\-ре\-мы Реньи для редеющих потоков~\cite{Gnedenko1996} и~классического 
результата тео\-рии экстремальных значений, связанного с~именами Пикандса, 
Балкемы и~Де Хаана~\cite{Balkema1974,Pickands1975}, которые позволяют 
избегать априорных предположений о~данных~\cite{Gorshenin2017a}. 
Из указанных выше тео\-рем следует, что распределение разностей моментов 
превышения порогового значения должно соответствовать экспоненциальному закону, 
а~величины превышений данного порога~--- обобщенному распределению Парето, 
которое имеет следующий вид ($\xi\hm\in\mathbb{R}$~--- параметр формы,  
$\mu\hm\in\mathbb{R}$~--- сдвига,  $\sigma\hm>0$~--- масштаба):
\begin{equation*}
F_{\xi, \sigma, \mu}(y)=
\begin{cases}
1-\left(1+\fr{\xi(y-\mu)}{\sigma}\right)^{-1/\xi}\,,& \!\!\mbox{если }\xi\neq 0\,;\\
1-e^{-({y-\mu})/{\sigma}}& \!\!\mbox{иначе.}
\end{cases}
\end{equation*}

Таким образом, пороговое значение может быть определено в~рамках статистической 
процедуры, в~которой для каждого уровня, начиная с~некоторого значения, 
например минимума в~данных, с~заранее заданным шагом должны проверяться 
последовательно две гипотезы об экспоненциальности и~паретовости описанных 
выше объектов. В~случае\linebreak\vspace*{-12pt}

\columnbreak

\noindent
 принятия обеих текущий уровень может считаться 
экстремальным пороговым значением. Назовем данный метод, следуя 
статьям~\cite{Gorshenin2016a,Gorshenin2017a}, \textit{восходящим} 
(в~соответствии с~направлением сдвига порогового значения в~процессе анализа данных).

Продемонстрируем описанный подход на примере наблюдений за объемом дневных 
осадков за почти 60-летний период (1950--2008~гг.)\ 
в~городах Потсдаме и~Элисте. Процедура поиска автоматизирована с~помощью 
программного решения, созданного на встроенном языке программирования 
пакета \verb"MATLAB". Получено, что для Потсдама критический уровень 
составляет~30{,}2~мм (шаг изменения уровня~--- 0{,}01~мм, 
уровень значимости статистического критерия при проверке по  $\chi^2$-тес\-ту 
выбран равным~0{,}01). Для экспоненциального распределения 
$p_{\mathrm{знач}}\hm=0{,}07$ (параметр оценивается значением~0{,}002), 
для обобщенного Парето  $p_{\mathrm{знач}}\hm=0{,}29$ (параметры: 0{,}226, 
9{,}196 и~30{,}2).


На рис.~\ref{FigPotsdamPOT} продемонстрировано визуальное соответствие 
между экспериментальными данными и~подогнанными распределениями (см.\
 гистограммы и~аппроксимирующие кривые на рис.~1,\,\textit{а} и~1,\,\textit{б}).\linebreak\vspace*{-12pt}
 
 \pagebreak
 
 \end{multicols}
 
 \begin{figure*} %fig2
\vspace*{1pt}
 \begin{center}
 \mbox{%
 \epsfxsize=160.191mm 
 \epsfbox{gok-2.eps}
 }
 \end{center}
\vspace*{-9pt}
\Caption{Пороговый уровень, полученный для дневных объемов восходящим методом, 
Элиста:
(\textit{а})~распределение интервалов между превышениями порога;
(\textit{б})~распределение превышений порога; (\textit{в})~суточные объемы;
\textit{1}~--- гистограмма; \textit{2}~--- экспоненциальное распределение 
с~параметром~0,002;
\textit{3}~--- обобщенное распределение Парето 
с~параметрами~0,038, 9,009 и~26,5;
\textit{4}~--- данные; \textit{5}~--- порог 26,5~мм
}\label{FigElistaPOT}
%\vspace*{3pt}
\end{figure*}

 
 \begin{multicols}{2}
 
 \noindent 
 На рис.~1,\,\textit{в} продемонстрировано, что полученное
 пороговое значение превышается относительно небольшое число раз за весь 
 период наблюдений; таким образом, данные пики могут  рассматриваться как 
 <<подозрительные на экстремальность>>.





Для Элисты критический уровень составляет~26{,}5~мм. 
Для экспоненциального распределения $p_{\mathrm{знач}}\hm=0{,}84$ 
(параметр оценивается значением~0{,}002), для обобщенного Парето 
 $p_{\mathrm{знач}}\hm=0{,}6$\linebreak
  (параметры: 0{,}038, 9{,}009 и~36{,}5). 
 На рис.~\ref{FigElistaPOT} отоб\-ра\-же\-но соответствие между экспериментальными 
 данными и~подогнанными распределениями (рис.~2,\,\textit{а} и~2,\,\textit{б}). 
 На рис.~2,\,\textit{в} продемонстрировано, что полученное 
 пороговое значение снова превышается относительно небольшое число раз за весь 
 период наблюдений.
 
 %\vspace*{-12pt}

\section{Использование PoT-методологии для~объемов осадков за~дождливые периоды}

Отметим, что описанный в~разд.~\ref{DailyData} восходящий метод может быть 
реализован и~в~обратном на\-прав\-ле\-нии (с~точки зрения смещения уровня в~процессе 
анализа данных). Для этого необходимо задать верхнюю границу (например, совпадающую 
с~максимумом наблюдений) и~последовательно проверять гипотезы об 
экспоненциальности и~па\-ре\-то\-вости. 

В~начале работы метода число превышений поро\-га 
будет недостаточным для построения гис\-то\-грамм и~корректного оценивания параметров, 
поэтому необходимо обеспечить минимально приемлемый объем соответствующей выборки. 

В~работе~\cite{Gorshenin2016a} описана схожая процедура, однако в~качестве 
примера рассмотрены кумулятивные данные. 
В~данном разделе откажемся от ограничения на тип рассматриваемых данных и~будем 
считать \textit{нисходящим} методом описанный выше алгоритм. 

В~качестве примера рассмотрим суммарные объемы осадков, зарегистрированные в~течение 
дождливого периода, однако данный метод может быть использован и~для исходных 
суточных на\-блю\-де\-ний.
{\looseness=1

}


Для объемов осадков за дождливые периоды в~Потсдаме верхний критический уровень 
(полу-\linebreak\vspace*{-12pt}

\pagebreak

\end{multicols}

\begin{figure*} %fig3
\vspace*{1pt}
 \begin{center}
 \mbox{%
 \epsfxsize=157.261mm 
 \epsfbox{gok-3.eps}
 }
 \end{center}
\vspace*{-9pt}
\Caption{Нисходящий, восходящий и~параметрический методы определения экстремальности:
 (\textit{а})~распределение интервалов между превышениями порога;
(\textit{б})~распределение превышений порога; (\textit{в})~объемы осадков за 
дождливые периоды, Потсдам; \textit{1}~--- гистограмма; 
\textit{2}~--- экспоненциальное распределение с~параметром~0,216;
\textit{3}~--- обобщенное распределение Парето 
с~параметрами~0,076,
13,406 и~14,41;
\textit{4}~--- данные; \textit{5}~--- порог~14,41~мм;
\textit{6}~--- порог~57,2~мм; \textit{7}~--- абсолютно экстремальные значения;
\textit{8}~--- промежуточные; \textit{9}~--- относительно экстремальные значения}
\label{FigPotsdamVolWet}
\vspace*{-6pt}
\end{figure*}

\begin{multicols}{2}

\noindent
чен нисходящим методом) составляет~57{,}2~мм (шаг
 изменения уровня~--- 0{,}01~мм, 
уровень зна\-чи\-мости
 статистического критерия при проверке по  $\chi^2$-тес\-ту 
выбран равным~0{,}01). Для экспоненциального\linebreak распределения $p_{\mathrm{знач}}\hm=0{,}1$ 
(параметр оценивается значением~0{,}014), для обобщенного Парето 
$p_{\mathrm{знач}}\hm=0{,}03$ (параметры: 0{,}097, 16{,}95 и~57{,}2). 
Нижний критический уровень (получен восходящим методом) составляет~14{,}41~мм. 
Для экспоненциального распределения $p_{\mathrm{знач}}\hm=0{,}058$ 
(параметр оценивается значением~0{,}216), для обобщенного Парето 
$p_{\mathrm{знач}}\hm=0{,}29$ (параметры: 0{,}076, 13{,}406 и~14{,}41).
{\looseness=1

}

На рис.~\ref{FigPotsdamVolWet} продемонстрировано визуальное соответствие 
между экспериментальными данными и~подогнанными распределениями (см.\
 гистограммы и~аппроксимирующие кривые на рис.~3,\,\textit{а}
 и~3,\,\textit{б}) для восходящего 
 метода. Рисунок~3,\,\textit{в} демонстрирует разницу пороговых значений, 
 полученных двумя способами. Очевидно, что в~отличие от случая разд.~\ref{DailyData} 
 (исходные данные) восходящий метод устанавливает критическую планку слишком низко 
 и~в~рассмотрении окажется избыточное количество пиков.

Для объемов осадков за дождливые периоды в~Элисте верхний критический уровень 
(получен нисходящим методом) составляет~28~мм (шаг изменения уровня~--- 0{,}01~мм, 
уровень значимости статистического критерия при проверке по  $\chi^2$-тес\-ту 
выбран равным~0{,}01). Для экспоненциального 
распределения $p_{\mathrm{знач}}\hm=0{,}082$ 
(параметр оценивается значением~0{,}029), для обобщенного 
Парето $p_{\mathrm{знач}}\hm=0{,}44$ 
(параметры: $-$0{,}095, 13{,}66 и~28). Нижний критический уровень 
(получен восходящим методом) составляет~10{,}71~мм.
 Для экспоненциального распределения $p_{\mathrm{знач}}\hm=0{,}062$ (параметр
  оценивается значением~0{,}183), для обобщенного Парето $p_{\mathrm{знач}}\hm=0{,}21$ 
  (параметры:  0{,}066, 9{,}586 и~10{,}71).



На рис.~\ref{FigElistaVolWet} продемонстрировано визуальное соответствие между 
экспериментальными данными и~подогнанными распределениями (см.\
 гистограммы и~аппроксимирующие кривые на рис.~4,\,\textit{а}
 и~4,\,\textit{б}) для восходящего метода. 
 Рисунок~4,\,\textit{в} демонстрирует разницу пороговых значений, 
 полученных двумя
 способами. Снова восходящий метод устанавливает критическую 
 планку слишком низко, и~более интересным представляется рассмотрение уровня 
 для нисходящего метода.
 
 \pagebreak
 
 \end{multicols}
 
 \begin{figure*} %fig4
\vspace*{1pt}
 \begin{center}
 \mbox{%
 \epsfxsize=157.761mm 
 \epsfbox{gok-4.eps}
 }
 \end{center}
\vspace*{-9pt}
\Caption{Нисходящий, восходящий и~параметрический методы определения экстремальности:
(\textit{а})~распределение интервалов между превышениями порога;
(\textit{б})~распределение превышений порога; 
(\textit{в})~объемы осадков за дождливые периоды, Элиста;
\textit{1}~--- гистограмма; 
\textit{2}~--- экспоненциальное распределение с~па\-ра\-мет\-ром~0,183;
\textit{3}~--- обобщенное распределение Парето с~па\-ра\-мет\-ра\-ми~0,066,
9,586 и~10,71;
\textit{4}~--- данные; \textit{5}~--- порог~10,71~мм;
\textit{6}~--- порог~28~мм; \textit{7}~--- абсолютно экстремальные значения;
\textit{8}~--- промежуточные; \textit{9}~--- относительно 
экстремальные значения}
\label{FigElistaVolWet}
%\vspace*{9pt}
\end{figure*}

\begin{multicols}{2}



\begin{figure*}[b] %fig5
\vspace*{4pt}
 \begin{center}
 \mbox{%
 \epsfxsize=106.504mm 
 \epsfbox{gok-5.eps}
 }
 \end{center}
\vspace*{-9pt}

\Caption{Алгоритм определения экстремальных значений}\label{FigMethods}
\end{figure*}


\section{Статистический тест экстремальности объемов, основанный 
на~распределении Снедекора--Фишера}
\label{StatExtremes}

На рис.~\ref{FigPotsdamVolWet},\,\textit{в} и~\ref{FigElistaVolWet},\,\textit{в} 
нанесены дополнительные маркеры (треугольники, круги и~квадраты), 
которыми отмечены экстремальные наблюдения, полученные в~соответствии 
со статистическим критерием, описанным в~работе~\cite{Gorshenin2018as}. 

Для перехода к~естественной временной шкале на со\-от\-вет\-ст\-ву\-ющие графики 
наносятся пики, по величине совпадающие с~объемами осадков, выпавших за 
дождливые периоды, а~их расположение на временн$\acute{\mbox{о}}$й шкале выбрано совпадающим с~днем 
начала выпадения осадков. Используемый для разметки параметрический критерий 
существенным образом использует установленные 
в~статьях~\cite{Gorshenin2017a,Gorshenin2017b,Gorshenin2017c,Gorshenin2017d,Gorshenin2018bs,Gorshenin2018c} 
факты о~соответствии распределений длительностей дождливых периодов отрицательному 
биномиальному распределению, а~их объемов~--- гам\-ма-рас\-пре\-де\-ле\-нию. 
%
Ниже опишем соответствующую процедуру, предложенную в~работе~\cite{Gorshenin2018as}.

Пусть $m\in\mathbb{N}$ и~$G^{(1)}_{r,\mu},G^{(2)}_{r,\mu},\ldots,G^{(m)}_{r,\mu}$~--- 
независимые случайные величины с~общим гам\-ма-рас\-пре\-де\-ле\-ни\-ем 
с~параметрами $r\hm>0$ и~$\mu\hm>0$. Рассмотрим статистику
\begin{equation*}
R_0=\fr{(m-1)G^{(1)}_{r,\mu}}{G^{(2)}_{r,\mu}+
\cdots+G^{(m)}_{r,\mu}}\eqd
\fr{k}{r}\,\fr{G_{r,\mu}}{G_{k,\mu}}\eqd\fr{k}{r}\,\fr{G_{r,1}}{G_{k,1}}\eqd
Q_{r,k}\,,
\end{equation*}
где $k=(m-1)r$; $Q_{k,r}$~---  случайная величина с~распределении 
Сне\-де\-ко\-ра--Фи\-ше\-ра с~параметрами $k\hm>0$ и~$r\hm>0$. Предположим, что 
$V_1,\ldots,V_m$~--- суммарные объемы осадков, выпавших за~$m$~дождли\-вых периодов. 
Рассмотрим следующий выборочный аналог величины~$R_0$:
\begin{equation*}
SR_{0,i}=\fr{(m-1)V_i}{\sum\nolimits_{j\neq i}V_j}\,.
\end{equation*}



Сформулируем гипотезу~$H_0$ в~следующем виде: <<объем осадков~$V_i$ 
не является аномально большим относительно остальных $m\hm-1$ величин>>.
 Тогда статистика~$SR_{0,i}$ имеет распределение Сне\-де\-ко\-ра--Фи\-ше\-ра 
 с~параметрами $r\hm>0$ и~$k\hm=(m\hm-1)r$. Обозначим через $q_{r,k}(1\hm-\alpha)$ 
 квантиль распределения Сне\-де\-ко\-ра--Фи\-ше\-ра уровня $(1\hm-\alpha)$, 
 $\alpha\hm\in(0,1)$. В~случае, если $SR_{0,i}\hm>q_{r,k}(1\hm-\alpha)$, гипотеза~$H_0$ 
 отвергается и~объем~$V_i$ должен быть признан экстремально большим. 
 Уровень значимости критерия установлен на уровне~$\alpha$.

Описанная процедура может быть расширена с~помощью метода скользящего окна. 
Задавая его ширину (т.\,е.\ число элементов в~выборке объемов дождливых периодов) 
равной~$m$ и~сдвигая каждый\linebreak
 раз на один элемент вправо (направление астрономического 
времени на графике с~экспериментальными данными), можно последовательно проверить, 
является ли каждый конкретный объем экстремаль\-ным относительно других в~смысле 
описанного выше критерия. Данная процедура может быть полезна в~ситуации, 
когда в~одно окно попадают достаточно близкие по абсолютной величине объемы, 
мало отличающиеся от максимального. 

Таким образом, в~рамках данного подхода 
каж\-дый элемент (начиная с~$m$) проверяется в~точ\-ности~$m$~раз. Тогда возможны 
сле\-ду\-ющие вари-\linebreak анты:\\[-15pt]
\begin{enumerate}[(1)]
\item элемент признается аномальным во всех~$m$~случаях 
и~соответствующее значение может считаться \textit{абсолютно} экстремальным;
\item 
элемент признается аномальным более чем в~половине случаев (т.\,е.\ 
не меньше чем на $\lceil  m/2\rceil$ положениях окна)  и~соответствующее значение 
может считаться \textit{промежуточным} экстремумом; 
\item элемент признается 
аномальным менее чем в~половине случаев и~соответствующее значение может 
считаться \textit{относительно} экстремальным.
\end{enumerate}

 Для приведения к~астрономическому 
времени необходимо найти среднюю продолжительность дождливых и~сухих периодов 
(используется модель на основе отрицательного биномиального распределения) и~умножить 
на размер окна. На рис.~\ref{FigPotsdamVolWet} и~\ref{FigElistaVolWet} 
расчеты проводились для периода в~360~дней (при этом $m\hm=62$).

\vspace*{-6pt}

\section{Алгоритм определения экстремальности значений~объемов}

Таким образом, алгоритм определения экстремальных значений 
(и~сравнения результатов, полученных разными способами) может быть пред\-став\-лен в~виде 
 блок-схе\-мы (рис.~\ref{FigMethods}).



К данным $\mathrm{Data}$ последовательно применяются восходящий, 
нисходящий и~параметрический методы поиска экстремальных значений. 
Для первых двух из них задается начальное значение ($\mathrm{level}\hm=0$ для нисходящего 
и~$\mathrm{LEVEL}\hm=\max(\mathrm{Data})$ для восходящего). Далее последовательно проверяются 
условия для теорем Реньи и~Пи\-канд\-са\,--\,Бал\-ке\-мы\,--\,Де Хаана при 
некотором заданном уровне зна\-чи\-мости~$\alpha$. Процедура \verb"StatExtremes()" 
реализует метод, описанный в~разд.~\ref{StatExtremes}. 
В~результате можно выявить наблюдения, которые признаются экстремальными с~помощью 
каждого из этих методов, как продемонстрировано на рис.~\ref{FigPotsdamVolWet} 
и~\ref{FigElistaVolWet}. Стоит отметить, что восходящий и~нисходящий методы 
могут быть использованы для любых данных, а~па\-ра\-мет\-ри\-че\-ский подход ориентирован 
на ряд дополнительных предположений о~распределениях характеристик наблюдений 
(а~также неотрицательность исходных значений).

%\vspace*{-6pt}

\section{Заключение}

\vspace*{-1pt}

Итак, в~работе рассмотрены восходящий и~нисходящий методы определения 
порогового (экстремального) уровня в~рамках идеологии \verb"PoT". 
Восходящий метод показывает отличные результаты для суточных объемов осадков, 
однако для дождливых периодов необходимо использовать нисходящий метод. 
Кроме того, предложено сравнение результатов со статистическим критерием, 
который построен в~рамках определенной (при этом весьма хорошо 
согласующейся с~реальными данными) вероятностной модели. 
Указанный метод позволяет улучшить полученные результаты, однако требует ряда 
априорных предположений о~распределениях характеристик исследуемых данных.

\vspace*{-6pt}

{\small\frenchspacing
 {%\baselineskip=10.8pt
 \addcontentsline{toc}{section}{References}
 \begin{thebibliography}{99}
 
% \vspace*{-1pt}
 
\bibitem{Groisman1999} 
\Au{Groisman~P.\,Y., Karl~T.\,R., Easterling~D.\,R., \textit{et al.}} Changes
in the probability of heavy precipitation: Important indicators of climatic change~// 
J.~Climate, 1999. Vol.~42. P.~243--285.

\bibitem{Zolina2008} 
\Au{Zolina~O., Simmer~C., Kapala~A., Bachner~S., Gulev~S., Maechel~H.} 
Seasonally dependent changes of precipitation extremes over Germany since~1950 
from a~very dense observational network~// J.~Geophys. Res., 2008. Vol.~113. Art. No.\,D06110.

\bibitem{Leadbetter1991} 
\Au{Leadbetter~M.\,R.} On a basis for ``Peaks over Threshold'' modeling~// 
Stat. Probabil. Lett., 1991. Vol.~12. Iss.~4. P.~357--362.

\bibitem{Mendez2006} 
\Au{Mendez~F.\,J., Menendez~M., Luceno~A., Losada~I.\,J.} 
Estimation of the long-term variability of extreme significant wave height 
using a~time-dependent Peak over Threshold (PoT) model~// 
J.~Geophys. Res. Oceans, 2006. Vol.~111. Iss.~C7. Art.\ No.\,C07024.

\bibitem{Kysely2010}
 \Au{Kysely~J., Picek~J., Beranova~R.} 
Estimating extremes in climate change simulations using the 
peaks-over-threshold method with a non-stationary threshold~// 
Global Planet. Change, 2010. Vol.~72. Iss.~1-2. P.~55--68.

\bibitem{Begueria2006} 
\Au{Begueria~S., Vicente-Serrano~S.\,M.} 
Mapping the hazard of extreme rainfall by peaks over threshold 
extreme value analysis and spatial regression techniques~// 
J.~Appl. Meteorol. Clim., 2006. Vol.~45. Iss.~1. P.~108--124.

\bibitem{Begueria2011} 
\Au{Begueria~S., Angulo-Martinez~M., Vicente-Serrano~S.\,M., 
Lopez-Moreno~I.\,J., El-Kenawy~A.} Assessing trends in extreme precipitation 
events intensity and magnitude using non-stationary peaks-over-threshold analysis: 
A~case study in northeast Spain from 1930 to 2006~// 
Int. J.~Climatol., 2011. Vol.~31. Iss.~142. P.~2102--2114.

\bibitem{Roth2012} 
\Au{Roth~M., Buishand~T.\,A., Jongbloed~G.,  Tank~A.\,M.\,G., 
van Zanten~J.\,H.} A~regional peaks-over-threshold model in a~nonstationary climate~// 
Water Resour. Res., 2012. Vol.~48. Art. No.\,W11533.

\bibitem{Gorshenin2016a} 
\Au{Gorshenin~A.\,K., Korolev~V.\,Yu.} 
A~methodology for the identification of extremal loading in data flows in 
information systems~// Comm. Com. Inf. Sc., 2016. Vol.~638. P.~94--103.

\bibitem{Gorshenin2018as} 
\Au{Korolev~V.\,Yu., Gorshenin~A.\,K., Belyaev~K.\,P.} 
Statistical tests for extreme precipitation volumes~// \mbox{ArXiv}:\linebreak
1802.02928v3 [stat.ME],  2018.

\bibitem{Gnedenko1996} 
\Au{Gnedenko~B.\,V., Korolev~V.\,Yu.} 
Random summation: Limit theorems and applications.~--- 
Boca Raton, FL, USA: CRC Press, 1996. 288~p.

\bibitem{Balkema1974} 
\Au{Balkema~A., de Haan~L.} 
Residual life time at great age~// Annals Probability, 1974. Vol.~2. No.\,5.
P.~792--804.

\bibitem{Pickands1975} 
\Au{Pickands~J.} 
Statistical inference using extreme order statistics~// 
Ann. Stat., 1975. Vol.~3. No.\,1. P.~119--131.

\bibitem{Gorshenin2017a} 
\Au{Горшенин~А.\,К.} О~некоторых математических
и программных методах построения структурных моделей информационных
потоков~// Информатика и~её применения, 2017. Т.~11. Вып.~1. C.~58--68.

\bibitem{Gorshenin2017b}  
\Au{Горшенин~А.\,К.} 
Анализ ве\-ро\-ят\-но\-ст\-но-ста\-ти\-сти\-че\-ских характеристик осадков на 
основе паттернов~// Информатика и~её применения, 2017. Т.~11. Вып.~4. C.~38--46.

\bibitem{Gorshenin2017c} 
\Au{Korolev~V.\,Yu., Gorshenin~A.\,K., Gulev~S.\,K.,
Belyaev~K.\,P., Grusho~A.\,A.} Statistical analysis of precipitation events~// AIP
Conf. Proc., 2017. Vol.~1863. P.~\mbox{090011-1}--\mbox{090011-4}.

\bibitem{Gorshenin2017d} 
\Au{Королев~В.\,Ю., Горшенин~А.\,К.} 
О~распределении вероятностей экстремальных осадков~// Докл. РАН,
2017. Т.~477. Вып.~5. C.~604--609.

\bibitem{Gorshenin2018bs} 
\Au{Gorshenin~A.\,K., Kuzmin~V.\,Yu.} Neural
network forecasting of precipitation volumes using patterns~// Pattern 
Recogn. Image
Anal., 2018. Vol.~28. No.\,3. P.~450--461.

\bibitem{Gorshenin2018c}
\Au{Gorshenin~A.\,K., Korolev~V.\,Yu.} Scale mixtures of
Frechet distributions as asymptotic approximations of extreme precipitation~// 
J.~Math. Sci., 2018. Vol.~234. No.\,6. P.~886--903.
 \end{thebibliography}

 }
 }

\end{multicols}

\vspace*{-11pt}

\hfill{\small\textit{Поступила в~редакцию 15.10.18}}

\vspace*{5pt}

%\pagebreak

%\newpage

%\vspace*{-28pt}

\hrule

\vspace*{2pt}

\hrule

\vspace*{-4pt}

\def\tit{DETERMINING THE~EXTREMES OF~PRECIPITATION VOLUMES BASED ON~THE~MODIFIED 
``PEAKS OVER THRESHOLD'' METHOD}

\def\titkol{Determining the~extremes of~precipitation volumes based on~the~modified 
``Peaks over Threshold'' method}


\def\aut{A.\,K.~Gorshenin$^{1,2}$ and V.\,Yu.~Korolev$^{1,2}$}

\def\autkol{A.\,K.~Gorshenin and V.\,Yu.~Korolev}

\titel{\tit}{\aut}{\autkol}{\titkol}

\vspace*{-11pt}




\noindent
$^1$Institute of Informatics Problems, Federal Research Center 
``Computer Science and Control'' of the Russian\linebreak
$\hphantom{^1}$Academy of Sciences,  44-2~Vavilov Str., Moscow 119333, Russian Federation

\noindent
$^2$Faculty of Computational Mathematics and Cybernetics, M.\,V.~Lomonosov Moscow
State University,  Leninskie\linebreak
$\hphantom{^1}$Gory,  Moscow 119991, GSP-1, Russian Federation

\def\leftfootline{\small{\textbf{\thepage}
\hfill INFORMATIKA I EE PRIMENENIYA~--- INFORMATICS AND
APPLICATIONS\ \ \ 2018\ \ \ volume~12\ \ \ issue\ 4}
}%
 \def\rightfootline{\small{INFORMATIKA I EE PRIMENENIYA~---
INFORMATICS AND APPLICATIONS\ \ \ 2018\ \ \ volume~12\ \ \ issue\ 4
\hfill \textbf{\thepage}}}

\vspace*{2pt}



\Abste{The problem of correct determination of extreme observations is 
very important when studying meteorological phenomena. The paper proposes 
ascending and descending methods for finding the threshold for extremes 
based on the R$\acute{\mbox{e}}$nyi theorem for thinning flows and the 
Pikands\,--\,Balkema\,--\,De Haan results. Using the observation data for~60~years 
for Potsdam and Elista, it is demonstrated that the ascending method can 
present excellent results for daily precipitation but for volumes 
of wet periods, the descending method should be used. The results of 
such nonparametric approaches are compared with the parametric criterion 
proposed in the previous papers by the authors.}


\KWE{precipitation; wet periods; extreme values; thresholds; R$\acute{\mbox{e}}$nyi theorem; 
Pickands\,--\,Balkema\,--\,de Haan theorem; testing statistical hypotheses;
 data analysis}


\DOI{10.14357/19922264180403}

\vspace*{-18pt}

\Ack
\noindent
The research is supported by the Russian Foundation 
for Basic Research (project~17-07-00851) and the 
RF Presidential scholarship program (No.\,538.2018.5).


%\vspace*{6pt}

  \begin{multicols}{2}

\renewcommand{\bibname}{\protect\rmfamily References}
%\renewcommand{\bibname}{\large\protect\rm References}

{\small\frenchspacing
 {%\baselineskip=10.8pt
 \addcontentsline{toc}{section}{References}
 \begin{thebibliography}{99}
\bibitem{1-gorkor}
\Aue{Groisman,~P.\,Y., T.\,R.~Karl, D.\,R.~Easterling, \textit{et al.}} 1999. Changes
in the probability of heavy precipitation: Important indicators of climatic change. 
\textit{J.~Climate} 42:243--285.

\bibitem{2-gorkor}
\Aue{Zolina,~O., C.~Simmer, A.~Kapala, S.~Bachner, S.~Gulev, and H.~Maechel.} 
2008. Seasonally dependent changes of precipitation extremes over Germany 
since 1950 from a~very dense observational network. 
\textit{J.~Geophys. Res.} 113: D06110.

\bibitem{3-gorkor}
\Aue{Leadbetter, M.\,R.} 1991. 
On a~basis for ``Peaks over Threshold'' modeling. 
\textit{Stat. Probabil. Lett.} 12(4):357--362.

\bibitem{4-gorkor}
\Aue{Mendez,~F.\,J., M.~Menendez, A.~Luceno, and I.\,J.~Losada.}
 2006. Estimation of the long-term variability of extreme significant wave 
 height using a time-dependent Peak over Threshold (PoT) model. \textit{J.~Geophys. 
 Res. Oceans} 111(C7):C07024.

\bibitem{5-gorkor}
\Aue{Kysely,~J., J.~Picek, and R.~Beranova.} 2010. 
Estimating extremes in climate change simulations using the peaks-over-threshold 
method with a non-stationary threshold. 
\textit{Global Planet. Change} 72(1--2):55--68.

\bibitem{6-gorkor}
\Aue{Begueria,~S., and S.\,M.~Vicente-Serrano.} 2006. Mapping the hazard 
of extreme rainfall by peaks over threshold extreme value analysis and 
spatial regression techniques. \textit{J.~Appl. Meteorol. Clim.} 
45(1):108--124.

\bibitem{7-gorkor}
\Aue{Begueria,~S., M.~Angulo-Martinez, S.\,M.~Vicente-Serrano, 
I.\,J.~Lopez-Moreno, and A.~El-Kenawy.} 2011. Assessing trends 
in extreme precipitation events intensity and magnitude using non-stationary 
peaks-over-threshold analysis: A~case study in northeast Spain from 1930 to~2006. 
\textit{Int. J.~Climatol.} 31(142):2102--2114.

\bibitem{8-gorkor}
\Aue{Roth,~M., T.\,A.~Buishand, G.~Jongbloed,  A.\,M.\,G.~Tank, and J.\,H.~van Zanten.}
 2012. A~regional peaks-over-threshold model in a~nonstationary climate. 
 \textit{Water Resour. Res.} 48:W11533.

\bibitem{9-gorkor}
\Aue{Gorshenin,~A.\,K., and V.\,Yu.~Korolev.} 2016. 
A~methodology for the identification of extremal loading in data flows in 
information systems. \textit{Comm. Com. Inf. Sc.} 638:94--103.

\bibitem{10-gorkor}
\Aue{Korolev,~V.\,Yu., A.\,K.~Gorshenin, and K.\,P.~Belyaev.} 
2018. Statistical tests for extreme precipitation volumes. 
1802.02928 [stat.ME]. Available at:
{\sf https://arxiv.org/\linebreak abs/1802.02928v3} (accessed December~3, 2018).

\bibitem{11-gorkor}
\Aue{Gnedenko,~B.\,V., and V.\,Yu.~Korolev.} 1996. \textit{Random summation: Limit 
theorems and applications}. Boca Raton, FL: CRC Press. 288~p.

\bibitem{12-gorkor}
\Aue{Balkema,~A., and L.~de Haan.} 1974. Residual life time at great age. 
\textit{Ann. Probab.} 2(5):792--804.

\bibitem{13-gorkor}
\Aue{Pickands,~J.} 1975. Statistical inference using extreme order statistics. 
\textit{Ann. Stat.} 3(1):119--131.

\bibitem{14-gorkor}
\Aue{Gorshenin,~A.\,K.} 2017. O~nekotorykh matematicheskikh i~programmnykh metodakh
postroeniya strukturnykh modeley informatsionnykh potokov [On some mathematical and
programming methods for construction of structural models of information flows]. 
\textit{Informatika i~ee Primeneniya ~--- Inform. Appl.} 11(1):58--68.

\bibitem{15-gorkor}
\Aue{Gorshenin,~A.\,K.} 2017. 
Analiz veroyatnostno-statisticheskikh kharakteristik osadkov 
na osnove patternov [Pattern-based analysis of probabilistic and 
statistical characteristics of precipitations]. 
\textit{Informatika i~ee Primeneniya~--- Inform. Appl.} 11(4):38--46.

\bibitem{16-gorkor}
\Aue{Korolev,~V.\,Yu., A.\,K.~Gorshenin, S.\,K.~Gulev, K.\,P.~Belyaev, and A.\,A.~Grusho.}
2017. Statistical analysis of precipitation events. \textit{AIP Conf. Proc.} 
1863:\mbox{090011-1}--\mbox{090011-4}.

\bibitem{17-gorkor}
\Aue{Korolev,~V.\,Yu., and A.\,K.~Gorshenin.} 2017. 
The probability distribution of extreme precipitation. \textit{Dokl. Earth Sci.} 
477(2):1461--1466.

\bibitem{18-gorkor}
\Aue{Gorshenin,~A.\,K., and V.\,Yu.~Kuzmin.}
 2018. Neural network forecasting of precipitation volumes using patterns. 
 \textit{Pattern Recogn. Image Anal.} 28(3):450--461.

\bibitem{19-gorkor}
\Aue{Gorshenin,~A.\,K., and V.\,Yu.~Korolev.} 2018. Scale mixtures of
Frechet distributions as asymptotic approximations of extreme precipitation. 
\textit{J.~Math. Sci.} 234(6):886--903.
\end{thebibliography}

 }
 }

\end{multicols}

\vspace*{-6pt}

\hfill{\small\textit{Received October 15, 2018}}

%\pagebreak

%\vspace*{-18pt}



\Contr

\noindent
\textbf{Gorshenin Andrey K.} (b.\ 1986)~--- Candidate of Science (PhD) in physics and
mathematics, associate professor, leading scientist, Institute of Informatics Problems,
Federal Research Center ``Computer Science and Control'' of the Russian Academy of
Sciences, 44-2~Vavilov Str., Moscow 119333, Russian Federation;  leading scientist, 
Faculty
of Computational Mathematics and Cybernetics, M.\,V.~Lomonosov Moscow State University, 
 Leninskie Gory,  Moscow 119991,  GSP-1, Russian Federation; \mbox{agorshenin@frccsc.ru}

\vspace*{3pt}

\noindent
\textbf{Korolev Victor Yu.} (b.\ 1954)~--- Doctor of Science (PhD) in physics and
mathematics, professor, Head of Department, 
Faculty of Computational Mathematics and Cybernetics, 
M.\,V.~Lomonosov Moscow State University,  Leninskie Gory,  Moscow 119991,  GSP-1,
Russian Federation; leading scientist, Institute of Informatics Problems, 
Federal Research Center ``Computer Science and Control'' of the 
Russian Academy of Sciences, 44-2~Vavilov Str., Moscow 119333, Russian Federation; 
professor, Hangzhou Dianzi University, Xiasha Higher Education Zone, Hangzhou 310018, 
China; \mbox{vkorolev@cs.msu.ru}
\label{end\stat}

\renewcommand{\bibname}{\protect\rm Литература}      %3
\def\stat{agalarov}


\def\tit{ПРИБЛИЖЕННЫЙ МЕТОД ВЫЧИСЛЕНИЯ ХАРАКТЕРИСТИК УЗЛА 
ТЕЛЕКОММУНИКАЦИОННОЙ СЕТИ С~ПОВТОРНЫМИ ПЕРЕДАЧАМИ}
\def\titkol{Приближенный метод вычисления характеристик узла 
телекоммуникационной сети с~повторными передачами} 

\def\autkol{Я.\,М.~Агаларов}
\def\aut{Я.\,М.~Агаларов$^1$}

\titel{\tit}{\aut}{\autkol}{\titkol}

%{\renewcommand{\thefootnote}{\fnsymbol{footnote}}\footnotetext[1]
%{Работа выполнена при поддержке РФФИ, проекты 08--07--00152 и 08--01--00567.}}

\renewcommand{\thefootnote}{\arabic{footnote}}
\footnotetext[1]{Институт проблем
информатики Российской академии наук, agglar@yandex.ru}

%\vspace*{-6pt}


\Abst{Рассмотрена модель узла коммутации пакетов c повторными передачами для двух 
схем распределения буферной памяти: полнодоступной и полного разделения. Предложен 
приближенный метод вычисления интенсивностей потоков и вероятностей блокировок узла. 
Получены необходимые и достаточные условия существования и единственности решения 
уравнения для потоков в узле при установившемся режиме работы и доказана сходимость 
итерационного метода решения указанного уравнения.}

\KW{узел коммутации пакетов; буферная память; повторные передачи; вероятности 
блокировок; итерационный метод}

      \vskip 18pt plus 9pt minus 6pt

      \thispagestyle{headings}

      \begin{multicols}{2}

      \label{st\stat}


\section{Введение}

    Одной из основных задач предварительного анализа 
телекоммуникационных сетей коммутации пакетов с ограниченной буферной 
памятью является расчет характеристик потоков и вероятностей блокировок в 
узлах связи. Важность указанных характеристик определяется тем, что от их 
значений существенным образом зависят другие основные показатели сети 
(пропускная способность, задержки пакетов и~др.). 

    Существует множество различных моделей узлов коммутации пакетов и 
методов их расчета (см., например,~[1--6]). Для моделей, рассматривающих 
узел с ограниченной буферной памятью как систему массового обслуживания 
(CMO) типа 
$
\begin{matrix}
M \\ \lambda
\end{matrix}
\left |
\begin{matrix}
M \\ \lambda
\end{matrix}
\right |
\overline{m} \vert N
$ или  $\vert PH\vert PH\vert 1\vert r$, в предположении отсутствия повторных 
передач пакетов получены точные методы вычисления характеристик 
узлов~[1, 3, 4, 6]. Приближенные методы расчета узлов, учитывающие повторные 
попытки передачи, используют модели типа $\vert PH\vert PH\vert 1\vert r$ или 
$
\begin{matrix}
M \\ \lambda
\end{matrix}
\left |
\begin{matrix}
M \\ \lambda
\end{matrix}
\right |
1 \vert N
$ и являются 
итерационными~[2, 3, 5, 7]. Для моделей типа 
$BM\!AP\vert PH\vert 1$, $M\vert G\vert 1\vert r$ и $M\!AP\vert 
(PH,PH)\vert 1$ с повторными заявками получены точные методы вычисления 
характеристик (например, в работах~[8--10]), которые также могут быть 
использованы при расчете узлов.

    Ниже будут рассмотрены модели узла коммутации пакетов с повторными 
передачами для двух схем распределения буферной памяти: с 
полнодоступными буферами и с полным разделением буферной памяти. 
Предлагается приближенный метод расчета характеристик, который в качестве 
модели узла использует СМО типа $
\begin{matrix}
M \\ \lambda
\end{matrix}
\left |
\begin{matrix}
M \\ \lambda
\end{matrix}
\right |
\overline{m} \vert N
$ с повторными заявками. Доказаны утверждения о 
достаточных и необходимых условиях существования и единственности 
решения уравнения для вероятности блокировки в установившемся режиме 
работы и сходимости предлагаемого итерационного метода. 

\section{Модель узла}

    Математическая модель узла представляется в виде СМО с ограниченной 
буферной памятью и различными потоками заявок, каждая из которых требует 
обслуживания только на одной из многоканальных линий связи. 

    Пусть $0<N<\infty$~--- число мест хранения в буферной памяти, $u$~--- 
узел связи, $v$~--- линия связи, $\Omega_u^+$~--- множество исходящих из 
узла~$u$ линий, $c_v$~--- канальная емкость линии~$v$. Поток заявок, 
тре\-бу\-ющих обслуживания на линии~$v$, назовем $v$-по\-то\-ком, заявки этого 
потока~--- $v$-за\-яв\-ка\-ми.


    Пусть выполняются следующие предположения: 
\begin{enumerate}[1.]
\item Места в буферной памяти распределяются согласно одной из двух 
схем:
\begin{enumerate}[($i$)]
\item полнодоступная схема~--- каждое свободное место хранения доступно 
любой заявке;
\item схема полного разделения памяти~--- $v$-за\-яв\-кам доступны всего 
$N_v$ мест, где $\sum\limits_{v\in\Omega_u^+} N_v=N$.
\end{enumerate}
\item Если в момент поступления $v$-заявки в буферной памяти есть 
доступное свободное место, то она сразу занимает это место. Если в момент 
поступления $v$-заявки в системе нет свободного доступного места 
хранения, то поступившая заявка через некоторое время повторно поступает 
на систему, оставаясь $v$-заявкой. 
\item Интенсивности первичных потоков $v$-заявок~--- заданные величины 
$0<\Lambda_v<\infty$, $v\in \Omega_u^+$. Суммарные потоки первичных и 
повторных $v$-заявок являются независимыми в совокупности 
пуассоновскими потоками. Для обслуживания $v$-заявки требуется 
одновременно одно место хранения и один канал типа~$v$, $v\in 
\Omega_u^+$.
\item Первичные нагрузки~--- реализуемые, т.\,е.\ в данном случае 
интенсивности входных первичных потоков равны интенсивностям 
выходных потоков выполненных заявок. 
\item Принятые в СМО $v$-заявки обслуживаются линией~$v$ в порядке 
поступления. 
\item Время занятия канала $v$-заявкой~--- экспоненциально 
распределенная случайная величина с параметром $0<\mu_v<\infty$, 
$v\in\Omega_u^+$, независимая от других случайных событий в узле.
\item Выполненная $v$-заявка с вероятностью~$B_v$ повторяется через 
заданное время~$\tau_v$ (тайм-аут) и с вероятностью $1-B_v$ покидает 
систему через время~$t_v$ навсегда, сразу освободив занятый канал и место 
буферной памяти.
\end{enumerate}

   Будем говорить, что узел блокирован для $v$-за\-яв\-ки, если в буферной 
памяти отсутствует доступное место хранения. Ставится задача вычисления 
вероятностей блокировок и интенсивностей потоков в узле.

\section{Вычисление вероятности блокировки и~интенсивностей~потоков} 

   Пусть $\Lambda_v^*$~--- интенсивность суммарного потока внешних 
заявок, требующих передачи по линии~$v$, $\pi_v$~--- вероятность блокировки 
узла для заявок, требующих передачи по исходящей из узла линии~$v$. 

    Пусть в узле используется полнодоступная схема распределения 
буферной памяти. Тогда, как следует из описания модели, $\pi_v 
=\pi_{v^\prime},\,v,\,v^\prime\in \Omega_u^+$, и для 
интенсивностей~$\Lambda_v^*$, $v\in\Omega_u^+$, справедливы соотношения:
\begin{equation*}
\Lambda_v^* = \fr{\Lambda_v}{1-\pi}\,,
%\label{e1aga}
\end{equation*}
    где
    $\pi =\pi_v$, $v\in\Omega_u^+$.

    Пусть 
    $\overline{k} = \{\overline{k}_v$, $v\in\Omega_u^+\}$~--- состояние 
буферной памяти узла, $\overline{k}_v =\left ( k_v,\,k_v^\prime,\,k_v^{\prime\prime}\right )$; 
$k_v$~--- число $v$-заявок в буферной 
памяти, ожидающих выполнения линией~$v$; $k^\prime_v$~--- число 
$v$-заявок в буферной памяти, ожидающих тайм-аут и неуспешно переданных 
в последующий узел; $k_v^{\prime\prime}$~--- число $v$-за\-явок в буферной 
памяти, успешно переданных в последующий узел и ожидающих 
потверждения; 
$A_m = \left \{ \overline{k}:\ \sum\limits_{v\in\Omega_u^+} \left ( 
k_v+k_v^\prime + k_v^{\prime\prime}\right ) =m \right \}$~--- множество различных 
состояний, при которых в памяти узла занято ровно $m$~буферов. Тогда с 
учетом введенных выше обозначений и предположений для ве\-ро\-ят\-ности 
блокировки узла можно написать формулу~\cite{1aga, 2aga}:
\begin{equation}
\pi = \fr{1}{G_N}\sum\limits_{\overline{k}\in A_N} 
p\left (\overline{k},\overline{\rho}^*\right )\,,
\label{e2aga}
\end{equation}
где  
\begin{gather}
p(\overline{k},\overline{\rho}^*) = \prod\limits_{v\in\Omega_u^+} z_v (\pi, 
\rho_v , k_v , k_v^\prime , k_v^{\prime\prime})\,;\\
z_v (\pi, \rho_v , k_v , k_v^\prime , k_v^{\prime\prime}) ={}\notag\\
\!\!{}=
\begin{cases}
 \fr{\rho_v^{\prime *k_v^\prime}}{k_v^{\prime}!}\,
\fr{\rho_v^{\prime\prime * k_v^{\prime\prime}}}{ k_v^{\prime\prime}!}  \,
\fr{\rho_v^{*k_v}}{ k_{v}!} 
&\mbox{при}\ k_v<c_v\,,\\
 \fr{\rho_v^{\prime * k_v^\prime}}{k_v^{\prime}!} \,
\fr{\rho_v^{\prime\prime * k_v^{\prime\prime}}} { k_v^{\prime\prime}!} 
\fr{\rho_v^{*k_v}}{ c_{v}!c_v^{k_v- c_v}} 
& \mbox{при}\ k_v\geq c_v\,;
\end{cases}\\
G_N = \sum\limits_{m=0}^N\sum\limits_{\overline{k}\in A_m}
p(\overline{k},\overline{\rho}^*)\,;\\ 
\overline{\rho}^*=\{\rho_v^*,\,v\in\Omega_u^+\}\,;\\
\rho_v^* = \fr{\rho_v}{1-\pi}\,;\quad \rho_v =\fr{\Lambda_v}{\mu_v(1- B_v)}\,;\\
\rho_v^{\prime *} =\rho_v^*\mu_v\tau_vB_v\,;\quad \rho_v^{\prime\prime *}=
p_v^* \mu_vt_v,\,\quad  v\in \Omega_u^+\,.\label{e3aga}
\end{gather}

Переобозначив $1-\pi$ через $y$, выражение в правой части равенства~(2)~--- через 
$p_{\overline{k}}(\overline{\rho},y)$, выражение в правой части равенства~(4)~--- 
через $g_N(\overline{\rho},y)$, а выражение в правой 
части равенства~(1)~--- через $1-q_N (\overline{\rho},y)$, 
где $\overline{\rho} = (\rho_v,\,v\in \Omega_u^+)$, $\rho_v = \rho_v^*y\;=$\linebreak 
$=\;\Lambda_v/(\mu_v(1-B_v))$, $v\in\Omega_u^+$, получим нелинейное уравнение 
относительно неизвестной переменной~$y$:
\begin{equation}
y=q_N(\overline{\rho},y)\,.
\label{e4aga}
\end{equation}

    Решим уравнение~(8). Как следует из~(2)--(7), верно 
равенство
\begin{equation}
q_N(\overline{\rho},y) = \fr{g_{N-1}(\overline{\rho},y )}{g_N(\overline{\rho},y)}\,.
\label{e5aga}
\end{equation}
Введем функцию  $d_n(\overline{\rho} ,y)$ среднего числа заявок в узле с 
буферной памятью емкости $n\geq 0$:
$$
d_n(\overline{\rho} ,y) = 
\fr{1}{g_n(\overline{\rho},y)}\,\sum\limits_{m=0}^n m\sum\limits_{\overline{k}\in 
A_m} p_{\overline{k}}(\overline{\rho},y)\,.
$$
Заметим, что $g_n$, $d_n$ и $q_n$, 
$n\geq 0$,~--- непрерывно-дифференцируемые функции по $y\in (0,\,1]$. Взяв 
производную функции~$g_n$ по~$y$, из~(2)--(7) получим
\begin{multline}
\fr{\partial g_n(\overline{\rho},y)}{\partial y} ={}\\
{}= -\sum\limits_{m=0}^n m 
\sum\limits_{\overline{k}\in A_m}\fr{\prod\limits_{v\in\Omega_u^+} z_n 
(0,\rho_v, k_v, k_v^\prime , k_v^{\prime\prime})}{y^{m+1}}={}\\
{}= -\fr{1}{y}\,g_n (\overline{\rho},y)d_n(\overline{\rho},y)\,.
\label{e6aga}
\end{multline}
Взяв производную функции $q_N$ по $y$, из~(\ref{e5aga}) и~(\ref{e6aga}) 
получим
\begin{equation}
\fr{\partial q_N(\overline{\rho},y)}{\partial y} = \fr{q_N(\overline{\rho},y)}{y}\left 
[ d_N (\overline{\rho},y)-d_{N-1}(\overline{\rho},y)\right ]\,.
\label{e7aga}
\end{equation}
    Докажем несколько утверждений о свойствах 
функции~$q_N(\overline{\rho},y)$.
\medskip

\noindent
\textbf{Утверждение 1.} \textit{Справедливы неравенства}
\begin{multline}
0<d_{n+1}(\overline{\rho},y)-d_n(\overline{\rho},y) <1\,,\\
\ \ \ \ \ \ \ \ \ \ \ \ \ \ \ \ \ \ \ \ y\in (0,\,1]\,, \ n\geq 0\,.
\label{e8aga}
\end{multline}


\noindent

Д\,о\,к\,а\,з\,а\,т\,е\,л\,ь\,с\,т\,в\,о\,.\ Подставив выражение для функции 
$d_n(\overline{\rho},y)$ и проведя преобразования, получим
\begin{multline*}
d_{n+1}(\overline{\rho},y) -d_n(\overline{\rho},y) = 
\fr{\sum\limits_{m=0}^{n+1}m\sum\limits_{\overline{k}\in A_m} 
p_{\overline{k}}(\overline{\rho},y)}
{\sum\limits_{m=0}^{n+1}
\sum\limits_{\overline{k}\in A_m} p_{\overline{k}}(\overline{\rho},y)} - {}\\
{}-
\fr{\sum\limits_{m=0}^n m \sum\limits_{\overline{k}\in A_m} p_{\overline{k}} 
(\overline{\rho},y)}{\sum\limits_{m=0}^n
\sum\limits_{\overline{k}\in A_m}p_{\overline{k}}(\overline{\rho},y)}={}\\
{}=\fr{\sum\limits_{m=1}^n m \sum\limits_{\overline{k}\in 
A_m}p_{\overline{k}}(\overline{\rho},y)+(n+1)\sum\limits_{\overline{k}\in 
A_{n+1}}  p_{\overline{k}}(\overline{\rho},y)}{\sum\limits_{m=0}^n\sum\limits_{\overline{k
}\in A_m}p_{\overline{k}}(\overline{\rho},y)+\sum\limits_{\overline{k}\in 
A_{n+1}}p_{\overline{k}}(\overline{\rho},y)} -{}
\end{multline*}
\begin{multline}
{}-
\fr{\sum\limits_{m=0}^n m 
\sum\limits_{\overline{k}\in A_m}p_{\overline{k}}(\overline{\rho},y)}
{\sum\limits_{m=0}^n\sum\limits_{\overline{k}\in A_m} 
p_{\overline{k}}(\overline{\rho},y)}={}\\
{}=\fr{(n+1)\sum\limits_{\overline{k}\in 
A_{n+1}}p_{\overline{k}}(\overline{\rho},y)g_n(\overline{\rho},y)}{g_{n+1}(\overline{\rho},y) g_n(\overline{\rho},y)} -{}\\
{}-
\fr{\sum\limits_{\overline{k}\in 
A_{n+1}}p_{\overline{k}}(\overline{\rho},y)\sum\limits_{m=0}^n  m 
\sum\limits_{\overline{k}\in A_m} p_{\overline{k}}(\overline{\rho},y) }
{g_{n+1}(\overline{\rho},y) g_n(\overline{\rho},y)}
={}\\
{}=\left [ 1-q_{n+1}(\overline{\rho},y)\right ] \left [n+1-d_n(\overline{\rho},y)\right ]\,.
\label{e9aga}
\end{multline}


    Докажем утверждение~1 методом индукции. При $n = 0$, как следует 
из~(\ref{e9aga}), имеем
$$
d_2(\overline{\rho},y) - d_1 (\overline{\rho},y) =1-q_1(\overline{\rho},y)\,,
$$
    т.\,е.\ утверждение~1 при $n = 0$ справедливо. 

    Пусть неравенства~(\ref{e8aga}) справедливы для некоторого $n > 0$. 
Докажем, что они справедливы и для $n + 1$. Из~(\ref{e9aga}) получаем
\begin{multline*}
d_{n+1}(\overline{\rho},y)- d_n(\overline{\rho},y)={}\\
{}=\left [ 1-
q_{n+1}(\overline{\rho},y)\right ] \left [n+1-d_n(\overline{\rho},y)\right ] ={}\\
{}= \left [ 1-
1-q_{n+1}(\overline{\rho},y)\right ] \left [ n-{}\right.\\
{}-\left. d_{n-1}(\overline{\rho},y)+d_{n-1}(\overline{\rho},y)-
d_n(\overline{\rho},y)+1\right ] ={}\\
{}=\left [ 1-q_{n+1}(\overline{\rho},y)\right ] 
\left [ n-d_{n-1}(\overline{\rho},y)-{}\right.\\
{}-\left. \left ( d_n(\overline{\rho},y)-d_{n-1}(\overline{\rho},y)\right )+1\right] = {}\\
{}=
\left [ 1-q_{n+1}(\overline{\rho},y)\right ]
\left [ 
\fr{d_n(\overline{\rho},y) -d_{n-1}(\overline{\rho},y)}{1-
q_n(\overline{\rho},y)}\right.-{}\\
{}-\left.
\left ( d_n(\overline{\rho},y)-d_{n-1}(\overline{\rho},y)\right )+1
\vphantom{\fr{d_n(\overline{\rho})}{(q_n)}}
\right ]={}\\
{}=
\left [ 1-q_{n+1}(\overline{\rho},y)\right ]
\left [ 
\vphantom{\fr{d_n(\overline{\rho})}{(q_n)}}
\left ( d_n(\overline{\rho},y\right)\right. -{}\\
 {}-\left.
d_{n-1}\left(\overline{\rho},y)\right )\fr{q_n(\overline{\rho},y)}{1-
q_n(\overline{\rho},y)}+1\right ]\,.
\end{multline*}
Так как по предположению $d_n (\overline{\rho},y) -d_{n-1}(\overline{\rho},y) 
>0$, то правая часть последнего равенства больше нуля; следовательно, 
$d_{n+1}(\overline{\rho},y)-d_n(\overline{\rho},y)>0$. 

    Продолжив преобразование правой части последнего равенства и 
учитывая предположение $d_n(\overline{\rho},y) -d_{n-1}(\overline{\rho},y)<1$, 
получим
\begin{multline*}
d_{n+1}((\overline{\rho},y) -d_n(\overline{\rho},y)<{}\\
{}< \left [ 1-
q_{n+1}(\overline{\rho},y)\right ]
\left ( \fr{q_n(\overline{\rho},y)}{1-q_n(\overline{\rho},y)}+1\right )={}\\
{}=
\fr{1-q_{n+1}(\overline{\rho},y)}{1-q_n(\overline{\rho},y)}<1\,,
\end{multline*}
так как $0< q_n(\overline{\rho},y)<q_{n+1}(\overline{\rho},y)<1$, $n>0$, $y\in 
(0,\,1]$.

Следовательно, утверждение~1 доказано.

\medskip

\noindent
\textbf{Утверждение 2.} $q_N(\overline{\rho},y)$~--- \textit{монотонно-воз\-рас\-та\-ющая 
функция по $y\in (0,\,1]$. При этом $0< q_N(\overline{\rho},y)\;\leq $\linebreak 
$\leq\;q_N(\overline{\rho},1) <1$, $y\in (0,\,1]$,  и $\underset{y\rightarrow 
0}{\mathrm{lim}}\,q_N(\overline{\rho},y) =0$}.

\medskip

\noindent
Д\,о\,к\,а\,з\,а\,т\,е\,л\,ь\,с\,т\,в\,о\,.\  Возрастание функции 
$q_N(\overline{\rho},y)$ следует непосредственно из~(\ref{e7aga}) и 
утверж\-де\-ния~1. Доказательство неравенств в условии утверждения очевидно 
следует из~(\ref{e5aga}) и вида функции $g_n (\overline{\rho},y)$, $n\geq 0$. 
Для предела имеем:
\begin{multline*}
\underset{y\rightarrow 0}{\mathrm{lim}}\,q_N(\overline{\rho},y) 
=\underset{y\rightarrow 0}{\mathrm{lim}}\,\fr{g_{N- 1}(\overline{\rho},y)}{g_N(\overline{\rho},y)} = {}\\
{}= \underset{y\rightarrow 0}{\mathrm{lim}}\,\left (
g_{N-1}(\overline{\rho},y)\Bigg / \left ( 
\vphantom{\prod\limits_{v\in\Omega_u^+}}
g_{N-1}(\overline{\rho},y)\right.\right.+{}\\
{}+\left.\left.\sum\limits_{\overline{k}\in A_N}\prod\limits_{v\in\Omega_u^+} 
\fr{z_v(0,\rho_v,k_v,k^\prime_v,k^{\prime\prime}_v)}{y^N}\right )\right ) = {}\\
{}= \underset{y\rightarrow 0}{\mathrm{lim}}\,\left (
y^N g_{N-1}(\overline{\rho},y)\Bigg / 
\left ( 
\vphantom{\prod\limits_{v\in\Omega_u^+}}
y^N g_{N-1}(\overline{\rho},y)+{}\right.\right.\\
{}+\left.\left.\sum\limits_{\overline{k}\in A_N}
\prod\limits_{v\in\Omega_u^+} z_v(0,\rho_v,k_v,k_v^\prime , k_v^{\prime\prime}) 
\right ) \right )=0\,.
\end{multline*}
    
\medskip

\noindent
\textbf{Утверждение 3.} \textit{Производная функции~$q_N (\overline{\rho},y)$ по 
$y\in (0,\,1]$ удовлетворяет следующим соотношениям}:
\begin{align}
\underset{y\rightarrow 0}{\mathrm{lim}}\fr{\partial q_N(\overline{p},y)}
{\partial  y} &= \fr{\sum\limits_{\overline{k}\in A_{N-1}} 
p_{\overline{k}}(\overline{\rho},1)}{\sum\limits_{\overline{k}\in 
A_N}p_{\overline{k}}(\overline{\rho},1)}\,;\label{e10aga}\\
\fr{\partial q_N(\overline{\rho},y)}{\partial y}\Big |_{y=1}&<1\,.\label{e11aga}
\end{align}

\medskip

\noindent
Д\,о\,к\,а\,з\,а\,т\,е\,л\,ь\,с\,т\,в\,о\,.\ Проведя преобразования 
функции~$q_N(\overline{\rho},y)$, получим:
\begin{multline*}
\underset{y\rightarrow 0}{\mathrm{lim}}\fr{q_N(\overline{\rho},y)}{y} = {}\\
\!\!{}=
\underset{y\rightarrow 0}{\mathrm{lim}}
\fr{\sum\limits_{m=0}^{N-1}\sum\limits_{\overline{k}\in A_m}
\!\!\left (\prod\limits_{v\in\Omega_u^+}\!\! 
z_v(0,\rho_v,k_v,k_v^\prime , k_v^{\prime\prime})\right )\!\!\Bigg /\!\! y^m}
{y\sum\limits_{m=0}^{N}\sum\limits_{\overline{k}\in A_m}
\!\!\left(\prod\limits_{v\in\Omega_u^+}\!\! z_v\left (0,\rho_v,k_v,k_v^\prime , 
k_v^{\prime\prime}\right )\right )\!\!\Bigg /\!\!y^m} = \!\!\!
\end{multline*}
\begin{multline*}
\!\!\!\!\!\!{}=\underset{y\rightarrow 0}{\mathrm{lim}}\,
\fr{\sum\limits_{m=0}^{N-1}\sum\limits_{\overline{k}\in A_m}
y^{N-1-m}\prod\limits_{v\in\Omega_u^+} z_v(0,\rho_v,k_v,k_v^\prime , 
k_v^{\prime\prime})}{\sum\limits_{m=0}^{N}\sum\limits_{\overline{k}
\in A_m} y^{N-m}
\prod\limits_{v\in\Omega_u^+} z_v(0,\rho_v,k_v,k_v^\prime , 
k_v^{\prime\prime})}={}\!\\
{}=\fr{\sum\limits_{\overline{k}\in A_{N-1}} p_{\overline{k}}(\overline{\rho},1)}{ 
\sum\limits_{\overline{k}\in A_{N}} p_{\overline{k}}(\overline{\rho},1)}\,.
\end{multline*}
Очевидно, $\underset{y\rightarrow 0}{\mathrm{lim}} \,[d_N (\overline{\rho},y) -
d_{N-1} (\overline{\rho},y)]=1$, так как $\underset{y\rightarrow 
0}{\mathrm{lim}}\,d_n (\overline{\rho},y)=n$, $n>0$.

Следовательно, учитывая~(\ref{e7aga}), получаем~(\ref{e10aga}). 
Справедливость~(\ref{e11aga}) непосредственно следует из~(\ref{e7aga}) и 
утверждения~1.

\medskip

\noindent
\textbf{Утверждение 4.} \textit{Пусть $y^*\in (0,\,1]$~--- решение 
уравнения}~(\ref{e4aga}). \textit{Тогда}
\begin{equation*}
\fr{\partial q_N(\overline{\rho},y)}{\partial y}\Big |_{y=y^*}<1\,.
%\label{e12aga}
\end{equation*}

\medskip

\noindent
Д\,о\,к\,а\,з\,а\,т\,е\,л\,ь\,с\,т\,в\,о\,.\ \ Доказательство следует из~(\ref{e7aga}), 
так как $q_N(\overline{\rho},y^*)/y^* =1$.
\medskip

\noindent
\textbf{Утверждение 5.} \textit{Уравнение}~(\ref{e4aga}) \textit{имеет решение $y^*\in 
(0,\,1)$ тогда и только тогда, когда} 
\begin{equation}
\fr{\sum\limits_{\overline{k}\in A_{N-1}} p_{\overline{k}}(\overline{\rho},1)}{ 
\sum\limits_{\overline{k}\in A_{N}} p_{\overline{k}}(\overline{\rho},1)} >1\,.
\label{e13aga}
\end{equation}
\textit{Если уравнение}~(\ref{e4aga}) \textit{имеет решение $y^*\in (0,\,1)$, то оно 
единственное положительное решение}.
\medskip

\noindent
Д\,о\,к\,а\,з\,а\,т\,е\,л\,ь\,с\,т\,в\,о\,.\ Пусть выполняется 
неравенство~(\ref{e13aga}). Тогда, как следует из утверждения~3, 
$\underset{y\rightarrow 0}{\mathrm{lim}} (\partial q_N(\overline{\rho},y)/\partial y) 
>1$. Кроме того, как следует из утверждения~2, 
$\underset{y\rightarrow 0}{\mathrm{lim}} q_N(\overline{\rho},y)=0$. Тогда, так 
как $q_N(\overline{\rho},y)$~--- непрерывно-дифференцируемая функция по 
$y\in (0,\,1]$, существует значение $y^\prime \in (0,\,1)$ такое, что 
$q_N(\overline{\rho},y)>y$ для всех $y\in (0,\,y^\prime]$ (следует из теоремы о 
конечном приращении~\cite{11aga}). В то же время, согласно утверждению~2, 
$q_N(\overline{\rho},y)<y$ в окрестности точки $y=1$ (рис.~\ref{f1aga},\,\textit{а}). 
Следовательно, кривая $x=q_N(\overline{\rho},y)$ пересекает прямую $x=y$ 
хотя бы в одной точке $y=y^*\in (0,\,1)$, т.\,е.\ уравнение~(\ref{e4aga}) имеет 
хотя бы одно решение $y^*\in (0,\,1)$.

\begin{figure*}
\vspace*{1pt}
\begin{center}
\vspace*{1pt}
\mbox{%
\epsfxsize=158mm
\epsfbox{aga-1.eps}
}
\end{center}
\vspace*{-9pt}
\Caption{Примеры кривых $x=q_N(\overline{\rho},y)$ и $x=y$ (\textit{а})~при существовании решения 
уравнения~(\ref{e5aga}) и (\textit{б})~при выполнении условий~(17)
\label{f1aga}}
\vspace*{6pt}
\end{figure*}

Пусть уравнение~(\ref{e4aga}) имеет решение $y^*\in (0,\,1)$ и 
\begin{equation}
\fr{\sum\limits_{\overline{k}\in A_{N-1}}p_{\overline{k}}(\overline{\rho},1)}{ 
\sum\limits_{\overline{k}\in A_{N}}p_{\overline{k}}(\overline{\rho},1)}\leq 
1\,.\label{e14aga}
\end{equation}
Тогда из условий утверждений~2 и~3 следует, что 
уравнение~(\ref{e4aga}) в интервале $(0,\,1)$ имеет более одного решения, что 
может быть только при существовании решения $y^\prime \in (0,\,1)$ такого, 
что в окрестности точки $y=y^\prime$ выполняются неравенства: 
$q_N(\overline{\rho},y)<y$ при $y<y^\prime$ и $q_N(\overline{\rho},y)>y$ при 
$y>y^\prime$, где $y$ принадлежит указанной окрест\-ности точки~$y^\prime$ 
(рис.~\ref{f1aga},\,\textit{б}). Тогда в точке $y=y^\prime$ производная функции 
$q_N(\overline{\rho},y)$ по $y$ больше~1, что противоречит утверждению~4. 
Следовательно, неравенство~(\ref{e13aga}) справедливо.


Пусть уравнение~(\ref{e4aga}) имеет более одного положительного 
решения. Рассуждая точно так же, как и выше (в случае выполнения 
условий~(\ref{e14aga})), получим противоречие утверждению~4. 
Следовательно, утверждение~5 справедливо.
\medskip

\noindent
\textbf{Следствие.} \textit{Неравенства}
\begin{gather*}
\fr{\mu_v c_v (1-B_v)}{\Lambda_v}>1\,,\quad \fr{1-B_v}{\Lambda_v \tau_v B_v}>1\,,\\ 
\fr{1-B_v}{\Lambda_v t_v}>1\,,\ v\in\Omega_u^+\,,
\end{gather*}
\textit{являются необходимым условием существования решения 
уравнения}~(\ref{e4aga}).

\medskip
\noindent
Д\,о\,к\,а\,з\,а\,т\,е\,л\,ь\,с\,т\,в\,о\,.\ Пусть $\overline{k}_v$~--- это 
набор~$\overline{k}$, у которого $k_v=0$. Преобразовав левую 
часть~(\ref{e13aga}), получим

\noindent
\begin{multline*}
\fr{\sum\limits_{\overline{k}\in A_{N-1}} p_{\overline{k}} (\overline{\rho},1)}
{ \sum\limits_{\overline{k}\in A_{N}} 
 p_{\overline{k}}(\overline{\rho},1)} 
={}
\\
{}=
\fr{\sum\limits_{\overline{k}\in A_{N-1}}\prod\limits_{v\in \Omega_u^+} 
z_v\left(0,\rho_v,k_v,k_v^\prime , k_v^{\prime\prime}\right)}
{\sum\limits_{\overline{k}\in A_{N}}
\prod\limits_{v\in \Omega_u^+} z_v\left (0,\rho_v,k_v,k_v^\prime , k_v^{\prime\prime}\right )} \leq{}
\\
{}\leq
\left ( 
\vphantom{\prod\limits_{v^\prime\in\Omega_u^+\backslash v}}
\fr{\mu_v c_v(1-B_v)}{\Lambda_v}\right. \times{}\\
{}\times \sum\limits_{k_v=0}^{N-1}\sum\limits_{\overline{k}_v\in A_{N-1-k_v}} z_v\left(0,\rho_v,k_v+1,k_v^\prime , 
k_v^{\prime\prime}\right )\times{}\\
{}\times \left.\prod\limits_{v^\prime\in\Omega_u^+\backslash v} z_v^\prime 
\left(0,\rho_v,k_v,k_v^\prime , k_v^{\prime\prime}\right) \right)
\Bigg /{}\\
\Bigg / \left ( 
\vphantom{\prod\limits_{v^\prime\in\Omega_u^+\backslash v}}
\sum\limits_{k_v=0}^{N-1} \sum\limits_{\overline{k}_v\in A_{N-1-k_v}} z_v 
\left (0,\rho_v,k_v+1,k_v^\prime , 
k_v^{\prime\prime}\right )\right. \times{}\\
{}\times \prod\limits_{v^\prime\in\Omega_u^+\backslash v} 
z_{v^\prime}\left(0,\rho_v,k_v,k^\prime , k_v^{\prime\prime}\right)+{}\\
{}+
\sum\limits_{\overline{k}_v\in A_N} z_v\left (0,\rho_v, 0,k_v^\prime , 
k_v^{\prime\prime}\right)\times{}\\
\left.{}\times \prod\limits_{v^\prime\in\Omega_u^+\backslash v}z_{v^\prime} 
\left(0,\rho_v,k_v,k_v^\prime , k_v^{\prime\prime}\right )\right )\,.
\end{multline*}
Как следует из правой части последнего неравенства, если 
$\mu_v c_v (1-B_v)/\Lambda_v \leq 1$, то она меньше~1. Поэтому, чтобы 
выполнилось условие~(\ref{e13aga}), необходимо выполнение первого 
неравенства в условии следствия для каждого $v\in\Omega_u^+$. Точно так же 
доказывается необходимость выполнения второго и третьего неравенств в 
условии следствия.

    Пусть $y[n]$, $n\geq 0$, последовательность, полученная по формуле 
$y[n+1]=q_N(\overline{\rho},y[n])$, $y[0]=1$.

\medskip

\noindent
\textbf{Утверждение 6.} \textit{Пусть $y^*\in (0,\,1)$~--- решение 
уравнения}~(8). \textit{Тогда последовательность $y[n]$, $n\geq 0$, сходится 
к решению~$y^*$}.

\medskip

\noindent
Д\,о\,к\,а\,з\,а\,т\,е\,л\,ь\,с\,т\,в\,о\,.\ Отметим, что $y[1]<y[0]$ (это следует из 
утверждения~2, так как $y[0]=1$). Пусть для некоторого $n>1$ выполняется 
условие $y[n]<y[n-1]$. Тогда, как следует из утверждения~2, указанное условие 
выполняется и для $n+1$, т.\,е.\ по индукции следует, что последовательность 
$y[n]$, $n\geq 0$, монотонно убывает. 

    Пусть для некоторого $n>0$ $y[n]>y^*$ (существование такого $n$ 
следует из равенства $y[0]=1$). Тогда, как следует из утверждения~2, 
$y[n+1]\;=$\linebreak $=\;q_N(\overline{\rho},y[n])>q_N(\overline{\rho},y^*) =y^*$, т.\,е.\ 
последовательность ограничена снизу величиной~$y^*$. Значит, существует 
$\underset{n\rightarrow \infty}{\mathrm{lim}}\,y[n]=y^0\geq y^*$. Так как 
$q_n(\overline{\rho},y)$~--- непрерывная по~$y$ функция, то можно написать 
$\underset{n\rightarrow 
\infty}{\mathrm{lim}}\,q_N(\overline{\rho},y[n])=q_N(\overline{\rho},y^0)=y^0$, 
т.\,е.\ $y^0$~--- решение уравнения~(\ref{e4aga}). Из единственности 
положительного решения уравнения~(\ref{e4aga}) получаем $y^0=y^*$.

    Пусть в узле используется схема полного разделения буферной памяти. 
Тогда для интенсив\-ностей~$\Lambda_v^*$, $v\in\Omega_u^+$, справедливы 
соотношения:
$$
\Lambda_v^* = \fr{\Lambda_v}{1-\pi_v}\,,
$$
где $v\in\Omega_u^+$.


Фиксируем произвольную линию сети~$v$. Пусть $\overline{k}_v = (k_v, 
k_v^\prime, k_v^{\prime\prime})$~--- состояние буферной памяти линии~$v$; 
$k_v$, $k_v^\prime$, $k_v^{\prime\prime}$ определены выше. Тогда с 
учетом введенных ранее предположений и обозначений для вероятности 
блокировки линии справедлива формула~\cite{4aga}:
\begin{equation}
\pi_v = \fr{1}{G_{N_v}}\sum\limits_{k_v=N_v} 
z_v(\pi_v,\rho_v,\overline{k}_v)\,,
\label{e15aga}
\end{equation}
где 
\begin{multline*}
z_v(\pi_v, \rho_v, \overline{k}_v)={}\\
{}=
\begin{cases}
\fr{\rho_v^{\prime * k_v^\prime}}{k_v^\prime !}\,
 \fr{\rho_v^{\prime\prime * k_v^{\prime\prime}}}{k_v^{\prime\prime}!}\,
 \fr{\rho_v^{*k_v}}{k_v !} & \mbox{при}\ k_v<c_v\,,\\
 \fr{\rho_v^{\prime *k_v^\prime}}{k_v^{\prime }! }
 \fr{\rho_v^{\prime\prime * k_v^{\prime\prime}}}{k_v^{\prime\prime}!}
\fr{\rho_v^{*k_v}}{c_v !c_v^{k_v-c_v}} & \mbox{при}\ k_v\geq c_v\,;
\end{cases}
\end{multline*}
\begin{align*}
G_{N_v} &= \sum\limits_{m=0}^{N_v} z_v (\pi_v ,\rho_v , \overline{k}_v)\,;\\ 
\rho_v^*&=\fr{\rho_v}{1-\pi_v}\,;
\end{align*}
$\rho_v$, $\rho_v^{\prime *}$, 
$\rho_v^{\prime\prime *}$, $v\in\Omega_u^+$ определены выше.
    
Пусть $y_v=1-\pi_v$, а $q_{N_v} (\rho_v, y_v)$~--- выражение в правой 
части~(\ref{e15aga}). Тогда из равенств~(\ref{e15aga}), взяв~$y_v$ в качестве 
неизвестной переменной, получим систему независимых уравнений:
\begin{equation}
y_v = q_{N_v}(\rho_v, y_v)\,, \quad v\in \Omega_u^+\,.
\label{e16aga}
\end{equation}
    
    Заметим, что для фиксированной $v$ и заданных параметров $\Lambda_v$, 
$\mu_v$, $\tau_v$, $t_v$, $N_v$, $v\in\Omega_u^+$, уравнение в~(\ref{e16aga}) 
является частным случаем уравнения~(\ref{e4aga}) и совпадает с последним, 
когда число исходящих линий из узла равно~1. Следовательно, для схемы 
полного разделения памяти также справедливы все приведенные выше 
утверждения~1--6 и следствие. Заметим, что неравенство~(\ref{e13aga}) в 
условии утверждения~5 при $B_v=0$ и $t_v=0$ преобразуется в неравенство 
$\Lambda_v / (\mu_v c_v) >1$, $v\in\Omega_u^+$. Последовательность 
$\overline{y}[n]$, $n\geq 0$, в утверждении~6 определяется по формуле:
    \begin{gather*}
    \overline{y}[n] =\{y_v[n],\ v\in\Omega_u^+\}\,,\
    y_v[n+1]=q_{N_v} (\rho_v,\,y_v[n])\,,\\
    y_v[0] =1\,,\quad n\geq 0\,,\quad v\in \Omega_u^+\,.
    \end{gather*}


\section{Алгоритм расчета} %4

    Для вычисления интенсивностей потоков и вероятностей блокировок в 
узле предлагается следующий алгоритм, описывающий изложенную выше 
итерационную процедуру. Введем обозначения:
$y_u^v$~--- вероятность блокировки узла для заявок, направляемых на 
линию~$v$,
\begin{gather*}
y_u^v  = 
\begin{cases}
y_u & \mbox{для}\ v\in\Omega_u^+\ \mbox{при}\\
&\mbox{полнодоступной схеме},\\
y_v & \mbox{при схеме полного распределения}\\
&\mbox{памяти};
\end{cases}
\\
q_N^v(\overline{\rho}_u^{-v}, y_u^v)  = 
\begin{cases}
q_N(\overline{\rho},y) & \mbox{для}\ v\in\Omega_u^+\ \mbox{при пол-}\\ 
&\mbox{нодоступной схеме},\\
q_{N_v}(\rho_v, y_v) & \mbox{при схеме полного}\\
&\mbox{распределения}\\ 
&\mbox{памяти},  v\in\Omega_u^+\,.
\end{cases}
\end{gather*}



Тогда уравнения~(\ref{e4aga}) и~(\ref{e16aga}) записываются в виде:
$$
y_u^v = q_N^v (\overline{\rho}^v_u, y^v_u)\,,\quad v\in \Omega_u^+\,.
$$
Для значений, вычисляемых на $k$-м шаге алгоритма, к 
обозначениям соответствующих параметров приписывается знак~$[k]$.
\pagebreak

\textbf{Шаг 0.} 
\begin{enumerate}[1.]
\item  \textit{Инициализация}. Вычисление начальных значений 
параметров~$\rho_v$, $v\in\Omega_u^+$: $\Lambda_v[0]=\Lambda_v$, 
$\rho_v[0]=\Lambda_v[0]/(\mu_v(1-B_v))$, $y_u^v[0]=1$.
\item \textit{Проверка условий существования решения}. Если для некоторой 
линии $v\in\Omega_u^+$ выполняется хотя бы одно неравенство $(c_v\mu_v(1-
B_v))/\Lambda_v[0]\;\leq$\linebreak $\leq\;1$, или $(1-B_v)/(\Lambda_v\tau_v B_v) \leq 1$, или 
$(t_v(1\;-$\linebreak $-\;B_v))/\Lambda_v[0] \leq 1$, то алгоритм заканчивает работу с 
результатом <<нагрузка не реализуема>>. Если в узле используется 
полнодоступная схема и $(c_v\mu_v(1-B_v))/\Lambda_v[0] > 1$, $(1-
B_v)/(\Lambda_v\tau_v B_v)\;>$\linebreak $>\;1$, $(t_v(1-B_v))/\Lambda_v[0] > 1$ для всех 
$v\in\Omega_u^+$, то проверяется условие~(\ref{e13aga}) для $\Lambda_v =
\Lambda_v[0]$, $v\in\Omega_u^+$, и при невыполнении этого условия алгоритм 
заканчивает работу с результатом <<нагрузка не реализуема>>.
\end{enumerate}

    При вычислении левой части неравенства~(\ref{e13aga}) рекомендуется 
использовать метод свертки Базена (см.~\cite{12aga}), позволяющий 
производить рекуррентные вычисления (подробно этот метод описан также 
в~[1, 3--6]).



\medskip
\textbf{Шаг~$k$} ($k > 0$):
\begin{enumerate}[1.]
\item \textit{Вычисление вероятностей блокировок}. Используя значения 
параметров $\overline{\rho}_u^v[k-1]$, $y_u^v[k-1]$, $v\in\Omega_u^+$, 
вычисление с помощью формул~(1)--(7) значений 
вероятностей $y[k]=1- \pi [k]$~--- в случае полнодоступной памяти, или 
$y_v[k]=1- \pi_v[k]$, $v\in\Omega_u^+$, с помощью формул~(\ref{e15aga})~--- в 
случае полного разделения памяти. При вычислении этих значений 
рекомендуется использовать метод свертки Базена.
    \item \textit{Проверка условий останова алгоритма}. Если хотя бы для 
одной $v\in\Omega_u^+$ для заданного значения точности   выполняется 
условие
$$
\fr{\vert \Lambda_v^*[k]-\Lambda_v^*[k-1]\vert}{\Lambda_v^*[k]}> \varepsilon\,,
$$
то вычисление параметров $\overline{\rho}_u^v[k]$, $v\in\Omega_u^+$, и 
переход к шагу~$k$, положив $k$ равным $k+1$, иначе алгоритм завершает 
работу. 
\end{enumerate}

    По завершении алгоритма либо выявится, что нагрузка в системе не 
реализуема, либо будут вычислены интенсивности потоков, поступающих на 
линии узла, и стационарные вероятности блокировок для заявок каждого типа. 
    
\section{Примеры расчета}

    Для проверки точности вычисления результатов с помощью 
предложенного выше алгоритма и приемлемости введенных предположений 
были проведены вычислительные эксперименты с использованием 
аналитических и имитационных моделей. Во всех рассмотренных ниже 
примерах потоки внешних заявок считаются пуассоновскими. 
В~табл.~1 приведены значения вероятности блокировок вновь 
поступивших извне заявок, полученные на основании точной формулы, 
приведенной в~\cite{4aga} для СМО типа $M\vert M\vert 1\vert 0$ с повторными 
заявками при экспоненциальном распределении интервала времени между 
повторными попытками (первая строка таблицы), алгоритма из подраздела~5 
настоящей статьи (вторая строка) и имитационной модели при постоянном 
интервале времени между повторными попытками, равном~10 (третья строка). 
Расчет табл.~1 проведен для узла с одной исходящей одноканальной 
линией при интенсивности первичного потока $\Lambda =1$ и емкости 
накопителя $N_v=1$. Таблицы~2 и~3 вычислены с помощью 
алгоритма из подраздела~5 и имитационной модели соответственно при одной 
исходящей линии с числом каналов~10.


    В табл.~\ref{t4aga} и~\ref{t5aga} приведены значения вероятности 
блокировки узла с тремя исходящими линиями канальной емкости~10 каждая 
при $\mu_v =0{,}2$, $v\in\Omega_u^+$,  вычисленные с помощью алгоритма из 
подраздела~5 и имитационной модели с интервалом повторной попытки, 
равным~10, соответственно. В табл.~\ref{t4aga} и~\ref{t5aga} знак <<--->> в 
ячейках означает, что предложенная нагрузка $\Lambda_v$, $v\in\Omega_u^+$, 
не реализуема.



В табл.~\ref{t6aga} отражены вероятности блокировки такого же узла с 
накопителем $N = 35$ при экспоненциальном распределении интервала 
времени между повторными попытками со средним значением~$\tau$. 


Результаты вычислительного эксперимента показывают, что с  увеличением 
длины интервала между повторными попытками  вероятность блокировки 
увеличивается и приближается к значению,\linebreak
вычисленному с помощью 
алгоритма из подраздела~5 (см.\ табл.~\ref{t4aga} и~\ref{t6aga}), т.\,е.\ при 
пуассоновском внешнем потоке заявок предположение, что суммарный 
входной поток заявок  является пуассоновским, вполне приемлемо для 
предварительного анализа характеристик узла (например, при  $\tau c_v\mu_v > 
10$). Как показывают табл.~1--3, вероятность блокировки 
узла существенно зависит от\linebreak 

\vspace*{6pt}
\noindent
%\begin{table*}\small %tabl1
{\small
{{\tablename~1}\ \ \small{Вероятности блокировок при одной исходящей одноканальной линии}}
%\label{t1aga}}
\vspace*{-3pt}

\begin{center}
{\tabcolsep=7.3pt
\begin{tabular}{|c|c|c|c|c|c|}
\hline
&\multicolumn{5}{c|}{$\mu$}\\
\cline{2-6}
\multicolumn{1}{|c|}{\raisebox{4pt}[0pt][0pt]{№}}&1{,}1&1{,}2&2&3&4\\
\hline
1&0,9091&0,8333&0,5000&0,3333&0,2500\\
2&0,9091&0,8333&0,5000&0,3333&0,2500\\
3&0,8867&0,8452&0,4944&0,3167&0,2396\\
\hline
\end{tabular}}
\end{center}
%\vspace*{-6pt}
%\end{table*}
}
%\bigskip
%\medskip
\addtocounter{table}{1}
\pagebreak

\end{multicols}

\renewcommand{\figurename}{\protect\bf Таблица}
%\renewcommand{\tablename}{\protect\bf Рис.}
\begin{figure*}
{\small
\begin{minipage}[t]{76mm}
%\begin{table*}\small %tabl2
\begin{center}
\Caption{Вероятности блокировок при одной исходящей многоканальной линии ($\varepsilon 
=0{,}0001$)
\label{t2aga}}
\vspace*{2ex}

\tabcolsep=6.5pt
\begin{tabular}{|c|c|c|c|c|c|}
\hline
&\multicolumn{5}{c|}{$\mu$}\\
\cline{2-6}
\multicolumn{1}{|c|}{\raisebox{4pt}[0pt][0pt]{$N$}}&0{,}11&0{,}12&0{,}2&0{,}3&0{,}4\\
\hline
10&0,4845&0,2935&0,0204&0,0017&0,0002\\
15&0,1181&0,0545&0,0006&0,0000&0,0000\\
20&0,0489&0,0167&0,0000&0,0000&0,0000\\
\hline
\end{tabular}
\end{center}
%\end{table*}
\end{minipage}
\hfill
\begin{minipage}[t]{76mm}
%\begin{table*}\small %tabl3
\begin{center}
\Caption{Вероятности блокировок при одной исходящей линии
\label{t3aga}}
\vspace*{2ex}

\tabcolsep=6.5pt
\begin{tabular}{|c|c|c|c|c|c|}
\hline
&\multicolumn{5}{c|}{$\mu_v$}\\
\cline{2-6}
\multicolumn{1}{|c|}{\raisebox{4pt}[0pt][0pt]{$N$}}&0{,}11&0{,}12&0{,}2&0{,}3&0{,}4\\
\hline
10&0,5247&0,3238&0,0219&0,0019&0,0001\\
15&0,1726&0,0912&0,0004&0,0001&0,0000\\
20&0,1180&0,0371&0,0000&0,0000&0,0000\\
\hline
\end{tabular}
\end{center}
%\end{table*}
\end{minipage}
}
\vspace*{6pt}
\end{figure*}

\renewcommand{\figurename}{\protect\bf Рис.}
\renewcommand{\tablename}{\protect\bf Таблица}
\addtocounter{table}{2}

\begin{table}\small %tabl4
\begin{center}
\parbox{400pt}{\Caption{Вероятности блокировок при трех исходящих линиях, вычисленные алгоритмом из 
подраздела~5 ($\varepsilon =0{,}0001$)
\label{t4aga}}
}

\vspace*{2ex}

\tabcolsep=8pt
\begin{tabular}{|c|c|c|c|c|c|c|c|c|c|}
\hline
&\multicolumn{9}{c|}{$\Lambda_v$}\\
\cline{2-10}
\multicolumn{1}{|c|}{\raisebox{4pt}[0pt][0pt]{$N$}}&1&1{,}1&1{,}2&1{,}3&1{,}4&1{,}5&1{,}6&1{,}7&1{,}8\\
\hline
20&0,0677&0,1423&0,2975&0,7653&---&---&---&---&---\\
25&0,0065&0,0173&0,0394&0,0827&0.1690&0.3827&---&---&---\\
30&0,0005&0,0019&0,0059&0,0155&0.0361&0.0790&0.1792&0,7259&---\\
35&0,0000&0,0002&0,0008&0,0030&0,0089&0,0234&0,0574&0,1505&---\\
40&0,0000&0,0000&0,0001&0,0005&0,0022&0,0075&0,0220&0,0617&0,2449\\
\hline
\end{tabular}
\end{center}
%\end{table}
\vspace*{6pt}
%\begin{table}\small %tabl5
\begin{center}
\parbox{400pt}{\Caption{Вероятности блокировок при трех исходящих линиях, вычисленные с помощью 
имитационной модели
\label{t5aga}}
}

\vspace*{2ex}

\tabcolsep=8pt
\begin{tabular}{|c|c|c|c|c|c|c|c|c|c|}
\hline
&\multicolumn{9}{c|}{$\Lambda_v$}\\
\cline{2-10}
\multicolumn{1}{|c|}{\raisebox{4pt}[0pt][0pt]{$N$}}&1&1{,}1&1{,}2&1{,}3&1{,}4&1{,}5&1{,}6&1{,}7&1{,}8\\
\hline
20&0,0786&0,1695&0,3549&0,7056&---&---&---&---&---\\
25&0,0069&0,0190&0,0441&0,0998&0,2266&0,4583&---&---&---\\
30&0,0007&0,0024&0,0075&0,0184&0,0462&0,1025&0,2380&0,6931&---\\
35&0,0000&0,0003&0,0007&0,0040&0,0129&0,0307&0,0890&0,2981&---\\
40&0,0000&0,0000&0,0000&0,0011&0,0041&0,0095&0,0346&0,0790&0,3179\\
\hline
\end{tabular}
\end{center}
%\end{table}
\vspace*{6pt}
%\begin{table}\small %tabl6
\begin{center}
\parbox{356pt}{\Caption{Зависимость вероятности блокировки при трех исходящих линиях, вы\-чис\-лен\-ные с 
помощью имитационной модели со случайным интервалом между повторными попытками
\label{t6aga}}
}

\vspace*{2ex}

\tabcolsep=8pt
\begin{tabular}{|c|c|c|c|c|c|c|c|c|}
\hline
&\multicolumn{8}{c|}{$\Lambda_v$}\\
\cline{2-9}
\multicolumn{1}{|c|}{\raisebox{6pt}[0pt][0pt]{$\tau$}}&1&1{,}1&1{,}2&1{,}3&1{,}4&1{,}5&1{,}6&1{,}7\\
\hline
\hphantom{9}1&0.0001&0,0001&0,0017&0,0063&0,0210&0,0733&0,1996&0,4222\\
\hphantom{9}5&0.0000&0,0002&0,0016&0,0036&0,0446&0,0159&0,1360&0,3273\\
10&0.0000&0,0002&0,0011&0,0036&0,0101&0,0430&0,0818&0,2774\\
20&0.0000&0,0003&0,0007&0,0029&0,0089&0,0257&0,0863&0,2045\\
     \hline
\end{tabular}
\end{center}
\end{table}


\begin{multicols}{2}


\noindent
числа каналов в линии при равной суммарной 
производительности. Кроме того, как видно из табл.~\ref{t5aga} и~\ref{t6aga}, 
вероятность блокировки в большей степени зависит от среднего значения 
длины интервала между повторными попытками передачи, чем от закона 
распределения длины интервала. Таким образом, предложенный в работе 
алгоритм позволяет вы\-чис\-лить с достаточной точностью вероятность 
блокировки узла, интенсивности повторных передач и предельную величину 
реализуемой нагрузки. Отметим, что полученные в данной статье результаты 
могут быть использованы для расчета нагрузок в телекоммуникационной сети с 
повторами заявок в предыдущем узле или из источника. 


{\small\frenchspacing
{%\baselineskip=10.8pt
\addcontentsline{toc}{section}{Литература}
\begin{thebibliography}{99}    
\bibitem{1aga}
\Au{Kamoun~F., Kleinrock~L.}
Analysis of shared finite storage in a computer networks node environment under 
general traffic conditions~// IEEE Trans. on Commun., 1980. Vol.~28. No.\,7. 
P.~992--1003.

\bibitem{6aga} %2
\Au{Агаларов~Я.\,М., Шоргин~С.\,Я.}
Рекуррентный метод вычисления параметров сетей связи~// Техника средств 
связи, 1986. Сер. <<Системы связи>>. Вып.~6. С.~42--46.

\bibitem{3aga}
\Au{Башарин Г.\,П., Бочаров~П.\,П., Коган~Я.\,А.}
Анализ очередей в вычислительных сетях.~--- М.: Наука, 1989. 

\bibitem{4aga}
\Au{Бочаров~П.\,П., Печинкин~А.\,В.}
Теория массового обслуживания.~--- М.: Изд-во РУДН, 1995. 

\bibitem{5aga}
\Au{Вишневский~В.\,М.} 
Теоретические основы проектирования компьютерных сетей.~--- М.: 
Техносфера, 2003. 

\bibitem{2aga} %6
\Au{Башарин Г.\,П.}
Лекции по математической теории телетрафика.~--- М.: Изд-во РУДН, 2007. 

\bibitem{7aga}
\Au{Таранцев~А.\,А.}
Инженерные методы теории массового обслуживания.~--- М.: Наука, 2007.

\bibitem{9aga} %8
\Au{D'Apice~C., De~Simone~T., Manzo~R., Rizelian~G.}
$M\vert G\vert 1\vert r$ retrial queueing system with priority service of primary 
customers and a customers-searching server~// Distributed Computer and 
Communication Networks. Stochastic Modelling and Optimization.~--- М.: 
Техносфера, 2003. P.~106--117.

\bibitem{8aga} %9
\Au{Klimenok~V.\,I., Kim~C.\,S.}
$BM\!AP$/$PH$/1 retrial system operating in random environment~// Proceedings of 
the 5th St.-Petersburg Workshop on Simulation, St.-Petersburg, June~26\,--\,July~2, 
2005.~--- St.-Petersburg: NII Chemistry St.-Petersburg University Publs., 
2005. P.~367--372.   

\bibitem{10aga}
\Au{Krishnamoorthy~A., Babu~S.}
$M\!AP\vert (PH,PH)/c$ retrial queue with selegeneration of priorities 
and non-preemptive service~// Proceedings of the 14th International Conference on 
Analytical and Stochastic Modeling Techniques and Applications, June~4--6, 
2007. Prague, Czech Republic.~--- Sbr.-Dudweiler: Digitaldruck Pirrot GmbH, 
2007. P.~70--74.

\bibitem{11aga}
\Au{Корн~Г., Корн~Т.}
Справочник по математике.~--- М.: Наука, 1974.

\label{end\stat}


\bibitem{12aga}
\Au{Buzen~J.\,P.}
Computational algorithm for closed queuing networks with exponential servers~// 
Communications ACM, 1973. Vol.~16. No.\,9. P.~527--531.
 \end{thebibliography}
}
}
\end{multicols}
 
 
  %4
\def\stat{kondranin+ushakov}

\def\tit{СИСТЕМА ОБСЛУЖИВАНИЯ С~ОТНОСИТЕЛЬНЫМ ПРИОРИТЕТОМ  И~ПРОФИЛАКТИКАМИ ПРИБОРА$^*$}

\def\titkol{Система обслуживания с~относительным приоритетом  и~профилактиками прибора}

\def\aut{Е.\,С.~Кондранин$^1$,  В.\,Г.~Ушаков$^2$}

\def\autkol{Е.\,С.~Кондранин,  В.\,Г.~Ушаков}

\titel{\tit}{\aut}{\autkol}{\titkol}

\index{Кондранин Е.\,С.}
\index{Ушаков В.\,Г.}
\index{Kondranin E.\,S.}
\index{Ushakov V.\,G.}




{\renewcommand{\thefootnote}{\fnsymbol{footnote}} \footnotetext[1]
{Работа выполнена при финансовой поддержке РФФИ (проект 18-07-00678).}}


\renewcommand{\thefootnote}{\arabic{footnote}}
\footnotetext[1]{Факультет вычислительной математики и~кибернетики Московского государственного 
университета им.\ М.\,В.~Ломоносова, \mbox{ekondranin@yandex.ru}}
\footnotetext[2]{Факультет вычислительной математики и~кибернетики
Московского государственного университета им.\ М.\,В.~Ломоносова;
Институт проб\-лем информатики Федерального исследовательского
центра <<Информатика и~управ\-ле\-ние>> Российской академии наук,
\mbox{vgushakov@mail.ru}}

\vspace*{-10pt}




\Abst{Изучена одноканальная система
массового обслуживания с~двумя типами требований, бесконечным
числом мест для ожидания, гиперэкспоненциальным входящим потоком 
и~профилактиками обслуживающего прибора при освобождении системы.
Тип  требования определяется случайно с~заданными вероятностями 
в~момент его поступления в~систему обслуживания. Требования первого
типа имеют относительный приоритет перед требованиями второго
типа. Найдено нестационарное совместное распределение числа
требований каждого типа в~системе. Профилактики прибора
заключаются в~том, что в~момент освобождения системы от требований
прибор на случайное время с~заданным распределением становится
недоступным для обслуживания. Если за время профилактики поступает
хотя бы одно требование, то начинается нормальное функционирование
системы. Если требования не поступают, то прибор отправляется на
новую профилактику. Такие системы хорошо описывают
функционирование большого числа реальных вычислительных и~информационных систем.}

\KW{гиперэкспоненциальный поток; профилактики
обслуживающего прибора; одноканальная система; относительный
приоритет; длина очереди}

\DOI{10.14357/19922264180405}
  
%\vspace*{4pt}


\vskip 10pt plus 9pt minus 6pt

\thispagestyle{headings}

\begin{multicols}{2}

\label{st\stat}

\section{Введение}

В классической системе массового обслуживания ожидание требований
в очереди связано только с~занятостью обслуживающего прибора. В~то
же время в~реальных системах сам  прибор может пребывать как 
в~активном, так и~в~неактивном состоянии. Такое неактивное
состояние прибора (в~литературе на английском языке используется
термин vacation, а~на русском~--- профилактика или прогулка) может
быть связано со многими причинами. В~част\-ности, сис\-те\-мы
обслуживания с~профилактиками прибора хорошо описывают
функционирование  реальных вычислительных и~информационных систем,
в которых наряду с~основными требованиями имеются второстепенные.
Второстепенные требования всегда присутствуют в~сис\-те\-ме, а~их
обслуживание может проводиться только тогда, когда нет основных,
т.\,е.\ в~фоновом режиме.

С точки зрения самого процесса профилактики прибора существует
несколько ее разновидностей. Во-пер\-вых, могут быть разными
правила, задающие условия начала профилактики: прибор может брать
перерыв только при  полном исчерпании требований в~очереди
(exhaustive service) либо при наличии определенного их числа
(nonexhaustive service). Во-вто\-рых, могут быть разными правила
возвращения прибора в~работу. С~этой точки зрения различают случаи
однократного (single vacation) и~многократного (multiple vacation)
перерыва в~работе. В~первом случае ушедший на профилактику прибор
после ее окончания находится в~рабочем состоянии независимо от
наличия требований в~системе. Во втором случае прибор, не
обнаружив новых требований в~очереди, уходит на новую
профилактику.


В работах~[1--4] можно найти обзор известных результатов, большое
число постановок задач, описание различных приложений и~обширную
библиографию по анализу систем с~профилактиками обслуживающего
прибора.


В настоящей работе исследуется совместное распределение длин
очередей в~нестационарном режиме в~однолинейной системе 
с~ожиданием, гиперэкспоненциальным входящим потоком, двумя типами
требований и~относительным приоритетом. Аналогичная неприоритетная
система обслуживания исследована в~[5].

\vspace*{-6pt}

\section{Описание модели}

Рассматривается однолинейная система массового обслуживания 
с~двумя приоритетными классами требований. Входящий поток~---
гиперэкспоненциальный с~функцией распределения интервалов между
поступлениями требований вида:
\begin{multline*}
A(t)=\sum\limits_{i=1}^kc_i\left(1-e^{-a_it}\right),\enskip t>0,\enskip
a_i>0,\enskip c_i>0,\\
a_i\ne a_j\,,\enskip i\ne j\,,\enskip  \sum\limits_{i=1}^k c_i=1\,.
\end{multline*}

Каждое поступившее требование направляется в~первый класс 
с~вероятностью~$p$ и~во второй класс с~вероятностью $1\hm-p$
независимо от остальных требований. Требования первого класса
обладают относительным приоритетом перед требованиями второго
класса. Длительности обслуживания требований $i$-го приоритетного
класса~--- независимые в~совокупности и~не зависящие от входящего
потока случайные величины с~функцией распределения~$B_i(x)$,
$i\hm=1,2.$
 Если в~некоторый момент времени система освободилась от требований, 
 то обслуживающий прибор
 отправляется на профилактику, которая длится случайное время с~функцией 
 распределения~$C(x).$
 Не ограничивая общности, будем считать, что $B_i(x)\hm<1$
 и~$C(x)\hm<1$  для любого~$x$ 
 и~существуют плотности
 распределения~$b_i(x)$ и~$c(x).$
  Обозначим:
$$
 \beta_i(s)=\int\limits_0^{\infty}e^{-sx}b_i(x)\,dx\,;\enskip 
  \gamma(s)=\int\limits_0^{\infty}e^{-sx}c(x)\,dx\,.
$$
Пока прибор находится на профилактике, он не доступен для
обслуживания. Если за время профилактики поступают требования,
после ее завершения начинается их обслуживание. Если ни одно
требование не поступает, то прибор отправляется на новую
профилактику. Длительности различных профилактик являются
независимыми случайными величинами 
и~не зависят от входящего потока и~времен обслуживания.

\section{Вспомогательные результаты}

  Рассмотрим многочлен по $\mu$ степени $k$ вида:
\begin{multline}
\label{1}
\prod\limits_{i=1}^k\left(\mu+a_i\right)-{}\\
{}-
\left(pz_1+(1-p)z_2\right)\sum\limits_{j=1}^kc_ja_j\prod\limits_{i\ne
j}\left(\mu+a_i\right)\,.
\end{multline}
Занумеруем его корни $\mu_1(z_1,z_2),\ldots,\mu_k(z_1,z_2)$ таким образом,
чтобы они были непрерывными функциями и~$\mu_1(1,1)\hm=0.$ Тогда
$\mathrm{Re}\, \mu_j\left(z_1,z_2\right)\hm<0$, $|z_1|\hm<1$, 
$|z_2|\hm<1,$ $\mu_i(z_1,z_2)\hm\ne \mu_j(z_1,z_2),$ $ i\hm\ne j$,
$j\hm=1,\ldots,k.$ Обозначим:
$$
\alpha_m(z_1,z_2)=\prod\limits_{j\ne m}\left(\mu_m\left(z_1,z_2\right)-
\mu_j\left(z_1,z_2\right)\right)\,.
$$
Справедливы следующие леммы.

\smallskip

\noindent
\textbf{Лемма~1.}\
\textit{Для любого $l=1,\ldots,\:k$ система уравнений}
$$
z_j=\beta_j(s-\mu_l(z_1,z_2)),\ \ j=1,2,
$$
\textit{имеет единственное решение $z_i=z_{il}(s)$ такое, 
что $|z_{il}(s)|\hm<1$ при $l\hm=2,\ldots, k,$ $\mathrm{Re}\, s\hm\geqslant 0,$ 
а~$z_{i1}(0)\hm=1$, $|z_{i1}(s)|\hm<1$ при} $\mathrm{Re}\, s\hm> 0$, $i\hm=1,2.$

\smallskip

\noindent
\textbf{Лемма~2.}\
\textit{При каждом $l\hm=1,\ldots,k$ уравнение}
$$
z_1=\beta_1\left(s-\mu_l(z_1,z_2)\right)
$$
\textit{имеет единственное решение $z_1\hm=z_{1l}(z_2,s),$ 
аналитическое в~области $\mathrm{Re}\, s\hm>0$, $|z_2|\hm<1.$
}

\smallskip

Положим
$$
\lambda_l(s)=\mu_l\left(z_{1l}(s),z_{2l}(s)\right)\,.
$$




\section{Распределение длины очереди}

  Гиперэкспоненциальный поток можно рас\-смат\-ри\-вать как
пуассоновский поток со случайной интен\-сив\-ностью~$a,$ которая
принимает $k$ различных значений $a_1,\ldots,a_k$  с~вероятностями
$c_1,\ldots,c_k.$ Текущее значение~$a$ разыгрывается в~момент
поступления требования и~не меняется между двумя соседними
поступлениями. Введем случайный процесс~$j(t)$ такой, что если
$a\hm=a_j$ в~момент времени $t,$ то $j(t)\hm=j.$

Целью работы является нахождение распределения случайного процесса
$\left(L_1(t),L_2(t)\right),$ где $L_i(t)$~--- число требований из
$i$-го приоритетного класса, находящихся в~системе в~момент
времени~$t.$

При сделанных предположениях относительно параметров изучаемой
системы обслуживания\linebreak процесс $\left(L_1(t),L_2(t)\right)$ не
является, вообще говоря, марковским. Пусть $i(t)=i$, $i\hm=1,2,$ если
в~момент времени~$t$ обслуживается требование из $i$-го
приоритетного класса, и~$i(t)\hm=0,$ если в~момент времени~$t$ прибор
находится на профилактике. Случайный процесс~$x(t)$ определим
следующим образом. Если $i(t)\hm\ne 0,$ то $x(t)$ есть
время, прошедшее с~начала обслуживания требования, находящегося на
приборе, до момента~$t.$ Если $i(t)\hm=0,$ то $x(t)$ есть время,
прошедшее с~начала профилактики прибора до момента~$t.$ Случайный
процесс $\left(L_1(t),L_2(t),i(t),j(t),x(t)\right)$ является
однородным марковским процессом. Положим
\begin{multline*}
P_{ij}(n_1,n_2,x,t)=\fr{\partial}{\partial x}
\mathbf{P}\left(L_1(t)=n_1,L_2(t)=n_2,\right.\\
\left. i(t)=i,j(t)=j,x(t)<x
\vphantom{L_1}\right)\,,\enskip 
 x\geqslant 0,\\ 
 j=1,\ldots,k,\enskip i=0,1,2;
\end{multline*}
\begin{gather*}
\eta_i(x)=\fr{b_i(x)}{1-B_i(x)},\ i=1,2;\enskip 
\eta_0(x)=\fr{c(x)}{1-C(x)}\,;\\
\delta_{i,j}=\begin{cases}
1,&\ i=j;\\ 
0,&\ i\ne j\,.
\end{cases}
\end{gather*}
Функции $P_{ij}(n_1,n_2,x,t)$  удовлетворяют при $x\hm>0$
системам дифференциальных уравнений:
\begin{multline}
\label{3}
\fr{\partial P_{ij}(n_1,n_2,x,t)}{\partial t}+\fr{\partial
P_{ij}(n_1,n_2,x,t)}{\partial
x}={}\\
{}=-(a_j+\eta_i(x))P_{ij}(n_1,n_2,x,t)+ {}\\
{}+
c_j\sum\limits_{l=1}^ka_l\left(p\:P_{il}(n_1-1,n_2,x,t)+{}\right.\\
\left.{}+
(1-p)P_{il}(n_1,n_2-1,x,t)\right)
\end{multline}
и краевым условиям при $x\hm=0$:
\begin{multline}
\label{5}
P_{0j}(n_1,n_2,0,t)=0,\ n_1+n_2>0;\\
P_{0j}(0,0,0,t)=\int\limits_0^{\infty}P_{0j}(0,0,x,t)\eta_0(x)\,dx+{}\\
 {}+\int\limits_0^{\infty}P_{1j}(1,0,x,t)\eta_1(x)dx+{}\\
 {}+
\int\limits_0^{\infty}P_{2j}(0,1,x,t)\eta_2(x)\,dx\,;
\end{multline}

\vspace*{-12pt}

\noindent
\begin{multline}
\label{6}
P_{1j}(n_1,n_2,0,t)+P_{2j}(n_1,n_2,0,t)={}\\
{}=\int\limits_0^{\infty}P_{1j}(n_1+1,n_2,x,t)\eta_1(x)\,dx+{}\\
{}+
\int\limits_0^{\infty}P_{2j}(n_1,n_2+1,x,t)\eta_2(x)\,dx+{}\\
{}+\int\limits_0^{\infty}P_{0j}(n_1,n_2,0,t)\eta_0(x)\,dx\,.
\end{multline}

Будем предполагать, что в~начальный момент времени $t\hm=0$ система
свободна от требований, а~с~начала профилактики прибора прошло
случайное время с~заданным распределением с~плотностью $d(x).$
Таким образом,
\begin{align*}
P_{ij}\left(n_1,n_2,x,0\right)&=0,\ i=1,2;
\\
P_{0j}\left(n_1,n_2,x,0\right)&=c_jd(x)\delta_{n_1+n_2,0},\ \
j=1,\ldots,k\,.
\end{align*}
Положим
\begin{multline*}
p_{ij}\left(z_1,z_2,x,s\right)={}\\
{}=\sum\limits_{n_1=0}^{\infty}
\sum\limits_{n_2=0}^{\infty}z_1^{n_1}z_2^{n_2}\!
\int\limits_0^{\infty}e^{-st}P_{ij}(n_1,n_2,x,t)\,dt\,;
\end{multline*}
$$
  \psi(s)=\int\limits_0^{\infty}e^{-sx}\,dx
  \int\limits_0^{\infty}\fr{c(u+x)d(u)}{1-C(u)}\,du\,.
$$
Тогда, учитывая начальные условия,  из \eqref{3}
получаем:
\begin{multline}
\label{7} 
\fr{\partial p_{ij}(z_1,z_2,x,s)}{\partial x}={}\\
{}=-\left(s+a_j+\eta_i(x)\right)p_{ij}
\left(z_1,z_2,x,s\right)+{}\\
{}+c_j\left(pz_1+(1-p)z_2\right)
\sum\limits_{l=1}^ka_lp_{il}\left(z_1,z_2,x,s\right),\\ 
i=1,2;
\end{multline}

\vspace*{-12pt}

\noindent
\begin{multline}
\label{8} 
\fr{\partial p_{0j}(z_1,z_2,x,s)}{\partial x}={}\\
{}=-\left(s+a_j+\eta_0(x)\right)p_{0j}\left(z_1,z_2,x,s\right)+{}\\
{}+c_j\left(pz_1+(1-p)z_2\right)\sum\limits_{l=1}^ka_lp_{0l}\left(z_1,z_2,x,s\right)+{}\\
{}+ c_jd(x).
\end{multline}
Решения \eqref{7} и~\eqref{8} имеют вид:
\begin{multline}
\label{9}
p_{ij}\left(z_1,z_2,x,s\right)=\left(1-B_i(x)\right)c_j\times{}\\
{}\times \sum\limits_{m=1}^k\fr{\gamma_i^{(m)}(z_1,z_2,s)}{\mu_m(z_1,z_2)+a_j}\,
e^{-(s-\mu_m(z_1,z_2))x}\,,\\
 i=1,2\,,
\end{multline}
\vspace*{-12pt}

\noindent
\begin{multline}
\label{10}
p_{0j}\left(z_1,z_2,x,s\right)={}\\
{}=\left(1-C(x)\right)
c_j\!\!\sum\limits_{m=1}^k\!\! e^{-(s-\mu_m(z_1,z_2))x}\!
\!\left(\!
\vphantom{\int\limits_{l=1}^k}
\delta^{(m)}\left(z_1,z_2,s\right)+{}\right.\\
%\left.
{}+\alpha_m^{-1}\left(z_1,z_2\right)
\prod\limits_{l=1}^k
\left(\mu_m\left(z_1,z_2\right)+a_l\right)\times{}\\
\left.{}\times \int\limits_0^x\!
e^{(s-\mu_m(z_1,z_2))u}
\fr{d(u)}{1-C(u)}\,du
\right)
\!\Bigg/ \!\left(\mu_m\left(z_1,z_2\right)+{}\right.\\
\left.{}+a_j\right)\,,
\end{multline}
где функции $\gamma_i^{(m)}(z_1,z_2,s)$  и~$\delta^{(m)}(z_1,z_2,s)$ являются
произвольными функциями указанных переменных и~определяются из
краевых условий. Из~\eqref{5} и~\eqref{6} получаем:
\begin{multline}
\label{11}
p_{1j}\left(z_1,z_2,0,s\right)+p_{2j}\left(z_1,z_2,0,s\right)={}\\
{}=z_1^{-1}\int\limits_0^{\infty}p_{1j}\left(z_1,z_2,x,s\right)\eta_1(x)\,dx+{}
\\
+z_2^{-1}\int\limits_0^{\infty}p_{2j}\left(z_1,z_2,x,s\right)\eta_2(x)\,dx+{}\\
{}+
\int\limits_0^{\infty}p_{0j}\left(z_1,z_2,x,s\right)\eta_0(x)\,dx
-p_{0j}\left(z_1,z_2,0,s\right)\,.
\end{multline}
Заметим, что $p_{0j}(z_1,z_2,0,s)$ не зависит от $z_1$ и~$z_2,$ т.\,е.\
$p_{0j}(z_1,z_2,0,s)\hm=q_j(s).$ 
Подставляя~\eqref{9} и~\eqref{10} в~\eqref{11}, получаем:
\begin{multline}
\label{12}
\gamma_1^{(m)}\left(z_1,z_2,s\right)\left(1-z_1^{-1}\beta_1(s-\mu_m(z_1,z_2))\right)+{}\\
{}+
\gamma_2^{(m)}(z_1,z_2,s)\left(1-z_2^{-1}\beta_2(s-\mu_m(z_1,z_2))\right)={}\\
{} =
\delta^{(m)}\left(z_1,z_2,s\right)\left(\gamma\left(s-\mu_m\left(z_1,z_2\right)\right)-1\right)+{}\\
{}+
\alpha_m^{-1}\left(z_1,z_2\right)\prod\limits_{l=1}^k
\left(\mu_m\left(z_1,z_2\right)+a_l\right)\psi\left(s-\mu_m(z_1,z_2)\right),\\
j=1,\ldots,k.
\end{multline}
В силу леммы~1 левая часть~\eqref{12} обращается в~0 при
$z_1\hm=z_{1m}(s)$ и~$z_2\hm=z_{2m}(s)$, $m\hm=1,\ldots,k.$ Следовательно,
\begin{multline}
\label{13}
\delta^{(m)}\left(z_{1m}(s),z_{2m}(s),s\right)={}\\
{}=\fr{\psi(s-\lambda_m(s))}{\alpha_m(z_{1m}(s),z_{2m}(s))
(1-\gamma(s-\lambda_m(s)))}\times{}\\
{}\times \prod\limits_{l=1}^k\left(\lambda_m(s)+a_l\right).
\end{multline}
Из \eqref{10} следует, что
$$
q_j(s)=c_j\sum\limits_{m=1}^k\fr{\delta^{(m)}(z_1,z_2,s)}{\mu_m(z_1,z_2)+a_j},\
j=1,\ldots,k .
$$
Решая эту систему уравнений относительно
$\delta^{(m)}(z_1,z_2,s),$ получаем:
\begin{multline}
\label{n1}
\delta^{(m)}(z_1,z_2,s)=\left(pz_1+(1-p)z_2\right)\times{}\\
{}\times
\fr{\prod\nolimits_{j=1}^k(\mu_m(z_1,z_2)+a_j)}
{\alpha_m(z_1,z_2)}\sum\limits_{l=1}^k\frac{a_lq_l(s)}{\mu_m(z_1,z_2)+a_l}.
\end{multline}
Подставляя в~\eqref{n1} $z_1\hm=z_{1m}(s)$ и~$z_2\hm=z_{2m}(s),$ имеем:
\begin{multline}
\label{14}
\delta^{(m)}\left(z_{1m}(s),z_{1m}(s),s\right)={}\\
{}=
\left(pz_{1m}(s)+(1-p)z_{2m}(s)\right)\times{}\\
{}\times
\fr{\prod\nolimits_{j=1}^k
(\lambda_m(s)+a_j)}{\alpha_m(z_{1m}(s),z_{1m}(s))}
\sum\limits_{l=1}^k\fr{a_lq_l(s)}{\lambda_m(s)+a_l}\,.
\end{multline}
Сравнивая два представления~\eqref{13} в~\eqref{14} для
$\delta^{(m)}(z_m(s),s),$ получаем систему уравнений для~$q_l(s)$:
\begin{multline*}
\sum\limits_{l=1}^k\fr{a_lq_l(s)}{\lambda_m(s)+a_l}={}\\
{}=\fr{\psi(s-\lambda_m(s))}{(pz_{1m}(s)+(1-p)z_{2m}(s))
(1-\gamma(s-\lambda_m(s)))},\\
m=1,\ldots,k\,,
\end{multline*}
из которой находим
\begin{multline}
\hspace*{-3pt}q_l(s)=c_l\prod\limits_{j=1}^k
\left(\lambda_l(s)+a_j\right) 
\sum\limits_{m=1}^k
%\fr
\psi(s-\lambda_m(s))\!\Bigg/ \!
\Bigg(\left(1-{}\right.\\
\left.
{}-\gamma\left(s-\lambda_m(s)\right)\right)(\lambda_m(s)+a_l)\times{}\\
{}\times \prod\limits_{n\ne m}(\lambda_m(s)-\lambda_n(s))\!\Bigg).
\label{15}
\end{multline}
Подставляя \eqref{15} в~\eqref{n1} и~учитывая~\eqref{1}, получаем:
\begin{multline*}
\delta^{(m)}(z_1,z_2,s)=\fr{(pz_1+(1-p)z_2)}{\alpha_m(z_1,z_2)}\times
\\
\times\sum\limits_{j=1}^k
\fr{\psi(s-\lambda_j(s))\prod\nolimits_{l=1}^k(\lambda_j(s)+a_l)}
{(pz_{1j}(s)+(1-p)z_{2j}(s))(1-\gamma(s-\lambda_j(s)))}\times{}\\
{}\times\prod\limits_{\nu\ne j}
\fr{\mu_m(z_1,z_2)-\lambda_{\nu}(s)}{\lambda_j(s)-\lambda_{\nu}(s)}\,.
\end{multline*}
Положим
$$
\lambda_m(z_2,s)=\mu_m\left(z_{1m}(z_2,s),z_2\right),\enskip m=1,\ldots,k\,.
$$
Подставляя в~\eqref{12} $z_1\hm=z_{1m}(z_2,s)$, имеем:
\begin{multline}
\label{1q}
\gamma_2^{(m)}\left(z_{1m}(z_2,s),z_2,s\right)={}\\
{}=\fr{\delta^{(m)}(z_{1m}(z_2,s),z_2,s)(\gamma_m(s-\lambda_m(z_2,s))-1)}
{1-z_2^{-1}\beta_2(s-\lambda_m(z_2,s))}+{}
\\
{}+\alpha_m^{-1}(z_{1m}(z_2,s),z_2)\psi(s-\lambda_m(z_2,s))
\prod\limits_{l=1}^k\left(\lambda_m(z_2,s)+{}\right.\\
\left.{}+a_l\right)\!\Bigg/\!
\left(
1-z_2^{-1}\beta_2(s-\lambda_m(z_2,s))\right).
\end{multline}
Далее, из~\eqref{9} следует:
$$
p_{2j}(z_1,z_2,0,s)=c_j\sum\limits_{m=1}^k
\fr{\gamma_2^{(m)}(z_1,z_2,s)}{\mu_m(z_1,z_2)+a_j}\,.
$$
Отсюда
\begin{multline}
\label{2q}
\gamma_2^{(m)}(z_1,z_2,s)=\fr{pz_1+(1-p)z_2}{\alpha_m(z_1,z_2)}\times{}\\
{}\times
\prod\limits_{j=1}^k(\mu_m(z_1,z_2)+a_j)
\sum\limits_{l=1}^k\fr{a_lp_{2l}(z_1,z_2,0,s)}{\mu_m(z_1,z_2)+a_l}\,.
\end{multline}
Так как $p_{2j}(z_1,z_2,0,s)$ не зависит от $z_1$, то
\begin{multline}
\label{3q}
p_{2j}\left(z_1,z_2,0,s\right)={}\\
{}=c_j
\sum\limits_{m=1}^k\fr{\gamma_2^{(m)}\left(z_{1m}(z_2,s),z_2,s\right)}{\lambda_m(z_2,s)+a_j}\,.
\end{multline}
Таким образом, соотношения~\eqref{1q}--\eqref{3q} полностью
определяют $\gamma_2^{(m)}(z_1,z_2,s)$ при любых $z_1$ и~$z_2$.
Теперь из~\eqref{12} можно найти $\gamma_2^{(m)}(z_1,z_2,s)$.

Все функции, необходимые для вычисления $p_{ij}(z_1,z_2,x,s)$,
$i\hm=0,1,2$, $j\hm=1,\ldots,k,$ найде-\linebreak\vspace*{-12pt}

\columnbreak

\noindent
ны. Искомая производящая функция
процесса $(L_1(t),L_2(t))$ равна:

\noindent
\begin{multline*}
\int\limits_0^{\infty}e^{-st}\mathbf{E}
z_1^{L_1(t)} z_2^{L_2(t)}\,dt={}\\
{}=
\sum\limits_{i=0}^2\sum\limits_{j=1}^k\int\limits_0^{\infty}p_{ij}
\left(z_1,z_2,x,s\right)\,dx\,.
\end{multline*}

\vspace*{-18pt}

{\small\frenchspacing
 {%\baselineskip=10.8pt
 \addcontentsline{toc}{section}{References}
 \begin{thebibliography}{9}
\bibitem{1-u}
\Au{Doshi B.\,T.} Queueing systems with vacations~--- a~survey~// 
Queueing Syst., 1986. Vol.~1.  P.~29--66.
\bibitem{2-u}
\Au{Takagi H.} Time-dependent analysis of $M\vert G\vert 1$ vacation models 
with exhaustive service~// Queueing Syst.,
1990. Vol.~6.  P.~369--390.
\bibitem{3-u}
\Au{Li J., Tian N., Zhang~Z.\,G. , Luh~H.\,P.} 
Analysis of the $M\vert G\vert 1$ queue with exponentially working vacations~--- 
a~matrix analytic approach~// Queueing Syst., 2009. Vol.~61.
P.~139--166.
\bibitem{4-u}
\Au{Bouman N., Borst S.\,C., Boxma~O.\,J., Leeuwaarden~J.\,S.\,H.} 
Queues with random back-offs~// Queueing Syst.,
2014. Vol.~77. P.~33--74.
\bibitem{5-u}
\Au{Ушаков~В.\,Г.} Система обслуживания с~гиперэкспоненциальным входящим потоком 
и~профилактиками прибора~// Информатика и~её применения, 2016. Т.~10. 
Вып.~2. С.~93--98.
 \end{thebibliography}

 }
 }

\end{multicols}

\vspace*{-9pt}

\hfill{\small\textit{Поступила в~редакцию 11.05.18}}

\vspace*{6pt}

%\pagebreak

%\newpage

%\vspace*{-28pt}

\hrule

\vspace*{2pt}

\hrule

%\vspace*{-2pt}

\def\tit{A~HEAD OF~THE~LINE PRIORITY QUEUE\\ WITH~WORKING VACATIONS}

\def\titkol{A head of the line priority queue with working vacations}

\def\aut{E.\,S.~Kondranin$^1$ and~V.\,G.~Ushakov$^{1,2}$}

\def\autkol{E.\,S.~Kondranin and~V.\,G.~Ushakov}

\titel{\tit}{\aut}{\autkol}{\titkol}

\vspace*{-11pt}


\noindent
$^1$Department of 
Mathematical Statistics, Faculty of Computational Mathematics and Cybernetics, 
M.\,V.~Lo\-mo-\linebreak
$\hphantom{^1}$no\-sov Moscow State University, 1-52~Leninskiye Gory, 
Moscow 119991, GSP-1, Russian Federation

\noindent
$^2$Institute of Informatics Problems, Federal Research Center 
``Computer Science and Control'' of the Russian\linebreak
$\hphantom{^1}$Academy of Sciences,  44-2~Vavilov Str., Moscow 119333, Russian Federation

\def\leftfootline{\small{\textbf{\thepage}
\hfill INFORMATIKA I EE PRIMENENIYA~--- INFORMATICS AND
APPLICATIONS\ \ \ 2018\ \ \ volume~12\ \ \ issue\ 4}
}%
 \def\rightfootline{\small{INFORMATIKA I EE PRIMENENIYA~---
INFORMATICS AND APPLICATIONS\ \ \ 2018\ \ \ volume~12\ \ \ issue\ 4
\hfill \textbf{\thepage}}}

\vspace*{3pt}



\Abste{The authors analyze the single-server queueing system with 
two types of customers, head of the line priority, hyperexponential 
input stream, and working vacations. The authors obtain the Laplace 
transform (with respect to an arbitrary point in time) of the joint 
distribution of server state, queue size, and elapsed time in that state. 
The authors restrict themselves to a~system with exhaustive service (the 
queue must be empty when the server starts a vacation) and multiple vacations. 
The queueing systems with vacations have been well studied because of their 
applications in modeling computer networks, communication, and manufacturing 
systems. For example, in many digital systems, the processor is multiplexed 
among a~number of jobs and, hence, is not available all the time to handle one job type. 
Besides such an application, theoretical interest in vacation models 
has been aroused with respect to their relationship with polling models.}

\KWE{hyperexponential input stream; working vacations; single server; 
head of the line priority; queue length}



\DOI{10.14357/19922264180405}

\vspace*{-14pt}

\Ack
\noindent
This work was supported by the Russian Foundation for Basic Research 
(project 18-07-00678).


%\vspace*{6pt}

  \begin{multicols}{2}

\renewcommand{\bibname}{\protect\rmfamily References}
%\renewcommand{\bibname}{\large\protect\rm References}

{\small\frenchspacing
 {%\baselineskip=10.8pt
 \addcontentsline{toc}{section}{References}
 \begin{thebibliography}{9}
\bibitem{1-u-1}
\Aue{Doshi, B.\,T.} 1986. Queueing systems with vacations~--- a~survey. 
\textit{Queueing Syst.} 1:29--66.
\bibitem{2-u-1}
\Aue{Takagi, H.} 1990. Time-dependent analysis of $M\vert G\vert M\vert 1$ 
vacation models with exhaustive service. \textit{Queueing Syst.} 6:369--390.
\bibitem{3-u-1}
\Aue{Li, J., N. Tian, Z.\,G.~Zhang,  and H.\,P.~Luh.} 2009. Analysis of the 
$M\vert G\vert 1$ queue with exponentially working vacations~--- 
a~matrix analytic approach. \textit{Queueing Syst.} 61:139--166.
{\looseness=1

}
\bibitem{4-u-1}
\Aue{Bouman, N., S.\,C.~Borst, O.\,J.~Boxma, and J.\,S.\,H.~Leeuwaarden.} 
2014. Queues with random back-offs. \textit{Queueing Syst.} 77:33--74.
\bibitem{5-u-1}
\Aue{Ushakov, V.\,G.} 2016. Sistema obsluzhivaniya s~gipereksponentsialnym 
vkhodyashchim potokom i~profilaktikami\linebreak pribora [Queueing system with working 
vacations and hyperexponential input stream]. 
\textit{Informatika i~ee Primeneniya~--- Inform. Appl.} 10(2):93--98.
\end{thebibliography}

 }
 }

\end{multicols}

\vspace*{-6pt}

\hfill{\small\textit{Received May 11, 2018}}

%\pagebreak

%\vspace*{-18pt}

\Contr

\noindent
\textbf{Kondranin Egor S.} (b.\ 1995)~---  MSc student, Department of 
Mathematical Statistics, Faculty of Computational Mathematics and Cybernetics, 
M.\,V.~Lomonosov Moscow State University, 1-52~Leninskiye Gory, 
Moscow 119991, GSP-1, Russian Federation; \mbox{ekondranin@yandex.ru}

\vspace*{6pt}

\noindent
\textbf{Ushakov Vladimir G.} (b.\ 1952)~--- 
Doctor of Science in physics and mathematics, professor, Department of Mathematical 
Statistics, Faculty of Computational Mathematics and Cybernetics, 
M.\,V.~Lomonosov Moscow State University, 1-52~Leninskiye Gory, Moscow 119991, 
GSP-1, Russian Federation; 
senior scientist, Institute of Informatics Problems, Federal Research Center 
``Computer Science and Control'' of the Russian Academy of Sciences, 
44-2~Vavilov Str., Moscow 119333, Russian Federation; \mbox{vgushakov@mail.ru}
\label{end\stat}

\renewcommand{\bibname}{\protect\rm Литература}        %5
\def\stat{grusho}

\def\tit{АРХИТЕКТУРНЫЕ РЕШЕНИЯ В~ЗАДАЧЕ ВЫЯВЛЕНИЯ МОШЕННИЧЕСТВА ПРИ~АНАЛИЗЕ 
ИНФОРМАЦИОННЫХ ПОТОКОВ В~ЦИФРОВОЙ ЭКОНОМИКЕ$^*$}

\def\titkol{Архитектурные решения в~задаче выявления мошенничества при~анализе 
информационных потоков в
%~цифровой 
экономике}

\def\aut{А.\,А.~Грушо$^1$, М.\,И.~Забежайло$^2$, Н.\,А.~Грушо$^3$, 
Е.\,Е.~Тимонина$^4$}

\def\autkol{А.\,А.~Грушо, М.\,И.~Забежайло, Н.\,А.~Грушо, 
Е.\,Е.~Тимонина}

\titel{\tit}{\aut}{\autkol}{\titkol}

\index{Грушо А.\,А.}
\index{Забежайло М.\,И.}
\index{Грушо Н.\,А.}
\index{Тимонина Е.\,Е.}
\index{Grusho A.\,A.}
\index{Zabezhailo M.\,I.}
\index{Grusho N.\,A.}
\index{Timonina E.\,E.}


{\renewcommand{\thefootnote}{\fnsymbol{footnote}} \footnotetext[1]
{Работа частично поддержана РФФИ (проекты 18-29-03081 и~18-07-00274).}}


\renewcommand{\thefootnote}{\arabic{footnote}}
\footnotetext[1]{Институт проблем информатики Федерального исследовательского центра <<Информатика и~управление>> 
Российской академии наук, grusho@yandex.ru}
\footnotetext[2]{Институт проблем информатики Федерального исследовательского центра <<Информатика и~управление>> 
Российской академии наук, m.zabezhailo@yandex.ru}
\footnotetext[3]{Институт проблем информатики Федерального исследовательского центра <<Информатика и~управление>> 
Российской академии наук, info@itake.ru}
\footnotetext[4]{Институт проблем информатики Федерального исследовательского центра <<Информатика и~управление>> 
Российской академии наук, eltimon@yandex.ru}

\vspace*{-12pt}
   

 
  
  \Abst{Cформулирован подход к~исследованию некоторых видов мошенничества в~цифровой 
экономике с~использованием причинно-следственных связей. Во всех видах рассматриваемых 
мошенничеств должно наблюдаться несоответствие между целями финансовых транзакций 
и~реальной стоимостью достижения этих целей. Данные о транзакциях можно собирать, 
наблюдая информационные потоки, в~которых отражаются эти транзакции. Архитектура сбора 
данных и~их анализа может быть организована с~помощью распределенных реестров 
с~централизованным консенсусом, что позволяет создать аналог электронной бухгалтерской 
книги, фиксирующей финансово-экономическую деятельность субъектов цифровой экономики в~регионе. 
  Рассматриваемые методы выявления мошенничества основаны на противоречиях 
между действиями, описанными в~транзакциях, и~информацией, содержащейся в~планах, 
стандартах, прецедентах и~др. Рассмотрен метод, основанный на некоторой упрощенной схеме 
реализации абстрактного проекта. Для выявления противоречий необходимо проводить анализ 
от следствия к~причине, т.\,е.\ искать аномалии в~информации, описывающей порождение 
наблюдаемых следствий. 
  Показано, как в~реализации проекта можно выделять простые <<необходимые условия>> 
нарушения при\-чин\-но-след\-ст\-вен\-ных связей, т.\,е.\ множество <<необходимых условий>>, 
нарушение которых свидетельствует о наличии мошенничества. Это множество <<необходимых 
условий>> можно назвать метаданными для контроля проекта на выявление мошенничества.} 
 
 
  \KW{цифровая экономика; информационные потоки; при\-чин\-но-след\-ст\-вен\-ные связи; 
выявление мошеннических схем} 

\DOI{10.14357/19922264190204}
  
\vspace*{-4pt}


\vskip 10pt plus 9pt minus 6pt

\thispagestyle{headings}

\begin{multicols}{2}

\label{st\stat}

\section{Введение}

\vspace*{3pt}

  В работе сформулирован подход к~исследованию некоторых видов 
мошенничества в~цифровой экономике с~использованием  
при\-чин\-но-след\-ст\-вен\-ных связей. Рассматриваются три вида мошенничества, 
а именно:
  \begin{enumerate}[(1)]
\item отмыв денег; 
\item обман при выполнении договорных обязательств при реализации 
технических проектов (строительные проекты и~др.); 
\item незаконный вывод денег. 
\end{enumerate}

  Названные виды мошенничества могут быть сведены к~решению одного типа 
задач. Для отмывания денег источник должен заключать фиктивные контракты, 
в~соответствии с~которыми будут переводиться средства за заведомо ненужную 
работу и~материалы. 
  
  Мошенничество, связанное с~невыполнением договорных обязательств, связано 
со снижением качества услуг, качества и~количества закупаемых 
материалов, выполнением работ с~ненадлежащим качеством. 
  
  Вывод денег связан с~переводом средств фир\-мам-од\-но\-днев\-кам, которые 
заведомо не могут выполнить обязательства по контрактам, за которые им 
переводятся средства. 
  
  Таким образом, во всех трех видах рассматриваемых мошенничеств должно 
наблюдаться несоответствие между целями финансовых транзакций и~реальной 
стоимостью достижения этих целей. Данные о транзакциях можно собирать, 
наблюдая информационные потоки, в~которых отражаются эти транзакции. 
  
  Однако для наблюдения таких информационных потоков необходимо создавать 
архитектуру\linebreak телекоммуникационной системы, позволяющей перехватывать 
и~собирать данные о всех транзакциях. Например, такая архитектура может быть 
организована с~помощью распределенных реестров с~централизованным 
консенсусом, т.\,е.\ все информационные потоки, сформированные в~цифровой 
экономике и~несущие информацию о транзакциях, проходят через некоторый 
центральный узел, запоминающий их в~форме распределенного реестра. Такие 
реестры могут дублироваться в~аналогичных центрах различных регионов, что 
позволяет создать аналог электронной бухгалтерской книги, фиксирующей 
фи\-нан\-со\-во-эко\-но\-ми\-че\-скую деятельность субъектов цифровой экономики. Такой 
подход предложено реализовать на базе системы ситуационных центров, что 
отражено в~работах~[1, 2].
  
  Собранная из информационных потоков информация о~транзакциях, т.\,е.\ 
о~контрактах, договорах, платежах, отчетах, закупленных материалах, 
характеристиках исполнителей работ и~др., собирается в~базе данных в~указанном 
центре. Согласно теории интеллектуальных сис\-тем~[3], эту базу данных можно 
называть базой фактов (БФ). Базу фактов можно представить как бинарную мат\-ри\-цу, 
строки которой описывают характеристики, входящие в~транзакции, а столбцы 
нумеруются характеристиками. Строки матрицы будем называть 
\textit{объектами}~[4, 5]. 
  
  Рассматриваемые в~работе методы выявления мошенничества будут основаны 
на противоречиях между действиями, описанными в~транзакциях, и~информацией, 
содержащейся в~планах, стандартах, прецедентах и~др. Для нахождения 
противоречий в~архитектуре центра предусмотрена другая база данных~--- база 
знаний (БЗ)~\cite{3-gr, 6-gr}, которая устроена так же, как БФ. 
  
  Информация в~БЗ собирается на основе положительного опыта или расчетов. 
Используя БЗ, можно выводить факты нарушения при\-чин\-но-след\-ст\-вен\-ных 
связей. Нарушения при\-чин\-но-след\-ст\-вен\-ных связей будем называть 
\textit{аномалиями}. 
  
  Для упрощения дальнейшее изложение будет вестись в~рамках поиска 
противоречий при выполнении некоторого абстрактного проекта. Выявление 
аномалий будет происходить на основе фактов из БФ с~помощью знаний из БЗ 
методами искусственного интеллекта и~интеллектуального анализа 
данных~\cite{6-gr}. 

\vspace*{-10pt}
  
  \section{Модели}
  
  \vspace*{-3pt}
  
  Наиболее сложная из рассмотренных выше задач~--- выявление противоречий, 
т.\,е.\ использование БЗ для получения новых знаний и~выявление аномалий из 
полученных фактов. 
  
  Все способы выявления противоречий основаны на определении 
  причинно-следственных связей. При этом противоречия в~параметрах транзакций по 
отношению к~требуемым в~БЗ составляют сущность аномалий. 
  
   Далее будет рассмотрен метод, основанный на некоторой упрощенной схеме 
реализации абстрактного проекта. 
  
  Каждый проект имеет цель: например, цель представляет собой построение 
некоторой системы. Воспользуемся структурным подходом, который позволяет 
строить проект на основе разбиения системы на подсистемы и~определения 
взаимодействий подсистем~\cite{7-gr}. При этом каждая подсистема также 
представима структурной моделью. 
  
  Как сама система, так и~каждая ее подсистема имеют свой функционал 
и~спецификацию, па\-ра\-мет\-ры настройки и~домены параметров настройки. Кроме 
этих характеристик существует множество характеристик, связанных 
с~<<жизненным циклом>> создания системы. Сюда входят работы, ресурсы, 
сроки выполнения работ по созданию подсистем и~самой системы, стоимости 
компонентов и~материалов, стоимости работ, схемы поставок, договорные 
обязательства и~др. Все характеристики связаны между собой, поэтому можно 
говорить о стоимости и~времени изготовления структурных компонентов системы. 
  
  Одной из важнейших характеристик является смета (система смет для 
подсистем). Смета сопоставляет каждому компоненту системы стоимость его 
изготовления и~настройки. 
  
  Схема построения системы может быть пред\-став\-ле\-на диаграммой, 
изображенной на рис.~1. 

{ \begin{center}  %fig1
 \vspace*{9pt}
   \mbox{%
 \epsfxsize=79mm 
 \epsfbox{gru-1.eps}
 }


\vspace*{9pt}


\noindent
{{\figurename~1}\ \ \small{Диаграмма достижения цели}}
\end{center}
}

\vspace*{9pt}

\addtocounter{figure}{1}
  
  


  Представленная на рис.~1 диаграмма позволяет описать основные классы 
возможных противоречий при достижении цели. Противоречия возникают, когда 
данные БФ не соответствуют требуемым характеристикам. 
  
  
  \section{Потенциальные классы аномалий при~достижении цели}
  
  Выделим четыре потенциальных класса противоречий, которые показывают, 
каким образом нужно искать эти противоречия.
  
 
  Противоречие цели и~проекта (рис.~2) возникает при отсутствии обоснования 
или в~случае логического противоречия между возможностями проектируемого 
функционала и~целью системы. Отметим, что в~проект входят сроки, перечень 
работ, материалы, настройки, которые описываются соответствующими 
параметрами и~допустимыми значениями этих параметров. Проект формируется 
на основе БЗ и~расчетов, исходя из информации, полученной по аналогии 
с~другими проектами и~решениями, которые считаются апробированными. 
  
  Отметим, что цель порождает проект и~в этом смысле является причиной 
проекта. Однако для анализа противоречий необходимо двигаться по штриховой 
стрелке диаграммы (см.\ рис.~2) от проекта к~цели. В~самом деле, любой компонент 
проекта направлен на теоретическое достижение цели. Цель~--- сложный объект, 
поэтому в~проекте могут возникнуть характеристики, противоречащие хотя бы 
некоторым характеристикам цели. Это делает проект противоречивым, но вывод 
об этом может быть сделан только на уровне описания цели. 
  

  Противоречия между проектом и~его реализацией, исключая настройки 
(рис.~3), могут возникать, например, при закупке исполнителем материалов более 
низкого качества по более низким ценам, при попытках достижения требуемых 
сроков работы за счет снижения качества выполнения работ, за счет нахождения 
<<объективных>> причин для увеличения сроков работы и,~следовательно, 
увеличения цены реализации проекта. 


  Для выявления указанных противоречий необходимо двигаться по диаграмме 
(см.\ рис.~3) в~обратную сторону в~соответствии со~штриховыми стрелками. 
Действительно, выявить противоречия между характеристиками закупленных 
материалов и~требуемыми по проекту можно только при обращении к~проекту 
и~его спецификациям. Манипуляции со сроками работы также можно выявить 
только при обращении к~соответствующим расчетам в~проекте. Задержки в~сроках 
работы, связанные с~поставками материалов, можно определить только на 
предыдущем этапе диаграммы (см.\ рис.~3) в~описании проекта. 


  


  Противоречия между реализацией проекта и~его настройкой (рис.~4) возникает, 
когда не удается добиться требуемых значений параметров функционала, не 
удается обеспечить необходимый уровень\linebreak\vspace*{-12pt}

{ \begin{center}  %fig2
 \vspace*{-6pt}
   \mbox{%
 \epsfxsize=16mm 
 \epsfbox{gru-2.eps}
 }


\vspace*{6pt}


\noindent
{{\figurename~2}\ \ \small{Противоречия цели и~проекта}}
\end{center}
}

%\vspace*{9pt}

\addtocounter{figure}{1}

{ \begin{center}  %fig3
 \vspace*{6pt}
    \mbox{%
 \epsfxsize=79mm 
 \epsfbox{gru-3.eps}
 }


\end{center}

\vspace*{-2pt}


\noindent
{{\figurename~3}\ \ \small{Противоречия проекта и~его реализации (без настройки)}}
}

\vspace*{6pt}

\addtocounter{figure}{1}

{ \begin{center}  %fig4
 \vspace*{1pt}
   \mbox{%
 \epsfxsize=54.5mm 
 \epsfbox{gru-4.eps}
 }


\end{center}


\noindent
{{\figurename~4}\ \ \small{Противоречия реализации проекта и~его на\-стройки}}
}

%\vspace*{9pt}

\addtocounter{figure}{1}

{ \begin{center}  %fig5
 \vspace*{5pt}
    \mbox{%
 \epsfxsize=79mm 
 \epsfbox{gru-5.eps}
 }


\end{center}



\noindent
{{\figurename~5}\ \ \small{Противоречия цели и~достигнутой реализации проекта}}
}

\vspace*{6pt}

\addtocounter{figure}{1}

\noindent
 качества реализации проекта. Для 
определения противоречия в~настройках надо опять же двигаться по диаграмме 
(см.\ рис.~4) в~обратную сторону по штриховым стрелкам, так как для выявления 
характеристик результатов работы, которые не дают возможности реализации 
определенного функционала, необходимо иметь информацию о результатах этой 
работы. 


  



  Противоречие между целью и~достигнутой реализацией проекта (рис.~5) 
возникает, когда реализованная система не позволяет достичь цели. В~этом случае 
опять противоречие нужно искать, двигаясь от цели к~реальному достигнутому 
функционалу по штриховой стрелке (см.\ рис.~5).
  
  Суммируя положения, изложенные в~данном разделе, приходим к~выводу, что 
для выявления противоречий необходимо проводить анализ от следствия 
к~причине, т.\,е.\ искать аномалии в~информации, описывающей порождение 
наблюдаемых следствий. 
  
  
  \section{Связь противоречий и~причин}
  
  Прежде чем построить связь между причинами и~противоречиями, кратко 
опишем простейшую модель связи этих понятий. Причины и~противоречия будут 
сформулированы для представления компонентов системы как объектов, 
обладающих наборами известных характеристик~\cite{4-gr, 5-gr}. 
  
  Пусть $U\hm=\{\alpha, \beta, \ldots\}$~--- совокупность характеристик 
(пространство характеристик). Согласно~\cite{4-gr} \textit{объектом}~$O$ 
называется любое подмножество характеристик $O\hm\subseteq U$. Рассмотрим 
последовательность объектов, возможно в~различных пространствах 
характеристик. 
  
  \smallskip
  
  \noindent
  \textbf{Определение~1.}\ Объект~$P$ с~числом характеристик, большим или 
равным~2, является \textit{причиной} объекта (\textit{свойства})~$B$ в~цепочке 
наблюдаемых объектов тогда и~только тогда, когда выполнены следующие 
условия:
  \begin{enumerate}[(1)]
\item для каждого объекта~$C$, если $P\hm\subseteq C$, то $C\mapsto B$, где 
$C\mapsto B$ означает, что объект~$B$ присутствует в~объекте, следующем за 
объектом~$C$;
\item объект~$P$ является минимальным объектом, удовлетворяющим 
условию~1, а~именно: $\forall \alpha\hm\in P$ объект~$P\backslash \{\alpha\}$ 
не является причиной, т.\,е.\ $\exists C:\ \alpha\not\in C$, $P\backslash 
\{\alpha\}\hm\subseteq C$ и~$C\not\mapsto B$, где $C\not\mapsto B$ означает, 
что~$B$ не может содержаться в~объекте, следующем за объектом~$C$. 
\end{enumerate}

  Приведенное определение причины является упрощением причин, 
возникающих в~реальном мире. Например, реальные причины могут возникать\linebreak 
как совокупность характеристик из разных пространств. Одно следствие может 
порождаться разными причинами или возникать из внешних\linebreak и~ненаблюдаемых 
характеристик. Однако пред\-став\-лен\-ная далее формализация позволяет доступно 
изложить при\-чин\-но-след\-ст\-вен\-ные истоки противоречий, которые 
инициируют в~дальнейшем глубокое исследование рассматриваемых процессов.
  
  Будем считать, что для любого интересующего нас свойства~$B$ существует 
причина. Тогда справедлива следующая теорема.
  
  \smallskip
  
  \noindent
  \textbf{Теорема~1.}\ \textit{Для любого свойства~$B$ существует 
единственная причина}. 
  
  \smallskip
  
  \noindent
  Д\,о\,к\,а\,з\,а\,т\,е\,л\,ь\,с\,т\,в\,о\,.\ \ Доказательство будем вести от противного, 
т.\,е.\ предположим, что существуют две причины свойства~$B$: $P$ 
и~$P^\prime$, $P\hm\not= P^\prime$. Тогда существует $\alpha\hm\in U$, которое 
удовлетворяет одному из двух условий:
  \begin{itemize}
\item[(а)] $\alpha\in P$, $\alpha\notin P^\prime$;
\item[(б)] $\alpha\notin P$, $\alpha \in P^\prime$.
\end{itemize}

  Пусть выполняется условие~(б). Тогда $P^\prime\backslash \{\alpha\}$ не 
является причиной по условию~2 определения~1, т.\,е.\ $\exists C$ такое, что 
$\alpha\notin C$, $P^\prime\backslash \{\alpha\}\hm\subseteq C$ и~$C\not\mapsto B$. 
Но если~$B$ произошло и~$P$ его причина, то $C\mapsto B$, что противоречит 
предположению. Теорема~1 доказана.
  
  \smallskip
  
  \noindent
  \textbf{Лемма.} \textit{Если $P$~--- причина появления свойства~$B$, то 
объект~$B$ определяет существование свойства~$P$ в~объекте, 
предшествующем~$B$. }
  
  \smallskip
  
  \noindent
  Д\,о\,к\,а\,з\,а\,т\,е\,л\,ь\,с\,т\,в\,о\,.\ \ Из предположения, что у~каж\-до\-го 
свойства~$B$ есть причина, и~условия, что~$P$ является причиной~$B$, следует, 
что при появлении в~данных свойства~$B$ объект~$C$, предшествующий 
появлению~$B$, содержит как часть объект~$P$. Это следует из теоремы~1 
и~определения причины. 
  
  Докажем принцип <<необходимого условия>>, который, несмотря на простоту 
доказательства, будет играть в~дальнейшем существенную роль.
  
  \smallskip
  
  \noindent
  \textbf{Теорема~2.} \textit{Если~$P$~--- причина появления свойства~$B$ 
и~$A\hm\subseteq P$, то объект~$B$ определяет наличие свойства~$A$ 
в~объекте, предшествующем~$B$}. 
  
  \smallskip
  
  \noindent
  Д\,о\,к\,а\,з\,а\,т\,е\,л\,ь\,с\,т\,в\,о\,.\ \ Пусть в~данных имеется объект~$B$ 
и~$P\mapsto B$, тогда в~силу существования и~единственности причины~$B$ 
в~данных должен существовать объект~$C$, предшествующий~$B$ 
и~содержащий причину~$P$. Поскольку $A\hm\subseteq P$ и~$B$ содержит 
причину~$P$, то $B\mapsto A$. С~учетом леммы теорема~2 доказана.
  
  \smallskip
  
  Пусть даны пространства $U_1, U_2,\ldots$ и~имеется последовательность 
данных (процесс выполнения этапов проекта в~соответствии с~рис.~1) $A, B, 
\ldots$, где каждый объект является подмножеством некоторого 
пространства~$U_i$, $i\hm=1,\ldots$ Тогда в~объекте~$A$ присутствует 
причина~$P$ появления интересующего нас свойства~$C$ в~объекте~$B$. Пусть 
$P\hm\subseteq A$, тогда по теореме~2 $\forall \alpha\hm\in P$:  
$C\mapsto \{\alpha\}$, т.\,е.\ из появления~$C$ следует появление 
характеристики~$\alpha$ в~предшествующем объекте. Это необходимое условие 
того, что~$C$ удовлетворяет причинно-следственным связям развития процесса 
выполнения проекта. Если для~$C$ нет характеристики~$\alpha$, которую можно 
отнести к~причине~$C$, то можно считать, что нарушена  
при\-чин\-но-след\-ст\-вен\-ная связь и~$C$~--- аномальный объект. 
  
  \smallskip
  
  \noindent
  \textbf{Пример.} Если объект~$C$ состоит в~получении суммы~$a$ 
фирмой~$K$, то согласно теореме~2 в~пред\-шест\-ву\-ющем объекте должна 
существовать причина перевода суммы~$a$ на фирму~$K$. Если эта причина 
в~проекте отсутствует, то это можно считать признаком мошеннической схемы. 
Все проекты по предположению собираются из <<кубиков>>, содержащихся в~БЗ. 
Тогда можно сравнить цену объекта~$C$, породившего получение суммы~$a$, 
и~сумму, присутствующую в~смете проекта. Если разница велика, то это либо 
ошибка проекта, либо признак мошеннической схемы.
  
  \section{Поиск противоречий на~основе~принципа <<необходимых~условий>>}
   
  Как было показано в~разд.~3, нахождение противоречий соответствуют 
движению от следствия к~причине. Для каждого объекта в~наблюдаемых данных 
выявление причин его появления является трудоемкой задачей. Кроме того, при 
реализации контроля соблюдения при\-чин\-но-след\-ст\-вен\-ных связей на 
большом множестве участников экономической деятельности задача анализа 
причин становится трудоемкой. Поэтому процедуру контроля необходимо разбить 
на два этапа, где первый этап состоит в~анализе простых <<необходимых 
условий>> проявления мошенничества, когда используется хотя бы одна 
известная характеристика причины. Второй этап (в~режиме офлайн) состоит 
в~выявлении причин, позволяющих провести анализ источников мошеннических 
схем. 
  
  Один из подходов к~выбору <<необходимых условий>> состоит в~построении 
множества подцелей исходной цели проекта (структурный метод построения 
проекта~\cite{7-gr}). Каждая подцель описывается диаграммой на рис.~1, 
и~реализации подцелей должны образовывать полный функционал цели. Это 
является необходимым, но не достаточным условием достижения цели, так как 
при таком подходе отсутствует компонент согласования всех подцелей в~единую 
систему. Однако такой подход значительно упрощает анализ выполнения проекта 
на предмет поиска мошенничества. Если признаки мошенничества будут 
обнаружены в~реализации хотя бы одной из подцелей, то это значит, что 
мошенничество присутствует в~реализации всего проекта. 
  
  Аналогично в~реализации каждого этапа в~любой из подцелей можно выделять 
простые <<необходимые условия>> нарушения при\-чин\-но-след\-ст\-венн\-ых 
связей. 
  
  Таким образом, получается множество <<необходимых условий>>, нарушение 
которых свидетельствует о наличии мошенничества. Это множество 
<<необходимых условий>> можно назвать метаданными~[8, 9] для контроля 
проекта на выявление мошенничества. 
  
  
  \section{Заключение }
  
  В поиске противоречий необходимо от транзакций, соответствующих 
следствиям при\-чин\-но-след\-ст\-вен\-ных связей, переходить к~анализу причин 
наблюдаемых следствий. Это сложная задача, которая связана с~описанием причин 
определенных свойств. 
  
  В работе представлена модель, позволяющая строить множество необходимых 
условий соответствия наблюдаемого следствия вызвавшей его причине. Этот 
подход делает поиск противоречий вполне вычислимой задачей, но не гарантирует 
успех. 
  
  {\small\frenchspacing
 {%\baselineskip=10.8pt
 \addcontentsline{toc}{section}{References}
 \begin{thebibliography}{9}
\bibitem{1-gr}
\Au{Грушо А.\,А., Зацаринный~А.\,А., Тимонина~Е.\,Е.} Блокчейны цифровой экономики на базе 
системы ситуационных центров и~централизованного консенсуса~// Радиолокация, навигация, 
связь: Мат-лы XXV Междунар. научн.-технич. конф.~---
Воронеж: Издательский дом ВГУ, 2019. Т.~6. С.~183--191. 
\bibitem{2-gr}
\Au{Grusho A., Zatsarinny~A., Timonina~E.} A~system approach to information security in 
distributed ledgers on the situational centers platform.~---
Lecture notes in computer science ser.~--- Springer, 2019 
(in press).
\bibitem{3-gr}
\Au{Финн В.\,К.} Искусственный интеллект: Методология, применения, философия.~--- М.: 
Красанд, 2011. 448~с.

\bibitem{5-gr} %4
\Au{Аншаков~О.\,М., Фабрикантова~Е.\,Ф.} ДСМ-ме\-тод автоматического порождения 
гипотез: Логические и~эпистемологические основания.~--- М.: Либроком, 2009. 432~с.

\bibitem{4-gr} %5
\Au{Poelmans J., Elzinga~P., Viaene~S., Dedene~G.} Formal concept analysis in knowledge 
discovery: A~survey~// Conceptual structures: From information to intelligence~/ Eds.\ M.~Croitoru, 
S.~Ferr$\acute{\mbox{e}}$, and D.~Lukose.~--- Lecture notes in computer science 
ser.~--- Berlin--Heidelberg: Springer, 2010. Vol.~6208.  P.~139--153.

\bibitem{6-gr}
\Au{Панкратова~Е.\,С., Финн~В.\,К.} Автоматическое по\-рож\-де\-ние гипотез в~интеллектуальных 
системах.~--- М.: Либроком, 2009. 528~с. 
\bibitem{7-gr}
\Au{Денисов А.\,А., Колесников~Д.\,Н.} Теория больших систем управления.~--- Л.: Энергоиздат, 1982. 488~с.

\bibitem{9-gr}
\Au{Грушо А.\,А., Грушо Н.\,А., Забежайло~М.\,И., Смирнов~Д.\,В., Тимонина~Е.\,Е.} 
Параметризация в~прикладных задачах поиска эмпирических причин~// Информатика и~её 
применения, 2018. Т.~12. Вып.~3. С.~62--66.

\bibitem{8-gr}
\Au{Грушо А.\,А., Грушо Н.\,А., Левыкин~М.\,В., Тимонина~Е.\,Е.} Методы идентификации 
захвата хоста в~распределенной ин\-фор\-ма\-ци\-он\-но-вы\-чис\-ли\-тель\-ной сис\-те\-ме, 
защищенной с~помощью метаданных~// Информатика и~её применения, 2018. Т.~12. Вып.~4. 
С.~41--45.

 \end{thebibliography}

 }
 }

\end{multicols}

\vspace*{-3pt}

\hfill{\small\textit{Поступила в~редакцию 03.04.19}}

%\vspace*{8pt}

%\pagebreak

\newpage

\vspace*{-28pt}

%\hrule

%\vspace*{2pt}

%\hrule

%\vspace*{-2pt}

\def\tit{ARCHITECTURAL DECISIONS IN~THE~PROBLEM 
OF~IDENTIFICATION OF~FRAUD IN~THE~ANALYSIS 
OF~INFORMATION FLOWS IN~DIGITAL ECONOMY\\[-5pt]}


\def\titkol{Architectural decisions in~the~problem 
of~identification of~fraud in~the~analysis 
of~information flows in~digital economy}

\def\aut{A.\,A.~Grusho, M.\,I.~Zabezhailo, N.\,A.~Grusho, and~E.\,E.~Timonina}

\def\autkol{A.\,A.~Grusho, M.\,I.~Zabezhailo, N.\,A.~Grusho, and~E.\,E.~Timonina}

\titel{\tit}{\aut}{\autkol}{\titkol}

\vspace*{-13pt}


 \noindent
   Institute of Informatics Problems, Federal Research Center ``Computer Sciences and 
Control'' of the Russian Academy of Sciences; 44-2~Vavilov Str., Moscow 119133, 
Russian Federation

\def\leftfootline{\small{\textbf{\thepage}
\hfill INFORMATIKA I EE PRIMENENIYA~--- INFORMATICS AND
APPLICATIONS\ \ \ 2019\ \ \ volume~13\ \ \ issue\ 2}
}%
 \def\rightfootline{\small{INFORMATIKA I EE PRIMENENIYA~---
INFORMATICS AND APPLICATIONS\ \ \ 2019\ \ \ volume~13\ \ \ issue\ 2
\hfill \textbf{\thepage}}}

\vspace*{3pt}


   
     
   \Abste{An approach to a~research of some types of fraud in digital economy with the usage of relationships of 
cause and effect is formulated. In all types of the considered frauds, the discrepancy between the 
purposes of financial transactions and actual cost of achievement of these purposes
has to be observed. Data on 
transactions can be collected by observing information flows in which these transactions are reflected. 
The architecture of data collection and their analysis can be organized by means of the distributed 
ledgers with the centralized consensus that allows creating an analog of the electronic account book 
fixing financial and economic activity of subjects of digital economy in the region. 
   The methods of fraud identification considered are based on the contradictions 
between actions described in transactions and information, which is contained in plans, standards, 
precedents, etc. 
   The method based on a~simplified scheme of implementation of the abstract project is considered. 
For identification of contradictions, it is necessary to carry out the analysis from the effect to the cause, 
i.\,e., to look for anomalies in information describing the generation of the observed effects. 
   It is shown how in implementation of the project it is possible to allocate simple ``necessary 
conditions'' of violation of cause and effect relationships, i.\,e., a~set of ``necessary conditions'' 
violation of which demonstrates fraud existence. It is possible to call this set of "necessary conditions" 
by metadata for control of the project for fraud identification.} 
   
   \KWE{digital economy; information flows; relationships of reason and effect; detection of 
fraudulent schemes}
   
  

 \DOI{10.14357/19922264190204}

\vspace*{-20pt}

 \Ack
   \noindent
   The work was partially supported by the Russian Foundation for Basic Research (projects  
18-29-03081 and 18-07-00274).



%\vspace*{6pt}

  \begin{multicols}{2}

\renewcommand{\bibname}{\protect\rmfamily References}
%\renewcommand{\bibname}{\large\protect\rm References}

{\small\frenchspacing
 {\baselineskip=10.5pt
 \addcontentsline{toc}{section}{References}
 \begin{thebibliography}{9}
\bibitem{1-gr-1}
\Aue{Grusho, A.\,A., A.\,A.~Zatsarinny, and E.\,E.~Timonina.} 2019. Blokcheyny tsifrovoy ekonomiki 
na baze sistemy situatsionnykh tsentrov i~tsentralizovannogo konsensusa [Blockchains of digital 
economy on the basis of the system of the situational centres and the centralized consensus]. 
\textit{25th Scientific and Technical Conference (International) ``Radar-Location, Navigation, 
Communication'' Proceedings}. Voronezh: VSU Publs. 6:183--191.
\bibitem{2-gr-1}
\Aue{Grusho, A., A.~Zatsarinny, and E.~Timonina.} 2019 (in press). 
A~system approach to information security 
in distributed ledgers on the situational centers platform. 
Lecture notes in computer science ser. Springer.
\bibitem{3-gr-1}
\Aue{Finn, V.\,K.} 2011. \textit{Iskusstvennyy intellekt: Metodologiya, primeneniya, filosofiya} 
[Artificial intelligence: Methodology, applications, philosophy]. Moscow: KRASAND. 448~p.

\bibitem{5-gr-1}
\Aue{Anshakov, O.\,M., and E.\,F.~Fabrikantova}. 2009. \textit{DSM-metod avtomaticheskogo porozhdeniya gipotez: Logicheskie 
i~epistemologicheskie osnovaniya} [JSM-method of automatic hypothesis generation: Logical and 
epistemological]. Moscow: KD LIBROKOM. 432~p.
\bibitem{4-gr-1} %5
\Aue{Poelmans, J., P.~Elzinga, S.~Viaene, and G.~Dedene.} 2010. Formal concept analysis in 
knowledge discovery: A~survey. \textit{Conceptual structures: From information to intelligence}. 
Eds.\ M.~Croitoru, S.~Ferr$\acute{\mbox{e}}$, and D.~Lukose. Lecture notes in 
computer science ser. Berlin--Heidelberg: Springer. 6208:139--153.

\bibitem{6-gr-1}
\Aue{Pankratov, E.\,S., and V.\,K.~Finn}. 
2009. \textit{Avtomaticheskoe porozhdenie gipotez v~intellektual'nykh 
sistemakh} [Automatic hypotheses generation in intelligent systems]. Moscow: KD 
\mbox{LIBROKOM}.  528~p. 
\bibitem{7-gr-1}
\Aue{Denisov, A.\,A., and D.\,N.~Kolesnikov.} 1982. \textit{Teoriya bol'shikh 
sistem upravleniya} [Theory of big control systems]. Leningrad: Energoizdat. 488~p.

\bibitem{9-gr-1}
\Aue{Grusho, A.\,A., N.\,A.~Grusho, M.\,I.~Zabezhailo, D.\,V.~Smirnov, and 
E.\,E.~Timonina.} 2018. 
Parametrizatsiya v~prikladnykh zadachakh poiska empiricheskikh prichin 
[Parametrization in applied 
problems of search of the empirical reasons]. 
\textit{Informatika i~ee Primeneniya~--- 
Inform. Appl.} 12(3):62--66.

\bibitem{8-gr-1}
\Aue{Grusho, A.\,A., N.\,A.~Grusho, M.\,V.~Levykin, and E.\,E.~Timonina.} 2018. Metody 
identifikatsii zakhvata khosta v~raspredelennoy informatsionno-vychislitel'noy sisteme, 
zashchishchennoy s~pomoshch'yu metadannykh [Methods of identification of host capture 
in the  distributed information system which is protected on the base of meta data].
\textit{Informatika i~ee 
Primeneniya~--- Inform. Appl.} 12(4):41--45.
{ %\looseness=1

}

\end{thebibliography}

 }
 }

\end{multicols}

\vspace*{-12pt}

\hfill{\small\textit{Received April 3, 2019}}

%\pagebreak

%\vspace*{-18pt}

\Contr

\noindent
\textbf{Grusho Alexander A.} (b.\ 1946)~--- Doctor of Science in physics and 
mathematics, professor, principal scientist, Institute of Informatics Problems, 
Federal Research Center ``Computer Sciences and Control'' of the Russian 
Academy of Sciences; 44-2~Vavilov Str., Moscow 119133, Russian Federation; 
\mbox{grusho@yandex.ru} 

\vspace*{3pt}

\noindent
\textbf{Zabezhailo Michael I.} (b.\ 1956)~--- Doctor of Science in physics and 
mathematics, principal scientist, Institute of Informatics Problems, Federal Research 
Center ``Computer Sciences and Control'' of the Russian Academy of Sciences;  
44-2~Vavilov Str., Moscow 119133, Russian Federation; 
\mbox{m.zabezhailo@yandex.ru} 

\vspace*{3pt}


\noindent
\textbf{Grusho Nikolai A.} (b.\ 1982)~--- Candidate of Science (PhD) in physics 
and mathematics, senior scientist, Institute of Informatics Problems, Federal 
Research Center ``Computer Sciences and Control'' of the Russian Academy of 
Sciences; 44-2~Vavilov Str., Moscow 119133, Russian Federation; 
\mbox{info@itake.ru} 

\vspace*{3pt}


\noindent
\textbf{Timonina Elena E.} (b.\ 1952)~--- Doctor of Science in technology, 
professor, leading scientist, Institute of Informatics Problems, Federal Research 
Center ``Computer Sciences and Control'' of the Russian Academy of Sciences;  
44-2~Vavilov Str., Moscow 119133, Russian Federation; 
\mbox{eltimon@yandex.ru} 

\label{end\stat}

\renewcommand{\bibname}{\protect\rm Литература}   %6
\def\stat{basok}

\def\tit{ИСПОЛЬЗОВАНИЕ ВЕРОЯТНОСТНОЙ МОДЕЛИ ВЫЧИСЛЕНИЙ 
ДЛЯ~ТЕСТИРОВАНИЯ ОДНОГО КЛАССА ГОТОВЫХ К~ИСПОЛЬЗОВАНИЮ 
ПРОГРАММНЫХ КОМПОНЕНТОВ ЛОКАЛЬНЫХ И~СЕТЕВЫХ СИСТЕМ$^*$}

\def\titkol{Использование вероятностной модели вычислений 
для~тестирования одного класса %готовых к~использованию 
программных компонентов} % локальных и~сетевых систем}

\def\aut{Б.\,М.~Басок$^1$, В.\,Н.~Захаров$^2$, С.\,Л.~Френкель$^3$}

\def\autkol{Б.\,М.~Басок, В.\,Н.~Захаров, С.\,Л.~Френкель}

\titel{\tit}{\aut}{\autkol}{\titkol}

\index{Басок Б.\,М.}
\index{Захаров В.\,Н.}
\index{Френкель С.\,Л.}
\index{Basok B.\,M.}
\index{Zakharov V.\,N.}
\index{Frenkel S.\,L.}




{\renewcommand{\thefootnote}{\fnsymbol{footnote}} \footnotetext[1]
{Работа выполнена при частичной финансовой поддержке 
РФФИ (проекты 18-07-00669, 18-29-03100).}}


\renewcommand{\thefootnote}{\arabic{footnote}}
\footnotetext[1]{МИРЭА~--- Российский технологический университет, VM\_E@mail.ru}
  \footnotetext[2]{Федеральный исследовательский центр <<Информатика и~управ\-ле\-ние>>
Российской академии наук,  \mbox{VZakharov@ipiran.ru}}
\footnotetext[3]{Институт проблем информатики Федерального исследовательского центра <<Информатика 
  и~управ\-ле\-ние>> Российской академии наук, \mbox{fsergei51@gmail.com}}

\vspace*{-8pt}

    
      
    
     
     \Abst{Обсуждается и~анализируется возможность обеспечения эффективного 
тестирования готовых к~использованию программных продуктов (ГИПП), решающих 
задачи вычисления функций, в~условиях отсутствия полной информации, необходимой 
для традиционного тестирования. Под эффективностью понимается возможность 
обеспечения сколь угодно высокой вероятности обнаружения возможных ошибок 
вычислений, не выявленных при выходном контроле, по мере роста чис\-ла проверок. 
В~качестве концептуальной модели предлагается использовать свойства функций со 
случайной самоприводимостью (random self-reducible function~--- RSR), т.\,е.\ функций, 
вычисление которых на конкретном входном наборе можно свести к~вы\-чис\-ле\-нию на 
нескольких случайно выбранных входных наборах.
     Обосновывается рациональность обеспечения свойств са\-мо\-тес\-ти\-ру\-емости 
в~коммерческих вычислительных ГИПП.}
     
     \KW{тестирование программ; самотестирование}
     
\DOI{10.14357/19922264180407}
  
\vspace*{-4pt}


\vskip 10pt plus 9pt minus 6pt

\thispagestyle{headings}

\begin{multicols}{2}

\label{st\stat}
     
\section{Введение}

    При приобретении ГИПП перед пользователем стоит задача убедиться в~его полной 
исправности, в~его способности без ошибок выполнять те функции, которые 
указаны в~прилагаемой к~нему документации. Для этого непосредственно 
перед эксплуатацией программный продукт (ПП) должен быть тщательно 
протестирован пользователями или передан для этого фирме, 
специализирующейся на тестировании~ПП.
    
    Данные действия должны осуществляться независимо от того, 
прилагаются к~ГИПП тесты разработчика или они отсутствуют. В~первом 
случае\linebreak следует убедиться, что ПП работает и~с другими данными, выполнить 
проверку наиболее сложных и~труднореализуемых функций, особенно тех, 
ошибки в~которых могут привести к~непредсказуемым последствиям. Во 
втором случае тестирование, безусловно, необходимо, поскольку только 
в~этом случае можно убедиться в~ра\-бо\-то\-спо\-соб\-ности ГИПП и~его 
возможностях.
    
        Технология тестирования ГИПП отличается от технологии 
тестирования ПП на этапе его разработки и~имеет свои особенности. Эти 
особенности подробно были разобраны в~работе~[1]. Среди данных 
особенностей можно в~первую очередь выделить следующие:
    \begin{itemize}
    \item доступ к~коду ГИПП для пользователей и~тестеров в~подавляющем 
большинстве случаев исключен. Также отсутствует возможность 
непосредственного контакта с~разработчиками ПП. Поэтому при 
тестировании используется метод <<черного ящика>> (BB~--- Black Box), 
основанный на системном функциональном или поведенческом 
тестировании. Кроме того, отсутствуют возможности создания 
и~эксплуатации модульных тестов и~тестов проверки межмодульных связей 
(интеграционных тестов);    
    \item ошибок в~программе существенно меньше, чем было на этапе ее 
разработки. По данным, приводимым в~[2], это число составляет примерно 
0,1--0,3~ошибки на~1000~строк кода;
    \item  оставшиеся в~ГИПП ошибки труднее выявляются, поскольку 
большинство ошибок было уже найдено при тестировании программы на 
этапе ее разработки и~естественно предположение о~том, что оставшиеся 
ошибки <<труд\-но\-тес\-ти\-ру\-емые>> (hard to test~[3]). Цена указанных ошибок 
достаточно велика, существенно выше цены найденных ранее ошибок 
примерно на~1--2~порядка. Поэтому от тестеров требуется повышенное 
внимание к~организации тестирования ГИПП перед началом его 
эксплуатации.
    \end{itemize}
    
    Обычно тесты~--- это последовательность шагов, каждый из которых 
содержит описание действий, предпринимаемых тестером, и~описание 
ожи\-да\-емых результатов, используемых при тестировании в~качестве 
эталонов. Разработка тестов ведется тестером на основе изучения работы 
ГИПП и~эксплуатационной документации.
    
    В связи с~тем что тестер может смоделировать лишь небольшую часть 
функций программы (так называемое остаточное моделирование)~\cite{7-bf}, 
возможны ошибки при задании как входных данных, так и~ожидаемых 
результатов (эталонных значений), получаемых вручную. Эти ошибки 
проявляются при реализации процедур сравнения полученных результатов 
тестирования и~эталонных данных и~принятия на их основе решений 
о~наличии ошибок в~ПП, называемых в~специальной литературе 
оракулами~\cite{4-bf}. При этом основную часть этих ошибок составляют 
ошибки, связанные с~заданием эталонов, поскольку их получение 
осуществляется тестером, как правило, вручную~\cite{2-bf}.
    
    Кроме того, в~связи с~тем что входные данные тестов ПП, как правило, 
представляют собой искусственно подобранные наборы данных, нет 
гарантии, что они достаточно полно проверяют все особенности ПП. В~то же 
время при выполнении реальных задач с~произвольными данными можно 
обнаружить некоторые отказы в~программе, поскольку при их выполнении 
осуществляется максимальный охват программного кода. Например, при 
тестировании текстового редактора реальные\linebreak
 тексты могут быть не менее 
полезными, чем специально подобранные буквосочетания. Однако 
определение ожидаемых результатов при обработке\linebreak
 программой реальных 
данных, особенно при реализации сложных вычислительных процессов, 
оказывается достаточно трудоемкой процедурой и,~как следствие этого, 
с~возможными ошибками.
    
    Таким образом, задача получения правильных эталонных данных при 
тестировании ГИПП является достаточно сложной и~актуальной.
    
    В предлагаемой статье рассматривается ряд принципиальных 
трудностей практики тестирования, связанных с~отсутствием полной 
спецификации программы и~информации о тестах разработчика, об 
эталонных значениях в~частности.
    
    Будет показано, что для программ, реа\-ли\-зу\-ющих функции, обладающие 
свойствами случайной самоприводимости 
(RSR~\cite{5-bf}), т.\,е.\ функций, вычисление которых на конкретном 
входном наборе можно свести к~вы\-чис\-ле\-нию на случайно выбранных 
входных наборах, часть трудностей, связанных с~отсутствием эталонов, 
можно преодолеть.
    
    Заметим, что вопрос использования свойств RSR далеко не новый, он 
широко обсуждался в~начале 1990-х~гг.\ в~рамках задач  
са\-мо\-тес\-ти\-ро\-ва\-ния/са\-мо\-про\-ве\-ря\-емости  
(self-testing/self-checking) программ~\cite{6-bf}. Принимая во внимание опыт 
по практическому тестированию программ~[1, 2], авторы считают полезным 
рассмотреть применение данной модели к~проблеме тестирования готовых 
к~использованию программ, особенно в~связи с~отмеченной проблемой 
обеспечения тестера информацией об эталонных значениях вычисляемых 
функций.

\vspace*{-6pt}
    
\section{Основные подходы к~тестированию готового к~использованию программного~продукта}

\vspace*{-2pt}
    
    Подходы к~тестированию ГИПП зависят в~основном от того, для 
решения каких задач они используются. Возможны следующие ситуации:
    \begin{enumerate}[(1)]
    \item ГИПП используется как продукт, предназначенный для 
самостоятельного решения конкретной прикладной задачи. В~этом случае 
тестирование состоит в~проверке правильности (согласно некоторому 
протоколу) его функционирования в~конкретной операционной среде 
и~в~проверке правильности решения задачи на рабочих данных;
    \item ГИПП представляет собой некоторый пакет программ, 
используемый для решения различных задач в~связанных прикладных 
областях (например, САПР (система автоматизированного
проектирования) для стро\-и\-тель\-но-мон\-таж\-ных работ);
    \item ГИПП должен использоваться как компонент некоторой сложной 
разрабатываемой программной или программно-аппаратной системы, 
в~частности в~виде сервисов в~сер\-вис-ори\-ен\-ти\-ро\-ван\-ных 
архитектурах.
    \end{enumerate}
    
    Основные современные подходы к~разработке тестов для различных 
вариантов данного списка задач можно разделить на методы <<черного 
ящика>>  и~инжекции ошибок (который может включать в~себя и~метод 
<<мутаций>> кодов и~входных данных~\cite{8-bf}).

\vspace*{-3pt}
    
    \subsection{Метод <<черного ящика>>}
    
    \vspace*{-1pt}
    
    Для тестирования методом BB~\cite{9-bf} требуется исполняемый 
компонент, входные данные и~<<оракул>>, представляющий собой 
спецификацию поведения компонента при разных входных данных, 
в~частности эталонные значения выходов (результатов работы) программы.
    
    Поскольку исходный код ГИПП, как правило, у~пользователя 
отсутствует, единственный источник информации для тестирования~--- это 
записи (<<журналы>>, <<лог-фай\-лы>>) различных активностей при 
выполнении программы на целевой платформе на некоторых тестовых 
входах, на которых значения вычисляемых функций должны играть роль 
эталонных значений при обнаружении ошибок рассматриваемым тестом.
 %   
    Однако при этом часто возникают указанные выше трудности, 
связанные с~отсутствием информации об истинных эталонных выходных 
значениях и~их связи с~конкретным операционным профилем, который может 
меняться в~зависимости от модификации операционной среды или от 
специфики работы рассматриваемой программы в~составе конкретной  
сер\-вис-ори\-ен\-ти\-ро\-ван\-ной архитектуры.
    
    Возможным выходом из этой ситуации считают использование 
исполняемой версии приложения с~аналогичной функциональностью, часть 
результатов которой можно принять за эталонные для новой  
ГИПП~\cite{2-bf, 11-bf}. Это соответствует ситуации, когда существуют 
прежние версии новой ГИПП или приложения, которые можно 
рассматривать как подобные рассматриваемой новой ГИПП (с точки зрения 
результатов вычислений). Часто, однако, такие версии не доступны 
приобретателю ГИПП.

\vspace*{-3pt}
    
    \subsection{Инжекция возможных программных ошибок 
и~системных сбоев}

\vspace*{-1pt}
    
    Под инжекцией ошибок понимают введение в~код программы на том 
или ином уровне представления (от исходного до исполнимого кода) 
искажений, которые могут приводить к~тем или иным видам некорректного 
поведения~\cite{8-bf}.
    
    Существует множество типов инжекций, которые могут отражать либо 
некоторые гипотетические сбои в~операционной системе (включая 
вызванные сбоями в~аппаратной части системы), либо списки возможных 
ошибок в~исходном коде программы.
    
    Очевидно, что ввиду отсутствия у пользователя ГИПП исходного кода 
второй способ инжекций интереса не представляет. Реальным приложением 
инжекций может быть имитация режимов отказа системы, таких как 
искажение данных операционной среды, повреждение данных, проходящих 
между вызовами компонент.
    
    В~\cite{10-bf} описана система инжекции ошибок при выполнении 
программы (runtime binary injection tool).
    %
    Этот инструмент имитирует ошибки в~запросах в~вызываемых функциях 
и~возвращает значения выполняемых при этом функций. Общий принцип 
использования таких инструментов основан на предположении, что тесты, 
обнаруживающие инжектированные ошибки во время выполнения 
программы, с~высокой вероятностью обнаружат также и~ошибки 
в~операторах исходного кода~\cite{11-bf}, результатом которых может быть, 
например, неправильная обработка того или иного системного сбоя. 
Следовательно, входные данные, определенные с~использованием 
инструмента инжекции как тестовые (т.\,е.\ обнаруживающие по результатам 
выполнения программы возможные ошибки), могут использоваться для 
тестирования ГИПП в~эксплуатационных условиях (например, при 
проведении регламентных работ).
    
    Трудности использования данного подхода связаны прежде всего 
с~высокой стоимостью систем инжекции, а также высокими требованиями 
к~квалификации тестера.
   % 
    При этом, разумеется, остается проблема определения правильных 
выходов выполняемых функций и~программы в~целом, используемых 
в~качестве эталонов при оценке результатов тестирования на специфических 
тестах.

\vspace*{-6pt}
    
\section{О~возможности тестирования программ с~неизвестной 
структурой и~не~полностью специфицированными функциями}

\vspace*{-2pt}
    
    Существенной трудностью тестирования ГИПП является отсутствие 
точных значений выходов программы при конкретных условиях ее 
выполнения (<<оракулов>>, в~терминах современной практики тес\-ти\-ро\-ва\-ния 
программ~\cite{4-bf}). Интересно было бы рассмотреть вопрос 
о~принципиальной воз\-мож\-ности решения этой проблемы.
    
    Представляется, что в~качестве теоретической и~концептуальной основы 
тестирования при неполной спецификации программы можно было бы\linebreak
 взять 
концепцию тестирования случайных самопри\-во\-ди\-мых функций, 
применяемую в~тео\-рии самотестируемых программ. Поскольку 
самотестируемость означает, что программа сама может определить отличие 
правильного результата от неправильного, естественно использовать 
правильное значение как эталон в~задаче внешнего (т.\,е.\ с~выбором 
исходных данных самим пользователем программы) тестирования.
    
    Функция является случайной самоприводимой на некотором множестве, 
если ее значение в~данной точке может быть эффективно реконструировано 
из ее оценки в~случайных точках~\cite{5-bf, 13-bf}.
    
    \smallskip
    
    \noindent
    \textbf{Определение.}\ Функция $f(x)$, определенная над 
множеством~$D$, называется случайной самоприводимой функцией, если 
существует функция~$\varphi$ и~множество функций $\sigma_1, \ldots , 
\sigma_n$ таких, что
    \begin{equation}
    f(x)=\varphi\left(x, r, f\left( \sigma_1\left(x,r\right)\right),\ldots , 
f\left(\sigma_k\left( x,r\right)\right)\right)\,,
    \label{e1-bs}
    \end{equation}
   где случайные переменные $r\hm=\{r_1, \ldots , r_m\}$ имеют известные 
распределения, а~функция~$\varphi$ и~множество функций $\sigma_1,\ldots , 
\sigma_k$ могут быть вычислены за полиномиальное время.

    Иными словами, если для функции $f(x)$ оценку ее значения на любом 
входе~$x$ можно свести за полиномиальное время к~оценке ее значения на 
одном или более случайных экземплярах входных переменных, то такая 
функция является самоприводимой.
    
    По сути, можно говорить о~\textit{восстановлении} значения функции 
$f(x)$ по случайно выбранным входным переменным $\{r\}$, так как это 
свойство позволяет восстановить значение функции, используя конечное 
число элементов, взятых из ее области определения, без ка\-ких-ли\-бо 
знаний о~реализации программы, которая выполняет вычисление.
    
    Например, пусть программа~$A$ должна вы\-чис\-лять линейную функцию 
$f(x) \hm= wx$, где $w$~--- любое действительное число (кроме~0), 
и~вы\-чис\-ле\-ни\-ям функций $\sigma_1,\ldots ,\sigma_4$ соответствуют вызовы 
программы:
    $A\left(w-r_1, x-r_2\right)$, 
    $A\left(w-r_1, r_2\right)$, $A\left(r_1, x-r_2\right)$ и~$A\left(r_1, 
r_2\right)$,
где $r_1$ и~$r_2$~--- равномерно распределенные в~области задания 
переменных~$w$ и~$x$ случайные величины.

    Если программа правильно вычисляет функцию с~указанными 
аргументами, то из приведенного ниже очевидного тождества следует:

\vspace*{-5pt}

\noindent
    \begin{multline}
    y_c = A\left(w-r_1, x-r_2\right) + A\left(w-r_1, r_2\right) + {}\\
    {}+A\left(r_1, x-r_2\right) + 
A\left(r_1, r_2\right) = \left(w-r_1\right)\left(x-r_2\right) + {}\\
{}+ \left(w-r_1\right)r_2 + r_1\left(x-r_2\right) + 
r_1r_2 = wx\,,
    \label{e2-bf}
    \end{multline}
    что демонстрирует выполнение условий определения RSR при любых 
случайных~$r_1$ и~$r_2$.

    В данном примере функции~$\varphi$ в~(1) соответствует сумма 
функций $\sigma_1, \ldots, \sigma_4$ с~соответствующими аргументами.
    
    Если каждый вызов программы выполняется с~небольшой вероятностью 
ошибки $\leq\alpha$ (скажем,~0,01), где $\alpha$~--- это доля входных 
наборов, на которых возможны ошибки вычисления (гарантированные, 
например, разработчиком программы), то вероятность ошибки вычисления 
суммы будет $\leq 4\alpha$ и,~соответственно, большинство значений~$y_c$ 
при достаточном большом числе случайно выбранных~$r_1$ и~$r_2$ будут 
равны истинному значению~$wx$, что означает возможность определения 
правильного значения выхода программы (эталона) как значения, которое 
получается на большинстве входных наборов программы, вычисляющей 
данную функцию.
    
    Таким образом, для реализации данного подхода необходимо выбрать 
статистически обоснованное число пар~$r_1, r_2$, чтобы обеспечить 
требуемую вероятность правильного вычисления с~достаточным уровнем 
доверия (величиной доверительного интервала~\cite{13-bf}).
 %   
    Для этого, предполагая, что последовательность правильных 
результатов образует последовательность Бернулли (поскольку случайные 
переменные~$r_1$ и~$r_2$~--- независимые, равномерно распределенные числа), 
получим вероятность того, что значение функции, получаемое на 
большинстве из~$n$~входных наборов переменных (т.\,е.\ $\geq [n/2]\hm+1$), 
есть точное значение функции:

\noindent
    \begin{equation}
    \mathrm{Pr} \left( k\geq L+1\right)=1-\sum\limits^L_{k=0} C_n^k p^k 
q^{n-k}\,,
    \label{e3-bf}
    \end{equation}
где $p>1/2$~--- вероятность успеха (правильного вычисления функции, 
$1\hm- 4\alpha$ в~рассматриваемом примере); $q\hm = 1 \hm- p$; $k$~--- 
число правильных ответов в~данной последовательности~$n$ испытаний, 
$L\hm= \lfloor n/2\rfloor\hm+1$.

    Границы этой вероятности определяются различными формами 
неравенства Чернова (Chernoff inequality), например:

\noindent
    \begin{equation}
    \mathrm{Pr}\left( k>\fr{n}{2}\right) \geq 1-\exp \left( -2n\left( p-
\fr{1}{2}\right)^2\right)\,,
    \label{e4-bf}
    \end{equation}
где~$k$, как и~выше,~--- число успешных исходов в~$n$~испытаниях Бернулли.

    Очевидно, что с~ростом числа испытаний~$n$ растет вероятность того, 
что большинство (больше половины всех исходов) будут успешными (т.\,е.\ 
функция будет вычислена правильно).
    
    Задавая требуемый статистический уровень доверия $1\hm-
\beta$~\cite{13-bf} (например,~0,99, что означает, что вероятность $\mathrm{Pr}\,(k\hm> 
n/2)$ на конечной выборке испытаний длиной~$n$ в~рассматриваемом 
примере будет оценена с~вероятностью ошибки не более~0,01), получают 
число испытаний Бернулли (т.\,е.\ число тестовых прогонов для вычисления 
функции), необходимых для определения вероятности большинства 
правильных ответов при данном уровне~$\beta$~\cite{12-bf, 19-bf}:

\noindent
    \begin{equation}
    n\geq  \fr{\ln(1/\beta)}{(p-1/2)^2}\,.
    \label{e5-bf}
    \end{equation}
    
    Иными словами, $n$~--- это число случайных наборов, которые дают 
достаточную вероятность того, что большая часть результатов вычислений 
тестируемой функции $f(x)$ будет правильной и~можно определить этот 
результат просто по большинству ответов при данной вероятности~$\alpha$ 
ошибки выполнения программы~$A$.
    
    Формула~(\ref{e5-bf}) применима для оценки необходимого числа 
запусков программы, вычисляющей любые функции, представимые как~(1).
    
    В настоящее время доказано~\cite{14-bf, 15-bf}, что значительный класс 
вычислительных задач может быть представлен как вычисление функций со 
свойствами RSR. Сюда, например, относятся:
    \begin{itemize}
  \item  преобразование Фурье;
  \item матричные вычисления (например, перемножение матриц, 
построение обратной матрицы, вычисление определителей, вычисление 
перманента матрицы);
  \item вычисление тригонометрических, степенных, логарифмических 
функций;
  \item операции (умножение-деление) с~полиномами нескольких 
переменных;
  \item значительное число теоретико-числовых функций, используемых 
в~поисковых алгоритмах и~криптографии~\cite{15-bf}.
  \end{itemize}
  
  Оставляя вопрос об автоматизированном построении функций~$\varphi$ 
и~$\sigma_i$ ($i\hm = 1, \ldots, k$) в~формуле~(1), отметим, что в~настоящее 
время в~литературе приведены различные примеры~$\varphi$ и~$\{\sigma_1, 
\ldots , \sigma_k\}$ для всех перечисленных выше RSR-функ\-ций, т.\,е.\ 
с~практической точки зрения можно говорить о~возможности использования 
<<биб\-лио\-те\-ки>> таких функций~\cite{6-bf}.
  
    Например, в~\cite{14-bf} доказана следующая
    
    \smallskip
    
    \noindent
    \textbf{Теорема.}\ \textit{Для каждого набора$\{a_1, \ldots , a_{d+1}\}$ 
попарно различных элементов конечного поля~$F$ с~числом элементов $\vert 
F\vert \hm> d\hm+1$ существует конечное число чисел $\{c_1, \ldots , 
c_{d+1})$, таких что для каждого полинома $P(x)$ степени~$d$, 
определенного на~$F$, справедливо}
    \begin{equation*}
    \forall x, r\in F,\ P(x)=\sum\limits_{i=1}^{d+1} c_i P\left( x+a_i r\right)\,.
   % \label{e6-bf}
    \end{equation*}
    
    {В этом случае функции $\sigma_i\hm = x\hm+ a_ir$, где~$r$, как 
и~выше,~--- случайное число. Если $(a_1, \ldots, a_{d+1}) \hm= (1, 2, \ldots, 
d+1)$, то $c_i\hm = (-1)^{i+1}C^i_{d+1}$, где $C^i_{d+1}$~--- биномиальные 
коэффициенты.}
    
    %\smallskip
    
    Существенно, что единственное, что надо знать в~данном случае 
о~вычисляемой функции,~--- это то, что она полином степени~$d$.
    
    Учитывая, что большинство непрерывных функций вычисляется через 
их полиномиальное представление, можно говорить о широких 
возможностях использования RSR в~задачах тестирования.
    
    Заметим, что, поскольку формально вычисления в~любых аппаратных 
средах являются при\-ближенными (конечность разрядной сетки про\-цессоров), 
в~теории RSR-функ\-ций введено также\linebreak
 понятие <<приближенной случайной 
самоприводимости>> (approximate-random self-reducible)~\cite{12-bf} для 
функции, значение в~некоторой точке~$x$ которой может быть представлено 
по формуле~(1) лишь с~некоторой точностью~$\varepsilon$. С~формальной 
точки зрения это вполне естественно для вычислений с~фиксированной 
и~плавающей точкой.

    В~\cite{12-bf} приведены примеры выполнения~(1) при вычислении 
показательной и~логарифмической функций для $k$-бит\-ных чисел  
с~$l_m$-бит\-ной мантиссой и~$l_{\mathrm{exp}}$-бит\-ной экспонентой.
    
    Соответственно, все сказанное выше о~вы\-чис\-ле\-нии значений функций 
тестируемыми программами может быть переформулировано в~терминах 
вычисления приближенных значений.
    
    Итак, если поставщик ГИПП (функциональное назначение которой 
состоит в~вычислении тех или иных функций) гарантирует достаточно 
высокую вероятность правильной работы программы, вычисляющей 
требуемую функцию, то, организовав вычисление функции 
на~$n$~случайных наборах из области определения тестируемой функции, где~$n$ 
определяется согласно~(\ref{e5-bf}), с~высокой вероятностью можно 
определить неизвестный из спецификации правильный результат вычисления 
функции.
    
    Очевидно, что при необходимости многократ\-ного тестирования ГИПП, 
например для оценки\linebreak не\-об\-хо\-ди\-мости дальнейшей модификации программы, 
или при тестировании в~процессе регла\-мент\-ной проверки оборудования нет 
не\-об\-хо\-ди\-мости прогона на указанных случайных наборах, поскольку 
достаточно использовать полученные ранее значения выхода программ как 
эталонные.

\vspace*{-9pt}
    
\section{О самотестировании готового к~использованию программного продукта}

\vspace*{-3pt}

    Как отмечалось во введении, подход, основанный на теории  
RSR-функ\-ций, был первоначально предложен для задач самотестирования 
и~самокоррекции программ. Соответственно, он может быть использован 
при решении задач самотестирования и~ГИПП~\cite{12-bf}, поскольку для 
самопроверки нужен лишь перебор достаточного числа входных значений 
без использования эталонных значений работы программы. Иными словами, 
можно использовать известные в~теории и~практике самокоррекции 
программ методы выбора правильного решения по большинству результатов 
(например, N-version programming). Более того, ввиду роста вероятности 
правильного вычисления по мере роста числа~$n$~случайных входных 
наборов $\{r_1,\ldots , r_k\}$ (формулы~(\ref{e2-bf})--(\ref{e5-bf})), можно 
говорить о повышении надежности программы (этот эффект называют 
<<усилением>> (amplification)~\cite{16-bf}).
    
    Обычно используют два вида процедур са\-мо\-тес\-ти\-ро\-ва\-ния программ: 
следящее и~активное~\cite{4-bf, 17-bf}. При реализации первого вида 
самотестирования осуществляется проверка результатов обработки данных 
в~процессе эксплуатации. Примерами следящего самотестирования могут 
служить подстановка найденных корней в~уравнение и~оценка разности 
между левой и~правой частью, проверка правильности сортировки путем 
анализа выходного массива, вычисление и~сравнение контрольных\linebreak
 сумм 
и~т.\,д. Главным недостатком данного вида\linebreak самотестирования ПП является 
невозможность хранения эталонных данных работы программы, 
необходимых для анализа результатов ее работы и~диагностирования отказов 
программы.
    
    При активном самотестировании осуществляется проверка результатов 
обработки специальных тестовых наборов входных данных путем сравнения 
полученных результатов с~хранимыми в~программе эталонными данными.
    
    Активное самотестирование может быть полезно при оценке работы ПП 
с~различной аппаратурой или в~различных операционных средах, при 
использовании различных браузеров, при проверке правильности работы 
программы после обновления некоторых ее функций. Недостатком данного 
вида самотестирования является то, что работа с~ограниченным объемом 
фиксированных входных данных тестирования не позволяет оценить полноту 
тестирования.
    
    Самотестирование на основе предлагаемого в~статье подхода включает 
в~себя ряд достоинств как первого, так и~второго вида самотестирования 
и~в~то же время свободно от присущих им недостатков. Прежде всего это 
обусловлено отсутствием необходимости заранее вычислять ожидаемые 
выходы, и~тем самым исключаются ошибки разработчиков тестов при 
подготовке эталонных данных для самотестирования.

\vspace*{-9pt}
    
\section{Заключение}

\vspace*{-3pt}

    Специфика тестирования приобретенных у~производителя ПП
     (ГИПП) широко обсуждается в~современных публикациях по 
программному обеспечению (именуемая в~англоязычной литературе как 
проблема тестирования ПП Ready to Use Software 
Product (RUSP)~\cite{7-bf} или как Commercial off-the-shelf  
(COTS)~\cite{18-bf}).
    
    Одна из основных проблем при этом состоит в~отсутствии достаточно 
подробных спецификаций продаваемых программ, что создает трудности 
покупателю в~тестировании этих программ, которое он, например, обязан 
выполнить перед включением их как компонентов в~более сложные системы с~повышенными требованиями к~надежности (авиационные, медицинские 
и~т.\,п.).
    
    В данной статье рассмотрена возможность преодоления проблемы 
отсутствия надежных данных о вычисляемых приобретаемыми программами 
функциях, используя при этом свойства случайной самоприводимости 
значительной части функций и~дополняя тем самым функциональное 
тестирование техникой случайного перебора входных наборов, что позволяет 
вычислять эталонные значения программ с~высокой достоверностью.
    
    Данный подход может быть использован для значительного числа 
вычислительных задач, таких\linebreak
 как преобразование Фурье, матричные 
вы\-чис\-ления, операции с~полиномами нескольких пе\-ре\-менных, вычисление 
различных полиномов над\linebreak
 конечными полями, широко используемых при 
вычислении хеш-функ\-ций в~задачах поиска и~хранения  
информации~\cite{15-bf} и~функций шифрования, используемых 
в~компьютерных сетях для обеспечения кибербезопасности. 
%
Существенно, 
что свойства случайной самоприводимости обеспечивают так-\linebreak же
самотестирование и~самокоррекцию со\-от\-вет\-ст\-ву\-ющих программ, поскольку 
вычисление правильных значений функций (определяемых по\linebreak
 большинству 
используемых наборов) можно интерпретировать и~как под\-тверж\-де\-ние 
появления правильных результатов вычисления (тестирование), и~как 
ис\-прав\-ле\-ние (за счет использования <<правила большинства>>). А~если 
в~со\-ста\-ве предлагаемой на рынок программы будет преду\-смот\-рен генератор 
случайных чисел и~указанный механизм принятия решения, это может 
существенно снизить за\-тра\-ты на тес\-ти\-ро\-ва\-ние после приобретения 
(например, перед интегрированием в~более слож\-ные сис\-те\-мы) 
и,~соответственно, повысить их привлекательность для потенциального 
покупателя.

\vspace*{-8pt}
    
  {\small\frenchspacing
 {%\baselineskip=10.8pt
 \addcontentsline{toc}{section}{References}
 \begin{thebibliography}{99}
 
 \vspace*{-1pt}
 
     \bibitem{1-bf}
     \Au{Басок Б.\,М., Головин~С.\,А., Захаров~В.\,Н., Френкель~С.\,Л.} Тестирование 
готового к~использованию про\-грам\-много продукта~// ИТ-Стан\-дарт: Электронный 
научный журнал, 2018. №\,1. 7~с.
{\sf  
http://journal.tc22.ru/\linebreak wp-content/uploads/2018/05/testirovanie\_gotovogo\_k\_\linebreak 
ispolzovaniyu\_programmnogo\_produkta.pdf}
     \bibitem{2-bf}
     \Au{Липаев В.\,В.} Тестирование компонентов и~комплексов программ.~---  
Москва--Берлин: Директ-Медиа, 2015. 528~с.
     \bibitem{3-bf}
     7~Types of software errors, that every tester should know~// Software Testing Help, 2018. 
{\sf www.softwaretestinghelp.\linebreak com/types-of-software-errors}.

\bibitem{7-bf} %4
     ГОСТ Р ИСО/МЭК 25051-2017. Информационные технологии. Системная 
и~программная инженерия. Требования и~оценка качества систем и~программно\-го 
обеспечения (SQuaRE). Требования к~качеству готового к~использованию программного 
продукта (RUSP) и~инструкции по тестированию.~--- М.: Стандартинформ, 2017. 32~с.
     \bibitem{4-bf} %5
     \Au{Barr E.\,T., Harman~M., McMinn~P., Shahbaz~M., Yoo~S.} The oracle problem in 
software testing: A~survey~// IEEE T.~Software Eng., 2015. Vol.~41. No.\,5.  
P.~507--525.
     \bibitem{5-bf} %6
     \Au{Lipton R.} New directions in testing~// 
      Distributed computing and cryptography~/
      Eds.\ J.~Feigenbaum, M.\,J.~Merritt.~---
      DIMACS ser. in discrete mathematics and 
theoretical computer science.~--- AMS, 1991. 
  Vol.~2. P.~191--202.
     \bibitem{6-bf} %7
     \Au{Blum M., Luby M., Rubinfeld~R.} Self-testing/correcting with applications to 
numerical problems~// 22nd ACM Symposium on Theory of Computing Proceedings.~--- New 
York, NY, USA: ACM Press, 1990. P.~73--83.
     
     \bibitem{8-bf} %8
     \Au{Natella R., Cotroneo D., Duraes~J.\,A., Madeira~H.} On fault representativeness of 
software fault injection~// IEEE T.~Software Eng., 2013. Vol.~39. No.\,1. P.~80--96.
     \bibitem{9-bf} %9
     \Au{Canfora G., Di Penta~M.} Testing services and service-centric systems: Challenges 
and opportunities~// IT Prof., 2006. Vol.~8. No.\,2. P.~10--17.

\bibitem{11-bf} %10
     \Au{Buck D., Hollingsworth~J.} An API for runtime code patching~// Int. 
J.~High Perform. C., 2000. Vol.~14. No.\,4. P.~317--329.
     \bibitem{10-bf} %11
     \Au{Barrantes E.\,G., Ackley~D.\,H., Forrest~S., Palmer~T.\,S., Stefanovic~D., 
Zovi~D.\,D.} Randomized instruction set emulation to disrupt binary code injection attacks~// 
10th ACM Conference on Computer and Communications Security Proceedings.~--- New York, 
NY, USA: ACM Press, 2003. P.~281--289.

 \bibitem{13-bf} %12
     \Au{Смирнов Н.\,В., Дунин-Барковский~И.\,В.} Курс теории вероятностей 
и~математической статистики для технических приложений.~--- Л.: Наука, 1969. 512~с.
     
     \bibitem{12-bf} %13
     \Au{Gemmell P., Lipton R., Rubinfeld~R., Sudan~M., Wigderson~A.}  
Self-testing/correcting for polynomials and for approximate functions~// 23rd ACM Symposium 
on the Theory of Computing Proceedings.~--- New York, NY, USA: ACM Press, 1991.  
P.~32--43.

\bibitem{19-bf} %14
     \Au{Bhattacharyya A., Dey~P.} Sample complexity for winner prediction in elections~// 
arXiv.org, 2016. \mbox{arXiv}:\linebreak 1502.04354 [cs.DS]. 
    
    
     \bibitem{15-bf} %15
     \Au{Carter L., Wegman~M.} Universal hash functions~// J.~Comput. Syst. Sci., 1979. 
Vol.~18. P.~143--154.

 \bibitem{14-bf} %16
     \Au{Nouber G., Nussbauer~H.} Self-correcting polynomial programs~// Reliab. 
Comput., 1996. Vol.~2. No.\,2. P.~139--145.
     \bibitem{16-bf} %17
     \Au{Dolev Sh., Frenkel~S.} Extending the scope of self-correcting~// 13th Conference 
(International) on Applied Stochastic Models and Data Analysis Proceedings. P.~458--462.
     \bibitem{17-bf} %18
     \Au{Басок Б.\,М., Красовский~В.\,Е.} Тестирование про\-грам\-мно\-го обеспечения.~--- 
М.: МИРЭА, 2010. 120~с.
     \bibitem{18-bf} %19
     \Au{Voas J., Charron F., Miller~K.} Robust software interfaces: Can COTS-based 
systems be trusted without them?~// 15th Conference (International) on Computer Safety, 
Reliability and Security Proceedings.~--- Vienna: Springer Verlag, 1996. P.~126--135.
     
 \end{thebibliography}

 }
 }

\end{multicols}

\vspace*{-3pt}

\hfill{\small\textit{Поступила в~редакцию 10.06.18}}

\vspace*{8pt}

%\pagebreak

%\newpage

%\vspace*{-28pt}

\hrule

\vspace*{2pt}

\hrule

%\vspace*{-2pt}

\def\tit{USING A PROBABILISTIC CALCULATION MODEL TO~TEST\\ ONE CLASS 
OF~READY-TO-USE SOFTWARE COMPONENTS\\ OF~LOCAL AND~NETWORK SYSTEMS}

\def\titkol{Using a probabilistic calculation model to test one class of 
ready-to-use software components of local and network systems}

\def\aut{B.\,M.~Basok$^1$, V.\,N.~Zakharov$^2$, and~S.\,L.~Frenkel$^3$}

\def\autkol{B.\,M.~Basok, V.\,N.~Zakharov, and~S.\,L.~Frenkel}

\titel{\tit}{\aut}{\autkol}{\titkol}

\vspace*{-11pt}


\noindent
$^1$MIREA~--- Russian Technological University, 78~Vernadskogo Ave., Moscow 119454, Russian 
Federation

\noindent
$^2$Federal Research Center ``Computer Science and Control'' of the Russian Academy of Sciences, 
44-2~Vavilov\linebreak
$\hphantom{^1}$Str., Moscow 119333, Russian Federation

\noindent
$^3$Institute of Informatics Problems, Federal Research Center ``Computer Sciences and Control'' of the 
Russian\linebreak
$\hphantom{^1}$Academy of Sciences, 44-2~Vavilov Str., Moscow 119333, Russian Federation


\def\leftfootline{\small{\textbf{\thepage}
\hfill INFORMATIKA I EE PRIMENENIYA~--- INFORMATICS AND
APPLICATIONS\ \ \ 2018\ \ \ volume~12\ \ \ issue\ 4}
}%
 \def\rightfootline{\small{INFORMATIKA I EE PRIMENENIYA~---
INFORMATICS AND APPLICATIONS\ \ \ 2018\ \ \ volume~12\ \ \ issue\ 4
\hfill \textbf{\thepage}}}

\vspace*{6pt}
    
    


\Abste{The paper discusses and analyzes the possibility of 
providing effective testing of ready-to-use software products that solve 
the task of calculating functions, in the absence of complete information 
necessary for traditional testing. Efficiency means the possibility of 
providing an arbitrarily high probability of detecting possible computational 
errors that were not detected by the output control as the number of 
inspections increases. As a~conceptual model of the proposed approach, 
the properties of functions with the random self-reducible function  
are used, that is, functions whose calculation on a~particular input set can 
be reduced to calculation on several randomly selected input sets.
The rationality of providing self-testability properties in ready-to-use 
software is substantiated.}

\KWE{software testing; self-testing}
    
\DOI{10.14357/19922264180407}

%\vspace*{-18pt}

\Ack
\noindent
The work was partly supported by
the Russian Foundation for Basic Research (projects 18-07-00669 and 18-29-03100).



%\vspace*{-3pt}

  \begin{multicols}{2}

\renewcommand{\bibname}{\protect\rmfamily References}
%\renewcommand{\bibname}{\large\protect\rm References}

{\small\frenchspacing
 {%\baselineskip=10.8pt
 \addcontentsline{toc}{section}{References}
 \begin{thebibliography}{99}
 


\bibitem{1-bf-1}
\Aue{Basok, B.\,M., S.\,A.~Golovin, V.\,N.~Zakharov, and S.\,L.~Frenkel.} 2018. 
Testirovanie gotovogo k ispol'zovaniyu programmnogo produkta [Testing of 
ready-to-use software product]. \textit{It-standart: Elektronnyy nauchnyy zh.} 
 [IT-Standart: Electronic Scientific~J.]  1. 7~p. Available at: {\sf  
http://journal.tc22.ru/wp-content/uploads/2018/05/\linebreak 
testirovanie\_gotovogo\_k\_ispolzovaniyu\_programmnogo\_\linebreak produkta.pdf} (accessed 
October~30, 2018).
\bibitem{2-bf-1}
\Aue{Lipaev, V.\,V.} 2015. \textit{Testirovanie komponentov i~kompleksov program} 
[Testing of components and software packages].  Moscow--Berlin: Direkt-Media. 
528~p.
\bibitem{3-bf-1}
\Aue{7~Types of software errors, that every tester should know}. \textit{Software 
Testing Help}. Available at: {\sf  
www.\linebreak softwaretestinghelp.com/types-of-software-errors} (accessed October~30, 
2018).

\bibitem{7-bf-1} %4
GOST R~ISO/MEK 25051-2017. Informatsionnyye tekhnologii. Sistemnaya 
i~programmnaya inzheneriya. Trebovaniya i~otsenka kachestva sistem 
i~programmnogo obespecheniya (SQuaRE). Trebovaniya k~kachestvu gotovogo 
k~ispol'zovaniyu programmnogo produkta (RUSP) i~instruktsii po testirovaniyu 
[Information technology. System and software engineering. Requirements and 
quality assessment of systems and software (SQuaRE). Requirements for the 
quality of ready-to-use software product (RUSP) and instructions for testing]. 
Moscow: Standardinform Publs. 32~p.

\bibitem{4-bf-1} %5
\Aue{Barr, E.\,T., M.~Harman, P.~McMinn, M.~Shahbaz, and S.~Yoo.} 2015. 
The oracle problem in software testing: A~survey. \textit{IEEE T.~Software Eng.} 
41(5):507--525. 
\bibitem{5-bf-1} %6
\Aue{Lipton, R.} 1991. New directions in testing.  \textit{Distributed 
computing and cryptography}. Eds.\ J.~Feigenbaum and M.\,J.~Merritt.
DIMACS ser. in 
discrete mathematics and theoretical computer science.  AMS. 2:191--202.
\bibitem{6-bf-1} %7
\Aue{Blum, M., M.~Luby, and R.~Rubinfeld.} 1990. Self-testing/correcting with 
applications to numerical problems. \textit{22nd ACM Symposium on Theory of 
Computing Proceedings}. New York, NY: ACM Press. 73--83.

\columnbreak

\bibitem{8-bf-1}
\Aue{Natella, R., D.~Cotroneo, J.\,A.~Duraes, and H.~Madeira.} 2013. On fault 
representativeness of software fault injection. \textit{IEEE T.~Software Eng.} 
39(1):80--96.
\bibitem{9-bf-1}
\Aue{Canfora, G. and V.~Di~Penta.} 2006. Testing services and servi ce-centric 
systems: Challenges and opportunities. \textit{IT Prof.} 8(2):10--17.

\bibitem{11-bf-1} %10
\Aue{Buck, D., and J.~Hollingsworth.} 2000.  An API for runtime code patching.   
\textit{Int. J.~High Perform. C.} 14(4):317--329.

\bibitem{10-bf-1} %11
\Aue{Barrantes, E.\,G., D.\,Y.~Ackley, T.\,S.~Palmer, D.~Stefanovic, and 
D.~Zovi.} 2003. Randomized instruction set emulation to disrupt binary code 
injection attacks. \textit{10th ACM Conference on Computer and Communications 
Security Proceedings.} New York, NY: ACM Press. 281--289.

\bibitem{13-bf-1} %12
\Aue{Smirnov, N.\,V., and I.\,V.~Dunin-Barkovskiy.} 1969. \textit{Kurs teorii 
veroyatnostey i~matematicheskoy statistiki dlya tekhnicheskikh prilozheniy} 
[Course in the probabilities theory and mathematical statistics for technical 
applications]. Leningrad: Nauka. 512~p.
\bibitem{12-bf-1} %13
\Aue{Gemmell, P., R.~Lipton, R.~Rubinfeld, M.~Sudan, and A.~Wigderson.} 
1991. Self-testing/correcting for polynomials and for approximate functions. 
\textit{23rd ACM Symposium on the Theory of Computing Proceedings}. New 
York, NY: ACM Press. 32--43.
\bibitem{19-bf-1} %14
\Aue{Bhattacharyya, A., and P.~Dey.} 2016. Sample complexity for winner 
prediction in elections. \mbox{\textit{arXiv.org}}.  arXiv:\linebreak 1502.04354 [cs.DS].


\bibitem{15-bf-1}
\Aue{Carter, L., and M.~Wegman.} 1979. Universal hash functions. 
\textit{J.~Comput. Syst. Sci.} 18:143--154.

\bibitem{14-bf-1} %16
\Aue{Nouber, G., and H.~Nussbauer.} 1996. Self-correcting polynomial programs. 
\textit{Reliab. Comput.} 2(2):139--145.
\bibitem{16-bf-1} %17
\Aue{Dolev, Sh., and S.~Frenkel.} 2009. Extending the scope of self-correcting. 
\textit{13th Conference (International) on Applied Stochastic Models and Data 
Analysis Proceedings}. 458--462.
\bibitem{17-bf-1} %18
\Aue{Basok, B.\,M., and V.\,E.~Krasovskiy.} 2010. Testirovanie programmnogo 
obespecheniya [Software testing].  Moscow: MIREA. 120~p.
\bibitem{18-bf-1} %19
\Aue{Voas J., F.~Charron, and K.~Miller.} 1996.  Robust software interfaces: Can 
COTS-based systems be trusted without them? \textit{15th Conference 
(International) on Computer Safety, Reliability and Security Proceedings}. Vienna: 
Springer Verlag. 126--135.
  
\end{thebibliography}

 }
 }

\end{multicols}

\vspace*{-6pt}

\hfill{\small\textit{Received September 20, 2018}}

%\pagebreak

\vspace*{-24pt}
      
    
    \Contr
    
\noindent
\textbf{Basok Boris M.} (b.\ 1948)~--- Candidate of Science (PhD) in technology, associate professor, 
MIREA~--- Russian Technological University, 78~Vernadskogo Ave., Moscow 119454, Russian Federation; 
\mbox{VM\_E@mail.ru} 

%\vspace*{3pt}

\noindent
\textbf{Zakharov Victor N.} (b.\ 1948)~--- Doctor of Science in technology, associate professor; Scientific 
Secretary, Federal Research Center ``Computer Science and Control'' of the Russian Academy of Sciences, 
44-2~Vavilov Str., Moscow 119333, Russian Federation; \mbox{vzakharov@ipiran.ru}


%\vspace*{3pt}

\noindent
\textbf{Frenkel Sergey L.} (b.\ 1951)~--- Candidate of Science (PhD) in technology, associate professor, 
senior scientist, Institute of Informatics Problems, Federal Research Center ``Computer Sciences and Control'' 
of the Russian Academy of Sciences, 44-2~Vavilov Str., Moscow 119333, Russian Federation; 
fsergei51@gmail.com 

    
\label{end\stat}

\renewcommand{\bibname}{\protect\rm Литература}     %7
%\newcommand{\eol}{\end{enumerate}\setlength{\itemsep}{-\parsep}}
%\newcommand{\ang}[1]{\langle{#1}\rangle}
%\newcommand{\infinity}{\infty}
%\newcommand{\mess}[1]{\mbox{\tt #1}}
%\newcommand{\var}[1]{\mbox{\it #1}}
%\newcommand{\order}[1]{\stackrel{#1}\fa}
%\newcommand{\orderr}[1]{\stackrel{#1}\Longrightarrow}
%\newcommand{\infrel}[1]{\stackrel{#1}\Longrightarrow}
%\newcommand{\prog}{\mbox{\tt Prog}}
%\newcommand{\comment}[1]{}
%\newcommand{\set}[1]{\{#1\}}
%\newcommand{\pair}[2]{\langle #1,#2 \rangle}
%\newcommand{\remove}[1]{}
%\renewcommand{\qed}{\hfill\rule{2mm}{2mm}}
%\newcommand{\bull}[1]{\begin{itemize}\item{#1}\end{itemize}}
%\newcommand{\marg}[1]{\marginpar{\small #1}}


\renewcommand{\figurename}{\protect\bf Figure}
\renewcommand{\tablename}{\protect\bf Table}

\def\stat{frenkel}


\def\tit{SEAMLESS ROUTE UPDATES IN SOFTWARE-DEFINED NETWORKING 
VIA QUALITY OF~SERVICE COMPLIANCE VERIFICATION}

\def\titkol{Seamless route updates in software-defined networking via 
quality of service compliance verification}

\def\autkol{S.\,L.~Frenkel and~D.~Khankin}

\def\aut{S.\,L.~Frenkel$^1$ and~D.~Khankin$^2$}

\titel{\tit}{\aut}{\autkol}{\titkol}

%{\renewcommand{\thefootnote}{\fnsymbol{footnote}}
%\footnotetext[1] {The 
%research of Yuri Kabanov was done under partial financial support of the grant 
%of RSF No.\,14-49-00079.}}

\renewcommand{\thefootnote}{\arabic{footnote}}
\footnotetext[1]{Institute of Informatics Problems, Federal Research 
Center ``Computer Science and Control'' of the Russian Academy of Sciences,
 44-2~Vavilov Str., Moscow 119333, Russian Federation, \mbox{fsergei51@gmail.com}}
\footnotetext[2]{Computer Science Department, Ben-Gurion University of the Negev, 
Beer-Sheva 84105, Israel, \mbox{danielkh@post.bgu.ac.il}}


\index{Frenkel S.\,L.}
\index{Khankin D.}
\index{Френкель С.}
\index{Ханкин Д.}

\def\leftfootline{\small{\textbf{\thepage}
\hfill INFORMATIKA I EE PRIMENENIYA~--- INFORMATICS AND
APPLICATIONS\ \ \ 2018\ \ \ volume~12\ \ \ issue\ 4}
}%
 \def\rightfootline{\small{INFORMATIKA I EE PRIMENENIYA~---
INFORMATICS AND APPLICATIONS\ \ \ 2018\ \ \ volume~12\ \ \ issue\ 4
\hfill \textbf{\thepage}}}

\vspace*{4pt}

\Abste{In software-defined networking (SDN), the control plane and the data 
plane are decoupled. This allows high flexibility by providing abstractions 
for network management applications and being directly programmable. 
However, reconfiguration and updates of a~network are sometimes inevitable due 
to topology changes, maintenance, or failures. In the scenario,  
a~current route~$C$ and a set of possible new routes~$\{N_i\}$, where one of the 
new routes is required to replace the current route, are given. There is a chance that 
a~new route $N_i$ is longer than a~different new route $N_j$, but $N_i$ is 
a~more reliable one and it will update faster or perform better after the update 
in terms of quality of service (QoS) demands. 
Taking into account the random nature of the network functioning, 
the present authors supplement the recently proposed algorithm by Delaet
\textit{et al}.\ for route updates with 
a~technique based on Markov chains (MCs). As such, an enhanced algorithm 
for complying QoS demands during route updates is proposed
in a~seamless fashion. First, 
an extension to the update algorithm of Delaet \textit{et al}.\ 
that describes the transmission of packets through a~chosen route and compares 
the update process for all possible alternative routes is suggested. Second, several 
methods for choosing a~combination of preferred subparts of new routes, resulting 
in an optimal, in the sense of QoS compliance, new route is provided.} 

\KWE{software-defined networking; Markov chains; quality of service}

\DOI{10.14357/19922264180408}


\vspace*{8pt}


\vskip 12pt plus 9pt minus 6pt

 \thispagestyle{myheadings}

 \begin{multicols}{2}

 \label{st\stat}

\section{Introduction}
\label{s:Intro}

\noindent
Software-defined networking is an emerging network paradigm, in which the 
control plane is decoupled from the data plane enabling centralized control 
logic. Such a~dynamic network may require frequent modifications and updates to 
the network topology and configuration. 
Also, the network topology is available to the centralized control entity, there, 
due to this flexibility, it is possible to perform offline optimized calculations.

Network functions virtualization (NFV) allows replacing traditional network 
devices with software that is running on commodity servers. This software 
implements the functionality that was previously provided by dedicated hardware. 
Network functions virtualization
 allows services to be composed of virtual network functions (VNF) hosted on 
different data centers. Software-defined networking, 
when applied to NFV, helps in addressing challenges 
of dynamic resource management and intelligent service 
orchestration~\cite{rao_sdn_2014}. Sometimes, traffic is often required to pass 
through and be processed by an ordered sequence of possibly remote 
VNFs~\cite{ghaznavi_service_2016}. For example, traffic may be required to pass 
through intrusion detection system, proxy, load balancer, or a~firewall. 
Such concatenation of services is called \textit{service function chaining} 
(SFC).

Consider, for example, two communicating parties in a~network featuring complex 
network topology (e.\,g., Small-world network), and the communication flow is 
passed over a~series of VNFs. It may be the case that the network operator is 
required to move the communicating flow to a~different path due to QoS 
requirements or other possible cost considerations. We are interested 
to model the anticipated expected number of steps until the update is complete 
given a~possible new route following the required QoS demands, e.\,g., 
delay, communication rounds, cost, etc. 

%Aforesaid dynamic networking requires frequent modifications and updates to the network. 
Let us consider a pair $(C, \{N_i\})$ where a~current route~$C$ from~$s$ to~$d$ 
is scheduled to be replaced by a new route from the set~$\{N_i\}$, each from~$s$ 
to~$d$ either. Let us model each route as an ordered list of network elements, such 
as VNFs (SFCs) or generally saying routers. Each new route~$N_i$ is constructed 
during the update process, and thus, certain delays may be introduced due to
 initial packet processing or due to possible losses. 
 %There, the eventual arrival of packets along the new route during the update process is critical for successful route update. Another possible example is when the routes are SFCs, and the requirement is to update a current chain to a new one, different service chains may exhibit different delays. 

The design goals must be achieved by constructing effective algorithms for 
efficient packet QoS routing in NFV/SDN computer network. Depending on the 
QoS metric, the lower (e.\,g., for reliability) or upper (e.\,g., for a~delay) 
constraints represent the desired bounds that the orchestration must meet. 
Since different configurations could meet these bounds, the designer should also 
optimize against a~specific metric by using these both ends of the extreme. 

Methods based on integer linear programming (ILP) were proposed in several works 
(see section~\ref{sec:related_work}). The difficulty of using tools based on ILP 
 in the operational work of an administrator is that in view of the possible 
 infeasibility of the resulting solution, it may take not a~few resources (time, efforts) 
 until acceptable QoS values can be ensured.

We consider the use of ``design via verification'' approach, suggesting a~method 
for complying QoS demands. The method is based on augmenting the update algorithm with
a~verification logic. Namely, we suggest the use of 
\textit{Probabilistic real-time Computation Tree Logic} 
(PCTL)~\cite{hansson_logic_1994} for expressing real-time and probability in systems. 
Using PCTL, we can express the probability for a~process to complete after 
a~certain number of steps along an execution path and verify the selected route 
for the update.


%Assume that packets are sent from a source node $s$ to a destination node $d$ along the current route. After the update process is finished, packets will be forwarded from $s$ to $d$ along the new route. 
Delaet \textit{et al.}\ proposed a~multicast-based scheme for seamlessly updating 
a~current route to a~new one~\cite{delaet_seamless_2015}. 
According to the multicast scheme, the controller instructs 
a~router to temporarily have two $(s,d)$ entries in the routing table. When 
a~router $r \neq d$ receives a~packet from~$s$ to~$d$, it sends the packet 
according to the forwarding instructions of all of its $(s,d)$ routing 
table entries. When two identical copies of a~packet that was multicasted 
over the current and new portion of a~route arrive, the controller can dismantle 
the current route, as the new route is ready. During the update process, packets 
should not be lost, no cycles should be formed, and communication should not 
be disrupted.

%Taking into account the random nature of the network functioning, we supplement the algorithm for route updates introduced by Delaet et al. in \cite{delaet_seamless_2015}, with a technique based on Markov chains. In our extension of the algorithm, we describe the transmission of packets through a chosen route and compare the update process for all the possible alternative routes that are candidates for replacement. 

Our contribution is a model for a successful route update, including its 
intermediate steps, as MC states, each with 
a~given probability. With our model, we are able to characterize the quality of 
an update by expected number of steps in the~MC. 
%We use Markov chains to characterize the quality of the update service, and represent the expected number of steps in the Markov chain as the quality of a successful update. While, the probability for an update event 

We suggest an enhanced update method for the network administrator to augment 
his decision regarding QoS demands in terms of various network parameters and 
possible failure of the update process. Moreover, in contrast to other works, 
we are able to provide a~version of an algorithm that can perform real-time QoS
 assessment during a~route update, for each subpart of a~route. At last, using 
 our method, it is possible that the active new route will be comprised of subparts 
 of different new routes, providing optimal route update service in regard of 
 required network QoS. 

%We assume that each new route is legal. 
%However, mixing subroutes belonging to different routes may result in inconsistent state or a cycle formed in the network. We use different 
%
%
%
%We model the update process as a service, namely as a VNF, and we use Markov chains to characterize the quality of the update service. Using the expected number of steps in the Markov chain representing the update, we abstract the quality of the update service. We calculate for each possible new (sub-)route the expected number of steps required to update an old (sub-)route successfully. Subsequently, the old route is updated to the new route which requires less number of steps with high probability. We supplement the seamless update algorithm proposed by the authors of \cite{delaet_seamless_2015} with the model in this work.

%The virtualized service implementing the update algorithm will provide a recommendation for an optimal choice of a route, based on the performed calculations. Fundamentally, we create a QoS VNF for seamlessly updating a route, regarding network parameters, and taking into consideration the complexity and possible failures of updating a route. In case there exist several alternatives for a route update, there is a chance that one of the possible new routes is much longer, however, a more reliable one, and as such will update faster. 
%
%
%One of the important requirements to modification process is that the update process should not form congestion in the network, nor result in time delays, and not lose any packets. 
%
%
%Additionally, we provide an enhanced version of an algorithm that can perform the quality of service assessment during the update process, for each subpart of the new route. 
%
%We propose a directed graph $G=(V,E)$, for representing the possible legal combinations of sub-routes. The set of common nodes to $(C, \{N_i\})$ subdivides the old route and each of the new routes to sub-routes. For two sub-routes represented by the nodes $u,v \in V$, the sub-route $v$ can be launched after $u$ if and only if there exists a directed edge $(u,v) \in E$. Otherwise, the launch of $v$ after $u$ is forbidden and can result in a cycle formed in the network.


%The results of this work helped to develop an operating strategy for a network administrator, supporting both, seamlessly updating a route, and providing QoS requirements. 

Extended abstract of this work appeared as a conference paper 
in~\cite{frenkel_predicting_2017} which presented preliminary results. 
In this work, we describe in detail the system settings and bring new results 
by providing two additional algorithms.
{\looseness=1

}

In the following section, we overview the related work. Next, we provide 
the required definitions and the system settings and describe the MC 
characterization of the network. Further, we describe different update setting, 
accordingly accompanying algorithms and data structures, used for QoS assessment 
during route updates.

\vspace*{-9pt}

\section{Related Work}
\label{sec:related_work}

\vspace*{-2pt}
%The design goals must be achieved by constructing effective algorithms for efficient packet QoS routing in NFV/SDN computer network. %These algorithms, which must enable an administrator to orchestrate the existing services exported by remote providers, were considered in \cite{martins_clickos_2014, zaalouk_orchsec:_2014}. Likewise, the functional behavior (e.g., services being deprecated by their providers), as well as changes in the non-functional behavior of the orchestrated services (e.g., an increased execution time) were also considered.

%Depending on the QoS metric, the lower (e.g., for reliability) or upper (e.g., for delay) constraints represent the desired bounds that the orchestration must meet. Since different configurations could meet these bounds, the designer must also optimize against a specific metric by using these both ends of extreme.

\noindent
Quality of service routing using multipath was proposed in~\cite{devi_approach_2015}. 
The routing algorithm, initially, eliminates all links that do not meet the 
bandwidth requirements. Then, it finds disjoint shortest paths based on 
the residual network graph in each iteration.

The work~\cite{egilmez_distributed_2012} proposed a~QoS optimized routing 
over multidomain OpenFlow networks managed by a~distributed control plane, 
where each controller performs optimal routing within its domain. 
The QoS routing problem was posed as a~constrained shortest path (CSP) problem, 
and the proposed solution computes a~near-optimal route, based on LARAC 
(Lagrange relaxation based aggregated cost)
algorithm~\cite{juttner_lagrange_2001}. The proposed algorithm is an approximation 
algorithm; it always gives a~suboptimal solution.

For traditional network architecture, a~routing strategy approach based on 
ILP was introduced in~\cite{yu_efficient_2013}.
 The main disadvantage of using ILP is that the problem is NP-hard. 
 Additionally, ILP cannot be applied to probabilistic values. 
 Using linear programming (not limited to integers) rounded to integer solutions 
 will not yield an optimal solution.
 

Route updates are extensively researched in SDN~\cite{foerster_survey_2016}, 
standing on the work by Reitblatt \textit{et al.}\ where requirements for SDN 
updates were examined. This work focused on per-packet consistency property, 
stating that packets have to be forwarded either using the initial configuration 
or the final configuration but never a~mixture of them, throughout the update 
process~\cite{reitblatt_consistent_2011}. The authors proposed 
a~2-phase commit technique which relies on packets tagging so that either of 
the rules is applied. However, such technique wastes critical network resources 
and complications are formed due to packet tagging~\cite{foerster_survey_2016}. 
Further, Delaet \textit{et al.}\ showed in~\cite{delaet_seamless_2015} 
that using a~careful multicast during route updates provides 
a~better working solution.

Hogan and Esposito propose in~\cite{hogan_stochastic_2017} the use of
 Bayesian networks for delay estimation as a~traffic engineering tool and model 
 the path selection problem using a~risk minimization technique. 
 However, the authors state that the accuracy of their model is limited by its 
 ability to correctly identify dependencies in the data. In our work, 
 we suggest a~general tool for probabilistic verification of any network parameter, 
 which does not depend on variance within the dataset.
 
 

In~\cite{mcgeer_safe_2012}, an update protocol proposed where packets are 
sent to the controller during updates; such approach adds 
a~significant cost to the control plane bandwidth~\cite{delaet_seamless_2015}. 
In~\cite{mcgeer_correct_2013}, an algorithm to find 
a~safe update sequence expressed as a~logic circuit has been proposed. 
However, the algorithm 
requires a~dedicated protocol which is not currently 
supported~\cite{foerster_survey_2016}. The authors 
of~\cite{katta_incremental_2013} propose to perform the 2-phase update 
scheme from~\cite{reitblatt_consistent_2011} incrementally, making the update longer. 
%For a thorough review of route updates, the reader is referred to \cite{foerster_survey_2016}.






Software-defined networking allows the involvement of the network administrator into the network 
management during route udpdates and, in particular, during packet transmission. 
Thus, it would be highly desirable to support the decision making process 
with the right tools. Our novelty is exactly such tool, for augmenting 
online decision making of the network administrator during network management 
in a~stochastic environment.
%In this work, we propose a technique to optimize the update process by selecting the preferred (sub-)route in order to reduce the update time. We use the expected number of steps for successfully completing the update as a QoS metric, and extend the algorithm by Delaet~et~al. with Discrete Time Markov Chains (DTMC) for finding (sub-)routes which are preferred in terms of QoS. % As such, we propose to use the route updates algorithm from \cite{delaet_seamless_2015} as a virtual service for network updates per QoS requirements.

%The interaction of software components have a greater weight in NFV context, which may lead to stochastic-like behavior 

%At present, certain routing algorithms (including $k$ Edge-Disjoint) are based on the shortest path (SP) problem solution \cite{wood_toward_2015}. However, the method proposed by Wood et al. is generic and valuable only in the case of request arrival, and do not consider certain additional important requirements, such as removal or priorities of requests. 

%Several approaches for efficient SP-based QoS routing have been recently proposed in \cite{buchbinder_improved_2006}, where the authors introduce and analyze a centralized algorithm for an online scheduling and routing of arbitrary sequence of communication requests. 

%Unsplittable (single-path) assignment for each request of QoS routing is probably competitive with the best possible splittable (multipath assignment).

The work by Delaet \textit{et al.}~[4] introduced the Make\&Activate-Before-Break 
approach for seamless
route update in SDN. The authors described in a~high-level the multicasting-based 
update, which we
employ in this work. Also, they introduced a~controller-based method for 
verifying the correctness
of a~new route before the traffic redirection. Dinitz \textit{et al.}~[16] 
extended the work~[4] to the general
case of several dependent (via shared links) routes pairs. The routes update 
problem was proved to
be NP-hard~\cite{17-aaa}. The authors of~[16] suggested the use of 
artificial intelligence (AI) methods for 
solving the problem. As a~basis for AI-based solutions, Dinitz 
\textit{et al.}\ proposed a dependence graph model describing the current
state of the problem instance at any replacement stage. 
In addition, route readiness verification similar
to that in~[4] was implemented in~[16] as a high-level network protocol.

In this work, we investigate a different problem; we consider the route updates 
problem from a~QoS
perspective and describe in high-level both the prediction and the update processes.

\vspace*{-9pt}

\section{Preliminaries and Definitions}

\vspace*{-2pt}

\noindent
The basic system settings are as follows. 
For a~(route) sequence~$X$, we denote by~$x_i$ the $i$th element in it.
In a~(directed) communication network, 
we are given a~route~$C$ from source~$s$ to destination~$d$. 
Additionally, we are given a~set of different new routes~$N_i$, each going from~$s$ 
to~$d$. We model each route as an ordered set of network nodes connected by network 
links. We assume that neither of the routes contains cycles. 
Each router in a~route matches a~packet from~$s$ to~$d$ 
and forwards the packet to the next router in order. After the update 
is complete, each router in the new route should forward the packets from~$s$ 
to~$d$ to the next router in order along the new route. 

In our work, we consider the route replacement problem as a~sequence of 
subroutes replacements.
The routes replacement subsystem was in great detail described by Dinitz 
\textit{et al.} in~\cite{dinitz_dependence_2017}. We borrow
from~[16] the relevant parts which we briefly describe here.

\smallskip

\noindent
\textbf{Definition~1.} We  define a~subset from $a\in X$ to $b\in X$ of an ordered
set~$X$, when $a$ precedes~$b$, as~a~subroute from~$a$ to~$b$, and denote such subroute by
$[a,b]$.

\smallskip

 

\textbf{Subroutes.} The current route~$C$ subdivides each new route 
to~$k$~common subroutes (a~subroute may consist of one router in the simplest case) 
and $k-1$ noncommon subroutes. 
For illustration, see Fig.~1.
In Fig.~1 and figures below, the current route is depicted
in a~light grey color full nodes, connected with
solid edges. The new route is depicted in white colored nodes, connected with
dashed edges. The common nodes are depicted as shaded. 
If there are several new
routes, the nodes of each route are filled with a~designating pattern. 
Additionally, for easier reading,
when it is possible, we denote subroutes of some route~$X$ as~$X^\prime$, $X^{\prime\prime}$, 
etc. In other cases, a~subroute~$j$
of a~new (current) route~$i$ is denoted as $N_j^i (C_i^j)$. 
Similarly, routers of some route~$X$ are denoted by~$r^\prime$,
$r^{\prime\prime}$, etc.

 { \begin{center}  %fig1
\vspace*{1pt}
 \mbox{%
 \epsfxsize=78.631mm 
 \epsfbox{fre-1.eps}
 }


\vspace*{3pt}


\noindent
{{\figurename~1}\ \ \small{Route $C$ with two possible new routes sharing a~link}}
\end{center}
}

\vspace*{6pt}






In the example in Fig.~1, 
noncommon new subroutes 
of route~$N_1$ are denoted by~$N^1_1=[s,r_2]$ and~$N^2_1=[r_2,d]$, while the noncommon new 
subroutes of~$N_2$ are denoted by~$N^1_2=[s,r_1]$, $N^2_2=[r_1,r_3]$, 
$N^3_2=[r_3,r_2]$, and~$N^4_2=[r_2,d]$. 

Note that in general, the order of common subroutes along~$C$ and along~$N$ 
can be different. See, for example, the common subroutes of~$C$ and~$N_2$ in 
%Figure \ref{fig:two_routes}.
Fig.~1.

\smallskip

\noindent
\textbf{Definition~2.} A~new noncommon subroute of~$N$ from router~$a$
to router~$b$ is legitimate for update only if~$a$ precedes~$b$ on the route~$C$.

\smallskip

Definition~2 guides us on which subroutes can be launched without creating routing cycles in the
network system. (See~[4] for details.)


When an update of a~subroute~$N^\prime$ from router~$r$ to~$r^\prime$ is finished, 
the update flow goes along~$C$ from~$s$ to~$r$, continues along~$N^\prime$ up to~$r^\prime$, 
and finishes along~$C$ from~$r^\prime$
 to~$d$. 
For illustration, see the result of launching~$N^4_2$ in Fig.~2.

 { \begin{center}  %fig2
\vspace*{-1pt}
 \mbox{%
 \epsfxsize=78.631mm 
 \epsfbox{fre-2.eps}
 }


\vspace*{3pt}


\noindent
{{\figurename~2}\ \ \small{$N^4_2$ was launched}}
\end{center}
}

\vspace*{4pt}


 

 Note that launching a~currently nonlegitimate new subroute, for example,~$N^3_2$ 
 in Fig.~1, is forbidden since it will form a~cycle 
 resulting in packets circulating and overwhelming the network. 

\textbf{Dynamics of the system.}
%\label{sec:dynamics} 
Dinitz \textit{et al.}\ performed a~detailed analysis on the dynamics of a~subroutes
system. After an update of a~subroute is complete, the set of current subroutes~$C$ 
and the set
of new subroutes~$N$ are recalculated. This may result in different system of subroutes. For example,
see Fig.~2 where after the launch of $N^4_2$ from the example in Fig.~1, 
the sets of subroutes are
recalculated. As a~result, we obtain different subroutes (for clarity, the previous labels are kept). See
also~[16] for details and extensive analysis.

\vspace*{-4pt}

\subsection{Markov chain characterization of~the~network~states}

\noindent
We characterize execution of some (sub)route in the network by 
a~packet delay time between the (sub)route's common sender and common destination 
routers as well the probability of a~packet drop. Let us for now define our 
network routing model (conceptual model) informally in the following terms. 
Delay of a~packet is obtained using a~physical delay and the total processing 
time in the router. We consider that transmission of packets in 
a~network can have a~random behavior, caused by the random character of both, 
the input, and possible loss of packets. There we are interested in 
a~probabilistic model, namely, a~Markov model. In order to fully characterize 
the network as an~MC, the internal state of each router 
(and, in particular, the buffer occupancies), as well as the characteristics
 of all flows, need to be expressed as states in the chain. 

However, such approach would result in an enormous and intractable number of states. 
Therefore, to simplify these computations, let us characterize the delay time as 
an abstract variable~$t$. This abstract variable can be interpreted in different ways, 
e.\,g., the current processing queue length and a~packet transmission rate of the link, 
or possibly a~fixed value, such as an interval between the beginning of 
a~packet transmission after being processed in some node and the end of processing 
at the next node. 

We describe the functioning of the network in the transmission of packets 
as transitions of a~discrete-time MC (DTMC). The state space corresponds to the set 
of nodes such that 
the transmission of a~packet from a~node that has finished processing the packet 
to the next node corresponds to the transition of the chain to the next state.


Discrete-time MC is defined as a~tuple $D\linebreak =(S, s_0, P)$. In the tuple, $S$ is 
the finite set of states, $s_0\in S$ is the initial
state, $P:S \times S \rightarrow [0, 1]$ is the transition probability matrix in 
which $\forall s\in S$, $\sum\nolimits_{s' \in S} P(s,s') = 1$. 
For any two states $s, s' \in S$, if $P(s,s')>0$, then~$s'$ is the successor of~$s$. 
For a~subset of states $T \subseteq S$, the probability of moving from a~state~$s$ 
to any state $t \in T$ in a~single step is denoted by $P(s, T)$ and is given by 
$P(s,T)=\sum\nolimits_{t \in T} P(s, t)$. 
%The row $P(s,:)$, in the transition matrix $P$, contains the probabilities of moving from $s$ to its successors, while the column $P(:, s)$ contains the probabilities of entering the state $s$ from any other state.

\vspace*{-6pt}

\subsection{Verification syntax}

\noindent
For implementation of our PCTL-based model, we use PRISM~--- 
probabilistic model checker~\cite{kwiatkowska_prism_2011}. There, we follow 
PRISM property specification language. Here, we briefly describe the essential 
syntax while more details can be found in~\cite{noauthor_prism_nodate}.

Given a property~$\Psi$, we say that~$\Psi$ is true with probability~$p$ 
and write that as
$P_p [ \Psi ]$. If the probability~$p$ is unknown, PRISM allows, for DTMC, 
writing properties queries of the form $P_{=?}[ \Psi ]$, meaning 
``what is the probability that~$\Psi$ is true?''. Additionally, it is possible 
to use a~time bound and write properties queries such as 
$P_{=?}[F^{\leq T} \Psi]$, meaning ``what is the probability that~$\Psi$ 
is true after less than~$T$~steps?''. At last, it is possible to compute 
properties such as expected time or expected number of steps. 
For example, $R_{=?}[F \Psi]$, meaning ``what is the expected number of 
steps until $\Psi$ is true?''. 
%\section{Model Settings}
%, and a subroute of route $X$ from router $a$ to router $b$ is specified by $[a,b]_X$

%When a new subroute of $N$ that is scheduled to update a current sub-route of $C_i$ is launched, the route $C$ is updated such that the updated sub-route is replaced by launched sub-route, and the new sub-route is now part of the current route $C$.

\setcounter{figure}{3}
\begin{figure*}[b] %fig4
\vspace*{-6pt}
 \begin{center}
 \mbox{%
 \epsfxsize=149.177mm 
 \epsfbox{fre-3.eps}
 }
 \end{center}
\vspace*{-9pt}

 \Caption{New routes~$N_1$~(\textit{a}) and $N_2$~(\textit{b}) and
 MC states for~$N_1$~(\textit{c}) 
and~$N_2$~(\textit{d})}
 \label{fig:routes_dtmc_example}
\end{figure*}



\vspace*{-6pt}

\section{Prediction of Preferred Update}
%\section{Prediction of Preferred Update}
\label{sec:dtmc}

\noindent
The states of a~DTMC describe the nodes in the new route and the transition 
probabilities in the chain represent the possible delay or 
a~packet loss in the routers along the new route. The
states are defined as 
$\{s_1, \ldots , s_n\}$ where~$n$ is the number
  of nodes in the new route. 
The network achieves the state~$s_i$ if a packet has reached the $i$th node. 
For example, in Fig.~3, the self-transition 
edge represents the probability for a~delay due to packet loss, rules installation 
at the router, or congestion on the router-controller link, while the 
forward transition edge represents the probability for 
a~successful transition to the next state. These probabilities can be estimated 
from network statistics (see, for example,~\cite{hogan_stochastic_2017}). 
The labels on edges are the probability values, when edge has no label
 means probability~1.
 
 The initial probability distribution of states is given by the vector~$P_0$ of size~$n$. 
We can determine the prob-\linebreak\vspace*{-12pt}
 
 %\linebreak\vspace*{-12pt}

{ \begin{center}  %fig3
\vspace*{-0.5pt}
  \mbox{%
 \epsfxsize=77.518mm 
 \epsfbox{fre-4.eps}
 }


\end{center}

\vspace*{-3pt}

\noindent
{{\figurename~3}\ \ \small{Probability as a~function of number of steps to update routes~$N_1$~(\textit{1})
 and~$N_2$~(\textit{2})}}
}

\vspace*{12pt}



\noindent
ability that a~particular route delays the update process 
by~$k$, that is, the number of steps required for a~successful update is given by 
$p(k)=P_0 P^k$. Using this characteristic, which is, in fact, the 
probability distribution of the number of steps $P(k < x)$, one can 
calculate various properties like average delay time for the new route, 
maximum or minimum number of steps to update, etc.
 
 Consider the example illustrated in Fig.~4. 
Figure~4\textit{a} illustrates the current route~$C$ and a candidate new route~$N_1$. 
Figure~4\textit{b} shows the same current route~$C$ with another candidate 
new route~$N_2$. 
Figures~4\textit{c} and~4\textit{d} 
show the MCs for new routes~$N_1$ and~$N_2$, accordingly, with given transition 
probabilities.

During the update process, packets are sent along the current and the new routes. 
Since the new route is\linebreak\vspace*{-9.5pt}

\columnbreak

\noindent
 not operational yet, packets can be delayed due to 
congestion on certain nodes or due to switch configurations. 
%
For example, if routing rules have not yet been installed in some switch, then an 
arriving packet is sent to the controller~\cite{onf_openflow_2015}. The controller 
then decides reactively on further actions whether to install an appropriate rule 
for the packet. Also, the controller may be busy with other work and not respond 
immediately. Those packet processing actions may delay the update process. 
In the case buffer becomes full, for example, if the network is being congested, 
packets may be dropped. There, the transition to the next state during the 
update process depends on the likelihood of a~delay or a~loss of a~packet in the 
current state. 

In the example, the number of steps required for launching~$N_2$ is smaller than 
the number of steps required for launching~$N_1$. However, due to a higher likelihood 
of delays along the route~$N_2$, it is possible that~$N_1$ is preferred having 
a~higher probability for a~successful update. The network administrator may ask 
which new route is recommended for the update process, considering the expected 
number of steps required for the update. 
%
That is, updating paths requires the operator to decide 
on the possible choice of a~subroute for the next step. 
One should consider the possibility of including a~decision tool augmenting the 
controller during route updates. 

There were many attempts to use the LP/ILP 
approach, as it was already mentioned above (see, e.\,g.,~\cite{juttner_lagrange_2001}), 
but they have encountered the same difficulties, especially when taking 
into account online implementation. We show that it is possible to describe 
the routing process as DTMC. Thus, taking into consideration~$O(n^3)$ worst case 
computation complexity, we consider using the ``design via verification'' 
mentioned above based on PCTL verification, similar to the one used in 
PRISM~\cite{kwiatkowska_prism_2011}.


We have calculated the probability for a~successful update as a~function of 
number of steps for routes~$N_1$ and~$N_2$ from the example in 
Fig.~\ref{fig:routes_dtmc_example}. See Fig.~3 
where this function is shown. Curve~\textit{1}
represents the plot for~$N_1$ and curve~\textit{2} represents
 the plot for~$N_2$. 

Observe that after~20~steps, both new routes will be launched with probability~1 
which can be written as 
$$
P_{1}\left[F^{>20}N_1\right]=P_{1}\left[F^{>20}N_2\right]=1\,.
$$
The expected number of steps required for~$N_1$ is smaller than the required for~$N_2$:
$$
R \left[F~N_1\right] < R \left[F~N_2\right]\,.
$$
However, the probability for successfully updating in less than~15~steps 
is higher for route~$N_2$ ($0.55 \pm 0.040$ for~$N_1$ and 
$0.717 \pm 0.036$ for~$N_2$, based on~99\% confidence level):
$P_{0.717 \pm 0.036}\left[F^{\leq 15} N_2 \right].$

\vspace*{-6pt}


\section{Route Updates per~Quality~of~Service}
\label{sec:updates_qos}

\vspace*{-2pt}

\noindent
In this section, we show algorithm that we propose for various settings. 
First, we show an enhancement for the sequential update algorithm 
from~\cite{delaet_seamless_2015}, which during the update process decides on 
preferred subroute from the set of possible subroutes as part of QoS requirements. 
In the multicast-based update, several methods were proposed 
in~\cite{delaet_seamless_2015} for eliminating duplicated packets. 
In the case the common destination router is not able to immediately eliminate 
duplicated packets, the algorithm begins the update from the end, 
ensuring a~correct update process~[4].



\begin{algorithm*} %alg1
 \setlength{\algowidth}{100mm}
 \setlength{\hsize}{\algowidth}
 \caption{Update per QoS Algorithm}
 \label{alg:update_per_qos}

%\hrule
%\vspace*{2pt}
%\centerline
%{\textbf{Algorithm~1:} Update per QoS Algorithm}\par

%\vspace*{2pt}

%\hrule
 \small
 
 %\Input
 {directed graph $G$} 
 
 \BlankLine
 \tcc{$A$ is a collection of nodes} $A \leftarrow$ choose nodes from $G$ with in-degree $0$ \\
 
 \Repeat {out-degree of node $N^t_i > 0$}
 {
 \ForEach{$v \in A$ \label{alg:inner_loop}}
 {
 calculate $R[F~v]$ \\
% calculate the expected QoS for this node as described in Section \ref{sec:updates_qos} \\
 }\label{alg:end_inner_loop}
 
% $N^t_i \leftarrow$ choose the node that maximizes QoS \label{alg:choose_qos}\\ 
 $N^t_i \leftarrow \argmax_{v} (R[F~v])$ \label{alg:choose_qos} \\
 launch $N^t_i$ \\
 update $C$ accordingly \\
 merge any new and common subroutes as described in section~3 \\ 
 $A \leftarrow$ choose nodes neighboring to $N^t_i$ \\ 
 }
 
 \BlankLine 
 
\end{algorithm*}





 
%The algorithm starts from any node with in-degree 0 since it means that such node has no precedence dependence. Updating is completed when the algorithm arrives to a node with out-degree zero, which would be the last subroute to launch.


After that, we show an algorithm that chooses the subroutes for update arbitrary, 
assuming that the common destination node will not leak duplicated packets. 
However, the packets sending rate along the new subroute need to be temporarily limited~[4].

At last, we present a supplementing algorithm that suggests which subroutes can 
be updated in parallel.

%The set of common nodes for each pair of routes subdivides the routes to sub-routes relatively to each other. 

\vspace*{12pt}

\subsection{Sequential update}

\noindent
Let us begin the update from the end, namely, from the last alternative 
subroute of any new route. Provably, this prevents the formation of 
cycles~\cite{delaet_seamless_2015}. In order to represent all possible choices 
of a~path from a current state of the update process to the end of the update process, 
we propose to use a directed graph which nodes are the new, legitimate for launching, 
subroutes of the network. The edges of the graph represent a~legal order of launching 
new subroutes. Each path in this graph from a~current node to the last node in 
the path represents a~legal combination of chosen subroutes. The update process is 
continued as long as there is a~possible node to transition to. 

Let us examine the two possible new routes~$N_1$ and~$N_2$ that can replace the 
current route~$C$ from the example depicted in Fig.~1. 
The new route~$N_1$ is composed of~$N^1_1$ and~$N^2_1$, while the new route~$N_2$ 
composed of~$N^1_2$, $N^2_2$, $N^3_2$, and~$N^4_2$. Starting from the end, the only 
new subroutes that are allowable to launch are~$N^2_1$ and~$N^4_2$. 
Assume that based on the DTMC calculations performed as described in section~4, 
the subroute~$N^4_2$ is chosen for update. After the update of the subroute is 
complete, the current route~$C$ is composed of not updated yet part of the old 
route and~$N^4_2$. See Fig.~2 where the change in~$C$ 
is depicted.

After the subroute~$N^4_2$ is launched, we arrive at a~smaller problem in which 
less subroutes are left to update. Due to dynamics of the system 
(see section~3), some new subroutes can merge into a~single new subroute.
See Fig.~2 where after~$N^4_2$ was launched, the 
new subroutes~$N^3_2$ and~$N^2_2$ are merged into a~single subroute. Now, one 
can launch either~$N^1_1$ or~$N^2_2$ merged with~$N^3_2$. Assume that we choose to 
launch~$N^1_1$, which launch
 finishes the update. The route~$C$ updated to~$N^1_1$ 
and~$N^4_2$. See Fig.~5 illustrating that.


Figure~6 shows the directed graph that represents 
the possible update sequences. Initially, the subroutes that %\linebreak\vspace*{-12pt}
 are legal 
for launch are~$N^2_1$ and~$N^4_2$. As such, these are
the only subroutes that
 have in-degree~0. Launching~$N^3_2$
 is forbidden; hence, there is no node in the 
 graph~$G$ that represents this subroute. After launching~$N^4_2$, we\linebreak\vspace*{-12pt}
 
 \setcounter{figure}{4}

{ \begin{center}  %fig5
\vspace*{12pt}
 \mbox{%
 \epsfxsize=78.631mm 
 \epsfbox{fre-5.eps}
 }


\vspace*{3pt}


\noindent
{{\figurename~5}\ \ \small{$N^1_1$ was launched}}
\end{center}
}

\vspace*{6pt}

{ \begin{center}  %fig6
\vspace*{1pt}
 \mbox{%
 \epsfxsize=36.428mm 
 \epsfbox{fre-6.eps}
 }


\end{center}


\noindent
{{\figurename~6}\ \ \small{Graph 
representation for possible update paths for routes update example from Fig.~1}}

}

%\vspace*{6pt}

\noindent
  can 
 proceed by launching~$N^1_1$ or~$N^2_2$. However, if~$N^2_1$ was launched first, 
 it would be forbidden to launch~$N^2_2$ since it shares a~common edge with~$N^2_1$. 
 This is reflected in the graph~$G$ by not having a~directed edge from the
  node~$N^2_1$ to the node~$N^2_2$. We finish the update process
 by arriving either 
 to~$N^1_1$ or to~$N^1_2$. Notably, these nodes have out-degree~0.

 Algorithm~1 updates subroutes according to calculated QoS for each new subroute, by
 choosing at each step the new subroute that maximizes QoS.


The algorithm starts by selecting the initial set of subroute nodes. 
These are nodes with in-degree~0. The algorithm continues traversing the graph up 
to arrival at a node with out-degree~0 which would be the last subroute to launch. 
The inner loop at lines~\ref{alg:inner_loop}--\ref{alg:end_inner_loop} 
calculates the QoS for each neighboring node. Afterward, at 
line~\ref{alg:choose_qos}, the algorithm chooses the node that maximizes QoS. 
Then launches this node and updates the route~$C$, accordingly (see 
Figs.~1--5 for illustration). 
Afterward, the algorithm selects the next neighboring nodes.

After execution of Algorithm~1, the resulting new route maximally complies QoS 
requirements.

%\vspace*{12pt}

\subsection{Arbitrary subroutes selection} 
%\label{sec:arbitrary}

%\vspace*{-12pt}

\noindent
In this subsection, we assume that immediate duplicate packets elimination is possible. 
It may be that some of the subroutes are not ready for an update yet. 
Thus, meanwhile, the administrator may want to proceed with the update process 
to other subroutes or see possible variations of the update. 
For such scenario, we provide an algorithm which can select a~subroute for 
update arbitrary and continue the update process from there. 
We create a~forest graph of all possible update combinations from which the 
desired update sequence can be chosen. 
{\looseness=1

}
 


Figure~7 shows all possible combinations from example 
in Fig.~1. Noticeable, as mentioned earlier, some\linebreak\vspace*{-12pt}

{ \begin{center}  %fig7
\vspace*{1pt}
  \mbox{%
 \epsfxsize=71.694mm 
 \epsfbox{fre-7.eps}
 }


\end{center}


\noindent
{{\figurename~7}\ \ \small{Forest graph representing execution combinations for example from 
 Fig.~1}}
}

\vspace*{12pt}


\noindent
 combinations 
exhibit fewer steps, though possible that its QoS compliance is worse than others.



Algorithm~2 starts by iterating over all roots of the forest graph and 
calculating QoS using Algorithm~1 each tree. Afterward, launch the update 
of the tree that maximizes QoS.

\begin{algorithm*} %alg2
\setlength{\algowidth}{100mm}
 \setlength{\hsize}{\algowidth}
 \caption{Arbitrary Selection Update}
 \label{alg:arbitrary_update}
 \small
 
% \Input
{directed graph $G$} 
 
 %\BlankLine
 
 $A_0 \leftarrow$ choose nodes from $G$ with in-degree $0$ \\
 $Q \leftarrow \{\}$ \\
 
 \BlankLine
 \tcc{iterate over all roots of trees in the forest $G$}
 \ForEach{$v_r \in A_0$}
 {
 $q \leftarrow$ get the expected QoS using Algorithm~1 for $v_r$ \\
 $Q \leftarrow Q \cup \{q \rightarrow \mathrm{root} \}$ \\
 }

 \BlankLine
 $q_{\max} \leftarrow \max_{\mathrm{QoS}}(Q)$ \\
 launch maximizing QoS update order in $\mathrm{root}=Q[q_{\max}]$ \\ 
 
 
\end{algorithm*}


%\columnbreak

\vspace*{12pt}





\subsection{Parallel update}

\noindent
In certain cases, it is possible to update in parallel several subroutes 
and, as such, decrease update time. However, launching subroutes in parallel 
is not always possible
 since subroute may share a~link and, thus, leads to congestion 
during the update process, close a~cycle, or lead to an inconsistent state of the 
system. In~\cite{delaet_seamless_2015}, it was shown that two new subroutes~$N'$ 
from~$a$ to~$b$ and~$N''$ from~$c$ to~$d$ can be launched in parallel only if~$c$ 
succeeds~$b$ or~$a$ succeeds~$d$.



%\begin{proposition}
% Let $N'$ from $a$ to $b$ and $N''$ from $c$ to $d$ be two legitimate new subroutes. $N'$ and $N''$ can be launched in parallel only if $c$ succeeds $b$ or $a$ succeeds $d$.
%%Two subroutes that are each legitimate can be launched in parallel only if they share at most one common subroute.
%\end{proposition}
%\begin{proof}
% \textbf{Direction}: $\Rightarrow$ Let $N'$ from router $a$ to $b$ and $N''$ from router $c$ to $d$, be two new legitimate sub-routes. The only way for them to share more than one common sub-route is if $b$ succeeds $c$ on $C$. In such case, launching $N'$ will eliminate the part of $C$ from $c$ to $b$ with no proper connection from $b$ to $c$, which leaves the system in an inconsistent state. The same occurs if $N''$ is launched. \\
% \textbf{Direction}: $\Leftarrow$ Let $N'$ from router $a$ to $b$ and $N''$ from router $c$ to $d$, be two new sub-routes, not necessary part of the same new route, such that $b$ precedes $c$ or $b=c$. If $a$ precedes $b$, than $N'$ is legal for launching independently of $N''$. Similarly, if $c$ precedes $b$, than $N''$ is legal for launching independently of $N'$. Thus, since $N'$ can be launched independently from $N''$, they can be launched in parallel. Symmetric considerations lead to same result in case $a$ succeeds $d$.
% 
%\noindent Generalization to more than two sub-routes is trivial.
%\end{proof}



\begin{algorithm*}[b] %[t] %alg3
\setlength{\algowidth}{100mm}
 \setlength{\hsize}{\algowidth}
 \caption{Parallel Update}
 \label{alg:parallel_update}
 \small
 
 %\Input
 {weighted graph $G_S$} 
 
 \BlankLine
 
 \While{there are still current subroutes to update}
 {
 $A \leftarrow$ find maximum-weight independent set in $G_S$ \\
 
 \BlankLine 
 \tcc{do in parallel} 
 \ForEach{$N^t_i \in A$} 
 { 
 launch $N^t_i$ \\
 }
 }
 
 \vspace*{6pt}
 
\end{algorithm*}

We create a supplementary graph~$G_S$, in which nodes are the new legitimate 
for launching subroutes, and edges represent restrictions on parallel 
launching of subroutes. See Fig.~8 for illustration, 
depicting subroutes from example in Fig.~1 and their parallel 
restrictions. For example, $N^4_2$ and~$N^1_2$ can be launched in parallel since 
there is no edge connecting them.

Clearly, any independent set of subroutes from the supplementary 
graph contains subroutes that can be launched in parallel. 
This can be further enhanced by setting QoS calculated values as weights 
on nodes of the graph and finding the subroutes that can be launched 
in parallel by finding a~maximum-weight independent set of the graph~$G_S$. 
Since~$G_S$ has few
 number of nodes (several tens), it is possible to find 
the
 maximum-weight independent set even by enumerating
 all possible independent 
sets~\cite{wu_review_2015} and comparing their total weights.
{\looseness=-1



{ \begin{center}  %fig8
\vspace*{12pt}
  \mbox{%
 \epsfxsize=36.666mm 
 \epsfbox{fre-8.eps}
 }


\end{center}


\noindent
{{\figurename~8}\ \ \small{Supplementary graph of the example in 
 Fig.~1, showing which subroutes cannot be run in parallel}
}}

%\vspace*{12pt}



} 



Important, the parallel method should not be launched on its own. 
For example, assume that at the first iteration of Algorithm~3, 
the independent sets of nodes are~$A_1$ and~$A_2$. Let us assume that~$A_1$ complies 
better to QoS demands than~$A_2$ and, thus, $A_1$ will be selected. 
Also, let us assume that~$B_1$ is the next independent set in the graph 
if~$A_1$ was selected and~$B_2$ if~$A_2$ was selected. 
Also, let us assume that~$B_1$ is
the next independent set in the graph if~$A_1$ was selected and~$B_2$ if~$A_2$ 
was selected.
It is possible that due to the dynamics of the system (see section~3), 
we could obtain overall higher QoS results if we initially launched the 
subroutes from the sets~$A_2$ and~$B_2$ afterwards than from the sets~$A_1$ and~$B_1$.
 

Therefore, the graph that we create in this section for parallelization constraints 
is a~supplementary graph which must be used in conjunction with the graphs from 
previous sections. Optimal results will be obtained when used in conjunction with 
the forest graph from subsection~5.2.

It is also important to note that, in the worst case, when there are 
no disjoint subroutes, the parallel method is reduced to the sequential 
method thought with a higher running time.

\vspace*{-12pt} 


\section{Implementation}

\noindent
We implemented the update algorithms from~\cite{delaet_seamless_2015} as 
services for our QoS verification module. The update algorithm itself 
was not modified. In other words, we treated the update itself as 
an atomic action. The route updates
 algorithms are implemented as 
applications interacting with the northbound interface of an SDN controller. 
We used POX~\cite{kaur_network_2014} as a~platform for controller development and 
Mininet~\cite{lantz_network_2010} for network topology emulation. 
Figure~9 depicts the schematic arrangement of the 
functional elements. 



We created networks with topology of random graph and small-world features. 
During each simulation trial, a~pair of common source and destination nodes $(s,d)$ 
were selected. A~path connecting~$s$ and~$d$ was selected as a~current route and 
a~set of~4~new routes connecting $(s,d)$, to replace the current route, were 
selected, possibly with shared links among themselves and the current route. 

We considered latency due to the formed congestion as QoS demands for the update, 
implemented by forming congestion on randomly selected subroutes. Route 
update was executed by the update algorithm from~\cite{delaet_seamless_2015} for 
each pair of current and new routes. Further, one of the enhanced versions 
was executed, updating to the
 preferred combination of subroutes, by identifying 
the congested subroutes (e.\,g., by estimating latency).

{ \begin{center}  %fig9
\vspace*{8pt}
  \mbox{%
 \epsfxsize=58.544mm 
 \epsfbox{fre-9.eps}
 }

\vspace*{3pt}


\noindent
{{\figurename~9}\ \ \small{Description of the system}
}
\end{center}}

%\vspace*{12pt}



%\vspace*{-45pt}

\section{Concluding Remarks}

\noindent
The study in this paper illustrates a~feasibility of modeling and 
designing the route update process via verification using DTMC. The goal was to 
strengthen the network administrator involvement in management and decision 
making during route update. In the present model, the network administrator is able 
to consider network parameters such as packet losses, delay, communication 
rounds, flow table updates, congestion, and other inherent unreliabilities of 
the network. 

We extended the updating algorithm with the ability to compute QoS as the 
MC characteristics, where the MC corresponds to the states 
of the update process. Using this MC computation ability, it is 
possible to predict the expected number of steps (delay time) required to 
complete the update process. These prediction results allow the administrator 
to make a~decision whether a~new route can satisfy the user requirements per QoS 
or a~more reliable route will be selected.

We provided sequential update algorithm and an arbitrary order algorithm 
when for the later, it is assumed that immediate duplicate packets elimination 
is possible. Further, we suggest a supplementary graph and algorithm for launching 
updates in parallel when it is possible.

This paper proposes a~conceptual approach. In future research, we will focus 
on optimization of predictions supplementing the network administrator with 
a~powerful tool which will be able to enhance the update process 
with fine grained analysis of the network.

\vspace*{-12pt}


\Ack
\noindent
The first author has partially been supported by the 
Russian Foundation for Basic Research under grants RFBR 18-07-00669 and 18-29-03100. 
The second author has partially been supported by the Rita Altura Trust Chair in
Computer Sciences; The Lynne and William Frankel Center for Computer
Science.

%\bigskip


The authors thank Prof.\ Shlomi Dolev 
for his valuable input and Prof.\ Yefim Dinitz for his comments.
 
\renewcommand{\bibname}{\protect\rmfamily References}

%\vspace*{-6pt}

\vspace*{-6pt}

{\small\frenchspacing
{\baselineskip=10.35pt
\begin{thebibliography}{99}



\bibitem{rao_sdn_2014}  %1
\Aue{Rao, S.\,K.} 2014. SDN and its use-cases~--- NV and NFV:
A~state-of-the-art survey. NEC Technologies India Ltd. 25~p.

\bibitem{ghaznavi_service_2016}  %2
\Aue{Ghaznavi, M., N.~Shahriar, R.~Ahmed, and R.~Boutaba}. 2016. 
Service function chaining simplified. {arXiv.org}. arXiv:1601.00751.

\bibitem{hansson_logic_1994}  %3
\Aue{Hansson, H., and B.~Jonsson}. 
1994. A~logic for reasoning about time and reliability. 
\textit{Form. Asp. Comput.} 6(5):512--535.

\bibitem{delaet_seamless_2015}  %4
\Aue{Delaet, S., S.~Dolev, D.~Khankin, S.~Tzur-David, and T.~Godinger}. 
2015. Seamless SDN route updates. \textit{IEEE 14th Symposium (International)
on Network Computing and Applications}. IEEE. 120--125.

\bibitem{frenkel_predicting_2017} 
\Aue{Frenkel, S., D.~Khankin, and A.~Kutsyy}. 
2017. Predicting and choosing alternatives of route updates per QoS VNF in SDN. 
\textit{IEEE 16th Symposium (International) on Network Computing and Applications}. 
IEEE. 1--6. 

\bibitem{devi_approach_2015} 
\Aue{Devi, G., and S.~Upadhyaya}. 2015. 
An approach to distributed multi-path QoS routing. 
\textit{Indian J.~Sci. Technol.} 8(20):1--14. 
doi: 10.17485/ijst/2015/v8i20/49253.

\bibitem{egilmez_distributed_2012} 
\Aue{Egilmez, H.\,E., S.~Civanlar, and A.\,M.~Tekalp}. 2012. 
A~distributed QoS routing architecture for scalable video streaming over multi-domain 
OpenFlow networks. \textit{19th IEEE Conference (International) on Image Processing}.
IEEE. 2237--2240.

\bibitem{juttner_lagrange_2001} 
\Aue{Juttner, A., B.~Szviatovski, I.~Mecs, and Z.~Rajko}. 2001. 
Lagrange relaxation based method
for the QoS routing problem. \textit{IEEE Conference on Computer Communications. 
20th Annual Joint Conference of the IEEE Computer and Communications Society
 Proceedings}. IEEE. 2:859--868.

\bibitem{yu_efficient_2013} %9
\Aue{Yu, Z., F.~Ma, J.~Liu, B.~Hu, and Z.~Zhang}. 2013. 
An efficient approximate algorithm for disjoint QoS routing.
\textit{Math. Probl. Eng.} 2013:489149. 9~p. 
doi: 10.1155/2013/489149.

\bibitem{foerster_survey_2016} 
\Aue{Foerster, K.-T., S.~Schmid, and S.~Vissicchio} 2016. 
A~survey of consistent network updates. \mbox{Arxiv.org}. \mbox{arXiv}:\linebreak 1609.02305.

\bibitem{reitblatt_consistent_2011} 
\Aue{Reitblatt, M., N.~Foster, J.~Rexford, and D.~Walker}. 
2011. Consistent updates for software-defined networks: Change you can believe in! 
\textit{10th ACM Workshop on Hot Topics in Networks Proceedings}.
New York, NY: ACM. Art.\ No.\,7. doi: 10.1145/2070562.2070569.

\bibitem{hogan_stochastic_2017} 
\Aue{Hogan, M., and F.~Esposito}. 
2017. Stochastic delay forecasts for edge traffic engineering via Bayesian networks. 
\textit{IEEE 16th Symposium (International) on Network Computing and Applications}. 
IEEE. 1--4.

\bibitem{mcgeer_safe_2012} %15
\Aue{McGeer, R.} 2012. A~safe, efficient Update Protocol for Openflow Networks. 
\textit{1st Workshop on Hot Topics in Software Defined Networks Proceedings}. 
New York, NY: ACM. 12:61--66.
\bibitem{mcgeer_correct_2013} 
\Aue{McGeer, R.} 2013. A~correct, zero-overhead protocol for network updates. 
\textit{2nd ACM SIGCOMM Workshop on Hot Topics in Software Defined Networking
Proceedings}. New York, NY: ACM. 13:161--162.
\bibitem{katta_incremental_2013} 
\Aue{Katta, N.\,P., J.~Rexford, and D.~Walker}. 
2013. Incremental consistent updates. \textit{2nd ACM SIGCOMM Workshop on Hot Topics 
in Software Defined Networking Proceedings}.
New York, NY: ACM. 13:49--54.

\bibitem{dinitz_dependence_2017}  %16
\Aue{Dinitz, Y., S.~Dolev, and D.~Khankin}. 
2017. Dependence graph and master switch for seamless dependent routes 
replacement in SDN. \textit{IEEE 16th Symposium 
(International) on Network Computing and Applications}. IEEE. 1--7.

\bibitem{17-aaa}
\Aue{Amiri, S.\,A., S.~Dudycz, S.~Schmid, and S.~Wiederrecht}.
2016. Congestion-free rerouting of flows
on DAGs. \mbox{ArXiv}.org. arXiv:1611.09296.
% [cs, math], Nov. 2016, arXiv: 1611.09296. [Online]. Available:
%http://arxiv.org/abs/1611.09296

\bibitem{kwiatkowska_prism_2011}  %17
\Aue{Kwiatkowska, M., G.~Norman, and D.~Parker}. 2011. 
PRISM~4.0: Verification of probabilistic real-time systems. 
\textit{Computer aided verification}.
Eds. G.~Gopalakrishnan and S.~Qadeer.
Lecture notes in computer science ser. Springer.
6806:585--591.

\bibitem{noauthor_prism_nodate}  %18
\Aue{Kwiatkowska, M., G.~Norman, and D.~Parker}. 2018. 
{PRISM manual}. Available at:
{\sf http://www.\linebreak prismmodelchecker.org/manual/}
(accessed December~10, 2018).

\bibitem{onf_openflow_2015} %19
{Open Networking Foundation}. 2015. 
OpenFlow Switch Specification Ver~1.5.1. 


\bibitem{wu_review_2015}  %20
\Aue{Wu, Q., and J.-K.~Hao}. 2015. 
A~review on algorithms for maximum clique problems. 
\textit{Eur. J.~Oper. Res.} 242(3):693--709.

\bibitem{kaur_network_2014}  %21
\Aue{Kaur, S., J.~Singh, and N.\,S.~Ghumman}. 2014. 
Network programmability using POX controller. 
\textit{Conference (International) on Communication, Computing and Systems}.
138.

\bibitem{lantz_network_2010}  %22
\Aue{Lantz, B., B.~Heller, and N.~McKeown}. 2010. 
A~network in a~laptop: Rapid prototyping for software-defined networks. 
\textit{9th ACM SIGCOMM Workshop on Hot Topics in Networks Proceedings}. 
New York, NY: ACM.  Art.\ No.\,19. doi: 10.1145/1868447.1868466.
\end{thebibliography} } }

\end{multicols}

\vspace*{-9pt}

\hfill{\small\textit{Received October 9, 2018}}

\vspace*{-22pt}

\Contr

\vspace*{-3pt}

\noindent
\textbf{Frenkel Sergey L.} (b.\ 1951)~--- 
Candidate of Science (PhD) in technology, associate professor, 
senior scientist, Institute of Informatics Problems, Federal Research Center 
``Computer Sciences and Control'' of the Russian Academy of Sciences, 
44-2~Vavilov Str., Moscow 119333, Russian Federation; \mbox{fsergei51@gmail.com}

\vspace*{1pt}

\noindent
\textbf{Khankin D.} (b.\ 1983)~--- MSc, doctorate student, Department of Computer 
Science, Ben-Gurion University of the Negev, Beer-Sheva 84105, Israel; 
\mbox{danielkh@post.bgu.ac.il}

\vspace*{4pt}

\hrule

\vspace*{2pt}

\hrule

\vspace*{-7pt}

%\newpage

%\vspace*{-28pt}

\def\tit{НЕПРЕРЫВНЫЕ ОБНОВЛЕНИЯ МАРШРУТА В~SDN С~ИСПОЛЬЗОВАНИЕМ ПРОВЕРКИ СООТВЕТСТВИЯ 
КАЧЕСТВУ~ОБСЛУЖИВАНИЯ$^*$\\[-7pt]}

\def\titkol{Непрерывные обновления маршрута в~SDN с~использованием проверки соответствия 
качеству обслуживания}

\def\aut{С.\,Л.~Френкель$^1$, Д.~Ханкин$^2$\\[-7pt]}

\def\autkol{С.\,Л.~Френкель, Д.~Ханкин}

{\renewcommand{\thefootnote}{\fnsymbol{footnote}} \footnotetext[1]
{Работа была частично поддержана РФФИ (гранты 18-07~00669 и~18-29-03100), 
а~также Rita Altura Trust Chair in
Computer Sciences; The Lynne and William Frankel Center for Computer
Science.}}



\titel{\tit}{\aut}{\autkol}{\titkol}

\vspace*{-22pt}

\noindent
$^1$Институт проблем информатики Федерального исследовательского центра 
<<Информатика и~управление>>\linebreak
$\hphantom{^1}$Российской академии наук
%, fsergei51@gmail.com 

\noindent
$^2$Университет им.\ Бен-Гуриона в Негеве, Беэр-Шева, Израиль
%, danielkh@post.bgu.ac.il 

\vspace*{1pt}

\def\leftfootline{\small{\textbf{\thepage}
\hfill ИНФОРМАТИКА И ЕЁ ПРИМЕНЕНИЯ\ \ \ том\ 12\ \ \ выпуск\ 4\ \ \ 2018}
}%
 \def\rightfootline{\small{ИНФОРМАТИКА И ЕЁ ПРИМЕНЕНИЯ\ \ \ том\ 12\ \ \ выпуск\ 4\ \ \ 2018
\hfill \textbf{\thepage}}}

\vspace*{-1pt}


 
\Abst{В программно-определяемой сети (SDN~--- software-defined networking) 
уровень управ\-ле\-ния 
и~уровень данных разделены. Это обеспечивает высокую гибкость эксплуатации, 
предоставляя абстракции для управления сетью приложений 
и~возможность непосредственного программирования маршрутов.
Однако из-за изменений топологии, процедуры обслуживания или происходящих 
сбоев иногда необходима реконфигурация и~обновление сети. 
В~предлагаемом сценарии рассматривается текущий маршрут~$C$
и~набор возможных новых маршрутов~~$\{N_i\}$, где для замены текущего 
маршрута требуется 
один из\linebreak\vspace*{-12pt}}

\Abstend{новых маршрутов. Существует вероятность того, что новый маршрут~$N_i$ 
окажется длиннее некоторого другого нового маршрута~$N_j$, но при этом~$N_i$ 
будет более надежным и~он будет обновляться быстрее или работать лучше 
после обновления с~точки зрения требований качества обслуживания (QoS~---
quality of service). Принимая 
во внимание случайный характер функционирования сети, авторы дополнили недавно 
предложенный алгоритм обновления маршрута Delaet с~соавт.\ методом оценки соблюдения 
требований QoS во время непрерывного обновления маршрута, основанным на 
использовании цепей Маркова. При этом, во-пер\-вых, предлагается расширить 
алгоритм передачи пакетов по выбранному маршруту, сравнивая процесс обновления 
для возможных альтернатив маршрута. Во-вто\-рых, предлагается несколько 
способов выбора комбинаций предпочтительных отрезков путей новых маршрутов, 
что приводит к оптимальному в~смысле соответствия QoS маршруту.}


\KW{программно-определяемые сети; цепи Маркова; качество обслуживания}

\DOI{10.14357/19922264180408}



%\vspace*{-3pt}


 \begin{multicols}{2}

\renewcommand{\bibname}{\protect\rmfamily Литература}
%\renewcommand{\bibname}{\large\protect\rm References}

{\small\frenchspacing
{\baselineskip=10.5pt
\begin{thebibliography}{99}
%\vspace*{-3pt}


\bibitem{2-fr-1}
\Au{Rao S.\,K.} SDN and its use-cases~--- NV and NFV: A~state-of-the-art survey.~--- 
NEC Technologies India Ltd., 2014. 25~p.
\bibitem{3-fr-1}
\Au{Ghaznavi M., Shahriar~N., Ahmed~R., Boutaba~R.} 
Service function chaining simplified~// Arxiv.org, 2016. \mbox{arXiv}:1601.00751cs.
\bibitem{4-fr-1}
\Au{Hansson H., Jonsson~B.} A~logic for reasoning about time and reliability~// 
Form. Asp. Comput., 1994. Vol.~6. No.\,5. P.~512--535.

\bibitem{1-fr-1} %4
\Au{Delaet S., Dolev~S., Khankin~D., Tzur-David~S., Godinger~T.}
Seamless SDN route updates~// IEEE 14th Symposium (International)
 on Network Computing and Applications.~--- IEEE, 2015. P.~120--125.
 
 
\bibitem{5-fr-1}
\Au{Frenkel S., Khankin D., Kutsyy~A.} Predicting and choosing alternatives 
of route updates per QoS VNF in SDN~// IEEE 16th Symposium (International)
on Network Computing and Applications.~--- IEEE, 2017. P.~1--6.
\bibitem{6-fr-1}
\Au{Devi G., Upadhyaya~S.} An approach to distributed multi-path QoS routing~// 
Indian J.~Sci. Technol., 2015. Vol.~8. Iss.~20. P.~1--14. 
doi: 10.17485/ijst/2015/v8i20/49253.
\bibitem{7-fr-1}
\Au{Egilmez H.\,E., Civanlar S., Tekalp~A.\,M.} 
A~distributed QoS routing architecture for scalable video streaming over multi-domain 
OpenFlow networks~// 19th IEEE Conference (International)
on Image Processing.~--- IEEE, 2012. P.~2237--2240.
\bibitem{8-fr-1}
\Au{Juttner A., Szviatovski B., Mecs~I., Rajko~Z.}
Lagrange relaxation based method for the QoS routing problem~// 
IEEE INFOCOM 2001 Conference on Computer Communications. 20th 
Annual Joint Conference of the IEEE Computer and Communications Society
Proceedings.~--- IEEE, 2001. Vol.~2. P.~859--868.
\bibitem{9-fr-1}
\Au{Yu Z., Ma F., Liu~J., Hu~B., Zhang~Z.}
An efficient approximate algorithm for disjoint QoS routing~// 
Math. Probl. Eng., 2013. Vol.~2013. Art.\ No.\,489149. 9~p. 
doi: 10.1155/2013/489149.
\bibitem{10-fr-1}
\Au{Foerster K.-T., Schmid S., Vissicchio~S.}
A~survey of consistent network updates~// Arxiv.org, 2016. arXiv:1609.02305.
\bibitem{11-fr-1}
\Au{Reitblatt M., Foster N., Rexford J., Walker~D.} 
Consistent updates for software-defined networks: Change you can believe in!~// 
10th ACM Workshop on Hot Topics in Networks Proceedings.~--- New York, NY, USA: ACM, 
2011. Art.\ No.\,7. doi: 10.1145/2070562.2070569.
\bibitem{12-fr-1}
\Au{Hogan M., Esposito F.} Stochastic delay forecasts for edge traffic engineering 
via Bayesian Networks~// IEEE 16th Symposium (International)
on Network Computing and Applications.~--- IEEE, 2017. P.~1--4.
\bibitem{13-fr-1}
\Au{McGeer R.} A~safe, efficient Update Protocol for Openflow Networks~// 
1st Workshop on Hot Topics in Software Defined Networks Proceedings.~--- 
New York, NY, USA: ACM, 2012. Vol.~12. P.~61--66.
\bibitem{14-fr-1}
\Au{McGeer R.} 2013. A~correct, zero-overhead protocol for network updates~// 
2nd Workshop on Hot Topics in Software Defined Networking Proceedings.~--- 
New York, NY, USA: ACM, 2013. Vol.~13. P.~161--162.
\bibitem{15-fr-1}
\Au{Katta N.\,P., Rexford J., Walker~D.} Incremental consistent updates~// 
2nd Workshop on Hot Topics in Software Defined Networking Proceedings.~--- 
New York, NY, USA: ACM, 2013. Vol.~13. P.~49--54.
\bibitem{16-fr-1}
\Au{Dinitz Y., Dolev S., Khankin~D.}
 Dependence graph and master switch for seamless dependent 
 routes replacement in SDN~// IEEE 16th Symposium 
 (International) on Network Computing and Applications.~--- IEEE, 2017. P.~1--7.
 \bibitem{17-aaa-1}
\Au{Amiri~S.\,A., Dudycz~S., Schmid~S., Wiederrecht~S}.
 Congestion-free rerouting of flows
on DAGs~// ArXiv.org, 2016. arXiv:1611.09296.
% [cs, math], Nov. 2016, arXiv: 1611.09296. [Online]. Available:
%http://arxiv.org/abs/1611.09296

\bibitem{17-fr-1}
\Au{Kwiatkowska M., Norman~G., Parker~D.}
 PRISM~4.0: Verification of probabilistic real-time systems~//
 Computer aided verification~/
 Eds. G.~Gopalakrishnan, S.~Qadeer.~---
Lecture notes in computer science ser.~--- Springer, 2011. 
 Vol.~6806. P.~585--591.
\bibitem{18-fr-1}
\Au{Kwiatkowska M., Norman G., Parker~D.}
 PRISM manual, 2018. 
{\sf http://www.prismmodelchecker.org/manual}.
\bibitem{19-fr-1}
Open Networking Foundation. OpenFlow Switch Specification Ver~1.5.1, 2015. 

\bibitem{21-fr-1}
\Au{Wu Q., Hao J.-K.} A~review on algorithms for maximum clique problems~// 
Eur. J.~Oper. Res., 2015. Vol.~242. No.\,3. P.~693--709.

\bibitem{20-fr-1}
\Au{Kaur S., Singh J., Ghumman~N.\,S.}
 Network programmability using POX controller~// Conference
 (International) on Communication, Computing and Systems, 2014. P.~138.
\bibitem{22-fr-1}
\Au{Lantz B., Heller B., McKeown~N.} 
A~network in a~laptop: Rapid prototyping for software-defined networks~// 
9th ACM SIGCOMM Workshop on Hot Topics in Networks Proceedings.~--- 
New York, NY, USA: ACM, 2010. Art.\ No.\,19. doi: 10.1145/1868447.1868466.
\end{thebibliography}
} }

\end{multicols}

 \label{end\stat}

 \vspace*{-9pt}

\hfill{\small\textit{Поступила в~редакцию 09.10.2018}}


%\renewcommand{\bibname}{\protect\rm Литература}
\renewcommand{\figurename}{\protect\bf Рис.}
\renewcommand{\tablename}{\protect\bf Таблица} %8
\def\stat{strijov}

\def\tit{ВОССТАНОВЛЕНИЕ МАТРИЦЫ СУПЕРПОЗИЦИИ В~ЗАДАЧЕ~СИМВОЛЬНОЙ РЕГРЕССИИ$^*$}

\def\titkol{Восстановление матрицы суперпозиции в~задаче символьной регрессии}

\def\aut{Р.\,Г.~Нейчев$^1$, И.\,А.~Шибаев$^2$, В.\,В.~Стрижов$^3$}

\def\autkol{Р.\,Г.~Нейчев, И.\,А.~Шибаев, В.\,В.~Стрижов}

\titel{\tit}{\aut}{\autkol}{\titkol}

\index{Нейчев Р.\,Г.}
\index{Шибаев И.\,А.}
\index{Стрижов В.\,В.}
\index{Neychev R.\,G.}
\index{Shibaev I.\,A.}
\index{Strijov V.\,V.}


{\renewcommand{\thefootnote}{\fnsymbol{footnote}} \footnotetext[1]
{Работа выполнена при поддержке РФФИ (проекты 20-37-90050 и~20-07-00990).}}


\renewcommand{\thefootnote}{\arabic{footnote}}
\footnotetext[1]{Московский физико-технический институт, 
\mbox{neychevr@gmail.com}}
\footnotetext[2]{Московский физико-технический институт, 
\mbox{shibaev.kesha@gmail.com}}
\footnotetext[3]{Федеральный исследовательский центр <<Информатика 
и~управ\-ле\-ние>> Российской академии наук, \mbox{strijov@phystech.edu}}

\vspace*{-12pt}
 



\Abst{Исследуется проблема порождения структуры регрессионной модели. 
Модель представляет собой суперпозицию базовых функций. Структура модели 
описывается взвешенным цвет\-ным графом. Каждая вершина графа соответствует 
некоторой базовой функции. Ребро задает суперпозицию двух функций. Вес ребра 
равен вероятности суперпозиции. Для создания оптимальной модели необходимо 
восстановить ее структуру по матрице смежности графа. Пред\-ла\-га\-емый алгоритм 
восстанавливает минимальное остовное дерево из взвешенного цветного графа. 
Пред\-став\-ле\-но новое решение, основанное на алгоритме дерева Штейнера. 
Алгоритм сравнивается с~альтернативами.}


\KW{символьная регрессия; линейное программирование; 
суперпозиция; минимальное остовное дерево; мат\-ри\-ца смеж\-ности}

\DOI{10.14357/19922264230105} 
  
\vspace*{-8pt}


\vskip 10pt plus 9pt minus 6pt

\thispagestyle{headings}

\begin{multicols}{2}

\label{st\stat}

\section{Введение}

Символьная регрессия~--- это метод по\-стро\-ения нелинейной модели, 
аппроксимирующей выборку. Структура модели определяется суперпозицией базовых 
функций. Набор базовых функций фиксируется для конкретной прикладной задачи. 
Структуры альтернативных моделей генерируются алгоритмом оптимизации для выбора 
оптимальной модели. В данной статье предлагается определять структуру модели 
с~по\-мощью вероятностного графа. Остовное дерево в~графе определяет некоторую 
суперпозицию. Для выбора оптимальной модели необходимо реконструировать 
минимальное остовное дерево по графу.

Методы генетического программирования~\cite{koza1992genetic} находят оптимальное 
подмножество в~наборе суперпозиций базовых функций, но имеют высокую 
вычислительную сложность. В~\cite{searson2010gptips} описаны методы, понижающие 
сложность. Они используют дополнительные ограничения на суперпозиции, например 
используют линейные комбинации базовых функций. Символьная регрессия, 
описанная~в~\cite{stanley2002evolving}, используется для оптимизации структуры 
суперпозиции. Методы решения задачи символьной регрессии основаны на матричном 
представлении структуры модели~\cite{bochkarev2017generation}. Однако эти методы 
не содержат ограничений на чис\-ло аргументов базовых функций и~на структуру 
графа, обеспечивающую допустимую суперпозицию. В~данной работе решается задача 
построения модели с~помощью символьной регрессии.

Требуется восстановить допустимую суперпозицию из предсказанной мат\-ри\-цы 
смежности с~вероятностями ребер. Решается задача вос\-ста\-нов\-ле\-ния~$k$-минимального 
остовного дерева $k$-MST (\textit{англ.}\ Minimum-cost Spanning Tree). Эта задача NP-слож\-ная, 
поэтому применимы только при\-бли\-жен\-ные решения~\cite{ravi1996spanning}. 
Алгоритм~$k$-MST эквивалентен проб\-ле\-ме дерева Штейнера PCST (\textit{англ.}\ 
Prize-Collecting Steiner Tree) из-за его эквивалентности ослабленной формулировке 
постановки задачи линейного программирования~\cite{chudak2004approximate}. 
В~работах~\cite{ravi1996spanning,awerbuch1998new,arora20062+} пред\-став\-ле\-ны 
приближенные решения задачи \mbox{$k$-MST}.



Предлагаемое решение основано на упрощенной версии задачи~$k$-MST, которая 
трансформируется в~задачу PCST с~постоянными призами, одинаковыми для всех 
вершин. Быст\-рый алгоритм PSCT описан в~\cite{hegde2014fast}. Альтернативное 
решение основано на алгоритме~$(2-\varepsilon)$-аппроксимации для задачи PSCT. 
Она сравнивается с~другими алгоритмами, включая алгоритмы обхода дерева в~глубину, обхода дерева в~ширину, алгоритмы Прима.

\begin{table*}[b]\small  %tabl1
\vspace*{-12pt}
\begin{center}
        \parbox{262pt}{\Caption{Вероятности суперпозиций в~матрице смежности порождают 
ориентированный граф}

}
    \label{restored_adjacency_matrix}
\vspace*{2ex}

        \begin{tabular}{|c|c|ccccccc|}
            \hline
            Арность&Функция&$\ast$&$+$&$\ln$&$\sin$&$\times$&$\exp$&$x$\\
            \hline
            $1$&$\ast$ &0,2&{\bf 0,7}&0,5&0,4&0,5&0,3&0,2\\
            $3$&$+$    &0,3&0,2&{\bf 1,0}&{\bf 0,8}&0,6&0,3&{\bf 0,7}\\
            $1$&$\ln$  &0,3&0,2&0,0&0,0&0,1&0,5&{\bf 0,5}\\
            $1$&$\sin$ &0,1&0,4&0,0&0,5&{\bf 0,9}&0,2&0,5\\
            $2$&$\times$&0,3&0,0&0,3&0,5&0,0&{\bf 0,8}&{\bf 0,6}\\
            $1$&$\exp$ &0,3&0,3&0,4&0,1&0,5&0,4&{\bf 0,4}\\
            \hline
        \end{tabular}
\end{center}
\end{table*}

\vspace*{-12pt}


\section{Задача выбора регрессионной модели}

\vspace*{-3pt}

Требуется выбрать регрессионную модель~$\varphi$ из набора альтернативных 
моделей. Модель описывает выборку~$D=\{(x_i,y_i)\}$ и~минимизирует ошибку

\noindent
\begin{equation}
\hat{\varphi}(D)=\mathop{\argmin}\limits_\varphi\sum\limits_{i=1}^m\left(\varphi(x_i)-
y_i\right)^2.
\label{task_1}
\end{equation}
Модель представляет собой суперпозицию базовых функций из некоторого заданного 
набора. На рис.~1\linebreak\vspace*{-12pt}

{ \begin{center}  %fig1
 \vspace*{-3pt}
    \mbox{%
\epsfxsize=37.447mm
\epsfbox{str-1.eps}
}

\end{center}

\vspace*{-2pt}

\noindent
{{\figurename~1}\ \ \small{Структура регрессионной модели представляет собой ориентированный 
граф
}}}

\vspace*{6pt}

\addtocounter{figure}{1}


\noindent
 показан ее пример. Структура модели~$\varphi$, 
суперпозиция, соответствует графу~$G=(V,E)$, где базовые функции находятся 
в~вершинах~$V$. {Корневая} вершина обозначается через~$\ast$. Модель:

\vspace*{1pt}

\noindent
$$
\varphi(D) =  \ln(x) + x + \sin\left( x\times \exp(x)\right).
$$

\vspace*{-4pt}

\noindent
 Еe структура в~виде матрицы 
смежности графа пред\-став\-ле\-на~в табл.~\ref{restored_adjacency_matrix}.
Базовые функции перечислены в~первой строке. Элементами матрицы являются 
вероятности ребер~$E$ дерева. Жир\-ным шриф\-том выделены ребра восстановленного 
дерева~$M$, образующие суперпозицию~$\varphi$. Для восстановления структуры 
модели~$\varphi$ как суперпозиции, заданной деревом~$M$, необходимы только 
графовое пред\-став\-ле\-ние~$G$~и~базовые функции.



Поставим задачу восстановления структуры модели. Задано множество 
выборок~$\{D_k\}$. Каждой выборке~$D_k$ соответствует своя модель. Эта модель 
имеет структуру~$M_k$. Таким образом, имеется набор пар $\{(D_k, M_k)\}$, 
выборка и~структура.
Обозначим через~$P$ отображение, которое предсказывает вероятности узлов 
в~графе~$G$ по выборке~$D$. Для выбора модели~$\varphi(D)$ необходимо восстановить 
структуру модели~$M$ по графу~$G$. Обозначим алгоритм восстановления дерева 
через~$R$. Регрессионная модель~$\hat{\varphi}(D)$, которая решает 
задачу~(\ref{task_1}), определяется формулой
$
\hat{M}=R\left(P(D)\right).
$
Поскольку дерево~$M$ играет центральную роль в~этой работе, критерий качества 
алгоритма восстановления дерева имеет вид:


\vspace*{-3pt}

\noindent
$$
\min_{M_k \in G} \fr{1}{K}\sum\limits_{k=1}^K \left[ \hat{M_k} = M_k\right].
$$

\vspace*{-4pt}

\noindent
Восстановленное дерево должно быть эквивалентно заданному дереву, следовательно, 
выбранная модель регрессии при\-бли\-жа\-ет выборку.

\vspace*{-10pt}

\section{Задача восстановления дерева суперпозиции}

\vspace*{-3pt}

Требуется восстановить дерево~$M_k$, задающее  суперпозицию и~решающее 
задачу~(\ref{task_1}). Задан ориентированный взвешенный граф~$G\hm=(V,E)$ 
с~раскрашенными вершинами~$v_i$ и~корневой вершиной~$r$. Каждая вершина~$v_i \hm\in 
V$ имеет свой цвет~$t(v_i)\hm=t_i$. Каждое реб\-ро~$e_i\in E$ имеет свой 
вес~\mbox{$w(e_i)\hm=c_i\hm\in[0,1]$}.

Требуется восстановить ориентированное дерево минимального веса с~корнем~$r$. 
Оно должно покрывать не менее~$k$ вершин в~заданном графе~$G$. Чис\-ло ребер, 
выходящих из вершины~$v_i$ дерева, должно быть меньше или равно~$t_i$. 
Корень~$r$ имеет одно ребро,~$t_r=1$.

Сформулируем это условие в~виде задачи линейного программирования 
с~целочисленными ограничениями:

\vspace*{-5pt}

\noindent
\begin{multline}
\underset{\substack{{x_e, z_S} \\ e\in E,\\ S\subseteq V\backslash 
\{r\}}}{\mbox{minimize}}  \displaystyle \sum\limits_{e\in E}c_ex_e \\[-3pt]
\mbox{s.t.}\  \displaystyle  \sum\limits_{\substack{{e\in\delta(S):}\\ e=(\ast,v_i),\\ v_i\in\delta(S)}} \!\!\!\! x_e + 
\sum\limits_{T:T\supseteq S}  \!\!\!\! z_T\geqslant 1,\enskip  S\subseteq 
V\backslash \{r\};\\[-3pt]
 \displaystyle \sum\limits_{e\in E:~e=(\ast,v)} \! x_e\leqslant 1,\enskip v\in V;\\[-3pt]
 \displaystyle \sum\limits_{e\in E:~e=(v,\ast)}x_e\leqslant t_i,\enskip  v\in V;\\[-3pt]
 \displaystyle \sum\limits_{S\subseteq V\backslash \{r\}}|S|z_S \leqslant n-k,\enskip  x_e\in\{0,1\},\enskip 
 z_S\in\{0,1\},\\[-3pt]
  e\in E,\enskip   S\subseteq V\backslash \{r\},
\label{ilp_our}
\end{multline}
где
$$
x_e =\begin{cases}
 1, &\mbox{если\ ребро}\ e\ \mbox{входит\ в~финальную}\\
 &\mbox{суперпозицию};\\
 0 & \mbox{в~противном\ случае};
 \end{cases}
 $$
  $z_S\hm = 1$ для всех вершин, исключенных из финальной 
суперпозиции. Обозначим через~$e\hm=(\ast, v)$ ориентированное ребро с~листом~$v$. 
Обозначим через $e\hm=(v, \ast)$ ориентированное ребро с~вершиной~$v$.

Первое ограничение~(\ref{ilp_our})  определяет структуру графа решения в~виде 
дерева с~корнем~$r$. Второе ограничение определяет ориентацию дерева: каждая 
вершина имеет не более одного входящего ребра. Третье ограничение определяет 
арность используемых базовых функций, поэтому число ребер, имеющих определенную 
вершину в~качестве источника, фиксировано. Четвертое ограничение говорит, что 
итоговое дерево имеет не менее~$k$ вершин. Если все веса неотрицательны, то 
четвертое ограничение на минимальное число вершин принимает более строгий вид: 
число вершин должно быть равно~$k$. Однако более слабое ограничение позволяет 
найти возможные связи с~другими оптимизационными задачами. Ограничения 
в~(\ref{ilp_our}) преследуют ту же цель.

\vspace*{-9pt}

\section{Алгоритмы восстановления дерева $k$-MST и~PCST}

\vspace*{-3pt}

\noindent
\textbf{Определение~1} (\textbf{$\bm{k}$-минимальное остовное дерево,\linebreak $\bm{k}$-MST}).
Задан взвешенный граф~$G\hm=(V,E)$ с~корнем~$r$ и~весами ребер~$w(e_i)\hm=c_i\hm\geqslant 
0$, $e_i\hm\in E$. Требуется построить ориентированное дерево минимального веса 
с~корнем~$r$, покрывающее не менее~$k$ вершин в~$G$.

\smallskip

Если та же задача ставится для ориентированных графов, то конечное дерево 
с~корнем~$r$ должно быть ориентированным. Задача линейного программирования для 
направленного~$k$-MST исключает \mbox{третье} условие в~(\ref{ilp_our}).
В~таком виде задача~$k$-MST отличается от исходной задачи восстановления 
дерева суперпозиций~(\ref{ilp_our}) отсутствием третьего ограничения на арность 
базовых функций. Это эквивалентно ограничению на число ребер, выходящих из 
вершины.

\smallskip

\noindent
\textbf{Определение~2} (\textbf{призовое дерево Штейнера, $\text{PCST}$}).\linebreak
Задан взвешенный граф $G\hm=(V,E)$ с~корнем~$r$ и~весами ребер~$w(e_i)\hm=c_i\hm\geqslant  0$, $e_i\hm\in E$, где каждой вершине~$v_i \hm\in V$ присвоен 
{приз} $\pi(v_i)\hm=\pi_i\geqslant 0$. Требуется построить дерево~$T$ с~корнем~$r$, 
которое \mbox{минимизирует} функционал
$\sum\nolimits_{e\in E}c_ex_e \hm+ \sum\nolimits_{S\subseteq V\backslash\{r\}} 
\pi(S)z_S,$
где~$x_e\in\{0, 1\}$, $x_e\hm=1$, если~$e\hm\in E$ входит в~тройку~$T$; $z_S\hm\in\{0, 1\}$, 
$z_S\hm=1$ для всех вершин, исключенных из дерева~$T$; $S \hm= V\backslash V(T)$; $\pi(S)\hm= \sum\nolimits_{v\in S}\pi(v)$.

\smallskip

В случае ориентированных графов эта задача обобщается до~асимметричной задачи 
A-PCST. Задача линейного программирования для~A-PCST принимает вид:

\vspace*{-4pt}

\noindent
\begin{multline}
\underset{\substack{x_e,z_S \\ e\in E,\\ S\subseteq V\backslash \{r\}}}{\mbox{minimize}} 
\displaystyle \sum\limits_{e\in E} c_e x_e + \sum\limits_{S\subseteq V\backslash\{r\}}  \!\!\!\!\!\pi(S)z_S \\
\mbox{s.t.}\ \displaystyle \sum\limits_{\substack{e\in\delta(S):\\e=(\ast,v_i),\\ v_i\in\delta(S)}} \!\!\!\!\!\! x_e + 
\sum\limits_{T:T\supseteq S}  \!\!\! z_T\geqslant 1,\enskip  S\subseteq  V\backslash \{r\};\\
\displaystyle \sum\limits_{e\in E:~e=(\ast,v)}\!\!\!\!  x_e\leqslant 1,\enskip
x_e\in\{0,1\},\enskip z_S\in\{0,1\},\enskip  v\in V,\\
e\in E,\enskip S\subseteq V\backslash \{r\}.
\label{ilp_pcst_ord}
\end{multline}

\vspace*{-3pt}

\noindent
Если последнее ограничение из~(\ref{ilp_our}) входит в~оптимизируемый 
функционал, задачи $k$-MST и~A-PCST имеют эквивалентные 
ограничения и~отличаются только оптимизируемым функционалом. Такое 
преобразование возможно согласно условиям Ка\-ру\-ша--Ку\-на--Так\-ке\-ра~\cite{ras2017approximate}. Если значения призов 
эквивалентны $\pi(v) \hm=  \lambda$, единственное отличие состоит в~постоянном члене~$\lambda(n\hm-k)$. Таким 
образом, задачи оптимизации~$k$-MST и~A-PCST принимают вид:

\vspace*{-4pt}

\noindent
\begin{align*}
\underset{\substack{x_e,z_S \\ e\in E,\\ S\subseteq V\backslash \{r\}}}{\mbox{minimize}} & 
\sum\limits_{e\in E}c_ex_e + \lambda\left(\sum\limits_{S\subseteq V\backslash \{r\}}|S|z_S - (n-k)\right);\\ 
\underset{\substack{x_e,z_S \\ e\in E,\\ S\subseteq V\backslash \{r\}}}{\mbox{minimize}} & 
\sum\limits_{e\in E}c_ex_e + \lambda\sum\limits_{S\subseteq V\backslash\{r\}}|S|z_S\,. 
\end{align*}
%
Константа~$\lambda$ обозначает неотрицательный множитель Лагранжа в~задаче~$k$-MST и~приз за вершину\linebreak 
в~задаче~A-PCST. 
Существуют несколько алгоритмов для решения проблемы~PCST, но не для 
решения проб\-ле\-мы A-PCST. Возможное решение~--- снять 
ограничения на ориентацию графа, чтобы\linebreak алгоритм~PCST мог позже 
восстановить ориентацию дерева.

\vspace*{-9pt}

\section{Решение задачи восстановления ограниченного леса с~помощью алгоритма 
$(2-\varepsilon)$-приближения}

\vspace*{-3pt}

Обзор методов решения задачи восстановления ограниченного леса представлен 
в~\cite{goemans1995general}. Задан взвешенный неориентированный граф~$G\hm=(V,E)$. 
Все его веса~$w(e_i)\hm=c_i\geqslant 0$, $e_i\hm\in E$. Задана некоторая 
функция~$f:2^{V}\to \{0, 1\}$. Требуется решить задачу линейного 
программирования с~целочисленными ограничениями:

\vspace*{-4pt}

\noindent
\begin{multline}
\underset{x_e:~e\in E}{\mbox{minimize}} \displaystyle \sum\limits_{e\in E}c_ex_e\\
\mbox{s.t.}\  x\left(\delta(S)\right)\geqslant f(S),\enskip  S \subset V, \enskip S \not= \emptyset,\\
 x_e\in\{0,1\},\enskip  e\in E.
\label{ilp_cfp}
\end{multline}

\vspace*{-3pt}

\noindent
Здесь
$$
x(\delta(S))=\sum\limits_{e\in \delta(S)}x_e,
$$
где $x_e\hm=1$, если 
ребро~$e$ входит в~финальное решение. Функция~$\delta(S)$ обозначает все ребра 
из~$E$ такие, что только одна из смежных вершин входит в~$S$.

Предположим, что отображение~$f$ удовлетворяет условиям

\vspace*{-3pt}

\noindent
\begin{gather*}
f(V) = 0,\\
 \underbrace{f(S)=f(V\backslash S)}_{\mathrm{симметричность}},\\
\underbrace{A,B\!\subset\! V\!: A\!\cap\! B\! =\! \emptyset, f(A)\!=\!f(B)\!=\!0\!\to\! f(A\!\cup\! B)\! =\! 0}_{\mathrm{дизъюнктивность}}.
\end{gather*}

\vspace*{-2pt}

\noindent
При выполнении этих условий~$f$ задает число ребер, начинающихся в~множестве 
вершин~$S$.

\smallskip

\noindent
\textbf{Лемма 1.}
\textit{Пусть $B\subseteq S\subset V$. Тогда $f(S) \hm= 0$ и~$f(B) \hm= 0$ приводит к}~$f(S\backslash B) \hm= 0$.

\smallskip

Задача с~таким описанием относится к~\textit{задачам поиска оптимального леса с~ограничениями}. 
Такая постановка задачи~(\ref{ilp_cfp}) с~соответствующим 
отображением~$f$ подходит для многих известных задач взвешенных графов, 
например: минимальный магистральный поиск, $st$-путь, задача Штейнера на 
минимальном дереве. Последняя задача является NP-полной, поэтому применим 
приближенный алгоритм.

\smallskip

\noindent
\textbf{Определение 3} (\textbf{алгоритм $\bm{\alpha}$-аппроксимации}).
Эвристический полиномиальный алгоритм, дающий\linebreak решение некоторой задачи 
оптимизации, называется $\alpha$-ап\-прок\-си\-ма\-ци\-ей, если он гарантирует 
удовлетворяющее ограничениям решение этой задачи оптимизации с~коэффициентом, 
меньшим или равным~$\alpha$, так что решение отличается от оптимального не более 
чем в~$\alpha$ раз по оптимизируемому функционалу.


\smallskip

Чтобы предложить приближенный алгоритм, целочисленные ограничения 
в~(\ref{ilp_cfp}) должны быть ослаблены:

\vspace*{-3pt}

\noindent
\begin{multline*}
\underset{x_e:~e\in E}{\mbox{minimize}}\  \displaystyle \sum\limits_{e\in E}c_ex_e \\
\mbox{s.t.}\  \displaystyle \sum\limits_{e\in \delta(S)}x_e\geqslant f(S),\enskip S \subset V\,, \enskip S \not= \emptyset\,,\\
 x_e>0,\enskip  e\in E,
%\label{rlp_cfp}
\end{multline*}
Двойственная задача принимает вид:

\vspace*{-4pt}

\noindent
\begin{multline}
\underset{y_S:~S \subset V, \; S \not= \emptyset}{\mbox{maximize}}\  
\displaystyle \sum\limits_{S\subset V}f(S)y_S \\
\mbox{s.t.}\  \displaystyle \sum\limits_{S:~e\in \delta(S)}y_S\leqslant c_e,\enskip  e\in E\,,\\
 y_S>0,\enskip  S \subset V, \enskip S \not= \emptyset\,,
\label{rd_cfp}
\end{multline}

\vspace*{-3pt}

\noindent
относительно дополнительного условия
$$
y_S \left(\sum\limits_{e\in \delta(S)}x_e - f(S)\right) = 0\,,\enskip S\subset  V\,.
$$

Обозначим множество вершин $A=\{v\hm\in V: f(\{v\})\hm=1\}$. Предлагается адаптивный 
жадный алгоритм $\left(2-{2}/{\vert A\vert }\right)$-ап\-прок\-си\-ма\-ции для задач 
вида~(\ref{ilp_cfp}). Алгоритм состоит из двух этапов. На первом этапе он жадно 
объединяет кластеры вершин, увеличивая двойственные переменные~$y_S$. Изначально 
каждая вершина принадлежит своему клас\-те\-ру. Если сле\-ду\-ющее реб\-ро~$e$ достигает 
равенства в~ограничениях в~(\ref{rd_cfp}), это ребро добавляется к~множеству~$S$ и~связанные клас\-те\-ры объединяются. Этот этап аналогичен алгоритму минимального 
остовного дерева Крускала. На втором этапе из конечного множества~$S$ удаляются 
некоторые ребра. Если обрезка ребра не нарушает ограничений, то это реб\-ро должно 
быть удалено.


Индекс $Z_{\mathrm{DRLP}}$ в~алгоритме~1 обозначает линейное 
программирование с~двойной релаксацией. Начальное значение $F:=\emptyset$ 
в~алгоритме~1 эквивалентно предположению $x_e \hm= 0$, $ e \hm\in E$. 
По условиям нежесткости $y_S \hm= 0$, $S \hm\subset V$,  $S \hm\not= \emptyset$.

На каждом шаге алгоритма кластер $\mathcal{C}$ содержит две компоненты 
$\mathcal{C} \hm= \mathcal{C}_i \hm\cup \mathcal{C}_a$, где $C\hm\in\mathcal{C }_a$, если 
$f(C) \hm= 1$, и~$C\hm\in\mathcal{C}_i$ в~противном случае. Назовем~$\mathcal{C}_a$ 
активным компонентом.
Переменные~$d(v)$ в~этом алгоритме связаны с~переменными~$y_S$ из~(\ref{rd_cfp}) 
соотношением
$$
d(i) = \sum\limits_{S:i\in S}y_S.
$$ 

Рассмотрим две различные компоненты $C_q$ и~$C_p$, $C_q\cap C_p\hm=\emptyset$, на 
некоторой итерации первого этапа алгоритма. Все~$y_S$ должны быть равномерно 
распределены по некоторому~$\varepsilon$ без нарушения ограничений
$$
\sum\limits_ {S:~e\in \delta(S)}y_S\leqslant c_e. 
$$
В терминах $d(v)$ это условие принимает вид:
$$
\sum\limits_{S:~e\in \delta(S)}y_S = d\left(v_1\right)+d\left(v_2\right),\enskip e=\left( v_1,v_2\right),
$$
поэтому $y_S\hm=0$ для любого~$S$ такого, что $v_1, v_2\hm\in S$, потому что 
компоненты растут только на первом этапе. Увеличение некоторых компонент на~$\varepsilon$ приводит к~уравнению
$$
d(v_1)+d(v_2)+\varepsilon \left(f(C_q)+f(C_p)\right)\leqslant 
c_e,\ e=\left(v_1,v_2\right), 
$$
что приводит к~формуле, используемой в~строке~$10$ алгоритма~1. 
В~случае когда в~состав входит следующее ребро, сумма $\sum\nolimits_{S:~e\in 
\delta (S)}y_S$ не будет увеличиваться, поэтому ограничения выполняются.

Ребра, которые можно удалить из~$F$ без добавления новых активных компонентов, 
удаляются на втором этапе алгоритма. Следующая лемма определяет свойства 
компонент связ\-ности в~$F'$.


\smallskip

\noindent
\textbf{Лемма~2.}\
\textit{Для каждой компоненты связ\-ности~$N$ из~$F'$ выполняется равенство}: $f(N)\hm=0$.

\smallskip

Следующая теорема утверж\-да\-ет, что решение, полученное с~помощью описанного 
алгоритма, удовле\-тво\-ря\-ет ограничениям исходной задачи линейного 
программирования.

\smallskip

\noindent
\textbf{Теорема~1.}
\textit{Набор ребер $F'$, полученный алгоритмом~$1$, удовлетворяет всем 
ограничениям исходной задачи}~(\ref{ilp_cfp}).


\smallskip

\noindent
\textbf{Лемма~3.}\
\textit{Обозначим граф $H$, каждая вершина которого соответствует одной из компонент 
связ\-ности $C\in\mathcal{C}$ на фиксированном шаге алгоритма. Ребро $(v_1,v_2)$ 
присутствует, если существует ребро $\hat{e}$ исходного графа, входящее в~$F'$: 
$\hat{e} \in F'$, поэтому граф $H$~--- это лес. Внут\-ри $H$ нет листовых вершин, 
со\-от\-вет\-ст\-ву\-ющих неактивным вершинам исходного графа}.

\smallskip

\noindent
\textbf{Теорема 2.}
\textit{Алгоритм~$1$ представляет собой $\alpha$-при\-бли\-жен\-ный алгоритм для 
задачи}~(\ref{ilp_cfp}) \textit{с}~$\alpha \hm= 2 - {2}/{|A|}$, \textit{где} $A\hm=\{v\  V: 
f(\{v\})=1\}$.

\smallskip

Несмотря на эту теоретическую основу, не существует подходящей функции $f$ для 
постановки задачи PCST, указанной в~(\ref{ilp_cfp}). Чтобы быть 
применимым в~этих условиях, алгоритм~1 нуждается в~нескольких 
модификациях.

\vspace*{-9pt}

\section{Модифицированная постановка задачи для~PCST}

\vspace*{-3pt}

Как и~в случае A-PCST, упрощенный вид задачи линейного 
программирования PCST принимает вид:
\begin{multline*}
\underset{\substack{x_e,s_v \\ e\in E, v\in V\backslash \{r\}}}{\mbox{minimize}}\  
\displaystyle \sum\limits_{e\in E}c_ex_e + \sum\limits_{v\in V\backslash\{r\}} \left(1-s_v\right)\pi_v \\
\mbox{s.t.}\  \displaystyle \sum\limits_{e\in\delta(S)} \!\! x_e\geqslant s_v,\enskip S\subseteq V\backslash \{r\},\enskip v\in S,\\
x_e\geqslant 0,\enskip e\in E,\enskip s_v\geqslant 0,\enskip v\in V\backslash \{r\}.
%\label{rlp_pcst_inord}
\end{multline*}
Эта постановка задачи отличается от исходной~(\ref{ilp_pcst_ord}) тем, что с~ней 
возможно согласовать задачу $k$-MST. Индикаторы~$s_v$ показывают, что 
вершина~$v$ включена в~дерево.

Двойственная задача принимает вид:

\vspace*{-3pt}

\noindent
\begin{multline*}
\underset{\substack{y_S:~S\subset V\backslash\{r\}}}{\mbox{maximize}}\ 
\displaystyle \sum\limits_{S\in V\backslash\{r\}}y_S \\
\mbox{s.t.}\  \displaystyle \sum\limits_{S:e\in\delta(S)}y_S\leqslant c_e ,\enskip e\in E;\\
 \displaystyle \sum\limits_{S\subseteq T}y_S\leqslant \sum\limits_{v\in T}\pi_v,\enskip  T\subset  V\backslash\{r\},\\
 y_S\geqslant 0,\enskip  S\subset V\backslash\{r\}.
%\label{rd_pcst_inord}
\end{multline*}

\vspace*{-3pt}

Алгоритм~2 решает эту задачу. Он похож на 
алгоритм~1. Двойные переменные должны обновляться равномерно 
с~дополнительными ограничениями. Тогда~$\varepsilon$ примет минимальное из двух 
значений в~соответствии с~обеими группами ограничений.
Более широкий анализ аппроксимационных свойств обновленного алгоритма 
представлен в~\cite{goemans1995general}. Алгоритм~2 представляет 
собой $\alpha$-приближенный алгоритм для задачи PCST с~$\alpha \hm= 2 \hm- 
{2}/({n-1})$, где $n$~--- число вершин в~графе~$G$.

\vspace*{-9pt}

\section{Вычислительный эксперимент}

\vspace*{-3pt}

Основная цель эксперимента~--- восстановить дерево суперпозиции. Алгоритмы, 
используемые для восстановления, перечислены ниже.

\vspace*{-14pt}

\paragraph*{DFS, BFS.}
Алгоритмы жадного дерева обхода в~глубину и~жадного дерева обхода в~ширину. 
Обход ребер с~наибольшим весом эквивалентен выбору наиболее вероятного пути. 
Алгоритм обхода останавливается, когда число ребер, исходящих из некоторой 
вершины, становится равным арности соответствующей функции.

\vspace*{-14pt}

\paragraph*{Алгоритм Прима.}
Алгоритм восстанавливает минимальное остовное дерево для графа с~дополнительными 
ограничениями на арность базовых функций. Эти ограничения задают минимальный вес 
ребра. После добавления вершины все лис\-то\-вые ребра этой вершины исключаются, 
чтобы сохранить направление дерева. Если число ребер, начинающихся в~какой-либо 
вершине, превышает соответствующую арность, то остальные ребра исключаются из 
множества возможных ребер в~этой вершине. Алгоритм не зависит от процедуры 
обхода. В случае небольшого шума в~матрице смежности этот алгоритм способен 
восстановить дерево суперпозиции без ошибок. 


\vspace*{-14pt}

\paragraph*{Алгоритмы на основе PCST.}
Матрица смеж\-ности~$M$ должна быть приведена к~неориентиро-\linebreak\vspace*{-12pt}

\pagebreak

\noindent
ванному виду. 
Использована квад\-рат\-ная мат\-ри\-ца~$M'$ без последнего столбца. PCST 
принимает мат\-ри\-цу смеж\-ности $1 \hm- ({1}/{2})(M' \hm+ M'^{\mathsf{T}})$ с~призовым 
значением~0,5 для каж\-дой вершины.
Призовое значение рав\-но~0,5, поскольку при меньших значениях дерево будет 
обрезано: если шум равен~0,5, некоторые вершины могут быть обрезаны по ошибке. 
В~случае больших призовых значений
дерево PCST может содержать ненужные 
вершины. Дерево восстанавливается по одному из опи-\linebreak\vspace*{-12pt}

{ \begin{center}  %fig2
 \vspace*{9pt}
    \mbox{%
\epsfxsize=79mm
\epsfbox{str-2.eps}
}
\end{center}



\noindent
{{\figurename~2}\ \ \small{Качество алгоритмов восстановления с~базовыми функциями небольших 
арностей: \textit{1}~--- DFS; \textit{2}~--- BFS; \textit{3}~--- алгоритм Прима;
\textit{4}~--- $k$-MST; \textit{5}~--- $k$-MST--DFS; \textit{6}~--- $h$-MST--BFS; \textit{7}~--- $k$-MST\,--\,ал\-го\-ритм Прима
}}}

\vspace*{6pt}

\addtocounter{figure}{1}

%\begin{table*}\small  %tabl2
\begin{center}
\parbox{75mm}{{{\tablename~2}\ \ \small{Качество алгоритмов реконструкции с~равномерным шумом, близким 
к~0,5
}}
}
    
    
\vspace*{6pt}

  {\small  \begin{tabular}{|l|ccccc|}
      \hline
                  & \multicolumn{5}{c|}{Шум}\\%& & Шум & & \\
       \cline{2-6}
        \multicolumn{1}{|c|}{\raisebox{6pt}[0pt][0pt]{Алгоритм}}                          
&0,50&0,52&0,54&0,56&0,58\\
                    \hline
      DFS        &0,20 &0,20 &0,19 &0,18 &0,16\\
      BFS        &0,60 &0,58 &0,51 &0,46 &0,40\\
      Прима    &1,00 &0,94&0,81&0,69&0,57\\
      $k$-MST     &0,17 &0,16 &0,14 &0,12 &0,10\\
      $k$-MST--DFS   &0,17 &0,16 &0,16 &0,14 &0,14 \\
      $k$-MST--BFS   &0,43 &0,40 &0,36 &0,33 &0,29 \\
      $k$-MST--Прима  &0,44 &0,39 &0,34 &0,33 &0,27 \\
      \hline
    \end{tabular}
    }
\end{center}
%\end{table*}




\noindent
 санных алгоритмов. Результаты 
$\text{PCST}$ можно использовать в~качестве априорных для других подходов, $M':=({1}/{2})(M_{\mathrm{PCST}}' + M')$,
поэтому результаты \mbox{PCST} обновляются~$M'$.


Процедура генерации данных имеет следующие допущения: арности функций 
генерируются биномиальным распределением, поэтому существуют много функций 
с~малой арностью, все базовые функции имеют только один вход. Любой случай 
с~частичной реконструкцией считается ошибкой. Качество алгоритмов реконструкции:
$$
\fr{1}{K}\sum\limits_{k=1}^K \left[ R\left( \bar{N}(M_k)\right)=M_k\right],
$$
где~$R$ ~--- алгоритм реконструкции;
$\bar{N}\hm=\left(N - \min(N)\right)/\left(\max(N)\hm-\min(N)\right)$~--- нормированная мат\-ри\-ца шума. 
Мат\-ри\-ца~$N$ генерируется как~$N(M)\hm=M\hm+U(-\alpha,\alpha)$.
Генератор случайных чисел возвращает матрицу того же вида, что и~$M$, где каждый 
элемент является независимой переменной из равномерного распределения 
в~сегменте~$[-\alpha,\alpha]$.

Вот список из семи сравниваемых алгоритмов:
DFS,
BFS,
алгоритм Прима,
$k$-MST через PCST,
$k$-MST\;+\;DFS,
$k$-MST\;+\;BFS,
$k$-MST\;+\;ал\-го\-ритм Прима.
На рис.~2 показана ошибка алгоритмов реконструкции 
с~шумом, близ\-ким к~порогу~0,5. Наилучшие результаты дает алгоритм Прима. Второе по 
точности решение основано на~$\text{BFS}$. Таб\-ли\-ца~2 
соответствует~рис.~2 и~показывает качество реконструкции 
семи алгоритмов для значений граничного шума~0,50--0,58.





\vspace*{-9pt}

\section{Заключение}

\vspace*{-3pt}

Предлагаются и~сравниваются  алгоритмы вос\-ста\-нов\-ле\-ния суперпозиции для задачи 
символьной регрессии. Алгоритм Прима дает наиболее точ\-ные результаты и~устойчив 
к~небольшому шуму в~данных. Пред\-ла\-га\-емый алгоритм дает точные результаты, но он 
более подвержен шуму в~мат\-ри\-це суперпозиции. Алгоритмы, основанные на BFS и~DFS, 
не могут вос\-ста\-но\-вить исходную суперпозицию с~зашумленными мат\-ри\-ца\-ми 
суперпозиции. Алгоритм PCST с~BFS, используемый для реконструкции мат\-ри\-цы 
суперпозиции, показывает приемлемые для практического использования результаты.

{\small\frenchspacing
 {%\baselineskip=10.8pt
 %\addcontentsline{toc}{section}{References}
 \begin{thebibliography}{99}
\bibitem{koza1992genetic}  %1
\Au{Koza J.\,R.} Genetic programming as a means for programming computers by 
natural selection~// Stat. Comput., 1994. Vol.~4. P.~87--112.

\bibitem{searson2010gptips} %2
\Au{Searson~D.\,P., Leahy~D.\,E., Willis~M.\,J.} GPTIPS: An open source 
genetic programming toolbox for multigene  symbolic regression~// 
Multiconference (International) of Engineers and Computer Scientists Proceedings, 
2010. Vol.~1. P.~77--80.

\bibitem{stanley2002evolving} %3
\Au{Stanley~K.\,O., Miikkulainen~R.} Evolving neural networks through 
augmenting topologies~// Evol. Comput., 2002. Vol.~10. 
Iss.~2. P.~99--127.

\bibitem{bochkarev2017generation}
\Au{Бочкарев~А.\,М., Софронов~И.\,Л., Стрижов~В.\,В.} По\-рож\-де\-ние экс\-перт\-но-ин\-тер\-пре\-ти\-ру\-емых 
моделей для прогноза проницаемости горной породы~// Системы и~средства информатики, 2017. Т.~27. №\,3. С.~74--87.
%

\bibitem{ravi1996spanning}
\Au{Ravi~R., Sundaram~R., Marathe~M.\,V., Rosenkrantz~D.\,J., Ravi~S.\,S.} 
Spanning trees~--- short or small~// SIAM J.~Discrete Math., 
1996. Vol.~9. Iss.~2. P.~178--200.

\bibitem{chudak2004approximate}
\Au{Chudak~F.\,A.,  Roughgarden~T., Williamson~D.\,P.} Approximate $k$-MSTS 
and $k$-Steiner trees via the primal-dual method and Lagrangean 
relaxation~// Math. Program., 2004. Vol.~100. Iss.~2. P.~411--421.

\bibitem{awerbuch1998new}
\Au{Awerbuch~B., Azar~Y., Blum~A., Vempala~S.} New approximation guarantees 
for minimum-weight $k$-trees and prize-collecting salesmen~// SIAM J. 
Comput., 1998. Vol.~28. Iss.~1. P.~254--262.

\bibitem{arora20062+}
\Au{Aror~S., Karakostas~G.} A~$2+\varepsilon$ approximation algorithm for the 
$k$-MST problem~// Math. Program., 2006. Vol.~107. 
Iss.~3. P.~491--504.

\bibitem{hegde2014fast}
\Au{Hegde~C., Indyk~P., Schmidt~L.} A~fast, adaptive variant of the 
Goemans--Williamson scheme for the prize-collecting steiner tree problem~// 11th DIMACS Implementation Challenge Workshop Proceedings, 2014. P.~1--32.
{\sf http://people. csail.mit.edu/ludwigs/papers/dimacs14\_fastpcst.pdf}.

\bibitem{ras2017approximate}
\Au{Ras~C., Swanepoel~K., Thomas~D.\,A.} Approximate Euclidean Steiner 
trees~// J.~Optimiz. Theory App., 2017. Vol.~172. 
Iss.~3. P.~845--873.

\bibitem{goemans1995general}
\Au{Goemans~M.\,X., Williamson~D.\,P.} A~general approximation technique for 
constrained forest problems~// SIAM J. Comput., 1995. Vol.~24. 
Iss.~2. P.~296--317.
\end{thebibliography}

 }
 }

\end{multicols}

\vspace*{-6pt}

\hfill{\small\textit{Поступила в~редакцию 23.01.22}}

\vspace*{8pt}

%\pagebreak

%\newpage

%\vspace*{-28pt}

\hrule

\vspace*{2pt}

\hrule

%\vspace*{-2pt}

\def\tit{OPTIMAL SPANNING TREE RECONSTRUCTION IN~SYMBOLIC~REGRESSION}


\def\titkol{Optimal spanning tree reconstruction in~symbolic regression}


\def\aut{R.\,G.~Neychev$^1$, I.\,A.~Shibaev$^1$, and~V.\,V.~Strijov$^2$}

\def\autkol{R.\,G.~Neychev, I.\,A.~Shibaev, and~V.\,V.~Strijov}

\titel{\tit}{\aut}{\autkol}{\titkol}

\vspace*{-8pt}


\noindent
$^1$Moscow Institute of Physics and Technology, 9~Institutskiy Per., Dolgoprudny, Moscow Region 141700, Russian\linebreak
$\hphantom{^1}$Federation

\noindent
$^2$Federal Research Center ``Computer Science and Control'' of the Russian Academy of Sciences, 44-2~Vavilov Str.,\linebreak
$\hphantom{^1}$Moscow 119333, Russian Federation

\def\leftfootline{\small{\textbf{\thepage}
\hfill INFORMATIKA I EE PRIMENENIYA~--- INFORMATICS AND
APPLICATIONS\ \ \ 2023\ \ \ volume~17\ \ \ issue\ 1}
}%
 \def\rightfootline{\small{INFORMATIKA I EE PRIMENENIYA~---
INFORMATICS AND APPLICATIONS\ \ \ 2023\ \ \ volume~17\ \ \ issue\ 1
\hfill \textbf{\thepage}}}

\vspace*{3pt} 



\Abste{The paper investigates the problem of regression model generation. A~model is a~superposition of primitive functions. 
The model structure is described by a~weighted colored graph. Each graph vertex corresponds to a~primitive function. 
An edge assigns a~superposition of two functions. The weight of an edge is equal to the probability of superposition. 
To generate an optimal model, one has to reconstruct its structure from its graph adjacency matrix. 
The proposed algorithm reconstructs the minimum spanning tree from the weighted colored graph. 
The paper presents a~novel solution based on the prize-collecting Steiner tree algorithm. This algorithm is compared with its alternatives.}


\KWE{symbolic regression; linear programming; superposition; minimum spanning tree; adjacency matrix}



\DOI{10.14357/19922264230105} 

\vspace*{-16pt}

\Ack

\vspace*{-3pt}


\noindent
This work was supported by the Russian Foundation for Basic Research, projects 20-37-90050 and 20-07-00990.
  

\vspace*{6pt}

  \begin{multicols}{2}

\renewcommand{\bibname}{\protect\rmfamily References}
%\renewcommand{\bibname}{\large\protect\rm References}

{\small\frenchspacing
 {%\baselineskip=10.8pt
 \addcontentsline{toc}{section}{References}
 \begin{thebibliography}{99} 

\bibitem{1-str}
\Aue{Koza, J.\,R.}
 1994. Genetic programming as a means for programming computers by natural selection. \textit{Stat. Comput.} 4:87--112.

\bibitem{2-str}
\Aue{Searson, D.\,P., D.\,E.~Leahy, and M.\,J.~Willis.}
 2010. \mbox{GPTIPS}: An open source genetic programming toolbox for multigene symbolic regression. 
 \textit{Multiconference (International) of Engineers and Computer Scientists Proceedings}. 1:77--80. 

\bibitem{3-str}
\Aue{Stanley, K.\,O., and R.~Miikkulainen.} 2002. Evolving neural networks through augmenting topologies. 
\textit{Evol. Comput.} 10(2):99--127.

\bibitem{4-str}
\Aue{Bochkarev, A.\,M., I.\,L.~Sofronov, and V.\,V.~Strijov.}
 2017. Po\-rozh\-de\-nie eks\-pert\-no-inter\-pre\-ti\-ru\-emykh mo\-de\-ley dlya prog\-no\-za pro\-ni\-tsa\-emosti gor\-noy po\-ro\-dy 
 [Generation of expertly-interpreted models for prediction of core permeability]. \textit{Sistemy i~Sredstva Informatiki~--- Systems and Means of Informatics}
  27(3):74--87.

\bibitem{5-str}
\Aue{Ravi, R., R.~Sundaram, M.\,V.~Marathe, D.\,J.~Rosenkrantz, and S.\,S.~Ravi.}
 1996. Spanning trees~--- short or small. \textit{SIAM J. Discrete Math.} 9(2):178--200.

\bibitem{6-str}
\Aue{Chudak, F.\,A., T.~Roughgarden, and D.\,P.~Williamson.}
 2004. Approximate k-MSTS and k-Steiner trees via the primal-dual method and Lagrangean relaxation. 
 \textit{Math. Program.} 100(2):411--421.

\bibitem{7-str}
\Aue{Awerbuch, B., Y.~Azar, A.~Blum, and S.~Vempala.}
 1998. New approximation guarantees for minimum-weight \mbox{k-trees} and prize-collecting salesmen.
 \textit{SIAM J. Comput.} 28(1):254--262.

\bibitem{8-str}
\Aue{Arora, S., and G.~Karakostas.} 2006. A~$2+\varepsilon$ approximation algorithm for the $k$-MST problem. 
\textit{Math. Program.} 107(3):491--504.

\bibitem{9-str}
\Aue{Hegde, C., P.~Indyk, and L.~Schmidt.} 2014. 
A~fast, adaptive variant of the Goemans--Williamson scheme for the prize-collecting Steiner tree problem. 
\textit{11th DIMACS Implementation Challenge Workshop Proceedings}. 1--32.
Available at: 
{\sf http://people.csail.mit.edu/ludwigs/papers/\linebreak dimacs14\_fastpcst.pdf} (accessed January~10, 2023).

\bibitem{10-str}
\Aue{Ras, C., K.~Swanepoel, and D.\,A.~Thomas.} 
2017. Approximate Euclidean Steiner trees. \textit{J.~Optimiz. Theory  App.} 172(3):845--873.

\bibitem{11-str}
\Aue{Goemans, M.\,X., and D.\,P.~Williamson.} 1995. 
A~general approximation technique for constrained forest problems. \textit{SIAM J. Comput.} 24(2):296--317.
 \end{thebibliography}

 }
 }

\end{multicols}

\vspace*{-6pt}

\hfill{\small\textit{Received January 23, 2022}}

\Contr

\noindent
\textbf{Neychev Radoslav G.} (b.\ 1994)~--- 
PhD student, Moscow Institute of Physics and Technology, 9~Institutskiy Per., Dolgoprudny, Moscow Region 141701, Russian Federation;
\mbox{neychev@phystech.edu}

\vspace*{3pt}

\noindent
\textbf{Shibaev Innokentii A.} (b.\ 1997)~--- 
PhD student, Moscow Institute of Physics and Technology, 9~Institutskiy Per., Dolgoprudny, Moscow Region 141701, Russian Federation; 
\mbox{shibaev.kesha@gmail.com}

\vspace*{3pt}

\noindent
\textbf{Strijov Vadim V.} (b.\ 1967)~--- 
Doctor of Science in physics and mathematics, leading scientist, A.\,A.~Dorodnicyn Computing Center, 
Federal Research Center ``Computer Science and Control'' of the Russian Academy of Sciences, 40~Vavilov Str., Moscow 119333, Russian Federation;
\mbox{strijov@phystech.edu}


\label{end\stat}

\renewcommand{\bibname}{\protect\rm Литература}   %9
\newcommand{\Tsf}{^{\mathsf T}}
\newcommand{\rank}{\mathrm{rank}\,}

\def\stat{logachev}

\def\tit{ПОЛИНОМИАЛЬНЫЕ АЛГОРИТМЫ ВЫЧИСЛЕНИЯ ЛОКАЛЬНЫХ АФФИННОСТЕЙ КВАДРАТИЧНЫХ 
БУЛЕВЫХ ФУНКЦИЙ$^*$}

\def\titkol{Полиномиальные алгоритмы вычисления локальных аффинностей квадратичных 
булевых функций}

\def\aut{О.\,А.~Логачев$^1$, А.\,А.~Сукаев$^2$, С.\,Н.~Федоров$^3$}

\def\autkol{О.\,А.~Логачев, А.\,А.~Сукаев, С.\,Н.~Федоров}

\titel{\tit}{\aut}{\autkol}{\titkol}

\index{Логачев О.\,А.}
\index{Сукаев А.\,А.}
\index{Федоров С.\,Н.}
\index{Logachev O.\,A.}
\index{Sukayev A.\,A.}
\index{Fedorov S.\,N.}


{\renewcommand{\thefootnote}{\fnsymbol{footnote}} \footnotetext[1]
{Работа выполнена при частичной поддержке РФФИ (проект 18-29-03124~мк).}}


\renewcommand{\thefootnote}{\arabic{footnote}}
\footnotetext[1]{Московский государственный университет им.\
М.\,В.~Ломоносова; Институт проб\-лем информатики Федерального исследовательского
центра <<Информатика и~управ\-ле\-ние>> Российской академии наук, \mbox{logol@iisi.msu.ru}}
\footnotetext[2]{Московский государственный университет им.\ 
М.\,В.~Ломоносова, \mbox{asukaev@gmail.com}}
\footnotetext[3]{Московский государственный университет им.\
М.\,В.~Ломоносова, \mbox{s.n.feodorov@yandex.ru}}

\vspace*{-12pt}

 
\Abst{Аффинная нормальная форма позволяет рассматривать произвольную булеву функцию на 
определенных плоскостях (так называемых локальных аффинностях) как аффинную. Данное 
пред\-став\-ле\-ние~--- по сути, аффинная аппроксимация~--- булевых функций может 
помочь в~решении систем нелинейных уравнений над полем из двух элементов. Задача 
решения таких систем (специального вида), среди прочего, используется в~ряде 
методов синтеза и~анализа средств обеспечения информационной безопасности.
В~статье описывается способ нахождения локальных аффинностей для квадратичных 
булевых функций, основанный на теореме Диксона. Тем самым решается задача 
построения аффинных нормальных форм для таких функций. Кроме того, обсуждаются 
вопросы эффективности подобных алгоритмов.
Основная цель данной статьи~--- подготовить базу для готовящейся к~публикации 
работы, предлагающей метод решения систем квадратичных булевых уравнений 
с~помощью <<аппроксимирования>> соответствующих функций их аффинными нормальными 
формами.}


\KW{булева функция; система квадратичных булевых уравнений; 
разбиение векторного пространства; плоскость; локальная аффинность; теорема 
Диксона; аффинная нормальная форма; алгебраический криптоанализ}

\DOI{10.14357/19922264190110}
  
\vspace*{-1pt}


\vskip 10pt plus 9pt minus 6pt

\thispagestyle{headings}

\begin{multicols}{2}

\label{st\stat}


\section{Введение}

\vspace*{-2pt}

Центральная идея алгебраического криптоанализа состоит в~том, чтобы описать 
используемые в~анализируемой криптосхеме преобразования сис\-те\-мой алгебраических 
уравнений (с~некоторой сек\-рет\-ной информацией в~качестве неизвестных) над\linebreak 
конечным полем и~затем решить эту систему.
В~данной статье рассматриваются только булевы системы уравнений, хотя часть 
пред\-став\-лен\-ных здесь результа\-тов может иметь место и~для сис\-тем ал\-геб\-ра\-и\-че\-ских 
уравнений над произвольными конечными полями.

Из теории сложности вычислений известно, что вычислительная задача определения 
совместности систем нелинейных булевых уравнений является NP-пол\-ной~\cite{GJ1982, GT2017}, 
а~вычислительная задача решения систем нелинейных булевых 
уравнений является NP-труд\-ной~\cite{GJ1982,GT2017}.
Однако в~специальных случаях эти задачи могут решаться эффективно (см., 
например,~\cite{GT2017,Smi2000}).

Кроме того, существуют полиномиальные алгоритмы построения по произвольной 
системе уравнений системы с~фиксированной алгебраической 
степенью~\cite[\S\;11.4.2]{Bard2009}, что позволяет, в~частности, ограничиться 
рассмотрением только квадратичных систем уравнений.

Можно выделить несколько основных классов методов, используемых в~криптоанализе 
для решения (или оценки трудоемкости решения) систем полиномиальных булевых 
уравнений:
использование базисов Грёбнера~\cite[section~12.2]{Bard2009}, применение 
программных систем поиска выполняющего набора булевой формулы 
(SAT-solvers)~\cite{BCJ2007}, вероятностные и~тео\-ре\-ти\-ко-ко\-до\-вые 
методы~\cite{LSSYa2015}, а~также  методы линеаризации~\cite[section~12.3]{Bard2009}.
Основная идея методов линеаризации состоит в~применении <<линейных>> методов 
к~нелинейным системам, т.\,е.\ в~построении сис\-тем линейных уравнений, решение 
которых дает возможность найти решение исходной нелинейной системы.

Важным параметром метода линеаризации служит число переменных в~синтезируемых 
линейных системах уравнений. Как правило, речь идет об увеличении (не~более чем 
полиномиальном) количества переменных.
Метод, основанный на рассмотренных в~данной работе идеях, по своей сути, 
осуществляет линеаризацию, но при этом он остав\-ля\-ет число переменных неизменным.

Этот метод решения квадратичных систем булевых уравнений использует локальные 
аффинности уравнений системы и~состоит из двух этапов.
Первый этап (предварительный) содержательно представляет собой описание семейств 
локальных аффинностей уравнений.
Второй этап метода заклю\-ча\-ет\-ся собственно в~решении исходной сис\-те\-мы посредством 
анализа сис\-тем линейных уравнений, полученных с~помощью этих локальных 
аффинностей.

Настоящая работа (в~силу ограниченности\linebreak объема публикации) посвящена 
исследованию первого этапа и,~в~частности, вопросам его эффективности. 
Результаты исследований с~оценкой эф\-фек\-тив\-ности и~описанием параметров второго
\mbox{этапа} предлагаемого метода предполагается опуб\-ли\-ко\-вать в~одном из сле\-ду\-ющих 
выпусков журнала.

\vspace*{-4pt}

\section{Необходимые понятия и~обозначения}

В данной работе булев куб $\{0,1\}^n$ отождествляется с~$n$-мерным векторным 
пространством~$V_n$ над полем из двух элементов~$\mathbb{F}_2$.
Векторы из $V_n$ будет удобнее записывать \textit{строками} длины~$n$. Значок~$\Tsf$ 
используется для операции транспонирования матриц.
Всюду далее~$x$ обозначает вектор $(x_1,x_2,\ldots,x_n)$.

Знак $\oplus$ будет использоваться для записи суммы по модулю~$2$ булевых 
переменных и~операций сложения в~$\mathbb{F}_2$ и~покомпонентного сложения 
в~$V_n$.

Множество всех невырожденных аффинных преобразований (отображений в~себя) 
пространства~$V_n$ обозначается через $\mathrm{GA}(V_n)$. В~матричном 
представлении действие элемента~$\alpha\in\mathrm{GA}(V_n)$ на векторах 
пространства имеет вид $\alpha(x)\hm=xA\oplus b$, где $x$~пробегает~$V_n$; $A$~--- 
невырожденная $(n\times n)$-мат\-ри\-ца над~$\mathbb{F}_2$; $b\hm\in V_n$.

Множество всех булевых функций от $n$~переменных обозначим через
$$
\mathcal{F}_n=\{f\colon V_n\to \mathbb{F}_2\}\,.
$$
Как известно, произвольную булеву функцию~$f$ от переменных $x_1,\ldots,x_n$ 
можно представить (единственным образом) в~виде полинома Жегалкина:
$$
f(x)=\bigoplus_{\varepsilon\in\{0,1\}^n} a_{\varepsilon}x^{\varepsilon}\,,
$$
где %\label{Zhegalkin}
$\varepsilon\hm=(\varepsilon_1,\ldots,\varepsilon_n)$,
$a_{\varepsilon}\hm\in\mathbb{F}_2$ и~$x^{\varepsilon}\hm=x_1^{\varepsilon_1}\cdots x_n^{\varepsilon_n}$ (считаем, 
$x_i^0\hm=1$, $x_i^1\hm=x_i$).
Далее под булевой функцией будет, как правило, подразумеваться ее запись в~виде 
полинома.

Если $\varphi$~--- некоторое преобразование пространства~$V_n$, то его действие на 
функцию~$f\hm\in\mathcal{F}_n$ будем определять и~обозначать так: 
$f^{\varphi}(x)\hm=f(\varphi(x))$.
В~частности, в~случае аффинных преобразований пространства будет рассматриваться 
множество $\mathrm{Orb}_f(\mathrm{GA}(V_n))\hm=\{f^{\varphi}\mid 
\varphi\hm\in\mathrm{GA}(V_n)\}$~--- орбита функции~$f$ относительно действия 
группы~$\mathrm{GA}(V_n)$.
Имея в~виду, что произведение~$\alpha_1\alpha_2$ элементов из $\mathrm{GA}(V_n)$ 
есть композиция $\alpha_1\circ\alpha_2(x)\hm=\alpha_1(\alpha_2(x))$, заметим, что 
действие~$\alpha_1\alpha_2$ на произвольную функцию $f\hm\in\mathcal{F}_n$ 
корректно определять следующим образом:
$$
f^{\alpha_1\alpha_2}(x)=\left(f^{\alpha_1}\right)^{\alpha_2}(x)=f^{\alpha_1}
\left(\alpha_2(x)\right)
=f\left(\alpha_1\alpha_2(x)\right),
$$
поскольку~$\alpha_i$ действуют на булеву функцию преобразованием \textit{ее 
аргумента}.

%Когда мы делаем невырожденную аффинную замену переменных $x'=\alpha(x)=xA\oplus 
%b$, функция~$f(x)$, при подставлении в~нее выражений старых переменных через 
%новые, преобразуется к~виду $f^{\alpha^{-1}}(x')$.


\textit{Алгебраической степенью} булевой функции~$f$ от $n$~переменных называют 
величину

\noindent
$$
\deg f = \max\left\{\sum\limits_{i=1}^n \varepsilon_i\mid a_{\varepsilon}=1\right\}
$$
(суммирование~--- в~$\mathbb{Z}$), т.\,е.\ максимальное число различных 
переменных в~мономах данного представления.

В множестве~$\mathcal{F}_n$ всех булевых функций от~$n$~переменных выделим 
подмножество

\noindent
$$
\mathcal{A}_n=\left\{f\in\mathcal{F}_n\mid \deg f\leqslant 1\right\}.
$$
Составляющие это подмножество функции называются линейными (в~математической 
логике и~кибернетике) или аф\-фин\-но-ли\-ней\-ны\-ми (в~ал\-геб\-ре), однако по сложившейся 
в~криптологии традиции в~данной работе они называются \textit{аффинными}, т.\,е.\ 
понимаются как частный случай аффинного \textit{отоб\-ра\-же\-ния} $n$-мер\-но\-го 
пространства в~одномерное.


Булеву функцию~$f$ c $\deg f\hm\leqslant 2$ будем называть 
\textit{квадратичной}\footnote{В~алгебре такие функции называют 
аффинно-квад\-ра\-тич\-ны\-ми. Квадратичными при этом называют функции, представляемые 
\textit{однородными} полиномами второй степени.}.
По определению квадратичная функция~$f\hm\in\mathcal{F}_n$ (ее полином Жегалкина) 
имеет вид:

\noindent
$$
f(x)= \bigoplus_{1\leqslant i<j\leqslant n} q_{ij}x_ix_j\oplus
\bigoplus_{1\leqslant k\leqslant n} 
l_kx_k \oplus c\,,
$$
где $q_{ij},l_k,c\in\mathbb{F}_2$.

В настоящей работе рассматриваются системы уравнений

\noindent
\begin{equation}
\left.
\begin{array}{c}
        f_1(x_1,\ldots,x_n)=0\,;\\
        f_2(x_1,\ldots,x_n)=0\,;\\
        \vdots\\
        f_m(x_1,\ldots,x_n)=0\\
    \end{array}
    \right\}
    \label{system}
\end{equation}
с квадратичными булевыми функциями~$f_i$, $1\hm\leqslant i\hm\leqslant m$, и~$m\hm>n$.
%Мы предполагаем, что все рассматриваемые нами системы квадратичных уравнений 
%имеют единственное решение.

\pagebreak

В матричном виде квадратичная функция записывается следующим образом:
$$
f(x)=x Q_f x\Tsf\oplus l_f x\Tsf \oplus c\,,
$$
где $Q_f$~--- верхнетреугольная $(n\times n)$-мат\-ри\-ца с~нулевой главной 
диагональю; $l_f\in\mathbb{F}_2^n$; $c\in\mathbb{F}_2$.
Рас\-смат\-ри\-ва\-ют также симметричную матрицу
$$
\tilde{Q}_f=Q_f\oplus Q_f\Tsf\,.
$$
Она определяет билинейную форму
$$
q_f(u,v)=u\tilde{Q}_f v\Tsf=f(u\oplus v)\oplus f(u) \oplus f(v)\oplus c\,,
$$
называемую \textit{ассоциированной с~квадратичной функцией~$f$}.

Булева билинейная форма $q(u,v)$, $u,v\hm\in V_n$, удовле\-тво\-ря\-ющая условиям
$$
q(u,u)=0\,;\qquad q(u,v)=q(v,u)\,,
$$
называется \textit{симплектической}.
Такие билинейные формы находятся во взаимно однозначном соответствии с~булевыми 
симметричными матрицами с~нулевой главной диагональю, называемыми 
\textit{симплектическими матрицами}.

Таким образом, для произвольной квадратичной булевой функции~$f$ 
матрица~$\tilde{Q}_f$~--- сим\-плектическая. Также очевидно, что билинейная\linebreak 
форма~$q_f(u,v)$, ассоциированная с~$f$, является симплектической.

\smallskip

\noindent
\textbf{Предложение~1}\
[6, лемма~3.3.1; 7, \S\;15.2, лемма~3].
\textit{Ранг симплектической матрицы четен.}


\smallskip

\textit{Плоскость}~$\pi$ в~$V_n$~--- это множество вида $v\hm+L$, где~$v$ и~$L$~--- 
соответственно вектор и~подпространство пространства~$V_n$. Другими словами, 
плоскость~--- аффинное подпространство в~$V_n$.
\textit{Размерность плоскости} совпадает с~размерностью соответствующего 
подпространства: $\mathrm{dim}\,\pi=\mathrm{dim}\,L$.
Как известно, любая плоскость является решением некоторой системы линейных 
уравнений, и~на\-обо\-рот: решение произвольной системы линейных уравнений~--- 
плоскость в~соответствующем пространстве.
%То есть, плоскость является линейным многообразием.

Сужение функции $f\hm\in\mathcal{F}_n$ на плоскость~$\pi$ будем обозначать 
через~$f|_{\pi}$. Таким образом, $f|_{\pi}\colon \pi\hm\to\mathbb{F}_2$ 
и~$f|_{\pi}(u)\hm=f(u)$ для всех $u\hm\in\pi$.



\section{Локальная аффинность и~аффинная нормальная форма~булевой~функции}

В этом разделе вводятся понятия, связанные с~представлением произвольной булевой 
функции совокупностью аффинных функций, заданных для определенных плоскостей 
в~векторном пространстве. Более общее изложение этой теории можно найти 
в~работе~\cite{LYaD2007}.

\textit{Локальной аффинностью} функции~$f\hm\in\mathcal{F}_n$ будем называть такую 
плоскость~$\pi$, что $f|_{\pi}$ можно продолжить до аффинной функции, т.\,е.\ 
существует $l\hm\in\mathcal{A}_n$ со свойством $f|_{\pi}\hm=l|_{\pi}$.
Очевидно, для любой булевой функции существует разбиение пространства~$V_n$ на 
ее локальные аффинности.

Возьмем произвольное разбиение $\Pi\hm=\{\pi_1,\ldots,\pi_{\lambda}\}$ 
пространства~$V_n$ на плоскости, являющиеся локальными аффинностями булевой 
функции~$f$ от $n$~переменных.
Будем называть \textit{аффинной нормальной формой} функции~$f$ выражение вида
\begin{equation}
\label{AffNF}
f(x)=\bigoplus_{j=1}^{\lambda}\chi_{\pi_j}(x) l_j(x)\,,
\end{equation}
где для каждого~$j$, $1\hm\leqslant j\hm\leqslant\lambda$, функция~$l_j$ аффинна 
и~$f|_{\pi_j}(x)\hm=l_j|_{\pi_j}(x)$, а~$\chi_{\pi_j}$~--- характеристическая 
функция (индикатор) множества~$\pi_j$.
Функции~$l_j$ из этого выражения для краткости назовем\linebreak
 \textit{аффинными 
аппроксимациями} функции~$f$.
\textit{Длиной аффинной нормальной формы} называется число плоскостей 
в~разбиении~$\Pi$, далее она будет обозначаться через~$\lambda(\Pi)$.

\smallskip

\noindent
\textbf{Замечание~1.}
  Характеристические функции плоскостей известны также под именем 
<<мультиаффинных функций>> \cite{GT2017}, играющих важную роль при описании 
классов эффективно решаемых систем булевых уравнений.


\smallskip

Характеристическая функция плоскости в~пространстве~$V_n$ имеет вполне 
определенный вид. Любая плоскость~$\pi$, как уже отмечалось, может быть задана 
как множество решений системы $d$~линейных уравнений (для некоторого~$d$):
\begin{equation}
\left.
\begin{array}{c}
        h_1(x_1,\ldots,x_n)=0\,;\\
        h_2(x_1,\ldots,x_n)=0\,;\\
        \vdots\\
        h_d(x_1,\ldots,x_n)=0\,,\\
    \end{array}
    \right\}
    \label{chi-system}
\end{equation}
где все $h_i(x)\hm\in\mathcal{A}_n$. Поскольку вектор~$x$ принадлежит 
плоскости~$\pi$ тогда и~только тогда, когда все~$h_i$, $1\hm\leqslant i\hm\leqslant d$, 
обращаются в~нуль на нем, характеристическая функция~$\pi$ выражается следующим 
образом:
$$
\chi_{\pi}(x)=\prod\limits_{i=1}^d (h_i(x)\oplus 1)\,.
$$
Если система линейных уравнений задана в~мат\-рич\-ной форме: $xH\oplus 
(b_1,\ldots,b_d)\hm=0$, $b_i\hm=h_i(0)$, то выражение будет иметь вид:

\noindent
$$
\chi_{\pi}(x)=\prod\limits_{i=1}^d (xH_i\oplus b_i\oplus 1)\,,
$$
где $H_i$~--- столбцы матрицы~$H$.

Как видно из определения, аффинная нормальная форма представляет собой 
в~некотором смыс\-ле ку\-соч\-но-аф\-фин\-ную аппроксимацию булевой функции. На каждой 
локальной аф\-фин\-ности~$\pi_j$ из разбиения $\Pi$ все, кроме одного, слагаемые 
в~выражении~\eqref{AffNF} обращаются в~нуль, и~функция принимает вид 
$f(x)\hm=\chi_{\pi_j}(x)l_j(x)\hm=l_j(x)$ для всех $x\hm\in\pi_j$.

Возможность заменить на плоскости~$\pi_j$ квадратичное уравнение $f(x)\hm=0$ 
линейным уравнением $l_j(x)\hm=0$ вместе с~дописанной к~нему системой~\eqref{chi-system} 
будет использоваться при решении систем полиномиальных уравнений 
в~следующей статье.
Как сказано в~замечании~1, функции~$\chi_{\pi}(x)$, а~также 
и~слагаемые в~аффинной нормальной форме~\eqref{AffNF} являются мультиаффинными 
функциями. Тео\-ре\-ти\-ко-слож\-ност\-ные вопросы, связанные, в~частности, с~решением 
систем мультиаффинных уравнений, а~также оценки числа таких функций 
рассматриваются в~работе~\cite{Gor1995}.

При аффинном преобразовании пространства аффинные нормальные формы сохраняются 
в~том смысле, что выражение, полученное после применения преобразования к~этой 
форме, тоже будет аффинной нормальной формой для некоторой функции.

\smallskip

\noindent
\textbf{Предложение~2.}
\textit{Пусть $\varphi\in\mathrm{GA}(V_n)$ и~$f(x)\hm=
\bigoplus_{j=1}^{\lambda(\Pi)}\chi_{\pi_j}(x) l_j(x)$~--- некоторая 
аффинная нормальная форма функции~$f$.
  Тогда} 
$$
f^{\varphi}(x)=f(\varphi(x))=\bigoplus_{j=1}^{\lambda(\Pi)}\chi_{\pi_j}(\varphi(x)) 
l_j(\varphi(x))$$ 
\textit{есть аффинная нормальная форма функции~$f^{\varphi}$}.


\smallskip

\noindent
Д\,о\,к\,а\,з\,а\,т\,е\,л\,ь\,с\,т\,в\,о\,.\ \ 
Множество $\Pi'\hm=\{\pi'_j\hm=\varphi^{-1}(\pi_j) \mid \pi_j\in\Pi\}$ является 
разбиением пространства~$V_n$ на $\lambda(\Pi)$ плоскостей, поскольку~$\varphi$~--- 
не\-вы\-рож\-ден\-ное аффинное преобразование.
Заметим,\linebreak
 что $\varphi(x)\in\pi_j$ тогда и~только тогда, когда $x\hm\in\varphi^{-1}
(\pi_j)\hm=\pi'_j$.
Поэтому $\chi^{\varphi}_{\pi_j}$~--- характеристическая функция 
плоскости~$\pi'_j$.
Выражение для~$f^{\varphi}$ в~новых обозначениях выглядит следующим образом:
$$
f^{\varphi}(x)=\bigoplus_{j=1}^{\lambda(\Pi')}\chi_{\pi'_j}(x) l_j^{\varphi}(x)\,.
$$
Так как, очевидно, функции $l_j^{\varphi}(x)\hm=l_j(\varphi(x))$ аффинны, полученное 
выражение представляет собой аффинную нормальную форму.


\section{Теорема Диксона и~приведение квадратичных функций к~каноническому 
виду}\label{Dickson}

Благодаря теореме Диксона можно для любой квадратичной булевой функции~$f$ найти 
ее каноническое представление, в~котором она выглядит наиболее просто. Как будет 
видно ниже, это представление~--- элемент из орбиты данной функции 
$\mathrm{Orb}_f(\mathrm{GA}(V_n))$.
Канонический вид квадратичной функции, в~свою очередь, подсказывает прос\-той 
способ построения ее аффинной нормальной \mbox{формы.}

\smallskip

\noindent
\textbf{Теорема~1}\ [10, \S\;199].
\textit{Для любой квад\-ра\-тич\-ной функции~$f\hm\in\mathcal{F}_n$ с~ненулевой 
матрицей~$\tilde{Q}_f$ существует аффинное 
преобразование~$\alpha\hm\in\mathrm{GA}(V_n)$, которое приводит~$f$ к~одному из  
канонических представлений}:
$$f^{\alpha}(x)=x_1x_2\oplus x_3x_4\oplus\cdots\oplus x_{2r-1}x_{2r}\oplus c
$$
\textit{или}
$$f^{\alpha}(x)=x_1x_2\oplus x_3x_4\oplus\cdots\oplus x_{2r-1}x_{2r}\oplus 
x_{2r+1}\,,
$$
где $2r=\rank \tilde{Q}_f$ и~$c\hm\in\mathbb{F}_2$.

\smallskip

Доказательство этого утверждения помимо авторского варианта можно найти также 
в~[6, \S\;3.3; 7, \S\;15.2].

На практике приведение полинома Жегалкина квадратичной булевой функции 
к~каноническому виду можно осуществить следующим способом.

Предположим, не ограничивая общности, что в~полиноме Жегалкина функции~$f$ 
присутствует моном~$x_1x_2$ (иначе с~помощью аффинного преобразования координат 
<<перенумеруем>> переменные).
Представим функцию в~виде:
\begin{multline*}
f(x)=x_1x_2\oplus x_1l_1(x_3,\ldots,x_n) \oplus x_2l_2(x_3,\ldots,x_n) \oplus{}\\
{}\oplus 
q_1(x_3,\ldots,x_n)\,,
\end{multline*}
где $l_1,l_2\in\mathcal{A}_{n-2}$, а~$q_1$~--- некоторая квадратичная функция.
Возьмем отображение~$\varphi_2$ пространства~$V_n$, задаваемое равенством:
\begin{multline*}
\varphi_2(x)=\left(x_1\oplus l_2(x_3,\ldots,x_n),\ x_2\oplus {}\right.\\
\left.{}\oplus
l_1(x_3,\ldots,x_n),\ x_3,\ldots,\ x_n\right)\,,
\end{multline*}
и рассмотрим следующую функцию:
$$
f^{(2)}(x)=x_1x_2\oplus q_2(x_3,\ldots,x_n)\,,
$$
где $q_2=q_1\oplus l_1l_2$~--- квадратичная функция.
Заметим, что $(f^{(2)})^{\varphi_2}\hm=f$.

Затем аналогично предыдущему выделим первые две переменные в~функции 
$q_2(x_3,\ldots,x_n)$.
Здесь берется отображение
\begin{multline*}
\varphi_4(x)=\bigl(x_1,\ x_2,\  x_3\oplus l_4(x_5,\ldots,x_n),\\
x_4\oplus 
l_3(x_5,\ldots,x_n),\ x_5,\ldots,\ x_n\bigr)
\end{multline*}
и функция
$$
f^{(4)}(x)=x_1x_2\oplus  x_3x_4\oplus q_4(x_5,\ldots,x_n)\,,
$$
так что $(f^{(4)})^{\varphi_4}=f^{(2)}$.

Проделываем это до тех пор, пока на некотором шаге не получим аффинную функцию
$$
q_{2r}(x_{2r+1},\ldots,x_n)=\bigoplus_{i=2r+1}^{n}b_ix_i\oplus c
$$
для некоторых $b_i,c\hm\in\mathbb{F}_2$.
Если $b_i\hm=0$ для всех~$i$, $2r\hm+1\hm\leqslant i\hm\leqslant n$, 
то искомый канонический вид 
найден: это функция~$f^{(2r)}$.
Иначе считаем, без ограничения общности, что $b_{2r+1}\hm=1$ и~полагаем
\begin{multline*}
\varphi_{2r+1}(x)=\left(x_1,\ldots,\ x_{2r},\ 
q_{2r}\left(x_{2r+1},\ldots,x_n\right),\right.\\ 
\left.x_{2r+2},\ldots,\ x_n\right)\,.
\end{multline*}
Тогда канонический вид для~$f$~--- это функция
\begin{multline*}
g(x)={}\\
{}=f^{(2r+1)}(x)=x_1x_2\oplus  x_3x_4\oplus \cdots \oplus x_{2r-1}x_{2r} 
\oplus x_{2r+1},
\end{multline*}
причем если положить $\varphi\hm=\varphi_{2r+1}\varphi_{2r}
\varphi_{2r-2}\cdots\varphi_2$, то
$$
g^{\varphi}=\left(\cdots(g^{\varphi_{2r+1}})^{\varphi_{2r}}\cdots\right)^{\varphi_2}=f.$$

%$g^{\varphi}(x)=g\bigl(\varphi_{2r+1}(\ldots\varphi_2(x)\ldots)\bigr)$.
Преобразование~$\varphi$, очевидно, аффинно, невырожденно и~имеет вид:
$$    \hspace*{-33mm}\varphi(x) ={}\hspace*{33mm}
$$
\begin{equation*}
      \begin{split}
    {}=
    x &
    {
      \begin{pmatrix}
\makebox[1.5em]{$1$} &\rule{1.5em}{0pt} & \rule{1.5em}{0pt}   & 
\rule{1.5em}{0pt}   & \rule{1.5em}{0pt}   &  & \rule{1.5em}{0pt}   & 
\rule{1.5em}{0pt}   & \rule{1.5em}{0pt}   &  \rule{1.5em}{0pt}   \\
        0 & 1 &   &   &   &   &   &   &   &   \\
        * & * &\smash[t]{\ddots}&&   &   &   &   & 
\smash[t]{\mbox{\Huge{$0$}}}  &   \\
        * & * & \smash[t]{\ddots}  & 1 &   &   &   &   &   &      \\
        * & * &\smash[t]{\ddots}   & 0 & 1 &   &   &   &   &      \\
        * & * &\smash[t]{\ddots}   & * & * & 1 &   &   &   &   \\
        * & * &\smash[t]{\ddots}   & * & * & \makebox[1.5em]{$b_{2r+2}$}  & 1 &   &  &     \\
        * & * & \smash[t]{\ddots}  & * & * & \makebox[1.5em]{$b_{2r+3}$}  & 0 
&\smash[t]{\ddots}&& \\
        \vdots  &\vdots   & \smash[t]{\ddots}  &\vdots   &\vdots   & \vdots  & \vdots  &\ddots& 
1 &     \\
        * & * &\cdots   & * & * &b_n& 0 &\cdots & 0 & 1\\
      \end{pmatrix}} \oplus \\
   \oplus &
      \;\begin{pmatrix}
      \makebox[1.5em]{$*$}&\makebox[1.5em]{$*$}&\makebox[1.5em]{$\cdots$}&\makebox[1.5
em]{$*$}&\makebox[1.5em]{$*$}&\makebox[1.5em]{$c$}&\makebox[1.5em]{$0$}&\makebox
[1.5em]{$\cdots$}&\makebox[1.5em]{$0$}&\makebox[1.5em]{$0$} \\
      \end{pmatrix}
  \end{split}
\end{equation*}
(здесь знак~$*$ заменяет собой один из элементов~$\mathbb{F}_2$, каждый раз 
свой). Соответственно, преобразование~$\alpha$ из формулировки теоремы Диксона 
является обратным к~$\varphi$.

Как будет показано в~разд.~\ref{canonic-to-ANF}, представление функций~$f_i$ 
в~таком виде, т.\,е.\ нахождение подходящих представителей орбиты 
$\mathrm{Orb}_{f_i}(\mathrm{GA}(V_n))$, позволяет легко выписать аффинные 
нормальные формы для~$f_i$.

\section{Построение аффинной нормальной формы для~квадратичной 
функции}\label{canonic-to-ANF}

Обозначим через $\varphi_i$, $1\hm\leqslant i\hm\leqslant m$, невырожденные аффинные преобразования 
пространства~$V_n$, с~помощью которых функции~$f_i$ приводятся к~каноническому 
виду~$g_i$:
$$
g_i(x)=f_i^{\varphi_i}(x)=x_1x_2\oplus\cdots\oplus x_{2r_i-1}x_{2r_i}\oplus 
b_ix_{2r_i+1}\oplus c_i\,,
$$
где $2r_i=\rank \tilde{Q}_{f_i}$, а $b_i, c_i\hm\in \mathbb{F}_2$.

Очевидно, что если среди первых~$2r_i$ переменных взять все переменные с~четными 
индексами или все с~нечетными и~зафиксировать их значения, то получится 
плоскость, являющаяся локальной аффинностью функции~$g_i$.
Рассмотрим, например, $2^{r_i}$ плоскостей, заданных уравнениями:
$$    \begin{array}{l@{\,}c@{\ }l}
        x_1&=&\delta_1;\\
        x_3&=&\delta_2;\\
        \vdots\\
        x_{2r_i-1}&=&\delta_{r_i},\\
    \end{array}
$$
где $\delta_j\in\mathbb{F}_2$, $1\hm\leqslant j\hm\leqslant r_i$.
Каждую из этих плоскостей обозначим через~$\pi'_{i,\delta}$ со сложным индексом 
$\delta\hm=(\delta_1,\dots,\delta_{r_i})\in\mathbb{F}_2^{r_i}$.
Размерность~$\pi'_{i,\delta}$ равна $n\hm-r_i$, а мощность, соответственно, 
$2^{n-r_i}$.
Нетрудно видеть, что $\Pi'_i\hm=\{\pi'_{i,\delta}\}_{\delta\in\mathbb{F}_2^{r_i}}$ 
является разбиением пространства~$V_n$.

Обозначаемое ниже через~$l'_{i,\delta}$ сужение функции~$g_i$ на каждую из 
плоскостей разбиения~--- аффинно:
\begin{multline*}
l'_{i,\delta}(x)={}\\
{}=g_i|_{\pi'_{i,\delta}}(x)=\delta_1x_2\oplus\delta_2x_4\cdots\oplus\delta_{r_i}x_{2r_i}\oplus b_ix_{2r_i+1}\oplus c_i,
\hspace*{-0.80452pt}
\end{multline*}
а характеристическая функция соответствующей плоскости имеет вид:
$$
\chi_{\pi'_{i,\delta}}(x)=\prod\limits_{k=1}^{r_i}\left(x_{2k-1}\oplus\delta_k\oplus1\right)\,.
$$

С помощью аффинной нормальной формы
$$
g_i(x)=\bigoplus_{\delta\in\mathbb{F}_2^{r_i}} 
\chi_{\pi'_{i,\delta}}(x)l'_{i,\delta}(x)
$$
для канонического представления функции~$f_i$ можно аффинным преобразованием, 
обратным к~$\varphi_i$, получить аффинную нормальную форму для исходной функции:

\vspace*{1pt}

\noindent
$$
f_i(x)=g_i^{\varphi_i^{-1}}(x) = \bigoplus_{\delta\in\mathbb{F}_2^{r_i}} 
\chi_{\pi_{i,\delta}}(x)l_{i,\delta}(x)\,,
$$

\vspace*{-3pt}

\noindent
где $\pi_{i,\delta}\hm=\varphi_i(\pi'_{i,\delta})$ 
и~$l_{i,\delta}(x)\hm=l'_{i,\delta}(\varphi_i^{-1}(x))$.

Разумеется, если алгебраическая степень какой-либо функции~$f_i$ оказалась 
равной~$1$, то искать ничего не нужно: ее полином Жегалкина является ее аффинной 
нормальной формой для тривиального разбиения $\Pi_i\hm=\{V_n\}$.

\smallskip

\noindent
\textbf{Замечание~2}.
    Подобный способ построения аффинной нормальной формы можно использовать 
    и~непосредственно для квадратичной\footnote{Для функций более высоких степеней 
такой подход тоже работает, но описать его строго гораздо сложнее и~полученные 
таким образом локальные аффинности, скорее всего, будут слишком маленькой 
размерности.} функции~$f$ в~ее исходном виде. Нужно просто фиксировать значения 
переменных так, чтобы в~каждом мономе оставалось не более одной свободной 
переменной. Для этого удобнее рассмотреть матрицу~$Q_f$, выбрать в~ней столбец 
или строку с~максимальным числом единиц среди всех столбцов и~строк (пусть это 
будет $k$-я строка) и~зафиксировать~$x_k$. Затем то же проделать, исключив из 
рассмотрения $k$-ю строку и~$k$-й столбец матрицы, и~так далее, пока единицы 
в~матрице не кончатся.
Однако, несмотря на то что здесь имеет место экономия на приведении функции 
к~каноническому виду, такой способ представляется менее эффективным в~следующем 
смысле. Канонический вид квадратичной функции содержит минимальное число мономов 
степени~$2$, поэтому для исходной (неканонической) функции придется фиксировать, 
как правило, большее число переменных. Но с~каждой дополнительно зафиксированной 
переменной размерность локальных аффинностей функции~$f$ уменьшается на~$1$, 
а~их число, соответственно, увеличивается вдвое.

\vspace*{-4pt}


\section{<<Локальные>> системы линейных уравнений}

\vspace*{-2pt}

Идея метода решения систем квадратичных булевых уравнений состоит в~следующем.
Пусть для всех~$f_i$, $1\hm\leqslant i\hm\leqslant m$, 
определены некоторые аффинные нормальные 
формы

\vspace*{1pt}

\noindent
\begin{equation*}
\label{AffNF_ij}
    f_i(x) = \bigoplus_{j=1}^{\lambda(i)} \chi_{\pi_{ij}}(x)l_{ij}(x)\,.
\end{equation*}

\vspace*{-3pt}

\noindent
Исходя из этих аффинных нормальных форм, можно для каждой пары~$i,j$ записать 
эквивалентную уравнению $f_i\hm=0$ на~$\pi_{ij}$ систему линейных уравнений:

\columnbreak

\noindent
\begin{equation*}
    \begin{array}{r@{\ }c@{\ }l}
        l_{ij}(x)&=&0;\\
        h_{ij}^1(x)&=&0;\\
        \vdots\\
        h_{ij}^{d(i,j)}(x)&=&0,\\
    \end{array}
  \label{approx}
\end{equation*}
в которой первое уравнение выражает равенство~$f_i\hm=0$ через аффинную 
аппроксимацию~$l_{ij}(x)$ функции~$f_i(x)$ на плоскости~$\pi_{ij}$, а остальные 
$d(i,j)$ уравнений задают эту плоскость.


\smallskip

\noindent
\textbf{Замечание~3.}\
Если аффинная нормальная форма получена описанным выше способом~--- через 
канонический вид квадратичной функции,~--- то характеристическая функция будет 
иметь вид:

\vspace*{1pt}

\noindent
$$
\chi_{\pi_{i,\delta}}(x)=\prod_{k=1}^{r_i}(\varphi_i^{-1}(x)e_{2k-1}
\Tsf\oplus\delta_k\oplus 1)\,,
$$

\vspace*{-3pt}

\noindent
где $e_{2k-1}$~--- $(2k-1)$-й базисный вектор, т.\,е.\ $\varphi_i^{-1}(x)e_{2k-1}
\Tsf$~--- $(2k-1)$-я компонента вектора~$\varphi_i^{-1}(x)$.
Значит, соответствующую плоскость задают уравнения
 $\{ \varphi_i^{-1}(x)e_{2k-1}\Tsf \oplus \delta_k \hm= 0 
 \mid 1\hm\leqslant k\hm\leqslant r_i \}$.

\smallskip

Таким образом, имеется набор <<локальных>> линейных систем для каждого уравнения 
исходной системы и~для каждой его локальной аффинности.
Метод состоит в~том, чтобы подобрать комбинацию <<локальных>> систем разных 
квадратичных уравнений, в~совокупности дающую решение исходной системы. Если 
решение квадратичной системы единственно (а~это естественное предположение для 
криптоанализа), ровно одна такая комбинация будет иметь решение, и~от того, как 
быстро удастся ее обнаружить, зависит эффективность метода.

\vspace*{-4pt}

\section{О~трудоемкости построения аффинной нормальной формы}

\vspace*{-2pt}

Напомним, что для функций из системы~\eqref{system} $r_i\hm=({1}/{2})\rank 
\tilde{Q}_{f_i}\hm\leqslant {n}/{2}$, $1\hm\leqslant i\hm\leqslant m$,~--- 
параметр, введенный в~разд.~\ref{canonic-to-ANF}.
Алгоритм приведения $m$~функций к~каноническому виду (см.\ разд.~4) 
имеет трудоемкость, оцениваемую выражением $O(\sum\nolimits_{i=1}^m n^2r_i)$, 
а~учитывая 
неравенство $r_i\hm\leqslant {n}/{2}$, имеем~$O(mn^3)$.

При построении аффинных нормальных форм для функции~$f_i$ в~разд.~5 
потребуется порядка $r_i2^{r_i}\hm+ n^3$ операций. 
Значит, для всех~$m$~функций имеем оценку $O(mn^3\hm+\sum\nolimits_{i=1}^m r_i2^{r_i})$.

Таким образом, в~худшем случае, когда все $r_i\hm={n}/{2}$ или даже когда хотя 
бы $r_i\hm=O(n)$ для некоторого~$i$, предложенный алгоритм экспоненциален.
Однако можно рассчитывать, что во встречающихся на практике системах 
квадратичных уравнений параметр~$r_i$ растет (с~увеличением~$n$) медленнее, 
и~тогда можно говорить о полиномиальности алгоритма построения аффинных нормальных 
форм.

В случае, когда система вида~\eqref{system} переопределенная, т.\,е.\ $n\hm\ll m$ 
(переопределенные системы достаточно часто рассматриваются в~задачах 
информатики, теории кодирования и~криптографии), можно рассчитывать на 
существование подсистемы (из~$l$~уравнений с~номерами $i_1,\ldots,i_l$), для 
которой трудоемкость построения аффинных нормальных форм меньше, чем 
экспоненциальная. Например, когда $r_{i_j}\hm=O(\sqrt{n})$, $ 1\hm\leqslant j\hm\leqslant l$, 
оценка со\-от\-вет\-ст\-ву\-ющей трудоемкости для системы~\eqref{system} имеет 
субэкспоненциальный характер.

Рассмотрим в~качестве еще одного примера класс~$\mathcal{K}_m$ систем 
$m$~квадратичных булевых уравнений от $n$~неизвестных вида~\eqref{system}, где 
$m\hm=m(n)$~--- некоторый полином от~$n$ и~где $r_i=
O(\log_2 n)$ для всех~$i$, $1\hm\leqslant i\hm\leqslant m$.


\vspace*{2pt}


\noindent
\textbf{Предложение~3.}
\textit{Для систем~\eqref{system} квадратичных булевых уравнений из 
класса~$\mathcal{K}_m$ существует полиномиальный} (\textit{по~$n$}) \textit{алгоритм построения 
аффинных нормальных форм для функций~$f_i$.}


\smallskip

Для доказательства этого утверждения достаточно рассмотреть предложенный 
в~статье алгоритм построения аффинных нормальных форм для квад\-ра\-тич\-ных булевых 
функций. В~полученной выше оценке  $O(mn^3\hm+\sum\nolimits_{i=1}^m r_i2^{r_i})$
данные в~условии ограничения на~$m$ и~на~$r_i$ дают полиномиальную оценку трудоемкости 
алгоритма.

%Отметим, что если рассматривать систему квадратичных уравнений, описывающую 
%функционирование произвольного фильтрующего генератора, то у всех уравнений 
%системы будет одно и~то же значение $r_i$, определяемое рангом матрицы 
%$\tilde{Q}_{f'}$, где $f'$ "--- ... для фильтрующей функции~$f$. Поэтому

\vspace*{-12pt}

{\small\frenchspacing
 {%\baselineskip=10.8pt
 \addcontentsline{toc}{section}{References}
 \begin{thebibliography}{99}

    \bibitem{GJ1982}
        \Au{Гэри~М., Джонсон~Д.}
        Вычислительные машины и~труднорешаемые задачи~/ Пер. с~англ.~---
        М.: Мир, 1982. 416~с.
        (\Au{Garey~M.\,R., Johnson~D.\,S.} Computers and intractability: 
A~guide to the theory of NP-completeness.~--- San Francisco, CA, USA: W.\,H.~Freeman 
and Co., 1979. 348~p.).

    \bibitem{GT2017}
        \Au{Горшков~С.\,П., Тарасов~А.\,В.}
        Сложность решения сис\-тем булевых уравнений.~---
        М.: Курс, 2017. 192~с.

    \bibitem{Smi2000}
        \Au{Смирнов~В.\,Г.}
        {Некоторые классы эффективно ре\-ша\-емых систем булевых уравнений}~//
        Труды по дискретной математике, 2000. Т.~3. С.~269--282.

    \bibitem{Bard2009}
        \Au{Bard~G.\,V.}
        Algebraic cryptanalysis.~--- Springer, 2009. 389~p.

    \bibitem{BCJ2007}
        \Au{Bard~G., Courtois~N., Jefferson~C.}
        {Efficient methods for conversion and solution of sparse systems of 
        low-degree multivariate polynomials over $\mathrm{GF}(2)$ via SAT-solvers}~//
        Cryptology ePrint Archive. Report 2007/024.
        {\sf http://eprint.iacr.org/2007/024.pdf}.

    \bibitem{LSSYa2015}
        \Au{Логачев~О.\,А., Сальников~А.\,А., Смышляев~С.\,В., 
Ященко~В.\,В.}
        Булевы функции в~теории кодирования и~крип\-то\-ло\-гии.~---
        М.: ЛЕНАНД, 2015. 576~с.

    \bibitem{MWS1979}
        \Au{Мак-Вильямс~Ф.\,Дж., Слоэн~Н.\,Дж.\,А.}
        Теория кодов, исправляющих ошибки~/ Пер. с~англ.~---
        М.: Связь, 1979. 743~с.
        (\Au{MacWilliams~F.\,J., Sloane~N.\,J.\,A.} The theory of 
        error-correcting codes.~--- 
        North-Holland mathematical library ser.~---
        North-Holland Publishing Co., 1977.  774~p.)

    \bibitem{LYaD2007}
        \Au{Logachev~O.\,A., Yashchenko~V.\,V., Denisenko~M.\,P.}
        {Local affinity of Boolean mappings}~//
        Boolean functions in cryptology and information security: Proceedings of the 
NATO Advanced Study Institute.~---
        IOS Press, 2008. P.~148--172.

    \bibitem{Gor1995}
        \Au{Горшков~С.\,П.}
        {Применение теории NP-пол\-ных задач для оценки сложности решения систем 
булевых уравнений}~//
        Обозрение прикладной и~промышленной математики, 1995. Т.~2. Вып.~3. 
С.~325--398.

    \bibitem{Dickson1901}
        \Au{Dickson~L.\,E.}
        Linear groups: With an exposition of the Galois field theory.~---
        Leipzig: B.\,G.\,Teubner, 1901. 322~p.

   % \bibitem{KSh1999}
       % \Au{Kipnis~A., Shamir~A.}
      %  {Cryptanalysis of the HFE public key cryptosystem by relinearization}~//
     %   Advances in cryptology~/
    %    Ed.\ M.\,J.~Wiener.~---
   %     Lectures notes in computer science ser.~---
   %     Springer, 1999. Vol.~1666. P.~19--30.

   % \bibitem{CShPK2000}
  %      \Au{Courtois~N., Klimov~A., Patarin~J., Shamir~A.}
 %       {Efficient algorithms for solving overdefined systems of multivariate 
%polynomial equations}~// Advances in cryptology~/
%Ed.\ B.~Preneel.~---
%         Lectures notes in computer science ser.~--- Springer, 2000. Vol.~1807. 
%P.~392--407.

   % \bibitem{FY1980}
  %      \Au{Fraenkel~A.\,S., Yesha~Y.}
 %       {Complexity of solving algebraic equations}~//
 %       Inform. Process. Lett., 1980. Vol.~10. Iss.~4-5. P.~178--179.

\end{thebibliography} 
 }
 }

\end{multicols}

\vspace*{-3pt}

\hfill{\small\textit{Поступила в~редакцию 11.01.19}}

\vspace*{8pt}

%\pagebreak

%\newpage

%\vspace*{-28pt}

\hrule

\vspace*{2pt}

\hrule

%\vspace*{-2pt}

\def\tit{POLYNOMIAL ALGORITHMS FOR~CONSTRUCTING LOCAL AFFINITIES OF~QUADRATIC BOOLEAN FUNCTIONS}

\def\titkol{Polynomial algorithms for~constructing local affinities of~quadratic Boolean functions}

\def\aut{O.\,A.~Logachev$^{1,2}$, A.\,A.~Sukayev$^1$, and~S.\,N.~Fedorov$^1$}

\def\autkol{O.\,A.~Logachev, A.\,A.~Sukayev, and~S.\,N.~Fedorov}

\titel{\tit}{\aut}{\autkol}{\titkol}

\vspace*{-11pt}


\noindent
$^1$Information Security Institute,  M.\,V.~Lomonosov Moscow State University, 
1~Michurinskiy Prosp., Moscow\linebreak
$\hphantom{^1}$119192, Russian Federation

\noindent
$^2$Institute of Informatics Problems, 
Federal Research Center ``Computer Science and Control'' 
of the Russian\linebreak
$\hphantom{^1}$Academy of Sciences, 44-2~Vavilov Str., Moscow 119333, 
Russian Federation

\def\leftfootline{\small{\textbf{\thepage}
\hfill INFORMATIKA I EE PRIMENENIYA~--- INFORMATICS AND
APPLICATIONS\ \ \ 2019\ \ \ volume~13\ \ \ issue\ 1}
}%
 \def\rightfootline{\small{INFORMATIKA I EE PRIMENENIYA~---
INFORMATICS AND APPLICATIONS\ \ \ 2019\ \ \ volume~13\ \ \ issue\ 1
\hfill \textbf{\thepage}}}

\vspace*{6pt}


\Abste{Due to the affine normal form, one can consider a~Boolean function 
as affine on certain flats in its domain~--- so-called local affinities. 
This Boolean function representation~--- affine approximation~---
could be
useful 
for solving systems of nonlinear equations over two-element field. The problem 
of solving these systems
(of a~special sort) arises, in particular, in some methods 
of the information security tools design and analysis.
The
paper describes an approach to finding local affinities for quadratic Boolean 
functions which is based on Dickson's\linebreak\vspace*{-12pt}}

\Abstend{theorem. By this, one obtains affine 
normal forms for such functions. Besides, the paper concerns the efficiency of 
corresponding algorithms.
This approach can be profitable for constructing efficient methods of solving 
systems of quadratic Boolean equations via ``approximation'' of corresponding 
Boolean functions by their affine normal forms.}

\KWE{Boolean function; system of quadratic Boolean equations; vector 
space partition; flat; local affinity; Dickson's theorem; 
affine normal form (ANF) of Boolean function; algebraic cryptanalysis}






\DOI{10.14357/19922264190110}

\vspace*{-14pt}

\Ack
\noindent
The paper was partly supported by the Russian Foundation for Basic Research 
(project 18-29-03124~mk).





  \begin{multicols}{2}

\renewcommand{\bibname}{\protect\rmfamily References}
%\renewcommand{\bibname}{\large\protect\rm References}

{\small\frenchspacing
 {%\baselineskip=10.8pt
 \addcontentsline{toc}{section}{References}
 \begin{thebibliography}{99}
\bibitem{1-log-1}
\Aue{Garey, M.\,R., and D.\,S.~Johnson.} 1979. \textit{Computers and intractability: 
A~guide to the theory of NP-completeness.} San Francisco, CA: W.\,H.~Freeman and Co. 348~p.
\bibitem{2-log-1}
\Aue{Gorshkov, S.\,P., and A.\,V.~Tarasov.} 2017. \textit{Slozhnost' re\-she\-niya 
sistem bulevykh uravneniy} [Complexity of solving the systems of 
Boolean equations]. Moscow: Kurs. 192~p.
\bibitem{3-log-1}
\Aue{Smirnov, V.\,G.} 2000. Nekotorye klassy effektivno reshaemykh 
sistem bulevykh uravneniy [Some classes of Boolean equation systems 
permitting effective solution]. 
\textit{Trudy po diskretnoy matematike} [Proceedings on Discrete Mathematics] 3:269--282.
\bibitem{4-log-1}
\Aue{Bard, G.\,V.} 2009. \textit{Algebraic cryptanalysis}. Springer. 389~p.
\bibitem{5-log-1}
\Aue{Bard, G., N.~Courtois, and C.~Jefferson.} 2007. 
Efficient methods for conversion and solution of sparse systems of 
low-degree multivariate polynomials over GF(2) via SAT-solvers. 
\textit{Cryptology ePrint Archive}. Report 2007/024. Available at: 
{\sf http://eprint.iacr.org/2007/024.pdf} (accessed August~30, 2018).
\bibitem{6-log-1}
\Aue{Logachev, O.\,A., A.\,A.~Sal'nikov, S.\,V.~Smyshlyaev, and V.\,V.~Yashchenko.} 
2015. \textit{Bulevy funktsii v~teorii kodirovaniya i~kriptologii} 
[Boolean functions in coding theory and cryptology]. Moscow: LENAND. 576~p.
\bibitem{7-log-1}
\Aue{MacWilliams, F.\,J., and N.\,J.\,A.~Sloane.} 1977. 
\textit{The theory of error-correcting codes}. 
North-Holland mathematical library ser.
North-Holland Publishing Co. 774~p.
\bibitem{8-log-1}
\Aue{Logachev, O.\,A., V.\,V.~Yashchenko, and M.\,P.~Denisenko.} 2008. 
Local affinity of Boolean mappings. 
\textit{Boolean functions in cryptology and information security: 
Proceedings of the NATO Advanced Study Institute.} IOS Press. 148--172.
\bibitem{9-log-1}
\Aue{Gorshkov, S.\,P.} 1995. Primenenie teorii NP-polnykh zadach 
dlya otsenki slozhnosti resheniya sistem bulevykh uravneniy 
[Application of the NP-complete problem theory to assessment 
of complexity of solving the systems of Boolean equations]. 
\textit{Obozrenie prikladnoy i~promyshlennoy matematiki} 
[Applied and Industrial Mathematics Review] 2(3):325--398.
\bibitem{10-log-1}
\Aue{Dickson, L.\,E.} 1901. \textit{Linear groups: 
With an exposition of the Galois field theory}. Leipzig: B.\,G.~Teubner. 322~p.
%\bibitem{11-log-1}
%\Aue{Kipnis, A., and A.~Shamir.} 1999. 
%Cryptanalysis of the HFE public key cryptosystem by relinearization. 
%\textit{Advances in cryptology}. Ed. M.\,J.~Wiener.
% Lecture notes in computer science ser.  Springer. 1666:19--30.
%\bibitem{12-log-1}
%\Aue{Courtois, N., A.~Klimov, J.~Patarin, and A.~Shamir.} 2000. 
%Efficient algorithms for solving overdefined systems of multivariate polynomial 
%equations. \textit{Advances in cryptology}. Ed.\ B.~Preneel.
%Lecture notes in computer science ser.  Springer. 1807:392--407.
%\bibitem{13-log-1}
%\Aue{Fraenkel, A.\,S., and Y.~Yesha.} 1980. 
%Complexity of solving algebraic equations. 
%\textit{Inform. Process. Lett.} 10(4-5):178--179.
\end{thebibliography}

 }
 }

\end{multicols}

\vspace*{-6pt}

\hfill{\small\textit{Received January 11, 2019}}

%\pagebreak

%\vspace*{-18pt}

\Contr

\noindent
\textbf{Logachev Oleg A.} (b.\ 1950)~--- 
Candidate of Science (PhD) in physics and mathematics, head of department, 
Information Security Institute, M.\,V.~Lomonosov Moscow State University, 
1~Michurinskiy Prosp., Moscow 119192, Russian Federation; 
senior scientist, Institute of Informatics Problems, 
Federal Research Center ``Computer Science and Control'' 
of the Russian Academy of Sciences, 44-2~Vavilov Str., Moscow 119333, 
Russian Federation; \mbox{logol@iisi.msu.ru }

 



\vspace*{3pt}

\noindent
\textbf{Sukayev Al'bert A.} (b.\ 1994)~--- 
student, Information Security Institute, Moscow State University, 
1~Michurinskiy Prosp., Moscow 119192, Russian Federation; 
\mbox{asukaev@gmail.com}

\vspace*{3pt}

\noindent
\textbf{Fedorov Sergey~N.} (b.\ 1982)~--- 
Candidate of Science (PhD) in physics and mathematics, senior scientist, 
Information Security Institute, M.\,V.~Lomonosov Moscow State University, 
1~Michurinskiy Prosp., Moscow 119192, Russian Federation; 
\mbox{s.n.feodorov@yandex.ru}
\label{end\stat}

\renewcommand{\bibname}{\protect\rm Литература}         %10
%\renewcommand{\r}{\mathbb R}
%\newcommand{\eqd}{\stackrel{d}{=}}
%\newcommand{\pto}{\stackrel{P}{\longrightarrow}}

\def\stat{kor+gor+zei}

\def\tit{НОВЫЕ ПРЕДСТАВЛЕНИЯ ОБОБЩЕННОГО РАСПРЕДЕЛЕНИЯ МИТТАГ-ЛЕФФЛЕРА 
В~ВИДЕ СМЕСЕЙ И~ИХ ПРИЛОЖЕНИЯ$^*$}

\def\titkol{Новые представления обобщенного распределения Миттаг-Леффлера 
в~виде смесей и~их приложения}

\def\aut{В.\,Ю.~Королев$^1$, А.\,К.~Горшенин$^2$, А.\,И.~Зейфман$^3$}

\def\autkol{В.\,Ю.~Королев, А.\,К.~Горшенин, А.\,И.~Зейфман}

\titel{\tit}{\aut}{\autkol}{\titkol}

\index{Королев В.\,Ю.}
\index{Горшенин А.\,К.}
\index{Зейфман А.\,И.}
\index{Korolev V.\,Yu.}
\index{Gorshenin A.\,K.} 
\index{Zeifman A.\,I.}



{\renewcommand{\thefootnote}{\fnsymbol{footnote}} \footnotetext[1]
{Работа выполнена при поддержке РФФИ (проект 17-07-00717).}}


\renewcommand{\thefootnote}{\arabic{footnote}}
\footnotetext[1]{Факультет вычислительной математики и~кибернетики
Московского государственного университета им.\ М.\,В.~Ломоносова;
Институт проблем информатики Федерального исследовательского
центра <<Информатика и~управ\-ле\-ние>> Российской академии наук; 
Hangzhou Dianzi University, Китай, \mbox{vkorolev@cs.msu.ru}}
\footnotetext[2]{Институт проблем информатики Федерального исследовательского
центра <<Информатика и~управ\-ле\-ние>> Российской академии наук; факультет
вычислительной математики и~кибернетики Московского государственного университета им.\ 
М.\,В.~Ломоносова,
\mbox{agorshenin@frccsc.ru}}
\footnotetext[3]{Вологодский государственный университет; 
Институт проб\-лем информатики Федерального исследовательского
центра <<Информатика и~управление>> Российской академии наук;
Вологодский научный центр Российской академии наук, \mbox{a\_zeifman@mail.ru}}

%\vspace*{8pt}


\Abst{Приведены новые представления обобщенного
распределения Мит\-таг-Леф\-фле\-ра в~виде смесей. В~част\-ности, показано,
что при значениях <<обобщающего>> параметра, не превосходящих
единицы, обобщенное распределение Мит\-таг-Леф\-фле\-ра является
масштабной смесью полунормальных законов, масштабной смесью
<<обычных>> распределений Мит\-таг-Леф\-фле\-ра или масштабной смесью
обобщенных распределений Мит\-таг-Леф\-фле\-ра с~большими значениями
характеристического показателя. Во всех случаях приведены явные
выражения для смешивающих величин. Полученные представления
позволяют предложить новые алгоритмы моделирования случайных величин (с.в.)\
с обобщенным распределением Мит\-таг-Леф\-фле\-ра и~сформулировать новые
предельные теоремы, в~которых указанное распределение выступает в~качестве предельного.}

\KW{обобщенное распределение Мит\-таг-Леф\-фле\-ра;
масштабная смесь; обобщенное гам\-ма-рас\-пре\-де\-ле\-ние; полунормальное
распределение; устойчивое распределение}

\DOI{10.14357/19922264180411}
  
%\vspace*{4pt}


\vskip 10pt plus 9pt minus 6pt

\thispagestyle{headings}

\begin{multicols}{2}

\label{st\stat}

\section{Введение}

Данная статья продолжает исследования, начатые в~работах~[1--4]. В~статье 
приведены новые представления обобщенного распределения Мит\-таг-Леф\-фле\-ра 
в~виде смесей. Это распределение представляет особый интерес как 
<<тяжелохвостая>>\linebreak модель статистических закономерностей, при которых 
большие значения наблюдаемых характери-\linebreak стик встречаются намного 
чаще, чем предписывает классическая экспоненциальная модель. Оно возникает 
в~некоторых задачах, связанных с~дифференциальными уравнениями дробного порядка 
в физике, астрономии, финансовой математике и~других областях~[5--8]. 
<<Обычное>> распределение Мит\-таг-Леф\-фле\-ра традиционно рассматривается 
вместе с~распределением Линника, поскольку\linebreak характеристическая функция 
(х.ф.)\ распределения Линника имеет такой же аналитический вид, как 
преобразование Лап\-ла\-са--Стилть\-еса (п.~Л.--С.)\ распределения 
Мит\-таг-Леф\-фле\-ра. Поэтому эти распределения обладают многими сходными 
свойствами. В~частности, они геометрически устойчивы, так как являются 
предельными для геометрических случайных сумм независимых одинаково 
распределенных с.в.\ с~бесконечными дисперсиями и~потому представимы в~виде 
масштабных смесей устойчивых законов, в~которых смешивающим выступает 
распределение Вейбулла. Соответственно, обоб\-щен\-ные распределения Линника 
и~Мит\-таг-Леф\-фле\-ра представляют собой масштабные смеси устойчивых законов, 
в~которых смешивающим служит обобщенное гам\-ма-рас\-пре\-де\-ление.

В данной работе приведены альтернативные представления обобщенного
распределения Мит\-таг-Леф\-фле\-ра в~виде смесей. В~частности, показано,
что при значениях <<обобщающего>> параметра, не превосходящих
единицы, обобщенное распределение Мит\-таг-Леф\-фле\-ра~--- это масштабная смесь полунормальных законов, <<обычных>> распределений Миттаг-Леффлера или
обобщенных распределений Мит\-таг-Леф\-фле\-ра с~б$\acute{\mbox{о}}$льшими значениями
характеристического показателя. Во всех случаях приведены явные
выражения для смешивающих величин. Полученные представления
позволяют предложить новые алгоритмы моделирования с.в.\
с~обобщенным распределением Мит\-таг-Леф\-фле\-ра и~сформулировать новые
предельные теоремы, в~которых указанное распределение выступает 
в~качестве предельного.

Аналогичные результаты относительно обобщенного распределения
Линника приведены в~\cite{KorolevGorsheninZeifman2018}, где,
в~частности, показано, что обобщенное распределение Линника~--- масштабная 
смесь нормальных законов со смешивающим распределением типа обобщенного 
распределения Мит\-таг-Леф\-фле\-ра. Здесь этот результат будет использован 
для вывода некоторых свойств обобщенного распределения Мит\-таг-Леф\-флера.

\vspace*{-4pt}

\section{Распределения Миттаг-Леффлера и~Линника}

Пусть
$\alpha\in(0,1]$ и~$M_{\alpha}$~--- неотрицательная с.в.\ 
с~п.~Л.--С.:
\begin{equation}
\psi_{\alpha}(s)\equiv {\sf
E}\exp\{-sM_{\alpha}\}=\left(1+s^{\alpha}\right)^{-1},\enskip
s\ge0\,.\label{e1-gzk}
\end{equation}
Распределения с~п.~Л.--С.~(1) принято называть \textit{распределениями
Мит\-таг-Леф\-фле\-ра}. Происхождение этого названия связано с~тем, что
плотность, соответствующая п.~Л.--С.~(1), имеет вид:

\noindent
\begin{multline*}
f_{\alpha}^{M}(x)=\fr{1}{x^{1-\alpha}}\sum\limits_{n=0}^{\infty}\fr{(-1)^nx^{\alpha
n}}{\Gamma(\alpha n+1)}=-\fr{d}{dx}\,E_{\alpha}(-x^{\alpha})\,,\\
x\ge0,
\end{multline*}

\vspace*{-2pt}

\noindent
где $E_{\alpha}(z)$~--- функция Мит\-таг-Леф\-фле\-ра индекса~$\alpha$,
определяемая как степенной ряд
$$
E_{\alpha}(z)=\sum\limits_{n=0}^{\infty}\fr{z^n}{\Gamma(\alpha
n+1)}\,,\enskip \alpha>0\,,\ \ z\in\mathbb{Z}\,.
$$
Функция распределения (ф.р.), соответствующая плотности
$f_{\alpha}^{M}(x)$, будет обозначаться~$F_{\alpha}^{M}(x)$.

При $\alpha=1$ распределение Мит\-таг-Леф\-фле\-ра превращается 
в~стандартное показательное распределение: $M_1\hm\eqd W_1$. Но при
$\alpha\hm<1$ плот\-ность~(1) имеет хвост, убывающий степенн$\acute{\mbox{ы}}$м
образом: если $0\hm<\alpha\hm<1$, то
$\lim\nolimits_{x\to\infty}
x^{\alpha+1}f_\alpha^{M}(x)\hm=\pi^{-1}\Gamma(\alpha\hm+1)\sin\pi\alpha$
(см., например,~\cite{Kilbas2014}).

Моменты с.в.\ с~распределением Мит\-таг-Леф\-фле\-ра порядков
$\beta\hm\ge\alpha$ бесконечны, но если $0\hm<\beta\hm<\alpha\hm<1$, то ${\sf E}
M_{\alpha}^{\beta}\hm={\Gamma(1\hm+{\beta}/{\alpha})\Gamma(1\hm-{\beta}/{\alpha})}$.

Пусть $\nu>0$, $\alpha\in(0,1]$. Распределение неотрицательной с.в.\
$M_{\alpha,\,\nu}$, соответствующее п.~Л.--С.
$$
\psi_{\alpha,\nu}(s)\equiv {\sf E}
e^{-sM_{\alpha,\,\nu}}=\left(1+s^{\alpha}\right)^{-\nu}\,,\enskip s\ge0\,,
$$
называется \textit{обобщенным распределением Мит\-таг-Леф\-фле\-ра} 
(см.~\cite{MathaiHaubold2011, Joseetal} и~ссылки в~этих работах).

Распределения с~х.ф.~$\mathfrak{f}^{L}_{\alpha}(t)\hm=\left(1\hm+|t|^{\alpha}\right)^{-1}$,
$t\hm\in\mathbb{R}$, где $0\hm<\alpha\hm\le2$, принято называть 
\textit{распределениями Линника} (в~работе~\cite{Pillai1985} предложено
альтернативное менее употребительное название \textit{$\alpha$-Laplace
distribution}). Они были введены Ю.\,В.~Линником в~1953~г.~\cite{Linnik1953}. 
При $\alpha\hm=2$ распределение Линника
превращается в~распределение Лапласа, соответствующее плотности
\begin{equation}
f^{\Lambda}(x)=\fr{1}{2}\,e^{-|x|}\,,\enskip
x\in\mathbb{R}\,.
\label{e2-kgz}
\end{equation}
Лапласовская с.в.\ с~плотностью~(2) и~ее ф.р.\ будут соответственно
обозначаться~$\Lambda$ и~$F^{\Lambda}(x)$.

Случайная величина, имеющая распределение Линника с~параметром~$\alpha$, ее ф.р.\
и~плот\-ность будут соответственно обозначаться~$L_{\alpha}$,
$F_{\alpha}^{L}$ и~$f_{\alpha}^{L}$. При этом $F_2^{L}(x)\hm\equiv
F^{\Lambda}(x)$, $x\hm\in\mathbb{R}$.

Распределения Линника обладают многими интересными свойствами,
которые описаны, например, в~работах~\cite{KotzOstrovskiiHayfavi1995a,
KotzOstrovskiiHayfavi1995b, Lin1994, Anderson1992, Devroye1990}.
Абсолютные моменты порядков $\beta\hm<\alpha$ с.в.~$L_{\alpha}$ имеют
вид:
$$
{\sf E}\left\vert L_{\alpha}\right\vert^{\beta}=
\fr{2^{\beta}}{\sqrt{\pi}}\,\fr{\Gamma\left(1+{\beta}/{\alpha}\right)
\Gamma\left(({1+\beta})/{2}\right)\Gamma\left(1-{\beta}/{\alpha}\right)}
{\Gamma\left(1-{\beta}/{2}\right)}.
$$
В работе~\cite{Jacquesetal1999} показано, что при $0\hm<\alpha\hm<2$
хвосты распределения Линника убывают степенн$\acute{\mbox{ы}}$м образом:

\noindent
$$
\lim\limits_{x\to\infty}x^{\alpha}\left[1-F^{L}_{\alpha}(x)\right]=
\pi^{-1}\Gamma(\alpha)\sin\left(\fr{\pi\alpha}{2}\right)\,.
$$


В работах~\cite{Pakes1998, KotzOstrovskii1996, KorolevZeifman2016,
KorolevZeifmanKMJ} получены разнообразные представления
распределений Линника в~виде смесей. Некоторые из этих представлений
будут приведены и~использованы ниже.

В работе~\cite{Pakes1998} замечено, что \textit{обобщенные
распределения Линника}, задаваемые х.ф.

\vspace*{-6pt}

\noindent
\begin{multline}
\phi_{\alpha,\nu,\theta}(t)=
\left(1+e^{-i\theta\,{\mathrm{sgn}\,t}}|t|^{\alpha}\right)^{-\nu},\\
t\in\mathbb{R}\,,\enskip
|\theta|\le\min\left\{\fr{1}{2}\,\pi\alpha,\,\pi-\fr{1}{2}\,\pi\alpha\right\},\
\nu>0\,,
\label{e3-kgz}
\end{multline}

\vspace*{-1pt}

\noindent
играют видную роль в~некоторых характеризационных задачах
математической статистики. Среди работ, посвященных свойствам этих
распределений и~их применениям, следует упомянуть~[5, 6, 13, 19, 21--25].

В данной работе будут рассматриваться сим\-мет\-рич\-ные распределения,
для которых в~соотношении~(\ref{e3-kgz}) $\theta\hm=0$.

\vspace*{-6pt}

\section{Вспомогательные сведения}

\vspace*{-3pt}

В дальнейшем удобнее вести изложение
не в~терминах распределений, а~в~терминах с.в., предполагая, что
все они заданы на одном вероятностном пространстве
$(\Omega,\mathfrak{A}, {\sf P})$.

Случайная величина со стандартной показательной ф.р.\ будет 
обозначаться~$W_1$: 

\noindent
$$
{\sf P}\left(W_1\hm<x\right)=\left[1\hm-e^{-x}\right]
{\bf 1}(x\hm\ge0)
$$ 
(здесь и~далее символ~${\bf 1}(C)$ обозначает индикатор
множества~$C$). Случайная величина со стандартной нормальной ф.р.~$\Phi(x)$
будет обозначаться~$X$:
$$
{\sf P}(X<x)=\Phi(x)=\fr{1}{\sqrt{2\pi}}\int\limits_{-\infty}^{x}e^{-z^2/2}\,dz\,,
\enskip x\in\mathbb{R}\,.
$$
Функция распределения 
и~плот\-ность строго устойчивого распределения
с~характеристическим показателем~$\alpha$ и~параметром формы~$\theta$, 
определяемого характеристической функцией 
$$
\mathfrak{f}_{\alpha,\theta}(t)=
\exp\left\{-|t|^{\alpha}\exp\{-(1/2)i\pi\theta\alpha\,\mathrm{sign}\,t\}\right\},\
t\in\mathbb{R}\,,
$$
 где $0\hm<\alpha\hm\le2$,
$|\theta|\hm\le\min\left\{1,{2}/{\alpha}\hm-1\right\}$, будут
соответственно обозначаться $G_{\alpha,\theta}(x)$ 
и~$g_{\alpha,\theta}(x)$ (см., например,~\cite{Zolotarev1983}). Любую
с.в.\ с~ф.р.\ $G_{\alpha,\theta}(x)$ будем обозначать
$S_{\alpha,\theta}$. Симметричным строго устойчивым распределениям
соответствует значение $\theta\hm=0$ и~х.ф.~$\mathfrak{f}_{\alpha,0}(t)
\hm=e^{-|t|^{\alpha}}$, $t\hm\in\mathbb{R}$. Отсюда
несложно видеть, что $S_{2,0}\hm\eqd\sqrt{2}X$.

Односторонним строго устойчивым законам, сосредоточенным на
неотрицательной полуоси, соответствуют значения $\theta\hm=1$ 
и~$0\hm<\alpha\hm\le1$. Пары $\alpha\hm=1$, $\theta\hm=\pm1$ отвечают
распределениям, вы\-рож\-ден\-ным в~$\pm1$ соответственно. Остальные
устойчивые распределения абсолютно непрерывны. Явные выражения
устойчивых плотностей в~терминах элементарных функций отсутствуют за
четырьмя исключениями (нормальный закон ($\alpha\hm=2$, $\theta\hm=0$),
распределение Коши ($\alpha\hm=1$, $\theta\hm=0$), распределение Леви
($\alpha\hm=1/2$, $\theta\hm=1$) и~распределение, симметричное 
к~распределению Леви ($\alpha\hm=1/2$, $\theta\hm=-1$)). Выражения
устойчивых плотностей в~терминах функций Фокса (обобщенных
$G$-функ\-ций Мейера) можно найти в~\cite{Schneider1986, UchaikinZolotarev1999}.

Хорошо известно, что если $0\hm<\alpha\hm<2$, то ${\sf E}
|S_{\alpha,\theta}|^{\beta}\hm<\infty$ для любого
$\beta\in(0,\alpha)$, но моментов с.в.~$S_{\alpha,\theta}$ порядков
$\beta\hm>\alpha$ не существует (см., например,~\cite{Zolotarev1983}).
Несмотря на отсутствие явных выражений плотностей устойчивых
распределений в~терминах элементарных функций, можно показать~\cite{KorolevWeibull2016}, 
что для $0\hm<\beta\hm<\alpha\hm<2$
\begin{equation}
{\sf E}\left\vert S_{\alpha,0}\right\vert^{\beta}=\fr{2^{\beta}}{\sqrt{\pi}}\,
\fr{\Gamma\left(({\beta+1})/{2}\right)\Gamma\left(1-{\beta}/{\alpha}\right)}
{\Gamma\left({2}/{\beta}-1\right)}
\label{e4-kgz}
\end{equation}
и для $0<\beta<\alpha\hm< 1$
\begin{equation}
{\sf E}S_{\alpha,1}^{\beta}=
\fr{\Gamma\left(1-{\beta}/{\alpha}\right)}{\Gamma(1-\beta)}\,.
\label{e5-kgz}
\end{equation}

Символы $\eqd$ и~$\Longrightarrow$ будут соответственно обозначать
совпадение распределений и~сходимость по распределению.

\smallskip

\noindent
\textbf{Лемма~1}~\cite[теорема~3.3.1]{Zolotarev1983}. \textit{Пусть
$\alpha\hm\in(0,2]$, $\alpha'\hm\in(0,1]$. Тогда $S_{\alpha\alpha',0}\hm\eqd
S_{\alpha,0}S_{\alpha',1}^{1/\alpha}$, где с.в.\ в~правой части
независимы}.

\smallskip

\noindent
\textbf{Следствие~1.}\ Симметричное строго устойчивое распределение
с характеристическим показателем~$\alpha$ является масштабной смесью
нормальных законов, в~которой смешивающим служит одностороннее
строго устойчивое распределение с~характеристическим показателем
$\alpha/2$:
\begin{equation}
S_{\alpha,0}\eqd X\sqrt{2S_{\alpha/2,1}}\,, 
\label{e6-kgz}
\end{equation}
где с.в.\ в~правой части независимы.

\smallskip

Случайная величина, име\-ющая гам\-ма-рас\-пре\-де\-ле\-ние с~параметром формы $\nu\hm>0$ 
и~параметром масштаба $\lambda\hm>0$, будет обозначаться~$G_{\nu,\lambda}$,
\begin{equation*}
{\sf P}(G_{\nu,\lambda}<x)=\int\limits_{0}^{x}g(z;\nu,\lambda)\,dz\,,
\end{equation*}
где
$$
g(x;\nu,\lambda)=\fr{\lambda^\nu}{\Gamma(\nu)}\,x^{\nu-1}e^{-\lambda
x}\,,\enskip x\ge0\,,
$$
В этих обозначениях, очевидно, $G_{1,1}\hm\eqd W_1$.

Гамма-распределение~--- это частный случай обобщенных
гамма-распределений, введенных в~работе~\cite{Stacy1962} как единый
класс, одновременно содержащий гам\-ма-рас\-пре\-де\-ле\-ние и~распределение
Вейбулла. \textit{Обобщенное гам\-ма-рас\-пре\-де\-ле\-ние}~--- это абсолютно
непрерывное распределение, плотность которого имеет вид:
$$
\overline{g}(x;\nu,\alpha,\lambda)=\fr{|\alpha|\lambda^\nu}{\Gamma(\nu)}\,x^{\alpha
\nu-1}e^{-\lambda x^{\alpha}}\,,\enskip x\ge0\,,
$$
где $\alpha\in\mathbb{R}$, $\lambda\hm>0$, $\nu\hm>0$.

Случайная величина с~плот\-ностью $\overline{g}(x;\nu,\alpha,\lambda)$ будет
обозначаться $\overline{G}_{\nu,\alpha,\lambda}$. Легко видеть, что
$$
\overline{G}_{\nu,\alpha,\mu}\eqd G_{\nu,\mu}^{1/\alpha}\eqd
\mu^{-1/\alpha}G_{\nu,1}^{1/\alpha}
\eqd\mu^{-1/\alpha}\overline{G}_{\nu,\alpha,1}\,.
$$

Для с.в.\ с~\textit{распределением Вейбулла}, частным случаем
обобщенных гам\-ма-рас\-пре\-де\-ле\-ний, соответствующим плотности
$\overline{g}(x;1,\alpha,1)$ и~ф.р.\
$\left[1\hm-e^{-x^{\alpha}}\right]{\bf 1}(x\hm\ge0)$ с~$\alpha\hm>0$, будет
использовано особое обозначение~$W_{\alpha}$. Таким образом,
$G_{1,1}\hm\eqd W_1$. Очевидно, $W_1^{1/\alpha}\hm\eqd W_{\alpha}$.

Несложно убедиться, что если $\gamma\hm>0$ и~$\gamma'\hm>0$, то 
\begin{multline*}
{\sf P} \left(W_{\gamma'}^{1/\gamma}\ge x\right)=
{\sf P}\left(W_{\gamma'}\ge x^{\gamma}\right)=
e^{-x^{\gamma\gamma'}}={}\\
{}={\sf P}\left(W_{\gamma\gamma'}\ge x\right)\,,\enskip
x\ge 0\,,
\end{multline*}
 т.\,е.\ при любых $\gamma\hm>0$ и~$\gamma'\hm>0$
\begin{equation}
W_{\gamma\gamma'}\eqd W_{\gamma'}^{1/\gamma}\,.
\label{e7-kgz}
\end{equation}

В статье~\cite{Gleser1989} было показано, что каждое
гам\-ма-рас\-пре\-де\-ле\-ние с~параметром формы, не пре\-вос\-ходящим единицы,
является смешанным по\-ка\-зательным. Сформулируем этот результат в~виде\linebreak
сле\-ду\-ющей леммы.

\smallskip

\noindent
\textbf{Лемма~2}~\cite{Gleser1989}. \textit{Плотность гам\-ма-рас\-пре\-де\-ле\-ния
$g(x;\nu,\mu)$ с~$0\hm<\nu\hm<1$ может быть представлена в~виде}

\noindent
$$
g(x;\nu,\mu)=\int\limits_{0}^{\infty}ze^{-zx}p(z;\nu,\mu)\,dz\,,
$$

\vspace*{-2pt}

\noindent
\textit{где}

\noindent
\begin{equation}
p(z;\nu,\mu)=\fr{\mu^\nu}{\Gamma(1-\nu)\Gamma(\nu)}\,
\fr{\mathbf{1}(z\ge\mu)}{(z-\mu)^\nu z}\,.\label{e8-kgz}
\end{equation}
\textit{Более того, гам\-ма-рас\-пре\-де\-ле\-ние с~параметром формы $\nu\hm>1$ не может
быть представлено в~виде смешанного показательного закона.}

\smallskip

\noindent
\textbf{Лемма~3}~\cite{Korolev2017}. \textit{Для $\nu\hm\in(0,1)$ пусть
$G_{\nu,\,1}$ и~$G_{1-\nu,\,1}$~--- независимые гам\-ма-рас\-пре\-де\-лен\-ные
с.в. Пусть $\mu\hm>0$. Тогда плотность $p(z;\nu,\mu)$, определенная выражением}~(\ref{e8-kgz}), 
\textit{соответствует с.в.}
\begin{multline*}
Z_{\nu,\mu}=\fr{\mu(G_{\nu,\,1}+G_{1-\nu,\,1})}{G_{\nu,\,1}}\eqd{}\\
{}\eqd \mu
Z_{\nu,1}\eqd\mu\left(1+\fr{1-\nu}{\nu}\, Q_{1-\nu,\nu}\right)\,,
\end{multline*}
\textit{где $Q_{1-\nu,\nu}$~--- с.в.\ с~распределением Сне\-де\-ко\-ра--Фи\-ше\-ра,
соответствующим плотности}
\begin{multline*}
q(x;1-\nu,\nu)=\fr{(1-\nu)^{1-\nu}\nu^\nu}{\Gamma(1-\nu)\Gamma(\nu)}
\,\fr{1}{x^{\nu}[\nu+(1-\nu)x]},\\  x\ge0.
\end{multline*}


\smallskip

Несложно видеть, что $G_{\nu,1}\hm+G_{1-\nu,1}\hm\eqd W_1$. Однако
числитель и~знаменатель в~определении с.в.~$Z_{\nu,\mu}$ не
являются независимыми~с.в.

\smallskip

Фактически леммы~2 и~3 означают, что если $\nu\in(0,1)$, то
\begin{equation}
G_{\nu,\,\mu}\eqd W_1  Z_{\nu,\,\mu}^{-1}\,,\label{e9-kgz}
\end{equation}
где с.в.\ в~правой части независимы.

\smallskip

Следующее утверждение уже стало фольклором. Без претензий на
первенство его доказательство приведено в~\cite{KorolevZeifmanKMJ}
как упражнение.

\smallskip

\noindent
\textbf{Лемма~4.} \textit{При каждом $\delta\hm\in(0,1]$ распределение
Мит\-таг-Леф\-фле\-ра с~параметром~$\delta$ является масштабной смесью
одностороннего устойчивого закона, в~которой смешивающее
распределение~--- распределение Вейбулла с~параметром}
$\delta/2$$:$ 

\noindent
$$
M_{\delta}\eqd S_{\delta,1}W_{\delta}\eqd
S_{\delta,1}\sqrt{W_{\delta/2}}\,,
$$
 \textit{где с.в.\ в~правой части
независимы}.

\smallskip

Пусть $\rho\in(0,1)$. В~статье~\cite{Kozubowski1998} было показано,
что функция
\begin{equation}
f_{\rho}^{K}(x)=\fr{\sin(\pi\rho)}{\pi\rho[x^2+2x\cos(\pi\rho)+1]}\,,\enskip\!
 x\in(0,\infty),\!\!\!\label{e10-kgz}
\end{equation}
является плотностью вероятностей на $(0,\infty)$. Случайную величину 
с~плот\-ностью~(\ref{e10-kgz}) обозначим~$K_{\rho}$.

\smallskip

\noindent
\textbf{Лемма~5}~\cite{Kozubowski1998}. \textit{Пусть
$0\hm<\delta\hm<\delta'\hm\le1$ и~$\rho\hm=\delta/\delta'\hm<1$. Тогда
$M_{\delta}\hm\eqd M_{\delta'}K_{\rho}^{1/\delta}$, где с.в.\ в~правой
части независимы.}

\smallskip

В статье~\cite{KorolevZeifmanKMJ} было показано, что при любом
$\delta\hm\in(0,1)$

\noindent
\begin{equation}
K_{\delta}^{1/\delta}\eqd
\fr{S_{\delta,1}}{S'_{\delta,1}}\,,\label{e11-kgz}
\end{equation}

\vspace*{-1pt}

\noindent
где $S'_{\delta,1}\eqd S_{\delta,1}$ и~где с.в.\ в~правой части
независимы. Таким образом, при $\delta'\hm=1$ из леммы~5 вытекает

\smallskip

\noindent
\textbf{Следствие~2}~\cite{Kozubowski1998, KorolevZeifmanKMJ}. 
Пусть $0<\delta<1$. Тогда распределение Миттаг-Леффлера с~параметром
$\delta$ является смешанным показательным, т.\,е.\ справедливо представление:

\noindent
$$
M_{\delta}\eqd W_1 K_{\delta}^{1/\delta}\eqd
W_1 \fr{S_{\delta,1}}{S'_{\delta,1}},
$$

\vspace*{-2pt}

\noindent
где с.в.\ в~правой части независимы.

\smallskip

Пусть $0<\alpha\hm<\alpha'\hm\le2$. В~статье~\cite{KotzOstrovskii1996}
было показано, что функция
$$
f_{\alpha,\alpha'}^{Q}(x)= \fr{\alpha'\sin(\pi\alpha/\alpha')
x^{\alpha-1}}{\pi[1+x^{2\alpha}+2x^{\alpha}\cos(\pi\alpha/\alpha')]}\,,\enskip
 x>0\,,
$$
является плотностью вероятностей на $(0,\infty)$. Пусть
$Q_{\alpha,\alpha'}$~--- с.в.\ с~плот\-ностью
$f_{\alpha,\alpha'}^{Q}(x)$.

\smallskip

\noindent
\textbf{Лемма~6}~\cite{KotzOstrovskii1996}. \textit{Пусть
$0\hm<\alpha\hm<\alpha'\hm\le2$. Тогда $L_{\alpha}\hm\eqd
L_{\alpha'}Q_{\alpha,\alpha'}$, где с.в.\ в~правой части
независимы}.

\smallskip

При $\alpha'=2$ получаем

\smallskip

\noindent
\textbf{Следствие~3}~\cite{KotzOstrovskii1996}. Пусть
$0\hm<\alpha\hm<2$. Тогда распределение Линника с~параметром~$\alpha$
является масштабной смесью распределений Лапласа, соответствующих
плотности~(2): $L_{\alpha}\hm\eqd \Lambda Q_{\alpha,2}$, где с.в.\ 
в~правой части независимы.

\smallskip

В работе~\cite{Devroye1990} было доказано следующее утверждение.
Здесь оно уточнено с~учетом~(\ref{e7-kgz}).

\smallskip

\noindent
\textbf{Лемма~7}~\cite{Devroye1990}. \textit{При любом} $\alpha\hm\in(0,2]$
$$
L_{\alpha}\eqd S_{\alpha,0} W_1^{1/\alpha}\,,
$$ 
где с.в.\  \textit{в~правой части независимы}.

\smallskip

\noindent
\textbf{Лемма~8}~\cite{KorolevZeifmanKMJ}. \textit{Пусть $\alpha\hm\in(0,2]$,
$\alpha'\hm\in(0,1]$. Тогда}
$$
L_{\alpha\alpha'}\eqd S_{\alpha,0}M_{\alpha'}^{1/\alpha}\,.
$$

%\smallskip

\noindent
\textbf{Следствие~4}~\cite{KorolevZeifmanKMJ}.  При каждом
$\alpha\hm\in(0,2]$ распределение Линника с~параметром~$\alpha$
является масштабной смесью нормальных законов, в~которой сме\-ши\-ва\-ющее
распределение~---  распределение Мит\-таг-Леф\-фле\-ра с~параметром
$\alpha/2$:
\begin{equation}
L_{\alpha}\eqd X\sqrt{2M_{\alpha/2}}\,,
\label{e12-kgz}
\end{equation}
где с.в.\ в~правой части независимы.

\smallskip

\noindent
\textbf{Лемма~9}~\cite{KorolevZeifmanKMJ}. При каждом
$\alpha\hm\in(0,2]$ справедливо представление:
$$
L_{\alpha}\eqd \Lambda\sqrt{\fr{S_{\alpha/2,1}}{S'_{\alpha/2,1}}}\,,
$$
где с.в.\ в~правой части независимы.

\smallskip

В статье~\cite{KorolevZeifmanKMJ} показано, что если $S_{\alpha,1}$
и~$S'_{\alpha,1}$~--- независимые с.в.\ с~одним и~тем же
односторонним строго устойчивым распределением с~характеристическим
показателем $\alpha\hm\in(0,1)$, то $S_{\alpha,1}/S'_{\alpha,1}\hm\eqd
K_{\alpha}^{1/\alpha}\hm\eqd Q_{2\alpha,2}^2$, т.\,е.\ плот\-ность~$p_{\alpha}(x)$ 
отношения $S_{\alpha,1}/S'_{\alpha,1}$ имеет вид:
$$
p_{\alpha}(x)=f_{\alpha,1}^{Q}(x)=\fr{\sin(\pi\alpha)x^{\alpha-1}}
{\pi[1+x^{2\alpha}+2x^{\alpha}\cos(\pi\alpha)]}\,,\ 
 x>0\,.
$$

\smallskip

\noindent
\textbf{Лемма~10}~\cite{KorolevZeifmanKMJ}. \textit{При любом
$\delta\hm\in(0,1]$ распределение Мит\-таг-Леф\-фле\-ра с~параметром~$\delta$
является масштабной смесью полунормальных законов}:
$$
M_{\delta}\eqd
|X| \sqrt{2W_1}\, \fr{S_{\delta,1}}{S'_{\delta,1}},
$$
\textit{где с.в.\ в~правой части независимы}.

\section{Представления обобщенных распределений Миттаг-Леффлера 
и~Линника в~виде смесей}

Представленные здесь теоремы обобщают 
и~уточняют некоторые результаты работ~\cite{Pakes1998, LimTeo2009,
Mathai2010, MathaiHaubold2011, Joseetal}. Некоторые известные
результаты сформулируем в~виде лемм.

\smallskip

\noindent
\textbf{Лемма~11}~\cite{Devroye1990, Pakes1998}. \textit{Пусть
$\alpha\hm,\in(0,2]$, $\nu\hm>0$. Тогда} 
$$
L_{\alpha,\nu}\eqd S_{\alpha,0} G_{\nu,1}^{1/\alpha}\eqd
S_{\alpha,0}\overline{G}_{\nu,\alpha,1}\,.
$$


\smallskip

Из леммы~11 и~соотношения~(\ref{e4-kgz}) получаем

\smallskip

\noindent
\textbf{Следствие~5.} Для $0\hm<\beta\hm<\alpha\hm<2$
$$
{\sf E}\left\vert L_{\alpha,\nu}\right\vert^{\beta}=
\fr{2^{\beta}\Gamma(({\beta+1})/{2})\Gamma(1-{\beta}/{\alpha})
\Gamma(\nu+{\beta}/{\alpha})}{\sqrt{\pi}
\Gamma({2}/{\beta}-1)\Gamma(\nu)}\,.
$$

\smallskip

\noindent
\textbf{Лемма~12}~\cite{Mathai2010, MathaiHaubold2011, Joseetal}. 
\textit{Пусть $\delta\hm\in(0,1]$ и~$\nu\hm>0$. Тогда} 
$$
M_{\delta,\nu}\eqd S_{\delta,1}\overline{G}_{\nu,\delta,1}\,.
$$

\smallskip

Из леммы~12 и~соотношения~(\ref{e5-kgz}) получаем

\smallskip

\noindent
\textbf{Следствие~6.} Для $0<\beta<\delta<1$
$$
{\sf E}
M_{\delta,\nu}^{\beta}=\fr{\Gamma(1-{\beta}/{\delta})\Gamma
(\nu+{\beta}/{\delta})}{\Gamma(1-\beta)\Gamma(\nu)}\,.
$$

\smallskip

Из следствия~1 (см.\ соотношение~(\ref{e6-kgz})) вытекает, что для $\nu\hm>0$ 
и~$\alpha\hm\in(0,2]$
$$
L_{\alpha,\nu}\eqd X \sqrt{2S_{\alpha/2,1}}\,
G_{\nu,1}^{1/\alpha}\eqd
X \sqrt{2S_{\alpha/2,1} \overline{G}_{\nu,\alpha/2,1}}\,,
$$
т.\,е.\ обобщенное распределение Линника является масштабной смесью
нормальных законов. При этом согласно лемме~12 смешивающим
распределением служит обобщенное распределение Мит\-таг-Леф\-фле\-ра.
Таким образом, по аналогии со следствием~4 получаем следующее
утверждение.

\smallskip

\noindent
\textbf{Теорема~1.} \textit{Если $\alpha\hm\in(0,2]$ и~$\nu\hm>0$, то
$L_{\alpha,\nu}\eqd$\linebreak $\eqd X \sqrt{2M_{\alpha/2,\,\nu}}$, где с.в.\ 
в~правой части независимы}.

\smallskip

Пусть $\alpha\hm\in(0,2]$, $\alpha'\hm\in(0,1)$ и~$\nu\hm>0$. Используя леммы~1 и~11, 
получаем следующую цепочку соотношений:
\begin{multline*}
L_{\alpha\alpha',\,\nu}\eqd S_{\alpha\alpha',\,0}
G_{\nu,1}^{1/\alpha}\eqd S_{\alpha,\,0}
S_{\alpha',\,1}^{1/\alpha} G_{\nu,1}^{1/\alpha}\eqd{}\\
{}\eqd
L_{\alpha,\,\nu} S_{\alpha',\,1}^{1/\alpha}\,.
\end{multline*}
Следовательно, справедливо следующее утверждение.

\smallskip

\noindent
\textbf{Теорема~2.} \textit{Пусть $\alpha\hm\in(0,2]$, $\alpha'\hm\in(0,1)$ 
и~$\nu\hm>0$. Тогда обобщенное распределение Линника является масштабной
смесью обобщенных распределений Линника с~б$\acute{\mbox{о}}$льшим
характеристическим параметром}: 
$$
L_{\alpha\alpha',\,\nu}\eqd
L_{\alpha,\,\nu} S_{\alpha',\,1}^{1/\alpha}\,,
$$
 \textit{где с.в.\ 
в~правой части независимы}.

\smallskip

Пусть теперь $\nu\in(0,1]$. Из представления~(\ref{e9-kgz}) и~леммы~7 получаем
цепочку соотношений:
\begin{multline*}
L_{\alpha,\nu}\eqd S_{\alpha,0} G_{\nu,1}^{1/\alpha}\eqd
S_{\alpha,0} W_1^{1/\alpha} Z_{\nu,1}^{-1/\alpha}\eqd{}\\
{}\eqd
S_{\alpha,0} W_{\alpha} Z_{\nu,1}^{-1/\alpha}\eqd
L_{\alpha} Z_{\nu,1}^{-1/\alpha}\,,
\end{multline*}
из которой вытекает следующее утверждение, связывающее обобщенное 
и~<<обычное>> распределения Линника.

\smallskip

\noindent
\textbf{Теорема~3}. \textit{Если $\nu\hm\in(0,1)$ и~$\alpha\hm\in(0,2]$, то}
\begin{equation}
L_{\alpha,\nu}\eqd L_{\alpha} Z_{\nu,1}^{-1/\alpha}\,,\label{e13-kgz}
\end{equation}
\textit{где с.в.\ в~правой части независимы. Другими словами, при
$\nu\hm\in(0,1]$ и~$\alpha\hm\in(0,2]$ обобщенное распределение Линника
является масштабной смесью обычных распределений Линника.}

\smallskip

Из~(\ref{e13-kgz}) и~леммы~9 получаем следующее пред\-став\-ле\-ние обобщенного
распределения Линника в~виде масштабной смеси распределений Лапласа:
$$
L_{\alpha,\nu}\eqd \Lambda
Z_{\nu,1}^{-1/\alpha}\sqrt{\fr{S_{\alpha/2,1}}{S'_{\alpha/2,1}}}\,.
$$
Более того, из следствия~4 (см.\ соотношение~(\ref{e12-kgz})) вытекает, что при
$\nu\hm\in(0,1)$ и~$\alpha\hm\in(0,2]$
\begin{equation}
L_{\alpha,\nu}\eqd X
Z_{\nu,1}^{-1/\alpha}\sqrt{2M_{\alpha/2}}\,.\label{e14-kgz}
\end{equation}
Поскольку масштабные смеси нормальных законов идентифицируемы~\cite{Teicher1961}, 
из~(\ref{e14-kgz}) и~теоремы~1 получаем следующее
представление обобщенного распределения Мит\-таг-Леф\-фле\-ра в~виде
масштабной смеси <<обычных>> распределений Мит\-таг-Леф\-флера.

\smallskip

\noindent
\textbf{Теорема~4}. \textit{Пусть $\nu\hm\in(0,1)$ и~$\delta\hm\in(0,1]$. Тогда}
$$
M_{\delta,\nu}\eqd Z_{\nu,1}^{-1/\delta} M_{\delta}\,,
$$ \textit{где с.в.\  в~правой части независимы}.

\smallskip

Пусть $\delta\in(0,1]$. Из леммы~12 вытекает, что
$$
M_{\delta,\nu}\eqd S_{\delta,1} \overline{G}_{\nu,\delta,\,1}\eqd
S_{\delta,1} G_{\nu,1}^{1/\delta}\,.
$$

 Теперь есть все
инструменты, позволяющие получить аналог теоремы~2 для распределений
Мит\-таг-Леф\-флера.

Пусть $\alpha\in(0,2]$, $\alpha'\hm\in(0,1)$ и~$\nu\hm>0$. Из тео\-ремы~1
вытекает, что $L_{\alpha\alpha',\nu}\eqd
X\sqrt{2M_{\alpha\alpha'/2,\nu}}$ и~$L_{\alpha,\nu}\eqd
X\sqrt{2M_{\alpha/2,\nu}}$. Из теоремы~2 вытекает, что
\begin{multline*}
X\sqrt{2M_{\alpha\alpha'/2,\nu}}\eqd
L_{\alpha\alpha',\nu}\eqd{}\\
{}\eqd L_{\alpha,\nu}
S_{\alpha',1}^{1/\alpha}\eqd X\sqrt{2M_{\alpha/2,\nu}}
S_{\alpha',1}^{1/\alpha}.
\end{multline*}
Следовательно, в~силу идентифицируемости масштабных смесей
нормальных законов $M_{\alpha\alpha'/2,\nu}\hm\eqd
M_{\alpha/2,\nu} S_{\alpha',1}^{2/\alpha}$. Поэтому,
переобозначив $\alpha/2\hm=\delta$, $\alpha'\hm=\delta'$, получаем
следующий результат.

\smallskip

\noindent
\textbf{Теорема~5.} \textit{Пусть $\delta\hm\in(0,1]$, $\delta'\hm\in(0,1)$ 
и~$\nu\hm>0$. Тогда}
$$
M_{\delta\delta',\nu}\eqd M_{\delta,\nu}
S_{\delta',1}^{1/\delta}\,,
$$
\textit{где с.в.\ в~правой части независимы}.

\smallskip

Другими словами, любое обобщенное распределение Мит\-таг-Леф\-фле\-ра
является масштабной \mbox{смесью} обобщенных распределений Мит\-таг-Леф\-фле\-ра
с~б$\acute{\mbox{о}}$льшим характеристическим параметром.

\smallskip

Комбинируя утверждения теорем~4 и~5, по\-лу\-чаем

\smallskip

\noindent
\textbf{Следствие~7.} Пусть $\delta\hm\in(0,1]$, $\delta'\hm\in(0,1)$ 
и~$\nu\hm\in(0,1)$. Тогда
$$
M_{\delta\delta',\,\nu}\eqd M_{\delta}
\left(\fr{S_{\delta',\,1}}{Z_{\nu,\,1}}\right)^{1/\delta}\,,
$$
где с.в.\ в~правой части независимы.

\smallskip

Из теоремы~4, лемм~4 и~5 вытекает, что обобщенное распределение
Мит\-таг-Леф\-фле\-ра допускает представление в~виде смешанного
распределения Вейбулла.

\smallskip

\noindent
\textbf{Теорема~6.} \textit{Если $\nu\hm\in(0,1)$ и~$0\hm<\delta\hm<\delta'\hm\le1$,
то}

\noindent
$$
M_{\delta,\nu}\eqd W_{\delta'}
S_{\delta',1}\left(\fr{K_{\delta/\delta'}}{Z_{\nu,\,1}}\right)^{1/\delta}\,,
$$
\textit{где все с. в. в~правой части независимы.}

\smallskip

Из теоремы~4 и~леммы~10 вытекает, что обобщенное распределение
Мит\-таг-Леф\-фле\-ра допускает представление в~виде масштабной смеси
полунормальных законов.

\smallskip

\noindent
\textbf{Теорема~7}. \textit{Пусть $\nu\hm\in(0,1)$ и~$\delta\hm\in(0,1]$. Тогда}
$$
M_{\delta,\nu}\eqd |X|
\fr{\sqrt{2W_1}}{Z_{\nu,1}^{1/\delta}}\,
\fr{S_{\delta,1}}{S'_{\delta,1}},
$$
\textit{где с.в.\ в~правой части независимы.}

\vspace*{-6pt}

\section{Сходимость распределений экстремальных порядковых статистик 
в~выборках случайного объема к~обобщенному распределению Миттаг-Леффлера}

Хорошо известно, что при достаточно общих условиях
распределение Вейбулла может быть предельным для линейно
преобразованных экстремальных порядковых статистик. Этот факт вкупе
с теоремой~6 позволяет убедиться, что обобщенное распределение
Мит\-таг-Леф\-фле\-ра может служить предельным для экстремальных
порядковых статистик в~выборках случайного объема.

В книге~\cite{GnedenkoKorolev1996} предложено описывать эволюцию
неоднородных хаотических стохастических процессов при помощи моделей
вида обобщенных дважды стохастических пуассоновских процессов
(обобщенных процессов Кокса). В~соответствии с~таким\linebreak
 подходом поток
информативных событий, каж\-дое из которых генерирует очередное
наблюдение, описывается стохастическим точечным процессом~$P(U(t))$,
где $P(t)$, $t\hm\geq0$,~--- однородный\linebreak
 пуассоновский процесс с~единичной
интенсивностью, а~$U(t)$, $t\hm\geq0$,~--- независимый от~$P(t)$
случайный процесс, такой что $U(0)\hm=0$, ${\sf P}(U(t)\hm<\infty)\hm=1$ для
любого $t\hm>0$, траектории~$U(t)$ не убывают и~непрерывны справа.
Процесс~$P(U(t))$, $t\hm\geq0$, называется дважды стохастическим
пуассоновским процессом (процессом Кокса)~\cite{Grandell1976}.

В рамках такой модели при каждом~$t$ с.в.~$P(U(t))$ имеет смешанное
пуассоновское распределение. Для наглядности рассмотрим ситуацию 
с~дискретным временем~$t$: $U(t)\hm=U(n)\hm=U_n$, $n\hm\in\mathbb{N}$, где
$\{U_n\}_{n\ge1}$~--- неограниченно возрастающая последовательность
неотрицательных с.в.\ такая, что $U_{n+1}(\omega)\hm\ge U_{n}(\omega)$
для каждого $\omega\hm\in\Omega$, $n\hm\ge1$. При этом асимптотика
$n\to\infty$ может быть интерпретирована как то, что интенсивность
потока информативных событий неограниченно возрастает.

Из сделанных выше предположений вытекает, что с.в.~$U_n$ независима
от стандартного пуассоновского процесса~$P(t)$, $t\hm\ge0$. Для каждого
$n\hm\in\mathbb{N}$ положим $N_n\hm=P(U_n)$, $n\hm\ge1$. Очевидно, что так
определенная с.в.~$N_n$ имеет смешанное пуассоновское распределение
\begin{multline*}
{\sf P}(N_n=k)={\sf P}\left(P(U_n)=k\right)={}\\
{}=\int\limits_{0}^{\infty}
e^{-nz}\fr{(nz)^k}{k!}\,d{\sf P}(U_n<z)\,,\enskip
 k=0,1,\ldots
\end{multline*}

Пусть $X_1,X_2,\ldots $~--- независимые одинаково распределенные с.в.\ 
с~общей ф.р.\ $F(x)\hm={\sf P}(X_i\hm<x)$, $x\hm\in\mathbb{R}$, $i\hm\ge1$.
Обозначим $\mathrm{lext}(F)\hm=\inf\{x:\,F(x)>0\}$. Предположим, что
при\linebreak
 каждом $k\hm\in\mathbb{N}$ с.в.~$N_k$ независима от
последовательности $X_1,X_2,\ldots$ В~книге~\cite{KorolevSokolov2008} 
доказано сле\-ду\-ющее утверждение.

\smallskip

\noindent
\textbf{Лемма~13.} \textit{Предположим, что
существуют неограниченно возрастающая последовательность
положительных чисел $\{d_n\}_{n\ge1}$ и~неотрицательная с.в.~$U$
такие, что $U_n/d_n\hm\Longrightarrow U$. Также пусть $\mathrm{lext}
(F)\hm>-\infty$ и~ф.р.\ $A_F(x)\hm=F\left( \mathrm{lext}
(F)\hm-x^{-1}\right)$ удовлетворяет условию$:$ при каждом} $x\hm>0$

\noindent
\begin{equation}
\lim_{y\to\-\infty}\fr{A_F(yx)}{A_F(y)}=x^{-\delta'}\label{e15-kgz}
\end{equation}
\textit{для некоторого положительного числа~$\delta'$. Тогда существуют
числа~$a_n$ и~$b_n$ такие, что}
\begin{multline*}
{\sf P}\left(\min_{1\le j\le N_n}X_j-a_n<b_nx\right) \Longrightarrow{}\\
{} \Longrightarrow
\left[1-\int\limits_{0}^{\infty}e^{-u x^{\delta'}}d{\sf P}
(U<u)\right]\mathbf{1}(x\ge 0)\,.
\end{multline*}
\textit{При этом числа~$a_n$ и~$b_n$ можно определить как}
\begin{equation}
\left.
\begin{array}{l}
\hspace*{-2mm}a_n={\mathrm{lext}}(F)\,;\\[6pt]
  \hspace*{-2mm}b_n=\sup\left\{x:\ F(x)\le
d_n^{-1}\right\}-{\mathrm{lext}}(F)\,,\  n\ge1\,.
\end{array}\!\!
\right\}
\label{e16-kgz}
\end{equation}

\smallskip

\noindent
\textbf{Теорема~8.} \textit{Пусть $\nu\hm\in(0,1)$, $\delta\hm\in(0,1)$. Для
того чтобы существовали числа $a_n\hm\in\mathbb{R}$ и~$b_n\hm>0$ такие, что}
$$
\fr{1}{b_n}\left(\min\limits_{1\le j\le N_n}X_j-a_n\right)\Longrightarrow
M_{\delta,\nu}\,,
$$
\textit{достаточно, чтобы были выполнены следующие условия}:
\begin{enumerate}[(1)]
\item \textit{существует число $\delta'\hm\in(\delta,1]$ такое, что
ф.р.~$F$ принадлежит области $\min$-при\-тя\-же\-ния распределения
Вейбулла с~параметром формы $\delta'\hm\in(0,1]$, т.\,е.\ $\mathrm{lext}(F)
\hm>-\infty$ и~выполнено условие}~(\ref{e15-kgz});

\item \textit{существует неограниченно возрастающая
последовательность $\{d_n\}_{n\ge1}$ такая, что}
$U_n/d_n\hm\Longrightarrow S_{\delta',1}^{-\delta'}
\left(K_{\delta/\delta'}Z_{\nu,1}\right)^{\delta'/\delta}$.
\end{enumerate}

\textit{При этом числа~$a_n$ и~$b_n$ могут быть определены 
в~соответствии с}~(\ref{e16-kgz}).

\smallskip

\noindent
Д\,о\,к\,а\,з\,а\,т\,е\,л\,ь\,с\,т\,в\,о\,.\ \ Требуемое утверждение является
непосредственным следствием леммы~13 и~тео\-ре\-мы~6 с~учетом
соотношения $K_{\delta/\delta'}^{-1}\hm\eqd K_{\delta/\delta'}$,
вытекающего из~(\ref{e11-kgz}).

\vspace*{-6pt}

\section{Сходимость распределений максимальных случайных сумм 
к~обобщенному распределению Миттаг-Леффлера}

В этом разделе будет
показано, что обобщенное распределение Мит\-таг-Леф\-фле\-ра может быть
предельным для максимальных или минимальных\linebreak
 случайных сумм, а~также
абсолютных величин\linebreak случайных сумм независимых с.в.\ \textit{с~конечными\linebreak
дис\-пер\-си\-ями}. Основную роль будет играть теоре-\linebreak ма~7,
 уста\-нав\-ли\-ва\-ющая
возможность представления обобщен\-но\-го распределения Мит\-таг-Леф\-фле\-ра
в~виде масштабной смеси полунормальных распределений.

Рассмотрим независимые, не обязательно одинаково распределенные с.в.\
 $X_1,X_2,\ldots $ с~${\sf E}X_i\hm=0$ и~$0\hm<\sigma^2_i\hm=
 {\sf D}X_i\hm<\infty$, $i\hm\in\mathbb{N}$. Для $n\hm\ge1$ обозначим 
\begin{align*}
 \overline S^*_n&=\max\limits_{1\le i\le n}S^*_i;\enskip 
 \underline S^*_n=\min\limits_{1\le i\le
n}S^*_i;\\
B_n^2&=\sigma_1^2+\cdots+\sigma_n^2\,.
\end{align*}

Предположим, что с.в.\
 $X_1,X_2,\ldots $ удовлетворяют условию Линдеберга: для любого
$\tau\hm>0$
\begin{equation}
\lim\limits_{n\to\infty}\fr{1}{B^2_n}\sum\limits_{i=1}^{n}\int\limits_{|x|\ge\tau
B_n} x^2\,d{\sf P}(X_i<x)=0\,.\label{e17-kgz}
\end{equation}
Как известно, при таких условиях 
\begin{align*}
{\sf P}\big(\overline S^*_n<B_nx\big)&\Longrightarrow \Psi(x)\equiv{\sf P}
(|X|<x)=2\Phi(x)-1;\\
{\sf P}\big(\underline
S^*_n<B_nx\big)&\Longrightarrow 1-\Psi(-x)\,.
\end{align*}

Пусть $N_1,N_2,\ldots $~--- последовательность неотрицательных
целочисленных с.в.\ таких, что при каж\-дом $n\hm\in\mathbb{N}$ с.в.\
$N_n,X_1,X_2,\ldots $ независимы. Для $n\hm\in\mathbb{N}$ положим
\begin{align*}
S^*_{N_n}&=X_1+\cdots +X_{N_n}\,;\\
\overline S^*_{N_n}&= \max\limits_{1\le i\le N_n}S_i^*\,;\enskip 
\underline S^*_{N_n}\hm=\min\limits_{1\le i\le N_n}S_i^*
\end{align*}
(для определенности считаем, что $S^*_0\hm=\overline S^*_0\hm=\underline
S^*_0\hm=0$). 

Пусть $\{d_n\}_{n\ge1}$~--- неограниченно возрастающая
последовательность положительных чисел.

\smallskip

\noindent
\textbf{Лемма~14}~\cite{Korolev1994}. \textit{Предположим, что с.в.\
$X_1,X_2,\ldots$ и~$N_1,N_2,\ldots$ удовлетворяют приведенным выше
условиям. В частности, пусть выполнено условие Линдеберга}~(\ref{e17-kgz}).
\textit{Более того, пусть $N_n\pto\infty$. Тогда распределение нормированных
экстремальных случайных сумм и~абсолютных величин случайных сумм
сходятся к~некоторым распределениям, т.\,е.\ существуют с.в.~$Y$,
$\overline Y$ и~$\underline Y$ такие, что}
$$
\fr{\overline S^*_{N_n}}{d_n}\Longrightarrow \overline Y\,;\enskip 
\fr{\underline S^*_{N_n}}{d_n}\Longrightarrow \underline Y\,; \enskip
\fr{\left\vert S^*_{N_n}\right\vert}{d_n}\Longrightarrow |Y|
$$
\textit{тогда и~только тогда, когда существует неотрицательная с.в.~$U$
такая, что $d_n^{-2}B^2_{N_n}\hm\Longrightarrow U$. При этом}

\noindent
\begin{gather*}
{\sf P}\big(\overline Y<x\big)={\sf P}\left(|Y|<x\right)= {\sf E}
\Psi\left(\fr{x}{\sqrt{U}}\right)\,;\\ 
{\sf P}\big(\underline
Y<x\big) =1-{\sf E}\Psi\left(-\fr{x}{\sqrt{U}}\right),\enskip
x\in\mathbb{R}\,.
\end{gather*}

%\smallskip

Из леммы~14 и~теоремы~7 вытекает следующее утверждение.

\smallskip

\noindent
\textbf{Теорема~9.} \textit{Пусть $\delta\hm\in(0,1]$, $\nu\hm\in(0,1)$.
Предположим, что с.в.\ $X_1,X_2,\ldots$ и~$N_1,N_2,\ldots$
удовлетворяют приведенным выше условиям. В~частности, пусть
выполнено условие Линдеберга}~(\ref{e17-kgz}). \textit{Более того, пусть
$N_n\hm\pto\infty$. Тогда следующие утверждения эквивалентны}:
\begin{gather*}
\fr{\overline S^*_{N_n}}{d_n}\Longrightarrow M_{\delta,\nu}; \ \ 
\fr{\underline S^*_{N_n}}{d_n}\Longrightarrow
-M_{\delta,\nu};\ \ 
 \fr{|S^*_{N_n}|}{d_n}\Longrightarrow
M_{\delta,\nu};\\ 
\fr{B^2_{N_n}}{d^2_n}\Longrightarrow
\fr{2W_1}{Z_{\nu,1}^{2/\delta}}\left(\fr{S_{\delta,1}}{S'_{\delta,1}}\right)^2\,.
\end{gather*}

\vspace*{-12pt}


{\small\frenchspacing
 {%\baselineskip=10.8pt
 \addcontentsline{toc}{section}{References}
 \begin{thebibliography}{99}
\bibitem{KorolevZeifmanKorchagin}
\Au{Королев~В.\,Ю., Зейфман~А.\,И., Корчагин~А.\,Ю.} Несимметричные
дву\-сто\-ронние распределения Миттаг-Леффлера как предельные законы
для случайных сумм независимых случайных величин с~конечными
дисперсиями~// Статистические методы оценивания и~пpовеpки гипотез.~---
Пермь: Пермский гос. ун-т, 2016. 
Т.~27. С.~69--89.
%Информатика и~ее применения, 2016. Т.~10. Вып.~4. C.~21--33.

\bibitem{KorolevZeifman2016} 
\Au{Korolev~V.\,Yu., Zeifman~A.\,I.} A~note on mixture
representations for the Linnik and Mittag-Leffler distributions and
their applications~// J.~Math. Sci., 2017. Vol.~218. No.\,3. P.~314--327.

\bibitem{KorolevZeifmanKMJ} 
\Au{Korolev V.\,Yu., Zeifman A.\,I.} Convergence of statistics
constructed from samples with random sizes to the Linnik and
Mittag-Leffler distributions and their generalizations~// J.~Korean 
Stat. Soc., 2017. Vol.~46. No.\,2. P.~161--181.

\bibitem{KorolevGorsheninZeifman2018} 
\Au{Korolev~V.\,Yu., Gorshenin~A.\,K., Zeifman~A.\,I.} 
On mixture representations for the generalized Linnik distribution and their
applications in limit theorems~// arXiv, 2018.

\bibitem{MittnikRachev1993} 
\Au{Mittnik~S., Rachev~S.\,T.} Modeling asset returns with
alternative stable distributions~// Economet. Rev.,~1993. Vol.~12. P.~261--330.

\bibitem{Kotz2001} 
\Au{Kotz~S., Kozubowski~T.\,J., Podgorski~K.} The Laplace
distribution and generalizations: A~revisit with applications to communications, 
economics, engineering, and finance.~--- Boston, MA, USA: Birkhauser, 2001. 349~p.

\bibitem{GorenfloMainardi2006} 
\Au{Gorenflo R., Mainardi~F.} Continuous time random walk, Mittag-Leffler 
waiting time and fractional diffusion: Mathematical aspects~// Anomalous transport: 
Foundations and applications~/ Eds. R.~Klages, G.~Radons, I.\,M.\,Sokolov.~--- 
Weinheim, Germany: Wiley-VCH, 2008. P.~93--127.

\bibitem{Kilbas2014}  %8
\Au{Gorenflo~R., Kilbas~A.\,A., Mainardi~F., Rogosin~S.\,V.}
Mittag-Leffler functions, related topics and applications.~--- 
Berlin/New York: Springer, 2014. 443~p.



\bibitem{Joseetal} %9
\Au{Jose~K.\,K., Uma~P., Lekshmi~V.\,S., Haubold~H.\,J.} Generalized
Mittag-Leffler distributions and processes for applications in astrophysics and time 
series modeling~// Astrophysics Space, 2010. Iss.~202559. 
P.~79--92.

\bibitem{MathaiHaubold2011}  %10
\Au{Mathai A.\,M., Haubold~H.\,J.} Matrix-variate statistical
distributions and fractional calculus~// Fract. Calc.  Appl. Anal., 2011. 
Vol.~14. No.\,1. P.~138--155.

\bibitem{Pillai1985}  %11
\Au{Pillai R.\,N.} Semi-$\alpha$-Laplace distributions~//
Commun.  Stat. Theory, 1985. Vol.~14. P.~991--1000.

\bibitem{Linnik1953} %12
\Au{Линник Ю.\,В.} Линейные формы и~статистические критерии.  I, II~//
Украинский математический~ж., 1953.
Т.~5. Вып.~2. С.~207--243; Вып.~3. С.~247--290.



\bibitem{Devroye1990} %13
\Au{Devroye~L.} A~note on Linnik's distribution~// Stat. Probabil. Lett., 
1990. Vol.~9. P.~305--306.



\bibitem{Anderson1992} %14
\Au{Anderson~D.\,N.} A multivariate Linnik distribution~//
Stat.  Probabil. Lett., 1992. Vol.~14. P.~333--336.

\bibitem{Lin1994}  %15
\Au{Lin~G.\,D.} Characterizations of the Laplace and related
distributions via geometric compound~// Sankhya Ser.~A, 
1994. Vol.~56. P.~1--9.

\bibitem{KotzOstrovskiiHayfavi1995a}  %16
\Au{Kotz~S., Ostrovskii~I.\,V., Hayfavi~A.} Analytic and asymptotic properties 
of Linnik's probability densities, I~// J.~Math. Anal. Appl., 1995. Vol.~193. 
P.~353--371.

\bibitem{KotzOstrovskiiHayfavi1995b}  %17
\Au{Kotz~S., Ostrovskii~I.\,V., Hayfavi~A.} 
Analytic and asymptotic properties of Linnik's probability densities, II~// 
J.~Math. Anal.  Appl., 1995. Vol.~193. P.~497--521.

\bibitem{Jacquesetal1999}  %18
\Au{Jacques~C., R$\acute{\mbox{e}}$millard~B., Theodorescu~R.} Estimation of
Linnik law parameters~// Statistics Risk Modeling, 1999. Vol. ~17. No.\,3. P.~213--236.



\bibitem{KotzOstrovskii1996} %19
\Au{Kotz~S., Ostrovskii~I.\,V.} A~mixture representation of the
Linnik distribution~// Stat. Probabil. Lett., 1996. Vol.~26. P.~61--64.

\bibitem{Pakes1998}  %20
\Au{Pakes~A.\,G.} Mixture representations for symmetric generalized
Linnik laws~// Stat. Probabil. Lett., 1998. Vol.~37. P.~213--221.

\bibitem{Anderson1993} %21
\Au{Anderson~D.\,N., Arnold~B.\,C.} Linnik distributions and
processes~// J.~Appl. Probab., 1993. Vol.~30. P.~330--340.

\bibitem{Jayakumar1995}  %22
\Au{Jayakumar~K., Kalyanaraman~K., Pillai~R.\,N.} $\alpha$-Laplace processes~// 
Math. Comput. Model., 1995. Vol.~22. P.~109--116.

\bibitem{BaringhausGrubel1997}  %23
\Au{Baringhaus~L., Grubel~R.} On 
a~class of characterization problems for random convex combinations~// 
Ann. I.~Stat. Math., 1997. Vol.~49. P.~555--567.



\bibitem{Kozubowski1998} %24
\Au{Kozubowski~T.\,J.} Mixture representation of Linnik distribution
revisited~// Stat. Probabil. Lett., 1998. Vol.~38. P.~157--160.

\bibitem{Lin1998} %25
\Au{Lin G.\,D.} A~note on the Linnik distributions~// J.~Math. Anal. Appl., 1998. Vol.~217. P.~701--706.

\bibitem{Zolotarev1983} %26
\Au{Золотарев~В.\,М.} Одномерные устойчивые распределения.~--- 
Теория вероятностей и~математическая статистика сер.~---
М.: Наука, 1983. 304~c. 
%(Translation of Mathematical Monographs. Vol.~65).

\bibitem{Schneider1986}  %27
\Au{Schneider~W.\,R.} Stable distributions: Fox function
representation and generalization~//  Stochastic processes in classical and quantum
systems~/ Eds. S.~Albeverio, G.~Casati, D.~Merlini.~--- 
Berlin: Springer, 1986. P.~497--511.

\bibitem{UchaikinZolotarev1999} %28
\Au{Uchaikin~V.\,V., Zolotarev~V.\,M.} Chance and stability.~--- 
Utrecht: VSP, 1999. 596~p.

\bibitem{KorolevWeibull2016} 
\Au{Korolev~V.\,Yu.} Product representations for random variables with the
 Weibull distributions and their applications~// J.~Math. Sci., 2016. Vol.~218. No.\,3. P.~298--313.

\bibitem{Stacy1962} %30
\Au{Stacy E.\,W.} A generalization of the gamma distribution~//
Ann. Math. Stat., 1962. Vol.~33. P.~1187--1192.

\bibitem{Gleser1989} 
\Au{Gleser~L.\,J.} The gamma distribution as a mixture of
exponential distributions~// Am. Stat., 1989. Vol.~43. P.~115--117.

\bibitem{Korolev2017} 
\Au{Королев~В.\,Ю.} Аналоги теоремы Глезера для отрицательных биномиальных 
и~обобщенных гам\-ма-рас\-пре\-де\-ле\-ний и~некоторые их приложения~// 
Информатика и~её применения, 2017. Т.~11. Вып.~3. C.~2--17.

\bibitem{LimTeo2009} {\it Lim~S.\,C., Teo~L.\,P.} Analytic and asymptotic properties of
multivariate generalized Linnik's probability densities~// J.~Fourier Anal.
Appl., 2010. Vol.~16. Iss.~5. P.~715--747.

\bibitem{Mathai2010} 
\Au{Mathai~A.\,M.} Some properties of Mittag-Leffler functions and
matrix-variate analogues: A~statistical perspective~// Fract. Calc. 
Appl. Anal., 2010. Vol.~13. No.\,2. P.~113--132.

\bibitem{Teicher1961} 
\Au{Teicher~H.} Identifiability of mixtures~// Ann. Math. Stat., 1961. 
Vol.~32. P.~244--248.

\bibitem{GnedenkoKorolev1996}
\Au{Gnedenko~B.\,V., Korolev~V.\,Yu.} Random summation: Limit theorems and applications.~--- 
Boca Raton, FL, USA: CRC Press, 1996. 288~p.

\bibitem{Grandell1976} 
\Au{Grandell~J.} Doubly stochastic Poisson processes.~--- 
Lecture notes in mathematics book ser.~--- 
Berlin\,--\,Heidelberg\,--\,New York: Springer, 1976. 
Vol.~529. 244~p. 

\bibitem{KorolevSokolov2008}
\Au{Королев~В.\,Ю., Соколов~И.\,А.} Математические модели неоднородных 
потоков экстремальных событий.~--- М.: ТОРУС ПРЕСС, 2008. 192~c.

\bibitem{Korolev1994} 
\Au{Королев~В.\,Ю.} Сходимость случайных последовательностей с~независимыми
случайными индексами. I~// Теория вероятностей и~ее 
применения, 1994. Т.~39. №\,2. С.~313--333.
 \end{thebibliography}

 }
 }

\end{multicols}

\vspace*{-3pt}

\hfill{\small\textit{Поступила в~редакцию 15.10.18}}

%\vspace*{8pt}

%\pagebreak

\newpage

\vspace*{-28pt}

%\hrule

%\vspace*{2pt}

%\hrule

%\vspace*{-2pt}

\def\tit{NEW MIXTURE REPRESENTATIONS OF~THE~GENERALIZED MITTAG-LEFFLER DISTRIBUTION 
AND~THEIR APPLICATIONS}

\def\titkol{New mixture representations of~the~generalized Mittag-Leffler distribution 
and~their applications}

\def\aut{V.\,Yu.~Korolev$^{1,2,3}$, A.\,K.~Gorshenin$^{1,2}$, 
and~A.\,I.~Zeifman$^{2,4,5}$}

\def\autkol{V.\,Yu.~Korolev, A.\,K.~Gorshenin, and~A.\,I.~Zeifman}

\titel{\tit}{\aut}{\autkol}{\titkol}

\vspace*{-11pt}


\noindent
$^1$Faculty of Computational Mathematics and Cybernetics, M.\,V.~Lomonosov Moscow
State University, GSP-1,\linebreak
$\hphantom{^1}$Leninskie Gory, Moscow 119991, Russian Federation

\noindent
$^2$Institute of Informatics Problems, Federal Research Center 
``Computer Science and Control'' of the Russian\linebreak
$\hphantom{^1}$Academy of Sciences, 44-2~Vavilov Str., 
Moscow 119333, Russian Federation

\noindent
$^3$Hangzhou Dianzi University, Xiasha Higher Education Zone, Hangzhou 310018, China

\noindent
$^4$Vologda State University, 15~Lenin Str., Vologda 160000, Russian Federation

\noindent
$^5$Vologda Research Center of the Russian Academy of Sciences, 56-A~Gorky Str.,
Vologda 160001, Russian\linebreak
$\hphantom{^1}$Federation



\def\leftfootline{\small{\textbf{\thepage}
\hfill INFORMATIKA I EE PRIMENENIYA~--- INFORMATICS AND
APPLICATIONS\ \ \ 2018\ \ \ volume~12\ \ \ issue\ 4}
}%
 \def\rightfootline{\small{INFORMATIKA I EE PRIMENENIYA~---
INFORMATICS AND APPLICATIONS\ \ \ 2018\ \ \ volume~12\ \ \ issue\ 4
\hfill \textbf{\thepage}}}

\vspace*{6pt}


\Abste{The article provides new mixture represenations for the generalized 
Mittag-Leffler distribution. In particular, it is shown that for values of the 
``generalizing'' parameter not exceeding one, the generalized Mittag-Leffler 
distribution is a~scale mixture of the half-normal distribution laws, classic 
Mittag-Leffler distributions, or generalized Mittag-Leffler distributions with 
the larger values of the characteristic index. The explicit expressions for mixing 
quantities are given for all cases. The obtained representations allow proposing new 
algorithms for modeling random variables with the generalized Mittag-Leffler 
distribution and formulating new limit theorems in which such distributions appear 
as the limit ones.}


\KWE{generalized Mittag-Leffler distribution; scale mixture; generalized gamma distribution; 
half-normal distribution; stable distribution}




\DOI{10.14357/19922264180411}

\vspace*{-20pt}

\Ack
\noindent
The research is supported by the Russian Foundation for Basic Research 
(project~17-07-00717).


%\vspace*{6pt}

  \begin{multicols}{2}

\renewcommand{\bibname}{\protect\rmfamily References}
%\renewcommand{\bibname}{\large\protect\rm References}

{\small\frenchspacing
 {%\baselineskip=10.8pt
 \addcontentsline{toc}{section}{References}
 \begin{thebibliography}{99}
\bibitem{1-kgz}
\Aue{Korolev,~V.\,Yu., A.\,I.~Zeifman, and A.\,Yu.~Korchagin.}
 2016. Nesimmetrichnye dvustoronnie raspredeleniya Mittag-Lefflera 
 kak predel'nye zakony dlya sluchaynykh summ nezavisimykh sluchaynykh 
 velichin s~konechnymi dispersiyami [Nonsymmetric two-sided Mittag-Leffler 
 distributions as limit laws for random sums of independent random variables 
 with finite variances]. \textit{Statisticheskie metody otsenivaniya i~provepki gipotez}
 [Statistical methods for evaluating and testing hypothesis].
 Perm: Perm State University. 
 27:69--89. %Информатика и~ее применения, 2016. Т.~10. Вып.~4. C.~21--33.

\bibitem{2-kgz}
\Aue{Korolev,~V.\,Yu., and A.\,I.~Zeifman.} 2017. A~note on mixture
representations for the Linnik and Mittag-Leffler distributions and 
their applications. \textit{J.~Math. Sci.} 218(3):314--327.

\bibitem{3-kgz}
\Aue{Korolev,~V.\,Yu., and A.\,I.~Zeifman.} 2017. Convergence of statistics 
constructed from samples with random sizes to the Linnik and Mittag-Leffler 
distributions and their generalizations. \textit{J.~Korean Stat. Soc.} 
46(2):161--181.

\bibitem{4-kgz}
\Aue{Korolev,~V.\,Yu., A.\,K.~Gorshenin, and A.\,I.~Zeifman.} 2018. 
On mixture representations for the generalized Linnik distribution and their
applications in limit theorems. \textit{arXiv}.

\bibitem{5-kgz}
\Aue{Mittnik,~S., and S.\,T.~Rachev.} 
1993. Modeling asset returns with alternative stable distributions. 
\textit{Economet. Rev.} 12:261--330.

\bibitem{6-kgz}
\Aue{Kotz,~S., T.\,J.~Kozubowski, and K.~Podgorski.} 2001.
\textit{The Laplace
distribution and generalizations: A~revisit with applications to communications, 
economics, engineering, and finance}. Boston, MA: Birkhauser. 349~p.

\bibitem{7-kgz}
\Aue{Gorenflo, R., and F.~Mainardi.} 2008. 
{Continuous time random walk, Mittag-Leffler waiting time and fractional 
diffusion: Mathematical aspects}. 
\textit{Anomalous transport: Foundations and applications}.
Eds.\ R.~Klages, G.~Radons, and I.\,M.~Sokolov. 
 Weinheim, Germany: Wiley-VCH. 93--127.

\bibitem{8-kgz}
\Aue{Gorenflo,~R., A.\,A.~Kilbas, F.~Mainardi, and S.\,V.~Rogosin.} 2014.
\textit{Mittag-Leffler functions, related topics and applications}. 
Berlin--New York: Springer.  443~p.



\bibitem{10-kgz} %9
\Aue{Jose,~K.\,K., P.~Uma, V.\,S.~Lekshmi, and H.\,J.~Haubold.}  
2010. Generalized Mittag-Leffler distributions and processes 
for applications in astrophysics and time series modeling. 
\textit{Astrophysics Space} 202559:79--92.

\bibitem{9-kgz} %10
\Aue{Mathai,~A.\,M., and H.\,J.~Haubold.} 2011. Matrix-variate statistical 
distributions and fractional calculus. \textit{Fract. Calc. Appl. Anal.} 
14(1):138--155.

\bibitem{11-kgz}
\Aue{Pillai,~R.\,N.} 1985. Semi-$\alpha$-Laplace distributions.
\textit{Commun. Stat. Theory} 14:991--1000.

\bibitem{12-kgz}
\Aue{Linnik,~Yu.\,V.} 
1953. Lineynye formy i~statisticheskie kriterii.~I, II 
[Linear forms and statistical criteria.~I. II]. 
\textit{Ukr. Math.~J.} 5(2):207--243; 5(3):247--290.


\bibitem{17-kgz} %13
\Aue{Devroye,~L.} 1990. A~note on Linnik's distribution. 
\textit{Stat. Probabil. Lett.} 9:305--306.

\bibitem{16-kgz} %14
\Aue{Anderson,~D.\,N.} 1992. A~multivariate Linnik distribution.
\textit{Stat. Probabil. Lett.} 14:333--336.

\bibitem{15-kgz} %15
\Aue{Lin,~G.\,D.} 1994. Characterizations of the Laplace and related
distributions via geometric compound. \textit{Sankhya Ser.~A}
 56:1--9.


\bibitem{13-kgz} %16
\Aue{Kotz,~S., I.\,V.~Ostrovskii, and A.~Hayfavi.} 1995. 
Analytic and asymptotic properties of Linnik's probability densities,~I. 
\textit{J.~Math. Anal. Appl.} 193:353--371.

\bibitem{14-kgz} %17
\Aue{Kotz,~S., I.\,V.~Ostrovskii, and A.~Hayfavi.}
 1995. Analytic and asymptotic properties of Linnik's probability densities,~II. 
 \textit{J.~Math. Anal. Appl.} 193:497--521.


\bibitem{18-kgz} %18
\Aue{Jacques,~C., B.~R$\acute{\mbox{e}}$millard, and R.~Theodorescu.}
1999. Estimation of Linnik law parameters. 
\textit{Statistics Risk Modeling} 17(3):213--236.

\bibitem{20-kgz} %19
\Aue{Kotz,~S., and I.\,V.~Ostrovskii.} 1996. 
A~mixture representation of the Linnik distribution. 
\textit{Stat. Probabil. Lett.} 26:61--64.

\bibitem{19-kgz} %20
\Aue{Pakes,~A.\,G.} 1998. Mixture representations for symmetric generalized
Linnik laws. \textit{Stat. Probabil. Lett.} 37:213--221.



\bibitem{21-kgz} %21
\Aue{Anderson,~D.\,N., and B.\,C.~Arnold.} 1993. Linnik distributions and
processes. \textit{J.~Appl. Probab.} 30:330--340.

\bibitem{23-kgz} %22
\Aue{Jayakumar,~K., K.~Kalyanaraman, and R.\,N.~Pillai.}
 1995. $\alpha$-Laplace processes. \textit{Math. 
 Comput. Model.} 22:109--116.

\bibitem{22-kgz} %23
\Aue{Baringhaus,~L., and R.~Grubel.} 1997. 
On a~class of characterization problems for random convex combinations. 
\textit{Ann. I.~Stat. Math.} 49:555--567.



\bibitem{24-kgz}
\Aue{Kozubowski,~T.\,J.} 1998. Mixture representation of Linnik distribution. 
\textit{Stat. Probabil. Lett.} 38:157--160.

\bibitem{25-kgz}
\Aue{Lin, G.\,D.} 1998. A~note on the Linnik distributions. 
\textit{J.~Math. Anal. Appl.} 217:701--706.

\bibitem{26-kgz}
\Aue{Zolotarev,~V.\,M.} 1986. \textit{One-dimensional stable distributions}.
 Translation of mathematical monographs ser.
 Providence, RI:
 American Mathematical Society. Vol.~65. 284~p. 


\bibitem{27-kgz}
\Aue{Schneider,~W.\,R.} 1986. Stable distributions: Fox function
representation and generalization. 
\textit{Stochastic processes in classical and quantum
systems}. Eds. S.~Albeverio, G.~Casati, and D.~Merlini. 
Berlin: Springer. 497--511.

\bibitem{28-kgz}
\Aue{Uchaikin,~V.\,V., and V.\,M.~Zolotarev.} 1999. 
\textit{Chance and stability}. Utrecht: VSP. 596~p.

\bibitem{29-kgz}
\Aue{Korolev,~V.\,Yu.} 2016. Product representations for random variables with 
the Weibull distributions and their applications. 
\textit{J.~Math. Sci.} 218(3):298--313.

\bibitem{30-kgz}
\Aue{Stacy,~E.\,W.} 1962. A~generalization of the gamma distribution.
\textit{Ann. Math. Stat.} 33:1187--1192.

\bibitem{31-kgz}
\Aue{Gleser,~L.\,J.} 1989. The gamma distribution as a~mixture of
exponential distributions. \textit{Am. Stat.} 43:115--117.

\bibitem{32-kgz}
\Aue{Korolev,~V.\,Yu.} 2017. Analogi teoremy Glezera dlya ot\-ri\-tsa\-tel'\-nykh 
binomial'nykh i~obobshchennykh gamma-raspredeleniy i~nekotorye ikh prilozheniya 
[Analogs of Gleser's theorem for negative binomial and generalized
 gamma distributions and some their applications]. 
 \textit{Informatika i~ee Primeneniya~--- Inform. Appl.} 11(3):2--17.

\bibitem{33-kgz}
\Aue{Lim,~S.\,C., and L.\,P.~Teo.} 2010. Analytic and asymptotic properties 
of multivariate generalized Linnik's probability densities. 
\textit{J.~Fourier Anal. Appl.} 16(5):715--747.

\bibitem{34-kgz}
\Aue{Mathai,~A.\,M.} 2010. Some properties of Mittag-Leffler functions and
matrix-variate analogues: A~statistical perspective. 
\textit{Fract. Calc. Appl. Anal.} 13(2):113--132.

\bibitem{35-kgz}
\Aue{Teicher,~H.} 1961. Identifiability of mixtures. 
\textit{Ann. Math. Stat.} 32:244--248.

\bibitem{36-kgz}
\Aue{Gnedenko,~B.\,V., and V.\,Yu.~Korolev.} 1996. 
\textit{Random summation: Limit theorems and applications}. 
Boca Raton, FL: CRC Press. 288~p.

\bibitem{37-kgz}
\Aue{Grandell,~J.} 1976. \textit{Doubly stochastic poisson processes}. 
Lecture notes in mathematics book ser. Berlin\,--\,Heidelberg\,--\,New York: 
Springer.  Vol.~529. 244~p.

\bibitem{38-kgz}
\Aue{Korolev,~V.\,Yu., and I.\,A.~Sokolov.} 2008. \textit{Ma\-te\-ma\-ti\-che\-skie
modeli neodnorodnykh potokov ekstremal'nykh sobytiy}
[Mathematical 
models of nonhomogeneous flows of extremal events]. Moscow: TORUS PRESS. 192~p.  
%(in Russian)

\bibitem{39-kgz}
\Aue{Korolev,~V.\,Yu.} 1994. Convergence of random sequences with the
independent random indexes.~I. \textit{Theor. Probab. Appl.} 39(2):282--297.
\end{thebibliography}

 }
 }

\end{multicols}

\vspace*{-7pt}

\hfill{\small\textit{Received October 15, 2018}}

\vspace*{-16pt}

\Contr

\vspace*{-4pt}

\noindent
\textbf{Korolev Victor Yu.} (b.\ 1954)~--- 
Doctor of Science (PhD) in physics and
mathematics, professor, Head of Department, Faculty of Computational Mathematics 
and Cybernetics, M.\,V.~Lomonosov Moscow State University, GSP-1, Leninskie Gory, 
Moscow 119991, Russian Federation; leading scientist, 
Institute of Informatics Problems, Federal Research Center 
``Computer Science and Control'' of the Russian Academy of Sciences, 
44-2~Vavilov Str., Moscow 119333, Russian Federation; 
professor, Hangzhou Dianzi University, Xiasha Higher Education Zone, 
Hangzhou 310018, China; \mbox{vkorolev@cs.msu.ru}

\vspace*{1pt}

\noindent
\textbf{Gorshenin Andrey K.} (b.\ 1986)~--- Candidate of Science (PhD) in physics and
mathematics, associate professor, leading scientist, Institute of Informatics Problems,
Federal Research Center ``Computer Science and Control'' of the Russian Academy of
Sciences, 44-2~Vavilov Str., Moscow 119333, Russian Federation;  
leading scientist, Faculty of Computational Mathematics and Cybernetics, 
M.\,V.~Lomonosov Moscow State University, GSP-1, Leninskie Gory, Moscow 119991, 
Russian Federation; \mbox{agorshenin@frccsc.ru}

\vspace*{1pt}

\noindent
\textbf{Zeifman Alexander I.} (b.\ 1954)~--- 
Doctor of Science in physics and mathematics, professor, Head of Department, 
Vologda State University, 15~Lenin Str., Vologda 160000, Russian Federation; 
senior scientist, Institute of Informatics Problems, Federal Research Center 
``Computer Science and Control'' of the Russian Academy of Sciences, 
44-2~Vavilov Str.,Moscow 119333, Russian Federation; 
principal scientist, Vologda Research Center of the Russian Academy of Sciences, 
56-A~Gorky Str., Vologda 160001, Russian Federation; \mbox{a\_zeifman@mail.ru}
\label{end\stat}

\renewcommand{\bibname}{\protect\rm Литература}         %11
\def\stat{kor+dor}

\def\tit{О НЕРАВНОМЕРНЫХ ОЦЕНКАХ ТОЧНОСТИ НОРМАЛЬНОЙ АППРОКСИМАЦИИ 
ДЛЯ~РАСПРЕДЕЛЕНИЙ НЕКОТОРЫХ СЛУЧАЙНЫХ СУММ ПРИ~ОСЛАБЛЕННЫХ МОМЕНТНЫХ
УСЛОВИЯХ$^*$}

\def\titkol{О~неравномерных оценках точности нормальной аппроксимации для
распределений некоторых случайных сумм} % при ослабленных моментных условиях}

\def\aut{В.\,Ю.~Королев$^1$, А.\,В.~Дорофеева$^2$}

\def\autkol{В.\,Ю.~Королев, А.\,В.~Дорофеева}

\titel{\tit}{\aut}{\autkol}{\titkol}

\index{Королев В.\,Ю.}
\index{Дорофеева А.\,В.}
\index{Korolev V.\,Yu.}
\index{Dorofeeva A.\,V.}




{\renewcommand{\thefootnote}{\fnsymbol{footnote}} \footnotetext[1]
{Работа выполнена при поддержке РФФИ (проект 18-07-01405).}}


\renewcommand{\thefootnote}{\arabic{footnote}}
\footnotetext[1]{Факультет вычислительной математики 
и~кибернетики, Московский государственный университет им.\
М.\,В.~Ломоносова; Институт проб\-лем информатики Федерального
исследовательского центра <<Информатика и~управ\-ле\-ние>> Российской
академии наук; Hangzhou Dianzi University, China, \mbox{vkorolev@cs.msu.ru}}
\footnotetext[2]{Факультет вычислительной
математики и~кибернетики, Московский государственный университет
им.\ М.\,В.~Ломоносова, \mbox{alex.dorofeyeva@gmail.com}}

%\vspace*{-6pt}



\Abst{Представлены неравномерные оценки скорости
сходимости в~центральной предельной теореме для сумм случайного
числа независимых одинаково распределенных случайных величин для
случаев, когда индекс суммирования (число слагаемых в~сумме) имеет
биномиальное или пуассоновское распределение и~стохастически
независим от слагаемых. Рассматривается ситуация, 
в~которой доступна информация лишь о существовании моментов второго
порядка у~слагаемых. Указаны конкретные числовые значения абсолютных
констант, входящих в~оценки. Попутно анонсируется уточнение
абсолютной константы в~неравномерной оценке скорости сходимости 
в~центральной предельной теореме для сумм неслучайного числа
независимых одинаково распределенных случайных величин с~моментами
порядков не выше второго.}

\KW{центральная предельная теорема; нормальная
аппроксимация; случайная сумма; биномиальное распределение;
распределение Пуассона; теорема Пуассона}

\DOI{10.14357/19922264180412}
  
\vspace*{6pt}


\vskip 10pt plus 9pt minus 6pt

\thispagestyle{headings}

\begin{multicols}{2}

\label{st\stat}

\section{Введение}

Оценки точности нормальной аппроксимации для
распределений сумм случайных величин традиционно являются объектом
пристального внимания среди специалистов в~области тео\-рии
вероятностей, поскольку они играют важную роль во многих прикладных
задачах. Такие оценки помогают осознанно принимать решения об
адекватности или неадекватности нормальной модели для наблюдаемых
статистических закономерностей. При этом особый интерес представляет
ситуация, в~которой доступна лишь минимальная информация 
о~существовании моментов второго порядка у~слагаемых. Именно такой
случай и~рассматривается в~данной заметке.

Пусть $X_1,X_2,\ldots$~--- независимые случайные величины с~${\sf E}
X_i\hm=0$ и~$0\hm<{\sf E} X_i^2\hm\equiv\sigma_i^2\hm<\infty$, $i\hm=1,2,\ldots $
Существование моментов случайных\linebreak величин $X_1,X_2,\ldots$ порядков
выше второго не предполагается. Для $n\hm\in\mathbb{N}$ обозначим
$S_n\hm=X_1+\cdots +X_n$ и~$B_n^2\hm=\sigma_1^2+\cdots +\sigma_n^2$.
Стандартную нормальную функцию распределения обозначим~$\Phi(x)$:

\noindent
$$
\Phi(x)=\fr{1}{\sqrt{2\pi}}\int\limits_{-\infty}^{x}e^{-z^2/2}\,dz\,,\enskip
x\in\mathbb{R}\,.
$$
Обозначим
\begin{align*}
\Delta_n(x)&=\left\vert{\sf P}\left(S_n<xB_n\right)-\Phi(x)\right\vert\,;\\
\Delta_n&=\sup\limits_x\left\vert{\sf P}\left(S_n<xB_n\right)-\Phi(x)\right\vert\,.
\end{align*}
Всюду далее символ~$\mathbb{I}(A)$ будет обозначать индикаторную
функцию события~$A$. Для $\varepsilon\hm\in(0,\infty)$ обозначим:
\begin{align*}
L_n(\varepsilon)&=\fr{1}{B_n^2}\sum\limits_{i=1}^n{\sf E}
X_i^2\mathbb{I}\left(\left\vert X_i\right\vert \ge \varepsilon B_n\right)\,;\
L_n=L_n(1)\,;
\\
M_n(\varepsilon)&=\fr{1}{B_n^3}\sum\limits_{i=1}^n{\sf E}
\left\vert X_i\right\vert^3\mathbb{I}\left(\left\vert X_i\right\vert < 
\varepsilon B_n\right)\,; \\
&\hspace*{47mm} M_n=M_n(1)\,.
\end{align*}
Оценкам величины~$\Delta_n$ при указанных выше минимально возможных
моментных условиях посвящены работы~[1--13] (см.\ также книги~\cite{Petrov1972, 
Petrov1987}. В~частности, для любого
$\varepsilon\hm\in(0,\infty)$ справедлива оценка:
\begin{equation}
\Delta_n\le 1,86\left(L_n(\varepsilon)+M_n(\varepsilon)\right)\le
1{,}86\left(L_n(\varepsilon)+\varepsilon\right)\,.
\label{e1-kd}
\end{equation}
Детальная история уточнения верхних оценок величины~$\Delta_n$,
изобилующая интересными результатами и~курьезами, описана 
в~работах~\cite{KorolevDorofeevaLMJ, Shevtsova}, в~которых
подчеркнуто, что оценки типа~(\ref{e1-kd}) разумно считать \textit{естественными}, 
поскольку они связывают скорость сходимости 
в~центральной предельной тео\-ре\-ме с~критерием схо\-ди\-мости.

В данной работе сосредоточимся на верхних оценках величины~$\Delta_n(x)$. 
В~1979~г.\ В.\,В.~Петров~\cite{Petrov1979} показал,
что существует конечная положительная постоянная~$C$, гарантирующая
выполнение неравенства:
\begin{multline}
\Delta_n(x)\le C\sum\limits_{k=1}^n \left[
\fr{{\sf E} X_k^2\mathbf{1}\left(|X_k|\ge
(1+|x|)B_n\right)}{B_n^2(1+|x|)^2}+{}\right.\\
\left.{}+\fr{{\sf E}
|X_k|^3\mathbf{1}\left(|X_k|<(1+|x|)B_n\right)}{B_n^3(1+|x|)^3}\right]\,.
\label{e2-kd}
\end{multline}
В 2001 г.\ неравенство~(\ref{e2-kd}) было передоказано другим методом 
в~статье~\cite{ChenShao2001}. В~неcкольких работах предпринимались
попытки оценить значение константы~$C$ в~неравенстве~(\ref{e2-kd}). 
В~част\-ности, в~работах~\cite{TN2007, NT2007} была получена
оценка $C\hm\le 76{,}17$. Эта оценка была существенно уточнена в~работе~\cite{PopovDisser}, 
где было показано, что в~случае одинаково
распределенных слагаемых, рас\-смат\-ри\-ва\-емом в~на\-сто\-ящей статье,
константа не превосходит~39,25. Следующее утверждение содержит
уточненную оценку абсолютной константы.

\smallskip

\noindent
\textbf{Теорема~1.}\ \textit{Пусть $X_1,X_2,\ldots$~---
независимые одинаково распределенные случайные величины с~${\sf E} X_1\hm=0$
и~$0\hm<{\sf E} X_1^2\hm\equiv\sigma^2\hm<\infty$. Тогда для любого
$x\hm\in\mathbb{R}$ справедливо неравенство}:
\begin{multline*}
\Delta_n(x)\le 36{,}62\left[\fr{{\sf E}
X_1^2\mathbf{1}\left(|X_1|\ge(1+|x|)\sigma\sqrt{n}\right)}{\sigma^2(1+|x|)^2}+{}\right.\\
\left.{}+
\fr{{\sf E}
|X_1|^3\mathbf{1}\left(|X_1|<(1+|x|)\sigma\sqrt{n}\right)}
{\sigma^3\sqrt{n}(1+|x|)^3}\right]\,.
%\label{e3-kd}
\end{multline*}

%\smallskip

\noindent
Д\,о\,к\,а\,з\,а\,т\,е\,л\,ь\,с\,т\,в\,о\,.\ \  
Для уточнения абсолютной константы здесь
частично использовались методы,\linebreak
 описанные в~\cite{PopovDisser}, 
с~учетом текущих наилучших оценок констант в~неравенстве
Бер\-ри--Эс\-се\-ена~\cite{ShDAN}, его неравномерном аналоге~\cite{NSh} 
и~неравенстве~(\ref{e1-kd})~\cite{KorolevDorofeevaLMJ}. Подробное описание
алгоритма будет представлено в~одной из следующих статей.

\smallskip

Цель настоящей работы~--- распространить утверждение теоремы 1 на
случайные суммы, в~которых число слагаемых имеет биномиальное или
пуассоновское распределение. При этом будет существенно
использоваться подход, развитый в~работе~\cite{KorolevDorofeevaLMJ}.

\section{Неравномерные оценки для~биномиальных случайных~сумм}

Всюду далее рассматриваются независимые одинаково распределенные
случайные величины $X_1,X_2,\ldots$ с~${\sf E} X_i\hm=0$ и~$0\hm<{\sf E}
X_i^2\hm\equiv \sigma^2<\infty$. Пусть $p\hm\in(0,1]$~--- произвольно.
Пусть $\xi_1,\ldots,\xi_n$~--- независимые случайные величины, такие
что
$$
\xi_j=
\begin{cases}1 & \mbox{ с~вероятностью } p\,,\\
                    0 & \mbox{ с~вероятностью } 1-p\,,
      \end{cases}\enskip
       j=1,\ldots,n\,.
$$
Случайная величина $N_{n,p}\hm=\xi_1+\cdots+\xi_n$ может
интерпретироваться как число успехов в~схеме испытаний Бернулли 
с~вероятностью успеха~$p$. Cлучайная величина~$N_{n,p}$ имеет
биномиальное распределение с~параметрами~$n$ и~$p$:
$$
{\sf P}(N_{n,p}=k)=C_n^kp^k(1-p)^{n-k}\,,\enskip k=0,\ldots,n\,.
$$
Предположим, что при каждом $n\hm\in\mathbb{N}$ случайные величины
$N_{n,p},X_1,X_2,\ldots$ взаимно независимы. В~данном разделе
основным объектом изучения будут \textit{биномиальные случайные суммы}
вида
$$
S_{N_{n,p}}=X_1+\cdots+X_{N_{n,p}}.
$$
При этом если $N_{n,p}\hm=0$, то $S_{N_{n,p}}\hm=0$.

Для $j\in\mathbb{N}$ введем случайные величины~$\widetilde{X}_j$, полагая
$$
\widetilde{X}_j=
\begin{cases}X_j & \mbox{ с~вероятностью } p\,,\\
0 & \mbox{ с~вероятностью } 1-p\,.
\end{cases}
$$
Несложно видеть, что $\widetilde{X}_j \eqd \xi_jX_j$, где сомножители в~правой
части независимы (здесь и~далее символ~$\eqd$ обозначает совпадение
распределений).

Пусть $F(x)$~--- общая функция распределения случайных величин~$X_j$,
$E_0(x)$~--- функция распределения с~единственным единичным скачком 
в~нуле. Тогда, очевидно,
$$
{\sf P}\left(\widetilde{X}_j<x\right)=pF(x)+(1-p)E_0(x)\,,\enskip
x\in\mathbb{R}\,,\ j\in\mathbb{N}\,.
$$
При этом ${\sf E}\widetilde{X}_j\hm=0$,
\begin{equation}
{\sf D}\widetilde{X}_j={\sf E}\widetilde{X}_j^2=p\sigma^2\,.
\label{e4-kd}
\end{equation}


\smallskip

\noindent
\textbf{Лемма~1.}\ \textit{Для любых $n\hm\in\mathbb{N}$ и~$p_j\hm\in(0,1]$}
\begin{equation}
S_{N_{n,p}}\eqd \widetilde{X}_1+\cdots+\widetilde{X}_n\,,
\label{e5-kd}
\end{equation}
\textit{где случайные величины в~правой части}~(\ref{e5-kd}) \textit{не\-за\-ви\-симы.}

\smallskip

\noindent
Д\,о\,к\,а\,з\,а\,т\,е\,л\,ь\,с\,т\,в\,о\,.\ \
Доказательство представляет собой простое
упражнение на свойства характеристических функций.

\smallskip

С~учетом~(\ref{e4-kd}) и~(\ref{e5-kd}) легко заметить, что
$$
{\sf D}S_{N_{n,p}}=np\sigma^2\,.
$$
Обозначим
$$
\Delta_{n,p}(x)=\left \vert{\sf P}
\left(S_{N_{n,p}}<x\sigma\sqrt{np}\right)-\Phi(x)\right\vert\,.
$$

\smallskip

\noindent
\textbf{Теорема~2.}\ \textit{Для любых $n\hm\in\mathbb{N}$ и~$p\hm\in(0,1]$ справедливо
неравенство}:
\begin{multline*}
\Delta_{n,p}(x)\le 36{,}62\left[\fr{{\sf E}
X_1^2\mathbf{1}\left(|X_1|\ge(1+|x|)\sigma\sqrt{np}\right)}{\sigma^2(1+|x|)^2}+{}\right.
\hspace*{-1.22522pt}\\
\left.{}+
\fr{{\sf E}
|X_1|^3\mathbf{1}\left(|X_1|<(1+|x|)\sigma\sqrt{np}\right)}{\sigma^3\sqrt{np}(1+|x|)^3}
\right]\le{}
\\
{}\le \fr{36{,}62}{\sigma^2(1+|x|)^2}\cdot {\sf
E}X_1^2\min\left\{ 1,\,\fr{|X_1|}{\sigma\sqrt{np}(1+|x|)}\right\}.
\end{multline*}

\smallskip

\noindent
Д\,о\,к\,а\,з\,а\,т\,е\,л\,ь\,с\,т\,в\,о\,.\ \ Из леммы~1 и~соотношения~(\ref{e4-kd}) 
вытекает, что
$$
\Delta_{n,p}=\left\vert{\sf
P}\left(\widetilde{X}_1+\cdots+\widetilde{X}_n<x\sigma\sqrt{np}\right)-\Phi(x)\right\vert\,.
$$
Правую часть этого соотношения оценим с~по\-мощью теоремы~1 и~получим:
\begin{multline*}
\left\vert {\sf
P}\left(\widetilde{X}_1+\cdots+\widetilde{X}_n<
x\sigma\sqrt{\theta_n}\right)-\Phi(x)\right\vert \le{}
\\[1pt]
\le 36{,}62\left[\fr{{\sf
E}\widetilde{X}_1^2\mathbb{I}\left(|\widetilde{X}_1|\ge(1+|x|)\sigma\sqrt{np}\right)}
{p\sigma^2(1+|x|)^2}+{}\right.\\[1pt]
\left.{}+
\fr{{\sf E}|\widetilde{X}_1|^3\mathbb{I}\left(|\widetilde{X}_1|<(1+|x|)
\sigma\sqrt{np}\right)}{\sigma^3p^{3/2}\sqrt{n}(1+|x|)^3}\right]={}
\\[1pt]
{}\le 36{,}62\left[\fr{{\sf E}X_1^2\mathbb{I}\left(|X_1|
\ge(1+|x|)\sigma\sqrt{np}\right)}{\sigma^2(1+|x|)^2}+{}\right.\\[1pt]
\left.{}+
\fr{{\sf E}|X_1|^3\mathbb{I}\left(|X_1|<(1+|x|)\sigma\sqrt{np}\right)}
{\sigma^3\sqrt{np}(1+|x|)^3}\right]=
\\[1pt]
{}= \fr{36{,}62}{\sigma^2(1+|x|)^2}\,{\sf E}X_1^2\min
\left\{1,\,\fr{|X_1|}{\sigma\sqrt{np}(1+|x|)}\right\},
\end{multline*}
что и~требовалось доказать.

\section{Неравномерные оценки для~пуассоновских случайных~сумм}

Для $\lambda\hm>0$ пусть $N_{\lambda}$~--- случайная величина, имеющая
распределение Пуассона с~параметром~$\lambda$:
$$
{\sf P}\left(N_{\lambda}=k\right)=e^{-\lambda}\fr{\lambda^k}{k!}\,,\enskip
k\in\mathbb{N}\cup \{0\}\,.
$$
Предположим, что при каждом $\lambda\hm>0$ случайные величины
$N_{\lambda},X_1,X_2,\ldots$ независимы. Рассмотрим \textit{пуассоновскую случайную сумму}
$$
S_{N_{\lambda}}=X_1+\cdots+X_{N_{\lambda}}\,.
$$
Если $N_{\lambda}\hm=0$, то полагаем $S_{N_{\lambda}}\hm=0$. Несложно
убедиться, что ${\sf E}S_{N_{\lambda}}\hm=0$ и~${\sf D}S_{N_{\lambda}}\hm=
\lambda\sigma^2$. Точность нормальной
аппроксимации для рас\-пре\-де\-лений пуассоновских случайных сумм
изучалась многими\linebreak автора\-ми (см.\ исторические обзоры 
в~рабо\-тах~\cite{KorolevShevtsova, ShevtsovaPoisson}). Равномерные оценки
точности нормальной аппроксимации для распределений пуассоновских
случайных сумм при ослабленных моментных условиях получены 
в~статье~\cite{KorolevDorofeevaLMJ}. Насколько известно авторам,
неравномерные оценки для такой ситуации еще не проводились.

В данном разделе будут построены верхние оценки величины
$$
\Delta_{\lambda}(x)=\left\vert {\sf P}\left(S_{N_{\lambda}}<x\sigma
\sqrt{\lambda}\right)-\Phi(x)\right\vert\,.
$$
Зафиксируем $\lambda$ и~наряду с~$N_{\lambda}$ рассмотрим случайную
величину~$N_{n,p}$, имеющую биномиальное распределение с~\textit{произвольной} 
парой параметров~$n$ и~$p\hm\in(0,1]$, удовлетворяющей
условию $np=\lambda$. При этом
$$
{\sf D}S_{N_{\lambda}}={\sf D}S_{N_{n,p}}=\sigma^2\lambda=\sigma^2np\,.
$$
Поэтому для любого $x\hm\in\mathbb{R}$ по неравенству треугольника
\begin{multline}
\Delta_{\lambda}(x)\le{}\\
{}\le\Delta_{n,p}(x)+
\left\vert {\sf P}(S_{N_{\lambda}}<x)-{\sf P}(S_{N_{n,p}}<x)\right\vert\,.
\label{e6-kd}
\end{multline}
Первое слагаемое в~правой части~(\ref{e6-kd}) оценим с~помощью теоремы~2 и~получим:
\begin{multline}
\Delta_{n,p}(x)\le{}\\
{}\le \fr{36{,}62}{\sigma^2(1+|x|)^2}\, {\sf E}X_1^2\min
\left\{1,\,\fr{|X_1|}{\sigma\sqrt{\lambda}(1+|x|)}\right\}\,.
\label{e7-kd}
\end{multline}
Рассмотрим второе слагаемое в~правой части~(\ref{e6-kd}). Имеем:
\begin{multline}
\left\vert {\sf P}\left(S_{N_{\lambda}}<x\right)-{\sf P}
\left(S_{N_{n,p}}<x\right)\right\vert \le{}\\
{}\le\sup\limits_x
\sum\limits_{k=0}^{\infty}{\sf P}
\left(\sum\limits_{j=1}^kX_j<x\right)\left\vert 
{\sf P}(N_{n,p}=k)-{}\right.\\
\hspace*{-4.5mm}\left.{}-{\sf P}\left(N_{\lambda}=k\right)\right\vert \le
\sum\limits_{k=0}^{\infty}\left\vert {\sf P}(N_{n,p}=k)-{\sf P}
\left(N_{\lambda}=k\right)\right\vert\,.\!\!\!\!
\label{e8-kd}
\end{multline}
Правую часть неравенства~(\ref{e8-kd}) оценим с~помощью неравенства
Бар\-бу\-ра--Хол\-ла~\cite{BarbourHallPoisson}, в~соответствии с~которым
\begin{equation}
\sum\limits_{k=0}^{\infty}\left\vert {\sf P}(N_{n,p}=k)-{\sf P}
\left(N_{\lambda}=k\right)\right\vert \le 2p\min\{1,\lambda\}\,.
\label{e9-kd}
\end{equation}
Таким образом, из~(\ref{e6-kd}), (\ref{e7-kd}) и~(\ref{e9-kd}) 
вытекает, что \textit{для любых}~$n$ и~$p$, 
удовлетворяющих условию $np\hm=\lambda$, и~для любого
$x\hm\in\mathbb{R}$ справедливо неравенство:
\begin{multline}
\hspace*{-1.5mm}\Delta_{\lambda}(x)\le \fr{36{,}62}{\sigma^2(1+|x|)^2}\, {\sf E}
X_1^2\min\left\{\!1,\,\fr{|X_1|}{\sigma\sqrt{\lambda}(1+|x|)}\!\right\} +{}\hspace*{-0.44373pt}\\
{}+
\fr{2}{n}\,\lambda\min\{1,\lambda\}\,.
\label{e10-kd}
\end{multline}
Теперь, устремляя в~(\ref{e10-kd}) $n\hm\to\infty$, получаем окончательный
результат.

\smallskip

\noindent
\textbf{Теорема~3.}\ \textit{Для любых $\lambda\hm>0$ и~$x\hm\in\mathbb{R}$
справедлива оценка}:
$$
\Delta_{\lambda}(x)\le \fr{36{,}62}{\sigma^2(1+|x|)^2}\, {\sf E}
X_1^2\min\left\{1,\,\fr{|X_1|}{\sigma\sqrt{\lambda}(1+|x|)}\right\}\,.
$$

{\small\frenchspacing
 {%\baselineskip=10.8pt
 \addcontentsline{toc}{section}{References}
 \begin{thebibliography}{99}

\bibitem{Katz1963}
\Au{Katz~M.} Note on the Berry--Esseen theorem~// Ann. Math.
Stat., 1963. Vol.~39. No.\,4. P.~1348--1349.

\bibitem{Petrov1965}
\Au{Петров~В.\,В.} Одна оценка отклонения распределения суммы
независимых случайных величин от нормального закона~// Докл. АН
СССР, 1965. Т.~160. Вып.~5. С.~1013--1015.

\bibitem{Osipov1966}
\Au{Осипов~Л.\,В.} Уточнение теоремы Линдеберга~// Теория
вероятностей и~ее применения, 1966. Т.~11. Вып.~2. С.~339--342.

\bibitem{Feller1968}
\Au{Feller~W.} On the Berry--Esseen theorem~// Z.~Wahrscheinlichkeit., 1968. Bd.~10. S.~261--268.

\bibitem{Paditz1980}
\Au{Paditz~L.} Bemerkungen zu einer Fehlerabsch$\ddot{\mbox{a}}$tzung im
zentralen Grenzwertsatz~// Wiss. Z.~Hochsch. Verkehrswesen
Friedrich List Dres., 1980. Vol.~27. No.\,4. P.~829--837.

\bibitem{Paditz1984} %6
\Au{Paditz~L.} On error-estimates in the central limit theorem for
generalized linear discounting~// Math. Operationsforsch. Stat. 
Ser. Stat., 1984. Vol.~15. No.~4. P.~601--610.

\bibitem{BarbourHall1984} %7
\Au{Barbour~A.\,D., Hall~P.} Stein's method and the Berry--Esseen
theorem~// Aust. J.~Stat., 1984. Vol.~26.
P.~8--15.

\bibitem{Paditz1986} %8
{\it Paditz~L.} $\ddot{\mbox{U}}$ber eine Fehlerabsch$\ddot{\mbox{a}}$tzung im zentralen
Grenzwertsatz~// Wiss. Z. Hochsch. Verkehrswesen
Friedrich List Dres., 1986. Vol.~33. No.\,2. P.~399--404.



\bibitem{ChenShao2001} %9
\Au{Chen~L.\,H.\,Y., Shao~Q.\,M.} A~non-uniform Berry--Esseen bound
via Stein's method~// Probab. Theory Rel., 2001.
Vol.~120. P.~236--254.

\bibitem{KP2011_3}
\Au{Королев~В.\,Ю., Попов~С.\,В.} Уточнение оценок ско\-рости
сходимости в~центральной предельной теореме при отсутствии моментов
порядков, больших второго~// Тео\-рия вероятностей и~ее применения,
2011. Т.~56. Вып.~4. С.~797--805.

\bibitem{KorolevPopovDAN}
\Au{Королев~В.\,Ю., Попов~С.\,В.} Уточнение оценок ско\-рости
сходимости в~центральной предельной тео\-ре\-ме при ослабленных
моментных условиях~// Докл. РАН, 2012. Т.~445. Вып.~3. С.~265--270.

\bibitem{PopovDisser}
\Au{Попов~С.\,В.} Оценки скорости сходимости в~центральной
предельной теореме при ослабленных моментных условиях: Дис.\ \ldots\ канд.
физ.-мат. наук.~--- М.: МГУ, 2012.

\bibitem{KorolevDorofeevaLMJ}
\Au{Korolev~V., Dorofeeva~A.} Bounds of the accuracy of the normal
approximation to the distributions of random sums under relaxed
moment conditions~//  Lith. Math.~J., 2017. Vol.~57. No.\,1. P.~38--58.

\bibitem{Petrov1972}
\Au{Петров.~В.\,В.} Суммы независимых случайных величин.~--- М.:
Наука, 1972. 416~с.

\bibitem{Petrov1987}
\Au{Петров~В.\,В.} Предельные теоремы для сумм независимых
случайных величин.~--- М.: Наука, 1987. 320~с.

\bibitem{Shevtsova}
\Au{Шевцова~И.\,Г.} Моментное неравенство с~применением к~оценкам
скорости сходимости в~глобальной ЦПТ для пуас\-сон-би\-но\-ми\-аль\-ных
случайных сумм~// Тео\-рия вероятностей и~ее применения, 2017. Т.~62.
Вып.~2. С.~345--364.

\bibitem{Petrov1979}
\Au{Петров~В.\,В.} Одна предельная теорема для сумм независимых
неодинаково распределенных случайных величин~// Записки научных
семинаров ЛОМИ, 1979. Т.~85. С.~188--192.

\bibitem{TN2007}
\Au{Thongtha~P., Neammanee~K.} Refinement of the constants in the
non-uniform version of the Berry--Esseen theorem~// Thai J.~Math., 2007. Vol.~5. P.~1--13.

\bibitem{NT2007} %19
\Au{Neammanee~K., Thongtha~P.} Improvement of the non-uniform
version of the Berry--Esseen inequality via Paditz--Shiganov
theorems~// J.~Inequalities Pure  \mbox{Appl.} Math.,
2007. Vol.~8. No.\,4. Art.~92.

\bibitem{ShDAN}
\Au{Шевцова И.\,Г.} Об абсолютных константах в~неравенствах
Бер\-ри--Эс\-се\-ена~//
Докл.\ РАН, 2014. Т.~456. №\,6. С.~650--654.

\bibitem{NSh} %21
\Au{Нефедова~Ю.\,С., Шевцова~И.\,Г.} О~неравномерных
оценках ско\-рости схо\-ди\-мости в~цент\-раль\-ной предельной
%\linebreak\vspace*{-12pt}
%\pagebreak
%\noindent
теореме~// Тео\-рия вероятностей и~ее применения, 2012. Т.~57. №\,1. С.~62--97.

\pagebreak

\bibitem{KorolevShevtsova}
\Au{Korolev~V.\,Yu., Shevtsova~I.\,G.} An improvement of the
Berry--Esseen inequality with applications to Poisson and mixed
Poisson random sums~// Scand. Actuar.~J., 2012. No.\,2.
P.~81--105.

\bibitem{ShevtsovaPoisson}
\Au{Шевцова И.\,Г.} О~точ\-ности нормальной аппроксимации для
обобщенных пуассоновских распределений~// Теория вероятностей и~ее
применения, 2013.
Т.~58. №\,1. С.~152--176.

\bibitem{BarbourHallPoisson}
\Au{Barbour~A.\,D., Hall~P.} On the rate of Poisson convergence~// 
Math. Proc. Cambridge,
1984. Vol.~95. P.~473--480.
 \end{thebibliography}

 }
 }

\end{multicols}

\vspace*{-3pt}

\hfill{\small\textit{Поступила в~редакцию 15.10.18}}

\vspace*{8pt}

%\pagebreak

%\newpage

%\vspace*{-28pt}

\hrule

\vspace*{2pt}

\hrule

%\vspace*{-2pt}

\def\tit{ON NONUNIFORM ESTIMATES OF~ACCURACY OF~NORMAL
APPROXIMATION FOR DISTRIBUTIONS OF~SOME RANDOM SUMS UNDER~RELAXED
MOMENT CONDITIONS}

\def\titkol{On nonuniform estimates of~accuracy of~normal
approximation for distributions of~some random sums} % under~relaxed moment conditions}

\def\aut{V.\,Yu.~Korolev$^{1,2,3}$ and~A.\,V.~Dorofeeva$^1$}

\def\autkol{V.\,Yu.~Korolev and~A.\,V.~Dorofeeva}

\titel{\tit}{\aut}{\autkol}{\titkol}

\vspace*{-11pt}


\noindent
$^1$Faculty of Computational Mathematics
and Cybernetics, M.\,V.~Lomonosov Moscow State University, 
1-52~Lenin-\linebreak
$\hphantom{^1}$skiye Gory, GSP-1, Moscow 119991, Russian Federation

\noindent
$^2$Institute
of Informatics Problems, Federal Research Center ``Computer Science
and Control'' of the Russian\linebreak
 $\hphantom{^1}$Academy of Sciences, 44-2~Vavilov Str.,
Moscow 119333, Russian Federation

\noindent
$^3$Hangzhou Dianzi University, Xiasha Higher Education Zone, Hangzhou 310018, 
China


\def\leftfootline{\small{\textbf{\thepage}
\hfill INFORMATIKA I EE PRIMENENIYA~--- INFORMATICS AND
APPLICATIONS\ \ \ 2018\ \ \ volume~12\ \ \ issue\ 4}
}%
 \def\rightfootline{\small{INFORMATIKA I EE PRIMENENIYA~---
INFORMATICS AND APPLICATIONS\ \ \ 2018\ \ \ volume~12\ \ \ issue\ 4
\hfill \textbf{\thepage}}}

\vspace*{6pt}



\Abste{Nonuniform estimates are
presented for the rate of convergence in the central limit theorem
for sums of a random number of independent identically distributed
random variables. Two cases are studied in which the summation index
(the number of summands in the sum) has the binomial or Poisson
distribution. The index is assumed to be independent of the
summands. The situation is considered where the information that
only the second moments of the summands exist is available. Particular numerical
values of the absolute constants are presented explicitly. Also, the
sharpening of the absolute constant in the nonuniform estimate of
the rate of convergence in the central limit theorem for sums of 
a~nonrandom number of independent identically distributed random
variables is announced for the case where the summands possess only
second moments.}

\smallskip

\KWE{central limit theorem; normal approximation; random
sum; binomial distribution; Poisson distribution; Poisson theorem}


\DOI{10.14357/19922264180412}

\vspace*{-14pt}

\Ack
\noindent
This work was financially supported by the Russian Foundation for 
Basic Research (grant No.\,118-07-01405).


%\vspace*{6pt}

  \begin{multicols}{2}

\renewcommand{\bibname}{\protect\rmfamily References}
%\renewcommand{\bibname}{\large\protect\rm References}

{\small\frenchspacing
 {%\baselineskip=10.8pt
 \addcontentsline{toc}{section}{References}
 \begin{thebibliography}{99}

\bibitem{1-kd}
\Aue{Katz,~M.} 1963. Note on the Berry--Esseen theorem. 
\textit{Ann. Math. Stat.} 39(4):1348--1349.

\bibitem{2-kd}
\Aue{Petrov,~V.\,V.} 1965. Odna otsenka otkloneniya raspredeleniya summy
nezavisimykh sluchaynykh velichin ot normal'nogo zakona
[One estimate of the deviation of distribution of the sum of independent
random variables from the normal law]. 
\textit{Sov. Math.} 160(5):1013--1015.

\bibitem{3-kd}
\Aue{Osipov,~L.\,V.} 1966. Refinement of Lindeberg's theorem. 
\textit{Theor. Probab. Appl.} 11(2):299--302.

\bibitem{4-kd}
\Aue{Feller,~W.} 1968. On the Berry--Esseen theorem. 
\textit{Z.~Wahrscheinlichkeit.} 10:261--268.

\bibitem{5-kd}
\Aue{Paditz,~L.} 1980. Bemerkungen zu einer Fehlerabsch$\ddot{\mbox{a}}$tzung 
im zentralen Grenzwertsatz. \textit{Wiss. Z.~Hochsch. 
 Verkehrswesen Friedrich List Dres.} 27(4):829--837.

\bibitem{6-kd}
\Aue{Paditz,~L.} 1984. On error-estimates in the central limit theorem for
generalized linear discounting. \textit{Math. Operationsforsch. 
Stat. Ser. Stat.} 15(4):601--610.



\bibitem{8-kd} %7
\Aue{Barbour,~A.\,D., and P.~Hall.} 1984. Stein's method and the Berry--Esseen
theorem. \textit{Aust. J.~Stat}. 26:8--15.

\bibitem{7-kd} %8
\Aue{Paditz,~L.} 1986. $\ddot{\mbox{U}}$ber eine Fehlerabsch{\"a}tzung im zentralen
Grenzwertsatz. \textit{Wiss. Z.~Hochsch. Verkehrswesen
Friedrich List Dres.} 33(2):399--404.

\bibitem{9-kd}
\Aue{Chen,~L.\,H.\,Y., and Q.\,M.~Shao.} 2001. A~non-uniform Berry--Esseen bound via
Stein's method. \textit{Probab. Theory Rel.} 120:236--254.

\bibitem{10-kd}
\Aue{Korolev,~V.\,Yu., and S.\,V.~Popov.} 
2012. An improvement of convergence rate estimates 
in the central limit theorem under absence of moments higher than the
second. 
\textit{Theor. Probab. Appl.} 56(4):682--691.

\bibitem{11-kd}
\Aue{Korolev,~V.\,Yu., and S.\,V.~Popov.} 2012. 
Improvement of convergence rate estimates in the 
central limit theorem under weakened moment conditions.
\textit{Dokl. Math.} 86(1):506--511.

\bibitem{12-kd}
\Aue{Popov,~S.\,V.} 2012. Utochneniye ocnok skorosti skhodimosti 
v~tsentral'noy predel'noy teoreme pri  oslablennykh momentnykh usloviyakh
[Improvement of convergence rate estimates in the 
central limit theorem under weakened moment conditions]. 
 Moscow: MSU. PhD Diss.

\bibitem{13-kd}
\Aue{Korolev,~V., and A.~Dorofeeva.} 2017. Bounds of the accuracy of the
normal approximation to the distributions of random sums under
relaxed moment conditions. \textit{Lith. Math.~J.~} 57(1):38--58.

\bibitem{14-kd}
\Aue{Petrov,~V.\,V.} 1972. 
\textit{Summy nezavisimykh sluchaynykh velichin} 
[Sums of independent random variables]. Moscow: Nauka. 416~p.

\bibitem{15-kd}
\Aue{Petrov,~V.\,V.} 1987. 
\textit{Predel'nye teoremy dlya summ ne\-za\-vi\-si\-mykh sluchaynykh velichin} 
[Limit theorems for sums of independent random variables]. Moscow: Nauka. 320~p.

\bibitem{16-kd}
\Aue{Schevtsova,~I.\,G.} 2018. 
A~moment inequality with application to convergence rate estimates 
in the global CLT for Poisson-binomial random sums. 
\textit{Theor. Probab. Appl.} 62(2):278--294.

\bibitem{17-kd}
\Aue{Petrov,~V.\,V.} 1979. Odna predel'naya teorema dlya summ nezavisimykh
neodinakovo raspredelennykh sluchaynykh velichin 
[One limit theorem for sums of independent
unequally distributed random variables]. \textit{J.~Mathematical Sciences} 85:188--192.

\bibitem{18-kd}
\Aue{Thongtha,~P., and K.~Neammanee.} 2007. Refinement of the constants in the
non-uniform version of the Berry--Esseen theorem. \textit{Thai J.~Math.} 5:1--13.

\bibitem{19-kd}
\Aue{Neammanee,~K., and P.~Thongtha.} 2007. Improvement of the non-uniform version
of the Berry--Esseen inequality via Paditz--Shiganov theorems.
\textit{J.~Inequalities  Pure  Appl. Math.} 8(4): 92.

\bibitem{20-kd}
\Aue{Shevtsova,~I.\,G.} 2014. On the absolute constants in the Berry--Esseen-type
inequalities. \textit{Dokl. Math.} 89(3):378--381.

\bibitem{21-kd}
\Aue{Nefedova,~Yu.\,S., and I.\,G.~Shevtsova.} 2013. On nonuniform convergence rate
estimates in the central limit theorem. \textit{Theor. Probab. Appl.} 57(1):28--59.

\bibitem{22-kd}
\Aue{Korolev,~V.\,Yu., and  I.\,G.~Shevtsova.} 2012. An improvement of the
Berry--Esseen inequality with applications to Poisson and mixed
Poisson random sums. \textit{Scand. Actuar.~J.} 2:81--105.

\bibitem{23-kd}
\Aue{Shevtsova,~I.\,G.} 2014. On the accuracy of the normal approximation to
compound Poisson distributions. \textit{Theor. Probab. Appl.} 58(1):138--158.

\bibitem{24-kd}
\Aue{Barbour,~A.\,D., and P.~Hall.} 1984. On the rate of Poisson convergence.
\textit{Math. Proc. Cambridge} 95:473--480.

\end{thebibliography}

 }
 }

\end{multicols}

\vspace*{-6pt}

\hfill{\small\textit{Received October 15, 2018}}

%\pagebreak

%\vspace*{-18pt}

\Contr

\noindent
\textbf{Korolev Victor Yu.} (b.\ 1954)~--- Doctor of Science 
in physics and mathematics, professor, Head of the Department of Mathematical 
Statistics, Faculty of Computational Mathematics and Cybernetics, M.\,V.~Lomonosov 
Moscow State University, 1-52~Leninskiye Gory, GSP-1, Moscow 119991, 
Russian Federation; leading scientist, Institute of Informatics Problems, 
Federal Research Center ``Computer Science and Control'' 
of the Russian Academy of Sciences, 44-2~Vavilov Str., Moscow 119333, 
Russian Federation; professor, Hangzhou Dianzi University, Xiasha Higher 
Education Zone, Hangzhou 310018, China; \mbox{vkorolev@cs.msu.su}

\vspace*{3pt}

\noindent
\textbf{Dorofeeva Alexandra V.} (b.\ 1991)~--- 
PhD student, Faculty of Computational Mathematics and Cybernetics, 
M.\,V.~Lomonosov Moscow State University, 1-52~Leninskiye Gory, GSP-1, Moscow 119991, 
Russian Federation; \mbox{alex.dorofeyeva@gmail.com}
\label{end\stat}

\renewcommand{\bibname}{\protect\rm Литература}        %12
\def\stat{kudr}

\def\tit{ПРИБЛИЖЕННЫЕ МЕТОДЫ РЕШЕНИЯ ЗАДАЧИ ДИАГНОСТИКИ ПЛОСКИМ 
ЗОНДОМ СИЛЬНОИОНИЗОВАННОЙ ПЛАЗМЫ С~УЧЕТОМ КУЛОНОВСКИХ 
СТОЛКНОВЕНИЙ}

\def\titkol{Приближенные методы решения задачи диагностики плоским 
зондом сильноионизованной плазмы} %с~учетом Кулоновских  столкновений}

\def\autkol{И.\,А.~Кудрявцева, А.\,В.~Пантелеев}
\def\aut{И.\,А.~Кудрявцева$^1$, А.\,В.~Пантелеев$^2$}

\titel{\tit}{\aut}{\autkol}{\titkol}

%{\renewcommand{\thefootnote}{\fnsymbol{footnote}}\footnotetext[1]
%{Работа поддержана Российским фондом фундаментальных исследований
%(проекты 11-01-00515а и 11-07-00112а), а также Министерством
%образования и науки РФ в рамках ФЦП <<Научные и
%научно-педагогические кадры инновационной России на 2009--2013~годы>>.}}


\renewcommand{\thefootnote}{\arabic{footnote}}
\footnotetext[1]{Московский авиационный институт, irina.home.mail@mail.ru}
\footnotetext[2]{Московский авиационный институт, avpanteleev@inbox.ru}

\vspace*{-2pt}

\Abst{Сформирована математическая модель, описывающая динамику сильноионизованной 
плазмы с учетом столкновений заряженных частиц вблизи плоского зонда. Модель включает уравнение 
Фоккера--Планка и уравнение Пуассона. Предложено два подхода к решению задачи: на основе метода 
статистических испытаний Мон\-те-Кар\-ло и на основе композиции метода крупных частиц и метода 
расщепления.} 

\vspace*{-2pt}

\KW{телекоммуникационные системы; метод Монте-Карло; метод крупных частиц; метод 
расщепления; зонд; уравнение Фоккера--Планка; уравнение Пуассона} 

\vspace*{-4pt}

 \vskip 8pt plus 9pt minus 6pt

      \thispagestyle{headings}

      \begin{multicols}{2}
      
            \label{st\stat}

\section{Введение}

В настоящее время в области телекоммуникаций все более востребованными становятся 
информационные технологии, основанные на использовании математических моделей и численных 
методов физики плазмы. Поэтому особенно актуальным является решение разнообразных задач анализа 
поведения плазмы, включающих в себя формирование новых моделей и методов их исследования. 
Помимо этого, в разработке телекоммуникационного оборудования эффективно используются 
собственно физические свойства плазмы. В~частности, изготовлена антенна, работа которой основана 
на газовом разряде низкотемпературной плазмы~[1], интенсивно ведутся разработки по созданию и 
усовершенствованию источников бесперебойного питания на основе плазменных элементов~[2, 3]. 
      
      Одним из наиболее перспективных направлений для построения систем оптической 
беспроводной связи является использование лазеров~\cite{4-k, 5-k}. В~этой связи большое внимание 
уделяется использованию плазмы при разработке импульсных сильноточных коммутаторов~\cite{6-k}, 
так как практическое применение подобных разработок требует повышения уровня надежности и 
быстродействия лазерных систем.
      
      Исследования низкотемпературной плазмы также связаны с разработками в области дальней 
космической связи, так как моделирование процессов взаимодействия заряженного тела с верхними 
слоями атмосферы позволяет предлагать способы улучшения существующих систем радиосвязи с 
космическими летательными аппаратами~\cite{7-k}. 
      
      Наряду с этим актуальными также являются задачи диагностики плазмы, поскольку перспективы 
ее использования в области телекоммуникаций после более полного изучения физических свойств 
могут значительно расшириться. 

Для диагностики плазмы применяют зондовые методы исследования~[8--11]. Эти методы относятся к 
классу контактных методов; как следствие, возникает сложность в исследовании пристеночной области 
вблизи зонда, которая характеризуется достаточно сложным распределением потенциала и функциями 
распределения, отличными от максвелловских. 

Данная работа посвящена исследованию переходного режима обтекания заряженного тела плазмой. Для 
переходного режима выполняется следующее условие: длина свободного пробега иона до столкновения 
с нейтральным атомом или другим ионом невелика по сравнению с характерными размерами тела. 
В~этом случае возникает необходимость учета столкновений заряженных частиц с нейтральными 
атомами и кулоновских столкновений. В~работах~\cite{10-k, 11-k} подробно рассмотрена модель с 
учетом столкновений заряженных частиц с нейтральными атомами. В~настоящей статье представлена 
теоретическая модель, описывающая влияния ион-ионных и ион-элек\-т\-рон\-ных столкновений на 
измеряемые характеристики плазмы, что ранее детально не исследовалось.
      
      В~рамках данной работы предлагается модель, описывающая динамику сильноионизованной 
плазмы с учетом кулоновских столкновений. Эта модель учитывает такие процессы взаимодействия, 
как перенос частиц и столкновения между заряженными частицами типа <<ион--ион>> и 
      <<ион--электрон>> под влиянием макроскопического электрического поля. Перечисленные 
процессы описываются самосогласованной системой уравнений, включающей уравнение 
      Фок\-ке\-ра--План\-ка и уравнение Пуассона~[12].
      
      Вычислительная модель задачи строится на основе двух методов: метода статистических 
испытаний Мон\-те-Кар\-ло и композиции метода крупных частиц и метода расщепления. Приведены 
результаты численного моделирования, полученные с использованием вышеперечисленных методов.

\vspace*{-4pt}

\section{Постановка задачи}

\vspace*{-2pt}

Рассматривается следующая физическая постановка зондовой задачи~[11]. В~невозмущенную 
бесконечно протяженную плазму, состоящую из электронов и однозарядных ионов, внесена большая\linebreak 
заряженная до потенциала $\varphi_p$ плоскость. Плоскость, расположенная поперек потока плазмы, 
является идеально поглощающей для электронов. Ионы при ударе о плоскость нейтрализуются. 
Предполагается, что частицы в плазме движутся под действием внешнего электрического поля, 
магнитное поле отсутствует. Концентрации ионов $n_{i\infty}$ и электронов $n_{e\infty}$, а также 
температуры данных час\-тиц~$T_{i\infty}$ 
и~$T_{e\infty}$ в невозмущенной плазме заданы. За начальные 
функции распределения обоих типов час\-тиц принимаются функции распределения Максвелла. 
      
      Требуется с учетом столкновений между заряженными частицами найти напряженность 
самосогласованного электрического поля $\vec{E}(\vec{r},t)$, функции распределения однозарядных 
ионов $f_i(\vec{r}, \vec{v}, t)$ и электронов $f_e(\vec{r}, \vec{v}, t)$, 
а также их моменты (плотности 
токов ионов и электронов  $j_i(\vec{r},t)\hm
=q\int f_i(\vec{r}, \vec{v}, t)\vec{v}\,d\vec{v}$, $j_e(\vec{r},t) 
\hm={\sf e}\int f_e(\vec{r},\vec{v},t)\vec{v}\,d\vec{v}$, где $q=Z_i{\sf e}$, $Z_i=1$~--- заряд иона, ${\sf 
e}$~--- заряд электрона; концентрации ионов и электронов $n_i(\vec{r},t)\hm=\int 
f_i(\vec{r},\vec{v},t)\,d\vec{v}$, $n_e(\vec{r},t)\hm=\int f_e(\vec{r},\vec{v}, t)\,d\vec{v}$). 
Поведение частиц во 
времени~$t$ характеризуется ра\-ди\-ус-век\-то\-ром~$\vec{r}$ и вектором скорости~$\vec{v}$.
      
      Математическая модель, соответствующая данной физической постановке задачи, имеет 
вид~\cite{11-k, 13-k}:

\noindent
      \begin{equation}
      \left.
      \begin{array}{c}
      \fr{\partial f_\alpha (\vec{r},\vec{v},t)}{\partial t}+
      \vec{v}\fr{\partial f_\alpha (\vec{r},\vec{v},t)}{ 
\partial \vec{r}}+
\fr{\vec{F}_\alpha(\vec{r},t)}{m_\alpha}\times{}\\[4pt]
{}\times\fr{\partial f_\alpha(\vec{r},\vec{v},t)}{ \partial 
\vec{v}}=
\left(\fr{\partial f_\alpha(\vec{r},\vec{v},t)}{ \partial t}\right)_{\mathrm{с}}+S_\alpha 
(\vec{r},\vec{v},t)\,;\\[6pt]
      \Delta\varphi(\vec{r},t)=-\fr{{\sf e}}{\varepsilon_0}\left( n_i(\vec{r},t)-n_e(\vec{r},t)\right)\,;\\[6pt]
      \vec{E}(\vec{r},t)=-\nabla \varphi(\vec{r},t)\,.
      \end{array}\!\!
      \right\}\!\!
      \label{e1-k}
      \end{equation}
Здесь первое уравнение~--- уравнение Фок\-ке\-ра--План\-ка для частиц сорта~$\alpha$ ($\alpha=i,e$), 
второе~--- уравнение Пуассона для самосогласованного электрического поля; 
$f_\alpha(\vec{r},\vec{v},t)$~--- функция\linebreak
распределения час\-тиц сорта~$\alpha$; $(\partial 
f_\alpha(\vec{r},\vec{v},t)/\partial t)_{\mathrm{с}}$~--- 
оператор столкновений Фок\-ке\-ра--План\-ка; 
функция~$S_\alpha(\vec{r},\vec{v},t)$ описывает источники или стоки\linebreak
 час\-тиц; 
$\vec{F}_\alpha(\vec{r},t)=q_\alpha\vec{E}(\vec{r},t)$, где $\vec{E}(\vec{r},t)$~--- напряженность 
самосогласованного электрического поля, 
$$
q_\alpha =
\begin{cases}
-{\sf e}\,, & \alpha=e\,,\\
{\sf e}\,, & \alpha=i\,;
\end{cases}
$$
$\varphi(\vec{r},t)$~--- потенциал самосогласованного электрического поля; $n_\alpha(\vec{r},t)$ ($\alpha 
\hm=i,e$)~--- концентрация частиц сорта~$\alpha$; $m_\alpha$~--- масса частицы сорта~$\alpha$; 
$\varepsilon_0$~--- электрическая постоянная. 

Оператор столкновений Фок\-ке\-ра--План\-ка имеет вид~\cite{13-k, 14-k}
\begin{multline*}
\fr{1}{\Gamma_\alpha}\left( \fr{\partial f_\alpha}{\partial t}\right)_{\mathrm{с}} 
=\fr{1}{2}\,\nabla_v\nabla_v:\left(f_\alpha\nabla_v\nabla_vg_\alpha(\vec{r},\vec{v},t)\right)-{}\\
{}-
\nabla_v\cdot\left(f_\alpha\nabla_v h_\alpha\right)\,,
\end{multline*}
где $\nabla_v\nabla_v g_\alpha(\vec{r},\vec{v},t)$~--- ковариантная тензорная производная второго ранга, 
знак двоеточия ($:$) обозначает операцию двойного суммирования:
\begin{gather*}
\Gamma_\alpha=\fr{Z_\alpha^4 {\sf e}^4}{4\pi \varepsilon_0^2 m^2_\alpha}\,\ln D_\alpha\,;
\\
D_\alpha =\fr{12\pi\varepsilon_0 kT_{\alpha\infty}}{Z_\alpha^2 {\sf e}^2}\left( \fr{\varepsilon_0 k 
T_{e\infty}}{n_{e\infty} {\sf e}^2}\right)^{1/2}\,;\\
g_\alpha (\vec{r},\vec{v},t)=\sum\limits_{b=i,e}\left( \fr{Z_b}{Z_\alpha}\right) \int f_b 
(\vec{r},{\vec{v}}^{\,\prime},t)\left\vert \vec{v}-{\vec{v}}^{\,\prime}\right\vert\,d\vec{v}^{\,\prime}\,;\\
h_\alpha (\vec{r},\vec{v},t)=\sum\limits_{b=i,e} \fr{m_\alpha+m_b}{m_b} 
\left(\fr{Z_b}{Z_\alpha}\right)
\int
\fr{f_b(\vec{r},{\vec{v}}^{\,\prime}, t)}{\vert \vec{v}-{\vec{v}}^{\,\prime}\vert}
\,d{\vec{v}}^{\,\prime}\,;\\
Z_\alpha =1\,, \quad \alpha=i,e\,.
\end{gather*}
 
К системе уравнений~(\ref{e1-k}) необходимо добавить начальные и краевые условия:
\begin{equation}
\!\left.
\begin{array}{rrl}
t=0:\ & f_\alpha(\vec{r},\vec{v},0)&=f_\alpha^{\mathrm{maksv}}\,,\enskip \alpha=i,e;\\[9pt]
\vec{r}\in \Omega_p:\ & f_\alpha(\vec{r},\vec{v},t)\big\vert_{\vec{r}\in\Omega_p}&=0\,,\enskip \alpha=i,e\,;\\[9pt]
&\varphi(\vec{r},t)\big\vert_{\vec{r}\in\Omega_p}&=\varphi_p\,;\\[9pt]
\vec{r}\in\Omega_\infty:\ & 
f_\alpha(\vec{r},\vec{v},t)\big\vert_{\vec{r}\in\Omega_\infty}&= %{}\\[9pt]
f_\alpha^{\mathrm{maksv}}\,,\enskip \alpha=i,e\,;\\[9pt]
&\varphi(\vec{r},t)\big\vert_{\vec{r}\in\Omega_\infty}&=0\,,
\end{array}\!\!
\right\}\!\!\!\!
\label{e2-k}
\end{equation}
    где 
    
    \noindent
    \begin{multline*}
    f_\alpha^{\mathrm{maksv}}=n_{\alpha\infty}\left(\fr{m_\alpha}{2k\pi T_{\alpha\infty}}\right)^{3/2}\times{}\\
    {}\times
    \exp\left( -
\fr{m_\alpha}{2kT_{\alpha\infty}}\left\vert\vec{v}-\vec{v}_\infty\right\vert^2\right)\,,
\enskip \alpha=i, e\,;
\end{multline*} 
$\Omega_p$ и $\Omega_\infty$~--- множество радиус-векторов час\-тиц, концы которых принадлежат плоскости зонда и 
границе возмущенной зоны соответственно.

Для решения поставленной задачи введем декартову систему координат таким образом, чтобы 
заряженная плоскость совпала с плоскостью~$0xz$. Тогда положение частицы в пространстве будет 
определяться координатами $x,y,z$, а скорость~--- координатами $v_x, v_y, v_z$. В~силу того что 
плоскость является бесконечно большой в сравнении с характерным размером задачи, функции 
распределения частиц будут зависеть только от переменных $y, v_y, t$.

Поставленную задачу предлагается решать независимо двумя методами. Первый метод основывается на 
методе статистических испытаний Мон\-те-Кар\-ло, второй метод является композицией метода 
расщепления и метода крупных частиц.

\section{Применение метода Монте-Карло}

Запишем самосогласованную систему уравнений~(\ref{e1-k}) и~(\ref{e2-k}) в декартовой системе 
координат с учетом сделанных предположений:
\begin{equation}
\left.
\begin{array}{l}
\fr{\partial f_\alpha}{\partial t}+
v_y\fr{\partial f_\alpha}{\partial y}+\fr{F_y^\alpha}{m_\alpha}\,\fr{\partial 
f_\alpha}{\partial v_y}=\fr{1}{2}\,\fr{\partial^2 }{\partial [v_y]^2}\times{}\\
{}\times \left( 
f_\alpha\fr{\partial^2 g_\alpha  }{\partial [v_y]^2}\right) -
\fr{\partial}{\partial v_y}\left( f_\alpha\fr{\partial h_\alpha}{\partial v_y}\right)\,,
\enskip \alpha=i,e\,;\\[6pt]
    \fr{\partial^2\varphi}{\partial y^2} =-\fr{{\sf e}}{\varepsilon_0}\left(n_i-n_e\right)\,;
    \enskip E_y=-
\fr{\partial\varphi}{\partial y}\,;\\[6pt]
\hspace*{3.1mm}    t=0:\  \hspace*{2.6mm}f_\alpha(y,v_y,0)=f_\alpha^{\mathrm{maksv}}\,,\ \alpha=i,e\,;\\[9pt]
\hspace*{2.9mm} y=0:\ \hspace*{2.8mm}f_\alpha(0,v_y,t)=0\,,\ \alpha=i,e\,;\\[9pt]
\hspace*{24.3mm}\varphi(0,t)=\varphi_p\,;\\[9pt]
y=y_\infty:\ f_\alpha(y_\infty, v_y, t)=f_\alpha^{\mathrm{maksv}}\,,\ \alpha=i,e\,;\\[9pt]
\hspace*{21.5mm}\varphi(y_\infty, t)=0\,.
\end{array}
\right \}
\label{e3-k}
\end{equation}

В полученной системе уравнений~(\ref{e3-k}) перейдем к безразмерным величинам, применив 
соотношение $X=M_X \hat{X}$, где $M_X$~--- масштаб размерной величины~$X$, $\hat{X}$~--- 
безразмерная величина~$X$. В~качестве используемых масштабов были взяты следующие: радиус 
Дебая, скорость теплового движения частиц, концентрация частиц в невозмущенной плазме, потенциал, 
возникающий при разделении зарядов в дебаевской сфере, и производные от них величины.

Система безразмерных уравнений имеет следующий вид:
%\noindent
\begin{equation}
\left.
\begin{array}{l}
\fr{\partial 
\hat{f}_\alpha}{\partial\hat{t}}+A_\alpha\fr{\partial\hat{f}_\alpha}{\partial\hat{y}}+
B_\alpha\hat{E}_y\fr{\partial\hat{f}_\alpha}{\partial \hat{v}_y}={}\\
\!{}=
\fr{\partial^2}{\partial[\hat{v}_y]^2}\left(D_\alpha 
\hat{f}_\alpha\right)-\fr{\partial}{\partial\hat{v}_y}\left(K_\alpha \hat{f}_\alpha\right),\enskip 
\alpha=i,e;\\[9pt]
\fr{\partial^2\hat{\varphi}}{\partial\hat{y}^2}=-\left(\hat{n}_i-\hat{n}_e\right)\,;\enskip \hat{e}_y=-
\fr{\partial\hat\varphi}{\partial\hat{y}}\,;\\[9pt]
\hspace*{3.1mm}\hat{t}=0:\ \hspace*{2.6mm}\hat{f}_\alpha(\hat{y},\hat{v}_y,0)=\hat{f}_\alpha^{\mathrm{maksv}}\,,\enskip \alpha-i,e\,;\\[9pt]
\hspace*{2.9mm}\hat{y}=0:\ \hspace*{2.8mm}\hat{f}_\alpha(0,\hat{v}_y,\hat{t})=0\,,\enskip \alpha=i,e\,;\\[9pt]
\hspace*{24.3mm}\hat\varphi(0,\hat{t})=\hat{\varphi}_p\,;\\[9pt]
\hat{y}=\hat{y}_\infty:\ \hat{f}_\alpha(\hat{y}_\infty, \hat{v}_y, \hat{t})=\hat{f}^{\mathrm{maksv}}_\alpha\,,\enskip 
\alpha=i,e\,;\\[9pt]
\hspace*{21.5mm}\hat\varphi(\hat{y}_\infty,\hat{t})=0\,.
\end{array}
\right\}
\label{e4-k}
\end{equation}
Здесь 

\vspace*{-2pt}

\noindent
\begin{gather*}
A_\alpha=\sqrt{\delta_\alpha }\,\hat{v}_y\,;\enskip 
B_\alpha=\sqrt{\delta_\alpha}\,\fr{z_\alpha}{2\varepsilon_\alpha}\,;\\
\delta_\alpha=\fr{\varepsilon_\alpha}{\mu_\alpha}\,;\enskip 
\varepsilon_\alpha=\fr{T_{\alpha\infty}}{T_{i\infty}}\,;\\
\mu_\alpha=\fr{m_\alpha}{m_i}\,;\enskip 
D_\alpha=A_g^\alpha\fr{\partial^2\hat{g}_\alpha}{\partial  [\hat{v}_y]^2}\,;\\
K_\alpha=A_h^\alpha \fr{\partial \hat{h}_\alpha}{\partial \hat{v}_y}\,,\enskip \alpha=i,e\,,
\end{gather*}
где $A_g^\alpha$ и $A_h^\alpha$~--- коэффициенты, определяемые характерными параметрами 
задачи~\cite{15-k}.

Поиск решения самосогласованной системы уравнений~(\ref{e4-k}) осуществляется по следующей 
схе-\linebreak ме. Вначале находятся значения напряженности\linebreak
 электрического поля по значениям потенциала, 
полученным из граничной задачи для уравнения Пуассона. Далее, используя найденные значения 
напряженности, решается уравнение Фок\-ке\-ра--План\-ка путем перехода к стохастическому 
дифференциальному уравнению (СДУ) Ито:

\noindent
\begin{multline*}
d\Theta_\alpha(\hat{t}) = a_\alpha \left(\hat{t},\Theta_\alpha(\hat{t})\right)+{}\\
{}+\sigma\left(
\hat{t},\Theta_\alpha(\hat{t})\right)\,dW(\hat{t})\,,\quad \alpha=i,e\,,
%\label{e5-k}
\end{multline*}
где 

\noindent
\begin{align*}
\Theta_\alpha(\hat{t})&=\begin{bmatrix}
\hat{y}(\hat{t})\\ \hat{v}_y(\hat{t})
\end{bmatrix}\,;\\
a_\alpha\left(\hat{t},\Theta_\alpha(\hat{t})\right)&=\begin{bmatrix}
-A_\alpha\\ -K_\alpha -B_\alpha \hat{E}_y
\end{bmatrix}\,;\\
\sigma_\alpha\left(\hat{t},\Theta_\alpha(\hat{t})\right)\sigma_\alpha^{\mathrm{T}}\left( 
\hat{t},\Theta_\alpha(\hat{t})\right)&=D_\alpha\,,\enskip \alpha=i,e\,;
\end{align*} 
$W(\hat{t})$~--- стандартный винеровский случайный процесс.
\pagebreak

Для нахождения значений вектора состояния~$\Theta_\alpha(\hat{t})$ применим явную разностную 
схему стохастического метода Эйлера~\cite{16-k}:
\begin{multline*}
\Theta_\alpha^{n+1}=\Theta_\alpha^n +h_\tau a_\alpha \left( \hat{t}_n, \Theta_\alpha^n\right)+\sigma_\alpha 
\left( \hat{t}_n, \Theta_\alpha^n\right)\Delta W_n\,,\\ 
n=0,\ldots , N\,,\ \alpha=i,e\,,
%\label{e6-k}
\end{multline*}
где $\Theta_\alpha^n$, $n=0,\ldots , N$,~--- приближенное значение вектора 
состояния~$\Theta_\alpha(\hat{t})$, $\alpha=i,e$, в момент времени $\hat{t}\hm=\hat{t}_n$, 
$\hat{t}_n\hm=n h_\tau$, $n=0,\ldots , N$; $h_\tau$~--- достаточно малый шаг интегрирования; $\Delta 
W_n$, $n=0,\ldots ,N$,~--- величина приращения винеровского процесса~$W(\hat{t})$ на отрезке $\left[ 
\hat{t}_n,\,\hat{t}_{n+1}\right]$, по определению независимая от~$\Theta_\alpha^0$, 
$\Delta W_0,\ldots , 
\Delta W_{n-1}$: $\Delta W_n\hm=W(\hat{t}_{n-1})\hm-W(\hat{t}_n)$; $\Delta W_n\hm\sim N(0,\,h_\tau)$, 
т.\,е.\ $\Delta W_n$ представляют собой гауссовские случайные величины с нулевыми математическими 
ожиданиями и дисперсиями, равными шагу интегрирования; $\Theta_\alpha^0$~--- значение вектора 
состояния $\Theta_\alpha(\hat{t})$, $\alpha\hm=i,e$, в момент времени $\hat{t}=0$, 
$\Theta_\alpha^0\hm\sim \hat{f}_\alpha^{\mathrm{maksv}}$. 

Частные производные $\partial^2\hat{g}_\alpha/\partial[\hat{v}_y]^2$ и $\partial \hat{h}_\alpha/\partial 
\hat{v}_y$, являющиеся составляющими матрицы $\sigma_\alpha (\hat{t}_n, 
\Theta_\alpha^n)\sigma_\alpha^{\mathrm{T}}(\hat{t}_n,\Theta_\alpha^n)$ и вектора $a_\alpha(\hat{t}_n, 
\Theta_\alpha^n)$ соответственно, аппроксимируются со вторым порядком точности на трехточечном 
шаблоне на основе значений~$\hat{g}_\alpha$ и~$\hat{h}_\alpha$~\cite{17-k}.
      
      В выражения для функций~$\hat{g}_\alpha$ и~$\hat{h}_\alpha$ входят интегралы, которые 
вычисляются методом Мон\-те-Кар\-ло с использованием набора значений скоростной компоненты 
вектора состояния~$\hat{v}_y$, полученных из решения СДУ Ито:
      \begin{equation*}
      \int \hat{f}_\alpha \left\vert \hat{v}_y-
\hat{v}_y^\prime\right\vert\,dv_y^\prime=M\left(\zeta\left(\hat{V}_y\right)\right)\,,
\end{equation*}
где
$$
      \zeta\left(\hat{V}_y\right)=\left\vert \hat{v}_y-\hat{V}_y\right\vert\,,\enskip \hat{V}_y\sim 
\hat{f}_\alpha\,.
  $$
      
      Для вычисления напряженности самосогласованного электрического поля $\hat{E}_y=-
\partial\hat{\varphi}/\partial\hat{y}$, входящей в вектор $a_\alpha(\hat{t}_n, \Theta_\alpha^n)$, необходимо 
аналогично аппроксимировать со вторым порядком точности производную 
$\partial\hat{\varphi}/\partial\hat{y}$ на трехточечном шаблоне с использованием значений 
потенциала~$\hat{\varphi}$~\cite{17-k}. Значения потенциала~$\hat\varphi$ находятся из решения 
уравнения Пуассона. 
      
      Граничную задачу для уравнения Пуассона 
      \begin{align*}
      \fr{\partial^2 \hat\varphi}{\partial \hat{y}^2} & = -\left(\hat{n}_i-\hat{n}_e\right)\,;\\
      \hat{\varphi}\big|_{\hat{y}=0} &=\hat{\varphi}_p\,;\\
      \hat{\varphi}\big|_{\hat{y}_\infty=0} &=0
      \end{align*}
    предлагается решать путем перехода к конечно-разностной системе с последующим ее решением 
методом прогонки~\cite{17-k}:

\noindent
\begin{gather*}
\hat{\varphi}^n_{l-1}+2\hat{\varphi}_l^n+\hat{\varphi}^n_{l+1}=
h_y\hat{\delta}_l^n\,,\enskip l=1,\ldots , 
N_y\,;\\
\hat{\delta}_l^n=-\left( \hat{n}^n_{i,l}-\hat{n}^n_{e,l}\right)\,;\enskip 
\hat{\varphi}_0=\hat{\varphi}_p\,;\enskip \hat{\varphi}_{N_y}=0\,,
\end{gather*}
где $N_y$~--- число шагов по переменной~$\hat{y}$, $h_y$~--- величина шагов разбиения по~$\hat{y}$. 
      
      Концентрации $\hat{n}_\alpha$, $\alpha=i,e$, и плотности токов частиц на зонд~$\hat{f}_\alpha$, 
$\alpha=i,e$, вычисляются согласно описанному выше методу Мон\-те-Карло.

\section{Применение метода расщепления и~метода крупных~частиц}

Решение задачи в данном случае предлагается начать с записи правой части уравнения 
Фок\-ке\-ра--План\-ка в декартовой системе координат в виде:
$$
\mathbf{Q} f_\alpha = \fr{1}{2}\,\fr{\partial^2 f_\alpha}{\partial [v_y]^2}\,\fr{\partial^2 g_\alpha}{\partial 
[v_y]^2}+\fr{\partial f_\alpha}{\partial v_y}\,\fr{\partial C_\alpha}{\partial v_y}+H_\alpha\,,\enskip 
\alpha=i,e\,,
$$  
где 
\begin{align*}
C_\alpha(\vec{r},\vec{v},t)&=
\begin{cases}
\fr{1-\gamma}{Z_i^2}\int\fr{f_e(\vec{r},{\vec{v}}^{\,\prime},t)}{|\vec{v}-{\vec{v}}^{\,\prime} |}\,d{\vec{v}}^{\,\prime}\,, 
&\alpha=i\,;\\[9pt]
\fr{Z_i^2(\gamma-1)}{\gamma}\int \fr{f_i(\vec{r},{\vec{v}}^{\,\prime}, t)}
{|\vec{v}-{\vec{v}}^{\,\prime} 
|}\,d{\vec{v}}^{\,\prime}\,, &\alpha=e\,;
\end{cases} 
\\
H_\alpha&=
\begin{cases}
4\pi \left( \fr{\gamma f_e}{Z_i^2}+f_i\right)f_i\,, & \alpha=i\,;\\[9pt]
4\pi\left(\fr{Z_i^2 f_i}{\gamma}+f_e\right)f_e\,, &\alpha=e\,.
\end{cases}
\end{align*}
Тогда при переходе к безразмерным величинам (см.\ разд.~3) система~(\ref{e1-k}) запишется 
следующим образом:
      \begin{equation}
      \left.
\!\!\begin{array}{l}
      \fr{\partial 
\hat{f}_\alpha}{\partial\hat{t}}+A_\alpha\fr{\partial\hat{f}_\alpha}{\partial\hat{y}}+
B_\alpha  \hat{E}_y
\fr{\partial\hat{f}_\alpha}{\partial\hat{v}_\alpha}=\tilde{\mathbf{Q}}\hat{f}_\alpha\,,\enskip 
\alpha=i,e;\\[9pt]
      \fr{\partial^2\hat{\varphi}}{\partial\hat{y}^2}=-\left( \hat{n}_i-\hat{n}_e\right)\,,\enskip \hat{E}_y=-
\fr{\partial\hat\varphi}{\partial\hat{y}}\,,\\[9pt]
\hspace*{3.1mm}\hat{t}=0:\ \hspace*{2.6mm}\hat{f}_\alpha(\hat{y},\hat{v}_y, 0)=\hat{f}_\alpha^{\mathrm{maksv}}\,,\enskip \alpha=i,e\,,\\[9pt]
\hspace*{2.9mm} \hat{y}=0:\ \hspace*{2.8mm}\hat{f}_\alpha(0,\hat{v}_y,\hat{t})=0\,,\enskip \alpha=i,e\,;\\[9pt]
\hspace*{24.3mm}\hat\varphi(0,\hat{t})=\hat{\varphi}_p\,;\\[9pt]
      \hat{y}=\hat{y}_\infty:\ \hat{f}_\alpha(\hat{y}_\infty, 
\hat{v}_y,\hat{t})=\hat{f}_\alpha^{\mathrm{maksv}}\,,\enskip \alpha=i,e\,;\\[9pt]
\hspace*{21.5mm}\hat{\varphi}(\hat{y}_\infty,\hat{t})=0\,,\\[9pt]
    \end{array}
\right\}\!\!
\label{e7-k}
\end{equation}
где 
\begin{gather*}
\tilde{\mathbf{Q}} \hat{f}_\alpha=D_\alpha\fr{\partial^2\hat{f}_\alpha}{\partial 
[\hat{v}_y]^2}+K_\alpha\fr{\partial\hat{f}_\alpha}{\partial\hat{v}_y}+H_\alpha\,;\\
D_\alpha=A_g^\alpha\fr{\partial^2\hat{g}_\alpha}{\partial [\hat{v}_y]^2}\,;\enskip 
K_\alpha=A_h^\alpha \fr{\partial \hat{h}_\alpha}{\partial\hat{v}_y}\,,\ \alpha=i,e\,.
\end{gather*}

Для решения системы уравнений~(\ref{e7-k}) применяется модификация метода 
расщепления~\cite{17-k}, согласно которой исходная задача разбивается на две вспомогательные. Такое 
разбиение можно осуществить, переписав уравнение Фок\-ке\-ра--План\-ка в следующем виде:
$$
\fr{\partial\hat{f}_\alpha}{\partial\hat{t}} =
\tilde{\mathbf{Q}}_1\hat{f}_\alpha+\tilde{\mathbf{Q}}_2\hat{f}_\alpha\,,
$$
где 
\begin{align*}
\tilde{\mathbf{Q}}_1\hat{f}_\alpha &=-
\left(A_\alpha\fr{\partial\hat{f}_\alpha}{\partial\hat{y}}+
B_\alpha\fr{\partial\hat{f}_\alpha}{\partial\hat{y}}
\right)\,;\\
\tilde{\mathbf{Q}}_2\hat{f}_\alpha 
&=\left(D_\alpha\fr{\partial^2\hat{f}_\alpha}{\partial[\hat{v}_y]^2}+K_\alpha\fr{\partial 
\hat{f}_\alpha}{\partial\hat{v}_y}+H_\alpha\right)\,.
\end{align*}

      Правая часть уравнения Фок\-ке\-ра--План\-ка представляет собой сумму двух операторов, 
первый из которых отвечает за перенос частиц, второй~--- за столкновения заряженных частиц. 
В~результате образуются следующие задачи, которые решаются последовательно:
      \begin{itemize}
\item первая задача:
\begin{align*}
&\fr{\partial w_\alpha(\hat{y},\hat{v}_y,\hat{t})}{\partial\hat{t}} =\mathbf{Q}_1 
w_\alpha(\hat{y},\hat{v}_y,\hat{t})\,,\enskip \alpha=i,e\,;\\[9pt]
&\fr{\partial^2\hat\varphi}{\partial\hat{y}^2}=-\left(\hat{n}_i-\hat{n}_e\right)\,;\enskip
\hat{E}_y=-
\fr{\partial\hat\varphi}{\partial\hat{y}}\,;\\[9pt]
&w_\alpha(\hat{y},\hat{v}_y,\hat{t}^n)=\hat{f}_\alpha(\hat{y},\hat{v}_y,\hat{t}^n)\,,\enskip n=0,\ldots ,N-
1\,;\\[9pt]
&\hspace{2.9mm}\hat{y}=0:\ \hspace*{2.9mm}w_\alpha(0,\hat{v}_y,\hat{t})=0\,,\enskip \alpha=i,e\,;\\[9pt]
&\hspace*{25.1mm}\hat\varphi(0,\hat{t})=\hat{\varphi}_p\,;\\[9pt]
&\hat{y}=\hat{y}_\infty:\ w_\alpha(\hat{y}_\infty, \hat{v}_y, \hat{t})=
\hat{f}_\alpha^{\mathrm{maksv}}\,,\enskip 
\alpha=i,e\,;\\[9pt]
&\hspace*{22.5mm}\hat\varphi(\hat{y}_\infty,\hat{t})=0\,;
\end{align*}
\item вторая задача:
\begin{align*}
\!\!\!\!\!\!\!\fr{\partial s_\alpha(\hat{y},\hat{v}_y,\hat{t})}{\partial \hat{t}} &=\mathbf{Q}_2 
s_\alpha(\hat{y},\hat{v}_y,\hat{t})\,, & \alpha&=i,e\,;\\
\!\!\!\!\!\!\!s_\alpha (\hat{y},\hat{v}_y,\hat{t}^n) &=w_\alpha (\hat{y},\hat{v}_y, \hat{t}^{n+1}),& n&=0,\ldots ,N-
1.
\end{align*}
\end{itemize}

Первая задача представляет собой систему безразмерных уравнений Вла\-со\-ва--Пуас\-со\-на. Для ее 
решения применяется метод крупных частиц~\cite{18-k}. Согласно этому методу решение задачи 
осуществляется путем расщепления на два этапа: на первом этапе не учитываются конвективные члены 
и решение получается обычным интегрированием на неподвижной эйлеровой сетке, а на втором этапе 
рассматривается система, которая описывает перенос частиц в лагранжевой системе координат. Кроме 
того, на первом этапе необходимо решить уравнение Пуассона для получения значений потенциала 
самосогласованного электрического поля. Для этого применяется метод, описанный в разд.~3. 

Вторая задача решается путем перехода к ко\-неч\-но-раз\-ност\-ной сис\-те\-ме. При этом частные 
производные $\partial^2\hat{g}_\alpha/\partial[\hat{v}_y]^2$ и $\partial\hat{h}_\alpha/\partial\hat{v}_y$ 
аппроксимируются со вторым порядком точности с использованием трехточечного шаблона, а 
производная $\partial s_\alpha/\partial\hat{t}$ аппроксимируется на двухточечном шаблоне с первым 
порядком точности~\cite{16-k}. К~полученной системе разностных уравнений предлагается применить 
один из классических методов решения систем линейных уравнений, например метод 
Гаусса~\cite{19-k}.
      
      Решением первой задачи является функция $w_\alpha(\hat{y}, \hat{v}_y, \hat{t}^n)$, 
$n\hm=0,\ldots ,N$, , которая дает начальное условие для второй задачи. Решая вторую задачу, находим 
функцию $s_\alpha(\hat{y},\hat{v}_y,\hat{t}^n)\hm=\hat{f}_\alpha(\hat{y},\hat{v}_y,\hat{t}^n)$, 
$n=1,\ldots ,N$, $\alpha=i,e$, которая определяет решение $\hat{f}_\alpha(\hat{y},\hat{v}_y,\hat{t}^n)$, 
$\alpha=i,e$, исходной системы~(\ref{e7-k}) для рассматриваемых моментов времени $n=1,\ldots ,N$.

Моменты функций распределения $\hat{f}_\alpha$, $\alpha=i,e$, находятся с помощью методов 
численного интегрирования, например метода трапеций~\cite{19-k}.

\section{Результаты численного моделирования}

Для двух описанных выше методов реализованы две отдельные программы в среде {Matlab~7.0}. 
Эти программы позволяют по заданным значениям концентраций и температур частиц $n_{i\infty}$, 
$n_{e\infty}$, $T_{i\infty}$ и~$T_{e\infty}$ в невозмущенной плазме, а также потенциала~$\varphi_p$, 
подаваемого на зонд, изучить эволюцию во времени плотностей тока частиц~$j_i$ и~$j_e$, концентраций 
частиц~$n_i$  и~$n_e$ в произвольной точке пространства в возмущенной зоне, а также динамику 
изменения напряженности~$E_y$ самосогласованного электрического поля во времени и пространстве.

С использованием разработанных программ проведены серии расчетных экспериментов, в которых 
значение концентраций варьировалось в пределах $n_{i\infty} \hm = n_{e\infty}\hm =10^{18}\div 
10^{22}$~м$^{-3}$. Значение температур было выбрано неизменным и равным $T_{i\infty}\hm = 
T_{e\infty}\hm=3000$~K, а значения потенциала, подаваемого на зонд, изменялись в пределах 
$\varphi_p\hm=0\div 2{,}6$~В.

На рис.~1  и~2 приведены графики изменения напряженности самосогласованного электрического
 поля (см.\ рис.~1) и плотности токов ионов (см.\linebreak\vspace*{-12pt}

\pagebreak

\end{multicols}

\begin{figure} %fig1
\vspace*{1pt}
\begin{center}
\mbox{%
\epsfxsize=162.594mm
\epsfbox{kud-1.eps}
}
\end{center}
\vspace*{-9pt}
\Caption{Динамика изменения плотности тока ионов во времени в фиксированной точке возмущенной 
зоны для значений потенциала: \textit{1}~--- $\varphi_p=-6$; 
\textit{2}~--- $\varphi_p=-16$; \textit{3}~--- $\varphi_p=- 30$ 
в случае применения методов Монте-Карло~(\textit{а}) 
и крупных частиц~(\textit{б})}
\end{figure}

\begin{figure} %fig2
\vspace*{1pt}
\begin{center}
\mbox{%
\epsfxsize=162.713mm
\epsfbox{kud-2.eps}
}
\end{center}
\vspace*{-9pt}
\Caption{Динамика изменения напряженности электрического поля во времени в фиксированной точке 
возмущенной зоны для значений потенциала: 
\textit{1}~--- $\varphi_p=-6$; \textit{2}~--- $\varphi_p=-16$; 
\textit{3}~--- $\varphi_p=-30$ в случае применения методов Монте-Карло~(\textit{а}) и
крупных частиц~(\textit{б})
}
\end{figure}

\begin{multicols}{2}

\noindent
 рис.~2) во времени в фиксированной точке пространства 
возмущенной зоны в случае применения обоих разработанных алгоритмов.


На основании полученных результатов можно отметить похожее поведение зависимостей 
напряженности электрического поля и плотности тока от времени в двух рассматриваемых случаях. 
Графики кривых сначала убывают, затем начинают возрастать, выходя в некоторый момент 
времени~$t^\prime$ (момент установления) на стационарные значения. 

Одинаковое поведение 
напряженности и плот\-ности тока можно объяснить из следующих соображений: плотность тока ионов в 
данной области пространства равна произведению концентрации ионов на их направленную скорость и 
на заряд иона. Скорость ионов, в свою очередь, зависит от заряда, массы и напряженности 
электрического поля. 
%\columnbreak

При внесении в плазму отрицательно заряженного зонда возникает электрическое поле, которое 
нарушает квазинейтральность плазмы. Для того чтобы компенсировать действие внешнего 
электрического поля, ионы устремляются к зонду, а электроны~--- от зонда. Это приводит к дисбалансу 
концентраций вблизи зонда и, как следствие, к увеличению разности потенциалов; график 
напряженности электрического поля убывает. Вскоре разделение зарядов компенсирует внешнее 
электрическое поле; график выходит на стационарное значение. 

Также можно отметить, что значения 
напряженности электрического поля и плотности тока частиц на зонд в момент установления для двух 
методов совпадают. 

Момент установления~$t^\prime$ зависит от при\-ме\-ня\-емо\-го метода решения. В~случае метода 
Мон\-те-Кар\-ло $t^\prime=3{,}5\div 4$~ед., а для метода крупных частиц совместно с методом 
расщепления $t^\prime\hm=5\div 5{,}5$~ед. Используя ко\-неч\-но-раз\-ност\-ный метод, можно 
получить динамику изменения функций распределения частиц~$f_\alpha$, $\alpha=i,e$, во времени и 
пространстве. Функции распределения позволяют наглядно представить влияние на картину 
распределения частиц вблизи зонда самой поверхности зонда и электрического поля.

\section{Заключение}
      
      В работе найдено решение задачи диагностики плоским зондом сильноионизованной плазмы с 
учетом столкновений заряженных частиц. Разработана математическая модель исследуемого явления, 
описываемая уравнениями Фок\-ке\-ра--План\-ка и Пуассона. Решение получено двумя методами:\linebreak 
статистическим и ко\-неч\-но-раз\-ност\-ным на основе\linebreak сформированных алгоритмов. Приведены 
резуль-\linebreak таты численного моделирования при различных\linebreak характерных параметрах задачи.
 Из  проведенных 
вычислительных экспериментов вытекает, что искомые величины: напряженность 
электрического поля, плотности токов частиц на зонд, концентрации частиц вблизи зонда~--- как по 
характеру зависимости, так и по числовым значениям совпадают. При применении метода 
      Мон\-те-Кар\-ло момент установления наступает быстрее по сравнению с конечно-разностным 
методом, однако конечно-разностный метод позволяет получить более наглядные результаты.

{\small\frenchspacing
{%\baselineskip=10.8pt
\addcontentsline{toc}{section}{Литература}
\begin{thebibliography}{99}

\bibitem{1-k}
\Au{Alexeff I., Anderson T.}
Experimental and theoretical results with plasma antenna~// IEEE Trans. Plasma Sci., 2006. Vol.~34. 
No.\,2. P.~166--172.

\bibitem{2-k}
\Au{Сысун В.\,И.}
Сильноионизованная низкотемпературная плазма в приборах электронной техники: Методы 
исследования, свойства, применение. Дисс. \ldots д-ра физ.-мат. наук в форме науч. докл.: 
01.04.08.~--- Пет\-ро\-за\-водск, 1996.

\bibitem{3-k}
\Au{Тухас В.\,А.}
Методология создания средств измерений и испытаний на устойчивость к кондуктивным помехам~// 
Мат-лы VI Междунар. симп. по электромагнитной совместимости и 
электромагнитной экологии.~--- СПб., 2005. С.~231--234.

\bibitem{4-k}
\Au{Гудзенко Л.\,И., Яковленко С.\,И.}
Плазменные лазеры.~--- М.: Атомиздат, 1978.  256~с.

\bibitem{5-k}
\Au{Звелто О.}
Принципы лазеров.~--- М.: Мир, 1990.  560~с.

\bibitem{6-k}
\Au{Сысун В.\,И., Хромой Ю.\,Д.}
Расширение канала мощного импульсного разряда в парах ртути~// Электронная техника, 1974. 
Сер.~4. Вып.~10. С.~80--85. 

\bibitem{7-k}
\Au{Винклер Дж.\,Р.}
Искусственные пучки частиц в космической плазме.~--- М.: Мир, 1985.  451~с.

\bibitem{8-k}
\Au{Bernstein I.\,B., Rabinowitz I.\,N.}
Theory of electrostatic probes in low-density plasma~// Phys. Fluids, 1959. Vol.~2. No.\,2. P.~112--121. 

\bibitem{9-k}
\Au{Альперт Я.\,Л., Гуревич А.\,В., Питаевский~Л.\,П.}
Искусственные спутники в разреженной плазме.~--- М.: Наука, 1964.  282~с.

\bibitem{10-k}
\Au{Чан П., Тэлбот Л., Турян~К.}
Электрические зонды в неподвижной и движущейся плазме.~--- М.: Мир, 1978.  202~с.

\bibitem{11-k}
\Au{Алексеев Б.\,В., Котельников В.\,А.}
Зондовый метод диагностики плазмы.~--- М.: Энергоатомиздат, 1989.  240~с.

\bibitem{12-k}
\Au{Пантелеев А.\,В., Кудрявцева И.\,А.}
Формирование математической модели двухкомпонентной плазмы с учетом столкновений 
заряженных частиц в случае плоского зонда~// Теоретические вопросы вычислительной техники и 
программного обеспечения: Межвузовский сб. научн. тр.~--- М.: МИРЭА, 2006. С.~11--21.

\bibitem{13-k}
\Au{Олдер Б.}
Вычислительные методы в физике плазмы.~--- М.: Мир, 1974.  111~с.

\bibitem{14-k}
\Au{Montgomery D.\,C., Tidman D.\,A.}
Plasma kinetic theory.~--- New York, 1964. 

\bibitem{15-k}
\Au{Кудрявцева И.\,А., Пантелеев А.\,В.}
Применение метода Мон\-те-Кар\-ло для анализа поведения двухкомпонентной плазмы с учетом 
столкновений между заряженными частицами~// Теоретические вопросы\linebreak
вычислительной техники и 
программного обеспечения: Межвузовский сб. научн. тр.~--- М.: МИРЭА, 2008. С.~122--128. 

\bibitem{16-k}
\Au{Семенов В.\,В., Пантелеев А.\,В., Руденко~Е.\,А., Бор\-та\-ков\-ский~А.\,С.}
Методы описания, анализа и синтеза нелинейных систем управления.~--- М.: МАИ, 1993.  312~с.

\bibitem{17-k}
\Au{Киреев В.\,И., Пантелеев А.\,В.}
Численные методы в примерах и задачах.~--- М.: Высшая школа, 2006.  480~с.

\bibitem{18-k}
\Au{Белоцерковский О.\,М., Давыдов~Ю.\,М.}
Метод крупных частиц в газовой динамике. Вычислительный эксперимент.~--- М.: Наука, 
Физматгиз, 1982.

\label{end\stat}

\bibitem{19-k}
\Au{Вержбицкий В.\,М.}
Основы численных методов.~--- М.: Высшая школа, 2002.  840~с.
 \end{thebibliography}
}
}


\end{multicols}         %13
\include{gonch+zatsman}  %14
\def\stat{gorshenin}

\def\tit{ЗАШУМЛЕНИЕ ДАННЫХ КОНЕЧНЫМИ СМЕСЯМИ НОРМАЛЬНЫХ 
И~ГАММА-РАСПРЕДЕЛЕНИЙ\\ С~ПРИМЕНЕНИЕМ К~ЗАДАЧЕ ОКРУГЛЕНИЯ НАБЛЮДЕНИЙ$^*$}

\def\titkol{Зашумление данных конечными смесями нормальных 
и~гамма-распределений с~применением к~задаче округления} % наблюдений}

\def\aut{А.\,К.~Горшенин$^1$}

\def\autkol{А.\,К.~Горшенин}

\titel{\tit}{\aut}{\autkol}{\titkol}

\index{Горшенин А.\,К.}
\index{Gorshenin A.\,K.}


{\renewcommand{\thefootnote}{\fnsymbol{footnote}} \footnotetext[1]
{Работа выполнена при поддержке РНФ (проект 18-71-00156).}}


\renewcommand{\thefootnote}{\arabic{footnote}}
\footnotetext[1]{Институт проблем информатики Федерального исследовательского центра 
<<Информатика и~управление>> Российской академии наук, \mbox{agorshenin@frccsc.ru}}

\vspace*{-12pt}




\Abst{Во многих реальных задачах проводится статистический анализ данных, 
содержащих дополнительные ошибки измерения, в~том числе в~виде округления, 
что в~ряде ситуаций может приводить к~достаточно существенным искажениям. 
В~настоящей статье для одной из возможных моделей округления получены оценки 
для неизвестного математического ожидания наблюдений в~предположении, что 
исходные данные дополнительно зашумлены с~по\-мощью случайных величин, 
име\-ющих распределения типа конечных смесей нормальных и~гам\-ма-за\-ко\-нов. 
Построены доверительные интервалы для неизвестного математического ожидания 
с~использованием уточненной оценки для дисперсии целой части случайной величины. 
Обсуждается алгоритм определения значения параметра для искусственного шума, 
добавление которого к~исходным данным способствует повышению качества работы 
метода скользящего разделения смесей.}

\KW{зашумленные данные; округленные наблюдения; конечные смеси нормальных 
распределений; конечные смеси гам\-ма-рас\-пре\-де\-ле\-ний; доверительные интервалы;  
метод скользящего разделения смесей}

\DOI{10.14357/19922264180304}
  
\vspace*{-4pt}


\vskip 10pt plus 9pt minus 6pt

\thispagestyle{headings}

\begin{multicols}{2}

\label{st\stat}


\section{Введение}

Во многих реальных задачах данные, являющиеся непрерывными по своей сути, 
регистрируются с~помощью инструментов, вносящих дополнительные ошибки 
измерения, в~том чис\-ле в~виде округления. Таким образом, статистический 
анализ проводится не для исходных, а для преобразованных некоторым 
случайным образом наблюдений, что в~ряде ситуаций может приводить к~достаточно
 существенным искажениям.

Для преодоления данной проблемы развивались различные подходы, в~том числе 
на основе смешанных моделей (см., например, статью~\cite{Wright2003}, в~которой 
различные компоненты  используются для пред\-став\-ле\-ния уровней округления). 
В~работе~\cite{Bai2009} приводятся результаты для моделей авторегрессии и~скользящего 
среднего для округленных данных, а~в~статье~\cite{Zhang2010} эти результаты 
развиваются и~исследуются их асимптотические свойства. 
В~статье~\cite{Zhao2012} исследован метод оценивания па\-ра\-мет\-ров конечных смесей 
вероятностных распределений (в~том чис\-ле, и~многомерных) 
на основе использования EM (expectation-maximization) 
алгоритма~\cite{Korolev2011-i} с~\mbox{целью} получения состоятельных 
и~асимптотически нормальных оценок.

В настоящей статье развиваются результаты для моделей округления, 
описанных в~работах~\cite{Ushakov2015,Ushakov2017a,Ushakov2017b}. 
В~их рамках будут получены оценки для неизве\-ст\-ного математического ожидания 
наблюдений в~предположении, что исходные данные зашумлены с~по\-мощью случайных 
величин, имеющих распределения типа конечных смесей нормальных и~гам\-ма-за\-ко\-нов. 
Это позволяет учесть большее количество случайных факторов, влия\-ющих на величину 
<<дополнительной>> ошибки. Также будут построены доверительные интервалы для 
неизвестного математического ожидания. Выражения для гам\-ма-рас\-пре\-де\-ле\-ний 
получены впервые. Также обсуждается алгоритм определения значения па\-ра\-мет\-ра для 
искусственного шума, добавление которого к~исходным данным способствует 
повышению качества работы метода скользящего разделения смесей~\cite{Gorshenin2016}.

\vspace*{-12pt}

\section{Предположения и~базовые отношения}

Для сокращения формулировок теорем в~сле\-ду\-ющих разделах сделаем ряд 
предположений, на которые будем ссылаться в~дальнейшем. Итак, пусть:
\begin{itemize}
\item[(A)] $X_1,X_2,\ldots$~--- независимые одинаково распределенные 
случайные величины с~неизвестным математическим ожиданием ${\sf E}_X\hm<+\infty$;
\item[(B)] $\varepsilon_1,\varepsilon_2,\ldots$~--- независимые одинаково 
распределенные случайные величины с~математическим ожиданием 
${\sf E}_\varepsilon\hm<+\infty$; %\label{B}
\item[(C)] $X_1,X_2,\ldots$ и~$\varepsilon_1,\varepsilon_2,\ldots$ 
являются независимыми;
\item[(D)] $Y_j=\left[X_j+\varepsilon_j+1/2\right]$ для всех $j\hm=1,2,\ldots$ 
представляют собой округление значения суммы случайных величин $X_j\hm+\varepsilon_j$ 
до ближайшего целого сверху (при этом запись~$[\cdot]$ соответствует целой 
части выражения).
\end{itemize}

В рамках данных предположений в~статье будут рассмотрены вопросы качества 
приближения неизвестного математического ожидания~${\sf E}_X$ для исходных данных 
в~ситуации, когда наблюдения для анализа получены с~аддитивной ошибкой c известными 
распределениями (см.\ предположение~(B)) и~дополнительно округляются до 
ближайшего целого (см.\ предположение~(D)).

Заметим, что в~силу усиленного закона больших чисел справедливы следующие выражения:
\begin{multline}
\fr{1}{n}\sum\limits_{j=1}^n Y_j\xrightarrow[n\to\infty]{\text{п.н.}}
{\sf E}_Y\equiv\mathbb{E}\left[X_1+\varepsilon_1+\fr{1}{2}\right]={}\\
{}=\mathbb{E}\left(X_j+\varepsilon_j+\fr{1}{2}\right)-\mathbb{E}
\left\{X_j+\varepsilon_j+\fr{1}{2}\right\}={}\\
{}={\sf E}_X+{\sf E}_\varepsilon+\fr{1}{2}-\mathbb{E}\left\{X_j+\varepsilon_j+\fr{1}{2}\right\}. 
\label{Law}
\end{multline}

Запись $\{\cdot\}$ в~формуле~\eqref{Law} соответствует дробной 
части выражения, а~п.н.\ обозначает сходимость в~смысле почти наверное.

Для доказательства результатов в~дальнейшем потребуется следующее 
представления для дробной части  абсолютно непрерывной случайной величины~$Z$ 
с~абсолютно  интегрируемой характеристической функцией~$\varphi_Z(t)$
 (см., например, Лемму~4 в~работе~\cite{Ushakov2017b}):
\begin{equation}
\label{Fract}
\mathbb{E}\{Z\}=\fr{1}{2}-\sum\limits_{n=1}^\infty 
\fr{\mathrm{Im}\left (\varphi_Z(2\pi n)\right)}{\pi n}\,.
\end{equation}

Через $\mathrm{Im}\,(\cdot)$ в~формуле~\eqref{Fract} обозначена мнимая часть 
соответствующей функции.

При построении доверительных интервалов в~дальнейшем будет 
использована следующая оценка, справедливая для любой случайной величины~$Z$:
\begin{equation}
\mathbb{D}[Z]\leqslant \left(\sqrt{\mathbb{D} Z}+\fr{1}{2}\right)^2.
\label{Var}
\end{equation}
Она может быть проверена непосредственно с~учетом представления 
$\mathbb{D} [Z]\hm=\mathbb{D}\left(Z\hm-\{Z\}\right)$, неравенства 
Ко\-ши--Бу\-ня\-ков\-ско\-го для ковариации и~соотношения 
 $\mathbb{D}\{Z\}\hm\leqslant 1/4$, справедливого для любой случайной величины~$Z$ 
 (см., например, статью~\cite{Ushakov2017b}). Отметим, что данная оценка 
 является более точной по сравнению с~использованным для аналогичных 
 целей в~работе~\cite{Ushakov2017b} соотношением 
 $\mathbb{D} [Z]\hm\leqslant 2\mathbb{D} Z\hm+1/2$. Действительно,
\begin{equation*}
2\mathbb{D} Z+\fr{1}{2}-\left(\sqrt{\mathbb{D} Z}+\fr{1}{2}\right)^2=
\left(\sqrt{\mathbb{D} Z}-\fr{1}{2}\right)^2\geqslant0\,,
\end{equation*}
причем для всех $\sqrt{\mathbb{D} Z}\hm\neq {1}/{2}$ 
данное неравенство является строгим.

\section{Конечные смеси нормальных законов}

Для случайной величины~$X$, имеющей распределение типа 
конечной смеси нормальных законов~\cite{Korolev2011-i} с~параметрами 
${\bf a}\hm=(a_1,\ldots, a_k)$, $a_j\hm\in \mathbb{R}$, 
$\boldsymbol{\sigma}\hm=(\sigma_1,\ldots, \sigma_k)$, 
$\sigma_j\hm>0$, ${\bf p}\hm=(p_1,\ldots, p_k)$, $p_j\hm\geqslant 0$, 
$\sum\nolimits_{j=1}^{k}p_j\hm=1$, плот\-ность которого задается выражением
\begin{equation}
f_X(x)=\sum\limits_{j=1}^{k}\fr{p_j}{\sigma_j\sqrt{2\pi}}\,e^{-(x-a_j)^2/(2\sigma_j^2)}\,,
\label{FinNormMixt}
\end{equation}
характеристическая функция имеет вид:
\begin{equation}
\varphi_X(t)=\int\limits_{-\infty}^{+\infty}\!\!e^{itx} f_X(x)\, dx = 
\sum\limits_{j=1}^{k}p_j e^{ita_j-\sigma_j^2 t^2/2}.
\label{ChiFinNormMixt}
\end{equation}

Абсолютная интегрируемость  $\varphi_X(t)$ вытекает из свойств 
характеристической функции нормального распределения. 
Заметим, что в~точке $t\hm=2\pi n$ выражение~\eqref{ChiFinNormMixt} принимает 
сле\-ду\-ющий вид:
\begin{equation}
\label{ChiFinNormMixt2npi}
\varphi_X(2\pi n)= \sum\limits_{j=1}^{k}p_j e^{-2\pi^2 \sigma_j^2 n^2}\,.
\end{equation}

Рассмотрим вопрос точности оценивания неизвестного математического ожидания~${\sf E}_X$ 
при до\-бав\-ле\-нии зашумления.

\smallskip

\noindent
\textbf{Теорема~1.}\ 
\textit{Пусть выполнены предположения}~(A)--(D), 
\textit{причем случайные величины~$\varepsilon_j$, $j\hm=1,2,\ldots$, 
имеют распределение типа конечной $k$-ком\-по\-нент\-ной смеси нормальных законов 
вида}~\eqref{FinNormMixt} \textit{с~па\-ра\-мет\-ра\-ми~${\bf a}$, $\boldsymbol{\sigma}$ 
и~${\bf p}$. Тогда}
\begin{equation}
\label{Th1Eq}
\left\lvert {\sf E}_Y-{\sf E}_X\right\rvert \leqslant 
A+\fr{1}{\pi}\left(1+\fr{1}{4\pi^2\sigma^2}\right)e^{-2\pi^2\sigma^2}\,, 
\end{equation}
\textit{где} $A=\max(|a_1|,\ldots,|a_k|)$, $\sigma\hm=\min(\sigma_1,\ldots,\sigma_k)$.

\smallskip


\noindent
Д\,о\,к\,а\,з\,а\,т\,е\,л\,ь\,с\,т\,в\,о\,.\ \
С~учетом пред\-став\-ле\-ний~\eqref{Law},~\eqref{Fract} и~\eqref{ChiFinNormMixt2npi}, 
ограниченности модуля характеристической функции, а~также не\-за\-ви\-си\-мости 
случайных величин~$X_j$ и~$\varepsilon_j$ имеем:
\begin{multline*}
\left\lvert {\sf E}_Y-{\sf E}_X\right\rvert =
\left\lvert {\sf E}_\varepsilon+\fr{1}{2}-\mathbb{E}\left\{X_j+
\varepsilon_j+\fr{1}{2}\right\}\right\rvert={}\\
{}=\left\lvert {\sf E}_\varepsilon+\sum\limits_{n=1}^\infty
\fr{\mathrm{Im} \left(\varphi_{X_j}(2\pi n)\varphi_{\varepsilon_j}(2\pi n)
\varphi_{1/2}(2\pi n)\right)}{\pi n}\right\rvert={}\\
=\left\lvert 
\vphantom{\fr{(-1)^n\sum\nolimits_{j=1}^{k}p_j e^{-2\pi^2 \sigma_j^2 n^2} 
\mathrm{Im} \left(\varphi_{X_j}(2\pi n)\right)}{\pi n}}
{\sf E}_\varepsilon+{}\right.\\
\left.{}+\sum\limits_{n=1}^\infty
\fr{\mathrm{Im} \left(\varphi_{X_j}(2\pi n) 
\sum\nolimits_{j=1}^{k}p_j e^{-2\pi^2 \sigma_j^2 n^2} 
e^{\pi n}\right)}{\pi n}\right\rvert={}\\
{}=\left\lvert 
\vphantom{\fr{(-1)^n\sum\nolimits_{j=1}^{k}p_j e^{-2\pi^2 \sigma_j^2 n^2} 
\mathrm{Im} \left(\varphi_{X_j}(2\pi n)\right)}{\pi n}}
{\sf E}_\varepsilon+{}\right.\\
\left.{}+\sum\limits_{n=1}^\infty
\fr{(-1)^n\sum\nolimits_{j=1}^{k}p_j e^{-2\pi^2 \sigma_j^2 n^2} 
\mathrm{Im} \left(\varphi_{X_j}(2\pi n)\right)}{\pi n}\right\rvert\leqslant{}\\
{}\leqslant \left\lvert {\sf E}_\varepsilon\right\rvert+\left\lvert\
\sum\limits_{j=1}^{k}p_j\sum\limits_{n=1}^\infty 
\fr{1}{\pi n} e^{-2\pi^2 \sigma_j^2 n^2}\right\rvert\leqslant {}\\
\\
{}\leqslant
\max\left(|a_1|,\ldots,|a_k|\right)+{}\\
{}+\sum\limits_{j=1}^{k} 
\fr{p_j}{\pi} \left(\!1+\fr{1}{4\pi^2\sigma_j^2}\!\right)\!e^{-2\pi^2\sigma_j^2}\leqslant{}\\
{}\leqslant
A+\fr{1}\pi\left(1+\fr{1}{4\pi^2\sigma^2}\right)e^{-2\pi^2\sigma^2}\,.
\end{multline*}

Справедливость использованной оценки 
\begin{equation*}
\sum\limits_{n=1}^\infty
\fr{e^{-2\pi^2 \sigma_j^2 n^2}}{n}\leqslant 
\left(1+\fr{1}{4\pi^2\sigma_j^2}\right)e^{-2\pi^2\sigma_j^2}
\end{equation*}
может быть проверена непосредственно (например, см.\ доказательство Теоремы~6 
в~статье~\cite{Ushakov2017b}).~\hfill$\square$

\smallskip

\noindent
\textbf{Замечание~1.}
В~случае, если зашумление производится нормально распределенными случайными 
величинами c нулевыми средними (т.\,е.\ в~формуле~\eqref{Th1Eq} необходимо считать 
$A\hm=0$, $k\hm=1$), то Тео\-ре\-ма~1 совпадает с~результатом, 
полученным в~работе~\cite{Ushakov2017b}.


\smallskip

Рассмотрим вопросы построения доверительного интервала для неизвестного 
математического ожидания~${\sf E}_X$ в~предположении, что случайные величины~$X_j$ не 
содержат ошибок измерения, а~все погрешности учтены исключительно в~за\-шум\-ля\-ющих 
элементах~$\varepsilon_j$.

\smallskip

\noindent
\textbf{Теорема~2.}\ 
\textit{Пусть выполнены предположения}~(A)--(D), 
\textit{причем случайные величины~$\varepsilon_j$, $j\hm=1,2,\ldots$, имеют 
распределение типа конечной $k$-ком\-по\-нент\-ной смеси нормальных законов 
вида}~\eqref{FinNormMixt} \textit{с~параметрами~${\bf a}$, $\boldsymbol{\sigma}$ 
и~${\bf p}$, а~случайные величины} $X_j\stackrel{\text{п.н.}}{=}{\sf E}_X$, $j\hm=1,2,\ldots$ 
\textit{Тогда доверительный интервал для~${\sf E}_X$ при условии $0\hm<\alpha\hm<1$ имеет вид}:
\begin{equation} 
\label{Th2Eq}
\hat{{\sf E}}_X - f({\bf a},\boldsymbol{\sigma},\alpha,n) 
\leqslant {\sf E}_X \leqslant  \hat{{\sf E}}_X + f({\bf a},\boldsymbol{\sigma},\alpha,n),
\end{equation}
\textit{где}

\vspace*{-2pt}

\noindent
\begin{align}
\hat{{\sf E}}_X&=\fr{1}{n} \sum\limits_{j=1}^{n} Y_j\,; \label{Th2hatE}\\
f({\bf a},\boldsymbol{\sigma},\alpha,n)&=
\fr{z_{1-{\alpha}/2}}{\sqrt{n}} \left(\sqrt{A^2+\Sigma^2}+\fr{1}{2}\right) +{}\notag\\
&{}+A+\fr{1}\pi\left(1+\fr{1}{4\pi^2\sigma^2}\right)e^{-2\pi^2\sigma^2}\,;
  \label{Th2f}
\end{align}
\textit{$z_{1-{\alpha}/2}$~--- $\left(1-{\alpha}/2\right)$-кван\-тиль 
стандартного нормального распределения; $A\hm=\max(|a_1|,\ldots,|a_k|)$; 
$\Sigma\hm=\max(\sigma_1,\ldots,\sigma_k)$; $\sigma\hm=\min(\sigma_1,\ldots,\sigma_k)$}. 


\smallskip

\noindent
\noindent
Д\,о\,к\,а\,з\,а\,т\,е\,л\,ь\,с\,т\,в\,о\,.\ \
Из центральной предельной тео\-ре\-мы с~учетом условия~(A) следует, 
что величина~$\hat{{\sf E}}_X$~\eqref{Th2hatE} асимптотически нормальна с~математическим 
ожиданием 
\begin{equation}
{\sf E}_Y\equiv \mathbb{E}\left[{\sf E}_X+\varepsilon_1+\fr{1}{2}\right] \label{EY}
\end{equation}
и дисперсией
\begin{equation}
\fr{1}{n} {\sf D}_Y\equiv \fr{1}{n}\mathbb{D}\left[{\sf E}_X+\varepsilon_1+
\fr{1}{2}\right]. \label{DY}
\end{equation}

Воспользовавшись оценкой~\eqref{Var}, получим:

\vspace*{-2pt}

\noindent
\begin{multline*}
{\sf D}_Y \leqslant  \left(\sqrt{\mathbb{D} \left({\sf E}_X+\varepsilon_1+\fr{1}{2}\right)}+
\fr{1}{2}\right)^2={}\\
{}=
\left(\sqrt{\mathbb{D}\varepsilon_1}+\fr{1}{2}\right)^2= {}\\
{}= \left(\sqrt{\sum\limits_{j=1}^{k}p_j\left(\left(a_j-\sum\limits_{t=1}^{k}
p_t a_t\right)^2+\sigma_j^2\right)}+\fr{1}{2}\right)^2\leqslant {}\\ 
{}\leqslant \left(\sqrt{A^2+\Sigma^2}+\fr{1}{2}\right)^2\,.
\end{multline*}
Тогда доверительный интервал уровня $1\hm-\alpha$ для математического ожидания~${\sf E}_Y$ 
имеет вид:
\begin{equation*}
\mathbb{P}\left(\left\lvert \hat{{\sf E}}_X-{\sf E}_Y\right\rvert \leqslant 
\fr{z_{1-{\alpha}/2}}{\sqrt{n}} 
\left(\sqrt{A^2+\Sigma^2}+\fr{1}{2}\right)\right)\geqslant 1-\alpha\,.
\end{equation*}

\begin{table*}[b]\small
\begin{center}

\begin{tabular}{|c|c|c|c|c|c|c|c|}
\multicolumn{7}{p{100mm}}{Численные решения уравнений~\eqref{f1} и~\eqref{f2} относительно 
параметра~$\sigma$ для некоторых значений~$n$ и~$\alpha$}\\
\multicolumn{7}{c}{\ }\\[-6pt]
\hline
\multicolumn{1}{|c|}{Размер}  & \multicolumn{2}{c|}{Уровень $\alpha=0{,}1$}& 
\multicolumn{2}{c|}{Уровень $\alpha=0{,}05$}& 
\multicolumn{2}{c|}{Уровень $\alpha=0{,}01$}\\
\cline{2-7}
\multicolumn{1}{|c|}{выборки $n$}&$\sigma_1$&$\sigma_2$&$\sigma_1$&$\sigma_2$&$\sigma_1$&$\sigma_2$\\
\hline
$\hphantom{000}100$&$0{,}4302$&$0{,}435$&$0{,}419$&$0{,}425$&$0{,}4002$&$0{,}408$\\
%\hline
$\hphantom{000}200$&$0{,}452$&$0{,}455$ &$0{,}441$&$0{,}445$&$0{,}424$&$0{,}429$\\
%\hline
$\hphantom{00}1000$&$0{,}499$&$0{,}499$ &$0{,}489$&$0{,}489$&$0{,}473$&$0{,}475$\\
%\hline
$\hphantom{0}10000$&$0{,}558$&$0{,}556$ &$0{,}549$&$0{,}547$&$0{,}536$&$0{,}534$\\
%\hline
$100000$&$0{,}611$&$0{,}607$ &$0{,}603$&$0{,}599$&$0{,}591$&$0{,}588$\\
\hline
\end{tabular}
\end{center}
\end{table*}


\noindent
Откуда следует справедливость соотношения~\eqref{Th2Eq} c~уче\-том 
очевидного неравенства

\pagebreak

\noindent
\begin{equation*}
\left\lvert \hat{{\sf E}}_X-{\sf E}_X\right\rvert \leqslant 
\left\lvert \hat{{\sf E}}_X-{\sf E}_Y\right\rvert +\left\lvert {\sf E}_Y-{\sf E}_X\right\rvert 
\end{equation*}
и оценки~\eqref{Th1Eq} из Теоремы~1.~\hfill$\square$

\smallskip

\noindent
\textbf{Замечание~2.}
В~работе~\cite{Gorshenin2016} было продемонстрировано повышение точ\-ности 
работы метода скользящего разделения конечных нормальных смесей за счет 
введения дополнительной компоненты, имеющей нормальное 
распределение $\mathcal{N}(0,\sigma^2)$ с~математическим ожиданием, равным~$0$, 
и~стандартным отклонением~$\sigma$. При этом была отмечена сложность выбора 
параметра~$\sigma$ для сохранения структуры выборки, близкой к~исходной. 
Результат Теоремы~2 может быть использован с~данной целью, если положить $k\hm=1$, 
$a_j\hm=0$ для всех $j\hm=1,2,\ldots$ и~выбирать величину~$\sigma$ как 
минимизирующую длину доверительного интервала~\eqref{Th2Eq}. Для 
этого необходимо найти производную функции $f(0,\sigma,\alpha,n)$~\eqref{Th2f} 
и~численно решить уравнение
\begin{multline}
f_\sigma'(0,\sigma,\alpha,n)\equiv \fr{z_{1-{\alpha}/2}}{\sqrt{n}} - {}\\
{}-
e^{-2\pi^2\sigma^2}\left(4\pi\sigma+\fr{1}{2\pi^3\sigma^3}+
\fr{1}{\pi\sigma}\right)=0
\label{f1}
\end{multline}
относительно неизвестного параметра~$\sigma$ при выбранных значениях величин~$n$ 
и~$\alpha$. В~качестве альтернативы можно использовать вид доверительного интервала 
из статьи~\cite{Ushakov2017b}, полученный с~помощью неравенства $\mathbb{D} [Z]
\hm\leqslant 2\mathbb{D} Z\hm+{1}/{2}$, и~искать решение уравнения вида:
\begin{multline}
\hspace*{-2.90578pt}\fr{2\sigma z_{1-{\alpha}/2}}{\sqrt{n (2\sigma^2+{1}/{2})}} -
 e^{-2\pi^2\sigma^2}\left(4\pi\sigma+\fr{1}{2\pi^3\sigma^3}+
 \fr{1}{\pi\sigma}\right)={}\\
 {}=0\,.\label{f2}
\end{multline}

Примеры найденных значений~$\sigma$ для типичных размеров выборок в~методе 
скользящего разделения смесей (учитываются как возможная ширина окна, 
так и~общее количество наблюдений в~анализируемом ряде) приведены в~таблице 
(использован метод оптимизации \verb"Trust-Region Dogleg" пакета \verb"MATLAB" 
c~настройками по умолчанию), в~которой через~$\sigma_1$ обозначено приближенное  
решение уравнения~\eqref{f1}, a~через $\sigma_2$~--- уравнения~\eqref{f2}.


Проверка практической эффективности данного подхода в~качестве 
критерия выбора параметров зашумляющего распределения для повышения 
точности работы метода скользящего разделения смесей может быть отмечена 
как задача для дальнейших исследований.


\section{Конечные смеси гамма-распределений}

Для случайной величины~$X$, имеющей распределение типа конечной смеси 
гам\-ма-рас\-пре\-де\-ле\-ний с~параметрами ${\bf r}\hm=(r_1,\ldots, r_k)$,
 $r_j\hm>0$, $\boldsymbol{\lambda}\hm=(\lambda_1,\ldots, \lambda_k)$, $\lambda_j\hm>0$, 
 ${\bf p}\hm=(p_1,\ldots, p_k)$, $p_j\hm\geqslant 0$, $\sum\nolimits_{j=1}^{k}p_j\hm=1$, 
 плот\-ность которого задается выражением
\begin{equation}
f_X(x)=\sum\limits_{j=1}^{k}p_j\fr{\lambda_j^{r_j} e^{-\lambda_j x}}
{\Gamma(r_j)}\,x^{r_j-1}\,,
\label{FinGammaMixt}
\end{equation}
характеристическая функция имеет следующий вид:
%характеристическая функция задается следующим выражением:
\begin{equation}
\varphi_X(t)=\!\int\limits_{-\infty}^{+\infty}\!\!\!e^{itx} f_X(x)\, dx = \!
\sum\limits_{j=1}^{k}p_j \left(\!1-\fr{it}{\lambda_j}\right)^{-r_j}\!.\!
\label{ChiFinGammaMixt}
\end{equation}

Отметим, что подобные модели зашумления разумно использовать в~случае, 
если известно, что данные сосредоточены на положительной полуоси, например 
при анализе различных информационных потоков (см., в~част\-ности, 
 работу~\cite{Gorshenin2013}). 

Проверим абсолютную интегрируемость функции $\varphi_X(t)$~\eqref{ChiFinGammaMixt}. 
Имеем:
\begin{multline*}
\int\limits_{-\infty}^{+\infty}\left\lvert\varphi_X(t)\right\rvert\, dt\leqslant 
\sum\limits_{j=1}^{k}p_j \int\limits_{-\infty}^{+\infty}\left\lvert \left(
1-\fr{it}{\lambda_j}\right)^{-r_j}\right\rvert \, dt={}\\
{}=\sum\limits_{j=1}^{k}p_j \int\limits_{-\infty}^{+\infty} \left\lvert\left(
\fr{\lambda_j(\lambda_j+it)}{\lambda_j^2+t^2}\right)^{r_j}\right\rvert\, dt \leqslant{}\\
{}\leqslant\sum\limits_{j=1}^{k}p_j \lambda_j \int\limits_{-\infty}^{+\infty}\left(
1+y^2\right)^{-{r_j}/{2}}\, dy\,.
\end{multline*}

Подынтегральное выражение при $r_j\hm\geqslant 2$ может быть оценено сверху 
функцией $1/({1+y^2})$, при этом соответствующий интеграл равен~$\pi$, что влечет 
абсолютную интегрируемость характеристической функции для конечной смеси 
гам\-ма-рас\-пре\-де\-ле\-ний. Поэтому в~дальнейшем будем предполагать,
 что $r_j\hm\geqslant 2$ для всех возможных значений $j\hm=1,2,\ldots$

Рассмотрим вопрос точ\-ности оценивания неизвестного математического ожидания ${\sf E}_X\hm>0$ 
при добавлении зашумления.

\smallskip

\noindent
\textbf{Теорема~3.}
\textit{Пусть выполнены предположения}~(A)--(D), 
\textit{причем случайные величины~$\varepsilon_j$, $j\hm=1,2,\ldots$, имеют 
распределение типа конечной $k$-ком\-по\-нент\-ной смеси 
гам\-ма-рас\-пре\-де\-ле\-ний вида}~\eqref{FinGammaMixt} 
\textit{с~па\-ра\-мет\-ра\-ми~${\bf r}$, $\boldsymbol{\lambda}$ и~${\bf p}$. Тогда}
\begin{equation}
\label{Th3Eq}
\left\lvert {\sf E}_Y-{\sf E}_X\right\rvert \leqslant \fr{R}{\lambda}+
\fr{\Lambda^{R}}{2^{r}\pi^{r+1}}\left(1+\frac1{r}\right)\,,
\end{equation}
\textit{где} $r=\min(r_1, \ldots,r_k)$; $R\hm=\max(r_1, \ldots,r_k)$; 
$\lambda\hm=\max(\lambda_1, \ldots,\lambda_k)$; 
$\Lambda\hm=\max(\lambda_1, \ldots,\lambda_k)$.

\smallskip

\noindent
Д\,о\,к\,а\,з\,а\,т\,е\,л\,ь\,с\,т\,в\,о\,.\ \
С~учетом пред\-став\-ле\-ний~\eqref{Law} и~\eqref{Fract}, ограниченности 
модуля характеристической функции, перехода от тригонометрической к~показательной 
записи комплексных чисел, а~также независимости случайных величин~$X_j$ 
и~$\varepsilon_j$ \mbox{имеем}:
\begin{multline*}
\left\lvert {\sf E}_Y-{\sf E}_X\right\rvert
\leqslant \left\lvert {\sf E}_\varepsilon\right\rvert+ {}\\
{}+\left\lvert\sum\limits_{n=1}^\infty
\left(
(-1)^n\mathrm{Im} \left(\sum\limits_{j=1}^{k}p_j \varphi_{X_j}(2\pi n)\left(
\vphantom{\fr{2\pi n}{\lambda_j}}
1-{}\right.\right.\right.\right.\\
\left.\left.\left.\left.{}-i\left(\fr{2\pi n}{\lambda_j}\right)\right)^{-r_j}\right)
\Bigg/ ({\pi n})
\vphantom{\sum\limits_{j=1}^{k}}
\right)\right\rvert={}\\
{}=\left\lvert {\sf E}_\varepsilon\right\rvert+ 
\left\lvert\sum\limits_{n=1}^\infty
\left(\!(-1)^n\mathrm{Im} \!\left(\sum\limits_{j=1}^{k}p_j \left(\!
1+\fr{4\pi^2 n^2}{\lambda_j^2}\right)^{- {r_j}/2}\!\times{}\right.\right.\right.\hspace*{-2.8663pt}\\
\left.\left.\left.{}\times \varphi_{X_j}(2\pi n)\,
e^{-ir_j\mathrm{arctan}\,({{t}/{\lambda_j}})}\right)
\Bigg/
({\pi n})
\vphantom{\left(
1+\fr{4\pi^2 n^2}{\lambda_j^2}\right)^{- {r_j}/2}}
\right)\right\rvert\leqslant{}\\
{}\leqslant \left\lvert {\sf E}_\varepsilon\right\rvert+\sum\limits_{j=1}^{k}
p_j\sum\limits_{n=1}^\infty\fr{1}{\pi n}\left(
1+\fr{4\pi^2 n^2}{\lambda_j^2}\right)^{-{r_j}/2}\leqslant{}\\
{}\leqslant  \fr{R}\lambda + \sum\limits_{j=1}^{k}p_j
\sum\limits_{n=1}^\infty\left(\fr{1}{\pi n}\,
\fr{\lambda_j^{r_j}}{(2\pi)^{r_j} n^{r_j}}\right)\leqslant {}
\\
{}\leqslant  \fr{R}{\lambda} + \sum\limits_{j=1}^{k}p_j 
\fr{\lambda_j^{r_j}}{2^{r_j}\pi^{r_j+1}}\left(1+\int\limits_{1}^{\infty}
\fr{1}{ x^{r_j+1}}\,dx\right)
\leqslant{}\\
{}\leqslant \fr{R}{\lambda}+\fr{\Lambda^{R}}{2^{r}\pi^{r+1}}\left(1+\fr{1}{r}\right).
\end{multline*}

При переходе от суммы к~интегралу используется факт убывания функции как переменной~$n$ 
(или~$x$).~\hfill$\square$


\smallskip

\noindent
\textbf{Замечание~3.}\
Теорема~3 описывает соответ\-ст\-ву\-ющий результат для гам\-ма-рас\-пре\-де\-лен\-ных 
за\-шум\-ля\-ющих случайных величин, если положить $k\hm=1$ в~выражении~\eqref{Th3Eq}. 
При этом, очевидно, $r\hm\equiv R$ и~$\lambda\hm\equiv \Lambda$.


\smallskip

Рассмотрим вопросы построения доверительного интервала для неизвестного 
математического ожидания ${\sf E}_X\hm>0$ в~предположении, что случайные величины~$X_j$ 
не содержат ошибок измерения, а все погрешности учтены исключительно в~за\-шум\-ля\-ющих 
элементах~$\varepsilon_j$.

\smallskip

\noindent
\textbf{Теорема~4.}
\textit{Пусть выполнены предположения}~(A)--(D), 
\textit{причем случайные величины~$\varepsilon_j$, $j\hm=1,2,\ldots$, имеют 
распределение типа конечной $k$-ком\-по\-нент\-ной смеси 
гам\-ма-рас\-пре\-де\-ле\-ний вида}~\eqref{FinGammaMixt} 
\textit{с~па\-ра\-мет\-ра\-ми~${\bf r}$, $\boldsymbol{\lambda}$ и~${\bf p}$, 
а~случайные величины} $X_j\stackrel{\text{п.н.}}{=}{\sf E}_X$, $j=1,2,\ldots$ 
\textit{Тогда доверительный интервал для~${\sf E}_X$ при условии $0\hm<\alpha\hm<1$ имеет вид}:
\begin{equation} 
\label{Th4Eq}
\left\lvert {\sf E}_X - \hat{{\sf E}}_X\right\rvert \leqslant  
f({\bf r},\boldsymbol{\lambda},\alpha,n),
\end{equation}
\textit{где}

\vspace*{-9pt}

\noindent
\begin{align}
\hat{{\sf E}}_X&=\fr{1}{n} \sum\limits_{j=1}^{n} Y_j\,; \label{Th4hatE}\\[-4pt]
f({\bf r}, \boldsymbol{\lambda},\alpha,n)&=\fr{z_{1-{\alpha}/2}}{\sqrt{n}} \left(
\sqrt{\fr{R(R+1)}{\lambda^2}-\fr{r^2}{\Lambda^2}}+\fr{1}{2}\right) +{}\notag\\[-1pt]
&\hspace*{20mm}{}+
\fr{R}{\lambda}+\fr{\Lambda^{R}}{2^{r}\pi^{r+1}}\left(1+\fr{1}{r}\right); \notag
\end{align}
\textit{$z_{1-{\alpha}/2}$~--- $\left(1-{\alpha}/2\right)$-кван\-тиль 
стандартного нормального распределения; $r\hm=\min(r_1, \ldots,r_k)$; 
$R\hm=\max(r_1, \ldots,r_k)$; $\lambda\hm=\max(\lambda_1, \ldots,\lambda_k)$; 
$\Lambda\hm=\max(\lambda_1, \ldots,\lambda_k)$}. 

\smallskip

\noindent
Д\,о\,к\,а\,з\,а\,т\,е\,л\,ь\,с\,т\,в\,о\,.\ \
Из центральной предельной теоремы с~учетом условия~(A) 
следует, что величина~$\hat{{\sf E}}_X$~\eqref{Th4hatE} асимптотически нормальна 
с~математическим ожиданием~${\sf E}_Y$~\eqref{EY} и~дисперсией $(1/n){\sf D}_Y$~\eqref{DY}. 
Пользуясь определением и~свойствами гам\-ма-функ\-ции, а~также оценкой~\eqref{Var} 
получим:

\noindent
\begin{multline*}
{\sf D}_Y \leqslant \left(\sqrt{\sum\limits_{j=1}^k p_j
\fr{\lambda_j^{r_j}}{\Gamma(r_j)} \int\limits_{0}^{+\infty} 
e^{\lambda_j x}x^{r_j+1}\, dx}+\fr{1}{2}\right)^2= {}\\[-0.5pt]
= \left(\sqrt{\sum\limits_{j=1}^{k}p_j
\fr{r_j(r_j+1)}{\lambda_j^2}-\left(\sum\limits_{j=1}^{k}p_j
\fr{r_j}{\lambda_j}\right) ^2}+\fr{1}{2}\right)^2\leqslant {}\\[-1.5pt]
{}\leqslant \left(\sqrt{\fr{R(R+1)}{\lambda^2}-\fr{r^2}{\Lambda^2}}+\fr{1}{2}\right)^2\,.
\end{multline*}

Аналогично доказательству Тео\-ре\-мы~2 с~учетом оценки~\eqref{Th3Eq} 
отсюда следует справедливость соотношения~\eqref{Th4Eq}.~\hfill$\square$

\vspace*{-12pt}

\section{Заключение}

Итак, в~работе получены оценки для математического ожидания наблюдений в~предположении 
зашумления конечными смесями нормальных\linebreak (Тео\-ре\-ма~1) 
и~гам\-ма-рас\-пре\-де\-ле\-ний (Тео\-ре\-ма~3). 
%
Построены доверительные интервалы 
для неизвестного математического ожидания в~этих случаях с~использованием 
уточненной оценки~\eqref{Var} 
(Тео\-ре\-мы~2 и~4 соответственно). Отметим, что соответствующие соотношения 
зависят только от <<экстремальных>> значений параметров смесей, но не от числа 
компонент и~весов в~распределении зашумляющих наблюдений. 
%
Замечание~2 
предлагает подход, который  может быть использован для определения неизвестного 
параметра искусственно добавляемого к~исходным данным шума для улучшения качества 
работы метода скользящего разделения смесей.

\smallskip
Автор выражает признательность доктору фи\-зи\-ко-ма\-те\-ма\-ти\-че\-ских наук, 
профессору Виктору Юрьевичу Королеву за идею использования оценки 
дисперсии вида~\eqref{Var} и~другие полезные обсуждения в~рамках 
работы над данной статьей.

\vspace*{-12pt}

{\small\frenchspacing
 {%\baselineskip=10.8pt
 \addcontentsline{toc}{section}{References}
 \begin{thebibliography}{99}
\bibitem{Wright2003} \Au{Wright~D.\,E., Bray~I.} 
A~mixture model for rounded data~// J.~Roy. Stat. Soc.~D 
Sta., 2003. Vol.~52. P.~3--13.

\columnbreak

\bibitem{Bai2009} \Au{Bai~Z., Zheng~S., Zhang~B., Hu~G.} 
Statistical analysis for rounded data~// J.~Stat. Plan.  Infer., 2009. 
Vol.~139. Iss.~8. P.~2526--2542.

\bibitem{Zhang2010} \Au{Zhang~B., Liu~T., Bai~Z.\,D.} 
Analysis of rounded data from dependent sequences~// 
Ann. I.~Stat. Math., 2010. Vol.~62. Iss.~6. P.~1143--1173.

\bibitem{Zhao2012} \Au{Zhao~N., Bai~Z.} 
Analysis of rounded data in mixture normal model~// Stat. Pap., 2012. 
Vol.~53. P.~895--914.

\bibitem{Korolev2011-i} \Au{Королев~В.\,Ю.} 
Ве\-ро\-ят\-но\-ст\-но-ста\-ти\-сти\-че\-ские методы декомпозиции волатильности 
хаотических процессов.~--- М.: Изд-во Моск. ун-та, 2011. 512~с.

\bibitem{Ushakov2015} \Au{Ушаков В.\,Г., Ушаков Н.\,Г.} 
Об усреднении округленных данных~// Информатика и~её применения, 2015. Т.~9. 
Вып.~4. С.~106--109.

\bibitem{Ushakov2017a} \Au{Ушаков~В.\,Г., Ушаков~Н.\,Г.} 
Границы точ\-ности восстановления информации, 
теряемой при округлении результатов наблюдений~// 
Вестник Московского университета. Серия~15: Вычислительная математика и~кибернетика, 
2017. №\,2. С.~26--30.

\bibitem{Ushakov2017b} \Au{Ushakov~N.\,G., Ushakov~V.\,G.} 
Statistical analysis of rounded data: Recovering of information lost due to rounding~// 
J.~Korean Stat. Soc., 2017.  Vol.~46. No.\,3. P.~426--437.

\bibitem{Gorshenin2016} \Au{Gorshenin~A.\,K., Korolev~V.\,Yu.} 
A~noising method for the identification of the stochastic structure of 
information flows~// Comm. Com. Inf. Sc., 2017. 
Vol.~678. P.~279--289.

\bibitem{Gorshenin2013} 
\Au{Gorshenin~A., Korolev~V.} Modelling of statistical
fluctuations of information flows by mixtures of gamma distributions~// 
27th European Conference on Modelling and Simulation Proceedings.~--- 
Dudweiler, Germany: Digitaldruck Pirrot GmbHP, 2013. P.~569--572.
 \end{thebibliography}

 }
 }

\end{multicols}

\vspace*{-6pt}

\hfill{\small\textit{Поступила в~редакцию 03.08.18}}

\vspace*{6pt}

%\newpage

%\vspace*{-24pt}

\hrule

\vspace*{2pt}

\hrule

\vspace*{-2pt}


\def\tit{DATA NOISING BY FINITE NORMAL AND~GAMMA MIXTURES WITH~APPLICATION 
TO~THE~PROBLEM OF~ROUNDED OBSERVATIONS}


\def\titkol{Data noising by finite normal and~gamma mixtures with~application 
to~the~problem of~rounded observations}



\def\aut{A.\,K.~Gorshenin}

\def\autkol{A.\,K.~Gorshenin}

\titel{\tit}{\aut}{\autkol}{\titkol}

\vspace*{-11pt}


\noindent
Institute of Informatics Problems, Federal Research Center ``Computer Science and
Control'' of the Russian Academy of Sciences, 44-2~Vavilov Str., Moscow 119333,
Russian Federation


\def\leftfootline{\small{\textbf{\thepage}
\hfill INFORMATIKA I EE PRIMENENIYA~--- INFORMATICS AND
APPLICATIONS\ \ \ 2018\ \ \ volume~12\ \ \ issue\ 3}
}%
 \def\rightfootline{\small{INFORMATIKA I EE PRIMENENIYA~---
INFORMATICS AND APPLICATIONS\ \ \ 2018\ \ \ volume~12\ \ \ issue\ 3
\hfill \textbf{\thepage}}}

\vspace*{3pt}



\Abste{In many real problems, statistical analysis of data containing additional 
measurement errors, including rounding, is performed, which in some situations can 
lead to sufficiently significant distortions. In this paper, estimates for an 
unknown expectation of observations are obtained for one of the possible 
rounding models under the assumption that the original data are additionally 
noised with random variables having distributions of the type of finite 
mixtures of normal and gamma laws. Confidence intervals for an 
unknown expectation are constructed using the refined estimate for 
the variance of the integer part of the random variable. An algorithm 
for determining the value of the parameter of artificial noise, which 
can be added to the initial data to improve the quality of the 
method of moving separation of mixtures, is discussed.}


\KWE{noisy data; rounded data; finite normal mixtures; finite gamma mixtures; 
confidence intervals; moving separation of mixtures}



\DOI{10.14357/19922264180304}

%\vspace*{-14pt}

\Ack
\noindent
The research was supported by the Russian Science Foundation (project 18-71-00156).



%\vspace*{6pt}

  \begin{multicols}{2}

\renewcommand{\bibname}{\protect\rmfamily References}
%\renewcommand{\bibname}{\large\protect\rm References}

{\small\frenchspacing
 {%\baselineskip=10.8pt
 \addcontentsline{toc}{section}{References}
 \begin{thebibliography}{99}
\bibitem{1-gor-1}
\Aue{Wright,~D.\,E., and I.~Bray.} 2003.
A~mixture model for rounded data.  \textit{J.~Roy. Stat. Soc.~D Sta.} 52:3--13.

\bibitem{2-gor-1}
\Aue{Bai,~Z., S.~Zheng, B.~Zhang, and G.~Hu.} 2009. 
Statistical analysis for rounded data. \textit{J.~Stat. Plan. 
Infer.} 139(8):2526--2542.

\bibitem{3-gor-1}
\Aue{Zhang,~B., T.~Liu, and Z.\,D.~Bai.} 2010. 
Analysis of rounded data from dependent sequences. 
\textit{Ann. I.~Stat. Math.} 62(6):1143--1173.

\bibitem{4-gor-1}
\Aue{Zhao,~N., and Z.~Bai.} 2012. Analysis of rounded data in mixture normal model. 
\textit{Stat. Pap.} 53:895--914.

\bibitem{5-gor-1}
\Aue{Korolev, V.\,Yu.} 2011. 
\textit{Veroyatnostno-statisticheskie metody dekompozitsii volatil'nosti 
khaoticheskikh protsessov} [Probabilistic and statistical methods of 
decomposition of volatility of chaotic processes]. 
Moscow: Moscow University Publishing House. 512~p.

\bibitem{6-gor-1}
\Aue{Ushakov, V.\,G., and N.\,G.~Ushakov.} 
2015. Ob usrednenii okruglennykh dannykh [On averaging of rounded data].
\textit{Informatika i~ee Primeneniya~--- Inform. Appl.} 9(4):106--109.

\bibitem{7-gor-1}
\Aue{Ushakov,~V.\,G., and N.\,G.~Ushakov.} 2017. 
Boundaries of the precision of restoring information lost after rounding
 the results from observations. 
 \textit{Moscow University Computational Math. Cybernetics} 41(2):76--80.

\bibitem{8-gor-1}
\Aue{Ushakov,~N.\,G., and  V.\,G.~Ushakov.} 2017. 
Statistical analysis of rounded data: Recovering of information lost due to rounding. 
\textit{J.~Korean Stat. Soc.} 46(3):426--437.

\bibitem{9-gor-1}
\Aue{Gorshenin,~A.\,K., and V.\,Yu.~Korolev.} 2016. 
A~noising method for the identification of the stochastic structure of information 
flows. \textit{Comm. Com. Inf. Sc.} 678:279--289.

\bibitem{10-gor-1}
\Aue{Gorshenin,~A., and V.~Korolev.} 2013.  Modelling of statistical fluctuations of
information flows by mixtures of gamma distributions. 
\textit{27th European Conference on Modelling and Simulation Proceedings}. 
Dudweiler, Germany: Digitaldruck Pirrot GmbHP. 569--572.

\end{thebibliography}

 }
 }

\end{multicols}

\vspace*{-6pt}

\hfill{\small\textit{Received August 3, 2018}}

%\pagebreak

%\vspace*{-18pt}

\Contrl

\noindent
\textbf{Gorshenin Andrey K.} (b.\ 1986)~--- Candidate of Science (PhD) in physics and
mathematics, associate professor, leading scientist, Institute of Informatics Problems,
Federal Research Center ``Computer Science and Control'' of the Russian Academy of
Sciences, 44-2 Vavilov Str., Moscow 119333, Russian Federation; 
\mbox{agorshenin@frccsc.ru}
\label{end\stat}

\renewcommand{\bibname}{\protect\rm Литература}       %15




%%%%%%%%%%%%%%%%%%%%%%%%%%%%%%%%%%%%%%%%%%%%%%%

%\def\stat{rez}
{%\hrule\par
%\vskip 7pt % 7pt
\raggedleft\Large \bf%\baselineskip=3.2ex
Р\,Е\,Ц\,Е\,Н\,З\,И\,И \vskip 17pt
    \hrule
    \par
\vskip 6pt plus 6pt minus 3pt }

%\thispagestyle{headings} %с верхним колонтитулом
%\thispagestyle{myheadings} %с нижним колонтитулом, но в верхнем РЕЦЕНЗИИ

\def\tit{НОВАЯ КНИГА И.\,Н.~СИНИЦЫНА, А.\,С.~ШАЛАМОВА <<ЛЕКЦИИ ПО ТЕОРИИ 
ИНТЕГРИРОВАННОЙ ЛОГИСТИЧЕСКОЙ ПОДДЕРЖКИ>> (М.: ТОРУС ПРЕСС, 2012. 624~с.)}

%1
\def\aut{Д.ф.-м.н., профессор С.\,Я.~Шоргин}

\def\auf{\ }

\def\leftkol{\ % РЕЦЕНЗИИ
}

\def\rightkol{ \ } 

%\def\leftkol{\ } % ENGLISH ABSTRACTS}

%\def\rightkol{\ } %ENGLISH ABSTRACTS}

%\def\leftkol{РЕЦЕНЗИИ}

%\def\rightkol{РЕЦЕНЗИИ}

\titele{\tit}{\aut}{\auf}{\leftkol}{\rightkol}
\vspace*{-18pt}


     \label{st\stat}

     \begin{multicols}{2}
     {\small
     {\baselineskip=10.1pt
     

      В книге представлено системное изложение теоретических основ одного из новейших 
направлений в \mbox{об\-ласти} экономики послепродажного обслуживания изделий наукоемкой 
продукции (ИНП) длительного пользования~--- интегрированной логистической поддержки
(ИЛП). 
{\looseness=1

}

Приведены также результаты новых работ, выполненных в Институте проблем информатики 
Российской академии наук в рамках научного направления <<Информационные технологии и 
анализ сложных сис\-тем>>.
 {%\looseness=1

}
     
      Излагаемые в книге научные подходы позво\-ляют карди\-наль\-но реформировать 
существующие системы производства и эксплуатации ИНП путем создания и внед\-ре\-ния 
методов рационального и оптимального управ\-ле\-ния процессами расходования 
вре\-мен\-н$\acute{\mbox{ы}}$х, 
мате\-ри\-аль\-ных, трудовых и других ресурсов на всех стадиях жизненного цикла изделий (ЖЦИ) по 
критериям экономической целесообразности и эф\-фек\-тив\-ности.
  {\looseness=1

}
    
      В книге приведен краткий обзор причин возник\-новения и
      развития CALS-методологии как основы 
современных международных стандартов по созданию и функционированию глобальных 
ин\-фор\-ма\-ци\-он\-но-ком\-му\-ни\-ка\-ци\-он\-ных систем, ее ключевых возможностей и эффективности 
результатов ее использования. 
Авторы %\linebreak 
предлагают ряд научных обоснований для разработки 
единой теории проектирования и управления систем ИЛП для полноценного использования 
преимуществ %\linebreak
 суще\-ст\-ву\-ющей методологии, определяют \mbox{общую} структурную схему 
комплексной системы <<ИНП-СППО>> и необходимость разработки для ее описания 
гибридных стохастических моделей.
{%\looseness=1

}

%\columnbreak
      
      Книга состоит из пяти частей, где последовательно излагается материал по каждой из 
следующих тем: <<Интегрированная логистическая поддержка>>, <<Теория гибридных 
стохастических систем и компьютерная поддержка исследований и разработок>>, <<Основы 
математического моделирования, анализа и синтеза систем послепродажного обслуживания>>, 
<<Определение и анализ показателей экспортного потенциала ИНП при проектировании>>, 
<<Задачи управления поддержкой послепродажного обслуживания>>, а также 
<<Моделирование инвестиционных процессов ИЛП в условиях неравновесных финансовых 
рынков>>. 
   
      В конце каждой главы приведены выводы и даны вопросы и задания для 
самоконтроля. В~приложениях содержатся основные определения по программам работ по 
анализу ИЛП, логистическим базам данных и компьютерным решениям, эквивалентной статистической 
линеаризации нелинейных преобразований ИЛП, справочный материал, а также развернутые 
уравнения для вероятностных характеристик.


      \def\leftkol{РЕЦЕНЗИИ}

\def\rightkol{РЕЦЕНЗИИ} 

      
      Книга заинтересует широкий круг специалистов и может быть использована научными 
проектными организациями в сфере промышленного производства ИНП. Большое количество 
иллюстраций, примеров и вопросов, обращенных к читателю, позволяет использовать книгу 
также в качестве учебного пособия для студентов и аспирантов машиностроительных, 
транспортных и~других специальностей, а также для самостоятельного изучения. 
{%\looseness=-1

}

Книга 
представляет несомненный интерес для специалистов и студентов в области прикладной 
математики и информатики.
    

}

}
\end{multicols}

%\newpage

\def\stat{authorsrus}
{%\hrule\par
%\vskip 7pt % 7pt
\raggedleft\Large \bf%\baselineskip=3.2ex
О\,Б\ \ А\,В\,Т\,О\,Р\,А\,Х \vskip 17pt
    \hrule
    \par
\vskip 21pt plus 8pt minus 4pt }


\def\tit{\ }

\def\aut{\ }

\def\auf{\ }

\def\leftkol{\ } % ENGLISH ABSTRACTS}

\def\rightkol{ОБ АВТОРАХ} %ENGLISH ABSTRACTS}

\titele{\tit}{\aut}{\auf}{\leftkol}{\rightkol}
      
            \label{st\stat}



\vspace*{24pt}

\begin{multicols}{2}




\noindent
\textbf{Архипов Олег Петрович} (р.\ 1948)~---
кандидат технических наук, директор Орловского филиала Института проб\-лем информатики
Российской академии наук
%302025, г.Орел, Московское шоссе, д.137

\vspace*{3pt}

\noindent
\textbf{Бирюкова Татьяна Константиновна} (р.\ 1968)~---
кандидат фи\-зи\-ко-ма\-те\-ма\-ти\-че\-ских наук, старший научный сотрудник Института проб\-лем информатики
Российской академии наук

\vspace*{3pt}

\noindent 
\textbf{Бобков  Сергей Геннадьевич} (р.\ 1955)~---
доктор технических наук,  заведующий отделением На\-уч\-но-ис\-сле\-до\-ва\-тель\-ско\-го 
института системных исследований Российской академии наук
%117218, Москва, Нахимовский просп., 36, к.1 

\vspace*{3pt}

\noindent \textbf{Васильев Николай Семенович} (р.\ 1952)~--- доктор 
фи\-зи\-ко-ма\-те\-ма\-ти\-че\-ских наук, профессор, 
МГТУ им.\ Н.\,Э.~Баумана 
%, Москва 105005, 2-я Бауманская ул., д.~5,

\vspace*{3pt}

\noindent
\textbf{Гершкович Максим Михайлович} (р.\ 1968)~---
старший научный сотрудник Института проб\-лем информатики
Российской академии наук

\vspace*{3pt}

\noindent 
\textbf{Дьяченко Юрий Георгиевич} (р.\ 1958)~--- кандидат технических наук, 
старший научный сотрудник Института проб\-лем информатики
Российской академии наук

\vspace*{3pt}

\noindent 
\textbf{Ерошенко Александр Андреевич} (р.\ 1989)~--- аспирант кафедры 
математической статистики факультета вычисли\-тельной математики и кибернетики 
Московского государственного университета им.\ М.\,В.~Ломоносова
%119991, Москва ГСП-1, Ленинские горы, д.\ 1, стр. 52

\vspace*{3pt}
 
\noindent 
\textbf{Захаров Виктор Николаевич} (р.\ 1948)~--- 
доктор технических наук, доцент, ученый секретарь Института проб\-лем информатики
Российской академии наук

\vspace*{3pt}

\noindent
\textbf{Зейфман Александр Израилевич} (р.\ 1954)~---
доктор фи\-зи\-ко-ма\-те\-ма\-ти\-че\-ских наук, профессор, 
заведующий кафедрой Вологодского государственного университета; 
старший научный сотрудник Института проб\-лем информатики
Российской академии наук; главный научный сотрудник ИСЭРТ Российской академии наук

\vspace*{3pt}

\noindent
\textbf{Зыкин Сергей Владимирович} (р.\ 1959)~--- 
доктор технических наук, профессор, заведующий лабораторией Института математики 
им.\ С.\,Л.~Соболева Сибирского отделения Российской академии наук, Новосибирск 
%630090, пр.\ ак.\ Коптюга, 4 

\vspace*{4pt}

\noindent
\textbf{Киреев Владимир Иванович} (р.\ 1938)~---
доктор фи\-зи\-ко-ма\-те\-ма\-ти\-че\-ских наук, профессор Московского 
государственного горного университета
%Адрес: Россия, 119991, г. Москва, Ленинский проспект, д. 6

%\columnbreak

\vspace*{4pt}

\noindent
\textbf{Козеренко Елена Борисовна} (р.\ 1959)~---
кандидат филологических наук, заведующая лабораторией Института проб\-лем информатики
Российской академии наук

\vspace*{4pt}

\noindent
\textbf{Королев Виктор Юрьевич} (р.\ 1954)~--- доктор
фи\-зи\-ко-ма\-те\-ма\-ти\-че\-ских наук, профессор кафедры математической 
статистики факультета вычисли\-тельной математики и кибернетики 
Московского государственного университета; 
ведущий научный сотрудник Института проб\-лем информатики
Российской академии наук

\vspace*{4pt}

\noindent
\textbf{Коротышева Анна Владимировна} (р.\ 1988)~---
старший преподаватель Вологодского государственного университета

\vspace*{4pt}

\noindent 
\textbf{Кун Де Турк} (р.\ 1981)~--- научный сотрудник 
исследовательской группы SMACS факультета телекоммуникаций и обработки информации
Университета Гента, Бельгия
%В-9000 Гент, Бельгия

\vspace*{4pt}

\noindent
\textbf{Лупенцов Олег Сергеевич} (р.\ 1986)~---
аспирант Омского государственного института сервиса
%Омск 644043, ул.\ Певцова 13

\vspace*{4pt}

\noindent
\textbf{Лучко Олег Николаевич} (р.\ 1961)~---
кандидат педагогических наук, профессор, заведующий кафедрой 
Омского государственного института сервиса
%Омск 644043, ул.\ Певцова 13

\vspace*{4pt}

\noindent
\textbf{Малашенко Юрий Евгеньевич} (р.\ 1946)~---
доктор фи\-зи\-ко-ма\-те\-ма\-ти\-че\-ских наук, заведующий сектором 
Вычислительного центра им.\ А.\,А.~Дородницына Российской академии наук
%Адрес: 119333, Москва, ул. Вавилова, 40,

\vspace*{4pt}

\noindent
\textbf{Маньяков Юрий Анатольевич} (р.\ 1984)~---
кандидат технических наук, научный сотрудник Орловского филиала Института проб\-лем информатики
Российской академии наук
%302025, г.Орел, Московское шоссе, д.137

\vspace*{4pt}

\noindent
\textbf{Маренко Валентина Афанасьевна} (р.\ 1951)~---
кандидат технических наук, доцент, старший научный сотрудник 
Института математики им.\ С.\,Л.~Соболева Сибирского отделения Российской академии наук
%Новосибирск 630090, пр. ак. Коптюга, 4 

\vspace*{3pt}

\noindent 
\textbf{Морозов Евсей Викторович} (р.\ 1947)~--- доктор 
фи\-зи\-ко-ма\-те\-ма\-ти\-че\-ских, профессор, ведущий научный сотрудник 
Института прикладных математических исследований Карельского научного центра Российской
академии наук; 
%%185910 Россия, Республика Карелия, г.\ Петрозаводск, ул.\ Пушкинская, 11
профессор Петрозаводского государственного университета, Петрозаводск
%185910 Россия, Республика Карелия, г.\ Петрозаводск, пр.\ Ленина, 33

%\pagebreak

\vspace*{3pt}

\noindent
\textbf{Назарова Ирина Александровна} (р.\ 1966)~---
кандидат фи\-зи\-ко-ма\-те\-ма\-ти\-че\-ских наук, 
научный сотрудник Вычислительного центра им.\ А.\,А.~Дородницына Российской академии наук 
%Адрес: 119333, Москва, ул. Вавилова, 40

\vspace*{3pt}

\noindent
\textbf{Павлов Игорь Валерианович} (р.\ 1945)~--- 
доктор фи\-зи\-ко-ма\-те\-ма\-ти\-че\-ских наук, профессор МГТУ им.\ Н.\,Э.~Баумана 
%Москва 105005, 2-я Бауманская ул., д.~5 

%\pagebreak

\vspace*{3pt}

\noindent 
\textbf{Потахина Любовь Викторовна} (р.\ 1989)~--- аспирантка
Института прикладных математических исследований Карельского научного центра
Российской академии наук; 
%%185910 Россия, Республика Карелия, г.\ Петрозаводск, ул.\ Пушкинская, 11
инженер Петрозаводского государственного университета, Петрозаводск
%185910 Россия, Республика Карелия, г.\ Петрозаводск, пр.\ Ленина, 33

\vspace*{3pt}

\noindent 
\textbf{Рождественский Юрий Владимирович} (р.\ 1952)~--- 
кандидат технических наук, заведующий сектором Института проб\-лем информатики
Российской академии наук

\vspace*{3pt}

\noindent 
\textbf{Синицын Игорь Николаевич} (р.\ 1940)~--- доктор технических наук,
профессор, заслуженный деятель\linebreak\vspace*{-12pt}

\columnbreak

\noindent
 науки РФ, заведующий отделом Института проб\-лем информатики
Российской академии наук

\vspace*{7pt}


\noindent
\textbf{Сиротинин Денис Олегович} (р.\ 1984)~---
кандидат технических наук, научный сотрудник Орловского филиала Института проб\-лем информатики
Российской академии наук
%302025, г.Орел, Московское шоссе, д.137

\vspace*{7pt}

%\columnbreak

\noindent 
\textbf{Соколов  Игорь Анатольевич} (р.\ 1954)~--- академик (действительный член) Российской 
академии наук, доктор технических наук, директор Института проб\-лем информатики
Российской академии наук

\vspace*{7pt}

\noindent
\textbf{Степченков Юрий Афанасьевич} (р.\ 1951)~---
кандидат технических наук, заведующий отделом Института проб\-лем информатики
Российской академии наук

\vspace*{7pt}

\noindent
\textbf{Сурков Алексей Викторович} (р.\ 1978)~--- 
старший научный сотрудник На\-уч\-но-ис\-сле\-до\-ва\-тель\-ско\-го 
института системных исследований Российской академии наук
%117218, Москва, Нахимовский просп., 36, к.1 

\vspace*{7pt}

\noindent 
\textbf{Шестаков Олег Владимирович} (р.\ 1976)~--- доктор 
фи\-зи\-ко-ма\-те\-ма\-ти\-че\-ских, доцент кафедры математической статистики 
факультета вычисли\-тельной математики и кибернетики Московского 
государственного университета им.\ М.\,В.~Ломоносова; 
%119991, Москва ГСП-1, Ленинские горы, д.\ 1, стр. 52
старший научный сотрудник Института проб\-лем информатики
Российской академии наук
%, Москва 119333, ул. Вавилова, д.~44, корп.~2

\vspace*{7pt}

\noindent 
\textbf{Шоргин Сергей Яковлевич} (р.\ 1952.)~--- доктор
фи\-зи\-ко-ма\-те\-ма\-ти\-че\-ских наук, профессор, заместитель директора Института 
проб\-лем информатики Российской академии наук





%%%%%%%%%%%%%%%%%%%%%%%%%%%%%%%%%%%%%%%%%%%%%%%%%%%%%%%%%%%%%%%%%%%%%%%%%%%%%%%




%\def\rightkol{ОБ АВТОРАХ}
%\def\leftkol{ОБ АВТОРАХ}

 \label{end\stat}





%\def\leftfootline{\small{\textbf{\thepage}
%\hfill ИНФОРМАТИКА И ЕЁ ПРИМЕНЕНИЯ\ \ \ том~7\ \ \ выпуск~1\ \ \ 2013}
%}%
% \def\rightfootline{\small{ИНФОРМАТИКА И ЕЁ ПРИМЕНЕНИЯ\ \ \ том~7\ \ \ выпуск~1\ \ \ 2013
%\hfill \textbf{\thepage}}}


%\thispagestyle{myheadings}



\end{multicols}

\newpage  

%\def\stat{cont}
{%\hrule\par
%\vskip 7pt % 7pt
\raggedleft\Large \bf%\baselineskip=3.2ex
А\,В\,Т\,О\,Р\,С\,К\,И\,Й\ \ У\,К\,А\,З\,А\,Т\,Е\,Л\,Ь\ \ З\,А\ \ 2\,0\,0\,7 г. \vskip 17pt
    \hrule
    \par
\vskip 21pt plus 6pt minus 3pt }

\label{st\stat}

\def\tit{\ }

\def\aut{\ }
\def\auf{\ }

\def\leftkol{\ } % ENGLISH ABSTRACTS}

\def\rightkol{\ } %ENGLISH ABSTRACTS}

\titele{\tit}{\aut}{\auf}{\leftkol}{\rightkol}


\contentsline {chapter}{\ }{Выпуск \quad Стр.} 
\contentsline {section}{\textbf{Батракова Д.\,А., Королев В.\,Ю., Шоргин С.\,Я.}\ \ Новый метод вероятностно-ста\-ти\-сти\-че\-ско\-го анализа информационных потоков в\nobreakspace {}телекоммуникационных сетях}{\qquad 1 \qquad 40} 
\contentsline {section}{\textbf{Борисов А.\,В.}\ \ Байесовское оценивание в системах наблюдения с\nobreakspace {}марковскими скачкообразными процессами: игровой подход}{\qquad 2 \qquad 65}
\contentsline {section}{\textbf{Босов А.\,В., Иванов А.\,В.}\ \ Программная инфраструктура информационного Web-пор\-тала}{\qquad 2 \qquad 50}
\contentsline {section}{\textbf{Захаров В.\,Н., Калиниченко Л.\,А., Соколов И.\,А., Ступников С.\,А.}\ \ Конструирование канонических информационных моделей для интегрированных информационных систем}{\qquad 2 \qquad 15}
\contentsline {section}{\textbf{Захаров В.\,Н., Козмидиади В.\,А.}\ \ Средства обеспечения отказоустойчивости при\-ло\-жений}{\qquad 1 \qquad 14} 
\contentsline {section}{\textbf{Иванов А.\,В.}\ \ см. Босов А.\,В.\hfill\hfill\hfill\hfill\hfill\hfill\hfill\hfill\hfill\hfill\hfill\hfill\hfill\hfill\hfill\hfill\hfill\hfill\hfill\hfill\hfill\hfill\hfill\hfill\hfill\hfill\hfill\hfill\hfill\hfill\hfill\hfill\hfill\hfill\hfill}{\ }
\contentsline {section}{\textbf{Ильин В.\,Д., Соколов И.\,А.}\ \ Символьная модель системы знаний информатики в\nobreakspace {}че\-ло\-ве\-ко-автоматной среде}{\qquad 1 \qquad 66} 
\contentsline {section}{\textbf{Калиниченко Л.\,А.}\ \ см. Захаров В.\,Н.\hfill\hfill\hfill\hfill\hfill\hfill\hfill\hfill\hfill\hfill\hfill\hfill\hfill\hfill\hfill\hfill\hfill\hfill\hfill\hfill\hfill\hfill\hfill\hfill\hfill\hfill\hfill\hfill\hfill\hfill\hfill\hfill\hfill\hfill\hfill}{\ }
\contentsline {section}{\textbf{Козеренко Е.\,Б.}\ \ Лингвистическое моделирование для систем машинного перевода и обработки знаний}{\qquad 1 \qquad 54} 
\contentsline {section}{\textbf{Козмидиади В.\,А.}\ \ см. Захаров В.\,Н.\hfill\hfill\hfill\hfill\hfill\hfill\hfill\hfill\hfill\hfill\hfill\hfill\hfill\hfill\hfill\hfill\hfill\hfill\hfill\hfill\hfill\hfill\hfill\hfill\hfill\hfill\hfill\hfill\hfill\hfill\hfill\hfill\hfill\hfill\hfill }{\ } 
\contentsline {section}{\textbf{Королев В.\,Ю.}\ \ см. Батракова Д.\,А.\hfill\hfill\hfill\hfill\hfill\hfill\hfill\hfill\hfill\hfill\hfill\hfill\hfill\hfill\hfill\hfill\hfill\hfill\hfill\hfill\hfill\hfill\hfill\hfill\hfill\hfill\hfill\hfill\hfill\hfill\hfill\hfill\hfill\hfill\hfill}{\ } 
\contentsline {section}{\textbf{Кудрявцев А.\,А., Шоргин С.\,Я.}\ \ Байесовский подход к\nobreakspace {}анализу систем массового обслуживания и\nobreakspace {}показателей надежности}{\qquad 2 \qquad 76}
\contentsline {section}{\textbf{Печинкин А.\,В., Соколов И.\,А., Чаплыгин В.\,В.}\ \ Многолинейная система массового обслуживания с конечным накопителем и ненадежными приборами}{\qquad 1 \qquad 27} 
\contentsline {section}{\textbf{Печинкин А.\,В., Соколов И.\,А., Чаплыгин В.\,В.}\ \ Стационарные характеристики многолинейной\nobreakspace {}системы массового обслуживания с\nobreakspace {}одновременными отказами приборов}{\qquad 2 \qquad 39}
\contentsline {section}{\textbf{Синицын И.\,Н.}\ \ Корреляционные методы построения аналитических информационных моделей флуктуаций полюса Земли по априорным данным}{\qquad 2 \qquad \hphantom{9}2}
\contentsline {section}{\textbf{Синицын И.\,Н.}\ \ Развитие теории фильтров Пугачева для оперативной обработки информации в стохастических системах}{{\qquad 1 \qquad \hphantom{9}3}} 
\contentsline {section}{\textbf{Соколов И.\,А.}\ \ см. Захаров В.\,Н.\hfill\hfill\hfill\hfill\hfill\hfill\hfill\hfill\hfill\hfill\hfill\hfill\hfill\hfill\hfill\hfill\hfill\hfill\hfill\hfill\hfill\hfill\hfill\hfill\hfill\hfill\hfill\hfill\hfill\hfill\hfill\hfill\hfill\hfill\hfill}{\ }
\contentsline {section}{\textbf{Соколов И.\,А.}\ \ см. Ильин В.\,Д.\hfill\hfill\hfill\hfill\hfill\hfill\hfill\hfill\hfill\hfill\hfill\hfill\hfill\hfill\hfill\hfill\hfill\hfill\hfill\hfill\hfill\hfill\hfill\hfill\hfill\hfill\hfill\hfill\hfill\hfill\hfill\hfill\hfill\hfill\hfill}{\ } 
\contentsline {section}{\textbf{Соколов И.\,А.}\ \ см. Печинкин А.\,В.\hfill\hfill\hfill\hfill\hfill\hfill\hfill\hfill\hfill\hfill\hfill\hfill\hfill\hfill\hfill\hfill\hfill\hfill\hfill\hfill\hfill\hfill\hfill\hfill\hfill\hfill\hfill\hfill\hfill\hfill\hfill\hfill\hfill\hfill\hfill}{\ } 
\contentsline {section}{\textbf{Соколов И.\,А.}\ \ см. Печинкин А.\,В.\hfill\hfill\hfill\hfill\hfill\hfill\hfill\hfill\hfill\hfill\hfill\hfill\hfill\hfill\hfill\hfill\hfill\hfill\hfill\hfill\hfill\hfill\hfill\hfill\hfill\hfill\hfill\hfill\hfill\hfill\hfill\hfill\hfill\hfill\hfill}{\ }
\contentsline {section}{\textbf{Ступников С.\,А.}\ \ см. Захаров В.\,Н.\hfill\hfill\hfill\hfill\hfill\hfill\hfill\hfill\hfill\hfill\hfill\hfill\hfill\hfill\hfill\hfill\hfill\hfill\hfill\hfill\hfill\hfill\hfill\hfill\hfill\hfill\hfill\hfill\hfill\hfill\hfill\hfill\hfill\hfill\hfill}{\ }
\contentsline {section}{\textbf{Чаплыгин В.\,В.}\ \ см. Печинкин А.\,В.\hfill\hfill\hfill\hfill\hfill\hfill\hfill\hfill\hfill\hfill\hfill\hfill\hfill\hfill\hfill\hfill\hfill\hfill\hfill\hfill\hfill\hfill\hfill\hfill\hfill\hfill\hfill\hfill\hfill\hfill\hfill\hfill\hfill\hfill\hfill}{\ } 
\contentsline {section}{\textbf{Чаплыгин В.\,В.}\ \ см. Печинкин А.\,В.\hfill\hfill\hfill\hfill\hfill\hfill\hfill\hfill\hfill\hfill\hfill\hfill\hfill\hfill\hfill\hfill\hfill\hfill\hfill\hfill\hfill\hfill\hfill\hfill\hfill\hfill\hfill\hfill\hfill\hfill\hfill\hfill\hfill\hfill\hfill}{\ }
\contentsline {section}{\textbf{Шоргин С.\,Я.}\ \ см. Батракова Д.\,А.\hfill\hfill\hfill\hfill\hfill\hfill\hfill\hfill\hfill\hfill\hfill\hfill\hfill\hfill\hfill\hfill\hfill\hfill\hfill\hfill\hfill\hfill\hfill\hfill\hfill\hfill\hfill\hfill\hfill\hfill\hfill\hfill\hfill\hfill\hfill}{\ } 
\contentsline {section}{\textbf{Шоргин С.\,Я.}\ \ см. Кудрявцев А.\,А.\hfill\hfill\hfill\hfill\hfill\hfill\hfill\hfill\hfill\hfill\hfill\hfill\hfill\hfill\hfill\hfill\hfill\hfill\hfill\hfill\hfill\hfill\hfill\hfill\hfill\hfill\hfill\hfill\hfill\hfill\hfill\hfill\hfill\hfill\hfill}{\ }
%\thispagestyle{myheadings}
\def\leftfootline{\small{\textbf{\thepage}
\hfill ИНФОРМАТИКА И ЕЁ ПРИМЕНЕНИЯ\ \ \ том~1\ \ \ выпуск~2\ \ \ 2007}
}%
 \def\rightfootline{\small{ИНФОРМАТИКА И ЕЁ ПРИМЕНЕНИЯ\ \ \ том~1\ \ \ выпуск~2\ \ \ 2007
 \hfill \textbf{\thepage}}}
 \label{end\stat} 
                     
%\def\stat{cont-e}
{%\hrule\par
%\vskip 7pt % 7pt
\raggedleft\Large \bf%\baselineskip=3.2ex
2\,0\,0\,7\ \ A\,U\,T\,H\,O\,R\ \ I\,N\,D\,E\,X \vskip 17pt
    \hrule
    \par
\vskip 21pt plus 6pt minus 3pt }

\label{st\stat}

\def\tit{\ }

\def\aut{\ }
\def\auf{\ }

\def\leftkol{\ } % ENGLISH ABSTRACTS}

\def\rightkol{\ } %ENGLISH ABSTRACTS}

\titele{\tit}{\aut}{\auf}{\leftkol}{\rightkol}


\contentsline {chapter}{\ }{Issue \quad Page} 
\contentsline {subsection}{\textbf{Batrakova D.\,A., Korolev V.\,Yu., Shorgin S.\,Ya.}\ \ A New Method for the Probabilistic and Statistical Analysis of Information Flows in Telecommunication Networks}{\qquad 1 \qquad 40} 
\contentsline {subsection}{\textbf{Borisov A.\,V.}\ \ Bayesian Estimation in\nobreakspace {}Observation Systems with\nobreakspace {}Markov Jump Processes: Game-Theoretic Approach}{\qquad 2 \qquad 65} 
\contentsline {subsection}{\textbf{Bosov A.\,V., Ivanov A.\,V.}\ \ Linguistic Simulation for Machine Translation and Knowledge Management Systems}{\qquad 2 \qquad 50} 
\contentsline {subsection}{\textbf{Chaplygin V.\,V.} see Pechinkin A.\,V.\hfill\hfill\hfill\hfill\hfill\hfill\hfill\hfill\hfill\hfill\hfill\hfill\hfill\hfill\hfill\hfill\hfill\hfill\hfill\hfill\hfill\hfill\hfill\hfill\hfill\hfill\hfill\hfill\hfill\hfill\hfill\hfill\hfill\hfill\hfill}{\ }
\contentsline {subsection}{\textbf{Chaplygin V.\,V.} see Pechinkin A.\,V.\hfill\hfill\hfill\hfill\hfill\hfill\hfill\hfill\hfill\hfill\hfill\hfill\hfill\hfill\hfill\hfill\hfill\hfill\hfill\hfill\hfill\hfill\hfill\hfill\hfill\hfill\hfill\hfill\hfill\hfill\hfill\hfill\hfill\hfill\hfill}{\ }
\contentsline {subsection}{\textbf{Ilyin V.\,D., Sokolov I.\,A.}\ \ The Symbol Model of Informatics Knowledge System in Human-Automaton Environment}{\qquad 1 \qquad 66} 
\contentsline {subsection}{\textbf{Ivanov A.\,V.} see Bosov A.\,V.\hfill\hfill\hfill\hfill\hfill\hfill\hfill\hfill\hfill\hfill\hfill\hfill\hfill\hfill\hfill\hfill\hfill\hfill\hfill\hfill\hfill\hfill\hfill\hfill\hfill\hfill\hfill\hfill\hfill\hfill\hfill\hfill\hfill\hfill\hfill}{\ }
\contentsline {subsection}{\textbf{Kalinichenko L.\,A.} see Zakharov V.\,N.\hfill\hfill\hfill\hfill\hfill\hfill\hfill\hfill\hfill\hfill\hfill\hfill\hfill\hfill\hfill\hfill\hfill\hfill\hfill\hfill\hfill\hfill\hfill\hfill\hfill\hfill\hfill\hfill\hfill\hfill\hfill\hfill\hfill\hfill\hfill}{\ }
\contentsline {subsection}{\textbf{Korolev V.\,Yu.} see Batrakova D.\,A.\hfill\hfill\hfill\hfill\hfill\hfill\hfill\hfill\hfill\hfill\hfill\hfill\hfill\hfill\hfill\hfill\hfill\hfill\hfill\hfill\hfill\hfill\hfill\hfill\hfill\hfill\hfill\hfill\hfill\hfill\hfill\hfill\hfill\hfill\hfill}{\ }
\contentsline {subsection}{\textbf{Kozerenko E.\,B.}\ \ Linguistic Simulation for Machine Translation and Knowledge Management Systems}{\qquad 1 \qquad 54} 
\contentsline {subsection}{\textbf{Kozmidiady V.\,A.} see Zakharov V.\,N.\hfill\hfill\hfill\hfill\hfill\hfill\hfill\hfill\hfill\hfill\hfill\hfill\hfill\hfill\hfill\hfill\hfill\hfill\hfill\hfill\hfill\hfill\hfill\hfill\hfill\hfill\hfill\hfill\hfill\hfill\hfill\hfill\hfill\hfill\hfill}{\ }
\contentsline {subsection}{\textbf{Kudryavtsev A.\,A., Shorgin S.\,Ya.}\ \ Bayesian Approach to Queueing Systems and Reliability Characteristics}{\qquad 2 \qquad 76} 
\contentsline {subsection}{\textbf{Pechinkin A.\,V., Sokolov I.\,A., Chaplygin V.\,V.}\ \ Multichannel Queuing System with Finite Buffer and Unreliable Servers}{\qquad 1 \qquad 27} 
\contentsline {subsection}{\textbf{Pechinkin A.\,V., Sokolov I.\,A., Chaplygin V.\,V.}\ \ Stationary Characteristics of a Multichannel Queueing System with\nobreakspace {}Simultaneous Refusals of Servers}{\qquad 2 \qquad 39} 
\contentsline {subsection}{\textbf{Shorgin S.\,Ya.} see Batrakova D.\,A.\hfill\hfill\hfill\hfill\hfill\hfill\hfill\hfill\hfill\hfill\hfill\hfill\hfill\hfill\hfill\hfill\hfill\hfill\hfill\hfill\hfill\hfill\hfill\hfill\hfill\hfill\hfill\hfill\hfill\hfill\hfill\hfill\hfill\hfill\hfill}{\ }
\contentsline {subsection}{\textbf{Shorgin S.\,Ya.} see Kudryavtsev A.\,A.\hfill\hfill\hfill\hfill\hfill\hfill\hfill\hfill\hfill\hfill\hfill\hfill\hfill\hfill\hfill\hfill\hfill\hfill\hfill\hfill\hfill\hfill\hfill\hfill\hfill\hfill\hfill\hfill\hfill\hfill\hfill\hfill\hfill\hfill\hfill}{\ }
\contentsline {subsection}{\textbf{Sinitsyn I.\,N.}\ \ Correlational Methods for Analytical Informational Models of the Earth Pole Fluctuations Design Based on a priori Data}{\qquad 2 \qquad \hphantom{9}2}
\contentsline {subsection}{\textbf{Sinitsyn I.\,N.}\ \ Development of Pugachev Filtering for Stochastic Systems}{\qquad 1 \qquad \hphantom{9}3}
\contentsline {subsection}{\textbf{Sokolov I.\,A.} see Ilyin V.\,D.\hfill\hfill\hfill\hfill\hfill\hfill\hfill\hfill\hfill\hfill\hfill\hfill\hfill\hfill\hfill\hfill\hfill\hfill\hfill\hfill\hfill\hfill\hfill\hfill\hfill\hfill\hfill\hfill\hfill\hfill\hfill\hfill\hfill\hfill\hfill}{\ }
\contentsline {subsection}{\textbf{Sokolov I.\,A.} see Pechinkin A.\,V.\hfill\hfill\hfill\hfill\hfill\hfill\hfill\hfill\hfill\hfill\hfill\hfill\hfill\hfill\hfill\hfill\hfill\hfill\hfill\hfill\hfill\hfill\hfill\hfill\hfill\hfill\hfill\hfill\hfill\hfill\hfill\hfill\hfill\hfill\hfill}{\ }
\contentsline {subsection}{\textbf{Sokolov I.\,A.} see Pechinkin A.\,V.\hfill\hfill\hfill\hfill\hfill\hfill\hfill\hfill\hfill\hfill\hfill\hfill\hfill\hfill\hfill\hfill\hfill\hfill\hfill\hfill\hfill\hfill\hfill\hfill\hfill\hfill\hfill\hfill\hfill\hfill\hfill\hfill\hfill\hfill\hfill}{\ }
\contentsline {subsection}{\textbf{Sokolov I.\,A.} see Zakharov V.\,N.\hfill\hfill\hfill\hfill\hfill\hfill\hfill\hfill\hfill\hfill\hfill\hfill\hfill\hfill\hfill\hfill\hfill\hfill\hfill\hfill\hfill\hfill\hfill\hfill\hfill\hfill\hfill\hfill\hfill\hfill\hfill\hfill\hfill\hfill\hfill}{\ }
\contentsline {subsection}{\textbf{Stupnikov S.\,A.} see Zakharov V.\,N.\hfill\hfill\hfill\hfill\hfill\hfill\hfill\hfill\hfill\hfill\hfill\hfill\hfill\hfill\hfill\hfill\hfill\hfill\hfill\hfill\hfill\hfill\hfill\hfill\hfill\hfill\hfill\hfill\hfill\hfill\hfill\hfill\hfill\hfill\hfill}{\ }
\contentsline {subsection}{\textbf{Zakharov V.\,N., Kalinichenko L.\,A., Sokolov I.\,A., Stupnikov S.\,A.}\ \ Development of Canonical Information Models for Integrated Information Systems}{\qquad 2 \qquad 15} 
\contentsline {subsection}{\textbf{Zakharov V.\,N., Kozmidiady V.\,A.}\ \ Means Providing Applications Fault Tolerance}{\qquad 1 \qquad 14} 
\def\leftfootline{\small{\textbf{\thepage}
\hfill ИНФОРМАТИКА И ЕЁ ПРИМЕНЕНИЯ\ \ \ том~1\ \ \ выпуск~2\ \ \ 2007}
}%
 \def\rightfootline{\small{ИНФОРМАТИКА И ЕЁ ПРИМЕНЕНИЯ\ \ \ том~1\ \ \ выпуск~2\ \ \ 2007
 \hfill \textbf{\thepage}}}
 \label{end\stat} 


%\end{document}

%
\def\stat{rekl}
%\label{preobr}

%\def\tit{АКАДЕМИК ПУГАЧЁВ  ВЛАДИМИР СЕМЁНОВИЧ\\
%25.03.1911--25.03.1998}


%   \vspace*{-48pt}
%   \begin{center}\LARGE
%Академик Пугачёв  Владимир Семёнович\\ (25.03.1911--25.03.1998)
%   \end{center}

   %\vspace*{2.5mm}

   \begin{center}

{\prgsh\LARGE
ЮБИЛЕИ}

\end{center}
%\hrule

\vspace*{6pt}


   \vspace*{8mm}

   \thispagestyle{empty}


%\def\stat{emel}


\section*{К 70-летию заместителя директора ИПИ РАН,\\ члена редколлегии журнала
<<Информатика и её применения>>\\ доктора технических наук В.\,И.~Будзко}

\vspace*{18pt}




          \begin{multicols}{2}

%            \label{st\stat}

\begin{center}
\vspace*{1pt}
\mbox{%
\epsfxsize=78mm
\epsfbox{bud-1.eps}
}
\end{center}

\vspace*{12pt}

      14 августа 2014~г.\ исполнилось 70~лет за\-мес\-ти\-те\-лю директора ИПИ РАН по
научной работе доктору технических наук Владимиру Игоревичу Будзко.

      Владимир Игоревич Будзко родился в г.~Москве. Высшее образование получил на факультете
элект\-рон\-но-вы\-чис\-ли\-тель\-ных устройств в Московском
ин\-же\-нер\-но-фи\-зи\-че\-ском институте
(МИФИ), который он окончил в 1968~г., после чего был на\-прав\-лен для прохождения
службы в одну из войс\-ко\-вых частей, где прошел путь от инженера до первого заместителя
командира войсковой части.

      С приходом В.\,И.~Будзко в ИПИ РАН (2001~г.)\ в институте
сформировалось новое научное на\-прав\-ле\-ние теоретических исследований~--- <<Постро\-ение
ин\-фор\-ма\-ци\-он\-но-те\-ле\-ком\-му\-ни\-ка\-ци\-он\-ных\linebreak сис\-тем
высокой до\-ступ\-ности>>. В~рамках этого
направления выполнен широкий круг фундаментальных исследований по поиску подходов и
определению принципов построения средств обеспечения доступности, конфиденциальности
и целостности современных крупномасштабных
ин\-фор\-ма\-ци\-он\-но-те\-ле\-ком\-му\-ни\-ка\-ци\-он\-ных
сис\-тем (ИТС). Разработаны основные сис\-тем\-но-тех\-ни\-че\-ские принципы и базовые
архитектурные решения построения перспективных для условий России ИТС с
централизованной обработкой и хранением информации, сочетающих в себе свойства
высокой доступности, отказо- и катастрофоустойчивости, информационной защищенности.
Определены принципы, методы и математические основы рационального построения и
оптимизации средств восстановления функционирования центров обработки данных (ЦОД)
после возникновения отказов и катастроф, передачи и хранения данных, обеспечения
информационной безопасности при достижении минимальной совокупной стоимости
владения такими системами. Результаты нашли практическое воплощение при реализации
проектов в интересах ряда отечественных государственных и негосударственных
организаций, таких как Банк России (БР), Внешторгбанк, ОАО <<ГМК <<Норильский Никель>>,
<<Газпром>>, Минэкономразвития России, Правительство Москвы, а также ряд силовых
ведомств.

      Под руководством В.\,И.~Будзко начиная с 2001~г.\ выполнен комплекс
      на\-уч\-но-ис\-сле\-до\-ва\-тель\-ских и
      опыт\-но-кон\-ст\-рук\-тор\-ских работ (свыше 100~проектов),
направленных на развитие электронной информационной технологии БР.
Разработаны концепции развития ИТС БР сначала до 2008~г., а затем до 2013~г., которые
были приняты в качестве основы проведения технической политики. За реализацию проекта
<<Катастрофоустойчивая тер\-ри\-то\-ри\-аль\-но-рас\-пре\-де\-лен\-ная
      ин\-фор\-ма\-ци\-он\-но-те\-ле\-ком\-му\-ни\-ка\-ци\-он\-ная сис\-те\-ма централизованной
обработки банковской информации>> В.\,И.~Будзко удостоен Премии Правительства РФ в
области науки и техники за 2010~г.

      В.\,И.~Будзко возглавлял и возглавляет работы по ряду других прикладных проектов,
связанных с созданием, совершенствованием и развитием крупномасштабных ИТС.

      В.\,И.~Будзко~--- генерал-майор, доктор технических наук, член-кор\-рес\-пон\-дент
Академии криптографии РФ, известный ученый в области информатики и применения
информационных технологий при построении территориально распределенных ИТС
различного назначения. Является автором свыше 250~научных работ, опубликованных в
на\-уч\-но-тех\-ни\-че\-ских и специальных изданиях.

    \thispagestyle{empty}

      В.\,И.~Будзко уделяет большое внимание подготовке научных кадров. Под его
руководством защищено 6~диссертаций на соискание ученой степени кандидата
технических наук. Свыше 30~лет он читает лекции в ИКСИ Академии ФСБ, профессор
кафедры НИЯУ МИФИ. Является членом двух диссертационных советов, главным
редактором журнала <<Системы высокой доступности>> и членом редколлегии журнала
<<Информатика и её применения>>.

      \bigskip

      Редакционный совет и Редакционная коллегия журнала <<Информатика и её
применения>> сердечно поздравляют Владимира Игоревича Будзко с 70-ле\-ти\-ем и желают
крепкого здоровья и новых научных достижений.

\end{multicols}

%Информатика и её применения
%Том 12   Выпуск 1-4   Год 2018

\def\stat{cont}
{%\hrule\par
%\vskip 7pt % 7pt
\raggedleft\Large \bf%\baselineskip=3.2ex
А\,В\,Т\,О\,Р\,С\,К\,И\,Й\ \ У\,К\,А\,З\,А\,Т\,Е\,Л\,Ь\ \ З\,А\ \ 2\,0\,1\,8 г. \vskip 17pt
 \hrule
 \par
\vskip 21pt plus 6pt minus 3pt }

\label{st\stat}

\def\tit{\ }

\def\aut{\ }
\def\auf{\ }

\def\leftkol{\ } % ENGLISH ABSTRACTS}

\def\rightkol{\ } %АВТОРСКИЙ УКАЗАТЕЛЬ ЗА 2018 г.} %ENGLISH ABSTRACTS}

\titele{\tit}{\aut}{\auf}{\leftkol}{\rightkol}
\addcontentsline{toc}{subsection}{\textrm\textbf Авторский указатель за 2018 г.}

\vspace*{-12pt}
\vspace*{-36pt}

\noindent
{\tabcolsep=3pt
\begin{tabular}{p{397pt}cc}
&\textbf{Вып.} & \textbf{Стр.}\\[6pt]
\Avtors{Агаларов~Я.\,М.} Оптимизация объема буферной памяти узла коммутации при схеме\linebreak
\\[-12pt]
\hspace*{23pt}полного разделения памяти&4&25--32\\
\Avtors{Агасандян~Г.\,А.} Континуальный критерий VaR на сценарных рынках&1&31--39\\
\Avtors{Алешин~И.\,С.} О формальной постановке задач поиска сгущений в разреженных булевых\linebreak
\\[-12pt]
\hspace*{23pt}матрицах&1&40--48\\
\Avtors{Арутюнов~Е.\,Н., Кудрявцев~А.\,А., Титова~А.\,И.} Гамма-вейбулловский случай априорных\linebreak
\\[-12pt]
\hspace*{23pt}распределений в~байесовских моделях массового обслуживания&4&92--95\\
\Avtors{Атаева~О.\,М., Серебряков~В.\,А.} Онтология цифровой семантической библиотеки LibMeta&1&\hphantom{1}2--10\\
\Avtors{Басок~Б.\,М., Захаров~В.\,Н., Френкель~С.\,Л.} Использование вероятностной модели вычислений для тестирования одного класса готовых к~использованию программных\linebreak
\\[-12pt]
\hspace*{23pt}компонентов локальных и~сетевых систем&4&44--51\\
\Avtors{Батенков~А.\,А., Маньяков~Ю.\,А., Гасилов~А.\,В., Яковлев~О.\,А.} Математическая модель\linebreak
\\[-12pt]
\hspace*{23pt}оптимальной триангуляции&2&69--74\\
\Avtors{Бахтеев~О.\,Ю.} см.~Огальцов~А.\,В.&&\\
\Avtors{Бахтеев~О.\,Ю.} см.~Смердов~А.\,Н.&&\\
\Avtors{Борисов~А.\,В.} Фильтрация состояний марковских скачкообразных процессов по дискре-\linebreak
\\[-12pt]
\hspace*{23pt}тизованным наблюдениям&3&115--121\\
\Avtors{Босов~А.\,В., Игнатов~А.\,Н., Наумов~А.\,В.} Модель передвижения поездов и маневровых локомотивов на железнодорожной станции в приложении к оценке и анализу\linebreak
\\[-12pt]
\hspace*{23pt}вероятности бокового столкновения&3&107--114\\
\Avtors{Босов~А.\,В., Стефанович~А.\,И.} Управление выходом стохастической дифференциальной системы по квадратичному критерию. I.~Оптимальное решение методом динами-\linebreak
\\[-12pt]
\hspace*{23pt}ческого программирования&3&\hphantom{1}99--106\\
\Avtors{Бунтман~Н.\,В., Гончаров~А.\,А., Зацман~И.\,М., Нуриев~В.\,А.} Количественный анализ\linebreak
\\[-12pt]
\hspace*{23pt}результатов машинного перевода с~использованием надкорпусных баз данных&4&\hphantom{1}96--105\\
\Avtors{Бунтман~Н.\,В.} см.~Нуриев~В.\,А.&&\\
\Avtors{Быковец~Е.\,В., Лаврентьев~В.\,В., Назаров~Л.\,В.} Вероятностная модель влияния книги\linebreak
\\[-12pt]
\hspace*{23pt}заказов на процесс цены&2&29--34\\
\Avtors{Васильева~С.\,Н., Кан~Ю.\,С.} Алгоритм визуализации плоского ядра вероятностной меры&2&60--68\\
\Avtors{Виноградов~Д.\,В.} Учет предварительных оценок скорости порождения сходств спарива-\linebreak
\\[-12pt]
\hspace*{23pt}ющей цепью Маркова&1&49--54\\
\Avtors{Вышинский~Л.\,Л., Флеров~Ю.\,А., Широков~Н.\,И.} Автоматизированная система весового\linebreak
\\[-12pt]
\hspace*{23pt}проектирования самолетов&1&18--30\\
\Avtors{Гайдамака~Ю.\,В.} см.~Горбунова~А.\,В.&&\\
\Avtors{Гайдамака~Ю.\,В.} см.~Самуйлов~К.\,Е.&&\\
\Avtors{Гасилов~А.\,В.} см.~Батенков~А.\,А.,&&\\
\Avtors{Гончаров~А.\,А.} см.~Бунтман~Н.\,В.&&\\
\Avtors{Горбунова~А.\,В., Наумов~В.\,А., Гайдамака~Ю.\,В., Самуйлов~К.\,Е.} Ресурсные системы\linebreak
\\[-12pt]
\hspace*{23pt}массового обслуживания как модели беспроводных систем связи&3&48--55\\
\Avtors{Горшенин~А.\,К.} Зашумление данных конечными смесями нормальных и гамма-рас\-пре-\linebreak
\\[-12pt]
\hspace*{23pt}де\-ле\-ний с применением к задаче округления наблюдений&3&28--34\\
\Avtors{Горшенин~А.\,К.} Развитие сервисов цифровых платформ для преодоления нефинансовых\linebreak
\\[-12pt]
\hspace*{23pt}барьеров&4&106--112\\
\Avtors{Горшенин~А.\,К., Королев~В.\,Ю.} Определение экстремальности объемов осадков на основе\linebreak
\\[-12pt]
\hspace*{23pt}модифицированного метода превышения порогового значения&4&16--24\\
\Avtors{Горшенин~А.\,К.} см.~Королев~В.\,Ю.&&\\
\end{tabular}
}

\pagebreak

\def\leftkol{АВТОРСКИЙ УКАЗАТЕЛЬ ЗА 2018 г.} % ENGLISH ABSTRACTS}

\def\rightkol{АВТОРСКИЙ УКАЗАТЕЛЬ ЗА 2018 г.} %ENGLISH ABSTRACTS}

%\thispagestyle{myheadings}
\def\leftfootline{\small{\textbf{\thepage}
\hfill ИНФОРМАТИКА И ЕЁ ПРИМЕНЕНИЯ\ \ \ том~12\ \ \ выпуск~4\ \ \ 2018}
}%
 \def\rightfootline{\small{ИНФОРМАТИКА И ЕЁ ПРИМЕНЕНИЯ\ \ \ том~12\ \ \ выпуск~4\ \ \ 2018
 \hfill \textbf{\thepage}}}


\noindent
{\tabcolsep=3pt
\begin{tabular}{p{394pt}cc}
&\textbf{Вып.} & \textbf{Стр.}\\[3pt]
\Avtors{Грушо~А.\,А., Грушо~Н.\,А., Забежайло~М.\,И., Смирнов~Д.\,В., Тимонина~Е.\,Е.} Параметриза-\linebreak
\\[-12pt]
\hspace*{23pt}ция в прикладных задачах поиска эмпирических причин&3&62--66\\
\Avtors{Грушо~А.\,А., Грушо~Н.\,А., Левыкин~М.\,В., Тимонина~Е.\,Е.} Методы идентификации захвата хоста в~распределенной информационно-вычислительной сис\-те\-ме, защищенной\linebreak
\\[-12pt]
\hspace*{23pt}с~по\-мощью метаданных&4&39--43\\
\Avtors{Грушо~А.\,А., Забежайло~М.\,И., Зацаринный~А.\,А., Тимонина~Е.\,Е.} О некоторых возможностях управления ресурсами при организации проактивного противодействия\linebreak
\\[-12pt]
\hspace*{23pt}компьютерным атакам&1&62--70\\
\Avtors{Грушо~А.\,А., Тимонина~Е.\,Е., Шоргин~С.\,Я.} Иерархический метод порождения метадан-\linebreak
\\[-12pt]
\hspace*{23pt}ных для управления сетевыми соединениями&2&44--49\\
\Avtors{Грушо~Н.\,А.} см.~Грушо~А.\,А.&&\\
\Avtors{Грушо~Н.\,А.} см.~Грушо~А.\,А.&&\\
\Avtors{Дорофеева~А.\,В.} см.~Королев~В.\,Ю.&&\\
\Avtors{Егоров~А.\,Ю.} см.~Шнурков~П.\,В.&&\\
\Avtors{Егоров~А.\,Ю.} см.~Шнурков~П.\,В.&&\\
\Avtors{Жуков~Д.\,О., Хватова~Т.\,Ю., Лесько~С.\,А., Зальцман~А.\,Д.} Влияние плотности связей на кластеризацию и порог перколяции при распространении информации в~со\-ци\-аль-\linebreak
\\[-12pt]
\hspace*{23pt}ных сетях&2&90--97\\
\Avtors{Забежайло~М.\,И.} см.~Грушо~А.\,А.&&\\
\Avtors{Забежайло~М.\,И.} см.~Грушо~А.\,А.&&\\
\Avtors{Зальцман~А.\,Д.} см.~Жуков~Д.\,О.&&\\
\Avtors{Захаров~В.\,Н.} см.~Басок~Б.\,М.&&\\
\Avtors{Захаров~В.\,Н.} см.~Шанин~И.\,А.&&\\
\Avtors{Зацаринный~А.\,А., Сучков~А.\,П.} Система ситуационного управления как мультисервисная\linebreak
\\[-12pt]
\hspace*{23pt}технология в облачной среде&1&78--88\\
\Avtors{Зацаринный~А.\,А.} см.~Грушо~А.\,А.&&\\
\Avtors{Зацман~И.\,М.} Имплицированные знания: основания и технологии извлечения&3&74--82\\
\Avtors{Зацман~И.\,М.} см.~Бунтман~Н.\,В.&&\\
\Avtors{Зейфман~А.\,И.} см.~Королев~В.\,Ю.&&\\
\Avtors{Зубарев~Д.\,В.} см.~Соченков~И.\,В.&&\\
\Avtors{Игнатов~А.\,Н.} см.~Босов~А.\,В.&&\\
\Avtors{Инькова~О.\,Ю., Кружков~М.\,Г.} Статистический анализ лингвоспецифичности коннек-\linebreak
\\[-12pt]
\hspace*{23pt}торов (на материале параллельных корпусов)&3&83--90\\
\Avtors{Инькова~О.\,Ю.} см.~Нуриев~В.\,А.&&\\
\Avtors{Кан~Ю.\,С.} см.~Васильева~С.\,Н.&&\\
\Avtors{Ковалёв~С.\,П.} Теория категорий как математическая прагматика модельно-ори\-ен\-ти-\linebreak
\\[-12pt]
\hspace*{23pt}ро\-ван\-ной системной инженерии&1&\hphantom{1}95--104\\
\Avtors{Козеренко~Е.\,Б., Кузнецов~К.\,И., Романов~Д.\,А.} Семантическая обработка неструктури-\linebreak
\\[-12pt]
\hspace*{23pt}рованных текстовых данных на основе лингвистического процессора PullEnti&3&91--98\\
\Avtors{Кондранин~Е.\,С., Ушаков~В.\,Г.} Система обслуживания с~относительным приоритетом\linebreak
\\[-12pt]
\hspace*{23pt}и~профилактиками прибора&4&33--38\\
\Avtors{Коновалов~М.\,Г., Разумчик~Р.\,В.} Сравнение двух механизмов активного управления\linebreak
\\[-12pt]
\hspace*{23pt}очередью в~системе $M/D/1/N$&4&\hphantom{1}9--15\\
\Avtors{Коновалов~М.\,Г., Разумчик~Р.\,В.} Управление случайным блужданием с эталонным\linebreak
\\[-12pt]
\hspace*{23pt}стационарным распределением&3&\hphantom{1}2--13\\
\Avtors{Королев~В.\,Ю., Горшенин~А.\,К., Зейфман~А.\,И.} Новые представления обобщенного\linebreak
\\[-12pt]
\hspace*{23pt}распределения Миттаг-Леффлера в~виде смесей и~их приложения&4&75--85\\
\Avtors{Королев~В.\,Ю., Дорофеева~А.\,В.} О~неравномерных оценках точности нормальной аппроксимации для распределений некоторых случайных сумм при ослабленных\linebreak
\\[-12pt]
\hspace*{23pt}моментных условиях&4&86--91\\
\Avtors{Королев~В.\,Ю.} см.~Горшенин~А.\,К.&&\\
\Avtors{Кривенко~М.\,П.}\ Обучаемая классификация данных с учетом анализа главных компонент&3&56--61\\
\Avtors{Кривенко~М.\,П.}\ Реконструкция осей главных компонент&1&71--77\\
\Avtors{Кружков~М.\,Г.} см.~Инькова~О.\,Ю.&&\\
\end{tabular}
}

\pagebreak

\def\leftkol{АВТОРСКИЙ УКАЗАТЕЛЬ ЗА 2018 г.} % ENGLISH ABSTRACTS}

\def\rightkol{АВТОРСКИЙ УКАЗАТЕЛЬ ЗА 2018 г.} %ENGLISH ABSTRACTS}

%\thispagestyle{myheadings}
\def\leftfootline{\small{\textbf{\thepage}
\hfill ИНФОРМАТИКА И ЕЁ ПРИМЕНЕНИЯ\ \ \ том~12\ \ \ выпуск~4\ \ \ 2018}
}%
 \def\rightfootline{\small{ИНФОРМАТИКА И ЕЁ ПРИМЕНЕНИЯ\ \ \ том~12\ \ \ выпуск~4\ \ \ 2018
 \hfill \textbf{\thepage}}}


\noindent
{\tabcolsep=3pt
\begin{tabular}{p{394pt}cc}
&\textbf{Вып.} & \textbf{Стр.}\\[3pt]
\Avtors{Кудрявцев~А.\,А.} Байесовские модели баланса&3&18--27\\
\Avtors{Кудрявцев~А.\,А., Шестаков~О.\,В.} Байесовские модели тестирования больших групп\linebreak
\\[-12pt]
\hspace*{23pt}обслуживающих приборов&1&105--108\\
\Avtors{Кудрявцев~А.\,А., Шестаков О.\,В.} Минимизация ошибок вычисления вейвлет-ко\-эф\-фи-\linebreak
\\[-12pt]
\hspace*{23pt}ци\-ен\-тов при решении обратных задач&2&17--23\\
\Avtors{Кудрявцев~А.\,А.} см.~Арутюнов~Е.\,Н.&&\\
\Avtors{Кузнецов~К.\,И.} см.~Козеренко~Е.\,Б.&&\\
\Avtors{Лаврентьев~В.\,В.} см.~Быковец~Е.\,В.&&\\
\Avtors{Лебедев~А.\,В.} Максимальные ветвящиеся процессы в случайной среде&2&35--43\\
\Avtors{Левыкин~М.\,В.} см.~Грушо~А.\,А.&&\\
\Avtors{Лери~М.\,М., Павлов~Ю.\,Л.} Об устойчивости конфигурационных графов в случайной\linebreak
\\[-12pt]
\hspace*{23pt}среде&2&\hphantom{1}2--10\\
\Avtors{Лесько~С.\,А.} см.~Жуков~Д.\,О.&&\\
\Avtors{Логачев~О.\,А.} Теоретико-информационная характеризация совершенно уравновешен-\linebreak
\\[-12pt]
\hspace*{23pt}ных функций&4&70--74\\
\Avtors{Малашенко~Ю.\,Е., Назарова~И.\,А., Новикова~Н.\,М.} Анализ разрезных повреждений\linebreak
\\[-12pt]
\hspace*{23pt}в~многополюсных сетях&3&35--41\\
\Avtors{Малашенко~Ю.\,Е., Назарова~И.\,А., Новикова~Н.\,М.} Диаграммы уязвимости потоковых\linebreak
\\[-12pt]
\hspace*{23pt}сетевых систем&1&11--17\\
\Avtors{Маньяков~Ю.\,А.} см.~Батенков~А.\,А.&&\\
\Avtors{Мирзабеков~Я.\,М., Шихиев~Ш.\,Б.} Дискретный анализ в синтаксическом анализе&2&\hphantom{1}98--104\\
\Avtors{Мистрюков~А.\,В., Ушаков~В.\,Г.} Достаточные условия эргодичности приоритетных\linebreak
\\[-12pt]
\hspace*{23pt}систем массового обслуживания&2&24--28\\
\Avtors{Назаров~Л.\,В.} см.~Быковец~Е.\,В.&&\\
\Avtors{Назарова~И.\,А.} см.~Малашенко~Ю.\,Е.&&\\
\Avtors{Назарова~И.\,А.} см.~Малашенко~Ю.\,Е.&&\\
\Avtors{Наумов~А.\,В.} см.~Босов~А.\,В.&&\\
\Avtors{Наумов~В.\,А.} см.~Горбунова~А.\,В.&&\\
\Avtors{Наумов~В.\,А.} см.~Сопин~Э.\,С.&&\\
\Avtors{Новикова~Н.\,М.} см.~Малашенко~Ю.\,Е.&&\\
\Avtors{Новикова~Н.\,М.} см.~Малашенко~Ю.\,Е.&&\\
\Avtors{Нуриев~В.\,А., Бунтман~Н.\,В., Инькова~О.\,Ю.} Ошибки и~неточности машинного перевода\linebreak
\\[-12pt]
\hspace*{23pt}русских коннекторов на~французский язык&2&105--113\\
\Avtors{Нуриев~В.\,А.} см.~Бунтман~Н.\,В.&&\\
\Avtors{Огальцов~А.\,В., Бахтеев~О.\,Ю.} Автоматическое извлечение метаданных из научных\linebreak
\\[-12pt]
\hspace*{23pt}PDF-документов&2&75--82\\
\Avtors{Павлов~Ю.\,Л.} см.~Лери~М.\,М.&&\\
\Avtors{Разумчик~Р.\,В.} см.~Коновалов~М.\,Г.&&\\
\Avtors{Разумчик~Р.\,В.} см.~Коновалов~М.\,Г.&&\\
\Avtors{Романов~Д.\,А.} см.~Козеренко~Е.\,Б.&&\\
\Avtors{Самуйлов~К.\,Е., Гайдамака~Ю.\,В., Шоргин~С.\,Я.} Применение моделей случайного\linebreak
\\[-12pt]
\hspace*{23pt}блуждания при моделировании перемещения устройств в~беспроводной сети&4&2--8\\
\Avtors{Самуйлов~К.\,Е.} см.~Горбунова~А.\,В.&&\\
\Avtors{Самуйлов~К.\,Е.} см.~Сопин~Э.\,С.&&\\
\Avtors{Серебряков~В.\,А.} см.~Атаева~О.\,М.&&\\
\Avtors{Синицын~И.\,Н.} Метод интерполяционного аналитического моделирования одномерных\linebreak
\\[-12pt]
\hspace*{23pt}распределений в стохастических системах&1&55--61\\
\Avtors{Смердов~А.\,Н., Бахтеев~О.\,Ю., Стрижов~В.\,В.} Выбор оптимальной модели рекуррентной\linebreak
\\[-12pt]
\hspace*{23pt}сети в~задачах поиска парафраза&4&63--69\\
\Avtors{Смирнов~Д.\,В.} см.~Грушо~А.\,А.&&\\
\Avtors{Сопин~Э.\,С., Наумов~В.\,А., Самуйлов~К.\,Е.} Об инвариантности стационарного распределения системы массового обслуживания с ограниченными ресурсами и~с~ин\-тен-\linebreak
\\[-12pt]
\hspace*{23pt}сив\-ностями поступления и~обслуживания, зависящими от состояния системы&3&42--47\\
\Avtors{Соченков~И.\,В., Зубарев~Д.\,В., Тихомиров~И.\,А.} Эксплоративный патентный поиск&1&89--94\\
\end{tabular}
}

\pagebreak

\def\leftkol{АВТОРСКИЙ УКАЗАТЕЛЬ ЗА 2018 г.} % ENGLISH ABSTRACTS}

\def\rightkol{АВТОРСКИЙ УКАЗАТЕЛЬ ЗА 2018 г.} %ENGLISH ABSTRACTS}

%\thispagestyle{myheadings}
\def\leftfootline{\small{\textbf{\thepage}
\hfill ИНФОРМАТИКА И ЕЁ ПРИМЕНЕНИЯ\ \ \ том~12\ \ \ выпуск~4\ \ \ 2018}
}%
 \def\rightfootline{\small{ИНФОРМАТИКА И ЕЁ ПРИМЕНЕНИЯ\ \ \ том~12\ \ \ выпуск~4\ \ \ 2018
 \hfill \textbf{\thepage}}}


\noindent
{\tabcolsep=3pt
\begin{tabular}{p{394pt}cc}
&\textbf{Вып.} & \textbf{Стр.}\\[3pt]
\Avtors{Стефанович~А.\,И.} см.~Босов~А.\,В.&&\\
\Avtors{Стрижов~В.\,В.} см.~Смердов~А.\,Н.&&\\
\Avtors{Ступников~С.\,А.} см.~Шанин~И.\,А.&&\\
\Avtors{Сурина~А.\,А.} см.~Тырсин~А.\,Н.&&\\
\Avtors{Сучков~А.\,П.} см.~Зацаринный~А.\,А.&&\\
\Avtors{Сюнтюренко~О.\,В.} Финансирование фундаментальных исследований: концептуальный облик системы поддержки принятия решений с использованием методов\linebreak
\\[-12pt]
\hspace*{23pt}наукометрии и анализа данных&1&118--127\\
\Avtors{Тимонина~Е.\,Е.} см.~Грушо~А.\,А.&&\\
\Avtors{Тимонина~Е.\,Е.} см.~Грушо~А.\,А.&&\\
\Avtors{Тимонина~Е.\,Е.} см.~Грушо~А.\,А.&&\\
\Avtors{Тимонина~Е.\,Е.} см.~Грушо~А.\,А.&&\\
\Avtors{Титова~А.\,И.} см.~Арутюнов~Е.\,Н.&&\\
\Avtors{Тихомиров~И.\,А.} см.~Соченков~И.\,В.&&\\
\Avtors{Тырсин~А.\,Н., Сурина~А.\,А.} Модели управления риском в гауссовских стохастических\linebreak
\\[-12pt]
\hspace*{23pt}системах&2&50--59\\
\Avtors{Ушаков~В.\,Г.} см.~Кондранин~Е.\,С.&&\\
\Avtors{Ушаков~В.\,Г.} см.~Мистрюков~А.\,В.&&\\
\Avtors{Флеров~Ю.\,А.} см.~Вышинский~Л.\,Л.&&\\
\Avtors{Френкель~С.\,Л., Ханкин~Д.} Непрерывные обновления маршрута в~SDN с~использованием\linebreak
\\[-12pt]
\hspace*{23pt}проверки соответствия качеству обслуживания&4&52--62\\
\Avtors{Френкель~С.\,Л.} см.~Басок~Б.\,М.&&\\
\Avtors{Ханкин~Д.} см.~Френкель~С.\,Л.&&\\
\Avtors{Хватова~Т.\,Ю.} см.~Жуков~Д.\,О.&&\\
\Avtors{Шанин~И.\,А., Ступников~С.\,А., Захаров~В.\,Н.} Методы и средства обнаружения нештатных\linebreak
\\[-12pt]
\hspace*{23pt}ситуаций, возникающих на элементах жилищно-коммунальной инфраструктуры&3&67--73\\
\Avtors{Шестаков~О.\,В.} Несмещенная оценка риска стабилизированной жесткой пороговой\linebreak
\\[-12pt]
\hspace*{23pt}обработки в модели с долгосрочной зависимостью&2&11--16\\
\Avtors{Шестаков~О.\,В.} Среднеквадратичный риск пороговой обработки при случайном объеме\linebreak
\\[-12pt]
\hspace*{23pt}выборки&3&14--17\\
\Avtors{Шестаков~О.\,В.} см.~Кудрявцев~А.\,А.&&\\
\Avtors{Шестаков~О.\,В.} см.~Кудрявцев~А.\,А.&&\\
\Avtors{Широков~Н.\,И.} см.~Вышинский~Л.\,Л.&&\\
\Avtors{Шихиев~Ш.\,Б.} см.~Мирзабеков~Я.\,М.&&\\
\Avtors{Шнурков~П.\,В., Егоров~А.\,Ю.} Разработка и предварительное исследование стохастической полумарковской модели управления запасом непрерывного продукта при\linebreak
\\[-12pt]
\hspace*{23pt}постоянно происходящем потреблении&1&109--117\\
\Avtors{Шнурков~П.\,В., Егоров~А.\,Ю.} Решение проблемы оптимального управления запасом непрерывного продукта при постоянно происходящем потреблении в стохастической\linebreak
\\[-12pt]
\hspace*{23pt}полумарковской модели&2&83--89\\
\Avtors{Шоргин~С.\,Я.} см.~Грушо~А.\,А.&&\\
\Avtors{Шоргин~С.\,Я.} см.~Самуйлов~К.\,Е.&&\\
\Avtors{Яковлев~О.\,А.} см.~Батенков~А.\,А.&&\\
\end{tabular}
}

%\thispagestyle{myheadings}
\def\leftfootline{\small{\textbf{\thepage}
\hfill ИНФОРМАТИКА И ЕЁ ПРИМЕНЕНИЯ\ \ \ том~12\ \ \ выпуск~4\ \ \ 2018}
}%
 \def\rightfootline{\small{ИНФОРМАТИКА И ЕЁ ПРИМЕНЕНИЯ\ \ \ том~12\ \ \ выпуск~4\ \ \ 2018
 \hfill \textbf{\thepage}}}

 \label{end\stat}

\newpage

%Информатика и её применения
%Том 12   Выпуск 1-4   Год 2018

\def\stat{cont-e}
{%\hrule\par
%\vskip 7pt % 7pt
\raggedleft\Large \bf%\baselineskip=3.2ex
2\,0\,1\,8\ \ A\,U\,T\,H\,O\,R\ \ I\,N\,D\,E\,X \vskip 17pt
 \hrule
 \par
\vskip 21pt plus 6pt minus 3pt }

\label{st\stat}

\def\tit{\ }

\def\aut{\ }
\def\auf{\ }

\def\leftkol{\ } %2018 AUTHOR INDEX} % ENGLISH ABSTRACTS}

\def\rightkol{\ } %2018 AUTHOR INDEX} %ENGLISH ABSTRACTS}

\titele{\tit}{\aut}{\auf}{\leftkol}{\rightkol}
\addcontentsline{toc}{subsection}{\textrm\textbf 2018 Author Index}

\def\leftfootline{\small{\textbf{\thepage}
\hfill INFORMATIKA I EE PRIMENENIYA~--- INFORMATICS AND APPLICATIONS\ \ \ 2018\
\ \ volume~12\ \ \ issue\ 4}
}%
 \def\rightfootline{\small{INFORMATIKA I EE PRIMENENIYA~--- INFORMATICS AND APPLICATIONS\ \ \ 2018\ \ \ volume~12\ \ \ issue\ 4
\hfill \textbf{\thepage}}}

\vspace*{-12pt}
\vspace*{-18pt}

\noindent
{\tabcolsep=3pt
\begin{tabular}{p{396pt}cc}
&\textbf{Issue} & \textbf{Page}\\[6pt]
\Avtors{Agalarov~Yа.\,M.} Optimization of buffer memory size of switching node in mode of full memory\linebreak
\\[-12pt]
\hspace*{23pt}sharing&4&25--32\\
\Avtors{Agasandyan~G.\,A.} Continuous VaR-criterion in scenario markets&1&31--39\\
\Avtors{Aleshin~I.\,S.} On the formalization of tasks searching dense submatrices in boolean sparse\linebreak
\\[-12pt]
\hspace*{23pt}matrices&1&40--48\\
\Avtors{Arutyunov~E.\,N., Kudryavtsev~A.\,A., and~Titova~A.\,I.} Gamma-Weibull \textit{a~priori} distributions\linebreak
\\[-12pt]
\hspace*{23pt}in~Bayesian queuing models&4&92--95\\
\Avtors{Ataeva~O.\,M.} see~Serebryakov~V.\,A.&&\\
\Avtors{Bakhteev~O.\,Y.} see~Ogaltsov~A.\,V.&&\\
\Avtors{Bakhteev~O.\,Y.} see~Smerdov~A.\,N.&&\\
\Avtors{Basok~B.\,M., Zakharov~V.\,N., and~Frenkel~S.\,L.} Using a probabilistic calculation model to test\linebreak
\\[-12pt]
\hspace*{23pt}one class of ready-to-use software components of local and network systems&4&44--51\\
\Avtors{Batenkov~A.\,A., Maniakov Yu.\,A., Gasilov A.\,V., and Yakovlev O.\,A.} Mathematical model\linebreak
\\[-12pt]
\hspace*{23pt}of~optimal triangulation&2&69--74\\
\Avtors{Borisov~A.\,V.} Filtering of Markov jump processes by discretized observations&3&115--121\\
\Avtors{Bosov~A.\,V., Ignatov~A.\,N., and Naumov~A.\,V.} Model of transportation of trains and shunting\linebreak
\\[-12pt]
\hspace*{23pt}locomotives at a railway station for evaluation and analysis of side-collision probability&3&107--114\\
\Avtors{Bosov~A.\,V.\ and Stefanovich~A.\,I.} Stochastic differential system output control by the quadratic\linebreak
\\[-12pt]
\hspace*{23pt}criterion. I.~Dynamic programming optimal solution&3&\hphantom{1}99--106\\
\Avtors{Buntman~N.\,V., Goncharov~A.\,A., Zatsman~I.\,M., and~Nuriev~V.\,A.} Using supracorpora databases\linebreak
\\[-12pt]
\hspace*{23pt}for quantitative analysis of machine translations&4&\hphantom{1}96--105\\
\Avtors{Buntman~N.\,V.} see~Nuriev~V.\,A., &&\\
\Avtors{Bykovets~E.\,V.} see~Nazarov~L.\,V.&&\\
\Avtors{Dorofeeva~A.\,V.} see~Korolev~V.\,Yu.&&\\
\Avtors{Egorov~A.\,Y.} see~Shnurkov~P.\,V.&&\\
\Avtors{Egorov~A.\,Y.} see~Shnurkov~P.\,V.&&\\
\Avtors{Flerov~Yu.\,A.} see~Vyshinsky~L.\,L.&&\\
\Avtors{Frenkel~S.\,L.\ and Khankin~D.} Seamless route updates in software-defined networking via quality\linebreak
\\[-12pt]
\hspace*{23pt}of~service compliance verification &4&52--62\\
\Avtors{Frenkel~S.\,L.} see~Basok~B.\,M.&&\\
\Avtors{Gaidamaka~Yu.\,V.} see~Gorbunova~A.\,V.&&\\
\Avtors{Gaidamaka~Yu.\,V.} see~Samouylov~K.\,E.&&\\
\Avtors{Gasilov A.\,V.} see~Batenkov~A.\,A.&&\\
\Avtors{Goncharov~A.\,A.} see~Buntman~N.\,V.&&\\
\Avtors{Gorbunova~A.\,V., Naumov~V.\,A., Gaidamaka~Yu.\,V., and Samouylov~K.\,E.} Resource queuing\linebreak
\\[-12pt]
\hspace*{23pt}systems as models of wireless communication systems&3&48--55\\
\Avtors{Gorshenin~A.\,K.} Data noising by finite normal and gamma mixtures with application to~the~prob-\linebreak
\\[-12pt]
\hspace*{23pt}lem of rounded observations&3&28--34\\
\Avtors{Gorshenin~A.\,K.} Development of services of digital platforms to overcome nonfinancial barriers&4&106--112\\
\Avtors{Gorshenin~A.\,K.\ and~Korolev~V.\,Yu.} Determining the extremes of precipitation volumes based\linebreak
\\[-12pt]
\hspace*{23pt}on~the~modified ``Peaks over Threshold'' method&4&16--24\\
\Avtors{Gorshenin~A.\,K.} see~Korolev~V.\,Yu.&&\\
\Avtors{Grusho~A.\,A., Grusho~N.\,A., Levykin~M.\,V., and~Timonina~E.\,E.} Methods of identification of host\linebreak
\\[-12pt]
\hspace*{23pt}capture in a distributed information system which is protected on the basis of meta data&4&39--43\\
\Avtors{Grusho~A.\,A., Grusho~N.\,A., Zabezhailo~M.\,I., Smirnov~D.\,V., and Timonina~E.\,E.} Parametrization\linebreak
\\[-12pt]
\hspace*{23pt}in applied problems of search of empirical reasons&3&62--66\\
\end{tabular}
}
\pagebreak

\def\leftfootline{\small{\textbf{\thepage}
\hfill INFORMATIKA I EE PRIMENENIYA~--- INFORMATICS AND APPLICATIONS\ \ \ 2018\
\ \ volume~12\ \ \ issue\ 4}
}%
 \def\rightfootline{\small{INFORMATIKA I EE PRIMENENIYA~---
INFORMATICS AND APPLICATIONS\ \ \ 2018\ \ \ volume~12\ \ \ issue\ 4
\hfill \textbf{\thepage}}}

\def\leftkol{2018 AUTHOR INDEX} % ENGLISH ABSTRACTS}

\def\rightkol{2018 AUTHOR INDEX} %ENGLISH ABSTRACTS}


\noindent
{\tabcolsep=3pt
\begin{tabular}{p{395.48108pt}cc}
&\textbf{Issue} & \textbf{Page}\\[6pt]
\Avtors{Grusho~A.\,A., Timonina~E.\,E., and Shorgin~S.\,Ya.} Hierarchical method of meta data generation\linebreak
\\[-12pt]
\hspace*{23pt}for control of network connections&2&44--49\\
\Avtors{Grusho~A.\,A., Zabezhailo~M.\,I., Zatsarinny~A.\,A., and Timonina~E.\,E.} On some possibilities\linebreak
\\[-12pt]
\hspace*{23pt}of~resource management for organizing active counteraction to computer attacks&1&62--70\\
\Avtors{Grusho~N.\,A.} see~Grusho~A.\,A.&&\\
\Avtors{Grusho~N.\,A.} see~Grusho~A.\,A.&&\\
\Avtors{Ignatov~A.\,N.} see~Bosov~A.\,V.&&\\
\Avtors{Inkova~O.\,Yu.\ and Kruzhkov~M.\,G.} Statistical analysis of language specificity of connectives\linebreak
\\[-12pt]
\hspace*{23pt}based on parallel texts&3&83--90\\
\Avtors{Inkova~O.\,Yu.} see~Nuriev~V.\,A., &&\\
\Avtors{Kan~Yu.\,S.} see~Vasil'eva~S.\,N.&&\\
\Avtors{Khankin~D.} see~Frenkel~S.\,L.&&\\
\Avtors{Khvatova~T.\,Yu.} see~Zhukov~D.\,O.&&\\
\Avtors{Kondranin~E.\,S.\ and~Ushakov~V.\,G.} A~head of the line priority queue with working vacations&4&33--38\\
\Avtors{Konovalov~M.\,G.\ and Razumchik~R.\,V.} Comparison of two active queue management schemes\linebreak
\\[-12pt]
\hspace*{23pt}through the $M/D/1/N$ queue&4&\hphantom{1}9--15\\
\Avtors{Konovalov~M.\,G.\ and Razumchik~R.\,V.} Finding control policy for one discrete-time Markov\linebreak
\\[-12pt]
\hspace*{23pt}chain on [0,1] with a given invariant measure&3&\hphantom{1}2--13\\
\Avtors{Korolev~V.\,Yu.\ and~Dorofeeva~A.\,V.} On nonuniform estimates of accuracy of normal approxima-\linebreak
\\[-12pt]
\hspace*{23pt}tion for distributions of some random sums under relaxed moment conditions&4&86--91\\
\Avtors{Korolev~V.\,Yu., Gorshenin~A.\,K., and~Zeifman~A.\,I. } New mixture representations of~the~general-\linebreak
\\[-12pt]
\hspace*{23pt}ized Mittag-Leffler distribution and their applications&4&75--85\\
\Avtors{Korolev~V.\,Yu.} see~Gorshenin~A.\,K.&&\\
\Avtors{Kovalyov~S.\,P.} Category theory as a mathematical pragmatics of model-based systems engineer-\linebreak
\\[-12pt]
\hspace*{23pt}ing&1&\hphantom{1}95--104\\
\Avtors{Kozerenko~E.\,B., Kuznetsov~K.\,I., and Romanov~D.\,A.} Semantic processing of unstructured\linebreak
\\[-12pt]
\hspace*{23pt}textual data based on the linguistic processor PullEnti&3&91--98\\
\Avtors{Krivenko~M.\,P.} Principal axes reconstruction&1&71--77\\
\Avtors{Krivenko~M.\,P.} Supervised learning classification of data taking into account principal compo-\linebreak
\\[-12pt]
\hspace*{23pt}nent analysis&3&56--61\\
\Avtors{Kruzhkov~M.\,G.} see~Inkova~O.\,Yu.&&\\
\Avtors{Kudryavtsev~A.\,A.} Bayesian balance models&3&18--27\\
\Avtors{Kudryavtsev~A.\,A.\ and Shestakov~O.\,V.} Bayesian models for testing large groups of service devices&1&105--108\\
\Avtors{Kudryavtsev~A.\,A.\ and Shestakov~O.\,V.} Minimization of errors of calculating wavelet coefficients\linebreak
\\[-12pt]
\hspace*{23pt}while solving inverse problems&2&17--23\\
\Avtors{Kudryavtsev~A.\,A.} see~Arutyunov~E.\,N.&&\\
\Avtors{Kuznetsov~K.\,I.} see~Kozerenko~E.\,B.&&\\
\Avtors{Lavrentyev~V.\,V.} see~Nazarov~L.\,V.&&\\
\Avtors{Lebedev~A.\,V.} Maximal branching processes in random environment&2&35--43\\
\Avtors{Leri~M.\,M.\ and Pavlov~Yu.\,L.} On the robustness of configuration graphs in a random environment&2&\hphantom{1}2--10\\
\Avtors{Lesko~S.\,A.} see~Zhukov~D.\,O.&&\\
\Avtors{Levykin~M.\,V.} see~Grusho~A.\,A.&&\\
\Avtors{Logachev~O.\,A.} An information based criterion for perfectly balanced functions&4&70--74\\
\Avtors{Malashenko~Yu.\,E., Nazarova~I.\,A., and Novikova~N.\,M.} Analysis of cutting damages to multipolar\linebreak
\\[-12pt]
\hspace*{23pt}networks&3&35--41\\
\Avtors{Malashenko~Yu.\,E., Nazarova~I.\,A., and Novikova~N.\,M.} Diagrams of the functional vulnerability\linebreak
\\[-12pt]
\hspace*{23pt}of flow network systems&1&11--17\\
\Avtors{Maniakov Yu.\,A.} see~Batenkov~A.\,A.&&\\
\Avtors{Mirzabekov~Ya.\,M.\ and Shihiev~Sh.\,B.} Discrete analysis in parsing&2&\hphantom{1}98--104\\
\Avtors{Mistryukov~A.\,V.\ and Ushakov~V.\,G.} Sufficient ergodicity conditions for priority queues&2&24--28\\
\Avtors{Naumov~A.\,V.} see~Bosov~A.\,V.&&\\
\Avtors{Naumov~V.\,A.} see~Gorbunova~A.\,V.&&\\
\Avtors{Naumov~V.\,A.} see~Sopin~E.\,S.&&\\
\end{tabular}
}
\pagebreak

\def\leftfootline{\small{\textbf{\thepage}
\hfill INFORMATIKA I EE PRIMENENIYA~--- INFORMATICS AND APPLICATIONS\ \ \ 2018\
\ \ volume~12\ \ \ issue\ 4}
}%
 \def\rightfootline{\small{INFORMATIKA I EE PRIMENENIYA~---
INFORMATICS AND APPLICATIONS\ \ \ 2018\ \ \ volume~12\ \ \ issue\ 4
\hfill \textbf{\thepage}}}

\def\leftkol{2018 AUTHOR INDEX} % ENGLISH ABSTRACTS}

\def\rightkol{2018 AUTHOR INDEX} %ENGLISH ABSTRACTS}


\noindent
{\tabcolsep=3pt
\begin{tabular}{p{395.48108pt}cc}
&\textbf{Issue} & \textbf{Page}\\[6pt]
\Avtors{Nazarov~L.\,V., Lavrentyev~V.\,V., and Bykovets~E.\,V.} A~probability model of the influence\linebreak
\\[-12pt]
\hspace*{23pt}of~the~order book on the price process&2&29--34\\
\Avtors{Nazarova~I.\,A.} see~Malashenko~Yu.\,E.&&\\
\Avtors{Nazarova~I.\,A.} see~Malashenko~Yu.\,E.&&\\
\Avtors{Novikova~N.\,M.} see~Malashenko~Yu.\,E.&&\\
\Avtors{Novikova~N.\,M.} see~Malashenko~Yu.\,E.&&\\
\Avtors{Nuriev~V.\,A., Buntman~N.\,V., and Inkova~O.\,Yu.} Machine translation of russian connectives into\linebreak
\\[-12pt]
\hspace*{23pt}french: Errors and quality failures&2&105--113\\
\Avtors{Nuriev~V.\,A.} see~Buntman~N.\,V.&&\\
\Avtors{Ogaltsov~A.\,V.\ and Bakhteev~O.\,Y.} Automatic metadata extraction from scientific PDF documents&2&75--82\\
\Avtors{Pavlov~Yu.\,L.} see~Leri~M.\,M.&&\\
\Avtors{Razumchik~R.\,V.} see~Konovalov~M.\,G.&&\\
\Avtors{Razumchik~R.\,V.} see~Konovalov~M.\,G.&&\\
\Avtors{Romanov~D.\,A.} see~Kozerenko~E.\,B.&&\\
\Avtors{Samouylov~K.\,E., Gaidamaka~Yu.\,V., and~Shorgin~S.\,Ya.} Modeling movement of devices in\linebreak
\\[-12pt]
\hspace*{23pt}a~wireless network by random walk models&4&2--8\\
\Avtors{Samouylov~K.\,E.} see~Gorbunova~A.\,V.&&\\
\Avtors{Samouylov~K.\,Е.} see~Sopin~E.\,S.&&\\
\Avtors{Serebryakov~V.\,A.\ and Ataeva~O.\,M.} Ontology of the digital semantic library LibMeta&1&\hphantom{1}2--10\\
\Avtors{Shanin~I.\,A., Stupnikov~S.\,A., and Zakharov~V.\,N.} Methods and tools for fault detection\linebreak
\\[-12pt]
\hspace*{23pt}on~elements of housing and utility infrastructure&3&67--73\\
\Avtors{Shestakov~O.\,V.} Mean-square thresholding risk with a random sample size&3&14--17\\
\Avtors{Shestakov~O.\,V.} Unbiased risk estimate of stabilized hard thresholding in the model with\linebreak
\\[-12pt]
\hspace*{23pt}a~long-range dependence&2&11--16\\
\Avtors{Shestakov~O.\,V.} see~Kudryavtsev~A.\,A.&&\\
\Avtors{Shestakov~O.\,V.} see~Kudryavtsev~A.\,A.&&\\
\Avtors{Shihiev~Sh.\,B.} see~Mirzabekov~Ya.\,M.&&\\
\Avtors{Shirokov~N.\,I.} see~Vyshinsky~L.\,L.&&\\
\Avtors{Shnurkov~P.\,V.\ and Egorov~A.\,Y.} Development and preliminary study of a~stochastic semi-Markov model of continuous supply of product management under the condition of\linebreak
\\[-12pt]
\hspace*{23pt}constant consumption&1&109--117\\
\Avtors{Shnurkov~P.\,V.\ and Egorov~A.\,Y.} Solution to the problem of optimal control of a~stochastic semi-Markov model of continuous supply of product management under the condition\linebreak
\\[-12pt]
\hspace*{23pt}of~constantly happening consumption&2&83--89\\
\Avtors{Shorgin~S.\,Ya.} see~Grusho~A.\,A.&&\\
\Avtors{Shorgin~S.\,Ya.} see~Samouylov~K.\,E.&&\\
\Avtors{Sinitsyn~I.\,N.} Method of interpolational analytical modeling of processes in stochastic systems&1&55--61\\
\Avtors{Smerdov~A.\,N., Bakhteev~O.\,Y., and~Strijov~V.\,V.} Optimal recurrent neural network model\linebreak
\\[-12pt]
\hspace*{23pt}in~paraphrase detection&4&63--69\\
\Avtors{Smirnov~D.\,V.} see~Grusho~A.\,A.&&\\
\Avtors{Sochenkov~I.\,V., Zubarev~D.\,V., and Tikhomirov~I.\,A.} Exploratory patent search&1&89--94\\
\Avtors{Sopin~E.\,S., Naumov~V.\,A., and Samouylov~K.\,Е.} On the insensitivity of the stationary distribution\linebreak
\\[-12pt]
\hspace*{23pt}of the limited resources queuing system with state-dependent arrival and service rates&3&42--47\\
\Avtors{Stefanovich~A.\,I.} see~Bosov~A.\,V.&&\\
\Avtors{Strijov~V.\,V.} see~Smerdov~A.\,N.&&\\
\Avtors{Stupnikov~S.\,A.} see~Shanin~I.\,A.&&\\
\Avtors{Suchkov~A.\,P.} see~Zatsarinny~A.\,A.&&\\
\Avtors{Surina~A.\,A.} see~Tyrsin~A.\,N.&&\\
\Avtors{Syuntyurenko~O.\,V.} Financing of basic research: Conceptual shape of a system of support\linebreak
\\[-12pt]
\hspace*{23pt}of~decision-making with use of methods of scientometrics and analysis of data&1&118--127\\
\Avtors{Tikhomirov~I.\,A.} see~Sochenkov~I.\,V.&&\\
\Avtors{Timonina~E.\,E.} see~Grusho~A.\,A.&&\\
\Avtors{Timonina~E.\,E.} see~Grusho~A.\,A.&&\\
\end{tabular}
}
\pagebreak

\def\leftfootline{\small{\textbf{\thepage}
\hfill INFORMATIKA I EE PRIMENENIYA~--- INFORMATICS AND APPLICATIONS\ \ \ 2018\
\ \ volume~12\ \ \ issue\ 4}
}%
 \def\rightfootline{\small{INFORMATIKA I EE PRIMENENIYA~---
INFORMATICS AND APPLICATIONS\ \ \ 2018\ \ \ volume~12\ \ \ issue\ 4
\hfill \textbf{\thepage}}}

\def\leftkol{2018 AUTHOR INDEX} % ENGLISH ABSTRACTS}

\def\rightkol{2018 AUTHOR INDEX} %ENGLISH ABSTRACTS}


\noindent
{\tabcolsep=3pt
\begin{tabular}{p{395.48108pt}cc}
&\textbf{Issue} & \textbf{Page}\\[6pt]
\Avtors{Timonina~E.\,E.} see~Grusho~A.\,A.&&\\
\Avtors{Timonina~E.\,E.} see~Grusho~A.\,A.&&\\
\Avtors{Titova~A.\,I.} see~Arutyunov~E.\,N.&&\\
\Avtors{Tyrsin~A.\,N.\ and Surina~A.\,A.} A~model of risk management in Gaussian stochastic systems&2&50--59\\
\Avtors{Ushakov~V.\,G.} see~Kondranin~E.\,S.&&\\
\Avtors{Ushakov~V.\,G.} see~Mistryukov~A.\,V.&&\\
\Avtors{Vasil'eva~S.\,N.\ and Kan~Yu.\,S.} A~visualization algorithm for the plane probability measure kernel&2&60--68\\
\Avtors{Vinogradov~D.\,V.} Influence of preliminary estimates on the speed of search of similarities by\linebreak
\\[-12pt]
\hspace*{23pt}the~coupling Markov chain&1&49--54\\
\Avtors{Vyshinsky~L.\,L., Flerov~Yu.\,A., and Shirokov~N.\,I.} Computer-aided system of aircraft weight\linebreak
\\[-12pt]
\hspace*{23pt}design&1&18--30\\
\Avtors{Yakovlev O.\,A.} see~Batenkov~A.\,A.&&\\
\Avtors{Zabezhailo~M.\,I.} see~Grusho~A.\,A.&&\\
\Avtors{Zabezhailo~M.\,I.} see~Grusho~A.\,A.&&\\
\Avtors{Zakharov~V.\,N.} see~Basok~B.\,M.&&\\
\Avtors{Zakharov~V.\,N.} see~Shanin~I.\,A.&&\\
\Avtors{Zaltsman~A.\,D.} see~Zhukov~D.\,O.&&\\
\Avtors{Zatsarinny~A.\,A.\ and Suchkov~A.\,P.} The situational management system as a multiservice\linebreak
\\[-12pt]
\hspace*{23pt}technology in the cloud&1&78--88\\
\Avtors{Zatsarinny~A.\,A.} see~Grusho~A.\,A.,&&\\
\Avtors{Zatsman~I.\,M.} Implied knowledge: Foundations and technologies of explication&3&74--82\\
\Avtors{Zatsman~I.\,M.} see~Buntman~N.\,V.&&\\
\Avtors{Zeifman~A.\,I.} see~Korolev~V.\,Yu.&&\\
\Avtors{Zhukov~D.\,O., Khvatova~T.\,Yu., Lesko~S.\,A., and Zaltsman~A.\,D.} The influence of the connections' density on clusterization and percolation threshold during information distribution in social\linebreak
\\[-12pt]
\hspace*{23pt}networks&2&90--97\\
\Avtors{Zubarev~D.\,V.} see~Sochenkov~I.\,V.&&\\
\end{tabular}
}

%\thispagestyle{myheadings}
\def\leftfootline{\small{\textbf{\thepage}
\hfill INFORMATIKA I EE PRIMENENIYA~--- INFORMATICS AND APPLICATIONS\ \ \ 2018\
\ \ volume~12\ \ \ issue\ 4}
}%
 \def\rightfootline{\small{INFORMATIKA I EE PRIMENENIYA~---
INFORMATICS AND APPLICATIONS\ \ \ 2018\ \ \ volume~12\ \ \ issue\ 4
\hfill \textbf{\thepage}}}

 \label{end\stat}

\newpage

   \vspace*{-48pt}

\begin{center}
\vspace*{6pt}
\mbox{%
\epsfxsize=53.502mm
\epsfbox{foto-1.eps}
}
\end{center}

\vspace*{6pt} %Академик


   \begin{center}
\fbox{\Large\textbf{Профессор Игорь Алексеевич Ушаков}}\\[12pt]
\textbf{\large 22.01.1935--27.02.2015}
   \end{center}


   %\vspace*{2.5mm}

   \vspace*{5mm}

   \thispagestyle{empty}

%\

%\vspace*{-12pt}


Редакционный совет и редакционная коллегия журнала <<Информатика и~её применения>> с~глубоким прискорбием извещают, что 27~февраля 2015~г.\ после тяжелой
и~продолжительной болезни скончался Игорь Алексеевич Ушаков~--- доктор технических наук, профессор, член редколлегии журнала <<Информатика и ее применения>>.

Игорь Алексеевич Ушаков окончил Московский авиационный институт, в~1963~г.\ защитил кандидатскую, а~в~1968~г.~--- докторскую диссертацию. С~1958 по 1989~гг.\ работал в~ряде научно-исследовательских организаций СССР, в~том числе руководил отделами в~НИИ АА и~ВЦ АН СССР; с 1969 по 1989 гг. преподавал в~МФТИ (был профессором, а~затем заведующим кафедрой) и~в~МЭИ. С~1989~г.~---- в~США: являлся профессором университета Дж.\ Вашингтона, университета Дж.\ Мэйсона и~Калифорнийского университета, сотрудником компаний MCI, Qualcomm и Hughes.

И.\,А.~Ушаков с момента основания журнала <<Надежность и~контроль качества>> был заместителем ответственного редактора, а~затем на протяжении многих лет членом редколлегии. В~2006~г.\ основал электронный международный журнал ``Reliability: Theory \& Application'', главным редактором которого оставался до конца жизни.

Учебниками и справочниками по теории надежности, написанными И.\,А.~Ушаковым, пользовались и~пользуются несколько поколений ученых и~специалистов в~разных странах мира.

Игорь Алексеевич всегда уделял огромное внимание работе с~молодежью; более~50 его учеников защитили докторские и~кандидатские диссертации.

И.\,А.~Ушаков вел активную научно-про\-све\-ти\-тель\-скую деятельность. В~частности, он был одним из организаторов и~руководителей Московского кабинета качества и~надежности при Политехническом музее (целью этого Кабинета было оказание консультаций работникам промышленных предприятий и~чтение курсов лекций для инженеров, занимающихся проблемой надежности). Находясь в~США, И.\,А.~Ушаков создал международный ин\-тер\-нет-фо\-рум им.\ Б.\,В.~Гнеденко, объединивший около~400~видных специалистов по приложениям теории вероятностей и~математической статистики, преимущественно в~об\-ласти теории надежности и~анализа риска, из десятков стран мира; коллективным членов этого Форума является и~наш журнал. Цели Форума~--- содействие контактам между специалистами из разных стран, организация обмена профессиональными 
новостями и~информацией (новые публикации, предстоящие события и~др.). Также необходимо отметить большое число на\-уч\-но-по\-пу\-ляр\-ных работ, опубликованных И.\,А.~Ушаковым.

И.\,А.~Ушаков обладал большим личным обаянием, имел широкий круг интересов. Все знавшие И.\,А.~Ушакова всегда будут помнить его как замечательного ученого и~прекрасного человека.

\bigskip

Редакционный совет и редакционная коллегия журнала <<Информатика и~её применения>> 
выражают глубокие соболезнования родным и близким покойного, всем, кто его знал и~работал с~ним.


%\def\stat{cont}
{%\hrule\par
%\vskip 7pt % 7pt
\raggedleft\Large \bf%\baselineskip=3.2ex
А\,В\,Т\,О\,Р\,С\,К\,И\,Й\ \ У\,К\,А\,З\,А\,Т\,Е\,Л\,Ь\ \ З\,А\ \ 2\,0\,1\,0 г. \vskip 17pt
    \hrule
    \par
\vskip 21pt plus 6pt minus 3pt }

\label{st\stat}

\def\tit{\ }

\def\aut{\ }
\def\auf{\ }

\def\leftkol{\ } % ENGLISH ABSTRACTS}

\def\rightkol{\ } %АВТОРСКИЙ УКАЗАТЕЛЬ ЗА 2010 г.} %ENGLISH ABSTRACTS}

\titele{\tit}{\aut}{\auf}{\leftkol}{\rightkol}

\vspace*{-12pt}

{\tabcolsep=3pt
\begin{tabular}{p{388pt}rr}
&\textbf{Выпуск} & \textbf{Стр.}\\[6pt]
\hangindent=23pt\noindent\textbf{Арутюнян~А.\,Р.} Моделирование влияния деформаций отпечатков пальцев на 
точность\linebreak
\vspace*{-12pt}\\
\hspace*{23pt}дактилоскопической идентификации$\dotfill$&1&51\\
\hangindent=23pt\noindent\textbf{Архипов~О.\,П., Зыкова~З.\,П.} Интеграция гетерогенной информации о цветных 
пикселях\linebreak
\vspace*{-12pt}\\
\hspace*{23pt}и их цветовосприятии$\dotfill$&4&15\\
\hangindent=23pt\noindent\textbf{Баранов~С.\,И., Френкель~С.\,Л., Захаров~В.\,Н.} Полуформальная верификация 
цифрового устройства с конвейером, основанная на использовании алгоритмических машин\linebreak
\vspace*{-12pt}\\
\hspace*{23pt}состояния$\dotfill$&4&49\\
\textbf{Бекетова~И.\,В.} см.~Каратеев~С.\,Л.&&\\
\textbf{Белоусов~В.\,В.} см.~Синицын~И.\,Н.&&\\
\hangindent=23pt\noindent\textbf{Бенинг~В.\,Е., Королев~Р.\,А.} О предельном поведении мощностей критериев в 
случае\linebreak
\vspace*{-12pt}\\
\hspace*{23pt}распределения Лапласа$\dotfill$&2&63\\
\hangindent=23pt\noindent\textbf{Бенинг~В.\,Е., Сипина~А.\,В.} Асимптотическое разложение для мощности 
критерия,\linebreak
\vspace*{-12pt}\\
\hspace*{23pt}основанного на выборочной медиане, в случае распределения Лапласа$\dotfill$&1&18\\
\textbf{Бондаренко~А.\,В.} см.~Каратеев~С.\,Л.&&\\
\hangindent=23pt\noindent\textbf{Бородина~А.\,В., Морозов~Е.\,В.} Об оценивании асимптотики вероятности 
большого\linebreak
\vspace*{-12pt}\\
\hspace*{23pt}уклонения стационарной регенеративной очереди с одним прибором$\dotfill$&3&29\\
\hangindent=23pt\noindent\textbf{Бунтман~Н.\,В., Минель~Ж.-Л., Ле~Пезан~Д., Зацман~И.\,М.} Типология и 
компьютерное\linebreak
\vspace*{-12pt}\\
\hspace*{23pt}моделирование трудностей перевода$\dotfill$&3&77\\
\textbf{Визильтер~Ю.\,В.} см.~Каратеев~С.\,Л.&&\\
\hangindent=23pt\noindent\textbf{Гавриленко~С.\,В.} Оценки скорости сходимости распределений случайных сумм с 
безгранично делимыми индексами к нормальному закону$\dotfill$&4&81\\
\hangindent=23pt\noindent\textbf{Григорьева~М.\,Е., Шевцова~И.\,Г.} Уточнение неравенства 
Каца--Берри--Эссеена$\dotfill$&2&75\\
\hangindent=23pt\noindent\textbf{Грушо~А.\,А., Грушо~Н.\,А., Тимонина~Е.\,Е.} Поиск конфликтов в политиках 
безопасности: модель случайных графов$\dotfill$&3&38\\
\textbf{Грушо~Н.\,А.} см.~Грушо~А.\,А.&&\\
\hangindent=23pt\noindent\textbf{Гудков~В.\,Ю.} Математические модели изображения отпечатка пальца на основе 
описания линий$\dotfill$&1&58\\
\textbf{Гуртов~А.\,В.} см.~Лукьяненко~А.\,С.&&\\
\textbf{Желтов~С.\,Ю.} см.~Каратеев~С.\,Л.&&\\
\hangindent=23pt\noindent\textbf{Захаров~А.\,А., Серебряков~В.\,А.} Система управления электронной библиотекой 
LibMeta$\dotfill$&4&2\\
\textbf{Захаров~В.\,Н.} см.~Баранов~С.\,И.&&\\
\textbf{Захарова~Т.\,В.} см.~Матвеева~С.\,С.&&\\
\hangindent=23pt\noindent\textbf{Зацаринный~А.\,А., Чупраков~К.\,Г.} Некоторые аспекты выбора технологии для 
постро-\linebreak
\vspace*{-12pt}\\
\hspace*{23pt}ения систем отображения информации ситуационного центра$\dotfill$&3&59\\
\textbf{Зацман~И.\,М.} см.~Бунтман~Н.\,В.&&\\
\hangindent=23pt\noindent\textbf{Зейфман~А.\,И., Коротышева~А.\,В., Сатин~Я.\,А., Шоргин~С.\,Я.} Об 
устойчивости нестаци-\linebreak
\vspace*{-12pt}\\
\hspace*{23pt}онарных систем обслуживания с катастрофами$\dotfill$&3&9\\
\textbf{Зыкова~З.\,П.} см.~Архипов~О.\,П.&&\\
\hangindent=23pt\noindent\textbf{Илюшин~Г.\,Я., Соколов~И.\,А.} Организация управляемого доступа пользователей 
к\linebreak
\vspace*{-12pt}\\
\hspace*{23pt}разнородным ведомственным информационным ресурсам$\dotfill$&1&24\\
\hangindent=23pt\noindent\textbf{Кавагучи~Ю., Ульянов~В.\,В., Фуджикоши~Я.} Приближения для статистик, 
описывающих\linebreak
\vspace*{-12pt}\\
\hspace*{23pt}геометрические свойства данных большой размерности, с оценками 
ошибок$\dotfill$&1&12\\
\hangindent=23pt\noindent\textbf{Каратеев~С.\,Л., Бекетова~И.\,В., Ососков~М.\,В., Князь~В.\,А., 
Визильтер~Ю.\,В., Бондаренко~А.\,В., Желтов~С.\,Ю.} Автоматизированный контроль 
качества цифровых\linebreak
\vspace*{-12pt}\\
\hspace*{23pt}изображений для персональных документов$\dotfill$&1&65\\
\end{tabular}
}

\pagebreak

\def\leftkol{АВТОРСКИЙ УКАЗАТЕЛЬ ЗА 2010 г.} % ENGLISH ABSTRACTS}

\def\rightkol{АВТОРСКИЙ УКАЗАТЕЛЬ ЗА 2010 г.} %ENGLISH ABSTRACTS}

{\tabcolsep=3pt
\begin{tabular}{p{388pt}rr}
&\textbf{Выпуск} & \textbf{Стр.}\\[3pt]
\hangindent=23pt\noindent\textbf{Козеренко~Е.\,Б.} Лингвистические фильтры в статистических моделях машинного\linebreak
\vspace*{-12pt}\\
\hspace*{23pt}перевода$\dotfill$&2&83\\
\hangindent=23pt\noindent\textbf{Козеренко~Е.\,Б., Кузнецов~И.\,П.} Когнитивно-лингвистические представления в 
систе-\linebreak
\vspace*{-12pt}\\
\hspace*{23pt}мах обработки текстов$\dotfill$&3&69\\
\textbf{Князь~В.\,А.} см.~Каратеев~С.\,Л.&&\\
\hangindent=23pt\noindent\textbf{Колесников~А.\,В., Солдатов~С.\,А.} Алгоритм координации для гибридной 
интеллектуальной системы решения сложной задачи оперативно-производственного\linebreak
\vspace*{-12pt}\\
\hspace*{23pt}планирования$\dotfill$&4&61\\
\hangindent=23pt\noindent\textbf{Коновалов~М.\,Г.} О планировании потоков в системах вычислительных 
ресурсов$\dotfill$&2&3\\
\textbf{Конушин~А.\,С.} см.~Конушин~В.\,С.&&\\
\hangindent=23pt\noindent\textbf{Конушин~В.\,С., Кривовязь~Г.\,Р., Конушин~А.\,С.} Алгоритм распознавания людей 
в видео-\linebreak
\vspace*{-12pt}\\
\hspace*{23pt}последовательности по одежде$\dotfill$&1&74\\
\textbf{Корепанов~Э.\, Р.} см.~Синицын~И.\,Н.&&\\
\textbf{Королев~В.\,Ю.} см.~Соколов~И.\,А.&&\\
\textbf{Королев~Р.\,А.} см.~Бенинг~В.\,Е.&&\\
\textbf{Коротышева~А.\,В.} см.~Зейфман~А.\,И.&&\\
\hangindent=23pt\noindent\textbf{Кривенко~М.\,П.} Непараметрическое оценивание элементов байесовского 
клас\-си-\linebreak
\vspace*{-12pt}\\
\hspace*{23pt}фикатора$\dotfill$&2&13\\
\textbf{Кривовязь~Г.\,Р.} см.~Конушин~В.\,С.&&\\
\textbf{Крылов~А.\,С.} см.~Павельева~Е.\,А.&&\\
\hangindent=23pt\noindent\textbf{Крылов~В.\,А.} Моделирование и классификация многоканальных дистанционных\linebreak
\vspace*{-12pt}\\
\hspace*{23pt}изображений с использованием копул$\dotfill$&4&34\\
\hangindent=23pt\noindent\textbf{Крючин~О.\,В.} Разработка параллельных эвристических алгоритмов подбора 
весовых\linebreak
\vspace*{-12pt}\\
\hspace*{23pt}коэффициентов искусственной нейтронной сети$\dotfill$&2&53\\
\hangindent=23pt\noindent\textbf{Кудрявцев~А.\,А., Шоргин~С.\,Я.} Байесовские модели массового обслуживания и 
надеж-\linebreak
\vspace*{-12pt}\\
\hspace*{23pt}ности: характеристики среднего числа заявок в системе $M\vert M \vert 1\vert 
\infty$$\dotfill$&3&16\\
\hangindent=23pt\noindent\textbf{Кузнецов~А.\,А.} Связь между временными и структурно-топологическими 
характери-\linebreak
\vspace*{-12pt}\\
\hspace*{23pt}стиками диаграмм ритма сердца здоровых людей$\dotfill$&4&39\\
\textbf{Кузнецов~И.\,П.} см.~Козеренко~Е.\,Б.&&\\
\textbf{Ле~Пезан~Д.} см.~Бунтман~Н.\,В.&&\\
\hangindent=23pt\noindent\textbf{Лукьяненко~А.\,С., Морозов~Е.\,В., Гуртов~А.\,В.} Анализ сетевого протокола с общей 
функ-\linebreak
\vspace*{-12pt}\\
\hspace*{23pt}цией расширения окна передачи сообщения при конфликтах$\dotfill$&2&46\\
\hangindent=23pt\noindent\textbf{Лямин~О.\,О.} О предельном поведении мощностей критериев в случае обобщенного\linebreak
\vspace*{-12pt}\\
\hspace*{23pt}распределения Лапласа$\dotfill$&3&47\\
\hangindent=23pt\noindent\textbf{Маркин~А.\,В., Шестаков~О.\,В.} Асимптотики оценки риска при пороговой 
обработке\linebreak
\vspace*{-12pt}\\
\hspace*{23pt}вейвлет-вейглет коэффициентов в задаче томографии$\dotfill$&2&36\\
\hangindent=23pt\noindent\textbf{Матвеева~С.\,С., Захарова~Т.\,В.} Сети массового обслуживания с наименьшей 
длиной\linebreak
\vspace*{-12pt}\\
\hspace*{23pt}очереди$\dotfill$&3&22\\
\hangindent=23pt\noindent\textbf{Матюшенко~С.\,И.} Стационарные характеристики двухканальной системы 
обслужива-\linebreak
\vspace*{-12pt}\\
\hspace*{23pt}ния с переупорядочиванием заявок и распределениями фазового типа$\dotfill$&4&68\\
\textbf{Минель~Ж.-Л.} см.~Бунтман~Н.\,В.&&\\
\textbf{Морозов~Е.\,В.} см.~Бородина~А.\,В.&&\\
\textbf{Морозов~Е.\,В.} см.~Лукьяненко~А.\,С.&&\\
\textbf{Ососков~М.\,В.} см.~Каратеев~С.\,Л.&&\\
\hangindent=23pt\noindent\textbf{Павельева~Е.\,А., Крылов~А.\,С.} Поиск и анализ ключевых точек радужной 
оболочки\linebreak
\vspace*{-12pt}\\
\hspace*{23pt}глаза методом преобразования Эрмита$\dotfill$&1&79\\
\textbf{Печинкин~А.\,В.} см.~Френкель~С.\,Л.,&&\\
\hangindent=23pt\noindent\textbf{Протасов~В.\,И.} Составление субъективного портрета с использованием 
эволюционно-\linebreak
\vspace*{-12pt}\\
\hspace*{23pt}го морфинга и квалиметрия метода$\dotfill$&1&83\\
\hangindent=23pt\noindent\textbf{Рудаков~К.\,В., Торшин~И.\,Ю.} Вопросы разрешимости задачи распознавания 
вторичной\linebreak
\vspace*{-12pt}\\
\hspace*{23pt}структуры белка$\dotfill$&2&25\\
\textbf{Сатин~Я.\,А.} см.~Зейфман~А.\,И.&&\\
\hangindent=23pt\noindent\textbf{Сейфуль-Мулюков~Р.\,Б.} Нефть как носитель информации о своем 
происхождении,\linebreak
\vspace*{-12pt}\\
\hspace*{23pt}структуре и эволюции$\dotfill$&1&41\\
\end{tabular}
}

{\tabcolsep=3pt
\begin{tabular}{p{388pt}rr}
&\textbf{Выпуск} & \textbf{Стр.}\\[6pt]
\textbf{Семендяев~Н.\,Н.} см.~Синицын~И.\,Н.&&\\
\textbf{Серебряков~В.\,А.} см.~Захаров~А.\,А.&&\\
\textbf{Синицын~В.\,И.} см.~Синицын~И.\,Н.&&\\
\hangindent=23pt\noindent\textbf{Синицын~И.\,Н., Синицын~В.\,И., Корепанов~Э.\, Р., Белоусов~В.\,В., 
Семендяев~Н.\,Н.} Оперативное построение информационных моделей движения полюса 
Земли\linebreak
\vspace*{-12pt}\\
\hspace*{23pt}методами линейных и линеаризованных фильтров$\dotfill$&1&2\\
\textbf{Сипина~А.\,В.} см.~Бенинг~В.\,Е.&&\\
\hangindent=23pt\noindent\textbf{Соколов~И.\,А.} О работах заслуженного деятеля науки Российской Федерации 
И.\,Н.~Синицына в области информационных технологий и автоматизации (к 70-летию\linebreak
\vspace*{-12pt}\\
\hspace*{23pt}со дня рождения)$\dotfill$&3&84\\
\textbf{Соколов~И.\,А.} см.~Илюшин~Г.\,Я.&&\\
\hangindent=23pt\noindent\textbf{Соколов~И.\,А., Королев~В.\,Ю.} Предисловие$\dotfill$&2&2\\
\textbf{Солдатов~С.\,А.} см.~Колесников~А.\,В.&&\\
\hangindent=23pt\noindent\textbf{Степанов~С.\,Ю.} Использование координатного метода фрагментации 
коммутаторной\linebreak
\vspace*{-12pt}\\
\hspace*{23pt}нейронной сети для сокращения трафика$\dotfill$&2&57\\
\textbf{Тимонина~Е.\,Е.} см.~Грушо~А.\,А.&&\\
\textbf{Торшин~И.\,Ю.} см.~Рудаков~К.\,В.&&\\
\textbf{Ульянов~В.\,В.} см.~Кавагучи~Ю.&&\\
\textbf{Фазекаш~И.} см.~Чупрунов~А.\,Н.&&\\
\textbf{Френкель~С.\,Л.} см.~Баранов~С.\,И.&&\\
\hangindent=23pt\noindent\textbf{Френкель~С.\,Л., Печинкин~А.\,В.} Оценка времени самовосстановления в 
цифровых\linebreak
\vspace*{-12pt}\\
\hspace*{23pt}системах после сбоев, вызываемых переходными помехами$\dotfill$&3&2\\
\textbf{Фуджикоши~Я.} см.~Кавагучи~Ю.&&\\
\hangindent=23pt\noindent\textbf{Цискаридзе~А.\,К.} Математическая модель и метод восстановления позы человека 
по\linebreak
\vspace*{-12pt}\\
\hspace*{23pt}стереопаре силуэтных изображений$\dotfill$&4&27\\
\hangindent=23pt\noindent\textbf{Чупраков~К.\,Г.} К вопросу о размещении коллективных средств отображения в 
ситуа-\linebreak
\vspace*{-12pt}\\
\hspace*{23pt}ционном зале с заданными параметрами$\dotfill$&4&89\\
\textbf{Чупраков~К.\,Г.} см.~Зацаринный~А.\,А.&&\\
\hangindent=23pt\noindent\textbf{Чупрунов~А.\,Н., Фазекаш~И.} Законы повторного логарифма для числа 
безошибочных\linebreak
\vspace*{-12pt}\\
\hspace*{23pt}блоков при помехоустойчивом кодировании$\dotfill$&3&42\\
\textbf{Шевцова~И.\,Г.} см.~Григорьева~М.\,Е.&&\\
\hangindent=23pt\noindent\textbf{Шестаков~О.\,В.} Аппроксимация распределения оценки риска пороговой 
обработки вейвлет-коэффициентов нормальным распределением при использовании 
выбо-\linebreak
\vspace*{-12pt}\\
\hspace*{23pt}рочной дисперсии$\dotfill$&4&73\\
\textbf{Шестаков~О.\,В.} см.~Маркин~А.\,В.&&\\
\textbf{Шоргин~С.\,Я.} см.~Зейфман~А.\,И.&&\\
\textbf{Шоргин~С.\,Я.} см.~Кудрявцев~А.\,А.&&\\
\end{tabular}
}

%\thispagestyle{myheadings}
\def\leftfootline{\small{\textbf{\thepage}
\hfill ИНФОРМАТИКА И ЕЁ ПРИМЕНЕНИЯ\ \ \ том~4\ \ \ выпуск~4\ \ \ 2010}
}%
 \def\rightfootline{\small{ИНФОРМАТИКА И ЕЁ ПРИМЕНЕНИЯ\ \ \ том~4\ \ \ выпуск~4\ \ \ 2010
 \hfill \textbf{\thepage}}}
 \label{end\stat}
%
%Том 10 Выпуск 1-4 Год 2016

\def\stat{cont-e}
{%\hrule\par
%\vskip 7pt % 7pt
\raggedleft\Large \bf%\baselineskip=3.2ex
2\,0\,1\,6\ \ A\,U\,T\,H\,O\,R\ \ I\,N\,D\,E\,X \vskip 17pt
 \hrule
 \par
\vskip 21pt plus 6pt minus 3pt }

\label{st\stat}

\def\tit{\ }

\def\aut{\ }
\def\auf{\ }

\def\leftkol{\ } %2016 AUTHOR INDEX} % ENGLISH ABSTRACTS}

\def\rightkol{\ } %2016 AUTHOR INDEX} %ENGLISH ABSTRACTS}

\titele{\tit}{\aut}{\auf}{\leftkol}{\rightkol}

\def\leftfootline{\small{\textbf{\thepage}
\hfill INFORMATIKA I EE PRIMENENIYA~--- INFORMATICS AND APPLICATIONS\ \ \ 2016\
\ \ volume~10\ \ \ issue\ 4}
}%
 \def\rightfootline{\small{INFORMATIKA I EE PRIMENENIYA~--- INFORMATICS AND APPLICATIONS\ \ \ 2016\ \ \ volume~10\ \ \ issue\ 4
\hfill \textbf{\thepage}}}

\vspace*{-12pt}
\vspace*{-18pt}

{\tabcolsep=2.8pt
\begin{tabular}{p{382pt}cc}
&\textbf{Issue} & \textbf{Page}\\[6pt]
\Avtors{Agalarov~M.\,Ya.} see~Agalarov~Ya.\,M.&&\\
\Avtors{Agalarov~Ya.\,M., Agalarov~M.\,Ya., and
Shorgin~V.\,S.} About the optimal threshold of queue\linebreak
\\[-12pt]
\hspace*{23pt}length in a~particular problem of profit maximization
in the $M/G/1$ queuing system&2&70--79\\
\Avtors{Alexeyevsky~D.\,A.} BioNLP ontology extraction from 
a~restricted language corpus with\linebreak
\\[-12pt]
\hspace*{23pt}context-free grammars&1&119--128\\
\Avtors{Andreev~S.\,D.} see~Gaidamaka~Yu.\,V.&&\\
\Avtors{Andreev~S.\,D.} see~Ometov~A.\,Ya.&&\\
\Avtors{Arkhipov~O.\,P., Arkhipov~P.\,O., and Sidorkin~I.\,I.} The
option to create a~local coordinate\linebreak
\\[-12pt]
\hspace*{23pt}system for synchronization of selected images&3&91--97\\
\Avtors{Arkhipov~P.\,O.} see~Arkhipov~O.\,P.&&\\
\Avtors{Belousov~V.\,V.} see~Shnurkov~P.\,V.&&\\
\Avtors{Belousov~V.\,V.} see~Shnurkov~P.\,V.&&\\
\Avtors{Bening~V.\,E.} Calculation of~the~asymptotic deficiency
of~some statistical procedures based\linebreak
\\[-12pt]
\hspace*{23pt}on~samples with~random sizes&4&34--45\\
\Avtors{Borisov~A.\,V., Bosov~A.\,V., and Miller~G.\,B.} Modeling and
monitoring of VoIP connection&2&\hphantom{1}2--13\\
\Avtors{Bosov~A.\,V.} see~Borisov~A.\,V.&&\\
\Avtors{Briukhov~D.\,O.} see~Stupnikov~S.\,A.&&\\
\Avtors{Callaos~N.\,K.\ and Seyful-Mulyukov~R.\,B.} Complexity and
its information content&1&129--139\\
\Avtors{Chertok~A.\,V., Kadaner~A.\,I., Khazeeva~G.\,T., and
Sokolov~I.\,A.} Regime switching detection\linebreak
\\[-12pt]
\hspace*{23pt}for~the~Levy driven
Ornstein--Uhlenbeck process using CUSUM methods&4&46--56\\
\Avtors{Chichagov~V.\,V.} Asymptotic expansions of mean absolute
error of uniformly minimum variance unbiased and maximum likelihood
estimators on the one-parameter exponential\linebreak
\\[-12pt]
\hspace*{23pt}family model of lattice distributions&3&66--76\\
\Avtors{Danishevsky~V.\,I.} see~Kolesnikov A.\,V.&&\\
\Avtors{Fazliev~A.\,Z.} see~Kalinichenko~L.\,A.&&\\
\Avtors{Fedoseev~A.\,A.} What is behind the concept of ``knowledge in
small packages''&3&105--110\\
\Avtors{Gaidamaka~Yu.\,V., Andreev~S.\,D., Sopin~E.\,S.,
Samouylov~K.\,E., and Shorgin~S.\,Ya.} Interference analysis
of~the~device-to-device communications model with~regard to~a~signal\linebreak
\\[-12pt]
\hspace*{23pt}propagation environment&4&\hphantom{1}2--10\\
\Avtors{Gasilov~A.\,V.} see~Yakovlev~O.\,A.&&\\
\Avtors{Goncharov~A.\,V.\ and Strijov~V.\,V.} Metric time series
classification using weighted dynamic\linebreak
\\[-12pt]
\hspace*{23pt}warping relative to centroids of classes&2&36--47\\
\Avtors{Gordov~E.\,P.} see~Kalinichenko~L.\,A.&&\\
\Avtors{Gorshenin~A.\,K.} Concept of online service for stochastic
modeling of real processes&1&72--81\\
\Avtors{Gorshenin~A.\,K.} see~Shnurkov~P.\,V.&&\\
\Avtors{Gorshenin~A.\,K.} see~Shnurkov~P.\,V.&&\\
\Avtors{Grusho~A.\,A., Grusho~N.\,A., Zabezhailo~M.\,I., and
Timonina~E.\,E.} Integration of statistical and\linebreak
\\[-12pt]
\hspace*{23pt}deterministic methods for
analysis of information security&3&2--8\\
\Avtors{Grusho~A.\,A., Zabezhailo~M.\,I., and Zatsarinny~A.\,A.} On
the advanced procedure to reduce\linebreak
\\[-12pt]
\hspace*{23pt}calculation of Galois closures&4&\hphantom{1}96--104\\
\Avtors{Grusho~N.\,A.} see~Grusho~A.\,A.&&\\
\Avtors{Havanskov~V.\,A.} see~Minin~V.\,A.&&\\
\Avtors{Inkova~O.\,Yu.} see~Zatsman~I.\,M.&&\\
\Avtors{Isachenko~R.\,V.\ and Strijov~V.\,V.} Metric learning in
multiclass time series classification\linebreak
\\[-12pt]
\hspace*{23pt}problem&2&48--57\\
\end{tabular}
}
\pagebreak

\def\leftfootline{\small{\textbf{\thepage}
\hfill INFORMATIKA I EE PRIMENENIYA~--- INFORMATICS AND APPLICATIONS\ \ \ 2016\
\ \ volume~10\ \ \ issue\ 4}
}%
 \def\rightfootline{\small{INFORMATIKA I EE PRIMENENIYA~---
INFORMATICS AND APPLICATIONS\ \ \ 2016\ \ \ volume~10\ \ \ issue\ 4
\hfill \textbf{\thepage}}}

\def\leftkol{2016 AUTHOR INDEX} % ENGLISH ABSTRACTS}

\def\rightkol{2016 AUTHOR INDEX} %ENGLISH ABSTRACTS}


{\tabcolsep=2.83pt
\begin{tabular}{p{382pt}cc}
&\textbf{Issue} & \textbf{Page}\\[6pt]
\Avtors{Kadaner~A.\,I.} see~Chertok~A.\,V.&&\\[.255pt]
\Avtors{Kalinichenko~L.\,A., Volnova~A.\,A., Gordov~E.\,P.,
Kiselyova~N.\,N., Kovaleva~D.\,A., Malkov~O.\,Yu., Okladnikov~I.\,G.,
Podkolodnyy~N.\,L., Pozanenko~A.\,S., Ponomareva~N.\,V.,
Stupnikov~S.\,A.,} \textbf{and Fazliev~A.\,Z.} Data access challenges for data
intensive\linebreak
\\[-12pt]
\hspace*{23pt}research in Russia&1& 2--22\\[.255pt]
\Avtors{Karasikov~M.\,E.\ and Strijov~V.\,V.} Feature-based
time-series classification&4&121--131\\[.255pt]
\Avtors{Khazeeva~G.\,T.} see~Chertok~A.\,V.&&\\[.255pt]
\Avtors{Khokhlov~Yu.\,S.} Multivariate fractional Levy motion and its
applications&2&\hphantom{1}98--106\\[.255pt]
\Avtors{Kirikov~I.\,A., Kolesnikov~A.\,V., Listopad~S.\,V., and
Rumovskaya~S.\,B.} Fine-grained hybrid\linebreak
\\[-12pt]
\hspace*{23pt}intelligent systems. Part 2:
Bidirectional hybridization&1&\hphantom{1}96--105\\[.255pt]
\Avtors{Kirikov~I.\,A., Kolesnikov~A.\,V., Listopad~S.\,V., and
Rumovskaya~S.\,B.} ``Virtual council''~---\linebreak
\\[-12pt]
\hspace*{23pt}source environment
supporting complex diagnostic decision making&3&81--90\\[.255pt]
\Avtors{Kiselyova~N.\,N.} see~Kalinichenko~L.\,A.&&\\[.255pt]
\Avtors{Kolesnikov A.\,V., Listopad~S.\,V., Rumovskaya~S.\,B., and
Danishevsky~V.\,I.} Informal axiomatic\linebreak
\\[-12pt]
\hspace*{23pt}theory of~the~role visual models&4&114--120\\[.255pt]
\Avtors{Kolesnikov~A.\,V.} see~Kirikov~I.\,A.&&\\[.255pt]
\Avtors{Kolesnikov~A.\,V.} see~Kirikov~I.\,A.&&\\[.255pt]
\Avtors{Kolin~K.\,K.} Humanitarian aspects of information
security&3&111--121\\[.255pt]
\Avtors{Konovalov~M.\,G.\ and Razumchik~R.\,V.} Dispatching
to~two parallel nonobservable queues using\linebreak
\\[-12pt]
\hspace*{23pt}only static
information&4&57--67\\[.255pt]
\Avtors{Korchagin~A.\,Yu.} see~Korolev~V.\,Yu.&&\\[.255pt]
\Avtors{Korchagin~A.\,Yu.} see~Korolev~V.\,Yu.&&\\[.255pt]
\Avtors{Korepanov~E.\,R.} see~Sinitsyn~I.\,N.&&\\[.255pt]
\Avtors{Korepanov~E.\,R.} see~Sinitsyn~I.\,N.&&\\[.255pt]
\Avtors{Korolev~V.\,Yu., Korchagin~A.\,Yu., and Zeifman~A.\,I.} The
Poisson theorem for Bernoulli trials\linebreak
\\[-12pt]
\hspace*{23pt}with~a~random probability
of~success and~a~discrete analog of~the~Weibull distribution&4&11--20\\[.255pt]
\Avtors{Korolev~V.\,Yu., Zeifman~A.\,I., and Korchagin~A.\,Yu.}
Asymmetric Linnik distributions as~limit\linebreak
\\[-12pt]
\hspace*{23pt}laws for~random sums
of~independent random variables with~finite variances&4&21--33\\[.255pt]
\Avtors{Koucheryavy~E.\,A.} see~Ometov~A.\,Ya.&&\\[.255pt]
\Avtors{Kovaleva~D.\,A.} see~Kalinichenko~L.\,A.&&\\[.255pt]
\Avtors{Kovalyov~S.\,P.} Metaprogramming to increase
manufacturability of large-scale software-\linebreak
\\[-12pt]
\hspace*{23pt}intensive systems&1&56--66\\[.255pt]
\Avtors{Krivenko~M.\,P.} Significance tests of feature selection for
classification&3&32--40\\[.255pt]
\Avtors{Kruzhkov~M.\,G.} see~Zalizniak~Anna~A.&&\\[.255pt]
\Avtors{Kruzhkov~M.\,G.} see~Zatsman~I.\,M.&&\\[.255pt]
\Avtors{Kudryavtsev~A.\,A.} Bayesian queueing and reliability models:
\textit{A~priori} distributions with\linebreak
\\[-12pt]
\hspace*{23pt}compact support&1&67--71\\[.255pt]
\Avtors{Kudryavtsev~A.\,A.} Characteristics dependent on the balance
coefficient in Bayesian models\linebreak
\\[-12pt]
\hspace*{23pt}with compact support of \textit{a priori}
distributions&3&77--80\\[.255pt]
\Avtors{Kudryavtsev~A.\,A.\ and Palionnaia~S.\,I.} Bayesian recurrent
model of reliability growth:\linebreak
\\[-12pt]
\hspace*{23pt}Parabolic distribution of parameters&2&80--83\\[.255pt]
\Avtors{Kudryavtsev~A.\,A.\ and Titova~A.\,I.} Bayesian queuing
and~reliability models: Degenerate-\linebreak
\\[-12pt]
\hspace*{23pt}Weibull case&4&68--71\\[.255pt]
\Avtors{Leontyev~N.\,D.\ and Ushakov~V.\,G.} Analysis of a queueing
system with autoregressive arrivals\linebreak
\\[-12pt]
\hspace*{23pt}and nonpreemptive priority&3&15--22\\[.255pt]
\Avtors{Listopad~S.\,V.} see~Kirikov~I.\,A.&&\\[.255pt]
\Avtors{Listopad~S.\,V.} see~Kirikov~I.\,A.&&\\[.255pt]
\Avtors{Listopad~S.\,V.} see~Kolesnikov A.\,V.&&\\[.255pt]
\Avtors{Malkov~O.\,Yu.} see~Kalinichenko~L.\,A.&&\\[.255pt]
\Avtors{Markov~A.\,S., Monakhov~M.\,M., and
Ulyanov~V.\,V.} Generalized Cornish--Fisher expansions\linebreak
\\[-12pt]
\hspace*{23pt}for distributions of statistics based on samples
of random size&2&84--91\\[.255pt]
\Avtors{Melnikov~A.\,K.\ and Ronzhin~A.\,F.} Generalized statistical
method of~text analysis based\linebreak
\\[-12pt]
\hspace*{23pt}on~calculation of~probability distributions
of~statistical values&4&89--95\\
\end{tabular}
}
\pagebreak

\def\leftfootline{\small{\textbf{\thepage}
\hfill INFORMATIKA I EE PRIMENENIYA~--- INFORMATICS AND APPLICATIONS\ \ \ 2016\
\ \ volume~10\ \ \ issue\ 4}
}%
 \def\rightfootline{\small{INFORMATIKA I EE PRIMENENIYA~---
INFORMATICS AND APPLICATIONS\ \ \ 2016\ \ \ volume~10\ \ \ issue\ 4
\hfill \textbf{\thepage}}}

\def\leftkol{2016 AUTHOR INDEX} % ENGLISH ABSTRACTS}

\def\rightkol{2016 AUTHOR INDEX} %ENGLISH ABSTRACTS}


{\tabcolsep=3pt
\begin{tabular}{p{381pt}cc}
&\textbf{Issue} & \textbf{Page}\\[6pt]
\Avtors{Meykhanadzhyan~L.\,A.} Stationary characteristics of the finite
capacity queueing system with\linebreak
\\[-12pt]
\hspace*{23pt}inverse service order and generalized
probabilistic priority&2&123--131\\[.23pt]
\Avtors{Miller~G.\,B.} see~Borisov~A.\,V.&&\\[.23pt]
\Avtors{Minin~V.\,A., Zatsman~I.\,M., Havanskov~V.\,A., and
Shubnikov~S.\,K.} Intensity of citation of scientific publications in
inventions on information and computer technologies patented\linebreak
\\[-12pt]
\hspace*{23pt}in Russia by domestic and foreign applicants&2&107--122\\[.23pt]
\Avtors{Monakhov~M.\,M.} see~Markov~A.\,S.&&\\[.23pt]
\Avtors{Naumov~V.\,A.\ and Samouylov~K.\,E.} On relationship
between queuing systems with resources\linebreak
\\[-12pt]
\hspace*{23pt}and Erlang networks&3&\hphantom{1}9--14\\[.23pt]
\Avtors{Okladnikov~I.\,G.} see~Kalinichenko~L.\,A.&&\\[.23pt]
\Avtors{Ometov~A.\,Ya., Andreev~S.\,D., Turlikov~A.\,M., and
Koucheryavy~E.\,A.} Performance analysis of\linebreak
\\[-12pt]
\hspace*{23pt}a wireless data
aggregation system with contention for contemporary sensor
networks&3&23--31\\[.23pt]
\Avtors{Palionnaia~S.\,I.} see~Kudryavtsev~A.\,A.&&\\[.23pt]
\Avtors{Podkolodnyy~N.\,L.} see~Kalinichenko~L.\,A.&&\\[.23pt]
\Avtors{Ponomareva~N.\,V.} see~Kalinichenko~L.\,A.&&\\[.23pt]
\Avtors{Popkova~N.\,A.} see~Zatsman~I.\,M.&&\\[.23pt]
\Avtors{Pozanenko~A.\,S.} see~Kalinichenko~L.\,A.&&\\[.23pt]
\Avtors{Razumchik~R.\,V.} see~Konovalov~M.\,G.&&\\[.23pt]
\Avtors{Ronzhin~A.\,F.} see~Melnikov~A.\,K.&&\\[.23pt]
\Avtors{Rumovskaya~S.\,B.} see~Kirikov~I.\,A.&&\\[.23pt]
\Avtors{Rumovskaya~S.\,B.} see~Kirikov~I.\,A.&&\\[.23pt]
\Avtors{Rumovskaya~S.\,B.} see~Kolesnikov A.\,V.&&\\[.23pt]
\Avtors{Samouylov~K.\,E.} see~Gaidamaka~Yu.\,V.&&\\[.23pt]
\Avtors{Samouylov~K.\,E.} see~Naumov~V.\,A.&&\\[.23pt]
\Avtors{Serebryanskii~S.\,M.} see~Tyrsin~A.\,N.&&\\[.23pt]
\Avtors{Seyful-Mulyukov~R.\,B.} see~Callaos~N.\,K.&&\\[.23pt]
\Avtors{Shestakov~O.\,V.} Statistical properties of the denoising method
based on the stabilized hard\linebreak
\\[-12pt]
\hspace*{23pt}thresholding&2&65--69\\[.23pt]
\Avtors{Shestakov~O.\,V.} The strong law of large numbers for the risk
estimate in the problem of\linebreak
\\[-12pt]
\hspace*{23pt}tomographic image reconstruction from
projections with a correlated noise&3&41--45\\[.23pt]
\Avtors{Shestakov~O.\,V.} see~Zakharova~T.\,V.&&\\[.23pt]
\Avtors{Shnurkov~P.\,V., Gorshenin~A.\,K., and Belousov~V.\,V.}
Analytical solution of~the~optimal control\linebreak
\\[-12pt]
\hspace*{23pt}task of~a~semi-Markov
process with~finite set of~states&4&72--88\\[.23pt]
\Avtors{Shnurkov~P.\,V., Zasypko~V.\,V., Belousov~V.\,V., and
Gorshenin~A.\,K.} Development of the algorithm of numerical solution
of the optimal investment control problem\linebreak
\\[-12pt]
\hspace*{23pt}in the closed dynamical model of three-sector economy&1&82--95\\[.23pt]
\Avtors{Shorgin~S.\,Ya.} see~Gaidamaka~Yu.\,V.&&\\[.23pt]
\Avtors{Shorgin~V.\,S.} see~Agalarov~Ya.\,M.&&\\[.23pt]
\Avtors{Shubnikov~S.\,K.} see~Minin~V.\,A.&&\\[.23pt]
\Avtors{Sidorkin~I.\,I.} see~Arkhipov~O.\,P.&&\\[.23pt]
\Avtors{Sinitsyn~I.\,N.} Analytical modeling of processes in stochastic
systems with complex fractional\linebreak
\\[-12pt]
\hspace*{23pt}order Bessel nonlinearities&3&55--65\\[.23pt]
\Avtors{Sinitsyn~I.\,N.} Orthogonal supoptimal filters for nonlinear
stochastic systems on manifolds&1&34--44\\[.23pt]
\Avtors{Sinitsyn~I.\,N.\ and Korepanov~E.\,R.} Normal Pugachev
conditionally-optimal filters and extra-\linebreak
\\[-12pt]
\hspace*{23pt}polators for state linear stochastic systems&2&14--23\\[.23pt]
\Avtors{Sinitsyn~I.\,N.\ and Sinitsyn~V.\,I.} Analytical modeling of
distributions in stochastic systems on\linebreak
\\[-12pt]
\hspace*{23pt}manifolds based on ellipsoidal approximation&1&45--55\\[.23pt]
\Avtors{Sinitsyn~I.\,N., Sinitsyn~V.\,I., and
Korepanov~E.\,R.} Ellipsoidal suboptimal filters for nonlinear\linebreak
\\[-12pt]
\hspace*{23pt}stochastic systems on manifolds&2&24--35\\[.23pt]
\Avtors{Sinitsyn~V.\,I.} see~Sinitsyn~I.\,N.&&\\[.23pt]
\Avtors{Sinitsyn~V.\,I.} see~Sinitsyn~I.\,N.&&\\[.23pt]
\Avtors{Skvortsov~N.\,A.} see~Stupnikov~S.\,A.&&\\[.23pt]
\Avtors{Sokolov~I.\,A.} see~Chertok~A.\,V.&&\\
\end{tabular}
}
\pagebreak

\def\leftfootline{\small{\textbf{\thepage}
\hfill INFORMATIKA I EE PRIMENENIYA~--- INFORMATICS AND APPLICATIONS\ \ \ 2016\
\ \ volume~10\ \ \ issue\ 4}
}%
 \def\rightfootline{\small{INFORMATIKA I EE PRIMENENIYA~---
INFORMATICS AND APPLICATIONS\ \ \ 2016\ \ \ volume~10\ \ \ issue\ 4
\hfill \textbf{\thepage}}}

\def\leftkol{2016 AUTHOR INDEX} % ENGLISH ABSTRACTS}

\def\rightkol{2016 AUTHOR INDEX} %ENGLISH ABSTRACTS}


{\tabcolsep=3pt
\begin{tabular}{p{382pt}cc}
&\textbf{Issue} & \textbf{Page}\\[6pt]
\Avtors{Sopin~E.\,S.} see~Gaidamaka~Yu.\,V.&&\\
\Avtors{Strijov~V.\,V.} see~Goncharov~A.\,V.&&\\
\Avtors{Strijov~V.\,V.} see~Isachenko~R.\,V.&&\\
\Avtors{Strijov~V.\,V.} see~Karasikov~M.\,E.&&\\
\Avtors{Stupnikov~S.\,A., Briukhov~D.\,O., and Skvortsov~N.\,A.}
Co-lending systemic risk analysis over\linebreak
\\[-12pt]
\hspace*{23pt}heterogeneous data collections&1&23--33\\
\Avtors{Stupnikov~S.\,A.} see~Kalinichenko~L.\,A.&&\\
\Avtors{Suchkov~A.\,P.} see~Zatsarinny~A.\,A.&&\\
\Avtors{Timonina~E.\,E.} see~Grusho~A.\,A.&&\\
\Avtors{Titova~A.\,I.} see~Kudryavtsev~A.\,A.&&\\
\Avtors{Turlikov~A.\,M.} see~Ometov~A.\,Ya.&&\\
\Avtors{Tyrsin~A.\,N.\ and Serebryanskii~S.\,M.} Recognition of
dependences on the basis of inverse\linebreak
\\[-12pt]
\hspace*{23pt}mapping&2&58--64\\
\Avtors{Ulyanov~V.\,V.} see~Markov~A.\,S.&&\\
\Avtors{Ushakov~V.\,G.} Queueing system with working vacations and
hyperexponential input stream&2&92--97\\
\Avtors{Ushakov~V.\,G.} see~Leontyev~N.\,D.&&\\
\Avtors{Volnova~A.\,A.} see~Kalinichenko~L.\,A.&&\\
\Avtors{Yakovlev~O.\,A.\ and Gasilov~A.\,V.} Speeded-up stereo
matching using geodesic support weights&3&\hphantom{1}98--104\\
\Avtors{Zabezhailo~M.\,I.} see~Grusho~A.\,A.&&\\
\Avtors{Zabezhailo~M.\,I.} see~Grusho~A.\,A.&&\\
\Avtors{Zakharova~T.\,V.\ and Shestakov~O.\,V.} Precision analysis of
wavelet processing of aerodynamic\linebreak
\\[-12pt]
\hspace*{23pt}flow patterns&3&46--54\\
\Avtors{Zalizniak~Anna~A.\ and Kruzhkov~M.\,G.} Database
of~Russian impersonal verbal constructions&4&132--141\\
\Avtors{Zasypko~V.\,V.} see~Shnurkov~P.\,V.&&\\
\Avtors{Zatsarinny~A.\,A.\ and Suchkov~A.\,P.} Systems engineering
approaches to~the~establishment of\linebreak
\\[-12pt]
\hspace*{23pt}a~system for~decision support based
on~situational analysis&4&105--113\\
\Avtors{Zatsarinny~A.\,A.} see~Grusho~A.\,A.&&\\
\Avtors{Zatsman~I.\,M., Inkova~O.\,Yu., Kruzhkov~M.\,G., and
Popkova~N.\,A.} Representation of cross-\linebreak
\\[-12pt]
\hspace*{23pt}lingual knowledge about
connectors in supracorpora databases&1&106--118\\
\Avtors{Zatsman~I.\,M.} see~Minin~V.\,A.&&\\
\Avtors{Zeifman~A.\,I.} see~Korolev~V.\,Yu.&&\\
\Avtors{Zeifman~A.\,I.} see~Korolev~V.\,Yu.&&\\
\end{tabular}
}

%\thispagestyle{myheadings}
\def\leftfootline{\small{\textbf{\thepage}
\hfill INFORMATIKA I EE PRIMENENIYA~--- INFORMATICS AND APPLICATIONS\ \ \ 2016\
\ \ volume~10\ \ \ issue\ 4}
}%
 \def\rightfootline{\small{INFORMATIKA I EE PRIMENENIYA~---
INFORMATICS AND APPLICATIONS\ \ \ 2016\ \ \ volume~10\ \ \ issue\ 4
\hfill \textbf{\thepage}}}

 \label{end\stat}

\newpage

%\def\stat{rekl}
%\label{preobr}

%\def\tit{АКАДЕМИК ПУГАЧЁВ  ВЛАДИМИР СЕМЁНОВИЧ\\
%25.03.1911--25.03.1998}


%   \vspace*{-48pt}
%   \begin{center}\LARGE
%Академик Пугачёв  Владимир Семёнович\\ (25.03.1911--25.03.1998)
%   \end{center}
   
   %\vspace*{2.5mm}
   
   \begin{center}

{\prgsh\LARGE
ОБЪЯВЛЕНИЯ О КОНФЕРЕНЦИЯХ}

\end{center}
%\hrule

\vspace*{6pt}

   
   \vspace*{10mm}
   
   \thispagestyle{empty}

\noindent
\begin{tabular}{cc}
%\begin{center}
\multicolumn{1}{c}{\raisebox{-40pt}[0pt][0pt]{\mbox{%
\epsfxsize=33mm
\epsfbox{vspu.eps}
}}}
%\end{center}
&
\tabcolsep=0pt\begin{tabular}{c}
{\prg{\Large\textbf{XII Всероссийское совещание}}}\\[6pt]
{\prg{\Large\textbf{по проблемам управления}}}\\[12pt]
{\prg{\large 16--19 июня 2014~г.}}\\[6pt] 
{\prg{\large Институт проблем управления имени В.\,А.~Трапезникова РАН}}\\[6pt]
{\prg{\large Москва, Россия}}
\end{tabular}
\end{tabular}

\vspace*{60pt}

     
 { %\large    
 XII Всероссийское совещание по проблемам управления (ВСПУ XII), посвященное 75-летию 
Института проблем управления (ИПУ) имени В.\,А.~Трапезникова РАН, проводится 16--19~июня 
2014~г.\ 
в ИПУ РАН (г.~Москва, Россия). ВСПУ XII организуется ИПУ РАН при поддержке РФФИ, Отделения 
энергетики, машиностроения, механики и процессов управления Российской академии наук, 
Российского 
национального комитета по автоматическому управлению, Академии навигации и управ\-ле\-ния 
движением, 
Научного совета РАН по комплексным проблемам управления и автоматизации, Совета по 
мехатронике и робототехнике РАН. Официальный язык Совещания~--- русский.

\vspace*{24pt}
     
     \textbf{Направления работы}
     \begin{enumerate}[1.]
\item Теория систем управления
\item Управление подвижными объектами и навигация
\item Интеллектуальные системы управления
\item Управление в промышленности, транспортом и логистикой
\item Управление системами междисциплинарной природы
\item Средства измерения, вычислений и контроля в управлении
\item Системный анализ и принятие решений в задачах управления
\item Информационные технологии в управлении
\item Проблемы образования в области управления: современное содержание и технологии обучения
\end{enumerate}

\vspace*{24pt}

     Подробная информация о Совещании находится на сайте {\sf http://vspu2014.ipu.ru}. Срок 
окончательной подачи докладов через систему подачи докладов на сайте~--- \textbf{30~ноября} 
2013~г.
}

%\include{rekl-1}

%\end{document}

%   \vspace*{-48pt}

\begin{center}
\vspace*{6pt}
\mbox{%
\epsfxsize=53.502mm
\epsfbox{foto-1.eps}
}
\end{center}

\vspace*{6pt} %Академик


   \begin{center}
\fbox{\Large\textbf{Профессор Игорь Алексеевич Ушаков}}\\[12pt]
\textbf{\large 22.01.1935--27.02.2015}
   \end{center}


   %\vspace*{2.5mm}

   \vspace*{5mm}

   \thispagestyle{empty}

%\

%\vspace*{-12pt}


Редакционный совет и редакционная коллегия журнала <<Информатика и~её применения>> с~глубоким прискорбием извещают, что 27~февраля 2015~г.\ после тяжелой
и~продолжительной болезни скончался Игорь Алексеевич Ушаков~--- доктор технических наук, профессор, член редколлегии журнала <<Информатика и ее применения>>.

Игорь Алексеевич Ушаков окончил Московский авиационный институт, в~1963~г.\ защитил кандидатскую, а~в~1968~г.~--- докторскую диссертацию. С~1958 по 1989~гг.\ работал в~ряде научно-исследовательских организаций СССР, в~том числе руководил отделами в~НИИ АА и~ВЦ АН СССР; с 1969 по 1989 гг. преподавал в~МФТИ (был профессором, а~затем заведующим кафедрой) и~в~МЭИ. С~1989~г.~---- в~США: являлся профессором университета Дж.\ Вашингтона, университета Дж.\ Мэйсона и~Калифорнийского университета, сотрудником компаний MCI, Qualcomm и Hughes.

И.\,А.~Ушаков с момента основания журнала <<Надежность и~контроль качества>> был заместителем ответственного редактора, а~затем на протяжении многих лет членом редколлегии. В~2006~г.\ основал электронный международный журнал ``Reliability: Theory \& Application'', главным редактором которого оставался до конца жизни.

Учебниками и справочниками по теории надежности, написанными И.\,А.~Ушаковым, пользовались и~пользуются несколько поколений ученых и~специалистов в~разных странах мира.

Игорь Алексеевич всегда уделял огромное внимание работе с~молодежью; более~50 его учеников защитили докторские и~кандидатские диссертации.

И.\,А.~Ушаков вел активную научно-про\-све\-ти\-тель\-скую деятельность. В~частности, он был одним из организаторов и~руководителей Московского кабинета качества и~надежности при Политехническом музее (целью этого Кабинета было оказание консультаций работникам промышленных предприятий и~чтение курсов лекций для инженеров, занимающихся проблемой надежности). Находясь в~США, И.\,А.~Ушаков создал международный ин\-тер\-нет-фо\-рум им.\ Б.\,В.~Гнеденко, объединивший около~400~видных специалистов по приложениям теории вероятностей и~математической статистики, преимущественно в~об\-ласти теории надежности и~анализа риска, из десятков стран мира; коллективным членов этого Форума является и~наш журнал. Цели Форума~--- содействие контактам между специалистами из разных стран, организация обмена профессиональными 
новостями и~информацией (новые публикации, предстоящие события и~др.). Также необходимо отметить большое число на\-уч\-но-по\-пу\-ляр\-ных работ, опубликованных И.\,А.~Ушаковым.

И.\,А.~Ушаков обладал большим личным обаянием, имел широкий круг интересов. Все знавшие И.\,А.~Ушакова всегда будут помнить его как замечательного ученого и~прекрасного человека.

\bigskip

Редакционный совет и редакционная коллегия журнала <<Информатика и~её применения>> 
выражают глубокие соболезнования родным и близким покойного, всем, кто его знал и~работал с~ним.



%\end{document}

%\include{IPPM-25}

\def\stat{cont-rus}
{%\hrule\par
%\vskip 7pt % 7pt
\vspace*{-24pt}
\raggedleft\Large \bf%\baselineskip=3.2ex
Правила подготовки рукописей  для публикации в журнале
<<Информатика~и~её~применения>> \vskip 8pt
    \hrule
    \par
\vskip 14pt plus 6pt minus 3pt }

\label{st\stat}

\def\tit{\ }

\def\aut{\ }
\def\auf{\ }

\def\leftkol{\ }
% Правила подготовки рукописей  для публикации в журнале
%<<Информатика и её применения>>

\def\rightkol{\ }
%Правила подготовки рукописей  для публикации в журнале
%<<Информатика и её применения>>}


\titele{\tit}{\aut}{\auf}{\leftkol}{\rightkol}


\vspace*{-60pt}
{ %\small

Журнал <<Информатика и её применения>>
публикует теоретические, обзорные и дискуссионные статьи,
посвященные научным исследованиям и разработкам в области
информатики и ее приложений.

Журнал издается на русском языке. По специальному решению
редколлегии отдельные статьи могут печататься на английском языке.

Тематика журнала охватывает следующие направления:
\begin{itemize}
\item теоретические основы информатики;\\[-15pt]
      \item
математические методы исследования сложных систем и процессов;\\[-15pt]
           \item
информационные системы и сети;\\[-15pt]
                \item
информационные технологии;\\[-15pt]
                     \item
архитектура и программное обеспечение вычислительных комплексов и сетей.\\[-15pt]
\end{itemize}


\noindent
\begin{enumerate}[1.]
\item В журнале печатаются статьи, содержащие результаты, ранее не опубликованные и
не предназначенные к одновременной публикации в других изданиях.

%Публикация не должна нарушать закон об авторских правах.
Публикация предоставленной автором(ами) рукописи не должна нарушать 
положений глав~69, 70 раздела~VII части~IV Гражданского кодекса, 
которые определяют права на результаты интеллектуальной деятельности 
и~средства индивидуализации, в~том числе авторские права, в~РФ.

Ответственность за нарушение авторских прав, в~случае предъявления претензий к~редакции журнала,  
несут авторы статей.



Направляя рукопись в редакцию, авторы сохраняют свои права на данную
рукопись и при этом передают учредителям и редколлегии журнала неисключительные права на
издание статьи на русском языке 
(или на языке статьи, если он отличен от рус\-ско\-го) и~на перевод ее на английский
язык, а~также на
ее распространение в России и за рубежом. 
Каждый автор должен представить в~редакцию подписанный 
с~его стороны <<Лицензионный договор о~передаче неисключительных прав 
на использование произведения>>, текст которого размещен по адресу 
{\sf http://www.ipiran.ru/publications/licence.doc}. 
Этот договор может быть пред\-став\-лен в~бумажном (в~2-х экз.)\ 
или в~электронном виде (отсканированная копия заполненного и~подписанного документа).




Редколлегия вправе запросить у авторов экспертное заключение о возможности
пуб\-ли\-ка\-ции пред\-став\-лен\-ной статьи в открытой печати.\\[-13.5pt]

\item К статье прилагаются данные автора (авторов) (см.\ п.~8). При наличии нескольких
авторов указывается фамилия автора, ответственного за переписку с редакцией.\\[-13.5pt]

\item Редакция журнала осуществляет экспертизу присланных статей в соответствии с
принятой в журнале процедурой рецензирования.

Возвращение рукописи на доработку не означает ее принятия к печати.

Доработанный вариант с ответом на замечания рецензента необходимо прислать в
редакцию.\\[-13.5pt]

\item Решение редколлегии о публикации статьи или ее отклонении сообщается авторам.

Редколлегия может также направить авторам текст рецензии на их статью. Дискуссия по
поводу отклоненных статей не ведется.\\[-13.5pt]

%\pagebreak

\item Редактура статей высылается авторам для просмотра. Замечания к редактуре должны
быть присланы авторами в кратчайшие сроки.\\[-13.5pt]

\item Рукопись предоставляется в электронном виде в форматах MS WORD (.doc или
.docx) или \LaTeX\  (.tex), дополнительно~--- в формате .pdf, на дискете, лазерном диске
или электронной почтой. Предоставление бумажной рукописи необязательно.\\[-13.5pt]

\item При подготовке рукописи в MS Word рекомендуется использовать следующие
настройки.

Параметры страницы:
формат~--- А4; ориентация~--- книжная; поля (см): внутри~--- 2,5, снаружи~--- 1,5,
сверху~--- 2, снизу~--- 2, от края до нижнего колонтитула~--- 1,3.

Основной текст: стиль~--- <<Обычный>>, шрифт~--- Times New Roman, размер~---
14~пунк\-тов, абзацный отступ~--- 0,5~см, 1,5~интервала, выравнивание~--- по ширине.

\pagebreak

\def\leftkol{Правила подготовки рукописей  для публикации в журнале
<<Информатика и её применения>>}

\def\rightkol{Правила подготовки рукописей  для публикации в журнале
<<Информатика и её применения>>}



Рекомендуемый объем рукописи~--- не свыше 10~страниц указанного формата.
При превышении указанного объема редколлегия вправе потребовать от 
автора сокращения объема рукописи.


Сокращения слов, помимо стандартных, не допускаются. Допускается минимальное
количество аббревиатур.


Все страницы рукописи нумеруются.

Шаблоны оформления представлены в интернете:

\noindent
 {\sf
http://www.ipiran.ru/journal/template\_iiep\_ssi\_2024.zip}\\[-14pt]

\item Статья должна содержать следующую информацию на {\bfseries\textit{русском и
английском языках}}:\\[-16pt]

\begin{itemize}
\item название статьи;\\[-15pt]
\item Ф.И.О.\ авторов, на английском можно только имя и фамилию;\\[-15pt]
\item место работы, с указанием почтового адреса организации и электронного адреса каждого
автора;\\[-15pt]
\item сведения об авторах, в соответствии с форматом, образцы которого
представлены на страницах:



\def\leftfootline{\small{\textbf{\thepage}
\hfill ИНФОРМАТИКА И ЕЁ ПРИМЕНЕНИЯ\ \ \ том\ 18\ \ \ выпуск\ 3\ \ \ 2024}
}%
 \def\rightfootline{\small{ИНФОРМАТИКА И ЕЁ ПРИМЕНЕНИЯ\ \ \ том\ 18\ \ \ выпуск\ 3\ \ \ 2024
\hfill \textbf{\thepage}}}



{\sf http://www.ipiran.ru/journal/issues/2013\_07\_01/authors.asp} и

{\sf http://www.ipiran.ru/journal/issues/2013\_07\_01\_eng/authors.asp};
\item аннотация (не менее 100~слов на каждом из языков). Аннотация~--- это краткое
резюме работы, которое может публиковаться отдельно. Она является основным
источником информации в~ин\-фор\-ма\-ци\-он\-ных системах и базах данных. Английская
аннотация должна быть оригинальной, может не быть дословным переводом русского
текста и должна быть написана хорошим английским языком. В~аннотации не должно
быть ссылок на литературу и, по возможности, формул;\\[-15pt]
\item ключевые слова~--- желательно из принятых в мировой
на\-уч\-но-тех\-ни\-че\-ской литературе тематических тезаурусов. Предложения не
могут быть ключевыми словами;\\[-15pt]
\item источники финансирования работы (ссылки на гранты, проекты,
поддерживающие организации и~т.\,п.).
\end{itemize}



%\pagebreak

\item  Требования к спискам литературы.\\[-14pt]

Ссылки на литературу в тексте статьи нумеруются (в квадратных скобках) и
располагаются в каждом из списков литературы в порядке  первых упоминаний. Если источник имеет DOI и/или EDN,
то их необходимо указывать.

Списки литературы представляются в двух вариантах:\\[-14pt]


\noindent
\begin{enumerate}[(1)]
\item \textbf{Список литературы к русскоязычной части}. Русские и английские
работы~---  на языке и в алфавите оригинала;\\[-14.5pt]
\item  \textbf{References}. Русские работы и работы на других языках~--- в латинской
транслитерации с переводом на английский язык; английские работы и работы на других
языках~--- на языке оригинала.
\end{enumerate}

Необходимо для составления списка ``References'' пользоваться размещенной на сайте
{\sf http://www. translit.net/ru/bgn/} бесплатной программой транслитерации русского
 текста в~латиницу. %, при этом в~за\-клад\-ке <<варианты\ldots>> следует выбратьопцию BGN.

Список литературы ``References'' приводится полностью отдельным блоком, повторяя все
позиции из списка литературы к русскоязычной части, независимо от того, имеются или
нет в нем иностранные источники. Если в списке литературы к русскоязычной части есть
ссылки на иностранные публикации, набранные латиницей, они полностью повторяются в
списке ``References''.

Ниже приведены примеры ссылок на различные виды публикаций в списке ``References''.

\def\leftfootline{\small{\textbf{\thepage}
\hfill ИНФОРМАТИКА И ЕЁ ПРИМЕНЕНИЯ\ \ \ том\ 18\ \ \ выпуск\ 3\ \ \ 2024}
}%
 \def\rightfootline{\small{ИНФОРМАТИКА И ЕЁ ПРИМЕНЕНИЯ\ \ \ том\ 18\ \ \ выпуск\ 3\ \ \ 2024
\hfill \textbf{\thepage}}}

{\small

\noindent
\textbf{Описание статьи из журнала:}

\Aue{Zagurenko, A.\,G., V.\,A.~Korotovskikh, A.\,A.~Kolesnikov, A.\,V.~Timonov, and D.\,V.~Kardymon}. 2008.
Tekhniko-ekonomicheskaya optimizatsiya dizayna gidrorazryva plasta [Technical and
economic optimization of the design
of hydraulic fracturing]. \textit{Neftyanoe hozyaystvo} [\textit{Oil Industry}] 11:54--57.

\Aue{Zhang, Z., and D.~Zhu}. 2008. Experimental research on the localized
electrochemical micromachining. \textit{Russ. J.~Electrochem.}  44(8):926--930.
{\sf doi:10.1134/S1023193508080077}.

\noindent
\textbf{Описание статьи из электронного журнала:}

\Aue{Swaminathan, V., E.~Lepkoswka-White, and B.\,P.~Rao}. 1999. Browsers or buyers in cyberspace? An
investigation of electronic factors influencing electronic exchange. \textit{JCMC}
5(2). Available at: {\sf http://www.ascusc.org/jcmc/vol5/issue2/} (accessed April~28, 2011).

\def\leftkol{Правила подготовки рукописей  для публикации в журнале
<<Информатика и её применения>>}

\def\rightkol{Правила подготовки рукописей  для публикации в журнале
<<Информатика и её применения>>}


\noindent
\textbf{Описание статьи из продолжающегося издания (сборника трудов):}

\Aue{Astakhov, M.\,V., and T.\,V.~Tagantsev}. 2006. Eksperimental'noe
issledovanie prochnosti soedineniy ``stal'--kompozit'' [Experimental study of
the strength of joints ``steel--composite'']. \textit{Trudy MGTU
``Matematicheskoe modelirovanie slozhnykh tekh\-ni\-che\-skikh sistem''}
[\textit{Bauman MSTU ``Mathematical Modeling of Complex Technical
Systems'' Proceedings}]. 593:125--130.


\pagebreak



\noindent
\textbf{Описание материалов конференций:}

\Aue{Usmanov, T.\,S., A.\,A.~Gusmanov, I.\,Z.~Mullagalin, R.\,Ju.~Muhametshina, A.\,N.~Chervyakova, and
A.\,V.~Sveshnikov}. 2007. Osobennosti proektirovaniya razrabotki mestorozhdeniy
s primeneniem gidrorazryva
plasta [Features of the design of field development with the use of hydraulic fracturing].
\textit{Trudy 6-go
Mezhdu\-na\-rod\-no\-go Simpoziuma ``Novye resursosberegayushchie tekhnologii nedropol'zovaniya i povysheniya
neftegazootdachi''} [\textit{6th  Symposium (International) ``New Energy Saving Subsoil Technologies and
the Increasing of the Oil and Gas Impact'' Proceedings}]. Moscow. 267--272.



\def\leftfootline{\small{\textbf{\thepage}
\hfill ИНФОРМАТИКА И ЕЁ ПРИМЕНЕНИЯ\ \ \ том\ 18\ \ \ выпуск\ 3\ \ \ 2024}
}%
 \def\rightfootline{\small{ИНФОРМАТИКА И ЕЁ ПРИМЕНЕНИЯ\ \ \ том\ 18\ \ \ выпуск\ 3\ \ \ 2024
\hfill \textbf{\thepage}}}



\noindent
\textbf{Описание книги (монографии, сборники):}



Lindorf, L.\,S., and L.\,G.~Mamikoniants, eds. 1972.
\textit{Ekspluatatsiya turbogeneratorov s neposredstvennym
okhlazhdeniem} [\textit{Operation of turbine generators with direct cooling}].
Moscow: Energy Publs. 352~p.


\Aue{Latyshev, V.\,N.} 2009. \textit{Tribologiya rezaniya. Kn.~1: Friktsionnye protsessy
pri rezanii metallov}
[\textit{Tribology of cutting. Vol.~1: Frictional processes in metal cutting}]. Ivanovo: Ivanovskii
State Univ. 108~p.

\def\leftkol{Правила подготовки рукописей  для публикации в журнале
<<Информатика и её применения>>}

\def\rightkol{Правила подготовки рукописей  для публикации в журнале
<<Информатика и её применения>>}

\noindent
\textbf{Описание переводной книги}
(в списке литературы к русскоязычной части необходимо указать:~/ Пер.\ с англ.~---
после названия книги, а в конце ссылки указать оригинал книги в круглых скобках):
\begin{enumerate}[1.]
\item  В русскоязычной части:

\def\leftfootline{\small{\textbf{\thepage}
\hfill ИНФОРМАТИКА И ЕЁ ПРИМЕНЕНИЯ\ \ \ том\ 18\ \ \ выпуск\ 3\ \ \ 2024}
}%
 \def\rightfootline{\small{ИНФОРМАТИКА И ЕЁ ПРИМЕНЕНИЯ\ \ \ том\ 18\ \ \ выпуск\ 3\ \ \ 2024
\hfill \textbf{\thepage}}}

\Au{Тимошенко С.\,П., Янг Д.\,Х., Уивер~У.}
Колебания в инженерном деле~/ Пер.\ с англ.~--- М.: Машиностроение, 1985. 472~с.
(\Au{Timoshenko~S.\,P., Young~D.\,H., Weaver~W.}
Vibration problems in engineering.~--- 4th ed.~--- New York, NY, USA: Wiley, 1974. 521~p.)\\[-13.5pt]
\item  В англоязычной части:

\Aue{Timoshenko, S.\,P., D.\,H.~Young, and W.~Weaver}.
1974. \textit{Vibration problems in engineering}. 4th ed. New York: 
Wiley. 521~p.
\end{enumerate}

\vspace*{-3pt}


\noindent
\textbf{Описание неопубликованного документа:}


\Aue{Latypov, A.\,R., M.\,M.~Khasanov, and V.\,A.~Baikov}.
2004 (unpubl.). Geologiya i~dobycha (NGT GiD) [Geology and production (NGT GiD)]. Certificate on official registration of the computer program
No.\,2004611198. 

\noindent
\textbf{Описание интернет-ресурса:}


Pravila tsitirovaniya istochnikov [Rules for the citing of sources]. Available at: {\sf
http://www.scribd.com/doc/1034528/} (accessed February~7, 2011).

%\pagebreak

\noindent
\textbf{Описание диссертации или автореферата диссертации:}

\Aue{Semenov, V.\,I.}
2003. Matematicheskoe modelirovanie plazmy v sisteme kompaktnyy tor [Mathematical
modeling of the plasma in the compact torus].  Moscow.  D.Sc.\ Diss. 272~p.

\Aue{Kozhunova, O.\,S.} 2009. Tekhnologiya razrabotki semanticheskogo
slovarya informatsionnogo monitoringa [Technology of development of
semantic dictionary of information monitoring system].  Moscow: IPI RAN. PhD Thesis. 23~p.


\noindent
\textbf{Описание ГОСТа:}

GOST 8.586.5-2005. 2007. Metodika vypolneniya izmereniy. Izmerenie raskhoda i~kolichestva zhidkostey i~gazov
s~pomoshch'yu standartnykh suzhayushchikh ustroystv [Method of measurement.
Measurement of flow rate and volume of liquids and gases by means of orifice devices]. Moscow:
Standardinform  Publs. 10~p.

\noindent
\textbf{Описание патента:}

\Aue{Bolshakov, M.\,V., A.\,V.~Kulakov, A.\,N.~Lavrenov, and M.\,V.~Palkin}.
2006. Sposob orientirovaniya po krenu letatel'nogo
apparata s opti\-che\-skoy golovkoy
samonavedeniya [The way to orient on the roll of aircraft with optical homing head].
Patent RF No.\,2280590.
}

\item Присланные в редакцию материалы авторам не возвращаются.\\[-13.5pt]

\item При отправке файлов по электронной почте просим придерживаться следующих
правил:
\begin{itemize}
\item указывать в поле subject (тема) название журнала и фамилию автора;\\[-13.5pt]
\item указывать в тексте письма название статьи, авторов и~журнал, в~который направляется статья;\\[-13.5pt]
\item использовать attach (присоединение);\\[-13.5pt]
\item в состав электронной версии статьи должны входить: файл, содержащий текст
статьи, и файл(ы), содержащий(е) иллюстрации.\\[-13.5pt]
\end{itemize}

\item Журнал <<Информатика и её применения>> является некоммерческим изданием.
Плата за публикацию не взимается, гонорар авторам не выплачивается.
\end{enumerate}



\def\leftfootline{\small{\textbf{\thepage}
\hfill ИНФОРМАТИКА И ЕЁ ПРИМЕНЕНИЯ\ \ \ том\ 18\ \ \ выпуск\ 3\ \ \ 2024}
}%
 \def\rightfootline{\small{ИНФОРМАТИКА И ЕЁ ПРИМЕНЕНИЯ\ \ \ том\ 18\ \ \ выпуск\ 3\ \ \ 2024
\hfill \textbf{\thepage}}}


\vspace*{-1mm}

\begin{center}

\textbf{Адрес редакции журнала <<Информатика и её применения>>:} \\




Москва 119333, ул.~Вавилова, д.~44, корп.~2, ФИЦ ИУ РАН\\[-10pt]

\

Тел.: +7\,(499)\,135-86-92\ \ Факс:  +7\,(495)\,930-45-05\\[-10pt]

 \

e-mail:   {\sf iiep@frccsc.ru} (Стригина Светлана Николаевна)\\[-10pt]

\

{\sf http://www.ipiran.ru/journal/issues/}
\end{center}
}


\def\leftkol{Правила подготовки рукописей  для публикации в журнале
<<Информатика и её применения>>}

\def\rightkol{Правила подготовки рукописей  для публикации в журнале
<<Информатика и её применения>>}


\def\leftfootline{\small{\textbf{\thepage}
\hfill ИНФОРМАТИКА И ЕЁ ПРИМЕНЕНИЯ\ \ \ том\ 18\ \ \ выпуск\ 3\ \ \ 2024}
}%
 \def\rightfootline{\small{ИНФОРМАТИКА И ЕЁ ПРИМЕНЕНИЯ\ \ \ том\ 18\ \ \ выпуск\ 3\ \ \ 2024
\hfill \textbf{\thepage}}} 
\def\stat{podg-e}
{%\hrule\par
%\vskip 7pt % 7pt
\vspace*{-24pt}
\raggedleft\Large \bf%\baselineskip=3.2ex
Requirements for manuscripts submitted to Journal
``Informatics~and~Applications'' \vskip 8pt
    \hrule
    \par
\vskip 21pt plus 6pt minus 3pt }

\label{st\stat}

\def\tit{\ }

\def\aut{\ }
\def\auf{\ }

\def\leftkol{\ }

\def\rightkol{\ }
%Requirements for manuscripts submitted to Journal
%``Informatics~and~Applications''}

\titele{\tit}{\aut}{\auf}{\leftkol}{\rightkol}

\def\leftfootline{\small{\textbf{\thepage}
\hfill INFORMATIKA I EE PRIMENENIYA~--- INFORMATICS AND APPLICATIONS\ \ \ 2019\
\ \ volume~13\ \ \ issue\ 4}
}%
 \def\rightfootline{\small{INFORMATIKA I EE PRIMENENIYA~--- INFORMATICS AND APPLICATIONS\ \ \ 2019\ \ \ volume~13\ \ \ issue\ 4
\hfill \textbf{\thepage}}}

\vspace*{-60pt}

{\small

\noindent
Journal ``Informatics and Applications'' (Inform.\ Appl.)
publishes theoretical, review, and discussion
articles on the research and development in the
field of informatics and its applications.

The journal is published in Russian.
By a special decision of the editorial
board, some articles can be published in English.


The topics covered include the following areas:
\begin{itemize}
               \item
     theoretical fundamentals of informatics; \\[-14pt]
\item
mathematical methods for studying complex systems and processes; \\[-14pt]
\item
information systems and networks;\\[-14pt]
\item
information technologies; and \\[-14pt]
\item
architecture and software of computational complexes and networks. \\[-14pt]
\end{itemize}

\noindent
\begin{enumerate}[1.]
\item The Journal publishes original articles which have not been published before and are not
intended for simultaneous publication in other editions. An article submitted to the Journal must not violate the
Copyright law. Sending the manuscript to the Editorial Board, the authors retain all rights of the
owners of the manuscript and transfer the nonexclusive rights to publish the article in Russian
(or the language of the article, if not Russian) and its distribution in Russia and abroad to the
Founders and the Editorial Board. Authors should submit a letter to the Editorial Board in the
following form:

{\bfseries\textit{Agreement on the transfer of rights to publish:}}

``\textit{We, the undersigned authors of the manuscript ``\ldots'', pass to the
Founder and the Editorial Board of the Journal ``Informatics and Applications''
the nonexclusive right to publish the manuscript of the article in Russian (or
in English) in both print and electronic versions of the Journal. We affirm
that this publication does not violate the Copyright of other persons or
organizations.}

\textit{Author(s) signature(s): (name(s), address(es), date).}

This agreement should be submitted in paper form or in the form of a scanned copy (signed by
the authors).


%The Editorial Board has the right to request from the authors an official expert conclusion that
%the submitted article has no secret data prohibited for publication. \\[-13.5pt]
\item
A submitted article should be attached with \textbf{the data on the author(s)} (see item~8). If
there are several authors, the contact person should be indicated who is responsible for
correspondence with the Editorial Board and other authors about revisions and final approval
of the proofs.\\[-13.5pt]

\item The Editorial Board of the Journal examines the article according to the established
reviewing procedure. If the authors receive their article for correction after reviewing, it does not
mean that the article is approved for publication. The corrected article should be sent to the
Editorial Board for the subsequent review and approval.\\[-13.5pt]

\item The decision on the article publication or its rejection is communicated to the authors. The
Editorial Board may also send the reviews on the submitted articles to the authors. Any
discussion upon the rejected articles is not possible.\\[-13.5pt]

\item The edited articles will be sent to the authors for proofread. The comments of the authors
to the edited text of the article should be sent to the Editorial Board as soon as possible.\\[-13.5pt]

\item The manuscript of the article should be presented electronically in the MS WORD (.doc or
.docx) or \LaTeX\ (.tex) formats, and additionally in the .pdf format. All documents
 may be sent
by e-mail or provided on a CD or diskette. A~hard copy submission is not necessary.\\[-13.5pt]

\item The recommended typesetting instructions for manuscript.

Pages parameters: format A4, portrait orientation, document margins (cm): left~--- 2.5, right~---
1.5, above~--- 2.0, below~--- 2.0, footer 1.3.

Text: font~---Times New Roman, font size~--- 14, paragraph indent~--- 0.5, line spacing~--- 1.5,
justified alignment.

The recommended manuscript size: not more than 15~pages of the specified format.
If the specified size exceeded, the editorial board is entitled to require the author
to reduce the manuscript.

Use only standard abbreviations. Avoid  abbreviations in the title and
abstract. The full term for which an abbreviation stands should precede
its first use in the text unless it is a standard unit of measurement.

All pages of the manuscript should be numbered.

The templates for the manuscript typesetting are presented on site: {\sf
http://www.ipiran.ru/journal/template.doc}.\\[-13.5pt]


%\def\leftkol{Requirements for manuscripts submitted to Journal
%``Informatics~and~Applications''}

\item The articles should enclose data both in \textbf{Russian and English}:
\begin{itemize}
\item title;\\[-13.5pt]
\item author's name and surname;\\[-13.5pt]
\item affiliation~--- organization, its address with ZIP code, city, country, and
official e-mail address;\\[-13.5pt]
\item data on authors according to the format: (see site)

{\sf http://www.ipiran.ru/journal/issues/2013\_07\_01/authors.asp}  and

{\sf  http://www.ipiran.ru/journal/issues/2013\_07\_01\_eng/authors.asp};\\[-13.5pt]

\pagebreak

\def\leftfootline{\small{\textbf{\thepage}
\hfill INFORMATIKA I EE PRIMENENIYA~--- INFORMATICS AND APPLICATIONS\ \ \ 2019\
\ \ volume~13\ \ \ issue\ 4}
}%
 \def\rightfootline{\small{INFORMATIKA I EE PRIMENENIYA~--- INFORMATICS AND APPLICATIONS\ \ \ 2019\ \ \ volume~13\ \ \ issue\ 4
\hfill \textbf{\thepage}}}


%\def\leftkol{Requirements for manuscripts submitted to Journal
%``Informatics~and~Applications''}

%\def\rightkol{Requirements for manuscripts submitted to Journal
%``Informatics~and~Applications''}



\item abstract (not less than 100 words) both in Russian and in English. Abstract is a short
summary of the article that can be published separately. The abstract is the
main source of information on the article and it could be included in leading information
systems and data bases. The abstract in English has to be an original text and should
not be an exact translation of the Russian one. Good English is required.
In abstracts, avoid references and formulae;\\[-13.5pt]
\item indexing is performed on the basis of keywords. The use of keywords from the
internationally accepted thematic Thesauri is recommended.

%\def\leftkol{Requirements for manuscripts submitted to Journal
%``Informatics~and~Applications''}

%\def\rightkol{Requirements for manuscripts submitted to Journal
%``Informatics~and~Applications''}

Important! Keywords must not be sentences;
\item Acknowledgments.
\end{itemize}

\item References. Russian references have to be presented both in English translation and Latin
transliteration (refer {\sf http://www.translit.net/ru/bgn/}).

Please take into account the following examples of Russian references appearance:

\noindent
\textbf{Article in journal:}

\Aue{Zhang, Z., and D.~Zhu}. 2008. Experimental research on the localized electrochemical
micromachining.
\textit{Rus. J.~Electrochem.}  44(8):926--930. {\sf doi:10.1134/S1023193508080077}.


\noindent
\textbf{Journal article in electronic format:}

\Aue{Swaminathan, V., E.~Lepkoswka-White, and B.\,P.~Rao}. 1999. Browsers or buyers in
cyberspace? An
investigation of electronic factors influencing electronic exchange. \textit{JCMC}
5(2). Available at: {\sf http://www.ascusc.org/jcmc/vol5/issue2/} (accessed April~28, 2011).




\noindent
\textbf{Article from the continuing publication (collection of works, proceedings):}

\Aue{Astakhov, M.\,V., and T.\,V.~Tagantsev}. 2006. Eksperimental'noe
issledovanie prochnosti soedineniy ``stal'--kompozit'' [Experimental study of
the strength of joints ``steel--composite'']. \textit{Trudy MGTU
``Matematicheskoe modelirovanie slozhnykh tekh\-ni\-che\-skikh sistem''}
[\textit{Bauman MSTU ``Mathematical Modeling of Complex Technical
Systems'' Proceedings}]. 593:125--130.

\def\leftfootline{\small{\textbf{\thepage}
\hfill INFORMATIKA I EE PRIMENENIYA~--- INFORMATICS AND APPLICATIONS\ \ \ 2019\
\ \ volume~13\ \ \ issue\ 4}
}%
 \def\rightfootline{\small{INFORMATIKA I EE PRIMENENIYA~--- INFORMATICS AND APPLICATIONS\ \ \ 2019\ \ \ volume~13\ \ \ issue\ 4
\hfill \textbf{\thepage}}}

\def\leftkol{Requirements for manuscripts submitted to Journal
``Informatics~and~Applications''}

\def\rightkol{Requirements for manuscripts submitted to Journal
``Informatics~and~Applications''}

\noindent
\textbf{Conference proceedings:}

\Aue{Usmanov, T.\,S., A.\,A.~Gusmanov, I.\,Z.~Mullagalin, R.\,Ju.~Muhametshina,
A.\,N.~Chervyakova, and
A.\,V.~Sveshnikov}. 2007. Osobennosti proektirovaniya razrabotki mestorozhdeniy
s primeneniem gidrorazryva
plasta [Features of the design of field development with the use of hydraulic fracturing].
\textit{Trudy 6-go
Mezhdu\-na\-rod\-no\-go Simpoziuma ``Novye resursosberegayushchie tekhnologii
nedropol'zovaniya i povysheniya
neftegazootdachi''} [\textit{6th  Symposium (International) ``New Energy Saving Subsoil
Technologies and
the Increasing of the Oil and Gas Impact'' Proceedings}]. Moscow. 267--272.


\noindent
\textbf{Books and other monographs:}




Lindorf, L.\,S., and L.\,G.~Mamikoniants, eds. 1972.
\textit{Ekspluatatsiya turbogeneratorov s neposredstvennym
okhlazhdeniem} [\textit{Operation of turbine generators with direct cooling}].
Moscow: Energy Publs. 352~p.


%\Aue{Latyshev, V.\,N.} 2009. \textit{Tribologiya rezaniya. Kn.~1: Frikcionnye prosessy
%pri rezanii metallov}
%[\textit{Tribology of cutting. Vol.~1: Frictional processes in metal cutting}]. Ivanovo: Ivanovskii
%State Univ. 108~p.


%\noindent
%\textbf{Unpublished material:}

%\Aue{Latypov, A.\,R., M.\,M.~Khasanov, and V.\,A.~Baikov}.
%2004. Geology and production (NGT GiD). Certificate on official registration of the computer
%program
%No.\,2004611198. (In Russian, unpubl.)

%\noindent
%\textbf{Internet-source:}

%APA Style. 2011. Available at: {\sf http://www.apastyle.org/apa-style-help.aspx} (accessed
%February~5, 2011).

%Pravila citirovaniya istochnikov [Rules for the citing of sources]. Available at: {\sf
%http://www.scribd.com/doc/1034528/} (accessed February~7, 2011).


\noindent
\textbf{Dissertation and Thesis:}

%\Aue{Semenov, V.\,I.}
%2003. Matematicheskoe modelirovanie plazmy v sisteme kompaktnyy tor. [Mathematical
%modeling of the plasma in the compact torus]. D.Sc.\ Diss. Moscow. 272~p.

\Aue{Kozhunova, O.\,S.} 2009. Tekhnologiya razrabotki semanticheskogo
slovarya informatsionnogo monitoringa [Technology of development of
semantic dictionary of information monitoring system]. PhD Thesis. Moscow: IPI RAN. 23~p.


\noindent
\textbf{State standards and patents:}

GOST 8.586.5-2005. 2007. Metodika vypolneniya izmereniy. Izmerenie raskhoda i~kolichestva
zhidkostey i gazov 
s~pomoshch'yu standartnykh suzhayushchikh ustroystv [Method of measurement.
Measurement of flow rate and volume of liquids and gases by means of orifice devices]. M.:
Standardinform
Publs. 10~p.

%\noindent
%\textbf{Patent:}

\Aue{Bolshakov, M.\,V., A.\,V.~Kulakov, A.\,N.~Lavrenov, and M.\,V.~Palkin}.
2006. Sposob orientirovaniya po krenu letatel'nogo
apparata s opti\-che\-skoy golovkoy
samonavedeniya [The way to orient on the roll of aircraft with optical homing head].
Patent RF No.\,2280590.

References in Latin transcription are presented in the original language.

References in the text are numbered according to the order of their
first appearance; the number is
placed in square brackets. All items from the reference list should be
cited.\\[-13.5pt]

\item Manuscripts and additional materials are not returned to Authors by the Editorial Board.\\[-13.5pt]

\item Submissions of files by e-mail must include:\\[-13.5pt]
\begin{itemize}
\item   the journal title and author's name in the ``Subject'' field; \\[-13.5pt]
\item   an article and additional materials have to be attached using the ``attach'' function;\\[-13.5pt]
\item   an electronic version of the article should contain the file with the text and a separate file
with figures.\\[-13.5pt]
\end{itemize}

\item ``Informatics and Applications'' journal is not a profit publication. There are no
charges for the authors as well as there are no royalties.\\[-13.5pt]
\end{enumerate}

\def\leftfootline{\small{\textbf{\thepage}
\hfill INFORMATIKA I EE PRIMENENIYA~--- INFORMATICS AND APPLICATIONS\ \ \ 2019\
\ \ volume~13\ \ \ issue\ 4}
}%
 \def\rightfootline{\small{INFORMATIKA I EE PRIMENENIYA~--- INFORMATICS AND APPLICATIONS\ \ \ 2019\ \ \ volume~13\ \ \ issue\ 4
\hfill \textbf{\thepage}}}

\def\leftkol{Requirements for manuscripts submitted to Journal
``Informatics~and~Applications''}

\def\rightkol{Requirements for manuscripts submitted to Journal
``Informatics~and~Applications''}


%\vspace*{5mm}


\begin{center}
\textbf{Editorial Board address:} \\

%ABOUT AUTHORS



FRC CSC RAS, 44, block~2, Vavilov Str., Moscow 119333, Russia\\[-10pt]

\

Ph.: +7\,(499)\,135\,86\,92,\ \ Fax: +7\,(495)\,930\,45\,05\\[-10pt]

\

 e-mail: {\sf rust@ipiran.ru} (to Prof.\ Rustem Seyful-Mulyukov)\\[-10pt]

\

 {\sf http://www.ipiran.ru/english/journal.asp}
\end{center}
 }
%\thispagestyle{myheadings}

\def\leftkol{Requirements for manuscripts submitted to Journal
``Informatics~and~Applications''}

\def\rightkol{Requirements for manuscripts submitted to Journal
``Informatics~and~Applications''}

\def\leftfootline{\small{\textbf{\thepage}
\hfill INFORMATIKA I EE PRIMENENIYA~--- INFORMATICS AND APPLICATIONS\ \ \ 2019\
\ \ volume~13\ \ \ issue\ 4}
}%
 \def\rightfootline{\small{INFORMATIKA I EE PRIMENENIYA~--- INFORMATICS AND APPLICATIONS\ \ \ 2019\ \ \ volume~13\ \ \ issue\ 4
\hfill \textbf{\thepage}}}

 \label{end\stat}

\newpage

%\vspace*{-60pt} {\small
{\baselineskip=9.1pt
\section*{Правила подготовки рукописей статей для публикации в журнале
<<Информатика и её применения>>}

\thispagestyle{empty}

 Журнал <<Информатика и её применения>> публикует
теоретические, обзорные и дискуссионные статьи, посвященные научным
исследованиям и разработкам в области информатики и ее приложений. Журнал
издается на русском языке. По специальному решению редколлегии отдельные статьи,
в виде исключения, могут печататься на английском языке.
Тематика журнала охватывает следующие направления:
\begin{itemize}
\item теоретические основы информатики; %\\[-13.5pt]
\item математические методы исследования сложных систем и процессов; %\\[-13.5pt]
\item информационные системы и сети; %\\[-13.5pt]
\item информационные технологии; %\\[-13.5pt]
\item архитектура и программное
обеспечение вычислительных комплексов и сетей.
\end{itemize}
\begin{enumerate}
\item В журнале печатаются результаты, ранее не
опубликованные и не предназначенные к одновременной публикации в других
изданиях. Публикация не должна нарушать закон об авторских правах. Направляя
свою рукопись в редакцию, авторы автоматически передают учредителям и
редколлегии неисключительные права на издание данной статьи на русском языке и
на ее распространение в России и за рубежом. При этом за авторами сохраняются
все права как собственников данной рукописи. В связи с этим авторами должно
быть представлено в редакцию письмо в следующей форме:
Соглашение о передаче права на публикацию:

\textit{<<Мы, нижеподписавшиеся, авторы рукописи <<$\qquad\qquad$>>, передаем
учредителям и редколлегии журнала <<Информатика и её применения>>
неисключительное право опубликовать данную рукопись статьи на русском языке как
в печатной, так и в электронной версиях журнала. Мы подтверждаем, что данная
публикация не нарушает авторского права других лиц или организаций. Подписи
авторов: (ф.\,и.\,о., дата, адрес)>>.}

Указанное соглашение может быть представлено 
как в бумажном виде, так и в виде отсканированной копии (с подписями авторов).


Редколлегия вправе запросить у авторов экспертное заключение о возможности
опубликования представленной статьи в открытой печати. %\\[-13.5pt]
\item Статья
подписывается всеми авторами. На отдельном листе представляются данные автора
(или всех авторов): фамилия, полные имя и отчество, телефон, факс, e-mail,
почтовый адрес. Если работа выполнена несколькими авторами, указывается фамилия
одного из них, ответственного за переписку с редакцией. %\\[-13.5pt]
\item Редакция журнала
осуществляет самостоятельную экспертизу присланных статей. Возвращение рукописи
на доработку не означает, что статья уже принята к печати. Доработанный вариант
с ответом на замечания рецензента необходимо прислать в редакцию. %\\[-13.5pt]
\item Решение
редакционной коллегии о принятии статьи к печати или ее отклонении сообщается
авторам. Редколлегия не обязуется направлять рецензию авторам отклоненной
статьи. %\\[-13.5pt]
\item Корректура статей высылается авторам для просмотра. Редакция
просит авторов присылать свои замечания в кратчайшие сроки. %\\[-13.5pt]
\item При
подготовке рукописи в MS Word рекомендуется использовать следующие настройки.
Параметры страницы: формат~--- А4; ориентация~--- книжная; поля (см): внутри~---
2,5, снаружи~--- 1,5, сверху~--- 2, снизу~--- 2, от края до нижнего
колонтитула~--- 1,3. Основной текст: стиль~--- <<Обычный>>: шрифт Times New
Roman, размер 14~пунктов, абзацный отступ~--- 0,5~см, 1,5 интервала,
выравнивание~--- по ширине. Рекомендуемый объем рукописи~--- не свыше
25~страниц указанного формата. Ознакомиться с шаблонами, содержащими примеры
оформления, можно по адресу в Интернете:
\textsf{http://www.ipiran.ru/journal/template.doc}.
\item К рукописи, предоставляемой в 2-х
экземплярах, обязательно прилагается электронная версия статьи (как правило, в
форматах MS WORD (.doc) или \LaTeX\ (.tex), а также~--- дополнительно~--- в
формате .pdf) на дискете, лазерном диске или по электронной почте. Сокращения
слов, кроме стандартных, не применяются. Все страницы рукописи должны быть
пронумерованы. %\\[-13.5pt]
\item Статья должна содержать следующую информацию на русском и
английском языках: название, Ф.И.О. авторов, места работы авторов и их
электронные адреса, подробные сведения об авторах, оформленные в соответствии с форматом, 
определяемым файлами {\sf http://www.ipiran.ru/journal/issues/2011\_05\_01/authors.asp} и 
{\sf http://www.ipiran.ru/journal/issues/2011\_01\_eng/authors.asp},
аннотация (не более 100~слов), ключевые слова. Ссылки на
литературу в тексте статьи нумеруются (в квадратных скобках) и располагаются в
порядке их первого упоминания. В~списке литературы не должно быть позиций, на которые нет ссылки в тексте статьи.
Все фамилии авторов, заглавия статей, названия
книг, конференций и~т.\,п.\ даются на языке оригинала, если этот язык
использует кириллический или латинский алфавит. %\\[-13.5pt]
\item Присланные в редакцию материалы авторам не возвращаются.
\item При отправке файлов по электронной
почте просим придерживаться следующих правил:
\begin{itemize}
\item указывать в поле subject (тема) название журнала и фамилию автора; %\\[-13.5pt]
\item использовать attach (присоединение); %\\[-13.5pt]
\item в случае больших объемов информации возможно
использование общеизвестных архиваторов (ZIP, RAR); %\\[-13.5pt]
\item в состав электронной версии статьи должны входить: файл, содержащий текст статьи, и файл(ы),
содержащий(е) иллюстрации. %\\[-13.5pt]
\end{itemize}
\item Журнал <<Информатика и её применения>> является некоммерческим изданием. 
Плата за публикацию с авторов не взимается, гонорар авторам не выплачивается.
\end{enumerate}
\thispagestyle{empty}
\textbf{Адрес редакции:} Москва 119333,
ул.~Вавилова, д.~44, корп.~2, ИПИ РАН\\
\hphantom{\textbf{Адрес редакции:} }Тел.: +7 (499) 135-86-92\ \
Факс:  +7 (495) 930-45-05\ \  E-mail:   rust@ipiran.ru }
}

%\include{ipi-ind}

%\tableofcontents

\end{document}

%\tableofcontents

%\end{document}

%\tableofcontents


\end{document}

\newcommand{\Ack}{\subsection*{\protect\large\bf Acknowledgments}}