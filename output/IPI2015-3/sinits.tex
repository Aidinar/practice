\def\ss2{\mathop {\sum\sum}}


\def\stat{sinits}

\def\tit{АНАЛИТИЧЕСКОЕ МОДЕЛИРОВАНИЕ
РАСПРЕДЕЛЕНИЙ МЕТОДОМ
ОРТОГОНАЛЬНЫХ РАЗЛОЖЕНИЙ В~НЕЛИНЕЙНЫХ СТОХАСТИЧЕСКИХ
СИСТЕМАХ НА МНОГООБРАЗИЯХ$^*$}

\def\titkol{Аналитическое моделирование
распределений методом
ортогональных разложений в~нелинейных СтС} %стохастических системах на многообразиях}

\def\aut{И.\,Н.~Синицын$^1$}

\def\autkol{И.\,Н.~Синицын}

\titel{\tit}{\aut}{\autkol}{\titkol}

{\renewcommand{\thefootnote}{\fnsymbol{footnote}} \footnotetext[1]
{Работа выполнена при финансовой поддержке  РФФИ (проект 15--07--02244).}}


\renewcommand{\thefootnote}{\arabic{footnote}}
\footnotetext[1]{Институт проблем информатики Федерального исследовательского
центра <<Информатика и~управление>> Российской академии наук,
sinitsin@dol.ru}

\Abst{Рассматриваются вопросы оценки точности и~чувствительности алгоритмов
параметрического на базе методов ортогональных разложений (МОР) и~квазимоментов
(МКМ) аналитического моделирования одномерных распределений
в~стохастических системах (СтС) на многообразиях (МСтС) с~винеровскими
и~пуассоновскими шумами. На основе обобщенной формулы Ито выведены уравнения
точности и~чувствительности алгоритмов МОР  для МСтС. Особое внимание уделено
МКМ для нормальных МСтС. Полученные методические результаты положены в~основу
разрабатываемого инструментального символьного программного обеспечения
в~среде MATLAB-MAPLE. Рассмотрены вопросы сокращения числа уравнений МОР и~МКМ.
В~качестве иллюстративного примера изучена одномерная нелинейная МСтС
с~мультипликативным гаусссовским (нормальным) белым шумом. Для типовых задач
оценки на\-деж\-ности и~безопасности технических систем предложены алгоритмы оценки
точности и~чувствительности. Сформулированы некоторые возможные обобщения.}

\KW{метод аналитического моделирования (МАМ);
метод квазимоментов (МКМ);
метод ортогональных разложений (МОР);
обобщенная формула Ито;
плотность одномерного распределения;
полиномы Эрмита;
стохастическая система на многообразиях (МСтС);
уравнения точности МОР и~МКМ;
уравнения чувствительности МОР и МКМ}

\DOI{10.14357/19922264150302}



\vskip 14pt plus 9pt minus 6pt

\thispagestyle{headings}

\begin{multicols}{2}

\label{st\stat}



\section{Введение}


Известные методы аналитического моделирования (МАМ) распределений процессов
в~СтС, описываемых дифференциальными стохастическими
уравнениями Ито с~винеровскими и~пуассоновскими шумами, основанные на
параметризации их распределений, подробно описаны в~[1, 2].
Обобщение результатов~[1, 2] на случай многоканальных круговых
и~сферических СтС выполнено в~[3--12].\linebreak
Статья~\cite{8-sin} посвящена развитию дискретных методов параметрического
статистического и~анали\-тического моделирования в~МСтС.
В~ней рассмот\-рены уравнения МСтС, приближенные методы\linebreak
 статистического моделирования
 (МСМ) различной точности и~МАМ,
 основанные на ортогональных разложениях. Подробно развита нелинейная
 корреляционная теория МСМ и~МАМ.
В~[9--12] развиты методы и~алгоритмы аналитического моделирования гауссовских
(нормальных) процессов в~МСтС.

Применительно к типовым задачам оценки надежности и безопасности технических
систем рассмотрены вопросы оценки точности и чувствительности параметрических
алгоритмов МАМ одномерных распределений процессов по 
МОР и~МКМ в~нелинейных негауссовских дифференциальных МСтС.

Статья включает в себя: введение, 4~раздела, заключение и список литературы.
В~разд.~2 получены уравнения МОР (теорема~1). Уравнения МКМ как частный случай
получены из теоремы~1 при использовании многомерных полиномов Эрмита
и~со\-став\-ля\-ют основу теоремы~2 разд.~3. В~разд.~4 получены уравнения точности
и~чувствительности МОР и~МКМ. Раздел~5 содержит иллюстративный пример.
В~заключении кратко сформулированы основные результаты и указаны некоторые
возможные обобщения.

\section{Метод ортогональных разложений}

Как известно~\cite{1-sin, 8-sin}, для
дифференциальных  СтС в
конечномерных пространствах, в том числе и~МСтС, используется
дифференциальное стохастическое уравнение Ито вида

\noindent
    \begin{multline}
    dY_t= a (Y_t,\Theta,t) \,dt + b(Y_t,\Theta,t)\, d W_0 +{}\\
    {}+
    \int\limits_{R_0^q} c(Y_t,\Theta,t,v)\, dP^0 (\Theta, t,
    dv)\,.\label{e1-sin}
\end{multline}
Здесь $Y_t$~--- $p$-мер\-ный вектор состояния, $Y_t\hm\in \Delta^y$
($\Delta^y$~--- многообразие состояний); $\Theta$~--- вектор  случайных
параметров размерности~$p^\Theta$;\linebreak
$a\hm=a (y_t,\Theta,t)$ и~$b\hm= b(y_t,\Theta,t)$~---
известные $(p\times 1)$-мер\-ная и~$(p\times r)$-мер\-ная
функции вектора~$Y_t$\linebreak и~времени~$t$; $W_0\hm= W_0(\Theta, t)$~---
$r$-мер\-ный винеровский стохастический процесс интенсивности $\nu_0\hm=
\nu_0(\Theta,t)$; $c(y_t,\Theta, t,v)$~--- $(p\times 1)$-мер\-ная функция~$y_t$,
$t$ и~вспомогательного
$(q\times 1)$-мер\-но\-го параметра~$v$; $\int\limits d P^0 (\Theta, t,A)$~---
центрированная пуассоновская мера:
    $$
    \int\limits_{\Delta} d P^0 (\Theta, t,A)=
    \int\limits_{\Delta} d P (\Theta, t,A)-
    \int\limits_{\Delta} \nu_P (\Theta, t,A) \,dt\,,
    $$
где $\int\limits_{\Delta} d P (\Theta, t,A)$~--- число скачков пуассоновского
процесса в интервале времени~$\Delta$; $\nu_P (\Theta, t,A)$~--- интенсивность
пуассоновского процесса $P(t,A)$;
$A$~--- некоторое борелевское множество пространства~$R^q_0$ с~выколотым началом координат.
Интеграл~(\ref{e1-sin}) в~общем случае распространяется на все пространство~$R_0^q$
с~выколотым началом координат.
Начальное значение~$Y_0$ вектора~$Y_t$ представляет
собой случайную величину, не зависящую от приращений винеровского
процесса $W_0(\Theta, t)$ и~пуассоновского процесса $P(\Theta, t,A)$ на интервалах
времени $\Delta\hm= (t_1, t_2]$, следующих за~$t_0$, $t_0\le t_1\le t_2$,
для любого множества~$A$.


Для вычисления вероятностей событий,
связанных со случайными функциями, в прикладных задачах достаточно
знания многомерных распределений. Поэтому центральной
задачей теории МСтС
является задача вероятностного анализа одномерных
распределений процессов, удовлетворяющих дифференциальному стохастическому
уравнению Ито~(\ref{e1-sin}) при
соответствующих начальных условиях.
В~теории МСтС различают два принципиально разных подхода
к вычислению распределений. Первый общий подход основан на
статистическом моделировании, т.\,е.\ на прямом численном решении~(\ref{e1-sin})
с~последующей статистической обработкой результатов. Второй общий
подход основан на теории непрерывных марковских процессов и
предполагает аналитическое моделирование, т.\,е.\ решение
детерминированных уравнений в функциональных пространствах
(уравнений Фок\-ке\-ра--План\-ка--Кол\-мо\-го\-ро\-ва, Фел\-ле\-ра--Кол\-мо\-го\-ро\-ва,
Пугачёва и~др.)\ для одномерных распределений.
В~практических задачах часто используют и комбинированные методы. При
этом будем предполагать, что существуют  одномерные
плотности процессов в МСтС~(\ref{e1-sin}). Достаточные условиях их
существования можно найти, например, в~\cite{13-sin}.


Будем полагать, что, во-пер\-вых, одномерные плотности распределений существуют
и,~во-вто\-рых, плотности можно параметризовать
с~по\-мощью условных параметров $\vartheta^y \hm=\vartheta^y(\Theta, t)$:
вероятностных моментов, квазимоментов, семиинвариантов, коэффициентов
ортогонального разложения плот\-ности и~др.~\cite{1-sin, 2-sin}.

Следуя~\cite{2-sin}, представим уравнения МОР
 в~виде следующего отрезка разложения:
    \begin{multline}
    f=f(y_t, \Theta, \vartheta^y, t) \approx {}\\
    {}\approx w(y_t, \Theta) \left[ 1 +
    \sss^n_{l=3}\sss_{|\nu |=l} c_{\nu t} (\Theta, t)p_\nu (y_t)\right].
    \label{e2-sin}
    \end{multline}
Здесь  $w=w(y_t; \Theta)$~--- эталонная одномерная плотность,
выбираемая из условия совпадения первых двух вероятностных моментов для~$w$ и~$f$;
$\{p_\nu (y_t) , q_\nu (y_t)\}$~--- система биортонормальных полиномов с~весом~$w$,
т.\,е.\ удовлетворяющих\linebreak условию
   \begin{equation}
    {\mathsf M}^{w}_{\Delta^y}
    \left[ p_\nu (Y_t) q_\mu (Y_t)\right] = \delta_{\nu\mu} =\begin{cases}
    0 &\ \mbox{при}\ \mu\ne \nu\,;\\
    1 &\ \mbox{при}\ \mu=\nu\,,
    \end{cases}
\label{e3-sin}
\end{equation}
где ${\mathsf M}_{\Delta^y}^w$~--- символ математического ожидания
в~об\-ласти~$\Delta^y$; $\vartheta^y \hm= \lf m_t, K_t, c_{\nu t}\rf$~---
условные па\-ра\-мет\-ры МОР, т.\,е.\ вектор условного математического ожидания
$m_t\hm= m(\Theta, t)$, условная ковариационная матрица $K_t\hm=K(\Theta, t)$
и~условная матрица коэффициентов ортогонального разложения
$c_{\nu t}\hm= c_{\nu t}(\Theta, t)$, удовлетворяющая условию
   \begin{equation}
    c_{\nu t} = {\sf M}_{\Delta^y}^f \left[ q_\nu (Y_t)\right] =
    q_\nu (\alp)\,,\label{e4-sin}
    \end{equation}
где $q_\nu (\alp)$  представляет собой комбинацию условных начальных
моментов~$\alp_t$, полученную из $q_\nu (x)$ заменой всех одночленов
$x_1^{k_1}, \ldots , x_r^{k_r}$ соответствующими начальными вероятностными
моментами $\alp_{k_1}, \ldots,  \alp_{k_r}$.

Для биортогональной системы полиномов условие~(\ref{e3-sin})
выполняется только для $\mu\hm\ne \nu$. Всякая биортогональная система
приводится к биортонормальной путем деления полиномов~$p_\nu$ и~$q_\nu$ на
множители~$\xi_\nu$ и~$\eta_\nu$, произведение которых равно интегралу~(\ref{e3-sin})
при соответствующем~$\nu$ и~$\mu\hm=\nu$. Очевидно, что при каждом~$\nu$
один из множителей $\xi_\nu,\eta_\nu$ может быть выбран произвольно.

Как известно~\cite{1-sin, 2-sin}, существование всех вероятностных моментов
для плот\-ности необходимо и~достаточно для существования интегралов~(\ref{e3-sin}).

Будем пользоваться векторной нумерацией полиномов
$\lf p_\nu, q_\nu\rf$, так чтобы сумма координат
$|\nu | \hm= \nu_1 +\cdots +\nu_r$ векторного индекса
$\nu \hm= [\nu_1 \cdots \nu_r]^{\mathrm{T}}$ была равна степени полиномов.
Тогда число линейно независимых полиномов данной степени $\nu_* \hm= |\nu |$
будет равно числу независимых одночленов степени~$\nu_*$, т.\,е.\
$C_{r+\nu_*-1}^{\nu_*}$.

Между начальными  моментами~$\alp_{\nu t}$, семиинвариантами~$\kappa_{\nu t}$
и~коэффициентами ортогонального разложения~$c_{\nu t}$
существуют следующие формулы связи~\cite{1-sin, 2-sin}:
    \begin{gather}
    \!\!\alp_{l_1, \ldots,  l_r, t} =
    \alp_{l_1, \ldots , l_r, t}^w + \sss_{l=1}^n
    \sss_{|\nu |=l} c_{\nu t} p_{\nu, l_1,\ldots,  l_r} (\alp^w)\,;\notag
%\label{e5-sin}
    \\
    p_{\nu, l_1, \ldots , l_r} (y_t) = y_{1t}^{l_1} \ldots y_{rt}^{l_r}
    p_\nu (y_t)\,,
   \label{e6-sin}
    \end{gather}
где
   \begin{equation*}
    \fr{c_{\nu t}}{\nu_1 ! \cdots \nu_r !} =
    \sss_{h=1}^{[|\nu |/3]} \fr{1}{H!}\,\fr{\kappa_{q_{1t}} \cdots \kappa_{q_{ht}}}
    {q_{11} ! \cdots q_{hr} ! }\enskip (|\nu | = 6,7,\ldots).
%    \label{e7-sin}
    \end{equation*}
Здесь  $p_{\nu,l_1, \ldots ,  l_r} (\alp^w)$ получается из
$p_{\nu,l_1,\ldots , l_r} (y_t)$  так же, как
$q_\nu (\alp)$  и~$q_\nu (y_t)$; $l_1,\ldots , l_r \hm= 0,1,\ldots , n$;
$l_1+\cdots+l_r\hm=
3, \ldots , n$; $ q_h \hm= \lk q_{h1}\cdots q_{hr}\rk^{\mathrm{T}}$;
$\nu \hm=\lk \nu_1 \cdots \nu_r\rk^{\mathrm{T}}$.

Пользуясь известной обобщенной формулой Ито~\cite{1-sin, 2-sin, 13-sin}
для дифференциала нелинейной функции $\vrp (Y_t , \Theta, t)$:
   \begin{multline*}
    d\vrp \left(Y_t, \Theta, t\right) ={}\\[2pt]
{}=
    \left\{ \vphantom{\fr{1}{2}}\vrp_t \left(Y_t, \Theta, t\right)+
    \vrp_Y \left(Y_t, \Theta, t\right)^{\mathrm{T}}
    a \left(Y_t, \Theta, t\right)+\fr{1}{2}\times{}\right.\\[2pt]
\left.\hspace*{-3.5pt}\times   \mathrm{tr} \!\left[
\vrp_{YY} \!\left(
Y_t, \Theta, t\right) b\left(Y_t, \Theta, t\right)\nu_0(\Theta, t)
b\left(Y_t, \Theta, t\right)^{\mathrm{T}}
\right]\!\right\}\!  dt+\hspace*{-1.33pt}\\[2pt]
{}+
\iii_{R_0^q} \left[ \vrp  \left(
\vphantom{\left(Y_t\right)^{\mathrm{T}}}
Y_t + c (Y_t, \Theta, t,v) -{}\right.\right.
\vrp \left(Y_t, \Theta, t\right)-{}\\[2pt]
\left.{}- \vrp \left(Y_t, \Theta, t\right)^{\mathrm{T}}
c \left(Y_t, \Theta, t,v\right),t \right) - \vrp \left(Y_t, \Theta, t\right)-{}\\[2pt]
\left.{}- \vrp_Y (Y_t, \Theta, t)^{\mathrm{T}}
c\left(Y_t, \Theta, t,v\right)
\vphantom{\left(Y_t\right)^{\mathrm{T}}}
\right] \nu_P (\Theta, dt, dv)+{}\\[2pt]
{}+
\vrp_Y(Y_t, \Theta, t)^{\mathrm{T}} b \left(Y_t, \Theta, t\right) dW_0 (\Theta, t) +{}\\[2pt]
{}+\iii_{R_0^q} \left[ \vrp\left(Y_t,+c\left(Y_t, \Theta, t,v\right), \Theta, t
\right) -{}\right.\\[-4pt]
\left.{}-
\vrp \left(Y_t, \Theta, t\right)\right] P_0 (\Theta, dt,dv)\,,
%\label{e8-sin}
\end{multline*}
получаем, что условные
параметры МОР для МСтС~(\ref{e1-sin}) удовлетворяют следующим
обыкновенным дифференциальным уравнениям:

\noindent
\begin{equation}
\left.
\begin{array}{l}
    \dot m_t =\vrp_{10} \left(m_t, K_t, \Theta, t\right) +{}\\[6pt]
    \hspace*{10mm}{}+
    \displaystyle\sss_{l=3}^n \sss_{| \nu | =l} \vrp_{1\nu}\left(m_t, K_t, \Theta, t\right)
    c_{\nu t}\,,\\[6pt]
    \hspace*{40mm}m\left(t_0\right) = m_0\,;
 \\[6pt]
 \dot K_t =\vrp_{20}\left(m_t, K_t, \Theta, t\right)+{}\\[6pt]
             \hspace*{10mm} \displaystyle {}+ \sss_{l=3}^n
    \sss_{| \nu | =l}  \vrp_{2\nu} \left(m_t, K_t, \Theta, t\right)c_{\nu t}\,, \\[6pt]
    \hspace*{40mm}K(t_0)= K_0\,;\\[6pt]
    \dot c_{\kappa t} =\vrp_{\kappa 0} (m_t, K_t, \Theta, t)+{}\\[6pt]
        \hspace*{8mm}{}+    \psi_{\kappa 0}^{m_t} (m_t, K_t, \Theta, t)^{\mathrm{T}} \dot m_t +{}\\[6pt]
        \hspace*{10mm}{}+ \fr{1}{2} \,{\rm tr}\, \left[ \psi_{\kappa 0}^{K_t} (m_t, K_t, \Theta, t)
    \dot K_t\right]+{}\\[6pt]
        \hspace*{11mm}{}+ \fr{1}{ 2} \,\mathrm{tr}\,
    \left[ \psi_{\kappa \nu}^{K_t} (m_t, K_t, \Theta, t) \dot K_t\right]+{}\\[6pt]
        \hspace*{12mm}\displaystyle{}+\sss_{l=3}^n \sss_{| \nu | =l} c_{\nu t} \left[ \vrp_{\kappa \nu}
\left(m_t, K_t, \Theta, t\right) +{}\right.\\[6pt]
\left.{}+\psi_{\kappa \nu}^{m_t}(m_t, K_t, \Theta, t)^t
\dot m_t\right]\,,\enskip c_{\kappa} (t_0) = c_{\kappa 0}\,.
\end{array}
\right\}
    \label{e9-sin}
    \end{equation}
Здесь приняты следующие обозначения:
    \begin{equation}
    \left.
    \begin{array}{rl}
    \vrp_{10} \left(m_t, K_t, \Theta, t\right)&={\sf M}_{\Delta^y}^w
    \left[ a \left(Y_t, \Theta, t\right)\right]\,;\\[6pt]
    \vrp_{1\nu} \left(m_t, K_t, \Theta, t\right)&= {\sf M}_{\Delta^y}^{w p_\nu}
    \left[ a \left(Y_t , \Theta, t\right)\right]\,;
    \end{array}
    \right\}
    \label{e10-sin}
    \end{equation}
    \begin{equation}
    \left.
    \begin{array}{l}
       \vrp_{20} \left(m_t, K_t, \Theta, t\right)={}\\[6pt]
       \hspace*{10mm}{}={\sf M}_{\Delta^y}^w
    \left[ a \left(Y_t, \Theta, t\right) \left(Y_t - m_y\right)^{\mathrm{T}}+{}\right.\\[6pt]
\left.    {}+
    \left(Y_t - m_t\right) a \left(Y_t, \Theta, t\right)^{\mathrm{T}} +
    \bar\sigma \left(Y_t, \Theta, t\right)\right]\,;
\\[6pt]
    \vrp_{2\nu} \left(m_t, K_t, \Theta, t\right)={}\\[6pt]
    \hspace*{10mm}{}={\sf M}_{\Delta^y}^{wp_\nu}
    \left[ a \left(Y_t, \Theta, t\right)\left(Y_t-m_t\right)^{\mathrm{T}} +{}\right.\\[6pt]
\left.    {}+
    \left(Y_t- m_t\right) a \left(Y_t, \Theta, t\right)^{\mathrm{T}} +
    \bar\sigma \left(Y_t, \Theta, t\right)\right]\,;
    \end{array}
    \right\}
    \label{e11-sin}
    \end{equation}
    \begin{equation}
    \left.
    \begin{array}{l}
    \bar\sigma \left(Y_t, \Theta, t\right)= \sigma\left(Y_t, \Theta, t\right)+{}\\[6pt]
    {}+
    \displaystyle\iii_{R_0^q}c\left(Y_t, \Theta, t,v\right)
    c \left(Y_t, \Theta, t,v\right)^{\mathrm{T}} \nu_P (t, \Theta)\, dv\,;
\\[6pt]
    \sigma \left(Y_t, \Theta, t\right)= b \left(Y_t, \Theta, t\right)\nu_0
    (\Theta, t) b \left(Y_t, \Theta, t\right)^{\mathrm{T}}\,;
    \end{array}
    \right\}
    \label{e12-sin}
    \end{equation}

    \vspace*{-12pt}

    \noindent
\begin{multline}
    \vrp_{\kappa \nu}\left( m_t, K_t, \Theta, t\right) ={}\\
    {}=
    {\sf M}_{\Delta^y}^{wp_\nu} \left\{
    \fr{\prt^{\mathrm{T}} q_\kappa (Y_t)}{\prt Y_t} a \left( Y_t, \Theta, t\right) +{}\right.\\
    {}+
    \fr {1}{2}\,\mathrm{tr} \left[ \fr{\prt}{\prt Y_t} \,
    \fr{\prt^{\mathrm{T}}}{\prt Y_t} q_\kappa \left(Y_t\right) \sigma \left( Y_t, \Theta, t\right)
    \right]+{}\\
{}+ \iii_{R_0^q} \left[ \vphantom{\fr{\prt}{\prt Y_t}}
q_\kappa \left(Y_t + c\left( Y_t, \Theta, t,v\right)\right) -
q_\kappa\left( Y_t\right) - \right.\\
\left.\left.{}-\fr{\prt q_\kappa (Y_t)}{\prt Y_t} c
\left( Y_t, \Theta, t,v\right)\right] \nu_P (\Theta, t, dv)\right\};\label{e13-sin}
\end{multline}
    \begin{equation}
    \left.
    \begin{array}{rl}
    \psi_{\kappa\nu}^{m_t} \left( m_t, K_t, \Theta, t\right) &= {\sf M}_{\Delta^y}^{wp_\nu}
    \left[ q_\kappa^{m_t} (Y_t)\right]\,;\\[6pt]
    \psi_{\kappa\nu}^{K_t} \left( m_t, K_t, \Theta, t\right)&=
    {\sf M}_{\Delta^y}^{wp_\nu} \left[ q_\kappa^{K_t} (Y_t)\right]\,,
    \end{array}
    \right\}
    \label{e14-sin}
    \end{equation}
а интегралы $\psi_{\kappa 0}^{m_t} ( m_t, K_t, \Theta, t)$
и~$\psi_{\kappa 0}^{K_t} ( m_t, K_t, \Theta, t)$ выража\-ются согласно~(\ref{e14-sin})
при $p_\nu (y_t) \hm= p_0 (y_t) =1$.

Таким образом, \textit{если существует одномерное распределение процесса~$Y_t$
в~МСтС}~(\ref{e1-sin}), \textit{то при фиксированном векторе параметров~$\Theta$
и~полиномиальных  $\lf p_\nu (y_t), q_\nu (y_t)\rf$ уравнения}~(\ref{e2-sin}),
(\ref{e9-sin}) \textit{при условиях}~(\ref{e3-sin}), (\ref{e4-sin})
\textit{и~конечности интегралов}~(\ref{e10-sin})--(\ref{e14-sin})
\textit{лежат в основе алгоритма МОР} (\textbf{теорема~1}).

\smallskip

Уравнения теоремы~1 для $\dot m_t$ и~$\dot K_t$ линейны относительно~$c_{\kappa t}$,
 в~то время как уравнения для $\dot c_{\kappa t}$  в~силу~(\ref{e4-sin}) нелинейны.

Отметим, что для рассмотрения стационарных распределений
с~параметрами $\vartheta^{*y} \hm= \lf m^*; K^*; c_\nu^*\rf$ достаточно
правые части уравнений~(\ref{e6-sin}) приравнять к нулю.


Основной трудностью практического применения МОР для многомерных
дифференциальных МСтС является быстрый рост числа уравнений для $c_{\nu t}$
с~увеличением размерности~$p$ вектора состояния~$Y_t$.
В~\cite{1-sin, 2-sin} содержатся таблицы, отражающие эти закономерности.
Например, для МСтС размерности  $p\hm=10$, порядка учитываемого момента $N\hm=4$
и~отрезка разложения одномерной плотности $n\hm=4$ количество  уравнений
составляет $Q^{\mathrm{МОР}}\hm=85$.

Описанные в~\cite{1-sin, 2-sin} универсальные методы сокращения числа
уравнений МОР, основанные на использовании метода С.\,В.~Мальчикова
и~его обобщений, мо\-мент\-но-се\-ми\-ин\-ва\-ри\-ант\-ных соотношений,
соответствующего ряда Эджуорта, а~также их комбинации могут быть использованы
и~для МСтС~(\ref{e1-sin}).
В~тех случаях, когда известна аналитическая природа нелинейной задачи,
применяют метод нормальных координат или непосредственно в~уравнениях
МСтС~(\ref{e1-sin}), или в~уравнениях~(\ref{e9-sin}).

\section{Метод квазимоментов}

При аппроксимации одномерной плотности~$f$ отрезком разложения по многомерным
полиномам Эрмита $\lf H_\nu , G_\nu\rf$ имеем согласно~\cite{1-sin, 2-sin}
\begin{gather}
\left.
\begin{array}{rl}
p_\nu (y_t) &= \fr{H_\nu \left(y_t-m_t\right)}{\nu_1 ! \cdots \nu_p !};\\[6pt]
q_\nu (y_t )&= G_\nu \left(y_t-m_t\right);\!\!
\end{array}
\right\}
\label{e15-sin}
\\
c_{\nu t} = q_\nu (\alp) = G_\nu(\mu)\,;\label{e16-sin}
\\
q_\kappa^{m_t} =0\,;\enskip q_\kappa^{K_t} (\alp)=0 \enskip
( | \kappa |=3)\,;\label{e17-sin}
\\
    \left.
    \begin{array}{rl}
        \hspace*{-4pt}q_{\kappa r}^{m_t} &= -\kappa_r c_{\kappa - e_r, t} \enskip
    (r=\overline{1,p}\,;\ |\kappa | = 4, \ldots , N)\,;\\[6pt]
    \hspace*{-4pt}q_{\kappa rr} &=-\kappa_r \kappa_s c_{\kappa - e_r - e_s ,t}\\[6pt]
 &(r,s = \overline{1,p};\ s>r;\enskip |\kappa | = 5, \ldots , N).
 \end{array}\!\!
 \right\}
 \label{e18-sin}
    \end{gather}

Таким образом, \textit{в основе МКМ для МСтС}~(\ref{e1-sin})
\textit{лежат уравнения теоремы~$1$ при условиях}~(\ref{e15-sin})--(\ref{e18-sin})
(\textbf{теорема~2}).

\vspace*{-6pt}

\section{Оценка точности и~чувствительности методов ортогональных разложений
и~квазимоментов}


В задачах надежности и безопасности технических систем [14--16] для оценки точности 
МОР и~МКМ можно, следуя~[1, 2], применить метод сравнения вероятностей попадания 
на множества определенного класса или метод оценки вероятностных моментов четвертого 
порядка.

Метод теории чувствительности в инженерной практике
широко применяется для приближенного анализа точности СтС со
случайными параметрами~$\Theta$ в предположении малых дисперсий этих
параметров по сравнению с их математическими ожиданиями~\cite{17-sin, 18-sin}.
Применяя описанный в~разд.~2 МОР при фиксированных значениях~$\Theta$,
найдем условные параметры $\vartheta^y \hm=\lf m_t; K_t; c_{\nu t} \rf$
одномерной плот\-ности. Уравнения функций чувствительности условных
параметров~$\vartheta^y$ для МОМ получаются путем дифференцирования правых
и~левых частей уравнений теоремы~1 по~$\Theta$. В~этом случае получаются
следующие уравнения для $\nabla^\Theta m_t \left( \prt /\prt \Theta\right)$
и~$\nabla^\Theta K_t \hm= \left( \prt / \prt \Theta\right) K_t$:
  \begin{equation}
  \left.
  \begin{array}{rl}
    \nabla^\Theta \dot m_t &=\nabla^\Theta \vrp_{10} +
    \displaystyle\sss_{l=3}^n \sss_{|\nu |=l} \left( \nabla^\Theta  
    \vrp_{1\nu} m_t^{c_\nu} +{}\right.\\[6pt]
&\left.{}+    \vrp_{1\nu} \nabla^\Theta  c_{\nu t}\right)\,, \enskip
 \nabla^\Theta  m (t_0) =0\,;
\\[6pt]
    \nabla^\Theta \dot K_t &= \nabla^\Theta  \vrp_{20} +
       \displaystyle \sss_{l=3}^n \sss_{|\nu  |=l} \left(
       \nabla^\Theta \vrp_{2\nu} +{}\right.\\[6pt]
       &\left.{}+\vrp_{2\nu}
    \nabla^\Theta  c_{\nu t}\right)\,,\enskip 
\nabla^\Theta K(t_0) =0\,;
\\[6pt]
\nabla^\Theta  \dot c_{\kappa t}& = \nabla^\Theta  \vrp_{\kappa 0} +{}\\[6pt]
&{}+\nabla^\Theta \left\{ \psi_{\kappa 0}^{m_t} \dot m_t + \fr{1}{2}\,\mathrm{tr}
\left ( \psi_{\kappa 0}^{m_t}\dot K_t\right) +{}\right.\\[6pt]
&\left.{}+\fr{1}{2}\,\mathrm{tr}
\left(\psi_{\kappa \nu}^{K_t} \dot K_t\right) \right\} + {}\\[6pt]
&    \hspace*{-10mm}\displaystyle{}+ \sss_{l=3}^n \sss_{|\nu | =l} \left\{
\nabla^\Theta  c_{\nu t} \left[ \vrp_{\kappa\nu} +
\left(\psi_{\kappa\nu}^{m_t}\right)^{\mathrm{T}} \dot m_t\right] +{}\right.\\[6pt]
&\hspace*{-10mm}\left.{}+
c_{\nu t} \nabla^\Theta  \left[ (\vrp_{\kappa \nu}^{m_t})^{\mathrm{T}} \dot m_t
\vphantom{\vrp_{\kappa\nu} +
\left(\psi_{\kappa\nu}^{m_t}\right)^{\mathrm{T}} \dot m_t}
\right]
\right\}\,,\enskip \nabla^\Theta c_\kappa \left(t_0\right)=0\,.
\end{array}
\right\}
\label{e19-sin}
\end{equation}


При дифференцировании $\vartheta^y$ по~$\Theta$ порядок уравнений
возрастает пропорционально числу производных.

Аналогично выписываются уравнения для $\nabla^\Theta (\nabla^\Theta )^{\mathrm{T}} m_t$,
$\nabla^\Theta (\nabla^\Theta )^{\mathrm{T}} K_t$ 
и~$\nabla^\Theta (\nabla^\Theta )^{\mathrm{T}} c_{\nu t}$.


Таким образом, \textit{в условиях теоремы~$1$ уравнения алгоритма чувствительности
МОР имеют вид}~(\ref{e19-sin}) (\textbf{теорема~3}),
\textit{а~в~условиях теоремы~$2$ уравнения алгоритма чувствительности
МКМ имеют вид}~(\ref{e19-sin}) \textit{при условиях}~(\ref{e15-sin})--(\ref{e18-sin})
(\textbf{теорема~4}).

\smallskip

Наряду с частными функциями чувствитель\-ности часто используют
обобщенные функции чувствительности~\cite{17-sin, 18-sin}.
Допустим, что для оценки качества МСтС при гауссовских~$\Theta$ выбрана
условная функция потерь~$\rho$, допускающая квадратическую аппроксимацию

\vspace*{-3pt}

\noindent
\begin{multline*}
    \rho = \rho(\Theta) = \rho \left(m^\Theta\right) +
    \sss_{i=1}^{p^\Theta} \rho_i' \left(m^\Theta\right) \Theta_i^0 +{}\\
    {}+
    \ss2_{i, j =1}^{p^\Theta} \rho_{ij}'' \left(m^\Theta \right)
    \Theta_i^0\Theta_j^0\,.
    %\label{e14-1-sin}
    \end{multline*}
Здесь $\rho(m^\Theta)$~--- значение функции потерь при $\Theta\hm=
m^\Theta$; $ \rho_i' \hm=\rho_i' (m^\Theta)$  и~$\rho_{ij}'' \hm=
\rho_{ij}'' (m^\Theta)$~--- вектор и~матрица первых и~вторых функций
чувст\-ви\-тель\-ности; $p^\Theta$~--- число параметров~$\Theta$;
$\Theta_i^0 \hm=\Theta_i \hm- m_i^\Theta$~--- случайные гауссовские
отклонения  параметров~$\Theta$ с~математическим ожиданием $m^\Theta
\hm= \lk m_1^\Theta \cdots m_{p}^\Theta\rk^{\mathrm{T}}$ и~ковариационной матрицей
$K^\Theta \hm= \lk K_{ij}^\Theta\rk$. Тогда применяют следующие оценки~\cite{17-sin, 18-sin}:
\begin{equation}
    \eps = \eps_2^{1/4},\enskip
    \eps_2 ={\sf M}^N \left[ \rho (\Theta)\right]^2 - \rho
    \left(m^\Theta\right)^2,\label{e21-sin}
    \end{equation}
где
\begin{equation}
{\sf M}^N \left[ \rho(\Theta)\right] > \rho (m^\Theta)\,;\label{e22-sin}
\end{equation}

\vspace*{-12pt}

\noindent
\begin{multline*}
{\sf M}^N \left[ \rho(\Theta)^2\right] =
\rho \left(m^\Theta\right)^2 + \rho' \left(m^\Theta\right)^{\mathrm{T}} K^\Theta \rho'
\left(m^\Theta\right) + {}\\[-1pt]
{}+2 \rho \left(m^\Theta\right) \mathrm{tr} \left[
\rho'' \left(m^\Theta\right) K^\Theta\right] +{}\\[-1pt]
{}+ \left\{ \mathrm{tr} \left[ \rho '' \left(m^\Theta\right) K^\Theta\right]
\right\}^2 + 2 \mathrm{tr} \left[ \rho'' \left(m^\Theta\right) K^\Theta\right]^2.
%\label{e23-sin}
\end{multline*}

Соотношение~(\ref{e22-sin}) показывает, что случайный разброс параметров~$\Theta$
ухудшает качество МСтС, причем количественная оценка ухудшения качества
проводится согласно~(\ref{e21-sin}).

При известной функции потерь~$\rho$ для вычисления функций
чувствительности  $\rho_i'$ и~$\rho_{ij}''$ применяются известные численные
методы аппроксимации в~точке, в~ряде точек и~в~об\-ласти~\cite{17-sin}.

Для задач оценки надежности и безопасности типовые бейесовские критерии~$\rho$
в~виде условного и~среднего риска приведены в~\cite{17-sin, 18-sin}.

\vspace*{-8pt}

\section{Пример}

\vspace*{-2pt}

Рассмотрим СтС вида~\cite{1-sin, 2-sin}

\vspace*{2pt}

\noindent
\begin{equation*}
\dot Y_t =-Y_t^3 + Y_t V(\Theta)\,,\enskip Y\left(t_0\right)=Y_0\,,
%\label{e24-sin}
\end{equation*}
где $V=V(\Theta,t)$~--- гауссовский белый шум, зависящий от
случайного параметра~$\Theta$ интенсивности\linebreak\vspace*{-12pt}

\columnbreak

\noindent
 $\nu \hm= \nu (\Theta)$.
Уравнения МОР с учетом вероятностных моментов четвертого порядка
$N\hm=4$ имеют следующий вид:

\noindent
    \begin{equation}
    \left.
    \begin{array}{l}
    \hspace*{-3mm}\dot m_t = - m_t \left(m_t^2 + 3 D_t\right) - c_{3t}\,,\enskip
    m\left(t_0\right) = m_0\,;
\\[4pt]
        \hspace*{-3mm}\dot D_t = \left[ \nu(\Theta) - 6 D_t\right] \left(m_t^2 + D_t\right) -
    6 m_t c_{3t} -{}
\\[4pt]
\hspace*{22mm} {}-2 c_{yt}\,,\enskip D\left(t_0\right) = D_0\,;
\\[4pt]
        \hspace*{-3mm}\dot c_{3t} = 6\left[ \nu (\Theta) - 3 D_t\right] m_t D_t +{}\\[4pt]
\hspace*{2mm}{}+\;3 \left[ \nu(\Theta) - 3 m_t^2 - 9 D_t \right] c_{3t} - 9 m_t c_{yt}\,,\\[4pt]
\hspace{37mm}c_3 \left(t_0\right) = c_{30}\,;\\[4pt]
        \hspace*{-3mm}\dot c_{4t} = 6\left[ \nu(\Theta) + 2 m_t^2\right] D_t^2 +
    36 D_t^3 + 4 c_{3t}^2 +{}\\[4pt]
\hspace*{17mm}{}+ 12 \left[ \nu (\Theta) - 6 D_t\right] m_t c_{3t} +{}\\[6pt]
{}+ 6 \left[ \nu(\Theta) - 2 m_t^2 - 8 D_t \right]
c_{4t}\,,\enskip c_4 \left(t_0\right) = c_{4}\,.
\end{array}
\right\}
\label{e25-sin}
\end{equation}
Дифференцируя уравнения~(\ref{e25-sin}) по параметру $\nu\hm = \nu(\Theta)$,
получим систему уравнений для производных от  $m_t$, $D_t$, $c_{3t}$ и~$c_{4t}$
по~$\nu$ при известных
$m_t^*$, $D_t^*$, $c_{3t}^*$ и~$c_{4t}^*$:

\noindent
    \begin{equation}
    \left.
    \begin{array}{l}
    \hspace*{-1mm}\nabla\dot m_t = -3\left( m_t^{*2} +  D_t^*\right) -
    3 m_t^* \nabla D_t -\nabla c_{3t}\,,\\[6pt]
    \hspace*{29mm}\nabla c_3 (t_0) =0\,;\\[6pt]
        \hspace*{-1mm}\nabla \dot D_t = m_t^{*2} + D_t^*+ 2 \left[ m_t^* \left(\nu - 6 D_t^*\right) -{}\right.\\[6pt]
\left.    \hspace*{-1mm}{}-    3 c_{3t}^*\right] \nabla m_t + 
\left[ \nu - 6 D^*_t - 6 \left( m_t^{*2} + D_t^*\right) \right]\times{}\\[6pt]
    \hspace*{-1mm}{}\times \nabla D_t -
6 m_t^* \nabla c_{2t} - 2 \nabla c_{4t}\,,\enskip
\nabla D \left(t_0\right) =0\,;
\\[6pt]
        \hspace*{-1mm}\nabla \dot c_{3t} = 3\left( 2 m_t^* D_t^* + c_{3t}^*\right) +{}\\[6pt]
        \hspace*{-1mm}{}+
    \left[ 6 \left(\nu - 3 D_t^*\right) D_t^* - 18 m_t^* c_{3t}^* - 9 c_{4t}^* \right]
     \nabla m_t+{}\\[6pt]
    \hspace*{-1mm}{}+ 3 \left[ - 6 m_t^* D_t^* + 2 \left(\nu - 3 D_t^*\right) m_t^* -
9 c_{3t}^*\right]\times{}\\[6pt]
{}\times\nabla D_t+ 3 \left(\nu - 3 m_t^{*2} - 9 D_t^*\right) \nabla c_{3t} -{}\\[6pt]
    \hspace*{-1mm}{}-9 m_t^* \nabla c_{4t}m\,,\enskip
\nabla c_4 \left(t_0\right) =0\,;
\\[6pt]
        \hspace*{-1mm}\nabla \dot c_{4t} = 6 \left( D_t^{*2} + 2 m_t^* c_{3t}^* + c_{4t}^*\right) +{}\\[6pt]
        \hspace*{-1mm}{}+
    12 \left[ m_t^* D_t^* + \left(\nu - 6 D_t^*\right) c_{3t}^* \right] \nabla m_t+{}\\[6pt]
    \hspace*{-1mm}{}+4 \left[ 3 \left(\nu + 2 m_t^{*2} \right) D_t^* + 27 D_t^{*2} - 18 m_t^* c_{3t}* -{}\right.\\[6pt]
\left.    \hspace*{-1mm}{}-
12 c_{4t}^* \right] \nabla D_t + 4 \left[
2 c_{3t}^* + 3 \left(\nu - 6 D_t^*\right) m_t^* \right]\times{}\\[6pt]
    \hspace*{-1mm}{}\times \nabla c_{3t} +6 \left(\nu - 2 m_t^{*2} - 8 D_t^*\right) \nabla c_{4t}\,,\\[6pt]
\hspace*{40mm}\nabla c_4 \left(t_0\right) =0\,.
\end{array}\!
\right\}\!\!\!
\label{e26-sin}
\end{equation}

Анализ уравнений~(\ref{e26-sin}) показывает:
\begin{enumerate}[(1)]
\item уравнения для  $\nabla \dot m_t$ однородны и линейны
относительно $\nabla m_t, \nabla D_t$ и $\nabla c_{3t}$;

\item  уравнения для  $\nabla \dot D_t$ неоднородны
и~линейны относительно всех переменных;

\item  уравнения для  $\nabla \dot c_{3t}$ и~$\nabla \dot c_{4t}$
неоднородны и~линейны относительно всех переменных.
\end{enumerate}

Для стационарного случая надо приравнять правые части~(\ref{e25-sin}) к нулю.

\vspace*{-6pt}

\section{Заключение}

Для нелинейных дифференциальных стохастических систем, в~том числе
и~на многообразиях, понимаемых в~смысле Ито, разработаны  методы параметрического
аналитического моделирования точности и~чувствительности. Предполагается, что
в~качестве параметров одномерных распределений выбраны коэффициенты ортогональных
разложений плотности. Особое внимание уделено методу квазимоментов.
Полученные результаты положены в~основу разрабатываемого инструментального
символьного программного обеспечения в~среде MATLAB-MAPLE для линейных, линейных
с~мультипликативными шумами и~нелинейных МСтС.

Для задач надежности и~безопасности технических систем имеют
важное значение вопросы сокращения числа уравнений точности и~чувствительности МОР
и~МКМ  путем использования соответствующих нормальных координат.


В качестве обобщений можно рассмотреть задачи оценки точности
и~чувствительности МОР и~МКМ на основе многомерных распределений для различных
моментов времени и~критериев качества технических систем.

\vspace*{-6pt}


{\small\frenchspacing
 {%\baselineskip=10.8pt
 \addcontentsline{toc}{section}{References}
 \begin{thebibliography}{99}

\bibitem{1-sin}
\Au{Пугачев В.\,С., Синицын И.\,Н.}
Стохастические дифференциальные системы. Анализ и~фильтрация.~--- М.:
Наука,  1990.  632~с. [\Au{Pugachev V.\,S., Sinitsyn~I.\,N.}. Stochastic differential systems.
Analysis and filtering.~--- Chichester, New York, NY, USA: Jonh Wiley, 1987.
549~p.]


\bibitem{2-sin}
\Aue{Pugachev, V.\,S., and I.\,N.~Sinitsyn}. 2001.
\textit{Stochastic systems. Theory and  applications.}~---
Singapore: Worlds Scientific. 908~p.


\bibitem{3-sin}
\Au{Синицын И.\,Н.}
Стохастические информационные технологии для исследования нелинейных
круговых стохастических систем~// Информатика и~её применения, 2011. Т.~5.
Вып.~4. С.~78--89.

\bibitem{4-sin}
\Au{Sinitsyn I.\,N., Belousov V.\,V., Konashenkova~T.\,D.}
Software tools for circular stochastic systems analysis~//
29th  Seminar (International) on Stability Problems for Stochastic Models:
Abstracts.~--- Svetlogorsk, Russia, 2011. Р.~86--87.

\bibitem{5-sin}
\Aue{Синицын И.\,Н.}
Математическое обеспечение для анализа нелинейных многоканальных круговых
стохастических систем, основанное на параметризации распределений~//
Информатика и~её применения, 2012. Т.~6. Вып.~1. С.~12--18.

\bibitem{6-sin}
\Au{Синицын И.\,Н., Корепанов Э.\,Р., Белоусов~В.\,В., Конашенкова~Т.\,Д.}
Развитие математического обеспечения для анализа нелинейных многоканальных
круговых стохастических систем~// Системы и~средства информатики, 2012.
Вып.~22. №\,1. С.~29--40.

\bibitem{7-sin}
\Au{Sinitsyn I.\,N., Belousov V.\,V., Konashenkova~T.\,D.}
Software tools for spherical stochastic systems analysis and filtering~//
30th Seminar (International) on Stability Problems for Stochastic Models 
and 6th Workshop (International) ``Applied Problems in Theory of Probabilities 
and Mathematical Statistics Related to Modeling of Information Systems: 
Book of Abstracts~/ Eds. V.\,Yu.~Korolev, S.\,Ya.~Shorgin.~---
Moscow: IPI RAN, 2012. P.~91--93.

\bibitem{8-sin}
\Au{Синицын И.\,Н.}
Параметрическое статистическое и~аналитическое моделирование распределений
в~нелинейных стохастических системах на многообразиях~// Информатика
и~её применения, 2013. Т.~7. Вып.~2. С.~4--16.


\bibitem{9-sin}
\Au{Синицын И.\,Н., Синицын В.\,И.}
Аналитическое моделирование нормальных процессов в~стохастических сис\-те\-мах
со сложными нелинейностями~//  Информатика и~её применения, 2014. Т.~8.
Вып.~3. С.~2--4.


\bibitem{10-sin}
\Au{Синицын И.\,Н., Синицын В.\,И., Сергеев~И.\,В., Белоусов~В.\,В., Шоргин~В.\,С.}
Математическое обеспечение аналитического моделирования стохастических сис\-тем
со сложными нелинейностями~// Системы и~средства информатики, 2014. Т.~24. №\,3.
С.~4--29.

\bibitem{11-sin}
\Au{ Синицын И.\,Н., Синицын В.\,И., Корепанов~Э.\,Р.}
Моделирование нормальных процессов в стохастических
системах со сложными иррациональными нелинейностями~// Информатика и~её
применения, 2015. Т.~9. Вып.~1. С.~2--8.

\bibitem{12-sin}
\Au{Синицын И.\,Н., Синицын В.\,И., Сергеев~И.\,В., Корепанов~Э.\,Р.,
Белоусов~В.\,В., Шоргин~В.\,С. }
Математическое обеспечение моделирования нормальных процессов
в~стохастических системах со сложными иррациональными нелинейностями~//
Системы и средства информатики, 2015. Т.~25. №\,2. С.~45--61.

\bibitem{13-sin}
\Au{Ватанабэ С., Икэда Н.} Стохастические
дифференциальные уравнения и диффузионные процессы.~--- М.: Наука, 1986.
474~с.

\bibitem{16-sin} %14
ГОСТ 23743-88. Изделия авиационной техники.
Номенклатура показателей безопасности полета, надежности,
контролепригодности, эксплуатационной и~ремонтной технологичности.


\bibitem{15-sin}
\Au{Болотин В.\,В.}
Теория надежности машин~// Машиностроение: Энциклопедия. Т.~IV-3.
Надежность машин.~--- М.: Машиностроение, 1998. 38~с.

\bibitem{14-sin} %16
\Au{Александровская Л.\,Н., Аронов~И.\,З., Круглов~В.\,И. и~др.}
 Безопасность и надежность технических сис\-тем.~---
 М.: Университетская книга, Логос, 2008. 376~с.

\bibitem{17-sin}
\Au{Евланов А.\,Г., Константинов~В.\,М. }
Системы со случайными параметрами.~--- М.: Наука, 1976. 568~с.

\bibitem{18-sin}
Справочник по теории автоматического управления~/ Под ред. А.\,А.~Красовского.~---
М.: Наука, 1987. 712~с.
 \end{thebibliography}

 }
 }

\end{multicols}

\vspace*{-10pt}

\hfill{\small\textit{Поступила в~редакцию 21.05.15}}

\newpage

\vspace*{-24pt}

%\hrule

%\vspace*{2pt}

%\hrule

%\vspace*{12pt}

\def\tit{ANALYTICAL MODELING
IN~STOCHASTIC SYSTEMS ON~MANIFOLDS BASED ON~ORTHOGONAL EXPANSIONS}

\def\titkol{Analytical modeling
in~stochastic systems on~manifolds based on~orthogonal expansions}

\def\aut{I.\,N.~Sinitsyn}

\def\autkol{I.\,N.~Sinitsyn}

\titel{\tit}{\aut}{\autkol}{\titkol}

\vspace*{-9pt}


\noindent
Institute of Informatics Problems,
Federal Research Center ``Computer Science and Control'' of
the Russian Academy of Sciences, 44-2 Vavilov Str.,
Moscow 119333, Russian Federation


\def\leftfootline{\small{\textbf{\thepage}
\hfill INFORMATIKA I EE PRIMENENIYA~--- INFORMATICS AND
APPLICATIONS\ \ \ 2015\ \ \ volume~9\ \ \ issue\ 3}
}%
 \def\rightfootline{\small{INFORMATIKA I EE PRIMENENIYA~---
INFORMATICS AND APPLICATIONS\ \ \ 2015\ \ \ volume~9\ \ \ issue\ 3
\hfill \textbf{\thepage}}}

\vspace*{3pt}



\Abste{Problems of accuracy and sensitivity of
one-dimensional distributions by parametrical analytical modeling
algorithms on the basis of the orthogonal expansion method (OEM) and
the quasi-moment method (QMM) in stochastic systems on manifolds (MStS) are
considered. Stochastic system on manifolds is described by Ito linear, linear with multiplicative
noises and nonlinear equations with Wiener and Poisson noises.
The OEM and QMM equations are derived by generalized Ito formula. Methodological
results are the basis of the original symbolic software tools for
MATLAB-MAPLE. The problems of reduction of number of OEM and QMM equations are
discussed, reliability and security algorithms are presented.
Scalar nonlinear MStS with multiplicative white noise is investigated.
Some possible generalizations are formulated.}

\KWE{analytical modeling method (AMM);
generalized Ito formula;
Hermite polynomials;
OEM and QMM accuracy equations;
OEM and QMM sensitivity equations;
orthogonal expansion method (OEM);
quasi-moment method (QMM);
stochastic system on manifold (MStS)}


\DOI{10.14357/19922264150302}

\Ack
\noindent
The work was financially supported by the Russian Foundation for Basic Research 
(project 15--07--02244).



%\vspace*{3pt}

  \begin{multicols}{2}

\renewcommand{\bibname}{\protect\rmfamily References}
%\renewcommand{\bibname}{\large\protect\rm References}

{\small\frenchspacing
 {%\baselineskip=10.8pt
 \addcontentsline{toc}{section}{References}
 \begin{thebibliography}{99}


\bibitem{1-sin-1}
\Aue{Pugachev, V.\,S., and  I.\,N.~Sinitsyn}.  1987.
\textit{Stochastic differential systems. Analysis and filtering.}
Chichester, New York, NY: Jonh Wiley. 549~p.

\bibitem{2-sin-1}
\Aue{Pugachev, V.\,S., and I.\,N.~Sinitsyn}. 2001.
\textit{Stochastic systems. Theory and  applications.}
Singapore: Worlds Scientific. 908~p.


\bibitem{3-sin-1}
\Aue{Sinitsyn, I.\,N.} 2011.
Stokhasticheskie informatsionnye tekhnologii dlya issledovaniya
nelineynykh krugovykh stokhasticheskikh sistem [Stochastic informational
technologies for circular stochastic systems investigation].
\textit{Informatika i~ee Primeneniya}~--- \textit{Inform Appl.} 5(4):78--89.

\bibitem{4-sin-1}
\Aue{Sinitsyn, I.\,N., V.\,V.~Belousov, and T.\,D.~Konashenkova.} 2011.
Software tools for circular stochastic systems analysis.
\textit{29th  Seminar (International) on Stability Problems for Stochastic Models:
Abstracts}. Svetlogorsk, Russia. 86--87.


\bibitem{5-sin-1}
\Aue{Sinitsin, I.\,N.} 2012.
Matematicheskoe obespechenie dlya analiza nelineynykh mnogokanal'nykh
krugovykh stokhasticheskikh sistem, osnovannoe na paramet\-ri\-za\-tsii raspredeleniy
[Mathematical software for analysis of nonlinear multichannel circular
stochastic systems based on distributions parametrization].
\textit{Informatika i~ee Primeneniya}~--- \textit{Inform Appl.} 6(1):12--18.

\bibitem{6-sin-1}
\Aue{Sinitsyn, I.\,N., E.\,R.~Korepanov, V.\,V.~Belousov, and T.\,D.~Konashenkova.}
2012.
Razvitie matematicheskogo obespecheniya dlya analiza nelineynykh mnogokanal'nykh
krugovykh stokhasticheskikh sistem [Development of mathematical software
for analysis of nonlinear mutlichannel circular stochastic systems].
\textit{Sistemy i~Sredstva Informatiki}~--- \textit{Systems and Means of Informatics}
 22(1):29--40.

\bibitem{7-sin-1}
\Aue{Sinitsyn, I.\,N., V.\,V.~Belousov, and T.\,D.~Konashenkova.}  2012.
Software tools for spherical stochastic systems analysis and
filtering.  \textit{30th Seminar (International) on Stability Problems 
for Stochastic Models and 6th Workshop (International) ``Applied Problems in 
Theory of Probabilities and Mathematical Statistics Related to Modeling of 
Information Systems: Book of Abstracts}. Eds. V.\,Yu.~Korolev and S.\,Ya.~Shorgin.
Moscow:   IPI RAN. 91--93.

\bibitem{8-sin-1}
\Aue{Sinitsyn, I.\,N.} 2013.
Parametricheskoe statisticheskoe i~analiticheskoe modelirovanie raspredeleniy
v~nelineynykh stokhasticheskikh sistemakh na mnogo\-ob\-ra\-zi\-yakh [Parametric
statistical and analytical modeling of distributions in nonlinear stochastic
systems on manifolds]. \textit{Informatika i~ee Primeneniya}~---
\textit{Inform. Appl.} 7(2):4--16.


\bibitem{9-sin-1}
\Aue{Sinitsyn, I.\,N., and V.\,I.~Sinitsyn.} 2014.
Analiticheskoe modelirovanie normal'nykh protsessov v~sto\-kha\-sti\-che\-skikh sistemakh
so slozhnymi nelineynostyami [Analytical modeling of normal processes stochastic
systems with complex nonlinearities]. \textit{Informatika i~ee Primeneniya}~---
\textit{Inform. Appl.} 8(3):2--4.

\bibitem{10-sin-1}
\Aue{Sinitsyn, I.\,N., V.\,I.~Sinitsyn, I.\,V.~Sergeev, V.\,V.~Belousov,
and V.\,S.~Shorgin.} 2014.
Matematicheskoe obespechenie analiticheskogo modelirovaniya sto\-kha\-sti\-che\-skikh
sistem so slozhnymi nelineynostyami [Mathematical software for analytical
modeling of stochastic systems with complex nonlinearities].
\textit{Sistemy i~Sredstva Informatiki}~--- \textit{Systems and Means of
Informatics}  24(3):4--29.

\bibitem{11-sin-1}
\Aue{Sinitsyn, I.\,N., V.\,I.~Sinitsyn, and E.\,R.~Korepanov.} 2015.
Modelirovanie normal'nykh protsessov v~sto\-kha\-sti\-che\-skikh sistemakh
so slozhnymi irratsional'nymi neli\-ney\-no\-stya\-mi  [Modeling of normal processes
in stochastic systems with complex irrational
nonlinearities]. \textit{Informatika i~ee Primeneniya}~---
\textit{Inform. Appl}. 9(1):2--8.

\bibitem{12-sin-1}
\Aue{Sinitsyn, I.\,N., V.\,I.~Sinitsyn, I.\,V.~Sergeev, E.\,R.~Korepanov,
V.\,V.~Belousov, and V.\,S.~Shorgin.} 2015.
Ma\-te\-ma\-ti\-che\-skoe obespechenie modelirovaniya normal'nykh protsessov
v~stokhasticheskikh sistemakh so slozhnymi irratsional'nymi nelineynostyami
[Mathematical software for modeling of normal processes in stochastic systems
with complex irrational nonlinearities]. \textit{Sistemy i~Sredstva Informatiki}~---
\textit{Systems and Means of Informatics}  25(2):45--61.

\bibitem{13-sin-1}
\Aue{Vatanabe, S., and N.~Ikeda.} 1986.
\textit{Stokhasticheskie dif\-fe\-ren\-tsi\-al'\-nye uravneniya i~diffuzionnye protsessy}
[Stochastic differential equations and diffusion processes]. Moscow:   Nauka. 474~p.

\bibitem{14-sin-1} %16
\Aue{Aleksandrovskaya, L.\,N., I.\,Z. Aronov, V.\,I.~Kruglov, \textit{et al.}}
2008.
\textit{Bezopasnost' i~nadezhnost' tekhnicheskikh sistem} [Security and reliability of
technical systems].
 Moscow:  Universitetskaya kniga, Logos.  376~p.


\bibitem{15-sin-1}
\Aue{Bolotin, V.\,V. } 1998.
{Teoriya nadezhnosti mashin}  [Theory of machine realiability].
\textit{Mashinostroenie.  Entsiklopetsiya} [Machinebuilding.
Encyclopedia]. Vol.~IV-3.
%Red. sovet K.\,V.~Frolov (pred.) i~dr.
Nadezhnost' mashin [Reliability of machines]. Moscow:  Mashi\-no\-stro\-enie.
38~p.

\bibitem{16-sin-1} %14
GOST 23743-88. Izdeliya aviatsionnoy tekhniki.
Nomenklatura pokazateley bezopasnosti poleta, nadezhnosti,
kontroleprigodnosti, ekspluatatsionnoy i~remontnoy tekhnologichnosti
[Aircraft products. Nomenclature of flight safety, reliability, 
testability, operational and repair manufacturability].


\bibitem{17-sin-1}
\Aue{Evlanov, A.\,G., and V.\,M.~Konstantinov.} 1976
\textit{Sistemy so slozhnymi parametrami} [Systems with random parameters].
Moscow. Nauka. 568~p.

\bibitem{18-sin-1}
Krasovskskiy, A.\,A., ed. 1987.
 \textit{Spravochnik po teorii avtomaticheskogo upravleniya}
 [Handbook for automatic control].  Moscow:  Nauka. 712~p.

\end{thebibliography}

 }
 }

\end{multicols}

\vspace*{-3pt}

\hfill{\small\textit{Received May 21, 2015}}

\Contrl

\noindent
\textbf{Sinitsyn Igor N.} (b.\ 1940)~---
Doctor of Science in technology, professor, Honored scientist of RF,
Head of Department, Institute of Informatics Problems,
Federal Research Center ``Computer Science and Control'' of
the Russian Academy of Sciences, 44-2 Vavilov Str.,
Moscow 119333, Russian Federation; sinitsin@dol.ru

\label{end\stat}


\renewcommand{\bibname}{\protect\rm Литература}