\def\stat{agalarov}

\def\tit{ПОРОГОВАЯ СТРАТЕГИЯ ОГРАНИЧЕНИЯ ДОСТУПА К~РЕСУРСАМ В~СИСТЕМЕ МАССОВОГО ОБСЛУЖИВАНИЯ
{\boldmath{$M/D/1$}} С~ФУНКЦИЕЙ ШТРАФОВ ЗА~НЕСВОЕВРЕМЕННОЕ ОБСЛУЖИВАНИЕ
ЗАЯВОК$^*$}

\def\titkol{Пороговая стратегия ограничения доступа к~ресурсам в~СМО
$M/D/1$ с~функцией штрафов} % за несвоевременное обслуживание заявок}

\def\aut{Я.\,М.~Агаларов$^1$}

\def\autkol{Я.\,М.~Агаларов}

\titel{\tit}{\aut}{\autkol}{\titkol}

{\renewcommand{\thefootnote}{\fnsymbol{footnote}} \footnotetext[1]
{Работа выполнена при поддержке РФФИ (проекты 15-07-03406 и~13-07-00223).}}


\renewcommand{\thefootnote}{\arabic{footnote}}
\footnotetext[1]{Институт проблем информатики Федерального исследовательского
центра <<Информатика и~управление>> Российской академии наук,
agglar@yandex.ru}

\vspace*{-12pt}


  \Abst{Рассматривается задача оптимизации управления доступом на множестве
динамических пороговых стратегий в~системе $M/D/1$, в~которой прием заявок
в~систему
прекращается, если число одновременно находящихся в~ней заявок превышает пороговое
значение, и~прием заявок возобновляется, если число заявок станет равно или меньше этого
значения. В~качестве целевой функции используется среднее значение предельного дохода
системы, получаемого в~единицу времени в~стационарном режиме, при условии, что от
каждой принятой на обслуживание заявки получает фиксированную плату, за
несвоевременно обслуженную заявку система платит фиксированный штраф, а~при
отклонении заявки система не получает плату и~не платит штраф. Получены оценки для
оптимального значения целевой функции и~оптимального порогового значения.}

  \KW{система массового обслуживания; пороговая стратегия; оптимизация}

  \DOI{10.14357/19922264150306}

\vspace*{-4pt}

\vskip 12pt plus 9pt minus 6pt

\thispagestyle{headings}

\begin{multicols}{2}

\label{st\stat}

\section{Введение}

  Одним из механизмов исключения перегрузок в~системах, предназначенных
для обработки стохастических потоков заданий, является контроль за
перегрузкой и~ограничение доступа для входной нагрузки на ресурсы системы.
В большинстве\linebreak существу\-ющих алгоритмов ограничения доступа используются
пороговые стратегии управления доступом, суть которых заключается в~том,
что для некоторых параметров, определяющих качество работы сети,
устанавливаются пороговые значения, при переходе которых
соответствующими параметрами в~большую или меньшую сторону по
специальному алгоритму осуществляется ограничение или расширение доступа
для входной нагрузки к~ресурсам~[1].

  Основные результаты по оценке эффектив\-ности и~оптимизации пороговых
стратегий получены ме\-то\-да\-ми теории очередей в~виде явных и~\mbox{неявных}
выражений для расчета стационарных
  ве\-ро\-ят\-но\-ст\-но-вре\-мен\-н$\acute{\mbox{ы}}$х характеристик (средней
задержки, стационарного распределения вероятностей, интенсивности
выходной нагрузки и~т.\,д.)\ систем массового обслуживания (СМО) различного типа ($M/G/1$, $M\_2/G/1$,
$MMPP/G/1$, $M^x/G/1$, $G/M/m/K$, $MQ|M|m$ и~др.)~[2--7]. Задача
оптимизации пороговых стратегий для СМО, за исключением, может быть,
самых простых, аналитически не решена, предлагаются либо эвристические
решения, либо решение численными методами. В~качестве целевой функции,
как правило, используются усредненные характеристики системы: средняя
задержка заявок в~системе, среднее число заявок в~системе, интенсивность
потерь заявок.

  В данной работе исследована задача выбора оптимального порога для
системы $M/D/1$, в~которой прием заявок в~систему прекращается, когда число
одновременно находящихся в~ней заявок превышает порог, и~прием заявок
возобновляется, если число заявок опустится до порога или ниже. В~качестве
целевой функции используется среднее значение предельного дохода системы,
получаемого в~единицу времени в~стационарном режиме, отнесенного к~одной
поступившей заявке, при условии, что от каждой принятой на обслуживание
заявки система получает фиксированную плату, за несвоевременно
обслуженную заявку платит фиксированный штраф, а~при отклонении заявки
система не получает плату и~не платит штраф. Аналогичная задача с~такой же
целевой функцией рассмотрена в~работе~\cite{4-aga} для системы $G/M/m$.
В~работе~\cite{6-aga} исследована оптимальность пороговой стратегии для
системы $M/G/1$ в~случае, когда система платит фиксированный штраф в~каждую единицу времени ожидания в~очереди независимо от длительности
пребывания.

\section{Описание системы и~постановка задачи}

  Рассматривается одноканальная СМО типа $M/D/1$, на которую поступает
пуассоновский поток однородных заявок с~детерминированным временем
обслуживания. Вновь поступившая заявка занимает свободный прибор, если
таковой есть, иначе становится в~очередь в~порядке поступления согласно
заданной стратегии управления доступом~$f$. Система получает плату
$C_0\hm>0$, если время выполнения заявки (время пребывания заявки в~системе) не превышает заданную величину (<<дедлайн>>)~$t_0$, и~платит
штраф $C_1\hm>0$, если время выполнения заявки превышает~$t_0$. Если
поступившая заявка согласно стратегии управления отклоняется, то система не
получает плату и~не платит штраф. Величину платы, получаемой системой при
обслуживании заявки, за вычетом штрафа будем называть доходом от
обслуживания заявки.

  Среднее значение дохода от поступления заявки зависит от принятой в~системе 
  стратегии управления доступом ($f$), и~его предельное значение
определяется в~виде предела
  \begin{equation}
  Q^f= \lim\limits_{T\to\infty} \sum\limits_{n=1}^{N_{\mathrm{вх}}}
\fr{d_n^f}{N_{\mathrm{вх}}(T)}\,,
  \label{e1-aga}
  \end{equation}
где $d_n^f$~--- доход, получаемый системой при обслуживании $n$-й в~порядке 
поступления заявки при стратегии~$f$; $N_{\mathrm{вх}}(T)$~---
число заявок, поступивших на вход системы за отрезок времени $[0,T]$.

  Пороговые стратегии управления доступом, рассматриваемые в~данной
статье, составляют класс динамических стационарных стратегий $F_d\hm=
\{f^{\overline{k}}\}$, определяемые следующим образом: $f^{\overline{k}}\hm= \{f_i^{k_i},
i\hm\geq 0\}$, где $i$~--- состояние системы (число заявок в~системе,
остающихся в~момент завершения обслуживания заявки), $f_i^{k_i}\hm=1$ при
$0\hm\leq i\hm\leq k_i\hm-1$, $f_i^{k_i}\hm=0$ при $k_i\hm\leq i$, $\infty \hm>
K\hm>k_i\hm>0$,~--- пороговое значение (натуральное число), назначаемое в~момент завершения обслуживания заявки. При этом если в~момент поступления
заявки в~систему $f_i^{k_i}\hm=1$, то заявка ставится в~очередь или сразу
начинает обслуживаться (если прибор свободен), иначе она отклоняется и~покидает систему. В~дальнейшем для упрощения обозначений всюду, за
исключением некоторых случаев, вместо слов
<<стратегия~$f^{\overline{k}}$>> (<<политика~$f_i^{k_i}$>>) будем писать
<<стратегия~$\overline{k}$>> (<<политика~$k_i$>>), вместо
обозначений~$f^{\overline{k}}$ и~$f_i^{k_i}$ будем использовать соответственно
$\overline{k}$ и~$k_i$. Чтобы исключить тривиальный случай, предположим, что
$k_0\hm>1$.

  Как известно (см., например,~\cite{8-aga}), процесс обслуживания заявок в~данной системе описывается цепью Mаркова, где переходы цепи определяются
моментами окончания обслуживания, и~состояние системы есть число заявок,
остающихся в~системе в~момент ухода обслуженной заявки с~прибора.
Предположим, что цепь Маркова при любой стратегии $f\hm\in F_d$ является
цепью с~одним положительным эргодическим классом~\cite{8-aga}. Заметим,
что цепь Маркова отвечает этому предположению, если, например, для класса
стратегий~$F_d$ выполняется условие: для любого состояния цепи~$i$
существует состояние~$j$ такое, что $i\hm\leq k_j\hm-1$.

  Введем обозначения:
\begin{description}
\item[\,]     $\pi_i^{\overline{k}}, I-1\geq i\geq0$,~--- стационарное распределение
вероятностей цепи при стратегии~$\overline{k}$ ($\pi_i^{\overline{k}}$~---
вероятность того, что цепь находится в~состоянии~$i$),
$I\hm= \max\limits_i \{k_i\}$;\\[-13.5pt]
\item[\,]
     $g^{\overline{k}}$~--- средний предельный доход, получаемый за один
шаг марковской цепи (за интервал времени между двумя соседними моментами
окончания обслуживания заявок) при стратегии~$\overline{k}$;\\[-13.5pt]
\item[\,]
     $q_i^{\overline{k}}$~--- средний доход, получаемый системой в~состоянии~$i$ (за время пребывания в~состоянии~$i$) при
стратегии~$\overline{k}$, $i\hm\geq 0$.
\end{description}

  Соотношение~(1) можно написать в~виде:
  
  \noindent
  \begin{multline*}
  Q^f= \lim\limits_{T\to\infty} \sum\limits_{n+1}^{N_{\mathrm{вх}}(T)}
\fr{d_n^{\overline{k}}} {N_{\mathrm{вх}}(T)}={}\\
{}=
  \lim\limits_{T\to\infty} \fr{N_{\mathrm{вых}}(T)} {N_{\mathrm{вх}}(T)}
\sum\limits_{n+1}^{N_{\mathrm{вх}}(T)}
\fr{d_n^{\overline{k}}}{N_{\mathrm{вых}}(T)}={}\\
  {}= \lim\limits_{T\to\infty} \fr{N_{\mathrm{вых}}(T)} {N_{\mathrm{вх}}(T)}
\lim\limits_{T\to \infty}
  \sum\limits_{n=1}^{N_{\mathrm{вх}}(T)}
\fr{d_n^{\overline{k}}}{N_{\mathrm{вых}}(T)}\,,
\end{multline*}
где $N_{\mathrm{вых}}(T)$~--- число заявок, обслуженных за отрезок времени
$[0,T]$. Как следует из теории управляемых марковских
процессов~\cite{8-aga}, предел $\lim\limits_{N_{\mathrm{вх}}\to\infty}
\sum\limits_{n=1}^{N_{\mathrm{вх}}} d_n^{\overline{k}}/N_{\mathrm{вых}}$
равен среднему доходу за один шаг рассматриваемой цепи Маркова и~$$
\lim\limits_{N_{\mathrm{вх}}\to\infty} \sum\limits_{n=1}^{N_{\mathrm{вх}}}
\fr{d_n^{\overline{k}}}{N_{\mathrm{вых}}} = \sum\limits_{i=0}^{K-1} \pi_i^k
q_i^{\overline{k}}\,.
$$

  Ставится задача: найти оптимальную стратегию $\overline{k}^*\hm\in F_d$
такую, что

\noindent
  \begin{equation}
  \max\limits_{\overline{\overline{k}}\in F_d} Q^{\overline{k}}
=Q^{\overline{k}^*}\,.
  \label{e2-aga}
  \end{equation}
Здесь $Q^{\overline{k}} = (1-\theta^{\overline{k}}) \sum\limits_{i=0}^{I-1}
\pi_i^{\overline{k}} q_i^{\overline{k}}$, где $\theta^{\overline{k}}$~---
стационарная вероятность того, что система находится в~состоянии~$i$ таком,
что $f_i\hm=1$.

\vspace*{-3pt}

\section{Метод решения}

  Получим выражения для расчета величины дохода~$q_i^{\overline{k}}$.
Введем обозначения:
\begin{description}
\item[\,]     $l$-заявка~--- заявка, которая в~данный момент имеет номер~$l$ 
     в~порядке поступления (будем считать, что заявки, находящиеся одновременно 
     в~сис\-те\-ме, в~любой момент пронумерованы числами $1, 2,\ldots$ в~порядке
поступления, начиная с~номера~1);
\item[\,]
     $t_0\geq \nu>0$~--- время обслуживания заявки;
\item[\,]
     $\tau_l$~--- случайная величина длины интервала времени между
моментами поступления $l$-заявки и~($l-1$)-за\-явки;
\item[\,]
     $\infty >\lambda >0$~--- интенсивность поступления заявок в~систему;
\item[\,]
     $\rho=\lambda\nu$~--- величина поступающей нагрузки;
\item[\,]
     $p_{ij}^{k_i}$~--- вероятность перехода цепи из состояния~$i$ в~состояние~$j$ при пороге~$k_i$;
\item[\,]
     $U_l^i(m,\nu)$~--- случайная величина времени пребывания в~системе
$l$-за\-яв\-ки (интервал времени с~момента поступления до момента окончания
обслуживания), поступившей за время на\-хож\-де\-ния сис\-те\-мы в~состоянии~$i$,
при условии, что за время~$\nu$ в~систему поступило ровно $m$ заявок.
\end{description}

  Для $U_l^i (m,\nu)<l\leq m$ справедливо соотношение:
  
  \noindent
  \begin{equation}
  U_l^i(m,\nu) =(l+i)\nu -\sum\limits_{j=1}^l \tau_j\,.
  \label{e3-aga}
  \end{equation}
Из~(\ref{e3-aga}) получаем:
\begin{equation}
U_{l+1}^i (m,\nu) =U_l^i(m,\nu) +\nu -\tau_{l+1}\,.
\label{e4-aga}
\end{equation}

  Обозначим: $B_m$~--- событие, состоящее в~том, что за время~$\nu$
поступило ровно $m$ заявок, $m\hm>0$; $A_{l,m}$~--- событие, состоящее в~том, что $l$-за\-яв\-ка обслужится неуспешно при условии~$B_m$, $l\hm=
1,\ldots, \min \{m,k_i\hm-i\}$; $E_{j,m}$~--- событие, состоящее в~том, что за
время~$\nu$ произошло ровно $j$ неуспешных обслуживаний заявок при
условии~$B_m$, $j\hm= 0,\ldots, \min \{m,k_i\hm-i\}$.

  Обозначим $l_i^{\max} \hm= \min \{m, k_i\hm-i\}$ Тогда из~(\ref{e3-aga})
и~(\ref{e4-aga}) следует
  \begin{equation}
  E_{l_i^{\max}-l,m}=A_{l+1,m} \overline{A}_{l,m},\ m>0,\ l_i^{\max} \geq
l\geq 0\,,
  \label{e5-aga}
   \end{equation}
где $\overline{A}_{0,m}=\Omega$; $A_{m+1,m}\hm= \Omega$;
$$
E_{l_i^{\max},m}=\Omega\,,\enskip E_{j,m}=\overline{\Omega}\,,
$$
 если $i\hm> t_0/\nu$, $m\hm> 0$, $l_i^{\max}\hm> j\hm\geq 0$.
 
 \columnbreak

  Величина дохода, получаемого системой при пороге $k_i\hm>1$ за время
нахождения в~состоянии $i\hm>0$ (т.\,е.\ за один шаг соответствующей цепи
Маркова) при условии~$B_m$ определяется как сумма доходов, получаемых
при обслуживании допущенных в~систему за время~$\nu$ новых заявок и~выражается формулами:
  \begin{multline}
  q_i^{k_i}(\nu,m;t_0) ={}\\
  {}=
  \begin{cases}
 mC_0-C_1 \displaystyle\sum\limits_{l=0}^m (m-l) P\left( E_{m-l,m}\right), &\\
   \hspace*{35mm}m\leq k_i-i\,;&\\
   (k_i-i) C_0 -{}\\
   {}-C_1\displaystyle\sum\limits_{l=0}^{k_i-i} (k_i-i-l) P\left( E_{k_i-i-l,m}\right)\,,\\
      \hspace*{35mm} m\geq k_i-i\,,&\\
   \hspace*{30mm}0<i\leq k_i-1\,;&\\
   0\,, &\hspace*{-28mm}i\geq k_i\,.
  \end{cases}
  \label{e6-aga}
  \end{multline}
Величина дохода, получаемого системой при пороге $k_i\hm=1$ в~состоянии
$i\hm=0$, равна $g^{\overline{k}}\hm= q_0^{k_0}\hm=C_0$; величина дохода,
получаемого системой при пороге $k_i\hm>1$ в~состоянии $i\hm=0$, составляет
$q_0^{k_0}\hm= q_1^{k_1}\hm+C_0$.

  Рассмотрим случай $0\hm< i\hm\leq t_0/\nu$, $k_i\hm>1$. Как следует
из~(\ref{e3-aga})--(\ref{e5-aga}), выполнение события $E_{l_i^{\max}-l,m}$
эквивалентно совместному выполнению неравенств:
  \begin{equation}
  \left.
  \begin{array}{rl}
  \displaystyle\sum\limits_{j=1}^l \tau_j &\geq (i+l)\nu -t_0\,;\\[6pt]
  \displaystyle \sum\limits_{j=1}^{l+1} \tau_j &< (i+l+1)\nu -t_0\,,
  \end{array}
  \right\}
  \label{e7-aga}
  \end{equation}
где $0\leq \sum\limits_{j=1}^m \tau_j\leq \nu$. Из~(\ref{e5-aga}) и~(\ref{e7-aga})
следует: $P(E_{l_i^{\max}-l,m})\hm=0$ при $(i\hm+l)\hm> t_0/\nu\hm+1$ или
$(i\hm+l)\hm\leq t_0/\nu\hm-1$. Отсюда вытекает, что для $0\hm< i\hm\leq
t_0/\nu$, $k_i\hm>0$ выражения~(\ref{e6-aga}) эквивалентны выражениям:
\begin{multline}
q_i^{k_i}(\nu,m;t_0) ={}\\
\hspace*{-2pt}{}=\!
\begin{cases}
\displaystyle mC_0-C_1
\sum\limits_{l=[t_0/\nu]^+ -i-1}^{\min\{m,[t_0/\nu]-i+1\}}\hspace*{-8mm}(m-l) P\left(
E_{m-l,m}\right), &\\
 \hspace*{43mm}m\leq k_i-i\,;&\\
\displaystyle (k_i-i)C_0-{}&\\
{}\!\!- C_1\!\! \displaystyle
\sum\limits_{l=[t_0/\nu]^+ -i-1}^{\min\{k_i-i, [t_0/\nu]-i+1\}}\hspace*{-11mm}(k_i-i-
l)P\left( E_{k_i-i-l,m}\right),& \\
 \hspace*{43mm}m\geq k_i-i\,,&
\end{cases}\!\!
\label{e8-aga}
\end{multline}
где $[t_0/\nu]$~--- целая часть числа $t_0/\nu$;
$$
\left[ \fr{t_0}{\nu}\right]^+ = \begin{cases}
\fr{t_0}{\nu}\,, &\ \mbox{если } \left[ \fr{t_0}{\nu}\right]=\fr{t_0}{\nu}\,;\\[10pt]
\left[ \fr{t_0}{\nu}\right]+1\,, &\ \mbox{если } \left[
\fr{t_0}{\nu}\right]<\fr{t_0}{\nu}\,.
\end{cases}
$$

  Как следует из~(\ref{e5-aga}), при $i\hm> t_0/\nu$ для любого $m\hm>0$
имеют место равенства 
$$
P(E_{l_i^{\max},m})=1\,;
$$

\vspace*{-12pt}

\noindent
 \begin{multline*}
  q_i^{k_i}(\nu,m;t_0) = {}\\
  {}=\begin{cases}
  m\left(C_0-C_1\right)\,, & 0<m\leq k_i-i\,;\\
  (k_i-i)\left(C_0-C_1\right), &\hphantom{0<}{}\; m\geq k_i-i\,.
  \end{cases}
%  \label{e9-aga}
  \end{multline*}

  Величина полного дохода, получаемого системой в~состоянии~$i$ при
пороге $k_i\hm>1$ и~$i\hm\leq t_0/\nu$, определяется с~учетом~(\ref{e8-aga}) по
формуле полной вероятности:
  \begin{multline}
  q_i^{k_i} =\sum\limits_{m=0}^\infty \fr{\rho^m}{m!}\,e^{-\rho} q_i^{k_i}
(\nu,m;t_0)={}\\
  {}=\sum\limits_{m=1}^{k_i-i-1} \fr{\rho^m}{m!}\,e^{-\rho} \left[
  \vphantom{\sum\limits_{l=m_i^+-1}^{\min\{m,m_i+1\}}}
  mC_0 - {}\right.\\
\left.  {}-C_1
\sum\limits_{l=m_i^+-1}^{\min\{m,m_i+1\}} \hspace*{-5mm}(m-l)P\left( E_{m-l,m}\right)
\right]+{}\\
  {}+\sum\limits_{m=k_i-i}^\infty \fr{\rho^m}{m!}\,e^{-\rho} \left[
  \vphantom{\sum\limits_{l=m_i^+-1}^{\min\{m,m_i+1\}}}
  \left(k_i-i\right) C_0-{}\right.\\
\left.  {}-
C_1\sum\limits_{l=m_i^+-1}^{\min\{k_i-i,m_i+1\}} \hspace*{-7mm}(k_i-i-l)P\left( E_{k_i-i-
l,m}\right)\right]={}\\
  {}=(k_i-i)C_0+C_0e^{-\rho} \sum\limits_{m=1}^{k_i-i-1} (m-k_i+i)
\fr{\rho^m}{m!}-{}\\
  {}-C_1\sum\limits_{m=1}^{k_i-i-1} \hspace*{-1mm}\fr{\rho^m}{m!}\,e^{-\rho}
\sum\limits_{l=m_i^+-1}^{\min\{m,m_i+1\}} \hspace*{-5mm}(m-l)P\left( E_{m-l,m}\right)-{}\\
  \hspace*{-3mm}{}-
  C_1e^{-\rho}\hspace*{-3mm}\sum\limits_{l=m_i^+-1}^{\min\{ k_i-i,m_i+1\}}\hspace*{-8mm} (k_i-i-l)\hspace*{-2mm}
\sum\limits_{m=k-i}^\infty \hspace*{-2mm}\fr{\rho^m}{m!}\,P\left( E_{k_i-i-l,m}\right),\!\!\!\!
  \label{e10-aga}
  \end{multline}
где $m_i^+ = [t_0/\nu]^+-i$; $m_i\hm= [t_0/\nu]\hm-i$.

  Для краткости изложения в~дальнейшем обозначим $\nu^+_{i,l}\hm= \min
\left\{ 1, (i+l+1) - t_0/\nu\right\}$; $\nu^-_{i,l}\hm= \max \left\{ 0, (i+l)-
t_0/\nu\right\}$. Отметим еще раз (см.~(\ref{e5-aga})), что для $i,l$ таких, что
$\nu^+_{i,l}\hm\leq 0$, $P(E_{j,m})\hm=0$, $j\hm\leq \min \{k_i\hm-i,m\}$.

  Вероятность $P(E_{m-l,m})$ для случая $i\hm\leq t_0/\nu$,\linebreak $k\hm>1$
вычисляется по формуле
  \begin{multline}
  P\left( E_{m-l,m}\right) ={}\\
  {}= \fr{1}{\nu^m} \int\limits_{\max\{0,(i+l)\nu-
t_0\}}^{\min\{\nu,(i+l+1)\nu-t_0\}}\int\limits_{x_1}^{\min\{\nu,(i_l+l+1)\nu-t_0\}}
 \fr{m!}{(l-1)!}\times{}\\
 {}\times \fr{x_1^{l-1}(\nu-x_2)^{m-l-1}}{(m-l-
1)!}\,dx_2dx_1={}\\
  {}=
  \int\limits_{\nu_{i,l}^-}^{\nu_{i,l}^+} \int\limits_x^{\nu_{i,l}^+} \fr{m!}{(l-
1)!}\,\fr{x^{l-1}(1-y)^{m-l-1}}{(m-l-1)!}\,dy dx
  \label{e11-aga}
  \end{multline}
при $k_i-i\geq m\geq l>0$,
\begin{multline*}
P\left( E_{m.m}\right) =\fr{m}{\nu^m} \int\limits^{\min\{\nu,(i+1)\nu-
t_0\}}_{\max\{0,i\nu-t_0\}} (\nu-x_1)^{m-1}\,dx_1={}\\
{}=m\int\limits_0^{\nu_{i,0}^+} (1-x)^{m-1}\,dx=1-\left(1-\nu^+_{i,0}\right)^m
\end{multline*}
при $l=0$, $k_i-i\geq m$. Формулы~(\ref{e11-aga}) следуют из~(\ref{e7-aga}) и~из того, что распределение величин $S_l\hm= \sum\limits_{j=1}^l \tau_j$, $l\hm=
1,\ldots ,m$, совпадает с~распределением порядковых статистик из
выборки~$n$, взятой из равномерного распределения на $[0,\nu]$~\cite{7-aga}.

  Взяв в~(\ref{e11-aga}) интеграл по~$y$ и~подставив формулу
  \begin{multline}
  x^{l-1}(1-x)^{m-l} ={}\\
  {}= \sum\limits_{j=0}^{m-l} (-1)^j \fr{(m-l)!} {j!(m-l-
j)!}\,x^{l+j-1}\,,
  \label{e12-aga}
  \end{multline}
получим
\begin{multline}
P\left( E_{m-l,m}\right) = \fr{m!}{(l-1)!(m-l-1)!}\times{}\\
{}\times \int\limits_{\nu^-
_{i,l}}^{\nu^+_{i,l}} x^{l-1}\fr{(1-x)^{m-l}-\left(1-\nu^+_{i,l}\right)^{m-l}}{m-
l}\,dx={}\\
{}=\fr{m!}{(l-1)!(m-l)!} \left[
\vphantom{\fr{\Delta_{i,l}(l)\left(1-\nu^+_{i,l}\right)^{m-l}}{l}}
\sum\limits_{j=0}^{m-l} (-1)^j \fr{(m-l)!}{j!(m-l-
j)!(l+j)}\times\right.\\
\left.{}\times\Delta_{i,l}(l+j)-
\fr{\Delta_{i,l}(l)\left(1-\nu^+_{i,l}\right)^{m-l}}{l}\right]\,,
\label{e13-aga}
\end{multline}

 \end{multicols}
  
%  \hrule
  
\noindent
где $\Delta_{i,l}(j) =\left( \nu^+_{i,l}\right)^j\hm- \left( \nu^-_{i,l}\right)^j$,
$0\hm< l\hm\leq m\hm\leq k_i\hm-i$.
%
Подставив~(\ref{e11-aga}) в~сумму $\sum\limits_{m=k_i-i}^\infty  (\rho^m/m!)
P\left( E_{k_i-i-l,m}\right)$, получим
\begin{multline}
\sum\limits^\infty_{m=k-i} \fr{\rho^m}{m!}\int\limits_{\nu^-_{i,l}}^{\nu^+_{i,l}}
\fr{x^{l-1}}{(l-1)!} \int\limits_x^{\nu_{i,l}^+} \fr{m!(1-y)^{m-l-1}}{(m-l-
1)!}\,dydx= \int\limits_{\nu^-_{i,l}}^{\nu^+_{i,l}} \fr{\rho^{l+1} x^{l-1}}{(l-1)!}
\int\limits_x^{\nu^+_{i,l}}\left[ \sum\limits_{m=l+1}^\infty \fr{\rho^{m-l-1}(1-
y)^{m-l-1}}{(m-l-1)!}-{}\right.\\
\left.{}- \sum\limits_{m=l+1}^{k_i-i-1} \fr{\rho^{m-l-1}(1-y)^{m-l-1}}{(m-l-1)!}
\right] dydx= \int\limits_{\nu^-_{i,l}}^{\nu^+_{i,l}} \fr{\rho^{l+1} x^{l-1}}{(l-1)!}
\int\limits_x^{\nu^+_{i,l}} \left[ 
\vphantom{\sum\limits_{m=l+1}^{k_i-i-1}}
e^{\rho(1-y)} \right.
\left.-\sum\limits_{m=l+1}^{k_i-i-1} \fr{\rho^{m-l-1}(1-y)^{m-l-1}}{(m-l-1)!}
\right] dydx={}\\
{}= \int\limits_{\nu^-_{i,l}}^{\nu^+_{i,l}} \fr{\rho^l x^{l-1}}{(l-1)!} \left[
e^{\rho(1-x)}-e^{\rho(1-\nu^+_{i,l})}\right]dx-
\int\limits_{\nu^-_{i,l}}^{\nu^+_{i,l}} \sum\limits_{m=l+1}^{k_i-i} \fr{\rho^m
x^{l-1}}{(l-1)! (m-l)!} \left[ 
%\vphantom{(1-\nu^+_{i,l})^{m-l}}
(1-x)^{m-l}-
(1-\nu^+_{i,l})^{m-l}\right]dx= {}\\
{}=
-e^{\rho(1-\nu^+_{i,l})} \fr{\rho^l}{(l-1)!}
 \int\limits_{\nu^-_{i,l}}^{\nu^+_{i,l}} x^{l-1}dx +e^\rho \fr{\rho^l}{(l-1)!} \int\limits_{\nu^-
_{i,l}}^{\nu^+_{i,l}} x^{l-1} e^{-\rho x}dx+{}\\
{}+ \sum\limits_{m=l+1}^{k_i-i} (1-\nu^+_{i,l})^{m-l} \fr{\rho^m}{(m-l)! (l-1)!}
\int\limits_{\nu^-_{i,l}}^{\nu^+_{i,l}} x^{l-1}dx- %{}\\
 \fr{\rho^m}{(m-l)! (l-1)!} \sum\limits_{m=l+1}^{k_i-i} 
 \int\limits_{\nu^-_{i,l}}^{\nu^+_{i,l}} x^{l-1} (1-x)^{m-l}dx.
\label{e14-aga}
\end{multline}
Использовав в~(\ref{e14-aga}) формулы~(\ref{e12-aga}) 
и
\begin{equation*}
\int x^{l-1}e^{-\rho x} dx = 
-\fr{e^{-\rho x}}{\rho} \left[ x^{l-1}+(l-1)!
\sum\limits_{j=1}^{l-2} \fr{x^{l-1-j}}{j!\rho^j} +\fr{(l-1)!}{\rho^{l-1}}\right]
\end{equation*}
и обозначения
\begin{equation}
\left.
\begin{array}{rl}
\Delta_{i,l}(j) &=\left( \nu^+_{i,l}\right)^j -\left(\nu^-_{i,l}\right)^j\,;\\
D_{i,l}(j) &= e^{-\rho \nu^+_{i,l}}\left( \nu^+_{i,l}\right)^j -
e^{-\rho
\nu^-_{i,l}}\left( \nu^-_{i,l}\right)^j\,,
\end{array}
\right\}
\label{e15-aga}
\end{equation}
после интегрирования получим:
\begin{multline*}
\sum\limits_{m=k-i}^\infty \fr{\rho^m}{m!} \left[ 1-(1-\nu^+_{i,0})^m\right]=
 \sum\limits_{m=0}^\infty \fr{\rho^m}{m!}\left[ 1-(1-\nu^+_{i,0})^m\right] -{}\\
 {}-
\sum\limits_{m=0}^{k_i-i-1} \fr{\rho^m}{m!}\left[ 1-(1-\nu^+_{i,0})^m\right]=
e^\rho \left(1-e^{-\rho \nu^+_{i,0}}\right) +
\sum\limits_{m=0}^{k_i-i-1} \fr{\rho^m}{m!}
\left[ \left(1 -\nu^+_{i,0}\right)^m-1\right]
\end{multline*}
при $l=0$;
\begin{multline}
\sum\limits_{m=k-i}^\infty \fr{\rho^m}{m!} P\left( E_{m-l,m}\right)
= -e^{\rho(1-\nu^+_{i,l})} \fr{\rho^l}{l!}\,\Delta_{i,l}(l)- 
e^\rho \rho^{l-1} \sum\limits_{j=0}^{l-2} \fr{D_{i,l}(l-1-j)}{(l-j)! \rho^j} -e^\rho D_{i,l}(0)+{}\\
{}+ \fr{\rho^m}{l!} \sum\limits_{m=l+1}^{k_i-i} \left( 1-\nu^+_{i,l}\right)^{m-l}
\fr{\Delta_{i,l}(l)}{(m-l)!}-
 \fr{\rho^m}{(l-1)!} \sum\limits_{m=l+1}^{k_i-i} \sum\limits_{j=0}^{m-l}
(-1)^j \fr{\Delta_{i,l}(l+j)}{j!(m-l-j)!(l+j)}
\label{e16-aga}
\end{multline}
  при $0<l\leq \min\{k_i-i,m\}$.
  
 

  Из~(\ref{e10-aga}), подставив~(\ref{e16-aga}), получим для $i\hm\leq t_0/\nu$,
$k_i\hm>1$:
\begin{equation}
q_i^{k_i}=\begin{cases}
 (k_i-i) C_0 +C_0 e^{-\lambda \nu} \displaystyle\hspace*{-2mm}\sum\limits_{m=0}^{k_i-i-1}
 \hspace*{-1mm} (m-k_i+i)\fr{\rho^m}{m!} %{}\\
- C_1 \displaystyle\hspace*{-2mm}\sum\limits_{m=0}^{k_i-i-1} \fr{\rho^m}{m!}\, 
e^{-\lambda\nu}
\displaystyle\hspace*{-1mm}\sum\limits_{l=m_i^+-1}^{\min\{ m,m_i+1\}}\hspace*{-5mm} (m-l)P\left( E_{m-l,m}\right)-{}&\\
{}- C_1 \displaystyle\sum\limits_{l=m_i^+-1}^{\min\{ k_i-i, m_i+1\}} (k_i-l+i)
\left[
\vphantom{\displaystyle\sum\limits_{m=l+1}^{k_i-i} \sum\limits_{j=0}^{m-l}}
e^{-\rho \nu^+_{i,l}} \fr{\rho^{l+1}}{l!}\,\Delta_{i,l}(l)-\right.
\fr{\rho^l}{(l-1)!}\,D_{i,l}(l-1) -{}&\\
{}-\rho^{l-1}
\displaystyle\sum\limits_{j=1}^{l-2} \fr{D_{i,l}(l-1-j)}{j!\rho^j(l-j)} %{}&\\
-\rho D_{i,l}(0)- e^{-\rho} \left(1-\nu^+_{i,l}\right)
\fr{1-(1-\nu^+_{i,l})^{k_i-i-l-1}}{\nu^+_{i,l}} \,
\fr{\rho^{l+1}}{(l-1)!}\,
\Delta_{i,l}(l)+{}&\\
\left.{}{}+e^{-\rho} \fr{\rho^{l+1}}{(l-1)!}
\displaystyle\sum\limits_{m=l+1}^{k_i-i} \sum\limits_{j=0}^{m-l} (-1)^j
\fr{(m-l)!}{j!(m-l-j)!(m-j)}\,\Delta_{i,l}(m-j)\right]
& \hspace*{-42mm}\mbox{ при } 0<i\leq k_i-1\,,\\ 
0 &  \hspace*{-42mm}\mbox{ при } i\geq k_i\,;
\end{cases}\!\!\!
\label{e17-aga}
\end{equation}
\begin{equation}
q_0^{k_0} =C_0+ q_1^{k_1}\,,
\label{e17-1-aga}
\end{equation}
где $P(E_{m-l,m})$ вычисляется по формуле~(\ref{e13-aga});
$\Delta_{i,l}(j)$ и~$D_{i,l}(j)$ определены в~(\ref{e15-aga}).



Для $i>t_0/\nu$, преобразовав~(\ref{e10-aga}), получим
\begin{equation*}
q_i^{k_i}=\begin{cases}
\displaystyle \sum\limits_{m=0}^\infty
\fr{\rho^m}{m!}\,e^{-\rho} q_i^{k_i}(\nu,m;t_0)= \left(C_0-C_1\right) \sum\limits_{m=0}^{k_i-i-1} m\fr{\rho^m}{m!}\,
e^{-\rho} +{}&\\
{}+\left( k_i-i\right) \left( C_0-C_1\right)
\displaystyle \hspace*{-2mm}\sum\limits_{m=k_i-i}^\infty \fr{\rho^m}{m!}\,e^{-\rho}= %{}&\\
%{}=
\left( C_0-C_1\right) \left[ k_i-i-\displaystyle \hspace*{-2mm}
\sum\limits_{m=0}^{k_i-i-1} \left( k_i-i-m\right)
\fr{\rho^m}{m!}\,e^{-\rho}\right] &\hspace*{-2mm}\mbox{ при } i\leq k_i-1;\\
0 & \hspace*{-2mm}\mbox{ при } i\geq k_i.
\end{cases}
%\label{e18-aga}
\end{equation*}



  Перейдем к~рассмотрению параметров~$\pi_j^{\overline{k}}$, $j\hm= 0,\ldots,
I\hm-1$, $\overline{k}\hm\geq 1$, $k_0\hm>1$, и~$\theta^{\overline{k}}$.

  Для вероятностей переходов $p_{ij}^{\overline{k}}$ справедливы формулы:
  \begin{equation}
  \left.
  \begin{array}{rl}
  p_{ij}^{\overline{k}} &=
  \begin{cases}
  r_{j-i+1}\,, &\ i\leq j\leq k_i-2\,;\\
 \displaystyle 1-\sum\limits_{l=0}^{k_i-i-1} r_l\,, &\ j=k_i-1\,,
  \end{cases}
\mbox{ если } 1\leq i\leq k_i-1;
\\[12pt]
p_{0j}^{\overline{k}}& = 
\begin{cases}
r_j\,, &\ i\leq j\leq k_0-2\,;\\[6pt]
 \displaystyle 1-\sum\limits_{l=0}^{k_0-2} r_l\,, &\ j=k_0-1\,;
\end{cases}\\[12pt]
  p_{i,i-1}^{\overline{k}}& = 
  \begin{cases}
  1\,, &\ k_i\leq i\,,\  i\geq 1\,;\\[6pt]
  r_0\,, &\ k_i>i\geq 1\,;
  \end{cases}
  \\[14pt]
  p_{ij}^k &=0 \ \ \mbox{ при }\ \ i>k_i-1\,,\ \ j>k_i-1\,,\ j<i-1\,,
  \end{array}
  \right\}
    \label{e19-aga}
  \end{equation}
  где
  \begin{equation*}
  r_l=\fr{\rho^l}{l!}\,e^{-\rho}\ \ \mbox{при\ \ } 0\leq l\leq k_i-2\,.
  \end{equation*}
  
  
  
  Для рассматриваемой цепи Маркова при стратегии~$\overline{k}$
стационарное распределение вероятностей является единственным решением
системы уравнений~\cite{8-aga}: %\sum_{\substack{{i=\overline{1,n}}\\ {j=\overline{1,l}}}}
  \begin{equation}
  \left.
  \begin{array}{c}
  \pi_j^{\overline{k}} = \sum\limits_{\substack{{i:\ 0\leq i\leq j},\\ {k_i>j}}}
  \hspace*{-2mm}\pi_i^{\overline{k}}
p_{ij}^k +\pi_{j+1}^{\overline{k}}  p_{j+1,j}^{\overline{k}}\,,\enskip j=0,\ldots, I-
2\,;\\[6pt]
     \displaystyle \sum\limits_{j=0}^{I-1} \pi_j^{\overline{k}} = 1\,.
      \end{array}
      \right\}
      \label{e20-aga}
\end{equation}

%\hrule

  \begin{multicols}{2}
  
  \noindent
Из~(\ref{e19-aga}) и~(\ref{e20-aga}) следуют рекуррентные формулы для
расчета стационарного распределения при стратегии~$\overline{k}$:
\begin{equation}
\pi_j^{\overline{k}} =\pi_0^{\overline{k}}R_j\,,\enskip j=1,\ldots, I-1\,,
\label{e21-aga}
\end{equation}
где
\begin{gather*}
\pi_0^k = \left( \sum\limits_{i=0}^{I-1} R_i\right)\,; \ R_0=1\,;\ R_1=\fr{1-
r_0}{r_0}\,;\\
R_{i+1} = \fr{1}{p_{i+1,i}^{\overline{k}}}\left( R_i - r_i-
\sum\limits_{\substack{{j:\ 0\leq j\leq i},\\
{k_j>i}}}\hspace*{-3mm} R_j r_{i-j+1}\right)\,,\\ 
\hspace*{43mm}i=1,\ldots, I-2\,.
\end{gather*}

  Для вычисления $\theta^{\overline{k}}$ рассмотрим вложенную цепь
Маркова, которую образует последовательность $\{v_n,\ n\hm\geq 0\}$, где
$v_n$~--- число заявок в~системе в~момент $t_n, t_{n+1}\hm= t_n+\nu,
t_0\hm=0$~\cite{9-aga}. Для состояний $i,j\hm= 0,\ldots, I\hm-2$ вероятности
переходов совпадают с~$p_{ij}^{\overline{k}}$ в~(\ref{e19-aga}), т.\,е.\
уравнения равновесия для состояний $i\hm= 0,\ldots, I-2$ (первые $I-1$
уравнений) совпадают с~соответствующими уравнениями в~(\ref{e20-aga}).
Обозначим стационарные вероятности состояний новой вложенной цепи
через~$\theta_j^{\overline{k}}$, $j\hm= 0,\ldots , I$. Добавив уравнения
$\sum\limits_{j=0}^I \theta_j^{\overline{k}}\hm=1$ (уравнение полной
вероят\-ности) и~$\rho\hm= 1\hm-
\theta_0^{\overline{k}}\hm+\rho\theta^{\overline{k}}$ (уравнение баланса
нагрузки), где $\theta^{\overline{k}}\hm= \sum\limits_{j:\
j=k_j}\theta_j^{\overline{k}}$, получим $I\hm+1$ независимых уравнений
равновесия для вычисления $I\hm+1$ неизвестных переменных~$\theta_j^k$,
$j\hm= 0,\ldots, I$. Аналогично~(\ref{e21-aga}) получим:
  \begin{align*}
  \theta_0^{\overline{k}} &= \fr{1}{\rho\sum\nolimits_{j:\ j\not= k_j} R_j+1}\,;\\
\theta_j^{\overline{k}}&=\theta_0^{\overline{k}} R_j\,,\ j=1,\ldots, I-1\,;\\
  \theta_I^{\overline{k}} &=1-\fr{1-\theta_0^{\overline{k}}}{\rho} -\hspace*{-3mm}\sum\limits_{j:
j=k_j,\ j\not=I}\hspace*{-3mm} \theta_j^{\overline{k}}\,,\quad \theta_j^{\overline{k}}=1-\fr{1-
\theta_0^{\overline{k}}}{\rho}\,.
  \end{align*}

  Упростим задачу~(\ref{e2-aga}), сократив класс допустимых пороговых
стратегий, а~именно: будем ниже рассматривать множество~$F_c$ статических
пороговых стратегий $\overline{k}\hm= \{ k_i,\ i\hm= 0,\ldots , I-1\}$, $k_i\hm=
k\hm>1$, $i\hm= 0,\ldots, I-1$, $I\hm=k$, и~вместо обозначения~$\overline{k}$
будем писать~$k$. Для исследования решения задачи~(\ref{e2-aga})
воспользуемся результатами теории управления марковскими
процессами~\cite{8-aga}. Пусть набор $V_j^k$, $j\hm\geq0$,~--- решение
следующей системы уравнений:

\noindent
  \begin{equation}
  \left.
  \begin{array}{c}
 \displaystyle q_i^k +\sum\limits_{j=i-1}^{k-1} r_{j-i+1} V_j^k= V_i^k +g^k\,,\\[6pt]
 \hspace*{30mm}k-1\geq i\geq 1\,;\\[6pt]
 q_0^k+ \displaystyle\sum\limits_{j=0}^{k-1} r_j V_j^k =V_0^k +g^k\,;\\[9pt]
  V_{i-1}^k = V_i^k +g^k\,,\enskip i>k-1\,.
  \end{array}
  \right\}
    \label{e22-aga}
  \end{equation}

  Отметим, что каждое уравнение в~(\ref{e22-aga}) является линейной
комбинацией остальных $k\hm-1$ уравнений, и~поэтому один из
параметров~$V_j^k$ является произвольным. Действительно, если $V_j^k$~---
решение системы, то $V_j^k\hm+const$ также являются решением. Если,
например, каждое $i$-е уравнение умножить на~$\pi_i$ для $i\hm= 1,\ldots,
k\hm-1$ и~просуммировать, то получим уравнение для $i\hm=0$.

  \smallskip

  \noindent
  \textbf{Утверждение~1.} \textit{Пусть для порогов $k\hm>0$ и~$k\hm+1\hm>0$ 
  справедлива система неравенств}
  
  \noindent
  \begin{multline}
  q_i^k -q_i^{k+1} \leq -g^k \left( 1-\sum\limits_{j=i-1}^{k-1} r_{j-
i+1}\right)\,,\\ 
0\leq i\leq k\,.
  \label{e23-aga}
  \end{multline}
\textit{Тогда если хотя бы одно неравенство в~$(\ref{e23-aga})$ является
строгим, то справедливо неравенство $g^k\hm<g^{k+1}$, иначе} $g^k\hm=
g^{k+1}$.

\smallskip

\noindent
  Д\,о\,к\,а\,з\,а\,т\,е\,л\,ь\,с\,т\,в\,о\,.\ \ Фиксируем порог~$k$ и~решим
систему уравнений~(\ref{e22-aga}). Из~(\ref{e22-aga}), проведя
преобразования, получим
  \begin{multline*}
  q_i^k +\sum\limits_{j=i-1}^{k-2} r_{j-i+1} V_j^k + \left( 1-\sum\limits_{j=i-
1}^{k-2} r_{j-i+1}\right) V_{k-1}^k={}\\
  {}= V_i^k +g^k\,,\enskip k-1\geq i\geq1\,;
  \end{multline*}
  $$
  g^k -q_i^k +\sum\limits_{j=i-1}^{k-2} r_{j-i+1}\left( V^k_{k-1}-V_j^k\right)
+\left( V_i^k -V^k_{k-1}\right) =0\,.
  $$
Положив $V^k_{k-1}=0$, получим

 \begin{figure*}[b] %fig1
       \vspace*{1pt}
 \begin{center}
 \mbox{%
 \epsfxsize=163.657mm
 \epsfbox{aga-1.eps}
 }
 \end{center}
 \vspace*{-9pt}
       \Caption{Зависимость предельного дохода от порогового значения
       ($\nu\hm=1$; $t_0\hm= 5{,}1$; $C_0\hm=2$) при $C_1\hm=1{,}2$~(\textit{а}) и~3~(\textit{б}):
       \textit{1}~--- $\lambda\hm=0{,}2$; \textit{2}~--- 0,5; \textit{3}~--- $\lambda\hm=1$}
      \end{figure*}


\noindent
\begin{align*}
V^k_{k-2} & = \fr{g^k-q^k_{k-1}}{r_0}\,;\\
V^k_{i-1} &= \fr{1}{r_0}\,
\left( \vphantom{\sum\limits_{j=i+1}^{k-2}}
g^k-q^k_{i-1} +(1-r_1)V_i^k -{}\right.\\
&\hspace*{5mm}\left.{}-\sum\limits_{j=i+1}^{k-2} r_{j-
i+1}V_j^k\right),\enskip 1\leq i\leq k-2\,.
%\label{e24-aga}
\end{align*}
Из~(\ref{e22-aga}) также получим $V^k_{i-1}\hm= V_i^k\hm+ g^k$,  т.\,е.
\begin{equation}
V_i^k= V_k^k= -g^k\,,\quad i\leq k 
\label{e25-aga}
\end{equation}
(так как $V^k_{k-1}=0$).
Далее воспользуемся теоремой~2.2 в~\cite{10-aga}. Согласно этой теореме,
если для стратегий $f\hm>0$ и~$f^\prime\hm>0$ выполняются неравенства
\begin{equation}
q_i^f +\sum\limits_j p^f_{ij} V_j^f \leq {q_i^{f}}^{\prime} + \sum\limits_j
{p_{ij}^{f}}^{\prime} V_j^f
\label{e26-aga}
\end{equation}
и хотя бы одно неравенство в~(\ref{e26-aga}) является строгим, то справедливо
неравенство $g^f\hm<{g^{f}}^{\prime}$, иначе $g^f\hm= {g^{f}}^{\prime}$.

  Рассмотрим неравенства~(\ref{e26-aga}) для порогов~$k$\linebreak и~$k\hm+1$:
  \begin{multline*}
  V_i^k +g^k =q_i^k +\sum\limits_{j=i-1}^{k-2} r_{j-i+1} V_j^k+{}\\
  {}+\left(\! 1- \hspace*{-0.5mm}
\sum\limits_{j=i-1}^{k-2}\hspace*{-1mm} r_{j-i+1}\!\right) V^k_{k-1}\leq q_i^{k+1} +\hspace*{-0.5mm}
\sum\limits_{j=i-1}^{k-1}\hspace*{-1mm} r_{j-i+1} V_j^k +{}\\
{}+\left( 1-
\sum\limits_{j=i-1}^{k-1} r_{j-i+1}\right) V_k^k\,,\enskip 0\leq i\leq k\,.
  \end{multline*}
Сократив одинаковые слагаемые в~левой и~правой части неравенств и~заменив
$V^k_{k-1}\hm=0$ и~$V_k^k\hm= -g^k$,  получим, что
неравенства~(\ref{e26-aga}) для порогов~$k$ и~$k\hm+1$ эквивалентны
неравенствам~(\ref{e23-aga}).

  Пусть $1\leq k\leq [t_0/\nu]$. При $k\hm\leq [t_0/\nu]$ выполняются равенства
$P(E_{i,l,m})\hm= 1$, $l\hm=1,\ldots, k-i$ (так как $i\hm+l\hm\leq k$). Тогда при
$k\hm \leq [t_0/\nu]$ из~(\ref{e11-aga}) (или из~(\ref{e17-aga}) и~(\ref{e17-1-aga}))
 следует:
  \begin{equation*}
  q_i^k=\begin{cases}
 \displaystyle C_0\left[ k-i-\sum\limits_{m=0}^{k-i-1}(k-i-m) \fr{\rho^m}{m!}\,e^{-\rho}\right]
&\\
&\hspace*{-40mm}\mbox{при } i=1,\ldots, k-1\,,\ k>1\,;\\
  0 &\hspace*{-40mm}\mbox{при } i\geq k\,;\end{cases}
  \end{equation*}
  \begin{equation*}
  q_0^k=\begin{cases}
  q_1^k+C_0 &\ \mbox{при } k>1\,;\\
  C_0 &\ \mbox{при } k=1\,.
  \end{cases}
  \end{equation*}
Использовав~(\ref{e26-aga}), получим:
\begin{multline*}
q_i^k -q_i^{k+1} = C_0\left( -1+\sum\limits_{j=0}^{k-i} r_j\right)\,,\\
 0\leq i\leq k\leq \left[ \fr{t_0}{\nu}\right]-1\,.
\end{multline*}
Подставив правую часть последнего равенства в~(\ref{e25-aga}), получим
неравенства
$$
C_0\left( -1+\sum\limits_{j=0}^{k-i} r_j\right) \leq -g^k\left( 1-
\sum\limits_{j=0}^{k-i} r_j\right)\,,\enskip 0\leq i\leq k\,,
$$
которые эквивалентны неравенству $C_0\hm\geq g^k$, справедливому для
любых $k\hm\geq 1$. Так как $g^1\hm= C_0$, то согласно утверждению~1 для
всех $1\hm\leq k\hm\leq [t_0/\nu]\hm-1$ выполняются равенства $g^k\hm=
g^{k+1}\hm=C_0$. Так как $\theta_k^k\hm> \theta_{k+1}^{k+1}$, то из
последних неравенств и~(\ref{e3-aga}) следует, что для оптимального
порога~$k^\prime$ выполняется неравенство $k^\prime \hm\geq [t_0/\nu]$. Для
максимального значения целевой функции получаем оценку снизу:
$$
\max\limits_{\overline{\overline{k}}\in F_c} Q^{\overline{k}} \geq C_0\left( 1-
\theta_{k_0}^{k_0}\right)\,,
$$
где $k_0\hm= [t_0/\nu]$, $\theta_{k_0}^{k_0}\hm=1- (1-\theta_0^{k_0})/\rho$,
$\theta_0^{k_0}\hm= 1/\left(\rho \sum\nolimits_{j=0}^{k_0-1} R_j+1\right)$, $R_j$, $j\hm= 0,\ldots, k_0-1$,
вы\-чис\-ля\-ют\-ся по формулам~(\ref{e21-aga}).

  Для поиска оптимального порогового значения на~$F_c$ предлагается метод
простого перебора: последовательно для $k\hm= k_0, k_0+1,\ldots$ вычислить
значение целевой функции~$Q^k$, пока не выполнится условие $Q^k\hm>
Q^{k+1}$.

\section{Примеры}

  В качестве примеров рассмотрены задачи оптимизации статической
пороговой стратегии управления доступом в~СМО типа $M/D/1$ со
следующими вариантами параметров ($\nu\hm=1$; $t_0\hm=5{,}1$; $C_0\hm=2$):
%  \begin{enumerate}[(1)]
 % \item 
 (1)~$\lambda\hm=0{,}2$; $C_1\hm= 1{,}2$ (рис.~1,\,\textit{а}, кривая~\textit{1});
  %\item 
  (2)~$\lambda\hm= 0{,}5$; $C_1\hm= 1{,}2$ (см.\ рис.~1,\,\textit{a}, кривая~\textit{2});
  %\item 
  (3)~$\lambda\hm=1$;  $C_1\hm=
1{,}2$ (см.\ рис.~1,\,\textit{а}, кривая~\textit{3});
  %\item 
  (4)~$\lambda=0{,}2$;  $C_1\hm= 3$
(см.\ рис.~1,\,\textit{б}, кривая~\textit{1});
  %\item 
  (5)~$\lambda=0{,}5$; $C_1\hm=3$
(см.\ рис.~1,\,\textit{б}, кривая~\textit{2});
  %\item 
  (6)~$\lambda=1$;  $C_1\hm=3$
(см.\ рис.~1,\,\textit{б}, кривая~\textit{3}).
%  \end{enumerate}

  На рис.~1 приведены графики зависимости предельного дохода
относительно одной поступившей заявки без учета длины интервала времени
между соседними моментами поступления.
     
  На рис.~2 приведены графики зависимости предельного дохода в~единицу
времени с~учетом длины интервала времени между моментами поступления для СМО типа $M/D/1$ 
с~параметрами ($\nu\hm=1$; $t_0\hm=5{,}1$; $C_0\hm=2$; $C_1\hm=3$):
(1)~$\lambda\hm=0{,}2$ (кривая~\textit{1}); (2)~$\lambda\hm=0{,}5$ 
(кривая~\textit{2}); (3)~$\lambda\hm=1$ (кривая~\textit{3}).
      
  Во всех рассмотренных выше примерах при $\lambda\hm=1$ значение
оптимального порога составляет $k_0\hm=5$, в~остальных случаях $k_0\hm=6$.

\begin{center}  %fig2
\vspace*{5pt}
\mbox{%
 \epsfxsize=78.026mm
 \epsfbox{aga-3.eps}
 }

\end{center}

\vspace*{-2pt}

\noindent
{{\figurename~2}\ \ \small{Зависимость предельного дохода в~единицу времени 
от значения порога ($\nu\hm=1$; $t_0\hm=5{,}1$; $C_0\hm=2$; $C_1\hm=3$):
\textit{1}~--- $\lambda\hm=0{,}2$; \textit{2}~--- 0,5; \textit{3}~--- $\lambda\hm=1$}}

 \vspace*{-6pt}

\section{Заключение}

  Исследование задачи оптимизации динамической пороговой стратегии
ограничения доступа в~СМО $M/D/1$, проведенное в~данной работе, приводит
к следующим выводам:
  \begin{enumerate}[1.]
\item Рассмотренная выше задача оптимизации~(\ref{e2-aga}) является
типичной задачей управления цепями Маркова.
\item Доказательство утверждения о том, что оптимальное решение
задачи~(\ref{e2-aga}) единственное и~принадлежит~$F_c$, остается
открытым (в~отличие от постановки задачи в~\cite{6-aga}).\\[-13pt]
\item Оптимальное решение $\overline{k}^\prime$ задачи~(\ref{e2-aga})
удовлетворяет условию $k_i^\prime \hm\geq [t_0/\nu]$, $i\hm= 1,\ldots,
I\hm-1$.\\[-13pt]
\item Численные эксперименты подтверждают гипотезу~\cite{4-aga}
о~том, что целевая функция в~задаче~(\ref{e2-aga}) является
унимодальной выпуклой функцией.
\end{enumerate}

  Результаты работы могут быть применены при исследовании систем, в~качестве модели которых может быть использована СМО $M/D/1$ (например,
фрагментов компьютерных сетей) с~целью повышения их эффективности.

\vspace*{-12pt}

    {\small\frenchspacing
 {%\baselineskip=10.8pt
 \addcontentsline{toc}{section}{References}
 \begin{thebibliography}{99}
\bibitem{1-aga}
\Au{Floyd S., Jacobson V.} Random early detection gateways for congestion
avoidance~// IEEE/ACM Trans. Network., 1993. Vol.~1. No.\,4. P.~397--413.
\bibitem{2-aga}
\Au{Welzl M.} Network congestion control.~--- New York, NY, USA: Wiley,
2005. 282~p.
\bibitem{7-aga} %3
\Au{Агаларов Я.\,М., Соколов И.\,А.} Динамическая стратегия распределения
буферной памяти АТМ коммутатора~// Информационные технологии и~вычислительные системы, 2008. №\,3, С.~14--21.

\bibitem{5-aga} %4
\Au{Hong Y., Huang C., Yan~J.} A~comparative study of SIP overload control
algorithms network and traffic engineering in emerging distributed computing
applications~// IGI Global, 2012. 20~p. {\sf
http://arxiv.org/ftp/ arxiv/papers/1210/1210.1505.pdf}.

\bibitem{3-aga} %5
\Au{Печинкин А.\,В., Разумчик Р.\,В.} Стационарные характеристики
системы $M2\vert G\vert 1\vert r$ с~гистерезисной политикой управления
интенсивностью входящего потока~// Информационные процессы, 2013.
Т.~3. №\,3. С.~125--140.
\bibitem{4-aga} %6
\Au{Коновалов М.\,Г.} Об одной задаче оптимального управ\-ле\-ния нагрузкой на
сервер~// Информатика и~её применения, 2013. Т.~7. Вып.~4. С.~34--43.

\bibitem{6-aga} %7
\Au{Гришунина Ю.\,Б.} Оптимальное управление очередью в~системе
$M\vert G\vert 1\vert \infty$ с~возможностью ограничения приема заявок~//
Автоматика и~телемеханика, 2015. №\,3. С.~79--93.

\bibitem{8-aga}
\Au{Карлин С.} Основы теории случайных процессов~/
Пер. с~англ.~--- М.: Мир, 1971. 536~с.
(\Au{Karlin~S.}  {A~first course in stochastic processes}.~---
New York and London: Academic Press, 1968. 502~p.)
\bibitem{9-aga}
\Au{Бочаров П.\,П., Печинкин А.\,В.} Теория массового обслуживания.~---
М.: РУДН, 1995. 529~с.
\bibitem{10-aga}
\Au{Майн Х., Осаки С.} Марковские процессы принятия решений~~/
Пер. с~англ.~--- М.: Наука, 1977. 176~с.
(\Au{Mine~H., Osaki~S.} {Markovian decision processes}.~---
New York, NY, USA: American Elsevier Publishing Co., 1970. 142~p.)
 \end{thebibliography}

 }
 }

\end{multicols}

\vspace*{-12pt}

\hfill{\small\textit{Поступила в~редакцию 21.05.15}}

\newpage

%\vspace*{12pt}

%\hrule

%\vspace*{2pt}

%\hrule

\vspace*{-24pt}

\def\tit{THE THRESHOLD STRATEGY FOR~RESTRICTING ACCESS 
IN~THE~$M/D/1$ QUEUEING SYSTEM WITH~PENALTY FUNCTION FOR~LATE SERVICE}

\def\titkol{The threshold strategy for restricting access 
in the $M/D/1$ queueing system with penalty function for late service}

\def\aut{Ya.\,M.~Agalarov}

\def\autkol{Ya.\,M.~Agalarov}

\titel{\tit}{\aut}{\autkol}{\titkol}

\vspace*{-9pt}


\noindent
\noindent
Institute of Informatics Problems,
Federal Research Center ``Computer Science and Control'' of
the Russian Academy of Sciences, 44-2 Vavilov Str.,
Moscow 119333, Russian Federation


\def\leftfootline{\small{\textbf{\thepage}
\hfill INFORMATIKA I EE PRIMENENIYA~--- INFORMATICS AND
APPLICATIONS\ \ \ 2015\ \ \ volume~9\ \ \ issue\ 3}
}%
 \def\rightfootline{\small{INFORMATIKA I EE PRIMENENIYA~---
INFORMATICS AND APPLICATIONS\ \ \ 2015\ \ \ volume~9\ \ \ issue\ 3
\hfill \textbf{\thepage}}}

\vspace*{3pt}


\Abste{The paper considers the problem of optimizing the access control on 
a~set of dynamic threshold strategies in an~$M/D/1$ system. If the number 
of concurrent requests in a system is more than the threshold, then the 
system stops accepting requests. If the number of requests is less or equal 
to this value, then the system resumes accepting requests. As a target function, 
the average value of the marginal revenue obtained per time unit in the stationary 
mode is used. It is assumed that the system receives a fixed fee for each accepted 
request and pays a fixed penalty for each overdue service of a request. The system 
does not receive a fee and does not pay a penalty for each rejected request. 
Estimates of the optimal value of the target function and the optimal threshold 
value are obtained.}


\KWE{queueing system; threshold strategy; optimization}

\DOI{10.14357/19922264150306}

\vspace*{-9pt}

\Ack
\noindent
The work was supported by the Russian Foundation for Basic
Research (projects 15-07-03406 and 13-07-00223).



%\vspace*{3pt}

  \begin{multicols}{2}

\renewcommand{\bibname}{\protect\rmfamily References}
%\renewcommand{\bibname}{\large\protect\rm References}

{\small\frenchspacing
 {%\baselineskip=10.8pt
 \addcontentsline{toc}{section}{References}
 \begin{thebibliography}{99}

\bibitem{1-aga-1}
\Aue{Floyd, S., and V.~Jacobson}. 1993.
Random early detection gateways for congestion avoidance.
\textit{IEEE ACM Trans. Network.} 1(4):397--413.
\bibitem{2-aga-1}
\Aue{Welzl, M.} 2005. \textit{Network congestion control}.  New
York, NY: Wiley. 282~p.

\bibitem{7-aga-1} %3
\Aue{Agalarov, Ya.\,M., and I.\,A.~Sokolov}. 2008. Dinamicheskaya strategiya
raspredeleniya bufernoy pamyati ATM kommutatora [Dynamic allocation strategy
of buffer memory in ATM commutator]. \textit{Informatsionnye Tekhnologii
i~Vychislitel'nye Sistemy} [Information Technology and Computer Systems] 3:14--21.

\bibitem{5-aga-1} %4
\Aue{Hong, Y., C. Huang, and J.~Yan}. 2012.
A~comparative study of SIP overload control algorithms network and traffic
engineering in emerging distributed computing applications.
\textit{IGI Global}. 20~p. Available at:
{\sf http://arxiv.org/ftp/arxiv/papers/1210/1210.1505.pdf} (accessed June~24, 2015).

\bibitem{3-aga-1} %5
\Aue{Pechinkin, A.\,V., and R.\,V.~Razumchik}.
2013. Sta\-tsi\-o\-nar\-nye kharakteristiki sistemy $M2\vert G\vert 1\vert r$
s~gisterezisnoy politikoy upravleniya intensivnost'yu vkhodyashchego
potoka [Stationary characteristics of $M2\vert G\vert 1\vert r$
queue with hysteric control policy of arrival rate].
\textit{Informatsionnye Protsessy} [Information Processes] 3(3):125--140.
\bibitem{4-aga-1} %6
\Aue{Konovalov, M.\,G.} 2013. Ob odnoy zadache optimal'nogo upravleniya
nagruzkoy na server  [About one task of overload control].
\textit{Informatika i~ee Primeneniya}~--- \textit{Inform. Appl}. 7(4):34--43.

\bibitem{6-aga-1} %7
\Aue{Grishunina, Ju.\,B.} 2015.
Optimal'noe upravlenie ochered'yu v sisteme $M\vert G\vert 1\vert \infty$
s~vozmozhnost'yu ogranicheniya priema zayavok [Optimal control of queue in the
$M\vert G\vert 1\vert \infty$ system with possibility of customer admission
restriction]. \textit{Avtomatika i~Telemekhanika} [Automation and Remote Control]
3:79--93.

\bibitem{8-aga-1}
\Aue{Karlin, S.} 1968. \textit{A~first course in stochastic processes}.
New York and London: Academic Press. 502~p.
\bibitem{9-aga-1}
\Aue{Bocharov, P.\,P., and A.\,V.~Pechinkin}. 1995.
\textit{Teoriya massovogo obsluzhivaniya} [Queueing theory]. Moscow: RUDN. 529~p.
\bibitem{10-aga-1}
\Aue{Mine, H., and S. Osaki}. 1970. \textit{Markovian decision processes}.
New York, NY: American Elsevier Publishing Co. 142~p.

\end{thebibliography}

 }
 }

\end{multicols}

\vspace*{-9pt}

\hfill{\small\textit{Received May 21, 2015}}

\Contrl

\noindent
\textbf{Agalarov Yaver M.} (b.\ 1952)~--- Candidate of Science (PhD) in technology, associate
professor; leading scientist, Institute of Informatics Problems, 
Federal Research Center ``Computer
Science and Control'' of the Russian Academy of Sciences, 44-2 Vavilov Str., Moscow 119333,
Russian Federation;  agglar@yandex.ru


\label{end\stat}


\renewcommand{\bibname}{\protect\rm Литература}