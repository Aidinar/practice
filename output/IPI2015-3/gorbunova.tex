\def\stat{gorbunova}

\def\tit{АППРОКСИМАЦИЯ ВРЕМЕНИ ОТКЛИКА СИСТЕМЫ ОБЛАЧНЫХ 
ВЫЧИСЛЕНИЙ$^*$}

\def\titkol{Аппроксимация времени отклика системы облачных 
вычислений}

\def\aut{А.\,В.~Горбунова$^1$, И.\,С.~Зарядов$^2$, С.\,И.~Матюшенко$^3$, 
К.\,Е.~Самуйлов$^4$, С.\,Я.~Шоргин$^5$ }

\def\autkol{А.\,В.~Горбунова, И.\,С.~Зарядов, С.\,И.~Матюшенко и~др.} 
%К.\,Е.~Самуйлов, С.\,Я.~Шоргин}

\titel{\tit}{\aut}{\autkol}{\titkol}

{\renewcommand{\thefootnote}{\fnsymbol{footnote}} \footnotetext[1]
{Работа выполнена при частичной финансовой поддержке РФФИ (проект 15-07-03051).}}


\renewcommand{\thefootnote}{\arabic{footnote}}
\footnotetext[1]{Российский университет дружбы народов, avgorbunova@rambler.ru}
\footnotetext[2]{Российский университет дружбы народов, izaryadov@sci.pfu.edu.ru}
\footnotetext[3]{Российский университет дружбы народов, matushenko@list.ru}
\footnotetext[4]{Российский университет дружбы народов, ksam@sci.pfu.edu.ru}
\footnotetext[5]{Федеральный исследовательский центр <<Информатика и~управление>> Российской академии наук, 
sshorgin@ipiran.ru}

\vspace*{-6pt}

\Abst{Облачные вычисления представляют собой эволюционную технологию, позволяющую 
удаленному пользователю получать доступ к~различным ресурсам посредством 
интернет-сервисов. В~статье рассматривается система облачных вычислений, в~которую поступают 
сложные пользовательские запросы, состоящие из нескольких подзапросов. Для обработки 
каждого подзапроса выделяется одна единица ресурса. Точная оценка эффективности 
облачных вычислительных ресурсов является необходимым условием обеспечения 
требуемого качества обслуживания. Время отклика системы, служащее одним из важнейших 
показателей этого качества, требует адекватного моделирования. Для его анализа построена 
математическая модель функционирования облачной системы в~виде системы массового 
обслуживания (СМО), состоящей из нескольких подсистем, на которую поступает единый 
пуассоновский поток запросов. Каждый из поступивших запросов в~момент поступления 
делится на несколько частей, обслуживаемых своей подсистемой массового обслуживания с~
накопителем неограниченной емкости и~экспоненциальным распределением времени 
обслуживания на приборе. С~по\-мощью данной математической модели получена 
аналитическая формула для нахождения приближения среднего значения и~дисперсии 
времени отклика системы~--- максимального времени обработки составляющих запрос 
подзапросов. Для оценки точности приближения выполнено имитационное моделирование 
средствами GPSS (General Purpose Simulation System).}

\KW{система облачных вычислений; время отклика; обработка сложных запросов; система 
массового обслуживания; аппроксимация; имитационное моделирование}

\DOI{10.14357/19922264150304}

%\vspace*{-6pt}

\vskip 12pt plus 9pt minus 6pt

\thispagestyle{headings}

\begin{multicols}{2}

\label{st\stat}

\section{Введение}

  Облачные вычисления (cloud computing) представляют собой технологию, 
позволяющую удаленному пользователю в~режиме реального времени по 
требованию получать доступ к~вычислительным ресурсам (программным 
приложениям, серверам, устройствам хранения данных, сервисам и~др.) через 
Интернет в~рамках согласованного качества обслуживания для выбранной 
ценовой категории. В~задачи поставщика таких услуг входит не только 
обеспечение требуемого уровня оказываемых услуг, но и, как следствие, 
избежание перегрузки ресурсов при обработке пользовательских 
требований~[1--3].
  
  Рассмотрим облачный центр, состоящий из нескольких физических машин, 
которые резервируются пользователями в~порядке поступления запросов и~
могут использоваться совместно для их обработки посредством метода 
виртуализации. Разделение доступа к~данным, виртуальным машинам и~
контроль их использования представляют собой основные задачи поставщика 
облачных услуг. Если же говорить о качестве обслуживания, то его показателем 
в большинстве случаев считается время отклика системы, т.\,е.\ время 
обработки запроса пользователя~[4]. 

В~статье рассматриваются сложные 
запросы пользователей, содержащие несколько задач, для обработки каждой из 
которых требуется одна виртуальная машина: посылаемый запрос разбивается 
на~$K$~подзапросов и~для его обработки выделяется~$K$~виртуальных 
машин. Запрос считается выполненным, т.\,е.\ пользователь получает отклик 
системы, после обработки всех его подзапросов. 

Таким образом, время отклика 
системы облачных вычислений определяется как максимальное время 
обработки составляющих его заданий или подзапросов~[5, 6].
  
  Следует отметить, что основой для построения описываемой аналитической 
модели послужила так называемая fork-join система (система с~расщеплением), 
которая, в~свою очередь, является естественной моделью для многих реальных 
систем. В~качестве примеров можно привести сферу обслуживания, 
медицинские приложения, производственные системы и~др.~\cite{7-gor}. Если 
же говорить об информационных технологиях, то прежде всего стоит 
упомянуть распределенные вычисления, технология которых оказала 
значительное влияние на концепцию облачных вычислений, а~также 
параллельные вычисления, например обработку пакетов данных, телефонных 
вызовов и~др.~[8].
  
  Но, несмотря на довольно большой перечень прикладных задач, которые 
могут быть решены с~помощью такого рода модели, на сегодняшний день 
точные аналитические результаты для среднего времени отклика существуют 
только для случая расщепления на два подзапроса, причем с~учетом некоторых 
дополнительных ограничений, а~именно: условия равенства интенсивностей 
обслуживания~[9]. Для случая же $K\hm>2$ большинство авторов в~своих 
работах концентрируются непосредственно на различных методах 
аппроксимации искомой величины~\cite{7-gor, 10-gor, 11-gor}. Это связано со 
сложностями, возникающими при поиске точного аналитического решения в~
рамках данной модели из-за существующей зависимости между очередями 
подзапросов вследствие общих моментов поступления. Отметим, что таких 
сложностей не возникает для случая, когда каждая подсистема состоит из 
бесконечного числа приборов~[12, 13].
  
  В большинстве статей из интуитивных соображений аппроксимация 
основывается на допущении о~$K$~параллельно функционирующих и~
независимых системах массового обслуживания, на вход которых поступают 
независимые потоки подзапросов. Подробный обзор существующих 
публикаций по данной тематике представлен в~\cite{14-gor}. В~данной работе 
с помощью анализа системы уравнений равновесия (СУР) получено логическое 
обоснование причин такого перехода и~приведен достаточно подробный 
сравнительный анализ полученных результатов применительно к~системе 
облачных вычислений.
  
  Статья организована следующим образом: в~разд.~2 описывается общая 
математическая модель функционирования рассматриваемой облачной 
системы, в~разд.~3 проводится анализ ее времени отклика, в~разд.~4 
представлены численные результаты и, наконец, в~разд.~5 кратко подведены 
итоги и~сформулированы дальнейшие планы исследований.

\vspace*{-6pt}

\section{Математическая модель}

\vspace*{-3pt}

  Итак, рассмотрим СМО с~$K$~независимыми 
параллельно работающими приборами, каждый из которых имеет накопитель 
неограниченной ем\-кости. В~эту систему поступает пуассоновский по-\linebreak ток заявок 
(запросов) с~интенсивностью~$\lambda$. При этом в~момент поступления 
заявка делится на~$K$ подзапросов от 1-го до $K$-го типа и~посылает их\linebreak 
в~каж\-дую из~$K$~очередей. Заявка считается обслуженной в~момент окончания 
обслуживания последнего из составляющих ее подзапросов. Време\-на 
обслужива\-ния заявок-под\-за\-про\-сов являются независимыми случайными 
величинами, распределенными по экспоненциальному закону 
с~па\-ра\-мет\-ром~$\mu_k$ для подзапроса $k$-го типа, $k\hm=\overline{1,K}$.
  
  Пусть $x_k(t)$~--- число подзапросов $k$-го типа, находящихся в~системе 
в~момент времени~$t$. Обозначим $X(t)\hm= (x_1(t),\ldots, x_K(t))$. 
Случайный процесс $\{ X(t),\ t\hm\geq 0\}$, описывающий поведение системы 
во времени, является марковским (МП) с~множеством состояний $X\hm= \{ 
\vec{n}{:=}( n_1, \ldots , n_k, \ldots , n_K)$, $n_1\hm\geq 0,\ldots , n_k\hm\geq 
0,\ldots , n_K\hm\geq 0\}$.
  
  В предположении о его существовании стационарное распределение должно 
удовлетворять следующей СУР:
  \begin{multline*}
  \left( \lambda +\sum\limits_{k=1}^K \mu_k u\left( n_k\right) \right) p\left( 
\vec{n}\right) ={}\\
{}=\lambda p\left( \vec{n}-\vec{1}\right) \prod\limits_{k=1}^K u\left( 
n_k\right) +\sum\limits_{k=1}^K \mu_k p\left( \vec{n}+\vec{e}_k\right),\enskip 
\vec{n}\in X,\hspace*{-1.6638pt}
  \end{multline*}
где $\vec{e}_k$~--- вектор, все элементы которого равны нулю, кроме $k$-го, 
который равен~1, $\vec{1}$~--- единичный вектор и
$$
u\left( n_k\right) = \begin{cases}
0\,, &\ \mbox{если } n_k\leq0\,;\\
1\,, &\ \mbox{если } n_k>0\,.
\end{cases}
$$ 
  
  Здесь не исследуется очевидное условие эргодичности рассматриваемой 
системы и~считается далее, что $\rho= \max\limits_{1\leq k\leq K} \rho_k<1$, где 
$\rho_k\hm= \lambda/\mu_k$, $k\hm=\overline{1,K}$.

  \begin{figure*}[b] %fig1
         \vspace*{18pt}
\begin{minipage}[t]{79mm}
 \begin{center}
 \mbox{%
% \epsfxsize=77.81mm
% \epsfbox{gor-1.eps}
 \epsfxsize=73.4mm
 \epsfbox{gor-1t.eps}
 }
 \end{center}
 \vspace*{-6pt}
  \Caption{Среднее время отклика: 
  \textit{1}~--- аналитическая модель; \textit{2}~--- имитационная модель}
%  \end{figure*}
%  \begin{figure*} %fig2
\end{minipage}
\hfill
            \vspace*{1pt}
            \begin{minipage}[t]{79mm}
 \begin{center}
 \mbox{%
 \epsfxsize=77.9mm
 \epsfbox{gor-3.eps}
 }
 \end{center}
 \vspace*{-6pt}
  \Caption{Среднеквадратическое отклонение времени отклика: 
  \textit{1}~--- аналитическая модель; \textit{2}~--- имитационная модель}
\end{minipage}
  \end{figure*}
  
\section{Анализ времени отклика системы}

  Исследуя СУР, замечаем, что суммированием ее уравнений по всем 
индексам, кроме $n_k$-го, приходим к~СУР, соответствующей 
СМО $M_\lambda/M_{\mu_k}/1$. Следовательно, можем 
получить выражения для маргинальных вероятностей макросостояний МП 
$X(t)$: 
  $$
  p_{n_k} =\left( 1-\rho_k\right) \rho_k^{n_k}\,,\enskip n_k\geq 0\,.
  $$
  
  Далее выписываем функцию распределения времени отклика с~учетом 
допущения о независимости времен пребывания подзапросов в~под\-сис\-те\-мах:
  \begin{multline*}
  {\sf P}\left( \eta<y\right) ={\sf P}\left( \max \left( \xi_1,\ldots , \xi_K\right) <y\right) ={}\\[2pt]
  {}=
{\sf  P}\left( \xi_1<y,\ldots , \xi_K<y\right) ={}\\[2pt]
  {}=
  {\sf P}\left( \xi_1<y\right) \cdots P\left( \xi_K<y\right) ={}\\[2pt]
  {}= \left( 1-e^{-(\mu_1-\lambda)y}\right) \cdots \left( 1- e^{-(\mu_K-
\lambda)y}\right)\,.
  \end{multline*}
    Тогда среднее время отклика и~его дисперсия будут иметь вид:
  \begin{multline*}
{\sf  E}\eta = \sum\limits^K_{l=1} \fr{1}{\mu_l-\lambda} - \sum\limits_{l\ne m} 
\fr{1}{\mu_l+\mu_m-2\lambda} +{}\\
{}+\sum\limits_{l\ne m\ne n} \fr{1}{\mu_l 
+\mu_m+\mu_n-3\lambda}+\cdots{}\\
  {}\cdots + (-1)^{2K-1} \fr{1}{\mu_1+\mu_2+\cdots+\mu_K-K\lambda}\,;
  \end{multline*}
  
  \vspace*{-12pt}
  
  \noindent
  \begin{multline*}
{\sf   D}\eta = \sum\limits^K_{l=1} \fr{2}{(\mu_l-\lambda)^2} -\sum\limits_{l\ne m} 
\fr{2}{(\mu_l+\mu_m-2\lambda)^2} +{}\\
{}+ \sum\limits_{l\ne m\ne n} 
\fr{2}{(\mu_l+\mu_m+\mu_n-3\lambda)^2}+\cdots{}\\
  {}\cdots+ (-1)^{2K-1} \fr{2}{(\mu_1+\mu_2+\cdots+ \mu_K - K\lambda)^2} -{}\\
  {}-
  \left( \sum\limits^K_{l=1} \fr{1}{\mu_l-\lambda}- \sum\limits_{l\ne m} 
\fr{1}{\mu_l+\mu_m-2\lambda}+{}\right.
\end{multline*}

\noindent
  \begin{multline*}
   {}+ \sum\limits_{l\ne m\ne n} \fr{1}{\mu_l+\mu_m+\mu_n-3\lambda} 
+\cdots{}\\
\left.{}\cdots + (-1)^{2K-1} \fr{1}{\mu_1 +\mu_2 +\cdots + \mu_K-K\lambda}
\vphantom{\sum\limits^K_{l=1} \fr{1}{\mu_l-\lambda}}
\right)^2\,.
  \end{multline*}

  Кроме того, можно записать выражения и~для других характеристик СМО, 
используя аналогичные результаты для системы типа  $M/M/1$.

\vspace*{-6pt}

\section{Численный пример}

\vspace*{-3pt}

  Для оценки полученной аппроксимации с~помощью программных средств 
GPSS была построена имитационная модель. Поскольку интерес представляют 
характеристики системы в~стационарном режиме ее работы, имитационное 
моделирование проводилось до тех пор, пока не стабилизировались параметры 
модели. Стабилизация наступила после прохождения через систему около 
500\,000 запросов.
  

  
  На рис.~1 и~2 представлены графики зависи\-мости среднего 
  и~среднеквадратического отклонения времени отклика системы облачных 
вычислений от интенсивности поступления запросов~$\lambda$. Расчеты были 
выполнены для случая расщепления пользовательского запроса на 
3~подзапроса ($K\hm=3$) для симметричного случая $\mu_1\hm= \mu_2\hm= 
10$~с$^{-1}$ при меняющейся интенсивности входящего потока~$\lambda$. 
Видно, что полученная с~помощью аналитической модели оценка среднего 
времени отклика является оценкой <<сверху>>, а~среднеквадратического 
отклонения~--- оценкой <<снизу>>. Заметим, что для несимметричного случая 
качественных изменений численных результатов не наблюдается.

\begin{figure*}
\vspace*{1pt}
\begin{minipage}[t]{79mm}
\begin{center}  %fig3
\mbox{%
 \epsfxsize=77.81mm
 \epsfbox{gor-2.eps}
 }
\end{center}
 \vspace*{-9pt}
\Caption{Математическое ожидание (сплошные кривые)
 и~среднеквадратическое отклонение (штриховые кривые) времени отклика: 
 \textit{1}~--- экспоненциальное обслуживание; \textit{2}~--- детерминированное обслуживание}
\end{minipage}
\hfill
%\begin{figure*} %fig4
            \vspace*{1pt}
            \begin{minipage}[t]{79mm}
 \begin{center}
 \mbox{%
 \epsfxsize=77.707mm
 \epsfbox{gor-4.eps}
 }
 \end{center}
 \vspace*{-9pt}
\Caption{Погрешность аппроксимации для среднего времени отклика: \textit{1}~--- 
$K\hm=2$;
\textit{2}~--- 3; \textit{3}~--- 4; \textit{4}~--- 5; \textit{5}~--- 6; \textit{6}~--- 
$K\hm=7$}
\end{minipage}
\end{figure*}

\begin{table*}\small
  \begin{center}
  \begin{tabular}{|l|c|c|c|c|c|c|}
  \multicolumn{7}{c}{Квантили случайной величины времени отклика системы}\\[6pt]
   \hline
  \multicolumn{1}{|c|}{Квантиль случайной величины} &\multicolumn{6}{c|}{Значения 
квантилей в~зависимости от нагрузки $\rho$}\\
\cline{2-7}
  \multicolumn{1}{|c|}{времени отклика}&\ \ 0,10\ \  &\ \  0,30\ \  &\ \  0,50\ \  &
  \ \  0,70\ \  &\ \  0,80\ \  &0,90\\
\hline
Квантиль уровня 0,97&0,51&0,66&0,92&1,53&2,30&4,60\\
Квантиль уровня 0,98&0,56&0,71&1,00&1,67&2,50&5,00\\
Квантиль уровня 0,99&0,63&0,81&1,14&1,90&2,85&5,70\\
Квантиль уровня 0,999&0,89&1,14&1,60&2,67&4,00&8,01\\
Квантиль уровня 0,9999&1,15&1,47&2,06&3,44&5,15& 10,31\hphantom{9}\\
\hline
\end{tabular}
\end{center}
\vspace*{4pt}
\end{table*}
 

  Рисунок~3 иллюстрирует разницу в~значениях времени отклика, полученных 
с~по\-мощью имитационного моделирования, для случаев экспоненциального 
и~детерминированного обслуживания. С~увеличением нагрузки математическое 
ожидание времени отклика при экспоненциальном обслуживании возрастает 
в~среднем в~два раза по сравнению с~детерминированным обслуживанием. 
Заметим, что аналогичный результат аналитически следует из формулы  
Пол\-ла\-че\-ка--Хин\-чи\-на для СМО $M/M/1$ и~$M/D/1$.
   
   
  
  На рис.~4 изображены графики для погрешности аппроксимации 
математического ожидания времени отклика в~зависимости от интенсивности 
входящего потока~$\lambda$ для различных значений $K\hm= \overline{2,7}$. 
Видно, что с~увеличением числа подзапросов и~интенсивности входящего 
потока увеличивается ошибка приближения.
  
  
  
  В таблице представлены значения верхней границы для времени отклика 
  с~заданными вероятностями (квантили функции распределения времени отклика) 
для различных значений интенсивности нагрузки~$\rho$ в~случае $K\hm=3$. 
Результаты, отраженные в~таблице, подтверждают сделанное выше замечание 
об увеличении значений времени отклика с~ростом нагрузки.

\begin{center}  %fig5
\vspace*{-1pt}
 \mbox{%
 \epsfxsize=78.908mm
 \epsfbox{gor-5.eps}
 }

\end{center}

\vspace*{-6pt}

\noindent
{{\figurename~5}\ \ \small{Графики эмпирической~(\textit{1}) и~аналитической~(\textit{2}) функций 
распределения}}
 
 \vspace*{15pt}
  

  
  На рис.~5 приведены графики эмпирической функции распределения, 
построенной по данным имитационного моделирования, и~аналитической 
функции распределения для времени отклика. 

Из рисунка следует, что 
эмпирическая функция распределения аппроксимирует аналитическую 
функцию распределения сверху, что сочетается с~результатами, 
иллюстрируемыми рис.~1.
  
\section{Заключение}

  В статье рассмотрена и~проанализирована модель системы облачных 
вычислений, в~которую поступают сложные запросы. Получена формула для 
расчета приближения времени отклика системы, реализована имитационная 
модель и~проведен сравнительный анализ.
  
  Помимо получения изложенных в~предыдущем разделе численных 
результатов с~помощью имитационного моделирования исследовалась цепь\linebreak 
Маркова, вложенная по моментам поступления запросов. Полученные 
значения стационарных вероятностей вложенной цепи Маркова оказались близкими 
к~значениям стационарных вероятностей по МП, поэтому не следует ожидать, 
что полученная аппроксимация не точна из-за того, что вывод формул для 
характеристик времени отклика основывался именно на вероятностях по 
процессу. Также были вычислены попарные корреляции между очередями. 
Эксперимент проводился на ограниченном объеме данных, но можно увидеть, 
что корреляция положительна и~составляет около 50\%, т.\,е.\ очереди 
коррелированы и~поэтому расчет времени отклика по маргинальным 
распределениям фактически означает расчет в~предположении об их 
независимости. Но из полученных значений для корреляции следует, что 
очереди зависимы, и~поэтому аналитически получается аппроксимация, которая 
дает положительный с~точки зрения инженерных расчетов результат. 
Погрешность в~параметрах, близких к~реальным, составляет не более 14\%.
  
  В~дальнейшем планируется улучшить аналитические и~имитационные 
результаты, а~также исследовать модели с~другим типом входящего потока 
и~иными законами распределения времени обслуживания, провести расчет их 
основных ве\-ро\-ят\-но\-ст\-но-вре\-мен\-н$\acute{\mbox{ы}}$х характеристик 
и показателей качества обслуживания.

{\small\frenchspacing
 {%\baselineskip=10.8pt
 \addcontentsline{toc}{section}{References}
 \begin{thebibliography}{99}
\bibitem{1-gor}
\Au{Xiong K., Perros H.} Service performance and analysis in cloud computing~// 
5th IEEE World Congress on Services (Services-1 '09) Proceedings.~--- Los 
Angeles: IEEE, 2009. P.~693--700.
\bibitem{3-gor} %2
\Au{Buyya R., Broberg J., Goscinski~A.\,M.} Introduction to cloud computing~// Cloud 
computing: Principles and paradigms.~--- John Wiley\,\&\,Sons, 2011. P.~3--42.
\bibitem{2-gor} %3
\Au{Satyanarayana А., Suresh Varma~P., Rama Sundari~M.\,V., Sarada Varma~P.} 
Performance analysis of cloud computing under non homogeneous conditions~// 
Int. J.~Adv. Res. Comput. Sci. Softw. Eng., 2013. Vol.~3. P.~969--974.

\bibitem{4-gor}
\Au{Khazae H., Misic J., Misic~V.\,B.} A~fine-grained performance model of cloud 
computing centers~// IEEE Trans. Parall. Distr. Syst., 
2012. Vol.~24. P.~2138--2147.
\bibitem{5-gor}
\Au{Мокров Е.\,В., Самуйлов~К.\,Е.} Модель системы облачных вычислений в~виде 
системы массового обслуживания с~несколькими очередями и~с групповым 
поступлением заявок~// T-comm~--- Телекоммуникации и~транспорт, 2013. Т.~7. 
№\,11. С.~139--141.
\bibitem{6-gor}
\Au{Basharin G.\,P., Gaidamaka~Yu.\,V., Samouylov~K.\,E.}
Mathematical theory of 
teletraffic and its application to the analysis of multiservice communication of 
next generation networks~// Autom. Control Comp. Sci., 2013. 
Vol.~47. No.\,2. P.~62--69.
\bibitem{7-gor}
\Au{Kemper B., Mandjes~M.} Mean sojourn time in two-queue fork-join systems: Bounds 
and approximations~// OR Spectrum, 2012. Vol.~34. P.~723--742.
\bibitem{8-gor}
\Au{Ko S.\,S., Serfozo R.\,F.} 
Response times in $M/M/s$ Fork-join networks~// Adv. 
Appl. Probab., 2004. Vol.~36. No.\,3. P.~854--871.
\bibitem{9-gor}
\Au{Nelson R., Tantawi A.\,N.} Approximate analysis of fork/join synchronization in 
parallel queues~// IEEE Trans. Comput., 1988. Vol.~37. P.~739--743.
\bibitem{10-gor}
\Aue{Baccelli F., Makowski~A.\,M., Shwartz~A.} The fork-join queue and related 
systems with synchronization constraints: Stochastic ordering and computable 
bounds~// Adv. Appl. Probab., 1989. Vol.~21. No.\,3. P.~629--660.
\bibitem{11-gor}
\Au{Kness C.} On the diffusion approximation to a fork and join queueing model~// 
\textit{SIAM J. Appl. Math.}, 1991. Vol.~51. P.~160--171.
\bibitem{12-gor}
\Au{Ивановская И.\,А., Моисеева~С.\,П.} Исследование математической модели 
параллельного обслуживания заявок смешанного типа~// Известия Томского 
политехнического ун-та. Управление, вычислительная техника и~информатика, 2010. 
Т.~317. №\,5. С.~32--34.
\bibitem{13-gor}
\Au{Ивановская И.\,А., Моисеева~С.\,П.} Исследование модели параллельного 
обслуживания кратных заявок в~нестационарном режиме~// Вестник Томского 
гос. ун-та. Управление, вычислительная техника 
и~информатика, 2010. №\,3(12). С.~21--28.
\bibitem{14-gor}
\Au{Thomasian A.} Analysis of fork-join and related queueing systems~// ACM 
Comput. Surv. (CSUR), 2014. Vol.~47. No.\,2. P.~1--71.
 \end{thebibliography}

 }
 }

\end{multicols}

\vspace*{-3pt}

\hfill{\small\textit{Поступила в~редакцию 21.07.15}}

\newpage

\vspace*{-24pt}

%\hrule

%\vspace*{2pt}

%\hrule

%\vspace*{12pt}

\def\tit{THE APPROXIMATION OF RESPONSE TIME OF~A~CLOUD COMPUTING SYSTEM}

\def\titkol{The approximation of response time of a cloud computing system}

\def\aut{A.\,V.~Gorbunova$^1$, I.\,S.~Zaryadov$^1$, S.\,I.~Matyushenko$^1$, 
K.\,E.~Samouylov$^1$,  and~S.\,Ya.~Shorgin$^2$}

\def\autkol{A.\,V.~Gorbunova, I.\,S.~Zaryadov, S.\,I.~Matyushenko, et al.}

\titel{\tit}{\aut}{\autkol}{\titkol}

\vspace*{-9pt}

\noindent
$^1$Peoples' Friendship University of Russia,
6 Miklukho-Maklaya Str., Moscow 117198, Russian Federation


\noindent
$^2$Federal Research Center ``Computer Science and Control'' of
the Russian Academy of Sciences, 44-2 Vavilov\linebreak
$\hphantom{^1}$Str., Moscow 119333, Russian Federation


\def\leftfootline{\small{\textbf{\thepage}
\hfill INFORMATIKA I EE PRIMENENIYA~--- INFORMATICS AND
APPLICATIONS\ \ \ 2015\ \ \ volume~9\ \ \ issue\ 3}
}%
 \def\rightfootline{\small{INFORMATIKA I EE PRIMENENIYA~---
INFORMATICS AND APPLICATIONS\ \ \ 2015\ \ \ volume~9\ \ \ issue\ 3
\hfill \textbf{\thepage}}}

\vspace*{3pt}



\Abste{Cloud computing is an evolutionary technology that allows a~remote 
user to gain access to resources through Internet services. The article discusses 
the cloud computing system, which receives complex user queries consisting of several 
subqueries. The allocation of one unit of the resource is required for processing 
each subquery. Accurate assessment of effectiveness of cloud computing resources is 
a~prerequisite to ensure the required quality of service. Response time, i.\,e., 
the maximum of subqueries service times, is chosen as an indication of effectiveness 
of a~cloud computing system. To analyze the characteristics of response time, 
a~simplified mathematical model of a~cloud system was constructed as 
a~queuing system with single Poisson input flow of requests and several subsystems 
(a~buffer and a~server). Each request at the instant of arrival is divided into 
several parts, each of which is served by its queuing subsystem with unlimited 
storage capacity with exponentially distributed service time. The analytical formulas 
for approximation of mean response time and its variance are presented. To assess the 
accuracy of the approximation, a~GPSS (General Purpose Simulation System) 
model was constructed.}

\KWE{cloud computing system; response time; complex queries processing; 
queueing system; approximation; simulation}


\DOI{10.14357/19922264150304}

\Ack
\noindent
This work is partially supported by the Russian Foundation for Basic
Research (grant No.\,15-07-03051).



%\vspace*{3pt}

  \begin{multicols}{2}

\renewcommand{\bibname}{\protect\rmfamily References}
%\renewcommand{\bibname}{\large\protect\rm References}

{\small\frenchspacing
 {%\baselineskip=10.8pt
 \addcontentsline{toc}{section}{References}
 \begin{thebibliography}{99}

\bibitem{1-gor-1}
\Aue{Xiong, K., and H. Perros}. 2009. 
Service performance and analysis in cloud computing. 
\textit{5th IEEE World Congress on Services (Services-1'09) Proceedings}. 
Los Angeles. 693--700.

\bibitem{3-gor-1} %2
\Aue{Buyya, R., J. Broberg, and A.\,M.~Goscinski}. 2011. \textit{Cloud computing: 
Principles and paradigms}. 1st ed. John Wiley\,\&\,Sons. 664~p.

\bibitem{2-gor-1} %3
\Aue{Satyanarayana, А., P.~Suresh Varma, M.\,V.~Rama Sundari, and P.~Sarada Varma}. 
2013. Performance analysis of cloud computing under non homogeneous conditions. 
\textit{Int. J.~Adv. Res. Comput. Sci. Softw. Eng.} 3:969--974.

\bibitem{4-gor-1}
\Aue{Khazae, H., J. Misic, and V.\,B.~Misic}. 2012. 
A~fine-grained performance model of cloud computing centers. 
\textit{IEEE Trans. Parall. Distr. Syst.} 24:2138--2147.
\bibitem{5-gor-1}
\Aue{Mokrov, E.\,V., and K.\,E.~Samuylov}. 2013. 
Model' sistemy oblachnykh vychisleniy v~vide sistemy massovogo obsluzhivaniya 
s~neskol'kimi ocheredyami i~s~gruppovym postupleniem zayavok [Modeling of cloud 
computing as a~queueuing system with batch arrival]. 
\textit{T-Comm~--- Telecommunications and Transport} 7(11):139--141.
\bibitem{6-gor-1}
\Aue{Basharin, G.\,P., Yu.\,V. Gaidamaka, and K.\,E.~Samouylov}. 
2013. Mathematical theory of teletraffic and its application to the analysis 
of multiservice communication of next generation networks. 
\textit{Autom. Control Comp. Sci.} 47(2):62--69.
\bibitem{7-gor-1}
\Aue{Kemper, B., and M. Mandjes}. 2012. Mean sojourn time in two-queue fork-join systems: 
Bounds and approximations. \textit{OR Spectrum} 34:723--742.
\bibitem{8-gor-1}
\Aue{Ko, S.\,S., and R.\,F. Serfozo}. 2004. Response times in $M/M/s$ fork-join networks. 
\textit{Adv. Appl. Probab.} 36(3):854--871.
\bibitem{9-gor-1}
\Aue{Nelson, R., and A.\,N. Tantawi}. 1988. 
Approximate analysis of fork/join synchronization in parallel queues. 
\textit{IEEE Trans. Comput.} 37:739--743.
\bibitem{10-gor-1}
\Aue{Baccelli, F., A.\,M. Makowski, and A.~Shwartz}. 1989. 
The fork-join queue and related systems with synchronization constraints: 
Stochastic ordering and computable bounds. 
\textit{Adv. Appl. Probab.} 21(3):629--660.
\bibitem{11-gor-1}
\Aue{Kness, C.} 1991. On the diffusion approximation to a fork and join 
queueing model. \textit{SIAM J.~Appl. Math.} 51:160--171.
\bibitem{12-gor-1}
\Aue{Ivanovskaya, I.\,A., and S.\,P.~Moiseeva}. 2010. Issledovanie
 matematicheskoy modeli parallel'nogo obsluzhivaniya zayavok smeshannogo tipa 
 [Studying of mathematical model of mixed type claim parallel service]. 
\textit{Bulletin of the Tomsk Polytechnic University} 317(5):32--34.
\bibitem{13-gor-1}
\Aue{Ivanovskaya, I.\,A., and S.\,P.~Moiseeva}. 2010. Issledovanie modeli 
parallel'nogo obsluzhivaniya kratnykh zayavok v~nestatsionarnom rezhime [Analysis of 
the parallel service model of multiple service requests operating in unsteady mode]. 
\textit{Vestnik Tomskogo gos. un-ta. Upravlenie, vychislitel'naya 
tekhnika i~informatika} [Tomsk State University J.~Control and Computer Science] 
3(12):21--28.
\bibitem{14-gopr-1}
\Aue{Thomasian, A.} 2014. Analysis of fork-join and related queueing systems. 
\textit{ACM Comput. Surv. (CSUR)} 47(2):1--71.
 
\end{thebibliography}

 }
 }

\end{multicols}

\vspace*{-3pt}

\hfill{\small\textit{Received July 21, 2015}}


\Contr


\noindent
\textbf{Gorbunova Anastasiya V.} (b.\ 1986)~---
PhD student, %Applied Probability and Informatics Department, 
Peoples' Friendship University of Russia,
6 Miklukho-Maklaya Str., Moscow 117198, Russian Federation; 
avgorbunova@rambler.ru

\vspace*{3pt}

\noindent
\textbf{Zaryadov Ivan S.} (b.\ 1981)~---
Candidate of Science (PhD) in physics and mathematics, associate professor, 
%Applied Probability and Informatics Department, 
Peoples' Friendship University of Russia,
6 Miklukho-Maklaya Str., Moscow 117198, Russian Federation;  
izaryadov@sci.pfu.edu.ru

 
\vspace*{3pt}

\noindent
\textbf{Matyushenko Sergey I.} (b.\ 1963)~---
Candidate of Science (PhD) in physics and mathematics, associate professor, 
%Applied Probability and Informatics Department, 
Peoples' Friendship University of Russia,
6 Miklukho-Maklaya Str., Moscow 117198, Russian Federation;  
matushenko@list.ru

\vspace*{3pt}

\noindent
\textbf{Samuylov Konstantin E.} (b.\ 1955)~---
Doctor of Science in technology, professor, Head of Department, 
%Applied Probability and Informatics Department, 
Peoples' Friendship University of Russia,
6 Miklukho-Maklaya Str., Moscow 117198, Russian Federation;  
ksam@sci.pfu.edu.ru
 
\vspace*{3pt}

\noindent
\textbf{Shorgin Sergey Ya.} (b.\ 1952)~---
Doctor of Science in physics and mathematics, professor, 
Deputy Director, Federal Research Center ``Computer Science and Control'' of 
the Russian Academy of Sciences, 44-2 Vavilov Str., 
Moscow 119333, Russian Federation; sshorgin@ipiran.ru 

 
\label{end\stat}


\renewcommand{\bibname}{\protect\rm Литература}