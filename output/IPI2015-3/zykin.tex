\def\stat{zykin}

\def\tit{ССЫЛОЧНАЯ ЦЕЛОСТНОСТЬ ДАННЫХ В~КОРПОРАТИВНЫХ
ИНФОРМАЦИОННЫХ СИСТЕМАХ}

\def\titkol{Ссылочная целостность данных в~корпоративных
информационных системах}

\def\aut{В.\,С.~Зыкин$^1$}

\def\autkol{В.\,С.~Зыкин}

\titel{\tit}{\aut}{\autkol}{\titkol}

{\renewcommand{\thefootnote}{\fnsymbol{footnote}} \footnotetext[1]
{Работа выполнена при поддержке РГНФ, грант №\,14-16-70008~a(p).}}


\renewcommand{\thefootnote}{\arabic{footnote}}
\footnotetext[1]{Омский государственный технический университет, vszykin@omgtu.ru}



  \Abst{Решается прикладная задача автоматизации построения неизбыточного
множества ссылочных ограничений на данные. Эти ограничения позволяют
регламентировать на предприятии биз\-нес-пра\-ви\-ла в~использовании информации, которая
хранится в~реляционной базе данных (БД) и~обслуживается системой управления БД (СУБД).
Теоретической основой ограничений служат зависимости включения, которые в~данной
статье получили обобщение, позволяющее использовать неопределенные значения. Такое
обобщение является следствием их практической значимости. Для корректного решения
указанной проблемы вводится и~исследуется понятие ациклических схем БД.
Введена интерпретация ациклических схем БД в~виде ассоциированных
гиперграфов, доказана теорема о~цикличности таких гиперграфов. Представлено
автоматическое построение множества всевозможных ссылочных ограничений и~предложен
алгоритм автоматизации удаления избыточных ссылочных ограничений целостности.}

  \KW{ссылочная целостность; неопределенные значения; ациклические схемы}

  \DOI{10.14357/19922264150310}



\vskip 14pt plus 9pt minus 6pt

\thispagestyle{headings}

\begin{multicols}{2}

\label{st\stat}

\section{Введение}

  Целостность БД (database integrity)~--- соответствие имеющейся
в~БД информации ее внутренней логике, структуре и~всем явно
заданным правилам. Каждое правило, налагающее некоторое ограничение на
возможное состояние БД, называется ограничением це\-лост\-ности
(integrity constraint). Биз\-нес-про\-цес\-сы оперируют данными, которые могут
быть изменены и~обновлены в~любое время. Данные, участвующие
  в~биз\-нес-про\-цессе, могут быть связаны с~хранимыми данными и~должны
удовлетворять правилам, которые регулируют поведение компании~[1].
Ограничения целостности оказывают существенное влияние на процесс
интеграции схем БД при слиянии компаний или расширении сферы их
деятельности~[2].

  Ссылочные ограничения целостности на данные (referential integrity)~--- один
из основных видов ограничений в~БД, которые позволяют
сохранить структурную целостность БД. В~большинстве существующих
СУБД поддерживается такой вид
ограничений и~задаются эти ограничения в~виде связей (relationship) на схеме
БД. При <<ручном>> проектировании схемы БД ссылочные ограничения
задаются проектировщиком <<вручную>> по мере формирования очередных
отношений (relation) БД. Однако в~настоящее время стали широко
использоваться средства автоматизации проектирования схем БД. При этом
связи между автоматически сформированными отношениями на схеме БД
  по-преж\-не\-му необходимо формировать либо корректировать вручную,
а~проблема идентификации и~удаления избыточных связей вообще не
решается. Таким образом, становится актуальной проблема построения
корректного и~неизбыточного набора ссылочных ограничений целостности.

  Ответственность за поддержание ограничений целостности в~технологии БД
возлагается на СУБД. Любая операция по изменению состояния БД,
нарушающая ка\-кое-ли\-бо ограничение целостности, будет заблокирована
СУБД.

  Традиционно в~технологии БД используются следующие виды ограничений
целостности.

\vspace*{-6pt}

\subsection{Ограничение домена}

  При определении структуры отношений БД могут быть заданы ограничения
на допустимые зна\-чения в~столбцах. Определение типа атрибута в~отноше\-нии
задает базовое ограничение, которое контролируется СУБД. В~указанном
столбце не может появиться значение, противоречащее выбранному типу,
например символьная строка в~столбце, для которого указан тип <<Дата>>.
Кроме того, на значения атрибутов в~столбце отношения могут быть дополнены
ограничения, например дата должна быть задана в~определенном интервале.
Попытка ввес\-ти значение этого атрибута, лежащее за пределами указанного
интервала, будет блокирована СУБД.

\subsection{Целостность сущностей}

  В каждом отношении должен быть задан первичный ключ, который имеет
уникальное непустое значение в~каждой строке отношения. Основанием для
определения первичного ключа служит множество функциональных
зависимостей, формиру\-емых проектировщиком БД. Определенному таким
образом набору атрибутов отношения ставится в~соответствие свойство
PRIMARY KEY. В~пределах отношения это свойство может указываться
только один раз. Другим способом гарантировать уникальность значений
подмножества атрибутов для альтернативных ключей отношения является
присвоение этому подмножеству свойства UNIQUE.

\subsection{Ссылочная целостность}

  Если в~некотором отношении значения атрибутов в~строке могут принимать
только те значения, для которых есть совпадающие значения соответствующих
атрибутов в~другом отношении, то такое ограничение называется ссылочной
це\-лост\-ностью. Для его реализации в~СУБД имеется аппарат внешних ключей
(FOREIGN KEY). При этом определяется главное отношение и~подчиненное
отношение: в~подчиненном отношении не может быть строк, которым нет
соответствующей строки в~главном отношении. Система управ\-ле\-ния БД не позволит выполнить
операцию дополнения строки в~подчиненное отношение, если в~главном
отношении нет соответствующего кортежа, и~операция удаления строки в~главном отношении будет отвергнута, если в~подчиненном отношении имеется
соответствующая связанная строка (запрет висячих ссылок).

  Необходимо отметить, что ограничения целостности могут
взаимодействовать между собой и, более того, противоречить друг другу.
В~литературе (см., например,~[3--5]) активно об\-суж\-да\-ет\-ся проблема
взаимодействия функциональных зависимостей и~зависимостей включения и,
как следствие, взаимодействия целостности сущностей и~ссылочной
целостности. Ограничения домена, в~свою очередь, могут существенно усилить
ссылочную целостность вплоть до блокировки ввода записей в~подчиненное
отношение. В~данной работе предполагается отсутствие влияния на ссылочные
ограничения других ограничений. Это не лишает актуальности исследования,
поскольку используются типизированные зависимости включения, которые
имеют огромное практическое значение, но при этом не вза\-и\-мо\-дей\-ст\-ву\-ют с~функциональными зависимостями. Ограничения домена могут только сузить
область допустимых значений, но это не противоречит результатам данной
статьи. Проблема взаимодействия ограничений может быть изучена в~дальнейшем как расширение представленных результатов исследования.

  В настоящее время известны различные варианты автоматизации построения
первых двух\linebreak видов ограничений целостности, а~ссылочная це-\linebreak ло\-стность остается
без должного внимания. Од\-нако при определении ссылочной целостности
могут быть использованы средства автоматизации. Это обусловлено тем, что
внешние ключи являются следствием декомпозиции и~синтеза схемы БД 
с~использованием функциональных зависи-\linebreak мостей. 

При правильном
(классическом) проектировании схемы БД~\cite{4-z, 3-z} в~большинстве
случаев внешние ключи устанавливаются на атрибутах первичного ключа
главного отношения. Этот факт и~будет использован для автоматизации
построения связей на схеме БД.

  Классический способ проектирования схемы БД~\cite{4-z, 3-z} основан на
зависимостях: функ\-ци\-о\-наль\-ных, многозначных, соединения и~включения. При
этом сущности (отношения) БД формируются в~процессе проектирования, что
позволяет реализовать принцип независимости данных и, как следствие,
устойчивость проекта БД при последующей модернизации. 

Альтернативным
подходом к~со\-зда\-нию проекта БД служит объект\-но-ори\-ен\-ти\-ро\-ван\-ный подход.
В~этом случае сначала создаются сущности (объекты) БД. Единственным
гарантом правильности их формирования является интуиция проектировщика.
Зачастую объекты дублируют структуру документооборота предприятия, что
нарушает принцип независимости данных и~приводит к разрушению структуры
БД при необходимости ее модернизации. Затем сразу создаются связи между
объектами, также на основе интуиции проектировщика (ER-диа\-грам\-ма~---
entity--relationship).
Типичной ошибкой при таком подходе становится <<нагрузка>> связей
семантикой приложения, т.\,е.\ в~связях <<прячутся>> объекты БД. Несмотря
на все недостатки объ\-ект\-но-ори\-ен\-ти\-ро\-ван\-но\-го подхода, он получил
достаточно широкое распространение на практике за счет двух своих
преимуществ: наглядности проектирования и~низкого уровня требований 
к~квалификации проектировщика. В~качестве примеров инструментария таких
систем можно привести ERWin, BPWin и~др.

  При преобразовании ER-диа\-грам\-мы в~схему БД используется механизм
автоматического формирования связей. Правила преобразования при этом
являются эвристическими с~множеством исключений и~нереализуемых
ситуаций, причина которых в~неоднозначной интерпретации сущностей и~связей. Следовательно, говорить о корректности результатов этого процесса нет
смысла.

  Цель данной работы~--- исследование свойств ссылочных ограничений
целостности и~реализация программного обеспечения, позволяющего
автоматизировать построение корректного и~неизбыточного набора связей на
схеме БД, реализующего ссылочные ограничения целостности. На основе
разработки нового математического аппарата и~на основе известного механизма
использования связей на схеме БД разработаны алгоритмы автоматического
формирования ссылочных ограничений целостности. Доказаны корректность и~неизбыточность результатов построений.

\section{Постановка задачи}

  Ссылочные ограничения целостности на данные~--- это один из основных
видов ограничений в~БД, которые поддерживаются большинством
существующих СУБД. Для этого в~СУБД используются связи, установленные
на схеме БД. Путь $U\hm= \{ A_1, A_2,\ldots , A_n\}$~--- множество атрибутов,
определенных в~БД; $[R_i]$~--- множество атрибутов, на которых
определено отношение~$R_i$, $[R_i]\subseteq U$; $\mathfrak{R}\hm= (R_1,
R_2, \ldots , R_k)$~--- БД; $S\hm= \{ [R_1], [R_2], \ldots , [R_k]\}$~---
схема БД, $1\hm\leq i\hm\leq k$. Значения атрибутов, по которым связаны отношения,
заимствуются из главных отношений (справочников) в~подчиненные
отношения. Между ними устанавливается связь $\mathbf{1}:\mathbf{М}$ либо
$\mathbf{1}:\mathbf{1}$ по направлению от главного отношения
к~подчиненному. Формальным основанием для установления связей служат
зависимости включения (inclusion dependencies)~\cite{5-z}:

  \smallskip

  \noindent
  \textbf{Определение~1.} Пусть $[R_i]$ и~$[R_j]$~--- схемы отношений 
  (необязательно различные), $V\subseteq [R_i]$ и~$W\subseteq [R_j]$, $\vert V\vert
\hm= \vert W\vert$, тогда соотношение $R_i[V]\subseteq R_j[W]$ называется
зависимостью включения.

  \smallskip

  В этом определении $\vert V\vert$~--- мощность множества~$V$;
$R_i[V]\hm= \pi_V(R_i)$~--- проекция отношения~$R_i$ по атрибутам~$V$.

Рассмотрим практические аспекты использования
зависимостей включения. Будем предполагать, что условие $V\hm= W$
является необходимым для установления связи. Такие зависимости включения
называются типизированными (typed)~\cite{6-z, 7-z}. Это дополнительное
ограничение вполне согласуется с~общепринятым свойством связей на схеме
БД: связи отражают количественное соотнесение кортежей в~отношениях и~не
обладают ка\-кой-ли\-бо семантикой. Необходимость связывания различных по
смыслу атрибутов, скорее всего, является признаком потери ка\-кой-ли\-бо
функциональной зависимости для связываемых атрибутов либо, как\linebreak
 в~примере~1.1 из~\cite{7-z}, 
следствием того, что остав\-шаяся зависимость
включения, после удаления избыточных зависимостей, установлена для
ат\-ри\-бу\-тов-си\-но\-ни\-мов и~им соответствует тривиальная функциональная
зависимость, т.\,е.\ решение семантических проблем на предварительном
этапе проектирования схемы БД в~большинстве случаев позволяет
избавиться от не\-об\-хо\-ди\-мости использования нетипизированных зависимостей
включения.

  В прикладных задачах часто приходится иметь дело с~неопределенными
данными, когда часть характеристик объектов, кроме идентификатора
(первичного ключа), неизвестна. В~этом случае условие $\pi_V(R_i)\subseteq
\pi_W(R_j)$ в~определении~1 может быть не выполнено, т.\,е.\
в~отношении~$R_i$ могут быть кортежи с~неопределенными значениями.
Этим кортежам соответствуют кортежи в~$R_j$, но с~определенными
значениями. В~технологии БД такое соответствие называется
расширенной связью. Далее определение~1 будет изменено в~соответствии с~этим дополнением.

  Рассмотрим пример, когда между отношениями должна быть установлена
связь $\mathbf{1}:\mathbf{1}$.

  \medskip

  \noindent
  \textbf{Пример~1.} Пусть $A_1$~--- <<Табельный номер сотрудника>>,
$A_2$~--- <<ФИО сотрудника>>, $A_3$~--- <<Дата увольнения сотрудника>>.
Существуют функциональные зависимости: $A_1\hm\to A_2$ и~$A_1\hm\to
A_3$. Области определения этих зависимостей различны: первая зависимость
определена для всего множества сотрудников, а~вторая~--- только для уволенных.
Это и~должно служить основанием для построения декомпозиции: $R_i[A_1,
A_2]$ и~$R_j[A_1, A_3]$. Между~$R_i$ и~$R_j$ устанавливается связь
$\mathbf{1}:\mathbf{1}$, где~$R_i$~--- главное отношение, а~$R_j$~---
подчиненное. Заметим, что такая декомпозиция решает проблему заведомо
неопределенных значений: в~объединенном отношении $R_l[A_1, A_2, A_3]$
значение атрибута~$A_3$ будет иметь неопределенное значение для всех
неуволен\-ных сотрудников. Формальное определение области зависимости
включения является трудоемкой задачей, требующей отдельного исследования,
в~данной \mbox{статье} ограничимся данным примером для понимания термина
<<область определения функциональной зависимости>>.

  \medskip

  Введем обозначения: $\mathrm{PK}(R_i)$, или просто $\mathrm{PK}(i)$,~--- множество
атрибутов, служащее первичным ключом в~отношении~$R_i$; $L(i,j,X)$~---
связь $\mathbf{1}:\mathbf{1}$ либо $\mathbf{1}:\mathbf{M}$ от~$R_i$
к~$R_j$, установленная по множеству атрибутов~$X$, где $R_i$~--- главное
отношение, а~$R_j$~--- подчиненное отношение; $L_1(i,j,X)$~--- связь
$\mathbf{1}:\mathbf{1}$ от~$R_i$ к~$R_j$; $L_M(i,j,X)$~--- связь
$\mathbf{1}:\mathbf{M}$ от~$R_i$ к~$R_j$. Заметим, что в~отношении~$R_i$
может существовать несколько альтернативных первичных ключей
и~множество связей, в~которых~$R_i$ будет главным или подчиненным.

  Далее рассмотрим определение связей между отношениями БД, которые
будем использовать в~качестве расширенных ссылочных ограничений
целостности, учитывающих наличие неопределенных значений.


  \smallskip

  \noindent
  \textbf{Определение~2.} Между отношениями~$R_i$ и~$R_j$ допустима
связь $L_1(i,j,X)$, если $X\hm= \mathrm{PK}(R_i) \hm= \mathrm{PK}(R_j)$ и~для любых
реализаций~$R_i$ и~$R_j$ выполнено $\pi_X(R_j)\subseteq \pi_X(R_i)$.


  В этом определении невозможно наличие неопределенных значений,
поскольку атрибуты~$X$ в~обоих отношениях являются компонентами
первичного ключа.

  \smallskip

  \noindent
  \textbf{Определение~3.} Между отношениями~$R_i$ и~$R_j$ допустима
связь $L_M(i,j,X)$, если $\mathrm{PK}(R_i)\not= \mathrm{PK}(R_j)$ 
и~$\mathrm{PK}(R_i)\subseteq [R_j]$.

  \smallskip

  В определении~3 отношение~$R_j$ может содержать неопределенные
значения для атрибутов, не принадлежащих первичному ключу. Заметим, что
определения~2 и~3 соответствуют типизированным зависимостям включения,
которые поддерживаются СУБД за счет создания внешних ключей (foreign key).
Ограничение целостности, задаваемое связью $L_M(i,j,X)$, не подразумевает
выполнения условия $\pi_X(R_j)\subseteq \pi_X(R_i)$, где $X\hm= R_i \cap R_j$,
поскольку атрибуты~$X$, не принадлежащие $\mathrm{PK}(R_j)$, могут принимать
неопределенные значения, тогда как в~$R_i$ им соответствуют определенные
значения. Смысл ограничения в~том, что неопределенное значение ка\-ко\-го-ли\-бо
атрибута в~$R_j$ может быть заменено только тем определенным значением,
которое есть в~$R_i$. Поиск связей, соответствующих определениям~2 и~3,
достаточно просто алгоритмизируется, что позволяет выявлять большинство
ссылочных ограничений целостности в~автоматическом ре-\linebreak жиме.

  В схемах БД~\cite{8-z} кроме рассмотренных используются следующие
связи:
\begin{enumerate}[(1)]
\item <<one or more mandatory>>; 
\item <<one and only one>>; 
\item <<zero or one
optional>>; 
\item <<zero or many optional>>. 
\end{enumerate}

Поскольку связи на схеме БД
являются парными, то связь~1 служит частным случаем связи
$\mathbf{1}:\mathbf{M}$ со стороны второго отношения, связь~2~--- частным
случаем связи $\mathbf{1}:\mathbf{1}$ со стороны обоих отношений, связь~3
совпадает со связью $\mathbf{1}:\mathbf{1}$ со стороны второго отношения,
связь~4 совпадает со связью $\mathbf{1}:\mathbf{M}$ со стороны второго
отношения. Другие варианты связей, имеющие тип связи~3 со стороны первого
отношения, противоречат определению зависимости включения и~не являются
ограничением ссылочной целостности данных. Такие связи будем далее
игнорировать. Особый случай представляет вариант связи
$\mathbf{1}:\mathbf{M}$, который допускает наличие кортежей с~неопределенными значениями общих атрибутов во втором отношении, когда
они не являются ключевыми. Поскольку неопределенное значение не равно
определенному значению, то этот вариант связи противоречит зависимости
включения, но остается ограничением целостности. Оставим такую связь в~рассмотрении, назвав ее расширенной связью.

  Таким образом, задача состоит в~разработке алгоритмов для формирования
ссылочных ограничений целостности в~соответствии с~определениями~2 и~3.

\vspace*{-6pt}


\section{Ациклические схемы баз~данных}

  В работе~\cite{6-z} условие ацикличности зависимостей включения
формулируется следующим образом: множество зависимостей включения
является ациклическим, если не существует объекта $R[X]\subseteq R[Y]$
и~$X\not= Y$ и~не существует отношений $R_1, R_2, \ldots, R_n$ ($n\hm>1$)
таких, что $R_1[X_1]\hm\subseteq R_2[Y_2], R_2[X_2]\subseteq R_3[Y_3]$,\ldots ,
$R_n[X_n]\subseteq R_1[Y_1]$. Тогда условие ацикличности зависимостей
включения можно проверять с~использованием полиномиального алгоритма
построения замыкания множества отношений. 

Рассмотрим простейший пример
с типизированными зависимостями включения $R_1[X_1]\hm\subseteq R_2[X_1]$,
$R_2[X_1]\subseteq R_3[X_1]$ и~пусть атрибуты~$X_1$ в~отношении~$R_2$
допускают неопределенные значения. Тогда замыкание $R_1[X_1]\subseteq
R_3[X_1]$ не выполнено, поскольку не выполнена зависимость\linebreak
$R_2[X_1]\subseteq R_3[X_1]$. Это говорит о том, что учет расширенных
связей не позволяет использовать ап\-парат выводимости и~построение
замыкания.\linebreak В~предлагаемом подходе условие $R_2[X_1]\subseteq R_3[X_1]$
выполнено, если~$X_1$ является первичным ключом в~$R_2$, а~в~первичном
ключе неопределенные значения не допускаются. Это несколько сужает
область применимости, но позволяет формировать корректные построения с~учетом расширенных связей.

  Сформулируем условие ацикличности для совокупности отношений по
аналогии с~условием ацик\-лич\-ности зависимостей включения~\cite{6-z}, но
учитывающее расширенные связи.

  \smallskip

  \noindent
  \textbf{Определение~4.} Совокупность отношений~$\mathfrak{R}$ будем
называть ациклической, если не существует упорядоченного подмножества
отношений

\noindent
  \begin{equation*}
  \left\{ R_{m(1)}, R_{m(2)}, \ldots, R_{m(s)} \right\} \subseteq \mathfrak{R}
%  \label{e1-z}
  \end{equation*}
такого, что имеются связи

\vspace*{-4pt}

\noindent
\begin{multline}
L\left(m(1), m(2), X_1\right), L\left(m(2), m(3), X_2\right),\ldots\\
\ldots , L\left(m(s), m(1),
X_s\right)\,,
\label{e2-z}
\end{multline}
$s>1$ и~$X_1\subseteq X_2\subseteq \cdots \subseteq X_s$, в~противном случае
совокупность отношений~$\mathfrak{R}$ будем называть циклической.
Последовательность~(\ref{e2-z}) может содержать расширенные связи.

  В работе~\cite{9-z} рассмотрено несколько различных условий
ацикличности~$\mathfrak{R}$ и~доказана их эквивалентность. Одно из
условий, рассмотренное также в~работе~\cite{3-z}, устанавливает, что
совокупность~$\mathfrak{R}$ является ациклической, если ацикличен
ас\-со\-ци\-иро\-ван\-ный с~ней гиперграф. Поскольку предполагается, что
в~$\mathfrak{R}$ отсутствуют отношения, для которых $[R_i]\subseteq [R_j]$
при $i\not= j$, то гиперграф будет редуцированным.

\vspace*{2pt}

  \noindent
  \textbf{Определение~5.} В~контексте БД гиперграфом называется
пара $G\hm= (U,S)$, где вершинами~$U$ служат атрибуты БД,
а~элементами~$S$ являются гиперребра $[R_i]$, которые объединяют
вершины~--- атрибуты, входящие в~$[R_i]$. Гиперграф будет циклическим,
если существует атрибут~$A$ и~последовательность $[R_1^\prime],
[R_2^\prime],\ldots , [R_m^\prime]$ такая, что $A\hm\in [R_1^\prime]$
и~$A\hm\in [R_m^\prime]$, и~$[R_i^\prime]\cap [R^\prime_{i+1}]\not= \emptyset$
для $1\hm\leq i\hm\leq m\hm-1$.

\vspace*{2pt}

  \noindent
  \textbf{Теорема~1.} \textit{Если совокупность~$\mathfrak{R}$ циклическая,
то ассоциированный с~ней гиперграф также будет цик\-лическим.}

\vspace*{2pt}

  \noindent
  Д\,о\,к\,а\,з\,а\,т\,е\,л\,ь\,с\,т\,в\,о\,.\ \ Подмножество отношений\linebreak
   $\{R_{m(1)}, R_{m(2)}, \ldots, R_{m(s)}\}\subseteq \mathfrak{R}$ соответствует 
реб\-рам в~ассоциированном гиперграфе. Ребра свя\-зы-\linebreak вают совокупности атрибутов~---
узлы гиперграфа. Существование последовательности $L\left(m(1)\right.$,
$\left.m(2),X_1\right)\!, 
L\left(m(2), m(3),X_2\right)\!, \ldots , L\left( m(s), m(1)\!,X_s\right)$
гарантирует, что $R_{m(i)}\cap R_{m(i+1)}\not=\emptyset$, $i\hm= 1,2,\ldots , s$,
где $m(s+1)\hm= m(1)$. Следовательно, подмножество $\{ R_{m(1)}, R_{m(2)},
\ldots , R_{m(s)}\}\subseteq \mathfrak{R}$ вмес\-те со связыва\-мыми узлами
образует цикл в~общем гиперграфе. А~значит, общий гиперграф
для~$\mathfrak{R}$ будет циклическим.

\vspace*{2pt}

  Исследование соотношения условий ацикличности в~\cite{9-z}
с~определением~4 продолжим рас\-смот\-ре\-нием примера~13.15 из~\cite{3-z}:
$\mathfrak{R}\hm= \{R_1, R_2, R_3\}$, где $R_1[ABC]$; $R_2[BD]$;
$R_3[CD]$. Дополним этот пример функциональными зависимостями
$\{BC\hm\to A, B\hm\to D, C\hm\to D\}$, которые могли быть использованы при
построении~$\mathfrak{R}$. Ассоциированный с~$\mathfrak{R}$ гиперграф
является циклическим, тогда как по определению~4
совокупность~$\mathfrak{R}$ является ациклической, поскольку для нее
существуют только две связи: $L_M(2,1,B)$ и~$L_M(3,1,C)$. Таким образом,
класс ациклических схем БД по определению~4 шире, чем по условиям
в~\cite{9-z}.

\section{Формирование неизбыточного набора связей}

  Далее будем предполагать, что совокупность~$\mathfrak{R}$ является
ациклической в~соответствии с~определением~4.

  \smallskip

  \noindent
  \textbf{Определение~6.}
  Связь $L(i,j,X)$ задает множество допустимых кортежей~$T_i$ в~$R_j$.
Тогда связь $L(i,j,X)$ будет избыточной, если существуют связи $L(l,j,Y_l)$,
$l\hm= 1,\ldots ,m$, где $\mathop{\cap}\limits_{l=1}^m T_l\subseteq T_i$.

  \smallskip

  Заметим, что под определение~6 не попадают\linebreak избыточ\-ные связи,
являющиеся следствием взаимо\-действия функциональных зависимостей и~зависимостей включения~\cite{5-z, 7-z}, поскольку для типизированных
зависимостей включения такое взаимодействие удаляется на этапе
проектирования схем отношений БД. По этой же причине между парой
отношений может быть установлена только одна связь (другая может быть
только циклической). Далее будем рассматривать только такие
множества~$\mathfrak{R}$.

  Для формирования неизбыточного набора связей на схеме БД сначала
формируется множество отношений $\mathfrak{R}\hm= \{R_1, R_2, \ldots ,
R_q\}$ в~соответствии с~нормальной формой Бой\-са--Код\-да. По
определениям~2 и~3 между сформированными отношениями устанавливаются
связи $L_M(i,j,X)$ и~$L_1(i,j,X)$. Затем осуществляется удаление избыточных
связей.

  Cвязи $L_M(i,j,X)$ не могут быть установлены простым алгоритмом
перебора по определению~3, поскольку наличие связи предполагает ссылочное
ограничение целостности, а~оно, в~свою очередь, зависит от семантики
прикладной области. Для демонстрации этого факта рассмотрим пример.

  \smallskip

  \noindent
  \textbf{Пример~2.} Пусть задано некоторое множество отношений~---
фрагмент схемы БД учебного заведения, где подчеркнуты ключевые атрибуты
отношений:
\begin{description}
\item[\,]
  $R_1$\;=\;\textit{Студенты} (\underline{№~студента}, \underline{№~группы}, 
  ФИО~студента);

\item[\,]
  $R_2$\;=\;\textit{Список групп} (\underline{№~группы}, Код группы, №~специальности,
№~курса);
\item[\,]
  $R_3$\;=\;\textit{Предметы} (\underline{№~предмета}, Предмет);
\item[\,]
  $R_4$\;=\;\textit{Оценки} (\underline{№~студента}, 
  \underline{№~группы}, \underline{№~предмета}, Оценка);
\item[\,]
  $R_5$\;=\;\textit{Аттестация} (\underline{№~студента}, 
  \underline{№~группы}, \underline{№~предмета}, \underline{Вид
аттестации}, Балл).
\end{description}

  На схеме должны быть установлены следующие связи:
\begin{description}
\item[\,]  $L_M(2,1,\langle \mbox{№~группы}\rangle)$;
\item[\,]  $L_M(2,4,\langle \mbox{№~группы}\rangle)$;

\pagebreak


\item[\,] $L_M(2,5,\langle \mbox{№~группы}\rangle)$;
\item[\,] $L_M(1,4,\langle\mbox{№~студента,\ №~группы}\rangle)$;
\item[\,]  $L_M(1,5,\langle\mbox{№~студента,\ №~группы}\rangle)$;
\item[\,]  $L_M(3,4,\langle\mbox{№~предмета}\rangle)$;
\item[\,] $L_M(3,5,\langle\mbox{№~предмета}\rangle)$.
\end{description}

 Формально, по определению~2
должна быть уста\-нов\-ле\-на связь $L_M(4,5,\langle\mbox{№~студента},\
\mbox{№~группы}$, $\mbox{№~предмета}\rangle)$, которая содержательно задает следующее
ограничение целостности: аттестация студентов может осуществляться только
по тем предметам, по которым проставлены оценки. Однако это противоречит
прикладной области, и~связь не должна присутствовать на схеме БД.
Проектировщику БД в~этом случае необходимо ответить на вопрос: нужна или
нет данная связь?

  Для установления связи $L_1(i,j,X)$ дополнительно надо определить: какое
отношение в~этой связи является главным, а~какое~--- подчиненным. При
пустой БД, на этапе проектирования, включение $\pi_V(R_j)\subseteq
\pi_V(R_i)$, где $V$~--- первичный ключ отношений~$R_i$ и~$R_j$, не может
быть установлено алгоритмически. Поэтому при установлении связей
необходимо учитывать семантику прикладной области, которую определяет
проектировщик схемы БД. При определении включения между~$R_i$ и~$R_j$
проектировщику БД достаточно ответить на вопрос: <<Как соотносятся
области значений атрибутов $R_i\cap R_j$ в~отношениях~$R_i$ и~$R_{j}$?>>,
предоставив выбор одного из трех вариантов: 
\begin{description}
\item[\,] <<1.~Значения в~$R_i$ являются
подмножеством значений в~$R_j$ по множеству атрибутов~$V$>>;
\item[\,] <<2.~Значения в~$R_j$ являются подмножеством значений в~$R_i$ по
множеству атрибутов~$V$>>; 
\item[\,] <<3.~Не выполнены условия 1-го и~2-го
вариантов>>.
\end{description}

 При выборе первого варианта отношение~$R_j$ становится
главным для~$R_i$, второго~--- $R_i$ становится главным для~$R_j$,
третьего~--- отсутствие связи между~$R_i$ и~$R_j$. Однако для решения
указанной проб\-ле\-мы достаточно спросить: нужна связь $L(i,j,V)$ либо связь
$L(j,i,V)$. Такое решение будет пригодно для обоих видов связей. Допустим,
что этот выбор реализуется функцией $\mathrm{Ch}(R_i, R_j, V)$, имеющей значения
$L(i,j,V)$ либо пусто ($\emptyset$).

  В соответствии с~рассмотренным обоснованием алгоритм формирования
связей для множества~$\mathfrak{R}$, где $L$~--- множество всех связей,
может быть записан в~следующем виде:

\noindent
$L=\emptyset$

\noindent
do $i=1$ to~$k$

\hspace*{2mm}for each $\mathrm{PK}(R_i)$ in $R_i$

\hspace*{4mm}do $j=1$ to $k$

\hspace*{6mm}if $i\not= j$ then

\hspace*{8mm}if $\mathrm{PK}(R_i)\in R_j$ then

\hspace*{9mm}$L=L\cup \mathrm{Ch}(R_i,R_j,V)$

\hspace*{8mm} endif

\hspace*{6mm}endif

\hspace*{4mm}enddo

\hspace*{2mm}endfor

\noindent
enddo

  Сложность данного алгоритма является полиномиальной по времени:
$O(k^2)$.

  В работе~\cite{6-z} без доказательства сформулировано условие избыточной
зависимости включения. Докажем аналогичное условие для связей, в~том числе
и~расширенных, опираясь на реализации отношений, а~не на выводимость.


  \smallskip

  \noindent
  \textbf{Теорема~2.} \textit{Связь   является избыточной, если
существуют связи}

\noindent
  \begin{multline}
  L\left(i,m(1), X_0\right), L\left( m(1), m(2), X_1\right),\ldots \\
  \ldots, L\left(
m(p),j,X_p\right)
  \label{e3-z}
  \end{multline}
\textit{и}

\noindent
\begin{equation}
X\subseteq \mathrm{PK}(i)\subseteq X_s\subseteq R_{m(s)}\,,\enskip s=2,3,\ldots, p\,,
\label{e4-z}
\end{equation}
\textit{где $m$~--- массив номеров отношений.}

\smallskip

\noindent
  Д\,о\,к\,а\,з\,а\,т\,е\,л\,ь\,с\,т\,в\,о\,.\ \ Заметим, что условия $\mathrm{PK}(i)\subseteq
R_j$ и~$\mathrm{PK}(i)\subseteq R_{m(1)}$ являются следствием наличия связей
$L(i,j,X)$ и~$L(i, m(1), X_0)$ соответственно.
  \begin{enumerate}[1.]
  \item Пусть выполнены условия~(\ref{e3-z}) и~(\ref{e4-z}). Поскольку
множество атрибутов $PK(i)$ присутствует в~каж\-дом отношении цепочки
  \begin{equation}
  R_i, R_{m(1)}, R_{m(2)},\ldots , R_{m(p)}, R_j\,,
  \label{e5-z}
  \end{equation}
то по определению оно участвует во всех связях
последовательности~(\ref{e3-z}); следовательно, 
\begin{multline*}
\pi_{\mathrm{PK}(i)}(R_i)\supseteq
\pi_{\mathrm{PK}(i)}(R_{m(p)}\supseteq \cdots\\
\cdots
 \supseteq  \pi_{\mathrm{PK}(i)}(R_{m(p)})\supseteq
\pi_{\mathrm{PK}(i)} (R_j)\,. 
\end{multline*}

Данная последовательность включений гарантирует, что
в~$R_j$ не появится кортежа со зна-\linebreak чением $PK(i)$, не содержащимся в~других\linebreak
отношениях цепочки~(\ref{e5-z}). Это служит более\linebreak \mbox{сильным} ограничением,
чем задает связь $L(i,j,X)$ и,~следовательно, она избыточна 
в~соответствии с~определением~6. Пусть связь $L(m(p),j,X_p)$
в~по\-сле\-до\-ва\-тель\-ности~(\ref{e3-z}) является расширенной, а~связь $L(i,j,X)$ не
является расширенной. Другие связи в~последовательности~(\ref{e3-z}) не
могут быть расширенными, так как общие атрибуты во вторых отношениях
должны быть ключевыми. В~отношении~$R_j$ не появится кортеж с~пустыми
значениями всех атрибутов $\mathrm{PK}(i)$, поскольку данное ограничение
уста\-нав\-ливается внутри~$R_j$. Поэтому связь $L(i,j,X)$ может быть удалена.

\pagebreak

  \item Предположим, что в~цепочке~(\ref{e5-z}) имеется
отношение~$R_{m(s)}$, которое не содержит полностью множество атрибутов
$\mathrm{PK}(i)$. Тогда при переходе от~$R_{m(s-1)}$ к~$R_{m(s)}$ снимаются
ограничения на атрибуты $\mathrm{PK}(i)\hm- [R_{m(s)}]$, обусловленные
последовательностью связей~(\ref{e3-z}). Следовательно, эти атрибуты могут
принимать произвольные значения, в~том числе и~не принадлежащие проекции
$\pi_{\mathrm{PK}(i)}(R_i)$. Поэтому последовательность~(\ref{e3-z}) не является
основанием для удаления связи $L(i,j,X)$.
  \item Если выполнено условие~(\ref{e4-z}), но отсутствует
последовательность~(\ref{e3-z}), то и~ограничения на $\mathrm{PK}(i)$ в~$R_j$,
обусловленные этой последовательностью, отсутствуют.
  \end{enumerate}

  %\smallskip

  \noindent
  \textbf{Следствие.} Если существует~(\ref{e3-z}) и~$p\hm=1$, то связь
$L(i,j,X)$ избыточна.

  \smallskip

  \noindent
  \textbf{Замечание.} Последовательность нескольких связей не может быть
избыточна. Допустим, что существуют последовательности~(\ref{e3-z})
и~$L(i,v,X_i)$, $L(v,j,X_v)$, $v\not= m(s)$, $s\hm= 1,2, \ldots, p$. Удаление
связей $L(i,v,X_i)$ и~$L(v,j,X_v)$, независимо от условия~(\ref{e4-z}),
означало бы снятие с~отношения~$R_v$ ограничений, накладываемых связью
$L(i,v,X_i)$.


  \smallskip

  Поиск избыточных связей с~использованием\linebreak соотношений~(\ref{e3-z})
и~(\ref{e4-z}) будет экспоненциальной задачей. Поскольку связи являются
реализацией типизированных зависимостей включения, то к~поиску
избыточных связей применим алгоритм по\-стро\-ения замыкания
отношений~\cite{2-z}, полиномиальный по времени. Рассмотрим адаптацию
этого алгоритма для удаления избыточных связей без использования графа
зависимостей включения~\cite{6-z}. Пусть $\mathrm{PK}(R_i$~--- ключ
отношения~$R_i$, соответст\-ву\-ющий связи~$L(i,j,X)$. Тогда

\noindent
for each $L(i,j,X)$ in $L$

\hspace*{2mm}$l=1$

\hspace*{2mm}$m(l)=i$

\hspace*{2mm}$iterations=\mathrm{true}$

\hspace*{2mm}do while $iterations$

\hspace*{3mm}for each $L(v,w,X_l)\ \mbox{in\ } L$

\hspace*{3mm}\mbox{where\ } $L(i,j,X)\not= L(v,w,X_l)$

\hspace*{4mm}$substitution=\mathrm{false}$

\hspace*{4mm}if $v\in m[1,\ldots ,l]$ and $\mathrm{PK}(R_i)\subseteq R_w$ then

\hspace*{5mm}if $w=j$ then

\hspace*{6mm}$L=L-L(i,j,X)$

\hspace*{6mm}exit do

\hspace*{5mm}else

\hspace*{6mm}if $w\not\in m[1,\ldots ,l]$ then

\hspace*{7mm}$ l=l+1$

\hspace*{7mm}$m(l)=w$

\hspace*{7mm}$substitution=\mathrm{true}$

\hspace*{6mm}endif

\hspace*{5mm}endif

\hspace*{4mm}endif

\hspace*{3mm}endfor

\hspace*{4mm}if not $substitution$ then $iterations=\mathrm{false}$

\hspace*{2mm}enddo

\noindent
endfor

  Пусть $\vert L\vert$~--- число связей в~исходном множестве~$L$. С~учетом
того, что на каждой итерации (цикл do while) к~массиву~$m$ присоединяется
не менее одного номера отношения (их $k$~штук), рассмотренный алгоритм
будет иметь полиномиальную сложность: $O(\vert L\vert^2 k^2)$.
Дополнительная степень при~$k$ обусловлена выполнением операций $v\hm\in
m[1,\ldots, l]$ и~$w\not\in m[1,\ldots, l]$. Доказательство корректности
алгоритма выполняется индукцией по числу итераций по аналогии с~доказательством теоремы~5.2 в~\cite{4-z}.

  Если между отношениями установлены связи $L_1(m(1), m(2), X_1),\ldots ,
L_1(m(p-1),m(p),X_{p-1})$, то эти последовательности могут иметь общие узлы
(отношения). Если исходное отношение~$R_i$ в~соответствии с~определением~3 имело связь $L_M(i,j,X)$, то связи $L_M(m(l),j,X_l)$, $l\hm=
1,\ldots ,p$, также будут иметь место. Алгоритм удаления избыточных связей
оставит исходящие связи $L_M(m(p),j,X_p)$, т.\,е.\linebreak
 исходящие связи из
последних отношений после\-довательностей. Эти связи будут задавать более\linebreak
жесткие ограничения, чем отношение~$R_i$, т.\,е.\ сократится количество
допустимых значений. Следовательно, связи необходимо <<передвинуть>>
к~тем отношениям последовательностей, которые соответствуют реальным
ограничениям. Число таких связей регламентируется следующим очевидным
свойством, вытекающим из теоремы~2.

  \smallskip

  \noindent
  \textbf{Свойство~1.} Последовательность отношений $R_{m(1)}$,
$R_{m(2)},\ldots , R_{m(p)}$, между которыми установлены связи $L_1(m(1),
m(2), X_1),\ldots , L_1(m(p\hm-1),m(p), X_{p-1})$, и~произвольное
отношение~$R_j$ могут иметь не более одной связи $L_M(m(l),j,X_l)$.

  Таким образом, число связей $L_M(m(l),j,X_l)$ с~отношением~$R_i$ из
указанных последовательностей связей может быть не больше, чем
соответствующих последовательностей $R_{m(1)}, R_{m(2)}, \ldots, R_{m(p)}$.
Дополнительная проблема заключается в~том, что все эти связи в~совокупности
задают конъюнктивное ограничение на значения атрибутов $\mathrm{PK}(R_i)$ 
в~отношении~$R_j$. Однако может потребоваться дизъюнктивное ограничение
для ка\-кой-ли\-бо \mbox{группы} связей. В~БД не используются связи,
устанавливаемые для трех и~более отношений и~поз\-во\-ля\-ющие задавать
ограничения в~виде дизъюнктивной (ДНФ)
или конъюнктивной  (КНФ) нормальной форме. Схема БД в~этом случае
была бы гиперграфом (не надо путать с~гиперграфом из~\cite{6-z, 9-z},
поскольку связей там вообще нет). В~рассматриваемом случае для упомянутой
группы связей придется создать фиктивное отношение с~атрибутами $\mathrm{PK}(R_i)$,
которое объединит эти связи и~будет встроено во все соответствующие этим
связям последовательности. Заметим, что, пользуясь определением
выводимости зависимостей включения без учета неопределенных значений,
невозможно получить аналогичное преобразование.

\section{Заключение}

  Область применения предложенного подхода ограничивается
определениями~2 и~3, что вполне достаточно для корректно сформированных
отношений БД. Обоснование этого утверждения может быть сделано только на
уровне анализа семантики конкретных БД. Если же проектировщик дополнит
связи на схеме БД, не удовлетворяющие определениям~2 и~3, то тогда ему
будет необходимо перейти к построению нормальной формы IDNF~\cite{5-z}.
Если же проектировщик захочет усилить ограничения, используя
  связи~\cite{8-z}, то для этого будет достаточно подправить (усилить)
свойства существующих связей.

  В работе рассмотрены простейшие виды парных ограничений целостности
данных, заимствованные из объектно-ориентированных моделей данных.
Однако на практике имеются структурные ограничения на данные, которые
связывают не только пары, но и~большее количество компонентов БД в~одном
ограничении, например для пятой нормальной формы, когда связывать
необходимо три и~более отношений, полученных после декомпозиции. Для
решения указанной задачи потребуется развить формальную теорию
зависимостей включения (теоретическую основу ссылочной целостности) на
случай множества отношений: построение неизбыточного множества
зависимостей, ацикличность, разработка алгоритмов автоматического
формирования программ (триггеров), обслуживающих ограничения, и~т.\,д.
В~дальнейшем эти исследования позволят существенно расширить
функциональные возможности приложений, работающих с~БД.

{\small\frenchspacing
 {%\baselineskip=10.8pt
 \addcontentsline{toc}{section}{References}
 \begin{thebibliography}{9}
\bibitem{1-z}
\Au{G$\acute{\mbox{o}}$mez-L$\acute{\mbox{o}}$pez~M.\,T., Gasca~R.\,M.,
P$\acute{\mbox{e}}$rez-$\acute{\mbox{\!\!A}}$lvarez~J.\,M.} Compliance validation
and diagnosis of business data constraints in business processes~// Inform.
Syst., 2015. Vol.~48. P.~26--43.
\bibitem{2-z}
\Au{Visser J.} Coupled transformation of schemas, documents, queries, and
constraints~// Electronic Notes in
Theoretical Computer Sci., 2008. Vol.~200. Iss.~3. P.~3--23.


\bibitem{5-z} %3
\Au{Casanova M., Fagin R., Papadimitriou~C.} Inclusion dependencies and their
interaction with functional~// J.~Comp. Syst. Sci., 1984. Vol.~28. Iss.~1.
P.~29--59.
\bibitem{6-z} %4
\Au{Missaoui R., Godin R.} The implication problem for inclusion dependencies:
A~graph approach~// SIGMOD Record, 1990. Vol.~19. Iss.~1. P.~36--40.
\bibitem{7-z} %5
\Au{Levene M., Vincent M.\,W.} Justification for inclusion dependency normal
form~// IEEE Trans. Knowledge Data Eng., 2000. Vol.~12. Iss.~2.
P.~281--291.

\bibitem{4-z} %6
\Au{Ульман Дж.} Основы систем баз данных.~--- М.: Финансы и~статистика,
1983. 334~с.

\bibitem{3-z} %7
\Au{Мейер Д.} Теория реляционных баз данных~/ Пер. с~англ.~--- М.: Мир, 1987. 608~с.
(\Au{Maier~D.} The theory of relational databases.~--- Computer Science Press, 1983. 656~p.)

\bibitem{8-z}
\Au{Garmany J., Walker J., Clark~T.} Logical database design principles.~--- New
York, NY, USA: Auerbach Publications, 2005. 200~p.
\bibitem{9-z}
\Au{Beeri C., Fagin R., Maier~D., Yannakakis~M.} On the desirability of acyclic
database schemes~// ACM, 1983. Vol.~30. Iss.~3. P.~479--513.


  \end{thebibliography}

 }
 }

\end{multicols}

\vspace*{-3pt}

\hfill{\small\textit{Поступила в~редакцию 05.01.15}}

%\newpage

\vspace*{12pt}

\hrule

\vspace*{2pt}

\hrule

%\vspace*{12pt}

\def\tit{REFERENTIAL INTEGRITY OF~DATA IN~CORPORATE INFORMATION SYSTEMS}

\def\titkol{Referential integrity of data in corporate information systems}

\def\aut{V.\,S.~Zykin}

\def\autkol{V.\,S.~Zykin}

\titel{\tit}{\aut}{\autkol}{\titkol}

\vspace*{-9pt}


\noindent
Omsk State Technical University, 11~Mira Av., Omsk 644050, Russian
Federation



\def\leftfootline{\small{\textbf{\thepage}
\hfill INFORMATIKA I EE PRIMENENIYA~--- INFORMATICS AND
APPLICATIONS\ \ \ 2015\ \ \ volume~9\ \ \ issue\ 3}
}%
 \def\rightfootline{\small{INFORMATIKA I EE PRIMENENIYA~---
INFORMATICS AND APPLICATIONS\ \ \ 2015\ \ \ volume~9\ \ \ issue\ 3
\hfill \textbf{\thepage}}}

\vspace*{3pt}


\Abste{The paper deals with the task of construction of a nonredundant set of 
referential constraints on data. This set of constraints allows regulating the business 
rules of using information on the enterprise, which is supported by
a~database 
management system and is located in a relation database. The inclusion dependences are 
the theoretical bases of the restrictions and they have obtained generalization in this 
paper; so, it is possible to use null values. This
generalization is a consequence of 
their practical significance. The term ``acyclic database schemes'' is introduced\linebreak\vspace*{-12pt}}

\Abstend{and 
investigated for correct solution of this problem. Some attention is given to 
interpretation of acyclic schemas with hypergraphs, the theorem of acyclic hypergraph 
is proved. It is proposed to construct the set of all referential constraints 
automatically by using the rule of decomposition of relations. An algorithm of removing 
redundant referential integrity constraints is presented.}


\KWE{referential integrity; undefined values; acyclic schemes}

\DOI{10.14357/19922264150310}

%\Ack
%\noindent




%\vspace*{3pt}

  \begin{multicols}{2}

\renewcommand{\bibname}{\protect\rmfamily References}
%\renewcommand{\bibname}{\large\protect\rm References}

{\small\frenchspacing
 {%\baselineskip=10.8pt
 \addcontentsline{toc}{section}{References}
 \begin{thebibliography}{9}
\bibitem{1-z-1}
\Aue{G$\acute{\mbox{o}}$mez-L$\acute{\mbox{o}}$pez, M.\,T., R.\,M.~Gasca,
and J.\,M.~P$\acute{\mbox{e}}$rez-$\acute{\mbox{A}}$lvarez}. 2015. Compliance
validation and diagnosis of business data constraints in business processes.
\textit{Inform. Syst.} 48:26--43.
\bibitem{2-z-1}
\Aue{Visser, J.} Coupled transformation of schemas, documents, queries, and
constraints. \textit{Electronic Notes Theoretical Computer Sci.} 200(3):3--23.

\bibitem{5-z-1} %3
\Aue{Casanova, M.} 1984. Inclusion dependencies and their interaction with
functional. \textit{J.~Comp. System Sci.} 28(1):29--59.
\bibitem{6-z-1} %4
\Aue{Missaoui, R.} 1990. The implication problem for inclusion dependencies:
A~graph approach. \textit{SIGMOD Record} 19(1):36--40.
\bibitem{7-z-1} %5
\Aue{Levene, M.} 2000. Justification for inclusion dependency normal form.
\textit{IEEE Trans. Knowledge Data Eng.} 12(2):281--291.

\bibitem{4-z-1} %6
\Aue{Ul'man, J.} 1983. \textit{Osnovy sistem baz dannykh} [Fundamentals of
database systems]. Мoscow: Finance and Statistics. 334~p.
\bibitem{3-z-1} %7
\Aue{Maier, D.} 1983.  \textit{The theory of
relational databases}. Computer Science Press. 656~p.
\bibitem{8-z-1}
\Aue{Garmany, J., J. Walker, and T.~Clark}. 2005. \textit{Logical database design
principles}. Auerbach Publications. 200 p.
\bibitem{9-z-1}
\Aue{Beeri, C., R. Fagin, D.~Maier, and M.~Yannakakis}. 1983. On the desirability
of acyclic database schemes. \textit{ACM} 30(3):479--513.
\end{thebibliography}

 }
 }

\end{multicols}

\vspace*{-3pt}

\hfill{\small\textit{Received January 5, 2015}}


  \Contrl

  \noindent
  \textbf{Zykin Vladimir S.} (b.\ 1992)~---
  assistant professor, Omsk State Technical University, 11~Mira Av., Omsk 644050, Russian
Federation; vszykin@omgtu.ru


\label{end\stat}


\renewcommand{\bibname}{\protect\rm Литература}