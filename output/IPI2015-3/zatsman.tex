\def\stat{zatsman}

\def\tit{ПРОЦЕССЫ ЦЕЛЕНАПРАВЛЕННОЙ ГЕНЕРАЦИИ И РАЗВИТИЯ КРОСС-ЯЗЫКОВЫХ 
ЭКСПЕРТНЫХ ЗНАНИЙ: СЕМИОТИЧЕСКИЕ~ОСНОВАНИЯ~МОДЕЛИРОВАНИЯ$^*$}

\def\titkol{Процессы целенаправленной генерации и развития кросс-языковых 
экспертных знаний: семиотические основания} % моделирования}

\def\aut{И.\,М.~Зацман$^1$}

\def\autkol{И.\,М.~Зацман}

\titel{\tit}{\aut}{\autkol}{\titkol}

{\renewcommand{\thefootnote}{\fnsymbol{footnote}} \footnotetext[1]
{Работа выполнена при поддержке РФФИ 
(проекты 14-07-00785, 13-06-00403) и РГНФ (проект 15-04-00507).}


\renewcommand{\thefootnote}{\arabic{footnote}}
\footnotetext[1]{Институт проблем информатики Федерального исследовательского
центра <<Информатика и~управление>> Российской академии наук,
iz\_ipi@a170.ipi.ac.ru}

 
   \Abst{Представлены результаты разработки семиотических оснований для создания 
моделей процессов целенаправленной генерации и~развития новых экспертных знаний 
и~разработки технологий, обеспечивающих эти процессы. Необходимость разработки таких 
технологий проявляется наиболее наглядно в~ситуациях, когда имеющиеся системы 
экспертных знаний не удовлетворяют новым социально или технологически значимым 
целям, отражающим новые или изменившиеся потребности общества. В~статье речь идет 
не о~хорошо известных в~области искусственного интеллекта методах и~моделях 
представления знаний, процессах управления формами их представления, а~о~разработке 
новых моделей процессов целенаправленной генерации знаний, отражающих динамику их 
формирования. Рассматриваемый подход к~моделированию этих процессов и~разработке 
обеспечивающих их технологий ориентирован на те прикладные области, где знания 
генерируются экспертами в~процессе анализа текстов или других объектов интерпретации, 
которые могут изменяться во времени, с~последующим пред\-став\-ле\-ни\-ем экспертных знаний 
в~надкорпусных базах данных (НБД). Отличительная черта предлагаемого подхода 
к~моделированию заключается в~явном описании отношений между новыми экспертными 
знаниями и~теми объектами интерпретации, на основе анализа которых были 
сгенерированы элементы новых знаний. Другая отличительная черта заключается в~явном 
описании изменяемых во времени элементов знаний, соответствующих объектам 
интерпретации. Реализуемость такого подхода демонстрируется на примере 
экспериментальной информационной технологии, которая поддерживает 
целенаправленную генерацию экспертами кросс-язы\-ко\-вых знаний о~переводах глагольных 
конструкций русского языка на французский. Эти кросс-язы\-ко\-вые знания формируются 
в~процессе анализа параллельных текстов на русском и~французском языках, пары 
выровненных предложений которых являются объектами интерпретации.}
   
   \KW{кросс-языковые экспертные знания; компьютерное моделирование; генерация 
знаний; объекты интерпретации; семиотические основания; модели процессов генерации 
знаний; надкорпусные базы данных}

\DOI{10.14357/19922264150311 }


\vspace*{-6pt}

\vskip 12pt plus 9pt minus 6pt

\thispagestyle{headings}

\begin{multicols}{2}

\label{st\stat}

\section{Введение}

      Модели процессов генерации и развития новых знаний (далее~--- модели генерации) 
стали активно разрабатываться в~последнем десятилетии прошлого века. Наиболее 
известную модель генерации, названную автором спиральной, предложил Икуджиро 
Нонака~[1, 2]. В~процессе ее построения Нонака рассматривал личностные знания человека 
и~коллективные (согласованные) знания группы людей, которые были разделены на 
выражаемые (explicit) и~невыражаемые знания (tacit). Таким образом, спиральная 
модель генерации включает в~рассмотрение следующие четыре понятия и~соответствующие 
им четыре множества знаний (рис.~1):
      \begin{enumerate}[(1)]
\item личностные невыражаемые знания (individual tacit knowledge);\\[-15pt]
\item коллективные невыражаемые знания (group tacit knowledge);\\[-15pt]
\item личностные выражаемые знания (individual explicit knowledge);\\[-15pt]
\item коллективные выражаемые знания (group explicit knowledge).
\end{enumerate}

\begin{figure*} %fig1
       \vspace*{1pt}
 \begin{center}
 \mbox{%
 \epsfxsize=87.663mm 
 \epsfbox{zac-1.eps}
 }
 \end{center}
 \vspace*{-12pt}
\Caption{Спиральная модель генерации знаний Икуджиро Нонака~\cite[с.~69]{3-zat}}
\vspace*{-4pt}
\end{figure*}


      Наряду с~этими четырьмя понятиями были определены следующие четыре вида 
процессов:
      \begin{enumerate}[(1)]
\item социализация личностных невыражаемых знаний;\\[-15pt]
\item экстернализация коллективных невыража\-емых знаний;\\[-15pt]
\item синтез личностных выража\-емых знаний;
\item интернализация личностных выражаемых знаний.
\end{enumerate}



      Используя эти четыре процесса, Нонака ввел метафорическое понятие спирали 
генерации знаний, каждый виток которой включает следующую последовательность:  
со\-циа\-ли\-за\-ция\;$\to$\;экстер\-на\-ли-\linebreak за\-ция\;$\to$\;син\-тез\;$\to$\;ин\-тер\-на\-ли\-за\-ция\;$\to$\;со\-циа\-ли\-за\-ция 
(как начало следующего витка спирали). 
Было показано на примерах, что эта спираль может служить качественной моделью 
итерационного процесса генерации новых знаний во время <<мозгового штурма>>.
      
      Обобщение и существенное развитие спиральной модели генерации знаний было 
предложено в~работах Йошитеру Накамори и~Анджея Вежбицкого в~рамках создаваемой 
ими научной дисциплины, которую они называют <<Наука о знаниях>>~[3--8]. В~этих 
работах знания разделены на личностные знания человека, коллективные 
и~конвенциональные знания. Это деление они называют социальным аспектом, или 
измерением, так как в~результате обобщения было определено три уровня социализации 
знаний (от первого личностного уровня до третьего конвенционального). С~учетом деления 
на выражаемые и~невыражаемые знания в~результате обобщения были определены еще два 
новых множества знаний, которых нет в~спиральной модели: конвенциональные 
невыражаемые и~конвенциональные выражаемые знания.
      
      Вежбицкий и Накамори определили систему\linebreak
       отношений между множествами 
знаний. Свою модель, включающую шесть множеств знаний и~сис\-те\-му отношений между 
ними, они в~совокупности назвали \textit{креативным пространством} (далее в~\mbox{статье} 
термины <<креативное пространство>> и~<<модель Веж\-биц\-ко\-го--На\-ка\-мо\-ри>> 
будут использоваться как синонимы). Кроме шести множеств знаний ими были также 
определены множества эмоций (личностные, коллективные и конвенциональные), которые 
в статье не рассматриваются. 

В~предлагаемых далее в~статье моделях не используются три 
множества невыражаемых знаний человека, которые по определению непосредст\-венно не 
поддаются экспликации. Однако в~процессе компьютерного моделирования эти три 
множества могут использоваться опосредованно.\linebreak Например, в~технологии генерации новых 
кросс-язы\-ко\-вых знаний, рас\-смат\-ри\-ва\-емой далее в~статье, анализируются объекты 
интерпретации, которые сформированы в~процессе перевода текстов на русском языке 
и~являются результатом применения переводчиком одновременно как конвенциональных 
знаний, так и его личностных невыражаемых знаний\footnote{В статье по смысловому 
содержанию разделяются понятия невыражаемых (tacit) и подразумеваемых (implicit) знаний. Невыражаемые 
и незакрепленные в~знаковой форме знания используются переводчиком часто неявно и могут быть известны 
только ему, т.\,е.\ такие знания могут являться личностными. Подразумеваемые знания косвенно выражаются 
в знаковой форме. Например, фраза <<фирма закрыла отдел разработки прикладных программ>> 
подразумевает, что раньше в~этой фирме существовал отдел прикладного программирования. Такие знания 
могут быть коллективными или конвенциональными.}. Этот анализ является примером извле\-чения и 
опосредованного использования послед-\linebreak них.

\begin{figure*} %fig2
       \vspace*{1pt}
 \begin{center}
 \mbox{%
 \epsfxsize=160.234mm
 \epsfbox{zac-2.eps}
 }
 \end{center}
 \vspace*{-9pt}
\Caption{Система терминов для описания объектов трех сред предметной области информатики и 
интерфейсов между ними~\cite{10-zat, 12-zat} (\textit{денотат по определению является компонентом, 
отношением или свойством объекта интерпретации}$^1$)}
\end{figure*}


      
      
      Отметим, что в~модели Веж\-биц\-ко\-го--На\-ка\-мо\-ри и спиральной модели, 
которые относятся к~категории качественных, нет явно определенной оси времени. Это не 
дает возможности фиксировать моменты времени генерации каждого нового структурного 
элемента экспертных знаний. Кроме того, в~этих моделях не рассматриваются объекты 
интерпретации, служащие источниками новых знаний.

\pagebreak

\renewcommand{\thefootnote}{\arabic{footnote}}
\footnotetext[1]{Отличие денотата от объекта интерпретации будет описано далее 
в~примере генерации  кросс-язы\-ко\-вых знаний.}


      Основная цель статьи заключается в~описании разработанных семиотических 
оснований для создания количественных моделей процессов целенаправленной генерации 
и~развития экспертных знаний, а также для разработки обеспечивающих их технологий и~баз 
данных. С~использованием этих оснований в~статье предлагается развитие модели 
Веж\-биц\-ко\-го--На\-ка\-мо\-ри в~следующих двух на\-прав\-ле\-ниях:
      \begin{enumerate}[(1)]
\item вводится ось времени, на которой фиксируют\-ся дискретные моменты времени, 
в~которые по\-рож\-да\-ют\-ся новые элементы вы\-ра\-жа\-емых экспертных знаний, а~также 
фиксируются моменты времени их изменения;
\item в~явном виде специфицируется связь каждого нового элемента знаний с~тем 
объектом интерпретации, в~результате анализа которого формируется этот элемент 
и~дается формализованное его описание.
\end{enumerate}




\section{Семиотические основания}

      В процессе развития модели Веж\-биц\-ко\-го--На\-ка\-мо\-ри использовалась ранее 
разработанная система терминов~\cite{9-zat, 10-zat}. Она включает, в~част\-ности,\linebreak
термины и~дефиниции для таких понятий, как\linebreak
 кон\-цеп\-ты знаний человека (личностные, коллективные, 
конвенциональные), семантическая информация, сен\-сор\-но вос\-при\-ни\-ма\-емые данные, 
циф\-ро\-вая информация, циф\-ро\-вые данные, циф\-ро\-вые коды нескольких категорий и~ряд 
других терминов, а~также задает их распределение по трем\linebreak средам предметной области 
информатики (ментальной, со\-ци\-аль\-но-ком\-му\-ни\-ка\-ци\-он\-ной и цифровой) 
и~описание отношений между этими терминами. Первая отличительная черта этой системы\linebreak 
терминов заключается в~том, что значения терминов <<знания>>, <<семантическая 
информация>>, <<сен\-сор\-но вос\-при\-ни\-ма\-емые данные>>, <<цифровая инфор\-мация>> 
и~<<цифровые данные>> четко разграничены и они по определению не пересекаются по их 
смысловому содержанию (рис.~2). Вторая ее отличительная черта состоит в~том, что она 
является <<масштабируемой>> по числу сред, так как определен принцип добавления 
новых сред в~предметную область информатики как ин\-фор\-ма\-ци\-он\-но-ком\-пьютерной науки, 
а~также соответствующего расширения системы терминов за счет именования объектов 
каждой новой среды и их интерфейсов с~объектами уже существующих сред. Этот принцип 
получил название аксиомы герметичности сред предметной области 
информатики~\cite{12-zat, 11-zat}.
      
      Основная идея развития модели Веж\-биц\-ко\-го--На\-ка\-мо\-ри заключается 
      в~установлении связи каж\-до\-го нового концепта с~объектом интерпретации. При этом анализ 
объектов интерпретации выполняется в~соответствии с~явно определенными целями 
формирования новых знаний, так как статья посвящена именно целенаправленным 
процессам формирования новых выражаемых знаний на основе извлечения и экспликации 
невыражаемых.





      Развитие модели Веж\-биц\-ко\-го--На\-ка\-мо\-ри было выполнено в~несколько 
этапов. Сначала на первом этапе в~модель были добавлены следующие шесть множеств:
      \begin{enumerate}[(1)]
\item формы представления личностных концептов (структурированные тексты, 
изображения или другие виды семантической информации);
\item формы представления коллективных концептов;
\item формы представления конвенциональных концептов;
\item цифровые коды личностных концептов и~форм их представления;
\item цифровые коды коллективных концептов и~форм их представления;
\item цифровые коды конвенциональных концептов и~форм их представления.
\end{enumerate}

      В последних трех множествах разделяются коды концептов и~их имен (как частного случая
      форм представления концептов), например 
коды для пред\-став\-ле\-ния смыслового содержания слов и~последовательностей их литер 
в~цифровой среде относятся к~разным категориям (см.\ рис.~2).
      
      Затем для количественного описания динамики социализации, определяемой 
процессами согласования личностных концептов в~группе экспертов, была введена ось 
с~числами от нуля до бесконечности. Единица на оси социализации обозначает\linebreak
 личностный 
концепт, $N\hm>1$~--- коллективный,\linebreak который согласован группой из~$N$~экспертов, 
а~бесконечность~--- конвенциональный концепт. Необходимость в~нуле на этой оси 
возникла в~процессе проведения эксперимента для обозначения тех <<бывших>> 
личностных концептов, от которых со временем отказались их авторы. Ось социализации 
позволяет кодировать степень согласованности между экспертами результатов анализа 
динамически изменяемых объектов интерпретации и изменение степени согласованности во 
времени.
      
      Таким образом, на первом этапе развития модели Веж\-биц\-ко\-го--На\-ка\-мо\-ри 
были определены шесть новых множеств знаний и введена ось социализации. Эта ось 
служит для обозначения уровня социализации не только выражаемых знаний, но также 
форм их представления и их цифровых кодов. Отметим, что система отношений между 
множествами знаний, определенная Вежбицким и Накамори, не охватывает шесть новых 
множеств знаний. Поэтому далее потребуется доопределить или построить новую систему 
отношений.
      
      На втором этапе была добавлена ось времени, на которой фиксируются моменты 
порождения новых личностных, коллективных и конвенциональных концептов, а~также 
моменты времени их изменений. Следствием этого этапа является то, что появляется 
возможность фиксировать на оси времени не только моменты порождения и изменения 
концептов, но также форм их представления и~их цифровых кодов.
      
      Третий этап заключается в~добавлении множества объектов интерпретации, каждый 
из которых имеет уникальный цифровой код. Следствием этого этапа является то, что 
появляется потенциальная возможность фиксировать на оси времени не только моменты 
порождения и изменения концептов, форм их представления и их цифровых кодов, но 
также динамику изменения соответствующих им объектов интерпретации.
      
      Результаты трех перечисленных этапов развития модели 
Веж\-биц\-ко\-го--На\-ка\-мо\-ри позволяют фиксировать на оси времени:
      \begin{itemize}
\item моменты изменения экспертами объектов интерпретации, являющихся 
источниками новых знаний;\\[-14pt]
\item моменты порождения новых концептов в~процессе интерпретации объектов 
и интроспекции результатов интерпретации;\\[-14pt]
\item моменты изменения формируемых концептов, форм их представления (их 
имен) и их цифровых кодов.
\end{itemize}

      Определим новую систему отношений между объектами интерпретации, концептами 
как эле\-мен\-тами множеств знаний, их именами и~цифро\-вы\-ми кодами, являющимися их 
идентифика\-торами. Она необходима потому, что система отношений в~модели  
Веж\-биц\-ко\-го--На\-ка\-мо\-ри не охватывает шесть новых множеств знаний. Для 
описания отношений между объектами интерпретации, концептами и~именами предлагается 
использовать треугольник Фреге~[13--15], что представляет собой 
\textit{первое семиотическое основание} моделирования процессов генерации знаний. 
Семиотический треугольник Фреге по определению связывает сам объект интерпретации 
(точнее, некоторый денотат, определяемый в~процессе анализа объекта интерпретации; как 
правило, это его компонент, отношение или свойство), понимание денотата (его смысловое 
содержание), т.\,е.\ его концепт, а также некоторое имя как текстовую, или невербальную, 
форму обозначения денотата и~его концепта. Для циф\-ро\-вого кодирования денотатов, 
концептов и~имен предлагается использовать циф\-ро\-вой семиотический треугольник, что 
представляет собой \textit{второе семиотическое основание} моделирования процессов 
генерации знаний. Его определение дано в~работе~\cite{16-zat}, в~которой рис.~6 
иллюстрирует взаимосвязи этих двух треугольников.
   
   Основная идея цифрового семиотического треугольника заключается в~том, что для 
каждой из трех вершин треугольника Фреге используется своя категория цифровых кодов, 
в~том числе и~для концептов. Предлагаемое введение отдельной кодировки для концептов 
дает возможность строить взаимно однозначные отношения между концептами 
и~циф\-ро\-вы\-ми кодами, являющимися их идентификаторами. Построение таких отношений 
является основой компьютерного моделирования процессов генерации знаний. Важно 
отметить, что циф\-ро\-вой семиотический треугольник дает возможность строить взаимно 
однозначные отношения, но остав\-ля\-ет открытым вопрос о~конкретных методах 
приписывания цифровых кодов концептам. В~ряде задач компьютерного моделирования 
процессов генерации знаний существует свобода выбора метода назначения кодов. Однако 
в~задачах оценивания релевантности новых знаний явно определенным целям их генерации 
этот метод может быть во многом обусловлен заданными целями генерации. Примеры 
таких целей рассматриваются далее в~статье.
      
      Перечислим кратко основные результаты, которые были получены в~процессе 
развития модели Веж\-биц\-ко\-го--На\-ка\-мо\-ри. С~помощью цифр~1 и~2 в~списке 
отмечены положения, взятые из модели Веж\-биц\-ко\-го--На\-ка\-мо\-ри~(1) и~полученные 
в результате развития этой модели~(2):
      \begin{itemize}
\item три множества выражаемых знаний (личностные, коллективные и~
конвенциональные)~(1), которые состоят из концептов соответству\-ющих 
категорий~(2);
\item три множества имен (форм представления концептов)~(2);
\item множество объектов интерпретации~(2);
\item три множества цифровых кодов концептов~(2) (личностные, коллективные 
и конвенциональные);
\item три множества цифровых кодов форм пред\-став\-ле\-ния концептов~(2);
\item множество цифровых кодов, построенное на основе уникальных 
идентификаторов объектов интерпретации~(2);
\item ось времени для отражения динамики процессов генерации знаний~(2);
\item ось социализации выражаемых экспертных знаний (1 и~2, так как в~модели 
Веж\-биц\-ко\-го--На\-ка\-мо\-ри нет детализации коллективных знаний в~
зависимости от числа экспертов в~группе, а в~результате развития этой модели 
добавлена их детализация);
\item система отношений между объектами интерпретации, денотатами, 
концептами вы\-ра\-жа\-емых знаний, именами и~их цифровыми кодами (1 и~2, так 
как система отношений задана в~модели Веж\-биц\-ко\-го--На\-ка\-мо\-ри только 
между знаниями с~учетом трех уровней их социализации; в~результате развития 
модели система отношений дополнена треугольником Фреге и~цифровым 
семиотическим треугольником).
\end{itemize}

      Для описания еще одного, третьего, семиотического основания необходимо 
вернуться к~определению классического треугольника Фреге. В~семиотике он определяется 
как треугольник с~тремя вершинами (денотат, концепт как идеальная вершина 
треугольника, имя как форма представления концепта), находящимися в~отношениях 
устойчивой связи, опосредованной сознанием, представляет собой устойчивое единство, 
которое посредством сенсорно воспринимаемой формы \textit{конвенционально} 
репрезентирует концепт и~денотат. Строго говоря, в~новой системе отношений 
классический треугольник Фреге применим только для случая конвенциональной 
репрезентации концепта сенсорно воспринимаемой формой. Следовательно, дополнительно 
необходим некоторый способ для построения личностного и~коллективного семиотических 
треугольников Фреге, аналогичных классическому треугольнику Фреге и~применимых в~
случае генерации личностных и~коллективных концептов. При этом необходимо учитывать 
то обстоятельство, что личностные и~коллективные концепты понимают только их авторы, 
так как для них отсутствует конвенциональная репрезентация их формой. Следовательно, 
для экспертов, участвующих в~процессе формирования новых знаний, но не являющихся 
авторами генерируемых новых концептов, они будут недоступны, если их авторы не 
эксплицируют свою личностную или коллективную репрезентацию в~форме, доступной 
другим экспертам тем или иным способом.
      
      Отметим, что в~процессе личностной репрезентации кроме объекта интерпретации, 
денотата, концепта и~имени <<задействовано>> персональное авторское сознание. 
В~приведенном определении классического треугольника Фреге говорится об устойчивой 
связи, \textit{опосредованной сознанием}, но ничего не говорится о том, как и~где (в какой 
среде или средах) эта связь закреплена материально. Остановимся кратко на истории этого 
вопроса, чтобы затем предложить способ экспликации личностной репрезентации концепта 
в форме, доступной другим экспертам.

\begin{figure*} %fig3
       \vspace*{1pt}
 \begin{center}
 \mbox{%
 \epsfxsize=138.786mm
 \epsfbox{zac-3.eps}
 }
 \end{center}
 \vspace*{-9pt}
\Caption{Семиотический нейротетраэдр и~нейроквадрат~\cite{12-zat}}
\end{figure*}
      
      В 1988~г.\ была сформирована рабочая группа <<FRamework of Information System 
COncepts~--- FRISCO>> в~рамках Международной федерации по обработке информации 
(International Federation for Information Processing~--- IFIP). Основной целью этой группы 
было создание системы определений для базовых терминов, которую затем можно было бы 
предложить использовать как терминологическую основу разработки и~описания 
информационных систем. Итоги ее работы опубликованы в~виде отчета 
      в~1998~г.~\cite{17-zat}. В~результате работы группы FRISCO, в~частности, было 
определено понятие семиотического тетраэдра с~вершинами <<объект, концепт, имя 
объекта и~интерпретатор>>. Отметим, что в~определении этого понятия есть субъект, 
который \textit{интерпретирует} объект, генерирует концепт и~имя объекта, а также 
\textit{устанавливает} связь между ними~\cite{18-zat}. Идея 
      субъ\-ек\-та-ин\-тер\-пре\-та\-то\-ра или интерпретанта\footnote{Интерпретатор~--- это 
только человек, который анализирует предмет, генерирует концепт и~его имя, а интерпретант~--- это не 
обязательно человек. В~общей семиотике функция абстрактного интерпретанта заключается в~интерпретации 
предметов, генерации концептов и~присвоении им имен~\cite[с.~15, 16]{19-zat}.}, который \textit{является 
носителем} этой связи, зародилась еще раньше в~общей семиотике и~в~рамках этой 
научной дисциплины оказалась весьма продуктивной~\cite{19-zat, 20-zat}.
      
      Однако с~точки зрения разработки и~описания информационных систем и~технологий 
включение субъектов в~дефиниции терминов вместо объектов обладает рядом недостатков. 
Основной из них заключа\-ется в~том, что в~задачах когнитивной информатики 
и~нейроинформатики, например при разработке нейрокоммуникаторов, необходимо 
различать объекты \textit{ментальной}, \textit{нейрофизиологической}, 
со\-ци\-аль\-но-ком\-му\-ни\-ка\-ци\-он\-ной и~цифровой сред. В~традиционных моделях 
с~субъек\-та\-ми-ин\-тер\-пре\-та\-то\-ра\-ми и~интерпретантами это различие между объектами трудно 
провести, так как в~них ментальная среда и~нейрофизиологическая среда (далее~--- 
нейросреда), а~также их объекты, как правило, не различаются. Поэтому 
      в~работе~\cite{12-zat} эти две среды было предложено рассматривать в~информатике 
раздельно, а систему терминов дополнить следующими понятиями (рис.~3):
      \begin{itemize}
\item <<нейроинформация>>, в~частности фик\-си\-ру\-ющая постоянные или 
временные связи между объектом интерпретации, концептом (личностным, 
коллективным или конвенциональным) и~именем объекта;
\item <<нейросемиотический тетраэдр>> с~вершинами объект, концепт, имя 
и~нейроинформация;
\item <<нейроквадрат>> как четверка кодов следующих категорий:
\begin{description}
\item[\,] К1~--- для концептов ментальных знаний человека;
\item[\,] К2~--- для слов как имен объектов и~других знаковых форм для 
обозначения ментальных знаний;
\item[\,] К3~--- для кодирования предметов матери\-альной сферы (в~общем 
случае~--- для кодирования любых объектов, в~результате семантиче\-ской 
интерпретации которых человеком определяются денотаты и~генерируются 
концепты);
\item[\,] К4~--- для кодирования в~цифровой среде нейроинформации,  
с~по\-мощью которой фиксируются связи между объектом интерпретации, 
концептом и~именем;
\end{description}
\item <<нейрокод>> как аналог компьютерных таблиц кодировки символов для 
нейроинформации.
\end{itemize}

      Определение нейросемиотического тетраэдра по своему содержанию во многом 
совпадает с~терми\-ном <<психосемиотический тетраэдр>> Ф.\,Е.~Ва\-силюка с~вершинами 
<<предмет, личностный\linebreak концепт, имя предмета и~чувственная ткань (которая
свя\-зы\-ва\-ет 
воедино первые три вершины)>>~\cite{21-zat}. Различие этих двух тетраэдров заключается 
в том, что первый описывает любые кон\-цеп\-ты (личностный, коллективный 
и~конвенциональный), а~второй был определен Ф.\,Е.~Василюком только для личностных 
концептов. В~нейросемиотическом тет\-ра\-эд\-ре три из четырех его вершин являются 
сущностями трех разных сред: ментальной, со\-ци\-аль\-но-ком\-му\-ни\-ка\-ци\-он\-ной и~нейросреды, к~
которой принадлежит нейроинформация. Таким образом, в~определении 
нейросемиотического тетраэдра использовался альтернативный подход по сравнению 
с~традиционной идеей субъ\-ек\-та-ин\-тер\-пре\-та\-то\-ра или интерпретанта.
      
      Сопоставление рис.~3 и~рис.~6 из работы~\cite{16-zat}, в~которой даны определения 
понятий <<формокод>> и~<<семокод>>, наглядно иллюстрирует то, что нейроквадрат 
является обобщением цифрового се-\linebreak миотического треугольника. Итак, нейротетраэдр 
и~нейро\-квадрат являются \textit{третьим семиотическим основанием} моделирования 
процессов генерации знаний.



\section{Семиотические модели}

      В этом разделе три семиотических основания, рассмотренные в~предыдущем 
разделе, будут использованы в~процессе построения двух моделей генерации знаний: 
модели фиксированного состояния и~нестационарной модели. Они являются результатом 
обобщения двух ранее разработанных моделей, основанных на треугольнике Фреге 
и~циф\-ро\-вом семиотическом треугольнике~\cite{16-zat, 22-zat}.
      
      Кроме трех семиотических оснований исходными данными для обобщения является 
следующее описание этих двух моделей, разработанных для итерационных процессов 
генерации знаний экспертами, которые используют информационную систему для 
фиксации эволюции объектов интерпретации, определения денотатов, сгенерированных 
ими концептов, описания которых эксперты формируют в~процессе интроспекции, имен 
концептов и~денотатов. По определению из работы~\cite{16-zat} первая модель фиксирует 
\textit{состояние процесса генерации в~момент времени}, соответствующий окончанию 
некоторой итерации этого процесса, и~состоит из:
      \begin{itemize}
\item трех сред предметной области информатики: ментальной,  
со\-ци\-аль\-но-ком\-му\-ни\-ка\-ци\-он\-ной и~цифровой;
\item треугольника Фреге, включающего объект интерпретации, концепт, 
сгенерированный или измененный на этой итерации некоторым экспертом, и~имя;
\item цифрового семиотического треугольника, включающего коды объекта 
интерпретации, концепта и~имени, сгенерированные в~этот же момент времени 
информационной системой, обеспечивающей работу экспертов.
\end{itemize}

      Согласно первой модели три вершины треугольника Фреге кодируются тремя 
цифровыми кодами разных категорий, которые генерируются в~конце каждой итерации:
      \begin{itemize}
\item семантическим кодом концепта (К1);
\item информационным кодом его имени, если оно создано экспертом 
(в~противном случае этот код равен нулю) (К2);
\item объектным кодом объекта интерпретации (К3).
\end{itemize}

      Таким образом, после завершения каждой итерации в~информационной системе по 
некоторому заданному алгоритму генерируются три цифровых\linebreak кода разных категорий. Если 
на некоторой итерации принял участие один эксперт, то создается\linebreak только одна запись 
с~результатами личностной семантической интерпретации рассматриваемого объекта и~три 
кода, а если несколько экспертов, то для каж\-до\-го из них~--- одна запись и~три кода.
      
      В процессе целенаправленной генерации знаний существует отдельный вид 
итераций для согласования личностных концептов и~имен, которые могут меняться 
в~пределах итераций, но не между ними. На этих итерациях эксперты ставят своей целью 
согласовать между собой свои личностные интерпретации и~сформировать коллективные 
концепты и~имена. Если это удается сделать, то в~информационной системе создается еще 
одна запись с~результатами коллективной семантической интерпретации и~еще три кода 
с~указанием идентификаторов всех экспертов, которые приняли учас\-тие в~процессе 
согласования и~выработали единую позицию. В~общем случае генерация и~согласование 
концептов могут быть совмещены в~рамках одной комплексной итерации. Отметим, что 
между двумя любыми итерациями объекты интерпретации могут меняться экспертами, но 
не в~пределах итераций. Регистрация в~информационной системе изменений объектов 
интерпретации между итерациями, описаний новых концептов и~результатов их 
согласования дает возможность восстановить ретроспективно все этапы процесса генерации 
знаний.
      
      Вторая модель предназначена для описания динамики процесса 
      генерации~\cite{22-zat}. Концепты, име-\linebreak на и~объекты интерпретации могут 
изменяться\linebreak в~широком диапазоне в~процессе согласования экспертами их личностных 
концептов и~имен. По определению вторая модель описывает динамику процесса генерации 
концептов одним экспертом и~состоит из:
      \begin{itemize}
\item трех сред предметной области информатики: ментальной,  
со\-ци\-аль\-но-ком\-му\-ни\-ка\-ци\-он\-ной и~цифровой;
\item треугольников Фреге, построенных экспертом в~моменты времени 
окончания итераций~$t_i$, $i \hm= 1, 2,\ldots$;
\item цифровых семиотических треугольников, построенных информационной 
системой в~эти же моменты времени~$t_i$.
\end{itemize}

      На основе этой модели динамики процесса генерации концептов 
      в~работе~\cite{22-zat} было дано определение пространства Фреге как 4-мерного 
множества точек для трех кодов разных категорий $\{t_i$, семантический код ($t_i$), 
информационный код ($t_i$), объектный код ($t_i$) при $i \hm= 1, 2,\ldots\}$, 
сгенерированных информационной системой в~процессе работы одного эксперта. 
Аналогичную модель и~соответствующее пространство Фреге можно определить для случая 
генерации знаний коллективом экспертов, пример которого рассматривается 
      в~работе~\cite{23-zat}.
      
      Пространство Фреге имеет три оси координат цифровых кодов: семантическую, 
информационную и~объектную, а также четвертую~--- ось времени, содержащую 
дискретный набор точек начала и~окончания итераций генерации концептов,\linebreak их 
согласования или комплексных итераций. Пространст\-во Фреге дает возможность 
представить графически динамику процесса, используя последовательности значений 
семантических, информационных и~объектных кодов, сгенерированные информационной 
системой в~дискретные моменты времени окончания итераций.
      
      Отметим, что вид дискретных траекторий точек будет определяться алгоритмами 
назначения семантических, информационных и~объектных кодов. Можно ли задать 
некоторую метрику в~пространстве Фреге? В~настоящее время этот вопрос остается 
открытым. Пока этот вопрос находится в~стадии изучения, термин <<пространство Фреге>> 
определен только как семиотическое понятие, но не математическое. При этом если в~
информационной системе фиксируется содержательная эволюция объектов интерпретации, 
сгенерированных концептов, описания которых эксперты формируют в~процессе 
интроспекции, и~имен, то пространство Фреге служит для количественного описания 
процесса генерации концептов.
      
      Прежде чем приступить к~построению модели фиксированного состояния 
      и~нестационар-\linebreak ной модели на основе обобщения двух ранее 
      раз\-работанных моделей, отметим, 
что в~последних использова\-лись только три среды предметной об\-ласти информати\-ки 
(ментальная, со\-ци\-аль\-но-ком\-му\-ни\-ка\-ци\-он\-ная и~цифровая среды) и~первые два 
семиотических основания для моделирования процессов генерации знаний (треугольник 
Фреге\linebreak и~цифровой семиотический треугольник). Пере\-чис\-лим исходные данные построения 
модели фиксированного состояния и~нестационарной модели на основе обобщения двух 
ранее разработанных моделей:
      \begin{itemize}
\item определение семиотического нейротетраэдра и~нейроквадрата, которые 
служат третьим семиотическим основанием;
\item первая исходная модель, которая фиксирует состояние процесса генерации 
концептов в~момент времени, соответствующий окончанию некоторой итерации 
этого процесса;
\item вторая исходная модель, которая описывает динамику процесса генерации 
концептов.
\end{itemize}

      Для обобщения перечисленных двух моделей рассмотрим четыре среды предметной 
области информатики как ин\-фор\-ма\-ци\-он\-но-компью\-тер\-ной науки: ментальную, 
со\-ци\-аль\-но-ком\-му\-ни\-ка\-ци\-он\-ную, нейро- и~цифровую среды~\cite{12-zat}. 
С~формальной 
точки зрения это обобщение представляет собой замену треугольника Фреге на 
семиотический тет\-раэдр, а цифрового семиотического треугольника~--- на нейроквадрат кодов 
(см.\ рис.~3). Сделав такую замену, получаем два следующих обобщения.

      \begin{figure*}[b] %fig4
\vspace*{9pt}
      \begin{center}
      {\small
      \begin{tabular}{|c|p{65mm}|p{65mm}|}
      \hline
\multicolumn{1}{|c|}{Номер пары}&
\multicolumn{1}{c|}{Оригинальный текст}&
\multicolumn{1}{c|}{Перевод}\\
\hline
\hphantom{9}9&Цвет лица у Ильи Ильича не был ни румяный, ни смуглый, ни положительно бледный, а 
безразличный или казался таким, может быть, потому, что Обломов как-то обрюзг не по 
летам: от недостатка ли движения или воздуха, а может быть, того и~другого.&Le teint d'Ilia 
Ilitch n'$\acute{\mbox{e}}$tait ni rose, ni h$\hat{\mbox{a}}$l$\acute{\mbox{e}}$, 
ni carr$\acute{\mbox{e}}$ment p$\hat{\mbox{a}}$le, mais indiff$\acute{\mbox{e}}$rent ou, du 
moins, il le paraissait. Peut-$\hat{\mbox{e}}$tre parce que la chair d'Oblomov 
$\acute{\mbox{e}}$tait pr$\acute{\mbox{e}}$matur$\acute{\mbox{e}}$ment flasque: faute 
d'exercice ou manque d'air, peut-$\hat{\mbox{e}}$tre l'un et l'autre.\\
\hline
18&Халат имел в~глазах Обломова тьму неоцененных достоинств: он мягок, гибок; тело не 
чувствует его на себе; он, как послушный раб, покоряется самомалейшему движению 
тела.&Aux yeux d'Oblomov cette robe de chambre avait une foule de qualit$\acute{\mbox{e}}$s 
inappr$\acute{\mbox{e}}$ciables: elle $\acute{\mbox{e}}$tait douce, souple, ne pesait pas sur le 
corps; telle une esclave docile, elle se pliait au moindre mouvement.\\
\hline
21&Лежанье у Ильи Ильича не было ни необходимостью, \textbf{как} у~больного или как 
\textbf{у~человека, который хочет спать}, ни случайностью, как у~того, кто устал, ни 
наслаждением, как у лентяя: это было его нормальным состоянием.&La position 
allong$\acute{\mbox{e}}$e n'$\acute{\mbox{e}}$tait pour Ilia Ilitch ni 
n$\acute{\mbox{e}}$cessaire, \textbf{comme} pour un malade ou \textbf{pour un homme qui 
veut dormir}, ni accidentelle, comme pour une personne fatigu$\acute{\mbox{e}}$e, ni 
voluptueuse comme chez le fain$\acute{\mbox{e}}$ant; c'$\acute{\mbox{e}}$tait son 
$\acute{\mbox{e}}$tat normal.\\
\hline
\end{tabular}
}
\end{center}

\vspace*{-3pt}

\Caption{Три предложения параллельных текстов на русском языке и~их переводы ({полужирным 
шрифтом выделен контекст, используемый далее на рис.}~6)}
\end{figure*}
      
      Для случая четырех сред модель фиксированного состояния процесса генерации 
знаний со\-сто\-ит~из:
      \begin{itemize}
\item ментальной, со\-ци\-аль\-но-ком\-му\-ни\-ка\-ци\-он\-ной, цифровой сред и~
нейросреды;
\item семиотического тетраэдра, включающего объект интерпретации, концепт, 
сгенерированный или измененный на этой итерации некоторым экспертом, имя 
объекта интерпретации, которое одновременно является и~именем концепта 
в~этот момент времени, а также нейроинформацию о связях между объектом, 
концептом и~их именем;
\item нейроквадрата, включающего цифровые коды объекта, концепта, имени 
и~связывающей их нейроинформации, сгенерированные в~этот же момент времени 
информационной системой, обеспечивающей работу экспертов.
\end{itemize}

      Нестационарная модель динамики процесса генерации знаний одним экспертом 
состоит из:
      \begin{itemize}
\item тех же самых четырех сред предметной области информатики;
\item семиотических тетраэдров, построенных экспертом в~моменты времени 
окончания итераций~$t_i$, $i \hm= 1, 2,\ldots$;
\item нейроквадратов, построенных в~эти же моменты времени~$t_i$.
\end{itemize}

      На основе этой модели динамики процесса аналогично можно определить 
обобщенное пространство Фреге как 5-мер\-ное множество точек для четырех кодов разных 
категорий $\{t_i$, семантический код ($t_i$), информационный код ($t_i$), объектный код 
($t_i$), код нейроинформации ($t_i$) при $i \hm= 1, 2,\ldots\}$, сгенерированных 
информационной системой на $i$-й итерации работы эксперта.
      


      Главное содержание приведенного формального обобщения состоит в~замене 
треугольника Фреге на семиотический тетраэдр, основное отличие которого от 
треугольника Фреге заключается в~наличии нейроинформации. По определению она 
фиксирует связи между объектом, концептом и~именем. Однако остается открытым вопрос 
о практических способах получения нейроинформации об этих связях в~процессе решения 
прикладных задач. Раньше, когда такие задачи решались с~использованием моделей, 
которые охватывали объекты только трех сред, объекты интерпретации были доступны 
экспертам для изменений и~анализа, так как они представляли собой изменяемые во 
времени:
      \begin{itemize}
\item компьютерные программы и~данные, используемые для вычисления 
значений новых индикаторов~\cite{9-zat, 23-zat};
\item фрагменты параллельных текстов на русском и~французском языке (рис.~4), 
в результате контрастивного анализа которых определялись денотаты и~
формировались их кросс-язы\-ко\-вые концепты в~процессе анализа 
параллельных фрагментов~\cite{24-zat, 25-zat}.
      \end{itemize}
      
      Новые концепты знаний, принадлежащие ментальной среде, описывались 
экспертами в~результате субъективной интроспекции с~последующим присвоением имен 
сформированным ими концептам. Иначе говоря, в~моделях, которые охватывали объекты 
трех сред, эксперты сами анализировали объекты интерпретации, описывали концепты 
и~давали имена. После расширения числа сред до четырех в~обобщенных моделях появляется 
нейроинформация о связях между объектом интерпретации, концептом и~именем, которая 
экспертам недоступна. Поэтому и~возникает вопрос о~способах получения 
нейроинформации в~процессе решения практических задач.
      
      Сегодня есть возможность отобразить в~компьютерной форме уровень активности 
разных участ\-ков мозга экспертов в~режиме реального времени, используя метод 
функциональной маг\-нит-\linebreak но-ре\-зо\-нанс\-ной томографии (functional Magnetic Resonance 
Imaging~--- fMRI)~\cite{26-zat}. Этот метод позволяет использовать объективные 
индикаторы уровня активности, наблюдая количественные измерения мозговой 
деятельности, одновременно фиксируя и~описывая концепты как результаты личностного 
анализа экспертами объектов интерпретации в~процессе субъективной интроспекции.
      
      Но и~здесь возникают вполне закономерные вопросы. Можно ли использовать 
объективные индикаторы уровня активности в~процессе генерации новых концептов для 
описания связей между объектом интерпретации, концептом и~именем, а также для 
сопоставления с~результатами личностного анализа? Можно ли их использовать для 
описания процесса согласования новых концептов между экспертами?
      
      В настоящее время действительно есть возможность наблюдать одновременно и~
количественные\linebreak данные измерения мозговой деятельности, и~результаты субъективного 
мышления в~процессе субъективной интроспекции, но из первых сегодня трудно получить 
именно ту нейроинформацию, которая соответствует личностным или согласованным 
концептам экспертов, чтобы провести ее сопоставление с~результатами личностного 
субъективного анализа. Кроме того, есть гипотеза и~подтверждающие ее 
экспериментальные данные, что у экспертов часть нейроинформации, соответствующей 
устоявшимся конвенциональным концептам знаний, носит структурный характер, скорее 
всего, на уровне связей между нейронами долговременной памяти, что не фиксируется 
fMRI и~другими современными методами. Стремительное развитие когнитивной 
нейронауки и~ее инструментальных средств позволяет надеяться, что в~будущем станет 
возможным соотнести структурную нейроинформацию и~количественные нейроданные 
измерений мозговой деятельности с~устоявшимися и~новыми концептами экспертных 
знаний~[26--29].

      
      Однако в~настоящее время при разработке информационных технологий и~решении 
практических задач, когда недоступна \textit{объективная нейроинформация} о~связях 
между объектом интерпретации, концептом и~именем, предлагается по-преж\-не\-му 
использовать результаты \textit{субъективной интроспекции}. Иначе говоря, в~процессе 
итерационной генерации новых знаний эксперты в~информационной системе должны 
описывать не только свои кон\-цеп\-ты и~присваивать им имена, но также устанавливать 
и~фиксировать их связи с~объектами интерпретации, определенными ими денотатами 
и~присвоенными именами.
      
      Предлагаемый подход позволяет уже сегодня использовать обобщенные модели при 
разработке информационных технологий, поддерживающих генерацию новых знаний, 
и~решении практических задач. Кроме того, использование цифровой среды информационной 
системы как носителя этих связей обеспечит доступ всех экспертов к~описаниям 
личностных и~коллективных концептов. Другими словами, анализировать и~обсуждать 
описания таких концептов смогут все эксперты, а не только их авторы, если в~процессе 
итерационной генерации новых знаний эксперты в~информационной системе описывают 
свои концепты, устанавливают и~фиксируют их связи с~объектами интерпретации, 
определенными ими денотатами и~присвоенными именами.


\section{Технология, обеспечивающая генерацию знаний}


      Модели фиксированного состояния и~динамики процесса генерации знаний были 
использованы при разработке информационной технологии,\linebreak
 обеспечивающей 
целенаправленную генерацию и~развитие кросс-язы\-ко\-вых знаний коллективом 
экспертов. Необходимость разработки подобной технологии проявляется наиболее 
наглядно в~ситуации, когда необходимо повысить качество машинного перевода и~для этого 
требуется существенное развитие контрастивных грамматик на основе формирования 
новых кросс-язы\-ко\-вых знаний. При этом направления развития контрастивных 
грамматик должны определяться явно эксплицированными целями, достижение которых 
и~должно непосредственно способствовать повышению качества машинного перевода. 

\begin{figure*}[b] %fig5
\vspace*{1pt}
 \begin{center}
 \mbox{%
 \epsfxsize=165.062mm
 \epsfbox{zac-5.eps}
 }
 \end{center}
 \vspace*{-9pt}
\Caption{Основные этапы технологии (нейросреда не показана;  
со\-ци\-аль\-но-ком\-му\-ни\-ка\-ци\-он\-ная среда для краткости обозначена как 
<<Информационная среда>>)}
\end{figure*}

%\begin{multicols}{2}

      
      При таком подходе кроме моделей состояния и~динамики процесса генерации 
знаний необходимо использовать некоторый способ описания целей. Как было уже 
отмечено, рассматрива\-емый подход к~моделированию процесса генерации знаний при 
явном описании целей и~разработке обеспечивающей технологии ориентирован на те 
прикладные области, где генерируемые экспертные знания являются результатом анализа 
объектов\linebreak интерпретации. В~рассматриваемом примере це\-ле\-на\-прав\-лен\-но\-го формирования 
кросс-язы\-ко\-вых знаний объектами интерпретации являются предложения параллельных 
текстов на рус\-ском и~французском языках, а~денотатами~--- пары тех параллельных 
фрагментов, которые выделяются\linebreak экспертами согласно рассматриваемому ими 
направлению развития контрастивной грамматики (выделенные полужирным шрифтом на 
рис.~4 параллельные фрагменты в~паре №\,21 станут далее одним из объектов анализа).

%\pagebreak

%\end{multicols}




      
      
      Отличительная черта предлагаемого подхода к~моделированию заключается в~явном 
описании
 отношений между новыми экспертными знаниями, объектами интерпретации 
и~денотатами, на основе анализа которых могут быть сгенерированы элементы новых знаний (т.\,е.\ не 
каждый анализируемый объект и~определенный в~процессе анализа денотат всегда 
порождают новый концепт). Реализуемость такого подхода была продемонстрирована 
в~процессе выполнения контрастивных исследований, включающих задачи целенаправленной 
генерации кросс-язы\-ко\-вых знаний:
      \begin{itemize}
\item о переводах глагольных конструкций русского языка на французский;
\item о возможных вариантах перевода лингвоспецифичных слов русского языка 
на французский.
\end{itemize}

      При проведении этих контрастивных исследований кросс-язы\-ко\-вые знания 
формировались экспертами в~процессе анализа параллельных текстов на русском 
и~французском языках с~использованием НДБ~\cite{30-zat, 31-zat}. 
Отметим, что переводной текст является результатом применения переводчиком как 
конвенциональных знаний (в~этом случае анализ соответствующих параллельных текстов 
не приводит к~генерации новых концептов), так и~его невыражаемых знаний, что может 
привести к~генерации новых концептов. Невыражаемые знания могут использоваться 
переводчиками неявно, при этом быть новыми и~неописанными в~контрастивных 
грамматиках в~явной (эксплицитной) форме. В~приведенных далее примерах 
рас\-смат\-ри\-ва\-ют\-ся оригинальные тексты на русском языке, при переводе которых на 
французский язык невыражаемые знания использовались переводчиками, что и~нашло свое 
отражение в~результатах перевода. Поэтому результаты сопоставления оригинальных 
текстов на русском языке и~их переводов могут помочь сформировать и~описать новые 
знания.

     
      
      Разработанная технология~\cite{24-zat, 32-zat, 33-zat}, обеспечивающая генерацию и~
целенаправленное формирование кросс-язы\-ко\-вых знаний, основана на методике, 
созданной Анной А.~Зализняк~[32--34], и~включает следующие 
основные этапы (рис.~5):
      \begin{itemize}
\item из корпуса параллельных текстов отбираются пары предложений как 
объекты интерпретации, содержащие исследуемые языковые объекты (см.\ пару 
выделенных фрагментов на рис.~4, которая является примером 
денотата)\footnote{На рис.~4 приведена пара предложений №\,21 с~глаголом настоящего 
времени русского языка, который в~приведенном далее примере станет исследуемым языковым 
объектом. На момент проведения эксперимента, описанного в~статье, в~рус\-ско-фран\-цуз\-ском 
подкорпусе Национального корпуса русского языка было около 4~тыс.\ пар с~глаголами 
настоящего времени (сейчас их около 20~тыс.).};
\item в~каждой из отобранных пар предложений эксперты анализируют перевод 
исследуемого языково\-го объекта на французский язык и~определяют его 
функционально эквивалентный фрагмент (ФЭФ), в~терминологии 
Д.\,О.~Добровольского;
\item языковой объект текста оригинала сопоставляется с~его ФЭФ согласно 
заданному направлению развития контрастивной грамматики\footnote[2]{Например, 
целью развития контрастивной грамматики может быть формирование расширенного списка 
вариантов перевода глагольных конструкций, включая низкочастотные варианты, 
в том числе такие 
варианты, которые могут отсутствовать в~суще\-ст\-ву\-ющих описаниях контрастивной 
грамматики~\cite{35-zat, 36-zat}, но которые могут использоваться переводчиками. Тогда их 
можно извлекать в~процессе сопоставления текстов оригинала и~перевода.};
\item результат сопоставления описывается экспертами в~формализованном виде 
с использованием методики Анны А.~Зализняк;
\item если вариант перевода исследуемого языкового объекта из текста оригинала 
уже включен в~существующие контрастивные грамматики (это случай пары 
выделенных фрагментов на рис.~4), то формализованное описание 
анализируемого экземпляра переводного соответствия вводится в~НБД 
без дополнения (развития) контрастивной грамматики;
\item если вариант перевода исследуемого языкового объекта из текста оригинала 
экспертами считается новым, то формализованное описание этого экземпляра 
переводного соответствия вводится в~НБД, а в~типологию включается новый вид 
соответствия исследуемого языкового объекта и~его ФЭФ;
\item одновременно НБД генерирует четыре цифровых идентификатора (см.\ рис.~3) 
для обозначения:
\begin{itemize}
\item объекта интерпретации с~выделенным в~нем денотатом, который 
представляет собой пару параллельных текстовых фрагментов;
\item концепта денотата как исследуемого языкового объекта или явления, 
который эксперты описывают в~формализованном виде;
\item имени концепта, которое одновременно является и~именем денотата;
\item связей между объектом интерпретации, включающим денотат, 
концептом и~именем.
\end{itemize}
\end{itemize}

 \begin{figure*} %fig6
      \begin{center}
      {\small
      \begin{tabular}{|p{40mm}| p{30mm}| p{40mm}| p{30mm}|}
      \hline
\multicolumn{1}{|c|}{\tabcolsep=0pt\begin{tabular}{c}Контекст\\ глагольной\\ конструкции\end{tabular}}&
\multicolumn{1}{c|}{\tabcolsep=0pt\begin{tabular}{c}Вид глагольной\\ конструкции\\ и~грамматические\\ 
признаки ее контекста\end{tabular}}&\multicolumn{1}{c|}{Контекст ФЭФ}&
\multicolumn{1}{c|}{\tabcolsep=0pt\begin{tabular}{c}Вид  конструкции ФЭФ\\
и~грамматические\\ признаки его контекста\end{tabular}}\\
\hline
как $[$\ldots$]$ у человека, который \textbf{хочет спать}, 
&\multicolumn{1}{l|}{\tabcolsep=0pt\begin{tabular}{l}\textbf{НастВ}\\
$\langle$~SubInf-IPF~$\rangle$\\
$\langle$~SubAttr~$\rangle$\end{tabular}}&
comme $[$\ldots$]$ pour un homme qui \textbf{veut dormir}, &
\multicolumn{1}{l|}{\tabcolsep=0pt\begin{tabular}{l}\textbf{Present}\\ $\langle$~SubInf~$\rangle$\\ 
$\langle$~SubAttr~$\rangle$\end{tabular}}\\
\hline
 \end{tabular}}
\end{center}
\Caption{Формализованное описание глагольной конструкции и~ее ФЭФ с~известным типологическим видом 
соответствия (НастВ, pr$\acute{\mbox{e}}$sent) ({контекст извлечен из пары №\,$21$ на рис.}~4)}
\end{figure*}


      
      Реализуемость разработанной технологии была проверена в~процессе проведения 
эксперимента по анализу переводов глагольных конструкций русского языка на 
французский~\cite{25-zat}. Целью анализа было развитие типологии видов соответствия 
исследуемых глагольных конструкций и~их ФЭФ. Из рус\-ско-фран\-цуз\-ско\-го 
подкорпуса Национального корпуса русского языка (НКРЯ) были отобраны\linebreak 
около~4000~пар предложений, содержащих глагольные конструкции настоящего времени 
русского языка и~их переводы на французский. Согласно работам~\cite{35-zat, 36-zat} они 
могут быть переведены с~помощью следующих~9~грамматических конструкций: 
pr$\acute{\mbox{e}}$sent, imparfait, infinitif, pass$\acute{\mbox{e}}$ 
compos$\acute{\mbox{e}}$, futur simple, subjonctif pr$\acute{\mbox{e}}$sent, 
g$\acute{\mbox{e}}$rondif, futur imm$\acute{\mbox{e}}$diat и~imp$\acute{\mbox{e}}$ratif.
      
      Таким образом, до начала эксперимента типология видов включала девять записей 
для русского настоящего времени (НастВ). Первая запись имела вид (НастВ, pr$\acute{\mbox{e}}$sent), 
вторая~--- (НастВ, imparfait) и~т.\,д.\ до (НастВ, imp$\acute{\mbox{e}}$ratif). Во время эксперимента эксперты 
сравнивали оригинальный и~переведенный текст в~отобранных парах предложений, выделяя 
глагольную конструкцию НастВ в~тексте оригинала и~ее ФЭФ в~тексте перевода, которые 
в~совокупности являются денотатом.
      
      Если вариант перевода глагольной конструкции НастВ уже был изначально включен 
в типологию (видов соответствия), то формализованное описание этой конструкции и~ее 
ФЭФ (рис.~6) до\-бав\-ля\-ют\-ся в~НБД без изменения типологии видов. Если вариант 
перевода глагольной конструкции НастВ эксперты считают новым, то формализованное 
описание этой конструкции и~ее ФЭФ добавляются в~НБД, а~в~типологию включается 
новый вид (см.\ рис.~5). Формализованное описание создается экспертами на основе 
смыслового содержания соответствия глагольной конструкции и~ее ФЭФ. Это смысловое 
содержание вида соответствия и~является тем концептом, который формируется в~процессе 
анализа этого соответствия. Если концепт оказывается новым, то типология дополняется 
его именем в~формате (вид глагольной конструкции, вид ее ФЭФ).

\begin{table*}\small
\begin{center}
\Caption{Четыре новых типологических вида}
\vspace*{2ex}

\begin{tabular}{|c|l|c|c|}
\hline
№ п/п &\multicolumn{1}{|c|}
{\tabcolsep=0pt\begin{tabular}{c}Типологический\\ вид соответствия\end{tabular}}&
\tabcolsep=0pt\begin{tabular}{c}Число\\ экземпляров\\ вида в~НБД\end{tabular}&
\tabcolsep=0pt\begin{tabular}{c}Статус вида\\ 
(\textit{известный} до начала эксперимента\\ или \textit{новый})\end{tabular}\\
\hline
1&(НастВ, pr$\acute{\mbox{e}}$sent)&311\hphantom{99}&\textit{известный}\\
2&(НастВ, imparfait)&53\hphantom{9}&\textit{известный}\\
3&(НастВ, infinitif)&15\hphantom{9}&\textit{известный}\\
4&(НастВ, pass$\acute{\mbox{e}}$ compos$\acute{\mbox{e}}$)&8&\textit{известный}\\
5&(НастВ, conditionnel pr$\acute{\mbox{e}}$sent)&7&\textit{новый}\\
6&(НастВ, futur simple)&5&\textit{известный}\\
7&(НастВ, subjonctif pr$\acute{\mbox{e}}$sent)&4&\textit{известный}\\
8&(НастВ, participe pass$\acute{\mbox{e}}$)&2&\textit{новый}\\
9&(НастВ, g$\acute{\mbox{e}}$rondif)&2&\textit{известный}\\
10\hphantom{9}&(НастВ, subjonctif imparfait)&1&\textit{новый}\\
11\hphantom{9}&(НастВ, plus-que-parfait)&1&\textit{новый}\\
12\hphantom{9}&(НастВ, futur imm$\acute{\mbox{e}}$diat)&0&\textit{известный}\\
13\hphantom{9}&(НастВ, imp$\acute{\mbox{e}}$ratif)&0&\textit{известный}\\
\hline
\multicolumn{2}{|l|}{Всего записей}&409\hphantom{99}&\\
\hline
\end{tabular}
\end{center}
\end{table*}


      
      Данные эксперимента по извлечению и~описанию новых концептов, полученные на 
первом и~втором его этапах, приведены в~табл.~1 и~2 соответственно.
      

\begin{table*}\small
\begin{center}
\Caption{Восемь новых типологических видов}
\vspace*{2ex}

\begin{tabular}{|c|l|c|c|}
\hline
№ п/п&\multicolumn{1}{|c|}
{\tabcolsep=0pt\begin{tabular}{c}Типологический\\ вид соответствия\end{tabular}}&
\tabcolsep=0pt\begin{tabular}{c}Число\\ экземпляров\\ вида в~НБД\end{tabular}&
\tabcolsep=0pt\begin{tabular}{c}Статус вида\\ 
(\textit{известный} до начала эксперимента\\ или \textit{новый})\end{tabular}\\
\hline
1&(НастВ, pr$\acute{\mbox{e}}$sent)&1587\hphantom{99}&\textit{известный}\\
2&(НастВ, imparfait)&328\hphantom{9}&\textit{известный}\\
3&(НастВ, infinitif)&71&\textit{известный}\\
4&(НастВ, pass$\acute{\mbox{e}}$ compos$\acute{\mbox{e}}$)&30&\textit{известный}\\
5&(НастВ, conditionnel pr$\acute{\mbox{e}}$sent)&23&\textit{новый}\\
6&(НастВ, participe pass$\acute{\mbox{e}}$)&22&\textit{новый}\\
7&(НастВ, subjonctif pr$\acute{\mbox{e}}$sent)&19&\textit{известный}\\
8&(НастВ, futur simple)&19&\textit{известный}\\
9&(НастВ, participe pr$\acute{\mbox{e}}$sent)&19&\textit{новый}\\
10\hphantom{9}&(НастВ, g$\acute{\mbox{e}}$rondif)&15&\textit{известный}\\
11\hphantom{9}&(НастВ, futur imm$\acute{\mbox{e}}$diat)&10&\textit{известный}\\
12\hphantom{9}&(НастВ, pass$\acute{\mbox{e}}$ simple)&10&\textit{новый}\\
13\hphantom{9}&(НастВ, plus-que-parfait)&\hphantom{9}8&\textit{новый}\\
14\hphantom{9}&(НастВ, subjonctif imparfait)&\hphantom{9}6&\textit{новый}\\
15\hphantom{9}&(НастВ, imp$\acute{\mbox{e}}$ratif)&\hphantom{9}5&\textit{известный}\\
16\hphantom{9}&(НастВ, infinitif pass$\acute{\mbox{e}}$)&\hphantom{9}3&\textit{новый}\\
17\hphantom{9}&(НастВ, pass$\acute{\mbox{e}}$ imm$\acute{\mbox{e}}$diat)&\hphantom{9}1&\textit{новый}\\
\hline
\multicolumn{2}{|l|}{Всего записей}&2176\hphantom{99}&\\
\hline
\end{tabular}
\end{center}
\end{table*}

      
      Таблица~1 содержит результаты первого этапа анализа 409~пар предложений 
из~4000, т.\,е.\ приблизительно 10\% от общего их числа. На этом этапе эксперты выявили и~
описали четыре новых типологических вида перевода русского настоящего времени, 
которым присвоили следующие имена: (НастВ, conditionnel pr$\acute{\mbox{e}}$sent), 
(НастВ, participe pass$\acute{\mbox{e}}$), (НастВ, subjonctif imparfait) и~(НастВ, 
      plus-que-parfait). В~то же самое время они не нашли примеры вариантов перевода с~
глагольными конструкциями futur imm$\acute{\mbox{e}}$diat и~imp$\acute{\mbox{e}}$ratif.
      
      Таким образом, первый вариант описания цели развития этой типологии видов мог 
бы состоять в~том, чтобы найти примеры для всех изначально известных девяти 
типологических видов и~описать найденные новые виды (как минимум найти и~описать 
один новый вид). В~этом случае обработка 409~пар недостаточна, так как для достижения 
такой цели эксперты должны продолжать искать примеры французских переводов с~
глагольными конструкциями futur imm$\acute{\mbox{e}}$diat и~imp$\acute{\mbox{e}}$ratif 
(см.\ табл.~1).
      
      Таблица~2 содержит результаты второго этапа анализа 2176~пар предложений 
из~4000, т.\,е.\ приблизительно 54\% от общего их числа. Эксперты выявили и~описали еще 
четыре новых типологических вида перевода русского настоящего времени: (НастВ, 
participe pr$\acute{\mbox{e}}$sent), (НастВ, pass$\acute{\mbox{e}}$ simple), (НастВ, infinitif 
pass$\acute{\mbox{e}}$) и~(НастВ, pass$\acute{\mbox{e}}$ imm$\acute{\mbox{e}}$diat) 
(см.\ табл.~2).
      

      
      Одновременно они нашли французские переводы с~глагольными конструкциями 
futur imm$\acute{\mbox{e}}$diat и~imp$\acute{\mbox{e}}$ratif. В~итоге проведенного 
эксперимента все восемь новых типологических видов были добавлены экспертами 
к~девяти уже имеющимся в~типологии. Дополненная типология включала 17~видов после 
обработки 2176~пар предложений из~4000 (см.\ табл.~2).
      
      Второй вариант описания цели развития этой типологии видов мог бы состоять 
      в~том, чтобы \mbox{найти} примеры для всех изначально известных девяти видов, 
      описать 
найденные новые типологические виды (как минимум найти и~описать один новый вид) при 
условии, что эксперты должны обработать не менее чем 55\% от всех пар предложений 
с~глагольной конструкцией НастВ, имеющихся в~рус\-ско-фран\-цуз\-ском подкорпусе НКРЯ, 
т.\,е.\ 55\% от~4000 на момент проведения эксперимента. В~этом случае анализ 2176~пар 
был недостаточен и~эксперты должны были бы продолжить свою работу, пока не будет 
обработано 2200~пар.
      
      Кроме развития типологии видов соответствия глагольной конструкции и~ее ФЭФ 
разработанная технология в~настоящее время используется для формирования списков 
возможных вариантов перевода лингвоспецифичных слов русского языка на французский 
язык~\cite{37-zat}, а также для формирования методологии контрастивного корпусного 
исследования категории безличности в~русском языке. Таким образом, компьютерная 
поддержка процессов генерации и~целенаправленного развития экспертами  
кросс-язы\-ко\-вых знаний была опробована в~процессе развития контрастивной  
рус\-ско-фран\-цуз\-ской грамматики для глагольных конструкций и~продолжает 
применяться для контрастивных исследований лингвоспецифичных слов русского языка.

\section{Заключение}

      Двуязычные параллельные корпуса, в~которых каждому тексту на русском языке 
соответствует один или несколько его переводов на другой язык, являются потенциальным 
и неисчерпаемым источником генерации новых, но трудно извлека\-емых кросс-язы\-ко\-вых 
знаний. Являясь уникальным и~постоянно пополняемым, он может быть использован для 
существенного повышения качества машинного перевода, актуализации моно- 
и~двуязычных грамматик, а также для обновления широкого спектра образовательных курсов 
по лингвистике, теории и~практике перевода.
      
      Однако функциональность традиционных электронных корпусов не обеспечивает 
извлечения тех невыражаемых знаний переводчиков, которые применялись ими в~процессе 
перевода. Эти знания могут быть личностными или коллективными и~передаваться 
в~процессе их социализации (см.\ рис.~1), например в~процессе демонстрации образцов 
перевода в~процессе обучения, но при этом они могут продолжать оставаться 
невыражаемыми и~неэксплицированными. Наблюдается парадокс: с~одной\linebreak стороны, 
в~электронных корпусах есть образцы переводов, полученные с~применением невыра-\linebreak жа\-емых 
знаний переводчиков; с~другой стороны, традиционные параллельные корпуса не могут 
поддержать процессы извлечения и~экспликации этих знаний. Поэтому понадобилось 
существенное дополнение функциональности традиционных корпусов за счет разработки 
новой информационной технологии целенаправленной генерации знаний.
      
      Разработка этой технологии была связана с~развитием семиотических оснований 
информатики как ин\-фор\-ма\-ци\-он\-но-компью\-тер\-ной науки и~созданием новых 
моделей целенаправленной генерации и~развития новых знаний с~использованием четырех 
сред ее предметной области. К~ментальной, со\-ци\-аль\-но-ком\-му\-ни\-ка\-ци\-он\-ной и~цифровой 
средам была добавлена нейросреда. Ее добавление стало основой для определения 
нейросемиотического тет\-ра\-эдра, что представляет собой качественно новое развитие 
понятия семиотического тет\-ра\-эдра, предложенного группой FRISCO в~конце прошлого 
века.
      
      Суть этого развития в~том, что предложено разделять объекты ментальной среды 
      и~нейросреды в~предметной области информатики. На практике такое деление уже 
используется в~процессе разработки ряда когнитивных и~нейрокоммуникационных 
технологий. Следовательно, это должно найти свое отражение и~в теоретических 
основаниях информатики, а также в~образовательных курсах по ее изучению в~системе 
среднего и~высшего профессионального образования. Отметим, что разделение объектов 
ментальной среды и~нейросреды существенно увеличивает спектр интерфейсов, которые 
являются новыми объектами исследований в~информатике~\cite{12-zat}.
      
      Адаптация разработанной технологии для проведения контрастивных исследований 
повлекла за собой необходимость в~новой категории инфор\-мационных лингвистических 
ресурсов, получивших название НБД, методы формирования которых 
разработаны М.\,Г.~Кружковым~\cite{30-zat, 31-zat, 33-zat, 34-zat}. С~прикладной точки 
зрения реализация в~этих базах данных моделей и~технологии целена\-прав\-лен\-ной генерации 
и развития новых знаний дала возможность существенно дополнить функциональность 
электронных корпусов текстов и~тем самым обеспечить извлечение тех невыражаемых 
и~труд\-нодоступных знаний переводчиков, которые применялись ими, являясь 
неэксплицированными и~новыми в~контрастивной лингвистике.
      
{\small\frenchspacing
 {%\baselineskip=10.8pt
 \addcontentsline{toc}{section}{References}
 \begin{thebibliography}{99}
\bibitem{1-zat}
\Au{Nonaka I.} The knowledge-creating company~// Harvard Bus. Rev., 1991. 
Vol.~69. No.\,6. P.~96--104.
\bibitem{2-zat}
\Au{Nonaka I.} A dynamic theory of organizational knowledge creation~// Organ. 
Sci., 1994. Vol.~5. No.\,1. P.~14--37.
\bibitem{3-zat}
\Au{Wierzbicki A.\,P., Nakamori~Y.} Basic dimensions of creative space // Creative space: 
Models of creative processes for knowledge civilization age~/ Eds. A.\,P.~Wierzbicki, 
Y.~Nakamori.~--- Berlin--Heidelberg: Springer Verlag, 2006. P.~59--90.
\bibitem{4-zat}
\Au{Wierzbicki A.\,P., Nakamori Y.} Knowledge sciences: Some new developments~// 
Zeitschrift f$\ddot{\mbox{u}}$r Betriebswirtschaft, 2007. Vol.~77. No.\,3. P.~271--295.
\bibitem{5-zat}
\Au{Wierzbicki A.\,P., Nakamori Y.} The importance of multimedia principle and emergence 
principle for the knowledge civilisation age~// J.~Syst. Sci. Syst. Eng., 2008. 
Vol.~17. No.\,3. P.~297--318.
\bibitem{6-zat}
\Au{Nakamori Y.} Methodology for knowledge synthesis~// 
Cutting-edge research topics on  multiple criteria decision making~/ 
Eds.\ Y.~Shi, S.~Wang, Y.~Peng, J.~Li, Y.~Zeng.~---
Communications in computer and information science ser.~--- 
Berlin: Springer, 2009. Vol.~35. P.~311--317.
\bibitem{7-zat}
Knowledge science~--- modeling the knowledge creation process~/
Ed. Y.~Nakamori.~--- London\,--\,New York: CRC Press, 2011. 
177~p.
\bibitem{8-zat}
\Au{Nakamori Y.} Knowledge and systems science~--- enabling systemic knowledge 
synthesis.~--- London\,--\,New York: CRC Press, 2013. 234~p.
\bibitem{9-zat}
\Au{Zatsman I., Buntman P.} Theoretical framework and denotatum-based models of 
knowledge creation for monitoring and evaluating R\&D program implementation~// 
Int. J.~Softw. Sci. Comput. Intell., 2013. Vol.~5. No.\,1. 
P.~15--31.
\bibitem{10-zat}
\Au{Зацман И.\,М.} Построение системы терминов 
ин\-фор\-ма\-ци\-он\-но-компью\-тер\-ной науки: 
проб\-лем\-но-ори\-ен\-ти\-ро\-ван\-ный подход~// Теория и~практика общественной 
научной информации.~--- М.: \mbox{ИНИОН} РАН, 2013. Вып.~21. С.~120--159.

\bibitem{12-zat} %11
\Au{Зацман И.\,М.} Таблица интерфейсов информатики как  
ин\-фор\-ма\-ци\-он\-но-компью\-тер\-ной науки~// 
На\-уч.-тех\-нич. информация. Сер.~1: Организация и~методика информационной 
работы, 2014. №\,11. С.~1--15.

\bibitem{11-zat} %12
\Au{Зацман И.\,М.} Ин\-фор\-ма\-ци\-он\-но-компью\-тер\-ная наука: технологические 
предпосылки становления~// Информационные технологии, 2014. №\,3. С.~3--12.

\bibitem{13-zat}
\Au{Успенский В.\,А.} К~публикации статьи Г.~Фреге <<Смысл и~денотат>>~// 
Семиотика и~информатика, 1997. Вып.~35. 
С.~351--352.
\bibitem{14-zat}
\Au{Фреге Г.} Смысл и~денотат~// Семиотика и~информатика, 1997. Вып.~35. С.~352--379.
\bibitem{15-zat}
\Au{Фреге Г.} Понятие и~вещь~// Семиотика и~информатика, 1997. Вып.~35. С.~380--396.
\bibitem{16-zat}
\Au{Зацман И.\,М.} Семиотическая модель взаимосвязей концептов, информационных 
объектов и~компьютерных кодов~// Информатика и~её применения, 2009. Т.~3. Вып.~2. 
С.~65--81.
\bibitem{17-zat}
A~framework of information system concepts (Web edition): The FRISCO Report.~--- IFIP, 
1998. {\sf http://www.mathematik.uni-marburg.de/$\sim$hesse/\linebreak papers/fri-full.pdf}.
\bibitem{18-zat}
\Au{Hesse W., Verrijn-Stuart A.\,A.} Towards a~theory of information systems: The FRISCO 
approach~// Information modelling and knowledge bases~XII~/ Eds. H.~Kangassalo, 
H.~Jaakkola, E.~Kawaguchi.~--- Amsterdam: IOS Press, 2001. P.~81--91.
\bibitem{19-zat}
\Au{Eco U.} A~theory of semiotics.~--- Bloomington: Indiana University Press, 1976. 356~p.
\bibitem{20-zat}
\Au{Пирс Ч.} Логические основания теории знаков~/
Пер. с~англ.~---  СПб.: Алетейя, 2000. 352~с.

\bibitem{21-zat}
\Au{Василюк Ф.\,Е.} Структура образа~// Вопросы психологии, 1993. №\,5. С.~5--19.
\bibitem{22-zat}
\Au{Зацман И.\,М.} Нестационарная семиотическая модель компьютерного кодирования 
концептов, информационных объектов и~денотатов~// Информатика и~её применения, 
2009. Т.~3. Вып.~4. С.~87--101.
\bibitem{23-zat}
\Au{Зацман И.\,М., Бунтман П.\,С.} Проектирование индикаторов мониторинга в~сфере 
науки: теоретические основания и~модели~// Онтология проектирования, 2014. №\,3(13). 
С.~32--51.
\bibitem{24-zat}
\Au{Zatsman I., Buntman N., Kruzhkov M., Nuriev~V., Zalizniak Anna~A.} Conceptual 
framework for development of computer technology supporting cross-linguistic knowledge 
discovery~// 15th European Conference on Knowledge Management Proceedings.~---  
Reading: Academic Publishing International Ltd., 2014. Vol.~3. P.~1063--1071.
\bibitem{25-zat}
\Au{Zatsman I., Buntman N.} Outlining goals for discovering new knowledge and 
computerised tracing of emerging meanings discovery~// 16th European Conference on 
Knowledge Management Proceedings.~--- Reading: Academic Publishing International 
Ltd., 2015. P.~851--860.
\bibitem{26-zat}
\Au{Баарс~Б., Гейдж~Н.} Мозг, познание, разум: введение в~когнитивные 
нейронауки~/ Пер.\ с~англ.~--- М.: БИНОМ. Лаборатория знаний, 2014. Ч.~1. 544~с.; Ч.~2. 464~с.
(\Au{Baars~B., Gage~N.} Cognition, brain, and consciousness: Introduction to cognitive 
neuroscience.~--- Burlington, MA, USA: Academic Press/Elsevier, 2010. 677~p.)
\bibitem{27-zat}
\Au{Секерина И.\,А.} Метод вызванных потенциалов мозга в~экспериментальной 
психолингвистике~// Вопросы языкознания, 2006. №\,3. С.~22--45.
\bibitem{28-zat}
\Au{De Charms R.\,C.} Applications of real-time fMRI~// Nat. Rev. Neurosci., 
2008. Vol.~9. No.\,9. P.~720--729.
\bibitem{29-zat}
\Au{Kumaran D., Summereld J.\,J., Hassabis~D., Maguire~E.\,A.} Tracking the emergence of 
conceptual knowledge during human decision-making~// Neuron, 2009. Vol.~63. No.\,6. 
P.~889--901.
\bibitem{30-zat}
\Au{Зализняк А.\,А., Зацман И.\,М., Инькова~О.\,Ю., Кружков~М.\,Г.} Надкорпусные 
базы данных как лингвистический ресурс~// Корпусная лингвистика-2015: Тр. 7-й 
Междунар. конф.~--- СПб.: СПбГУ, 2015. С.~211--218.
\bibitem{31-zat}
\Au{Кружков М.\,Г.} Информационные ресурсы контрастивных лингвистических 
исследований: электронные корпуса текстов~// Системы и~средства информатики, 2015. 
Т.~25. Вып.~2. С.~140--159.
\bibitem{32-zat}
\Au{Loiseau~S., Sitchinava~D.\,V., Zalizniak~Anna~A., Zatsman~I.\,M.} Information 
technologies for creating the database of equivalent verbal forms in the Russian-French 
multivariant parallel corpus~// Информатика и~её применения, 2013. Т.~7. №\,2. 
С.~100--109.
\bibitem{33-zat}
\Au{Kruzhkov M.\,G., Buntman N.\,V., Loshchilova~E.\,Ju., Sitchinava~D.\,V., Zalizniak 
Anna~A., Zatsman~I.\,M.} A~database of Russian verbal forms and their French translation 
equivalents~// Компьютерная лингвистика и~интеллектуальные технологии: По мат-лам 
ежегодной Междунар. конф. <<Диалог>>.~--- М.: РГГУ, 2014. Вып.~13(20).
С.~284--297.
\bibitem{34-zat}
\Au{Бунтман Н.\,В., Зализняк Анна~A., Зацман~И.\,M., Кружков~М.\,Г., 
Лощилова~Е.\,Ю., Сичинава~Д.\,В.} Инфор\-ма\-ционные технологии корпусных 
исследований:\linebreak принципы построения кросслингвистических баз данных~// Информатика 
и её применения, 2014. Т.~8. Вып.~2. С.~98--110.
\bibitem{35-zat}
\Au{Гак В.\,Г.} Русский язык в~сопоставлении с~французским.~--- М.: УРСС, 2006. 264~с.
\bibitem{36-zat}
\Au{Kouznetsova I.\,N.} Grammaire contrastive du francais et du russe.~--- Moscow: Nestor 
Academic Publs., 2009. 272~p.
\bibitem{37-zat}
\Au{Зализняк Анна А.} Лингвоспецифичные единицы русского языка в~свете 
контрастивного корпусного анализа~// Компьютерная лингвистика и~интеллектуальные 
технологии: По мат-лам ежегодной Междунар. конф. <<Диалог>>.~--- М.: РГГУ, 2015. 
Вып.~14(21). Т.~1. С.~683--695.

 \end{thebibliography}

 }
 }

\end{multicols}

\vspace*{-3pt}

\hfill{\small\textit{Поступила в~редакцию 22.07.15}}

%\newpage

\vspace*{12pt}

\hrule

\vspace*{2pt}

\hrule

%\vspace*{12pt}

\def\tit{GOAL-ORIENTED PROCESSES OF~CROSS-LINGUAL EXPERT KNOWLEDGE CREATION:
SEMIOTIC FOUNDATIONS FOR~MODELING}

\def\titkol{Goal-oriented processes of cross-lingual expert knowledge creation:
Semiotic foundations for modeling}

\def\aut{I.\,M.~Zatsman}

\def\autkol{I.\,M.~Zatsman}

\titel{\tit}{\aut}{\autkol}{\titkol}

\vspace*{-9pt}


\noindent
Institute of Informatics Problems, Federal Research Center ``Computer Science and Control'' of 
the Russian Academy of Sciences, 44-2 Vavilov Str., Moscow 119333, Russian Federation


\def\leftfootline{\small{\textbf{\thepage}
\hfill INFORMATIKA I EE PRIMENENIYA~--- INFORMATICS AND
APPLICATIONS\ \ \ 2015\ \ \ volume~9\ \ \ issue\ 3}
}%
 \def\rightfootline{\small{INFORMATIKA I EE PRIMENENIYA~---
INFORMATICS AND APPLICATIONS\ \ \ 2015\ \ \ volume~9\ \ \ issue\ 3
\hfill \textbf{\thepage}}}

\vspace*{3pt}

\Abste{The results of development of semiotic foundations for modeling goal-oriented processes 
of cross-lingual expert knowledge creation are described. The technology supporting these 
processes is outlined. The demand for such technologies is obvious in situations where present 
systems of expert knowledge do not answer to new socially or technologically significant 
purposes, corresponding to new or changed requirements of modern society. Instead of centering 
on the well-known artificial intelligence methods and models of information processing for 
knowledge representation, this paper focuses on development of new models of goal-oriented 
processes of expert knowledge creation reflecting dynamics of its formation. The suggested 
approach to modeling these processes and to development of technologies supporting them is 
focused on those applied areas where expert knowledge is elicited from domain experts. The 
experts analyze texts or other interpretation objects which can vary over time and enter\linebreak\vspace*{-12pt}}

\Abstend{the 
results of analysis into supracorpus databases. The distinguishing feature of the semiotic approach 
to modeling is the explicit description of relations between new expert knowledge and those 
interpretation objects, from which parts of new knowledge were generated. Other important 
feature is the explicit description of parts of knowledge corresponding to interpretation objects that 
may vary over time. Feasibility of the approach is demonstrated on
the example of information 
technology, which supports the processes of creation of cross-lingual expert knowledge based on 
French translations of Russian verbal constructions. Cross-lingual knowledge is generated in the 
course of analysis of parallel texts in Russian and French languages.}

\KWE{cross-lingual expert knowledge; computer modeling; knowledge creation; interpretation 
objects; semiotic foundations; models of knowledge creation processes}

\DOI{10.14357/19922264150311}

\Ack
\noindent
The work was supported by the Russian Foundation for Basic Research 
(projects 14-07-00785, 13-06-00403) and the Rusian Foundation for Humanities 
(project 15-04-00507).

%\vspace*{3pt}

  \begin{multicols}{2}

\renewcommand{\bibname}{\protect\rmfamily References}
%\renewcommand{\bibname}{\large\protect\rm References}

{\small\frenchspacing
 {%\baselineskip=10.8pt
 \addcontentsline{toc}{section}{References}
 \begin{thebibliography}{99}
\bibitem{1-zat-1}
\Aue{Nonaka, I.} 1991. The knowledge-creating company. \textit{Harvard Bus. Rev.} 
69(6):96--104.
\bibitem{2-zat-1}
\Aue{Nonaka, I.} 1994. A~dynamic theory of organizational knowledge creation. 
\textit{Organ. Sci.} 5(1):14--37.
\bibitem{3-zat-1}
\Aue{Wierzbicki, A.\,P., and Y.~Nakamori}. 2006. Basic dimensions of creative space. 
\textit{Creative space: Models of creative processes for knowledge civilization age}. 
Eds. A.\,P.~Wierzbicki,  and
Y.~Nakamori. Berlin--Heidelberg: Springer Verlag. 59--90.
\bibitem{4-zat-1}
\Aue{Wierzbicki, A.\,P., and Y.~Nakamori}. 2007. Knowledge sciences: Some new 
developments. \textit{Zeitschrift f$\ddot{\mbox{u}}$r Betriebswirtschaft} 77(3):271--295.
\bibitem{5-zat-1}
\Aue{Wierzbicki, A.\,P., and Y. Nakamori}. 2008. The importance of multimedia principle 
and emergence principle for the knowledge civilisation age. \textit{J.~Syst. Sci. Syst.
Eng.} 17(3):297--318.
\bibitem{6-zat-1}
\Aue{Nakamori, Y.} 2009. Methodology for knowledge synthesis. \textit{Cutting-edge 
research topics on multiple criteria decision making}. 
Eds.\ Y.~Shi, S.~Wang, Y.~Peng, J.~Li, and Y.~Zeng. Communications in computer and 
information science ser. Berlin: Springer. 35:311--317.
\bibitem{7-zat-1}
Nakamori, Y., ed. 2011. \textit{Knowledge science~--- modeling the knowledge creation 
process}. London\,--\,New York: CRC Press. 177~p.
\bibitem{8-zat-1}
\Aue{Nakamori, Y.} 2013. \textit{Knowledge and systems science~--- enabling systemic 
knowledge synthesis}. London\,--\,New York: CRC Press. 234~p.
\bibitem{9-zat-1}
\Aue{Zatsman, I., and P. Buntman}. 2013. Theoretical framework and denotatum-based 
models of knowledge creation for monitoring and evaluating R\&D program implementation. 
\textit{Int. J.~Softw. Sci. Comput. Intell.} 5(1):15--31.
\bibitem{10-zat-1}
\Aue{Zatsman, I.} 2013. Postroenie sistemy terminov informatsionno-komp'yuternoy nauki: 
problemno-orientirovannyy podkhod [Construction of the system of terms of information and 
computer science: A~problem-oriented approach]. \textit{Teoriya i~praktika obshchestvennoy 
nauchnoy informatsii} [Theory and practice of scientific information for social sciences]. 
Moscow: INION RAS. 120--159.

\bibitem{12-zat-1} %11
\Aue{Zatsman, I.} 2014. Tablitsa interfeysov informatiki kak informatsionno-komp'yuternoy 
nauki [A~table of interfaces of informatics as computer and information science]. 
\textit{Nauchno-tekhnicheskaya informatsiya. Ser.~1:\linebreak Organizatsiya i~metodika 
informatsionnoy raboty} [Scientific and Technical Information. Ser.~1: Management and 
methodology of information work] (11):1--15.

\bibitem{11-zat-1} %12
\Aue{Zatsman, I.} 2014. Informatsionno-komp'yuternaya nauka: Tekhnologicheskie 
predposylki stanovleniya [Information and computer science: Technological prerequisites of 
formation]. \textit{Informatsionnye Tekhnologii} [Information Technologies] (3):3--12.

\bibitem{13-zat-1}
\Aue{Uspenskiy, V.\,A.} 1997. K~publikatsii stat'i G.~Frege ``Smysl i~denotat'' [To the 
publication of the paper of G.~Frege ``Sense and reference'']. \textit{Semiotika i Informatika} 
[Semiotics and Informatics] 35:351--352.
\bibitem{14-zat-1}
\Aue{Frege, G.} 1997. Smysl i denotat [Sense and reference]. \textit{Semiotika i~Informatika} 
[Semiotics and Informatics] 35:352--379.
\bibitem{15-zat-1}
\Aue{Frege, G.} 1997. Ponyatie i~veshch' [Concept and thing]. \textit{Semiotika 
i~Informatika} [Semiotics and Informatics] 35:380--396.
\bibitem{16-zat-1}
\Aue{Zatsman, I.} 2009. Semioticheskaya model' vzaimosvyazey kontseptov, 
informatsionnykh ob"ektov i komp'yuternykh kodov [Semiotic model of relationships of 
concepts, information objects, and computer codes]. \textit{Informatika i~ee Primeneniya}~--- 
\textit{Inform. Appl.} 3(2):65--81.
\bibitem{17-zat-1}
FRISCO.  1998. A framework of information system concepts. 
Report. Available at: {\sf 
http://www.mathematik.uni-marburg.de/$\sim$hesse/papers/fri-full.pdf} (accessed July~29, 
2015).
\bibitem{18-zat-1}
\Aue{Hesse, W., and A.\,A.~Verrijn-Stuart}. 2001. {Towards a~theory of information 
systems: The FRISCO approach}. \textit{Information modelling and knowledge bases~XII}. 
Eds. H.~Kangassalo,  H.~Jaakkola, and E.~Kawaguchi.
Amsterdam: IOS Press. 81--91.
\bibitem{19-zat-1}
\Aue{Eco, U.} 1976. \textit{A~theory of semiotics}. Bloomington: Indiana University Press. 
356~p.
\bibitem{20-zat-1}
\Aue{Peirce, Ch.\,S.} 1931--1958. \textit{Collected papers of Charles S.~Peirce}. 
Cambridge: Harvard University Press. 8~vols.
\bibitem{21-zat-1}
\Aue{Vasilyuk, F.\,E.} 1993. Struktura obraza [Structure of image]. \textit{Voprosy 
Psikhologii} [Questions of Psychology] (5):5--19.
\bibitem{22-zat-1}
\Aue{Zatsman, I.} 2009. Nestatsionarnaya semioticheskaya model' komp'yuternogo 
kodirovaniya kontseptov, informatsionnykh ob"ektov i~denotatov [Nonstationary semiotic 
model of computer coding of concepts, information objects and denotata]. \textit{Informatika 
i~ee Primeneniya}~--- \textit{Inform. Appl.} 3(4):87--101.
\bibitem{23-zat-1}
\Aue{Zatsman, I., and P. Buntman}. 2014. Proektirovanie indikatorov monitoringa v~sfere 
nauki: Teoreticheskie osnovaniya i~modeli [Design of indicators for monitoring in science: 
Theoretical foundations and models]. \textit{Ontologiya Proektirovaniya} [Ontology of 
Design] (3):32--51.
\bibitem{24-zat-1}
\Aue{Zatsman, I., N. Buntman, M.~Kruzhkov, V.~Nuriev, and Anna A.~Zalizniak}. 2014. 
Conceptual framework for development of computer technology supporting cross-linguistic 
knowledge discovery. \textit{15th European Conference on Knowledge Management 
Proceedings}. Reading: Academic Publishing International Ltd. 3:1063--1071.
\bibitem{25-zat-1}
\Aue{Zatsman, I., and N. Buntman}. 2015. Outlining goals for discovering new knowledge 
and computerised tracing of emerging meanings discovery. \textit{16th European Conference 
on Knowledge Management Proceedings}. Reading: Academic Publishing International 
Ltd. 851--860.
\bibitem{26-zat-1}
\Aue{Baars, B., and N. Gage}. 2010. \textit{Cognition, brain, and consciousness: Introduction 
to cognitive neuroscience}. Burlington, MA: Academic Press/Elsevier. 677~p.
\bibitem{27-zat-1}
\Aue{Sekerina, I.} 2006. Metod vyzvannykh potentsialov mozga v~eksperimental'noy 
psikholingvistike [Method of evoked potentials of brain in experimental psycholinguistics]. 
\textit{Voprosy Yazykoznaniya} [Topics in the Study of Language] 3:22--45.
\bibitem{28-zat-1}
\Aue{De Charms, R.\,C.} 2008. Applications of real-time fMRI. \textit{Nat. Rev. 
Neurosci.} 9(9):720--729.
\bibitem{29-zat-1}
\Aue{Kumaran, D., J.\,J. Summereld, D.~Hassabis, and E.\,A.~Maguire}. 2009. Tracking the 
emergence of conceptual knowledge during human decision-making. \textit{Neuron} 
63(6):889--901.
\bibitem{30-zat-1}
\Aue{Zalizniak, Anna A., I.~Zatsman, O.~Inkova, and M.~Kruzhkov}. 2015. Nadkorpusnye 
bazy dannykh kak lingvisticheskiy resurs [Supracorpus database as linguistic resource]. 
\textit{Tr. 7-y konf. po Korpusnoy Lingvistike} [7th Conference on Corpus Linguistics 
Proceedings]. St.\ Petersburg. 211--218.
\bibitem{31-zat-1}
\Aue{Kruzhkov, M.} 2015. Informatsionnye resursy kontrastivnykh lingvisticheskikh 
issledovaniy: Elektronnye korpusa tekstov [Information resources for contrastive studies: 
Digital text corpora]. \textit{Sistemy i~Sredstva Informatiki}~--- \textit{Systems and Means of 
Informatics} 25(2):140--159.
\bibitem{32-zat-1}
\Aue{Loiseau, S., D.\,V. Sitchinava, Anna A.~Zalizniak, and I.\,M.~Zatsman}. 2013. 
Information technologies for creating the database of equivalent verbal forms in the 
Russian-French multivariant parallel corpus. \textit{Informatika i~ee Primeneniya}~--- 
\textit{Inform.s Appl.} 7(2):100--109.
\bibitem{33-zat-1}
\Aue{Kruzhkov, M.\,G., N.\,V. Buntman, E.\,Ju.~Loshchilova, D.\,V.~Sitchinava, Anna 
A.~Zalizniak, and I.\,M.~Zatsman}. 2014. A~database of Russian verbal forms and their 
French translation equivalents. \textit{Komp'yuternaya Lingvistika i~Intellektual'nye 
Tekhnologii. Po mat-lam ezhegodnoy Mezhdunar. konf. ``Dialog-2014''} [Computational 
Linguistics and Intellectual Technologies: Conference (International) ``Dialog-2014'' 
Proceedings]. Moscow. 13(20):284--297.
\bibitem{34-zat-1}
\Aue{Buntman, N.\,V., Anna A.~Zaliznyak, I.\,M.~Zatsman, M.\,G.~Kruzhkov, 
E.\,Yu.~Loshchilova, and D.\,V.~Sitchinava}. 2014. Informatsionnye tekhnologii korpusnykh 
issledovaniy: Printsipy postroeniya krosslingvisticheskikh baz dannykh [Information 
technologies for corpus studies: Underpinnings for cross-linguistic database creation]. 
\textit{Informatika i~ee Primeneniya}~--- \textit{Inform. Appl.} 8(2):98--110.
\bibitem{35-zat-1}
\Aue{Gak, V.\,G.} 2006. \textit{Russkiy yazyk v~sopostavlenii s~fran\-tsuz\-skim} [Russian in 
comparison to French]. Moscow: URSS. 264~p.
\bibitem{36-zat-1}
\Aue{Kouznetsova, I.\,N.} 2009. \textit{Grammaire contrastive du francais et du russe}. 
Moscow: Nestor Academic Publs. 272~p.
\bibitem{37-zat-1}
\Aue{Zalizniak, Anna~A.} Lingvospetsifichnye edinitsy russkogo yazyka v~svete 
kontrastivnogo korpusnogo analiza [Russian language-specific words in light of the contrastive 
corpus analysis]. \textit{Komp'yuternaya Lingvistika i~Intellektual'nye Tekhnologii. Po 
mat-lam ezhegodnoy Mezhdunar. konf. ``Dialog-2015''} [Computational Linguistics and 
Intellectual Technologies: Conference (International) ``Dialog-2015'' Proceedings]. Moscow.  
14(21):683--695.
\end{thebibliography}

 }
 }

\end{multicols}

\vspace*{-3pt}

\hfill{\small\textit{Received July 22, 2015}}

\Contrl

\noindent
\textbf{Zatsman Igor M.} (b.\ 1952)~--- 
Doctor of Science in technology, Head of Department, Institute of Informatics Problems, Federal Research Center 
``Computer Science and Control'' of the Russian Academy of Sciences, 44-2 Vavilov Str., Moscow 119333, 
Russian Federation;  iz\_ipi@a170.ipi.ac.ru
\label{end\stat}


\renewcommand{\bibname}{\protect\rm Литература}