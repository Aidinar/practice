\def\vo{\;\mathop{\to}\limits_{r}\;}
\def\by{\begin{array}{llllllllllllll}}
\def\ey{\end{array}}
\def\bullettri{\bigtriangleup}

\def\stat{mironov}

\def\tit{РЕАЛИЗУЕМОСТЬ ВЕРОЯТНОСТНЫХ РЕАКЦИЙ КОНЕЧНЫМИ ВЕРОЯТНОСТНЫМИ АВТОМАТАМИ$^*$}

\def\titkol{Реализуемость вероятностных реакций конечными вероятностными автоматами}

\def\aut{А.\,М.~Миронов$^1$}

\def\autkol{А.\,М.~Миронов}

\titel{\tit}{\aut}{\autkol}{\titkol}

{\renewcommand{\thefootnote}{\fnsymbol{footnote}} \footnotetext[1]
{Работа выполнена при частичной поддержке  РФФИ
(проект 12-07-00109).}}


\renewcommand{\thefootnote}{\arabic{footnote}}
\footnotetext[1]{Институт проблем информатики Федерального исследовательского
центра <<Информатика и~управление>> Российской академии наук, amironov66@gmail.com}


\Abst{Формулируется и~доказывается
критерий реализуемости вероятностной реакции (ВР) конечным вероятностным
автоматом (ВА). Этот критерий усиливает соответствующий критерий
Р.\,Г.~Бухараева и~Х.~Хомута, который имеет следующий вид:
ВР  реализуется в~конечном ВА
тогда и~только тогда, когда существует конечное опорное множество
для множества состояний этой ВР,
выпуклое относительно полугруппы всех вращений.
Сформулированный и~доказанный в~настоящей работе  критерий реализуемости ВР
в~конечном ВА
не связан с~построением множества состояний для
ВР и~имеет следующий  вид: ВР
реализуема конечным ВА тогда
и~только тогда, когда она принадлежит конусу над некоторым
конечным множеством ВР, устойчивому
относительно сдвигов. Доказательство этого критерия
имеет существенно более простой вид,
чем доказательство критерия Бу\-ха\-ра\-ева--Хо\-му\-та.}


\KW{вероятностные автоматы; вероятностные реакции; случайные функции}

\DOI{10.14357/19922264150309}



\vskip 14pt plus 9pt minus 6pt

\thispagestyle{headings}

\begin{multicols}{2}

\label{st\stat}


\section{Введение}

\vspace*{2pt}

\subsection{Понятие вероятностного автомата}

\vspace*{2pt}

Понятие \textbf{вероятностного автомата} впервые было
сформулировано в~1963~г.\ в~основополагающей работе
М.~Рабина~\cite{1-mir}. Данное понятие возникло как синтез понятий конечного
детерминированного автомата~\cite{2-mir} и~цепи Маркова~\cite{3-mir}
и~было предназначено для построения математических моделей
динамических сис\-тем, в~которых присутствует неопределенность,
описываемая статистическими закономерностями. Эта неопределенность
связана:
\begin{itemize}
\item с~неточностью знаний о состояниях, в~которых
моделируемые сис\-те\-мы находятся в~процессе своего функционирования;
\item  с~недетерминированностью правил изменения этих состояний.
\end{itemize}
Неопределенность в~ВА может быть вызвана различными причинами,
которые подразделяются на два класса.
\begin{enumerate}
\item Причины из первого
класса связаны с~природой сис\-те\-мы, моделируемой 
ВА. К~ним относятся:
\begin{itemize}
\item влияние случайных факторов на
функционирование сис\-те\-мы, например: случайные сбои компонентов
сис\-те\-мы или отказы в~их работе,  случайное изменение  условий
функционирования анализируемой сис\-те\-мы, случайность потока заявок в~сис\-те\-ме массового обслуживания и~т.~п.;
\item несовершенство (или
невозможность)  точного измерения состояний этой сис\-темы.
\end{itemize}
\item
Второй класс причин связан с~преднамеренным внесением неточности и~неопределен\-ности в~математические модели анализируемых сис\-тем. Это
делается в~тех случаях, когда точные модели анализируемых сис\-тем
имеют неприемлемо высокую сложность и~проведение анализа поведения
таких сис\-тем возможно только с~использованием их упрощенных моделей,
в которых некоторые компоненты состояний этих сис\-тем  игнорируются.
В~частности, анализ поведения сложной программной сис\-те\-мы (например,
операционной сис\-те\-мы компьюте-\linebreak ра) в~большинстве случаев возможен
только с~использова\-нием таких упрощенных математических моделей этих
сис\-тем, в~которых принимаются во внимание значения лишь некоторых
программных переменных, от которых существен\-но зависит поведение
анализируемой программной сис\-темы.
\end{enumerate}

Как правило, моделирование  сис\-тем при помощи ВА производится:
\begin{itemize}
\item
либо с~целью анализа свойств этих сис\-тем (к~чис\-лу которых относятся,
например, корректность, безопасность, надежность, устойчивость
функционирования в~непредусмотренных ситуациях и~др.),
\item либо с~целью вы\-чис\-ле\-ния различных количественных характеристик
анализируемых сис\-тем, среди которых могут быть, например, следующие:
\begin{itemize}
\item частота выполнения тех или иных действий или переходов в~анализируемых сис\-те\-мах;
\item вероятность отказа компонентов
анализируемых сис\-тем;
\item  вероятность вторжения злоумышленника в~компьютерную сеть;
\item  математическое ожидание времени отклика
веб-сер\-виса.
\end{itemize}
\end{itemize}

\subsection{Исторический обзор}

Первоначальное понятие ВА, введенное в~работе М.~Рабина~\cite{1-mir},
было предназначено главным образом для изучения
вопросов представимости регулярных языков вероятностными автоматами.
Затем оно было обобщено до такого понятия, которое позволило
моделировать вероятностные преобразователи информации. Определение
ВА в~общей форме было введено независимо в~работах Дж.~Карлайла~\cite{4-mir},
Р.\,Г.~Бухараева~\cite{5-mir} и~П.~Штарке~\cite{6-mir}.

С  начала возникновения  понятия  ВА исследовательская деятельность
в этой области отличалась высокой активностью. Результаты первых лет
исследований в~области ВА были сис\-те\-ма\-ти\-зи\-ро\-ва\-ны в~книге~\cite{7-mir}.
Подробный список  (около 500) ссылок на работы с~наиболее
существенными теоретическими и~практическими результатами по ВА,
полученными до~1985~г., можно найти в~фундаментальной монографии
Р.\,Г.~Бухараева~\cite{8-mir}, которую можно рассматривать как
итог первого периода развития теории ВА, продолжавшегося более двух
десятилетий.

В последующие годы произошло некоторое снижение
активности исследований в~этой области,
но в~настоящее время теория ВА
вновь находится в~состоянии подъема.
Возрождение исследовательской
активности в~области ВА в~значительной степени
связано с~тем, что в~связи с~бурным развитием современных информационных
технологий возник широкий круг новых задач, в~решении которых ВА могут служить
  эффективным инструментом. К~чис\-лу таких задач
  относятся задачи в~следующих областях:
\begin{itemize}
\item верификация программ и~протоколов передачи данных в~компьютерных сетях;
\item информационный поиск в~Интернете;
\item финансово-экономический анализ;
\item  обработка и~извлечение знаний из больших массивов данных
(data mining и~process mining);
 в~частности в~задачах анализа биз\-нес-\-про\-цес\-сов,
 биоинженерии и~биоинформатики;
\item  извлечение смысла из текстов на естественных языках;
\item  машинное зрение и~обработка
изображений   и~др.
\end{itemize}

Началом современного этапа развития
теории ВА можно считать работу~\cite{9-mir},
в которой рас\-смот\-ре\-ны ВА, возникающие при моделировании
параллельных вы\-чис\-ли\-тель\-ных сис\-тем с~асинхронным взаимодействием.
В~качестве вводных текстов в~современную теорию ВА можно назвать
работы~\cite{10-mir, 11-mir}.

Главное отличие нового понятия ВА от того, которое изучалось
в~предшествующий период, заключается в~том, что в~новом понимании ВА определяется
как \textbf{сис\-те\-ма переходов} ({transition system}),
с~которой связано некоторое множество переменных.
Вероятностный автомат
функционирует путем выполнения переходов, после каждого из которых
происходит обновление значений переменных этого ВА.
Можно доказать, что если множество переменных
ВА конечно и~множества значений этих переменных
тоже конечны, то новое и~старое понятия ВА будут эквивалентны.

Наряду с~упомянутыми выше понятиями ВА существуют и~другие модели
динамических сис\-тем со случайным поведением, например
скрытые марковские модели (hidden Markov models)~\cite{12-mir},
байесовские сети (Bayesian networks)~\cite{13-mir},
вероятностные графические модели~\cite{14-mir}, марковские решающие процессы
(Mar\-kov decision processes)~\cite{15-mir},
вероятностные I/O автоматы (pro\-ba\-bi\-li\-s\-tic    I/O automata)~\cite{16-mir}.
Все эти модели являются частными случаями
исходного понятия ВА общего вида~\cite{8-mir}.
{\looseness=1

}

Наряду с~перечисленными выше моделями в~последние годы изучаются модели
динамических сис\-тем со случайным поведением,
переходы в~которых могут быть ассоциированы
не только с~вероятностями их выполнения,
но и~с~модальностями must и~may, которые позволяют
существенно усилить выразительные возможности
этих моделей по сравнению с~другими упомянутыми выше моделями.
Основные концепции и~методы, относящиеся к~таким моделям,
содержатся в~статье~\cite{17-mir}.

Также изучаются и~другие обобщения понятия ВА, в~частности
вероятностные сети Петри~\cite{18-mir, 19-mir}, ВА  с~непрерывным временем~\cite{20-mir},
вероятностные процессные алгебры~\cite{21-mir}.


\subsection{Содержание статьи}
%\label{sdafsafsadf}

Главный результат настоящей работы (теорема~4)
представляет собой усиление критерия реализуемости многотактного
канала в~конечном ВА, полученного независимо 
Х.~Хо\-му\-том~\cite{23-mir} и~Р.\,Г.~Бухараевым~\cite{22-mir}.

Понятие \textbf{многотактного канала} (МК) было введено независимо
Р.\,Г.~Бухараевым~\cite{5-mir} и~П.~Штарке~\cite{6-mir}
соответственно. Это понятие служит характеристикой внешнего
поведения ВА  и~представляет собой вероятностную зависимость
между входными сигналами, поступающими в~ВА, и~выходными сигналами,
выдаваемыми этим ВА. Бухараев~\cite{5-mir} 
и~П.~Штарке~\cite{6-mir} ввели  условие на МК, которое они называют условием
автоматности. Автор настоящей работы счел целесообразным ввести
новое понятие~--- \textbf{вероятностной реакции}. Можно доказать
эквивалентность понятий ВР и~МК, удовлетворяющего
условию автоматности. Основной результат настоящей работы
формулируется в~терминах ВР.

Суть критерия Бу\-ха\-ра\-ева--Хо\-му\-та реали\-зу\-емости МК в~конечном 
ВА~\cite[теорема~2.2.1, с.~50]{8-mir} заключается
в~следующем. Вводятся понятия множества~$S_f$ состояний МК~$f$
 и~опорного множества для~$S_f$.
Критерий имеет следующий вид: МК~$f$,
удовлетворяющий условию автоматности, реализуется в~конечном ВА
тогда и~только тогда, когда существует конечное опорное множество
для~$S_f$, выпуклое относительно полугруппы всех вращений.
Проверка данного критерия представляется затруднительной
по той причине, что для этого необходимо вы\-чис\-лить все  множество~$S_f$,
что сложно сделать в~том случае, когда данное множество  бесконечно.

Доказательство этого критерия основано на представлении
МК как точек счетномерного линейного пространства с~индексацией координат
парами элементов свободных полугрупп
над входным и~выходным алфавитами. В~доказательстве используются
бесконечномерные матрицы, индексированные парами, компонентами которых являются
 пары элементов этих свободных полугрупп.

 Сформулированный и~доказанный в~настоящей работе
 критерий реализуемости ВР в~конечном ВА не связан
 с~построением бесконечного множества состояний для
ВР и~имеет следующий  вид: ВР реализуема конечным ВА тогда и~только тогда,
когда она принадлежит конусу над некоторым конечным множеством ВР, устойчивому
относительно сдвигов. Доказательство этого критерия
имеет существенно более простой вид, чем
доказательство критерия Бухараева--Хо\-му\-та,
 и~не использует бесконечномерных матриц.

\subsection{Мотивация и~актуальность работы}

Проблема реализуемости ВР в~конечных ВА является вероятностным
обобщением известной проблемы реализуемости отображений реакции
в~конечных детерминированных автоматах, т.\,е.\ проб\-ле\-мы построения
автоматов по информации о~выходных сигналах, которые они выдают
в~ответ на входные воздействия. Данная проблема была самой первой
проблемой, относящейся к~теории автоматов (первой работой,
посвященной данной проблеме, является~\cite{25-mir}). На всем
протяжении развития теории автоматов специализация данной проблемы
на различные классы автоматов играла ключевую роль в~развитии теории
автоматов.

В настоящее время важность проблемы реализуемости ВР
в~классе ВА обосновывается значительным чис\-лом
актуальных прикладных задач, приводящих к~данной проблеме.
Например, большой класс задач распознавания образов
сводится к~задаче построения распознающей сис\-те\-мы
(в~качестве которой может выступать, например, ВА),
реакция которой на заданный класс входных воздействий
должна удовлетворять некоторым условиям (например, сис\-те\-ма
должна правильно реагировать на некоторые
входные воздействия с~заданной вероятностью ошибки).
Математической формализацией данной задачи
является проблема построения ВА, реакция которого обладает
заданными свойствами. Иногда не существует ни одного ВА, реакция которого обладает
заданными свойствами. Поэтому еще до синтеза ВА, соответствующего некоторой ВР,
необходимо обосновать существование такого ВА.
Доказываемая в~настоящей работе теорема~4
связана именно с~этой задачей.  Теорема~4 сводит проблему  обоснования
существования ВА, реакция которого совпадает с~заданной ВР~$f$,
к~проблеме нахождения конечного множества~$\Gamma_f$ ВР, обла\-да\-юще\-го
 свойствами, упомянутыми в~конце п.~1.3.
Из этого результата не вытекает существования
алгоритма построения множества~$\Gamma_f$
в~том случае, когда это множество существует.
Не исключено, что задача нахождения множества~$\Gamma_f$
является алгоритмически неразрешимой.
Однако достоинством теоремы~4 является то, что она сужает направление
поисков в~решении задачи распознавания реализуемости заданной ВР~$f$.

Отметим, что с~момента появления в~начале 1970-х~гг.\
критерия Бу\-ха\-ра\-ева--Хо\-му\-та реали\-зу\-емости ВР
не появилось никаких  результатов,
связанных с~усилением данного критерия.
В~недавней работе Бухараева~\cite{24-mir} этот критерий
упомянут  в~той же самой формулировке, в~которой
он был приведен в~\cite{22-mir, 23-mir}.


\section{Вспомогательные понятия}

\subsection{Случайные функции и~распределения} %\label{defsdffg}

Пусть задана пара множеств $X,Y$. \textbf{Случайной функцией} (СФ)
из~$X$ в~$Y$ называется произвольная функция~$f$ вида
\begin{equation*}
%\label{dfgfdsgfds}
f: X\times Y \to [0,1]\,,
\end{equation*} 
такая что $\forall\,x\hm\in X$ множество
$\{y\hm\in Y\mid f(x,y)\hm>0\}$  конечно или счетно и~$\forall\,x\hm\in
X\quad \sum\limits_{y\in Y}f(x,y)\hm=1$.

Для любых $x\hm\in X$ и~$y\hm\in Y$
значение $f(x,y)$ можно интерпретировать
как вероятность того, что СФ~$f$ отображает~$x$ в~$y$.

Если $f$~--- СФ из~$X$ в~$Y$, то  будем обозначать этот факт
записью $f: X\vo Y$.
Будем называть~$X$ \textbf{об\-ластью определения} СФ~$f$,
а~$Y$~--- \textbf{\bf обл\-астью значений} СФ~$f$.

Случайная функция называется \textbf{конечной} (КСФ),
если ее область определения и~область значений являются конечными множествами.


Пусть задана КСФ $f: X\vo Y$  и~на~$X$ и~$Y$ заданы  упорядочения их элементов,
которые имеют вид $(x_1,\ldots, x_m)$ и~$(y_1,\ldots, y_n)$
соответственно.
Тогда~$f$ можно представить в~виде матрицы
(обозначаемой тем же символом~$f$):
\begin{equation}
\label{dfgdsghdsfgdsg}
f=
\begin{pmatrix}
f(x_1,y_1)&\cdots&f(x_1,y_n)\\
\vdots&\vdots&\vdots\\
f(x_m,y_1)&\cdots&f(x_m,y_n)
\end{pmatrix}\,.
\end{equation}

Ниже  будем отождествлять каждую КСФ~$f$\linebreak
с~соответствующей ей матрицей~(\ref{dfgdsghdsfgdsg}).
Будем пред\-полагать, что для каждого множества~$X$,
 явля\-юще\-го\-ся областью определения или областью значений
ка\-кой-ли\-бо из рассматриваемых КСФ, на~$X$
задано фиксированное упорядочение его элементов. Таким образом,
для каждой рассматриваемой КСФ соответствующая  ей матрица
определена од\-но\-значно.


\textbf{Вероятностным распределением} (или просто \textbf{распределением})
на множестве~$X$ называется  СФ вида $\xi:\;\Ik \vo X$,
где $\Ik$~--- множество, состоящее из одного элемента, который
 будем обозначать символом~$e$.
Совокупность всех распределений на~$X$  будем
обозначать записью $X^\bigtriangleup$.
Для каждого $x\hm\in X$ и~каждого $\xi\hm\in X^\bigtriangleup$
значение $\xi(e,x)$   будем обозначать более коротко записью~$x^\xi$.
Для каждого $x\hm\in X$  будем обозначать записью~$\xi_x$ распределение
из~$X^\bigtriangleup$,
определяемое следующим образом:
$\forall\, y\in X\;\;y^{\xi_x}\eam 1$, если $y\hm=x$,
и~$y^{\xi_x}\eam 0$, если $y\not =x$.

\subsection{Строки}

Для каждого множества~$X$  будем обозначать записью~$X^*$
совокупность всех конечных строк, компонентами
которых являются элементы~$X$. Множество~$X^*$ содержит \textbf{пустую строку}, она обозначается
символом~$\varepsilon$.
Для каждого $x\hm\in X$ строка, состоящая из одного
этого элемента, обозначается той же записью~$x$.
Для каждой строки $u\hm\in X^*$
ее \textbf{длиной} называется чис\-ло компонентов этой строки. Длина пустой строки равна нулю.
Длина строки~$u$ обозначается записью~$|u|$.
Для каждой пары строк $u,v\hm\in X^*$
их \textbf{конкатенацией} называется строка, обозначаемая записью~$uv$
и~определяемая следующим образом:
$u \varepsilon\eam \varepsilon u\eam u$,
и~если $u\hm=x_1\cdots x_n$ и~$v\hm=x'_1\cdots x'_m$,
то $u v\eam x_1\cdots x_nx'_1\ldots x'_m$.


\section{Вероятностные автоматы}

\subsection{Понятие вероятностного автомата}

\textbf{Вероятностный автомат}~--- это пятерка~$A$~вида
\begin{equation}
\label{dfsgdsfgdsffd44555}
A=\left(X,Y,S,P, \xi^0\right)\,,
\end{equation}
компоненты которой имеют следующий смысл:
\begin{enumerate}[(1)]
\item $X$, $Y$ и~$S$~--- конечные множества, элементы
которых называются соответственно
\textbf{входными сигналами},
\textbf{выходными сигналами} и~\textbf{состояниями} ВА~$A$;
\item $P$~--- СФ вида $P:S\times X\vo S\times Y$, называемая \textbf{поведением}
ВА~$A$. $\forall\,(s,x,s',y) \hm\in S\times X\times S\times Y$
значение $P(s,x,s',y)$ понимается как
   вероятность того, что
   если  в~текущий момент времени~$(t)$ $A$ находится
   в~состоянии~$s$ и~в~этот момент времени на его вход
   поступил сигнал~$x$,
то  в~следующий момент времени $(t+1)$~$A$~будет находиться
   в~состоянии~$s'$ и~в~момент време-\linebreak ни~$t$
   выходной сигнал~$A$ равен~$y$;
\item $\xi^0$~--- распределение на~$S$,
называемое  \textbf{начальным распределением} ВА~$A$.
$\forall\,s\hm\in S$ значение $s^{\xi^0}$ понимается как
вероятность того, что в~начальный момент времени $(t\hm=0)$
ВА~$A$ находится в~состоянии~$s$.
\end{enumerate}

Пусть $A$~--- ВА вида~(\ref{dfsgdsfgdsffd44555})
и упорядочение множества~$S$ его состояний имеет вид
$(s_1,\ldots, s_n)$. Для любых $x\hm\in X$ и~$y\hm\in Y$
 будем обозначать записью $A^{xy}$ матрицу порядка~$n$
\begin{equation}
\label{dsfdsafdsf5566}
\begin{pmatrix}
P(s_1,x,s_1,y)&\cdots&P(s_1,x,s_n,y)\\
\vdots&\vdots&\vdots\\
P(s_n,x,s_1,y)&\cdots&P(s_n,x,s_n,y)
\end{pmatrix}
\end{equation}
и~для любой пары строк $u\hm\in X^*$, $v\hm\in Y^*$
 будем обозначать записью $A^{u,v}$
(запятая в~этой записи может опускаться) матрицу
порядка~$n$, определяемую следующим образом:
\begin{itemize}
\item $A^{\varepsilon,\varepsilon} \hm= {\sf E}$ (единичная матрица),
 \item если $|u|\neq |v|$,  то $A^{u,v} \hm= 0$ (нулевая матрица),
\item если $u\hm=x_1\cdots x_k$ и~$v\hm=y_1\cdots y_k$,
то $A^{u,v}\hm = A^{x_1y_1}\cdots A^{x_ky_k}$.
\end{itemize}

Для любых $s$, $s'\hm\in S$  будем обозначать
за\-писью $A^{u,v}_{s,s'}$
коэффициент матрицы~$A^{u,v}$, находящийся в~строке,
соответствующей состоянию~$s$, и~в~столбце, соответствующем состоянию~$s'$.

Если строки $u\hm\in X^*$ и~$v\hm\in Y^*$ имеют вид
$x_0\cdots x_k$ и~$y_0\cdots y_k$ соответственно,
то~$A^{u,v}_{s,s'}$ можно понимать как
   вероятность того, что
 если   в~текущий    момент~($t$)~$A$~находился в~состоянии~$s$
   и~начиная с~этого момента  на вход~$A$ последовательно
   поступали элементы строки~$u$
   (т.\,е.\ в~момент~$t$ поступил сигнал~$x_0$,
   в~момент $t\hm+1$ поступил сигнал~$x_1$ и~т.\,д.),
 то в~моменты $t,t+1,\ldots, t+k$
   выходные сигналы~$A$ равны $y_0,\ldots, y_k$ соответственно и~в~момент
   $t\hm+k\hm+1$ $A$~будет находиться в~состоянии~$s'$.

Для каждого ВА $A=(X,Y,S,P,\xi^0)$ \textbf{графовым представлением} ВА~$A$
называется граф~$\Gamma_A$ с~множеством
вершин~$S$,
 каждая вершина~$s$ которого помечена чис\-лом~$s^{\xi^0}$
и~ребра которого помечены тройками вида $(x,y,a)$, где $x\hm\in X$,
$y\hm \in Y$, $a\hm\in [0,1]$:
если $A^{xy}_{s,s'}\hm=a\hm>0$, то~$\Gamma_A$ содержит
ребро из~$s$ в~$s'$  с~меткой $(x,y,a)$.

Данное графовое представление ВА полностью аналогично графовому
представлению обычных автоматов.
В~частности, для ВА тоже можно ввести понятие
\textbf{вы\-чис\-ле\-ния (computation)}, которое пред\-став\-ля\-ет
собой путь в~графе~$\Gamma_A$.
Если путь в~графе~$\Gamma_A$ имеет вид:
\begin{multline*}
s_0\xrightarrow{(x_1,y_1,a_1)}s_1\xrightarrow{(x_2,y_2,a_2)}\cdots
\xrightarrow{(x_n,y_n,a_n)}s_n,
\end{multline*}
то произведение $a_1\cdots a_n$ можно интерпретировать
как вероятность того, что при подаче
на вход ВА~$A$ слова $x_1\cdots x_n$
соответствующее выходное слово будет иметь вид $y_1\cdots y_n$.


Для каждого $y\hm\in Y$
\textbf{принимающим вы\-чис\-ле\-ни\-ем (accepting computation)} относительно сигнала~$y$
является произвольное вы\-чис\-ле\-ние в~графе~$\Gamma_A$,
име\-ющее вид:
\begin{multline*}
s_0\xrightarrow{(x_1,y_1,a_1)}s_1\xrightarrow{(x_2,y_2,a_2)}\cdots\\
\cdots
\xrightarrow{(x_n,y_n,a_n)} s_n \xrightarrow{(x,y,a)} s_{n+1}\,.
\end{multline*}

\subsection{Реакция вероятностного автомата}
%\label{asdfasdfsafa}

Пусть задан ВА $A$ вида~(\ref{dfsgdsfgdsffd44555})
и распределение $\xi\hm\in S^\bullettri$.
Будем говорить, что  \textbf{ВА $A$  в~момент времени~$t$
 имеет распределение~$\xi$},
 если  для каждого состояния $s\hm\in S$ вероятность того,
 что~$A$~в момент времени~$t$ находится в~состоянии~$s$,
 равна~$s^\xi$.
\textbf{Реакцией ВА}~$A$ в~распределении~$\xi$ называется функция
$$
A^\xi: X^*\times Y^*\to\mathbb{R}\,,
$$
определяемая следующим образом:
$$
\forall\,u\in X^*, \,\forall\,v \in Y^*\quad
A^\xi(u,v)\eam \xi A^{u,v} I\,,
$$
где запись~$I$ обозначает
век\-тор-стол\-бец порядка~$|S|$, все компоненты которого равны~1.

\textbf{Реакцией ВА $A$}  будем называть реакцию этого
ВА в~его начальном распределении. Будем обозначать
реакцию ВА~$A$ записью~$f_A$.
Если строки $u\hm\in X^*$ и~$v\hm\in Y^*$
имеют вид
$x_0\cdots x_k$ и~$y_0\cdots y_k$
соответственно, то  $f_A(u,v)$
можно понимать как    вероятность того, что если начиная с~момента~0
      на вход~$A$ последовательно    поступали элементы строки~$u$
   (т.\,е.\ в~момент~$0$ поступил сигнал~$x_0$,
   в~момент~$1$ поступил сигнал~$x_1$
   и~т.\,д.), то в~моменты $0,1,\ldots, k$
   выходные сигналы~$A$ равны $y_0,\ldots, y_k$ соответственно.

\smallskip

\noindent
\textbf{Теорема~1.}\
\textit{Если $A$~---  ВА вида~(\ref{dfsgdsfgdsffd44555})
и~$\xi\hm\in S^\bullettri$, то $A^\xi$~--- СФ.}

\smallskip

\noindent
Д\,о\,к\,а\,з\,а\,т\,е\,л\,ь\,с\,т\,в\,о\,.\ \
Поскольку $\forall\,u\hm\in X^*, \forall\,v\hm\in X^*\quad A^\xi(u,v)\geq
0$, то для доказательства теоремы достаточно доказать, что
$$
\forall\,u\in X^*\;\;\sum\limits_{v\in Y^*} A^\xi(u,v)=1\,,
$$
 т.\,е.\
\begin{equation}
\label{dsfdsdsafs33}
\forall\,u\in X^*\quad\sum\limits_{v\in Y^*}\xi
A^{u,v} I=1\,.
\end{equation}

Поскольку $A^{u,v}=0$ при $|u|\hm\neq|v|$, то~(\ref{dsfdsdsafs33})
эквивалентно условию:
\begin{equation}
\label{dsfdsdsafs331}
 \forall\,k\geq 0,\ \forall\,u\in X^k\quad
\sum\limits_{v\in Y^k}\xi A^{u,v} I =1\,.
\end{equation}

Докажем~(\ref{dsfdsdsafs331}) индукцией по~$k$. Если $k\hm=0$, то~(\ref{dsfdsdsafs331})
следует из того, что
$A^{\varepsilon,\varepsilon}\hm={\sf E}$ и~$\xi {\sf E} I\hm=\xi I=1$ (так как $\xi\hm\in
S^\bullettri$).

Пусть (\ref{dsfdsdsafs331}) верно для некоторого~$k$. Докажем, что
\begin{equation}
\label{dsfdsdsafs333331}
\forall\,u\in X^{k+1}\quad \sum\limits_{v\in Y^{k+1}}\xi A^{u,v} I =1\,.
\end{equation}
Соотношение~(\ref{dsfdsdsafs333331}) эквивалентно соотношению
\begin{equation}
\label{dsfdsdsafs333332}
\forall\,u\in X^{k}, \;\forall\,x\in X\quad \sum\limits_{v\in Y^{k},\;y\in Y}\xi
A^{ux,vy} I =1.
\end{equation}
Так как $A^{ux,vy} = A^{u,v}  A^{xy}$, то
(\ref{dsfdsdsafs333332}) можно переписать в~виде
\begin{equation}
\label{dsfdsdsafs333333}
\forall\,u\in X^{k}, \;\forall\,x\in X\quad \sum\limits_{v\in
Y^{k}}\xi A^{u,v} \left(\sum\limits_{y\in Y}A^{xy}
 I\right) =1\,.
 \end{equation}
Соотношение (\ref{dsfdsdsafs333333}) следует из~(\ref{dsfdsdsafs331}) и~из равенства
\begin{equation}
\label{dsfdsdsafs333334}
\sum\limits_{y\in Y}A^{xy} I=I\,,
\end{equation}
которое
верно потому, что если $A^{xy}$ имеет вид~(\ref{dsfdsafdsf5566}), то
$\forall\,i\hm\in\{1,\ldots,n\}$ элемент с~индексом~$i$ столбца
$\sum\limits_{y\in Y}A^{xy} I$ равен сумме
$$
\sum\limits_{y\in Y,\;j=1,\ldots,n}\hspace*{-1mm}P(s_i,x,s_j,y)\,,
$$
которая равна~1, так как $P$ есть СФ вида:
$$
~~~~~~P:S\times X\vo S\times Y\,.~~~~~~\square
$$


\section{Вероятностные реакции}

\subsection{Понятие вероятностной реакции}

Пусть $X$ и~$Y$~--- конечные множества.

\textbf{Вероятностной реакцией} из~$X$ в~$Y$ называется СФ
$f: X^*\vo Y^*$, удовлетворяющая
условию: $\forall\,u\hm\in X^*$, $\forall\,v\hm\in Y^*$,
\begin{equation}
\left.
\begin{array}{c}
\mbox{если }|u|\neq |v|,
\mbox{ то }f(u,v)=0\,,\\[6pt]
\forall\,x\in X\quad
f(u,v)=\sum\limits_{y\in Y}f(u x,v y)\,.
\end{array}\right\}
\label{safdsdfdsafsafarrr}
\end{equation}
Запись $R(X,Y)$ обозначает совокупность всех ВР из $X$ в~$Y$.\\

\smallskip

\noindent
\textbf{Теорема 2.}\
\textit{Множество $R(X,Y)$ замкнуто относительно выпуклых комбинаций, т.\,е.\
если $f_1,\ldots, f_n\hm\in R(X,Y)$ и~$a_1,\ldots, a_n\hm\in [0,1]$, где
$\sum\limits_{i=1}^na_i\hm=1$, то}
$$
\sum\limits_{i=1}^n a_i f_i\in R(X,Y)\,.
$$

\noindent
Д\,о\,к\,а\,з\,а\,т\,е\,л\,ь\,с\,т\,в\,о\,.\ \
$\sum\limits_{i=1}^na_if_i$~--- СФ, поэтому
\begin{itemize}
\item
$\forall\,u\in X^*,\forall\,v\in Y^*$, если $|u|\neq |v|$, то\\
$\forall\,i=1,\ldots, n\;\;
f_i(u,v)=0 \;\Rightarrow\;\sum\limits_{i=1}^na_if_i(u,v)=0$;
\item $\forall\,x\in X$
$\sum\limits_{y\in Y}\sum\limits_{i=1}^na_if_i(u x,v y)\hm=
\sum\limits_{i=1}^n a_i \times$\linebreak $\times\sum\limits_{y\in Y}f_i(u x,v y)\hm=
\sum\limits_{i=1}^n a_i f_i(u,v)\hm=
f(u,v).$\hfill$\square$
\end{itemize}

\noindent
\textbf{Теорема~3.} % \arabic{theorem}\label{th0121}}.
\textit{Для каждого ВА
$A=(X,Y,S,P, \xi^0)$
и~каж\-до\-го} $\xi\hm\in S^\bullettri$
$$
A^{\xi} \in R(X,Y)\,.
$$

\noindent
Д\,о\,к\,а\,з\,а\,т\,е\,л\,ь\,с\,т\,в\,о\,.\ \
Докажем, что  $\forall\,u\hm\in X^*,\forall\,v\hm\in Y^*$
СФ $f\eam A^{\xi}$
удовлетворяет  условию~(\ref{safdsdfdsafsafarrr}):
\begin{itemize}
\item если $|u|\hm\neq |v|$, то $A^{u,v}\hm=0$;
поэтому $A^{\xi}(u,v) \hm= \xi A^{u,v} I\hm=0$,
\item $\forall\,x\in X$
\begin{multline}
\label{dsvsdfsdr444}
\sum\limits_{y\in Y}A^{\xi}(u x,v y)=
\sum\limits_{y\in Y}\xi A^{ux,vy} I=\\=
\sum\limits_{y\in Y}\xi A^{u,v} A^{xy} I=
\xi A^{u,v} (\sum\limits_{y\in Y}A^{xy} I)=\\=
\xi A^{u,v} I = A^\xi(u,v)
\end{multline}
(в (\ref{dsvsdfsdr444}) используется
равенство~(\ref{dsfdsdsafs333334})).\hfill$\square$
\end{itemize}

\subsection{Реализуемость вероятностных реакций}

Вероятностная реакция~$f$ называется \textbf{реализу\-емой}, если
$\exists$~ВА $A: \;f_A\hm=f$.


Пусть $X$ и~$Y$~--- конечные множества.
Будем использовать следующие определения и~обозначения:

\begin{itemize}
\item запись $[0,1]^{X^*\times Y^*}$ обозначает множество
 функций вида
 $$
 f:X^*\times Y^*\hm\to [0,1]\,;
 $$
\item для каждого $\Gamma\hm\subseteq [0,1]^{X^*\times Y^*}$
\textbf{конусом} над~$\Gamma$ называется подмножество
$C_0(\Gamma)\hm\subseteq [0,1]^{X^*\times Y^*}$,
состоящее из функций вида $\sum\limits_{i=1}^n
a_if_i$, где
\begin{itemize}
\item $\forall\,i=1,\ldots, n$ $a_i\hm\in [0,1]$,
$f_i\hm\in\Gamma$,
$\sum\limits_{i=1}^na_i\leq 1$;
\item $\forall\,(u,v)\in
X^*\times Y^*$ $\left(\sum\limits_{i=1}^n
a_if_i\right)(u,v)\eam$\linebreak $\eam \sum\limits_{i=1}^n
a_if_i(u,v)$;
\end{itemize}
\item $\forall\,x\in X$, $\forall\,y \hm\in Y$ запись $D^{xy}$
обозначает отображение вида
$$
D^{xy}: [0,1]^{X^*\times Y^*}\to [0,1]^{X^*\times Y^*}\,,
$$
называемое \textbf{сдвигом} и~сопоставляющее каждой функции~$f$ из $[0,1]^{X^*\times Y^*}$
функцию, обозначаемую записью $fD^{xy}$, где
\begin{multline}
\forall\,u\in X^*, \forall\,v \in Y^*
\\ \left(fD^{xy}\right)(u,v)\eam f(xu,yv)\,;\!\!
\label{sadfsadfdsfrrrr5556}
\end{multline}

\item подмножество
$\Gamma\hm\subseteq [0,1]^{X^*\times Y^*}$
называется \textbf{устойчивым относительно сдвигов},
если
$$
\forall\,f\in \Gamma,\;
\forall\,x\in X, \forall\,y\in Y\enskip fD^{xy}\in C_0(\Gamma)\,.
$$
\end{itemize}


\noindent
\textbf{Теорема~4.}\
\textit{Пусть~$X$ и~$Y$~--- конечные множества
и~$f\hm\in R(X,Y)$. Следующие условия эквивалентны}:
\begin{itemize}
\item $f$ \textit{реализуема};
\item \textit{существует конечное  множество $\Gamma_f\hm\subseteq
R(X,Y)$, устойчивое относительно сдвигов и~такое, что} $f\hm\in C_0(\Gamma_f)$.
\end{itemize}

\noindent
Д\,о\,к\,а\,з\,а\,т\,е\,л\,ь\,с\,т\,в\,о\,.\ \
Пусть~$f$ реализуема, т.\,е.\ $\exists$ ВА $A\hm=(X,Y,S,P, \xi^0)$:
$$
\forall\,u\in X^*, \forall\,v\in Y^*\enskip f(u,v)=\xi^0A^{u,v}I\,.
$$

$\forall\,s\in S$ обозначим записью~$A_s$
ВА $(X,Y,S,P, \xi_s)$.
В~качестве искомого~$\Gamma_f$ можно взять множество
$\{f_{A_s}\mid s\hm\in S\}$.

$f\in C_0(\Gamma_f)$, так как $f\hm=\sum\limits_{s\in S}s^{\xi^0}
f_{A_{s}}$ и~$\Gamma_f\hm\subseteq R(X,Y)$ (по теореме~3).


Докажем, что $\Gamma_f$ устойчиво относительно сдвигов, т.\,е.\
$\forall\,s\hm\in S$, $\forall\,x\hm\in X$, $\forall\,y\hm\in Y$
$f_{A_s}D^{xy}\hm\in C_0(\Gamma_f)$. Согласно~(\ref{sadfsadfdsfrrrr5556}),
\begin{multline*}
\forall\,u\in X^*, \forall\,v \in Y^*\ \ 
(f_{A_s}D^{xy})(u,v)={}\\
{}= f_{A_s}(xu,yv)=\xi_sA^{xu,yv}I=
\xi_sA^{xy}A^{u,v}I\,.
%\label{fdsafrefrefwerfewrf}
\end{multline*}
Нетрудно видеть, что
$$
\xi_sA^{xy}A^{u,v}I= \sum\limits_{s'\in S}a_{s'}
f_{A_{s'}}(u,v)\,,
$$
где $\forall\,s'\in S\;\; a_{s'}$~--- компонента
век\-тор-стро\-ки $\xi_sA^{xy}$, соответствующая состоянию~$s'$ (т.\,е.\
элемент матрицы $A^{xy}$, находящийся в~строке~$s$ и~столбце~$s'$).
Свойства $\forall\,s'\hm\in S$ $a_{s'}\hm\in [0,1]$ и~$\sum\limits_{s'\in S}a_{s'}\hm\leq 1$ являются следствием
соответствующих свойств матрицы~$A^{xy}$.

Обратно, пусть $f\hm\in C_0(\Gamma_f)$, где $\Gamma_f
\hm=\{f_1,\ldots, f_n\}\hm\subseteq R(X,Y)$
и~$\Gamma_f$ устойчиво относительно сдвигов.
Определим~$A$ как ВА
\begin{equation}
\label{sfddsfsafsadgs}
A\eam (X,Y,S,P, \xi^0)\,,
\end{equation}
компоненты которого имеют следующий вид:
\begin{itemize}
\item $S\eam \{1,\ldots, n\}$;
\item $\xi^0=(a_1,\ldots, a_n)$,
где $a_1,\ldots, a_n$~--- коэффициенты представления~$f$ в~виде суммы
\begin{equation}
\label{sadfsadfasdfdsa34344367}
f = \sum\limits_{i=1}^na_if_i\,,
\end{equation}
где 
$\forall\,i=1,\ldots, n$ $a_i\hm\geq 0$,
$\sum\limits_{i=1}^na_i\hm\leq 1$.

\columnbreak

По предположению, $f\hm\in R(X,Y)$, в~частности
 $f(\varepsilon, \varepsilon)\hm=1$,
откуда следует равенство $\sum\limits_{i=1}^na_i\hm=1$,
поэтому $\xi^0\hm\in S^\bullettri$;
\item поведение $P:S\times X\times S\times Y\to[0,1]$
ВА~(\ref{sfddsfsafsadgs})
определяется матрицами~$A^{xy}$ порядка~$n$ ($x\hm\in X$, $y\hm\in Y$):

\noindent
$$
P(i,x,j,y)\eam A^{xy}_{ij}\,,
$$
где $\forall\,x\in X$, $\forall\,y\hm\in Y$, $\forall\,i\hm=1,\ldots, n$
строка~$i$ матрицы~$A^{xy}$  состоит из коэффициентов
$a_{i1},\ldots, a_{in}$
представления функции $f_iD^{xy}$ в~виде суммы
$\sum\limits_{j=1}^na_{ij}f_j$
$(\forall\,i,j$ $a_{ij}\hm\geq 0$,
$\sum\limits_{j=1}^na_{ij} \hm\leq 1$).

Докажем, что $P$ является СФ вида $S\times X\vo S\hm\times Y$.
Данное утверждение эквивалентно соотношению
$\left(\sum\limits_{y\in Y}A^{xy}\right)I\hm=I.$

$\forall\,i=1,\ldots, n$ из

\noindent
\begin{equation}
\label{sdfsdgsfdg55566}
f_iD^{xy}=\sum\limits_{j=1}^nA^{xy}_{ij}f_j
\end{equation}
следует, что

\noindent
\begin{equation}
\label{sdfsdfsdfdsaf}
\left(f_iD^{xy}\right)(\varepsilon, \varepsilon)=
\sum\limits_{j=1}^nA^{xy}_{ij}f_j(\varepsilon, \varepsilon)\,.
\end{equation}
Так как $\forall\,i=1,\ldots, n$  $f_j\hm\in R(X,Y)$, то
$f_j(\varepsilon, \varepsilon)\hm=1$. Кроме того, левая часть~(\ref{sdfsdfsdfdsaf})
равна $f_i(x,y)$. Поэтому~(\ref{sdfsdfsdfdsaf})
можно переписать  в~виде
$f_i(x,y)\hm= \sum\limits_{j=1}^nA^{xy}_{ij}$, откуда следует
соотношение

\noindent
\begin{equation}
\label{fdgfsdgfsdgsd5544}
\sum\limits_{y\in Y}f_i(x,y)=
\sum\limits_{y\in Y}\sum\limits_{j=1}^nA^{xy}_{ij}.
\end{equation}
Так как $f_i\hm\in R(X,Y)$, то согласно второму соотношению
в~(\ref{safdsdfdsafsafarrr})
левая часть~(\ref{fdgfsdgfsdgsd5544}) равна $f_i(\varepsilon,
\varepsilon)$, т.\,е.\ равна~1. Учитывая это и~меняя порядок
суммирования в~правой части~(\ref{fdgfsdgfsdgsd5544}), получаем
соотношение

\noindent
\begin{equation}
\label{fdgf4sdgfsdgsd5544}
\sum\limits_{j=1}^n\sum\limits_{y\in Y}A^{xy}_{ij}=1\,.
\end{equation}
Нетрудно видеть, что истинность~(\ref{fdgf4sdgfsdgsd5544})
$\forall\,i\hm=1,\ldots, n$ эквивалентна доказываемому равенству
$\left(\sum\limits_{y\in Y}A^{xy}\right)I\hm=I.$
\end{itemize}

Докажем, что реакция ВА~(\ref{sfddsfsafsadgs}) совпадает с~$f$, т.\,е.\
\begin{equation}
\label{sfdgfsdfdsf555}
\forall\,u\in X^*, \forall\,v\in Y^*\enskip
\xi^0A^{u,v}I = f(u,v)\,.
\end{equation}
Если $|u|\neq |v|$, то левая часть
равенства в~(\ref{sfdgfsdfdsf555}) равна~0 по определению матриц вида~$A^{u,v}$
и~правая часть равенства в~(\ref{sfdgfsdfdsf555}) равна~0
согласно предположению $f\hm\in R(X,Y)$ и~первому соотношению
в~(\ref{safdsdfdsafsafarrr}).

Пусть $|u|=|v|$. Докажем (индукцией по~$|u|$), что
\begin{equation}
\label{sadfdsafsadgfsadfsa}
A^{u,v}I=\begin{pmatrix}
f_1(u,v)\\
\vdots\\
f_n(u,v) \end{pmatrix}\,.
\end{equation}

Если $u=v=\varepsilon$, то обе части~(\ref{sadfdsafsadgfsadfsa})
равны~$I$.

Если $u=xu'$ и~$v\hm=yv'$, то, предполагая верным равенство~(\ref{sadfdsafsadgfsadfsa}),
в~котором~$u$ и~$v$ заменены на~$u'$ и~$v'$,
имеем
\begin{multline}
\label{sdafsadfasd55566}
A^{u,v}I = A^{xu',yv'}I = A^{xy}A^{u',v'}I ={}\\
{}=A^{xy}
\begin{pmatrix}
 f_1(u',v')\\
 \vdots\\
 f_n(u',v')
\end{pmatrix}=
\begin{pmatrix}
 \sum\limits_{i=1}^n A^{xy}_{1j}f_j(u',v')\\
 \vdots\\
\sum\limits_{i=1}^n A^{xy}_{nj}f_j(u',v')
\end{pmatrix}\,.
\end{multline}
Из~(\ref{sdfsdgsfdg55566}) следует, что правую часть
в~(\ref{sdafsadfasd55566}) можно переписать в~виде
\begin{equation}
\label{sdfsadfg2255566}
\begin{pmatrix}
 \left(f_1D^{xy}\right)(u',v')\\
\vdots\\
(f_nD^{xy})(u',v')\,.
\end{pmatrix}
\end{equation}
Согласно определению~(\ref{sadfsadfdsfrrrr5556})
функций вида $fD^{xy}$, столбец~(\ref{sdfsadfg2255566})
совпадает с~правой частью
доказыва\-емо\-го равенства~(\ref{sadfdsafsadgfsadfsa}).

Таким образом, равенство~(\ref{sadfdsafsadgfsadfsa}) доказано.
Согласно этому равенству
левая часть доказываемого равенства~(\ref{sfdgfsdfdsf555}) равна
\begin{equation}
\label{sdfsdfdsfs336667}
\xi^0 \begin{pmatrix}
 f_1(u,v)\\
 \vdots\\
 f_n(u,v)\end{pmatrix}
= \sum\limits_{i=1}^n\xi^0_if_i(u,v)\,.
\end{equation}

По определению $\xi^0$ (см.~(\ref{sadfsadfasdfdsa34344367})) правая
часть~(\ref{sdfsdfdsfs336667}) равна $f(u,v)$, т.\,е.\  правой части
доказываемого равенства~(\ref{sfdgfsdfdsf555}).\hfill$\square$

\section{Примеры реализуемых и~нереализуемых вероятностных реакций}

\subsection{Пример реализуемой вероятностной реакции}

В этом пункте  приводится пример ВР, реализу\-емость которой
обосновывается  теоремой~4.

Пусть $X=\{x\}$, $Y\hm=\{y_1,y_2\}$ и~реакция $f: X^*\times Y^*\to[0,1]$
имеет следующий вид: 
$$
f(\varepsilon,\varepsilon)\eam 1
$$
и~$\forall\,u\hm\in X^*$, $\forall\,v\hm\in Y^*$,
\begin{itemize}
\item если $|u|\neq |v|$, то $f(u,v)\hm=0$;
\item если $|u| = |v|>0$, то~$v$~можно представить в~виде
одной из следующих конкатенаций:
\begin{equation}
\hspace*{-3mm}\left.
\begin{array}{l}
y_1^{p_1}\quad(p_1>0)\,;\\[6pt]
y_2^{q_1}\quad(q_1>0)\,;\\[6pt]
y_1^{p_1}y_2^{q_1}\quad(p_1>0,q_1>0)\,;\\[6pt]
y_1^{p_1}y_2^{q_1}\cdots
 y_k^{p_k}y_2^{q_k}
\ \left(\begin{array}{l}
k>1,p_1\geq 0,q_k\geq 0,\\[2pt]
\mbox{остальные }
p_i,q_i>0\end{array}\right)\!,
\end{array}\!
\right\}\!\!
\label{sdfsdgdsfgsdf}
\end{equation}
 где $\forall\,n\geq 0$
$y_i^n$ является строкой из~$n$ символов~$y$.


Обозначим $p\eam \sum\limits_{i=1}^k p_i$, $q\eam \sum\limits_{i=1}^k
q_i$ и~определим
\begin{equation}
\label{sfdsdfgsfdgfsd}
f(u,v)\eam c\left(\fr{1}{5}\right)^{p}\left(\fr{4}{5}\right)^q\,,
\end{equation}
где чис\-ло $c$ определяется в~соответствии с~видом~$v$ из
пе\-ре\-чис\-ле\-ния~(\ref{sdfsdgdsfgsdf}), т.\,е.~$c$~равно соответственно
\begin{equation}
\label{sdfsadfasdgsd}
\left.
\begin{array}{l}
1+\fr{4p_1}{3}\,;\\[6pt]
\fr{2}{3}\,;\\[6pt]
\fr{2}{3}+\fr{4p_1}{3}\,;\\[6pt]
\mbox{если}\ p_1=0, \mbox{то\ } \fr{2}{3},\
 \mbox{иначе}\ \fr{2}{3}+\fr{4p_1}{3}
\end{array}
\right\}
\end{equation}
(строки в~(\ref{sdfsadfasdgsd})
соответствуют строкам в~(\ref{sdfsdgdsfgsdf})).
\end{itemize}

Определим $\Gamma_f\eam \{f_1,f_2\}$, где
$f_i(\varepsilon, \varepsilon)\hm=1$ $(i\hm=1,2)$~и
\begin{itemize}
\item $f_1$ имеет такой же вид, как~$f$ в~(\ref{sfdsdfgsfdgfsd}),
но чис\-ло~$c$ определяется иначе
и~равно соответственно
$$
\begin{array}{l}
1+4p_1\,;\\[6pt]
0\,;\\[6pt]
4p_1\,;\\[6pt]
\mbox{если $p_1=0$, то $0$, иначе $4p_1$\,;}
\end{array}
$$
\item $f_2$ имеет такой же вид, как~$f$ в~(\ref{sfdsdfgsfdgfsd}),
где $c\eam 1$.
\end{itemize}

Докажем, что определенное выше множество~$\Gamma_f$ удовлетворяет условиям теоремы~4.
\begin{enumerate}[1.]
\item Докажем, что~$\Gamma_f$ устойчиво относительно сдвигов.
Нетрудно видеть, что если $|u|\hm=|v|$, то
\begin{itemize}
\item $\displaystyle\left(f_1D^{xy_1}\right)(u,v)=f_1(xu,y_1v)= 
c\left(\fr{1}{5}\right)^{p+1}\times$\linebreak $\times\left(\fr{4}{5}\right)^q$,


\noindent
где~$p$ и~$q$~--- количество вхождений~$y_1$ и~$y_2$
в~$v$,~и
$$
c\eam 
\begin{cases}
1+4(p+1), &\hspace*{-0.15894pt} \mbox{если } v=y_1^{p}\;(p\geq 0);\\
4(p+1), & \hspace*{-0.15894pt}\mbox{если } v=y_1^{p}y_2w\;\\
&\hspace*{10mm}(p> 0, w\in Y^*);\\
4, & \hspace*{-0.15894pt}\mbox{если } v=y_2w\;(w\in Y^*);
\end{cases}.
$$
\item $\left(f_1D^{xy_2}\right)(u,v)\hm=f_1(xu,y_2v)=0;$
\item  $\displaystyle
\left(f_2D^{xy_1}\right)(u,v)\hm=f_2(xu,y_1v)\hm= \left(\fr{1}{5}\right)^{p+1}\hm\times
\left(\fr{4}{5}\right)^{q}$,

\noindent
где~$p$ и~$q$~--- количество вхождений~$y_1$ и~$y_2$ в~$v$;
\item $\displaystyle
\left(f_2D^{xy_2}\right)(u,v)=f_2\left(xu,y_2v\right)=
\left(\fr{1}{5}\right)^{p}\hm\times\left(\fr{4}{5}\right)^{q+1}$.
\end{itemize}

Из вышеприведенных соотношений следуют равенства:
\begin{align*}
\left(f_1D^{xy_1}\right)(u,v)&=\fr{1}{5}\,f_1(u,v)+\fr{4}{5}\,f_2(u,v);\\
\left(f_1D^{xy_2}\right)(u,v)&=0;\\
\left(f_2D^{xy_1}\right)(u,v)&=\fr{1}{5}\,f_2(u,v);\\
\left(f_2D^{xy_2}\right)(u,v)&=\fr{4}{5}\,f_2(u,v),
\end{align*}
которые означают, что~$\Gamma_f$ устойчиво относительно сдвигов.

\item Докажем, что $f\hm\in C_0(\Gamma_f)$. Данное свойство
является непосредственным следствием легко проверяемого равенства
$$
f(u,v)=\fr{1}{3}\,f_1(u,v)+\fr{2}{3}\,f_2(u,v)\,.
$$
\end{enumerate}
Таким образом, на основании теоремы можно заключить, что
определенная выше функция~4 являет\-ся реализуемой.
В~доказательстве теоремы~4 содержится алгоритм построения
ВА, реакция которого совпадает с~$f$.\hfill$\square$

\medskip


Отметим, что для применения к~данной функции~$f$
критерия Бу\-ха\-ра\-ева--Хо\-му\-та необходимо построить
множество~$S_f$ состояний для МК~$f$, которое, как нетрудно доказать,
в данном случае будет бесконечным.


\subsection{Пример нереализуемой вероятностной реакции}

Пусть множество~$X$ имеет вид $\{a,b\}$,
а множество~$Y$ имеет вид $\{0,1\}$.
Определим функцию $f:X^*\times Y^*\hm\to [0,1]$
следующим образом: $f(\varepsilon,\varepsilon)\eam 1$
и~$\forall\,u\hm\in X^*$, $\forall \,v\hm\in Y^*$
\begin{itemize}
\item если $|u|\neq |v|$, то $f(u,v)\eam 0$;
\item если строки $u$ и~$v$ имеют вид
$x_1\cdots x_k$ и~$y_1\cdots y_k$ соответственно, где
$k\hm\geq 1$ и~$\forall\,i\hm=1,\ldots, k$ $x_i\hm\in X$ и~$y_i\hm\in Y$,
то $\forall\,i\hm=1,\ldots, k$
компонента~$y_i$ строки~$v$ имеет следующий вид:
\begin{itemize}
\item
если чис\-ло вхождений символа~$a$ в~префикс
$x_1 \cdots x_i$ строки~$u$ совпадает
с~чис\-лом вхож\-де\-ний символа~$b$ в~этот префикс, то $y_i\hm=1$,
\item иначе $y_i=0$.
\end{itemize}
\end{itemize}

Нетрудно видеть, что определенная выше функция~$f$ является
\begin{itemize}
\item
СФ, так как $\forall\, u\in X^*$, $\exists\, v\hm\in Y^*:
f(u,v)\hm=1$,  $\forall\,v'\hm\in Y^*\setminus\{v\}$ $f(u,v')\hm=0$;
\item  ВР, так как $\forall\,u\in X^*$, $\forall\,v\hm\in Y^*$ первое соотношение в~условии~(\ref{safdsdfdsafsafarrr}) из определения понятия ВР верно по
определению функции~$f$, а~второе соотношение
\begin{equation}
\label{dfsasdfsdfgsdfsd}
\forall\,x\in X\quad f(u,v)=\sum\limits_{y\in
Y}f(u x,v y)
\end{equation}
верно потому, что
\begin{itemize}
\item если $f(u,v) = 0$, то
$\forall\,x\hm\in X$, $\forall\,y\hm\in Y$  $f(ux,vy)\hm=0$;
 \item если $f(u,v) = 1$,  то $\forall\,x\hm\in X$
$\exists\,y\hm\in Y$:
$$
f(ux,vy)=1\,, \enskip \forall\,y'\in Y\setminus\{y\}\enskip
f(ux,vy')=0\,,$$
поэтому в~данном случае соотношение~(\ref{dfsasdfsdfgsdfsd})
также верно.
\end{itemize}
\end{itemize}

Можно доказать, что не существует ВА, реакция которого
совпадает с~ВР~$f$.

\vspace*{-6pt}

\section{Заключение}

Критерий реализуемости вероятностных реакций конечными
вероятностными автоматами, изложенный в~настоящей работе, является
более прос\-тым, чем соответствующий критерий Бу\-ха\-ра\-ева--Хо\-му\-та. Однако проверка этого критерия для заданной ВР $f$ может
представлять некоторые трудности, поскольку для доказательства
реализуемости~$f$ необходимо построить конечное мно-\linebreak\vspace*{-12pt}

\pagebreak

\noindent
жество ВР~$\Gamma_f$,
удовлетворяющее условию теоремы~4. Одним из
направлений развития изложенного в~настоящей работе результата может
быть нахождение стратегий построения для заданной ВР~$f$
соответствующего множества ВР~$\Gamma_f$.

Кроме того, поскольку множество~$\Gamma_f$ для заданной ВР~$f$
можно рассматривать как множество состояний одного из ВА, реакция которого совпадает с~$f$, то, следовательно,
к проблеме построения  для заданной ВР~$f$ соответствующего
множества~$\Gamma_f$ с~наименьшим
возможным чис\-лом
элементов сводится проблема построения
для заданного ВА~$A$
такого ВА, реакция которого
совпадает с~реакцией ВА~$A$ и~который содержит
наименьшее возможное чис\-ло состояний (поскольку в~качестве
исходной ВР~$f$ можно рассматривать реакцию ВА~$A$).
Данная проблема известна в~литературе по теории автоматов
как проблема минимизации автоматов и~является одним из
наиболее популярных предметов исследований в~области
теории ВА. Среди последних
результатов, относящихся
к решению данной проблемы, отметим работы~[26--28]. Одним из путей развития данных результатов
может быть  разработка на их основе  методов
построения для заданной ВР~$f$ соответствующего множества~$\Gamma_f$,
которое содержит как можно меньшее чис\-ло элементов.

{\small\frenchspacing
 {%\baselineskip=10.8pt
 \addcontentsline{toc}{section}{References}
 \begin{thebibliography}{99}
\bibitem{1-mir}
\Au{Rabin M.\,O.}
Probabilistic automata~// Information Control, 1963.
Vol.~6. No.\,3. P.~230--245.

\bibitem{2-mir}
\Au{Хопкрофт Д., Мотвани Р., Ульман~Дж.}
Введение в~тео\-рию автоматов, языков и~вы\-чис\-ле\-ний~/
Пер.\ с~англ.~--- М.: Вильямс, 2002.
528~с.
(\Au{Hopcroft J.\,E.,  Motwani~R., Ullman~J.\,D.}
2006. {Introduction to automata theory, languages, and computation}.~---
3rd ed.~--- Pearson. 750~p.)

\bibitem{3-mir}
\Au{Кемени Дж., Снелл Дж.}
Конечные цепи Маркова~/ Пер. с~англ.~--- М.: Наука, 1970. 274~c.
(\Au{Kemeny~J.\,G., Snell~J.\,L.}
1976. {Finite Markov chains.}~---
New York\,--\,Berlin\,--\,Heidelberg\,--\,Tokyo: Springer-Verlag.
225~p.)

\bibitem{4-mir}
\Au{Сarlуle J.\,W.}
Reduced forms for stochastic sequential machines~//
J.~Math. Anal. Appl., 1963.  Vol.~7. No.\,2. P.~167--175.

\bibitem{5-mir}
\Au{Бухараев Р.\,Г.}
Некоторые эквивалентности в~теории вероятностных автоматов~//
Ученые записки Казанcкого университета, 1964. Т.~124. №\,2.
C.~45--65.

\bibitem{6-mir}
\Au{Starke P.\,H.}
Theorie stochastischen Automaten~//
Elektronische Informationsverarbeitung und Kybernetik, 1965.
 Vol.~1. No.\,2. P.~5--32.

\bibitem{7-mir}
\Au{Paz A. }
Introduction to probabilistic automata.~--- New York, NY, USA: Academic Press, 1971.
228~p.

\bibitem{8-mir}
\Au{Бухараев Р.\,Г. }
Основы теории вероятностных автоматов.~--- М.: Наука, 1985.
288~c.

\bibitem{9-mir}
\Au{Segala R., Lynch N.\,A.}
Probabilistic simulations for probabilistic processes~// Nordic J.~Computing, 1995.
Vol.~2. No.\,2. P.~250--273.

\bibitem{10-mir}
\Au{Stoelinga M. }
An introduction to probabilistic automata~//
Bull.  Eur. Assoc. Theor. Comput. Sci.,
2002. Vol.~78. P.~176--198.

\bibitem{11-mir}
\Au{Sokolova A., de Vink~E.\,P.}
Probabilistic automata: System types, parallel
composition and comparison~// Validation of stochastic systems~--- 
a~guide to current research~/  Eds. Ch.~Baier, B.\,R.~Haverkort, 
H.~Hermanns, J.-P.~Katoen,  M.~Siegle.~---
Lecture notes in computer science ser.~--- Springer, 2004.
Vol.~2925. P.~1--43.

\bibitem{12-mir}
\Au{Rabiner L.\,R.}
A tutorial on hidden Markov models and selected applications in speech recognition~//
Proc. IEEE, 1989. Vol.~77. No.\,2. P.~257--286.

\bibitem{13-mir}
\Au{Darwiche А.}
Modeling and reasoning with Bayesian networks.~--- Cambridge: Cambridge University Press, 2009.
562~p.

\bibitem{14-mir}
\Au{Koller D., Friedman N.}
Probabilistic graphical models.
Principles and techniques.~--- Massachusetts: MIT Press, 2009.
1280~p.

\bibitem{15-mir}
Handbook of Markov decision processes~/
Eds. E.\,A.~Feinberg, A.~Shwartz.~--- Boston, MA, USA: Kluwer, 2002.
562~p.

\bibitem{16-mir}
\Au{Wu S.-H., Smolka S.\,A., Stark~E.\,W.}
Composition and behaviors of probabilistic I/O automata~//
Theor. Comput. Sci., 1997.  Vol.~176. P.~1--38.

\bibitem{17-mir}
\Au{Delahaye B., Katoen J.-P., Larsen~K.\,G., Legay~A., Pedersen~M.\,L.,
Sher~F., Wasowski~A.} Abstract probabilistic  automata~//
Inform. Comput., 2013. Vol.~232. P.~66--116.
{\looseness=1

}

\bibitem{18-mir}
\Au{Kudlek M.}
Probability in Petri nets~// Fund. Inform., 2005. Vol.~67. No.\,1.
P.~121--130.

\bibitem{19-mir}
\Au{Liu Y., Miao H., Zeng~H., Li~Z.}
Probabilistic Petri net and its logical semantics~//
9th  Conference (International) on Software Engineering Research, Management
and Applications Proceedings.~--- Baltimore: IEEE Computer Society, 2011.
P.~73--78.

\bibitem{20-mir}
\Au{Eisentraut C., Hermanns H., Zhang~L.}
On probabilistic automata in continuous time~//
25th Annual IEEE Symposium on Logic
in Computer Science (LICS) Proceedings, 2010.
P.~342--351.

\bibitem{21-mir}
\Au{Jonsson B., Larsen~K.\,G., Yi~W.}
Probabilistic  extensions of process algebras~// Handbook of process algebras.~---
North Holland: Elsevier, 2001. P.~685--710.

\bibitem{23-mir} %22
\Au{Homuth H.\,H.}
A type of stochastic automation applicable to the communication channel~//
Angewandte Informatik, 1971. No.\,8. P.~362--372.

\bibitem{22-mir} %23
\Au{Бухараев Р.\,Г.}
Теория абстрактных вероятностных автоматов~// Проблемы кибернетики, 1975.
Вып.~30. C.~147--198.


\bibitem{25-mir} %24
\Au{Мур Э.\,Ф.}
Умозрительные эксперименты с~последовательностными машинами~//
Автоматы.~--- М.: ИЛ, 1956.
C.~179--210.

\pagebreak

\bibitem{24-mir} %25
\Au{Бухараев Р.\,Г.}
Сети вероятностных процессоров~// Математические вопросы кибернетики, 2007.
Вып.~16. С.~57--72.

\bibitem{28-mir} %26
\Au{Mateus P., Qiu D., Li~L.}
On the complexity of minimizing probabilistic and
quantum automata~// Inform. Comput., 2012. Vol.~218.
P.~36--53.

\bibitem{26-mir} %27
\Au{Миронов А.\,М., Френкель~С.\,Л.}
Минимизация вероятностных моделей программ~//
Фундаментальная и~прикладная математика, 2014. Т.~19. Вып.~1.
C.~121--163.

\bibitem{27-mir} %28
\Au{Kiefer S., Wachter B.}
Stability and complexity of minimising probabilistic automata~//
Automata, languages, and programming~/ Eds. J.~Esparza, P.~Fraigniaud, 
Th.~Husfeldt, E.~Koutsoupias.~--- 
Lecture notes in computer science ser.~--- Springer, 2014.
Vol.~8573. P.~268--279.
 \end{thebibliography}

 }
 }

\end{multicols}

\vspace*{-3pt}

\hfill{\small\textit{Поступила в~редакцию 05.05.15}}

%\newpage

\vspace*{12pt}

\hrule

\vspace*{2pt}

\hrule

%\vspace*{12pt}

\def\tit{REALIZABILITY OF PROBABILISTIC REACTIONS BY~FINITE~PROBABILISTIC AUTOMATA}

\def\titkol{Realizability of probabilistic reactions by finite probabilistic automata}

\def\aut{A.\,M.~Mironov}

\def\autkol{A.\,M.~Mironov}

\titel{\tit}{\aut}{\autkol}{\titkol}

\vspace*{-9pt}


\noindent
Institute of Informatics Problems,
Federal Research Center ``Computer Science and Control'' of
the Russian Academy of Sciences, 44-2 Vavilov Str.,
Moscow 119333, Russian Federation



\def\leftfootline{\small{\textbf{\thepage}
\hfill INFORMATIKA I EE PRIMENENIYA~--- INFORMATICS AND
APPLICATIONS\ \ \ 2015\ \ \ volume~9\ \ \ issue\ 3}
}%
 \def\rightfootline{\small{INFORMATIKA I EE PRIMENENIYA~---
INFORMATICS AND APPLICATIONS\ \ \ 2015\ \ \ volume~9\ \ \ issue\ 3
\hfill \textbf{\thepage}}}

\vspace*{3pt}


\Abste{The paper considers the problem of optimizing the access control on 
a set of dynamic threshold strategies in an $M/D/1$ system. If the number of concurrent 
requests in a system is more than the threshold then the system stops accepting 
requests. If the number of requests is less or equal to this value, then the 
system resumes accepting requests. As a target function, the average value of the 
marginal revenue obtained per time unit in the stationary mode is used. It is assumed 
that the system receives a fixed fee for each accepted request and pays a fixed penalty 
for each overdue service of a request. The system does not receive a fee and does not 
pay a penalty for each rejected request. Estimates of the optimal value of the target 
function and the optimal threshold value are obtained.}


\KWE{probabilistic automata; probabilistic reaction; random functions}

\DOI{10.14357/19922264150309}

\Ack
\noindent
The work was partially supported by the Russian Foundation for Basic Research
(project 12-07-00109).



%\vspace*{3pt}

  \begin{multicols}{2}

\renewcommand{\bibname}{\protect\rmfamily References}
%\renewcommand{\bibname}{\large\protect\rm References}

{\small\frenchspacing
 {%\baselineskip=10.8pt
 \addcontentsline{toc}{section}{References}
 \begin{thebibliography}{99}
 \bibitem{rabin}
\Aue{Rabin, M.\,O.} 1963. Probabilistic automata.
\textit{Information Control} 6(3):230--245.

\bibitem{hmu}
\Aue{Hopcroft, J.\,E., R. Motwani, and J.\,D.~Ullman.}
2006. \textit{Introduction to automata theory, languages, and computation}
3rd ed. Pearson. 750~p.

\bibitem{kskcm}
\Aue{Kemeny, J.\,G., and J.\,L. Snell}.
1976. \textit{Finite Markov chains.}
New York\,--\,Berlin\,--\,Heidelberg\,--\,Tokyo: Springer-Verlag.
225~p.

\bibitem{245}
\Aue{Carlyle, J.\,W.}
1963. Reduced forms for stochastic sequential machines.
\textit{J.~Math. Anal. Appl.}
7(2):167--175.

\bibitem{21}
\Aue{Bukharaev, R.\,G.}
1964. Nekotorye ekvivalentnosti v~teorii veroyatnostnykh avtomatov
[Certain equivalences in the theory of probabilistic automata].
\textit{Uchenye zapiski Kazanckogo Universiteta}
[Lecture Notes of Kazan University] 124(2):45--65.

\bibitem{415}
\Aue{Starke, P.\,H.}
1965. Theorie stochastischen Automaten.
\textit{Elektronische Informationsverarbeitung und Kybernetik} 1(2):5--32.

\bibitem{paz}
\Aue{Paz, A.}
1971. \textit{Introduction to probabilistic automata.}
New York, NY: Academic Press. 228~p.

\bibitem{buharaevk}
\Aue{Bukharaev, R.\,G.}
1985. \textit{Osnovy teorii veroyatnostnykh avtomatov}
[Foundations of probabilistic automata theory]. Moscow: Nauka.
288~p.

\bibitem{segala}
\Aue{Segala, R., and N.\,A.~Lynch}.
1995. Probabilistic simulations for probabilistic processes.
\textit{Nordic J.~Computing} 2(2):250--273.

\bibitem{Stoelinga} %10
\Aue{Stoelinga, M.}
2002. An introduction to probabilistic automata.
\textit{Bull. Eur. Assoc.  Theor. Comput. Sci.}
78:176--198.
{\looseness=1

}

\bibitem{SokolovaVink} %11
\Aue{Sokolova, A., and E.\,P.~de Vink.}
2004.
Probabilistic automata: System types, parallel composition and comparison.
\textit{Validation of stochastic systems~--- a~guide to current research}.
Eds. Ch.~Baier, B.\,R.~Haverkort, H.~Hermanns, J.-P.~Katoen, and M.~Siegle.
Lecture notes in computer science ser. Springer. 2925:1--43.

\bibitem{hmm} %12
\Aue{Rabiner, L.\,R.}
1989. A~tutorial on hidden Markov models and selected applications in
speech recognition. \textit{Proc. IEEE}
77(2):257--286.

\bibitem{bayes}
\Aue{Darwiche, A.}
2009. \textit{Modeling and reasoning with Bayesian networks.}
Cambridge University Press. 562~p.

\bibitem{gms}
\Aue{Koller, D., and N.~Friedman.}
2009. \textit{Probabilistic graphical models. Principles and techniques.}
Massachusetts: MIT Press. 1280~p.

\bibitem{mdp} %15
Feinberg, E.\,A., and A.~Shwartz, eds.
2002. \textit{Handbook of Markov decision processes}.
Boston, MA: Kluwer. 562~p.

\bibitem{ioauto} %16
\Aue{Wu, S.-H., S.\,A. Smolka, and E.\,W.~Stark}.
1997. Composition and behaviors of probabilistic I/O automata.
\textit{Theor. Comput. Sci.} 176(1)1--38.

\bibitem{apapaper}
\Aue{Delahaye, B., J.-P. Katoen, K.\,G.~Larsen, A.~Legay,
M.\,L.~Pedersen, F.~Sher, and A.~Wasowski}.
2013. Abstract probabilistic  automata.
\textit{Inform. Comput.} 232:66--116.

\bibitem{petri1}
\Aue{Kudlek, M.}
2005. Probability in Petri nets.
\textit{Fund. Inform.} 67(1):121--130.

\bibitem{petri2} %19
\Aue{Liu, Y., H.~Miao, H.~Zeng, and Z.~Li.}
2011. Probabilistic Petri net and its logical semantics.
\textit{9th  Conference (International) on Software Engineering Research,
Management and Applications Proceedings}. 73--78.

\bibitem{conttime} %20
\Aue{Eisentraut, C., H. Hermanns, and L.~Zhang}.
2010. On probabilistic automata in continuous time.
\textit{25th Annual IEEE Symposium on Logic
in Computer Science (LICS) Proceedings.} 342--351.

\bibitem{probprocalg} %21
\Aue{Jonsson, B., K.\,G.~Larsen,  and W.~Yi}.
2001. Probabilistic  extensions of process algebras.
\textit{Handbook of process algebra.}
North Holland: Elsevier. 685--710.

\bibitem{296} %22
\Aue{Homuth, H.\,H.}
1971. A~type of stochastic automation applicable to the communication channel.
\textit{Angewandte Informatik} 8:362--372.

\bibitem{35} %23
\Aue{Bukharaev, R.\,G.}
1975. Teoriya abstraktnykh veroyatnostnykh avtomatov
[A theory of abstract probabilistic automata].
\textit{Problemy Kibernetiki}
[Problems Cybernetics]  30:147--198.

\bibitem{mur} %24
\Aue{Moore, E.\,F.}
1956. Gedanken-experiments on sequential machines.
\textit{Automata Studies. Annals of Mathematical Studies.}
Princeton University Press.
34:129--153.

\bibitem{svp07} %25
\Aue{Bukharaev, R.\,G.}
2007. Seti veroyatnostnykh protsessorov [Networks of probabilistic processors].
\textit{Matematicheskie Voprosy Kibernetiki}
[Mathematical Issues of Cybernetics] 16:57--72.

\bibitem{mqll} %26
\Aue{Mateus, P., D. Qiu, and L.~Li.}
2012. On the complexity of minimizing probabilistic and quantum automata.
\textit{Inform. Comput.} 218:36--53.

\bibitem{mironov} %27
\Aue{Mironov, A.\,M., and S.\,L.~Frenkel}.
2014. Mini\-mi\-za\-tsiya veroyatnostnykh modeley programm
[Minimization of probabilistic models of programs].
\textit{Fundamental'naya i~Prikladnaya Matematika}
[Fundamental Applied Mathematics] 19(1):121--163.


\bibitem{scmpa} %28
\Aue{Kiefer, S., and B. Wachter}.
2014. Stability and complexity of minimising probabilistic automata.
\textit{Automata, languages, and programming}.
Eds. J.~Esparza, P.~Fraigniaud, Th.~Husfeldt, and E.~Koutsoupias.
Lecture notes in computer science ser. Springer. 8573:268--279.



 \end{thebibliography}

 }
 }



\end{multicols}

\vspace*{-3pt}

\hfill{\small\textit{Received May 5, 2015}}


\Contrl

\noindent
\textbf{Mironov Andrew M.} (b.\ 1966)~---
Candidate of Science (PhD) in physics and mathematics,
senior scientist, Institute of Informatics Problems,
Federal Research Center ``Computer Science and Control'' of
the Russian Academy of Sciences, 44-2 Vavilova Str., Moscow 119333,
Russian Federation;  amironov66@gmail.com

\label{end\stat}


\renewcommand{\bibname}{\protect\rm Литература}