\def\stat{abramov}

\def\tit{ВЛИЯНИЕ МОРФОЛОГИЧЕСКИХ ОПЕРАЦИЙ НА~РАСПОЗНАВАНИЕ
ФИГУРЫ ДВИЖУЩЕГОСЯ ЧЕЛОВЕКА ПО~ПОТОКУ
ИЗОБРАЖЕНИЙ$^*$}

\def\titkol{Влияние морфологических операций на распознавание
фигуры движущегося человека по потоку
изображений}

\def\aut{М.\,О.~Абрамов$^1$, М.\,Ю.~Катаев$^2$}

\def\autkol{М.\,О.~Абрамов, М.\,Ю.~Катаев}

\titel{\tit}{\aut}{\autkol}{\titkol}

{\renewcommand{\thefootnote}{\fnsymbol{footnote}} \footnotetext[1]
{Работа выполнена при поддержке РГНФ, грант №\,14-16-70008~a(p).}}


\renewcommand{\thefootnote}{\arabic{footnote}}
\footnotetext[1]{Национальный исследовательский Томский государственный
университет, maxim\_amo@mail.ru}
\footnotetext[2]{Томский государственный университет систем управления
и~радиоэлектроники; Юргинский технологический
институт (филиал) Национального исследовательского Томского политехнического университета,
kataev.m@sibmail.com}

\Abst{Приведено описание методики выделения фигуры движущегося человека по потоку
изоб\-ра\-же\-ний с~использованием морфологических операций. Поставленная задача
определяется получением максимальной точности выделения фигуры человека на бинарном
изображении. Приведены численные результаты, полученные на основе целого ряда
методов. Сделаны выводы и~даны рекомендации по распознаванию фигуры человека на
изображении.}

\KW{морфологические операции; одна камера; распознавание фигуры человека; цифровая
обработка; бинарные изображения}

\DOI{10.14357/19922264150312}


\vspace*{6pt}


\vskip 14pt plus 9pt minus 6pt

\thispagestyle{headings}

\begin{multicols}{2}

\label{st\stat}

\section{Введение}

  Цифровая обработка изображений~[1--6] находит широкое применение
практически во всех \mbox{об\-ластях} человеческой деятельности. Одними из наиболее
сложных в~этом направлении являются задачи автоматического извлечения
информации из изображений в~различных условиях их получения. Особую
уникальность приобретают методы обработки потоков изображений, так как
они сложны для автоматического анализа из-за множества параметров, которые
определяются условиями формирования динамического изображения, а~также
множества вариантов их компенсации или выделения. Каждый из методов
обработки изображений может быть реализован с~помощью различных
алгоритмов, а~их выбор и~сочетание определенно влияют на быстродействие и~точность.

  Рассматриваемая в~статье задача является развитием цикла работ~[7--9],
связанных с~оценкой двигательной активности человека (на примере его
походки). Параметры двигательной активности: скорость, энергия, длина и~симметричность шагов и~др.~--- восстанавливаются методом, связанным с~оценкой центра масс по изображению фигуры. Точность метода существенно
зависит от качества выделения фигуры.

  Изменение свойств динамического фона от кад\-ра к~кадру затрудняет
применение типизованных математических процедур обработки изображений,
что, например, при выделении фигуры человека приводит к~ее изменению.
Изменение фигуры человека проявляется в~разрывах фигуры, уменьшении
площади или появлении дополнительных артефактов, чаще всего связанных с~тенью, и~др.

  Для получения оценок изменения центра масс в~данной статье
рассматриваются особенности, которые связаны с~применением
морфологических операций.

\section{Постановка задачи }

  В последние годы ведутся активные работы в~области приложения
математических методов к~обработке видеоизображений, что обусловлено
быст\-рым развитием техники. Появляются новые способы использования уже
известных методов, что требует проведения специальных исследований
относительно поставленной задачи. Одной из важнейших является задача
обнаружения и~выделения человека на изображении, сложность которой
повышается по разным причинам: высокой измен\-чивости освещенности
различных участков изображения, появления тени, динамического фона,
различного числа объектов фона, различных форм фигуры человека, его
положения относительно камеры и~пр.~[10]. Однако при решении конкретных
задач контроля, медицины, спорта необходимо иметь четкие и~однозначные
оценки направления движения человека, его антропометрических
характеристик при различных условиях освещенности, типа фона, положения и~наклона тела.

  Данное исследование проводится в~медицинских целях и~посвящено
разработке автоматизированной программной системы оценки двигательной
способности постинсультных пациентов~\cite{7-ab, 8-ab}. От точности работы
программной системы зависит оценка двигательной активности, которая
определяет выбор тех или иных реабилитационных мероприятий. Медицинская
направленность приложения накладывает жесткие условия на точность и~скорость работы алгоритмов обработки изображений, которые должны
работать в~режиме, близком к~реальному времени.

  Оценка двигательной активности опирается на метод, связанный с~изучением
положения при движении центра масс~\cite{8-ab}. Для точного определения
центра масс фигура человека в~процессе движения должна быть определена
максимально близко к~реальной. Авторами было проведено исследование
влияния методов предобработки на качество выделения фигуры движущегося
человека на изображениях~\cite{9-ab}. В~результате был получен диапазон
данных по изменению восстановленной фигуры человека, отображающий
влияние комбинаций тех или иных методов обработки на качество выделения.
Оценка качества восстановления проводилась по критерию сравнения площади
полученной фигуры и~реальной фигуры, выделенной вручную максимально
близко к~оригиналу.

  Очевидно, что условия измерений и~сцена определяют тип и~стабильность
фона. Особую проблему для выделения фигуры человека создает динамический
фон, когда наблюдаются изменения сце-\linebreak ны, освещенности, возникающие тени и~др. 
Эти\linebreak особенности учитывались при разработке про\-граммно-ал\-го\-рит\-ми\-че\-ской части. Ранее авторами было исследовано влияние различных
методов пред\-об\-ра\-бот\-ки и~обработки на конечный результат~\cite{9-ab}. Анализ
этой работы позволил выявить, что площадь полученной фигуры после
обработки занижена по сравнению со своими реальными значениями. Одним из
вариантов решения этой проб\-ле\-мы является применение морфологических
операций. Дело в~том, что в~некоторых случаях фигура человека после
выделения теряет свою целостность, т.\,е.\ некоторые части фигуры получаются
отделенными друг от друга. Нарушение це\-лост\-ности фигуры приводит к~тому,
что при фильтрации шумов на изображении отдельные части фигуры могут
быть просто отброшены как случайные помехи. Морфологические операции
позволяют объединить соседние части фигуры и~заполнить получающиеся
артефакты (пус\-то\-ты) внутри фигуры.

\section{Математическая морфология}

  Термин <<морфология>>~\cite{11-ab} относится к~описанию свойств формы и~структуры ка\-ких-ли\-бо объектов. В~контексте машинного зрения этот
термин относится к~описанию свойств формы областей на изображении.
Операции математической морфологии изначально были определены как
операции над множествами, в~дальнейшем было определено, что они также
полезны в~задачах обработки мно\-жества точек в~двумерном пространстве.
Множествами в~математической морфологии пред\-став\-ля\-ют\-ся объекты на
изображении.

  Так как в~рассматриваемом случае математическая морфология используется
применительно к~зада\-че распознавания объектов на двумерном изоб\-ра\-же\-нии,
то в~первую очередь представляет интерес бинарная морфология, описывающая\linebreak
действия над бинарными изображениями, пред\-став\-лен\-ны\-ми в~виде
упорядоченного множества точек~--- черных и~белых, задаваемых как~1 и~0
соответственно. Под областью изображения обычно понимается подмножество
единиц изображения. Каждая операция двоичной морфологии является
некоторым преобразованием этого множества. В~качестве исходных данных
принимаются двоичное изображение~$S$ и~некоторый структурный элемент
(СЭ), или примитив,~$B$. Результатом операции также является двоичное
изображение.

  Структурный элемент представляет собой некоторое двоичное изображение,
которое имеет произвольную форму и~размер и~у~которого определен один
элемент, называемый ядром (началом). Как правило, используются
симметричные СЭ в~форме квадрата, креста или круга, а~их ядром является
точка, находящаяся посередине. Размер СЭ в~пикселах примем за~$r$.

  Базовыми операциями морфологии являются расширение (дилатация) и~сужение (эрозия). Расширение двоичного изображения~$S$ на СЭ~$B$
обозначается как
  $$
  (S\oplus  B) (u,v) =\max\limits_{(i,j)\in B} \left\{ S(u+i,v+j)+B(i,j)\right\}
  $$
и понимается как объединение всех копий СЭ, центрированного по каждому
пикселу изображения~$S$. Сужение двоичного изображения~$S$ на СЭ~$B$
обозначается как
$$
(S!\, B) (u,v) =\min\limits_{(i,j)\in B} \left\{ S(u+i, v+j)+B(i,j)\right\}\,.
$$

  Операции сужения и~расширения редко используются по отдельности, так
как они изменяют размеры областей изображения. Однако если обе опера\-ции
используются последовательно, то это позволяет добиться различных
результатов в~зависимости от порядка их следования. Тип морфологической
операции, когда сначала производится расширение, а~потом сужение,
называется закрытием (замыканием). Обратному порядку соответствует
открытие (размыкание). Открытие обозначается следующим образом:
  \begin{equation}
  S\circ B = (S!\, B)\oplus B\,,
  \label{e1-ab}
  \end{equation}
объекты, которые меньше, чем СЭ~$B$, исчезают, а~большие объекты
остаются.

  Морфологическое закрытие:
  \begin{equation}
  S\bullet B =(S\oplus B)!\, B\,,
  \label{e2-ab}
  \end{equation}
отверстия на области изображения, которые меньше~$B$, будут
заполнены~\cite{12-ab}.

\section{Исследование морфологических операций}

  В рассматриваемой работе морфологическая обработ\-ка применяется для
следующих целей: удаление шумов, объединение разделенных частей фигуры и~заполнение пустот внутри фигуры. В~результате работы морфологических
операций на выходе должна получиться замкнутая фигура человека, которая
максимально точно соответствует по форме и~площади реальной фигуре.

Чтобы исследовать особенности использования морфологических
операций применительно к~данной задаче, был проведен ряд тестов,
показы\-ва\-ющих их основные достоинства и~недостатки. На вход для обработки
подавались изображения с~различными комбинациями проблемных для
обработки ситуаций, возникающих в~реальных условиях (например, мелкие
помехи, крупные помехи, слабая фрагментация фигуры человека, сильная
фрагментация фигуры человека и~т.\,д.).

  Основными морфологическими методами были выбраны операции
закрытия~(\ref{e2-ab}) и~открытия~(\ref{e1-ab})~\cite{13-ab}. Исследовалось
как их отдельное использование, так и~последовательное, изменялся размер
матрицы СЭ и~ее форма. По результатам исследования определялись наиболее
подходящие параметры для решения поставленной задачи (рис.~1 и~2).


  Визуальная оценка результатов показывает, что наиболее подходящей
комбинацией для решения поставленной задачи является последовательное
выполнение над изображением операций открытия и~закрытия (рис.~3). При
использовании такого подхода удалось очистить изображение от мелких\linebreak\vspace*{-12pt}
\begin{center}  %fig1
\vspace*{-1pt}
\mbox{%
 \epsfxsize=78mm
 \epsfbox{abr-1.eps}
 }

\end{center}

%\vspace*{9pt}

\noindent
{{\figurename~1}\ \ \small{Действие операции морфологического закрытия на изображение
((\textit{a})~исходное
изображение) фрагментированного силуэта человека при различном размере СЭ: 
 (\textit{б})~$r\hm= 10$; (\textit{в})~15; (\textit{г})~$r\hm=20$}}
 
 \vspace*{12pt}


\begin{center}  %fig2
\vspace*{-1pt}
 \mbox{%
 \epsfxsize=54.821mm
 \epsfbox{abr-2.eps}
 }

\end{center}

%\vspace*{3pt}

\noindent
{{\figurename~2}\ \ \small{Действие операции морфологического открытия на изображение
фрагментированного силуэта человека при различном размере СЭ: (\textit{а})~$r\hm= 5$; 
(\textit{б})~10; (\textit{в})~$r\hm=15$}}

 \vspace*{12pt}

\begin{center}  %fig3
\vspace*{-1pt}
  \mbox{%
 \epsfxsize=78mm
 \epsfbox{abr-3.eps}
 }

\end{center}

%\vspace*{3pt}

\noindent
{{\figurename~3}\ \ \small{Последовательное выполнение морфологического открытия и~закрытия:
(\textit{а})~исходное изображение; (\textit{б})~морфологическое открытие $r\hm = 10$;
(\textit{в})~морфологическое закрытие $r\hm = 30$}}





\vspace*{18pt}


\addtocounter{figure}{3}




\noindent
артефактов, размеры которых меньше размерности матрицы СЭ, не изменяя
части фигуры человека. Также удалось добиться соединения разделенных
участков фигуры на расстоянии, не превышающем размеров матрицы СЭ.
Наиболее удачной формой СЭ был определен круг, так как контур фигуры
человека состоит в~основном из плавных изгибов и~непрямолинейных линий.

\begin{table*}[b]\small
\begin{center}
\begin{tabular}{|c|c|c|c|r|r|c|c|c|}
\multicolumn{9}{c}{Сравнение результатов, полученных с~использованием
морфологических операций и~без них}\\
\multicolumn{9}{c}{\ }\\[-4pt]
\hline
&&&&&&&&\\[-9pt]
№ метода&$\Delta \overline{S}$, &$\Delta \overline{S}^\prime$, &№ метода&
\multicolumn{1}{c|}{$\Delta
\overline{S}$,} &\multicolumn{1}{c|}{$\Delta \overline{S}^\prime$,}  &№ метода&
$\Delta \overline{S}$, &$\Delta \overline{S}^\prime$, \\
\cline{1-1}
\cline{4-4}
\cline{7-7}
RGB&\%&\%&HSV&\multicolumn{1}{c|}{\%}&\multicolumn{1}{c|}{\%}&YUV&\%&\%\\
\hline
1&$-29{,}61$ &6,03 &17&$-20{,}41$ &43,66 &33&$-51{,}55$ &5,23\\
2&$-23{,}63$ &5,04 &18&$-40{,}88$ &4,84 &34&$-24{,}49$ &10,04\\
3&$-29{,}54$ &6,05 &19&\cellcolor[gray]{.6}$-0{,}68$ &\cellcolor[gray]{.6}36,56 &35
&$-46{,}94$ &5,45\\
4&$-23{,}56$ &4,95 &20&10,62 &18,31 &36&$-25{,}25$ &9,64\\
5&$-28{,}91$ &5,24 &21&$-18{,}83$ &36,73 &37&$-51{,}40$ &4,52\\
6&$-24{,}16$ &3,70 &22&$-39{,}38$ &33,16 &38&$-26{,}30$ &6,97\\
7&$-28{,}98$ &5,23 &23&\cellcolor[gray]{.6}$-1{,}94$ &\cellcolor[gray]{.6}26,48
&39&$-46{,}63$ &5,25\\
8&$-24{,}07$ &4,44 &24&15,56 &$-0{,}41$ &40&$-26{,}09$ &7,09\\
9&$-29{,}27$ &4,19 &25&$-23{,}37$ &18,47 &41&$-51{,}51$&3,97\\
10\hphantom{9}&$-25{,}03$ &3,32 &26&$-47{,}44$ &$-39{,}70$ &42&$-27{,}62$ &5,03\\
11\hphantom{9}&$-29{,}22$ &4,26 &27&$-6{,}50$ &14,80 &43&$-47{,}26$ &5,05\\
12\hphantom{9}&$-24{,}91$ &3,34 &28&9,03 &2,88 &44&$-28{,}09$ &4,45\\
13\hphantom{9}&$-30{,}46$ &5,48 &29&$-20{,}17$ &51,14 &45&$-53{,}20$ &4,09\\
14\hphantom{9}&$-22{,}53$ &4,90 &30&$-44{,}76$ &31,63 &46&$-25{,}24$ &9,04\\
15\hphantom{9}&$-30{,}39$ &5,48 &31&\cellcolor[gray]{.6}1,09 &\cellcolor[gray]{.6}41,10
&47&$-47{,}30$ &5,33\\
16\hphantom{9}&$-22{,}28$ &5,06 &32&11,48 &24,93 &48&$-24{,}86$ &9,46\\
\hline
\end{tabular}
\end{center}
\end{table*}

  Таким образом, для достижения максимальной эффективности
морфологической обработки было принято решение использовать
последовательно операции открытия и~закрытия с~различным размером
матрицы ядра. Данный подход позволяет очистить изображение от мелкого
шума без существенного влияния на фигуру, а~также объединить
фрагментированные участки фигуры человека и~заполнить возникающие
пустоты, тем самым улучшая качество всего процесса распознавания фигуры
человека.

\section{Результаты}

  Авторами был разработан программно-ап\-па\-рат\-ный комплекс,
предназначенный для исследования эффективности различных комбинаций
методов предобработки, обработки и~постобработки с~целью распознавания
человека на потоке изоб\-ра\-же\-ний и~дальнейшего анализа. Все этапы обработки
проходят последовательно над всеми изоб\-ра\-же\-ни\-ями из потока. На каждом
этапе можно использовать до~4--5~разных вариантов обработки, таких как
выбор цветового пространства, метод фильтрации, алгоритм распознавания,
эквализация гистограммы и~др. Суммарное число возможных комбинаций всех
вариантов методов (добавленных в~программно-аппаратный комплекс)
равно~48. Для распознавания фигуры в~данном исследовании используется
модифицированный алгоритм на базе адаптивной гауссовской смешанной
модели~\cite{14-ab}. На вход подаются последовательно все кадры видеоряда,
начиная с~кадра, на котором нет исследуемого субъекта. Более подробно
используемые алгоритмы и~методы описаны в~работе~\cite{9-ab}.

  Все выбранные методы обработки применялись к~базе
видеопоследовательностей, собранной авторами статьи. На данный момент в~нее входят 
изоб\-ра\-же\-ния 30~человек. Съемки проводились врачами местных
поликлиник~\cite{8-ab} с~разрешения пациентов с~нарушениями походки и~без,
в разные периоды времени. Разрешение видео~--- $640\times 480$~пикселов.
Средняя продолжительность~--- 8--12~с с~частотой~25~кадр/с.

  Для исследования влияния морфологических операций на качество
распознавания фигуры человека на изображении использовались алгоритмы на
основе формул~(\ref{e1-ab}) и~(\ref{e2-ab}). Для численной оценки точности
выделения фигуры было проведено сравнение результатов предыдущей
работы~\cite{9-ab} с~результатами, полученными после проведения
морфологических операций. Критерием точности служило отклонение
найденной площади фигуры человека от эталонной, размер которой известен.

  Морфологическая обработка применялась на финальной стадии работы
  про\-граммно-ап\-па\-рат\-но\-го комплекса. Использовались
последовательно операции открытия с~диаметром СЭ $r\hm = 5$ и~закрытия
с~$r \hm= 20$ в~указанном порядке.

  В приведенной таблице в~колонке $\Delta \overline{S}$ 
  указано отличие в~процентах полученной площади фигу-\linebreak ры от эталонного значения~\cite{9-ab}.
Несмотря на со-\linebreak хранение тенденции к~занижению результатов от реальной
площади, наблюдается существенное улучшение по сравнению с~циклом
обработки без морфологических операций. Это улучшение отражено в~колонке
$\Delta \overline{S}^\prime$ таблицы в~процентах от результатов, полученных
без морфологических опера-\linebreak ций.

  Для того чтобы оценить возможности различных цветовых пространств для
выделения фигуры человека на изображении, проводились одинаковые расчеты
для таких цветовых пространств, как RGB, HSV и~IUV (см.\ таблицу).



\section{Выводы}

  Среднее процентное отклонение (в~абсолютном выражении) площади
восстановленной фигуры при использовании морфологических операций от
эталонной составляет 25,98\%, что на 12,86\% лучше, чем без их использования
(38,84\%~\cite{9-ab}). Помимо значительного среднестатистического
улучшения результата при использовании методов~19, 23 и~31 удалось
добиться максимального приближения значения восстановленной фигуры 
к~эталонной (отклонение менее 2\%). Таким образом, именно эти методы
являются наиболее подходящими для решения поставленной задачи. Обращаем
внимание на то, что помимо цветового пространства HSV (методы 17--32, см.\
таблицу) все эти методы объединяет обработка методом коррекции баланса
белого <<Серый мир>> и~использование методов фильтрации для снижения
уровня шума.

  Стоит отметить, что в~зависимости от используемого цветового пространства
изменяются особенности полученной фигуры человека. При использовании
пространства RGB (методы~1--16, см.\ таблицу) получаются наиболее
стабильные результаты, а~при визуальной оценке фигура человека наиболее
четко соответствует реальной. Однако данное цветовое пространство наименее
восприимчиво к~внешним факторам, таким как схожесть цветового тона
элементов фона и~исследуемого объекта. На всех полученных изображениях с~использованием RGB те части тела человека, что были свободны от одежды,
имели схожие тона с~фоном (светлая кожа и~светлый фон), поэтому после всех
процедур обработки эти части относились алгоритмом к~фону и~обрезались.
Данную проблему легко решить при наличии на человеке одежды, средне или
сильно отличающейся от фона. Поэтому можно рекомендовать использование
этого цветового пространства в~том случае, если есть возможность влиять на
объект или среду исследования. Пространство HSV в~меньшей степени
чувствительно к~внешним факторам, оно позволило распознать все части
исследуемого объекта. Однако, в~отличие от RGB, силуэт полученного человека
получается искаженным, нечеткие линии на краях и~различного рода артефакты
делают это пространство очень чувствительным к~используемым методам
предобработки. Наилучшие результаты получились при его использовании,
поэтому можно рекомендовать выбор этого цветового пространства в~тех
случаях, когда исследователь не может контролировать среду или объект
записи. Пространство YUV (методы~33--48, см.\ таблицу) при визуальном
анализе и~согласно таблице показало наихудшие результаты. Несмотря на
стабильность метода при использовании контрастирования на этапе
предобработки, следует воздержаться от использования данного цветового
пространства для распознавания движущихся объектов на изображении.

  Все сказанное выше дает право заключить, что использование
морфологических операций приводит к~улучшению качества распознавания
фигуры движущегося человека на изображении. В~данной работе сделан упор
на описание практического действия морфологических операций, поэтому в~совокупности с~предыдущей публикацией по предобработке~\cite{9-ab} она
может оказать содействие желающим в~создании своей собственной системы
распознавания или в~изучении действия методов предобработки и~морфологических операций на изображения.

\vspace*{-6pt}

{\small\frenchspacing
 {%\baselineskip=10.8pt
 \addcontentsline{toc}{section}{References}
 \begin{thebibliography}{99}
 
 \bibitem{5-ab} %1
\Au{Грузман И.\,С., Киричук В.\,С., Косых~В.\,П., Перетягин~Г.\,И.,
Спектор~А.\,А.} Цифровая обработка изоб\-ражений в~информационных системах.~--- Новосибирск: НГТУ, 2002. 352~c.

\bibitem{4-ab} %2
\Au{Форсайт А., Понс Дж.} Компьютерное зрение. Современный подход.~--- М.: Вильямс,
2004. 928~с.

\bibitem{2-ab} %3
\Au{Яне Б.} Цифровая обработка изображений~/ Пер. с~англ.~---
М.: Техносфера, 2007. 584~с.

 \bibitem{1-ab} %4
 \Au{Гонсалес Р., Вудс P., Эддинс~C.}
 Цифровая обработка изображений в~среде MATLAB.~---
М.: Техносфера, 2006. 616~с.

(\Au{Jane B.} Digital image processing.~--- 6th ed.~---
 Berlin: Springer, 2005. 608~p.)
\bibitem{3-ab} %5
\Au{Красильников Н.\,Н.} Цифровая обработка 2D- и~3D-изображений.~---
СПб.: БХВ-Пе\-тер\-бург, 2011. 608~с.


\bibitem{6-ab}
\Au{Захаров Р.\,К.}
Методы повышения качества изображений в~задачах распознавания~//
Современные научные исследования и~инновации, 2012. №\,8. С.~9--10.
\bibitem{7-ab}
\Au{Катаева Н.\,Г., Катаев М.\,Ю., Чистякова~В.\,А.}
Автоматизированная оценка степени
нарушения ходьбы после инсульта~// Медицинская техника, 2012. №\,1. С.~40--43.
\bibitem{8-ab}
\Au{Катаев М.\,Ю., Катаева Н.\,Г., Катаев~С.\,Г., Абрамов~М.\,О., Чистякова~В.\,А.}
Определение и~анализ двигательной активности постинсультного пациента по видеопотоку~//
Бюллетень сибирской медицины, 2014. Т.~13. №\,5. С.~36--41.
\bibitem{9-ab}
\Au{Абрамов М.\,О., Катаев М.\,Ю.} Влияние методов пред\-об\-ра\-бот\-ки на восстановление
фигуры движущегося человека из потока изображений~// Докл. ТУСУР, 2014. №\,4(33).
С.~65--69.
\bibitem{10-ab}
\Au{Бойко И.\,А., Гурьянов Р.\,А.} Распознавание объектов на основе видеосигнала,
полученного с~камеры, уста\-нов\-лен\-ной на подвижной платформе~// Молодой ученый, 2013.
№\,6. С.~34--36.
\bibitem{11-ab}
\Au{Огнев И.\,В., Сидорова Н.\,А.} Обработка изображений методами математической
морфологии в~ассоциативной осцилляторной среде~// Известия вузов. Поволжский регион.
Технические науки, 2007. №\,4. С.~87--97.
\bibitem{12-ab}
\Au{Burger W., Burge M.\,J.} Digital image processing: An algorithmic
introduction using Java.~---  New York, NY, USA:
Springer Science \& Business Media, 2007. 566~p.
\bibitem{13-ab}
\Au{Najman L.} Mathematical morphology: From theory to applications.~---
London: Wiley, 2010. 520~p.
\bibitem{14-ab}
\Au{Kaew Trakulpong P., Bowden R.}
An improved adaptive background mixture model for real-time tracking
with shadow detection~// Computer
Vision and Distributed Processing:  2nd European Workshop on Advanced
Video-based Surveillance Systems, AVBS01, Proceedings.~---
Kingston: Kluwer Academic Publs., 2002. P.~135--144.
 \end{thebibliography}

 }
 }

\end{multicols}

\vspace*{-3pt}

\hfill{\small\textit{Поступила в~редакцию 26.01.15}}

%\newpage

\vspace*{12pt}

\hrule

\vspace*{2pt}

\hrule

%\vspace*{12pt}

\def\tit{INFLUENCE OF MORPHOLOGICAL OPERATIONS
ON~RECOGNITION OF~A~MOVING HUMAN FIGURE FROM~A~SET~OF~IMAGES}

\def\titkol{Influence of morphological operations on recognition of a moving human figure from a set of images}

\def\aut{M.\,O.~Abramov$^1$ and M.\,Yu.~Kataev$^{2,3}$}

\def\autkol{M.\,O.~Abramov and M.\,Yu.~Kataev}

\titel{\tit}{\aut}{\autkol}{\titkol}

\vspace*{-9pt}


\noindent
$^1$National Research Tomsk State University,
 36~Lenin Av., Tomsk 634050, Russian Federation

\noindent
$^2$Tomsk State University of Control Systems and Radioelectronics, 40~Lenin Av., Tomsk 634050, Russian Federation

\noindent
$^3$Yurga Institute of Technology (Branch) of National Research Tomsk Polytechnic 
University, 26~Leningradskaya\linebreak
$\hphantom{^1}$Str., Yurga, Kemerovo Region 652055, Russian Federation


\def\leftfootline{\small{\textbf{\thepage}
\hfill INFORMATIKA I EE PRIMENENIYA~--- INFORMATICS AND
APPLICATIONS\ \ \ 2015\ \ \ volume~9\ \ \ issue\ 3}
}%
 \def\rightfootline{\small{INFORMATIKA I EE PRIMENENIYA~---
INFORMATICS AND APPLICATIONS\ \ \ 2015\ \ \ volume~9\ \ \ issue\ 3
\hfill \textbf{\thepage}}}

\vspace*{3pt}


\Abste{The article describes the methodology of moving human figure recognition
from set of images with the help of morphological operations. Accuracy of recognition
results is determined by the difference between the true and recovering figure
squares from binary images. Comparison of the obtained accuracy for different
morphological operations for some color spaces (RGB, HSV, and YUV) is investigated.
The paper provides recommendations for improvement of human figure recognition
from set of images.}


\KWE{morphological operations; one camera; human figure recognition;
digital processing; binary images}

\DOI{10.14357/19922264150312}

\Ack
\noindent
Work was performed with financial support of the Russian Foundation for Humanities,
project No.\,14-16-70008)~a(r).


%\vspace*{3pt}

  \begin{multicols}{2}

\renewcommand{\bibname}{\protect\rmfamily References}
%\renewcommand{\bibname}{\large\protect\rm References}

{\small\frenchspacing
 {%\baselineskip=10.8pt
 \addcontentsline{toc}{section}{References}
 \begin{thebibliography}{99}
 
 \bibitem{5-ab-1} %1
\Aue{Gruzman, I.\,S., V.\,S. Kirichuk, V.\,P.~Kosykh, G.\,I.~Peretyagin,
and A.\,A.~Spektor}. 2002.
\textit{Tsifrovaya obrabotka izob\-ra\-zheniy v~informatsionnykh sistemakh}
[Digital image processing in information systems]. Novosibisrk:  NGTU Publ. 352~p.

\bibitem{4-ab-1} %2
\Aue{Forsayt, A. and J.~Pons}. 2004.
\textit{Komp'yuternoe zrenie. Sovremennyy podkhod}
[Computer vision. Modern approach]. Moscow: Vil'yams. 928~p.

\bibitem{2-ab-1} %3
\Aue{Jane, B.} 2005. \textit{Digital image processing}. 6th ed.
Berlin: Springer. 608~p.
\bibitem{1-ab-1} %4
\Aue{Gonsales, P., P. Vuds, and C.~Eddins}.
2006. \textit{Tsifrovaya obrabotka izobrazheniy v~srede MATLAB} [Digital image
processing in MATLAB]. Moscow: Technosphere. 616~p.

\bibitem{3-ab-1} %5
\Aue{Krasil'nikov, N.\,N.} 2011. \textit{Tsifrovaya obrabotka 2D i~3D
izobrazheniy} [Digital 2D and 3D image processing].
St.\ Petersburg: BKhV-Peterburg. 608~p.


\bibitem{6-ab-1}
\Aue{Zakharov, R.\,K.}
2012. Metody povysheniya kachestva izob\-ra\-zhe\-niy v~zadachakh raspoznavaniya
[Methods of improvement quality of images in the recognition tasks].
\textit{Sovremennye Nauchnye Issledovaniya i~Innovatsii}
[Modern Scientific Researches and Innovations] 8:9--10.
\bibitem{7-ab-1}
\Aue{Kataeva, N.\,G., M.\,Yu. Kataev, V.\,A.~Chistyakova, and Ya.\,A.~Khamaganov}.
2012. Avtomatizirovannaya otsenka stepeni narusheniya khod'by posle insul'ta
[Automated estimation of the severity of walking disorders in patients
after stroke]. \textit{Meditsinskaya Tekhnika} [Biomedical Engineering] 1:40--43.
\bibitem{8-ab-1}
\Aue{Kataev, M.\,Yu., N.\,G. Kataeva, S.\,G.~Kataev, M.\,O.~Abramov, and
V.\,A.~Chistyakova}. 2014.
Opredelenie i~analiz dvigatel'noy aktivnosti postinsul'tnogo patsienta po
videopotoku [Definition and analysis of motion activity of afterstroke patient
from the video stream]. \textit{Byulleten' Sibirskoy Meditsiny}
[Bulletin of Siberian Medicine] 5:36--41.
\bibitem{9-ab-1}
\Aue{Abramov, M.\,O., and M.\,Yu.~Kataev}. 2014.
Vliyanie metodov predobrabotki na vosstanovlenie figury dvi\-zhu\-shche\-go\-sya cheloveka
iz potoka izobrazheniy [The effect of preprocessing methods on figure
of the moving human retrieving from an images sequence].
\textit{Dokl. TUSURa} [Proc. TUSUR] 4:65--69.
\bibitem{10-ab-1}
\Aue{Boyko, I.\,A., and R.\,A.~Gur'yanov}. 2013.
Raspoznavanie ob"ektov na osnove videosignala, poluchennogo
s~kamery, ustanovlennoy na podvizhnoy platforme
[Object recognition based on the video signal obtained from a~camera
placed on mobile platform]. \textit{Molodoy Uchenyy} [Young Scientist] 6:34--36.
\bibitem{11-ab-1}
\Aue{Ognev, I.\,V., and N.\,A.~Sidorova}. 2007. Obrabotka izob\-ra\-zhe\-niy metodami
matematicheskoy morfologii v~assotsiativnoy ostsillyatornoy srede
[Image processing by means of mathematical morphology in the associative
oscillatory medium]. \textit{Izvestiya VUZov. Povolzhskiy region.
Tekh\-ni\-che\-skie Nauki} [University Proceedings. Volga Region. Technical Sciences]
4:87--97.
\bibitem{12-ab-1}
\Aue{Burger, W., and M.\,J.~Burge}. 2007.
\textit{Digital image processing: An algorithmic introduction using Java}.
New York, NY: Springer Science \& Business Media. 566~p.
\bibitem{13-ab-1}
\Aue{Najman, L.} 2010. \textit{Mathematical morphology:
From theory to applications}. London: Wiley. 520~p.
\bibitem{14-ab-1}
\Aue{Kaew Trakulpong, P., and R.~Bowden.} 2002.
An improved adaptive background mixture model
for real-time tracking with shadow detection.
\textit{Computer
Vision and Distributed Processing:  2nd European Workshop on Advanced
Video-based Surveillance Systems, AVBS01, Proceedings}.
 Kingston: Kluwer Academic Publs. 135--144.


\end{thebibliography}

 }
 }

\end{multicols}

\vspace*{-3pt}

\hfill{\small\textit{Received January 26, 2015}}

\Contr

\noindent
\textbf{Abramov Maksim O.} (b.\ 1990)~---
PhD student, National Research Tomsk State University,
 36~Lenin Av., Tomsk 634050, Russian Federation;  maxim\_amo@mail.ru

\vspace*{3pt}

\noindent
\textbf{Kataev Mikhail Yu.} (b.\ 1961)~---
Doctor of Science in technology;
professor,  Tomsk State University of Control Systems and Radioelectronics, 
40~Lenin Av., Tomsk 634050, Russian Federation; professor, Yurga Institute of 
Technology (Branch) of National Research Tomsk Polytechnic University, 
26~Leningradskaya Str., Yurga, Kemerovo Region 652055, Russian Federation;
kataev.m@sibmail.com


\label{end\stat}


\renewcommand{\bibname}{\protect\rm Литература}