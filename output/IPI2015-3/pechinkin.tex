\def\stat{pech-raz}

\def\tit{СОВМЕСТНОЕ СТАЦИОНАРНОЕ РАСПРЕДЕЛЕНИЕ
ЧИСЛА ЗАЯВОК В~{\boldmath{$m$}}~ОЧЕРЕДЯХ В~{\boldmath{$N$}}-КАНАЛЬНОЙ СИСТЕМЕ
ОБСЛУЖИВАНИЯ С~ПЕРЕУПОРЯДОЧЕНИЕМ ЗАЯВОК$^*$}

\def\titkol{Совместное стационарное распределение
числа заявок в~$m$ очередях в~$N$-канальной системе
обслуживания} % с~переупорядочением заявок$^*$}

\def\aut{\fbox{А.\,В.~Печинкин}, Р.\,В.~Разумчик$^1$}

\def\autkol{А.\,В.~Печинкин, Р.\,В.~Разумчик}

\titel{\tit}{\aut}{\autkol}{\titkol}

{\renewcommand{\thefootnote}{\fnsymbol{footnote}} \footnotetext[1]
{Работа выполнена при поддержке РФФИ (проект 13-07-00223).}}


\renewcommand{\thefootnote}{\arabic{footnote}}
\footnotetext[1]{Институт проблем информатики Федерального исследовательского
центра <<Информатика и~управление>> Российской академии наук;
Российский
университет дружбы народов, rrazumchik@gmail.com}


\Abst{Рассматривается функционирующая в~непрерывном
времени $N$-канальная система обслуживания с~накопителем бесконечной
емкости и~переупорядочением заявок. В систему поступает
пуассоновский поток заявок, время обслуживания каждым прибором
распределено по экспоненциальному закону с~одним и~тем же
параметром. При поступлении в~систему всем заявкам  присваивается
порядковый номер. На выходе из системы сохраняется порядок между
заявками, установленный при входе в~нее. Заявки, завершившие
обслуживание и~нарушившие установленный порядок, формируют в~бункере
переупорядочения (БП) (неограниченной емкости) разные очереди. Если на
приборах находится~$n$ ($n\hm=\overline{1,N}$) заявок, то заявкой 1-го
уровня будем называть ту из них, которая в~систему поступила
последней, 2-го уровня~--- предпоследней, $\ldots,$ $n$-го уровня~--- первой. 
Находящиеся в~БП заявки,
поступившие между заявками 2-го и~1-го уровней, будем называть
заявками 1-й очереди, заявки, поступившие между заявками 3-го
и~2-го уровней,~--- заявками 2-й очереди, $\ldots,$ заявки,
поступившие между заявками $N$-го и~$(N-1)$-го уровней, --- заявками
$(N-1)$-й очереди. Найдены математические соотношения в~терминах
производящих функций (ПФ), позволяющие алгоритмически вычислять
совместное стационарное распределение числа заявок в~накопителе и~на
приборах, а~также в~1-й, 2-й, \ldots, $m$-й очередях в~БП ($m\hm=\overline{1,N-1}$).}


\KW{многолинейная система массового обслуживания;
переупорядочение; раздельные очереди; совместное стационарное распределение}

\DOI{10.14357/19922264150303}



\vskip 14pt plus 9pt minus 6pt

\thispagestyle{headings}

\begin{multicols}{2}

\label{st\stat}

\section{Введение}

Производительность многосерверных систем с~возможностью параллельной
обработки информации существенным образом может зависеть от
необходимости сохранения порядка в~потоке передаваемых сообщений.
Нарушение порядка может быть вызвано как внешними, так и~внутренними
причинами, и~в~результате складывается ситуация, когда задержка
сообщений, передаваемых через сис\-те\-му, увеличивается, поскольку
прежде, чем покинуть систему, они должны ожидать друг друга для
восстановления порядка следования (т.\,е.\ для переупорядочения). 

К~настоящему времени для изучения влияния переупорядочения на
характеристики производительности сис\-тем уже предложено много
аналитических моделей. Некоторый обзор моделей и~тео\-ре\-ти\-че\-ских
результатов, полученных до 1997~г., можно найти в~\cite{new7, new8}. 
Один из подходов к~изуче\-нию сис\-тем с~переупорядочением,
опирающийся на методы тео\-рии массового обслуживания, предполагает
представление изучаемой сис\-те\-мы в~виде сис\-те\-мы/се\-ти массового
обслуживания. На выходе из системы/сети имеется очередь (БП), 
в~которой хранятся заявки, нарушившие порядок следования,
установленный при входе. С~использованием данного подхода было
получено много результатов, касающихся распределения числа заявок 
в~системе и~БП
 при различных предположениях о входящем потоке 
 и~процессе обслуживания, распределения времен пребывания заявок
в~системе и~БП и~др. Среди наиболее интересных работ можно 
отметить~[3--8].

Настоящее исследование является продолжением работ~\cite{p1, a8}, в~которых
рассматривалась система массового обслуживания (СМО) 
с~переупорядочением в~виде марковской многоканальной
системы обслуживания неограниченной ем\-кости и~БП, также имеющим неограниченную
емкость. В~этих работах была получена система уравнений равновесия для
совместного стационарного распределения числа заявок в~системе
и~(суммарного числа заявок) в~БП,
допускающая рекуррентное решение, и~приведены некоторые результаты
численных расчетов. В~недавней работе~\cite{pr2015} было замечено,
что заявки, ожидающие в~БП в~рассматриваемой СМО,
 можно разделить на несколько групп в~зависимости от того, сколько
$\langle$обслуживаний$\rangle$ осталось прождать заявке, прежде чем она
 сможет покинуть БП (т.\,е.\ систему). Поскольку максимальное число $\langle$обслуживаний$\rangle$,
 которое может прождать заявка в~БП, равно $(N\hm-1)$, а~минимальное
 равно~1, то очередь в~БП можно представить состоящей из нескольких
отдельных очередей, общее число которых зависит от числа занятых приборов.
Если занят только один прибор, то в~БП не может ожидать ни одна заявка;
при двух занятых приборах в~БП может быть только одна очередь (из заявок,
ожидающих окончания обслуживания на приборе, который был занят самым первым).
Легко видеть, что при $N$ занятых приборах в~БП имеется $(N-1)$ очередь.
Общее число заявок в~БП есть суммарное число заявок в~отдельных очередях.
Несомненный интерес представляет нахождение
совместного стационарного распределения числа заявок в~накопителе 
и~на приборах, а~также числа заявок в~отдельных очередях в~БП.
В~\cite{pr2015} приводится система уравнений
равновесия для соответствующего совместного стационарного распределения
и упоминается, что она допускает рекуррентное решение.
В~данной работе покажем, что
традиционный подход, предполагающий нахождение
стационарных вероятностей состояний
в~терминах ПФ, применим также и~в~данном \mbox{случае}.

Статья организована следующим образом.
В~разд.~2 дается описание системы.
В~разд.~3 приводится система уравнений равновесия для совместного стационарного
распределения. Раздел~4 посвящен анализу системы равновесия
в~терминах ПФ.
Здесь показано, что ПФ можно также находить рекуррентным образом.
В~заключении сформулированы основные результаты работы.

\section{Описание системы}

Рассмотрим функционирующую в~непрерывном времени
$N$-ли\-ней\-ную ($N\hm\ge 2$) СМО с~на\-ко\-пителем
неограниченной емкости, входящим пу\-ассоновским
потоком заявок интенсивности~$\lambda$\linebreak
 и~экспоненциальным распределением времени
обслуживания заявки каждым прибором с~пара\-мет\-ром~$\mu$.


При поступлении в~систему всем заявкам  присваивается
порядковый номер.
На выходе из СМО сохраняется порядок между заявками,
установленный при входе в~нее.
Заявки, завершившие обслуживание и~нарушившие
установленный порядок, накапливаются на выходе
сис\-те\-мы в~БП и~покидают СМО только
после того, как закончится обслуживание всех заявок 
с~меньшими номерами.
Такая СМО носит название системы с~переупорядочением заявок.

Предполагается также выполнение необходимого и~достаточного условия 
${\tilde{\rho}\hm=\rho/N\hm<1}$ существования
стационарного режима функционирования СМО,
где ${\rho\hm=\lambda/\mu}$.

\section{Система уравнений равновесия для стационарных вероятностей состояний}

Пусть на приборах находится~$n$ ($n\hm=\overline{1,N}$) заявок. Тогда заявкой
1-го уровня будем называть ту из них, которая в~систему поступила
последней, второго уровня~--- предпоследней, $\ldots,$ $n$-го уровня~--- 
первой. При этом если $n\hm=N$ (все приборы заняты), то находящиеся
в~БП заявки, поступившие между заявками 2-го и~1-го уровней,
будем называть заявками 1-й очереди, заявки, поступившие между
заявками 3-го и~2-го уровней,~--- заявками 2-й очереди,
$\ldots,$ заявки, поступившие между заявками $N$-го и~$(N-1)$-го
уровней,~--- заявками $(N\hm-1)$-й очереди. Если же $n\hm<N$, то  заявками
1-й очереди будем называть заявки из БП, поступившие после заявки
1-го уровня, заявками 2-й очереди~--- заявки, поступившие
между заявками 2-го и~3-го уровней, и~т.\,д.

При $n\ge N$ обозначим через
$p_{n;i_1,\ldots,i_m}$, $m\hm=\overline{1,N-1}$, $i_1,\ldots,i_m\hm\ge 0$,
стационарную вероятность того, что в~системе на
приборах и~в накопителе находится~$n$~заявок, а~в~БП
имеется~$i_1$~заявок 1-й очереди, $i_2$ заявок
2-й очереди, $\ldots,$ $i_{m}$ заявок $m$-й очереди.
Через $p_{n;i_1,\ldots,i_m}$, $m\hm=\overline{1,n}$,
$i_1,\ldots,i_m\hm\ge 0$, обозначим аналогичную
стационарную вероятность при $n\hm=\overline{1,N-1}$.
Наконец, через~$p_0$ обозначим стационарную вероятность отсутствия заявок в~системе.

Через $p_n$, ${n\hm\ge 0}$, обозначим
стационарную веро\-ят\-ность того, что в~системе на
приборах и~в~нако\-пи\-те\-ле (без учета числа заявок в~БП) находится~$n$~заявок.
Очевидно, что стационарные вероятности~$p_n$ определяются теми же самыми формулами, что 
и~в~обычной марковской СМО $M/M/N/\infty$
(см., например,~\cite{rior}):
\begin{align}
%\label{3-1}
p_{0} &= \left( \sum\limits_{i=0}^{N-1} \fr{\rho^i}{i!}
+ \fr{\rho^N }{(N-1)!\, (N-\rho)}
\right)^{\!-1}\, ;\notag
\\
%\label{3-2}
p_{i} &= \fr{\rho^i}{i!}\, p_{0}\,, \enskip i=\overline{1,N}\,;\notag
\\
\label{3-3}
p_{i} &= \fr{\rho^i}{N!\, N^{i-N}}\, p_{0}
= \tilde{\rho}^{i-N} p_{N}\,, \enskip i\ge N+1\,.
\end{align}

Поскольку целью настоящей работы является нахождение вероятностей
в~терминах ПФ, то не будем подробно останавливаться
на системе уравнений равновесия и~лишь заметим следующее.
Для того чтобы ее выписать, сначала рассматриваются
 уравнения глобального баланса для вероятностей
$p_{n;i_1}$, $n\hm\ge 1$, $i_1 \hm\ge 0$,
затем для $p_{n;i_1,i_2}$, $n\hm\ge2$, $i_1,i_2 \hm\ge 0$,
и~т.\,д. В~последнюю очередь составляются уравнения для вероятностей
$p_{n;i_1,\ldots,i_{N-1}}$, $n\hm\ge N-1$, $i_1,\ldots,i_{N-1} \hm\ge 0$.
Произведя данные действия, можно получить следующий набор
уравнений, которые приводим без пояснений.

Для вероятностей $p_{n;i}$, $n\hm\ge N$, $i\hm \ge 0$, справедливы уравнения:
\begin{align}
\label{eq-1-1}
 p_{n;0} (\lambda+N\mu) &=
p_{n-1;0} \lambda + p_{n+1} (N-1) \mu\,,\notag\\
&\hspace*{25mm}n\ge N\,;
\\
\label{eq-1-2} 
p_{n;i} (\lambda+N\mu) &=
p_{n-1;i} \lambda + p_{n+1;i-1} \mu\,,\ \notag\\ 
&\hspace*{25mm}n\ge N\,,\ \ i \ge 1\,.
\end{align}

Соотношения для вероятностей $p_{N-1;i}$, $i \hm\ge 0$,
имеют вид:
\begin{align}
\label{eq-1-3}
p_{N-1;0} [\lambda+(N-1)\mu]
&= p_{N-2} \lambda + p_{N} (N-1)\mu \,;
\\
\label{eq-1-4}
p_{N-1;i} [\lambda+(N-1)\mu] &= p_{N;i-1} \mu \,,\ \ i \ge 1\,.
\end{align}

Наконец, для вероятностей $p_{n;i}$, $n\hm=\overline{1,N-2}$,
$i \hm\ge 0$, имеют место следующие уравнения:
\begin{align}
\label{eq-1-5}
p_{n;0} (\lambda+n\mu) &= p_{n-1} \lambda +
p_{n+1;0} n\mu \,,\notag\\
& \hspace*{15mm}n=\overline{1,N-2}\,;
\\
p_{n;i} (\lambda+n\mu) &= p_{n+1;i} n\mu +
\sum\limits_{j=0}^{i-1} p_{n+1;i-j-1,j} \mu \,,\notag\\ 
&\hspace*{15mm}n=\overline{1,N-2}\,,\ \ i \ge 1\,.
\label{eq-1-6}
\end{align}

Начиная с~вероятностей
$p_{n;i_1,i_2}$, $n\hm\ge2$, $i_1,i_2 \hm\ge 0$,
и~до вероятностей $p_{n;i_1,\ldots,i_{N-1}}$, $n\hm\ge N-1$, $i_1,\ldots,i_{N-1} \hm\ge 0$,
уравнения получаются практически одинаковыми.
Поэтому приведем их сразу в~общем виде.
Далее для сокращения записи примем соглашение, что
$\sum\limits_{i=0}^{-1} a_i \hm= 0$.
Тогда уравнения для вероятностей $p_{n;i_1,\ldots,i_{m}}$, $m\hm=\overline{2,N-1}$, 
$n \hm\ge m$, $i_1,\ldots,i_{N-1} \hm\ge 0$, можно записать в~следующем виде:
\begin{multline}
\label{bat-1}
p_{n;0,i_2,\ldots,i_m} (\lambda+N\mu) = p_{n-1;0,i_2,\ldots,i_m} \lambda+{}\\
{}+ p_{n+1;i_2,\ldots,i_m} (N-m) \mu +{}
\\
{}+ \sum\limits_{j=0}^{i_2-1} p_{n+1;j,i_2-j-1,i_3,\ldots,i_m} \mu
+ \cdots {}\\
{}\cdots + \sum\limits_{j=0}^{i_m-1} p_{n+1;i_2,\ldots,i_{m-1},j,i_m-j-1} \mu\,,\\
 n\ge N\,,\ \ i_2,\ldots,i_m \ge 0\,;
\end{multline}

\vspace*{-12pt}

\noindent
\begin{multline}
\label{bat-2}
p_{n;i_1,\ldots,i_{m}} (\lambda+N\mu) ={}\\
{}= p_{n-1;i_1,\ldots,i_{m}} \lambda
+
p_{n+1;i_1-1,i_2,\ldots,i_{m}} \mu\,,\\ 
n\ge N\,,\ \ i_1\ge 1\,,\ \ i_2,\ldots,i_{m}\ge 0\,;
\end{multline}

\vspace*{-12pt}

\noindent
\begin{multline}
\label{bat-3}
p_{N-1;0,i_2,\ldots,i_{m}} [\lambda+(N-1)\mu] =
p_{N-2;i_2,\ldots,i_{m}} \lambda+{}\\
{}+
p_{N;i_2,\ldots,i_{m}} (N-m) \mu
+{}\\
{}+
\sum\limits_{j=0}^{i_2-1} p_{N;j,i_2-j-1,i_3,\ldots,i_{m}} \mu
+ \cdots{}\\
{}\cdots + \sum\limits_{j=0}^{i_m-1} p_{N;i_2,\ldots,i_{m-1},j,i_m-j-1} \mu\,,\\ 
i_2,\ldots,i_{m} \ge 0\,;
\end{multline}

\vspace*{-12pt}

\noindent
\begin{multline}
\label{bat-4}
p_{N-1;i_1,\ldots,i_{m}} [\lambda+(N-1)\mu]
=
p_{N;i_1-1,i_2,\ldots,i_{m}} \mu\,,\\ 
i_1\ge 1\,,\ \ i_2,\ldots,i_{m}\ge 0\,,
\end{multline}

\vspace*{-12pt}

\noindent
\begin{multline}
\label{bat-5}
p_{n;0,i_2,\ldots,i_{m}} (\lambda+n\mu)
= p_{n-1;i_2,\ldots,i_{m}} \lambda
+{}\\
{}+
p_{n+1;0,i_2,\ldots,i_{m}} (n-m+1) \mu
+{}\\
{}+
\sum\limits_{j=0}^{i_2-1}
p_{n+1;0,j,i_2-j-1,i_3,\ldots,i_{m}} \mu
+ \cdots{}\\
{}\cdots + \sum\limits_{j=0}^{i_m-1} p_{n+1;0,i_2,\ldots,i_{m-1},j,i_m-j-1} \mu\,,\\ 
m\ne N-1\,, \ \ n=\overline{m,N-2}\,,\ \ i_2,\ldots,i_{m} \ge 0\,;
\end{multline}

\vspace*{-12pt}

\noindent
\begin{multline}
\label{bat-6}
p_{n;i_1,\ldots,i_{m}} (\lambda+n\mu)
= p_{n+1;i_1,\ldots,i_{m}} (n-m+1) \mu
+ {}\\
{}+\sum\limits_{j=0}^{i_1-1} p_{n+1;j,i_1-j-1,i_2,\ldots,i_{m}} \mu
+  \cdots {}\\
{}\cdots +
\sum\limits_{j=0}^{i_m-1} p_{n+1;i_1,\ldots,i_{m-1},j,i_m-j-1,} \mu\,,\ \  
m\ne N-1\,,\\
 n=\overline{m,N-2}\,,\ \ i_1 \ge 1\,,\ \ i_2,\ldots,i_{m}\ge 0\,.
\end{multline}

Как указано в~\cite{pr2015}, данную систему можно решить рекуррентным образом.
Однако традиционный подход, предполагающий нахождение
стационарных вероятностей состояний
в~терминах ПФ, применим также и~для рас\-смат\-ри\-ва\-емой системы,
что и~будет показано в~сле\-ду\-ющем разделе.
Забегая вперед, отметим, что решение получается алгоритмическим.

\section{Уравнения для~производящих функций}

Введем ПФ
\begin{multline*}
p_{n}(z_1,\ldots,z_m) = \sum\limits_{i_1,\dots , i_m=0}^{\infty}
z_1^{i_1} \cdots z_m^{i_m} p_{n;i_1,\ldots,i_m}\,,\\ 
n\ge1, \  m=\overline{1,\min(n,N-1)},  \  0<z_k \le 1, \  k=\ov{1,m};
\end{multline*}

\vspace*{-24pt}

\noindent
\begin{multline*}
P(u,z_1,\ldots,z_m) = \sum\limits_{n=N}^{\infty} u^{n-N} p_{n}(z_1,\ldots,z_m)\,, \\ 
m=\overline{1,\min(n,N-1)}\,,  \ \ 0< u \le 1\,.
\end{multline*}
Учитывая~\eqref{3-3}, положим также
$$ 
P(u) = \sum\limits_{n=N}^{\infty} u^{n-N} p_{n} =
\fr{1}{1 - \tilde{\rho} u}\, p_N \,, \ \ 0< u \le 1\,.
$$

Умножая теперь \eqref{eq-1-1} и~\eqref{eq-1-2} на $z^i$ 
и~суммируя по всем возможным значениям $i$, получаем
\begin{multline*}
%\label{eq-z-1}
(\lambda+N\mu) p_{n}(z) = {}\\
{}=\lambda p_{n-1}(z) + (N-1) \mu p_{n+1} 
+
\mu z p_{n+1}(z) \,,\ \ n\ge N\,.
\end{multline*}
Умножая последнее выражение
%\eqref{eq-z-1}
на $u^{N-n}$ и~суммируя по~$n$ от~$N$ до бесконечности, приходим к~уравнению
\begin{multline}
\label{eq-z-2}
(\lambda+N\mu) P(u,z) = \lambda p_{N-1}(z) +
\lambda u P(u,z) + {}\\
{}+\fr{(N-1) \mu}{u} \left[P(u) - p_{N}\right]
+ {}\\
{}+\fr{\mu z }{u}\, \left[P(u,z)-p_{N}(z)\right] \,.
\end{multline}
Обратимся к~уравнениям~\eqref{eq-1-3} и~\eqref{eq-1-4}.
Умножая их на~$z^i$ и~производя традиционные преобразования, получаем уравнение
\begin{multline}
\label{eq-z-3}
[\lambda+(N-1)\mu] p_{N-1}(z) ={}\\
{}= \lambda p_{N-2} +
(N-1)\mu p_{N} + \mu z p_N(z)\,.
\end{multline}
Наконец, из уравнений~\eqref{eq-1-5} и~\eqref{eq-1-6}
после домножения на~$z^i$ и~суммирования получаем соотношение
\begin{multline}
\label{eq-z-4}
(\lambda+n\mu) p_{n}(z) = \lambda p_{n-1}+ n\mu p_{n+1}(z)+
\mu z p_{n+1}(z,z) \,,\\ 
n=\overline{1,N-2}\,.
%,\ \ m=\ov{0,N-1}
%,\ \ i_1,\ldots,i_m\ge 0.
\end{multline}
Отметим, что нахождение уравнений, которым удовле\-тво\-ря\-ют ПФ $p_{n}(z_1,\ldots,z_m)$ 
и~$P(u,z_1,\ldots,z_m)$ при фиксированном~$m$, осуществляется в~4~шага:
\begin{enumerate}[(1)]
\item находится ПФ $p_{n}(z_1,\ldots,z_{m})$ при $n \hm\ge N$;
\item находится выражение для двойной $ПФ P(u,z_1,\ldots,z_m)$;
\item находится ПФ $p_{N-1}(z_1,\ldots,z_{m})$; 
\item находится ПФ $p_{n}(z_1,\ldots,z_{m})$, $n\hm=\overline{m,N-2}$.
\end{enumerate}
 Действуя подобным образом, уравнения~\eqref{bat-1}--\eqref{bat-6} в~терминах введенных ПФ
 можно привести к~виду:
 \begin{multline*}
%\label{bat-z-1}
(\lambda+N\mu) p_{n}\left(z_1,\ldots,z_{m}\right) =
\lambda p_{n-1}\left(z_1,\ldots,z_{m}\right) +{}\\
{}+
(N-m) \mu p_{n+1}(z_2,\ldots,z_{m}) +{}
\\
{}+
\mu z_1 p_{n+1}\left(z_1,\ldots,z_{m}\right) +
\mu z_2 p_{n+1}\left(z_2,z_2,\ldots,z_{m}\right) +{}\\
{}+
\mu z_3 p_{n+1}\left(z_2,z_3,z_3,\ldots,z_{m}\right) + \cdots{}\\
{}\cdots + \mu z_m p_{n+1}\left(z_2,\ldots,z_{m},z_{m}\right)\,,\ \ n\ge N\,;
\end{multline*}

\vspace*{-24pt}

\noindent
\begin{multline}
\label{bat-z-2}
(\lambda+N\mu) P\left(u,z_1,\ldots,z_{m}\right)
= \lambda p_{N-1}\left(z_1,\ldots,z_{m}\right) +{}\\
{}+
\lambda u P\left(u,z_1,\ldots,z_{m}\right) +{}
\\
+
\fr{(N-m) \mu}{u}
\left[P\left(u,z_2,\ldots,z_{m}\right) - p_{N}\left(z_2,\ldots,z_{m}\right)\right] +{}\\
{}+
\fr{\mu z_1}{u}\left[P\left(u,z_1,\ldots,z_{m}\right) - p_{N}\left(z_1,\ldots,z_{m}\right)\right]
+{}
\\
{}+ \fr{\mu z_2}{u}\left[P\left(u,z_2,z_2,\ldots,z_{m}\right)
- p_{N}\left(z_2,z_2,\ldots,z_{m}\right)\right]+{}
\\
{}+
\fr{\mu z_3}{u}\left[P(u,z_2,z_3,z_3,\ldots,z_{m}) -{}\right.\\
\left.{}-
p_{N}(z_2,z_3,z_3,\ldots,z_{m})\right]+\cdots{}
\\
{} \cdots + \fr{\mu z_m}{u}\left[P\left(u,z_2,\ldots,z_{m},z_m\right)-\right.\\
\left.{}-p_{N}\left(z_2,\ldots,z_{m},z_m\right)\right]\,,
\end{multline}

\vspace*{-12pt}

\noindent
\begin{multline}
\label{bat-z-3}
\left[\lambda+(N-1)\mu\right] p_{N-1}\left(z_1,\ldots,z_{m}\right)={}\\
{}=
\lambda p_{N-2}\left(z_2,\ldots,z_{m}\right)+{}\\
{}+(N-m) \mu p_{N}\left(z_2,\ldots,z_{m}\right)
+{}
\\
{}+ \mu z_1 p_{N}\left(z_1,\ldots,z_{m}\right) + \mu z_2 p_{N}\left(z_2,z_2,\ldots,z_{m}\right)
+{}
\\
{}+
\mu z_3 p_{N}\left(z_2,z_3,z_3,\ldots,z_{m}\right) +
\cdots{}\\
{}\cdots + \mu z_m p_{N}\left(z_2,\ldots,z_{m},z_m\right)\,;
\end{multline}

\vspace*{-12pt}

\noindent
\begin{multline}
\label{bat-z-4}
(\lambda+n\mu) p_{n}\left(z_1,\ldots,z_{m}\right)
= \lambda p_{n-1}(z_2,\ldots,z_{m}) +{}\\
{}+
(n-m+1) \mu p_{n+1}\left(z_1,\ldots,z_{m}\right)
+{}
\\
{}+
\mu z_1 p_{n+1}\left(z_1,z_1,\ldots,z_{m}\right)+{}\\
{}+ \mu z_2 p_{n+1}\left(z_1,z_2,z_2,\ldots,z_{m}\right)
+\cdots{}
\\
{} \cdots +
\mu z_m p_{n+1}\left(z_1,\ldots,z_{m},z_m\right)\,,\\ 
m \ne N-1\,, \ \ 
n=\overline{m,N-2}\,.
\end{multline}

Нахождение выражений для ПФ $P(u,z)$, $p_{N-1}(z)$, $p_{n}(z)$, 
$P(u,z_1,\ldots,z_{m})$,  $p_{N-1}(z_1,\ldots,z_{m})$, $m\hm=\overline{2,N-1}$, 
$p_{n}(z_1,\ldots,z_{m})$, $m\hm=\overline{2,N-2}$, $n\hm=\overline{m,N-2}$,
можно свести к~последовательному решению систем линейных уравнений.

Сначала находятся ПФ числа заявок в~накопителе
и суммарного числа заявок в~бункере.
Если положить $z_1\hm=\cdots=z_m\hm=z$ в~ПФ
$P(u,z_1,\ldots,z_{m})$, то функция
$P(u,z,\ldots,z)$ будет представлять собой
не что иное, как ПФ суммарного числа заявок 
в~\mbox{1-й}, \mbox{2-й}, $\ldots,$ $m$-й очередях 
и~вероятность того, что заняты все приборы (в системе
на приборах и~в очереди не менее~$N$~заявок).
Аналогично $p_n(u,z,\ldots,z)$~---
ПФ суммарного числа заявок в~\mbox{1-й}, \mbox{2-й},\
$\ldots,$ $m$-й очередях и~вероятность того, что
занято ровно~$n$~приборов.
Как показано в~\cite{p1},
при $z_1=\cdots=z_m=z$ система уравнений~\eqref{eq-z-2}--\eqref{eq-z-4}
и~\eqref{bat-z-2}--\eqref{bat-z-4} решается рекуррентным образом.

Алгоритм нахождения
$P(u,z_1,\ldots,z_{m})$, $m\hm=\overline{m,N-1}$,  $p_{N-1}(z_1,\ldots,z_{m})$, 
$n\hm=\overline{m,N-1}$, $p_{n}(z_1,\ldots,z_{m})$, $m\hm=\overline{2,N-2}$, 
$n\hm=\overline{m,N-2}$,
при фиксированном~$N$ состоит из $(N\hm-1)$ шага.
На первом шаге находятся ПФ    $p_{N}(z_1,z_{2})$, $p_{N-1}(z_1,z_2)$, $P(u,z_1,z_2)$.
На втором шаге сначала последовательно находятся ПФ:
\begin{itemize}
\item $p_{N-1}(z_1,z_2,z_2)$,  $p_{N}(z_1,z_{2},z_2)$, $P(u,z_1,z_2,z_2)$;

\item $p_{N-1}(z_1,z_1,z_2)$,  $p_{N}(z_1,z_{1},z_2)$, $p_{N-2}(z_1,z_2)$, $P(u,z_1,z_1,z_2)$;

\item $p_{N-1}(z_1,z_2,z_3)$,  $p_{N}(z_1,z_{2},z_3)$, $P(u,z_1,z_2,z_3)$.
\end{itemize}

На третьем шаге ПФ рассчитываются в~сле\-ду\-ющей последовательности:
\begin{itemize}

\item $p_{N-1}(z_1,z_2,z_2,z_2)$, $p_{N}(z_1,z_{2},z_2,z_2)$,\\
 $P(u,z_1,z_2,z_2,z_2)$;

\item $p_{N-1}(z_1,z_1,z_2,z_2)$,  $p_{N}(z_1,z_{1},z_2,z_2)$,\\
 $p_{N-2}(z_1,z_2,z_2)$, $P(u,z_1,z_1,z_2,z_2)$;

\item $p_{N-1}(z_1,z_1,z_1,z_2)$,  $p_{N}(z_1,z_{1},z_1,z_2)$,\\
 $p_{N-2}(z_1,z_1,z_2)$, $p_{N-3}(z_1,z_2)$, $P(u,z_1,z_1,z_1,z_2)$;

\item $p_{N-1}(z_1,z_2,z_2,z_3)$,  $p_{N}(z_1,z_{2},z_2,z_3)$,\\
 $P(u,z_1,z_2,z_2,z_3)$;

\item $p_{N-1}(z_1,z_2,z_3,z_3)$,  $p_{N}(z_1,z_{2},z_3,z_3)$,\\
 $P(u,z_1,z_2,z_3,z_3)$;

\item $p_{N-1}(z_1,z_1,z_2,z_3)$,  $p_{N}(z_1,z_{1},z_2,z_3)$,\\
 $p_{N-2}(z_1,z_2,z_3)$, $P(u,z_1,z_1,z_2,z_3)$;

\item $p_{N-1}(z_1,z_2,z_3,z_4)$,  $p_{N}(z_1,z_{2},z_3,z_4)$,\\
 $P(u,z_1,z_2,z_3,z_4)$.
\end{itemize}

Дальнейшие шаги, практически не отличающиеся от третьего шага, позволяют
рассчитывать все остальные ПФ, последней из которых является ПФ 
$P(u,z_1,\ldots,z_{N-1})$.
Алгоритм вычисления ПФ можно описать и~по-дру\-го\-му, если
ввести упорядочение, или отношение линейного
порядка на множестве векторов 
$$
{\vec z}_{nm}=(\underbrace{z_1,z_2,\dots, z_m}_n),\ 
n = \overline{1,N-1},\ m = \overline{1,N-1}.
$$
 Здесь $n$~--- размерность, 
а~$m$~--- порядок вектора.
Если два вектора ${\vec z}_{n_1,m}$ и~${\vec z}_{n_2,m}$ имеют порядок $n_1\hm<n_2$,
то будем писать ${\vec z}_{n_1,m} \prec {\vec z}_{n_2,m}$. Для двух
векторов  ${\vec z}_{n,m_1}$ и~${\vec z}_{n,m_2}$ одинаковой размерности
порядка $m_1\hm<m_2$ будем писать
${\vec z}_{n,m_1} \prec {\vec z}_{n,m_2}$.
Если же векторы ${\vec z}^{\,(1)}_{nm}$ и~${\vec z}^{\,(2)}_{nm}$
имеют одинаковую размерность и~порядок, то
${\vec z}^{(1)}_{nm} \prec {\vec z}^{(2)}_{nm}$,
если индексы их первых~$k$~координат
совпадают, а~индекс $(k+1)$-й координаты ${\vec z}^{\,(1)}_{nm}$
больше индекса $(k+1)$-й координаты ${\vec z}^{\,(2)}_{nm}$.
Таким образом, для заданной размерности~$n$ и~порядка~$m$
в~совокупности векторов ${\vec z}_{nm}$
имеет место следующий порядок:

\noindent
\begin{align*}
{\vec z}_{nm}&=\left(z_1,z_2,\dots,z_{m-2}, z_{m-1}, z_m, z_m, \dots, z_m, z_m\right)\,;
\\
{\vec z}_{nm}&=\\
&\hspace*{-3.5mm}{}=\left(z_1,z_2,\dots,z_{m-2}, z_{m-1}, z_{m-1}, z_m, \dots, z_m, z_m\right)\,;
\\
{\vec z}_{nm}&=\\
&\hspace*{-7mm}{}=\left(z_1,z_2,\dots,z_{m-2}, z_{m-1}, z_{m-1}, z_{m-1}, \dots, z_m, z_m\right)\,;
\\
\dots&  \dots \dots\dots\dots\dots\dots\dots\dots\dots\dots\dots\dots\dots\dots
\\
{\vec z}_{nm}&=\\
&\hspace*{-3.5mm}{}=\left(z_1,z_2,\dots,z_{m-2}, z_{m-2}, z_{m-1}, z_m, \dots, z_m, z_m\right)\,;
\\
\dots&  \dots \dots\dots\dots\dots\dots\dots\dots\dots\dots\dots\dots\dots\dots
\\
{\vec z}_{nm}&=\left(z_1,z_1,\dots,z_{1}, z_{2}, z_3, \dots, z_{m-1}, z_m\right)\,.
\end{align*}
Теперь ПФ 
\begin{multline*}
P\left(u,z_1,\ldots,z_{m}\right)\hm=P(u,{\vec z}_{nm})\,,\\
n=\overline{m,N-1}\,, 
\enskip
m=\overline{m,N-1}\,,
\end{multline*}
можно вычислить последовательно от минимального вектора ${\vec z}_{11}$ к~максимальному
${\vec z}_{nn}$ по порядку ${\vec z}_{11} \prec {\vec z}_{21} \prec {\vec z}_{22} 
\prec {\vec z}_{31} \prec \dots \prec {\vec z}_{nn}$. Тогда на каж\-дом этапе вычислений 
будут использоваться уже известные функции.

\section{Заключение}

В настоящей работе рассмотрена функционирующая в~непрерывном времени $N$-ка\-наль\-ная 
сис\-те\-ма обслуживания с~накопителем бесконечной емкости и~переупорядочением заявок.
В~систему\linebreak поступа\-ет пуассоновский поток заявок, время
обслуживания каждым прибором распределено по экспоненциальному закону с~одним и~тем же параметром.
Заявки, завершившие обслуживание и~нарушившие
установленный порядок, формируют в~БП
(неограниченной емкости) очереди разных уровней.
В~работе показано, что совместное стационарное распределение
общего числа заявок в~накопителе и~на приборах, а~также числа заявок
в~1-й, 2-й, $\ldots,$ $m$-й очередях в~БП
можно найти в~терминах ПФ, причем
решение получается алгоритмическим.

Несомненный интерес представляет обобщение уже полученных результатов
на случай неоднородных приборов, что может стать предметом дальнейших исследований.

{\small\frenchspacing
 {%\baselineskip=10.8pt
 \addcontentsline{toc}{section}{References}
 \begin{thebibliography}{99}

\bibitem{new7}
\Au{Boxma O., Koole G., Liu Z.} 
Queueing-theoretic solution methods for models of parallel and distributed systems~// 
Performance evaluation of parallel and distributed systems: Solution methods.~--- 
CWI tract ser.~---  Torino, Italy: CWI, 1994.
Vol.~105-106. P.~1--24. {\sf http://oai.cwi.nl/oai/asset/1479/1479a.pdf}.
\bibitem{new8}
\Au{Dimitrov B.\,D., Green Jr.\,D., Rykov~V.\,V., Stanchev~P.\,L.}
On performance evaluation and optimization problems in queues with resequencing~//
Advances in stochastic modelling~/
Eds. J.\,R.~Artalejo, A.~Krishnamoorthy.~---
NJ, USA: Notable Publications Inc.,  2002. P.~55--72.


\bibitem{new11}
\Au{Gogate N.\,R., Panwar S.\,S.} 
Assigning customers to two parallel servers with resequencing~//
IEEE Commun. Lett., 1999. Vol.~3. No.\,4. P.~119--121.

\bibitem{new5} %4
\Au{Huisman T., Boucherie R.\,J.} 
The sojourn time distribution in an infinite server resequencing queue with dependent interarrival and service times~//
J.~Appl. Probab., 2002. Vol.~39. No.\,3. P.~590--603.


\bibitem{new9} %5
\Au{Lelarge M.}
Packet reordering in networks with heavy-tailed delays~//
Math. Method. Oper. Res., 2008. Vol.~67. P.~341--371.

\bibitem{cpr23} %6
\Au{Xia~Y., Tse D.\,N.\,C.}
On the large deviations of resequencing
queue size: 2-$M/M/1$ case~// IEEE Trans. Inform. Theory, 2008.
Vol.~54. No.\,9. P.~4107--4118.

\bibitem{cpr13-1} %7
\Au{Leung K., Li V.\,O.\,K.}
A~resequencing model for high-speed packet-switching networks~//
Comput. Commun., 2010. Vol.~33. Iss.~4. P.~443--453.

\bibitem{cpr22} %8
\Au{Матюшенко С.\,И.} 
Стационарные характеристики двухканальной системы обслуживания 
с~переупорядочиванием заявок и~распределениями фазового типа~// 
Информатика и~её применения, 2010. Т.~4. Вып.~4. С.~67--71.


\bibitem{p1} %9
\Au{Печинкин А.\,В., Разумчик~Р.\,В.}
Совместное стационарное распределение числа заявок в~накопителе 
и~в~бункере переупорядочения в~многоканальной сис\-те\-ме обслуживания 
с~переупорядочением заявок~// Информатика и~её применения, 2014. Т.~8. Вып.~4. С.~3--10.

\bibitem{a8} %10
\Au{Pechinkin A.\,V., Caraccio~I., Razumchik~R.\,V.}
On joint stationary distribution in exponential
multiserver reordering queue~// 12th  Conference (International) on
Numerical Analysis and Applied Mathematics Proceedings, 2015. 
Vol.~1648. 250003. 4~p.

\bibitem{pr2015} %11
\Au{Pechinkin A.\,V., Razumchik~R.\,V.} Joint stationary distribution
of queues in multiserver resequencing queue~// 
18th  Conference (International) on Distributed Computer and
Communication Networks: Control, Computation, Communications Proceedings, 2015
(in print).

\bibitem{rior}
\Au{Riordan J.} \textit{Stochastic service systems.}~--- 
New York, NY, USA: Wiley, 1962. 139~p.
 \end{thebibliography}

 }
 }

\end{multicols}

\vspace*{-3pt}

\hfill{\small\textit{Поступила в~редакцию 1.07.15}}

%\newpage

\vspace*{12pt}

\hrule

\vspace*{2pt}

\hrule

%\vspace*{12pt}

\def\tit{JOINT STATIONARY DISTRIBUTION OF~{\boldmath{$m$}}~QUEUES 
 IN~THE~{\boldmath{$N$}}-SERVER QUEUEING SYSTEM WITH~REORDERING}

\def\titkol{Joint stationary distribution of~$m$~queues 
 in~the~$N$-server queueing system with~reordering}

\def\aut{\fbox{A.\,V.~Pechinkin} and R.\,V.~Razumchik$^{1,2}$}

\def\autkol{A.\,V.~Pechinkin and R.\,V.~Razumchik}

\titel{\tit}{\aut}{\autkol}{\titkol}

\vspace*{-9pt}


\noindent
$^1$Institute of Informatics Problems,
Federal Research Center ``Computer Science and Control'' of
the Russian\linebreak
$\hphantom{^1}$Academy of Sciences, 44-2 Vavilov Str.,
Moscow 119333, Russian Federation 

\noindent
$^2$Peoples' Friendship University
of Russia, 6 Miklukho-Maklaya Str., Moscow 117198, Russian Federation


\def\leftfootline{\small{\textbf{\thepage}
\hfill INFORMATIKA I EE PRIMENENIYA~--- INFORMATICS AND
APPLICATIONS\ \ \ 2015\ \ \ volume~9\ \ \ issue\ 3}
}%
 \def\rightfootline{\small{INFORMATIKA I EE PRIMENENIYA~---
INFORMATICS AND APPLICATIONS\ \ \ 2015\ \ \ volume~9\ \ \ issue\ 3
\hfill \textbf{\thepage}}}

\vspace*{3pt}


\Abste{The paper considers a continuous-time $N$-server queueing system with 
a~buffer of infinite capacity and customer reordering. The Poisson flow of customers 
arrives at the system. Service times of customers at each server are exponentially 
distributed with the same parameter. Each customer obtains a~sequential number upon 
arrival. The order of customers upon arrival should be preserved upon departure 
from the system. Customers which violated the order form different queues in the 
reordering buffer which has infinite capacity. If there are $n$, $n=\overline{1,N}$, 
customers in servers, then the latest customer to occupy a server  is called the 1st 
level customer, the last but one~--- the 2nd  level customer,\ \ldots , 
the first~--- the $n$th level customer. Customers in the reordering buffer that 
arrived between the 1st  level and the 2nd level customers, form the queue number one. 
Customers, which arrived between the 2nd level and the 3rd level customers, form the 
queue number two, etc. Customers, which arrived between the $N$th level and the $(N-1)$th 
level customers, form the queue number $(N-1)$ in the reordering buffer. Mathematical 
relations in terms of Z-transform, which make it possible to calculate the joint 
stationary distribution of the number of customers in the buffer, servers, and in 
the 1st, 2nd,\ \ldots, $m$th queues ($m=\overline{1,N-1}$) 
in the reordering buffer, are obtained.}


\KWE{multiserver queueing system; reordering; separate queues; joint stationary distribution}

\DOI{10.14357/19922264150303}

\Ack
\noindent
The work was supported by the Russian Foundation for Basic Research (project 13-07-00223).



%\vspace*{3pt}

  \begin{multicols}{2}

\renewcommand{\bibname}{\protect\rmfamily References}
%\renewcommand{\bibname}{\large\protect\rm References}

{\small\frenchspacing
 {%\baselineskip=10.8pt
 \addcontentsline{toc}{section}{References}
 \begin{thebibliography}{99}


\bibitem{new7-1}
\Aue{Boxma, O., G.~Koole, and Z.~Liu.} 1994. 
Queueing-theoretic solution methods for models of parallel and distributed systems.
\textit{Performance evaluation of parallel and distributed systems: Solution methods}. 
CWI tract ser. Torino, Italy: CWI. 105-106:1--24. Available at: 
{\sf http://oai.cwi.nl/oai/asset/1479/1479a.pdf} (accessed August~7, 2015).


\bibitem{new8-1}
\Aue{Dimitrov, B.\,D., D.~Green, Jr., V.\,V.~Rykov, and P.\,L.~Stanchev.} 2002.
On performance evaluation and optimization problems in queues with resequencing.
\textit{Advances in stochastic modelling}. Eds. J.\,R.~Artalejo
and A.~Krishnamoorthy.
 NJ: Notable Publications Inc. 55--72.

\bibitem{new11-1}
\Aue{Gogate, N.\,R., and S.\,S.~Panwar.} 1999. 
Assigning customers to two parallel servers with resequencing.
\textit{IEEE Commun. Lett.} 3(4):119--121.

\bibitem{new5-1}
\Aue{Huisman, T., and R.\,J.~Boucherie.} 2002. 
The sojourn time distribution in an infinite server resequencing queue with dependent interarrival and service times.
\textit{J.~Appl. Probab}. 39(3):590--603.



\bibitem{new9-1} %5
\Aue{Lelarge, M.} 2008.
Packet reordering in networks with heavy-tailed delays.
\textit{Math. Method. Oper. Res.} (67):341--371.

\bibitem{cpr23-1} %6
\Aue{Xia,~Y., and D.N.C. Tse} 2008.
On the large deviations of resequencing
queue size: 2-$M/M/1$ case. \textit{IEEE Trans. Information Theory} 54(9):4107--4118.


\bibitem{cpr13-1-1} %7
\Aue{Leung, K., and V.\,O.\,K.~Li.} 2010.
A~resequencing model for high-speed packet-switching networks.
\textit{Comput. Commun.} 33(4):443--453.


\bibitem{cpr22-1} %8
\Aue{Matyushenko, S.\,I.} 2010.
Statsionarnye kha\-rak\-te\-ri\-sti\-ki dvukhkanal'noy sistemy obsluzhivaniya 
s~pe\-re\-upo\-rya\-do\-chi\-va\-ni\-em zayavok i~raspredeleniyami fazovogo tipa 
[Stationary characteristics of the two-channel queueing system with
reordering customers and distributions of phase type]. 
\textit{Informatika i~ee Primeneniya}~--- \textit{Inform. Appl}. 4(4):67--71.

\bibitem{p1-1} %9
\Aue{Pechinkin, A.\,V., and R.\,V.~Razumchik.} 2014.
Sovmestnoe statsionarnoe raspredelenie chisla zayavok v~nakopitele i~v~bunkere 
pereuporyadocheniya v~mnogokanal'noy sisteme obsluzhivaniya s~pereuporyadocheniem 
zayavok [Joint stationary distribution of the number of customers in the system and 
reordering buffer in the multiserver reordering queue].
\textit{Informatika i~ee Primeneniya}~--- \textit{Inform. Appl}. 8(4):3--10.

\bibitem{a8-1} %10
\Aue{Pechinkin, A.\,V., I. Caraccio, and R.\,V.~Razumchik.} 2015.
On joint stationary distribution in exponential
multiserver reordering queue
\textit{12th  Conference (International) on
Numerical Analysis and Applied Mathematics Proceedings}. 1648:250003. 4~p.


\bibitem{pr2015-1} %11
\Aue{Pechinkin, A.\,V., and R.\,V.~Razumchik.} 2015 (in print).
Joint stationary distribution of queues in multiserver resequencing
queue. \textit{18th  Conference (International) on Distributed Computer and 
Communication Networks: Control, Computation, Communications Proceedings}. Moscow. 

\bibitem{rior-1}
\Aue{Riordan, J.} 1962.
\textit{Stochastic service systems}. New York, NY: Wiley. 139~p.
\end{thebibliography}

 }
 }

\end{multicols}

\vspace*{-3pt}

\hfill{\small\textit{Received July 1, 2015}}


\Contr

\noindent
\textbf{Pechinkin Alexander V.} (1946--2014)~--- professor, Doctor
of Science in physics and mathematics

\vspace*{3pt}

\noindent
\textbf{Razumchik Rostislav V.} (b.\ 1984) --- Candidate
of Science (PhD) in physics and mathematics,
senior scientist, Institute of Informatics Problems, 
Federal Research Center ``Computer Science and Control'' of the Russian Academy of 
Sciences,
44-2 Vavilov Str.,
Moscow 119333, Russian Federation; associate professor,
Peoples' Friendship University of Russia,
6 Miklukho-Maklaya Str., Moscow 117198, Russian Federation; rrazumchik@ieee.org


\label{end\stat}


\renewcommand{\bibname}{\protect\rm Литература}