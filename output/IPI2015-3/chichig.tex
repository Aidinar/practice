  \newcommand{\Do}{\noindent\mbox{Д\,о\,к\,а\,з\,а\,т\,е\,л\,ь\,с\,т\,в\,о\,.\ \ }}


 \newcommand{\Np}{\mathbb{N}}
 \newcommand{\Vp}{{\sf V}}
 \newcommand{\Ep}{{\sf E}}
 \newcommand{\Rp}{\mathbb{R}}
 \newcommand{\Op}{\mathbf{O}}
 \newcommand{\op}{\mathbf{o}}



\def\stat{chich}

\def\tit{АСИМПТОТИЧЕСКИЕ РАЗЛОЖЕНИЯ ВЫСОКОГО ПОРЯДКА ДЛЯ~НЕСМЕЩЕННЫХ ОЦЕНОК
 И~ИХ ДИСПЕРСИЙ В~МОДЕЛИ ОДНОПАРАМЕТРИЧЕСКОГО ЭКСПОНЕНЦИАЛЬНОГО
 СЕМЕЙСТВА$^*$}

\def\titkol{Асимптотические разложения высокого порядка для несмещенных оценок
 и~их дисперсий} % в~модели однопараметрического экспоненциального  семейства}

\def\aut{В.\,В.~Чичагов$^1$}

\def\autkol{В.\,В.~Чичагов}

\titel{\tit}{\aut}{\autkol}{\titkol}

{\renewcommand{\thefootnote}{\fnsymbol{footnote}} \footnotetext[1]
{Работа выполнена при финансовой поддержке
 Министерства образования и~науки РФ, проект №\,2096.}}


\renewcommand{\thefootnote}{\arabic{footnote}}
\footnotetext[1]{Пермский государственный национальный исследовательский университет,  chichagov@psu.ru}

%\vspace*{-12pt}

\Abst{Рассмотрена модель повторной выборки
объема~$n$ из распределения, принадлежащего естественному
однопараметрическому экспоненциальному семейству. При неограниченном
возрастании объема выборки изучено предельное поведение несмещенной
оценки c~равномерно минимальной дисперсией (НОРМД) заданной
параметрической функции и~НОРМД дисперсии этой оценки. Получены
асимптотические разложения высокого порядка как для функций,
определяющих несмещенные оценки, так и~для дисперсий этих оценок.
Результаты представлены как в~канонической, так и~в~mean-па\-ра\-мет\-ри\-зации.}

\KW{естественное экспоненциальное семейство;
несмещенная оценка; асимптотическое разложение}

\DOI{10.14357/19922264150308}

\vspace*{-6pt}

\vskip 12pt plus 9pt minus 6pt

\thispagestyle{headings}

\begin{multicols}{2}

\label{st\stat}

\section{Введение}

  Несмещенные оценки широко применяются при решении задач теории надежности
  и~ста\-ти\-стического контроля качества, а~также при построении моделей массового обслуживания.
  Проб\-ле\-ма построения точечных несмещенных оценок в~значительной степени
 себя исчерпала.  Воиновым и~Никулиным в~[1]
 описаны основные методы построения несмещенных оценок, а~также пред\-став\-ле\-ны
 в~виде таблиц несмещенные оценки и~функции от них для многих вероятностных законов.

  Однако изучение предельного поведения несмещенных оценок и~их характеристик
 применительно к~естественным экспоненциальным семействам распределений
 далеко не закончено.
Portnoy в~[2] установил асимптотическую нормальность и~асимптотическую эффективность несмещенной оценки заданной параметрической
 функции $g[\theta]$ c~каноническим параметром $\theta\hm\in \Rp^p$
 при $g'[\theta]\hm\neq 0$.

  Случай $g'[\theta]\hm=0$ рассмотрели Lopez-Blazquez и~Salamanca-Mino
  в~[3] для скалярного параметра~$\theta$,
 а~Blazquez и~Rubio в~[4]~--- случай многомерного параметра~$\theta.$
%
  Решение проблемы получено с~использованием счетной системы ассоциированных
  с~экспоненциальным семейством ортогональных полиномов, которые предложил
 Morris~[5].
  Задачу сравнения несмещенной оценки и~оценки максимального правдоподобия с~помощью
 асимптотических разложений их среднеквадратических ошибок изучали Hwang
 и~Hu~[6].

  Совершенно иной подход к~изучению предельного поведения несмещенных оценок
 предложен в~работах автора~[7, 8].
  Он основывается на применении локальных предельных теорем и~разложения
  Эджворта~\cite{10-ch, 9-ch} к~плотности условного распределения, определяющей несмещенную оценку плотности
 заданного распределения~\cite{1-ch, 11-ch}.
  На основе этого подхода в~\cite{12-ch} получены стохастические разложения
 как для НОРМД заданной однопа\-ра\-мет\-ри\-че\-ской функции, так 
 и~для НОРМД дисперсии этой оценки. Другие возможности изложенного подхода 
 представлены в~работах~\cite{13-ch, 14-ch}.

  В данной работе, в~отличие от~\cite{12-ch}, для функций, определяющих НОРМД
  плотности распределения и~заданной параметрической функции, найде\-ны
  разложения более высокого порядка. Это позволило получить асимптотические разложения дисперсий несмещенных
 оценок для более широкого, чем в~\cite{6-ch}, класса оценок,
 не требуя при этом существования разложений несмещенных оценок
 в~сходящиеся ряды Тейлора.
 
 \vspace*{-9pt}

\section{Основная модель наблюдений и~ее параметризации}

\vspace*{-3pt}

   Имеется  $\vec{X}=(X_1 ,\ldots ,X_n) $~--- повторная выборка, элементы которой имеют то же распределение, что и~наблюдаемая случайная величина~$\xi$.
   Распределение вероятностей случайной величины~$\xi$ принадлежит естественному экспоненциальному семейству распределений~$\mathcal{M}$ (см., например,~\cite{15-ch}), которое параметризуется либо с~по\-мощью канонического параметра,  либо с~помощью параметра <<среднее>>.
   В~зависимости от типа параметра распределения~$\xi$ будем различать
   2~формы параметризации семейства~$\mathcal{M}$.

  \textbf{Каноническая параметризация.} Естественное однопараметрическое
  экспоненциальное семейство~$\mathcal{M}$ с~каноническим параметром
  $\theta  \hm\in \Theta \hm=\textbf{Int} [\widetilde{\Theta}]\subseteq \Rp$
  определяется плотностью распределения
 \begin{multline}
 f[x;\theta] =  \exp \left\{\theta T[x] - \kappa[\theta]+d[x]\right\}\,,\\
x\in {\mathbb X}_G \subseteq \Rp\,,
 \label{2:1}
 \end{multline}
  относительно меры $\mu[x]$. Здесь ${\mathbb X}_G$~--- 
  носитель распределения~(\ref{2:1}); $d[x]$ 
  и~$T[x]$~--- известные борелевские функции;
  $\kappa[\theta]$~--- кумулянтное преобразование данного распределения
  (см., например,~\cite[c.~13]{16-ch});
  $\mu[x]$~--- мера Лебега, если случайная величина~$\xi$
  имеет абсолютно непрерывное распределение, или считающая мера,
  если~$\xi$  имеет решетчатое распределение. Функция $\kappa[\theta]$
  в~соответствии с~теоремой~2.2 из~\cite{16-ch} имеет производную любого порядка.

 Как обычно, $\widetilde{\Theta}$~--- множество  значений~$\theta,$ при которых
 \begin{equation*}
 \int\limits \exp \left\{\theta  T[x]+d[x]\right\}\,d\mu[x]<\infty\,,
 \end{equation*}
 называемое естественным параметрическим множеством семейства~$\mathcal{M}$.

\textbf{Mean-параметризация.}
  Величину $a \hm= \Ep T[\xi]\hm=\kappa'[\theta]:\,\Theta\hm\rightarrow \mathbb {A}\hm\in \Rp$
  называют параметром <<среднего значения>>~\cite{15-ch, 16-ch} семейства~$\mathcal{M}$.
    В~случае естественного экспоненциального семейства
  дисперсия $\Vp T[\xi]\hm=\kappa''[\theta]\hm>0$ при любом $\theta\hm\in\Theta$,
  а~потому между параметрами~$\theta$ и~$a$ существует взаимно однозначное
  соответствие. Пусть $\theta\hm=\Phi_1[a]$~--- функция, обратная
  к~$a \hm= \kappa'[\theta]$.
  Обе функции $\theta\hm=\Phi_1[a]$ и~$a \hm= \kappa'[\theta]$ 
  бесконечное число раз дифферен\-ци\-ру\-емы, при этом $\Phi'_1[a] \hm> 0$ для любого $a \hm\in
  \mathbb{A}$.

   Функция плотности~\eqref{2:1} при mean-па\-ра\-мет\-ри\-за\-ции принимает вид:
 \begin{multline}
  f[x;\Phi_1[a]] =
   \exp \left\{ \Phi_1[a]  T[x]
  - \kappa[\Phi_1[a]] + d[x ]\right\}\,,
\\ x\in \mathbb {X}_G \subset \Rp\,.
 \label{2:2}
 \end{multline}


\section{Основные предположения и~обозначения}

  \begin{description}
  \item[$\mathbf{(A_1)}$] Распределение случайной величины~$\xi$
  принадлежит естественному экспоненциальному семейству, определяемому
  выражениями~\eqref{2:1} и~\eqref{2:2}.
  \item[$\mathbf{(A_2)}$]
 Если $\mu[x]$~--- мера Лебега, то для каждого $\theta\hm\in \Theta$ существует
 $n_0\hm\in \mathbb{N}$, при котором случайная величина
  $$
  Z_n=\fr{S_n-n a}{b\sqrt{n}}\,
  $$
  где
  $$
  S_n=\sum\limits^n_{i=1}T[X_i]\,; \ a=\Ep T[\xi]\,;\ b=\sqrt{\Vp T[\xi]}\,,
  $$
 имеет ограниченную плотность $f_{Z_n}[x;\theta]$.
 Если $\mu[x]$~--- считающая мера, то носитель распределения 
 $\mathbb{X}_G\subset \mathbb{Z}$, но не содержится ни в~какой подрешетке решетки~$\mathbf{Z}$.
 \end{description}


  Введем обозначения:
\begin{description}
\item[\,]
 $g(\theta)$, $\theta\in \Theta$ ($G[a]\hm=g[\Phi_1[a]]$, $a\hm\in \mathbb {A}$, при 
 mean-па\-ра\-мет\-ри\-за\-ции)~--- заданная параметрическая функция;
\item[\,]
 $\widehat{g}[\theta]=\widehat{g}_n[\theta|S_n]$
 ($\widehat{G}[a]\hm=\widehat{G}_n[a|S_n]$)~--- несмещенная оценка функции
 $g(\theta)$ ($G[a]$) по достаточной статистике~$S_n$ (отметим, что $\widehat{g}[\theta]\hm\equiv \widehat{G}[a]$);
\item[\,]
 $\widehat{g}_n[\theta;Z_n]$ ($\widehat{G}_n[a;Z_n]$)~--- несмещенная оценка
 $\widehat{g}_n[\theta|S_n]$ ($\widehat{G}_n[a|S_n]$) как функция от нормированной суммы~$Z_n$;
\item[\,]
 $h'[z]$,  $h''[z]$, $h^{(j)}[z]$~--- соответственно производные 1-го, 2-го и~$j$-го порядков функции~$h[z]$;
\item[\,]
 $\Np_0=\mathbb{N} \cup \{0\}$~--- расширенное множество натуральных чисел;
\item[\,]
 $\kappa_j =\kappa^{(j)} [\theta]\hm=\kappa^{(j)} [\Phi_1[a]]$~--- кумулянт $j$-го порядка распределения $T[\xi]$;
\item[\,]
 $\rho_j =\kappa_j/b^j$~--- нормированный кумулянт $j$-го порядка распределения $T[\xi]$;
\item[\,]
 $\varphi[z]=({1}/{\sqrt{2\pi}}) e^{-z^2/2}$~--- плотность стандартного нормального распределения;
\item[\,]
 $H_j[z]$~--- полином Че\-бы\-шё\-ва--Эр\-ми\-та $j$-го по\-рядка;
\item[\,]
 $H_2[z]=z^2-1$; 
 \item[\,]
 $H_3[z]\hm=z^3-3z$;
\item[\,]
 $I[A]$~--- индикатор события~$A$.
 \end{description}

\section{Интегральное представление несмещенных оценок и~несмещенно оцениваемых параметрических функций}

 В условиях $\bf (A_1)$ и~$\bf(A_2)$ сумма~$S_n$ является полной достаточной статистикой
 для параметра $\theta\hm\in \Theta$ ($a\hm\in \mathbb {A}$). При этом если несмещенная оценка заданной параметрической функции является функцией от полной статистики, то она имеет равномерно минимальную дисперсию. В~дальнейшем речь будет идти только о~несмещенных оценках с~равномерно минимальной дисперсией.
 Отметим также, что носитель распределения~$S_n$ не зависит от параметра~$\theta$
 (см.\ упражнение~2.11 в~\cite{15-ch}).

 Пусть несмещенная оценка $\widehat{g}_n\left[\theta|S_n\right]$
 функции  $g(\theta)$, $\theta\hm\in \Theta$, при $n\hm\ge L_0$
 удовлетворяет уравнению несмещенности
  \begin{equation}
  g[\theta]=\int\limits_{\mathbb{R}} \widehat{g}_n[\theta|s] f_{S_n}[s;\theta] \,d\mu[s].
   \label{4:1}
  \end{equation}
 Здесь и~далее $L_0$~--- минимальное значение~$n$, при котором выполняется~\eqref{4:1},
 $f_{S_n}[s;\theta]$~--- плотность распределения статистики~$S_n$ относительно
 меры $\mu[\cdot]$.

 В условиях $\bf (A_1)$ и~$\bf(A_2)$ функция $f_{S_n}[s;\theta]$ имеет вид:
  \begin{equation*}
   f_{S_n}[s;\theta]=\exp\left\{\theta s-n\kappa[\theta]+d_n[s]\right\},
%   \label{4:2}
  \end{equation*}
 где $d_n[s]$~--- некоторая борелевская функция.

 Если использовать $\widehat{g}_{L}[\theta|S_L]$~--- несмещенную оценку функции
 $g(\theta)$ по первым~$L$~элементам выборки~$\vec{X}$,
 то уравнение несмещенности~\eqref{4:1} примет вид:
   \begin{equation}
  g[\theta]=\int\limits_{\mathbb{R}} \widehat{g}_{L}[\theta|t] f_{S_L}[t;\theta]\, d\mu[t]\,,\enskip   L_0\le L <n\,.
   \label{4:3}
   \end{equation}

 При сделанных предположениях НОРМД функции $g(\theta)$ определяется
 следующими выражениями~\cite{1-ch, 4-ch}:
  \begin{align}
  \widehat{g}_n\left[\theta|S_n\right]&=\int\limits_{\mathbb{R}}
   \widehat{g}_{L}[\theta|t] \widehat{f}_{S_L}[t;\theta|S_n]\, d\mu[t]\,;
   \label{4:4}\\
     \widehat{f}_{S_L}[t|S_n]&=\fr{f_{S_L}[t;\theta_0]
    f_{S_{n-L}}[S_n-t;\theta_0]} {f_{S_n}[S_n;\theta_0]}\,,\notag
%   \label{4:5}
  \end{align}
  где $\widehat{f}_{S_L}[t;\theta|S_n]$~--- несмещенная оценка плотности
  $f_{S_L}[t;\theta]$; $\theta_0$~--- произвольное значение параметра~$\theta$
  из области~$\Theta$ (будем полагать $\theta_0\hm=\theta$).

 Отметим, что значение оценки~\eqref{4:4} в~силу полноты  статистики~$S_n$ не зависит от выбора значения~$L$.

\section{Вспомогательные результаты}


\noindent
\textbf{Лемма 5.1.}\
\textit{Пусть выполнены условия $\mathbf{(A_1)}$ и~$\mathbf{(A_2)}$. Тогда
 центральные моменты $\mu_\ell\hm=\mu_\ell[a]\hm=\Ep(T[\xi]-a)^\ell$, 
 $\ell\hm = 0,1,\ldots,5$, определяются выражениями}:
 \begin{gather*}
\mu_0[a]=1\,;\
 \mu_1[a]=0\,;\
 \mu_2[a]=\fr{1}{\Phi_1'[a]}\,;\\
 \mu_3[a]= -\fr{\Phi_1''[a]}{(\Phi_1'[a])^3}\,;
\\
\mu_4[a]= \fr{3 (\Phi_1'[a])^3+3 (\Phi_1''[a])^2
         -\Phi_1'[a] \Phi_1^{(3)}[a]}{(\Phi_1'[a])^5}\,;
\\
\mu_5[a]= -\left(\vphantom{\Phi_1^{(4)}}
10 (\Phi_1'[a])^3 \Phi_1''[a]+15 (\Phi_1''[a])^3
         -\right.\\
         \left.{}-10 \Phi_1'[a] \Phi_1''[a] \Phi_1^{(3)}[a]
         +(\Phi_1'[a])^2 \Phi_1^{(4)}[a]\right)\Big /
         (\Phi_1'[a])^7.
 \end{gather*}

  Для доказательства леммы~5.1 достаточно воспользоваться полученным в~работе~\cite{6-ch} рекуррентным соотношением
 \begin{equation*}
 \mu_{\ell+1}[a]=\fr{\ell\mu_{\ell-1}[a]+\mu_\ell'[a]}{\Phi_1'[a]}.
 %\label{L:5.1-1}
  \end{equation*}

\noindent
\textbf{Лемма 5.2.}\
\textit{ Первые 5 кумулянтов и~моментов связаны соотношениями}:
 \begin{gather*}
 \kappa_1=a\,;\ \kappa_2 = b^2 \,;\ \kappa_3 = \mu _3\,;\\
 \kappa_4 = \mu_4  - 3\mu_2^2\,;\enskip  \kappa_5  = \mu_5 - 10\mu_3 \mu_2\,.
 \end{gather*}


\noindent
\textbf{Лемма 5.3.}\
\textit{Пусть выполнены условия $\mathbf{(A_1)}$ и~$\mathbf{(A_2)}$. Тогда
 нормированные кумулянты $\rho_\ell$, $\ell \hm= 3,4,5$, имеют вид}
 \begin{gather*}
 \rho_3= -\fr{\Phi_1''[a]}{(\Phi_1'[a])^{3/2}}\,; \enskip
 \rho_4= \fr{3 (\Phi_1''[a])^2 -\Phi_1'[a]\Phi_1^{(3)}[a]}{(\Phi_1'[a])^3}\,; \\
   \rho_5= -\fr{15 (\Phi_1''[a])^3}{(\Phi_1'[a])^{9/2}}
 +\fr{10 \Phi_1''[a] \Phi_1^{(3)}[a]}{(\Phi_1'[a])^{7/2}}
 -\fr{\Phi_1^{(4)}[a]}{(\Phi_1'[a])^{5/2}}.
 \end{gather*}

 \noindent
 \textbf{Определение.} Назовем взвешенным моментом
 $j$-го порядка центрированной суммы~$S_L$ величину
 \begin{equation}
 \nu_j[\theta;L]=\Ep\left( \left( S_L - La\right)^j \widehat{g}_{L}[\theta|S_L]
 \right),\enskip j\in\mathbb{N}_0\,.
 \label{5.1}
 \end{equation}

\noindent
\textbf{Теорема 5.1.}\
\textit{Пусть выполнены условия $\mathbf{(A_1)}$ и~$\mathbf{(A_2)}$,
 $\Vp \widehat{g}_{L}[\theta|S_L]\hm<\infty$, $L_0\hm\le L\hm< n$.
 Тогда последовательность взвешенных моментов
  $\left\{\nu_j[\theta;L], j\hm\in\mathbb{N}_0\right\}$ удовлетворяет соотношениям}:
 \begin{equation}
 \left.
 \begin{array}{rl}
  \nu_0[\theta;L] &=g[\theta]\,;\enskip \nu_1[\theta;L]=g'[\theta]\,;
\\[6pt]
  \nu_{j+1}[\theta;L]&= \nu'_{j}[\theta;L]+jLb^2\nu_{j-1}[\theta;L]\,,\\[6pt] 
  &\hspace*{25mm}j=1,2,\ldots ,
    \end{array}
  \right\}
 \label{T:5.1-1}
 \end{equation}
 \textit{при этом $\nu_j[\theta;L]<\infty$ для любого} $j\hm\in\mathbb{N}_0$.

 \smallskip

 \Do
 Согласно~\cite{16-ch} при сделанных предположениях существуют моменты любого порядка случайной величины $T[\xi]$.
 Поэтому $\Ep\left|  S_L\hm - La\right|^{j}\hm<\infty$ при любом $j\hm\in\Np_0$
 и~в~силу неравенства Ко\-ши--Бу\-ня\-ков\-ско\-го и~ограниченности дисперсии
 $\Vp \widehat{g_L}[\theta|S_L]$  имеем для любого $\theta\hm\in\Theta$
 \begin{multline*}
\left|\Ep \left(\left( S_L - La\right)^j
  \widehat{g}_{L}[\theta|S_L] \right)\right|\le{}\\
  {}\le
  \sqrt{\Ep \left( S_L - La\right)^{2j} 
  \Ep \left(\widehat{g}_{L}[\theta|S_L]\right)^2}<\infty.
 \end{multline*}
 А значит, при каждом $j\hm\in\Np_0$ соотношение
 \begin{multline*}
  \nu_j[\theta;L]={}\\
  {}=
  \!\int\limits_{-\infty}^\infty \!\left( t - La\right)^j \widehat{g}_{L}[\theta|t]\,
  \exp\left\{\theta t-L\kappa[\theta]+d_L[t]\right\}\,d\mu[t]\hspace*{-4.45374pt}
 %\label{T:5.1-2}
 \end{multline*}
 можно почленно дифференцировать по~$\theta$ любое число раз.

 Дифференцируя это соотношение и~учитывая, что $a'\hm=a'[\theta]\hm=\kappa''[\theta]$,  получим~\eqref{T:5.1-1}.

\smallskip

\noindent
\textbf{Следствие 5.1.}\
В условиях теоремы~5.1
 \begin{align*}
 \nu_1[\theta;L]&   =g'[\theta]\,;
\\
 \nu_2[\theta;L]&   =b^2 L g[\theta]+g''[\theta]\,;
\\
 \nu_3[\theta;L]&   =b^3 L\rho_3 g[\theta] +3 b^2 L g'[\theta]+g^{(3)}[\theta]\,;
\\
 \nu_4[\theta;L]&   =3 b^4 L^2 g[\theta]+b^4 L\rho_4 g[\theta] +
   4 b^3 L \rho_3 g'[\theta]+{}\\
   &\hspace*{25mm}{}+ 6 b^2 L g''[\theta]+g^{(4)}[\theta]\,;
\\
 \nu_5[\theta;L]&    =10 b^5 L^2\rho_3 g[\theta] +b^5 L\rho_5 g[\theta] +
   15 b^4 L^2 g'[\theta]+{}\\
&   \hspace*{-12.3mm}{}+5 b^4 L \rho_4 g'[\theta] +10 b^3 L \rho_3 g''[\theta]+10 b^2 L g^{(3)}[\theta]+g^{(5)}[\theta].
 \end{align*}


\noindent
\textbf{Следствие 5.2.}\
 Пусть $\gamma_j\hm=\Ep\left(Z_n^j\right)$,
$j\hm=1,2,\ldots,6$. Тогда в~условиях теоремы~5.1 справедливы выражения:
 \begin{gather*}
 \gamma_1=0\,;\ \gamma_2=1\,;\ \gamma_3=\fr{\rho_3}{n^{1/2}}\,;\
 \gamma_4=3+\fr{\rho_4}{n}\,;
\\
 \gamma_5
   =\fr{10\rho_3}{n^{1/2}} +\fr{\rho_5}{n^{3/2}}\,;\
 \gamma_6=15+\fr{10\rho_3^2+15\rho_4}{n}+\fr{\rho_6}{n^2}.
% \label{C:5.2-1}
 \end{gather*}

\smallskip

  Утверждение следствия~5.2 легко получить, воспользовавшись следствием~5.1,
 имея в~виду, что $\gamma_j\hm=\nu_j[\theta;L]/(b\sqrt{n})^j$
 при $g[\theta]\hm\equiv 1$, $L\hm=n$.

\smallskip

\noindent
\textbf{Следствие 5.3.}\
 В~условиях теоремы~5.1 любая несмещенно
 оцениваемая функция $g[\theta]$ имеет производные любого порядка.

\smallskip

 Далее $p^{[k]}_n[z;\theta]$~--- приближение Эджворта порядка~$k$ для плотности 
 распределения нормированной суммы~$Z_n$, определяемое выражением:
 \begin{equation}
p^{[k]}_n[z;\theta]=\sum\limits_{j=0}^k \fr{q_j[z]}{n^{j/2}}\,.
 \label{6:0}
 \end{equation}
 Здесь
 $$
 q_j[z]=\varphi[z]\,\varrho_j[z]\,,\enskip \varrho_0[z]\equiv 1\,,
 $$
 где
 \begin{multline}
 \varrho_j[z]=\sum H_{j+2s}[z] \prod_{m=1}^j
 \fr{1}{k_m !}\left(\fr{\rho_{m+2}}{(m+2)!} \right)^{k_m}\,,\\
 j=\overline{1,k}\,.
 \label{6:2}
 \end{multline}

  Суммирование в~\eqref{6:2} производится по всем целым неотрицательным решениям
  $(k_1,\ldots,k_j)$ уравнения
 $k_1\hm+2k_2+\cdots+j k_j\hm=j$ при $s\hm=k_1\hm+k_2+\cdots+k_j$.

 Основываясь на хорошо известном соотношении
 \begin{equation*}
 \left(\varphi[z]\,H_k[z]\right)^{(\ell)}=(-1)^\ell
 \varphi[z]\,H_{k+\ell}[z]\,,\ \ell\in \mathbb{N}_0\,,
  %\label{6:7}
 \end{equation*}
 и~используя~\eqref{6:0}, \eqref{6:2}, нетрудно получить следу\-ющее утверждение.

\smallskip

\noindent
\textbf{Лемма~5.4.}\
\textit{Для любых} $z\hm\in\mathbb{R}$, $m\hm\in\Np_0$
 \begin{align*}
  q_0^{(m)}[z]&=(-1)^m\varphi[z]H_m[z]\,;\\
  q_1^{(m)}[z]&=(-1)^m\varphi[z]\fr{\rho_3 H_{3+m}[z]}{6}\,;\\
 q_2^{(m)}[z]&=(-1)^m \varphi[z]\left\{
 \fr{\rho_4 H_{4+m}[z]}{24}+\fr{\rho_3^2 H_{6+m}[z]}{72}
 \right\}\,;\\
 q_3^{(m)}[z]&=(-1)^m\varphi[z] \left\{
 \fr{\rho_5 H_{5+m}[z]}{5!}+{}\right.\\
&\hspace*{15mm}\left. {}+\fr{\rho_3 \rho_4 H_{7+m}[z]}{3!4!}+
 \fr{\rho_3^3 H_{9+m}[z]}{(3!)^4}
 \right\}\,;\\
 q_4^{(m)}[z]&=(-1)^m\varphi[z]   \left\{
 \fr{\rho_6 H_{6+m}[z]}{6!}+\right.\\
&\left. \hspace*{8mm}{}+
 \left( \fr{\rho_3 \rho_5}{3!\,5!}+\fr{\rho_4^2}{(4!)^2 2!}\right) H_{8+m}[z]+ {} \right.\\
&\left.
\hspace*{13mm}{}  +\fr{\rho_3^2 \rho_4 H_{10+m}[z]}{(3!)^2\,2!\, 4!}+
 \fr{\rho_3^4\,H_{12+m}[z]}{(3!)^4\,4!}
 \right\}.
 \end{align*}


 Обозначим
  \begin{equation}
  Q_k[z;n]=\sum\limits_{j=0}^k \fr{q_j[z]}{n^{j/2}} +\Op\left(n^{-(k+1)/2}\right).
  \label{6:9}
  \end{equation}

\noindent
\textbf{Лемма 5.5.}\
\textit{Если $A_0,\ldots,A_k$~--- некоторые постоянные, то
  для любых  $\alpha>0,$ $0<\delta<0{,}5$ при $|z|\le \sqrt{2\alpha\ln n}$,
$(k+1)/2-\alpha\geq 5/2$, $m=\overline{0,4},\,n\rightarrow \infty$
справедливы следующие соотношения}:
 \begin{multline*}
 \left(Q_k[z;n]\right)^{-1}
 \sum\limits_{j=0}^k \fr{A_j q_j^{(m)}[z]}{n^{j/2}} ={}\\
 {}=
 \sum\limits_{j=0}^{4-m} \fr{a_j[z]}{n^{j/2}}
 +\Op\left(n^{-5/2+m/2+\delta}\right)=
\Op\left(\ln^{m/2} n\right),
% \label{L:5.5-1}
 \end{multline*}
\textit{где}
 \begin{gather*}
 a_0[z]=\fr{A_0 q_0^{(m)}[z]}{\varphi[z]}\,;\\
 a_j[z]=\fr{A_j q_j^{(m)}[z]}{\varphi[z]}
-\sum\limits_{i=0}^{j-1} \fr{a_i[z] q_{j-i}[z]}{\varphi[z]}\,,\
j=1,\ldots,k\,.
 \end{gather*}
 \textit{При этом для любого} $k\hm\in \mathbb{N}_0$
  \begin{equation}
  \left(Q_k[z;n]\right)^{-1}= \Op\left(n^{\alpha}\right).
 \label{L:5.5-2}
 \end{equation}


\smallskip

 \Do
 Заметим,
 что $\varphi[\sqrt{2\alpha\ln n}]\hm=\varphi[0]\,n^{-\alpha}$,
 $\left\{a_j[z],\ j\hm=\overline{0,k}\right\}$~--- многочлены от~$z$,
 а~функции $\left\{q_j^{(m)}[z],\ j\hm=\overline{0,k},\ m\hm=\overline{0,4}\right\}$ равномерно ограничены на числовой прямой.
 Поэтому для любого $j\hm=\overline{0,k}$ при $|z|\hm\le \sqrt{2\alpha\ln n}$
 верна оценка $a_j[z]\hm=\op\left(n^{\delta}\right)$.
    При этом справедливы соотношения:
 \begin{multline*}
  \left(Q_k[z;n]\right)^{-1}={}\\
  {}=
 \fr{1}{\varphi[z](1+\op(1)) +\Op\left(n^{-(k+1)/2}\right)}=
 \Op\left(n^\alpha\right)\,;
\end{multline*}
\begin{equation*}
 \sum\limits_{j_1,j_2=0}^{k}\! \fr{q_{j_1}[z] a_{j_2}[z]}{n^{(j_1+j_2)/2}}
 = \sum\limits_{j=0}^k \! \fr{A_{j}\,q_{j}^{(m)}[z]}{n^{j/2}}+
 \Op\!\left(\!n^{-(k+1)/2+\delta}\!\right)\!,
 \end{equation*}

 \vspace*{-12pt}

 \noindent
 \begin{multline*}
 \left(Q_k[z;n]\right)^{-1}\left\{\sum\limits_{j=0}^k \fr{A_j q_j^{(m)}[z]}{n^{j/2}}-{}\right.\\
\left. {}-
 \sum\limits_{j_1,j_2=0}^{k} \fr{q_{j_1}[z]\,a_{j_2}[z]}{n^{(j_1+j_2)/2}}
 + \Op\left(n^{-(k+1)/2+\delta}\right)
\vphantom{\sum\limits_{j=0}^k}
\right\} ={}
\\
 {} =\Op\left(n^{-(k+1)/2+\alpha+\delta}\right)=
  \Op\left(n^{-5/2+\delta}\right).
 \end{multline*}

 Осталось заметить, что
 \begin{multline*}
 \sum\limits_{j=0}^4 \fr{a_j[z]}{n^{j/2}}+\Op\left(n^{-5/2+\delta}\right)
 - \sum\limits_{j=0}^{4-m} \fr{a_j[z]}{n^{j/2}}={}\\
 {}=\Op\left(n^{-5/2+m/2+\delta}\right)\,,\enskip
 \left|a_0[z]\right|=\Op\left(\ln^{m/2} n\right).
 \end{multline*}


\noindent
\textbf{Лемма 5.6.}\
\textit{Для любых $\alpha>0,$ $0<\delta<0{,}5$
 при $|z|\le \sqrt{2\alpha\ln n}$, $n\to \infty,$
 $(k+1)/2-\alpha\geq 5/2$ справедливо разложение}:
 \begin{multline*}
  \left(Q_k[z;n]\right)^{-1}\sum\limits_{j=0}^k \fr{(j+1)(j+3)q_j^{(0)}[z]}{n^{j/2}} ={}\\
  {}=
  3+\Op(n^{-1/2+\delta})\,.
  %\label{L:5.6-1}
 \end{multline*}


 Утверждение леммы~5.6 нетрудно получить, воспользовавшись леммами~5.4 и~5.5
 при $A_j\hm=(j+1)(j+3)$, $a_0[z]\hm=3$, $m\hm=0$.

 \smallskip

\noindent
\textbf{Лемма 5.7.}\
\textit{Для любых $\alpha\hm>0,$ $0\hm<\delta\hm<0{,}5$
 при $|z|\hm\le \sqrt{2\alpha\ln n}$, $n\hm\to \infty,$
 $(k+1)/2\hm-\alpha\hm\geq 5/2$, $m\hm=\overline{0,2}$ справедливо разложение}:
 \begin{multline*}
  \left(Q_k[z;n]\right)^{-1}\sum\limits_{j=0}^k \fr{(j+1)q_j^{(m)}[z]}{n^{j/2}} ={}\\
  {}=
  \sum\limits_{j=0}^{2-m} \fr{b_{m,j}[z]}{n^{j/2}}+
  \Op(n^{-3/2+m/2+\delta})\,,
 %\label{L:5.7-1}
 \end{multline*}
\textit{где}
 \begin{gather*}
 b_{0,0}[z]=1\,;\enskip  b_{0,1}[z]=\fr{\rho_3 H_3[z]}{6}\,;
\\
 \hspace*{-11mm}b_{0,2}[z]=   \fr{1}{12} \left\{\rho_3^2\left(-3z^4+12z^2-5\right)+{}\right.\\
 \left.\hspace*{20mm}{}+
  \rho_4\left(z^4-6z^2+3\right)\right\}\,;
\\
 b_{1,0}[z]=-z\,;\enskip
 b_{1,1}[z]=-\fr{\rho_3}{6}\left(z^4-9z^2+6\right)\,;\\
 b_{2,0}[z]=H_2[z]\,.
 \end{gather*}

 Утверждение леммы~5.7 нетрудно получить, воспользовавшись леммами~5.4 и~5.5 при
 $A_j\hm=j\hm+1$, $a_j[z]\hm=b_{m,j}[z]$,  $m\hm=0,1,2$.

 \smallskip

\noindent
\textbf{Лемма 5.8.}\
\textit{Для любых $\alpha\hm>0,$ $0\hm<\delta\hm<0{,}5$
 при $|z|\hm\le \sqrt{2\alpha\ln n}$, $n\hm\to \infty,$
 $(k+1)/2\hm-\alpha\hm\geq 5/2$, $m\hm=\overline{0,4}$ справедливо разложение}:
 \begin{multline*}
 \left(Q_k[z;n]\right)^{-1}\sum\limits_{j=0}^k \fr{q_j^{(m)}[z]}{n^{j/2}}={}\\
 {}=
  \sum\limits_{j=0}^{4-m} \fr{c_{m,j}[z]}{n^{j/2}}+\Op(n^{-5/2+m/2+\delta})\,,
  %\label{L:5.8-1}
 \end{multline*}
\textit{где}
 \begin{gather*}
  c_{0,0}[z]=1\,;\enskip c_{0,j}[z]=0\,,\enskip j=\overline{1,4}\,;
\\
 c_{1,0}[z]=-z\,;\enskip  c_{1,1}[z]=\fr{\rho_3 H_2[z]}{2}\,;\\
 c_{1,2}[z]=\fr{z}{6} \left\{-3\left(z^2-2\right)\rho_3^2+
 \left(z^2-3\right)\rho_4\right\}\,;
\\
 \hspace*{-15mm}c_{1,3}[z]=\fr{1}{24} \left\{3 \left(5z^4-16z^2+5\right) \rho_3^3-{}\right.\\
\left. {}-2
 \left(5z^4-21z^2+8\right) \rho_3 \rho_4+\rho_5\left(z^4-6z^2+3\right) \right\}\,;
\\
 c_{2,0}[z]=H_2[z]\,;\enskip  c_{2,1}[z]=-z\rho_3\left(z^2-2\right)\,;
\\
\hspace*{-10mm}c_{2,2}[z]=   \fr{1}{12} \left\{3\rho_3^2\left(5z^4-16z^2+5\right)-{}\right.\\
\left. \hspace*{15mm}{}-
  2\rho_4\left(2z^4-9z^2+3\right)\right\}\,;
\\
 c_{3,0}[z]=-H_3[z]\,;\enskip
 c_{3,1}[z]=\fr{\rho_3}{2}\left(3z^4-12z^2+5\right)\,;\\
 c_{4,0}[z]=z^4-6 z^2+3\,.
 \end{gather*}

 Утверждение леммы~5.8 нетрудно получить, воспользовавшись леммами~5.4 и~5.5 при
 $A_j\hm\equiv 1$, $a_j[z]\hm=c_{m,j}[z]$, $m\hm=\overline{0,4}$.

\smallskip

\noindent
\textbf{Лемма 5.9.}\
\textit{При $u\hm=\sqrt{{n}/(n-L)}\left(z-{\Delta}/{\sqrt{n}}\right)$,
  $L\hm\in \Np$,  $\Delta\hm=\Op(n^\nu)$, $0\hm<\nu\hm<0{,}5$,
  $|z|\hm\le \sqrt{2\alpha\ln n},$ $n\hm\rightarrow\infty$,
  для любых $m\hm=\overline{0,4}$, $\alpha\hm>0$ справедливо разложение}:
 \begin{multline}
(u-z)^m=A_m[z,\Delta,n,L]+P_{m,1}[\Delta]\Op\left(
\fr{|z|^m}{n^{5/2}}\right)={}\\
{}=
 P_{m,2}[\Delta]\,\Op\left( \fr{1}{n^{m/2}}\right)\,,
 \label{L:5.9-1}
 \end{multline}
 \textit{где $\{P_{m,j}[\Delta],\ j\hm=1,2\}$~--- некоторые многочлены от~$\Delta$ степени~$m$},
 \begin{gather*}
A_1[z,\Delta,n,L]=\fr{3 L^2 z}{8 n^2}+\fr{L z}{2 n}
  -\left(\fr{L}{2 n^{3/2}}+\fr{1}{\sqrt{n}}\right) \Delta\,;
\\
 A_2[z,\Delta,n,L]=\fr{L^2 z^2}{4 n^2}- \fr{L z}{n^{3/2}} \Delta +\left(\fr{L}{n^2}  +\fr{1}{n}\right) \Delta ^2\,;
\\
 A_3[z,\Delta,n,L]=\fr{3 L z}{2 n^2}  \Delta^2-\fr{1}{n^{3/2}}\Delta^3\,;\\
 A_4[z,\Delta,n,L]= \fr{\Delta^4}{n^2}\,,\enskip
 A_5[z,\Delta,n,L]\equiv 0\,.
 \end{gather*}

 \textit{При этом для любого $z_*\hm\in \left[\min (u,z),\max (u,z)\right]$ верна оценка}:
 \begin{equation*}
\fr{ \varphi[z_*]}{\varphi[z]}=\Op(1)\,.
% \label{L:5.9-2}
 \end{equation*}

 \Do
 Соотношение~\eqref{L:5.9-1} легко получить c~помощью элементарных
 преобразований из разложения
 \begin{multline*}
 u-z= {}\\
 {}= \left(1+\fr{L}{2 n}+\fr{3L^2}{8 n^2}+\Op(n^{-3})\right)
  \left(z-\fr{\Delta }{\sqrt{n}}\right)-z={}
\\
{} =\fr{L z}{2 n}+\fr{3 L^2 z}{8 n^2}
 -\Delta \left(\fr{1}{\sqrt{n}}+\fr{L}{2 n^{3/2}}\right)
 +\Delta \Op\left(\frac{1}{n^{5/2}}\right)\,.
 \end{multline*}

 Теперь заметим, что при 
 $$
 z_*\in \left(\min (u,z),\max (u,z)\right)
 $$ 
 верны  неравенства:
 \begin{align*}
  \left|z_*-z \right|&\le  \left|u-z \right|=
  \Op\left(\fr{\Delta}{\sqrt{n}}\right)\,;\\
 \left|z_*+z \right|&\le  \left|z_*-z \right|+2\left|z \right|=\Op(\sqrt{\ln n})\,.
 \end{align*}
 Отсюда
 \begin{multline*}
 \fr{\varphi[z_*]}{\varphi[z]}=
 \exp\left\{-\fr{\left(z_*-z \right)\left(z_*+z \right)}{2} \right\}={}\\
 {}=
 \exp\left\{\Op\left(\fr{z\Delta}{\sqrt{n}}\right) \right\}=\Op(1)\,,
 \end{multline*}
 что завершает доказательство леммы~5.9.

\columnbreak

\noindent
\textbf{Лемма 5.10.}\
\textit{Пусть выполнены условия $\mathbf{(A_1)}$ и~$\mathbf{(A_2)}$.
 Тогда для любых $\alpha>0,\,\beta>0$ справедливы следующие оценки}:
 \begin{gather}
 \Ep |Z_n|^\beta \le c_\beta\,;
 \label{L:5.10-1}\\
 {\sf P}\left(|Z_n|\ge \sqrt{2\alpha\ln n}\right)=\op\left(n^{-\alpha}\right)
 \mbox{ при } n\to\infty\,;
 \label{L:5.10-2}
  \end{gather}

  \vspace*{-12pt}

  \noindent
  \begin{multline*}
 \Ep \left( |Z_n|^\beta I\left[|Z_n|\ge \sqrt{2\alpha\ln n}\right]\right)
 = \op\left(n^{-\alpha/2}\right)\\  \mbox{ при } n\to\infty\,,
% \label{L:5.10-3}
 \end{multline*}
 \textit{где $c_\beta$~--- положительная постоянная, зависящая только от}~$\beta$.

\vspace*{2pt}

 \Do Оценку~\eqref{L:5.10-1} получим, воспользовавшись теоремой~20
 из~\cite[с.~89]{17-ch} при $\beta\hm\ge 2$ и~неравенством Ляпунова при $0\hm<\beta\hm<2$:
 \begin{align*}
  \Ep |Z_n|^\beta&=\fr{\Ep |S_n-na|^\beta}{(b\sqrt{n})^\beta}\le{}\\
&  \hspace*{5mm}{}\le
 \fr{C[\beta]n^{\beta/2-1}n\Ep |T[X_1]-a|^\beta}{(b\sqrt{n})^\beta};
\\
 \Ep |Z_n|^\beta &\le \left( \Ep |Z_n|^2\right)^{\beta/2}=1\,,
 \end{align*}
 где $C[\beta]$~--- некоторая постоянная, зависящая только от~$\beta$.

  Применив теорему Амосовой~\cite[с.~309]{10-ch} при $c\hm=\sqrt{2\alpha}$, оценим
 \begin{multline*}
   {\sf P}\left(|Z_n|\ge c\sqrt{\ln n}\right)={}\\
   {}=
 \fr{2}{cn^{c^2/2}\,\sqrt{2\pi\ln n}}
 \left(1+\Op\left(\fr{1}{\ln n}\right)\right)={}
\\
  {}  = \fr{1}{n^{\alpha}\,\sqrt{\alpha\pi\ln n}}
 \left(1+\Op\left(\fr{1}{\ln n}\right)\right)= \op\left(n^{-\alpha}\right)\,.
 \end{multline*}

 Используя неравенство Ко\-ши--Бу\-ня\-ков\-ско\-го и~\eqref{L:5.10-1}--\eqref{L:5.10-2},
 получим:
 
 \noindent
 \begin{multline*}
  \Ep \left( |Z_n|^\beta I\left[|Z_n|\ge \sqrt{2\alpha\ln n}\right]\right)
 \le{}\\
 {}\le
 \sqrt{\Ep |Z_n|^{2\beta}
  {\sf P}\left(|Z_n|\ge \sqrt{2\alpha\ln n}\right)}=
 \op\left(n^{-\alpha/2}\right)\,.
 \end{multline*}
 
 \vspace*{-2pt}


\noindent
\textbf{Лемма 5.11.}\
\textit{Пусть выполнены условия}~$\mathbf{(A_1)}$ и~$\mathbf{(A_2)}$, $L_0\hm\le L \hm<n$.
\begin{enumerate}[1.]
\item
\textit{Если} 
$$
\hspace*{-4pt}\delta_0>0;\ \beta>0;\
\Ep \left|\widehat{g}_L[\theta|S_L]-g[\theta] \right|^{\max\{1,\beta+\delta_0\}}
 <\infty\,,
 $$
  \textit{то}
 \begin{multline}
 \Ep \left(\left|\widehat{g}_n[\theta|S_n]-g[\theta] \right|^{\beta}
 I\left[|Z_n|\ge \sqrt{2\alpha\ln n}\right]\right)={}\\
 {}=
 \op\left(n^{-\alpha\delta_0/(\beta+\delta_0)}\right)\,.
 \label{L:5.11-1}
 \end{multline}
 \item \textit{Если} $\beta\ge 2,$
 $\Ep \left|\widehat{g}_L[\theta|S_L]\hm-g[\theta] \right|^\beta\hm<\infty,$
 $n_L$~--- \textit{целая часть числа~$n/L$, то}
  \begin{multline}
 \Ep \left|\widehat{g}_n[\theta|S_n]-g[\theta] \right|^{\beta}
 \le{}\\
 {}\le
 d_\beta n_L^{-\beta/2}
 \Ep \left|\widehat{g}_L[\theta|S_L]-g[\theta] \right|^{\beta},
 \label{L:5.11-2}
 \end{multline}
 \textit{где $d_\beta$~--- положительная постоянная, зависящая только от~$\beta$, и}
  \begin{equation}
 \Ep \left|\widehat{g}_n[\theta|S_n]\right|^{\beta}
 =\left|g[\theta]\right|^{\beta}+
 {\bf O}\left(\fr{1}{\sqrt{n}} \right)\,.
 \label{L:5.11-3}
 \end{equation}
 \end{enumerate}

 \Do
 Если $r\hm=\beta\hm+\delta_0\hm<1$, то, используя неравенство Ляпунова
 и~тео\-ре\-му Рао--Блэк\-вэл\-ла--Кол\-мо\-го\-ро\-ва, получим:
  \begin{multline*}
\Ep \left|\widehat{g}_n[\theta|S_n]-g[\theta] \right|^{r}\le
 \left(\Ep \left|\widehat{g}_n[\theta|S_n]-g[\theta] \right|\right)^{r}\le{}\\
 {}\le
 \left(\Ep \left|\widehat{g}_L[\theta|S_L]-g[\theta]
  \right|\right)^{r}<\infty\,.
 \end{multline*}
 При $r\ge 1$, используя теорему Рао--Блэк\-вэл\-ла--Кол\-мо\-го\-ро\-ва, имеем:
  \begin{equation*}
\Ep \left|\widehat{g}_n[\theta|S_n]-g[\theta] \right|^{r}\le
 \Ep \left|\widehat{g}_L[\theta|S_L]-g[\theta] \right|^{r}<\infty\,.
 \end{equation*}

 Докажем~\eqref{L:5.11-1}, воспользовавшись неравенством Гельдера с~$p\hm=1\hm+\delta_0/\beta$ и~\eqref{L:5.10-2}:
 \begin{multline*}
{\sf E} \left(\left|\widehat{g}_n[\theta|S_n]-g[\theta] \right|^{\beta}\,
 I\left[|Z_n|\ge \sqrt{2\alpha\ln n}\right]\right)\le{}
\\
{} \le
 \left\{{\sf E} \left|\widehat{g}_n[\theta|S_n]-g[\theta] \right|^{\beta+\delta_0}
 \right\}^{1/p}\times{}\\
 {}\times
 \left\{{\sf E} I\left[|Z_n|\ge \sqrt{2\alpha\ln n}\right]\right\}^{(p-1)/p}={}
\\
 {}= \left\{\op\left(n^{-\alpha}\right)\right\}^{(p-1)/p}=
 \op\left(n^{-\alpha\delta_0/(\beta+\delta_0)}\right).
 \end{multline*}

 Теперь по каждой подвыборке
 $\left(X_{1+(j-1)L},\right.$ $\left.X_{2+(j-1)L},\ldots,X_{L+(j-1)L}\right)$ построим несмещенную оценку c~минимальной дисперсией
 $\widehat{g}_{L,j}[\theta|S_{L,j}]$, $j\hm=1,\ldots,n_L$, где
 $S_{L,j}\hm=\sum\limits_{i=1}^L X_{i+(j-1)L}.$
 Используя теорему Рао--Блэк\-вэл\-ла--Кол\-мо\-го\-ро\-ва
 и~теорему~20~\cite[с.~89]{17-ch},
 убедимся в~справедливости~\eqref{L:5.11-2} при $\beta\hm\ge 2$:
 \begin{multline*}
\Ep \left|\widehat{g}_n[\theta|S_n]-g[\theta] \right|^{\beta}\le{}\\
{}\le
 \Ep \left|\fr{1}{n_L} \sum\limits_{j=1}^{n_L} \widehat{g}_{L,j}[\theta|S_{L,j}]
 -g[\theta] \right|^{\beta}\le{}
\\
 {}\le
 \fr{d_\beta}{n_L^{\beta/2}} \,
 \Ep \left|\widehat{g}_L[\theta|S_L]-g[\theta] \right|^{\beta}.
 \end{multline*}

 Справедливость~\eqref{L:5.11-3} при $\beta\hm\ge 2$ следует из оценок:
  \begin{multline*}
  \left(\Ep \left|\widehat{g}_n[\theta|S_n]\right|^{\beta}\right)^{1/\beta}\le{}\\
  {}\le
 \left(\Ep \left|\widehat{g}_n[\theta|S_n]-g[\theta]\right|^{\beta}\right)^{1/\beta}+
 \left(\Ep \left|g[\theta] \right|^{\beta}\right)^{1/\beta}\,;
\end{multline*}

\vspace*{-12pt}

\noindent
\begin{multline*}
 \left(\Ep \left|g[\theta] \right|^{\beta}\right)^{1/\beta}-
 \left(\Ep \left|\widehat{g}_n[\theta|S_n]\right|^{\beta}\right)^{1/\beta}\le{}\\
 {}\le
 \left(\Ep\left|\widehat{g}_n[\theta|S_n]-g[\theta]\right|^{\beta}\right)^{1/\beta},
 \end{multline*}
 которые получены с~по\-мощью неравенства Минковского.

 \smallskip

\noindent
\textbf{Лемма 5.12.}\
\textit{Пусть выполнены условия $\mathbf{(A_1)}$ и~$\mathbf{(A_2)}$,
  $\Vp \widehat{g}_L[\theta|S_L]\hm<\infty$ при некотором $L_0\hm\le L \hm<n$,
  $B\hm=\left\{t:\,|t-La|\hm\le n^\nu\right\}$ при $0\hm<\nu\hm<0{,}5$,
$\Delta[t]\hm=(t\hm-La)/b.$
 Тогда для любых $\left\{\zeta_i\hm>0,\ i\hm=1,2,3\right\}$, $j\hm\in\Np_0,$ при
 $n\hm\to\infty$ справедливы следующие оценки}:
 \begin{align}
 &\Ep\left\{ I\left[S_L\notin B\right]\right\}={\sf P}\left(\left|S_L - 
 La \right|>n^{\nu}\right)={}\notag\\
 &\hspace*{50mm}{}=\Op\left( n^{-\zeta_1}\right)\,;
\label{L:5.12-1}
 \\
  &
  \Ep\left\{ I\left[S_L\notin B\right]\cdot
  \left| \widehat{g}_L[\theta|S_L]-g[\theta]\right|\right\}
  =\Op\left( n^{-\zeta_2}\right)\,;
 \label{L:5.12-2}\\
 &   \Ep\left\{ I\left[S_L\notin B\right]\cdot
  \left|\Delta^{j}[S_L]\, \widehat{g}_L[\theta|S_L]\right|\right\}
  =\Op\left( n^{-\zeta_3}\right).\!\!
 \label{L:5.12-3}
  \end{align}

 \Do
  Поскольку существует момент любого порядка случайной величины~$S_L$, 
  то для любого $\beta\hm>0$ при $n\hm\to\infty$ с~учетом неравенства Маркова имеем:
 \begin{multline*}
{\sf P}\left(\left|S_L - La \right|>n^{\nu}\right)\le
  \fr{\Ep \left|S_L-La \right|^\beta}{n^{\nu \beta}}={}\\
  {}=
  \fr{(b\sqrt{L})^\beta\Ep \left|Z_L\right|^\beta}{n^{\beta\nu }}=
  \Op\left( n^{-\beta\nu}\right).
 \end{multline*}


  Отсюда в~силу неравенства Ко\-ши--Бу\-ня\-ков\-ско\-го и~тео\-ре\-мы~5.1
  \begin{multline*}
    \Ep\left\{ I\left[S_L\notin B\right]\cdot
  \left| \widehat{g}_L[\theta|S_L]-g[\theta]\right|\right\}\le{}
  \\
  {}  \le\sqrt{\Vp \widehat{g}_L[\theta|S_L]\,
  \Ep I\left[S_L\notin B\right] } =
   \Op\left( n^{-{\beta\nu/2}}\right)\,;
  \end{multline*}

\vspace*{-12pt}

\noindent
\begin{multline*}
  \Ep\left\{ I\left[S_L\notin B\right]\cdot
  \left|\Delta^{j}[S_L]\, \widehat{g}_L[\theta|S_L]\right|\right\}
  ={}\\
  {}=\Op\left( n^{-{\beta\nu/4}}\right),\enskip  j\in\Np_0\,.
  \end{multline*}
  Произвольность выбора~$\beta$ в~таком случае означает справедливость~\eqref{L:5.12-1}--\eqref{L:5.12-3}.


\section{Асимптотические разложения функций, определяющих
  несмещенные оценки}

\noindent
\textbf{Теорема~6.1.}\
\textit{Пусть выполнены условия $\mathbf{(A_1)}$ и~$\mathbf{(A_2)}$, $L_0\hm\le L\hm<n$,
$\Delta\hm=(t-La)/b\hm=\mathbf {O}(n^\nu)$ при $0\hm<\nu\hm<0{,}5$.
Тогда для любых $\alpha\hm>0,$ $0\hm<\delta\hm<0{,}5$ при
$|z|\hm\le \sqrt{2\alpha\ln n}$, $s\hm=na\hm+zb\sqrt{n}$, $n\hm\rightarrow\infty$
справедливо асимптотическое разложение}:

\pagebreak

\noindent
 \begin{multline}
\fr{f_{S_{n-L}}[s-t;\theta]}{f_{S_n}[s;\theta]}
  =1+\sum\limits_{j=1}^4 \fr{c_{jf}[z,L,\Delta]}{n^{j/2}}+{}\\
  {}+
  P_5[\Delta]\,\Op\left(n^{-5/2+\delta} \right),
 \label{T:6.1-1}
 \end{multline}
  \textit{где $P_5[\Delta]$~--- некоторый многочлен от~$\Delta$ степени~5},
 \begin{align*}
& c_{1f}[z,L,\Delta]=z\Delta\,;\\
& c_{2f}[z,L,\Delta]=\fr{H_2[z]}{2}\left( \Delta^2-\rho_3\Delta-L\right)\,;
\\
 &c_{3f}[z,L,\Delta]
  =L\rho_3\left(\fr{z^3}{3}-\fr{z}{2}\right)+{}\\
&\hspace*{5mm}{}+
   \Delta \left\{ H_3[z] \left(\fr{\rho_3^2-L}{2}-\fr{\rho_4}{6}\right)+
   \fr{\rho_3^2 z}{2}\right\}+{}\\
&\hspace*{15mm}{}   +\Delta^2\rho_3\left(z-\fr{z^3}{2}\right)+
    \Delta^3\fr{H_3[z]}{6}\,;
 \end{align*}

 \vspace*{-12pt}

 \begin{multline*}
  c_{4f}[z,L,\Delta]
  =L^2\left(\fr{3}{8}-\fr{3 z^2}{4}+\fr{z^4}{8}\right)
  -{}\\
  {}- L\rho_3^2\left(\fr{5}{24}-z^2+\fr{3 z^4}{8}\right)+L\rho_4\left(\fr{1}{8}-\fr{z^2}{2}+\fr{z^4}{8}\right)
   +{}\\
   {}+\Delta \!\left\{\!L\rho_3\!\left(\fr{3}{4}-2 z^2+\fr{7z^4}{12}\right)
   -\rho_3^3\!\left(\fr{5}{8}-2 z^2+\fr{5 z^4}{8}\right)
   +{}\right.
\\
   \left.
  {} +\rho_3\rho_4\left(\fr{2}{3}-\fr{7z^2}{4}+\fr{5z^4}{12}\right)
   -\rho_5\left(\fr{1}{8}-\fr{z^2}{4}+\fr{z^4}{24}\right)\right\}+{}
\\
  {}   +\Delta^2 \left\{
    \rho_3^2\left(\fr{5}{8}-2z^2+\fr{5z^4}{8}\right)
   -L\left(\fr{3}{4}-\fr{3z^2}{2}+\fr{z^4}{4}\right)
   -{}\right.\\
\left.   {}-\rho_4\left(\fr{1}{4}-\fr{3z^2}{4}+\fr{z^4}{6}\right)\right\}-\Delta^3 \rho_3\left(\fr{5}{12}-z^2+\fr{z^4}{4}\right)+{}\\
{}+
    \Delta^4\left(\fr{1}{8}-\fr{z^2}{4}+\fr{z^4}{24}\right)\,.
  \end{multline*}
\textit{При этом для любого} $t\hm\in\mathbb{R}$
  \begin{equation}
\fr{f_{S_{n-L}}[s-t;\theta]}{f_{S_n}[s;\theta]}
  =\Op\left(n^{\alpha} \right)\,.
 \label{T:6.1-2}
 \end{equation}

 \Do
 Положим 
 $$
 u=\left(\fr{n}{n-L}\right)^{1/2}\left(z-\fr{\Delta}{\sqrt{n}}\right)\,.
 $$
 
 Используя уточнения локальной предельной тео\-ре\-мы 
 (тео\-ремы~13 и~15 из~\cite{10-ch}) и~приближение Эджворта~\eqref{6:0},
 левую часть~$\eqref{T:6.1-1}$ при любом $k\hm\in\mathbb{N}$ можно представить в~виде:
 \begin{multline}
\Psi_n=\fr{f_{S_{n-L}}[s-t;\theta]}{f_{S_n}[s;\theta]}={}\\
{}=
 \sqrt{\fr{n}{n-L}} \,
  \fr{p_{n-L}^{[k]}[u;\theta]  +\Op(n^{-(k+1)/2})}
  {p_{n}^{[k]}[z;\theta]+\Op(n^{-(k+1)/2})},
 \label{T:6.1-3}
 \end{multline}
 поскольку в~случае абсолютно непрерывного распределения~$\xi$
 \begin{multline*}
 f_{S_n}[s;\theta]=(b\sqrt{n})^{-1} f_{Z_n}[z;\theta]={}\\
 {}=
 (b\sqrt{n})^{-1}\left(p_{n}^{[k]}[z;\theta]+\Op(n^{-(k+1)/2})\right),
 \end{multline*}
 а~в случае решетчатого распределения~$\xi$ верно соотношение:
 \begin{equation*}
 b\sqrt{n}f_{S_n}[s;\theta]=p_{n}^{[k]}[z;\theta]+\Op(n^{-(k+1)/2})\,.
 \end{equation*}

   Выберем $k$ таким, что $(k\hm+1)/2\hm-\alpha\hm\ge 5/2$,
   и~осуществим разложение в~точке~$z$ следующего выражения:
 \begin{multline*}
   \sqrt{\fr{n}{n-L}}\, p_{n-L}^{[k]}[u;\theta]=
  \sum\limits_{j=0}^k \fr{q_j[u]}{n^{j/2}}\left(\fr{n}{n-L}\right)^
  {(j+1)/2}={}
\\ {}  =
  \sum\limits_{m=0}^4 \fr{(u-z)^m}{m!}\sum_{j=0}^k \fr{q_j^{(m)}[z]}{n^{j/2}}
  \left(\fr{n}{n-L}\right)^{(j+1)/2}+{}\\
  {}+
  \fr{(u-z)^5}{5!}\, \varphi\left[z_*\right] \Op\left(1\right),
% \label{T:6.1-4}
 \end{multline*}
 где $z_*\in \left[\min (u,z),\max (u,z) \right]$.

 Используя~\eqref{6:9} и~разложение
 \begin{equation*}
\left( \fr{n}{n-L}\right)^{j/2}
  = 1+\fr{jL}{2n}+\fr{j(j+2)L^2}{8n^2}+\Op(n^{-3})\,,
 \end{equation*}
 представим левую часть~$\eqref{T:6.1-3}$ в~виде суммы $\Psi_n\hm=\sum\limits_{i=1}^6 J_i,$
 где
 \begin{align*}
 J_1&=\left(Q_k[z;n]\right)^{-1}\sum\limits_{m=0}^4 \fr{(u-z)^m}{m!}\sum_{j=0}^k \fr{q_j^{(m)}[z]}{n^{j/2}}\,;
\\
 J_2&=\left(Q_k[z;n]\right)^{-1}\fr{L}{2n}\sum\limits_{m=0}^4 
 \fr{(u-z)^m}{m!}\times{}\\
 &\hspace*{27mm}{}\times \sum\limits_{j=0}^k \fr{(j+1)q_j^{(m)}[z]}{n^{j/2}}\,;
\\
 J_3&=\left(Q_k[z;n]\right)^{-1}\fr{L^2}{8n^2}\sum\limits_{m=0}^4 \fr{(u-z)^m}{m!}\times{}\\
 &\hspace*{25mm}{}\times\sum\limits_{j=0}^k \fr{(j+1)(j+3)q_j^{(m)}[z]}{n^{j/2}}\,;
\\
 J_4&=\left(Q_k[z;n]\right)^{-1}\Op\left(n^{-3}\right)\sum_{m=0}^4 \fr{(u-z)^m}{m!}\sum\limits_{j=0}^k \fr{q_j^{(m)}[z]}{n^{j/2}};
\\
 J_5&=  \left(Q_k[z;n]\right)^{-1}\fr{(u-z)^5}{5!}\, \varphi[z_*]\Op\left(1\right)\,;\\  
 J_6&=\left(Q_k[z;n]\right)^{-1}\Op(n^{-(k+1)/2})\,.
 \end{align*}

 Применяя леммы~5.5--5.9, получим при $0\hm<\delta_i\hm<0{,}5$, $i\hm=1,2,3,$ 
 следующие оценки слагаемых $J_1,\ldots,J_6$:

 \noindent
 \begin{align}
 J_1&=\sum\limits_{m=0}^4  \fr{A_m[z,\Delta,n,L]}{m!} \sum\limits_{j=0}^{4-m}
  \fr{c_{m,j}[z]}{n^{j/2}}+{}\notag\\
  &\hspace*{25mm}{}+P_{4,1}[\Delta]\Op\left(n^{-5/2+\delta_1}\right)\,;
 \label{T:6.1-6}\\
 J_2&=\fr{L}{2n} \sum\limits_{m=0}^2 \fr{A_m[z,\Delta,n,L]}{m!}
 \sum\limits_{j=0}^{2-m} \fr{b_{m,j}[z]}{n^{j/2}}
 +{}\notag\\
 &\hspace*{25mm}{}+P_{4,2}[\Delta]\Op\left(n^{-5/2+\delta_2}\right)\,;
 \label{T:6.1-7}\\
J_3&= \fr{3 L^2}{8n^2}+P_{4,3}[\Delta]\Op\left(n^{-5/2+\delta_3}\right)\,;
 \label{T:6.1-8}\\
J_4&=\Op\left(n^{-3}\right)\,;\notag\\
 J_5&=P_{5,2}[\Delta]\Op\left(n^{-5/2}\right)\,;\notag\\
 J_6&= \Op\left(n^{-5/2}\right)\,,\notag
 \end{align}
 где $A_0[z,\Delta,n,L]\equiv 1$, $P_{i,j}[\Delta]$~--- некоторый многочлен порядка~$i$ от~$\Delta$.

 Подставляя в~\eqref{T:6.1-6}--\eqref{T:6.1-8}  выражения коэффициентов
 $c_{m,j}[z]$ из леммы~5.8,
 $A_m[z,\Delta,n,L]$~--- из леммы~5.9,
 $b_{m,j}[z]$~--- из леммы~5.7 и~используя
 лемму~5.6, установим справедливость разложения~\eqref{T:6.1-1}.

 Справедливость~\eqref{T:6.1-2} следует из~\eqref{L:5.5-2}, \eqref{T:6.1-3}
 и~ограниченности на числовой прямой функций $\left\{q_j[z], \ j\hm=\overline{0,k}\right\}.$

\smallskip

\noindent
\textbf{Следствие 6.1.}\
В~условиях теоремы~6.1 верно следующее разложение функции,
  определяющей несмещенную оценку
 плотности распределения $f_{S_L}[t;\theta]$:
  \begin{multline*}
 \widehat{f}_{S_L}[t;\theta|s]=f_{S_L}[t;\theta]
 \left\{\vphantom{  \sum\limits_{j=1}^4} 1
+  \sum\limits_{j=1}^4 \fr{c_{jf}[z,L,\Delta]}{n^{j/2}}+{}\right.\\
\left.{}+  P_5[\Delta]\Op\left(n^{-5/2+\delta} \right)
\vphantom{  \sum\limits_{j=1}^4}
\right\}.
%  \label{C:6.1-1}
 \end{multline*}

\noindent
\textbf{Теорема 6.2.}\
\textit{Пусть выполнены условия $\mathbf{(A_1)}$ и~$\mathbf{(A_2)}$,
 ${\sf V} \widehat{g}_{L}[\theta|S_L]\hm<\infty$ при некотором $L_0\hm\le L \hm<n$.
 Тогда для любых $\alpha\hm>0$ и~$0\hm<\delta\hm<0{,}5$ при $|z|\hm\le \sqrt{2\alpha\ln n}$, $n\hm\rightarrow\infty$ справедливо следующее асимптотическое разложение
 функции, определяющей несмещенную оценку $\widehat{g}_n[\theta;Z_n]$ функции}
 $g[\theta]$:
  \begin{equation}
\widehat{g}_n[\theta;z]=g[\theta]+
 \sum\limits_{j=1}^4 \fr{c_{jg}[z]}{n^{j/2}}+\Op\left(n^{-5/2+\delta}\right)\,,
  \label{T:6.2-1}
 \end{equation}
\textit{где}
$$
  c_{1g}[z]   =\fr{z g'[\theta]}{b}\,;\enskip  c_{2g}[z] =\fr{H_2[z]\left(g''[\theta]-b\rho_3 g'[\theta] \right)}{2b^2}\,;
  $$
 \begin{align*}
 &c_{3g}[z]   =(z^3-2z)\left(\fr{\rho _3^2 g'[\theta]}{2 b}
    -\fr{\rho_3 g''[\theta]}{2 b^2}  \right)
  +{}\\
  &\hspace*{35mm}{}+H_3[z]\left(\fr{g^{(3)}[\theta]}{6 b^3}-\fr{\rho_4 g'[\theta]}{6 b}\right)\,;
\\
& c_{4g}[z]  =\left(5-16 z^2+5 z^4\right)\left(
 \fr{\rho_3^2 g''[\theta]}{8 b^2}-\fr{ \rho_3^3 g'[\theta]}{8 b}
 \right)+{}
\\
& \hspace*{8mm}{} +\left(3-6 z^2+z^4\right)\left(\fr{g^{(4)}[\theta]}{24 b^4}
 -\fr{ \rho _5 g'[\theta]}{24 b}\right)
 +{}\\
&\hspace*{16mm} {}+\fr{\left(-5+12 z^2-3 z^4\right) \rho_3 g^{(3)}[\theta]}{12
 b^3}+{}\\
 &\hspace*{24mm}{}+
\fr{\left(-3 +9 z^2-2 z^4\right) \rho_4 g''[\theta]}
 {12 b^2}+{}\\
& \hspace*{32mm}{}+  \fr{\left(8-21 z^2+5 z^4\right) \rho_3\rho_4 g'[\theta]}{12 b}\,.
 \end{align*}


  \Do
  Положим 
\begin{gather*}
  B=\left\{t:\,|t-La|\le n^\nu\right\}\ \mbox{при\ } 0<\nu<0{,}5\,;\\
\Delta[t]=\fr{t-La}{b}\,;\enskip s=na+zb\sqrt{n}\,;\enskip
h[S_L]=\widehat{g}_{L}[\theta|S_L]\,.
\end{gather*}
  Тогда, воспользовавшись леммой~5.12, соотношениями~\eqref{T:6.1-2} и~\eqref{L:5.12-2} 
  при $\zeta_2\hm\ge -5/2\hm-\alpha$, получим оценку:
 \begin{multline*}
  \Ep  \left|I\left[S_L\notin B\right]h[S_L]
  \left(\widehat{f}_{S_L}[S_L;\theta|s]-f_{S_L}[S_L;\theta] \right)\right|
  \le{}
\\
  {}\le\Op\left(n^\alpha\right)\,
  \Ep\left\{I\left[S_L\notin B\right]
  |h[S_L]|\right\}=\Op\left( n^{-5/2}\right).
  \end{multline*}
  Отсюда c~учетом следствия~6.1, леммы~5.12, формул~\eqref{4:3} и~\eqref{4:4}
  найдем:
 \begin{multline}
 \widehat{g}_n[\theta;z]-g[\theta]=
 \Ep  \left\{I\left[S_L\in B\right]h[S_L]
  \left(\widehat{f}_{S_L}[S_L;\theta|s]-\right.\right.\\
\left.\left.  {}-f_{S_L}[S_L;\theta] \right)\right\}+
  \Op\left( n^{-5/2}\right)={}\\
  {}=
\sum\limits_{j=1}^4
  \fr{1}{n^{j/2}}\,
  \Ep \left\{\widehat{g}_{L}[\theta|S_L] c_{jf}[z,L,\Delta[S_L]]\right\}+{}\\
  {}+
  \Op\left(n^{-5/2+\delta}\right).
  \label{T:6.2-2}
  \end{multline}
  Имея в~виду, что для любого $j\hm\in\Np_0$ согласно~\eqref{5.1}
   \begin{equation*}
 \Ep \left\{\widehat{g}_{L}[\theta|S_L] \Delta^j[S_L]\right\}=
 \fr{\nu_j[\theta;L]}{b^j},
 \end{equation*}
  и~подставляя в~\eqref{T:6.2-2} выражения коэффициентов
  $\left\{c_{jf}[\,\cdot\,],\ \overline{j=1,4}\right\}$ из теоремы~6.1, после элементарных преобразований получим~\eqref{T:6.2-1}.

  \smallskip

\noindent
\textbf{Следствие 6.2.}\
 В~условиях теоремы~6.2 справедливо следующее асимптотическое разложение
  функции $\widehat{G}_n[a;Z_n]$, определяющей несмещенную оценку функции
  $G[a]\hm=g\left[\Phi_1[a]\right]$:
  \begin{multline*}
 \widehat{G}_n[a;z]=G[a]+\fr{z G'[a]}{\sqrt{n\Phi_1'[a]}}+
 \fr{H_2[z] G''[a]}{2n \Phi_1'[a]} +{}\\
 {}+
 \sum\limits_{i=3}^4 \fr{c_{iG}[z]}{n^{i/2}}+\Op\left(n^{-5/2+\delta}\right)\,,
% \label{C:6.2-1}
 \end{multline*}
 где
 \begin{align*}
 c_{3G}[z]&=  \fr{z G''[a] \Phi_1''[a]}{2 (\Phi_1'[a])^{5/2}}+
 \fr{H_3[z] G^{(3)}[a]}{6 (\Phi_1'[a])^{3/2}}\,;
\\
 c_{4G}[z]&=  -\fr{H_2[z] G''[a] (\Phi_1''[a])^2}{2 (\Phi_1'[a])^4}+{}\\
& \hspace*{-8mm}{}+
 \fr{\left(3z^2-2\right) \Phi_1''[a] G^{(3)}[a]}{6 (\Phi_1'[a])^3}
 +\fr{H_2[z] G''[a] \Phi_1^{(3)}[a]}{4 (\Phi_1'[a])^3}+{}\\
&\hspace*{28mm} {}+
 \fr{\left(z^4-6 z^2+3\right) G^{(4)}[a]}{24 (\Phi_1'[a])^2}.
 \end{align*}


  Далее
  $\widehat{g^2_n}[\theta|S_n]\hm=\widehat{g^2_n}[\theta;Z_n]$ и~$\widehat{\Vp} \widehat{g}_n[\theta|S_n]\hm=\widehat{\Vp} \widehat{g}_n[\theta;Z_n]$~--- НОРМД соответственно функции $g^2[\theta]$ и~дисперсии несмещенной оценки $\Vp\widehat{g}_n[\theta|S_n]$ по выборке объема~$n$.
  Отметим, что оценка $\widehat{g^2_n}[\theta|S_n]$ заведомо существует при
  $n\hm> 2 L \hm\ge L_0,$ если существует дисперсия
  $\Vp \widehat{g}_n[\theta|S_n]$, поскольку
 \begin{equation*}
 \widehat{g^2_n}[\theta|S_n]=
 \Ep  \left\{
 \widehat{g}_{L,1}[\theta|S_{L,1}]\,
 \widehat{g}_{L,2}[\theta|S_{L,2}]|\, S_n \right\},
 \end{equation*}
 где
 $ S_{L,1}\hm=\sum\limits_{i=1}^L T[X_i]$;
 $S_{L,2}\hm=\sum\limits_{i=L+1}^{2L} T[X_i]$.

  Через
  $\widehat{G^2_n}[a|S_n]\hm=\widehat{G^2_n}[a;Z_n]$ и~$\widehat{\Vp} \widehat{G}_n[a|S_n]\hm=\widehat{\Vp}\widehat{G}_n[a;Z_n]$
  обозначим НОРМД соответственно функции $G^2[a]$ и~дисперсии несмещенной
  оценки $\Vp\widehat{G}_n[a|S_n]$.

\smallskip

\noindent
\textbf{Теорема 6.3.}\
\textit{Пусть выполнены условия $\mathbf{(A_1)}$ и~$\mathbf{(A_2)}$,
 ${\sf V}\, \widehat{g}_L[\theta|S_L]\hm<\infty$ при некотором $L\hm\in\mathbb{N}$,
 ${\sf V}\, \widehat{g^2_M}[\theta|S_M]\hm<\infty$ при некотором $M\hm\in\mathbb{N}$.
 Тогда для любых $\alpha\hm>0$, $|z|\hm\le \sqrt{2\alpha\ln n}$ и~$0\hm<\delta\hm<0{,}5$ при $n\hm\to\infty$ справедливо следующее асимптотическое разложение функции,
 определяющей несмещенную оценку дисперсии несмещенной оценки}
 $\widehat{g}_n[\theta;Z_n]$:
 \begin{equation}
 \widehat{\Vp} \widehat{g}_n[\theta;z]=
  \sum_{i=2}^4 \fr{c_{i\widehat{V}\widehat{g}}[z]}{n^{i/2}}+
  \Op\left( n^{-5/2+\delta}\right),
 \label{T:6.4-1}
 \end{equation}
\textit{где}
 \begin{align*}
  c_{2\widehat{V}\widehat{g}}[z]
&   =\fr{(g'[\theta])^2}{b^2}\,;\\
   c_{3\widehat{V}\widehat{g}}[z]&    =
   \fr{2z g'[\theta] g''[\theta]}{b^3}-\fr{z \rho_3 (g'[\theta])^2}{b^2}\,;
\end{align*}
 \begin{align*}
  c_{4\widehat{V}\widehat{g}}[z]
   &=\fr{(g'[\theta])^2
   \left(3 z^2 \rho_3^2-2\rho_3^2+\rho_4 -  z^2 \rho_4 \right)}{2 b^2}
  +{}\\
  &\hspace*{10mm}{}+\fr{g'[\theta] g''[\theta]\left(2 \rho_3 -3 z^2 \rho_3
  \right)}{b^3}+{}
\\
& \hspace*{5mm}{} +\fr{(g''[\theta])^2 (2z^2-1)}{2 b^4}
   +\fr{H_2[z] g'[\theta] g^{(3)}[\theta]}{b^4}.
 \end{align*}

  \Do Чтобы получить~\eqref{T:6.4-1}, достаточно
  заменить несмещенные оценки $\widehat{g}_n[\theta|S_n]$ 
  и~$\widehat{g^2_n}[\theta|S_n]$ в~хорошо известном представлении
 несмещенной оценки дисперсии
 \begin{equation*}
 \widehat{\Vp} \widehat{g}_n[\theta|S_n]=
 \left(\widehat{g}_n[\theta|S_n]\right)^2-\widehat{g^2_n}[\theta|S_n]
% \label{T:6.4-2}
 \end{equation*}
  на их асимптотические разложения.
  Разложение оценки $\widehat{g}_n[\theta|S_n]$ определяется формулой~\eqref{T:6.2-1},
  разложение оценки $\widehat{g^2_n}[\theta|S_n]$ может быть получено
  после замены $g[\theta]$ на $g^2[\theta]$ в~\eqref{T:6.2-1}.

  \smallskip

\noindent
\textbf{Следствие 6.3.}\
Пусть выполнены условия $\mathbf{(A_1)}$ и~$\mathbf{(A_2)}$,
 $\Vp \widehat{G}_L[a|S_L]\hm<\infty$ при некотором $L\hm\in\mathbb{N}$,
 ${\sf V} \widehat{G^2_M}[a|S_M]\hm<\infty$ при некотором $M\hm\in\mathbb{N}$.
 Тогда для любых $\alpha\hm>0$, $|z|\hm\le \sqrt{2\alpha\ln n}$
 и~$0\hm<\delta\hm<0{,}5$ при $n\hm\to\infty$ справедливо асимптотическое разложение функции,
 определяющей несмещенную оценку дисперсии несмещенной оценки
 $\widehat{\Vp}\widehat{G}_n[a;Z_n]$,
 \begin{equation*}
 \widehat{\Vp}\widehat{G}_n[a;z]=
 \sum\limits_{i=2}^4 \fr{c_{i\widehat{V}\widehat{G}}[z]}{n^{i/2}}+\Op\left( n^{-5/2+\delta}\right),
% \label{C:6.3-1}
 \end{equation*}
 где
  \begin{align*}
 c_{2\widehat{V}\widehat{G}}[z]&=\fr{(G'[a])^2}{\Phi_1'[a]}\,;\\
 c_{3\widehat{V}\widehat{G}}[z]&=\fr{2 z G'[a] G''[a]}{(\Phi_1'[a])^{3/2}}-
 \fr{z (G'[a])^2 \Phi_1''[a]}{(\Phi_1'[a])^{5/2}}\,;
\\
 c_{4\widehat{V}\widehat{G}}[z]&=
 \fr{\left(2z^2-1\right) (G''[a])^2+2H_2[z] G'[a] G^{(3)}[a]}
      {2(\Phi_1'[a])^2}-{}
\\
&\hspace*{-8mm}{} -\fr{ H_2[z] \left(4G'[a] G''[a] \Phi_1''[a]+
 (G'[a])^2 \Phi_1^{(3)}[a]\right)}{2(\Phi_1'[a])^3}+{}\\
 &\hspace*{30mm}{}+
 \fr{H_2[z] (G'[a])^2 (\Phi_1''[a])^2}{(\Phi_1'[a])^4}.
 \end{align*}

\section{Асимптотические разложения дисперсий
  несмещенных оценок}

\noindent
\textbf{Теорема~7.1.}\
\textit{Пусть выполнены условия $\mathbf{(A_1)}$ и~$\mathbf{(A_2)}$, а~также
 существуют $L\hm\in\mathbb{N}$ и~$\delta_0\hm>0$ такие, что
 $C_L\hm=\Ep \left|\widehat{g}_L[\theta|S_L]\hm-g[\theta] \right|^{2+\delta_0}\hm<\infty.$
 Тогда для любого $0\hm<\delta\hm<0{,}5$ при $n\hm\rightarrow\infty$ 
 справедливо асимптотическое разложение дисперсии несмещенной оценки}:
 \begin{multline*}
 \Vp \widehat{g}_n[\theta|S_n]=
 \fr{(g'[\theta])^2}{n\,b^2} +{}\\
 {}+
 \fr{\left(b \rho_3 g'[\theta]-g''[\theta ]\right)^2}{2 n^2 b^4}
 + \Op\left( n^{-5/2+\delta}\right).
% \label{T:7.1-1}
 \end{multline*}

 \Do
   Представим при произвольном $\alpha\hm>0$ дисперсию несмещенной оценки в~виде суммы
 \begin{equation}
 \Vp \widehat{g}_n[\theta|S_n]=J_1+J_2,
 \label{T:7.1-2}
 \end{equation}
 где
 \begin{align*}
  J_1& = \Ep \left(\left|\widehat{g}_n[\theta;Z_n]-g[\theta] \right|^2
  I\left[|Z_n|\le \sqrt{2\alpha\ln n}\right]\right);
\\
  J_2& =\Ep \left(\left|\widehat{g}_n[\theta;Z_n]-g[\theta] \right|^2
  I\left[|Z_n|> \sqrt{2\alpha\ln n}\right]\right).
 \end{align*}

 Используя неравенство Гельдера при $p\hm=1\hm+\delta_0/2$, леммы~5.10--5.11 и~теорему Рао--Блэк\-вел\-ла--Кол\-мо\-го\-ро\-ва, оценим
 \begin{multline}
 J_2\le \left\{\Ep \left|
  \widehat{g}_n[\theta;Z_n]-g[\theta] \right|^{2+\delta_0}\right\}^{1/p}\times{}\\
  {}\times
 \left\{\Ep I\left[|Z_n|> \sqrt{2\alpha\ln n}\right]\right\}^{(p-1)/p}\le{}
\\
{} \le
 C_L^{1/p} \mathbf{o}\left(n^{-\alpha\delta_0/(2+\delta_0)}\right)
  =\mathbf{o}\left(n^{-\alpha\delta_0/(2+\delta_0)}\right).
 \label{T:7.1-3}
 \end{multline}
  Используя леммы~5.10--5.11 и~теорему~6.2, представим
 \begin{multline}
 J_1 =\Ep \left\{\left( \sum\limits_{i=1}^4 \fr{c_{ig}[Z_n]}{n^{i/2}}+
 \Op\left(n^{-5/2+\delta}\right)\right)^2\times{}\right.\\
\left. {}\times
  I\left[|Z_n|\le \sqrt{2\alpha\ln n}\right]
  \vphantom{\left(\sum\limits_{i=1}^4\right)^2}\right\} ={}
\\
  {}=\Ep P_4[Z_n]+\Op\left(n^{-5/2+\delta}\right)+\op\left(n^{-\alpha}\right),
  \label{T:7.1-4}
 \end{multline}
 где
  \begin{multline*}
  P_4[z]=\fr{z^2 (g'[\theta])^2}{n\,b^2}
  +{}\\
  {}+\fr{(z^3-z)\left(
  g'[\theta] g''[\theta]-b \rho _3 (g'[\theta])^2\right)}{b^3 n^{3/2}}+{}
\\
 {}+\fr{1}{12n^2 b^4}\left\{
 \vphantom{+g'[\theta] g^{(3)}[\theta]\left(
  4 z^4-12 z^2 \right)}
 (g'[\theta])^2 b^2 \left(
  3 \rho _3^2-30 z^2 \rho _3^2+15 z^4 \rho _3^2
  +{}\right.\right.\\
\left.  {}+12 z^2 \rho _4 -4 z^4 \rho _4\right)+{}
 \\
  {}+g'[\theta] g''[\theta] b\left(36 z^2 \rho_3
  -6 \rho_3-18 z^4 \rho_3\right)+{}\\
  {}+
  (g''[\theta])^2\left(3-6 z^2+3 z^4 \right)+{}
\\
\left.{}  +g'[\theta] g^{(3)}[\theta]\left(
  4 z^4-12 z^2 \right)\right\}.
 \end{multline*}

 Применяя следствие~5.2, вычислим
 \begin{multline}
 \Ep P_{4}[Z_n]= \fr{(g'[\theta])^2}{n\,b^2}
 +{}\\
 {}+\fr{\left(b \rho_3 g'[\theta]-g''[\theta ]\right)^2}{2n^2 b^4}
 + \Op\left( n^{-5/2+\delta}\right).
  \label{T:7.1-5}
 \end{multline}

 Подставляя~\eqref{T:7.1-5} в~\eqref{T:7.1-2}--\eqref{T:7.1-4} 
 и~выбирая $\alpha\hm>0$ достаточно большим, завершим доказательство тео\-ре\-мы~7.1.

\vspace*{2pt}

\noindent
\textbf{Следствие 7.1.}\
 В~условиях теоремы~7.1
 дисперсия несмещенной оценки $\Vp \widehat{G}_n[a|S_n]$
 имеет следующее разложение:
 \begin{equation*}
\Vp \widehat{G}_n[a|S_n]=
 \fr{(G'[a])^2}{n\Phi_1'[a]}+
 \fr{(G''[a])^2}{2n^2 (\Phi'_1[a])^2}+ \Op\left(n^{-5/2+\delta}\right).
% \label{8:2}
 \end{equation*}


 Следуя ходу доказательства теоремы~7.1, но используя разложения более высокого порядка, можно получить следующее утверждение.

 \vspace*{2pt}

 \noindent
 \textbf{Теорема 7.2.}\
\textit{Пусть выполнены условия $\mathbf{(A_1)}$ и~$\mathbf{(A_2)}$, а~также
 существуют $L\hm\in\Np$ и~$\delta_0\hm>0$ такие, что}
 $$
 C_L=\Ep \left|\widehat{\Vp}\widehat{G}_L[a|S_L]-
 \Vp\widehat{G}_L[a|S_L] \right|^{2+\delta_0}<\infty.
 $$
 \textit{Тогда для любых $\alpha\hm>0$ и~$0\hm<\delta\hm<0{,}5$ при $|z|\hm\le \sqrt{2\alpha\ln n}$, $n\hm\rightarrow\infty$ 
 справедливо асимптотическое разложение дисперсии несмещенной оценки дисперсии}:
 \begin{equation*}
 \Ep\left(\widehat{\Vp}\widehat{G}_n[a|S_n]-
 \Vp\widehat{G}_n[a|S_n]\right)^2
 =\fr{c_{1}}{n^3}+\fr{c_{2}}{n^4}+
 \op\!\left(\fr{1}{n^{4 }}\right),
% \label{T:7.2-1}
 \end{equation*}
\textit{где}
  \begin{align*}
  c_{1}&=
 \fr{4 (G'[a] G''[a])^2}{(\Phi_1'[a])^3}-
 \fr{4 (G'[a])^3 G''[a] \Phi_1''[a]}{(\Phi_1'[a])^4}+{}\\
 &\hspace*{40mm}{}+
 \fr{(G'[a])^4 (\Phi_1''[a])^2}{(\Phi_1'[a])^5}\,;
\\
 c_{2}&=
 \fr{2 (G''[a])^4}{(\Phi_1'[a])^4}
 + \fr{2 (G'[a] G^{(3)}[a])^2}{(\Phi_1'[a])^4}
 - {}\\
 &\hspace*{30mm}{}-\fr{2 (G'[a])^3 G^{(3)}[a] \Phi_1^{(3)}[a]}{(\Phi_1'[a])^5}+{}\\
&\hspace*{-11.32532pt}{} +\fr{4 G'[a] (G''[a])^2 \left(-3 G''[a] \Phi_1''[a]+
 2 \Phi_1'[a] G^{(3)}[a]\right)}{(\Phi_1'[a])^5}-{}
\\
&\hspace*{-4.79753pt}{} -\fr{2 (G'[a])^2 G''[a] \left(5 \Phi_1''[a] G^{(3)}[a]+G''[a]
 \Phi_1^{(3)}[a]\right)}{(\Phi_1'[a])^5} +{}
 \end{align*}
  \begin{align*}
 &{}+\fr{14 (G'[a] G''[a] \Phi_1''[a])^2}{(\Phi_1'[a])^6}
 + \fr{(G'[a])^4 (\Phi_1^{(3)}[a])^2}{2 (\Phi_1'[a])^6}+{}
\\
&{} +\fr{4 (G'[a])^3 \Phi_1''[a] \left(\Phi_1''[a] G^{(3)}[a]+
  G''[a] \Phi_1^{(3)}[a]\right)}{(\Phi_1'[a])^6}-{}
\\
& \hspace*{20mm}{}- \fr{8 (G'[a])^3 G''[a] (\Phi_1''[a])^3}{(\Phi_1'[a])^7}
 -{}\\
 &{}-\fr{2 (G'[a])^4 (\Phi_1''[a])^2 \Phi_1{}^{(3)}[a]}{(\Phi_1'[a])^7}
 +\fr{2 (G'[a] \Phi_1''[a])^4}{(\Phi_1'[a])^8}.
 \end{align*}



 В заключение отметим, что предложенный в~работе подход
 позволяет получать для несмещенных оценок и~их характеристик
 асимптотические разложения любого порядка. Возникающая при этом проблема проведения громоздких преобразований может быть решена с~по\-мощью одной из систем аналитических вычислений.



{\small\frenchspacing
 {%\baselineskip=10.8pt
 \addcontentsline{toc}{section}{References}
 \begin{thebibliography}{99}
\bibitem{1-ch}
\Au{Воинов В.\,Г., Никулин М.\,С.} Несмещенные оценки и~их
  применения.~--- М.: Наука, 1989. 440~с.
 %2
\bibitem{2-ch}
\Au{Portnoy~S.} Asymptotic efficiency of minimum variance unbiased
 estimators~// Ann.  Stat., 1977. Vol.~5. No.\,3.  P.~522--529.
 %3
\bibitem{3-ch}
\Au{Lopez-Blazquez F., Salamanca-Mino~B.} Limit distribution of
 unbiased estimators in natural exponential families~// Statistics,
 2002. Vol.~14. No.\,4.  P.~329--338.
 %4
\bibitem{4-ch}
\Au{Blazquez F.\,L., Rubio D.\,G.} Unbiased estimation in the
  multivariate natural exponential family with simple quadratic variance function~// 
  J.~Multivariate Anal., 2003. Vol.~86. P.~1--13.
 %5
\bibitem{5-ch}
\Au{Morris C.\,N.} Natural exponential families with quadratic
 variance functions~// Ann. Stat., 1982. Vol.~10.
 No.\,1.  P.~65--80.
 %6
\bibitem{6-ch}
\Au{Hwang T.-Y., Hu~C.-Y.} More comparisons of MLE with UMVUE for
 exponential families~// Ann. Inst. Statist. Math., 1990. Vol.~42.
 P.~65--75.
 %7
\bibitem{7-ch}
\Au{Чичагов В.\,В. } Об асимптотическом поведении несмещенных
  оценок вероятностей для решетчатых распределений, достаточной статистикой
  которых является среднее~// Статистические методы оценивания и~проверки гипотез: Межвуз. сб. науч. тр.~--- Пермь: ПГУ, 2002. C.~106--120.
 %8
\bibitem{8-ch}
\Au{Chichagov V.\,V.} Concerning asymptotic normality
  of a~class of unbiased estimators in the case of absolutely continuous
  distributions. Statistical methods of estimation and testing of hypotheses~//
  J.~Math. Sci., 2004. Vol.~119. No.\,3. P.~336--341.



\bibitem{10-ch} %9
\Au{Петров В.\,В. } Суммы независимых случайных величин.~--- М.: Наука, 1972.
416~с.

\bibitem{9-ch} %10
\Au{Барндорф-Нильсен~О., Кокс~Д.} Асимптотические методы в~математической статистике~/
Пер. с~англ.~--- М.: Мир, 1999. 255~с. (\Au{Barndorff-Nielsen~O.\,E.,
Cox~D.\,R.} Asymptotic techniques for use in statistics.~---
London: Chapman and Hall, 1989. 252~p.)

 %11
\bibitem{11-ch}
\Au{Лумельский Я.\,П., Сапожников~П.\,Н.} Несмещенные оценки
  для плотностей распределений~// Теория вероятностей и~ее применение,
  1969.  T.~14. №\,2.  С.~372--380.
 %12
\bibitem{12-ch}
\Au{Чичагов В.\,В.} Стохастические разложения несмещенных оценок в~случае однопараметрического экспоненциального семейства~//
 Информатика и~её применения, 2008. Т.~2. Вып.~2. С.~62--70.
 %13
\bibitem{13-ch}
\Au{Чичагов В.\,В.} О~несмещенной оценке вероятности
  ${\sf P}(X<Y)$ в~модели на\-груз\-ка--проч\-ность~//
  Теория вероятностей  и~ее приложения: Тезисы докл.
  Междунар. конф., посвященной 100-ле\-тию со дня
  рождения Б.\,В.~Гнеденко.~--- М.: ЛЕНАНД, 2012. С.~264.
 %14
\bibitem{14-ch}
\Au{Chichagov~V.} Asymptotic of the mean absolute error of UNVUE
 and MLE in the case of one-parameter exponential family lattice
 distributions~// 31st  Seminar (International) on Stability Problems for
 Stochastic Models: Book of abstracts.~--- Moscow: IPI RAN, 2013. P.~13--15.
 %15
\bibitem{15-ch}
\Au{J\!\mbox{{\fontsize{10pt}{10pt}\selectfont\ptb{\!\!\o}}}\,rgensen B.} The theory of dispersion models.~---  London: Chapman \& Hall, 1997. 256~p.
 %16
\bibitem{16-ch}
\Au{Brown L.\,D.} Fundamentals of statistical exponential
 families with applications in statistical decision theory~//
 Lecture notes~--- monograph ser.~--- 
 Hayward, CA, USA: Institute of Mathematical Statistics, 1986.  Vol.~9. 284~p.
 %17
\bibitem{17-ch}
\Au{Петров В.\,В. } Предельные теоремы для сумм независимых случайных величин.~--- М.: Наука, 1987. 320~с.
 \end{thebibliography}

 }
 }

\end{multicols}

\vspace*{-9pt}

\hfill{\small\textit{Поступила в~редакцию 10.04.14}}

%\newpage

\vspace*{12pt}

\hrule

\vspace*{2pt}

\hrule

%\vspace*{12pt}

\def\tit{HIGHER-ORDER ASYMPTOTIC EXPANSIONS OF~UNBIASED ESTIMATORS AND~THEIR VARIANCES ON~THE~ONE-PARAMETER EXPONENTIAL FAMILY MODEL}

\def\titkol{Higher-order asymptotic expansions of unbiased estimators and
their variances on the one-parameter exponential family model}

\def\aut{V.\,V.~Chichagov}

\def\autkol{V.\,V.~Chichagov}

\titel{\tit}{\aut}{\autkol}{\titkol}

\vspace*{-9pt}


\noindent
Perm State University, 15~Bukireva Str., Perm 614990, Russian Federation



\def\leftfootline{\small{\textbf{\thepage}
\hfill INFORMATIKA I EE PRIMENENIYA~--- INFORMATICS AND
APPLICATIONS\ \ \ 2015\ \ \ volume~9\ \ \ issue\ 3}
}%
 \def\rightfootline{\small{INFORMATIKA I EE PRIMENENIYA~---
INFORMATICS AND APPLICATIONS\ \ \ 2015\ \ \ volume~9\ \ \ issue\ 3
\hfill \textbf{\thepage}}}

\vspace*{3pt}


\Abste{The paper considers a model of duplicate sampling with the fixed size $n$ from a~distribution belonging to the natural one-parameter exponential family.
A~limiting behavior of the uniformly minimum variance unbiased\linebreak\vspace*{-12pt}}

\Abstend{estimator (UMVUE) of the 
given parametric function and the UMVUE variance of this estimator is studied in the 
case of infinite size of the sample. Higher-order asymptotic expansions are obtained 
for functions defining unbiased estimators and variances of these estimators. 
The results are presented for both the canonical parameterization and the mean 
parameterization.}


\KWE{natural exponential family; unbiased estimate; asymptotic expansion}

\DOI{10.14357/19922264150308}

\vspace*{-6pt}

\Ack
\noindent
The research was financially supported by the
Russian Ministry of Education and Science (project No.\,2096).

%\vspace*{3pt}

  \begin{multicols}{2}

\renewcommand{\bibname}{\protect\rmfamily References}
%\renewcommand{\bibname}{\large\protect\rm References}

{\small\frenchspacing
 {%\baselineskip=10.8pt
 \addcontentsline{toc}{section}{References}
 \begin{thebibliography}{99}
\bibitem{1-ch-1}
\Aue{Voinov, V.\,G., and M.\,S.~Nikulin}. 1989.
\textit{Nesmeshchennye otsenki i~ikh primeneniya}
 [Unbiased estimators and their applications].
 Moscow: Nauka. 440~p. \emph{ }
 %2
\bibitem{2-ch-2}
\Aue{Portnoy, S.} 1977.
 Asymptotic efficiency of minimum variance unbiased estimators.
\textit{Ann. Stat.} 5(3):522--529.
 %3
\bibitem{3-ch-1}
\Aue{Lopez-Blazquez, F., and B.~Salamanca-Mino}. 2002.
 Limit distribution of unbiased estimators in natural exponential families.
\textit{Statistics} 14(4):329--338.
 %4
\bibitem{4-ch-1}
\Aue{Blazquez, F.\,L., and D.\,G.~Rubio}. 2003.
 Unbiased estimation in the multivariate natural exponential family with simple quadratic variance function.
\textit{J.~Multivariate Anal.} 86:1--13.
 %5
\bibitem{5-ch-1}
\Aue{Morris, C.\,N.} 1982.
 Natural exponential families with quadratic variance functions.
\textit{Ann. Stat.} 10(1):65--80.
 %6
\bibitem{6-ch-1}
\Aue{Hwang, T.-Y., and C.-Y.~Hu}. 1990.
 More comparisons of MLE with UMVUE for exponential families.
\textit{Ann. Inst. Statist. Math.} 42(1):65--75.
 %7
\bibitem{7-ch-1}
\Aue{Chichagov, V.\,V.} 2002.
 Ob asimptoticheskom povedenii nesmeshchennykh otsenok veroyatnostey dlya reshetchatykh raspredeleniy,
 dostatochnoy statistikoy kotorykh yavlyaetsya srednee
 [On asymptotic behavior of unbiased probability estimators for lattice distributions with the mean
 as a sufficient statistic].
\textit{Statisticheskie Metody Otsenivaniya i~Proverki Gipotez: 
Mezhvuzovskiy Sbornik Nauchnykh Trudov}
[Statistical methods for estimating and hypothesis
 testing: Interuniversity Collection of Research Papers].
 Perm. 16:106--120.
 %8
\bibitem{8-ch-1}
\Aue{Chichagov, V.\,V.} 2004.
 Concerning asymptotic normality of a class of unbiased estimators in the case of absolutely continuous
 distributions. Statistical methods of estimation and testing of hypotheses.
\textit{J. Math. Sci.} 119(3):336--341.



\bibitem{10-ch-1}
\Aue{Petrov, V.\,V.} 1972.
\textit{Summy nezavisimykh sluchaynykh velichin}
 [Sums of independent random variables].
 Moscow: Nauka. 416~p.
 
 \bibitem{9-ch-1} %10
\Aue{Barndorff-Nielsen,~O.\,E., and D.\,R.~Cox.} 1989. \textit{Asymptotic techniques for use in statistics}.
London: Chapman and Hall. 252~p.

 %11
\bibitem{11-ch-1}
\Aue{Lumel'skiy, Ya.\,P., and P.\,N.~Sapozhnikov}. 1969.
 Nesmeshchennye otsenki dlya plotnostey raspredeleniy
 [Unbiased estimators for distribution densities].
\textit{Teoriya veroyatnostey i~ee primenenie}
[Theory Probab. Appl.] 14(2):372--380.
 %12
\bibitem{12-ch-1}
\Aue{Chichagov, V.\,V.} 2008.
 Stokhasticheskie razlozheniya nesmeshchennykh otsenok v~sluchae odnopara\-met\-ri\-che\-sko\-go eksponentsial'nogo semeystva
 [Stochastic expansions of unbiased estimators for the case of one-parameter exponential family].
\textit{Informatika i~ee primeneniya}~--- \textit{Inform. Appl.} 2(2):62--70.
 %13
\bibitem{13-ch-1}
\Aue{Chichagov, V.\,V.} 2012.
 O~nesmeshchennoy otsenke ve\-ro\-yat\-nosti ${\sf P}(X<Y)$ v~modeli nagruzka--prochnost'
 [An unbiased estimator of the probability ${\sf P}(X<Y)$ in stress--strength model].
\textit{Teoriya veroyatnostey  i~ee prilozheniya: Tezisy dokl.
 Mezhdunar. konf., posvyashchennoy 100-letiyu so dnya
  rozhdeniya B.\,V.~Gnedenko}
 [Conference (International) ``Probability Theory and Its Applications''
 in Commemoration of the Centennial of B.\,V.~Gnedenko].
 Moscow: LENAND. 264.
 %14
\bibitem{14-ch-1}
\Aue{Chichagov, V.} 2013.
 Asymptotic of the mean absolut error of UNVUE and MLE in the case of one-parameter exponential family lattice
 distributions.
\textit{31st  Seminar (International) on Stability Problems for
 Stochastic Models: Book of Abstracts.}
 Moscow. 13--15.
 %15
\bibitem{15-ch-1}
\Aue{J\mbox{{\fontsize{10pt}{10pt}\selectfont\ptb{\!\o}}}rgensen, B.} 1997.
\textit{The theory of dispersion models.}
 London: Chapman and Hall/CRC. 256~p.
 %16
\bibitem{16-ch-1}
\Aue{Brown, L.\,D.} 1986.
\textit{Fundamentals of statistical exponential families with applications in statistical decision theory.}
 Lecture notes~--- monograph ser. 
 Hayward, CA: Institute of Mathematical Statistics. Vol.~9. 284~p.
 %17
\bibitem{17-ch-1}
\Aue{Petrov, V.\,V.} 1987.
\textit{Predel'nye teoremy dlya summ nezavisimykh sluchaynykh velichin}
 [Limit theorems for sums of independent random variables].
 Moscow: Nauka. 320~p.
\end{thebibliography}

 }
 }

\end{multicols}

\vspace*{-6pt}

\hfill{\small\textit{Received April 10, 2014}}






\Contrl

\noindent
\textbf{Chichagov Vladimir V.} (b.\ 1955)~---
 Candidate of Science (PhD) in physics and mathematics, associate professor, 
 Perm State University, 15~Bukireva Str., Perm 614990, Russian Federation; 
 chichagov@psu.ru

\label{end\stat}


\renewcommand{\bibname}{\protect\rm Литература}