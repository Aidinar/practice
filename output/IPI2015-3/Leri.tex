\def\stat{leri}

\def\tit{ПОЖАР НА КОНФИГУРАЦИОННОМ ГРАФЕ СО~СЛУЧАЙНЫМИ ПЕРЕХОДАМИ ОГНЯ ПО~РЕБРАМ$^*$}

\def\titkol{Пожар на конфигурационном графе со~случайными переходами огня по~ребрам}

\def\aut{М.\,М.~Лери$^1$}

\def\autkol{М.\,М.~Лери}

\titel{\tit}{\aut}{\autkol}{\titkol}

{\renewcommand{\thefootnote}{\fnsymbol{footnote}} \footnotetext[1]
{Работа выполнена при поддержке РФФИ (проект 13-01-00009).}}


\renewcommand{\thefootnote}{\arabic{footnote}}
\footnotetext[1]{Институт прикладных математических
исследований КарНЦ РАН, leri@krc.karelia.ru}

\Abst{Рассматривается случайный процесс распространения огня по реб\-рам двух
видов конфигурационных графов со случайными степенями вершин. Степени вершин имеют
либо степенное распределение, либо распределение Пуассона. Процесс распространения
огня происходит в~случайной среде, в~которой вероятности перехода огня по реб\-рам
графа имеют стандартное равномерное распределение. Оценены оптимальные значения параметров
распределений степеней вершин, обеспечивающие максимальную сохранность вершин графа
при двух типах возгорания: целенаправленном поджоге, когда пожар начинается с~вершины,
имеющей наибольшую степень, и~случайном возгорании равновероятно выбранной вершины.
В~условиях одинаковых объемов рассматриваемых графов проведен сравнительный анализ
степенной модели и~модели с~пуассоновским распределением с~точки зрения чис\-ла
уцелевших в~пожаре вершин при разных условиях возгорания.}

\KW{конфигурационные графы; степенное распределение;
распределение Пуассона; устойчивость; модель лесного пожара}

\DOI{10.14357/19922264150307}



\vskip 14pt plus 9pt minus 6pt

\thispagestyle{headings}

\begin{multicols}{2}

\label{st\stat}

\section{Введение}

Одним из актуальных направлений в~теории случайных графов в последние годы
стали исследования их устойчивости к~разрушающим воздействиям, которые
передаются по реб\-рам графов~[1--4]. Такое воздействие может
рассматриваться как распространение огня во время лесного пожара (см.,
например,~[5--7]), а~также при изучении банковских
кризисов~\cite{Ari}.
В~[9--11] были построены модели на конфигурационных
графах, впервые предложенных в~\cite{Bol1} и~широко используемых для изучения структуры
и~динамики сложных сетей коммуникаций, например сети Интернет и~систем мобильной
связи~\cite{Dur, Hof}. Наблюдения за реальными сетями (см., например,~\cite{Fa,RN1}) показали, что
в~таких моделях степени вершин графов можно
считать независимыми одинаково распределенными случайными величинами. Кроме того,
чис\-ло вершин степени~$k$ при больших значениях~$k$ пропорционально
$k^{-(\tau+1)}$, где $\tau\hm>0$. Это означает, что распределение случайной
величины~$\xi$,
равной степени произвольной вершины, можно задать следующим образом:
\begin{equation}
\label{EQ:1}
\P\{\xi \geqslant k\} = h(k)k^{-\tau}\,,
\end{equation}
где $k=1,2,\dots$, а~$h(k)$~--- медленно меняющаяся функция. Такое
распределение означает, что граф не имеет изолированных вершин. Изучение
свойств конфигурационных графов с~распределением степеней вершин~(\ref{EQ:1})
показало~\cite{RN1}, что при $k\hm\rightarrow\infty$ предельное поведение важнейших
чис\-ло\-вых характеристик не зависит от выбора функции $h(k)$. Поэтому при
исследовании динамики случайных графов с~растущим чис\-лом вершин можно ограничиться
простейшим случаем $h(k)\hm\equiv const$.

\vspace*{-4pt}

\section{Конфигурационные графы со~случайными степенями вершин}

\vspace*{-1pt}

Рассмотрим конфигурационные графы с~$N$~вершинами, степени которых
$\xi_1,\xi_2,\ldots,\xi_N$ являются независимыми одинаково распределенными
случайными величинами с~общим распределением. В~настоящей работе рассматриваются
два варианта распределений степеней вершин графа. Во-пер\-вых,
это степенн$\acute{\mbox{о}}$е распределение,
заданное ра\-вен\-ством
\begin{equation}
\label{EQ:2}
\P\{\xi \geqslant k\} = k^{-\tau}\,, \enskip  k=1,2,\ldots\,, \ \tau > 1\,;
\end{equation}
во-вто\-рых, распределение Пуассона с единичным сдвигом:

\noindent
\begin{equation}
\label{EQ:3}
\P\{\xi = k+1\} = \fr{\lambda^k}{k!}\, e^{-\lambda}\,, \enskip
k=0,1,\ldots\,, \ \lambda > 0\,.
\end{equation}

\begin{figure*}[b] %fig1
 \vspace*{1pt}
 \begin{center}
 \mbox{%
 \epsfxsize=146.199mm
 \epsfbox{ler-1.eps}
 }
\end{center}
 \vspace*{-9pt}
\Caption{Фрагменты топологий вспомогательного графа при $m\hm=6{,}53$~(\textit{а})
и~3~(\textit{б})}
\end{figure*}

При построении графа сначала определяются степени вершин $1,2,\ldots,N$ как
реализации случайной величины $\xi$, имеющей одно из двух рассматриваемых
распределений~(\ref{EQ:2}) или~(\ref{EQ:3}). Будем считать (следуя \cite{RN1}), что из каждой
вершины~$i$, $i\hm=1,\ldots,N$, выходит~$\xi_i$~так называемых <<по\-лу\-ре\-бер>>,
т.\,е.\ ре\-бер графа, для которых смежные вершины еще не определены.
Все по\-лу\-реб\-ра считаются
различными (пронумерованными). В~любом графе сумма степеней вершин должна быть четной,
поэтому, если величина $\eta_N\hm=\xi_1+\cdots +\xi_N$ оказывается нечетной,
к~равновероятно
выбранной вершине добавляется еще одно по\-лу\-реб\-ро. В~\cite{RN1} отмечается, что
введение такого вспомогательного по\-лу\-реб\-ра не влияет на поведение графа
при больших значениях~$N$. Далее попарными равновероятными соединениями по\-лу\-ре\-бер
образуются реб\-ра графа, завершая его построение. Нетрудно заметить, что такие
графы могут содержать кратные реб\-ра и петли.

Структура описанной конструкции конфигурационного графа и~ее
асимптотическое поведение в~случае, когда распределение степеней вершин имеет
вид~(\ref{EQ:2}), подробно изучены в~\cite{RN1}, где считалось, что параметр~$\tau$
распределения~(\ref{EQ:2}) принадлежит интервалу $(1,2)$. Этот выбор интервала
объясняется тем, что именно такие значения~$\tau$ типичны для сложных
сетей коммуникаций, к~чис\-лу которых относится сеть Интернет.

\vspace*{-4pt}

\section{Исследование устойчивости: модель лесного пожара}

\vspace*{-1pt}

Моделирование <<лесного пожара>> в~настоящей работе производилось следующим
образом. Будем рассматривать вершины конфигурационного графа как деревья, растущие
на некоторой ограниченной территории. Ребра графа означают возможные пути перехода
<<огня>> от одной вершины к~другой.\linebreak В~[9--11] для графов со степенным
распределением степеней вершин~(\ref{EQ:2}) была решена задача нахождения оптимальных
значений параметра~$\tau$, зависящих от объема графа~$N$ и~от заданных вероятностей
перехода огня по реб\-ру~$p$, где $0\hm<p\hm\leq 1$, при которых чис\-ло уцелевших при пожаре
вершин максимально.
С~целью согласования топологии кон\-фи\-гу\-ра\-ционного графа с моделью <<лесного пожара>>
введем вспомогательные графы следующего вида. Будем считать, что вершины графа находятся
в~узлах квад\-рат\-ной це\-ло\-чис\-лен\-ной решетки, как показано на рис.~1 для
некоторых значений средней степени внутренней вершины (обозначим ее~$m$).

\begin{figure*} %fig2
 \vspace*{1pt}
 \begin{minipage}[t]{79mm}
 \begin{center}
 \mbox{%
 \epsfxsize=77.828mm
 \epsfbox{ler-2.eps}
 }
\end{center}
 \vspace*{-9pt}
\Caption{Регрессионная зависимость $\tau$ от~$N$}
%\end{figure*}
%\begin{figure*} %fig3
\end{minipage}
\hfill
 \vspace*{1pt}
  \begin{minipage}[t]{80mm}
 \begin{center}
 \mbox{%
 \epsfxsize=78.612mm
 \epsfbox{ler-3.eps}
 }
\end{center}
 \vspace*{-9pt}
\Caption{Регрессионная зависимость $\lambda$ от~$N$}
\end{minipage}
\end{figure*}

В настоящей работе рассматривается решетка размера $100\times 100$,
чтобы чис\-ло вершин было достаточно большим. Таким образом, общее чис\-ло
<<деревьев>> на некоторой ограниченной <<территории>> не превосходит
10\,000.
Ребра графа соединяют только ближайшие вершины, т.\,е.\ находящиеся на
расстоянии~1 друг от друга по горизонтали и~вертикали и~на расстоянии~$\sqrt{2}$
по диагонали.
Следовательно, степень любой вершины, находящейся внутри решетки, будет не более~8.
Использование вспомогательных графов обусловлено не\-об\-хо\-ди\-мостью выявления
зависимости между па\-ра\-мет\-ра\-ми распределений степеней вершин~($\tau$ для~(\ref{EQ:2})
и~$\lambda$ для~(\ref{EQ:3})) и~чис\-лом исходных вершин конфигурационного графа~$N$.
Для установления соответствия вспомогательного решетчатого графа описанному ранее
конфигурационному графу были вы\-чис\-ле\-ны средние значения $m\hm=m(N)$ степеней вершин при
различных значениях $N\hm\leq 10\,000$ и различных вариантах размещения~$N$ вершин на
решетке~\cite{Ler3}. Число вершин конфигурационного графа~$N$ в~данном случае равно
чис\-лу вершин у~соответствующего ему графа на решетке. Нетрудно проверить,
что математическое
ожидание распределения~(\ref{EQ:2}) равно $\zeta(\tau)$, где $\zeta(\tau)$~---
дзе\-та-функ\-ция
Римана. Принимая во внимание тот факт, что равенство $m(N)\hm=\zeta(\tau)$ должно выполняться
хотя бы приблизительно, была найдена регрессионная зависимость~$\tau$ от~$N$
(рис.~2) следующего вида (с коэффициентом детерминации $R^2\hm=0{,}87$):
\begin{equation}
\label{EQ:4}
\tau = 16\,813N^{-1{,}067}\,.
\end{equation}
Кроме того, поскольку математическое ожидание распределения~(\ref{EQ:3})
равно $\lambda \hm+1$, на основе равенства $m(N)\hm=\lambda \hm+1$ (рис.~3)
была найдена зависимость параметра~$\lambda$ от~$N$:
\begin{equation}
\label{EQ:5}
\lambda = 0{,}0011N-3{,}61\,,
\end{equation}
где $N\geqslant 3350$ и~$R^2\hm=0{,}98$.


Уравнения (\ref{EQ:4}) и~(\ref{EQ:5}) позволяют находить значения
параметров~$\tau$ и~$\lambda$ распределений степеней вершин конфигурационного
графа, наиболее близко соответствующие распределению~$N$~вершин на квадратной
решетке. Отличие конфигурационного графа от решетчатого состоит еще и~в~том, что
как распределение~(\ref{EQ:2}), так и~распределение~(\ref{EQ:3}) допускают сколь
угодно большие степени вершин, тогда как на решетке степень любой вершины не
превосходит~8. Однако это отличие можно считать не очень существенным по двум причинам.
Во-пер\-вых, вероятности больших степеней чрезвычайно малы и~такие вершины
встречаются довольно редко. Во-вто\-рых, кратные реб\-ра, соединяющие две вершины
конфигурационного графа, существенно увеличивают степени этих вершин, в~то
время как в~решетчатом графе между двумя соседними вершинами существует только
одно реб\-ро, означающее возможность перехода огня непосредственно от
одной вершины к~другой. Все вышесказанное поз\-во\-ля\-ет предположить, что соответствие
между двумя видами графов можно использовать для оценки значений
параметров~$\tau$ и~$\lambda$ распределений~(\ref{EQ:2}) и~(\ref{EQ:3})
с~помощью полученных
регрессионных уравнений~(\ref{EQ:4}) и~(\ref{EQ:5}) соответственно в~зависимости
от исходного чис\-ла вершин~$N$. В~этом случае параметр распределения степеней вершин
определяет не только точный вид распределения, но и~в~ка\-кой-то мере характеризует
топологию рассматриваемого конфигурационного графа.

Установленная связь между параметрами~$\tau$ и~$\lambda$ распределений~(\ref{EQ:2})
и~(\ref{EQ:3}) и~чис\-лом вершин на решетке позволяет найти зависимости чис\-ла~$N$
исходных вершин конфигурационных графов от параметров распределений степеней вершин.
Для распределения~(\ref{EQ:2}) была получена следующая модель~\cite{Ler1}:
\begin{equation}
\label{EQ:10}
N = \left[9256\tau^{-1,05}\right]\,,
\end{equation}
а~для распределения~(\ref{EQ:3}):
\begin{equation}
\label{EQ:11}
N = \left[907{,}5\lambda+2509{,}4\right]\,.
\end{equation}
Коэффициенты детерминации уравнений~(\ref{EQ:10}) и~(\ref{EQ:11})
равны~0,97 и~0,98 соответственно.

\begin{figure*} %fig4
 \vspace*{1pt}
  \begin{center}
 \mbox{%
 \epsfxsize=158.949mm
 \epsfbox{ler-4.eps}
 }
\end{center}
 \vspace*{-9pt}
\Caption{Зависимость чис\-ла несгоревших вершин~$g$ от параметра~$\tau$:
(\textit{a})~при
случайном возгорании, $0{,}2\hm\leq p\hm\leq 1$;
(\textit{б})~при направленном поджоге, $0\hm<p\hm\leq 1$}
\vspace*{6pt}
\end{figure*}


\begin{figure*}[b] %fig5
 \vspace*{7pt}
 \begin{center}
 \mbox{%
 \epsfxsize=159.045mm
 \epsfbox{ler-6.eps}
 }
\end{center}
 \vspace*{-9pt}
\Caption{Зависимость чис\-ла несгоревших вершин~$g$ от параметра~$\lambda$, 
$0\hm<p\hm\leq 1$: (\textit{а})~при случайном возгорании;
(\textit{б})~при направленном поджоге}
%\end{figure*}
\end{figure*}

Далее рассматривался процесс распространения <<огня>> по реб\-рам конфигурационных
графов двух типов: степенного и~Пуассона. Предполагались две возможности <<начала
пожара>>~--- случайное возгорание, при котором первой <<загорается>> равновероятно
выбранная вершина графа, и~це\-ле\-на\-прав\-лен\-ный поджог вершины, имеющей максимальную степень.
При возгорании первой вершины огонь переходит по реб\-рам на смежные вершины с~вероятностью
$0\hm<p\hm\leq 1$, т.\,е.\ с~вероятностью $1\hm-p$ реб\-ро становится несгораемым и~огонь на смежную
вершину не переходит. В~настоящей работе, в~отличие от
предшествующих~[9--11],
процесс распространения огня происходит в~так называемой случайной среде.
Это означает,
что вероятность~$p$ не является фиксированной величиной, она равномерно распределена на
интервале $(0,1]$ и~своя для каждого реб\-ра.
Цель работы состояла в~том, чтобы для каждого типа графа (степенн$\acute{\mbox{о}}$го
и~Пуассона) выяснить,
какой из двух вариантов возгорания (случайное или целенаправленное) и~при каких значениях
параметров~$\tau$ и~$\lambda$ обеспечивает наибольшее чис\-ло уцелевших при пожаре вершин.

Имитационные эксперименты проводились на графах, задаваемых параметрами распределений
степеней вершин $1\hm<\tau\hm\leq 3{,}5$ и~$0{,}3\hm\leq\lambda\hm\leq 3$,
объемы которых вы\-чис\-ля\-лись
с~по\-мощью уравнений~(\ref{EQ:10}) и~(\ref{EQ:11}). Для по\-стро\-ения графов был использован
алгоритм, описанный в~\cite{Tang}, и~датчик псевдослучайных чисел Mersenne
twister~\cite{MT}.
Были построены уравнения зависимости чис\-ла сохранившихся после пожара вершин $g$ от
параметра распределения степеней вершин $\tau$. 

В~случае случайного возгорания была получена
следующая модель ($R^2\hm=0{,}99$):
\begin{equation}
\label{EQ:6}
g = \left[2705{,}42\tau - 679{,}52\tau^2 + 1040{,}43\right]\,,
\end{equation}
а при целенаправленном поджоге ($R^2\hm=0{,}92$):
\begin{equation}
\label{EQ:7}
g = \left[917\tau - 408{,}43\tau^2 + 3955{,}53\right]\,.
\end{equation}

\begin{figure*}[b] %fig6
\vspace*{1pt}
 \begin{center}
 \mbox{%
 \epsfxsize=163.589mm
 \epsfbox{ler-8.eps}
 }
\end{center}
 \vspace*{-9pt}
\Caption{Зависимость величины~$V$ от объема графа~$N$,
$0\hm<p\hm\leq 1$: (\textit{а})~при случайном возгорании; 
(\textit{б})~при направленном поджоге}
\end{figure*}

Рисунок~4 отражает зависимости чис\-ла сохранившихся после пожара вершин~$g$ от
параметра распределения степеней вершин~$\tau$.

На рис.~4,\,\textit{а} видно, что при случайном возгорании облако регрессии распадается на~2~части. Это происходит вследствие того, что при одних и тех же значениях параметра распределения
степеней вершин $\tau$ существует некоторая доля графов, которые почти не сгорают,
 т.\,е.\ после
пожара в~них остается более 98\% вершин. Такая ситуация возникает, когда возгорание начинается
с~вершины, являющейся \mbox{частью} очень маленькой компоненты связности (не более~5~вершин), 
и, следовательно, огонь не достигнет остальных вершин графа.

Построенные уравнения~(\ref{EQ:6}) и~(\ref{EQ:7}) позволяют найти оптимальные значения
параметра~$\tau$, обеспечивающие максимальную сохранность вершин. Так, в случае направленного поджога оно
оказалось равным 1,99, а~при случайном возгорании~--- 1,12.
Следует отметить тот факт, что
было введено ограничение снизу вероятности перехода огня по реб\-ру значением~0,2,
которое
было обусловлено тем, что при меньших значениях распад облака регрессии
увеличивается
и~становится невозможным нахождение оптимального значения параметра~$\tau$
в~пределах интервала его изменения.

Аналогичные эксперименты были проведены на конфигурационных графах с~пуассоновским
распределением степеней вершин~(\ref{EQ:5}). Были получены следующие модели
зависимости
чис\-ла уцелевших вершин~$g$ от параметра~$\lambda$: при случайном возгорании
(рис.~5,\,\textit{а})
\begin{equation}
\label{EQ:8}
g = \left[1510{,}75\lambda - 747{,}81\lambda^2 + 2552{,}62\right]\,,
\end{equation}
где $R^2=0{,}97$, и при целенаправленном поджоге (рис.~5,\,\textit{б}) 
\begin{equation}
\label{EQ:9}
g = \left[2089{,}81\lambda - 779{,}04\lambda^2 + 2150\right]\,,
\end{equation}
где $R^2\hm=0{,}92$.



Из~(\ref{EQ:8}) и~(\ref{EQ:9}) нетрудно получить, что оптимальное значение
параметра~$\lambda$, обеспечивающее максимальную сохранность вершин графа
при пожаре в~случае
направленного поджога равно~1,01, а~при случайном возгорании~--- 1,34.

\vspace*{-4pt}

\section{Исследование устойчивости: сравнение топологий}

\vspace*{-1pt}

Далее рассматривалась задача сравнения двух моделей конфигурационных графов:
со степенным распределением~(\ref{EQ:2}) и~с~распределением Пуассона~(\ref{EQ:3}).
Имитационные эксперименты проводились на графах одинаковых объемов
$3000\hm\leq N\hm\leq 10\,000$ с~разными топологиями взаимного расположения вершин
и~ре\-бер, зависящими от заданного распределения степеней вершин. Параметры~$\tau$
и~$\lambda$ распределений~(\ref{EQ:2}) и~(\ref{EQ:3}) находились из
уравнений~(\ref{EQ:4}) и~(\ref{EQ:5}). Шаг изменения
объема графа~$N$ был равен~100.
С~целью сравнения топологий находились величины $V \hm= V(N)$, равные разности
между чис\-лом оставшихся после пожара вершин в~модели~(\ref{EQ:2})
и~в~модели~(\ref{EQ:3}). На рис.~6,\,\textit{а} приведены значения~$V$ в~зависимости от~$N$
при случайном возгорании, а~на рис.~6,\,\textit{б}~--- при целенаправленном поджоге.




Проведенное исследование показало, что как при случайном возгорании, так и~в~случае
на\-прав\-лен\-но\-го поджога степенной закон распределения степеней вершин~(\ref{EQ:2})
обеспечивает лучшую сохранность вершин во время пожара по сравнению с~законом
распределения Пуассона~(\ref{EQ:3}). Полученные в~настоящей работе результаты
для целенаправленного поджога отличаются от результатов сравнения топологий,
приведенных в~\cite{Ler4}, где рассматривалась модель <<лесного пожара>> при заданной
фиксированной вероятности~$p$ перехода огня по реб\-рам. В~\cite{Ler4} было показано, что
при некоторых значениях~$p$ и~$N$ пуассоновский закон распределения степеней вершин графа
обеспечит их лучшую сохранность во время пожара по сравнению со степенным законом.

\smallskip

Автор выражает благодарность профессору Ю.\,Л.~Павлову за помощь в~постановке задачи
и~обсуждение полученных результатов.

\vspace*{-4pt}

{\small\frenchspacing
 {%\baselineskip=10.8pt
 \addcontentsline{toc}{section}{References}
 \begin{thebibliography}{99}
 
 \vspace*{-1pt}
 
 \bibitem{Coh}
\Au{Cohen R., Erez K., Ben-Avraham~D., Havlin~S.}
Resilience of the Internet to random breakdowns~// Phys. Rev. Lett.,
2000. Vol.~85. P.~4626--4628.

\bibitem{Bol2}
\Au{Bollobas B., Riordan O.} Robustness and vulnerability of
scale-free random graphs~// Internet Math., 2004. Vol.~1.
No.\,1. P.~1--35.



\bibitem{Dur}
\Au{Durrett R.} Random graph dynamics.~--- Cambridge: Cambridge University
Press, 2007. 221~p.

\bibitem{Hof}
\Au{Hofstad R.} Random graphs and complex networks.~--- Eindhoven:
Eindhoven University of Technology, 2011. 363~p.

\bibitem{Dros}
\Au{Drossel B., Schwabl F.} Self-organized critical forest-fire
model~// Phys. Rev. Lett., 1992. Vol.~69. P.~1629--1632.

\bibitem{Ber1}
\Au{Bertoin J.} Burning cars in a parking lot~// Commun. Math.
Phys., 2011. Vol.~306. P.~261--290.

\bibitem{Ber2}
\Au{Bertoin J.} Fires on trees~// Annales de l'Institut Henri
Poincare Probabilites et Statistiques, 2012. Vol.~48. No.\,4.
P.~909--921.

\bibitem{Ari}
\Au{Arinaminparty N., Kapadia~S., May~R.} Size and complexity
model financial systems~// Proc. Natl. Acad. Sci.
USA, 2012. Vol.~109. P.~18338--18343.

\bibitem{Ler1}
\Au{Лери М.\,М., Павлов Ю.\,Л.} Лесной пожар на случайном
графе со сгораемыми реб\-ра\-ми~// Ученые записки ПетрГУ. Сер.
Естественные и технические науки, 2013. №~2(131). С.~96--99.

\bibitem{Ler2}
\Au{Leri M., Pavlov Yu.} Power-law graphs robustness and forest
fires~// Computer Data Analysis and Modeling: Theoretical and
Applied Stochastics:  10th
Conference (International) Proceedings.~--- Minsk, 2013. Vol.~1. P.~74--77.

\bibitem{Ler3}
\Au{Leri M., Pavlov Yu.} Power-law random graphs' robustness:
Link saving and forest fire model~// Austrian J.~Stat.,
2014. Vol.~43. No.\,4. P.~229--236.

\bibitem{Bol1}\textit{Bollobas B.~A.} A probabilistic proof of an asymptotic formula
for the number of labelled regular graphs // / Eur. J. Combust., 1980. Vol.~1. P.~311--316.

\bibitem{Fa}
\Au{Faloutsos C., Faloutsos P., Faloutsos~M.} On power-law
relationships of the Internet topology~// Comp. Comm. Rev., 1999.
Vol.~29. P.~251--262.

\bibitem{RN1}
\Au{Reittu H., Norros I.} On the power-law random
graph model of massive data networks~// Perform. Evaluation,
2004. Vol.~55. P.~3--23.

\bibitem{Tang}
\Au{Tangmunarunkit H., Govindan~R., Jamin~S., Shenker~S., Willinger~W.}
Network topology generators: Degree-based vs.\ structural~//
SIGCOMM'02 Proceedings.~--- Pittsburgh, USA, 2002. P.~147--159.

\bibitem{MT}
\Au{Matsumoto M., Nishimura T.} Mersenne twister: A~623-dimensionally
equidistributed uniform pseudorandom number generator~// ACM Trans.
Model. Comput. Simul., 1998. Vol.~8. No.\,1. P.~3--30.

\bibitem{Ler4}
\Au{Leri M., Pavlov Yu.} Forest fire models on configuration random graphs~//
3rd Russian--Finnish Symposium on Discrete Mathematics: Extended Abstracts.~---
Petrozavodsk: Karelian Research Centre RAS, 2014. P.~68--70.

 \end{thebibliography}

 }
 }

\end{multicols}

\vspace*{-6pt}

\hfill{\small\textit{Поступила в~редакцию 30.03.15}}

%\newpage

\vspace*{18pt}

\hrule

\vspace*{2pt}

\hrule

\vspace*{6pt}

\def\tit{FOREST FIRE ON~A~CONFIGURATION GRAPH WITH~RANDOM~FIRE~PROPAGATION}

\def\titkol{Forest fire on a configuration graph with random fire propagation}

\def\aut{M.\,M.~Leri}

\def\autkol{M.\,M.~Leri}

\titel{\tit}{\aut}{\autkol}{\titkol}

\vspace*{-9pt}


\noindent
Institute of Applied Mathematical Research,
Karelian Research Centre, Russian Academy of Sciences,
11~Pushkinskaya Str., Petrozavodsk 185910, Russian Federation


\def\leftfootline{\small{\textbf{\thepage}
\hfill INFORMATIKA I EE PRIMENENIYA~--- INFORMATICS AND
APPLICATIONS\ \ \ 2015\ \ \ volume~9\ \ \ issue\ 3}
}%
 \def\rightfootline{\small{INFORMATIKA I EE PRIMENENIYA~---
INFORMATICS AND APPLICATIONS\ \ \ 2015\ \ \ volume~9\ \ \ issue\ 3
\hfill \textbf{\thepage}}}

\vspace*{6pt}




\Abste{The paper considers a~random process of fire propagation over
links of two types of configuration graphs with random node degrees.
Node degrees follow either a~power law or the Poisson distribution.
The process takes place in a~random environment where the probabilities
of fire propagation follow the standard uniform distribution. The optimal
values of the node degree distribution parameters that ensure maximum
node survival in case of a~fire were estimated. The results were obtained
for two cases of fire start: targeted start~--- when a~fire starts from
the node with the highest degree and random ignition~---
when a~fire starts from an equiprobably chosen node. A~comparative analysis
of two graph models (power law and Poisson) in terms of the number of nodes
remained after the fire was performed.}

%\vspace*{0.5pt}

\KWE{configuration graphs; power-law distribution;
Poisson distribution; robustness; forest fire model}

%\vspace*{0.5pt}

\DOI{10.14357/19922264150307}

\vspace*{-6pt}

\Ack
The research was supported by the Russian Foundation for Basic Research
(project 13-01-00009).

\pagebreak

  \begin{multicols}{2}

\renewcommand{\bibname}{\protect\rmfamily References}
%\renewcommand{\bibname}{\large\protect\rm References}

{\small\frenchspacing
 {%\baselineskip=10.8pt
 \addcontentsline{toc}{section}{References}
 \begin{thebibliography}{99}
 
 \bibitem{2-li}
\Aue{Cohen, R., K.~Erez, D.~Ben-Avraham, and S.~Havlin.} 2000.
Resilience of the Internet to random breakdowns. \textit{Phys. Rev. Lett.}
85:4626--4628.

\bibitem{1-li}
\Aue{Bollobas, B., and O.~Riordan}. 2004. Robustness and vulnerability
of scale-free random graphs. \textit{Internet Math.} 1(1):1--35.



\bibitem{3-li}
\Aue{Durrett, R.} 2007. \textit{Random graph dynamics}.
Cambridge: Cambridge University Press.  221~p.

\bibitem{4-li}
\Aue{Hofstad, R.} 2011. \textit{Random graphs and complex networks}.
Eindhoven: Eindhoven University of Technology. 363~p.

\bibitem{5-li}
\Aue{Drossel, B., and F.~Schwabl.} 1992. Self-organized critical forest-fire
model. \textit{Phys. Rev. Lett.} 69:1629--1632.

%\pagebreak

\bibitem{6-li}
\Aue{Bertoin, J.} 2011. Burning cars in a parking lot. \textit{Commun. Math.
Phys.} 306:261--290.

\bibitem{7-li}
\Aue{Bertoin, J.} 2012. Fires on trees. \textit{Annales de l'Institut Henri
Poincare Probabilites et Statistiques} 48(4):909--921.

\bibitem{8-li}
\Aue{Arinaminparty, N., S.~Kapadia, and R.~May.} 2012. Size and complexity
model financial systems. \textit{Proc.\ Natl. Acad. Sci. USA} 109:18338--18343.

\bibitem{9-li}
\Aue{Leri, M.\,M., and Yu.\,L.~Pavlov.} 2013. Lesnoy pozhar na sluchaynom grafe
so sgoraemymi rebrami [Forest fire on random graph with inflammable links].
\textit{Uchenye Zapiski PetrGU. Estestvennye i~tekhnicheskie nauki ser.}
[Proceedings of Petrozavodsk State University. Natural and engeneering sciences ser.]
2(131):96--99.

\bibitem{10-li}
\Aue{Leri, M., and Yu.~Pavlov.} 2013. Power-law graphs robustness and forest
fires. \textit{10th  Conference (International) ``Computer Data Analysis
and Modeling: Theoretical and Applied Stochastics'' Proceedings}, Minsk. 1:74--77.

\bibitem{11-li}
\Aue{Leri, M., and Yu.~Pavlov.} 2014. Power-law random graphs' robustness:
Link saving and forest fire model. \textit{Austrian J.~Stat.} 43(4):229--236.

\bibitem{12-li}
\Aue{Bollobas, B.\,A.} 1980. A~probabilistic proof of an asymptotic formula
for the number of labelled regular graphs. \textit{Eur. J.~Combust.} 1:311--316.

\bibitem{13-li}
\Aue{Faloutsos, C., P.~Faloutsos, and M.~Faloutsos.} 1999. On power-law
relationships of the internet topology. \textit{Comp. Comm. Rev.}
29:251--262.

\bibitem{14-li}
\Aue{Reittu, H., and I.~Norros} 2004. On the power-law random
graph model of massive data networks. \textit{Perform. Evaluation} 55:3--23.

\bibitem{15-li}
\Aue{Tangmunarunkit, H., R.~Govindan, S.~Jamin, S.~Shenker, and W.~Willinger.}
2002. Network topology generators: Degree-based vs.\ structural.
\textit{SIGCOMM'02 Proceedings}.
Pittsburgh, USA. 147--159.

\bibitem{16-li}
\Aue{Matsumoto, M., and T.~Nishimura.} 1998. Mersenne twister: A~623-dimensionally
equidistributed uniform pseudorandom number generator.
\textit{ACM Trans. Modeling Comput. Simul.} 8(1):3--30.

\bibitem{17-li}
\Aue{Leri, M., and Yu.~Pavlov.} 2014. Forest fire models on
configuration random graphs.
\textit{3rd Russian Finnish Symposium on
Discrete Mathematics: Extended Abstracts}. Pet\-ro\-za\-vodsk. 68--70.
\end{thebibliography}

 }
 }

\end{multicols}

\vspace*{-3pt}

\hfill{\small\textit{Received March 30, 2015}}

\Contrl


\noindent
\textbf{Leri Marina M.} (b.\ 1969)~---
  Candidate of Science (PhD) in technology,
  scientist, Institute of Applied Mathematical Research
  of Karelian Research Centre,
  Russian Academy of Sciences, 11 Pushkinskaya Str., Petrozavodsk 185910, Russian Federation; leri@krc.karelia.ru

\label{end\stat}


\renewcommand{\bibname}{\protect\rm Литература}