\def\stat{lebedev}

\def\tit{ЭКСТРЕМАЛЬНЫЕ ИНДЕКСЫ В СХЕМЕ СЕРИЙ И~ИХ~ПРИЛОЖЕНИЯ$^*$}

\def\titkol{Экстремальные индексы в~схеме серий и~их~приложения}

\def\aut{А.\,В. Лебедев$^1$}

\def\autkol{А.\,В. Лебедев}

\titel{\tit}{\aut}{\autkol}{\titkol}

{\renewcommand{\thefootnote}{\fnsymbol{footnote}} \footnotetext[1]
{Работа поддержана РФФИ (проект 14-01-00075).}}


\renewcommand{\thefootnote}{\arabic{footnote}}
\footnotetext[1]{Московский государственный университет имени М.\,В.~Ломоносова,
ме\-ха\-ни\-ко-ма\-те\-ма\-ти\-че\-ский факультет, avlebed@уandex.ru}


\vspace*{-6pt}

\Abst{Проводится обобщение понятия экстремального индекса стационарной случайной последовательности
на схему серий одинаково распределенных случайных величин со случайными длинами серий,
стремящимися к~бесконечности по вероятности. Введены новые экстремальные индексы с~по\-мощью
двух определений, обобщающих основные свойства классического экстремального индекса.
Доказан ряд полезных свойств новых экстремальных индексов. Показано, как поведение максимумов
суммарных активностей на случайных графах (в~моделях информационных сетей) и~поведение максимумов
признаков частиц в~ветвящихся процессах (в~моделях биологических популяций) могут быть описаны в~терминах новых экстремальных индексов. Получены также новые результаты для моделей с~копулами
и~пороговых моделей. Показано, что новые индексы могут принимать разные значения для одной системы,
а~также значения, большие единицы.}

\KW{экстремальный индекс; схема серий; случайный граф;
информационная сеть; ветвящийся процесс;  копула}

\DOI{10.14357/19922264150305}

\vspace*{-6pt}

\vskip 12pt plus 9pt minus 6pt

\thispagestyle{headings}

\begin{multicols}{2}

\label{st\stat}

\section{Введение}

Экстремальный индекс стационарной (в узком смысле) случайной
последовательности $\{\xi_n\}$
определяется следующим образом~\cite[\S 3.7]{LLR}.

\noindent
\textbf{Определение A.}\footnote[2]{Далее через~$\vee$ обозначается максимум,
через~$\wedge$~--- минимум, чертой сверху над функцией распределения~--- хвост:
${\bar F}(x)\hm=1\hm-F(x)$,
через $f^{-1}$~--- обратная функция к~$f$, а~для функций распределения~---
обобщенная обратная:
$F^{-1}(y)\hm=\inf\{x:F(x)\hm\ge y\}$, через $f(x)^n$~--- $n$-я степень от $f(x)$.}
Пусть $\xi_n$, $n\hm\ge 1$, имеют распределение~$F$ и~$M_n\hm=\vee_{k=1}^n\xi_k$.
Если для каждого $\tau\hm>0$ существует такая числовая
последовательность $u_n(\tau)$, что $n{\bar F}(u_n(\tau))\hm\to \tau$
и~${\sf P}(M_n\le u_n(\tau))\hm\to e^{-\theta\tau}$,
то~$\theta$ называется экстремальным индексом.

\smallskip

При этом возможно любое значение $\theta\hm\in [0,1]$.

Заметим, что если взять максимумы ${\hat M}_n$ последовательности
независимых случайных величин с~тем же распределением~$F$, то
$$
\lim_{n\to\infty}{\sf P}({\hat M}_n\le u_n(\tau))=e^{-\tau}\,,
$$
откуда следует
\begin{equation}
\hspace*{-2mm}\lim\limits_{n\to\infty}{\sf P}(M_n\le u_n(\tau))=
\left(\lim\limits_{n\to\infty}{\sf P}({\hat M}_n\le u_n(\tau))\right)^{\!\theta}\!,\!\!
\label{Mfunc}
\end{equation}
т.\,е.\ предельные функции распределения~$M_n$ и~${\hat M}_n$ связаны
степенной зависимостью,
\begin{multline}
\label{Mtheta}
\lim\limits_{n\to\infty}{\sf P}(M_n\le u_n(\tau))=
\lim\limits_{n\to\infty}{\sf P}({\hat M}_{[\theta n]}\le u_n(\tau))\,,\\
\theta>0\,,
\end{multline}
т.\,е.\ $M_n$ растет асимптотически как максимум $[\theta n]$
независимых случайных величин при $n\hm\to\infty$ и
\begin{equation}
\label{Mhat}
\lim\limits_{n\to\infty}{\sf P}(M_n\le u_n(\tau))\ge \lim\limits_{n\to\infty}{\sf P}({\hat M}_n\le u_n(\tau))\,,
\end{equation}
т.\,е.\ $M_n$ стохастически не превосходит максимума независимых
случайных величин (в~пределе).

Интерес к~экстремальному индексу связан от\-час\-ти с~тем,
что его наличие сохраняет экстремальный тип
предельного распределения максимумов. Напомним, что если для
некоторых числовых последовательностей
$a_n\hm>0$, $b_n$, $n\hm\ge 1$, и~невырожденного распределения~$G$
имеет место предел
$$
\lim\limits_{n\to\infty}{\sf P}({\hat M}_n\le
a_nx+b_n)=G(x),\quad \forall x\in{\mathbb{R}}\,,
$$
то $G$ относится к~одному из трех экстремальных типов, а~именно:
$G(x)\hm=G_i(ax\hm+b)$ для
некоторых $a\hm>0$ и~$b$, где $G_1(x)\hm=\exp\{-e^{-x}\}$ (тип Гумбеля);
$G_2(x)\hm=\exp\{-x^{-\alpha}\}$, $x\hm>0$,
$\alpha\hm>0$ (тип Фреше); $G_3(x)\hm=\exp\{-(-x)^\alpha\}$, $x\hm\le 0$,
$\alpha\hm>0$ (тип Вейбулла). Такие
распределения~$G$ называют мак\-си\-мум-устой\-чи\-вы\-ми (max-stable)
или распределениями экстремальных
значений (extreme value distributions). Для любого $s\hm>0$
существуют такие $a(s)\hm>0$, $b(s)$, что
$G^s(x)\hm=G(a(s)x\hm+b(s))$. Таким образом, возведение в~степень $\theta\hm>0$
предельной функции распределения,
возникающее в~силу свойства~(\ref{Mfunc}), сохраняет экстремальный тип.

Одна из интерпретаций экстремального индекса заключается в~том, что превышения высокого уровня
в~последовательности происходят не по одиночке, 
а~группами (кластерами) средней величины~$1/\theta$.
В~приложениях это может означать природные катастрофы, отказы технических систем, потерю данных при
передаче информации, финансовые потери и~др. Понятно, что, когда такие события происходят несколько
раз подряд, это гораздо опаснее, чем единичные случаи, и~должно учитываться в~управлении рисками.
Более подробно об этом можно
прочитать в~[1--4]. %\cite{LLR, Gal, EKM, HF}


С~1980-х гг.\ активные исследования в~данной области ведутся
в~двух главных направлениях: вычисление
экстремального индекса для различных случайных последовательностей
и~разработка статистических
оценок экстремального индекса по наблюдениям.

С обзором результатов и~библиографией можно ознакомиться,
например, в~\cite[\S 8.1]{EKM} и~\cite[\S 5.5]{HF}.
Некоторым обобщениям классического понятия экстремального индекса
и~его статистическим оценкам
посвящен раздел диссертации~\cite[\S 1.2]{Novak-2013}.
В~частности, можно дать следующее определение.

\smallskip

\noindent
\textbf{Определение Б.} Пусть~$\xi_n$, $n\hm\ge 1$, имеют распределение~$F$
и~$M_n\hm=\vee_{k=1}^n\xi_k$.
Если для каждой числовой последовательности~$u_n$, $n\hm\ge 1$, такой что
$$
0<\liminf\limits_{n\to\infty}n{\bar F}(u_n)\le
\limsup\limits_{n\to\infty}n{\bar F}(u_n)<\infty\,,
$$
верно ${\sf P}(M_n\le u_n)\hm-F(u_n)^{\theta n}\hm\to 0$, $n\hm\to\infty$,
то~$\theta$ называется экстремальным индексом.

\smallskip


Такое определение позволяет расширить понятие экстремального индекса на некоторые стационарные
последовательности случайных величин с~дискретным распределением (например, геометриче\-ским),
а~для непрерывных эквивалентно определению~А.

Исследованиями экстремумов и~превышений высокого уровня в~связи с~моделями телекоммуникаций
посвящены работы~\cite{Markovich1, Markovich2},
а~в~\cite{Markovich-new} изучались распределения и~зависимость экстремумов в~процессах
выборочного исследования информационных сетей (network sampling processes), в~том числе
экстремальные индексы.

В диссертации~\cite{Gold} получены новые интересные результаты по экстремальным индексам
последовательностей вида
$$
X_n=A_nX_{n-1}+B_n\,,
$$
где $(A_n,B_n)$, $n\ge 1$,~--- независимые случайные пары со значениями
в~$\mathbb{R}_+^2$. В~некоторых
случаях найдены в~явном виде экстремальные индексы и~распределения размеров кластеров превышений
высокого уровня, в~более общем случае получены верхние и~нижние границы для экстремального индекса.
Доказана непрерывность экстремального индекса относительно некоторой сходимости распределений
коэффициентов. Введены и~изучены индексы многомерных последовательностей с~тяжелыми хвостами.
Часть полученных результатов представлена в~\cite{Gold1, Gold2}.

Однако на практике существует необходимость в~изучении максимумов на более сложных структурах,
чем множество натуральных чисел. Связанные с~этим трудности обсуждались еще в~\cite[\S 3.9, \S 3.12]{Gal}.
Например, если речь идет о~продолжительностях жизни компонент сложной системы (надежностной схемы),
то непонятно, как пронумеровать их так, чтобы использовать модель стационарной последовательности, да
и~возможно ли это вообще. Несколько проще обстоит дело со случайными полями.

Экстремальный индекс естественным образом обобщается со случайных последовательностей на случайные
поля на решетках $\mathbb{N}^d$~\cite{Choi}. Пусть, например, задано
такое поле $\{\xi_{n_1,n_2}\}$ в~$\mathbb{N}^2$ и~$M_{n_1,n_2}\hm=\vee_{k_1=1}^{n_1}\vee_{k_2=1}^{n_1}\xi_{k_1,k_2}$.
Тогда если для каждого $\tau\hm>0$
существует такое $u_{n_1,n_2}(\tau)$, что $n_1n_2{\bar F}(u_{n_1,n_2}(\tau))\hm\to
\tau$ и~${\sf P}(M_{n_1,n_2}\hm\le u_{n_1,n_2}(\tau))\hm\to e^{-\theta\tau}$,
то~$\theta$~называется экстремальным индексом.
Вычислению экстремального индекса случайного поля в~$\mathbb{N}^2$ посвящена работа~\cite{Exp1},
а~в~\cite{Exp2} изучалось асимптотическое расположение максимума случайного поля с~некоторым
экстремальным индексом.

Поскольку вышеупомянутые результаты не имеют прямого отношения к~теме статьи,
не будем останавливаться
на них более подробно, а~подчеркнем лишь актуальность исследований
экстремального индекса в~различных моделях и~приложениях.

Далее представлено новое обобщение экстремального индекса на схему серий со случайными
длинами, которое позволит работать с~более широким классом стохастических структур.
Более того, дадим два различных определения.

Пусть задан набор случайных величин $\xi_{n,m}$, $n\hm\ge 1$, $m\hm\ge 1$,
с~распределениями~$F_n$ (здесь~$n$~--- номер серии; $m$~---
номер случайной величины в~серии),
а~также последовательность целочисленных случайных величин (длин серий)
$\nu_n\stackrel{{\sf P}}{\to}\hm+\infty$, $n\hm\to\infty$,
и~$M_n\hm=\vee_{m=1}^{\nu_n}\xi_{n,m}$.

\smallskip

\noindent
\textbf{Определение 1.}\footnote{По сравнению с~определением~А сделана замена
$s\hm=e^{-\tau}$
и~соответственно переопределены функции~$u_n$, $n\hm\ge 1$.}
Пусть для каждого $s\hm\in (0,1)$ существует такая последовательность $u_n(s)$,
что ${\sf E}F_n(u_n(s))^{\nu_n}\to s$, и~${\sf P}(M_n\hm\le u_n(s))\hm\to\psi(s)$,
$n\hm\to\infty$.
Тогда~$\psi$ назовем экстремальной функцией.
Если $\psi(s)\hm=s^\theta$, то~$\theta$~назовем экстремальным индексом.

\smallskip

В общем случае можем определить частичные индексы
$$
\theta^+=\sup\limits_{s\in (0,1)}\log_s\psi(s)\,;\quad
\theta^-=\inf\limits_{s\in (0,1)}\log_s\psi(s)\,,
$$
тогда $\theta^+\hm\ge \theta^-$ и~$s^{\theta^+}\hm\le\psi(s)\hm\le s^{\theta^-}$,
$s\hm\in (0,1)$.

Смысл определения~1 заключается в~сравнении предельных распределений~$M_n$
и~максимумов~${\hat M}_n$ независимых случайных величин в~количестве~$\nu_n$
(не зависящем от них)
при одинаковой нормировке, определяемой условием
${\sf P}({\hat M}_n\hm\le u_n(s))\hm\to s$, $n\hm\to\infty$.
Тем самым обобщается свойство~(\ref{Mfunc}).

Понятно, что индексы, как и~ранее, принимают неотрицательные значения, однако ограничение
сверху единицей снимается, по крайней мере, для~$\theta^+$, как будет показано далее
(примеры~5.5, 6.1 и~6.2).
Это связано с~возможностью нарушения неравенства~(\ref{Mhat}).
Максимумы по сериям могут расти асимптотически быстрее,
чем максимумы независимых случайных величин, взятых в~тех же количествах,
что соответствует случаю $\psi(s)\hm<s$, $s\hm\in (0,1)$.

\smallskip

\noindent
\textbf{Определение 2.}\
Пусть для каждого $s\hm\in (0,1)$ существует такая последовательность $u_n(s)$,
что 
\begin{align*}
{\sf E}F_n(u_n(s))^{\nu_n}&\to s\,;\\
{\sf P}(M_n\le u_n(s))-{\sf E}F_n(u_n(s))^{\theta\nu_n}&\to 0\,,\ 
n\to\infty\,,
\end{align*}
 тогда~$\theta$~назовем экстремальным индексом.

\smallskip

Смысл определения~2 заключается в~подборе такого значения~$\theta$,
что предельные распределения~$M_n$ и~максимумов независимых случайных величин
в~количестве $[\theta\nu_n]$ (не зависящем от них)
совпадают при той же нормировке, что и~в~определении~1 (при $\theta\hm>0$).
Тем самым обобщается свойство~(\ref{Mtheta}).

Существование экстремального индекса по определению~2 фактически означает, что
экстремальная функция из определения~1 допускает представление
$$
\psi(s)=\lim\limits_{n\to\infty}{\sf E}F_n(u_n(s))^{\theta\nu_n}\,.
$$

Возникает вопрос, зачем давать два определения, нельзя ли обойтись ка\-ким-то одним. Действительно,
во многих случаях оба индекса эквивалентны: они существуют и~равны между собой (см.\ разд.~3, пример~5.1).
Но бывает также, что обоих индексов не существует, а~по определению~1 существует экстремальная
функция и~частичные индексы (примеры~5.2, 5.3, 6.1 и~6.2); бывает, что
существует индекс по определению~2
и~не существует индекса по определению~1, а~экстремальная функция
и~частичные индексы по-преж\-не\-му
существуют (см.\ разд.~4); наконец, бывает удивительная ситуация,
когда оба индекса существуют,
но принимают \textit{разные} значения (см.\ пример~5.4). Таким образом, это
действительно две разные характеристики системы, не сводящиеся к~одной.

Заметим, что ранее максимумы в~схеме серий рассматривались
в~\cite{Sav} для случайных величин, связанных IT-ко\-пу\-ла\-ми (копулами
преобразования независимости) и~выводились условия, при которых
максимумы растут асимптотически, как в~случае независимых величин,
т.\,е.\ в~терминах данной \mbox{статьи} $\theta\hm=1$.

Для определенности терминологии далее будем говорить об экстремальных индексах
\textit{системы} (случайных величин), обозначаемой через $\{\xi_{n,m};\nu_n\}$.

В разд.~2 доказаны основные свойства экстремальных индексов,
в~разд.~3 представлены
их приложения к~моделям информационных сетей, в~разд.~4~---
к~моделям биологических популяций,
в~разд.~5~--- к~моделям с~копулами, в~разд.~6~--- к~пороговым моделям.

\section{Основные свойства экстремальных индексов}

Экстремальные индексы обладают следующими свойствами.

\noindent
\textbf{Свойство 1.} \textit{Пусть~$\eta_n$, $n\hm\ge 1$,~---
стационарная последовательность с~экстремальным индексом~$\theta$ по определению~А.
Положим $\xi_{n,m}\hm=\eta_m$, $m\hm\ge 1$, и~пусть задана
целочисленная последовательность $l_n\hm\to +\infty$, тогда
система $\{\xi_{n,m};l_n\}$ имеет экстремальный индекс~$\theta$ по определениям~$1$
и~$2$}.

\smallskip

\noindent
Д\,о\,к\,а\,з\,а\,т\,е\,л\,ь\,с\,т\,в\,о\,.\ \
Обозначим через $u^0_n(\tau)$, $n\hm\ge 1$, последовательность,
существующую по определению~А,
и~положим $u_n(s)\hm=u^0_{l_n}(-\ln s)$, тогда
$F(u_n(s))^{l_n}\hm\to s$, ${\sf P}(M_n\hm\le u_n(s))\hm\to s^\theta$
и~$F(u_n(s))^{\theta l_n}\hm\to s^\theta$, что дает тот же
экстремальный индекс по обоим определениям.\hfill$\square$


\smallskip

\noindent
\textbf{Свойство 2.} \textit{Пусть система $\{\xi_{n,m};\nu_n\}$
имеет экстремальный индекс по одному из определений~$1$ и~$2$
(или экстремальную функцию) и~задана последовательность функций $g_n(x)$,
$n\hm \ge 1$, непрерывных и~строго возрастающих на множестве точек роста~$F_n$.
Положим ${\tilde\xi}_{n,m}\hm=g_n(\xi_{n,m})$, тогда система
$\{{\tilde\xi}_{n,m};\nu_n\}$
имеет тот же экстремальный индекс (экстремальную функцию).}


\smallskip

\noindent
Д\,о\,к\,а\,з\,а\,т\,е\,л\,ь\,с\,т\,в\,о\,.\ \
Для новой системы ${\tilde F}_n(x)\hm=F_n(g^{-1}_n(x))$. Пусть
${\tilde u}_n(s)\hm=g_n(u_n(s))$,
тогда ${\tilde F}_n({\tilde u}_n(s))\hm=F_n(u_n(s))$
и~${\sf P}({\tilde M}_n\le {\tilde u}_n(s))\hm=
{\sf P}(M_n\hm\le u_n(s))$, так что все пределы (в~определениях~1 и~2)
сохраняются.\hfill$\square$

\smallskip

\noindent
\textbf{Свойство~3.} \textit{Пусть система $\{\xi_{n,m};\nu_n\}$
имеет экстремальный индекс по одному из определений~$1$ и~$2$
и~существует такая последовательность $c_n\hm\to +\infty$, что
$\nu_n/c_n\stackrel{{\sf P}}{\to} 1$, $n\hm\to\infty$, тогда система имеет тот
же экстремальный индекс по другому определению.}

\smallskip


\noindent
Д\,о\,к\,а\,з\,а\,т\,е\,л\,ь\,с\,т\,в\,о\,.\ \
В~этом случае из
${\sf E}F_n(u_n(s))^{\nu_n}\hm={\sf E}(F_n(u_n(s))^{c_n})^{\nu_n/c_n}
\hm\to s\hm\in (0,1)$ следует $F_n(u_n(s))^{c_n}\to s$
и~${\sf E}F_n(u_n(s))^{r\nu_n}\hm\to s^r$, $r\hm\ge 0$, $n\hm\to\infty$.
Таким образом, если~$\theta$~--- экстремальный индекс
по определению~1, то из ${\sf P}(M_n\le u_n(s))\hm\to s^\theta$ следует
${\sf E}F_n(u_n(s))^{\theta\nu_n}\hm\to s^\theta$, а~значит, $\theta$~---
экстремальный индекс по определению~2. И~наоборот, если $\theta$~---
экстремальный индекс
по определению~2, то из ${\sf E}F_n(u_n(s))^{\theta\nu_n}\hm\to s^\theta$ следует
${\sf P}(M_n\le u_n(s))\hm\to s^\theta$, а~значит, $\theta$~--- экстремальный индекс
по определению~1.\hfill$\square$

\smallskip

\noindent
\textbf{Свойство~4.} \textit{Пусть имеется система $\{\xi_{n,m};\nu_n\}$
с~экстремальным индексом $\theta\hm>0$ по определению~$2$, у~которой}:
\begin{enumerate}
\item[(a)] $F_n\equiv F$;
\item[(б)] \textit{для некоторого мак\-си\-мум-устой\-чи\-во\-го закона~$G$ и~функций
$a(r)\hm>0$, $b(r)$, $r\hm>0$, верно}
$$
F^r(a(r)x+b(r))\to G(x)\,,\quad r\to\infty\,;
$$

\item[(в)] \textit{существует такая последовательность $c_n\hm\to +\infty$,
что} $\nu_n/c_n\stackrel{d}{\to}\zeta\hm>0$, $n\hm\to\infty$;

\item[(г)] \textit{в определении~$2$ можно взять
$u_n(s)\hm=A_nH^{-1}(s)\hm+B_n$, где $A_n\hm=a(c_n)$; $B_n\hm=b(c_n)$;
$H(x)$~--- непрерывная функция распределения}.
\end{enumerate}


\textit{Тогда\footnote{Если $\zeta$ вырождена и~равна константе $c\hm>0$ почти наверное 
(п.~н.),
то очевидно ${\sf E}u^\zeta\hm=u^c$.} 
\begin{gather*}
H(x)={\sf E}G(x)^\zeta\,;
\\
{\sf P}(M_n\le A_nx+B_n)\to H(ax+b)\,,\quad n\to\infty\,,
\end{gather*}
где $a>0$ и~$b$ определяются из тождества $G(x)^\theta\hm=G(ax\hm+b)$.
При этом экстремальная функция $\psi(s)\hm=H(aH^{-1}(s)\hm+b)$ по определению}~1.

\smallskip

\noindent
Д\,о\,к\,а\,з\,а\,т\,е\,л\,ь\,с\,т\,в\,о\,.\ \
Согласно~\cite[следствие 1.3.2]{LLR} для любого мак\-си\-мум-устой\-чи\-во\-го
закона существуют такие
$a\hm>0$ и~$b$, что $G(x)^\theta\hm=G(ax\hm+b)$. Пусть $x\hm=H^{-1}(s)$.
Поскольку
\begin{multline*}
{\sf E}F(A_nx+B_n)^{\nu_n}={\sf E}(F(a_{l_n}x+b_{l_n})^{l_n})^{\nu_n/l_n}\to{}\\
{}\to
{\sf E}G(x)^\zeta\,,\quad n\to\infty\,,
\end{multline*}
то 
$$
H(x)={\sf E}G(x)^\zeta\,.
$$
Тогда
\begin{multline*}
{\sf E}F(A_nx+B_n)^{\theta\nu_n}\to
{\sf E}G(x)^{\theta\zeta}={\sf E}G(ax+b)^\zeta={}\\
{}=
H(ax+b)\,,\quad n\to\infty\,.
\end{multline*}
По определению~2 отсюда следует ${\sf P}(M_n\le A_nx\hm+B_n)\hm\to
H(ax\hm+b)$, $n\hm\to\infty$, а~по определению~1 получаем
$\psi(s)\hm=H(aH^{-1}(s)\hm+b)$.\hfill$\square$

\smallskip

\noindent
\textbf{Свойство 5.} \textit{Пусть имеется система $\{\xi_{n,m};\nu_n\}$,
у~которой}:
\begin{enumerate}
\item[(a)] $F_n\equiv F$;
\item[(б)] \textit{существует такая последовательность $c_n\hm\to +\infty$,
что} $\nu_n/c_n\stackrel{d}{\to}\zeta\hm>0$, $n\hm\to\infty$;

\item[(в)] \textit{для непрерывного распределения~$G$
и~коэффициентов $A_n>0$, $B_n$ верно}

\vspace*{-9pt}

\noindent
\begin{align*}
F(A_nx+B_n)^{c_n}&\to G(x)\,;\\
{\sf P}(M_n\le A_nx+B_n)&\to {\sf E}G(x)^{\theta\zeta}\,,\quad n\to\infty\,.
\end{align*}
\end{enumerate}

\textit{Тогда $\theta$ является экстремальным индексом по определению}~2.



\noindent
Д\,о\,к\,а\,з\,а\,т\,е\,л\,ь\,с\,т\,в\,о\,.\ \
Прежде всего получаем ${\sf E}F(A_nx\hm+B_n)^{\nu_n}\hm\to {\sf E}G(x)^\zeta$.
Обозначим
$H(x)\hm={\sf E}G(x)^\zeta$~--- это непрерывная функция, про\-бе\-га\-ющая все значения
в~$(0,1)$. Полагаем $x\hm=H^{-1}(s)$, $u_n(s)\hm=A_nx\hm+B_n$, тогда
\begin{equation*}
{\sf E}F(u_n(s))^{\nu_n}\to s\,;\ 
{\sf E}F(u_n(s))^{\theta\nu_n}\to {\sf E}G(x)^{\theta\zeta}\,.
\end{equation*}
Поскольку по условию
верно также ${\sf P}(M_n\hm\le u_n(s))\hm\to {\sf E}G(x)^{\theta\zeta}$, то
$$
{\sf P}(M_n\le u_n(s))-{\sf E}F(u_n(s))^{\theta\nu_n}\to 0\,,
\ n\to\infty\,,
$$
 и~$\theta$ является экстремальным индексом по
определению~2.\hfill$\square$


\smallskip

\noindent
\textbf{Свойство 6.}\ \textit{Пусть имеется система $\{\xi_{n,m}; l_n\}$
с~экстремальным индексом~$\theta$ по одному
из определений, у~которой}:
\begin{enumerate}
\item[(а)] $F_n\equiv F$~--- \textit{непрерывное распределение};
\item[(б)] $l_n$, $n\hm\ge 1$,~--- \textit{целочисленная последовательность,
$l_n\hm\to +\infty$, $l_n\sim n^{\alpha}L(n)$, $n\hm\to\infty$, $\alpha\hm>0$,
$L(x)$~--- медленно меняющаяся функция на}~${\mathbb{R}}_+$.
\end{enumerate}

\textit{Пусть $\nu_n/l_n\stackrel{{\sf P}}{\to} 1$, $n\hm\to\infty$,
тогда система $\{\xi_{n,m}; \nu_n\}$ имеет тот же экстремальный индекс
по обоим определениям.}

\smallskip


\noindent
Д\,о\,к\,а\,з\,а\,т\,е\,л\,ь\,с\,т\,в\,о\,.\ \
По свойству~3 каждая из сис\-тем имеет одинаковые экстремальные индексы по
 обоим определениям.
Обозначим через~$M_n$ максимумы для системы
$\{\xi_{n,m}; l_n\}$ и~через~${\tilde M}_n$ для
системы $\{\xi_{n,m}; \nu_n\}$.



Для любого $\rho\hm>0$ верно $l_{[n\rho]}\sim\rho^\alpha l_n$, $n\hm\to\infty$.
Поэтому из $\nu_n/l_n\stackrel{{\sf P}}{\to} 1$, $n\hm\to\infty$,
следует
$$
{\sf P}\left(M_{[n(1-\varepsilon)]}\le {\tilde M}_n\le M_{[n(1+\varepsilon)]}\right)
\to 1\,,\quad n\to\infty\,,
$$

\vspace*{-4pt}

\pagebreak



\noindent
для любого $\varepsilon\hm>0$. Поскольку $F(u_n(s))^{l_n}\hm\to s$, то
$$
u_n(s)=F^{-1}\left(1+\fr{(1+o(1))\ln s}{l_n}\right)\,,\quad n\to\infty\,,
$$
и для любого $\rho\hm>0$ в~силу определения~1 верно
\begin{multline*}
{\sf P}(M_{[n\rho]}\le u_n(s))={}\\
{}={\sf P}\left(M_{[n\rho]}\le F^{-1}\left(1+\fr{(1+o(1))\ln s}{l_n}\right)\right)={}\\
{}=
{\sf P}\left(
M_{[n\rho]}\le  F^{-1}\left(1+
\fr{(1+o(1))\ln s^{1/\rho^\alpha}}{l_{[n\rho]}}\right)\right)\to {}\\
{}\to
s^{\theta/\rho^\alpha}\,,\quad n\to\infty\,.
\end{multline*}
Полагая $\rho=1\pm\varepsilon$, $\varepsilon\hm>0$, получаем
\begin{multline*}
s^{\theta/(1+\varepsilon)^\alpha}\le\liminf\limits_{n\to\infty}{\sf P}
\left({\tilde M}_n\le u_n(s)\right)\le{}\\
{}\le
\limsup\limits_{n\to\infty}{\sf P}\left({\tilde M}_n\le u_n(s)\right)\le
s^{\theta/(1-\varepsilon)^\alpha}\,.
\end{multline*}
Переходя к~пределу по $\varepsilon\hm\to 0$, получаем
$\lim_{n\to\infty}{\sf P}({\tilde M}_n\le u_n(s))\hm=s^\theta$;
значит, $\theta$ является экстремальным индексом системы
$\{\xi_{n,m}; \nu_n\}$ по определению~1.\hfill$\square$

\smallskip

\noindent
\textbf{Свойство 7.} \textit{Пусть имеется система $\{\xi_{n,m};\nu_n\}$
с~экстремальным индексом~$\theta$ по определению~$1$,
у которой $F_n\hm\equiv F$~--- непрерывное распределение,
$\nu_n/n\stackrel{{\sf P}}{\to} c\hm>0$, $n\hm\to\infty$,
и~пусть имеются целочисленная случайная
последовательность~$\eta_n$, не зависящая от $\{\xi_{n,m};\nu_n\}$,
и~числовая последовательность $\mu_n\hm\to +\infty$ такие, что
$\eta_n/\mu_n\stackrel{d}{\to}\zeta\hm>0$, $n\hm\to\infty$.
Положим ${\tilde\xi}_{n,m}\hm=\xi_{\eta_n,m}$,
${\tilde\nu_n}\hm=\nu_{\eta_n}$, тогда система
$\{{\tilde\xi}_{n,m};{\tilde\nu}_n\}$ имеет экстремальный
индекс~$\theta$ по определению}~2.

\noindent
Д\,о\,к\,а\,з\,а\,т\,е\,л\,ь\,с\,т\,в\,о\,.\ \
Для последовательности $u_n(s)$ из определения~1 для системы $\{\xi_{n,m};\nu_n\}$
имеет место сходимость ${\sf E}F(u_n(s))^{\nu_n}\hm\to s$,
откуда $F(u_n(s))^{cn}\hm\to s$, $n\hm\to\infty$, так что
$$
u_n(s)=F^{-1}\left(1+\fr{(1+o(1))\ln s}{cn}\right)\,,\quad n\to\infty\,.
$$
Для любого $x\hm\in (0,1)$ имеем
\begin{multline*}
{\sf E}F(u_{[\mu_n]}(x))^{{\tilde\nu}_n}={\sf E}F\left(u_{[\mu_n]}(x)\right)^{{\tilde\nu}_n}={}\\
{}={\sf E}F(u_{[\mu_n]}(x))^{(\nu_{\eta_n}/\eta_n)
\left(\eta_n/\mu_n\right)\mu_n}\to {\sf E}x^\zeta,\enskip n\to\infty.
\end{multline*}
Обозначим 
$$
H(x)={\sf E}x^\zeta;\ x=H^{-1}(s);\quad
{\tilde u}_n(s)=u_{[\mu_n]}(x),
$$
тогда
$$
{\sf E}F({\tilde u}_n(s))^{{\tilde\nu}_n}\to s\,;\enskip
{\sf E}F({\tilde u}_n(s))^{\theta{\tilde\nu}_n}\to {\sf E}x^{\theta\zeta}\,,\
n\to\infty\,.
$$


\noindent
С другой стороны, по определению~1 для системы $\{\xi_{n,m};\nu_n\}$ получаем


\noindent
\begin{multline*}
{\sf P}({\tilde M}_n\le{\tilde u}_n)={\sf P}\left(M_{\eta_n}\le u_{[\mu_n]}(x)\right)={}\\
{}={\sf P}\left(M_{\eta_n}\le F^{-1}\left(1+\fr{(1+o(1))\ln x}{c\mu_n}\right)\right)={}\\
{}={\sf P}\left(
M_{\eta_n}\le F^{-1}\left(
1+\fr{(1+o(1))\ln x^{\eta_n/\mu_n}}{c\eta_n}\right)\right)
\to{}\\
{}\to  {\sf E}x^{\theta\zeta},\quad n\to\infty.
\end{multline*}
Следовательно, 

\noindent
$$
{\sf P}({\tilde M}_n \le{\tilde u}_n)-
{\sf E}F({\tilde u}_n(s))^{\theta{\tilde\nu}_n}\to 0,\ n\to\infty,
$$
и~$\theta$ является экстремальным индексом сис\-те\-мы
$\{{\tilde\xi}_{n,m};{\tilde\nu}_n\}$ по определению~2.\hfill$\square$

\smallskip

Прокомментируем доказанные свойства.

Свойство~1 означает, что введенные индексы действительно являются обобщениями классического
экстремального индекса (по определению~А) и~совпадают с~ним, если в~качестве серий брать
растущие детерминированным образом куски стационарной последовательности.

Свойство~2 означает инвариантность экстремальных индексов относительно непрерывных строго
возрастающих преобразований серий. Это означает, например, что
в~случае непрерывных случайных
величин их все можно привести к~равномерному распределению на $[0,1]$ преобразованием
с~$g_n\hm=F_n$.
Аналогичное свойство имеет место для классического экстремального индекса
(по определению~А),
когда речь идет об одном непрерывном строго возрастающем преобразовании ко всем членам
последовательности.

Свойство~3 определяет ограничение на случайность длин серий, при котором оба новых индекса
эквивалентны. Длины должны расти асимптотически эквивалентно неслучайной 
по\-сле\-до\-ва\-тель\-ности.

Свойство~4 обобщает известное утверждение для классического
экстремального индекса~\cite[следствие 3.7.3]{LLR}: предельное
распределение максимумов стационарной последовательности имеет тот
же экстремальный тип, что и~предельное распределение максимумов
независимых случайных величин с~тем же частным распределением.
В~данном случае предельный закон уже не обязан быть
мак\-си\-мум-устой\-чи\-вым, однако его тип сохраняется. Максимум-устойчив
он только при вырожденной случайной величине $\zeta$ (константе),
т.\,е.\ при условии свойства~3.


Свойство~5 позволяет интерпретировать некоторый параметр
предельного распределения
максимумов зависимых случайных величин как экстремальный индекс по определению~2.



Свойство~6 дает достаточное условие на скорость роста длин серий, при котором индексы
для случайных и~неслучайных длин совпадают.

Свойство~7 показывает, что рандомизацией по случайно растущему номеру серии можно
перейти от экстремального индекса по определению~1 к~тому же индексу по определению~2.

\section{Приложения к~моделям информационных сетей}

В работах автора~[16--18]
%\cite{Leb3, Leb4, Leb-Nc}
рассматривались максимумы суммарной активности
в~информационных сетях, описываемых степенными случайными графами
(которые в~отечественной литературе называют также ин\-тер\-нет-гра\-фа\-ми
или графами ин\-тер\-нет-типа).

В качестве примеров недавних работ отечественных авторов о~степенных графах
можно указать~\cite{Pavl, Leri} и~обзор~\cite{Raig}. Рекомендуем также зарубежный
электронный учебник~\cite{Hofstad}.

Рассмотрим следующий пример модели информационной сети~\cite{Leb4}.

Пусть каждый узел сети обладает индивидуальной случайной
информационной активностью\linebreak (интенсив\-ностью производства информации).
Предположим, что активности узлов независимы и~одинаково
распределены, причем их распределение $A$ имеет тяжелый (правильно
меняющийся) хвост, т.\,е.\ ${\bar A}(x)\sim x^{-a}L(x)$, $x\hm\to\infty$,
$a\hm>0$, где $L(x)$~--- медленно меняющаяся функция. Активности
и~степени вершин (узлов) полагаются независимыми, и~это существенное
предположение.

Рассмотрим модель ориентированного
случайного графа, где направления ребер соответствуют направлениям
передачи информации. Пусть имеется~$n$~вершин и~заданы независимые
не\-от\-ри\-ца\-тельные целочисленные случайные величины $K_1,\ldots , K_n$,
имеющие одинаковое
распределение, заданное вероятностями $p_k\sim ck^{-\beta}$, $k\hm\to\infty$,
$\beta\hm>2$. Положим $D_i\hm=\min\{K_i,n-1\}$. Для $i$-й вершины выберем
случайным образом (равновероятно и~независимо от выбора для других
вершин) $D_i$ различных вершин из числа остальных (кроме $i$-й)
и~выпустим из них ребра, направленные в~$i$-ю вершину. Полученный
в~результате граф можно отнести к~степенным в~том смысле, что число
входящих ребер распределено асимптотически по степенному закону.
Суммарной активностью в~узле в~данном случае будем считать сумму
собственной активности узла и~всех узлов, из которых в~него
поступает информация (его входящих со\-седей).
{\looseness=1

}

Подчеркнем, что активность никак не связана с~понятиями <<качества>> или <<веса>> вершины,
используемых в~современных моделях формирования случайных графов. Здесь граф
сформирован по описанному выше алгоритму, а~индивидуальные активности представляют
собой независимые дополнения к~графу. Если говорить о~социальных сетях, то бывает,
что пользователь пишет много, но мало кого читает (или его мало кто читает),
и~на\-обо\-рот,
бывает, что он пишет мало, но читает многих (или его читают многие). Что касается
суммарной активности, то она может оказаться большой за счет малого числа входящих
соседей с~большой индивидуальной активностью (или даже одного соседа), и~наоборот,
небольшой при большом числе входящих соседей с~малой индивидуальной активностью.
Как известно, для тяжелых хвостов характерно, что большие значения суммы
принимаются за счет одного большого (максимального) слагаемого. Это свойство в~данном случае обобщается и~на суммы случайного числа слагаемых.

Обозначим через~$M_n$ максимум суммарных активностей. Пусть
$v(r)$~--- положительная неубывающая
функция такая, что верно $r{\bar A}(v(r))\hm\to 1$, $r\hm\to\infty$.
Заметим, что $v(r)$ заведомо существует
и~является правильно меняющейся с~показателем $1/a$~\cite[\S 1.5]{Sen}.

Тогда при $a<\beta\hm-2$, если $2\hm<\beta\hm<3$, и~при
$a\hm<(\beta-1)/2$, если $\beta\hm\ge 3$,
имеет место предельный закон Фреше:

\vspace*{3pt}

\noindent
$$
{\sf P}\left(\fr{M_n}{v(n)}\le x\right)=\exp\{-x^{-a}\},\ x>0,\ n\to\infty.
$$ 

\vspace*{-2pt}

\noindent
Отметим,
что этот предельный закон обусловлен тем, что максимум суммарных активностей
растет асимптотически эквивалентно (по ве\-ро\-ят\-ности) максимуму индивидуальных
активностей по сети, что и~доказано в~\cite[теорема~1]{Leb4}.

С другой стороны, если число
входящих соседей описывается случайной величиной~$K$, не зависящей от активности,
то предельное распределение~$F$ суммарной активности в~каждом узле имеет хвост

\vspace*{3pt}

\noindent
$$
{\bar F}(x)\sim (1+{\sf E}K){\bar A}(x)\,,\enskip x\to\infty\,,
$$

\vspace*{-2pt}

\noindent
при условии (которое в~данном случае выпол\-ня\-ется)

\vspace*{3pt}

\noindent
$$
{\sf E}K^{1\vee (a+\varepsilon)}<\infty\,,\enskip \varepsilon>0\,,
$$

\vspace*{-2pt}

\noindent
согласно результатам~\cite{Stam} о~распределении суммы случайного числа независимых
случайных величин с~тяжелыми хвостами. Поэтому

\vspace*{-3pt}

\noindent
\begin{multline*}
F(xv(n))^n\to\exp\left\{-(1+{\sf E}K)x^{-a}\right\}\,,\\ x>0\,,\enskip n\to\infty\,.
\end{multline*}
Обозначив 
$$
s=\exp\{-(1+{\sf E}K)x^{-a}\}\in (0,1);\quad  
u_n(s)=xv(n),
$$
приходим к~выводу, что система
суммарных активностей имеет экстремальный индекс $\theta\hm=1/(1\hm+{\sf E}K)$
по определению~1
(а~значит, и~по определению~2 в~силу свойства~3, поскольку $\nu_n\hm=n$).

Для последовательностей значение $\theta\in (0,1)$ означает,
что превышения высокого уровня происходят не по одиночке, а~группами
(кластерами) средней величины $1/\theta$~\cite[\S 8.1]{EKM}.
В~рассматриваемом случае также можно
предположить образование подобных кластеров.

Применительно к~информационным сетям
речь может идти о группах узлов с~высокими суммарными активностями, вызванными
высокой индивидуальной активностью одного узла, являющегося их
общим входящим соседом.

Проверка наличия подобного эффекта в~реальных сетях, разумеется,
требует экспериментального исследования, выходящего
за рамки данной работы, которая имеет теоретический характер.

Следует также признать, что выбор модели случайного графа
в~\cite{Leb4} был обусловлен
не ка\-ки\-ми-то ее преимуществами в~описании реальных сетей по сравнению с~другими
современными моделями (например, моделями преимущественного присоединения),
а~относительной простотой доказательства асимптотической эквивалентности роста
максимумов суммарных и~индивидуальных активностей методами работы автора~\cite{Leb-2005c}.
При этом просто знания степенного закона для числа входящих вершин совершенно недостаточно
и каждая модель случайного графа должна рассматриваться в~индивидуальном порядке. Например,
важна скорость роста максимальной степени вершины в~графе. Если искусственно урезать степени
вершин на растущем (c~числом вершин) пороге, можно получить класс моделей с~одинаковыми предельными
распределениями степеней вершин, но разными скоростями роста максимальной степени,
для которых будут возникать разные ограничения для~$a$ в~зависимости от~$\beta$.
Играют также роль другие нюансы.

Недавно автором получены новые результаты для простых моделей с~весами~\cite{Leb-Nc}.
Похожие модели рассматривались
в~\cite[гл. 6]{Hofstad} как обобщенные случайные графы
(generalized random graphs). Предполагается,
что вершинам приписываются независимые веса $w_i$, $1\hm\le i\hm\le n$,
одинаково распределенные как
неотрицательная случайная величина $W$, ${\sf E}W^\beta\hm<\infty$, $\beta\hm\ge 1$.

В модели~1 полагаем $p_i\hm=\varphi(w_in^{-s/2})$, где $0\hm<s\hm\le 1$,
и~для~$\varphi$ на
${\mathbb{R}}_+$ верно $0\hm\le \varphi(x)\hm\le \min\{1,x\}$, $\varphi(x)\sim x$,
$x\hm\to 0$.
При известных $w_i$, $1\hm\le i\hm\le n$, каждая пара вершин~$i$ и~$j$
соединяется реб\-ром
с~вероятностью $p_ip_j$ независимо от других пар. Здесь граф полагается
неориентированным, информация передается по реб\-ру в~обе стороны. Применительно
к~социальным сетям веса могут отражать общительность пользователей.

В модели~2 при тех же предположениях о~весах полагаем $p_i\hm=
\varphi(w_in^{-s})$, $0\hm<s\hm\le 1$.
При известных~$w_i$, $1\hm\le i\hm\le n$, в~$i$-ю вершину входит реб\-ро из любой другой вершины
с~вероятностью~$p_i$ независимо от других ребер. Здесь граф полагается ориентированным,
информация передается по направлению реб\-ра. Применительно
к~социальным сетям веса могут отражать любознательность пользователей.

В обоих случаях доказана асимптотическая эквивалентность роста
максимумов суммарных и~индивидуальных активностей при некоторых ограничениях
на~$a$ в~зависимости от~$\beta$ и~$s$.
При $s\hm=1$ существуют предельные распределения числа соседей (входящих соседей),
а~полученные результаты можно интерпретировать как наличие экстремальных индексов:
$\theta\hm=1/(1\hm+({\sf E}W)^2)$
в~модели~1 и~$\theta\hm=1/(1\hm+{\sf E}W)$ в~модели~2
(по обоим определениям), аналогично предыдущему примеру.

Приложение разработанных методов к~другим, более сложным и~популярным
моделям информационных сетей представляет собой дело будущего.

\section{Приложение к~моделям биологических популяций}

В качестве моделей биологических популяций часто используются ветвящиеся процессы.
При этом элементы популяции традиционно называют частицами. Частицы могут обладать
ка\-ки\-ми-то случайными (количественными) признаками.

В случае живых организмов речь может идти о~размерах, весе и~других характеристиках, например
удоях коров, яйценоскости кур, урожайности растений, чувствительности организмов к~вредным
и~опасным факторам и~др.

Ветвящимися процессами может также описываться распространение компьютерных вирусов.
Полиморфные компьютерные вирусы способны не только размножаться, но и~изменять свой код
(подобно мутациям живых организмов). В~качестве признаков могут рассматриваться ка\-кие-то
характеристики кода вируса или его деятельности.

В~\cite{AV, Pakes} изучались максимумы независимых случайных признаков частиц
в~ветвящихся
процессах. При этом в~качестве приложений в~\cite{AV} речь шла о~росте людей,
а~в~\cite{Pakes}
упоминалось разведение скаковых лошадей, с~призовыми очками в~качестве признака.

Приведем еще один пример. Если имеется колония вредных организмов с~различными
индивидуальными порогами чувствительности к~ка\-ко\-му-то фактору (яду, антибиотику или др.),
то для
уничтожения всей колонии нужна максимальная концентрация, иначе часть организмов
колонии выживет и~вновь размножится.

В ряде работ автора рассматривались максимумы
случайных признаков частиц в~надкритических ветвящихся процессах
без вырождения (с конечными средним и~дисперсией числа потомков).
Так, в~\cite{Leb5} рассмотрен процесс с~непрерывным временем, а~в~\cite{Leb60, Leb6}~---
с~дискретным. Однако
при этом признаки разных частиц считались независимыми. В~\cite{Leb7} впервые изучалась
модель с~зависимостью признаков частиц в~поколении, обусловленной их общей наследственностью.

Сначала рассмотрен случай, когда признаки имеют стандартное нормальное
распределение, а~коэффициент корреляции признаков пары частиц мажорируется величиной~$r^k$,
$r\hm\in (0,1)$, если эти частицы имеют ближайшего общего предка~$k$~поколений назад. Показано, что
максимумы по поколениям растут асимптотически так же, как в~случае независимых признаков,
что соответствует $\theta\hm=1$.

Далее рассмотрен случай, когда признаки имеют распределение с~правильно меняющимся
хвостом, а~наследственность явно описывается процессом линейной авторегрессии первого порядка:
\begin{equation}
\label{lireg}
\xi_{n,m}=a\xi_{n-1,\kappa(n,m)}+b\xi^*_{n,m}\,,\enskip a\in (0,1)\,,\ b>0\,,
\end{equation}
где $\xi_{n,m}$~--- признак $m$-й частицы в~$n$-м поколении;
$\kappa(n,m)$~--- номер предка этой частицы в~предыду\-щем поколении,
а~случайные величины $\xi^*_{n,m}$, $m\hm\ge 1$, $n\hm\ge 1$, независимы
и~имеют одинаковое распределение~$A$, которое удовлетворяет условиям
\begin{equation}
\left.
\begin{array}{c}
{\bar A}(x)\sim x^{-\gamma}L(x)\,;\quad \fr{A(-x)}{{\bar A}(x)}\to p\ge 0\,;
\\[6pt]
 x\to\infty\,;\quad \gamma>0\,,
 \end{array}
 \right\}
 \label{prav}
\end{equation}
где $L(x)$~--- медленно меняющаяся функция.

В~модели~(\ref{lireg}) существует и~единственно стационарное
распределение~$F$. Предполагается, что все признаки частиц имеют это
стационарное распределение.
Для выявления роли наследственности <<в~чис\-том виде>> желательно
обеспечить независимость распределения признаков от коэффициентов
авторегрессии (как это имело место в~гауссовском случае).
Здесь можно добиться этого только для строго устойчивых распределений
с~$0\hm<\gamma\hm<2$, полагая
\begin{equation}
\label{uab}
a^\gamma+b^\gamma=1\,.
\end{equation}
Для произвольных~$A$, удовлетворяющих~(\ref{prav}), условие~(\ref{uab})
обеспечивает асимптотическую эквивалентность хвостов:
${\bar F}(x)\sim {\bar A}(x)$, $x\hm\to\infty$.
Предполагается, что это условие выполнено.

В этом случае фактически показано, что
$$
\theta=\fr{1-a^\gamma}{1-a^\gamma/\mu}\in (0,1)
$$ 
по определению~2
(по свойству~5), где $\mu\hm>1$~--- среднее число потомков. Заметим,
что~$\theta$~стремится к~1 как при уменьшении параметра зависимости~$a$,
так и~при уменьшении среднего числа потомков~$\mu$.
Экстремального индекса по определению~1 в~данном случае не существует.

Здесь также можно предположить образование кластеров.
Очевидно, речь идет о~родственных группах частиц,
имеющих общего предка с~аномально большим признаком
и~унаследовавшим эту мутацию. Этот вывод иллюстрируется
компьютерным моделированием в~\cite{Leb7}.

\section{Модели с~копулами}

Напомним понятия из теории копул~[32, гл.~5, \S~7.5; 33].

Копулой~$C$ называется функция многомерного распределения на $[0,1]^d$,
$d\hm\ge 2$, если все частные
распределения являются равномерными на $[0,1]$. Согласно теореме Скляра
любая функция многомерного
распределения в~${\mathbb{R}}^d$ представима в~виде:
$$
G\left(x_1,\dots x_d\right)=C\left(G_1\left(x_1\right),\dots G_d\left(x_d\right)\right)\,,
$$
где $G_i$, $1\hm\le i\hm\le d$,~--- функции частных распределений. Таким образом,
всякому многомерному
распределению можно поставить в~соответствие его копулу. Если частные распределения
непрерывны, то такое представление единственно.

Вектору с~независимыми компонентами соответствует копула независимости
$$
C(y_1,\dots,y_d)=y_1\dots y_d\,.
$$

В настоящее время математический аппарат копул активно используется в~самых разных
приложениях, в~том числе проникает и~в~информатику.
Отметим работу по рекуррентным нейронным сетям~\cite{Chat},
где использовались копулы Стьюдента, Клейтона и~Гумбеля. На их основе производилось
успешное обучение человекоподобного робота.

В общем случае копулы могут описывать зависимость в~поведении
компонент сложных систем, обусловленную их взаимодействием или
влиянием общих внешних факторов. В~моделях предыдущих разделов
зависимости суммарных активностей в~сети и~признаков частиц
в~поколении также описываются некими копулами, которые, однако,
затруднительно выписать в~явном виде и~поэтому лучше использовать
иные методы. Отметим, что общение пользователей сети может приводить
к~зависимости их индивидуальных информационных активностей, что не
учитывалось в~\cite{Leb4}. В~технических системах износ или выход из
строя одних деталей может сказываться на других деталях, а~на все
вместе может влиять общий режим эксплуатации (температура, влажность
и~т.\,п.).

В финансах копулы используются для описания зависимости между колебаниями курсов различных
акций и~валют~\cite{QRM}. Эту зависимость следует учитывать как в~финансовых расчетах,
 так и~при программировании торговых (финансовых) роботов (black boxes).

Далее для простоты будем полагать, что $\nu_n\hm=n$ (треугольная схема),
$F_n(x)\hm\equiv x$,
$x\hm\in [0,1]$, а~случайные величины~$\xi_{n,m}$, $1\hm\le m\hm\le n$,
связаны $n$-мер\-ной копулой~$C_n$.
Напомним, что к~равномерному распределению можно перейти от любого непрерывного в~силу
свойства~2.

Пусть для любого $s\hm\in (0,1)$ последовательность $u_n(s)$ такова,
что $u_n(s)^n\hm\to s$, $n\hm\to\infty$.
Тогда 
$$
u_n(s)=1+\fr{(1+o(1))\ln s}{n},\enskip n\to\infty\,.
$$

\smallskip

\noindent
\textbf{Пример 5.1.\ Копула Гум\-бе\-ля--Хоу\-га\-ар\-да}.
Данная копула имеет вид:
$$
C(y_1,\dots,y_d)=\exp\left\{\!-\left(\sum\limits_{i=1}^d(-\ln y_i)^\alpha
\right)^{\!\!1/\alpha}\right\}\!,\  \alpha\ge 1,
$$
откуда следует
$$
C(y,\dots,y)=y^{d^{1/\alpha}}\,.
$$
Полагая $C_n$ копулой Гум\-бе\-ля--Хоу\-га\-ар\-да
с~$\alpha_n\hm\ge 1$ и~$(\alpha_n-1)\ln n\hm\to\gamma\hm\ge 0$, получаем
$$
{\sf P}(M_n\le u_n(s))=u_n(s)^{n^{1/\alpha_n}}\to s^\theta,\  \
 \theta=e^{-\gamma}\in [0,1].
 $$
Данная копула относится к~классу копул экстремальных значений
(или мак\-си\-мум-устой\-чи\-вых).
В~общем случае для них имеет место представление
Пикандса~\cite[c. 312, теорема 7.45]{QRM}:
\begin{multline*}
C(y_1,\dots,y_d)={}\\
{}=\exp\!\left\{B\!\left(\fr{\ln y_1}
{\sum\nolimits_{i=1}^d\ln y_i},\dots,
\fr{\ln y_1}{\sum\nolimits_{i=1}^d\ln y_i}\!\right)\!
\sum\limits_{i=1}^d\ln y_i\!\right\}.\hspace*{-8.86847pt}
\end{multline*}
Здесь
$$
B\left(w_1,\dots w_d\right)=\int\limits_{S^d}
\left(\bigvee_{i=1}^d x_iw_i\right)\,dH(x)\,,
$$
где $H$~--- конечная мера на $S^d\hm=\{x=(x_1,\dots x_d): x_i\hm\ge 0,
\sum_{i=1}^d x_d\hm=1\}$, причем эта
мера должна быть нормирована так, что $\int\limits_{S^d}x_i\,dH(x)\hm=1$ для всех
$1\hm\le i\hm\le d$ (о~чем в~\cite{QRM} забыли упомянуть).

Заметим, что функция~$B$ однородна первого порядка. Таким образом,
в~общем случае верно
$$
C(y,\dots,y)=y^{B(1,\dots 1)}\,.
$$
Обозначим $\beta_n\hm=B_n(1,\dots 1)$, тогда если $\beta_n/n\hm\to\theta$,
то $\theta$~--- экстремальный индекс
(по обоим определениям). Поскольку $0\hm\le\beta_n\hm\le n$, то $\theta\hm\in [0,1]$.

\smallskip

\noindent
\textbf{Пример~5.2. Копула Клейтона.} Данная копула имеет вид:
$$
C(y_1,\dots,y_d)=\left(\sum\limits_{i=1}^d y_i^{-\alpha}-d+1\right)^{-1/\alpha}\,,
\enskip \alpha\ge 0\,,
$$
где вырожденный случай $\alpha\hm=0$ соответствует копуле независимости,
возникающей в~пределе при $\alpha\hm\to 0$. Отсюда
$$
C(y,\dots,y)=\left(d (y^{-\alpha}-1)+1\right)^{-1/\alpha}\,.
$$
Пусть $C_n$~--- копула Клейтона с~$\alpha_n\hm\equiv \alpha>0$, тогда
\begin{multline*}
{\sf P}(M_n\le u_n(s))=\left(n(u_n(s)^{-\alpha}-1)+1\right)^{-1/\alpha}\to{}\\
\to \left(1-\alpha\ln s\right)^{-1/\alpha}=\psi(s)\,.
\end{multline*}
Здесь $\theta^-\hm=0$, $\theta^+\hm=1$.

\smallskip

\noindent
\textbf{Пример~5.3.\ Копула Франка}. Данная копула имеет вид:
$$
C(y_1,\dots,y_d)=-\fr{1}{\alpha}\ln\left(\!1-\fr{\prod_{i=1}^d(1-e^{-\alpha y_i})}
{(1-e^{-\alpha})^{d-1}}\!\right)\!,
\enskip \alpha\ge 0,
$$
где вырожденный случай $\alpha\hm=0$ соответствует копуле независимости,
возникающей в~пределе при $\alpha\hm\to 0$. Отсюда
$$
C(y,\dots,y)=-\fr{1}{\alpha}\ln\left(1-
\fr{(1-e^{-\alpha y})^d}{(1-e^{-\alpha})^{d-1}}\right)\,.
$$
Пусть $C_n$~--- копула Франка с~$\alpha_n\hm\equiv \alpha\hm>0$,
тогда, переходя к~пределу, получаем:
\begin{multline*}
{\sf P}\left(M_n\le u_n(s)\right)\to{}\\
{}\to -
\fr{1}{\alpha}\ln\left(1-\left(1-e^{-\alpha}\right)s^{\alpha/(e^\alpha-1)}\right)
=\psi(s)\,.
\end{multline*}
В этом случае
$$
\lim\limits_{s\to 0}\log_s\psi(s)=
\fr{\alpha}{e^\alpha-1}\,;\quad \lim\limits_{s\to 1}\log_s\psi(s)=1\,,
$$
а на интервале $(0,1)$ функцией принимаются промежуточные значения. Поэтому
$\theta^-\hm=\alpha/(e^\alpha-1)\hm\in (0,1)$, $\theta^+\hm=1$.

Во всех трех примерах имеют место строго архимедовы копулы.
Напомним, что строго архимедовой называется копула вида
\begin{equation}\label{copu}
C(y_1,\dots,y_d)=\varphi^{-1}\left(\sum\limits_{i=1}^d \varphi(y_i)\right)\,,
\end{equation}
где $\varphi$~--- убывающая функция на $[0,1]$, называемая генератором,
$\varphi(0)\hm=+\infty$, $\varphi(1)\hm=0$.
При $d\hm=2$ достаточно, чтобы эта функция была выпуклой. Если потребовать,
чтобы функция~$\varphi^{-1}$
была вполне монотонной на $(0,+\infty)$, то формула~(\ref{copu})
определяет копулу при любом $d\hm\ge 2$~\cite[теорема 4.6.2]{Nel}.
Далее будем считать это условие на~$\varphi$ выполненным.

С другой стороны, функция~$f$ является преобразованием
Лап\-ла\-са--Стилть\-еса некоторого распределения тогда и~только тогда,
когда~$f$~впол\-не монотонна и~$f(0)\hm=1$~\cite[гл. 13, \S 4, теорема~1]{Fel}.
Отсюда следует, что функция~$\varphi^{-1}$ должна быть
преобразованием Лап\-ла\-са--Стилть\-еса некоторого распределения, причем
в~силу условия $\varphi(0)\hm=+\infty$, а~значит,
и~$\varphi^{-1}(+\infty)\hm=0$, это распределение не должно иметь атомов
в~нуле. Таким образом, существует некоторая случайная величина
$\zeta\hm>0$ п.~н.\ такая, что
$$
\varphi^{-1}(u)={\sf E}e^{-u\zeta}\,,\quad u\ge 0\,.
$$
Введем обозначения:
$$
x_0=\inf\{x>0: {\sf P}(\zeta\le x)>0\}\,,\enskip \mu={\sf E}\zeta\,.
$$
Будем для краткости обозначать $f(u)\hm=\varphi^{-1}(u)$.

\smallskip

\noindent
\textbf{Теорема~5.1.}\ \textit{Пусть $\mu\hm<\infty$, тогда экстремальная функция
$\psi(s)\hm=f(-(\ln s)/\mu)\hm={\sf E}s^{\zeta/\mu}$,
$\theta^+\hm=1$, $\theta^-\hm=x_0/\mu$.}


\noindent
Д\,о\,к\,а\,з\,а\,т\,е\,л\,ь\,с\,т\,в\,о\,.\ \
 Поскольку $1-f(u)\sim \mu u$, $u\hm\to 0+0$, то $\varphi(1-t)\sim t/\mu$,
 $t\hm\to 1-0$.
Имеем:
\begin{multline*}
{\sf P}\left(M_n\le u_n(s)\right)=f(n\varphi(u_n(s)))={}\\
{}=f\left(n\varphi\left(1+\fr{(1+o(1))\ln s}{n}\right)\right)\to 
f\left(-\fr{\ln s}{\mu}\right)\,,\\ 
n\to\infty\,.
\end{multline*}

Из неравенства Иенсена получаем:
$$
\psi(s)={\sf E}s^{\zeta/\mu}\ge
s^{{\sf E}\zeta}=s\,.
$$ 
С~другой стороны, поскольку $\zeta\hm>0$ п.~н., то
$\psi(s)\hm\le s^{x_0/\mu}$. Следовательно, $\theta^+\hm\le 1$
и~$\theta^-\hm\ge x_0/\mu$. Кроме того, получаем:
$$
\lim\limits_{s\to 0}\log_s\psi(s)=\fr{x_0}{\mu}\,;\quad
\lim\limits_{s\to 1}\log_s\psi(s)=1\,,
$$
так что эти оценки
достигаются в~пределе и~верно $\theta^+\hm=1$; $\theta^-\hm=x_0/\mu$.\hfill$\square$

\smallskip

В случае копулы Клейтона генератор $\varphi(t)\hm=t^{-\alpha}-1$
и~обратная функция $f(u)\hm=1/(1+u)^{1/\alpha}$
соответствуют гам\-ма-рас\-пре\-де\-ле\-нию с~параметром формы~$1/\alpha$,
для которого $x_0\hm=0$, так что
$\theta^-\hm=0$; $\theta^+\hm=1$.

В случае копулы Франка генератор
$\varphi(t)\hm=-\ln((1-e^{-\alpha t})/(1-e^{-\alpha}))$ и~обратная функция
$f(u)\hm=-(1/\alpha)\ln(1-(1-e^{-\alpha})e^{-u})$ соответствуют
дискретному распределению с~вероятностями
${\sf P}(\zeta=k)\hm=(1-e^{-\alpha})^k/(\alpha k)$, $k\hm\ge 1$. Тогда
$x_0\hm=1$ и~$\mu\hm=f'(0)\hm=(e^\alpha-1)/\alpha$,
откуда $\theta^-\hm=\alpha/(e^\alpha-1)$; $\theta^+\hm=1$.

В условия теоремы~5.1 из рассмотренных не вписывается
только пример копулы Гум\-бе\-ля--Хоу\-га\-ар\-да,
поскольку она имеет генератор $\varphi(t)\hm=(-\ln t)^\alpha$
c~обратной функцией $f(u)\hm=\exp\{-u^{1/\alpha}\}$, $\alpha\hm\ge 1$,
что соответствует асимметричному $(1/\alpha)$-устой\-чи\-во\-му
распределению на~${\mathbb{R}}_+$, не имеющему конечного среднего.

Для изучения таких случаев применим сле\-ду\-ющую модификацию.

Заметим, что если $\varphi(t)$~--- генератор c~вполне монотонной обратной функцией,
то $\varphi(t)^\beta$, $\beta\hm\ge 1$,~--- также генератор  с~вполне
монотонной обратной функцией~\cite[лемма 4.6.4]{Nel}.

\smallskip

\noindent
\textbf{Теорема~5.2.}\ \textit{Пусть $n$-мер\-ная копула~$C_n$ имеет
генератор $\varphi_n(t)\hm=\varphi(t)^{\beta_n}$,
где $\beta_n\hm\ge 1$, $(\beta_n-1)\ln n\hm\to\gamma\hm\ge 0$,
и~для генератора $\varphi(t)$ верно $\mu\hm<\infty$.
Тогда }
$$
\psi(s)=f\left(-\fr{e^{-\gamma}\ln s}{\mu}\right);\ \ 
\theta^-=\fr{x_0}{\mu}\,e^{-\gamma};\ \ \theta^+=e^{-\gamma}.
$$

\noindent
Д\,о\,к\,а\,з\,а\,т\,е\,л\,ь\,с\,т\,в\,о\,.\ \
Из $\varphi_n(t)\hm=\varphi(t)^{\beta_n}$ следует $f_n(u)\hm=f(u^{1/\beta_n})$. Имеем
\begin{multline*}
{\sf P}(M_n\le u_n(s))=f_n\left(n\varphi_n(u_n(s))\right)={}\\
{}=
f\left(n^{1/\beta_n}\varphi\left(1+\fr{(1+o(1))\ln s}{n}\right)\right)={}\\
{}=f\left(e^{((1-\beta_n)\ln n)/\beta_n}\left(-\fr{\ln s}{\mu}\right)\right)\to{}\\
{}\to
f\left(-\fr{e^{-\gamma}\ln s}{\mu}\right)\,,\quad n\to\infty\,.
\end{multline*}
Частичные индексы получаем из соотношения:
\begin{multline*}
\log_s\psi(s)=\fr{\ln f\left(-e^{-\gamma}(\ln s)/\mu\right)}{\ln s}={}\\
{}=
e^{-\gamma}\fr{\ln f(-(\ln r)/\mu)}{\ln r}\,,
\enskip r=s^{e^{-\gamma}}\in (0,1)\,,
\end{multline*}
где дробь в~правой части представляет собой логарифм экстремальной
функции из теоремы~5.1, принимающий значения от $x_0/\mu$ до~1.\hfill$\square$

\smallskip

В частности, результат примера~5.1 для копулы Гум\-бе\-ля--Хоу\-га\-ар\-да
получаем при $\varphi_n(t)\hm=(-\ln t)^{\alpha_n}$, где
$\varphi(t)\hm=-\ln t$ соответствует $\zeta\hm=1$ п.~н.\ и~копуле
независимости.

Таким образом, видно, что в~рассмотренных моделях
с~копулами могут быть любые $\theta\hm\in [0,1]$
и~любые $0\hm\le\theta^-\hm<\theta^+\hm\le 1$.

Рассмотрим теперь один поучительный пример модели
с~копулами и~случайными длинами серий.

\smallskip

\noindent
\textbf{Пример~5.4.}\ Пусть длины серий удовлетворяют условию
$\nu_n/n\stackrel{d}{\to}\zeta$, $n\hm\to\infty$, где~$\zeta$~имеет
устойчивое распределение с~преобразованием
Лап\-ла\-са--Стилть\-еса ${\sf E}e^{-u\zeta}\hm=e^{-u^\beta}$,
$0\hm<\beta\hm<1$, и~в~каж\-дой серии случайные величины
(не зависящие от~$\nu_n$) связаны копулой Гум\-бе\-ля--Хоу\-га\-ар\-да
с~$\alpha_n\hm>1$, $(\alpha_n-1)\ln n\hm\to\gamma>0$,
$n\hm\to\infty$ (см.\ пример~5.1).

Предположим сначала, что 
$$
u_n(s)^n\to e^{-\tau},\ n\to\infty,\ \tau>0.
$$ 
Тогда
\begin{multline*}
{\sf E}u_n(s)^{\nu_n}={\sf E}\left(u_n(s)^n\right)^{\nu_n/n}\to
{\sf E}e^{-\tau\zeta}=e^{-\tau^\beta}\,,\\
 n\to\infty\,.
\end{multline*}
Возьмем $\tau=(-\ln s)^{1/\beta}$, тогда ${\sf E}u_n(s)^{\nu_n}\hm\to s$, что и~требовалось.

Далее имеем:
\begin{multline}
\label{p335}
{\sf P}(M_n\le u_n(s))=u_n(s)^{\nu_n^{1/\alpha_n}}={}\\
{}=
\left(u_n(s)^{n^{1/\alpha_n}}\right)^{(\nu_n/n)^{1/\alpha_n}}\to{}\\
{}\to{\sf E}e^{-\tau e^{-\gamma}\zeta}=
e^{-(e^{-\gamma}\tau)^\beta}=s^{e^{-\gamma\beta}}\,,\enskip n\to\infty\,.
\end{multline}
Отсюда экстремальный индекс по определению~1 равен~$e^{-\gamma\beta}$.

С другой стороны, для любого $\theta\hm>0$ верно
$$
{\sf E}u_n(s)^{\theta\nu_n}\to
{\sf E}e^{-\tau\theta\zeta}=e^{-(\theta\tau)^\beta}=s^{\theta^\beta}\,,\enskip
n\to\infty\,,
$$
откуда и~из~(\ref{p335}) следует, что экстремальный индекс по определению~2
равен~$e^{-\gamma}$.

Таким образом, система имеет два \textit{разных} экстремальных индекса
по двум разным определениям!

Во всех рассмотренных ранее моделях классическое свойство~(\ref{Mhat})
сохраняло силу в~форме $\psi(s)\hm\ge s$ при всех $s\hm\in [0,1]$.
В~заключение рассмотрим пример, когда оно нарушается. При этом
симметричную зависимость случайных величин в~серии можно описать некоторой копулой,
но проще сделать это конструктивно, построением.

\smallskip

\noindent
\textbf{Пример 5.5.} Пусть $\eta_{n,m}$, $m\hm\ge 1$, $n\hm\ge 1$,
независимы и~имеют равномерное распределение на $[0,1]$, $\nu_n\hm=n$,
$\kappa_n$ принимают значения от~1 до~$n$ равновероятно и~не
зависят от~$\eta_{n,m}$,
$1\hm\le m\hm\le n$; $\gamma\hm>0$. Положим
$$
\xi_{n,m}=\begin{cases}
\eta_{n,m}^{1/(\gamma n)}\,, &\ m=\kappa_n\,;\\
\eta_{n,m}\,, &\ m\ne\kappa_n\,.
\end{cases}
$$
Тогда совместная функция распределения $\xi_{n,m}$, $1\hm\le m\hm\le n$, имеет вид:
$$
F^{(n)}(x_1,\dots x_n)=\left(\prod\limits_{m=1}^n x_m\right)
\left(\fr{1}{n}\sum\limits_{m=1}^n x_m^{\gamma n-1}\right)\,,
$$
откуда
\begin{gather*}
F_n(x)=x\left(1+\fr{x^{\gamma n-1}-1}{n}\right)\,;\\
{\sf P}(M_n\le x)=x^{(1+\gamma)n-1}\,.
\end{gather*}
Полагая 
$$
u_n(s)=1-\fr{\left(1+o(1)\right)\tau}{n}\,,\enskip 
\tau>0\,,
$$
 получаем при $n\hm\to\infty$:
\begin{align*}
F_n(u_n(s))^n&\to e^{-\tau}\exp\{e^{-\gamma\tau}-1\}=s\,;\\
 {\sf P}(M_n\le u_n(s))&\to e^{-(1+\gamma)\tau}=\psi(s)\,.
 \end{align*}
Отсюда можно найти в~явном виде обратную экстремальную функцию:
$$
\psi^{-1}(u)=u^{1/(1+\gamma)}\exp\left\{u^{\gamma/(1+\gamma)}-1\right\}\,,
$$
для которой верно $\psi^{-1}(u)\hm>u$ при всех $u\hm\in (0,1)$,
а~значит, $\psi(s)\hm<s$ при всех $s\hm\in (0,1)$.
В~этом случае можно показать, что $\theta^-\hm=1$, $\theta^+\hm=1\hm+\gamma>1$.

\section{Пороговые модели}

До сих пор рассматривались модели, в~которых величина~$\nu_n$
определялась внешними причинами по отношению к~величинам $\{\xi_{n,m}\}$.
Теперь введем
модели, в~которых~$\nu_n$ представляет собой момент остановки относительно
последовательности
$\{\xi_{n,m}$, $m\hm\ge 1\}$, где~$\xi_{n,m}$, $m\hm\ge 1$, независимы
и~имеют равномерное распределение
на $[0,1]$, а~остановка происходит в~момент превышения очередной случайной величиной
некоторого порога.

Подобные модели могут возникать, например, в~задачах автоматизированного
поиска объектов, обладающих некоторыми свойствами, простым перебором.

Заметим, что в~\cite{Novak-2013, Novak-1991} рассматривалась модель максимумов
случайных величин, в~которой остановка
происходила в~момент превышения порога (а~именно, времени~$t$) не очередной
величиной, а~их накопленной суммой. Это обобщение классической задачи о~наиболее
длинной серии успехов в~испытаниях Бернулли~\cite[\S 8.5]{EKM}. Однако в~этом случае
момент остановки растет просто асимптотически пропорционально порогу
и~не происходит таких интересных эффектов, как в~рас\-смат\-ри\-ва\-емых моделях.

\smallskip

\noindent
\textbf{Пример~6.1.}\
Пусть задана числовая последовательность
$a_n\hm\in (0,1)$, $n\hm\ge 1$, и~$a_n\hm\to 1$, $n\hm\to \infty$.
Обозначим $\varepsilon_n\hm=1\hm-a_n\hm>0$, тогда
$\varepsilon_n\hm\to 0$, $n\hm\to\infty$.
Положим $\nu_n\hm=\min\{m\ge 1: \xi_{n,m}>a_n\}$. Тогда
\begin{multline}
\label{mot}
{\sf P}(M_n\le u_n(s))=
{\sf P}(\xi_{n,\nu_n}\le u_n(s))={}\\
{}=
{\sf P}(\xi_{1,1}\le u_n(s)|\xi_{1,1}>a_n)={}\\
{}=0\vee \fr{u_n(s)-a_n}{\varepsilon_n}=0\vee \left(1-\fr{1-u_n(s)}{\varepsilon_n}\right),
\end{multline}
где $u_n(s)$ определяются из условия:
\begin{equation}
\label{eot}
{\sf E}u_n(s)^{\nu_n}=
\fr{\varepsilon_n u_n(s)}{1-(1-\varepsilon_n)u_n(s)}\to s\,,\enskip n\to\infty\,,
\end{equation}
поскольку $\nu_n$ имеет геометрическое распределение
(начиная от единицы) с~параметром~$\varepsilon_n$.
Из~(\ref{eot}) следует:
\begin{equation}
\label{1mun}
1-u_n(s)\sim \varepsilon_n\fr{1-s}{s}\,,\quad n\to\infty\,.
\end{equation}
Подставляя это в~(\ref{mot}) и~переходя к~пределу, получаем
$\psi(s)\hm=0\vee (2-1/s)$ по определению~1. В~этом случае, как
и~в~примере~5.5, $\psi(s)\hm<s$ при всех $s\hm\in (0,1)$. Имеем $\psi(s)\hm=0$,
а~значит, $\log_s\psi(s)\hm=+\infty$ при $s\hm\in [0,1/2]$,
$\log_s\psi(s)\hm>1$ при $s\hm\in (1/2,1)$, а~также $\log_s\psi(s)\hm\to 1$,
$s\hm\to 1$. Отсюда $\theta^-\hm=1$, $\theta^+\hm=+\infty$.

Удивительно, что результат не зависит от выбора
последовательности~$a_n$, $n\hm\ge 1$.

Что можно сказать в~этом случае об экстремальном индексе по определению~2?
Из~(\ref{eot}) с~учетом~(\ref{1mun}) получаем:
\begin{multline*}
{\sf E}u_n(s)^{\theta\nu_n}=
\fr{\varepsilon_n u_n(s)^\theta}{1-(1-\varepsilon_n)u_n(s)^\theta}\to{}\\
{}\to
\fr{s}{\theta+(1-\theta)s}\,,\enskip n\to\infty\,,
\end{multline*}
однако экстремальная функция не имеет такого вида; следовательно,
экстремального индекса по определению~2 не существует.

Рассмотрим теперь модель со случайными порогами~$\zeta_n$, $n\hm\ge 1$.
Пусть $0\hm<\zeta_n\hm<1$~п.~н.; $\xi_{n,m}$, $m\hm\ge 1$, не зависят от~$\zeta_n$
и~$\nu_n\hm=\min\{m\ge 1: \xi_{n,m}>\zeta_n\}.$

\smallskip

\noindent
\textbf{Теорема~6.1.}\ \textit{Пусть $n(1-\zeta_n)\stackrel{L_1}{\to}\zeta\hm>0$,
$n\hm\to\infty$, ${\sf E}\zeta\hm=1$.
Тогда $\psi(s)\hm=g(f^{-1}(s))$, где
$f(t)\hm={\sf E}(\zeta/(t+\zeta))$ и~$g(t)\hm={\sf E}(\zeta-t)_+$.}


\noindent
Д\,о\,к\,а\,з\,а\,т\,е\,л\,ь\,с\,т\,в\,о\,.\ \
 При условии, что $\zeta_n\hm=x\in (0,1)$, длина серии~$\nu_n$
 имеет геометрическое распределение с~параметром $1\hm-x$. Отсюда следует,
 что $(1\hm-\zeta_n)\nu_n\stackrel{d}{\to}\eta$
 и~$\nu_n/n\stackrel{d}{\to}\eta/\zeta$, $n\hm\to\infty$, где~$\eta$~имеет
 стандартное показательное
распределение и~не зависит от~$\zeta$. Обозначим $f(t)\hm={\sf E}e^{-t\eta/\zeta}$,
тогда $f(t)\hm={\sf E}(\zeta/(t+\zeta))$.

Пусть $\tau\hm>0$, тогда
\begin{multline*}
{\sf E}\left(1-\fr{\tau}{n}\right)^{\nu_n}={\sf E}
\left(\left(1-\fr{\tau}{n}\right)^n\right)^{\nu_n/n}\to{}\\
{}\to
{\sf E}e^{-\tau\eta/\zeta}=f(\tau)\,,\quad n\to\infty\,.
\end{multline*}
Так что из ${\sf E}u_n(s)^{\nu_n}\hm\to s $ следует
$$
u_n(s)=1-\fr{(1+o(1))f^{-1}(s)}{n}\,,\enskip n\to\infty\,.
$$

Получаем
\begin{multline*}
{\sf P}(M_n\le u_n(s))={\sf P}(\xi_{1,1}\le u_n(s)|\xi_{1,1}>\zeta_n)={}\\
{}=\fr{{\sf P}(\zeta_n<\xi_{1,1}\le u_n(s))}{\varepsilon_n}={}\\
{}=\fr{{\sf E}(1-(1+o(1))f^{-1}(s)/n-\zeta_n)_+}{\varepsilon_n}\to{}\\
{}\to {\sf E}(\zeta-f^{-1}(s))_+,\enskip n\to\infty\,.
\end{multline*}

%\smallskip

Напомним удобную для вычислений формулу
\begin{equation}
\label{forepl}
{\sf E}(\zeta-t)_+=\int\limits_t^{+\infty}{\bar F}_\zeta(x)\,dx\,,
\end{equation}
получаемую интегрированием по частям.

\smallskip

\noindent
\textbf{Пример~6.2.}\
Пусть~$\zeta$ равновероятно принимает значения $1\hm-\delta$ и~$1\hm+\delta$,
$0\hm<\delta\hm<1$
(случай $\delta\hm=0$ сводится к~примеру~6.1).
Тогда
\begin{multline*}
f(t)=\fr{1}{2}\left(\fr{1}{1+t/(1-\delta)}+\fr{1}{1+t/(1+\delta)}\right)={}\\
{}=
\fr{(1+t)-\delta^2}{(1+t)^2-\delta^2}\,,
\end{multline*}
откуда
\begin{gather*}
f^{-1}(s)=\fr{1+\sqrt{1-4s(1-s)\delta^2}}{2s}-1\,;
\\
\psi(s)=\fr{1}{2}\left(1-\delta-f^{-1}(s)\right)_+
+\fr{1}{2}\left(1+\delta-f^{-1}(s)\right)_+.
\end{gather*}

Имеем $\psi(s)=0$ при $0\hm<s\hm<f(1+\delta)\hm=(2\hm-\delta)/4$, так что
$\theta^+\hm=+\infty$.

В общем случае можно исследовать асимптотическое поведение
экстремальной функции при
$s\hm\to 0$ и~$s\hm\to 1$. Обозначим:
$$
\theta_0=\lim\limits_{s\to 0}\log_s\psi(s)\,;\quad
\theta_1=\lim\limits_{s\to 1}\log_s\psi(s)\,.
$$
Можно утверждать, что $\theta^-\hm\le \theta_0\wedge \theta_1$
и~$\theta^+\hm\ge \theta_0\vee \theta_1$.

\smallskip

\noindent
\textbf{Следствие~6.1.}
\begin{enumerate}[(1)]
\item \textit{Если ${\bar F}_\zeta(x)\hm\sim Cx^{-\alpha}$, $x\hm\to\infty$,
$C\hm>0$, $\alpha\hm>1$, то
$\psi(s)\hm\sim Cs^{\alpha-1}/(\alpha\hm-1)$, $s\hm\to 0$,
и~$\theta_0\hm=\alpha-1$.
Если ${\bar F}_\zeta(x)$ убывает быстрее любой степени, то} $\theta_0\hm=+\infty$.
\item
\textit{Если ${\sf E}\zeta^{-1}\hm<\infty$, то $\theta_1\hm=1/{\sf E}\zeta^{-1}$.
Если ${\sf E}\zeta^{-1}\hm=\infty$, то} $\theta_1\hm=0$.
\end{enumerate}



\noindent
Д\,о\,к\,а\,з\,а\,т\,е\,л\,ь\,с\,т\,в\,о\,.\ \
\begin{enumerate}[(1)]
\item Заметим, что 

\noindent
$$
f(t)={\sf E}\fr{\zeta}{\zeta+t}\sim {\sf E}\fr{\zeta}{t}=\fr{1}{t}\,,\enskip 
t\to\infty\,.
$$
Поэтому $f^{-1}(s)\sim {1}/{s}$, $s\hm\to 0$.
Из ${\bar F}_\zeta(x)\hm\sim Cx^{-\alpha}$, $x\hm\to\infty$,
по формуле~(\ref{forepl}) следует 

\noindent
$$
g(t)\sim \fr{Ct^{-(\alpha-1)}}{\alpha-1}\,,\enskip
t\to\infty\,.
$$
 Следовательно,
 
 \noindent
$$
\psi(s)=g(f^{-1}(s))\sim \fr{Cs^{\alpha-1}}{\alpha-1},\ s\to 0,
$$
и~$\theta_0\hm=\alpha\hm-1$.

Если ${\bar F}_\zeta(x)\hm=o(x^N)$, $x\hm\to\infty$, то $g(t)\hm=o(t^{N-1})$,
$t\hm\to\infty$, и~$\psi(s)\hm=o(s^{N-1})$,
$s\hm\to 0$, для любого $N\hm>0$, откуда $\theta_0\hm=+\infty$.
\item
При ${\sf E}\zeta^{-1}\hm<\infty$ имеем

\noindent
$$
1-f(t)={\sf E}\left(\fr{t}{\zeta+t}\right)\sim t{\sf E}\zeta^{-1},\ t\to 0.
$$
 Отсюда
 
 \noindent
$$
f^{-1}(s)\sim \fr{1-s}{{\sf E}\zeta^{-1}},\ s\to 1.
$$ 
Кроме того,
$1\hm-g(t)\sim t$, $t\hm\to 0$. Следовательно,


\noindent
$$
1-\psi(s)\sim \fr{1-s}{{\sf E}\zeta^{-1}},\ s\to 0,
$$

\noindent
откуда
$\theta_1\hm=1/{\sf E}\zeta^{-1}$. Результат для
${\sf E}\zeta^{-1}\hm=\infty$ получаем предельным переходом.
\end{enumerate}


По следствию~6.1 в~примере~6.2 получаем $\theta_0\hm=+\infty$;
$\theta_1\hm=1\hm-\delta^2\hm\in (0,1)$.

Понятно, что если из показателей~$\theta_0$ и~$\theta_1$ один больше единицы,
а~другой меньше,
то график $\psi(s)$ неизбежно пересекает диагональ. При этом свойство~(\ref{Mhat})
в~форме $\psi(s)\hm\ge s$ выполняется при некоторых $s\hm\in (0,1)$,
а~при некоторых~--- нарушается.

\vspace*{-9pt}

\section{Заключение}

\vspace*{-2pt}

В работе проведено
обобщение понятия экстремального индекса стационарной случайной последовательности
на схему серий со случайными длинами (двумя определениями). Изучены свойства новых
экстремальных индексов. Рассмотрены их различные приложения для моделей
информационных сетей и~биологических популяций, моделей с~копулами
и~пороговых моделей. Приведены примеры, когда
существуют оба экстремальных индекса, только один из них или ни одного. В~случаях,
когда экстремальный индекс по первому определению не существует, найдены
частичные индексы.
Таким образом, сделан ряд важных шагов в~построении нового математического
аппарата, имеющего
теоретическое и~прикладное значение для описания экстремального поведения
различных систем.

Разумеется, исследование новых экстремальных индексов, их свойств
и~приложений не может
исчерпываться рамками одной статьи. Данная работа скорее призвана открыть
цикл статей,
а~возможно, и~целое научное направление, к~которому могут присоединиться
и~другие авторы,
подобно тому как это происходит в~исследованиях классического экстремального индекса.

\vspace*{-9pt}

{\small\frenchspacing
 {%\baselineskip=10.8pt
 \addcontentsline{toc}{section}{References}
 \begin{thebibliography}{99}
 
 \vspace*{-2pt}
 
 
\bibitem{LLR} %1
\Au{Лидбеттер М., Линдгрен~Г., Ротсен~Х.} Экстремумы случайных
последовательностей и~процессов~/ Пер. с~англ.~--- М.: Мир, 1989.
392~c. (\Au{Leadbetter M.\,R., Lindgren~G., Rootzen~H.} Extremes and
related properties of random sequences and processes.~--- Springer,
1986. 336~p.)
\bibitem{Gal} %2
\Au{Галамбош Я.\,И.} Асимптотическая теория экстремальных порядковых
статистик~/ Пер. с~англ.~--- М.: Наука, 1984. 304~c. (\Au{Galambos~J.}
The asymptotic theory of extreme order statistics.~--- New York, NY, USA:
Wiley, 1978. 352~p.)
\bibitem{EKM} %3
\Au{Embrechts P., Kl$\ddot{\mbox{u}}$uppelberg~C., Mikosh~T.} Modelling extremal
events for insurance and finance.~--- Springer, 2003. 638~p.
\bibitem{HF} %4
\Au{De Haan L., Ferreira~A.} Extreme value theory. An introduction.~---
Springer, 2006. 420~p.

\pagebreak

\bibitem{Novak-2013} %5
\Au{Новак С.\,Ю.} Предельные теоремы и~оценки ско\-рости сходимости
в~теории экстремальных значений: Дис.~\ldots\ докт. физ.-мат. наук.~---
СПб.: ПОМИ РАН, 2014. 230~c.
\bibitem{Markovich1} %6
\Au{Markovich N.\,M.} Modeling clusters of extreme values~// Extremes, 2013.
Vol.~17. No.\,1. P.~97--125.
\bibitem{Markovich2} %7
\Au{Markovich N.\,M.} Quality assessment of the packet transport of
peer-to-peer video traffic in high-speed networks~// Perform. Evaluation, 2013.
Vol.~70. No.\,1. P.~28--44.
\bibitem{Markovich-new} %8
\Au{Avrachenkov K., Markovich~N.\,M., Sreedharan~J.\,K.} Distribution
and dependence of extremes in network sampling processes. INRIA
Research Report No.\,8578, 2014. 25~p.
{\sf http://arxiv.org/abs/1408.2529}.
\bibitem{Gold} %9
\Au{Голдаева А.\,А.} Тяжелые хвосты, экстремумы и~кластеры линейных
стохастических рекуррентных последовательностей: Дис.\ \ldots\ канд.
физ.-мат. наук.~--- М.: МГУ, 2014. 94~c.
\bibitem{Gold1} %10
\Au{Голдаева А.\,А.} Равномерная оценка экстремального индекса стохастических
рекуррентных последовательностей~// Вестник Моск. ун-та. Сер.~1. Математика.
Механика, 2012. №\,2. С.~51--55.
\bibitem{Gold2}%11
\Au{Голдаева А.\,А.} Экстремальные индексы и~кластеры в~линейных
стохастических рекуррентных последовательностях~// Теория
вероятностей и~ее применения, 2013. Т.~58. №\,4. С. 795--804.
\bibitem{Choi} %12
\Au{Choi H.} Central limit theory and extremes of random fields.~--- Chapel Hill:
University of North Carolina at Chapel Hill, 2002.  PhD Diss.
\bibitem{Exp1} %13
\Au{Ferreira H., Pereira~L.} How to compute the extremal index of stationary
random fields~//
Stat. Probabil. Lett., 2008. Vol.~78. P.~1301--1304.
\bibitem{Exp2}
\Au{Pereira L.} The asymptotic location of the maximum of a stationary random field~//
Stat. Probabil. Lett., 2009. Vol.~79. P.~2166--2169.
\bibitem{Sav}
\Au{Савинов Е.\,А.} Предельная теорема для максимума случайных величин,
связанных IT-ко\-пу\-ла\-ми $t$-рас\-пре\-де\-ле\-ния Стьюдента~//
Теория вероятностей
и~ее применения, 2014. Т. 59. №\,3. С. 594--602.
\bibitem{Leb3}
\Au{Лебедев А. В.} Максимумы активности в~случайных сетях
в~случае тяжелых хвостов~// Проблемы передачи информации, 2008. T.~44.
№\,2. С.~96--100.
\bibitem{Leb4}
\Au{Лебедев А.\,В.} Максимумы активности в~безмасштабных случайных сетях
с~тяжелыми хвостами~// Информатика и~её применения, 2011. T.~5. Вып.~4. С.~13--16.
\bibitem{Leb-Nc}
\Au{Лебедев А.\,В.} Максимумы активности в~некоторых моделях информационных сетей
со случайными весами и~тяжелыми хвостами~// Проблемы передачи информации, 2015. Т.~51.
№\,1. С.~72--81.
\bibitem{Pavl} %19
\Au{Павлов Ю.\,Л.} О~предельных распределениях степеней вершин
в~условных ин\-тер\-нет-гра\-фах~// Дискретная математика, 2009. T.~21. №\,3.
C.~14--23.
\bibitem{Leri}
\Au{Лери М.\,М., Чеплюкова И.\,А.} Об одной статистической задаче для случайных
графов ин\-тер\-нет-ти\-па~// Информатика и~её применения, 2011. Т.~5. Вып.~3. С.~34--40.
\bibitem{Raig}
\Au{Райгородский А.\,М.} Модели случайных графов и~их применения~// Труды МФТИ,
2010. Т.~2. №\,4. С.~130--140.
\bibitem{Hofstad}
\Au{Van der Hofstad R.} Random graphs and complex networks.~---
Eindhoven University of Technology, 2014.  Vol.~1. 328~p.
{\sf http://www.win.tue.nl/ $\sim$\/rhofstad/NotesRGCN.pdf}.
\bibitem{Sen}
\Au{Сенета Е.} Правильно меняющиеся функции~/ Пер. с~англ.~--- М.:
Наука, 1985. 144~c. (\Au{Seneta~E.} Regularly varying functions.~--- Springer, 1976.
116~p.)
\bibitem{Stam} %24
\Au{Stam A.\,J.} Regular variation of the tail of a~subordinated probability
distribution~// Adv. Appl. Probab., 1973. Vol.~5. P.~308--327.
\bibitem{Leb-2005c} %25
\Au{Лебедев А.\,В.} Общая схема максимумов сумм независимых
случайных величин и~ее приложения~// Математические заметки, 2005.
T.~77. №\,4. C.~544--550.
\bibitem{AV} %26
\Au{Arnold B.\,C., Villasenor~J.\,A.} The tallest man in the world~//
Statistical theory and applications: Papers in honor of Herbert~A.~David~/
Eds. H.\,N.~Nagaraja, P.\,K.~Sen, D.\,F.~Morrison.~---
New York, NY, USA: Springer-Verlag New York, 1996. P.~81--88.
\bibitem{Pakes}
\Au{Pakes A.\,G.} Extreme order statistics on Galton--Watson trees~//
Metrika, 1998. Vol.~47. No.\,1. P.~95--117.
\bibitem{Leb5}
\Au{Lebedev A.\,V.} Maxima of random particles scores in Markov
branching processes with continuous time~// Extremes, 2008. Vol.~11. No.\,2. P.~203--216.
\bibitem{Leb60}
\Au{Лебедев А.\,В.} Максимумы случайных признаков час\-тиц
в~надкритических ветвящихся процессах~// Вестник Моск. ун-та.  Сер.~1.
Математика. Механика, 2008. №\,5. С.~3--6.
\bibitem{Leb6}
\Au{Лебедев А.\,В.} Многомерные экстремумы случайных признаков частиц
в~надкритических ветвящихся
процессах~// Теория вероятностей и~ее применения, 2012. Т.~57. №\,4. С.~788--794.
\bibitem{Leb7} %31
\Au{Лебедев А.\,В.} Асимптотическое поведение экстремумов случайных
признаков частиц в~ветвящихся процессах с~наследственностью~//
Ярославский педагогический вестник. Сер. Фи\-зи\-ко-ма\-те\-ма\-ти\-че\-ские
и~естественные науки, 2010. №\,1. С.~7--14.

\bibitem{QRM} %32
\Au{McNeil A.\,J., Frey~R., Embrechts~P.}
Quantitative risk management.~--- Princeton University Press, 2005. 538~p.

\bibitem{Nel} %33
\Au{Nelsen R.} An introduction to copulas.~--- Springer, 2006. 276~p.

\bibitem{Chat} %34
\Au{Chatzis S.\,P., Demiris~Y.} The copula echo state network~//
Pattern Recogn., 2012. Vol.~45. P.~570--577.
\bibitem{Fel}
\Au{Феллер В.} Введение в~теорию вероятностей и~ее приложения~/
Пер. с~англ.~--- М.: Мир, 1984.  Т.~2. 752~c.
(\Au{Feller~W.} Аn introduction to probability and its applications.~---
New York, NY, USA: Wiley, 1971.   Vol.~2. 668~p.)
\bibitem{Novak-1991}
\Au{Новак С.\,Ю.} О~распределении максимума случайного числа случайных
величин~// Теория вероятностей и~ее применения, 1991. Т.~36. №\,4. С.~675--681.
 \end{thebibliography}

 }
 }

\end{multicols}

\vspace*{-6pt}

\hfill{\small\textit{Поступила в~редакцию 13.02.15}}

\newpage

%\vspace*{12pt}

%\hrule

%\vspace*{2pt}

%\hrule

\vspace*{-24pt}

\def\tit{EXTREMAL INDICES IN~A~SERIES SCHEME AND~THEIR~APPLICATIONS}

\def\titkol{Extremal indices in~a~series scheme and~their applications}

\def\aut{A.\,V.~Lebedev}

\def\autkol{A.\,V.~Lebedev}

\titel{\tit}{\aut}{\autkol}{\titkol}

\vspace*{-9pt}


\noindent
Faculty of Mechanics and Mathematics, M.\,V.~Lomonosov Moscow State
University, 1-52~Leninskiye Gory, GSP-1, Moscow 119991, Russian Federation


\def\leftfootline{\small{\textbf{\thepage}
\hfill INFORMATIKA I EE PRIMENENIYA~--- INFORMATICS AND
APPLICATIONS\ \ \ 2015\ \ \ volume~9\ \ \ issue\ 3}
}%
 \def\rightfootline{\small{INFORMATIKA I EE PRIMENENIYA~---
INFORMATICS AND APPLICATIONS\ \ \ 2015\ \ \ volume~9\ \ \ issue\ 3
\hfill \textbf{\thepage}}}

\vspace*{3pt}


\Abste{The concept of an extremal index of a stationary random sequence
is generalized to a series scheme of identically distributed random variables
with random series sizes tending to infinity in probability. The new extremal
indices are introduced through two definitions generalizing the basic properties
of the classical extremal index. Some useful properties of the new extremal
indices are proved. The paper shows how the behavior of aggregate activity
maxima on random graphs (in information network models) and the behavior of
maxima of random particles scores in branching processes (in biological populations
models) can be described in terms of the new extremal indices. New results on
models with copulas and threshold models are obtained. The paper shows that the
new indices can take different values for one system and the values greater than one.}

\KWE{extremal index; series scheme; random graph;
information network; branching process; copula}


\DOI{10.14357/19922264150305}

\Ack
The research was supported by the Russian Foundation for Basic Research
(project 14-01-00075).


%\vspace*{3pt}

  \begin{multicols}{2}

\renewcommand{\bibname}{\protect\rmfamily References}
%\renewcommand{\bibname}{\large\protect\rm References}

{\small\frenchspacing
 {%\baselineskip=10.8pt
 \addcontentsline{toc}{section}{References}
 \begin{thebibliography}{99}
\bibitem{LLR-1}
\Aue{Leadbetter, M.\,R., G.~Lindgren, and H.~Rootzen}.
1986. \textit{Extremes and related properties
of random sequences and processes.} Spinger. 336~p.
\bibitem{Gal-1}
\Aue{Galambos, J.} 1978. \textit{The asymptotic
theory of extreme order statistics}. New York, NY: Wiley. 352~p.
\bibitem{EKM-1}
\Aue{Embrechts, P., C.~Kl$\ddot{\mbox{u}}$uppelberg, and T.~Mikosh}.
2003. \textit{Modelling
extremal events for insurance and finance}. Springer. 638~p.
\bibitem{HF-1}
\Aue{De Haan, L., and A.~Ferreira}.
2006. \textit{Extreme value theory. An introduction.} Springer. 420~p.
\bibitem{Novak-2013-1} %5
\Aue{Novak, S.\,Yu.} 2014. Predelnye teoremy i~otsenki
skorosti skhodimosti v~teorii ekstremal'nykh znacheniy
[Limit theorems and convergence rate estimation in the extreme value theory].
 St.\ Petersburg.  D.Sc. Diss. 230~p.
\bibitem{Markovich1-1}
\Aue{Markovich, N.\,M.} 2013. Modeling clusters of extreme values.
\textit{Extremes} 17(1):97--125.
\bibitem{Markovich2-1}
\Aue{Markovich, N.\,M.} 2013.
Quality assessment of the packet transport of peer-to-peer video traffic
in high-speed networks. \textit{Perform. Evaluation} 70(1):28--44.
\bibitem{Markovich-new-1} %8
\Aue{Avrachenkov, K., N.\,M.~Markovich, and J.\,K.~Sreedharan}.
2014. Distribution and dependence of extremes
in network sampling processes. INRIA Research Report No.\,8578. 25~p.
Available at: {http://arxiv.org/abs/1408.2529} (accessed August~7, 2015).
\bibitem{Gold-1}
\Aue{Goldaeva, A.\,A.} 2014. Tyazhelye khvosty, ekstremumy i~klastery lineynykh
stokhasicheskikh rekurrentnykh
posledovatel'nostey [Heavy tails, extremes, and clusters of linear stochastic
recursive sequences].
 Moscow: MGU.  PhD Diss. 94~p.
\bibitem{Gold1-1}
\Aue{Goldaeva, A.\,A.} 2012. Uniform estimator of the extremal
index of stochastic recurrent sequences.
\textit{Moscow Univ. Math. Bull.} 67(2):82--85.
\bibitem{Gold2-1} %11
\Aue{Goldaeva, A.\,A.} 2013. Extremal indices and clusters
it the linear resursive stochastic sequences. 
\textit{Teoriya Ve\-ro\-yat\-no\-stey i~ee Primenenia}
[Theory Probab. Appl.] 58(4):689--698.
\bibitem{Choi-1} %12
\Aue{Choi, H.} 2002. Central limit theory and extremes of random fields.
 Chapel Hill: University of North Carolina at Chapel Hill. PhD Diss.
\bibitem{Exp1-1}
\Aue{Ferreira, H., and L.~Pereira}. 2008. How to compute the extremal
index of stationary random fields.
\textit{Stat. Probabil. Lett.} 78:1301--1304.
\bibitem{Exp2-1}
\Aue{Pereira, L.} 2009. The asymptotic location of the maximum of
a~stationary random field.
\textit{Stat. Probabil. Lett.} 79:2166--2169.
\bibitem{Sav-1} %15
\Aue{Savinov, E.\,A.} 2014. Predelnaya teorema dlya maksimuma sluchaynykh velichin,
svyazannykh IT-kopulami \mbox{$t$-ras}\-pre\-de\-leniya St'yudenta [Limit theorem for the maximum
of random variables connected by IT-copulas of Student's distribution]
\textit{Teoriya Veroyatnostey i~ee Primenenia} [Theory Probab. Appl.]
59(3):594--602.
%\bibitem{Panch-86}
%Pancheva, E. 1987. General limit theorems for maximum of independent random variables
%{\it Theory Probab. Appl.} 31(4): 645--657.
%\bibitem{Chup}
%Chuprunov, A.N. 2000. On convergence in law of maxima of independent identically distributed random
%variables with random coefficients {\it Theory Probab. Appl.} 44(1): 93--97.
\bibitem{Leb3-1} %16
\Aue{Lebedev, A.\,V.} 2008. Activity maxima in random networks
 in the heavy tail case. \textit{Problems Information Transmission}
 44(2):156--160.
\bibitem{Leb4-1} %17
\Aue{Lebedev, A.\,V.} 2011. Maksimumy aktivnosti v~bezmasshtabnykh sluchaynykh
setyakh s~tyazhelymi khvostami
[Activity maxima in free-scale random networks with heavy tails].
\textit{Informatika i~ee Primenenia}~--- \textit{Inform. Appl.} 5(4):13--16.
\bibitem{Leb-Nc-1}
\Aue{Lebedev, A.\,V.} 2015. Activity maxima in some models of information
networks with random weights and heavy tails. \textit{Problems
Information Transmission} 51(1):66--74.
\bibitem{Pavl-1}
\Aue{Pavlov, Yu.\,L.} 2009. On the limit distributions of the vertex degrees
of conditional Internet graphs. \textit{Discrete Math. Appl.}
19(4):349--359.
\bibitem{Leri-1} %20
\Aue{Leri, M.\,M.} 2011. Ob odnoy statisticheskoy zadache dlya
sluchaynykh grafov internet-tipa
[On a statistical problem for random Internet-type graphs].
\textit{Informatika i~ee Primenenia}~--- \textit{Inform. Appl.}  5(3):34--40.
\bibitem{Raig-1}
\Aue{Raigorodskii, A.\,M.} 2010. Modeli sluchaynykh grafov i~ikh
primeneniya [Models of random graphs
and their applications]. \textit{Trudy MFTI} [MIPT Proceedings] 2(4):130--140.
\bibitem{Hofstad-1} %22
\Aue{Van der Hofstad, R.} 2014. \textit{Random graphs and complex networks.} 
Eindhoven University of Technology. Vol.~1. 328~p. Available at:
{\sf  http://www.win.tue.nl/ $\sim$\/rhofstad/NotesRGCN.pdf}
(accessed August~7, 2015).
\bibitem{Sen-1}
\Aue{Seneta, E.} 1976. \textit{Regularly varying functions.} Springer. 116~p.
\bibitem{Stam-1}
\Aue{Stam, A.\,J.} 1973. Regular variation of the tail of
a~subordinated probability distribution.
\textit{Adv. Appl. Probab.} 5:308--327.
\bibitem{Leb-2005c-1}
\Aue{Lebedev, A.\,V.} 2005. General scheme of maxima of sums of independent
random variables and its applications.
\textit{Mathematical Notes} 77(4):503--509.
\bibitem{AV-1}
\Aue{Arnold, B.\,C., and J.\,A.~Villasenor}. 1996. The tallest man in the world.
\textit{Statistical theory and applications. Papers in honor of Herbert~A.~David.}
Eds.\ H.\,N.~Nagaraja, P.\,K.~Sen, and D.\,F.~Morrison.
New York, NY: Springer-Verlag New York. 81--88.
\bibitem{Pakes-1} %27
\Aue{Pakes, A.\,G.} 1998. Extreme order statistics on Galton--Watson trees.
\textit{Metrika} 47(1):95--117.
\bibitem{Leb5-1} %28
\Aue{Lebedev, A.\,V.} 2008. Maxima of random particles scores in Markov
branching processes with continuous time. \textit{Extremes} 11(2):203--216.
\bibitem{Leb60-1}
\Aue{Lebedev, A.\,V.}  2008. Maxima of random properties of particles in supercritical branching processes.
\textit{Moscow Univ. Math. Bull.} 63(5):175--178.
\bibitem{Leb6-1} %30
\Aue{Lebedev, A.\,V.} 2013.
Multivariate extremes of random properties of particles in supercritical branching
processes.
\textit{Teoriya Veroyatnostey i~ee Primeneniya} [Theory Probab. Appl.] 57(4):678--683.
\bibitem{Leb7-1} %31
\Aue{Lebedev, A.\,V.} 2010. Asimptoticheskoe povedenie eks\-tre\-mu\-mov slu\-chay\-nukh
priz\-na\-kov chastits v~vet\-vya\-shchikh\-sya protsessakh s~nasledstvennost'yu
[The asymptotic behavior of extremes of
random particles scores in branching processes with a~heredity]
\textit{Yaroslavskiy pedagogicheskiy
vestnik. Ser. Fiziko-matematicheskie i~estestvennye nauki}
[Yaroslavl Pedagogical Bull. Physics
and mathematics and natural sciences] 1:7--14. 

\bibitem{QRM-1} %32
\Aue{McNeil, A.\,J., R.~Frey, and P.~Embrechts}. 2005.
\textit{Quantitative risk management}. Princeton University Press. 538~p.

\bibitem{Nel-1} %33
\Aue{Nelsen, R.} 2006. \textit{An introduction to copulas.} Springer. 276~p.

\bibitem{Chat-1} %34
\Aue{Chatzis, S.\,P., and Y.~Demiris}. 2012.
The copula echo state network. \textit{Pattern Recogn.} 45:570--577.
\bibitem{Fel-1}
\Aue{Feller, W.} 1971. \textit{An introduction to probability and its applications.}
 New York, NY: Wiley. Vol.~2. 668~p.
\bibitem{Novak-1991-1}
\Aue{Novak, S.\,Yu.} 1992. On the distribution of the maximum of
a~random number of random variables.
\textit{Teoriya Veroyatnostey i~ee Primeneniya} [Theory Probab. Appl.]
36(4):714--721.
\end{thebibliography}

 }
 }

\end{multicols}

\vspace*{-3pt}

\hfill{\small\textit{Received February 13, 2015}}


\Contrl

\noindent
\textbf{Lebedev Alexey V.} (b.\ 1971)~---
Candidate of Science (PhD) in physics and mathematics, associate professor,
Faculty of Mechanics and Mathematics, M.\,V.~Lomonosov Moscow State
University, 1-52~Leninskiye Gory, GSP-1, Moscow 119991, Russian Federation;
avlebed@уandex.ru
\label{end\stat}


\renewcommand{\bibname}{\protect\rm Литература}