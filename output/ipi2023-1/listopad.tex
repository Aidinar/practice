\def\stat{listopad}

\def\tit{МЕТОД НА ОСНОВЕ НЕЧЕТКИХ ПРАВИЛ ДЛЯ~УПРАВЛЕНИЯ КОНФЛИКТАМИ 
АГЕНТОВ В~ГИБРИДНЫХ ИНТЕЛЛЕКТУАЛЬНЫХ МНОГОАГЕНТНЫХ СИСТЕМАХ}

\def\titkol{Метод на основе нечетких правил для управления конфликтами 
агентов в~ГиИМАС} %гибридных интеллектуальных многоагентных системах}

\def\aut{С.\,В.~Листопад$^1$, И.\,А.~Кириков$^2$}

\def\autkol{С.\,В.~Листопад, И.\,А.~Кириков}

\titel{\tit}{\aut}{\autkol}{\titkol}

\index{Листопад С.\,В.}
\index{Кириков И.\,А.}
\index{Listopad S.\,V.}
\index{Kirikov I.\,A.}


%{\renewcommand{\thefootnote}{\fnsymbol{footnote}} \footnotetext[1]
%{Работа выполнена при поддержке Министерства науки и~высшего образования
%Российской федерации, грант №\,075-15-2020-799.}}


\renewcommand{\thefootnote}{\arabic{footnote}}
\footnotetext[1]{Федеральный исследовательский центр <<Информатика и~управление>> 
Российской академии наук, \mbox{ser-list-post@yandex.ru}}
\footnotetext[2]{Федеральный исследовательский центр <<Информатика и~управление>> 
Российской академии наук, \mbox{baltbipiran@mail.ru}}

%\vspace*{-6pt}
 
 
  \Abst{Работа продолжает исследования по компьютерному моделированию гибридными 
интеллектуальными многоагентными сис\-те\-ма\-ми (ГиИМАС) работы коллектива специалистов различных 
профилей, решающих проблемы за круглым столом. Агенты таких сис\-тем~--- автономные 
программные сущности, имитирующие рассуждения реальных специалистов. 
Моделирование в~единой интеллектуальной сис\-те\-ме разнородных знаний, целей и~точек 
зрения агентов на поставленную проблему обусловливает их столкновение, возникновение 
конфликтов по аналогии с~тем, как это происходит в~моделируемых коллективах. Не каждый 
конфликт между агентами носит деструктивный характер и~требует подавления: управление 
конфликтом в~ГиИМАС, как и~в~коллективе, 
предполагает идентификацию ситуации принятия решений, при необходимости стимуляцию 
и~последующее разрешение конструктивных форм конфликта, а~также предотвращение его 
деструктивных форм. Для управ\-ле\-ния конфликтами между агентами 
в~ГиИМАС предлагается метод на основе 
нечетких правил.}
   
  \KW{конфликт; гибридная интеллектуальная многоагентная сис\-те\-ма; коллектив 
специалистов; управ\-ле\-ние конфликтами}

\DOI{10.14357/19922264230109} 
  
\vspace*{6pt}


\vskip 10pt plus 9pt minus 6pt

\thispagestyle{headings}

\begin{multicols}{2}

\label{st\stat}

\section{Введение}

\vspace*{-3pt}
  
  Как показано в~[1], для множества практических проблем, традиционно 
решаемых коллективом специалистов под руководством лица, принимающего 
решения, актуальна разработка \mbox{ГиИМАС}, моделирующих работу такого коллектива с~\mbox{целью} ускорения выработки 
решений в~условиях высокой динамичности внешней среды и~временн$\acute{\mbox{ы}}$х 
ограничений на принятие решения. Характерная особенность как указанных 
коллективов, так и~их компьютерных моделей~--- возникновение внутри них 
конфликтов различного характера и~необходимость задействовать механизмы 
управления ими для обеспечения эффективной работы коллектива. 
В~последние несколько десятилетий конфликты в~коллективах 
рассматриваются не только как препятствия, но и~как потенциальные 
источники для повышения эффективности групповой работы~[2]. Наличие 
противоречивых эмпирических данных о~роли конф\-лик\-тов в~коллективной 
работе обусловливает необходимость дальнейших исследований~[3--8]. 
  
  В~[9] показано, что конструктивная роль конфликтов проявляется 
преимущественно тогда, когда цели и~потребности конф\-лик\-ту\-ющих 
специалистов не противоположны целям, ценностям и~нормам коллектива,  
а~со\-ци\-аль\-но-пси\-хо\-ло\-ги\-че\-ская структура коллектива относительно 
гибкая,\linebreak т.\,е.\ для коллектива характерна относительная открытость подгрупп
 и~динамичные связи между ними и~не включенными в~них членами~[9]. 
Конструктивные конфликты~--- преимущественно \mbox{инструментальные} 
конфликты, возникающие по поводу проблемы или процесса ее решения. 
Открытое обсуждение и~споры по этим вопросам повышают результативность 
деятельности \mbox{группы}.
{\looseness=1

}
  
  Деструктивный конфликт возникает, когда цели и~по\-треб\-но\-сти 
конф\-лик\-ту\-ющих субъектов радикально расходятся с~целями, ценностями 
и~нормами коллектива, при этом коллектив обладает \mbox{жесткой} структурой, 
т.\,е.\ для него характерна закрытость подгрупп по отношению к~коллективу 
и~явное доминирование дезынтегративных связей между ними~[9]. Чаще всего 
деструктивные конфликты~--- конфликты по поводу отношений, т.\,е.\ 
разногласия между специалистами по личным вопросам и~не относящимся 
к~выполняемой работе проб\-ле\-мам, которые обусловлены не\-со\-вмес\-ти\-мостью 
и~враж\-деб\-ностью между ними.
  
  Управление конфликтом в~малых коллективах специалистов состоит 
в~предотвращении деструктивных конфликтов, но стимулировании 
и~разрешении конструктивных. В~[1] для управ\-ле\-ния конфликтами, 
возникающими между агентами, по аналогии с~тем, как это происходит 
в~коллективах\linebreak специалистов, предложена модель \mbox{ГиИМАС}  
с~проб\-лем\-но- и~про\-цес\-сно-ори\-ен\-ти\-ро\-ван\-ны\-ми конфликтами. 
\mbox{Цель} на\-сто\-ящей работы~--- разработка в~рамках данной модели метода 
управ\-ле\-ния \mbox{конфликтами} между агентами на основе нечетких правил.

\section{Моделирование конфликтов в~гибридных 
интеллектуальных многоагентных системах}
  
  Подробное описание модели \mbox{ГиИМАС} с~проб\-лем\-но-
  и~про\-цес\-сно-ори\-ен\-ти\-ро\-ван\-ны\-ми конфликтами пред\-став\-ле\-но 
в~[10]. Рассмотрим основные ее элементы, необходимые для описания 
процесса управ\-ле\-ния конфликтами. 
  
  Формально \mbox{ГиИМАС} в~целом определяется следующим 
образом~[10]:
  \begin{equation}
  \mathrm{himas} =\langle \mathrm{AG}^*, 
\mathrm{env},\mathrm{INT}^*,\mathrm{ORG}, \mathrm{MLP}\rangle\,,
  \label{e1-lis}
  \end{equation}
где $\mathrm{AG}^*\hm= \left\{ \mathrm{ag}_1, \ldots , \mathrm{ag}_n, 
\mathrm{ag}^{\mathrm{dm}}, \mathrm{ag}^{\mathrm{fc}}\right\}$~--- множество агентов, 
включающее $n$ аген\-тов-спе\-циа\-ли\-стов (АС) $\mathrm{ag}_i$, $i\hm\in 
\mathbb{N}$, $1\hm\leq i\hm\leq n$, агента, принимающего решения (АПР), 
$\mathrm{ag}^{\mathrm{dm}}$ и~аген\-та-фа\-си\-ли\-та\-то\-ра (АФ), управ\-ля\-юще\-го 
взаимодействиями агентов при решении проб\-ле\-мы и~конфликтами между 
ними, $\mathrm{ag}^{\mathrm{fc}}$; $\mathrm{env}$~--- концептуальная модель 
внешней среды сис\-те\-мы; $\mathrm{INT}^*$~--- множество элементов 
структурирования взаимодействий агентов~[10], содержащее среди прочего 
модель предметной об\-ласти $\mathrm{ont}$;\linebreak $\mathrm{ORG}$~--- множество архитектур 
\mbox{ГиИМАС}; $\mathrm{MLP}\hm= \{ \mathrm{cnffm}, \mathrm{gdid}\}$~--- множество 
концептуальных моделей макроуровневых процессов в~\mbox{ГиИМАС},\linebreak 
содержащее модель $\mathrm{cnffm}$ процесса нечеткого управ\-ле\-ния конфликтами 
агентов, которая представлена выражением~(\ref{e2-lis}), и~модель~$\mathrm{gdid}$ 
идентификации взаимозависимости целей агентов, описанную 
выражением~(\ref{e4-lis}). 

  Агент $\mathrm{ag}_{\mathrm{id}}\hm\in \mathrm{AG}^*$ из формулы~(\ref{e1-lis}) 
описывается выражением:
  $$
  \mathrm{ag}_{\mathrm{id}} =\left\langle \mathrm{id}_{\mathrm{id}}^{\mathrm{ag}}, \mathrm{gl}_{\mathrm{id}}^{\mathrm{ag}}, 
\mathrm{ACT}_{\mathrm{id}}^{\mathrm{ag}} \right\rangle\,,
  $$
где $\mathrm{id}_{\mathrm{id}}^{\mathrm{ag}}$~--- идентификатор агента; $\mathrm{gl}_{\mathrm{id}}^{\mathrm{ag}}$~--- нечеткая\linebreak цель 
агента, т.\,е.\ нечеткое множество с~функцией принадлежности 
$\mu_{\mathrm{id}}(\mathrm{pr}_1, \ldots , \mathrm{pr}_{\mathrm{NPRG}})$, заданной на 
подмножестве целевых кон\-цеп\-тов-свойств
$\mathrm{PR}^g\hm= \{ 
\mathrm{pr}_1, \ldots , \mathrm{pr}_{\mathrm{NPRG}}\}$ множества  
кон\-цеп\-тов-\linebreak\vspace*{-12pt}

\columnbreak

\noindent
 свойств $\mathrm{PR}^g\hm\subseteq \mathrm{PR}$ модели 
предметной области $\mathrm{ont}$; $\mathrm{ACT}_{\mathrm{id}}^{\mathrm{ag}}$~--- множество 
действий агента.


  Модель нечеткого управ\-ле\-ния конфликтами агентов описывается 
выражением:
  \begin{multline}
  \mathrm{cnffm} ={}\\
  {}=\left\langle \mathbf{CNF}, \mathrm{cnfcl}, \mathrm{ACT}^{\mathrm{afcfm}}, 
\mathrm{ACT}^{\mathrm{agcs}}, \mathrm{ACT}^{\mathrm{agcr}}\right\rangle,
  \label{e2-lis}
  \end{multline}
где $\mathbf{CNF}$~--- матрица конфликтов~(\ref{e3-lis}) между парами 
агентов; $\mathrm{cnfcl}$~--- классификатор конфликтов агентов~[10], формирующий  
мат\-ри\-цу~$\mathbf{CNF}$;\linebreak $\mathrm{ACT}^{\mathrm{afcfm}}\hm= \{ 
\mathrm{act}^{\mathrm{cnffm}}, \mathrm{act}^{\mathrm{cnfi}}, \mathrm{act}^{\mathrm{cnfs}}, 
\mathrm{act}^{\mathrm{cnfr}}\}$~--- множество функций АФ по управ\-ле\-нию 
конфликтами АС, содержащее функцию <<нечеткое управ\-ле\-ние конфликтом>> 
$\mathrm{act}^{\mathrm{cnffm}}$, обеспечивающую \mbox{идентификацию} 
$\mathrm{act}^{\mathrm{cnfi}}$ конфликтов с~по\-мощью классификатора $\mathrm{cnfcl}$ 
и~инициализацию функции стимуляции $\mathrm{act}^{\mathrm{cnfs}}$ или разрешения 
$\mathrm{act}^{\mathrm{cnfr}}$ конфликтов; $\mathrm{ACT}^{\mathrm{agcs}}$~--- множество 
действий АС, выполняемых при стимулировании противоречий~[11] АФ, 
$\mathrm{ACT}^{\mathrm{agcs}} \hm\subseteq \cup_{\mathrm{ag}_{\mathrm{id}}\in 
\mathrm{AG}^*}  \mathrm{ACT}_{\mathrm{id}}^{\mathrm{ag}}$; $\mathrm{ACT}^{\mathrm{agcr}}$~--- 
множество допустимых действий АС по разрешению противоречий~[12], 
$\mathrm{ACT}^{\mathrm{agcr}} \hm\subseteq \cup_{\mathrm{ag}_{\mathrm{id}}\in 
\mathrm{AG}^*} \mathrm{ACT}_{\mathrm{id}}^{\mathrm{ag}}$.
  
  Конфликт между агентами, т.\,е.\ элемент матрицы $\mathbf{CNF}$ из 
выражения~(\ref{e2-lis}), описывается следующей формулой:

\vspace*{-7pt}

\noindent
  \begin{multline}
  \mathrm{cnf}_{i\,j\,\mathrm{cnft}}={}\\
  {}= \left\langle \mathrm{ag}_i,\mathrm{ag}_j, \mathrm{cnfin}, \mathrm{cnft}, 
\mathrm{ACT}_i^{\mathrm{agcr}}, \mathrm{ACT}_j^{\mathrm{agcr}}\right\rangle,
  \label{e3-lis}
  \end{multline}
  
\vspace*{-3pt}

  \noindent
где $\mathrm{ag}_i$ и $\mathrm{ag}_j$~--- агенты-субъ\-ек\-ты конфликта, 
$i,j\hm\in \mathbb{N}$, $1\hm \leq i, j \hm\leq n$, $i\not= j$; $\mathrm{cnfin}$~--- 
напряженность конфликта в~виде скалярной величины $\mathrm{cnfin} \hm\in [0,1]$, 
вычисляемой классификатором конфликтов $\mathrm{cnfcl}$ в~соответствии 
с~реализуемой им мерой напряженности конфликта, в~зависимости от того, 
предлагают АС частные решения проб\-ле\-мы или альтернативные решения 
проблемы в~целом, могут использоваться меры на основе совместимости 
частных решений~[13] или на основе ранжирования альтернатив~[14, 15]; 
$\mathrm{cnft}$~--- символьная переменная <<тип конфликта>>, определенная на 
множестве $\mathrm{CNFT}\hm= \{  
\mathrm{cnfprb}$\;=\;<<{проб\-лем\-но-ори\-ен\-ти\-ро\-ван\-ный}>>, 
$\mathrm{cnfprc}$\;=\;<<{про\-цес\-сно-ори\-ен\-ти\-ро\-ван\-ный}>>$\}$; 
$\mathrm{ACT}_i^{\mathrm{agcr}}$ и~$\mathrm{ACT}_j^{\mathrm{agcr}}$~--- множество 
допустимых действий агентов~$\mathrm{ag}_i$ и~$\mathrm{ag}_j$ 
соответственно по разрешению противоречий, $\mathrm{ACT}_i^{\mathrm{agcr}} 
\hm\subseteq \mathrm{ACT}_i^{\mathrm{ag}}$, $\mathrm{ACT}_j^{\mathrm{agcr}}\hm\subseteq 
\mathrm{ACT}_j^{\mathrm{ag}}$, $\mathrm{ACT}_i^{\mathrm{agcr}}, \mathrm{ACT}_j^{\mathrm{agcr}} 
\hm\subseteq \mathrm{ACT}^{\mathrm{agcr}}$.

  Модель идентификации взаимозависимости целей описывается выражением
  
  \vspace*{2pt}
  
  \noindent
  \begin{equation}
\mathrm{gdid} =\langle \mathbf{GD}, \mathrm{gdi}\rangle.
  \label{e4-lis}
  \end{equation}
  
  \vspace*{-2pt}
  
  \noindent
Здесь $\mathbf{GD}$~--- мат\-ри\-ца показателей взаимоза\-ви\-си\-мости целей пар 
агентов, $\mathrm{gd}_{i\,j} \hm\in [-1,1]$: 

\noindent
$$
\mathrm{gd}_{i\,j}=\begin{cases}
-1 & \mbox{---~максимальная\ отрицательная}\\
&\hphantom{\mbox{---~}}\mbox{взаимозависимость};\\
\hphantom{-}0 & \mbox{---~независимость}; \\
\hphantom{-}1 &\mbox{---~максимальная\ положительная}\\
&\hphantom{\mbox{---~}}\mbox{взаимозависимость\ целей}\\
&\hphantom{\mbox{---~}}\mbox{пары\ агентов}\  \mathrm{\mathrm{ag}}_i\ \mbox{и}\ \mathrm{\mathrm{ag}}_j;
\end{cases}
$$
 $\mathrm{gdi}$~--- показатель 
взаимозависимости целей агентов, формирующий мат\-ри\-цу~$\mathbf{GD}$ 
в~случае одномерных целей, т.\,е.\ $\vert \mathrm{PR}^g\vert \hm= 1$ 
в~соответствии с~выражением
\begin{multline}
\mathrm{gd}_{i\,j} =\mathrm{gdi} \left( \mathrm{gl}_i^{\mathrm{ag}}, \mathrm{gl}_j^{\mathrm{ag}}\right) = 
\fr{\int\nolimits_{v_{\min}}^{v_{\max}} \mu_{i\cap j}(\mathrm{pr}) 
d(\mathrm{pr})}{\int\nolimits_{v_{\min}}^{v_{\max}} \mu_i(\mathrm{pr}) 
d(\mathrm{pr})}+{}\\
{}+\fr{\int\nolimits_{v_{\min}}^{v_{\max}} \mu_{i\cap 
j}(\mathrm{pr}) d(\mathrm{pr})}{\int\nolimits_{v_{\min}}^{v_{\max}} \mu_j 
(\mathrm{pr}) d(\mathrm{pr})} -1\,,
\label{e5-lis}
\end{multline}
где $v_{\min}$ и~$v_{\max}$~--- минимальное и~максимальное значение 
свойства~$\mathrm{pr}$; $\mu_{i\cap j} (\mathrm{pr})$~--- пересечение 
функций принадлежности нечетких целей агентов~$\mathrm{ag}_i$ 
и~$\mathrm{ag}_j$.

\section{Нечеткое управление конфликтами агентов 
в~гибридных интеллектуальных многоагентных системах}
  
  Для нечеткого управления конфликтами АФ использует метод на основе 
нечеткого вывода Мамдани со следующими лингвистическими переменными: 
<<конфликт>>, <<взаимозависимость целей>>, <<длитель\-ность>>, 
<<стадия>>. При формировании нечетких множеств используются 
колоколообразные
\begin{equation*}
\mathrm{bmf}(u,a,b,c)= \left( 1+\vert u-c\vert^{2b} \vert a\vert^{-2b}\right)^{-1}
%\label{e6-lis}
  \end{equation*}
  и~сигмоидальные функции
  \begin{equation*}
\mathrm{sigmf}(u,b,c)= \left( 1+e^{-b(u-c)}\right)^{-1}\,,
%\label{e7-lis}
  \end{equation*}
где $u$~--- элемент нечеткого множества; $a$, $b$ и~$c$~--- параметры функции 
принадлежности.
  
  Кроме того, используется детерминированная переменная <<номер 
итерации>> $\mathrm{it} \hm\in \mathbb{N}_0\hm= \{0\}\cup \mathbb{N}$, описывающая 
число повторений действий по анализу текущей ситуации, выполняемых АФ 
при управлении конфликтами.
  
  Лингвистическая переменная <<конфликт>> для оценки напряженности 
конфликта между агентами \mbox{ГиИМАС} задается выражением
  \begin{equation}
  \mathrm{cnfl} =\left\langle \beta_{\mathrm{cnfl}}, T_{\mathrm{cnfl}}, U_{\mathrm{cnfl}}, G_{\mathrm{cnfl}}, 
M_{\mathrm{cnfl}}\right\rangle,
  \label{e8-lis}
  \end{equation}
где $\beta_{\mathrm{cnfl}}$\;=\;<<{конфликт}>>~--- наименование 
линг\-ви\-сти\-че\-ской переменной; $T_{\mathrm{cnfl}}$\;=\;$\{$<<{нет}>>; 
<<{слабый}>>; <<{умеренный}>>; <<{острый}>>$\}$~---  
терм-мно\-жест\-во ее значений, названий нечеткой переменной;\linebreak 
$U_{\mathrm{cnfl}}\hm=[0;1]$~--- универсум нечетких переменных; $G_{\mathrm{cnfl}}\hm= 
\varnothing$~--- процедура образования из элементов множества~$T_{\mathrm{cnfl}}$ 
новых термов;\linebreak $M_{\mathrm{cnfl}}$\;=\;$\{ \mu_{\mathrm{нет}}, 
\mu_{\mathrm{слабый}}, \mu_{\mathrm{умеренный}}, 
\mu_{\mathrm{острый}}\}$~--- процедура, ставящая в~соответствие каж\-до\-му
терму\linebreak  множества~$T_{\mathrm{cnfl}}$ осмысленное содержание путем\linebreak формирования 
нечеткого множества: $\mu_{\mathrm{нет}}\hm= \mathrm{bmf}(u_{\mathrm{cnfl}}; 0{,}15;5;0)$, 
$\mu_{\mathrm{слабый}}\hm= \mathrm{bmf} ( u_{\mathrm{cnfl}}; 0{,}15; 5;\linebreak 0{,}3)$, $\mu_{\mathrm{умеренный}} 
\hm= \mathrm{bmf} ( u_{\mathrm{cnfl}}; 0{,}15; 5; 0{,}6)$, $\mu_{\mathrm{острый}}\} \hm = \mathrm{bmf} ( u_{\mathrm{cnfl}}; 
0{,}25;5;1)$. Параметры этих и~других функций принадлежности, 
рассмотренных в~работе, должны быть уточнены в~ходе тестирования сис\-темы.
  
  Лингвистическая переменная <<взаимозависимость целей>> описывается по 
аналогии с~формулой~(\ref{e8-lis}) следующим выражением:
  $$
 \mathrm{gdl}=\left\langle \beta_{\mathrm{gdl}}, T_{\mathrm{gdl}}, U_{\mathrm{gdl}}, G_{\mathrm{gdl}}, M_{\mathrm{gdl}}\right\rangle,
  $$
где $\beta_{\mathrm{gdl}}$\;=\;<<{взаи\-мо\-за\-ви\-си\-мость целей}>>; 
$T_{\mathrm{gdl}}$\;=\;$\{$<<{от\-ри\-ца\-тель\-ная}>>; <<{нет}>>; 
<<{по\-ло\-жи\-тель\-ная}>>$\}$; $U_{\mathrm{gdl}}=\linebreak$
$= [-1;1]$; $G_{\mathrm{gdl}}\hm= 
\varnothing$; $M_{\mathrm{gdl}}=\{ \mu_{\mathrm{от\-ри\-ца\-тель\-ная}},
\mu_{\mathrm{нет}},\linebreak 
 \mu_{\mathrm{по\-ло\-жи\-тель\-ная}}\}$, 
$\mu_{\mathrm{от\-ри\-ца\-тель\-ная}}\hm= \mathrm{sigmf} (u_{\mathrm{dcnfl}}; -10;\linebreak -0{,}3)$, $\mu_{\mathrm{нет}} 
\hm= \mathrm{bmf}(u_{\mathrm{dcnfl}}; 0{,}3; 3; 0)$, $\mu_{\mathrm{по\-ло\-жи\-тель\-ная}}\hm= \mathrm{sigmf} (u_{\mathrm{dcnfl}}; 10; 0{,}3)$.
  
  Лингвистическая переменная <<длительность>>, используемая для оценки 
продолжительности стадии управления конфликтом, представляется 
выражением
  $$
  \mathrm{durl} =\left\langle \beta_{\mathrm{durl}}, T_{\mathrm{durl}}, U_{\mathrm{durl}}, G_{\mathrm{durl}}, 
M_{\mathrm{durl}}\right\rangle,
  $$
где $\beta_{\mathrm{durl}}$\;=\;<<{длительность}>>; 
$T_{\mathrm{durl}}$\;=\;$\{$<<{ма\-лая}>>;\linebreak <<{большая}>>$\}$; 
$U_{\mathrm{durl}}\hm= \mathbb{N}_0$; $G_{\mathrm{durl}}\hm=\varnothing$; $M_{\mathrm{durl}}\hm=  
\{\mu_{\mathrm{малая}}, \mu_{\mathrm{большая}}\}$, 
$\mu_{\mathrm{малая}} \hm= \mathrm{sigmf}\,(u_{\mathrm{durl}}; -0{,}2;\linebreak 10\cdot n)$, 
$\mu_{\mathrm{большая}} \hm= \mathrm{sigmf}\, (u_{\mathrm{durl}};0{,}5; 10\cdot n)$.
  
  Лингвистическая переменная <<стадия>> описывается выражением
  $$
  \mathrm{stgl} =\left\langle \beta_{\mathrm{stgl}}, T_{\mathrm{stgl}}, U_{\mathrm{stgl}}, G_{\mathrm{stgl}}, M_{\mathrm{stgl}}\right\rangle,
  $$
где $\beta_{\mathrm{stgl}}$\;=\;<<{ста\-дия}>>;  
$T_{\mathrm{stgl}}$\;=\;$\{$<<{на\-ча\-ло}>>; <<{сти\-му\-ля\-ция}>>; 
<<{раз\-ре\-ше\-ние}>>; <<{ко\-нец}>>$\}$; $U_{\mathrm{stgl}}\hm= [0;1]$;\linebreak 
$G_{\mathrm{stgl}}\hm= \varnothing$; $M_{\mathrm{stgl}}\hm= \{ \mu_{\mathrm{начало}}, 
\mu_{\mathrm{стимуляция}}, \mu_{\mathrm{разрешение}},\linebreak 
\mu_{\mathrm{конец}}\}$, $\mu_{\mathrm{начало}}\hm= \mathrm{bmf}\,(u_{\mathrm{stgl}}; 6^{-1}; 5; 0)$, 
$\mu_{\mathrm{стимуляция}}\hm= \mathrm{bmf}\,(u_{\mathrm{stgl}}; 6^{-1}; 5; 3^{-1})$, 
$\mu_{\mathrm{разрешение}} \hm= \mathrm{bmf}\,(u_{\mathrm{stgl}}; 6^{-1}; 5; 2\cdot 3^{-1})$, 
$\mu_{\mathrm{конец}}\hm= \mathrm{bmf}\,(u_{\mathrm{stgl}}; 6^{-1};\linebreak 5; 1)$.
  
  Таким образом, функция <<нечеткое управление конфликтом>> 
$\mathrm{act}^{\mathrm{cnffm}}$ АФ может быть представлена следующей 
последовательностью шагов:
  \begin{enumerate}[(1)]
  \item установить начальные значения переменных $\mathrm{it}\hm=0$, $u_{\mathrm{stgl}}^{\mathrm{it}} 
\hm=0$ и~$u_{\mathrm{durl}}\hm=0$;
  \item используя выражение~(\ref{e5-lis}), вычислить среднее арифметическое 
показателей взаимозависимости целей агентов по формуле
  $$
  \mathrm{gd}^{\mathrm{himas}} =\sum\limits^n_{i=1} \sum\limits^n_{j=i+1} 2 \mathrm{gd}_{i\,j} 
(n-2)! (n!)^{-1}\,;
  $$
  \item ожидать поступления сообщений от АС или АПР;
  \item если получено сообщение от АПР о~завершении работы, закончить 
работу;
  \item если получено сообщение от АС, содержащее решение проблемы или ее 
части, перейти к~п.\,6, иначе сообщить отправителю об ошибке и~перейти 
к~п.\,12;
  \item увеличить значение счетчиков $\mathrm{it}\hm= \mathrm{it}\hm+1$, $u_{\mathrm{durl}}\hm= 
u_{\mathrm{durl}}\hm+1$;
  \item запустить функцию идентификации конфликтов $\mathrm{act}^{\mathrm{cnfi}}$, 
чтобы с~помощью классификатора\linebreak конфликтов $\mathrm{cnfcl}$ на основе решений, 
предложенных АС, сформировать мат\-ри\-цу конф\-лик\-тов~$\mathbf{CNF}$ между 
парами агентов~[10] и~вы\-чис\-лить для \mbox{ГиИМАС} общие показатели 
\mbox{напряженности} конф\-лик\-тов по каж\-до\-му типу:
  \begin{align*}
  \mathrm{cnf}_{\mathrm{cnfprb}}^{\mathrm{himas}} &= \sum\limits^n_{i=1} \sum\limits^n_{j=i+1} 
2 \mathrm{cnf}_{i\,j\,\mathrm{cnfprb}} (n-2)! (n!)^{-1}\,;\\
  \mathrm{cnf}^{\mathrm{himas}}_{\mathrm{cnfprc}} &= \sum\limits^n_{i=1} \sum\limits^n_{j=i+1} 
2\mathrm{cnf}_{i\,j\,\mathrm{cnfprc}} (n-2)! (n!)^{-1}\,;
  \end{align*}
    
  \item фаззифицировать четкие значения лингвистических переменных 
  $u_{\mathrm{cnfl}}^{\mathrm{cnfprb}} \hm= \mathrm{cnf}_{\mathrm{cnfprb}}^{\mathrm{himas}}$, $u_{\mathrm{cnfl}}^{\mathrm{cnfprc}} 
\hm= \mathrm{cnf}_{\mathrm{cnfprc}}^{\mathrm{himas}}$ и~$u_{\mathrm{gdl}}\hm= \mathrm{gd}^{\mathrm{himas}}$;
  \item установить стадию управления конфликтами $\mathrm{stgl}^{\mathrm{it}}$ в~результате 
нечеткого вывода Мамдани~\cite{16-lis} по следующим правилам:
  \begin{description}
\item[правило~1:] ЕСЛИ ($\mathrm{stgl}^{\mathrm{it}-1}$ ЕСТЬ <<{начало}>>\\
ИЛИ $\mathrm{stgl}^{\mathrm{it}-1}$ ЕСТЬ <<{стимуляция}>>) \\ 
И ($\mathrm{cnfl}_{\mathrm{cnfprc}}$ ЕСТЬ <<{нет}>> ИЛИ $\mathrm{cnfl}_{\mathrm{cnfprc}}$\\
ЕСТЬ <<{слабый}>>), \\ 
ТО $\mathrm{stgl}^{\mathrm{it}}$ ЕСТЬ <<{стимуляция}>>;
\item[правило~2:] ЕСЛИ ($\mathrm{stgl}^{\mathrm{it}-1}$ ЕСТЬ <<{начало}>>\\
ИЛИ $\mathrm{stgl}^{\mathrm{it}-1}$ ЕСТЬ <<{стимуляция}>>)\\  
И ($\mathrm{cnfl}_{\mathrm{cnfprb}}$ ЕСТЬ <<{нет}>> ИЛИ $\mathrm{cnfl}_{\mathrm{cnfprb}}$\\ 
ЕСТЬ <<{слабый}>>), \\ 
ТО $\mathrm{stgl}^{\mathrm{it}}$ ЕСТЬ <<{стимуляция}>>;
\item[правило~3:] ЕСЛИ ($\mathrm{stgl}^{\mathrm{it}-1}$ ЕСТЬ <<{начало}>>\\
ИЛИ $\mathrm{stgl}^{\mathrm{it}-1}$ ЕСТЬ <<{стимуляция}>>)\\  
И ($\mathrm{cnfl}_{\mathrm{cnfprc}}$ ЕСТЬ <<{умеренный}>>\\
ИЛИ $\mathrm{cnfl}_{\mathrm{cnfprc}}$ ЕСТЬ <<{острый}>>),\\  
ТО $\mathrm{stgl}^{\mathrm{it}}$ ЕСТЬ <<{разрешение}>>;
\item[правило~4:] ЕСЛИ ($\mathrm{stgl}^{\mathrm{it}-1}$ ЕСТЬ <<{начало}>>\\
ИЛИ  $\mathrm{stgl}^{\mathrm{it}-1}$ ЕСТЬ <<{стимуляция}>>) \\ 
И $\mathrm{cnfl}_{\mathrm{cnfprb}}$ ЕСТЬ <<{острый}>>\\
ТО $\mathrm{stgl}^{\mathrm{it}}$ ЕСТЬ  <<{разрешение}>>;
\item[правило~5:] ЕСЛИ ($\mathrm{stgl}^{\mathrm{it}-1}$ ЕСТЬ <<{начало}>>\\
ИЛИ $\mathrm{stgl}^{\mathrm{it}-1}$ ЕСТЬ <<{стимуляция}>>)\\  
И $\mathrm{gdl}$ ЕСТЬ <<{положительная}>>\\
И $\mathrm{cnfl}_{\mathrm{cnfprb}}$ ЕСТЬ <<t{умеренный}>>, \\ 
ТО $\mathrm{stgl}^{\mathrm{it}}$ ЕСТЬ <<{стимуляция}>>;
\item[правило~6:] ЕСЛИ ($\mathrm{stgl}^{\mathrm{it}-1}$ ЕСТЬ <<{начало}>>\\
ИЛИ $\mathrm{stgl}^{\mathrm{it}-1}$ ЕСТЬ <<{стимуляция}>>)\\  
И $\mathrm{gdl}$ ЕСТЬ <<{отрицательная}>>\\
И $\mathrm{cnfl}_{\mathrm{cnfprb}}$ ЕСТЬ <<{умеренный}>>,\\  
ТО $\mathrm{stgl}^{\mathrm{it}}$ ЕСТЬ <<{разрешение}>>;
\item[правило~7:] ЕСЛИ $\mathrm{stgl}^{\mathrm{it}-1}$ ЕСТЬ <<{стимуляция}>>\\
И $\mathrm{durl}$  ЕСТЬ <<{большая}>>, \\ 
ТО $\mathrm{stgl}^{\mathrm{it}}$ ЕСТЬ <<{разрешение}>>;
\item[правило~8:] ЕСЛИ $\mathrm{stgl}^{\mathrm{it}-1}$ ЕСТЬ <<{разрешение}>>\\
И $\mathrm{durl}$  ЕСТЬ <<{большая}>>,\\  
ТО $\mathrm{stgl}^{\mathrm{it}}$ ЕСТЬ <<{конец}>>;
\end{description}
  \item  если $\mathrm{stgl}^{\mathrm{it}-1}\not= \mathrm{stgl}^{\mathrm{it}}$, установить $u_{\mathrm{durl}}\hm=0$;
  \item  если $\mathrm{stgl}^{\mathrm{it}-1}$\;=\;``сти\-му\-ля\-ция'', выполнить функцию 
<<стимуляция конфликтов>> $\mathrm{\mathrm{act}}^{\mathrm{cnfs}}$, если же 
$\mathrm{stgl}^{\mathrm{it}}$\;=\;``раз\-ре\-ше\-ние'', выполнить функцию <<разрешение 
конфликтов>> $\mathrm{\mathrm{act}}^{\mathrm{cnfr}}$;
  \item если $\mathrm{stgl}^{\mathrm{it}}$\;=\;``{конец}'' или после выполнения функций 
<<стимуляция конфликтов>> $\mathrm{\mathrm{act}}^{\mathrm{cnfs}}$ или <<разрешение 
конфликтов>> $\mathrm{act}^{\mathrm{cnfr}}$ уста\-нов\-лен флаг завершения работы, то 
отправить АПР сообщение о~необходимости завершения работы 
\mbox{ГиИМАС} и~закончить работу, иначе перейти к~п.\,3.
  \end{enumerate}
  
  Как показывает анализ функции <<нечеткое управление конфликтом>> 
$\mathrm{act}^{\mathrm{cnffm}}$ АФ, порядок смены стадий не предопределен, 
а~зависит от ситуации решения проблемы в~\mbox{ГиИМАС}, в~част\-ности от 
длительности использования метода, интенсивности конфликтов 
и~соотношения целей АС. С~использованием данной функции, а~также 
действий по стимулированию $\mathrm{ACT}^{\mathrm{agcs}}$ и~разрешению 
$\mathrm{ACT}^{\mathrm{agcr}}$ противоречий АФ может повышать напряженность 
конфликтов в~\mbox{ГиИМАС} и~разнообразие рассматриваемых в~ходе 
решения проблемы альтернатив либо переходить к~этапу согласования 
позиций АС, если между ними возникает большое число конфликтов высокой 
интенсивности. Таким образом, благодаря динамической комбинации 
различных технологий искусственного интеллекта, реа\-ли\-зу\-емых агентами, 
\mbox{ГиИМАС} при решении очередной проб\-ле\-мы вырабатывает наиболее 
релевантный ей метод. Применение нечеткого вывода Мамдани при управ\-ле\-нии 
конфликтами в~\mbox{ГиИМАС} обеспечивает прозрачность базы знаний для 
интерпретации человеком, в~отличие от систем нечеткого вывода  
Та\-ка\-ги--Су\-ге\-но~[17] или нейронечетких гибридов: ANFIS~[18], 
GARIC~[19], SONFIN~[20], за счет того, что в~консеквенте правила 
присутствуют лингвистические переменные. Кроме того, сохраняется 
возможность более тонкой настойки базы знаний в~ходе тестирования  
сис\-те\-мы с~применением методов подстройки параметров функций 
принадлежности по алгоритму обратного распространения ошибки,  
рас\-смот\-рен\-ных в~[21].

\section{Заключение}

  Рассмотрены особенности возникновения и~управления конфликтами 
в~коллективах реальных специалистов и~их компьютерных моделей, 
решающих практические проб\-ле\-мы за круглым столом. Представлена модель 
\mbox{ГиИМАС} с~управ\-ле\-ни\-ем проб\-лем\-но-  
и~про\-цес\-сно-ори\-ен\-ти\-ро\-ван\-ны\-ми конфликтами на основе анализа их 
напряженности,\linebreak типа и~соотношения целей агентов. Предложен\linebreak метод 
управления конфликтами в~\mbox{ГиИМАС} с~использованием нечеткого 
вывода Мамдани, что обеспечивает прозрачность базы знаний, а~также\linebreak 
воз\-мож\-ность более тонкой ее настойки в~ходе тес-\linebreak тирования сис\-те\-мы. 
Управ\-ле\-ние конф\-лик\-та\-ми в~\mbox{ГиИМАС} заключается в~релевантном 
ситуа\-ции стимулировании или разрешении проб\-лем\-но-  
и~про\-цес\-сно-ори\-ен\-ти\-ро\-ван\-ных конфликтов. Стимулирование 
конструктивных \mbox{конфликтов} в~\mbox{ГиИМАС} позволяет расширить 
множество рас\-смат\-ри\-ва\-емых альтернатив, учесть различные точки зрения 
и~интересы специалистов, мо\-де\-ли\-ру\-емых агентами сис\-те\-мы. Последующее 
снижение ин\-тен\-сив\-ности и~разрешение конфликтов, возникших между 
агентами \mbox{ГиИМАС} на первых стадиях рас\-смот\-ре\-ния по\-став\-лен\-ной 
проблемы, поз\-во\-ля\-ет выработать согласованную позицию с~учетом различных 
точек зрения и~интересов специалистов, мо\-де\-ли\-ру\-емых агентами сис\-те\-мы, 
повышая таким образом релевантность сис\-те\-мы реальному коллективу 
специалистов.
  
{\small\frenchspacing
 {%\baselineskip=10.8pt
 %\addcontentsline{toc}{section}{References}
 \begin{thebibliography}{99}
  \bibitem{1-lis}
  \Au{Листопад С.\,В., Кириков~И.\,А.} Моделирование конфликтов агентов в~гиб\-рид\-ных 
интеллектуальных многоагентных сис\-те\-мах~// Сис\-те\-мы и~средства информатики, 2019. 
Т.~29. №\,3. С.~139--148. doi: 10.14357/08696527190312.
  \bibitem{2-lis}
  \Au{Maltarich M.\,A., Kukenberger~M., Reilly~G., Mathieu~J.} Conflict in teams: Modeling 
early and late conflict states and the interactive effects of conflict processes~// Group  
Organ. Manage., 2018. Vol.~43. Iss.~1. P.~6--37.

 \bibitem{7-lis} %3
  \Au{Jehn K.\,A., Northcraft~G.\,B., Neale~M.\,A.} Why differences make a difference: A~field 
study of diversity, conflict, and performance in workgroups~// Admin. Sci. Quart., 
1999. Vol.~44. P.~741--763.
  \bibitem{8-lis} %4
  \Au{Rahim M.\,A.} Toward a~theory of managing organizational conflict~// Int. J.~Confl. 
Manage., 2002. Vol.~13. Iss.~3. P.~206--235.

 \bibitem{6-lis} %5
  \Au{Емельянов С.\,М.} Практикум по конфликтологии.~--- СПб.: Питер, 2009. 384~с.

 
  \bibitem{4-lis} %6
  \Au{Greer L.\,L., Caruso~H.\,M., Jehn~K.\,A.} The bigger they are, the harder they fall: Linking 
team power, team conflict, and performance~// Organ. Behav. Hum. Dec., 2011. Vol.~116. P.~116--128.

 \bibitem{3-lis} %7
  \Au{De Wit F.\,R.\,C., Greer~L.\,L., Jehn~K.\,A.} The paradox of intragroup conflict:  
A~meta-analysis~// J.~Appl. Psychol., 2012. Vol.~97. P.~360--390.

  \bibitem{5-lis} %8
  \Au{Хохлов А.\,С.} Конфликтология: История. Теория. Практика.~--- 
Самара: СФ МГПУ, 2014. 312~с.
 
 
  \bibitem{9-lis}
  \Au{Сидоренков А.\,В.} Конфликт в~малой группе: понятие, функции, виды и~модель~//  
Се\-ве\-ро-Кав\-каз\-ский психологический вестник, 2008. Т.~6. №\,4. С.~22--28.
  \bibitem{10-lis}
  \Au{Листопад С.\,В., Кириков~И.\,А.} Метод идентификации конфликтов агентов 
 в~гиб\-рид\-ных интеллектуальных многоагентных системах~// Сис\-те\-мы и~средства 
информатики, 2020. Т.~30. №\,1. С.~56--65. doi: 10.14357/08696527200105.
  \bibitem{11-lis}
  \Au{Листопад С.\,В., Кириков~И.\,А.} Стимуляция конфликтов агентов  
в~гиб\-рид\-ных интеллектуальных многоагентных сис\-те\-мах~// Сис\-те\-мы и~средства 
информатики, 2021. Т.~31. №\,2. С.~47--58. doi: 10.14357/ 08696527210205.
  \bibitem{12-lis}
  \Au{Листопад С.\,В., Кириков~И.\,А.} Разрешение конфликтов 
 в~гиб\-рид\-ных интеллектуальных многоагентных сис\-те\-мах~// Информатика и~её 
применения, 2022. Т.~16. Вып.~1. С.~54--60. doi: 10.14357/19922264220108.
  \bibitem{13-lis}
  \Au{Kolesnikov A.\,V., Listopad~S.\,V.} Hybrid intelligent multiagent system of heterogeneous 
thinking for solving the problem of restoring the distribution power grid after failures~// Open 
Semantic Technologies for Intelligent Systems: Research Papers Collection.~--- Minsk: BGUIR, 
2019. P.~133--138.
 
  \bibitem{15-lis} %14
  \Au{Bana e~Costa C.} The use of multi-criteria decision analysis to support the search for less 
conflicting policy options in a multi-actor context: Case study~// J.~Multi-Criteria Decision 
Analysis, 2001. Vol.~10. Iss.~2. P.~111--125.

 \bibitem{14-lis} %15
  \Au{Fasth T., Larsson~A., Ekenberg~L., Danielson~M.} Measuring conflicts using cardinal 
ranking: An application to decision analytic conflict evaluations~// Advances Operations Research, 
2018. Vol.~2018. Art.~8290434. 14~p.

  \bibitem{16-lis}
  \Au{Леоненков А.\,В.} Нечеткое моделирование в~среде MATLAB и~fuzzyTECH.~--- СПб.: 
БХВ Петербург, 2005. 736~с.
  \bibitem{17-lis}
  \Au{Abonyi J., Nagy~L., Szeifert~F.} Adaptive fuzzy inference system and its application in 
modelling and model\linebreak\vspace*{-12pt}

\pagebreak

\noindent
 based control~// Chem. Eng. Res. Des., 1999. Vol.~77. 
Iss.~4. P.~281--290.
  \bibitem{18-lis}
  \Au{Jang J.-S.\,R.} ANFIS: Adaptive-network-based fuzzy inference systems~// IEEE 
Transactions on Systems, Man, and Cybernetics, 1993. Vol.~23. P.~665--685.
  \bibitem{19-lis}
  \Au{Berenji H.\,R., Khedkar~P.} Learning and tuning fuzzy logic controllers through 
reinforcements~// IEEE Trans. Neural Networks, 1992. Vol.~3. P.~724--740.

%\columnbreak 

  \bibitem{20-lis}
  \Au{Feng J.\,C., Teng~L.\,C.} An online self constructing neural fuzzy inference network and its 
applications~// IEEE T. Fuzzy Syst., 1998. Vol.~6. Iss.~1. P.~12--32.
  \bibitem{21-lis}
  \Au{Колесников А.\,В., Кириков~И.\,А., Листопад~С.\,В.} Гиб\-рид\-ные интеллектуальные 
сис\-те\-мы с~самоорганизацией: координация, согласованность, спор.~--- М.: ИПИ РАН, 2014. 
189~с.

\end{thebibliography}

 }
 }

\end{multicols}

\vspace*{-6pt}

\hfill{\footnotesize\textit{Поступила в~редакцию 10.01.23}}

\vspace*{8pt}

%\pagebreak

%\newpage

%\vspace*{-28pt}

\hrule

\vspace*{2pt}

\hrule

%\vspace*{-2pt}

\def\tit{FUZZY RULES BASED METHOD FOR~AGENT CONFLICT 
MANAGEMENT IN~HYBRID INTELLIGENT MULTIAGENT 
SYSTEMS}


\def\titkol{Fuzzy rules based method for~agent conflict 
management in~hybrid intelligent multiagent 
systems}


\def\aut{S.\,V.~Listopad and~I.\,A.~Kirikov}

\def\autkol{S.\,V.~Listopad and~I.\,A.~Kirikov}

\titel{\tit}{\aut}{\autkol}{\titkol}

\vspace*{-8pt}


\noindent
Federal Research Center ``Computer Science and Control'' of the Russian Academy of Sciences, 44-2~Vavilov
Str., Moscow 119333, Russian Federation


\def\leftfootline{\small{\textbf{\thepage}
\hfill INFORMATIKA I EE PRIMENENIYA~--- INFORMATICS AND
APPLICATIONS\ \ \ 2023\ \ \ volume~17\ \ \ issue\ 1}
}%
 \def\rightfootline{\small{INFORMATIKA I EE PRIMENENIYA~---
INFORMATICS AND APPLICATIONS\ \ \ 2023\ \ \ volume~17\ \ \ issue\ 1
\hfill \textbf{\thepage}}}

\vspace*{15pt} 

  
   
   \Abste{The paper continues research on computer simulation with hybrid intelligent multiagent systems 
of a~teamwork of specialists of various profiles who solve problems at a~round table. The agents of such 
systems are autonomous software entities that imitate the reasoning of real specialists. Modeling of 
heterogeneous knowledge, goals, and points of view of agents on the problem posed within single intelligent 
system causes their collision, the emergence of conflicts by analogy with how it happens in simulated teams. 
Not every conflict between agents is destructive and requires suppression: conflict management in a~hybrid 
intelligent multiagent system as well as in a~team involves the identification of a~decision-making 
situation, if necessary, stimulation and subsequent resolution of constructive forms of conflict as well as the 
prevention of its destructive forms. The paper proposes the method based on fuzzy rules to manage conflicts 
between agents in hybrid intelligent multiagent systems.}
   
   \KWE{conflict; hybrid intelligent multiagent system; team of specialists; conflict management}
   
\DOI{10.14357/19922264230109} 

%\vspace*{-16pt}

%\Ack
%\noindent

  

\vspace*{4pt}

  \begin{multicols}{2}

\renewcommand{\bibname}{\protect\rmfamily References}
%\renewcommand{\bibname}{\large\protect\rm References}

{\small\frenchspacing
 {%\baselineskip=10.8pt
 \addcontentsline{toc}{section}{References}
 \begin{thebibliography}{99} 
  \bibitem{1-lis-1}
   \Aue{Listopad, S.\,V., and I.\,A.~Kirikov.} 2019. Mo\-de\-li\-ro\-va\-nie\linebreak konfliktov agen\-tov v~gib\-rid\-nykh 
in\-tel\-lek\-tu\-al'\-nykh mno\-go\-agent\-nykh sis\-te\-makh [Modeling of agent conflicts in hybrid intelligent multiagent 
systems]. \textit{Sis\-te\-my i~Sredstva\linebreak Informatiki~--- Systems and Means of Informatics} 29(3):139--148. doi: 10.14357/08696527190312.
  \bibitem{2-lis-1}
   \Aue{Maltarich, M.\,A., M.~Kukenberger, G.~Reilly, and J.~Mathieu.} 2018. Conflict in teams: 
Modeling early and late conflict states and the interactive effects of conflict processes. \textit{Group  
Organ. Manage.} 43(1):6--37.
  
  
  

%\columnbreak

\bibitem{7-lis-1} %3
   \Aue{Jehn, K.\,A., G.\,B.~Northcraft, and M.\,A.~Neale.} 1999. Why differences make a difference: 
A~field study of diversity, conflict, and performance in workgroups. \textit{Admin. Sci. 
Quart.} 44:741--763.
  \bibitem{8-lis-1} %4
   \Aue{Rahim, M.\,A.} 2002. Toward a~theory of managing organizational conflict. \textit{Int.  
J.~Confl. Manage.} 13(3):206--235.

  \bibitem{6-lis-1} %5
   \Aue{Emel'yanov, S.\,M.} 2009. \textit{Praktikum po konfliktologii} [Tutorial at conflictology]. St.\ 
Petersburg: Piter. 384~p. 

\bibitem{4-lis-1} %6
   \Aue{Greer, L.\,L., H.\,M.~Caruso, and K.\,A.~Jehn.} 2011. The bigger they are, the harder they fall: 
Linking team power, team conflict, and performance. \textit{Organ. Behav. Hum. Dec.} 116:116--128.
  
  
  \bibitem{3-lis-1} %7
   \Aue{De Wit, F.\,R.\,C., L.\,L.~Greer, and K.\,A.~Jehn.} 2012. The paradox of intragroup conflict: 
A~meta-analysis. \textit{J.~Appl. Psychol.} 97:360--390.

\bibitem{5-lis-1} %8
   \Aue{Khokhlov, A.\,S.} 2014. \textit{Konfliktologiya: Istoriya. Teoriya. Praktika} [Conflictology: 
History. Theory. Practice]. Samara: SF MGPU. 312~p.

  \bibitem{9-lis-1}
   \Aue{Sidorenkov, A.\,V.} 2008. Konflikt v~ma\-loy grup\-pe: po\-nya\-tie, funk\-tsii, vi\-dy i~mo\-del' [Conflict in 
small group: Concept, functions, forms and model]. \textit{Severo-Kavkazskiy psikhologicheskiy vestnik} 
[North Caucasian Psychological~J.] 6(4):22--28.
  \bibitem{10-lis-1}
   \Aue{Listopad, S.\,V., and I.\,A.~Kirikov.} 2020. Metod iden\-ti\-fi\-ka\-tsii konf\-lik\-tov agen\-tov v~gib\-rid\-nykh 
in\-tel\-lek\-tu\-al'\-nykh mno\-go\-agent\-nykh sis\-te\-makh [Agent conflict identification
   method in hybrid intelligent multiagent systems]. \textit{Sis\-te\-my i~Sredstva Informatiki~--- Systems 
and Means of Informatics} 30(1):56--65. doi: 10.14357/08696527200105.
   
  \bibitem{11-lis-1}
   \Aue{Listopad, S.\,V., and I.\,A.~Kirikov.} 2021. Stimulyatsiya konfliktov agentov v~gibridnykh 
intellektual'nykh mnogoagentnykh sis\-te\-makh [Stimulation of agent conflicts in hybrid intelligent 
multiagent systems]. \textit{Sis\-te\-my i~Sredstva}\linebreak\vspace*{-12pt}

\pagebreak

\noindent
\textit{Informatiki~--- Systems and Means of Informatics} 
31(2):47--58.
   doi: 10.14357/08696527210205.
  \bibitem{12-lis-1}
  \Aue{Listopad, S.\,V., and I.\,A.~Kirikov.} 2022. Razreshenie konfliktov v~gibridnykh intellektual'nykh 
mnogoagentnykh sis\-te\-makh [Conflict resolution in hybrid intelligent multi-agent systems]. 
\textit{Informatika i~ee Primeneniya~--- Inform. Appl.} 16(1):54--60. doi: 10.14357/19922264220108.
   
  \bibitem{13-lis-1}
   \Aue{Kolesnikov, A.\,V., and S.\,V.~Listopad.} 2019. Hybrid intelligent multiagent system of 
heterogeneous thinking for solving the problem of restoring the distribution power grid after failures. 
\textit{Open Semantic Technologies for Intelligent Systems: Research Papers Collection}. Minsk: BGUIR. 
133--138.
  
  \bibitem{15-lis-1} %14
   \Aue{Bana e Costa, C.} 2001. The use of multi-criteria decision analysis to support the search for less 
conflicting policy options in a multi-actor context: Case study. \textit{J.~Multi-Criteria Decision Analysis} 
10(2):111--125.

\bibitem{14-lis-1} %15
   \Aue{Fasth, T., A.~Larsson, L.~Ekenberg, and M.~Danielson.} 2018. Measuring conflicts using 
cardinal ranking: an application to decision analytic conflict evaluations. \textit{Advances Operations 
Research} 2018:8290434. 14~p.

\columnbreak

  \bibitem{16-lis-1}
   \Aue{Leonenkov, A.\,V.} 2005. \textit{Nechetkoe modelirovanie v~srede MATLAB i~fuzzyTECH} 
[Fuzzy modeling in MATLAB and fuzzyTECH]. St.\ Petersburg: BHV-Peterburg. 736~p.
  \bibitem{17-lis-1}
   \Aue{Abonyi, J., L.~Nagy, and F.~Szeifert.} 1999. Adaptive fuzzy inference system and its application 
in modelling and model based control. \textit{Chem. Eng. Res. Des.} 77(4):281--290.
  \bibitem{18-lis-1}
   \Aue{Jang, J.-S.\,R.} 1993. ANFIS: Adaptive-network-based fuzzy inference systems. \textit{IEEE 
Transactions Systems, Man, and Cybernetics} 23:665--685.
  \bibitem{19-lis-1}
   \Aue{Berenji, H.\,R., and P.~Khedkar.} 1992. Learning and tuning fuzzy logic controllers through 
reinforcements. \textit{IEEE Trans. Neural Networks} 3:724--740.
  \bibitem{20-lis-1}
   \Aue{Feng, J.\,C., and L.\,C.~Teng.} 1998. An online self constructing neural fuzzy inference network 
and its applications. \textit{IEEE T. Fuzzy Syst.} 6(1):12--32.
  \bibitem{21-lis-1}
   \Aue{Kolesnikov, A.\,V., I. A.~Kirikov, and S.\,V.~Listopad.} 2014. \textit{Gibridnye intellektual'nye  
sis\-te\-my s~samoorganizatsiey: koordinatsiya, soglasovannost', spor} [Hybrid intelligent systems with  
self-organization: Coordination, consistency, and dispute]. Moscow: IPI RAN. 189~p.
\end{thebibliography}

 }
 }

\end{multicols}

\vspace*{-6pt}

\hfill{\footnotesize\textit{Received January 10, 2023}}   
   
   \Contr
   
   \noindent
   \textbf{Listopad Sergey V.} (b.\ 1984)~--- Candidate of Science (PhD) in technology, senior scientist, 
Kaliningrad Branch of the Federal Research Center ``Computer Science and Control'' of the Russian 
Academy of Sciences, 5~Gostinaya Str., Kaliningrad 236000, Russian Federation;  
\mbox{ser-list-post@yandex.ru}
   
   \vspace*{3pt}
   
   \noindent
   \textbf{Kirikov Igor A.} (b.\ 1955)~--- Candidate of  Sciences (PhD) in technology; director, 
Kaliningrad Branch of the Federal Research Center ``Computer Science and Control'' of the Russian 
Academy of Sciences, 5~Gostinaya Str., Kaliningrad 236000, Russian Federation; 
\mbox{baltbipiran@mail.ru}
   
 

   
\label{end\stat}

\renewcommand{\bibname}{\protect\rm Литература} 