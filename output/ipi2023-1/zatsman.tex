\def\stat{zatsman}

\def\tit{ДАННЫЕ, ИНФОРМАЦИЯ И~ЗНАНИЕ\\ В~НАУЧНОЙ ПАРАДИГМЕ~ИНФОРМАТИКИ$^*$}

\def\titkol{Данные, информация и~знание в~научной парадигме 
информатики}

\def\aut{И.\,М.~Зацман$^1$}

\def\autkol{И.\,М.~Зацман}

\titel{\tit}{\aut}{\autkol}{\titkol}

\index{Зацман И.\,М.}
\index{Zatsman I.\,M.}


{\renewcommand{\thefootnote}{\fnsymbol{footnote}} \footnotetext[1]
{Работа выполнялась с использованием инфраструктуры Центра коллективного
пользования <<Высокопроизводительные вы\-чис\-ле\-ния и~большие данные>> (ЦКП
<<Информатика>>) ФИЦ ИУ РАН (г.~Москва).}}


\renewcommand{\thefootnote}{\arabic{footnote}}
\footnotetext[1]{Федеральный исследовательский центр <<Информатика и~управление>> Российской академии наук, 
\mbox{izatsman@yandex.ru}}

\vspace*{-10pt}





  \Abst{Рассматриваются три базовых понятия информатики: данные, информация 
  и~знание. Предлагается вариант специфицирования этих понятий в~рамках построения системы 
терминов научной парадигмы информатики как фундаментальной науки. С одной стороны, 
понятия <<данные>>, <<информация>> и~<<знание>> широко используются в~научной 
литературе и~учебниках по информатике, в~частности при описании ее теоретических 
оснований. С~другой стороны, до сих пор отсутствует консенсус по их смысловому 
содержанию. Сложившаяся ситуация, скорее всего, обусловлена распространенной 
пресуппозицией (имплицитным предположением), что рассматриваемые понятия выражают 
некоторые объективные сущности предметной области информатики. В~статье 
предполагается, что они выражают интерсубъективные сущности, которые по своей природе 
возникают как предметы мысли в~результате договоренности, т.\,е.\ не являются объективно 
существующими. С~точки зрения треугольника Фреге  
({пред\-мет}--{по\-ня\-тие}--{сло\-во}, которое выражает понятие 
и~обозначает предмет) для объективных сущностей первичной вершиной треугольника 
служит {предмет}, в~результате изучения которого появляются {понятие} 
и~{слово}. Для интерсубъективных сущностей первичной вершиной служит 
{понятие}, варианты дефиниции которого необходимо обсудить в~интересах 
достижения консенсуса. Если его удается достичь, то именно в~процессе обсуждения 
появляются {предмет} мысли, {слово}, его обозначающее и~выражающее 
понятие, которые вместе и~образуют треугольник Фреге. Цель статьи~--- специфицировать 
базовые понятия информатики как выражающие интерсубъективные сущности 
и~являющиеся первичными вершинами треугольника Фреге, распределить эти сущности по 
средам ее предметной области, выделяя границы между ними, и~рассмотреть отношения 
между этими сущностями на выделенных границах.}
  
\KW{научная парадигма; информатика как фундаментальная наука; данные; информация; 
знание; интерсубъективные сущности информатики; треугольник Фреге}

 \DOI{10.14357/19922264230115} 
  
\vspace*{12pt}


\vskip 10pt plus 9pt minus 6pt

\thispagestyle{headings}

\begin{multicols}{2}

\label{st\stat}

\section{Введение}

%\vspace*{-6pt}


Томас Элиот строками поэмы <<The Rock>> [1] сформулировал следующие 
вопросы, используя слова <<информация>>, <<знание>> и~<<мудрость>>\footnote[2]{Строки поэмы даны в~переводе автора статьи. Поэма на английском языке 
включает следующие строки~[1]:\\
Where is the wisdom that we have lost in knowledge?\\
   Where is the knowledge that we have lost in information?}:\\
  `Где та мудрость, которую мы потеряли в~знании?\\
  Где то знание, которое мы потеряли в~информации?'
  
  В 1989 г.~Рассел Акофф добавил к~ним чет\-вер\-тое слово~--- <<данные>>, 
ввел понятие <<иерархия\footnote[3]{Иногда используют словосочетание <<пирамида 
DIKW>>, но в~статье Акоффа говорится именно об иерархии~[2].} DIKW>> (data, information, 
knowledge, wisdom~--- данные, информация, знания, мудрость) и~описал 
отношения между ними так: <<Мудрость находится на вершине 
иерархии$\ldots$ По нисходящей от мудрости идут понимание, знание, 
информация и,~в~самом низу, данные. Каждое из перечисленных понятий 
[кроме данных] содержит в~себе нижестоящие, например не может быть 
мудрости без понимания и~понимания без знания>>~[2].
  
  Отметим, что при использовании иерархии DIKW рассматривают чаще 
только четыре ее уровня (данные, информация, знания, мудрость), т.\,е.\ для 
\textit{понимания} отдельный уровень, как правило, не выделяется~[3]. Иногда 
ограничиваются рас\-смот\-ре\-ни\-ем первых трех понятий~--- данные, информация, 
знания~--- и~описанием отношений между ними~[4].
{\looseness=1

}
  
  Акофф приводит их толкование, что помогает понять смысл его фразы о том, 
что <<\textit{пе\-ре\-чис\-лен\-ные понятия содержат в~себе нижестоящие}>>. 
Данные определяются им как наборы символов, которые \textit{характеризуют 
свойства объектов и~событий, а~также их окружение}. Эти наборы 
формируются в~процессе наблюдения или зондирования. Информация 
определяется как \textit{результат анализа данных}, вы\-пол\-ня\-емый в~том чис\-ле 
и~с~по\-мощью компьютеров. Знание рассматривается им в~контексте 
управления организационными системами в~сфере экономики: это то, что 
делает возможным \textit{преобразование информации в~инструкции}, и~это 
делает возможным управ\-ле\-ние такими сис\-те\-ма\-ми. Данные Акофф сравнивает 
с~металлической рудой, которая ценится меньше по сравнению с~результатом 
ее преобразования (где металл выступает как метафора ин\-фор\-ма\-ции)~[2].
{\looseness=1

}
  
  Таким образом, когда Акофф говорит о~том, что вышестоящие понятия 
включают нижестоящие, то речь идет о~процессах их трансформации, а~не 
о~включенности их смыслового содержания, которая описывается 
родовидовыми или иными иерархическими отношениями~[5]. На эту 
особенность описания иерархии DIKW обратила внимание Дженнифер Роули 
в~обзоре работ, посвященных иерархии DIKW: <<Обычно [в иерархии] 
информация определяется в~терминах данных, знание в~терминах информации 
и~мудрость в~терминах знаний, но существует меньше консенсуса в~описании 
процессов, которые трансформируют сущности, расположенные ниже 
в~иерархии, в~те, которые находятся над ними, что приводит к~отсутствию 
чет\-кости определений этих терминов>>~[3].
  
  Иерархия DIKW как упрощенная модель отношений между данными, 
информацией и~знанием имеет как своих сторонников, так и~критиков~[6, 7]. 
Например, Давид Вайнбергер ее недостатки описал так: <<Знание~--- это не 
просто результат фильтрации [сущностей нижних уровней иерархии] или 
применения некоторых алгоритмов. Оно~--- результат гораздо более сложного 
процесса [его генерации], который является \textit{социальным}, 
\textit{целенаправленным}, \textit{контекстно} и~\textit{культурно} 
обусловленным$\ldots$ Наиболее важным в~этом отношении является то, что 
там, где решения сложные и~знание [необходимое для решения] получить 
трудно, \textit{оно не определяется информацией}, поскольку именно процесс 
познания в~первую очередь определяет, какая информация необходима и~как ее 
следует использовать$\ldots$ Представление о том, что знание (а~тем более 
мудрость) является результатом применения фильтров на каж\-дом уровне, 
рисует неверную картину$\ldots$ Знание~--- это результат \textit{креативного} 
процесса, и~оно может иметь \textit{лакуны}>>~\cite{6-zac} (курсив мой.~--- 
И.\,З.).
  
  Цель статьи~--- рассмотреть базовые понятия информатики как выражающие 
интерсубъективные сущности и~являющиеся первичными вершинами 
треугольника Фреге, распределить эти сущности по средам ее предметной 
области, выделяя границы между ними, и~рассмотреть отношения между этими 
сущностями на выделенных границах в~рамках предлагаемой научной 
парадигмы информатики~[8].
  
  Необходимость в~более детальной спецификации базовых понятий 
информатики (по сравнению\linebreak с~иерархией DIKW) обусловлена тем, что при 
описании ее научных на\-прав\-ле\-ний отличные по своей природе сущ\-ности, 
а~иногда име\-ющие \textit{взаимоисключающие свойства}, часто называют 
\mbox{одинаково}, например словом <<информация>>~[9, 10]. Таким образом, 
кардинальные понятийные различия сущностей предметной об\-ласти 
информатики час\-то не выражаются лексически, что существенно усложняет 
описание отношений между этими сущностями. В~статье различные по своей 
природе и~принципиальным свойствам сущ\-ности предлагается 
специфицировать терминологически за счет использования отличающихся 
лексических средств их описания.

%\vspace*{-12pt}
  
\section{Контекст построения научной~парадигмы}

%\vspace*{-3pt}

  Сложность построения и~сопоставления вариантов научной парадигмы 
информатики обуслов\-лена отсутствием конвенциональных границ ее 
предметной области и~консенсуса по смысловому\linebreak содержанию ее базовых 
понятий~[11--15]. Дискуссии о~природе данных, информации, знания ведутся 
уже десятки лет. Согласно Ю.\,А.~Шрейдеру, <<споры о~предмете 
[информатики]~--- это не столько споры об объективной истине, сколько 
стремление отстоять свой взгляд на предмет, сделать его фактом 
общественного сознания сообщества исследователей. Это не значит, что 
представление о~предмете произвольно~--- оно долж\-но выражать плодотворный 
и~эвристичный взгляд на реально осу\-щест\-в\-ля\-емую дея\-тель\-ность, воз\-мож\-но\-сти 
ее развития и~перспективы использования>>~[16]. Однако <<плодотворные 
взгляды>> могут различаться концептуально, и~в~\mbox{статье} излагается только один 
из возможных вариантов.
  
  Для построения предлагаемого варианта научной парадигмы информатики 
выбрано десять пуб\-ли\-ка\-ций как источников его  
зарождения~\cite{16-zac, 18-zac, 19-zac, 20-zac, 21-zac, 22-zac, 23-zac, 24-zac, 25-zac, 26-zac}, из которых взяты следующие ключевые по\-ло\-же\-ния:
  \begin{itemize}
\item  согласно Ю.\,А.~Шрейдеру, определение базовых понятий информатики~--- это результат 
договоренности в~сообществе исследователей~\cite{16-zac}, а~не итоги изучения объективно существующих 
предметов\footnote{См.\ в~\cite{17-zac} подход к~определению информации, основанный на интерпретации 
объективных сущностей.}, и~поэтому в~рамках предлагаемой\linebreak\vspace*{-12pt}
\end{itemize}

\pagebreak

\begin{itemize}
\item[\,]парадигмы они и~позиционируются как 
\textit{интерсубъективные сущности}\footnote[1]{С~точки зрения треугольника Фреге  
(\textit{предмет}--\textit{по\-ня\-тие}--\textit{сло\-во}) для объективных сущностей первичной вершиной 
треугольника служит \textit{предмет}, в~результате изучения которого появляются \textit{понятие} 
и~\textit{слово}. Для интерсубъективных сущностей первичной вершиной служит \textit{понятие}, варианты 
дефиниции которого необходимо обсудить в~интересах достижения консенсуса. Если его удается достичь, то 
именно в~процессе обсуждения появляются \textit{предмет} мысли, \textit{слово}, его обозначающее 
и~выражающее \textit{понятие}, которые вместе и~образуют треугольник Фреге в~результате генезиса 
интерсубъективной сущности.} (курсив мой.~---\linebreak  И.\,З.);
  \item  согласно Р.\,С.~Гиляревскому, <<если \textit{данные воспринимаются} и~\textit{интерпретируются человеком}, то они становятся для него 
\textit{информацией}>> (курсив мой.~--- И.\,З.)~\cite[с.~19]{18-zac};
\item в~отчете <<Глубинное изменение~--- технологические переломные 
моменты и~социальное воздействие>>~\cite{19-zac}\footnote[2]{В~подготовке 
материалов для этого отчета, который был разработан под эгидой Всемирного экономического форума (Давос, 
Швейцария), принимали участие около 800~экспертов и~руководителей отрасли информационных 
и~коммуникационных технологий.} предложен перечень новых информационных 
технологий (ИТ), которые определяют кардинальный характер преобразования 
общества и~экономики, получившего\linebreak название <<Четвертая промышленная 
революция>>~\cite{20-zac}, и~сказано, что прогнозируемый характер 
преобразования во многом будет обуслов\-лен, в~частности, теми ИТ, которые 
охватывают \textit{сущности, принадлежащие средам разной природы} 
(курсив мой.~--- И.\,З.);
  \item в~обзоре <<Informatics Education in Europe>>~\cite{21-zac}\footnote[3]{В~обзор включены результаты двухлетнего мониторинга состояния систем 
преподавания информатики в~регионах и~странах Европы, включая РФ, а также в~Израиле.}, 
опубликованном в~2017~г.\ и~предваряющем разработку Европейской 
стратегии компьютерного образования <<Informatics for All>>~\cite{22-zac}, 
дана следующая характеристика современной информатики: <<В~то время как 
естественные науки определяются применительно к~миру (world), в~котором мы 
живем, информатику как научную дисциплину определить сложнее; \textit{у 
нее нет эмпирических основ}, как у~естественных наук; это нечто 
\textit{большее, чем мышление формальными символами}, как в~математике; 
и~это далеко \textit{не просто компиляция инженерных принципов 
и~технологий}>> (курсив мой.~--- И.\,З.), что также обуслов\-ли\-ва\-ет 
интерсубъективность базовых понятий информатики;
  \item в~статье~\cite{23-zac} Нонака разделяет явное и~неявное знание и~дает 
схему процессов их преобразования, которую планируется использовать при 
определении одной из со\-став\-ля\-ющих парадигмы информатики, а~именно: ее 
базовых мето-\linebreak\vspace*{-12pt}
\end{itemize}

\columnbreak

\noindent
\begin{itemize}
\item[\,]
дов и~моделей (эту составляющую планируется описать в~отдельной статье);
  \item в~работах~\cite{24-zac, 25-zac} Деннинг и~Розенблюм предложили 
сгруппировать научные дисциплины\linebreak в~четырех отраслях знания и~включить 
исследование информационных трансформаций в~технических, живых 
и~социальных системах в~чет\-вер\-тую отрасль знания;
  \item в~докладе Кристена Нюгора~\cite{26-zac}\footnote[4]{Доклад был представлен 
на Всемирном конгрессе IFIP (International Federation of Information Processing~--- Международная 
федерация по обработке информации) в~Дублине в~1986~г.} дано следующее определение 
информатики: <<это наука, область [исследований] которой охватывает 
информационные процессы и~\textit{связанные с~ними феномены 
в~артефактах, обществе и~природе}>> (курсив мой.~--- И.\,З.).
  \end{itemize}
  
  В работе Кристена Нюгора дается ссылка на определение понятия 
<<феномен>> в~словаре Webster\footnote[5]{Феномен~--- любой факт, обстоятельство или 
событие, которые сенсорно воспринимаются и~которые могут быть научно описаны или 
оценены>>~\cite{27-zac}. Отметим, что толкование понятия <<феномен>>, существенно 
расширенное по объему его значения, позже было включено в~современный он\-лайн-сло\-варь 
Merriam-Webster (в~новом определении удалена сенсорная воспринимаемость для феноменов, 
поддающихся научному описанию)~\cite{28-zac}.}, а~потом на примерах он дает 
расширенное толкование этого понятия, которое концептуально отличается от 
определения по словарю Webster 1960~г.: <<Важными примерами 
феноменов являются: живые организмы, неодушевленные объекты$\ldots$ Мы 
также можем говорить о~\textit{когнитивных феноменах}, происходящих 
в~сознании людей, в~отличие от \textit{явных} [\textit{сенсорно 
воспринимаемых}] \textit{феноменов}, находящихся вне сознания>> (курсив 
 мой~--- И.\,З.)~\cite{26-zac}.
  
  Это расширенное толкование имплицитно вводит в~предметную область 
информатики по Нюгору разделение ее сущностей на категории разной 
природы: \textit{ментальной} (например, результаты процессов генерации 
научного знания, происходящих в~сознании исследователей) и~\textit{сенсорно 
воспринимаемой}, к~которой относятся, в~частности, знаковые формы 
представления результатов процессов генерации знания после его деления на\linebreak 
кон\-цепты.


  На основе определения информатики по Нюгору в~работе~\cite{29-zac} было 
дано описание основания для по\-стро\-ения верхнего уровня классификации 
сущностей ее предметной об\-ласти как феноменов разной природы 
и~формирования следующих~5~ее\linebreak\vspace*{-12pt}

\pagebreak

\noindent
  сред\footnote{Другое значение термина 
<<среда>> было определено К.\,К.~Колиным для описания структуры научных исследований, 
относящихся к~комплексу наук об информации~\cite{30-zac}, а~не для определения классов 
сущностей как феноменов разной природы, относящихся к~предметной об\-ласти информатики.}, 
включающих сущности одной и~той же природы:
  \begin{enumerate}[(1)]
\item \textit{ментальная среда}~--- это совокупность когнитивных 
феноменов, формируемых в~процессах познания, происходящих в~сознании 
людей (далее~--- концепты);
\item \textit{информационная среда}~--- это совокупность сенсорно 
воспринимаемых феноменов, находящихся вне сознания;
\item \textit{цифровая среда}~--- это совокупность компьютерных кодов;
\item \textit{нейросреда}~--- это электрические потенциалы и~магнитные 
поля, генерируемые мозгом, которые используются в~ИТ управления 
роботизированной рукой~\cite{31-zac} и~в~других ИТ, применяющих 
интерфейсы <<мозг--компью\-тер>>;
\item \textit{ДНК-среда}~--- это совокупность естественных цепочек РНК 
и~ДНК\footnote{Например, модели трансляции естественных ДНК, созданные 
микробиологами, используются при разработке методов записи и~хранения данных 
с~использованием синтезированных цепочек ДНК.}.
  \end{enumerate}
  
  В соответствии с~перечисленными средами верхний уровень классификации 
сущностей предметной области информатики включает как 
минимум~5~классов, каждый из которых содержит объекты одной среды: ментальной, 
информационной, цифровой, нейро- или ДНК-сре\-ды. При этом с~ростом 
разнообразия природы сущностей верхний уровень классификации может 
пополняться новыми классами, природа сущностей которых отличается от 
природы сред, ранее включенных в~этот уровень классификации~\cite{8-zac}.
  
  Это может произойти, например, в~том случае, когда при проектировании ИТ 
встретятся сущности, которые по своей природе не относятся ни к~одной из 
ранее уже выделенных сред~\cite{32-zac}. Таким образом, в~предлагаемом 
варианте парадигмы информатики верхний уровень классификации ее 
пред\-мет\-ной об\-ласти предлагается сделать открытым, что обусловлено 
разнообразием природы ее сущностей и~возможным включением в~будущем 
в~ее предметную об\-ласть сущностей ранее не рас\-смат\-ри\-вав\-шей\-ся природы 
(откры\-тость уров\-ня классификации).

\section{Составляющие научной парадигмы информатики}

  Согласно А.~Соломонику, научная парадигма <<зрелой>> науки состоит из 
следующих четырех составляющих, которые могут разрабатываться отдельно, 
но объединяются в~единую и~цель\-ную конструкцию~\cite[с.~23--24]{33-zac}:
  \begin{enumerate}[(1)]
\item философские основания;
  \item аксиоматика;
  \item классификация исследуемых объектов и~процессов;
  \item система терминов.
  \end{enumerate}
  
  Сам термин <<научная парадигма>> трактуется им в~соответствии с~тео\-ри\-ей 
Т.~Куна, которая описывает процесс смены научных парадигм~\cite{34-zac}. 
При этом А.~Соломоник отмечает тот факт, что в~книге Куна мы не находим 
ответа на вопрос: <<Из чего должна состоять парадигма любой <<зрелой>> 
науки?>>~\cite[с.~23]{33-zac}.
  
  Отметим, что в~предлагаемом варианте парадигмы информатики планируется 
добавить пятую составляющую: базовые методы и~модели. Благодаря этому 
появится возможность дать дефиниции ключевых понятий и~описать 
отношения между ними в~рамках системы терминов, тогда как процессы 
взаимного преобразования обозначаемых ими сущностей будут описаны 
в~рамках базовых моделей и~методов. Иными словами, описания понятий, 
терминов, которые их обозначают, и~процессов преобразования обозна\-ча\-емых 
ими сущностей будут распределены по разным составляющим па\-ра\-дигмы.
{\looseness=-1

}
  
  В этой статье спецификация значений трех интер\-субъ\-ек\-тив\-ных сущностей 
(данные, информация, знание), широко используемых, но не \mbox{име\-ющих} 
конвенциональных дефиниций в~информатике, ограничена только тремя 
средами ее предметной об\-ласти: ментальной, информационной и~циф\-ро\-вой. 
Иначе говоря, в~статье следующий уровень классификации сущностей 
предметной об\-ласти информатики специфицирует только сущности 
ментальной, информационной и~циф\-ро\-вой природы, а~сущности остальных 
сред предметной об\-ласти не рас\-смат\-ри\-ва\-ются.

  \begin{figure*}[b] %fig1
  \vspace*{1pt}
\begin{center}
   \mbox{%
\epsfxsize=163mm
\epsfbox{zac-1.eps}
}

\vspace*{6pt}

{\small Три среды предметной области информатики, восемь их сущностей и~отношения 
между ними
  }
  \end{center}
  \end{figure*}
  
  Важно отметить, что согласно предложенному определению 
информационной среды она содержит сенсорно воспринимаемые данные, 
например\linebreak кардиограмму, сформированную в~процессе ре\-гист\-ра\-ции 
электрических полей, образующихся при работе сердца. Информация из 
заключения кардиолога, созданная им как результат \mbox{содержательного} анализа 
данных кардиограммы, также является сенсорно воспринимаемой, знаковой 
и~принадлежит к~той же среде. Таким образом, информационная среда 
содержит как минимум \textit{знаковую информацию} и~\textit{сенсорно} 
\textit{воспринимаемые данные}. При их компьютерном кодировании получаем 
соответственно \textit{цифровую информацию} и~\textit{цифровые данные}, т.\,е.\
 две принципиально разные сущности циф\-ро\-вой природы. При этом 
в~предлагаемом делении предметной об\-ласти информатики цифровая среда 
включает также результаты кодирования кортежей вида $\langle$концепт; 
знаковая форма его представления$\rangle$. Совокупность компьютерных 
кодов таких кортежей предлагается назвать \textit{цифровым представлением 
знания}. Они включают и~коды дефиниций концеп-\linebreak тов, и~коды слов, которые их 
обозначают~\cite{35-zac}.\linebreak Использование кодов кортежей служит основой 
разрешения проблемы асимметрии в~цифровой среде при кодировании 
синонимов и~омонимов (например, коды кортежей $\langle$(заплетенные 
волосы); коса$\rangle$, $\langle$(\mbox{с.-х.}\ орудие, используемое при скашивании); 
коса$\rangle$ и~$\langle$(намывная полоса суши, причлененная к~берегу~[36]); 
коса$\rangle$ будут отличаться из-за разных кодов трех концептов, но коды 
слова <<коса>> будут одинаковыми в~кодах кортежей).
  
  Отметим, что процесс содержательного анализа данных кардиограммы 
и~подготовки заключения состоит из нескольких этапов, в~процессе описания 
которых дадим определения сущностям, вы\-де\-ля\-емым в~трех средах. Сначала 
кардиолог сенсорно воспринимает \textit{данные} кардиограммы, полученной 
в~процессе \textit{цифрового мониторинга} работы сердца и~последующего 
\textit{декодирования} его результатов. На следующем этапе появляются 
\textit{ментальные образы данных} в~сознании кардиолога как 
\textit{результат сенсорного восприятия данных}. Затем генерируется 
\textit{знание} кардиолога о~наличии или отсутствии нарушений в~работе 
сердца как \textit{результат креативного процесса понимания ментальных 
образов данных}. Потом следует этап подготовки заключения на некотором 
естественном языке, включающий генерацию \textit{концептов} и~как результат 
\textit{деление полученного знания} и~их выражение словами этого языка. Если 
готовится двуязычное заключение, например параллельно на русском 
и~английском языке, то деление знания на концепты выполняется 
 по-раз\-но\-му в~сис\-те\-мах этих двух языков. Другими словами, деление одного и~того же знания на концепты зависит от используемого естественного языка 
как вербальной знаковой сис\-те\-мы. Отметим, что виды знаковых сис\-тем, 
обусловливающих деление знания на концепты, относятся к~третьему по сче\-ту 
уровню классификации, который планируется рас\-смот\-реть в~отдельной статье.
  

  
  Подведем итоги построения второго уровня классификации объектов 
предметной об\-ласти информатики, на котором в~этой \mbox{статье} определены восемь 
сущностей (см.\ рисунок), а~так\-же описаны и~отношения между ними:
\begin{itemize}
  \item  ментальная среда содержит как минимум \textit{ментальные образы 
данных} и~\textit{знание}, деление которого на \textit{концепты} определяется 
используемым естественным языком (в~общем случае деление на концепты 
определяется используемой знаковой сис\-те\-мой того или иного вида);
  \item информационная среда содержит как минимум \textit{сенсорно 
воспринимаемые данные} и~\textit{знаковую информацию};
  \item цифровая среда содержит как минимум \textit{цифровые данные}, 
\textit{цифровую информацию и~коды кортежей} вида $\langle$концепт; 
знаковая форма его пред\-став\-ле\-ния$\rangle$.
  \end{itemize}
  
  Приведенный пример с~кардиограммой ил\-люст\-ри\-ру\-ет только отдельные 
виды отношений между восемью описанными сущностями. Обобщенный 
вариант отношений между ними приведен на рисунке, который включает 
и~част\-ный случай отношений между сущностями в~примере с~кардиограммой. 
Отметим, что кардиолог может использовать данные предыдущих кардиограмм 
и~ранее сделанные заключения, компьютерные коды которых хранятся 
в~информационной медицинской сис\-теме.
{\looseness=1

}

 Пример с~геофизическими данными и~генерацией на их основе \textit{невербальной знаковой информации} приведен 
в~\cite{37-zac}.
  
  Опишем кратко отношения между восемью сущностями трех сред (см.\ рисунок):
  \begin{itemize}
\item \textit{концепты знания} человека~--- это либо результат деления этого 
знания на составляющие с~использованием некоторой знаковой системы, 
либо результат восприятия и~семантической интерпретации данных, либо 
результат понимания знаковой информации (на рисунке процесс понимания 
не показан);
\item \textit{ментальные образы данных}~--- это результат сенсорного 
восприятия данных человеком;
\item \textit{знаковая информация}~--- это результат вы\-ра\-жения концептов 
знания с~использованием не\-которой знаковой сис\-те\-мы или результат 
де\-кодирования цифровой информации с~\mbox{использованием} кодовых таб\-лиц 
символов;
\item \textit{сенсорно воспринимаемые данные}~--- это результаты процессов 
наблюдения, мониторинга или зондирования, воспринимаемые органами 
чувств человека;
\item \textit{цифровая информация}~--- это результат компьютерного 
кодирования знаковой информации с~использованием кодовых таблиц 
символов;
\item \textit{цифровые данные}~--- это результаты компьютерного 
кодирования сенсорно воспринимаемых данных или циф\-ро\-во\-го мониторинга 
(зондирования), который их порождает;
\item \textit{цифровое представление знания}~--- это результат 
компьютерного кодирования кортежей (с~теоретической точки зрения их 
кодовые таб\-ли\-цы~--- это триединые по своей природе 
сущности\footnote{Использование кодовых таб\-лиц кортежей как триединых по своей 
природе сущностей, которые располагаются в~точке соприкосновения трех сред (на рисунке это 
ментальная, информационная и~цифровая среды), описано в~работах~\cite{31-zac, 35-zac}.}).
\end{itemize}

\section{Заключение}

  Верхний уровень классификации объектов информатики как одна из 
со\-став\-ля\-ющих ее парадигмы, рас\-смот\-рен\-ный в~работе~\cite{8-zac} на примере 
пяти ее сред, дает возможность увидеть спектр \mbox{тео\-ре\-ти\-че\-ски} воз\-мож\-ных 
интерфейсов между объектами разной природы и~при проектировании ИТ, 
и~в~процессе преподавания информатики. В~данной \mbox{статье} начато описание 
следующего (второго) уровня классификации на примере трех сред 
(ментальной, информационной и~циф\-ро\-вой). При этом второй уровень 
классификации предлагается сделать также открытым для расширения. Это 
даст воз\-мож\-ность ввести еще один вид данных, формируемых при работе 
искусственных нейронных сетей, которые предлагается назвать ИИ-дан\-ными.

  
  В статье рассмотрены два вида сущностей информационной среды и~по три 
вида сущностей ментальной среды (знание, концепты и~ментальные образы 
данных) и~циф\-ро\-вой среды (циф\-ро\-вые данные, цифровая информация и~коды 
кортежей). Дефиниции этих восьми сущностей предлагается включить не 
в~классификацию, а~в~другую тесно связанную с~ней со\-став\-ля\-ющую научной 
парадигмы\linebreak информатики~--- в~систему терминов. Их распределение по средам 
дает возможность показать принципиальное тео\-ре\-ти\-че\-ское различие 
в~средствах \mbox{кодирования} \textit{символов} на границе между информационной 
и~циф\-ро\-вой средами (например, с~по\-мощью традиционных кодовых таб\-лиц 
Unicode) и~концептов в~точке соприкосновения трех сред (например, 
с~использованием кодов значений слов в~тезаурусах~\cite{5-zac}).
  
  В заключение отметим, что необходимость перехода от схемы иерархии 
DIKW к~более детально специфицированным сущностям предметной об\-ласти 
информатики обуслов\-ле\-на, в~част\-ности, актуальным на\-прав\-ле\-ни\-ем 
в~информатике, получившем название визуальной  
аналитики~\cite{4-zac, 38-zac}.
  
  \bigskip
  
  Автор признателен А.\,А.~Гончарову, С.\,Н.~Гринченко и~С.\,К.~Дулину за 
обсуждение предлагаемого варианта научной парадигмы информатики и~их 
предложения по редактированию ее положений.
   
{\small\frenchspacing
 {\baselineskip=10.5pt
 %\addcontentsline{toc}{section}{References}
 \begin{thebibliography}{99}
\bibitem{1-zac}
\Au{Eliot T.\,S.} The rock.~--- London: Faber and Faber Ltd., 1934. 86~p.
\bibitem{2-zac}
\Au{Ackoff R.} From data to wisdom~// J.~Appl. Syst. Anal., 1989. Vol.~16. P.~3--9.
\bibitem{3-zac}
\Au{Rowley J.} The wisdom hierarchy: representations of the DIKW hierarchy~// J.~Inf. Sci., 2007. 
Vol.~33. No.\,2. P.~163--180.
\bibitem{4-zac}
\Au{Chen M., Ebert~D., Hagen~H., Laramee~R., Van Liere~R., Ma~\mbox{K.-L.}, Ribarsky~W., 
Scheuermann~G., Silver~D.} Data, information, and knowledge in visualization~// IEEE Comput. 
Graph., 2009. Vol.~29. No.\,1. P.~12--19.
\bibitem{5-zac}
\Au{Лукашевич Н.\,В.} Тезаурусы в~задачах информационного поиска.~--- М.: Изд-во 
Московского ун-та, 2011. 512~с.
\bibitem{6-zac}
\Au{Weinberger D.} The problem with the data--information--knowledge--wisdom hierarchy~// 
Harvard Bus. Rev., 2010. {\sf https://hbr.org/2010/02/data-is-to-info-as-info-is-not}.
\bibitem{7-zac}
\Au{Frick$\acute{\mbox{e}}$~M.\,H.} Data--Information--Knowledge--Wisdom (DIKW) pyramid, 
framework, continuum~// Encyclopedia of big data~/ Eds. L.~Schintler, C.~McNeely.~--- Cham: 
Springer, 2018. 4~p.  doi: 10.1007/978-3-319-32001-4\_\mbox{331-1}.
\bibitem{8-zac}
\Au{Зацман И.\,М.} О~научной парадигме информатики: верхний уровень классификации 
объектов ее предметной об\-ласти~// Информатика и~её применения, 2022. Т.~16. Вып.~4. 
С.~108--114.
\bibitem{9-zac}
\Au{Newman J.} Some observations on the semantics of ``Information''~// Inform. Syst. 
Front., 2001. Vol.~3. No.\,2. P.~155--167.
\bibitem{10-zac}
\Au{D$\acute{\mbox{\ptb{\!\!\i}}}$az Nafr$\acute{\mbox{\ptb{\!\!\i}}}$a~J.\,M.} What is information? 
A~multidimensional concern~// Triple~С, 2010. Vol.~8. No.\,1. P.~77--108.

\bibitem{14-zac} %11
\Au{Denning P.\,J.} Opening statement~// Ubiquity Symposium ``What is Computation?'' 
Proceedings, 2010. Vol.~10. {\sf https://ubiquity.acm.org/article.cfm?id=1880067}.

\bibitem{13-zac} %12
\Au{Conery J.\,S.} Computation is symbol manipulation~// Ubiquity Symposium ``What is 
Computation?'' Proceedings, 2010. Vol.~10. {\sf https://ubiquity.acm.org/article. cfm?id=1889839}.


\bibitem{12-zac} %13
\Au{Frailey D.\,J.} Computation is process~//  Ubiquity Symposium ``What is Computation?'' 
Proceedings, 2010. Vol.~10. {\sf https://ubiquity.acm.org/article.cfm?id=1891341}.


\bibitem{15-zac} %14
\Au{Bajcsy R.} Computation and information~// Ubiquity Symposium ``What is Computation?'' 
Proceedings, 2010. Vol.~10. {\sf https://ubiquity.acm.org/article. cfm?id=1899473}.

\bibitem{11-zac} %15
\Au{Rosenbloom P.} The computing sciences and STEM education~// Ubiquity Symposium ``The 
Science in Computer Science'' Proceedings, 2014. Vol.~14. {\sf 
http://\linebreak ubiquity.acm.org/article.cfm?id=2590530}.

\bibitem{16-zac} %16
\Au{Шрейдер Ю.\,А.} Информация и~знание~// Сис\-тем\-ная концепция информационных 
процессов.~--- М.: ВНИИСИ, 1988. С.~47--52.

\bibitem{26-zac} %17
\Au{Nygaard K.} Program development as a~social activity~// 10th 
World Computer Congress Proceedings~/ Ed.~\mbox{H.-J.}~Kugler.~--- North Holland: Elsevier Science 
Publishers B.\,V., IFIP, 1986. P.~189--198.

\bibitem{23-zac} %18
\Au{Nonaka I}. The knowledge-creating company~// Harvard Bus. Rev., 1991. Vol.~69. No.\,6. 
P.~96--104.

\bibitem{18-zac} %19
Информатика как наука об информации~/ Под ред. Р.\,С.~Гиляревского.~---  
М.: ФАИР-ПРЕСС, 2006. 592~с.

\bibitem{24-zac} %20
\Au{Denning~P., Rosenbloom~P.} Computing: The fourth great domain of science~// 
Commun. ACM, 2009. Vol.~52. No.\,9. P.~27--29.

\bibitem{25-zac} %21
\Au{Rosenbloom P.\,S.} On computing: The fourth great scientific domain.~--- Cambridge, MA, 
USA: MIT Press, 2013. 307~p.


\bibitem{19-zac} %22
Deep shift~--- technology tipping points and societal impact~//
World Economic Forum.~--- Geneva, Switzerland, %: World Economic Forum, 
2015. {\sf 
http://www3.weforum.org/docs/WEF\_ GAC15\_Technological\_Tipping\_Points\_report\_2015.pdf}.

\bibitem{20-zac} %23
\Au{Шваб К.} Четвертая промышленная революция~/ Пер. с~англ.~--- М.: Эксмо, 2018. 288~с. 
(\Au{Schwab~K.} The fourth industrial revolution.~--- Geneva, Switzerland: World Economic 
Forum, 2016. 172~p.)

\bibitem{21-zac} %24
Informatics Education in Europe: Are we all in the same boat?~--- New York, NY, USA: ACM, 2017. 
 Technical Report of the Committee 
on European Computing Education. 251~p.
\bibitem{22-zac} %25
\Au{Caspersen M.\,E., Gal-Ezer~J., McGettrick~A., Nardelli~E.} Informatics for all: The 
strategy.~--- New York, NY, USA: ACM, 2018. 16~p.

\bibitem{17-zac} %26
\Au{D$\acute{\mbox{\ptb{\!\!\i}}}$az Nafr$\acute{\mbox{\ptb{\!\!\i}}}$a J.\,M., Zimmermann~R.\,E.} 
Emergence and evolution of meaning: The general definition of information (GDI) revisiting 
program~--- Part~2: The regressive perspective: Bottom-up~// Information, 2013. Vol.~4.  
P.~240--261.


\bibitem{27-zac} %27
Webster's New World dictionary of the American language~/
Eds. D.\,B.~Guralnik, J.\,H.~Friend.~--- New York, NY, USA: The World Publishing Company, 
1960. 1760~p.
\bibitem{28-zac}
Definition of phenomenon (meaning 2c)~//
Merriam-Webster's dictionary.  {\sf  
https://www.merriam-webster. com/dictionary/phenomenon}.
\bibitem{29-zac}
\Au{Зацман И.\,М.} Теоретические основания компьютерного образования: среды 
предметной об\-ласти информатики как основание классификации ее объектов~// Сис\-те\-мы 
и~средства информатики, 2022. Т.~32. №\,4. С.~77--89.
\bibitem{30-zac}
\Au{Колин К.\,К.} О~структуре научных исследований по комплексной проб\-ле\-ме 
<<Информатика>>~// Социальная информатика.~--- М.: ВКШ при ЦК ВЛКСМ, 
1990. С.~19--33.
\bibitem{31-zac}
\Au{Зацман И.\,М.} Интерфейсы третьего порядка в~информатике~// Информатика и~её 
применения, 2019. Т.~13. Вып.~3. С.~82--89.
\bibitem{32-zac}
\Au{Зацман И.\,М.} Таблица интерфейсов информатики как  
ин\-фор\-ма\-ци\-он\-но-компью\-тер\-ной науки~// На\-уч\-но-тех\-ни\-че\-ская информация. 
Сер.~1: Организация и~методика информационной работы, 2014. №\,11. С.~1--15.
\bibitem{33-zac}
\Au{Соломоник А.} Парадигма семиотики.~--- Минск: МЕТ, 2006. 335~с.
\bibitem{34-zac}
\Au{Кун Т.} Структура научных революций~/ Пер. c~англ.~--- М.: Прогресс, 1977. 302~с. 
(\Au{Kuhn~T.} The structure of scientific revolutions.~--- Chicago, IL, USA: University of Chicago Press, 
1962. 264~p.)
\bibitem{35-zac}
\Au{Зацман И.\,М.} Кодирование концептов в~циф\-ро\-вой среде~// Информатика и~её 
применения, 2019. Т.~13. Вып.~4. С.~97--106.
\bibitem{36-zac}
Географический энциклопедический словарь. Понятия и~термины~/ Под ред.\ 
А.\,Ф.~Трёшникова.~--- М.: Сов. энциклопедия, 1988. 432~с.
\bibitem{37-zac}
\Au{Зацман И.\,М.} Концептуальный поиск и~качество информации.~--- М.: Наука, 2003. 
272~с.
\bibitem{38-zac}
\Au{Federico P., Wagner~M., Rind~A., Amor-Amoros~A., Miksch~S., Aigner~W.} The role of 
explicit knowledge: A~conceptual model of knowledge-assisted visual analytics~// IEEE 
Conference on Visual Analytics Science and Technology Proceedings.~--- New York, NY, USA: 
IEEE, 2017. P.~92--103.

\end{thebibliography}

 }
 }

\end{multicols}

\vspace*{-6pt}

\hfill{\small\textit{Поступила в~редакцию 14.01.23}}

\vspace*{8pt}

%\pagebreak

%\newpage

%\vspace*{-28pt}

\hrule

\vspace*{2pt}

\hrule

%\vspace*{-2pt}

\def\tit{ON THE SCIENTIFIC PARADIGM OF~INFORMATICS:\\ DATA, INFORMATION, 
AND~KNOWLEDGE}


\def\titkol{On the scientific paradigm of~informatics: Data, information, 
and~knowledge}


\def\aut{I.\,M.~Zatsman}

\def\autkol{I.\,M.~Zatsman}

\titel{\tit}{\aut}{\autkol}{\titkol}

\vspace*{-15pt}


\noindent
Federal Research Center ``Computer Science and Control'' of the Russian Academy of Sciences, 
44-2~Vavilov Str., Moscow 119333, Russian Federation

\def\leftfootline{\small{\textbf{\thepage}
\hfill INFORMATIKA I EE PRIMENENIYA~--- INFORMATICS AND
APPLICATIONS\ \ \ 2023\ \ \ volume~17\ \ \ issue\ 1}
}%
 \def\rightfootline{\small{INFORMATIKA I EE PRIMENENIYA~---
INFORMATICS AND APPLICATIONS\ \ \ 2023\ \ \ volume~17\ \ \ issue\ 1
\hfill \textbf{\thepage}}}

\vspace*{3pt} 
     




\Abste{Three basic notions of informatics~--- data, information, and knowledge~--- are considered. The 
variant of specification of these notions within the framework of constructing a system of terms of 
the scientific paradigm of informatics as a fundamental science is proposed. On the one hand, the 
notions of ``data,'' ``information,'' and ``knowledge'' are widely used in the scientific literature and 
textbooks on informatics, in particular, when describing its theoretical foundations. On the other 
hand, there is still no consensus on their semantic content. The current situation is most likely due 
to the widespread presupposition (implicit assumption) that the notions in question express some 
objective entities of the subject domain of informatics. The paper assumes that they express 
intersubjective entities that by their nature arise as objects of thought during an agreement process, 
that is, they are not objectively existing. From the point of view of the Frege's triangle (subject\,--\,notion (concept)\,--\,word 
that expres the notion and denote the subject), for objective entities, 
the primary vertex of the triangle is the subject as a~result of the study of which the notion and the 
word appear. For intersubjective entities, the primary vertex is the notion, the definitions of which 
must be discussed in the interests of reaching consensus. If it can be achieved, then it is during the 
process of discussion that the subject of thought appears, the word denoting it and expressing the 
notion which together form the Frege's triangle. The aim of the paper is to specify the basic notions 
of informatics as expressing intersubjective entities and being the primary vertices of the Frege's 
triangle, to distribute them among the media of the subject domain of informatics, highlighting the 
boundaries between the media, and to consider the relationship between the specified notions on 
these boundaries.}

\KWE{scientific paradigm; informatics as fundamental science; data; information; knowledge; 
intersubjective entities of informatics; Frege's triangle}



 \DOI{10.14357/19922264230115} 

\vspace*{-16pt}

\Ack

\vspace*{-4pt}


\noindent
The research was carried out using the infrastructure of the Shared Research Facilities ``High 
Performance Computing and Big Data'' (CKP ``Informatics'') of FRC CSC RAS (Moscow).


  

%\vspace*{4pt}

  \begin{multicols}{2}

\renewcommand{\bibname}{\protect\rmfamily References}
%\renewcommand{\bibname}{\large\protect\rm References}

{\small\frenchspacing
 {%\baselineskip=10.8pt
 \addcontentsline{toc}{section}{References}
 \begin{thebibliography}{99} 

\bibitem{1-zac-1}
\Aue{Eliot, T.\,S.} 1934. \textit{The rock}. London: Faber and Faber Ltd. 86~p.
\bibitem{2-zac-1}
\Aue{Ackoff, R.} 1989. From data to wisdom. \textit{J.~Appl. Syst.Anal.} 16(1):3--9.

\vspace*{-2pt}

\bibitem{3-zac-1}
\Aue{Rowley, J.} 2007. The wisdom hierarchy: Representations of the DIKW hierarchy. 
\textit{J.~Inf. Sci.} 33(2):163--180.
\bibitem{4-zac-1}
\Aue{Chen, M., D.~Ebert, H.~Hagen, R.~Laramee, R.~van Liere, \mbox{K.-L.}~Ma, W.~Ribarsky, 
G.~Scheuermann, and D.~Silver.} 2009. Data, information, and knowledge in visualization. 
\textit{IEEE Comput. Graph.} 29(1):12--19.
\bibitem{5-zac-1}
\Aue{Loukachevitch, N.\,V.} 2011. \textit{Tezaurusy v~zadachakh informatsionnogo poiska} 
[Thesauri in information retrieval tasks]. Moscow: Izd-vo Moskovskogo un-ta. 512~p.
\bibitem{6-zac-1}
\Aue{Weinberger, D.} 2010. The problem with the data--information--knowledge--wisdom hierarchy. 
\textit{Harvard Bus. Rev.} Available at: {\sf https://hbr.org/2010/02/data-is-to-info-as-info-is-not} 
(accessed January~24, 2023).
\bibitem{7-zac-1}
\Aue{Frick$\acute{\mbox{e}}$, M.\,H.} 2018. Data--Information--Knowledge--Wisdom (DIKW) 
pyramid, framework, continuum. \textit{Encyclopedia of big data.} Eds. L.~Schintler and 
C.~\mbox{McNeely}. Cham: Springer. 4~p.  doi: 10.1007/978-3-319-32001-4\_331-1.
\bibitem{8-zac-1}
\Aue{Zatsman, I.} 2022. O~nauchnoy paradigme informatiki: verkhniy uroven' klassifikatsii 
ob''ektov ee predmetnoy oblasti [On the scientific paradigm of informatics: The classification 
high level of its objects]. \textit{Informatika i~ee Primeneniya~--- Inform. Appl.} 16(4):108--114.
\bibitem{9-zac-1}
\Aue{Newman, J.} 2001. Some observations on the semantics of ``Information.'' \textit{Inform. 
Syst. Front.} 3(2):155--167.
\bibitem{10-zac-1}
\Aue{D$\acute{\!\mbox{\ptb{\i}}}$az Nafr$\acute{\!\mbox{\ptb{\i}}}$a, J.\,M.} 2010. What is information? 
A~multidimensional concern. \textit{Triple~С} 8(1):77--108.

\bibitem{14-zac-1} %11
\Aue{Denning, P.\,J.} 2010. Opening statement. \textit{Ubiquity Symposium ``The Science in 
Computer Science'' Proceedings}. 10. Available at: {\sf  
https://ubiquity.acm.org/article. cfm?id=1880067} (accessed January~24, 2023).

\bibitem{13-zac-1} %12
\Aue{Conery, J.\,S.} 2010. Computation is symbol manipulation. \textit{Ubiquity Symposium ``The 
Science in Computer Science'' Proceedings}. 10. Available at: 
{\sf https://ubiquity.acm.org/ article.cfm?id=1889839} (accessed January~24, 2023).



\bibitem{12-zac-1} %13
\Aue{Frailey, D.\,J.} 2010. Computation is process. \textit{Ubiquity Symposium ``The Science in 
Computer Science'' Proceedings}. 10. Available at: {\sf  
https://ubiquity.acm.org/article.cfm?id= 1891341} (accessed January~24, 2023).

\bibitem{15-zac-1} %14
\Aue{Bajcsy, R.} 2010. Computation and information. \textit{Ubiquity Symposium ``The Science in 
Computer Science'' Proceedings}. 10. Available at: {\sf  
https://ubiquity.acm.org/article. cfm?id=1899473} (accessed January~24, 2023).


\bibitem{11-zac-1} %15
\Aue{Rosenbloom, P.\,S.} 2014. The computing sciences and STEM education. \textit{Ubiquity 
Symposium ``The Science in Computer Science'' Proceedings}. 14. Available at:  
{\sf http:// ubiquity.acm.org/article.cfm?id=2590530} (accessed January~24, 2023).



\bibitem{16-zac-1}
\Aue{Shreyder, Yu.\,A.} Informatsiya i~znanie [Information and knowledge]. \textit{Sistemnaya 
kontseptsiya informatsionnykh protsessov} [System conception of information processes]. Moscow: 
VNIISI Publs. 47--52.

\bibitem{26-zac-1} %17
\Aue{Nygaard, K.} 1986. Program development as a social activity. \textit{10th World Computer 
Congress Proceedings}. Ed. \mbox{H.-J.}~Kugler. North Holland: Elsevier Science Publs. B.\,V., IFIP. 
189--198.



\bibitem{23-zac-1} %18
\Aue{Nonaka, I.} 1991. The knowledge-creating company. \textit{Harvard Bus. Rev.}  
69(6):96--104.

\bibitem{18-zac-1} %19
Gilyarevskiy, R.\,S., ed. 2006. \textit{Informatika kak nauka ob informatsii} [Informatics as 
information science]. Moscow: FAIR-PRESS. 592~p.


\bibitem{24-zac-1} %20
\Aue{Denning, P., and P.~Rosenbloom.} 2009. Computing: The fourth great domain of science. 
\textit{Commun. ACM} 52(9):27--29.
\bibitem{25-zac-1} %21
\Aue{Rosenbloom, P.\,S.} 2013. \textit{On computing: The fourth great scientific domain}. 
Cambridge, MA: MIT Press. 307~p.

\bibitem{19-zac-1} %22
 Deep shift~--- technology tipping points and societal impact. 2015. \textit{World Economic Forum}. Geneva, 
Switzerland. Available at: {\sf  
http://www3.weforum.org/docs/WEF\_ GAC15\_Technological\_Tipping\_Points\_report\_2015.pdf} (accessed 
January~24, 2023).

\bibitem{20-zac-1} %23
\Aue{Schwab, K.} 2016. \textit{The fourth industrial revolution}. Geneva, Switzerland: World 
Economic Forum. 172~p.
\bibitem{21-zac-1} %24
The Committee on European Computing Education. 2017. \textit{Informatics education in Europe: Are we 
all in the same boat?} New York, NY: ACM. Technical Report. 251~p. Available at: {\sf 
https://dl.acm.org/citation. cfm?id=3106077} (accessed January~24, 2023).
\bibitem{22-zac-1} %25
\Aue{Caspersen, M.\,E., J.~Gal-Ezer, A.~McGettrick, and E.~Nardelli}. 2018. \textit{Informatics 
for all: The strategy}. New York, NY: ACM. 16~p.


\bibitem{17-zac-1} %26
\Aue{D$\acute{\!\mbox{\ptb{\i}}}$az Nafr$\acute{\!\mbox{\ptb{\i}}}$a, J.\,M., and R.\,E.~Zimmermann.} 
2013. Emergence and evolution of meaning: The general definition of information (GDI) revisiting 
program~--- Part~2: The regressive perspective: Bottom-up. \textit{Information} 4:240--261.

\bibitem{27-zac-1} %27
Guralnik, D.\,B., and J.\,H.~Friend, eds. 1960. \textit{Webster's New World dictionary of the 
American language}. New York, NY: The World Publishing Co. 1760~p.
\bibitem{28-zac-1}
 Definition of phenomenon (meaning 2c). \textit{Merriam-Webster's dictionary}. Available at: {\sf 
https://www.merriam-webster.com/dictionary/phenomenon} (accessed January~24, 2023).
\bibitem{29-zac-1}
\Aue{Zatsman, I.} 2022. Teo\-re\-ti\-che\-skie os\-no\-va\-niya komp'yuter\-no\-go ob\-ra\-zo\-va\-niya: sre\-dy 
pred\-met\-noy ob\-lasti in\-for\-ma\-ti\-ki kak osnovanie klassifikatsii ee ob''\-ek\-tov [Theoretical 
foundations of digital education: Subject domain media of informatics as the base of its objects' 
classification]. \textit{Sistemy i~Sredstva Informatiki~--- Systems and Means of Informatics} 
32(4):77--89.
\bibitem{30-zac-1}
\Aue{Kolin, K.\,K.} 1990. O~strukture nauchnykh issledovaniy po kompleksnoy probleme 
``Informatika'' [On the structure of scientific research on the complex problem of ``Informatics'']. 
\textit{Sotsial'naya informatika} [Social informatics]. Moscow: 
VKSh. 19--33.
\bibitem{31-zac-1}
\Aue{Zatsman, I.\,M.} 2019. Interfeysy tret'ego poryadka v informatike [Third-order interfaces in 
informatics]. \textit{Informatika i~ee Primeneniya~--- Inform. Appl.} 13(3):82--89.
\bibitem{32-zac-1}
\Aue{Zatsman, I.} 2014. A~table of interfaces of informatics as computer and information science. 
\textit{Scientific Technical Information Processing} 41(4):233--246.
\bibitem{33-zac-1}
\Aue{Solomonick, A.} 2006. \textit{Paradigma semiotiki} [The paradigm of semiotics]. Minsk: MET 
Publs. 335~p.

\pagebreak

\bibitem{34-zac-1}
\Aue{Kuhn, T.} 1962. \textit{The structure of scientific revolutions}. Chicago, IL: University of 
Chicago Press. 264~p.
\bibitem{35-zac-1}
\Aue{Zatsman, I.\,M.} 2019. Kodirovanie kontseptov v~tsifrovoy srede [Digital encoding of 
concepts]. \textit{Informatika i~ee Primeneniya~--- Inform. Appl.} 13(4):97--106.
\bibitem{36-zac-1}
Treshnikov, A.\,F., ed. 1988. \textit{Geograficheskiy en\-tsi\-klo\-pe\-di\-che\-skiy slovar'. Ponyatiya 
i~terminy} [Geographical encyclopedic dictionary. Concepts and terms]. Moscow: The Soviet 
Encyclopedia Publs. 432~p.
\bibitem{37-zac-1}
\Aue{Zatsman, I.} 2003. \textit{Kontseptual'nyy poisk i~kachestvo informatsii} [Conceptual 
retrieval and quality of information]. Moscow: Nauka. 272~p.
\bibitem{38-zac-1}
\Aue{Federico, P., M.~Wagner, A.~Rind, A.~Amor-Amoros, S.~Miksch, and W.~Aigner.} 2017. 
The role of explicit knowledge: A conceptual model of knowledge-assisted visual analytics. 
\textit{IEEE Conference on Visual Analytics Science and Technology Proceedings}.
New York, NY: IEEE. 92--103.
\end{thebibliography}

 }
 }

\end{multicols}

\vspace*{-6pt}

\hfill{\small\textit{Received January 14, 2023}}

\Contrl

\noindent
\textbf{Zatsman Igor M.} (b.\ 1952)~--- Doctor of Science in technology, head of department, 
Institute of Informatics Problems, Federal Research Center ``Computer Science and Control'' of the 
Russian Academy of Sciences, 44-2~Vavilov Str., Moscow 119333, Russian Federation; 
\mbox{izatsman@yandex.ru}

     

\label{end\stat}

\renewcommand{\bibname}{\protect\rm Литература} 