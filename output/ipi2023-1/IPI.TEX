\documentclass[10pt]{book}
\usepackage[utf8]{inputenc}

\usepackage{latexsym,amssymb,amsfonts,amsmath,amsxtra,dsfont,
indentfirst,shapepar,%fleqn,%
picinpar,shadow,floatflt,enumerate,multicol,colortbl,moreverb,cite,ipi}

\usepackage{rotating}
\usepackage{mathrsfs}
\usepackage[noend]{algorithmic}
\usepackage{ulem}
\usepackage{graphicx}
%\usepackage{algorithm2e}
\usepackage[linesnumbered,boxed,ruled]{algorithm2e}
%\usepackage{xypic}
\usepackage{oldgerm}
\usepackage{epic}
\usepackage{eepic}

\SetAlgorithmName{Algorithm}{алгоритм}{Список алгоритмов}

%из Дюковой

\newcommand{\algKeyword}[1]{{\bf #1}}
\newcommand{\Proc}[1]{\text{\tt #1}}
\def\CALL{\algKeyword{call}~}

\newenvironment{AlgProcedure}[1]
{
\small
\medskip
%    \hrule
\medskip
\algKeyword{PROCEDURE} #1
\begin{algorithmic}[1]}
{\end{algorithmic}
%    \hrule
\bigskip
}

\def\CALL{\algKeyword{call}~}

%конец для Дюковой

%\RequirePackage[ruled]{algorithm}


\input{epsf}

%\nofiles

%\includeonly{avtor}    %
%\includeonly{podgot-rus-site,podgot-eng-site}  
%\includeonly{podgot-rus,podgot-eng}  
%\includeonly{ipi-ind} 
%\includeonly{index-16i}
%\includeonly{toc-rus, toc-en}
%\includeonly{toc-rus}
%\includeonly{toc-en} 
%\includeonly{popravka}



%\includeonly{ushakov}              %pdf+авт+
%\includeonly{agasand}              %pdf письмо получил+авт+
%\includeonly{malashenko}           %pdf+авт+
%\includeonly{dulin}                %pdf+авт+
%\includeonly{strijov}              %pdf+авт+
%\includeonly{sinitsin}             %pdf+авт+
%\includeonly{arkhipov}             %pdf+авт+
%\includeonly{listopad}             %pdf+авт+
%\includeonly{leri}                 %pdf+авт+
%\includeonly{grusho}               %pdf+авт+
%\includeonly{bosov}                %pdf+авт+
%\includeonly{kabanov}              %pdf+авт+повт отпр
%\includeonly{zatsman}              %pdf+авт+
%\includeonly{agalarov}             %pdf+авт+
%\includeonly{adu}                  %pdf+авт+




%%%%%%%%%%%%%%%%%%%\includeonly{nekrolog-new}



%\includeonly{rekl}




\usepackage{acad}
%\usepackage{courier}
\usepackage{decor}
\usepackage{newton}
\usepackage{pragmatica}
\usepackage{zapfchan}
\usepackage{petrotex}
\usepackage{bm}                     % полужирные греческие буквы
\usepackage{upgreek}                % прямые греческие буквы \upalpha
\usepackage{eufrak}
\usepackage{verbatim}

\renewcommand{\bottomfraction}{0.99}
\renewcommand{\topfraction}{0.99}
\renewcommand{\textfraction}{0.01}

\setcounter{secnumdepth}{1} %здесь - 3 + chapter = 4

\arraycolsep=1.5pt

%\usepackage[pdftex]{graphicx}

%\usepackage{oz}

%NEW COMMANDS



\renewcommand*{\hm}[1]{#1\nobreak\discretionary{}%
            {\hbox{$\mathsurround=0pt #1$}}{}} %% Дублирует знаки операций
                               %при переносе в формуле (перед знаком, который
                               %надо продублировать ставится команда \hm)
                               
                               \newcommand{\PRB}{\begin{picture}(22.5,11)
      \spline(1,8)(4,10)(7,10.5)(10,11)(13,11)(16,10.5)(19,10)(22,8)
               \put(0,0){$P_{i-1}P_{t_{t-1}}$} \end{picture}}

\newcommand{\prb}{\begin{picture}(15.5,9)
      \spline(1,6)(3,8)(5,8.5)(7,9)(9,9)(11,8.5)(13,8)(15,6)
               \put(0,0){$PP_t$} \end{picture}}
               
                 \newcommand{\PRDN}{\begin{picture}(40,11)
      \spline(4,11.5)(7,10.5)(12,10)(16,9)(20,9)(24,10)(29,10.5)(32,11.5)
               \put(0,0){$P_{i-1}P_{t_{t-1}}$} \end{picture}}

\newcommand{\prdn}{\begin{picture}(18,11)
      \spline(3,10.5)(4,10)(6,9)(8,8.5)(10,8.5)(12,9)(14,10)(15,10.5)
               \put(0,0){$PP_t$} \end{picture}}




%\newcommand{\endproof}{\hfill$\Box$}
%\renewcommand{\r}{\mathbb{R}}
%\newcommand{\I}{{\rm I\hspace{-0.7mm}I}}
%\newcommand{\Ikl}{{\tt{1}}\hspace*{-1.44mm}\mathtt{1}}
\newcommand{\Ik}{\mbox{{\small \tt {1}}\hspace{-1.3mm}{\tt 1}}}
\newcommand{\argmin}{\mathop{\mathrm{arg}\,\mathrm{min}}}
\newcommand{\argmax}{\mathop{\mathrm{arg}\,\mathrm{max}}}
%\newcommand{\capr}{\mathop{\cap\,}}
%\newcommand{\cupr}{\mathop{\cup\,}}
%\def\argmin{\mathop{arg\,min}}

\def\vrp{\varphi}
\def\prt{\partial}
\def\mm{{\sf M}}
\def\modnop#1{\mathop{#1}\limits_{n}}
\def\eam{\mathbin{{\mathop{=}\limits^{\mathrm{def}}}}}
\def\dey#1#2{#1 (#2)}
\def\deyc#1#2{#1 \cdot  #2}
\def\ra#1{\;\mathop{\to}\limits^{#1}\;}
\def\raz#1{\;\mathop{\longrightarrow}\limits^{\!\!\!#1}\;}
\def\ral#1{\;\mathop{\longrightarrow}\limits^{#1}\;}

\newcommand{\Nor}{\mathcal{N}}
\newcommand{\T}{\mathbb{T}}
\newcommand{\Z}{\mathbb{Z}}



\newcommand{\il}[2]{\int\limits_{#1}^{#2}}%интеграл с пределами #1 и #2

\def\sm2{\mathop {\sum\limits^{n^\Theta}\sum\limits^{n^\Theta}}}
\def\sss{\sum\limits}
\def\tr{,\,\ldots\,,\,}
\def\rk{\right]}
\def\lk{\left[}
\def\rf{\right\}}
\def\lf{\left\{}
\def\lv{\,\left\vert}
\def\rv{\right\vert\,}
\def\iii{\int\limits}
\def\iin{\int\limits_{-\infty}^\infty}
\def\rrv{\right\vert}


\def\ee{{\cal E}}
\def\ww{{\cal W}}
\def\yy{{\cal Y}}
\def\vv{{\cal V}}

\newcommand{\R}{\mathbb R}
\newcommand{\E}{\mathbb E}
\newcommand{\N}{\mathbb N}

\renewcommand{\P}{\mathbb{P}}

\newcommand{\h}{{\bf H}}
\newcommand{\p}{{\sf P}}  % вероятность

\newcommand{\e}{{\sf E}}  % мат. ожидание
\newcommand{\D}{{\sf D}}  % дисперсия
\newcommand{\eps}{\varepsilon}
\newcommand{\vw}{{\mathbf w}}
\newcommand{\vp}{{\mathbf p}}
\newcommand{\vz}{{\mathbf z}}
\newcommand{\vx}{{\mathbf x}}
\newcommand{\vf}{{\mathbf f}}
\newcommand{\F}{{\mathcal F}}
\def\ap{{\mathrm{ЭР}}}
\newcommand{\ud}{\Delta_n} %uniform ditance
\newcommand{\nud}{\Delta_n(x)}
%\renewcommand{\Re}{\mathrm{Re}\,}

\newcommand{\abs}[1]{\left\vert#1\right\vert}

\newcommand{\norm}[1]{\left\Vert#1\right\Vert}
\def\da{(\Delta_t,A)}

\newcommand{\corr}{\mathrm{corr}}

\newcommand{\cov}{\mathrm{cov}}
\newcommand{\Expect}{\mathbb{E}}

\def\w{\omega}
\def\W{\Omega}

\def\inh{\int\limits_{nh}^{(n+1)h}}

\def\sumin{\sum_{i=1}^N}


\def\bxt{(Y,t)}
\def\xt{(y,t)}

\def\ovth{{\fr{\tau-nh}{h}}}
\def\ov{\overline}
\def\tm{\tilde m}
\def\tl{\tilde\lambda}
\def\tB{\widetilde B}
\def\tb{\tilde b}
\def\ld{\ldots}
\def\cd{\cdots}


\DeclareMathOperator{\sign}{sign}

%\newcommand{\gr}{{\geqslant}}


\newcommand{\g}{\mbox{\textit{g}}}

\renewcommand{\la}{\lambda}
\newcommand{\si}{\sigma}
\newcommand{\alp}{\alpha}

\newcommand{\pto}{\stackrel{P}{\longrightarrow}} % сходимость по веpоятности

\newcommand{\eqd}{\stackrel{\mathrm{d}}{=}} % равенство по pаспpеделению
\newcommand{\eqdelta}{\stackrel{\triangle}{=}} % равенство по pаспpеделению

\def\be#1{\begin{equation}\label{#1}}
\def\ee{\end{equation}}
\def\re#1{(\ref{#1})}

\def\bn{\begin{enumerate}}
\def\en{\end{enumerate}}
\def\bi{\begin{itemize}}
\def\ei{\end{itemize}}
%\def\i{\item}

%\newcommand{\kp}{\kappa}
%\def\Q{{\cal Q}} \def\H{{\cal H}}
%\newcommand{\bet}{\beta_{2+\delta}}


%\newtheorem{definition}{Определение}
%\renewcommand{\thedefinition}{\arabic{definition}.}
%END NEW COMMANDS

%\renewcommand{\baselinestretch}{1.2}

%\pagestyle{myheadings}

\setlength{\textwidth}{167mm}      % 122mm
\setlength{\textheight}{658pt}
%\setlength{\textheight}{635.6pt}
\setlength{\columnsep}{4.5mm}

\setcounter{secnumdepth}{4}

%\addtolength{\headheight}{2pt}
%\addtolength{\headsep}{-2mm}

\addtolength{\topmargin}{-7mm}  % for printing


%\hoffset=-30mm  % From Yap
\hoffset=-23mm  % From Acrobat

%\voffset=0mm % From Yap
\voffset=-5mm   % From Acrobat

%\addtolength{\evensidemargin}{-2.5mm} % for printing
%\addtolength{\oddsidemargin}{2.5mm}  % for printing

\addtolength{\evensidemargin}{-12mm} % for printing
\addtolength{\oddsidemargin}{8mm}  % for printing

%\renewcommand{\thefootnote}{\fnsymbol{footnote}}
%\renewcommand{\thefootnote}{\arabic{footnote}}
\renewcommand{\figurename}{\protect\bf Рис.}
\renewcommand{\tablename}{\protect\bf Таблица}

\newcommand{\Caption}[1]{\caption{\protect\small %\baselineskip=2.5ex
#1}}

\renewcommand{\thefigure}{\arabic{figure}}
\renewcommand{\thetable}{\arabic{table}}
\renewcommand{\theequation}{\arabic{equation}}
\renewcommand{\thesection}{\arabic{section}}

\renewcommand{\contentsname}{СОДЕРЖАНИЕ}
\newcommand{\fr}[2]{\displaystyle\frac{\displaystyle #1\mathstrut}{\displaystyle #2\mathstrut}}

%\renewcommand{\thefootnote}{\fnsymbol{footnote}}
%\newcommand{\g}{\mbox{\textit{g}}}

%\newcommand{\Caption}[1]{\caption{\protect\small\baselineskip=2ex #1}}
\newcounter{razdel}
\setcounter{razdel}{0}

\def\god{2023}
\def\tom{17}
\def\vyp{1}


\newcommand{\titel}[4]{%
\

\vspace*{5pt}

\ifodd\therazdel {\raggedright\noindent\Large\textrm\textbf
 \lineskip .75em
  \baselineskip=3.2ex #1 \par}
\vskip 1em {\noindent\large\textrm\textbf #2 \par}
\addcontentsline{toc}{subsection}{{\textrm\textbf #1}\protect\newline #2}
\def\rightheadline{\underline{\noindent\hbox to \textwidth{\hfill\small\textrm{#4}
%\hfill \large\bf\thepage
}}}
\def\leftheadline{\underline{\noindent\parbox{\textwidth}{
%\raggedleft\large\bf\thepage \hfill
\small\textit{#3}\hfill}}}
\def\leftfootline{\small{\textbf{\thepage}
\hfill ИНФОРМАТИКА И ЕЁ ПРИМЕНЕНИЯ\ \ \ том~\tom\ \ \ выпуск~\vyp\ \ \ \god}
}%
 \def\rightfootline{\small{ИНФОРМАТИКА И ЕЁ ПРИМЕНЕНИЯ\ \ \ том~\tom\ \ \ выпуск~\vyp\ \ \ \god
\hfill \textbf{\thepage}}}
\vskip 2em \setcounter{figure}{0}
\setcounter{table}{0}
\setcounter{equation}{0}
\setcounter{section}{0}
\setcounter{subsection}{0}
\setcounter{subsubsection}{0}
\setcounter{footnote}{0}
\setcounter{razdel}{0}
%\end{flushleft}
\else {
 \raggedright\noindent\Large\textrm\textbf
 \lineskip .75em
\baselineskip=3.2ex #1 \par} \vskip 1em
%\begin{flushleft}
{\noindent\large\textrm\textbf #2 \par}
\addcontentsline{toc}{subsection}{{\textrm\textbf #1}\protect\newline #2}
\def\rightheadline{\underline{\noindent\hbox to \textwidth{\hfill\small\textrm{#4}
%\hfill \large\bf\thepage
}}}
\def\leftheadline{\underline{\noindent\parbox{\textwidth}{%\raggedleft\large\bf\thepage \hfill
\small\textit{#3}\hfill}}}
\def\leftfootline{\small{\textbf{\thepage}
\hfill ИНФОРМАТИКА И ЕЁ ПРИМЕНЕНИЯ\ \ \ том~\tom\ \ \ выпуск~\vyp\ \ \ \god}
}%
 \def\rightfootline{\small{ИНФОРМАТИКА И ЕЁ ПРИМЕНЕНИЯ\ \ \ том~17\ \ \ выпуск~\vyp\ \ \ 2023
\hfill \textbf{\thepage}}} \vskip 2em \setcounter{figure}{0}
\setcounter{table}{0} \setcounter{equation}{0} \setcounter{section}{0}
\setcounter{subsection}{0} \setcounter{subsubsection}{0}
\setcounter{footnote}{0}
%\end{flushleft}
\fi}

\newcommand{\titelr}[2]{%
\

\vspace*{5pt}

\ifodd\therazdel {\raggedright\noindent%\Large\textrm\textbf
 \lineskip .75em
  \baselineskip=3.2ex #1 \par}
\vskip 1em {\noindent\normalsize\textrm\textbf #2 \par}
\else {
 \raggedright\noindent\Large\textrm\textbf
 \lineskip .75em
\baselineskip=3.2ex #1 \par} \vskip 1em
%\begin{flushleft}
{\noindent\large\textrm\textbf #2 \par
%\noindent\normalsize\textrm\textbf #2 \par
} \fi}

\newcommand{\titele}[5]{%
\

%\vspace*{5pt}

\ifodd\therazdel {\raggedright\noindent\large
\textrm\textbf
 \lineskip .75em
%  \baselineskip=3.2ex
#1 \par}
\vskip .5em {\noindent\large\textrm\textbf #2 \par}
\vskip .5em
 {\noindent\textrm #3 \par}
\addcontentsline{toc}{subsection}{{\textrm\textbf #1}\protect\newline #2}
\def\rightheadline{\underline{\noindent\hbox to \textwidth{\hfill\small\textrm{#4}
%\hfill \large\bf\thepage
}}}
\def\leftheadline{\underline{\noindent\parbox{\textwidth}{
%\raggedleft\large\bf\thepage \hfill
\small\textrm{#5}\hfill}}}
\def\leftfootline{\small{\textbf{\thepage}
\hfill ИНФОРМАТИКА И ЕЁ ПРИМЕНЕНИЯ\ \ \ том~17\ \ \ выпуск~1\ \ \ 2023}
}%
 \def\rightfootline{\small{ИНФОРМАТИКА И ЕЁ ПРИМЕНЕНИЯ\ \ \ том~17\ \ \ выпуск~1\ \ \ 2023
\hfill \textbf{\thepage}}} \vskip 1em \setcounter{figure}{0}
\setcounter{table}{0} \setcounter{equation}{0} \setcounter{section}{0}
\setcounter{subsection}{0} \setcounter{subsubsection}{0}
\setcounter{footnote}{0} \setcounter{razdel}{0}
%\end{flushleft}
\else {
 \raggedright\noindent\large
 \textrm\textbf
 \lineskip .75em
%\baselineskip=3.2ex
#1 \par} \vskip .5em
%\begin{flushleft}
{\noindent\large\textrm\textbf #2 \par} \vskip .5em
 {\noindent\textrm #3 \par}
\addcontentsline{toc}{subsection}{{\textrm\textbf #1}\protect\newline #2}
\def\rightheadline{\underline{\noindent\hbox to \textwidth{\hfill\small\textrm{#4}
%\hfill \large\bf\thepage
}}}
\def\leftheadline{\underline{\noindent\parbox{\textwidth}{%\raggedleft\large\bf\thepage \hfill
\small\textrm{#5}\hfill}}}
\def\leftfootline{\small{\textbf{\thepage}
\hfill ИНФОРМАТИКА И ЕЁ ПРИМЕНЕНИЯ\ \ \ том~17\ \ \ выпуск~1\ \ \ 2023}
}%
 \def\rightfootline{\small{ИНФОРМАТИКА И ЕЁ ПРИМЕНЕНИЯ\ \ \ том~17\ \ \ выпуск~1\ \ \ 2023
\hfill \textbf{\thepage}}} \vskip 1em \setcounter{figure}{0}
\setcounter{table}{0} \setcounter{equation}{0} \setcounter{section}{0}
\setcounter{subsection}{0} \setcounter{subsubsection}{0}
\setcounter{footnote}{0}
%\end{flushleft}
\fi}

\def\Abst#1{
\begin{center}\small\nwt
\parbox{150mm}{%\baselineskip=2.5ex
\textbf{Аннотация:}\ \
%\hspace*{\parindent}
#1}
\end{center}}
\def\Abste#1{
\begin{center}\small\nwt
\parbox{150mm}{%\baselineskip=2.5ex
\textbf{Abstract:}\ \
%\hspace*{\parindent}
#1}
\end{center}}

\def\DOI#1{
\begin{center}\small\nwt
\parbox{150mm}{%\baselineskip=2.5ex
\textbf{DOI:}\ \
%\hspace*{\parindent}
#1}
\end{center}}

\def\Abstend#1{
\begin{center}\small\nwt
\parbox{150mm}{%\baselineskip=2.5ex
%\hspace*{\parindent}
#1}
\end{center}}


\def\KW#1{
\begin{center}\small\nwt
\parbox{150mm}{%\baselineskip=2.5ex
\textbf{Ключевые слова:}\ \ #1}
\end{center}}

\def\KWE#1{
\begin{center}\small\nwt
\parbox{150mm}{%\baselineskip=2.5ex
\textbf{Keywords:}\ \ #1}
\end{center}}


\def\KWN#1{
%\begin{center}
%\small
%\parbox{150mm}\end{center}
}

\newcommand{\Avtors}[1]{%\smallskip
%\vspace*{.5pt}
\hangindent=23pt\noindent
%\nwt
{\bfseries#1}\
}


\renewcommand{\thesubsection}{\thesection.\arabic{subsection}\hspace*{-5pt}}
\renewcommand{\thesubsubsection}{\thesubsection\hspace*{5pt}.\arabic{subsubsection}\hspace*{-3pt}}

\newcommand{\Ack}{\section*{\protect\rmfamily Acknowledgments}\noindent}
\newcommand{\Contr}{\section*{\protect\rmfamily Contributors}\noindent}
\newcommand{\Contrl}{\section*{\protect\rmfamily Contributor}\noindent}

\makeindex


\begin{document}
\Rus

\nwt
%\ptb


%\renewcommand{\contentsname}{\protect\Large\bf Содержание}

\setcounter{tocdepth}{2}

%\tableofcontents

\renewcommand{\bibname}{\protect\rmfamily Литература}
  \def\Au#1{{\it #1}}
    \def\Aue#1{{#1}}

%\newcommand{\No}{№}
  \newcommand{\tg}{\,\mathrm{tg}\,}
    \newcommand{\ctg}{\,\mathrm{ctg}\,}
  \newcommand{\arctg}{\,\mathrm{arctg}\,}

\def\forallb{\mathop{\forall}}
\def\cupb{\mathop{\cup}}
\def\existsb{\mathop{\exists}}


\newpage
\addtocounter{razdel}{1}
%\def\razd{РЕГУЛИРУЕМЫЙ ЭЛЕКТРОПРИВОД ДЛЯ ЭЛЕКТРОЭНЕРГЕТИКИ}


\setcounter{page}{2}

%   { %\Large  
   { %\baselineskip=16.6pt
   
   \vspace*{-48pt}
   \begin{center}\LARGE
   \textit{Предисловие}
   \end{center}
   
   %\vspace*{2.5mm}
   
   \vspace*{25mm}
   
   \thispagestyle{empty}
   
   { %\small 

    
Вниманию читателей журнала <<Информатика и её применения>> предлагается 
очередной тематический выпуск <<Вероятностно-статистические методы и 
задачи информатики и информационных технологий>>. Предыдущие тематические 
выпуски журнала по данному направлению вышли в 2008~г.\ (т.~2, вып.~2), 
в 2009~г.\ (т.~3, вып.~3) и в 2010~г.\ (т.~4, вып.~2). 

Статьи, собранные в данном журнале, посвящены разработке новых вероятностно-статистических 
методов, ориентированных на применение к решению конкретных задач информатики и информационных 
технологий, а также~--- в ряде случаев~--- и других прикладных задач. Проблематика, охватываемая 
публикуемыми работами, развивается в рамках научного сотрудничества между Институтом проблем 
информатики Российской академии наук (ИПИ РАН) и Факультетом вычислительной математики и 
кибернетики Московского государственного университета им.\ М.\,В.~Ломоносова в ходе работ 
над совместными научными проектами (в том числе в рамках функционирования 
Научно-образовательного центра <<Вероятностно-статистические методы анализа рисков>>). 
Многие из авторов статей, включенных в данный номер журнала, являются активными участниками 
традиционного международного семинара по проблемам устойчивости стохастических моделей, 
руководимого В.\,М.~Золотаревым и В.\,Ю.~Королевым; регулярные сессии этого семинара 
проводятся под эгидой МГУ и ИПИ РАН (в 2011~г.\ указанный семинар проводится в октябре 
в Калининградской области РФ). 

Наряду с представителями ИПИ РАН и МГУ в число авторов данного выпуска журнала входят 
ученые из Научно-исследовательского института системных исследований РАН, Института 
проблем технологии микроэлектроники и особочистых материалов РАН, Института 
прикладных математических исследований Карельского НЦ РАН, Московского 
авиационного института, Вологодского государственного педагогического университета, 
НИИММ им.\ Н.\,Г.~Чеботарева, Казанского государственного университета, Дебреценского 
университета (Венгрия).

Несколько статей выпуска посвящено разработке и применению стохастических методов и 
информационных технологий для решения различных прикладных задач. В~работе В.\,Г.~Ушакова 
и О.\,В.~Шестакова рассмотрена задача определения вероятностных характеристик случайных 
функций по распределениям интегральных преобразований, возникающих в задачах эмиссионной 
томографии. В~статье Д.\,О.~Яковенко и М.\,А.~Целищева рассмотрены некоторые вопросы 
математической теории риска и предложен новый подход к диверсификации инвестиционных 
портфелей. Работа И.\,А.~Кудрявцевой и А.\,В.~Пантелеева посвящена построению и 
исследованию математической модели, описывающей динамику сильноионизованной плазмы. 
В~статье П.\,П.~Кольцова изучается качество работы ряда алгоритмов сегментации изображений. 
Статья А.\,Н.~Чупрунова и И.~Фазекаша посвящена вероятностному анализу числа без\-оши\-бочных 
блоков при помехоустойчивом кодировании; получены усиленные законы больших чисел для указанных 
величин.

В данном выпуске традиционно присутствует тематика, весьма активно разрабатываемая в течение 
многих лет специалистами ИПИ РАН и МГУ,~--- методы моделирования и управления для 
информационно-телекоммуникационных и вычислительных систем, в частности методы 
теории массового обслуживания. В~статье А.\,И.~Зейфмана с соавторами рассматриваются 
модели обслуживания, описываемые марковскими цепями с непрерывным временем в случае 
наличия катастроф. В~работе М.\,М.~Лери и И.\,А.~Чеплюковой рассматриваются случайные 
графы Интернет-типа, т.\,е.\ графы, степени вершин которых имеют степенные распределения; 
такие задачи находят применение при исследовании глобальных сетей передачи данных. 
Работа Р.\,В.~Разумчика посвящена исследованию систем массового обслуживания специального 
вида~--- с отрицательными заявками и хранением вытесненных заявок.

Ряд статей посвящен развитию перспективных теоретических 
вероятностно-статистических методов, которые находят широкое применение в различных 
задачах информатики и информационных технологий. В~работе В.\,Е.~Бенинга, А.\,К.~Горшенина 
и В.\,Ю.~Королева рассмотрена задача статистической проверки гипотез о числе компонент 
смеси вероятностных распределений, приводится конструкция асимптотически наиболее мощного 
критерия. Результаты этой работы найдут применение в ряде прикладных задач, использующих 
математическую модель смеси вероятностных распределений (в информатике, моделировании 
финансовых рынков, физике турбулентной плазмы и~т.\,д.). В~статье В.\,Ю.~Королева, 
И.\,Г.~Шевцовой и С.\,Я.~Шоргина строится новая, улучшенная оценка точности нормальной 
аппроксимации для пуассоновских случайных сумм; как известно, указанные случайные суммы 
широко используются в качестве моделей многих реальных объектов, в том числе в информатике, 
физике и других прикладных областях. Работа В.\,Г.~Ушакова и Н.\,Г.~Ушакова посвящена 
исследованию ядерной оценки плотности распределения; эти результаты могут применяться, 
в част\-ности, при анализе трафика в телекоммуникационных системах. Серьезные приложения 
в статистике могут получить результаты работы О.\,В.~Шестакова, в которой доказаны оценки 
скорости сходимости распределения выборочного абсолютного медианного отклонения к нормальному 
закону. 

\smallskip

Редакционная коллегия журнала выражает надежду, что данный тематический  выпуск 
будет интересен специалистам в области теории вероятностей и математической статистики 
и их применения к решению задач информатики и информационных технологий.
     
     %\vfill 
     \vspace*{20mm}
     \noindent
     Заместитель главного редактора журнала <<Информатика и её 
применения>>,\\
     директор ИПИ РАН, академик  \hfill
     \textit{И.\,А.~Соколов}\\
     
     \noindent
     Редактор-составитель тематического выпуска,\\
     профессор кафедры математической статистики факультета\\
      вычислительной математики и кибернетики МГУ им.\ М.\,В.~Ломоносова,\\
     ведущий научный сотрудник ИПИ РАН,\\ 
доктор физико-математических наук \hfill
      \textit{В.\,Ю.~Королев}
     
     } }
     }

\def\stat{sinits}

\def\tit{АНАЛИТИЧЕСКОЕ МОДЕЛИРОВАНИЕ  РАСПРЕДЕЛЕНИЙ С~ИНВАРИАНТНОЙ
МЕРОЙ В~СТОХАСТИЧЕСКИХ СИСТЕМАХ С~РАЗРЫВНЫМИ ХАРАКТЕРИСТИКАМИ$^*$}

\def\titkol{Аналитическое моделирование  распределений с~инвариантной
мерой в~стохастических системах} % с~разрывными характеристиками}

\def\autkol{И.\,Н.~Синицын}

\def\aut{И.\,Н.~Синицын$^1$}

\titel{\tit}{\aut}{\autkol}{\titkol}

{\renewcommand{\thefootnote}{\fnsymbol{footnote}}\footnotetext[1]
{Работа выполнена при финансовой поддержке РФФИ
(проект №\,13-07-00036) и программой <<Интеллектуальные информационные 
технологии, системный анализ и автоматизация>> (проект~1.7).}}

\renewcommand{\thefootnote}{\arabic{footnote}}
\footnotetext[1]{Институт проблем информатики Российской академии наук, sinitsin@dol.ru}



\Abst{На базе методов нормальной аппроксимации и статистической линеаризации разработаны 
точные и приближенные алгоритмы аналитического моделирования плотностей стохастических 
режимов с инвариантной мерой в гауссовых и негауссовых стохастических системах (СтС)
с разрывными 
характеристиками. Рассмотрены особенности моделирования в СтС с 
пуассоновскими шумами. На тестовых примерах показана достаточная для многих приложений 
точность алгоритмов.}

\KW{автокоррелированная помеха; аналитическое моделирование;
интегродифференциальные уравнения Пугачёва; метод нормальной аппроксимации;
метод статистической линеаризации; нелинейная гауссовская и негауссовская стохастическая система в смысле Ито;
пуассоновская стохастическая сис\-те\-ма; распределение с инвариантной мерой;
стохастический режим}

\vskip 14pt plus 9pt minus 6pt

      \thispagestyle{headings}

      \begin{multicols}{2}

            \label{st\stat}



\section{Введение}

Следуя [1, 2], будем рассматривать нестационарный стохастических режим $Z\hm=Z(t)$ 
в нелинейной дифференциальной СтС, понимаемой в смысле Ито:
    \begin{equation}
    \dot Z = a(Z,t) + b (Z,t) V\,, \enskip Z(t_0) = Z_0\,.
    \label{e1.1-sin}
    \end{equation}
Здесь $Z$~--- $k$-мер\-ный вектор состояния СтС, $Z\hm\in \Delta$ ($\Delta$~--- 
многообразие состояний); $a\hm=a(Z,t)$ и $b\hm= b(Z,t)$~--- детерминированные  
$(k\times 1)$- и $(k\times m)$-мер\-ные  функции  отмеченных аргументов; 
$V\hm=V(t)$~--- $m$-мер\-ный вектор негауссовских (в общем случае) белых шумов 
с нулевыми математическими ожиданиями и представляющий собой среднеквадратичную 
(с.к.)\ производную процесса с независимыми приращениями  $W\hm=W(t)$, 
$V\hm=\dot W$. Обозначим через $\chi\hm=\chi(\mu;t)$ логарифмическую производную 
одномерной характеристической функции $h_1\hm=h_1(\mu;t)$ процесса $W\hm=W(t)$, определяемую формулой
    \begin{equation}
    \chi(\mu;t)=\fr{\prt \ln h_1 (\mu;t)}{\prt t}=
    \fr{1}{h_1(\mu;t)}\,\fr{\prt h_1(\mu;t)}{\prt t}\,.
    \label{e1.2-sin}
    \end{equation}

Начальное состояние $Z_0$ будем считать случайной величиной (СВ), не зависящей 
от $W(t)$ для $t\hm>t_0$. Предположим, что стохастический режим $Z(t)$ является 
сильным решением~(\ref{e1.1-sin}), а функции $a,b$ и $\chi$ удовлетворяют известным 
условиям существования и единственности~[1, 2].

Пусть существуют одно- и $n$-мерные плот\-ности\linebreak $f_1\hm=f_1(z;t)$ и 
$f_n\hm= f_n(z_1\tr z_n; t_1 \tr t_n)$ и характеристические функции $g_1\hm=g_1(\la;t)$ и 
$g_n\hm=g_n(\la_1\tr \la_n; t_1\tr t_n)$ $(n\hm\ge 2)$, удовлетворяющие 
интегродифференциальным уравнениям\linebreak Пугачева~[1, 2]:
    \begin{multline}
    \fr{\prt f_1(z;t)}{\prt t}+\fr{\prt^{\mathrm{T}}}{\prt z}\lk a(z,t)f_1(z;t)\rk = 
\fr{1}{(2 \pi)^k} \times{}\\
{}\times \iin\iin \chi(b(\xi,t)^{\mathrm{T}}\la;t) e^{i\la^{\mathrm{T}}(\xi-z)} f_1(z;t) \,
    d\xi d\la\,;
    \label{e1.3-sin}
    \end{multline}
   \begin{equation}
    f_1(z;t_0)=f_0(z)\,;\label{e1.4-sin}
    \end{equation}
    
    \vspace*{-12pt}
    
    \begin{multline*}
\fr{\prt f_n(z_1\tr z_n;t_1\tr t_n)}{\prt t_n}+{}\\
{}+\fr{\prt^{\mathrm{T}}}{\prt z_n}\left[
a(z_n, t_n) f_n (z_1\tr z_n; t_1\tr t_n)\right]={}\\
{}= \fr{1}{(2\pi)^{kn}} \iin\iin \chi(b(\xi_n, t_n)^{\mathrm{T}} \la_n;t_n) \times{}\\
{}\times \exp\lf i \sss_{l=1}^n \la_l^{\mathrm{T}} (\xi_l-z_l)\rf \times{}\\
{}\times
f_n (\xi_1\tr \xi_n; t_1\tr t_n)\,d\xi_1\cdots d\xi_n d\la_1\cdots d\la_n\,;
%\label{e1.5-sin}
\end{multline*}

\vspace*{-12pt}

\begin{multline*}
f_n(z_1\tr z_{n-1},z_n;t_1\tr t_{n-1},t_{n})={}\\
{}= f_{n-1} (z_1\tr z_{n-1};t_1\tr t_{n-1})\delta (z_n - z_{n-1})\,;
%\label{e1.6-sin}
\end{multline*}
       
        
\noindent        
\begin{multline}
\fr{\prt g_1 (\la;t)}{\prt t} -{}\\
{}-\fr{1}{(2\pi)^k} \iin \iin i\la^{\mathrm{T}} a (z,t) 
e^{i(\la^{\mathrm{T}} -\mu^{\mathrm{T}})z} g_1 (\mu;t)\, d\mu dz={}\\
{}=\fr{1}{(2\pi)^k} \iin \iin \chi(b(z,t)^{\mathrm{T}} \la^{\mathrm{T}};t) 
e^{i(\la^{\mathrm{T}} -\mu^{\mathrm{T}})z} \times{}\\
{}\times
g_1 (\mu;t)\, d\mu dz\,;
\label{e1.7-sin}
\end{multline}
\begin{equation}
g_1(\la;t_0) = g_0(\la)\,\,; \label{e1.8-sin}
\end{equation}

\vspace*{-12pt}

\begin{multline*}
\fr{\prt g_n (\la_1\tr \la_n; t_1\tr t_n)}{\prt t_n} -{}\\
{}-
\fr{1}{(2\pi)^{kn}} \iin \cdots \iin i\la^{\mathrm{T}} a (z_n,t_n) \times{}\\
{}\times \exp \lk i \sss\limits_{k=1}^n (\la_k^{\mathrm{T}} - \mu_k^{\mathrm{T}}) z_k\rk \times{}\\
{}\times g_n 
(\mu_1\tr \mu_n; t_1\tr t_n)\, d\mu_1 \cdots d \mu_n dz_1\cdots dz_n={}\\
{}= \fr{1}{(2\pi)^{kn}} \iin\cdots \iin \chi (b(z_n;t)^{\mathrm{T}} \la_n;t_n)\times{}\\
{}\times\exp \lk i \sss_{k=1}^n (\la_k^{\mathrm{T}} - \mu_k^{\mathrm{T}}) z_k\rk \times{}\\
{}\times g_n 
(\mu_1\tr \mu_n; t_1\tr t_n) \,d\mu_1 \cdots d \mu_n dz_1\cdots dz_n;
%\label{e1.9-sin}
\end{multline*}

\vspace*{-12pt}

\noindent
\begin{multline*}
g_n (\la_1\tr \la_n; t_1\tr t_{n-1},t_{n-1})= {}\\
{}=
g_{n-1} (\la_1\tr \la_{n-2},\la_{n-1}+\la_n; t_1\tr t_{n-1})\,, %\label{e1.10-sin}
\end{multline*}
 $$
        t_1\le t_2 \le \cdots \le t_n,\enskip n=2,3,\ldots
        $$

При этом одно- и $n$-мер\-ные плотности и характеристические функции связаны 
между собой известными соотношениями:
\begin{equation*}
f_1(z;t) = \fr{1}{(2\pi)^{k}} \iin e^{-i\mu^{\mathrm{T}} z} g_1(\mu;t) d\mu\,; %\label{e1.11-sin}
    \end{equation*}
      \begin{equation*}
   g_1(\la;t) = \iin e^{i\la^{\mathrm{T}} z} f_1(z;t)\, dz\,; %\label{e1.12-sin}
   \end{equation*}
   
   \vspace*{-12pt}

\noindent
\begin{multline}
f_n( z_1\tr z_n; t_1\tr t_n) ={}\\
{}=
\fr{1}{(2\pi)^{kn}} 
\iin\cdots \iin \exp \lf - i \sss_{l=1}^n \la_l^{\mathrm{T}} z_l\rf \times{}\\
{}\times g_n (\la_1\tr \la_n; t_1\tr t_n)\, d\la_1\cdots d\la_n\,;\label{e1.13-sin}
\end{multline}


\vspace*{-12pt}

\noindent
\begin{multline*}
g_n (\la_1\tr \la_n; t_1\tr t_n) ={}\\
{}=\iin\cdots \iin \exp\lf i \sss_{l=1}^n \la_l^{\mathrm{T}} z_l\rf \times{}\\
{}\times f_n (z_1\tr z_n; t_1\tr t_n)\, dz_1\cdots dz_n\,. %\label{e1.14-sin}
\end{multline*}

Для нахождения одномерных плотностей $f_1(z,t) \hm= f_1^* (z)$ и характеристических функций 
$g_1(\la;t) \hm= g_1^* (\la)$ стохастических режимов в стационарных СтС~(\ref{e1.1-sin}) при
    \begin{equation}
    a(z,t) = a^*(z)\,;\ b(z,t)=b^*(z)\,;\ \chi(\mu;t)= \chi^*(\mu)
    \label{e1.15-sin}
    \end{equation}
следует в~(\ref{e1.3-sin}) и~(\ref{e1.7-sin}) положить 
$\prt f_1/\prt t \hm= 0$ и $\prt g_1/ \prt t \hm=0$. В~результате получим соответственно
\begin{multline*}
\fr{\prt^{\mathrm{T}}}{\prt z}\lk a^* (z) f_1^* (z)\rk = {}\\
{}=
\fr{1}{(2\pi)^k} \iin \iin \chi^* (b^*(\xi)^{\mathrm{T}} \la) e^{i\la^{\mathrm{T}}(\xi-z)} f_1^* (\xi)\, d\xi d\la\,;
%\label{e1.16-sin}
\end{multline*}

\vspace*{-12pt}

\noindent
\begin{multline*}
-\fr{1}{(2\pi)^k} \iin  \iin i\la^{\mathrm{T}} a^*(z) e^{i(\la^{\mathrm{T}}-\mu^{\mathrm{T}})z} g_1^*(\mu)\, d\mu dz={}\\
{}=\fr{1}{(2\pi)^k} \iin  \iin \chi^*(b^*(z)^{\mathrm{T}}\la) e^{i(\la^{\mathrm{T}}-\mu^{\mathrm{T}})z} g_1^*(\mu)\, d\mu dz.
%\label{e1.17-sin}
\end{multline*}
Поставим задачу разработки точных и приближенных  алгоритмов
аналитического моделирования распределений (плотностей и
характеристических функций) стохастических режимов  $Z\hm=Z(t)$ в
нелинейных гауссовских и негауссовских СтС~(\ref{e1.1-sin})  с разрывными
характеристиками $a\hm=a(z,t)$ и $b\hm=b(z,t)$, обладающих свойством
сохранения инвариантной меры, т.\,е.\ удовлетворяющих уравнениям~(\ref{e1.3-sin})
и~(\ref{e1.7-sin}) при $\chi\hm=0$.

Условия сохранения инвариантной меры можно представить в следующем развернутом виде:
\begin{equation}
\left.
\begin{array}{c}
\displaystyle\fr{\prt f_1 (z;t)}{\prt t} + A_a f_1 (z;t) =0\,;\\[9pt] 
\hspace*{-4.5mm}\displaystyle A_a f_1(z;t) = 
    \fr{\prt^{\mathrm{T}}}{\prt z} \lk a(z,t) f_1(z;t)\rk = \mathrm{div}\, \pi(z;t)\,;
    \end{array}
    \right\}
    \label{e1.18-sin}
    \end{equation}
\begin{equation}
\left.
\begin{array}{c}
A_a^* f_1^*(z) =0\,;\\[9pt]
\displaystyle A_a^* f_1^* (z) = \fr{\prt^{\mathrm{T}}}{ \prt z} \lk a^* 
(z) f_1^* (z)\rk =\mathrm{div}\, \pi^* (z)\,;
\end{array}
\right\}
\label{e1.19-sin}
\end{equation}
$$
\fr{\prt g_1 (\la;t)}{\prt t} - B_a g_1(\la;t) =0\,;
$$

\vspace*{-14pt}

\noindent
\begin{multline}
B_a g_1(\la;t) ={}\\[2pt]
{}=\fr{1}{(2\pi)^k} \iin\iin i\la^{\mathrm{T}} a(z,t) e^{i(\la^{\mathrm{T}}-\mu^{\mathrm{T}})z}
 g_1(\mu;t)\, d\mu dz={}\\[2pt]
{}= \iin i\la^{\mathrm{T}} a(z,t) e^{i\la^{\mathrm{T}}z} f_1(z;t)\, dz={}\\[2pt]
{}= \iin e^{i\la^{\mathrm{T}} z} i\la^{\mathrm{T}} \pi(z;t)\, dz\,;
\label{e1.20-sin}
\end{multline}

\vspace*{-9pt}

\noindent
\begin{equation}
\left.
\begin{array}{c}
\hspace*{-45mm}B_a^* g_1^* (\la)=0\,;\\[12pt]
\hspace*{-48mm}B_a^* g_1^* (\la) = {}\\[10pt]
\hspace*{-3mm}{}=\fr{1}{(2\pi)^k} \iin\! i\la^{\mathrm{T}} a^* (z) e^{i(\la^{\mathrm{T}} -\mu^{\mathrm{T}})z} g_1^* (\mu)\, d\mu dz={}\\[10pt]
{}=\displaystyle\iin\! i\la^{\mathrm{T}} a^*(z) e^{i\la^{\mathrm{T}}z} f_1^* (z)\, dz = {}\\[10pt]
\displaystyle{}=
\iin e^{i\la^{\mathrm{T}} z} i\la^{\mathrm{T}} \pi^* (z)\, dz\,.
\end{array}
\right\}
\label{e1.21-sin}
\end{equation}
Для гауссовских (нормальных) СтС с гладкими характери\-стиками точные и приближенные 
методы  и алгоритмы аналитического моделирования рассмотрены в~[1--15]. 

Особое внимание 
уделим приближенным методам, основанным на методах нор\-маль\-ной аппроксимации и статистической 
линеаризации. Подробно рассмотрим их применение к пуассоновским СтС.



\section{Точные методы и~алгоритмы аналитического моделирования распределений 
с~инвариантной мерой}

Пусть функция~$a$ в СтС~(\ref{e1.1-sin}) допускает пред\-став\-ле\-ние
\begin{equation}
a= a(z,t) = a_1(z,t) +a_2 (z,t) \label{e2.1-sin}
\end{equation}
такое, что функция  $f_1\hm=f_1(z;t)$ является плот\-ностью инвариантной меры 
невозмущенной шумами системы, описываемой векторным обыкновенным дифференциальным 
уравнением вида
   \begin{equation}
   \dot z = a_1 (z,t)\,,\label{e2.2-sin}
   \end{equation}
т.\,е.\ удовлетворяет условию~(\ref{e1.18-sin}):
\begin{equation}
\fr{\prt f_1 (z;t)}{\prt t}+ \fr{\prt^{\mathrm{T}}}{\prt z} \lk a_1 (z,t) f_1(z;t)\rk =0\,.
\label{e2.3-sin}
\end{equation}

Для гладких функций $a_1\hm=a_1(z,t)$ вопросы существования и основные свойства 
интегральных 
 инвариантов изучены в~\cite{16-sin, 17-sin}. При этом в~(\ref{e2.1-sin}) 
функция $a_2 \hm= a_2(z,t)$ определяется путем решения следующего интегродифференциального 
уравнения:
\begin{multline}
\fr{\prt^{\mathrm{T}}}{\prt z}\lk a_2 (z,t) f_1(z;t) \rk 
=
\fr{1}{(2\pi)^k}\times{}\\
{}\times \iin\iin \chi(b(\xi,t)^{\mathrm{T}} \la;t) 
e^{i\la^{\mathrm{T}}(\xi-z)} f_1(\xi;t)\, d\xi d\la\,.\label{e2.4-sin}
\end{multline}
В общем случае нахождение функций $a_1$ и~$a_2$ в~(\ref{e2.1-sin})~--- такая же
трудная задача, как решение уравнений~(\ref{e1.3-sin}) и~(\ref{e1.4-sin}).

Для стационарных СтС, когда выполнены условия~(\ref{e1.15-sin}), 
уравнения~(\ref{e2.1-sin})--(\ref{e2.4-sin}) имеют вид:
\begin{align}
a(z)&= a_1(z) + a_2(z)\,;\label{e2.5-sin}
\\
\dot z &= a_1(z)\,,\label{e2.6-sin}
\\
\fr{\prt^{\mathrm{T}}}{\prt z}\lk a_2^*(z) f_1^*(z)\rk &= {}\notag\\
&\hspace*{-28mm}{}=
\fr{1}{(2\pi)^k} \!\!\iin \iin\!\! \chi^* (b^*(\xi)^{\mathrm{T}} \la) 
e^{i\la^{\mathrm{T}}(\xi-z)} f_1^*(\xi)\, d\xi d\la\,.\!\!\!\label{e2.7-sin}
\end{align}
В этом случае можно выбирать невозмущенную сис\-те\-му~(\ref{e2.6-sin}) так, чтобы
она имела первые интегралы.

В терминах характеристических функций соотношения~(\ref{e2.3-sin}), (\ref{e2.4-sin})
и~(\ref{e2.7-sin}) могут быть записаны следующим образом:

\noindent
\begin{equation}
\fr{\prt g_1 (\la;t)}{\prt t} - B_{a_1} g_1(\la;t) =0\,;\label{e2.8-sin}
\end{equation}
\begin{equation*}
B_{a_1}^* g_1^*(\la) =0\,. %\label{e2.9-sin}
\end{equation*}
Для составляющих $a_2(z,t)$ и $a_2^*(z)$ имеют место уравнения
\begin{multline}
B_{a_2} g_1(\la;t) 
= \fr{1}{(2\pi)^k} \times{}\\
\hspace*{-2.5mm}{}\times\iin\iin \!\chi(b(z,t)^{\mathrm{T}} \la;t) 
e^{i(\la^{\mathrm{T}}-\mu^{\mathrm{T}})z} g_1(\mu;t) \,d\mu dz;\label{e2.10-sin}
\end{multline}

\vspace*{-16pt}

\noindent
\begin{multline}
B_{a_2}^* g_1^*(\la) 
= \fr{1}{(2\pi)^k} \times{}\\
{}\times\iin\iin 
\chi^*(b^*(z)^{\mathrm{T}} \la) e^{i(\la^{\mathrm{T}}-\mu^{\mathrm{T}})z} g_1^*(\mu)\, d\mu dz\,.
\label{e2.11-sin}
\end{multline}

Отсюда вытекают конструктивные точные алгоритмы аналитического
моделирования распределений с инвариантной мерой. В~их основе лежат
следующие теоремы.

%\pagebreak

\medskip

\noindent
\textbf{Теорема~2.1.} \textit{Функция $f_1\hm=f_1(z;t)$ будет решением}~(\ref{e1.3-sin})
\textit{и}~(\ref{e1.4-sin}) \textit{тогда и только тогда, когда $a\hm=a(z,t)$ допускает
представление}~(\ref{e2.1-sin}) \textit{такое, что $f_1\hm=f_1(z;t)$ является плотностью
инвариантной меры обыкновенного дифференциального уравнения}~(\ref{e2.2-sin}),
\textit{т.\,е.\ удовле\-тво\-ря\-ет условию}~(\ref{e2.3-sin}). \textit{При этом со\-став\-ля\-ющая $a_2$
определяется из решения интегродифференциального уравнения}~(\ref{e2.4-sin}).

\medskip

\noindent
\textbf{Теорема~2.2.} \textit{Функция $f_1^*\hm=f_1^*(z)$ будет решением}~(\ref{e1.3-sin}) 
\textit{тогда и только тогда, когда $a^*\hm=a^*(z)$ допускает
представление}~(\ref{e2.5-sin}) \textit{такое, что $f_1^*\hm=f_1^*(z)$ является плотностью
инвариантной меры}~(\ref{e2.6-sin}). \textit{При этом составляющая $a_2^{*}$
определяется из решения  уравнения}~(\ref{e2.7-sin}).

\medskip

\noindent
\textbf{Теорема~2.3.} \textit{Функция $g_1\hm=g_1(\la;t)$ будет ре\-ше\-нием}~(\ref{e1.7-sin}), 
(\ref{e1.8-sin}) \textit{тогда и только тогда, когда недиф\-фе\-ренцируемая функция
$a\hm=a(z,t)$  допускает пред\-став\-ление}~(\ref{e2.1-sin}) \textit{такое, что
$g_1\hm=g_1(\la;t)$ является ха\-рак\-теристической функцией инвариантной
меры \mbox{уравнения}}~(\ref{e2.2-sin}), \textit{т.\,е.\ удовлетворяет условию}~(\ref{e2.8-sin}). 
\textit{При этом составляющая $a_2$ определяется из уравнения}~(\ref{e2.10-sin}).

\medskip

\noindent
\textbf{Теорема 2.4.} \textit{Функция $g_1^*\hm=g_1^*(\la)$  будет решением}~(\ref{e1.13-sin}) 
\textit{тогда и только тогда, когда недифференцируемая функция $a^*\hm=a^*(z)$  
допускает представление}~(\ref{e2.5-sin}) \textit{такое, что $g_1^*$ является  
характеристической функцией инвариантной меры}~(\ref{e2.2-sin}). 
\textit{При этом $a_2^*$ определяется из решения}~(\ref{e2.11-sin}).

\smallskip

Теоремы~2.1--2.4 легко обобщаются на случай многомерных распределений с инвариантной мерой.

\section{Приближенные методы и~алгоритмы аналитического моделирования распределений 
с~инвариантной мерой, основанные на~нормальной аппроксимации и статистической линеаризации}

Пусть нелинейная СтС~(\ref{e1.1-sin}) допускает применение метода нормальной аппроксимации 
(МНА)~[1, 2]. Тогда одно- и двумерные нормальные плот\-ности $f_1^{\mathrm{МНА}}$,
 $f_2^{\mathrm{МНА}}$ и характеристические функции  $g_1^{\mathrm{МНА}}$,  
 $g_2^{\mathrm{МНА}}$, а также вектор математического ожидания $m_t = M^{\mathrm{МНА}} Z(t)$, 
 ковариационная мат\-ри\-ца $K_t \hm= M^{\mathrm{МНА}} Z^{0\mathrm{T}} Z^0 (t)$ 
 $(Z^0 (t) \hm= Z(t) \hm- m_t)$ и матрица ковариационных функций 
 $K(t_1, t_2) \hm= M^{\mathrm{МНА}} Z^{0\mathrm{T}} (t_1) Z^0 (t_2)$ $(t_1\hm< t_2)$ определяются 
 следующими уравнениями:
    \begin{multline}
    f_1^{\mathrm{МНА}} = f_1^{\mathrm{МНА}} (z;t, m_t, K_t) =
    \lk (2\pi)^k |K_t|\rk^{-1/2}\times{}\\
    {}\times \exp \lf -  \fr{1}{ 2} 
    \left(z^{\mathrm{T}} - m_t^{\mathrm{T}}\right) K_t^{-1}(z-m_t)\rf\,;\label{e3.1-sin}
    \end{multline}
    
    \vspace*{-12pt}
    
    \noindent
\begin{multline}
f_2^{\mathrm{МНА}} ={}\\
= f_2^{\mathrm{МНА}} (z_1, z_2;t_1, t_2, m_{t_1}, m_{t_2}, K_{t_1}, K_{t_2}, K(t_1, t_2))=\\
{}=\lk (2\pi)^k |\bar K_2|\rk^{-1/2}\times{}\\
\hspace*{-2mm}{}\times \exp \lf - 
([z_1^{\mathrm{T}} z_2^{\mathrm{T}}] - \bar m_2^{\mathrm{T}}) 
\bar K_2^{-1}([z_1^{\mathrm{T}} z_2^{\mathrm{T}}]^{\mathrm{T}}-\bar m_2)\rf;
\!\!\label{e3.2-sin}
\end{multline}
\begin{equation}
g_1^{\mathrm{МНА}} (\la;t)=
\exp\lf i\la^{\mathrm{T}} m- \fr{1}{2}\,\la^{\mathrm{T}} K_t \la\rf\,;\label{e3.3-sin}
\end{equation}

\vspace*{-12pt}

\noindent
\begin{multline}
g_2^{\mathrm{МНА}} (\la_1, \la_2; t_1,t_2) ={}\\
{}= \exp \lf i \bar \la^{\mathrm{T}} \bar m_2 - 
    \fr{1}{2} \,\bar \la^{\mathrm{T}} \bar K_2 \bar \la\rf\,;\label{e3.4-sin}
    \end{multline}
$$
    \bar \la =\lk \la_1^{\mathrm{T}} \la_2^{\mathrm{T}}\rk^{\mathrm{T}}\,;\enskip 
    \bar m_2 =\lk m_{t_1}^{\mathrm{T}} m_{t_2}^{\mathrm{T}}\rk^{\mathrm{T}}\,;
    $$
    $$
    \bar K_2 =\begin{bmatrix}
        K(t_1, t_1)&K(t_1, t_2)\\[3pt]
        K(t_2, t_1)& K(t_2, t_2)
        \end{bmatrix}\,;
        $$
  \begin{multline}
  \dot m_t = a_1 (t, m_t, K_t) ={}\\
  {}=\iin a(z,t) f_1^{\mathrm{МНА}} (z; t, m_t, K_t) \,dz\,;
  \label{e3.5-sin}
  \end{multline}

\vspace*{-12pt}

\noindent
\begin{multline}
\dot K_t = a_2(t, m_t, K_t) = a_{21} + a_{12}+a_{22}={}\\
{}=\left[ \iin a(z,t) (z^{\mathrm{T}}-m_t^{\mathrm{T}}) + (z-m_t) a^{\mathrm{T}} (z,t) +{}\right.\\
\left.{}+ \sigma (z,t)
\vphantom{\iin}\right] f_1^{\mathrm{МНА}} (z;t, m_t, K_t)\, dz\,;
\label{e3.6-sin}
\end{multline}

\vspace*{-12pt}

\noindent
\begin{multline}
\fr{\prt K(t_1, t_2)}{\prt t_2} ={}\\
{}= a_3 (t_1, t_2, m_{t_1},m_{t_2}, K_{t_1}, K_{t_2}, K(t_1,t_2))={}\\
{}=\lk (2\pi)^{2k} |\bar K_2|\rk^{-1/2}\times{}\\
{} \times\iin\iin (z_1-m_{t_1}) a(z_2, t_2)
\exp\left\{ - ([z_1^{\mathrm{T}} z_2^{\mathrm{T}}]-\bar m_2^{\mathrm{T}})\times{}\right.\\
\left.{}\times\bar K_2^{-1} 
([z_1^{\mathrm{T}} z_2^{\mathrm{T}}]-\bar m_2)\right\} dz_1 dz_2\,.
\label{e3.7-sin}
\end{multline}
Здесь введены следующие обозначения:
\begin{equation}
\left.
\begin{array}{c}
z_1=z_{t_1}\,;\enskip  z_2=z_{t_2}\,;\enskip \bar m_2 =\lk m_{t_1}^{\mathrm{T}} m_{t_2}^{\mathrm{T}}\rk^{\mathrm{T}}\,;\\[9pt]
\displaystyle \bar K_2 =\begin{bmatrix}
        K(t_1,t_1)&K(t_1, t_2)\\[3pt]
        K(t_2, t_1)& K(t_2, t_2)
        \end{bmatrix}\,,
        \end{array}
        \right\}
        \label{e3.8-sin}
        \end{equation}
\begin{equation}
\sigma(z,t) = b(z,t) \nu(t) b(z,t)^{\mathrm{T}}\,,\label{e3.9-sin}
\end{equation}
где $\nu=\nu(t)$~--- интенсивность негауссовского белого шума $V\hm=V(t)$.

Для стационарных СтС  при $\dot m^* \hm=0$, $\dot K^* \hm=0$, 
$K(t_1, t_2)\hm= k(\tau)$ $(\tau\hm=t_1-t_2)$  соотношения~(\ref{e3.5-sin})--(\ref{e3.9-sin}) 
принимают вид:
\begin{equation}
a_1^* (m^*, K^*) =0\,;\label{e3.10-sin}
\end{equation}
\begin{equation}
    a_2^*(m^*, K^*) =0\,;\label{e3.11-sin}
    \end{equation}
    \begin{equation}
    \fr{dk(\tau) }{d\tau} = a_{11}^{\mathrm{МНА}} (m^*, K^*) k(\tau)\,;\label{e3.12-sin}
    \end{equation}
$$
k(\tau) = k(-\tau^{\mathrm{T}})\,;\enskip k(0)=K\,.
$$
Из уравнения~(\ref{e3.12-sin}) следует, что алгоритм МНА будет устойчивым, если матрица 
$a_{11}^{\mathrm{МНА}} (m_t, K_t, t)$ будет асимптотически устойчива.

Для $m$ и $K$ уравнения метода статистической линеаризации (МСЛ) в 
нелинейных СтС  при аддитивных шумах, когда $b(z,t) \hm= b_0(t)$, $b^*(z)\hm=b_0^*$ 
получаются из~(\ref{e3.5-sin})--(\ref{e3.7-sin}) и (\ref{e3.10-sin})--(\ref{e3.12-sin}) 
как частный случай.

Условия наличия нормального распределения с инвариантной мерой~(\ref{e1.18-sin}) 
и~(\ref{e1.19-sin}), если заменить $a(z,t)$ статистически
линеаризованным выраже\-нием
\begin{equation*}
    a(Z,t)\approx a_{10}^{\mathrm{МНА}} (t, m_t, K_t) + a_{11}^{\mathrm{МНА}} (t, m_t, K_t) 
    (Z-m_t)\,, %\label{e3.13-sin}
    \end{equation*}
где
\begin{equation*}
a_{10}^{\mathrm{МНА}} =a_{10}^{\mathrm{МНА}} (t, m_t, K_t)\equiv a_1\,; %\label{e3.14-sin}
\end{equation*}
    
    
   
    \noindent
    \begin{multline*}
    a_{11}^{\mathrm{МНА}}=a_{11}^{\mathrm{МНА}} (t, m_t, K_t) = {}\\
    {}=\lk \iin a(z,t) (z^{\mathrm{T}}-m_t^{\mathrm{T}}) 
        f_1^{\mathrm{МНА}} (z; t , m_t, K_t)\, dz\rk\times{}\\
        {}\times K_t^{-1} 
=\left(\fr{\prt}{\prt m_t} a_1^{\mathrm{T}}\right)^{\mathrm{T}}\,, %\label{e3.15-sin}
\end{multline*}
приводят к следующим соотношениям:
        \begin{multline}
\fr{\prt f_1^{\mathrm{МНА}} (z; t, m_t, K_t)}{\prt t} +\fr{\prt^{\mathrm{T}}}{ \prt z} 
\left\{ \left[ a_{10}^{\mathrm{МНА}} (t, m_t, K_t) 
+{}\right.\right.\\
\left.{}+ a_{11}^{\mathrm{МНА}} (t, m_t, K_t) (z-m_t) \vphantom{a_{10}^{\mathrm{МНА}}}
\right]\times{}\\
\left.{}\times 
     f_1^{\mathrm{МНА}} ( z; t , m_t, K_t)\right\} =0\,;
     \label{e3.16-sin}
     \end{multline}
     
     
     \noindent
\begin{multline}
\hspace*{-9.81628pt}\fr{\prt^{\mathrm{T}}}{\prt z} \left\{ \left[ a_{10}^{*{\mathrm{МНА}}}(m^*, K^*) + 
 a_{11}^{*{\mathrm{МНА}}}(m^*, K^*) (z-m^*)\right] \times{}\right.\\
\left.{}\times f_1^{*{\mathrm{МНА}}}(z; m^*, K^*)\right\} =0\,,\label{e3.17-sin}
 \end{multline}
где
\begin{multline*}
f_1^{*{\mathrm{МНА}}} (z; m^*, K^*) = \lk (2\pi)^k |K^*|\rk^{-1/2}\times{}\\
{}\times \exp \lf -
    \fr{1}{2} (z^{\mathrm{T}}-m^{*\mathrm{T}})(K^*)^{-1} (z-m^*)\rf\,.
    \end{multline*}

Аналогично в развернутом виде выписываются условия~(\ref{e1.20-sin}) и~(\ref{e1.21-sin}):
\begin{multline}
\fr{\prt g_1^{\mathrm{МНА}} (\la;t)}{\prt t} -\iin i\la^{\mathrm{T}} \left[ a_{10}^{\mathrm{МНА}} 
    (m_t, K_t, t) +{}\right.\\[2pt]
\left.    {}+ a_{11}^{\mathrm{МНА}} (m_t, K_t, t) (z- m_t) \right]\times{}\\[2pt]
{}\times e^{i\la^{\mathrm{T}} z} f_1^{\mathrm{МНА}} (z; m_t, K_t, t)\, dz=0\,;\label{e3.18-sin}
\end{multline}


\noindent
\begin{multline}
\iin i\la^{\mathrm{T}} \left[ a_{10}^{*{\mathrm{МНА}} } (m^*, K^*) 
+{}\right.\\[2pt]
\left.{}+a_{11}^{*{\mathrm{МНА}} } 
    (m^*, K^*) (z-m^*)\right]\times{}\\[2pt]
    {}\times
     e^{i\la^{\mathrm{T}}z} f_1^{*{\mathrm{МНА}} } (z; m^*, K^*)\, dz =0\,.
    \label{e3.19-sin}
    \end{multline}

Отсюда вытекают следующие теоремы.

\bigskip

\noindent
\textbf{Теорема~3.1.}\ \textit{Если существуют одно- и двумерные  плотности
стохастического режима, а  матрица $a_{11}^{\mathrm{МНА}}$ коэффициентов
статистической (нормальной) линеаризации асимптотически устойчива,
то приближенный алгоритм аналитического моделирования МНА
нестационарных стохастических режимов в СтС}~(\ref{e1.1-sin}) \textit{с инвариантной
мерой определяется выражениями}~(\ref{e3.1-sin})--(\ref{e3.7-sin}) и~(\ref{e3.16-sin}).

\bigskip

\noindent
\textbf{Теорема 3.2.}\ \textit{Если существуют стационарные одно- и
двумерные плотности стохастического режима, а матрица
$a_{11}^{*{\mathrm{МНА}}}$  коэффициентов статистической (нормальной)
линеаризации асимптотически устойчива, то приближенный алгоритм
аналитического моделирования стационарных стохастических режимов с
инвариантной мерой в стационарной СтС}~(\ref{e1.1-sin}) \textit{определяется 
выражениями}~(\ref{e3.10-sin})--(\ref{e3.12-sin}) и~(\ref{e3.17-sin}).

\bigskip

Как известно~[1, 2], одно- и двумерные нормальные распределения
определяют и все  $n$-мер\-ные распределения $(n\hm\ge 3)$, поэтому МНА и
МСЛ дают приближенные алгоритмы для любых многомерных плотностей
стохастических режимов, если они существуют. Аналогично
формулируются теоремы~3.3 и~3.4 на основе условий~(\ref{e3.18-sin}) и~(\ref{e3.19-sin}).


\section{О других приближенных методах и~алгоритмах аналитического моделирования 
распределений с~инвариантной мерой}

\vspace*{-2pt}

 Обобщением МНА являются различные
приближенные методы, основанные на параметризации распределений~[1, 2].
Аппроксимируя одномерную характеристическую функцию $g_1 (\la;t)$
и соответствующую плотность $f_1 (z,t)$ известными функциями
 $g_1^* (\la;\theta)$, $f_1^* (z;\theta)$,  зависящими от
конечномерного векторного параметра~$\theta$, можно свести задачу
приближенного определения одномерного распределения к выводу из
уравнения для характеристических функций обыкновенных
дифференциальных уравнений, определяющих~$\theta$ как функцию
времени. Это относится и к остальным многомерным распределениям.
При аппроксимации многомерных распределений целесообразно выбирать
последовательности функций $\{ f_n^* (z_1,\ldots,z_n;\theta_n)\}$ и 
$\{g_n^* (\la_1\tr \la_n;\theta_n)\}$, каждая пара
которых находилась бы в такой  зависимости от векторного параметра~$\theta_n$, 
чтобы при любом~$n$ множество параметров, образующих
вектор~$\theta_n$, включало в качестве подмножества множество
параметров, образующих вектор~$\theta_{n-1}$. Тогда при
аппроксимации $n$-мер\-но\-го распределения придется определять только
те координаты вектора~$\theta_n$, которые не были определены ранее
при аппроксимации функций $f_1, g_1\tr f_{n-1}$, $g_{n-1}$.

В зависимости от того, что представляют собой параметры, от
которых зависят функции $f_n^* (z_1\tr z_n;\theta_n)$ и 
$g_n^* (\la_1\tr \la_n;\theta_n)$, аппрок-\linebreak симирующие неизвестные
многомерные плотности $f_n (z_1,  \ldots,z_n; t_1 \tr t_n)$ и
характеристические функции $g_n (\la_1\tr \la_n; t_1,\ldots,t_n)$,
используются различные приближенные методы решения
 уравнений при условиях~(9)--(12), определяющих\linebreak многомерные
распределения вектора состояния сис\-те\-мы~$X_t$, в частности методы
моментов (ММ), семиинвариантов (МСИ), ортогональных разложений
(МОР), квазимоментов (МКМ) и~др.~[1, 2].

\vspace*{-6pt}


\section{Обобщение на~случай стохастических систем с~автокоррелированными шумами}

\vspace*{-2pt}

Пусть  СтС описывается нелинейным, в общем случае векторным дифференциальным 
стохастическим уравнением Ито~\cite{1-sin, 2-sin, 15-sin, 18-sin}

\noindent
\begin{equation}
\left.
\begin{array}{c}
    \dot Z = a(Z,t) + b_U(Z,t) U\,;\\[6pt] 
\displaystyle    \sss_{i=0}^l \alpha_i U^{(i)} =
\displaystyle\sss_{j=0}^h \beta_j V^{(j)}\enskip (h<l)\,.
\end{array}
\right\}
    \label{e5.1-sin}
    \end{equation}
    Здесь $U=U(t)$~--- векторная помеха размерности  $m\times 1$; $V\hm=V(t)$~--- 
    негауссовский белый шум с нулевым математическим ожиданием и известной функцией  
    $\chi\hm=\chi(\mu;t)$. В~таком случае в за\-ви\-си\-мости от степени <<гладкости>> 
    стохастического режима $Z\hm=Z(t)$ и помехи $U\hm=U(t)$ уравнения~(\ref{e5.1-sin})  
    путем расширения вектора состояния согласно~[1, 2] приводятся к виду~(\ref{e1.1-sin}) 
    для расширенного вектора состояния~$\bar Z$. Тогда, но уже для расширенного вектора 
    состояния СтС, при решении уравнений~(\ref{e5.1-sin}) могут быть использованы точные 
    (разд.~2) и приближенные (разд.~3) методы и алгоритмы аналитического моделирования 
    нестационарных и стационарных распределений с инвариантной мерой.

\section{Особенности аналитического моделирования распределений с~инвариантной мерой 
в~пуассоновских стохастических системах}

Рассмотрим СтС~(\ref{e1.1-sin}) при $b(z,t) \hm=I_m$ для обобщенного пуассоновского 
белого шума  $V^{\mathrm{OP}}\hm=  V^{\mathrm{OP}}(t)$, когда функция~(\ref{e1.2-sin}) 
определяется формулой
\begin{equation*}
\chi^{\mathrm{OP}} (\mu;t) =\lk g_c^{\mathrm{OP}} (\mu) -
1\rk \nu^{\mathrm{OP}} (t)\,, %\label{e6.1-sin}
\end{equation*}
где $g_c^{\mathrm{OP}} \hm=g_c^{\mathrm{OP}} (\mu)$~--- характеристическая 
функция скачков; $\nu^{\mathrm{OP}} \hm= \nu^{\mathrm{OP}} (t)$~--- 
интенсивность пуассоновского белого шума 
$V^{\mathrm{OP}}\hm=V^{\mathrm{OP}} (t)$. Обозначим через $f_c^{\mathrm{OP}} \hm=
 f_c^{\mathrm{OP}} (z)$ плотность скачков обобщенного пуассоновского процесса. 
 Тогда~(\ref{e1.3-sin}) будет представлять собой известное уравнение Фел\-ле\-ра--Кол\-мо\-го\-ро\-ва
\begin{multline}
\fr{\prt f_1(z;t)}{\prt t} + \fr{\prt^{\mathrm{T}}}{\prt z} 
    \lk a(z,t) f_1(z;t)\rk ={}\\
    \hspace*{-3mm}{}= \nu^{\mathrm{OP}} (t) \lk \iin f_c^{\mathrm{OP}} (z-\xi) f_1 (\xi;t)\, d\xi - f_1(z;t)\rk
    \label{e6.2-sin}
    \end{multline}
с начальным условием~(\ref{e1.4-sin}). В~случае простого пуассоновского белого шума 
с единичными скачками $g_c (\mu) \hm= e^{i\mu}$.

Для  стационарной пуассоновской СтС~(\ref{e1.1-sin}) уравнение~(\ref{e6.2-sin}) имеет следующий вид:
\begin{multline}
\fr{\prt^{\mathrm{T}}}{\prt z} \lk a^* (z) f_1^* (z)\rk = {}\\
{}=
\nu^{\mathrm{OP} *} \lk \iin f_c^{\mathrm{OP}} (z-\xi) f_1^* (\xi)\, d\xi- 
f_1^* (z)\rk\,.\label{e6.3-sin}
\end{multline}

Пользуясь уравнениями~(\ref{e6.2-sin}), (\ref{e6.3-sin})  
и результатами разд.~1 и~2, нетрудно сформулировать следующие утверждения.

\medskip

\noindent
\textbf{Теорема 6.1.}\ \textit{Функция $f_1 \hm= f_1(z;t)$ будет
нестационарным решением}~(\ref{e6.2-sin}), (\ref{e1.4-sin}) \textit{тогда и только тогда, 
когда $a$ допускает представление}~(\ref{e2.1-sin}) \textit{такое, что $f_1$ является плот\-ностью
инвариантной меры обыкновенного дифференциального уравнения}~(\ref{e2.2-sin}),
\textit{т.\,е.\ удовле\-тво\-ря\-ет условию}~(\ref{e2.3-sin}), \textit{а составляющая $a_2$ определяется
из решения следующего уравнения}:
\begin{multline*}
    \fr{\prt^{\mathrm{T}}}{\prt z} \lk a_2 (z,t) f_1 (z;t)\rk =
     \fr{1}{(2\pi)^k}\times{}\\
     {}\times \iin\iin \chi^{\mathrm{OP}} 
    \left(b(\xi,t)^{\mathrm{T}} \la;t\right) e^{i\la^{\mathrm{T}}(\xi-z)} f_1(\xi,t)\,d\xi d\la\,.
%    \label{e6.4-sin}
    \end{multline*}

%\smallskip

\noindent
\textbf{Теорема 6.2.}\ \textit{Функция $f_1^* \hm= f_1^* (z)$ будет стационарным 
решением}~(\ref{e6.3-sin}) \textit{тогда и только тогда, когда $a_2^*$ допускает 
представление}~(\ref{e2.5-sin}) \textit{такое, что  $f_1^*$ является плот\-ностью 
инвариантной меры}~(\ref{e2.6-sin}), \textit{а составляющая $a_2^{*}$ определяется 
из решения следующего уравнения}:
\begin{multline*}
\fr{\prt^{\mathrm{T}} }{\prt z} \lk a_2^{*} (z) f_1^* (z)\rk ={}\\
{}=
    \fr{1}{(2\pi)^k} \iin\iin \chi^{\mathrm{OP} *} (b(\xi)^{\mathrm{T}} \la) 
    e^{i\la^{\mathrm{T}}(\xi-z)} f_1^*(\xi)\,d\xi d\la\,.
%    \label{e6.5-sin}
    \end{multline*}

При использовании МНА и МСЛ для пуассоновских СтС непосредственно применяются теоремы~3.1--3.4, 
причем в формулу~(\ref{e3.9-sin}) для  
$\sigma(z,t)$ входит интенсивность 
$\nu^{\mathrm{OP}} (t)$ обобщенного пуассоновского белого шума.

\section{Тестовые примеры}

\noindent
\textbf{Пример~1}. Рассмотрим осциллятор Дуффинга в обобщенной пуассоновской 
стохастической среде:
\begin{equation}
\ddot X +\w^2 X -\mu X^3 =-\delta^{\mathrm{OP}} \dot X + V^{\mathrm{OP}} (t)\,.\label{e7.1-sin}
\end{equation}
Уравнения МСЛ для~(\ref{e7.1-sin}) имеют следующий вид:
\begin{equation}
\dot m_X = m_{\dot X}\,;\enskip 
\dot m_{\dot X} =- \w_{\mathrm{э}}^2 m_X -\delta^{\mathrm{OP}} m_{\dot X}\,;
\label{e7.2-sin}
\end{equation}
    \begin{equation}
    \left.
    \begin{array}{rl}
    \dot D_{X} &= 2 K_{X\dot X}\,;\\[6pt] 
    \dot D_{\dot X} &=\nu^{\mathrm{OP}} - 2 (\w_{1 \mathrm{э}}^2 K_{X\dot X} + 
    \delta^{\mathrm{OP}} D_{\dot X})\,;\\[6pt]
\dot K_{X\dot X} &= D_{\dot X} -\w_{1 \mathrm{э}}^2 D_X - 
\delta^{\mathrm{OP}} K_{X\dot X}\,.
\end{array}
\right\}
 \label{e7.3-sin}
\end{equation}
Здесь кубическая функция $X^3$ была заменена на статистически линеаризованную при 
гауссовом распределении с дисперсией  $D_X$ согласно~[1, 2]:
\begin{equation*}
X^3 \approx k_0 (m_X, D_X) m_X + k_1 (m_X, D_X) X^0\,,\label{e7.4-sin}
\end{equation*}
где
\begin{align*}
k_0 (m_X, D_X) &= m_X^2 + 3 D_X\,;\\ 
k_1 (m_X, D_X) &= 3 (m_X^2 + D_X)\,;\\
%\label{e7.5-sin}
\w_{\mathrm{э}}^2 &=\w^2 \lk 1- \fr{\mu (m_X^2 + 3D_X)}{\w^2}\rk\,;\\
\w_{1 \mathrm{э}}^2 &=\w^2 \lk 1-  \fr{3\mu (m_X^2 + D_X)}{\w^2}\rk \enskip 
(\w_{\mathrm{э}}>\w_{1 \mathrm{э}})\,.
\end{align*}
%\label{e7.6-sin}
Из~(\ref{e7.2-sin}) и~(\ref{e7.3-sin}) в стационарном режиме имеем:
\begin{gather*}
m_X^* =0\,;\enskip 
m_{\dot X}^* =0\,;\enskip 
K_{X\dot X}^* =0\,;\\
D_{\dot X}^* =\vartheta\,;\enskip 
\vartheta =  \fr{\nu^{\mathrm{OP}}}{ 2\delta^{\mathrm{OP}}}\,,
\end{gather*}
%\label{e7.7-sin}
а $D_X^*$ определяется из уравнения:
    \begin{equation*}
    \w_{1 \mathrm{э}}^2 (D_X^*) D_X^* =\vartheta\,. %\label{e7.8-sin}
    \end{equation*}
Условие наличия стационарного распределения с инвариантной мерой~(\ref{e3.17-sin}) 
требует консерватизма линеаризованной левой части~(\ref{e7.1-sin}). 
Процесс установления стационарных стохастических колебаний происходит 
в два этапа: сначала устанавливается $D_{\dot X}^*$, а затем $D_X^*$.

Интересно отметить, что уравнения МСЛ~(\ref{e7.2-sin}) и~(\ref{e7.3-sin}) сохраняют свой
вид и для любого белого шума интенсивности  $\nu(t)$,
представляющего собой с.к., производную от произвольного процесса с
независимыми приращениями~$W(t)$. Для гауссовского белого шума
$\nu\hm=\nu^G$ соответствующие результаты получены в~\cite{1-sin, 2-sin, 15-sin}. Как
показали вычислительные эксперименты для значений~$\mu$, отвечающих
стохастическим колебаниям, точность составляет около 10\%~\cite{15-sin}.

\medskip

\noindent
\textbf{Пример~2}.\  Для осциллятора Дуффинга в автокоррелированной  пуассоновской среде, когда
\begin{equation*}
\ddot X+ \w^2 X -\mu X^3 =-\delta^{\mathrm{OP}} \dot X + U\,;\enskip 
\dot U +\gamma U =V^{\mathrm{OP}} (t)\,, %\label{e7.9-sin}
\end{equation*}
уравнения МСЛ для  $Z\hm= [X\dot X U]^{\mathrm{T}}$ имеют вид~(\ref{e3.5-sin}) и~(\ref{e3.6-sin}) при
    \begin{gather*}
   a_1 = \begin{bmatrix}
        m_{\dot X}\\
        -\w_{ \mathrm{э}}^2 m_X-\delta^{\mathrm{OP}} m_{\dot X}\\
        -m_U\end{bmatrix}\,;\\
    \alpha=  \begin{bmatrix}
            0&1&0\\
            -\w_{1 \mathrm{э}}^2&-\delta^{\mathrm{OP}}&0\\
            0&0&-\gamma\end{bmatrix}\,;\enskip
    \beta= \begin{bmatrix}
        0&0&0\\
        0&0&0\\
        0&0&1\end{bmatrix}\,;
%        \label{e7.10-sin}
\\
a_2 =\alpha K_t+ K_t \alpha^{\mathrm{T}} +\beta \nu^{\mathrm{OP}} \beta^{\mathrm{T}}\,.
        \end{gather*}
Здесь $\nu^{\mathrm{OP}} =\nu^{\mathrm{OP}}(t)$~--- интенсивность белого шума 
$V^{\mathrm{OP}}(t)$. 
Отсюда аналитическим мо\-де\-ли\-ро\-ванием определяются стационарные
режимы, а также режимы их установления. Так же, как в\linebreak случае
автокоррелированных гауссовских белых шумов~\cite{1-sin, 2-sin, 15-sin}, точность МСЛ
за счет <<профильтрованности>> помех значительно повышается и
достигает 2\%--4\%. Результат справедлив и для произвольных
негауссовских белых шумов.

\medskip

\noindent
\textbf{Пример 3}.\  Для релейного осциллятора в гауссовской стохастической среде
\begin{equation}
\ddot X + \w^2 {\mathrm{sgn}} X = -\delta^G \dot X + V^G + U_0\label{e7.11-sin}
\end{equation}
плотность распределения стационарного режима стохастических колебаний при $U_0\hm=0$ 
определяется формулой Гиббса~[1, 2]:
\begin{equation}
f^* (x,\dot x) = c \exp \lf - 
    \fr{H(x,\dot x)}{\vartheta^G}\rf\,,\enskip \vartheta^G = 
    \fr{\nu^G}{ 2\delta^G}\,.\label{e7.12-sin}
    \end{equation}
Здесь через
\begin{equation*}
H(x,\dot x) = \fr{\dot x^2}{2} +\Pi(x)\,,\enskip \Pi (x) =\w^2 |x|\,, %\label{e7.13-sin}
\end{equation*}
обозначена полная энергия осциллятора.

Для~(\ref{e7.11-sin}) при  $U_0\hm\ne 0$, если заменить релейную характеристику 
статистически линеаризованной, согласно~[1, 2]
\begin{equation*}
\mathrm{sgn}\, X = k_0 (m_X, D_X) m_X + k_1 (m_X, D_X) (X^0 - m_X)\,; %\label{e7.14-sin}
\end{equation*}
    $$
    k_0(m_X, D_X) =\fr{2}{ m_X} \Phi \left( \fr{m_X}{\sqrt{D_X}}\right)\,;
    $$
    $$ 
    k_1 (m_X,D_X) = \fr{1}{\sqrt{D_X}} \sqrt{\fr{2}{\pi}}\, \exp \left( -\fr{m_X^2}{2D_X}\right)\,;
    $$
\begin{equation}
\Phi (\tau) = \fr{1}{2\pi} \int\limits_0^\tau e^{-t^2/2} dt\,.\label{e7.15-sin}
\end{equation}
Тогда уравнения МСЛ будут иметь вид:
\begin{equation}
\left.
\begin{array}{rl}
\dot m_X &= m_{\dot X}\,;\\[9pt]
\dot m_X &= U_0 - \w^2 k_0 (m_X, D_X) m_X -\delta m_{\dot X}\,;
\end{array}
\right\}
\label{e7.16-sin}
\end{equation}
    \begin{equation}
\left.
\hspace*{-3.5mm}\begin{array}{c}
    \dot D_X = 2 K_{X\dot X}\,;
\\
    \dot D_{\dot X} = \nu^G - 2\lk \delta D_{\dot X} + \w^2 k_1(m_X,D_X) K_{X\dot X}\rk\,;\\[9pt]
    \dot K_{X\dot X} = D_{\dot X} - \w^2 k_1 (m_X, D_X) D_X - \delta K_{X\dot X}\,,
    \end{array}
    \right\}\!\!
    \label{e7.17-sin}
    \end{equation}
где $\delta \hm= \delta^G$, $\nu\hm=\nu^G$.
Отсюда для стационарных стохастических колебаний имеем связанную систему уравнений:
\begin{equation}
m_{\dot X}^* =0\,;\enskip \w^2 k_0 (m_X^*, D_X^*) = U_0\,;\label{e7.18-sin}
\end{equation}
\begin{equation}
\left.
\begin{array}{c}
K_{X\dot X}^* =0\,;\enskip 
D_X^* =\vartheta=\displaystyle \fr{\nu}{ 2\delta}\,;\\[9pt]
k_1(m_X^*, D_X^*) D_X^* =\rho= \displaystyle \fr{\vartheta}{\w^2} =\fr{\nu}{ 2\delta \w^2}\,.
\end{array}
\right\}
\label{e7.19-sin}
\end{equation}

При $U_0 =0$ из~(\ref{e7.15-sin}), (\ref{e7.18-sin}) и~(\ref{e7.19-sin}) находим:
\begin{equation*}
m_X^* =0\,;\enskip 
m_{\dot X}^* =0\,; \enskip 
D_{\dot X}^* =\vartheta\,;\enskip 
D_X^* =  \fr{\pi}{2}\,\rho^2\,. %\label{e7.20-sin}
\end{equation*}
Отсюда видно, что стационарная дисперсия скорости совпадает с точным
решением~(\ref{e7.12-sin}). Стационарная дисперсия координаты, найденная
согласно МСЛ, отличается от следующего точного решения, полученного
согласно~(\ref{e7.12-sin}). При $\rho\hm \le 1$ относительная ошибка составляет
10\%. Стационарные колебания по~$X$ и $\dot X$ не коррелированы.

Уравнения~(\ref{e7.16-sin}) и~(\ref{e7.17-sin}) показывают, что процесс установления 
режима стохастических колебаний происходит в две стадии: сначала устанавливается 
стационарное распределение по ско\-рости~$\dot X$, а затем по координате~$X$.

\medskip

\noindent
\textbf{Пример 4}.  В~условиях примера~3, но для пуассоновской среды, когда
    \begin{equation*}
    \ddot X +\w^2 {\mathrm{sgn}} X =-\delta^{\mathrm{OP}} \dot X + 
    V^{\mathrm{OP}} + U_0\,,
%    \label{e7.21-sin}
    \end{equation*}
уравнения МСЛ имеют вид~(\ref{e7.16-sin}), (\ref{e7.17-sin}), если принять 
$\delta\hm= \delta^{\mathrm{OP}}$, $ \nu\hm=\nu^{\mathrm{OP}}$, 
$\vartheta\hm=\vartheta^{\mathrm{OP}}\hm=\nu^{\mathrm{OP}}/(2\delta^{\mathrm{OP}})$, 
$\rho \hm=\vartheta^{\mathrm{OP}}/\w^2$. Точного аналитического уравнения 
Фел\-ле\-ра--Кол\-мо\-го\-ро\-ва не обнаружено.

Другие тестовые примеры можно найти в~[10, 12--14].

\section{Заключение}

Дано обобщение точных и приближенных (основанных на параметризации распределений)\linebreak 
методов и алгоритмов теории распределений с инвари\-антной мерой на случай нелинейных 
дифференциальных гауссовых и негауссовых стохастических систем с гладкими и разрывными 
характеристиками.

Особое внимание уделено пуассоновским стохастическим системам с разрывными характеристиками.

На тестовых примерах показана достаточная точность для практических приложений в стохастической 
информатике.

{\small\frenchspacing
{%\baselineskip=10.8pt
\addcontentsline{toc}{section}{Литература}
\begin{thebibliography}{99}
\bibitem{1-sin}
\Au{Пугачёв В.\,С., Синицын И.\,Н.} Стохастические дифференциальные системы. 
Анализ и фильтрация.~--- 2-е изд., доп.~--- М.: Наука, 1990.

\bibitem{2-sin}
\Au{Пугачёв В.\,С., Синицын И.\,Н.} Теория стохастических систем.~--- 2-е изд.~--- М.: Логос,  2004.

\bibitem{3-sin}
\Au{Moshchuk N.\,K., Sinitsyn I.\,N.} On stationary distributions in nonlinear 
stochastic differential systems: Preprint.~--- Coventry, UK: 
University of Warwick, Mathematics Institute, 1989. 15~p.

\bibitem{4-sin}
\Au{Moshchuk N.\,K., Sinitsyn I.\,N.} On stochastic nonholonomic systems: Preprint.~--- 
Coventry, UK: University of Warwick, Mathematics Institute, 1989. 32~p.

\bibitem{5-sin}
\Au{Мощук Н.\,К., Синицын И.\,Н.} О~стохастических неголономных системах~// 
Прикладная механика и математика, 1990. Т.~54. Вып.~2. С.~213--223.

\bibitem{6-sin}
\Au{Moshchuk N.\,K., Sinitsyn I.\,N.} On stationary distributions in 
nonlinear stochastic differential systems~// Quart. J. Mech. Appl. Math., 1991. Vol.~44.  
Pt.~4.  P.~571--579.

\bibitem{7-sin}
\Au{Мощук Н.\,К., Синицын И.\,Н.} О~стационарных и приводимых к стационарным 
режимах в нормальных стохастических системах~// 
Прикладная механика и математика, 1991. Т.~55. Вып.~6. С.~895--903.

\bibitem{8-sin}
\Au{Мощук Н.\,К., Синицын И.\,Н.} Распределения с инвариантной мерой в механических 
стохастических нормальных сис\-те\-мах~// Докл. АН СССР, 1992. Т.~322. №\,4. С.~662--667.

\bibitem{9-sin}
\Au{Синицын И.\,Н.} Конечномерные распределения с инвариантной мерой в стохастических 
механических сис\-те\-мах~// Докл. РАН, 1993. Т.~328. №\,3. С.~308--310.

\bibitem{13-sin} %10
\Au{Soize C.} The Fokker--Plank equation for stochastic dynamical systems 
and its explicit steady state solutions.~--- Singapore: World Scientific,  1994.

\bibitem{10-sin} %11
\Au{Синицын И.\,Н.} Конечномерные распределения с инвариантной мерой в 
стохастических нелинейных дифференциальных системах.~--- М.: Диалог--МГУ, 1997. С.~129--140.

\bibitem{11-sin} %12
\Au{Синицын И.\,Н., Корепанов Э.\,Р., Белоусов~В.\,В.} 
Точные методы расчета стационарных режимов с инвариантной мерой в стохастических 
сис\-те\-мах управ\-ле\-ния~// Кибернетика и технологии XXI~ве\-ка: Тр.\ II Междунар. 
науч.-техн. конф. C\&T'2002.~--- Воронеж: Саквое, 2002. С.~124--131.

\bibitem{12-sin} %13
\Au{Синицын И.\,Н., Корепанов Э.\,Р., Белоусов~В.\,В.} 
Точные аналитические методы в статистической динамике нелинейных 
ин\-фор\-ма\-ци\-он\-но-управ\-ля\-ющих сис\-тем~// Сис\-те\-мы и средства информатики. 
Спец. вып. Математическое и алгоритмическое обеспечение 
ин\-фор\-ма\-ци\-он\-но-те\-ле\-ком\-му\-ни\-ка\-ци\-он\-ных сис\-тем.~--- М.: Наука, 2002. С.~112--121.

\bibitem{14-sin}
\Au{Синицын И.\,Н.} Развитие методов аналитического моделирования распределений с 
инвариантной мерой в стохастических сис\-те\-мах~// Современные проб\-ле\-мы 
прикладной математики, информатики и автоматизации: Тр. Междунар. науч.-техн. семинара.~--- 
Севастополь, 2012. С.~24--35.

\bibitem{15-sin}
\Au{Синицын И.\,Н.} Аналитическое моделирование распределений с инвариантной мерой 
в стохастических сис\-те\-мах с автокоррелированными шумами~// 
Информатика и её применения, 2012. Т.~6. Вып.~4. С.~4--8.

\bibitem{16-sin}
\Au{Немыцкий В.\,В., Степанов В.\,В.} Качественная теория дифференциальных уравнений.~--- 
М.--Л.: Гостехиздат, 1949.


\bibitem{17-sin}
\Au{Козлов В.\,В.} О~существовании интегрального инварианта гладких динамических систем~// 
ПММ, 1987. №\,1. С.~538--545.

\label{end\stat}

\bibitem{18-sin}
\Au{Синицын И.\,Н.} Фильтры Калмана и Пугачёва.~--- 2-е изд.~--- М.: Логос, 2007.
\end{thebibliography}
}
}

\end{multicols}   %1+
\renewcommand{\figurename}{\protect\bf Figure}
\renewcommand{\tablename}{\protect\bf Table}

\def\stat{kabanov}


\def\tit{ON UNIQUENESS OF CLEARING VECTORS REDUCING~THE~SYSTEMIC RISK}

\def\titkol{On uniqueness of clearing vectors reducing the systemic risk}

\def\autkol{Kh.\ El Bitar,  Yu.~Kabanov, and~R.~Mokbel}

\def\aut{Kh.\ El Bitar$^1$,  Yu.~Kabanov$^2$, and~R.~Mokbel$^3$}

\titel{\tit}{\aut}{\autkol}{\titkol}

%{\renewcommand{\thefootnote}{\fnsymbol{footnote}}
%\footnotetext[1] {The 
%research of Yuri Kabanov was done under partial financial support   of the grant 
%of  RSF No.\,14-49-00079.}}

\renewcommand{\thefootnote}{\arabic{footnote}}
\footnotetext[1]{Laboratoire de Math$\acute{\mbox{e}}$matiques, Universit$\acute{\mbox{e}}$ de 
Franche-Comt$\acute{\mbox{e}}$, 16~Route de Gray, 25030 \mbox{Besan{\!\ptb{\c{c}}}on}, CEDEX, France, 
\mbox{khalilbitar\_aw@hotmail.com}}
\footnotetext[2]{Laboratoire de 
Math$\acute{\mbox{e}}$matiques, Universit$\acute{\mbox{e}}$ de
 Franche-Comt$\acute{\mbox{e}}$, 16~Route de Gray, 25030 
\mbox{Besan{\!\ptb{\c{c}}}on}, CEDEX, France; 
Institute of Informatics Problems, Federal Research 
Center ``Computer Science and Control'' of the Russian Academy of Sciences, 
44-2~Vavilov Str., Moscow 119333, Russian Federation; 
National Research University 
``MPEI,'' 14~Krasnokazarmennaya Str., Moscow, 111250, Russian Federation, 
\mbox{Youri.Kabanov@univ-fcomte.fr}}
\footnotetext[3]{Laboratoire de 
Math$\acute{\mbox{e}}$matiques, Universit$\acute{\mbox{e}}$ de 
Franche-Comt$\acute{\mbox{e}}$, 
16~Route de Gray, 25030  \mbox{Besan{\!\ptb{\c{c}}}on}, CEDEX, France,
\mbox{ritamokbel@hotmail.com}}

\index{El Bitar Kh.}
\index{Kabanov Yu.}
\index{Mokbel R.}
\index{Эль Битар Х.}
\index{Кабанов Ю.}
\index{Мокбель Р.}


\vspace*{-12pt}

\def\leftfootline{\small{\textbf{\thepage}
\hfill INFORMATIKA I EE PRIMENENIYA~--- INFORMATICS AND APPLICATIONS\ \ \ 2017\ \ \ volume~11\ \ \ issue\ 1}
}%
 \def\rightfootline{\small{INFORMATIKA I EE PRIMENENIYA~--- INFORMATICS AND APPLICATIONS\ \ \ 2017\ \ \ volume~11\ \ \ issue\ 1
\hfill \textbf{\thepage}}}




\Abste{Clearing of financial system, i.\,e., of a~network of interconnecting banks, is 
a~procedure of simultaneous repaying debts to reduce their total volume. The 
vector whose components are  repayments of each bank
is called clearing vector.  In  simple models  considered  by Eisenberg and Noe 
(2001) and, independently,  by Suzuki (2002), it was shown that
the  clearing  to the minimal value of debts  accordingly to natural rules  can 
be formulated as fixpoint problems.
The existence
of their solutions, i.\,e., of clearing vectors,  is rather straightforward and can 
be obtained by a~direct reference to the Knaster--Tarski or Brouwer theorems.  
The uniqueness of clearing vectors is a~more delicate problem which was solved 
by Eisenberg and Noe  using a~graph structure of the financial network.  
The uniqueness  results have been proved in two generalizations of the  Eisenberg--Noe model:  
in the Elsinger model with seniority of liabilities and in the Amini--Filipovic--Minca 
type model with several
types of illiquid assets whose firing sale has a~market impact.}

\KWE{systemic risk;  financial networks; clearing; Knaster--Tarski 
theorem; Eisenberg--Noe model; debt seniority; price impact}

\DOI{10.14357/19922264170110} 

\vspace*{7pt}


\vskip 12pt plus 9pt minus 6pt

      \thispagestyle{myheadings}

      \begin{multicols}{2}

                  \label{st\stat}


\section{Introduction}

\noindent
To explain the clearing problem, let us start with the simplest example of 
a~financial
system with two agents each having in a~cash 10 dollars. The first agent gets 
from the second a~credit of~1M  dollars, the second gets from the first  a~credit 
of~1~M and 1~dollars. Apparently, as a~result, both agents have a~huge liabilities with 
respect to each other. Of course, the agents can be asked to reduce their 
liabilities by reimbursing credits partially (e.\,g., to the levels~0.5~M and 
0.5~M\;+\;1 in liabilities and~10~dollars both in cash) or completely, with zero 
liabilities and cash reserves~11 and~9~dollars, respectively. Intuitively, the 
situation where the liability is reduced (i.\,e., the system is cleared) seems to 
be less risky: if one of the agents became bankrupt and only the percentage of the 
huge debt value  can be reimbursed, the creditor's losses will be also huge. For 
complex financial systems involving large numbers of agents
with chains of borrowing,  the clearing problem, that is, the reduction of 
absolute values by reimbursement, looks much more complicated.
{ %\looseness=1

}

In the influential paper~\cite{Eisenberg-Noe} published in 2001, Eisenberg and 
Noe suggested a~clearing procedure in the model describing a~financial system 
composed by~$N$~banks (under ``banks''  can be understood  various financial 
institutions); a~more general model was introduced independently at the same 
time by Suzuki~\cite{Suzuki}.   The assets of the bank are cash and interbank 
exposures which are, in turn, liabilities for its debtors.  The clearing 
consists in simultaneous paying all debts. Each bank pays to its counterparties 
the debts \textit{pro rata} of their relative volume using its cash reserve and 
money collected from the credited banks. The rule is: either all debts are payed 
in full or the zero level of the equity is attained and the bank defaults. The 
totals reimbursed by banks form an $N$-dimensional clearing vector. A~remarkable 
feature is that this vector is a~fixed point of a~monotone mapping of a~complete 
lattice into itself and its existence follows immediately from the 
Knaster--Tarski  theorem, a~beautiful and fairy simple result which proof needs only 
a~few lines of arguments~\cite{Tarski}. The uniqueness of the clearing vector is a~more 
delicate 
result involving the graph structure of the system.

The ideas of the  Eisenberg--Noe paper happened to be very fruitful and their 
model
was generalized in many directions having not only financial importance but 
posing  interesting mathematical questions. One of them is the question on 
uniqueness of clearing vector or   equilibrium  on financial market.

The first theorem provides a~new sufficient condition for the Elsinger model of 
clearing with debts priority structure. This model is given by a~set of 
liability matrices corresponding to each seniority. The idea of the present approach is 
to use the largest clearing vector which always exists to construct a~new 
liability matrix generating a~graph structure with which one can work in 
a~similar way as in the Eisenberg--Noe model.
The second theorem deals with the uniqueness of  equilibrium in a~clearing  
model with several illiquid assets and a~market impact.  In the presence of 
several illiquid assets,  the banks are faced the choice of  asset selling 
strategies. The proportional scheme of selling similar to that in the 
paper by Cont--Wagalath~\cite{Cont-Wag} has been used 
leaving game-theoretical versions for 
future studies.  In the case of one illiquid asset, the
obtained result is close to that  
of the study by Amini--Filipovic--Minca~\cite{AFM}, but the present definition of the 
equilibrium is different (but equivalent).

The structure of the note is as follows. In the introductory section~2, 
 the general principle and results are discussed briefly in the framework of the 
Eisenberg--Noe model. To facilitate the comparison with further development, 
also, short proofs are provided.
In section~3, a~uniqueness of the clearing vector for the Elsinger model 
where senior  liabilities should be reimbursed before the juniors ones. Section~4 
contains
the sufficient condition  for the uniqueness of the equilibrium in the model
where clearing requires selling of the illiquid assets with price impact.  
Economically speaking, it is  oriented to the recovering of the market  after 
fire sales.  For the reader convenience,  in Appendix, 
a~short information about the Knaster--Tarski theorem adapted to 
the present authors' needs is provided.



\noindent
\textbf{Notations.}\ The partial ordering in~$\mathbb{R}^n$ and its 
subsets  induced by the cone~$\mathbb{R}^n_+$ is denoted by $\ge$. In other words, the inequality $y\ge x$ 
is understood componentwise. Also, the symbols $x\wedge y$ and $x\vee y$ mean, 
respectively, the componentwise minimum and maximum, $x^+:=x\vee 0$ and\linebreak 
$x^-:=(- x)^+$.
The notation $[x,z]$ is used for the order interval, i.\,e.,
$[x,z]=\{y\in \mathbb{R}^n:\ x\le y\le z \}$.
If $A\subseteq [x,z]$, then $\inf A$ is the unique element $\underline y\in 
[x,z]$ such
that $\underline y\le y$ for all $y\in A$ and for any $\tilde y$  such that 
$\tilde y\le y$ for all $y\in A$, one has $\tilde y\le \underline y$, that 
is, the component $\underline y^i=\inf \{y^i:\ y\in A\}$ for  $i=1,\dots,n$.

The matrix notations are used where the vectors are columns, $'$ is the symbol of 
transpose, and\linebreak  ${\bf 1}':=(1,\dots,1)$ (the dimension of the vector is supposed 
to be clear from the context).

\vspace*{-9pt}

\section{The Eisenberg--Noe Model}

\noindent
In~\cite{Eisenberg-Noe}, Eisenberg and Noe investigated the model 
describing a~financial system composed of $N$ banks (under ``banks"  can be 
understood  various financial institutions). In the aggregate oversimplified  
form, the balance sheet of the bank $i$ can be split into two parts: assets and 
liabilities. The assets are of two types:  interbank assets (exposures)~$\tilde X^i$ 
and cash~$e^i$.  The liabilities are: interbank debts (liabilities)~$\tilde L^i$ 
and the equity~$C ^i$ (or proper capital reserve) equalizing the two sides 
of the balance sheet:
\vspace*{2pt}

\noindent
$$
e^i+\tilde X^i= \tilde L^i + C^i\,.
$$

\vspace*{-2pt}

\noindent
All these values are assumed to be greater or equal to zero. The condition that 
$C^i\ge 0$ means that the bank is solvent.

More detailed balance sheet provides the information on the values  of 
liabilities of the bank  $i$ to the bank $j$, namely,  vectors 
$(L^{i1},\ldots,L^{iN})'$ of liabilities and  $(X^{i1},\ldots,X^{iN})$ of exposures.
 With this, one 
has  $\tilde X^i=X^{i1}+\cdots+X^{iN}$ and $\tilde L^i =L^{i1}+\cdots+L^{iN}$.

The matrix $L=(L^{ij})$ with positive entries and zero diagonal defines the
total interbank exposures. Since the value of the exposure of~$i$ to~$j$ is the 
value of the liability of~$j$ to~$i$, one has that $L'=X$.  So, 
the matrix $L$ and the vector~$e$ give a~description of a~financial system in 
this model.

Put

\vspace*{-3pt}

\noindent
$$
\Pi^{ij}:=
\begin{cases}
\fr {L^{ij}}{\tilde L^i}=\displaystyle \fr {L^{ij}}{\sum\nolimits_j L^{ij}} 
&\ \mbox{if } \tilde L^i\neq 0\,; \\
\delta^{ij} &\  \mbox{otherwise}
\end{cases}
$$
where the Kronecker symbol $\delta^{ij}=0$ for $i\neq j$ and $\delta^{ii}=1$.
Then,~$\Pi^{ij}$  describes the proportion of the value debtor $i$ due to the 
creditor~$j$ of the total interbank debt of~$i$; $\Pi=(\Pi^{ij})$  is called 
relative liabilities matrix. Note that in this definition, to get a~stochastic 
matrix $\Pi$, we deviate from~\cite{Eisenberg-Noe} where $\Pi^{ii}=0$ when 
$L^i= 0$.

%As an example consider the simplest system with two banks where 
%$L^{12}=L^{21}+\varepsilon$ where $\e<0$ can be thought small with respect to~$L^{21}$.  
%After paying debts in the cleared system the matrix of liabilities will have the 
%entries $L^{12}_{c}=\e$, $L^{21}_c=0$. That is the values of debts are reduced 
%and so are eventual values of losses in the case of defaults of a~partner.

In general, financial system   may have a~complicated structure with cyclical 
interdependences and  banks may have large exposures within cycles. To reduce 
them, one can impose a~clearing mechanism satisfying several natural 
requirements: limited liability and proportionality. Formally,  this  leads to 
the concept of a~\textit{clearing payment vector} $p^*\in \prod_i[0,\tilde L^i]$ 
satisfying the following properties:
\begin{itemize}
\item[$a.$] \textit{Limiting liability}. For every $i$,
$$
p_i^*\le e^i+\sum\limits_j\Pi^{ji}p_j^*\,.
$$

\item[$b.$] \textit{Absolute priority.} For every $i$, either $p^*_i=\tilde L^i$, or
$$
p_i^*= e^i+ \sum\limits_j\Pi^{ji}p_j^*.
$$
\end{itemize}
One may think that the  central clearing authority forces  each bank to make 
a~``fair'' payment of debts in such\linebreak\vspace*{-12pt}

\pagebreak

\noindent
 a~way that, having  the total payment~$p_i^*$, 
the bank~$i$ remains solvent and  pays to~$j$ the fraction $p_i^*\Pi^{ij}$ in 
such a~way that either its total debts are paid,  or all the resources are 
exhausted.

Alternatively, the conditions~$a$ and~$b$ can be written in the following way:
\begin{equation}
\label{p^*} p^*=\min \left\{ e+\Pi' p^*, \tilde L\right\}
\end{equation}
where the minimum is understood in the componentwise sense, i.\,e., accordingly to 
the partial ordering defined by the cone~${\mathbb{R}}^N_+$.

The main result of Eisenberg and Noe asserts that the set of clearing vectors is 
nonempty. Moreover, there are the minimal and the maximal clearing vectors,  
denoted here~$\underline p$ and~$\bar p$, respectively.  This assertion follows
immediately from the Knaster--Tarski fixed point theorem: the monotone mapping 
$f:p\mapsto (e+\Pi'p)\wedge \tilde L$ of a~complete lattice $[0,\tilde L]$ into 
itself has the largest and the smallest fixed points (for 
details, see section~5). The set $[0,\tilde L]$ is convex and compact and~$f$ is a~continuous 
mapping. So, the existence of its fixed point follows also from the classical  
Brouwer theorem.

Using the obvious identity $(x-y)^+=x -x\wedge y$, one can rewrite 
Eq.~(\ref{p^*}) in the following equivalent form:
\begin{equation}
\label{alt1}
\left(e+\Pi'p^*-\tilde L\right)^+=e+\Pi'p^*-p^*
\end{equation}
where the left-hand side is the equity vector of the system after clearing.

%After clearing by an arbitrary clearing (outflow) vector $p^*$ the equity 
%vector of the system  is
%$$(e+\Pi'p^*-\tilde L)^+=e+\Pi'p^*-p^*; $$
%this equality is nothing but the equation (\ref{p^*}) written in an equivalent 
%form.

An important but simple observation: {\it the equity (after clearing) does not 
depend on the clearing vector}.
Indeed,~$\Pi$~being  a~stochastic matrix, ${\bf 1}'\Pi'={\bf 1}'$.  Therefore, 
multiplying  the above representation~(\ref{alt1}) from the left by~${\bf 1}'$, 
one gets that  the sum of equities
$$
{\bf 1}'\left(e+\Pi'p^*-\tilde L\right)^+={\bf 1}'e
$$
is equal to the sum of the initial cash reserves, that is, invariant with respect 
to the choice of the clearing vector.
On the other hand, by monotonicity, one has that
$$
\left(e+\Pi'p^*-\tilde L\right)^+\le \left(e+\Pi'\bar p-\tilde L\right)^+.
$$
If the both sides here are not equal, then
$$
{\bf 1}'\left(e+\Pi'p^* - \tilde L\right)^+< {\bf 1}'\left(e+\Pi'\bar p-\tilde L\right)^+$$
in contradiction with the invariance of the  total of equities.

\smallskip

\noindent
\textbf{Sufficient condition for the uniqueness of the clearing vector.}
As in~\cite{Eisenberg-Noe}, let us assume for simplicity that $\tilde L^i>0$ 
for all~$i$.

For a~stochastic matrix~$\Pi$,  we say that
$I\subseteq  \{1,\ldots,N\}$ is
a~($\Pi$-)\textit{surplus set} if $\Pi^{ij}=0$ for all $i\in I$, $j\in I^c$, 
and~$\sum_{j\in I}e^j>0$.

\columnbreak

Recall that~$j$ is the creditor of $i$ if $\Pi^{ij}>0$ (i.\,e., $\Pi^{ij}>0$); in 
this case, let us use, as in the  theory of Markov chains or in the graph 
theory,  the notation $i\to j$.

Let us denote by $o(i)$ {\it the orbit of $i$} that is the set of all~$j$ for which 
there is a~directed path 
$$
i\to i_1\to i_2\to\cdots\to j\,,$$ 
i.\,e.,  $o(i)$ is the set 
of all direct or indirect creditors of~$i$.

Note that the orbit $o(i)$ with $\sum_{j\in I}e^j>0$ is a~surplus set. Indeed,
if\ $\Pi^{jj'}>0$ for some $j\in o(i)$, $j'\notin o(i)$, i.\,e.,  $j\to j'$, then 
there is
a~path 
$$i\to i_1\to i_2\to\cdots\to j \to  j'\,.
$$


\noindent
\textbf{Lemma~1.}\
%\label{equity>0}
 \textit{Suppose that the market is cleared by a~vector $p^*\in [0,\tilde L]$. Let~$I$ 
be a~surplus set.  Then, at least one node of~$I$ has a~strictly positive equity 
value}.

\textit{In particular,
any orbit~$o(i)$ with $\sum_{j\in o(i)}e^j>0$ has an element with strictly  
positive equity value}.

\smallskip

\noindent
P\,r\,o\,o\,f\,.\ \  Multiplying the identity~(\ref{alt1}) by~${\bf 1}'_I$ and noticing 
that
$({\bf 1}'_I\Pi')^i=1$ for $i\in I$,
one obtains that
$$
{\bf 1}'_I \left(e+\Pi'p^*-\tilde L\right)^+\ge {\bf 1}'_I e>0
$$
implying the claim.~$\square$

\smallskip

A financial system is called \textit{regular} if for  every~$i$, the orbit~$o(i)$ is 
a~surplus set.

\smallskip

\noindent
\textbf{Theorem~1.}\
%\label{uni1}
\textit{Suppose that the financial system is regular.
Then}, $\underline p=\bar p$.

\smallskip

\noindent
P\,r\,o\,o\,f\,.\ \  Suppose that~$\underline p$ and~$\bar p$ are not equal, i.\,e., 
$\underline p\le \bar p$ but for some~$i$, one 
has the strict inequality  $\underline p^i<\bar p^i$.
Denote by~$C$ the vector of equities (it is common for all clearing vectors).
By assumption, the orbit~$o(i)$ is a~surplus set and by Lemma~1, it 
contains an element~$m$ with the equity value $C^m>0$. By definition of the 
orbit, there is a~path $i\to i_1\to \cdots \to m$ and one may assume without loss of 
generality that in this path,~$m$ is  the first node with strictly positive 
equity value.

First, let us prove that  one may consider only the case where the path
consists of one step,  i.\,e., $i\to m$.  To this end, let us check that
$\underline p^{i_1}<\bar p^{i_1}$ if $i_1\neq m$. In other words, the property 
that $\underline p^i\neq \bar p^i$ propagates along the path.

Suppose that $\bar p^{i_1}< \tilde L^{i_1}$. Then, also, $\underline p^{i_1}< 
\tilde L^{i_1}$.  In such a~case,

\vspace*{3pt}

\noindent
$$
 \underline p^{i_1}=e^{i_1}+ \sum\limits_j\Pi^{ji_1}\underline p^j\,, \enskip \bar 
p^{i_1}=e^{i_1}+\sum\limits_j\Pi^{ji_1}\bar p^j
$$
and one has  that

\vspace*{3pt}

\noindent
$$
\bar p^{i_1}-\underline p^{i_1}=\sum\limits_j\Pi^{ji_1}\left(\bar p^j-\underline p^j\right)>0
$$

\vspace*{-6pt}

\noindent
because   $\Pi^{ii_1}>0$, that is, $\underline p^{i_1}<  \bar p^{i_1}$. This 
inequality also holds trivially, if
$\bar p^{i_1}= \tilde L^{i_1}$ but $\underline p^{i_1}< \tilde L^{i_1}$.
 The remaining\linebreak\vspace*{-12pt}
 
 \pagebreak
 
 \noindent
  case where
$\underline p^{i_1}=\bar p^{i_1}=\tilde L^{i_1}$ is excluded as it is supposed that 
$C^{i_1}=0$.  Indeed, according to~(\ref{alt1}),  this leads to the equalities:
$$
e^{i_1}+ \sum\limits_j\Pi^{ji_1}\bar p^j - \tilde L^{i_1}=0\,;\enskip
e^{i_1}+  \sum\limits_j\Pi^{ji_1}\underline p^j - \tilde L^{i_1}=0\,,
$$
implying the identity
$$
\sum\limits_j\Pi^{ji_1}\left(\bar p^j-\underline p^j\right)=0
$$
which cannot be true since in the above sum, the term corresponding to $j=i$ is 
strictly positive.

So, it is sufficient to consider only one-step case. Since $C^m>0$, one has the 
representations:
\begin{align*}
C^m&=e^{m}+ \sum\limits_j\Pi^{jm}\underline p^j - \tilde L^{m}\,; \\
C^m&=e^{m}+ \sum\limits_j\Pi^{jm}\bar p^j- \tilde L^{m}\,.
\end{align*}
As above, one again obtains the impossible equality:
$$
\sum\limits_j\Pi^{jm}\left(\bar p^j-\underline p^j\right)=0\,.
$$
Therefore, the  assumption $\underline p^i<\bar p^i$ leads to a~contradiction. 
The
uniqueness of clearing vector is proven.~$\square$


\smallskip

\noindent
\textbf{Remark~1.}\
The above  theorem reveals that the problem to find a~clearing 
vector is ill-posed. Indeed, adding an infinitesimally small amount $\varepsilon>0$ 
(say,  one cent) to the initial endowments leads to a~unique clearing vector. Similar 
effect will have small increase in liabilities. One can think that the ``true'' 
liability matrix has all elements strictly positive and that in the model matrix, zero 
elements appeared because liabilities are neglected.
These phenomena are related to the ill-posedness of the spectral problem for 
stochastic matrices. Another question is which clearing vector is natural.


\smallskip


The above proof  is rather straightforward and uses graph-theoretical language.  
One can get another one  appealing to the contraction property of the mapping 
$f:p\mapsto (e+\Pi'p)\wedge \tilde L$ defined on the set $[0,\tilde L]$ equipped 
with $l_1$-distance $|p-\tilde p|_1$.

\smallskip

\noindent
\textbf{Proposition.}\
For every $p,\tilde p\in [0,\tilde L]$
\begin{equation*}
%\label{non-exp}
\left\vert f(p)-f(\tilde p)\right\vert_1\le \left\vert\Pi' (p-\tilde p)
\right\vert_1\le \left\vert p-\tilde p\right\vert_1\,.
\end{equation*}
Moreover, the first relation above is the equality if and only if the
union of subsets $A:=\{i:\ (\Pi'p)^i=(\Pi'\tilde p)^i\}$ and $B:=\{i:\ 
(\Pi'p)^i,(\Pi'\tilde p)^i\le \tilde L^i-e^i\}$ is the set of indices 
$\{1,\dots, N\}$.

\smallskip
%Moreover, if for each $i$ the sum $\sum_{j\in o(i)} e^i>0$, then the mapping 
%$f$ is a~contraction %on the set ${\rm Fix}_f$, i.e. the above inequality is 
%strict when the fixed points $p\neq \tilde p$.

\noindent
P\,r\,o\,o\,f\,.\ \ Using the elementary inequality $|a\wedge c-b\wedge c|$\linebreak $\le |a-b|$ 
which holds as  the
equality if and only if when\linebreak $a=b$ or $a,b\le c$, one obtains that
$|f(p)-f(\tilde p)|_1$\linebreak $\le |\Pi'p-\Pi'\tilde p|_1$
where the equality holds if and only if for every~$i$, one has 
$(\Pi'p)^i=(\Pi'\tilde p)^i$ or
$(\Pi'p)^i,(\Pi'\tilde p)^i$\linebreak $\le \tilde L^i-e^i$. Since $|\Pi'y|_1\le 
|\Pi'|_1|y|_1$ and $|\Pi'|_1=1$, one has the claim.~$\square$

\smallskip

Let us consider  the case where the matrix~$\Pi$ is irreducible. Suppose that 
${\bf 1}'e>0$ and~$p$ and~$\tilde p$ are two different fixed points of the 
mapping~$f$. According to above proposition,
$$
\sum\limits_{j\in B}\Pi^{ji}\left(p^j-\tilde p^j\right)=p^ i-\tilde p^i\,, \enskip i\in B\,.
$$
This means that  the nonzero vector with the coordinates $p^ i-\tilde p^i$, 
$i\in B$, is a~left eigenvector of the matrix
$(\Pi^{ij})_{i,j\in B}$ corresponding to unit eigenvalue. This is possible only 
if the latter matrix coincides with~$\Pi$. Thus, $p=f(p)=e+\Pi'p$. Since  
${\bf 1}'\Pi'p={\bf 1}'p$, one gets that ${\bf 1}'e=0$
which is a~contradiction.  Using the decomposition of the matrix~$\Pi$ on the 
irreducible component, one gets that  the clearing vector  is unique if for any 
irreducible component, there is a~node with strictly positive initial endowment.



\section{The Elsinger Model}

\noindent
In the present paper,  a~simplified version of the Elsinger model
introduced in~\cite{Elsinger2011}, where the interbank debts may be 
senior and junior, is considered. In this model, the system of~$N$ banks is described by the 
vector
of cash reserves and by~$M$~matrices $L_1=(L^{ij}_1), \ldots, L_M=(L^{ij}_M)$ 
representing the hierarchy of liabilities with decreasing seniority,  that is, 
the element~$L^{ij}_1$ represents the debt of the bank~$i$ to the bank~$j$ of the 
highest seniority, etc.,  $\sum_jL^{ij}_S$ is the total of  debts of the bank~$i$ 
of the seniority~$S$.

The relative liabilities are defined by  the matrix~$\Pi_S$ with
$$
\Pi_S^{ij}=\fr {L_S^{ij}}{\tilde L_S^i}=\fr {L_S^{ij}}{\sum\nolimits_j L_S^{ij}}\,.
$$
The clearing procedure requires the complete reimbursement of the debts starting 
from the highest priority and for each seniority level, the distribution is 
proportional
to the volume of debts of this seniority. For the bank~$i$, let us denote  by $p^i_S$ 
the value distributed to cover the debts of the seniority~$S$. So, the clearing 
can be described by the set of vectors~$p_S$, $S=1,\ldots, M$, which can be 
considered as a~``long'' vector from~$(\mathbb{R}^N)^M$  satisfying the system of 
equations:
\begin{equation*}
p_{1}^{i}=\min \left\{e^i+\sum\limits_S \sum\limits_j\Pi_S^{ji}p_{S}^{j}, \tilde L_1^i 
\right\}\,;
\end{equation*}
\begin{align*}
p_{S}^{i}&=\min\left\{\left(e^i+\sum\limits_S \sum\limits_j\Pi_S^{ji}p_{S}^{j}-
\sum\limits_{r<S}\tilde 
L_r^i\right)^+, \tilde L_S^i \right\}\,,  \\
&\hspace*{57mm}1<S\le M\,.
\end{align*}
In a~vector form, these equations can be written as follows:

\vspace*{-4pt}

\noindent
\begin{multline}
\label{SM}
p_{S}^{}=\left(e+\sum\limits_S \hspace*{-1.2pt}
\Pi_S'p_{S}-\sum\limits_{r<S}\hspace*{-1.2pt}\tilde L_r\right)^+\wedge  
\tilde  L_S\,,  \\ S=1,\ldots,M\,.
\end{multline}
It is clear that for the partial ordering in~$(\mathbb{R}^N)^M$ induced by the 
cone~$(\mathbb{R}^N_+)^M$, the function

\vspace*{-4pt}

\noindent
\begin{multline*}
\left(p_1,\ldots,p_M\right)\mapsto \left(
\left(e+\sum\limits_S \Pi_S'p_{S}^* \right)^+\wedge \tilde L_1 
,\ldots\right.\\
\left.\ldots,\left(e+\sum\limits_S \Pi_S'p_{S}^*-\sum\limits_{r<M}\tilde L_r\right)^+ 
\wedge L_M 
\right)
\end{multline*}
is a~monotone mapping of the order interval 
$$
[0,\tilde L_1] \times\cdots\times 
[0,\tilde L_M]\subset (\mathbb{R}^N)^M
$$ 
into itself.
 Thus, according to the Knaster--Tarski theorem, the set of fixed points of this 
mapping, i.\,e., the solutions of Eq.~(\ref{SM}), is nonempty and has the 
maximal and the minimal elements.

In the case of liabilities of different seniority after clearing by the vector 
$p\in (\mathbb{R}^N)^M$,  the equity vector $C\in \mathbb{R}^N$ has the form:
$$
C=\left(e+\sum\limits_S \Pi_S'p_{S}-\sum\limits_S \tilde L_S\right)^+\,.
$$

%\smallskip

\noindent
\textbf{Lemma~2.}\
\textit{The equity vector does not depend on the clearing vector}.

\vspace*{2pt}

\noindent
P\,r\,o\,o\,f\,.\ \  Note that
$$
\left(e+\sum\limits_S\Pi'_Sp_S\right)\wedge \sum\limits_S \tilde L^i_S=\sum\limits_S p_S\,.
$$
Therefore,
$$
\left(e+\sum\limits_S \Pi_S'p_{S}-\sum\limits_S \tilde L_S\right)^+=
e+\sum\limits_S \Pi_S'p_{S}-\sum\limits_S  p_{S}\,.
$$
With this identity, the reasoning is analogous to that with a~single seniority 
class.~$\square$

\vspace*{2pt}

The aim of this section is to provide a~sufficient condition for the uniqueness 
of clearing vector using a~specific graph structure induced by the matrices~$\Pi_S$.

For a~given clearing vector~$p$, let us define the \textit{default index}~$d^i$ of the 
node~$i$ as the smallest~$r$  such that
$$
\bar p_r^i=e^i+ \sum\limits_S \sum\limits_j\Pi_S^{ji}\bar p_{S}^j-\sum\limits_{r'< r}\tilde 
L_{r'}^{i}\,.
$$
In another words,~$d^i$ is the lowest seniority for which the bank equity after 
clearing is equal to zero. Define the matrix $\Delta=\Delta(p)$ by putting 
$$
\Delta^{ij}=
\begin{cases}
1 &\ \mbox{if\ \ } \Pi_{d(i)}^{ij}>0\,;\\
0 &\ \mbox{otherwise}.
\end{cases}
$$

%\columnbreak

\noindent
Let us use 
the notation $i\leadsto j$ if $\Delta^{ij}=1$ and  denote by $O(i)$ \textit{the 
$\Delta $-orbit of $i$} that is the set of all~$j$ for which there is 
a~directed path $i\leadsto i_1\leadsto i_2\leadsto\cdots\leadsto j$.

\vspace*{2pt}

\noindent
\textbf{Theorem~2.}\
\textit{Suppose that for the clearing vector $\bar p$, any $\Delta $-orbit is a~surplus 
set.
Then, the clearing vector is unique}.

\vspace*{2pt}

\noindent
P\,r\,o\,o\,f\,.\ \  By definition, the default index
$$
d^i:=\min\left\{r:\ \bar p_r^i=e^i+ \sum\limits_S \sum\limits_j\Pi_S^{ji}
\bar p_{S}^j-\sum\limits_{r'<  r}\tilde L_{r'}^{i}\right \}\,.
$$
It follows that $\bar p_r^i=0$; hence,  $\underline p_r^i=0$ for every $r>d^i$.
Suppose that
$\underline p_{d^i}^i<\bar p_{d^i}^i$ and consider a~path 
$$
i\leadsto  i_1\leadsto i_2\leadsto\cdots \leadsto m
$$ 
ending up at the node with strictly  positive equity value.

First, let us show that at least for one seniority $\underline p^{i_1}_S<\bar 
p^{i_1}_S$.

Let $r':=d^{i_1}$.  By definition, one has: 
$$
\bar p^{i_1}_r=\begin{cases}
\tilde L^{i_1}_r\,, & r\le r'\,;\\
0\,,  & r>r'\,.
\end{cases}
$$
 The claim 
holds, if  $\underline p^{i_1}_r<\tilde L^{i_1}_r$
for some $r<r'$. Thus, it remains to consider only the case where $\underline 
p^{i_1}_r=\bar p^{i_1}_r = \tilde L^{i_1}_r$
for all $r<r'$ and prove that  $\underline p^{i_1}_{r'}<\bar p^{i_1}_{r'}$.
One has the alternative: either $\underline p^{i_1}_{r'}<\bar p^{i_1}_{r'}\le  
\tilde L^{i_1}_r$ (what is needed), or
$\underline p^{i_1}_{r'}=\bar p^{i_1}_{r'}\le  \tilde L^{i_1}_r$. The second 
case is impossible, since the equalities

\noindent
\begin{align*}
\bar p^{i_1}_{r'}&=e^{i_1}+ \sum\limits_S \sum\limits_j\Pi_S^{ji_1}\bar p_{S}^j-
\sum\limits_{r<  r'}\tilde L_{r}^{i_1}\,;\\
\underline p^{i_1}_{r'}&=e^{i_1}+ \sum\limits_S 
\sum\limits_j\Pi_S^{ji_1}\underline p_{S}^j-
\sum\limits_{r< r'}\tilde L_{r}^{i_1}
\end{align*}
imply that

\noindent
\begin{multline*}
\bar p^{i_1}_{r'}-\underline p^{i_1}_{r'}=\sum\limits_S \sum\limits_j\Pi_S^{ji_1}
\left(\bar  p_{S}^j-\underline  p_{S}^j\right)\\
{}\ge \Pi_{d^i}^{ii_1}
\left(\bar p_{d^i}^i-\underline   p_{d^i}^i\right)>0\,.
\end{multline*}
This is a~contradiction.

\pagebreak

The above argument reduces the problem to the case $i\leadsto m$ and the node~$m$ 
has a~strictly positive equity.  The equity~$C^m$ does not depend on the 
clearing vector.  Therefore,

\noindent
\begin{align*}
C^m&=e^{m}+ \sum\limits_S \sum\limits_j\Pi_S^{jm}\bar p_{S}^j-
\sum\limits_{S}\tilde L_{S}^{m}\,;\\
C^m&=e^{m}+ \sum\limits_S \sum\limits_j\Pi_S^{jm}\underline p_{S}^j-
\sum\limits_{S}\tilde L_{S}^{m}\,.
\end{align*}


\noindent
It follows that
$$
0=\sum\limits_S \sum\limits_j\Pi_S^{jm}\left(\bar p_{S}^j-\underline p_{S}^j\right)\ge 
\Pi_{d^i}^{im}\left(\bar p_{d^i}^i-\underline  p_{d^i}^i\right)>0\,.
$$
This contradiction shows that $\underline p=\bar p$.

\subsection{Example~1}

\noindent
Let us consider the system consisting of~3~nodes with the initial cash 
endowments
given by the vector $e=(0.1,0,0)$ and the liability and the "distribution"  
matrices corresponding to senior and junior debts:
\begin{alignat*}{2}
L_S&=
\begin{pmatrix}
0 & 1 & 0\\
1 & 0 & 1\\
0 & 2 & 0
\end{pmatrix}\,; &\enskip
L_J&=\begin{pmatrix}
0 & 0 & 0\\
0& 0 & 2\\
0 & 0 & 0
\end{pmatrix}\,;
\\[9pt]
\Pi_S&=
\begin{pmatrix}
0 & 1 & 0\\
0.5 & 0 & 0.5\\
0 & 1 & 0
\end{pmatrix}\,; &\enskip
\Pi_J&=\begin{pmatrix}
0 & 0 & 0\\
0& 0 & 1\\
0 & 0 & 0
\end{pmatrix}.
\end{alignat*}
For this model, the vectors of total liabilities corresponding to the senior and 
junior debts are, respectively, $\tilde L_S=(1,2,2)$ and   $\tilde L_J=(0,2,0)$.

The equations for clearing vectors are:
\begin{align*}
p_S^1 & =  \left(0.1+0.5\, p_S^2 \right)\wedge 1\,;\\
p_S^2 & =  \left(p_S^1+p_S^3 \right)\wedge 2\,;\\
p_S^3 & =  \left(0.5\, p_S^2+p_J^2\right)\wedge 2\,;\\
p_J^1 & = 0\,;\\
p_J^2 & = \left(p_S^1+p_S^3-2\right)^+\wedge 2\,;\\
p_J^3 & = 0.
\end{align*}
It is not difficult to check that there are infinite set of clearing vectors.
Namely, one has that $p_S=(1,2,1+t)$ and $p_J=(0,t,0)$ where $t\in [0,1]$.
The minimal clearing vector corresponds to $t=0$ and the maximal corresponds to 
$t=1$.

\subsection{Example~2}

\noindent
The vector of cash endowments and the matrix of the senior debts  is the same as 
in Example~1. The junior debts matrix $L_J$ and the corresponding 
distribution matrix~$\Pi_J$ are now:
$$
L_J=\begin{pmatrix}
0 & 0 & 0\\
0.4& 0 & 1.6\\
0 & 0 & 0
\end{pmatrix}\,;
 \enskip
\Pi_J=\begin{pmatrix}
0 & 0 & 0\\
0.2& 0 & 0.8\\
0 & 0 & 0
\end{pmatrix}\,.
$$
We are looking for positive solutions of the following  equations:
\begin{align*}
p_S^1 & =  \left(0.1+0.5\, p_S^2 + 0.2\, p_J^2\right)\wedge 1\,;\\
p_S^2 & =  \left(p_S^1+p_S^3 \right)\wedge 2\,;\\
p_S^3 & =  \left(0.5\, p_S^2+0.8\, p_J^2\right)\wedge 2\,;\\
p_J^1 & = 0\,;\\
p_J^2 & = \left(p_S^1+p_S^3-2\right)^+\wedge 2\,;\\
p_J^3 & = 0\,.
\end{align*}
Note that $p_S^1\le 1$ and $p_S^2\le 2$; hence, $p_J^2\le 1$ and the 3rd equation 
is linear:
\begin{equation}
\label{pS3}
p_S^3  =  0.5\, p_S^2+0.8\, p_J^2.
\end{equation}
Substituting into the 2nd equation this expression for~$p_S^3$ and the 
expression for~$p_S^1$ from the 1st equation, one gets that
\begin{equation*}
p_S^2 \!=\!\left(\!\left(0.1+0.5\, p_S^2 + 0.2\, p_J^2\right)\wedge 1+
0.5\, p_S^2+0.8\, p_J^2 \right)\wedge 2.
\end{equation*}
The inequality $p_S^1< 1$ is impossible since in this case, $0.1+0.5\, p_S^2 + 
0.2\, p_J^2<1$, implying that
$$
p_S^2 =\left(0.1+p_S^2 + p_J^2\right)\wedge 2\,.
$$
For positive values of unknown variables, the last equality may hold only if  
$p_S^2=2$ but then, the 1st equation tells one that  $p_S^1=1$.

Thus, it was determined that $p_S^1=1$.

Combining the 2nd equation with~(\ref{pS3}), one obtains the equality
$$
p_S^2  =  \left(1+0.5\, p_S^2+0.8\, p_J^2\right)\wedge 2
$$
implying that $p_S^2=2$.

Available information allows one to reduce
the 5th equation to the simple one of the  form
$p_J^2  = 0.8\left(p_J^2\right)^+\wedge 2$ having the unique solution  $p_J^2=0$.

Summarizing, one gets that  $p_S=(1,2,1)$ and $p_J=(0,0,0)$.

\smallskip

\noindent
\textbf{Comment.} In the first example, the bank 1 has met all liabilities and 
finished with a~positive equity,  the bank~2 has payed the senior liabilities 
but defaulted on the junior debts, the bank~3 has defaulted already at the 
senior debts; and the 
bank~2 has no junior liabilities with the bank~1.  So, the $\Delta$-orbit of the 
banks~2 and~3 are not surplus sets and there are infinite many clearing vectors. 
In the second example, the bank~2 has a~junior debt to bank~1, 
all  $\Delta$-orbits are surplus sets, and the clearing vector is unique.


\section{Models with Illiquid Assets and~a~Price Impact}

\noindent
Let us consider the clearing problem without seniority structure where the bank~$i$ 
owns not only cash~$e^i$ but also~$K$~illiquid assets, in quantities 
$y^{i1},\dots y^{iK}$ represented in  the model by the row~$i$ of the matrix 
$Y=(y^{im})$, $i\le N$, $m\le K$. The nominal prices per unit  of illiquid 
assets are strictly positive  numbers $Q^1,\ldots,Q^K$.  The clearing might  
require their partial   sale  influencing   the market price. If the bank sells  
$u^{im}\in [0,y^{im}]$ units of the $m$th assets for the price~$q_m$, its 
total increase in cash is
$$
(Uq)^i=\sum\limits_{m=1}^K u^{im}q^m\,.
$$

\textbf{The price formation}  is modeled by the inverse demand function 
$F_0:\mathbb{R}^K\to \mathbb{R}^K$ assumed to be continuous and monotone 
decreasing ($F_0(z)\le F_0(x)$ when $z\ge x$ in the sense of partial ordering 
defined by~$\mathbb{R}^K_+$) and 
such that $F_0(0)=Q$ and $F^m_0(Y'{\bf 1})>0$ for $m=1,\ldots , K$.  The first 
condition means that in the absence
of supply, the prices are just the nominal prices while  the second one shows 
that even in the case of total sale, the prices of illiquid assets remain strictly 
positive.


\textbf{The clearing rules:} each bank pays  debts in accordance to the matrix of 
relative liabilities
and sells illiquid assets if it has insufficient amount of cash. The result of 
clearing should be: all
debts of the bank are covered or its equity falls down  to zero.



In the case of several illiquid assets,  there is a~problem how the banks chose 
their strategies of selling. In principle, one can imagine the situation that 
they have  full freedom and, acting in the noncooperative way, drop down the 
market of  illiquid assets because of an excessive supply. It seems reasonable 
that the central authority may  impose extra rules on selling illiquid assets. 
Let us suppose that this is done by prescribing that the bank~$i$ must sell all 
assets in the same proportion~$\alpha^{i}$:
\begin{equation*}
\alpha^i(q)=\fr{\left(\tilde L^i -e^i - \sum\nolimits_j\Pi^{ji}p^j \right)^+}
{ \sum\nolimits_k  y^{ik} q^k}\,\wedge 1\,,\enskip i=1,\dots, N\,.
\end{equation*}
This formula means that for a~fixed market price, the bank does not sell illiquid 
assets
if its  cash reserve together with collected debts covers the liabilities.
In the another extreme case where
$$
\tilde L^i -e^i - \sum\limits_j\Pi^{ji}p^j \ge \sum\limits_k y^{ik} q^k=(Yq)^i\,,
$$
all illiquid assets have to be sold and the bank defaults. In the intermediate
case, the bank sells a~share $\alpha^i\in ]0,1[$ of the $m$th asset adding to its 
cash an extra amount
$(({\tilde L^i -e^i - \sum\nolimits_j\Pi^{ji}p^j})/{\sum\nolimits_k y^{ik} 
q^k})\,y^{im}q_m$.
The total increase in cash allows to cover the liabilities.

Under such a~rule, the  $i$th bank sells~$u^{im}:=u^{im}(p,q)$ units of the $m$th asset where
\begin{equation*}
u^{im}
{}:=\fr{y^{im}\left(\tilde L^i -e^i - \sum\nolimits_j\Pi^{ji}p^j 
\right)^+}{ \sum\nolimits_k y^{ik} q^k}\,\wedge y^{im}.
\end{equation*}
The total supply of the illiquid assets is given by the vector ${\bf 1}'U(p,q)$ 
where
$U(p,q)$ is the matrix with entries given by the above formula.

Define the equilibrium vector 
$$
\left(p^*,q^*\right)\in \left[0,\tilde L\right] \times \left[ F_0(1Y),Q\right]
$$ 
as 
the solution of the system of $N+K$ equations written in the matrix form as
\begin{align}
\label{firstM}
p&=(e+U(p,q)q+\Pi'p)\wedge \tilde L\,;\\
\label{secondM}
q&=F_0(U'(p,q){\bf 1})\,.
\end{align}
The existence of the equilibrium is easy. Indeed,
check that 
\begin{gather*}
U'(p,q){\bf 1}\ge U'\left(\tilde p,\tilde q\right){\bf 1}\,;\\
U(p,q)q+\Pi'p\le  U\left(\tilde p,\tilde q\right)\tilde q+\Pi'\tilde p
\end{gather*}
when $(\tilde p,\tilde q)\ge (p,q)$. Denoting  $F(p,q)$ the right-hand side 
of the first equation, one obtains that  
$$
(p,q)\mapsto \left(F(p,q),F_0\left(U'(p,q)\right){\bf  1}\right)
$$ 
is a~monotone  mapping of the order interval $[0,\tilde L]\linebreak\times [ F_0(1Y),Q]$ into 
itself.  According to Knaster--Tarski theorem, the set of its fixed points is nonempty 
and contains the minimal and maximal elements $(\underline p^*, \underline q^*)$ 
and $(\bar p^*,\bar q^*)$.

For a~fixed $q$, the function $p\to F(p,q)$ is monotone. Thus, by the 
Knaster--Tarski theorem, the set of solutions of Eq.~(\ref{firstM}) is nonempty 
and contains, in particular, the maximal element~$\bar p(q)$.

For any fixed $q\in [F_0(Y),Q]$, the largest solution $\bar p=\bar p(q)$ 
of~(\ref{firstM}) is given by formula:
$$
\bar p=\sup\left\{p\in [0,\tilde L]:\ p\le \left(e+U(p,q)q+\Pi'p\right)\wedge \tilde L\right\}
$$
implying that $q\mapsto \bar p(q)$ is an increasing (and continuous) function on 
$[F_0(Y),Q]$.  It follows that the supply function
$$
q\mapsto \zeta(q):=U'(\bar p(q),q){\bf 1}
$$
is decreasing and, therefore, $q\mapsto F_0(\zeta(q))$ is an increasing 
(and continuous) mapping of the interval  $[F_0(Y),Q]$ into itself and, 
therefore, it has  the minimal and maximal fixed points that will be denoted by~$q_1$ 
and~$q_2$.

\smallskip

\noindent
\textbf{Lemma~3.}\
\textit{Suppose that the scalar function $x\to x'F_0(x)$ is strictly increasing on 
$[F_0(Y),Q]$. Then, the
solution of the equation  $q=F_0(\zeta(q))$ is unique, i.\,e.}, $q_1=q_2$.

\smallskip

\noindent
P\,r\,o\,o\,f\,.\ \
Arguing by contradiction, suppose that  $q_1\neq q_2$.     Since $q_1\le q_2$ 
and $\zeta(\cdot)$ is decreasing,   $\zeta(q_1)\ge \zeta(q_2)$. Moreover, 
$\zeta(q_1)\neq \zeta(q_2)$ as the values of~$F_0$ at these points are~$q_1$ 
and~$q_2$.
 The assumed strict monotonicity  implies that
 $$
 \zeta'(q_1)F_0( \zeta(q_1))> \zeta'(q_2)F_0( \zeta(q_2)).
 $$
It follows that
$$
\zeta'\left(q_1\right) q_1> \zeta'\left(q_2\right)q_2\,.
 $$
To get a~contradiction, it is sufficient to show that
$$
\Delta:= \zeta'\left(q_2\right)q_2-\zeta'\left(q_1\right)q_1\ge 0\,.
$$
Let $\bar p_k:=\bar p(q_k)$ and let
$$
D_k:=\left\{i:\ \left(\tilde L-e-\Pi'\bar p\left(q_k\right)\right)^i\ge 
\left(Yq_k\right)^i\right\}\,,
$$
i.\,e., $D_k$ is the set of banks that are forced to sell all their illiquid assets 
for the price~$q_k$, $k=1,2$. Since~$\bar p(\cdot)$ is increasing, $D_2\subseteq D_1$.  
With the 
notation~${\bf 1}'_{A}$ for the row-vector representing the indicator function
of the subset $A\subseteq \{1,\dots, N\}$, one has, taking into account that 
$a^+=a+a^-$:
\begin{multline*}
\zeta'\left(q_k\right)q_k={\bf 1}'_{D_k}Yq_k\\
{}+{\bf 1}'_{D_k^c}\left(\tilde L-e-\Pi'\bar 
p_k\right)+{\bf 1}'_{D_k^c}\left(\tilde L-e-\Pi'\bar p_k\right)^-.
\end{multline*}
This formula leads to the representation:
\begin{multline*}
\Delta={\bf 1}'_{D_2}Y(q_2-q_1)-{\bf 1}'_{D_1\setminus D_2}Yq_1\\
{}- {\bf 1}'_{D_1^c}
\Pi'\left(\bar p_2-  \bar p_1\right)+{\bf 1}'_{D_2^c\setminus D_1^c}
\left(\tilde L-e -\Pi'\bar p_2\right)\\
{}+ {\bf 1}'_{D_1^c}\left(\left(\tilde L-e -\Pi'\bar p_2\right)^- -
\left(\tilde L-e -\Pi'\bar p_1\right)^-\right)\\
+
{\bf 1}'_{D_2^c\setminus D_1^c}\left(\tilde L-e -\Pi'\bar p_2\right)^-.
\end{multline*}
Since the function $x\to x^-$ (on ${\mathbb{R}}^N$) is positive and decreasing, the 
last two terms in the right-hand side are positive. Regrouping  the third and 
the forth  terms, one gets that
\begin{multline}
\label{ineq1}
\Delta\ge{\bf 1}'_{D_2}Y\left(q_2-q_1\right)-{\bf 1}'_{D_1\setminus D_2}q_1Y
- {\bf 1}'_{D_2^c}\Pi'(\bar p_2-
 \bar p_1)\\
 {}+{\bf 1}'_{D_1\setminus D_2}\left(\tilde L-e -\Pi'\bar p_1\right)\,.
\end{multline}
From Eq.~(\ref{firstM}), it follows that
\begin{multline*}
{\bf 1}'\Pi'\left(\bar p_2-  \bar p_1\right)=
{\bf 1}'\left(\bar p_2-  \bar p_1\right)={\bf 1}'_{D_1}\left(\bar p_2-  \bar p_1\right)\\
{}={\bf 1}'_{D_2}\left(q_2u\left(\bar p_2,q_2\right)-q_1u
\left(\bar p_1,q_1\right)+\Pi'\left(\bar p_2-  \bar p_1\right)\right)\\
{}+{\bf 1}'_{D_1\setminus D_2}\left(\tilde L -\left(e+q_1u\left(\bar p_1,q_1
\right) +\Pi'\bar p_1\right)\right)
\end{multline*}
implying that

\columnbreak

\noindent
\begin{multline*}
 {\bf 1}'_{D_2^c}\Pi'\left(\bar p_2-
 \bar p_1\right)={\bf 1}'_{D_2}\left(U\left(\bar p_2,q_2\right)q_2\right.\\
 \left.{}-
 U\left(\bar p_1,q_1\right)q_1\right)-{\bf  1}'_{D_1\setminus D_2}
 U\left(\bar p_1,q_1\right)q_1\\
{}+{\bf 1}'_{D_1\setminus D_2}\left(\tilde L-e -\Pi'\bar p_1\right)\,.
\end{multline*}
Substituting this expression in~(\ref{ineq1}), one has:
\begin{multline*}
\Delta\ge{\bf 1}'_{D_2}Y\left(q_2-q_1\right)-{\bf 1}'_{D_1\setminus D_2}Yq_1\\
{}-{\bf 1}'_{D_2}\left(U\left(
\bar p_2,q_2\right)q_2-U\left(\bar p_1,q_1\right)q_1\right)\\
{}+
{\bf 1}'_{D_1\setminus D_2}q_1u\left(\bar p_1,q_1\right)=0
\end{multline*}
since the cash increment $(U(\bar p_2,q_2)q_2)^i=(Yq)^i$ for the bank $i\in D_2$ 
and $(U(\bar p_1,q_1)q_1)^i=(Yq_1)^i$ for $i\in D_1\supseteq D_2$.~$\square$


\smallskip

\noindent
\textbf{Theorem~3.}\
\textit{Suppose that the scalar function $x\to x'F_0(x)$ is strictly increasing on 
$[F_0(Y),Q]$. Then, there is $q^*$ such that  the set of solutions of the 
system}~(\ref{firstM}),  (\ref{secondM})
\textit{is contained in the interval  with the extremities $(\underline p(q^*),q^*)$ and 
$(\bar p(q^*),q^*)$.
In particular, if for each~$q$ the solution of}~(\ref{firstM}) \textit{is unique, then 
the solution of the system is also unique}.

\smallskip

\noindent
P\,r\,o\,o\,f\,.\ \ 
Let~$\Gamma$ be the set of~$q$ for which $(p,q)$ is a~solution  of 
the system~(\ref{firstM}),  (\ref{secondM}). If $q^*\in \Gamma$, then $(\bar 
p(q^*),q^*)$
is the solution of~(\ref{firstM}),  (\ref{secondM}). According to  
lemma~3, the point~$q^*$ is uniquely defined. This implies the result.~$\square$

\smallskip

Note that the uniqueness of the solution of~(\ref{firstM}) is guarantied if  for 
each~$i$,
the orbit of~$i$ contains an element with positive cash reserve.

\smallskip

\noindent
\textbf{Remark~2.}
In~\cite{AFM},  it was considered  a~model coinciding with studied 
above
in the case of a~single illiquid asset. The difference is that in the cited 
paper, the equilibrium is defined  as a~vector $(p,q)$ satisfying the
system of equations:
\begin{align}
\label{firstAFM}
p&=\left(e+qy+ \Pi'p\right)^+\wedge \tilde L\,; \\
%\label{secondAFM}
q&=F_0\left({\bf 1}'\left(\left(q^{-1}
\left(\tilde L-e-\Pi'p\right)^+\right)\wedge y\right)\right).\notag
\end{align}
To our opinion, the definition of the equilibrium given 
by the system~(\ref{firstM}), 
(\ref{secondM}), which is in the one liquid asset case has the  form:
\begin{align}
p&=\left(e+\left(\tilde L-e-\Pi'p\right)^+\wedge (qy)+ 
\Pi'p\right)\wedge \tilde L\,; \label{firstAFM1}
\\
%\label{secondAFM1}
q&=F_0\left({\bf 1}'\left(\left(q^{-1}\left(\tilde L-e-\Pi'p\right)^+
\right)\wedge y\right)\right), \notag
\end{align}
 is more natural.  In fact, the   right-hand sides of~(\ref{firstAFM}) 
 and~(\ref{firstAFM1}) as functions $R_1(p,q)$ and $R_2(p,q)$ defined
 on $[0,\tilde L]\times [ F_0(1Y),Q]$ coincide.  To see this, fix~$i$ and  
consider the three possible cases.
\begin{enumerate}[1.]
\item  Let  $e^i+qy+ (\Pi'p)^i\le \tilde L^i$. Then, the expressions for 
$R^i_1(p,q)$ and $R^i_2(p,q)$ have the same form  $e^i+qy+ (\Pi'p)^i$.

\item Let $e^i+qy+ (\Pi'p)^i> \tilde L^i$ and $\tilde L^i-e^i - (\Pi'p)^i\ge 0$. 
Then, the values $R^i_1(p,q)$ and $R^i_2(p,q)$ are equal to~$\tilde L^i$.

\item Let $e^i+qy+ (\Pi'p)^i> \tilde L^i$ and $\tilde L^i-e^i - (\Pi'p)^i<0$. 
Then, the value of $R^i_1(p,q)$ is $\tilde L^i$ and the value of $R^2_1(p,q)$ is 
$(e^i + (\Pi'p)^i)\wedge \tilde L^i=\tilde L^i$.
\end{enumerate}

\vspace*{-18pt}


{\small
\section*{\raggedleft Appendix}

%\vspace*{-6pt}

\subsection*{Knaster--Tarski Fixpoint Theorem}
%\label{app}

\noindent
Let $X$ be a~set with a~partial ordering~$\ge$ and let~$A$ be its nonempty 
subset.
By definition,~$\sup A$ is an element~$\bar x$ such that $\bar x\ge x$ for all 
$x\in A$ and if~$\bar x'$ is such that  $\bar x'\ge x$ for all $x\in A$, then 
$\bar x'\ge \bar x$. The definition of~$\inf A$ follows the same pattern but 
with the dual ordering~$\le$.  A~partially ordered set~$X$ is  {\it complete 
lattice} if for any its nonempty subset~$A$,
there exist~$\inf A$ and~$\sup A$.

\smallskip

\noindent
\textbf{Theorem~4.}\
\textit{Let $X$ be a~complete lattice and let $f : X \mapsto X$ be an order-preserving 
mapping, $L:=\{x:\  f(x)\le x\}$, $U:=\{x:\ f(x)\ge x\}$.   The set
$L\cap U$ of fixed points of~$f$
is nonempty and has the smallest and the largest fixed points  which are, 
respectively, $\underline x:=\inf L$ and}   $\bar x:=\sup U$.

\smallskip

 \noindent
P\,r\,o\,o\,f\,.\ \  
Note that $L\neq \emptyset$ since it contains the element~$\sup X$.
Take arbitrary $x\in L$. Then, $\underline x\le x$
implying that $f(\underline x)\le f(x) \le x$. Thus, $f(\underline x)\le 
\underline x$ as~$\underline x$ is~$\inf L$. So, $\underline x\in L$. 
Since $f(L)\subseteq L$, 
also $f(\underline x)\in L$; hence,  $\underline x\le f(\underline x)$, i.\,e., 
$\underline x= f(\underline x)$. All fixed points belong to~$L$ and, 
therefore,~$\underline x$ is the smallest one.

The proof of the statement for the largest fixed point is analogous.~$\square$

\smallskip

 \noindent
 \textbf{Corollary.}\
\textit{Let $f(\cdot;y)$ be an order-preserving mapping of a~complete lattice $(X,\ge)$ into 
itself, depending on the parameter~$y$ from a~partially ordered set 
$(Y,\succeq)$.
Suppose that $f(\cdot,y)$ is increasing in~$y$, that is, $f(x,y')\ge f(x,y)$ for all 
$x\in X$ when $y'\succeq y$.  Then, the smallest and the largest fixed points are 
also increasing in}~$y$.

\smallskip

\noindent
P\,r\,o\,o\,f\,.\ \ The claim is obvious because the sets   
$$
L(y):=\{x:\  f(x,y)\le x\}
$$ 
are decreasing and the sets 
$$
U(y):=\{x:\ f(x,y)\ge x\}$$ 
are increasing in~$y$
(see~\cite{Milgrom-Roberts}).

These general results are applied to the order intervals $[a,b]\subset \mathbb{R}^d$
with the ordering induced by~$\mathbb{R}^d_+$.

}

\vspace*{-6pt}

\Ack
\noindent
The 
research of Yuri Kabanov was done under partial financial support   of the grant 
of  the Russian Science Foundation No.\,14-49-00079.


\renewcommand{\bibname}{\protect\rmfamily References}

\vspace*{-6pt}

{\small\frenchspacing
{%\baselineskip=10.8pt
\begin{thebibliography}{9}

\bibitem{Eisenberg-Noe} %1
\Aue{Eisenberg, L., and T.\,H.~Noe}. 2001. Systemic risk in financial systems. 
\textit{Manag. Sci.} 47(2):236--249.

\bibitem{Suzuki} %2
\Aue{Suzuki, T.} 2002. Valuing corporate debt: The effect of cross-holdings of stock 
and debt. \textit{J.~Oper. Res. Soc. Japan} 45(2):123--144.

\bibitem{Tarski} %3
\Aue{Tarski, A.} 1955. A~lattice-theoretical fixpoint theorem and its applications. 
\textit{Pacific J.~Math.} 5(2):285--309.


\bibitem{Cont-Wag} %4
\Aue{Cont, R., and L.~Wagalath}. 2015. Fire sale forensics: Measuring endogenous risk. 
\textit{Math. Finance} 26:835--866. %doi: 10.1111/mafi.12071.

\bibitem{AFM} %5
\Aue{Amini, H., D.~Filipovi$\acute{\mbox{c}}$, and A.~Minca.} 2015. To fully net or not to net: Adverse 
effects of partial multilateral netting. %Swiss Finance Institute Research Paper  series. No.~14-63. Forthcoming in ``
\textit{Oper. Res.} 64(5):1135--1142.

\bibitem{Elsinger2011} %6
\Aue{Elsinger, H.} 2009. Financial networks, cross holdings, and limited liability. 
Working paper from Oesterreichische Nationalbank.

\bibitem{Milgrom-Roberts} %7
\Aue{Milgrom, J., and J.~Roberts.} 1994. Comparing equilibria. 
\textit{Am. Econ. Rev.}  84:441--454.




\end{thebibliography} } }

\end{multicols}

\vspace*{-6pt}

\hfill{\small\textit{Received September 25, 2016}}

\vspace*{-18pt}

\Contr

%\vspace*{-3pt}

\noindent
\textbf{El Bitar  Khalil} (b.\ 1981)~--- 
PhD student, Laboratoire de Mathematiques, Universite de Franche-Comte, 
16~Route de Gray, 25030, \mbox{Besan{\!\ptb{\c{c}}}on}, CEDEX, France; 
\mbox{khalilbitar\_aw@hotmail.com}  

 \vspace*{1pt}
 
 \noindent
 \textbf{Kabanov Yuri M.} (b.\ 1948)~---
  professor, Laboratoire de Mathematiques, Universite de Franche-Comte, 
  16~Route de Gray, 25030, Besancon, CEDEX, France; leading scientist, 
  Institute of Informatics Problems, Federal Research Center 
  ``Computer Science and Control'' of the Russian Academy of Sciences,  
  44-2~Vavilov Str., Moscow 119333, Russian Federation; 
  National Research University ``MPEI,'' 14~Krasnokazarmennaya Str., 
  Moscow 111250, Russian Federation; \mbox{Youri.Kabanov@univ-fcomte.fr} 

\vspace*{1pt}
 
 \noindent
 \textbf{Mokbel Rita} (b.\ 1981)~--- 
 PhD student, Laboratoire de Mathematiques, Universite de Franche-Comte, 
 16~Route de Gray, 25030, Besancon, CEDEX, France; \mbox{ritamokbel@hotmail.com}




%\vspace*{8pt}

%\hrule

%\vspace*{2pt}

%\hrule

\newpage

\vspace*{-24pt}



\def\tit{О~ЕДИНСТВЕННОСТИ КЛИРИНГОВЫХ ВЕКТОРОВ, РЕДУЦИРУЮЩИХ 
СИСТЕМНЫЙ РИСК$^*$}

\def\aut{Х.~Эль Битар$^1$, Ю.~Кабанов$^{1,2,3}$, Р.~Мокбель$^1$}


\def\titkol{О~единственности клиринговых векторов, редуцирующих 
системный риск}

\def\autkol{Х.~Эль Битар, Ю.~Кабанов, Р.~Мокбель}

{\renewcommand{\thefootnote}{\fnsymbol{footnote}}
\footnotetext[1]{Представленные в настоящей статье результаты исследований, проведенных 
Ю.\,М.~Кабановым, были получены при частичной финансовой поддержке 
Российского научного фонда (проект №\,14-49-00079).}}


\titel{\tit}{\aut}{\autkol}{\titkol}

\vspace*{-12pt}

\noindent
$^1$Лаборатория математики Университета Франш-Кон\-те, г.~Безансон, Франция

\noindent
$^2$Институт проблем информатики Федерального исследовательского
центра <<Информатика и~управление>>\linebreak
$\hphantom{^1}$Российской академии наук, Российский
университет дружбы народов

\noindent
$^3$Национальный исследовательский университет <<МЭИ>>

\vspace*{6pt}

\def\leftfootline{\small{\textbf{\thepage}
\hfill ИНФОРМАТИКА И ЕЁ ПРИМЕНЕНИЯ\ \ \ том\ 11\ \ \ выпуск\ 1\ \ \ 2017}
}%
 \def\rightfootline{\small{ИНФОРМАТИКА И ЕЁ ПРИМЕНЕНИЯ\ \ \ том\ 11\ \ \ выпуск\ 1\ \ \ 2017
\hfill \textbf{\thepage}}}


\Abst{В~финансовых системах, т.\,е.\ в сети взаимосвязанных банков, 
процедура взаимозачета, или клиринга, состоит в~одновременной выплате 
задолженностей с~целью уменьшения общей их суммы в~системе. Вектор, компоненты 
которого есть суммарные выплаты каждого банка системы, называется клиринговым 
вектором. В~простых моделях, предложенных Айзенбергом и Ноэ (2001) и~независимо 
Судзуки (2002) было показано, что полный клиринг описывается вектором, который 
является неподвижной точкой некоторого отображения. Существование клирингового 
вектора может быть получено прямой ссылкой на теоремы о~неподвижной точке 
Кнас\-те\-ра--Тар\-скo\-го или Брауэра. Вопрос о~его единственности является более 
деликатным. Айзенберг и Ноэ получили достаточное условие единственности 
в~терминах графа связей финансовой системы. В~настоящей работе доказывается 
единственность для двух более общих моделей: модели Эльсингера с~приоритетами 
долгов и~модели типа Ами\-ни--Фи\-ли\-по\-ви\-ча--Мин\-ки, 
в~которой банки имеют неликвидные 
активы, продажа которых влияет на их рыночную цену.}

\KW{системный риск; финансовые сети; клиринг; теорема 
Кнас\-те\-ра--Тар\-ско\-го; модель Ай\-зен\-бер\-га--Ноэ; приоритет финансовых обязательств; 
влияние на ценообразование}



\DOI{10.14357/19922264170110}

%\vspace*{6pt}


 \begin{multicols}{2}

\renewcommand{\bibname}{\protect\rmfamily Литература}
%\renewcommand{\bibname}{\large\protect\rm References}

{\small\frenchspacing
{%\baselineskip=10.8pt
\begin{thebibliography}{9}
\bibitem{3-kab} %1
\Au{Eisenberg L., Noe~T.\,H.} Systemic risk in financial systems~// 
Manag. Sci., 2001. Vol.~47. No.\,2. P.~236--249.
\bibitem{6-kab} %2
\Au{Suzuki T.} Valuing corporate debt: The effect of cross-holdings of stock and debt~// 
J.~Oper. Res. Soc. Japan, 2002. Vol.~45. No.\,2. P.~123--144.
\bibitem{7-kab} %3
\Au{Tarski A.} A~lattice-theoretical fixpoint theorem and its applications~// 
Pacific J.~Math., 1955. Vol.~5. No.\,2. P.~285--309.

\bibitem{2-kab} %4
\Au{Cont R., Wagalath~L.} Fire sale forensics: Measuring endogenous risk~// 
Math.  Finance, 2015. Vol.~26. P.~835--866. %doi: 10.1111/mafi.12071.
\bibitem{1-kab} %5
\Au{Amini H., Filipovi$\acute{\mbox{c}}$~D., Minca~A.} To fully net or not to net: 
Adverse effects of partial multilateral netting~// Oper. Res., 2015. Vol.~62.
No.\,5. P.~1135--1142.

\bibitem{4-kab} %6
\Au{Elsinger H.} Financial networks, cross holdings, and limited liability. 
Working paper from Oesterreichische Nationalbank, 2009.
\bibitem{5-kab} %7
\Au{Milgrom J., Roberts~J.} Comparing equilibria~// Am. Econ. Rev., 1994. 
Vol.~84. P.~441--454.


\end{thebibliography}
} }

\end{multicols}

 \label{end\stat}

 \vspace*{-3pt}

\hfill{\small\textit{Поступила в редакцию  25.09.2016}}
%\renewcommand{\bibname}{\protect\rm Литература}
\renewcommand{\figurename}{\protect\bf Рис.}
\renewcommand{\tablename}{\protect\bf Таблица}    %2+
\def\stat{kondranin+ushakov}

\def\tit{СИСТЕМА ОБСЛУЖИВАНИЯ С~ОТНОСИТЕЛЬНЫМ ПРИОРИТЕТОМ  И~ПРОФИЛАКТИКАМИ ПРИБОРА$^*$}

\def\titkol{Система обслуживания с~относительным приоритетом  и~профилактиками прибора}

\def\aut{Е.\,С.~Кондранин$^1$,  В.\,Г.~Ушаков$^2$}

\def\autkol{Е.\,С.~Кондранин,  В.\,Г.~Ушаков}

\titel{\tit}{\aut}{\autkol}{\titkol}

\index{Кондранин Е.\,С.}
\index{Ушаков В.\,Г.}
\index{Kondranin E.\,S.}
\index{Ushakov V.\,G.}




{\renewcommand{\thefootnote}{\fnsymbol{footnote}} \footnotetext[1]
{Работа выполнена при финансовой поддержке РФФИ (проект 18-07-00678).}}


\renewcommand{\thefootnote}{\arabic{footnote}}
\footnotetext[1]{Факультет вычислительной математики и~кибернетики Московского государственного 
университета им.\ М.\,В.~Ломоносова, \mbox{ekondranin@yandex.ru}}
\footnotetext[2]{Факультет вычислительной математики и~кибернетики
Московского государственного университета им.\ М.\,В.~Ломоносова;
Институт проб\-лем информатики Федерального исследовательского
центра <<Информатика и~управ\-ле\-ние>> Российской академии наук,
\mbox{vgushakov@mail.ru}}

\vspace*{-10pt}




\Abst{Изучена одноканальная система
массового обслуживания с~двумя типами требований, бесконечным
числом мест для ожидания, гиперэкспоненциальным входящим потоком 
и~профилактиками обслуживающего прибора при освобождении системы.
Тип  требования определяется случайно с~заданными вероятностями 
в~момент его поступления в~систему обслуживания. Требования первого
типа имеют относительный приоритет перед требованиями второго
типа. Найдено нестационарное совместное распределение числа
требований каждого типа в~системе. Профилактики прибора
заключаются в~том, что в~момент освобождения системы от требований
прибор на случайное время с~заданным распределением становится
недоступным для обслуживания. Если за время профилактики поступает
хотя бы одно требование, то начинается нормальное функционирование
системы. Если требования не поступают, то прибор отправляется на
новую профилактику. Такие системы хорошо описывают
функционирование большого числа реальных вычислительных и~информационных систем.}

\KW{гиперэкспоненциальный поток; профилактики
обслуживающего прибора; одноканальная система; относительный
приоритет; длина очереди}

\DOI{10.14357/19922264180405}
  
%\vspace*{4pt}


\vskip 10pt plus 9pt minus 6pt

\thispagestyle{headings}

\begin{multicols}{2}

\label{st\stat}

\section{Введение}

В классической системе массового обслуживания ожидание требований
в очереди связано только с~занятостью обслуживающего прибора. В~то
же время в~реальных системах сам  прибор может пребывать как 
в~активном, так и~в~неактивном состоянии. Такое неактивное
состояние прибора (в~литературе на английском языке используется
термин vacation, а~на русском~--- профилактика или прогулка) может
быть связано со многими причинами. В~част\-ности, сис\-те\-мы
обслуживания с~профилактиками прибора хорошо описывают
функционирование  реальных вычислительных и~информационных систем,
в которых наряду с~основными требованиями имеются второстепенные.
Второстепенные требования всегда присутствуют в~сис\-те\-ме, а~их
обслуживание может проводиться только тогда, когда нет основных,
т.\,е.\ в~фоновом режиме.

С точки зрения самого процесса профилактики прибора существует
несколько ее разновидностей. Во-пер\-вых, могут быть разными
правила, задающие условия начала профилактики: прибор может брать
перерыв только при  полном исчерпании требований в~очереди
(exhaustive service) либо при наличии определенного их числа
(nonexhaustive service). Во-вто\-рых, могут быть разными правила
возвращения прибора в~работу. С~этой точки зрения различают случаи
однократного (single vacation) и~многократного (multiple vacation)
перерыва в~работе. В~первом случае ушедший на профилактику прибор
после ее окончания находится в~рабочем состоянии независимо от
наличия требований в~системе. Во втором случае прибор, не
обнаружив новых требований в~очереди, уходит на новую
профилактику.


В работах~[1--4] можно найти обзор известных результатов, большое
число постановок задач, описание различных приложений и~обширную
библиографию по анализу систем с~профилактиками обслуживающего
прибора.


В настоящей работе исследуется совместное распределение длин
очередей в~нестационарном режиме в~однолинейной системе 
с~ожиданием, гиперэкспоненциальным входящим потоком, двумя типами
требований и~относительным приоритетом. Аналогичная неприоритетная
система обслуживания исследована в~[5].

\vspace*{-6pt}

\section{Описание модели}

Рассматривается однолинейная система массового обслуживания 
с~двумя приоритетными классами требований. Входящий поток~---
гиперэкспоненциальный с~функцией распределения интервалов между
поступлениями требований вида:
\begin{multline*}
A(t)=\sum\limits_{i=1}^kc_i\left(1-e^{-a_it}\right),\enskip t>0,\enskip
a_i>0,\enskip c_i>0,\\
a_i\ne a_j\,,\enskip i\ne j\,,\enskip  \sum\limits_{i=1}^k c_i=1\,.
\end{multline*}

Каждое поступившее требование направляется в~первый класс 
с~вероятностью~$p$ и~во второй класс с~вероятностью $1\hm-p$
независимо от остальных требований. Требования первого класса
обладают относительным приоритетом перед требованиями второго
класса. Длительности обслуживания требований $i$-го приоритетного
класса~--- независимые в~совокупности и~не зависящие от входящего
потока случайные величины с~функцией распределения~$B_i(x)$,
$i\hm=1,2.$
 Если в~некоторый момент времени система освободилась от требований, 
 то обслуживающий прибор
 отправляется на профилактику, которая длится случайное время с~функцией 
 распределения~$C(x).$
 Не ограничивая общности, будем считать, что $B_i(x)\hm<1$
 и~$C(x)\hm<1$  для любого~$x$ 
 и~существуют плотности
 распределения~$b_i(x)$ и~$c(x).$
  Обозначим:
$$
 \beta_i(s)=\int\limits_0^{\infty}e^{-sx}b_i(x)\,dx\,;\enskip 
  \gamma(s)=\int\limits_0^{\infty}e^{-sx}c(x)\,dx\,.
$$
Пока прибор находится на профилактике, он не доступен для
обслуживания. Если за время профилактики поступают требования,
после ее завершения начинается их обслуживание. Если ни одно
требование не поступает, то прибор отправляется на новую
профилактику. Длительности различных профилактик являются
независимыми случайными величинами 
и~не зависят от входящего потока и~времен обслуживания.

\section{Вспомогательные результаты}

  Рассмотрим многочлен по $\mu$ степени $k$ вида:
\begin{multline}
\label{1}
\prod\limits_{i=1}^k\left(\mu+a_i\right)-{}\\
{}-
\left(pz_1+(1-p)z_2\right)\sum\limits_{j=1}^kc_ja_j\prod\limits_{i\ne
j}\left(\mu+a_i\right)\,.
\end{multline}
Занумеруем его корни $\mu_1(z_1,z_2),\ldots,\mu_k(z_1,z_2)$ таким образом,
чтобы они были непрерывными функциями и~$\mu_1(1,1)\hm=0.$ Тогда
$\mathrm{Re}\, \mu_j\left(z_1,z_2\right)\hm<0$, $|z_1|\hm<1$, 
$|z_2|\hm<1,$ $\mu_i(z_1,z_2)\hm\ne \mu_j(z_1,z_2),$ $ i\hm\ne j$,
$j\hm=1,\ldots,k.$ Обозначим:
$$
\alpha_m(z_1,z_2)=\prod\limits_{j\ne m}\left(\mu_m\left(z_1,z_2\right)-
\mu_j\left(z_1,z_2\right)\right)\,.
$$
Справедливы следующие леммы.

\smallskip

\noindent
\textbf{Лемма~1.}\
\textit{Для любого $l=1,\ldots,\:k$ система уравнений}
$$
z_j=\beta_j(s-\mu_l(z_1,z_2)),\ \ j=1,2,
$$
\textit{имеет единственное решение $z_i=z_{il}(s)$ такое, 
что $|z_{il}(s)|\hm<1$ при $l\hm=2,\ldots, k,$ $\mathrm{Re}\, s\hm\geqslant 0,$ 
а~$z_{i1}(0)\hm=1$, $|z_{i1}(s)|\hm<1$ при} $\mathrm{Re}\, s\hm> 0$, $i\hm=1,2.$

\smallskip

\noindent
\textbf{Лемма~2.}\
\textit{При каждом $l\hm=1,\ldots,k$ уравнение}
$$
z_1=\beta_1\left(s-\mu_l(z_1,z_2)\right)
$$
\textit{имеет единственное решение $z_1\hm=z_{1l}(z_2,s),$ 
аналитическое в~области $\mathrm{Re}\, s\hm>0$, $|z_2|\hm<1.$
}

\smallskip

Положим
$$
\lambda_l(s)=\mu_l\left(z_{1l}(s),z_{2l}(s)\right)\,.
$$




\section{Распределение длины очереди}

  Гиперэкспоненциальный поток можно рас\-смат\-ри\-вать как
пуассоновский поток со случайной интен\-сив\-ностью~$a,$ которая
принимает $k$ различных значений $a_1,\ldots,a_k$  с~вероятностями
$c_1,\ldots,c_k.$ Текущее значение~$a$ разыгрывается в~момент
поступления требования и~не меняется между двумя соседними
поступлениями. Введем случайный процесс~$j(t)$ такой, что если
$a\hm=a_j$ в~момент времени $t,$ то $j(t)\hm=j.$

Целью работы является нахождение распределения случайного процесса
$\left(L_1(t),L_2(t)\right),$ где $L_i(t)$~--- число требований из
$i$-го приоритетного класса, находящихся в~системе в~момент
времени~$t.$

При сделанных предположениях относительно параметров изучаемой
системы обслуживания\linebreak процесс $\left(L_1(t),L_2(t)\right)$ не
является, вообще говоря, марковским. Пусть $i(t)=i$, $i\hm=1,2,$ если
в~момент времени~$t$ обслуживается требование из $i$-го
приоритетного класса, и~$i(t)\hm=0,$ если в~момент времени~$t$ прибор
находится на профилактике. Случайный процесс~$x(t)$ определим
следующим образом. Если $i(t)\hm\ne 0,$ то $x(t)$ есть
время, прошедшее с~начала обслуживания требования, находящегося на
приборе, до момента~$t.$ Если $i(t)\hm=0,$ то $x(t)$ есть время,
прошедшее с~начала профилактики прибора до момента~$t.$ Случайный
процесс $\left(L_1(t),L_2(t),i(t),j(t),x(t)\right)$ является
однородным марковским процессом. Положим
\begin{multline*}
P_{ij}(n_1,n_2,x,t)=\fr{\partial}{\partial x}
\mathbf{P}\left(L_1(t)=n_1,L_2(t)=n_2,\right.\\
\left. i(t)=i,j(t)=j,x(t)<x
\vphantom{L_1}\right)\,,\enskip 
 x\geqslant 0,\\ 
 j=1,\ldots,k,\enskip i=0,1,2;
\end{multline*}
\begin{gather*}
\eta_i(x)=\fr{b_i(x)}{1-B_i(x)},\ i=1,2;\enskip 
\eta_0(x)=\fr{c(x)}{1-C(x)}\,;\\
\delta_{i,j}=\begin{cases}
1,&\ i=j;\\ 
0,&\ i\ne j\,.
\end{cases}
\end{gather*}
Функции $P_{ij}(n_1,n_2,x,t)$  удовлетворяют при $x\hm>0$
системам дифференциальных уравнений:
\begin{multline}
\label{3}
\fr{\partial P_{ij}(n_1,n_2,x,t)}{\partial t}+\fr{\partial
P_{ij}(n_1,n_2,x,t)}{\partial
x}={}\\
{}=-(a_j+\eta_i(x))P_{ij}(n_1,n_2,x,t)+ {}\\
{}+
c_j\sum\limits_{l=1}^ka_l\left(p\:P_{il}(n_1-1,n_2,x,t)+{}\right.\\
\left.{}+
(1-p)P_{il}(n_1,n_2-1,x,t)\right)
\end{multline}
и краевым условиям при $x\hm=0$:
\begin{multline}
\label{5}
P_{0j}(n_1,n_2,0,t)=0,\ n_1+n_2>0;\\
P_{0j}(0,0,0,t)=\int\limits_0^{\infty}P_{0j}(0,0,x,t)\eta_0(x)\,dx+{}\\
 {}+\int\limits_0^{\infty}P_{1j}(1,0,x,t)\eta_1(x)dx+{}\\
 {}+
\int\limits_0^{\infty}P_{2j}(0,1,x,t)\eta_2(x)\,dx\,;
\end{multline}

\vspace*{-12pt}

\noindent
\begin{multline}
\label{6}
P_{1j}(n_1,n_2,0,t)+P_{2j}(n_1,n_2,0,t)={}\\
{}=\int\limits_0^{\infty}P_{1j}(n_1+1,n_2,x,t)\eta_1(x)\,dx+{}\\
{}+
\int\limits_0^{\infty}P_{2j}(n_1,n_2+1,x,t)\eta_2(x)\,dx+{}\\
{}+\int\limits_0^{\infty}P_{0j}(n_1,n_2,0,t)\eta_0(x)\,dx\,.
\end{multline}

Будем предполагать, что в~начальный момент времени $t\hm=0$ система
свободна от требований, а~с~начала профилактики прибора прошло
случайное время с~заданным распределением с~плотностью $d(x).$
Таким образом,
\begin{align*}
P_{ij}\left(n_1,n_2,x,0\right)&=0,\ i=1,2;
\\
P_{0j}\left(n_1,n_2,x,0\right)&=c_jd(x)\delta_{n_1+n_2,0},\ \
j=1,\ldots,k\,.
\end{align*}
Положим
\begin{multline*}
p_{ij}\left(z_1,z_2,x,s\right)={}\\
{}=\sum\limits_{n_1=0}^{\infty}
\sum\limits_{n_2=0}^{\infty}z_1^{n_1}z_2^{n_2}\!
\int\limits_0^{\infty}e^{-st}P_{ij}(n_1,n_2,x,t)\,dt\,;
\end{multline*}
$$
  \psi(s)=\int\limits_0^{\infty}e^{-sx}\,dx
  \int\limits_0^{\infty}\fr{c(u+x)d(u)}{1-C(u)}\,du\,.
$$
Тогда, учитывая начальные условия,  из \eqref{3}
получаем:
\begin{multline}
\label{7} 
\fr{\partial p_{ij}(z_1,z_2,x,s)}{\partial x}={}\\
{}=-\left(s+a_j+\eta_i(x)\right)p_{ij}
\left(z_1,z_2,x,s\right)+{}\\
{}+c_j\left(pz_1+(1-p)z_2\right)
\sum\limits_{l=1}^ka_lp_{il}\left(z_1,z_2,x,s\right),\\ 
i=1,2;
\end{multline}

\vspace*{-12pt}

\noindent
\begin{multline}
\label{8} 
\fr{\partial p_{0j}(z_1,z_2,x,s)}{\partial x}={}\\
{}=-\left(s+a_j+\eta_0(x)\right)p_{0j}\left(z_1,z_2,x,s\right)+{}\\
{}+c_j\left(pz_1+(1-p)z_2\right)\sum\limits_{l=1}^ka_lp_{0l}\left(z_1,z_2,x,s\right)+{}\\
{}+ c_jd(x).
\end{multline}
Решения \eqref{7} и~\eqref{8} имеют вид:
\begin{multline}
\label{9}
p_{ij}\left(z_1,z_2,x,s\right)=\left(1-B_i(x)\right)c_j\times{}\\
{}\times \sum\limits_{m=1}^k\fr{\gamma_i^{(m)}(z_1,z_2,s)}{\mu_m(z_1,z_2)+a_j}\,
e^{-(s-\mu_m(z_1,z_2))x}\,,\\
 i=1,2\,,
\end{multline}
\vspace*{-12pt}

\noindent
\begin{multline}
\label{10}
p_{0j}\left(z_1,z_2,x,s\right)={}\\
{}=\left(1-C(x)\right)
c_j\!\!\sum\limits_{m=1}^k\!\! e^{-(s-\mu_m(z_1,z_2))x}\!
\!\left(\!
\vphantom{\int\limits_{l=1}^k}
\delta^{(m)}\left(z_1,z_2,s\right)+{}\right.\\
%\left.
{}+\alpha_m^{-1}\left(z_1,z_2\right)
\prod\limits_{l=1}^k
\left(\mu_m\left(z_1,z_2\right)+a_l\right)\times{}\\
\left.{}\times \int\limits_0^x\!
e^{(s-\mu_m(z_1,z_2))u}
\fr{d(u)}{1-C(u)}\,du
\right)
\!\Bigg/ \!\left(\mu_m\left(z_1,z_2\right)+{}\right.\\
\left.{}+a_j\right)\,,
\end{multline}
где функции $\gamma_i^{(m)}(z_1,z_2,s)$  и~$\delta^{(m)}(z_1,z_2,s)$ являются
произвольными функциями указанных переменных и~определяются из
краевых условий. Из~\eqref{5} и~\eqref{6} получаем:
\begin{multline}
\label{11}
p_{1j}\left(z_1,z_2,0,s\right)+p_{2j}\left(z_1,z_2,0,s\right)={}\\
{}=z_1^{-1}\int\limits_0^{\infty}p_{1j}\left(z_1,z_2,x,s\right)\eta_1(x)\,dx+{}
\\
+z_2^{-1}\int\limits_0^{\infty}p_{2j}\left(z_1,z_2,x,s\right)\eta_2(x)\,dx+{}\\
{}+
\int\limits_0^{\infty}p_{0j}\left(z_1,z_2,x,s\right)\eta_0(x)\,dx
-p_{0j}\left(z_1,z_2,0,s\right)\,.
\end{multline}
Заметим, что $p_{0j}(z_1,z_2,0,s)$ не зависит от $z_1$ и~$z_2,$ т.\,е.\
$p_{0j}(z_1,z_2,0,s)\hm=q_j(s).$ 
Подставляя~\eqref{9} и~\eqref{10} в~\eqref{11}, получаем:
\begin{multline}
\label{12}
\gamma_1^{(m)}\left(z_1,z_2,s\right)\left(1-z_1^{-1}\beta_1(s-\mu_m(z_1,z_2))\right)+{}\\
{}+
\gamma_2^{(m)}(z_1,z_2,s)\left(1-z_2^{-1}\beta_2(s-\mu_m(z_1,z_2))\right)={}\\
{} =
\delta^{(m)}\left(z_1,z_2,s\right)\left(\gamma\left(s-\mu_m\left(z_1,z_2\right)\right)-1\right)+{}\\
{}+
\alpha_m^{-1}\left(z_1,z_2\right)\prod\limits_{l=1}^k
\left(\mu_m\left(z_1,z_2\right)+a_l\right)\psi\left(s-\mu_m(z_1,z_2)\right),\\
j=1,\ldots,k.
\end{multline}
В силу леммы~1 левая часть~\eqref{12} обращается в~0 при
$z_1\hm=z_{1m}(s)$ и~$z_2\hm=z_{2m}(s)$, $m\hm=1,\ldots,k.$ Следовательно,
\begin{multline}
\label{13}
\delta^{(m)}\left(z_{1m}(s),z_{2m}(s),s\right)={}\\
{}=\fr{\psi(s-\lambda_m(s))}{\alpha_m(z_{1m}(s),z_{2m}(s))
(1-\gamma(s-\lambda_m(s)))}\times{}\\
{}\times \prod\limits_{l=1}^k\left(\lambda_m(s)+a_l\right).
\end{multline}
Из \eqref{10} следует, что
$$
q_j(s)=c_j\sum\limits_{m=1}^k\fr{\delta^{(m)}(z_1,z_2,s)}{\mu_m(z_1,z_2)+a_j},\
j=1,\ldots,k .
$$
Решая эту систему уравнений относительно
$\delta^{(m)}(z_1,z_2,s),$ получаем:
\begin{multline}
\label{n1}
\delta^{(m)}(z_1,z_2,s)=\left(pz_1+(1-p)z_2\right)\times{}\\
{}\times
\fr{\prod\nolimits_{j=1}^k(\mu_m(z_1,z_2)+a_j)}
{\alpha_m(z_1,z_2)}\sum\limits_{l=1}^k\frac{a_lq_l(s)}{\mu_m(z_1,z_2)+a_l}.
\end{multline}
Подставляя в~\eqref{n1} $z_1\hm=z_{1m}(s)$ и~$z_2\hm=z_{2m}(s),$ имеем:
\begin{multline}
\label{14}
\delta^{(m)}\left(z_{1m}(s),z_{1m}(s),s\right)={}\\
{}=
\left(pz_{1m}(s)+(1-p)z_{2m}(s)\right)\times{}\\
{}\times
\fr{\prod\nolimits_{j=1}^k
(\lambda_m(s)+a_j)}{\alpha_m(z_{1m}(s),z_{1m}(s))}
\sum\limits_{l=1}^k\fr{a_lq_l(s)}{\lambda_m(s)+a_l}\,.
\end{multline}
Сравнивая два представления~\eqref{13} в~\eqref{14} для
$\delta^{(m)}(z_m(s),s),$ получаем систему уравнений для~$q_l(s)$:
\begin{multline*}
\sum\limits_{l=1}^k\fr{a_lq_l(s)}{\lambda_m(s)+a_l}={}\\
{}=\fr{\psi(s-\lambda_m(s))}{(pz_{1m}(s)+(1-p)z_{2m}(s))
(1-\gamma(s-\lambda_m(s)))},\\
m=1,\ldots,k\,,
\end{multline*}
из которой находим
\begin{multline}
\hspace*{-3pt}q_l(s)=c_l\prod\limits_{j=1}^k
\left(\lambda_l(s)+a_j\right) 
\sum\limits_{m=1}^k
%\fr
\psi(s-\lambda_m(s))\!\Bigg/ \!
\Bigg(\left(1-{}\right.\\
\left.
{}-\gamma\left(s-\lambda_m(s)\right)\right)(\lambda_m(s)+a_l)\times{}\\
{}\times \prod\limits_{n\ne m}(\lambda_m(s)-\lambda_n(s))\!\Bigg).
\label{15}
\end{multline}
Подставляя \eqref{15} в~\eqref{n1} и~учитывая~\eqref{1}, получаем:
\begin{multline*}
\delta^{(m)}(z_1,z_2,s)=\fr{(pz_1+(1-p)z_2)}{\alpha_m(z_1,z_2)}\times
\\
\times\sum\limits_{j=1}^k
\fr{\psi(s-\lambda_j(s))\prod\nolimits_{l=1}^k(\lambda_j(s)+a_l)}
{(pz_{1j}(s)+(1-p)z_{2j}(s))(1-\gamma(s-\lambda_j(s)))}\times{}\\
{}\times\prod\limits_{\nu\ne j}
\fr{\mu_m(z_1,z_2)-\lambda_{\nu}(s)}{\lambda_j(s)-\lambda_{\nu}(s)}\,.
\end{multline*}
Положим
$$
\lambda_m(z_2,s)=\mu_m\left(z_{1m}(z_2,s),z_2\right),\enskip m=1,\ldots,k\,.
$$
Подставляя в~\eqref{12} $z_1\hm=z_{1m}(z_2,s)$, имеем:
\begin{multline}
\label{1q}
\gamma_2^{(m)}\left(z_{1m}(z_2,s),z_2,s\right)={}\\
{}=\fr{\delta^{(m)}(z_{1m}(z_2,s),z_2,s)(\gamma_m(s-\lambda_m(z_2,s))-1)}
{1-z_2^{-1}\beta_2(s-\lambda_m(z_2,s))}+{}
\\
{}+\alpha_m^{-1}(z_{1m}(z_2,s),z_2)\psi(s-\lambda_m(z_2,s))
\prod\limits_{l=1}^k\left(\lambda_m(z_2,s)+{}\right.\\
\left.{}+a_l\right)\!\Bigg/\!
\left(
1-z_2^{-1}\beta_2(s-\lambda_m(z_2,s))\right).
\end{multline}
Далее, из~\eqref{9} следует:
$$
p_{2j}(z_1,z_2,0,s)=c_j\sum\limits_{m=1}^k
\fr{\gamma_2^{(m)}(z_1,z_2,s)}{\mu_m(z_1,z_2)+a_j}\,.
$$
Отсюда
\begin{multline}
\label{2q}
\gamma_2^{(m)}(z_1,z_2,s)=\fr{pz_1+(1-p)z_2}{\alpha_m(z_1,z_2)}\times{}\\
{}\times
\prod\limits_{j=1}^k(\mu_m(z_1,z_2)+a_j)
\sum\limits_{l=1}^k\fr{a_lp_{2l}(z_1,z_2,0,s)}{\mu_m(z_1,z_2)+a_l}\,.
\end{multline}
Так как $p_{2j}(z_1,z_2,0,s)$ не зависит от $z_1$, то
\begin{multline}
\label{3q}
p_{2j}\left(z_1,z_2,0,s\right)={}\\
{}=c_j
\sum\limits_{m=1}^k\fr{\gamma_2^{(m)}\left(z_{1m}(z_2,s),z_2,s\right)}{\lambda_m(z_2,s)+a_j}\,.
\end{multline}
Таким образом, соотношения~\eqref{1q}--\eqref{3q} полностью
определяют $\gamma_2^{(m)}(z_1,z_2,s)$ при любых $z_1$ и~$z_2$.
Теперь из~\eqref{12} можно найти $\gamma_2^{(m)}(z_1,z_2,s)$.

Все функции, необходимые для вычисления $p_{ij}(z_1,z_2,x,s)$,
$i\hm=0,1,2$, $j\hm=1,\ldots,k,$ найде-\linebreak\vspace*{-12pt}

\columnbreak

\noindent
ны. Искомая производящая функция
процесса $(L_1(t),L_2(t))$ равна:

\noindent
\begin{multline*}
\int\limits_0^{\infty}e^{-st}\mathbf{E}
z_1^{L_1(t)} z_2^{L_2(t)}\,dt={}\\
{}=
\sum\limits_{i=0}^2\sum\limits_{j=1}^k\int\limits_0^{\infty}p_{ij}
\left(z_1,z_2,x,s\right)\,dx\,.
\end{multline*}

\vspace*{-18pt}

{\small\frenchspacing
 {%\baselineskip=10.8pt
 \addcontentsline{toc}{section}{References}
 \begin{thebibliography}{9}
\bibitem{1-u}
\Au{Doshi B.\,T.} Queueing systems with vacations~--- a~survey~// 
Queueing Syst., 1986. Vol.~1.  P.~29--66.
\bibitem{2-u}
\Au{Takagi H.} Time-dependent analysis of $M\vert G\vert 1$ vacation models 
with exhaustive service~// Queueing Syst.,
1990. Vol.~6.  P.~369--390.
\bibitem{3-u}
\Au{Li J., Tian N., Zhang~Z.\,G. , Luh~H.\,P.} 
Analysis of the $M\vert G\vert 1$ queue with exponentially working vacations~--- 
a~matrix analytic approach~// Queueing Syst., 2009. Vol.~61.
P.~139--166.
\bibitem{4-u}
\Au{Bouman N., Borst S.\,C., Boxma~O.\,J., Leeuwaarden~J.\,S.\,H.} 
Queues with random back-offs~// Queueing Syst.,
2014. Vol.~77. P.~33--74.
\bibitem{5-u}
\Au{Ушаков~В.\,Г.} Система обслуживания с~гиперэкспоненциальным входящим потоком 
и~профилактиками прибора~// Информатика и~её применения, 2016. Т.~10. 
Вып.~2. С.~93--98.
 \end{thebibliography}

 }
 }

\end{multicols}

\vspace*{-9pt}

\hfill{\small\textit{Поступила в~редакцию 11.05.18}}

\vspace*{6pt}

%\pagebreak

%\newpage

%\vspace*{-28pt}

\hrule

\vspace*{2pt}

\hrule

%\vspace*{-2pt}

\def\tit{A~HEAD OF~THE~LINE PRIORITY QUEUE\\ WITH~WORKING VACATIONS}

\def\titkol{A head of the line priority queue with working vacations}

\def\aut{E.\,S.~Kondranin$^1$ and~V.\,G.~Ushakov$^{1,2}$}

\def\autkol{E.\,S.~Kondranin and~V.\,G.~Ushakov}

\titel{\tit}{\aut}{\autkol}{\titkol}

\vspace*{-11pt}


\noindent
$^1$Department of 
Mathematical Statistics, Faculty of Computational Mathematics and Cybernetics, 
M.\,V.~Lo\-mo-\linebreak
$\hphantom{^1}$no\-sov Moscow State University, 1-52~Leninskiye Gory, 
Moscow 119991, GSP-1, Russian Federation

\noindent
$^2$Institute of Informatics Problems, Federal Research Center 
``Computer Science and Control'' of the Russian\linebreak
$\hphantom{^1}$Academy of Sciences,  44-2~Vavilov Str., Moscow 119333, Russian Federation

\def\leftfootline{\small{\textbf{\thepage}
\hfill INFORMATIKA I EE PRIMENENIYA~--- INFORMATICS AND
APPLICATIONS\ \ \ 2018\ \ \ volume~12\ \ \ issue\ 4}
}%
 \def\rightfootline{\small{INFORMATIKA I EE PRIMENENIYA~---
INFORMATICS AND APPLICATIONS\ \ \ 2018\ \ \ volume~12\ \ \ issue\ 4
\hfill \textbf{\thepage}}}

\vspace*{3pt}



\Abste{The authors analyze the single-server queueing system with 
two types of customers, head of the line priority, hyperexponential 
input stream, and working vacations. The authors obtain the Laplace 
transform (with respect to an arbitrary point in time) of the joint 
distribution of server state, queue size, and elapsed time in that state. 
The authors restrict themselves to a~system with exhaustive service (the 
queue must be empty when the server starts a vacation) and multiple vacations. 
The queueing systems with vacations have been well studied because of their 
applications in modeling computer networks, communication, and manufacturing 
systems. For example, in many digital systems, the processor is multiplexed 
among a~number of jobs and, hence, is not available all the time to handle one job type. 
Besides such an application, theoretical interest in vacation models 
has been aroused with respect to their relationship with polling models.}

\KWE{hyperexponential input stream; working vacations; single server; 
head of the line priority; queue length}



\DOI{10.14357/19922264180405}

\vspace*{-14pt}

\Ack
\noindent
This work was supported by the Russian Foundation for Basic Research 
(project 18-07-00678).


%\vspace*{6pt}

  \begin{multicols}{2}

\renewcommand{\bibname}{\protect\rmfamily References}
%\renewcommand{\bibname}{\large\protect\rm References}

{\small\frenchspacing
 {%\baselineskip=10.8pt
 \addcontentsline{toc}{section}{References}
 \begin{thebibliography}{9}
\bibitem{1-u-1}
\Aue{Doshi, B.\,T.} 1986. Queueing systems with vacations~--- a~survey. 
\textit{Queueing Syst.} 1:29--66.
\bibitem{2-u-1}
\Aue{Takagi, H.} 1990. Time-dependent analysis of $M\vert G\vert M\vert 1$ 
vacation models with exhaustive service. \textit{Queueing Syst.} 6:369--390.
\bibitem{3-u-1}
\Aue{Li, J., N. Tian, Z.\,G.~Zhang,  and H.\,P.~Luh.} 2009. Analysis of the 
$M\vert G\vert 1$ queue with exponentially working vacations~--- 
a~matrix analytic approach. \textit{Queueing Syst.} 61:139--166.
{\looseness=1

}
\bibitem{4-u-1}
\Aue{Bouman, N., S.\,C.~Borst, O.\,J.~Boxma, and J.\,S.\,H.~Leeuwaarden.} 
2014. Queues with random back-offs. \textit{Queueing Syst.} 77:33--74.
\bibitem{5-u-1}
\Aue{Ushakov, V.\,G.} 2016. Sistema obsluzhivaniya s~gipereksponentsialnym 
vkhodyashchim potokom i~profilaktikami\linebreak pribora [Queueing system with working 
vacations and hyperexponential input stream]. 
\textit{Informatika i~ee Primeneniya~--- Inform. Appl.} 10(2):93--98.
\end{thebibliography}

 }
 }

\end{multicols}

\vspace*{-6pt}

\hfill{\small\textit{Received May 11, 2018}}

%\pagebreak

%\vspace*{-18pt}

\Contr

\noindent
\textbf{Kondranin Egor S.} (b.\ 1995)~---  MSc student, Department of 
Mathematical Statistics, Faculty of Computational Mathematics and Cybernetics, 
M.\,V.~Lomonosov Moscow State University, 1-52~Leninskiye Gory, 
Moscow 119991, GSP-1, Russian Federation; \mbox{ekondranin@yandex.ru}

\vspace*{6pt}

\noindent
\textbf{Ushakov Vladimir G.} (b.\ 1952)~--- 
Doctor of Science in physics and mathematics, professor, Department of Mathematical 
Statistics, Faculty of Computational Mathematics and Cybernetics, 
M.\,V.~Lomonosov Moscow State University, 1-52~Leninskiye Gory, Moscow 119991, 
GSP-1, Russian Federation; 
senior scientist, Institute of Informatics Problems, Federal Research Center 
``Computer Science and Control'' of the Russian Academy of Sciences, 
44-2~Vavilov Str., Moscow 119333, Russian Federation; \mbox{vgushakov@mail.ru}
\label{end\stat}

\renewcommand{\bibname}{\protect\rm Литература}           %3+
\def\stat{leri}

\def\tit{СРЕДНЕЕ РАССТОЯНИЕ В~КОНФИГУРАЦИОННЫХ ГРАФАХ СО~СТЕПЕННЫМ РАСПРЕДЕЛЕНИЕМ$^*$}

\def\titkol{Среднее расстояние в~конфигурационных графах со~степенным распределением}

\def\aut{М.\,М.~Лери$^1$}

\def\autkol{М.\,М.~Лери}

\titel{\tit}{\aut}{\autkol}{\titkol}

\index{Лери М.\,М.}
\index{Leri M.\,M.}


{\renewcommand{\thefootnote}{\fnsymbol{footnote}} \footnotetext[1]
{Финансовое обеспечение исследований осуществлялось из средств федерального
бюджета на выполнение государственного задания Карельского научного центра Российской академии наук
(Институт прикладных математических исследований КарНЦ РАН).}}


\renewcommand{\thefootnote}{\arabic{footnote}}
\footnotetext[1]{Институт прикладных математических исследований Карельского научного центра
Российской академии наук, \mbox{leri@krc.karelia.ru}}

%\vspace*{-2pt}








\Abst{В случайных конфигурационных графах с~дискретным степенным распределением степеней вершин
с фиксированным параметром рассматривается среднее расстояние в~графе, которое вычисляется
как среднее арифметическое расстояний между всеми парами вершин графа.
Эта характеристика оценивается с~по\-мощью методов имитационного моделирования. В~силу вычислительных
ограничений рассматриваются графы в~доасимптотической области (в~настоящей работе это графы объемом
до 7000~вершин). По\-стро\-ены модели зависимостей сред\-не\-го рас\-сто\-яния от объема графа и~па\-ра\-мет\-ра распределения степеней вершин.
Проведено сравнение полученных результатов с~результатами тео\-ре\-ти\-че\-ских исследований типичного расстояния
в графе в~асимп\-то\-ти\-ке (т.\,е.\ когда число вершин графа стремится к~бес\-ко\-неч\-ности), приведенными в~работах
Р.~Хофстада.}

\KW{конфигурационные графы; степенное распределение;
сред\-нее рас\-сто\-яние в~графе; имитационное моделирование}

\DOI{10.14357/19922264230104} 
  
\vspace*{-6pt}


\vskip 10pt plus 9pt minus 6pt

\thispagestyle{headings}

\begin{multicols}{2}

\label{st\stat}

\section{Введение}

\vspace*{-1pt}

Изучение структуры и~функционирования сложных сетей продолжает оставаться одним из важных
направлений исследований в~науке и~технике~\cite{Dur,Hof1}. Примерами таких сетей, окружающих
нас в~повседневной жизни, служат интернет, электрические и~телекоммуникационные сети, сети
социальных отношений, соавторства и~цитирования и~др.
Их быст\-рое и~динамичное развитие и~нарастающая популярность легли в~основу многих фундаментальных
исследований в~об\-ласти топологии таких сетей (см., например,~\cite{Dur,Hof1,Hof2,New1,New2}).
В~качестве моделей слож\-ных сетей широко используются случайные графы, причем их разнообразие
касается как определения степеней вершин графа, так и~уста\-нов\-ле\-ния связей между этими вершинами.
В~частности, было показано (см., например,~\cite{Fa,RN}), что модели случайных графов с~независимыми
одинаково распределенными степенями вершин с~общим дискретным законом распределения подходят для
моделирования сети Интернет в~случае, когда в~качестве узлов сети рассматриваются автономные системы.

Увеличение размеров сетей и~изменчивость сетевой структуры дают понять, что для адекватного отражения
их топологии и~функционирования в~ходе по\-стро\-ения их математических моделей необходимо учитывать не
только распределение степеней вершин в~со\-от\-вет\-ст\-ву\-ющей модели случайного графа, но также
принимать во внимание и~другие не менее важ\-ные характеристики исследуемых сетей~\cite{Hof1, New1}.
В связи с~этим различные структурные характеристики слож\-ных сетей пред\-став\-ля\-ют определенный интерес
как при моделировании их топологии, так и~при изучении динамических процессов, происходящих в~таких сетях
по мере их рос\-та или под внешними воздействиями. Одна из таких характеристик~---
рас\-сто\-яние между двумя произвольными вершинами графа~\cite{Dur, Hof2, New1}.

 Мож\-но рас\-смат\-ри\-вать различные
виды расстояний в~графе: расстояние между двумя заданными вершинами, рас\-сто\-яние между произвольно\linebreak
выбранными вершинами, все расстояния между каж\-дой парой вершин, наименьшее воз\-мож\-ное {рас\-сто\-яние},
наибольшее, или диаметр графа, и~т.\,д. 

В~на\-сто\-ящей работе рассматривается среднее расстояние в~графе,
которое вы\-чис\-ля\-ет\-ся как среднее арифметическое расстояний между всеми парами вершин графа.
Цель работы со\-сто\-яла в~на\-хож\-де\-нии зависимостей среднего расстояния в~графе от числа его вершин и~па\-ра\-мет\-ра 
распределения степеней вершин, а~также в~сравнении результатов работы с~результатами тео\-ре\-ти\-че\-ских
исследований рас\-сто\-яния в~графе, приведенными в~\cite{Hof2}.
Исследование проводилось по\-средст\-вом методов имитационного моделирования с~по\-сле\-ду\-ющей статистической
обработкой данных с~помощью программного обеспечения Statistica 10 и~Wolfram Mathematica 9.0.

\section{Описание модели}

Рассматривается случайный граф, со\-сто\-ящий из~$N$~вершин. Через $\xi_1,\xi_2,\ldots,\xi_N$ обозначим
степени вершин, которые являются независимыми одинаково распределенными случайными величинами
со сле\-ду\-ющим дискретным степенным распределением~\cite{RN}:
\begin{equation}
\label{eq1}
{\bf P}\{\xi = k\} = k^{-\tau} - (k+1)^{-\tau}, \quad k=1,2,\ldots,
\end{equation}
с фиксированным параметром~$\tau\hm>1$. Легко показать, что математическое ожидание
распределения~(\ref{eq1}) рав\-но ${\bf E}\xi\hm=\zeta(\tau)$, где $\zeta(\tau)$~--- значение дзе\-та-функ\-ции
Римана в~точке~$\tau$. Что касается дисперсии, то при $\tau\hm>2$ она конечна, а при
$\tau\hm\in(1,2]$~-- бесконечна.
В~работе рассматриваются конфигурационные графы~\cite{Bol}, построение которых происходит сле\-ду\-ющим
образом.
Для каждой из $N$ вершин графа задаются степени в~соответствии  с~распределением~(\ref{eq1}) с~выбранным
значением па\-ра\-мет\-ра~$\tau$. Степени определяют чис\-ло различимых полуребер~\cite{RN} (под полуребром понимают
ребро, инцидентное данной вершине графа, для которого смежная вершина еще не определена), занумерованных в~произвольном порядке. 
Для формирования ребер все полуребра попарно и~равновероятно соединяют между собой.
Сумма степеней вершин рас\-смат\-ри\-ва\-емо\-го графа является случайной величиной. Очевидно, что она должна быть чет\-ной,
поэтому, если это не так, для построения недостающего ребра к~равновероятно выбранной
вершине добавляется одно полуребро, увеличивая степень этой вершины на~1. Граф, по\-стро\-ен\-ный таким
образом, может иметь пет\-ли, цик\-лы и~кратные \mbox{ребра}.

Известно (см., например,~\cite{Dur,Hof1,RN}), что степенной конфигурационный граф, значение па\-ра\-мет\-ра
распределения степеней вершин которого $\tau\hm>1$, асимптотически почти наверное содержит больше одной
компоненты связности, причем при $\tau\hm\in(1,2)$ в~таком графе существует, и~она единственна, так
называемая гигантская компонента связ\-ности, чис\-ло вершин в~которой пропорционально~$N$
при $N\hm\rightarrow\infty$, а~объем любой другой компоненты такого графа бесконечно мал по
сравнению с~объемом гигантской компоненты.


\vspace*{-6pt}

\section{Среднее расстояние в~графе}

\vspace*{-2pt}

Расстояния между узлами сложной сети служат важными числовыми характеристиками сетевой
топологии (см., например,~\cite{Hof2, Chu}).
Пусть $G \hm= (V,E)$~--- неориентированный граф, в~котором~$V$~--- множество вершин, а~$E$~--- множество ребер.
Обозначим через~$l(v,u)$ чис\-ло ребер простой цепи, со\-еди\-ня\-ющей вершины~$v$ и~$u$ графа~$G$
($v,u\hm\in V$ и~$v\hm\neq u$). Если вершины~$v$ и~$u$ принадлежат разным компонентам связ\-ности, то $l(v,u)$
полагают равным~$\infty$. Длину цепи от вершины~$v$ до вершины~$u$ наименьшей длины называют расстоянием
между этими вершинами:
$$
d(v,u)=\min\limits_{l(v,u)\neq\infty}l(v,u).
$$

Пусть $k$~--- чис\-ло рас\-сто\-яний $d(v,u)\hm\neq\infty$ между всеми парами вершин~$v$ и~$u$ ($v\hm\neq u$).
Среднее расстояние вы\-чис\-ля\-ет\-ся как среднее арифметическое всех рас\-сто\-яний $d(v,u)$ графа~$G$:
\begin{equation*}
\mathrm{dist} = \mathrm{dist}\,(G) = \fr{\sum\nolimits_{v,u\in V(v\neq u)}d(v,u)}{k}.
\end{equation*}

Из теорем~7.2 и~7.1 в~\cite{Hof2} следует, что при $N\hm\rightarrow\infty$ 
\begin{equation}
\label{eq2}
d(v,u)\sim\fr{2\ln\ln N}{|\ln(\tau-1)|}\,,
\end{equation}
если $1\hm<\tau\hm<2$,
и
\begin{equation}
\label{eq3}
d(v,u)\sim\fr{\ln N}{\ln\nu}\,,
\end{equation}
если $\tau>2$, 
где $\nu={\bf E}\xi(\xi-1)/({\bf E}\xi)$ и~$\nu\hm>1$.
Выражения~(\ref{eq2}) и~(\ref{eq3}) носят асимптотический характер и~получены для <<типичных рас\-сто\-яний>>
(где под типичным рас\-сто\-яни\-ем понимается математическое ожидание рас\-сто\-яния)~\cite{Hof2} в~конфигурационных
графах с~бесконечной и~с~конечной дис\-пер\-си\-ями соответственно.
Легко показать, что для конфигурационных графов с~распределением~(\ref{eq1})
$$
\nu=\fr{2\zeta(\tau-1)}{\zeta(\tau)}-2\,,
$$
причем $\nu>1$ при $2\hm<\tau\hm\leq 2,8106\ldots$

Для нахождения зависимости среднего расстояния от числа вершин графа~$N$ и~па\-ра\-мет\-ра распределения
степеней вершин~$\tau$ были получены оценки средних расстояний в~конфигурационных графах различных
размерностей с~разными па\-ра\-мет\-ра\-ми распределения степеней вершин. По полученным результатам были
построены зависимости $\mathrm{dist}$ от чис\-ла вершин графа~$N$ при конкретных значениях па\-ра\-мет\-ра распределения
степеней вершин~$\tau$, а~также зависимости $\mathrm{dist}$ от~$N$ и~$\tau$ на интервалах $\tau\hm\in(1,2)$ и~$\tau\hm\in(2,\,2,8]$.
Рассматривались конфигурационные графы сле\-ду\-ющих размерностей:
$10\hm\leq N\hm\leq 100$ с~шагом~$10$, $100\hm\leq N\hm\leq 1000$ с~шагом~$50$, $1000\hm\leq N\hm\leq 7000$ с~шагом~$500$.
Значения параметра~$\tau$ изменялись с~шагом $0{,}1$ в~двух интервалах: $1{,}1\hm\leq\tau\hm<2$ и~$2\hm<\tau\hm\leq 2{,}8$,
а~так\-же были взяты два дополнительных значения: $\tau\hm=1{,}99$ и~$2{,}01$. Для
каждой пары значений $(N,\tau)$
генерировалось\linebreak\vspace*{-12pt}

\pagebreak

\noindent
  по~$100$~графов, т.\,е.\ $40\,000$ и~$36\,000$ графов на интервалах $1{,}1\hm\leq\tau\hm\leq 1{,}99$
и~$2{,}01\hm\leq\tau\hm\leq 2{,}8$ соответственно.
Расстояния в~графе находились с~применением алгоритма Дейкстры~\cite{Dijk}.

\subsection{Результаты для интервала $\tau\in(1,2)$}

Сначала для рассматриваемых в~настоящей работе графов были по\-стро\-ены зависимости среднего рас\-сто\-яния $\mathrm{dist}$ от
чис\-ла вершин $N$ при фиксированных значениях па\-ра\-мет\-ра~$\tau$, т.\,е.\ для каждого из рас\-смот\-рен\-ных значений
$1{,}1\hm\leq\tau\hm\leq 1{,}99$ были получены регрессионные зависимости вида
\begin{equation}
\label{eq4}
\mathrm{dist} = a \ln\ln N + b\,.
\end{equation}
Здесь и~далее коэффициенты всех регрессионных уравнений находили по\-средст\-вом метода наименьших
квадратов, зна\-чи\-мость коэффициентов проверяли с~помощью критерия Стьюдента. Для оценки степени подгонки
регрессионной модели к~данным вы\-чис\-ля\-ли коэффициент детерминации этой модели и~с~по\-мощью критерия Фишера
проверяли гипотезу $H_0: R^2\hm=0$. Проверка всех статистических гипотез осуществлялась на 5\%-ном уровне
зна\-чи\-мости.

Обозначим $a_f={2}/{|\ln(\tau-1)|}$ и~сравним эти значения с~коэффициентами~$a$ уравнений~(\ref{eq4})
для каждого~$\tau$.
В табл.~1 приведены значения~$a_f$, коэффициенты~$a$ и~$b$ регрессионных урав\-не\-ний вида~(\ref{eq4})
и~соответствующие коэффициенты детерминации~$R^2$ этих уравнений. Все коэффициенты~$a$ и~$b$ в~табл.~1
значимы, а~гипотезы $H_0: R^2\hm=0$ отвергаются для всех уравнений.




Таким образом, при фиксированных~$\tau$ сред\-нее\linebreak рас\-сто\-яние в~графе с~рос\-том чис\-ла его вершин\linebreak
рас\-тет как $\ln\ln N$, так же как и~расстояние в~асимп\-то\-ти\-ке~(\ref{eq2}). Из табл.~1 видно,
что значения коэффициента~$a$ ниже значений~$a_f$ на всем
интервале\linebreak  изменения~$\tau$, однако эта разница не остается неизменной,
а~возрастает с~рос\-том~$\tau$.
Более того, сам угловой коэффициент~$a$ возрастает с~рос\-том
па\-ра-\linebreak\vspace*{-12pt}

%\begin{table*}\small %tabl1
\begin{center}

\vspace*{6pt}

\noindent
\parbox{64mm}{{{\tablename~1}\ \ \small{Значения $a_f$, коэффициенты $a$ и~$b$ зависимостей вида~(\ref{eq4})
и коэффициенты детерминации $R^2$ этих уравнений
}}}


\vspace*{6pt}


{\small 
\begin{tabular}{|c|c|c|c|c|}
\hline
&&&&\\[-10pt]
$\tau$ & $a_f$ & $a$ & $b$ & $R^2$ \\ 
\hline
1,1 & 0,869 &   0,745 & \hphantom{$-$}1,702 & 0,88 \\
1,2 & 1,243 &   0,975 & \hphantom{$-$}1,513 & 0,91 \\
1,3 & 1,661 &   1,255 & \hphantom{$-$}1,244 & 0,94 \\
1,4 & 2,183 &   1,596 & \hphantom{$-$}0,869 & 0,96 \\
1,5 & 2,885 &   1,959 & \hphantom{$-$}0,494 & 0,96 \\
1,6 & 3,915 &   2,421 & $-$0,065 & 0,98 \\
1,7 & 5,607 &   3,048 & $-$0,909 & 0,98 \\
1,8 & 8,963 &   3,668 & $-$1,710 & 0,98 \\
1,9 & 18,982\hphantom{9} & 4,417 & $-$2,806 & 0,98 \\
\hphantom{9}1,99 & 198,998\hphantom{99} & 5,262 & $-$4,045 & 0,96\\
\hline
\end{tabular}
}
\end{center}
%\end{table*}

{ \begin{center}  %fig1
 \vspace*{-2pt}
    \mbox{%
\epsfxsize=78.504mm
\epsfbox{ler-1.eps}
}

\end{center}



\noindent
{{\figurename~1}\ \ \small{График экспериментальных
значений $\mathrm{dist}$ от $N$ и~$1{,}1\hm\leq\tau\hm<2$
}}}

\vspace*{8pt}

\addtocounter{figure}{1}
\addtocounter{table}{1}



\noindent
мет\-ра~$\tau$,
 и~это  на\-гляд\-но видно
на рис.~1, где показана за\-ви\-си\-мость
экспериментальных значений $\mathrm{dist}$ от~$N$ и~$\tau$.


Далее задача состояла в~том, чтобы найти за\-ви\-си\-мость сред\-не\-го рас\-сто\-яния $\mathrm{dist}$ от обеих переменных:
$N$ и~$\tau$.
Сначала была построена за\-ви\-си\-мость в~виде $\mathrm{dist}\hm={2\ln\ln N}/({|\ln(\tau-1)|})\hm+b$. Получено
сле\-ду\-ющее регрессионное уравнение:
\begin{equation*}
\mathrm{dist} = \fr{2\ln\ln N}{|\ln(\tau-1)|}-39{,}604\,.
\end{equation*}
К сожалению, коэффициент детерминации полученной за\-ви\-си\-мости оказался очень низ\-ким
($R^2\hm=0{,}01$). Гипотеза о~равенстве~$R^2$ нулю не отвергается, коэффициент~$b$
статистически не значим, поэтому такую модель использовать для прогноза не имеет смысла.



\begin{figure*} %fig2
\vspace*{1pt}
\begin{minipage}[t]{80mm}
\begin{center}
   \mbox{%
\epsfxsize=79mm
\epsfbox{ler-2-a.eps}
}
\end{center}
\vspace*{-9pt}
\Caption{Регрессионная зависимость~(\ref{eq6}) среднего расстояния $\mathrm{dist}$ от $N$ при фиксированных
значениях $1{,}1\hm\leq\tau\hm<2$: \textit{1}~--- $\tau\hm=1{,}1$;
\textit{2}~--- 1,3; \textit{3}~--- 1,5; \textit{4}~--- 1,7; \textit{5}~--- $\tau\hm= 1{,}99$
}
\end{minipage}
%\end{figure*}
\hfill
%\begin{figure*} %fig3
\vspace*{1pt}
\begin{minipage}[t]{80mm}
\begin{center}
   \mbox{%
\epsfxsize=77.81mm
\epsfbox{ler-2-b.eps}
}
\end{center}
\vspace*{-9pt}
\Caption{Регрессионная зависимость~(\ref{eq6}) среднего расстояния $\mathrm{dist}$
от~$\tau$ при фиксированных значениях $10\hm\leq N\hm\leq 7000$:) \textit{1}~--- $N\hm=10$; \textit{2}~--- 100; \textit{3}~--- 1000;
\textit{4}~--- 5000; \textit{5}~--- $N\hm=7000$}
\end{minipage}
\vspace*{-4pt}
\end{figure*}



Далее была построена регрессия вида $\mathrm{dist}\hm={a\ln\ln N}/({|\ln(\tau-1)|})\hm+b$ и~получена
за\-ви\-си\-мость
\begin{equation}
\label{eq5}
\mathrm{dist} = \fr{0{,}0104\ln\ln N}{|\ln(\tau-1)|}+3{,}922
\end{equation}
с~коэффициентом детерминации $R^2\hm=0{,}21$. В~данном случае гипотеза $H_0: R^2\hm=0$ отвергается,
а~что касается коэффициентов~$a$ и~$b$ регрессионного уравнения~(\ref{eq5}), то коэффициент~$a$
оказался статистически значим, а~коэффициент~$b$ нет.
Поиск наилучшей регрессионной за\-ви\-си\-мости был продолжен и~привел к~получению сле\-ду\-ющей
модели за\-ви\-си\-мости сред\-не\-го рас\-сто\-яния конфигурационного графа $\mathrm{dist}$ от объема графа~$N$ 
и~па\-ра\-мет\-ра распределения степеней вершин~$\tau$:
\begin{equation}\label{eq6}
\mathrm{dist} = \fr{2(4{,}488-3{,}077\tau+0{,}417\tau^2)\ln\ln N}{|\ln(\tau-1)|}
\end{equation}
с коэффициентом детерминации $R^2\hm=0{,}88$ и~значимыми коэффициентами регрессии. Графически
за\-ви\-си\-мость~(\ref{eq6}) пред\-став\-ле\-на на рис.~2 и~3.

Оценка значимости различия между коэффициентами множественной корреляции $r\hm=\sqrt{R^2}$ регрессионных моделей~(\ref{eq5}) и~(\ref{eq6})
 на уровне зна\-чи\-мости~0,05 показала, что нулевая гипотеза $H_0:\linebreak r_{(5)}\hm=r_{(6)}$
($r_{(5)}$ и~$r_{(6)}$~--- коэффициенты множественной корреляции зависимостей~(\ref{eq5}) и~(\ref{eq6})
соответственно) отвергается; следовательно, различие между коэффициентами корреляции значимо.
Остатки обеих моделей распределены нормально, но сравнение сумм квад\-ра\-тов
остатков $\mathrm{SSR}_{(5)}\hm=62951{,}1$ и~$\mathrm{SSR}_{(6)}\hm=24786{,}9$ показывает, что $\mathrm{SSR}_{(5)}\hm>\mathrm{SSR}_{(6)}$, т.\,е.\ модель~(\ref{eq6}) 
<<лучше>> в~смыс\-ле описания изуча\-емо\-го явления и~для прогнозирования.
Таким образом, в~качестве наиболее под\-хо\-дя\-щей модели за\-ви\-си\-мости $\mathrm{dist}$ от~$N$ и~$\tau$ для до\-асимп\-то\-ти\-че\-ской
об\-ласти предлагается за\-ви\-си\-мость, описываемая уравнением~(\ref{eq6}).



На рис.~2 и~3 линии внут\-ри затененных областей соответствуют зависимостям $\mathrm{dist}$ от~$N$ (см.\ рис.~2)
и~от~$\tau$ (см.\ рис.~3) при некоторых (отраженных в~легендах) значениях па\-ра\-мет\-ра~$\tau$
или объема графа $N$ соответственно. Заметим, что кривые зависимостей $\mathrm{dist}$ от~$N$ на рис.~2
расположены одна над другой по мере роста значения параметра $1,1\leq\tau<2$ в~пределах его граничных значений.
Аналогично кривые зависимостей $\mathrm{dist}$ от~$\tau$ на рис.~3 также расположены друг над другом по
мере воз\-рас\-та\-ния чис\-ла вершин графа $10\hm\leq N\hm\leq 7000$.


\vspace*{-6pt}

\subsection{Результаты для интервала $\tau\in(2,\,2{,}8]$}

\vspace*{-2pt}


Исследование зависимости сред\-не\-го рас\-сто\-яния от~$N$ и~$\tau\hm\in(2,\,2{,}8]$ в~до\-асимп\-то\-ти\-че\-ской об\-ласти
было проведено аналогично предыду\-ще\-му исследованию для $\tau\hm\in(1,2)$.
Сначала для фиксированных значений па\-ра\-мет\-ра $2{,}01\hm\leq\tau\hm\leq 2{,}8$ были построены зависимости сред\-не\-го
рас\-сто\-яния $\mathrm{dist}$ от чис\-ла вершин графа~$N$ сле\-ду\-юще\-го вида:
\begin{equation}
\label{eq7}
\mathrm{dist} = a \ln N + b\,.
\end{equation}

Обозначим 
$$
a_f=\fr{1}{\ln\left({2\zeta(\tau-1)}/{\zeta(\tau)}-2\right)}\,.
$$
 Для сравнения этих
значений с~коэффициентами~$a$ уравнений~(\ref{eq7}) для каждого~$\tau$ все они приведены
в~табл.~2 наряду с~коэффициентами~$b$ и~со\-от\-вет\-ст\-ву\-ющи\-ми коэффициентами детерминации
$R^2$ этих уравнений. Все коэффициенты~$a$ и~$b$ значимы, а~гипотезы о~ра\-венст\-ве нулю коэффициентов
детерминации полученных моделей отвергаются.


\setcounter{figure}{4}
\begin{figure*}[b] %fig5
\vspace*{1pt}
\begin{minipage}[t]{81mm}
\begin{center}
   \mbox{%
\epsfxsize=80mm
\epsfbox{ler-4-a.eps}
}
\end{center}
\vspace*{-13pt}
\Caption{Регрессионная зависимость~(\ref{eq9}) среднего рас\-сто\-яния $\mathrm{dist}$ от~$N$ при фиксированных
значениях $2\hm<\tau\hm\leq 2{,}8$:
\textit{1}~--- $\tau\hm= 2{,}01$; \textit{2}~--- 2,2288\ldots; \textit{3}~--- 2,4; \textit{4}~--- 2,6; \textit{5}~--- $\tau\hm= 2{,}8$}
%\end{figure*}
\end{minipage}
\hfill
%\begin{figure*} %fig6
\vspace*{1pt}
\begin{minipage}[t]{79.94mm}
\begin{center}
   \mbox{%
\epsfxsize=78.94mm
\epsfbox{ler-4-b.eps}
}
\end{center}
\vspace*{-13pt}
\Caption{Регрессионная зависимость~(\ref{eq9}) среднего рас\-сто\-яния $\mathrm{dist}$ 
от~$\tau$ при фиксированных значениях $10\hm\leq N\hm\leq 7000$:
\textit{1}~--- $N\hm=10$; \textit{2}~--- 100; \textit{3}~--- 1000;
\textit{4}~--- 5000; \textit{5}~--- $N\hm=7000$}
\end{minipage}
\end{figure*}

Таким образом, при фиксированных~$\tau$ из интервала $(2,\,2{,}8]$ сред\-нее рас\-сто\-яние в~графе
возрастает логарифмически с~рос\-том чис\-ла его вершин $N$, так
же как и~рас\-сто\-яние в~асимп\-то\-ти\-ке
(см.\ выраже-\linebreak\vspace*{-12pt}

%\begin{table*}\small  %tabl2
\begin{center}

\vspace*{6pt}

\noindent
\parbox{62mm}{{{\tablename~2}\ \ \small{Значения $a_f$, коэффициенты~$a$ и~$b$ зависимостей вида~(\ref{eq7})
и~коэффициенты детерминации $R^2$ этих уравнений
}}
}

\vspace*{6pt}


{\small \begin{tabular}{|c|c|c|c|c|}
\hline
&&&&\\[-10pt]
$\tau$ & $a_f$ & $a$ & $b$ & $R^2$ \\
 \hline
\hphantom{9}2,01 & 0,209 & 0,967 & $-$0,684 & 0,96 \\
2,1 &   0,408 & 1,151 & $-$1,661 & 0,98 \\
2,2 &   0,586 & 1,344 & $-$2,777 & 0,98 \\
2,3 &   0,800 & 1,541 & $-$3,994 & 0,95 \\
2,4 &   1,096 & 1,573 & $-$4,428 & 0,91 \\
2,5 &   1,565 & 1,430 & $-$4,078 & 0,88 \\
2,6 &   2,459 & 1,076 & $-$2,673 & 0,89 \\
2,7 &   4,942 & 0,760 & $-$1,387 & 0,91 \\
2,8 &   53,870\hphantom{9} & 0,507 & $-$0,352 & 0,94\\
\hline
\end{tabular}
}
\vspace*{3pt}
\end{center}
%\end{table*}

%\vspace*{3pt}

{ \begin{center}  %fig4
 \vspace*{-2pt}
   \mbox{%
\epsfxsize=78.504mm
\epsfbox{ler-3.eps}
}

\end{center}

\noindent
{{\figurename~4}\ \ \small{График экспериментальных
значений $\mathrm{dist}$ от $N$ и~$2\hm<\tau\hm\leq 2{,}8$
}}}

\vspace*{6pt}

\addtocounter{figure}{1}
\addtocounter{table}{1}


\noindent
 ние~(\ref{eq3})).
Сравнение значений коэффициентов~$a$ с~$a_f$ показывает, что для $2{,}01\leq\tau\leq 2{,}4$ значения~$a$ 
выше значений $a_f$, а~при $2{,}5\hm\leq\tau\hm\leq 2{,}8$ ниже (см.\ табл.~2).
Изменение углового коэффициента~$a$ в~данном случае показывает, что сред\-нее расстояние в~графе
с~рос\-том значения~$\tau$ сначала возрастает, достигая максимума в~промежутке от~2,2 до~2,4, 
а~затем убывает. На\-гляд\-но это мож\-но видеть на рис.~4, где показана за\-ви\-си\-мость
экспериментальных значений $\mathrm{dist}$ от~$N$ и~$\tau$.





Поиск зависимости среднего расстояния $\mathrm{dist}$ от переменных $N$ и~$\tau$ 
проходил по аналогии с~предыду\-щим интервалом изменения параметра распределения степеней вершин.
Сначала была по\-стро\-ена за\-ви\-си\-мость в~виде 
$$
\mathrm{dist}=\fr{\ln N}{\ln\left({2\zeta(\tau-1)}/{\zeta(\tau)}-2\right)}+b
$$
и получено сле\-ду\-ющее регрессионное уравнение:
\begin{equation*}
\mathrm{dist} = \fr{\ln N}{\ln\left({2\zeta(\tau-1)}/{\zeta(\tau)}-2\right)}-40{,}936\,.
\end{equation*}
К сожалению, на этом интервале коэффициент детерминации полученной за\-ви\-си\-мости оказался очень низким
($R^2\hm=0{,}0005$), гипотеза о~равенстве~$R^2$ нулю не отвергается и~коэффициент~$b$ не значим.
Следовательно, такую модель не имеет смыс\-ла использовать для прогноза.
Поэтому была осуществлена попытка по\-стро\-ить регрессию вида
$$
\mathrm{dist}=\fr{a\ln N}{\ln\left({2\zeta(\tau-1)}/{\zeta(\tau)}-2\right)}+b
$$ 
и~была получена зависимость
\begin{equation}
\label{eq8}
\mathrm{dist} = 4{,}962 - \fr{0{,}005\ln N}{\ln\left({2\zeta(\tau-1)}/{\zeta(\tau)}-2\right)}
\end{equation}
с коэффициентом детерминации $R^2\hm=0{,}06$. Несмотря на столь низ\-кое значение~$R^2$, гипотеза о~его
равенстве нулю отвергается, однако оценка зна\-чи\-мости коэффициентов~$a$ и~$b$
регрессионного уравнения~(\ref{eq8}) показала, что коэффициент~$a$ статистически значим, тогда как~$b$~-- нет.
Дальнейший поиск наилучшей регрессии привел к~получению сле\-ду\-ющей за\-ви\-си\-мости
сред\-не\-го рас\-сто\-яния конфигурационного графа $\mathrm{dist}$ от $N$ и~$\tau$:
\begin{equation}
\label{eq9}
\mathrm{dist} = \fr{(31{,}706-22{,}076\tau+3{,}841\tau^2)\ln N}{\ln\left({2\zeta(\tau-1)}/{\zeta(\tau)}-2\right)}\,,
\end{equation}
где все коэффициенты модели значимы, а $R^2\hm=0{,}74$. Зависимость~(\ref{eq9}) отражена графически на рис.~5 и~6.

Для моделей~(\ref{eq8}) и~(\ref{eq9}) была оценена зна\-чи\-мость различия между коэффициентами множественной
корреляции этих моделей при 5\%-ном уровне зна\-чи\-мости. В~результате $H_0:r_{(8)}=r_{(9)}$ была отвергнута, т.\,е.\
различие между коэффициентами корреляции оказалось значимым.
Проверка остатков регрессий~(\ref{eq8}) и~(\ref{eq9}) на нормальность показала, что нормальное распределение
имеют только остатки модели~(\ref{eq9}). Кроме того, остаточная сумма квад\-ра\-тов модели~(\ref{eq8})
$\mathrm{SSR}_{(8)}\hm=245793{,}9$ больше, чем $\mathrm{SSR}_{(9)}\hm=102369{,}9$. Поэтому мож\-но сделать вывод о~том, что модель~(\ref{eq9})
лучше подходит для прогноза, чем модель~(\ref{eq8}).
Таким образом, при значениях па\-ра\-мет\-ра $2\hm<\tau\hm\leq 2{,}8$ в~качестве наиболее подходящей модели за\-ви\-си\-мости
среднего рас\-сто\-яния $\mathrm{dist}$ от~$N$ и~$\tau$ в~до\-асимп\-то\-ти\-че\-ской об\-ласти предлагается за\-ви\-си\-мость, опи\-сы\-ва\-емая
уравнением~(\ref{eq9}).



На рис.~5 и~6 линии, находящиеся внут\-ри затененных областей и~отраженные в~легендах, соответствуют
зависимостям $\mathrm{dist}$ от~$N$ (см.\ рис.~5) и~от~$\tau$ (см.\ рис.~6) при некоторых
значениях па\-ра\-мет\-ра $\tau$ или объема графа~$N$ соответственно.
На рис.~5 ниж\-няя граница об\-ласти соответствует $\tau\hm=2{,}01$, верхняя~--- максимуму функции~(\ref{eq9}) 
по параметру $\tau$: $\tau^*\hm=2{,}2288\ldots$, а~кривые зависимостей $\mathrm{dist}$ от~$N$ внут\-ри затененной
об\-ласти расположены сле\-ду\-ющим образом: по воз\-рас\-та\-нию значений $\mathrm{dist}$ при увеличении значений~$\tau$ от~1,1
до~$\tau^*$ и~по убыванию $\mathrm{dist}$ при рос\-те~$\tau$ от~$\tau^*$ до~2,8. А~на рис.~6 кривые
зависимостей $\mathrm{dist}$ от~$\tau$ расположены друг над другом по мере воз\-рас\-та\-ния чис\-ла вершин графа
$10\hm\leq N\hm\leq 7000$ в~пределах граничных значений.

\vspace*{-9pt}


\subsection{Результаты при $\tau=2$}

\vspace*{-2pt}

Заметим, что модели~(\ref{eq6}) и~(\ref{eq9}) не охватывают значение па\-ра\-мет\-ра распределения степеней
вершин $\tau\hm=2$. Однако по экспериментальным данным была по\-стро\-ена сле\-ду\-ющая регрессионная за\-ви\-си\-мость
$\mathrm{dist}$ от~$N$ при фиксированном $\tau\hm=2$ (все коэффициенты модели значимы):
\begin{equation}
\label{eq10}
\mathrm{dist} = 5{,}262\ln\ln N - 4{,}045 \quad \left(R^2=0{,}73\right).
\end{equation}

\vspace*{-14.5pt}

\section{Выводы}

\vspace*{-2.5pt}

Итак, экспериментальные результаты на степенных конфигурационных графах с~фиксированным
па\-ра\-мет\-ром~$\tau$ распределения~(\ref{eq1}) степеней вершин показывают, что на интервале $(1,2)$ 
с~рос\-том объема~$N$ среднее расстояние $\mathrm{dist}$ в~графе рас\-тет как $\ln\ln N$, а~на интервале $2\hm<\tau\hm\leq 2{,}8$
рас\-тет логарифмически в~доасимптотической об\-ласти (при $N\hm\leq 7000$) так же, как это было показано
Р.~Хофстадом~\cite{Hof2} для типичного расстояния в~графах при $N\hm\rightarrow\infty$.
Однако что касается за\-ви\-си\-мости среднего расстояния от переменных~$N$ и~$\tau$, то при малых
объемах графа предлагается использовать модели~(\ref{eq6}) и~(\ref{eq9}) в~соответствующих интервалах
изменения параметра~$\tau$, так как они лучше описывают данную за\-ви\-си\-мость, что было под\-тверж\-де\-но
в~настоящей работе с~по\-мощью методов статистического анализа
и предложена модель~(\ref{eq10}) за\-ви\-си\-мости сред\-не\-го рас\-сто\-яния от чис\-ла вершин~$N$ при $\tau\hm=2$
так\-же для графов в~до\-асимп\-то\-ти\-че\-ской об\-ласти.

{\small\frenchspacing
 {\baselineskip=10.7pt
 %\addcontentsline{toc}{section}{References}
 \begin{thebibliography}{99}
 
 %\vspace*{-6pt}
 
\bibitem{Dur} %1
\Au{Durrett~R.} Random graph dynamics.~--- Cambridge: Cambridge University
Press, 2007. 221~p. doi: 10.1017/ CBO9780511546594.

\bibitem{Hof1} %2
\Au{Hofstad~R.} Random graphs and complex networks.~--- Cambridge:
Cambridge University Press, 2017.  Vol.~1. 337~p. doi: 10.1017/9781316779422.



\bibitem{New1} %3
\Au{Newman~M.\,E.\,J.} Networks. An introduction.~--- Oxford: Oxford University Press, 2010. 772~p.
doi: 10.1093/ acprof:oso/9780199206650.001.0001.

\bibitem{New2} %4
\textit{Newman~M.\,E.\,J.} Networks.~--- 2nd ed.~--- Oxford: Oxford University Press, 2018. 800~p.
doi: 10.1093/oso/ 9780198805090.001.0001.

\bibitem{Hof2} %5
\textit{Hofstad~R.} Random graphs and complex networks~// Notes RGCNII, 2020.  Vol.~2.
314~p. {\sf https://www.win. tue.nl/$\sim$rhofstad/NotesRGCNII.pdf.}

\bibitem{Fa} %6
\Au{Faloutsos~C., Faloutsos~P., Faloutsos~M.} On power-law relationships of
the internet topology~// Comput. Commun. Rev., 1999. Vol.~29. P.~251--262.
doi: 10.1145/ 316194.316229.

\bibitem{RN} %7
\Au{Reittu~H., Norros~I.} On the power-law random graph model of massive
data networks~// Perform. Evaluation, 2004. Vol.~55. Iss.~1-2. P.~3--23.
doi: 10.1016/S0166-5316(03)00097-X.

\bibitem{Bol}
\Au{Bollobas~B.} A~probabilistic proof of an asymptotic formula for the number
of labelled regular graphs~// Eur. J.~Combin., 1980. Vol.~1.
Iss.~4. P.~311--316. doi: 10.1016/S0195-6698(80)80030-8.



\bibitem{Chu} %9
\Au{Chung~F., Lu~L.} The average distances in random graphs with given expected degrees~// 
P.~Natl. Acad. Sci. USA, 2002. Vol.~99. Iss.~25. P.~15879--15882.
doi: 10.1073/pnas.252631999.

\bibitem{Dijk}
\Au{Dijkstra~E.\,W.} A~note on two problems in connexion with graphs~// 
Numer. Math., 1959. Vol.~1. Iss.~1. P.~269--271. doi: 10.1007/BF01386390.
\end{thebibliography}

 }
 }

\end{multicols}

\vspace*{-7pt}

\hfill{\small\textit{Поступила в~редакцию 21.03.22}}

%\vspace*{8pt}

%\pagebreak

\newpage

\vspace*{-28pt}

%\hrule

%\vspace*{2pt}

%\hrule

%\vspace*{-2pt}

\def\tit{AN AVERAGE DISTANCE IN~THE~POWER-LAW CONFIGURATION GRAPHS}


\def\titkol{An average distance in~the~power-law configuration graphs}


\def\aut{M.\,M.~Leri}

\def\autkol{M.\,M.~Leri}

\titel{\tit}{\aut}{\autkol}{\titkol}

\vspace*{-8pt}


\noindent
Institute of Applied Mathematical Research of the Karelian Research Center of the Russian Academy of Sciences, 
11~Pushkinskaya Str., Petrozavodsk 185910, Russian Federation

\def\leftfootline{\small{\textbf{\thepage}
\hfill INFORMATIKA I EE PRIMENENIYA~--- INFORMATICS AND
APPLICATIONS\ \ \ 2023\ \ \ volume~17\ \ \ issue\ 1}
}%
 \def\rightfootline{\small{INFORMATIKA I EE PRIMENENIYA~---
INFORMATICS AND APPLICATIONS\ \ \ 2023\ \ \ volume~17\ \ \ issue\ 1
\hfill \textbf{\thepage}}}

\vspace*{3pt} 


\Abste{In random configuration graphs with a~discrete power-law vertex degree distribution with a~fixed parameter, 
the average distance in the graph is considered, i.\,e., the arithmetic mean of distances between all pairs of graph nodes. 
This characteristic is estimated using simulation methods. Due to computational constraints, the author considers graphs
 in the pre-asymptotic domain (in this paper, these are the graphs up to 7000~nodes). The models of dependencies of the average distance on 
 the graph size and the parameter of vertex degree distribution are reseived. The obtained results are compared with the results 
 of theoretical studies of the typical distance in a graph in the asymptotics (i.\,e., when the number of graph vertices tends to infinity), 
 given in the works by R.~Hofstad.}

\KWE{configuration graph; power-law distribution;
average distance in a graph; simulation}



\DOI{10.14357/19922264230104} 

\vspace*{-16pt}

\Ack
\noindent
The study was carried out under state order to the Karelian Research Center 
of the Russian Academy of Sciences (Institute of Applied Mathematical Research KarRC RAS).

\vspace*{6pt}

  \begin{multicols}{2}

\renewcommand{\bibname}{\protect\rmfamily References}
%\renewcommand{\bibname}{\large\protect\rm References}

{\small\frenchspacing
 {%\baselineskip=10.8pt
 \addcontentsline{toc}{section}{References}
 \begin{thebibliography}{99} 
\bibitem{1-leri-1}
\Aue{Durrett, R.} 2007. \textit{Random graph dynamics.} Cambridge: Cambridge University
Press. 221~p. doi: 10.1017/ CBO9780511546594.

\bibitem{2-leri-1}
\Aue{Hofstad, R.} 2017. \textit{Random graphs and complex networks.} Cambridge:
Cambridge University Press.   Vol.~1. 337~p. doi: 10.1017/9781316779422.

\bibitem{4-leri-1} %3
\Aue{Newman, M.\,E.\,J.} 2010. \textit{Networks. An introduction.} Oxford: Oxford
University Press. 772~p. doi: 10.1093/ acprof:oso/9780199206650.001.0001.

\bibitem{5-leri-1} %4
\Aue{Newman, M.\,E.\,J.} 2018. \textit{Networks.} 2nd ed. Oxford: Oxford
University Press. 800~p. doi: 10.1093/oso/ 9780198805090.001.0001.

\bibitem{3-leri-1} %5
\Aue{Hofstad, R.} 2020. Random graphs and complex networks.  \textit{Notes RGCNII}. Vol.~2. 314~p.
Available at: {\sf https:// www.win.tue.nl/$\sim$rhofstad/NotesRGCNII.pdf} (accessed January~18, 2023)



\bibitem{6-leri-1}
\Aue{Faloutsos, C., P.~Faloutsos, and M.~Faloutsos.} 1999. On power-law relationships of
the internet topology. \textit{Comput. Commun. Rev.} 29:251--262.
doi: 10.1145/316194.316229.

\bibitem{7-leri-1}
\Aue{Reittu, H., and I.~Norros.} 2004. On the power-law random graph model of massive data
networks. \textit{Perform. Evaluation} 5(1-2)5:3--23.
doi: 10.1016/S0166-5316(03)00097-X.

\bibitem{8-leri-1}
\Aue{Bollobas, B.} 1980. A~probabilistic proof of an asymptotic formula for the number
of labelled regular graphs. \textit{Eur. J.~Combin.} 1(4):311--316.
doi: 10.1016/S0195-6698(80)80030-8.

\bibitem{9-leri-1}
\Aue{Chung, F., and L.~Lu.} 2002. The average distances in random graphs with given expected
degrees. \textit{P.~Natl. Acad. Sci. USA} 99(25):15879--15882.
doi: 10.1073/pnas.252631999.

\bibitem{10-leri-1}
\Aue{Dijkstra, E.\,W.} 1959. A~note on two problems in connexion with graphs.
\textit{Numer. Math.} 1(1):269--271. doi: 10.1007/BF01386390.
\end{thebibliography}

 }
 }

\end{multicols}

\vspace*{-6pt}

\hfill{\small\textit{Received March 21, 2022}}


\Contrl

\noindent
\textbf{Leri Marina M.} (b.\ 1969)~--- Candidate of Science (PhD) in technology, scientist,
Institute of Applied Mathematical Research of the Karelian Research Center of the Russian Academy of Sciences,
11~Pushkinskaya Str., Petrozavodsk 185910, Russian Federation; \mbox{leri@krc.karelia.ru}


\label{end\stat}

\renewcommand{\bibname}{\protect\rm Литература}        %4+
\def\stat{strijov}

\def\tit{ВОССТАНОВЛЕНИЕ МАТРИЦЫ СУПЕРПОЗИЦИИ В~ЗАДАЧЕ~СИМВОЛЬНОЙ РЕГРЕССИИ$^*$}

\def\titkol{Восстановление матрицы суперпозиции в~задаче символьной регрессии}

\def\aut{Р.\,Г.~Нейчев$^1$, И.\,А.~Шибаев$^2$, В.\,В.~Стрижов$^3$}

\def\autkol{Р.\,Г.~Нейчев, И.\,А.~Шибаев, В.\,В.~Стрижов}

\titel{\tit}{\aut}{\autkol}{\titkol}

\index{Нейчев Р.\,Г.}
\index{Шибаев И.\,А.}
\index{Стрижов В.\,В.}
\index{Neychev R.\,G.}
\index{Shibaev I.\,A.}
\index{Strijov V.\,V.}


{\renewcommand{\thefootnote}{\fnsymbol{footnote}} \footnotetext[1]
{Работа выполнена при поддержке РФФИ (проекты 20-37-90050 и~20-07-00990).}}


\renewcommand{\thefootnote}{\arabic{footnote}}
\footnotetext[1]{Московский физико-технический институт, 
\mbox{neychevr@gmail.com}}
\footnotetext[2]{Московский физико-технический институт, 
\mbox{shibaev.kesha@gmail.com}}
\footnotetext[3]{Федеральный исследовательский центр <<Информатика 
и~управ\-ле\-ние>> Российской академии наук, \mbox{strijov@phystech.edu}}

\vspace*{-12pt}
 



\Abst{Исследуется проблема порождения структуры регрессионной модели. 
Модель представляет собой суперпозицию базовых функций. Структура модели 
описывается взвешенным цвет\-ным графом. Каждая вершина графа соответствует 
некоторой базовой функции. Ребро задает суперпозицию двух функций. Вес ребра 
равен вероятности суперпозиции. Для создания оптимальной модели необходимо 
восстановить ее структуру по матрице смежности графа. Пред\-ла\-га\-емый алгоритм 
восстанавливает минимальное остовное дерево из взвешенного цветного графа. 
Пред\-став\-ле\-но новое решение, основанное на алгоритме дерева Штейнера. 
Алгоритм сравнивается с~альтернативами.}


\KW{символьная регрессия; линейное программирование; 
суперпозиция; минимальное остовное дерево; мат\-ри\-ца смеж\-ности}

\DOI{10.14357/19922264230105} 
  
\vspace*{-8pt}


\vskip 10pt plus 9pt minus 6pt

\thispagestyle{headings}

\begin{multicols}{2}

\label{st\stat}

\section{Введение}

Символьная регрессия~--- это метод по\-стро\-ения нелинейной модели, 
аппроксимирующей выборку. Структура модели определяется суперпозицией базовых 
функций. Набор базовых функций фиксируется для конкретной прикладной задачи. 
Структуры альтернативных моделей генерируются алгоритмом оптимизации для выбора 
оптимальной модели. В данной статье предлагается определять структуру модели 
с~по\-мощью вероятностного графа. Остовное дерево в~графе определяет некоторую 
суперпозицию. Для выбора оптимальной модели необходимо реконструировать 
минимальное остовное дерево по графу.

Методы генетического программирования~\cite{koza1992genetic} находят оптимальное 
подмножество в~наборе суперпозиций базовых функций, но имеют высокую 
вычислительную сложность. В~\cite{searson2010gptips} описаны методы, понижающие 
сложность. Они используют дополнительные ограничения на суперпозиции, например 
используют линейные комбинации базовых функций. Символьная регрессия, 
описанная~в~\cite{stanley2002evolving}, используется для оптимизации структуры 
суперпозиции. Методы решения задачи символьной регрессии основаны на матричном 
представлении структуры модели~\cite{bochkarev2017generation}. Однако эти методы 
не содержат ограничений на чис\-ло аргументов базовых функций и~на структуру 
графа, обеспечивающую допустимую суперпозицию. В~данной работе решается задача 
построения модели с~помощью символьной регрессии.

Требуется восстановить допустимую суперпозицию из предсказанной мат\-ри\-цы 
смежности с~вероятностями ребер. Решается задача вос\-ста\-нов\-ле\-ния~$k$-минимального 
остовного дерева $k$-MST (\textit{англ.}\ Minimum-cost Spanning Tree). Эта задача NP-слож\-ная, 
поэтому применимы только при\-бли\-жен\-ные решения~\cite{ravi1996spanning}. 
Алгоритм~$k$-MST эквивалентен проб\-ле\-ме дерева Штейнера PCST (\textit{англ.}\ 
Prize-Collecting Steiner Tree) из-за его эквивалентности ослабленной формулировке 
постановки задачи линейного программирования~\cite{chudak2004approximate}. 
В~работах~\cite{ravi1996spanning,awerbuch1998new,arora20062+} пред\-став\-ле\-ны 
приближенные решения задачи \mbox{$k$-MST}.



Предлагаемое решение основано на упрощенной версии задачи~$k$-MST, которая 
трансформируется в~задачу PCST с~постоянными призами, одинаковыми для всех 
вершин. Быст\-рый алгоритм PSCT описан в~\cite{hegde2014fast}. Альтернативное 
решение основано на алгоритме~$(2-\varepsilon)$-аппроксимации для задачи PSCT. 
Она сравнивается с~другими алгоритмами, включая алгоритмы обхода дерева в~глубину, обхода дерева в~ширину, алгоритмы Прима.

\begin{table*}[b]\small  %tabl1
\vspace*{-12pt}
\begin{center}
        \parbox{262pt}{\Caption{Вероятности суперпозиций в~матрице смежности порождают 
ориентированный граф}

}
    \label{restored_adjacency_matrix}
\vspace*{2ex}

        \begin{tabular}{|c|c|ccccccc|}
            \hline
            Арность&Функция&$\ast$&$+$&$\ln$&$\sin$&$\times$&$\exp$&$x$\\
            \hline
            $1$&$\ast$ &0,2&{\bf 0,7}&0,5&0,4&0,5&0,3&0,2\\
            $3$&$+$    &0,3&0,2&{\bf 1,0}&{\bf 0,8}&0,6&0,3&{\bf 0,7}\\
            $1$&$\ln$  &0,3&0,2&0,0&0,0&0,1&0,5&{\bf 0,5}\\
            $1$&$\sin$ &0,1&0,4&0,0&0,5&{\bf 0,9}&0,2&0,5\\
            $2$&$\times$&0,3&0,0&0,3&0,5&0,0&{\bf 0,8}&{\bf 0,6}\\
            $1$&$\exp$ &0,3&0,3&0,4&0,1&0,5&0,4&{\bf 0,4}\\
            \hline
        \end{tabular}
\end{center}
\end{table*}

\vspace*{-12pt}


\section{Задача выбора регрессионной модели}

\vspace*{-3pt}

Требуется выбрать регрессионную модель~$\varphi$ из набора альтернативных 
моделей. Модель описывает выборку~$D=\{(x_i,y_i)\}$ и~минимизирует ошибку

\noindent
\begin{equation}
\hat{\varphi}(D)=\mathop{\argmin}\limits_\varphi\sum\limits_{i=1}^m\left(\varphi(x_i)-
y_i\right)^2.
\label{task_1}
\end{equation}
Модель представляет собой суперпозицию базовых функций из некоторого заданного 
набора. На рис.~1\linebreak\vspace*{-12pt}

{ \begin{center}  %fig1
 \vspace*{-3pt}
    \mbox{%
\epsfxsize=37.447mm
\epsfbox{str-1.eps}
}

\end{center}

\vspace*{-2pt}

\noindent
{{\figurename~1}\ \ \small{Структура регрессионной модели представляет собой ориентированный 
граф
}}}

\vspace*{6pt}

\addtocounter{figure}{1}


\noindent
 показан ее пример. Структура модели~$\varphi$, 
суперпозиция, соответствует графу~$G=(V,E)$, где базовые функции находятся 
в~вершинах~$V$. {Корневая} вершина обозначается через~$\ast$. Модель:

\vspace*{1pt}

\noindent
$$
\varphi(D) =  \ln(x) + x + \sin\left( x\times \exp(x)\right).
$$

\vspace*{-4pt}

\noindent
 Еe структура в~виде матрицы 
смежности графа пред\-став\-ле\-на~в табл.~\ref{restored_adjacency_matrix}.
Базовые функции перечислены в~первой строке. Элементами матрицы являются 
вероятности ребер~$E$ дерева. Жир\-ным шриф\-том выделены ребра восстановленного 
дерева~$M$, образующие суперпозицию~$\varphi$. Для восстановления структуры 
модели~$\varphi$ как суперпозиции, заданной деревом~$M$, необходимы только 
графовое пред\-став\-ле\-ние~$G$~и~базовые функции.



Поставим задачу восстановления структуры модели. Задано множество 
выборок~$\{D_k\}$. Каждой выборке~$D_k$ соответствует своя модель. Эта модель 
имеет структуру~$M_k$. Таким образом, имеется набор пар $\{(D_k, M_k)\}$, 
выборка и~структура.
Обозначим через~$P$ отображение, которое предсказывает вероятности узлов 
в~графе~$G$ по выборке~$D$. Для выбора модели~$\varphi(D)$ необходимо восстановить 
структуру модели~$M$ по графу~$G$. Обозначим алгоритм восстановления дерева 
через~$R$. Регрессионная модель~$\hat{\varphi}(D)$, которая решает 
задачу~(\ref{task_1}), определяется формулой
$
\hat{M}=R\left(P(D)\right).
$
Поскольку дерево~$M$ играет центральную роль в~этой работе, критерий качества 
алгоритма восстановления дерева имеет вид:


\vspace*{-3pt}

\noindent
$$
\min_{M_k \in G} \fr{1}{K}\sum\limits_{k=1}^K \left[ \hat{M_k} = M_k\right].
$$

\vspace*{-4pt}

\noindent
Восстановленное дерево должно быть эквивалентно заданному дереву, следовательно, 
выбранная модель регрессии при\-бли\-жа\-ет выборку.

\vspace*{-10pt}

\section{Задача восстановления дерева суперпозиции}

\vspace*{-3pt}

Требуется восстановить дерево~$M_k$, задающее  суперпозицию и~решающее 
задачу~(\ref{task_1}). Задан ориентированный взвешенный граф~$G\hm=(V,E)$ 
с~раскрашенными вершинами~$v_i$ и~корневой вершиной~$r$. Каждая вершина~$v_i \hm\in 
V$ имеет свой цвет~$t(v_i)\hm=t_i$. Каждое реб\-ро~$e_i\in E$ имеет свой 
вес~\mbox{$w(e_i)\hm=c_i\hm\in[0,1]$}.

Требуется восстановить ориентированное дерево минимального веса с~корнем~$r$. 
Оно должно покрывать не менее~$k$ вершин в~заданном графе~$G$. Чис\-ло ребер, 
выходящих из вершины~$v_i$ дерева, должно быть меньше или равно~$t_i$. 
Корень~$r$ имеет одно ребро,~$t_r=1$.

Сформулируем это условие в~виде задачи линейного программирования 
с~целочисленными ограничениями:

\vspace*{-5pt}

\noindent
\begin{multline}
\underset{\substack{{x_e, z_S} \\ e\in E,\\ S\subseteq V\backslash 
\{r\}}}{\mbox{minimize}}  \displaystyle \sum\limits_{e\in E}c_ex_e \\[-3pt]
\mbox{s.t.}\  \displaystyle  \sum\limits_{\substack{{e\in\delta(S):}\\ e=(\ast,v_i),\\ v_i\in\delta(S)}} \!\!\!\! x_e + 
\sum\limits_{T:T\supseteq S}  \!\!\!\! z_T\geqslant 1,\enskip  S\subseteq 
V\backslash \{r\};\\[-3pt]
 \displaystyle \sum\limits_{e\in E:~e=(\ast,v)} \! x_e\leqslant 1,\enskip v\in V;\\[-3pt]
 \displaystyle \sum\limits_{e\in E:~e=(v,\ast)}x_e\leqslant t_i,\enskip  v\in V;\\[-3pt]
 \displaystyle \sum\limits_{S\subseteq V\backslash \{r\}}|S|z_S \leqslant n-k,\enskip  x_e\in\{0,1\},\enskip 
 z_S\in\{0,1\},\\[-3pt]
  e\in E,\enskip   S\subseteq V\backslash \{r\},
\label{ilp_our}
\end{multline}
где
$$
x_e =\begin{cases}
 1, &\mbox{если\ ребро}\ e\ \mbox{входит\ в~финальную}\\
 &\mbox{суперпозицию};\\
 0 & \mbox{в~противном\ случае};
 \end{cases}
 $$
  $z_S\hm = 1$ для всех вершин, исключенных из финальной 
суперпозиции. Обозначим через~$e\hm=(\ast, v)$ ориентированное ребро с~листом~$v$. 
Обозначим через $e\hm=(v, \ast)$ ориентированное ребро с~вершиной~$v$.

Первое ограничение~(\ref{ilp_our})  определяет структуру графа решения в~виде 
дерева с~корнем~$r$. Второе ограничение определяет ориентацию дерева: каждая 
вершина имеет не более одного входящего ребра. Третье ограничение определяет 
арность используемых базовых функций, поэтому число ребер, имеющих определенную 
вершину в~качестве источника, фиксировано. Четвертое ограничение говорит, что 
итоговое дерево имеет не менее~$k$ вершин. Если все веса неотрицательны, то 
четвертое ограничение на минимальное число вершин принимает более строгий вид: 
число вершин должно быть равно~$k$. Однако более слабое ограничение позволяет 
найти возможные связи с~другими оптимизационными задачами. Ограничения 
в~(\ref{ilp_our}) преследуют ту же цель.

\vspace*{-9pt}

\section{Алгоритмы восстановления дерева $k$-MST и~PCST}

\vspace*{-3pt}

\noindent
\textbf{Определение~1} (\textbf{$\bm{k}$-минимальное остовное дерево,\linebreak $\bm{k}$-MST}).
Задан взвешенный граф~$G\hm=(V,E)$ с~корнем~$r$ и~весами ребер~$w(e_i)\hm=c_i\hm\geqslant 
0$, $e_i\hm\in E$. Требуется построить ориентированное дерево минимального веса 
с~корнем~$r$, покрывающее не менее~$k$ вершин в~$G$.

\smallskip

Если та же задача ставится для ориентированных графов, то конечное дерево 
с~корнем~$r$ должно быть ориентированным. Задача линейного программирования для 
направленного~$k$-MST исключает \mbox{третье} условие в~(\ref{ilp_our}).
В~таком виде задача~$k$-MST отличается от исходной задачи восстановления 
дерева суперпозиций~(\ref{ilp_our}) отсутствием третьего ограничения на арность 
базовых функций. Это эквивалентно ограничению на число ребер, выходящих из 
вершины.

\smallskip

\noindent
\textbf{Определение~2} (\textbf{призовое дерево Штейнера, $\text{PCST}$}).\linebreak
Задан взвешенный граф $G\hm=(V,E)$ с~корнем~$r$ и~весами ребер~$w(e_i)\hm=c_i\hm\geqslant  0$, $e_i\hm\in E$, где каждой вершине~$v_i \hm\in V$ присвоен 
{приз} $\pi(v_i)\hm=\pi_i\geqslant 0$. Требуется построить дерево~$T$ с~корнем~$r$, 
которое \mbox{минимизирует} функционал
$\sum\nolimits_{e\in E}c_ex_e \hm+ \sum\nolimits_{S\subseteq V\backslash\{r\}} 
\pi(S)z_S,$
где~$x_e\in\{0, 1\}$, $x_e\hm=1$, если~$e\hm\in E$ входит в~тройку~$T$; $z_S\hm\in\{0, 1\}$, 
$z_S\hm=1$ для всех вершин, исключенных из дерева~$T$; $S \hm= V\backslash V(T)$; $\pi(S)\hm= \sum\nolimits_{v\in S}\pi(v)$.

\smallskip

В случае ориентированных графов эта задача обобщается до~асимметричной задачи 
A-PCST. Задача линейного программирования для~A-PCST принимает вид:

\vspace*{-4pt}

\noindent
\begin{multline}
\underset{\substack{x_e,z_S \\ e\in E,\\ S\subseteq V\backslash \{r\}}}{\mbox{minimize}} 
\displaystyle \sum\limits_{e\in E} c_e x_e + \sum\limits_{S\subseteq V\backslash\{r\}}  \!\!\!\!\!\pi(S)z_S \\
\mbox{s.t.}\ \displaystyle \sum\limits_{\substack{e\in\delta(S):\\e=(\ast,v_i),\\ v_i\in\delta(S)}} \!\!\!\!\!\! x_e + 
\sum\limits_{T:T\supseteq S}  \!\!\! z_T\geqslant 1,\enskip  S\subseteq  V\backslash \{r\};\\
\displaystyle \sum\limits_{e\in E:~e=(\ast,v)}\!\!\!\!  x_e\leqslant 1,\enskip
x_e\in\{0,1\},\enskip z_S\in\{0,1\},\enskip  v\in V,\\
e\in E,\enskip S\subseteq V\backslash \{r\}.
\label{ilp_pcst_ord}
\end{multline}

\vspace*{-3pt}

\noindent
Если последнее ограничение из~(\ref{ilp_our}) входит в~оптимизируемый 
функционал, задачи $k$-MST и~A-PCST имеют эквивалентные 
ограничения и~отличаются только оптимизируемым функционалом. Такое 
преобразование возможно согласно условиям Ка\-ру\-ша--Ку\-на--Так\-ке\-ра~\cite{ras2017approximate}. Если значения призов 
эквивалентны $\pi(v) \hm=  \lambda$, единственное отличие состоит в~постоянном члене~$\lambda(n\hm-k)$. Таким 
образом, задачи оптимизации~$k$-MST и~A-PCST принимают вид:

\vspace*{-4pt}

\noindent
\begin{align*}
\underset{\substack{x_e,z_S \\ e\in E,\\ S\subseteq V\backslash \{r\}}}{\mbox{minimize}} & 
\sum\limits_{e\in E}c_ex_e + \lambda\left(\sum\limits_{S\subseteq V\backslash \{r\}}|S|z_S - (n-k)\right);\\ 
\underset{\substack{x_e,z_S \\ e\in E,\\ S\subseteq V\backslash \{r\}}}{\mbox{minimize}} & 
\sum\limits_{e\in E}c_ex_e + \lambda\sum\limits_{S\subseteq V\backslash\{r\}}|S|z_S\,. 
\end{align*}
%
Константа~$\lambda$ обозначает неотрицательный множитель Лагранжа в~задаче~$k$-MST и~приз за вершину\linebreak 
в~задаче~A-PCST. 
Существуют несколько алгоритмов для решения проблемы~PCST, но не для 
решения проб\-ле\-мы A-PCST. Возможное решение~--- снять 
ограничения на ориентацию графа, чтобы\linebreak алгоритм~PCST мог позже 
восстановить ориентацию дерева.

\vspace*{-9pt}

\section{Решение задачи восстановления ограниченного леса с~помощью алгоритма 
$(2-\varepsilon)$-приближения}

\vspace*{-3pt}

Обзор методов решения задачи восстановления ограниченного леса представлен 
в~\cite{goemans1995general}. Задан взвешенный неориентированный граф~$G\hm=(V,E)$. 
Все его веса~$w(e_i)\hm=c_i\geqslant 0$, $e_i\hm\in E$. Задана некоторая 
функция~$f:2^{V}\to \{0, 1\}$. Требуется решить задачу линейного 
программирования с~целочисленными ограничениями:

\vspace*{-4pt}

\noindent
\begin{multline}
\underset{x_e:~e\in E}{\mbox{minimize}} \displaystyle \sum\limits_{e\in E}c_ex_e\\
\mbox{s.t.}\  x\left(\delta(S)\right)\geqslant f(S),\enskip  S \subset V, \enskip S \not= \emptyset,\\
 x_e\in\{0,1\},\enskip  e\in E.
\label{ilp_cfp}
\end{multline}

\vspace*{-3pt}

\noindent
Здесь
$$
x(\delta(S))=\sum\limits_{e\in \delta(S)}x_e,
$$
где $x_e\hm=1$, если 
ребро~$e$ входит в~финальное решение. Функция~$\delta(S)$ обозначает все ребра 
из~$E$ такие, что только одна из смежных вершин входит в~$S$.

Предположим, что отображение~$f$ удовлетворяет условиям

\vspace*{-3pt}

\noindent
\begin{gather*}
f(V) = 0,\\
 \underbrace{f(S)=f(V\backslash S)}_{\mathrm{симметричность}},\\
\underbrace{A,B\!\subset\! V\!: A\!\cap\! B\! =\! \emptyset, f(A)\!=\!f(B)\!=\!0\!\to\! f(A\!\cup\! B)\! =\! 0}_{\mathrm{дизъюнктивность}}.
\end{gather*}

\vspace*{-2pt}

\noindent
При выполнении этих условий~$f$ задает число ребер, начинающихся в~множестве 
вершин~$S$.

\smallskip

\noindent
\textbf{Лемма 1.}
\textit{Пусть $B\subseteq S\subset V$. Тогда $f(S) \hm= 0$ и~$f(B) \hm= 0$ приводит к}~$f(S\backslash B) \hm= 0$.

\smallskip

Задача с~таким описанием относится к~\textit{задачам поиска оптимального леса с~ограничениями}. 
Такая постановка задачи~(\ref{ilp_cfp}) с~соответствующим 
отображением~$f$ подходит для многих известных задач взвешенных графов, 
например: минимальный магистральный поиск, $st$-путь, задача Штейнера на 
минимальном дереве. Последняя задача является NP-полной, поэтому применим 
приближенный алгоритм.

\smallskip

\noindent
\textbf{Определение 3} (\textbf{алгоритм $\bm{\alpha}$-аппроксимации}).
Эвристический полиномиальный алгоритм, дающий\linebreak решение некоторой задачи 
оптимизации, называется $\alpha$-ап\-прок\-си\-ма\-ци\-ей, если он гарантирует 
удовлетворяющее ограничениям решение этой задачи оптимизации с~коэффициентом, 
меньшим или равным~$\alpha$, так что решение отличается от оптимального не более 
чем в~$\alpha$ раз по оптимизируемому функционалу.


\smallskip

Чтобы предложить приближенный алгоритм, целочисленные ограничения 
в~(\ref{ilp_cfp}) должны быть ослаблены:

\vspace*{-3pt}

\noindent
\begin{multline*}
\underset{x_e:~e\in E}{\mbox{minimize}}\  \displaystyle \sum\limits_{e\in E}c_ex_e \\
\mbox{s.t.}\  \displaystyle \sum\limits_{e\in \delta(S)}x_e\geqslant f(S),\enskip S \subset V\,, \enskip S \not= \emptyset\,,\\
 x_e>0,\enskip  e\in E,
%\label{rlp_cfp}
\end{multline*}
Двойственная задача принимает вид:

\vspace*{-4pt}

\noindent
\begin{multline}
\underset{y_S:~S \subset V, \; S \not= \emptyset}{\mbox{maximize}}\  
\displaystyle \sum\limits_{S\subset V}f(S)y_S \\
\mbox{s.t.}\  \displaystyle \sum\limits_{S:~e\in \delta(S)}y_S\leqslant c_e,\enskip  e\in E\,,\\
 y_S>0,\enskip  S \subset V, \enskip S \not= \emptyset\,,
\label{rd_cfp}
\end{multline}

\vspace*{-3pt}

\noindent
относительно дополнительного условия
$$
y_S \left(\sum\limits_{e\in \delta(S)}x_e - f(S)\right) = 0\,,\enskip S\subset  V\,.
$$

Обозначим множество вершин $A=\{v\hm\in V: f(\{v\})\hm=1\}$. Предлагается адаптивный 
жадный алгоритм $\left(2-{2}/{\vert A\vert }\right)$-ап\-прок\-си\-ма\-ции для задач 
вида~(\ref{ilp_cfp}). Алгоритм состоит из двух этапов. На первом этапе он жадно 
объединяет кластеры вершин, увеличивая двойственные переменные~$y_S$. Изначально 
каждая вершина принадлежит своему клас\-те\-ру. Если сле\-ду\-ющее реб\-ро~$e$ достигает 
равенства в~ограничениях в~(\ref{rd_cfp}), это ребро добавляется к~множеству~$S$ и~связанные клас\-те\-ры объединяются. Этот этап аналогичен алгоритму минимального 
остовного дерева Крускала. На втором этапе из конечного множества~$S$ удаляются 
некоторые ребра. Если обрезка ребра не нарушает ограничений, то это реб\-ро должно 
быть удалено.


Индекс $Z_{\mathrm{DRLP}}$ в~алгоритме~1 обозначает линейное 
программирование с~двойной релаксацией. Начальное значение $F:=\emptyset$ 
в~алгоритме~1 эквивалентно предположению $x_e \hm= 0$, $ e \hm\in E$. 
По условиям нежесткости $y_S \hm= 0$, $S \hm\subset V$,  $S \hm\not= \emptyset$.

На каждом шаге алгоритма кластер $\mathcal{C}$ содержит две компоненты 
$\mathcal{C} \hm= \mathcal{C}_i \hm\cup \mathcal{C}_a$, где $C\hm\in\mathcal{C }_a$, если 
$f(C) \hm= 1$, и~$C\hm\in\mathcal{C}_i$ в~противном случае. Назовем~$\mathcal{C}_a$ 
активным компонентом.
Переменные~$d(v)$ в~этом алгоритме связаны с~переменными~$y_S$ из~(\ref{rd_cfp}) 
соотношением
$$
d(i) = \sum\limits_{S:i\in S}y_S.
$$ 

Рассмотрим две различные компоненты $C_q$ и~$C_p$, $C_q\cap C_p\hm=\emptyset$, на 
некоторой итерации первого этапа алгоритма. Все~$y_S$ должны быть равномерно 
распределены по некоторому~$\varepsilon$ без нарушения ограничений
$$
\sum\limits_ {S:~e\in \delta(S)}y_S\leqslant c_e. 
$$
В терминах $d(v)$ это условие принимает вид:
$$
\sum\limits_{S:~e\in \delta(S)}y_S = d\left(v_1\right)+d\left(v_2\right),\enskip e=\left( v_1,v_2\right),
$$
поэтому $y_S\hm=0$ для любого~$S$ такого, что $v_1, v_2\hm\in S$, потому что 
компоненты растут только на первом этапе. Увеличение некоторых компонент на~$\varepsilon$ приводит к~уравнению
$$
d(v_1)+d(v_2)+\varepsilon \left(f(C_q)+f(C_p)\right)\leqslant 
c_e,\ e=\left(v_1,v_2\right), 
$$
что приводит к~формуле, используемой в~строке~$10$ алгоритма~1. 
В~случае когда в~состав входит следующее ребро, сумма $\sum\nolimits_{S:~e\in 
\delta (S)}y_S$ не будет увеличиваться, поэтому ограничения выполняются.

Ребра, которые можно удалить из~$F$ без добавления новых активных компонентов, 
удаляются на втором этапе алгоритма. Следующая лемма определяет свойства 
компонент связ\-ности в~$F'$.


\smallskip

\noindent
\textbf{Лемма~2.}\
\textit{Для каждой компоненты связ\-ности~$N$ из~$F'$ выполняется равенство}: $f(N)\hm=0$.

\smallskip

Следующая теорема утверж\-да\-ет, что решение, полученное с~помощью описанного 
алгоритма, удовле\-тво\-ря\-ет ограничениям исходной задачи линейного 
программирования.

\smallskip

\noindent
\textbf{Теорема~1.}
\textit{Набор ребер $F'$, полученный алгоритмом~$1$, удовлетворяет всем 
ограничениям исходной задачи}~(\ref{ilp_cfp}).


\smallskip

\noindent
\textbf{Лемма~3.}\
\textit{Обозначим граф $H$, каждая вершина которого соответствует одной из компонент 
связ\-ности $C\in\mathcal{C}$ на фиксированном шаге алгоритма. Ребро $(v_1,v_2)$ 
присутствует, если существует ребро $\hat{e}$ исходного графа, входящее в~$F'$: 
$\hat{e} \in F'$, поэтому граф $H$~--- это лес. Внут\-ри $H$ нет листовых вершин, 
со\-от\-вет\-ст\-ву\-ющих неактивным вершинам исходного графа}.

\smallskip

\noindent
\textbf{Теорема 2.}
\textit{Алгоритм~$1$ представляет собой $\alpha$-при\-бли\-жен\-ный алгоритм для 
задачи}~(\ref{ilp_cfp}) \textit{с}~$\alpha \hm= 2 - {2}/{|A|}$, \textit{где} $A\hm=\{v\  V: 
f(\{v\})=1\}$.

\smallskip

Несмотря на эту теоретическую основу, не существует подходящей функции $f$ для 
постановки задачи PCST, указанной в~(\ref{ilp_cfp}). Чтобы быть 
применимым в~этих условиях, алгоритм~1 нуждается в~нескольких 
модификациях.

\vspace*{-9pt}

\section{Модифицированная постановка задачи для~PCST}

\vspace*{-3pt}

Как и~в случае A-PCST, упрощенный вид задачи линейного 
программирования PCST принимает вид:
\begin{multline*}
\underset{\substack{x_e,s_v \\ e\in E, v\in V\backslash \{r\}}}{\mbox{minimize}}\  
\displaystyle \sum\limits_{e\in E}c_ex_e + \sum\limits_{v\in V\backslash\{r\}} \left(1-s_v\right)\pi_v \\
\mbox{s.t.}\  \displaystyle \sum\limits_{e\in\delta(S)} \!\! x_e\geqslant s_v,\enskip S\subseteq V\backslash \{r\},\enskip v\in S,\\
x_e\geqslant 0,\enskip e\in E,\enskip s_v\geqslant 0,\enskip v\in V\backslash \{r\}.
%\label{rlp_pcst_inord}
\end{multline*}
Эта постановка задачи отличается от исходной~(\ref{ilp_pcst_ord}) тем, что с~ней 
возможно согласовать задачу $k$-MST. Индикаторы~$s_v$ показывают, что 
вершина~$v$ включена в~дерево.

Двойственная задача принимает вид:

\vspace*{-3pt}

\noindent
\begin{multline*}
\underset{\substack{y_S:~S\subset V\backslash\{r\}}}{\mbox{maximize}}\ 
\displaystyle \sum\limits_{S\in V\backslash\{r\}}y_S \\
\mbox{s.t.}\  \displaystyle \sum\limits_{S:e\in\delta(S)}y_S\leqslant c_e ,\enskip e\in E;\\
 \displaystyle \sum\limits_{S\subseteq T}y_S\leqslant \sum\limits_{v\in T}\pi_v,\enskip  T\subset  V\backslash\{r\},\\
 y_S\geqslant 0,\enskip  S\subset V\backslash\{r\}.
%\label{rd_pcst_inord}
\end{multline*}

\vspace*{-3pt}

Алгоритм~2 решает эту задачу. Он похож на 
алгоритм~1. Двойные переменные должны обновляться равномерно 
с~дополнительными ограничениями. Тогда~$\varepsilon$ примет минимальное из двух 
значений в~соответствии с~обеими группами ограничений.
Более широкий анализ аппроксимационных свойств обновленного алгоритма 
представлен в~\cite{goemans1995general}. Алгоритм~2 представляет 
собой $\alpha$-приближенный алгоритм для задачи PCST с~$\alpha \hm= 2 \hm- 
{2}/({n-1})$, где $n$~--- число вершин в~графе~$G$.

\vspace*{-9pt}

\section{Вычислительный эксперимент}

\vspace*{-3pt}

Основная цель эксперимента~--- восстановить дерево суперпозиции. Алгоритмы, 
используемые для восстановления, перечислены ниже.

\vspace*{-14pt}

\paragraph*{DFS, BFS.}
Алгоритмы жадного дерева обхода в~глубину и~жадного дерева обхода в~ширину. 
Обход ребер с~наибольшим весом эквивалентен выбору наиболее вероятного пути. 
Алгоритм обхода останавливается, когда число ребер, исходящих из некоторой 
вершины, становится равным арности соответствующей функции.

\vspace*{-14pt}

\paragraph*{Алгоритм Прима.}
Алгоритм восстанавливает минимальное остовное дерево для графа с~дополнительными 
ограничениями на арность базовых функций. Эти ограничения задают минимальный вес 
ребра. После добавления вершины все лис\-то\-вые ребра этой вершины исключаются, 
чтобы сохранить направление дерева. Если число ребер, начинающихся в~какой-либо 
вершине, превышает соответствующую арность, то остальные ребра исключаются из 
множества возможных ребер в~этой вершине. Алгоритм не зависит от процедуры 
обхода. В случае небольшого шума в~матрице смежности этот алгоритм способен 
восстановить дерево суперпозиции без ошибок. 


\vspace*{-14pt}

\paragraph*{Алгоритмы на основе PCST.}
Матрица смеж\-ности~$M$ должна быть приведена к~неориентиро-\linebreak\vspace*{-12pt}

\pagebreak

\noindent
ванному виду. 
Использована квад\-рат\-ная мат\-ри\-ца~$M'$ без последнего столбца. PCST 
принимает мат\-ри\-цу смеж\-ности $1 \hm- ({1}/{2})(M' \hm+ M'^{\mathsf{T}})$ с~призовым 
значением~0,5 для каж\-дой вершины.
Призовое значение рав\-но~0,5, поскольку при меньших значениях дерево будет 
обрезано: если шум равен~0,5, некоторые вершины могут быть обрезаны по ошибке. 
В~случае больших призовых значений
дерево PCST может содержать ненужные 
вершины. Дерево восстанавливается по одному из опи-\linebreak\vspace*{-12pt}

{ \begin{center}  %fig2
 \vspace*{9pt}
    \mbox{%
\epsfxsize=79mm
\epsfbox{str-2.eps}
}
\end{center}



\noindent
{{\figurename~2}\ \ \small{Качество алгоритмов восстановления с~базовыми функциями небольших 
арностей: \textit{1}~--- DFS; \textit{2}~--- BFS; \textit{3}~--- алгоритм Прима;
\textit{4}~--- $k$-MST; \textit{5}~--- $k$-MST--DFS; \textit{6}~--- $h$-MST--BFS; \textit{7}~--- $k$-MST\,--\,ал\-го\-ритм Прима
}}}

\vspace*{6pt}

\addtocounter{figure}{1}

%\begin{table*}\small  %tabl2
\begin{center}
\parbox{75mm}{{{\tablename~2}\ \ \small{Качество алгоритмов реконструкции с~равномерным шумом, близким 
к~0,5
}}
}
    
    
\vspace*{6pt}

  {\small  \begin{tabular}{|l|ccccc|}
      \hline
                  & \multicolumn{5}{c|}{Шум}\\%& & Шум & & \\
       \cline{2-6}
        \multicolumn{1}{|c|}{\raisebox{6pt}[0pt][0pt]{Алгоритм}}                          
&0,50&0,52&0,54&0,56&0,58\\
                    \hline
      DFS        &0,20 &0,20 &0,19 &0,18 &0,16\\
      BFS        &0,60 &0,58 &0,51 &0,46 &0,40\\
      Прима    &1,00 &0,94&0,81&0,69&0,57\\
      $k$-MST     &0,17 &0,16 &0,14 &0,12 &0,10\\
      $k$-MST--DFS   &0,17 &0,16 &0,16 &0,14 &0,14 \\
      $k$-MST--BFS   &0,43 &0,40 &0,36 &0,33 &0,29 \\
      $k$-MST--Прима  &0,44 &0,39 &0,34 &0,33 &0,27 \\
      \hline
    \end{tabular}
    }
\end{center}
%\end{table*}




\noindent
 санных алгоритмов. Результаты 
$\text{PCST}$ можно использовать в~качестве априорных для других подходов, $M':=({1}/{2})(M_{\mathrm{PCST}}' + M')$,
поэтому результаты \mbox{PCST} обновляются~$M'$.


Процедура генерации данных имеет следующие допущения: арности функций 
генерируются биномиальным распределением, поэтому существуют много функций 
с~малой арностью, все базовые функции имеют только один вход. Любой случай 
с~частичной реконструкцией считается ошибкой. Качество алгоритмов реконструкции:
$$
\fr{1}{K}\sum\limits_{k=1}^K \left[ R\left( \bar{N}(M_k)\right)=M_k\right],
$$
где~$R$ ~--- алгоритм реконструкции;
$\bar{N}\hm=\left(N - \min(N)\right)/\left(\max(N)\hm-\min(N)\right)$~--- нормированная мат\-ри\-ца шума. 
Мат\-ри\-ца~$N$ генерируется как~$N(M)\hm=M\hm+U(-\alpha,\alpha)$.
Генератор случайных чисел возвращает матрицу того же вида, что и~$M$, где каждый 
элемент является независимой переменной из равномерного распределения 
в~сегменте~$[-\alpha,\alpha]$.

Вот список из семи сравниваемых алгоритмов:
DFS,
BFS,
алгоритм Прима,
$k$-MST через PCST,
$k$-MST\;+\;DFS,
$k$-MST\;+\;BFS,
$k$-MST\;+\;ал\-го\-ритм Прима.
На рис.~2 показана ошибка алгоритмов реконструкции 
с~шумом, близ\-ким к~порогу~0,5. Наилучшие результаты дает алгоритм Прима. Второе по 
точности решение основано на~$\text{BFS}$. Таб\-ли\-ца~2 
соответствует~рис.~2 и~показывает качество реконструкции 
семи алгоритмов для значений граничного шума~0,50--0,58.





\vspace*{-9pt}

\section{Заключение}

\vspace*{-3pt}

Предлагаются и~сравниваются  алгоритмы вос\-ста\-нов\-ле\-ния суперпозиции для задачи 
символьной регрессии. Алгоритм Прима дает наиболее точ\-ные результаты и~устойчив 
к~небольшому шуму в~данных. Пред\-ла\-га\-емый алгоритм дает точные результаты, но он 
более подвержен шуму в~мат\-ри\-це суперпозиции. Алгоритмы, основанные на BFS и~DFS, 
не могут вос\-ста\-но\-вить исходную суперпозицию с~зашумленными мат\-ри\-ца\-ми 
суперпозиции. Алгоритм PCST с~BFS, используемый для реконструкции мат\-ри\-цы 
суперпозиции, показывает приемлемые для практического использования результаты.

{\small\frenchspacing
 {%\baselineskip=10.8pt
 %\addcontentsline{toc}{section}{References}
 \begin{thebibliography}{99}
\bibitem{koza1992genetic}  %1
\Au{Koza J.\,R.} Genetic programming as a means for programming computers by 
natural selection~// Stat. Comput., 1994. Vol.~4. P.~87--112.

\bibitem{searson2010gptips} %2
\Au{Searson~D.\,P., Leahy~D.\,E., Willis~M.\,J.} GPTIPS: An open source 
genetic programming toolbox for multigene  symbolic regression~// 
Multiconference (International) of Engineers and Computer Scientists Proceedings, 
2010. Vol.~1. P.~77--80.

\bibitem{stanley2002evolving} %3
\Au{Stanley~K.\,O., Miikkulainen~R.} Evolving neural networks through 
augmenting topologies~// Evol. Comput., 2002. Vol.~10. 
Iss.~2. P.~99--127.

\bibitem{bochkarev2017generation}
\Au{Бочкарев~А.\,М., Софронов~И.\,Л., Стрижов~В.\,В.} По\-рож\-де\-ние экс\-перт\-но-ин\-тер\-пре\-ти\-ру\-емых 
моделей для прогноза проницаемости горной породы~// Системы и~средства информатики, 2017. Т.~27. №\,3. С.~74--87.
%

\bibitem{ravi1996spanning}
\Au{Ravi~R., Sundaram~R., Marathe~M.\,V., Rosenkrantz~D.\,J., Ravi~S.\,S.} 
Spanning trees~--- short or small~// SIAM J.~Discrete Math., 
1996. Vol.~9. Iss.~2. P.~178--200.

\bibitem{chudak2004approximate}
\Au{Chudak~F.\,A.,  Roughgarden~T., Williamson~D.\,P.} Approximate $k$-MSTS 
and $k$-Steiner trees via the primal-dual method and Lagrangean 
relaxation~// Math. Program., 2004. Vol.~100. Iss.~2. P.~411--421.

\bibitem{awerbuch1998new}
\Au{Awerbuch~B., Azar~Y., Blum~A., Vempala~S.} New approximation guarantees 
for minimum-weight $k$-trees and prize-collecting salesmen~// SIAM J. 
Comput., 1998. Vol.~28. Iss.~1. P.~254--262.

\bibitem{arora20062+}
\Au{Aror~S., Karakostas~G.} A~$2+\varepsilon$ approximation algorithm for the 
$k$-MST problem~// Math. Program., 2006. Vol.~107. 
Iss.~3. P.~491--504.

\bibitem{hegde2014fast}
\Au{Hegde~C., Indyk~P., Schmidt~L.} A~fast, adaptive variant of the 
Goemans--Williamson scheme for the prize-collecting steiner tree problem~// 11th DIMACS Implementation Challenge Workshop Proceedings, 2014. P.~1--32.
{\sf http://people. csail.mit.edu/ludwigs/papers/dimacs14\_fastpcst.pdf}.

\bibitem{ras2017approximate}
\Au{Ras~C., Swanepoel~K., Thomas~D.\,A.} Approximate Euclidean Steiner 
trees~// J.~Optimiz. Theory App., 2017. Vol.~172. 
Iss.~3. P.~845--873.

\bibitem{goemans1995general}
\Au{Goemans~M.\,X., Williamson~D.\,P.} A~general approximation technique for 
constrained forest problems~// SIAM J. Comput., 1995. Vol.~24. 
Iss.~2. P.~296--317.
\end{thebibliography}

 }
 }

\end{multicols}

\vspace*{-6pt}

\hfill{\small\textit{Поступила в~редакцию 23.01.22}}

\vspace*{8pt}

%\pagebreak

%\newpage

%\vspace*{-28pt}

\hrule

\vspace*{2pt}

\hrule

%\vspace*{-2pt}

\def\tit{OPTIMAL SPANNING TREE RECONSTRUCTION IN~SYMBOLIC~REGRESSION}


\def\titkol{Optimal spanning tree reconstruction in~symbolic regression}


\def\aut{R.\,G.~Neychev$^1$, I.\,A.~Shibaev$^1$, and~V.\,V.~Strijov$^2$}

\def\autkol{R.\,G.~Neychev, I.\,A.~Shibaev, and~V.\,V.~Strijov}

\titel{\tit}{\aut}{\autkol}{\titkol}

\vspace*{-8pt}


\noindent
$^1$Moscow Institute of Physics and Technology, 9~Institutskiy Per., Dolgoprudny, Moscow Region 141700, Russian\linebreak
$\hphantom{^1}$Federation

\noindent
$^2$Federal Research Center ``Computer Science and Control'' of the Russian Academy of Sciences, 44-2~Vavilov Str.,\linebreak
$\hphantom{^1}$Moscow 119333, Russian Federation

\def\leftfootline{\small{\textbf{\thepage}
\hfill INFORMATIKA I EE PRIMENENIYA~--- INFORMATICS AND
APPLICATIONS\ \ \ 2023\ \ \ volume~17\ \ \ issue\ 1}
}%
 \def\rightfootline{\small{INFORMATIKA I EE PRIMENENIYA~---
INFORMATICS AND APPLICATIONS\ \ \ 2023\ \ \ volume~17\ \ \ issue\ 1
\hfill \textbf{\thepage}}}

\vspace*{3pt} 



\Abste{The paper investigates the problem of regression model generation. A~model is a~superposition of primitive functions. 
The model structure is described by a~weighted colored graph. Each graph vertex corresponds to a~primitive function. 
An edge assigns a~superposition of two functions. The weight of an edge is equal to the probability of superposition. 
To generate an optimal model, one has to reconstruct its structure from its graph adjacency matrix. 
The proposed algorithm reconstructs the minimum spanning tree from the weighted colored graph. 
The paper presents a~novel solution based on the prize-collecting Steiner tree algorithm. This algorithm is compared with its alternatives.}


\KWE{symbolic regression; linear programming; superposition; minimum spanning tree; adjacency matrix}



\DOI{10.14357/19922264230105} 

\vspace*{-16pt}

\Ack

\vspace*{-3pt}


\noindent
This work was supported by the Russian Foundation for Basic Research, projects 20-37-90050 and 20-07-00990.
  

\vspace*{6pt}

  \begin{multicols}{2}

\renewcommand{\bibname}{\protect\rmfamily References}
%\renewcommand{\bibname}{\large\protect\rm References}

{\small\frenchspacing
 {%\baselineskip=10.8pt
 \addcontentsline{toc}{section}{References}
 \begin{thebibliography}{99} 

\bibitem{1-str}
\Aue{Koza, J.\,R.}
 1994. Genetic programming as a means for programming computers by natural selection. \textit{Stat. Comput.} 4:87--112.

\bibitem{2-str}
\Aue{Searson, D.\,P., D.\,E.~Leahy, and M.\,J.~Willis.}
 2010. \mbox{GPTIPS}: An open source genetic programming toolbox for multigene symbolic regression. 
 \textit{Multiconference (International) of Engineers and Computer Scientists Proceedings}. 1:77--80. 

\bibitem{3-str}
\Aue{Stanley, K.\,O., and R.~Miikkulainen.} 2002. Evolving neural networks through augmenting topologies. 
\textit{Evol. Comput.} 10(2):99--127.

\bibitem{4-str}
\Aue{Bochkarev, A.\,M., I.\,L.~Sofronov, and V.\,V.~Strijov.}
 2017. Po\-rozh\-de\-nie eks\-pert\-no-inter\-pre\-ti\-ru\-emykh mo\-de\-ley dlya prog\-no\-za pro\-ni\-tsa\-emosti gor\-noy po\-ro\-dy 
 [Generation of expertly-interpreted models for prediction of core permeability]. \textit{Sistemy i~Sredstva Informatiki~--- Systems and Means of Informatics}
  27(3):74--87.

\bibitem{5-str}
\Aue{Ravi, R., R.~Sundaram, M.\,V.~Marathe, D.\,J.~Rosenkrantz, and S.\,S.~Ravi.}
 1996. Spanning trees~--- short or small. \textit{SIAM J. Discrete Math.} 9(2):178--200.

\bibitem{6-str}
\Aue{Chudak, F.\,A., T.~Roughgarden, and D.\,P.~Williamson.}
 2004. Approximate k-MSTS and k-Steiner trees via the primal-dual method and Lagrangean relaxation. 
 \textit{Math. Program.} 100(2):411--421.

\bibitem{7-str}
\Aue{Awerbuch, B., Y.~Azar, A.~Blum, and S.~Vempala.}
 1998. New approximation guarantees for minimum-weight \mbox{k-trees} and prize-collecting salesmen.
 \textit{SIAM J. Comput.} 28(1):254--262.

\bibitem{8-str}
\Aue{Arora, S., and G.~Karakostas.} 2006. A~$2+\varepsilon$ approximation algorithm for the $k$-MST problem. 
\textit{Math. Program.} 107(3):491--504.

\bibitem{9-str}
\Aue{Hegde, C., P.~Indyk, and L.~Schmidt.} 2014. 
A~fast, adaptive variant of the Goemans--Williamson scheme for the prize-collecting Steiner tree problem. 
\textit{11th DIMACS Implementation Challenge Workshop Proceedings}. 1--32.
Available at: 
{\sf http://people.csail.mit.edu/ludwigs/papers/\linebreak dimacs14\_fastpcst.pdf} (accessed January~10, 2023).

\bibitem{10-str}
\Aue{Ras, C., K.~Swanepoel, and D.\,A.~Thomas.} 
2017. Approximate Euclidean Steiner trees. \textit{J.~Optimiz. Theory  App.} 172(3):845--873.

\bibitem{11-str}
\Aue{Goemans, M.\,X., and D.\,P.~Williamson.} 1995. 
A~general approximation technique for constrained forest problems. \textit{SIAM J. Comput.} 24(2):296--317.
 \end{thebibliography}

 }
 }

\end{multicols}

\vspace*{-6pt}

\hfill{\small\textit{Received January 23, 2022}}

\Contr

\noindent
\textbf{Neychev Radoslav G.} (b.\ 1994)~--- 
PhD student, Moscow Institute of Physics and Technology, 9~Institutskiy Per., Dolgoprudny, Moscow Region 141701, Russian Federation;
\mbox{neychev@phystech.edu}

\vspace*{3pt}

\noindent
\textbf{Shibaev Innokentii A.} (b.\ 1997)~--- 
PhD student, Moscow Institute of Physics and Technology, 9~Institutskiy Per., Dolgoprudny, Moscow Region 141701, Russian Federation; 
\mbox{shibaev.kesha@gmail.com}

\vspace*{3pt}

\noindent
\textbf{Strijov Vadim V.} (b.\ 1967)~--- 
Doctor of Science in physics and mathematics, leading scientist, A.\,A.~Dorodnicyn Computing Center, 
Federal Research Center ``Computer Science and Control'' of the Russian Academy of Sciences, 40~Vavilov Str., Moscow 119333, Russian Federation;
\mbox{strijov@phystech.edu}


\label{end\stat}

\renewcommand{\bibname}{\protect\rm Литература}     %5+
\def\stat{grusho}

\def\tit{АРХИТЕКТУРНЫЕ РЕШЕНИЯ В~ЗАДАЧЕ ВЫЯВЛЕНИЯ МОШЕННИЧЕСТВА ПРИ~АНАЛИЗЕ 
ИНФОРМАЦИОННЫХ ПОТОКОВ В~ЦИФРОВОЙ ЭКОНОМИКЕ$^*$}

\def\titkol{Архитектурные решения в~задаче выявления мошенничества при~анализе 
информационных потоков в
%~цифровой 
экономике}

\def\aut{А.\,А.~Грушо$^1$, М.\,И.~Забежайло$^2$, Н.\,А.~Грушо$^3$, 
Е.\,Е.~Тимонина$^4$}

\def\autkol{А.\,А.~Грушо, М.\,И.~Забежайло, Н.\,А.~Грушо, 
Е.\,Е.~Тимонина}

\titel{\tit}{\aut}{\autkol}{\titkol}

\index{Грушо А.\,А.}
\index{Забежайло М.\,И.}
\index{Грушо Н.\,А.}
\index{Тимонина Е.\,Е.}
\index{Grusho A.\,A.}
\index{Zabezhailo M.\,I.}
\index{Grusho N.\,A.}
\index{Timonina E.\,E.}


{\renewcommand{\thefootnote}{\fnsymbol{footnote}} \footnotetext[1]
{Работа частично поддержана РФФИ (проекты 18-29-03081 и~18-07-00274).}}


\renewcommand{\thefootnote}{\arabic{footnote}}
\footnotetext[1]{Институт проблем информатики Федерального исследовательского центра <<Информатика и~управление>> 
Российской академии наук, grusho@yandex.ru}
\footnotetext[2]{Институт проблем информатики Федерального исследовательского центра <<Информатика и~управление>> 
Российской академии наук, m.zabezhailo@yandex.ru}
\footnotetext[3]{Институт проблем информатики Федерального исследовательского центра <<Информатика и~управление>> 
Российской академии наук, info@itake.ru}
\footnotetext[4]{Институт проблем информатики Федерального исследовательского центра <<Информатика и~управление>> 
Российской академии наук, eltimon@yandex.ru}

\vspace*{-12pt}
   

 
  
  \Abst{Cформулирован подход к~исследованию некоторых видов мошенничества в~цифровой 
экономике с~использованием причинно-следственных связей. Во всех видах рассматриваемых 
мошенничеств должно наблюдаться несоответствие между целями финансовых транзакций 
и~реальной стоимостью достижения этих целей. Данные о транзакциях можно собирать, 
наблюдая информационные потоки, в~которых отражаются эти транзакции. Архитектура сбора 
данных и~их анализа может быть организована с~помощью распределенных реестров 
с~централизованным консенсусом, что позволяет создать аналог электронной бухгалтерской 
книги, фиксирующей финансово-экономическую деятельность субъектов цифровой экономики в~регионе. 
  Рассматриваемые методы выявления мошенничества основаны на противоречиях 
между действиями, описанными в~транзакциях, и~информацией, содержащейся в~планах, 
стандартах, прецедентах и~др. Рассмотрен метод, основанный на некоторой упрощенной схеме 
реализации абстрактного проекта. Для выявления противоречий необходимо проводить анализ 
от следствия к~причине, т.\,е.\ искать аномалии в~информации, описывающей порождение 
наблюдаемых следствий. 
  Показано, как в~реализации проекта можно выделять простые <<необходимые условия>> 
нарушения при\-чин\-но-след\-ст\-вен\-ных связей, т.\,е.\ множество <<необходимых условий>>, 
нарушение которых свидетельствует о наличии мошенничества. Это множество <<необходимых 
условий>> можно назвать метаданными для контроля проекта на выявление мошенничества.} 
 
 
  \KW{цифровая экономика; информационные потоки; при\-чин\-но-след\-ст\-вен\-ные связи; 
выявление мошеннических схем} 

\DOI{10.14357/19922264190204}
  
\vspace*{-4pt}


\vskip 10pt plus 9pt minus 6pt

\thispagestyle{headings}

\begin{multicols}{2}

\label{st\stat}

\section{Введение}

\vspace*{3pt}

  В работе сформулирован подход к~исследованию некоторых видов 
мошенничества в~цифровой экономике с~использованием  
при\-чин\-но-след\-ст\-вен\-ных связей. Рассматриваются три вида мошенничества, 
а именно:
  \begin{enumerate}[(1)]
\item отмыв денег; 
\item обман при выполнении договорных обязательств при реализации 
технических проектов (строительные проекты и~др.); 
\item незаконный вывод денег. 
\end{enumerate}

  Названные виды мошенничества могут быть сведены к~решению одного типа 
задач. Для отмывания денег источник должен заключать фиктивные контракты, 
в~соответствии с~которыми будут переводиться средства за заведомо ненужную 
работу и~материалы. 
  
  Мошенничество, связанное с~невыполнением договорных обязательств, связано 
со снижением качества услуг, качества и~количества закупаемых 
материалов, выполнением работ с~ненадлежащим качеством. 
  
  Вывод денег связан с~переводом средств фир\-мам-од\-но\-днев\-кам, которые 
заведомо не могут выполнить обязательства по контрактам, за которые им 
переводятся средства. 
  
  Таким образом, во всех трех видах рассматриваемых мошенничеств должно 
наблюдаться несоответствие между целями финансовых транзакций и~реальной 
стоимостью достижения этих целей. Данные о транзакциях можно собирать, 
наблюдая информационные потоки, в~которых отражаются эти транзакции. 
  
  Однако для наблюдения таких информационных потоков необходимо создавать 
архитектуру\linebreak телекоммуникационной системы, позволяющей перехватывать 
и~собирать данные о всех транзакциях. Например, такая архитектура может быть 
организована с~помощью распределенных реестров с~централизованным 
консенсусом, т.\,е.\ все информационные потоки, сформированные в~цифровой 
экономике и~несущие информацию о транзакциях, проходят через некоторый 
центральный узел, запоминающий их в~форме распределенного реестра. Такие 
реестры могут дублироваться в~аналогичных центрах различных регионов, что 
позволяет создать аналог электронной бухгалтерской книги, фиксирующей 
фи\-нан\-со\-во-эко\-но\-ми\-че\-скую деятельность субъектов цифровой экономики. Такой 
подход предложено реализовать на базе системы ситуационных центров, что 
отражено в~работах~[1, 2].
  
  Собранная из информационных потоков информация о~транзакциях, т.\,е.\ 
о~контрактах, договорах, платежах, отчетах, закупленных материалах, 
характеристиках исполнителей работ и~др., собирается в~базе данных в~указанном 
центре. Согласно теории интеллектуальных сис\-тем~[3], эту базу данных можно 
называть базой фактов (БФ). Базу фактов можно представить как бинарную мат\-ри\-цу, 
строки которой описывают характеристики, входящие в~транзакции, а столбцы 
нумеруются характеристиками. Строки матрицы будем называть 
\textit{объектами}~[4, 5]. 
  
  Рассматриваемые в~работе методы выявления мошенничества будут основаны 
на противоречиях между действиями, описанными в~транзакциях, и~информацией, 
содержащейся в~планах, стандартах, прецедентах и~др. Для нахождения 
противоречий в~архитектуре центра предусмотрена другая база данных~--- база 
знаний (БЗ)~\cite{3-gr, 6-gr}, которая устроена так же, как БФ. 
  
  Информация в~БЗ собирается на основе положительного опыта или расчетов. 
Используя БЗ, можно выводить факты нарушения при\-чин\-но-след\-ст\-вен\-ных 
связей. Нарушения при\-чин\-но-след\-ст\-вен\-ных связей будем называть 
\textit{аномалиями}. 
  
  Для упрощения дальнейшее изложение будет вестись в~рамках поиска 
противоречий при выполнении некоторого абстрактного проекта. Выявление 
аномалий будет происходить на основе фактов из БФ с~помощью знаний из БЗ 
методами искусственного интеллекта и~интеллектуального анализа 
данных~\cite{6-gr}. 

\vspace*{-10pt}
  
  \section{Модели}
  
  \vspace*{-3pt}
  
  Наиболее сложная из рассмотренных выше задач~--- выявление противоречий, 
т.\,е.\ использование БЗ для получения новых знаний и~выявление аномалий из 
полученных фактов. 
  
  Все способы выявления противоречий основаны на определении 
  причинно-следственных связей. При этом противоречия в~параметрах транзакций по 
отношению к~требуемым в~БЗ составляют сущность аномалий. 
  
   Далее будет рассмотрен метод, основанный на некоторой упрощенной схеме 
реализации абстрактного проекта. 
  
  Каждый проект имеет цель: например, цель представляет собой построение 
некоторой системы. Воспользуемся структурным подходом, который позволяет 
строить проект на основе разбиения системы на подсистемы и~определения 
взаимодействий подсистем~\cite{7-gr}. При этом каждая подсистема также 
представима структурной моделью. 
  
  Как сама система, так и~каждая ее подсистема имеют свой функционал 
и~спецификацию, па\-ра\-мет\-ры настройки и~домены параметров настройки. Кроме 
этих характеристик существует множество характеристик, связанных 
с~<<жизненным циклом>> создания системы. Сюда входят работы, ресурсы, 
сроки выполнения работ по созданию подсистем и~самой системы, стоимости 
компонентов и~материалов, стоимости работ, схемы поставок, договорные 
обязательства и~др. Все характеристики связаны между собой, поэтому можно 
говорить о стоимости и~времени изготовления структурных компонентов системы. 
  
  Одной из важнейших характеристик является смета (система смет для 
подсистем). Смета сопоставляет каждому компоненту системы стоимость его 
изготовления и~настройки. 
  
  Схема построения системы может быть пред\-став\-ле\-на диаграммой, 
изображенной на рис.~1. 

{ \begin{center}  %fig1
 \vspace*{9pt}
   \mbox{%
 \epsfxsize=79mm 
 \epsfbox{gru-1.eps}
 }


\vspace*{9pt}


\noindent
{{\figurename~1}\ \ \small{Диаграмма достижения цели}}
\end{center}
}

\vspace*{9pt}

\addtocounter{figure}{1}
  
  


  Представленная на рис.~1 диаграмма позволяет описать основные классы 
возможных противоречий при достижении цели. Противоречия возникают, когда 
данные БФ не соответствуют требуемым характеристикам. 
  
  
  \section{Потенциальные классы аномалий при~достижении цели}
  
  Выделим четыре потенциальных класса противоречий, которые показывают, 
каким образом нужно искать эти противоречия.
  
 
  Противоречие цели и~проекта (рис.~2) возникает при отсутствии обоснования 
или в~случае логического противоречия между возможностями проектируемого 
функционала и~целью системы. Отметим, что в~проект входят сроки, перечень 
работ, материалы, настройки, которые описываются соответствующими 
параметрами и~допустимыми значениями этих параметров. Проект формируется 
на основе БЗ и~расчетов, исходя из информации, полученной по аналогии 
с~другими проектами и~решениями, которые считаются апробированными. 
  
  Отметим, что цель порождает проект и~в этом смысле является причиной 
проекта. Однако для анализа противоречий необходимо двигаться по штриховой 
стрелке диаграммы (см.\ рис.~2) от проекта к~цели. В~самом деле, любой компонент 
проекта направлен на теоретическое достижение цели. Цель~--- сложный объект, 
поэтому в~проекте могут возникнуть характеристики, противоречащие хотя бы 
некоторым характеристикам цели. Это делает проект противоречивым, но вывод 
об этом может быть сделан только на уровне описания цели. 
  

  Противоречия между проектом и~его реализацией, исключая настройки 
(рис.~3), могут возникать, например, при закупке исполнителем материалов более 
низкого качества по более низким ценам, при попытках достижения требуемых 
сроков работы за счет снижения качества выполнения работ, за счет нахождения 
<<объективных>> причин для увеличения сроков работы и,~следовательно, 
увеличения цены реализации проекта. 


  Для выявления указанных противоречий необходимо двигаться по диаграмме 
(см.\ рис.~3) в~обратную сторону в~соответствии со~штриховыми стрелками. 
Действительно, выявить противоречия между характеристиками закупленных 
материалов и~требуемыми по проекту можно только при обращении к~проекту 
и~его спецификациям. Манипуляции со сроками работы также можно выявить 
только при обращении к~соответствующим расчетам в~проекте. Задержки в~сроках 
работы, связанные с~поставками материалов, можно определить только на 
предыдущем этапе диаграммы (см.\ рис.~3) в~описании проекта. 


  


  Противоречия между реализацией проекта и~его настройкой (рис.~4) возникает, 
когда не удается добиться требуемых значений параметров функционала, не 
удается обеспечить необходимый уровень\linebreak\vspace*{-12pt}

{ \begin{center}  %fig2
 \vspace*{-6pt}
   \mbox{%
 \epsfxsize=16mm 
 \epsfbox{gru-2.eps}
 }


\vspace*{6pt}


\noindent
{{\figurename~2}\ \ \small{Противоречия цели и~проекта}}
\end{center}
}

%\vspace*{9pt}

\addtocounter{figure}{1}

{ \begin{center}  %fig3
 \vspace*{6pt}
    \mbox{%
 \epsfxsize=79mm 
 \epsfbox{gru-3.eps}
 }


\end{center}

\vspace*{-2pt}


\noindent
{{\figurename~3}\ \ \small{Противоречия проекта и~его реализации (без настройки)}}
}

\vspace*{6pt}

\addtocounter{figure}{1}

{ \begin{center}  %fig4
 \vspace*{1pt}
   \mbox{%
 \epsfxsize=54.5mm 
 \epsfbox{gru-4.eps}
 }


\end{center}


\noindent
{{\figurename~4}\ \ \small{Противоречия реализации проекта и~его на\-стройки}}
}

%\vspace*{9pt}

\addtocounter{figure}{1}

{ \begin{center}  %fig5
 \vspace*{5pt}
    \mbox{%
 \epsfxsize=79mm 
 \epsfbox{gru-5.eps}
 }


\end{center}



\noindent
{{\figurename~5}\ \ \small{Противоречия цели и~достигнутой реализации проекта}}
}

\vspace*{6pt}

\addtocounter{figure}{1}

\noindent
 качества реализации проекта. Для 
определения противоречия в~настройках надо опять же двигаться по диаграмме 
(см.\ рис.~4) в~обратную сторону по штриховым стрелкам, так как для выявления 
характеристик результатов работы, которые не дают возможности реализации 
определенного функционала, необходимо иметь информацию о результатах этой 
работы. 


  



  Противоречие между целью и~достигнутой реализацией проекта (рис.~5) 
возникает, когда реализованная система не позволяет достичь цели. В~этом случае 
опять противоречие нужно искать, двигаясь от цели к~реальному достигнутому 
функционалу по штриховой стрелке (см.\ рис.~5).
  
  Суммируя положения, изложенные в~данном разделе, приходим к~выводу, что 
для выявления противоречий необходимо проводить анализ от следствия 
к~причине, т.\,е.\ искать аномалии в~информации, описывающей порождение 
наблюдаемых следствий. 
  
  
  \section{Связь противоречий и~причин}
  
  Прежде чем построить связь между причинами и~противоречиями, кратко 
опишем простейшую модель связи этих понятий. Причины и~противоречия будут 
сформулированы для представления компонентов системы как объектов, 
обладающих наборами известных характеристик~\cite{4-gr, 5-gr}. 
  
  Пусть $U\hm=\{\alpha, \beta, \ldots\}$~--- совокупность характеристик 
(пространство характеристик). Согласно~\cite{4-gr} \textit{объектом}~$O$ 
называется любое подмножество характеристик $O\hm\subseteq U$. Рассмотрим 
последовательность объектов, возможно в~различных пространствах 
характеристик. 
  
  \smallskip
  
  \noindent
  \textbf{Определение~1.}\ Объект~$P$ с~числом характеристик, большим или 
равным~2, является \textit{причиной} объекта (\textit{свойства})~$B$ в~цепочке 
наблюдаемых объектов тогда и~только тогда, когда выполнены следующие 
условия:
  \begin{enumerate}[(1)]
\item для каждого объекта~$C$, если $P\hm\subseteq C$, то $C\mapsto B$, где 
$C\mapsto B$ означает, что объект~$B$ присутствует в~объекте, следующем за 
объектом~$C$;
\item объект~$P$ является минимальным объектом, удовлетворяющим 
условию~1, а~именно: $\forall \alpha\hm\in P$ объект~$P\backslash \{\alpha\}$ 
не является причиной, т.\,е.\ $\exists C:\ \alpha\not\in C$, $P\backslash 
\{\alpha\}\hm\subseteq C$ и~$C\not\mapsto B$, где $C\not\mapsto B$ означает, 
что~$B$ не может содержаться в~объекте, следующем за объектом~$C$. 
\end{enumerate}

  Приведенное определение причины является упрощением причин, 
возникающих в~реальном мире. Например, реальные причины могут возникать\linebreak 
как совокупность характеристик из разных пространств. Одно следствие может 
порождаться разными причинами или возникать из внешних\linebreak и~ненаблюдаемых 
характеристик. Однако пред\-став\-лен\-ная далее формализация позволяет доступно 
изложить при\-чин\-но-след\-ст\-вен\-ные истоки противоречий, которые 
инициируют в~дальнейшем глубокое исследование рассматриваемых процессов.
  
  Будем считать, что для любого интересующего нас свойства~$B$ существует 
причина. Тогда справедлива следующая теорема.
  
  \smallskip
  
  \noindent
  \textbf{Теорема~1.}\ \textit{Для любого свойства~$B$ существует 
единственная причина}. 
  
  \smallskip
  
  \noindent
  Д\,о\,к\,а\,з\,а\,т\,е\,л\,ь\,с\,т\,в\,о\,.\ \ Доказательство будем вести от противного, 
т.\,е.\ предположим, что существуют две причины свойства~$B$: $P$ 
и~$P^\prime$, $P\hm\not= P^\prime$. Тогда существует $\alpha\hm\in U$, которое 
удовлетворяет одному из двух условий:
  \begin{itemize}
\item[(а)] $\alpha\in P$, $\alpha\notin P^\prime$;
\item[(б)] $\alpha\notin P$, $\alpha \in P^\prime$.
\end{itemize}

  Пусть выполняется условие~(б). Тогда $P^\prime\backslash \{\alpha\}$ не 
является причиной по условию~2 определения~1, т.\,е.\ $\exists C$ такое, что 
$\alpha\notin C$, $P^\prime\backslash \{\alpha\}\hm\subseteq C$ и~$C\not\mapsto B$. 
Но если~$B$ произошло и~$P$ его причина, то $C\mapsto B$, что противоречит 
предположению. Теорема~1 доказана.
  
  \smallskip
  
  \noindent
  \textbf{Лемма.} \textit{Если $P$~--- причина появления свойства~$B$, то 
объект~$B$ определяет существование свойства~$P$ в~объекте, 
предшествующем~$B$. }
  
  \smallskip
  
  \noindent
  Д\,о\,к\,а\,з\,а\,т\,е\,л\,ь\,с\,т\,в\,о\,.\ \ Из предположения, что у~каж\-до\-го 
свойства~$B$ есть причина, и~условия, что~$P$ является причиной~$B$, следует, 
что при появлении в~данных свойства~$B$ объект~$C$, предшествующий 
появлению~$B$, содержит как часть объект~$P$. Это следует из теоремы~1 
и~определения причины. 
  
  Докажем принцип <<необходимого условия>>, который, несмотря на простоту 
доказательства, будет играть в~дальнейшем существенную роль.
  
  \smallskip
  
  \noindent
  \textbf{Теорема~2.} \textit{Если~$P$~--- причина появления свойства~$B$ 
и~$A\hm\subseteq P$, то объект~$B$ определяет наличие свойства~$A$ 
в~объекте, предшествующем~$B$}. 
  
  \smallskip
  
  \noindent
  Д\,о\,к\,а\,з\,а\,т\,е\,л\,ь\,с\,т\,в\,о\,.\ \ Пусть в~данных имеется объект~$B$ 
и~$P\mapsto B$, тогда в~силу существования и~единственности причины~$B$ 
в~данных должен существовать объект~$C$, предшествующий~$B$ 
и~содержащий причину~$P$. Поскольку $A\hm\subseteq P$ и~$B$ содержит 
причину~$P$, то $B\mapsto A$. С~учетом леммы теорема~2 доказана.
  
  \smallskip
  
  Пусть даны пространства $U_1, U_2,\ldots$ и~имеется последовательность 
данных (процесс выполнения этапов проекта в~соответствии с~рис.~1) $A, B, 
\ldots$, где каждый объект является подмножеством некоторого 
пространства~$U_i$, $i\hm=1,\ldots$ Тогда в~объекте~$A$ присутствует 
причина~$P$ появления интересующего нас свойства~$C$ в~объекте~$B$. Пусть 
$P\hm\subseteq A$, тогда по теореме~2 $\forall \alpha\hm\in P$:  
$C\mapsto \{\alpha\}$, т.\,е.\ из появления~$C$ следует появление 
характеристики~$\alpha$ в~предшествующем объекте. Это необходимое условие 
того, что~$C$ удовлетворяет причинно-следственным связям развития процесса 
выполнения проекта. Если для~$C$ нет характеристики~$\alpha$, которую можно 
отнести к~причине~$C$, то можно считать, что нарушена  
при\-чин\-но-след\-ст\-вен\-ная связь и~$C$~--- аномальный объект. 
  
  \smallskip
  
  \noindent
  \textbf{Пример.} Если объект~$C$ состоит в~получении суммы~$a$ 
фирмой~$K$, то согласно теореме~2 в~пред\-шест\-ву\-ющем объекте должна 
существовать причина перевода суммы~$a$ на фирму~$K$. Если эта причина 
в~проекте отсутствует, то это можно считать признаком мошеннической схемы. 
Все проекты по предположению собираются из <<кубиков>>, содержащихся в~БЗ. 
Тогда можно сравнить цену объекта~$C$, породившего получение суммы~$a$, 
и~сумму, присутствующую в~смете проекта. Если разница велика, то это либо 
ошибка проекта, либо признак мошеннической схемы.
  
  \section{Поиск противоречий на~основе~принципа <<необходимых~условий>>}
   
  Как было показано в~разд.~3, нахождение противоречий соответствуют 
движению от следствия к~причине. Для каждого объекта в~наблюдаемых данных 
выявление причин его появления является трудоемкой задачей. Кроме того, при 
реализации контроля соблюдения при\-чин\-но-след\-ст\-вен\-ных связей на 
большом множестве участников экономической деятельности задача анализа 
причин становится трудоемкой. Поэтому процедуру контроля необходимо разбить 
на два этапа, где первый этап состоит в~анализе простых <<необходимых 
условий>> проявления мошенничества, когда используется хотя бы одна 
известная характеристика причины. Второй этап (в~режиме офлайн) состоит 
в~выявлении причин, позволяющих провести анализ источников мошеннических 
схем. 
  
  Один из подходов к~выбору <<необходимых условий>> состоит в~построении 
множества подцелей исходной цели проекта (структурный метод построения 
проекта~\cite{7-gr}). Каждая подцель описывается диаграммой на рис.~1, 
и~реализации подцелей должны образовывать полный функционал цели. Это 
является необходимым, но не достаточным условием достижения цели, так как 
при таком подходе отсутствует компонент согласования всех подцелей в~единую 
систему. Однако такой подход значительно упрощает анализ выполнения проекта 
на предмет поиска мошенничества. Если признаки мошенничества будут 
обнаружены в~реализации хотя бы одной из подцелей, то это значит, что 
мошенничество присутствует в~реализации всего проекта. 
  
  Аналогично в~реализации каждого этапа в~любой из подцелей можно выделять 
простые <<необходимые условия>> нарушения при\-чин\-но-след\-ст\-венн\-ых 
связей. 
  
  Таким образом, получается множество <<необходимых условий>>, нарушение 
которых свидетельствует о наличии мошенничества. Это множество 
<<необходимых условий>> можно назвать метаданными~[8, 9] для контроля 
проекта на выявление мошенничества. 
  
  
  \section{Заключение }
  
  В поиске противоречий необходимо от транзакций, соответствующих 
следствиям при\-чин\-но-след\-ст\-вен\-ных связей, переходить к~анализу причин 
наблюдаемых следствий. Это сложная задача, которая связана с~описанием причин 
определенных свойств. 
  
  В работе представлена модель, позволяющая строить множество необходимых 
условий соответствия наблюдаемого следствия вызвавшей его причине. Этот 
подход делает поиск противоречий вполне вычислимой задачей, но не гарантирует 
успех. 
  
  {\small\frenchspacing
 {%\baselineskip=10.8pt
 \addcontentsline{toc}{section}{References}
 \begin{thebibliography}{9}
\bibitem{1-gr}
\Au{Грушо А.\,А., Зацаринный~А.\,А., Тимонина~Е.\,Е.} Блокчейны цифровой экономики на базе 
системы ситуационных центров и~централизованного консенсуса~// Радиолокация, навигация, 
связь: Мат-лы XXV Междунар. научн.-технич. конф.~---
Воронеж: Издательский дом ВГУ, 2019. Т.~6. С.~183--191. 
\bibitem{2-gr}
\Au{Grusho A., Zatsarinny~A., Timonina~E.} A~system approach to information security in 
distributed ledgers on the situational centers platform.~---
Lecture notes in computer science ser.~--- Springer, 2019 
(in press).
\bibitem{3-gr}
\Au{Финн В.\,К.} Искусственный интеллект: Методология, применения, философия.~--- М.: 
Красанд, 2011. 448~с.

\bibitem{5-gr} %4
\Au{Аншаков~О.\,М., Фабрикантова~Е.\,Ф.} ДСМ-ме\-тод автоматического порождения 
гипотез: Логические и~эпистемологические основания.~--- М.: Либроком, 2009. 432~с.

\bibitem{4-gr} %5
\Au{Poelmans J., Elzinga~P., Viaene~S., Dedene~G.} Formal concept analysis in knowledge 
discovery: A~survey~// Conceptual structures: From information to intelligence~/ Eds.\ M.~Croitoru, 
S.~Ferr$\acute{\mbox{e}}$, and D.~Lukose.~--- Lecture notes in computer science 
ser.~--- Berlin--Heidelberg: Springer, 2010. Vol.~6208.  P.~139--153.

\bibitem{6-gr}
\Au{Панкратова~Е.\,С., Финн~В.\,К.} Автоматическое по\-рож\-де\-ние гипотез в~интеллектуальных 
системах.~--- М.: Либроком, 2009. 528~с. 
\bibitem{7-gr}
\Au{Денисов А.\,А., Колесников~Д.\,Н.} Теория больших систем управления.~--- Л.: Энергоиздат, 1982. 488~с.

\bibitem{9-gr}
\Au{Грушо А.\,А., Грушо Н.\,А., Забежайло~М.\,И., Смирнов~Д.\,В., Тимонина~Е.\,Е.} 
Параметризация в~прикладных задачах поиска эмпирических причин~// Информатика и~её 
применения, 2018. Т.~12. Вып.~3. С.~62--66.

\bibitem{8-gr}
\Au{Грушо А.\,А., Грушо Н.\,А., Левыкин~М.\,В., Тимонина~Е.\,Е.} Методы идентификации 
захвата хоста в~распределенной ин\-фор\-ма\-ци\-он\-но-вы\-чис\-ли\-тель\-ной сис\-те\-ме, 
защищенной с~помощью метаданных~// Информатика и~её применения, 2018. Т.~12. Вып.~4. 
С.~41--45.

 \end{thebibliography}

 }
 }

\end{multicols}

\vspace*{-3pt}

\hfill{\small\textit{Поступила в~редакцию 03.04.19}}

%\vspace*{8pt}

%\pagebreak

\newpage

\vspace*{-28pt}

%\hrule

%\vspace*{2pt}

%\hrule

%\vspace*{-2pt}

\def\tit{ARCHITECTURAL DECISIONS IN~THE~PROBLEM 
OF~IDENTIFICATION OF~FRAUD IN~THE~ANALYSIS 
OF~INFORMATION FLOWS IN~DIGITAL ECONOMY\\[-5pt]}


\def\titkol{Architectural decisions in~the~problem 
of~identification of~fraud in~the~analysis 
of~information flows in~digital economy}

\def\aut{A.\,A.~Grusho, M.\,I.~Zabezhailo, N.\,A.~Grusho, and~E.\,E.~Timonina}

\def\autkol{A.\,A.~Grusho, M.\,I.~Zabezhailo, N.\,A.~Grusho, and~E.\,E.~Timonina}

\titel{\tit}{\aut}{\autkol}{\titkol}

\vspace*{-13pt}


 \noindent
   Institute of Informatics Problems, Federal Research Center ``Computer Sciences and 
Control'' of the Russian Academy of Sciences; 44-2~Vavilov Str., Moscow 119133, 
Russian Federation

\def\leftfootline{\small{\textbf{\thepage}
\hfill INFORMATIKA I EE PRIMENENIYA~--- INFORMATICS AND
APPLICATIONS\ \ \ 2019\ \ \ volume~13\ \ \ issue\ 2}
}%
 \def\rightfootline{\small{INFORMATIKA I EE PRIMENENIYA~---
INFORMATICS AND APPLICATIONS\ \ \ 2019\ \ \ volume~13\ \ \ issue\ 2
\hfill \textbf{\thepage}}}

\vspace*{3pt}


   
     
   \Abste{An approach to a~research of some types of fraud in digital economy with the usage of relationships of 
cause and effect is formulated. In all types of the considered frauds, the discrepancy between the 
purposes of financial transactions and actual cost of achievement of these purposes
has to be observed. Data on 
transactions can be collected by observing information flows in which these transactions are reflected. 
The architecture of data collection and their analysis can be organized by means of the distributed 
ledgers with the centralized consensus that allows creating an analog of the electronic account book 
fixing financial and economic activity of subjects of digital economy in the region. 
   The methods of fraud identification considered are based on the contradictions 
between actions described in transactions and information, which is contained in plans, standards, 
precedents, etc. 
   The method based on a~simplified scheme of implementation of the abstract project is considered. 
For identification of contradictions, it is necessary to carry out the analysis from the effect to the cause, 
i.\,e., to look for anomalies in information describing the generation of the observed effects. 
   It is shown how in implementation of the project it is possible to allocate simple ``necessary 
conditions'' of violation of cause and effect relationships, i.\,e., a~set of ``necessary conditions'' 
violation of which demonstrates fraud existence. It is possible to call this set of "necessary conditions" 
by metadata for control of the project for fraud identification.} 
   
   \KWE{digital economy; information flows; relationships of reason and effect; detection of 
fraudulent schemes}
   
  

 \DOI{10.14357/19922264190204}

\vspace*{-20pt}

 \Ack
   \noindent
   The work was partially supported by the Russian Foundation for Basic Research (projects  
18-29-03081 and 18-07-00274).



%\vspace*{6pt}

  \begin{multicols}{2}

\renewcommand{\bibname}{\protect\rmfamily References}
%\renewcommand{\bibname}{\large\protect\rm References}

{\small\frenchspacing
 {\baselineskip=10.5pt
 \addcontentsline{toc}{section}{References}
 \begin{thebibliography}{9}
\bibitem{1-gr-1}
\Aue{Grusho, A.\,A., A.\,A.~Zatsarinny, and E.\,E.~Timonina.} 2019. Blokcheyny tsifrovoy ekonomiki 
na baze sistemy situatsionnykh tsentrov i~tsentralizovannogo konsensusa [Blockchains of digital 
economy on the basis of the system of the situational centres and the centralized consensus]. 
\textit{25th Scientific and Technical Conference (International) ``Radar-Location, Navigation, 
Communication'' Proceedings}. Voronezh: VSU Publs. 6:183--191.
\bibitem{2-gr-1}
\Aue{Grusho, A., A.~Zatsarinny, and E.~Timonina.} 2019 (in press). 
A~system approach to information security 
in distributed ledgers on the situational centers platform. 
Lecture notes in computer science ser. Springer.
\bibitem{3-gr-1}
\Aue{Finn, V.\,K.} 2011. \textit{Iskusstvennyy intellekt: Metodologiya, primeneniya, filosofiya} 
[Artificial intelligence: Methodology, applications, philosophy]. Moscow: KRASAND. 448~p.

\bibitem{5-gr-1}
\Aue{Anshakov, O.\,M., and E.\,F.~Fabrikantova}. 2009. \textit{DSM-metod avtomaticheskogo porozhdeniya gipotez: Logicheskie 
i~epistemologicheskie osnovaniya} [JSM-method of automatic hypothesis generation: Logical and 
epistemological]. Moscow: KD LIBROKOM. 432~p.
\bibitem{4-gr-1} %5
\Aue{Poelmans, J., P.~Elzinga, S.~Viaene, and G.~Dedene.} 2010. Formal concept analysis in 
knowledge discovery: A~survey. \textit{Conceptual structures: From information to intelligence}. 
Eds.\ M.~Croitoru, S.~Ferr$\acute{\mbox{e}}$, and D.~Lukose. Lecture notes in 
computer science ser. Berlin--Heidelberg: Springer. 6208:139--153.

\bibitem{6-gr-1}
\Aue{Pankratov, E.\,S., and V.\,K.~Finn}. 
2009. \textit{Avtomaticheskoe porozhdenie gipotez v~intellektual'nykh 
sistemakh} [Automatic hypotheses generation in intelligent systems]. Moscow: KD 
\mbox{LIBROKOM}.  528~p. 
\bibitem{7-gr-1}
\Aue{Denisov, A.\,A., and D.\,N.~Kolesnikov.} 1982. \textit{Teoriya bol'shikh 
sistem upravleniya} [Theory of big control systems]. Leningrad: Energoizdat. 488~p.

\bibitem{9-gr-1}
\Aue{Grusho, A.\,A., N.\,A.~Grusho, M.\,I.~Zabezhailo, D.\,V.~Smirnov, and 
E.\,E.~Timonina.} 2018. 
Parametrizatsiya v~prikladnykh zadachakh poiska empiricheskikh prichin 
[Parametrization in applied 
problems of search of the empirical reasons]. 
\textit{Informatika i~ee Primeneniya~--- 
Inform. Appl.} 12(3):62--66.

\bibitem{8-gr-1}
\Aue{Grusho, A.\,A., N.\,A.~Grusho, M.\,V.~Levykin, and E.\,E.~Timonina.} 2018. Metody 
identifikatsii zakhvata khosta v~raspredelennoy informatsionno-vychislitel'noy sisteme, 
zashchishchennoy s~pomoshch'yu metadannykh [Methods of identification of host capture 
in the  distributed information system which is protected on the base of meta data].
\textit{Informatika i~ee 
Primeneniya~--- Inform. Appl.} 12(4):41--45.
{ %\looseness=1

}

\end{thebibliography}

 }
 }

\end{multicols}

\vspace*{-12pt}

\hfill{\small\textit{Received April 3, 2019}}

%\pagebreak

%\vspace*{-18pt}

\Contr

\noindent
\textbf{Grusho Alexander A.} (b.\ 1946)~--- Doctor of Science in physics and 
mathematics, professor, principal scientist, Institute of Informatics Problems, 
Federal Research Center ``Computer Sciences and Control'' of the Russian 
Academy of Sciences; 44-2~Vavilov Str., Moscow 119133, Russian Federation; 
\mbox{grusho@yandex.ru} 

\vspace*{3pt}

\noindent
\textbf{Zabezhailo Michael I.} (b.\ 1956)~--- Doctor of Science in physics and 
mathematics, principal scientist, Institute of Informatics Problems, Federal Research 
Center ``Computer Sciences and Control'' of the Russian Academy of Sciences;  
44-2~Vavilov Str., Moscow 119133, Russian Federation; 
\mbox{m.zabezhailo@yandex.ru} 

\vspace*{3pt}


\noindent
\textbf{Grusho Nikolai A.} (b.\ 1982)~--- Candidate of Science (PhD) in physics 
and mathematics, senior scientist, Institute of Informatics Problems, Federal 
Research Center ``Computer Sciences and Control'' of the Russian Academy of 
Sciences; 44-2~Vavilov Str., Moscow 119133, Russian Federation; 
\mbox{info@itake.ru} 

\vspace*{3pt}


\noindent
\textbf{Timonina Elena E.} (b.\ 1952)~--- Doctor of Science in technology, 
professor, leading scientist, Institute of Informatics Problems, Federal Research 
Center ``Computer Sciences and Control'' of the Russian Academy of Sciences;  
44-2~Vavilov Str., Moscow 119133, Russian Federation; 
\mbox{eltimon@yandex.ru} 

\label{end\stat}

\renewcommand{\bibname}{\protect\rm Литература}       %6+
\def\stat{arkhipov}

\def\tit{ВАРИАНТ СОЗДАНИЯ ЛОКАЛЬНОЙ СИСТЕМЫ КООРДИНАТ 
ДЛЯ~СИНХРОНИЗАЦИИ ИЗОБРАЖЕНИЙ ВЫБРАННЫХ СНИМКОВ}

\def\titkol{Вариант создания локальной системы координат 
для~синхронизации изображений выбранных снимков}

\def\aut{О.\,П.~Архипов$^1$, П.\,О.~Архипов$^2$, И.\,И.~Сидоркин$^3$}

\def\autkol{О.\,П.~Архипов, П.\,О.~Архипов, И.\,И.~Сидоркин}

\titel{\tit}{\aut}{\autkol}{\titkol}

\index{Архипов О.\,П.}
\index{Архипов П.\,О.}
\index{Сидоркин И.\,И.}
\index{Arkhipov O.\,P.}
\index{Arkhipov P.\,O.}
\index{Sidorkin I.\,I.}


%{\renewcommand{\thefootnote}{\fnsymbol{footnote}} \footnotetext[1]
%{Работа выполнена при частичной поддержке РФФИ (проект 16-07-00272 А).}}


\renewcommand{\thefootnote}{\arabic{footnote}}
\footnotetext[1]{Орловский филиал Федерального исследовательского центра <<Информатика и~управление>> 
Российской академии наук, \mbox{arkhipov12@yandex.ru}}
\footnotetext[2]{Орловский филиал Федерального исследовательского центра <<Информатика и~управление>> 
Российской академии наук, \mbox{arpaul@mail.ru}}
\footnotetext[3]{Орловский филиал Федерального исследовательского центра <<Информатика и~управление>> 
Российской академии наук, \mbox{voronecburgsiti@mail.ru}}

\Abst{Рассмотрены проблемы сравнения пар изображений, имеющих 
искажения поворота и~сдвига сцен друг относительно друга. Разработан 
алгоритм создания локальной системы координат (ЛСК) для пар сравниваемых 
изображений.}

\KW{алгоритм; методика; локальная система координат; цветное 
изображение; синхронизация; пиксель; цветное пятно; фильтрация}

\DOI{10.14357/19922264160312} 


\vskip 12pt plus 9pt minus 6pt

\thispagestyle{headings}

\begin{multicols}{2}

\label{st\stat}

  \section{Введение}
  
  Важным этапом обработки кадров видеопотока является построение 
ЛСК для синхронизации обрабатываемых 
изображений. Под синхронизацией понимается процедура совмещения пары 
обрабатываемых кадров путем смещения одного изображения относительно 
другого для достижения совпадения одинаковых устойчивых робастных 
структур. В качестве общих робастных структур могут выступать границы 
объектов, имеющихся на полутоновых изображениях, и~центры одинаковых 
по площади цветных пятен соответствующих цветных изображений. 
В~случае необходимости сравнения пары кадров, полученных с~различных 
точек съемки либо с~отличным углом съемки, в~результате чего изображения 
оказались смещены относительно друг друга, синхронизация может стать 
единственно возможным решением для осуществления возможности 
машинного сравнения изображений. 

В~данной статье описывается процесс 
создания ЛСК для синхронизации пар 
обрабатываемых изоб\-ра\-же\-ний. Актуальность работы обусловлена 
необходимостью сравнения пар изоб\-ра\-же\-ний, которые были получены 
с~разных точек съемки, что привело к~искажениям поворота и~смещения. 

Целью данной работы является разработка варианта создания 
ЛСК для синхронизации пар изображений выбранных 
снимков. Основная идея работы состоит в~том, что для синхронизации двух 
изображений необходимо отыскать на этих изображениях робастные 
структуры, которые повторялись бы на каждом из этих изображений, а~затем 
выполнить создание ЛСК с~сохранением лишь 
общей части обрабатываемой пары изображений. Предполагается, что даже 
будучи смещенными друг относительно друга и/или повернутыми на 
произвольный угол, данные изображения, имеющие общую сов\-па\-да\-ющую 
часть, могут быть синхронизированы путем создания ЛСК
 и~преобразованием одного из изображений. Независимо от угла 
поворота и~смещения изображений, имеющих общую часть, робастные 
структуры данных изображений будут сов\-падать. 

\vspace*{-6pt}

  \section{Обзор аналогов}
  
  \vspace*{-2pt}
  
  Одними из наиболее распространенных методов определения 
геометрического рассогласования изображений являются корреляционные 
методы~[1, 2]. Данные методы позволяют рассчитать коэффициент 
корреляции для всех возможных вариантов смещения изображений друг 
относительно друга и~выбрать одно пиковое значение, которое будет 
соответствовать наибольшему совпадению двух сравниваемых изображений. 
Еще одним примером определения взаимного сдвига изображений являются 
статические методы, в~основе которых лежит процесс вычисления 
евклидовой меры взаимного рассогласования изображений~[3]. Однако 
данные методы являются весьма чувствительными к~шумам на 
изображениях, которые являются их неотъем-\linebreak\vspace*{-12pt}

\pagebreak

\noindent
лемой частью, и,~что более 
существенно, они не позволяют выполнить согласование изображений, 
имеющих искажение поворота. 
  
  Так как при съемке изображений нестационарной камерой получаемые 
изображения имеют именно искажения сдвига и~поворота, то пе\-ре\-чис\-лен\-ные 
выше методы не могут быть использованы для синхронизации таких 
изображений. В~данной статье предлагается метод, основанный на 
выявлении робастных характеристик, имеющих сходство на обоих 
обрабатываемых изображениях, который позволит выполнять 
синхронизацию изображений, подвергнутых искажениям сдвига и~поворота. 

\vspace*{-6pt}

  \section{Создание локальной системы координат 
для~синхронизации изображений выбранных снимков}

\vspace*{-2pt}
  
  Цветное изображение представляется в~виде двумерной 
последовательности пикселей вида
  \begin{multline*}
  \mathrm{Image}_i = \!\left\{\!
  \begin{matrix}
  p_{i,1,1}(\mathrm{R,G,B}), & p_{i,1,2}(\mathrm{R,G,B}), &\ldots\\ 
  \ldots  &\ldots  &\ldots\\
  p_{i,h,1}(\mathrm{R,G,B}), & p_{i,h,2}(\mathrm{R,G,B}), &\ldots\end{matrix}\right.\\ 
\hspace*{40mm}\left.\begin{matrix}\ldots, &p_{i,1,w}(\mathrm{R,G,B})\\
\ldots &\ldots\\
\ldots, &p_{i,h,w}(\mathrm{R,G,B})
  \end{matrix}\!
  \right\}\!\!,\hspace*{-0.7966pt}\\
   i\in [1, 2]\,,\enskip
  w\in [1, W_i]\,,\enskip h\in [1, H_i]\,,
%  \label{e1-ar}
  \end{multline*}
где Image$_i$~--- изображение снимка~$i$;
$p$~--- пиксели с~цветовыми координатами (R, G, B);
$W_i$ и~$H_i$~--- ширина и~высота изображения снимка~$i$ в~пикселях. 

  Для сравнения цветных изображений предлагается использовать цветные 
пятна изображений и~робастные характеристики этих изображений. Для 
этого необходимо выполнить процедуру получения полутоновых 
изображений для получения наборов робастных характеристик каждого из 
обра\-ба\-ты\-ва\-емых изображений вида

\noindent
  \begin{multline*}
  \mathrm{Im}_i ={}\\
  {}=Q\left(\varphi_{a,1}(\mathrm{Image}_i),   
\varphi_{a,2 }(\mathrm{Image}_i), 
\varphi_{a,3}(\mathrm{Image}_i)\right)\,,\\
  i\in [1, 2]\,,
%  \label{e2-ar}
  \end{multline*}
где Im$_i$~--- полутоновое изоб\-ра\-же\-ние снимка~$i$;
$Q$~--- функция объединения полутоновых преобразований;
$\varphi$~--- функция выполнения полутоновых преобразований~\cite{4-ar}.
  
  Перед выполнением сегментации цветных изоб\-ра\-же\-ний необходимо 
выполнить огрубление цветовых составляющих изображений до~256~цветов, 
что позволит получить более удобные для сегментации
изображения 
с~четким контрастированием цвето-\linebreak\vspace*{-12pt}

\columnbreak

\noindent вых пятен~\cite{5-ar}. Процедура 
аппроксимации изображений выполняется в~два этапа:
\begin{enumerate}[(1)]
\item  аппроксимация 
изображений до~4096~цветов вида

\noindent
  \begin{equation*}
  \mathrm{Img}_i=\left\{
  \Psi_{\mathrm{app}_{4096}} (\mathrm{Image}_i ,
\mathrm{Pal}_{4096})\right\}\,,\enskip i\in [1, 2]\,,
%  \label{3-ar}
  \end{equation*}
где Img$_i$~--- аппроксимированное до~4096~цветов изображение 
снимка~$i$;
$\Psi_{\mathrm{app}_{4096}}$~--- функция получения множества  
(R, G, B)-пик\-се\-лей в~результате аппроксимации к~4096~цветам;
$\mathrm{Pal}_{4096}$~--- палитра~4096~цветов;
\item 
аппроксимация изображений до~256~цветов вида

\noindent
\begin{equation*}
\mathrm{Img}_i=\left\{
\Psi_{\mathrm{app}_{256}}(\mathrm{Image}_i, 
\mathrm{Pal}_{256})\right\}\,,\enskip i\in [1, 2]\,,
%\label{e4-ar}
\end{equation*}
где $\Psi_{\mathrm{app}_{256}}$~--- функция получения множества  
(R, G, B)-пик\-се\-лей в~результате аппроксимации к~256~цветам;
Pal$_{256}$~--- палитра 256~цветов.
\end{enumerate}
  
  Полученные в~результате выполнения двух этапов аппроксимации 
изображения должны быть сегментированы с~целью формирования 
последовательности цветных пятен каждого изображения. 
Последовательность цветных пятен изображений можно представить в~виде: 
  \begin{multline*}
  \hspace*{-5pt}\Psi_{\mathrm{segm}_i}=\{\Psi_{i,j}\} = \left\{
  \begin{matrix}
  \psi_{i,1}(p_{i,1,1}),&\ldots,& \psi_{i,1}(p_{i,1,t}),\ldots\\
  \ldots&\ldots&\ldots\\
  \psi_{i,n}(p_{i,h,1}),&\ldots, &\psi_{i,n}(p_{i,h,t}),\ldots\end{matrix}\right.\\
\left.\begin{matrix}
\ldots, &\psi_{i,u}(p_{i,1,w-g}),&\ldots ,& \psi_{i,k}(p_{i,1,w})\\
  \ldots&\ldots&\ldots&\ldots\\
\ldots,& \psi_{i,j}(p_{i,h,w-d}),&\ldots ,& \psi_{i,j}(p_{i,h,w})
  \end{matrix} \right\}
%  \label{e5-ar}
  \end{multline*}
  
  \vspace*{-12pt}
  
  \noindent
  \begin{gather*}
  i\in [1,  2]\,,\enskip
  j\in [0,  J_j]\,,\enskip
  t\in [1, T_i]\,,\\
  d\in [1, D_i]\,,\enskip
  g\in [1, G_i]\,,\enskip u\in [1, U_i]\,,\\
  w\in [1, W_i]\,,\enskip h\in [1, H_i]\,,\enskip
  n\in [1, N_i]\,,\\
  N_i\leq J_i\,,\enskip U_i\leq J_i\,, \enskip T_i\leq J_i\,,\enskip
  D_i<W_i\,,
  \end{gather*}
где $\Psi_{\mathrm{segm}_i}$~--- множество сегментов изоб\-ра\-же\-ния 
снимка~$i$;
$\psi_{i,j}$~--- сегмент с~номером~$j$ цветного изоб\-ра\-же\-ния снимка~$i$;
$J_i$~--- максимальное количество цветных сегментов изоб\-ра\-же\-ния 
снимка~$i$.
  
  Полученное множество цветных пятен представляет собой 
последовательность из всех цветоразличимых сегментов обрабатываемых 
изоб\-ра\-же\-ний~[6--8]. Для осуществления синхронизации 
изображений путем определения и~сопоставления робастных структур пары 
обрабатываемых изображений необходимо выполнить фильтрацию 
полученного множества цветных пятен. Сравнительно маленькие и~большие 
по площади пятна на изоб\-ра\-же\-ни\-ях не позволяют выполнить синхронизацию 
изображений из-за того, что маленькие пятна могут с~большой вероятностью 
повторяться на паре обрабатываемых изоб\-ра\-же\-ний либо вовсе пропа-\linebreak\vspace*{-12pt}

\pagebreak

\noindent
дать, 
а~большие пятна могут оказаться на границах изоб\-ра\-же\-ний, что приведет 
к~невозможности точного определения их центров. Следовательно, 
необходимо выбрать для дальнейшего рассмотрения только цветные пятна 
средних размеров, имеющих граничные точки, полученные в~результате 
полутоновых преобразований.
  
  Средняя величина пятен определяется как суммарная величина значений 
площадей всех цветных пятен изображения, деленная на количество цветных 
пятен данного изображения. 
  
  Ввиду того что точное совпадение размеров пятен маловероятно, выбирать 
следует при фильт\-ра\-ции пятна, площадь которых будет принадлежать 
промежутку от $\mathrm{AveSize}/2$ до 3AveSize, где\linebreak AveSize~--- средний 
размер цветного пятна изоб\-ра\-же\-ния. Выбор данного интервала увеличивает 
количество цветных пятен, подлежащих рассмотрению, и~повышает 
вероятность успешной синхронизации изображений.
  
  Отфильтрованная последовательность цветных пятен обрабатываемых 
изображений может быть представлена в~виде:

\noindent
  \begin{multline*}
  \theta_{\mathrm{segm}_i}= \{\theta_{i,k}\}\subset \Psi_{\mathrm{segm}_i}:\ i\in 
[1, 2]\,,\\
  k\in [0,  K]\,,\enskip j\in [0,  J_i]\,,\enskip K_i\leq J_i\,,
%  \label{e6-ar}
  \end{multline*}
где $\theta_{\mathrm{segm}_i}$~--- множество цветных сегментов, оставшихся 
после фильтрации цветных сегментов изоб\-ра\-же\-ния снимка~$i$.

  После того как получена последовательность цветных пятен, 
удовлетворяющая всем условиям фильтрации: площадь и~наличие граничных 
точек, необходимо выполнить процедуру сравнения множеств~$\theta_{\mathrm{segm}_i}$ 
для первого и~второго изображений. Сравнение 
производится путем определения совпадения площадей цветных пятен 
и~взаимного удаления данных пятен от других на каждом изоб\-ра\-же\-нии. При 
этом, найдя цветное пятно $\theta_{1,b}:\ b\hm\leq K_1$, принадлежащее 
первому изображению, и~цветное пятно $\theta_{2,v}:\ v\hm\leq K_2$, 
площадь которого соответствует~$\theta_{1,b}$, необходимо выполнить 
проверку удаленности от всех цветных пятен на каждом изображении 
относительно данных пятен. Если $Y_{\mathrm{segm}}\hm=\emptyset$, то данная пара 
заносится в~множество как совпадающие друг с~другом сегменты. Если 
расстояния до большей части пятен, которые были занесены в~$Y_{\mathrm{segm}}$, 
совпадают, то необходимо добавить в~множество~$Y_{\mathrm{segm}}$ данную пару 
цветных пятен, иначе они будут признаны как ошибочно выбранные 
совпадающими друг с~другом. Получа\-емую последовательность  можно 
представить в~виде:
  \begin{multline*}
  Y_{\mathrm{segm}} =\{\tau_m\}\subset \theta_{\mathrm{segm}_i}:\
  i\in [1, 2]\,,\\
  m\in [0,M_i]\,,\enskip K_1\geq M_i\leq K_2\,,
%  \label{e7-ar}
  \end{multline*}
  
  \columnbreak
  
\noindent
где $Y_{\mathrm{segm}}$~--- множество цветных сегментов, выбранных в~качестве 
совпадающих на паре изображений выбранных снимков.

  Поскольку первый элемент множества~$Y_{\mathrm{segm}}$ не может быть 
проверен на удаленность от других цветных пятен, а второй и~последующий 
сравниваются только с~предыдущими элементами, то необходимо выполнить 
дополнительную проверку\linebreak уда\-лен\-ности каждого элемента 
множества~$Y_{\mathrm{segm}}$ для каждого изоб\-ра\-же\-ния, тем самым исключив 
возможные случайные ошибки. Результирующее множество сегментов, 
имеющих соответствие на паре обрабатываемых изоб\-ра\-же\-ний, может быть 
представлено в~виде:
  \begin{equation*}
  S_{\mathrm{segm}}= \{s_c\}\subset Y_{\mathrm{segm}}:\ c\in [0,C_i]\,,\ K_1\geq C_i\leq K_2\,,
%  \label{e8-ar}
  \end{equation*}
  
  \vspace*{-2pt}
  
  \noindent
где $S_{\mathrm{segm}}$~--- множество цветных сегментов, оставшихся после 
фильтрации цветных сегментов, выбранных в~качестве совпадающих на паре 
изображений выбранных снимков.

  Успешная синхронизация изображений возможна, если имеется три 
и~более сегментов, име\-ющих соответствие на паре обрабатываемых 
изоб\-ра\-же\-ний. При этом чем дальше данные цветовые\linebreak пятна будут 
располагаться друг от друга, тем выше точность синхронизации и~создания 
ЛСК. Если число соответствующих друг другу 
сегментов три и~более, то выполняется процедура создания 
ЛСК для обрабатываемой пары изображений путем 
определения угла поворота одного изображения относительно другого 
и~вычисления расстояний до краев общей области с~переносом пикселей 
каждого из изображений. Таким образом, при выполнении условия наличия 
минимум трех цветных пятен, которые совпадают на паре обрабатываемых 
изображений, за счет выполнения процедуры

\vspace*{2pt}

\noindent
  \begin{equation*}
  \mathrm{Im}\_s_i= \mathrm{Qs}_i(\mathrm{Image}, S_{\mathrm{segm}})\,,\enskip i\in [1, 
2]\,,
%  \label{e9-ar}
  \end{equation*}
  
  \vspace*{-2pt}
  
  \noindent
где Im\_s$_i$~--- изображения, преобразованные к~ЛСК;
Qs$_i$~--- процедура построения ЛСК для 
изоб\-ра\-же\-ния снимка~$i$;
получаем два новых изображения, которые будут иметь новые 
ЛСК, совпадающие на обоих изоб\-ра\-же\-ниях.
  
  На рис.~1 приведена схема алгоритма создания 
ЛСК для синхронизации изображений выбранных кадров.



  Процессы на рис.~1:
{1}~--- получение полутоновых представлений изоб\-ра\-же\-ний
Image$_1$ и~Image$_2$;
  {2}~--- аппроксимация изоб\-ра\-же\-ний Image$_1$ и~Image$_2$
  к~палитре~4096~цветов;
  {3}~--- аппроксимация изобра\-же\-ний Img$_1$ и~Img$_2$ 
к~палитре~256~цветов;
  {4}~---  сегментация изоб\-ра\-же\-ний Img$_1$ и~Img$_2$;
  {5}~--- фильт\-ра\-ция цветных сегментов $\Psi_\mathrm{segm_1}$ 
и~$\Psi_\mathrm{segm_2}$;\linebreak\vspace*{-12pt} 

  
  \pagebreak
  
  \end{multicols}
  
  \begin{figure} %fig1
\vspace*{1pt}
 \begin{center}  
\mbox{%
 \epsfxsize=140.988mm
 \epsfbox{arh-1.eps}
 }
\end{center} 
\vspace*{-9pt}
\Caption{Схема алгоритма создания ЛСК для синхронизации 
изображений выбранных кадров}
\end{figure}
  
  \begin{multicols}{2}
  
  \noindent
  {6}~--- определение соответствия цветных сегментов 
$\theta_{\mathrm{segm}_1}$ и~$\theta_{\mathrm{segm}_2}$ пары изоб\-ра\-же\-ний
Img$_1$ и~Img$_2$ вне зависимости от угла поворота и~смещения;
  {7}~--- фильт\-ра\-ция обнаруженных сегментов множества~$Y_{\mathrm{segm}}$;
  {8}~--- определение числа совпадающих пятен на паре сравниваемых 
изоб\-ра\-же\-ний и~возможности построения ЛСК;
  {9}~--- построение ЛСК;
  {10}~--- выдача ошибки построения ЛСК;
  {11}~--- завершение работы.
  
\vspace*{-6pt}

    \section{Результаты вычислительных экспериментов}
    
  Для тестирования предлагаемого варианта создания 
ЛСК для синхронизации изображений выбранных снимков была 
выбрана пара кадров, представленных на рис.~2. 
     
\begin{figure*} %fig2
\vspace*{1pt}
 \begin{center}  
\mbox{%
 \epsfxsize=156.221mm
 \epsfbox{arh-2.eps}
 }
\end{center} 
\vspace*{-9pt}
      \Caption{Изображения первого~(\textit{а}) и~второго~(\textit{а}) снимков}
      \end{figure*}
      
  После выполнения сегментации и~фильтрации сегментов оставшиеся 
сегменты на изображениях первого и~второго снимков выделены зеленым 
цветом. Для наглядности вручную были отмечены красными линиями 
и~стрелками части изображений, по которым наиболее отчетливо видно 
относительное смещение изображений (рис.~3).
  
\begin{figure*} %fig3
\vspace*{1pt}
 \begin{center}  
\mbox{%
 \epsfxsize=156.304mm
 \epsfbox{arh-3.eps}
 }
\end{center} 
\vspace*{-9pt}
  \Caption{Сегменты для синхронизации, визуализация относительного смещения 
изображений}
  \end{figure*}
  
  После выполнения вышеописанных процедур было 
обнаружено~6~совпадающих цветных сегментов, за счет которых была 
построена ЛСК и~преобразованы оба изображения к~виду, пригодному для 
сравнения. Полученные изображения пред\-став\-ле\-ны на
рис.~4,\,\textit{а} и~4,\,\textit{б}.
  
  \begin{figure*} %fig4
  \vspace*{1pt}
 \begin{center}  
\mbox{%
 \epsfxsize=162.337mm
 \epsfbox{arh-4.eps}
 }
\end{center} 
\vspace*{-9pt}
  \Caption{Преобразованные изображения первого~(\textit{а}) и~второго~(\textit{б}) снимков}
  \end{figure*}
  
  \vspace*{-12pt}
     
  \section{Заключение}
  
  В данной статье был предложен вариант создания 
ЛСК для синхронизации изображений выбранных снимков.
  %
  В результате его тестирования были получены положительные результаты 
для изображений, имеющих искажение сдвига и~поворота. 
  
{\small\frenchspacing
 {%\baselineskip=10.8pt
 \addcontentsline{toc}{section}{References}
 \begin{thebibliography}{9}


\bibitem{2-ar}
\Au{Прэтт У.} Цифровая обработка изображений~/ Пер. с~англ.~--- М.: Мир, 
1982. (Pratt~W.\,K. Digital image processing.~---Wiley-Interscience Publication, 
1978. 750~p.)
\bibitem{1-ar}
\Au{Форсайт Д., Понс Ж.} Компьютерное зрение. Современный подход.~--- 
М.: Вильямс, 2004. 928~с. 
\bibitem{3-ar}
\Au{Гонсалес Р., Вудс Р.} Цифровая обработка изображений~/ Пер. с~англ.~--- 
М.: Техносфера, 2005. 1070~с. (Gonzalez~R.\,C.,  Woods~R.\,E. Digital image 
processing.~--- Wiley-Interscience Publication, 2002. 793~p.)
\bibitem{4-ar}
\Au{Архипов О.\,П., Зыкова З.\,П.} Применение полутоновых представлений 
при анализе изменений цветных изоб\-ра\-же\-ний~// Информатика и~её 
применения, 2014. Т.~8. Вып.~3. С.~90--99.
\bibitem{5-ar}
\Au{Архипов О.\,П., Зыкова З.\,П.} Интеграция гетерогенной информации 
о~пикселях и~их цветовосприятии~// Информатика и~её применения, 2010. 
T.~4. Вып.~4. С.~14--25.
\bibitem{6-ar}
\Au{Архипов О.\,П., Зыкова З.\,П.} Функциональное описание 
индивидуального цветовосприятия~// Известия ОрелГТУ. Сер. 
Информационные системы и~технологии, 2010. №\,5. С.~5--12.
\bibitem{7-ar}
\Au{Архипов О.\,П., Зыкова З.\,П.} RGB-ха\-рак\-те\-ри\-за\-ция пространства 
цветовосприятия~// Системы и~средства информатики, 2010. Вып.~20. №\,1. 
С.~72--89.
\bibitem{8-ar}
\Au{Архипов О.\,П., Зыкова З.\,П.} Равноконтрастные градационные 
преобразования ступенчатых тоновых шкал~// Информационные системы 
и~технологии, 2011. №\,4. С.~39--46.
\end{thebibliography}

 }
 }

\end{multicols}

\vspace*{-6pt}

\hfill{\small\textit{Поступила в~редакцию 12.07.16}}

\vspace*{10pt}

%\newpage

%\vspace*{-24pt}

\hrule

\vspace*{2pt}

\hrule

%\vspace*{8pt}



\def\tit{THE OPTION TO CREATE A~LOCAL COORDINATE SYSTEM 
FOR~SYNCHRONIZATION OF~SELECTED IMAGES}

\def\titkol{The option to create a~local coordinate system 
for~synchronization of selected images}

\def\aut{O.\,P.~Arkhipov, P.\,O.~Arkhipov, and~I.\,I.~Sidorkin}

\def\autkol{O.\,P.~Arkhipov, P.\,O.~Arkhipov, and~I.\,I.~Sidorkin}

\titel{\tit}{\aut}{\autkol}{\titkol}

\vspace*{-9pt}

\noindent
Orel Branch of the 
Federal Research Center ``Computer Science and Control'' of the Russian Academy 
of Sciences, 137~Moskovskoe Sh., Orel 302025, Russian Federation


\def\leftfootline{\small{\textbf{\thepage}
\hfill INFORMATIKA I EE PRIMENENIYA~--- INFORMATICS AND
APPLICATIONS\ \ \ 2016\ \ \ volume~10\ \ \ issue\ 3}
}%
 \def\rightfootline{\small{INFORMATIKA I EE PRIMENENIYA~---
INFORMATICS AND APPLICATIONS\ \ \ 2016\ \ \ volume~10\ \ \ issue\ 3
\hfill \textbf{\thepage}}}

\vspace*{9pt}



\Abste{While comparing pairs of images, in most cases, the problem of 
misalignment of images arises in which one image is distortion of translation and 
rotation relative to another image. Such an image is quite difficult to compare in 
automatic mode. Existing methods of image pairs  synchronize a~large 
number of constraints due to which most of them are rarely used. The proposed 
option to create a~local coordinate system for synchronizing images is based on the 
analysis of color spots presented on the images. It is assumed that successful 
synchronization of two images on their total amount of colored spots is to be 
found that match on the data images. For comparison, it is suggested to use
colored spots and 
distances between spots. In order to successfully synchronize, one needs at 
least three colored spots, which would coincide in all modes of filtration. The 
experiments show acceptable results of synchronization.}

\KWE{algorithm; local coordinate system; color image; synchronization; pixel; 
colored spot; filtration}

\DOI{10.14357/19922264160312} 

\vspace*{9pt}

%\Ack
%\noindent



%\vspace*{6pt}

  \begin{multicols}{2}

\renewcommand{\bibname}{\protect\rmfamily References}
%\renewcommand{\bibname}{\large\protect\rm References}

{\small\frenchspacing
 {%\baselineskip=10.8pt
 \addcontentsline{toc}{section}{References}
 \begin{thebibliography}{9}

  
\bibitem{2-ar-1}
\Aue{Pratt, W.\,K.} 1978. \textit{Digital image processing}. Wiley-Interscience 
Publication. 750~p.
\bibitem{1-ar-1}
  \Aue{Forsyth, D.\,A., and J.~Ponce}. 2002. \textit{Computer vision: A~modern 
approach}. Prentice Hall Professional Technical Reference. 720~p.
\bibitem{3-ar-1}
\Aue{Gonzalez, R.\,C., and R.\,E.~Woods}. 2002. \textit{Digital image 
processing}. Wiley-Interscience Publication. 793~p.
  \bibitem{4-ar-1}
  \Aue{Arhipov, O.\,P., and Z.\,P.~Zykova}. 2014. Primenenie polutonovykh 
predstavleniy pri analize izmeneniy tsvetnykh izobrazheniy [The use of half-tone 
representations in the analysis of changes in color images]. \textit{Informatika i~ee 
Primeneniya~--- Inform. Appl.} 8(3):90--99.
  \bibitem{5-ar-1}
  \Aue{Arhipov, O.\,P., and Z.\,P.~Zykova}. 2010. Integratsiya geterogennoy 
informatsii o pikselyakh i~ikh tsvetovospriyatii [Integration of heterogeneous 
information about pixels and their color perception]. \textit{Informatika i~ee 
Primeneniya~--- Inform. Appl.} 4(4):14--25.
  \bibitem{6-ar-1}
  \Aue{Arhipov, O.\,P., and Z.\,P.~Zykova}. 2010. Funktsional'noe opisanie 
individual'nogo tsvetovospriyatiya [Characteristics of color perceptual space]. 
\textit{Izvestiya OrTGU. Ser. Informatsionnye sistemy i~tekhnologii} [Herald
of Oryol Technical State University. Ser. 
information systems and technologies] 5:5--12.

\pagebreak

  \bibitem{7-sr-1}
  \Aue{Arhipov, O.\,P., and Z.\,P.~Zykova}. 2010. RGB-kharakterizatsiya 
prostranstva tsvetovospriyatiya [RGB-characterization of color space]. 
\textit{Sistemy i~Sredstva Informatiki~--- Systems and Means of Informatics} 
1(20):\linebreak 72--89.
  \bibitem{8-ar-1}
  \Aue{Arhipov, O.\,P., and Z.\,P.~Zykova}. 2011. Ravnokontrastnye 
gradatsionnye preobrazovaniya stupenchatykh tonovykh shkal [Equal contrast 
graded transformation of step tinted scales]. \textit{Informatsionnye Sistemy 
i~Tekhnologii} [Information Systems and Technologies] 4:39--46.
\end{thebibliography}

 }
 }

\end{multicols}

\vspace*{-6pt}

\hfill{\small\textit{Received July 12, 2016}}

\vspace*{-3pt}


\Contr

\noindent
\textbf{Arkhipov Oleg P.}\ (b.\ 1948)~--- Candidate of Science (PhD) in technology, Director, Oryol Branch of  
Federal Research Center  ``Computer Science  and Control'' of the 
Russian Academy of Sciences, 137~Moskovskoe Sh., Oryol 
302025, Russian Federation; \mbox{arkhipov12@yandex.ru}

\vspace*{4pt}

\noindent
\textbf{Arkhipov Pavel O.}\ (b.\ 1979)~--- Candidate of Science (PhD) in technology, senior scientist, Oryol Branch 
of  Federal Research Center  ``Computer Science  and Control'' of the Russian 
Academy of Sciences, 137~Moskovskoe Sh., Oryol 
302025, Russian Federation; \mbox{arpaul@mail.ru}

\vspace*{4pt}

\noindent
\textbf{Sidorkin Ivan I.}\ (b.\ 1990)~--- engineer-researcher, Orel Branch of the 
Federal Research Center ``Computer Science and Control'' of the Russian Academy 
of Sciences, 137~Moskovskoe Sh., Orel 302025, Russian Federation; 
\mbox{voronecburgsiti@mail.ru}
  \label{end\stat}
  
  
  \renewcommand{\bibname}{\protect\rm Литература}   %7+
\renewcommand{\figurename}{\protect\bf Figure}
\renewcommand{\tablename}{\protect\bf Table}

\def\stat{dulin}


\def\tit{INFORMATION FUSION OF~DOCUMENTS}

\def\titkol{Information fusion of~documents}

\def\autkol{S.\,K.~Dulin, N.\,G.~Dulina, and~P.\,V.~Ermakov}

\def\aut{ S.\,K.~Dulin$^1$, N.\,G.~Dulina$^2$, and~P.\,V.~Ermakov$^3$}

\titel{\tit}{\aut}{\autkol}{\titkol}



\renewcommand{\thefootnote}{\arabic{footnote}}
\footnotetext[1]{Institute of Informatics Problems, Federal Research Center ``Computer Science and Control'' 
of the Russian Academy of Sciences, 44-2~Vavilov Str., Moscow 119333, Russian Federation, 
skdulin@mail.ru}
\footnotetext[2]{A.\,A.~Dorodnicyn Computing Center, Federal Research Center ``Computer Science and 
Control'' of the Russian Academy of Sciences, 40~Vavilov Str., Moscow 119333, Russian Federation, 
ngdulina@mail.ru}
\footnotetext[3]{ TeleRetail GmbH, 30~\mbox{Markenstra{\!\ptb{\ss}}e}, 
D$\ddot{\mbox{u}}$sseldorf 40227,  Germany; petcazay@gmail.com}


\index{Dulin S.\,K.}
\index{Dulina N.\,G.}
\index{Ermakov P.\,V.}
\index{Дулин С.\,К.}
\index{Дулина Н.\,Г.}
\index{Ермаков П.\,В.}


\def\leftfootline{\small{\textbf{\thepage}
\hfill INFORMATIKA I EE PRIMENENIYA~--- INFORMATICS AND
APPLICATIONS\ \ \ 2020\ \ \ volume~14\ \ \ issue\ 1}
}%
 \def\rightfootline{\small{INFORMATIKA I EE PRIMENENIYA~---
INFORMATICS AND APPLICATIONS\ \ \ 2020\ \ \ volume~14\ \ \ issue\ 1
\hfill \textbf{\thepage}}}

%\vspace*{-2pt}



       \Abste{The paper considers the problems associated with the 
creation of an expert base of documents that require prompt 
processing of incoming information and, as a consequence, 
restructuring of the knowledge base. The authors propose procedures 
that reduce the search of the optimal consistent state of 
interrelated documents. An approach to assessing the relationship of 
text documents and informational messages as poorly structured 
objects was developed. The practical implementation of this approach 
is described.}
      
      \KWE{information fusion; controlled data and knowledge consistency; 
knowledge base restructuring}
      
\DOI{10.14357/19922264200117} 
      
      %\vspace*{8pt}
      
      
      \vskip 12pt plus 9pt minus 6pt
      
       \thispagestyle{myheadings}
      
       \begin{multicols}{2}
      
       \label{st\stat}
     
     \section{Introduction}
      
     \noindent
     Combining information of various origins for integrative analysis and 
processing has been called ``Information Fusion''[1], implying that the synthesized 
data carrying information combine type properties of source data and possess 
more information than merely conjunction of information sources considered 
separately. The main difficulty of the synthesis problem is that information sources 
contain heterogeneous data represented by various formats and structures and 
employed in different types of platforms.
     
     The main factors of data heterogeneity and their sources are: various types 
of data, diversity in data origin, various models of database representation, various 
data presentation formats, differentiating in the organization of data storage 
systems, differences in the degree of reliability and accuracy of data, and
variety of  a~degree and form of data structure.
     
     The process of information fusion is a~multilevel process that includes five 
basic stages~\cite{2-d, 3-d, 4-d}:
     \begin{itemize}
\item zero stage~--- the stage of combining sensor signals, designed to obtain 
data indicating semantically clear and interpretable attributes of objects and 
participating in the applications of the research being performed;
\item the first stage is aimed at processing data of the zero stage in order to 
make a decision on the classes of the objects in question and the states of these 
objects;
\item the second stage of Information Fusion, designed to assess the situation, 
including the zero and the first stages. It is used to assess the situational 
interaction of objects considered as a whole;
\item the third stage~--- the stage of evaluation of the interaction ``Impact 
Assessment,'' designed to perform an antagonistic assessment, based on the 
prediction of the situation;
\item the fourth stage~--- the stage of feedbacks, evaluating the possibility of 
using feedbacks in the system in question; and
\item the fifth stage~--- the final stage, the level of man--machine interaction, 
performing correctional actions of the operator for the sake of the system 
control.
\end{itemize}

     Research in the field of Information Fusion mainly focuses on the synthesis 
of data represented by digital images and arrays of data and  
documents~\cite{1-d, 4-d, 5-d}.
     
     Current trends in the development of corporative informational systems 
show that, along with traditional informational resources, the results of intelligent 
activity of experts and analysts become very important for the successful operation 
of large and middle-sized companies. A~unified informational environment of the 
company incorporates these formalized results in an accumulated form such that 
all executives can jointly use this resource in the context of their assignments. The 
role played by the knowledge accumulated in such a~way in the enterprise-wide 
systems allows us to consider this knowledge as very valuable and a~notably 
important resource for a~company, which, together with the traditional resources, 
such as financial, material, human, etc., characterizes the reliability of the 
company. The totality of this knowledge, presented mainly in text form, is the 
intelligent assets of the company, and the competitiveness of the company and its 
adaptability to changing the business environment depends on how efficiently this 
resource is used.
{\looseness=-1

}
     
     An intelligent asset is a~specific resource that requires specialized 
knowledge management systems. These systems enable the search, accumulation, 
and processing of knowledge by experts in solving various analytical problems. 
This tendency in knowledge engineering appeared relatively recently, but interest 
in the development and usage of such systems is permanently growing. This is 
largely due to the significant results achieved by some companies that have 
successfully implemented knowledge management systems into their 
manufacturing activity.
     
     Complex technological solutions designed to support various stages of 
composition and usage of corporative data and knowledge have been embodied in 
the knowledge management systems. At each of these stages, individual problems 
are solved, with the most important of them being associated with tasks related to 
searching, processing documents, and extracting knowledge from them.
     
     Text processing tasks are solved in practically all fields of human activity, 
and the analysis of the current environment is an integral part of practically 
each 
corporative management system securing a timely and adequate reaction to 
changes in the business environment. Actually, operativeness is the basic 
characteristics of monitoring problems, which distinguishes them from the problems 
related to prediction, planning, etc., because the main goal of the monitoring is the 
timely reaction of corresponding management subsystems of the general 
technological scheme of company functioning to changes of internal or external 
factors.
     
     In the general case, the purpose of text processing tasks is to accumulate 
necessary information from different sources, process it analytically, and, on this 
basis, generate corresponding decisions. The character of text processing tasks is 
permanent in the sense that the environment and the parameters of the company 
operation are subject to permanent changes, which requires regular (or periodic) 
sampling of ever changing information.
     
     Text processing tasks can conventionally be divided into two classes: internal 
monitoring and external monitoring.
     
     Internal monitoring is associated mainly with the monitoring of internal 
operation parameters, e.\,g., regular monitoring of the operation of complex installa-
tions, cargo moving, etc. Possible examples are control systems for energy plants, 
freight management, etc. The typical feature of these problems is a relatively 
constant set of parameters used to estimate the state of the process (production, 
physical parameters of an installation, etc.).
     
     In contrast to the internal monitoring, the external monitoring is mainly 
related to the estimation of the state of the environment and external conditions of 
the company operation. As an example, an analysis of consumer demand carried 
out by a commodity-producing company falls into this category. The typical 
feature of these problems is that, first, the parameters to be estimated are poorly 
formalized and, second, the set of these parameters is variable. The latter factor 
requires the restructuring of the analyst knowledge according to the changed 
conditions. All this makes us consider the ``restructurability'' of the expert 
knowledge base as one of the characteristic features of the problems of external 
monitoring.
     
     In the problems of external monitoring, special requirements must be 
imposed on the sources of information used by experts for the localization of 
required knowledge and data. The development of informational technologies 
during recent years has strongly suggested that the Internet is gradually becoming 
the most important source of information in solving analytical problems in 
practically all areas of human activity. Coming up to printed and electronic mass 
media, Internet is often ranked first in operativeness, which makes the Internet the 
most valuable information source in monitoring problems. It is for this reason that, 
in this work, special attention is paid to the solution of monitoring problems 
associated with search and processing of text information in Internet.
     
     \section{Approach to~Provision of~Knowledge Consistency}
     
     \noindent
     In previous works (see~\cite{4-d, 7-d, 6-d}), the authors put forward a procedure 
providing the consistency of the knowledge base dynamically formed by an 
expert, which is based on the analysis of structural interrelations between separate 
components of the knowledge base with subsequent restructuring of it aimed at 
reducing existing inconsistency. In so doing, the basic criterion of structural 
consistency was a concept of polyconsonance of power~$n$~\cite{2-d}.
     
     Consider a knowledge base formed on the basis of search and analysis of 
Internet information. In solving the monitoring problems associated with the 
formation of such a knowledge base, the application of this procedure faces certain 
difficulties resulting from poor formalization and an obscure or ambiguous 
structure of the data (text or multimedia documents). Besides, for the monitoring 
problems considered here, a large number of informational messages directed to the 
expert for analytical processing and replenishment of the knowledge base are 
characteristic. As a result, the amount of resources (especially, time) required for 
the restructuring of a dynamically changing knowledge base is increased 
significantly, which is, perhaps, the main obstacle to the successful practical 
implementation of any procedure of the above type.
     
     One of the major disadvantages of the algorithm proposed in~\cite{4-d} is 
that it is oriented to problems of the search type; that is why, the authors made 
special efforts to reduce the search and thus increase the algorithm efficiency in its 
practical implementation. The results presented below are aimed at the solution of 
the latter problem.
     
     Consider a set of mutually related objects $O = \{o_i\}$ with a similarity 
function~$f$~\cite{3-d} satisfying the condition
     $$
     0\leq f\left( o_i, o_j\right)\leq 1\,.
     $$
     
     Numbers $\alpha$ and~$\beta$ will denote the lower and upper similarity 
thresholds, respectively, satisfying the condition
     $$
     0\leq \alpha\leq \beta\leq 1\,.
     $$
     
    Now, let us introduce the concepts of a negative, positive, and indifferent link 
between two arbitrary elements~$o_i$ and~$o_j$ of the set~$O$. The link is called 
``negative'' if its value does not exceed the lower similarity threshold: $0\leq 
f(o_i,o_j)\leq \alpha$; it is called ``positive'' if the value of the similarity function is 
not less than the upper similarity threshold: $\beta\leq f(o_i,o_j)\leq 1$; and, if 
$\alpha<f<\beta$, it is called ``indifferent'' (zero).
     
     Consider a partition of the given set into a number of nonempty subsets 
$K_1,\ldots , K_n$.
     
     A link between two arbitrary elements~$o_i$ and~$o_j$ of the entire 
set~$O$ is called ``bad'' if one of the following conditions is satisfied:
     \begin{enumerate}[(1)] 
     \item the elements~$o_i$ and~$o_j$ belong to the same subset~$K_x$, and 
the link between them is negative; or
\item the elements~$o_i$ and~$o_j$ belong to different subsets~$K_1$ 
and~$K_2$, and the link between them is positive.
\end{enumerate}

     Using this definition, let us to each object~$o_k$ from the set 
considered   assign the number~$v_k$ of its bad links for a~given partition into subsets. 
Now, let us construct a~vector~$V$ consisting of these values (this vector has 
a~dimension equal to the number of objects in the set) and call it the nodewise 
difference vector (NDV)~\cite{4-d}. The sum of the elements of this vector is 
denoted by $S_{\mathrm{NDV}}$.
     
     Clearly, different partitions of the original set correspond to different NDVs 
and different values of $S_{\mathrm{NDV}}$. According to the algorithm considered, 
the main problem is to find a partition of the given set~$O$ such that the sum 
$S_{\mathrm{NDV}}$ 
takes its minimal value; i.\,e., the total number of bad links tends to zero.
     
     The algorithm~\cite{4-d} developed by the authors consists in 
successive transformations of the set of informational objects on the basis of the 
condition
     $$
     S_{\mathrm{NDV}} > \fr{n(N-n)}{2}
     $$
     where $S_{\mathrm{NDV}}$ is the sum of nodewise differences for the given 
set of~$n$ elements belonging to a pair of consonant subsets of the total 
cardinality~$N$.  If this condition is fulfilled, then the restructuring of the 
considered set results in a decrease of the total sum~$S_{\mathrm{NDV}}$.
     \smallskip
     
     \noindent
     \textbf{Theorem~1.} \textit{Let~$K_1$ and~$K_2$ be two subsets of 
a~given set of mutually related objects~$O$}:
     \begin{align*}
     K_1 &= \left\{ o_i\right\}\,,\ i=1,\ldots, n_1\,;\\
     K_2&= \left\{ o_j\right\}\,, \ j=1,\ldots , n_2\,.
     \end{align*}
     
     \textit{A set containing~$m$~elements from these two subsets satisfies the 
condition of the algorithm if, and only if, the set consisting of all remaining 
elements of these two subsets satisfies the same condition.}
     
     \smallskip
     
     \noindent
     P\,r\,o\,o\,f\,.\ \  First, let us prove the necessity. Let the set of 
objects~$\{o_k\}$, $k = 1,\ldots , m$, satisfy the condition of the algorithm:
     $$
     \sum v_k> \fr{m(n_1+n_2-m)}{2}
     $$
     where $v_k$ are the NDV values for the element with the number~$k$. 
This formula can be transformed to the form:
     $$
     \sum v_k > \fr{\left(n_1+n_2-m\right)
     \left(\left(n_1+n_2\right)-\left(n_1+n_2-m\right)\right)}{2}
     $$
     which means that the set of $n_1+n_2-m$ vectors not belonging to the 
original set also satisfies the condition of the algorithm.
     
     The sufficiency of the condition is proved similarly. The theorem is proved.
     
     \smallskip
     
     \noindent
     \textbf{Corollary.} In order to find a set of objects from two given subsets 
that satisfies the condition of the algorithm, it is sufficient to check the fulfillment 
of this condition only for the subsets consisting of $(n_1+n_2)/2$ objects. In other 
words, only subsets with cardinalities not exceeding half of the sum of the 
cardinalities of the original subsets~$K_1$ and~$K_2$ should be checked.
     \smallskip
     
     \noindent
     P\,r\,o\,o\,f\,.\ \ Indeed, if some set consisting of more than $(n_1+n_2)/2$ 
elements satisfies the condition, then the complement to it also satisfies this 
condition, with the cardinality of the complement being not greater than 
$(n_1+n_2)/2$.

\begin{figure*}[b] %fig1
\vspace*{1pt}
    \begin{center}  
  \mbox{%
 \epsfxsize=160.967mm 
 \epsfbox{dul-1.eps}
 }
\end{center}
\vspace*{-10pt}
\Caption{Determination of vocabulary groups}
\end{figure*}

     
     \smallskip
     
     \noindent
     \textbf{Theorem~2.}\  \textit{Let~$K_1$ and~$K_2$ be two subsets of 
a~given set of mutually related objects~$O$}:
     \begin{align*}
     K_1 &= \left\{o_i\right\}\,,\ i=1,\ldots , n_1\,;\\
     K_2&= \left\{o_j\right\}\,,\ i=1,\ldots , n_2\,.
     \end{align*}
     \textit{Let a set $\{o_k\}$ of $m < (n_1 + n_2)/2$ elements belonging to 
these two subsets satisfy the condition of the algorithm. If a zero NDV element 
corresponds to some element~$o_x$ from this set, then the set of the vectors 
corresponding $O^*=\{o_1, \ldots, o_{x-1}, o_{x+1}, \ldots, o_m\}$ also satisfies the 
condition of the algorithm.}
     
     \smallskip
     
     \noindent
     P\,r\,o\,o\,f\,.\ \ According to the assumption of the theorem, the sum 
$S^*_{\mathrm{NDV}}$ for the set $O^*=\{o_1, \ldots\linebreak
\ldots, o_{x-1}, o_{x+1}, \ldots, o_m\}$ is 
equal to the sum $S_{\mathrm{NDV}}$ of the original set of the elements from the two 
subsets~$K_1$ and~$K_2$:
     $$
S^*_{\mathrm{NDV}} = S_{\mathrm{NDV}}\,.
$$
     
     Denote by~$N$ the total cardinality of the considered subsets: $N = 
n_1+n_2$. Then,
     $$
     (m-1)(N-(m-1)) = m(N-m)+(2m-N-1)\,.
     $$
     
     According to the assumption of the theorem, $m \leq N/2$; hence, $2m-N-1 
< 0$. To complete the proof, let us write the following inequality:
     \begin{multline*}
     S^*_{\mathrm{NDV}} = S_{\mathrm{NDV}} = \sum v_k >\fr{m(N - 
m)}{2} >{}\\
{}> \fr{(m-1)(N - (m-1))}{ 2}
   \end{multline*}
     which means that the set $\{o_1, \ldots, o_{x-1}, o_{x+1}, \ldots, o_m\}$ satisfies 
the condition of the algorithm.
     
     Obviously enough, it follows from this theorem that, in the practical 
implementation of the proposed algorithm, it is sufficient to search for a set of 
elements for the next iteration among those with nonzero NDV values.
{\looseness=1

}
     
\section{Thematic Role of~Similarity}

     \noindent
     The most significant factor affecting the operation of the algorithm 
considered is the similarity function on the basis of which interrelations between 
different elements of a given set are determined. As far as the support of 
monitoring problems is considered, with the texts (in particular, news) and the 
Internet being the elements and the main information source, respectively, the 
construction of the similarity function becomes a fairly difficult problem. Perhaps, 
one of the solutions to this problem could be the use of various methods of 
linguistic analysis to determine the degree of ``likeness'' of two different 
documents, although these methods are not free from some shortcomings 
associated with the hardship of their implementation, adjustment, etc. To 
determine the similarity function in practical applications, the authors have put 
forward another approach. One of the advantages of this new approach is the 
simplicity of implementation and the ``notional transparency.''
     
     The basis of this approach schematically shown in Fig.~1 is the 
determination of vocabulary groups~\cite{7-d}, which denote the sets of keywords 
defined by the expert. The expert assorts the keywords according to 
some criterion, e.\,g., ``thematic meaning:''
     $$
     G_k= \left\{w_i\right\},\enskip i = 1,\ldots ,n_k.
     $$

\begin{figure*}[b] %fig2
\vspace*{1pt}
    \begin{center}  
  \mbox{%
 \epsfxsize=94.043mm 
 \epsfbox{dul-2.eps}
 }
\end{center}
\vspace*{-10pt}
\Caption{A general scheme of operation of iiProcessor system}
\end{figure*}

     Consider an arbitrary element~$o_j$ from a given set~$O$. This object is a 
text document; so, it can be represented as an aggregate of lexical units, i.\,e., 
words. For~$o_j$, let us define its coefficient of correspondence with the dictionary 
group~$G_i$ as the ratio $S(G_i)_j$ of the number of keywords specified in this 
dictionary group and available in the text of the information object itself, to the 
total number of keywords from all dictionary groups, $S(G)_j$ found in this text. 
Then, one can define the factor of correspondence of the object~$o_j$ to the 
vocabulary group~$G_i$ as
     $$
     L^i_j = \fr{S(G_i)_j}{S(G)_j}.
        $$
     
     On the basis of these coefficients, let us define the degree of thematic coupling 
between two arbitrary informational objects as follows:
     \begin{itemize}
     \item[(A)] $f(o_k, o_l) = 1$ if $ S(G)_k = 0$ and  $S(G)_l = 0$;
     \item[(B)] $f(o_k, o_l) = 0$ if  $S(G)_k\not= S(G)_l$ 
     and $S(G)_k S(G)_l\linebreak = 0$; and
     \item[(C)] $f(o_k, o_l) = \max\left( \min\left(L^i_k, L^i_l\right) \right)$, $i = 1, 
\ldots, n$,  for $S(G)_k  S(G)_l\not= 0$
     where $n$ is the number of the vocabulary groups.
     \end{itemize}

     
     Note that the similarity function defined above takes the values on the 
interval from~0 to~1 but lacks associativity, because $0 \leq f(o_i, o_j) \leq 1$. In 
the works devoted to the theoretical grounds of the considered algorithm of 
structural transformations of a set of objects, the associativity of the similarity 
function has not been used; therefore, the fact that the function introduced above is 
not associative does not require any changes in the proposed algorithm. Moreover, 
the lack of associativity here has an additional meaning, which makes it possible to 
treat the function introduced above as a~\textit{thematic} similarity function.
     
     Indeed, if, in the considered text, there are keywords from different 
vocabulary groups, then all the coefficients~$L^i_j$ for this element will be less 
than one. Hence, the value of the similarity function~$f$ will also be less than one, 
and the more the number of the vocabulary groups, the less this value. In practice, 
this could mean that the considered document is of a review nature and, most 
probably, has no distinct ``thematic meaning.''

\begin{figure*} %fig3
\vspace*{1pt}
    \begin{center}  
  \mbox{%
 \epsfxsize=156.872mm 
 \epsfbox{dul-3.eps}
 }
\end{center}
\vspace*{-10pt}
\Caption{Example of use of vocabulary group technique to establish
links between different documents}
\end{figure*}
     
\section{Consistency Controlling Module iiProcessor}

     \noindent
     The authors' technique for providing structural consistency of the knowledge 
base in solving monitoring problems has been implemented in a specialized system 
called an iiProcessor. This system is designed to compose expert knowledge bases 
for social, political, and international sciences. The knowledge bases are 
constructed from the information supplied by various mass media through their 
Internet servers. The main purpose of the system is to accumulate informational 
messages (news) related to the themes of user's interest from various Internet 
sources, to integrate the information into a unified knowledge base, to create links 
between different elements of the knowledge base, and to make subsequent 
restructuring of the knowledge base on the basis of these links, with the result of 
this restructuring being the representation of the body of the information 
accumulated as a logical system of classes. The latter system can be treated as an 
informational model of the problem examined by the expert (for example, the 
social and political situation in a particular region of the world). A~general scheme 
of operation of the system is shown in Fig.~2.



     As a source of information, this system uses the CNN Internet site ({\sf 
http://cnn.com}). Several times a day, this site publishes information covering many 
aspects of social and political life in many countries. In most cases, the 
informational messages are weakly-structured text documents. In order to establish 
links between different documents, the vocabulary group technique described 
above is used (Fig.~3). If various informational messages contain common 
keywords belonging to different vocabulary groups, this technique estimates the 
``likeness'' of the messages. The similarity function classifies these links as 
positive or negative, which makes it possible to construct a~connectivity matrix on 
the set of the informational messages received by the user (see Fig.~3).
     
    


     The mode of ``Keywords'' allows one to get~10 of the most significant key 
words for a~given document with an indication of their weighting factors (Fig.~4).
     
     
     The mode of interrelations (``Correlations'') will allow to get several 
documents that have the greatest interrelations with selected document. This mode 
works only if the loaded document belongs to the current project of the iiProcessor 
system, in which the relationship was evaluated (Fig.~5).
    
     The choice of the CNN server as a source of information is explained by the 
fact that this server is one of the most informationally abundant servers providing 
real-time information. Of course, the choice of the sources of information is 
strongly determined by the character of the problem considered. In this sense, the 
CNN server is not universal. In view of the above considerations, the Restructor 
system is implemented as a~complex of two program modules. The rsn.exe module 
is the basic one. An auxiliary iip.class module executes a real-time search for new 
information in a specified information source in the Internet. With such an 
architecture, this\linebreak\vspace*{-12pt}

{ \begin{center}  %fig4
 \vspace*{-7pt}
     \mbox{%
 \epsfxsize=79mm 
 \epsfbox{dul-4.eps}
 }


\vspace*{4pt}


\noindent
{{\figurename~4}\ \ \small{``Keywords'' mode}}
\end{center}
}

\vspace*{2pt}


{ \begin{center}  %fig5
 \vspace*{-1pt}
    \mbox{%
 \epsfxsize=79mm 
 \epsfbox{dul-5.eps}
 }


\vspace*{4pt}


\noindent
{{\figurename~5}\ \ \small{``Correlation'' mode}}
\end{center}
}

%\vspace*{3pt}



\noindent
 system can be adopted to operation with any informational 
servers in the Internet (and beyond) by replacing only the auxiliary module, 
without changing its kernel where the major mathematical results of the authors' 
approach are implemented.
     
\section{Concluding Remarks}

\noindent
The implementation of the results of Theorems~1 and~2 in the inference engine 
made it possible to considerably reduce the time expenses of the built-in algorithm 
for restructuring the database. The use of the connectivity matrix as the major 
visualization means for the informational objects improved the clearness of the 
representation of the information model of the problem considered by an expert. 
The system has been tested in analyzing the events related to NASA's 
aerospace research.
     
    % \Ack
    % \noindent
    % This work was supported by the Russian Foundation for Basic Research, 
%project No.\,20-07-00329~А.
     
     \renewcommand{\bibname}{\protect\rmfamily References}
     
     
     \vspace*{-9pt}
     
     {\small\frenchspacing
     {\baselineskip=10.45pt
     \begin{thebibliography}{99}
     
     \bibitem{1-d} %1
\Aue{Dasarathy, B.} 2001. Information fusion~--- what, where, why, when, and how? 
\textit{Inform. Fusion} 2(2):75--76.
     
     \bibitem{4-d} %2
\Aue{Dulin, S.\,K.} 1995. The approach to structural consistency of situations' models in 
an active knowledge base. \textit{Workshop of 10th IEEE Symposium 
(International) on Intelligent Control Proceedings}. Monterey, CA: AdRem, Inc. 
253--258.

\bibitem{3-d} %3
\Aue{Duckham, M., and M.~Worboys.} 2007. Automated geographic information 
fusion and ontology alignment. \textit{Spatial data on the Web}. Eds. A.~Belussi, 
B.~Catania, E.~Clementini, and E.~Ferrari.
Berlin: Springer. Ch.~6:109--132. 

\bibitem{2-d} %4
\Aue{Pravia, M.} 2008. Generation of a~fundamental data set for hard/soft information 
fusion. \textit{11th Conference (International) on Information Fusion Proceedings}. 
Cologne: International Society of Information Fusion. 134--145.





\bibitem{5-d} %5
\Aue{Landauer, T.\,K., K.~Kireyev, and C.~Panaccione.} 2011. Word maturity: A~new 
metric for word knowledge. \textit{Sci. Stud. Read.} 15(1):92--108. 

\bibitem{7-d} %6
\Aue{Dulina, N., and O.~Kozhunova.} 2010. Information monitoring system: 
A~problem 
of linguistic resources consistency and verification. \textit{Problems of Cybernetics and 
Informatics: 3rd Conference (International) Proceedings}. Baku.  
56--58.
\bibitem{6-d} %7
\Aue{Dulin, S.\,K., and  N.\,G.~Dulina.} 2018. Ispol'zovanie disseminatsionnykh 
algoritmov dlya formirovaniya nestrukturirovannoy tekstovoy informatsii v~baze 
geodannykh [Using dissemination algorithms for the formation of unstructured textual 
information in the geodatabase]. \textit{Sistemy i~Sredstva Informatiki~--- Systems and 
Means of Informatics} 28(2):42--59.

\end{thebibliography}}}

\end{multicols}

\vspace*{-6pt}

\hfill{\small\textit{Received February 26, 2019}}

\vspace*{-16pt}

\Contr

%\vspace*{-3pt}

\noindent
\textbf{Dulin Sergey K.} (b.\ 1950)~--- Doctor of Science in technology, 
professor, leading scientist, Institute of Informatics Problems, Federal Research 
Center ``Computer Science and Control'' of the Russian Academy of Sciences,  
44-2~Vavilov Str., Moscow 119333, Russian Federation; principal scientist, 
Research \& Design Institute for Information Technology, Signalling and 
Telecommunications on Railway Transport (JSC NIIAS), 27-1~Nizhegorodskaya 
Str., Moscow 109029, Russian Federation; \mbox{skdulin@mail.ru} 

\vspace*{3pt}

\noindent
\textbf{Dulina Natalia G.} (b.\ 1947)~--- Candidate of Science (PhD) in 
technology, leading programmer, A.\,A.~Dorodnicyn Computing Center, Federal 
Research Center ``Computer Science and Control'' of the Russian Academy of 
Sciences, 40~Vavilov Str., Moscow 119333, Russian Federation; 
\mbox{ngdulina@mail.ru}
\vspace*{3pt}

\noindent
\textbf{Ermakov Petr V.} (b.\ 1985)~--- Senior Software Developer, TeleRetail 
GmbH, 30~\mbox{Markenstra{\!\ptb{\ss}}e}, D$\ddot{\mbox{u}}$sseldorf 
40227,  Germany; \mbox{petcazay@gmail.com}

 

%\newpage

\vspace*{8pt}

\hrule

\vspace*{2pt}

\hrule

%\vspace*{-7pt}

%\newpage

%\vspace*{-28pt}

\def\tit{ИНФОРМАЦИОННЫЙ СИНТЕЗ ДОКУМЕНТОВ}

\def\titkol{Информационный синтез документов}

\def\aut{С.\,К.~Дулин$^1$, Н.\,Г.~Дулина$^2$, П.\,В.~Ермаков$^3$}

\def\autkol{С.\,К.~Дулин, Н.\,Г.~Дулина, П.\,В.~Ермаков}

%{\renewcommand{\thefootnote}{\fnsymbol{footnote}} \footnotetext[1]
%{Работа was supported by the Russian Foundation for Basic Research, project No.\,20-07-00329~А.}}



\titel{\tit}{\aut}{\autkol}{\titkol}

\vspace*{-11pt}

\noindent
$^1$Институт проблем информатики Федерального исследовательского центра <<Информатика 
и~управление>>\linebreak
$\hphantom{^1}$Российской академии наук, \mbox{skdulin@mail.ru}

\noindent
$^2$Вычислительный центр им.\ А.\,А.~Дородницына Федерального исследовательского центра 
<<Информатика\linebreak
$\hphantom{^1}$и~управление>> Российской академии наук, \mbox{ngdulina@mail.ru}

\noindent
$^3$TeleRetail GmbH, D$\ddot{\mbox{u}}$sseldorf, Germany

\vspace*{1pt}

\def\leftfootline{\small{\textbf{\thepage}
\hfill ИНФОРМАТИКА И ЕЁ ПРИМЕНЕНИЯ\ \ \ том\ 14\ \ \ выпуск\ 1\ \ \ 2020}
}%
 \def\rightfootline{\small{ИНФОРМАТИКА И ЕЁ ПРИМЕНЕНИЯ\ \ \ том\ 14\ \ \ 
выпуск\ 1\ \ \ 2020
\hfill \textbf{\thepage}}}

\vspace*{-1pt}



\Abst{Рассматриваются проблемы, связанные с созданием экспертной 
базы документов, требующей оперативной обработки поступающей 
информации и, как следствие, реструктуризации базы знаний. 
Предложены процедуры, уменьшающие время поиска оптимального 
согласованного состояния взаимосвязанных документов. Был 
разработан подход к~оценке взаимосвязи текстовых документов 
и~информационных сообщений как плохо структурированных 
объектов. Описана практическая реализация этого подхода.}

\KW{информационный синтез; контролируемая согласованность 
данных и~знаний; реструктуризация базы знаний}


\DOI{10.14357/19922264200117} 

%\vspace*{-3pt}


 \begin{multicols}{2}

\renewcommand{\bibname}{\protect\rmfamily Литература}
%\renewcommand{\bibname}{\large\protect\rm References}

{\small\frenchspacing
{\baselineskip=10.5pt
\begin{thebibliography}{99}
%\vspace*{-3pt} 

\bibitem{1-d-1} %1
\Au{Dasarathy B.} Information fusion~--- what, where, why, when, and how?~// 
Inform. Fusion, 2001. Vol.~2. Iss.~2. P.~75--76.

\bibitem{4-d-1} %2
\Au{Dulin S.\,K.} The approach to structural consistency of situations' models in an 
active knowledge base~// Workshop of 10th IEEE Symposium 
(International) on Intelligent Control Proceedings.~--- Monterey, CA, USA: AdRem, 
Inc., 1995. P.~253--258.

\bibitem{3-d-1} %3
\Au{Duckham M., Worboys~M.} Automated geographic information fusion and 
ontology alignment~// Spatial data on the Web~/ Eds. A.~Belussi, B.~Catania, 
E.~Clementini, E.~Ferrari.~--- Berlin: Springer, 2007. Ch.~6. P.~109--132. 

\bibitem{2-d-1} %4
\Au{Pravia M.} Generation of a fundamental data set for hard/soft information 
fusion~// 11th Conference (International) on Information Fusion.~--- Cologne: 
International Society of Information Fusion, 2008. P.~134--145.




\bibitem{5-d-1} %5
\Au{Landauer T.\,K., Kireyev~K., Panaccione~C.} Word maturity: A~new metric 
for word knowledge~// Sci. Stud. Read., 2011. Vol.~15. Iss.~1. 
P.~92--108. 

\bibitem{7-d-1} %6
\Au{Dulina N., Kozhunova~O.} Information monitoring system: A~problem of 
linguistic resources consistency and verification~// Problems of Cybernetics and 
Informatics: 3rd Conference (International) Proceedings.~--- Baku, 2010.  
P.~56--58.
\bibitem{6-d-1} %7
\Au{Дулин С.\,К., Дулина~Н.\,Г.} Использование диссеминационных 
алгоритмов для формирования неструктурированной текстовой информации 
в базе геоданных~// Системы и средства информатики, 2018. Т.~28. №\,2. 
С.~42--59. 

\end{thebibliography}
} }

\end{multicols}

 \label{end\stat}

 \vspace*{-9pt}

\hfill{\small\textit{Поступила в~редакцию 26.02.2019}}


%\renewcommand{\bibname}{\protect\rm Литература}
\renewcommand{\figurename}{\protect\bf Рис.}
\renewcommand{\tablename}{\protect\bf Таблица}      %8+
\def\stat{listopad}

\def\tit{ЖИЗНЕННЫЙ ЦИКЛ МЕТОДОЛОГИИ ПОСТРОЕНИЯ РЕФЛЕКСИВНО-АКТИВНЫХ 
СИСТЕМ ИСКУССТВЕННЫХ ГЕТЕРОГЕННЫХ ИНТЕЛЛЕКТУАЛЬНЫХ АГЕНТОВ$^*$}

\def\titkol{Жизненный цикл методологии построения РАСИГИА} %рефлексивно-активных систем искусственных гетерогенных интеллектуальных агентов}

\def\aut{С.\,В.~Листопад$^1$}

\def\autkol{С.\,В.~Листопад}

\titel{\tit}{\aut}{\autkol}{\titkol}

\index{Листопад С.\,В.}
\index{Listopad S.\,V.}


{\renewcommand{\thefootnote}{\fnsymbol{footnote}} \footnotetext[1]
{Исследование выполнено за счет гранта Российского научного фонда №\,23-21-00218, 
{\sf https://rscf.ru/project/23-21-00218/}.}}


\renewcommand{\thefootnote}{\arabic{footnote}}
\footnotetext[1]{Федеральный исследовательский центр <<Информатика и~управ\-ле\-ние>> Российской академии наук, 
\mbox{ser-list-post@yandex.ru}}

%\vspace*{-12pt}

  
  

  \Abst{Представлена темпоральная структура (жизненный цикл) методологии построения 
рефлексивно-активных систем искусственных гетерогенных интеллектуальных агентов (\mbox{РАСИГИА}). 
Такие системы предназначены для компьютерного моделирования процессов и~эффектов, 
возникающих при решении практических проблем коллективами специалистов под 
руководством лица, принимающего решения. Искусственные гетерогенные 
интеллектуальные агенты реф\-лек\-сив\-но-ак\-тив\-ных сис\-тем~--- активные субъекты, 
способные к~рассуждениям, коммуникации и~рефлексии как умению моделировать 
рассуждения других агентов системы и~себя самих. Моделирование рефлексивных процессов 
обеспечивает выработку агентами согласованного представления об объекте управ\-ле\-ния, 
цели коллективной работы и~нормах взаимодействия, позволяя системе в~ходе 
самоорганизации генерировать заново релевантный гибридный интеллектуальный метод 
решения очередной проб\-лемы.} 
  
  \KW{рефлексия; методология; рефлексивно-активная сис\-те\-ма искусственных 
гетерогенных интеллектуальных агентов; гибридная интеллектуальная многоагентная 
система; коллектив специалистов}

\DOI{10.14357/19922264240112}{GUAMVE}
  
%\vspace*{-6pt}


\vskip 10pt plus 9pt minus 6pt

\thispagestyle{headings}

\begin{multicols}{2}

\label{st\stat}

\section{Введение}

  Компьютерное моделирование процессов и~эффектов, возникающих при 
решении практических проблем коллективами специалистов, каждый из 
которых обладает собственным опытом, знаниями и~пониманием предметной 
области,~--- перспективное на\-прав\-ле\-ние научных разработок, которое 
Д.\,А.~Поспелов выделял как одну из десяти горячих точек в~исследованиях по 
искусственному\linebreak интеллекту~[1]. Для компьютерного моделирования 
рас\-суж\-де\-ний коллективов специалистов предлагается создание \mbox{РАСИГИА} 
в~рамках многоагентного подхода~[2] на основе модели 
\mbox{ги\-брид\-ных} интеллектуальных многоагентных сис\-тем~[3]. Агенты 
\mbox{РАСИГИА}~--- активные программные сущности, способные 
рас\-суж\-дать, взаимодействовать и~рефлексировать. Рефлексивное 
моделирование агентами друг друга обеспечивает выработку согласованного 
пред\-став\-ле\-ния об объекте управ\-ле\-ния, \mbox{цели} коллективной работы и~нормах 
взаимодействия, а~также эволюцию \mbox{РАСИГИА} в~ходе 
самоорганизации в~сильном смыс\-ле. В~на\-сто\-ящей работе рас\-смат\-ри\-ва\-ют\-ся 
вопросы создания методологии разработки сис\-тем такого класса, которая 
понимается как учение об организации продуктивной де\-я\-тель\-ности 
в~це\-лост\-ную сис\-те\-му с~чет\-ко определенными характеристиками, логической 
структурой и~процессом ее осуществления (темпоральной структурой)~[4]. 
Характеристики (особенности и~принципы) и~логическая структура (субъект, 
объект, предмет, методы, средства, результат) методологии разработки 
\mbox{РАСИГИА} рас\-смот\-ре\-ны в~[5]. Данная работа по\-свя\-ще\-на разработке 
жизненного цик\-ла (темпоральной структуры) предлагаемой методологии.

\begin{figure*} %fig1
\vspace*{1pt}
      \begin{center}
     \mbox{%
\epsfxsize=148.855mm 
\epsfbox{lis-1.eps}
}
\end{center}
%\vspace*{-9pt}

{\small Темпоральная структура методологии построения РАСИГИА: \textit{1}~--- этап методологии; \textit{2}~--- стадия методологии;
\textit{3}~--- граница фазы методологии; \textit{4}~--- отношение следования при нормальном завершении этапа;  
\textit{5}~--- возврат к~предыдущим этапам при выявлении допущенных на них недочетов}
\end{figure*}

\vspace*{-6pt}
  
\section{Темпоральная структура методологии}

\vspace*{-6pt}

  Укрупненно в~жизненном цикле методологии построения \mbox{РАСИГИА}, 
показанном на рисунке, могут быть выделены проектная, технологическая 
и~рефлексивная фазы, которые со\-сто\-ят из стадий и~этапов. Как видно, 
последовательное выполнение этапов методологии приводит к~же\-ла\-емо\-му 
результату лишь в~идеальном случае, когда проектировщик сразу получает всю 
необходимую достоверную информацию, имеет необходимый арсенал методов, 
не совершает ошибок ни на одном из этапов и,~по сути, заранее знает, какой 
должна быть разрабатываемая \mbox{РАСИГИА}. В~реальности на каждом из 
этапов могут обнаруживаться ранее допущенные недочеты, требующие 
возврата к~соответствующему этапу, их исправления и~повторного выполнения 
проделанной работы с~новыми исходными данными. В~определенном смыс\-ле 
такой подход представляет собой метод проб и~ошибок, и~чем слож\-нее 
проблема, для которой проектируется \mbox{РАСИГИА}, с~точ\-ки зрения 
конкретного коллектива разработчиков, тем больше будет возвратов в~ходе 
проектирования системы~[6]. Рас\-смот\-рим по\-дроб\-нее каждую из фаз 
методологии.



\section{Проектная фаза}

  Проектная фаза включает в~себя стадии концептуального описания проб\-ле\-мы и~моделирования, выполняемые сис\-тем\-ны\-ми аналитиками из коллектива 
разработчиков. В~рамках первой стадии фазы на доформальном, 
содержательном уровне рас\-смат\-ри\-ва\-ет\-ся проб\-ле\-ма как отрицательное 
отношение субъекта к~реальности~[6] и~проблемная ситуация как объективное 
стечение обстоятельств, обуслов\-ли\-ва\-ющее проб\-ле\-му. Данная стадия со\-сто\-ит из 
сле\-ду\-ющих этапов:
  \begin{itemize}
\item формулирование проб\-ле\-мы, ее предварительное описание в~ходе 
интервьюирования лица, при\-ни\-ма\-юще\-го решение, его советников и~активных 
групп на естественном языке с~использованием привычных для них 
определений и~формулировок~[7];
  \item определение проб\-ле\-ма\-ти\-ки, т.\,е.\ комплекса проб\-лем, связанных 
с~рас\-смат\-ри\-ва\-емой~[4], чтобы учесть создаваемые ее решением последствия 
для каж\-дой из них. Необходимо охватить весь круг участников проб\-лем\-ной 
ситуации (стейкхолдеров, заинтересованных лиц): непосредственных 
участников ситуации, пред\-ста\-ви\-те\-лей проб\-ле\-мо\-раз\-ре\-ша\-ющих 
и~проб\-ле\-мо\-со\-дер\-жа\-щих сис\-тем, же\-ла\-емых помощников или союзников, 
субъектов, связанных с~ситуацией юридически, лиц с~возможным негативным 
отношением к~решению проб\-ле\-мы~[6]. Для по\-стро\-ения проб\-ле\-ма\-ти\-ки может 
быть использована, например, технология Дж.~Уор\-фил\-да, подходы 
с~использованием метафор организации, взгляда на проблему стейкхолдером 
с~раз\-ных точек зрения, рас\-смот\-ре\-ния проб\-ле\-мы в~рамках различных парадигм 
(функциональной, объяснительной, освободительной, пост\-мо\-дер\-нист\-ской)~[4, 6]. 
Формируется древовидная или сетевая структура в~виде диаграммы связей, 
концептуальной кар\-ты или аналогичных инструментов;
  \item определение целей проектирования \mbox{РАСИГИА}, 
пред\-по\-ла\-га\-ющее проведение собеседований с~каж\-дым стейк\-хол\-де\-ром, 
выяснение их целей и~пожеланий, формирование и~структурирование 
множества целей в~виде дерева или сетевидной структуры и~его 
визуализация~[4, 6]. Выделяются следующие уровни целей: ожи\-да\-емые 
в~плановом периоде результаты; задачи, которые не будут решены 
в~рас\-смат\-ри\-ва\-емом периоде, но будет достигнут существенный прогресс на 
пути к~ним; не\-до\-сти\-жи\-мые идеалы, к~которым следует стремиться~[8];
  \item выбор критериев, т.\,е.\ до\-ступ\-ных для наблюдения и~измерения 
характеристик, опи\-сы\-ва\-ющих важ\-ные особенности объектов или процессов 
и~поз\-во\-ля\-ющих сравнивать \mbox{пред\-ла\-га\-емые} альтернативы, контролировать 
процесс решения~[6]. Со\-во\-куп\-ность критериев долж\-на быть релевантной 
количественной моделью выделенных ранее качественных целей. Отдельно 
выделяются ограничения, фик\-си\-ру\-ющие условия, которые не могут нарушаться 
при до\-сти\-же\-нии цели;
  \item оценка концептуального описания проб\-ле\-мы в~ходе специально 
спланированного эксперимента. Если существует коллектив специалистов, 
решающий на практике по\-став\-ле\-нную или схожие проб\-ле\-мы, он выступает 
образцом, прототипом создаваемой сис\-те\-мы агентов. В~этом случае 
выполняется наблюдение за работой такого коллектива в~рамках решения 
реальных или тренировочных проб\-лем и~оценка релевантности 
зафиксированной информации сведениям, полученным в~ходе предыду\-щих 
этапов. Если выявлено существенное рас\-хож\-де\-ние, выполняется возврат 
к~этапу, в~рамках которого были получены некорректные сведения. Сведения 
о~составе участников коллектива, вы\-де\-ля\-емых ими под\-проб\-ле\-мах, методах их 
решения используются на по\-сле\-ду\-ющих этапах проектирования 
\mbox{РАСИГИА} при по\-стро\-ении со\-от\-вет\-ст\-ву\-ющих моделей проб\-ле\-мы 
и~сис\-те\-мы <<как есть сейчас>>. Данные о~качестве принятых решений 
и~дли\-тель\-ности их выработки используются в~дальнейшем как показатель 
эффекта от разработки и~внед\-ре\-ния \mbox{РАСИГИА}. Если подобных 
коллективов нет или не\-воз\-мож\-но реализовать со\-от\-вет\-ст\-ву\-ющий эксперимент, 
данный этап отсутствует.
  \end{itemize}
  
  Стадия моделирования предполагает разработку формализованного описания 
проб\-ле\-мы, коллектива специалистов, ре\-ша\-юще\-го проб\-ле\-му на момент 
разработки \mbox{РАСИГИА}, если он существует, и~самой 
\mbox{РАСИГИА}. Модели строятся с~использованием визуального 
метаязыка~[9], что позволяет наглядно их изобразить, а~так\-же поз\-во\-ля\-ет 
с~использованием заранее заданных соответствий однозначно отоб\-ра\-зить 
графическое пред\-став\-ле\-ние моделей в~формальное символьное пред\-став\-ле\-ние, 
пригодное для компьютерной интерпретации. Данная стадия со\-сто\-ит из 
сле\-ду\-ющих этапов:
  \begin{itemize}
  \item моделирование проб\-ле\-мы, которое обеспечивает ее формальное 
пред\-став\-ле\-ние на макро- и~мик\-ро\-уров\-не. Мак\-ро\-уров\-не\-вая модель описывает 
проб\-ле\-му как <<чер\-ный ящик>>, отражая ее место в~ме\-та\-проб\-ле\-ме (проб\-ле\-ме 
более высокого уровня), свойства как целого и~связи с~другими проб\-ле\-ма\-ми 
ме\-та\-проб\-ле\-мы. Атрибуты проблемы на макроуровне~--- цели, критерии 
(включая ограничения), исходные данные и~идентификатор. Мик\-ро\-уров\-не\-вая 
модель раскрывает со\-став и~структуру проб\-ле\-мы, описывает ее под\-проб\-ле\-мы 
и~связи между ними. Для каждой под\-проб\-ле\-мы специфицируются цели, 
критерии, исходные данные и~идентификатор, выполняется поиск релевантных 
методов решения. Если такие методы найдены, дальнейшая декомпозиция 
под\-проб\-ле\-мы не требуется, иначе выполняется по\-стро\-ение ее мик\-ро\-уров\-не\-вой 
модели, т.\,е.\ модели более глубокого уров\-ня иерархии. Таким образом, 
формируется многоуровневая иерархическая структура по\-став\-лен\-ной 
проб\-лемы;
  \item моделирование коллектива, которое отражает ситуацию решения 
проб\-ле\-мы <<как есть сейчас>> со всеми ее преимуществами и~недостатками. 
Модель коллектива~--- основа, образец для проектирования \mbox{РАСИГИА} и~оценки эф\-фек\-тив\-ности 
альтернативных конфигураций \mbox{РАСИГИА}. 
При моделировании коллектива специалистов фиксируется его со\-став в~виде 
множества ролей участников, час\-ти проб\-ле\-мы, ре\-ша\-емые каж\-дым из 
участников с~определенной ролью, знания и~методы, ис\-поль\-зу\-емые 
участниками для решения своей части проб\-ле\-мы, а~так\-же порядок и~нормы 
взаимодействия участников коллектива; 
  \item моделирование \mbox{РАСИГИА}, фор\-ми\-ру\-ющее идеализированное 
пред\-став\-ле\-ние <<как должно стать>> о~коллективе интеллектуальных агентов, 
ре\-ша\-ющих по\-став\-лен\-ную проб\-ле\-му. В~ходе моделирования \mbox{РАСИГИА} 
должны быть специфицированы со\-став и~иерархия ролей агентов, множество 
агентов, ис\-поль\-зу\-емые протоколы взаимодействия, под\-дер\-жи\-ва\-емые языки 
передачи сообщений, базовая онтология как осно\-ва для интерпретации 
семантики пе\-ре\-да\-ва\-емых сообщений, модель окру\-жа\-ющей среды, содержащая 
в~том чис\-ле пул, из которого агенты могут привлекаться сис\-те\-мой по мере 
не\-об\-хо\-ди\-мости и~в~который попадают ис\-клю\-ча\-емые из нее агенты, множество 
моделей архитектур \mbox{РАСИГИА}, множество необходимых моделей 
мак\-ро\-уров\-не\-вых эффектов. В~множестве агентов должны присутствовать 
агенты, пред\-став\-ля\-ющие стейк\-хол\-де\-ров с~их целями, критериями достижения 
цели и~ограничениями. Если на предыду\-щем этапе была по\-стро\-ена модель 
коллектива, то одна из архитектур \mbox{РАСИГИА} долж\-на соответствовать 
данной модели. 
  \end{itemize}
  
\section{Технологическая фаза}

  Технологическая фаза включает в~себя разработку эскизного проекта 
\mbox{РАСИГИА}, ее технического проекта и~программной реализации. 
Стадия разработки эскизного проекта \mbox{РАСИГИА} обеспечивает 
пред\-став\-ле\-ние создаваемой сис\-те\-мы и~ее внеш\-ней среды в~виде 
взаимосвязанных мо\-ду\-лей-бло\-ков в~соответствии с~моделью 
\mbox{РАСИГИА}, по\-стро\-ен\-ной на стадии проектирования. Данная стадия 
со\-сто\-ит из сле\-ду\-ющих этапов:
  \begin{itemize}
  \item разработка функциональной структуры, в~ходе которой строится 
множество взаимосвязанных схем-диа\-грамм, определяющих под\-сис\-те\-мы 
РАСИГИА, распределение агентов по ним, функционал агентов, до\-пус\-ти\-мые 
языки передачи сообщений и~протоколы взаимодействия для каж\-дой пары или 
группы ролей агентов, технологические элементы сис\-те\-мы, потоки 
информации и~управ\-ле\-ния, а~также отношения, воз\-ни\-ка\-ющие между агентами 
в~процессе решения проб\-лем. Для каждой роли указывается множество 
релевантных ей уже существующих (разработанных ранее для других сис\-тем) 
агентов, если таковые имеются. В~случае отсутствия релевантных агентов они 
должны быть разработаны на сле\-ду\-ющих этапах. Кроме того, 
специфицируются функциональные мо\-ду\-ли-бло\-ки, от\-ве\-ча\-ющие за организацию 
макроуровневых эффектов в~\mbox{РАСИГИА};
  \item разработка структуры внешней среды по аналогии с~разработкой 
функциональной структуры \mbox{РАСИГИА} предполагает построение схем-диа\-грамм, описывающих виртуальную внеш\-нюю среду, ее под\-сис\-те\-мы, роли 
агентов и~способы взаимодействия \mbox{РАСИГИА} с~ними, т.\,е.\ языки 
передачи сообщений и~протоколы взаимодействия, отношения, потоки 
информации и~управ\-ле\-ния. Для каж\-дой роли указываются су\-щест\-ву\-ющие 
релевантные ей агенты, если они имеются;
  \item разработка архитектур агентов выполняется для тех ролей 
в~функциональной структуре и~структуре внеш\-ней среды, для которых не 
найдено релевантных реализованных агентов. Архитектура агента~--- схема, 
описывающая со\-став, структуру и~взаимосвязь функ\-ций-бло\-ков, 
ре\-а\-ли\-зу\-емых агентом, обеспечивающая выполнение им своего предназначения. 
Для каждой функ\-ции-бло\-ка указывается метод или алгоритм, с~по\-мощью 
которого она реализуется, в~случае если таковые отсутствуют, они долж\-ны 
быть разработаны в~рамках сле\-ду\-ющей стадии.
  \end{itemize}
  
  Стадия разработки технического проекта \mbox{РАСИГИА} обеспечивает 
создание недостающих блоков для ее агентов или технологических элементов. 
При этом может по\-тре\-бо\-вать\-ся разработка методов решения под\-проб\-лем, 
алгоритмов на основе метода, баз данных, онтологий и~др. Порядок их 
разработки не регламентируется на\-сто\-ящей методологией в~связи 
с~существенным разнообразием и~не\-воз\-мож\-ностью совместного рас\-смот\-ре\-ния. 
На данной стадии должен быть сформирован технический проект, 
опи\-сы\-ва\-ющий для каждого блока со\-став, структуру и~форму пред\-став\-ле\-ния 
входных и~выходных данных, алгоритм его функционирования, спецификацию 
необходимых технических средств~[10].
  
  Стадия программной реализации и~отладки предполагает разработку 
программного кода \mbox{РАСИГИА} и~его тестирование на предмет 
корректной работы с~\mbox{целью} формирования полноценного программного 
продукта, а~так\-же разработку программной документации. Данная стадия 
со\-сто\-ит из сле\-ду\-ющих этапов:
  \begin{itemize}
  \item программная реализация и~разработка документации выполняется 
с~использованием платформы JaCaMo~[11], объединяющей технологию Jason 
для программирования автономных агентов, Cartago для программирования 
элементов внеш\-ней среды и~Moise для программирования многоагентных 
организаций. Кроме того, применяется язык Java для программирования 
отдельных элементов сис\-те\-мы и~тонкой настройки механизмов 
платформы~[12];
  \item тестирование и~отладка обеспечивают выявление и~устранение 
основных дефектов в~сис\-те\-ме. Ввиду того что полное тестирование  
сколь\-ко-ни\-будь слож\-ной программы не\-воз\-мож\-но~[13], выполняется 
выборочное тестирование в~сле\-ду\-ющем порядке: отдельные функ\-ции и~блоки 
из состава аген\-тов и~технологических элементов, межмодульные связи, агенты 
и~технологические элементы в~целом, протоколы взаимодействия агентов, 
\mbox{РАСИГИА} в~целом. В~тес\-ти\-ро\-ва\-нии принимают участие 
представители всех ролей команды разработчиков, так как каж\-дый из них 
выполняет поиск ошибок разного рода~[14]. При этом выделяется отдельная 
роль тестировщика, опре\-де\-ля\-юще\-го стратегию тес\-ти\-ро\-ва\-ния,  
тест-тре\-бо\-ва\-ния и~тест-пла\-ны для каждой из фаз проекта; он выполняет 
тестирование сис\-те\-мы, собирает и~анализирует отчеты о~про\-хож\-де\-нии 
тестирования. 
\end{itemize}

\section{Рефлексивная фаза}

  Рефлексивная фаза предназначена для оценки показателей реализованной 
\mbox{РАСИГИА} и~процесса ее разработки, выявления ее недостатков и~при 
не\-об\-хо\-ди\-мости до\-ра\-бот\-ки как сис\-те\-мы, так и~методологии ее построения. 
Стадия оценки эф\-фек\-тив\-ности \mbox{РАСИГИА} предполагает сбор 
показателей работы сис\-те\-мы и~их сравнение с~целевыми значениями. Если 
выявляется их несоответствие, выполняется анализ причин отклонений, 
переход к~этапу методологии, вызвавшему их, и~повторное выполнение данного и~по\-сле\-ду\-ющих этапов с~учетом тре\-бу\-емых корректировок. Кроме того, на этой 
стадии продолжается отладка сис\-те\-мы. Данная стадия выполняется в~три этапа:
  \begin{enumerate}[(1)]
  \item оценка в~лабораторных условиях командой разработчиков, когда 
система работает в~виртуальной внешней среде, решая тестовые проб\-ле\-мы. На 
данной стадии оценка сис\-те\-мы выполняется вычислительными моделями 
стейкхолдеров, реализованными со\-от\-вет\-ст\-ву\-ющи\-ми агентами виртуальной 
внеш\-ней среды; 
  \item оценка по результатам тестовой эксплуатации, когда \mbox{РАСИГИА} 
функционирует в~реальной внеш\-ней среде параллельно с~традиционным 
методом решения проб\-ле\-мы и~выполняется сравнение их эф\-фек\-тив\-ности 
пользователями и~реальными стейк\-хол\-де\-ра\-ми. Первоначально 
у~\mbox{РАСИГИА} должна быть отключена воз\-мож\-ность оказывать ка\-кое-ли\-бо воздействие на реальную внеш\-нюю среду, а~результатом ее\linebreak
 работы 
долж\-ны стать рекомендации по оказанию таких воздействий. После 
удовлетворительной оценки пользователей и~стейк\-хол\-де\-ров \mbox{РАСИГИА} 
может быть переведена в~\mbox{автоматический} режим взаимодействия со средой, 
а~традиционный метод решения проб\-ле\-мы используется в~качестве резервного 
для проверки ее работы еще в~течение некоторого времени. Длительности 
каждого из этих периодов долж\-ны определяться заказчиками сис\-те\-мы для 
решения конкретной проб\-ле\-мы совместно с~коллективом разработчиков; 
  \item сопровождение после внед\-ре\-ния поз\-во\-ля\-ет собирать жалобы, замечания и~предложения в~процессе эксплуатации \mbox{РАСИГИА}, в~том чис\-ле от 
людей, которые ошибочно не были включены в~со\-став стейкхолдеров.\\[-13pt] 
  \end{enumerate}
  
  Стадия оценки и~корректировки методологии в~определенном смысле длится 
на протяжении всего проекта, так как для ее реализации долж\-ны вес\-тись 
протоколы де\-я\-тель\-ности разработчиков,\linebreak в~которых отмечается дли\-тель\-ность 
реализации каж\-до\-го этапа, выполненные возвраты и~их причины. Однако 
именно по завершении проекта выполняется рефлексия проделанной работы, 
когда разработчики долж\-ны проанализировать удачные и~провальные решения, 
причины рас\-хож\-де\-ния результатов с~планами, возвратов к~предыду\-щим этапам 
и~фазам разработки \mbox{РАСИГИА}, затягивания отдельных этапов 
разработки, из\-бы\-точ\-ность или, наоборот, не\-ин\-фор\-ма\-тив\-ность 
по\-стро\-ений~\cite{4-lis}. По результатам анализа в~методологию вносятся 
изменения в~статусе <<предложение>>, которые после под\-тверж\-де\-ния 
эф\-фек\-тив\-ности в~новых проектах закрепляются в~новой версии методологии.

\vspace*{-9pt}

\section{Заключение}

\vspace*{-3pt}

  В работе представлена темпоральная структура (жизненный цикл) 
разработки \mbox{РАСИГИА}, опи\-сы\-ва\-ющая процессы сис\-тем\-но\-го анализа 
проб\-ле-\linebreak мы, моделирования, эскизного и~технического \mbox{проектирования} сис\-те\-мы, 
ее программной реализации, отладки и~тестирования. 
Основной результат 
организации работ в~соответствии с~предложенной методологией~--- 
программная реализация \mbox{РАСИГИА}, релевантно моделирующая 
коллектив специалистов, со\-вмест\-но ре\-ша\-ющих по\-став\-лен\-ную проб\-ле\-му 
с~учетом ее слабой формализации, не\-од\-но\-род\-ности, сетевого характера условий 
и~целей, не\-опре\-де\-лен\-ности и~ди\-на\-мич\-ности~\cite{5-lis}. Кроме того, в~результате 
рефлексивной стадии методологии формируется ее новая версия или 
подтверждается эф\-фек\-тив\-ность су\-щест\-ву\-ющей, что представляется\linebreak 
дополнительным результатом работ. Таким образом, методология предполагает 
свое развитие, потенциально обеспечивающее ее ре\-ле\-вант\-ность \mbox{актуальным}
подходам к~проектированию и~реализации интеллектуальных информационных 
сис\-тем.

\vspace*{-9pt}
  
{\small\frenchspacing
 { %\baselineskip=10.6pt
 %\addcontentsline{toc}{section}{References}
 \begin{thebibliography}{99}
 
 \vspace*{-3pt}
 
  \bibitem{1-lis}
   \Au{Поспелов Д.\,А.} Десять <<горячих точек>> в~исследованиях по искусственному 
интеллекту~// Искусственный\linebreak\vspace*{-12pt}

\columnbreak

\noindent
 интеллект и~принятие решений, 2019. №\,4. С.~3--9. doi: 
10.14357/20718594190401. EDN: BAUHFV.
  
  \bibitem{2-lis}
\Au{Тарасов В.\,Б.} От многоагентных сис\-тем к~интеллектуальным организациям: 
философия, психология, информатика.~--- М.: Эдиториал УРСС, 2002. 348~с.
  \bibitem{3-lis}
  \Au{Колесников А.\,В., Кириков~И.\,А., Листопад~С.\,В.} Ги\-брид\-ные интеллектуальные 
сис\-те\-мы с~самоорганизацией: координация, со\-гла\-со\-ван\-ность, спор.~--- М.: ИПИ РАН, 2014. 
189~с.
  \bibitem{4-lis}
  \Au{Новиков А.\,М., Новиков~Д.\,А.} Методология.~--- М.: Синтег, 2007. 668~с.
  \bibitem{5-lis}
  \Au{Листопад С.\,В.} Характеристики и~логическая структура методологии по\-стро\-ения  
реф\-лек\-сив\-но-ак\-тив\-ных сис\-тем искусственных гетерогенных интеллектуальных 
агентов~// Сис\-те\-мы и~средства \mbox{информатики}, 2023. Т.~33. №\,4. С.~16--27. doi: 
10.14357/ 08696527230402. EDN: TRTHEI.
  \bibitem{6-lis}
  \Au{Тарасенко Ф.\,П.} Прикладной сис\-те\-мный анализ.~--- М.: 
КНОРУС, 2010. 224~с.
  \bibitem{7-lis}
  \Au{Ларичев О.\,И.} Вербальный анализ решений.~--- М.: Наука, 2006. 181~с.
  \bibitem{8-lis}
  \Au{Акофф Р.} Акофф о менеджменте~/ Пер.\ с~англ.~--- СПб.: Питер, 2002. 448~с.
  (\Au{Akoff~R.\,L.} Ackoff's best: His classic writings on management.~--- New 
York, NY, USA: Wiley, 1999. 368~p.)
  \bibitem{9-lis}
  \Au{Колесников А.\,В., Листопад~С.\,В., Румовская~С.\,Б., Данишевский~В.\,И.} 
Неформальная аксиоматическая тео\-рия ролевых визуальных моделей~// Информатика и~её 
применения, 2016. Т.~10. Вып.~4. С.~114--120.  doi: 10.14357/19922264160412. EDN: XGSIVN.
  \bibitem{10-lis}
  \Au{Черушева Т.\,В.} Проектирование программного обеспечения.~--- Пенза: ПГУ, 2014. 
172~с.
  \bibitem{11-lis}
  \Au{Boissier O., Bordini~R.\,H., Hubnerand~J., Ricci~A.} Multi-agent oriented programming: 
Programming multi-agent systems using JaCaMo.~--- Intelligent robotics and autonomous agents 
series.~--- Cambridge: The MIT Press, 2020. 264~p.
  \bibitem{12-lis}
  \Au{Смирнов С.\,С., Смольянинова~В.\,А.} Введение в~разработку многоагентных сис\-тем 
в~среде Jason. Основы программирования на языке AgentSpeak.~--- М.: \mbox{МИРЭА}, 2009. 136~с.
  \bibitem{13-lis}
  \Au{Канер~С., Фолк~Д., Нгуен~Е.\,К.} Тестирование про\-грам\-мно\-го обеспечения. 
Фундаментальные концепции менеджмента биз\-нес-при\-ло\-же\-ний~/
Пер. с~англ.~--- Киев: ДиаСофт, 
2001. 544~с. (\Au{Kaner~С., Falk~J., Nguyen~H.\,Q.} {Testing computer software}.~--- 
International Thomson Computer Press,  1999. 496~p.)
  \bibitem{14-lis}
  \Au{Романькова Т.\,Л.} Тестирование программного обеспечения. {\sf 
https://elib.gstu.by/bitstream/handle/220612/ 9860/416.pdf}.

\end{thebibliography}

 }
 }

\end{multicols}

\vspace*{-6pt}

\hfill{\small\textit{Поступила в~редакцию 25.11.23}}

%\vspace*{8pt}

%\pagebreak

\newpage

\vspace*{-28pt}

%\hrule

%\vspace*{2pt}

%\hrule



\def\tit{LIFE CYCLE OF METHODOLOGY FOR~CONSTRUCTING REFLEXIVE-ACTIVE SYSTEMS OF~ARTIFICIAL HETEROGENEOUS INTELLIGENT AGENTS}


\def\titkol{Life cycle of methodology for~constructing reflexive-active systems of~artificial heterogeneous intelligent agents}


\def\aut{S.\,V.~Listopad}

\def\autkol{S.\,V.~Listopad}

\titel{\tit}{\aut}{\autkol}{\titkol}

\vspace*{-8pt}


\noindent
Federal Research Center ``Computer Science and Control'' of the Russian Academy of 
Sciences, 44-2~Vavilov Str., Moscow 119333, Russian Federation

\def\leftfootline{\small{\textbf{\thepage}
\hfill INFORMATIKA I EE PRIMENENIYA~--- INFORMATICS AND
APPLICATIONS\ \ \ 2024\ \ \ volume~18\ \ \ issue\ 1}
}%
 \def\rightfootline{\small{INFORMATIKA I EE PRIMENENIYA~---
INFORMATICS AND APPLICATIONS\ \ \ 2024\ \ \ volume~18\ \ \ issue\ 1
\hfill \textbf{\thepage}}}

\vspace*{4pt}
  
  
   
   \Abste{The paper presents the temporal structure (life cycle) of the methodology for 
constructing reflexive-active systems of artificial heterogeneous intelligent agents. These systems 
are designed for computer modeling of processes and effects that arise when solving practical 
problems by teams of specialists under the guidance of a~decision maker. Artificial heterogeneous 
intelligent agents of reflexive-active systems are active subjects capable of reasoning, 
communication, and reflection as the ability to model the reasoning of other agents of the system 
and themselves. Modeling of reflexive processes ensures the development by agents of a~consistent 
understanding of the control object, the purpose of collective work, and the norms of interaction 
allowing the system to self-organize and re-develop a relevant hybrid intelligent method for solving 
the next problem.}
   
   \KWE{reflection; methodology; reflexive-active system of artificial heterogeneous intelligent 
agents; hybrid intelligent multiagent system; team of specialists}
   
 
   
\DOI{10.14357/19922264240112}{GUAMVE}

\vspace*{-8pt}

\Ack

\vspace*{-1pt}


     \noindent
     This work was supported by the Russian Science Foundation, project No.\,23-21-00218.


\vspace*{6pt}

  \begin{multicols}{2}

\renewcommand{\bibname}{\protect\rmfamily References}
%\renewcommand{\bibname}{\large\protect\rm References}

{\small\frenchspacing
 {\baselineskip=11.5pt
 \addcontentsline{toc}{section}{References}
 \begin{thebibliography}{99} 
  \bibitem{1-lis-1}
   \Aue{Pospelov, D.\,A.} 2019. Desyat' ``goryachikh tochek'' v~issledovaniyakh po 
iskusstvennomu intellektu [Ten hot topics in AI research]. \textit{Is\-kus\-stven\-nyy in\-tel\-lekt 
i~pri\-nya\-tie re\-she\-niy} [Artificial Intelligence and Decision Making] 4:3--9. doi: 
10.14357/20718594190401. EDN: BAUHFV.
  \bibitem{2-lis-1}
   \Aue{Tarasov, V.\,B.} 2002. \textit{Ot mnogoagentnykh sis\-tem k~in\-tel\-lek\-tu\-al'\-nym 
or\-ga\-ni\-za\-tsi\-yam: fi\-lo\-so\-fiya, psi\-kho\-lo\-giya, in\-for\-ma\-ti\-ka} [From multiagent systems to intelligent 
organizations: Philosophy, psychology, and computer science]. Moscow: Editorial URSS. 348~p.
  \bibitem{3-lis-1}
   \Aue{Kolesnikov, A.\,V., I.\,A.~Kirikov, and S.\,V.~Listopad.} 2014. \textit{Gib\-rid\-nye 
in\-tel\-lek\-tu\-al'\-nye sis\-te\-my s~sa\-mo\-or\-ga\-ni\-za\-tsiey: ko\-or\-di\-na\-tsiya, so\-gla\-so\-van\-nost', spor} [Hybrid 
intelligent systems with self-organization: Coordination, consistency, and dispute]. Moscow: IPI 
RAN. 189~p.
  \bibitem{4-lis-1}
   \Aue{Novikov, A.\,M., and D.\,A.~Novikov.} 2007. \textit{Me\-to\-do\-lo\-giya} [Methodology]. 
Moscow: SINTEG. 668~p.
  \bibitem{5-lis-1}
   \Aue{Listopad, S.\,V.} 2023. Kharakteristiki i~logicheskaya struk\-tu\-ra me\-to\-do\-lo\-gii po\-stro\-eniya 
refleksivno-aktivnykh sis\-tem is\-kus\-stven\-nykh ge\-te\-ro\-gen\-nykh in\-tel\-lek\-tu\-al'\-nykh agen\-tov 
[Characteristics and logical structure of the methodology for constructing reflexive-active systems 
of artificial heterogeneous intelligent agents]. \textit{Sistemy i~Sredstva Informatiki~--- Systems 
and Means of Informatics} 33(4):16--27. doi: 10.14357/08696527230402. EDN: TRTHEI.
  \bibitem{6-lis-1}
   \Aue{Tarasenko, F.\,P.} 2010. \textit{Pri\-klad\-noy sis\-tem\-nyy ana\-liz} 
[Applied systems analysis]. Moscow: KNORUS. 224~p.
  \bibitem{7-lis-1}
   \Aue{Larichev, O.\,I.} 2006. \textit{Ver\-bal'\-nyy ana\-liz re\-she\-niy} [Verbal analysis of decisions]. 
Moscow: Nauka. 181~p.
  \bibitem{8-lis-1}
   \Aue{Akoff, R.\,L.} 1999. \textit{Ackoff's best: His classic writings on management}. New 
York, NY: Wiley. 368~p.
  \bibitem{9-lis-1}
   \Aue{Kolesnikov, A.\,V., S.\,V.~Listopad, S.\,B.~Rumovskaya, and V.\,I.~Danishevskiy.} 
2016. Ne\-for\-mal'\-naya ak\-sio\-ma\-ti\-che\-skaya teo\-riya ro\-le\-vykh vi\-zu\-al'\-nykh mo\-de\-ley [Informal axiomatic 
theory of the role visual models]. \textit{Informatika i~ee Primeneniya~--- Inform. Appl.} 
10(4):114--120. doi: 10.14357/19922264160412. EDN: XGSIVN.
  \bibitem{10-lis-1}
   \Aue{Cherusheva, T.\,V.} 2014. \textit{Pro\-ek\-ti\-ro\-va\-nie pro\-gram\-mno\-go obes\-pe\-che\-niya} 
[Software design]. Penza: PGU. 172~p.
  \bibitem{11-lis-1}
   \Aue{Boissier, O., R.\,H.~Bordini, J.~Hubnerand, and A.~Ricci}. 2020. \textit{Multi-agent 
oriented programming: Programming multi-agent systems using JaCaMo}. Intelligent robotics and 
autonomous agents ser. Cambridge: The MIT Press. 264~p.
  \bibitem{12-lis-1}
   \Aue{Smirnov, S.\,S., and V.\,A.~Smol'yaninova}. 2009. \textit{Vve\-de\-nie v~raz\-ra\-bot\-ku 
mno\-go\-agent\-nykh sis\-tem v~sre\-de Jason. Osno\-vy pro\-gram\-mi\-ro\-va\-niya na yazy\-ke AgentSpeak} 
[Introduction to the development of multiagent systems in the Jason environment. Fundamentals of 
programming in the AgentSpeak language]. Moscow: MIREA. 136~p.
  \bibitem{13-lis-1}
   \Aue{Kaner, С., J.~Falk, and H.\,Q.~Nguyen}. 1999. \textit{Testing computer software}. 
International Thomson Computer Press. 496~p.
  \bibitem{14-lis-1}
   \Aue{Romankova, T.\,L.} 2014. Tes\-ti\-ro\-va\-nie pro\-gram\-mno\-go obes\-pe\-che\-niya [Software testing]. 
Available at: {\sf https://}\linebreak\vspace*{-12pt}

\columnbreak

\noindent
 {\sf elib.gstu.by/bitstream/handle/220612/9860/416.pdf} (accessed January~16, 
2024).
   
  \end{thebibliography}

 }
 }

\end{multicols}

\vspace*{-6pt}

\hfill{\small\textit{Received November 25, 2023}} 

%\vspace*{-18pt}
     
     \Contrl
     
 %    \vspace*{-3pt}
   
   \noindent
   \textbf{Listopad Sergey V.} (b.\ 1984)~--- Candidate of Science (PhD) in technology, senior 
scientist, Federal Research Center ``Computer Science and Control'' of the Russian Academy of 
Sciences, 44-2~Vavilov Str., Moscow 119133, Russian Federation;  
\mbox{ser-list-post@yandex.ru}
   
    
\label{end\stat}

\renewcommand{\bibname}{\protect\rm Литература}    %9+
\def\stat{bosov+stef}

\def\tit{УПРАВЛЕНИЕ ВЫХОДОМ СТОХАСТИЧЕСКОЙ ДИФФЕРЕНЦИАЛЬНОЙ СИСТЕМЫ 
ПО~КВАДРАТИЧНОМУ КРИТЕРИЮ. I.~ОПТИМАЛЬНОЕ РЕШЕНИЕ МЕТОДОМ 
ДИНАМИЧЕСКОГО ПРОГРАММИРОВАНИЯ$^*$}

\def\titkol{Управление выходом стохастической дифференциальной системы 
по~квадратичному критерию. I}
%.~Оптимальное решение методом 
%динамического программирования}

\def\aut{А.\,В.~Босов$^1$, А.\,И.~Стефанович$^2$}

\def\autkol{А.\,В.~Босов, А.\,И.~Стефанович}

\titel{\tit}{\aut}{\autkol}{\titkol}

\index{Босов А.\,В.}
\index{Стефанович А.\,И.}
\index{Bosov A.\,V.}
\index{Stefanovich A.\,I.}




{\renewcommand{\thefootnote}{\fnsymbol{footnote}} \footnotetext[1]
{Работа выполнена при частичной поддержке РФФИ (проект 16-07-00677).}}


\renewcommand{\thefootnote}{\arabic{footnote}}
\footnotetext[1]{Институт проблем информатики Федерального исследовательского центра <<Информатика 
и~управление>> Российской академии наук, \mbox{AVBosov@ipiran.ru}}
\footnotetext[2]{Институт проблем информатики Федерального исследовательского центра <<Информатика 
и~управление>> Российской академии наук, \mbox{AStefanovich@frccsc.ru}}

%\vspace*{8pt}



  
  \Abst{Решается задача оптимального управления для диффузионного процесса 
Ито и~линейного управ\-ля\-емо\-го выхода. Рассматриваемая постановка близка 
к~классической ли\-ней\-но-квад\-ра\-тич\-ной гауссовской задаче управления 
(linear-quadratic Gaussian (LQG) control). Отличия состоят в~том, что состояние описывается нелинейным 
дифференциальным уравнение Ито $dy_t\hm= A_t(y_t) \,dt\hm+ \Sigma_t(y_t)\,dv_t$ 
и~не зависит от управ\-ле\-ния~$u_t$, оптимизации подлежит управ\-ля\-емый 
линейный выход $dz_t\hm= a_t y_t\,dt\hm+ b_t z_t \,dt\hm+ c_t u_t \,dt\hm+ \sigma_t\, 
dw_t$. Дополнительные обобщения внесены в~квад\-ра\-тич\-ный критерий качества 
с~целью воз\-мож\-ности постановки таких задач, как отслеживание выходом 
состояния или управ\-ле\-ни\-ем~--- линейной комбинации состояния и~выхода. Для 
решения используется метод динамического программирования. Функцию 
Беллмана позволяет найти предположение о~ее структуре вида $V_t(y,z)\hm= 
\alpha_t z^2\hm+ \beta_t(y)z \hm+\gamma_t(y)$. Решение дают три 
дифференциальных уравнения для коэффициентов~$\alpha_t$, $\beta_t(y)$ 
и~$\gamma_t(y)$. Эти уравнения со\-став\-ля\-ют оптимальное решение 
рас\-смат\-ри\-ва\-емой задачи.}
  
  \KW{стохастическое дифференциальное уравнение; оптимальное управ\-ле\-ние; 
динамическое программирование; функция Беллмана; уравнение Риккати; 
линейные уравнения параболического типа}

\DOI{10.14357/19922264180314}
  
%\vspace*{4pt}


\vskip 10pt plus 9pt minus 6pt

\thispagestyle{headings}

\begin{multicols}{2}

\label{st\stat}

\section{Введение}

     Ключевые результаты в~области оптимизации стохастических 
динамических систем, со\-став\-ля\-ющие классическую теорию управления, 
получены более~40~лет назад (такова работа~[1] в~отношении задачи 
управ\-ле\-ния ли\-ней\-но-гаус\-сов\-ски\-ми стохастическими сис\-те\-ма\-ми по 
квад\-ра\-тич\-но\-му критерию). К~классической тео\-рии следует относить 
линейные модели стохастических сис\-тем и~квадратичный критерий качества. 
Это исходный базис, на котором основано множество успешно 
исследованных и~решенных задач стохастического управ\-ле\-ния 
и~оптимизации. 

Дальнейшее развитие~--- это новые модели и~критерии, но 
прежде всего это новые методы: от тео\-рии линейных регуляторов, метода 
динамического программирования и~принципа максимума к~адаптивному 
и~минимаксному подходу, импульсному управ\-ле\-нию и~т.\,д. Множество 
инноваций как в~час\-ти моделей, так и~в~час\-ти математического аппарата, 
имевших мес\-то в~по\-сле\-ду\-ющие годы, существенно обогатили тео\-рию 
управ\-ле\-ния. Но и~до настоящего времени линейные модели и~квадратичный 
критерий, несмотря на всю справедливую критику в~отношении их 
аде\-кват\-ности и~гиб\-кости, сохраняют исследовательский интерес и~находят 
современные области приложения.
     
     Не претендуя на сколь\-ко-ни\-будь полное обосно\-ва\-ние последнего 
тезиса, приведем несколько примеров, показавшихся наиболее ин\-те\-рес\-ными. 

Так, в~[2] решается ли\-ней\-но-квад\-ра\-тич\-ная за\-да\-ча в~игровой 
постановке с~запаздыванием. В~близ\-кой по модели работе~[3] задача 
управ\-ле\-ния ставится в~терминах $H_\infty$-ро\-баст\-ности. Точнее \mbox{называть} 
эту тематику $H_2/H_\infty$-управ\-ле\-ни\-ем, и~работ по этой теме очень 
много. Аккуратности ради следует уточнить, что под линейными 
понимаются модели с~мультипликативными по состоянию воз\-му\-ще\-ниями. 

Совсем другой класс моделей, особо популярных в~по\-след\-ние годы, 
составляют скачкообразные процессы. Например, линейные уравнения 
в~сочетании с~пуассоновскими скачками в~[4] используются в~моделях, 
описывающих различные показатели функционирования сетевых протоколов 
передачи данных транспортного уровня. Телекоммуникации представляют 
в~последние годы самый популярный прикладной материал для 
исследований, работ по этой проб\-ле\-ма\-ти\-ке множество, математические 
техники привлекаются самые разные и~самые современные, но и~линейным 
моделям место находится. Еще один любопытный пример исследования 
скачкообразного процесса и~оптимизации на основе квад\-ра\-тич\-но\-го критерия 
можно найти в~[5] применительно к~задаче инвестирования на финансовом 
рынке. Наконец, упомянем еще работу~[6], подводящую итог исследований 
в~отношении классической детерминированной  
ли\-ней\-но-квад\-ра\-тич\-ной задачи с~использованием техники матричных 
неравенств.
     
     В данной работе также эксплуатируются привлекательные свойства 
линейных моделей и~квад\-ра\-тич\-но\-го критерия, причем в~стохастической 
постановке. На\-прав\-ле\-ни\-ем для обобщения \mbox{выбрана} модель динамики 
сис\-те\-мы: основные усилия на\-прав\-ле\-ны на то, чтобы сделать ее нелинейной. 
Кроме того, пред\-став\-лен\-ная постановка может рас\-смат\-ри\-вать\-ся и~как 
обобщение ранее решенной задачи в~дискретном времени~[7, 8] на время 
непрерывное. В~упомянутых работах помимо собственно модельной 
постановки важна еще и~привлекаемая прикладная об\-ласть~--- 
функционирование сложных программных сис\-тем. Результатов, 
ориентированных непосредственно на такие приложения, к~настоящему 
времени пренебрежимо мало, поэтому~[7, 8]~--- это еще и~прикладное 
обоснование рас\-смат\-ри\-ва\-емой далее задачи.
     
     Оптимизируемая динамическая сис\-те\-ма описывается двумя 
уравнениями. Состояние задается нелинейным стохастическим 
дифференциальным уравнением Ито, не содержащим управ\-ля\-емой 
переменной. Возмущение здесь описывается стандартным винеровским 
процессом, накладываются простые условия существования 
и~един\-ст\-вен\-ности решения. Поскольку состояние не управ\-ля\-ет\-ся, то уместно 
его интерпретировать как слож\-ное внешнее возмущение. Вторая 
переменная~--- управ\-ля\-емый выход~--- задается линейным стохастическим 
дифференциальным уравнением. Цель оптимизации выхода формируется 
квадратичным критерием общего вида. Формальная постановка задачи 
приведена в~сле\-ду\-ющем разделе.
     
     Для решения задачи используется метод динамического 
программирования, решается уравнение Беллмана~[9]. Соответственно, 
в~результате получаются аналитические выражения и~для оптимального 
управ\-ле\-ния, и~для значения функционала качества. Технически 
традиционный, стандартный подход к~задаче обременен, пожалуй, 
единственной проблемой~--- поиском верного пред\-став\-ле\-ния структуры 
функции Беллмана. Справиться с~этой проблемой в~большей степени удается 
за счет результата, полученного при решении дискретного по времени 
аналога рассматриваемой постановки~\cite{8-bos}. Конечные соотношения 
для оптимального решения, как и~во всех подобных задачах, включая 
классическую ли\-ней\-но-квад\-ра\-тич\-ную, содержат решения 
определенных дифференциальных уравнений (обыкновенных и~в~частных 
производных). Вывод этих уравнений и~со\-став\-ля\-ет содержание первой час\-ти 
данной работы. Во второй части будет обсуждаться их приближенное 
чис\-лен\-ное решение и~компьютерные эксперименты.
     
     Кратко обозначим основные положения, при\-вле\-ка\-емые далее 
к~решению задачи, следуя в~основном обозначениям 
и~терминологии~\cite{9-bos}, а~именно: будем рассматривать задачу 
оптимального управления в~стохастической динамической сис\-те\-ме по полной 
информации, применяя метод динамического программирования. В~качестве 
целевого функционала, опре\-де\-ля\-юще\-го качество управ\-ле\-ния $U_0^T\hm= \{ 
u_t,\ 0\leq t\leq T\}$, выступает
     \begin{equation}
     J\left(U_0^T\right)={\sf E}\left\{ \int\limits_0^T L_t \left(x_t, u_t\right)\,dt+ 
l\left(x_T\right)\right\}\,.
     \label{e1-bos}
     \end{equation}
Здесь ${\sf E}\{\cdot\}$~--- оператор математического ожидания; $x_t$~--- 
случайный процесс, описываемый стохастическим дифференциальным 
уравнением Ито
     \begin{equation}
     dx_t=m_t\left( x_t, u_t\right) dt+ \sigma_t\left( x_t\right)dW_t\,,\enskip 
x_0=X\,,
     \label{e2-bos}
     \end{equation}
где $W_t$~--- стандартный винеровский процесс подходящей раз\-мер\-ности; 
$X$~--- случайный вектор.

     $U_0^T$ будем выбирать из класса допустимых неупреждающих (по 
отношению к~$W_t$) управлений~\cite{9-bos}. Соответственно, 
относительно функций сноса и~диффузии~$m_t$ и~$\sigma_t$  
в~(\ref{e2-bos}) будем предполагать выполненными ка\-кие-ли\-бо условия 
существования сильного решения для заданного до\-пус\-ти\-мо\-го управ\-ле\-ния. 
Например, для управ\-ле\-ния с~обратной связью $u_t\hm= u_t(x_t)$ будем 
считать, что $m_t(x,u_t(x))$ и~$\sigma_t(x)$ удовлетворяют условию 
линейного рос\-та и~локальному условию Липшица по~$x$ равномерно 
по~$t$ (т.\,е.\ условиям Ито).
     
     Для поиска оптимального управления, минимизирующего $J(U_0^T)$, 
рас\-смат\-ри\-ва\-ет\-ся функция Беллмана
     \begin{equation}
     V_t(x)=\left.\mathop{\mathrm{inf}}\limits_{U_t^T} {\sf E} \left\{ \int\limits_t^T 
L_t \left( x_t, u_t\right)\,dt+l\left( x_T\right) \right\vert \mathcal{F}_t^x\right\}\,,
     \label{e3-bos}
     \end{equation}
где $\mathcal{F}_t^x$~--- $\sigma$-ал\-геб\-ра, по\-рож\-ден\-ная~$x_\tau$, 
$0\hm\leq \tau\hm\leq t$, ${\sf E}\{\cdot\vert \mathcal{F}\}$~--- оператор условного 
математического ожидания относительно~$\mathcal{F}$. Соответственно, 
в~качестве достаточного условия оп\-ти\-маль\-ности воспользуемся уравнением 
динамического программирования
\begin{multline}
\fr{\partial V_t(x)}{\partial t} +\fr{1}{2}\sum\limits^n_{i,j=1} \sigma^2_{t_{ij}}
\fr{\partial^2 V_t(x)}{\partial x_i \partial x_j}+{}\\
{}+\min\limits_u\left[  
\sum\limits^n_{i=1} m_{t_i} \fr{\partial V_t(x)}{\partial x_i} + L_t(x,u)\right] 
=0\,,\\
V_T(x)=l(x)\,,
\label{e4-bos}
\end{multline}
где $m_{t_i}$~--- $i$-й элемент век\-тор-функ\-ции~$m_t(x,u)$; 
$\sigma^2_{t_{ij}} \hm= \sum\nolimits^m_{k=1} 
\sigma_{t_{ik}}\sigma_{t_{ki}}$, $\sigma_{t_{ij}}$~--- $i$-й по строке, $j$-й 
по столб\-цу элемент мат\-рич\-ной функции~$\sigma_t(x)$; $n$ и~$m$~--- 
размерности~$x_t$ и~$W_t$ соответственно.

     Традиционно в~рамках применения метода динамического 
программирования будем предполагать, что функции~$L_t$, $l$, $m_t$ 
и~$\sigma_t$ обеспечивают существование хотя бы одного решения 
уравнения~(\ref{e4-bos}), а~следовательно, и~оптимального 
управления~$u_t^*$, $0\hm\leq t\hm\leq T$, до\-став\-ля\-юще\-го минимум 
целевому функционалу~(\ref{e1-bos}). Задача оптимизации далее получается 
путем указания конкретных выражений для~$L_t$, $l$, $m_t$ и~$\sigma_t$.

\section{Постановка задачи управления выходом}

     Рассматриваемые далее случайные функции будут предполагаться 
скалярными. Такое упрощение позволит разгрузить выкладки и~итоговые 
выражения от не самых существенных деталей.
     
     Рассмотрим стохастическую дифференциальную сис\-те\-му, со\-сто\-яние 
которой представляет диффузи\-он\-ный процесс~$y_t$, описываемый 
нелинейным стохастическим дифференциальным уравнением Ито
     \begin{equation}
     dy_t=A_t\left( y_t\right) dt +\Sigma_t \left( y_t\right) dv_t\,,\enskip 
y_0=Y\,,
     \label{e5-bos}
     \end{equation}
где $v_t$~--- стандартный (одномерный) винеровский процесс; $Y$~--- 
случайная величина с~конечным вторым моментом; функции~$A_t$ 
и~$\Sigma_t$ удовлетворяют условиям Ито:
\begin{equation*}
\left\vert A_t(y)\right\vert +\left\vert \Sigma_t(y)\right\vert \leq C(1+\vert y\vert )\ 
\mbox{для\ всех } 0\leq t\leq T\,;
\end{equation*}

\vspace*{-12pt}

\noindent
\begin{multline*}
\hspace*{-2.10051pt}\left\vert A_t\left(y_1\right) -A_t \left( y_2\right) \right\vert +\left\vert 
\Sigma_t\left( y_1\right) -\Sigma_t \left(y_2\right)\right\vert \leq
C\left\vert y_1-y_2\right\vert\\
 \mbox{для\ всех\ } 0\leq t\leq T\ \mbox{и } 
y_1,y_2\in \mathbb{R}^1\,,
\end{multline*}
обеспечивающим существование единственного сильного (потраекторного) 
решения уравнения.
     
     Будем считать, что~$y_t$ описывает состояние некоторой 
динамической системы. Соответственно, поведение этой сис\-те\-мы опишем 
выходом, линейно связанным с~со\-сто\-янием:
     \begin{equation}
     dz_t=a_t y_t \,dt+ b_t z_t \,dt+ c_t u_t \,dt+\sigma_t \,dw_t\,,\enskip
     z_0=Z\,.
     \label{e6-bos}
     \end{equation}
Здесь $w_t$~--- не зависящий от~$v_t$, $Y$ и~$Z$ стандартный (одномерный) 
винеровский процесс; $Z$~--- случайная величина с~конечным вторым 
моментом; $u_t$~--- допустимое неупреждающее управ\-ле\-ние, качество 
которого определяется целевым функционалом следующего вида:
\begin{multline}
\!\hspace*{-3.98538pt}J\left( U_0^T\right) ={\sf E}\left\{ \int\limits_0^T \!\left( S_t\left( s_ty_t-g_t z_t -h_t 
u_t\right)^2 +G_t z_t^2+{}\right.\right.\\
\left.\left.{}+ H_t u_t^2
\vphantom{S_t\left( s_ty_t-g_t z_t -h_t 
u_t\right)^2}
\right) dt+S_T\left( s_T y_T -g_T 
z_T\right)^2+G_T z_T^2
\vphantom{\int\limits_0^T}\right\}\,,
\label{e7-bos}
\end{multline}
где $S_t$, $G_t$ и~$H_t$~--- неотрицательные функции\linebreak
$0\hm\leq t\hm\leq T$. 
Такой критерий отражает физический смысл задачи распределения ресурсов 
со\-глас\-но аналогичной~(\ref{e5-bos})--(\ref{e7-bos}) задаче для дис\-крет\-но\-го 
времени, рас\-смот\-рен\-ной в~\cite{7-bos}. В~част\-ности,  
функци\-онал~(\ref{e7-bos}) поз\-во\-ля\-ет ставить задачи отслеживания
 выходом 
со\-сто\-яния сис\-те\-мы, используя сла\-га\-емое $(y_t\hm- z_t)^2$, или 
управлением~--- линейной комбинации со\-сто\-яния и~выхода, сла\-га\-емое типа\linebreak 
$(y_t\hm+ z_t\hm- u_t)^2$. Поскольку задача формулируется 
в~предположении наличия пол\-ной информации о~со\-сто\-янии~$y_t$ 
и~выходе~$z_t$ (соответствующую $\sigma$-ал\-геб\-ру 
обозначим~$\mathcal{F}_t^{y,z}$), то допустимое управ\-ле\-ние ищется 
в~классе~$\mathcal{F}_t^{y,z}$-из\-ме\-ри\-мых неупреждающих функций 
(и,~как будет показано далее, оказывается управ\-ле\-ни\-ем с~обратной связью).

     Функции~$a_t$, $b_t$, $c_t$ и~$\sigma_t$ будем предполагать 
ограниченными: $\vert a_t\vert \hm+ \vert b_t\vert \hm+\vert c_t\vert \hm+ \vert 
\sigma_t \vert \hm\leq C$ для всех $0\hm\leq t\hm\leq T$, процесс  
управления~--- допустимым не\-упреж\-да\-ющим~\cite{9-bos}, обеспечивая, 
таким образом, существование сильного решения урав\-не\-ния~(\ref{e6-bos}) 
для любого допустимого управ\-ления.
     
     Задачу составляет поиск~$u_t^*$~--- допустимого управ\-ле\-ния, 
доставляющего минимум квад\-ра\-тич\-но\-му функционалу~$J(U_0^T)$.
      
     Поставленная задача очевидным образом формулируется в~терминах 
введенных выше в~(\ref{e1-bos})--(\ref{e3-bos}) обозначений, а~именно: 
     требуется обозначить
     \begin{gather*}
      x_t=\begin{pmatrix}
     y_t\\ z_t\end{pmatrix};\quad  m_t(x_t, u_t)=\begin{pmatrix}
     A_t(y_t)\\ a_t y_t +b_t z_t +c_t u_t\end{pmatrix};\\
     \sigma_t(x_t)= \begin{pmatrix}
     \Sigma_t(y_t)& 0\\
     0& \sigma_t\end{pmatrix};\quad W_t=\begin{pmatrix}
     v_t \\ w_t\end{pmatrix}
     %     \label{e8-bos}
     \end{gather*}
для записи уравнения со\-сто\-яния типа~(\ref{e2-bos}) и
\begin{align*}
L_t(x,u)&= L_t(y,z,u) ={}\\
&\hspace*{3mm}{}=S_t\left( s_t y-g_t z -h_t u\right)^2 +G_t z^2 +H_t  u^2\,;\\
l(x)&= l(y,z) =S_T \left( S_T y-g_T z\right)^2 +G_T z^2
%\label{e9-bos}
\end{align*}
для записи целевого функционала в~виде~(\ref{e1-bos}).

     Функция Беллмана~(\ref{e3-bos}) принимает вид 
     $V_t(x)\hm= V_t(y,z)$. Для записи со\-от\-вет\-ст\-ву\-юще\-го~(\ref{e4-bos}) 
уравнения Беллмана для~$V_t(y,z)$ заметим, что
     $$
     \left( \sigma^2_{t_{ij}}\right)_{i,j=1,2}= \begin{pmatrix}
     \Sigma_t^2(y) & 0\\
     0 & \sigma_t^2\end{pmatrix}\,.
     $$
     
     С~учетом перечисленных обозначений урав\-не\-ние динамического 
программирования~(\ref{e4-bos}) принимает вид:
     \begin{multline}
     \fr{\partial V_t(y,z)}{\partial t} +\fr{1}{2}\left( \Sigma_t^2(y) \fr{\partial^2 
V_t(y,z)} {\partial y^2}+\sigma_t^2\fr{\partial^2 V_t(y,z)} {\partial 
z^2}\right)+{}\\
    {}+\min\limits_u\! \left[ A_t(y) \fr{\partial V_t(y,z)}{\partial y}+\left( a_t 
y+b_t z+c_t u\right) \fr{\partial V_t(y,z)}{\partial z} +{}\right.\hspace*{-3pt}\\
\left.{}+ S_t\left( s_t y-g_t z-h_t 
u\right)^2+G_t z^2+H_t u^2
     \vphantom{\fr{\partial V_t(y,z)}{\partial y}}\right] =0\,,\\
     V_T(y,z)=S_T\left( s_T y-g_T z\right)^2+G_T z^2\,.
     \label{e10-bos}
     \end{multline}
     Это и~есть то самое уравнение, которое требуется решить: 
существование решения данного урав\-не\-ния суть достаточное условие 
оптимальности; оптимальное управ\-ле\-ние при этом~--- точ\-ка минимума 
со\-от\-вет\-ст\-ву\-юще\-го сла\-га\-емого.
     
\section{Динамическое программирование и~оптимальное 
управление}

     В рассматриваемой постановке линейность\linebreak выхода и~квадратичность 
критерия дают те же преимущества, что и~в~классической  
ли\-ней\-но-квад\-ра\-тич\-ной задаче управ\-ле\-ния~\cite{1-bos}, а~именно: 
позволяют сразу определить вид оптимального управ\-ле\-ния и~фактические 
условия его существования. Действительно, со\-хра\-няя в~(\ref{e10-bos}) под 
знаком $\min\nolimits_u$ только члены, зависящие от~$u$, получаем
     \begin{multline*}
     \fr{\partial V_t(y,z)}{\partial t} +\fr{1}{2}\left( \Sigma_t^2(y) \fr{\partial^2 
V_t(y,z)} {\partial y^2}+\sigma_t^2\fr{\partial^2 V_t(y,z)} {\partial 
z^2}\right)+{}\\
     {}+A_t(y)\fr{\partial V_t(y,z)}{\partial y}+\left( a_t y+b_t z\right) 
\fr{\partial V_t(y,z)}{\partial z}+{}\\
{}+S_t\left( s_t y-g_t z\right)^2 +G_t z^2+{}
\end{multline*}

\noindent
\begin{multline*}
     {}+\min\limits_u \left[ \left( c_t \fr{\partial V_t(y,z)}{\partial z}-2S_t \left( 
s_t y-g_t z\right) h_t\right)u +{}\right.\\
\left.{}+\left( S_t h_t^2+H_t\right) u^2
\vphantom{\fr{\partial V_t(y,z)}{\partial z}}
\right]=0\,,
     %\label{e11-bos}
     \end{multline*}
откуда в~предположении $S_t h_t^2\hm+ H_t\hm>0$ следует, что существует 
оптимальное управ\-ле\-ние, которое определяется равенством
\begin{multline}
u_t^* = u_t^*(y,z)=-\fr{1}{2}\left( S_t h_t^2 +H_t\right)^{-1} \left( c_t 
\fr{\partial V_t(y,z)}{\partial z}-{}\right.\\
\left.{}-2S_t\left( s_t y-g_t z\right) h_t
\vphantom{\fr{\partial V_t(y,z)}{\partial z}}
\right)
\label{e12-bos}
\end{multline}
и доставляет минимум соответствующему сла\-га\-емо\-му в~урав\-не\-нии Беллмана, 
равный
$-\left( S_t h_t^2\hm+\right.$\linebreak
$\left.{}+H_t\right)^{-1} \left( c_t 
{\partial V_t(y,z)}/{\partial 
z}\hm-2S_t\left( s_t y \hm-g_t z\right) h_t \right)^2/4.
$ 
     
     Отметим, что, как и~в~классической ли\-ней\-но-квад\-ра\-тич\-ной 
задаче, управ\-ле\-ние из класса до\-пус\-ти\-мых не\-упреж\-да\-ющих получилось 
управ\-ле\-ни\-ем с~обратной связью.
     
     Таким образом, функция Беллмана описывается сле\-ду\-ющим 
дифференциальным уравнением:
     \begin{multline}
     \fr{\partial V_t(y,z)}{\partial t} +\fr{1}{2}\left( \Sigma_t^2(y) \fr{\partial^2 
V_t(y,z)} {\partial y^2}+\sigma_t^2\fr{\partial^2 V_t(y,z)} {\partial 
z^2}\right)+{}\\
     {}+ A_t(y) \fr{\partial V_t(y,z)}{\partial y}+\left( a_t y+b_t z\right) 
\fr{\partial V_t(y,z)}{\partial z}+{}\\
{}+ S_t \left( s_t y- g_t z\right)^2 +G_t z^2-
 \fr{1}{4}\left( S_t h_t^2+H_t\right)^{-1}\times{}\\
 {}\times \left( c_t \fr{\partial V_t(y,z)} 
{\partial z}-2S_t\left( s_t y -g_t z\right) h_t \right)^2=0\,.
     \label{e13-bos}
     \end{multline}
     
     Возводя в~квадрат по\-след\-нее сла\-га\-емое в~(\ref{e13-bos}), перепишем 
его в~виде:
     \begin{multline}
     \fr{\partial V_t(y,z)}{\partial t} +\fr{1}{2}\left( \Sigma_t^2(y) \fr{\partial^2 
V_t(y,z)} {\partial y^2}+\sigma_t^2\fr{\partial^2 V_t(y,z)} {\partial 
z^2}\!\right)+{}\\
{}+A_t(y) \fr{\partial V_t(y,z)}{\partial y}
+ \left( 
\vphantom{\left( S_t h_t^2 +H_t\right)^{-1}}
a_t y+b_t z+{}\right.\\
\left.{}+\left( S_t h_t^2 +H_t\right)^{-1}
 c_t S_t \left( s_t y-g_t z\right) h_t
\right) 
     \fr{\partial V_t(y,z)}{\partial z}+{}\\
     {}+\left( S_t-\left( S_t h_t^2 +H_t\right)^{-1} S_t^2 h_t^2\right)\left( s_t y -
g_t z\right)^2+{}\\
     \!\!{}+
     G_t z^2 -\fr{1}{4}\left( S_t h_t^2+H_t\right)^{-1}\! c_t^2
     \left(\fr{\partial V_t(y,z)}{\partial z}\right)^{\!2}=0\,.\!\!
     \label{e14-bos}
     \end{multline}
     
     Рассматривая полученное уравнение, заметим, что его решение может 
быть пред\-став\-ле\-но в~виде:
   \begin{equation}
     V_t(y,z)= \alpha_t z^2+\beta_t(y) z +\gamma_t(y)\,,
     \label{e15-bos}
     \end{equation}
т.\,е.\ будем искать решение при дополнительном предположении 
о~квад\-ра\-тич\-ности функции Белл\-ма\-на по переменной~$z$, и~сведем, таким 
образом, поиск оптимального решения к~уравнениям относительно функций 
$\alpha_t$, $\beta_t(y)$ и~$\gamma_t(y)$. Отметим сразу, что явный вид 
функции~$\gamma_t(y)$ для реализации оптимального управ\-ле\-ния не 
требуется, однако далее будет предложен вариант вы\-чис\-ле\-ния и~этой 
функции, что пред\-став\-ля\-ет\-ся небесполезным, поскольку позволит выполнять 
расчет минимума целевого функционала. Источником для 
предложения~(\ref{e15-bos}) является уже упоминавшаяся аналогичная 
задача для случая дис\-крет\-но\-го времени~\cite{7-bos, 8-bos}. В~той задаче 
выражение для функции Беллмана получается формально без 
дополнительных усилий. При этом форма~(\ref{e15-bos}) обнаруживается 
как свойство оптимального решения. В~рассматриваемом случае 
непрерывного времени~(\ref{e15-bos}) постулируется, а~пра\-виль\-ность 
постулата под\-тверж\-да\-ет\-ся далее ре\-зуль\-ти\-ру\-ющи\-ми уравнениями 
для~$\alpha_t$, $\beta_t(y)$ и~$\gamma_t(y)$ Кроме того, данное 
предположение пред\-став\-ля\-ет\-ся вы\-те\-ка\-ющим из линейной структуры задачи 
в~отношении переменной~$z$, в~част\-ности, тем фактом, что такой вид 
функции Беллмана обеспечивает линейность оптимального 
управ\-ле\-ния~(\ref{e12-bos}) по~$z$.

     Граничное условие при выбранном предположении~(\ref{e15-bos}) 
принимает вид:

\noindent
     \begin{multline*}
     V_T(y,z)= S_T\left( s_T y- g_T z\right)^2+G_T z^2 ={}\\[-0.5pt]
     {}=\alpha_T z^2 
+\beta_T(y) z +\gamma_T(y)\,,
    \end{multline*}
т.\,е.

\noindent
\begin{align*}
\alpha_T&= S_T g_T^2 +G_T\,;\\[-0.5pt]
\beta_T(y)&=-2S_T s_T g_T y\,;\\[-0.5pt]
\gamma_T(y)&=S_T s_T^2 y^2\,.
%\label{e16-bos}
\end{align*}
          При этом само оптимальное управ\-ле\-ние, определенное 
выражением~(\ref{e12-bos}), оказывается управ\-ле\-ни\-ем с~обратной связью 
по~$y_t$ и~$z_t$:

\noindent
     \begin{multline}
     u_t^*=u_t^*(y,z) ={}\\[-0.5pt]
     {}=
     -\fr{1}{2}\left( S_t h_t^2 +H_t\right)^{-1}
     \left( c_t \left( 2\alpha_t z +\beta_t(y)\right) +{}\right.\\[-0.5pt]
    \left. {}+2S_t\left( s_t y-g_t z\right) 
h_t\right)\,.
     \label{e17-bos}
     \end{multline}
          Подставляем $V_t(y,z)\hm= \alpha_t z^2 \hm+ \beta_t(y) 
z\hm+\gamma_t(y)$ в~(\ref{e14-bos}):

\noindent
     \begin{multline*}
     \fr{\partial \alpha_t}{\partial t}\, z^2 +
     \fr{\partial \beta_t(y)}{\partial t}\,z +
     \fr{\partial \gamma_t(y)}{\partial t}+{}\\[-0.5pt]
     {}+\fr{1}{2}\left( \Sigma_t^2(y) \left( 
\fr{\partial^2\beta_t(y)}{\partial y^2}\,z +\fr{\partial^2 \gamma_t(y)}{\partial 
y^2}\right) +2\sigma_t^2\alpha_t\right)+{}\\[-0.5pt]
 {}+A_t(y)\left(\fr{\partial \beta_t(y)}{\partial y}\,z + \fr{\partial 
\gamma_t(y)}{\partial y}\right) +{}\\[-0.5pt]
\hspace*{-0.22987pt}{}+\left( a_t y+b_t z+\left( S_t h_t^2 +H_t\right)^{-1} c_t S_t \left( s_t y-
g_t z\right) h_t\right)\times{}
\end{multline*}

\noindent
\begin{multline*}
         {}\times \left( 2\alpha_t z+\beta_t(y)\right)+{}\\
     {}+\left( S_t-\left( S_t h_t^2 +H_t\right)^{-1} S_t^2 h_t^2\right)\left( s_t y-
g_t z\right)^2+{}\\
     {}+ G_t z^2 -\fr{1}{4}\left( S_t h_t^2 +H_t\right)^{-1} c_t^2 \left( 
2\alpha_t z+\beta_t(y)\right)^2=0\,.
     \end{multline*}
          Далее выделяем слагаемые при~$z^2$, $z$ и~$z^0$
          
          \noindent
     \begin{multline*}
     \fr{\partial \alpha_t}{\partial t}\, z^2 +\fr{\partial \beta_t(y)}{\partial t}\,z +
     \fr{\partial \gamma_t(y)}{\partial 
t}+\fr{1}{2}\,\Sigma_t^2(y)\fr{\partial^2\beta_t(y)}{\partial y^2}\,z+ {}\\
{}+
\fr{1}{2}\,\Sigma_t^2(y)\fr{\partial^2\gamma_t(y)}{\partial 
y^2}+\sigma_t^2\alpha_t+A_t(y)\fr{\partial \beta_t(y)}{\partial y}\,z +{}\\
{}+A_t(y) \fr{\partial 
\gamma_t(y)}{\partial y}+{}\\
{}+ 2\alpha_t \left( b_t -\left( S_t h_t^2+H_t\right)^{-1} c_t 
S_t h_t g_t \right)z^2+{}\\
     {}+
     \left( 2\alpha_t\left( \alpha_t+\left( S_t h_t^2+H_t\right)^{-1} c_t S_t h_t 
s_t\right)y +{}\right.\\
\left.{}+\beta_t(y) \left( b_t-\left( S_t h_t^2+H_t\right)^{-1} c_t S_t h_t 
g_t\right) \right) z+{}\\
     {}+\beta_t(y)\left( a_t +\left( S_t h_t^2+H_t\right)^{-1} c_t S_t h_t s_t\right) 
y+{}\\
{}+ \left( S_t -\left( S_t h_t^2+H_t\right)^{-1} S_t^2 h_t^2\right) g_t^2 z^2-{}\\
     {}- 2\left( S_t -\left( S_t h_t^2+H_t\right)^{-1} S_t^2 h_t^2\right) s_t g_t yz 
+{}\\
{}+
     \left( S_t-\left( S_t h_t^2+H_t\right)^{-1} S_t^2 h_t^2\right) s_t^2 y^2+{}\\
     {}+G_t z^2 -\left( S_t h_t^2 +H_t\right)^{-1} c_t^2 \alpha_t^2 z^2 -{}\\
     {}-\left( 
S_t h_t^2+H_t\right)^{-1} c_t^2 \alpha_t \beta_t(y) z-{}\\
{}-
\fr{1}{4}\left( S_t h_t^2+H_t\right)^{-1}  c_t^2 \beta_t^2(y)=0\,,
     \end{multline*}
группируем их и~получаем сле\-ду\-ющие уравнения:
\begin{itemize}
\item  для~$\alpha_t$:

\noindent
\begin{multline}
\fr{\partial\alpha_t}{\partial t}+2\alpha_t\left( b_t-\left( S_t h_t^2+H_t\right)^{-1} c_t 
S_t h_t g_t\right)+{}\\
{}+ \left( S_t- \left( S_t h_t^2+H_t\right)^{-1} S_t^2 h_t^2\right) 
g_t^2+G_t-{}\\
\hspace*{-8mm}{}-\left( S_t h_t^2+H_t\right)^{-1} c_t^2 \alpha_t^2 =0\,,\enskip \alpha_T=S_T 
g_t^2+G_T\,;\!\!
\label{e18-bos}
\end{multline}
\item для $\beta_t$:

\noindent
\begin{multline}
\fr{\partial\beta_t(y)}{\partial 
t}+\fr{1}{2}\,\Sigma_t^2(y)\fr{\partial^2\beta_t(y)}{\partial y^2} 
+A_t(y)\fr{\partial \beta_t(y)}{\partial y}+{}\\
{}+ 2\alpha_t\left( a_t +\left( S_t h_t^2+H_t\right)^{-1} c_t S_t h_t s_t\right) y+{}\\
{}+
\beta_t(y)\left( b_t -\left( S_t h_t^2 +H_t\right)^{-1} c_t S_t h_t g_t\right)-{}\\
{}-2\left( S_t-\left( S_t h_t^2+H_t\right)^{-1} S_t^2 h_t^2\right) s_t g_t y-{}
\\
{}-
\left( S_t h_t^2+H_t\right)^{-1} c_t^2 \alpha_t \beta_t(y)=0\,,\\
\beta_T(y)=-2S_T s_T g_T y\,;
\label{e19-bos}
\end{multline}
\item  для $\gamma_t$:
\begin{multline}
\hspace*{-0.8pt}\fr{\partial \gamma_t(y)}{\partial t}+\fr{1}{2}\,\Sigma_t^2(y)
\fr{\partial^2 \gamma_t(y)}{\partial y^2} +\sigma_t^2 \alpha_t +A_t(y)
\fr{\partial \gamma_t(y)}{\partial y}+{}\\
{}+ \beta_t(y)\left( a_t +\left( S_t h_t^2+H_t\right)^{-1} c_t S_t h_t s_t\right) y+{}\\
{}+
\left( S_t-\left( S_t h_t^2+H_t\right)^{-1} S_t^2 h_t^2\right)  s_t^2 y^2-{}\\
{}-\fr{1}{4}\left( S_t h_t^2+H_t\right)^{-1} c_t^2 \beta_t^2(y) =0\,,\\
\gamma_T(y)=S_T s_T^2 y^2\,.
\label{e20-bos}
\end{multline}
\end{itemize}
     
     Уравнение~(\ref{e18-bos}), легко заметить, является уравнением 
Риккати, которое в~силу сформулированного выше условия   
имеет единственное неотрицательное решение для всех $0\hm\leq t\hm\leq T$. 
Этот факт требует дополнительного комментария. Для получения 
уравнения~(\ref{e18-bos}) рас\-смот\-рим исходную задачу при дополнительных 
условиях $a_t\hm=0$ и~$s_t\hm=0$ для всех $0\hm\leq t\hm\leq T$. Нетрудно 
видеть, что эти условия рассматриваемую по\-ста\-нов\-ку сводят фактически 
к~классической ли\-ней\-но-квад\-ра\-тич\-ной задаче. Имеющуюся 
в~рассматриваемой формулировке чуть более общую форму целевой 
функции (принципиального значения это обобщение, конечно, не имеет) 
сведем к~классической еще одним предположением: $S_t\hm=0$ для всех 
$0\hm\leq t\hm\leq T$. Теперь уравнение для~$\alpha_t$ принимает хорошо 
известный вид:
     \begin{equation}
     \fr{\partial \alpha_t}{\partial t}+2\alpha_t b_t +G_t- H_t^{-1} c_t^2 
\alpha_t^2=0\,,\enskip \alpha_T=G_T\,.
     \label{e21-bos}
     \end{equation}

     В таком случае, как известно~\cite{10-bos}, существует единственное 
оптимальное управление~--- линейное с~обратной связью по выходу~$z_t$, 
с~коэффициентом усиления, опи\-сы\-ва\-емым уравнением  
Риккати~(\ref{e21-bos}). Именно этот результат дают  
уравнения~(\ref{e18-bos})--(\ref{e20-bos}) и~описываемая ими функция 
Беллмана~(\ref{e15-bos}), так как из $a_t\hm=0$ и~$s_t\hm=0$ немедленно 
следует, что $\beta_t(y)\hm=0$, откуда, в~свою очередь, с~учетом 
не\-за\-ви\-си\-мости решения от~$y_t$ следует, что $\gamma_t(y)\hm=\gamma_t$, 
т.\,е.\ не зависит от~$y$ и~задается уравнением: 
     $$
     \fr{\partial \gamma_t(y)}{\partial t} +\sigma^2_t \alpha_t=0\,,\enskip 
\gamma_T=0\,.
     $$ 
     Оптимальное управ\-ле\-ние при этом 
     $$
     u_t^*= -H_t^{-1} c_t \alpha_t z_t\,,
     $$
      т.\,е.\ все полностью совпадает с~известным классическим решением.
     
     С уравнениями~(\ref{e19-bos}) и~(\ref{e20-bos}) ситуация, естественно, 
обстоит сложнее. Это линейные уравнения второго порядка параболического 
типа, поскольку\linebreak
 $\Sigma_t^2(y)\hm>0$. Фактически отсутствуют 
конструктивные условия, гарантирующие существование их\linebreak
 решений 
(требовать, чтобы все фигурирующие в~уравнениях коэффициенты были 
представлены аналитическими функциями на всем пространстве значений, 
вряд ли целесообразно), поэтому далее будем предполагать, что данные 
уравнения имеют на рас\-смат\-ри\-ва\-емом интервале $0\hm\leq t\hm\leq T$ хотя 
бы одно ограниченное решение и~именно эти условия будем рас\-смат\-ри\-вать 
как достаточные условия существования оптимального решения 
рассматриваемой задачи.
     
     Таким образом, доказана следующая тео\-рема.
     
     \smallskip
     
     \noindent
     \textbf{Теорема.}\ \textit{Пусть для диффузионного 
процесса}~(\ref{e5-bos}) \textit{выполнены условия Ито, для 
     процесса}~(\ref{e6-bos})~--- \textit{ограничены коэффициенты, 
уравнения}~(\ref{e18-bos})--(\ref{e20-bos}) \textit{имеют ограниченные 
решения для $0\hm\leq t\hm\leq T$. Тогда минимум  
функционалу}~(\ref{e7-bos}) \textit{доставляет оптимальное 
управ\-ле\-ние}~(\ref{e17-bos}), \textit{где} $y\hm= y_t$; $z\hm=z_t$.
     
\section{Заключение}

     Рассмотренная задача оптимизации в~целом близка и~по модели, и~по 
критерию к~классической ли\-ней\-но-квад\-ра\-тич\-ной постановке. 
Принципиальным отличием является нелинейная модель для описания 
со\-сто\-яния динамической сис\-те\-мы, в~которой отсутствует управ\-ля\-ющее 
воздействие.\linebreak
 Такую модель наряду с~традиционной интер\-пре\-тацией  
<<со\-сто\-яние--вы\-ход>> мож\-но понимать как\linebreak модель неконтролируемого 
слож\-но\-го внешнего воздействия. Небольшое дополнительное отличие дает 
предложенная форма квад\-ра\-тич\-но\-го критерия, поз\-во\-ля\-ющая, в~част\-ности, 
ставить такие задачи, как отслеживание выходом или управ\-ле\-ни\-ем со\-сто\-яния 
сис\-те\-мы или ее выхода.
     
     Поскольку обсуждать возможности точного решения уравнений, 
определяющих оптимальное управ\-ле\-ние, не имеет смыс\-ла, наиболее 
актуальной далее является задача их приближенного чис\-лен\-но\-го решения 
и~анализа воз\-мож\-ности практической реализации. Этому посвящена вторая 
часть данной работы, пла\-ни\-ру\-емая к~выходу в~ближайшее время.

{\small\frenchspacing
 {%\baselineskip=10.8pt
 \addcontentsline{toc}{section}{References}
 \begin{thebibliography}{99}
\bibitem{1-bos}
\Au{Athans M.} Editorial on the LQG problem~// IEEE~T. Automat. Contr., 1971. Vol.~16. 
No.\,6. P.~528--552. doi: 10.1109/TAC.1971.1099845.
\bibitem{2-bos}
\Au{Wu Z.} Forward-backward stochastic differential equations, linear quadratic stochastic 
optimal control and nonzero sum differential games~// J.~Syst. Sci. Complex., 2005. Vol.~18. 
No.\,2. P.~179--192.
\bibitem{3-bos}
\Au{Chen B.\,S., Zhang~W.} Stochastic H2/H1 control with state-dependent noise~// IEEE 
T.~Automat. Contr., 2004. Vol.~49. No.\,1. P.~45--56. doi: 10.1109/TAC.2003.821400.
\bibitem{4-bos}
\Au{Bohacek S.} A~stochastic model of TCP and fair video transmission~// IEEE 
INFOCOM, 2003. Vol.~2. P.~1134--1144. doi: 10.1109/INFCOM.2003.1208950.
\bibitem{5-bos}
\Au{Домбровский В.\,В., Объедко~Т.\,Ю.} Управление с~прогнозированием системами 
с~марковскими скачками при ограничениях и~применение к~оптимизации 
инвестиционного портфеля~// Автомат. телемех., 2011. №\,5. С.~96--112. doi: 
10.1134/S0005117911050079.
\bibitem{6-bos}
\Au{Баландин Д.\,В., Коган~М.\,М.} Оптимальное линейно-квад\-ра\-тич\-ное управление: от 
матричных уравнений к~линейным матричным неравенствам~// Автомат. телемех., 2011. 
№\,11. С.~60--69. doi: 10.1134/ S0005117911110038.
\bibitem{7-bos}
\Au{Босов А.\,В.} Обобщенная задача распределения ресурсов программной системы~// 
Информатика и~её применения, 2014. Т.~8. Вып.~2. С.~39--47. doi: 
10.14357/19922264140204.
\bibitem{8-bos}
\Au{Босов А.\,В.} Управление линейным выходом дискретной стохастической системы по 
квадратичному критерию~// Изв. РАН. Теория и~системы управления, 2016. №\,3.  
С.~19--35. doi: 10.1134/S1064230716030060.
\bibitem{9-bos}
\Au{Флеминг У., Ришел~Р.} Оптимальное управление детерминированными 
и~стохастическими системами~/ Пер. с~англ.~--- М.: Мир, 1978. 316~с. 
(\Au{Fleming~W.\,H., Rishel~R.\,W.} Deterministic and stochastic optimal control.~--- New 
York, NY, USA: Springer-Verlag, 1975. 222~p.)
\bibitem{10-bos}
\Au{Девис М.\,Х.\,А.} Линейное оценивание и~стохастическое управление~/ Пер. с~англ.~--- 
М.: Наука, 1984. 206~с. (\Au{Davis~M.\,H.\,A.} Linear estimation and stochastic control.~--- 
London: Chapman and Hall, 1977. 224~p.)

 \end{thebibliography}

 }
 }

\end{multicols}

\vspace*{-6pt}

\hfill{\small\textit{Поступила в~редакцию 30.03.18}}

\vspace*{4pt}

%\newpage

%\vspace*{-24pt}

\hrule

\vspace*{2pt}

\hrule

\vspace*{-2pt}


\def\tit{STOCHASTIC DIFFERENTIAL SYSTEM OUTPUT CONTROL 
BY~THE~QUADRATIC CRITERION.~I.~DYNAMIC\\ PROGRAMMING 
OPTIMAL SOLUTION}


\def\titkol{Stochastic differential system output control 
by~the~quadratic criterion. I.~Dynamic programming 
optimal solution}

\def\aut{A.\,V.~Bosov and~A.\,I.~Stefanovich}

\def\autkol{A.\,V.~Bosov and~A.\,I.~Stefanovich}

\titel{\tit}{\aut}{\autkol}{\titkol}

\vspace*{-11pt}


\noindent
Institute of Informatics Problems, Federal Research Center ``Computer Science 
and Control'' of the Russian Academy of Sciences, 44-2~Vavilov Str., Moscow 
119333, Russian Federation


\def\leftfootline{\small{\textbf{\thepage}
\hfill INFORMATIKA I EE PRIMENENIYA~--- INFORMATICS AND
APPLICATIONS\ \ \ 2018\ \ \ volume~12\ \ \ issue\ 3}
}%
 \def\rightfootline{\small{INFORMATIKA I EE PRIMENENIYA~---
INFORMATICS AND APPLICATIONS\ \ \ 2018\ \ \ volume~12\ \ \ issue\ 3
\hfill \textbf{\thepage}}}

\vspace*{3pt}



\Abste{The problem of optimal control for the Ito diffusion 
process and a~controlled linear output is solved. The considered 
statement is close to the classical linear-quadratic Gaussian 
control  (LQG control) problem. Differences consist in the fact 
that the state is described by the nonlinear differential Ito equation  $dy_y = A_t(y_t) 
\,dt+\Sigma_t(y_t)\,dv_t$ and does not depend on the control~$u_t$, 
optimization subject is controlled linear output 
 $dz_t=a_ty_t\,dt +b_tz_t\,dt +c_t u_t\,dt +\sigma_t \,dw_t$. 
Additional generalizations are included in the quadratic 
quality criterion for the purpose of statement such problems 
as state tracking by output or a linear combination of state 
and output tracking by control. The method of dynamic programming 
is used for the solution. 
The assumption about Bellman function in the form  $V_t(y,z)= \alpha_t 
z^2+\beta_t(y) z+\gamma_t(y)$ allows one to find it. 
Three differential equations for the coefficients $\alpha_t$,  $\beta_t(y)$,
and $\gamma_t(y)$ give the solution. 
These equations constitute the optimal solution of the problem under consideration.}

\KWE{stochastic differential equation; optimal control; dynamic programming; 
Bellman function; Riccati equation; linear differential equations of parabolic type}


\DOI{10.14357/19922264180314}

\vspace*{-12pt}

\Ack
\noindent
This work was partially supported by the Russian Science Foundation (grant  
16-07-00677).



%\vspace*{6pt}

  \begin{multicols}{2}

\renewcommand{\bibname}{\protect\rmfamily References}
%\renewcommand{\bibname}{\large\protect\rm References}

{\small\frenchspacing
 {%\baselineskip=10.8pt
 \addcontentsline{toc}{section}{References}
 \begin{thebibliography}{99}
\bibitem{1-bos-1}
\Aue{Athans, M.} 1971. Editorial on the LQG problem. \textit{IEEE~T. 
Automat. Contr.} 16(6):528--552. doi: 10.1109/ TAC.1971.1099845.
\bibitem{2-bos-1}
\Aue{Wu, Z.} 2005. Forward-backward stochastic differential equations, linear 
quadratic stochastic optimal control and\linebreak\vspace*{-12pt}

\columnbreak

\noindent
 nonzero sum differential games. 
\textit{J.~Syst. Sci. Complex.} 18(2):179--192.
\bibitem{3-bos-1}
\Aue{Chen, B.\,S. and W.~Zhang.} 2004. Stochastic H2/H1 control with  
state-dependent noise. \textit{IEEE~T. Automat. Contr.} 49(1):45--56.
doi: 10.1109/TAC.2003.821400.
\bibitem{4-bos-1}
\Aue{Bohacek, S.} 2003. A~stochastic model of TCP and fair video 
transmission. \textit{IEEE INFOCOM}. 2:1134--1144.
doi: 10.1109/INFCOM.2003.1208950.
\bibitem{5-bos-1}
\Aue{Dombrovskii, V.\,V., and T.\,Yu.~Ob''edko.} 2011. Predictive control of 
systems with Markovian jumps under constraints and its application to the 
investment portfolio optimization. \textit{Automat. Rem. Contr.}  
72(5):989--1003.
\bibitem{6-bos-1}
\Aue{Balandin, D.\,V., and M.\,M.~Kogan.} 2011. Optimal linear-quadratic 
control: From matrix equations to linear matrix inequalities. \textit{Automat. 
Rem. Contr.} 72(11):2276--2284.
\bibitem{7-bos-1}
\Aue{Bosov, A.\,V.} 2014. Obobshchennaya zadacha raspredeleniya resursov 
programmnoy sistemy [The generalized problem of software system resources 
distribution]. \textit{Informatika i~ee Primeneniya~--- Inform. Appl.}  
8(2):39--47. doi: 
10.14357/19922264140204.
\bibitem{8-bos-1}
\Aue{Bosov, A.\,V.} 2016. Discrete stochastic system linear output control 
with respect to a quadratic criterion. \textit{J.~Comput. Syst. Sc. 
Int.} 55(3):349--364.
\bibitem{9-bos-1}
\Aue{Fleming, W.\,H., and R.\,W.~Rishel.} 1975. \textit{Deterministic and 
stochastic optimal control.} New York, NY: Springer-Verlag. 222~p.
\bibitem{10-bos-1}
\Aue{Davis, M.\,H.\,A.} 1977. \textit{Linear estimation and stochastic 
control.} London: Chapman and Hall. 224~p.
\end{thebibliography}

 }
 }

\end{multicols}

\vspace*{-6pt}

\hfill{\small\textit{Received March 30, 2018}}

%\pagebreak

%\vspace*{-18pt}
     
     \Contr
     
       \noindent
       \textbf{Bosov Alexey V.} (b.\ 1969)~--- Doctor of Science in technology, 
principal scientist, Institute of Informatics Problems, Federal Research 
Center ``Computer Science and Control'' of the Russian Academy of Sciences, 
44-2~Vavilov Str., Moscow 119333, Russian Federation; 
\mbox{AVBosov@ipiran.ru}
       
       \vspace*{3pt}
       
       \noindent
       \textbf{Stefanovich Alexey I.} (b.\ 1983)~--- principal specialist, 
Institute of Informatics Problems, Federal Research Center ``Computer Science 
and Control'' of the Russian Academy of Sciences, 44-2~Vavilov Str., Moscow 
119333, Russian Federation; \mbox{AStefanovich@frccsc.ru}
\label{end\stat}

\renewcommand{\bibname}{\protect\rm Литература}       

            %10+
\def\stat{malashenko}

\def\tit{ПОСЛЕДОВАТЕЛЬНЫЙ АНАЛИЗ И~МЕТРИЧЕСКИЕ ОЦЕНКИ ПРЕДЕЛЬНЫХ
РАСПРЕДЕЛЕНИЙ МЕЖУЗЛОВЫХ ПОТОКОВ В~МНОГОПОЛЬЗОВАТЕЛЬСКОЙ СЕТИ}

\def\titkol{Последовательный анализ и~метрические оценки предельных
распределений межузловых потоков в %~многопользовательской 
сети}

\def\aut{Ю.\,Е. Малашенко$^1$}

\def\autkol{Ю.\,Е. Малашенко}

\titel{\tit}{\aut}{\autkol}{\titkol}

\index{Малашенко Ю.\,Е.}
\index{Malashenko Yu.\,E.}


%{\renewcommand{\thefootnote}{\fnsymbol{footnote}} \footnotetext[1]
%{Исследование выполнено при финансовой поддержке Российского научного фонда (проект 
%<<Информатика>> ФИЦ ИУ РАН, Москва).}}


\renewcommand{\thefootnote}{\arabic{footnote}}
\footnotetext[1]{Федеральный исследовательский центр <<Информатика и~управление>> Российской академии 
\mbox{mala-yur@yandex.ru}}


%\vspace*{-6pt}



\Abst{Для оценки функциональных возможностей
многопользовательской сети связи аналилизируется множество векторов межузловых потоков при предельных распределениях ресурсов
сети. В~рамках многопродуктовой модели про\-пуск\-ные спо\-соб\-ности ребер рас\-смат\-ри\-ва\-ют\-ся 
как компоненты вектора ресурсов различных
типов, которые требуются для передачи потоков различных видов.
При проведении вычислительных экспериментов на каждой итерации вычисляются нормы векторов совместно допустимых межузловых
потоков, при передаче которых полностью используется пропускная спо\-соб\-ность всех ребер сети. Полученные последовательности
метрических оценок позволяют анализировать особенности множества до\-сти\-жи\-мости и~эф\-фек\-тив\-ность использования ресурсов сети при
уравнительном распределении про\-пуск\-ной спо\-соб\-ности между корреспондентами.}

\KW{многопродуктовая потоковая сетевая
модель; множество достижимых межузловых потоков; предельные
распределения пропускной способности}

\DOI{10.14357/19922264220306} 
  
%\vspace*{-3pt}


\vskip 10pt plus 9pt minus 6pt

\thispagestyle{headings}

\begin{multicols}{2}

\label{st\stat}

\section{Введение}

Данная работа продолжает исследования функциональных характеристик
сетевых сис\-тем связи~[1]. В~настоящее время математические модели
передачи многопродуктового потока применяются для поиска
распределений потоков и~ресурсов в~многопользовательских
телекоммуникационных\linebreak сетях~[2--4]. Разрабатываются методы анализа
с~учетом вектора требований всех \textit{равноправных} 
и~невзаимозаменяемых корреспондентов~[5]. С~позиций\linebreak методологии
исследования операций изучаются справедливые распределения потоков
и~ресурсов~[6].

Соответствующие \textit{недискриминирующие} правила управления
потоками являются решениями задач на максмин и/или получаются 
в~результате использования процедур последовательной
лексикографически упорядоченной оптимизации~[7].

В~настоящей работе пути соединения корреспондентов прокладываются
через со\-от\-вет\-ст\-ву\-ющие минимальные разрезы. Указанный метод\linebreak \mbox{можно}
рассматривать как возможный вариант решения задачи о~построении
SPLIT-марш\-ру\-тов~[8,~9]. В~рамках вычислительных экспериментов\linebreak на
многопродуктовой модели анализируются распределения межузловых
потоков  и~пропускной способ\-ности сети.  Для оценки функциональных
возможностей многопользовательской сети используется вектор
совместно допустимых межузловых потоков. Под ресурсом, выделяемым
некоторой паре узлов-кор\-рес\-пон\-ден\-тов,  понимается суммарное
значение тре\-бу\-емых пропускных способностей на всех ребрах,
расположенных на всех маршрутах при прохождении межузлового\linebreak потока
данного вида.  Сумма соответствующих реберных потоков трактуется
как полная нагрузка на сеть, возникающая при передаче заданного
межузлового потока. Рас\-смат\-ри\-ва\-ют\-ся распределения пропускной
способности и~межузловых потоков при предельной загрузке сети.
При проведении вычислительных экспериментов на каждой  итерации
вычисляется норма  вектора совместно допустимых межузловых
потоков.   Для оценки величины требуемых ресурсов при соединении
корреспондентов по различным путям для каж\-дой пары узлов
определяется максимальный однопродуктовый поток. Марш\-ру\-ты передачи
всех допустимых межузловых потоков  проходят по ребрам
соответствующих минимальных разрезов. Вычислительные эксперименты
проводились  для получения последовательности  мет\-ри\-че\-ских оценок
векторов межузловых потоков, принадлежащих множеству до\-сти\-жи\-мости
многопользовательской сети.

\section{Математическая модель}

В качестве математической модели для описания
многопользовательской сетевой системы используется следующая
формальная запись условий и~ограничений, которые должны
выполняться при одновременной передаче потоков различных видов
между всеми парами улов-корреспондентов:

Сеть $G(\mathbf{d})$ задается множествами $\langle V,
R,U,P\rangle$:
\begin{itemize}
\item  узлов (вершин) сети 
$$
V=\left \{v_{1}, v_{2},\dots,v_{n},\dots,v_{N}\right\};
$$
\item  неориентированных ребер 
$$
R=\left\{r_{1}, r_{2}, \dots, r_{k}, \dots,
r_{E}\right\}.
$$
\end{itemize}

Ребро $r_{k}$ \textit{соединяет} концевые вершины~$v_{n_k}$ и~$v_{j_k}$. 
Реб\-ру~$r_{k}$ ставятся в~соответствие две
ориентированные дуги $\{u_{k},u_{k+E}\}$ из множества
ориентированных дуг $U\hm=\{u_{1}, u_{2}, \dots, u_{k}, \dots,
u_{2E}\}$. Дуги $\{u_{k}, u_{k+E}\}$ определяют прямое и~обратное
на\-прав\-ле\-ние передачи потока по реб\-ру~$r_{k}$ между концевыми
вершинами $\{v_{n_k}, v_{j_k}\}$.

В многопользовательской сети~$G(\mathbf{d})$ рассматривается
$M\hm=N(N\hm-1)$ независимых, невзаимозаменяемых и~равноправных потоков
различных видов, которые передаются между уз\-ла\-ми-кор\-рес\-пон\-ден\-та\-ми
из множества 
$$
P=\left\{p_{1}, p_{2}, \dots, p_{M}\right\}.
$$

По определению, каждой паре уз\-лов-кор\-рес\-пон\-ден\-тов~$p_{m}$
соответствуют:
\begin{itemize}
\item вершина-ис\-точ\-ник с~номером~$s_{m}$, через которую входной поток
$m$-го вида~$z_{m}$ поступает в~сеть;
\item  вершина-при\-ем\-ник с~номером~$t_{m}$, из которой поток $m$-го
вида~$z_{m}$ покидает сеть.
\end{itemize}

В множестве~$P$ выделяется подмножество $P(R^{+})$ пар
уз\-лов-кор\-рес\-пон\-ден\-тов, расположенных в~концевых вершинах
ребра~$r_{k}$, $k\hm=\overline{1,E}$. Вводятся сле\-ду\-ющие обозначения:
пусть ребро~$r_{k}$  с~номером~$k$ соединяет вершины с~номерами~$n$ и~$j$ такими, что $n\hm< j$. Для со\-от\-вет\-ст\-ву\-ющей пары
уз\-лов-кор\-рес\-пон\-ден\-тов~$p_{k}$, расположенных в~узлах $\{v_{n},
v_{j}\}$, узел~$v_{n}$ считается источником, а узел~$v_{j}$~---
приемником потока $z_{k}$ $k$-го вида, который передается из узла
c номером~$n$ в~узел с~номером~$j$ для пары~$p_{k}$ с~номером~$k$.
Для пары $p^{}_{k+E} \Longleftrightarrow \{v_{j},v_{n}\}$ узел~$v_{j}$ 
считается источником~$s_{k+E}$, а~узел $v_m$~--- приемником~$t_{k+E}$ для пары с~номером~$p_{k+E}$. Формируется
$R^+\hm=\{1,2,3,\dots,E,E+1,\dots,2E\}$~--- список номеров смежных
пар.

Пары $p_{k}$ из подмножества~$P(R^{+})$ называются
\textit{смежными} уз\-ла\-ми-кор\-рес\-пон\-ден\-та\-ми. Все остальные
\textit{несмежные} пары уз\-лов-кор\-рес\-пон\-ден\-тов относятся к~множеству~$P(R^{-})$:
\begin{equation*}
P=P(R^{+})\cup P(R^{-});\quad
P(R^{+}) \cap P(R^{-}) = \emptyset.
\end{equation*}

Введем обозначения:
\begin{description}
\item[\,]
$z_{m}$~--- величина \textit{межузлового} потока $m$-го вида,
который поступает в~сеть из узла с~номером~$s_{m }$ и~покидает из
узла с~номером~$t_{m}$;
\item[\,]
$S(v_{n})$~--- множество номеров исходящих дуг, по которым поток
покидает узел~$v_{n}$;
\item[\,]
$T(v_{n})$~--- множество номеров входящих дуг, по которым поток
поступает в~узел~$v_{n}$.
\end{description}

Во всех узлах $v_{n}\in V$, $n\hm=\overline{1,N}$, для всех видов
потоков должны выполняться условия сохранения потоков:
\begin{multline}
\label{eq1} 
\sum\limits_{i\in S(v_n )} x_{mi}-\sum\limits_{i\in T(v_n )} x_{mi}
={}\\
{}=\begin{cases}
z_m, & \mbox{если } v=v^{}_{S_m}; \\
-z_m,&\mbox{если } v=v_{t_m}; \\
0&\mbox{в остальных случаях}, \\
\end{cases}
\end{multline}
$n=\overline{1,N}$, $m\hm=\overline{1,M}$, $x_{mi}\hm\ge 0$,
$z_{m}\hm\ge0$.

Величина {z}$_{m}$ равна входному потоку $m$-го вида, который
пропускается от источника к~приемнику пары $p_{m}$ при
распределении потоков $x_{mi}$ по дугам сети.

Каждому ребру $r_{k}\hm\in R$ приписывается неотрицательное число~$d_{k}$, 
определяющее суммарный предельно допустимый поток,
который можно передать по реб\-ру~$r_{k}$ в~обоих на\-прав\-ле\-ни\-ях. 
В~исходной сети компоненты вектора про\-пуск\-ных способностей
$\mathbf{d}\hm=(d_{1}, d_{2},\dots, d_{k}, \dots, d_{E})$~--- наперед
заданные положительные числа $d_{k}
\hm> 0$. Вектором $\mathbf{d}$ определяются сле\-ду\-ющие ограничения на сумму
дуговых потоков всех видов, пе\-ре\-да\-ва\-емых по реб\-ру~$r_{k}$:
\begin{multline}
\sum\limits_{m=1}^M (x_{mk}+x_{m(k+E)}) \le d_k,\\
 x_{mk}\ge 0\,,\enskip
 x_{m(k+E)}\ge 0\,, \enskip k=\overline {1,E}\,.
 \label{eq2} 
\end{multline}
В рамках данной модели пропускная спо\-соб\-ность ребер сети~--- вектор~$\mathbf{d}$~--- трактуется как <<\textit{ресурсное ограничение}>>,
а~сумма дуговых
 потоков рас\-смат\-ри\-ва\-ет\-ся как показатель использования
<<\textit{ресурсов}>> сети при передаче межузловых потоков
различных видов.

Для всех $z_{m}$ и~$x_{mi}$, удовлетворяющих
условиям~\eqref{eq1} и~\eqref{eq2}, вычисляются суммарные потоки:
\begin{equation}
 y_{m }=\sum\limits_{i=1}^{2E} {x}_{mi},\quad
m=\overline{1,M}\,.
\label{eq3}
\end{equation}

Суммарный реберный поток~$y_{m}$ характеризует
<<\textit{нагрузку}>> на сеть при передаче межузлового потока
величины $z_{m}$ из уз\-ла-ис\-точ\-ни\-ка~$s_{m}$ в~узел-при\-ем\-ник~$t_{m}$. 
Величина~$y_{m}$ показывает, какой суммарный
\textit{ресурс}~-- пропускная спо\-соб\-ность сети~-- требуется для
передачи межузлового потока~$z_{m}$, а~отношение
$w_{m}\hm={y_m}/{z_m}$,  $m\hm=\overline{1,M},$
показывает, какие \textit{ресурсы} необходимы для передачи
единичного потока $m$-го вида между узлами~$s_{m}$ и~$t_{m}$.

Ограничения~\eqref{eq1}--\eqref{eq3} задают подмножество
допустимых значений компонент вектора межузловых потоков
$\mathbf{z}\hm=\left(z_{1}, z_{2},\dots,z_{m},\dots,z_{M}\right)$:
\begin{equation*}
{Z}(\mathbf{d})=\left\{\mathbf{z} \ge 0 \mid
(\mathbf{z},\mathbf{x},\mathbf{y}) \ \mbox{удовлетворяют~\eqref{eq1}--\eqref{eq3}}
\right\}\!,
\!\!
%\label{eq4} 
\end{equation*}
а все допустимые распределения ресурсов принадлежат подмножеству
\begin{equation*}
{Y}(\mathbf{d})=\left\{\mathbf{y} \ge 0 \mid
(\mathbf{z},\mathbf{x},\mathbf{y}) \ \mbox{удовлетворяют~\eqref{eq1}--\eqref{eq3}}\right\}\!.
%\!\!\!\label{eq5}
\end{equation*}


\section{Метрические оценки предельных распределений}

Для оценки функциональных возможностей сис\-те\-мы рассматриваются
допустимые распределения межузловых потоков при предельных
загрузках ребер сети.

В рамках данного модельного описания монопольным режимом
называется способ управления, при котором все ресурсы сети
используются для передачи потока одной выделенной пары
уз\-лов-кор\-рес\-пон\-ден\-тов $p_{a}\hm\in P(R^-)$, а для всех
остальных потоки полагаются равными нулю.

Предельно допустимый поток, который можно передать между
фиксированной парой уз\-лов-кор\-рес\-пон\-ден\-тов $p_{a}$ в~монопольном
режиме, является решением стандартной, в~данном случае
однопродуктовой, задачи о~максимальном потоке.

\smallskip

\noindent
\textbf{Задача 1.} Найти
$z_a^0\hm=\max\limits_{\langle z,x\rangle \in Z(d)} z_a
$
при условии $z_{i}=0$, $i\hm=\overline{1,M}$, $i\hm\ne a$.

При решении задачи~1 для пары $p_{a}$ вы\-чис\-ля\-ют\-ся: межузловой
поток~$z_a^0$; дуговые потоки $\{x^{0}_{ak};x^{0}_{a(k+E)}\}$,
$k\hm=\overline{1,E}$; суммарное значение реберного
потока~$y_{a}^{0}\hm=\sum\nolimits_{i=1}^{2E} {x}_{ai}^{0}$.

Поток величины $z_a^0$ является \textit{максимальным потоком},
пе\-ре\-да\-ва\-емым в~\textit{монопольном режиме} для пары
уз\-лов-кор\-рес\-пон\-ден\-тов~$p_{a}$, $p_{a}\hm\in P(R^-)$, в~сети~$G(d)$.

Задача~1 решается последовательно для всех $p_{m}\in P(R^-)$,
вы\-чис\-ля\-ют\-ся значения $z_{m}^{0}(t)$.

При проведении вычислительных экспериментов использовалась
итерационная процедура для оценки функциональных возможностей
сис\-те\-мы при передаче межузловых потоков по нескольким маршрутам.
На предварительном этапе шага~$t$ в~сети~$G(t)$ при заданных
значениях пропускной спо\-соб\-ности ребер~$d_k(t)$ для каждой \mbox{пары}
уз\-лов-кор\-рес\-пон\-ден\-тов $p_m\hm\in P(R^-)$ определяется максимально
допустимый однопродуктовый поток~$z^0_m(t)$, со\-от\-вет\-ст\-ву\-ющие
дуговые потоки $(x_{mk}^0(t),x_{m(k+E)}^0(t))$, $p_m\hm\in P(R^-)$, и~коэффициенты нормировки
$\xi_m^0(t)\hm={1}/{z_m^0(t)}$ для всех  $p_m\hm \in P(R^-)$,
таких что $z^0_m(t)\hm>0$, $y_m^0(t)\hm>0$.
Коэффициенты~$\xi_m^0(t)$ используются для поиска текущих
совместно допустимых квот на передачу потоков одновременно между
всеми парами $p_m\in P(R^-)$.

\smallskip

\noindent
\textbf{Задача 2.} Найти $\alpha^*(t)=\max\limits_\alpha \alpha$
при условиях
$$
\alpha\!\!\sum\limits_{m\in R^-}\! \xi_m^0\left(x_{mk}^0(t)+x_{m(k+E)}^0(t)\right)\le d_k(t),\enskip
k=\overline{1,E}\,.
$$

На основании $\alpha^*(t)$ вычисляются совместно допустимые
дуговые потоки:
\begin{multline*}
x_{mk}^*(t)=\alpha^*(t)\xi^0_m(t)x^0_{mk}(t),\\
x^*_{m(k+E)}(t)=\alpha^*(t)\xi^0_m(t)x^0_{m(k+E)}(t),
\\
m=\overline{1,M}\,,\enskip k=\overline{1,E}\,,
\end{multline*}
и остаточная пропускная способность ребер в~сети $G(t+1)$:
\begin{multline*}
d_k(t+1)=d_k(t)-\sum_{m\in R^-} (x_{mk}^*(t)+x_{m(k+E)}(t)),\\
k=\overline{1,E}\,,\enskip p_m\in P(R^-).
\end{multline*}
Формируется вектор допустимых межузловых потоков:
\begin{align*}
z_k^+(t)&=d_k(t+1),\enskip p_k\in P(R^+),\enskip k=\overline{1,E}\,;
\\
z_m^-(t)&=\sum\limits_{\tau=1}^t\alpha^*(\tau)\xi_m^0(\tau) z_m^0(\tau), \enskip p_m\in P(R^-).
\end{align*}

По построению, на шаге~$t$ при передаче вектора межузлового потока
$\mathbf{z}(t)=\{\mathbf{z}^+(t), \mathbf{z}^-(t)\}$ достигается
предельная загрузка, и~пропускная способность всех ребер  сети
используется полностью.

Точка с~координатами $\mathbf{z}(t)$ принадлежит множеству~$Z(d)$.

Расстояние точки от начала координат определяется как норма
соответствующего вектора:
\begin{align*}
\rho^+(t)&=\|\mathbf{z}^+(t)\|=
\left[\,\sum\limits_{k=1}(\mathbf{z}^+(t))^2\right]^{1/2};
\\
\rho^-(t)&=\|\mathbf{z}^-(t)\|= \left[\sum\limits_{p_m\in P(R^-)}(\mathbf{z}_m^-(t))^2\right]^{1/2}.
\end{align*}

Если при выполнении шага $(t+1)$ окажется, что $z_m^0(t+1)=0$ для
всех $p_m\in P(R^-)$, то про\-изойдет останов и~сформируются
массивы финальных данных:
\begin{align*}
z_m^-(T)&=\sum\limits_{\tau=1}^t \alpha^*(\tau)\xi_m^0(\tau) z_m^0(\tau),\enskip 
p_m\in P(R^-),\\
z_k^+(T)&=d_k(t+1),\enskip p_k\in P(R^+),\enskip k=\overline{1,E}\,.
\end{align*}

Вышеописанная вычислительная процедура далее обозначается как
MFPL-про\-це\-ду\-ра (от англ.\ \textit{max-flow-peak-load}).

При проведении второй серии вычислительных экспериментов
MFPL-про\-це\-ду\-ра использовалась для оценки функциональных
характеристик сис\-те\-мы при \textit{уравнительном} поэтапном
распределении пропускной способности между всеми
па\-ра\-ми-кор\-рес\-пон\-ден\-тами.

При реализации MFPL-процедуры выполнение каждого шага разбивается
на несколько этапов. На предварительном этапе шага~$t$ 
в~сети~$G(t)$ при заданных значениях пропускной способности ребер~$d_k(t)$ 
для каждой пары уз\-лов-кор\-рес\-пон\-ден\-тов $p_m\hm\in P(R^-)$
определяется максимально допустимый однопродуктовый
поток~$z_m^0(t)$, соответствующие дуговые потоки
$\left(x_{mk}^0(t),x_{m(k+E)}^0(t)\right)$, $p_m\hm\in P(R^-)$, и~суммарная
реберная нагрузка
$$
y_m^0(t)=\sum\limits_{k=1}^E (x_{mk}^0(t),x_{m(k+E)}^0(t)),\enskip p_m\in P(R^-).
$$

Для каждой пары $p_m\hm\in P(R^-)$ вычисляются коэффициенты
нормировки
$\theta_m^0(t)\hm={1}/{y_m^0(t)}$ для всех  
$p_m\hm\in P(R^-)$, таких что  $z^0_m(t)\hm>0$,
$y_m^0(t)\hm>0$.
Коэффициенты~$\theta_m^0(t)$ используются для поиска совместно
допустимых дуговых потоков для всех $p_m\hm\in P(R^-)$.

\smallskip

\noindent
\textbf{Задача 3.} Найти $\beta^*(t)=\max\nolimits_\beta \beta$ при
условиях
$$
\beta\!\!\!\!\sum\limits_{p_m\in P(R^-)}\!\!
\theta_m^0(x_{mk}^0(t)+x_{m(k+E)}^0(t))\le d_k(t),\enskip
k=\overline{1,E}\,.
$$

 С помощью $\beta^*(t)$ (решения задачи~3) вычисляются текущие допустимые значения дуговых потоков:
\begin{multline*}
x_{mk}^*(t)=\beta^*(t)\theta^0_m(t)x^0_{mk}(t),\\
x^*_{m(k+E)}(t)=\beta^*(t)\theta^0_m(t)x^0_{m(k+E)}(t), \enskip
k=\overline{1,E},
\end{multline*}
и реберных нагрузок при одновременной передаче межузловых потоков:

\noindent
\begin{multline*}
y_m^*(t)=\sum\limits_{i=1}^E
\left[x_{mi}^*(t)+x^*_{m(i+E)}(t)\right]={}\\
{}= \fr{\beta^*(t)}{y_m^0(t)} \sum\limits_{i=1}^E
\left[x_{mi}^0(t)+x^0_{m(i+E)}(t)\right]=\beta^*(t), \\
 p_m\in P(R^-).
\end{multline*}
Таким образом на каждом шаге определенная часть имеющегося ресурса
(пропускной спо\-соб\-ности) делится строго по\-ров\-ну меж\-ду всеми
корреспондентами $p_m\in P(R^-)$, для которых существует путь
передачи в~$G(t)$.

Формируется вектор допустимых межузловых потоков:
\begin{gather*}
\hspace*{-30mm}z_k^{++}(t)=d_k(t+1)={}\hspace*{10mm}\\
{}=d_k(t)-\!\!\! \sum\limits_{p_m\in P(R^-)}\!\!\!
\left(x_{mk}^*(t)+x_{m(k+E)}(t)\right),\\
\hspace*{35mm}k=\overline{1,E}, \enskip
p_k\in P(R^+);\\
z_m^{(=)}(t)\overset{\Delta}{=}\sum\limits_{\tau=1}^t\beta^*(\tau)
\theta_m^0(\tau) z_m^0(\tau), \enskip p_m\in P(R^-).
\end{gather*}

\noindent
Определяются расстояния:
\begin{align*}
\rho^{++}(t)&=\|\mathbf{z}^{++}(t)\|\overset{\Delta}{=}
\left[\sum\limits_{k=1}^E\left(d_k(t+1)\right)^2\right]^{1/2};\\
\rho^{(=)}(t)&=\|\mathbf{z}^{=}(t)\|= \left[\sum\limits_{p_m\in
P(R^-)}\left(z_m^{(=)}(t)\right)^2\right]^{1/2}.
\end{align*}

Если на предварительном этапе на шаге $(t+1)$ окажется, что в~сети~$G(t+1)$ для всех $p_m\hm\in P(R^-)$ все значения
$z_m^0(t+1)\hm=0$, то произойдет останов и~сформируются финальные
массивы:
\begin{align*}
z_k^{(++)}(T)&=d_k(t+1), \enskip
p_k\in P(R^+), \enskip k=\overline{1,E};
\\
z_m^{(=)}(t)&=\sum\limits_{\tau=1}^{t+1}\beta^*(\tau)
\theta_m^0(\tau) z_m^0(\tau), \enskip p_m\in P(R^-).
\end{align*}



\section{Вычислительный эксперимент}

Результаты вычислительных экспериментов, описанные ниже, служат
продолжением исследований, начатых в~[1]. Вычислительные
эксперименты проводились на моделях сетевых сис\-тем, пред\-став\-лен\-ных
на рис.~1 и~2. В~каждой сети~69~узлов. Пропускные спо\-соб\-но\-сти
ребер~-- значения $d_k$~-- выбирались случайным образом из отрезка
$[900,999]$ и~совпадали для ребер, при\-сут\-ст\-ву\-ющих в~обеих сетях.
В~кольцевой сети пропускная спо\-соб\-ность каждого из добавленных
ребер равнялась~900.

\begin{figure*} %fig1
\vspace*{1pt}
\begin{minipage}[t]{80mm}
  \begin{center}  
    \mbox{%
\epsfxsize=69.408mm
\epsfbox{mal-1.eps}
}

\end{center}
\vspace*{-6pt}
\Caption{Базовая сеть}
\end{minipage}
%\end{figure*}
\hfill
%\begin{figure*} %fig2
\vspace*{1pt}
\begin{minipage}[t]{80mm}
  \begin{center}  
    \mbox{%
\epsfxsize=69.408mm
\epsfbox{mal-2.eps}
}

\end{center}
\vspace*{-6pt}
\Caption{Кольцевая сеть}
\end{minipage}
\end{figure*}

\begin{table*}[b]\small %tabl1
\vspace*{-12pt}
\begin{center}

%\renewcommand{\arraystretch}{1.1}
\Caption{Базовая сеть}
\vspace*{2ex}

\begin{tabular}{|c||c|c|c||c|c|c|} 
\hline
&&&&&&\\[-9pt]
$t$  & $\rho^{-}(t)$ & $\rho^{+}(t)$ & $d^{+}(t+1)$ &
$\rho^{=}(t)$ & $\rho^{++}(t)$&  $d^{++}(t+1)$ \\ 
\hline
\hphantom{99}0  & \hphantom{99}0   & 8048&  68256&  \hphantom{9}0   &  8048&   68256\\
1  & \hphantom{9}63  & 4182&  26544&  \hphantom{9}95  &  3881&   24476\\
$\cdots$  & $\cdots$   & $\cdots$   &  $\cdots$    &  $\cdots$   &  $\cdots$   &   $\cdots$\\
11 & \hphantom{9}70  & 3975&  21469&  \hphantom{9}101\hphantom{9} &  3707&   20155\\
$\cdots$& $\cdots$   & $\cdots$   &  $\cdots$    & $\cdots$   &  $\cdots$   &  $\cdots$\\
22 & \hphantom{9}83  & 3861&  19623&  \hphantom{9}122\hphantom{9} &  3586&   18260\\
$\cdots$ & $\cdots$  & $\cdots$   &  $\cdots$   &  $\cdots$   &  $\cdots$  &   $\cdots$\\
33 & \hphantom{9}103\hphantom{9} & 3778&  18827&  \hphantom{9}139\hphantom{9} &  3522&   17601\\
$\cdots$ &$\cdots$  &$\cdots$  & $\cdots$  & $\cdots$   &  $\cdots$  &  $\cdots$\\
44 & \hphantom{9}\bf 190\hphantom{9} & \bf3553&  \bf17503&  \hphantom{9}\bf203\hphantom{9} &  \bf3285&   \bf16201\\
45 & \hphantom{9}\bf1452\hphantom{99}& \bf2166&  \hphantom{9}\bf7069 &  \hphantom{9}\bf1376\hphantom{99}&  \bf2020&   \hphantom{9}\bf6584\\
46 & \hphantom{9}\bf1498\hphantom{99}& \bf2158&  \hphantom{9}\bf6707 &  \hphantom{9}\bf1388\hphantom{99}&  \bf2017&   \hphantom{9}\bf6483\\
$\cdots$ & $\cdots$   & $\cdots$   &  $\cdots$    & $\cdots$   &  $\cdots$   &  $\cdots$\\
52 & \hphantom{9}1535\hphantom{99}& 2155&  \hphantom{9}6413 & \hphantom{9}1442\hphantom{99} &  2011&   \hphantom{9}6059\\
\hline
\end{tabular}
\end{center}
 %\end{table*}
% \begin{table*}\small %tabl2
\begin{center}
\Caption{Кольцевая сеть}
\vspace*{2ex}


\begin{tabular}{|c||c|c|c||c|c|c|} 
\hline
&&&&&&\\[-9pt]
$t$  & $\rho^{-}(t)$ & $\rho^{+}(t)$ & $d^{+}(t+1)$ &
$\rho^{=}(t)$ & $\rho^{++}(t)$&  $d^{++}(t+1)$ \\
 \hline
\hphantom{9}0  &\hphantom{99}0    & 8440  & 75456   &\hphantom{9}0      &8440   &75456\\
\hphantom{9}1  &\hphantom{9}68   & 5317  & 43038   &92     &5045   &40716 \\ 
$\cdots$ &$\cdots$    & $\cdots$     & $\cdots$   &$\cdots$      &$\cdots$      &$\cdots$      \\
11 &\hphantom{9}95   & 3608  & 20459   &124    &3397   &19080  \\
$\cdots$ &$\cdots$   & $\cdots$    & $\cdots$      &$\cdots$     &$\cdots$     &$\cdots$   \\
22 &\hphantom{9}101\hphantom{9}  & 3540  & 19530   &130    &3350   &18338 \\
$\cdots$ &$\cdots$  & $\cdots$   &$\cdots$      &$\cdots$     &$\cdots$   &$\cdots$    \\
33 &\hphantom{9}135\hphantom{9}  & 3346  & 17561   &154    &3220   &17003 \\
$\cdots$  &$\cdots$   & $\cdots$    & $\cdots$      &$\cdots$     &$\cdots$    &$\cdots$    \\
44 &\hphantom{9}234\hphantom{9}  & 3094  & 14881   &269    &2918   &13848 \\
$\cdots$ &$\cdots$   & $\cdots$    &$\cdots$      &$\cdots$     &$\cdots$     &$\cdots$    \\
50 &\hphantom{9}\bf 413\hphantom{9}  & \bf2770  & \bf12901   &\bf329    &\bf2792   &\bf13079 \\
51 &\hphantom{9}\bf1040\hphantom{99} & \bf2299  & \hphantom{9}\bf8801    &\bf334    &\bf2784   &\bf13034 \\
52 &\hphantom{9}\bf1062\hphantom{99} & \bf2297  & \hphantom{9}\bf8672    &\bf974    &\bf2262   &\hphantom{9}\bf8768  \\
$\cdots$ &$\cdots$   &$\cdots$    & $\cdots$      &$\cdots$      &$\cdots$     &$\cdots$    \\
55 &\hphantom{9}1069\hphantom{99} & 2297  & \hphantom{9}8630    &1010\hphantom{9}   &2259   &\hphantom{9}8553  \\
\hline
 \end{tabular}
\end{center}
 \end{table*}




Для базовой сети исходная сумма пропускных способностей:
$D^+(0)\hm=68\,256$, а~для кольцевой сети $D^{++}(0)=75\,456$.
Соответствующие значения $\rho^+(0)$ и~$\rho^{++}(0)$ указаны в~<<нулевой>> строке 
в~табл.~1 и~2, где собраны результаты
вычислительных экспериментов. В~ходе эксперимента при
уравнительном распределении остаточных ресурсов соблюдается
\textit{равномерное} убывание остаточной пропускной спо\-соб\-ности и~<<\textit{длины}>> вектора~$\rho^+(t)$. 
Однако между 44--46
итерациями для базовой и~50--52 для кольцевой сети наблюдается
резкий скачок величин~$\rho^-(t)$, $\rho^{=}(t)$ и~$d^+(t)$,
$d^{++}(t)$.

На указанных шагах полностью используется пропускная способность
ребер в~центральной час\-ти сети. Сеть \textit{распадается} на
несвязные компоненты, и~для $80\%$ корреспондентов пропадают пути
соединения, а~остаточный ресурс распределяется поровну между
оставшимися парами узлов.

Анализ результатов показал, что почти равные значения потоков
достигаются для~80\% корреспондентов и~требуют 60\%--70\%
ресурсов. Однако для~2\% смежных  пар узлов межузловые потоки на
два порядка выше медианных значений, а~затраты пропускной
способности  со\-став\-ля\-ют~20\%--30\%.








\section{Заключение}

Предложенный метод и~проведенные вычислительные эксперименты
показали, что уравнительное поэтапное распределение   приводит 
к~неравномерному  распределению   потоков  для разных групп\linebreak
корреспондентов.    Метрические оценки, полученные  в~ходе
экспериментов, продемонстрировали\linebreak \textit{деформацию} множества
достижимых потоков. В~рамках модели   предполагалось, что  все
корреспонденты  равноправны, а~потоки невзаимозаменяемы,  однако
при уравнительном предельном  распределении  смежные  пары узлов
оказывались в~привилегированном положении при использовании
остаточной пропускной способности. Пропускные способности  ребер
рассматривались  как вектор   ресурсов  различных типов,  которые
распределяются между корреспондентами   при передаче  потоков
различных видов.  По построению, на каж\-дом шаге норма вектора
смежных   межузловых    потоков численно равна   модулю вектора
остаточных  пропускных способностей.   Полученные мет\-ри\-че\-ские
значения  можно использовать  для   оценки функциональных
возможностей сети  в~режиме  предельной загрузки.

{\small\frenchspacing
 {%\baselineskip=10.8pt
 %\addcontentsline{toc}{section}{References}
 \begin{thebibliography}{9}

\bibitem{1-mal}
\Au{Малашенко Ю.\,Е., Назарова И.\,А.} Неоднородность
распределения   потоков при предельной  загрузке
многопользовательской сети~//  Известия РАН. Теория и~сис\-те\-мы
управления,  2022. №\,3. С.~81--96.

\bibitem{4-mal} %2
\Au{Luss H.} Equitable resource allocation: Models,
algorithms, and applications.~--- Hoboken, NJ, USA: John Wiley \& Sons, 2012.
420~p.

\bibitem{2-mal} %3
\Au{Ogryczak W., Luss~H., Pioro~M., Nace~D., Tomaszewski~A.}   Fair
optimization and networks: A~aurvey~// J.~Appl. Math., 2014. Vol.~2014. Art.~ID~612018. 25~p. doi: 10.1155/ 2014/612018.

\bibitem{3-mal} %4
\Au{Salimifard K., Bigharaz~S.} The multicommodity network
flow problem: State of the art classification, applications, and
solution methods~// J.~Oper. Res., 2020. Vol.~18. Iss.~3. P.~1--47.



\bibitem{5-mal}
\Au{Balakrishnan A., Li~G., Mirchandani~P.}  Optimal
network design with end-to-end service requirements~// Oper. Res.,
2017. Vol.~65. Iss.~3. P.~729--750.

\bibitem{6-mal}
\Au{Nace D., Doan~L.\,N., Klopfenstein~O., Bashllari~A.} Max-min
fairness in multicommodity flows~// Comput. Oper. Res., 2008.
Vol.~35. Iss.~2. P.~557--573.

\bibitem{7-mal}
\Au{Ros-Giralt J., Tsai~W.\,K.} A~lexicographic optimization
framework to the flow control problem~// IEEE T.
Inform. Theory, 2010. Vol.~56. Iss.~6. P.~2875--2886.

\bibitem{8-mal}
\Au{Baier G., Kohler~E., Skutella~M.}  The \mbox{k-splittable}
flow problem~//  Algorithmica, 2005. Vol.~42. Iss.~3-4.
P.~231--248.

\bibitem{9-mal}
\Au{Bialon P.\,A.} Randomized rounding approach to 
a~\mbox{k-splittable} multicommodity flow problem with lower path flow
bounds affording solution quality guarantees~// Telecommun. Syst.,
2017. Vol.~64. Iss.~3. P.~525--542.
\end{thebibliography}

 }
 }

\end{multicols}

\vspace*{-6pt}

\hfill{\small\textit{Поступила в~редакцию 10.06.22}}

\vspace*{8pt}

%\pagebreak

%\newpage

%\vspace*{-28pt}

\hrule

\vspace*{2pt}

\hrule

%\vspace*{-2pt}

\def\tit{SEQUENTIAL ANALYSIS AND METRIC ESTIMATES\\ OF~PEAK LOAD FLOWS IN~THE~MULTIUSER NETWORK}


\def\titkol{Sequential analysis and metric estimates of~peak load flows in~the~multiuser network}


\def\aut{Yu.\,E.~Malashenko}

\def\autkol{Yu.\,E.~Malashenko}

\titel{\tit}{\aut}{\autkol}{\titkol}

\vspace*{-8pt}


\noindent
Federal Research Center ``Computer Science and Control'' of the Russian Academy of Sciences, 
44-2~Vavilov Str., Moscow 119333, Russian Federation



\def\leftfootline{\small{\textbf{\thepage}
\hfill INFORMATIKA I EE PRIMENENIYA~--- INFORMATICS AND
APPLICATIONS\ \ \ 2022\ \ \ volume~16\ \ \ issue\ 3}
}%
 \def\rightfootline{\small{INFORMATIKA I EE PRIMENENIYA~---
INFORMATICS AND APPLICATIONS\ \ \ 2022\ \ \ volume~16\ \ \ issue\ 3
\hfill \textbf{\thepage}}}

\vspace*{3pt} 



\Abste{The set of vectors of internodal flows in a~multiuser communication network under peak load is analyzed. Within the framework of
 the multicommodity model, the throughput capacities of edges are considered as the components of a~vector of resources of various types that 
 are required for the transmission of various kinds of
 flows. When conducting computational experiments, at each iteration, the
  norms of vectors of jointly permissible internodal flows are calculated, during the transmission of which the capacity of 
  all network edges is fully used.\linebreak\vspace*{-12pt}}
 
 \Abstend{The proposed method and computational experiments have shown that the equalizing phased 
  distribution leads to an uneven distribution of flows for different groups of correspondents. Metric values obtained during experiments 
  indicate deformation of the sets of accessible flows. Within the framework of the model, all correspondents are tantamount 
  and the flows are noninterchangeable; however, in the case of an equalizing peak load distribution, adjacent pairs 
  of nodes are in a privileged position when using residual capacity. The obtained metric values can be used to 
  evaluate the functional characteristics of the transmission network in the finite capacity loading mode.}

\KWE{multicommodity flow network model; set of achievable internodal flows; peak load distribution}


\DOI{10.14357/19922264220306} 

%\vspace*{-16pt}

%\Ack
%\noindent



%\vspace*{4pt}

  \begin{multicols}{2}

\renewcommand{\bibname}{\protect\rmfamily References}
%\renewcommand{\bibname}{\large\protect\rm References}

{\small\frenchspacing
 {%\baselineskip=10.8pt
 \addcontentsline{toc}{section}{References}
 \begin{thebibliography}{9}
\bibitem{1-mal-1}
\Aue{Malashenko, Yu.\,E., and I.\,A.~Nazarova.}
2022. Heterogeneous flow distribution at the peak load in the multiuser network. \textit{J.~Comput. Sys. Sc. Int.} 61:372--387.

\bibitem{4-mal-1} %2
\Aue{Luss, H.} 2012. \textit{Equitable resource allocation: Models, algorithms, and applications}.
Hoboken, NJ: John Wiley \& Sons. 420~p.

\bibitem{2-mal-1} %3
\Aue{Ogryczak, W., H.~Luss, M.~Pioro, D.~Nace, and A.~Tomaszewski.}
 2014. Fair optimization and networks: A~survey. \textit{J.~Appl. Math.} 2014:612018. 25~p. doi: 10.1155/ 2014/612018.
\bibitem{3-mal-1} %4
\Aue{Salimifard, K., and S.~Bigharaz.}
 2020. The multicommodity network flow problem: State of the art classification, applications, and solution methods. 
 \textit{J.~Oper. Res.} 18(3):\linebreak 1--47.

\bibitem{5-mal-1}
\Aue{Balakrishnan, A., G.~Li, and P.~Mirchandani.} 2017. Optimal network design with end-to-end service requirements. 
\textit{Oper. Res.} 65(3):729--750.
\bibitem{6-mal-1}
\Aue{Nace, D., L.\,N.~Doan, O.~Klopfenstein, and A.~Bashllari.} 2008. Max-min fairness in multicommodity flows. 
\textit{Comput. Oper. Res.} 35(2):557--573.
\bibitem{7-mal-1}
\Aue{Ros-Giralt, J., and W.\,K.~Tsai.} 2010. A~lexicographic optimization framework to the flow control problem. 
\textit{IEEE T.~Inform. Theory} 56(6):2875--2886.
\bibitem{8-mal-1}
\Aue{Baier, G., E.~Kohler, and M.~Skutella.}
 2005. The k-splittable flow problem. \textit{Algorithmica} 42(3-4):231--248.
\bibitem{9-mal-1}
\Aue{Bialon, P.} 2017. A~randomized rounding approach to a~\mbox{k-splittable} multicommodity flow problem with lower path flow bounds affording solution quality guarantees. 
\textit{Telecommun. Syst.} 64(3):525--542.
 \end{thebibliography}

 }
 }

\end{multicols}

\vspace*{-6pt}

\hfill{\small\textit{Received June 10, 2022}}

\Contrl

\noindent
\textbf{Malashenko Yuri E.} (b.\ 1946)~--- 
Doctor of Science in physics and mathematics, principal scientist, Federal Research Center ``Computer Science and Control'' 
of the Russian Academy of Sciences, 44-2~Vavilov Str., Moscow 119333, Russian Federation; \mbox{malash09@ccas.ru} 


\label{end\stat}

\renewcommand{\bibname}{\protect\rm Литература}   %11+
\def\stat{agalarov}


\def\tit{ПРИБЛИЖЕННЫЙ МЕТОД ВЫЧИСЛЕНИЯ ХАРАКТЕРИСТИК УЗЛА 
ТЕЛЕКОММУНИКАЦИОННОЙ СЕТИ С~ПОВТОРНЫМИ ПЕРЕДАЧАМИ}
\def\titkol{Приближенный метод вычисления характеристик узла 
телекоммуникационной сети с~повторными передачами} 

\def\autkol{Я.\,М.~Агаларов}
\def\aut{Я.\,М.~Агаларов$^1$}

\titel{\tit}{\aut}{\autkol}{\titkol}

%{\renewcommand{\thefootnote}{\fnsymbol{footnote}}\footnotetext[1]
%{Работа выполнена при поддержке РФФИ, проекты 08--07--00152 и 08--01--00567.}}

\renewcommand{\thefootnote}{\arabic{footnote}}
\footnotetext[1]{Институт проблем
информатики Российской академии наук, agglar@yandex.ru}

%\vspace*{-6pt}


\Abst{Рассмотрена модель узла коммутации пакетов c повторными передачами для двух 
схем распределения буферной памяти: полнодоступной и полного разделения. Предложен 
приближенный метод вычисления интенсивностей потоков и вероятностей блокировок узла. 
Получены необходимые и достаточные условия существования и единственности решения 
уравнения для потоков в узле при установившемся режиме работы и доказана сходимость 
итерационного метода решения указанного уравнения.}

\KW{узел коммутации пакетов; буферная память; повторные передачи; вероятности 
блокировок; итерационный метод}

      \vskip 18pt plus 9pt minus 6pt

      \thispagestyle{headings}

      \begin{multicols}{2}

      \label{st\stat}


\section{Введение}

    Одной из основных задач предварительного анализа 
телекоммуникационных сетей коммутации пакетов с ограниченной буферной 
памятью является расчет характеристик потоков и вероятностей блокировок в 
узлах связи. Важность указанных характеристик определяется тем, что от их 
значений существенным образом зависят другие основные показатели сети 
(пропускная способность, задержки пакетов и~др.). 

    Существует множество различных моделей узлов коммутации пакетов и 
методов их расчета (см., например,~[1--6]). Для моделей, рассматривающих 
узел с ограниченной буферной памятью как систему массового обслуживания 
(CMO) типа 
$
\begin{matrix}
M \\ \lambda
\end{matrix}
\left |
\begin{matrix}
M \\ \lambda
\end{matrix}
\right |
\overline{m} \vert N
$ или  $\vert PH\vert PH\vert 1\vert r$, в предположении отсутствия повторных 
передач пакетов получены точные методы вычисления характеристик 
узлов~[1, 3, 4, 6]. Приближенные методы расчета узлов, учитывающие повторные 
попытки передачи, используют модели типа $\vert PH\vert PH\vert 1\vert r$ или 
$
\begin{matrix}
M \\ \lambda
\end{matrix}
\left |
\begin{matrix}
M \\ \lambda
\end{matrix}
\right |
1 \vert N
$ и являются 
итерационными~[2, 3, 5, 7]. Для моделей типа 
$BM\!AP\vert PH\vert 1$, $M\vert G\vert 1\vert r$ и $M\!AP\vert 
(PH,PH)\vert 1$ с повторными заявками получены точные методы вычисления 
характеристик (например, в работах~[8--10]), которые также могут быть 
использованы при расчете узлов.

    Ниже будут рассмотрены модели узла коммутации пакетов с повторными 
передачами для двух схем распределения буферной памяти: с 
полнодоступными буферами и с полным разделением буферной памяти. 
Предлагается приближенный метод расчета характеристик, который в качестве 
модели узла использует СМО типа $
\begin{matrix}
M \\ \lambda
\end{matrix}
\left |
\begin{matrix}
M \\ \lambda
\end{matrix}
\right |
\overline{m} \vert N
$ с повторными заявками. Доказаны утверждения о 
достаточных и необходимых условиях существования и единственности 
решения уравнения для вероятности блокировки в установившемся режиме 
работы и сходимости предлагаемого итерационного метода. 

\section{Модель узла}

    Математическая модель узла представляется в виде СМО с ограниченной 
буферной памятью и различными потоками заявок, каждая из которых требует 
обслуживания только на одной из многоканальных линий связи. 

    Пусть $0<N<\infty$~--- число мест хранения в буферной памяти, $u$~--- 
узел связи, $v$~--- линия связи, $\Omega_u^+$~--- множество исходящих из 
узла~$u$ линий, $c_v$~--- канальная емкость линии~$v$. Поток заявок, 
тре\-бу\-ющих обслуживания на линии~$v$, назовем $v$-по\-то\-ком, заявки этого 
потока~--- $v$-за\-яв\-ка\-ми.


    Пусть выполняются следующие предположения: 
\begin{enumerate}[1.]
\item Места в буферной памяти распределяются согласно одной из двух 
схем:
\begin{enumerate}[($i$)]
\item полнодоступная схема~--- каждое свободное место хранения доступно 
любой заявке;
\item схема полного разделения памяти~--- $v$-за\-яв\-кам доступны всего 
$N_v$ мест, где $\sum\limits_{v\in\Omega_u^+} N_v=N$.
\end{enumerate}
\item Если в момент поступления $v$-заявки в буферной памяти есть 
доступное свободное место, то она сразу занимает это место. Если в момент 
поступления $v$-заявки в системе нет свободного доступного места 
хранения, то поступившая заявка через некоторое время повторно поступает 
на систему, оставаясь $v$-заявкой. 
\item Интенсивности первичных потоков $v$-заявок~--- заданные величины 
$0<\Lambda_v<\infty$, $v\in \Omega_u^+$. Суммарные потоки первичных и 
повторных $v$-заявок являются независимыми в совокупности 
пуассоновскими потоками. Для обслуживания $v$-заявки требуется 
одновременно одно место хранения и один канал типа~$v$, $v\in 
\Omega_u^+$.
\item Первичные нагрузки~--- реализуемые, т.\,е.\ в данном случае 
интенсивности входных первичных потоков равны интенсивностям 
выходных потоков выполненных заявок. 
\item Принятые в СМО $v$-заявки обслуживаются линией~$v$ в порядке 
поступления. 
\item Время занятия канала $v$-заявкой~--- экспоненциально 
распределенная случайная величина с параметром $0<\mu_v<\infty$, 
$v\in\Omega_u^+$, независимая от других случайных событий в узле.
\item Выполненная $v$-заявка с вероятностью~$B_v$ повторяется через 
заданное время~$\tau_v$ (тайм-аут) и с вероятностью $1-B_v$ покидает 
систему через время~$t_v$ навсегда, сразу освободив занятый канал и место 
буферной памяти.
\end{enumerate}

   Будем говорить, что узел блокирован для $v$-за\-яв\-ки, если в буферной 
памяти отсутствует доступное место хранения. Ставится задача вычисления 
вероятностей блокировок и интенсивностей потоков в узле.

\section{Вычисление вероятности блокировки и~интенсивностей~потоков} 

   Пусть $\Lambda_v^*$~--- интенсивность суммарного потока внешних 
заявок, требующих передачи по линии~$v$, $\pi_v$~--- вероятность блокировки 
узла для заявок, требующих передачи по исходящей из узла линии~$v$. 

    Пусть в узле используется полнодоступная схема распределения 
буферной памяти. Тогда, как следует из описания модели, $\pi_v 
=\pi_{v^\prime},\,v,\,v^\prime\in \Omega_u^+$, и для 
интенсивностей~$\Lambda_v^*$, $v\in\Omega_u^+$, справедливы соотношения:
\begin{equation*}
\Lambda_v^* = \fr{\Lambda_v}{1-\pi}\,,
%\label{e1aga}
\end{equation*}
    где
    $\pi =\pi_v$, $v\in\Omega_u^+$.

    Пусть 
    $\overline{k} = \{\overline{k}_v$, $v\in\Omega_u^+\}$~--- состояние 
буферной памяти узла, $\overline{k}_v =\left ( k_v,\,k_v^\prime,\,k_v^{\prime\prime}\right )$; 
$k_v$~--- число $v$-заявок в буферной 
памяти, ожидающих выполнения линией~$v$; $k^\prime_v$~--- число 
$v$-заявок в буферной памяти, ожидающих тайм-аут и неуспешно переданных 
в последующий узел; $k_v^{\prime\prime}$~--- число $v$-за\-явок в буферной 
памяти, успешно переданных в последующий узел и ожидающих 
потверждения; 
$A_m = \left \{ \overline{k}:\ \sum\limits_{v\in\Omega_u^+} \left ( 
k_v+k_v^\prime + k_v^{\prime\prime}\right ) =m \right \}$~--- множество различных 
состояний, при которых в памяти узла занято ровно $m$~буферов. Тогда с 
учетом введенных выше обозначений и предположений для ве\-ро\-ят\-ности 
блокировки узла можно написать формулу~\cite{1aga, 2aga}:
\begin{equation}
\pi = \fr{1}{G_N}\sum\limits_{\overline{k}\in A_N} 
p\left (\overline{k},\overline{\rho}^*\right )\,,
\label{e2aga}
\end{equation}
где  
\begin{gather}
p(\overline{k},\overline{\rho}^*) = \prod\limits_{v\in\Omega_u^+} z_v (\pi, 
\rho_v , k_v , k_v^\prime , k_v^{\prime\prime})\,;\\
z_v (\pi, \rho_v , k_v , k_v^\prime , k_v^{\prime\prime}) ={}\notag\\
\!\!{}=
\begin{cases}
 \fr{\rho_v^{\prime *k_v^\prime}}{k_v^{\prime}!}\,
\fr{\rho_v^{\prime\prime * k_v^{\prime\prime}}}{ k_v^{\prime\prime}!}  \,
\fr{\rho_v^{*k_v}}{ k_{v}!} 
&\mbox{при}\ k_v<c_v\,,\\
 \fr{\rho_v^{\prime * k_v^\prime}}{k_v^{\prime}!} \,
\fr{\rho_v^{\prime\prime * k_v^{\prime\prime}}} { k_v^{\prime\prime}!} 
\fr{\rho_v^{*k_v}}{ c_{v}!c_v^{k_v- c_v}} 
& \mbox{при}\ k_v\geq c_v\,;
\end{cases}\\
G_N = \sum\limits_{m=0}^N\sum\limits_{\overline{k}\in A_m}
p(\overline{k},\overline{\rho}^*)\,;\\ 
\overline{\rho}^*=\{\rho_v^*,\,v\in\Omega_u^+\}\,;\\
\rho_v^* = \fr{\rho_v}{1-\pi}\,;\quad \rho_v =\fr{\Lambda_v}{\mu_v(1- B_v)}\,;\\
\rho_v^{\prime *} =\rho_v^*\mu_v\tau_vB_v\,;\quad \rho_v^{\prime\prime *}=
p_v^* \mu_vt_v,\,\quad  v\in \Omega_u^+\,.\label{e3aga}
\end{gather}

Переобозначив $1-\pi$ через $y$, выражение в правой части равенства~(2)~--- через 
$p_{\overline{k}}(\overline{\rho},y)$, выражение в правой части равенства~(4)~--- 
через $g_N(\overline{\rho},y)$, а выражение в правой 
части равенства~(1)~--- через $1-q_N (\overline{\rho},y)$, 
где $\overline{\rho} = (\rho_v,\,v\in \Omega_u^+)$, $\rho_v = \rho_v^*y\;=$\linebreak 
$=\;\Lambda_v/(\mu_v(1-B_v))$, $v\in\Omega_u^+$, получим нелинейное уравнение 
относительно неизвестной переменной~$y$:
\begin{equation}
y=q_N(\overline{\rho},y)\,.
\label{e4aga}
\end{equation}

    Решим уравнение~(8). Как следует из~(2)--(7), верно 
равенство
\begin{equation}
q_N(\overline{\rho},y) = \fr{g_{N-1}(\overline{\rho},y )}{g_N(\overline{\rho},y)}\,.
\label{e5aga}
\end{equation}
Введем функцию  $d_n(\overline{\rho} ,y)$ среднего числа заявок в узле с 
буферной памятью емкости $n\geq 0$:
$$
d_n(\overline{\rho} ,y) = 
\fr{1}{g_n(\overline{\rho},y)}\,\sum\limits_{m=0}^n m\sum\limits_{\overline{k}\in 
A_m} p_{\overline{k}}(\overline{\rho},y)\,.
$$
Заметим, что $g_n$, $d_n$ и $q_n$, 
$n\geq 0$,~--- непрерывно-дифференцируемые функции по $y\in (0,\,1]$. Взяв 
производную функции~$g_n$ по~$y$, из~(2)--(7) получим
\begin{multline}
\fr{\partial g_n(\overline{\rho},y)}{\partial y} ={}\\
{}= -\sum\limits_{m=0}^n m 
\sum\limits_{\overline{k}\in A_m}\fr{\prod\limits_{v\in\Omega_u^+} z_n 
(0,\rho_v, k_v, k_v^\prime , k_v^{\prime\prime})}{y^{m+1}}={}\\
{}= -\fr{1}{y}\,g_n (\overline{\rho},y)d_n(\overline{\rho},y)\,.
\label{e6aga}
\end{multline}
Взяв производную функции $q_N$ по $y$, из~(\ref{e5aga}) и~(\ref{e6aga}) 
получим
\begin{equation}
\fr{\partial q_N(\overline{\rho},y)}{\partial y} = \fr{q_N(\overline{\rho},y)}{y}\left 
[ d_N (\overline{\rho},y)-d_{N-1}(\overline{\rho},y)\right ]\,.
\label{e7aga}
\end{equation}
    Докажем несколько утверждений о свойствах 
функции~$q_N(\overline{\rho},y)$.
\medskip

\noindent
\textbf{Утверждение 1.} \textit{Справедливы неравенства}
\begin{multline}
0<d_{n+1}(\overline{\rho},y)-d_n(\overline{\rho},y) <1\,,\\
\ \ \ \ \ \ \ \ \ \ \ \ \ \ \ \ \ \ \ \ y\in (0,\,1]\,, \ n\geq 0\,.
\label{e8aga}
\end{multline}


\noindent

Д\,о\,к\,а\,з\,а\,т\,е\,л\,ь\,с\,т\,в\,о\,.\ Подставив выражение для функции 
$d_n(\overline{\rho},y)$ и проведя преобразования, получим
\begin{multline*}
d_{n+1}(\overline{\rho},y) -d_n(\overline{\rho},y) = 
\fr{\sum\limits_{m=0}^{n+1}m\sum\limits_{\overline{k}\in A_m} 
p_{\overline{k}}(\overline{\rho},y)}
{\sum\limits_{m=0}^{n+1}
\sum\limits_{\overline{k}\in A_m} p_{\overline{k}}(\overline{\rho},y)} - {}\\
{}-
\fr{\sum\limits_{m=0}^n m \sum\limits_{\overline{k}\in A_m} p_{\overline{k}} 
(\overline{\rho},y)}{\sum\limits_{m=0}^n
\sum\limits_{\overline{k}\in A_m}p_{\overline{k}}(\overline{\rho},y)}={}\\
{}=\fr{\sum\limits_{m=1}^n m \sum\limits_{\overline{k}\in 
A_m}p_{\overline{k}}(\overline{\rho},y)+(n+1)\sum\limits_{\overline{k}\in 
A_{n+1}}  p_{\overline{k}}(\overline{\rho},y)}{\sum\limits_{m=0}^n\sum\limits_{\overline{k
}\in A_m}p_{\overline{k}}(\overline{\rho},y)+\sum\limits_{\overline{k}\in 
A_{n+1}}p_{\overline{k}}(\overline{\rho},y)} -{}
\end{multline*}
\begin{multline}
{}-
\fr{\sum\limits_{m=0}^n m 
\sum\limits_{\overline{k}\in A_m}p_{\overline{k}}(\overline{\rho},y)}
{\sum\limits_{m=0}^n\sum\limits_{\overline{k}\in A_m} 
p_{\overline{k}}(\overline{\rho},y)}={}\\
{}=\fr{(n+1)\sum\limits_{\overline{k}\in 
A_{n+1}}p_{\overline{k}}(\overline{\rho},y)g_n(\overline{\rho},y)}{g_{n+1}(\overline{\rho},y) g_n(\overline{\rho},y)} -{}\\
{}-
\fr{\sum\limits_{\overline{k}\in 
A_{n+1}}p_{\overline{k}}(\overline{\rho},y)\sum\limits_{m=0}^n  m 
\sum\limits_{\overline{k}\in A_m} p_{\overline{k}}(\overline{\rho},y) }
{g_{n+1}(\overline{\rho},y) g_n(\overline{\rho},y)}
={}\\
{}=\left [ 1-q_{n+1}(\overline{\rho},y)\right ] \left [n+1-d_n(\overline{\rho},y)\right ]\,.
\label{e9aga}
\end{multline}


    Докажем утверждение~1 методом индукции. При $n = 0$, как следует 
из~(\ref{e9aga}), имеем
$$
d_2(\overline{\rho},y) - d_1 (\overline{\rho},y) =1-q_1(\overline{\rho},y)\,,
$$
    т.\,е.\ утверждение~1 при $n = 0$ справедливо. 

    Пусть неравенства~(\ref{e8aga}) справедливы для некоторого $n > 0$. 
Докажем, что они справедливы и для $n + 1$. Из~(\ref{e9aga}) получаем
\begin{multline*}
d_{n+1}(\overline{\rho},y)- d_n(\overline{\rho},y)={}\\
{}=\left [ 1-
q_{n+1}(\overline{\rho},y)\right ] \left [n+1-d_n(\overline{\rho},y)\right ] ={}\\
{}= \left [ 1-
1-q_{n+1}(\overline{\rho},y)\right ] \left [ n-{}\right.\\
{}-\left. d_{n-1}(\overline{\rho},y)+d_{n-1}(\overline{\rho},y)-
d_n(\overline{\rho},y)+1\right ] ={}\\
{}=\left [ 1-q_{n+1}(\overline{\rho},y)\right ] 
\left [ n-d_{n-1}(\overline{\rho},y)-{}\right.\\
{}-\left. \left ( d_n(\overline{\rho},y)-d_{n-1}(\overline{\rho},y)\right )+1\right] = {}\\
{}=
\left [ 1-q_{n+1}(\overline{\rho},y)\right ]
\left [ 
\fr{d_n(\overline{\rho},y) -d_{n-1}(\overline{\rho},y)}{1-
q_n(\overline{\rho},y)}\right.-{}\\
{}-\left.
\left ( d_n(\overline{\rho},y)-d_{n-1}(\overline{\rho},y)\right )+1
\vphantom{\fr{d_n(\overline{\rho})}{(q_n)}}
\right ]={}\\
{}=
\left [ 1-q_{n+1}(\overline{\rho},y)\right ]
\left [ 
\vphantom{\fr{d_n(\overline{\rho})}{(q_n)}}
\left ( d_n(\overline{\rho},y\right)\right. -{}\\
 {}-\left.
d_{n-1}\left(\overline{\rho},y)\right )\fr{q_n(\overline{\rho},y)}{1-
q_n(\overline{\rho},y)}+1\right ]\,.
\end{multline*}
Так как по предположению $d_n (\overline{\rho},y) -d_{n-1}(\overline{\rho},y) 
>0$, то правая часть последнего равенства больше нуля; следовательно, 
$d_{n+1}(\overline{\rho},y)-d_n(\overline{\rho},y)>0$. 

    Продолжив преобразование правой части последнего равенства и 
учитывая предположение $d_n(\overline{\rho},y) -d_{n-1}(\overline{\rho},y)<1$, 
получим
\begin{multline*}
d_{n+1}((\overline{\rho},y) -d_n(\overline{\rho},y)<{}\\
{}< \left [ 1-
q_{n+1}(\overline{\rho},y)\right ]
\left ( \fr{q_n(\overline{\rho},y)}{1-q_n(\overline{\rho},y)}+1\right )={}\\
{}=
\fr{1-q_{n+1}(\overline{\rho},y)}{1-q_n(\overline{\rho},y)}<1\,,
\end{multline*}
так как $0< q_n(\overline{\rho},y)<q_{n+1}(\overline{\rho},y)<1$, $n>0$, $y\in 
(0,\,1]$.

Следовательно, утверждение~1 доказано.

\medskip

\noindent
\textbf{Утверждение 2.} $q_N(\overline{\rho},y)$~--- \textit{монотонно-воз\-рас\-та\-ющая 
функция по $y\in (0,\,1]$. При этом $0< q_N(\overline{\rho},y)\;\leq $\linebreak 
$\leq\;q_N(\overline{\rho},1) <1$, $y\in (0,\,1]$,  и $\underset{y\rightarrow 
0}{\mathrm{lim}}\,q_N(\overline{\rho},y) =0$}.

\medskip

\noindent
Д\,о\,к\,а\,з\,а\,т\,е\,л\,ь\,с\,т\,в\,о\,.\  Возрастание функции 
$q_N(\overline{\rho},y)$ следует непосредственно из~(\ref{e7aga}) и 
утверж\-де\-ния~1. Доказательство неравенств в условии утверждения очевидно 
следует из~(\ref{e5aga}) и вида функции $g_n (\overline{\rho},y)$, $n\geq 0$. 
Для предела имеем:
\begin{multline*}
\underset{y\rightarrow 0}{\mathrm{lim}}\,q_N(\overline{\rho},y) 
=\underset{y\rightarrow 0}{\mathrm{lim}}\,\fr{g_{N- 1}(\overline{\rho},y)}{g_N(\overline{\rho},y)} = {}\\
{}= \underset{y\rightarrow 0}{\mathrm{lim}}\,\left (
g_{N-1}(\overline{\rho},y)\Bigg / \left ( 
\vphantom{\prod\limits_{v\in\Omega_u^+}}
g_{N-1}(\overline{\rho},y)\right.\right.+{}\\
{}+\left.\left.\sum\limits_{\overline{k}\in A_N}\prod\limits_{v\in\Omega_u^+} 
\fr{z_v(0,\rho_v,k_v,k^\prime_v,k^{\prime\prime}_v)}{y^N}\right )\right ) = {}\\
{}= \underset{y\rightarrow 0}{\mathrm{lim}}\,\left (
y^N g_{N-1}(\overline{\rho},y)\Bigg / 
\left ( 
\vphantom{\prod\limits_{v\in\Omega_u^+}}
y^N g_{N-1}(\overline{\rho},y)+{}\right.\right.\\
{}+\left.\left.\sum\limits_{\overline{k}\in A_N}
\prod\limits_{v\in\Omega_u^+} z_v(0,\rho_v,k_v,k_v^\prime , k_v^{\prime\prime}) 
\right ) \right )=0\,.
\end{multline*}
    
\medskip

\noindent
\textbf{Утверждение 3.} \textit{Производная функции~$q_N (\overline{\rho},y)$ по 
$y\in (0,\,1]$ удовлетворяет следующим соотношениям}:
\begin{align}
\underset{y\rightarrow 0}{\mathrm{lim}}\fr{\partial q_N(\overline{p},y)}
{\partial  y} &= \fr{\sum\limits_{\overline{k}\in A_{N-1}} 
p_{\overline{k}}(\overline{\rho},1)}{\sum\limits_{\overline{k}\in 
A_N}p_{\overline{k}}(\overline{\rho},1)}\,;\label{e10aga}\\
\fr{\partial q_N(\overline{\rho},y)}{\partial y}\Big |_{y=1}&<1\,.\label{e11aga}
\end{align}

\medskip

\noindent
Д\,о\,к\,а\,з\,а\,т\,е\,л\,ь\,с\,т\,в\,о\,.\ Проведя преобразования 
функции~$q_N(\overline{\rho},y)$, получим:
\begin{multline*}
\underset{y\rightarrow 0}{\mathrm{lim}}\fr{q_N(\overline{\rho},y)}{y} = {}\\
\!\!{}=
\underset{y\rightarrow 0}{\mathrm{lim}}
\fr{\sum\limits_{m=0}^{N-1}\sum\limits_{\overline{k}\in A_m}
\!\!\left (\prod\limits_{v\in\Omega_u^+}\!\! 
z_v(0,\rho_v,k_v,k_v^\prime , k_v^{\prime\prime})\right )\!\!\Bigg /\!\! y^m}
{y\sum\limits_{m=0}^{N}\sum\limits_{\overline{k}\in A_m}
\!\!\left(\prod\limits_{v\in\Omega_u^+}\!\! z_v\left (0,\rho_v,k_v,k_v^\prime , 
k_v^{\prime\prime}\right )\right )\!\!\Bigg /\!\!y^m} = \!\!\!
\end{multline*}
\begin{multline*}
\!\!\!\!\!\!{}=\underset{y\rightarrow 0}{\mathrm{lim}}\,
\fr{\sum\limits_{m=0}^{N-1}\sum\limits_{\overline{k}\in A_m}
y^{N-1-m}\prod\limits_{v\in\Omega_u^+} z_v(0,\rho_v,k_v,k_v^\prime , 
k_v^{\prime\prime})}{\sum\limits_{m=0}^{N}\sum\limits_{\overline{k}
\in A_m} y^{N-m}
\prod\limits_{v\in\Omega_u^+} z_v(0,\rho_v,k_v,k_v^\prime , 
k_v^{\prime\prime})}={}\!\\
{}=\fr{\sum\limits_{\overline{k}\in A_{N-1}} p_{\overline{k}}(\overline{\rho},1)}{ 
\sum\limits_{\overline{k}\in A_{N}} p_{\overline{k}}(\overline{\rho},1)}\,.
\end{multline*}
Очевидно, $\underset{y\rightarrow 0}{\mathrm{lim}} \,[d_N (\overline{\rho},y) -
d_{N-1} (\overline{\rho},y)]=1$, так как $\underset{y\rightarrow 
0}{\mathrm{lim}}\,d_n (\overline{\rho},y)=n$, $n>0$.

Следовательно, учитывая~(\ref{e7aga}), получаем~(\ref{e10aga}). 
Справедливость~(\ref{e11aga}) непосредственно следует из~(\ref{e7aga}) и 
утверждения~1.

\medskip

\noindent
\textbf{Утверждение 4.} \textit{Пусть $y^*\in (0,\,1]$~--- решение 
уравнения}~(\ref{e4aga}). \textit{Тогда}
\begin{equation*}
\fr{\partial q_N(\overline{\rho},y)}{\partial y}\Big |_{y=y^*}<1\,.
%\label{e12aga}
\end{equation*}

\medskip

\noindent
Д\,о\,к\,а\,з\,а\,т\,е\,л\,ь\,с\,т\,в\,о\,.\ \ Доказательство следует из~(\ref{e7aga}), 
так как $q_N(\overline{\rho},y^*)/y^* =1$.
\medskip

\noindent
\textbf{Утверждение 5.} \textit{Уравнение}~(\ref{e4aga}) \textit{имеет решение $y^*\in 
(0,\,1)$ тогда и только тогда, когда} 
\begin{equation}
\fr{\sum\limits_{\overline{k}\in A_{N-1}} p_{\overline{k}}(\overline{\rho},1)}{ 
\sum\limits_{\overline{k}\in A_{N}} p_{\overline{k}}(\overline{\rho},1)} >1\,.
\label{e13aga}
\end{equation}
\textit{Если уравнение}~(\ref{e4aga}) \textit{имеет решение $y^*\in (0,\,1)$, то оно 
единственное положительное решение}.
\medskip

\noindent
Д\,о\,к\,а\,з\,а\,т\,е\,л\,ь\,с\,т\,в\,о\,.\ Пусть выполняется 
неравенство~(\ref{e13aga}). Тогда, как следует из утверждения~3, 
$\underset{y\rightarrow 0}{\mathrm{lim}} (\partial q_N(\overline{\rho},y)/\partial y) 
>1$. Кроме того, как следует из утверждения~2, 
$\underset{y\rightarrow 0}{\mathrm{lim}} q_N(\overline{\rho},y)=0$. Тогда, так 
как $q_N(\overline{\rho},y)$~--- непрерывно-дифференцируемая функция по 
$y\in (0,\,1]$, существует значение $y^\prime \in (0,\,1)$ такое, что 
$q_N(\overline{\rho},y)>y$ для всех $y\in (0,\,y^\prime]$ (следует из теоремы о 
конечном приращении~\cite{11aga}). В то же время, согласно утверждению~2, 
$q_N(\overline{\rho},y)<y$ в окрестности точки $y=1$ (рис.~\ref{f1aga},\,\textit{а}). 
Следовательно, кривая $x=q_N(\overline{\rho},y)$ пересекает прямую $x=y$ 
хотя бы в одной точке $y=y^*\in (0,\,1)$, т.\,е.\ уравнение~(\ref{e4aga}) имеет 
хотя бы одно решение $y^*\in (0,\,1)$.

\begin{figure*}
\vspace*{1pt}
\begin{center}
\vspace*{1pt}
\mbox{%
\epsfxsize=158mm
\epsfbox{aga-1.eps}
}
\end{center}
\vspace*{-9pt}
\Caption{Примеры кривых $x=q_N(\overline{\rho},y)$ и $x=y$ (\textit{а})~при существовании решения 
уравнения~(\ref{e5aga}) и (\textit{б})~при выполнении условий~(17)
\label{f1aga}}
\vspace*{6pt}
\end{figure*}

Пусть уравнение~(\ref{e4aga}) имеет решение $y^*\in (0,\,1)$ и 
\begin{equation}
\fr{\sum\limits_{\overline{k}\in A_{N-1}}p_{\overline{k}}(\overline{\rho},1)}{ 
\sum\limits_{\overline{k}\in A_{N}}p_{\overline{k}}(\overline{\rho},1)}\leq 
1\,.\label{e14aga}
\end{equation}
Тогда из условий утверждений~2 и~3 следует, что 
уравнение~(\ref{e4aga}) в интервале $(0,\,1)$ имеет более одного решения, что 
может быть только при существовании решения $y^\prime \in (0,\,1)$ такого, 
что в окрестности точки $y=y^\prime$ выполняются неравенства: 
$q_N(\overline{\rho},y)<y$ при $y<y^\prime$ и $q_N(\overline{\rho},y)>y$ при 
$y>y^\prime$, где $y$ принадлежит указанной окрест\-ности точки~$y^\prime$ 
(рис.~\ref{f1aga},\,\textit{б}). Тогда в точке $y=y^\prime$ производная функции 
$q_N(\overline{\rho},y)$ по $y$ больше~1, что противоречит утверждению~4. 
Следовательно, неравенство~(\ref{e13aga}) справедливо.


Пусть уравнение~(\ref{e4aga}) имеет более одного положительного 
решения. Рассуждая точно так же, как и выше (в случае выполнения 
условий~(\ref{e14aga})), получим противоречие утверждению~4. 
Следовательно, утверждение~5 справедливо.
\medskip

\noindent
\textbf{Следствие.} \textit{Неравенства}
\begin{gather*}
\fr{\mu_v c_v (1-B_v)}{\Lambda_v}>1\,,\quad \fr{1-B_v}{\Lambda_v \tau_v B_v}>1\,,\\ 
\fr{1-B_v}{\Lambda_v t_v}>1\,,\ v\in\Omega_u^+\,,
\end{gather*}
\textit{являются необходимым условием существования решения 
уравнения}~(\ref{e4aga}).

\medskip
\noindent
Д\,о\,к\,а\,з\,а\,т\,е\,л\,ь\,с\,т\,в\,о\,.\ Пусть $\overline{k}_v$~--- это 
набор~$\overline{k}$, у которого $k_v=0$. Преобразовав левую 
часть~(\ref{e13aga}), получим

\noindent
\begin{multline*}
\fr{\sum\limits_{\overline{k}\in A_{N-1}} p_{\overline{k}} (\overline{\rho},1)}
{ \sum\limits_{\overline{k}\in A_{N}} 
 p_{\overline{k}}(\overline{\rho},1)} 
={}
\\
{}=
\fr{\sum\limits_{\overline{k}\in A_{N-1}}\prod\limits_{v\in \Omega_u^+} 
z_v\left(0,\rho_v,k_v,k_v^\prime , k_v^{\prime\prime}\right)}
{\sum\limits_{\overline{k}\in A_{N}}
\prod\limits_{v\in \Omega_u^+} z_v\left (0,\rho_v,k_v,k_v^\prime , k_v^{\prime\prime}\right )} \leq{}
\\
{}\leq
\left ( 
\vphantom{\prod\limits_{v^\prime\in\Omega_u^+\backslash v}}
\fr{\mu_v c_v(1-B_v)}{\Lambda_v}\right. \times{}\\
{}\times \sum\limits_{k_v=0}^{N-1}\sum\limits_{\overline{k}_v\in A_{N-1-k_v}} z_v\left(0,\rho_v,k_v+1,k_v^\prime , 
k_v^{\prime\prime}\right )\times{}\\
{}\times \left.\prod\limits_{v^\prime\in\Omega_u^+\backslash v} z_v^\prime 
\left(0,\rho_v,k_v,k_v^\prime , k_v^{\prime\prime}\right) \right)
\Bigg /{}\\
\Bigg / \left ( 
\vphantom{\prod\limits_{v^\prime\in\Omega_u^+\backslash v}}
\sum\limits_{k_v=0}^{N-1} \sum\limits_{\overline{k}_v\in A_{N-1-k_v}} z_v 
\left (0,\rho_v,k_v+1,k_v^\prime , 
k_v^{\prime\prime}\right )\right. \times{}\\
{}\times \prod\limits_{v^\prime\in\Omega_u^+\backslash v} 
z_{v^\prime}\left(0,\rho_v,k_v,k^\prime , k_v^{\prime\prime}\right)+{}\\
{}+
\sum\limits_{\overline{k}_v\in A_N} z_v\left (0,\rho_v, 0,k_v^\prime , 
k_v^{\prime\prime}\right)\times{}\\
\left.{}\times \prod\limits_{v^\prime\in\Omega_u^+\backslash v}z_{v^\prime} 
\left(0,\rho_v,k_v,k_v^\prime , k_v^{\prime\prime}\right )\right )\,.
\end{multline*}
Как следует из правой части последнего неравенства, если 
$\mu_v c_v (1-B_v)/\Lambda_v \leq 1$, то она меньше~1. Поэтому, чтобы 
выполнилось условие~(\ref{e13aga}), необходимо выполнение первого 
неравенства в условии следствия для каждого $v\in\Omega_u^+$. Точно так же 
доказывается необходимость выполнения второго и третьего неравенств в 
условии следствия.

    Пусть $y[n]$, $n\geq 0$, последовательность, полученная по формуле 
$y[n+1]=q_N(\overline{\rho},y[n])$, $y[0]=1$.

\medskip

\noindent
\textbf{Утверждение 6.} \textit{Пусть $y^*\in (0,\,1)$~--- решение 
уравнения}~(8). \textit{Тогда последовательность $y[n]$, $n\geq 0$, сходится 
к решению~$y^*$}.

\medskip

\noindent
Д\,о\,к\,а\,з\,а\,т\,е\,л\,ь\,с\,т\,в\,о\,.\ Отметим, что $y[1]<y[0]$ (это следует из 
утверждения~2, так как $y[0]=1$). Пусть для некоторого $n>1$ выполняется 
условие $y[n]<y[n-1]$. Тогда, как следует из утверждения~2, указанное условие 
выполняется и для $n+1$, т.\,е.\ по индукции следует, что последовательность 
$y[n]$, $n\geq 0$, монотонно убывает. 

    Пусть для некоторого $n>0$ $y[n]>y^*$ (существование такого $n$ 
следует из равенства $y[0]=1$). Тогда, как следует из утверждения~2, 
$y[n+1]\;=$\linebreak $=\;q_N(\overline{\rho},y[n])>q_N(\overline{\rho},y^*) =y^*$, т.\,е.\ 
последовательность ограничена снизу величиной~$y^*$. Значит, существует 
$\underset{n\rightarrow \infty}{\mathrm{lim}}\,y[n]=y^0\geq y^*$. Так как 
$q_n(\overline{\rho},y)$~--- непрерывная по~$y$ функция, то можно написать 
$\underset{n\rightarrow 
\infty}{\mathrm{lim}}\,q_N(\overline{\rho},y[n])=q_N(\overline{\rho},y^0)=y^0$, 
т.\,е.\ $y^0$~--- решение уравнения~(\ref{e4aga}). Из единственности 
положительного решения уравнения~(\ref{e4aga}) получаем $y^0=y^*$.

    Пусть в узле используется схема полного разделения буферной памяти. 
Тогда для интенсив\-ностей~$\Lambda_v^*$, $v\in\Omega_u^+$, справедливы 
соотношения:
$$
\Lambda_v^* = \fr{\Lambda_v}{1-\pi_v}\,,
$$
где $v\in\Omega_u^+$.


Фиксируем произвольную линию сети~$v$. Пусть $\overline{k}_v = (k_v, 
k_v^\prime, k_v^{\prime\prime})$~--- состояние буферной памяти линии~$v$; 
$k_v$, $k_v^\prime$, $k_v^{\prime\prime}$ определены выше. Тогда с 
учетом введенных ранее предположений и обозначений для вероятности 
блокировки линии справедлива формула~\cite{4aga}:
\begin{equation}
\pi_v = \fr{1}{G_{N_v}}\sum\limits_{k_v=N_v} 
z_v(\pi_v,\rho_v,\overline{k}_v)\,,
\label{e15aga}
\end{equation}
где 
\begin{multline*}
z_v(\pi_v, \rho_v, \overline{k}_v)={}\\
{}=
\begin{cases}
\fr{\rho_v^{\prime * k_v^\prime}}{k_v^\prime !}\,
 \fr{\rho_v^{\prime\prime * k_v^{\prime\prime}}}{k_v^{\prime\prime}!}\,
 \fr{\rho_v^{*k_v}}{k_v !} & \mbox{при}\ k_v<c_v\,,\\
 \fr{\rho_v^{\prime *k_v^\prime}}{k_v^{\prime }! }
 \fr{\rho_v^{\prime\prime * k_v^{\prime\prime}}}{k_v^{\prime\prime}!}
\fr{\rho_v^{*k_v}}{c_v !c_v^{k_v-c_v}} & \mbox{при}\ k_v\geq c_v\,;
\end{cases}
\end{multline*}
\begin{align*}
G_{N_v} &= \sum\limits_{m=0}^{N_v} z_v (\pi_v ,\rho_v , \overline{k}_v)\,;\\ 
\rho_v^*&=\fr{\rho_v}{1-\pi_v}\,;
\end{align*}
$\rho_v$, $\rho_v^{\prime *}$, 
$\rho_v^{\prime\prime *}$, $v\in\Omega_u^+$ определены выше.
    
Пусть $y_v=1-\pi_v$, а $q_{N_v} (\rho_v, y_v)$~--- выражение в правой 
части~(\ref{e15aga}). Тогда из равенств~(\ref{e15aga}), взяв~$y_v$ в качестве 
неизвестной переменной, получим систему независимых уравнений:
\begin{equation}
y_v = q_{N_v}(\rho_v, y_v)\,, \quad v\in \Omega_u^+\,.
\label{e16aga}
\end{equation}
    
    Заметим, что для фиксированной $v$ и заданных параметров $\Lambda_v$, 
$\mu_v$, $\tau_v$, $t_v$, $N_v$, $v\in\Omega_u^+$, уравнение в~(\ref{e16aga}) 
является частным случаем уравнения~(\ref{e4aga}) и совпадает с последним, 
когда число исходящих линий из узла равно~1. Следовательно, для схемы 
полного разделения памяти также справедливы все приведенные выше 
утверждения~1--6 и следствие. Заметим, что неравенство~(\ref{e13aga}) в 
условии утверждения~5 при $B_v=0$ и $t_v=0$ преобразуется в неравенство 
$\Lambda_v / (\mu_v c_v) >1$, $v\in\Omega_u^+$. Последовательность 
$\overline{y}[n]$, $n\geq 0$, в утверждении~6 определяется по формуле:
    \begin{gather*}
    \overline{y}[n] =\{y_v[n],\ v\in\Omega_u^+\}\,,\
    y_v[n+1]=q_{N_v} (\rho_v,\,y_v[n])\,,\\
    y_v[0] =1\,,\quad n\geq 0\,,\quad v\in \Omega_u^+\,.
    \end{gather*}


\section{Алгоритм расчета} %4

    Для вычисления интенсивностей потоков и вероятностей блокировок в 
узле предлагается следующий алгоритм, описывающий изложенную выше 
итерационную процедуру. Введем обозначения:
$y_u^v$~--- вероятность блокировки узла для заявок, направляемых на 
линию~$v$,
\begin{gather*}
y_u^v  = 
\begin{cases}
y_u & \mbox{для}\ v\in\Omega_u^+\ \mbox{при}\\
&\mbox{полнодоступной схеме},\\
y_v & \mbox{при схеме полного распределения}\\
&\mbox{памяти};
\end{cases}
\\
q_N^v(\overline{\rho}_u^{-v}, y_u^v)  = 
\begin{cases}
q_N(\overline{\rho},y) & \mbox{для}\ v\in\Omega_u^+\ \mbox{при пол-}\\ 
&\mbox{нодоступной схеме},\\
q_{N_v}(\rho_v, y_v) & \mbox{при схеме полного}\\
&\mbox{распределения}\\ 
&\mbox{памяти},  v\in\Omega_u^+\,.
\end{cases}
\end{gather*}



Тогда уравнения~(\ref{e4aga}) и~(\ref{e16aga}) записываются в виде:
$$
y_u^v = q_N^v (\overline{\rho}^v_u, y^v_u)\,,\quad v\in \Omega_u^+\,.
$$
Для значений, вычисляемых на $k$-м шаге алгоритма, к 
обозначениям соответствующих параметров приписывается знак~$[k]$.
\pagebreak

\textbf{Шаг 0.} 
\begin{enumerate}[1.]
\item  \textit{Инициализация}. Вычисление начальных значений 
параметров~$\rho_v$, $v\in\Omega_u^+$: $\Lambda_v[0]=\Lambda_v$, 
$\rho_v[0]=\Lambda_v[0]/(\mu_v(1-B_v))$, $y_u^v[0]=1$.
\item \textit{Проверка условий существования решения}. Если для некоторой 
линии $v\in\Omega_u^+$ выполняется хотя бы одно неравенство $(c_v\mu_v(1-
B_v))/\Lambda_v[0]\;\leq$\linebreak $\leq\;1$, или $(1-B_v)/(\Lambda_v\tau_v B_v) \leq 1$, или 
$(t_v(1\;-$\linebreak $-\;B_v))/\Lambda_v[0] \leq 1$, то алгоритм заканчивает работу с 
результатом <<нагрузка не реализуема>>. Если в узле используется 
полнодоступная схема и $(c_v\mu_v(1-B_v))/\Lambda_v[0] > 1$, $(1-
B_v)/(\Lambda_v\tau_v B_v)\;>$\linebreak $>\;1$, $(t_v(1-B_v))/\Lambda_v[0] > 1$ для всех 
$v\in\Omega_u^+$, то проверяется условие~(\ref{e13aga}) для $\Lambda_v =
\Lambda_v[0]$, $v\in\Omega_u^+$, и при невыполнении этого условия алгоритм 
заканчивает работу с результатом <<нагрузка не реализуема>>.
\end{enumerate}

    При вычислении левой части неравенства~(\ref{e13aga}) рекомендуется 
использовать метод свертки Базена (см.~\cite{12aga}), позволяющий 
производить рекуррентные вычисления (подробно этот метод описан также 
в~[1, 3--6]).



\medskip
\textbf{Шаг~$k$} ($k > 0$):
\begin{enumerate}[1.]
\item \textit{Вычисление вероятностей блокировок}. Используя значения 
параметров $\overline{\rho}_u^v[k-1]$, $y_u^v[k-1]$, $v\in\Omega_u^+$, 
вычисление с помощью формул~(1)--(7) значений 
вероятностей $y[k]=1- \pi [k]$~--- в случае полнодоступной памяти, или 
$y_v[k]=1- \pi_v[k]$, $v\in\Omega_u^+$, с помощью формул~(\ref{e15aga})~--- в 
случае полного разделения памяти. При вычислении этих значений 
рекомендуется использовать метод свертки Базена.
    \item \textit{Проверка условий останова алгоритма}. Если хотя бы для 
одной $v\in\Omega_u^+$ для заданного значения точности   выполняется 
условие
$$
\fr{\vert \Lambda_v^*[k]-\Lambda_v^*[k-1]\vert}{\Lambda_v^*[k]}> \varepsilon\,,
$$
то вычисление параметров $\overline{\rho}_u^v[k]$, $v\in\Omega_u^+$, и 
переход к шагу~$k$, положив $k$ равным $k+1$, иначе алгоритм завершает 
работу. 
\end{enumerate}

    По завершении алгоритма либо выявится, что нагрузка в системе не 
реализуема, либо будут вычислены интенсивности потоков, поступающих на 
линии узла, и стационарные вероятности блокировок для заявок каждого типа. 
    
\section{Примеры расчета}

    Для проверки точности вычисления результатов с помощью 
предложенного выше алгоритма и приемлемости введенных предположений 
были проведены вычислительные эксперименты с использованием 
аналитических и имитационных моделей. Во всех рассмотренных ниже 
примерах потоки внешних заявок считаются пуассоновскими. 
В~табл.~1 приведены значения вероятности блокировок вновь 
поступивших извне заявок, полученные на основании точной формулы, 
приведенной в~\cite{4aga} для СМО типа $M\vert M\vert 1\vert 0$ с повторными 
заявками при экспоненциальном распределении интервала времени между 
повторными попытками (первая строка таблицы), алгоритма из подраздела~5 
настоящей статьи (вторая строка) и имитационной модели при постоянном 
интервале времени между повторными попытками, равном~10 (третья строка). 
Расчет табл.~1 проведен для узла с одной исходящей одноканальной 
линией при интенсивности первичного потока $\Lambda =1$ и емкости 
накопителя $N_v=1$. Таблицы~2 и~3 вычислены с помощью 
алгоритма из подраздела~5 и имитационной модели соответственно при одной 
исходящей линии с числом каналов~10.


    В табл.~\ref{t4aga} и~\ref{t5aga} приведены значения вероятности 
блокировки узла с тремя исходящими линиями канальной емкости~10 каждая 
при $\mu_v =0{,}2$, $v\in\Omega_u^+$,  вычисленные с помощью алгоритма из 
подраздела~5 и имитационной модели с интервалом повторной попытки, 
равным~10, соответственно. В табл.~\ref{t4aga} и~\ref{t5aga} знак <<--->> в 
ячейках означает, что предложенная нагрузка $\Lambda_v$, $v\in\Omega_u^+$, 
не реализуема.



В табл.~\ref{t6aga} отражены вероятности блокировки такого же узла с 
накопителем $N = 35$ при экспоненциальном распределении интервала 
времени между повторными попытками со средним значением~$\tau$. 


Результаты вычислительного эксперимента показывают, что с  увеличением 
длины интервала между повторными попытками  вероятность блокировки 
увеличивается и приближается к значению,\linebreak
вычисленному с помощью 
алгоритма из подраздела~5 (см.\ табл.~\ref{t4aga} и~\ref{t6aga}), т.\,е.\ при 
пуассоновском внешнем потоке заявок предположение, что суммарный 
входной поток заявок  является пуассоновским, вполне приемлемо для 
предварительного анализа характеристик узла (например, при  $\tau c_v\mu_v > 
10$). Как показывают табл.~1--3, вероятность блокировки 
узла существенно зависит от\linebreak 

\vspace*{6pt}
\noindent
%\begin{table*}\small %tabl1
{\small
{{\tablename~1}\ \ \small{Вероятности блокировок при одной исходящей одноканальной линии}}
%\label{t1aga}}
\vspace*{-3pt}

\begin{center}
{\tabcolsep=7.3pt
\begin{tabular}{|c|c|c|c|c|c|}
\hline
&\multicolumn{5}{c|}{$\mu$}\\
\cline{2-6}
\multicolumn{1}{|c|}{\raisebox{4pt}[0pt][0pt]{№}}&1{,}1&1{,}2&2&3&4\\
\hline
1&0,9091&0,8333&0,5000&0,3333&0,2500\\
2&0,9091&0,8333&0,5000&0,3333&0,2500\\
3&0,8867&0,8452&0,4944&0,3167&0,2396\\
\hline
\end{tabular}}
\end{center}
%\vspace*{-6pt}
%\end{table*}
}
%\bigskip
%\medskip
\addtocounter{table}{1}
\pagebreak

\end{multicols}

\renewcommand{\figurename}{\protect\bf Таблица}
%\renewcommand{\tablename}{\protect\bf Рис.}
\begin{figure*}
{\small
\begin{minipage}[t]{76mm}
%\begin{table*}\small %tabl2
\begin{center}
\Caption{Вероятности блокировок при одной исходящей многоканальной линии ($\varepsilon 
=0{,}0001$)
\label{t2aga}}
\vspace*{2ex}

\tabcolsep=6.5pt
\begin{tabular}{|c|c|c|c|c|c|}
\hline
&\multicolumn{5}{c|}{$\mu$}\\
\cline{2-6}
\multicolumn{1}{|c|}{\raisebox{4pt}[0pt][0pt]{$N$}}&0{,}11&0{,}12&0{,}2&0{,}3&0{,}4\\
\hline
10&0,4845&0,2935&0,0204&0,0017&0,0002\\
15&0,1181&0,0545&0,0006&0,0000&0,0000\\
20&0,0489&0,0167&0,0000&0,0000&0,0000\\
\hline
\end{tabular}
\end{center}
%\end{table*}
\end{minipage}
\hfill
\begin{minipage}[t]{76mm}
%\begin{table*}\small %tabl3
\begin{center}
\Caption{Вероятности блокировок при одной исходящей линии
\label{t3aga}}
\vspace*{2ex}

\tabcolsep=6.5pt
\begin{tabular}{|c|c|c|c|c|c|}
\hline
&\multicolumn{5}{c|}{$\mu_v$}\\
\cline{2-6}
\multicolumn{1}{|c|}{\raisebox{4pt}[0pt][0pt]{$N$}}&0{,}11&0{,}12&0{,}2&0{,}3&0{,}4\\
\hline
10&0,5247&0,3238&0,0219&0,0019&0,0001\\
15&0,1726&0,0912&0,0004&0,0001&0,0000\\
20&0,1180&0,0371&0,0000&0,0000&0,0000\\
\hline
\end{tabular}
\end{center}
%\end{table*}
\end{minipage}
}
\vspace*{6pt}
\end{figure*}

\renewcommand{\figurename}{\protect\bf Рис.}
\renewcommand{\tablename}{\protect\bf Таблица}
\addtocounter{table}{2}

\begin{table}\small %tabl4
\begin{center}
\parbox{400pt}{\Caption{Вероятности блокировок при трех исходящих линиях, вычисленные алгоритмом из 
подраздела~5 ($\varepsilon =0{,}0001$)
\label{t4aga}}
}

\vspace*{2ex}

\tabcolsep=8pt
\begin{tabular}{|c|c|c|c|c|c|c|c|c|c|}
\hline
&\multicolumn{9}{c|}{$\Lambda_v$}\\
\cline{2-10}
\multicolumn{1}{|c|}{\raisebox{4pt}[0pt][0pt]{$N$}}&1&1{,}1&1{,}2&1{,}3&1{,}4&1{,}5&1{,}6&1{,}7&1{,}8\\
\hline
20&0,0677&0,1423&0,2975&0,7653&---&---&---&---&---\\
25&0,0065&0,0173&0,0394&0,0827&0.1690&0.3827&---&---&---\\
30&0,0005&0,0019&0,0059&0,0155&0.0361&0.0790&0.1792&0,7259&---\\
35&0,0000&0,0002&0,0008&0,0030&0,0089&0,0234&0,0574&0,1505&---\\
40&0,0000&0,0000&0,0001&0,0005&0,0022&0,0075&0,0220&0,0617&0,2449\\
\hline
\end{tabular}
\end{center}
%\end{table}
\vspace*{6pt}
%\begin{table}\small %tabl5
\begin{center}
\parbox{400pt}{\Caption{Вероятности блокировок при трех исходящих линиях, вычисленные с помощью 
имитационной модели
\label{t5aga}}
}

\vspace*{2ex}

\tabcolsep=8pt
\begin{tabular}{|c|c|c|c|c|c|c|c|c|c|}
\hline
&\multicolumn{9}{c|}{$\Lambda_v$}\\
\cline{2-10}
\multicolumn{1}{|c|}{\raisebox{4pt}[0pt][0pt]{$N$}}&1&1{,}1&1{,}2&1{,}3&1{,}4&1{,}5&1{,}6&1{,}7&1{,}8\\
\hline
20&0,0786&0,1695&0,3549&0,7056&---&---&---&---&---\\
25&0,0069&0,0190&0,0441&0,0998&0,2266&0,4583&---&---&---\\
30&0,0007&0,0024&0,0075&0,0184&0,0462&0,1025&0,2380&0,6931&---\\
35&0,0000&0,0003&0,0007&0,0040&0,0129&0,0307&0,0890&0,2981&---\\
40&0,0000&0,0000&0,0000&0,0011&0,0041&0,0095&0,0346&0,0790&0,3179\\
\hline
\end{tabular}
\end{center}
%\end{table}
\vspace*{6pt}
%\begin{table}\small %tabl6
\begin{center}
\parbox{356pt}{\Caption{Зависимость вероятности блокировки при трех исходящих линиях, вы\-чис\-лен\-ные с 
помощью имитационной модели со случайным интервалом между повторными попытками
\label{t6aga}}
}

\vspace*{2ex}

\tabcolsep=8pt
\begin{tabular}{|c|c|c|c|c|c|c|c|c|}
\hline
&\multicolumn{8}{c|}{$\Lambda_v$}\\
\cline{2-9}
\multicolumn{1}{|c|}{\raisebox{6pt}[0pt][0pt]{$\tau$}}&1&1{,}1&1{,}2&1{,}3&1{,}4&1{,}5&1{,}6&1{,}7\\
\hline
\hphantom{9}1&0.0001&0,0001&0,0017&0,0063&0,0210&0,0733&0,1996&0,4222\\
\hphantom{9}5&0.0000&0,0002&0,0016&0,0036&0,0446&0,0159&0,1360&0,3273\\
10&0.0000&0,0002&0,0011&0,0036&0,0101&0,0430&0,0818&0,2774\\
20&0.0000&0,0003&0,0007&0,0029&0,0089&0,0257&0,0863&0,2045\\
     \hline
\end{tabular}
\end{center}
\end{table}


\begin{multicols}{2}


\noindent
числа каналов в линии при равной суммарной 
производительности. Кроме того, как видно из табл.~\ref{t5aga} и~\ref{t6aga}, 
вероятность блокировки в большей степени зависит от среднего значения 
длины интервала между повторными попытками передачи, чем от закона 
распределения длины интервала. Таким образом, предложенный в работе 
алгоритм позволяет вы\-чис\-лить с достаточной точностью вероятность 
блокировки узла, интенсивности повторных передач и предельную величину 
реализуемой нагрузки. Отметим, что полученные в данной статье результаты 
могут быть использованы для расчета нагрузок в телекоммуникационной сети с 
повторами заявок в предыдущем узле или из источника. 


{\small\frenchspacing
{%\baselineskip=10.8pt
\addcontentsline{toc}{section}{Литература}
\begin{thebibliography}{99}    
\bibitem{1aga}
\Au{Kamoun~F., Kleinrock~L.}
Analysis of shared finite storage in a computer networks node environment under 
general traffic conditions~// IEEE Trans. on Commun., 1980. Vol.~28. No.\,7. 
P.~992--1003.

\bibitem{6aga} %2
\Au{Агаларов~Я.\,М., Шоргин~С.\,Я.}
Рекуррентный метод вычисления параметров сетей связи~// Техника средств 
связи, 1986. Сер. <<Системы связи>>. Вып.~6. С.~42--46.

\bibitem{3aga}
\Au{Башарин Г.\,П., Бочаров~П.\,П., Коган~Я.\,А.}
Анализ очередей в вычислительных сетях.~--- М.: Наука, 1989. 

\bibitem{4aga}
\Au{Бочаров~П.\,П., Печинкин~А.\,В.}
Теория массового обслуживания.~--- М.: Изд-во РУДН, 1995. 

\bibitem{5aga}
\Au{Вишневский~В.\,М.} 
Теоретические основы проектирования компьютерных сетей.~--- М.: 
Техносфера, 2003. 

\bibitem{2aga} %6
\Au{Башарин Г.\,П.}
Лекции по математической теории телетрафика.~--- М.: Изд-во РУДН, 2007. 

\bibitem{7aga}
\Au{Таранцев~А.\,А.}
Инженерные методы теории массового обслуживания.~--- М.: Наука, 2007.

\bibitem{9aga} %8
\Au{D'Apice~C., De~Simone~T., Manzo~R., Rizelian~G.}
$M\vert G\vert 1\vert r$ retrial queueing system with priority service of primary 
customers and a customers-searching server~// Distributed Computer and 
Communication Networks. Stochastic Modelling and Optimization.~--- М.: 
Техносфера, 2003. P.~106--117.

\bibitem{8aga} %9
\Au{Klimenok~V.\,I., Kim~C.\,S.}
$BM\!AP$/$PH$/1 retrial system operating in random environment~// Proceedings of 
the 5th St.-Petersburg Workshop on Simulation, St.-Petersburg, June~26\,--\,July~2, 
2005.~--- St.-Petersburg: NII Chemistry St.-Petersburg University Publs., 
2005. P.~367--372.   

\bibitem{10aga}
\Au{Krishnamoorthy~A., Babu~S.}
$M\!AP\vert (PH,PH)/c$ retrial queue with selegeneration of priorities 
and non-preemptive service~// Proceedings of the 14th International Conference on 
Analytical and Stochastic Modeling Techniques and Applications, June~4--6, 
2007. Prague, Czech Republic.~--- Sbr.-Dudweiler: Digitaldruck Pirrot GmbH, 
2007. P.~70--74.

\bibitem{11aga}
\Au{Корн~Г., Корн~Т.}
Справочник по математике.~--- М.: Наука, 1974.

\label{end\stat}


\bibitem{12aga}
\Au{Buzen~J.\,P.}
Computational algorithm for closed queuing networks with exponential servers~// 
Communications ACM, 1973. Vol.~16. No.\,9. P.~527--531.
 \end{thebibliography}
}
}
\end{multicols}
 
 
   %12
\def\stat{adu}

\def\tit{АНАЛИЗ СХЕМЫ ДОСТУПА С~ПРЕРЫВАНИЕМ ПРИ НАРЕЗКЕ 
РАДИОРЕСУРСОВ СЕТИ ПЯТОГО ПОКОЛЕНИЯ$^*$}

\def\titkol{Анализ схемы доступа с~прерыванием при нарезке 
радиоресурсов сети пятого поколения}

\def\aut{К.\,И.\,Б.~Аду$^1$, Е.\,В.~Маркова$^2$, Ю.\,В.~Гайдамака$^3$, С.\,Я.~Шоргин$^4$}

\def\autkol{К.\,И.\,Б.~Аду, Е.\,В.~Маркова, Ю.\,В.~Гайдамака, С.\,Я.~Шоргин}

\titel{\tit}{\aut}{\autkol}{\titkol}

\index{Аду К.\,И.\,Б.}
\index{Маркова Е.\,В.}
\index{Гайдамака Ю.\,В.}
\index{Шоргин С.\,Я.}
\index{Adou K.\,Y.\,B.}
\index{Markova E.\,V.}
\index{Gaidamaka Yu.\,V.}
\index{Shorgin S.\,Ya.}


{\renewcommand{\thefootnote}{\fnsymbol{footnote}} \footnotetext[1]
{Исследование выполнено за счет гранта Российского научного 
фонда №\,22-79-10053, {\sf https://rscf.ru/project/22-79-10053/}.}}


\renewcommand{\thefootnote}{\arabic{footnote}}
\footnotetext[1]{Российский университет дружбы народов, \mbox{adu-k@rudn.ru}}
\footnotetext[2]{Российский университет дружбы народов, markova-ev@rudn.ru}
\footnotetext[3]{Российский университет дружбы народов; Федеральный 
исследовательский центр <<Информатика и~управление>> Российской академии наук, 
\mbox{gaydamaka-yuv@rudn.ru}}
\footnotetext[4]{Федеральный исследовательский центр <<Информатика и~управление>> Российской академии наук, 
\mbox{sshorgin@ipiran.ru}}

\vspace*{-2pt}
 
 

\Abst{Активно исследуемая в~последние годы технология <<нарезки радиоресурсов сети>> 
(NS~--- Network Slicing), основанная на представлении общей сетевой инфраструктуры 
в~виде различных настраиваемых логических сетей, называемых слайсами, предполагает 
разделение операторов мобильной сети на две группы~--- провайдеры физической 
сетевой инфраструктуры InPs (Infrastructure Providers) и~операторы мобильной 
виртуальной сети (MVNOs~--- Mobile Virtual Network Operators). Последние арендуют 
физические ресурсы InPs для создания собственных слайсов с~целью предоставления 
своим пользователям услуг с~различными требованиями к~качеству обслуживания. 
В~статье для сети с~технологией NS предложена схема доступа к~радиоресурсам сети, 
предоставляющей пользователям услуги с~гарантированной скоростью передачи данных 
(GBR~--- Guaranteed Bit Rate) и~приоритетным управлением, основанным на реализации 
механизма прерывания обслуживания пользователей. Для оценки эффективности 
предлагаемой схемы проведен сравнительный анализ ее характеристик 
с~характеристиками схемы доступа, основанной на механизме резервирования ресурсов.}

\KW{5G; нарезка сети; качество обслуживания; ключевые 
показатели эффективности; приоритетное управление; прерывание обслуживания; 
итерационный метод}

\DOI{10.14357/19922264230113} 
  
\vspace*{2pt}


\vskip 10pt plus 9pt minus 6pt

\thispagestyle{headings}

\begin{multicols}{2}

\label{st\stat}

\section{Введение}

Ввиду ограниченности спектра частотного диапазона мобильных сетей для 
предоставления пользователям услуг с~требуемым качеством обслуживания необходимо 
внедрение новых технологий. В~последние годы активно исследуется технология 
<<нарезки радиоресурсов сети>> NS, основанная на представлении 
общей сетевой инфраструктуры в~виде различных настраиваемых логических сетей, 
называемых слайсами. Одной из \mbox{важнейших} проб\-лем реализации технологии NS 
является проб\-ле\-ма эффективного распределения радиоресурсов~--- полосы 
пропускания или физических ресурсных\linebreak блоков PRB (Physical Resource Block). 
Радиоре\-сурсы должны быть распределены между несколькими слайсами в~соответствии 
с~динамически\linebreak меняющимися требованиями пользователей~--- операторов 
мобильной виртуальной сети MVNOs (Mobile Virtual Network Operators), при этом 
долж\-ны быть выполнены ключевые требования к~изоляции слайсов~\cite{3gpp.21.916, 3gpp.22.864}, 
в~част\-ности трафик, об\-слу\-жи\-ва\-емый в~рамках одного слайса, не должен оказывать негативного влияния на трафик, 
об\-слу\-жи\-ва\-емый в~других слайсах. Например, в~работах~\cite{3gpp.28.554, Yarkina2022} для реализации межслайсовой изоляции 
предложены подходы, основанные на резервировании ресурсов.

В данной работе рассмотрена модель схемы распределения ресурсов базовой 
станции (БС) соты сети между несколькими слайсами, пользователям которых 
предоставляются услуги реального времени с~GBR, например видео- и~голосовая 
телефония.
Особенность предложенной модели заключается в~управлении доступом RAC 
(Radio Admission\linebreak Control), основанном на реализации совместного доступа к~части 
имеющихся ресурсов и~введении приоритетного обслуживания. Для реализации 
межслайсовой изоляции приоритетное\linebreak управление использует механизм прерывания 
обслуживания пользователей, при этом в~\mbox{статье} раз\-работан итерационный метод 
расчета числа прерванных запросов. Чтобы оценить эффективность предлагаемой 
схемы доступа с~прерыванием обслуживания пользователей (далее~--- ПС,  схема 
с~прерыванием), проведен сравнительный анализ ее основных ключевых показателей 
эффективности (КПЭ) с~КПЭ схемы доступа, основанной на реализации механизма 
резервирования ресурсов (далее~--- РС, схема с~резервированием)~\cite{Luu2022,Rehman2022}.

\begin{table*}[b]\small %tabl1
\vspace*{-16pt}
\begin{center}
\Caption{Основные символы системы}
\label{tab:Notations}
\vspace*{2ex}

\begin{tabular}{|c|l|c|}
\hline
Обозначение & \multicolumn{1}{c}{Описание} & \tabcolsep=0pt\begin{tabular}{c}Единица \\ измерения\end{tabular}\\ 
\hline
$\mathcal{S}$ & Множество слайсов в~сис\-те\-ме, 
$\mathcal{S}\subset\mathbb{N}\setminus\{0\}, \mathbb{N}=\{0,1,2,\ldots\}$ & --- \\
\hline
$C$        & Общая емкость системы & ед.\ емкости \\
\hline
$C_s$      & Максимальная емкость слайса $s$, $s\hm\in\mathcal{S}$, 
$\sum\nolimits_{s\in\mathcal{S}} C_s \hm\geq C$ & ед.\ емкости \\[2pt]
\hline
&&\\[-10pt]
$Q_s$      & Гарантированная емкость слайса $s$, $s\hm\in\mathcal{S}$, $Q_s \hm\leq C_s$ и~$\sum\nolimits_{s\in\mathcal{S}} Q_s\hm \leq C$ & ед.\ емкости \\[2pt]
\hline
&&\\[-10pt]
$\lambda_s$ & Интенсивность поступления запросов в~слайс~$s$, $s\hm\in\mathcal{S}$, 
$\boldsymbol\lambda = 
\left(\lambda_1,\ldots,\lambda_{\lvert\mathcal{S}\rvert}\right)$ & запросов/ед.\ вр. \\[2pt]
\hline
$\mu_s^{-1}$ & \tabcolsep=0pt\begin{tabular}{l}Среднее время обслуживания одного запроса в~слайсе 
$s$, $s\hm\in\mathcal{S}$, $\boldsymbol\mu \hm= {}$\\
${}=
\left(\mu_1,\ldots,\mu_{\lvert\mathcal{S}\rvert}\right)$\end{tabular} & ед.\ вр. \\
\multicolumn{1}{|c|}{\ }&&\multicolumn{1}{c|}{\ }\\[-10pt]
\hline
$\rho_s\hm=\lambda_s/\mu_s$ & Предложенная нагрузка, создаваемая 
запросами в~слайсе~$s$, $s\hm\in\mathcal{S}$ & --- \\
\hline
$b_s$      & \tabcolsep=0pt\begin{tabular}{l}Требование к~ресурсам, необходимым для обслуживания одного запроса\\
 в~слайсе $s$, $s\hm\in\mathcal{S}$, $b_s \leq Q_s$, $\mathbf{b}\hm= 
\left(b_1,\ldots,b_{\lvert\mathcal{S}\rvert}\right)$ \end{tabular}& ед.\ емкости \\
\multicolumn{1}{|c|}{\ }&&\multicolumn{1}{c|}{\ }\\[-10pt]
\hline
&&\\[-10pt]
$\left\lfloor C_s/b_s \right\rfloor$ & \tabcolsep=0pt\begin{tabular}{l}Максимальное число запросов в~слайсе 
$s,s\in\mathcal{S}$, $\mathbf{N}^{\max} \hm= \left(\left\lfloor C_1/b_1 
\right\rfloor,\ldots\right.$\\ 
$\left.\ldots,\left\lfloor 
C_{\lvert\mathcal{S}\rvert}/b_{\lvert\mathcal{S}\rvert} \right\rfloor \right)$ \end{tabular}& 
--- \\
\multicolumn{1}{|c|}{\ }&&\multicolumn{1}{c|}{\ }\\[-10pt]
\hline
$\left\lfloor Q_s/b_s \right\rfloor$ &\tabcolsep=0pt\begin{tabular}{l} Максимальное число запросов, которое 
может быть обслужено с~ис\-поль\-зо-\\ ва\-ни\-ем гарантированной емкости слайса 
$s$, $s\hm\in\mathcal{S}$, $\mathbf{N}^{{g}}\hm = \left(\left\lfloor Q_1/b_1 
\right\rfloor,\ldots \right.$\\
$\left.\ldots ,\left\lfloor Q_{\lvert\mathcal{S}\rvert}/b_{\lvert\mathcal{S}\rvert} \right\rfloor \right)$\end{tabular} & 
--- \\
\multicolumn{1}{|c|}{\ }&&\multicolumn{1}{c|}{\ }\\[-10pt]
\hline
$n_s$      & \tabcolsep=0pt\begin{tabular}{l}Число запросов в~слайсе $s$, $s\hm\in\mathcal{S}$, когда система 
находится в~состоянии\\
 $\mathbf{n} \hm= \left(n_1,\ldots,n_{\lvert\mathcal{S}\rvert}\right)$\end{tabular} & --- \\
 \multicolumn{1}{|c|}{\ }&&\multicolumn{1}{c|}{\ }\\[-10pt]
\hline
$\mathbf{e}_s$ & Строка $s$, $s\hm\in\mathcal{S}$, единичной матрицы 
$\lvert\mathcal{S}\rvert \times  \lvert\mathcal{S}\rvert$ & --- \\
\hline
\end{tabular}
\end{center}
\end{table*}


\section{Описание модели}
%\label{sec:sysModel}

Рассмотрим работу БС соты сети, принадлежащей одному провайдеру InP и~имеющей 
емкость~$C$. Емкость БС используется несколькими мобильными операторами MVNOs, 
\mbox{предоставляющими} услуги своим пользователям, при этом под каждую услугу 
оператору MVNO выделяется так называемый <<слайс>>~--- часть от общей 
ем\-кости~$C$. Обозначим множество слайсов через $\mathcal{S}$, 
$\mathcal{S}\hm\subset\mathbb{N}\setminus\{0\}, \mathbb{N} \hm= \{0,1,2,\ldots\}$. Для 
каждого слайса $s$ определена максимальная емкость~$C_s$, причем $C_s\hm \leq C$ 
и~$\sum\nolimits_{s\in\mathcal{S}} C_s \hm\geq C$. Емкость~$C_s$ слайса $s$ включает в~себя 
гарантированную часть емкостью~$Q_s$ и~общедоступную часть $C_s \hm- Q_s$, при этом 
$Q_s \hm\leq C_s$ и~$\sum\nolimits_{s\in\mathcal{S}} Q_s \hm\leq C$. Поток запросов 
пользователей на предоставление услуги $s$ моделируется с~по\-мощью пуассоновского 
потока запросов типа~$s$ с~интенсивностью~$\lambda_s$, $s\hm\in\mathcal{S}$.

Основные обозначения представлены в~табл.~\ref{tab:Notations}. Указанные параметры 
позволяют реализовать схе-\linebreak му ПС управления доступом с~механизмом прерывания 
обслуживания пользователей.
Отличие предлагаемой схемы ПС управления доступом с~механизмом прерывания от 
классической неполнодоступной схемы с~потолками в~общей части ресурса (Sharing 
with Maximum Queue Length and Minimum Allocation)~\cite{Basharin1982,kermani1977analysis,Kamoun1980} заключается в~отсутствии 
индивидуальной зоны для запросов типа~$s$, $s\hm\in\mathcal{S}$, в~которую не 
допускаются запросы других типов. Отметим, что в~гарантированной части 
незагруженного слайса~$s$, $s\hm\in\mathcal{S}$, может начать обслуживаться принятый в~сис\-те\-му запрос произвольного типа 
$\hat{s}$, $\hat{s}\hm\in\mathcal{S}\setminus\{s\}$, при этом с~точки зрения 
слайса~$s$ запрос типа~$\hat{s}$ становится так на\-зы\-ва\-емым <<нарушителем>>. Если 
при по\-сле\-ду\-ющем поступлении запроса типа~$s$ чис\-ло об\-слу\-жи\-ва\-емых в~этом слайсе 
запросов окажется меньше гарантированного значения $\left\lfloor Q_s/b_s 
\right\rfloor$, а~объем доступного ресурса БС меньше требуемого~$b_s$, то запрос 
будет принят на обслуживание за счет прерывания обслуживания одного или 
нескольких за\-про\-сов-<<на\-ру\-ши\-те\-лей>> типа 
$\hat{s}$, $\hat{s}\hm\in\mathcal{S}\setminus\{s\}$. Для реализации механизма 
прерывания каждому слайсу присвоен приоритет в~обслуживании, что также отличает 
предложенную схему ПС от классической неполнодоступной схемы с~потолками в~общей 
час\-ти ресурса. В~предположении о~различающихся приоритетах у слайсов 
перенумеруем слайсы в~порядке убывания приоритета, т.\,е.\ высший приоритет 
в~обслуживании получат запросы, находящиеся в~слайсе с~номером~<<$1$>>, низший~--- 
в~слайсе с~номером <<$\lvert\mathcal{S}\rvert$>>. Введем век\-тор-функ\-цию 
пре\-ры\-вания
\begin{equation*}
\mathbf{z}\left(s,\mathbf{n}\right) = \left( 
z_{\hat{s}}\left(s,\mathbf{n}\right) \right) = \left( 
z_1\left(s,\mathbf{n}\right), \ldots, 
z_{\lvert\mathcal{S}\rvert}\left(s,\mathbf{n}\right) \right),\!
\end{equation*} 

\vspace*{-3pt}

\noindent
определяющую число об\-слу\-жи\-ва\-емых запросов слайса 
$\hat{s}$, $\hat{s}\hm\in\mathcal{S}$, которое необходимо прервать для приема одного 
запроса в~слайс~$s$, $s\hm\in\mathcal{S}$.

\begin{figure*} %fig1
\vspace*{1pt}
\begin{center}
   \mbox{%
\epsfxsize=161.239mm
\epsfbox{adu-1.eps}
}
\end{center}
\vspace*{-9pt}
\Caption{Блок-схема для иллюстрации правил управления доступом к~радиоресурсам сис\-те\-мы}
\label{fig:generalRACScheme}
\end{figure*}

Таким образом, при поступлении в~слайс~$s$, $s\hm\in\mathcal{S}$, запроса на 
предоставление услуги возможны три случая:\\[-14pt]
\begin{enumerate}[(1)]
\item запрос немедленно будет принят на обслуживание, если число об\-слу\-жи\-ва\-емых 
запросов в~данном слайсе меньше $\left\lfloor C_s/b_s \right\rfloor$, а~объем 
доступного ресурса БС больше или равен~$b_s$, т.\,е. 
$\left(\mathbf{n}\hm+\mathbf{e}_s\hm-\mathbf{N}^{\max}\right) \cdot \mathbf{e}_s \hm\leq 
0 \land \left(\mathbf{n}\hm + \mathbf{e}_s\right) \cdot \mathbf{b} \hm\leq C$;
\item запрос будет принят на обслуживание за счет прерывания обслуживания 
за\-про\-сов-<<на\-ру\-ши\-те\-лей>> из других слайсов 
$\hat{s}$, $\hat{s}\hm\in\mathcal{S}\setminus\{s\}$, число которых 
определяется с~по\-мощью век\-тор-функ\-ции прерывания 
$\mathbf{z}\left(s,\mathbf{n}\right)$, если число об\-слу\-жи\-ва\-емых запросов в~слайсе~$s$ 
меньше $\left\lfloor Q_s/b_s \right\rfloor$, а~объем доступных 
ресурсов БС меньше~$b_s$, т.\,е.\ $\left(\mathbf{n}\hm+\mathbf{e}_s \hm-
\mathbf{N}^{g}\right) \cdot \mathbf{e}_s\hm \leq 0 \hm\land \left(\mathbf{n} \hm+ 
\mathbf{e}_s \hm- \mathbf{z}\left(s,\mathbf{n}\right)\right) \cdot \mathbf{b} \hm\leq 
C$;
\item запрос будет заблокирован, если число об\-слу\-жи\-ва\-емых запросов в~слайсе~$s$ 
больше или рав\-но $\left\lfloor Q_s/b_s \right\rfloor$, а~объем доступного 
ресурса БС меньше~$b_s$, т.\,е.\ $\left( \mathbf{n}\hm + \mathbf{e}_s \hm-
\mathbf{N}^{g} \right) \cdot \mathbf{e}_s \hm> 0 \land \left( \mathbf{n} \hm+ 
\mathbf{e}_s \right) \cdot \mathbf{b} \hm> C$.
\end{enumerate}

Отметим, что, в~отличие от классических схем\linebreak управ\-ле\-ния доступом 
с~приоритизацией, предлагаемая схема ПС для приема запроса в~гарантированную часть 
соответствующего слайса \mbox{предусмат\-ри\-ва\-ет} необходимость прервать обслуживание 
за\-про\-са-<<на\-ру\-ши\-те\-ля>> (т.\,е.\ запроса, об\-слу\-жи\-ва\-емо\-го вне гарантированной час\-ти 
своего слайса) не только более низкого приоритета, но и~более высокого 
приоритета.


%\smallskip

\noindent
\textbf{Утверждение 1.}
При начальном условии $\mathbf{z}\left(s,\mathbf{n}\right)=\mathbf{0}$, числа 
$z_{\hat{s}}\left(s,\mathbf{n}\right)$, $\hat{s}\hm\in\mathcal{S}$, запросов, 
обслуживание которых необходимо будет прервать, можно вычислить с~по\-мощью 
рекуррентного соотношения

\vspace*{-6pt}

\noindent
\begin{multline}
\label{eq:capabilityFunc}
z_{\hat{s}}\left(s,\mathbf{n}\right) = \min
\biggl\{
R \left(\left(\mathbf{n}-\mathbf{N}^{g}\right) \cdot 
\mathbf{e}_{\hat{s}}\right), \\[-3pt]
R \left(\left\lceil \frac{\left(\mathbf{n}+\mathbf{e}_s - 
\mathbf{z}\left(s,\mathbf{n}\right) \right)\cdot\mathbf{b} - 
C}{\mathbf{e}_{\hat{s}}\cdot\mathbf{b}} \right\rceil\right)\!\!
\biggl\},\\[-3pt]
\hat{s} = \lvert\mathcal{S}\rvert,\ldots,1,
\end{multline}

\columnbreak

\noindent
где $R\left(x\right)=xH\left(x\right)$~--- функция 
Рампы\footnote{{\sf https://mathworld.wolfram.com/RampFunction.html.}}, 
а~$H\left(x\right)$~--- функция 
Хевисайда\footnote{{\sf https://mathworld.wolfram.com/HeavisideStepFunction.html.}}.

\smallskip

Отметим, что число~$z_s\left(s,\mathbf{n}\right), 
s\in\mathcal{S}$, всегда равно нулю, так как запрос не может быть принят на 
обслуживание за счет прерывания об\-слу\-жи\-ва\-емо\-\mbox{го(-ых)} за\-про\-са(-ов) в~этом же 
слайсе.

Для наглядности схему управ\-ле\-ния до\-сту\-пом можно описать с~по\-мощью блок-схе\-мы, 
пред\-став\-лен\-ной на рис.~\ref{fig:generalRACScheme}.

\vspace*{-14pt}


\section{Построение математической модели}\label{sec:mathModel}

\vspace*{-4pt}

В соответствии с~описанной в~разд.~2 схемой 
управ\-ле\-ния доступом к~радиоресурсам сети поведение системы описывает 
$\lvert\mathcal{S}\rvert$-мер\-ный случайный процесс (СП) 
$\mathbf{X}\left(t\right) \hm= \left( X_1\left(t\right),\ldots, 
X_{\lvert\mathcal{S}\rvert}\left(t\right), t\hm>0 \right)$, где 
$X_s\left(t\right)$, $s\hm\in\mathcal{S}$,~--- чис\-ло об\-слу\-жи\-ва\-емых запросов в~слайсе~$s$ 
в~момент времени~$t$ над пространством со\-сто\-яний

\noindent
\begin{equation*}
%\label{eq:StateSpace}
\Omega =
\left\{
\mathbf{n} \in \mathbb{N}^{\lvert\mathcal{S}\rvert} :
\left( \mathbf{n} - \mathbf{N}^{\max} \right) \cdot \mathbf{j} \leq 0
\land
\mathbf{n} \cdot \mathbf{b} \leq C
\right\},
\end{equation*}

\vspace*{-4pt}

\noindent
где $\mathbb{N}^{\lvert\mathcal{S}\rvert}$~--- множество всевозможных 
$\lvert\mathcal{S}\rvert$-мер\-ных век\-то\-ров-строк с~натуральными элементами, 
а~$\mathbf{j}$~--- единичная матрица размера $1\times \lvert\mathcal{S}\rvert$.

Схема соответствующей мультисервисной сис\-те\-мы массового 
обслуживания (СМО)~\cite{Basharin2013} изображена на~рис.~2. 





Для дальнейшего анализа модели введем следующие подмножества пространства 
состояний~$\Omega$ системы:
\begin{enumerate}[(1)]
\item $\Omega_s^{\mathrm{dad}},s\in\mathcal{S}$,~--- множество состояний системы, в~которых поступающий в~слайс~$s$ запрос немедленно будет принят на 
обслуживание:
\end{enumerate}

{ \begin{center}  %fig2
 \vspace*{-3pt}
    \mbox{%
\epsfxsize=77.767mm
\epsfbox{adu-2.eps}
}


\vspace*{3pt}


\noindent
{{\figurename~2}\ \ \small{Схема СМО
}}
\end{center}
}

\vspace*{-12pt}

\addtocounter{figure}{1}

\begin{enumerate}[(1)]
\setcounter{enumi}{1}
\item[\,]

\noindent
\begin{multline}
\label{eq:SetAdmission}
\Omega_s^{\mathrm{dad}} =
\left\{
\mathbf{n} \in \Omega: \right.\\ \left.
\left(\mathbf{n}-\mathbf{N}^{\max}\right) \cdot \mathbf{e}_s < 0
\land
\left(\mathbf{n} + \mathbf{e}_s \right) \cdot \mathbf{b} \leq C
\right\};
\end{multline}
\item $\Omega_s^{\mathrm{vpad}}$, $s\hm\in\mathcal{S}$,~--- множество состояний 
системы, в~которых поступающий в~слайс~$s$ запрос будет принят на обслуживание 
за счет прерывания об\-слу\-жи\-ва\-емо\-го(-ых) за\-про\-са(-ов) в~слайсах 
$\hat{s}$, $\hat{s}\hm\in\mathcal{S}\setminus\{s\}$:
\begin{multline}
\label{eq:SetPreemptionCapability}
\Omega_s^{\mathrm{vpad}} =
\left\{
\mathbf{n} \in \Omega: \right.\\ \left.
\left(\mathbf{n}-\mathbf{N}^{{g}}\right) \cdot \mathbf{e}_s < 0
\land
\left(\mathbf{n} + \mathbf{e}_s \right) \cdot \mathbf{b} > C
\right\};
\end{multline}
\item $\Omega_s^{\mathrm{block}}$, $s\hm\in\mathcal{S}$,~--- множество состояний 
системы, в~которых поступающий в~слайс~$s$ запрос будет заблокирован:
\begin{subequations}
\label{eq:SetBlocking}
\begin{multline}
\Omega_s^{\mathrm{block}} =
\left\{
\mathbf{n} \in \Omega: \right.\\ \left.
\left(\mathbf{n}-\mathbf{N}^{{g}}\right) \cdot \mathbf{e}_s \geq 0
\land
\left(\mathbf{n} + \mathbf{e}_s \right) \cdot \mathbf{b} > C
\right\};
\end{multline}
или
\begin{equation}
\Omega_s^{\mathrm{block}} = \Omega \setminus \left( \Omega_s^{\mathrm{dad}} \cup 
\Omega_s^{\mathrm{vpad} }\right),
\end{equation}
где $\Omega_s^{\mathrm{dad}} \cup \Omega_s^{\mathrm{vpad}}$~--- множество состояний 
системы, в~которых поступающий в~слайс~$s$ запрос будет принят на обслуживание.
\end{subequations}
\end{enumerate}



Пример для иллюстрации данных подмножеств для $\mathcal{S} = 
\{1,2\}$ пред\-став\-лен в~разд.~4.1 (см.\ рис.~4).

Подмножества~(\ref{eq:SetAdmission})--(\ref{eq:SetBlocking}) позволяют разделить все состояния 
$\mathbf{n},\mathbf{n}\hm\in\Omega$, системы на шесть групп:
\begin{enumerate}[(1)]
\item $\bigcap\nolimits_{s\in\mathcal{S}}\Omega_s^{\mathrm{dad}}$~--- множество 
состояний сис\-те\-мы, в~которых любой поступающий в~систему запрос 
немедленно будет принят на обслуживание;
\item $\Omega_s^{\mathrm{dad}}\setminus 
\bigcup\nolimits_{\hat{s}\hm\in\mathcal{S}\setminus\{s\}} 
\Omega_{\hat{s}}^{\mathrm{dad}}$, $s\hm\in\mathcal{S}$,~--- множество состояний 
сис\-те\-мы, в~которых только по\-сту\-па-\linebreak\vspace*{-16pt}
\end{enumerate}

\columnbreak

\begin{enumerate}[(1)]
\setcounter{enumi}{2}
\item[\,] ющий в~слайс~$s$ запрос немедленно будет 
принят на обслуживание;
\item $\Omega_s^{\mathrm{vpad}} \cap 
\bigcup\nolimits_{\hat{s}\hm\in\mathcal{S}\setminus\{s\}} 
\Omega_{\hat{s}}^{\mathrm{vpad}}$, $s\hm\in\mathcal{S}$,~--- множество состояний 
системы, в~которых поступающие в~слайсы~$s$ и~$\hat{s}$,
$\hat{s}\hm\in\mathcal{S}\setminus\{s\}$, запросы будут приняты на 
обслуживание за счет прерывания об\-слу\-жи\-ва\-емо\-го(-ых) за\-про\-са(-ов) в~слайсах~$\tilde{s}$,
$\tilde{s}\hm\in\mathcal{S}\setminus\{s,\hat{s}\}$;
\item $\Omega_s^{\mathrm{vpad}} \setminus 
\bigcup\nolimits_{\hat{s}\hm\in\mathcal{S}\setminus\{s\}} 
\Omega_{\hat{s}}^{\mathrm{vpad}}$, $s\hm\in\mathcal{S}$,~--- множество состояний 
системы, в~которых только поступающий в~слайс~$s$ запрос будет принят на 
обслуживание за счет прерывания одного или нескольких об\-слу\-жи\-ва\-емых запросов 
в~слайсе~$\hat{s}$, $\hat{s}\hm\in\mathcal{S}\setminus\{s\}$;
\item $\bigcap\nolimits_{s\in\mathcal{S}}\Omega_s^{\mathrm{block}}$~--- множество 
состояний системы, в~которых любой поступающий в~систему запрос будет 
заблокирован;
\item $\Omega_s^{\mathrm{block}}\setminus 
\bigcup\nolimits_{\hat{s}\hm\in\mathcal{S}\setminus\{s\}} 
\Omega_{\hat{s}}^{\mathrm{block}}$, $s\hm\in\mathcal{S}$,~--- множество состояний 
системы, в~которых только поступающий в~слайс~$s$ запрос будет 
заблокирован.
\end{enumerate}


Диаграмма интенсивностей переходов для состояния системы 
$\mathbf{n},\mathbf{n}\hm\in\Omega$, имеет вид, представленный 
на рис.~3.




Функция управления доступом к~ресурсам сис\-те\-мы определяется следующим образом:
\begin{multline*}
f_s\left(\mathbf{n}\right) =
\begin{cases}
1,  &\ \mbox{если}\ 
\mathbf{n}\in\left(\Omega_s^{\mathrm{dad}}\cup\Omega_s^{\mathrm{vpad}}\right);\\
0           &\ \mbox{в\ противном\ случае},\ 
\mathbf{n}\in\Omega_s^{\mathrm{block}}.
\end{cases}\\
s\in\mathcal{S}.
\end{multline*}

Согласно диаграмме интенсивностей переходов рис.~3, 
рассматриваемый СП описывается сле\-ду\-ющей сис\-те\-мой урав\-не\-ний рав\-но\-ве\-сия:

{ \begin{center}  %fig3
 \vspace*{6pt}
    \mbox{%
\epsfxsize=76.958mm
\epsfbox{adu-3.eps}
}

\end{center}



\noindent
{{\figurename~3}\ \ \small{Диаграмма интенсивностей переходов для состояния 
системы~$\mathbf{n}$, $\mathbf{n}\hm\in\Omega$
}}}

%\vspace*{6pt}

\addtocounter{figure}{1}

\noindent
\begin{multline*}
%\label{eq:EquilibriumEquationsSystem}
\!\!\!\! P\left(\mathbf{n}\right)\! \left( \boldsymbol\lambda \cdot 
\sum_{s\in\mathcal{S}}{\left( I_{\Omega_s^{\mathrm{dad}}}\left(\mathbf{n}\right) + 
I_{\Omega_s^{\mathrm{vpad}}}\left(\mathbf{n}\right) \right) \mathbf{e}_s + 
\mathbf{n}\cdot\boldsymbol\mu} \right) = {}\hspace*{-2.69116pt}\\
{}= \boldsymbol\lambda \cdot \sum_{s\in\mathcal{S}}
\biggl(
P\left(\mathbf{n}-\mathbf{e}_s\right)\, 
I_{\Omega_s^{\mathrm{dad}}}\left(\mathbf{n}-\mathbf{e}_s\right) + {}\hspace*{-2.69116pt}\\
{}+ P\left(\mathbf{n}-\mathbf{e}_s+\mathbf{z}\left(s,\mathbf{n}\right)\right)\, 
I_{\Omega_s^{\mathrm{vpad}}}\left(\mathbf{n}-
\mathbf{e}_s+\mathbf{z}\left(s,\mathbf{n}\right)\right)\!\!
\biggl)
\mathbf{e}_s +{}\hspace*{-2.69116pt}\\
{}+ \boldsymbol\mu \cdot 
\sum_{s\in\mathcal{S}}{\left(P\left(\mathbf{n}+\mathbf{e}_s\right)\, 
I_{\Omega_s^{\mathrm{dad}}}\left(\mathbf{n}\right)\, 
\left(\mathbf{n}+\mathbf{e}_s\right)\cdot\mathbf{e}_s\right)\mathbf{e}_s},
\end{multline*}
где $P\left(\mathbf{n}\in\Omega\right)$~--- стационарная вероятность того, что 
система находится в~состоянии $\mathbf{n}$, а $I_{\circ}\left(\ast\right)$~--- 
функ\-ция-ин\-ди\-ка\-тор\footnote{{\sf https://mathworld.wolfram.com/CharacteristicFunction.html.}}.

В связи с~реализацией механизма прерывания обслуживания запросов СП 
$\mathbf{X}\left(t\right)_{t>0}$, опи\-сы\-ва\-ющий рас\-смат\-ри\-ва\-емую сис\-те\-му, не 
является обратимым марковским процессом. В~этом случае для вы\-чис\-ле\-ния 
стационарного распределения вероятностей со\-сто\-яний сис\-те\-мы 
$\mathbf{P}\hm=\left(P\left(\mathbf{n}\right)\right)_{\mathbf{n}\hm\in\Omega}$~--- 
век\-то\-ра-столб\-ца размера $\lvert\Omega\rvert$~--- может быть применен один из 
численных методов, например итерационный метод~\cite{Zhou2022,Stepanov}:
\begin{equation*}
%\label{eq:InfinitesimalGeneratorMarkov}
\mathbf{A}^\top\, \mathbf{P} = \mathbf{0}\,,\quad \mathbf{P}\cdot \mathbf{j} = 1\,,
\end{equation*}
где $\mathbf{A}$~--- инфинитезимальная матрица размера $\lvert\Omega\rvert^2$, 
элементы $A\left(\mathbf{n},\hat{\mathbf{n}}\right)$, $\mathbf{n}\hm\in\Omega$, 
$\hat{\mathbf{n}}\hm\in\Omega$, которой определяются следующим образом:

\noindent
\begin{subequations}
%\label{eq:SolutionEquilibrium}
при $\mathbf{n}\neq\hat{\mathbf{n}}$
\begin{multline*}
A\left(\mathbf{n},\hat{\mathbf{n}}\right) = \\
\begin{cases}
\boldsymbol\lambda \cdot \mathbf{e}_s,  &\!\!\!\!\!\!\!\!  \mbox{если}\ 
\hat{\mathbf{n}}=\mathbf{n}+\mathbf{e}_s,\, 
\mathbf{n}\in\Omega_s^{\mathrm{dad}},\\
&\!\!\!\!\!\!\!\! \mbox{или}\ \hat{\mathbf{n}}=\mathbf{n}+\mathbf{e}_s-
\mathbf{z}\left(s,\mathbf{n}\right),\, \mathbf{n}\in\Omega_s^{\mathrm{vpad}};\\
\left(\mathbf{n}\odot\boldsymbol\mu\right)\cdot\mathbf{e}_s,    &\!\!\! \mbox{если}\ 
\hat{\mathbf{n}}=\mathbf{n}-\mathbf{e}_s,\, 
\hat{\mathbf{n}}\in\Omega_s^{\mathrm{dad}};\\
0           & \!\!\!\!\!\!\!\! \mbox{в противном случае},\, 
\hat{\mathbf{n}}\in\Omega\setminus\{\mathbf{n}\},
\end{cases}\\
s=1,\ldots,\lvert\mathcal{S}\rvert;
\end{multline*}
при $\mathbf{n}=\hat{\mathbf{n}}$
\begin{equation*}
A\left(\mathbf{n},\mathbf{n}\right) = -
\sum_{\hat{\mathbf{n}}\in\Omega\setminus\{\mathbf{n}\}}{A\left(\mathbf{n},\hat{\mathbf{n}}\right)}.
\end{equation*}
\end{subequations}

Рассчитав стационарное распределение вероятностей состояний системы, можно 
вычислить следующие КПЭ системы:
\begin{itemize}
\item среднее число запросов, об\-слу\-жи\-ва\-емых в~сис\-те\-ме
\setcounter{equation}{4}
\begin{equation}
\label{eq:meanN}
N = \sum\limits_{\mathbf{n}\in\Omega} P\left(\mathbf{n}\right)\: \mathbf{n} \cdot 
\mathbf{j}\,;
\end{equation}
\item вероятность блокировки запроса, поступающего в~систему
\begin{equation}
\label{eq:Pblock}
P^{\mathrm{block}} = \sum\limits_{\mathbf{n}\in \mathcal{B}}P(\mathbf{n}); \quad 
\mathcal{B} = \bigcap_{s=1}^{\lvert\mathcal{S}\rvert}\Omega_s^{\mathrm{block}}\,;
\end{equation}
\item средний занятый ресурс,
\begin{equation}\label{eq:meanK}
K = \sum\limits_{\mathbf{n}\in\Omega} P\left(\mathbf{n}\right)\: \mathbf{n} \cdot 
\mathbf{b}\,.
\end{equation}
\end{itemize}

\vspace*{-24pt}


\section{Примеры расчета вектор-функции прерывания 
обслуживания}
%\label{sec:exampleModels}

\vspace*{-12pt}

Проиллюстрируем работу итерационного метода расчета числа и~типа запросов-<<на\-ру\-ши\-те\-лей>>, 
которые должны быть прерваны для приема запроса, поступающего 
в~гарантированную часть своего слайса, на примере моделей с~двумя и~тремя 
слайсами, для которых вычислим век\-тор-функ\-цию прерывания 
обслуживания~\eqref{eq:capabilityFunc} запросов.

\vspace*{-12pt}


\subsection{Модель сети с~двумя слайсами}
%\label{subsec:2Dmodel}

\vspace*{-12pt}

Для случая $\mathcal{S}\hm=\{1,2\}$ с~учетом основных подмножеств системы~\eqref{eq:SetAdmission}--\eqref{eq:SetBlocking} 
пространство состояний модели 
имеет вид, изображенный на рис.~4.

Приведем пример расчета век\-тор-функ\-ции прерывания 
обслуживания~\eqref{eq:capabilityFunc} для приема запроса\linebreak\vspace*{-12pt}

\begin{table*}\small %tabl2 
%\vspace*{-12pt}
\begin{center}
\Caption{Исходные данные для сравнительного анализа~\cite{schoolar2019}}
\label{tab:SlicesAndCharacteristics}
\vspace*{2ex}

\tabcolsep=4pt
\begin{tabular}{|c|c|c|c|c|c|c|}
\hline
  & \multicolumn{3}{c|}{СХЕМА С РЕЗЕРВИРОВАНИЕМ (РС)} & 
\multicolumn{3}{c|}{СХЕМА С ПРЕРЫВАНИЕМ (ПС)}\\
\cline{2-7}
\multicolumn{1}{|c|}{\raisebox{6pt}[0pt][0pt]{СЛАЙС/УСЛУГА }}& Параметр  & Значение &  \tabcolsep=0pt\begin{tabular}{c}Единица\\ измерения\end{tabular}  & Параметр  & Значение & \tabcolsep=0pt\begin{tabular}{c}Единица\\ измерения\end{tabular} \\ 
\hline
\multicolumn{1}{|c|}{\raisebox{-18pt}[0pt][0pt]{1/3K Cloud VR (Game)}} & $C_1$ & 1,0 & Гбит/с & $C_1$ & 1,25 & Гбит/с \\
                       & $Q_1$ & 1,0 & Гбит/с & $Q_1$ & 1,0  & Гбит/с \\
                       & $b_1$ & 0,1 & Гбит/с & $b_1$ & 0,1  & Гбит/с \\
                       & $\mu_1^{-1}$ & 3600 & с~& $\mu_1^{-1}$ & 3600  & с~\\
\hline
\multicolumn{1}{|c|}{\raisebox{-18pt}[0pt][0pt]{2/4K Live News Pushing (30~fps)}} & $C_2$ & 0,5 & Гбит/с & $C_2$ & 0,65 & Гбит/с 
\\
                                 & $Q_2$ & 0,5 & Гбит/с & $Q_2$ & 0,5  & Гбит/с 
\\
                                 & $b_2$ & 0,04 & Гбит/с & $b_2$ & 0,04  &  Гбит/с \\
                                 & $\mu_2^{-1}$ & 1200 & с~& $\mu_2^{-1}$ & 1200  
& с~\\
\hline
\multicolumn{1}{|c|}{\raisebox{-18pt}[0pt][0pt]{3/4K On-Demand Video}} & $C_3$ & 0,5 & Гбит/с & $C_3$ & 0,65 & Гбит/с \\
                       & $Q_3$ & 0,5 & Гбит/с & $Q_3$ & 0,5  & Гбит/с \\
                       & $b_3$ & 0,03 & Гбит/с & $b_3$ & 0,03  & Гбит/с \\
                       & $\mu_3^{-1}$ & 1800 & с~& $\mu_3^{-1}$ & 1800  & с~\\
\hline
\multicolumn{1}{|c|}{\raisebox{-18pt}[0pt][0pt]{4/4K Live Broadcast Pushing (50fps)}} & $C_4$ & 0,5 & Гбит/с & $C_4$ & 0,65 & Гбит/с \\
                                      & $Q_4$ & 0,5 & Гбит/с & $Q_4$ & 0,5  & 
Гбит/с \\
                                      & $b_4$ & 0,063 & Гбит/с & $b_4$ & 0,063  
& Гбит/с \\
                                      & $\mu_4^{-1}$ & 5400 & с~& $\mu_4^{-1}$ & 
5400  & с\\
\hline
\end{tabular}
\end{center}
\vspace*{5pt}

\begin{center}
\begin{tabular}{|c|c|c|c|c|c|c|c|}
%\hline
\hline
  &  & \multicolumn{2}{c|}{СЦЕН.\ 1} & \multicolumn{2}{c|}{СЦЕН.\ 2} & 
\multicolumn{2}{c|}{СЦЕН.\ 3} \\ 
\cline{3-8}
\multicolumn{1}{|c|}{\raisebox{6pt}[0pt][0pt]{$\rho$}} &
\multicolumn{1}{c|}{\raisebox{6pt}[0pt][0pt]{ $\lambda_s$}} & СЛАЙСЫ & $C$ & СЛАЙСЫ  & $C$ & СЛАЙСЫ  & $C$ \\ 
\hline
от 1 до 25 & $\rho\mu_s$ запросов/с & 1 и~2 & 1,5 Гбит/с & 1, 2 и~3 & 2,0 Гбит/с & 1, 2, 3 и~4 & 2,5 Гбит/с \\
\hline
\end{tabular}
\end{center}
%\vspace*{-12pt}
\end{table*}









{ \begin{center}  %fig4
 \vspace*{6pt}
   \mbox{%
\epsfxsize=76.222mm
\epsfbox{adu-4.eps}
}


\vspace*{12pt}

{\small \begin{tabular}{|c|c|c|c|}
\hline
(a) & $\Omega_1^{\mathrm{dad}}\cap \Omega_2^{\mathrm{dad}}$ & (d) & $\Omega_1^{\mathrm{vpad}}$\\
(b) & $\Omega_1^{\mathrm{dad}}\backslash \Omega_2^{\mathrm{dad}}$ & (e) & $\Omega_2^{\mathrm{vpad}}$\\
(c) & $\Omega_2^{\mathrm{dad}}\backslash \Omega_1^{\mathrm{dad}}$ & (f) & $\Omega_1^{\mathrm{block}}\cap \Omega_2^{\mathrm{block}}$\\
\hline
\end{tabular}}
\end{center}

%\vspace*{6pt}

\noindent
{{\figurename~4}\ \ \small{Пространство состояний модели с~двумя слайсами с~учетом основных подмножеств системы
}}}

\vspace*{9pt}

\addtocounter{figure}{1}

\pagebreak




\noindent
 в~первый слайс
$\mathbf{z}\left(1,\mathbf{n}\right)\hm=\left(z_1\left(1,\mathbf{n}\right), 
z_2\left(1,\mathbf{n}\right)\right)$.
Рас\-смот\-рим со\-сто\-яние сис\-те\-мы
$$
\mathbf{n} = \left(\left\lfloor \left\lfloor C/b_1 \right\rfloor \left(1-
\fr{\left\lfloor C_2/b_2 \right\rfloor}{\left\lfloor C/b_2 \right\rfloor} 
\right) \right\rfloor, \left\lfloor C_2/b_2 \right\rfloor \right),
$$
%\linebreak\vspace*{-12pt}
которое согласно диаграмме на рис.~4 принадлежит множеству 
$\Omega_1^{\mathrm{vpad}}$, т.\,е.\ множеству со\-сто\-яний сис\-те\-мы, в~которых 
поступающий в~первый слайс запрос
 будет принят на обслуживание за счет 
прерывания об\-слу\-жи\-ва\-емо\-го(-ых) за\-про\-са(-ов) во втором \mbox{слайсе}.

Для исходных данных <<СЦЕН.~1>>, пред\-став\-лен\-ных 
в~табл.~2, получим $\mathbf{n}\hm = \left(8,16\right)$, 
$\mathbf{n}\hm\in\Omega_1^{\mathrm{vpad}}$.


Рассчитаем число $z_2\left(1,\mathbf{n}\right)$ об\-слу\-жи\-ва\-емых запросов 
второго слайса, которое необходимо прервать для приема одного запроса в~первый 
слайс. Воспользуемся начальным условием 
$\mathbf{z}\left(1,\mathbf{n}\right)\hm=\left(0,0\right)$, получим

\noindent
\begin{multline*}
z_2\left(1,\mathbf{n}\right) = \min
\biggl\{
R \left(\left(\mathbf{n}-\mathbf{N}^{{g}}\right) \cdot \mathbf{e}_2\right), 
\\
R \left(\left\lceil \fr{\left(\mathbf{n}+\mathbf{e}_1 \right)\cdot\mathbf{b} - 
C}{\mathbf{e}_2\cdot\mathbf{b}} \right\rceil\right)
\biggl\} =
\min \left\{ R \left(4\right),R \left(1\right) \right\} = {}\\
{}= \min \left\{4,1\right\} = 1.
\end{multline*}

Перейдем к~расчету числа $z_1\left(1,\mathbf{n}\right)$ об\-слу\-жи\-ва\-емых 
запросов первого слайса, которое необходимо прервать для приема одного запроса в~первый слайс. Очевидно, что число $z_1\left(1,\mathbf{n}\right)$ должно 
быть равно~$0$, так как запрос не может быть принят в~слайс на обслуживание за 
счет прерывания об\-слу\-жи\-ва\-емо\-\mbox{го(-ых)} за\-про\-са(-ов) в~этом же слайсе. 
Воспользуемся текущим значением 
$\mathbf{z}\left(1,\mathbf{n}\right)\hm=\left(0,z_2\left(1,\mathbf{n}\right)\right)
=\left(0,1\right)$, получим

\vspace*{-6pt}

\noindent
\begin{multline*}
z_1\left(1,\mathbf{n}\right) =
\min
\biggl\{
R \left(\left(\mathbf{n}-\mathbf{N}^{{g}}\right) \cdot \mathbf{e}_1\right), 
\\
R \left(\left\lceil \frac{\left(\mathbf{n}+\mathbf{e}_1 - 
\mathbf{z}\left(1,\mathbf{n}\right) \right)\cdot\mathbf{b} - 
C}{\mathbf{e}_1\cdot\mathbf{b}} \right\rceil\right)
\biggl\} = {}\\
{}= \min \left\{R \left(-2\right),R \left(0\right) \right\} = \min 
\left\{0,0\right\} = 0\,.
\end{multline*}

\vspace*{-4pt}

Таким образом, в~состоянии системы $\mathbf{n}\hm=\left(8,16\right)$, 
$\mathbf{n}\hm\in\Omega_1^{\mathrm{vpad}}$, век\-тор-функ\-ция прерывания 
$\mathbf{z}\left(1,\mathbf{n}\right)\hm=\left(z_1\left(1,\mathbf{n}\right),z_2\left
(1,\mathbf{n}\right)\right)\hm=\left(0,1\right)$,
т.\,е.\ поступающий в~первый слайс запрос будет принят на обслуживание за счет 
прерывания одного запроса, об\-слу\-жи\-ва\-емо\-го во втором слайсе.


\subsection{Модель сети с~тремя слайсами}

Перейдем к~трехмерному случаю $\mathcal{S}=\{1,2,3\}$. Рассмотрим пример расчета 
век\-тор-функ\-ции прерывания обслуживания~\eqref{eq:capabilityFunc} для приема 
запроса во второй слайс 
$\mathbf{z}\left(2,\mathbf{n}\right)\hm=\left(z_1\left(2,\mathbf{n}\right), 
z_2\left(2,\mathbf{n}\right), z_3\left(2,\mathbf{n}\right)\right)$, 
$\mathbf{n}\hm\in\Omega_2^{\mathrm{vpad}}$.

Сведем исходные данные для примера в~табл.~3.

\pagebreak

%\begin{table*}\small %tabl3
\noindent
{{\tablename~3}\ \ \small{Исходные данные для примера с~тремя слайсами
}}
%\label{tab:NumExampleParams3D}

\vspace*{3pt}

\begin{center}
{\small 
\tabcolsep=11pt
\begin{tabular}{|c|c|c|c|}
\hline
$C$, Мбит/с & $\mathbf{b}$, Мбит/с& $\mathbf{n}$ & $\mathbf{N}^g$\\
\hline
13& (1,3,1) & (6,1,4)& (2,2,2)\\
\hline
\end{tabular}
}
\end{center}
%\end{table*}

\vspace*{6pt}

\addtocounter{table}{1}


Рассчитаем число $z_3\left(2,\mathbf{n}\right)$ об\-слу\-жи\-ва\-емых запросов третьего 
слайса~--- слайса с~низшим приоритетом, которое необходимо прервать для приема 
одного запроса во второй слайс. Воспользуемся начальным условием 
$\mathbf{z}\left(2,\mathbf{n}\right)\hm=\left(0,0,0\right)$, получим
\begin{multline*}
z_3\left(2,\mathbf{n}\right) = \min
\biggl\{
R \left(\left(\mathbf{n}-\mathbf{N}^{{g}}\right) \cdot \mathbf{e}_3\right), 
\\
R \left(\left\lceil \fr{\left(\mathbf{n}+\mathbf{e}_2 \right)\cdot\mathbf{b} - 
C}{\mathbf{e}_3\cdot\mathbf{b}} \right\rceil\right)
\biggl\} =
\min \left\{ R \left(2\right),R \left(3\right) \right\} ={} \\
{}= \min \left\{2,3\right\} = 2.
\end{multline*}

Покажем, что число $z_2\left(2,\mathbf{n}\right)$ об\-слу\-жи\-ва\-емых запросов второго 
слайса, которое необходимо прервать для приема одного запроса во второй слайс, 
равно~$0$. Воспользуемся текущим значением функции 
$\mathbf{z}\left(2,\mathbf{n}\right)\hm=\left(0,0,z_3\left(2,\mathbf{n}\right)\right)=\left(0,0,2\right)$, получим
\begin{multline*}
z_2\left(2,\mathbf{n}\right) = \min
\biggl\{
R \left(\left(\mathbf{n}-\mathbf{N}^{{g}}\right) \cdot \mathbf{e}_2\right), 
\\
R \left(\left\lceil \fr{\left(\mathbf{n}+\mathbf{e}_2 - 
\mathbf{z}\left(2,\mathbf{n}\right) \right)\cdot\mathbf{b} - 
C}{\mathbf{e}_2\cdot\mathbf{b}} \right\rceil\right)
\biggl\} ={} \\
{}= \min \left\{R \left(-1\right),R \left(1\right) \right\} = \min 
\left\{0,1\right\} = 0.
\end{multline*}

Согласно схеме управления доступом ПС при исчерпании возможности освободить 
ресурс за счет прерывания запросов-<<нарушителей>> более низкого приоритета для 
приема запроса в~гарантированную часть второго слайса должно быть прервано 
обслуживание за\-про\-сов-<<на\-ру\-ши\-те\-лей>> более высокого приоритета. Так как число 
об\-слу\-жи\-ва\-емых запросов первого слайса превышает гарантированное значение, для 
приема одного запроса во второй слайс должно быть прервано обслуживание 
$z_1\left(2,\mathbf{n}\right)$ запросов первого слайса. С~учетом текущего 
значения век\-тор-функ\-ции прерывания 
$$
\mathbf{z}\left(2,\mathbf{n}\right)=\left(0, 
z_2\left(2,\mathbf{n}\right), 
z_3\left(2,\mathbf{n}\right)\right)\hm=\left(0,0,2\right)
$$ 
получим
\begin{multline*}
z_1\left(2,\mathbf{n}\right) = \min
\biggl\{
R \left(\left(\mathbf{n}-\mathbf{N}^{{g}}\right) \cdot \mathbf{e}_1\right), 
\\
R \left(\left\lceil \fr{\left(\mathbf{n}+\mathbf{e}_2 - 
\mathbf{z}\left(2,\mathbf{n}\right) \right)\cdot\mathbf{b} - 
C}{\mathbf{e}_1\cdot\mathbf{b}} \right\rceil\right)
\biggl\} ={} \\
{}= \min \left\{R \left(4\right),R \left(1\right) \right\} = \min 
\left\{4,1\right\} = 1.
\end{multline*}
Таким образом, в~состоянии системы $\mathbf{n}\hm=\left(6,1,4\right)$, 
$\mathbf{n}\hm\in\Omega_2^{\mathrm{vpad}}$, век\-тор-функ\-ция прерывания обслуживания 
запросов имеет вид 
$$
\mathbf{z}\left(2,\mathbf{n}\right)=\left(z_1\left(2,\mathbf{n}\right), 
z_2\left(2,\mathbf{n}\right), 
z_2\left(2,\mathbf{n}\right)\right)=\left(1,0,2\right),
$$
 т.\,е.\ поступающий во 
второй слайс запрос будет принят на обслуживание за счет прерывания двух 
запросов, об\-слу\-жи\-ва\-емых в~третьем слайсе, и~одного~--- в~первом.

Далее перейдем к~анализу основных КПЭ сис\-те\-мы, описанных в~разд.~3.



\section{Численный анализ}
%\label{sec:numericalanalysis}

Для анализа эффективности предложенной в~работе схемы доступа ПС к~радиоресурсам сети, основанной на реализации механизма прерывания обслуживания 
пользователей, проведем \mbox{сравнительный} анализ ее основных КПЭ с~КПЭ известной 
схемы доступа РС, основанной на реализации механизма резервирования ресурсов. 
Сведем исходные данные для численного анализа 
в~табл.~\ref{tab:SlicesAndCharacteristics}.
Рас\-смот\-рим зависимость среднего числа запросов,\linebreak об\-слу\-жи\-ва\-емых в~сис\-те\-ме,~\eqref{eq:meanN}, вероятности блокировки запроса любого типа, 
поступающего в~сис\-те\-му,~\eqref{eq:Pblock} (вероятности блокировки сис\-те\-мы) и~среднего чис\-ла занятых ресурсов в~сис\-те\-ме~\eqref{eq:meanK} от \mbox{интенсивности} 
предложенной нагрузки~$\rho$, созда\-ва\-емой в~каж\-дом слайсе. Положим $\rho_s=\rho, 
s\in\mathcal{S}$. Результаты сравнительного анализа схем ПС и~РС представлены 
на~рис.~5 и~6.




На рис.~5,\,\textit{а} проиллюстрировано поведение среднего числа запросов~$N$, 
об\-слу\-жи\-ва\-емых в~сис\-те\-ме, в~за\-ви\-си\-мости от интенсивности предложенной нагрузки~$\rho$ для трех сценариев~--- с~двумя, тремя и~четырьмя слайсами. По графику 
видно, что чем больше слайсов в~сети, тем выше сред\-нее чис\-ло об\-слу\-жи\-ва\-емых 
запросов и~тем эф\-фек\-тив\-нее схема ПС, в~част\-ности (см.\ рис.~6,\,\textit{а}) 
для <<СЦЕН.~1>> эф\-фек\-тив\-ность схемы ПС может превышать эф\-фек\-тив\-ность схемы РС в~1,06~раза, а~для <<СЦЕН.~3>>~--- в~1,12 раза.



С ростом интенсивности предложенной нагрузки~$\rho$ вероятность блокировки 
системы~$P^{\mathrm{block}}$ увеличивается для всех рассматриваемых сценариев 
(см.\ рис.~5,\,\textit{б}), причем она минимальна для сценария с~наибольшем числом 
слайсов (<<СЦЕН.~3>>). Однако на рис.~5,\,\textit{б} и~6,\,\textit{б} 
видно, что схема ПС эффективнее схемы РС только в~диапазоне небольших нагрузок 
на сис\-те\-му, в~част\-ности (см.\ рис.~6,\,\textit{б}) для <<СЦЕН.~2>> 
и~<<СЦЕН.~3>>~--- диапазон от~1 до~4,43.



Анализ  рис.~5,\,\textit{в} и~6,\,\textit{в} показывает, что сис\-те\-ма более 
эффективно использует ресурсы при применении схемы ПС. В~част\-ности 
(см.\ рис.~6,\,\textit{в}), сред\-нее чис\-ло ресурсов, занятых в~сис\-те\-ме, при 
использовании схемы ПС для <<СЦЕН.~1>> может быть в~1,05~раза выше, чем при 
использовании схемы РС, а~для <<СЦЕН.~3>>~--- в~1,12~раза.

\end{multicols}

\begin{figure*} %fig5
\vspace*{1pt}
\begin{minipage}[t]{80mm}
\begin{center}
   \mbox{%
\epsfxsize=79mm
\epsfbox{adu-5.eps}
}
\end{center}
\vspace*{-9pt}
\Caption{Графики зависимостей среднего числа запросов, 
об\-слу\-жи\-ва\-емых в~сис\-те\-ме~(\textit{а}), вероятности блокировки 
системы~(\textit{б}) и~среднего занятого ресурса~(\textit{в})
от интенсивности предложенной нагрузки:
\textit{1}~--- СЦЕН.~1; \textit{2}~--- СЦЕН.~2;
\textit{3}~--- СЦЕН.~3;
сплошные кривые~--- ПС; штриховые кривые~--- РС} 
\label{fig:meanN}
\end{minipage}
%\end{figure*}
\hfill
%\begin{figure*} %fig6
\vspace*{1pt}
\begin{minipage}[t]{80mm}
\begin{center}
   \mbox{%
\epsfxsize=79.102mm
\epsfbox{adu-6.eps}
}
\end{center}
\vspace*{-9pt}
\Caption{Графики зависимостей соотношений средних чисел 
об\-слу\-жи\-ва\-емых в~сис\-те\-ме запросов~(\textit{а}),
вероятности блокировки 
системы~(\textit{б}) и~среднего занятого ресурса~(\textit{в})
  для двух схем от интенсивности предложенной 
нагрузки: \textit{1}~--- СЦЕН.~1; \textit{2}~--- СЦЕН.~2;
\textit{3}~--- СЦЕН.~3}
 \label{fig:meanNRatio}
 \end{minipage}
\end{figure*}


\begin{multicols}{2}



\section{Заключение}
%\label{sec:conclusion}

В работе предложена схема доступа запросов пользователей к~ресурсам беспроводной 
сети, основанная на реализации механизма прерывания обслуживания пользователей в~рамках технологии нарезки радиоресурсов сети Network Slicing. 

Проведен 
сравнительный анализ, по\-ка\-зы\-ва\-ющий эффективность предложенной схемы по сравнению с~известной схемой доступа, основанной на механизме резервирования ресурсов.  
Результаты численного эксперимента показали, что в~диапазоне небольших нагрузок 
на сис\-те\-му предложенная схема эффективнее схемы доступа с~реализацией механизма 
резервирования.
Результаты численного эксперимента выделяют сле\-ду\-ющие особенности применения 
предложенной схемы до\-сту\-па по сравнению со схемой на основе реализации механизма 
резервирования:
\begin{enumerate}[(1)]
\item 
эффективность в~диапазоне небольших нагрузок в~использовании физических 
ресурсов БС и~емкостей слайсов;
\item 
значимое повышение эффективности при поддержке предоставления услуг для 
большего числа слайсов.
\end{enumerate}


{\small\frenchspacing
 {%\baselineskip=10.8pt
 %\addcontentsline{toc}{section}{References}
 \begin{thebibliography}{99}


\bibitem{3gpp.22.864} %1
\Au{Sultan A., Pope~M.} Feasibility study on new 
services and markets technology enablers for network operation; Stage~1 (3GPP). 
Ver. 15.0.0, 2016. 
{\sf https://portal.3gpp. org/desktopmodules/Specifications/SpecificationDetail s.aspx?specificationId=3016}.

\bibitem{3gpp.21.916} %2
\Au{Meredith J., Firmin~F., Pope~M.} Release 16 
Description; Summary of Rel-16 Work Items (3GPP). Ver. 16.2.0, 2022. 
{\sf https://portal.3gpp.org/desktopmodules/Specifications /SpecificationDetails.aspx?specificationId=3493}.
%Pravila tsitirovaniya istochnikov [Rules for the 
%citing of sources]. Available at: http://www.scribd.com/doc/1034528/ (accessed 
%February 7, 2011).

\bibitem{3gpp.28.554} %3
\Au{Meredith J., Soveri~M., Pope~M.} Management and 
orchestration; 5G end to end Key Performance Indicators (KPI) (3GPP). Ver. 
18.0.0, 2022. 
{\sf https://portal.3gpp. org/desktopmodules/Specifications/SpecificationDetail s.aspx?specificationId=3415}.

\bibitem{Yarkina2022} %4
\Au{Yarkina N., Correia~L., Moltchanov~D., Gaidamaka~Y., Samouylov~K.} Multi-tenant resource sharing with equitable-priority-based 
performance isolation of slices for 5G cellular systems~// Comput. Commun., 
2022. Vol.~188. P.~39--51. doi: 10.1016/j.comcom.2022.02.019.



\bibitem{Luu2022} %5
\Au{Luu Q., Kerboeuf~S., Kieffer~M.} Admission control and 
resource reservation for prioritized slice requests with guaranteed SLA under 
uncertainties~// IEEE T. Netw. Serv. Man., 2022. Vol.~19. P.~3136--3153. 
doi: 10.1109/ tnsm.2022.3160352.

\bibitem{Rehman2022} %6
\Au{Rehman A., Mahmood~I., Kamran~M., Sanaullah~M., Ijaz~A., Ali~J., Ali~M.} 
Enhancement in quality-of-services using 5G cellular network 
using resource reservation protocol~// Phys. Commun.~--- Amst., 2022. Art.~101907. 10~p. doi: 
10.1016/j.phycom.2022.101907.







\bibitem{kermani1977analysis} %7
\Au{Kermani P., Kleinrock~L.} Analysis of 
buffer allocation schemes in a~multiplexing node~// Int. Conf. 
Comm., 1977. Vol.~2. P.~30--34.

\bibitem{Kamoun1980} %8
\Au{Kamoun F., Kleinrock~L.} Analysis of shared finite 
storage in a~computer network node environment under general traffic conditions~// 
IEEE T. Commun., 1980. Vol.~28. P.~992--1003. doi: 
10.1109/tcom.1980.1094756.

\bibitem{Basharin1982} %9
\Au{Башарин Г.\,П., Самуйлов~К.\,Е.} Об оптимальной 
структуре БП в~сетях передачи данных с~коммутацией пакетов.~--- М.: ВИНИТИ, 
1982. 70~с.


\bibitem{Basharin2013} %10
\Au{Basharin G., Gaidamaka~Y., Samouylov~K.} 
Mathematical theory of teletraffic and its application to the analysis of 
multiservice communication of next generation networks~// Autom. Control Comp.~S., 2013. Vol.~47. P.~62--69. doi: 10.3103/s0146411613020028.





\bibitem{Stepanov} %11
\Au{Степанов С.\,Н.} Теория телетрафика: концепции, модели, 
приложения.~--- М.: Горячая лин\-ия-Телеком, 2015. 868~с.

\bibitem{Zhou2022} %12
\Au{Zhou D., Chen~Z., Pan~E., Zhang~Y.} Dynamic 
statistical responses of gear drive based on improved stochastic iteration 
method~// Appl. Math. Model., 2022. Vol.~108. P.~46--65. doi: 
10.1016/j.apm.2022.03.020.

\bibitem{schoolar2019} %13
\Au{Schoolar~D., Lambert~P., Nanbin~W., Liang~Z.} 5G 
service experience-based network planning criteria (Ovum Consulting)~// 
Partnership with Huawei, 2019. 
{\sf https://carrier. huawei.com/$\sim$/media/CNBGV2/download/products/\linebreak servies/5G-Planning-Criteria-White-Paper.pdf}.

\end{thebibliography}

 }
 }

\end{multicols}

\vspace*{-6pt}

\hfill{\small\textit{Поступила в~редакцию 15.01.23}}

\vspace*{8pt}

%\pagebreak

%\newpage

%\vspace*{-28pt}

\hrule

\vspace*{2pt}

\hrule

%\vspace*{-2pt}

\def\tit{PREEMPTION-BASED PRIORITIZATION SCHEME\\ FOR~NETWORK RESOURCES SLICING IN~5G~SYSTEMS}


\def\titkol{Preemption-based prioritization scheme for~network resources slicing in~5G~systems}


\def\aut{K.\,Y.\,B.~Adou$^1$, E.\,V.~Markova$^1$, Yu.\,V.~Gaidamaka$^{1,2}$, and~S.\,Ya.~Shorgin$^2$}

\def\autkol{K.\,Y.\,B.~Adou, E.\,V.~Markova, Yu.\,V.~Gaidamaka, and~S.\,Ya.~Shorgin}

\titel{\tit}{\aut}{\autkol}{\titkol}

\vspace*{-8pt}


\noindent
$^1$Peoples' Friendship University of Russia (RUDN University), 6~Miklukho-Maklaya Str., Moscow 117198, Russian\linebreak
$\hphantom{^1}$Federation

\noindent
$^2$Federal Research Center ``Computer Science and Control'' of the Russian Academy of Sciences; 
44-2~Vavilov\linebreak
$\hphantom{^1}$Str., Moscow 119133, Russian Federation

\def\leftfootline{\small{\textbf{\thepage}
\hfill INFORMATIKA I EE PRIMENENIYA~--- INFORMATICS AND
APPLICATIONS\ \ \ 2023\ \ \ volume~17\ \ \ issue\ 1}
}%
 \def\rightfootline{\small{INFORMATIKA I EE PRIMENENIYA~---
INFORMATICS AND APPLICATIONS\ \ \ 2023\ \ \ volume~17\ \ \ issue\ 1
\hfill \textbf{\thepage}}}

\vspace*{3pt} 



\Abste{The network slicing (NS) technology, which has been actively studied in recent years, is based on the representation of 
a~common network infrastructure in the form of various customizable logical networks called slices and involves the division of mobile 
network operators into two groups~--- physical network infrastructure providers (InPs) and mobile virtual network operators (MVNOs). 
The MVNOs lease the physical resources of InPs\linebreak\vspace*{-12pt}}

\Abstend{to create their own slices to provide services to their users with different quality of service 
requirements. In the present paper, for a~network with NS technology, a~scheme for accessing its radio resources is proposed that
 provides users with services with a~guaranteed bit rate (GBR) and priority control based on the implementation of the user service interruption mechanism. 
 The authors propose a~scheme for accessing radio resources of a~network under NS technology that provides users with services with GBR
  and priority control based on the implementation of the user service interruption mechanism. 
 To evaluate the effectiveness of the proposed scheme, a~comparative analysis of its characteristics with the characteristics of the
  access scheme based on the resource reservation mechanism was carried out.}

\KWE{5G; network slicing; quality of service; key performance indicators; priority management; service interruption; iterative method}



 \DOI{10.14357/19922264230113} 

\vspace*{-16pt}


\Ack

%\vspace*{-4pt}


\noindent
The research was supported by the Russian Science Foundation grant No.\,22-79-10053, {\sf https://rscf.ru/en/\linebreak project/22-79-10053/}.
% (Conceptualization, Y.\,A., E.\,M. and Y.\,G.; Data curation, Y.\,A.; Formal analysis, Y.\,A.; Funding acquisition, E.\,M.; 
 %Investigation, Y.\,A.; Methodology, Y.\,A., E.\,M., and Y.\,G.; Project administration, 
% E.\,M. and Y.\,G.; Resources, Y.\,A.; Software, Y.\,A.; Supervision, E.\,M. and Y.\,G.; Validation, 
% Y.\,A.; Visualization, Y.\,A. and E.\,M.; Writing~--- original draft, Y.\,A.; Writing~--- review and editing, Y.\,A., E.\,M., and Y.\,G.). 

  

%\vspace*{4pt}

  \begin{multicols}{2}

\renewcommand{\bibname}{\protect\rmfamily References}
%\renewcommand{\bibname}{\large\protect\rm References}

{\small\frenchspacing
 {%\baselineskip=10.8pt
 \addcontentsline{toc}{section}{References}
 \begin{thebibliography}{99} 

\bibitem{2-adu} %1
\Aue{Sultan, A., and M.~Pope.}
%3GPP TR 22.864. v. 15.0.0. 
2016. Feasibility study on new services and markets technology enablers for network operation; Stage~1 (3GPP). Ver. 15.0.0. Available at: 
{\sf https://\linebreak portal.3gpp.org/desktopmodules/Specifications/Specifi cationDetails.aspx?specificationId=3016} (accessed January~30, 2023).

\bibitem{1-adu} %2
\Aue{Meredith, J., F.~Firmin, and M.~Pope.}
%3GPP TR 21.916. v. 16.2.0. 
2022. Release 16~description; Summary of Rel-16 work items (3GPP).  Ver. 16.2.0. Available at: 
{\sf https://portal.3gpp.org/desktop\linebreak  modules/Specifications/SpecificationDetails.aspx?speci\linebreak ficationId=3493} (accessed January~30, 2023).
\bibitem{3-adu} %3
\Aue{Meredith, J., M.~Soveri, and M.~Pope.}
%3GPP TR 28.554. v. 18.0.0. 
2022. Management and orchestration; 5G end to end Key Performance Indicators (KPI) (3GPP). Ver. 18.0.0. Available at:
{\sf  https://\linebreak portal.3gpp.org/desktopmodules/Specifications/Specifi\linebreak cationDetails.aspx?specificatio nId=3415} (accessed January~30, 2023).
\bibitem{4-adu}
\Aue{Yarkina, N., L.\,M.~Correia, D.~Moltchanov, Y.~Gaidamaka, and K.~Samouylov.}
 2022. Multi-tenant resource sharing with equitable-priority-based performance isolation of slices for 5G cellular systems. 
 \textit{Comput. Commun.} 188:39--51. doi: 10.1016/j.comcom.2022.02.019.
\bibitem{5-adu}
\Aue{Luu, Q., S.~Kerboeuf, and M.~Kieffer.} 
2022. Admission control and resource reservation for prioritized slice requests with guaranteed SLA under uncertainties. 
\textit{IEEE T. Netw. Serv. Man.} 19:3136--3153. doi: 10.1109/ tnsm.2022.3160352.
\bibitem{6-adu}
\Aue{Rehman, A., I.~Mahmood, M.~Kamran, M.~Sanaullah, A.~Ijaz, J.~Ali, and M.~Ali.}
 2022. Enhancement in quality-of-services using 5G cellular network using resource reservation protocol. \textit{Phys. Commun.~--- Amst.}
  55:101907. 10~p. doi: 10.1016/j.phycom.2022.101907. 

\bibitem{8-adu} %7
\Aue{Kermani, P., and L.~Kleinrock.}
 1977. Analysis of buffer allocation schemes in a~multiplexing node. \textit{Int. Conf. Comm.} 2:30--34.
\bibitem{9-adu} %8
\Aue{Kamoun, F, and L.~Kleinrock.}
 1980. Analysis of shared finite storage in a~computer networks node environment under general traffic conditions. 
 \textit{IEEE T. Commun.} 28(7):992--1003. doi: 10.1109/tcom.1980.1094756.
 
 \bibitem{7-adu} %9
\Aue{Basharin, G.\,P., and K.\,E.~Samouylov.}
 1982. \textit{Ob op\-ti\-mal'\-noy struk\-tu\-re bu\-fer\-noy pa\-mya\-ti v~se\-tyakh pe\-re\-da\-chi dan\-nykh 
 s~kom\-mu\-ta\-tsi\-ey pa\-ke\-tov} [On the optimal structure of BP in data transmission networks with packet commutation]. Moscow: VINITI. 70~p.
 
\bibitem{10-adu}
\Aue{Basharin, G.\,P., Yu.\,V.~Gaidamaka, and K.\,E.~Samouylov}.
 2013. Mathematical theory of teletraffic and its application to the analysis of multiservice communication of next generation networks. 
 \textit{Autom. Control Comp.~S.} 47(2):62--69. doi: 10.3103/s0146411613020028.

\bibitem{12-adu} %11
\Aue{Stepanov, S.\,N.} 2015. \textit{Teo\-riya te\-le\-tra\-fi\-ka: Kon\-tsep\-tsii, mo\-de\-li, pri\-lo\-zhe\-niya} 
[Theory of teletraffic: Concepts, models, and applications]. Moscow: Goryachaya liniya-Telekom. 868~p.

\bibitem{11-adu} %12
\Aue{Zhou, D., Z.~Chen, E.~Pan, and Y.~Zhang.} 
2022. Dynamic statistical responses of gear drive based on improved stochastic iteration method. \textit{Appl. Math. Model.} 108:46--65.
doi: 10.1016/j.apm.2022.03.020.
\bibitem{13-adu}
\Aue{Schoolar, D., P.~Lambert, W.~Nanbin, and Z.~Liang.} 2019. 
5G service experience-based network planning criteria  (Ovum Consulting). Partnership with Huawei. Available at: 
{\sf https://carrier.huawei.com/$\sim$/media/\linebreak CNBGV2/download/products/servies/5G-Planning-Criteria-White-Paper.pdf} (accessed January~30, 2023).
  \end{thebibliography}

 }
 }

\end{multicols}

\vspace*{-6pt}

\hfill{\small\textit{Received January 15, 2023}}

\vspace*{-14pt}

\Contr

\vspace*{-3pt}

\noindent
\textbf{Adou Kpangny Y.\,B.} (b.\ 1993)~--- 
PhD student, research assistant, Department of Applied Probability and Informatics, Peoples' Friendship University of Russia (RUDN University), 
6~Miklukho-Maklaya Str., Moscow 117198, Russian Federation; \mbox{adu-k@rudn.ru}

\pagebreak

\noindent
\textbf{Markova Ekaterina V.} (b.\ 1987)~--- 
Candidate of Science (PhD) in physics and mathematics, assistant professor, Department of Applied Probability and Informatics, 
Peoples' Friendship University of Russia (RUDN University), 6~Miklukho-Maklaya Str., Moscow 117198, Russian Federation; 
\mbox{markova-ev@rudn.ru}

\vspace*{6pt}

\noindent
\textbf{Gaidamaka Yuliya V.} (b.\ 1971)~--- 
Doctor of Science in physics and mathematics, professor, Department of Applied Probability and Informatics, Peoples' 
Friendship University of Russia (RUDN University), 6~Miklukho-Maklaya Str., Moscow 117198, Russian Federation; 
senior scientist, Institute of Informatics Problems, Federal Research Center ``Computer Science and Control'' 
of the Russian Academy of Sciences, 44-2~Vavilov Str., Moscow 119333, Russian Federation; \mbox{gaydamaka-yuv@rudn.ru}

\vspace*{6pt}

\noindent
\textbf{Shorgin Sergey Ya.} (b.\ 1952)~--- 
Doctor of Science in physics and mathematics, professor, principal scientist, Institute of Informatics Problems, Federal Research Center 
``Computer Science and Control'' of the Russian Academy of Sciences, 44-2~Vavilov Str., Moscow 119133, Russian Federation; 
\mbox{sshorgin@ipiran.ru}



   
\label{end\stat}

\renewcommand{\bibname}{\protect\rm Литература} 
          %13

\def\stat{agasandyan}

\def\tit{МНОГОМЕРНЫЕ БАТТЕРФЛЯИ\\ В~ЗАДАЧАХ ОПТИМИЗАЦИИ ПО CC-VaR}

\def\titkol{Многомерные баттерфляи в~задачах оптимизации 
по~CC-VaR}

\def\aut{Г.\,А.~Агасандян$^1$}

\def\autkol{Г.\,А.~Агасандян}

\titel{\tit}{\aut}{\autkol}{\titkol}

\index{Агасандян Г.\,А.}
\index{Agasandyan G.\,A.}


%{\renewcommand{\thefootnote}{\fnsymbol{footnote}} \footnotetext[1]
%{Работа выполнена при поддержке Министерства науки и~высшего образования
%Российской федерации, грант №\,075-15-2020-799.}}


\renewcommand{\thefootnote}{\arabic{footnote}}
\footnotetext[1]{Федеральный исследовательский центр <<Информатика и~управление>> Российской 
академии наук, \mbox{agasand17@yandex.ru}}

\vspace*{-6pt}
 
  
  \Abst{Работа продолжает исследование технических проблем, связанных с~применением  
континуального критерия VaR (CC-VaR) на многомерных рынках опционов. 
В~предположении, что на рынке сценарными баттерфляями непосредственно не торгуют, 
разрабатывается методика получения их реп\-ли\-ка\-ции из многомерных $\alpha$-оп\-ци\-онов~--- 
многомерного обобщения обычных одномерных опционов, таких как коллы и~путы. Работа 
служит непосредственным расширением предложенного в~предыдущей работе автора 
способа, позволяющего конструировать индикаторы базиса на многомерном сценарном 
рынке комбинациями многомерных бинарных опционов. Методика основывается на 
теоремах паритета для одномерного рынка традиционных опционов и~пригодна для рынков 
произвольной размерности, но ее фактическая реализация проводится для двумерных 
рынков. Приводятся конструкции базисов из $\alpha$-оп\-ци\-онов~--- как однотипных, так 
и~смешанных естественных с~выделенным цент\-ром рынка. Теоретические пред\-став\-ле\-ния 
оптимальных портфелей в~этих базисах иллюстрируются на примере конкретного 
двумерного рынка.}
   
  \KW{базовые активы; многомерный рынок; функция рисковых предпочтений инвестора; 
континуальный критерий VaR (CC-VaR); стоимостная и~прогнозная плотности; опционы 
колл и~пут; $\alpha$-оп\-ци\-оны; сценарные баттерфляи; базисы; центр рынка; портфели 
баттерфляев}

 \DOI{10.14357/19922264230114} 
  
\vspace*{-2pt}


\vskip 10pt plus 9pt minus 6pt

\thispagestyle{headings}

\begin{multicols}{2}

\label{st\stat}
   
  \section{Введение}
  
  Проблемы применения на рынках опционов введенного автором 
континуального критерия VaR (CC-VaR) рассматриваются в~[1--5]. Настоящую 
работу можно рассматривать как продолжение исследования~[6], в~котором 
предлагались варианты репликации индикаторов базиса на многомерном 
сценарном рынке комбинациями так называемых $\zeta$-оп\-ци\-онов 
(многомерных бинарных опционов).\linebreak Здесь подобная задача решается для более 
сложных инструментов~--- многомерных аналогов одномерных базисных 
баттерфляев, которые реплицируются комбинациями так называемых
  $\alpha$-оп\-ци\-онов~--- многомерных аналогов традиционных \mbox{опционов} типа 
колл и~пут. 
  
  Основания для такого рассмотрения и~его проб\-ле\-мы, связанные 
с~применением континуального критерия VaR (CC-VaR), приведены в~[6], там 
же вводятся многие обозначения, которые используются и~здесь. 
Теоретической моделью при построении $\alpha$-рын\-ка служит также 
многомерный $\delta$-ры\-нок~\cite{5-aga, 6-aga}. 
  
  В работе для многомерных рынков опционов решаются те же проблемы 
технического характера, что и~для $\zeta$-рын\-ков~--- рынков  многомерных 
бинарных опционов. Но на этот раз в~отношении своего инструментария они 
в~большей мере напоминают проб\-ле\-мы традиционных рынков опционов, на 
которых в~отсутствие баттерфляев в~качестве объектов непосредственной 
торговли предлагается получать их в~виде комбинаций коллов и~путов. 
  
  \section{Теоретический $\alpha$-рынок и~его~свойства}
  
  Вновь рассматривается многомерный $\delta$-ры\-нок (однопериодный, 
теоретический и~идеальный)\linebreak с~$n$ ($>1$) базовыми активами, векторы цен 
которых в~конце периода $\bm{x}\hm= (x_1, x_2, \ldots, x_n)$, $x_l\hm\in {\sf X}_l \hm\subset 
\mathfrak{R}$, $l\hm\in N\hm=\{1, \ldots , n\}$, образуют $n$-мер\-ное множество 
${\mathsf X}\hm=\prod_{l\in N} {\mathsf X}_l$. На~${\mathsf X}$ заданы 
\textit{прогнозная} $p(\bm{x})$ и~\textit{стоимостная} $c(\bm{x})$ плотности, 
по\-рож\-да\-ющие вероятностные меры~${\mathsf P}\{\cdot\}$ и~${\mathsf 
C}\{\cdot\}$. 
  
  Платежная функция произвольного инструмента~$\bm{I}$ обозначается 
$\pi(\bm{x}; \bm{I})$, его рыночная сто\-и\-мость и~средний, с~точки зрения 
инвестора, доход, рас\-счи\-тан\-ные по плотностям $c(\bm{x})$ и~$p(\bm{x})$ 
соответственно, определяются соотношениями: 
  $$
  \vert \bm{I}\vert =\int\limits_{\mathsf X} \pi (\bm{x};\bm{I}) 
c(\bm{x})\,d\bm{x}\,;\enskip
  \left\| \bm{I}\right\| =\int\limits_{\mathsf X} \pi(\bm{x};\bm{I}) 
p(\bm{x})\,d\bm{x}\,.
  $$
  
  Базис рынка составляют $\delta$-ин\-стру\-мен\-ты $\bm{D}(\bm{s})$, 
$\bm{s}\hm\in {\mathsf X}$, с~обобщенной $n$-мер\-ной $\delta$-функ\-ци\-ей 
относительно~$\bm{s}$ в~качестве платежной: 

\noindent
  \begin{multline}
    \bm{D}(\bm{s}) =\prod\limits_{l\in N} \bm{D}_l(s_l)\,,\\
  \pi(\bm{x};\bm{D}(\bm{s}))=\delta(\bm{x}-\bm{s})=\prod\limits_{l\in N} 
\delta(x_l-s_l).
  \label{1-aga}
  \end{multline}
  
  Инструмент $\bm{G}$ с~произвольной измеримой платежной 
функцией~$g(\bm{x})$ и~его стоимость имеют вид: 
  \begin{align*}
  \bm{G}&= \int\limits_{\mathsf X} g(\bm{s}) \bm{D}(\bm{s})\,d\bm{s}\,;\\
  \vert \bm{G}\vert &=\int\limits_{\mathsf X} g(\bm{s}) \vert \bm{D}(\bm{s}) \vert 
\,d\bm{s}= \int\limits_{\mathsf X} g(\bm{s}) c(\bm{s}) \,d\bm{s}\,.
  \end{align*}
  
  Наряду с~<<полноправными>> $n$-мер\-ны\-ми инструментами на рынке 
присутствуют и~их $k$-мер\-ные версии, у~которых $n\hm- k$ координатных 
базовых активов пред\-став\-ле\-ны в~форме одномерных единичных без\-рис\-ко\-вых 
инструментов. 
  
  Для индикаторов $\bm{H}\{M\}$, $M\hm\subset {\mathsf X}$, без\-рис\-ко\-во\-го 
актива $\bm{U}\hm=\bm{H}\{ {\mathsf X}\}$ и~их цен 
  \begin{gather*}
  \bm{H}\{ M\}=\int\limits_M \bm{D}(\bm{s})\, d\bm{s}\,;\enskip
   \vert \bm{H}\{M\}\vert =\int\limits_M c(\bm{s})\,d\bm{s}\,;\\
   \vert \bm{U}\vert ={\mathsf C}\{ {\mathsf X}\}= \int\limits_{\mathsf X} c(\bm{s}) 
\,d\bm{s}=\fr{1}{r}\,,
   \end{gather*}
где $r$~--- приравниваемый единице безрисковый доход за период. 
  
  На \textit{одномерном} рынке опционы пут~$\bm{P}_s$ и~колл~$\bm{C}_s$ 
со страйком~$s$ задаются своими платежными функциями: 
  \begin{equation}
  \left.
  \begin{array}{rl}
  \pi(x;\bm{P}_s)&=\max (0,s-x);\\[6pt]
  \pi(x;{\bm C}_x)&=\max (0,x-s),\ x,s \in {\mathsf X}\subset \mathfrak{R}\,.
  \end{array}
  \right\}
  \label{e2-aga}
  \end{equation}
  %
  Для них выполняется формула паритета ($\bm{X}$~--- вектор базовых 
активов)
  $$
  \bm{C}_s-\bm{P}_s= \bm{X}-s\bm{U}\,.
  $$
  
  Нормированными спрэдами быка с~парой страйков $s\hm-h$, $s \hm\in 
{\mathsf X}$ ($h \hm>0$)  и~медведя с~парой страйков~$s$, $s\hm+h \hm\in 
{\mathsf X}$ служат комбинации опционов соответственно 
  \begin{equation}
  \left.
  \begin{array}{rl}
 \!\!\!\! \bm{S}_{s;h}^{\mathrm{bull}} &= \fr{1}{h}\left( \bm{C}_{s-h}-\bm{C}_s\right) =
\bm{U}+\fr{1}{h}\left( \bm{P}_{s-h}-\bm{P}_s\right)\,;\\[6pt]
 \!\! \!\! \bm{S}_{s;h}^{\mathrm{bear}} &=\fr{1}{h}\left( \bm{P}_{s+h}-\bm{P}_s\right) 
=\bm{U}+\fr{1}{h} \left( \bm{C}_{s+h}-\bm{C}_s\right)
  \end{array}\!
  \right\}\!\!
  \label{e3-aga}
  \end{equation}
с платежными функциями 
\begin{equation}
\left.
\begin{array}{rl}
 \!\!\!\!\pi\left( x;\bm{S}_{s;h}^{\mathrm{bull}}\right) &= \min \left(\! 1,\fr{1}{h}\max (0,x-(s-h))\!\right);\\[9pt]
\! \!\!\!\pi\left( x;\bm{S}_{s;h}^{\mathrm{bear}}\right) &= \min \left(\! 1,\fr{1}{h}\max (0, (s+h)-x)\!\right).
\end{array}\!
\right\}\!
\label{e4-aga}
\end{equation}
  
  Нормированные симметричные баттерфляи с~тройкой страйков $s\hm-h$, $s$, 
$s\hm+h \hm\in {\mathsf X}$ образуются комбинациями 
  \begin{multline}
  \bm{B}_{s;h} = \fr{1}{h}\left( \bm{C}_{s-h} -2\bm{C}_s +\bm{C}_{s+h}\right) 
={}\\
  {}= \fr{1}{h} \left( \bm{P}_{s-h} -2\bm{P}_s+\bm{P}_{s+h}\right) 
={}\\
{}=\bm{U}+\fr{1}{h}\left( \bm{P}_{s-h} -\bm{P}_s -
\bm{C}_s+\bm{C}_{s+h}\right)
  \label{e5-aga}
  \end{multline}
с платежными функциями 
\begin{equation}
\pi \left( x;\bm{B}_{s;h}\right) =\fr{1}{h}\max ( 0, h-\vert x-s\vert).
\label{e6-aga}
\end{equation}
  
  В комбинациях~(\ref{e3-aga}) и~(\ref{e5-aga}) безрисковый 
инструмент~$\bm{U}$ выполняет функцию маржевого инструмента 
и~применяется инвестором в~соответствии с~требованиями рынка. Формально 
верно еще одно пред\-став\-ле\-ние:
  \begin{equation*}
  \bm{B}_{s;h}=\fr{1}{h}\left( \bm{C}_{s-h} -\bm{C}_s -\bm{P}_s 
+\bm{P}_{s+h}\right)\,,
 % \label{e7-aga}
  \end{equation*}
но оно не является \textit{естественным} (страйки коллов в~комбинации ниже 
страйков путов) и~потому далее не используется. 
  
  Можно было бы рассматривать и~не создающие принципиальных трудностей 
несимметричные баттерфляи (с~неравными по длине сценариями 
и~неравномерной линейкой страйков), но они, как правило, не применяются на 
рынках и~к~тому же сильно загромождали бы изложение. 
  
  С целью алгоритмической автоматизации дальнейших построений для 
одномерных опционов $\bm{P}_s$ и~$\bm{C}_s$ вводятся также обозначения 
$\bm{O}_s^-$ (и~$\bm{O}_{0;s}$) и~$\bm{O}_s^+$ (и~$\bm{O}_{1;s})$, которые 
могут обрастать дополнительными индексами координат $l\hm\in N$: 
  \begin{equation}
  \bm{O}_{0;s}\equiv \bm{O}_s^- \equiv \bm{P}_s\,;\enskip
  \bm{O}_{1;s} \equiv \bm{O}_s^+\equiv \bm{C}_s\,.
  \label{e8-aga}
  \end{equation}
  
  \textit{Многомерным} обобщением одномерных опционов $\bm{P}_s$ 
и~$\bm{C}_s$ служат $n$-мер\-ные $\alpha$-\textit{оп\-ци\-оны} 
$\bm{A}_{\bm{\alpha};\bm{s}}$ векторного типа~$\bm{\alpha}$ и~с~векторным 
страйком $\bm{s}\hm \in \mathfrak{R}^n$, задаваемые вместе с~платежными 
функциями соотношениями 
  \begin{multline}
  {A}_{\alpha;s}=\prod\limits_{i\in N} \bm{O}_{i\beta_i;s_i}\,,\\
  \pi\left(\bm{x}, A_{\alpha;s}\right)=
  \prod_{l\in N}\pi \left(x_l; \bm{O}_{l\beta_l;s_l}\right),\ \bm{x}\in 
\mathfrak{R}^n\,,\\[6pt]
  \pi\left( x_l; \bm{O}_{l\beta_l;s_l}\right) =\omega_{l\beta_l;s_l}(x_l)={}\\[6pt]
  \hspace*{10mm}{}=\max  \left(0,\alpha_l (x_l-s_l)\right), \enskip l\in N\,.
    \label{e9-aga}
  \end{multline}
  
  Как и~в~[6], вектор~$\bm{\alpha}$ с~компонентами $\alpha_l\hm= \pm1$, $l\hm\in N$, 
в~индексах инструментов (или просто~$\pm$) определяет векторный тип  
$\alpha$-оп\-ци\-онов~$\bm{A}_{\bm{\alpha};\bm{s}}$. Вектор $\bm{\beta}\hm= 
(\bm{\alpha}\hm+1)/2$, дублирующий~$\bm{\alpha}$, вводится для удобства по 
техническим причинам и~принимает для каждого $l\hm\in N$ два значения: 
$$
\beta_l= \begin{cases}
0 &\mbox{для\ пута;}\\ 
1 & \mbox{для\ колла.}
\end{cases}
$$ 
  
  Для каждого векторного страйка~$\bm{s}$ на $n$-мер\-ном рынке могут 
котироваться $2^n$ типов $\alpha$-оп\-ци\-онов. Рынок $n$-мер\-ных  
$\alpha$-оп\-ци\-онов с~их $k$-мер\-ны\-ми версиями, $k\hm< n$, называется  
$n$-мер\-ным $\alpha$-\textit{рын\-ком}. 
  
  \section{Двумерный дискретный $\alpha$-рынок}
   
  В основе дискретного $\alpha$-рын\-ка лежит \textit{сценарный} рынок~--- 
сценарная дискретизация двумерного тео\-ре\-ти\-че\-ско\-го $\delta$-рын\-ка. Как 
и~для бинарного рынка, используется в~большей мере адаптированная 
к~двумерному случаю очевидная сис\-те\-ма обозначений, но учитывается 
и~специфика требований $\alpha$-рынка. 
  
  Цены двух базовых активов \textit{теоретического} двумерного  
$\delta$-рын\-ка обозначаются~$x$ и~$y$, страйки опционов~--- 
соответственно~$s$ и~$t$, $x, s \hm\in {\mathsf X} \hm=[a_1,b_1)\hm\subset 
\mathfrak{R}$, $y,t \hm\in {\mathsf Y}\hm=[a_2,b_2) \hm\subset \mathfrak{R}$. 
Дискретизация осуществляется равномерным разбиением множества~${\mathsf 
X}$ на~$v_1$ интервалов (сценариев), ${\mathsf Y}$~--- на $v_2$ интервалов. 
Одномерные сценарии на~${\mathsf X}$ и~${\mathsf Y}$ даются формулами: 
  \begin{multline}
  S_i= \left[ x_{i-1},x_i\right),\ x_i=a_1+ih_1,\ h_1=\fr{b_1-a_1}{v_1},\\
   i\in  \bm{I},\ x_0=a_1; \label{e10-aga}
   \end{multline}
   
   \vspace*{-12pt}
   
   \noindent
   \begin{multline}
  T_j= \left [ y_{j-1},y_j\right),\ y_j= a_2+jh_2,\ h_2=\fr{b_2-a_2}{v_2},\\ j\in 
\bm{J},\ y_0=a_2\,,
  \label{e11-aga}
\end{multline}
где $\bm{I}=\{1,2, \ldots, v_1\}$, $\bm{J}\hm=\{1,2,\ldots , v_2\}$, а~номер 
сценария совпадает с~индексом его правой границы. Двумерными сценариями 
служат прямые произведения всех пар $S_i\times T_j$, $i\hm\in \bm{I}$, $j\hm\in 
\bm{J}$. 
  
  На сценарном рынке базис образуют индикаторы сценариев 
$\bm{D}_{ij}\hm=\bm{H}\{S_i\times T_j\}$, но для $\alpha$-рын\-ка при той же структуре 
сценариев уместнее использовать иной базис~--- из баттерфляев~$\bm{B}_{ij}$, 
задаваемых с~учетом определения~(\ref{e5-aga}), но для специально 
подобранных страйков. Страйками~$s_i$ и~$t_j$ одномерных опционов 
и~упомянутых баттерфляев служат середины сценариев~(\ref{e10-aga}) 
и~(\ref{e11-aga}): 
  $$
  s_i = \fr{x_{i-1} + x_i}{2}\,,\  i\in \bm{I}; \enskip   t_j = \fr{y_{j-1} + y_j}{2},\   j\in 
\bm{J}\,. 
  $$
  %
  При этом параметр~$h$ для баттерфляев~(\ref{e5-aga}), равный длине 
сценариев, определяется в~(\ref{e10-aga}) и~\ref{e11-aga}). Для удобства 
записи формул также доопределяются параметры $s_0 \hm= a_1$, $s_{v_1+1}\hm = 
b_1$, $t_0 \hm= a_2$, $t_{v_2+1}\hm = b_2$, но они страйками не являются. 
  
  Портфель с~вектором~$\bm{g}$ весов базисных баттерфляев в~двумерном 
случае приобретает вид: 
  \begin{equation}
  \bm{G}= \sum\limits_{i\in \bm{I}, j\in \bm{J}} g_{ij} \bm{B}_{ij}\,.
  \label{e12-aga}
  \end{equation}
  
  Двумерным обобщением обычных опционов служат инструменты, 
характеризуемые парой страйков $(s_i,t_j)$, или просто $(i,j)$, 
с~дополнительным указанием типа (лучше в~терминах $\bm{\beta}\hm=(\beta_1, 
\beta_2))$: 
\begin{multline*}
\bm{A}_{\beta_1\beta_2;ij} 
=\bm{O}_{\beta_1;1,i}\bm{O}_{\beta_2;2,j}=\bm{O}_{\beta_1;i\cdot} 
\bm{O}_{\beta_2;\cdot j}\,,\\ 
i\in \bm{I}\,,\ j\in \bm{J}\,.
\end{multline*}
  %
  Также рассматриваются и~их одномерные версии, обозначаемые~$\bm{A}_{i\cdot}$ 
и~$\bm{A}_{\cdot j}$ с~маркером <<точка>> в~позиции, отведенной координате 
безрискового актива. 
  
  Для представления произвольного инструмента~$\bm{G}$~(\ref{e12-aga}) 
в~базисе из $\alpha$-оп\-ци\-онов (избыточным в~сравнении с~базисом из 
баттерфляев) все нормированные баттерфляи в~(\ref{e12-aga}) следует 
реплицировать в~терминах $\alpha$-оп\-ци\-онов. 
  
  В соответствии с~(\ref{e9-aga}) для конструирования репликаций следует 
перемножать одномерные представления сценарных баттерфляев, выбирая 
подходящие сомножители из~(\ref{e3-aga}) и~(\ref{e5-aga}). Для одномерного 
рынка с~$v$~сценариями $i\hm\in \bm{I}$ справедливы такие репликации 
баттерфляев коллами и~путами:
  \begin{multline}
  \bm{B}_i={}\\
  \!\!\!\!{}=\begin{cases}
  \bm{U}-\fr{\bm{O}_1^+ -\bm{O}_2^+}{h}=\fr{\bm{O}_2^- -\bm{O}_1^-}{h}\,, & i=1;\\
  \fr{\bm{O}_{i-1}^+-2\bm{O}_i^+ +\bm{O}_{i+1}^+}{h}={}&\\
  \hspace*{7mm}{}= \fr{\bm{O}^-_{i-1}-2\bm{O}_i^- +\bm{O}^-_{i+1}}{h}={}&\\
  \hspace*{12mm}{}= \bm{U}-\fr{\bm{O}_i^- -\bm{O}^-_{i-1}}{h} -{}&\\
  \hspace*{15mm}{}- \fr{\bm{O}_i^+  - \bm{O}^+_{i+1}}{h}\,, &\hspace*{-7mm} 1<i<v\,;\\
  \fr{\bm{O}^+_{v-1} -\bm{O}_v^+}{h}=\bm{U}-\fr{\bm{O}_v^- -\bm{O}^-_{v-
1}}{h}\,, & i=v\,.
  \end{cases}\!
  \label{e13-aga}
  \end{multline}
  
  Базисные инструменты для $i\hm=\overline{1,v}$ являются спрэдами, но их для 
удобства также называем баттерфляями (\textit{усеченными}). 
Инструмент~$\bm{U}$ в~выписанных соотношениях, как в~(\ref{e3-aga}) 
и~(\ref{e5-aga}),  выполняет функцию маржевого инструмента. 
  
  Подобно сценарным базисам для $\zeta$-рынка~\cite{6-aga} построение 
двумерных базисов из $\alpha$-оп\-ци\-онов проводится на основе одномерных 
базисов, но их элементы на этот раз выбираются из~(\ref{e13-aga}). Строятся 
три варианта репликации базисов: два однотипных (один в~путах, другой 
в~коллах) и~третий~--- смешанный естественный. Если в~базисе $v$~сценариев, 
а~центральный страйк~$i_c$, то 
  \begin{itemize}
\item однотипный базис при $\alpha\hm=-1$ (в~путах):
\begin{equation}
\left.
\begin{array}{rl}
\bm{B}_1^- &=\fr{\bm{O}_2^*-\bm{O}_1^-}{h}\,;\\[6pt]
  \bm{B}_i^- &= \fr{\bm{O}^-_{i-1} -2\bm{O}_i^- +\bm{O}^-_{i+1}}{h}\,;\\[6pt]
   \bm{B}^-_v &=\bm{U}- \fr{\bm{O}_v^- -\bm{O}^-_{v-1}}{h}\,;
   \end{array}
   \right\}
\label{e14-aga}
\end{equation}
\item однотипный базис при $\alpha\hm=+1$ (в~коллах):
\begin{equation}
\left.
\begin{array}{rl}
\bm{B}_1^+ &\equiv \bm{U} - \fr{\bm{O}_1^+ -\bm{O}_2^+}{h}\,;\\[6pt]
  \bm{B}_i^+ &= \fr{\bm{O}^+_{i-1} -2\bm{O}_i^+ +\bm{O}^+_{i+1}}{h}\,;\\[6pt]
    \bm{B}_v^+ &\equiv \fr{\bm{O}^+_{v-1} -\bm{O}_v^+}{h}\,;
    \end{array}
    \right\}
\label{e15-aga}
\end{equation}
\item смешанный естественный базис:
\begin{equation}
\left.
\begin{array}{l}
\!\!\!\bm{B}_1^m\equiv \fr{\bm{O}_2^- -  \bm{O}_1^-}{h}\,;\\[6pt]
  \!\!\!\bm{B}_i^m \equiv \fr{\bm{O}^-_{i-1} -2\bm{O}_i^- +\bm{O}^-_{i+1}}{h}\,,\ 0< i< i_c;\\[6pt]
\!\!\!\bm{B}^m_{i_c} \equiv  \bm{U}-\fr{\bm{O}^-_{i_c-1} -\bm{O}^-_{i_c} -
\bm{O}^+_{i_c} +\bm{O}^+_{i_c+1}}{h}\,;\\[6pt]
\!\!\!\bm{B}_i^m\equiv \fr{\bm{O}^+_{i-1} -2\bm{O}_i^++\bm{O}^+_{i+1}}{h_i}\,,\ i_c<i<v\,;\\[6pt]
 \!\!\!\bm{B}_v^m\equiv \fr{\bm{O}^+_{v-1} -\bm{O}_v^+}{h}\,.
\end{array}\!
\right\}\!
\label{e16-aga}
\end{equation}
  \end{itemize}
  
  \section{Формирование базисов и~платежных функций 
портфелей $\alpha$-опционов}
  
  На основе соотношений~(\ref{e14-aga})--(\ref{e16-aga}) введенные 
  в~многомерном случае произвольной размерности конструкции здесь 
переписываются для двумерного  
$\alpha$-рын\-ка в~однотипных и~смешанных вариантах. 
  
  Поскольку каждый двумерный базисный баттерфляй определяется как 
произведение двух одномерных (что соответствует перемножению платежных 
функций), его репликации двумерными\linebreak $\alpha$-оп\-ци\-она\-ми находятся 
перемножением пары подходящих представлений  
из~(\ref{e14-aga})--(\ref{e16-aga}). 
  
  \textit{Однотипная} репликация сценарных баттерфляев проводится 
  $\alpha$-оп\-ци\-она\-ми единого типа $\bm{\alpha}\hm=\{\alpha_1, \alpha_2\}$. Он фиксируется 
заранее, и~потому используются более простые соотношения~(\ref{e14-aga}) 
и~(\ref{e15-aga}), а~обозначение типа опциона опускается. 
  
  Каждое перемножение сумм одномерных опционов в~(\ref{e14-aga}) 
или~(\ref{e15-aga}) дает сумму парных произведений этих опционов, которые 
затем следует замещать согласно~(\ref{e9-aga}) эквивалентными двумерными 
$\alpha$-оп\-ци\-она\-ми по правилам 
  \begin{multline}
  1\to \bm{U}\,,\enskip \bm{O}_{1,i}\bm{O}_{2,j}\to \bm{A}_{ij}\,,\\ 
\bm{O}_{1,i}\bm{U}_2\to \bm{A}_{i\cdot}\,,\enskip \bm{U}_1 \bm{O}_{2,j}\to 
\bm{A}_{\cdot j}\,.
  \label{e17-aga}
  \end{multline}
  
  \textit{Смешанная} репликация осуществляется аналогично, но указание типа в~обозначениях необходимо, и~потому правила трансформации приобретают 
вид: 
  \begin{multline}
  1\to \bm{U}\,,\ \bm{O}_{1,i}^{\alpha_1} \bm{O}_{2,j}^{\alpha_2} \to 
\bm{A}_{ij}^{\bm {\alpha}}=\bm{A}_{\beta_1,\beta_2;ij},\\
\bm{O}_{1,i}^{\alpha_1}\to \bm{A}_{\beta_1;i,\cdot} \left( = 
\bm{A}^{\alpha_1}_{i\cdot}\right)\,,\ \bm{O}^{\alpha_2}_{2,j} \to 
\bm{A}_{\beta_2;\cdot,j}\left( =\bm{A}^{\alpha_2}_{\cdot j}\right),\\
 \beta_1,\beta_2\in  \{0,1\}\,,
\label{e18-aga}
\end{multline}
  
  На двумерном $\alpha$-рын\-ке в~соответствии с~чис\-лом возможных 
векторов~$\bm{\alpha}$ насчитываются четыре варианта однотипных базисов 
и~один смешанный (естественный с~заданным центром рынка). 
  
  Для каждого варианта с~\textit{однотипным} базисом и~оптимальным 
портфелем фиксируется тип~$\bm{\alpha}$, и~он становится типом всех  
$\alpha$-оп\-ци\-онов варианта. В~двумерном случае таких типов четыре: $\{-1, -
1\}$; $\{-1,+1\}$; $\{+1,-1\}$; $\{+1,+1\}$. Последовательным применением 
правил~(\ref{e17-aga}) ко всем страйкам для каж\-до\-го значения векторного 
параметра~$\bm{\alpha}$ находятся искомые четыре базиса. В~однотипном случае 
для каждой компоненты рынка наличествуют три качественно различных 
представления по варианту страйка~--- двум крайним и~общему внутреннему, 
и~потому их $3^2\hm=9$. В~качестве примера приводится базис для 
$\bm{\alpha}\hm=\{-1,+1\}$, т.\,е.\ в~терминах $\alpha$-оп\-ци\-онов~$\bm{A}_{01}$ 
(остальные три \textit{однотипных} базиса выписываются сходным образом), 
при этом в~списке принимается $0\hm<i \hm< v_1$, $0 \hm< j \hm< v_2$: 
  \begin{align*}
  \bm{B}_{1,1}&=\fr{\bm{A}_{1,1}- \bm{A}_{1,2} -
\bm{A}_{2,1}+\bm{A}_{2,2}}{h_1h_2}+{}\\
&\hspace*{35mm}{}+ \fr{-\bm{A}_{1,\cdot} +\bm{A}_{2,\cdot}}{h_1}\,;\\
  \bm{B}_{1,j} &= \fr{-\bm{A}_{1,j-1} +2\bm{A}_{1,j} -\bm{A}_{1,j+1}}{h_1h_2}+{}\\
  &  \hspace*{15mm} {}+
\fr{\bm{A}_{2,j-1} -2\bm{A}_{2,j} +\bm{A}_{2,j+1}}{h_1h_2}\,;\\
   \bm{B}_{1,v_2}&= \fr{-\bm{A}_{1,v_2-1} +\bm{A}_{1,v_2} +\bm{A}_{2,v_2-1}- \bm{A}_{2,v_2}}{h_1h_2}\,;
  \end{align*}
  
\noindent
  \begin{align*}
   \bm{B}_{i,1}&= \fr{ -\bm{A}_{i-1,1} +\bm{A}_{i-1,2} +2\bm{A}_{i,1}}{h_1h_2}+{}\\
  &\hspace*{15mm}{}+   \fr{ -2\bm{A}_{i,2} -\bm{A}_{i+1,1} +\bm{A}_{i+1,2}}{h_1h_2}+{}\\
  &\hspace*{25mm}{}+ \fr{ \bm{A}_{i-1,\cdot } - 2\bm{A}_{i,\cdot} +\bm{A}_{i+1,\cdot}}{h_1}\,;\\
  \bm{B}_{i,j} &=\fr{\bm{A}_{i-1,j-1} -2\bm{A}_{i-1,j} +\bm{A}_{i-1,j+1}}{h_1h_2}+{}\\
  &\hspace*{11mm}{}+ \fr{-2\bm{A}_{i,j-1} +4\bm{A}_{i,j} -2\bm{A}_{i,j+1}}{h_1h_2} +{}\\
 &\hspace*{13mm}{}+  \fr{\bm{A}_{i+1,j-1} -2\bm{A}_{i+1,j}+\bm{A}_{i+1,j+1}}{h_1h_2}\,;\\
  \bm{B}_{i,v_2}&= \fr{\bm{A}_{i-1,v_2-1} -\bm{A}_{i-1,v_2} -2\bm{A}_{i,v_2-1}}{h_1h_2} +{}\\
 & \hspace*{16mm}{}+\fr{2\bm{A}_{i,v_2} +\bm{A}_{i+1,v_2-1} -\bm{A}_{i+1,v_2}}{h_1h_2}\,;\\
  \bm{B}_{v_1,1} &=\bm{U} +\fr{ -\bm{A}_{\cdot,1} +\bm{A}_{\cdot, 2}}{h_2}+ {}\\
 & \hspace*{2mm}{}+\fr{-\bm{A}_{v_1-1,1} +\bm{A}_{v_1-1,2} +\bm{A}_{v_1,1} -\bm{A}_{v_1,2}}{h_1h_2} +{}\\
&\hspace*{37mm}{}+\fr{\bm{A}_{v_1-1,\cdot}- \bm{A}_{v_1,\cdot}}{h_1}\,;\\
  \bm{B}_{v_1,j}&= \fr{\bm{A}_{\cdot,j-1}- 2\bm{A}_{\cdot,j} +\bm{A}_{\cdot,j+1}}{h_2} +{}\\
&\hspace*{2mm}{}+\fr{\bm{A}_{v_1-1,j-1}-2\bm{A}_{v_1-1,j}+\bm{A}_{v_1-1,j+1}}{h_1h_2}+{}\\
&\hspace*{15mm}{}+\fr{ -\bm{A}_{v_1,j-1}+2\bm{A}_{v_1,j}-\bm{A}_{v_1,j+1}}{h_1h_1}\,;\\
  \bm{B}_{v_1,v_2}&= \fr{\bm{A}_{\cdot,v_2-1}- \bm{A}_{\cdot,v_2}}{h_2} 
+{}\\
&\hspace*{-7mm}{}+\fr{\bm{A}_{v_1-1, v_2-1}- \bm{A}_{v_1-1,v_2} - \bm{A}_{v_1,v_2-1} 
+\bm{A}_{v_1,v_2}}{h_1h_2}\,.
  \end{align*}
    Здесь в~индексах опционов маркер <<точка>> отмечает координату 
безрискового актива, а~под~$\bm{A}_{i,\cdot}$ и~$\bm{A}_{\cdot,j}$, как уже обсуждалось 
выше, понимаются двумерные инструменты $\bm{A}_i\times \bm{U}_2$ 
и~$\bm{U}_1\times \bm{A}_j$ соответственно. 
  
  \textit{Смешанный} базис состоит из $5^2\hm=25$ качественно различных 
вариантов представления базисных инструментов, поскольку для каждой 
компоненты рынка вариантов страйка пять: два крайних, один центральный 
и~два внутренних, ниже и~выше центра. Их перечень получается применением 
правил~(\ref{e16-aga}). Приводим лишь часть базиса, связанную с~первым по 
отношению к~центру рынка квадрантом, т.\,е.\ для $1 \hm\leq i \hm\leq i_c$, $1 
\hm\leq j \hm\leq j_c$ (прочие части образуются аналогично):
  \begin{align*}
  \bm{B}_{1,1}&=\fr{\bm{A}_{00;1,1} - \bm{A}_{00;1,2} - \bm{A}_{00;2,1} + 
\bm{A}_{00;2,2}}{h_1h_2}\,; 
\end{align*}

  \noindent
  \begin{align*}
  \bm{B}_{1,j}&=\fr{-\bm{A}_{00;1,j-1} + 2\bm{A}_{00;1,j} - \bm{A}_{00;1,j+1}}{h_1h_2} + {}\\
&\hspace*{-3mm}{}+
\fr{\bm{A}_{00;2,j-1} - 2\bm{A}_{00;2,j} + \bm{A}_{00;2,j+1}}{h_1h_2}\,,\enskip 0<j<j_c\,; \\
  \bm{B}_{1,j_c}&=\fr{- \bm{A}_{00;1,j_c-1} + \bm{A}_{00;1,j_c} + \bm{A}_{00;2,j_c - 1}}{h_1h_2} +{}\\
  &\hspace*{-3mm}{}+  
  \fr{- \bm{A}_{00;2,j_c} + \bm{A}_{01;1,j_c} - \bm{A}_{01;1,j_c+1} - \bm{A}_{01;2,j_c} }{h_1h_2}+{}\\
&\hspace*{21mm}{}+ \fr{\bm{A}_{01;2,j_c+1}}{h_1h_2} + \fr{- \bm{A}_{0;1,\cdot} + \bm{A}_{0;2,\cdot}}{h_1}\,; \\
  \bm{B}_{i,1}&=\fr{-\bm{A}_{00;i-1,1} + \bm{A}_{00;i-1,2} + 2\bm{A}_{00;i,1}}{h_1h_2}+{}\\
  &\hspace*{-4mm}{}+ \fr{ -  2\bm{A}_{00;i,2} - \bm{A}_{00;i+1,1} + \bm{A}_{00;i+1,2}}{h_1h_2}\,,\enskip   0<i<i_c\,; \\
  \bm{B}_{i,j}&= \fr{\bm{A}_{00;i-1,j-1} - 2\bm{A}_{00;i-1,j} + \bm{A}_{00;i-1,j+1}}{h_1h_2}+{}\\
  &\hspace*{2mm}{}+ \fr{- 2\bm{A}_{00;i,j-1} + 4\bm{A}_{00;i,j} - 2\bm{A}_{00;i,j+1}}{h_1h_2} +{}\\
&\hspace*{3mm}{}+\fr{\bm{A}_{00;i+1,j-1} - 2\bm{A}_{00;i+1,j} + \bm{A}_{00;i+1,j+1}}{h_1h_2}\,,\\
    & \hspace*{35mm}0<i<i_c,\enskip  0<j<j_c\,; \\
  \bm{B}_{i,j_c}&=\fr{\bm{A}_{00;i-1,j_c-1} - \bm{A}_{00;i-1,j_c} - 2\bm{A}_{00;i,j_c-1}}{h_1h_2} +{} \\
 &\hspace*{1mm}{}+\fr{2\bm{A}_{00;i,j_c} + \bm{A}_{00;i+1,j_c-1} - \bm{A}_{00;i+1,j_c}}{h_1h_2}+{}\\
 &\hspace*{2mm}{}+ \fr{ -\bm{A}_{01;i-1,j_c} + \bm{A}_{01;i-1,j_c+1} + 2\bm{A}_{01;i,j_c}}{h_1h_2}+{}\\
& {}+\fr{ - 2\bm{A}_{01;i,j_c+1} - \bm{A}_{01;i+1,j_c} + \bm{A}_{01;i+1,j_c+1}}{h_1h_2} +{}\\
&\hspace*{4mm}{}+ \fr{\bm{A}_{0;i-1,\cdot} - 2\bm{A}_{0;i,\cdot} + \bm{A}_{0;i+1,\cdot}}{h_1},\enskip    0<i<i_c\,; \\
  \bm{B}_{i_c,1}&=\fr{-\bm{A}_{00;i_c-1,1} + \bm{A}_{00;i_c-1,2} + \bm{A}_{00;i_c,1}}{h_1h_2}+{}\\
  &\hspace*{5mm}{}+ \fr{ -\bm{A}_{00; i_c,2} + \bm{A}_{10;i_c,1} - \bm{A}_{10;i_c,2}}{h_1 h_2}+{}\\
  &\hspace*{-2mm}{}+\fr{ - \bm{A}_{10;i_c+1,1} +\bm{A}_{10;i_c+1,2}}{h_1h_2} +   \fr{- \bm{A}_{0;\cdot,1} + \bm{A}_{0;\cdot,2}}{h_2}\,; \\
   \bm{B}_{i_c,j}&=\fr{\bm{A}_{00;i_c-1,j-1} - 2\bm{A}_{00;i_c-1,j} + \bm{A}_{00;i_c-1,j+1}}{h_1h_2} + {}\\
   &\hspace*{2mm}{}+   \fr{-\bm{A}_{00;i_c,j-1} + 2\bm{A}_{00;i_c,j} - \bm{A}_{00;i_c,j+1}}{h_1h_2}+{}\\
   &\hspace*{3mm}{}+ \fr{ - \bm{A}_{10;i_c,j-1} + 2\bm{A}_{10;i_c,j} - \bm{A}_{10;i_c,j+1}}{h_1h_2} +{}\\
  &\hspace*{-1mm} {}+\fr{ \bm{A}_{10;i_c+1,j-1} - 2\bm{A}_{10;i_c+1,j} + \bm{A}_{10;i_c+1,j+1}}{h_1h_2} +{}\\
&\hspace*{1mm}{}+ \fr{\bm{A}_{0;\cdot,j-1} - 2\bm{A}_{0;\cdot,j} + \bm{A}_{0;\cdot,j+1}}{h_2}\,,\enskip    0<j<j_c\,; 
 \end{align*}
  
 \noindent
  \begin{align*}
  \bm{B}_{i_c,j_c}& =1 + 
  \fr{\bm{A}_{00;i_c-1,j_c-1} - \bm{A}_{00;i_c-1,j_c}}{h_1h_2}+{}\\[2pt]
  &{}+\fr{ - \bm{A}_{00;i_c,j_c-1} + 
\bm{A}_{00;i_c,j_c} - \bm{A}_{01;i_c-1,j_c}}{h_1h_2} +{}\\[2pt]
&{}+ \fr{\bm{A}_{01;i_c-1,j_c+1} + \bm{A}_{01;i_c,j_c} - \bm{A}_{01;i_c,j_c+1}}{h_1h_2}+{}\\[2pt]
&{}+ \fr{ - \bm{A}_{10;i_c,j_c-1} + \bm{A}_{10;i_c,j_c} + \bm{A}_{10;i_c+1,j_c-1}}{h_1h_2}+{}\\[2pt]
&{}+ \fr{ -\bm{A}_{10;i_c+1,j_c} + \bm{A}_{11;i_c,j_c} - \bm{A}_{11;i_c,j_c+1}}{h_1h_2}+{}\\[2pt]
&{}+\fr{ - \bm{A}_{11;i_c+1,j_c} + \bm{A}_{11;i_c+1,j_c+1}}{h_1h_2} + {}\\[2pt]
&{}+\fr{\bm{A}_{0;i_c-1,\cdot} - \bm{A}_{0;i_c, \cdot} - \bm{A}_{1;i_c,\cdot} + \bm{A}_{1;i_c+1,\cdot}}{h_1} + {}\\[2pt]
&\hspace*{3mm}{}+\fr{\bm{A}_{0;\cdot,j_c-1} - \bm{A}_{0;\cdot,j_c} - \bm{A}_{1;\cdot,j_c} + \bm{A}_{1;\cdot,j_c+1}}{h_2}\,. 
  \end{align*}
  %
  В этом списке присутствуют обозначения инструментов~$\bm{A}$ 
с~четырьмя и~тремя индексами. В~первой группе пара индексов до точки 
с~запятой означает тип двумерного $\alpha$-оп\-ци\-она~(\ref{e18-aga}), а~после 
нее~--- его страйк. Во второй группе представлены одномерные версии 
двумерных $\alpha$-оп\-ци\-онов. Индекс до точки с~запятой означает тип опциона, 
числовой индекс после нее~--- его страйк, а~позиция маркера <<точка>> 
показывает координату безрискового актива. 
  
  Сценарные баттерфляи, полученные из $\alpha$-оп\-ци\-онов, позволяют 
произвольный инструмент на рынке представить в~виде портфеля  
$\alpha$-оп\-ци\-онов. Для нахождения его доходов следует воспользоваться 
соотношениями~(\ref{e2-aga}) с~учетом переопределения~(\ref{e8-aga}). Так, 
в~однотипном случае платежная функция портфеля находится в~соответствии 
с~(\ref{e17-aga}) по правилам: 
  \begin{multline}
  \bm{U}\to 1\,,\enskip \bm{A}_{ij}\to \omega_{1;i}(x)\omega_{2;j}(y)\,,\\[2pt] 
\bm{A}_{i\cdot}\to \omega_{1;i}(x)\,,\enskip 
 \bm{A}_{\cdot j}\to \omega_{2;j}(y)\,;
  \label{e19-aga}
  \end{multline}
  
  \vspace*{-12pt}
  
  \noindent
  \begin{multline*}
  \omega_{\beta_1;i\cdot}(x)=\max \left( 0,\alpha_1(x-s_i)\right)\,,\
  i\in \bm{I};\\[2pt]
   \omega_{2;i}(y)=\max \left( 0,\alpha_2(y-t_j)\right), \enskip j\in \bm{J}\,.
 % \label{e20-aga}
  \end{multline*}
  
  Аналогично в~соответствии с~(\ref{e18-aga}) записываются в~смешанном 
случае правила формирования платежных функций: 
  \begin{multline*}
  \bm{U}\to 1\,,\enskip \bm{A}_{\beta;ij}\to \omega_{\beta_1,1;i}(x)\omega_{\beta_2,2;j}(y),\\[2pt]
  \bm{A}_{\beta_1;i,\cdot}\to \omega_{\beta_1,1;i}(x),\enskip
  \bm{A}_{\beta_2;\cdot,j}\to \omega_{\beta_2,2;j}(y)\,;
  \end{multline*}
  
  \vspace*{-12pt}
  
  \noindent
  \begin{multline}
  \omega_{\beta_1,1;i}(x)=\max \left( 0,\alpha_1(x-s_i)\right),\enskip i\in \bm{I},\\[2pt]
  \omega_{\beta_2,2;j}(y)=\max \left( 0,\alpha_2(y-t_j)\right),\enskip j\in\bm{J}.
  \label{e21-aga}
  \end{multline}
  
  \section{Иллюстративный пример}
  
  \vspace*{-1pt}
  
  Для построения мер ${\mathsf C}\{\cdot\}$ и~${\mathsf P}\{\cdot\}$ и~их 
сравнительного анализа данные в~примере заимствуются из~[6]. Так, 
принимается ${\mathsf X}\hm=[0,1)$, ${\mathsf Y}\hm=[0,1)$, а~для ${\mathsf 
F}_{\mathsf {CX}}(x)$ и~${\mathsf F}_{\mathsf{CY}}(y)$ выбираются  
бе\-та-рас\-пре\-де\-ле\-ния с~па\-ра\-мет\-ра\-ми $\{3/2,2\}$ и~$\{3/2,3\}$ 
соответственно, для ${\mathsf F}_{\mathsf PX}(x)$ и~${\mathsf F}_{\mathsf PY}(y)$~--- $\{2,3\}$ и~$\{2,4\}$:

\vspace*{-2pt}

\noindent
  \begin{align*}
  {\mathsf F}_{\mathsf {CX}} (x) &= \fr{x^{3/2}(5-3x)}{2}\,;\\  
  {\mathsf F}_{\mathsf {CY}}(y)&= \fr{y^{3/2}(35-42y+15y^2)}{8}\,;\\
  {\mathsf F}_{\mathsf {PX}}(x) &= x^2\left(6-8x+3x^2\right);\\
  {\mathsf F}_{\mathsf {PY}}(y)&= y^2\left( 10-20y+15y^2-4y^3\right).
  \end{align*}
  %
  
  \vspace*{-2pt}
  
  \noindent
  Из них совместные функции распределения для обеих мер строятся как
  
  \vspace*{-5pt}
  
  \noindent 
  \begin{multline}
  {\mathsf F}(x,y)={}\\
\hspace*{-3mm}  {}= {\mathsf F}_{\mathsf X}(x) {\mathsf F}_{\mathsf Y}(y) \left( 
1+3\kappa \left( 1-{\mathsf F}_{\mathsf X}(x)\right) \left(1-{\mathsf F}_{\mathsf 
Y}(y)\right)\right).
  \label{e22-aga}
  \end{multline}
  
\vspace*{-1pt}
 
  Искомые двумерные функции распределения ${\mathsf F}_{\mathsf C}(x,y)$ 
и~${\mathsf F}_{\mathsf P} (x,y)$ определяются подстановкой в~(\ref{e22-aga}) 
в~качестве параметра, отвечающего за корреляционную связь компонент, 
соответственно $\kappa_c\hm=0$ и~$\kappa_p\hm=0{,}2$. Из них простым 
смешанным дифференцированием по обеим переменным находятся плотности 
$c(x,y)$ и~$p(x,y)$, но ввиду громоздкости записей они здесь не приводятся. 
  
  Двумерная дискретизация множества ${\mathsf X}\times \mathsf{Y}$ 
в~примере проводится также при $v_1\hm=6$ и~$v_2\hm=5$, а~центральным 
выбирается страйк $i_c\hm=3$, $j_c\hm=3$. 
  
  В отличие от $\zeta$-рын\-ков~[6], для которых в~целях применения 
дискретного алгоритма находились стоимости сценарных индикаторов, для  
\mbox{$\alpha$-рын}\-ков следует вычислять стоимости сценарных баттерфляев. И~потому 
для адекватного сравнения относительных доходов естественно вычислять и~их 
средние доходы. И~те и~другие, а~это векторы $\bm{c}^B$ и~$\bm{p}^B$, 
определяются интегрированием платежных функций~(\ref{e4-aga})  
и~(\ref{e6-aga}) с~плотностями $c(x,y)$ и~$p(x,y)$ соответственно. 
  
  Применением к~этим векторам дискретного алгоритма оптимизации~[6], 
основанного на процедуре Ней\-ма\-на--Пир\-со\-на~\cite{7-aga}, определяется 
вектор весов базисных баттерфляев для оптимального двумерного портфеля. 
При этом в~качестве функции рисковых предпочтений выбирается 
$\varphi(\varepsilon)\hm=\varepsilon^2$, $\varepsilon\hm\in [0,1]$.

\begin{figure*} %fig1
\vspace*{1pt}
\begin{minipage}[t]{80mm}
\begin{center}
   \mbox{%
\epsfxsize=77.328mm
\epsfbox{aga-1.eps}
}
\end{center}
\vspace*{-9pt}
\Caption{Доходы оптимального опционного портфеля при дискретизации $6\times5$}
\end{minipage}
%\end{figure*}
\hfill
%\begin{figure*} %fig2
\vspace*{1pt}
\begin{minipage}[t]{80mm}
\begin{center}
   \mbox{%
\epsfxsize=77.328mm
\epsfbox{aga-2.eps}
}
\end{center}
\vspace*{-9pt}
\Caption{Доходы сценарного портфеля при дискретизации $40\times40$}
\end{minipage}
\end{figure*}

  
  В результате  получается вектор весов 
  
  \vspace*{-4pt}
  
  \noindent
 \begin{multline*}
  \bm{g}=\{0{,}118; 0{,}159; 0{,}0113; 0{,}000219; 0{,}000008; 0{,}414;\\
   1{,}0; 0{,}228; 0{,}0151; 0{,}000989; 0{,}0739; 0{,}788; 0{,}602;\\
      0{,}0873; 0{,}00175; 0{,}0069; 0{,}309; 0{,}495; 0{,}176; 0{,}00191;
\end{multline*}

%\vspace*{-3pt}

%\pagebreak

\noindent
 \begin{multline*}
   \hspace*{-4.8848pt}0{,}000907; 0{,}0254; 0{,}0405; 0{,}0291; 0{,}00159; 0{,}0000058;\\
   0{,}0000848; 0{,}00151; 0{,}00112;  0{,}000009\}. 
  \end{multline*}
  
  \vspace*{-2pt}

  
  Он порождает оптимальный портфель~(\ref{e12-aga}) с~инвестиционной 
суммой, средним доходом и~средней доходностью соответственно 
  $A\hm=0{,}28642$,  $R\hm=0{,}364418$  и~$y\hm=0{,}272317$. 
      График его платежной функции изображен на рис.~1. Для сравнения на 
рис.~2 приведен аналогичный график для сценарного рынка при дискретизации 
$40\times40$.
  

  По понятным причинам графики платежных функций на рис.~1 и~2 
демонстрируют большее взаимное сходство, чем аналогичная пара графиков 
из~[6]. Но и~различие между собой графиков на рис.~1 и~2 пред\-став\-ля\-ет\-ся 
естественным. 
  
  Остается определить оптимальные портфели в~терминах $\alpha$-оп\-ци\-онов 
рас\-смат\-ри\-ва\-емых типов. Для нахождения каждого такого пред\-став\-ле\-ния 
в~формулу~(\ref{e12-aga}) следует для всех пар $(i,j)$ под\-став\-лять вмес\-то 
индикаторов~$\bm{B}_{ij}$ со\-от\-вет\-ст\-ву\-ющие им пред\-став\-ле\-ния  
в~$\alpha$-оп\-ци\-онах. В~результате после упрощений получаются четыре 
однотипных портфеля и~один смешанный. Так, оптимальный 
\textit{однотипный} портфель для $\bm{\alpha}\hm= \{-1, +1\}$, т.\,е.\ образованный 
путами для первого актива и~коллами~--- для второго: 

\vspace*{-2pt}

\noindent
 \begin{multline*}
  \bm{G}_{01}=0{,}000006 \bm{U} + 16{,}369\bm{A}_{11} - 35{,}107\bm{A}_{12} + {}\\
  {}+12{,}681\bm{A}_{13} + 5{,}641\bm{A}_{14} + 0{,}416\bm{A}_{15} + 1{,}774\bm{A}_{1\cdot} - {}\\
{}-12{,}549\bm{A}_{21} + 48{,}875\bm{A}_{22} - 39{,}318\bm{A}_{23} + 1{,}263\bm{A}_{24} + {}\\
{}+ 1{,}729\bm{A}_{25} - 3{,}812\bm{A}_{2\cdot} - 16{,}162\bm{A}_{31} + 9{,}717\bm{A}_{32} + {}\\
{}+21{,}364\bm{A}_{33} - 15{,}431\bm{A}_{34} + 0{,}512\bm{A}_{35} + 1{,}636\bm{A}_{3\cdot} + {}\\
{}+ 4{,}007\bm{A}_{41} - 20{,}268\bm{A}_{42} + 19{,}612\bm{A}_{43} + 3{,}707\bm{A}_{44} - {}\\
{}-7{,}058\bm{A}_{45} + 0{,}366\bm{A}_{4\cdot} + 7{,}603\bm{A}_{51} - 2{,}899\bm{A}_{52} - {}\\
{}- 13{,}593\bm{A}_{53} + 5{,}279\bm{A}_{54} + 3{,}609\bm{A}_{55} + 0,031\bm{A}_{5\cdot} + {}\\
{}+ 0{,}731\bm{A}_{61} - 0{,}318\bm{A}_{62} - 0{,}745\bm{A}_{63} - 0{,}458\bm{A}_{64} + {}
\end{multline*}

\noindent
\begin{multline*}
{}+0{,}791\bm{A}_{65} + 0{,}005\bm{A}_{6\cdot} + 0{,}0004\bm{A}_{\cdot1} + 0{,}007\bm{A}_{\cdot 2} -{}\\
{}- 0{,}009\bm{A}_{\cdot 3} - 0{,}004\bm{A}_{\cdot 4} + 0{,}006\bm{A}_{\cdot 5}.
\end{multline*}

\vspace*{-2pt}


  
  Платежные функции всех однотипных портфелей получаются по 
правилам~(\ref{e19-aga}). Проведенные расчеты под\-тверж\-да\-ют вер\-ность 
алгоритма. Все они, несмотря на внешнее различие их пред\-став\-ле\-ний, на 
идеальном рынке должны иметь единую платежную функцию с~графиком, 
пред\-став\-лен\-ным на рис.~1. 
  
  Оптимальный \textit{смешанный портфель} строится вновь по 
формуле~(\ref{e12-aga}), но в~смешанном базисе\linebreak естественного происхождения с~выделенным 
цент\-раль\-ным страйком~$(3, 3)$. В~первом квад\-ран\-те\linebreak 
(относительно цент\-ра рынка) используются $\alpha$-оп\-ци\-оны~$\bm{A}_{11}$, во 
втором~--- $\bm{A}_{01}$, в~треть\-ем~--- $\bm{A}_{00}$, в~\mbox{чет\-вер\-том}~--- 
$\bm{A}_{10}$. 
  
  Вычисления с~применением~(\ref{e12-aga}) дают оптимальный 
\textit{смешанный} портфель:

\vspace*{-2pt}
 

\noindent
\begin{multline*}
  \bm{G}_m=0{,}602\bm{U} + 16{,}369\bm{A}_{00;11} - 35{,}107\bm{A}_{00;12} + {}\\
  {}+
18{,}737\bm{A}_{00;13} - 12{,}549\bm{A}_{00;21} + 48{,}876\bm{A}_{00;22} - {}\\
{}-
36{,}326\bm{A}_{00;23} - 3{,}82\bm{A}_{00;31} - 13{,}769\bm{A}_{00;32} +{}\\
{}+ 17{,}589\bm{A}_{00;33} - 
6{,}056\bm{A}_{01;13} + 5{,}641\bm{A}_{01;14} +{}\\
{}+ 0{,}416\bm{A}_{01;15} - 2{,}992\bm{A}_{01;23} + 
1{,}263\bm{A}_{01;24} + {}\\
{}+1{,}729\bm{A}_{01;25} + 9{,}049\bm{A}_{01;33} - 6{,}903\bm{A}_{01;34} - {}\\
{}-2{,}145\bm{A}_{01;35} - 12{,}342\bm{A}_{10;31} + 23{,}486\bm{A}_{10;32} - {}\\
{}-11{,}144\bm{A}_{10;33} + 4{,}007\bm{A}_{10;41} - 20{,}268\bm{A}_{10;42} + {}\\
{}+16{,}261\bm{A}_{10;43} + 7{,}603\bm{A}_{10;51} - 2{,}899\bm{A}_{10;52} -{}\\
{}-4{,}704\bm{A}_{10;53} +  0{,}731\bm{A}_{10;61} - 0{,}318\bm{A}_{10;62} - {}\\
{}-0{,}413\bm{A}_{10;63} + 5{,}871\bm{A}_{11;33} - 8{,}528\bm{A}_{11;34} +{}\\
{}+ 2{,}657\bm{A}_{11;35} + 3{,}351\bm{A}_{11;43} + 3{,}707\bm{A}_{11;44} - {}\\
{}-7{,}058\bm{A}_{11;45} - 8{,}889\bm{A}_{11;53} + 5{,}279\bm{A}_{11;54} +{}\\
{}+ 3{,}609\bm{A}_{11;55} - 0{,}333\bm{A}_{11;63} - 0{,}458\bm{A}_{11;64} +{}
\end{multline*}

\noindent
\begin{multline*}
{}+ 0{,}791\bm{A}_{11;65} + 1{,}3\bm{A}_{0;1\cdot} + 0{,}943\bm{A}_{0;2\cdot} - 2{,}243\bm{A}_{0;3\cdot} +{}\\
{}+ 3{,}568\bm{A}_{0;\cdot 1} - 4{,}497\bm{A}_{0;\cdot 2} + 0{,}929\bm{A}_{0;\cdot 3} - 0{,}642\bm{A}_{1;3\cdot} - {}\\
{}-
2{,}085\bm{A}_{1;4\cdot} + 2{,}492\bm{A}_{1;5\cdot} + 0{,}234\bm{A}_{1;6\cdot} - 
2{,}573\bm{A}_{1;\cdot 3} +{}\\
{}+ 2{,}145\bm{A}_{1;\cdot 4} + 0{,}428\bm{A}_{1;\cdot 5}. 
  \end{multline*}
  
  \vspace*{-2pt}
  
\noindent
  В этом портфеле 56~инструментов; среди них один безрисковый актив, 
42~двумерных $\alpha$-оп\-ци\-она~$\bm{A}_{00}$, $\bm{A}_{01}$, $\bm{A}_{10}$ 
и~$\bm{A}_{11}$ и~13~одномерных версий~$\bm{A}_0, \bm{A}_1$ в~количествах 
9, 9, 12, 12 и~6, 7 соответственно. 
  
  Двумерные $\alpha$-оп\-ци\-оны снабжены четырьмя индексами; первые два из 
них (до точки с~запятой) показывают тип опциона по каждой координате 
в~терминах~$\beta$, другие два индекса~--- номера страй\-ков. 
  
  Одномерные версии снабжены двумя индексами и~маркером <<точка>>. 
Один индекс до точки с~запятой указывает тип опциона, индекс после нее~---  
номер страй\-ка, а~маркер~--- координату под\-ра\-зу\-ме\-ва\-емо\-го безрискового 
актива.
  
  Платежная функция смешанного портфеля строится по  
правилам~(\ref{e21-aga}). Ее графиком служит все тот же график на рис.~1. 
  
  \section{Заключение }
  
  В работе решена задача алгоритмического на\-хож\-де\-ния пред\-став\-ле\-ний 
многомерных баттерфляев при сценарной дискретизации рынка и~\mbox{по\-стро\-ения} 
из них базиса в~терминах $\alpha$-оп\-ци\-онов~--- многомерного обобщения 
традиционных опционов колл и~пут. На конкретном примере двумерного рынка 
продемонстрирована работа этого алгоритма и~ее результат. Подобные расчеты 
могут быть без принципиальных трудностей реализованы и~для рынков 
большей размерности. Поскольку бат\-тер\-фляи для одномерного рынка 
образуются из трех страй\-ков, а~индикаторы~--- из двух, то их многомерные 
реп\-ли\-ка\-ции получаются еще более громоздкими. Фактические расчеты, 
проведенные для $n\hm=4$, показали, что уже объем перечня базисных 
инструментов однотипного базиса, например в~терминах~$\bm{A}_{0000}$, 
аналогичного~$\bm{A}_{00}$ из разд.~4, был соразмерен всему объему 
на\-сто\-ящей работы, а~потому и~на многомерных $\alpha$-рын\-ках лучше торговать 
непосредственно бат\-тер\-фля\-ями, а~не $\alpha$-оп\-ци\-онами. 
  
{\small\frenchspacing
 {%\baselineskip=10.8pt
 %\addcontentsline{toc}{section}{References}
 \begin{thebibliography}{9}
  \bibitem{1-aga}
  \Au{Агасандян~Г.\,А.} Применение континуального критерия VaR на 
финансовых рынках.~--- М.: ВЦ РАН, 2011. 299~с. 
  \bibitem{2-aga}
  \Au{Агасандян~Г.\,А.} Континуальный критерий VaR и~оптимальный 
портфель инвестора~// Управ\-ле\-ние большими сис\-те\-ма\-ми, 2018. Вып.~73. 
С.~6--26.
  \bibitem{3-aga}
  \Au{Агасандян~Г.\,А.} Континуальный критерий VaR на сценарных рынках~// 
Информатика и~её применения, 2018. Т.~12. Вып.~1. С.~32--40. 
  \bibitem{4-aga}
  \Au{Агасандян~Г.\,А.} Вычисление показателей оптимальных по CC-VaR 
портфелей на рынках опционов~// Информатика и~её применения, 2019. Т.~13. 
Вып.~3. С.~75--84. 
  \bibitem{5-aga}
  \Au{Агасандян~Г.\,А.} Многомерные рынки опционов и~оптимизация по  
CC-VaR~// Управ\-ле\-ние большими сис\-те\-ма\-ми, 2020. Вып.~88. С.~5--25.
  \bibitem{6-aga}
  \Au{Агасандян~Г.\,А.} Многомерные бинарные рынки и~CC-VaR~// 
Информатика и~её применения, 2022. Т.~16. Вып.~2. С.~2--10.
  \bibitem{7-aga}
  \Au{Крамер~Г.} Математические методы статистики~/ Пер. с~англ.~--- М.: 
Мир, 1975. 750~с. (\Au{Cramer~H.} Mathematical methods of statistics.~--- 
Princeton, NJ, USA: Princeton University Press, 1946. 575~p.)
\end{thebibliography}

 }
 }

\end{multicols}

\vspace*{-6pt}

\hfill{\small\textit{Поступила в~редакцию 09.03.22}}

\vspace*{8pt}

%\pagebreak

%\newpage

%\vspace*{-28pt}

\hrule

\vspace*{2pt}

\hrule

%\vspace*{-2pt}

\def\tit{MULTIDIMENSIONAL BUTTERFLIES IN~PROBLEMS 
OF~OPTIMIZATION ON CC-VaR}


\def\titkol{Multidimensional butterflies in~problems 
of~optimization on CC-VaR}


\def\aut{G.\,A.~Agasandyan}

\def\autkol{G.\,A.~Agasandyan}

\titel{\tit}{\aut}{\autkol}{\titkol}

\vspace*{-8pt}


\noindent
Federal Research Center ``Computer Science and Control'' of the Russian Academy 
of Sciences, 44-2~Vavilov Str., Moscow 119333, Russian Federation


\def\leftfootline{\small{\textbf{\thepage}
\hfill INFORMATIKA I EE PRIMENENIYA~--- INFORMATICS AND
APPLICATIONS\ \ \ 2023\ \ \ volume~17\ \ \ issue\ 1}
}%
 \def\rightfootline{\small{INFORMATIKA I EE PRIMENENIYA~---
INFORMATICS AND APPLICATIONS\ \ \ 2023\ \ \ volume~17\ \ \ issue\ 1
\hfill \textbf{\thepage}}}

\vspace*{3pt} 
  
  
  


  
  \Abste{The work continues studying problems of using continuous VaR-criterion (CC-VaR) in 
financial markets. Again some technical problems are concerned. However, they emerge this time 
not in multidimensional relatively simple binary markets but in multidimensional markets that are 
an extension of one-dimensional traditional
markets of options such as calls and puts. In assumption 
that scenario butterflies are not traded in markets directly, a~method of receiving their replication 
from multidimensional options, i.\,e., $\alpha$-options, is developed. It is based on options parity 
theorems and can be applied to markets of arbitrary dimension, but actual realization is conducted
for two-dimensional markets. The bases constructions in terms of $\alpha$-options both one-type and 
natural mixed with\linebreak\vspace*{-12pt}}

\Abstend{selected market center are produced. Theoretical representations of optimal 
portfolios in these bases accompanied with the payoffs diagram are illustrated by the distinctive 
example of a two-dimensional market.}
  
  \KWE{underliers; multidimensional market; investor's risk preferences function; continuous 
VaR-criterion; cost and forecast densities; scenario indicators; bases; binary options; one-type 
portfolio; market center; mixed portfolio}
  
 \DOI{10.14357/19922264230114} 

%\vspace*{-16pt}

%\Ack
%\noindent

  

%\vspace*{4pt}

  \begin{multicols}{2}

\renewcommand{\bibname}{\protect\rmfamily References}
%\renewcommand{\bibname}{\large\protect\rm References}

{\small\frenchspacing
 {%\baselineskip=10.8pt
 \addcontentsline{toc}{section}{References}
 \begin{thebibliography}{9} 
  \bibitem{1-aga-1}
  \Aue{Agasandyan, G.\,A.} 2011. \textit{Pri\-me\-ne\-nie kon\-ti\-nu\-al'\-no\-go kri\-te\-riya VaR 
na fi\-nan\-so\-vykh ryn\-kakh} [Application of continuous VaR-criterion in financial 
markets]. Moscow: CCRAS. 299~p.
  \bibitem{2-aga-1}
  \Aue{Agasandyan, G.\,A.} 2018. Kon\-ti\-nu\-al'\-nyy kri\-te\-riy VaR i~op\-ti\-mal'\-nyy 
port\-fel' in\-ves\-to\-ra [Continuous VaR-criterion and investor's optimal portfolio]. 
\textit{Upravlenie bol'shimi sistemami} [Large-Scale Systems Control] 73:6--26.
  \bibitem{3-aga-1}
  \Aue{Agasandyan, G.\,A.} 2018. Kon\-ti\-nu\-al'\-nyy kri\-te\-riy VaR na stse\-nar\-nykh 
ryn\-kakh [Continuous VaR-criterion in scenario markets]. \textit{Informatika i~ee 
Primeneniya~--- Inform. Appl.} 12(1):32--40.
  \bibitem{4-aga-1}
  \Aue{Agasandyan, G.\,A.} 2019. Vy\-chis\-le\-nie po\-ka\-za\-te\-ley op\-ti\-mal'\-nykh po  
CC-VaR port\-fe\-ley na ryn\-kakh op\-tsi\-o\-nov [Performance estimations for  
optimal-on-CC-VaR portfolios in option markets]. \textit{Informatika i~ee 
Primeneniya~--- Inform. Appl.} 13(3):75--84.
  \bibitem{5-aga-1}
  \Aue{Agasandyan, G.\,A.} 2020. Mno\-go\-mer\-nye ryn\-ki op\-tsi\-o\-nov i~op\-ti\-mi\-za\-tsiya 
po CC-VaR [Multidimensional option markets and optimization on CC-VaR]. 
\textit{Upravlenie bol'shimi sistemami} [Large-Scale Systems Control] 88:5--25.
  \bibitem{6-aga-1}
  \Aue{Agasandyan, G.\,A.} 2022. Mno\-go\-mer\-nye bi\-nar\-nye ryn\-ki i~CC-VaR 
[Multidimensional binary markets and CC-VaR]. \textit{Informatika i~ee 
Primeneniya~--- Inform. Appl.} 16(2):2--10.
  \bibitem{7-aga-1}
  \Aue{Cramer, H.} 1946. \textit{Mathematical methods of statistics}. Princeton, 
NJ: Princeton University Press. 575~p.

\end{thebibliography}

 }
 }

\end{multicols}

\vspace*{-6pt}

\hfill{\small\textit{Received March 9, 2022}}

  
  \Contrl
  
  \noindent
  \textbf{Agasandyan Gennady A.} (b.\ 1941)~--- Doctor of Science in physics and 
mathematics, leading scientist, A.\,A.~Dorodnicyn Computing Center, Federal 
Research Center ``Computer Science and Control'' of the Russian Academy of 
Sciences, 40~Vavilov Str., Moscow 119333, Russian Federation; 
\mbox{agasand17@yandex.ru}
  

   
\label{end\stat}

\renewcommand{\bibname}{\protect\rm Литература} 
       %14+
\def\stat{zatsman}

\def\tit{ТРАНСФОРМАЦИИ ОБЪЕКТОВ ПЕРВОГО И~ВТОРОГО ПОРЯДКА 
В~ЛЕКСИКОГРАФИЧЕСКОЙ ИНФОРМАЦИОННОЙ СИСТЕМЕ$^*$}

\def\titkol{Трансформации объектов первого и~второго порядка 
в~лексикографической информационной системе}

\def\aut{И.\,М.~Зацман$^1$}

\def\autkol{И.\,М.~Зацман}

\titel{\tit}{\aut}{\autkol}{\titkol}

\index{Зацман И.\,М.}
\index{Zatsman I.\,M.}


{\renewcommand{\thefootnote}{\fnsymbol{footnote}} \footnotetext[1]
{Исследование выполнено в~ФИЦ ИУ РАН за счет гранта Российского научного фонда №\,24-18-00155, {\sf 
https://rscf.ru/project/24-18-00155}. Работа выполнялась с~использованием инфраструктуры Центра 
коллективного пользования <<Высокопроизводительные вычисления и~большие данные>> (ЦКП 
<<Информатика>>) ФИЦ ИУ РАН (г.\ Москва).}}


\renewcommand{\thefootnote}{\arabic{footnote}}
\footnotetext[1]{ Федеральный исследовательский центр <<Информатика и~управление>> Российской академии наук, 
\mbox{izatsman@yandex.ru}}

\vspace*{-12pt}


  
  \Abst{Рассматриваются теоретические основания проектирования информационных 
технологий (ИТ) интеграции двуязычных словарей и~параллельных корпусов. Дано описание 
первых результатов создания третьего уровня классификации трансформаций объектов 
предметной области информатики, которую предполагается использовать при создании 
концепции лексикографической информационной системы, обеспечивающей интеграцию. 
Все сущности информатики в~статье разделены на два глобальных класса: объекты и~их 
трансформации. Для каждого такого класса конструируется своя классификация. Ранее были 
описаны два верхних уровня классификации трансформаций объектов предметной области. 
В~данной статье рассматривается третий уровень этой классификации. Основанием для 
построения самого верхнего ее уровня служило деление предметной области информатики 
на среды (ментальная, сенсорно воспринимаемая, цифровая и~ряд других сред), каждая из 
которых по определению включает объекты одной природы. Основанием для построения 
второго уровня классификации трансформаций объектов служила типология знаковых  
сис\-тем А.~Соломоника. Цель статьи состоит в~систематизации трансформаций первого 
и~второго порядка объектов предметной области на третьем уровне этой классификации. 
Основанием для систематизации служит средовая версия иерархии Акоффа.}
  
  \KW{объекты предметной области; трансформации объектов; классификация; данные; 
информация; знание; лексикографическая информационная сис\-тема}

\DOI{10.14357/19922264240211}{VZTGVV}
  
\vspace*{3pt}


\vskip 10pt plus 9pt minus 6pt

\thispagestyle{headings}

\begin{multicols}{2}

\label{st\stat}
  
\section{Введение}

\vspace*{-9pt}

  Возникновение параллельных корпусов, в~которых предложениям 
оригинального текста со\-по\-став\-ле\-ны предложения его перевода, обеспечило 
возможность контрастивного лингвистического\linebreak \mbox{анализа} на принципиально 
новом уровне полноты и~точности, недостижимом в~докорпусную эпоху. 
Пионерскими в~этой области стали работы \mbox{1990-х~гг}. Стига Йоханссона  
с~анг\-ло-нор\-веж\-ским корпусом~[1]. В России параллельные корпусы стали 
формироваться в~начале XXI~века в~рамках Национального корпуса русского 
языка~[2].
  
  Создатели двуязычных словарей используют параллельные корпусы для 
сбора материала и~эмпирической проверки своих гипотез, касающихся 
межъязы\-ко\-вой эквивалентности. Ценность параллельных корпусов 
определяется тем, что в~лингвистике этап сбора исходного материала считается 
наиболее трудоемким и~наименее творческим, а~параллельные корпусы 
позволяют значительно сэкономить время и~силы для творческого этапа 
создания словарей~[3].
 % 
  При этом двуязычные словари, создаваемые на основе исходного материала, 
извлеченного из параллельных корпусов, сейчас формируются без связей с~их 
текстами. Другими словами, онлайновые связи созданных словарей 
с~параллельными корпусами, которые служили источниками исходного 
материала, отсутствуют. 

Параллельные корпусы постоянно пополняются 
новыми текстами, в~предложениях которых можно обнаружить новые значения 
слов и~устойчивых словосочетаний. Однако при этом отсутствуют методы 
и~средства оперативного обновления словарей по корпусным данным. 
В~настоящее время проблема установления связей между двуязычными 
словарями и~параллельными корпусами (далее~--- проблема интеграции) 
находится на стадии поиска концептуальных подходов к~их интеграции на 
уровне значений.
  
  Подход к~решению проблемы интеграции, предлагаемый в~статье, учитывает 
  и~появление новых значений слов и~устойчивых словосочетаний, и~динамику 
смысловых значений, которая обусловлена развитием и~пополнением знания 
лингвистов, фиксирующих эти значения в~результате семантического анализа 
пополняемых корпусных данных. Проведенные эксперименты показали, что 
обнаружение нового лингвистического знания обусловливает и~формирование 
дефиниций новых значений, и~пересмотр уже существующих дефиниций~[4, 5].
  
  Например, в~проведенных экспериментах с~использованием ЦКП 
<<Информатика>> ФИЦ ИУ РАН фиксировалась эволюция значений немецких 
модальных глаголов, исходное состояние значений которых было описано 
в~не\-мец\-ко-рус\-ском словаре. В~экспериментальном массиве текстов как 
потенциальных источниках нового знания 16\,268 предложений содержали 
немецкие модальные глаголы и~в~2041 из них встречался глагол sollen. 
В~начале эксперимента в~словаре были описаны~12~значений этого модального 
глагола. По окончании эксперимента лингвисты обнаружили два новых его 
значения, согласовали их дефиниции и~описали эволюцию дефиниций~[6, 7].
  
  Таким образом, для решения проблемы интеграции требуется фиксировать 
новое знание, обнаруженное лингвистами в~текстовых данных параллельных 
корпусов, отслеживать эволюцию знания, представленного в~виде дефиниций 
значений слов и~устойчивых словосочетаний, и,~соответственно, 
актуализировать электронные двуязычные словари. Предлагаемый 
концептуальный подход к~интеграции, который планируется реализовать 
в~процессе проектирования лексикографической информационной сис\-те\-мы, 
фиксирующей эволюцию лингвистического знания, основан на решении 
следующих задач:\\[-14pt]
  \begin{itemize}
  \item категоризация трех базовых понятий информатики, включенных 
  в~иерархию Акоффа~[8] (данные, информация, знание), на объекты 
проектируемой сис\-те\-мы, которая необходима, чтобы фиксировать 
<<кванты>> нового знания и~отслеживать его эволюцию в~этой сис\-теме;\\[-15pt]
  \item  систематизация трансформаций объектов этой сис\-темы.\\[-14pt]
  \end{itemize}
  
  Цель статьи и~состоит в~решении двух задач: категоризации трех базовых 
понятий информатики на объекты лексикографической информационной  
сис\-те\-мы и~сис\-те\-ма\-ти\-за\-ции трансформаций первого и~второго порядка 
ее объектов.
  
  Трансформациями первого порядка, о которых сказано в~формулировке цели 
статьи, называются взаимные преобразования между двумя объектами  
сис\-те\-мы одной природы. Например, перевод в~сис\-те\-ме текста с~русского 
языка на английский относится к~ним. Трансформациями второго порядка 
и~выше называются взаимные преобразования между двумя и~более объектами 
разной природы. Например, кодирование символов текс\-та компьютерными 
кодами и~их декодирование относятся по определению к~трансформациям 
второго порядка.

%\vspace*{-9pt}
  
\section{Процессы трансформаций в~информатике}

%\vspace*{-3pt}

Процессы трансформаций, рассматриваемые в~статье, относятся к~теоретическому ядру информатики, а~не 
только к~проектированию лексикографической информационной сис\-те\-мы. Например, из трех основных 
подходов к~описанию предметной об\-ласти информатики\footnote{В статье предметная область информатики 
трактуется согласно концепции полиадического компьютинга Пола Розенблума~\cite{9-zac}.} (объектный, 
трансформационный и~синтетический) сис\-те\-ма\-ти\-за\-ция трансформаций ближе всего ко второму 
подходу. Примерами первого подхода, в~рамках которого основное внимание уделяется объектам предметной 
области информатики и~в~меньшей степени отношениям\linebreak между ними, могут служить  
работы~\cite{8-zac, 10-zac, 11-zac}; \mbox{примерами} второго подхода, в~рамках которого основное внимание 
уделяется трансформациям и~в~меньшей степени трансформируемым объектам,~---  
работы~\cite{12-zac, 13-zac}; примерами третьего, синтетического подхода, в~котором уделяется внимание 
и~объектам предметной об\-ласти информатики, и~отношениям между ними, могут служить работы~\cite{14-zac, 
15-zac, 16-zac, 17-zac, 18-zac}.

  Таким образом, для описания трансформаций объектов лексикографической 
информационной\linebreak системы предпочтительнее всего трансформационный 
подход, который упоминается и~в определениях информатики. Например, 
в~2009~г.\ П.~Деннинг и~П.~Розенблум сформулировали суть \mbox{информатики} как 
компьютинга следующим образом: <<$\ldots$информатика~--- это не просто 
алгоритмы и~структуры данных; это преобразования [трансформации] 
представлений>>~\cite{12-zac}. Чуть позже, в~контексте краткого описания 
парадигмы информатики как компьютинга, П.~Деннинг и~П.~Фриман изменили 
эту формулировку на такую: <<Центральный объект внимания в~информатике 
можно определить как информационные процессы~--- \textit{естественные или 
искусственные процессы, преобразующие информацию} (курсив мой~--- 
И.\,З.)>>~\cite{13-zac}. Согласно парадигме, предлагаемой авторами этой 
статьи, на начальном этапе проектирования автоматизированных систем 
базовыми элементами моделей их функционирования служат 
\textit{информационные про\-цессы}.
  
  Однако если 15~лет назад в~формулировке из работы~\cite{13-zac} шла речь 
о~процессах, преобразующих информацию, то в~последние~10~лет в~спектр 
процессов трансформаций все чаще стали включать процессы, преобразующие 
не только информацию, но также и~другие объекты автоматизированных 
систем, в~первую очередь данные и~знания~[19--21]. Например, Виктория 
Стодден, позиционируя науку о~данных как одну из дисциплин информатики, 
говорит, что центральный объект исследований в~науке о~данных~--- это 
<<изучение обобщаемого извлечения знания из данных>>~\cite{21-zac}. 
Увеличение и~чис\-ла объектов, и~спект\-ра процессов их трансформаций 
в~автоматизированных сис\-те\-мах обуслов\-ли\-ва\-ет не\-об\-хо\-ди\-мость 
систематизации и~объектов, и~процессов их трансформаций на начальном этапе 
проектирования сис\-тем.
  
  Для создания концепции лексикографической информационной сис\-те\-мы 
и~проектирования ИТ, обеспечивающих интеграцию 
двуязычных словарей и~параллельных корпусов, сначала выполним 
категоризацию на объекты этой сис\-те\-мы трех базовых понятий информатики 
(данные, информация, знание) в~контексте построения классификаций 
сущностей ее предметной об\-ласти.
  
  Необходимость использования классификаций информатики в~процессе 
создания концепции проиллюстрируем, используя иерархию  
Акоффа~\cite{8-zac}. Он использовал принцип их вертикального размещения 
в~иерархии снизу вверх: данные, информация и~знание. Еще в~ней есть термин 
<<мудрость>>, который в~статье не рассматривается. Такое размещение Акофф 
прокомментировал так: <<Каждое из пе\-ре\-чис\-лен\-ных понятий [кроме данных] 
содержит в~себе нижестоящие$\ldots$>>~\cite{8-zac}.
  
  Этому принципу размещения и~комментарию Акоффа свойственны 
недостатки, проанализированные, в~частности, в~работе~\cite{10-zac}. Главный 
вывод, к~которому пришла Роули после изучения иерархии Акоффа, 
заключается в~следующем: <<$\ldots$информация определяется в~терминах 
данных, знание~--- в~терминах информации$\ldots$ но существует меньше 
консенсуса в~описании трансформаций, которые преобразуют сущности, 
расположенные ниже в~иерархии, в~те, которые находятся над ними, что 
приводит к~их терминологической неопределенности>>~\cite{10-zac}. Причина 
этой неопределенности, скорее всего, в~том, что базовые понятия информатики 
включены в~иерархию Акоффа изолированно от общего контекста 
классификаций сущностей ее предметной об\-ласти.

%\vspace*{-9pt}
  
\section{Классификации сущностей информатики}


%\vspace*{-2pt}

  Все сущности предметной области информатики в~работах~[22, 23] 
разделены на два глобальных класса: ее объекты и~их трансформации. Для 
каждого такого класса была предложена своя классификация. 
В~работе~\cite{22-zac} дано описание классификации объектов предметной 
области информатики, первый уровень которой содержит базовые понятия ее 
предметной области (данные, информация, знания и~др.).  
В~работе~\cite{23-zac} дано описание двух верхних уровней классификации 
трансформаций объектов предметной об\-ласти (см.\ рисунок 
в~работе~\cite{23-zac}). Основанием для построения самого верхнего ее уровня послужило деление 
предметной области информатики на среды\footnote{В~работе~\cite{24-zac} дано описание пяти сред 
предметной области информатики (ментальная; сенсорно воспринимаемая, или информационная; 
цифровая; нейро- и~ДНК-среда), каждая из которых по определению включает объекты одной и~той же 
природы.} и~степень разнообразия природы объектов, вовлеченных в~трансформации:
\begin{itemize}
\item  первый класс верхнего уровня классификации включает 
трансформации объектов в~пределах среды только одной природы 
(трансформации первого порядка);
\item  второй класс включает трансформации объектов, относящихся 
к~двум средам разной природы (трансформации второго порядка);
\item третий и~последующие классы включают трансформации объектов, 
относящихся к~трем и~более средам разной природы (трансформации 
третьего и~более высоких порядков).
\end{itemize}

  В работе~\cite{23-zac} были приведены примеры для трех первых классов 
трансформаций, включая пример трансформаций объектов, относящихся 
к~двум средам разной природы (компьютерное кодирование символов текстов 
с~по\-мощью таб\-лиц Unicode).
  
Основанием для построения второго уровня классификации трансформаций объектов послужила типология 
знаковых сис\-тем А.~Соломоника~\cite[c.~131]{25-zac}: естественные знаковые сис\-те\-мы, образные,  
ес\-тест\-вен\-но-язы\-ко\-в$\acute{\mbox{ы}}$е,  
вер\-баль\-но-не\-сло\-вес\-ные сис\-те\-мы записи\footnote{Под системой записи понимается знаковая 
система, сочетающая вербальные знаки с~несловесными (языки нотной записи, карт, таблиц и~др.).} 
и~формализованные знаковые сис\-те\-мы, включая математические. Введем понятие обобщенного текста~--- 
это текст, который может быть создан в~любой из перечисленных знаковых систем. Тогда обобщенные тексты 
могут быть естественными, образными, ес\-тест\-вен\-но-язы\-ко\-в$\acute{\mbox{ы}}$\-ми,  
вер\-баль\-но-не\-сло\-вес\-ны\-ми и~формализованными. Второй уровень классификации трансформаций 
охватывает не все виды объектов предметной  
об\-ласти информатики, а~только перечисленные~5~видов текс\-тов и~их представления, вовлеченные 
в~процессы трансформаций в~одной или более средах вместе с~данными, знанием и~его концептами.

\begin{figure*}[b] %fig1
\vspace*{6pt}
      \begin{center}
     \mbox{%
\epsfxsize=121.191mm 
\epsfbox{zac-1.eps}
}
\end{center}
\vspace*{-6pt}
\Caption{Средовая версия иерархии Акоффа}
\end{figure*}

\section{Классификация трансформаций: построение~третьего 
уровня}

  Основанием для систематизации трансформаций первого и~второго порядка 
на третьем уровне этой классификации служит иерархия Акоффа~\cite{8-zac}, 
на основе которой и~была создана ее средов$\acute{\mbox{а}}$я версия~[26, 
27]. Для создания средов$\acute{\mbox{о}}$й версии была выполнена 
категоризация трех базовых понятий информатики (данные, информация, 
знания) на объекты лексикографической информационной сис\-те\-мы 
в~процессе создания ее концепции\linebreak (рис.~1).
  


  В отличие от классической иерархии Акоффа, в~ее 
средов$\acute{\mbox{о}}$й версии различаются три вида данных: сенсорно 
воспринимаемые, цифровые и~те данные, которые генерируются 
искусственными нейронными сетями (ИНС) в~системах искусственного интеллекта 
(далее~--- ИИ-дан\-ные). Последний вид данных необходим, например, для 
различения входа и~выхода процесса применения обученной 
ИНС в~цифровой модели генерации знания, описанию которой 
посвящена работа~\cite{27-zac}.
  
  Также предлагается различать два вида информации: сенсорно 
воспринимаемая и~цифровая. Кроме знания в~средов$\acute{\mbox{у}}$ю 
версию добавлены концепты и~ментальные образы сенсорно воспринимаемых 
данных. Последние служат промежуточной сущностью между сенсорно 
воспринимаемыми данными и~генерируемым знанием при описании процессов 
извлечения знания из текстовых данных лексикографической информационной 
системы. Описание объектов средов$\acute{\mbox{о}}$й версии иерархии 
Акоффа (см.\ рис.~1) и~отношений между ними дано в~работах~\cite{26-zac, 28-zac}.
  
  В средов$\acute{\mbox{о}}$й версии число объектов равно восьми. Если 
учитывать направления трансформаций, то между восемью объектами на 
рис.~1 она включает~16 их видов (трансформации на границе между сенсорно 
воспринимаемыми данными и~информацией, обозначенные символом~<<?>>, 
в~статье не рас\-смат\-ри\-ва\-ют\-ся). В~будущем число объектов 
в~средов$\acute{\mbox{о}}$й версии, которая выбрана как основание для 
сис\-те\-ма\-ти\-за\-ции трансформаций первого и~второго порядка, может быть 
увеличено. Для построения классификации трансформаций 
важ\-но не возможное увеличение числа объектов 
и~трансформаций между ними, а то, что их виды в~средов$\acute{\mbox{о}}$й 
версии распределены между трансформациями первого и~второго порядка. Из 
16~видов на рис.~1 шесть относятся к~трансформациям первого порядка, это\linebreak 
виды с~номерами~7, 8, 13--16 (далее~--- типология трансформаций первого 
порядка), а~десять~--- к~трансформациям второго порядка, это виды 
с~\mbox{номерами}~1--6 и~9--12 (далее~--- типология трансформаций второго 
порядка). Разместим обе типологии на третьем уровне классификации (см.\ ее 
схему на рис.~2). Перечислим виды трансформаций первой типологии, вводя 
в~скобках их краткие названия, используемые ниже на рис.~3:
  \begin{description}
  \item[\,] 7~--- членение знания на концепты с~помощью одной или нескольких 
знаковых систем (далее~--- членение знания);
  \item[\,] 8~--- формирование знания на основе концептов (формирование 
знания);
  \item[\,] 13~--- обучение ИНС;
  \end{description}
  
  \vspace*{-6pt}
  
  \pagebreak
  
  \end{multicols}
  
  \begin{figure*} %fig2
\vspace*{1pt}
      \begin{center}
     \mbox{%
\epsfxsize=127.513mm 
\epsfbox{zac-2.eps}
}
\end{center}
\vspace*{-9pt}
\Caption{Схема трех верхних уровней классификации трансформаций объектов (объединены 
по три слоя и~для второго, и~для третьего уровней этой классификации)}
\end{figure*}
  
  \begin{multicols}{2}
  
  \noindent
  \begin{description}
  \item[\,] 14~--- восстановление обучающей информации на основе 
содержания обученной ИНС (обращение ИНС);
  \item[\,] 15~--- использование обученной ИНС (использование ИНС);



  \item[\,] 16~--- восстановление исходных данных, соответствующих 
полученным результатам работы обучен\-ной ИНС (восстановление исходных данных 
по результатам ИНС).
  \end{description}
  
  
  Не все виды трансформаций 13--16 поддерживаются в~конкретных системах 
искусственного интеллекта, но с~теоретической точки зрения все их 
предлагается включить в~первую типологию для полноты спектра видов 
трансформаций.
  
  Перечислим виды трансформаций второй типологии:
  \begin{description}
  \item[\,] 1~--- декодирование цифровых данных в~компьютерных системах 
(декодирование данных);
  \item[\,]  2~--- кодирование сенсорно воспринимаемых данных (кодирование 
данных);
  \item[\,] 3~--- ментальное копирование сенсорно воспринимаемых данных 
(ментальное копирование);
  \item[\,] 4~--- восстановление сенсорно воспринимаемых данных по 
ментальным образам (восстановление по образам);
  \item[\,] 5~--- смысловая интерпретация без деления на концепты ментальных 
образов сенсорно воспринимаемых данных (смысловая интерпретация);
  \item[\,] 6~--- восстановление ментальных образов (восстановление образов);
  \item[\,] 9~--- представление концептов в~виде сенсорно воспринимаемой 
информации, например текс\-та\-ми, формулами, таблицами, рисунками и~т.\,д.\ 
(представление концептов);
  \item[\,] 10~--- понимание смысла сенсорно воспринимаемой информации 
(понимание смысла);
  \item[\,] 11~--- кодирование сенсорно воспринимаемой информации 
(кодирование информации);
\end{description}

\vspace*{-6pt}

\pagebreak

\end{multicols}

\begin{figure*} %fig3
\vspace*{1pt}
      \begin{center}
     \mbox{%
\epsfxsize=163mm 
\epsfbox{zac-3.eps}
}
\end{center}
\vspace*{-9pt}
\Caption{Схема частного случая классификации трансформаций объектов (трансформации 
пронумерованы согласно рис.~1)}
\end{figure*}

\begin{multicols}{2}

\noindent
\begin{description}

  \item[\,] 12~--- декодирование цифровой информации (декодирование 
информации).
  \end{description}
  
  Отметим, что в~существующих ИТ
  и~компьютерных системах наиболее часто используются виды 
трансформаций~13 и~15 типологии первого порядка и~1, 2, 11 и~12 типологии 
второго порядка. На рис.~2 в~первом слое третьего уровня классификации 
показаны типологии первого порядка без указания числа трансформаций в~них 
и~без детализации трансформируемых объектов.
  
  Во втором слое третьего уровня классификации условно (без названий) 
показаны типологии второго порядка. Также на рис.~2 в~третьем слое третьего 
уровня классификации условно (также без названий) показаны типологии 
третьего порядка, которые планируется рассмотреть в~отдельной статье. По 
определению они должны включать трансформации между тремя объектами 
разной природы, но средов$\acute{\mbox{а}}$я версия иерархии Акоффа 
включает трансформации только между двумя объектами разной природы. 
Поэтому потребуется другое основание для их систематизации (ранее были 
рассмотрены отдельные примеры трансформаций третьего 
порядка\footnote{Далеко не всегда трансформации третьего и~более высоких порядков можно 
рассматривать как последовательность трансформаций второго порядка. Примером этого могут 
служить трансформации в~процессе обучения пациента пользованию роботизированной рукой, 
охватывающие личностные концепты пациента, релевантные его намерениям, сигналы активности 
мозга как объекты нейросреды и~компьютерные коды~\cite{29-zac}.}~\cite{29-zac}).

\section{Классификация трансформаций: частный~случай}

  Выше было отмечено, что в~будущем число объектов 
в~средов$\acute{\mbox{о}}$й версии иерархии Акоффа может быть увеличено. 
Это означает, что увеличатся и~чис\-ло объектов, и~чис\-ло трансформаций между 
ними в~классификации трансформаций, так как эта средов$\acute{\mbox{а}}$я 
версия служит по определению основанием для систематизации 
трансформаций первого и~второго порядка. Поэтому на третьем уровне рис.~2 
указаны типологии без детализации объектов и~без указания числа 
трансформаций в~каждой из них. С~одной стороны, при таком подходе 
получаем достаточно общий вид этой классификации, так как она не зависит от 
числа объектов в~том или ином варианте средов$\acute{\mbox{о}}$й версии 
(и~это существенно упрощает рис.~2). С~другой стороны, на третьем уровне 
такой общей классификации подразумевается, но не эксплицируется природа 
трансформируемых объектов и~их возможные сочетания в~трансформациях. 

При проектировании лексикографической информационной системы важно 
эксплицировать природу трансформируемых объектов и~их возможные 
сочетания.
  %
  Поэтому в~парадигму информатики~\cite{30-zac} кроме общей 
классификации трансформаций предлагается включать и~ее частные случаи, 
эксплицирующие природу трансформируемых объектов. 

В~этом разделе 
рассмотрим один частный случай, когда используются только естественные 
знаковые сис\-те\-мы из типологии А.~Соломоника~\cite{25-zac} вместе 
с~данными, знанием и~его концептами. Чис\-ло естественных языков при этом не 
ограничено. И~этот частный случай классификации включает только три 
класса природных трансформаций (первого, второго и~третьего порядка, см.\ 
схему классификации на рис.~3).
  
  Первый и~второй уровни схемы общей классификации (см.\ рис.~2) можно 
объединить в~один уровень в~этом частном случае. Ниже этого уровня 
приведено содержание типологий первого и~второго порядка без содержания 
типологий третьего по\-рядка.




  Наполнение типологий первого и~второго порядка соответствует 
средов$\acute{\mbox{о}}$й версии иерархии Акоффа на рис.~1, содержащей 
6~видов трансформаций типологии первого порядка и~10~видов 
трансформаций типологии второго порядка (на рис.~3 стрелки указывают 
направления трансформаций согласно средов$\acute{\mbox{о}}$й версии на рис.~1).
  
  Таким образом, частный случай классификации содержит для этих двух 
типологий 16~теоретически возможных трансформаций, 6 из которых 
в~настоящее время в~существующих ИТ применяются наиболее часто: виды 
трансформаций~1, 2, 11 и~12 типологии второго порядка реализуются 
с~помощью тех или иных методов ко\-ди\-ро\-ва\-ния/де\-ко\-ди\-ро\-ва\-ния 
(например, с~использованием таблиц Unicode), а~виды трансформаций~13 и~15
 в~типологии первого порядка реализуются полностью с~по\-мощью процессов 
цифровой обработки компьютерами.
  
  Остальные виды трансформаций или применяются намного реже (это 
виды~3, 5, 7, 9 и~10), или находятся в~стадии поиска и~разработки (14 и~16) или 
в~настоящее время носят только теоретический характер, обеспечивая полноту 
первой и~второй типологий (4, 6 и~8). Знаком~<<?>> обозначены те виды 
трансформаций, которые по определению не существуют в~используемой 
парадигме информатики~\cite{30-zac}. Однако возможно, что в~других 
будущих подходах к~построению ее парадигмы эти виды трансформаций будут 
существовать.
  
\section{Заключение}

  На сегодняшний день процесс построения классификаций объектов 
предметной области информатики~\cite{22-zac} и~их  
трансформаций~\cite{23-zac} еще не завершен. Однако первые результаты их 
построения уже используются для создания концепции лексикографической 
информационной сис\-те\-мы, обеспечивающей интеграцию двуязычных 
словарей и~параллельных корпусов.
  
  \bigskip
  
  
  Автор признателен рецензентам за помощь в~улучшении статьи.
  
{\small\frenchspacing
 { %\baselineskip=10.6pt
 %\addcontentsline{toc}{section}{References}
 \begin{thebibliography}{99}
\bibitem{1-zac}
\Au{Aijmer K., Altenberg~B.} Advances in corpus-based contrastive linguistics. Studies in honour 
of Stig Johansson.~--- Amsterdam: John Benjamins, 2013. 295~p.  doi: 10.1075/scl.54.
\bibitem{2-zac}
\Au{Добровольский Д.\,О., Кретов~А.\, А., Шаров~С.\,А.} Корпус параллельных текстов~// 
Научная и~техническая информация. Сер.~2: Информационные процессы и~сис\-те\-мы, 2005. 
№\,6. С.~16--27.
\bibitem{3-zac}
\Au{Добровольский Д.\,О.} Корпус параллельных текстов и~сопоставительная 
лексикология~// Труды Института русского языка им.\ В.\,В.~Виноградова, 2015. №\,6. 
С.~413--449. EDN: VJQBHP.
\bibitem{4-zac}
\Au{Гончаров А.\,А., Зацман~И.\,М., Кружков~М.\,Г.} Эволюция классификаций 
в~надкорпусных базах данных~// Информатика и~её применения, 2020. Т.~14. Вып.~4. 
С.~108--116. doi: 10.14357/19922264200415.  
EDN: \mbox{GKWBZT}.
\bibitem{5-zac}
\Au{Гончаров А.\, А., Зацман И. \,М., Кружков~М.\, Г}. Представление новых 
лексикографических знаний в~динамических классификационных сис\-те\-мах~// 
Информатика и~её применения, 2021. Т.~15. Вып.~1. С.~86--93.  doi: 10.14357/19922264210112. EDN: OPEFXW.
\bibitem{6-zac}
\Au{Zatsman I.} Finding and filling lacunas in linguistic typologies~// 15th Forum (International) 
on Knowledge Asset Dynamics Proceedings.~--- Matera, Italy: Institute of Knowledge Asset 
Management, 2020. P.~780--793.
\bibitem{7-zac}
\Au{Zatsman I.} Three-dimensional encoding of emerging meanings in AI-systems~// 21st 
European Conference on Knowledge Management Proceedings.~--- Reading, U.K.: Academic 
Publishing International Ltd., 2020. P.~878--887.
\bibitem{8-zac}
\Au{Ackoff R.} From data to wisdom~// J.~Applied Systems Analysis, 1989. Vol.~16. No.\,1. P.~3--9.
\bibitem{9-zac}
\Au{Rosenbloom P.\,S.} On computing: The fourth great scientific domain.~--- Cambridge, MA, 
USA: MIT Press, 2013. 307~p.
\bibitem{10-zac}
\Au{Rowley J.} The wisdom hierarchy: Representations of the DIKW hierarchy~// J.~Inf. 
Sci., 2007. Vol.~33. Iss.~2. P.~163--180. doi: 10.1177/0165551506070706.
\bibitem{11-zac} 
\Au{Frick$\acute{\mbox{e}}$~M.\,H.} Data--Information--Knowledge--Wisdom (DIKW) pyramid, 
framework, continuum~// Encyclopedia of big data~/ Eds. L.~Schintler, C.~McNeely.~--- Cham: 
Springer, 2018. 4~p. doi: 10.1007/978-3-319-32001-4\_331-1.
\bibitem{12-zac}
\Au{Denning P., Rosenbloom~P.} Computing: The fourth great domain of science~// Commun. 
ACM, 2009. Vol.~52. Iss.~9. P.~27--29.
\bibitem{13-zac}
\Au{Denning P., Freeman~P.} Computing's paradigm~// Commun.  ACM, 2009. Vol.~52. 
Iss.~12. P.~28--30. doi: 10.1145/ 1610252.1610265.
\bibitem{17-zac} %14
\Au{Farradane J.} Knowledge, information, and information science~// J.~Inf. Sci., 
1980. Vol.~2. Iss.~2. P.~75--80. doi: 10.1177/01655515800020020.

\bibitem{15-zac}
\Au{Шрейдер Ю.\,А.} Информация и~знание~// Сис\-тем\-ная концепция информационных 
процессов.~--- М.: ВНИИСИ, 1988. С.~47--52.
\bibitem{16-zac}
\Au{Ingwersen P.} Information and information science~// Enclyclopaedie of library and 
information science~/ Eds. J.\,D.~McDonald, 
M.~Levine-Clark.~--- New York, NY, USA: Marcel Dekker Inc., 1992. Vol.~56. Sup.~19. 
P.~137--174.

\bibitem{14-zac} %17
Информатика как наука об информации: Информационный, документальный, 
технологический, экономический, социальный и~организационный аспекты~/ Под ред. 
Р.\,С.~Гиляревского.~--- М.: Фаир-Пресс, 2006. 592~с.

\bibitem{18-zac}
\Au{Hjorland B.} Library and information science: practice, theory, and philosophical basis~// 
Inform. Process. Manag., 2000. Vol.~36. Iss.~3. P.~501--531. doi:  
10.1016/S0306-\mbox{4573(99)00038-2}.
\bibitem{19-zac}
Deep shift~--- technology tipping points and societal impact.~--- Geneva: WE Forum, 2015. 44~p. 
{\sf http://www3.weforum.org/docs/WEF\_GAC15\_ Technological\_Tipping\_Points\_report\_2015.pdf}.
\bibitem{20-zac}
\Au{Berman F., Rutenbar~R., Hailpern~B., Christensen~H., Davidson~S., Estrin~D., 
Franklin~M., Martonosi~M., Raghavan~P., Stodden~V., Szalay~A.\,S.} Realizing the potential of 
data science~// Commun.  ACM, 2018. Vol.~61. Iss.~4. P.~67--72. doi: 10.1145/3188721.

\bibitem{21-zac}
\Au{Stodden V.} The data science life cycle: A~disciplined approach to advancing data science as 
a~science~// Commun.  ACM, 2020. Vol.~63. Iss.~7. P.~58--66. doi: 10.1145/ 3360646.


\bibitem{23-zac} %22
\Au{Зацман И.\,М.} Научная парадигма информатики: классификация трансформаций 
объектов предметной об\-ласти~// Системы и~средства информатики, 2023. Т.~33. №\,4. 
С.~126--138. doi: 10.14357/08696527230412. EDN: ZIKUWO.

\bibitem{22-zac} %23
\Au{Зацман И.\,М.} Научная парадигма информатики: классификация объектов предметной  
об\-ласти~// Информатика и~её применения, 2023. Т.~17. Вып.~4. С.~96--103. doi: 
10.14357/19922264230413. EDN: FIUQAT.

\bibitem{24-zac}
\Au{Зацман И.\,М.} О~научной парадигме информатики: верхний уровень классификации 
объектов ее предметной об\-ласти~// Информатика и~её применения, 2022. Т.~16. Вып.~4. 
С.~73--79. doi: 10.14357/ 19922264220411. EDN: XZNKVI.

\bibitem{25-zac}
\Au{Соломоник А.\,Б.} Философия знаковых систем и~язык.~--- М.: ЛКИ, 2011. 408~с.
\bibitem{26-zac}
\Au{Зацман И.\,М.} Трансформация иерархии Акоффа в~научной парадигме информатики~// 
Информатика и~её применения, 2023. Т.~17. Вып.~3. С.~107--113. doi: 
10.14357/19922264230315. EDN: UMVRRV.

\bibitem{27-zac}
\Au{Zatsman I.} Building digital spiral models of knowledge generation~// 19th Forum 
(International) on Knowledge Asset Dynamics Proceedings.~--- Matera, Italy: Arts for Business 
Institute, 2024. P.~2185--2196.
\bibitem{28-zac}
\Au{Zatsman I.} Digital spiral model of knowledge creation and encoding its dynamics~// 18th 
Forum (International) on Knowledge Asset Dynamics Proceedings.~--- Matera, Italy: Arts for 
Business Institute, 2023. P.~581--596.
\bibitem{29-zac}
\Au{Зацман И.\,М.} Интерфейсы третьего порядка в~информатике~// Информатика и~её 
применения, 2019. Т.~13. Вып.~3. С.~82--89. doi: 10.14357/19922264190312. EDN: 
EHRQLF.

\bibitem{30-zac}
\Au{Зацман И.\,М.} Научная парадигма информатики как третьей культуры~//  
На\-уч\-но-тех\-ни\-че\-ская информация. Сер.~1: Организация и~методика информационной 
работы, 2023. №\,11. С.~1--14.

\end{thebibliography}

 }
 }

\end{multicols}

\vspace*{-9pt}

\hfill{\small\textit{Поступила в~редакцию 14.04.24}}

\vspace*{4pt}

%\pagebreak

%\newpage

%\vspace*{-28pt}

\hrule

\vspace*{2pt}

\hrule



\def\tit{OBJECT TRANSFORMATIONS OF~THE~FIRST AND~SECOND ORDER
IN~A~LEXICOGRAPHIC INFORMATION SYSTEM\\[-5pt]}


\def\titkol{Object transformations of~the~first and~second order
in~a~lexicographic information system}


\def\aut{I.\,M.~Zatsman}

\def\autkol{I.\,M.~Zatsman}

\titel{\tit}{\aut}{\autkol}{\titkol}

\vspace*{-13pt}


\noindent
Federal Research Center ``Computer Science and Control'' of the Russian Academy of Sciences, 
44-2~Vavilov Str., Moscow 119133, Russian Federation


\def\leftfootline{\small{\textbf{\thepage}
\hfill INFORMATIKA I EE PRIMENENIYA~--- INFORMATICS AND
APPLICATIONS\ \ \ 2024\ \ \ volume~18\ \ \ issue\ 2}
}%
 \def\rightfootline{\small{INFORMATIKA I EE PRIMENENIYA~---
INFORMATICS AND APPLICATIONS\ \ \ 2024\ \ \ volume~18\ \ \ issue\ 2
\hfill \textbf{\thepage}}}

\vspace*{2pt}



\Abste{The theoretical foundations of the design of information technologies used for 
the integration of bilingual dictionaries and parallel corpora are considered. The 
description of the first outcomes of the creation of the third\linebreak\vspace*{-12pt}}

\Abstend{ level of object 
transformations classification in the subject domain of informatics, which is supposed 
to be used
in creating the lexicographic information system providing integration, is 
given. All the entities of informatics are divided into two global classes: objects and 
their transformations. For each such class, its own classification is constructed. 
Previously, the two upper levels of the object transformation classification in the subject 
domain have been described. The present paper discusses the third level of this classification. The 
basis for the construction of its highest level was the division of the subject domain of 
informatics into media (mental, sensory, digital, and a~number of other media), each 
of which by definition includes objects of the same nature. The Solomonick's 
typology of sign systems served as the basis for constructing the second level of the 
object transformation classification. The aim of the paper is to systematize object 
transformations of the first and second orders at the third level of this classification. 
The basis for systematization is the medium version of the Ackoff's hierarchy.}

\KWE{subject domain objects; object transformations; classification; data; 
information; knowledge; lexicographic information system}


\DOI{10.14357/19922264240211}{VZTGVV}

\vspace*{-12pt}

\Ack

\vspace*{-3pt}


\noindent
The reported study was funded by the Russian Science Foundation, project  
No.\,24-18-00155, {\sf 
https://rscf.ru/project/24-18-00155}. The research was carried out using the infrastructure of the Shared 
Research Facilities ``High Performance Computing and Big Data'' (CKP 
``Informatics'') of FRC CSC RAS (Moscow) .
   


  \begin{multicols}{2}

\renewcommand{\bibname}{\protect\rmfamily References}
%\renewcommand{\bibname}{\large\protect\rm References}

{\small\frenchspacing
 {%\baselineskip=10.8pt
 \addcontentsline{toc}{section}{References}
 \begin{thebibliography}{99} 
\bibitem{1-zac-1}
\Aue{Aijmer, K., and B.~Altenberg.} 2013. \textit{Advances in corpus-based 
contrastive linguistics. Studies in honour of Stig Johansson}. Amsterdam: John 
Benjamins. 295~p. doi: 10.1075/scl.54.
\bibitem{2-zac-1}
\Aue{Dobrovolskiy, D.\,O., A.\,A.~Kretov, and S.\,A.~Sharov.} 2005. Korpus 
parallel'nykh tekstov [Corpus of parallel texts]. \textit{Nauchnaya i~tekhnicheskaya 
informatsiya. Ser. 2. Informatsionnye protsessy i~sistemy} [Scientific and Technical 
Information. Ser.~2: Information Processes and Systems] 6:16--27.
\bibitem{3-zac-1}
\Aue{Dobrovolskiy, D.\,O.} 2015. Korpus parallel'nykh tekstov i~sopostavitel'naya 
leksikologiya [The corpus of parallel texts and contrastive lexicology]. \textit{Trudy 
Instituta russkogo yazyka im. V.\,V.~Vinogradova} [Proceedings of the 
V.\,V.~Vinogradov Russian Language Institute] 6:413--449. EDN: VJQBHP.
\bibitem{4-zac-1}
\Aue{Goncharov, A.\,A., I.\,M.~Zatsman, and M.\,G.~Kruzhkov.} 2020. Evolyutsiya 
klassifikatsiy v~nadkorpusnykh ba\-zakh dannykh [Evolution of classifications in 
supracorpora databases]. \textit{Informatika i~ee Primeneniya~--- Inform. \mbox{Appl.}}  
14(4):108--116. doi: 10.14357/19922264200415.  
EDN: GKWBZT.
\bibitem{5-zac-1}
\Aue{Goncharov, A.\,A., I.\,M.~Zatsman, and M.\,G.~Kruzhkov.} 2021. 
Predstavlenie novykh leksikograficheskikh znaniy v~dinamicheskikh 
klassifikatsionnykh sistemakh [Representation of new lexicographical knowledge in 
dynamic classification systems]. \textit{Informatika i~ee Primeneniya~--- Inform. 
Appl.} 15(1):86--93. doi: 10.14357/19922264210112. EDN: OPEFXW.
\bibitem{6-zac-1}
\Aue{Zatsman, I.} 2020. Finding and filling lacunas in linguistic typologies. 
\textit{15th Forum (International) on Knowledge Asset Dynamics Proceedings}. 
Matera, Italy: Institute of Knowledge Asset Management. 780--793.
\bibitem{7-zac-1}
\Aue{Zatsman, I.} 2020. Three-dimensional encoding of emerging meanings in  
AI-systems. \textit{21st European Conference on Knowledge Management 
Proceedings}. Reading, U.K.: Academic Publishing International Ltd. 878--887.
\bibitem{8-zac-1}
\Aue{Ackoff, R.} 1989. From data to wisdom. \textit{J.~Applied Systems Analysis} 
16(1):3--9.
\bibitem{9-zac-1}
\Aue{Rosenbloom, P.\,S.} 2013. \textit{On computing: The fourth great scientific 
domain}. Cambridge, MA: MIT Press. 307~p.
\bibitem{10-zac-1}
\Aue{Rowley, J.} 2007. The wisdom hierarchy: Representations of the DIKW 
hierarchy. \textit{J.~Inf. Sci.} 33(2):163--180. doi: 10.1177/0165551506070706.
\bibitem{11-zac-1}
\Aue{Frick$\acute{\mbox{e}}$, M.\,H.} 2018.  
Data-Information-Knowledge-Wisdom (DIKW) pyramid, framework, continuum. 
\textit{Encyclopedia of big data}. Eds. L.~Schintler and C.~McNeely. Cham: 
Springer. 4~p. doi: 10.1007/978-3-319-32001- 4\_331-1.
\bibitem{12-zac-1}
\Aue{Denning, P., and P.~Rosenbloom.} 2009. Computing: The fourth great domain 
of science. \textit{Commun. ACM} 52(9):27--29.
\bibitem{13-zac-1}
\Aue{Denning, P., and P.~Freeman.} 2009. Computing's paradigm. \textit{Commun. 
ACM} 52(12):28--30. doi: 10.1145/ 1610252.1610265.

\bibitem{17-zac-1} %14
\Aue{Farradane, J.} 1980. Knowledge, information, and information science. 
\textit{J.~Inf. Sci.} 2(2):75--80. doi: 10.1177/ 01655515800020020.

\bibitem{15-zac-1}
\Aue{Shreyder, Yu.\,A.} 1988. Informatsiya i~znanie [Information and knowledge]. 
\textit{Sistemnaya kontseptsiya in\-for\-ma\-tsi\-on\-nykh protsessov} [System concept of 
information processes]. Moscow: VNIISI. 47--52.
\bibitem{16-zac-1}
\Aue{Ingwersen, P.} 1995. Information and information science. 
\textit{Encyclopedia of library and information science}. Eds. J.\,D.~McDonald and 
M.~Levine-Clark. New York, NY: Marcel Dekker Inc. 56(19):137--174.

\bibitem{14-zac-1} %17
Gilyarevskiy, R.\,S., ed. 2006. \textit{Informatika kak nauka ob informatsii: 
informatsionnyy, dokumental'nyy, tekh\-no\-lo\-gi\-che\-skiy, ekonomicheskiy, sotsial'nyy 
i~organizatsionnyy aspekty} [Informatics as information science: Informational, 
documentary, technological, economic, social, and organizational dimensions]. 
Moscow: FAIR-PRESS. 592~p.

\bibitem{18-zac-1}
\Aue{Hjorland, B.} 2000. Library and information science: Practice, theory, and 
philosophical basis. \textit{Inform. Process. Manag.} 36(3):501--531. doi:  
10.1016/S0306-\mbox{4573(99)00038-2}.
\bibitem{19-zac-1}
Deep shift~--- technology tipping points and societal impact. 2015. \textit{World Economic 
Forum}. Geneva. 44~p. Available at: {\sf 
http://www3.weforum.org/docs/WEF\_ GAC15\_Technological\_Tipping\_Points\_report\_2015.pdf} (accessed May~20, 
2024).
\bibitem{20-zac-1}
\Aue{Berman, F., R.~Rutenbar, B.~Hailpern, H.~Christensen, S.~Davidson, 
D.~Estrin, M.~Franklin, M.~Martonosi, P.~Raghavan, V.~Stodden, and 
A.\,S.~Szalay.} 2018. Realizing the potential of data science. \textit{Commun. ACM} 
61(4):67--72. doi: 10.1145/3188721.
\bibitem{21-zac-1}
\Aue{Stodden, V.} 2020. The data science life cycle: A~disciplined approach to 
advancing data science as a~science. \textit{Commun. ACM} 
 63(7):58--66. doi: 10.1145/3360646.

\bibitem{23-zac-1} %22
\Aue{Zatsman, I.\,M.} 2023. Nauchnaya paradigma informatiki: klassifikatsiya 
transformatsiy ob''ektov predmetnoy oblasti [Scientific paradigm of informatics: 
Transformation classification of domain objects]. \textit{Sistemy i~Sredstva 
Informatiki~--- Systems and Means of Informatics} 33(4):126--138. doi: 
10.14357/08696527230412. EDN: ZIKUWO.

\bibitem{22-zac-1} %23
\Aue{Zatsman, I.\,M.} 2023. Nauchnaya paradigma informatiki: klassifikatsiya 
ob''ektov predmetnoy oblasti [Scientific paradigm of informatics: Classification of 
domain objects]. \textit{Informatika i~ee Primeneniya~--- Inform. Appl.} 
 17(4):96--103. doi: 10.14357/19922264230413. EDN: FIUQAT.
 
\bibitem{24-zac-1}
\Aue{   Zatsman, I.\,M.} 2022. O nauchnoy paradigme informatiki: verkhniy uroven' 
klassifikatsii ob''ektov ee predmetnoy oblasti [On the scientific paradigm of 
informatics: The classification high level of its objects]. \textit{Informatika i~ee 
Primeneniya~--- Inform. Appl.} 16(4):73--79. doi: 10.14357/19922264220411. EDN: 
XZNKVI.
\bibitem{25-zac-1}
\Aue{Solomonick, A.\,B.} 2011. \textit{Filosofiya znakovykh system i~yazyk} 
[Philosophy of sign systems and language]. Moscow: LKI. 408~p.
\bibitem{26-zac-1}
\Aue{Zatsman, I.\,M.} 2023. Transformatsiya ierarkhii Akoffa v~nauchnoy 
paradigme informatiki [Transformation of the Ackoff's hierarchy in the scientific 
paradigm of informatics]. \textit{Informatika i~ee Primeneniya~--- Inform. \mbox{Appl.}} 
17(3):107--113. doi: 10.14357/19922264230315. EDN: UMVRRV.
\bibitem{27-zac-1}
\Aue{Zatsman, I.} 2024. Building digital spiral models of knowledge 
generation. \textit{19th Forum (International) on Knowledge Asset Dynamics 
Proceedings}. Matera, Italy: Arts for Business Institute. 2185--2196.
\bibitem{28-zac-1}
\Aue{Zatsman, I.} 2023. Digital spiral model of knowledge creation and encoding its 
dynamics. \textit{18th Forum (International) on Knowledge Asset Dynamics 
Proceedings}. Matera, Italy: Arts for Business Institute. 581--596.
\bibitem{29-zac-1}
\Aue{Zatsman, I.\,M.} 2019. Interfeysy tret'ego poryadka v~informatike 
 [Third-order interfaces in informatics]. \textit{Informatika i~ee Primeneniya~--- 
Inform. Appl.} 13(3):82--89. doi: 10.14357/19922264190312. EDN: EHRQLF.
\bibitem{30-zac-1}
\Aue{Zatsman, I.} 2023. Scientific paradigm of informatics as a~third culture. 
\textit{Scientific Technical Information Processing} 50(4):246--258. doi: 
10.3103/S0147688223040111. EDN: CKHMYS.

\end{thebibliography}

 }
 }

\end{multicols}

\vspace*{-6pt}

\hfill{\small\textit{Received April 14, 2024}} 


\vspace*{-12pt}


\Contrl

\vspace*{-3pt}

\noindent
\textbf{Zatsman Igor M.} (b.\ 1952)~--- Doctor of Science in technology, head of 
department, Federal Research Center ``Computer Science and Control'' of the 
Russian Academy of Sciences, 44-2~Vavilov Str., Moscow 119333, Russian 
Federation; \mbox{izatsman@yandex.ru}





\label{end\stat}

\renewcommand{\bibname}{\protect\rm Литература}     %15+




%%%%%%%%%%%%%%%%%%%%%%%%%%%%%%%%%%%%%%%%

%\def\stat{rez}
{%\hrule\par
%\vskip 7pt % 7pt
\raggedleft\Large \bf%\baselineskip=3.2ex
Р\,Е\,Ц\,Е\,Н\,З\,И\,И \vskip 17pt
    \hrule
    \par
\vskip 6pt plus 6pt minus 3pt }

%\thispagestyle{headings} %с верхним колонтитулом
%\thispagestyle{myheadings} %с нижним колонтитулом, но в верхнем РЕЦЕНЗИИ

\def\tit{НОВАЯ КНИГА И.\,Н.~СИНИЦЫНА, А.\,С.~ШАЛАМОВА <<ЛЕКЦИИ ПО ТЕОРИИ 
ИНТЕГРИРОВАННОЙ ЛОГИСТИЧЕСКОЙ ПОДДЕРЖКИ>> (М.: ТОРУС ПРЕСС, 2012. 624~с.)}

%1
\def\aut{Д.ф.-м.н., профессор С.\,Я.~Шоргин}

\def\auf{\ }

\def\leftkol{\ % РЕЦЕНЗИИ
}

\def\rightkol{ \ } 

%\def\leftkol{\ } % ENGLISH ABSTRACTS}

%\def\rightkol{\ } %ENGLISH ABSTRACTS}

%\def\leftkol{РЕЦЕНЗИИ}

%\def\rightkol{РЕЦЕНЗИИ}

\titele{\tit}{\aut}{\auf}{\leftkol}{\rightkol}
\vspace*{-18pt}


     \label{st\stat}

     \begin{multicols}{2}
     {\small
     {\baselineskip=10.1pt
     

      В книге представлено системное изложение теоретических основ одного из новейших 
направлений в \mbox{об\-ласти} экономики послепродажного обслуживания изделий наукоемкой 
продукции (ИНП) длительного пользования~--- интегрированной логистической поддержки
(ИЛП). 
{\looseness=1

}

Приведены также результаты новых работ, выполненных в Институте проблем информатики 
Российской академии наук в рамках научного направления <<Информационные технологии и 
анализ сложных сис\-тем>>.
 {%\looseness=1

}
     
      Излагаемые в книге научные подходы позво\-ляют карди\-наль\-но реформировать 
существующие системы производства и эксплуатации ИНП путем создания и внед\-ре\-ния 
методов рационального и оптимального управ\-ле\-ния процессами расходования 
вре\-мен\-н$\acute{\mbox{ы}}$х, 
мате\-ри\-аль\-ных, трудовых и других ресурсов на всех стадиях жизненного цикла изделий (ЖЦИ) по 
критериям экономической целесообразности и эф\-фек\-тив\-ности.
  {\looseness=1

}
    
      В книге приведен краткий обзор причин возник\-новения и
      развития CALS-методологии как основы 
современных международных стандартов по созданию и функционированию глобальных 
ин\-фор\-ма\-ци\-он\-но-ком\-му\-ни\-ка\-ци\-он\-ных систем, ее ключевых возможностей и эффективности 
результатов ее использования. 
Авторы %\linebreak 
предлагают ряд научных обоснований для разработки 
единой теории проектирования и управления систем ИЛП для полноценного использования 
преимуществ %\linebreak
 суще\-ст\-ву\-ющей методологии, определяют \mbox{общую} структурную схему 
комплексной системы <<ИНП-СППО>> и необходимость разработки для ее описания 
гибридных стохастических моделей.
{%\looseness=1

}

%\columnbreak
      
      Книга состоит из пяти частей, где последовательно излагается материал по каждой из 
следующих тем: <<Интегрированная логистическая поддержка>>, <<Теория гибридных 
стохастических систем и компьютерная поддержка исследований и разработок>>, <<Основы 
математического моделирования, анализа и синтеза систем послепродажного обслуживания>>, 
<<Определение и анализ показателей экспортного потенциала ИНП при проектировании>>, 
<<Задачи управления поддержкой послепродажного обслуживания>>, а также 
<<Моделирование инвестиционных процессов ИЛП в условиях неравновесных финансовых 
рынков>>. 
   
      В конце каждой главы приведены выводы и даны вопросы и задания для 
самоконтроля. В~приложениях содержатся основные определения по программам работ по 
анализу ИЛП, логистическим базам данных и компьютерным решениям, эквивалентной статистической 
линеаризации нелинейных преобразований ИЛП, справочный материал, а также развернутые 
уравнения для вероятностных характеристик.


      \def\leftkol{РЕЦЕНЗИИ}

\def\rightkol{РЕЦЕНЗИИ} 

      
      Книга заинтересует широкий круг специалистов и может быть использована научными 
проектными организациями в сфере промышленного производства ИНП. Большое количество 
иллюстраций, примеров и вопросов, обращенных к читателю, позволяет использовать книгу 
также в качестве учебного пособия для студентов и аспирантов машиностроительных, 
транспортных и~других специальностей, а также для самостоятельного изучения. 
{%\looseness=-1

}

Книга 
представляет несомненный интерес для специалистов и студентов в области прикладной 
математики и информатики.
    

}

}
\end{multicols}

%\newpage

%\def\stat{popravka}



\def\tit{ПОПРАВКА К СТАТЬЕ О.\,В.~ШЕСТАКОВА 
<<ПОРОГОВЫЕ ФУНКЦИИ В~МЕТОДАХ ПОДАВЛЕНИЯ ШУМА, ОСНОВАННЫХ~НА~ВЕЙВЛЕТ-РАЗЛОЖЕНИИ СИГНАЛА>>\\
(Информатика и её применения, 2021. Т.\ 15.  Вып.\,3. C.\ 51--56)}



\def\titkol{Поправка к статье О.\,В.~Шестакова\\
<<Пороговые функции в~методах подавления шума, основанных
на~вейвлет-разложении сигнала>>}



  \def\aut{\ }

  \def\autkol{\ } 

\titel{\tit}{\aut}{\autkol}{\titkol}

\def\leftfootline{\small{\textbf{\thepage}
\hfill INFORMATIKA I EE PRIMENENIYA~--- INFORMATICS AND
APPLICATIONS\ \ \ 2021\ \ \ volume~15\ \ \ issue\ 4}
}%
 \def\rightfootline{\small{INFORMATIKA I EE PRIMENENIYA~---
INFORMATICS AND APPLICATIONS\ \ \ 2021\ \ \ volume~15\ \ \ issue\ 4
\hfill \textbf{\thepage}}}


 \label{st\stat}

 \thispagestyle{headings}
 
 \vspace*{-24pt}  

\noindent
{\textbf{DOI:} 10.14357/19922264210307}

\vspace*{20pt}

\def\leftfootline{\small{\textbf{\thepage}
\hfill INFORMATIKA I EE PRIMENENIYA~--- INFORMATICS AND
APPLICATIONS\ \ \ 2021\ \ \ volume~15\ \ \ issue\ 4}
}%
 \def\rightfootline{\small{INFORMATIKA I EE PRIMENENIYA~---
INFORMATICS AND APPLICATIONS\ \ \ 2021\ \ \ volume~15\ \ \ issue\ 4
\hfill \textbf{\thepage}}}


%%%%%%%%%

\medskip

\noindent
С.~55, вместо 

\bigskip

\noindent
{\large ANALYSIS OF THE UNBIASED MEAN-SQUARE RISK ESTIMATE\\[6pt]
 OF~THE~BLOCK THRESHOLDING METHOD}

 



\bigskip

\noindent
должно быть

\bigskip

\noindent
{\large THRESHOLDING FUNCTIONS IN~THE~NOISE SUPPRESSION METHODS\\[6pt] 
BASED ON~THE~WAVELET EXPANSION OF~THE~SIGNAL}

 



 
\vskip 10pt plus 9pt minus 6pt

 \thispagestyle{headings}
 
 %\vspace*{-22pt}
  

\label{end\stat}

\renewcommand{\bibname}{\protect\rm Литература} 


\vspace*{8pt}

\hrule

\vspace*{2pt}

\hrule 

\vspace*{12pt}


\def\stat{popravka-1}



\def\tit{ПОПРАВКА К СТАТЬЕ А.\,А.~КУДРЯВЦЕВА, О.\,В.~ШЕСТАКОВА, С.\,Я.~ШОРГИНА
<<МЕТОД ОЦЕНИВАНИЯ ПАРАМЕТРОВ ИЗГИБА, ФОРМЫ И~МАСШТАБА
ГАММА-ЭКСПОНЕНЦИАЛЬНОГО РАСПРЕДЕЛЕНИЯ>>\\
(Информатика и её применения, 2021. Т.\ 15.  Вып.\,3. C.\ 57--62)}



\def\titkol{Поправка к статье А.\,А.~Кудрявцева, О.\,В.~Шестакова, С.\,Я.~Шоргина
<<Метод оценивания параметров изгиба, формы и масштаба
гамма-экспоненциального распределения>>}



  \def\aut{\ }

  \def\autkol{\ } 

\titel{\tit}{\aut}{\autkol}{\titkol}


 \label{st\stat}

 \thispagestyle{headings}
 
 \vspace*{-24pt}  

\noindent
{\textbf{DOI:} 10.14357/19922264210308}

\vspace*{20pt}




%%%%%%%%%

\medskip

\noindent
С.~61, вместо 

\bigskip

\noindent
{\large PROBABILISTIC CHARACTERISTICS OF~BALANCE INDEX
OF~FACTORS\\[6pt] 
WITH~GENERALIZED GAMMA DISTRIBUTION}



 



\bigskip

\noindent
должно быть

\bigskip

\noindent
{\large A METHOD FOR~ESTIMATING BENT, SHAPE AND~SCALE PARAMETERS\\[6pt] 
OF~THE~GAMMA-EXPONENTIAL DISTRIBUTION} 



 



 
\vskip 10pt plus 9pt minus 6pt

 \thispagestyle{headings}
 
 %\vspace*{-22pt}
  

\label{end\stat}

\renewcommand{\bibname}{\protect\rm Литература}  
%\include{popravka-1}

\def\stat{authorsrus}
{%\hrule\par
%\vskip 7pt % 7pt
\raggedleft\Large \bf%\baselineskip=3.2ex
О\,Б\ \ А\,В\,Т\,О\,Р\,А\,Х \vskip 17pt
    \hrule
    \par
\vskip 21pt plus 8pt minus 4pt }


\def\tit{\ }

\def\aut{\ }

\def\auf{\ }

\def\leftkol{\ } % ENGLISH ABSTRACTS}

\def\rightkol{ОБ АВТОРАХ} %ENGLISH ABSTRACTS}

\titele{\tit}{\aut}{\auf}{\leftkol}{\rightkol}
      
            \label{st\stat}



\vspace*{24pt}

\begin{multicols}{2}




\noindent
\textbf{Архипов Олег Петрович} (р.\ 1948)~---
кандидат технических наук, директор Орловского филиала Института проб\-лем информатики
Российской академии наук
%302025, г.Орел, Московское шоссе, д.137

\vspace*{3pt}

\noindent
\textbf{Бирюкова Татьяна Константиновна} (р.\ 1968)~---
кандидат фи\-зи\-ко-ма\-те\-ма\-ти\-че\-ских наук, старший научный сотрудник Института проб\-лем информатики
Российской академии наук

\vspace*{3pt}

\noindent 
\textbf{Бобков  Сергей Геннадьевич} (р.\ 1955)~---
доктор технических наук,  заведующий отделением На\-уч\-но-ис\-сле\-до\-ва\-тель\-ско\-го 
института системных исследований Российской академии наук
%117218, Москва, Нахимовский просп., 36, к.1 

\vspace*{3pt}

\noindent \textbf{Васильев Николай Семенович} (р.\ 1952)~--- доктор 
фи\-зи\-ко-ма\-те\-ма\-ти\-че\-ских наук, профессор, 
МГТУ им.\ Н.\,Э.~Баумана 
%, Москва 105005, 2-я Бауманская ул., д.~5,

\vspace*{3pt}

\noindent
\textbf{Гершкович Максим Михайлович} (р.\ 1968)~---
старший научный сотрудник Института проб\-лем информатики
Российской академии наук

\vspace*{3pt}

\noindent 
\textbf{Дьяченко Юрий Георгиевич} (р.\ 1958)~--- кандидат технических наук, 
старший научный сотрудник Института проб\-лем информатики
Российской академии наук

\vspace*{3pt}

\noindent 
\textbf{Ерошенко Александр Андреевич} (р.\ 1989)~--- аспирант кафедры 
математической статистики факультета вычисли\-тельной математики и кибернетики 
Московского государственного университета им.\ М.\,В.~Ломоносова
%119991, Москва ГСП-1, Ленинские горы, д.\ 1, стр. 52

\vspace*{3pt}
 
\noindent 
\textbf{Захаров Виктор Николаевич} (р.\ 1948)~--- 
доктор технических наук, доцент, ученый секретарь Института проб\-лем информатики
Российской академии наук

\vspace*{3pt}

\noindent
\textbf{Зейфман Александр Израилевич} (р.\ 1954)~---
доктор фи\-зи\-ко-ма\-те\-ма\-ти\-че\-ских наук, профессор, 
заведующий кафедрой Вологодского государственного университета; 
старший научный сотрудник Института проб\-лем информатики
Российской академии наук; главный научный сотрудник ИСЭРТ Российской академии наук

\vspace*{3pt}

\noindent
\textbf{Зыкин Сергей Владимирович} (р.\ 1959)~--- 
доктор технических наук, профессор, заведующий лабораторией Института математики 
им.\ С.\,Л.~Соболева Сибирского отделения Российской академии наук, Новосибирск 
%630090, пр.\ ак.\ Коптюга, 4 

\vspace*{4pt}

\noindent
\textbf{Киреев Владимир Иванович} (р.\ 1938)~---
доктор фи\-зи\-ко-ма\-те\-ма\-ти\-че\-ских наук, профессор Московского 
государственного горного университета
%Адрес: Россия, 119991, г. Москва, Ленинский проспект, д. 6

%\columnbreak

\vspace*{4pt}

\noindent
\textbf{Козеренко Елена Борисовна} (р.\ 1959)~---
кандидат филологических наук, заведующая лабораторией Института проб\-лем информатики
Российской академии наук

\vspace*{4pt}

\noindent
\textbf{Королев Виктор Юрьевич} (р.\ 1954)~--- доктор
фи\-зи\-ко-ма\-те\-ма\-ти\-че\-ских наук, профессор кафедры математической 
статистики факультета вычисли\-тельной математики и кибернетики 
Московского государственного университета; 
ведущий научный сотрудник Института проб\-лем информатики
Российской академии наук

\vspace*{4pt}

\noindent
\textbf{Коротышева Анна Владимировна} (р.\ 1988)~---
старший преподаватель Вологодского государственного университета

\vspace*{4pt}

\noindent 
\textbf{Кун Де Турк} (р.\ 1981)~--- научный сотрудник 
исследовательской группы SMACS факультета телекоммуникаций и обработки информации
Университета Гента, Бельгия
%В-9000 Гент, Бельгия

\vspace*{4pt}

\noindent
\textbf{Лупенцов Олег Сергеевич} (р.\ 1986)~---
аспирант Омского государственного института сервиса
%Омск 644043, ул.\ Певцова 13

\vspace*{4pt}

\noindent
\textbf{Лучко Олег Николаевич} (р.\ 1961)~---
кандидат педагогических наук, профессор, заведующий кафедрой 
Омского государственного института сервиса
%Омск 644043, ул.\ Певцова 13

\vspace*{4pt}

\noindent
\textbf{Малашенко Юрий Евгеньевич} (р.\ 1946)~---
доктор фи\-зи\-ко-ма\-те\-ма\-ти\-че\-ских наук, заведующий сектором 
Вычислительного центра им.\ А.\,А.~Дородницына Российской академии наук
%Адрес: 119333, Москва, ул. Вавилова, 40,

\vspace*{4pt}

\noindent
\textbf{Маньяков Юрий Анатольевич} (р.\ 1984)~---
кандидат технических наук, научный сотрудник Орловского филиала Института проб\-лем информатики
Российской академии наук
%302025, г.Орел, Московское шоссе, д.137

\vspace*{4pt}

\noindent
\textbf{Маренко Валентина Афанасьевна} (р.\ 1951)~---
кандидат технических наук, доцент, старший научный сотрудник 
Института математики им.\ С.\,Л.~Соболева Сибирского отделения Российской академии наук
%Новосибирск 630090, пр. ак. Коптюга, 4 

\vspace*{3pt}

\noindent 
\textbf{Морозов Евсей Викторович} (р.\ 1947)~--- доктор 
фи\-зи\-ко-ма\-те\-ма\-ти\-че\-ских, профессор, ведущий научный сотрудник 
Института прикладных математических исследований Карельского научного центра Российской
академии наук; 
%%185910 Россия, Республика Карелия, г.\ Петрозаводск, ул.\ Пушкинская, 11
профессор Петрозаводского государственного университета, Петрозаводск
%185910 Россия, Республика Карелия, г.\ Петрозаводск, пр.\ Ленина, 33

%\pagebreak

\vspace*{3pt}

\noindent
\textbf{Назарова Ирина Александровна} (р.\ 1966)~---
кандидат фи\-зи\-ко-ма\-те\-ма\-ти\-че\-ских наук, 
научный сотрудник Вычислительного центра им.\ А.\,А.~Дородницына Российской академии наук 
%Адрес: 119333, Москва, ул. Вавилова, 40

\vspace*{3pt}

\noindent
\textbf{Павлов Игорь Валерианович} (р.\ 1945)~--- 
доктор фи\-зи\-ко-ма\-те\-ма\-ти\-че\-ских наук, профессор МГТУ им.\ Н.\,Э.~Баумана 
%Москва 105005, 2-я Бауманская ул., д.~5 

%\pagebreak

\vspace*{3pt}

\noindent 
\textbf{Потахина Любовь Викторовна} (р.\ 1989)~--- аспирантка
Института прикладных математических исследований Карельского научного центра
Российской академии наук; 
%%185910 Россия, Республика Карелия, г.\ Петрозаводск, ул.\ Пушкинская, 11
инженер Петрозаводского государственного университета, Петрозаводск
%185910 Россия, Республика Карелия, г.\ Петрозаводск, пр.\ Ленина, 33

\vspace*{3pt}

\noindent 
\textbf{Рождественский Юрий Владимирович} (р.\ 1952)~--- 
кандидат технических наук, заведующий сектором Института проб\-лем информатики
Российской академии наук

\vspace*{3pt}

\noindent 
\textbf{Синицын Игорь Николаевич} (р.\ 1940)~--- доктор технических наук,
профессор, заслуженный деятель\linebreak\vspace*{-12pt}

\columnbreak

\noindent
 науки РФ, заведующий отделом Института проб\-лем информатики
Российской академии наук

\vspace*{7pt}


\noindent
\textbf{Сиротинин Денис Олегович} (р.\ 1984)~---
кандидат технических наук, научный сотрудник Орловского филиала Института проб\-лем информатики
Российской академии наук
%302025, г.Орел, Московское шоссе, д.137

\vspace*{7pt}

%\columnbreak

\noindent 
\textbf{Соколов  Игорь Анатольевич} (р.\ 1954)~--- академик (действительный член) Российской 
академии наук, доктор технических наук, директор Института проб\-лем информатики
Российской академии наук

\vspace*{7pt}

\noindent
\textbf{Степченков Юрий Афанасьевич} (р.\ 1951)~---
кандидат технических наук, заведующий отделом Института проб\-лем информатики
Российской академии наук

\vspace*{7pt}

\noindent
\textbf{Сурков Алексей Викторович} (р.\ 1978)~--- 
старший научный сотрудник На\-уч\-но-ис\-сле\-до\-ва\-тель\-ско\-го 
института системных исследований Российской академии наук
%117218, Москва, Нахимовский просп., 36, к.1 

\vspace*{7pt}

\noindent 
\textbf{Шестаков Олег Владимирович} (р.\ 1976)~--- доктор 
фи\-зи\-ко-ма\-те\-ма\-ти\-че\-ских, доцент кафедры математической статистики 
факультета вычисли\-тельной математики и кибернетики Московского 
государственного университета им.\ М.\,В.~Ломоносова; 
%119991, Москва ГСП-1, Ленинские горы, д.\ 1, стр. 52
старший научный сотрудник Института проб\-лем информатики
Российской академии наук
%, Москва 119333, ул. Вавилова, д.~44, корп.~2

\vspace*{7pt}

\noindent 
\textbf{Шоргин Сергей Яковлевич} (р.\ 1952.)~--- доктор
фи\-зи\-ко-ма\-те\-ма\-ти\-че\-ских наук, профессор, заместитель директора Института 
проб\-лем информатики Российской академии наук





%%%%%%%%%%%%%%%%%%%%%%%%%%%%%%%%%%%%%%%%%%%%%%%%%%%%%%%%%%%%%%%%%%%%%%%%%%%%%%%




%\def\rightkol{ОБ АВТОРАХ}
%\def\leftkol{ОБ АВТОРАХ}

 \label{end\stat}





%\def\leftfootline{\small{\textbf{\thepage}
%\hfill ИНФОРМАТИКА И ЕЁ ПРИМЕНЕНИЯ\ \ \ том~7\ \ \ выпуск~1\ \ \ 2013}
%}%
% \def\rightfootline{\small{ИНФОРМАТИКА И ЕЁ ПРИМЕНЕНИЯ\ \ \ том~7\ \ \ выпуск~1\ \ \ 2013
%\hfill \textbf{\thepage}}}


%\thispagestyle{myheadings}



\end{multicols}

\newpage  

%\def\stat{cont}
{%\hrule\par
%\vskip 7pt % 7pt
\raggedleft\Large \bf%\baselineskip=3.2ex
А\,В\,Т\,О\,Р\,С\,К\,И\,Й\ \ У\,К\,А\,З\,А\,Т\,Е\,Л\,Ь\ \ З\,А\ \ 2\,0\,0\,7 г. \vskip 17pt
    \hrule
    \par
\vskip 21pt plus 6pt minus 3pt }

\label{st\stat}

\def\tit{\ }

\def\aut{\ }
\def\auf{\ }

\def\leftkol{\ } % ENGLISH ABSTRACTS}

\def\rightkol{\ } %ENGLISH ABSTRACTS}

\titele{\tit}{\aut}{\auf}{\leftkol}{\rightkol}


\contentsline {chapter}{\ }{Выпуск \quad Стр.} 
\contentsline {section}{\textbf{Батракова Д.\,А., Королев В.\,Ю., Шоргин С.\,Я.}\ \ Новый метод вероятностно-ста\-ти\-сти\-че\-ско\-го анализа информационных потоков в\nobreakspace {}телекоммуникационных сетях}{\qquad 1 \qquad 40} 
\contentsline {section}{\textbf{Борисов А.\,В.}\ \ Байесовское оценивание в системах наблюдения с\nobreakspace {}марковскими скачкообразными процессами: игровой подход}{\qquad 2 \qquad 65}
\contentsline {section}{\textbf{Босов А.\,В., Иванов А.\,В.}\ \ Программная инфраструктура информационного Web-пор\-тала}{\qquad 2 \qquad 50}
\contentsline {section}{\textbf{Захаров В.\,Н., Калиниченко Л.\,А., Соколов И.\,А., Ступников С.\,А.}\ \ Конструирование канонических информационных моделей для интегрированных информационных систем}{\qquad 2 \qquad 15}
\contentsline {section}{\textbf{Захаров В.\,Н., Козмидиади В.\,А.}\ \ Средства обеспечения отказоустойчивости при\-ло\-жений}{\qquad 1 \qquad 14} 
\contentsline {section}{\textbf{Иванов А.\,В.}\ \ см. Босов А.\,В.\hfill\hfill\hfill\hfill\hfill\hfill\hfill\hfill\hfill\hfill\hfill\hfill\hfill\hfill\hfill\hfill\hfill\hfill\hfill\hfill\hfill\hfill\hfill\hfill\hfill\hfill\hfill\hfill\hfill\hfill\hfill\hfill\hfill\hfill\hfill}{\ }
\contentsline {section}{\textbf{Ильин В.\,Д., Соколов И.\,А.}\ \ Символьная модель системы знаний информатики в\nobreakspace {}че\-ло\-ве\-ко-автоматной среде}{\qquad 1 \qquad 66} 
\contentsline {section}{\textbf{Калиниченко Л.\,А.}\ \ см. Захаров В.\,Н.\hfill\hfill\hfill\hfill\hfill\hfill\hfill\hfill\hfill\hfill\hfill\hfill\hfill\hfill\hfill\hfill\hfill\hfill\hfill\hfill\hfill\hfill\hfill\hfill\hfill\hfill\hfill\hfill\hfill\hfill\hfill\hfill\hfill\hfill\hfill}{\ }
\contentsline {section}{\textbf{Козеренко Е.\,Б.}\ \ Лингвистическое моделирование для систем машинного перевода и обработки знаний}{\qquad 1 \qquad 54} 
\contentsline {section}{\textbf{Козмидиади В.\,А.}\ \ см. Захаров В.\,Н.\hfill\hfill\hfill\hfill\hfill\hfill\hfill\hfill\hfill\hfill\hfill\hfill\hfill\hfill\hfill\hfill\hfill\hfill\hfill\hfill\hfill\hfill\hfill\hfill\hfill\hfill\hfill\hfill\hfill\hfill\hfill\hfill\hfill\hfill\hfill }{\ } 
\contentsline {section}{\textbf{Королев В.\,Ю.}\ \ см. Батракова Д.\,А.\hfill\hfill\hfill\hfill\hfill\hfill\hfill\hfill\hfill\hfill\hfill\hfill\hfill\hfill\hfill\hfill\hfill\hfill\hfill\hfill\hfill\hfill\hfill\hfill\hfill\hfill\hfill\hfill\hfill\hfill\hfill\hfill\hfill\hfill\hfill}{\ } 
\contentsline {section}{\textbf{Кудрявцев А.\,А., Шоргин С.\,Я.}\ \ Байесовский подход к\nobreakspace {}анализу систем массового обслуживания и\nobreakspace {}показателей надежности}{\qquad 2 \qquad 76}
\contentsline {section}{\textbf{Печинкин А.\,В., Соколов И.\,А., Чаплыгин В.\,В.}\ \ Многолинейная система массового обслуживания с конечным накопителем и ненадежными приборами}{\qquad 1 \qquad 27} 
\contentsline {section}{\textbf{Печинкин А.\,В., Соколов И.\,А., Чаплыгин В.\,В.}\ \ Стационарные характеристики многолинейной\nobreakspace {}системы массового обслуживания с\nobreakspace {}одновременными отказами приборов}{\qquad 2 \qquad 39}
\contentsline {section}{\textbf{Синицын И.\,Н.}\ \ Корреляционные методы построения аналитических информационных моделей флуктуаций полюса Земли по априорным данным}{\qquad 2 \qquad \hphantom{9}2}
\contentsline {section}{\textbf{Синицын И.\,Н.}\ \ Развитие теории фильтров Пугачева для оперативной обработки информации в стохастических системах}{{\qquad 1 \qquad \hphantom{9}3}} 
\contentsline {section}{\textbf{Соколов И.\,А.}\ \ см. Захаров В.\,Н.\hfill\hfill\hfill\hfill\hfill\hfill\hfill\hfill\hfill\hfill\hfill\hfill\hfill\hfill\hfill\hfill\hfill\hfill\hfill\hfill\hfill\hfill\hfill\hfill\hfill\hfill\hfill\hfill\hfill\hfill\hfill\hfill\hfill\hfill\hfill}{\ }
\contentsline {section}{\textbf{Соколов И.\,А.}\ \ см. Ильин В.\,Д.\hfill\hfill\hfill\hfill\hfill\hfill\hfill\hfill\hfill\hfill\hfill\hfill\hfill\hfill\hfill\hfill\hfill\hfill\hfill\hfill\hfill\hfill\hfill\hfill\hfill\hfill\hfill\hfill\hfill\hfill\hfill\hfill\hfill\hfill\hfill}{\ } 
\contentsline {section}{\textbf{Соколов И.\,А.}\ \ см. Печинкин А.\,В.\hfill\hfill\hfill\hfill\hfill\hfill\hfill\hfill\hfill\hfill\hfill\hfill\hfill\hfill\hfill\hfill\hfill\hfill\hfill\hfill\hfill\hfill\hfill\hfill\hfill\hfill\hfill\hfill\hfill\hfill\hfill\hfill\hfill\hfill\hfill}{\ } 
\contentsline {section}{\textbf{Соколов И.\,А.}\ \ см. Печинкин А.\,В.\hfill\hfill\hfill\hfill\hfill\hfill\hfill\hfill\hfill\hfill\hfill\hfill\hfill\hfill\hfill\hfill\hfill\hfill\hfill\hfill\hfill\hfill\hfill\hfill\hfill\hfill\hfill\hfill\hfill\hfill\hfill\hfill\hfill\hfill\hfill}{\ }
\contentsline {section}{\textbf{Ступников С.\,А.}\ \ см. Захаров В.\,Н.\hfill\hfill\hfill\hfill\hfill\hfill\hfill\hfill\hfill\hfill\hfill\hfill\hfill\hfill\hfill\hfill\hfill\hfill\hfill\hfill\hfill\hfill\hfill\hfill\hfill\hfill\hfill\hfill\hfill\hfill\hfill\hfill\hfill\hfill\hfill}{\ }
\contentsline {section}{\textbf{Чаплыгин В.\,В.}\ \ см. Печинкин А.\,В.\hfill\hfill\hfill\hfill\hfill\hfill\hfill\hfill\hfill\hfill\hfill\hfill\hfill\hfill\hfill\hfill\hfill\hfill\hfill\hfill\hfill\hfill\hfill\hfill\hfill\hfill\hfill\hfill\hfill\hfill\hfill\hfill\hfill\hfill\hfill}{\ } 
\contentsline {section}{\textbf{Чаплыгин В.\,В.}\ \ см. Печинкин А.\,В.\hfill\hfill\hfill\hfill\hfill\hfill\hfill\hfill\hfill\hfill\hfill\hfill\hfill\hfill\hfill\hfill\hfill\hfill\hfill\hfill\hfill\hfill\hfill\hfill\hfill\hfill\hfill\hfill\hfill\hfill\hfill\hfill\hfill\hfill\hfill}{\ }
\contentsline {section}{\textbf{Шоргин С.\,Я.}\ \ см. Батракова Д.\,А.\hfill\hfill\hfill\hfill\hfill\hfill\hfill\hfill\hfill\hfill\hfill\hfill\hfill\hfill\hfill\hfill\hfill\hfill\hfill\hfill\hfill\hfill\hfill\hfill\hfill\hfill\hfill\hfill\hfill\hfill\hfill\hfill\hfill\hfill\hfill}{\ } 
\contentsline {section}{\textbf{Шоргин С.\,Я.}\ \ см. Кудрявцев А.\,А.\hfill\hfill\hfill\hfill\hfill\hfill\hfill\hfill\hfill\hfill\hfill\hfill\hfill\hfill\hfill\hfill\hfill\hfill\hfill\hfill\hfill\hfill\hfill\hfill\hfill\hfill\hfill\hfill\hfill\hfill\hfill\hfill\hfill\hfill\hfill}{\ }
%\thispagestyle{myheadings}
\def\leftfootline{\small{\textbf{\thepage}
\hfill ИНФОРМАТИКА И ЕЁ ПРИМЕНЕНИЯ\ \ \ том~1\ \ \ выпуск~2\ \ \ 2007}
}%
 \def\rightfootline{\small{ИНФОРМАТИКА И ЕЁ ПРИМЕНЕНИЯ\ \ \ том~1\ \ \ выпуск~2\ \ \ 2007
 \hfill \textbf{\thepage}}}
 \label{end\stat} 
                     
%\def\stat{cont-e}
{%\hrule\par
%\vskip 7pt % 7pt
\raggedleft\Large \bf%\baselineskip=3.2ex
2\,0\,0\,7\ \ A\,U\,T\,H\,O\,R\ \ I\,N\,D\,E\,X \vskip 17pt
    \hrule
    \par
\vskip 21pt plus 6pt minus 3pt }

\label{st\stat}

\def\tit{\ }

\def\aut{\ }
\def\auf{\ }

\def\leftkol{\ } % ENGLISH ABSTRACTS}

\def\rightkol{\ } %ENGLISH ABSTRACTS}

\titele{\tit}{\aut}{\auf}{\leftkol}{\rightkol}


\contentsline {chapter}{\ }{Issue \quad Page} 
\contentsline {subsection}{\textbf{Batrakova D.\,A., Korolev V.\,Yu., Shorgin S.\,Ya.}\ \ A New Method for the Probabilistic and Statistical Analysis of Information Flows in Telecommunication Networks}{\qquad 1 \qquad 40} 
\contentsline {subsection}{\textbf{Borisov A.\,V.}\ \ Bayesian Estimation in\nobreakspace {}Observation Systems with\nobreakspace {}Markov Jump Processes: Game-Theoretic Approach}{\qquad 2 \qquad 65} 
\contentsline {subsection}{\textbf{Bosov A.\,V., Ivanov A.\,V.}\ \ Linguistic Simulation for Machine Translation and Knowledge Management Systems}{\qquad 2 \qquad 50} 
\contentsline {subsection}{\textbf{Chaplygin V.\,V.} see Pechinkin A.\,V.\hfill\hfill\hfill\hfill\hfill\hfill\hfill\hfill\hfill\hfill\hfill\hfill\hfill\hfill\hfill\hfill\hfill\hfill\hfill\hfill\hfill\hfill\hfill\hfill\hfill\hfill\hfill\hfill\hfill\hfill\hfill\hfill\hfill\hfill\hfill}{\ }
\contentsline {subsection}{\textbf{Chaplygin V.\,V.} see Pechinkin A.\,V.\hfill\hfill\hfill\hfill\hfill\hfill\hfill\hfill\hfill\hfill\hfill\hfill\hfill\hfill\hfill\hfill\hfill\hfill\hfill\hfill\hfill\hfill\hfill\hfill\hfill\hfill\hfill\hfill\hfill\hfill\hfill\hfill\hfill\hfill\hfill}{\ }
\contentsline {subsection}{\textbf{Ilyin V.\,D., Sokolov I.\,A.}\ \ The Symbol Model of Informatics Knowledge System in Human-Automaton Environment}{\qquad 1 \qquad 66} 
\contentsline {subsection}{\textbf{Ivanov A.\,V.} see Bosov A.\,V.\hfill\hfill\hfill\hfill\hfill\hfill\hfill\hfill\hfill\hfill\hfill\hfill\hfill\hfill\hfill\hfill\hfill\hfill\hfill\hfill\hfill\hfill\hfill\hfill\hfill\hfill\hfill\hfill\hfill\hfill\hfill\hfill\hfill\hfill\hfill}{\ }
\contentsline {subsection}{\textbf{Kalinichenko L.\,A.} see Zakharov V.\,N.\hfill\hfill\hfill\hfill\hfill\hfill\hfill\hfill\hfill\hfill\hfill\hfill\hfill\hfill\hfill\hfill\hfill\hfill\hfill\hfill\hfill\hfill\hfill\hfill\hfill\hfill\hfill\hfill\hfill\hfill\hfill\hfill\hfill\hfill\hfill}{\ }
\contentsline {subsection}{\textbf{Korolev V.\,Yu.} see Batrakova D.\,A.\hfill\hfill\hfill\hfill\hfill\hfill\hfill\hfill\hfill\hfill\hfill\hfill\hfill\hfill\hfill\hfill\hfill\hfill\hfill\hfill\hfill\hfill\hfill\hfill\hfill\hfill\hfill\hfill\hfill\hfill\hfill\hfill\hfill\hfill\hfill}{\ }
\contentsline {subsection}{\textbf{Kozerenko E.\,B.}\ \ Linguistic Simulation for Machine Translation and Knowledge Management Systems}{\qquad 1 \qquad 54} 
\contentsline {subsection}{\textbf{Kozmidiady V.\,A.} see Zakharov V.\,N.\hfill\hfill\hfill\hfill\hfill\hfill\hfill\hfill\hfill\hfill\hfill\hfill\hfill\hfill\hfill\hfill\hfill\hfill\hfill\hfill\hfill\hfill\hfill\hfill\hfill\hfill\hfill\hfill\hfill\hfill\hfill\hfill\hfill\hfill\hfill}{\ }
\contentsline {subsection}{\textbf{Kudryavtsev A.\,A., Shorgin S.\,Ya.}\ \ Bayesian Approach to Queueing Systems and Reliability Characteristics}{\qquad 2 \qquad 76} 
\contentsline {subsection}{\textbf{Pechinkin A.\,V., Sokolov I.\,A., Chaplygin V.\,V.}\ \ Multichannel Queuing System with Finite Buffer and Unreliable Servers}{\qquad 1 \qquad 27} 
\contentsline {subsection}{\textbf{Pechinkin A.\,V., Sokolov I.\,A., Chaplygin V.\,V.}\ \ Stationary Characteristics of a Multichannel Queueing System with\nobreakspace {}Simultaneous Refusals of Servers}{\qquad 2 \qquad 39} 
\contentsline {subsection}{\textbf{Shorgin S.\,Ya.} see Batrakova D.\,A.\hfill\hfill\hfill\hfill\hfill\hfill\hfill\hfill\hfill\hfill\hfill\hfill\hfill\hfill\hfill\hfill\hfill\hfill\hfill\hfill\hfill\hfill\hfill\hfill\hfill\hfill\hfill\hfill\hfill\hfill\hfill\hfill\hfill\hfill\hfill}{\ }
\contentsline {subsection}{\textbf{Shorgin S.\,Ya.} see Kudryavtsev A.\,A.\hfill\hfill\hfill\hfill\hfill\hfill\hfill\hfill\hfill\hfill\hfill\hfill\hfill\hfill\hfill\hfill\hfill\hfill\hfill\hfill\hfill\hfill\hfill\hfill\hfill\hfill\hfill\hfill\hfill\hfill\hfill\hfill\hfill\hfill\hfill}{\ }
\contentsline {subsection}{\textbf{Sinitsyn I.\,N.}\ \ Correlational Methods for Analytical Informational Models of the Earth Pole Fluctuations Design Based on a priori Data}{\qquad 2 \qquad \hphantom{9}2}
\contentsline {subsection}{\textbf{Sinitsyn I.\,N.}\ \ Development of Pugachev Filtering for Stochastic Systems}{\qquad 1 \qquad \hphantom{9}3}
\contentsline {subsection}{\textbf{Sokolov I.\,A.} see Ilyin V.\,D.\hfill\hfill\hfill\hfill\hfill\hfill\hfill\hfill\hfill\hfill\hfill\hfill\hfill\hfill\hfill\hfill\hfill\hfill\hfill\hfill\hfill\hfill\hfill\hfill\hfill\hfill\hfill\hfill\hfill\hfill\hfill\hfill\hfill\hfill\hfill}{\ }
\contentsline {subsection}{\textbf{Sokolov I.\,A.} see Pechinkin A.\,V.\hfill\hfill\hfill\hfill\hfill\hfill\hfill\hfill\hfill\hfill\hfill\hfill\hfill\hfill\hfill\hfill\hfill\hfill\hfill\hfill\hfill\hfill\hfill\hfill\hfill\hfill\hfill\hfill\hfill\hfill\hfill\hfill\hfill\hfill\hfill}{\ }
\contentsline {subsection}{\textbf{Sokolov I.\,A.} see Pechinkin A.\,V.\hfill\hfill\hfill\hfill\hfill\hfill\hfill\hfill\hfill\hfill\hfill\hfill\hfill\hfill\hfill\hfill\hfill\hfill\hfill\hfill\hfill\hfill\hfill\hfill\hfill\hfill\hfill\hfill\hfill\hfill\hfill\hfill\hfill\hfill\hfill}{\ }
\contentsline {subsection}{\textbf{Sokolov I.\,A.} see Zakharov V.\,N.\hfill\hfill\hfill\hfill\hfill\hfill\hfill\hfill\hfill\hfill\hfill\hfill\hfill\hfill\hfill\hfill\hfill\hfill\hfill\hfill\hfill\hfill\hfill\hfill\hfill\hfill\hfill\hfill\hfill\hfill\hfill\hfill\hfill\hfill\hfill}{\ }
\contentsline {subsection}{\textbf{Stupnikov S.\,A.} see Zakharov V.\,N.\hfill\hfill\hfill\hfill\hfill\hfill\hfill\hfill\hfill\hfill\hfill\hfill\hfill\hfill\hfill\hfill\hfill\hfill\hfill\hfill\hfill\hfill\hfill\hfill\hfill\hfill\hfill\hfill\hfill\hfill\hfill\hfill\hfill\hfill\hfill}{\ }
\contentsline {subsection}{\textbf{Zakharov V.\,N., Kalinichenko L.\,A., Sokolov I.\,A., Stupnikov S.\,A.}\ \ Development of Canonical Information Models for Integrated Information Systems}{\qquad 2 \qquad 15} 
\contentsline {subsection}{\textbf{Zakharov V.\,N., Kozmidiady V.\,A.}\ \ Means Providing Applications Fault Tolerance}{\qquad 1 \qquad 14} 
\def\leftfootline{\small{\textbf{\thepage}
\hfill ИНФОРМАТИКА И ЕЁ ПРИМЕНЕНИЯ\ \ \ том~1\ \ \ выпуск~2\ \ \ 2007}
}%
 \def\rightfootline{\small{ИНФОРМАТИКА И ЕЁ ПРИМЕНЕНИЯ\ \ \ том~1\ \ \ выпуск~2\ \ \ 2007
 \hfill \textbf{\thepage}}}
 \label{end\stat} 


%\end{document}

%
\def\stat{rekl}
%\label{preobr}

%\def\tit{АКАДЕМИК ПУГАЧЁВ  ВЛАДИМИР СЕМЁНОВИЧ\\
%25.03.1911--25.03.1998}


%   \vspace*{-48pt}
%   \begin{center}\LARGE
%Академик Пугачёв  Владимир Семёнович\\ (25.03.1911--25.03.1998)
%   \end{center}

   %\vspace*{2.5mm}

   \begin{center}

{\prgsh\LARGE
ЮБИЛЕИ}

\end{center}
%\hrule

\vspace*{6pt}


   \vspace*{8mm}

   \thispagestyle{empty}


%\def\stat{emel}


\section*{К 70-летию заместителя директора ИПИ РАН,\\ члена редколлегии журнала
<<Информатика и её применения>>\\ доктора технических наук В.\,И.~Будзко}

\vspace*{18pt}




          \begin{multicols}{2}

%            \label{st\stat}

\begin{center}
\vspace*{1pt}
\mbox{%
\epsfxsize=78mm
\epsfbox{bud-1.eps}
}
\end{center}

\vspace*{12pt}

      14 августа 2014~г.\ исполнилось 70~лет за\-мес\-ти\-те\-лю директора ИПИ РАН по
научной работе доктору технических наук Владимиру Игоревичу Будзко.

      Владимир Игоревич Будзко родился в г.~Москве. Высшее образование получил на факультете
элект\-рон\-но-вы\-чис\-ли\-тель\-ных устройств в Московском
ин\-же\-нер\-но-фи\-зи\-че\-ском институте
(МИФИ), который он окончил в 1968~г., после чего был на\-прав\-лен для прохождения
службы в одну из войс\-ко\-вых частей, где прошел путь от инженера до первого заместителя
командира войсковой части.

      С приходом В.\,И.~Будзко в ИПИ РАН (2001~г.)\ в институте
сформировалось новое научное на\-прав\-ле\-ние теоретических исследований~--- <<Постро\-ение
ин\-фор\-ма\-ци\-он\-но-те\-ле\-ком\-му\-ни\-ка\-ци\-он\-ных\linebreak сис\-тем
высокой до\-ступ\-ности>>. В~рамках этого
направления выполнен широкий круг фундаментальных исследований по поиску подходов и
определению принципов построения средств обеспечения доступности, конфиденциальности
и целостности современных крупномасштабных
ин\-фор\-ма\-ци\-он\-но-те\-ле\-ком\-му\-ни\-ка\-ци\-он\-ных
сис\-тем (ИТС). Разработаны основные сис\-тем\-но-тех\-ни\-че\-ские принципы и базовые
архитектурные решения построения перспективных для условий России ИТС с
централизованной обработкой и хранением информации, сочетающих в себе свойства
высокой доступности, отказо- и катастрофоустойчивости, информационной защищенности.
Определены принципы, методы и математические основы рационального построения и
оптимизации средств восстановления функционирования центров обработки данных (ЦОД)
после возникновения отказов и катастроф, передачи и хранения данных, обеспечения
информационной безопасности при достижении минимальной совокупной стоимости
владения такими системами. Результаты нашли практическое воплощение при реализации
проектов в интересах ряда отечественных государственных и негосударственных
организаций, таких как Банк России (БР), Внешторгбанк, ОАО <<ГМК <<Норильский Никель>>,
<<Газпром>>, Минэкономразвития России, Правительство Москвы, а также ряд силовых
ведомств.

      Под руководством В.\,И.~Будзко начиная с 2001~г.\ выполнен комплекс
      на\-уч\-но-ис\-сле\-до\-ва\-тель\-ских и
      опыт\-но-кон\-ст\-рук\-тор\-ских работ (свыше 100~проектов),
направленных на развитие электронной информационной технологии БР.
Разработаны концепции развития ИТС БР сначала до 2008~г., а затем до 2013~г., которые
были приняты в качестве основы проведения технической политики. За реализацию проекта
<<Катастрофоустойчивая тер\-ри\-то\-ри\-аль\-но-рас\-пре\-де\-лен\-ная
      ин\-фор\-ма\-ци\-он\-но-те\-ле\-ком\-му\-ни\-ка\-ци\-он\-ная сис\-те\-ма централизованной
обработки банковской информации>> В.\,И.~Будзко удостоен Премии Правительства РФ в
области науки и техники за 2010~г.

      В.\,И.~Будзко возглавлял и возглавляет работы по ряду других прикладных проектов,
связанных с созданием, совершенствованием и развитием крупномасштабных ИТС.

      В.\,И.~Будзко~--- генерал-майор, доктор технических наук, член-кор\-рес\-пон\-дент
Академии криптографии РФ, известный ученый в области информатики и применения
информационных технологий при построении территориально распределенных ИТС
различного назначения. Является автором свыше 250~научных работ, опубликованных в
на\-уч\-но-тех\-ни\-че\-ских и специальных изданиях.

    \thispagestyle{empty}

      В.\,И.~Будзко уделяет большое внимание подготовке научных кадров. Под его
руководством защищено 6~диссертаций на соискание ученой степени кандидата
технических наук. Свыше 30~лет он читает лекции в ИКСИ Академии ФСБ, профессор
кафедры НИЯУ МИФИ. Является членом двух диссертационных советов, главным
редактором журнала <<Системы высокой доступности>> и членом редколлегии журнала
<<Информатика и её применения>>.

      \bigskip

      Редакционный совет и Редакционная коллегия журнала <<Информатика и её
применения>> сердечно поздравляют Владимира Игоревича Будзко с 70-ле\-ти\-ем и желают
крепкого здоровья и новых научных достижений.

\end{multicols}


%Информатика Т 16 Год 2022-1\\
\def\stat{cont}
{%\hrule\par
%\vskip 7pt % 7pt
\raggedleft\Large \bf%\baselineskip=3.2ex
А\,В\,Т\,О\,Р\,С\,К\,И\,Й\ \ У\,К\,А\,З\,А\,Т\,Е\,Л\,Ь\ \ З\,А\ \ 2\,0\,2\,2 г. \vskip 17pt
 \hrule
 \par
\vskip 21pt plus 6pt minus 3pt }

\label{st\stat}

\def\tit{\ }

\def\aut{\ }
\def\auf{\ }

\def\leftkol{\ } % ENGLISH ABSTRACTS}

\def\rightkol{\ } %АВТОРСКИЙ УКАЗАТЕЛЬ ЗА 2021 г.} %ENGLISH ABSTRACTS}

\titele{\tit}{\aut}{\auf}{\leftkol}{\rightkol}
\addcontentsline{toc}{subsection}{\textrm\textbf Авторский указатель за 2022 г.}

\vspace*{-24pt}

\noindent
{\tabcolsep=3pt
\begin{tabular}{p{397pt}cc}
&\textbf{Вып.} & \textbf{Стр.}\\[6pt]
\Avtors{Абгарян~К.\,К., Гаврилов~Е.\,С.} Программный комплекс для 
многомасштабного модели-\linebreak
\\[-12pt]
\hspace*{23pt}рования структурных свойств композиционных 
материалов&1&88--97\\
\Avtors{Аблаев~Ф.\,М.} см.\ Андрианов~С.\,Н.&&\\
\Avtors{Агаларов Я.\,М.} Оптимальное управление подключением резервного прибора 
в~системе\linebreak
\\[-12pt]
\hspace*{23pt}массового обслуживания $G/M/1$&4&34--41\\
\Avtors{Агаларов~Я.\,М.} Оптимизация порогового управления переключением 
скорости обслу-\linebreak
\\[-12pt]
\hspace*{23pt}живания в~системе массового обслуживания $G/M/1$&1&73--81\\
\Avtors{Агасандян~Г.\,А.} Многомерные бинарные рынки и~CC-VaR&2&\hphantom{1}2--10\\
\Avtors{Алию~Б., Мачнев~Е.\,А., Мокров~Е.\,В.} Гистерезисное управление нагрузкой 
в~беспроводных\linebreak
\\[-12pt]
\hspace*{23pt}сенсорных сетях&3&83--89\\
\Avtors{Андрианов~С.\,Н., Андрианова~Н.\,С., Аблаев~Ф.\,М., Кочнева~Ю.\,Ю.} 
Контекстный поиск\linebreak
\\[-12pt]
\hspace*{23pt}на фотонах с~использованием тестов Белла&1&20--24\\
\Avtors{Андрианова~Н.\,С.} см.\ Андрианов~С.\,Н.&&\\
\Avtors{Базилевский М.\,П.} Обобщение метода выпрямления искаженных из-за 
мультиколлинеарности коэффициентов для~регрессионных моделей с~различной 
степенью\linebreak
\\[-12pt]
\hspace*{23pt}корреляции объясняющих переменных&4&20--25\\
\Avtors{Бесчастный~В.\,А., Острикова~Д.\,Ю., Шоргин~С.\,Я., Молчанов~Д.\,А., 
Гайдамака~Ю.\,В.} Анализ плотности базовых станций 5G NR для предоставления услуг 
виртуальной\linebreak
\\[-12pt]
\hspace*{23pt}и~дополненной реальности&2&102--108\\
\Avtors{Бесчастный~В.\,А. } см.\ Мачнев Е.\,А.&&\\
\Avtors{Битюков~Ю.\,И.} см.\ Босов~А.\,В.&&\\
\Avtors{Борисов А.\,В.} Общий порядок аппроксимации оценок фильтрации состояний 
марков-\linebreak
\\[-12pt]
\hspace*{23pt}ских скачкообразных процессов по~дискретизованным наблюдениям&4&8--13\\
\Avtors{Босов~А.\,В.} Применение самоорганизующихся нейронных сетей к~процессу 
формиро-\linebreak
\\[-12pt]
\hspace*{23pt}вания индивидуальной траектории обучения&3&\hphantom{1}7--15\\
\Avtors{Босов~А.\,В.} Управление линейным выходом марковской цепи по квадратичному 
крите-\linebreak
\\[-12pt]
\hspace*{23pt}рию. Случай полной информации&2&19--26\\
\Avtors{Босов~А.\,В., Битюков~Ю.\,И., Денискина~Г.\,Ю.} О~поиске оптимальной 
схемы 3D-печати\linebreak
\\[-12pt]
\hspace*{23pt}конструкций из композиционных материалов&1&10--19\\
\Avtors{Босов А.\,В., Иванов А.\,В.} Технология классификации типов контента 
электронного\linebreak
\\[-12pt]
\hspace*{23pt}учебника&4&63--72\\
\Avtors{Брюхов Д.\,О., Ступников~С.\,А.} Логическая реляционная модель структур 
данных для\linebreak
\\[-12pt]
\hspace*{23pt}решения задач в~предметной области управления 
землепользованием&4&93--98\\
\Avtors{Бурцева~С.\,А.} см.\ Власкина~А.\,С.&&\\
\Avtors{Васильев~Н.\,С.} О~достаточных условиях экстремума в~многомерных 
вариационных\linebreak
\\[-12pt]
\hspace*{23pt}задачах&3&39--44\\
\Avtors{Власкина~А.\,С., Бурцева~С.\,А., Кочеткова~И.\,А., Шоргин~С.\,Я.} Управляемая 
система массового обслуживания с~эластичным трафиком и~сигналами для анализа 
нарезки\linebreak
\\[-12pt]
\hspace*{23pt}ресурсов в~сети радиодоступа&3&90--96\\
\Avtors{Гаврилов~Е.\,С.} см.\ Абгарян~К.\,К.&&\\
\Avtors{Гайдамака~Ю.\,В.} см.\ Бесчастный~В.\,А.&&\\
\Avtors{Гайдамака~Ю.\,В.} см.\ Мачнев Е.\,А.&&\\
\Avtors{Горшенин~А.\,К., Гусейнова~Е.\,И.} Повышение доходности торговли на~FOREX 
с~помощью\linebreak
\\[-12pt]
\hspace*{23pt}LSTM-идентификации свечных паттернов и~индикатора тиковых 
объемов&3&26--38\\
\Avtors{Григорьев~О.\,Г.} см.\ Смирнов~И.\,В.&&\\
\end{tabular}
}

\pagebreak

\def\leftkol{АВТОРСКИЙ УКАЗАТЕЛЬ ЗА 2022 г.} % ENGLISH ABSTRACTS}

\def\rightkol{АВТОРСКИЙ УКАЗАТЕЛЬ ЗА 2022 г.} %ENGLISH ABSTRACTS}

%\thispagestyle{myheadings}
\def\leftfootline{\small{\textbf{\thepage}
\hfill ИНФОРМАТИКА И ЕЁ ПРИМЕНЕНИЯ\ \ \ том~16\ \ \ выпуск~4\ \ \ 2022}
}%
 \def\rightfootline{\small{ИНФОРМАТИКА И ЕЁ ПРИМЕНЕНИЯ\ \ \ том~16\ \ \ выпуск~4\ \ \ 2022
 \hfill \textbf{\thepage}}}


\noindent
{\tabcolsep=3pt
\begin{tabular}{p{394pt}cc}
&\textbf{Вып.} & \textbf{Стр.}\\[3pt]
\Avtors{Грушо~А.\,А., Грушо~Н.\,А., Забежайло~М.\,И., Зацаринный~А.\,А., 
Тимонина~Е.\,Е., Шор-}\linebreak
\\[-12pt]
\hspace*{23pt}\textbf{гин~С.\,Я.} Анализ цепочек причинно-следственных связей&2&68--74\\
\Avtors{Грушо А.\,А., Грушо Н.\,А., Забежайло~М.\,И., Смирнов~Д.\,В., Тимонина~Е.\,Е., 
Шоргин~С.\,Я.}\linebreak
\\[-12pt]
\hspace*{23pt}О~безопасной архитектуре вычислительной системы на основе 
микросервисов&4&87--92\\
\Avtors{Грушо~А.\,А., Грушо~Н.\,А., Тимонина~Е.\,Е.} Метаданные в~защищенном 
электронном\linebreak
\\[-12pt]
\hspace*{23pt}документообороте&3&\hphantom{1}97--102\\
\Avtors{Грушо~Н.\,А.} см.\ Грушо~А.\,А.&&\\
\Avtors{Грушо Н.\,А.} см.\ Грушо А.\,А.&&\\
\Avtors{Грушо~Н.\,А.} см.\ Грушо~А.\,А.&&\\
\Avtors{Гусейнова~Е.\,И.} см.\ Горшенин~А.\,К.&&\\
\Avtors{Денискина~Г.\,Ю.} см.\ Босов~А.\,В.&&\\
\Avtors{Драгунов~Н.\,А., Дюкова~Е.\,В.} О~поиске максимальных частых 
и~минимальных нечастых\linebreak
\\[-12pt]
\hspace*{23pt}наборов произведения частичных порядков&1&82--87\\
\Avtors{Дубанов~А.\,А., Нефедова~В.\,А.} Кинематические модели задач преследования 
на~плос-\linebreak
\\[-12pt]
\hspace*{23pt}кости методами параллельного сближения и~погони&3&103--109\\
\Avtors{Дунсяо Гу} см.\ Зацман И.\,М.&&\\
\Avtors{Дурново~А.\,А., Инькова~О.\,Ю., Попкова~Н.\,А.} Принципы описания 
показателей логико-\linebreak
\\[-12pt]
\hspace*{23pt}семантических отношений и~их иерархии&2&52--59\\
\Avtors{Дьяченко~Ю.\,Г.} см.\ Соколов И.\,А.&&\\
\Avtors{Дюкова А.\,П.} см.\ Дюкова Е.\,В.&&\\
\Avtors{Дюкова Е.\,В., Дюкова А.\,П.} О~сложности обучения логических процедур 
классификации&4&57--62\\
\Avtors{Дюкова~Е.\,В.} см.\ Драгунов~Н.\,А.&&\\
\Avtors{Забежайло~М.\,И.} см.\ Грушо А.\,А.&&\\
\Avtors{Забежайло~М.\,И.} см.\ Грушо~А.\,А.&&\\
\Avtors{Зацаринный~А.\,А.} см.\ Грушо~А.\,А.&&\\
\Avtors{Зацман И.\,М.} О~научной парадигме информатики: верхний уровень 
классификации\linebreak
\\[-12pt]
\hspace*{23pt}объектов ее предметной области&4&73--79\\
\Avtors{Зацман~И.\,М.} Средовые модели информационных технологий: теоретические 
основа-\linebreak
\\[-12pt]
\hspace*{23pt}ния построения&3&59--67\\
\Avtors{Зацман~И.\,М., Золотарев~О.\,В., Хакимова~А.\,Х.} Средовые модели извлечения 
из текста\linebreak
\\[-12pt]
\hspace*{23pt}новых терминов и~индикаторов настроений&2&60--67\\
\Avtors{Зацман И.\,М., Золотарев~О.\,В., Хакимова~А.\,Х., Дунсяо~Гу.} Модель 
и~технология\linebreak
\\[-12pt]
\hspace*{23pt}извлечения новых терминов из~медицинских текстов&4&80--86\\
\Avtors{Зейфман~А.\,И.} см.\ Ковалёв~И.\,А.&&\\
\Avtors{Зейфман~А.\,И.} см.\ Сатин~Я.\,А.&&\\
\Avtors{Золотарев~О.\,В.} см.\ Зацман И.\,М.&&\\
\Avtors{Золотарев~О.\,В.} см.\ Зацман~И.\,М.&&\\
\Avtors{Иванов А.\,В.} см.\ Босов А.\,В.&&\\
\Avtors{Инькова~О.\,Ю.} см.\ Дурново~А.\,А.&&\\
\Avtors{Кириков~И.\,А.} см.\ Листопад~С.\,В.&&\\
\Avtors{Кириков~И.\,А.} см.\ Румовская~С.\,Б.&&\\
\Avtors{Киселёв~Г.\,А.} см.\ Смирнов~И.\,В.&&\\
\Avtors{Ковалёв~И.\,А., Сатин~Я.\,А., Синицина~А.\,В., Зейфман~А.\,И.} Об одном 
подходе к~оцениванию скорости сходимости нестационарных марковских моделей систем 
обслужи-\linebreak
\\[-12pt]
\hspace*{23pt}вания&3&75--82\\
\Avtors{Ковалёв~С.\,П.} Алгебраическая спецификация графовых вычислительных 
структур&1&2--9\\
\Avtors{Коновалов~М.\,Г., Разумчик~Р.\,В.} Синтез управления двумерным случайным 
блужданием\linebreak
\\[-12pt]
\hspace*{23pt}с~эталонным стационарным распределением&2&109--117\\
\Avtors{Кочеткова~И.\,А.} см.\ Власкина~А.\,С.&&\\
\Avtors{Кочнева~Ю.\,Ю.} см.\ Андрианов~С.\,Н.&&\\
\Avtors{Кравцова~О.\,А.} Использование критериев стационарности для настройки 
моделей при\linebreak
\\[-12pt]
\hspace*{23pt}прогнозировании временных рядов&2&11--18\\
\Avtors{Кривенко~М.\,П.} Выбор модели при факторизации матрицы данных 
с~пропусками&3&52--58\\
\Avtors{Крюкова~А.\,Л.} см.\ Сатин~Я.\,А.&&\\
\end{tabular}
}

\pagebreak

\def\leftkol{АВТОРСКИЙ УКАЗАТЕЛЬ ЗА 2022 г.} % ENGLISH ABSTRACTS}

\def\rightkol{АВТОРСКИЙ УКАЗАТЕЛЬ ЗА 2022 г.} %ENGLISH ABSTRACTS}

%\thispagestyle{myheadings}
\def\leftfootline{\small{\textbf{\thepage}
\hfill ИНФОРМАТИКА И ЕЁ ПРИМЕНЕНИЯ\ \ \ том~16\ \ \ выпуск~4\ \ \ 2022}
}%
 \def\rightfootline{\small{ИНФОРМАТИКА И ЕЁ ПРИМЕНЕНИЯ\ \ \ том~16\ \ \ выпуск~4\ \ \ 2022
 \hfill \textbf{\thepage}}}


\noindent
{\tabcolsep=3pt
\begin{tabular}{p{394pt}cc}
&\textbf{Вып.} & \textbf{Стр.}\\[3pt]
\Avtors{Курузов~И.\,А.} см.\ Смирнов~И.\,В.&&\\[0.3pt]
\Avtors{Листопад~С.\,В., Кириков~И.\,А.} Разрешение конфликтов в~гибридных 
интеллектуальных\linebreak
\\[-12pt]
\hspace*{23pt}многоагентных системах&1&54--60\\[0.3pt]
\Avtors{Малашенко~Ю.\,Е.} Метрические оценки угловых точек множества достижимых 
межуз-\linebreak
\\[-12pt]
\hspace*{23pt}ловых потоков многопользовательской сети&1&25--31\\[0.3pt]
\Avtors{Малашенко~Ю.\,Е.} Последовательный анализ и~метрические оценки 
предельных рас-\linebreak
\\[-12pt]
\hspace*{23pt}пределений межузловых потоков в~многопользовательской сети&3&45--51\\[0.3pt]
\Avtors{Мачнев Е.\,А., Бесчастный~В.\,А., Острикова~Д.\,Ю., Гайдамака~Ю.\,В., 
Шоргин~С.\,Я.} Об оптимальном расположении антенн для~V2X-соединений 
в~субтерагерцевом диа-\linebreak
\\[-12pt]
\hspace*{23pt}пазоне&4&42--50\\
\Avtors{Мачнев~Е.\,А.} см.\ Алию~Б.&&\\[0.3pt]
\Avtors{Мигуля~М.\,А.} см.\ Шнурков~П.\,В.&&\\[0.3pt]
\Avtors{Мокров~Е.\,В.} см.\ Алию~Б.&&\\[0.3pt]
\Avtors{Молчанов~Д.\,А.} см.\ Бесчастный~В.\,А.&&\\[0.3pt]
\Avtors{Нефедова~В.\,А.} см.\ Дубанов~А.\,А.&&\\[0.3pt]
\Avtors{Нуриев~В.\,А.} Переводческий анализ текста с~применением информационных 
ресурсов:\linebreak
\\[-12pt]
\hspace*{23pt}редуцирование спектра моделей перевода в~надкорпусных базах 
данных&3&68--74\\[0.3pt]
\Avtors{Острикова~Д.\,Ю.} см.\ Бесчастный~В.\,А.&&\\[0.3pt]
\Avtors{Острикова~Д.\,Ю.} см.\ Мачнев Е.\,А.&&\\[0.3pt]
\Avtors{Ошушкова~В.\,С.} см.\ Сатин~Я.\,А.&&\\[0.3pt]
\Avtors{Палионная~С.\,И., Шестаков~О.\,В.} Использование FDR-метода множественной 
провер-\linebreak
\\[-12pt]
\hspace*{23pt}ки гипотез при обращении линейных однородных операторов&2&44--51\\[0.3pt]
\Avtors{Панов~А.\,И.} см.\ Смирнов~И.\,В.&&\\[0.3pt]
\Avtors{Пешкова И.\,В.} Границы экстремального индекса времени ожидания в~системе 
$M/G/1$\linebreak
\\[-12pt]
\hspace*{23pt}с~распределением времени обслуживания в~виде конечной смеси&4&26--33\\[0.3pt]
\Avtors{Пешкова~И.\,В.} Сравнение экстремальных индексов времен ожидания 
в~системах об-\linebreak
\\[-12pt]
\hspace*{23pt}служивания $M/G/1$&1&61--67\\[0.3pt]
\Avtors{Попкова~Н.\,А.} см.\ Дурново~А.\,А.&&\\[0.3pt]
\Avtors{Разумчик~Р.\,В.} см.\ Коновалов~М.\,Г.&&\\[0.3pt]
\Avtors{Рождественский~Ю.\,В.} см.\ Соколов И.\,А.&&\\[0.3pt]
\Avtors{Румовская~С.\,Б., Кириков~И.\,А.} Метод визуализации снижения интенсивности 
и~разре-\linebreak
\\[-12pt]
\hspace*{23pt}шения конфликтов в~гибридных интеллектуальных многоагентных 
системах&2&\hphantom{1}94--101\\[0.3pt]
\Avtors{Сатин~Я.\,А., Крюкова~А.\,Л., Ошушкова~В.\,С., Зейфман~А.\,И.} 
О~монотонности\linebreak
\\[-12pt]
\hspace*{23pt}некоторых классов марковских цепей&2&27--34\\[0.3pt]
\Avtors{Сатин~Я.\,А.} см.\ Ковалёв~И.\,А.&&\\[0.3pt]
\Avtors{Синицина~А.\,В.} см.\ Ковалёв~И.\,А.&&\\[0.3pt]
\Avtors{Синицын~И.\,Н.} Нормализация систем, стохастически не разрешенных 
относительно\linebreak
\\[-12pt]
\hspace*{23pt}производных&1&32--38\\[0.3pt]
\Avtors{Синицын~И.\,Н.} Совместная фильтрация и~распознавание нормальных 
процессов в~сто-\linebreak
\\[-12pt]
\hspace*{23pt}хастических системах, не разрешенных относительно 
производных&2&85--93\\
\Avtors{Смирнов~Д.\,В.} см.\ Грушо А.\,А.&&\\[0.3pt]
\Avtors{Смирнов~И.\,В., Панов~А.\,И., Чуганская~А.\,А., Суворова~М.\,И., 
Киселёв~Г.\,А., Курузов~И.\,А., Григорьев~О.\,Г.} Персональный когнитивный 
ассистент: планирование поведения\linebreak
\\[-12pt]
\hspace*{23pt}на основе сценариев деятельности&1&46--53\\[0.3pt]
\Avtors{Соколов И.\,А., Степченков~Ю.\,А., Дьяченко~Ю.\,Г., Рождественский~Ю.\,В.} 
Оценка надеж-\linebreak
\\[-12pt]
\hspace*{23pt}ности синхронного и~самосинхронного конвейеров&4&2--7\\[0.3pt]
\Avtors{Степченков~Ю.\,А.} см.\ Соколов И.\,А.&&\\[0.3pt]
\Avtors{Ступников~С.\,А.} см.\ Брюхов Д.\,О.&&\\[0.3pt]
\Avtors{Суворова~М.\,И.} см.\ Смирнов~И.\,В.&&\\[0.3pt]
\Avtors{Сучков А.\,П.} Единая модель государственных данных: сценарии 
развития&4&\hphantom{9}99--105\\[0.3pt]
\Avtors{Тимонина~Е.\,Е.} см.\ Грушо А.\,А.&&\\[0.3pt]
\Avtors{Тимонина~Е.\,Е.} см.\ Грушо~А.\,А.&&\\[0.3pt]
\Avtors{Тимонина~Е.\,Е} см.\ Грушо~А.\,А.&&\\
\end{tabular}
}

\pagebreak

\def\leftkol{АВТОРСКИЙ УКАЗАТЕЛЬ ЗА 2022 г.} % ENGLISH ABSTRACTS}

\def\rightkol{АВТОРСКИЙ УКАЗАТЕЛЬ ЗА 2022 г.} %ENGLISH ABSTRACTS}

%\thispagestyle{myheadings}
\def\leftfootline{\small{\textbf{\thepage}
\hfill ИНФОРМАТИКА И ЕЁ ПРИМЕНЕНИЯ\ \ \ том~16\ \ \ выпуск~4\ \ \ 2022}
}%
 \def\rightfootline{\small{ИНФОРМАТИКА И ЕЁ ПРИМЕНЕНИЯ\ \ \ том~16\ \ \ выпуск~4\ \ \ 2022
 \hfill \textbf{\thepage}}}


\noindent
{\tabcolsep=3pt
\begin{tabular}{p{394pt}cc}
&\textbf{Вып.} & \textbf{Стр.}\\[3pt]
\Avtors{Торшин~И.\,Ю.} О~применении топологического подхода к анализу плохо 
формализуемых задач для построения алгоритмов виртуального скрининга кван\-то\-во-ме\-ха\-ни\-че\-ских\linebreak
\\[-12pt]
\hspace*{23pt}свойств органических молекул I:~Основы проблемно ориентированной 
теории&1&39--45\\
\Avtors{Торшин~И.\,Ю.} О~применении топологического подхода к~анализу плохо 
формализуемых задач для построения алгоритмов виртуального скрининга кван\-то\-во-ме\-ха\-ни\-че\-ских 
свойств органических молекул II:~Сопоставление формализма 
с~конструктами\linebreak
\\[-12pt]
\hspace*{23pt}квантовой механики и экспериментальная апробация предложенных 
алгоритмов&2&35--43\\
\Avtors{Хакимова~А.\,Х.} см.\ Зацман И.\,М.&&\\
\Avtors{Хакимова~А.\,Х.} см.\ Зацман~И.\,М.&&\\
\Avtors{Хацкевич В.\,Л.} Нечеткие усредняющие операторы в~задаче агрегирования 
нечеткой\linebreak
\\[-12pt]
\hspace*{23pt}информации&4&51--56\\
\Avtors{Чуганская~А.\,А.} см.\ Смирнов~И.\,В.&&\\
\Avtors{Шведов~А.\,С.} Критерий непустоты эпсилон-ядер для нечетких игр с~нетрансферабель-\linebreak
\\[-12pt]
\hspace*{23pt}ной полезностью и~вычислительные процедуры&3&2--6\\
\Avtors{Шестаков О.\,В.} Несмещенная оценка риска пороговой обработки с~двумя 
пороговыми\linebreak
\\[-12pt]
\hspace*{23pt}значениями&4&14--19\\
\Avtors{Шестаков~О.\,В.} см.\ Палионная~С.\,И.&&\\
\Avtors{Шихиев~Ф.\,Ш.} см.\ Шихиев~Ш.\,Б.&&\\
\Avtors{Шихиев~Ш.\,Б., Шихиев~Ф.\,Ш.} Упрощенный язык зрительных 
образов&1&68--72\\
\Avtors{Шнурков~П.\,В.} Об аналитической структуре некоторых видов целевых 
функционалов,\linebreak
\\[-12pt]
\hspace*{23pt}связанных с~задачами управления полумарковскими случайными 
процессами&2&75--84\\
\Avtors{Шнурков~П.\,В., Мигуля~М.\,А.} Некоторые результаты анализа процесса 
изменения цены\linebreak
\\[-12pt]
\hspace*{23pt}бивалютной корзины на основе методов статистики случайных 
процессов&3&16--25\\
\Avtors{Шоргин~С.\,Я.} см.\ Бесчастный~В.\,А.&&\\
\Avtors{Шоргин~С.\,Я.} см.\ Власкина~А.\,С.&&\\
\Avtors{Шоргин~С.\,Я.} см.\ Грушо А.\,А.&&\\
\Avtors{Шоргин~С.\,Я.} см.\ Грушо~А.\,А.&&\\
\Avtors{Шоргин~С.\,Я.} см.\ Мачнев Е.\,А.&&\\
\end{tabular}
}

%\thispagestyle{myheadings}
\def\leftfootline{\small{\textbf{\thepage}
\hfill ИНФОРМАТИКА И ЕЁ ПРИМЕНЕНИЯ\ \ \ том~16\ \ \ выпуск~4\ \ \ 2022}
}%
 \def\rightfootline{\small{ИНФОРМАТИКА И ЕЁ ПРИМЕНЕНИЯ\ \ \ том~16\ \ \ выпуск~4\ \ \ 2022
 \hfill \textbf{\thepage}}}

 \label{end\stat}

\newpage

\def\stat{cont-e}
{%\hrule\par
%\vskip 7pt % 7pt
\raggedleft\Large \bf%\baselineskip=3.2ex
2\,0\,2\,2\ \ A\,U\,T\,H\,O\,R\ \ I\,N\,D\,E\,X \vskip 17pt
 \hrule
 \par
\vskip 21pt plus 6pt minus 3pt }

\label{st\stat}

\def\tit{\ }

\def\aut{\ }
\def\auf{\ }

\def\leftkol{\ } %2021 AUTHOR INDEX} % ENGLISH ABSTRACTS}

\def\rightkol{\ } %2021 AUTHOR INDEX} %ENGLISH ABSTRACTS}

\titele{\tit}{\aut}{\auf}{\leftkol}{\rightkol}
\addcontentsline{toc}{subsection}{\textrm\textbf 2022 Author Index}

\def\leftfootline{\small{\textbf{\thepage}
\hfill INFORMATIKA I EE PRIMENENIYA~--- INFORMATICS AND APPLICATIONS\ \ \ 2022\
\ \ volume~16\ \ \ issue\ 4}
}%
 \def\rightfootline{\small{INFORMATIKA I EE PRIMENENIYA~--- INFORMATICS AND APPLICATIONS\ \ \ 2022\ \ \ volume~16\ \ \ issue\ 4
\hfill \textbf{\thepage}}}

\vspace*{-24pt}

\noindent
{\tabcolsep=3pt
\begin{tabular}{p{395.89pt}cc}
&\textbf{Issue} & \textbf{Page}\\[6pt]
\Avtors{Abgaryan~K.\,K.\ and Gavrilov~E.\,S.} Software package for multiscale modeling of 
structural\linebreak
\\[-12pt]
\hspace*{23pt}properties of composite materials&1&88--97\\
\Avtors{Ablaev~F.\,M.} see Andrianov~S.\,N.&&\\
\Avtors{Agalarov Ya.\,M.} Optimal control of~a~queue-length dependent additional server 
in~$\mathrm{GI}/M/1$\linebreak
\\[-12pt]
\hspace*{23pt}queue&4&34--41\\
\Avtors{Agalarov~Ya.\,M.} Optimization of the threshold service speed control in the $G/M/1$ 
queue&1&73--81\\
\Avtors{Agasandyan~G.\,A.} Multidimensional binary markets and CC-VaR&2&\hphantom{1}2--10\\
\Avtors{Aliyu~B., Machnev~E.\,A., and Mokrov~E.\,V.} Hysteretic congestion control in 
wireless cloud\linebreak
\\[-12pt]
\hspace*{23pt}sensor networks&3&83--89\\
\Avtors{Andrianov~S.\,N., Andrianova~N.\,S., Ablaev~F.\,M., and Kochneva~Yu.\,Yu.} 
Context query on\linebreak
\\[-12pt]
\hspace*{23pt}photons with the use of Bell tests&1&20--24\\
\Avtors{Andrianova~N.\,S.} see Andrianov~S.\,N.&&\\
\Avtors{Bazilevskiy M.\,P.} Generalization of~a~method for~straightening coefficients 
distorted due~to~mul-\linebreak
\\[-12pt]
\hspace*{23pt}ticollinearity in~regression models with different degrees of~explanatory 
variables correlation&4&20--25\\
\Avtors{Beschastnyi~V.\,A., Ostrikova~D.\,Yu., Shorgin~S.\,Ya., Moltchanov~D.\,A., and 
Gaidamaka~Yu.\,V.}\linebreak
\\[-12pt]
\hspace*{23pt}Density analysis of mmWave NR deployments for delivering scalable 
AR/VR video services&2&102--108\\
\Avtors{Beschastnyi~V.\,A.} see Machnev E.\,A.&&\\
\Avtors{Bityukov~Yu.\,I.} see Bosov~A.\,V.&&\\
\Avtors{Borisov A.\,V.} Total approximation order for~Markov jump process filtering given 
discretized\linebreak
\\[-12pt]
\hspace*{23pt}observations&4&8--13\\
\Avtors{Bosov~A.\,V.} Application of self-organizing neural networks to the process of forming 
an individual\linebreak
\\[-12pt]
\hspace*{23pt}learning path&3&\hphantom{1}7--15\\
\Avtors{Bosov~A.\,V.} Linear output control of Markov chain by square criterion. Complete 
information\linebreak
\\[-12pt]
\hspace*{23pt}case&2&19--26\\
\Avtors{Bosov~A.\,V., Bityukov~Yu.\,I., and Deniskina~G.\,Yu.} About searching for the 
optimal 3D printing\linebreak
\\[-12pt]
\hspace*{23pt}scheme of structures from composite materials&1&10--19\\
\Avtors{Bosov A.\,V. and Ivanov~A.\,V.} Technology for~classification of~content types of~e-textbooks&4&63--72\\
\Avtors{Briukhov D.\,O. and Stupnikov~S.\,A.} Logical relational model of~data structures 
for~problem\linebreak
\\[-12pt]
\hspace*{23pt}solving in~land use management&4&93--98\\
\Avtors{Burtseva~S.\,A.} see Vlaskina~A.\,S.&&\\
\Avtors{Chuganskaya~A.\,A.} see Smirnov~I.\,V&&\\
\Avtors{Deniskina~G.\,Yu.} see Bosov~A.\,V.&&\\
\Avtors{Diachenko~Yu.\,G.} see Sokolov I.\,A.&&\\
\Avtors{Djukova~A.\,P.} see Djukova E.\,V.&&\\
\Avtors{Djukova E.\,V. and Djukova~A.\,P.} On the~complexity of~logical classification 
learning procedures&4&57--62\\
\Avtors{Djukova~E.\,V.} see Dragunov~N.\,A.&&\\
\Avtors{Dongxiao~Gu} see Zatsman I.\,M.&&\\
\Avtors{Dragunov~N.\,A.\ and Djukova~E.\,V.} Finding maximal frequent and minimal 
infrequent sets\linebreak
\\[-12pt]
\hspace*{23pt}in partially ordered data&1&82--87\\
\Avtors{Dubanov~A.\,A.\ and Nefedova~V.\,A.} Kinematic models of pursuit problems on the 
plane\linebreak
\\[-12pt]
\hspace*{23pt}by the methods of parallel approach and pursuit&3&103--109\\
\Avtors{Durnovo~A.\,A., Inkova~O.\,Yu., and Popkova~N.\,A.} Principles of describing 
markers of logical-\linebreak
\\[-12pt]
\hspace*{23pt}semantic relations and their hierarchy&2&52--59\\
\Avtors{Gaidamaka~Yu.\,V.} see Beschastnyi~V.\,A.&&\\
\Avtors{Gaidamaka~Yu.\,V.} see Machnev E.\,A.&&\\
\Avtors{Gavrilov~E.\,S.} see Abgaryan~K.\,K.&&\\

\end{tabular}
}
\pagebreak

\def\leftfootline{\small{\textbf{\thepage}
\hfill INFORMATIKA I EE PRIMENENIYA~--- INFORMATICS AND APPLICATIONS\ \ \ 2022\
\ \ volume~16\ \ \ issue\ 4}
}%
 \def\rightfootline{\small{INFORMATIKA I EE PRIMENENIYA~---
INFORMATICS AND APPLICATIONS\ \ \ 2022\ \ \ volume~16\ \ \ issue\ 4
\hfill \textbf{\thepage}}}

\def\leftkol{2022 AUTHOR INDEX} % ENGLISH ABSTRACTS}

\def\rightkol{2022 AUTHOR INDEX} %ENGLISH ABSTRACTS}


\noindent
{\tabcolsep=3pt
\begin{tabular}{p{395.5pt}cc}
&\textbf{Issue} & \textbf{Page}\\[6pt]
\Avtors{Gorshenin~A.\,K.\ and Guseynova~E.\,I.} Increasing FOREX trading profitability with 
LSTM\linebreak
\\[-12pt]
\hspace*{23pt}candlestick pattern recognition and tick volume indicator&3&26--38\\
\Avtors{Grigoriev~O.\,G.} see Smirnov~I.\,V&&\\[-0.1pt]
\Avtors{Grusho~A.\,A., Grusho~N.\,A., and Timonina~E.\,E.} Metadata in secure electronic 
document\linebreak
\\[-12pt]
\hspace*{23pt}management&3&\hphantom{1}97--102\\[-0.1pt]
\Avtors{Grusho A.\,A., Grusho~N.\,A., Zabezhailo~M.\,I., Smirnov~D.\,V., Timonina~E.\,E., 
and Shorgin~S.\,Ya.}\linebreak
\\[-12pt]
\hspace*{23pt}About the~secure architecture of~a~microservice-based computing 
system&4&87--92\\[-0.1pt]
\Avtors{Grusho~A.\,A., Grusho~N.\,A., Zabezhailo~M.\,I., Zatsarinny~A.\,A., 
Timonina~E.\,E.,}\linebreak
\\[-12pt]
\hspace*{23pt}\textbf{and Shorgin~S.\,Ya.} Cause-and-effect chain analysis&2&68--74\\
\Avtors{Grusho~N.\,A.} see Grusho A.\,A.&&\\[-0.1pt]
\Avtors{Grusho~N.\,A.} see Grusho~A.\,A.&&\\[-0.1pt]
\Avtors{Grusho~N.\,A.} see Grusho~A.\,A.&&\\[-0.1pt]
\Avtors{Guseynova~E.\,I.} see Gorshenin~A.\,K.&&\\
\Avtors{Inkova~O.\,Yu.} see Durnovo~A.\,A.&&\\[-0.1pt]
\Avtors{Ivanov~A.\,V.} see Bosov A.\,V.&&\\[-0.1pt]
\Avtors{Khakimova~A.\,K.} see Zatsman I.\,M.&&\\[-0.1pt]
\Avtors{Khakimova~A.\,K.} see Zatsman~I.\,M.&&\\[-0.1pt]
\Avtors{Khatskevich V.\,L.} Fuzzy averaging operators in~the~problem of~aggregating fuzzy 
information&4&51--56\\[-0.1pt]
\Avtors{Kirikov~I.\,A.} see Listopad~S.\,V.&&\\[-0.1pt]
\Avtors{Kirikov~I.\,A.} see Rumovskaya~S.\,B.&&\\[-0.1pt]
\Avtors{Kiselev~G.\,A.} see Smirnov~I.\,V&&\\[-0.1pt]
\Avtors{Kochetkova~I.\,A.} see Vlaskina~A.\,S.&&\\[-0.1pt]
\Avtors{Kochneva~Yu.\,Yu.} see Andrianov~S.\,N.&&\\[-0.1pt]
\Avtors{Konovalov~M.\,G.\ and Razumchik~R.\,V.} Controlling a bounded two-dimensional 
Markov chain\linebreak
\\[-12pt]
\hspace*{23pt}with a~given invariant measure&2&109--117\\[-0.1pt]
\Avtors{Kovalev~I.\,A., Satin~Y.\,A., Sinitcina~A.\,V., and Zeifman~A.\,I.} On an approach 
for estimating\linebreak
\\[-12pt]
\hspace*{23pt}the rate of convergence for nonstationary Markov models of queueing 
systems&3&75--82\\[-0.1pt]
\Avtors{Kovalyov~S.\,P.} Algebraic specification of graph computational structures&1&2--9\\
\Avtors{Kravtsova~O.\,A.} Model setting using stationarity criteria for time series 
forecasting&2&11--18\\[-0.1pt]
\Avtors{Krivenko~M.\,P.} Model selection for matrix factorization with missing 
components&3&52--58\\[-0.1pt]
\Avtors{Kryukova~A.\,L.} see Satin~Y.\,A.&&\\[-0.1pt]
\Avtors{Kuruzov~I.\,A.} see Smirnov~I.\,V&&\\[-0.1pt]
\Avtors{Listopad~S.\,V.\ and Kirikov~I.\,A.} Conflict resolution in hybrid intelligent multiagent 
systems&1&54--60\\[-0.1pt]
\Avtors{Machnev E.\,A., Beschastnyi~V.\,A., Ostrikova~D.\,Yu., Gaidamaka~Yu.\,V., and 
Shorgin~S.\,Ya.} On\linebreak
\\[-12pt]
\hspace*{23pt}the optimal antenna deployment for~subterahertz V2X 
communications&4&42--50\\[-0.1pt]
\Avtors{Machnev~E.\,A.} see Aliyu~B.&&\\[-0.1pt]
\Avtors{Malashenko~Yu.\,E.} Metric evaluations of the angular points of the set of attainable 
internodal\linebreak
\\[-12pt]
\hspace*{23pt}flows of multiuser network&1&25--31\\[-0.1pt]
\Avtors{Malashenko~Yu.\,E.} Sequential analysis and metric estimates of peak load flows in 
the multiuser\linebreak
\\[-12pt]
\hspace*{23pt}network&3&45--51\\[-0.1pt]
\Avtors{Migulya~M.\,A.} see Shnurkov~P.\,V.&&\\[-0.1pt]
\Avtors{Mokrov~E.\,V.} see Aliyu~B.&&\\[-0.1pt]
\Avtors{Moltchanov~D.\,A.} see Beschastnyi~V.\,A.&&\\[-0.1pt]
\Avtors{Nefedova~V.\,A.} see Dubanov~A.\,A.&&\\[-0.1pt]
\Avtors{Nuriev~V.\,A.} Computer-assisted textual analysis in translation: Reducing the 
spectrum of\linebreak
\\[-12pt]
\hspace*{23pt}translation models in supracorpora databases&3&68--74\\[-0.1pt]
\Avtors{Oshushkova~V.\,S.} see Satin~Y.\,A.&&\\[-0.1pt]
\Avtors{Ostrikova~D.\,Yu.} see Beschastnyi~V.\,A.&&\\[-0.1pt]
\Avtors{Ostrikova~D.\,Yu.} see Machnev E.\,A.&&\\[-0.1pt]
\Avtors{Palionnaya~S.\,I.\ and Shestakov~O.\,V.} The use of the FDR method of multiple 
hypothesis testing\linebreak
\\[-12pt]
\hspace*{23pt}when inverting linear homogeneous operators&2&44--51\\[-0.1pt]
\Avtors{Panov~A.\,I.} see Smirnov~I.\,V&&\\[-0.1pt]
\Avtors{Peshkova I.\,V.} On bounds of~the~stationary waiting time extremal index 
in~$M/G/1$ system\linebreak
\\[-12pt]
\hspace*{23pt}with mixture service times&4&26--33\\[-0.1pt]
\end{tabular}
}
\pagebreak

\def\leftfootline{\small{\textbf{\thepage}
\hfill INFORMATIKA I EE PRIMENENIYA~--- INFORMATICS AND APPLICATIONS\ \ \ 2022\
\ \ volume~16\ \ \ issue\ 4}
}%
 \def\rightfootline{\small{INFORMATIKA I EE PRIMENENIYA~---
INFORMATICS AND APPLICATIONS\ \ \ 2022\ \ \ volume~16\ \ \ issue\ 4
\hfill \textbf{\thepage}}}

\def\leftkol{2022 AUTHOR INDEX} % ENGLISH ABSTRACTS}

\def\rightkol{2022 AUTHOR INDEX} %ENGLISH ABSTRACTS}


\noindent
{\tabcolsep=3pt
\begin{tabular}{p{395.5pt}cc}
&\textbf{Issue} & \textbf{Page}\\[6pt]
\Avtors{Peshkova~I.\,V.} The comparison of waiting time extremal indexes in $M/G/1$ 
queueing systems&1&61--67\\[-0.1pt]
\Avtors{Popkova~N.\,A.} see Durnovo~A.\,A.&&\\[-0.1pt]
\Avtors{Razumchik~R.\,V.} see Konovalov~M.\,G.&&\\[-0.1pt]
\Avtors{Rogdestvenski~Yu.\,V.} see Sokolov I.\,A.&&\\[-0.1pt]
\Avtors{Rumovskaya~S.\,B.\ and Kirikov~I.\,A.} Visual representation of the decrease in 
conflict intensity\linebreak
\\[-12pt]
\hspace*{23pt}and its resolution in hybrid intelligent multiagent 
systems&2&\hphantom{1}94--101\\[-0.1pt]
\Avtors{Satin~Y.\,A., Kryukova~A.\,L., Oshushkova~V.\,S., and Zeifman~A.\,I.} On 
monotonicity of some\linebreak
\\[-12pt]
\hspace*{23pt}classes of Markov chains&2&27--34\\[-0.1pt]
\Avtors{Satin~Y.\,A.} see Kovalev~I.\,A.&&\\[-0.1pt]
\Avtors{Shestakov O.\,V.} Unbiased thresholding risk estimate with two threshold 
values&4&14--19\\[-0.1pt]
\Avtors{Shestakov~O.\,V.} see Palionnaya~S.\,I.&&\\[-0.1pt]
\Avtors{Shihiev~F.\,Sh.} see Shihiev~Sh.\,B.&&\\[-0.1pt]
\Avtors{Shihiev~Sh.\,B.\ and Shihiev~F.\,Sh.} Simplified language for visual images&1&68--72\\[-0.1pt]
\Avtors{Shnurkov~P.\,V.} On the analytical structure of some kinds of target functionals 
associated with\linebreak
\\[-12pt]
\hspace*{23pt}the control problems of semi-Markov stoсhastic processes&2&75--84\\[-0.1pt]
\Avtors{Shnurkov~P.\,V.\ and Migulya~M.\,A.} Some results of the analysis of the process of 
changing\linebreak
\\[-12pt]
\hspace*{23pt}the price of a dual currency basket based on random process statistics 
methods&3&16--25\\[-0.1pt]
\Avtors{Shorgin~S.\,Ya.} see Beschastnyi~V.\,A.&&\\[-0.1pt]
\Avtors{Shorgin~S.\,Ya.} see Grusho A.\,A.&&\\[-0.1pt]
\Avtors{Shorgin~S.\,Ya.} see Grusho~A.\,A.&&\\[-0.1pt]
\Avtors{Shorgin~S.\,Ya.} see Machnev E.\,A.&&\\[-0.1pt]
\Avtors{Shorgin~S.\,Ya.} see Vlaskina~A.\,S.&&\\[-0.1pt]
\Avtors{Shvedov~A.\,S.} A~condition for non-emptiness of the epsilon-core of 
a~nontransferable utility\linebreak
\\[-12pt]
\hspace*{23pt}fuzzy game and computational schemes&3&2--6\\[-0.1pt]
\Avtors{Sinitcina~A.\,V.} see Kovalev~I.\,A.&&\\[-0.1pt]
\Avtors{Sinitsyn~I.\,N.} Joint filtration and recognition of normal proсesses in stochastic 
systems with\linebreak
\\[-12pt]
\hspace*{23pt}unsolved derivatives&2&85--93\\[-0.1pt]
\Avtors{Sinitsyn~I.\,N.} Normalization of systems with stochastically unsolved 
derivatives&1&32--38\\[-0.1pt]
\Avtors{Smirnov~D.\,V.} see Grusho A.\,A.&&\\[-0.1pt]
\Avtors{Smirnov~I.\,V., Panov~A.\,I., Chuganskaya~A.\,A., Suvorova~M.\,I., Kiselev~G.\,A., 
Kuruzov~I.\,A., and}\linebreak
\\[-12pt]
\hspace*{23pt}\textbf{Grigoriev~O.\,G.} Personal cognitive assistant: Planning activity with 
scripts&1&46--53\\[-0.1pt]
\Avtors{Sokolov I.\,A., Stepchenkov Yu.\,A., Diachenko~Yu.\,G., 
and~Rogdestvenski~Yu.\,V.} Synchronous and\linebreak
\\[-12pt]
\hspace*{23pt}self-timed pipeline's reliability 
estimation&4&2--7\\[-0.1pt]
\Avtors{Stepchenkov Yu.\,A.} see Sokolov I.\,A.&&\\[-0.1pt]
\Avtors{Stupnikov~S.\,A.} see Briukhov D.\,O.&&\\[-0.1pt]
\Avtors{Suchkov A.\,P.} Unified model of national data: Development scenarios&4&\hphantom{9}99--105\\
\Avtors{Suvorova~M.\,I.} see Smirnov~I.\,V&&\\[-0.1pt]
\Avtors{Timonina~E.\,E.} see Grusho A.\,A.&&\\[-0.1pt]
\Avtors{Timonina~E.\,E.} see Grusho~A.\,A.&&\\[-0.1pt]
\Avtors{Timonina~E.\,E.} see Grusho~A.\,A.&&\\[-0.1pt]
\Avtors{Torshin~I.\,Yu.} On the application of a~topological approach to analysis of poorly 
formalized problems for constructing algorithms for virtual screening of quantum-mechanical 
properties\linebreak
\\[-12pt]
\hspace*{23pt}of organic molecules I:~The basics of the problem-oriented theory&1&39--45\\[-0.1pt]
\Avtors{Torshin~I.\,Yu.} On the application of a topological approach to analysis of poorly 
formalized problems for constructing algorithms for virtual screening of quantum-mechanical 
properties\linebreak
\\[-12pt]
\hspace*{23pt}of organic molecules II:~Comparison of formalism with constructions of quantum mechan-\linebreak
\\[-12pt]
\hspace*{23pt}ics and experimental approbation of the proposed algorithms&2&35--43\\[-0.1pt]
\Avtors{Vasilyev~N.\,S.} On extremum sufficient conditions in multidimensional variation 
calculus\linebreak
\\[-12pt]
\hspace*{23pt}problems&3&39--44\\[-0.1pt]
\Avtors{Vlaskina~A.\,S., Burtseva~S.\,A., Kochetkova~I.\,A., and Shorgin~S.\,Ya.} 
Controllable queuing system\linebreak
\\[-12pt]
\hspace*{23pt}with elastic traffic and signals for analyzing network 
slicing&3&90--96\\[-0.1pt]
\Avtors{Zabezhailo~M.\,I.} see Grusho A.\,A.&&\\[-0.1pt]
\Avtors{Zabezhailo~M.\,I.} see Grusho~A.\,A.&&\\[-0.1pt]
\Avtors{Zatsarinny~A.\,A.} see Grusho~A.\,A.&&\\[-0.1pt]
\end{tabular}
}
\pagebreak

\def\leftfootline{\small{\textbf{\thepage}
\hfill INFORMATIKA I EE PRIMENENIYA~--- INFORMATICS AND APPLICATIONS\ \ \ 2022\
\ \ volume~16\ \ \ issue\ 4}
}%
 \def\rightfootline{\small{INFORMATIKA I EE PRIMENENIYA~---
INFORMATICS AND APPLICATIONS\ \ \ 2022\ \ \ volume~16\ \ \ issue\ 4
\hfill \textbf{\thepage}}}

\def\leftkol{2022 AUTHOR INDEX} % ENGLISH ABSTRACTS}

\def\rightkol{2022 AUTHOR INDEX} %ENGLISH ABSTRACTS}


\noindent
{\tabcolsep=3pt
\begin{tabular}{p{395.5pt}cc}
&\textbf{Issue} & \textbf{Page}\\[6pt]
\Avtors{Zatsman~I.\,M.} Informatics' medium models of information technology: Theoretical 
foundations\linebreak
\\[-12pt]
\hspace*{23pt}for their creating&3&59--67\\
\Avtors{Zatsman I.\,M.} On the~scientific paradigm of~informatics: The~classification high 
level of~its~objects&4&73--79\\
\Avtors{Zatsman~I.\,M., Zolotarev~O.\,V., and Khakimova~A.\,K.} Medium models for 
discovering novel\linebreak
\\[-12pt]
\hspace*{23pt}terms and sentiments from texts&2&60--67\\
\Avtors{Zatsman I.\,M., Zolotarev~O.\,V., Khakimova~A.\,K., and~Dongxiao~Gu.} Model and 
technology\linebreak
\\[-12pt]
\hspace*{23pt}for discovering new terms in medical texts&4&80--86\\
\Avtors{Zeifman~A.\,I.} see Kovalev~I.\,A.&&\\
\Avtors{Zeifman~A.\,I.} see Satin~Y.\,A.&&\\
\Avtors{Zolotarev~O.\,V.} see Zatsman I.\,M.&&\\
\Avtors{Zolotarev~O.\,V.} see Zatsman~I.\,M.&&\\
\end{tabular}
}

%\thispagestyle{myheadings}
\def\leftfootline{\small{\textbf{\thepage}
\hfill INFORMATIKA I EE PRIMENENIYA~--- INFORMATICS AND APPLICATIONS\ \ \ 2022\
\ \ volume~16\ \ \ issue\ 4}
}%
 \def\rightfootline{\small{INFORMATIKA I EE PRIMENENIYA~---
INFORMATICS AND APPLICATIONS\ \ \ 2022\ \ \ volume~16\ \ \ issue\ 4
\hfill \textbf{\thepage}}}

 \label{end\stat}

\newpage

%
   \vspace*{-46pt}

\begin{center}
\vspace*{4pt}
\mbox{%

\epsfxsize=55mm %112.705
\epsfbox{zhur-2.eps}
}
%\end{center}

\vspace*{10pt} 


%   \begin{center}
\fbox{\large\textbf{Академик Юрий Иванович Журавлёв}}\\[10pt]
\textbf{\large 14.01.1935--14.01.2022}
   \end{center}


   %\vspace*{2.5mm}

   \vspace*{5mm}

   \thispagestyle{empty}

%\

%\vspace*{-12pt}
       


В январе этого года ушел из жизни главный научный сотрудник Федерального исследовательского 
центра <<Информатика и управление>> РАН, председатель Редакционного совета журнала 
<<Информатика и~её применения>> академик Юрий Иванович Журавлёв. В~его лице мировая 
наука потеряла одного из своих ярчайших представителей~--- выдающегося ученого-исследователя 
и~талантливого ученого-организатора.

Юрий Иванович родился в Воронеже в 1935~г.\ в семье ученого и врача. Среднее образование 
получил в школе №\,6 г.~Фрунзе (ныне Бишкек) Киргизской ССР. В~1952~г.\ поступил на 
ме\-ха\-ни\-ко-ма\-те\-ма\-ти\-че\-ский факультет МГУ им.\ М.\,В.~Ломоносова. В~1957~г.\ Юрий Иванович 
защищает диплом и продолжает обучение в аспирантуре Московского университета на кафедре 
вычислительной математики (возглавляемой тогда академиком С.\,Л.~Соболевым). После 
успешной защиты кандидатской диссертации (к.ф.-м.н., 1959 г., научный руководитель~--- 
А.\,А.~Ляпунов, оппоненты~--- чл.-корр.\ А.\,А.~Марков, к.ф.-м.н.\ О.\,Б.~Лупанов) и~до 
окончательного переезда в Москву в 1969~г.\ работал в Институте математики Сибирского 
отделения АН СССР, занимая в нем последовательно должности младшего научного сотрудника, 
заведующего отделом, заведующего отделением, заместителя директора по научной работе. 
В~этот период (1954--1966~гг.)\ им был опубликован цикл работ по решению задач алгебры и 
математической логики, причем полученные результаты применялись для создания эффективных 
программ для ЭВМ, конструирования схем и сетей для обработки информации. Наиболее значимый 
результат этого периода научной работы~--- обоснование нового направления исследований, 
общей теории локальных алгоритмов. В~ней были окончательно объединены топологические 
принципы и теория алгоритмов. Эта теория и легла в основу докторской диссертации Юрия 
Ивановича (д.ф.-м.н., 1965~г.)\ по еще тогда новой научной специальности <<Математическая 
кибернетика>>. Оппонировали ему как специалисты по кибернетике~--- академик 
В.\,М.~Глушков, член-корреспондент А.\,А.~Ляпунов и О.\,Б.~Лупанов, так и про\-фес\-сор-ал\-геб\-раист А.\,Д.~Тайманов. 

В 1969~г.\ Юрий Иванович переезжает в Москву и возглавляет в Вычислительном центре АН 
СССР лабораторию проблем распознавания. Впоследствии он~--- заместитель директора по 
научной работе. Научные интересы этого периода связаны с проблемами классификации или 
распознавания образов. В~1976--1978~гг.\ Юрий Иванович публикует цикл работ по ставшему 
вскоре знаменитым алгебраическому подходу к проблеме синтеза корректных алгоритмов. Эти 
работы определили современное состояние всей проблематики распознавания и многих смежных 
областей прикладной математики и информатики. В~своих основополагающих работах Юрий 
Иванович показал, что можно в явном виде строить экстремальные по качеству алгоритмы для 
решения очень широких классов плохо формализованных задач. 
{\looseness=-1

}





Научные заслуги Юрия Ивановича получили широкое признание. В~1966~г.\ он совместно с 
О.\,Б.~Лупановым и чле\-ном-кор\-рес\-пон\-ден\-том АН СССР С.\,В.~Яблонским были удостоены 
звания лауреата Ленинской премии в~об\-ласти науки и техники. В~1984~г.\ Юрий Иванович 
был избран членом-корреспондентом АН СССР (по специальности <<Информатика>>), 
а~в~1992~г.~--- академиком РАН (по той же специальности).\linebreak\vspace*{-12pt}

\pagebreak

\

\vspace*{-46pt}

\noindent
\begin{floatingfigure}{48mm}
\begin{center}
%\vspace*{6pt}
\mbox{%

\epsfxsize=46mm %112.705
\epsfbox{zhur-3.eps}
}
\end{center}
\vspace*{6pt}
\end{floatingfigure}

 \thispagestyle{empty}

\noindent
В~1986~г.\ за цикл прикладных 
работ ему и ряду его учеников была при\-суж\-де\-на премия Совета Министров СССР. Он являлся 
членом иностранных академий наук, председателем секции <<Прикладная математика
 и~информатика>> Отделения математических наук РАН, председателем экспертного совета ВАК 
России по управ\-ле\-нию и информатике, заслуженным профессором нескольких университетов, 
председателем Российской ассоциации <<Распознавание образов и обработка изображений>>, 
членом исполкома Международной ассоциации IAPR (распознавание образов и обработка 
изображений). Был награжден 8-ю орденами и медалями СССР и России.

Юрий Иванович проводил большую научно-литературную работу, являясь, в том числе, главным 
редактором международных научных журналов и членом редколлегий ряда рецензируемых 
научных журналов. 


Параллельно с активной научной деятельностью Юрий Иванович вел и преподавательскую 
работу. С~1961 по~1969~гг.~--- в Новосибирском государственном университете на кафедре 
алгебры и математической логики, которую возглавлял в то время академик А.\,И.~Мальцев. 
С~1970~г., будучи уже профессором (1967~г.),~--- в Московском физико-техническом институте 
на кафедре академика Н.\,Н.~Моисеева. В~1997~г.\ по предложению ректора МГУ им.\ 
М.\,В.~Ломоносова академика В.\,А.~Садовничего Юрий Иванович организовал на факультете 
Вычислительной математики и кибернетики новую кафедру <<Математические методы 
прогнозирования>>, которой и руководил до конца жизни. В~2008~г.\ ему была присуждена 
премия Совета Министров РФ в области образования. С~1965~г.\ Юрий Иванович периодически 
читал курсы лекций за рубежом, в университетах США, Франции, Финляндии, Швеции, Австрии, 
Польши, Болгарии, ГДР и других стран. Эта работа в существенной степени обеспечила широкое 
международное признание советской и российской науки в области дискретной математики и~распознавания образов. 

%\begin{floatingfigure}{60mm}
\begin{figure}[b]
\begin{center}
\vspace*{-6pt}
\mbox{%

\epsfxsize=112mm %90mm %112.705
\epsfbox{zhur-1.eps}
}
\end{center}
\end{figure}
%\end{floatingfigure}

Понимая важность вопроса воспитания подрастающего поколения для развития науки в стране, 
Юрий Иванович вскоре после защиты первой диссертации включился в работу по подготовке 
научных кадров. Им создана большая научная школа: под руководством Юрия Ивановича 
защищены более 100~кандидатских диссертаций по всевозможным разделам естествознания 
(математике, информатике, медицине, технике, экономике, геологии), не один десяток докторов 
наук. Он воспитал академиков и членов-корреспондентов РАН и академий государств СНГ. 
С~большим вниманием и участием Юрий Иванович относился к развитию научных школ страны 
в~об\-ласти обработки изображений, распознавания образов и компьютерной оптики. 

Для всех коллег и учеников Юрия Ивановича он останется примером замечательного человека, 
та\-лант\-ли\-во\-го педагога и выдающегося, преданного служению науке ученого. 


%\def\stat{cont}
{%\hrule\par
%\vskip 7pt % 7pt
\raggedleft\Large \bf%\baselineskip=3.2ex
А\,В\,Т\,О\,Р\,С\,К\,И\,Й\ \ У\,К\,А\,З\,А\,Т\,Е\,Л\,Ь\ \ З\,А\ \ 2\,0\,1\,0 г. \vskip 17pt
    \hrule
    \par
\vskip 21pt plus 6pt minus 3pt }

\label{st\stat}

\def\tit{\ }

\def\aut{\ }
\def\auf{\ }

\def\leftkol{\ } % ENGLISH ABSTRACTS}

\def\rightkol{\ } %АВТОРСКИЙ УКАЗАТЕЛЬ ЗА 2010 г.} %ENGLISH ABSTRACTS}

\titele{\tit}{\aut}{\auf}{\leftkol}{\rightkol}

\vspace*{-12pt}

{\tabcolsep=3pt
\begin{tabular}{p{388pt}rr}
&\textbf{Выпуск} & \textbf{Стр.}\\[6pt]
\hangindent=23pt\noindent\textbf{Арутюнян~А.\,Р.} Моделирование влияния деформаций отпечатков пальцев на 
точность\linebreak
\vspace*{-12pt}\\
\hspace*{23pt}дактилоскопической идентификации$\dotfill$&1&51\\
\hangindent=23pt\noindent\textbf{Архипов~О.\,П., Зыкова~З.\,П.} Интеграция гетерогенной информации о цветных 
пикселях\linebreak
\vspace*{-12pt}\\
\hspace*{23pt}и их цветовосприятии$\dotfill$&4&15\\
\hangindent=23pt\noindent\textbf{Баранов~С.\,И., Френкель~С.\,Л., Захаров~В.\,Н.} Полуформальная верификация 
цифрового устройства с конвейером, основанная на использовании алгоритмических машин\linebreak
\vspace*{-12pt}\\
\hspace*{23pt}состояния$\dotfill$&4&49\\
\textbf{Бекетова~И.\,В.} см.~Каратеев~С.\,Л.&&\\
\textbf{Белоусов~В.\,В.} см.~Синицын~И.\,Н.&&\\
\hangindent=23pt\noindent\textbf{Бенинг~В.\,Е., Королев~Р.\,А.} О предельном поведении мощностей критериев в 
случае\linebreak
\vspace*{-12pt}\\
\hspace*{23pt}распределения Лапласа$\dotfill$&2&63\\
\hangindent=23pt\noindent\textbf{Бенинг~В.\,Е., Сипина~А.\,В.} Асимптотическое разложение для мощности 
критерия,\linebreak
\vspace*{-12pt}\\
\hspace*{23pt}основанного на выборочной медиане, в случае распределения Лапласа$\dotfill$&1&18\\
\textbf{Бондаренко~А.\,В.} см.~Каратеев~С.\,Л.&&\\
\hangindent=23pt\noindent\textbf{Бородина~А.\,В., Морозов~Е.\,В.} Об оценивании асимптотики вероятности 
большого\linebreak
\vspace*{-12pt}\\
\hspace*{23pt}уклонения стационарной регенеративной очереди с одним прибором$\dotfill$&3&29\\
\hangindent=23pt\noindent\textbf{Бунтман~Н.\,В., Минель~Ж.-Л., Ле~Пезан~Д., Зацман~И.\,М.} Типология и 
компьютерное\linebreak
\vspace*{-12pt}\\
\hspace*{23pt}моделирование трудностей перевода$\dotfill$&3&77\\
\textbf{Визильтер~Ю.\,В.} см.~Каратеев~С.\,Л.&&\\
\hangindent=23pt\noindent\textbf{Гавриленко~С.\,В.} Оценки скорости сходимости распределений случайных сумм с 
безгранично делимыми индексами к нормальному закону$\dotfill$&4&81\\
\hangindent=23pt\noindent\textbf{Григорьева~М.\,Е., Шевцова~И.\,Г.} Уточнение неравенства 
Каца--Берри--Эссеена$\dotfill$&2&75\\
\hangindent=23pt\noindent\textbf{Грушо~А.\,А., Грушо~Н.\,А., Тимонина~Е.\,Е.} Поиск конфликтов в политиках 
безопасности: модель случайных графов$\dotfill$&3&38\\
\textbf{Грушо~Н.\,А.} см.~Грушо~А.\,А.&&\\
\hangindent=23pt\noindent\textbf{Гудков~В.\,Ю.} Математические модели изображения отпечатка пальца на основе 
описания линий$\dotfill$&1&58\\
\textbf{Гуртов~А.\,В.} см.~Лукьяненко~А.\,С.&&\\
\textbf{Желтов~С.\,Ю.} см.~Каратеев~С.\,Л.&&\\
\hangindent=23pt\noindent\textbf{Захаров~А.\,А., Серебряков~В.\,А.} Система управления электронной библиотекой 
LibMeta$\dotfill$&4&2\\
\textbf{Захаров~В.\,Н.} см.~Баранов~С.\,И.&&\\
\textbf{Захарова~Т.\,В.} см.~Матвеева~С.\,С.&&\\
\hangindent=23pt\noindent\textbf{Зацаринный~А.\,А., Чупраков~К.\,Г.} Некоторые аспекты выбора технологии для 
постро-\linebreak
\vspace*{-12pt}\\
\hspace*{23pt}ения систем отображения информации ситуационного центра$\dotfill$&3&59\\
\textbf{Зацман~И.\,М.} см.~Бунтман~Н.\,В.&&\\
\hangindent=23pt\noindent\textbf{Зейфман~А.\,И., Коротышева~А.\,В., Сатин~Я.\,А., Шоргин~С.\,Я.} Об 
устойчивости нестаци-\linebreak
\vspace*{-12pt}\\
\hspace*{23pt}онарных систем обслуживания с катастрофами$\dotfill$&3&9\\
\textbf{Зыкова~З.\,П.} см.~Архипов~О.\,П.&&\\
\hangindent=23pt\noindent\textbf{Илюшин~Г.\,Я., Соколов~И.\,А.} Организация управляемого доступа пользователей 
к\linebreak
\vspace*{-12pt}\\
\hspace*{23pt}разнородным ведомственным информационным ресурсам$\dotfill$&1&24\\
\hangindent=23pt\noindent\textbf{Кавагучи~Ю., Ульянов~В.\,В., Фуджикоши~Я.} Приближения для статистик, 
описывающих\linebreak
\vspace*{-12pt}\\
\hspace*{23pt}геометрические свойства данных большой размерности, с оценками 
ошибок$\dotfill$&1&12\\
\hangindent=23pt\noindent\textbf{Каратеев~С.\,Л., Бекетова~И.\,В., Ососков~М.\,В., Князь~В.\,А., 
Визильтер~Ю.\,В., Бондаренко~А.\,В., Желтов~С.\,Ю.} Автоматизированный контроль 
качества цифровых\linebreak
\vspace*{-12pt}\\
\hspace*{23pt}изображений для персональных документов$\dotfill$&1&65\\
\end{tabular}
}

\pagebreak

\def\leftkol{АВТОРСКИЙ УКАЗАТЕЛЬ ЗА 2010 г.} % ENGLISH ABSTRACTS}

\def\rightkol{АВТОРСКИЙ УКАЗАТЕЛЬ ЗА 2010 г.} %ENGLISH ABSTRACTS}

{\tabcolsep=3pt
\begin{tabular}{p{388pt}rr}
&\textbf{Выпуск} & \textbf{Стр.}\\[3pt]
\hangindent=23pt\noindent\textbf{Козеренко~Е.\,Б.} Лингвистические фильтры в статистических моделях машинного\linebreak
\vspace*{-12pt}\\
\hspace*{23pt}перевода$\dotfill$&2&83\\
\hangindent=23pt\noindent\textbf{Козеренко~Е.\,Б., Кузнецов~И.\,П.} Когнитивно-лингвистические представления в 
систе-\linebreak
\vspace*{-12pt}\\
\hspace*{23pt}мах обработки текстов$\dotfill$&3&69\\
\textbf{Князь~В.\,А.} см.~Каратеев~С.\,Л.&&\\
\hangindent=23pt\noindent\textbf{Колесников~А.\,В., Солдатов~С.\,А.} Алгоритм координации для гибридной 
интеллектуальной системы решения сложной задачи оперативно-производственного\linebreak
\vspace*{-12pt}\\
\hspace*{23pt}планирования$\dotfill$&4&61\\
\hangindent=23pt\noindent\textbf{Коновалов~М.\,Г.} О планировании потоков в системах вычислительных 
ресурсов$\dotfill$&2&3\\
\textbf{Конушин~А.\,С.} см.~Конушин~В.\,С.&&\\
\hangindent=23pt\noindent\textbf{Конушин~В.\,С., Кривовязь~Г.\,Р., Конушин~А.\,С.} Алгоритм распознавания людей 
в видео-\linebreak
\vspace*{-12pt}\\
\hspace*{23pt}последовательности по одежде$\dotfill$&1&74\\
\textbf{Корепанов~Э.\, Р.} см.~Синицын~И.\,Н.&&\\
\textbf{Королев~В.\,Ю.} см.~Соколов~И.\,А.&&\\
\textbf{Королев~Р.\,А.} см.~Бенинг~В.\,Е.&&\\
\textbf{Коротышева~А.\,В.} см.~Зейфман~А.\,И.&&\\
\hangindent=23pt\noindent\textbf{Кривенко~М.\,П.} Непараметрическое оценивание элементов байесовского 
клас\-си-\linebreak
\vspace*{-12pt}\\
\hspace*{23pt}фикатора$\dotfill$&2&13\\
\textbf{Кривовязь~Г.\,Р.} см.~Конушин~В.\,С.&&\\
\textbf{Крылов~А.\,С.} см.~Павельева~Е.\,А.&&\\
\hangindent=23pt\noindent\textbf{Крылов~В.\,А.} Моделирование и классификация многоканальных дистанционных\linebreak
\vspace*{-12pt}\\
\hspace*{23pt}изображений с использованием копул$\dotfill$&4&34\\
\hangindent=23pt\noindent\textbf{Крючин~О.\,В.} Разработка параллельных эвристических алгоритмов подбора 
весовых\linebreak
\vspace*{-12pt}\\
\hspace*{23pt}коэффициентов искусственной нейтронной сети$\dotfill$&2&53\\
\hangindent=23pt\noindent\textbf{Кудрявцев~А.\,А., Шоргин~С.\,Я.} Байесовские модели массового обслуживания и 
надеж-\linebreak
\vspace*{-12pt}\\
\hspace*{23pt}ности: характеристики среднего числа заявок в системе $M\vert M \vert 1\vert 
\infty$$\dotfill$&3&16\\
\hangindent=23pt\noindent\textbf{Кузнецов~А.\,А.} Связь между временными и структурно-топологическими 
характери-\linebreak
\vspace*{-12pt}\\
\hspace*{23pt}стиками диаграмм ритма сердца здоровых людей$\dotfill$&4&39\\
\textbf{Кузнецов~И.\,П.} см.~Козеренко~Е.\,Б.&&\\
\textbf{Ле~Пезан~Д.} см.~Бунтман~Н.\,В.&&\\
\hangindent=23pt\noindent\textbf{Лукьяненко~А.\,С., Морозов~Е.\,В., Гуртов~А.\,В.} Анализ сетевого протокола с общей 
функ-\linebreak
\vspace*{-12pt}\\
\hspace*{23pt}цией расширения окна передачи сообщения при конфликтах$\dotfill$&2&46\\
\hangindent=23pt\noindent\textbf{Лямин~О.\,О.} О предельном поведении мощностей критериев в случае обобщенного\linebreak
\vspace*{-12pt}\\
\hspace*{23pt}распределения Лапласа$\dotfill$&3&47\\
\hangindent=23pt\noindent\textbf{Маркин~А.\,В., Шестаков~О.\,В.} Асимптотики оценки риска при пороговой 
обработке\linebreak
\vspace*{-12pt}\\
\hspace*{23pt}вейвлет-вейглет коэффициентов в задаче томографии$\dotfill$&2&36\\
\hangindent=23pt\noindent\textbf{Матвеева~С.\,С., Захарова~Т.\,В.} Сети массового обслуживания с наименьшей 
длиной\linebreak
\vspace*{-12pt}\\
\hspace*{23pt}очереди$\dotfill$&3&22\\
\hangindent=23pt\noindent\textbf{Матюшенко~С.\,И.} Стационарные характеристики двухканальной системы 
обслужива-\linebreak
\vspace*{-12pt}\\
\hspace*{23pt}ния с переупорядочиванием заявок и распределениями фазового типа$\dotfill$&4&68\\
\textbf{Минель~Ж.-Л.} см.~Бунтман~Н.\,В.&&\\
\textbf{Морозов~Е.\,В.} см.~Бородина~А.\,В.&&\\
\textbf{Морозов~Е.\,В.} см.~Лукьяненко~А.\,С.&&\\
\textbf{Ососков~М.\,В.} см.~Каратеев~С.\,Л.&&\\
\hangindent=23pt\noindent\textbf{Павельева~Е.\,А., Крылов~А.\,С.} Поиск и анализ ключевых точек радужной 
оболочки\linebreak
\vspace*{-12pt}\\
\hspace*{23pt}глаза методом преобразования Эрмита$\dotfill$&1&79\\
\textbf{Печинкин~А.\,В.} см.~Френкель~С.\,Л.,&&\\
\hangindent=23pt\noindent\textbf{Протасов~В.\,И.} Составление субъективного портрета с использованием 
эволюционно-\linebreak
\vspace*{-12pt}\\
\hspace*{23pt}го морфинга и квалиметрия метода$\dotfill$&1&83\\
\hangindent=23pt\noindent\textbf{Рудаков~К.\,В., Торшин~И.\,Ю.} Вопросы разрешимости задачи распознавания 
вторичной\linebreak
\vspace*{-12pt}\\
\hspace*{23pt}структуры белка$\dotfill$&2&25\\
\textbf{Сатин~Я.\,А.} см.~Зейфман~А.\,И.&&\\
\hangindent=23pt\noindent\textbf{Сейфуль-Мулюков~Р.\,Б.} Нефть как носитель информации о своем 
происхождении,\linebreak
\vspace*{-12pt}\\
\hspace*{23pt}структуре и эволюции$\dotfill$&1&41\\
\end{tabular}
}

{\tabcolsep=3pt
\begin{tabular}{p{388pt}rr}
&\textbf{Выпуск} & \textbf{Стр.}\\[6pt]
\textbf{Семендяев~Н.\,Н.} см.~Синицын~И.\,Н.&&\\
\textbf{Серебряков~В.\,А.} см.~Захаров~А.\,А.&&\\
\textbf{Синицын~В.\,И.} см.~Синицын~И.\,Н.&&\\
\hangindent=23pt\noindent\textbf{Синицын~И.\,Н., Синицын~В.\,И., Корепанов~Э.\, Р., Белоусов~В.\,В., 
Семендяев~Н.\,Н.} Оперативное построение информационных моделей движения полюса 
Земли\linebreak
\vspace*{-12pt}\\
\hspace*{23pt}методами линейных и линеаризованных фильтров$\dotfill$&1&2\\
\textbf{Сипина~А.\,В.} см.~Бенинг~В.\,Е.&&\\
\hangindent=23pt\noindent\textbf{Соколов~И.\,А.} О работах заслуженного деятеля науки Российской Федерации 
И.\,Н.~Синицына в области информационных технологий и автоматизации (к 70-летию\linebreak
\vspace*{-12pt}\\
\hspace*{23pt}со дня рождения)$\dotfill$&3&84\\
\textbf{Соколов~И.\,А.} см.~Илюшин~Г.\,Я.&&\\
\hangindent=23pt\noindent\textbf{Соколов~И.\,А., Королев~В.\,Ю.} Предисловие$\dotfill$&2&2\\
\textbf{Солдатов~С.\,А.} см.~Колесников~А.\,В.&&\\
\hangindent=23pt\noindent\textbf{Степанов~С.\,Ю.} Использование координатного метода фрагментации 
коммутаторной\linebreak
\vspace*{-12pt}\\
\hspace*{23pt}нейронной сети для сокращения трафика$\dotfill$&2&57\\
\textbf{Тимонина~Е.\,Е.} см.~Грушо~А.\,А.&&\\
\textbf{Торшин~И.\,Ю.} см.~Рудаков~К.\,В.&&\\
\textbf{Ульянов~В.\,В.} см.~Кавагучи~Ю.&&\\
\textbf{Фазекаш~И.} см.~Чупрунов~А.\,Н.&&\\
\textbf{Френкель~С.\,Л.} см.~Баранов~С.\,И.&&\\
\hangindent=23pt\noindent\textbf{Френкель~С.\,Л., Печинкин~А.\,В.} Оценка времени самовосстановления в 
цифровых\linebreak
\vspace*{-12pt}\\
\hspace*{23pt}системах после сбоев, вызываемых переходными помехами$\dotfill$&3&2\\
\textbf{Фуджикоши~Я.} см.~Кавагучи~Ю.&&\\
\hangindent=23pt\noindent\textbf{Цискаридзе~А.\,К.} Математическая модель и метод восстановления позы человека 
по\linebreak
\vspace*{-12pt}\\
\hspace*{23pt}стереопаре силуэтных изображений$\dotfill$&4&27\\
\hangindent=23pt\noindent\textbf{Чупраков~К.\,Г.} К вопросу о размещении коллективных средств отображения в 
ситуа-\linebreak
\vspace*{-12pt}\\
\hspace*{23pt}ционном зале с заданными параметрами$\dotfill$&4&89\\
\textbf{Чупраков~К.\,Г.} см.~Зацаринный~А.\,А.&&\\
\hangindent=23pt\noindent\textbf{Чупрунов~А.\,Н., Фазекаш~И.} Законы повторного логарифма для числа 
безошибочных\linebreak
\vspace*{-12pt}\\
\hspace*{23pt}блоков при помехоустойчивом кодировании$\dotfill$&3&42\\
\textbf{Шевцова~И.\,Г.} см.~Григорьева~М.\,Е.&&\\
\hangindent=23pt\noindent\textbf{Шестаков~О.\,В.} Аппроксимация распределения оценки риска пороговой 
обработки вейвлет-коэффициентов нормальным распределением при использовании 
выбо-\linebreak
\vspace*{-12pt}\\
\hspace*{23pt}рочной дисперсии$\dotfill$&4&73\\
\textbf{Шестаков~О.\,В.} см.~Маркин~А.\,В.&&\\
\textbf{Шоргин~С.\,Я.} см.~Зейфман~А.\,И.&&\\
\textbf{Шоргин~С.\,Я.} см.~Кудрявцев~А.\,А.&&\\
\end{tabular}
}

%\thispagestyle{myheadings}
\def\leftfootline{\small{\textbf{\thepage}
\hfill ИНФОРМАТИКА И ЕЁ ПРИМЕНЕНИЯ\ \ \ том~4\ \ \ выпуск~4\ \ \ 2010}
}%
 \def\rightfootline{\small{ИНФОРМАТИКА И ЕЁ ПРИМЕНЕНИЯ\ \ \ том~4\ \ \ выпуск~4\ \ \ 2010
 \hfill \textbf{\thepage}}}
 \label{end\stat}
%
%Том 10 Выпуск 1-4 Год 2016

\def\stat{cont-e}
{%\hrule\par
%\vskip 7pt % 7pt
\raggedleft\Large \bf%\baselineskip=3.2ex
2\,0\,1\,6\ \ A\,U\,T\,H\,O\,R\ \ I\,N\,D\,E\,X \vskip 17pt
 \hrule
 \par
\vskip 21pt plus 6pt minus 3pt }

\label{st\stat}

\def\tit{\ }

\def\aut{\ }
\def\auf{\ }

\def\leftkol{\ } %2016 AUTHOR INDEX} % ENGLISH ABSTRACTS}

\def\rightkol{\ } %2016 AUTHOR INDEX} %ENGLISH ABSTRACTS}

\titele{\tit}{\aut}{\auf}{\leftkol}{\rightkol}

\def\leftfootline{\small{\textbf{\thepage}
\hfill INFORMATIKA I EE PRIMENENIYA~--- INFORMATICS AND APPLICATIONS\ \ \ 2016\
\ \ volume~10\ \ \ issue\ 4}
}%
 \def\rightfootline{\small{INFORMATIKA I EE PRIMENENIYA~--- INFORMATICS AND APPLICATIONS\ \ \ 2016\ \ \ volume~10\ \ \ issue\ 4
\hfill \textbf{\thepage}}}

\vspace*{-12pt}
\vspace*{-18pt}

{\tabcolsep=2.8pt
\begin{tabular}{p{382pt}cc}
&\textbf{Issue} & \textbf{Page}\\[6pt]
\Avtors{Agalarov~M.\,Ya.} see~Agalarov~Ya.\,M.&&\\
\Avtors{Agalarov~Ya.\,M., Agalarov~M.\,Ya., and
Shorgin~V.\,S.} About the optimal threshold of queue\linebreak
\\[-12pt]
\hspace*{23pt}length in a~particular problem of profit maximization
in the $M/G/1$ queuing system&2&70--79\\
\Avtors{Alexeyevsky~D.\,A.} BioNLP ontology extraction from 
a~restricted language corpus with\linebreak
\\[-12pt]
\hspace*{23pt}context-free grammars&1&119--128\\
\Avtors{Andreev~S.\,D.} see~Gaidamaka~Yu.\,V.&&\\
\Avtors{Andreev~S.\,D.} see~Ometov~A.\,Ya.&&\\
\Avtors{Arkhipov~O.\,P., Arkhipov~P.\,O., and Sidorkin~I.\,I.} The
option to create a~local coordinate\linebreak
\\[-12pt]
\hspace*{23pt}system for synchronization of selected images&3&91--97\\
\Avtors{Arkhipov~P.\,O.} see~Arkhipov~O.\,P.&&\\
\Avtors{Belousov~V.\,V.} see~Shnurkov~P.\,V.&&\\
\Avtors{Belousov~V.\,V.} see~Shnurkov~P.\,V.&&\\
\Avtors{Bening~V.\,E.} Calculation of~the~asymptotic deficiency
of~some statistical procedures based\linebreak
\\[-12pt]
\hspace*{23pt}on~samples with~random sizes&4&34--45\\
\Avtors{Borisov~A.\,V., Bosov~A.\,V., and Miller~G.\,B.} Modeling and
monitoring of VoIP connection&2&\hphantom{1}2--13\\
\Avtors{Bosov~A.\,V.} see~Borisov~A.\,V.&&\\
\Avtors{Briukhov~D.\,O.} see~Stupnikov~S.\,A.&&\\
\Avtors{Callaos~N.\,K.\ and Seyful-Mulyukov~R.\,B.} Complexity and
its information content&1&129--139\\
\Avtors{Chertok~A.\,V., Kadaner~A.\,I., Khazeeva~G.\,T., and
Sokolov~I.\,A.} Regime switching detection\linebreak
\\[-12pt]
\hspace*{23pt}for~the~Levy driven
Ornstein--Uhlenbeck process using CUSUM methods&4&46--56\\
\Avtors{Chichagov~V.\,V.} Asymptotic expansions of mean absolute
error of uniformly minimum variance unbiased and maximum likelihood
estimators on the one-parameter exponential\linebreak
\\[-12pt]
\hspace*{23pt}family model of lattice distributions&3&66--76\\
\Avtors{Danishevsky~V.\,I.} see~Kolesnikov A.\,V.&&\\
\Avtors{Fazliev~A.\,Z.} see~Kalinichenko~L.\,A.&&\\
\Avtors{Fedoseev~A.\,A.} What is behind the concept of ``knowledge in
small packages''&3&105--110\\
\Avtors{Gaidamaka~Yu.\,V., Andreev~S.\,D., Sopin~E.\,S.,
Samouylov~K.\,E., and Shorgin~S.\,Ya.} Interference analysis
of~the~device-to-device communications model with~regard to~a~signal\linebreak
\\[-12pt]
\hspace*{23pt}propagation environment&4&\hphantom{1}2--10\\
\Avtors{Gasilov~A.\,V.} see~Yakovlev~O.\,A.&&\\
\Avtors{Goncharov~A.\,V.\ and Strijov~V.\,V.} Metric time series
classification using weighted dynamic\linebreak
\\[-12pt]
\hspace*{23pt}warping relative to centroids of classes&2&36--47\\
\Avtors{Gordov~E.\,P.} see~Kalinichenko~L.\,A.&&\\
\Avtors{Gorshenin~A.\,K.} Concept of online service for stochastic
modeling of real processes&1&72--81\\
\Avtors{Gorshenin~A.\,K.} see~Shnurkov~P.\,V.&&\\
\Avtors{Gorshenin~A.\,K.} see~Shnurkov~P.\,V.&&\\
\Avtors{Grusho~A.\,A., Grusho~N.\,A., Zabezhailo~M.\,I., and
Timonina~E.\,E.} Integration of statistical and\linebreak
\\[-12pt]
\hspace*{23pt}deterministic methods for
analysis of information security&3&2--8\\
\Avtors{Grusho~A.\,A., Zabezhailo~M.\,I., and Zatsarinny~A.\,A.} On
the advanced procedure to reduce\linebreak
\\[-12pt]
\hspace*{23pt}calculation of Galois closures&4&\hphantom{1}96--104\\
\Avtors{Grusho~N.\,A.} see~Grusho~A.\,A.&&\\
\Avtors{Havanskov~V.\,A.} see~Minin~V.\,A.&&\\
\Avtors{Inkova~O.\,Yu.} see~Zatsman~I.\,M.&&\\
\Avtors{Isachenko~R.\,V.\ and Strijov~V.\,V.} Metric learning in
multiclass time series classification\linebreak
\\[-12pt]
\hspace*{23pt}problem&2&48--57\\
\end{tabular}
}
\pagebreak

\def\leftfootline{\small{\textbf{\thepage}
\hfill INFORMATIKA I EE PRIMENENIYA~--- INFORMATICS AND APPLICATIONS\ \ \ 2016\
\ \ volume~10\ \ \ issue\ 4}
}%
 \def\rightfootline{\small{INFORMATIKA I EE PRIMENENIYA~---
INFORMATICS AND APPLICATIONS\ \ \ 2016\ \ \ volume~10\ \ \ issue\ 4
\hfill \textbf{\thepage}}}

\def\leftkol{2016 AUTHOR INDEX} % ENGLISH ABSTRACTS}

\def\rightkol{2016 AUTHOR INDEX} %ENGLISH ABSTRACTS}


{\tabcolsep=2.83pt
\begin{tabular}{p{382pt}cc}
&\textbf{Issue} & \textbf{Page}\\[6pt]
\Avtors{Kadaner~A.\,I.} see~Chertok~A.\,V.&&\\[.255pt]
\Avtors{Kalinichenko~L.\,A., Volnova~A.\,A., Gordov~E.\,P.,
Kiselyova~N.\,N., Kovaleva~D.\,A., Malkov~O.\,Yu., Okladnikov~I.\,G.,
Podkolodnyy~N.\,L., Pozanenko~A.\,S., Ponomareva~N.\,V.,
Stupnikov~S.\,A.,} \textbf{and Fazliev~A.\,Z.} Data access challenges for data
intensive\linebreak
\\[-12pt]
\hspace*{23pt}research in Russia&1& 2--22\\[.255pt]
\Avtors{Karasikov~M.\,E.\ and Strijov~V.\,V.} Feature-based
time-series classification&4&121--131\\[.255pt]
\Avtors{Khazeeva~G.\,T.} see~Chertok~A.\,V.&&\\[.255pt]
\Avtors{Khokhlov~Yu.\,S.} Multivariate fractional Levy motion and its
applications&2&\hphantom{1}98--106\\[.255pt]
\Avtors{Kirikov~I.\,A., Kolesnikov~A.\,V., Listopad~S.\,V., and
Rumovskaya~S.\,B.} Fine-grained hybrid\linebreak
\\[-12pt]
\hspace*{23pt}intelligent systems. Part 2:
Bidirectional hybridization&1&\hphantom{1}96--105\\[.255pt]
\Avtors{Kirikov~I.\,A., Kolesnikov~A.\,V., Listopad~S.\,V., and
Rumovskaya~S.\,B.} ``Virtual council''~---\linebreak
\\[-12pt]
\hspace*{23pt}source environment
supporting complex diagnostic decision making&3&81--90\\[.255pt]
\Avtors{Kiselyova~N.\,N.} see~Kalinichenko~L.\,A.&&\\[.255pt]
\Avtors{Kolesnikov A.\,V., Listopad~S.\,V., Rumovskaya~S.\,B., and
Danishevsky~V.\,I.} Informal axiomatic\linebreak
\\[-12pt]
\hspace*{23pt}theory of~the~role visual models&4&114--120\\[.255pt]
\Avtors{Kolesnikov~A.\,V.} see~Kirikov~I.\,A.&&\\[.255pt]
\Avtors{Kolesnikov~A.\,V.} see~Kirikov~I.\,A.&&\\[.255pt]
\Avtors{Kolin~K.\,K.} Humanitarian aspects of information
security&3&111--121\\[.255pt]
\Avtors{Konovalov~M.\,G.\ and Razumchik~R.\,V.} Dispatching
to~two parallel nonobservable queues using\linebreak
\\[-12pt]
\hspace*{23pt}only static
information&4&57--67\\[.255pt]
\Avtors{Korchagin~A.\,Yu.} see~Korolev~V.\,Yu.&&\\[.255pt]
\Avtors{Korchagin~A.\,Yu.} see~Korolev~V.\,Yu.&&\\[.255pt]
\Avtors{Korepanov~E.\,R.} see~Sinitsyn~I.\,N.&&\\[.255pt]
\Avtors{Korepanov~E.\,R.} see~Sinitsyn~I.\,N.&&\\[.255pt]
\Avtors{Korolev~V.\,Yu., Korchagin~A.\,Yu., and Zeifman~A.\,I.} The
Poisson theorem for Bernoulli trials\linebreak
\\[-12pt]
\hspace*{23pt}with~a~random probability
of~success and~a~discrete analog of~the~Weibull distribution&4&11--20\\[.255pt]
\Avtors{Korolev~V.\,Yu., Zeifman~A.\,I., and Korchagin~A.\,Yu.}
Asymmetric Linnik distributions as~limit\linebreak
\\[-12pt]
\hspace*{23pt}laws for~random sums
of~independent random variables with~finite variances&4&21--33\\[.255pt]
\Avtors{Koucheryavy~E.\,A.} see~Ometov~A.\,Ya.&&\\[.255pt]
\Avtors{Kovaleva~D.\,A.} see~Kalinichenko~L.\,A.&&\\[.255pt]
\Avtors{Kovalyov~S.\,P.} Metaprogramming to increase
manufacturability of large-scale software-\linebreak
\\[-12pt]
\hspace*{23pt}intensive systems&1&56--66\\[.255pt]
\Avtors{Krivenko~M.\,P.} Significance tests of feature selection for
classification&3&32--40\\[.255pt]
\Avtors{Kruzhkov~M.\,G.} see~Zalizniak~Anna~A.&&\\[.255pt]
\Avtors{Kruzhkov~M.\,G.} see~Zatsman~I.\,M.&&\\[.255pt]
\Avtors{Kudryavtsev~A.\,A.} Bayesian queueing and reliability models:
\textit{A~priori} distributions with\linebreak
\\[-12pt]
\hspace*{23pt}compact support&1&67--71\\[.255pt]
\Avtors{Kudryavtsev~A.\,A.} Characteristics dependent on the balance
coefficient in Bayesian models\linebreak
\\[-12pt]
\hspace*{23pt}with compact support of \textit{a priori}
distributions&3&77--80\\[.255pt]
\Avtors{Kudryavtsev~A.\,A.\ and Palionnaia~S.\,I.} Bayesian recurrent
model of reliability growth:\linebreak
\\[-12pt]
\hspace*{23pt}Parabolic distribution of parameters&2&80--83\\[.255pt]
\Avtors{Kudryavtsev~A.\,A.\ and Titova~A.\,I.} Bayesian queuing
and~reliability models: Degenerate-\linebreak
\\[-12pt]
\hspace*{23pt}Weibull case&4&68--71\\[.255pt]
\Avtors{Leontyev~N.\,D.\ and Ushakov~V.\,G.} Analysis of a queueing
system with autoregressive arrivals\linebreak
\\[-12pt]
\hspace*{23pt}and nonpreemptive priority&3&15--22\\[.255pt]
\Avtors{Listopad~S.\,V.} see~Kirikov~I.\,A.&&\\[.255pt]
\Avtors{Listopad~S.\,V.} see~Kirikov~I.\,A.&&\\[.255pt]
\Avtors{Listopad~S.\,V.} see~Kolesnikov A.\,V.&&\\[.255pt]
\Avtors{Malkov~O.\,Yu.} see~Kalinichenko~L.\,A.&&\\[.255pt]
\Avtors{Markov~A.\,S., Monakhov~M.\,M., and
Ulyanov~V.\,V.} Generalized Cornish--Fisher expansions\linebreak
\\[-12pt]
\hspace*{23pt}for distributions of statistics based on samples
of random size&2&84--91\\[.255pt]
\Avtors{Melnikov~A.\,K.\ and Ronzhin~A.\,F.} Generalized statistical
method of~text analysis based\linebreak
\\[-12pt]
\hspace*{23pt}on~calculation of~probability distributions
of~statistical values&4&89--95\\
\end{tabular}
}
\pagebreak

\def\leftfootline{\small{\textbf{\thepage}
\hfill INFORMATIKA I EE PRIMENENIYA~--- INFORMATICS AND APPLICATIONS\ \ \ 2016\
\ \ volume~10\ \ \ issue\ 4}
}%
 \def\rightfootline{\small{INFORMATIKA I EE PRIMENENIYA~---
INFORMATICS AND APPLICATIONS\ \ \ 2016\ \ \ volume~10\ \ \ issue\ 4
\hfill \textbf{\thepage}}}

\def\leftkol{2016 AUTHOR INDEX} % ENGLISH ABSTRACTS}

\def\rightkol{2016 AUTHOR INDEX} %ENGLISH ABSTRACTS}


{\tabcolsep=3pt
\begin{tabular}{p{381pt}cc}
&\textbf{Issue} & \textbf{Page}\\[6pt]
\Avtors{Meykhanadzhyan~L.\,A.} Stationary characteristics of the finite
capacity queueing system with\linebreak
\\[-12pt]
\hspace*{23pt}inverse service order and generalized
probabilistic priority&2&123--131\\[.23pt]
\Avtors{Miller~G.\,B.} see~Borisov~A.\,V.&&\\[.23pt]
\Avtors{Minin~V.\,A., Zatsman~I.\,M., Havanskov~V.\,A., and
Shubnikov~S.\,K.} Intensity of citation of scientific publications in
inventions on information and computer technologies patented\linebreak
\\[-12pt]
\hspace*{23pt}in Russia by domestic and foreign applicants&2&107--122\\[.23pt]
\Avtors{Monakhov~M.\,M.} see~Markov~A.\,S.&&\\[.23pt]
\Avtors{Naumov~V.\,A.\ and Samouylov~K.\,E.} On relationship
between queuing systems with resources\linebreak
\\[-12pt]
\hspace*{23pt}and Erlang networks&3&\hphantom{1}9--14\\[.23pt]
\Avtors{Okladnikov~I.\,G.} see~Kalinichenko~L.\,A.&&\\[.23pt]
\Avtors{Ometov~A.\,Ya., Andreev~S.\,D., Turlikov~A.\,M., and
Koucheryavy~E.\,A.} Performance analysis of\linebreak
\\[-12pt]
\hspace*{23pt}a wireless data
aggregation system with contention for contemporary sensor
networks&3&23--31\\[.23pt]
\Avtors{Palionnaia~S.\,I.} see~Kudryavtsev~A.\,A.&&\\[.23pt]
\Avtors{Podkolodnyy~N.\,L.} see~Kalinichenko~L.\,A.&&\\[.23pt]
\Avtors{Ponomareva~N.\,V.} see~Kalinichenko~L.\,A.&&\\[.23pt]
\Avtors{Popkova~N.\,A.} see~Zatsman~I.\,M.&&\\[.23pt]
\Avtors{Pozanenko~A.\,S.} see~Kalinichenko~L.\,A.&&\\[.23pt]
\Avtors{Razumchik~R.\,V.} see~Konovalov~M.\,G.&&\\[.23pt]
\Avtors{Ronzhin~A.\,F.} see~Melnikov~A.\,K.&&\\[.23pt]
\Avtors{Rumovskaya~S.\,B.} see~Kirikov~I.\,A.&&\\[.23pt]
\Avtors{Rumovskaya~S.\,B.} see~Kirikov~I.\,A.&&\\[.23pt]
\Avtors{Rumovskaya~S.\,B.} see~Kolesnikov A.\,V.&&\\[.23pt]
\Avtors{Samouylov~K.\,E.} see~Gaidamaka~Yu.\,V.&&\\[.23pt]
\Avtors{Samouylov~K.\,E.} see~Naumov~V.\,A.&&\\[.23pt]
\Avtors{Serebryanskii~S.\,M.} see~Tyrsin~A.\,N.&&\\[.23pt]
\Avtors{Seyful-Mulyukov~R.\,B.} see~Callaos~N.\,K.&&\\[.23pt]
\Avtors{Shestakov~O.\,V.} Statistical properties of the denoising method
based on the stabilized hard\linebreak
\\[-12pt]
\hspace*{23pt}thresholding&2&65--69\\[.23pt]
\Avtors{Shestakov~O.\,V.} The strong law of large numbers for the risk
estimate in the problem of\linebreak
\\[-12pt]
\hspace*{23pt}tomographic image reconstruction from
projections with a correlated noise&3&41--45\\[.23pt]
\Avtors{Shestakov~O.\,V.} see~Zakharova~T.\,V.&&\\[.23pt]
\Avtors{Shnurkov~P.\,V., Gorshenin~A.\,K., and Belousov~V.\,V.}
Analytical solution of~the~optimal control\linebreak
\\[-12pt]
\hspace*{23pt}task of~a~semi-Markov
process with~finite set of~states&4&72--88\\[.23pt]
\Avtors{Shnurkov~P.\,V., Zasypko~V.\,V., Belousov~V.\,V., and
Gorshenin~A.\,K.} Development of the algorithm of numerical solution
of the optimal investment control problem\linebreak
\\[-12pt]
\hspace*{23pt}in the closed dynamical model of three-sector economy&1&82--95\\[.23pt]
\Avtors{Shorgin~S.\,Ya.} see~Gaidamaka~Yu.\,V.&&\\[.23pt]
\Avtors{Shorgin~V.\,S.} see~Agalarov~Ya.\,M.&&\\[.23pt]
\Avtors{Shubnikov~S.\,K.} see~Minin~V.\,A.&&\\[.23pt]
\Avtors{Sidorkin~I.\,I.} see~Arkhipov~O.\,P.&&\\[.23pt]
\Avtors{Sinitsyn~I.\,N.} Analytical modeling of processes in stochastic
systems with complex fractional\linebreak
\\[-12pt]
\hspace*{23pt}order Bessel nonlinearities&3&55--65\\[.23pt]
\Avtors{Sinitsyn~I.\,N.} Orthogonal supoptimal filters for nonlinear
stochastic systems on manifolds&1&34--44\\[.23pt]
\Avtors{Sinitsyn~I.\,N.\ and Korepanov~E.\,R.} Normal Pugachev
conditionally-optimal filters and extra-\linebreak
\\[-12pt]
\hspace*{23pt}polators for state linear stochastic systems&2&14--23\\[.23pt]
\Avtors{Sinitsyn~I.\,N.\ and Sinitsyn~V.\,I.} Analytical modeling of
distributions in stochastic systems on\linebreak
\\[-12pt]
\hspace*{23pt}manifolds based on ellipsoidal approximation&1&45--55\\[.23pt]
\Avtors{Sinitsyn~I.\,N., Sinitsyn~V.\,I., and
Korepanov~E.\,R.} Ellipsoidal suboptimal filters for nonlinear\linebreak
\\[-12pt]
\hspace*{23pt}stochastic systems on manifolds&2&24--35\\[.23pt]
\Avtors{Sinitsyn~V.\,I.} see~Sinitsyn~I.\,N.&&\\[.23pt]
\Avtors{Sinitsyn~V.\,I.} see~Sinitsyn~I.\,N.&&\\[.23pt]
\Avtors{Skvortsov~N.\,A.} see~Stupnikov~S.\,A.&&\\[.23pt]
\Avtors{Sokolov~I.\,A.} see~Chertok~A.\,V.&&\\
\end{tabular}
}
\pagebreak

\def\leftfootline{\small{\textbf{\thepage}
\hfill INFORMATIKA I EE PRIMENENIYA~--- INFORMATICS AND APPLICATIONS\ \ \ 2016\
\ \ volume~10\ \ \ issue\ 4}
}%
 \def\rightfootline{\small{INFORMATIKA I EE PRIMENENIYA~---
INFORMATICS AND APPLICATIONS\ \ \ 2016\ \ \ volume~10\ \ \ issue\ 4
\hfill \textbf{\thepage}}}

\def\leftkol{2016 AUTHOR INDEX} % ENGLISH ABSTRACTS}

\def\rightkol{2016 AUTHOR INDEX} %ENGLISH ABSTRACTS}


{\tabcolsep=3pt
\begin{tabular}{p{382pt}cc}
&\textbf{Issue} & \textbf{Page}\\[6pt]
\Avtors{Sopin~E.\,S.} see~Gaidamaka~Yu.\,V.&&\\
\Avtors{Strijov~V.\,V.} see~Goncharov~A.\,V.&&\\
\Avtors{Strijov~V.\,V.} see~Isachenko~R.\,V.&&\\
\Avtors{Strijov~V.\,V.} see~Karasikov~M.\,E.&&\\
\Avtors{Stupnikov~S.\,A., Briukhov~D.\,O., and Skvortsov~N.\,A.}
Co-lending systemic risk analysis over\linebreak
\\[-12pt]
\hspace*{23pt}heterogeneous data collections&1&23--33\\
\Avtors{Stupnikov~S.\,A.} see~Kalinichenko~L.\,A.&&\\
\Avtors{Suchkov~A.\,P.} see~Zatsarinny~A.\,A.&&\\
\Avtors{Timonina~E.\,E.} see~Grusho~A.\,A.&&\\
\Avtors{Titova~A.\,I.} see~Kudryavtsev~A.\,A.&&\\
\Avtors{Turlikov~A.\,M.} see~Ometov~A.\,Ya.&&\\
\Avtors{Tyrsin~A.\,N.\ and Serebryanskii~S.\,M.} Recognition of
dependences on the basis of inverse\linebreak
\\[-12pt]
\hspace*{23pt}mapping&2&58--64\\
\Avtors{Ulyanov~V.\,V.} see~Markov~A.\,S.&&\\
\Avtors{Ushakov~V.\,G.} Queueing system with working vacations and
hyperexponential input stream&2&92--97\\
\Avtors{Ushakov~V.\,G.} see~Leontyev~N.\,D.&&\\
\Avtors{Volnova~A.\,A.} see~Kalinichenko~L.\,A.&&\\
\Avtors{Yakovlev~O.\,A.\ and Gasilov~A.\,V.} Speeded-up stereo
matching using geodesic support weights&3&\hphantom{1}98--104\\
\Avtors{Zabezhailo~M.\,I.} see~Grusho~A.\,A.&&\\
\Avtors{Zabezhailo~M.\,I.} see~Grusho~A.\,A.&&\\
\Avtors{Zakharova~T.\,V.\ and Shestakov~O.\,V.} Precision analysis of
wavelet processing of aerodynamic\linebreak
\\[-12pt]
\hspace*{23pt}flow patterns&3&46--54\\
\Avtors{Zalizniak~Anna~A.\ and Kruzhkov~M.\,G.} Database
of~Russian impersonal verbal constructions&4&132--141\\
\Avtors{Zasypko~V.\,V.} see~Shnurkov~P.\,V.&&\\
\Avtors{Zatsarinny~A.\,A.\ and Suchkov~A.\,P.} Systems engineering
approaches to~the~establishment of\linebreak
\\[-12pt]
\hspace*{23pt}a~system for~decision support based
on~situational analysis&4&105--113\\
\Avtors{Zatsarinny~A.\,A.} see~Grusho~A.\,A.&&\\
\Avtors{Zatsman~I.\,M., Inkova~O.\,Yu., Kruzhkov~M.\,G., and
Popkova~N.\,A.} Representation of cross-\linebreak
\\[-12pt]
\hspace*{23pt}lingual knowledge about
connectors in supracorpora databases&1&106--118\\
\Avtors{Zatsman~I.\,M.} see~Minin~V.\,A.&&\\
\Avtors{Zeifman~A.\,I.} see~Korolev~V.\,Yu.&&\\
\Avtors{Zeifman~A.\,I.} see~Korolev~V.\,Yu.&&\\
\end{tabular}
}

%\thispagestyle{myheadings}
\def\leftfootline{\small{\textbf{\thepage}
\hfill INFORMATIKA I EE PRIMENENIYA~--- INFORMATICS AND APPLICATIONS\ \ \ 2016\
\ \ volume~10\ \ \ issue\ 4}
}%
 \def\rightfootline{\small{INFORMATIKA I EE PRIMENENIYA~---
INFORMATICS AND APPLICATIONS\ \ \ 2016\ \ \ volume~10\ \ \ issue\ 4
\hfill \textbf{\thepage}}}

 \label{end\stat}

\newpage

%\def\stat{rekl}
%\label{preobr}

%\def\tit{АКАДЕМИК ПУГАЧЁВ  ВЛАДИМИР СЕМЁНОВИЧ\\
%25.03.1911--25.03.1998}


%   \vspace*{-48pt}
%   \begin{center}\LARGE
%Академик Пугачёв  Владимир Семёнович\\ (25.03.1911--25.03.1998)
%   \end{center}
   
   %\vspace*{2.5mm}
   
   \begin{center}

{\prgsh\LARGE
ОБЪЯВЛЕНИЯ О КОНФЕРЕНЦИЯХ}

\end{center}
%\hrule

\vspace*{6pt}

   
   \vspace*{10mm}
   
   \thispagestyle{empty}

\noindent
\begin{tabular}{cc}
%\begin{center}
\multicolumn{1}{c}{\raisebox{-40pt}[0pt][0pt]{\mbox{%
\epsfxsize=33mm
\epsfbox{vspu.eps}
}}}
%\end{center}
&
\tabcolsep=0pt\begin{tabular}{c}
{\prg{\Large\textbf{XII Всероссийское совещание}}}\\[6pt]
{\prg{\Large\textbf{по проблемам управления}}}\\[12pt]
{\prg{\large 16--19 июня 2014~г.}}\\[6pt] 
{\prg{\large Институт проблем управления имени В.\,А.~Трапезникова РАН}}\\[6pt]
{\prg{\large Москва, Россия}}
\end{tabular}
\end{tabular}

\vspace*{60pt}

     
 { %\large    
 XII Всероссийское совещание по проблемам управления (ВСПУ XII), посвященное 75-летию 
Института проблем управления (ИПУ) имени В.\,А.~Трапезникова РАН, проводится 16--19~июня 
2014~г.\ 
в ИПУ РАН (г.~Москва, Россия). ВСПУ XII организуется ИПУ РАН при поддержке РФФИ, Отделения 
энергетики, машиностроения, механики и процессов управления Российской академии наук, 
Российского 
национального комитета по автоматическому управлению, Академии навигации и управ\-ле\-ния 
движением, 
Научного совета РАН по комплексным проблемам управления и автоматизации, Совета по 
мехатронике и робототехнике РАН. Официальный язык Совещания~--- русский.

\vspace*{24pt}
     
     \textbf{Направления работы}
     \begin{enumerate}[1.]
\item Теория систем управления
\item Управление подвижными объектами и навигация
\item Интеллектуальные системы управления
\item Управление в промышленности, транспортом и логистикой
\item Управление системами междисциплинарной природы
\item Средства измерения, вычислений и контроля в управлении
\item Системный анализ и принятие решений в задачах управления
\item Информационные технологии в управлении
\item Проблемы образования в области управления: современное содержание и технологии обучения
\end{enumerate}

\vspace*{24pt}

     Подробная информация о Совещании находится на сайте {\sf http://vspu2014.ipu.ru}. Срок 
окончательной подачи докладов через систему подачи докладов на сайте~--- \textbf{30~ноября} 
2013~г.
}

%\include{rekl-1}

%\end{document}

%\include{nekrolog-rb}


%\end{document}

%\include{IPPM-25}

\def\stat{cont-rus}
{%\hrule\par
%\vskip 7pt % 7pt
\vspace*{-24pt}
\raggedleft\Large \bf%\baselineskip=3.2ex
Правила подготовки рукописей  для публикации в журнале
<<Информатика~и~её~применения>> \vskip 8pt
    \hrule
    \par
\vskip 14pt plus 6pt minus 3pt }

\label{st\stat}

\def\tit{\ }

\def\aut{\ }
\def\auf{\ }

\def\leftkol{\ }
% Правила подготовки рукописей  для публикации в журнале
%<<Информатика и её применения>>

\def\rightkol{\ }
%Правила подготовки рукописей  для публикации в журнале
%<<Информатика и её применения>>}


\titele{\tit}{\aut}{\auf}{\leftkol}{\rightkol}


\vspace*{-60pt}
{ %\small

Журнал <<Информатика и её применения>>
публикует теоретические, обзорные и дискуссионные статьи,
посвященные научным исследованиям и разработкам в области
информатики и ее приложений.

Журнал издается на русском языке. По специальному решению
редколлегии отдельные статьи могут печататься на английском языке.

Тематика журнала охватывает следующие направления:
\begin{itemize}
\item теоретические основы информатики;\\[-15pt]
      \item
математические методы исследования сложных систем и процессов;\\[-15pt]
           \item
информационные системы и сети;\\[-15pt]
                \item
информационные технологии;\\[-15pt]
                     \item
архитектура и программное обеспечение вычислительных комплексов и сетей.\\[-15pt]
\end{itemize}


\noindent
\begin{enumerate}[1.]
\item В журнале печатаются статьи, содержащие результаты, ранее не опубликованные и
не предназначенные к одновременной публикации в других изданиях.

%Публикация не должна нарушать закон об авторских правах.
Публикация предоставленной автором(ами) рукописи не должна нарушать 
положений глав~69, 70 раздела~VII части~IV Гражданского кодекса, 
которые определяют права на результаты интеллектуальной деятельности 
и~средства индивидуализации, в~том числе авторские права, в~РФ.

Ответственность за нарушение авторских прав, в~случае предъявления претензий к~редакции журнала,  
несут авторы статей.



Направляя рукопись в редакцию, авторы сохраняют свои права на данную
рукопись и при этом передают учредителям и редколлегии журнала неисключительные права на
издание статьи на русском языке 
(или на языке статьи, если он отличен от рус\-ско\-го) и~на перевод ее на английский
язык, а~также на
ее распространение в России и за рубежом. 
Каждый автор должен представить в~редакцию подписанный 
с~его стороны <<Лицензионный договор о~передаче неисключительных прав 
на использование произведения>>, текст которого размещен по адресу 
{\sf http://www.ipiran.ru/publications/licence.doc}. 
Этот договор может быть пред\-став\-лен в~бумажном (в~2-х экз.)\ 
или в~электронном виде (отсканированная копия заполненного и~подписанного документа).




Редколлегия вправе запросить у авторов экспертное заключение о возможности
пуб\-ли\-ка\-ции пред\-став\-лен\-ной статьи в открытой печати.\\[-13.5pt]

\item К статье прилагаются данные автора (авторов) (см.\ п.~8). При наличии нескольких
авторов указывается фамилия автора, ответственного за переписку с редакцией.\\[-13.5pt]

\item Редакция журнала осуществляет экспертизу присланных статей в соответствии с
принятой в журнале процедурой рецензирования.

Возвращение рукописи на доработку не означает ее принятия к печати.

Доработанный вариант с ответом на замечания рецензента необходимо прислать в
редакцию.\\[-13.5pt]

\item Решение редколлегии о публикации статьи или ее отклонении сообщается авторам.

Редколлегия может также направить авторам текст рецензии на их статью. Дискуссия по
поводу отклоненных статей не ведется.\\[-13.5pt]

%\pagebreak

\item Редактура статей высылается авторам для просмотра. Замечания к редактуре должны
быть присланы авторами в кратчайшие сроки.\\[-13.5pt]

\item Рукопись предоставляется в электронном виде в форматах MS WORD (.doc или
.docx) или \LaTeX\  (.tex), дополнительно~--- в формате .pdf, на дискете, лазерном диске
или электронной почтой. Предоставление бумажной рукописи необязательно.\\[-13.5pt]

\item При подготовке рукописи в MS Word рекомендуется использовать следующие
настройки.

Параметры страницы:
формат~--- А4; ориентация~--- книжная; поля (см): внутри~--- 2,5, снаружи~--- 1,5,
сверху~--- 2, снизу~--- 2, от края до нижнего колонтитула~--- 1,3.

Основной текст: стиль~--- <<Обычный>>, шрифт~--- Times New Roman, размер~---
14~пунк\-тов, абзацный отступ~--- 0,5~см, 1,5~интервала, выравнивание~--- по ширине.

\pagebreak

\def\leftkol{Правила подготовки рукописей  для публикации в журнале
<<Информатика и её применения>>}

\def\rightkol{Правила подготовки рукописей  для публикации в журнале
<<Информатика и её применения>>}



Рекомендуемый объем рукописи~--- не свыше 10~страниц указанного формата.
При превышении указанного объема редколлегия вправе потребовать от 
автора сокращения объема рукописи.


Сокращения слов, помимо стандартных, не допускаются. Допускается минимальное
количество аббревиатур.


Все страницы рукописи нумеруются.

Шаблоны оформления представлены в интернете:

\noindent
 {\sf
http://www.ipiran.ru/journal/template\_iiep\_ssi\_2024.zip}\\[-14pt]

\item Статья должна содержать следующую информацию на {\bfseries\textit{русском и
английском языках}}:\\[-16pt]

\begin{itemize}
\item название статьи;\\[-15pt]
\item Ф.И.О.\ авторов, на английском можно только имя и фамилию;\\[-15pt]
\item место работы, с указанием почтового адреса организации и электронного адреса каждого
автора;\\[-15pt]
\item сведения об авторах, в соответствии с форматом, образцы которого
представлены на страницах:



\def\leftfootline{\small{\textbf{\thepage}
\hfill ИНФОРМАТИКА И ЕЁ ПРИМЕНЕНИЯ\ \ \ том\ 18\ \ \ выпуск\ 3\ \ \ 2024}
}%
 \def\rightfootline{\small{ИНФОРМАТИКА И ЕЁ ПРИМЕНЕНИЯ\ \ \ том\ 18\ \ \ выпуск\ 3\ \ \ 2024
\hfill \textbf{\thepage}}}



{\sf http://www.ipiran.ru/journal/issues/2013\_07\_01/authors.asp} и

{\sf http://www.ipiran.ru/journal/issues/2013\_07\_01\_eng/authors.asp};
\item аннотация (не менее 100~слов на каждом из языков). Аннотация~--- это краткое
резюме работы, которое может публиковаться отдельно. Она является основным
источником информации в~ин\-фор\-ма\-ци\-он\-ных системах и базах данных. Английская
аннотация должна быть оригинальной, может не быть дословным переводом русского
текста и должна быть написана хорошим английским языком. В~аннотации не должно
быть ссылок на литературу и, по возможности, формул;\\[-15pt]
\item ключевые слова~--- желательно из принятых в мировой
на\-уч\-но-тех\-ни\-че\-ской литературе тематических тезаурусов. Предложения не
могут быть ключевыми словами;\\[-15pt]
\item источники финансирования работы (ссылки на гранты, проекты,
поддерживающие организации и~т.\,п.).
\end{itemize}



%\pagebreak

\item  Требования к спискам литературы.\\[-14pt]

Ссылки на литературу в тексте статьи нумеруются (в квадратных скобках) и
располагаются в каждом из списков литературы в порядке  первых упоминаний. Если источник имеет DOI и/или EDN,
то их необходимо указывать.

Списки литературы представляются в двух вариантах:\\[-14pt]


\noindent
\begin{enumerate}[(1)]
\item \textbf{Список литературы к русскоязычной части}. Русские и английские
работы~---  на языке и в алфавите оригинала;\\[-14.5pt]
\item  \textbf{References}. Русские работы и работы на других языках~--- в латинской
транслитерации с переводом на английский язык; английские работы и работы на других
языках~--- на языке оригинала.
\end{enumerate}

Необходимо для составления списка ``References'' пользоваться размещенной на сайте
{\sf http://www. translit.net/ru/bgn/} бесплатной программой транслитерации русского
 текста в~латиницу. %, при этом в~за\-клад\-ке <<варианты\ldots>> следует выбратьопцию BGN.

Список литературы ``References'' приводится полностью отдельным блоком, повторяя все
позиции из списка литературы к русскоязычной части, независимо от того, имеются или
нет в нем иностранные источники. Если в списке литературы к русскоязычной части есть
ссылки на иностранные публикации, набранные латиницей, они полностью повторяются в
списке ``References''.

Ниже приведены примеры ссылок на различные виды публикаций в списке ``References''.

\def\leftfootline{\small{\textbf{\thepage}
\hfill ИНФОРМАТИКА И ЕЁ ПРИМЕНЕНИЯ\ \ \ том\ 18\ \ \ выпуск\ 3\ \ \ 2024}
}%
 \def\rightfootline{\small{ИНФОРМАТИКА И ЕЁ ПРИМЕНЕНИЯ\ \ \ том\ 18\ \ \ выпуск\ 3\ \ \ 2024
\hfill \textbf{\thepage}}}

{\small

\noindent
\textbf{Описание статьи из журнала:}

\Aue{Zagurenko, A.\,G., V.\,A.~Korotovskikh, A.\,A.~Kolesnikov, A.\,V.~Timonov, and D.\,V.~Kardymon}. 2008.
Tekhniko-ekonomicheskaya optimizatsiya dizayna gidrorazryva plasta [Technical and
economic optimization of the design
of hydraulic fracturing]. \textit{Neftyanoe hozyaystvo} [\textit{Oil Industry}] 11:54--57.

\Aue{Zhang, Z., and D.~Zhu}. 2008. Experimental research on the localized
electrochemical micromachining. \textit{Russ. J.~Electrochem.}  44(8):926--930.
{\sf doi:10.1134/S1023193508080077}.

\noindent
\textbf{Описание статьи из электронного журнала:}

\Aue{Swaminathan, V., E.~Lepkoswka-White, and B.\,P.~Rao}. 1999. Browsers or buyers in cyberspace? An
investigation of electronic factors influencing electronic exchange. \textit{JCMC}
5(2). Available at: {\sf http://www.ascusc.org/jcmc/vol5/issue2/} (accessed April~28, 2011).

\def\leftkol{Правила подготовки рукописей  для публикации в журнале
<<Информатика и её применения>>}

\def\rightkol{Правила подготовки рукописей  для публикации в журнале
<<Информатика и её применения>>}


\noindent
\textbf{Описание статьи из продолжающегося издания (сборника трудов):}

\Aue{Astakhov, M.\,V., and T.\,V.~Tagantsev}. 2006. Eksperimental'noe
issledovanie prochnosti soedineniy ``stal'--kompozit'' [Experimental study of
the strength of joints ``steel--composite'']. \textit{Trudy MGTU
``Matematicheskoe modelirovanie slozhnykh tekh\-ni\-che\-skikh sistem''}
[\textit{Bauman MSTU ``Mathematical Modeling of Complex Technical
Systems'' Proceedings}]. 593:125--130.


\pagebreak



\noindent
\textbf{Описание материалов конференций:}

\Aue{Usmanov, T.\,S., A.\,A.~Gusmanov, I.\,Z.~Mullagalin, R.\,Ju.~Muhametshina, A.\,N.~Chervyakova, and
A.\,V.~Sveshnikov}. 2007. Osobennosti proektirovaniya razrabotki mestorozhdeniy
s primeneniem gidrorazryva
plasta [Features of the design of field development with the use of hydraulic fracturing].
\textit{Trudy 6-go
Mezhdu\-na\-rod\-no\-go Simpoziuma ``Novye resursosberegayushchie tekhnologii nedropol'zovaniya i povysheniya
neftegazootdachi''} [\textit{6th  Symposium (International) ``New Energy Saving Subsoil Technologies and
the Increasing of the Oil and Gas Impact'' Proceedings}]. Moscow. 267--272.



\def\leftfootline{\small{\textbf{\thepage}
\hfill ИНФОРМАТИКА И ЕЁ ПРИМЕНЕНИЯ\ \ \ том\ 18\ \ \ выпуск\ 3\ \ \ 2024}
}%
 \def\rightfootline{\small{ИНФОРМАТИКА И ЕЁ ПРИМЕНЕНИЯ\ \ \ том\ 18\ \ \ выпуск\ 3\ \ \ 2024
\hfill \textbf{\thepage}}}



\noindent
\textbf{Описание книги (монографии, сборники):}



Lindorf, L.\,S., and L.\,G.~Mamikoniants, eds. 1972.
\textit{Ekspluatatsiya turbogeneratorov s neposredstvennym
okhlazhdeniem} [\textit{Operation of turbine generators with direct cooling}].
Moscow: Energy Publs. 352~p.


\Aue{Latyshev, V.\,N.} 2009. \textit{Tribologiya rezaniya. Kn.~1: Friktsionnye protsessy
pri rezanii metallov}
[\textit{Tribology of cutting. Vol.~1: Frictional processes in metal cutting}]. Ivanovo: Ivanovskii
State Univ. 108~p.

\def\leftkol{Правила подготовки рукописей  для публикации в журнале
<<Информатика и её применения>>}

\def\rightkol{Правила подготовки рукописей  для публикации в журнале
<<Информатика и её применения>>}

\noindent
\textbf{Описание переводной книги}
(в списке литературы к русскоязычной части необходимо указать:~/ Пер.\ с англ.~---
после названия книги, а в конце ссылки указать оригинал книги в круглых скобках):
\begin{enumerate}[1.]
\item  В русскоязычной части:

\def\leftfootline{\small{\textbf{\thepage}
\hfill ИНФОРМАТИКА И ЕЁ ПРИМЕНЕНИЯ\ \ \ том\ 18\ \ \ выпуск\ 3\ \ \ 2024}
}%
 \def\rightfootline{\small{ИНФОРМАТИКА И ЕЁ ПРИМЕНЕНИЯ\ \ \ том\ 18\ \ \ выпуск\ 3\ \ \ 2024
\hfill \textbf{\thepage}}}

\Au{Тимошенко С.\,П., Янг Д.\,Х., Уивер~У.}
Колебания в инженерном деле~/ Пер.\ с англ.~--- М.: Машиностроение, 1985. 472~с.
(\Au{Timoshenko~S.\,P., Young~D.\,H., Weaver~W.}
Vibration problems in engineering.~--- 4th ed.~--- New York, NY, USA: Wiley, 1974. 521~p.)\\[-13.5pt]
\item  В англоязычной части:

\Aue{Timoshenko, S.\,P., D.\,H.~Young, and W.~Weaver}.
1974. \textit{Vibration problems in engineering}. 4th ed. New York: 
Wiley. 521~p.
\end{enumerate}

\vspace*{-3pt}


\noindent
\textbf{Описание неопубликованного документа:}


\Aue{Latypov, A.\,R., M.\,M.~Khasanov, and V.\,A.~Baikov}.
2004 (unpubl.). Geologiya i~dobycha (NGT GiD) [Geology and production (NGT GiD)]. Certificate on official registration of the computer program
No.\,2004611198. 

\noindent
\textbf{Описание интернет-ресурса:}


Pravila tsitirovaniya istochnikov [Rules for the citing of sources]. Available at: {\sf
http://www.scribd.com/doc/1034528/} (accessed February~7, 2011).

%\pagebreak

\noindent
\textbf{Описание диссертации или автореферата диссертации:}

\Aue{Semenov, V.\,I.}
2003. Matematicheskoe modelirovanie plazmy v sisteme kompaktnyy tor [Mathematical
modeling of the plasma in the compact torus].  Moscow.  D.Sc.\ Diss. 272~p.

\Aue{Kozhunova, O.\,S.} 2009. Tekhnologiya razrabotki semanticheskogo
slovarya informatsionnogo monitoringa [Technology of development of
semantic dictionary of information monitoring system].  Moscow: IPI RAN. PhD Thesis. 23~p.


\noindent
\textbf{Описание ГОСТа:}

GOST 8.586.5-2005. 2007. Metodika vypolneniya izmereniy. Izmerenie raskhoda i~kolichestva zhidkostey i~gazov
s~pomoshch'yu standartnykh suzhayushchikh ustroystv [Method of measurement.
Measurement of flow rate and volume of liquids and gases by means of orifice devices]. Moscow:
Standardinform  Publs. 10~p.

\noindent
\textbf{Описание патента:}

\Aue{Bolshakov, M.\,V., A.\,V.~Kulakov, A.\,N.~Lavrenov, and M.\,V.~Palkin}.
2006. Sposob orientirovaniya po krenu letatel'nogo
apparata s opti\-che\-skoy golovkoy
samonavedeniya [The way to orient on the roll of aircraft with optical homing head].
Patent RF No.\,2280590.
}

\item Присланные в редакцию материалы авторам не возвращаются.\\[-13.5pt]

\item При отправке файлов по электронной почте просим придерживаться следующих
правил:
\begin{itemize}
\item указывать в поле subject (тема) название журнала и фамилию автора;\\[-13.5pt]
\item указывать в тексте письма название статьи, авторов и~журнал, в~который направляется статья;\\[-13.5pt]
\item использовать attach (присоединение);\\[-13.5pt]
\item в состав электронной версии статьи должны входить: файл, содержащий текст
статьи, и файл(ы), содержащий(е) иллюстрации.\\[-13.5pt]
\end{itemize}

\item Журнал <<Информатика и её применения>> является некоммерческим изданием.
Плата за публикацию не взимается, гонорар авторам не выплачивается.
\end{enumerate}



\def\leftfootline{\small{\textbf{\thepage}
\hfill ИНФОРМАТИКА И ЕЁ ПРИМЕНЕНИЯ\ \ \ том\ 18\ \ \ выпуск\ 3\ \ \ 2024}
}%
 \def\rightfootline{\small{ИНФОРМАТИКА И ЕЁ ПРИМЕНЕНИЯ\ \ \ том\ 18\ \ \ выпуск\ 3\ \ \ 2024
\hfill \textbf{\thepage}}}


\vspace*{-1mm}

\begin{center}

\textbf{Адрес редакции журнала <<Информатика и её применения>>:} \\




Москва 119333, ул.~Вавилова, д.~44, корп.~2, ФИЦ ИУ РАН\\[-10pt]

\

Тел.: +7\,(499)\,135-86-92\ \ Факс:  +7\,(495)\,930-45-05\\[-10pt]

 \

e-mail:   {\sf iiep@frccsc.ru} (Стригина Светлана Николаевна)\\[-10pt]

\

{\sf http://www.ipiran.ru/journal/issues/}
\end{center}
}


\def\leftkol{Правила подготовки рукописей  для публикации в журнале
<<Информатика и её применения>>}

\def\rightkol{Правила подготовки рукописей  для публикации в журнале
<<Информатика и её применения>>}


\def\leftfootline{\small{\textbf{\thepage}
\hfill ИНФОРМАТИКА И ЕЁ ПРИМЕНЕНИЯ\ \ \ том\ 18\ \ \ выпуск\ 3\ \ \ 2024}
}%
 \def\rightfootline{\small{ИНФОРМАТИКА И ЕЁ ПРИМЕНЕНИЯ\ \ \ том\ 18\ \ \ выпуск\ 3\ \ \ 2024
\hfill \textbf{\thepage}}} 
\def\stat{podg-e}
{%\hrule\par
%\vskip 7pt % 7pt
\vspace*{-24pt}
\raggedleft\Large \bf%\baselineskip=3.2ex
Requirements for manuscripts submitted to Journal
``Informatics~and~Applications'' \vskip 8pt
    \hrule
    \par
\vskip 21pt plus 6pt minus 3pt }

\label{st\stat}

\def\tit{\ }

\def\aut{\ }
\def\auf{\ }

\def\leftkol{\ }

\def\rightkol{\ }
%Requirements for manuscripts submitted to Journal
%``Informatics~and~Applications''}

\titele{\tit}{\aut}{\auf}{\leftkol}{\rightkol}

\def\leftfootline{\small{\textbf{\thepage}
\hfill INFORMATIKA I EE PRIMENENIYA~--- INFORMATICS AND APPLICATIONS\ \ \ 2019\
\ \ volume~13\ \ \ issue\ 4}
}%
 \def\rightfootline{\small{INFORMATIKA I EE PRIMENENIYA~--- INFORMATICS AND APPLICATIONS\ \ \ 2019\ \ \ volume~13\ \ \ issue\ 4
\hfill \textbf{\thepage}}}

\vspace*{-60pt}

{\small

\noindent
Journal ``Informatics and Applications'' (Inform.\ Appl.)
publishes theoretical, review, and discussion
articles on the research and development in the
field of informatics and its applications.

The journal is published in Russian.
By a special decision of the editorial
board, some articles can be published in English.


The topics covered include the following areas:
\begin{itemize}
               \item
     theoretical fundamentals of informatics; \\[-14pt]
\item
mathematical methods for studying complex systems and processes; \\[-14pt]
\item
information systems and networks;\\[-14pt]
\item
information technologies; and \\[-14pt]
\item
architecture and software of computational complexes and networks. \\[-14pt]
\end{itemize}

\noindent
\begin{enumerate}[1.]
\item The Journal publishes original articles which have not been published before and are not
intended for simultaneous publication in other editions. An article submitted to the Journal must not violate the
Copyright law. Sending the manuscript to the Editorial Board, the authors retain all rights of the
owners of the manuscript and transfer the nonexclusive rights to publish the article in Russian
(or the language of the article, if not Russian) and its distribution in Russia and abroad to the
Founders and the Editorial Board. Authors should submit a letter to the Editorial Board in the
following form:

{\bfseries\textit{Agreement on the transfer of rights to publish:}}

``\textit{We, the undersigned authors of the manuscript ``\ldots'', pass to the
Founder and the Editorial Board of the Journal ``Informatics and Applications''
the nonexclusive right to publish the manuscript of the article in Russian (or
in English) in both print and electronic versions of the Journal. We affirm
that this publication does not violate the Copyright of other persons or
organizations.}

\textit{Author(s) signature(s): (name(s), address(es), date).}

This agreement should be submitted in paper form or in the form of a scanned copy (signed by
the authors).


%The Editorial Board has the right to request from the authors an official expert conclusion that
%the submitted article has no secret data prohibited for publication. \\[-13.5pt]
\item
A submitted article should be attached with \textbf{the data on the author(s)} (see item~8). If
there are several authors, the contact person should be indicated who is responsible for
correspondence with the Editorial Board and other authors about revisions and final approval
of the proofs.\\[-13.5pt]

\item The Editorial Board of the Journal examines the article according to the established
reviewing procedure. If the authors receive their article for correction after reviewing, it does not
mean that the article is approved for publication. The corrected article should be sent to the
Editorial Board for the subsequent review and approval.\\[-13.5pt]

\item The decision on the article publication or its rejection is communicated to the authors. The
Editorial Board may also send the reviews on the submitted articles to the authors. Any
discussion upon the rejected articles is not possible.\\[-13.5pt]

\item The edited articles will be sent to the authors for proofread. The comments of the authors
to the edited text of the article should be sent to the Editorial Board as soon as possible.\\[-13.5pt]

\item The manuscript of the article should be presented electronically in the MS WORD (.doc or
.docx) or \LaTeX\ (.tex) formats, and additionally in the .pdf format. All documents
 may be sent
by e-mail or provided on a CD or diskette. A~hard copy submission is not necessary.\\[-13.5pt]

\item The recommended typesetting instructions for manuscript.

Pages parameters: format A4, portrait orientation, document margins (cm): left~--- 2.5, right~---
1.5, above~--- 2.0, below~--- 2.0, footer 1.3.

Text: font~---Times New Roman, font size~--- 14, paragraph indent~--- 0.5, line spacing~--- 1.5,
justified alignment.

The recommended manuscript size: not more than 15~pages of the specified format.
If the specified size exceeded, the editorial board is entitled to require the author
to reduce the manuscript.

Use only standard abbreviations. Avoid  abbreviations in the title and
abstract. The full term for which an abbreviation stands should precede
its first use in the text unless it is a standard unit of measurement.

All pages of the manuscript should be numbered.

The templates for the manuscript typesetting are presented on site: {\sf
http://www.ipiran.ru/journal/template.doc}.\\[-13.5pt]


%\def\leftkol{Requirements for manuscripts submitted to Journal
%``Informatics~and~Applications''}

\item The articles should enclose data both in \textbf{Russian and English}:
\begin{itemize}
\item title;\\[-13.5pt]
\item author's name and surname;\\[-13.5pt]
\item affiliation~--- organization, its address with ZIP code, city, country, and
official e-mail address;\\[-13.5pt]
\item data on authors according to the format: (see site)

{\sf http://www.ipiran.ru/journal/issues/2013\_07\_01/authors.asp}  and

{\sf  http://www.ipiran.ru/journal/issues/2013\_07\_01\_eng/authors.asp};\\[-13.5pt]

\pagebreak

\def\leftfootline{\small{\textbf{\thepage}
\hfill INFORMATIKA I EE PRIMENENIYA~--- INFORMATICS AND APPLICATIONS\ \ \ 2019\
\ \ volume~13\ \ \ issue\ 4}
}%
 \def\rightfootline{\small{INFORMATIKA I EE PRIMENENIYA~--- INFORMATICS AND APPLICATIONS\ \ \ 2019\ \ \ volume~13\ \ \ issue\ 4
\hfill \textbf{\thepage}}}


%\def\leftkol{Requirements for manuscripts submitted to Journal
%``Informatics~and~Applications''}

%\def\rightkol{Requirements for manuscripts submitted to Journal
%``Informatics~and~Applications''}



\item abstract (not less than 100 words) both in Russian and in English. Abstract is a short
summary of the article that can be published separately. The abstract is the
main source of information on the article and it could be included in leading information
systems and data bases. The abstract in English has to be an original text and should
not be an exact translation of the Russian one. Good English is required.
In abstracts, avoid references and formulae;\\[-13.5pt]
\item indexing is performed on the basis of keywords. The use of keywords from the
internationally accepted thematic Thesauri is recommended.

%\def\leftkol{Requirements for manuscripts submitted to Journal
%``Informatics~and~Applications''}

%\def\rightkol{Requirements for manuscripts submitted to Journal
%``Informatics~and~Applications''}

Important! Keywords must not be sentences;
\item Acknowledgments.
\end{itemize}

\item References. Russian references have to be presented both in English translation and Latin
transliteration (refer {\sf http://www.translit.net/ru/bgn/}).

Please take into account the following examples of Russian references appearance:

\noindent
\textbf{Article in journal:}

\Aue{Zhang, Z., and D.~Zhu}. 2008. Experimental research on the localized electrochemical
micromachining.
\textit{Rus. J.~Electrochem.}  44(8):926--930. {\sf doi:10.1134/S1023193508080077}.


\noindent
\textbf{Journal article in electronic format:}

\Aue{Swaminathan, V., E.~Lepkoswka-White, and B.\,P.~Rao}. 1999. Browsers or buyers in
cyberspace? An
investigation of electronic factors influencing electronic exchange. \textit{JCMC}
5(2). Available at: {\sf http://www.ascusc.org/jcmc/vol5/issue2/} (accessed April~28, 2011).




\noindent
\textbf{Article from the continuing publication (collection of works, proceedings):}

\Aue{Astakhov, M.\,V., and T.\,V.~Tagantsev}. 2006. Eksperimental'noe
issledovanie prochnosti soedineniy ``stal'--kompozit'' [Experimental study of
the strength of joints ``steel--composite'']. \textit{Trudy MGTU
``Matematicheskoe modelirovanie slozhnykh tekh\-ni\-che\-skikh sistem''}
[\textit{Bauman MSTU ``Mathematical Modeling of Complex Technical
Systems'' Proceedings}]. 593:125--130.

\def\leftfootline{\small{\textbf{\thepage}
\hfill INFORMATIKA I EE PRIMENENIYA~--- INFORMATICS AND APPLICATIONS\ \ \ 2019\
\ \ volume~13\ \ \ issue\ 4}
}%
 \def\rightfootline{\small{INFORMATIKA I EE PRIMENENIYA~--- INFORMATICS AND APPLICATIONS\ \ \ 2019\ \ \ volume~13\ \ \ issue\ 4
\hfill \textbf{\thepage}}}

\def\leftkol{Requirements for manuscripts submitted to Journal
``Informatics~and~Applications''}

\def\rightkol{Requirements for manuscripts submitted to Journal
``Informatics~and~Applications''}

\noindent
\textbf{Conference proceedings:}

\Aue{Usmanov, T.\,S., A.\,A.~Gusmanov, I.\,Z.~Mullagalin, R.\,Ju.~Muhametshina,
A.\,N.~Chervyakova, and
A.\,V.~Sveshnikov}. 2007. Osobennosti proektirovaniya razrabotki mestorozhdeniy
s primeneniem gidrorazryva
plasta [Features of the design of field development with the use of hydraulic fracturing].
\textit{Trudy 6-go
Mezhdu\-na\-rod\-no\-go Simpoziuma ``Novye resursosberegayushchie tekhnologii
nedropol'zovaniya i povysheniya
neftegazootdachi''} [\textit{6th  Symposium (International) ``New Energy Saving Subsoil
Technologies and
the Increasing of the Oil and Gas Impact'' Proceedings}]. Moscow. 267--272.


\noindent
\textbf{Books and other monographs:}




Lindorf, L.\,S., and L.\,G.~Mamikoniants, eds. 1972.
\textit{Ekspluatatsiya turbogeneratorov s neposredstvennym
okhlazhdeniem} [\textit{Operation of turbine generators with direct cooling}].
Moscow: Energy Publs. 352~p.


%\Aue{Latyshev, V.\,N.} 2009. \textit{Tribologiya rezaniya. Kn.~1: Frikcionnye prosessy
%pri rezanii metallov}
%[\textit{Tribology of cutting. Vol.~1: Frictional processes in metal cutting}]. Ivanovo: Ivanovskii
%State Univ. 108~p.


%\noindent
%\textbf{Unpublished material:}

%\Aue{Latypov, A.\,R., M.\,M.~Khasanov, and V.\,A.~Baikov}.
%2004. Geology and production (NGT GiD). Certificate on official registration of the computer
%program
%No.\,2004611198. (In Russian, unpubl.)

%\noindent
%\textbf{Internet-source:}

%APA Style. 2011. Available at: {\sf http://www.apastyle.org/apa-style-help.aspx} (accessed
%February~5, 2011).

%Pravila citirovaniya istochnikov [Rules for the citing of sources]. Available at: {\sf
%http://www.scribd.com/doc/1034528/} (accessed February~7, 2011).


\noindent
\textbf{Dissertation and Thesis:}

%\Aue{Semenov, V.\,I.}
%2003. Matematicheskoe modelirovanie plazmy v sisteme kompaktnyy tor. [Mathematical
%modeling of the plasma in the compact torus]. D.Sc.\ Diss. Moscow. 272~p.

\Aue{Kozhunova, O.\,S.} 2009. Tekhnologiya razrabotki semanticheskogo
slovarya informatsionnogo monitoringa [Technology of development of
semantic dictionary of information monitoring system]. PhD Thesis. Moscow: IPI RAN. 23~p.


\noindent
\textbf{State standards and patents:}

GOST 8.586.5-2005. 2007. Metodika vypolneniya izmereniy. Izmerenie raskhoda i~kolichestva
zhidkostey i gazov 
s~pomoshch'yu standartnykh suzhayushchikh ustroystv [Method of measurement.
Measurement of flow rate and volume of liquids and gases by means of orifice devices]. M.:
Standardinform
Publs. 10~p.

%\noindent
%\textbf{Patent:}

\Aue{Bolshakov, M.\,V., A.\,V.~Kulakov, A.\,N.~Lavrenov, and M.\,V.~Palkin}.
2006. Sposob orientirovaniya po krenu letatel'nogo
apparata s opti\-che\-skoy golovkoy
samonavedeniya [The way to orient on the roll of aircraft with optical homing head].
Patent RF No.\,2280590.

References in Latin transcription are presented in the original language.

References in the text are numbered according to the order of their
first appearance; the number is
placed in square brackets. All items from the reference list should be
cited.\\[-13.5pt]

\item Manuscripts and additional materials are not returned to Authors by the Editorial Board.\\[-13.5pt]

\item Submissions of files by e-mail must include:\\[-13.5pt]
\begin{itemize}
\item   the journal title and author's name in the ``Subject'' field; \\[-13.5pt]
\item   an article and additional materials have to be attached using the ``attach'' function;\\[-13.5pt]
\item   an electronic version of the article should contain the file with the text and a separate file
with figures.\\[-13.5pt]
\end{itemize}

\item ``Informatics and Applications'' journal is not a profit publication. There are no
charges for the authors as well as there are no royalties.\\[-13.5pt]
\end{enumerate}

\def\leftfootline{\small{\textbf{\thepage}
\hfill INFORMATIKA I EE PRIMENENIYA~--- INFORMATICS AND APPLICATIONS\ \ \ 2019\
\ \ volume~13\ \ \ issue\ 4}
}%
 \def\rightfootline{\small{INFORMATIKA I EE PRIMENENIYA~--- INFORMATICS AND APPLICATIONS\ \ \ 2019\ \ \ volume~13\ \ \ issue\ 4
\hfill \textbf{\thepage}}}

\def\leftkol{Requirements for manuscripts submitted to Journal
``Informatics~and~Applications''}

\def\rightkol{Requirements for manuscripts submitted to Journal
``Informatics~and~Applications''}


%\vspace*{5mm}


\begin{center}
\textbf{Editorial Board address:} \\

%ABOUT AUTHORS



FRC CSC RAS, 44, block~2, Vavilov Str., Moscow 119333, Russia\\[-10pt]

\

Ph.: +7\,(499)\,135\,86\,92,\ \ Fax: +7\,(495)\,930\,45\,05\\[-10pt]

\

 e-mail: {\sf rust@ipiran.ru} (to Prof.\ Rustem Seyful-Mulyukov)\\[-10pt]

\

 {\sf http://www.ipiran.ru/english/journal.asp}
\end{center}
 }
%\thispagestyle{myheadings}

\def\leftkol{Requirements for manuscripts submitted to Journal
``Informatics~and~Applications''}

\def\rightkol{Requirements for manuscripts submitted to Journal
``Informatics~and~Applications''}

\def\leftfootline{\small{\textbf{\thepage}
\hfill INFORMATIKA I EE PRIMENENIYA~--- INFORMATICS AND APPLICATIONS\ \ \ 2019\
\ \ volume~13\ \ \ issue\ 4}
}%
 \def\rightfootline{\small{INFORMATIKA I EE PRIMENENIYA~--- INFORMATICS AND APPLICATIONS\ \ \ 2019\ \ \ volume~13\ \ \ issue\ 4
\hfill \textbf{\thepage}}}

 \label{end\stat}

\newpage

%\vspace*{-60pt} {\small
{\baselineskip=9.1pt
\section*{Правила подготовки рукописей статей для публикации в журнале
<<Информатика и её применения>>}

\thispagestyle{empty}

 Журнал <<Информатика и её применения>> публикует
теоретические, обзорные и дискуссионные статьи, посвященные научным
исследованиям и разработкам в области информатики и ее приложений. Журнал
издается на русском языке. По специальному решению редколлегии отдельные статьи,
в виде исключения, могут печататься на английском языке.
Тематика журнала охватывает следующие направления:
\begin{itemize}
\item теоретические основы информатики; %\\[-13.5pt]
\item математические методы исследования сложных систем и процессов; %\\[-13.5pt]
\item информационные системы и сети; %\\[-13.5pt]
\item информационные технологии; %\\[-13.5pt]
\item архитектура и программное
обеспечение вычислительных комплексов и сетей.
\end{itemize}
\begin{enumerate}
\item В журнале печатаются результаты, ранее не
опубликованные и не предназначенные к одновременной публикации в других
изданиях. Публикация не должна нарушать закон об авторских правах. Направляя
свою рукопись в редакцию, авторы автоматически передают учредителям и
редколлегии неисключительные права на издание данной статьи на русском языке и
на ее распространение в России и за рубежом. При этом за авторами сохраняются
все права как собственников данной рукописи. В связи с этим авторами должно
быть представлено в редакцию письмо в следующей форме:
Соглашение о передаче права на публикацию:

\textit{<<Мы, нижеподписавшиеся, авторы рукописи <<$\qquad\qquad$>>, передаем
учредителям и редколлегии журнала <<Информатика и её применения>>
неисключительное право опубликовать данную рукопись статьи на русском языке как
в печатной, так и в электронной версиях журнала. Мы подтверждаем, что данная
публикация не нарушает авторского права других лиц или организаций. Подписи
авторов: (ф.\,и.\,о., дата, адрес)>>.}

Указанное соглашение может быть представлено 
как в бумажном виде, так и в виде отсканированной копии (с подписями авторов).


Редколлегия вправе запросить у авторов экспертное заключение о возможности
опубликования представленной статьи в открытой печати. %\\[-13.5pt]
\item Статья
подписывается всеми авторами. На отдельном листе представляются данные автора
(или всех авторов): фамилия, полные имя и отчество, телефон, факс, e-mail,
почтовый адрес. Если работа выполнена несколькими авторами, указывается фамилия
одного из них, ответственного за переписку с редакцией. %\\[-13.5pt]
\item Редакция журнала
осуществляет самостоятельную экспертизу присланных статей. Возвращение рукописи
на доработку не означает, что статья уже принята к печати. Доработанный вариант
с ответом на замечания рецензента необходимо прислать в редакцию. %\\[-13.5pt]
\item Решение
редакционной коллегии о принятии статьи к печати или ее отклонении сообщается
авторам. Редколлегия не обязуется направлять рецензию авторам отклоненной
статьи. %\\[-13.5pt]
\item Корректура статей высылается авторам для просмотра. Редакция
просит авторов присылать свои замечания в кратчайшие сроки. %\\[-13.5pt]
\item При
подготовке рукописи в MS Word рекомендуется использовать следующие настройки.
Параметры страницы: формат~--- А4; ориентация~--- книжная; поля (см): внутри~---
2,5, снаружи~--- 1,5, сверху~--- 2, снизу~--- 2, от края до нижнего
колонтитула~--- 1,3. Основной текст: стиль~--- <<Обычный>>: шрифт Times New
Roman, размер 14~пунктов, абзацный отступ~--- 0,5~см, 1,5 интервала,
выравнивание~--- по ширине. Рекомендуемый объем рукописи~--- не свыше
25~страниц указанного формата. Ознакомиться с шаблонами, содержащими примеры
оформления, можно по адресу в Интернете:
\textsf{http://www.ipiran.ru/journal/template.doc}.
\item К рукописи, предоставляемой в 2-х
экземплярах, обязательно прилагается электронная версия статьи (как правило, в
форматах MS WORD (.doc) или \LaTeX\ (.tex), а также~--- дополнительно~--- в
формате .pdf) на дискете, лазерном диске или по электронной почте. Сокращения
слов, кроме стандартных, не применяются. Все страницы рукописи должны быть
пронумерованы. %\\[-13.5pt]
\item Статья должна содержать следующую информацию на русском и
английском языках: название, Ф.И.О. авторов, места работы авторов и их
электронные адреса, подробные сведения об авторах, оформленные в соответствии с форматом, 
определяемым файлами {\sf http://www.ipiran.ru/journal/issues/2011\_05\_01/authors.asp} и 
{\sf http://www.ipiran.ru/journal/issues/2011\_01\_eng/authors.asp},
аннотация (не более 100~слов), ключевые слова. Ссылки на
литературу в тексте статьи нумеруются (в квадратных скобках) и располагаются в
порядке их первого упоминания. В~списке литературы не должно быть позиций, на которые нет ссылки в тексте статьи.
Все фамилии авторов, заглавия статей, названия
книг, конференций и~т.\,п.\ даются на языке оригинала, если этот язык
использует кириллический или латинский алфавит. %\\[-13.5pt]
\item Присланные в редакцию материалы авторам не возвращаются.
\item При отправке файлов по электронной
почте просим придерживаться следующих правил:
\begin{itemize}
\item указывать в поле subject (тема) название журнала и фамилию автора; %\\[-13.5pt]
\item использовать attach (присоединение); %\\[-13.5pt]
\item в случае больших объемов информации возможно
использование общеизвестных архиваторов (ZIP, RAR); %\\[-13.5pt]
\item в состав электронной версии статьи должны входить: файл, содержащий текст статьи, и файл(ы),
содержащий(е) иллюстрации. %\\[-13.5pt]
\end{itemize}
\item Журнал <<Информатика и её применения>> является некоммерческим изданием. 
Плата за публикацию с авторов не взимается, гонорар авторам не выплачивается.
\end{enumerate}
\thispagestyle{empty}
\textbf{Адрес редакции:} Москва 119333,
ул.~Вавилова, д.~44, корп.~2, ИПИ РАН\\
\hphantom{\textbf{Адрес редакции:} }Тел.: +7 (499) 135-86-92\ \
Факс:  +7 (495) 930-45-05\ \  E-mail:   rust@ipiran.ru }
}

%\include{ipi-ind}

%\tableofcontents

\end{document}

%\tableofcontents

%\end{document}

%\tableofcontents


\end{document}

\newcommand{\Ack}{\subsection*{\protect\large\bf Acknowledgments}}

\vphantom*{\int\limits_0^T}

{ \begin{center}  %fig1
 \vspace*{6pt}
    \mbox{%
 \epsfxsize=79mm 
 \epsfbox{gru-1.eps}
 }

\end{center}



\noindent
{{\figurename~1}\ \ \small{
}}}

%\vspace*{6pt}

\addtocounter{figure}{1}

$\acute{\mbox{о}}$

\linebreak