\def\stat{agalarov}

\def\tit{ОБ ОПТИМИЗАЦИИ РАБОТЫ РЕЗЕРВНОГО ПРИБОРА В~МНОГОЛИНЕЙНОЙ 
СИСТЕМЕ МАССОВОГО ОБСЛУЖИВАНИЯ}

\def\titkol{Об оптимизации работы резервного прибора в~многолинейной 
системе массового обслуживания}

\def\aut{Я.\,М.~Агаларов$^1$}

\def\autkol{Я.\,М.~Агаларов}

\titel{\tit}{\aut}{\autkol}{\titkol}

\index{Агаларов Я.\,М.}
\index{Agalarov Ya.\,M.}


%{\renewcommand{\thefootnote}{\fnsymbol{footnote}} \footnotetext[1]
%{Работа выполнена при поддержке Министерства науки и~высшего образования
%Российской федерации, грант №\,075-15-2020-799.}}


\renewcommand{\thefootnote}{\arabic{footnote}}
\footnotetext[1]{Федеральный исследовательский центр <<Информатика и~управление>> Российской 
академии наук, \mbox{agglar@yandex.ru}}

%\vspace*{-12pt}


  
  \Abst{Рассматривается задача оптимизации работы (подключения или отключения) 
резервного прибора управляемой системы массового обслуживания (СМО) типа $G/M/s$. Она 
сформулирована в~виде задачи нелинейного программирования со стоимостной целевой 
функцией, учитывающей плату за обслуживание заявок, затраты на техническое 
обслуживание резервного прибора, потери из-за задержки заявок в~очереди и~простоя 
резервного прибора. Процесс работы системы описан управляемой цепью Маркова, где 
состояние цепи определяется числом заявок в~сис\-те\-ме, а~последовательность решений, 
принимаемых на каждом шаге цепи в~зависимости от его состояния, определяет процесс 
управления резервным прибором. В~качестве допустимого множества управ\-ле\-ний 
рассмотрено множество стационарных стратегий управ\-ле\-ния цепью, решение в~которой 
в~каждом состоянии цепи о~подключении или отключении резервного прибора принимается 
по длине очереди. Для случая пуассоновского входного потока доказана тео\-ре\-ма 
о~необходимых и~достаточных условиях существования конечного оптимального 
порогового значения длины очереди и~унимодальность целевой функции от порогового 
значения, предложен простой алгоритм решения задачи.}
  
  \KW{многолинейная система массового обслуживания; оптимизация; резервный прибор}
  
 \DOI{10.14357/19922264230112} 
  
%\vspace*{-8pt}


\vskip 10pt plus 9pt minus 6pt

\thispagestyle{headings}

\begin{multicols}{2}

\label{st\stat}
  
\section{Введение}
  
  Одной из практических задач, при решении которой исследователи 
в~качестве математической модели используют многолинейные СМО с~резервными приборами, является задача выбора 
оптимального режима подключения (отключения) резервного элемента 
системы с~целью обеспечения требуемого качества работы системы при 
одновременном снижении затрат на ресурсы~[1--3]. Математические 
постановки некоторых подобных задач и~результаты их исследования 
приведены в~работах~[1--7]. Рассматриваемая в~данной работе задача 
представляет собой обобщение задачи максимизации дохода СМО $G/M/1$ 
с~резервным прибором, исследованной в~работе~\cite{7-al} с~по\-мощью метода 
вложенных цепей Маркова, где переходы цепи определяются моментами 
поступления заявок, а~со\-сто\-яние цепи~--- чис\-лом заявок, находящихся  
в~сис\-те\-ме в~момент поступления. Подключение (отключение) резервного 
прибора системы, рас\-смот\-рен\-ной в~работе~\cite{7-al}, происходит по 
достижении длиной очереди порогового значения~$h_1$. 
Доход представлен 
функцией вида:
  \begin{equation}
  D(h_1)= \sum\limits^\infty_{i=0} \fr{d_i^{h_1} \pi_i^{h_1}}{\overline{v}}\,,
  \label{e1-al}
  \end{equation}
  
\columnbreak
  
  \noindent
где $\left\{ \pi_i^{h_1},\ 0\hm \leq i\hm\leq \infty\right\}$~--- стационарное 
распределение вероятностей цепи при пороге~$h_1$ ($\pi_i^{h_1}$~--- 
стационарная вероятность того, что цепь находится в~со\-сто\-янии~$i$); 
$\overline{v}$~--- среднее время нахождения в~со\-сто\-янии~~$i$, 
$\overline{v}\hm < (\mu\hm+ \gamma)^{-1}$, $\mu\hm>0$~--- ин\-тен\-сив\-ность 
обслуживания основным прибором, $\gamma\hm>0$~--- ин\-тен\-сив\-ность 
обслуживания резервным прибором; $d_i^{h_1}$~--- средний доход, 
получаемый системой в~со\-сто\-янии~$i$ при пороге~$h_1$.

 Доход~$d_i^{h_1}$ 
вы\-чис\-ля\-ет\-ся по формуле:

\vspace*{-8pt}

\noindent
\begin{multline}
d_i^{h_1} =C_0 -C_1 \overline{T}_i^{h_1} - C_2\left( \overline{v} -
\overline{T}^{h_1}_{\mathrm{пр},i} \right) -C_3 
\overline{T}^{h_1}_{\mathrm{пр},i}\,,\\
0\leq i\,,
\label{e2-al}
\end{multline}

\vspace*{-8pt}

\noindent
где $\overline{T}_i^{h_1}$~--- среднее суммарное время задержки заявок 
в~очереди за время нахождения сис\-те\-мы в~со\-сто\-янии~$i$; 
$\overline{T}^{h_1}_{\mathrm{пр},i}$~--- среднее время простоя резервного 
прибора в~со\-сто\-янии~$i$; $C_0\hm\geq 0$~--- плата, по\-лу\-ча\-емая сис\-те\-мой, если 
поступившая заявка обслужена сис\-те\-мой (принята в~накопитель); $C_1\hm\geq 
0$~--- потери в~единицу времени за ожидание заявки в~сис\-те\-ме; $C_2\hm\geq 
0$~---  потери на амортизацию резервного прибора в~единицу времени, когда он 
подключен к~сис\-те\-ме и~занят заявкой; $C_3\hm\geq 0$~--- потери в~единицу 
времени простоя ($C_3\hm\leq C_2$).
  
  Предполагается, что доход сис\-те\-мы в~единицу времени при любом значении 
порога имеет некоторое конечное значение.

\pagebreak
  
  Основные результаты работы~\cite{7-al}~--- доказательство уни\-мо\-даль\-ности 
целевой функции~$D$ по~$h_1$ и~вывод условий существования конечного 
оптимального порогового значения в~случае СМО $G/M/1$.
{ %\looseness=-1

}

%\pagebreak
  
  В данной работе показано, что аналогичные результаты имеют место и~для 
многолинейной СМО с~резервным прибором.

%\vspace*{-4pt}
  
\section{Постановка задачи и~метод~решения}

%\vspace*{-4pt}

  Рассматривается СМО типа $G/M/s$ с~накопителем бесконечной ем\-кости, $s$ 
основными однотипными приборами и~одним резервным прибором 
обслуживания, на которую поступает рекуррентный поток заявок с~функцией 
распределения вероятностей~$V(t)$. Время обслуживания заявки на основном 
приборе распределено по экспоненциальному закону с~па\-ра\-мет\-ром $\mu\hm>0$, а~на 
резервном~--- с~параметром $\gamma\hm> 0$. Резервный прибор подключается 
к~сис\-те\-ме в~момент времени, когда в~сис\-те\-ме число заявок становится 
равным пороговому значению~$h_1$, и~отключается от сис\-те\-мы, когда 
число заявок становится равным $(h_1\hm-1)$, причем если это происходит 
в~момент завершения обслуживания очередной заявки на основном приборе, то 
заявка на резервном приборе перед его отключением переходит на 
освободившийся прибор и~заново обслуживается. По завершении 
обслуживания заявка уходит из сис\-те\-мы, освободив одновременно прибор 
и~занимаемое мес\-то в~накопителе. Пусть, как и~в~\cite{7-al}, показатель 
эффективности работы сис\-те\-мы представлен функцией дохода вида~(\ref{e1-al}). 
Ставится задача целочисленного программирования сле\-ду\-юще\-го вида: найти 
порог~$h^*$, такой что 
$$
\max\limits_{s<h_1} D\left(h_1\right) = D(h^*),
$$ 
где $D(h_1)$ определена в~(1).
   
  Подход, применяемый ниже к~решению данной задачи, аналогичен подходу, 
изложенному в~\cite{7-al}, суть которого заключается в~проверке выполнения 
условий тео\-ре\-мы~1 из~\cite{8-al} для целевой функции задачи для СМО 
c~ограниченным накопителем. 
  
  Пусть данная задача рассматривается для сис\-те\-мы $G/M/s/h_2$ 
c~ограниченным накопителем ем\-кости~$h_2$, $h_2\hm= h_1\hm+ a$, 
$h_1\hm\geq s\hm+1$, $a\hm=const\hm\geq 0$. Пусть заявка, которая застает 
накопитель пол\-ностью занятой, теряется и~в~этом случае сис\-те\-ма не получает 
плату за обслуживание. Ставится задача: найти порог~$h^*$, такой что 
  \begin{equation}
  \max\limits_{s<h_1} D(h_1,a) =D\left( h^*,a\right).
  \label{e3-al}
  \end{equation}
  
  \noindent
  Здесь
 \begin{equation*}
D(h_1,a) =\sum\limits_{i=0}^{h_2} d_i^{h_1} \pi_i^{h_1}\,,
\end{equation*}
где
\begin{multline*}
 d_i^{h_1}={}\\
\!\!\!{}= \!\begin{cases} 
C_0-C_1\overline{T}_i^{h_1} -C_2\left( \overline{v} - \overline{T}^{h_1}_{\mathrm{пр},i}\right) -C_3  \overline{T}^{h_1}_{\mathrm{пр},i}, &\\
& \hspace*{-27mm} 0\leq i\leq h_2-1\,;\\
d^{h_1}_{h_2-1} -C_0, & \hspace*{-21mm} i=h_2\,;
\end{cases}\!\!
%\label{e4-al}
\end{multline*}


\noindent
  $d_i^{h_1}$, $\pi_i^{h_1}$, $C_0$, $C_1$, $C_2$, $C_3$, $\overline{v}$, 
$\overline{T}_i^{h_1}$ и~$\overline{T}^{h_1}_{\mathrm{пр},i}$~--- обозначения, 
использованные в~разд.~1.

  Обозначим: 
  $$
  \mu_i^{h_1}= \begin{cases}
  s\mu & \mbox{при } s-1\leq i\leq h_1-2\,;\\
  s\mu +\gamma & \mbox{при } h_1\leq i\leq h_2\,;
  \end{cases}
  $$
$r_{i,m}^{h_1}$~--- вероятность того, что в~со\-сто\-янии~$i$ будут обслужены 
ров\-но $m$ заявок при условии, что в~этом со\-сто\-янии не произойдет отключение 
резервного прибора; $a^{h_1}_{i,m}$~--- вероятность того, что в~состоянии~$i$ 
будут обслужены ровно $m$ заявок при условии, что в~этом со\-сто\-янии 
произойдет отключение резервного прибора; $q_{ij}$~--- ве\-ро\-ят\-ность того, что в~состоянии~$i$ будут обслужены ров\-но ($i\hm+1\hm-j$) заявок при условиях 
$0\hm\leq i\hm\leq s\hm-1$ и~$0\hm\leq j\hm\leq i\hm+1$; $b_{ij}$~--- вероятность 
того, что в~со\-сто\-янии~$i$ будут обслужены ров\-но ($i\hm+1\hm-j$) заявок при 
условиях $h_1\hm- 1\hm\leq i\hm\leq h_2$ и~$0\hm\leq j\hm\leq s\hm-1$.
  
  Справедливы формулы:
  \begin{multline}
  r^{h_1}_{i,m} =\int\limits_0^\infty \fr{\left(\mu_i^{h_1} v\right)^m}{m!}\,e^{-\mu_i^{h_1}t}\,dV(v)\,,\\
   h_1-1\leq i\leq h_2\,,\enskip
  0 \leq m\leq i -h_1+1\\
   \mbox{или}\ s-1\leq i\leq h_1-2\,,\enskip 0\leq m\leq h_1-s-1\,;
   \label{e4-al}
  \end{multline}
   
   \vspace*{-12pt}
   
   \noindent
   \begin{multline}
   a^{h_1}_{i,m} =\int\limits_0^\infty \int\limits_0^v \fr{\mu_i^{h_1}\left( \mu_i^{h_1} t\right)^{i-h_1+1}} 
{(i-h_1+1)!}\,e^{-\mu_i^{h_1}t} \times{}\\
{}\times \fr{[\mu_1(v-t)]^{m-i+h_1-2}} {(m-i+h_1-2)!}\,e^{\mu_1(v-t)} \,dtdV(v)\,,\\
   h_1-1\leq i\leq h_2\,,\enskip s-1\leq j\leq h_1-1\,;
   \label{e5-al}
   \end{multline}
   
   \vspace*{-12pt}
   
   \noindent
   \begin{multline}
   q_{ij} =\int\limits_0^\infty \begin{pmatrix}
   i+1\\ j\end{pmatrix} \left[ 1-e^{-\mu v}\right]^{i+1-j} e^{-j\mu v}\,dV(v)\,,\\
   0\leq i\leq s-1\,,\enskip 0\leq j\leq i+1\,;
   \end{multline}
   
   %\vspace*{-12pt}
   
   \noindent
   \begin{multline}
   p_{ij}^{h_1} =\int\limits_0^\infty \begin{pmatrix}
   s\\ j\end{pmatrix} e^{-\mu v j} \Bigg[ \int\limits_0^v \fr{(s\mu t)^{i-s}} {(i-s)!}\left( e^{-\mu t}-{}\right.\\[3pt]
\left.   {}-e^{-\mu v}\right)^{s-j} s\mu\,dt\Bigg] dV(v)\,,\enskip 
   s\leq i\leq h_1-2\,,\\[3pt]
    0\leq j\leq s-1\,;
   \end{multline}
   
   \vspace*{-12pt}
   
   \noindent
   \begin{multline}
   b_{ij}^{h_1} =\int\limits_0^\infty \begin{pmatrix}
   s\\ j\end{pmatrix}
    \int\limits_0^v \int\limits_0^y \fr{\mu_i^{h_1}\left( \mu_i^{h_1}t\right)^{i-h_1+1}} 
{(i-h_1+1)!}\,e^{-\mu_i^{h_1} t}\times{}\\[3pt]
{}\times \fr{[s\mu(y-t)]^{h_1-s-2}}{(h_1-s-2)!} 
   e^{-s\mu(y-t)} s\mu\, dt \,e^{-\mu(v-y)j} \times{}\\[3pt]
   {}\times \left( 1-e^{-\mu(v-y)}\right)^{s-j} dydV(v)={}\\[3pt]
   {}=
   \int\limits_0^\infty \begin{pmatrix}
   s\\ j\end{pmatrix} 
   e^{-\mu v} \int\limits_0^v \int\limits_0^y \fr{\mu_i^{h_1}\left( \mu_i^{h_1} 
t\right)^{i-h_1+1}} {(i-h_1+1)!}\,e^{\mu_2 t}\times{}\\[3pt]
{}\times \fr{[s\mu(y-t)]^{h_1-s-2}} {(h_1-s-2)!}\left( e^{-\mu y} -e^{\mu v}\right)^{s-j} 
s\mu \,dt dy dV(v)\,,\\[3pt]
   h_1-1\leq i\leq h_2\,,\ 0\leq j\leq s-1\,.
   \label{e5-1-al}
   \end{multline}

  Из системы уравнений равновесия для вложенной цепи Маркова 
и~выражений~(\ref{e4-al})--(\ref{e5-1-al}) для па\-ра\-мет\-ров~$r_{i,m}^{h_1}$, $a_{i,m}^{h_1}$, 
$q_{ij}$, $p_{ij}^{h_1}$ и~$b_{ij}^{h_1}$ следует спра\-вед\-ли\-вость сле\-ду\-ющих 
рекуррентных формул для \mbox{стационарных} вероятностей 
со\-сто\-яний~$\pi_j^{h_1}$, $0\hm\leq j\hm\leq h_2$:
\begin{equation}
  \pi_j^{h_1}=\fr{R_j^{h_1}}{\sum\nolimits_{i=0}^{h_2} R_i^{h_1}}\,,\enskip 
0\leq j\leq h_2\,,
\label{e6-al}
  \end{equation}
  где
 \begin{equation*}
   R_{h_2}^{h_1} =1\,;\enskip R_{h_2-1}^{h_1}= \fr{1-r^{h_1}_{h_2,0}}{r^{h_1}_{h_2,0}}\,;
  \end{equation*}
  
  \vspace*{-12pt}
  
  \noindent
  \begin{multline*}
   R^{h_1}_{j-1}= \fr{R_j^{h_1} \left( 1-r^{h_1}_{j,1}\right)} {r^{h_1}_{j-1,0}}-{}\\
   {}-
\fr{\sum\nolimits_{i=j+1}^{h_2-1} R_i^{h_1} r^{h_1}_{j,i+1-j} +R^{h_1}_{h_2} r^{h_1}_{h_2, h_2-j}} {r^{h_1}_{j-1,0}}\,,\\
   h_1\leq j\leq h_2-1\,;
\end{multline*}

\vspace*{-12pt}

\noindent
   \begin{multline*}
   R^{h_1}_{h_1-2}= \fr{R^{h_1}_{h_1-1} (1-r_{h_1-1,1})} {r^{h_1}_{h_1-2,0}} -{}\\
   {}-\fr{\sum\nolimits^{h_2-1}_{i=h_1} R_i^{h_1} 
   r_{i,i+2-h_1} +R^{h_1}_{h_2} r_{h_2, h_2-h_1+1}} {r^{h_1}_{h_1-2,0}}\,;
  % \label{e6-al}
   \end{multline*}
   
   %\vspace*{-12pt}
   
   \noindent
   \begin{multline*}
   R^{h_1}_{j-1}= \fr{R_j^{h_1}(1-r_{j,1})}{r^{h_1}_{j-1,0}} - {}\\
   {}-
   \fr{\sum\nolimits_{i=j+1}^{h_1-2} R_i^{h_1} r_{i,i+1-j} +
\sum\nolimits_{i=h_1-1}^{h_2-1} R_i^{h_1} a_{i,i+1-j}} {r^{h_1}_{j-1,0}}-{}\\
   {}- \fr{R^{h_1}_{h_2} a_{h_2,h_2-j}}{r^{h_1}_{j-1,0}}\,,\enskip s\leq j\leq h_1-2\,;
   \end{multline*}
   
   \vspace*{-12pt}
   
   \noindent
   \begin{multline*}
  R_{j-1}^{h_1}= \fr{R_j^{h_1}(1-q_{ij}) } {q_{j-1,j}}-{}\\
  {}-\fr{\sum\nolimits^{s-1}_{i=j+1} R_i^{h_1} q_{ij} + \sum\nolimits_{i=s}^{h_1-2} R_i^{h_1} p_{ij}^{h_1} }{q_{j-1,j}}-{}\\
  {}- \fr{\sum\nolimits_{i=h_1-
1}^{h_2-1} R_i^{h_1} b_{ij}^{h_1}}{q_{j-1,j}}- \fr{R_{h_2}^{h_1} b^{h_1}_{h_2\,j}}{q_{j-1,j}}\,,\enskip 1\leq j\leq s-1\,.
  \end{multline*}
  
  Из~(\ref{e6-al}) следует равенство:
  \begin{equation}
  \pi_{i+1}^{k+1} =A_{k+1} \pi_i^k\,,\enskip i=s-1, \ldots , k\,.
  \label{e7-al}
  \end{equation}
  Здесь
  $$
   A_{k+1}=\fr{1-Q^{k+1}_{s-1}}{1-Q_{s-2}^k}\,, 
$$
  где
  \begin{equation*}
   Q_{s-1}^{k+1} =\sum\limits_{i=0}^{s-1} \pi_i^{k+1}\,;\enskip 
    Q^k_{s-2} =  \begin{cases}
  0, & 1\leq s\leq 2\,;\\
  \displaystyle\sum\limits_{i=0}^{s-2} \pi_i^k, & s\geq 2\,.
  \end{cases}
  \end{equation*}

  В дальнейшем для крат\-кости изложения будем пользоваться обозначениями:
  \begin{align*}
  r^{h_1}_{i,m}(v)&= \fr{\left(\mu_i^{h_1} v\right)^m}{m!}\,e^{-\mu_i^{h_1} t}\,;
  \\
     z_i^{h_1}(v,t,m) &=\fr{\mu_i^{h_1}\left(\mu_i^{h_1}t\right)^{i-h_1+1}}{(i-h_1+1)!}\,e^{-\mu_i^{h_1}t}\times{}\\
   &\hspace*{7mm}{}\times \fr{[\mu_1(v-t)]^{m-i+h_1-2}} {(m-i+h_1-2)!}\,e^{-\mu_1(v-t)}.
\end{align*}
  
  Ниже приведем ряд формул, доказательства справедливости которых для 
крат\-кости изложения не приводим (они аналогичны доказательствам, 
приведенным в~работе~\cite{7-al} для подобных формул).
  
  Среднее значение дохода~(\ref{e2-al}), по\-лу\-ча\-емо\-го сис\-те\-мой с~ограниченным 
накопителем при пороге~$h_1$ в~со\-сто\-янии~$i$, равно
  \begin{multline}
  d_i^{h_1} ={}\\
\!\!  {}=\!\begin{cases}
  C_0 -C_1 \overline{T}_i^{h_1} -C_2 \left(\overline{v} -
\overline{T}^{h_1}_{\mathrm{пр},i}\right) -C_3\overline{T}^{h_1}_{\mathrm{пр},i}\,, & \\
&\hspace*{-27mm}0\leq i\leq h_2-1\,;\\
  d^{h_1}_{h_2-1} -C_0\,,& \hspace*{-20.5mm}i=h_2\,,
  \end{cases}\!\!\!\!
  \label{e8-al}
  \end{multline}
где $\overline{T}_i^{h_1}$ и~$\overline{T}^{h_1}_{\mathrm{пр},i}$ определены 
в~(2) и~в~данном случае для них справедливы формулы:

\vspace*{-6pt}

\noindent
\begin{multline}
\overline{T}_i^{h_1} ={}\\
\!{}=\!
\begin{cases}
\fr{1}{\mu_i^{h_1}} \left[ (i+1-s) \displaystyle\sum\limits_{m=1}^{i+2-s} 
mr^{h_1}_{i,m} -{}\right.\\
\hspace*{15mm}{}-\fr{1}{2} \sum\limits_{m=1}^{i+2-s} (m-1) 
mr^{h_1}_{i,m}+{}\\
\left.{}+ \fr{1}{2}\left( i+1-s\right) (i+2-s) \!\!\sum\limits^\infty_{m=i+3-s} \!\!\!\!
r_{i,m}^{h_1}\right]
&\\
&\hspace*{-45mm}
  \mbox{при}\ s\leq i\leq h_1-2\\
  &\hspace*{-45mm}\mbox{и}\ \overline{T}_i^{h_1}=0\  \mbox{при }\ 
i\leq s-1;\\
\fr{1}{\mu_i^{h_1}} \left[ (i+1-s) \sum\limits_{m=1}^{i-h_1+3} mr^{h_1}_{i,m} -{}\right.\\
\left.  \hspace*{5mm}{}-\fr{1}{2} \sum\limits_{m=1}^{i-h_1+3} (m-1) 
mr^{h_1}_{i,m}\right]+{}\\
  \hspace*{5mm}{}+
  \displaystyle\sum\limits_{m=i-h_1+3}^{i+1-s} \int\limits_0^\infty \int\limits_0^v \left[ \fr{i-
h_1+1}{2}\,t +{}\right.\\
{}+(m\!-\!i\!+\!h_1\!-\!1)t+ \fr{1}{2}(m\!-\!i\!+\!h_1\!-\!2)\times{}\\
\left.{}\times(v-t)\right] z_i^{h_1}(v,t,m) \,dtdV(v)+{}\\
  {}+
 \displaystyle \!\!\! \sum\limits^{i-s}_{m=i-h_1+3} \int\limits_0^\infty \int\limits_0^v (i+1-m-s)v \times{}\\
 {}\times
z_i^{h_1}(v,t,m) \,dt dV(v)+{}\\
 \displaystyle {}+ \!\!\!\!\sum\limits^\infty_{m=i+2-s} \int\limits_0^\infty \int\limits_0^v \!\left[ \fr{i\!-\!h_1\!+\!1}{2}\,t +(h_1\!-\!1)t +{}\right.\\
\left.{}+\fr{(h_1-2)(h_1-1)(v-t)}{2(m-i+h_1-1)}\right]\times{}\\
{}\times   z_i^{h_1}(v,t,m)\,dtdV(v)  &\hspace*{-30mm}\mbox{при}\  i\geq h_1-1;\!
  \end{cases}\!\!
  \label{e9-al}
  \end{multline}
  
  \vspace*{-12pt}
  
  \noindent
  \begin{multline}
  \overline{T}^{h_1}_{\mathrm{пр}, i} ={}\\
 \!\!\!\! {}=\!\begin{cases}
  \overline{v} -\displaystyle \!\!\int\limits_0^\infty\! \!\int\limits_0^v t 
\fr{\mu_i^{h_1}\left( \mu_i^{h_1} t\right)^{i-h_1+1}}{(i-h_1+1)!}\, 
e^{-\mu_i^{h_1 t}} dtdV(v)\,, &\\
&\hspace*{-28mm}i\geq h_1-1\,;\\
  \overline{v}\,, & \hspace*{-34.5mm}0\leq i\leq h_1-2\,.
  \end{cases}\!\!\!\!\!\!
  \label{e10-al}
  \end{multline}
  
  \vspace*{-4pt}
  
  Среднее время отсутствия очереди при пороге~$h_1$ в~со\-сто\-янии~$i$ равно 
  
  \vspace*{-6pt}
  
  \noindent
  \begin{multline*}
  \overline{T}^{h_1}_{\mathrm{оч},i} ={}\\
  {}=\begin{cases}
  \displaystyle \sum\limits^\infty_{m=i+1} \int\limits_0^\infty \int\limits_0^v \left[ 
v-t-\fr{(h_1-1)(v-t)}{m-i+h_1-1}\right]\times{}\\
\hspace*{6mm}{}\times z_i^{h_1}(v,t,m)\,dtdV(v)\,, & \hspace*{-20mm}i\geq h_1-1\,;\\
  \displaystyle \fr{1}{\mu_1}\sum\limits^\infty_{m=i+2} (m-i-1) r^{h_1}_{i,m}\,, 
& \hspace*{-20mm} i\leq h_1-2\,.
  \end{cases}
  %\label{e11-al}
  \end{multline*}
  
  
  Из~(\ref{e8-al})--(\ref{e10-al}) после несложных преобразований получаем 
равенства:

\noindent
  \begin{multline}
  d^{h_1+1}_{i+1} -d_i^{h_1}= {}\\
\!\!\!  {}=\!
  \begin{cases}
  \displaystyle -\fr{C_1}{\mu_i^{h_1}} \sum\limits_{m=1}^{i-s+2} 
mr^{h_1}_{i,m} -{}&\\
\displaystyle {}-\fr{C_1(i+2-s)}{\mu_i^{h_1}} \sum\limits^\infty_{m=i-s+3} 
r^{h_1}_{i,m}   &\\
&\hspace*{-44mm} \mbox{при}\  s-1\leq i\leq h_1-2\,;\\
  \displaystyle -C_1\int\limits_0^\infty v \left[
  \vphantom{\int\limits_0^\infty}
   \sum\limits_{m=0}^{i-h_1+2} 
r^{h_1}_{i,m}(v) +{}\right.\\
\displaystyle \left.{}+\sum\limits_{m=i-h_1+3}^{i-s+1} \int\limits_0^v 
z_i^{h_1}(v,t,m)\,dt\right] dV(v)- {}&\\ 
\displaystyle {}-C_1 \!\!\!\!\!\sum\limits^\infty_{m=i-s+2} \int\limits_0^\infty \!\!\int\limits_0^v \left[ 
t+\fr{(h_1-1)(v-t)}{m-i+h_1-1}\right]\times{}&\\
{}\times z_i^{h_1}(v,t,m) \,dtdV(v) &\hspace*{-29mm}
\mbox{при}\  i\geq h_1-1\,.
\end{cases}\!\!\!\!
\label{e12-al}
\end{multline}
     
Введем обозначения:
\begin{multline}
w(h_1,a)=\sum\limits_{i=0}^{h_1-2} \pi_i^{h_1} \fr{1}{\mu_i^{h_1}} 
\sum\limits^\infty_{m=i-s+3} (m-i-1) r^{h_1}_{i,m} +{}\\
{}+
\sum\limits_{i=h_1-1}^{h_2-1}\! \!\!\pi_i^{h_1} \!\!\!\sum\limits^\infty_{m=i-s+2} 
\int\limits_0^\infty \left[ v -\!\!\int\limits_0^v \!\left[ t+\fr{(h_1-1) (v-t)}{m - i +h_1 -
1}\right]\times{}\right.\\
\left.{}\times z_i^{h_1}(v,t,m)\,dt\right] dV(v)+{}\\
{}+\pi_{h_2}^{h_1} \sum\limits^\infty_{m=h_2} \int\limits_0^\infty \left[ v- \!
\int\limits_0^v\! \left[ t+\fr{(h_1-1)(v-t)}{m-h_2+h_1}\right]\times{}\right.\\
\left.{}\times z_i^{h_1}(v,t,m)\,dt
\vphantom{\int\limits_0^\infty}
\right] 
dV(v)\,;
\label{e13-al}
\end{multline}

\vspace*{-12pt}

\noindent
\begin{multline}
f(h_1,a) =C_0-C_3\overline{v} -{}\\
{}-C_1\fr{A_{h_1+1}}{1-A_{h_1+1}} \left[ 
\overline{v} -w(h_1,a)\right].
\label{e14-al}
\end{multline}


Для разности доходов при порогах~$h_1$ и~$h_1\hm+1$ выполняются 
равенства (следуют из формул~(\ref{e1-al}), (\ref{e3-al}), (\ref{e7-al}),  
(\ref{e12-al})--(\ref{e14-al})):
\begin{multline*}
D\left(h_1,a\right) -D\left(h_1+1,a\right)={}\\
{}=\sum\limits_{i=0}^{h_2} \pi_i^{h_1} d_i^{h_1} -
\sum\limits_{i=0}^{h_2+1} \pi_i^{h_1+1} d_i^{h_1+1} ={}\\
{}= \left( C_0-C_3\overline{v}\right) \left( Q^{h_1}_{s-2} -Q_{s-1}^{h_1+1}\right) 
+ {}\\
{}+ \sum\limits^{h_2}_{i=s-1} \pi_i^{h_1} d_i^{h_1} -\sum\limits_{i=s}^{h_2+1} 
\pi_i^{h_1+1} d_i^{h_1+1}={}\\
\end{multline*}

\noindent
\begin{multline*}
{}=
     \left( C_0-C_3\overline{v}\right) \left( Q^{h_1}_{s-2} -Q_{s-1}^{h_1+1}\right)+ {}\\
     {}+
\fr{Q_{s-1}^{h_1+1} -Q_{s-2}^{h_1}}{1-Q^{h_1}_{s-2}}\left[ 
\sum\limits_{i=0}^{h_2} \pi_i^{h_1} d_i^{h_1} -\left( C_0-C_3\overline{v}\right) 
Q^{h_1}_{s-2}\right]-{}\\
  {}- 
  \fr{1-Q_{s-1}^{h_1+1}}{1-Q^{h_1}_{s-2}} \sum\limits^{h_2}_{i=s-1} 
\pi_i^{h_1}\left( d_{i+1}^{h_1+1} -d_i^{h_1}\right) ={}\\
{}= \left( 1-A_{h_1+1}\right) 
\left[ 
\vphantom{\sum\limits_{h_2}^{h_2}}
D(h_1,a) -\left( C_0 -C_3\overline{v}\right) -{}\right.\\
{}- \fr{A_{h_1+1}}{1-A_{h_1+1}} 
\left( \sum\limits_{i=s-1}^{h_2-1}  \pi_i^{h_1}\left( d_{i+1}^{h_1+1} -
d_i^{h_1}\right) +{}\right.\\
\left.\left.{}+\pi_{h_2}^{h_1} \left( d_{h_2}^{h_1+1} -d^{h_1}_{h_2-1}\right)\!
\vphantom{\sum\limits_{h_2}^{h_2}}
\right)\right]={}\\
  {}=
  \left(1-A_{h_1+1}\right) \left[ D(h_1,a) -f(h_1,a)\right].
  \end{multline*}
  
Как показывают многочисленные вы\-чис\-ли\-тель\-ные эксперименты, проведенные 
при различных значениях па\-ра\-мет\-ров СМО, величина $A_{h_1+1}$ воз\-рас\-та\-ет 
по~$h_1$. Ниже приведем доказательство этого положения для случая 
пуассоновского входного потока. Отметим, что доказательство спра\-вед\-ли\-вости 
такого свойства~$A_{h_1+1}$ в~случае произвольного рекуррентного 
входного потока остается открытым.

  Для случая пуассоновского входного потока процесс работы сис\-те\-мы 
описывается случайным процессом рож\-де\-ния и~гибели, для стационарных 
вероятностей со\-сто\-яний (чис\-ла заявок в~сис\-те\-ме) которого справедлива 
формула:
  $$
  \pi_i^{h_1} =\begin{cases}
  \fr{\rho^i}{i!}\,\pi_0^{h_1}, & \hspace*{-30mm}0\leq i\leq s\,;\\
  \fr{\rho^s}{s!} \left( \fr{\rho}{s}\right)^{i-s} \pi_0^{h_1}, & \hspace*{-30mm}s<i\leq h_1-1\,;\\
  \fr{\rho^s}{s!}\left(\fr{\rho}{s}\right)^{h_1-s-1} \left( 
\fr{\rho}{s+\alpha}\right)^{i-h_1+1} \pi_0^{h_1}, &\\
& \hspace*{-30mm}h_1\leq i\leq h_2,
  \end{cases}
  $$
где 
\begin{multline*}
\pi_0^{h_1} =\left[ \sum\limits^s_{j=0} \fr{\rho^j}{j!} +\fr{\rho^s}{s!} 
\sum\limits_{j=1}^{h_1-s-1} \left( \fr{\rho}{s}\right)^j +{}\right.\\
\left.{}+ \fr{\rho^s}{s!} \left( \fr{\rho}{s}\right)^{h_1-s-1} 
\sum\limits_{j=1}^{h_2-h_1-1} \left( \fr{\rho}{s+\alpha}\right)^j \right]^{-1}\!\!,\enskip
\alpha=\fr{\gamma}{\mu}\,.
\end{multline*}
  
  Рассмотрим раз\-ность $\left( A_{h_1+2}\hm- A_{h_1+1}\right)$, пред\-ста\-ви\-мую 
в~виде:

\noindent
  \begin{multline}
  \fr{\pi_0^{h_1+2} \left( \pi_0^{h_1+2}\right)^{-1} \left( 1-Q_{s-1}^{h_1+2}\right)} 
  {\pi_0^{h_1+1} \left( \pi_0^{h_1+1}\right)^{-1} \left( 1-Q_{s-2}^{h_1+1}\right)} -{}\\
  {}-
\fr{\pi_0^{h_1+1} \left( \pi_0^{h_1+1}\right)^{-1} \left(1- Q_{s-1}^{h_1+1}\right) } 
{\pi_0^{h_1}\left( \pi_0^{h_1}\right)^{-1} \left( 1-Q^{h_1}_{s-2}\right)}\,.
  \label{e15-al}
  \end{multline}
  %
  Выполняется равенство:
  \begin{multline*}
  \left(\pi_0^{h_1+2}\right)^{-1} \left( 1-Q_{s-1}^{h_1+2}\right) 
\left(\pi_0^{h_1}\right)^{-1} \left( 1-Q_{s-2}^{h_1}\right)={}\\
  {}= \left(\pi_0^{h_1+1}\right)^{-1} \left( 1-Q_{s-1}^{h_1+1}\right) 
\pi_0^{h_1+1} \times{}\\
{}\times \left(\pi_0^{h_1+1}\right)^{-1} \left( 1- Q_{s-2}^{h_1+1}\right).
  \end{multline*}
  %
  Отсюда следует, что знак выражения~(\ref{e15-al}) такой же, что 
и~у~раз\-ности $\pi_0^{h_1+2} \pi_0^{h_1}\hm-\left(\pi_0^{h_1+1}\right)^{2}$. 
Находим знак выражения $\left( \pi_0^{h_1+2} \pi_0^{h_1}\right)^{-1} \hm- \left( 
\pi_0^{h_1+1}\right)^{-2}$, противоположный знаку выражения $\pi_0^{h_1+2} 
\pi_0^{h_1} \hm- \left( \pi_0^{h_1+1}\right)^2$. Проведя упро\-ща\-ющие 
преобразования, находим:
  \begin{multline*}
  \left( \pi_0^{h_1+2} \pi_0^{h_1}\right)^{-1} -\left( \pi_0^{h_1+1}\right)^{-2} 
={}\\
  {}= \left( \sum\limits_{j=0}^s \fr{\rho^j}{(j-1)!s} -\sum\limits_{j=0}^s 
\fr{\rho^j}{j!} \right) \left( E_{h_1+1}- E_{h_1}\right),
  \end{multline*}
где 
\begin{multline*}
E_{h_1}= \fr{\rho^s}{s!} \sum\limits_{j=1}^{h_1-s-1} \left(\fr{\rho}{s}\right)^j +{}\\
{}+ \fr{\rho^s}{s!}  
\left(\fr{\rho}{s}\right)^{h_1-s-1} \sum\limits_{j=1}^{h_2-h_1-1} \left(\fr{\rho}{s+\alpha}\right)^j.
\end{multline*}
%
Как видим, первая скобка в~правой час\-ти имеет отрицательный знак. Для 
второй скоб\-ки имеем:
\begin{multline*}
E_{h_1+1} -E_{h_1} = \fr{\rho^{s+1}}{s!s}\left( \fr{\rho}{s}\right)^{h_1-s-1}\times{}\\
{}\times \left[
\left(\fr{\rho}{s}\right)^{h_2-h_1-1} +\fr{\rho/s+\rho/(s+\alpha)}{1+\rho/s}\right] 
>0\,.
\end{multline*}
Следовательно, $A_{h_1+2}\hm- A_{h_1+1}\hm>0$, так как $\left( 
\pi_0^{h_1+2} \pi_0^{h_1}\right)^{-1} \hm- \left( \pi_0^{h_1+1}
\right)^{-2}\hm<0$ (или $\pi_0^{h_1+2} \pi_0^{h_1} \hm- 
(\pi_0^{h_1+1})^2\hm>0$), т.\,е.\ $A_{h_1+1}$ возрастает по~$h_1$.
   
   Из этого свойства~$A_{h_1+1}$ и~из того, что сред\-нее время отсутствия 
очереди также убывает по~$h_1$, следует, что $f(h_1,a)$ убывает по~$h_1$. 
Следовательно, функция $D(h_1,a)$ удовлетворяет всем условиям тео\-ре\-мы~1 
из~\cite{8-al}. Далее, рас\-суж\-дая точ\-но так же, как и~в~работе~\cite{7-al}, 
находим, что условия тео\-ре\-мы~1 из~\cite{8-al} выполняются и~в случае 
бесконечного накопителя и~функция $D(h_1)$ унимодальна по пороговому 
значению $h_1\hm>s$.
    
  В заключение отметим также сле\-ду\-ющий результат, вытекающий из свойства 
уни\-мо\-даль\-ности функции $D(h_1)$: если существует конечный оптимальный 
порог, то для его поиска (решения исходной задачи) достаточно применить 
сле\-ду\-ющий прос\-той алгоритм.
  \begin{enumerate}[1.]
\item Положить $h_1\hm= s\hm+1$. 
\item До тех пор пока выполняется условие 
$$
D\left(h_1+1\right)> D\left(h_1\right),
$$ 
полагать $h_1 \hm= h_1\hm+1$.
\item Положить $h_1^*\hm=h_1$. 
\end{enumerate}

{\small\frenchspacing
 {%\baselineskip=10.8pt
 %\addcontentsline{toc}{section}{References}
 \begin{thebibliography}{9}
\bibitem{1-al}
\Au{Горцев А.\,М.} Система массового обслуживания с~произвольным чис\-лом резервных 
каналов и~гистерезисным управ\-ле\-ни\-ем вклю\-че\-ни\-ем и~вы\-клю\-че\-ни\-ем резервных 
каналов~// Автоматика и~телемеханика, 1977. Вып.~10. С.~30--37.
\bibitem{2-al}
\Au{Крылова Д.\,С., Головко~Н.\,И., Жук~Т.\,А.} Анализ СМО с~резервным прибором 
и~скачкообразной интенсивностью вход\-но\-го потока~// Вестник ВГУ. Сер. Физика. 
Математика, 2017. №\,4. С.~109--123.
\bibitem{3-al}
\Au{Клименок~В.\,И.} Многолинейная сис\-те\-ма массового обслуживания с~резервными 
приборами~// Ж.~Белорусского государственного университета. Математика. 
Информатика, 2019. №\,3. С.~57--70.
\bibitem{4-al}
\Au{Дудин А.\,Н.} О~задаче оптимального управ\-ле\-ния многоскоростной сис\-те\-мой 
массового обслуживания~// Автоматика и~телемеханика, 1980. Вып.~9. С.~43--51 
\bibitem{5-al}
\Au{Самочернова Е.\,С., Петров~Л.\,И.} Оптимизация сис\-те\-мы массового обслуживания 
с~однотипным резервным прибором~// Известия Томского политехнического университета, 
2010. Т.~317. №\,5. С.~28--31.
\bibitem{6-al}
\Au{Агаларов Я.\,М.} Оптимизация порогового управ\-ле\-ния переключением скорости 
обслуживания в~сис\-те\-ме массового обслуживания $G/M/1$~// Информатика и~её 
применения, 2022. Т.~16. Вып.~1. С.~73--81.
\bibitem{7-al}
\Au{ Агаларов Я.\,М.} Оптимальное управ\-ле\-ние подключением резервного прибора  
в~сис\-те\-ме массового обслуживания $G/M/1$~// Информатика и~её применения, 2022. 
Т.~16. Вып.~4. С.~34--41.
\bibitem{8-al}
\Au{Агаларов Я.\,М.} Признак унимодальности це\-ло\-чис\-лен\-ной функции одной переменной~// 
Обозрение при\-клад\-ной и~промышленной математики, 2019. Т.~26. Вып.~1. С.~65--66.
\end{thebibliography}

 }
 }

\end{multicols}

\vspace*{-9pt}

\hfill{\small\textit{Поступила в~редакцию 05.09.22}}

\vspace*{8pt}

%\pagebreak

%\newpage

%\vspace*{-28pt}

\hrule

\vspace*{2pt}

\hrule

%\vspace*{-2pt}

\def\tit{OPTIMIZATION OF A~QUEUE-LENGTH DEPENDENT ADDITIONAL SERVER 
IN~THE~MULTISERVER QUEUE}


\def\titkol{Optimization of a~queue-length dependent additional server 
in~the~multiserver queue}


\def\aut{Ya.\,M.~Agalarov}

\def\autkol{Ya.\,M.~Agalarov}

\titel{\tit}{\aut}{\autkol}{\titkol}

\vspace*{-15pt}


\noindent
Federal Research Center ``Computer Science and Control'' of the Russian Academy 
of Sciences, 44-2~Vavilov Str., Moscow 119333, Russian Federation


\def\leftfootline{\small{\textbf{\thepage}
\hfill INFORMATIKA I EE PRIMENENIYA~--- INFORMATICS AND
APPLICATIONS\ \ \ 2023\ \ \ volume~17\ \ \ issue\ 1}
}%
 \def\rightfootline{\small{INFORMATIKA I EE PRIMENENIYA~---
INFORMATICS AND APPLICATIONS\ \ \ 2023\ \ \ volume~17\ \ \ issue\ 1
\hfill \textbf{\thepage}}}

\vspace*{3pt}
  
  
  
  \Abste{The problem of optimal control of an additional server in a~stationary $G/M/s$ queue is 
considered. The additional server can be turned on and off at instants when the queue length is 
changed. It is formulated as the nonlinear optimization problem, in which the objective function 
accounts for amounts for service, losses due to the waiting of customers, maintenance, and 
downtime of the additional server. The functioning of the system is described as a~controlled 
Markov chain. Only stationary control policies are considered. For Poisson arrivals, necessary and 
sufficient conditions are given for the existence of the optimal decision point (threshold) and it is 
proved that the objective function is unimodal. A~simple algorithm for the computation of the 
threshold is provided.}
  
  \KWE{multiserver queuing system; optimization; additional server}
  
 \DOI{10.14357/19922264230112} 

%\vspace*{-16pt}

%\Ack
%\noindent

  

%\vspace*{4pt}

  \begin{multicols}{2}

\renewcommand{\bibname}{\protect\rmfamily References}
%\renewcommand{\bibname}{\large\protect\rm References}

{\small\frenchspacing
 {%\baselineskip=10.8pt
 \addcontentsline{toc}{section}{References}
 \begin{thebibliography}{9} 
\bibitem{1-al-1}
  \Aue{Gortsev, A.\,M.} 1978. A~queueing system with an arbitrary number of stand-by channels 
and hysteresis control of their connection and disconnection. \textit{Automat. Rem. Contr.} 
38(10):1451--1457.
\bibitem{2-al-1}
  \Aue{Krylova, D.\,S., N.\,I.~Golovko, and T.\,A.~Zhuk.} 2017. Ana\-liz SMO s~re\-zerv\-nym 
pri\-bo\-rom i~skach\-ko\-ob\-raz\-noy in\-ten\-siv\-nost'yu vkhod\-no\-go po\-to\-ka [Analysis of SMO
 backup device and the abrupt intensity of the input stream]. \textit{Vestnik VGU. Ser. Fizika. 
Matematika} [Proceedings of Voronezh State University. Ser. Physics. Mathematics] 4:109--123.
\bibitem{3-al-1}
  \Aue{Klimenok, V.\,I.} 2019. Mno\-go\-li\-ney\-naya sis\-te\-ma mas\-so\-vo\-go ob\-slu\-zhi\-van\-iya s~re\-zerv\-ny\-mi 
pri\-bo\-ra\-mi [Multi-server queueing system with reserve servers]. \textit{J.~Belarusian State 
University. Mathematics Informatics} 3:57--70.
\bibitem{4-al-1}
  \Aue{Dudin, A.\,N.} 1981. On optimal control of a~multi-rate service system. \textit{Automat. 
Rem. Contr.} 41(9):1221--1228.
\bibitem{5-al-1}
  \Aue{Samochernova, E.\,S., and L.\,I.~Petrov.} 2010. Op\-ti\-mi\-za\-tsiya sis\-te\-my mas\-so\-vo\-go 
ob\-slu\-zhi\-van\-iya s~od\-no\-tip\-nym re\-zerv\-nym pri\-bo\-rom [Optimization of the queuing system with the 
same type of backup device]. \textit{Bulletin Tomsk Polytechnic University} 317(5):28--31.
\bibitem{6-al-1}
  \Aue{Agalarov, Ya.\,M.} 2022. Op\-ti\-mi\-za\-tsiya po\-ro\-go\-vo\-go uprav\-le\-niya pe\-reklyu\-che\-niem 
sko\-rosti ob\-slu\-zhi\-va\-niya v~sis\-te\-me mas\-so\-vo\-go ob\-slu\-zhi\-va\-niya $G/M/1$ [Optimization of the 
threshold service speed control in the $G/M/1$ queue]. \textit{Informatika i~ee Primeneniya~--- 
Inform. Appl.} 16(1):\linebreak 73--81.
\bibitem{7-al-1}
  \Aue{Agalarov Ya.\,M.} 2022. Op\-ti\-mal'\-noe uprav\-le\-nie pod\-klyu\-che\-niem re\-zerv\-no\-go pri\-bo\-ra 
v~sis\-te\-me mas\-so\-vo\-go ob\-slu\-zhi\-va\-niya $G/M/1$ [Optimal control of a~queue-length dependent 
additional server in $\mathrm{GI}/M/1$ queue]. \textit{Informatika i~ee Primeneniya~--- Inform. Appl.} 
16(4):34--41.
\bibitem{8-al-1}
  \Aue{Agalarov, Ya.\,M.} 2019. Pri\-znak uni\-mo\-dal'\-nosti tse\-lo\-chis\-len\-noy funk\-tsii od\-noy 
pe\-re\-men\-noy [A~sign of unimodality of an integer function of one variable]. \textit{Obozrenie 
prikladnoy i~promyshlennoy matematiki} [Surveys on Applied and Industrial Mathematics]  
26(1):65--66.
 \end{thebibliography}

 }
 }

\end{multicols}

\vspace*{-6pt}

\hfill{\small\textit{Received September 5, 2022}} 
  
  \Contrl
  
  \noindent
  \textbf{Agalarov Yaver M.} (b.\ 1952)~--- Candidate of Science (PhD) in technology, associate 
professor, leading scientist, Institute of Informatics Problems, Federal Research Center ``Computer 
Science and Control'' of the Russian Academy of Sciences, 44-2~Vavilov Str., Moscow 119333, 
Russian Federation; \mbox{agglar@yandex.ru}
  
  

   
\label{end\stat}

\renewcommand{\bibname}{\protect\rm Литература} 
   