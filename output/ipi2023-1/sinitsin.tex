\def\tbxmdl{\left(t,\bar X_t, m^l, D^l\right)}
\def\tbxmdu{\left(t,\bar X_t, m^U, D^U\right)}


\def\stat{sinits}

\def\tit{АНАЛИТИЧЕСКОЕ МОДЕЛИРОВАНИЕ РАСПРЕДЕЛЕНИЙ
С~ИНВАРИАНТНОЙ МЕРОЙ В~СТОХАСТИЧЕСКИХ СИСТЕМАХ,
НЕ~РАЗРЕШЕННЫХ ОТНОСИТЕЛЬНО ПРОИЗВОДНЫХ}

\def\titkol{Аналитическое моделирование распределений
с~инвариантной мерой в~СтСНРОП}

\def\aut{И.\,Н.~Синицын$^1$}

\def\autkol{И.\,Н.~Синицын}

\titel{\tit}{\aut}{\autkol}{\titkol}

\index{Синицын И.\,Н.}
\index{Sinitsyn I.\,N.}


%{\renewcommand{\thefootnote}{\fnsymbol{footnote}} \footnotetext[1]
%{Работа выполнена при финансовой поддержке РАН (проект АААА-А19-119091990037-5).}}


\renewcommand{\thefootnote}{\arabic{footnote}}
\footnotetext[1]{Федеральный исследовательский центр <<Информатика и~управ\-ле\-ние>> Российской академии наук; 
Московский авиационный институт, \mbox{sinitsin@dol.ru}} %\mbox{kafedra802@yandex.ru}}

%\vspace*{-6pt}


\Abst{Рассматриваются точные и~приближенные методы аналитического моделирования гауссовских 
и~негауссовских стационарных и~нестационарных стохастических процессов (СтП) с~инвариантной мерой в~стохастических сис\-те\-мах, 
не~разрешенных относительно производных (СтСНРОП). Для скалярных и~векторных СтСНРОП, допускающих линейную регрессионную 
аппроксимацию нелинейных функций, содержащих старшие производные, разработаны методы сведения уравнений СтСНРОП 
к~уравнениям дифференциальных стохастических сис\-тем (СтС). Предложены два точных метода аналитического моделирования СтС с~инвариантной мерой. 
Разработаны приближенные методы аналитического моделирования, основанные на параметризации одно- и~многомерных распределений. 
Особое внимание уделено гауссовским СтП с~инвариантной мерой. В~качестве примера рассмотрен нелинейный осциллятор Дуффинга, в~котором 
вместо второй производной присутствует нелинейная функция от второй производной. Приведены уравнения нелинейной корреляционной теории.
 Изучены стационарные СтП, совпадающие с~СтП с~инвариантной мерой. 
Обсуждаются полученные результаты и~формулируются направления дальнейших исследований в~об\-ласти аналитического моделирования СтСНРОП.}

\KW{аналитическое моделирование; параметризация распределений; распределение с~инвариантной мерой;
стохастическая система (СтС); стохастическая система, не разрешенная относительно производной (СтСНРОП);
стохастический процесс (СтП)}

\DOI{10.14357/19922264230101} 
  
\vspace*{2pt}


\vskip 10pt plus 9pt minus 6pt

\thispagestyle{headings}

\begin{multicols}{2}

\label{st\stat}


\section{Введение}


В~[1] дан обзор точных методов аналитического моделирования одно- и~многомерных распределений стационарных и~нестационарных процессов, 
основанных на построении интегральных инвариантов специально подобранных обыкновенных дифференциальных уравнений. Особое внимание в~[1] 
уделено случаю СтС с~автокоррелированными шумами. В~[2] на основе теории потенциала предложены методы расчета 
распределений с~инвариантной мерой для гамильтоновых СтС. В~[3] дано обобщение~[1, 2] на случай разрывных характеристик в~дифференциальных СтС. 
Дальнейшее развитие точных методов аналитического моделирования распределений в~дифференциальных СтС, в~том числе  и~в~условиях 
автокоррелированных шумов, представлено в~[4].

Вопросы аналитического моделирования СтП в~СтСНРОП рас\-смот\-ре\-ны в~[5--8]. 
Особое внимание в~них уделено нормальным (гауссовским) СтП. В~\cite{9-sin} предложены методы нормализации сис\-тем, стохастически не разрешенных 
относительно производных.

Рассмотрим обобщение~[1--8] на случай, когда СтСНРОП приводятся к~уравнениям дифференциальной СтС с~гауссовскими и~негауссовскими шумами.
%
Представлены точные и~приближенные методы аналитического моделирования гауссовских и~негауссовских стационарных и~нестационарных СтП с~инвариантной
 мерой в~СтСНРОП, до\-пус\-ка\-ющих линейную регрессионную аппроксимацию нелинейных функций, содержащих старшие производные. Разработаны методы 
 сведения уравнений СтСНРОП к~уравнениям дифференциальных СтС. %\linebreak 
 
 Предложены два точных метода аналитического %\linebreak
  моделирования СтС с~инвариантной мерой. 
 Разработаны приближенные методы аналитического моделирования, основанные на параметризации одно- и~многомерных распределений. Особое внимание уделено
  гауссовским СтП с~инвариантной мерой. 
  
  В~качестве примера рассмотрен нелинейный осциллятор Дуффинга, в~котором вместо второй производной присутствует 
  нелинейная функция от второй производной. Приведены уравнения нелинейной корреляционной теории. Изучены стационарные СтП, совпадающие с~СтП с~инвариантной мерой. 
  В~заключении статьи обсуждаются полученные результаты и~формулируются направления дальнейших исследований в~об\-ласти аналитического моделирования СтСНРОП.


\section{Приведение стохастических сис\-те\-м, 
не~разрешенных относительно производных, к~дифференциальным стохастическим системам}


Рассмотрим сначала скалярную СтСНРОП следующего вида [8]:
\begin{align}
\varphi&= \varphi \left(t, X_t, p X_t\tr p^{(l)} X_t, U_t\right)=0\,;\label{e2.1-s}
\\
dU_t &= a^U \left(U_t,t\right) dt + b^U \left(U_t,t\right) dW_0 + {}\notag\\
&\hspace*{17mm}{}+
\iii_{R_0^q} c^U \left(U_t,t,v\right) dP^0 (t,dt).
\label{e2.2-s}
\end{align}
Здесь $p=d/dt$~--- оператор дифференцирования по времени; $X_t \hm= X(t)$ и~$p^{(l)} X_t$~--- 
скалярные СтП, понимаемые в~среднеквадратичном смысле; $U_t\hm= U(t)$~--- скалярный СтП, определяемый уравнением Ито~(\ref{e2.2-s}); 
$\varphi$~--- нелинейная, в~общем случае разрывная, функция отмеченных переменных, \mbox{допускающая}
 линейную регрессионную аппроксимацию относительно 
старшей $l$-й производной $U_t$ вида
    \begin{equation}
    \varphi \approx \varphi_0+ k_l^\varphi p^{(l)} X_t + k_U^\varphi U_t\,.\label{e2.3-s}
    \end{equation}
Здесь введены следующие обозначения:
\begin{align*}
\varphi_0 &=\varphi_0 \tbxmdl; %\label{e2.4-s}
\\
k_l^\varphi &= k_l^\varphi\tbxmdl; %\label{e2.5-s}
\\
k_U^\varphi&=k_U^\varphi \tbxmdu, %\label{e2.6-s}
\end{align*}
где 
\begin{gather*}
{\bar X}_t =\lk  X_t^{\mathrm{T}} \, X_t^{(l-1)} \rk^{\mathrm{T}};\\
m^l ={\sf M} X_t^{(l)};\enskip D^l = {\sf M} \lv X_t^{(l)} -m^l\rrv^2.
\end{gather*}
В~уравнении~(\ref{e2.2-s}) принято: $W_0 \hm= W_0(t)$~--- винеровский скалярный СтП интенсивности $\nu_0 \hm=\nu_0(t)$; $c(u,t,v)$~--- 
скалярная функция~$u$ и~$t$, а~также вспомогательной переменной~$v$; $\int\nolimits_R dP^0 (t,A) \,dt$~--- 
цент\-ри\-ро\-ван\-ная пуассоновская мера, удовлетворяющая условию:
    $$
    \iii_\Delta dP^0 (t,A)= \iii_\Delta dP(t,A) \,dt- \iii_\delta \nu_P (t,A) \,dt\,,
    $$
где $\int\nolimits_\Delta dP(t,A) \,dt$~--- число скачков пуассоновского СтП $P(t,A)$ в~интервале времени~$\Delta$; $\nu_P(t,A)$~--- 
интенсивность пуассоновского СтП $P(t,A)$; $A\hm=R_0$~--- 
прямая с~выколотым началом. Интеграл в~(\ref{e2.2-s}) в~общем случае распространяется на~$R_0$. Начальное значение $U(t_0) \hm= U_0$ 
пред\-став\-ля\-ет собой случайную величину (СВ), не зависящую от приращений СтП  $W_0(t)$ и~$P(t,A)$ на интервалах времени  $\Delta \hm= (t_1, t_2]$, 
сле\-ду\-ющих за~$t_0$, $t_0\hm\le t_1\hm\le t_2$ для любого~$A$.

Из условия конечности дисперсий скалярных переменных~$p^{(l)} X_t$ и~$U_t$ следует соотношение
  \begin{equation*}
  {\sf D}\lk p^{(l)} X_t\rk \sim {\sf D}U_t,\enskip k_l^\varphi \ne 0\,.
  %\label{e2.8-s}
  \end{equation*}
  
  \noindent
  \textbf{Теорема~2.1.} \textit{Пусть СтСНРОП}~(\ref{e2.1-s}) 
  \textit{допускает линейную регрессионную линеаризацию согласно}~(\ref{e2.3-s}). \textit{Тогда уравнение}~(\ref{e2.1-s}) 
  \textit{приводится к~виду}:
   \begin{multline*}
   pX_{h-1} = X_h \enskip \left(h=\overline{1,l-1}\right),\\
   pX_l =-\varphi_0 \left(k_l^\varphi\right)^{-1} - k_U^\varphi 
   \left(k_l^\varphi\right)^{-1} U_t\,.
   %\label{e2.9-s}
   \end{multline*}


В векторном виде имеем следующие уравнения:
    \begin{align}
    \bar\varphi &=\bar\varphi (t, \bar X_t,\dot{\bar X}_t,\bar U_t) =0\,;\label{e2.10-s}\\
    d\bar U_t &= \bar a^U \left(t,\bar U_t\right) dt + \bar b^U \left(t, \bar U_t\right) dW_0 +{}\notag\\
    &\hspace*{18mm}{}+ \iii_{R_0^q} c^U \left(t, U_t, v\right) d P_0 (t, dt);\label{e2.11-s}\\
    \bar \varphi &\approx \bar\varphi_0 + k_{\dot{\bar X}}^{\bar\varphi} \dot{\bar X}_t + k_U^{\bar \varphi} \bar U_t\,;\label{e2.12-s}\\
    \dot{\bar X}_t &\approx a^{\bar X} =- \bar\varphi_0 \left(k_{\dot{\bar X}}^{\bar\varphi}\right)^{-1} - 
    k_0^{\bar\varphi} U_t \left(k_{\dot{\bar X}}^{\bar\varphi}\right)^{-1}.\notag %\label{e2.13-s}
    \end{align}
Здесь матричные коэффициенты регрессионной линеаризации удовлетворяют условиям:
\begin{alignat*}{2}
k_{\dot{\bar X}}^{\bar\varphi}&= k_{\dot{\bar X}}^{\bar\varphi}\left(t,\bar X_t, m^{\dot{\bar X}}, K^{\dot{\bar X}}\right), & \enskip \det k_{\dot{\bar X}}^{\bar\varphi} 
&\ne 0\,; %\label{e2.14-s}
\\
    K_U^{\bar\varphi} &= k_U^{\bar\varphi} \left(t,\bar X_t, m^U, K^U\right), & \enskip \det K_U^{\bar\varphi} &\ne 0\,.%\label{e2.15-s}
    \end{alignat*}
В уравнении~(\ref{e2.11-s}) принято: $U_t$~--- $n^{\bar U}$-мер\-ный СтП; $a^{\bar U}$, $b^{\bar U}$ и~$c^{\bar U}$~--- 
известные функции отмеченных переменных; $W_0(t)$~--- $n^{W_0}$-мер\-ный винеровский СтП мат\-рич\-ной ин\-тен\-сив\-ности $\nu_0\hm=\nu_0(t)$; 
$P(t,A)$~--- пуассоновский СтП ин\-тен\-сив\-ности~$\nu_P (t,A)$; $A\hm=R_0^{n^P}$~--- некоторое борелевское множество пространства с~выколотым началом.

\smallskip

\noindent
\textbf{Теорема~2.2.} \textit{Пусть векторная СтСНРОП}~(\ref{e2.10-s}) 
\textit{допускает линейную регрессию}~(\ref{e2.12-s}). \textit{Тогда векторное уравнение}~(\ref{e2.10-s}) 
\textit{приводится к~виду}~(\ref{e2.12-s}), \textit{а~уравнение для~$U_t$ в}~(\ref{e2.11-s}) 
\textit{должно допускать решение с~конечными моментами второго порядка}.


\section{Точные методы аналитического моделирования распределений стохастических процессов с~инвариантной мерой}

Введем расширенный вектор состояния приведенной СтСНРОП $Z_t \hm= \lk \bar X_t^{\mathrm{T}}\, U_t^{\mathrm{T}}\rk^{\mathrm{T}}$ 
и~векторное стохастическое дифференциальное уравнение Ито для него:
  \begin{multline}  
  dZ_t = \bar a \left(Z_t, t\right) dt + b\left(Z_t,t\right) W_0 +{}\\
  {}+ \iii_{R^0} c\left(Z_t,t,v\right) dP^0(t,dv).\label{e3.1-s}
  \end{multline}
Здесь
\begin{gather*}
    a=a(Z_t,t) =\begin{bmatrix}
    a^X\\ a^U
\end{bmatrix};\enskip
 b=b(Z_t,t) =\begin{bmatrix}
    0\\ \bar b^U
\end{bmatrix};\\[6pt]
     c=c\left(Z_t,t,v\right) =\begin{bmatrix}
    0\\ c^{\bar U}
\end{bmatrix}.
% \label{e3.2-s}
\end{gather*}


Пусть существуют одно- и~$n$-мер\-ные плотности $f_1\hm=f_1(z;t)$ и~$f_n\hm= f_n(z_1\tr z_n; t_1 \tr t_n)$ 
и~характеристические функции $g_1\hm=g_1(\lambda;t)$ и~$g_n\hm=g_n(\lambda_1\tr \lambda_n; t_1\tr t_n)$ $(n\ge 2)$, 
удовлетворяющие ин\-тег\-ро-диф\-фе\-рен\-ци\-аль\-ным уравнениям Пугачёва~\cite{10-sin}:
\begin{multline} %8
\fr{\prt f_1(z;t)}{\prt t}+\fr{\prt^{\mathrm{T}}}{\prt z}\lk a(z,t)f_1(z;t)\rk = {}\\
\!\!\! {}=
\fr{1}{(2 \pi)^p}\!\! \iin\iin \!\chi^f(\la,\zeta,t) e^{i\la^{\mathrm{T}}(\zeta-z)} f_1(z;t) \,d\zeta d\la\,,
\!\! %\label{e3.3-s}
\\
f_1\left(z;t_0\right)=f_0(z);
\label{e3.4-s}
\end{multline}


\vspace*{-12pt}

\noindent
\begin{multline} %9
 \fr{\prt f_n(z_1\tr z_n;t_1\tr t_n)}{\prt t_n}={}\\
 {}=\fr{\prt^{\mathrm{T}}}{\prt z_n}[a(z_n, t_n) f_n (z_1\tr z_n; t_1\tr t_n)]={}\\
{}= \fr{1}{(2\pi)^{pn}} \iin\iin \chi_n^f (\la_n, \zeta_n, t_n)\times{}\\
{}\times \exp\lf i \sss_{l=1}^n \la_l^{\mathrm{T}} \left(\zeta_l-z_l\right)\rf f_n 
\left(\zeta_1\tr \zeta_n; \right.\\
\left. t_1\tr t_n\right)d\zeta_1\cdots d\zeta_n d\la_1\cdots d\la_n,
\\
f_n(z_1\tr z_{n-1},z_n;t_1\tr t_{n-1},t_{n-1})={}\\
{}= f_{n-1} (z_1\tr z_{n-1};t_1\tr t_{n-1})\delta (z_n - z_{n-1}),\\
t_1\le t_2 \le \cdots \le t_n,\enskip n=2,3,\ldots;\label{e3.6-s}
\end{multline}

%\vspace*{-12pt}

\noindent
\begin{multline} %10
\fr{\prt g_1 (\la;t)}{\prt t} -
\fr{1}{(2\pi)^p} \iin \iin i\la^{\mathrm{T}} a (z,t) e^{i(\la^{\mathrm{T}} -\mu^{\mathrm{T}})z} \times{}\\[-2pt]
{}\times g_1 (\mu;t) d\mu dz=
\fr{1}{(2\pi)^k} \iin \iin \chi^g(\la, z,t)\times{}\\[-1pt]
{}\times  e^{i(\la^{\mathrm{T}} -\mu^{\mathrm{T}})z} g_1 (\mu;t) d\mu dz\,, %\label{e3.7-s}
\\
g_1(\la;t_0) = g_0(\la);\label{e3.8-s}
\end{multline}

\vspace*{-14pt}

\noindent
\begin{multline*}
 \fr{\prt g_n (\la_1\tr \la_n; t_1\tr t_n)}{\prt t_n} -{}\\[-2pt]
 {}-\fr{1}{(2\pi)^{pn}} \iin \!\cdots \!\iin i\la^{\mathrm{T}} a (z_n,t_n) \times{}\\
{}\times \exp \lk i \sss_{k=1}^n (\la_k^{\mathrm{T}} - \mu_k^{\mathrm{T}}) z_k\rk g_n \left(\mu_1\tr \mu_n;\right.\\
\left. t_1\tr t_n\right) d\mu_1 \cdots d \mu_n dz_1\cdots dz_n={}\\
{}=\fr{1}{(2\pi)^{pn}} \iin\!\cdots\! \iin \chi^n (\la_n, z_n,t_n)\times{}\\[-2pt]
{}\times\exp \lk i \sss_{k=1}^n \left(\la_k^{\mathrm{T}} - \mu_k^{\mathrm{T}}\right) z_k\rk g_n 
\left(\mu_1\tr \mu_n;\right.\\
\left. t_1\tr t_n\right) d\mu_1 \cdots d \mu_n dz_1\cdots dz_n\,, %\label{e3.9-s}
\\
g_n \left(\la_1\tr \la_n; t_1\tr t_{n-1},t_{n-1}\right)={}\\
{}= g_{n-1} \left(\la_1\tr \la_{n-2},\la_{n-1}+\la_n; t_1\tr t_{n-1}\right)\\
t_1\le t_2 \le\cdots \le t_n, \enskip n=2,3,\ldots
%\label{e3.10-s}
\end{multline*}

\vspace*{-1pt}

\noindent
Здесь приняты следующие обозначения:

\vspace*{-9pt}

    \begin{multline*}
    \chi^f (\la,\zeta, t) =-\fr{1}{2}\, \la^{\mathrm{T}} b(\zeta,t) \nu_0(t) b(\zeta,t)^{\mathrm{T}} +{}\\[-1pt]
{}+\! \iii_{R_0^q}\! \lf \exp \lk i\la^{\mathrm{T}} c(\zeta,t,v) \rk -1 -i\la^{\mathrm{T}} c(\zeta, t,v)\rf \nu_P (t,dv);\hspace*{-4.47359pt}
%\label{e3.11-s}
\end{multline*}

\vspace*{-14pt}

\noindent
\begin{multline*}
\chi^f_n \left(\la_n,\zeta_n, t_n\right) =-\fr{1}{2} \la^{\mathrm{T}}_n b(\zeta_n,t) \nu_0(t) b(\zeta_n,t)^{\mathrm{T}} +{}\\[-1pt]
{}+ \iii_{R_0^q} \left\{ \exp \lk i\la^{\mathrm{T}}_n c\left(\zeta_n,t_n,v\right) \rk -{}\right.\\
\left.{}-1 -i\la^{\mathrm{T}}_n 
c\left(\zeta_n, t_n,v\right)\right\} \nu_P (t_n,dv);
%\label{e3.12-s}
\end{multline*}

\vspace*{-14pt}

\noindent
\begin{multline*}
\chi^g (\la,z, t) =-\fr{1}{2}\, \la^{\mathrm{T}} b(z,t) \nu_0(t) b(z,t)^{\mathrm{T}} +{}\\[-1pt]
{}+ \!\iii_{R_0^q}\! \lf \exp \lk i\la^{\mathrm{T}} c(z,t,v) \rk -1 -i\la^{\mathrm{T}} c(z, t,v)\rf \nu_P (t,dv);\hspace*{-4.42848pt}
%\label{e3.13-s}
\end{multline*}

\vspace*{-14pt}

\noindent
\begin{multline*}
\chi^g_n \left(\la_n,z_n, t_n\right) =-\fr{1}{2}\, \la^{\mathrm{T}}_n b(z_n,t) \nu_0(t) b(z_n,t)^{\mathrm{T}} +{}\\[1pt]
{}+ \iii_{R_0^q} \left\{ \exp \lk i\la^{\mathrm{T}}_n c\left(z_n,t_n,v\right) \rk -{}\right.\\[-1pt]
\left.{}-1 -i\la^{\mathrm{T}}_n 
c\left(z_n, t_n,v\right)\right\} \nu_P (t_n,dv).
%\label{e3.14-s}
\end{multline*}

  %\vspace*{-2pt}
  
  \noindent
При этом одно- и~$n$-мер\-ные плотности и~характеристические функции связаны между собой известными соотношениями~\cite{10-sin}.

Для нахождения одномерных плотностей $f_1(z,t) \hm= f_1^* (z)$ и~характеристических функций $g_1(\la;t)\hm = g_1^* (\la)$ СтП 
в~стационарных системах~(\ref{e3.1-s}), когда

  \vspace*{-6pt}
  
  \noindent
  \begin{gather*}
  a(z,t) = a^*(z);\quad b(z,t)=b^*(z);\\
  \chi(\mu;t)= \chi^f(\mu,\zeta, t)={\chi*}^f (\mu, \zeta), %\label{e3.15-s}
  \end{gather*}
  
    \vspace*{-2pt}
    
    \noindent
в~(\ref{e3.4-s}) и~(\ref{e3.6-s}) следует положить 
$$
\fr{\prt f_1}{\prt t}= 0; \quad \fr{\prt g_1}{\prt t}=0\,.
$$

  \vspace*{-2pt}

Пусть функция $a$ в~системе~(\ref{e3.1-s}) допускает представление
      \begin{equation}
      a= a(z,t) = a_1(z,t) +a_2 (z,t), \label{e3.16-s}
      \end{equation}
      
        \vspace*{-2pt}
        
        \noindent
при котором функция  $f_1\hm=f_1(z;t)$ является плот\-ностью инвариантной меры, не возмущенной шумами системы, 
описываемой векторным обыкновенным дифференциальным уравнением вида
    \begin{equation}
    \dot z = a_1 (z,t),\label{e3.17-s}
    \end{equation}
т.\,е.\ удовлетворяет следующему условию сохранения инвариантной меры:
  \begin{equation}
  \fr{\prt f_1 (z;t)}{\prt t}+ \fr{\prt^{\mathrm{T}}}{\prt z} \left[ a_1 (z,t) f_1(z;t)\right] =0.\label{e3.18-s}
  \end{equation}
  
  \vspace*{-2pt}

Для гладких функций $a_1\hm=a_1(z,t)$ вопросы существования и~основные свойства интегральных инвариантов и~инвариантных мер изучены 
в~\cite{1-sin, 4-sin}. При этом  функция $a_2 \hm= a_2(z,t)$ в~(\ref{e3.16-s}) определяется путем решения следующего 
ин\-тег\-ро-диф\-фе\-рен\-ци\-аль\-но\-го уравнения:

\vspace*{-6pt}

\noindent
  \begin{multline}
  \fr{\prt^{\mathrm{T}}}{\prt z}\left[ a_2 (z,t) f_1(z;t) \right] ={}\\
 \!\!\! {}= \fr{1}{(2\pi)^k} \iin\iin \!\!\chi^f     (\la,\zeta,t) e^{i\la^{\mathrm{T}}(\zeta-z)} f_1(\zeta;t)\, d\zeta d\la\,.
  \!\label{e3.19-s}
  \end{multline}
  
  \vspace*{-2pt}

Условия сохранения инвариантной меры можно представить в~следующем развернутом виде~\cite{3-sin}:
    \begin{equation}
     \fr{\prt f_1 (z;t)}{\prt t} + A_a f_1 (z;t) =0\,,
     \label{e3.20-s}
     \end{equation}
     где
    
     \noindent
     \begin{equation*}
      A_a f_1(z;t)=\fr{\prt^{\mathrm{T}}}{\prt z} \lk a_1(z,t) f_1(z;t)\rk = \mathrm{div}\, \pi(z;t);
      \end{equation*}
      
     
      
      \noindent
           \begin{equation}
\fr{\prt g_1 (\la;t)}{\prt t} - B_a g_1(\la;t) =0\,,
\label{e3.21-s}
\end{equation}



\noindent
где

\vspace*{-9pt}

\noindent
\begin{multline*}
B_a g_1(\la;t) =\fr{1}{(2\pi)^p} \iin\iin \! i\la^{\mathrm{T}} a_1(z,t)\times{}\\
{}\times  e^{i(\la^{\mathrm{T}}-\mu^{\mathrm{T}})z} g_1(\mu;t)\, d\mu dz={}\\
{}=\!\!\! \iin \!\! i\la^{\mathrm{T}} a(z,t) e^{i\la^{\mathrm{T}}z} f_1(z;t)\, dz= 
\!\! \!\iin\!\! e^{i\la^{\mathrm{T}} z} i\la^{\mathrm{T}} \pi(z;t)\, dz.\hspace*{-6.03pt}
\end{multline*}

\vspace*{-2pt}

Для разрывных функций $a_1 (z,t)$ в~терминах характеристических функций соотношение~(\ref{e3.18-s}) может быть записано в~виде~(\ref{e3.21-s}). При этом для
со\-став\-ля\-ющих~$a_2(z,t)$  имеет место уравнение

\vspace*{-6pt}

\noindent
  \begin{multline}
  B_{a_2} g_1(\la;t) = \fr{1}{(2\pi)^p} \iin\iin \!\chi^g(\la,z,t) \times{}\\
  {}\times e^{i(\la^{\mathrm{T}}-\mu^{\mathrm{T}})z} g_1(\mu;t)\, d\mu dz\,.
  \label{e3.22-s}
  \end{multline}
  
  \vspace*{2pt}
  
  \noindent
Отсюда вытекают точные алгоритмы аналитического
моделирования распределений с~инвариантной мерой. В~их основе лежат
следующие две теоремы.

\vspace*{1pt}

\noindent
\textbf{Теорема 3.1.} \textit{Функция $f_1\hm=f_1(z;t)$ будет решением}~(\ref{e3.4-s}) \textit{тогда и~только тогда, когда $a\hm=a(z,t)$ допускает
представление}~(\ref{e3.16-s}), \textit{такое что $f_1\hm=f_1(z;t)$ является плот\-ностью
инвариантной меры обыкновенного дифференциального уравнения}~(\ref{e3.19-s}),
\textit{т.\,е.\ удовле\-тво\-ря\-ет условию}~(\ref{e3.18-s}). \textit{При этом со\-став\-ля\-ющая~$a_2$
определяется из решения ин\-тег\-ро-диф\-фе\-рен\-ци\-аль\-но\-го уравнения}~(\ref{e3.19-s}).

\vspace*{1pt}

\noindent
\textbf{Теорема 3.2.} \textit{Функция $g_1\hm=g_1(\la;t)$ будет решением}~(\ref{e3.8-s}) 
\textit{тогда и~только тогда, когда недифференцируемая функция
$a\hm=a(z,t)$  допускает пред\-став\-ле\-ние}~(\ref{e3.16-s}), \textit{такое что
$g_1\hm=g_1(\la;t)$ является \mbox{характеристической} функцией инвариантной
меры уравнения}~(\ref{e3.17-s}), \textit{т.\,е.\ удовлетворяет условию}~(\ref{e3.20-s}). 
\textit{При этом
со\-став\-ля\-ющая~$a_2$ определяется из уравнения}~(\ref{e3.22-s}).

\vspace*{-9pt}

\section{Приближенные методы аналитического моделирования распределений стохастических процессов с~инвариантной мерой}

\vspace*{-3pt}

Пусть нелинейная система~(\ref{e3.1-s}) допускает применение метода нормальной аппроксимации (МНА)~\cite{10-sin}. 
Тогда одно- и~двумерные нормальные плотности $f_1^{\mathrm{МНА}}$
и~$f_2^{\mathrm{МНА}}$ и~характеристические функции  $g_1^{\mathrm{МНА}}$ и~ $g_2^{\mathrm{МНА}}$, 
 а~также вектор математического ожидания $m_t \hm= {\sf M}^{\mathrm{МНА}} Z(t)$, ковариационная матрица $K_t\hm = {\sf M}^{\mathrm{МНА}} Z^{0\mathrm{T}} Z^0 (t)$ 
 $(Z^0 (t) \hm= Z(t) \hm- m_t)$ и~матрица ковариационных функций $K(t_1, t_2)\hm = {\sf M}^{\mathrm{МНА}} Z^{0\mathrm{T}} (t_1) Z^0 (t_2)$ $(t_1\hm< t_2)$ 
 определяются следующими уравнениями:
 \begin{multline} %18
 f_1^{\mathrm{МНА}} = f_1^{\mathrm{МНА}} \left(z;t, m_t, K_t\right) =
 \lk (2\pi)^p |K_t|\rk^{-1/2} \times{}\\
 {}\times \exp \lf -\fr{1}{2} (z^{\mathrm{T}} - 
    m_t^{\mathrm{T}}) K_t^{-1}(z-m_t)\rf;
    \label{e4.1-s}
    \end{multline}
    
    \vspace*{-12pt}
    
    \noindent
    \begin{multline} %19
    f_2^{\mathrm{МНА}} =
     f_2^{\mathrm{МНА}} \left(z_1, z_2;t_1, t_2, m_{t_1}, m_{t_2},\right.\\
     \left. K_{t_1}, K_{t_2}, K(t_1, t_2)\right)=
     \lk (2\pi)^p |\bar K_2|\rk^{-1/2}\times{}\\
\!     \!\!{}\times \exp \lf - \left([z_1^{\mathrm{T}} z_2^{\mathrm{T}}] - \bar m_2^{\mathrm{T}}\right) 
\bar K_2^{-1}([z_1^{\mathrm{T}} z_2^{\mathrm{T}}]^{\mathrm{T}}-\bar m_2)\rf;\!\!\label{e4.2-s}
\end{multline}

  %\vspace*{-12pt}
    
    \noindent
    \begin{equation} %20
    g_1^{\mathrm{МНА}}(\la;t)=\exp\lf i\la^{\mathrm{T}} m- \fr{1}{2} \la^{\mathrm{T}} K_t \la\rf;\label{e4.3-s}
    \end{equation}
    
    \vspace*{-12pt}
    
    \noindent
    \begin{multline} %21
g_2^{\mathrm{МНА}} \left(\la_1, \la_2; t_1,t_2\right) = {}\\
{}=\exp \lf i \bar \la^{\mathrm{T}} \bar m_2 - \fr{1}{2} \bar \la^{\mathrm{T}} \bar K_2 \bar \la\rf;
\label{e4.4-s}
\end{multline}


\noindent
    \begin{equation} %22
    \left.
    \begin{array}{c}
    \bar \la =\lk \la_1^{\mathrm{T}} \la_2^{\mathrm{T}}\rk^{\mathrm{T}};\enskip 
    \bar m_2 =\lk m_{t_1}^{\mathrm{T}} m_{t_2}^{\mathrm{T}}\rk^{\mathrm{T}};
\\[6pt]
     \bar K_2 = \begin{bmatrix}
        K\left(t_1, t_1\right)&K\left(t_1, t_2\right)\\[3pt]
        K\left(t_2, t_1\right)& K\left(t_2, t_2\right)
       \end{bmatrix};
      \\[6pt]
          \hspace*{-20mm}\dot m_t = \Phi_1 \left(t, m_t, K_t\right) ={}\\
  \displaystyle  {}+\iin a(z,t) f_1^{\mathrm{МНА}} \left(z; t, m_t, K_t\right) dz\,;
    \end{array}
    \right\}
    \label{e4.5-s}
    \end{equation}


\vspace*{-12pt}
    
    \noindent
    \begin{multline}
    \dot K_t = \Phi_2\left(t, m_t, K_t\right) = \Phi_{21} + \Phi_{12}+\Phi_{22}={}\\
{}=\left[ \iin a(z,t) \left(z^{\mathrm{T}}-m_t^{\mathrm{T}}\right) +
 \left(z-m_t\right) 
    a^{\mathrm{T}} (z,t) +{}\right.\\
\left.    {}+ \bar \sigma (z,t)
    \vphantom{\iin}
    \right] f_1^{\mathrm{МНА}} \left(z;t, m_t, K_t\right) dz\,;
    \label{e4.6-s}
    \end{multline}
    
    \vspace*{-12pt}
    
    \noindent
    \begin{multline}
\fr{\prt K(t_1, t_2)}{\prt t_2} = {}\\
{}=\Phi_3 \left(t_1, t_2, m_{t_1},m_{t_2}, K_{t_1}, K_{t_2}, K(t_1,t_2)\right)={}\\
{}=\lk (2\pi)^{2p} |\bar K_2|\rk^{-1/2}\iin\iin (z_1-m_{t_1}) a\left(z_2, t_2\right)\times{}\\
{}\times \exp\left\{ - 
    \left(\left[z_1^{\mathrm{T}} z_2^{\mathrm{T}}\right]-\bar m_2^{\mathrm{T}}\right)\bar K_2^{-1}\times{} \right.\\
\left.    {}\times
    \left(\left[z_1^{\mathrm{T}} z_2^{\mathrm{T}}\right]-\bar m_2\right)\right\} dz_1 dz_2={}\\
{}= K\left(t_1, t_2\right) K\left(t_2\right)^{-1} \Phi_{21} \left(m(t_2), K(t_2), t_2\right)^{\mathrm{T}}.\label{e4.7-s}
\end{multline}
Здесь введены следующие обозначения:

\noindent
\begin{gather*}
z_1=z_{t_1};\enskip z_2=z_{t_2};\enskip \bar m_2 =\lk m_{t_1}^{\mathrm{T}} m_{t_2}^{\mathrm{T}}\rk^{\mathrm{T}};\\
 \bar K_2 =\begin{bmatrix}
        K(t_1,t_1)&K(t_1, t_2)\\
        K(t_2, t_1)& K(t_2, t_2)\end{bmatrix};
               % \label{e4.8-s}
        \\
    \bar \sigma (z,t) =\sigma(z,t) +\iii_{R_0^q} c(z,t,v) c(z,t,v)^{\mathrm{T}} \nu_P (t,dv),\\
     \hspace*{30mm}\sigma(z,t) = b(z,t) \nu_0(t) b(z,t)^{\mathrm{T}}.\label{e4.9-s}
     \end{gather*}

Условия наличия нормального распределения с~инвариантной мерой, если заменить $a(z,t)$ статистически
линеаризованным выражением вида

\vspace*{-3pt}

\noindent
   \begin{multline*}
a(Z,t)\approx a_{10}^{\mathrm{МНА}} \left(t, m_t, K_t\right) + {}\\
{}+a_{11}^{\mathrm{МНА}}\left(t, m_t, K_t\right) \left(Z-m_t\right),
%\label{e4.10-s}
\end{multline*}


\noindent
где
  \begin{equation*}
  a_{10}^{\mathrm{МНА}} =a_{10}^{\mathrm{МНА}} \left(t, m_t, K_t\right);
  %\label{e4.11-s}
  \end{equation*}
  
  \vspace*{-12pt}
  
  \noindent
  \begin{multline*}
a_{11}^{\mathrm{МНА}}=a_{11}^{\mathrm{МНА}} \left(t, m_t, K_t\right) = \left[
 \iin \! a(z,t) \times{}\right.\\
\left. {}\times \left(z^{\mathrm{T}}-m_t^{\mathrm{T}}\right) f_1^{\mathrm{МНА}} \left(z; t , m_t, K_t\right) dz
\vphantom{\iin}\right] K_t^{-1} ={}\\
{}=\lk \fr{\prt}{\prt m_t}\left( a_{10}^{\mathrm{МНА}}\right)^{\mathrm{T}}\rk^{\mathrm{T}},
%\label{e4.12-s}
\end{multline*}



\noindent
приводят к~следующим соотношениям:

\vspace*{-3pt}

\noindent
  \begin{multline}
  \fr{\prt f_1^{\mathrm{МНА}} \left(z; t, m_t, K_t\right)}{\prt t} +
  \fr{\prt^{\mathrm{T}}}{\prt z} \left\{ \left[ a_{10}^{\mathrm{МНА}} \left(t, m_t, K_t\right) + {}\right.\right.\\
 \hspace*{-2pt}\left.\left. {}+a_{11}^{\mathrm{МНА}} 
\left(t, m_t, K_t\right) \left(z-m_t\right) \right] f_1^{\mathrm{МНА}} \left( z; t , m_t, K_t\right)\right\} ={}\\
{}=0\,;\label{e4.13-s}
\end{multline}

\vspace*{-12pt}

\begin{multline}
\fr{\prt g_1^{\mathrm{МНА}} (\la;t)}{\prt t} -\iin i\la^{\mathrm{T}} \left[ a_{10}^{\mathrm{МНА}} 
\left(m_t, K_t, t\right) +{}\right.\\
\left.{}+ a_{11}^{\mathrm{МНА}} \left(m_t, K_t, t\right) \left(z- m_t\right) \right]\times{}\\
{}\times e^{i\la^{\mathrm{T}} z} f_1^{\mathrm{МНА}} \left(z; m_t, K_t, t\right) dz=0\,;
\label{e4.14-s}
\end{multline}

\vspace*{-14pt}

\begin{multline*}
\iin i\la^{\mathrm{T}} \left[ a_{10}^{*{\mathrm{МНА}} } (m^*, K^*) +{}\right.\\
\left.{}+ a_{11}^{*{\mathrm{МНА}} } 
\left(m^*, K^*\right) \left(z-m^*\right)\right] \times{}\\
{}\times e^{i\la^{\mathrm{T}}z} f_1^{*{\mathrm{МНА}} } \left(z; m^*, K^*\right) dz =0\,.
%\label{e4.19-s}
\end{multline*}



\noindent
Отсюда вытекает следующая теорема.

\smallskip

\noindent
\textbf{Теорема 4.1.} \textit{Если существуют одно- и~двумерные  плот\-ности
СтП, а~матрица $a_{11}^{\mathrm{МНА}}$ коэффициентов
статистической линеаризации асимптотически устойчива,
то приближенный алгоритм \mbox{аналитического} моделирования МНА
нестационарных СтП в~сис\-те\-ме}~(\ref{e3.1-s}) \textit{с~инвариантной
мерой определяется выражениями}~(\ref{e4.1-s})--(\ref{e4.7-s}) \textit{и}~(\ref{e4.13-s}).

\smallskip

Как известно~\cite{10-sin}, одно- и~двумерные нормальные распределения
определяют и~все  $n$-мер\-ные распределения $(n> 3)$. Поэтому МНА и~МСЛ  при $b(Y,t)\hm=b_0(t)$ 
и~$c(Y,t,z) \hm=c_0 (t,v)$ дают приближенные алгоритмы для любых многомерных плотностей
СтП, если они существуют.
% Аналогично формулируются в~терминах характеристических функций на основе условий~(\ref{e4.14-s}).



Обобщением МНА являются различные
приближенные методы, основанные на параметри-\linebreak\vspace*{-12pt}

\pagebreak

\noindent
зации распределений~\cite{10-sin}.
Аппроксимируя одномерную характеристическую функцию~$g_1 (\la;t)$
и~\mbox{соответствующую} плот\-ность $f_1 (z,t)$ известными функциями
 $g_1^* (\la;\theta)$ и~$f_1^* (z;\theta)$,  зависящими от
конечномерного векторного па\-ра\-мет\-ра~$\theta$, можно свести задачу
приближенного определения одномерного распределения к~выводу из
уравнения для характеристических функций обыкновенных
дифференциальных уравнений, определяющих $\theta$ как функцию
времени. Это относится и~к остальным многомерным распределениям~\cite{6-sin}.

\section{Пример }


Рассмотрим скалярную СтСНРОП второго порядка следующего вида:
   \begin{align*}
   \varphi_1 (\ddot X) + \w_0^2 X - \mu X^3 &=-\delta \dot X + U\,;\\ %\label{e5.1-s}\\
\dot U &=-\alpha U + \beta V\,,
%\label{e5.2-s}
\end{align*}
где $\varphi_1$~--- нелинейная функция, допускающая регрессионную линеаризацию
 \begin{equation*}
 \varphi_1 (\ddot X) \approx  \varphi_{10} \left(m^{\ddot X}, D^{\ddot X}\right)+k_{\ddot X}^{\varphi_1}(m^{\ddot X}, D^{\ddot X})\ddot X\
% \label{e5.3-s}
 \end{equation*}
при условии
\begin{equation*}
\det k_{\ddot X}^\varphi \ne 0\,; %\label{e5.4-s}
\end{equation*}
$\w_0$, $\gamma$, $\delta$, $\alpha$ и~$\beta$~--- постоянные па\-ра\-мет\-ры;  $V$~--- белый нормальный шум интенсивности $\nu_0$. 
В~переменных $X_1 \hm=X$, $\dot X_1\hm = X_2$ и~$X_3 \hm=U$ развернутые уравнения эквивалентной дифференциальной СтС имеют вид:
    \begin{equation}
    \left.
    \begin{array}{rl}
    \dot X_1 &= X_2\,;\\
    \dot X_2 &= -\bar\varphi_{10}- \bar\w_0^2 X_1 + \bar\mu X_1^3 - 
    \bar\delta X_2 +\bar\gamma X_3\,; \\[6pt]
     \dot X_3& =-\alpha X_3 +\beta \,.
     \end{array}
     \right\}
    \label{e5.5-s}
    \end{equation}
Здесь введены обозначения:
   \begin{gather*}
    \bar\varphi_{10} =\bar\varphi_{10}\left(t, m^{\dot X_2}, D^{\dot X_2}\right)\,; \quad 
    \bar \w_0^2 = \bar\w_0^2\bar\gamma\,;\\
    \bar \mu =\mu\bar\gamma;\quad \bar\delta =\delta\bar\gamma\,;
    \\
\bar\gamma = \lk k_{\dot X_2}^{\varphi_1} \left(t, m^{\dot X_2}, D^{\dot X_2}\right)\rk^{-1}.
%\label{e5.6-s}
\end{gather*}
Далее выполним статистическую линеаризацию кубической нелинейности согласно~\cite{10-sin}:
\begin{multline}
X_1^3 = m_1 \left(m_1^2+ 3 D_1\right) + 3\left(m_1^2 + D_1\right) X_1^0 ={}\\
{}=-2m_1^3 +3 \left(m_1^2 + D_1\right) X_1.\label{e5.7-s}
\end{multline}
Тогда~(\ref{e5.5-s}) приводятся к~виду:
    \begin{equation}
    \left.
    \begin{array}{rl}
    \dot X_1 &= X_2\,;\\[6pt]
    \dot X_2 &=-\tilde\varphi_{10} -\bar\w_{\mathrm{э}}^2 X_1 -\bar\delta X_2 + \bar\gamma X_3\,;\\[6pt]
     \dot X_3 &=-\alpha X_3 +\beta V\,.
     \end{array}
     \right\}
     \label{e5.8-s}
     \end{equation}
Здесь дополнительно обозначено
  \begin{equation}
  \left.
  \begin{array}{rl}
  \tilde\varphi_{10} &=-\bar\varphi_{10} - 2 m_1^3;\\[6pt]
  \bar\w_{\mathrm{э}}^2 &= \w_0^2 \lk1-\frac{3(m_1^2+D_1)\mu}{\w^2}\rk\bar\gamma\,.
  \end{array}
  \right\}\label{e5.10-s}
  \end{equation}
В силу~(\ref{e5.8-s}) и~(\ref{e5.10-s}) для вычисления $m^{\ddot X}\hm =m^{X_2}$ и~$D^{\ddot X} \hm=D^{X_2}$ используются сле\-ду\-ющие формулы связи:
\begin{align*}
m^{\ddot X}&= m^{\dot X_2} =-\tilde\varphi_{10} -\w_{\mathrm{э}}^2 m_1 -\bar\delta m_2 + \bar\gamma m_3\,;\\
D^{\ddot X}&= D^{\dot X_2} =\mathrm{M} \lk \lv -\bar\w_{\mathrm{э}}^2 X_1^0 -\bar\delta X_2^0 + \bar\gamma X_3^0\rrv^2\rk ={}\\
&\hspace*{5mm}{}=\bar\w_{\mathrm{э}}^4 D_1 +\bar\delta^2D^2 + \bar\gamma^2 D_3 + 
2\bar\w_{\mathrm{э}}^2\bar\delta K_{12} -{}\\
&\hspace*{30mm}{}-\w_{\mathrm{э}}^2\bar\gamma K_{13} -2 \bar\delta \bar\gamma K_{23}. %\label{e5.11-s}
\end{align*}
Уравнения~(\ref{e5.8-s}) для математических ожиданий~$m_i$, дисперсий и~ковариаций~$K_{ij}$ $(i,j\hm=1,2,3)$ 
связаны между собой параметрически вследствие как нелинейности~$\varphi_1$, так и~кубической нелинейности и~имеют сле\-ду\-ющий вид:
 \begin{equation}
 \left.
 \begin{array}{rl}
 \dot m_1 &= m_2\,;\\[6pt]
  \dot m_2 &=-\tilde\varphi_{10} -\bar\w_{\mathrm{э}}^2 m_1 -\bar\delta m_2 +\bar\gamma m_3;\\[6pt]
 \dot m_3& =-\alpha m_3\,;
 \end{array}
 \right\}
 \label{e5.12-s}
 \end{equation}
 \begin{equation}
 \left.
 \begin{array}{rl}
    \dot X_1^0 &= X_2^0\,;\\[6pt]
     \dot X_2^0 &=-\dot\w_{\mathrm{э}}^2 X_1^0 -\bar\delta X_2^0 +\bar\gamma X_3^0;\\[6pt]
    \dot X_3^0 &=-\alpha X_3^0 +V\,.
    \end{array}
    \right\}\label{e5.13-s}
    \end{equation}

Из второго и~третьего уравнений~(\ref{e5.12-s}) 
для $\alpha\hm>0$ и~$\bar\delta \hm>0$ при $t\gg t_0$ получаем соотношения для стационарных математических ожиданий:
\begin{equation}
m_3^* =0\,;\enskip m_2^* =0\,;\enskip \w_{\mathrm{э}}^2m_1^* =-\tilde\varphi_{10}.
\label{e5.14-s}
\end{equation}

Уравнения для дисперсий $D_i$ и~ковариаций $K_{ij}$ в~силу~(\ref{e5.13-s}) можно представить в~виде:
\begin{equation*}
\dot K = AK + KA^{\mathrm{T}} + b\nu_0 b^{\mathrm{T}},
%\label{e5.15-s}
\end{equation*}
где
    \begin{equation}
    A=\begin{bmatrix}
    0&1&0\\
    -\w_{\mathrm{э}}^2&-\bar\delta&\bar\gamma\\
    0&0&-\alpha
\end{bmatrix};\quad b=\begin{bmatrix}
    0\\ 0\\ \beta
\end{bmatrix},
\label{e5.16-s}
\end{equation}
или в~развернутом виде:
    \begin{equation}
    \left.
    \begin{array}{rl}
    \dot D_1 &=2K_{12}\,;\\[6pt]
    \dot D_2 &=2\left(-\bar\w_{1\mathrm{э}}^2 K_{12} -\bar\delta D_2 +\bar\gamma_1 D_3\right);\\[6pt]
        \dot D_3 &=-2\alpha D_3 +\beta^2 \nu_0\,;\\[6pt]
    \dot K_{12} &= D_2 -\bar\w_{1\mathrm{э}}^2 D_1 -\bar\delta K_{12} +\bar\gamma_1 K_{13}\,;\\[6pt]
    \dot K_{13} &= K_{23} -\alpha K_{13}\,;\\[6pt]
    \dot K_{23} &= -\bar\w_{1\mathrm{э}}^2 K_{13} -\bar\delta K_{23} +\bar\gamma D_3\,.
    \end{array}
    \right\}
    \label{e5.17-s}
    \end{equation}
Здесь дополнительно введены обозначения:
\begin{equation*}
\w_{1\mathrm{э}}^2 =\w^2 \bar\gamma \lk 1 -\fr{3(m_1^2+D_1)}{\w^2}\rk;\enskip 
\bar \gamma_1 = k_U^{\varphi_1} \bar\gamma\,.
%\label{e5.18-s}
\end{equation*}

В стационарном режиме для асимптотически устойчивой матрицы~$A$ в~(\ref{e5.16-s}) в~уравнениях~(\ref{e5.17-s}) 
следует положить правые части равными нулю. Тогда получим искомые совместные уравнения для стационарных дисперсий и~ковариаций. 
Решив эти уравнения, придем к~следующим со\-от\-но\-ше\-ниям:
{\looseness=1

}

\vspace*{1pt}

 \begin{equation}
 \left.
 \begin{array}{c}
    D_2^* = \fr{\beta^2 \nu_0 \bar\gamma_1}{2\alpha\bar\delta^*}\,;\enskip  D_3^* =\fr{\beta^2\nu_0}{2\alpha}\,;\\[9pt]
    K_{12}^* =0\,;\enskip K_{13}^* =\fr{\beta^2 \nu_0 \bar\gamma_2^*}{2\alpha (\bar\w_{1\mathrm{э}}^{*2} +\alpha\bar\delta^*)}\,;\\[10pt]
    K_{23}^* =\fr{\beta^2 \nu_0 \bar\gamma_1^*}{2(\bar\w_{1\mathrm{э}}^{*2} +\alpha \bar\delta^*)}\,;
    \end{array}
    \right\}
    \label{e5.19-s}
    \end{equation}
   % \vspace*{1pt}
  %
\begin{equation}
D_1^* \bar\w_{1\mathrm{э}}^{*2}=\fr{\beta^2 \nu_0 \bar\gamma_1^*}{2\alpha \bar\delta^*} \left[ 1+
    \fr{\bar\gamma_1^*\bar\delta^*}{(\bar\w_{1\mathrm{э}}^{*2}+\alpha\bar\delta^*)}\right].\label{e5.20-s}
    \end{equation}
Соотношения~(\ref{e5.19-s}) 
параметрически зависят от $m_1^*$ и~$D_1^*$  посредством $\bar\w_{1\mathrm{э}}^{*2},\bar\delta^*$ и~$\bar\gamma_1^*$. 
Количества~$m_1^*$ и~$D_1^*$ определяются~(\ref{e5.14-s}) и~(\ref{e5.20-s}).

Для разрывной нелинейной функции~$\varphi_1 (\ddot X)$ коэффициенты регрессионной линеаризации имеют следующий вид:
\begin{gather*}
\varphi_1 (\ddot X) =\mathrm{sgn}  \ddot X\approx \varphi_0(\zeta) +k\ddot X^0\,;\\
 \zeta = \fr{m^{\ddot X}}{\sqrt{D^{\ddot X}}}\,;\enskip 
 \Phi (\zeta) =\fr{1}{\sqrt{2\pi}} \iii_0^\zeta e^{-t^2/2} dt\,;\enskip 
 k=\fr{\prt \varphi_0}{\prt m^{\ddot X}}.
 %\eqno(5.21)
 \end{gather*}

Для кубической нелинейности имеем формулы~(\ref{e5.7-s}).

%\vspace*{-12pt}

\section{Заключение}

Получено обобщение точных и~приближенных методов аналитического моделирования стационарных 
и~нестационарных СтП с~инвариантной мерой в~гауссовских и~негауссовских СтСНРОП, приводимых к~дифференциальным СтС. 
Для негауссовских СтСНРОП особое внимание уделено применению методов нормальной аппроксимации и~статистической
 линеаризации для нахождения корреляционных характеристик нормальных распределений с~инвариантной мерой.
Полученные результаты могут быть использованы в~задачах установления эквивалентности гауссовских и~негауссовских СтСНРОП.

Разработан комплекс тестовых примеров для экспериментального инструментального 
программного обеспечения StS-IMD в~среде MATLAB.

Проведенные вычислительные эксперименты подтверждают достаточную точностью 
для инженерных применений~\cite{11-sin, 12-sin}.

Среди направлений повышения точ\-ности аналитического моделирования можно
 отметить методы, основанные на прямом численном интегрировании исходных уравнений СтСНРОП.

%\vspace*{-9pt}

{\small\frenchspacing
 {%\baselineskip=10.8pt
 %\addcontentsline{toc}{section}{References}
 \begin{thebibliography}{99}
 
 \vspace*{-3pt}
 
\bibitem{1-sin} 
\Au{ Синицын~И.\,Н.}
Аналитическое моделирование распределений с~инвариантной мерой в~стохастических системах с~автокоррелированными шумами~// Информатика 
и~её применения, 2012. Т.~6. Вып.~4. С.~4--8.

%2
\bibitem{2-sin}  
\Au{Soize C.}
The Fokker--Plank equaton for stochastic dynamical systems and its explicit 
steady state solutions.~--- Singapore: World Scientific, 1994. 321~р.

%3
\bibitem{3-sin} 
\Au{Синицын И.\,Н.}
Аналитическое моделирование распределений с~инвариантной мерой в~стохастических системах с~разрывными характеристиками~// 
Информатика и~её применения, 2013. Т.~7. Вып.~1. С.~3--11.

%4
\bibitem{4-sin}  
\Au{Синицын И.\,Н.}
Развитие методов аналитического моделирования распределений с~инвариантной 
мерой в~стохастических сис\-те\-мах~// Современные проб\-ле\-мы 
при\-клад\-ной математики, информатики, автоматизации, управ\-ле\-ния: 
Мат-лы Междунар. семинара.~--- Севастополь: СевНТУ, 2012. С.~24--35.

%5
\bibitem{5-sin}  
\Au{Синицын И.\,Н.}
Аналитическое моделирование широкополосных процессов в~стохастических системах, 
не разрешенных относительно производных~// Информатика и~её применения, 
2017. Т.~11. Вып.~1. С.~3--10.

%6
\bibitem{6-sin}  
\Au{Синицын И.\,Н.}
Параметрическое аналитическое моделирование процессов в~стохастических  сис\-те\-мах, 
не разрешенных относительно производных~// Сис\-те\-мы и~средства 
информатики, 2017. Т.~27. №~1. С.~21--45.

%7
\bibitem{7-sin}  
\Au{Sinitsyn I.\,N.}
Analytical modeling and estimation of normal processes defined by stochastic differential equations
 with unsolved derivatives~// 
J.~Mathematics Statistics Research, 2021. Vol.~3. Iss.~1. Art.~139. 7~p. doi: 10.36266/ \mbox{JMSR}/139.

%8
\bibitem{8-sin}  
\Au{Синицын И.\,Н.}
Аналитическое моделирование и~оценивание нестационарных нормальных процессов 
в~стохастических сис\-те\-мах, не разрешенных относительно производных~// 
Сис\-те\-мы и~средства информатики, 2022. Т.~32. №~2. С.~58--71.


%9
\bibitem{9-sin}  
\Au{Синицын И.\,Н.}
Нормализация сис\-тем, стохастически не разрешенных относительно производных~// 
Информатика и~её применения, 2022. Т.~16. Вып.~1. С.~32--38.


%10
\bibitem{10-sin}  
\Au{Пугачёв В.\,С., Синицын И.\,Н.}
Теория стохастических сис\-тем.~--- М.: Логос, 2004. 1000~с.

\pagebreak


%11
\bibitem{11-sin} 
\Au{Александровская Л.\, Н., Аронов И.\,З., Круглов В.\,И. и~др.}
Безопасность и~надежность технических сис\-тем.~--- М.: Университетская книга, Логос, 2008. 348~c.

%12
\bibitem{12-sin} 
\Au{Синицын И.\,Н., Шаламов А.\,С.}
Лекции по теории сис\-тем интегрированной логистической поддержки.~--- 2-е изд.~--- М.: ТОРУС ПРЕСС, 2019. 1072~c.
\end{thebibliography}

 }
 }

\end{multicols}

\vspace*{-6pt}

\hfill{\small\textit{Поступила в~редакцию 15.01.23}}

\vspace*{8pt}

%\pagebreak

%\newpage

%\vspace*{-28pt}

\hrule

\vspace*{2pt}

\hrule

%\vspace*{-2pt}

\def\tit{ANALYTICAL MODELING OF DISTRIBUTIONS WITH~INVARIANT~MEASURE IN~STOCHASTIC SYSTEMS WITH~UNSOLVED~DERIVATIVES}


\def\titkol{Analytical modeling of distributions with invariant measure in stochastic systems with unsolved derivatives}


\def\aut{I.\,N.~Sinitsyn$^{1,2}$}

\def\autkol{I.\,N.~Sinitsyn}

\titel{\tit}{\aut}{\autkol}{\titkol}

\vspace*{-11pt}


\noindent
$^1$Federal Research Center ``Computer Science and Control'' of the Russian Academy 
of Sciences, 44-2~Vavilov\linebreak
$\hphantom{^1}$Str., Moscow 119333, Russian Federation


\noindent
$^2$Moscow State Aviation Institute (National Research University), 4~Volokolamskoe Shosse, Moscow 125933,\linebreak
$\hphantom{^1}$Russian Federation

\def\leftfootline{\small{\textbf{\thepage}
\hfill INFORMATIKA I EE PRIMENENIYA~--- INFORMATICS AND
APPLICATIONS\ \ \ 2023\ \ \ volume~17\ \ \ issue\ 1}
}%
 \def\rightfootline{\small{INFORMATIKA I EE PRIMENENIYA~---
INFORMATICS AND APPLICATIONS\ \ \ 2023\ \ \ volume~17\ \ \ issue\ 1
\hfill \textbf{\thepage}}}

\vspace*{3pt}



\Abste{Exact and approximate analytical modeling methods for stochastic processes with invariant measure in Gaussian and non-Gaussian stochastic
 systems with unsolved derivatives are considered. The methods are based on the linear regression approximation of nonlinear functions with 
 unsolved derivatives and reduction to stochastic Ito differential equations. 
 Two exact methods for analytical modeling of one- and multidimensional distributions with invariant measure are described. 
 Special attention is paid to normal approximation and parametrization methods. A~test example for Duffing equation nonlinear in second derivative is
  given. The stationary and nonstationary regimes and asymptotic stability are investigated. The method of normal approximation for one- and two-dimensional
   distributions is accurate enough  for engineering applications. Some generalizations concerning numerical analytical modeling are considered.}


\KWE{analytical modeling; distribution parametrization; distribution with invariant measure; 
stochastic system; stochastic system with unsolved derivatives; stochastic process}


 \DOI{10.14357/19922264230101} 

%\vspace*{-20pt}

%\Ack

%\vspace*{-4pt}

%\noindent
%The research was supported by the Russian Academy of Sciences (project АААА-А19-119091990037-5).
  

%\vspace*{6pt}

  \begin{multicols}{2}

\renewcommand{\bibname}{\protect\rmfamily References}
%\renewcommand{\bibname}{\large\protect\rm References}

{\small\frenchspacing
 {%\baselineskip=10.8pt
 \addcontentsline{toc}{section}{References}
 \begin{thebibliography}{99} 
 
%\vspace*{-2pt}
 
\bibitem{1-sin-1}
\Aue{Sinitsyn, I.\,N.}
 2012. Analiticheskoe modelirovanie ras\-predeleniy s~invariantnoy meroy v~sto\-kha\-sti\-che\-skikh sis\-temakh s~av\-to\-kor\-re\-li\-ro\-van\-ny\-mi shu\-ma\-mi 
 [Analytical\linebreak modeling of distributions with invariant measures in stochastic systems with autocorrelated noises]. 
 \textit{Informatika i~ee Primeneniya~--- Inform. Appl.} 6(4):4--8.
\bibitem{2-sin-1}
\Aue{Soize, C.} 1994. 
 \textit{The Fokker--Plank equation for stochastic dynamical systems and its explicit steady state solutions}. Singapore: World Scientific. 321~p.
\bibitem{3-sin-1}
\Aue{Sinitsyn, I.\,N.} 2013. Analiticheskoe modelirovanie raspredeleniy s~invariantnoy meroy v~sto\-kha\-sti\-che\-skikh sis\-te\-makh 
s~razryvnymi kharakteristikami [Analytical modeling of distributions with invariant measure in stochastic systems with discontinuous characteristics]. 
 \textit{Informatika i~ee Primeneniya~--- Inform. Appl.} 7(1):3--11.
\bibitem{4-sin-1}
\Aue{Sinitsyn, I.\,N.} 2012. 
Razvitie metodov ana\-li\-ti\-che\-sko\-go mo\-de\-li\-ro\-va\-niya ras\-pre\-de\-le\-niy s~in\-va\-ri\-ant\-noy me\-roy v~sto\-kha\-sti\-che\-skikh sis\-te\-makh 
[Development of analytical modeling methods for distributions with invariant measure in stochastic systems].
 \textit{Sovremennye problemy pri\-klad\-noy ma\-te\-ma\-ti\-ki, informatiki, avtomatizatsii, upravleniya: Mat-ly Mezhdunar. seminara}
  [Modern Problems of Applied Mathematics Informatics, Atomization and Control: Seminar (International) Proceedings]. Sevastopol': SevNTU. 24--35.
\bibitem{5-sin-1}
\Aue{Sinitsyn, I.\,N.} 2017. Ana\-li\-ti\-che\-skoe mo\-de\-li\-ro\-va\-nie shi\-ro\-ko\-po\-los\-nykh pro\-tses\-sov 
v~sto\-kha\-sti\-che\-skikh sis\-te\-makh, ne raz\-re\-shen\-nykh ot\-no\-si\-tel'\-no pro\-iz\-vod\-nykh 
[Analytical modeling of wide band processes in stochastic systems with unsolved derivatives].
 \textit{Informatika i~ee Primeneniya~--- Inform. Appl.} 11(1):3--10.
\bibitem{6-sin-1}
\Aue{Sinitsyn, I.\,N.}
 2017. Pa\-ra\-met\-ri\-che\-skoe ana\-li\-ti\-che\-skoe mo\-de\-li\-ro\-va\-nie pro\-tses\-sov v~sto\-kha\-sti\-che\-skikh sis\-te\-makh, 
 ne raz\-re\-shen\-nykh ot\-no\-si\-tel'\-no pro\-iz\-vod\-nykh 
 [Parametric analytical modeling of processes in stochastic systems that are not allowed with respect to derivatives]. 
  \textit{Sistemy i~Sredstva Informatiki~--- Systems and Means of Informatics} 27(1):\linebreak21--45.
\bibitem{7-sin-1}
\Aue{Sinitsyn, I.\,N.} 2021. Analytical modeling and estimation of normal processes defined by stochastic differential equations with unsolved derivates.
 \textit{J.~Mathematics Statistics Research} 3(1):139. 7~p. 
\bibitem{8-sin-1}
\Aue{Sinitsyn, I.\,N.} 2022. Ana\-li\-ti\-che\-skoe mo\-de\-li\-ro\-va\-nie i~otse\-ni\-va\-nie ne\-sta\-tsi\-o\-nar\-nykh nor\-mal'\-nykh pro\-tses\-sov 
v~sto\-kha\-sti\-che\-skikh sis\-te\-makh, ne raz\-res\-hen\-nykh ot\-no\-si\-tel'\-no pro\-iz\-vod\-nykh [Anatlytical modeling and estmation of nonstationary normal processes with 
unsolved deriva-\linebreak\vspace*{-12pt}

\pagebreak

\noindent
tives].  \textit{Sistemy i~Sredstva Informatiki~--- Systems and Means of Informatics} 32(2):58--71.
\bibitem{9-sin-1}
\Aue{Sinitsyn, I.\,N.}
 2022. Nor\-ma\-li\-za\-tsiya sis\-tem, sto\-kha\-sti\-che\-ski ne raz\-re\-shen\-nykh ot\-no\-si\-tel'\-no pro\-iz\-vod\-nykh 
 [Normalization of systems with stochastically unsolved derivatives]. 
  \textit{Informatika i~ee Primeneniya~--- Inform. Appl.} 16(1):32--38.
\bibitem{10-sin-1}
\Aue{Pugachev, V.\,S., and I.\,N.~Sinitsyn.} 2000, 2004.  \textit{Teo\-riya sto\-kha\-sti\-che\-skikh sis\-tem}
 [Stochastic systems. Theory and applications]. Moscow: Logos. 1000~p.
\bibitem{11-sin-1}
\Aue{Aleksandrovskaya, L.\,N., I.\,Z.~Aronov, V.\,I.~Kruglov, \textit{et al.}}
 2008.  \textit{Bezopas\-nost' i~na\-dezh\-nost' tekh\-ni\-che\-skikh sis\-tem} [Security and safety of technical systems]. Moscow: Universitetskaya kniga, Logos. 348~p.
\bibitem{12-sin-1}
\Aue{Sinitsyn, I.\,N., and A.\,S.~Shalamov.}
 2019. \textit{Lek\-tsii po teo\-rii sis\-tem in\-teg\-ri\-ro\-van\-noy lo\-gi\-sti\-che\-skoy pod\-derzh\-ki}
  [Lectures on theory of integrated logistic support systems]. 2nd ed. Moscow: TORUS PRESS. 1072~p.
 \end{thebibliography}

 }
 }

\end{multicols}

\vspace*{-6pt}

\hfill{\small\textit{Received January 15, 2023}} 


\Contrl

\noindent
\textbf{Sinitsyn Igor N.} (b.\ 1940)~--- 
Doctor of Science in technology, professor, Honored scientist of RF, principal scientist, Institute of Informatics Problems, Federal Research Center 
``Computer Science and Control'' of the Russian Academy of Sciences, 44-2~Vavilov Str., Moscow 119333, Russian Federation; 
professor, Moscow State Aviation Institute (National Research University), 4~Volokolamskoe Shosse, Moscow 125933, Russian Federation; \mbox{sinitsin@dol.ru}


   
\label{end\stat}

\renewcommand{\bibname}{\protect\rm Литература} 