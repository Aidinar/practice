\def\stat{adu}

\def\tit{АНАЛИЗ СХЕМЫ ДОСТУПА С~ПРЕРЫВАНИЕМ ПРИ НАРЕЗКЕ 
РАДИОРЕСУРСОВ СЕТИ ПЯТОГО ПОКОЛЕНИЯ$^*$}

\def\titkol{Анализ схемы доступа с~прерыванием при нарезке 
радиоресурсов сети пятого поколения}

\def\aut{К.\,И.\,Б.~Аду$^1$, Е.\,В.~Маркова$^2$, Ю.\,В.~Гайдамака$^3$, С.\,Я.~Шоргин$^4$}

\def\autkol{К.\,И.\,Б.~Аду, Е.\,В.~Маркова, Ю.\,В.~Гайдамака, С.\,Я.~Шоргин}

\titel{\tit}{\aut}{\autkol}{\titkol}

\index{Аду К.\,И.\,Б.}
\index{Маркова Е.\,В.}
\index{Гайдамака Ю.\,В.}
\index{Шоргин С.\,Я.}
\index{Adou K.\,Y.\,B.}
\index{Markova E.\,V.}
\index{Gaidamaka Yu.\,V.}
\index{Shorgin S.\,Ya.}


{\renewcommand{\thefootnote}{\fnsymbol{footnote}} \footnotetext[1]
{Исследование выполнено за счет гранта Российского научного 
фонда №\,22-79-10053, {\sf https://rscf.ru/project/22-79-10053/}.}}


\renewcommand{\thefootnote}{\arabic{footnote}}
\footnotetext[1]{Российский университет дружбы народов, \mbox{adu-k@rudn.ru}}
\footnotetext[2]{Российский университет дружбы народов, markova-ev@rudn.ru}
\footnotetext[3]{Российский университет дружбы народов; Федеральный 
исследовательский центр <<Информатика и~управление>> Российской академии наук, 
\mbox{gaydamaka-yuv@rudn.ru}}
\footnotetext[4]{Федеральный исследовательский центр <<Информатика и~управление>> Российской академии наук, 
\mbox{sshorgin@ipiran.ru}}

\vspace*{-2pt}
 
 

\Abst{Активно исследуемая в~последние годы технология <<нарезки радиоресурсов сети>> 
(NS~--- Network Slicing), основанная на представлении общей сетевой инфраструктуры 
в~виде различных настраиваемых логических сетей, называемых слайсами, предполагает 
разделение операторов мобильной сети на две группы~--- провайдеры физической 
сетевой инфраструктуры InPs (Infrastructure Providers) и~операторы мобильной 
виртуальной сети (MVNOs~--- Mobile Virtual Network Operators). Последние арендуют 
физические ресурсы InPs для создания собственных слайсов с~целью предоставления 
своим пользователям услуг с~различными требованиями к~качеству обслуживания. 
В~статье для сети с~технологией NS предложена схема доступа к~радиоресурсам сети, 
предоставляющей пользователям услуги с~гарантированной скоростью передачи данных 
(GBR~--- Guaranteed Bit Rate) и~приоритетным управлением, основанным на реализации 
механизма прерывания обслуживания пользователей. Для оценки эффективности 
предлагаемой схемы проведен сравнительный анализ ее характеристик 
с~характеристиками схемы доступа, основанной на механизме резервирования ресурсов.}

\KW{5G; нарезка сети; качество обслуживания; ключевые 
показатели эффективности; приоритетное управление; прерывание обслуживания; 
итерационный метод}

\DOI{10.14357/19922264230113} 
  
\vspace*{2pt}


\vskip 10pt plus 9pt minus 6pt

\thispagestyle{headings}

\begin{multicols}{2}

\label{st\stat}

\section{Введение}

Ввиду ограниченности спектра частотного диапазона мобильных сетей для 
предоставления пользователям услуг с~требуемым качеством обслуживания необходимо 
внедрение новых технологий. В~последние годы активно исследуется технология 
<<нарезки радиоресурсов сети>> NS, основанная на представлении 
общей сетевой инфраструктуры в~виде различных настраиваемых логических сетей, 
называемых слайсами. Одной из \mbox{важнейших} проб\-лем реализации технологии NS 
является проб\-ле\-ма эффективного распределения радиоресурсов~--- полосы 
пропускания или физических ресурсных\linebreak блоков PRB (Physical Resource Block). 
Радиоре\-сурсы должны быть распределены между несколькими слайсами в~соответствии 
с~динамически\linebreak меняющимися требованиями пользователей~--- операторов 
мобильной виртуальной сети MVNOs (Mobile Virtual Network Operators), при этом 
долж\-ны быть выполнены ключевые требования к~изоляции слайсов~\cite{3gpp.21.916, 3gpp.22.864}, 
в~част\-ности трафик, об\-слу\-жи\-ва\-емый в~рамках одного слайса, не должен оказывать негативного влияния на трафик, 
об\-слу\-жи\-ва\-емый в~других слайсах. Например, в~работах~\cite{3gpp.28.554, Yarkina2022} для реализации межслайсовой изоляции 
предложены подходы, основанные на резервировании ресурсов.

В данной работе рассмотрена модель схемы распределения ресурсов базовой 
станции (БС) соты сети между несколькими слайсами, пользователям которых 
предоставляются услуги реального времени с~GBR, например видео- и~голосовая 
телефония.
Особенность предложенной модели заключается в~управлении доступом RAC 
(Radio Admission\linebreak Control), основанном на реализации совместного доступа к~части 
имеющихся ресурсов и~введении приоритетного обслуживания. Для реализации 
межслайсовой изоляции приоритетное\linebreak управление использует механизм прерывания 
обслуживания пользователей, при этом в~\mbox{статье} раз\-работан итерационный метод 
расчета числа прерванных запросов. Чтобы оценить эффективность предлагаемой 
схемы доступа с~прерыванием обслуживания пользователей (далее~--- ПС,  схема 
с~прерыванием), проведен сравнительный анализ ее основных ключевых показателей 
эффективности (КПЭ) с~КПЭ схемы доступа, основанной на реализации механизма 
резервирования ресурсов (далее~--- РС, схема с~резервированием)~\cite{Luu2022,Rehman2022}.

\begin{table*}[b]\small %tabl1
\vspace*{-16pt}
\begin{center}
\Caption{Основные символы системы}
\label{tab:Notations}
\vspace*{2ex}

\begin{tabular}{|c|l|c|}
\hline
Обозначение & \multicolumn{1}{c}{Описание} & \tabcolsep=0pt\begin{tabular}{c}Единица \\ измерения\end{tabular}\\ 
\hline
$\mathcal{S}$ & Множество слайсов в~сис\-те\-ме, 
$\mathcal{S}\subset\mathbb{N}\setminus\{0\}, \mathbb{N}=\{0,1,2,\ldots\}$ & --- \\
\hline
$C$        & Общая емкость системы & ед.\ емкости \\
\hline
$C_s$      & Максимальная емкость слайса $s$, $s\hm\in\mathcal{S}$, 
$\sum\nolimits_{s\in\mathcal{S}} C_s \hm\geq C$ & ед.\ емкости \\[2pt]
\hline
&&\\[-10pt]
$Q_s$      & Гарантированная емкость слайса $s$, $s\hm\in\mathcal{S}$, $Q_s \hm\leq C_s$ и~$\sum\nolimits_{s\in\mathcal{S}} Q_s\hm \leq C$ & ед.\ емкости \\[2pt]
\hline
&&\\[-10pt]
$\lambda_s$ & Интенсивность поступления запросов в~слайс~$s$, $s\hm\in\mathcal{S}$, 
$\boldsymbol\lambda = 
\left(\lambda_1,\ldots,\lambda_{\lvert\mathcal{S}\rvert}\right)$ & запросов/ед.\ вр. \\[2pt]
\hline
$\mu_s^{-1}$ & \tabcolsep=0pt\begin{tabular}{l}Среднее время обслуживания одного запроса в~слайсе 
$s$, $s\hm\in\mathcal{S}$, $\boldsymbol\mu \hm= {}$\\
${}=
\left(\mu_1,\ldots,\mu_{\lvert\mathcal{S}\rvert}\right)$\end{tabular} & ед.\ вр. \\
\multicolumn{1}{|c|}{\ }&&\multicolumn{1}{c|}{\ }\\[-10pt]
\hline
$\rho_s\hm=\lambda_s/\mu_s$ & Предложенная нагрузка, создаваемая 
запросами в~слайсе~$s$, $s\hm\in\mathcal{S}$ & --- \\
\hline
$b_s$      & \tabcolsep=0pt\begin{tabular}{l}Требование к~ресурсам, необходимым для обслуживания одного запроса\\
 в~слайсе $s$, $s\hm\in\mathcal{S}$, $b_s \leq Q_s$, $\mathbf{b}\hm= 
\left(b_1,\ldots,b_{\lvert\mathcal{S}\rvert}\right)$ \end{tabular}& ед.\ емкости \\
\multicolumn{1}{|c|}{\ }&&\multicolumn{1}{c|}{\ }\\[-10pt]
\hline
&&\\[-10pt]
$\left\lfloor C_s/b_s \right\rfloor$ & \tabcolsep=0pt\begin{tabular}{l}Максимальное число запросов в~слайсе 
$s,s\in\mathcal{S}$, $\mathbf{N}^{\max} \hm= \left(\left\lfloor C_1/b_1 
\right\rfloor,\ldots\right.$\\ 
$\left.\ldots,\left\lfloor 
C_{\lvert\mathcal{S}\rvert}/b_{\lvert\mathcal{S}\rvert} \right\rfloor \right)$ \end{tabular}& 
--- \\
\multicolumn{1}{|c|}{\ }&&\multicolumn{1}{c|}{\ }\\[-10pt]
\hline
$\left\lfloor Q_s/b_s \right\rfloor$ &\tabcolsep=0pt\begin{tabular}{l} Максимальное число запросов, которое 
может быть обслужено с~ис\-поль\-зо-\\ ва\-ни\-ем гарантированной емкости слайса 
$s$, $s\hm\in\mathcal{S}$, $\mathbf{N}^{{g}}\hm = \left(\left\lfloor Q_1/b_1 
\right\rfloor,\ldots \right.$\\
$\left.\ldots ,\left\lfloor Q_{\lvert\mathcal{S}\rvert}/b_{\lvert\mathcal{S}\rvert} \right\rfloor \right)$\end{tabular} & 
--- \\
\multicolumn{1}{|c|}{\ }&&\multicolumn{1}{c|}{\ }\\[-10pt]
\hline
$n_s$      & \tabcolsep=0pt\begin{tabular}{l}Число запросов в~слайсе $s$, $s\hm\in\mathcal{S}$, когда система 
находится в~состоянии\\
 $\mathbf{n} \hm= \left(n_1,\ldots,n_{\lvert\mathcal{S}\rvert}\right)$\end{tabular} & --- \\
 \multicolumn{1}{|c|}{\ }&&\multicolumn{1}{c|}{\ }\\[-10pt]
\hline
$\mathbf{e}_s$ & Строка $s$, $s\hm\in\mathcal{S}$, единичной матрицы 
$\lvert\mathcal{S}\rvert \times  \lvert\mathcal{S}\rvert$ & --- \\
\hline
\end{tabular}
\end{center}
\end{table*}


\section{Описание модели}
%\label{sec:sysModel}

Рассмотрим работу БС соты сети, принадлежащей одному провайдеру InP и~имеющей 
емкость~$C$. Емкость БС используется несколькими мобильными операторами MVNOs, 
\mbox{предоставляющими} услуги своим пользователям, при этом под каждую услугу 
оператору MVNO выделяется так называемый <<слайс>>~--- часть от общей 
ем\-кости~$C$. Обозначим множество слайсов через $\mathcal{S}$, 
$\mathcal{S}\hm\subset\mathbb{N}\setminus\{0\}, \mathbb{N} \hm= \{0,1,2,\ldots\}$. Для 
каждого слайса $s$ определена максимальная емкость~$C_s$, причем $C_s\hm \leq C$ 
и~$\sum\nolimits_{s\in\mathcal{S}} C_s \hm\geq C$. Емкость~$C_s$ слайса $s$ включает в~себя 
гарантированную часть емкостью~$Q_s$ и~общедоступную часть $C_s \hm- Q_s$, при этом 
$Q_s \hm\leq C_s$ и~$\sum\nolimits_{s\in\mathcal{S}} Q_s \hm\leq C$. Поток запросов 
пользователей на предоставление услуги $s$ моделируется с~по\-мощью пуассоновского 
потока запросов типа~$s$ с~интенсивностью~$\lambda_s$, $s\hm\in\mathcal{S}$.

Основные обозначения представлены в~табл.~\ref{tab:Notations}. Указанные параметры 
позволяют реализовать схе-\linebreak му ПС управления доступом с~механизмом прерывания 
обслуживания пользователей.
Отличие предлагаемой схемы ПС управления доступом с~механизмом прерывания от 
классической неполнодоступной схемы с~потолками в~общей части ресурса (Sharing 
with Maximum Queue Length and Minimum Allocation)~\cite{Basharin1982,kermani1977analysis,Kamoun1980} заключается в~отсутствии 
индивидуальной зоны для запросов типа~$s$, $s\hm\in\mathcal{S}$, в~которую не 
допускаются запросы других типов. Отметим, что в~гарантированной части 
незагруженного слайса~$s$, $s\hm\in\mathcal{S}$, может начать обслуживаться принятый в~сис\-те\-му запрос произвольного типа 
$\hat{s}$, $\hat{s}\hm\in\mathcal{S}\setminus\{s\}$, при этом с~точки зрения 
слайса~$s$ запрос типа~$\hat{s}$ становится так на\-зы\-ва\-емым <<нарушителем>>. Если 
при по\-сле\-ду\-ющем поступлении запроса типа~$s$ чис\-ло об\-слу\-жи\-ва\-емых в~этом слайсе 
запросов окажется меньше гарантированного значения $\left\lfloor Q_s/b_s 
\right\rfloor$, а~объем доступного ресурса БС меньше требуемого~$b_s$, то запрос 
будет принят на обслуживание за счет прерывания обслуживания одного или 
нескольких за\-про\-сов-<<на\-ру\-ши\-те\-лей>> типа 
$\hat{s}$, $\hat{s}\hm\in\mathcal{S}\setminus\{s\}$. Для реализации механизма 
прерывания каждому слайсу присвоен приоритет в~обслуживании, что также отличает 
предложенную схему ПС от классической неполнодоступной схемы с~потолками в~общей 
час\-ти ресурса. В~предположении о~различающихся приоритетах у слайсов 
перенумеруем слайсы в~порядке убывания приоритета, т.\,е.\ высший приоритет 
в~обслуживании получат запросы, находящиеся в~слайсе с~номером~<<$1$>>, низший~--- 
в~слайсе с~номером <<$\lvert\mathcal{S}\rvert$>>. Введем век\-тор-функ\-цию 
пре\-ры\-вания
\begin{equation*}
\mathbf{z}\left(s,\mathbf{n}\right) = \left( 
z_{\hat{s}}\left(s,\mathbf{n}\right) \right) = \left( 
z_1\left(s,\mathbf{n}\right), \ldots, 
z_{\lvert\mathcal{S}\rvert}\left(s,\mathbf{n}\right) \right),\!
\end{equation*} 

\vspace*{-3pt}

\noindent
определяющую число об\-слу\-жи\-ва\-емых запросов слайса 
$\hat{s}$, $\hat{s}\hm\in\mathcal{S}$, которое необходимо прервать для приема одного 
запроса в~слайс~$s$, $s\hm\in\mathcal{S}$.

\begin{figure*} %fig1
\vspace*{1pt}
\begin{center}
   \mbox{%
\epsfxsize=161.239mm
\epsfbox{adu-1.eps}
}
\end{center}
\vspace*{-9pt}
\Caption{Блок-схема для иллюстрации правил управления доступом к~радиоресурсам сис\-те\-мы}
\label{fig:generalRACScheme}
\end{figure*}

Таким образом, при поступлении в~слайс~$s$, $s\hm\in\mathcal{S}$, запроса на 
предоставление услуги возможны три случая:\\[-14pt]
\begin{enumerate}[(1)]
\item запрос немедленно будет принят на обслуживание, если число об\-слу\-жи\-ва\-емых 
запросов в~данном слайсе меньше $\left\lfloor C_s/b_s \right\rfloor$, а~объем 
доступного ресурса БС больше или равен~$b_s$, т.\,е. 
$\left(\mathbf{n}\hm+\mathbf{e}_s\hm-\mathbf{N}^{\max}\right) \cdot \mathbf{e}_s \hm\leq 
0 \land \left(\mathbf{n}\hm + \mathbf{e}_s\right) \cdot \mathbf{b} \hm\leq C$;
\item запрос будет принят на обслуживание за счет прерывания обслуживания 
за\-про\-сов-<<на\-ру\-ши\-те\-лей>> из других слайсов 
$\hat{s}$, $\hat{s}\hm\in\mathcal{S}\setminus\{s\}$, число которых 
определяется с~по\-мощью век\-тор-функ\-ции прерывания 
$\mathbf{z}\left(s,\mathbf{n}\right)$, если число об\-слу\-жи\-ва\-емых запросов в~слайсе~$s$ 
меньше $\left\lfloor Q_s/b_s \right\rfloor$, а~объем доступных 
ресурсов БС меньше~$b_s$, т.\,е.\ $\left(\mathbf{n}\hm+\mathbf{e}_s \hm-
\mathbf{N}^{g}\right) \cdot \mathbf{e}_s\hm \leq 0 \hm\land \left(\mathbf{n} \hm+ 
\mathbf{e}_s \hm- \mathbf{z}\left(s,\mathbf{n}\right)\right) \cdot \mathbf{b} \hm\leq 
C$;
\item запрос будет заблокирован, если число об\-слу\-жи\-ва\-емых запросов в~слайсе~$s$ 
больше или рав\-но $\left\lfloor Q_s/b_s \right\rfloor$, а~объем доступного 
ресурса БС меньше~$b_s$, т.\,е.\ $\left( \mathbf{n}\hm + \mathbf{e}_s \hm-
\mathbf{N}^{g} \right) \cdot \mathbf{e}_s \hm> 0 \land \left( \mathbf{n} \hm+ 
\mathbf{e}_s \right) \cdot \mathbf{b} \hm> C$.
\end{enumerate}

Отметим, что, в~отличие от классических схем\linebreak управ\-ле\-ния доступом 
с~приоритизацией, предлагаемая схема ПС для приема запроса в~гарантированную часть 
соответствующего слайса \mbox{предусмат\-ри\-ва\-ет} необходимость прервать обслуживание 
за\-про\-са-<<на\-ру\-ши\-те\-ля>> (т.\,е.\ запроса, об\-слу\-жи\-ва\-емо\-го вне гарантированной час\-ти 
своего слайса) не только более низкого приоритета, но и~более высокого 
приоритета.


%\smallskip

\noindent
\textbf{Утверждение 1.}
При начальном условии $\mathbf{z}\left(s,\mathbf{n}\right)=\mathbf{0}$, числа 
$z_{\hat{s}}\left(s,\mathbf{n}\right)$, $\hat{s}\hm\in\mathcal{S}$, запросов, 
обслуживание которых необходимо будет прервать, можно вычислить с~по\-мощью 
рекуррентного соотношения

\vspace*{-6pt}

\noindent
\begin{multline}
\label{eq:capabilityFunc}
z_{\hat{s}}\left(s,\mathbf{n}\right) = \min
\biggl\{
R \left(\left(\mathbf{n}-\mathbf{N}^{g}\right) \cdot 
\mathbf{e}_{\hat{s}}\right), \\[-3pt]
R \left(\left\lceil \frac{\left(\mathbf{n}+\mathbf{e}_s - 
\mathbf{z}\left(s,\mathbf{n}\right) \right)\cdot\mathbf{b} - 
C}{\mathbf{e}_{\hat{s}}\cdot\mathbf{b}} \right\rceil\right)\!\!
\biggl\},\\[-3pt]
\hat{s} = \lvert\mathcal{S}\rvert,\ldots,1,
\end{multline}

\columnbreak

\noindent
где $R\left(x\right)=xH\left(x\right)$~--- функция 
Рампы\footnote{{\sf https://mathworld.wolfram.com/RampFunction.html.}}, 
а~$H\left(x\right)$~--- функция 
Хевисайда\footnote{{\sf https://mathworld.wolfram.com/HeavisideStepFunction.html.}}.

\smallskip

Отметим, что число~$z_s\left(s,\mathbf{n}\right), 
s\in\mathcal{S}$, всегда равно нулю, так как запрос не может быть принят на 
обслуживание за счет прерывания об\-слу\-жи\-ва\-емо\-\mbox{го(-ых)} за\-про\-са(-ов) в~этом же 
слайсе.

Для наглядности схему управ\-ле\-ния до\-сту\-пом можно описать с~по\-мощью блок-схе\-мы, 
пред\-став\-лен\-ной на рис.~\ref{fig:generalRACScheme}.

\vspace*{-14pt}


\section{Построение математической модели}\label{sec:mathModel}

\vspace*{-4pt}

В соответствии с~описанной в~разд.~2 схемой 
управ\-ле\-ния доступом к~радиоресурсам сети поведение системы описывает 
$\lvert\mathcal{S}\rvert$-мер\-ный случайный процесс (СП) 
$\mathbf{X}\left(t\right) \hm= \left( X_1\left(t\right),\ldots, 
X_{\lvert\mathcal{S}\rvert}\left(t\right), t\hm>0 \right)$, где 
$X_s\left(t\right)$, $s\hm\in\mathcal{S}$,~--- чис\-ло об\-слу\-жи\-ва\-емых запросов в~слайсе~$s$ 
в~момент времени~$t$ над пространством со\-сто\-яний

\noindent
\begin{equation*}
%\label{eq:StateSpace}
\Omega =
\left\{
\mathbf{n} \in \mathbb{N}^{\lvert\mathcal{S}\rvert} :
\left( \mathbf{n} - \mathbf{N}^{\max} \right) \cdot \mathbf{j} \leq 0
\land
\mathbf{n} \cdot \mathbf{b} \leq C
\right\},
\end{equation*}

\vspace*{-4pt}

\noindent
где $\mathbb{N}^{\lvert\mathcal{S}\rvert}$~--- множество всевозможных 
$\lvert\mathcal{S}\rvert$-мер\-ных век\-то\-ров-строк с~натуральными элементами, 
а~$\mathbf{j}$~--- единичная матрица размера $1\times \lvert\mathcal{S}\rvert$.

Схема соответствующей мультисервисной сис\-те\-мы массового 
обслуживания (СМО)~\cite{Basharin2013} изображена на~рис.~2. 





Для дальнейшего анализа модели введем следующие подмножества пространства 
состояний~$\Omega$ системы:
\begin{enumerate}[(1)]
\item $\Omega_s^{\mathrm{dad}},s\in\mathcal{S}$,~--- множество состояний системы, в~которых поступающий в~слайс~$s$ запрос немедленно будет принят на 
обслуживание:
\end{enumerate}

{ \begin{center}  %fig2
 \vspace*{-3pt}
    \mbox{%
\epsfxsize=77.767mm
\epsfbox{adu-2.eps}
}


\vspace*{3pt}


\noindent
{{\figurename~2}\ \ \small{Схема СМО
}}
\end{center}
}

\vspace*{-12pt}

\addtocounter{figure}{1}

\begin{enumerate}[(1)]
\setcounter{enumi}{1}
\item[\,]

\noindent
\begin{multline}
\label{eq:SetAdmission}
\Omega_s^{\mathrm{dad}} =
\left\{
\mathbf{n} \in \Omega: \right.\\ \left.
\left(\mathbf{n}-\mathbf{N}^{\max}\right) \cdot \mathbf{e}_s < 0
\land
\left(\mathbf{n} + \mathbf{e}_s \right) \cdot \mathbf{b} \leq C
\right\};
\end{multline}
\item $\Omega_s^{\mathrm{vpad}}$, $s\hm\in\mathcal{S}$,~--- множество состояний 
системы, в~которых поступающий в~слайс~$s$ запрос будет принят на обслуживание 
за счет прерывания об\-слу\-жи\-ва\-емо\-го(-ых) за\-про\-са(-ов) в~слайсах 
$\hat{s}$, $\hat{s}\hm\in\mathcal{S}\setminus\{s\}$:
\begin{multline}
\label{eq:SetPreemptionCapability}
\Omega_s^{\mathrm{vpad}} =
\left\{
\mathbf{n} \in \Omega: \right.\\ \left.
\left(\mathbf{n}-\mathbf{N}^{{g}}\right) \cdot \mathbf{e}_s < 0
\land
\left(\mathbf{n} + \mathbf{e}_s \right) \cdot \mathbf{b} > C
\right\};
\end{multline}
\item $\Omega_s^{\mathrm{block}}$, $s\hm\in\mathcal{S}$,~--- множество состояний 
системы, в~которых поступающий в~слайс~$s$ запрос будет заблокирован:
\begin{subequations}
\label{eq:SetBlocking}
\begin{multline}
\Omega_s^{\mathrm{block}} =
\left\{
\mathbf{n} \in \Omega: \right.\\ \left.
\left(\mathbf{n}-\mathbf{N}^{{g}}\right) \cdot \mathbf{e}_s \geq 0
\land
\left(\mathbf{n} + \mathbf{e}_s \right) \cdot \mathbf{b} > C
\right\};
\end{multline}
или
\begin{equation}
\Omega_s^{\mathrm{block}} = \Omega \setminus \left( \Omega_s^{\mathrm{dad}} \cup 
\Omega_s^{\mathrm{vpad} }\right),
\end{equation}
где $\Omega_s^{\mathrm{dad}} \cup \Omega_s^{\mathrm{vpad}}$~--- множество состояний 
системы, в~которых поступающий в~слайс~$s$ запрос будет принят на обслуживание.
\end{subequations}
\end{enumerate}



Пример для иллюстрации данных подмножеств для $\mathcal{S} = 
\{1,2\}$ пред\-став\-лен в~разд.~4.1 (см.\ рис.~4).

Подмножества~(\ref{eq:SetAdmission})--(\ref{eq:SetBlocking}) позволяют разделить все состояния 
$\mathbf{n},\mathbf{n}\hm\in\Omega$, системы на шесть групп:
\begin{enumerate}[(1)]
\item $\bigcap\nolimits_{s\in\mathcal{S}}\Omega_s^{\mathrm{dad}}$~--- множество 
состояний сис\-те\-мы, в~которых любой поступающий в~систему запрос 
немедленно будет принят на обслуживание;
\item $\Omega_s^{\mathrm{dad}}\setminus 
\bigcup\nolimits_{\hat{s}\hm\in\mathcal{S}\setminus\{s\}} 
\Omega_{\hat{s}}^{\mathrm{dad}}$, $s\hm\in\mathcal{S}$,~--- множество состояний 
сис\-те\-мы, в~которых только по\-сту\-па-\linebreak\vspace*{-16pt}
\end{enumerate}

\columnbreak

\begin{enumerate}[(1)]
\setcounter{enumi}{2}
\item[\,] ющий в~слайс~$s$ запрос немедленно будет 
принят на обслуживание;
\item $\Omega_s^{\mathrm{vpad}} \cap 
\bigcup\nolimits_{\hat{s}\hm\in\mathcal{S}\setminus\{s\}} 
\Omega_{\hat{s}}^{\mathrm{vpad}}$, $s\hm\in\mathcal{S}$,~--- множество состояний 
системы, в~которых поступающие в~слайсы~$s$ и~$\hat{s}$,
$\hat{s}\hm\in\mathcal{S}\setminus\{s\}$, запросы будут приняты на 
обслуживание за счет прерывания об\-слу\-жи\-ва\-емо\-го(-ых) за\-про\-са(-ов) в~слайсах~$\tilde{s}$,
$\tilde{s}\hm\in\mathcal{S}\setminus\{s,\hat{s}\}$;
\item $\Omega_s^{\mathrm{vpad}} \setminus 
\bigcup\nolimits_{\hat{s}\hm\in\mathcal{S}\setminus\{s\}} 
\Omega_{\hat{s}}^{\mathrm{vpad}}$, $s\hm\in\mathcal{S}$,~--- множество состояний 
системы, в~которых только поступающий в~слайс~$s$ запрос будет принят на 
обслуживание за счет прерывания одного или нескольких об\-слу\-жи\-ва\-емых запросов 
в~слайсе~$\hat{s}$, $\hat{s}\hm\in\mathcal{S}\setminus\{s\}$;
\item $\bigcap\nolimits_{s\in\mathcal{S}}\Omega_s^{\mathrm{block}}$~--- множество 
состояний системы, в~которых любой поступающий в~систему запрос будет 
заблокирован;
\item $\Omega_s^{\mathrm{block}}\setminus 
\bigcup\nolimits_{\hat{s}\hm\in\mathcal{S}\setminus\{s\}} 
\Omega_{\hat{s}}^{\mathrm{block}}$, $s\hm\in\mathcal{S}$,~--- множество состояний 
системы, в~которых только поступающий в~слайс~$s$ запрос будет 
заблокирован.
\end{enumerate}


Диаграмма интенсивностей переходов для состояния системы 
$\mathbf{n},\mathbf{n}\hm\in\Omega$, имеет вид, представленный 
на рис.~3.




Функция управления доступом к~ресурсам сис\-те\-мы определяется следующим образом:
\begin{multline*}
f_s\left(\mathbf{n}\right) =
\begin{cases}
1,  &\ \mbox{если}\ 
\mathbf{n}\in\left(\Omega_s^{\mathrm{dad}}\cup\Omega_s^{\mathrm{vpad}}\right);\\
0           &\ \mbox{в\ противном\ случае},\ 
\mathbf{n}\in\Omega_s^{\mathrm{block}}.
\end{cases}\\
s\in\mathcal{S}.
\end{multline*}

Согласно диаграмме интенсивностей переходов рис.~3, 
рассматриваемый СП описывается сле\-ду\-ющей сис\-те\-мой урав\-не\-ний рав\-но\-ве\-сия:

{ \begin{center}  %fig3
 \vspace*{6pt}
    \mbox{%
\epsfxsize=76.958mm
\epsfbox{adu-3.eps}
}

\end{center}



\noindent
{{\figurename~3}\ \ \small{Диаграмма интенсивностей переходов для состояния 
системы~$\mathbf{n}$, $\mathbf{n}\hm\in\Omega$
}}}

%\vspace*{6pt}

\addtocounter{figure}{1}

\noindent
\begin{multline*}
%\label{eq:EquilibriumEquationsSystem}
\!\!\!\! P\left(\mathbf{n}\right)\! \left( \boldsymbol\lambda \cdot 
\sum_{s\in\mathcal{S}}{\left( I_{\Omega_s^{\mathrm{dad}}}\left(\mathbf{n}\right) + 
I_{\Omega_s^{\mathrm{vpad}}}\left(\mathbf{n}\right) \right) \mathbf{e}_s + 
\mathbf{n}\cdot\boldsymbol\mu} \right) = {}\hspace*{-2.69116pt}\\
{}= \boldsymbol\lambda \cdot \sum_{s\in\mathcal{S}}
\biggl(
P\left(\mathbf{n}-\mathbf{e}_s\right)\, 
I_{\Omega_s^{\mathrm{dad}}}\left(\mathbf{n}-\mathbf{e}_s\right) + {}\hspace*{-2.69116pt}\\
{}+ P\left(\mathbf{n}-\mathbf{e}_s+\mathbf{z}\left(s,\mathbf{n}\right)\right)\, 
I_{\Omega_s^{\mathrm{vpad}}}\left(\mathbf{n}-
\mathbf{e}_s+\mathbf{z}\left(s,\mathbf{n}\right)\right)\!\!
\biggl)
\mathbf{e}_s +{}\hspace*{-2.69116pt}\\
{}+ \boldsymbol\mu \cdot 
\sum_{s\in\mathcal{S}}{\left(P\left(\mathbf{n}+\mathbf{e}_s\right)\, 
I_{\Omega_s^{\mathrm{dad}}}\left(\mathbf{n}\right)\, 
\left(\mathbf{n}+\mathbf{e}_s\right)\cdot\mathbf{e}_s\right)\mathbf{e}_s},
\end{multline*}
где $P\left(\mathbf{n}\in\Omega\right)$~--- стационарная вероятность того, что 
система находится в~состоянии $\mathbf{n}$, а $I_{\circ}\left(\ast\right)$~--- 
функ\-ция-ин\-ди\-ка\-тор\footnote{{\sf https://mathworld.wolfram.com/CharacteristicFunction.html.}}.

В связи с~реализацией механизма прерывания обслуживания запросов СП 
$\mathbf{X}\left(t\right)_{t>0}$, опи\-сы\-ва\-ющий рас\-смат\-ри\-ва\-емую сис\-те\-му, не 
является обратимым марковским процессом. В~этом случае для вы\-чис\-ле\-ния 
стационарного распределения вероятностей со\-сто\-яний сис\-те\-мы 
$\mathbf{P}\hm=\left(P\left(\mathbf{n}\right)\right)_{\mathbf{n}\hm\in\Omega}$~--- 
век\-то\-ра-столб\-ца размера $\lvert\Omega\rvert$~--- может быть применен один из 
численных методов, например итерационный метод~\cite{Zhou2022,Stepanov}:
\begin{equation*}
%\label{eq:InfinitesimalGeneratorMarkov}
\mathbf{A}^\top\, \mathbf{P} = \mathbf{0}\,,\quad \mathbf{P}\cdot \mathbf{j} = 1\,,
\end{equation*}
где $\mathbf{A}$~--- инфинитезимальная матрица размера $\lvert\Omega\rvert^2$, 
элементы $A\left(\mathbf{n},\hat{\mathbf{n}}\right)$, $\mathbf{n}\hm\in\Omega$, 
$\hat{\mathbf{n}}\hm\in\Omega$, которой определяются следующим образом:

\noindent
\begin{subequations}
%\label{eq:SolutionEquilibrium}
при $\mathbf{n}\neq\hat{\mathbf{n}}$
\begin{multline*}
A\left(\mathbf{n},\hat{\mathbf{n}}\right) = \\
\begin{cases}
\boldsymbol\lambda \cdot \mathbf{e}_s,  &\!\!\!\!\!\!\!\!  \mbox{если}\ 
\hat{\mathbf{n}}=\mathbf{n}+\mathbf{e}_s,\, 
\mathbf{n}\in\Omega_s^{\mathrm{dad}},\\
&\!\!\!\!\!\!\!\! \mbox{или}\ \hat{\mathbf{n}}=\mathbf{n}+\mathbf{e}_s-
\mathbf{z}\left(s,\mathbf{n}\right),\, \mathbf{n}\in\Omega_s^{\mathrm{vpad}};\\
\left(\mathbf{n}\odot\boldsymbol\mu\right)\cdot\mathbf{e}_s,    &\!\!\! \mbox{если}\ 
\hat{\mathbf{n}}=\mathbf{n}-\mathbf{e}_s,\, 
\hat{\mathbf{n}}\in\Omega_s^{\mathrm{dad}};\\
0           & \!\!\!\!\!\!\!\! \mbox{в противном случае},\, 
\hat{\mathbf{n}}\in\Omega\setminus\{\mathbf{n}\},
\end{cases}\\
s=1,\ldots,\lvert\mathcal{S}\rvert;
\end{multline*}
при $\mathbf{n}=\hat{\mathbf{n}}$
\begin{equation*}
A\left(\mathbf{n},\mathbf{n}\right) = -
\sum_{\hat{\mathbf{n}}\in\Omega\setminus\{\mathbf{n}\}}{A\left(\mathbf{n},\hat{\mathbf{n}}\right)}.
\end{equation*}
\end{subequations}

Рассчитав стационарное распределение вероятностей состояний системы, можно 
вычислить следующие КПЭ системы:
\begin{itemize}
\item среднее число запросов, об\-слу\-жи\-ва\-емых в~сис\-те\-ме
\setcounter{equation}{4}
\begin{equation}
\label{eq:meanN}
N = \sum\limits_{\mathbf{n}\in\Omega} P\left(\mathbf{n}\right)\: \mathbf{n} \cdot 
\mathbf{j}\,;
\end{equation}
\item вероятность блокировки запроса, поступающего в~систему
\begin{equation}
\label{eq:Pblock}
P^{\mathrm{block}} = \sum\limits_{\mathbf{n}\in \mathcal{B}}P(\mathbf{n}); \quad 
\mathcal{B} = \bigcap_{s=1}^{\lvert\mathcal{S}\rvert}\Omega_s^{\mathrm{block}}\,;
\end{equation}
\item средний занятый ресурс,
\begin{equation}\label{eq:meanK}
K = \sum\limits_{\mathbf{n}\in\Omega} P\left(\mathbf{n}\right)\: \mathbf{n} \cdot 
\mathbf{b}\,.
\end{equation}
\end{itemize}

\vspace*{-24pt}


\section{Примеры расчета вектор-функции прерывания 
обслуживания}
%\label{sec:exampleModels}

\vspace*{-12pt}

Проиллюстрируем работу итерационного метода расчета числа и~типа запросов-<<на\-ру\-ши\-те\-лей>>, 
которые должны быть прерваны для приема запроса, поступающего 
в~гарантированную часть своего слайса, на примере моделей с~двумя и~тремя 
слайсами, для которых вычислим век\-тор-функ\-цию прерывания 
обслуживания~\eqref{eq:capabilityFunc} запросов.

\vspace*{-12pt}


\subsection{Модель сети с~двумя слайсами}
%\label{subsec:2Dmodel}

\vspace*{-12pt}

Для случая $\mathcal{S}\hm=\{1,2\}$ с~учетом основных подмножеств системы~\eqref{eq:SetAdmission}--\eqref{eq:SetBlocking} 
пространство состояний модели 
имеет вид, изображенный на рис.~4.

Приведем пример расчета век\-тор-функ\-ции прерывания 
обслуживания~\eqref{eq:capabilityFunc} для приема запроса\linebreak\vspace*{-12pt}

\begin{table*}\small %tabl2 
%\vspace*{-12pt}
\begin{center}
\Caption{Исходные данные для сравнительного анализа~\cite{schoolar2019}}
\label{tab:SlicesAndCharacteristics}
\vspace*{2ex}

\tabcolsep=4pt
\begin{tabular}{|c|c|c|c|c|c|c|}
\hline
  & \multicolumn{3}{c|}{СХЕМА С РЕЗЕРВИРОВАНИЕМ (РС)} & 
\multicolumn{3}{c|}{СХЕМА С ПРЕРЫВАНИЕМ (ПС)}\\
\cline{2-7}
\multicolumn{1}{|c|}{\raisebox{6pt}[0pt][0pt]{СЛАЙС/УСЛУГА }}& Параметр  & Значение &  \tabcolsep=0pt\begin{tabular}{c}Единица\\ измерения\end{tabular}  & Параметр  & Значение & \tabcolsep=0pt\begin{tabular}{c}Единица\\ измерения\end{tabular} \\ 
\hline
\multicolumn{1}{|c|}{\raisebox{-18pt}[0pt][0pt]{1/3K Cloud VR (Game)}} & $C_1$ & 1,0 & Гбит/с & $C_1$ & 1,25 & Гбит/с \\
                       & $Q_1$ & 1,0 & Гбит/с & $Q_1$ & 1,0  & Гбит/с \\
                       & $b_1$ & 0,1 & Гбит/с & $b_1$ & 0,1  & Гбит/с \\
                       & $\mu_1^{-1}$ & 3600 & с~& $\mu_1^{-1}$ & 3600  & с~\\
\hline
\multicolumn{1}{|c|}{\raisebox{-18pt}[0pt][0pt]{2/4K Live News Pushing (30~fps)}} & $C_2$ & 0,5 & Гбит/с & $C_2$ & 0,65 & Гбит/с 
\\
                                 & $Q_2$ & 0,5 & Гбит/с & $Q_2$ & 0,5  & Гбит/с 
\\
                                 & $b_2$ & 0,04 & Гбит/с & $b_2$ & 0,04  &  Гбит/с \\
                                 & $\mu_2^{-1}$ & 1200 & с~& $\mu_2^{-1}$ & 1200  
& с~\\
\hline
\multicolumn{1}{|c|}{\raisebox{-18pt}[0pt][0pt]{3/4K On-Demand Video}} & $C_3$ & 0,5 & Гбит/с & $C_3$ & 0,65 & Гбит/с \\
                       & $Q_3$ & 0,5 & Гбит/с & $Q_3$ & 0,5  & Гбит/с \\
                       & $b_3$ & 0,03 & Гбит/с & $b_3$ & 0,03  & Гбит/с \\
                       & $\mu_3^{-1}$ & 1800 & с~& $\mu_3^{-1}$ & 1800  & с~\\
\hline
\multicolumn{1}{|c|}{\raisebox{-18pt}[0pt][0pt]{4/4K Live Broadcast Pushing (50fps)}} & $C_4$ & 0,5 & Гбит/с & $C_4$ & 0,65 & Гбит/с \\
                                      & $Q_4$ & 0,5 & Гбит/с & $Q_4$ & 0,5  & 
Гбит/с \\
                                      & $b_4$ & 0,063 & Гбит/с & $b_4$ & 0,063  
& Гбит/с \\
                                      & $\mu_4^{-1}$ & 5400 & с~& $\mu_4^{-1}$ & 
5400  & с\\
\hline
\end{tabular}
\end{center}
\vspace*{5pt}

\begin{center}
\begin{tabular}{|c|c|c|c|c|c|c|c|}
%\hline
\hline
  &  & \multicolumn{2}{c|}{СЦЕН.\ 1} & \multicolumn{2}{c|}{СЦЕН.\ 2} & 
\multicolumn{2}{c|}{СЦЕН.\ 3} \\ 
\cline{3-8}
\multicolumn{1}{|c|}{\raisebox{6pt}[0pt][0pt]{$\rho$}} &
\multicolumn{1}{c|}{\raisebox{6pt}[0pt][0pt]{ $\lambda_s$}} & СЛАЙСЫ & $C$ & СЛАЙСЫ  & $C$ & СЛАЙСЫ  & $C$ \\ 
\hline
от 1 до 25 & $\rho\mu_s$ запросов/с & 1 и~2 & 1,5 Гбит/с & 1, 2 и~3 & 2,0 Гбит/с & 1, 2, 3 и~4 & 2,5 Гбит/с \\
\hline
\end{tabular}
\end{center}
%\vspace*{-12pt}
\end{table*}









{ \begin{center}  %fig4
 \vspace*{6pt}
   \mbox{%
\epsfxsize=76.222mm
\epsfbox{adu-4.eps}
}


\vspace*{12pt}

{\small \begin{tabular}{|c|c|c|c|}
\hline
(a) & $\Omega_1^{\mathrm{dad}}\cap \Omega_2^{\mathrm{dad}}$ & (d) & $\Omega_1^{\mathrm{vpad}}$\\
(b) & $\Omega_1^{\mathrm{dad}}\backslash \Omega_2^{\mathrm{dad}}$ & (e) & $\Omega_2^{\mathrm{vpad}}$\\
(c) & $\Omega_2^{\mathrm{dad}}\backslash \Omega_1^{\mathrm{dad}}$ & (f) & $\Omega_1^{\mathrm{block}}\cap \Omega_2^{\mathrm{block}}$\\
\hline
\end{tabular}}
\end{center}

%\vspace*{6pt}

\noindent
{{\figurename~4}\ \ \small{Пространство состояний модели с~двумя слайсами с~учетом основных подмножеств системы
}}}

\vspace*{9pt}

\addtocounter{figure}{1}

\pagebreak




\noindent
 в~первый слайс
$\mathbf{z}\left(1,\mathbf{n}\right)\hm=\left(z_1\left(1,\mathbf{n}\right), 
z_2\left(1,\mathbf{n}\right)\right)$.
Рас\-смот\-рим со\-сто\-яние сис\-те\-мы
$$
\mathbf{n} = \left(\left\lfloor \left\lfloor C/b_1 \right\rfloor \left(1-
\fr{\left\lfloor C_2/b_2 \right\rfloor}{\left\lfloor C/b_2 \right\rfloor} 
\right) \right\rfloor, \left\lfloor C_2/b_2 \right\rfloor \right),
$$
%\linebreak\vspace*{-12pt}
которое согласно диаграмме на рис.~4 принадлежит множеству 
$\Omega_1^{\mathrm{vpad}}$, т.\,е.\ множеству со\-сто\-яний сис\-те\-мы, в~которых 
поступающий в~первый слайс запрос
 будет принят на обслуживание за счет 
прерывания об\-слу\-жи\-ва\-емо\-го(-ых) за\-про\-са(-ов) во втором \mbox{слайсе}.

Для исходных данных <<СЦЕН.~1>>, пред\-став\-лен\-ных 
в~табл.~2, получим $\mathbf{n}\hm = \left(8,16\right)$, 
$\mathbf{n}\hm\in\Omega_1^{\mathrm{vpad}}$.


Рассчитаем число $z_2\left(1,\mathbf{n}\right)$ об\-слу\-жи\-ва\-емых запросов 
второго слайса, которое необходимо прервать для приема одного запроса в~первый 
слайс. Воспользуемся начальным условием 
$\mathbf{z}\left(1,\mathbf{n}\right)\hm=\left(0,0\right)$, получим

\noindent
\begin{multline*}
z_2\left(1,\mathbf{n}\right) = \min
\biggl\{
R \left(\left(\mathbf{n}-\mathbf{N}^{{g}}\right) \cdot \mathbf{e}_2\right), 
\\
R \left(\left\lceil \fr{\left(\mathbf{n}+\mathbf{e}_1 \right)\cdot\mathbf{b} - 
C}{\mathbf{e}_2\cdot\mathbf{b}} \right\rceil\right)
\biggl\} =
\min \left\{ R \left(4\right),R \left(1\right) \right\} = {}\\
{}= \min \left\{4,1\right\} = 1.
\end{multline*}

Перейдем к~расчету числа $z_1\left(1,\mathbf{n}\right)$ об\-слу\-жи\-ва\-емых 
запросов первого слайса, которое необходимо прервать для приема одного запроса в~первый слайс. Очевидно, что число $z_1\left(1,\mathbf{n}\right)$ должно 
быть равно~$0$, так как запрос не может быть принят в~слайс на обслуживание за 
счет прерывания об\-слу\-жи\-ва\-емо\-\mbox{го(-ых)} за\-про\-са(-ов) в~этом же слайсе. 
Воспользуемся текущим значением 
$\mathbf{z}\left(1,\mathbf{n}\right)\hm=\left(0,z_2\left(1,\mathbf{n}\right)\right)
=\left(0,1\right)$, получим

\vspace*{-6pt}

\noindent
\begin{multline*}
z_1\left(1,\mathbf{n}\right) =
\min
\biggl\{
R \left(\left(\mathbf{n}-\mathbf{N}^{{g}}\right) \cdot \mathbf{e}_1\right), 
\\
R \left(\left\lceil \frac{\left(\mathbf{n}+\mathbf{e}_1 - 
\mathbf{z}\left(1,\mathbf{n}\right) \right)\cdot\mathbf{b} - 
C}{\mathbf{e}_1\cdot\mathbf{b}} \right\rceil\right)
\biggl\} = {}\\
{}= \min \left\{R \left(-2\right),R \left(0\right) \right\} = \min 
\left\{0,0\right\} = 0\,.
\end{multline*}

\vspace*{-4pt}

Таким образом, в~состоянии системы $\mathbf{n}\hm=\left(8,16\right)$, 
$\mathbf{n}\hm\in\Omega_1^{\mathrm{vpad}}$, век\-тор-функ\-ция прерывания 
$\mathbf{z}\left(1,\mathbf{n}\right)\hm=\left(z_1\left(1,\mathbf{n}\right),z_2\left
(1,\mathbf{n}\right)\right)\hm=\left(0,1\right)$,
т.\,е.\ поступающий в~первый слайс запрос будет принят на обслуживание за счет 
прерывания одного запроса, об\-слу\-жи\-ва\-емо\-го во втором слайсе.


\subsection{Модель сети с~тремя слайсами}

Перейдем к~трехмерному случаю $\mathcal{S}=\{1,2,3\}$. Рассмотрим пример расчета 
век\-тор-функ\-ции прерывания обслуживания~\eqref{eq:capabilityFunc} для приема 
запроса во второй слайс 
$\mathbf{z}\left(2,\mathbf{n}\right)\hm=\left(z_1\left(2,\mathbf{n}\right), 
z_2\left(2,\mathbf{n}\right), z_3\left(2,\mathbf{n}\right)\right)$, 
$\mathbf{n}\hm\in\Omega_2^{\mathrm{vpad}}$.

Сведем исходные данные для примера в~табл.~3.

\pagebreak

%\begin{table*}\small %tabl3
\noindent
{{\tablename~3}\ \ \small{Исходные данные для примера с~тремя слайсами
}}
%\label{tab:NumExampleParams3D}

\vspace*{3pt}

\begin{center}
{\small 
\tabcolsep=11pt
\begin{tabular}{|c|c|c|c|}
\hline
$C$, Мбит/с & $\mathbf{b}$, Мбит/с& $\mathbf{n}$ & $\mathbf{N}^g$\\
\hline
13& (1,3,1) & (6,1,4)& (2,2,2)\\
\hline
\end{tabular}
}
\end{center}
%\end{table*}

\vspace*{6pt}

\addtocounter{table}{1}


Рассчитаем число $z_3\left(2,\mathbf{n}\right)$ об\-слу\-жи\-ва\-емых запросов третьего 
слайса~--- слайса с~низшим приоритетом, которое необходимо прервать для приема 
одного запроса во второй слайс. Воспользуемся начальным условием 
$\mathbf{z}\left(2,\mathbf{n}\right)\hm=\left(0,0,0\right)$, получим
\begin{multline*}
z_3\left(2,\mathbf{n}\right) = \min
\biggl\{
R \left(\left(\mathbf{n}-\mathbf{N}^{{g}}\right) \cdot \mathbf{e}_3\right), 
\\
R \left(\left\lceil \fr{\left(\mathbf{n}+\mathbf{e}_2 \right)\cdot\mathbf{b} - 
C}{\mathbf{e}_3\cdot\mathbf{b}} \right\rceil\right)
\biggl\} =
\min \left\{ R \left(2\right),R \left(3\right) \right\} ={} \\
{}= \min \left\{2,3\right\} = 2.
\end{multline*}

Покажем, что число $z_2\left(2,\mathbf{n}\right)$ об\-слу\-жи\-ва\-емых запросов второго 
слайса, которое необходимо прервать для приема одного запроса во второй слайс, 
равно~$0$. Воспользуемся текущим значением функции 
$\mathbf{z}\left(2,\mathbf{n}\right)\hm=\left(0,0,z_3\left(2,\mathbf{n}\right)\right)=\left(0,0,2\right)$, получим
\begin{multline*}
z_2\left(2,\mathbf{n}\right) = \min
\biggl\{
R \left(\left(\mathbf{n}-\mathbf{N}^{{g}}\right) \cdot \mathbf{e}_2\right), 
\\
R \left(\left\lceil \fr{\left(\mathbf{n}+\mathbf{e}_2 - 
\mathbf{z}\left(2,\mathbf{n}\right) \right)\cdot\mathbf{b} - 
C}{\mathbf{e}_2\cdot\mathbf{b}} \right\rceil\right)
\biggl\} ={} \\
{}= \min \left\{R \left(-1\right),R \left(1\right) \right\} = \min 
\left\{0,1\right\} = 0.
\end{multline*}

Согласно схеме управления доступом ПС при исчерпании возможности освободить 
ресурс за счет прерывания запросов-<<нарушителей>> более низкого приоритета для 
приема запроса в~гарантированную часть второго слайса должно быть прервано 
обслуживание за\-про\-сов-<<на\-ру\-ши\-те\-лей>> более высокого приоритета. Так как число 
об\-слу\-жи\-ва\-емых запросов первого слайса превышает гарантированное значение, для 
приема одного запроса во второй слайс должно быть прервано обслуживание 
$z_1\left(2,\mathbf{n}\right)$ запросов первого слайса. С~учетом текущего 
значения век\-тор-функ\-ции прерывания 
$$
\mathbf{z}\left(2,\mathbf{n}\right)=\left(0, 
z_2\left(2,\mathbf{n}\right), 
z_3\left(2,\mathbf{n}\right)\right)\hm=\left(0,0,2\right)
$$ 
получим
\begin{multline*}
z_1\left(2,\mathbf{n}\right) = \min
\biggl\{
R \left(\left(\mathbf{n}-\mathbf{N}^{{g}}\right) \cdot \mathbf{e}_1\right), 
\\
R \left(\left\lceil \fr{\left(\mathbf{n}+\mathbf{e}_2 - 
\mathbf{z}\left(2,\mathbf{n}\right) \right)\cdot\mathbf{b} - 
C}{\mathbf{e}_1\cdot\mathbf{b}} \right\rceil\right)
\biggl\} ={} \\
{}= \min \left\{R \left(4\right),R \left(1\right) \right\} = \min 
\left\{4,1\right\} = 1.
\end{multline*}
Таким образом, в~состоянии системы $\mathbf{n}\hm=\left(6,1,4\right)$, 
$\mathbf{n}\hm\in\Omega_2^{\mathrm{vpad}}$, век\-тор-функ\-ция прерывания обслуживания 
запросов имеет вид 
$$
\mathbf{z}\left(2,\mathbf{n}\right)=\left(z_1\left(2,\mathbf{n}\right), 
z_2\left(2,\mathbf{n}\right), 
z_2\left(2,\mathbf{n}\right)\right)=\left(1,0,2\right),
$$
 т.\,е.\ поступающий во 
второй слайс запрос будет принят на обслуживание за счет прерывания двух 
запросов, об\-слу\-жи\-ва\-емых в~третьем слайсе, и~одного~--- в~первом.

Далее перейдем к~анализу основных КПЭ сис\-те\-мы, описанных в~разд.~3.



\section{Численный анализ}
%\label{sec:numericalanalysis}

Для анализа эффективности предложенной в~работе схемы доступа ПС к~радиоресурсам сети, основанной на реализации механизма прерывания обслуживания 
пользователей, проведем \mbox{сравнительный} анализ ее основных КПЭ с~КПЭ известной 
схемы доступа РС, основанной на реализации механизма резервирования ресурсов. 
Сведем исходные данные для численного анализа 
в~табл.~\ref{tab:SlicesAndCharacteristics}.
Рас\-смот\-рим зависимость среднего числа запросов,\linebreak об\-слу\-жи\-ва\-емых в~сис\-те\-ме,~\eqref{eq:meanN}, вероятности блокировки запроса любого типа, 
поступающего в~сис\-те\-му,~\eqref{eq:Pblock} (вероятности блокировки сис\-те\-мы) и~среднего чис\-ла занятых ресурсов в~сис\-те\-ме~\eqref{eq:meanK} от \mbox{интенсивности} 
предложенной нагрузки~$\rho$, созда\-ва\-емой в~каж\-дом слайсе. Положим $\rho_s=\rho, 
s\in\mathcal{S}$. Результаты сравнительного анализа схем ПС и~РС представлены 
на~рис.~5 и~6.




На рис.~5,\,\textit{а} проиллюстрировано поведение среднего числа запросов~$N$, 
об\-слу\-жи\-ва\-емых в~сис\-те\-ме, в~за\-ви\-си\-мости от интенсивности предложенной нагрузки~$\rho$ для трех сценариев~--- с~двумя, тремя и~четырьмя слайсами. По графику 
видно, что чем больше слайсов в~сети, тем выше сред\-нее чис\-ло об\-слу\-жи\-ва\-емых 
запросов и~тем эф\-фек\-тив\-нее схема ПС, в~част\-ности (см.\ рис.~6,\,\textit{а}) 
для <<СЦЕН.~1>> эф\-фек\-тив\-ность схемы ПС может превышать эф\-фек\-тив\-ность схемы РС в~1,06~раза, а~для <<СЦЕН.~3>>~--- в~1,12 раза.



С ростом интенсивности предложенной нагрузки~$\rho$ вероятность блокировки 
системы~$P^{\mathrm{block}}$ увеличивается для всех рассматриваемых сценариев 
(см.\ рис.~5,\,\textit{б}), причем она минимальна для сценария с~наибольшем числом 
слайсов (<<СЦЕН.~3>>). Однако на рис.~5,\,\textit{б} и~6,\,\textit{б} 
видно, что схема ПС эффективнее схемы РС только в~диапазоне небольших нагрузок 
на сис\-те\-му, в~част\-ности (см.\ рис.~6,\,\textit{б}) для <<СЦЕН.~2>> 
и~<<СЦЕН.~3>>~--- диапазон от~1 до~4,43.



Анализ  рис.~5,\,\textit{в} и~6,\,\textit{в} показывает, что сис\-те\-ма более 
эффективно использует ресурсы при применении схемы ПС. В~част\-ности 
(см.\ рис.~6,\,\textit{в}), сред\-нее чис\-ло ресурсов, занятых в~сис\-те\-ме, при 
использовании схемы ПС для <<СЦЕН.~1>> может быть в~1,05~раза выше, чем при 
использовании схемы РС, а~для <<СЦЕН.~3>>~--- в~1,12~раза.

\end{multicols}

\begin{figure*} %fig5
\vspace*{1pt}
\begin{minipage}[t]{80mm}
\begin{center}
   \mbox{%
\epsfxsize=79mm
\epsfbox{adu-5.eps}
}
\end{center}
\vspace*{-9pt}
\Caption{Графики зависимостей среднего числа запросов, 
об\-слу\-жи\-ва\-емых в~сис\-те\-ме~(\textit{а}), вероятности блокировки 
системы~(\textit{б}) и~среднего занятого ресурса~(\textit{в})
от интенсивности предложенной нагрузки:
\textit{1}~--- СЦЕН.~1; \textit{2}~--- СЦЕН.~2;
\textit{3}~--- СЦЕН.~3;
сплошные кривые~--- ПС; штриховые кривые~--- РС} 
\label{fig:meanN}
\end{minipage}
%\end{figure*}
\hfill
%\begin{figure*} %fig6
\vspace*{1pt}
\begin{minipage}[t]{80mm}
\begin{center}
   \mbox{%
\epsfxsize=79.102mm
\epsfbox{adu-6.eps}
}
\end{center}
\vspace*{-9pt}
\Caption{Графики зависимостей соотношений средних чисел 
об\-слу\-жи\-ва\-емых в~сис\-те\-ме запросов~(\textit{а}),
вероятности блокировки 
системы~(\textit{б}) и~среднего занятого ресурса~(\textit{в})
  для двух схем от интенсивности предложенной 
нагрузки: \textit{1}~--- СЦЕН.~1; \textit{2}~--- СЦЕН.~2;
\textit{3}~--- СЦЕН.~3}
 \label{fig:meanNRatio}
 \end{minipage}
\end{figure*}


\begin{multicols}{2}



\section{Заключение}
%\label{sec:conclusion}

В работе предложена схема доступа запросов пользователей к~ресурсам беспроводной 
сети, основанная на реализации механизма прерывания обслуживания пользователей в~рамках технологии нарезки радиоресурсов сети Network Slicing. 

Проведен 
сравнительный анализ, по\-ка\-зы\-ва\-ющий эффективность предложенной схемы по сравнению с~известной схемой доступа, основанной на механизме резервирования ресурсов.  
Результаты численного эксперимента показали, что в~диапазоне небольших нагрузок 
на сис\-те\-му предложенная схема эффективнее схемы доступа с~реализацией механизма 
резервирования.
Результаты численного эксперимента выделяют сле\-ду\-ющие особенности применения 
предложенной схемы до\-сту\-па по сравнению со схемой на основе реализации механизма 
резервирования:
\begin{enumerate}[(1)]
\item 
эффективность в~диапазоне небольших нагрузок в~использовании физических 
ресурсов БС и~емкостей слайсов;
\item 
значимое повышение эффективности при поддержке предоставления услуг для 
большего числа слайсов.
\end{enumerate}


{\small\frenchspacing
 {%\baselineskip=10.8pt
 %\addcontentsline{toc}{section}{References}
 \begin{thebibliography}{99}


\bibitem{3gpp.22.864} %1
\Au{Sultan A., Pope~M.} Feasibility study on new 
services and markets technology enablers for network operation; Stage~1 (3GPP). 
Ver. 15.0.0, 2016. 
{\sf https://portal.3gpp. org/desktopmodules/Specifications/SpecificationDetail s.aspx?specificationId=3016}.

\bibitem{3gpp.21.916} %2
\Au{Meredith J., Firmin~F., Pope~M.} Release 16 
Description; Summary of Rel-16 Work Items (3GPP). Ver. 16.2.0, 2022. 
{\sf https://portal.3gpp.org/desktopmodules/Specifications /SpecificationDetails.aspx?specificationId=3493}.
%Pravila tsitirovaniya istochnikov [Rules for the 
%citing of sources]. Available at: http://www.scribd.com/doc/1034528/ (accessed 
%February 7, 2011).

\bibitem{3gpp.28.554} %3
\Au{Meredith J., Soveri~M., Pope~M.} Management and 
orchestration; 5G end to end Key Performance Indicators (KPI) (3GPP). Ver. 
18.0.0, 2022. 
{\sf https://portal.3gpp. org/desktopmodules/Specifications/SpecificationDetail s.aspx?specificationId=3415}.

\bibitem{Yarkina2022} %4
\Au{Yarkina N., Correia~L., Moltchanov~D., Gaidamaka~Y., Samouylov~K.} Multi-tenant resource sharing with equitable-priority-based 
performance isolation of slices for 5G cellular systems~// Comput. Commun., 
2022. Vol.~188. P.~39--51. doi: 10.1016/j.comcom.2022.02.019.



\bibitem{Luu2022} %5
\Au{Luu Q., Kerboeuf~S., Kieffer~M.} Admission control and 
resource reservation for prioritized slice requests with guaranteed SLA under 
uncertainties~// IEEE T. Netw. Serv. Man., 2022. Vol.~19. P.~3136--3153. 
doi: 10.1109/ tnsm.2022.3160352.

\bibitem{Rehman2022} %6
\Au{Rehman A., Mahmood~I., Kamran~M., Sanaullah~M., Ijaz~A., Ali~J., Ali~M.} 
Enhancement in quality-of-services using 5G cellular network 
using resource reservation protocol~// Phys. Commun.~--- Amst., 2022. Art.~101907. 10~p. doi: 
10.1016/j.phycom.2022.101907.







\bibitem{kermani1977analysis} %7
\Au{Kermani P., Kleinrock~L.} Analysis of 
buffer allocation schemes in a~multiplexing node~// Int. Conf. 
Comm., 1977. Vol.~2. P.~30--34.

\bibitem{Kamoun1980} %8
\Au{Kamoun F., Kleinrock~L.} Analysis of shared finite 
storage in a~computer network node environment under general traffic conditions~// 
IEEE T. Commun., 1980. Vol.~28. P.~992--1003. doi: 
10.1109/tcom.1980.1094756.

\bibitem{Basharin1982} %9
\Au{Башарин Г.\,П., Самуйлов~К.\,Е.} Об оптимальной 
структуре БП в~сетях передачи данных с~коммутацией пакетов.~--- М.: ВИНИТИ, 
1982. 70~с.


\bibitem{Basharin2013} %10
\Au{Basharin G., Gaidamaka~Y., Samouylov~K.} 
Mathematical theory of teletraffic and its application to the analysis of 
multiservice communication of next generation networks~// Autom. Control Comp.~S., 2013. Vol.~47. P.~62--69. doi: 10.3103/s0146411613020028.





\bibitem{Stepanov} %11
\Au{Степанов С.\,Н.} Теория телетрафика: концепции, модели, 
приложения.~--- М.: Горячая лин\-ия-Телеком, 2015. 868~с.

\bibitem{Zhou2022} %12
\Au{Zhou D., Chen~Z., Pan~E., Zhang~Y.} Dynamic 
statistical responses of gear drive based on improved stochastic iteration 
method~// Appl. Math. Model., 2022. Vol.~108. P.~46--65. doi: 
10.1016/j.apm.2022.03.020.

\bibitem{schoolar2019} %13
\Au{Schoolar~D., Lambert~P., Nanbin~W., Liang~Z.} 5G 
service experience-based network planning criteria (Ovum Consulting)~// 
Partnership with Huawei, 2019. 
{\sf https://carrier. huawei.com/$\sim$/media/CNBGV2/download/products/\linebreak servies/5G-Planning-Criteria-White-Paper.pdf}.

\end{thebibliography}

 }
 }

\end{multicols}

\vspace*{-6pt}

\hfill{\small\textit{Поступила в~редакцию 15.01.23}}

\vspace*{8pt}

%\pagebreak

%\newpage

%\vspace*{-28pt}

\hrule

\vspace*{2pt}

\hrule

%\vspace*{-2pt}

\def\tit{PREEMPTION-BASED PRIORITIZATION SCHEME\\ FOR~NETWORK RESOURCES SLICING IN~5G~SYSTEMS}


\def\titkol{Preemption-based prioritization scheme for~network resources slicing in~5G~systems}


\def\aut{K.\,Y.\,B.~Adou$^1$, E.\,V.~Markova$^1$, Yu.\,V.~Gaidamaka$^{1,2}$, and~S.\,Ya.~Shorgin$^2$}

\def\autkol{K.\,Y.\,B.~Adou, E.\,V.~Markova, Yu.\,V.~Gaidamaka, and~S.\,Ya.~Shorgin}

\titel{\tit}{\aut}{\autkol}{\titkol}

\vspace*{-8pt}


\noindent
$^1$Peoples' Friendship University of Russia (RUDN University), 6~Miklukho-Maklaya Str., Moscow 117198, Russian\linebreak
$\hphantom{^1}$Federation

\noindent
$^2$Federal Research Center ``Computer Science and Control'' of the Russian Academy of Sciences; 
44-2~Vavilov\linebreak
$\hphantom{^1}$Str., Moscow 119133, Russian Federation

\def\leftfootline{\small{\textbf{\thepage}
\hfill INFORMATIKA I EE PRIMENENIYA~--- INFORMATICS AND
APPLICATIONS\ \ \ 2023\ \ \ volume~17\ \ \ issue\ 1}
}%
 \def\rightfootline{\small{INFORMATIKA I EE PRIMENENIYA~---
INFORMATICS AND APPLICATIONS\ \ \ 2023\ \ \ volume~17\ \ \ issue\ 1
\hfill \textbf{\thepage}}}

\vspace*{3pt} 



\Abste{The network slicing (NS) technology, which has been actively studied in recent years, is based on the representation of 
a~common network infrastructure in the form of various customizable logical networks called slices and involves the division of mobile 
network operators into two groups~--- physical network infrastructure providers (InPs) and mobile virtual network operators (MVNOs). 
The MVNOs lease the physical resources of InPs\linebreak\vspace*{-12pt}}

\Abstend{to create their own slices to provide services to their users with different quality of service 
requirements. In the present paper, for a~network with NS technology, a~scheme for accessing its radio resources is proposed that
 provides users with services with a~guaranteed bit rate (GBR) and priority control based on the implementation of the user service interruption mechanism. 
 The authors propose a~scheme for accessing radio resources of a~network under NS technology that provides users with services with GBR
  and priority control based on the implementation of the user service interruption mechanism. 
 To evaluate the effectiveness of the proposed scheme, a~comparative analysis of its characteristics with the characteristics of the
  access scheme based on the resource reservation mechanism was carried out.}

\KWE{5G; network slicing; quality of service; key performance indicators; priority management; service interruption; iterative method}



 \DOI{10.14357/19922264230113} 

\vspace*{-16pt}


\Ack

%\vspace*{-4pt}


\noindent
The research was supported by the Russian Science Foundation grant No.\,22-79-10053, {\sf https://rscf.ru/en/\linebreak project/22-79-10053/}.
% (Conceptualization, Y.\,A., E.\,M. and Y.\,G.; Data curation, Y.\,A.; Formal analysis, Y.\,A.; Funding acquisition, E.\,M.; 
 %Investigation, Y.\,A.; Methodology, Y.\,A., E.\,M., and Y.\,G.; Project administration, 
% E.\,M. and Y.\,G.; Resources, Y.\,A.; Software, Y.\,A.; Supervision, E.\,M. and Y.\,G.; Validation, 
% Y.\,A.; Visualization, Y.\,A. and E.\,M.; Writing~--- original draft, Y.\,A.; Writing~--- review and editing, Y.\,A., E.\,M., and Y.\,G.). 

  

%\vspace*{4pt}

  \begin{multicols}{2}

\renewcommand{\bibname}{\protect\rmfamily References}
%\renewcommand{\bibname}{\large\protect\rm References}

{\small\frenchspacing
 {%\baselineskip=10.8pt
 \addcontentsline{toc}{section}{References}
 \begin{thebibliography}{99} 

\bibitem{2-adu} %1
\Aue{Sultan, A., and M.~Pope.}
%3GPP TR 22.864. v. 15.0.0. 
2016. Feasibility study on new services and markets technology enablers for network operation; Stage~1 (3GPP). Ver. 15.0.0. Available at: 
{\sf https://\linebreak portal.3gpp.org/desktopmodules/Specifications/Specifi cationDetails.aspx?specificationId=3016} (accessed January~30, 2023).

\bibitem{1-adu} %2
\Aue{Meredith, J., F.~Firmin, and M.~Pope.}
%3GPP TR 21.916. v. 16.2.0. 
2022. Release 16~description; Summary of Rel-16 work items (3GPP).  Ver. 16.2.0. Available at: 
{\sf https://portal.3gpp.org/desktop\linebreak  modules/Specifications/SpecificationDetails.aspx?speci\linebreak ficationId=3493} (accessed January~30, 2023).
\bibitem{3-adu} %3
\Aue{Meredith, J., M.~Soveri, and M.~Pope.}
%3GPP TR 28.554. v. 18.0.0. 
2022. Management and orchestration; 5G end to end Key Performance Indicators (KPI) (3GPP). Ver. 18.0.0. Available at:
{\sf  https://\linebreak portal.3gpp.org/desktopmodules/Specifications/Specifi\linebreak cationDetails.aspx?specificatio nId=3415} (accessed January~30, 2023).
\bibitem{4-adu}
\Aue{Yarkina, N., L.\,M.~Correia, D.~Moltchanov, Y.~Gaidamaka, and K.~Samouylov.}
 2022. Multi-tenant resource sharing with equitable-priority-based performance isolation of slices for 5G cellular systems. 
 \textit{Comput. Commun.} 188:39--51. doi: 10.1016/j.comcom.2022.02.019.
\bibitem{5-adu}
\Aue{Luu, Q., S.~Kerboeuf, and M.~Kieffer.} 
2022. Admission control and resource reservation for prioritized slice requests with guaranteed SLA under uncertainties. 
\textit{IEEE T. Netw. Serv. Man.} 19:3136--3153. doi: 10.1109/ tnsm.2022.3160352.
\bibitem{6-adu}
\Aue{Rehman, A., I.~Mahmood, M.~Kamran, M.~Sanaullah, A.~Ijaz, J.~Ali, and M.~Ali.}
 2022. Enhancement in quality-of-services using 5G cellular network using resource reservation protocol. \textit{Phys. Commun.~--- Amst.}
  55:101907. 10~p. doi: 10.1016/j.phycom.2022.101907. 

\bibitem{8-adu} %7
\Aue{Kermani, P., and L.~Kleinrock.}
 1977. Analysis of buffer allocation schemes in a~multiplexing node. \textit{Int. Conf. Comm.} 2:30--34.
\bibitem{9-adu} %8
\Aue{Kamoun, F, and L.~Kleinrock.}
 1980. Analysis of shared finite storage in a~computer networks node environment under general traffic conditions. 
 \textit{IEEE T. Commun.} 28(7):992--1003. doi: 10.1109/tcom.1980.1094756.
 
 \bibitem{7-adu} %9
\Aue{Basharin, G.\,P., and K.\,E.~Samouylov.}
 1982. \textit{Ob op\-ti\-mal'\-noy struk\-tu\-re bu\-fer\-noy pa\-mya\-ti v~se\-tyakh pe\-re\-da\-chi dan\-nykh 
 s~kom\-mu\-ta\-tsi\-ey pa\-ke\-tov} [On the optimal structure of BP in data transmission networks with packet commutation]. Moscow: VINITI. 70~p.
 
\bibitem{10-adu}
\Aue{Basharin, G.\,P., Yu.\,V.~Gaidamaka, and K.\,E.~Samouylov}.
 2013. Mathematical theory of teletraffic and its application to the analysis of multiservice communication of next generation networks. 
 \textit{Autom. Control Comp.~S.} 47(2):62--69. doi: 10.3103/s0146411613020028.

\bibitem{12-adu} %11
\Aue{Stepanov, S.\,N.} 2015. \textit{Teo\-riya te\-le\-tra\-fi\-ka: Kon\-tsep\-tsii, mo\-de\-li, pri\-lo\-zhe\-niya} 
[Theory of teletraffic: Concepts, models, and applications]. Moscow: Goryachaya liniya-Telekom. 868~p.

\bibitem{11-adu} %12
\Aue{Zhou, D., Z.~Chen, E.~Pan, and Y.~Zhang.} 
2022. Dynamic statistical responses of gear drive based on improved stochastic iteration method. \textit{Appl. Math. Model.} 108:46--65.
doi: 10.1016/j.apm.2022.03.020.
\bibitem{13-adu}
\Aue{Schoolar, D., P.~Lambert, W.~Nanbin, and Z.~Liang.} 2019. 
5G service experience-based network planning criteria  (Ovum Consulting). Partnership with Huawei. Available at: 
{\sf https://carrier.huawei.com/$\sim$/media/\linebreak CNBGV2/download/products/servies/5G-Planning-Criteria-White-Paper.pdf} (accessed January~30, 2023).
  \end{thebibliography}

 }
 }

\end{multicols}

\vspace*{-6pt}

\hfill{\small\textit{Received January 15, 2023}}

\vspace*{-14pt}

\Contr

\vspace*{-3pt}

\noindent
\textbf{Adou Kpangny Y.\,B.} (b.\ 1993)~--- 
PhD student, research assistant, Department of Applied Probability and Informatics, Peoples' Friendship University of Russia (RUDN University), 
6~Miklukho-Maklaya Str., Moscow 117198, Russian Federation; \mbox{adu-k@rudn.ru}

\pagebreak

\noindent
\textbf{Markova Ekaterina V.} (b.\ 1987)~--- 
Candidate of Science (PhD) in physics and mathematics, assistant professor, Department of Applied Probability and Informatics, 
Peoples' Friendship University of Russia (RUDN University), 6~Miklukho-Maklaya Str., Moscow 117198, Russian Federation; 
\mbox{markova-ev@rudn.ru}

\vspace*{6pt}

\noindent
\textbf{Gaidamaka Yuliya V.} (b.\ 1971)~--- 
Doctor of Science in physics and mathematics, professor, Department of Applied Probability and Informatics, Peoples' 
Friendship University of Russia (RUDN University), 6~Miklukho-Maklaya Str., Moscow 117198, Russian Federation; 
senior scientist, Institute of Informatics Problems, Federal Research Center ``Computer Science and Control'' 
of the Russian Academy of Sciences, 44-2~Vavilov Str., Moscow 119333, Russian Federation; \mbox{gaydamaka-yuv@rudn.ru}

\vspace*{6pt}

\noindent
\textbf{Shorgin Sergey Ya.} (b.\ 1952)~--- 
Doctor of Science in physics and mathematics, professor, principal scientist, Institute of Informatics Problems, Federal Research Center 
``Computer Science and Control'' of the Russian Academy of Sciences, 44-2~Vavilov Str., Moscow 119133, Russian Federation; 
\mbox{sshorgin@ipiran.ru}



   
\label{end\stat}

\renewcommand{\bibname}{\protect\rm Литература} 
   