\def\stat{bosov}

\def\tit{О ЗАДАЧЕ ОЦЕНКИ И~АНАЛИЗА РИСКА ТРАНСПОРТНЫХ ПРОИСШЕСТВИЙ  НА~РЕЛЬСОВОМ ТРАНСПОРТЕ$^*$}

\def\titkol{О задаче оценки и~анализа риска транспортных происшествий на~рельсовом транспорте}

\def\aut{А.\,В.~Босов$^1$, А.\,Н.~Игнатов$^2$}

\def\autkol{А.\,В.~Босов, А.\,Н.~Игнатов}

\titel{\tit}{\aut}{\autkol}{\titkol}

\index{Босов А.\,В.}
\index{Игнатов А.\,Н.}
\index{Bosov A.\,V.}
\index{Ignatov A.\,N.}


{\renewcommand{\thefootnote}{\fnsymbol{footnote}} \footnotetext[1]
{Работа выполнена при поддержке РФФИ (проект 20-07-00046~А).}}


\renewcommand{\thefootnote}{\arabic{footnote}}
\footnotetext[1]{Федеральный исследовательский центр <<Информатика и~управ\-ле\-ние>> Российской академии наук; 
Московский авиационный институт, \mbox{avbosov@ipiran.ru}}
\footnotetext[2]{Московский авиационный институт, \mbox{alexei.ignatov1@gmail.com}}

\vspace*{-6pt}


  
  
      
      \Abst{Рассматривается задача по оцениванию и~анализу риска транспортных 
происшествий на рельсовом транспорте. Предлагаются две функции интегрального риска, 
поз\-во\-ля\-ющие оценить опас\-ность движения на всем маршруте следования транспортного 
средства. В~качестве таких функций выбираются ве\-ро\-ят\-ность возникновения неблагоприятного 
события при транспортировке, а~так\-же сред\-ний ущерб. Предлагается концепция оценивания 
ве\-ро\-ят\-ности и~ущерба от неблагоприятных событий при движении грузовых поездов. На основе 
ранее обработанной статистики по движению грузовых поездов и~происходивших с ними 
неблагоприятных событий приводится содержательный пример расчета функций интегрального 
риска.}
       
      \KW{риск; неблагоприятное событие; рельсовый транспорт; вероятность; средний ущерб}

\DOI{10.14357/19922264230110}  
  
%\vspace*{-4pt}


\vskip 10pt plus 9pt minus 6pt

\thispagestyle{headings}

\begin{multicols}{2}

\label{st\stat}
      
\section{Введение}

    Типовой интерпретацией риска служит отклонение в~большую или 
меньшую сторону от целевого значения некоторого показателя вследствие 
неопределенностей~[1]. В~данной терминологии у~поня\-тия присутствует не 
только негативная коннотация, а~риск ничем не отличается от 
 убыт\-ка/при\-бы\-ли от ведения некоторой де\-я\-тель\-ности. В~то же время 
 в~межгосударственном стандарте~[2] постулируется, что риск~--- это сочетание 
ве\-ро\-ят\-ности нанесения ущер\-ба и~тя\-жести этого ущерба. Именно этим понятием 
рис\-ка и~будем пользоваться, не учитывая доход.

    В одной из первых работ, посвященных прогнозированию и~оценке 
по\-след\-ст\-вий железнодорожных происшествий, рас\-смат\-ри\-ва\-лась задача оценки 
влияния номера первой сошедшей с~рельсов по\-движ\-ной единицы (вагонов 
и~секций локомотива) на чис\-ло сошедших с~рельсов подвижных единиц~[3]. 
Предлагалось проводить оптимизацию расположения вагонов с~опас\-ны\-ми грузами 
с~\mbox{целью} уменьшения ве\-ро\-ят\-ности схода с~рельсов именно этих вагонов. 
В~дальнейшем~[4, 5] была предложена оценка ве\-ро\-ят\-ности схода с~рельсов, 
зависевшая от длины пути и~чис\-ла по\-движ\-ных единиц в~со\-ста\-ве поезда. В~[6] для 
прогнозирования чис\-ла по\-движ\-ных единиц в~сходе с~рельсов было использовано 
отрицательное биномиальное распределение. В~[7] была рас\-смот\-ре\-на похожая 
на~[6] модель, но с~б$\acute{\mbox{о}}$льшим числом факторов. В~[8] была 
рассмотрена задача прогнозирования дополнительного негативного последствия 
от схода с~рельсов помимо собственно повреждения подвижного со\-ста\-ва~--- 
выхода в~габарит соседнего пути хотя бы одной сошедшей с~рельсов по\-движ\-ной 
единицы. Такое событие влечет остановку движения на соседнем пути и,~кроме 
того, может привести к~столкновению со встречным поездом. 

Наиболее близ\-ки 
к~теме данной \mbox{статьи} работы~[9, 10]. В~них строится функция интегрального 
(общего на всем пути следования) риска с~\mbox{целью} оптимального расположения 
вагонов с~опасными грузами с~точ\-ки зрения минимизации суммарного рис\-ка. 
Данная функция зависит от ско\-рости (при итоговом расчете используется не 
реальная ско\-рость поезда в~пути, а~максимально допустимая на участ\-ке), а~также 
зависит от номера первой сошедшей с~рельсов по\-движ\-ной единицы и~причины 
схода с~рельсов. Следует отметить, что сход с~рельсов по конкретной причине~--- 
случайное событие, однако учет этого обстоятельства (например, при помощи 
формулы полной ве\-ро\-ят\-ности) в~[9, 10] опускается. Другие факторы, например 
профиль и~кривизна пути, не используются, а~это, в~свою очередь, снижает 
точ\-ность полученного результата. Кроме того, в~[9, 10] поезд пред\-став\-ля\-ет\-ся 
материальной точ\-кой, для которой не учитывается, что голова поезда может 
находиться, например, на стрелочном переводе, а~хвост еще нет. Также не 
рас\-смат\-ри\-ва\-ют\-ся другие помимо схода с~рельсов неблагоприятные события, 
которые могут произойти при транспортировке, а~в~качестве функции риска не 
по\-сту\-ли\-ру\-ет\-ся ве\-ро\-ят\-ность отсутствия всех неблагоприятных событий при 
движении. Устранению этих, а~так\-же некоторых других недостатков по\-свя\-ще\-на 
на\-сто\-ящая статья.

    Рассматривается задача построения функций интегрального риска, 
характеризующих опас\-ность возникновения различных неблагоприятных событий 
при движении на рельсовом транспорте. Предлагаются две функции 
интегрального риска: ве\-ро\-ят\-ность отсутствия всех неблагоприятных событий при 
движении и~сред\-ний ущерб при транспортировке. Пред\-став\-ле\-на концепция оценки 
ве\-ро\-ят\-ности и~ущерба неблагоприятных событий при движении грузовых поездов. 
На содержательном примере показывается важ\-ность выбора скоростного режима 
применительно к~снижению интегрального рис\-ка в~виде среднего ущерба.

\vspace*{-9pt}

\section{Постановка задачи}

\vspace*{-3pt}

    Пусть по рельсам по заданному маршруту длиной $S$~мет\-ров движется 
некоторое транспортное средство (трамвай, грузовой поезд или вагонетка). При 
движении происходит управ\-ле\-ние ско\-ростью: пусть~$v_s$~--- ско\-рость на $s$-м 
мет\-ре дистанции, $s\hm=\overline{1,S}$. Для упрощения модель допускает 
мгновенное изменение ско\-рости на участке. При этом ско\-рость считается 
заданной детерминированной величиной. Пусть на каж\-дом мет\-ре дистанции при 
движении могут произойти~$n$~неблагоприятных событий (транспортных 
происшествий). Обозначим через~$A_{si}$ событие, за\-клю\-ча\-юще\-еся в~том, что на 
$s$-м мет\-ре маршрута произойдет $i$-е событие, $s\hm= \overline{1,S}$, $i\hm= 
\overline{1,n}$. Например, такими событиями могут вы\-сту\-пать сход с~рельсов по 
причине неисправности по\-движ\-но\-го со\-ста\-ва, сход с~рельсов по причине 
не\-ис\-прав\-ности железнодорожного полотна. Будем предполагать, что на одном 
метре дистанции может произойти максимум одно неблагоприятное событие, 
причем если это событие наступает, то далее движение не осуществляется. Введем 
обо\-зна\-че\-ния: ${\sf P}_{1i}(v_1)\hm= \mathcal{P}(A_{1,i})$, ${\sf P}_{si}(v_s) \hm= 
\mathcal{P}(A_{s,i}\vert \prod\nolimits_{k=1}^{s-1} \prod\nolimits^n_{j=1} 
\overline{A}_{k,j})$, $s\hm= \overline{2,S}$, т.\,е.\ для $s\hm \geq 2$ 
${\sf P}_{si}(v_s)$~--- это ве\-ро\-ят\-ность того, что на $s$-м мет\-ре маршрута произойдет 
$i$-е неблагоприятное событие при условии, что на предыду\-щих $s\hm-1$ мет\-рах 
неблагоприятных событий не произошло. Если $P_{si}(v_s)$ не точные значения 
вероятностей, а~оценки, то равенства выше понимаются в~при\-бли\-жен\-ном смыс\-ле. 
Ущерб~$C_{si}(v_s)$ от возникновения $i$-го неблагоприятного события на $s$-м 
мет\-ре маршрута будем полагать случайной величиной с~известным 
распределением, па\-ра\-мет\-ры которого определяются только ско\-ростью. 
Дополнительно предположим, что математическое ожидание 
$\mathbf{M}[C_{si}(v_s)]$ конечно для любого значения ско\-рости~$v_s$. Имеет 
место сле\-ду\-ющая лемма.

\smallskip

\noindent
\textbf{Лемма.}\ \textit{Случайная величина $\Phi(v_1, \ldots , v_S)$~--- ущерб при 
осуществлении перевозок~--- имеет вид}: 
$$
\Phi(v_1, \ldots , v_S) =\sum\limits^S_{s=1} \sum\limits^n_{i=1} C_{si} (v_s) 
I(A_{s,i})\,,
$$
\textit{где $I(A_{s,i})$~--- индикатор события~$A_{s,i}$, ве\-ро\-ят\-ность которого} 

\vspace*{-4pt}

\noindent
\begin{multline*}
\mathcal{P}(A_{s,i})= {\sf P}_{si}(v_s)\prod\limits^{s-1}_{t=1} \left( 1-
\sum\limits^n_{j=1} {\sf P}_{tj}(v_t)\right)\,,\\
s=\overline{2,S}\,,\ i=\overline{1,n}\,.
\end{multline*}
     
     \noindent
     Д\,о\,к\,а\,з\,а\,т\,е\,л\,ь\,с\,т\,в\,о\,.\ \  Выберем некоторое $i\hm\in \{ 1, \ldots , n \}$. 
     По условию ${\sf P}_{1i} (v_1)\hm= \mathcal{P}(A_{1,i})$. Далее найдем 
ве\-ро\-ят\-ность того, что $i$-е неблагоприятное событие произойдет на 2-м мет\-ре 
маршрута. Для того чтобы это событие произошло, на 1-м мет\-ре 
неблагоприятных событий произойти не долж\-но. Поэтому 
     $$
     \mathcal{P}(A_{2,i})=\mathcal{P}\left( A_{2,i}\vert \overline{A}_{1,1} \cdots 
\overline{A}_{1,n}\right) \mathcal{P}\left( \overline{A}_{1,1}\cdots 
\overline{A}_{1,n}\right).
     $$
     
     Используя равенство 
     $$
     \mathcal{P}\left( \overline{A}_{1,1}\cdots 
\overline{A}_{1,n}\right) \hm= 1\hm- \mathcal{P}\left( A_{1,1}+\cdots+ 
A_{1,n}\right),
$$
 получаем 
 
 \vspace*{-4pt}
 
 \noindent
     \begin{multline*}
     \mathcal{P}\left( A_{2,i}\right) ={}\\
     {}=\mathcal{P}\left( A_{2,i}\vert 
\overline{A}_{1,1} \cdots \overline{A}_{1,n}\right) \left[ 1-\mathcal{P}\left( 
\sum\limits^n_{j=1} A_{1,j}\right)\right] ={}\\
{}={\sf P}_{2i}(v_2) \left[ 1-\mathcal{P}\left( 
\sum\limits^n_{j=1} A_{1,j}\right)\right].
     %\label{e1-bos}
     \end{multline*}
     
     Так как события $A_{1,1},\ldots , A_{1,n}$ несовместны по условию, 
получаем 
$$
\mathcal{P}\left( \sum\limits^n_{j=1}A_{1,j}\right) = 
\sum\limits^n_{j=1} \mathcal{P} \left( A_{1,j}\right).
$$
 Отсюда 
 
 \vspace*{-4pt}
 
 \noindent
\begin{multline*}
     \mathcal{P}\left( A_{2,i}\right) ={\sf P}_{2i} (v_2) \left( 1-\sum\limits^n_{j=1} 
\mathcal{P}\left( A_{1,j}\right)\right) ={}\\
{}={\sf P}_{2i}(v_2)\left[ 1-\sum\limits^n_{j=1} 
{\sf P}_{1j}(v_1)\right].
\end{multline*}
     
     Нетрудно видеть, что аналогичные рас\-суж\-де\-ния повторяются на 
по\-сле\-ду\-ющих мет\-рах маршрута и~дают по индукции 

\pagebreak

\noindent
     \begin{multline*}
     \mathcal{P}\left( A_{s,i}\right) ={\sf P}_{si} (v_s) \prod\limits_{t=1}^{s-1} \left( 1-
\sum\limits^n_{j=1} {\sf P}_{tj}(v_t)\right),\\
 s=\overline{2,S}\,,\ i=\overline{1,n}\,.
\end{multline*}
    
     По условию ущерб при возникновении события~$A_{s,i}$ со\-став\-ля\-ет 
$C_{si}(v_s)$, поэтому 
$$
\Phi(v_1, \ldots , v_S)= \sum\limits^S_{s=1} 
\sum\limits^n_{i=1} C_{si}(v_s) I(A_{s,i}).
$$
 Лемма доказана.
     
     \smallskip
    
Найдем теперь вероятность того, что с транспортным средством при движении не 
произойдет ни одного неблагоприятного события, т.\,е. 
\begin{multline*}
\mathcal{P}\left( \prod\limits^S_{s=1} \prod\limits^n_{j=1} \overline{A}_{s,j}\right) 
=1-\mathcal{P}\left( \sum\limits^S_{s=1} \sum\limits^n_{j=1} A_{s,j}\right)={}\\
{}=
1-\sum\limits^S_{s=1} \sum\limits^n_{j=1} \mathcal{P}(A_{s,j})={}\\
{}= 1-\sum\limits^n_{j=1} {\sf P}_{1j}(v_1) -\sum\limits^S_{s=2} 
\sum\limits^n_{j=1}\mathcal{P}(A_{s,j}) ={}\\
{}= 1-\!\sum\limits^n_{j=1}  {\sf P}_{1j}(v_1) - \sum\limits^n_{j=1} {\sf P}_{2j}(v_2)\!\left[ 1- \!\sum\limits^n_{j=1} {\sf P}_{1j}(v_1)\right] -{}\\
{}- 
\sum\limits^S_{s=3} \sum\limits^n_{j=1} \mathcal{P}(A_{s,j}) =\cdots = 
\prod\limits^S_{s=1} \left[ 1-\sum\limits^n_{j=1} {\sf P}_{sj}(v_s)\right].
\end{multline*}
     
     Теперь в~качестве функции интегрального риска мож\-но предложить 
сле\-ду\-ющие два варианта: 
     \begin{align*}
     R_1(v_1, \ldots , v_S) &= 1-\mathcal{P} \left( \prod\limits^S_{s=1} 
\prod\limits^n_{j=1} \overline{A}_{s,j}\right) ={}\\
&\hspace*{11mm}{}=1-\prod\limits^S_{s=1} \left[ 1- 
\sum\limits^n_{j=1} {\sf P}_{sj}(v_s)\right];\hspace*{-0.68048pt}\\
     R_2(v_1, \ldots , v_S) &= \mathbf{M} \left[ \Phi(v_1, \ldots , v_S)\right] = {}&\\
&     \hspace*{-21mm}{}=
\sum\limits^S_{s=1} \sum\limits^n_{i=1} \mathbf{M} \left[ C_{si}(v_s)\right] 
{\sf P}_{si}(v_s) \prod\limits_{t=1}^{s-1} \left[ 1- \sum\limits^n_{j=1} {\sf P}_{tj}(v_j)\right].\hspace*{-6.9pt}
     \end{align*}
    
     Функция $R_1(v_1, \ldots , v_S)$ при выборе режима движения, т.\,е.\ набора 
скоростей $v_1, \ldots , v_S$, характеризует ве\-ро\-ят\-ность того, что на маршруте 
произойдет неблагоприятное событие. Функ\-ция $R_2(v_1, \ldots , v_S)$ позволяет 
вы\-чис\-лить сред\-ний ущерб при транспортировке. В~конкретных задачах 
целесообразным может оказаться использование как первой, так и~второй 
функции, а~умест\-нее всего использовать их комбинацию. Это вызвано тем, что 
при выбранном наборе скоростей $v_1, \ldots , v_S$ ве\-ро\-ят\-ность возникновения 
неблагоприятного события $R_1(v_1, \ldots , v_S)$ может быть небольшой, но 
сред\-ний ущерб $R_2(v_1, \ldots , v_S)$ при выборе такого режима движения может 
оказаться неоправданно высоким. Напротив, может оказаться, что 
прог\-но\-зи\-ру\-емый сред\-ний ущерб при заданном наборе скоростей $v_1, \ldots , v_S$ 
невелик, однако неблагоприятные события возникают \mbox{часто}.

\section{Концепция оценивания вероятности неблагоприятных событий 
и~ущерба при~движении грузовых поездов}

     Зафиксируем некоторый промежуток времени $\mathcal{T}$ в~прош\-лом. 
Пусть~$L_1$~--- чис\-ло вагонов; $L_0$~--- чис\-ло секций локомотива; $d_l$~--- 
длина $l$-й по\-движ\-ной единицы от головы поезда, округ\-лен\-ная в~большую 
сторону до мет\-ров. Для удобства введем величину 
$$
S_0= d_1+ d_2+ \cdots +d_L\,,
$$
 где $L$~--- общее чис\-ло по\-движ\-ных единиц грузового поезда, т.\,е.\ 
$L\hm= L_0\hm+ L_1$. Введем так\-же обозначения:
     \begin{description}
\item[\,] $p_1$~--- число сходов с рель\-сов\,/\,кру\-ше\-ний за промежуток 
времени~$\mathcal{T}$ вне стрелочных переводов по причине не\-ис\-прав\-ности 
подвижного состава; 
\item[\,] $p_2$~--- число сходов с рель\-сов\,/\,кру\-ше\-ний за промежуток 
времени~$\mathcal{T}$ вне стрелочных переводов по причине не\-ис\-прав\-ности 
железнодорожного полотна; 
\item[\,] $p_3$~--- число сходов с рель\-сов\,/\,кру\-ше\-ний за промежуток 
времени~$\mathcal{T}$ на стрелочных переводах; 
\item[\,] $p_{\mathrm{общ}}$~--- общее число сходов с рель\-сов\,/\,кру\-ше\-ний 
за промежуток времени~$\mathcal{T}$; 
\item[\,] $Q_{1}^\prime$~--- чис\-ло вагоно-ки\-ло\-мет\-ров за промежуток 
времени~$\mathcal{T}$; 
\item[\,] $Q_{2}^\prime$~--- чис\-ло поездо-ки\-ло\-мет\-ров за промежуток 
времени~$\mathcal{T}$; 
\item[\,] $w$~--- вес поезда, т; 
\item[\,] $\tilde{\mu}=w/(69L_1)-1/3$~--- величина, ха\-рак\-те\-ри\-зу\-ющая степень 
за\-груз\-ки поезда~\cite{7-bos};
\item[\,] $\ptb{\ae}_s$~--- кривизна кривой на $s$-м мет\-ре пути (величина, 
обрат\-но пропорциональная радиусу кривизны, для прямой полагается рав\-ной 
нулю); 
\item[\,] $\gamma_s$~--- профиль пути на $s$-м метре пути, из\-ме\-ря\-емый 
в~тысячных, имеющий знак минус, ес-\linebreak\vspace*{-12pt}

\pagebreak

\noindent
ли уклон пред\-став\-ля\-ет спуск, знак плюс, 
если уклон пред\-став\-ля\-ет подъем, ед.; 
\item[\,] $\lambda_s$~--- величина, ха\-рак\-те\-ри\-зу\-ющая наличие стрелочного 
перевода на $s$-м мет\-ре пути (0~--- нет стрелочного перевода; 1~--- есть);
\item[\,] $\chi_s$~--- величина, ха\-рак\-те\-ри\-зу\-ющая наличие со\-сед\-не\-го/со\-сед\-них 
пу\-ти/п\-утей на $s$-м мет\-ре пути (0~--- нет со\-сед\-не\-го пути; 1~--- 
есть); 
\item[\,] $\eta_s$~--- величина, ха\-рак\-те\-ри\-зу\-ющая начало же\-лез\-но\-до\-рож\-но\-го 
переезда на \mbox{$s$-м} мет\-ре пути (0~--- железнодорожный переезд не начинается 
на \mbox{$s$-м} мет\-ре пути; 1~--- иначе); 
$\xi_s$~--- величина, ха\-рак\-те\-ри\-зу\-ющая начало стрелочного перевода на $s$-м 
мет\-ре пути в~пределах же\-лез\-но\-до\-рож\-ной станции (0~--- стрелочный перевод на 
$s$-м мет\-ре пути в~пределах станции; 1~--- иначе), $s\hm= \overline{2-S_0, S}$.
     \end{description}
     
     Будем рассматривать следующие неблагоприятные события:
     $$
     A_{s,i}=\! \begin{cases}\!
     \mbox{сход}\ \lfloor (i-1)/L\rfloor\hm+1\ \mbox{подвижных\ единиц}\\
    \mbox{по\ причине\ неисправности\ подвижного}\\
    \mbox{состава\ вне\ стрелочного\ перевода}\\
    \mbox{по\ прошествии}\ s\ \mbox{метров\ пути,\ начиная}\\
    \mbox{с}\ i-\lfloor (i-1)/L\rfloor L\ 
\mbox{подвижной\ единицы}, &\\
& \hspace*{-30mm}i=\overline{1,L^2}\,;\\
     \mbox{сход}\  \lfloor (i-1-L^2)/L\rfloor +1\ \mbox{подвижных}\\ 
     \mbox{единиц\ по\ причине\ неисправности}\\
     \mbox{железнодорожного\ пути\ вне\ стрелоч-}\\
     \mbox{ного\ перевода\ по прошествии}\ s\ \mbox{метров}\\ 
     \mbox{пути,\ начиная\ с}\ i\!-\!L^2\!-\!\lfloor (i\!-\!1\!-\!L^2)/L\rfloor L\\
          \mbox{подвижной\ единицы,}  & \hspace*{-30mm}i=\overline{L^2+1, 2L^2}\,;\\
     \mbox{сход}\ \lfloor (i-1-2L^2)/L\rfloor+1\ \mbox{подвижных}\\
     \mbox{единиц\ на\ стрелочном\ переводе\  по}\\ 
     \mbox{прошествии}\ s\ \mbox{метров\ пути,\ начиная}\\
     \mbox{с}\  i-2L^2 -\lfloor (i-1-2L^2)/L\rfloor L\\ 
     \mbox{подвижной\ единицы}, & 
\hspace*{-30mm}i=\overline{2L^2+1,3L^2}\,;
     \end{cases}
     \hspace*{-2.86766pt}
     $$
     
     \vspace*{-9pt}
     
     \noindent
\begin{multline*}
A_{s,3L^2+1} = \left\{ \mbox{столкновение\ на\ железнодорожном}\right.\\ 
\left.\mbox{переезде,\ начинающемся\  на}\ s\mbox{-м}\ \mbox{метре пути}\right\};
\end{multline*}

\vspace*{-12pt}

\noindent
\begin{multline*}
A_{s,3L^2+2} = \left\{ \mbox{столкновение\ на\ железнодородной}\right.\\ 
\hspace*{16mm}\mbox{cтанции\ на\ стрелочном\ переводе,}\\
\left.\mbox{начинающемся\ на}\ s\mbox{-м}\ \mbox{метре пути}\right\},
\end{multline*}

\vspace*{-3pt}

\noindent
где $\lfloor x\rfloor$ обозначает целую часть~$x$.

     
     Такой способ нумерации поз\-во\-ля\-ет одним индексом~$i$ пронумеровать все 
события в~группе с~учетом как номера по\-движ\-ной единицы, с~которой
начнется 
сход, так и~чис\-ла сошедших по\-движ\-ных единиц в~сходе, т.\,е.~$L^2$ вариантов, 
и,~кроме то-\linebreak\vspace*{-12pt}

\columnbreak

\noindent
 го, поз\-во\-ля\-ет учесть раз\-ни\-цу причин, вызвавших неблагоприятное 
событие.

Про\-ил\-люст\-ри\-ру\-ем принцип работы пред\-ло\-жен\-ной нумерации, введя для 
ла\-ко\-нич\-ности обозначения

\noindent
     $$
     q_i \overset{\mathrm{def}}{=}
     \begin{cases}
     \left\lfloor \fr{i-1}{L}\right\rfloor +1\,, & i=\overline{1,L^2}\,;\\[9pt]
     \left\lfloor \fr{i-1-L^2}{L}\right\rfloor+1\,, & i=\overline{L^2+1,2L^2}\,;\\[9pt]
     \left\lfloor \fr{i-1-2L^2}{L}\right\rfloor +1\,, & i=\overline{2L^2+1,3L^2}\,.
     \end{cases}
     $$ 

%\begin{table*}\small %tabl1

\vspace*{-12pt}

\begin{center}
\noindent
\parbox{76mm}{{{\tablename~1}\ \ \small{Потенциальное число по\-движ\-ных единиц в~сходе с рельсов и~номер пер\-вой сошедшей 
по\-движ\-ной единицы при $L\hm=10$
}}}

\vspace*{6pt}

{\small \begin{tabular}{|l|c|c|c|}
\hline
\multicolumn{1}{|c|}{Событие} &$i$ & $q_i$ & $f_i$\\
\hline
\tabcolsep=0pt\begin{tabular}{l}Сход вне стрелочного перевода\\
по причине неисправности\\ подвижного состава\end{tabular}&\tabcolsep=0pt\begin{tabular}{c}
\hphantom{9}1\\ \hphantom{9}2\\ $\cdots$\\ 10\\ 11\\ 12\\ $\cdots$ \\ 20\\ $\cdots$\\ 91\\ 92\\ $\cdots$\\ 100\hphantom{9}\end{tabular}&
\tabcolsep=0pt\begin{tabular}{c}
\hphantom{9}1\\ \hphantom{9}1\\ $\cdots$\\ \hphantom{9}1\\ \hphantom{9}2\\ \hphantom{9}2\\ $\cdots$\\ 2\\ $\cdots$\\ 10\\ 10\\ $\cdots$\\10\end{tabular}
& 
\tabcolsep=0pt\begin{tabular}{c}
\hphantom{9}1\\ \hphantom{9}2\\ $\cdots$ \\ 10\\ \hphantom{9}1\\ \hphantom{9}2\\ $\cdots$\\ 10\\$\cdots$\\ \hphantom{9}1\\ \hphantom{9}2\\ $\cdots$ \\ 10\end{tabular}
\\
\hline
\tabcolsep=0pt\begin{tabular}{l}Сход вне стрелочного перевода\\ по причине неисправности\\ железнодорожного 
пути\end{tabular}&
\tabcolsep=0pt\begin{tabular}{c}
101\hphantom{9}\\ 102\hphantom{9}\\ $\cdots$\\ 110\hphantom{9}\\ 111\hphantom{9}\\ 112\hphantom{9}\\ $\cdots$\\ 120\hphantom{9}\\ $\cdots$\\ 191\hphantom{9}\\ 192\hphantom{9}\\ $\cdots$\\ 200\hphantom{9}\end{tabular}
&
\tabcolsep=0pt\begin{tabular}{c}
\hphantom{9}1\\ \hphantom{9}1\\ $\cdots$\\ \hphantom{9}1\\ \hphantom{9}2\\ \hphantom{9}2\\ $\cdots$\\ \hphantom{9}2\\ $\cdots$\\ 10\\ 10\\ $\cdots$\\ 10\end{tabular} &
\tabcolsep=0pt\begin{tabular}{c}
\hphantom{9}1\\ \hphantom{9}2\\ $\cdots$\\ 10\\ \hphantom{9}1\\ \hphantom{9}2\\ $\cdots$\\ 10\\ $\cdots$\\ \hphantom{9}1\\ \hphantom{9}2\\ $\cdots$\\ 10\end{tabular}
\\
\hline
\tabcolsep=0pt\begin{tabular}{l}Сход на стрелочном переводе\end{tabular} &
\tabcolsep=0pt\begin{tabular}{c}
201\hphantom{9}\\ 202\hphantom{9}\\ $\cdots$\\ 210\hphantom{9}\\211\hphantom{9}\\ 212\hphantom{9}\\ $\cdots$ \\ 220\hphantom{9}\\$\cdots$ \\ 291\hphantom{9}\\ 291\hphantom{9}\\ $\cdots$\\ 300\hphantom{9}\end{tabular}
&
\tabcolsep=0pt\begin{tabular}{c}
\hphantom{9}1\\ \hphantom{9}1\\ $\cdots$\\ \hphantom{9}1\\ \hphantom{9}2\\ \hphantom{9}2\\ $\cdots$\\ \hphantom{9}2\\ $\cdots$ \\ 10\\ 10\\ $\cdots$\\ 10\end{tabular}
&
\tabcolsep=0pt\begin{tabular}{c}
\hphantom{9}1\\ \hphantom{9}2\\ $\cdots$\\ 10\\ \hphantom{9}1\\\hphantom{9}2\\ $\cdots$ \\ 10\\ $\cdots$\\ \hphantom{9}1\\ \hphantom{9}2\\ $\cdots$ \\ 10\end{tabular}\\
\hline
\end{tabular}
}
\end{center}
%\end{table*}



     
   %  \vspace*{-12pt}
    
    \noindent
    \begin{multline*}
f_i\overset{\mathrm{def}}{=} \\
{}\overset{\mathrm{def}}{=}\!
\begin{cases}
i-\left\lfloor \fr{i-1}{L}\right\rfloor  L\,, &\!\!\! i=\overline{1,L^2}\,;\\[9pt]
i-L^2 -\left\lfloor \fr{i-1-L^2}{L}\right\rfloor  L\,, &\!\!\! i=\overline{L^2+1,2L^2}\,;\\[9pt]
i-2L^2-\left\lfloor \fr{i-1-2L^2}{L}\right\rfloor L\,, &\!\!\! i=\overline{2L^2+1, 3L^2}\,.
\end{cases}\hspace*{-10.1835pt}
\end{multline*}
     
     Значение~$q_i$ характеризует чис\-ло по\-движ\-ных единиц в~сходе с рельсов, 
$f_i$~--- номер пер\-вой сошедшей по\-движ\-ной единицы.

В~табл.~1 представлен расчет для $L=10$.
     
\vspace*{-7pt}


\section{Оценивание вероятности неблагоприятных событий}

\vspace*{-3pt}

     В~[4, 5] была предложена сле\-ду\-ющая (с~учетом ис\-поль\-зу\-емых 
обозначений) оценка ве\-ро\-ят\-ности схода с~рельсов по\-движ\-ных единиц грузового 
поезда (при движении по конкретному клас\-су пути):
     \begin{multline*}
     \mathcal{P}(A_{\mathrm{сх}}) = 1-\exp \left\{ -S^{\prime\prime} \left( 
\fr{p_1+p_2+p_3}{Q_1^{\prime\prime}}\,L +{}\right.\right.\\
\left.\left.{}+\fr{p_{\mathrm{общ}} -p_1 -p_2-p_3} 
{Q_2^{\prime\prime}}\right)\right\},
  \end{multline*}
где $S^{\prime\prime}$~--- длина пути в~милях; $Q_1^{\prime\prime}$~--- чис\-ло  
ва\-го\-но-миль за промежуток времени~$\mathcal{T}$; $Q_2^{\prime\prime}$~--- 
чис\-ло по\-ез\-до-миль за промежуток времени~$\mathcal{T}$. Пусть $S^\prime$~--- 
длина пути в~ки\-ло\-мет\-рах. Тогда при пересчете миль в~ки\-ло\-мет\-ры и~мет\-ры имеем 
\begin{multline*}
\mathcal{P}(A_{\mathrm{сх}}) =1-\exp \left\{ - \fr{S^\prime}{1{,}609} \left( 
\fr{p_1+p_2+p_3}{Q_1^\prime/1{,}609}\,L +{}\right.\right.\\
\left.\left.{}+\fr{p_{\mathrm{общ}}-p_1-p_2-p_3} 
{Q_2^\prime/1{,}609}\right)\right\} ={}\\
{}= 1-\exp \left\{ -S\left( \fr{p_1+p_2+p_3}{1000Q_1^\prime}\,L+ {}\right.\right.\\
\left.\left.{}+\fr{p_{\mathrm{общ}}-p_1-p_2-p_3}{1000Q^\prime_2}\right)\right\}.
\end{multline*}
     
     Обозначив 
     \begin{equation*}
     Q_1 = \fr{p_1+p_2+p_3}{1000Q_1^\prime}\,;\enskip 
     Q_2=\fr{p_{\mathrm{общ}}-p_1-p_2-p_3}{1000Q_2^\prime}\,,
     \end{equation*}
     будем далее полагать 
     \begin{equation}
     \mathcal{P}(A_{\mathrm{сх}}) =1-\exp \left\{ -S\left( Q_1L+Q_2\right)\right\}.
     \label{e2-bos}
     \end{equation}
     
     Согласно~(\ref{e2-bos}) вероятность схода с рельсов на 1-м мет\-ре 
дис\-танции
     \begin{equation}
     \mathcal{P}\left( A_{\mathrm{сх},1}\right)= 1-\exp \left\{ -\left( Q_1L 
+Q_2\right)\right\}.
     \label{e3-bos}
     \end{equation}
     
     Условная вероятность того, что сход произойдет на 2-м мет\-ре дистанции 
при условии, что на 1-м\linebreak\vspace*{-12pt}

\columnbreak

\noindent
 мет\-ре схода не было, определяется при решении 
уравнения 

\vspace*{-4pt}

\noindent
     \begin{multline*}
     \mathcal{P}\left( A_{\mathrm{сх},1}\right) +\mathcal{P}\left( 
A_{\mathrm{сх},2} \vert \overline{A}_{\mathrm{сх},1}\right) \mathcal{P}\left( 
\overline{A}_{\mathrm{сх},1}\right) ={}\\
{}=1-\exp \left\{ -2\left( Q_1L+Q_2\right)\right\}.
     \end{multline*}
     
     Подставляя~(\ref{e3-bos}) в~последнее уравнение, записываем:
     
     \vspace*{-4pt}
     
     \noindent
     \begin{multline}
     1-\exp \left\{ -\left( Q_1L+Q_2\right)\right\} +{}\\
     {}+\mathcal{P}\left( 
A_{\mathrm{сх},2} \vert \overline{A}_{\mathrm{сх},1}\right) \exp \left\{ -\left( 
Q_1L+Q_2\right)\right\}={}\\
     {}= 1-\exp \left\{ -2\left( Q_1L +Q_2\right)\right\}.
     \label{e4-bos}
     \end{multline}
     
     Решая~(\ref{e4-bos}), получаем 
     $$
     \mathcal{P}\left( A_{\mathrm{сх},2} \vert \overline{A}_{\mathrm{сх},1} 
\right) =\mathcal{P}\left(A_{\mathrm{сх},1}\right)= 1- \exp \{-(Q_1L\hm+ Q_2)\}
$$ 
и~далее по индукции

\vspace*{-4pt}

\noindent
     \begin{multline}
     P_s\overset{\mathrm{def}}{=} \mathcal{P}\left( A_{\mathrm{сх},s} \left\vert 
\prod\limits_{t=1}^{s-1}\right. \overline{A}_{\mathrm{сх},t}\right) ={}\\
{}=1-\exp \left\{ -
\left( Q_1L+Q_2\right)\right\}\,,\enskip s=\overline{1,S}\,.
     \label{e5-bos}
     \end{multline}
     
     Введем вспомогательные вероятности, необходимые для 
задания~${\sf P}_{si}(v_s)$: $\overline{\sf P}^{s,l}$, $\tilde{\sf P}^{s,l}$, $\hat{\sf P}^{s,l}$~--- 
вероятность (оценка вероятности) того, что сход начнется с $l$-й по порядку 
следования подвижной единицы по причине: 
     \begin{itemize}
\item неисправности подвижного со\-ста\-ва вне стрелочного перевода 
($\overline{\sf P}^{s,l}$); 
\item неисправности железнодорожного по\-лот\-на вне стрелочного перевода 
($\overline{\sf P}^{s,l}$);
\item неисправности любого вида на стрелочном переводе ($\hat{\sf P}^{s,l}$) 
при условии схода по прошествии~$s$~мет\-ров пути, $s\hm= \overline{1,S}$, 
$l\hm= \overline{1,L}$.
\end{itemize}
     
     В предположении, что сход может рав\-но\-ве\-ро\-ят\-но начаться с~любой 
по\-движ\-ной единицы, а~также основываясь на час\-то\-те сходов по причине той или 
иной не\-ис\-прав\-ности и~учитывая на\-ли\-чие/от\-сут\-ст\-вие стрелочного перевода в~пред\-по\-ла\-га\-емом мес\-те схода, зададим 
     \begin{align*}
     \overline{\sf P}^{s,l} &=\fr{1}{L}\left( 1-\lambda_{s-(d_1+d_2+\cdots + d_{l-1})} 
\right) \fr{p_1}{p_1+p_2}\,;\\
     \tilde{\sf P}^{s,l}  &=\fr{1}{L}\left( 1-\lambda_{s-(d_1+d_2+\cdots +d_{l-1})} 
\right)\fr{p_2}{p_1+p_2}\,;\\
     \hat{\sf P}^{s,l} &=\fr{1}{L}\,\lambda_{s - (d_1+d_2+\cdots +d_{l-1})}.
     \end{align*}
     
     Очевидно, что вероятности $\overline{\sf P}^{s,l}$, $\tilde{\sf P}^{s,l}$ 
и~$\hat{\sf P}^{s,l}$ долж\-ны зависеть от возраста и~со\-сто\-яния рельсов и~возрас\-та 
и~условий эксплуатации той или иной по-\linebreak\vspace*{-12pt}

\pagebreak

\noindent
движной единицы, однако для прос\-то\-ты 
использованы оценки ве\-ро\-ят\-ности в~форме час\-то\-ты. Имеет смысл более точ\-ное 
оценивание, которое, однако, пред\-став\-ля\-ет собой отдельную задачу.
     
     Пусть $\overline{\sf P}_{s,l,k}(v_s)$, $\tilde{\sf P}_{s,l,k}(v_s)$ 
и~$\hat{\sf P}_{s,l,k}(v_s)$~--- ве\-ро\-ят\-ность (оцен\-ка ве\-ро\-ят\-ности) схода~$k$~по\-движ\-ных 
единиц грузового поезда при условии схода по прошествии $s$~мет\-ров пути 
с~учетом того, что сход начнется с $l$-й по порядку следования по\-движ\-ной единицы 
по причине: 
     \begin{itemize}
\item неисправности подвижного состава вне стрелочного перевода 
($\overline{\sf P}_{s,l,k}(v_s)$);
\item неисправности железнодорожного полотна вне стрелочного перевода 
($\tilde{\sf P}_{s,l,k}(v_s)$);
\item неисправности любого вида на стрелочном переводе 
($\hat{\sf P}_{s,l,k}(v_s)$), $s\hm= \overline{1,S}$, $l\hm= \overline{1,L}$, $k\hm= 
\overline{1,L}$.
\end{itemize}
     Следуя~\cite{6-bos, 7-bos}, зададим 
     \begin{multline*}
     \hspace*{-3pt}\overline{\sf P}_{s,l,k} (v_s) =\fr{\Gamma(k-1+1/\theta_1) } 
{\Gamma(k)\Gamma(1/\theta_1)}\left( 1+ \theta_1 g_1\left(
\vphantom{\ae_{s-\sum\nolimits^{l-1}_{r=1} d_r}}
a_1, v_s, l, w,\right.\right.\\ 
\left.\left. L_1, L, 
\tilde{\mu}, \ae_{s-\sum\nolimits^{l-1}_{r=1} d_r}, \gamma_{s-\sum\nolimits^{l-
1}_{r=1} d_r}\right)\right)^{-(k-1+1/\theta_1)}\times{}\\
     {}\times
     \left( \theta_1 g_1\left( a_1, v_s,l,w,L_1, L, \tilde{\mu}, \ae_{s-\sum\nolimits^{l-1}_{r=1} dr},\right.\right.\\
     \left.\left. 
     \gamma_{s-\sum\nolimits_{r=1}^{l-1} dr}\right)\right)^{k-1},
     \end{multline*}
где $a_1$ и~$\theta_1$~--- па\-ра\-мет\-ры; $g_1(\cdot)$~--- функция, под\-ле\-жа\-щая 
подбору на основе метода максимального прав\-до\-по\-до\-бия; $\Gamma(\cdot)$~---  
гам\-ма-функ\-ция. Аналогичным~(\ref{e5-bos}) образом зададим 
$\tilde{\sf P}_{s,l,k}(v_s)$ и~$\hat{\sf P}_{s,l,k}(v_s)$ с~точ\-ностью до замены па\-ра\-мет\-ров 
$a_1$ и~$\theta_1$ и~функции~$g_1(\cdot)$ на па\-ра\-мет\-ры $a_2$, $a_3$, $\theta_2$ 
и~$\theta_3$ и~функции $g_2(\cdot)$ и~$g_3(\cdot)$ соответственно.
     
     Роли вероятностей ${\sf P}_s$, $\overline{\sf P}^{s,l}$ и~$\overline{\sf P}_{s,l,k}(v_s)$ 
при вы\-чис\-ле\-нии~${\sf P}_{si}(v_s)$ ил\-люст\-ри\-ру\-ет сле\-ду\-ющая схема: 
\begin{multline*}
{\sf P}_{si}(v_s)= {}\\
{}\!=\!\mathcal{P}\!\left( \! \left. \! \left\{
\begin{array}{c}
\mathrm{сход}\ q_i\ \mathrm{подвижных}\\ 
\mathrm{единиц\ по\ причине}\\   
\mathrm{неисправности}\\
\mathrm{пути,\ начиная}\\ 
\mathrm{с}\ f_i\mathrm{\mbox{-}й}\\ 
\mathrm{по\ счету}\\
\mathrm{вне\ стрелки}\\ 
\mbox{после}\   s\ \mathrm{метров}
\end{array}\!
\right\} \! \right\vert \!\left\{\!
\begin{array}{c}
\mathrm{на}\\ 
\mbox{предыдущих}\\
s-1\ \mathrm{метрах}\\
\mathrm{не\ произошло}\\
\mathrm{неблагопри\mbox{-}}\\
\mathrm{ятных}\\
\mbox{событий}
\end{array}
\right\}\!
\right)={}
\\
{}=\!
\underbrace{\mathcal{P}\!\left(\! \left\{
\begin{array}{c}
\mathrm{сход}\\ 
\mathrm{после}\\
s\ \mathrm{метров}
\end{array}
\right\}\left\vert \!
\left\{
\begin{array}{c}
\mathrm{на\ предыдущих}\\ s-1\ \mathrm{метрах}\\
\mathrm{не\ произошло}\\
\mathrm{неблагоприятных}\\
\mathrm{событий}
\end{array}
\right\}\right.
\right)}_{={\sf P}^s\approx {\sf P}_s}\!\times{}\\
\end{multline*}

\vspace*{-30pt}

\columnbreak

\noindent
\begin{multline*}
\hspace*{-5.6508pt}{}\times
\underbrace{\mathcal{P}\!\left(\! \left.\left\{
\begin{array}{c}
\mathrm{сход\ по\ причине}\\
\mathrm{неисправности}\\
\mathrm{пути, начиная}\\
\mathrm{с}\ f_i\mathrm{\mbox{-}й\ по\ счету}\\
\mathrm{подвижной}\\
\mathrm{единицы}\\
\mathrm{вне\ стрелки}
\end{array}
\right\}
\right\vert 
\!\left\{ 
\begin{array}{c}
\mathrm{на\ предыдущих}\\
s-1\ \mathrm{метрах}\\
\mathrm{не\ произошло}\\
\mathrm{неблагоприят-}\\
\mathrm{ных\ событий,}\\
\mathrm{сход\ после}\\
s\ \mathrm{метров}
\end{array}
\right\}\!
\right)}_{=\overline{\sf P}^{s,f_i}}\!\times{}\\
{}\times \underbrace{\mathcal{P}\!\left( \!\left\{
\begin{array}{c}
\mathrm{сход}\ q_i\\
\mathrm{подвижных}\\
\mathrm{единиц}
\end{array}
\right\}
\left\vert 
\!
\left\{ 
\begin{array}{c}
\mathrm{на\ предыдущих}\\
s-1\ \mathrm{метрах}\\
\mathrm{не\ произошло}\\
\mathrm{неблагоприятных}\\
\mathrm{событий,\ сход}\\
\mathrm{после}\ s\ \mathrm{метров}\\
\mathrm{по\ причине\ неис-}\\
\mathrm{правности\ пути}\\
\mathrm{начиная\ с}\ f_i\mathrm{\mbox{-}й\ по}\\
\mathrm{счету\ подвижной}\\
\mathrm{единицы\ вне\ стрелки.} 
\end{array}
\right\}\!\right.
\right)
}_{=\overline{\sf P}_{s,f_i,q_i(v_s)}}\!.
\hspace*{-7.020969pt}\hspace*{-0.18872pt}
\end{multline*}
%%%%%%%%%%%%%%%%
 
 
     
     Прокомментируем приближенное равенство ${\sf P}^s\hm\approx {\sf P}_s$. Имеем
     \begin{multline*}
     {\sf P}_s= {}\\
     {}=\mathcal{P}\left(\! A_{\mathrm{сх},s} \left\vert \prod\limits_{t=1}^{s-1}\right. 
\overline{A}_{\mathrm{сх},t}\!\right) =\mathcal{P}\left( \!A_{\mathrm{сх},s}\left\vert 
\prod\limits_{t=1}^{s-1}\right. \prod\limits^{3L^2}_{i=1} \overline{A}_{t,i}\!\right)= {}\\
     {}=
     \mathcal{P} \left( A_{\mathrm{сх},s}\left\vert \prod\limits_{t=1}^{s-1}\right. 
\prod\limits_{i=1}^{3L^2} \overline{A}_{t,i} \left( 
\prod\limits_{i=3L^2+1}^{3L^2+2} \overline{A}_{t,i} 
+{}\right.\right.\\
\left.\left.{}+\sum\limits_{i=3L^2+1}^{3L^2+2} A_{t,i}\right)\right)={}\\
     {}= \mathcal{P}\left( A_{\mathrm{сх},s} \prod\limits_{t=1}^{s-1} 
\prod\limits_{i=1}^{3L^2+2} \overline{A}_{t,i}\right)\! \! \Bigg /\!\!
     \Bigg(\underbrace{\mathcal{P}\!\Bigg(\prod\limits_{t=1}^{s-1} 
\prod\limits_{i=1}^{3L^2+2} \overline{A}_{t,i}\Bigg)}_{\approx 1} +{}\\
{}+\underbrace{\mathcal{P}\Bigg( \prod\limits_{t=1}^{s-1} 
\prod\limits_{i=1}^{3L^2} \overline{A}_{t,i} \sum\limits_{i=3L^2+1}^{3L^2+2} 
A_{t,i}\Bigg)}_{\substack{\approx\;0,\ \mathrm{так\ как\ вероятность\ любого}\\
\mathrm{неблагоприятного\ события}\\
\mathrm{должна\ быть\ около\ нуля}~\mbox{---}\\
\mathrm{в\ противном\ случае\ начинать}\\
\mathrm{движение\ опасно}}}\Bigg)\approx{}\\
{} \approx \fr{\mathcal{P}\left( A_{\mathrm{сх},s} \prod\nolimits_{t=1}^{s-1} 
\prod\nolimits_{i=1}^{3L^2+2} \overline{A}_{t,i}\right)}{\mathcal{P}\left( 
\prod\nolimits_{t=1}^{s-1} \prod\nolimits_{i=1}^{3L^2+2} \overline{A}_{t,i}\right)} 
={}\\
{}=\mathcal{P}\left( A_{\mathrm{сх},s} \left\vert \prod\limits_{t=1}^{s-1} \right.
\prod\limits_{i=1}^{3L^2+2} \overline{A}_{t,i}\right) ={\sf P}^s.
     \end{multline*}
     
     Учитывая вышесказанное, получаем сле\-ду\-ющие оценки для 
величин~${\sf P}_{si} (v_s)$: 
     \begin{equation*}
    {\sf P}_{si}(v_s)= \begin{cases}
     {\sf P}_s \overline{\sf P}^{s,f_i} \overline{\sf P}_{s,f_i,q_i}(v_s)\,, & 
i=\overline{1,L^2}\,;\\
     {\sf P}_s\tilde{\sf P}^{s,f_i}\tilde{\sf P}_{s,f_i,q_i}(v_s)\,, & 
i=\overline{L^2+1,2L^2}\,;\\
     {\sf P}_s \hat{\sf P}^{s,f_i} \hat{\sf P}_{s,f_i,q_i}(v_s)\,, & i=\overline{2L^2+1,3L^2}\,.
     \end{cases}
     \end{equation*}
     
     Вероятность ${\sf P}_{s\cdot 3L^2+2}(v_s)$ мож\-но оценить, например, при 
помощи имитационного моделирования~\cite{11-bos} либо на основе 
пуассоновских потоков~\cite{12-bos}, вероятность ${\sf P}_{s\cdot 3L^2+2}(v_s)$ 
может быть оценена в~форме час\-то\-ты на основании ста\-ти\-сти\-ки по пересечениям 
грузовыми поездами переездов и~столк\-но\-ве\-ни\-ям на них.


     
\section{Оценивание ущерба}

     Будем рассматривать только материальный ущерб. Ущерб, связанный 
с~задержками поездов, выз\-ван\-ны\-ми возникновением неблагоприятных событий, 
ввиду слож\-ности пересчета в~материальный ущерб учитывать не будем. Иными 
словами, предполагается, что ущерб при за\-держ\-ках отсутствует. При сходе 
с~рель\-сов ущерб включает в~себя расходы, связанные  
с~ре\-мон\-том/спи\-са\-ни\-ем подвижных единиц грузового поезда. Поскольку 
сто\-и\-мость различных видов ремонта подвижных единиц грузовых поездов, 
участ\-во\-вав\-ших в~сходах и~крушениях, недоступна, предложим сле\-ду\-ющую оценку 
сред\-не\-го размера ущерба: 
     \begin{multline}
     \mathbf{M} \left[ C_{si}(v_s)\right] =10^5 P^a_{si}(v_s) +{}\\
     {}+ 4{,}5\cdot 10^6\cdot \fr{9}{200}\,v_s \cdot
     \begin{cases}
     q_i, & f_i+q_i-1\leq L\,;\\
     0 &\mbox{иначе},
     \end{cases}
     \label{e6-bos}
     \end{multline}
     
\noindent
где $P_{si}^a(v_s)$~--- оценка вероятности выхода в~габарит соседнего пути хотя 
бы одной по\-движ\-ной единицы грузового поезда, $i\hm= \overline{1,3L^2}$. 
Согласно~\cite{13-bos}, средний ущерб от столкновений со\-став\-ля\-ет 1599~тыс.\ 
руб., а~по открытым источникам, примерно каж\-дый \mbox{16-й} раз в~случае выхода 
в~габарит соседнего пути происходит столкновение со встреч\-ным поездом. Поэтому 
сред\-ний ущерб при выходе в~габарит соседнего пути оценивается 
в~$1\,599\,000/16\hm\approx 10^5$~руб., что и~пред\-став\-ля\-ет сомножитель 
${\sf P}^a_{si}(v_s)$ в~(\ref{e6-bos}). Отметим, что данный сомножитель, вообще 
говоря, зависит от ин\-тен\-сив\-ности встреч\-но\-го движения. Поэтому приведенное 
чис\-ло следует рас\-смат\-ри\-вать лишь как некоторое начальное при\-бли\-же\-ние. Далее, 
ско\-рость в~80~км/ч (как правило, выше этого порога ско\-рость грузового поезда 
недопустима) эквивалентна~200/9~м/с, поэтому сомножитель~9/200 используется 
в~формуле~(\ref{e6-bos}). Заметим, что 4,5~млн руб.~--- при\-бли\-зи\-тель\-ная 
сто\-и\-мость нового грузового вагона. При этом в~сходе с~рельсов может оказаться 
и~секция локомотива, однако по\-вреж\-де\-ние секции локомотива до исключения из 
инвентарного парка намного более ред\-кое событие, чем исключение вагона, 
поэтому для прос\-то\-ты предполагается, что максимальный ущерб от схода секции 
локомотива равен сто\-и\-мости нового вагона. Следует отметить, что с~рос\-том 
ско\-рости рас\-тет и~кинетическая энергия, а~следовательно, и~по\-след\-ст\-вия для 
по\-движ\-ных единиц от схода. Поэтому в~(\ref{e6-bos}) предполагается линейный 
рост размера ущерба с~рос\-том ско\-рости. Значение ${\sf P}^a_{si}(v_s)$ мож\-но 
оценить с~при\-вле\-че\-ни\-ем метода максимального прав\-до\-по\-до\-бия на основе 
исторических данных о~возникновения данного события~\cite{8-bos}: 
\begin{multline*}
{\sf P}^a_{si}(v_s)= {}\\
{}=\begin{cases}
\overline{p}_s \left( 
\vphantom{\ae_{s-\sum\nolimits_{r=1}^{f_i-1} d_r}}
v_s, f_i, q_i, b_1, w,L_1, L, \tilde{\mu},\right.&\\
 \hspace*{1.5mm}\left.\ae_{s-\sum\nolimits_{r=1}^{f_i-1} d_r}, \gamma_{s-\sum\nolimits_{r=1}^{f_i-1} d_r}\right), &
\hspace*{-1.5mm}i=\overline{1,L^2}\,;\\[3pt]
\tilde{p}_s\left(
\vphantom{\ae_{s-\sum\nolimits_{r=1}^{f_i-1} d_r}}
v_s, f_i, q_i, b_2, w,L_1, L, \tilde{\mu},\right.&\\
 \hspace*{1.5mm}\left.\ae_{s-\sum\nolimits_{r=1}^{f_i-1} d_r}, \gamma_{s-\sum\nolimits_{r=1}^{f_i-1} d_r}\right), & 
 \hspace*{-1.5mm}i=\overline{L^2+1, 2L^2}\,;\\[3pt]
1, & \hspace*{-1.5mm}i=\overline{2L^2+1, 3L^2}\,,\hspace*{-6.3864pt}\hspace*{-1.38638pt}
\end{cases}
\end{multline*}
где $\overline{p}_s(\cdot)$ и~$\tilde{p}_s(\cdot)$~--- подлежащие оцениванию 
функции; $b_1$ и~$b_2$~--- подлежащие оцениванию па\-ра\-метры.

     При столкновениях на железнодорожном переезде, как правило, виновным 
признается владелец автотранспорта, поэтому ущерб, вызванный столкновением, 
компенсируется, откуда 
$$
\mathbf{M}\left[C_{si}(v_s)\right] =0\,,\enskip s=\overline{1,S}\,,\enskip  
i= 3L^2+1\,.
$$
     
     Аналогично, согласно~\cite{13-bos}, мож\-но положить 
     $$
     \mathbf{M}\left[C_{si}(v_s)\right]= 1{,}599 \cdot 10^6,\enskip s= \overline{1,S}\,,\enskip 
i= 3L^2+2\,.
$$

\section{Пример}

     Выберем промежуток времени~$\mathcal{T}$ с~2013 по \mbox{2016~гг}. 
Согласно~\cite{7-bos}, $p_{\mathrm{общ}}\hm= 246$, $p_1\hm= 150$, $p_2\hm= 
38$ и~$p_3\hm= 46$. Данные о~величинах~$Q_1$ и~$Q_2$ в~открытых источниках 
за данный период отсутствуют. Предположим, что $Q_1^\prime\hm= 1{,}8\cdot 
10^{11}$ и~$Q_2^\prime\hm= 3\cdot 10^9$. Отметим, что при таких значениях чисел 
$Q_1^\prime$ и~$Q_2^\prime$ ве\-ро\-ят\-ность~(\ref{e2-bos}) имеет тот же порядок, 
что и~аналогичная ве\-ро\-ят\-ность из~\cite{4-bos}. Пусть $L_0\hm= 2$, $L_2\hm= 58$, 
$w\hm= 5000$ и~$S\hm= 250\,000$. Положим $d_1\hm= d_2\hm= 20$ и~$d_3 \hm= d_4=
\cdots = d_{L_0+L_1} \hm= 14$. Таким образом, длина со\-ста\-ва $S_0\hm= 852$. 
Пусть движение происходит по двухпутной железной дороге, т.\,е.\ $\chi_s\hm=1$, 
$s\hm= \overline{2-S_0,S}$, железнодорожных переездов нет, т.\,е.\ $\eta_s\hm=0$, 
$s\hm= \overline{2-S_0,S}$, а~стрелочные переводы располагаются не в~пределах 
станций ($\xi_s\hm=0$, $s\hm= \overline{2-S_0,S}$), но на каж\-дом ки\-ло\-мет\-ре пути 
так, что

\vspace*{-4pt}

\noindent
\begin{multline*}
\lambda_s={}\\
{}=\begin{cases}
1  & \!\!\mbox{если}\ s= \overline{(j\!-\!1)\cdot 1000+1{,}(j\!-\!1)\cdot 1000+30};\hspace*{-5.97627pt}\\ 
0 & \!\!\mbox{иначе},
\end{cases}
\end{multline*}

\vspace*{-3pt}

\noindent
 где $j\hm= \overline{1,\lfloor S/1000\rfloor}$.
    
Зададим в~табл.~2 кривизну и~уклон на пути движения.

     
     Выберем различные режимы движения, т.\,е.\ наборы скоростей (табл.~3).


     
     Отметим, что для всех режимов время проследования маршрута одинаково и~со\-став\-ля\-ет 20\,000~с.
     
     Для задания функций $g_1(\cdot)$ и~$g_3(\cdot)$, а~также па\-ра\-мет\-ров~$a_1$, 
$a_3$, $\theta_1$ и~$\theta_3$ воспользуемся приведенными в~\cite{7-bos} 
результатами. Функцию $g_2(\cdot)$ выберем отличную от~\cite{7-bos}, а~именно: 
положим 

\vspace*{-4pt}

\noindent
     \begin{multline*}
     g_2(a_2,v,l,w,L_1, L, \tilde{\mu},\ae,\gamma) = {}\\
     {}= \exp \left\{ -3{,}7+0{,}97\tilde{\mu} +0{,}2\ln (v) \mathbf{1}_{\gamma>0} 
+{}\right.\\
\left.{}+1{,}33\ln (L-l+1)\right\}.
     \end{multline*}
Здесь 

\vspace*{-4pt}

\noindent
$$
\mathbf{1}_{\gamma>0}=
\begin{cases}
 1, & \mbox{если}\  \gamma> 0;\\
 0 & \mbox{в~противном\ случае.}
 \end{cases}
 $$
 
 \vspace*{-3pt}
 
 \noindent
  Такая функ\-ция $g_2(\cdot)$ 
обеспечивает большее значение функ\-ции максимального прав\-до\-по\-до\-бия, нежели 
функ\-ция из~\cite{7-bos}.

     Вычисленные значения функций интегрального риска на рас\-смат\-ри\-ва\-емых 
режимах движения приведены в~табл.~4.
     

     
     Прокомментируем полученные в~табл.~4 результаты. Значения функции 
$R_1(\cdot)$ на рассмотренных режимах движения ожи\-да\-емо примерно рав\-ны 
(в~табл.~4 данные приведены с~округ\-ле\-ни\-ем), так как в~предложенной концепции 
оценки вероятностей неблагоприятных событий величина ${\sf P}_s$ не зависит от 
скорости, а~$\forall\,s\hm\in \{1,\ldots , S\}\ \forall\,v_s\hm>0$ имеем 
$\sum\nolimits_{i=1}^{3L^2} {\sf P}_{si} (v_s) \hm\approx {\sf P}_s$. Отметим, что здесь 
используется при\-бли\-жен\-ное, а~не точ\-ное равенство, так как у~при\-ме\-ня\-емо\-го при 
прогнозировании чис\-ла по\-движ\-ных единиц в~сходе с~рельсов отрицательного 
биномиального распределения бес\-ко\-неч\-ное чис\-ло значений.



     
     Комментируя значения функции $R_2(\cdot)$, следует обратить внимание на 
небольшие по абсолютной величине значения. Хотя по\-след\-ст\-вия схода могут 
исчисляться миллионами руб\-лей, сред\-ний ущерб на
рас\-смот\-рен\-ных данных 
со\-став\-ля\-ет порядка 170~руб. Это вызвано тем, что сход с~рельсов является крайне 
ред\-ким событием. Действительно, со\-глас\-но~(\ref{e2-bos})
вероятность схода 
со\-став\-ля\-ет $2{,}05\cdot 10^{-5}$. Далее,\linebreak\vspace*{-12pt}

%\setcounter{table}{1}
    % \begin{table*}[b]\small %tabl2
    \vspace*{-2pt}
\begin{center}
\noindent
{{\tablename~2}\ \ \small{Карта подъемов, спусков и~кривых
}}

\vspace*{3pt}

{\small \tabcolsep=10.2pt
\begin{tabular}{|c|c|c|}
\hline
$s$&$\ae_s$&$\gamma_s$\\
\hline
\hphantom{9999}$[1, 50000]$&1/2000&$-$0,0001\\                      
$[50001, 60000]$&1/1500&$-$0,009\hphantom{9}\\                   
$[60001, 70000]$&1/1500&$-$0,003\hphantom{9}\\                   
\hphantom{9}$[70001, 100000]$&0&0,01\hphantom{\,}\\                           
$[100001, 110000]$&1/800&$-$0,004\hphantom{9}\\                  
$[-850, 0]$\hphantom{$-$99}&0&0\hphantom{999}\\    
 $[110001, 150000]$&1/1000&\hphantom{,}0,008\\
 $[150001, 200000]$&0&0\hphantom{999}\\       
 $[200001, 205000]$&1/600&\hphantom{,}0,005\\
 $[205001, 220000]$&0&$-$0,01\hphantom{99}\\ 
 $[220001, 250000]$&0&\hphantom{,}0,005\\   
 \hline
\end{tabular}
}
\end{center}
\vspace*{-6pt}
%\end{table*}
%\begin{table*}\small %tabl3
\begin{center}

\noindent
\parbox{65mm}{{{\tablename~3}\ \ \small{Значения скоростей (в~м/c) при различных режимах движения
}}
}

\vspace*{3pt}

{\small \begin{tabular}{|c|c|c|c|}
\hline
$s$&$v_s^1$&$v_s^2$&$v_s^3$\\
\hline
\hphantom{9999}$[1, 50000]$&12,5&16&16,34\\                                               
$[50001, 60000]$&12,5&10&11,5\hphantom{9}\\                                            
$[60001, 70000]$&12,5&\hphantom{,9}12,5&12,5\hphantom{9}\\                                          
\hphantom{9}$[70001, 100000]$&12,5&10&10,5\hphantom{9}\\                                           
$[100001, 110000]$&12,5&10&12,5\hphantom{9}\\   
 $[110001, 150000]$&12,5&\hphantom{,9}12,5&10\hphantom{,99}\\     
 $[150001, 200000]$&12,5&20&16,28\\    
 $[200001, 205000]$&12,5&\hphantom{9}5&10,5\hphantom{9}\\      
 $[205001, 220000]$&12,5&\hphantom{9}6&10\hphantom{,99}\\        
 $[220001, 250000]$&12,5&16&\hphantom{9}11,692\\   
\hline
\end{tabular}
}
\end{center}
%\end{table*}
%\begin{table*}\small %tabl4
\vspace*{6pt}

\noindent
{{\tablename~4}\ \ \small{Интегральные функции рис\-ка при различных режимах движения
}}

\vspace*{3pt}

\begin{center}
{\small 
\tabcolsep=6.4pt
\begin{tabular}{|c|c|c|}
\hline
Режим движения & $R_1\left(v_1, \ldots , v_s\right)$ & $R_2\left(v_1, \ldots , v_s\right)$ \\
\hline
&&\\[-10pt]
1&$2{,}046\cdot 10^{-5}$ & 168,68\\
2& $2{,}046\cdot 10^{-5}$ & 183,6\hphantom{9}\\
3&$2{,}046\cdot 10^{-5}$ & 167,86\\
\hline
\end{tabular}
}
\end{center}
%\end{table*}

\vspace*{6pt}


\noindent
   согласно~\cite{7-bos} в~среднем сходит 
4,23~по\-движ\-ные единицы. При предполагаемом максимальном ущербе от схода 
по\-движ\-ной единицы в~4,5~млн руб.\ и~сред\-ней ско\-рости в~12,5~м/c получаем, что 
в~сред\-нем ущерб от схода со\-став\-ля\-ет порядка 220~руб. Таким образом, порядок 
цифр верен. Однако в~то же время возникает вопрос, зачем использовать 
рас\-смот\-рен\-ный в~\mbox{статье} слож\-ный аппарат для оценивания функции $R_2(\cdot)$. 
Ответ в~том, что благодаря пред\-став\-лен\-ной концепции выяснилось, что 
\textit{сред\-ний ущерб при различных режимах движения может отличаться на~10\%}. 
А~это значит, что выбором скоростного режима мож\-но 
существенно со\-кра\-тить сред\-ний ущерб от схода.


\vspace*{-9pt}
     
\section{Заключение}

\vspace*{-3pt}

     Разработаны функции интегрального рис\-ка возникновения различных 
неблагоприятных событий, со\-пут\-ст\-ву\-ющих движению на рельсовом транспорте. 
Предложена концепция оценивания ве\-ро\-ят\-ности и~ущер\-ба от различных 
неблагоприятных событий, слу\-ча\-ющих\-ся во время движения грузовых поездов. 
На содержательном примере показано, что выбор скоростного режима может 
существенно по\-вли\-ять на сред\-ний ущерб при осуществлении перевозок. Результат 
дает импульс к~дальнейшему развитию старых и~исследованию новых моделей:
     \begin{itemize}
  \item прогнозирования места схода с~рельсов;
  \item прогнозирования причины схода с~рельсов;
  \item прогнозирования номера пер\-вой сошедшей с~рельсов по\-движ\-ной единицы;
  \item оценивания материального ущерба от схода с~рельсов при различных 
факторах дви\-жения.
  \end{itemize}
  
  \vspace*{-6pt}
  
{\small\frenchspacing
 {%\baselineskip=10.8pt
 %\addcontentsline{toc}{section}{References}
 \begin{thebibliography}{99}
\bibitem{1-bos}
 ГОСТ Р ИСО 31000-2019. Менеджмент риска. Принципы и~руководство.~--- М.: Стандартинформ, 2020. 19~с.
\bibitem{2-bos}
 ГОСТ 33433-2015. Безопас\-ность функциональная. Управ\-ле\-ние рисками на железнодорожном 
транс\-порте.~--- М.: Стандартинформ, 2016. 34~с.
\bibitem{3-bos}
\Au{Saccomanno F.\,F., El-Hage~S.} Establishing derailment profiles by position for corridor 
shipments of dangerous goods~// Can. J. Civil Eng., 1991. Vol.~18. P.~67--75.
\bibitem{4-bos}
\Au{Anderson R.\,T., Barkan~C.\,P.\,L.} Derailment probability analyses and modeling of mainline 
freight trains~// 8th  Heavy Haul Conference (International) Proceedings.~--- Rio de Janeiro, Brazil, 
2005. P.~491--497.
\bibitem{5-bos} 
\Au{Bagheri M., Saccomanno~F., Chenouri~S., Fu~L.} Reducing the threat of in-transit 
derailments involving dangerous goods through effective placement along the train consist~// Accident 
Anal.  Prev., 2011. Vol.~43. No.\,3. P.~613--620.
\bibitem{6-bos}
\Au{Liu~X., Saat M.\,R., Qin~X., Barkan~C.\,P.\,L.} Analysis of U.S.\ freight-train derailment 
severity using  
zero-truncated negative binomial regression and quantile regression~// Accident Anal.  Prev., 
2013. Vol.~59. P.~87--93.
\bibitem{7-bos}
\Au{Zamyshliaev A.\,M., Ignatov~A.\,N., Kibzun~A.\,I., Novozhilov~E.\,O.} Functional 
dependency between the number of wagons derailed due to wagon or track defects and the traffic 
factors~// Dependability, 2018. Vol.~18. No.\,1. P.~53--60.
\bibitem{8-bos}
\Au{Zamyshliaev A.\,M., Ignatov~A.\,N., Kibzun~A.\,I., Novozhilov~E.\,O.} On traffic safety 
incidents caused by intrusion of derailed freight cars into the operational space of an adjacent track~// 
Dependability, 2018. Vol.~18. No.\,3. P.~39--45.
\bibitem{9-bos}
\Au{Bagheri M., Saccomanno~F., Fu~L.} Effective placement of dangerous goods cars in rail yard 
marshaling operation~// Can. J. Civil Eng., 2010. Vol.~37. P.~753--762.
\bibitem{10-bos}
\Au{Rahbar M., Bagheri~M.} Risk assessment framework for the rail transport of hazardous 
materials: Formulation and solution~// Transp. Res. Rec., 2014. Vol.~2411. P.~90--95.
\bibitem{11-bos}
\Au{Босов А.\,В., Игнатов~А.\,Н., Наумов~А.\,В.} Модель передвижения поездов и~маневровых 
локомотивов на железнодорожной станции в~приложении к оценке и~анализу ве\-ро\-ят\-ности 
бокового столк\-но\-ве\-ния~// Информатика и~её применения, 2018. Т.~12. Вып.~3. C.~107--114.
\bibitem{12-bos}
\Au{Игнатов А.\,Н., Кибзун~А.\,И., Платонов~Е.\,Н.}
 Оценка вероятности столкновения составов на железнодорожных станциях на основе пуассоновской модели~//
 Автоматика и~телемеханика, 2016.
№\,11. С.~43--59.
 EDN: XWTSYD.
 
  
\bibitem{13-bos}
\Au{Shubinsky I.\,B., Zamyshliaev~A.\,M., Ignatov~A.\,N., Kibzun~A.\,I.} Use of automatic 
signalling system for reduction of the risk of transportation incidents in railway stations~// 
Dependability, 2017. Vol.~17. No.\,3. P.~49--57.
\end{thebibliography}

 }
 }

\end{multicols}

\vspace*{-8pt}

\hfill{\small\textit{Поступила в~редакцию 24.10.22}}

\vspace*{6pt}

%\pagebreak

%\newpage

%\vspace*{-28pt}

\hrule

\vspace*{2pt}

\hrule

%\vspace*{-2pt}

\def\tit{ON THE PROBLEM OF~ASSESSING AND~ANALYZING TRAFFIC ACCIDENTS 
RISK ON~THE~RAIL TRANSPORT}


\def\titkol{On the problem of~assessing and~analyzing traffic accidents 
risk on~the~rail transport}


\def\aut{A.\,V.~Bosov$^{1,2}$ and~A.\,N.~Ignatov$^2$}

\def\autkol{A.\,V.~Bosov and~A.\,N.~Ignatov}

\titel{\tit}{\aut}{\autkol}{\titkol}

\vspace*{-15pt}


\noindent
$^1$Federal Research Center ``Computer Science and Control'' of the Russian Academy of Sciences, 44-2~Vavilov\linebreak
$\hphantom{^1}$Str., Moscow 119333, Russian Federation

\noindent
$^2$Moscow State Aviation Institute (National Research University), 4~Volokolamskoe Shosse, Moscow 125933,\linebreak
$\hphantom{^1}$Russian Federation


\def\leftfootline{\small{\textbf{\thepage}
\hfill INFORMATIKA I EE PRIMENENIYA~--- INFORMATICS AND
APPLICATIONS\ \ \ 2023\ \ \ volume~17\ \ \ issue\ 1}
}%
 \def\rightfootline{\small{INFORMATIKA I EE PRIMENENIYA~---
INFORMATICS AND APPLICATIONS\ \ \ 2023\ \ \ volume~17\ \ \ issue\ 1
\hfill \textbf{\thepage}}}

\vspace*{3pt} 




\Abste{The problem of assessing and analyzing traffic accidents risk on the rail 
transport is considered. Two functions of the integral risk are proposed that allow 
assessing danger of transportation along the entire route of a~transport. The probability 
of an unfavorable event occurring during transportation and the expected damage are 
chosen as such functions. The concept of assessing probability and damage from 
unfavorable events during the freight trains transportation is proposed. A~meaningful 
example of calculating integral risk functions is given on the basis of previously 
investigated statistics on the freight trains transportation and unfavorable events that 
occurred with them.}

\KWE{risk; unfavorable event; rail transport; probability; expected damage}



 \DOI{10.14357/19922264230110} 

\vspace*{-14pt}

\Ack

\vspace*{-4pt}

\noindent
The work was partially supported by the Russian Foundation for Basic Research (grant 
20-07-00046~А).

%\vspace*{4pt}

  \begin{multicols}{2}

\renewcommand{\bibname}{\protect\rmfamily References}
%\renewcommand{\bibname}{\large\protect\rm References}

{\small\frenchspacing
 {%\baselineskip=10.8pt
 \addcontentsline{toc}{section}{References}
 \begin{thebibliography}{99} 
\bibitem{1-bos-1}
 GOST R ISO 31000-2019. 2020. Menedzhment riska. Printsipy i~rukovodstvo [Risk 
management. Principles and guidelines]. Moscow: Standartinform. 19~p.
\bibitem{2-bos-1}
GOST 33433-2015. 2016. Bezopasnost' funktsional'naya. Upravlenie riskami na 
zheleznodorozhnom transporte [Functional safety. Risk control in railroad transport]. 
Moscow: Standartinform. 34~p. 
\bibitem{3-bos-1}
\Aue{Saccomanno, F.\,F., and S.~El-Hage.} 1991. Establishing derailment profiles by 
position for corridor shipments of dangerous goods. \textit{Can. J. Civil 
Eng.} 18: 67--75.
\bibitem{4-bos-1}
\Aue{Anderson, R.\,T., and C.\,P.\,L.~Barkan.} 2005. Derailment probability analyses 
and modeling of mainline freight trains. \textit{8th Heavy Haul Conference 
(International) Proceedings}. Rio de Janeiro, Brazil. 491--497.
\bibitem{5-bos-1}
\Aue{Bagheri, M., F.~Saccomanno, S.~Chenouri, and L.~Fu.} 2011. Reducing the threat 
of in-transit derailments in-volving dangerous goods through effective placement along 
the train consist. \textit{Accident Anal. Prev.} 43(3):613--620.
\bibitem{6-bos-1}
\Aue{Liu, X., M.\,R.~Saat, X.~Qin, and C.\,P.\,L.~Barkan.} 2013. Analysis of U.S.\ 
freight-train derailment severity using zero-truncated negative binomial regression and 
quantile regression. \textit{Accident Anal. Prev.} 59:87--93.
\bibitem{7-bos-1}
\Aue{Zamyshliaev, A.\,M., A.\,N.~Ignatov, A.\,I.~Kibzun, and E.\,O.~Novozhilov.} 
2018. Functional dependency between the number of wagons derailed due to wagon or 
track defects and the traffic factors. \textit{Dependability} 18(1):53--60.
\bibitem{8-bos-1}
\Aue{Zamyshliaev, A.\,M., A.\,N.~Ignatov, A.\,I.~Kibzun, and E.\,O.~Novozhilov.} 
2018. On traffic safety incidents caused by intrusion of derailed freight cars into the 
operational space of an adjacent track. \textit{Dependability} 18(3):39--45.
\bibitem{9-bos-1}
\Aue{Bagheri, M., F.~Saccomanno, and L.~Fu.} 2010. Effective placement of dangerous 
goods cars in rail yard marshaling operation. \textit{Can. J. Civil Eng.} 
37:753--762.
\bibitem{10-bos-1}
\Aue{Rahbar, M., and M.~Bagheri.} 2014. Risk assessment framework for the rail 
transport of hazardous materials: Formulation and solution. \textit{Transp. 
Res. Rec.} 2411(1):90--95.
\bibitem{11-bos-1}
\Aue{Bosov, A.\,V., A.\,N.~Ignatov, and A.\,V.~Naumov.} 2018. Mo\-del' pe\-re\-dvi\-zhe\-niya 
po\-ez\-dov i~ma\-nev\-ro\-vykh lo\-ko\-mo\-ti\-vov na zhe\-lez\-no\-do\-rozh\-noy stan\-tsii v~pri\-lo\-zhe\-nii 
k~otsen\-ke i~ana\-li\-zu ve\-ro\-yat\-nosti bo\-ko\-vo\-go stolk\-no\-ve\-niya [Transportation of trains and 
shunting locomotives at the railway station model for evaluating and analysis of side-collision probabilities]. 
\textit{Informatika i~ee Primeneniya~--- Inform. Appl.} 
12(3):107--114.
\bibitem{12-bos-1}
\Aue{Ignatov, A.\,N., A.\,I.~Kibzun, and E.\,N.~Platonov.} 2016. Estimating collision 
probabilities for trains on railroad stations based on a~Poisson model. \textit{Automat. 
Rem. Contr.} 77:1914--1927.
\bibitem{13-bos-1}
\Aue{Shubinsky, I.\,B., A.\,M.~Zamyshliaev, A.\,N.~Ignatov, and A.\,I.~Kibzun.} 2017. 
Use of automatic signalling system for reduction of the risk of transportation incidents in 
railway stations. \textit{Dependability} 17(3):49--57.
\end{thebibliography}

 }
 }

\end{multicols}

\vspace*{-6pt}

\hfill{\small\textit{Received October 24, 2022}}

\Contr

\noindent
\textbf{Bosov Alexey V.} (b.\ 1969)~--- Doctor of Science in technology, principal 
scientist, Institute of Informatics Problems, Federal Research Center ``Computer 
Science and Control'' of the Russian Academy of Sciences, 44-2~Vavilov Str., Moscow 
119333, Russian Federation; professor, Moscow State Aviation Institute (National 
Research University), 4~Volokolamskoe Shosse, Moscow 125933, Russian Federation; 
\mbox{AVBosov@ipiran.ru}

\vspace*{3pt}

\noindent
\textbf{Ignatov Aleksei N.} (b.\ 1991)~--- Candidate of Science in physics and 
mathematics, assistant professor, Moscow Aviation Institute (National Research 
University), 4~Volokolamskoe Shosse, Moscow 125933, Russian Federation; 
\mbox{alexei.ignatov1@gmail.com}


  
\label{end\stat}

\renewcommand{\bibname}{\protect\rm Литература} 