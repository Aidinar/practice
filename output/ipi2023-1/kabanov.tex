

\renewcommand{\figurename}{\protect\bf Figure}
\renewcommand{\tablename}{\protect\bf Table}

\def\stat{kabanov}


\def\tit{AN AXIOMATIC VIEWPOINT ON THE ROGERS--VERAART AND~SUZUKI--ELSINGER MODELS OF~SYSTEMIC RISK}

\def\titkol{An axiomatic viewpoint on the Rogers--Veraart and Suzuki--Elsinger models of systemic risk}

\def\autkol{Yu.\,M.~Kabanov and~A.\,P.~Sidorenko}

\def\aut{Yu.\,M.~Kabanov$^{1,2}$ and~A.\,P.~Sidorenko$^1$}


\titel{\tit}{\aut}{\autkol}{\titkol}

%{\renewcommand{\thefootnote}{\fnsymbol{footnote}}
%\footnotetext[1] {The first author has partially been supported by the Russian Foundation 
%for Basic Research under grant 15-07-05316. The second author has 
%partially been supported by the internal research program of the Shamoon College 
%of Engineering.}}

\renewcommand{\thefootnote}{\arabic{footnote}}
\footnotetext[1]{M.\,V.~Lomonosov Moscow State University, 1-52~Leninskie Gory, GSP-1, Moscow 119991, Russian Federation }
\footnotetext[2]{Federal Research Center ``Computer Science and Control'' of the Russian Academy of Sciences, 44-2~Vavilov Str., Moscow 119333, Russian Federation}

\vspace*{-9pt}

\def\leftfootline{\small{\textbf{\thepage}
\hfill INFORMATIKA I EE PRIMENENIYA~--- INFORMATICS AND APPLICATIONS\ \ \ 2023\ \ \ volume~17\ \ \ issue\ 1}
}%
 \def\rightfootline{\small{INFORMATIKA I EE PRIMENENIYA~--- INFORMATICS AND APPLICATIONS\ \ \ 2023\ \ \ volume~17\ \ \ issue\ 1
\hfill \textbf{\thepage}}}

\vspace*{-6pt}


\index{Kabanov Yu.\,M.}
\index{Sidorenko A.\,P.}
\index{Кабанов Ю.\,М.}
\index{Сидоренко А.\,П.}
 

\Abste{The authors study a~model of clearing in an interbank network with crossholdings and default charges.  
Following the  Eisenberg--Noe approach, the authors  define the model via a~set of natural financial regulations 
including those related to eventual default charges and derive a~finite family of fixpoint problems. 
These problems are parameterized by vectors of binary variables. The model combines features of the Ararat--Meimanjanov, 
Rogers--Veraart,  and Suzuki--Elsinger networks. 
The authors develop methods of computing the maximal and  minimal clearing pairs using the mixed integer-linear 
programming and a~Gaussian elimination algorithm.}  

   \KWE{systemic risks; financial networks; clearing; crossholdings; default charges} 

\DOI{10.14357/19922264230102} 

%\vspace*{-6pt}


\vskip 12pt plus 9pt minus 6pt

      \thispagestyle{myheadings}

      \begin{multicols}{2}

                  \label{st\stat}
                
    
\section{Introduction}

\noindent
In a~financial system with strongly interconnected institutions, a~shock touching even 
a~small number of entities can propagate through the network and lead to substantial losses. 
Clearing is a~procedure that diminishes the total amount of interbank liabilities and, as such, decreases the risk of 
a~system-wide breakdown. In the seminal paper~\cite{eisenberg2001systemic} by Eisenberg and Noe, the  notion of clearing vectors
 was introduced in an axiomatic way   via  limited liability and absolute priority rules.  
 It was shown that the set of clearing vectors is the set of fixed points of 
 a~simple nonlinear mapping and this set contains the maximal and minimal  elements. 
 The further development led to more sophisticated models including crossholdings, seniority of debts, and default charges (see, 
 e.\,g.,~[2--5] %\cite{suzuki2002valuing, elsinger2009financial, RV2013}) 
 and a~survey paper~\cite{kabanov2018clearing}). 
 Dynamic versions of the Eisenberg--Noe network are suggested, for instance, in~\cite{feinstein2019dynamic, djete2021mean}. 

The adding of default fees introduced in the paper by Rogers and Veraart \cite{RV2013} is of particular interest 
because it allows to study the problem of a~rescue consortium to aid  insolvent institutions. 
In general, default charges have a~crisis amplifying effect: external payoffs may increase the number 
of  failures. 
In the Rogers--Veraart model, the clearing vectors were defined directly as solutions of a~fixpoint equation.  
As was observed by Ararat and Meimanjanov in~\cite{Ararat2019}, such a~formulation does not coincide with a~formulation defined in terms  of financial regulations. 


In this paper, the authors consider  a~model with default charges and crossholdings using the Ararat--Meimanjanov
``axiomatic'' approach. The aim is to derive the fixpoint equations and suggest algorithms to find their 
minimal and maximal solutions.  In particular,  a~method using the mixed integer-linear programming
and  a~Gaussian-type dimensionality reduction algorithm is considered. 

\smallskip

\noindent
\textbf{Notations:} For vectors $a, b \in \mathbb{R}^N$, the symbol $a\le b$ denotes the componentwise partial ordering
 $a^i \leq b^i$ for every $i = 1, \dots, N$,  $[a,b]:=\{x\in 
\mathbb{R}^N\colon a\linebreak \le x\le b\}$, ${\bf 1}_{\{a<b\}}$
is the vector with the components  $I_{\{a^i< b^i\}}$,  ${\bf 1}:=(1,\dots,1)$, ${\bf 0}:=(0,\dots,0)$.    
The symbol  $a\circ b:=(a^1b^1,\dots,a^Nb^N)$ stands for the Hadamard (componentwise) product.  


 
%combining the amplifying defaualt  from Rogers--Veraart and Suzuki--Elsinger networks, namely, cost of default and cross holdings. We provide a~set of axioms for such a~model in the spirit of \cite{eisenberg2001systemic} and \cite{Ararat2019} and derive a~class of fixpoint problems whose solutions cover the set of clearing vectors. In particular, the equations for the maximal and minimal clearing vectors were obtained. It turns out that the equation for the maximal clearing vector coincides with the conventional fixpoint problems from \cite{RV2013} and \cite{elsinger2009financial}. The suggested set of axioms does not contradict with the conventional approaches, but rather complements them. On top of that, particular methods of solving the resulting fixpoint problems are discussed. A generalized Gaussian method introduced from \cite{kabanov2018clearing} was elabora\-ted. In the spirit of \cite{Ararat2019}, mixed
%integer-linear programming problems for the maximal and the minimal solutions were also formulated.

\vspace*{-2pt}

\section{Suzuki--Elsinger Model with~Default~Payments}

\vspace*{-4pt}


\noindent
Let us consider a~financial network consisting of $N>1$ banks. The bank~$i$ has a~cash reserve~$e^i$, a~liability~$l^{ij}$ towards the bank 
$j\neq i$ with the total  $l^i=\sum\nolimits_j l^{ij}$, and possesses a~share~$\theta ^{ji}$ of the bank~$j$.  
It is convenient to introduce the   relative liabilities matrix $\Pi=(\pi^{ij})$ with 

\vspace*{2pt}

\noindent
\begin{equation*}
%\label{sm}
\pi^{ij}:=\begin{cases}
\fr {l^{ij}}{l^i}=\fr{l^{ij}}{\sum\nolimits_j l^{ij}}\,, & \mbox{if \ }l^i\neq 0\,;\\
\delta^{ij} &  \mbox{otherwise} 
\end{cases}
\end{equation*}

\vspace*{-4pt}

\noindent
where the Kronecker symbol $\delta^{ij}=0$ for $i\neq j$ and $\delta^{ii}=1$.
Then~$\pi^{ij}$  describes the fraction of the value of the debtor~$i$ due to the creditor~$j$ of the total interbank debt of~$i$.  The value $\pi^{ii}=1$ means that 
the bank~$i$ has no interbank debts. 

The matrix $\Theta=(\theta^{ij})$ is assumed to be substochastic and such that unit is not its eigenvalue. 

 Fix $\alpha, \beta, \gamma \in [0,1]$.  
It is assumed that there is  Central Clearing Counterparty (CCP) calculating  \textit{clearing}\linebreak\vspace*{-12pt}

\pagebreak

\noindent
 and \textit{equity}  vectors, 
denoting~$p$ and~$V$, and provide  the settlement service for creditors and debtors.   

A \textit{clearing pair} $(p,V)\in [0,l]\times  \mathbb{R}^N_+$ is determined by the following rules.   
\begin{enumerate}[1.]
    \item
    %[$\bullet$] 
    {\bf Limited liability}: $p^i \leq (e + \Pi' p + \Theta' V)^i$ for every~$i$.
    \item
    %[$\bullet$] 
    {\bf Absolute priority}: for every $i$,
    either $p^i = l^i$ or\linebreak $p^i = \left(\alpha e + \beta \Pi' p + \gamma \Theta' V\right)^i$. 
    \item
    %[$\bullet$] 
    {\bf Equity evaluation}: for every $i$, if $p^i = l^i$, then\linebreak $V^i = \left(e + \Pi' p + \Theta' V - p\right)^i$, otherwise $V^i = 0$.
\end{enumerate}

The  first rule means that the clearing payment cannot exceed the available resources 
(cash~$e^i$ plus collected debts $\sum\nolimits_j \pi^{ji}p^j$ plus the total of owned shares $\sum\nolimits_j \theta^{ji} V^j$) while
the second rule stipulates that either debts are payed in full or all resources are distributed with the charges payed in the case of the default, namely, the amount 
$(1-\alpha)e^i + (1-\beta) \left(\Pi'p\right)^i+(1- \gamma)\left( \Theta' V\right)^i.
$ 
The third relation is just the definition of the equity:  $V^i$ is the total value of  assets of the bank~$i$ after clearing  (corresponding to the clearing vector~$p$). 
By virtue of the limited liability rule, all components $V^i\ge 0$. 

Denote by $\mathcal{P}$ the set of clearing pairs.

 Put 
\begin{align*}
G(x, V) &:=  \left(e + \Pi' x + \Theta' V - l\right) \circ {\bf 1}_{\{x = l\}};\\
G_+(x, V) &:=  \left(e + \Pi' x + \Theta' V - l\right)^+ \circ {\bf 1}_{\{x = l\}}. 
\end{align*}
In this notation, if $(p,V)$ is  a~clearing pair, (i.\,e., a~clearing vector and an equity vector), then   $V = G(p, V)$.

\smallskip

\noindent
\textbf{Lemma~2.1}\ %\label{lem21}
\textit{If  $x \in \mathbb{R}^N$, then  equations $V = G(x, V)$ and $V = G_+(x, V)$ have unique solutions.  
As functions of~$x$, the solution of the first equation is  affine and the solution of the second  one is  convex increasing}. 


\noindent 
P\,r\,o\,o\,f\,.\ \
Put $\Lambda := \mathrm{diag}\,{\bf 1}_{\{x=l\}}$, $a(x): = \Lambda (e + \Pi'x - x)$, and $B := \Theta \Lambda$.
In this notation,
$$ 
G(x, V) = a(x) + B' V. 
$$
The  linear equation $V=G(x,V)$ has a~unique solution if the matrix 
$I-B$ is invertible, i.\,e., unit is not an eigenvalue of~$B$ (recall that this property is  assumed). 

 
 Analogously, for $c(x) := \Lambda (e + \Pi'x - l)$, $G_+(x, V)$ can be written in the form 
$$ G_+(x, V) = (c(x) + B' V)^+.
 $$
It remains to apply   Lemma~4.3 from~\cite{kabanov2018clearing} claiming that for every $y\in \mathbb{R}^N$, 
the equation $v=(y+\tilde \Theta v)^+$, where $\tilde \Theta$ is a~substochastic matrix with invertible $I-\tilde \Theta$, 
has a~unique solution $v=v(y)$ which is a~monotone increasing  convex function and so is the function $H(x):=v(c(x))$.~\hfill$\square$ 

\smallskip

\noindent 
R\,e\,m\,a\,r\,k\,.\ \
In the above lemma, the existence can be deduced from the Knaster--Tarski theorem. Indeed, put 
$z := e + \Pi' l - l$, $K := (I - \Theta')^{-1} z^+$. Then, for any $p \in [0, l]$,
\begin{multline*}
G_+(p, K) \leq G_+(l, K) \\
{}= \left(z + \Theta' K\right)^+ \leq z^+ + \Theta' K =  K\,.  
\end{multline*}
It follows that  the monotone function  $V \mapsto G_+(p, V)$ maps $[0, K]$ into itself.



\smallskip

\noindent
\textbf{Lemma~2.2}\ %\label{lem22}
\textit{If $(p,V)\in\mathcal{P}$, then}
$V = G_+(p, V)$.

\smallskip

\noindent
P\,r\,o\,o\,f\,.\ \
As we have already observed, the equity evaluation means that   $V = G(p, V)$  and     
$p \le e + \Pi' p + \Theta' V$ because of the limited liability. The components~$G^i(p, V)$ and  $G^i_+(p, V)$
coincide if $p^i=l^i$ and are equal to zero if $p^i<l^i$. \hfill$\square$



\smallskip


If $\alpha=\beta=1$ and the  matrix $\Theta$ is zero (no crossholdings), the model 
is reduced to that of  Eisenberg--Noe.  If $\alpha,\beta\in ]0,1]$ and~$\Theta$ is zero,  one gets the Ararat--Meimanjanov 
model. The model of Suzuki--Elsinger has a~nontrivial substochastic matrix~$\Theta$ and  
$\alpha=\beta=\gamma=1$, i.\,e.,  default charges are not imposed. The relation with the Rogers--Veraart model we 
will discuss later. 

%\smallskip

To determine the clearing pairs, we  use the absolute priority rule. For $p\in [0,l]$,
we put   
\begin{align*}
\Phi^i_0(p, V) & :=  \left(\alpha e+\beta \Pi' p + \gamma \Theta' V\right)^i \wedge l^i;\\
\Phi^i_1(p, V) & :=  \left(\alpha e+\beta \Pi' p + \gamma \Theta' V\right)^i d^i+ \left(1 - d^i\right)l^i
\end{align*}
where 
$$
d^i := I_{\{ (e + \Pi' p + \Theta' V)^i < l^i \}}, \enskip i=1,\dots,N. 
$$ 

For a~binary vector  $b=(b^1,\ldots ,b^N) \in \{0, 1\}^N$ 
 on $\mathbb{R}^N\times \mathbb{R}^N$, we define the $\mathbb{R}^N$-valued function  $(p,V)\mapsto \Phi_b (p, V)$ with 
$$
\Phi_b (p, V) = \left(\Phi^1_{b^1}(p, V), \dots, \Phi^N_{b^N}(p, V) \right).
$$ 

In particular, for ${\bf 0}:=(0,\ldots ,0)$ and ${\bf 1}:=(1,\ldots ,1)$, we have 
\begin{align*} 
\Phi_{\bf 0}(p, V) &:=  \left(\alpha e+\beta \Pi' p  + \gamma \Theta' V\right)\wedge l\,; \\ 
\Phi_{\bf 1}(p, V) &:= \left(\alpha e+\beta \Pi' p + \gamma \Theta' V\right)\circ d  + ({\bf 1} - d) \circ l
\end{align*} 
where $d = (d^1, \dots, d^N)$. 

 


Let  $H(x)$ be the fixpoint $G_+(x, \cdot)$. By  Lemma~2.2, the mapping $H: \mathbb{R}^N \to \mathbb{R}^N_+$ is monotone.
Let $F^i_j(p) := \Phi^i_j(p, H(p))$, $i=1,\dots, N$,   $j=0,1$, and~let  
\begin{multline*}
F_b(p) := \Phi_b(p, H(p))={}\\
{}=\left( \Phi^1_b(p, H(p)),\dots,  \Phi^N_b(p, H(p))\right), \enskip b\in \{0,1\}^N.  
\end{multline*}
It is easily seen that  $(b,p)\mapsto F_b(p)$ is a~monotone mapping of 
$\{0,1\}^N\times [0,l]$ into $[0,l]$ with respect to the componentwise ordering.
By the Knaster--Tarski theorem (see, e.\,g.,~\cite{kabanov2018clearing}), the set of its fixpoints is
 nonempty and contains the minimal and maximal elements. By the same reason, for every $b\in \{0,1\}$, 
 the set of  fixpoints of the mapping $p\mapsto F_b(p)$ is nonempty and contains the minimal and maximal elements.
\smallskip

Let  $S:=\{p \in [0, l]\colon (p,V)\in \mathcal{P}\}$, i.\,e., $S$ is the projection of the subset~$\mathcal{P}$ of the ``plane''' 
$\mathbb{R}^N\times \mathbb{R}^N$ into the ``$x$-axis'' $\mathbb{R}^N$. 

The following  result gives a~characterization of the set~$S$ of  clearing vectors. 

\smallskip

\noindent
\textbf{Theorem 2.3}\ %\label{RV-t1}
 $S=\bigcup_b\{p \in [0, l]\colon p=F_b(p)\}$.

\noindent
P\,r\,o\,o\,f\,.\ \
\smallskip
Define the set $
 W_b := \bigcap_{i=1}^{N} W^i_{b^i}$, $b \in \{0, 1\}^N$, where  $W^i_j := \{ p \in S \colon p^i = F^i_j (p)\}$. 
 
 Let $p\in S$, that is, $(p,V) \in \mathcal{P}$ for some $V=H(p)$  (see Lemma~2.2). There are two cases:  $p^i = l^i$ and $p^i < l^i$. In the first one, 
 $$
 l^i \le \left(e + \Pi' p + \Theta' V\right)^i=\left(e + \Pi' p + \Theta' H(p)\right)^i
 $$
implying that $F^i_1 (p)=l^i=p^i$; hence, $p \in W^i_1$.
In the second case,   by the absolute  priority rule, 
$p^i = (\alpha e + \beta \Pi' p + \gamma \Theta' H(p))^i$, that is, $p^i = F^i_0(p)$ and $p \in W^i_0$. Thus, 
 $S = W^i_0 \cup W^i_1$ whatever is~$i$ and  
 
 \vspace*{-4pt}
 
 \noindent
\begin{multline*}
 S=\bigcap_i \left(W^i_0 \cup W^i_1\right)\\
 {}=\bigcup_{b \in \left\{0, 1\right\}^N}  W_b\subseteq   \bigcup_b\left\{p \in [0, l]\colon p=F_b(p)\right\} .
\end{multline*}

\vspace*{-4pt}

To prove the converse inclusion,  fix some~$b$ and take $x \in [0, l]$ such that $x = F_b(x)$. Put $V = H(x)$. 
It is easily seen that $(x, V)$ satisfies all the axioms.\hfill$\square$


\smallskip

\noindent
\textbf{Corollary 2.4}\
%\label{RV_c1}
\textit{Let  $\underline p_b$ and $\bar p_b$ be the minimal and maximal  fixpoints of the mapping~$F_b$. Then 
 $\underline p_{\bf 0}$ and $ \bar p_{\bf 1}$ are the minimal and maximal  fixpoints of}~$S$.

\smallskip

\noindent
P\,r\,o\,o\,f\,.\ \ 
Since functions $F_b$ are increasing in~$b$, so are the~$\underline p_b$ and~$\bar p_b$ (see, e.\,g.,~\cite{kabanov2018clearing}). This implies the result. 
\hfill$\square$


\smallskip

Let us elaborate on the above statements. The maximal clearing vector $p^*$ is the  component of the vector  $(p^*, V^*)\in [0,l]\times \mathbb{R}^N_+$ which is the maximal solution of the system 
\begin{align*}
    p& =  \left(\alpha e+\beta \Pi' p + \gamma \Theta' V\right) \circ d + l \circ ({\bf 1} - d); \\
     V &=  \left( e + \Pi' p +  \Theta' V - l\right)^+ \circ ({\bf 1} - d)
\end{align*}

\vspace*{-4pt}

\noindent
where $d := {\bf 1}_{\{e + \Pi' p + \Theta' V < l \}}$. 
 

The minimal clearing vector~$p_*$ is the  component of the vector  $(p_*, V_*)\in [0,l]\times \mathbb{R}^N_+$ which is the minimal solution of the system 
\begin{align*}
    p & =  \left(\alpha e+\beta \Pi' p + \gamma \Theta' V\right) \wedge l\,; \\
     V & = \left( e + \Pi' p +  \Theta' V - l\right)^+ \circ {\bf 1}_{\{p = l\}} .
\end{align*}

Consider two particular cases. Let $\Theta = 0$. Then~$V$ is irrelevant. The equation for the maximal fixpoint is
$$
p =   \left(\alpha e+\beta \Pi' p \right) \circ d + l \circ ({\bf 1} - d)
$$
where $d := {\bf 1}_{\{e + \Pi' p < l \}}$. 

\columnbreak


This equation coincides with that introduced in~\cite{RV2013} as the definition of the clearing vector. 
It means that the maximal clearing vector in the ``axiomatic'' definition of~\cite{Ararat2019} and in the  
definition via fixpoint of~\cite{RV2013} coincide. However, in the axiomatic approach, the minimal clearing vector is the minimal solution of 
a~different equation

\noindent
$$ 
p = \left(\alpha e+\beta \Pi' p \right) \wedge l
$$

\vspace*{-4pt}

\noindent
and may be different  from that coming via fixpoint equation of~\cite{RV2013}. 

Our arguments are intended to show that  the  ``axiomatic'' description also leads 
to a~fixpoint problem allowing, in principle, to find all clearing vectors and corresponding equities. Unfortunately, the number of equations is exponentially 
growing. For example, for a~small financial system with only ten banks,  we have  20\,480~equations. On the other hand, it seems that the largest 
clearing vector is of a~major practical interest and to get it, one  can consider the system of only $2N$ equations suggested by Rogers--Veraart.   


Comparably to the  Rogers--Veraart approach, the axiomatic one might seem rather perplexing because it leads to~$2^N$ cases equations
 instead of just one. There is no  prospect that a~significant proportion of these equations is redundant and one can rule out them. 
 Indeed, take arbitrary~$\Pi$, $l$ with all  $l^i > 0$, put $\Theta = 0$, $e = l$, $\gamma = 1$, and $\beta=\alpha$. 
 Choose~$\alpha$ such that $\alpha (l + \Pi' l)^i < l^i$ for all~$i$. Obviously, the solution of the equation $p = F_b(p)$ has  
 the component  $p^i = l^i$ if $b^i=1$ and $p^i < l^i$, otherwise. In this example, the sets~$W_b$ are the disjoint singletons.
 


Now, let us consider the case where $\alpha = \beta = \gamma = 1$ as in the  
Suzuki--Elsinger model. In this case, $F_{\bf 0} = F_{\bf 1}$ and clearing vectors are the solutions of the system 

\vspace*{-4pt}

\noindent
\begin{align*}
 p &= \left(e + \Pi' p + \Theta' V\right) \wedge l; \\
V &= \left(e + \Pi' p + \Theta' V - l\right)^+ \circ {\bf 1}_{\{p = l\}}.
\end{align*}

\vspace*{-4pt}

\noindent
Obviously, the second equation can be replaced by $V = (e + \Pi' p + \Theta' V - l)^+$. 
Hence, the proposed ``axiomatic" definition coincides with that of Suzuki--Elsinger model in~\cite{elsinger2009financial}.


\vspace*{2pt}

A short remark on the interpretation of the parameter~$b$.  
Assume for simplicity that $\alpha = \beta = \gamma < 1$.    
Let   $(p, V)$ be a~clearing pair corresponding to some binary vector~$b$ 
and let  $x(p,V) := e + \Pi' p + \Theta'V$. The component~$x^i(p,V)$ is the total  assets of bank~$i$. 
For $z \in \mathbb{R}_+$, put $h^1_0(z) := (\alpha z) \wedge l^1$ and $h^1_1 (z) := \alpha z {\bf 1}_{ \{ z < l^1 \} } + l^1 {\bf 1}_{ \{ z \geq l^1 \} }$.
The value $h^1_{b^1}(x^1(p, V)) = \Phi^1_{b^1}(p, V)$  is the debt payment of bank~1.  
If $x^1(p, V) \in [l^1, l^1/\alpha)$, then  $h^1_0(x^1(p, V))\linebreak < h^1_1(x^1(p, V)) = l^1$. 
Let us compare clearing pairs $(p,V)$ and $(\bar p,\bar V)$ corresponding to the binary vectors $(0,\tilde b)$ and $(1,\tilde b)$ 
with all components equal except the first one. Then, $(p,V)\le (\bar p,\bar V)$ and 
$x(p,V)\le x(\bar p,\bar V)$.  It may happen that 
$ x^1(\bar p, \bar V) \in [l^1, l^1/\alpha)$ but $x^1( p, V) <l^1$.  
This leads to the conclusion that the banks are motivated to require  CCP to calculate maximal  clearing pairs using $b={\bf 1}$. 


\vspace*{-4pt}

\section{Clearing Pairs via Integer-Linear Programming }

\vspace*{-4pt}

\noindent
In this section, we inroduce two integer-linear programming problems to find 
the maximal and  the minimal clearing pairs in the models with crossholdings and default charges. 
To this end, we consider a~linear function $f: \{ 0, 1 \}^N \times [0, l] \times 
\mathbb{R}^N_+ \to \mathbb{R}$ with 
$$
f(a, p, V) = \sum\limits_{i=1}^N f_{1i} a^i + f_{2i} p^i + f_{3i} V^i
$$
where all coefficients  $ f_{1i}$,  $f_{2i}$, and $f_{3i}$ are strictly positive.~Put 

\vspace*{-4pt}

\noindent
\begin{align*}
x &= x(p, V) = e + \Pi' p + \Theta' V\,; \\
 y &= y(p, V) = \alpha e + \beta \Pi' p + \gamma \Theta' V\,.
\end{align*}

\vspace*{-6pt}

\subsection{The maximal clearing pair}

\noindent
Let $\kappa:=||K||_\infty$ where $K$ is the vector introduced in the remark after Lemma~2.1. 
 Then, $H(l) \leq \kappa {\bf 1} $ where the function~$H$ was defined in Section 2. 

Problem {\bf P1}: maximize $f$ under the constraints 
\begin{align}
 \label{max1}
 p &\le y(p,V) + a \circ l; \\
 \label{max2}
 a \circ l &\le x(p,V); \\
\label{max3}
 V &\le x(p,V) - a \circ  l; \\
\label{max4}
 V &\leq \kappa a.
\end{align}
Immediately, if $(p, V)$ is a~clearing pair, then $({\bf 1}_{\{p = l\}}, p, V)$ is admissible for~{\bf P1}.

\smallskip

\noindent
\textbf{Lemma 3.1}\
\textit{If $(\hat a, \hat p, \hat V)$ solves}~\textbf{P1}, \textit{then} $\hat a = {\bf 1}_{\{\hat p = l\}}$.


\smallskip
\noindent
P\,r\,o\,o\,f\,.\ \  Suppose that 
 $\hat p^j < l^j$ but $\hat a^j=1$. In this case, we have from~(\ref{max1}) 
 that   $p^j\le y^j(\hat p,\hat V) + l^j $. We can replace  
 $\hat p^j$ by a~larger value $\hat p^j+\varepsilon$ with $\varepsilon\in (0,l^j-\hat p^j)$ 
 without violating the constraints.  Since the cost function~$f$ is strictly increasing, we get a~contradiction with
 the optimality of  $(\hat a, \hat p, \hat V)$. Thus,  $\hat a^j = 0$. 

Suppose that  $p^j = l^j$ but $\hat a^j = 0$. Then, the component $\hat V^j = 0$ by virtue of~(\ref{max4})
 and  $l^j \leq y^j (\hat p,\hat V)\linebreak \leq x^j(\hat p, \hat V)$ due to~(\ref{max1}). 
 Put $\tilde a^j = 1$ and $\tilde a^i = \hat a^i$ for $i \neq j$.
  Then,  $(\tilde a, \hat p, \hat V)$ is admissible and one again obtains a~contradiction with the optimality. \hfill$\square$

\smallskip

\noindent
\textbf{Theorem 3.2}\
\textit{If $(\hat a, \hat p, \hat V)$ solves {\bf P1}, then $(\hat p, \hat V)$ is the maximal clearing pair.}

%\columnbreak

\smallskip

\noindent
P\,r\,o\,o\,f\,.\ \ 
Let $(\hat a, \hat p, \hat V)$ solve~\textbf{P1}. 
It suffices to verify that $(\hat p, \hat V)$ is a~clearing pair.
 Due to the above lemma, we already know that $\hat{a} = {\bf 1}_{\{\hat p = l\}}$. Let us check that the required rules are met. 

\smallskip

\noindent
\textbf{Limited liability}. If $\hat p^j < l^j$, 
then $\hat p^j \leq y^j(\hat  p,\hat  V)$ from~(\ref{max1}). 
Thus, $\hat p^j \leq x^j(\hat  p,\hat  V)$. If $\hat p^j = l^j$, then $\hat a^j = 1$ and  $l^j \leq  x^j(\hat  p,\hat  V)$ from~(\ref{max2}).

\smallskip

 \noindent
 \textbf{Absolute priority}. 
 If $\hat p^j = l^j$, then there is nothing to prove. If $\hat p^j < l^j$, then 
 $\hat p^j \leq y^j(\hat  p,\hat  V)$ by~(\ref{max1}). If the inequality were strict, one could add a~small $\varepsilon>0$ to $\hat p^j$ 
 and obtain again an admissible vector.

\smallskip

\noindent
\textbf{Equity evaluation}. If $\hat p^j < l^j$ and $\hat a^j = 0$, then $\hat V^j = 0$ 
because of~(\ref{max4}). If $\hat p^j = l^j$ and $\hat a^j = 1$, then  $l^j \leq x^j(\hat  p,\hat  V)$ and $\hat V^j 
\leq x^j(\hat  p,\hat  V) - l^j$ from~(\ref{max2}). In the case where the latter inequality is strict, one can add a~small $\varepsilon>0$ 
to~$\hat V^j$ and get an admissible vector. 
 \hfill$\square$

\subsection{The minimal clearing pair}

\noindent
Put  $\kappa_1 := || e + \Pi'l +\kappa \Theta' {\bf 1} ||_\infty $. Then $x(l,H(l))\leq \kappa_1 {\bf 1}$.
%$e + \Pi'l + \Theta' H(l)\linebreak \leq \kappa_1 {\bf 1}$.  

Problem~{\bf P2}: minimize $f$ under the constraints 
\begin{align}
 \label{min1}
 p &\geq y(p,V) - \kappa_1 a; \\
 \label{min2}
 p &\geq a \circ l; \\
\label{min3}
 ( {\bf 1} - a) \circ l  + \kappa_1 a &\geq y(p,V) ; \\
\label{min4}
 V& \geq x(p,V) - l - \kappa({\bf 1} - a).
\end{align}


 We will show, under mild assumptions, that the solution $(\check a, \check p, \check V)$ 
 of {\bf P2} is the minimal clearing pair. More precisely, it is the minimal solution of the system 
\begin{align}
    \label{min_eq1}
    p &= \Phi_{\bf 0} (p, V); \\
    \label{min_eq2}
    V &= \check G_+(p, V)
\end{align}
where 
$$
\check G_+ (p, V) = (x(p,V) - l)^+ \circ {\bf 1}_{\{y(p,V)> l\}}.
$$


Note that  for every solution $( p,  V)$ of the above system, $({\bf 1}_{\{y(p,V)> l\}}, p, V)$ is admissible for~\textbf{P2}.

\smallskip

\noindent
\textbf{Lemma 3.3}\
\textit{The vector} $\check p = ({\bf 1} - \check a) \circ x(\check p, \check V) +  \check a \circ l$.

\smallskip

\noindent
P\,r\,o\,o\,f\,.\ \ If $\check a^j = 0$, then by~(\ref{min1}) $\check p^j \geq y^j(\check p,\check V)$.
 If the inequality is strict, one can decrease $\check p^j$ by a~small $\varepsilon>0$. Hence, 
$\check p^j = y^j(\check p,\check  V)$. If
 $\check a^j = 1$, then by~(\ref{min2}), $\check p^j = l^j$.  \hfill$\square$

\smallskip

\noindent
\textbf{Lemma 3.4}\
\textit{The vector}  $\check V = \check a \circ (x(\check  p,\check  V)-l)^+$.

\smallskip

\noindent
P\,r\,o\,o\,f\,.\ \
  Let $\check a^j = 0$. Then $\check V^j = 0$ (otherwise,  $(\check a, \check p, \check V)$ cannot be the solution of~\textbf{P2}). 

Let $\check a^j = 1$ and $x^j(\check  p,\check  V) - l^j \geq 0$. From~(\ref{min4}), 
$\check V^j \geq x^j(\check  p,\check  V) - l^j$. Since~$f$ attains its minimum at $(\check a, \check p,\check  V)$, necessarily, 
$V^j = x^j(\check  p,\check  V) - l^j$. 

Let $a^j = 1$, but $x^j(\check  p,\check  V) - l^j < 0$. Then, $\check V^j = 0$ by the same reasoning. \hfill$\square$

\smallskip

\noindent
\textbf{Lemma 3.5}\
\textit{The vector $\check a = {\bf 1}_{\{y(\check  p,\check  V) > l \} }$. In particular, $\check p = y(\check  p,\check  V) \wedge l$}.

\pagebreak

\noindent
P\,r\,o\,o\,f\,.\ \ 
Let $y^j (\check  p,\check  V)\leq l^j$.  Assume for a~minute that $\check a^j = 1$. 
Immediately, $\check p^j = l^j$. Put $\tilde a^j = 0$ and $\tilde a^i = \check a^i$ for
 $i \neq j$, then $(\tilde a, \check p, \check V)$ is admissible. A~contradiction. \\
Let  $y^j(\check  p,\check  V) > l^j$. Then $\check a^j = 1$ by~(\ref{min1}). \hfill$\square$


\smallskip

\noindent
\textbf{Theorem 3.6}\
\textit{Let $(\check a, \check p, \check V)$ be the solution of}~\textbf{P2}. 
\textit{Then $(\check p, \check V)$ is the solution of the system}~(\ref{min_eq1})--(\ref{min_eq2}). 
\textit{In particular, $\check p \leq \underline p_{\bf 0}$ where $\underline p_{\bf 0}$ was defined in Corollary}~2.4. 
\textit{If $y^j(\check p, \check V) \neq l^j$ for every $j$, then} $\check p =\underline p_{\bf 0}$.

\smallskip

\noindent
P\,r\,o\,o\,f\,.\ \
The fact that $(\check p, \check V)$ solves~\eqref{min_eq1}--\eqref{min_eq2} follows from the above lemmata. Note that 
$(\underline p_{\bf 0}, H(\underline p_{\bf 0}))$ solves the system

\vspace*{-4pt}

\noindent
\begin{align*}
p &= y(p, V) \wedge l\,; \\[3pt]
V &= (x(p, V) - l)^+ \circ {\bf 1}_{\{x=l\}}
\end{align*}

\vspace*{-4pt}

\noindent
that  is equivalent to the following one:

\vspace*{-4pt}

\noindent
\begin{align}
\label{min_eq3}
p &= y(p, V) \wedge l\,; \\[3pt]
\label{min_eq4}
V &= (x(p, V) - l)^+ \circ {\bf 1}_{\{y(p, V) \geq l\}}.
\end{align}

\vspace*{-4pt}

\noindent
Put $\tilde G_+(p, V) := (x(p, V) - l)^+ \circ {\bf 1}_{\{y(p, V) \geq l\}}$. 
Obviously, $\check G_+(p, V) \leq \tilde  G_+(p, V)$ for every $(p, V)$. 
Due to the Knaster--Tarski theorem, one has $\check p \leq \underline p_0$.
Furthermore, if $y^j(\check p, \check V) \neq l^j$ for every~$j$, then 
$\check G_+(p, V) = \tilde  G_+(p, V)$. Under this condition the pair $(\check p, \check V)$ is the solution of~\eqref{min_eq3}--\eqref{min_eq4}
 that is why $\check p = \underline p_{\bf 0}$. \hfill$\square$




\section{Gaussian Method}

\noindent
In this section, we present a~Gaussian-type   algorithm to find the maximal clearing vector in a~model with crossholding and 
default charges assuming that  $\alpha=\beta=\gamma$ and $||\Theta||_\infty<1$, 
%$\Theta {\bf 1}<{\bf 1} $,
 i.\,e., $\sum\nolimits_j \theta^{ij} < 1$ for every~$i$. 


In such a~case,   the mappings $\Phi_{\bf 1}:[0, l]\times  \mathbb{R}^N_+\to [0,l]$ and $G_+:[0, l]\times  \mathbb{R}^N_+\to \mathbb{R}^N_+$ 
are given by the formulae: 
\begin{align*}
 \Phi_{\bf 1}(p, V)&: = \alpha \left(e +  \Pi' p +  \Theta' V\right) \circ {\bf 1}_{D} + l \circ {\bf 1}_{\bar D}; \\
G_+(p, V) &:= \left(e + \Pi' p + \Theta' V - l\right)^+ \circ {\bf 1}_{\bar D}
\end{align*}
where $D := \{ e + \Pi' p + \Theta' V <  l  \}$. 

Our aim is to find the maximal solution of the  system 

\vspace*{-4pt}

\noindent
\begin{align*}
%\label{eq-n1}
p&=\Phi_{\bf 1}(p, V) ; \\
%\label{eq2-n}
V&=G_+(p, V) 
\end{align*}

\vspace*{-4pt}

\noindent
existing by virtue of the Knaster--Tarski theorem. 

In the case where 
 $(e + \Pi' l + \Theta' H(l))^i \ge   l^i $ for all~$i$,    the maximal clearing pair  is 
$(l,H(l))$. Of course, to check the condition,  one needs to compute the value $H(l)$ which 
is the unique solution of the system  $v=(c(l)+\tilde \Theta v)^+$ with $c(l):=e+P'l-l$.  It is known\linebreak\vspace*{-12pt}

\columnbreak

\noindent
 that the latter system
can be solved in a~finite number of steps (e.\,g.,  by a~Gaussian-type algorithms)~\cite{kabanov2018clearing}. 


In the opposite case where at least one inequality fails to be true,  we  assume, without loss of generality, 
that  $e^1 + (\Pi' l)^1 + (\Theta' V(l))^1 < l^1$ and $V^1=0$. The first equation  becomes linear:  
\begin{equation*}
%\label{s3_eq_for_p1}
\alpha e^1 + \alpha (\Pi' p)^1 + \alpha (\Theta' V)^1 = p^1.
\end{equation*}
 The vector equation $p=\Phi_{\bf 1}(p, V)$ can be written as the system 
\begin{align*}
p^1&=\alpha e^1+ \alpha  \pi^{11}p^1+  \alpha  T' \tilde p +  \alpha  M' \tilde V;\\
%\label{tildep}
\tilde p&=\alpha\left(\tilde e+p^1R'+\tilde \Pi'\tilde p +\tilde \Theta'\tilde V\right)\circ \tilde 1_{D} + \tilde l \circ \tilde 1_{\bar D}
\end{align*}
where $\tilde \Pi := (\pi^{ij})_{i, j = 2}^N$, 
$R: = (\pi^{1j})_{j=2}^N$, $T: = (\pi^{i1})_{i=2}^N$, 
$\tilde \Theta: = (\theta^{ij})_{i, j = 2}^N$, $M := (\theta^{i1})_{i=2}^N$,  
and the tilde indicates  vectors with the removed first component. 

If $\alpha \pi^{11}\neq 1$, then 
\begin{equation*}
%\label{p1}
p^1 = \alpha(1 - \alpha \pi^{11})^{-1} \left(e^1 +   T' \tilde p +  M' \tilde V\right).
\end{equation*}
Substituting this expression, we get that  
$$
\tilde e+p^1R'+\tilde \Pi'\tilde p +\tilde \Theta'\tilde V=e_1 +  \Pi'_1 \tilde p + \Theta'_1 \tilde V
$$
where 

\vspace*{-4pt}

\noindent
\begin{align*}
\Pi_1 &:= \tilde \Pi + \alpha\left(1 - \alpha \pi^{11}\right)^{-1}  TR; \\
\Theta_1 &:= \tilde \Theta + \alpha\left(1 - \alpha \pi^{11}\right)^{-1}  MR; \\ 
e_1 &:= \tilde e + \alpha\left(1 - \alpha \pi^{11}\right)^{-1}  e^1 R'.
\end{align*}

\vspace*{-4pt}

\noindent
It follows that $(\tilde p,\tilde V)$ is the solution of the system  
\begin{align*}
\tilde p&=\alpha\left(e_1 +  \Pi'_1 \tilde p + \Theta'_1 \tilde V\right) \circ \tilde 1_{D} + \tilde l \circ \tilde 1_{\bar D}; \\
\tilde V&= \left(e_1 + \Pi'_1 \tilde p + \Theta'_1 \tilde V - \tilde l\right)^+ \circ \tilde 1_{\bar D}.  
\end{align*}

Note that $TR \tilde {\bf 1} = T (R \tilde {\bf 1}) = ((\Pi{\bf 1})^1 - \pi^{11}) T$  and, therefore,   
\begin{multline*}
\Pi_1\tilde {\bf 1}=\tilde \Pi \tilde {\bf 1} + 
\left(\alpha \left(\Pi\tilde {\bf 1}\right)^1 - \alpha \pi^{11}\right) 
\left(1 -\alpha \pi^{11}\right)^{-1} T \\
{}\le\tilde \Pi \tilde {\bf 1} + T =\widetilde {\Pi{\bf 1}}\le \tilde {\bf 1}. 
\end{multline*}
Thus, the matrix~$\Pi_1$ is substochastic. By the same arguments, we get that $||\Theta_1||_\infty<1$.
Moreover,  $ \tilde {\bf 1}_{D}={\bf 1}_{D_1}$ where 
$D_1 := \{ e_1 + \Pi'_1\tilde  p + \Theta'_1 \tilde V <  l  \}$. 
 

If $\alpha=1$ and $\pi^{11}=1$, then we can take $p^1=l^1$ and continue with the system:  
\begin{align*}
\tilde p&=\left(e_1 +  \Pi'_1 \tilde p + \Theta'_1 \tilde V\right) \circ \tilde 1_{D} + \tilde l \circ \tilde 1_{\bar D}; \\
\tilde V&= \left(e_1 + \Pi'_1 \tilde p + \Theta'_1 \tilde V - \tilde l\right)^+ \circ \tilde 1_{\bar D}  
\end{align*}
where $e_1:=\tilde e+e^1R'$; $ \Pi_1:=\tilde P$; and $\Theta_1:=\tilde \Theta$.

\pagebreak

In both cases,  we have reduced the $2N$-dimensional problem to the $2(N-1)$-dimensional 
one of the same type. 

\vspace*{-12pt}


\Ack

\vspace*{-4pt}

\noindent
The research is funded by the grant of the Russian Science Foundation 20-68-47030 ``Econometric and probabilistic methods for the analysis of financial markets with complex structure.''
 


\renewcommand{\bibname}{\protect\rmfamily References}

\vspace*{-6pt}


{\small\frenchspacing
{%\baselineskip=10.8pt
\begin{thebibliography}{9}



\bibitem{eisenberg2001systemic} %1
\Aue{Eisenberg, L., and T.\,H.~Noe.} 2001. Systemic risk in financial systems.  \textit{Manage. Sci.} 47(2):236--249.

\bibitem{suzuki2002valuing} %2
\Aue{Suzuki, T.} 2002. Valuing corporate debt: The effect of cross-holdings of stock and debt. 
\textit{J.~Oper. Res. Soc. Jpn.} 45(2):123--144.

\bibitem{elsinger2009financial} %3
\Aue{Elsinger, H.} 2011. \textit{Financial networks, cross holdings, and limited liability}. 
Wien: Oesterreichische Nationalbank. Working Paper 156. 37~p.

\bibitem{RV2013} %4
\Aue{Rogers, L.\,C.\,G., and  L.\,A.\,M.~Veraart.} 2013. Failure and rescue in an interbank network. 
\textit{Manage. Sci.} 59(4):882--898.

\bibitem{weber2017joint} %8
\Aue{Weber, S., and  K.~Weske.}
2017. The joint impact of bankruptcy costs, fire sales and cross-holdings on systemic risk in financial networks.
\textit{Probability Uncertainty Quantitative Risk} 2(1):1--38.

\bibitem{kabanov2018clearing} %5
\Aue{Kabanov, Yu.\,M., R.~ Mokbel, and Kh.~El Bitar.} 2017. Clearing in financial networks. \textit{Theor. Probab. Appl.} 62(2):252--277.

\bibitem{feinstein2019dynamic} %6
\Aue{Feinstein, Z., and A.~Sojmark.} 2019. A~dynamic default contagion model: From Eisenberg--Noe to the mean field.  \textit{arXiv.org}. 45~p. 
doi: 10.48550/arXiv.1912.08695.
  


\bibitem{djete2021mean} %7
\Aue{Djete, M., and N.~Touzi.} 2021. Mean field game of mutual holding. \textit{arXiv.org}. 38~p. 
doi: 10.48550/arXiv.2104.03884.




\bibitem{Ararat2019}
\Aue{Ararat, C., and N.~Meimanjanov.} 2019. Computation of systemic risk measures: 
A~mixed-integer linear programming approach. \textit{arXiv.org}. 65~p. doi: 10.48550/arXiv.1903.08367.
\end{thebibliography} } }

\end{multicols}

\vspace*{-6pt}

\hfill{\small\textit{Received August 1, 2022}}

\vspace*{-18pt}



\Contr

%\vspace*{-4pt}

\noindent
\textbf{Kabanov Yuri M.} (b.\ 1948)~--- 
Doctor of Science in physics and mathematics, professor, M.\,V.~Lomonosov Moscow State University, 1-52~Leninskie Gory, 
GSP-1, Moscow 119991, Russian Federation; leading scientist, Institute of Informatics Problems, Federal Research Center 
``Computer Science and Control'' of the Russian Academy of Sciences, 44-2~Vavilov Str., Moscow 119333, Russian Federation; 
\mbox{youri.kabanov@univ-fcomte.fr}

\vspace*{3pt}

\noindent
\textbf{Sidorenko Artur P.} (b.\ 1998)~--- PhD student, Faculty of Mechanics and Mathematics, M.\,V.~Lomonosov 
Moscow State University, 1-52 Leninskie Gory, GSP-1, Moscow 119991, Russian Federation;  \mbox{Artur.Sidorenko@student.msu.ru}

\vspace*{6pt}

\hrule

\vspace*{2pt}

\hrule

\vspace*{4pt}

%\newpage

%\vspace*{-24pt}



\def\tit{АКСИОМАТИЧЕСКИЙ ВЗГЛЯД НА МОДЕЛИ СИСТЕМНОГО РИСКА РОДЖЕРСА--ВЕРААРТ 
И~СУДЗУКИ--ЭЛЬСИНГЕРА$^*$\\[-5pt]}

\def\aut{Ю.\,М.~Кабанов$^{1}$, А.\,П. Сидоренко$^2$}


\def\titkol{Аксиоматический взгляд на модели системного риска Роджерса--Вераарт 
и~Судзуки--Эльсингера}

\def\autkol{Ю.\,М.~Кабанов, А.\,П.~Сидоренко}

{\renewcommand{\thefootnote}{\fnsymbol{footnote}}
\footnotetext[1]{Работа выполнена при поддержке Российского научного фонда, проект 
20-68-47030 <<Эконометрические и~вероятностные методы для анализа финансовых рынков сложной структуры>>.}}


\titel{\tit}{\aut}{\autkol}{\titkol}

\vspace*{-8pt}

\noindent
$^1$Московский государственный университет имени М.\,В.~Ломоносова;
Федеральный исследовательский\linebreak
$\hphantom{^1}$центр <<Информатика и~управ\-ле\-ние>> Российской академии наук

\noindent
$^2$Московский государственный университет имени М.\,В.~Ломоносова

\vspace*{3pt}

\def\leftfootline{\small{\textbf{\thepage}
\hfill ИНФОРМАТИКА И ЕЁ ПРИМЕНЕНИЯ\ \ \ том\ 17\ \ \ выпуск\ 1\ \ \ 2023}
}%
 \def\rightfootline{\small{ИНФОРМАТИКА И ЕЁ ПРИМЕНЕНИЯ\ \ \ том\ 17\ \ \ выпуск\ 1\ \ \ 2023
\hfill \textbf{\thepage}}}


\Abst{Изучается модель клиринга  межбанковской сети с~кросс-хол\-дин\-га\-ми и~стоимостью дефолта. 
Следуя подходу Айзенберга и~Ноэ, авторы формулируют  модель с~по\-мощью 
естественных правил финансового регулирования, включающих описание выплат в~случае дефолтов.  
Эти правила определяют конечное семейство задач о~неподвижных точках, параметризованных векторами бинарных  переменных.  
Представленная модель обобщает известные  модели Ара\-ра\-та--Мей\-ман\-джа\-но\-ва, Род\-жер\-са--Ве\-ра\-арт и~Суд\-зу\-ки--Эль\-син\-ге\-ра. 
Предложены  методы вычисления максимальной и минимальной клиринговых пар с~использованием комбинации целочисленного и~линейного программирования, 
а~также обсуждается  алгоритм  последовательного исключения переменных.}

\KW{системный риск; финансовые сети; клиринг; кросс-хол\-динг; стоимость дефолта}

\DOI{10.14357/19922264230102} 

%\pagebreak

\vspace*{-12pt}


 \begin{multicols}{2}

\renewcommand{\bibname}{\protect\rmfamily Литература}
%\renewcommand{\bibname}{\large\protect\rm References}

{\small\frenchspacing
{%\baselineskip=10.8pt
\begin{thebibliography}{9}


\vspace*{-2pt}

\bibitem{eisenberg2001systemic-1} %1
\Au{Eisenberg L., Noe~T.\,H.} Systemic risk in financial systems~// Manage. Sci., 2001.  Vol.~47. Iss.~2. P.~236--249.

\bibitem{suzuki2002valuing-1} %2
\Au{Suzuki T.} Valuing corporate debt: The effect of cross-holdings of stock and debt~// J.~Oper. Res. Soc. Jpn., 2002. Vol.~45. Iss.~2. P.~123--144.

\bibitem{elsinger2009financial-1} %3
\Au{Elsinger H.} Financial networks, cross holdings, and limited liability.~---  Wien: Oesterreichische Nationalbank, 2011. 
 Working Paper~156. 37~p.
 
 \bibitem{RV2013-1} %4
\Au{Rogers L.\,C.\,G.,  Veraart L.\,A.\,M.} Failure and rescue in an interbank network~// Manage. Sci., 2013. Vol.~59. Iss.~4. P.~882--898.

\bibitem{weber2017joint-1}
\Au{Weber S., Weske~K.} The joint impact of bankruptcy costs, fire sales and cross-holdings on systemic risk in financial networks~// 
Probability Uncertainty Quantitative Risk, 2017.  Vol.~2. Iss.~1. P.~1--38.


\bibitem{kabanov2018clearing-1} %5
\Au{Кабанов Ю.\,М., Мокбель~Р., Эль Битар~Х.} Взаимозачет в финансовых сетях~// Теория вероятностей и ее применения, 2017. Т.~62. №\,2. С.~311--344.

\bibitem{feinstein2019dynamic-1} %6
\Au{Feinstein Z., Sojmark~A.} A~dynamic default contagion model: From Eisenberg--Noe to the mean field~// arXiv, 2019. 45~p. doi: 10.48550/arXiv.1912.08695.

\bibitem{djete2021mean-1} %7
\Au{Djete M., Touzi~N.} Mean field game of mutual holding~// arXiv, 2021. 38~p. doi: 10.48550/arXiv.2104.03884.






\bibitem{Ararat2019-1}
\Au{Ararat C., Meimanjanov~N.} Computation of systemic risk measures: A~mixed-integer linear programming approach~// 
arXiv, 2019. 65~p. doi: 10.48550/arXiv.1903.08367.


\end{thebibliography}
}}
\end{multicols}

 \label{end\stat}

 \vspace*{-6pt}

\hfill{\small\textit{Поступила в редакцию  01.08.2022}}
%\renewcommand{\bibname}{\protect\rm Литература}
\renewcommand{\figurename}{\protect\bf Рис.}
\renewcommand{\tablename}{\protect\bf Таблица}