\def\stat{leri}

\def\tit{СРЕДНЕЕ РАССТОЯНИЕ В~КОНФИГУРАЦИОННЫХ ГРАФАХ СО~СТЕПЕННЫМ РАСПРЕДЕЛЕНИЕМ$^*$}

\def\titkol{Среднее расстояние в~конфигурационных графах со~степенным распределением}

\def\aut{М.\,М.~Лери$^1$}

\def\autkol{М.\,М.~Лери}

\titel{\tit}{\aut}{\autkol}{\titkol}

\index{Лери М.\,М.}
\index{Leri M.\,M.}


{\renewcommand{\thefootnote}{\fnsymbol{footnote}} \footnotetext[1]
{Финансовое обеспечение исследований осуществлялось из средств федерального
бюджета на выполнение государственного задания Карельского научного центра Российской академии наук
(Институт прикладных математических исследований КарНЦ РАН).}}


\renewcommand{\thefootnote}{\arabic{footnote}}
\footnotetext[1]{Институт прикладных математических исследований Карельского научного центра
Российской академии наук, \mbox{leri@krc.karelia.ru}}

%\vspace*{-2pt}








\Abst{В случайных конфигурационных графах с~дискретным степенным распределением степеней вершин
с фиксированным параметром рассматривается среднее расстояние в~графе, которое вычисляется
как среднее арифметическое расстояний между всеми парами вершин графа.
Эта характеристика оценивается с~по\-мощью методов имитационного моделирования. В~силу вычислительных
ограничений рассматриваются графы в~доасимптотической области (в~настоящей работе это графы объемом
до 7000~вершин). По\-стро\-ены модели зависимостей сред\-не\-го рас\-сто\-яния от объема графа и~па\-ра\-мет\-ра распределения степеней вершин.
Проведено сравнение полученных результатов с~результатами тео\-ре\-ти\-че\-ских исследований типичного расстояния
в графе в~асимп\-то\-ти\-ке (т.\,е.\ когда число вершин графа стремится к~бес\-ко\-неч\-ности), приведенными в~работах
Р.~Хофстада.}

\KW{конфигурационные графы; степенное распределение;
сред\-нее рас\-сто\-яние в~графе; имитационное моделирование}

\DOI{10.14357/19922264230104} 
  
\vspace*{-6pt}


\vskip 10pt plus 9pt minus 6pt

\thispagestyle{headings}

\begin{multicols}{2}

\label{st\stat}

\section{Введение}

\vspace*{-1pt}

Изучение структуры и~функционирования сложных сетей продолжает оставаться одним из важных
направлений исследований в~науке и~технике~\cite{Dur,Hof1}. Примерами таких сетей, окружающих
нас в~повседневной жизни, служат интернет, электрические и~телекоммуникационные сети, сети
социальных отношений, соавторства и~цитирования и~др.
Их быст\-рое и~динамичное развитие и~нарастающая популярность легли в~основу многих фундаментальных
исследований в~об\-ласти топологии таких сетей (см., например,~\cite{Dur,Hof1,Hof2,New1,New2}).
В~качестве моделей слож\-ных сетей широко используются случайные графы, причем их разнообразие
касается как определения степеней вершин графа, так и~уста\-нов\-ле\-ния связей между этими вершинами.
В~частности, было показано (см., например,~\cite{Fa,RN}), что модели случайных графов с~независимыми
одинаково распределенными степенями вершин с~общим дискретным законом распределения подходят для
моделирования сети Интернет в~случае, когда в~качестве узлов сети рассматриваются автономные системы.

Увеличение размеров сетей и~изменчивость сетевой структуры дают понять, что для адекватного отражения
их топологии и~функционирования в~ходе по\-стро\-ения их математических моделей необходимо учитывать не
только распределение степеней вершин в~со\-от\-вет\-ст\-ву\-ющей модели случайного графа, но также
принимать во внимание и~другие не менее важ\-ные характеристики исследуемых сетей~\cite{Hof1, New1}.
В связи с~этим различные структурные характеристики слож\-ных сетей пред\-став\-ля\-ют определенный интерес
как при моделировании их топологии, так и~при изучении динамических процессов, происходящих в~таких сетях
по мере их рос\-та или под внешними воздействиями. Одна из таких характеристик~---
рас\-сто\-яние между двумя произвольными вершинами графа~\cite{Dur, Hof2, New1}.

 Мож\-но рас\-смат\-ри\-вать различные
виды расстояний в~графе: расстояние между двумя заданными вершинами, рас\-сто\-яние между произвольно\linebreak
выбранными вершинами, все расстояния между каж\-дой парой вершин, наименьшее воз\-мож\-ное {рас\-сто\-яние},
наибольшее, или диаметр графа, и~т.\,д. 

В~на\-сто\-ящей работе рассматривается среднее расстояние в~графе,
которое вы\-чис\-ля\-ет\-ся как среднее арифметическое расстояний между всеми парами вершин графа.
Цель работы со\-сто\-яла в~на\-хож\-де\-нии зависимостей среднего расстояния в~графе от числа его вершин и~па\-ра\-мет\-ра 
распределения степеней вершин, а~также в~сравнении результатов работы с~результатами тео\-ре\-ти\-че\-ских
исследований рас\-сто\-яния в~графе, приведенными в~\cite{Hof2}.
Исследование проводилось по\-средст\-вом методов имитационного моделирования с~по\-сле\-ду\-ющей статистической
обработкой данных с~помощью программного обеспечения Statistica 10 и~Wolfram Mathematica 9.0.

\section{Описание модели}

Рассматривается случайный граф, со\-сто\-ящий из~$N$~вершин. Через $\xi_1,\xi_2,\ldots,\xi_N$ обозначим
степени вершин, которые являются независимыми одинаково распределенными случайными величинами
со сле\-ду\-ющим дискретным степенным распределением~\cite{RN}:
\begin{equation}
\label{eq1}
{\bf P}\{\xi = k\} = k^{-\tau} - (k+1)^{-\tau}, \quad k=1,2,\ldots,
\end{equation}
с фиксированным параметром~$\tau\hm>1$. Легко показать, что математическое ожидание
распределения~(\ref{eq1}) рав\-но ${\bf E}\xi\hm=\zeta(\tau)$, где $\zeta(\tau)$~--- значение дзе\-та-функ\-ции
Римана в~точке~$\tau$. Что касается дисперсии, то при $\tau\hm>2$ она конечна, а при
$\tau\hm\in(1,2]$~-- бесконечна.
В~работе рассматриваются конфигурационные графы~\cite{Bol}, построение которых происходит сле\-ду\-ющим
образом.
Для каждой из $N$ вершин графа задаются степени в~соответствии  с~распределением~(\ref{eq1}) с~выбранным
значением па\-ра\-мет\-ра~$\tau$. Степени определяют чис\-ло различимых полуребер~\cite{RN} (под полуребром понимают
ребро, инцидентное данной вершине графа, для которого смежная вершина еще не определена), занумерованных в~произвольном порядке. 
Для формирования ребер все полуребра попарно и~равновероятно соединяют между собой.
Сумма степеней вершин рас\-смат\-ри\-ва\-емо\-го графа является случайной величиной. Очевидно, что она должна быть чет\-ной,
поэтому, если это не так, для построения недостающего ребра к~равновероятно выбранной
вершине добавляется одно полуребро, увеличивая степень этой вершины на~1. Граф, по\-стро\-ен\-ный таким
образом, может иметь пет\-ли, цик\-лы и~кратные \mbox{ребра}.

Известно (см., например,~\cite{Dur,Hof1,RN}), что степенной конфигурационный граф, значение па\-ра\-мет\-ра
распределения степеней вершин которого $\tau\hm>1$, асимптотически почти наверное содержит больше одной
компоненты связности, причем при $\tau\hm\in(1,2)$ в~таком графе существует, и~она единственна, так
называемая гигантская компонента связ\-ности, чис\-ло вершин в~которой пропорционально~$N$
при $N\hm\rightarrow\infty$, а~объем любой другой компоненты такого графа бесконечно мал по
сравнению с~объемом гигантской компоненты.


\vspace*{-6pt}

\section{Среднее расстояние в~графе}

\vspace*{-2pt}

Расстояния между узлами сложной сети служат важными числовыми характеристиками сетевой
топологии (см., например,~\cite{Hof2, Chu}).
Пусть $G \hm= (V,E)$~--- неориентированный граф, в~котором~$V$~--- множество вершин, а~$E$~--- множество ребер.
Обозначим через~$l(v,u)$ чис\-ло ребер простой цепи, со\-еди\-ня\-ющей вершины~$v$ и~$u$ графа~$G$
($v,u\hm\in V$ и~$v\hm\neq u$). Если вершины~$v$ и~$u$ принадлежат разным компонентам связ\-ности, то $l(v,u)$
полагают равным~$\infty$. Длину цепи от вершины~$v$ до вершины~$u$ наименьшей длины называют расстоянием
между этими вершинами:
$$
d(v,u)=\min\limits_{l(v,u)\neq\infty}l(v,u).
$$

Пусть $k$~--- чис\-ло рас\-сто\-яний $d(v,u)\hm\neq\infty$ между всеми парами вершин~$v$ и~$u$ ($v\hm\neq u$).
Среднее расстояние вы\-чис\-ля\-ет\-ся как среднее арифметическое всех рас\-сто\-яний $d(v,u)$ графа~$G$:
\begin{equation*}
\mathrm{dist} = \mathrm{dist}\,(G) = \fr{\sum\nolimits_{v,u\in V(v\neq u)}d(v,u)}{k}.
\end{equation*}

Из теорем~7.2 и~7.1 в~\cite{Hof2} следует, что при $N\hm\rightarrow\infty$ 
\begin{equation}
\label{eq2}
d(v,u)\sim\fr{2\ln\ln N}{|\ln(\tau-1)|}\,,
\end{equation}
если $1\hm<\tau\hm<2$,
и
\begin{equation}
\label{eq3}
d(v,u)\sim\fr{\ln N}{\ln\nu}\,,
\end{equation}
если $\tau>2$, 
где $\nu={\bf E}\xi(\xi-1)/({\bf E}\xi)$ и~$\nu\hm>1$.
Выражения~(\ref{eq2}) и~(\ref{eq3}) носят асимптотический характер и~получены для <<типичных рас\-сто\-яний>>
(где под типичным рас\-сто\-яни\-ем понимается математическое ожидание рас\-сто\-яния)~\cite{Hof2} в~конфигурационных
графах с~бесконечной и~с~конечной дис\-пер\-си\-ями соответственно.
Легко показать, что для конфигурационных графов с~распределением~(\ref{eq1})
$$
\nu=\fr{2\zeta(\tau-1)}{\zeta(\tau)}-2\,,
$$
причем $\nu>1$ при $2\hm<\tau\hm\leq 2,8106\ldots$

Для нахождения зависимости среднего расстояния от числа вершин графа~$N$ и~па\-ра\-мет\-ра распределения
степеней вершин~$\tau$ были получены оценки средних расстояний в~конфигурационных графах различных
размерностей с~разными па\-ра\-мет\-ра\-ми распределения степеней вершин. По полученным результатам были
построены зависимости $\mathrm{dist}$ от чис\-ла вершин графа~$N$ при конкретных значениях па\-ра\-мет\-ра распределения
степеней вершин~$\tau$, а~также зависимости $\mathrm{dist}$ от~$N$ и~$\tau$ на интервалах $\tau\hm\in(1,2)$ и~$\tau\hm\in(2,\,2,8]$.
Рассматривались конфигурационные графы сле\-ду\-ющих размерностей:
$10\hm\leq N\hm\leq 100$ с~шагом~$10$, $100\hm\leq N\hm\leq 1000$ с~шагом~$50$, $1000\hm\leq N\hm\leq 7000$ с~шагом~$500$.
Значения параметра~$\tau$ изменялись с~шагом $0{,}1$ в~двух интервалах: $1{,}1\hm\leq\tau\hm<2$ и~$2\hm<\tau\hm\leq 2{,}8$,
а~так\-же были взяты два дополнительных значения: $\tau\hm=1{,}99$ и~$2{,}01$. Для
каждой пары значений $(N,\tau)$
генерировалось\linebreak\vspace*{-12pt}

\pagebreak

\noindent
  по~$100$~графов, т.\,е.\ $40\,000$ и~$36\,000$ графов на интервалах $1{,}1\hm\leq\tau\hm\leq 1{,}99$
и~$2{,}01\hm\leq\tau\hm\leq 2{,}8$ соответственно.
Расстояния в~графе находились с~применением алгоритма Дейкстры~\cite{Dijk}.

\subsection{Результаты для интервала $\tau\in(1,2)$}

Сначала для рассматриваемых в~настоящей работе графов были по\-стро\-ены зависимости среднего рас\-сто\-яния $\mathrm{dist}$ от
чис\-ла вершин $N$ при фиксированных значениях па\-ра\-мет\-ра~$\tau$, т.\,е.\ для каждого из рас\-смот\-рен\-ных значений
$1{,}1\hm\leq\tau\hm\leq 1{,}99$ были получены регрессионные зависимости вида
\begin{equation}
\label{eq4}
\mathrm{dist} = a \ln\ln N + b\,.
\end{equation}
Здесь и~далее коэффициенты всех регрессионных уравнений находили по\-средст\-вом метода наименьших
квадратов, зна\-чи\-мость коэффициентов проверяли с~помощью критерия Стьюдента. Для оценки степени подгонки
регрессионной модели к~данным вы\-чис\-ля\-ли коэффициент детерминации этой модели и~с~по\-мощью критерия Фишера
проверяли гипотезу $H_0: R^2\hm=0$. Проверка всех статистических гипотез осуществлялась на 5\%-ном уровне
зна\-чи\-мости.

Обозначим $a_f={2}/{|\ln(\tau-1)|}$ и~сравним эти значения с~коэффициентами~$a$ уравнений~(\ref{eq4})
для каждого~$\tau$.
В табл.~1 приведены значения~$a_f$, коэффициенты~$a$ и~$b$ регрессионных урав\-не\-ний вида~(\ref{eq4})
и~соответствующие коэффициенты детерминации~$R^2$ этих уравнений. Все коэффициенты~$a$ и~$b$ в~табл.~1
значимы, а~гипотезы $H_0: R^2\hm=0$ отвергаются для всех уравнений.




Таким образом, при фиксированных~$\tau$ сред\-нее\linebreak рас\-сто\-яние в~графе с~рос\-том чис\-ла его вершин\linebreak
рас\-тет как $\ln\ln N$, так же как и~расстояние в~асимп\-то\-ти\-ке~(\ref{eq2}). Из табл.~1 видно,
что значения коэффициента~$a$ ниже значений~$a_f$ на всем
интервале\linebreak  изменения~$\tau$, однако эта разница не остается неизменной,
а~возрастает с~рос\-том~$\tau$.
Более того, сам угловой коэффициент~$a$ возрастает с~рос\-том
па\-ра-\linebreak\vspace*{-12pt}

%\begin{table*}\small %tabl1
\begin{center}

\vspace*{6pt}

\noindent
\parbox{64mm}{{{\tablename~1}\ \ \small{Значения $a_f$, коэффициенты $a$ и~$b$ зависимостей вида~(\ref{eq4})
и коэффициенты детерминации $R^2$ этих уравнений
}}}


\vspace*{6pt}


{\small 
\begin{tabular}{|c|c|c|c|c|}
\hline
&&&&\\[-10pt]
$\tau$ & $a_f$ & $a$ & $b$ & $R^2$ \\ 
\hline
1,1 & 0,869 &   0,745 & \hphantom{$-$}1,702 & 0,88 \\
1,2 & 1,243 &   0,975 & \hphantom{$-$}1,513 & 0,91 \\
1,3 & 1,661 &   1,255 & \hphantom{$-$}1,244 & 0,94 \\
1,4 & 2,183 &   1,596 & \hphantom{$-$}0,869 & 0,96 \\
1,5 & 2,885 &   1,959 & \hphantom{$-$}0,494 & 0,96 \\
1,6 & 3,915 &   2,421 & $-$0,065 & 0,98 \\
1,7 & 5,607 &   3,048 & $-$0,909 & 0,98 \\
1,8 & 8,963 &   3,668 & $-$1,710 & 0,98 \\
1,9 & 18,982\hphantom{9} & 4,417 & $-$2,806 & 0,98 \\
\hphantom{9}1,99 & 198,998\hphantom{99} & 5,262 & $-$4,045 & 0,96\\
\hline
\end{tabular}
}
\end{center}
%\end{table*}

{ \begin{center}  %fig1
 \vspace*{-2pt}
    \mbox{%
\epsfxsize=78.504mm
\epsfbox{ler-1.eps}
}

\end{center}



\noindent
{{\figurename~1}\ \ \small{График экспериментальных
значений $\mathrm{dist}$ от $N$ и~$1{,}1\hm\leq\tau\hm<2$
}}}

\vspace*{8pt}

\addtocounter{figure}{1}
\addtocounter{table}{1}



\noindent
мет\-ра~$\tau$,
 и~это  на\-гляд\-но видно
на рис.~1, где показана за\-ви\-си\-мость
экспериментальных значений $\mathrm{dist}$ от~$N$ и~$\tau$.


Далее задача состояла в~том, чтобы найти за\-ви\-си\-мость сред\-не\-го рас\-сто\-яния $\mathrm{dist}$ от обеих переменных:
$N$ и~$\tau$.
Сначала была построена за\-ви\-си\-мость в~виде $\mathrm{dist}\hm={2\ln\ln N}/({|\ln(\tau-1)|})\hm+b$. Получено
сле\-ду\-ющее регрессионное уравнение:
\begin{equation*}
\mathrm{dist} = \fr{2\ln\ln N}{|\ln(\tau-1)|}-39{,}604\,.
\end{equation*}
К сожалению, коэффициент детерминации полученной за\-ви\-си\-мости оказался очень низ\-ким
($R^2\hm=0{,}01$). Гипотеза о~равенстве~$R^2$ нулю не отвергается, коэффициент~$b$
статистически не значим, поэтому такую модель использовать для прогноза не имеет смысла.



\begin{figure*} %fig2
\vspace*{1pt}
\begin{minipage}[t]{80mm}
\begin{center}
   \mbox{%
\epsfxsize=79mm
\epsfbox{ler-2-a.eps}
}
\end{center}
\vspace*{-9pt}
\Caption{Регрессионная зависимость~(\ref{eq6}) среднего расстояния $\mathrm{dist}$ от $N$ при фиксированных
значениях $1{,}1\hm\leq\tau\hm<2$: \textit{1}~--- $\tau\hm=1{,}1$;
\textit{2}~--- 1,3; \textit{3}~--- 1,5; \textit{4}~--- 1,7; \textit{5}~--- $\tau\hm= 1{,}99$
}
\end{minipage}
%\end{figure*}
\hfill
%\begin{figure*} %fig3
\vspace*{1pt}
\begin{minipage}[t]{80mm}
\begin{center}
   \mbox{%
\epsfxsize=77.81mm
\epsfbox{ler-2-b.eps}
}
\end{center}
\vspace*{-9pt}
\Caption{Регрессионная зависимость~(\ref{eq6}) среднего расстояния $\mathrm{dist}$
от~$\tau$ при фиксированных значениях $10\hm\leq N\hm\leq 7000$:) \textit{1}~--- $N\hm=10$; \textit{2}~--- 100; \textit{3}~--- 1000;
\textit{4}~--- 5000; \textit{5}~--- $N\hm=7000$}
\end{minipage}
\vspace*{-4pt}
\end{figure*}



Далее была построена регрессия вида $\mathrm{dist}\hm={a\ln\ln N}/({|\ln(\tau-1)|})\hm+b$ и~получена
за\-ви\-си\-мость
\begin{equation}
\label{eq5}
\mathrm{dist} = \fr{0{,}0104\ln\ln N}{|\ln(\tau-1)|}+3{,}922
\end{equation}
с~коэффициентом детерминации $R^2\hm=0{,}21$. В~данном случае гипотеза $H_0: R^2\hm=0$ отвергается,
а~что касается коэффициентов~$a$ и~$b$ регрессионного уравнения~(\ref{eq5}), то коэффициент~$a$
оказался статистически значим, а~коэффициент~$b$ нет.
Поиск наилучшей регрессионной за\-ви\-си\-мости был продолжен и~привел к~получению сле\-ду\-ющей
модели за\-ви\-си\-мости сред\-не\-го рас\-сто\-яния конфигурационного графа $\mathrm{dist}$ от объема графа~$N$ 
и~па\-ра\-мет\-ра распределения степеней вершин~$\tau$:
\begin{equation}\label{eq6}
\mathrm{dist} = \fr{2(4{,}488-3{,}077\tau+0{,}417\tau^2)\ln\ln N}{|\ln(\tau-1)|}
\end{equation}
с коэффициентом детерминации $R^2\hm=0{,}88$ и~значимыми коэффициентами регрессии. Графически
за\-ви\-си\-мость~(\ref{eq6}) пред\-став\-ле\-на на рис.~2 и~3.

Оценка значимости различия между коэффициентами множественной корреляции $r\hm=\sqrt{R^2}$ регрессионных моделей~(\ref{eq5}) и~(\ref{eq6})
 на уровне зна\-чи\-мости~0,05 показала, что нулевая гипотеза $H_0:\linebreak r_{(5)}\hm=r_{(6)}$
($r_{(5)}$ и~$r_{(6)}$~--- коэффициенты множественной корреляции зависимостей~(\ref{eq5}) и~(\ref{eq6})
соответственно) отвергается; следовательно, различие между коэффициентами корреляции значимо.
Остатки обеих моделей распределены нормально, но сравнение сумм квад\-ра\-тов
остатков $\mathrm{SSR}_{(5)}\hm=62951{,}1$ и~$\mathrm{SSR}_{(6)}\hm=24786{,}9$ показывает, что $\mathrm{SSR}_{(5)}\hm>\mathrm{SSR}_{(6)}$, т.\,е.\ модель~(\ref{eq6}) 
<<лучше>> в~смыс\-ле описания изуча\-емо\-го явления и~для прогнозирования.
Таким образом, в~качестве наиболее под\-хо\-дя\-щей модели за\-ви\-си\-мости $\mathrm{dist}$ от~$N$ и~$\tau$ для до\-асимп\-то\-ти\-че\-ской
об\-ласти предлагается за\-ви\-си\-мость, описываемая уравнением~(\ref{eq6}).



На рис.~2 и~3 линии внут\-ри затененных областей соответствуют зависимостям $\mathrm{dist}$ от~$N$ (см.\ рис.~2)
и~от~$\tau$ (см.\ рис.~3) при некоторых (отраженных в~легендах) значениях па\-ра\-мет\-ра~$\tau$
или объема графа $N$ соответственно. Заметим, что кривые зависимостей $\mathrm{dist}$ от~$N$ на рис.~2
расположены одна над другой по мере роста значения параметра $1,1\leq\tau<2$ в~пределах его граничных значений.
Аналогично кривые зависимостей $\mathrm{dist}$ от~$\tau$ на рис.~3 также расположены друг над другом по
мере воз\-рас\-та\-ния чис\-ла вершин графа $10\hm\leq N\hm\leq 7000$.


\vspace*{-6pt}

\subsection{Результаты для интервала $\tau\in(2,\,2{,}8]$}

\vspace*{-2pt}


Исследование зависимости сред\-не\-го рас\-сто\-яния от~$N$ и~$\tau\hm\in(2,\,2{,}8]$ в~до\-асимп\-то\-ти\-че\-ской об\-ласти
было проведено аналогично предыду\-ще\-му исследованию для $\tau\hm\in(1,2)$.
Сначала для фиксированных значений па\-ра\-мет\-ра $2{,}01\hm\leq\tau\hm\leq 2{,}8$ были построены зависимости сред\-не\-го
рас\-сто\-яния $\mathrm{dist}$ от чис\-ла вершин графа~$N$ сле\-ду\-юще\-го вида:
\begin{equation}
\label{eq7}
\mathrm{dist} = a \ln N + b\,.
\end{equation}

Обозначим 
$$
a_f=\fr{1}{\ln\left({2\zeta(\tau-1)}/{\zeta(\tau)}-2\right)}\,.
$$
 Для сравнения этих
значений с~коэффициентами~$a$ уравнений~(\ref{eq7}) для каждого~$\tau$ все они приведены
в~табл.~2 наряду с~коэффициентами~$b$ и~со\-от\-вет\-ст\-ву\-ющи\-ми коэффициентами детерминации
$R^2$ этих уравнений. Все коэффициенты~$a$ и~$b$ значимы, а~гипотезы о~ра\-венст\-ве нулю коэффициентов
детерминации полученных моделей отвергаются.


\setcounter{figure}{4}
\begin{figure*}[b] %fig5
\vspace*{1pt}
\begin{minipage}[t]{81mm}
\begin{center}
   \mbox{%
\epsfxsize=80mm
\epsfbox{ler-4-a.eps}
}
\end{center}
\vspace*{-13pt}
\Caption{Регрессионная зависимость~(\ref{eq9}) среднего рас\-сто\-яния $\mathrm{dist}$ от~$N$ при фиксированных
значениях $2\hm<\tau\hm\leq 2{,}8$:
\textit{1}~--- $\tau\hm= 2{,}01$; \textit{2}~--- 2,2288\ldots; \textit{3}~--- 2,4; \textit{4}~--- 2,6; \textit{5}~--- $\tau\hm= 2{,}8$}
%\end{figure*}
\end{minipage}
\hfill
%\begin{figure*} %fig6
\vspace*{1pt}
\begin{minipage}[t]{79.94mm}
\begin{center}
   \mbox{%
\epsfxsize=78.94mm
\epsfbox{ler-4-b.eps}
}
\end{center}
\vspace*{-13pt}
\Caption{Регрессионная зависимость~(\ref{eq9}) среднего рас\-сто\-яния $\mathrm{dist}$ 
от~$\tau$ при фиксированных значениях $10\hm\leq N\hm\leq 7000$:
\textit{1}~--- $N\hm=10$; \textit{2}~--- 100; \textit{3}~--- 1000;
\textit{4}~--- 5000; \textit{5}~--- $N\hm=7000$}
\end{minipage}
\end{figure*}

Таким образом, при фиксированных~$\tau$ из интервала $(2,\,2{,}8]$ сред\-нее рас\-сто\-яние в~графе
возрастает логарифмически с~рос\-том чис\-ла его вершин $N$, так
же как и~рас\-сто\-яние в~асимп\-то\-ти\-ке
(см.\ выраже-\linebreak\vspace*{-12pt}

%\begin{table*}\small  %tabl2
\begin{center}

\vspace*{6pt}

\noindent
\parbox{62mm}{{{\tablename~2}\ \ \small{Значения $a_f$, коэффициенты~$a$ и~$b$ зависимостей вида~(\ref{eq7})
и~коэффициенты детерминации $R^2$ этих уравнений
}}
}

\vspace*{6pt}


{\small \begin{tabular}{|c|c|c|c|c|}
\hline
&&&&\\[-10pt]
$\tau$ & $a_f$ & $a$ & $b$ & $R^2$ \\
 \hline
\hphantom{9}2,01 & 0,209 & 0,967 & $-$0,684 & 0,96 \\
2,1 &   0,408 & 1,151 & $-$1,661 & 0,98 \\
2,2 &   0,586 & 1,344 & $-$2,777 & 0,98 \\
2,3 &   0,800 & 1,541 & $-$3,994 & 0,95 \\
2,4 &   1,096 & 1,573 & $-$4,428 & 0,91 \\
2,5 &   1,565 & 1,430 & $-$4,078 & 0,88 \\
2,6 &   2,459 & 1,076 & $-$2,673 & 0,89 \\
2,7 &   4,942 & 0,760 & $-$1,387 & 0,91 \\
2,8 &   53,870\hphantom{9} & 0,507 & $-$0,352 & 0,94\\
\hline
\end{tabular}
}
\vspace*{3pt}
\end{center}
%\end{table*}

%\vspace*{3pt}

{ \begin{center}  %fig4
 \vspace*{-2pt}
   \mbox{%
\epsfxsize=78.504mm
\epsfbox{ler-3.eps}
}

\end{center}

\noindent
{{\figurename~4}\ \ \small{График экспериментальных
значений $\mathrm{dist}$ от $N$ и~$2\hm<\tau\hm\leq 2{,}8$
}}}

\vspace*{6pt}

\addtocounter{figure}{1}
\addtocounter{table}{1}


\noindent
 ние~(\ref{eq3})).
Сравнение значений коэффициентов~$a$ с~$a_f$ показывает, что для $2{,}01\leq\tau\leq 2{,}4$ значения~$a$ 
выше значений $a_f$, а~при $2{,}5\hm\leq\tau\hm\leq 2{,}8$ ниже (см.\ табл.~2).
Изменение углового коэффициента~$a$ в~данном случае показывает, что сред\-нее расстояние в~графе
с~рос\-том значения~$\tau$ сначала возрастает, достигая максимума в~промежутке от~2,2 до~2,4, 
а~затем убывает. На\-гляд\-но это мож\-но видеть на рис.~4, где показана за\-ви\-си\-мость
экспериментальных значений $\mathrm{dist}$ от~$N$ и~$\tau$.





Поиск зависимости среднего расстояния $\mathrm{dist}$ от переменных $N$ и~$\tau$ 
проходил по аналогии с~предыду\-щим интервалом изменения параметра распределения степеней вершин.
Сначала была по\-стро\-ена за\-ви\-си\-мость в~виде 
$$
\mathrm{dist}=\fr{\ln N}{\ln\left({2\zeta(\tau-1)}/{\zeta(\tau)}-2\right)}+b
$$
и получено сле\-ду\-ющее регрессионное уравнение:
\begin{equation*}
\mathrm{dist} = \fr{\ln N}{\ln\left({2\zeta(\tau-1)}/{\zeta(\tau)}-2\right)}-40{,}936\,.
\end{equation*}
К сожалению, на этом интервале коэффициент детерминации полученной за\-ви\-си\-мости оказался очень низким
($R^2\hm=0{,}0005$), гипотеза о~равенстве~$R^2$ нулю не отвергается и~коэффициент~$b$ не значим.
Следовательно, такую модель не имеет смыс\-ла использовать для прогноза.
Поэтому была осуществлена попытка по\-стро\-ить регрессию вида
$$
\mathrm{dist}=\fr{a\ln N}{\ln\left({2\zeta(\tau-1)}/{\zeta(\tau)}-2\right)}+b
$$ 
и~была получена зависимость
\begin{equation}
\label{eq8}
\mathrm{dist} = 4{,}962 - \fr{0{,}005\ln N}{\ln\left({2\zeta(\tau-1)}/{\zeta(\tau)}-2\right)}
\end{equation}
с коэффициентом детерминации $R^2\hm=0{,}06$. Несмотря на столь низ\-кое значение~$R^2$, гипотеза о~его
равенстве нулю отвергается, однако оценка зна\-чи\-мости коэффициентов~$a$ и~$b$
регрессионного уравнения~(\ref{eq8}) показала, что коэффициент~$a$ статистически значим, тогда как~$b$~-- нет.
Дальнейший поиск наилучшей регрессии привел к~получению сле\-ду\-ющей за\-ви\-си\-мости
сред\-не\-го рас\-сто\-яния конфигурационного графа $\mathrm{dist}$ от $N$ и~$\tau$:
\begin{equation}
\label{eq9}
\mathrm{dist} = \fr{(31{,}706-22{,}076\tau+3{,}841\tau^2)\ln N}{\ln\left({2\zeta(\tau-1)}/{\zeta(\tau)}-2\right)}\,,
\end{equation}
где все коэффициенты модели значимы, а $R^2\hm=0{,}74$. Зависимость~(\ref{eq9}) отражена графически на рис.~5 и~6.

Для моделей~(\ref{eq8}) и~(\ref{eq9}) была оценена зна\-чи\-мость различия между коэффициентами множественной
корреляции этих моделей при 5\%-ном уровне зна\-чи\-мости. В~результате $H_0:r_{(8)}=r_{(9)}$ была отвергнута, т.\,е.\
различие между коэффициентами корреляции оказалось значимым.
Проверка остатков регрессий~(\ref{eq8}) и~(\ref{eq9}) на нормальность показала, что нормальное распределение
имеют только остатки модели~(\ref{eq9}). Кроме того, остаточная сумма квад\-ра\-тов модели~(\ref{eq8})
$\mathrm{SSR}_{(8)}\hm=245793{,}9$ больше, чем $\mathrm{SSR}_{(9)}\hm=102369{,}9$. Поэтому мож\-но сделать вывод о~том, что модель~(\ref{eq9})
лучше подходит для прогноза, чем модель~(\ref{eq8}).
Таким образом, при значениях па\-ра\-мет\-ра $2\hm<\tau\hm\leq 2{,}8$ в~качестве наиболее подходящей модели за\-ви\-си\-мости
среднего рас\-сто\-яния $\mathrm{dist}$ от~$N$ и~$\tau$ в~до\-асимп\-то\-ти\-че\-ской об\-ласти предлагается за\-ви\-си\-мость, опи\-сы\-ва\-емая
уравнением~(\ref{eq9}).



На рис.~5 и~6 линии, находящиеся внут\-ри затененных областей и~отраженные в~легендах, соответствуют
зависимостям $\mathrm{dist}$ от~$N$ (см.\ рис.~5) и~от~$\tau$ (см.\ рис.~6) при некоторых
значениях па\-ра\-мет\-ра $\tau$ или объема графа~$N$ соответственно.
На рис.~5 ниж\-няя граница об\-ласти соответствует $\tau\hm=2{,}01$, верхняя~--- максимуму функции~(\ref{eq9}) 
по параметру $\tau$: $\tau^*\hm=2{,}2288\ldots$, а~кривые зависимостей $\mathrm{dist}$ от~$N$ внут\-ри затененной
об\-ласти расположены сле\-ду\-ющим образом: по воз\-рас\-та\-нию значений $\mathrm{dist}$ при увеличении значений~$\tau$ от~1,1
до~$\tau^*$ и~по убыванию $\mathrm{dist}$ при рос\-те~$\tau$ от~$\tau^*$ до~2,8. А~на рис.~6 кривые
зависимостей $\mathrm{dist}$ от~$\tau$ расположены друг над другом по мере воз\-рас\-та\-ния чис\-ла вершин графа
$10\hm\leq N\hm\leq 7000$ в~пределах граничных значений.

\vspace*{-9pt}


\subsection{Результаты при $\tau=2$}

\vspace*{-2pt}

Заметим, что модели~(\ref{eq6}) и~(\ref{eq9}) не охватывают значение па\-ра\-мет\-ра распределения степеней
вершин $\tau\hm=2$. Однако по экспериментальным данным была по\-стро\-ена сле\-ду\-ющая регрессионная за\-ви\-си\-мость
$\mathrm{dist}$ от~$N$ при фиксированном $\tau\hm=2$ (все коэффициенты модели значимы):
\begin{equation}
\label{eq10}
\mathrm{dist} = 5{,}262\ln\ln N - 4{,}045 \quad \left(R^2=0{,}73\right).
\end{equation}

\vspace*{-14.5pt}

\section{Выводы}

\vspace*{-2.5pt}

Итак, экспериментальные результаты на степенных конфигурационных графах с~фиксированным
па\-ра\-мет\-ром~$\tau$ распределения~(\ref{eq1}) степеней вершин показывают, что на интервале $(1,2)$ 
с~рос\-том объема~$N$ среднее расстояние $\mathrm{dist}$ в~графе рас\-тет как $\ln\ln N$, а~на интервале $2\hm<\tau\hm\leq 2{,}8$
рас\-тет логарифмически в~доасимптотической об\-ласти (при $N\hm\leq 7000$) так же, как это было показано
Р.~Хофстадом~\cite{Hof2} для типичного расстояния в~графах при $N\hm\rightarrow\infty$.
Однако что касается за\-ви\-си\-мости среднего расстояния от переменных~$N$ и~$\tau$, то при малых
объемах графа предлагается использовать модели~(\ref{eq6}) и~(\ref{eq9}) в~соответствующих интервалах
изменения параметра~$\tau$, так как они лучше описывают данную за\-ви\-си\-мость, что было под\-тверж\-де\-но
в~настоящей работе с~по\-мощью методов статистического анализа
и предложена модель~(\ref{eq10}) за\-ви\-си\-мости сред\-не\-го рас\-сто\-яния от чис\-ла вершин~$N$ при $\tau\hm=2$
так\-же для графов в~до\-асимп\-то\-ти\-че\-ской об\-ласти.

{\small\frenchspacing
 {\baselineskip=10.7pt
 %\addcontentsline{toc}{section}{References}
 \begin{thebibliography}{99}
 
 %\vspace*{-6pt}
 
\bibitem{Dur} %1
\Au{Durrett~R.} Random graph dynamics.~--- Cambridge: Cambridge University
Press, 2007. 221~p. doi: 10.1017/ CBO9780511546594.

\bibitem{Hof1} %2
\Au{Hofstad~R.} Random graphs and complex networks.~--- Cambridge:
Cambridge University Press, 2017.  Vol.~1. 337~p. doi: 10.1017/9781316779422.



\bibitem{New1} %3
\Au{Newman~M.\,E.\,J.} Networks. An introduction.~--- Oxford: Oxford University Press, 2010. 772~p.
doi: 10.1093/ acprof:oso/9780199206650.001.0001.

\bibitem{New2} %4
\textit{Newman~M.\,E.\,J.} Networks.~--- 2nd ed.~--- Oxford: Oxford University Press, 2018. 800~p.
doi: 10.1093/oso/ 9780198805090.001.0001.

\bibitem{Hof2} %5
\textit{Hofstad~R.} Random graphs and complex networks~// Notes RGCNII, 2020.  Vol.~2.
314~p. {\sf https://www.win. tue.nl/$\sim$rhofstad/NotesRGCNII.pdf.}

\bibitem{Fa} %6
\Au{Faloutsos~C., Faloutsos~P., Faloutsos~M.} On power-law relationships of
the internet topology~// Comput. Commun. Rev., 1999. Vol.~29. P.~251--262.
doi: 10.1145/ 316194.316229.

\bibitem{RN} %7
\Au{Reittu~H., Norros~I.} On the power-law random graph model of massive
data networks~// Perform. Evaluation, 2004. Vol.~55. Iss.~1-2. P.~3--23.
doi: 10.1016/S0166-5316(03)00097-X.

\bibitem{Bol}
\Au{Bollobas~B.} A~probabilistic proof of an asymptotic formula for the number
of labelled regular graphs~// Eur. J.~Combin., 1980. Vol.~1.
Iss.~4. P.~311--316. doi: 10.1016/S0195-6698(80)80030-8.



\bibitem{Chu} %9
\Au{Chung~F., Lu~L.} The average distances in random graphs with given expected degrees~// 
P.~Natl. Acad. Sci. USA, 2002. Vol.~99. Iss.~25. P.~15879--15882.
doi: 10.1073/pnas.252631999.

\bibitem{Dijk}
\Au{Dijkstra~E.\,W.} A~note on two problems in connexion with graphs~// 
Numer. Math., 1959. Vol.~1. Iss.~1. P.~269--271. doi: 10.1007/BF01386390.
\end{thebibliography}

 }
 }

\end{multicols}

\vspace*{-7pt}

\hfill{\small\textit{Поступила в~редакцию 21.03.22}}

%\vspace*{8pt}

%\pagebreak

\newpage

\vspace*{-28pt}

%\hrule

%\vspace*{2pt}

%\hrule

%\vspace*{-2pt}

\def\tit{AN AVERAGE DISTANCE IN~THE~POWER-LAW CONFIGURATION GRAPHS}


\def\titkol{An average distance in~the~power-law configuration graphs}


\def\aut{M.\,M.~Leri}

\def\autkol{M.\,M.~Leri}

\titel{\tit}{\aut}{\autkol}{\titkol}

\vspace*{-8pt}


\noindent
Institute of Applied Mathematical Research of the Karelian Research Center of the Russian Academy of Sciences, 
11~Pushkinskaya Str., Petrozavodsk 185910, Russian Federation

\def\leftfootline{\small{\textbf{\thepage}
\hfill INFORMATIKA I EE PRIMENENIYA~--- INFORMATICS AND
APPLICATIONS\ \ \ 2023\ \ \ volume~17\ \ \ issue\ 1}
}%
 \def\rightfootline{\small{INFORMATIKA I EE PRIMENENIYA~---
INFORMATICS AND APPLICATIONS\ \ \ 2023\ \ \ volume~17\ \ \ issue\ 1
\hfill \textbf{\thepage}}}

\vspace*{3pt} 


\Abste{In random configuration graphs with a~discrete power-law vertex degree distribution with a~fixed parameter, 
the average distance in the graph is considered, i.\,e., the arithmetic mean of distances between all pairs of graph nodes. 
This characteristic is estimated using simulation methods. Due to computational constraints, the author considers graphs
 in the pre-asymptotic domain (in this paper, these are the graphs up to 7000~nodes). The models of dependencies of the average distance on 
 the graph size and the parameter of vertex degree distribution are reseived. The obtained results are compared with the results 
 of theoretical studies of the typical distance in a graph in the asymptotics (i.\,e., when the number of graph vertices tends to infinity), 
 given in the works by R.~Hofstad.}

\KWE{configuration graph; power-law distribution;
average distance in a graph; simulation}



\DOI{10.14357/19922264230104} 

\vspace*{-16pt}

\Ack
\noindent
The study was carried out under state order to the Karelian Research Center 
of the Russian Academy of Sciences (Institute of Applied Mathematical Research KarRC RAS).

\vspace*{6pt}

  \begin{multicols}{2}

\renewcommand{\bibname}{\protect\rmfamily References}
%\renewcommand{\bibname}{\large\protect\rm References}

{\small\frenchspacing
 {%\baselineskip=10.8pt
 \addcontentsline{toc}{section}{References}
 \begin{thebibliography}{99} 
\bibitem{1-leri-1}
\Aue{Durrett, R.} 2007. \textit{Random graph dynamics.} Cambridge: Cambridge University
Press. 221~p. doi: 10.1017/ CBO9780511546594.

\bibitem{2-leri-1}
\Aue{Hofstad, R.} 2017. \textit{Random graphs and complex networks.} Cambridge:
Cambridge University Press.   Vol.~1. 337~p. doi: 10.1017/9781316779422.

\bibitem{4-leri-1} %3
\Aue{Newman, M.\,E.\,J.} 2010. \textit{Networks. An introduction.} Oxford: Oxford
University Press. 772~p. doi: 10.1093/ acprof:oso/9780199206650.001.0001.

\bibitem{5-leri-1} %4
\Aue{Newman, M.\,E.\,J.} 2018. \textit{Networks.} 2nd ed. Oxford: Oxford
University Press. 800~p. doi: 10.1093/oso/ 9780198805090.001.0001.

\bibitem{3-leri-1} %5
\Aue{Hofstad, R.} 2020. Random graphs and complex networks.  \textit{Notes RGCNII}. Vol.~2. 314~p.
Available at: {\sf https:// www.win.tue.nl/$\sim$rhofstad/NotesRGCNII.pdf} (accessed January~18, 2023)



\bibitem{6-leri-1}
\Aue{Faloutsos, C., P.~Faloutsos, and M.~Faloutsos.} 1999. On power-law relationships of
the internet topology. \textit{Comput. Commun. Rev.} 29:251--262.
doi: 10.1145/316194.316229.

\bibitem{7-leri-1}
\Aue{Reittu, H., and I.~Norros.} 2004. On the power-law random graph model of massive data
networks. \textit{Perform. Evaluation} 5(1-2)5:3--23.
doi: 10.1016/S0166-5316(03)00097-X.

\bibitem{8-leri-1}
\Aue{Bollobas, B.} 1980. A~probabilistic proof of an asymptotic formula for the number
of labelled regular graphs. \textit{Eur. J.~Combin.} 1(4):311--316.
doi: 10.1016/S0195-6698(80)80030-8.

\bibitem{9-leri-1}
\Aue{Chung, F., and L.~Lu.} 2002. The average distances in random graphs with given expected
degrees. \textit{P.~Natl. Acad. Sci. USA} 99(25):15879--15882.
doi: 10.1073/pnas.252631999.

\bibitem{10-leri-1}
\Aue{Dijkstra, E.\,W.} 1959. A~note on two problems in connexion with graphs.
\textit{Numer. Math.} 1(1):269--271. doi: 10.1007/BF01386390.
\end{thebibliography}

 }
 }

\end{multicols}

\vspace*{-6pt}

\hfill{\small\textit{Received March 21, 2022}}


\Contrl

\noindent
\textbf{Leri Marina M.} (b.\ 1969)~--- Candidate of Science (PhD) in technology, scientist,
Institute of Applied Mathematical Research of the Karelian Research Center of the Russian Academy of Sciences,
11~Pushkinskaya Str., Petrozavodsk 185910, Russian Federation; \mbox{leri@krc.karelia.ru}


\label{end\stat}

\renewcommand{\bibname}{\protect\rm Литература} 