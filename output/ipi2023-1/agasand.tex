
\def\stat{agasandyan}

\def\tit{МНОГОМЕРНЫЕ БАТТЕРФЛЯИ\\ В~ЗАДАЧАХ ОПТИМИЗАЦИИ ПО CC-VaR}

\def\titkol{Многомерные баттерфляи в~задачах оптимизации 
по~CC-VaR}

\def\aut{Г.\,А.~Агасандян$^1$}

\def\autkol{Г.\,А.~Агасандян}

\titel{\tit}{\aut}{\autkol}{\titkol}

\index{Агасандян Г.\,А.}
\index{Agasandyan G.\,A.}


%{\renewcommand{\thefootnote}{\fnsymbol{footnote}} \footnotetext[1]
%{Работа выполнена при поддержке Министерства науки и~высшего образования
%Российской федерации, грант №\,075-15-2020-799.}}


\renewcommand{\thefootnote}{\arabic{footnote}}
\footnotetext[1]{Федеральный исследовательский центр <<Информатика и~управление>> Российской 
академии наук, \mbox{agasand17@yandex.ru}}

\vspace*{-6pt}
 
  
  \Abst{Работа продолжает исследование технических проблем, связанных с~применением  
континуального критерия VaR (CC-VaR) на многомерных рынках опционов. 
В~предположении, что на рынке сценарными баттерфляями непосредственно не торгуют, 
разрабатывается методика получения их реп\-ли\-ка\-ции из многомерных $\alpha$-оп\-ци\-онов~--- 
многомерного обобщения обычных одномерных опционов, таких как коллы и~путы. Работа 
служит непосредственным расширением предложенного в~предыдущей работе автора 
способа, позволяющего конструировать индикаторы базиса на многомерном сценарном 
рынке комбинациями многомерных бинарных опционов. Методика основывается на 
теоремах паритета для одномерного рынка традиционных опционов и~пригодна для рынков 
произвольной размерности, но ее фактическая реализация проводится для двумерных 
рынков. Приводятся конструкции базисов из $\alpha$-оп\-ци\-онов~--- как однотипных, так 
и~смешанных естественных с~выделенным цент\-ром рынка. Теоретические пред\-став\-ле\-ния 
оптимальных портфелей в~этих базисах иллюстрируются на примере конкретного 
двумерного рынка.}
   
  \KW{базовые активы; многомерный рынок; функция рисковых предпочтений инвестора; 
континуальный критерий VaR (CC-VaR); стоимостная и~прогнозная плотности; опционы 
колл и~пут; $\alpha$-оп\-ци\-оны; сценарные баттерфляи; базисы; центр рынка; портфели 
баттерфляев}

 \DOI{10.14357/19922264230114} 
  
\vspace*{-2pt}


\vskip 10pt plus 9pt minus 6pt

\thispagestyle{headings}

\begin{multicols}{2}

\label{st\stat}
   
  \section{Введение}
  
  Проблемы применения на рынках опционов введенного автором 
континуального критерия VaR (CC-VaR) рассматриваются в~[1--5]. Настоящую 
работу можно рассматривать как продолжение исследования~[6], в~котором 
предлагались варианты репликации индикаторов базиса на многомерном 
сценарном рынке комбинациями так называемых $\zeta$-оп\-ци\-онов 
(многомерных бинарных опционов).\linebreak Здесь подобная задача решается для более 
сложных инструментов~--- многомерных аналогов одномерных базисных 
баттерфляев, которые реплицируются комбинациями так называемых
  $\alpha$-оп\-ци\-онов~--- многомерных аналогов традиционных \mbox{опционов} типа 
колл и~пут. 
  
  Основания для такого рассмотрения и~его проб\-ле\-мы, связанные 
с~применением континуального критерия VaR (CC-VaR), приведены в~[6], там 
же вводятся многие обозначения, которые используются и~здесь. 
Теоретической моделью при построении $\alpha$-рын\-ка служит также 
многомерный $\delta$-ры\-нок~\cite{5-aga, 6-aga}. 
  
  В работе для многомерных рынков опционов решаются те же проблемы 
технического характера, что и~для $\zeta$-рын\-ков~--- рынков  многомерных 
бинарных опционов. Но на этот раз в~отношении своего инструментария они 
в~большей мере напоминают проб\-ле\-мы традиционных рынков опционов, на 
которых в~отсутствие баттерфляев в~качестве объектов непосредственной 
торговли предлагается получать их в~виде комбинаций коллов и~путов. 
  
  \section{Теоретический $\alpha$-рынок и~его~свойства}
  
  Вновь рассматривается многомерный $\delta$-ры\-нок (однопериодный, 
теоретический и~идеальный)\linebreak с~$n$ ($>1$) базовыми активами, векторы цен 
которых в~конце периода $\bm{x}\hm= (x_1, x_2, \ldots, x_n)$, $x_l\hm\in {\sf X}_l \hm\subset 
\mathfrak{R}$, $l\hm\in N\hm=\{1, \ldots , n\}$, образуют $n$-мер\-ное множество 
${\mathsf X}\hm=\prod_{l\in N} {\mathsf X}_l$. На~${\mathsf X}$ заданы 
\textit{прогнозная} $p(\bm{x})$ и~\textit{стоимостная} $c(\bm{x})$ плотности, 
по\-рож\-да\-ющие вероятностные меры~${\mathsf P}\{\cdot\}$ и~${\mathsf 
C}\{\cdot\}$. 
  
  Платежная функция произвольного инструмента~$\bm{I}$ обозначается 
$\pi(\bm{x}; \bm{I})$, его рыночная сто\-и\-мость и~средний, с~точки зрения 
инвестора, доход, рас\-счи\-тан\-ные по плотностям $c(\bm{x})$ и~$p(\bm{x})$ 
соответственно, определяются соотношениями: 
  $$
  \vert \bm{I}\vert =\int\limits_{\mathsf X} \pi (\bm{x};\bm{I}) 
c(\bm{x})\,d\bm{x}\,;\enskip
  \left\| \bm{I}\right\| =\int\limits_{\mathsf X} \pi(\bm{x};\bm{I}) 
p(\bm{x})\,d\bm{x}\,.
  $$
  
  Базис рынка составляют $\delta$-ин\-стру\-мен\-ты $\bm{D}(\bm{s})$, 
$\bm{s}\hm\in {\mathsf X}$, с~обобщенной $n$-мер\-ной $\delta$-функ\-ци\-ей 
относительно~$\bm{s}$ в~качестве платежной: 

\noindent
  \begin{multline}
    \bm{D}(\bm{s}) =\prod\limits_{l\in N} \bm{D}_l(s_l)\,,\\
  \pi(\bm{x};\bm{D}(\bm{s}))=\delta(\bm{x}-\bm{s})=\prod\limits_{l\in N} 
\delta(x_l-s_l).
  \label{1-aga}
  \end{multline}
  
  Инструмент $\bm{G}$ с~произвольной измеримой платежной 
функцией~$g(\bm{x})$ и~его стоимость имеют вид: 
  \begin{align*}
  \bm{G}&= \int\limits_{\mathsf X} g(\bm{s}) \bm{D}(\bm{s})\,d\bm{s}\,;\\
  \vert \bm{G}\vert &=\int\limits_{\mathsf X} g(\bm{s}) \vert \bm{D}(\bm{s}) \vert 
\,d\bm{s}= \int\limits_{\mathsf X} g(\bm{s}) c(\bm{s}) \,d\bm{s}\,.
  \end{align*}
  
  Наряду с~<<полноправными>> $n$-мер\-ны\-ми инструментами на рынке 
присутствуют и~их $k$-мер\-ные версии, у~которых $n\hm- k$ координатных 
базовых активов пред\-став\-ле\-ны в~форме одномерных единичных без\-рис\-ко\-вых 
инструментов. 
  
  Для индикаторов $\bm{H}\{M\}$, $M\hm\subset {\mathsf X}$, без\-рис\-ко\-во\-го 
актива $\bm{U}\hm=\bm{H}\{ {\mathsf X}\}$ и~их цен 
  \begin{gather*}
  \bm{H}\{ M\}=\int\limits_M \bm{D}(\bm{s})\, d\bm{s}\,;\enskip
   \vert \bm{H}\{M\}\vert =\int\limits_M c(\bm{s})\,d\bm{s}\,;\\
   \vert \bm{U}\vert ={\mathsf C}\{ {\mathsf X}\}= \int\limits_{\mathsf X} c(\bm{s}) 
\,d\bm{s}=\fr{1}{r}\,,
   \end{gather*}
где $r$~--- приравниваемый единице безрисковый доход за период. 
  
  На \textit{одномерном} рынке опционы пут~$\bm{P}_s$ и~колл~$\bm{C}_s$ 
со страйком~$s$ задаются своими платежными функциями: 
  \begin{equation}
  \left.
  \begin{array}{rl}
  \pi(x;\bm{P}_s)&=\max (0,s-x);\\[6pt]
  \pi(x;{\bm C}_x)&=\max (0,x-s),\ x,s \in {\mathsf X}\subset \mathfrak{R}\,.
  \end{array}
  \right\}
  \label{e2-aga}
  \end{equation}
  %
  Для них выполняется формула паритета ($\bm{X}$~--- вектор базовых 
активов)
  $$
  \bm{C}_s-\bm{P}_s= \bm{X}-s\bm{U}\,.
  $$
  
  Нормированными спрэдами быка с~парой страйков $s\hm-h$, $s \hm\in 
{\mathsf X}$ ($h \hm>0$)  и~медведя с~парой страйков~$s$, $s\hm+h \hm\in 
{\mathsf X}$ служат комбинации опционов соответственно 
  \begin{equation}
  \left.
  \begin{array}{rl}
 \!\!\!\! \bm{S}_{s;h}^{\mathrm{bull}} &= \fr{1}{h}\left( \bm{C}_{s-h}-\bm{C}_s\right) =
\bm{U}+\fr{1}{h}\left( \bm{P}_{s-h}-\bm{P}_s\right)\,;\\[6pt]
 \!\! \!\! \bm{S}_{s;h}^{\mathrm{bear}} &=\fr{1}{h}\left( \bm{P}_{s+h}-\bm{P}_s\right) 
=\bm{U}+\fr{1}{h} \left( \bm{C}_{s+h}-\bm{C}_s\right)
  \end{array}\!
  \right\}\!\!
  \label{e3-aga}
  \end{equation}
с платежными функциями 
\begin{equation}
\left.
\begin{array}{rl}
 \!\!\!\!\pi\left( x;\bm{S}_{s;h}^{\mathrm{bull}}\right) &= \min \left(\! 1,\fr{1}{h}\max (0,x-(s-h))\!\right);\\[9pt]
\! \!\!\!\pi\left( x;\bm{S}_{s;h}^{\mathrm{bear}}\right) &= \min \left(\! 1,\fr{1}{h}\max (0, (s+h)-x)\!\right).
\end{array}\!
\right\}\!
\label{e4-aga}
\end{equation}
  
  Нормированные симметричные баттерфляи с~тройкой страйков $s\hm-h$, $s$, 
$s\hm+h \hm\in {\mathsf X}$ образуются комбинациями 
  \begin{multline}
  \bm{B}_{s;h} = \fr{1}{h}\left( \bm{C}_{s-h} -2\bm{C}_s +\bm{C}_{s+h}\right) 
={}\\
  {}= \fr{1}{h} \left( \bm{P}_{s-h} -2\bm{P}_s+\bm{P}_{s+h}\right) 
={}\\
{}=\bm{U}+\fr{1}{h}\left( \bm{P}_{s-h} -\bm{P}_s -
\bm{C}_s+\bm{C}_{s+h}\right)
  \label{e5-aga}
  \end{multline}
с платежными функциями 
\begin{equation}
\pi \left( x;\bm{B}_{s;h}\right) =\fr{1}{h}\max ( 0, h-\vert x-s\vert).
\label{e6-aga}
\end{equation}
  
  В комбинациях~(\ref{e3-aga}) и~(\ref{e5-aga}) безрисковый 
инструмент~$\bm{U}$ выполняет функцию маржевого инструмента 
и~применяется инвестором в~соответствии с~требованиями рынка. Формально 
верно еще одно пред\-став\-ле\-ние:
  \begin{equation*}
  \bm{B}_{s;h}=\fr{1}{h}\left( \bm{C}_{s-h} -\bm{C}_s -\bm{P}_s 
+\bm{P}_{s+h}\right)\,,
 % \label{e7-aga}
  \end{equation*}
но оно не является \textit{естественным} (страйки коллов в~комбинации ниже 
страйков путов) и~потому далее не используется. 
  
  Можно было бы рассматривать и~не создающие принципиальных трудностей 
несимметричные баттерфляи (с~неравными по длине сценариями 
и~неравномерной линейкой страйков), но они, как правило, не применяются на 
рынках и~к~тому же сильно загромождали бы изложение. 
  
  С целью алгоритмической автоматизации дальнейших построений для 
одномерных опционов $\bm{P}_s$ и~$\bm{C}_s$ вводятся также обозначения 
$\bm{O}_s^-$ (и~$\bm{O}_{0;s}$) и~$\bm{O}_s^+$ (и~$\bm{O}_{1;s})$, которые 
могут обрастать дополнительными индексами координат $l\hm\in N$: 
  \begin{equation}
  \bm{O}_{0;s}\equiv \bm{O}_s^- \equiv \bm{P}_s\,;\enskip
  \bm{O}_{1;s} \equiv \bm{O}_s^+\equiv \bm{C}_s\,.
  \label{e8-aga}
  \end{equation}
  
  \textit{Многомерным} обобщением одномерных опционов $\bm{P}_s$ 
и~$\bm{C}_s$ служат $n$-мер\-ные $\alpha$-\textit{оп\-ци\-оны} 
$\bm{A}_{\bm{\alpha};\bm{s}}$ векторного типа~$\bm{\alpha}$ и~с~векторным 
страйком $\bm{s}\hm \in \mathfrak{R}^n$, задаваемые вместе с~платежными 
функциями соотношениями 
  \begin{multline}
  {A}_{\alpha;s}=\prod\limits_{i\in N} \bm{O}_{i\beta_i;s_i}\,,\\
  \pi\left(\bm{x}, A_{\alpha;s}\right)=
  \prod_{l\in N}\pi \left(x_l; \bm{O}_{l\beta_l;s_l}\right),\ \bm{x}\in 
\mathfrak{R}^n\,,\\[6pt]
  \pi\left( x_l; \bm{O}_{l\beta_l;s_l}\right) =\omega_{l\beta_l;s_l}(x_l)={}\\[6pt]
  \hspace*{10mm}{}=\max  \left(0,\alpha_l (x_l-s_l)\right), \enskip l\in N\,.
    \label{e9-aga}
  \end{multline}
  
  Как и~в~[6], вектор~$\bm{\alpha}$ с~компонентами $\alpha_l\hm= \pm1$, $l\hm\in N$, 
в~индексах инструментов (или просто~$\pm$) определяет векторный тип  
$\alpha$-оп\-ци\-онов~$\bm{A}_{\bm{\alpha};\bm{s}}$. Вектор $\bm{\beta}\hm= 
(\bm{\alpha}\hm+1)/2$, дублирующий~$\bm{\alpha}$, вводится для удобства по 
техническим причинам и~принимает для каждого $l\hm\in N$ два значения: 
$$
\beta_l= \begin{cases}
0 &\mbox{для\ пута;}\\ 
1 & \mbox{для\ колла.}
\end{cases}
$$ 
  
  Для каждого векторного страйка~$\bm{s}$ на $n$-мер\-ном рынке могут 
котироваться $2^n$ типов $\alpha$-оп\-ци\-онов. Рынок $n$-мер\-ных  
$\alpha$-оп\-ци\-онов с~их $k$-мер\-ны\-ми версиями, $k\hm< n$, называется  
$n$-мер\-ным $\alpha$-\textit{рын\-ком}. 
  
  \section{Двумерный дискретный $\alpha$-рынок}
   
  В основе дискретного $\alpha$-рын\-ка лежит \textit{сценарный} рынок~--- 
сценарная дискретизация двумерного тео\-ре\-ти\-че\-ско\-го $\delta$-рын\-ка. Как 
и~для бинарного рынка, используется в~большей мере адаптированная 
к~двумерному случаю очевидная сис\-те\-ма обозначений, но учитывается 
и~специфика требований $\alpha$-рынка. 
  
  Цены двух базовых активов \textit{теоретического} двумерного  
$\delta$-рын\-ка обозначаются~$x$ и~$y$, страйки опционов~--- 
соответственно~$s$ и~$t$, $x, s \hm\in {\mathsf X} \hm=[a_1,b_1)\hm\subset 
\mathfrak{R}$, $y,t \hm\in {\mathsf Y}\hm=[a_2,b_2) \hm\subset \mathfrak{R}$. 
Дискретизация осуществляется равномерным разбиением множества~${\mathsf 
X}$ на~$v_1$ интервалов (сценариев), ${\mathsf Y}$~--- на $v_2$ интервалов. 
Одномерные сценарии на~${\mathsf X}$ и~${\mathsf Y}$ даются формулами: 
  \begin{multline}
  S_i= \left[ x_{i-1},x_i\right),\ x_i=a_1+ih_1,\ h_1=\fr{b_1-a_1}{v_1},\\
   i\in  \bm{I},\ x_0=a_1; \label{e10-aga}
   \end{multline}
   
   \vspace*{-12pt}
   
   \noindent
   \begin{multline}
  T_j= \left [ y_{j-1},y_j\right),\ y_j= a_2+jh_2,\ h_2=\fr{b_2-a_2}{v_2},\\ j\in 
\bm{J},\ y_0=a_2\,,
  \label{e11-aga}
\end{multline}
где $\bm{I}=\{1,2, \ldots, v_1\}$, $\bm{J}\hm=\{1,2,\ldots , v_2\}$, а~номер 
сценария совпадает с~индексом его правой границы. Двумерными сценариями 
служат прямые произведения всех пар $S_i\times T_j$, $i\hm\in \bm{I}$, $j\hm\in 
\bm{J}$. 
  
  На сценарном рынке базис образуют индикаторы сценариев 
$\bm{D}_{ij}\hm=\bm{H}\{S_i\times T_j\}$, но для $\alpha$-рын\-ка при той же структуре 
сценариев уместнее использовать иной базис~--- из баттерфляев~$\bm{B}_{ij}$, 
задаваемых с~учетом определения~(\ref{e5-aga}), но для специально 
подобранных страйков. Страйками~$s_i$ и~$t_j$ одномерных опционов 
и~упомянутых баттерфляев служат середины сценариев~(\ref{e10-aga}) 
и~(\ref{e11-aga}): 
  $$
  s_i = \fr{x_{i-1} + x_i}{2}\,,\  i\in \bm{I}; \enskip   t_j = \fr{y_{j-1} + y_j}{2},\   j\in 
\bm{J}\,. 
  $$
  %
  При этом параметр~$h$ для баттерфляев~(\ref{e5-aga}), равный длине 
сценариев, определяется в~(\ref{e10-aga}) и~\ref{e11-aga}). Для удобства 
записи формул также доопределяются параметры $s_0 \hm= a_1$, $s_{v_1+1}\hm = 
b_1$, $t_0 \hm= a_2$, $t_{v_2+1}\hm = b_2$, но они страйками не являются. 
  
  Портфель с~вектором~$\bm{g}$ весов базисных баттерфляев в~двумерном 
случае приобретает вид: 
  \begin{equation}
  \bm{G}= \sum\limits_{i\in \bm{I}, j\in \bm{J}} g_{ij} \bm{B}_{ij}\,.
  \label{e12-aga}
  \end{equation}
  
  Двумерным обобщением обычных опционов служат инструменты, 
характеризуемые парой страйков $(s_i,t_j)$, или просто $(i,j)$, 
с~дополнительным указанием типа (лучше в~терминах $\bm{\beta}\hm=(\beta_1, 
\beta_2))$: 
\begin{multline*}
\bm{A}_{\beta_1\beta_2;ij} 
=\bm{O}_{\beta_1;1,i}\bm{O}_{\beta_2;2,j}=\bm{O}_{\beta_1;i\cdot} 
\bm{O}_{\beta_2;\cdot j}\,,\\ 
i\in \bm{I}\,,\ j\in \bm{J}\,.
\end{multline*}
  %
  Также рассматриваются и~их одномерные версии, обозначаемые~$\bm{A}_{i\cdot}$ 
и~$\bm{A}_{\cdot j}$ с~маркером <<точка>> в~позиции, отведенной координате 
безрискового актива. 
  
  Для представления произвольного инструмента~$\bm{G}$~(\ref{e12-aga}) 
в~базисе из $\alpha$-оп\-ци\-онов (избыточным в~сравнении с~базисом из 
баттерфляев) все нормированные баттерфляи в~(\ref{e12-aga}) следует 
реплицировать в~терминах $\alpha$-оп\-ци\-онов. 
  
  В соответствии с~(\ref{e9-aga}) для конструирования репликаций следует 
перемножать одномерные представления сценарных баттерфляев, выбирая 
подходящие сомножители из~(\ref{e3-aga}) и~(\ref{e5-aga}). Для одномерного 
рынка с~$v$~сценариями $i\hm\in \bm{I}$ справедливы такие репликации 
баттерфляев коллами и~путами:
  \begin{multline}
  \bm{B}_i={}\\
  \!\!\!\!{}=\begin{cases}
  \bm{U}-\fr{\bm{O}_1^+ -\bm{O}_2^+}{h}=\fr{\bm{O}_2^- -\bm{O}_1^-}{h}\,, & i=1;\\
  \fr{\bm{O}_{i-1}^+-2\bm{O}_i^+ +\bm{O}_{i+1}^+}{h}={}&\\
  \hspace*{7mm}{}= \fr{\bm{O}^-_{i-1}-2\bm{O}_i^- +\bm{O}^-_{i+1}}{h}={}&\\
  \hspace*{12mm}{}= \bm{U}-\fr{\bm{O}_i^- -\bm{O}^-_{i-1}}{h} -{}&\\
  \hspace*{15mm}{}- \fr{\bm{O}_i^+  - \bm{O}^+_{i+1}}{h}\,, &\hspace*{-7mm} 1<i<v\,;\\
  \fr{\bm{O}^+_{v-1} -\bm{O}_v^+}{h}=\bm{U}-\fr{\bm{O}_v^- -\bm{O}^-_{v-
1}}{h}\,, & i=v\,.
  \end{cases}\!
  \label{e13-aga}
  \end{multline}
  
  Базисные инструменты для $i\hm=\overline{1,v}$ являются спрэдами, но их для 
удобства также называем баттерфляями (\textit{усеченными}). 
Инструмент~$\bm{U}$ в~выписанных соотношениях, как в~(\ref{e3-aga}) 
и~(\ref{e5-aga}),  выполняет функцию маржевого инструмента. 
  
  Подобно сценарным базисам для $\zeta$-рынка~\cite{6-aga} построение 
двумерных базисов из $\alpha$-оп\-ци\-онов проводится на основе одномерных 
базисов, но их элементы на этот раз выбираются из~(\ref{e13-aga}). Строятся 
три варианта репликации базисов: два однотипных (один в~путах, другой 
в~коллах) и~третий~--- смешанный естественный. Если в~базисе $v$~сценариев, 
а~центральный страйк~$i_c$, то 
  \begin{itemize}
\item однотипный базис при $\alpha\hm=-1$ (в~путах):
\begin{equation}
\left.
\begin{array}{rl}
\bm{B}_1^- &=\fr{\bm{O}_2^*-\bm{O}_1^-}{h}\,;\\[6pt]
  \bm{B}_i^- &= \fr{\bm{O}^-_{i-1} -2\bm{O}_i^- +\bm{O}^-_{i+1}}{h}\,;\\[6pt]
   \bm{B}^-_v &=\bm{U}- \fr{\bm{O}_v^- -\bm{O}^-_{v-1}}{h}\,;
   \end{array}
   \right\}
\label{e14-aga}
\end{equation}
\item однотипный базис при $\alpha\hm=+1$ (в~коллах):
\begin{equation}
\left.
\begin{array}{rl}
\bm{B}_1^+ &\equiv \bm{U} - \fr{\bm{O}_1^+ -\bm{O}_2^+}{h}\,;\\[6pt]
  \bm{B}_i^+ &= \fr{\bm{O}^+_{i-1} -2\bm{O}_i^+ +\bm{O}^+_{i+1}}{h}\,;\\[6pt]
    \bm{B}_v^+ &\equiv \fr{\bm{O}^+_{v-1} -\bm{O}_v^+}{h}\,;
    \end{array}
    \right\}
\label{e15-aga}
\end{equation}
\item смешанный естественный базис:
\begin{equation}
\left.
\begin{array}{l}
\!\!\!\bm{B}_1^m\equiv \fr{\bm{O}_2^- -  \bm{O}_1^-}{h}\,;\\[6pt]
  \!\!\!\bm{B}_i^m \equiv \fr{\bm{O}^-_{i-1} -2\bm{O}_i^- +\bm{O}^-_{i+1}}{h}\,,\ 0< i< i_c;\\[6pt]
\!\!\!\bm{B}^m_{i_c} \equiv  \bm{U}-\fr{\bm{O}^-_{i_c-1} -\bm{O}^-_{i_c} -
\bm{O}^+_{i_c} +\bm{O}^+_{i_c+1}}{h}\,;\\[6pt]
\!\!\!\bm{B}_i^m\equiv \fr{\bm{O}^+_{i-1} -2\bm{O}_i^++\bm{O}^+_{i+1}}{h_i}\,,\ i_c<i<v\,;\\[6pt]
 \!\!\!\bm{B}_v^m\equiv \fr{\bm{O}^+_{v-1} -\bm{O}_v^+}{h}\,.
\end{array}\!
\right\}\!
\label{e16-aga}
\end{equation}
  \end{itemize}
  
  \section{Формирование базисов и~платежных функций 
портфелей $\alpha$-опционов}
  
  На основе соотношений~(\ref{e14-aga})--(\ref{e16-aga}) введенные 
  в~многомерном случае произвольной размерности конструкции здесь 
переписываются для двумерного  
$\alpha$-рын\-ка в~однотипных и~смешанных вариантах. 
  
  Поскольку каждый двумерный базисный баттерфляй определяется как 
произведение двух одномерных (что соответствует перемножению платежных 
функций), его репликации двумерными\linebreak $\alpha$-оп\-ци\-она\-ми находятся 
перемножением пары подходящих представлений  
из~(\ref{e14-aga})--(\ref{e16-aga}). 
  
  \textit{Однотипная} репликация сценарных баттерфляев проводится 
  $\alpha$-оп\-ци\-она\-ми единого типа $\bm{\alpha}\hm=\{\alpha_1, \alpha_2\}$. Он фиксируется 
заранее, и~потому используются более простые соотношения~(\ref{e14-aga}) 
и~(\ref{e15-aga}), а~обозначение типа опциона опускается. 
  
  Каждое перемножение сумм одномерных опционов в~(\ref{e14-aga}) 
или~(\ref{e15-aga}) дает сумму парных произведений этих опционов, которые 
затем следует замещать согласно~(\ref{e9-aga}) эквивалентными двумерными 
$\alpha$-оп\-ци\-она\-ми по правилам 
  \begin{multline}
  1\to \bm{U}\,,\enskip \bm{O}_{1,i}\bm{O}_{2,j}\to \bm{A}_{ij}\,,\\ 
\bm{O}_{1,i}\bm{U}_2\to \bm{A}_{i\cdot}\,,\enskip \bm{U}_1 \bm{O}_{2,j}\to 
\bm{A}_{\cdot j}\,.
  \label{e17-aga}
  \end{multline}
  
  \textit{Смешанная} репликация осуществляется аналогично, но указание типа в~обозначениях необходимо, и~потому правила трансформации приобретают 
вид: 
  \begin{multline}
  1\to \bm{U}\,,\ \bm{O}_{1,i}^{\alpha_1} \bm{O}_{2,j}^{\alpha_2} \to 
\bm{A}_{ij}^{\bm {\alpha}}=\bm{A}_{\beta_1,\beta_2;ij},\\
\bm{O}_{1,i}^{\alpha_1}\to \bm{A}_{\beta_1;i,\cdot} \left( = 
\bm{A}^{\alpha_1}_{i\cdot}\right)\,,\ \bm{O}^{\alpha_2}_{2,j} \to 
\bm{A}_{\beta_2;\cdot,j}\left( =\bm{A}^{\alpha_2}_{\cdot j}\right),\\
 \beta_1,\beta_2\in  \{0,1\}\,,
\label{e18-aga}
\end{multline}
  
  На двумерном $\alpha$-рын\-ке в~соответствии с~чис\-лом возможных 
векторов~$\bm{\alpha}$ насчитываются четыре варианта однотипных базисов 
и~один смешанный (естественный с~заданным центром рынка). 
  
  Для каждого варианта с~\textit{однотипным} базисом и~оптимальным 
портфелем фиксируется тип~$\bm{\alpha}$, и~он становится типом всех  
$\alpha$-оп\-ци\-онов варианта. В~двумерном случае таких типов четыре: $\{-1, -
1\}$; $\{-1,+1\}$; $\{+1,-1\}$; $\{+1,+1\}$. Последовательным применением 
правил~(\ref{e17-aga}) ко всем страйкам для каж\-до\-го значения векторного 
параметра~$\bm{\alpha}$ находятся искомые четыре базиса. В~однотипном случае 
для каждой компоненты рынка наличествуют три качественно различных 
представления по варианту страйка~--- двум крайним и~общему внутреннему, 
и~потому их $3^2\hm=9$. В~качестве примера приводится базис для 
$\bm{\alpha}\hm=\{-1,+1\}$, т.\,е.\ в~терминах $\alpha$-оп\-ци\-онов~$\bm{A}_{01}$ 
(остальные три \textit{однотипных} базиса выписываются сходным образом), 
при этом в~списке принимается $0\hm<i \hm< v_1$, $0 \hm< j \hm< v_2$: 
  \begin{align*}
  \bm{B}_{1,1}&=\fr{\bm{A}_{1,1}- \bm{A}_{1,2} -
\bm{A}_{2,1}+\bm{A}_{2,2}}{h_1h_2}+{}\\
&\hspace*{35mm}{}+ \fr{-\bm{A}_{1,\cdot} +\bm{A}_{2,\cdot}}{h_1}\,;\\
  \bm{B}_{1,j} &= \fr{-\bm{A}_{1,j-1} +2\bm{A}_{1,j} -\bm{A}_{1,j+1}}{h_1h_2}+{}\\
  &  \hspace*{15mm} {}+
\fr{\bm{A}_{2,j-1} -2\bm{A}_{2,j} +\bm{A}_{2,j+1}}{h_1h_2}\,;\\
   \bm{B}_{1,v_2}&= \fr{-\bm{A}_{1,v_2-1} +\bm{A}_{1,v_2} +\bm{A}_{2,v_2-1}- \bm{A}_{2,v_2}}{h_1h_2}\,;
  \end{align*}
  
\noindent
  \begin{align*}
   \bm{B}_{i,1}&= \fr{ -\bm{A}_{i-1,1} +\bm{A}_{i-1,2} +2\bm{A}_{i,1}}{h_1h_2}+{}\\
  &\hspace*{15mm}{}+   \fr{ -2\bm{A}_{i,2} -\bm{A}_{i+1,1} +\bm{A}_{i+1,2}}{h_1h_2}+{}\\
  &\hspace*{25mm}{}+ \fr{ \bm{A}_{i-1,\cdot } - 2\bm{A}_{i,\cdot} +\bm{A}_{i+1,\cdot}}{h_1}\,;\\
  \bm{B}_{i,j} &=\fr{\bm{A}_{i-1,j-1} -2\bm{A}_{i-1,j} +\bm{A}_{i-1,j+1}}{h_1h_2}+{}\\
  &\hspace*{11mm}{}+ \fr{-2\bm{A}_{i,j-1} +4\bm{A}_{i,j} -2\bm{A}_{i,j+1}}{h_1h_2} +{}\\
 &\hspace*{13mm}{}+  \fr{\bm{A}_{i+1,j-1} -2\bm{A}_{i+1,j}+\bm{A}_{i+1,j+1}}{h_1h_2}\,;\\
  \bm{B}_{i,v_2}&= \fr{\bm{A}_{i-1,v_2-1} -\bm{A}_{i-1,v_2} -2\bm{A}_{i,v_2-1}}{h_1h_2} +{}\\
 & \hspace*{16mm}{}+\fr{2\bm{A}_{i,v_2} +\bm{A}_{i+1,v_2-1} -\bm{A}_{i+1,v_2}}{h_1h_2}\,;\\
  \bm{B}_{v_1,1} &=\bm{U} +\fr{ -\bm{A}_{\cdot,1} +\bm{A}_{\cdot, 2}}{h_2}+ {}\\
 & \hspace*{2mm}{}+\fr{-\bm{A}_{v_1-1,1} +\bm{A}_{v_1-1,2} +\bm{A}_{v_1,1} -\bm{A}_{v_1,2}}{h_1h_2} +{}\\
&\hspace*{37mm}{}+\fr{\bm{A}_{v_1-1,\cdot}- \bm{A}_{v_1,\cdot}}{h_1}\,;\\
  \bm{B}_{v_1,j}&= \fr{\bm{A}_{\cdot,j-1}- 2\bm{A}_{\cdot,j} +\bm{A}_{\cdot,j+1}}{h_2} +{}\\
&\hspace*{2mm}{}+\fr{\bm{A}_{v_1-1,j-1}-2\bm{A}_{v_1-1,j}+\bm{A}_{v_1-1,j+1}}{h_1h_2}+{}\\
&\hspace*{15mm}{}+\fr{ -\bm{A}_{v_1,j-1}+2\bm{A}_{v_1,j}-\bm{A}_{v_1,j+1}}{h_1h_1}\,;\\
  \bm{B}_{v_1,v_2}&= \fr{\bm{A}_{\cdot,v_2-1}- \bm{A}_{\cdot,v_2}}{h_2} 
+{}\\
&\hspace*{-7mm}{}+\fr{\bm{A}_{v_1-1, v_2-1}- \bm{A}_{v_1-1,v_2} - \bm{A}_{v_1,v_2-1} 
+\bm{A}_{v_1,v_2}}{h_1h_2}\,.
  \end{align*}
    Здесь в~индексах опционов маркер <<точка>> отмечает координату 
безрискового актива, а~под~$\bm{A}_{i,\cdot}$ и~$\bm{A}_{\cdot,j}$, как уже обсуждалось 
выше, понимаются двумерные инструменты $\bm{A}_i\times \bm{U}_2$ 
и~$\bm{U}_1\times \bm{A}_j$ соответственно. 
  
  \textit{Смешанный} базис состоит из $5^2\hm=25$ качественно различных 
вариантов представления базисных инструментов, поскольку для каждой 
компоненты рынка вариантов страйка пять: два крайних, один центральный 
и~два внутренних, ниже и~выше центра. Их перечень получается применением 
правил~(\ref{e16-aga}). Приводим лишь часть базиса, связанную с~первым по 
отношению к~центру рынка квадрантом, т.\,е.\ для $1 \hm\leq i \hm\leq i_c$, $1 
\hm\leq j \hm\leq j_c$ (прочие части образуются аналогично):
  \begin{align*}
  \bm{B}_{1,1}&=\fr{\bm{A}_{00;1,1} - \bm{A}_{00;1,2} - \bm{A}_{00;2,1} + 
\bm{A}_{00;2,2}}{h_1h_2}\,; 
\end{align*}

  \noindent
  \begin{align*}
  \bm{B}_{1,j}&=\fr{-\bm{A}_{00;1,j-1} + 2\bm{A}_{00;1,j} - \bm{A}_{00;1,j+1}}{h_1h_2} + {}\\
&\hspace*{-3mm}{}+
\fr{\bm{A}_{00;2,j-1} - 2\bm{A}_{00;2,j} + \bm{A}_{00;2,j+1}}{h_1h_2}\,,\enskip 0<j<j_c\,; \\
  \bm{B}_{1,j_c}&=\fr{- \bm{A}_{00;1,j_c-1} + \bm{A}_{00;1,j_c} + \bm{A}_{00;2,j_c - 1}}{h_1h_2} +{}\\
  &\hspace*{-3mm}{}+  
  \fr{- \bm{A}_{00;2,j_c} + \bm{A}_{01;1,j_c} - \bm{A}_{01;1,j_c+1} - \bm{A}_{01;2,j_c} }{h_1h_2}+{}\\
&\hspace*{21mm}{}+ \fr{\bm{A}_{01;2,j_c+1}}{h_1h_2} + \fr{- \bm{A}_{0;1,\cdot} + \bm{A}_{0;2,\cdot}}{h_1}\,; \\
  \bm{B}_{i,1}&=\fr{-\bm{A}_{00;i-1,1} + \bm{A}_{00;i-1,2} + 2\bm{A}_{00;i,1}}{h_1h_2}+{}\\
  &\hspace*{-4mm}{}+ \fr{ -  2\bm{A}_{00;i,2} - \bm{A}_{00;i+1,1} + \bm{A}_{00;i+1,2}}{h_1h_2}\,,\enskip   0<i<i_c\,; \\
  \bm{B}_{i,j}&= \fr{\bm{A}_{00;i-1,j-1} - 2\bm{A}_{00;i-1,j} + \bm{A}_{00;i-1,j+1}}{h_1h_2}+{}\\
  &\hspace*{2mm}{}+ \fr{- 2\bm{A}_{00;i,j-1} + 4\bm{A}_{00;i,j} - 2\bm{A}_{00;i,j+1}}{h_1h_2} +{}\\
&\hspace*{3mm}{}+\fr{\bm{A}_{00;i+1,j-1} - 2\bm{A}_{00;i+1,j} + \bm{A}_{00;i+1,j+1}}{h_1h_2}\,,\\
    & \hspace*{35mm}0<i<i_c,\enskip  0<j<j_c\,; \\
  \bm{B}_{i,j_c}&=\fr{\bm{A}_{00;i-1,j_c-1} - \bm{A}_{00;i-1,j_c} - 2\bm{A}_{00;i,j_c-1}}{h_1h_2} +{} \\
 &\hspace*{1mm}{}+\fr{2\bm{A}_{00;i,j_c} + \bm{A}_{00;i+1,j_c-1} - \bm{A}_{00;i+1,j_c}}{h_1h_2}+{}\\
 &\hspace*{2mm}{}+ \fr{ -\bm{A}_{01;i-1,j_c} + \bm{A}_{01;i-1,j_c+1} + 2\bm{A}_{01;i,j_c}}{h_1h_2}+{}\\
& {}+\fr{ - 2\bm{A}_{01;i,j_c+1} - \bm{A}_{01;i+1,j_c} + \bm{A}_{01;i+1,j_c+1}}{h_1h_2} +{}\\
&\hspace*{4mm}{}+ \fr{\bm{A}_{0;i-1,\cdot} - 2\bm{A}_{0;i,\cdot} + \bm{A}_{0;i+1,\cdot}}{h_1},\enskip    0<i<i_c\,; \\
  \bm{B}_{i_c,1}&=\fr{-\bm{A}_{00;i_c-1,1} + \bm{A}_{00;i_c-1,2} + \bm{A}_{00;i_c,1}}{h_1h_2}+{}\\
  &\hspace*{5mm}{}+ \fr{ -\bm{A}_{00; i_c,2} + \bm{A}_{10;i_c,1} - \bm{A}_{10;i_c,2}}{h_1 h_2}+{}\\
  &\hspace*{-2mm}{}+\fr{ - \bm{A}_{10;i_c+1,1} +\bm{A}_{10;i_c+1,2}}{h_1h_2} +   \fr{- \bm{A}_{0;\cdot,1} + \bm{A}_{0;\cdot,2}}{h_2}\,; \\
   \bm{B}_{i_c,j}&=\fr{\bm{A}_{00;i_c-1,j-1} - 2\bm{A}_{00;i_c-1,j} + \bm{A}_{00;i_c-1,j+1}}{h_1h_2} + {}\\
   &\hspace*{2mm}{}+   \fr{-\bm{A}_{00;i_c,j-1} + 2\bm{A}_{00;i_c,j} - \bm{A}_{00;i_c,j+1}}{h_1h_2}+{}\\
   &\hspace*{3mm}{}+ \fr{ - \bm{A}_{10;i_c,j-1} + 2\bm{A}_{10;i_c,j} - \bm{A}_{10;i_c,j+1}}{h_1h_2} +{}\\
  &\hspace*{-1mm} {}+\fr{ \bm{A}_{10;i_c+1,j-1} - 2\bm{A}_{10;i_c+1,j} + \bm{A}_{10;i_c+1,j+1}}{h_1h_2} +{}\\
&\hspace*{1mm}{}+ \fr{\bm{A}_{0;\cdot,j-1} - 2\bm{A}_{0;\cdot,j} + \bm{A}_{0;\cdot,j+1}}{h_2}\,,\enskip    0<j<j_c\,; 
 \end{align*}
  
 \noindent
  \begin{align*}
  \bm{B}_{i_c,j_c}& =1 + 
  \fr{\bm{A}_{00;i_c-1,j_c-1} - \bm{A}_{00;i_c-1,j_c}}{h_1h_2}+{}\\[2pt]
  &{}+\fr{ - \bm{A}_{00;i_c,j_c-1} + 
\bm{A}_{00;i_c,j_c} - \bm{A}_{01;i_c-1,j_c}}{h_1h_2} +{}\\[2pt]
&{}+ \fr{\bm{A}_{01;i_c-1,j_c+1} + \bm{A}_{01;i_c,j_c} - \bm{A}_{01;i_c,j_c+1}}{h_1h_2}+{}\\[2pt]
&{}+ \fr{ - \bm{A}_{10;i_c,j_c-1} + \bm{A}_{10;i_c,j_c} + \bm{A}_{10;i_c+1,j_c-1}}{h_1h_2}+{}\\[2pt]
&{}+ \fr{ -\bm{A}_{10;i_c+1,j_c} + \bm{A}_{11;i_c,j_c} - \bm{A}_{11;i_c,j_c+1}}{h_1h_2}+{}\\[2pt]
&{}+\fr{ - \bm{A}_{11;i_c+1,j_c} + \bm{A}_{11;i_c+1,j_c+1}}{h_1h_2} + {}\\[2pt]
&{}+\fr{\bm{A}_{0;i_c-1,\cdot} - \bm{A}_{0;i_c, \cdot} - \bm{A}_{1;i_c,\cdot} + \bm{A}_{1;i_c+1,\cdot}}{h_1} + {}\\[2pt]
&\hspace*{3mm}{}+\fr{\bm{A}_{0;\cdot,j_c-1} - \bm{A}_{0;\cdot,j_c} - \bm{A}_{1;\cdot,j_c} + \bm{A}_{1;\cdot,j_c+1}}{h_2}\,. 
  \end{align*}
  %
  В этом списке присутствуют обозначения инструментов~$\bm{A}$ 
с~четырьмя и~тремя индексами. В~первой группе пара индексов до точки 
с~запятой означает тип двумерного $\alpha$-оп\-ци\-она~(\ref{e18-aga}), а~после 
нее~--- его страйк. Во второй группе представлены одномерные версии 
двумерных $\alpha$-оп\-ци\-онов. Индекс до точки с~запятой означает тип опциона, 
числовой индекс после нее~--- его страйк, а~позиция маркера <<точка>> 
показывает координату безрискового актива. 
  
  Сценарные баттерфляи, полученные из $\alpha$-оп\-ци\-онов, позволяют 
произвольный инструмент на рынке представить в~виде портфеля  
$\alpha$-оп\-ци\-онов. Для нахождения его доходов следует воспользоваться 
соотношениями~(\ref{e2-aga}) с~учетом переопределения~(\ref{e8-aga}). Так, 
в~однотипном случае платежная функция портфеля находится в~соответствии 
с~(\ref{e17-aga}) по правилам: 
  \begin{multline}
  \bm{U}\to 1\,,\enskip \bm{A}_{ij}\to \omega_{1;i}(x)\omega_{2;j}(y)\,,\\[2pt] 
\bm{A}_{i\cdot}\to \omega_{1;i}(x)\,,\enskip 
 \bm{A}_{\cdot j}\to \omega_{2;j}(y)\,;
  \label{e19-aga}
  \end{multline}
  
  \vspace*{-12pt}
  
  \noindent
  \begin{multline*}
  \omega_{\beta_1;i\cdot}(x)=\max \left( 0,\alpha_1(x-s_i)\right)\,,\
  i\in \bm{I};\\[2pt]
   \omega_{2;i}(y)=\max \left( 0,\alpha_2(y-t_j)\right), \enskip j\in \bm{J}\,.
 % \label{e20-aga}
  \end{multline*}
  
  Аналогично в~соответствии с~(\ref{e18-aga}) записываются в~смешанном 
случае правила формирования платежных функций: 
  \begin{multline*}
  \bm{U}\to 1\,,\enskip \bm{A}_{\beta;ij}\to \omega_{\beta_1,1;i}(x)\omega_{\beta_2,2;j}(y),\\[2pt]
  \bm{A}_{\beta_1;i,\cdot}\to \omega_{\beta_1,1;i}(x),\enskip
  \bm{A}_{\beta_2;\cdot,j}\to \omega_{\beta_2,2;j}(y)\,;
  \end{multline*}
  
  \vspace*{-12pt}
  
  \noindent
  \begin{multline}
  \omega_{\beta_1,1;i}(x)=\max \left( 0,\alpha_1(x-s_i)\right),\enskip i\in \bm{I},\\[2pt]
  \omega_{\beta_2,2;j}(y)=\max \left( 0,\alpha_2(y-t_j)\right),\enskip j\in\bm{J}.
  \label{e21-aga}
  \end{multline}
  
  \section{Иллюстративный пример}
  
  \vspace*{-1pt}
  
  Для построения мер ${\mathsf C}\{\cdot\}$ и~${\mathsf P}\{\cdot\}$ и~их 
сравнительного анализа данные в~примере заимствуются из~[6]. Так, 
принимается ${\mathsf X}\hm=[0,1)$, ${\mathsf Y}\hm=[0,1)$, а~для ${\mathsf 
F}_{\mathsf {CX}}(x)$ и~${\mathsf F}_{\mathsf{CY}}(y)$ выбираются  
бе\-та-рас\-пре\-де\-ле\-ния с~па\-ра\-мет\-ра\-ми $\{3/2,2\}$ и~$\{3/2,3\}$ 
соответственно, для ${\mathsf F}_{\mathsf PX}(x)$ и~${\mathsf F}_{\mathsf PY}(y)$~--- $\{2,3\}$ и~$\{2,4\}$:

\vspace*{-2pt}

\noindent
  \begin{align*}
  {\mathsf F}_{\mathsf {CX}} (x) &= \fr{x^{3/2}(5-3x)}{2}\,;\\  
  {\mathsf F}_{\mathsf {CY}}(y)&= \fr{y^{3/2}(35-42y+15y^2)}{8}\,;\\
  {\mathsf F}_{\mathsf {PX}}(x) &= x^2\left(6-8x+3x^2\right);\\
  {\mathsf F}_{\mathsf {PY}}(y)&= y^2\left( 10-20y+15y^2-4y^3\right).
  \end{align*}
  %
  
  \vspace*{-2pt}
  
  \noindent
  Из них совместные функции распределения для обеих мер строятся как
  
  \vspace*{-5pt}
  
  \noindent 
  \begin{multline}
  {\mathsf F}(x,y)={}\\
\hspace*{-3mm}  {}= {\mathsf F}_{\mathsf X}(x) {\mathsf F}_{\mathsf Y}(y) \left( 
1+3\kappa \left( 1-{\mathsf F}_{\mathsf X}(x)\right) \left(1-{\mathsf F}_{\mathsf 
Y}(y)\right)\right).
  \label{e22-aga}
  \end{multline}
  
\vspace*{-1pt}
 
  Искомые двумерные функции распределения ${\mathsf F}_{\mathsf C}(x,y)$ 
и~${\mathsf F}_{\mathsf P} (x,y)$ определяются подстановкой в~(\ref{e22-aga}) 
в~качестве параметра, отвечающего за корреляционную связь компонент, 
соответственно $\kappa_c\hm=0$ и~$\kappa_p\hm=0{,}2$. Из них простым 
смешанным дифференцированием по обеим переменным находятся плотности 
$c(x,y)$ и~$p(x,y)$, но ввиду громоздкости записей они здесь не приводятся. 
  
  Двумерная дискретизация множества ${\mathsf X}\times \mathsf{Y}$ 
в~примере проводится также при $v_1\hm=6$ и~$v_2\hm=5$, а~центральным 
выбирается страйк $i_c\hm=3$, $j_c\hm=3$. 
  
  В отличие от $\zeta$-рын\-ков~[6], для которых в~целях применения 
дискретного алгоритма находились стоимости сценарных индикаторов, для  
\mbox{$\alpha$-рын}\-ков следует вычислять стоимости сценарных баттерфляев. И~потому 
для адекватного сравнения относительных доходов естественно вычислять и~их 
средние доходы. И~те и~другие, а~это векторы $\bm{c}^B$ и~$\bm{p}^B$, 
определяются интегрированием платежных функций~(\ref{e4-aga})  
и~(\ref{e6-aga}) с~плотностями $c(x,y)$ и~$p(x,y)$ соответственно. 
  
  Применением к~этим векторам дискретного алгоритма оптимизации~[6], 
основанного на процедуре Ней\-ма\-на--Пир\-со\-на~\cite{7-aga}, определяется 
вектор весов базисных баттерфляев для оптимального двумерного портфеля. 
При этом в~качестве функции рисковых предпочтений выбирается 
$\varphi(\varepsilon)\hm=\varepsilon^2$, $\varepsilon\hm\in [0,1]$.

\begin{figure*} %fig1
\vspace*{1pt}
\begin{minipage}[t]{80mm}
\begin{center}
   \mbox{%
\epsfxsize=77.328mm
\epsfbox{aga-1.eps}
}
\end{center}
\vspace*{-9pt}
\Caption{Доходы оптимального опционного портфеля при дискретизации $6\times5$}
\end{minipage}
%\end{figure*}
\hfill
%\begin{figure*} %fig2
\vspace*{1pt}
\begin{minipage}[t]{80mm}
\begin{center}
   \mbox{%
\epsfxsize=77.328mm
\epsfbox{aga-2.eps}
}
\end{center}
\vspace*{-9pt}
\Caption{Доходы сценарного портфеля при дискретизации $40\times40$}
\end{minipage}
\end{figure*}

  
  В результате  получается вектор весов 
  
  \vspace*{-4pt}
  
  \noindent
 \begin{multline*}
  \bm{g}=\{0{,}118; 0{,}159; 0{,}0113; 0{,}000219; 0{,}000008; 0{,}414;\\
   1{,}0; 0{,}228; 0{,}0151; 0{,}000989; 0{,}0739; 0{,}788; 0{,}602;\\
      0{,}0873; 0{,}00175; 0{,}0069; 0{,}309; 0{,}495; 0{,}176; 0{,}00191;
\end{multline*}

%\vspace*{-3pt}

%\pagebreak

\noindent
 \begin{multline*}
   \hspace*{-4.8848pt}0{,}000907; 0{,}0254; 0{,}0405; 0{,}0291; 0{,}00159; 0{,}0000058;\\
   0{,}0000848; 0{,}00151; 0{,}00112;  0{,}000009\}. 
  \end{multline*}
  
  \vspace*{-2pt}

  
  Он порождает оптимальный портфель~(\ref{e12-aga}) с~инвестиционной 
суммой, средним доходом и~средней доходностью соответственно 
  $A\hm=0{,}28642$,  $R\hm=0{,}364418$  и~$y\hm=0{,}272317$. 
      График его платежной функции изображен на рис.~1. Для сравнения на 
рис.~2 приведен аналогичный график для сценарного рынка при дискретизации 
$40\times40$.
  

  По понятным причинам графики платежных функций на рис.~1 и~2 
демонстрируют большее взаимное сходство, чем аналогичная пара графиков 
из~[6]. Но и~различие между собой графиков на рис.~1 и~2 пред\-став\-ля\-ет\-ся 
естественным. 
  
  Остается определить оптимальные портфели в~терминах $\alpha$-оп\-ци\-онов 
рас\-смат\-ри\-ва\-емых типов. Для нахождения каждого такого пред\-став\-ле\-ния 
в~формулу~(\ref{e12-aga}) следует для всех пар $(i,j)$ под\-став\-лять вмес\-то 
индикаторов~$\bm{B}_{ij}$ со\-от\-вет\-ст\-ву\-ющие им пред\-став\-ле\-ния  
в~$\alpha$-оп\-ци\-онах. В~результате после упрощений получаются четыре 
однотипных портфеля и~один смешанный. Так, оптимальный 
\textit{однотипный} портфель для $\bm{\alpha}\hm= \{-1, +1\}$, т.\,е.\ образованный 
путами для первого актива и~коллами~--- для второго: 

\vspace*{-2pt}

\noindent
 \begin{multline*}
  \bm{G}_{01}=0{,}000006 \bm{U} + 16{,}369\bm{A}_{11} - 35{,}107\bm{A}_{12} + {}\\
  {}+12{,}681\bm{A}_{13} + 5{,}641\bm{A}_{14} + 0{,}416\bm{A}_{15} + 1{,}774\bm{A}_{1\cdot} - {}\\
{}-12{,}549\bm{A}_{21} + 48{,}875\bm{A}_{22} - 39{,}318\bm{A}_{23} + 1{,}263\bm{A}_{24} + {}\\
{}+ 1{,}729\bm{A}_{25} - 3{,}812\bm{A}_{2\cdot} - 16{,}162\bm{A}_{31} + 9{,}717\bm{A}_{32} + {}\\
{}+21{,}364\bm{A}_{33} - 15{,}431\bm{A}_{34} + 0{,}512\bm{A}_{35} + 1{,}636\bm{A}_{3\cdot} + {}\\
{}+ 4{,}007\bm{A}_{41} - 20{,}268\bm{A}_{42} + 19{,}612\bm{A}_{43} + 3{,}707\bm{A}_{44} - {}\\
{}-7{,}058\bm{A}_{45} + 0{,}366\bm{A}_{4\cdot} + 7{,}603\bm{A}_{51} - 2{,}899\bm{A}_{52} - {}\\
{}- 13{,}593\bm{A}_{53} + 5{,}279\bm{A}_{54} + 3{,}609\bm{A}_{55} + 0,031\bm{A}_{5\cdot} + {}\\
{}+ 0{,}731\bm{A}_{61} - 0{,}318\bm{A}_{62} - 0{,}745\bm{A}_{63} - 0{,}458\bm{A}_{64} + {}
\end{multline*}

\noindent
\begin{multline*}
{}+0{,}791\bm{A}_{65} + 0{,}005\bm{A}_{6\cdot} + 0{,}0004\bm{A}_{\cdot1} + 0{,}007\bm{A}_{\cdot 2} -{}\\
{}- 0{,}009\bm{A}_{\cdot 3} - 0{,}004\bm{A}_{\cdot 4} + 0{,}006\bm{A}_{\cdot 5}.
\end{multline*}

\vspace*{-2pt}


  
  Платежные функции всех однотипных портфелей получаются по 
правилам~(\ref{e19-aga}). Проведенные расчеты под\-тверж\-да\-ют вер\-ность 
алгоритма. Все они, несмотря на внешнее различие их пред\-став\-ле\-ний, на 
идеальном рынке должны иметь единую платежную функцию с~графиком, 
пред\-став\-лен\-ным на рис.~1. 
  
  Оптимальный \textit{смешанный портфель} строится вновь по 
формуле~(\ref{e12-aga}), но в~смешанном базисе\linebreak естественного происхождения с~выделенным 
цент\-раль\-ным страйком~$(3, 3)$. В~первом квад\-ран\-те\linebreak 
(относительно цент\-ра рынка) используются $\alpha$-оп\-ци\-оны~$\bm{A}_{11}$, во 
втором~--- $\bm{A}_{01}$, в~треть\-ем~--- $\bm{A}_{00}$, в~\mbox{чет\-вер\-том}~--- 
$\bm{A}_{10}$. 
  
  Вычисления с~применением~(\ref{e12-aga}) дают оптимальный 
\textit{смешанный} портфель:

\vspace*{-2pt}
 

\noindent
\begin{multline*}
  \bm{G}_m=0{,}602\bm{U} + 16{,}369\bm{A}_{00;11} - 35{,}107\bm{A}_{00;12} + {}\\
  {}+
18{,}737\bm{A}_{00;13} - 12{,}549\bm{A}_{00;21} + 48{,}876\bm{A}_{00;22} - {}\\
{}-
36{,}326\bm{A}_{00;23} - 3{,}82\bm{A}_{00;31} - 13{,}769\bm{A}_{00;32} +{}\\
{}+ 17{,}589\bm{A}_{00;33} - 
6{,}056\bm{A}_{01;13} + 5{,}641\bm{A}_{01;14} +{}\\
{}+ 0{,}416\bm{A}_{01;15} - 2{,}992\bm{A}_{01;23} + 
1{,}263\bm{A}_{01;24} + {}\\
{}+1{,}729\bm{A}_{01;25} + 9{,}049\bm{A}_{01;33} - 6{,}903\bm{A}_{01;34} - {}\\
{}-2{,}145\bm{A}_{01;35} - 12{,}342\bm{A}_{10;31} + 23{,}486\bm{A}_{10;32} - {}\\
{}-11{,}144\bm{A}_{10;33} + 4{,}007\bm{A}_{10;41} - 20{,}268\bm{A}_{10;42} + {}\\
{}+16{,}261\bm{A}_{10;43} + 7{,}603\bm{A}_{10;51} - 2{,}899\bm{A}_{10;52} -{}\\
{}-4{,}704\bm{A}_{10;53} +  0{,}731\bm{A}_{10;61} - 0{,}318\bm{A}_{10;62} - {}\\
{}-0{,}413\bm{A}_{10;63} + 5{,}871\bm{A}_{11;33} - 8{,}528\bm{A}_{11;34} +{}\\
{}+ 2{,}657\bm{A}_{11;35} + 3{,}351\bm{A}_{11;43} + 3{,}707\bm{A}_{11;44} - {}\\
{}-7{,}058\bm{A}_{11;45} - 8{,}889\bm{A}_{11;53} + 5{,}279\bm{A}_{11;54} +{}\\
{}+ 3{,}609\bm{A}_{11;55} - 0{,}333\bm{A}_{11;63} - 0{,}458\bm{A}_{11;64} +{}
\end{multline*}

\noindent
\begin{multline*}
{}+ 0{,}791\bm{A}_{11;65} + 1{,}3\bm{A}_{0;1\cdot} + 0{,}943\bm{A}_{0;2\cdot} - 2{,}243\bm{A}_{0;3\cdot} +{}\\
{}+ 3{,}568\bm{A}_{0;\cdot 1} - 4{,}497\bm{A}_{0;\cdot 2} + 0{,}929\bm{A}_{0;\cdot 3} - 0{,}642\bm{A}_{1;3\cdot} - {}\\
{}-
2{,}085\bm{A}_{1;4\cdot} + 2{,}492\bm{A}_{1;5\cdot} + 0{,}234\bm{A}_{1;6\cdot} - 
2{,}573\bm{A}_{1;\cdot 3} +{}\\
{}+ 2{,}145\bm{A}_{1;\cdot 4} + 0{,}428\bm{A}_{1;\cdot 5}. 
  \end{multline*}
  
  \vspace*{-2pt}
  
\noindent
  В этом портфеле 56~инструментов; среди них один безрисковый актив, 
42~двумерных $\alpha$-оп\-ци\-она~$\bm{A}_{00}$, $\bm{A}_{01}$, $\bm{A}_{10}$ 
и~$\bm{A}_{11}$ и~13~одномерных версий~$\bm{A}_0, \bm{A}_1$ в~количествах 
9, 9, 12, 12 и~6, 7 соответственно. 
  
  Двумерные $\alpha$-оп\-ци\-оны снабжены четырьмя индексами; первые два из 
них (до точки с~запятой) показывают тип опциона по каждой координате 
в~терминах~$\beta$, другие два индекса~--- номера страй\-ков. 
  
  Одномерные версии снабжены двумя индексами и~маркером <<точка>>. 
Один индекс до точки с~запятой указывает тип опциона, индекс после нее~---  
номер страй\-ка, а~маркер~--- координату под\-ра\-зу\-ме\-ва\-емо\-го безрискового 
актива.
  
  Платежная функция смешанного портфеля строится по  
правилам~(\ref{e21-aga}). Ее графиком служит все тот же график на рис.~1. 
  
  \section{Заключение }
  
  В работе решена задача алгоритмического на\-хож\-де\-ния пред\-став\-ле\-ний 
многомерных баттерфляев при сценарной дискретизации рынка и~\mbox{по\-стро\-ения} 
из них базиса в~терминах $\alpha$-оп\-ци\-онов~--- многомерного обобщения 
традиционных опционов колл и~пут. На конкретном примере двумерного рынка 
продемонстрирована работа этого алгоритма и~ее результат. Подобные расчеты 
могут быть без принципиальных трудностей реализованы и~для рынков 
большей размерности. Поскольку бат\-тер\-фляи для одномерного рынка 
образуются из трех страй\-ков, а~индикаторы~--- из двух, то их многомерные 
реп\-ли\-ка\-ции получаются еще более громоздкими. Фактические расчеты, 
проведенные для $n\hm=4$, показали, что уже объем перечня базисных 
инструментов однотипного базиса, например в~терминах~$\bm{A}_{0000}$, 
аналогичного~$\bm{A}_{00}$ из разд.~4, был соразмерен всему объему 
на\-сто\-ящей работы, а~потому и~на многомерных $\alpha$-рын\-ках лучше торговать 
непосредственно бат\-тер\-фля\-ями, а~не $\alpha$-оп\-ци\-онами. 
  
{\small\frenchspacing
 {%\baselineskip=10.8pt
 %\addcontentsline{toc}{section}{References}
 \begin{thebibliography}{9}
  \bibitem{1-aga}
  \Au{Агасандян~Г.\,А.} Применение континуального критерия VaR на 
финансовых рынках.~--- М.: ВЦ РАН, 2011. 299~с. 
  \bibitem{2-aga}
  \Au{Агасандян~Г.\,А.} Континуальный критерий VaR и~оптимальный 
портфель инвестора~// Управ\-ле\-ние большими сис\-те\-ма\-ми, 2018. Вып.~73. 
С.~6--26.
  \bibitem{3-aga}
  \Au{Агасандян~Г.\,А.} Континуальный критерий VaR на сценарных рынках~// 
Информатика и~её применения, 2018. Т.~12. Вып.~1. С.~32--40. 
  \bibitem{4-aga}
  \Au{Агасандян~Г.\,А.} Вычисление показателей оптимальных по CC-VaR 
портфелей на рынках опционов~// Информатика и~её применения, 2019. Т.~13. 
Вып.~3. С.~75--84. 
  \bibitem{5-aga}
  \Au{Агасандян~Г.\,А.} Многомерные рынки опционов и~оптимизация по  
CC-VaR~// Управ\-ле\-ние большими сис\-те\-ма\-ми, 2020. Вып.~88. С.~5--25.
  \bibitem{6-aga}
  \Au{Агасандян~Г.\,А.} Многомерные бинарные рынки и~CC-VaR~// 
Информатика и~её применения, 2022. Т.~16. Вып.~2. С.~2--10.
  \bibitem{7-aga}
  \Au{Крамер~Г.} Математические методы статистики~/ Пер. с~англ.~--- М.: 
Мир, 1975. 750~с. (\Au{Cramer~H.} Mathematical methods of statistics.~--- 
Princeton, NJ, USA: Princeton University Press, 1946. 575~p.)
\end{thebibliography}

 }
 }

\end{multicols}

\vspace*{-6pt}

\hfill{\small\textit{Поступила в~редакцию 09.03.22}}

\vspace*{8pt}

%\pagebreak

%\newpage

%\vspace*{-28pt}

\hrule

\vspace*{2pt}

\hrule

%\vspace*{-2pt}

\def\tit{MULTIDIMENSIONAL BUTTERFLIES IN~PROBLEMS 
OF~OPTIMIZATION ON CC-VaR}


\def\titkol{Multidimensional butterflies in~problems 
of~optimization on CC-VaR}


\def\aut{G.\,A.~Agasandyan}

\def\autkol{G.\,A.~Agasandyan}

\titel{\tit}{\aut}{\autkol}{\titkol}

\vspace*{-8pt}


\noindent
Federal Research Center ``Computer Science and Control'' of the Russian Academy 
of Sciences, 44-2~Vavilov Str., Moscow 119333, Russian Federation


\def\leftfootline{\small{\textbf{\thepage}
\hfill INFORMATIKA I EE PRIMENENIYA~--- INFORMATICS AND
APPLICATIONS\ \ \ 2023\ \ \ volume~17\ \ \ issue\ 1}
}%
 \def\rightfootline{\small{INFORMATIKA I EE PRIMENENIYA~---
INFORMATICS AND APPLICATIONS\ \ \ 2023\ \ \ volume~17\ \ \ issue\ 1
\hfill \textbf{\thepage}}}

\vspace*{3pt} 
  
  
  


  
  \Abste{The work continues studying problems of using continuous VaR-criterion (CC-VaR) in 
financial markets. Again some technical problems are concerned. However, they emerge this time 
not in multidimensional relatively simple binary markets but in multidimensional markets that are 
an extension of one-dimensional traditional
markets of options such as calls and puts. In assumption 
that scenario butterflies are not traded in markets directly, a~method of receiving their replication 
from multidimensional options, i.\,e., $\alpha$-options, is developed. It is based on options parity 
theorems and can be applied to markets of arbitrary dimension, but actual realization is conducted
for two-dimensional markets. The bases constructions in terms of $\alpha$-options both one-type and 
natural mixed with\linebreak\vspace*{-12pt}}

\Abstend{selected market center are produced. Theoretical representations of optimal 
portfolios in these bases accompanied with the payoffs diagram are illustrated by the distinctive 
example of a two-dimensional market.}
  
  \KWE{underliers; multidimensional market; investor's risk preferences function; continuous 
VaR-criterion; cost and forecast densities; scenario indicators; bases; binary options; one-type 
portfolio; market center; mixed portfolio}
  
 \DOI{10.14357/19922264230114} 

%\vspace*{-16pt}

%\Ack
%\noindent

  

%\vspace*{4pt}

  \begin{multicols}{2}

\renewcommand{\bibname}{\protect\rmfamily References}
%\renewcommand{\bibname}{\large\protect\rm References}

{\small\frenchspacing
 {%\baselineskip=10.8pt
 \addcontentsline{toc}{section}{References}
 \begin{thebibliography}{9} 
  \bibitem{1-aga-1}
  \Aue{Agasandyan, G.\,A.} 2011. \textit{Pri\-me\-ne\-nie kon\-ti\-nu\-al'\-no\-go kri\-te\-riya VaR 
na fi\-nan\-so\-vykh ryn\-kakh} [Application of continuous VaR-criterion in financial 
markets]. Moscow: CCRAS. 299~p.
  \bibitem{2-aga-1}
  \Aue{Agasandyan, G.\,A.} 2018. Kon\-ti\-nu\-al'\-nyy kri\-te\-riy VaR i~op\-ti\-mal'\-nyy 
port\-fel' in\-ves\-to\-ra [Continuous VaR-criterion and investor's optimal portfolio]. 
\textit{Upravlenie bol'shimi sistemami} [Large-Scale Systems Control] 73:6--26.
  \bibitem{3-aga-1}
  \Aue{Agasandyan, G.\,A.} 2018. Kon\-ti\-nu\-al'\-nyy kri\-te\-riy VaR na stse\-nar\-nykh 
ryn\-kakh [Continuous VaR-criterion in scenario markets]. \textit{Informatika i~ee 
Primeneniya~--- Inform. Appl.} 12(1):32--40.
  \bibitem{4-aga-1}
  \Aue{Agasandyan, G.\,A.} 2019. Vy\-chis\-le\-nie po\-ka\-za\-te\-ley op\-ti\-mal'\-nykh po  
CC-VaR port\-fe\-ley na ryn\-kakh op\-tsi\-o\-nov [Performance estimations for  
optimal-on-CC-VaR portfolios in option markets]. \textit{Informatika i~ee 
Primeneniya~--- Inform. Appl.} 13(3):75--84.
  \bibitem{5-aga-1}
  \Aue{Agasandyan, G.\,A.} 2020. Mno\-go\-mer\-nye ryn\-ki op\-tsi\-o\-nov i~op\-ti\-mi\-za\-tsiya 
po CC-VaR [Multidimensional option markets and optimization on CC-VaR]. 
\textit{Upravlenie bol'shimi sistemami} [Large-Scale Systems Control] 88:5--25.
  \bibitem{6-aga-1}
  \Aue{Agasandyan, G.\,A.} 2022. Mno\-go\-mer\-nye bi\-nar\-nye ryn\-ki i~CC-VaR 
[Multidimensional binary markets and CC-VaR]. \textit{Informatika i~ee 
Primeneniya~--- Inform. Appl.} 16(2):2--10.
  \bibitem{7-aga-1}
  \Aue{Cramer, H.} 1946. \textit{Mathematical methods of statistics}. Princeton, 
NJ: Princeton University Press. 575~p.

\end{thebibliography}

 }
 }

\end{multicols}

\vspace*{-6pt}

\hfill{\small\textit{Received March 9, 2022}}

  
  \Contrl
  
  \noindent
  \textbf{Agasandyan Gennady A.} (b.\ 1941)~--- Doctor of Science in physics and 
mathematics, leading scientist, A.\,A.~Dorodnicyn Computing Center, Federal 
Research Center ``Computer Science and Control'' of the Russian Academy of 
Sciences, 40~Vavilov Str., Moscow 119333, Russian Federation; 
\mbox{agasand17@yandex.ru}
  

   
\label{end\stat}

\renewcommand{\bibname}{\protect\rm Литература} 
   