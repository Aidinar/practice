\def\stat{dulin}

\def\tit{МОДЕЛИРОВАНИЕ СТРУКТУРЫ ИНТЕРОПЕРАБЕЛЬНОСТИ СРЕДСТВАМИ СТРУКТУРНОЙ 
СОГЛАСОВАННОСТИ}

\def\titkol{Моделирование структуры интероперабельности средствами структурной 
согласованности}

\def\aut{И.\,Н.~Розенберг$^1$, С.\,К.~Дулин$^2$, Н.\,Г.~Дулина$^3$}

\def\autkol{И.\,Н.~Розенберг, С.\,К.~Дулин, Н.\,Г.~Дулина}

\titel{\tit}{\aut}{\autkol}{\titkol}

\index{Розенберг И.\,Н.}
\index{Дулин С.\,К.}
\index{Дулина Н.\,Г.}
\index{Rozenberg I.\,N.}
\index{Dulin S.\,K.}
\index{Dulina N.\,G.}


%{\renewcommand{\thefootnote}{\fnsymbol{footnote}} \footnotetext[1]
%{Работа выполнена при поддержке Министерства науки и~высшего образования
%Российской федерации, грант №\,075-15-2020-799.}}


\renewcommand{\thefootnote}{\arabic{footnote}}
\footnotetext[1]{Научно-исследовательский и~проектно-конструкторский институт информатизации, автоматизации и~связи 
на железнодорожном транспорте, \mbox{I.Rozenberg@vniias.ru}}
\footnotetext[2]{Федеральный исследовательский центр <<Информатика и~управление>> Российской академии наук;  
На\-уч\-но-ис\-сле\-до\-ва\-тель\-ский и~про\-ект\-но-конст\-рук\-тор\-ский институт информатизации, 
автоматизации и~связи на железнодорожном транспорте, \mbox{skdulin@mail.ru}}
\footnotetext[3]{Федеральный исследовательский центр <<Информатика и~управ\-ле\-ние>> Российской академии наук, 
\mbox{ngdulina@mail.ru}}

\vspace*{-10pt}

      
      \Abst{Начальный синтаксический уровень интероперабельности предполагает 
коммуникацию с~соответствующим протоколом, аппаратные средства, программное 
обеспечение и~необходимый уровень со\-вмес\-ти\-мости данных. Исследованию уровня 
совместимости данных, опи\-сы\-ва\-ющих вза\-имо\-дей\-ст\-ву\-ющие элементы на основе вектора 
признаков, посвящена пред\-став\-лен\-ная работа. Для этого предлагается модель 
структурного соответствия, поз\-во\-ля\-ющая оценивать тенденцию к~установлению 
интероперабельности. Моделирование структурной ин\-тер\-опе\-ра\-бель\-ности на основе 
анализа знаков связей с~по\-мощью введенного критерия согласованности приводит 
к~на\-хож\-де\-нию ближайшего к~исходному множеству консонансного прообраза. 
Найденный консонансный прообраз своими подмножествами указывает на 
предпочтительную группировку элементов, при которой ин\-тер\-опе\-ра\-бель\-ность меж\-ду 
ними устанавливается с~наименьшей рас\-со\-гла\-со\-ван\-ностью относительно 
зафиксированных знаков связей. Поскольку рас\-смат\-ри\-ва\-емые элементы описаны 
вектором па\-ра\-мет\-ров, из сравнения которых мож\-но сделать вывод о~сход\-ст\-ве 
меж\-ду элементами, соответственно, на\-хож\-де\-ние элементов в~одном подмножестве 
говорит о~потенциальной мотивации к~ин\-тер\-опе\-ра\-бель\-ности.}
      
      \KW{интероперабельность; структурная согласованность; мат\-ри\-ца связ\-ности}
      
\DOI{10.14357/19922264230108} 
  
%\vspace*{-4pt}


\vskip 10pt plus 9pt minus 6pt

\thispagestyle{headings}

\begin{multicols}{2}

\label{st\stat}

\section{Введение}
     
     При интеграции и~глобализации информационных сис\-тем ключевым 
фактором становится ин\-тер\-опе\-ра\-бель\-ность как неотъемлемое свойство 
эффективного функционирования синтезированных \mbox{сис\-тем} и~элементов. По 
определению органов стандартизации~\cite{1-dul},  
<<ин\-тер\-опе\-ра\-бел\-ьность~--- спо\-соб\-ность двух или более информационных 
сис\-тем или компонентов к~обмену информацией и~к использованию 
информации, полученной в~результате обмена>>. Эталонная модель 
ин\-тер\-опе\-ра\-бель\-ности по ГОСТ Р~55062-2012 представляет собой 
трехуровневую модель, вклю\-ча\-ющую техническую ин\-тер\-опе\-ра\-бель\-ность, 
семантическую ин\-тер\-опе\-ра\-бель\-ность и~организационную 
ин\-тер\-опе\-ра\-бель\-ность.
     
     В настоящее время в~отечественной литературе обсуж\-да\-ют\-ся, главным 
образом, вопросы реализации технической ин\-тер\-опе\-ра\-бель\-ности. 
Семантический и~организационный уровни ин\-тер\-опе\-ра\-бель\-ности 
информационных сис\-тем обсуждаются только тео\-ре\-ти\-чески. 
     
     В работе~[2] представлена обобщенная модель ин\-тер\-опе\-ра\-бель\-ности, 
разработанная международным консорциумом организаций NCOIC~--- 
``Systems, Capabilities, Operations, Programs, and Enterprises Model for 
Interoperability Assessment'' (SCOPE-мо\-дель), а~в~\cite{3-dul} обоснован 
вариант декомпозиции па\-ра\-мет\-ров SCOPE-мо\-де\-ли и~их приведения 
к~эталонной модели, пред\-став\-лен\-ной в~ГОСТ Р~55062-2012 (рис.~1).



\vspace*{-6pt}

\section{Построение модели структурной согласованности}
     
     Помимо мотивации к~ин\-тер\-опе\-ра\-бель\-ности активных элементов 
существуют определенные признаки или характеристики в~структуре 
элементов,\linebreak которые способствуют или препятствуют достижению 
ин\-тер\-опе\-ра\-бель\-ности. Оценивая потенциальную возможность установления 
ин\-тер\-опе\-ра\-бель\-ности той или иной степени в~структуре\linebreak взаимосвязанных 
элементов, можно говорить о~структурной ин\-тер\-опе\-ра\-бель\-ности. Для 
изучения возможности информационных сис\-тем или элементов обладать 
тенденцией к~ин\-тер\-опе\-ра\-бель\-ности в~за\-ви\-си\-мости от соотнесения 
собственных признаков или характеристик мож\-но предложить некоторую 
модель структурного соответствия, поз\-во\-ля\-ющую оценивать группы 
потенциально близ\-ких друг к~другу элементов по ряду выбранных при-\linebreak\vspace*{-12pt}

\pagebreak

\end{multicols}

\begin{figure*} %fig1
\vspace*{1pt}
\begin{center}
   \mbox{%
\epsfxsize=163mm
\epsfbox{dul-1.eps}
}
\end{center}
\vspace*{-9pt}
\Caption{Общая структура интероперабельности в~соответствии с~ГОСТ Р~55062-2012}
\vspace*{-3pt}
\end{figure*}

\begin{multicols}{2}

\noindent
знаков, 
важ\-ных для установления ин\-тер\-опе\-ра\-бель\-ности. Для этого рас\-смот\-рим 
некоторое множество из $N$ элементов, которые могут быть пред\-став\-ле\-ны 
как агенты, информационные сис\-те\-мы или компоненты знаний. Эти 
элементы оказываются вовлечены во взаимодействие, уровень которого 
требуется оценить на предмет ин\-тер\-опе\-ра\-бель\-ности и~представить структуру 
ин\-тер\-опе\-ра\-бель\-ности на основании анализа предпочтительности 
установления отношений между элементами. Предлагается при\-влечь для 
этого аппарат структурной со\-гла\-со\-ван\-ности~\cite{4-dul}. 
     
     Поставленная задача, таким образом, определяет разбиение множества 
потенциально взаимодействующих элементов на наборы мотивированных 
к~взаимодействию элементов. Рас\-смат\-ри\-ва\-емые\linebreak элементы могут быть 
описаны вектором па\-ра\-мет\-ров (признаков), из срав\-не\-ния которых можно 
\mbox{сделать} вывод о~сходстве между элементами и,~соответственно, 
потенциальной мотивации к~ин\-тер\-опе\-ра\-бель\-ности.
     
     Другими словами, рас\-смат\-ри\-ва\-ет\-ся множество элементов $O\hm = 
\{o_i\}$ ($i \hm= 1, \ldots , N$), где описание каж\-до\-го элемента пред\-став\-ле\-но 
в~виде вектора из $m$ приз\-на\-ков-ат\-ри\-бу\-тов: $o_i \hm= (p_1^i, \ldots, p_m^i)$. 
Сравнивая любые два элемента этого множества на основе этих признаков, 
мож\-но оценить их сходство. Следует отметить, что допустимо использовать 
либо все $m$ признаков, либо подмножество из $k\hm\leq m$ признаков, 
поз\-во\-ля\-ющее дать оцен\-ку для каж\-дой пары элементов.
     
     Поступая стандартным образом, зададим~$F$~--- функцию сходства 
элементов по $k$ признакам, нормированную на максимальный диапазон 
значений признака. Для двух элементов~$o_i$ и~$o_j$, сходство которых 
уста\-нав\-ли\-ва\-ет\-ся на осно\-ве $k$ признаков~$\{p_k\}$, функция~$F$ имеет вид:
     $$
     F(o_i,o_j) =\fr{1}{k}\sum\limits_{i=1}^k  w_{ml}\fr{\vert p^i_{ml}-
p^j_{ml}\vert}{\max \vert p^i_{ml}-p^j_{ml}\vert}\,,
     $$
где $0 \leq w_{ml} \hm\leq 1$~--- вес $ml$-го признака, а $\max \vert p^i_{ml} \hm- 
p_{ml}^j\vert$~--- диапазон значений $ml$-го признака.
     
     Функция~$F$ принимает значения из $[0,1]$, так что $F \hm=1$~--- это 
абсолютное сходство элементов~$o_i$ и~$o_j$, а~$F\hm=0$~--- абсолютное 
различие. Остальные значения функции трак\-ту\-ют\-ся как оциф\-ро\-ван\-ные 
степени сходства пар элементов по~$k$~признакам, позволяя пред\-ста\-вить 
множество как граф со взвешенными связями. 
     
     Далее предлагается перейти к~знаковому графу, поз\-во\-ля\-юще\-му на 
основе анализа со\-гла\-со\-ван\-ности его связей по вы\-бран\-но\-му критерию 
со\-гла\-со\-ван\-ности сгруп\-пи\-ро\-вать $N$ элементов в~$P$~групп. Знаковый граф 
получается в~результате выбора порогового значения~$\alpha$  для 
функции~$F$ так, что, когда $0 \hm\leq F(o_i, o_j) \hm\leq \alpha$, элементы 
$o_i$ и~$o_j$ считаются несходными по $k$ признакам, а~в~случае $\alpha \hm <  
F(o_i, o_j) \hm\leq 1$~--- сходными. Со\-по\-став\-ляя знак минус связям с~$0\hm\ 
\leq F(o_i, o_j) \hm\leq \alpha$ и~знак плюс~--- остальным, получим 
знаковую структуру, пред\-став\-ля\-ющую собой дис\-крет\-ную знаковую модель 
множества потенциально вза\-имо\-дей\-ст\-ву\-ющих элементов. 
     
     Знаковому графу можно поставить в~соответствие мат\-ри\-цу связ\-ности 
     и~сформулировать задачу структурной со\-гла\-со\-ван\-ности ин\-тер\-опе\-ра\-бель\-ности. 
Стандартный подход к~анализу \mbox{со\-гла\-со\-ван\-ности} заключается в~оценке 
бинарных отношений между элементами. Но взаимное влияние связей 
требует учитывать тернарные отношения, которые могут быть отнесены 
к~со\-гла\-со\-ван\-но\-му или рас\-со\-гла\-со\-ван\-но\-му со\-сто\-янию. В~\cite{5-dul} эти 
со\-сто\-яния названы консонансным и~диссонансным соответственно. Если 
пред\-ста\-вить знаковый граф\linebreak в~виде со\-во\-куп\-ности тернарных отношений, то 
получится сис\-те\-ма анализа структурной со\-гла\-со\-ван\-ности потенциально 
вза\-и\-мо\-дей\-ст\-ву\-ющих элементов. В~качестве критерия со\-гла\-со\-ван\-ности 
\mbox{пред\-ла\-га\-ет\-ся} вы\-брать тернарный критерий~\cite{5-dul}, относящий 
тернарные отношения (треугольники) к~консонансным, если положительная 
связь меж\-ду любыми парами вершин задает тож\-дест\-вен\-ность связей этих 
вершин с~третьей, в~про\-тив\-ном случае~--- к~диссонансным. 
     
     При разбиении знакового графа на треугольники и~исследовании 
со\-гла\-со\-ван\-ности множества по критерию Хай\-де\-ра выясняется, что 
консонансное множество~$M_K$ (со\-сто\-ящее только из консонансных 
треугольников) пред\-ста\-ви\-мо в~виде двух подмножеств~$M_1$ и~$M_2$: 
$M_K\hm= M_1\cup M_2$, так что любые два элемента~$o_i$ и~$o_j$ из 
одного подмножества связаны положительной связью, а~при\-над\-ле\-жа\-щие 
разным подмножествам~--- отрицательной. Если~$M_1$ или~$M_2$ пустое, 
то такое множество называется тривиальным,~--- оно содержит только 
положительные связи. 
     
     Непредсказуемое изменение знаков связей консонансного множества 
может привести к~ассонансному множеству, так как некоторые треугольники\linebreak 
становятся диссонансными. Инвертированное изменение вернет ассонансное 
множество в~консонанс. Можно говорить о~наборе связей, однократное 
изменение знаков которого преобразует \mbox{ассонансное} множество в~консонанс. 
Назовем такие связи сильными диссонансными и~определим поиск 
со\-гла\-со\-ван\-но\-го со\-сто\-яния для ассонансного множества как поиск его 
силь\-ных диссонансных связей.
     
     Существуют $[N/2]\hm+1$ типов консонансного множества из $N$ 
элементов, что указывает на множественность наборов сильных 
диссонансных связей у~каждого ассонансного множества по отношению 
к~типам консонанса. Среди су\-ще\-ст\-ву\-ющих наборов сильных диссонансных 
связей можно найти минимальный набор, который приводит заданное 
ассонансное множество к~некоторому консонансному типу. Соответственно, 
назовем консонансным прообразом ассонансного множества консонансное 
множество, по\-лу\-ча\-юще\-еся переводом ассонансного множества по\-средст\-вом 
изменения знаков минимального набора сильных диссонансных 
связей~\cite{5-dul}.
     
     Приведение ассонансного множества в~консонанс по Хай\-де\-ру означает 
воз\-мож\-ность пред\-став\-ле\-ния его в~виде двух классов эк\-ви\-ва\-лент\-ности 
элементов. Для поиска согласованного множества, со\-сто\-яще\-го более чем из 
двух подмножеств, необходимо расширить критерий Хай\-де\-ра. Введем 
понятия поликонсонанса и~консонанса степени~$P$~\cite{5-dul}. 
Поликонсонанс степени~$P$ соответствует со\-гла\-со\-ван\-но\-му со\-сто\-янию 
множества, со\-сто\-яще\-го из не\linebreak более чем $P$ подмножеств, так что элементы 
внут\-ри каждого подмножества связаны только положительными связями, 
а~из разных подмножеств~--- только отрицательными. Под консонансом 
\mbox{степени}~$P$ будем понимать поликонсонанс степени~$P$, со\-сто\-ящий 
в~точ\-ности из $P$ классов (рис.~2).
     
%o1&+&-&-&-&-&&o1&+&+&-&-&-
%+&o2&-&-&-&-&&+&o2&+&-&-&-
%-&-&o3&+&-&-&&+&+&o3&-&-&-
%-&-&+&o4&-&-&&-&-&-&o4&+&+
%-&-&-&-&o5&+&&-&-&-&+&o5&+
%-&-&-&-&+&o6&&-&-&-&+&+&o6
%Консонанс 3 степени&&Консонанс

\begin{figure*} %fig2
\vspace*{1pt}
\begin{center}
   \mbox{%
\epsfxsize=115.348mm
\epsfbox{dul-2.eps}
}
\end{center}
\vspace*{-9pt}
\Caption{Примеры матриц связности консонансных множеств: (\textit{а})~консонанс степени~3;
(\textit{б})~консонанс степени~2}
\end{figure*}
     
     Предельный случай поликонсонанса при $P\hm= N$, где~$N$~--- чис\-ло 
элементов множества, пред\-став\-ля\-ет собой тривиальное диссонансное 
множество. Тем самым диссонансное множество \mbox{пред\-став\-ля\-ет\-ся} как очень 
слабо со\-гла\-со\-ван\-ное, и~вопрос преобразования в~со\-гла\-со\-ван\-ное со\-сто\-яние 
мож\-но ставить только относительно ассонансного множества. Эвристические 
методы могут уменьшить перебор при поиске поликонсонанса. Одним из 
таких прос\-тых методов является учет при раз\-би\-ении множества на 
подмножества поликонсонанса тож\-дест\-вен\-ности структур элементов, 
находящихся в~одном и~том же подмножестве как классе эк\-ви\-ва\-лент\-ности по 
структурному признаку. Другими словами, любые два элемента 
с~тож\-дест\-вен\-ны\-ми связями со всеми элементами из некоторого 
подмножества и~име\-ющие положительную связь друг с~другом, будут 
располагаться в~этом же подмножестве. Это поз\-во\-ля\-ет выявлять элементы 
с~тож\-дест\-вен\-ны\-ми структурами связей, пред\-став\-ляя их в~виде 
интегрированного элемента, что со\-кра\-ща\-ет чис\-ло пе\-ре\-би\-ра\-емых элементов. 
Слож\-ность такой задачи оценивается примерно как слож\-ность задачи 
сортировки элементов. 
     
     Понятие поликонсонанса степени $P\hm>1$ корректирует понятие 
ассонансного множества как множества, не относящегося ни 
к~консонансным, ни к~диссонансным. 
     
     Пусть дано множество, мат\-ри\-ца связ\-ности которого изображена на 
рис.~2,\,\textit{а}. 
          Это множество соответствует поликонсонансу степени~3, но 
в~условиях консонанса степени~2 оно становится ассонансным. Понижение 
степени консонанса, как правило, переводит консонансное множество 
в~ассонансное, а~с~повышением степени консонанса, наоборот, ассонансное 
множество может перейти в~консонанс. 
     
     Центральная задача пред\-став\-лен\-но\-го исследования~--- это уменьшение 
рас\-со\-гла\-со\-ван\-ности по\-средст\-вом поиска консонансного прообраза 
минимальным набором силь\-ных диссонансных связей. 
     
     Структура консонансного множества такова, что любые два 
элемента~$o_i$ и~$o_j$ из одного и~того же подмножества обладают 
тож\-дест\-вен\-ной структурой связей и~неразличимы по структуре связей среди 
других элементов этого подмножества. С~другой стороны, если ~$o_i$ 
и~$o_j$ взяты из разных подмножеств, то связи с~элементами из 
противоположного подмножества у~них инвертированы по знаку. Отсюда 
следует правило перевода элемента из одного подмножества в~другое: 
следует изменить знаки всех его связей с~элементами подмножества, 
в~котором он находится, на противоположные. 
     
     Перевод элемента в~другое подмножество может трактоваться как 
переброс вершины со\-от\-вет\-ст\-ву\-юще\-го множеству знакового графа из одного 
под\-гра\-фа в~другой. Эта операция называется \textit{повершинным 
перебросом}. Любая по\-сле\-до\-ва\-тель\-ность повершинных перебросов не 
изменяет вид со\-сто\-яния множества: консонанс, диссонанс или ассонанс не 
чувствительны к~этой операции. 

\section{Поиск ближайшей консонансной структуры}
     
     Алгоритм поиска структурного соответствия, поз\-во\-ля\-ющий оценивать 
связ\-ность групп потенциально близ\-ких друг к~другу элементов, заключается\linebreak 
в~оценке со\-сто\-яния ассонансного множества, мо\-де\-ли\-ру\-юще\-го 
потенциальную воз\-мож\-ность элементов к~ин\-тер\-опе\-ра\-бель\-ности. Полученное 
в~результате моделирования ассонансное множество \mbox{срав\-ни\-ва\-ет\-ся} 
с~некоторым консонансным прообразом и~преобразуется с~по\-мощью 
операции повершинного переброса так, что\-бы \mbox{найти} ближайший по чис\-лу 
раз\-ли\-ча\-емых связей консонансный прообраз. 
     
     Пусть заданы два множества~$M_1$ и~$M_2$, со\-сто\-ящие из одних 
и~тех же $N$ элементов, у~которых могут различаться знаки связей меж\-ду 
парами одних и~тех же элементов. Для связей каж\-дой пары элементов ~$o_i$ 
и~$o_j$ в~двух множествах зададим чис\-ло~$r_{ij}$:
$$
r_{ij} = \begin{cases}
0,  & \mbox{если } i=j\,;\\
0, & \mbox{если } i\not=j\ \mbox{и~связи\ между } o_i\ \mbox{и}\ o_j\  \mbox{в~этих}\\
& \mbox{множествах\ различны};\\
1, & \mbox{если } i\not= j\  \mbox{и~связи\ между }o_i\ \mbox{и}\ o_j\ \mbox{в~этих}\\
& \mbox{множествах\ совпадают}.
\end{cases}
$$

\begin{figure*} %fig3
\vspace*{1pt}
\begin{center}
   \mbox{%
\epsfxsize=160.127mm
\epsfbox{dul-4.eps}
}
\end{center}
\vspace*{-9pt}
\Caption{Пример различных множеств~$A$ и~$B$ с~одним и~тем же век\-то\-ром 
повершинных различий от~$C$}
\end{figure*}
     
     Для оценки различия множеств~$M_1$ и~$M_2$ введем 
последовательность из $N$ чисел~$v_1, v_2, \ldots, v_N$, где $v_i\hm= 
\sum\nolimits^N_{j=1} r_{ij}$ для $i \hm= 1, \ldots , N$, и~определим вектор 
$V\hm = (v_1, v_2, \ldots , v_N)$, ха\-рак\-те\-ри\-зу\-ющий различие в~знаках связей 
у~пар одних и~тех же элементов в~двух множествах. Каждую компоненту 
вектора~$V$ можно интерпретировать как сумму различий в~связях 
некоторой вершины в~графах, со\-от\-вет\-ст\-ву\-ющих~$M_1$ и~$M_2$. Благодаря 
такой интерпретации вектор~$V$ мож\-но назвать вектором повершинных 
раз\-ли\-чий~$M_1$ и~$M_2$. Сумма $v_1, v_2, \ldots , v_N$ этого вектора 
$S_v\hm= \sum\nolimits^N_{i=1} v_i$ равна удвоенной сумме знаковых 
раз\-ли\-чий заданных множеств.
    % 
     Поскольку $0 \hm\leq v_i \hm\leq N\hm-1$ для любого~$i$, то сумма 
повершинных различий~$S_v$ не превышает $N(N\hm-1)$ и~всегда \mbox{четна}.
     
     Определение множества на основе вектора повершинных различий от 
другого множества в~общем случае неоднозначно, что мож\-но 
про\-ил\-люст\-ри\-ро\-вать на сле\-ду\-ющем примере (рис.~3). 
     
%o1&+&+&+&&o1&+&-&+&&o1&+&-&-&&1
%+&o2&+&+&&+&o2&+&-&&+&o2&-&-&&1
%+&+&o3&+&&-&+&o3&+&&-&-&o3&+&&1
%+&+&+&o4&&+&-&+&o4&&-&-&+&o4&&1
%A&&B&&C&&V

     
     Множество, заданное на основе мат\-ри\-цы связ\-ности~$B$, имеет 
век\-тор повершинных различий~$V$, все компоненты которого~--- единицы, 
по срав\-не\-нию с~тривиальным консонансным множеством с~со\-от\-вет\-ст\-ву\-ющей 
мат\-ри\-цей~$A$. Этот же вектор повершинных отличий определяет срав\-не\-ние 
множества~$B$ с~консонансным множеством с~со\-от\-вет\-ст\-ву\-ющей 
мат\-ри\-цей~$C$.
     
     Минимальное отличие суммы инвертированных связей 
в~множестве~$M$ выражается в~минимальной сумме компонентов вектора 
повершинных различий с~$M$. Выведем соотношение для уменьшения 
суммы повершинных различий при повершинных перебросах. 
     
     Возьмем множество из $N$ элементов и~ка\-кое-ни\-будь консонансное 
множество из тех же элементов, тогда, согласно определению, вектор 
повершинных различий для этих множеств:
     $$
     V= \left(v_1, v_2, \ldots , v_N\right)\,, 
     $$
     где $v_i\hm= \sum\nolimits^N_{j=1} r_{ij}$.
     
Сумма повершинных различий:
     $$
     S_v= \sum\limits^N_{i=1} v_i= \sum\limits^N_{i=1} \sum\limits^N_{j=1} 
r_{ij}.
     $$
     
     Если существует элемент, у которого сумма его несовпадающих связей 
больше половины всех связей этого элемента, то, не нарушая общ\-ности, 
мож\-но считать, что это элемент~$o_1$, и~тогда $v_1\hm= (N\hm-1)/2$. 
Инвертирование знаков связей этого элемента изменяет вектор повершинных 
различий сле\-ду\-ющим образом:

\noindent
    \begin{multline*}
     S_v^\prime =\sum\limits^N_{i=1} v_i^\prime =v^\prime_1 
+\sum\limits_{i=2}^N \sum\limits^N_{j=1} r_{ij}^\prime = {}\\
{}= v_1^\prime + \sum\limits_{i=2}^N r^\prime_{i1} +\sum\limits_{i=2}^N \sum\limits^N_{j=2} r^\prime_{ij}\,.
\end{multline*}
     
     Поскольку при повершинном перебросе все связи~$o_1$ изменяются 
на противоположные, то $v_1^\prime \hm= N\hm- 1 \hm- v_1$. Так как 
$r_{ij}\hm = r_{ji}$, имеем
     $$
     \sum\limits^N_{i=2} r^\prime_{i1} =\sum\limits^N_{j=2} r^\prime_{1j} 
=v_1^\prime = N-1-v_1.
     $$
При повершинном перебросе элемента~$o_1$ связи между~$o_i$ и~$o_j$  ($i 
\hm> 1$ и~$j \hm> 1$) не изменяются; следовательно, 
$$
\sum\limits^N_{i=2} \sum\limits^N_{j=2} r^\prime_{ij} =\sum\limits^N_{i=2} 
\sum\limits^N_{j=2} r_{ij}.
$$
Подставляя в~$S_v^\prime$, получаем
$$
S_v^\prime =  \left( N-1-v_1\right) +\left( N-1-v_1\right) +\left( S_v-2v_1\right).
$$

\begin{figure*} %fig4
\vspace*{1pt}
\begin{center}
   \mbox{%
\epsfxsize=120.915mm
\epsfbox{dul-5.eps}
}
\end{center}
\vspace*{-9pt}
\Caption{Два минимальных вектора повершинных различий с~различными 
суммами~$v_i$}
\end{figure*}

\noindent
Отсюда
$$
S_v^\prime -S_v= 2\left( N-1-v_1\right) -2v_1.
$$
Значит, для того чтобы повершинный переброс элемента привел 
к~уменьшению суммы повершинных различий, нуж\-но брать элементы, для 
которых компоненты вектора суммы повершинных различий больше 
половины знаков связей:
$$
v_i> \fr{N-1}{2}\,.
$$
     
     Последовательные повершинные перебросы в~общем случае не 
обеспечивают до\-сти\-же\-ние минимально удаленного консонансного 
прообраза. Можно продемонстрировать это, рас\-смот\-рев ассонансное 
множество на рис.~4.
     
      %                                $V_1$    &     $V_2$
%o1&+&+&+&+&+&+&+&-&&3&&2\\
%+&o2&+&+&-&-&+&+&-&&3&&2\\
%+&+&o3&+&-&+&-&-&+&&3&&2\\
%+&+&+&o4&-&+&-&-&+&&3&&2\\
%+&-&-&-&o5&-&-&-&-&&3&&1\\
%+&-&+&+&-&o6&+&+&+&&3&&1\\
%+&+&-&-&-&+&o7&+&+&&2&&2\\
%+&+&-&-&-&+&+&o8&+&&2&&2\\
%-&-&+&+&-&+&+&+&o9&&2&&2\\


     
     Матрица связности этого ассонансного множества со\-сто\-ит из 9 
элементов. Вектор повершинных различий~$V_1$ характеризует отличие по 
связям данного множества от консонансного множества $(\{o_1, o_2, o_3, o_4, 
o_5\}; \{o_6, o_7, o_8, o_9\})$ из тех же элементов. Ни одна из 
компонент~$V_1$ не превышает половины чис\-ла связей соотносимых 
элементов, и,~значит, любой повершинный переброс в~данном множестве 
будет увеличивать~$S_v$. При этом найденное консонансное множество не 
является минимально удаленным от рас\-смат\-ри\-ва\-емо\-го ассонансного 
множества, так как существует множество $(\{o_5\}: \{o_2, o_3, o_4, o_1, o_6, 
o_7, o_8, o_9\})$ с~вектором повершинных различий~$V_2$, обла\-да\-ющим 
суммой~$S_v$, меньшей, чем для вектора~$V_1$. 
     
     Становится ясно, что алгоритм, в~основе которого лежит 
по\-сле\-до\-ва\-тель\-ный повершинный переброс элемента с~компонентом вектора 
повершинных различий, б$\acute{\mbox{о}}$льшим, чем половина связей 
\mbox{этого}\linebreak элемента, находит только локальный минимум удаления от консонанса. 
Следовательно, алгоритм последовательного уменьшения повершинных\linebreak 
\mbox{различий} должен находить все возможные неулучшаемые состояния. Для 
реализации такой воз\-мож\-ности нужно сформулировать некоторые 
специальные условия, поз\-во\-ля\-ющие найти минимально\linebreak удаленный 
консонансный прообраз.
     
     Если вернуться к~примеру ассонансного множества из 9 
элементов, то мож\-но проанализировать получение двух векторов 
повершинных различий для двух консонансных прообразов. Консонансный 
прообраз, со\-от\-вет\-ст\-ву\-ющий вектору~$V_2$, предпочтительнее первого 
прообраза. Чтобы преобразовать пер\-вый консонансный прообраз во второй 
консонансный прообраз, нуж\-но инвертировать знаки связей элементов~$o_1$, 
$o_2$, $o_3$ и~$o_4$, что эквивалентно операции повершинного переброса 
применительно к~каж\-до\-му из этих элементов. Возникает характерная 
ситуация: последовательные повершинные перебросы элементов~$o_1$, 
$o_2$, $o_3$ или~$o_4$, приводят к~увеличению~$S_v$, а~груп\-по\-вой 
повершинный переброс, проведенный для~$o_1$, $o_2$, $o_3$ и~$o_4$ 
одновременно, поз\-во\-ля\-ет уменьшить~$S_v$. 
     
     Значит, необходимо искать такую группу из $k$ элементов, 
одновременный повершинный переброс которой уменьшает~$S_v$. 
Рас\-смот\-рим в~связи с~этим некое ассонансное множество из~$N$~элементов: 
$M\hm= \{o_i\}$, $i \hm= 1, \ldots , N$. По\-ста\-вим ему в~соответствие  
ка\-кой-ни\-будь консонансный прообраз с~вектором повершинных различий 
$V\hm = (v_1, v_2, \ldots , v_N)$ с~суммой элементов
     $$
     S_v= \sum\limits_{i=1}^N v_i= \sum\limits^N_{i=1} \sum\limits^N_{j=1} 
r_{ij}.
     $$
Выберем произвольные $k$ элементов. Не нарушая общ\-ности, мож\-но 
выбрать $o_1, \ldots , o_k$. Осуществив для группы из $k$ элементов операции 
повершинных перебросов, преобразуем вектор повершинных различий 
$V^\prime \hm= (v_1^\prime, v_2^\prime, \ldots , v_N^\prime)$ и~$S_v^\prime 
\hm= \sum\nolimits^N_{i=1} v_i^\prime\hm= \sum\nolimits^N_{i=1} 
\sum\nolimits^N_{j=1} r^\prime_{ij}$. Распишем эту сумму в~виде четырех 
сумм:
\begin{multline*}
S_v^\prime= \sum\limits_{i=1}^k \sum\limits_{j=1}^N r^\prime_{ij} 
+\sum\limits^N_{i=k+1} \sum\limits^N_{j=1} r^\prime_{ij} =  \sum\limits^k_{i=1} \sum\limits^k_{j=1} r^\prime_{ij} +{}\\
{}+
\sum\limits^N_{i=k+1} \sum\limits^k_{j=1} r^\prime_{ij}+
\sum\limits^k_{i=1} \sum\limits^N_{j=k+1} r^\prime_{ij}+
\sum\limits^N_{i=k+1} \sum\limits^N_{j=k+1} r^\prime_{ij}.
\end{multline*}
     
     Так как при проведении повершинных перебросов связи между 
элементами~$o_i$ и~$o_j$ ($i \hm> k$ и~$j \hm> k$) не затрагиваются, 
     $$
     \sum\limits^N_{i=k+1} \sum\limits^N_{j=k+1} r^\prime_{ij} = 
\sum\limits^N_{i=k+1} \sum\limits_{j=k+1}^N r_{ij}.
     $$
     
     В то же время связи между элементами~$o_i$ и~$o_j$ ($i \hm\leq k$ 
и~$j \hm\leq k$) изменяются дваж\-ды: сначала при повершинных изменениях 
для элемента~$o_i$, затем для элемента~$o_j$; следовательно, связи между 
элементами из выбранной группы останутся неизменными:
     $$
     \sum\limits^k_{i=1} \sum\limits^k_{j=1} r^\prime_{ij} =\sum\limits^k_{i=1} 
\sum\limits^k_{j=1} r_{ij}.
     $$
     
     Симметричность связей между элементами поз\-во\-ля\-ет полагать
     $$
     \sum\limits^N_{i=k+1} \sum\limits^k_{j=1} r^\prime_{ij} 
=\sum\limits^k_{i=1} \sum\limits^N_{j=k+1} r^\prime_{ij}.
     $$
     
     Учитывая вышесказанное, имеем
     $$
     S_v^\prime -S_v= 2\sum\limits^k_{i=1} \sum\limits^N_{j=k+1} 
r^\prime_{ij} -2\sum\limits^k_{i=1} \sum\limits^N_{j=k+1} r_{ij}.
     $$
     
     Пусть $x_i\hm= \sum\nolimits^N_{j=k+1} r_{ij}$~--- число диссонансных 
связей элемента $o_i$ из группы в~$k$ элементов с~элементами, не входящими 
в~эту группу. Так как эти связи инвертируются, имеем $x_i^\prime\hm= N\hm- k\hm= x_i$.
 Отсюда для уменьшения суммы повершинных различий 
необходимо, чтобы

\noindent
     \begin{multline*}
     S_v^\prime -S_v= 2\sum\limits^N_{i=1} x_i^\prime -2\sum\limits^N_{i=1} 
x_i =2\sum\limits^N_{i=1} \left( N-k-x_i\right) -{}\\
{}- 2\sum\limits^N_{i=1} x_i= 2k(N-k) -4\sum\limits^N_{i=1} x_i<0\,,
     \end{multline*}
а это справедливо при условии
$$
\sum\limits^N_{i=1} x_i > \fr{k(N-k)}{2}\,.
$$
     
     Если группа состоит из одного элемента ($k\hm=1$), то приходим 
к~сформулированному ранее условию $v_i\hm> (N-1)/2$.
     
     Программная реализация алгоритма уменьшения рас\-со\-гла\-со\-ван\-ности 
ослож\-ня\-ет\-ся тем, что его тру\-до\-ем\-кость оценивается как $2^N$~\cite{4-dul}. 
Для консонансного или диссонансного множества это не станет проб\-ле\-мой: 
ввиду опре\-де\-лен\-ности структуры можно указать эффективный алгоритм 
приведения к~нуж\-но\-му консонансу. Для ассонансного множества важ\-но 
правильно задать начальный консонансный прообраз исходного множества: 
чис\-ло итераций ре\-ша\-ющим образом зависит от его бли\-зости к~исходному 
множеству. 
     
     На выбор начального прообраза оказывают существенное влияние 
многие факторы, в~том чис\-ле и~субъективные. Можно использовать только 
математические методы задания начального прообраза, но значительное 
вли\-яние на эф\-фек\-тив\-ность алгоритма~\cite{4-dul} оказывают 
и~дополнительные факторы. Использование этих факторов в~программной 
реализации алгоритма уменьшения рас\-со\-гла\-со\-ван\-ности обеспечило 
приемлемые временн$\acute{\mbox{ы}}$е характеристики. 

\section{Заключение}

     Моделирование структурной ин\-тер\-опе\-ра\-бель\-ности на основе анализа 
знаков связей с~по\-мощью введенного критерия со\-гла\-со\-ван\-ности приводит 
к~нахождению ближайшего к~исходному множеству консонансного 
прообраза. Найденный консонансный прообраз своими подмножествами 
указывает на предпочтительную группировку элементов, при которой 
ин\-тер\-опе\-ра\-бель\-ность меж\-ду ними уста\-нав\-ли\-ва\-ет\-ся с~наименьшей 
рас\-со\-гла\-со\-ван\-ностью относительно зафиксированных знаков связей. 
Поскольку рас\-смат\-ри\-ва\-емые элементы могут быть описаны вектором 
па\-ра\-мет\-ров, из сравнения которых мож\-но сделать вывод о~сходстве меж\-ду 
элементами, соответственно, на\-хож\-де\-ние элементов в~одном подмножестве 
говорит о~потенциальной мотивации к~ин\-тер\-опе\-ра\-бель\-ности.

{\small\frenchspacing
 {%\baselineskip=10.8pt
 %\addcontentsline{toc}{section}{References}
 \begin{thebibliography}{9}
\bibitem{1-dul}
ГОСТ Р 55062-2012. Информационные технологии (ИТ). Сис\-те\-мы промышленной 
автоматизации и~их интеграция. Ин\-тер\-опе\-ра\-бель\-ность. Основ\-ные положения.~--- М.: 
Стандартинформ, 2014. 12~с.
\bibitem{2-dul}
Systems, Capabilities, Operations, Programs, and Enterprises (SCOPE) Model for 
Interoperability Assessment. Version~1.0.~--- NCOIC, 2008. 154~p.
\bibitem{3-dul}
\Au{Макаренко С.\,И., Соловьева~О.\,С.} Основные положения концепции семантической 
ин\-тер\-опе\-ра\-бель\-ности сетецентрических сис\-тем~// Ж.~радиоэлектроники, 2021. №\,4. 
24~с. 
\bibitem{4-dul}
\Au{Дулин С.\,К., Розенберг~И.\,Н.} Об одном подходе к~структурной со\-гла\-со\-ван\-ности 
гео\-дан\-ных~// Мир транспорта, 2005. Т.~3. №\,3. С.~16--29.
\bibitem{5-dul}
\Au{Дулин С.\,К.} Введение в~теорию структурной со\-гла\-со\-ван\-ности.~--- М.: ВЦ РАН, 
2005. 135~с.

\end{thebibliography}

 }
 }

\end{multicols}

\vspace*{-6pt}

\hfill{\small\textit{Поступила в~редакцию 22.02.22}}

\vspace*{8pt}

%\pagebreak

%\newpage

%\vspace*{-28pt}

\hrule

\vspace*{2pt}

\hrule

%\vspace*{-2pt}

\def\tit{MODELING THE STRUCTURE OF~INTEROPERABILITY BY~MEANS~OF~STRUCTURAL~CONSISTENCY}


\def\titkol{Modeling the structure of~interoperability by~means of~structural 
consistency}


\def\aut{I.\,N.~Rozenberg$^1$, S.\,K.~Dulin$^{1,2}$, and~N.\,G.~Dulina$^2$}

\def\autkol{I.\,N.~Rozenberg, S.\,K.~Dulin, and~N.\,G.~Dulina}

\titel{\tit}{\aut}{\autkol}{\titkol}

\vspace*{-8pt}


\noindent
$^1$Research \& Design Institute for Information Technology, Signalling and Telecommunications on Railway\linebreak
$\hphantom{^1}$Transport, 27-1~Nizhegorodskaya Str., Moscow 109029, Russian Federation

\noindent
$^2$Federal Research Center ``Computer Science and Control'' of the Russian Academy of Sciences,  
44-2~Vavilov\linebreak
$\hphantom{^1}$Str., Moscow 119133, Russian Federation

\def\leftfootline{\small{\textbf{\thepage}
\hfill INFORMATIKA I EE PRIMENENIYA~--- INFORMATICS AND
APPLICATIONS\ \ \ 2023\ \ \ volume~17\ \ \ issue\ 1}
}%
 \def\rightfootline{\small{INFORMATIKA I EE PRIMENENIYA~---
INFORMATICS AND APPLICATIONS\ \ \ 2023\ \ \ volume~17\ \ \ issue\ 1
\hfill \textbf{\thepage}}}

\vspace*{3pt} 




\Abste{The initial syntactic level of interoperability involves communication with the appropriate 
protocol, hardware, software, and necessary level of data compatibility. The work is devoted 
to the study of the level of compatibility of data describing interacting elements based on the feature 
vector. To do this, a model of structural correspondence is proposed which allows assessing the tendency 
to establish interoperability. Modeling structural interoperability based on the analysis of signs of 
connections using the introduced consistency criterion leads to finding the closest consonant pre-image to 
the original set. The found consonant pre-image with its subsets indicates the preferred grouping of 
elements in which the interoperability between them is established with the least mismatch with respect 
to the fixed signs of connections. Since the elements under consideration are described by a~vector of 
parameters, from the comparison of which one can infer the similarity between the elements, respectively, 
the presence of elements in the same subset indicates a potential motivation for interoperability.}

\KWE{interoperability; structural consistency; connectivity matrix}


\DOI{10.14357/19922264230108} 

%\vspace*{-16pt}

%\Ack
%\noindent

  

%\vspace*{12pt}

  \begin{multicols}{2}

\renewcommand{\bibname}{\protect\rmfamily References}
%\renewcommand{\bibname}{\large\protect\rm References}

{\small\frenchspacing
 {%\baselineskip=10.8pt
 \addcontentsline{toc}{section}{References}
 \begin{thebibliography}{9} 
 \bibitem{1-dul-1}
GOST R 55062-2012. 2012. {Informatsionnye tekhnologii (IT). Sis\-te\-my pro\-mysh\-len\-noy av\-to\-ma\-ti\-za\-tsii i~ikh in\-teg\-ra\-tsiya. 
Inter\-ope\-ra\-bel'\-nost'. Osnovnye polozheniya} [Information technologies (IT). Systems of industrial automation and their integration. 
Interoperability. Basic statements]. Moscow: Standartinform Publs. 20~p. 
\bibitem{2-dul-1}
NCOIC.  2008. {Systems, Capabilities, Operations, Programs, and 
Enterprises (SCOPE). Model for Interoperability Assessment}. Version~1. 154~p. Available at: {\sf 
https://\linebreak jfsowa.com/ikl/scope08.pdf} (accessed December~29, 2022).

%\columnbreak

\bibitem{3-dul-1}
\Aue{Makarenko, S.\,I., and O.\,S.~Solovieva.} 2021. Osnov\-nye po\-lo\-zhe\-niya kon\-tsep\-tsii 
se\-man\-ti\-che\-skoy in\-ter\-ope\-ra\-bel'\-nosti se\-te\-tsent\-ri\-che\-skikh sis\-tem [Basic provisions of the concept of 
semantic interoperability of net-centric systems]. \textit{Zh.\ radioelektroniki} [J.~Radio Electronics] 
4:1--24. 


\bibitem{4-dul-1}
\Aue{Dulin, S.\,K., and I.\,N.~Rozenberg.} 2005. Ob od\-nom pod\-kho\-de k~struk\-tur\-noy so\-gla\-so\-van\-nosti 
geo\-dan\-nykh [An approach to the structural consistency of geographical data]. \textit{Mir transporta} [World of 
Transport and Transportation] 3(3):16--29.
\bibitem{5-dul-1}
\Aue{Dulin, S.\,K.} 2005. \textit{Vve\-de\-nie v~teo\-riyu struk\-tur\-noy so\-gla\-so\-van\-nosti} [Introduction to 
structural consistency theory]. Moscow: CC RAS. 135~p.

\end{thebibliography}

 }
 }

\end{multicols}

\vspace*{-6pt}

\hfill{\small\textit{Received February 22, 2022}}

\Contr
\noindent
\textbf{Rozenberg Igor N.} (b.\ 1965)~--- Doctor of Science in technology, professor, Corresponding 
Member of the Russian Academy of Sciences, research advisor, Research \& Design Institute for 
Information Technology, Signalling and Telecommunications on Railway Transport,  
27-1~Nizhegorodskaya Str., Moscow 109029, Russian Federation; \mbox{I.Rozenberg@vniias.ru}




\vspace*{6pt}

\noindent
\textbf{Dulin Sergey K.} (b.\ 1950)~--- Doctor of Science in technology, professor; leading scientist, 
Institute of Informatics Problems, Federal Research Center ``Computer Science and Control'' of the 
Russian Academy of Sciences, 44-2~Vavilov Str., Moscow 119333, Russian Federation; principal 
scientist, Research \& Design Institute for Information Technology, Signalling and Telecommunications 
on Railway Transport, 27-1~Nizhegorodskaya Str., Moscow 109029, Russian Federation; 
\mbox{skdulin@mail.ru}

\vspace*{6pt}

\noindent
\textbf{Dulina Natalia G.} (b.\ 1947)~--- Candidate of Science (PhD) in technology, leading programmer, 
A.\,A.~Dorodnicyn Computing Center, Federal Research Center ``Computer Science and Control'' of the
Russian Academy of Sciences, 40~Vavilov Str., Moscow 119333, Russian Federation; 
\mbox{ngdulina@mail.ru}


\label{end\stat}

\renewcommand{\bibname}{\protect\rm Литература} 