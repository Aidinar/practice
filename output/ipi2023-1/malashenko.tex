\def\ol{\overline}

\def\stat{malashenko}

\def\tit{ОЦЕНКИ РАСПРЕДЕЛЕНИЯ РЕСУРСОВ В~МНОГОПОЛЬЗОВАТЕЛЬСКОЙ СЕТИ  ПРИ~РАВНЫХ~МЕЖУЗЛОВЫХ НАГРУЗКАХ}

\def\titkol{Оценки распределения ресурсов в~многопользовательской сети  при~равных межузловых нагрузках}

\def\aut{Ю.\,Е. Малашенко$^1$,  И.\,А.~Назарова$^2$}

\def\autkol{Ю.\,Е. Малашенко,  И.\,А.~Назарова}

\titel{\tit}{\aut}{\autkol}{\titkol}

\index{Малашенко Ю.\,Е.}
\index{Назарова И.\,А.}
\index{Malashenko Yu.\,E.}
\index{Nazarova I.\,A.}


%{\renewcommand{\thefootnote}{\fnsymbol{footnote}} \footnotetext[1]
%{Работа выполнена при поддержке Министерства науки и~высшего образования
%Российской федерации, грант №\,075-15-2020-799.}}


\renewcommand{\thefootnote}{\arabic{footnote}}
\footnotetext[1]{Федеральный исследовательский центр <<Информатика и~управление>> Российской 
академии наук, \mbox{malash09@ccas.ru}}
\footnotetext[2]{Федеральный исследовательский центр <<Информатика и~управление>> Российской 
академии наук, \mbox{irina-nazar@yandex.ru}}

\vspace*{-10pt}



\Abst{Предлагается метод  оценки ресурсов при уравнительном распределении межузловых нагрузок в~многопользовательской сети. 
В~рамках формальной математической модели пропускные способности ребер рассматриваются как компоненты вектора ресурсов, 
которые требуются для передачи потоков  разных видов. Предлагается алгоритмическая процедура перераспределения и~использования
 пропускных способностей при равных объемах  ресурсов, выделяемых всем корреспондентам.
  При поиске соответствующих реберных загрузок определяются значения максимальных однопродуктовых потоков для каждой пары узлов. 
  В~ходе вычислительных экспериментов суммарный ресурс считается заданным для сетей с~различными структурными особенностями.}

\KW{многопродуктовая потоковая модель; распределение ресурсов  и~межузловых нагрузок; предельная загрузка сети}

 \DOI{10.14357/19922264230111} 
  
\vspace*{-2pt}


\vskip 10pt plus 9pt minus 6pt

\thispagestyle{headings}

\begin{multicols}{2}

\label{st\stat}


\section{Введение}

Данная работа продолжает исследование методов анализа  функциональных возможностей  телекоммуникационных сис\-тем~[1, 2].  
В~рамках многопродуктовой сетевой модели рассматривается \mbox{вычислительная} процедура оценки ресурсов, которые требуются при уравнительном 
распределении межузловых  нагрузок. Под ресурсом,  выделяемым  некоторой паре уз\-лов-кор\-рес\-пон\-ден\-тов, 
понимается суммарная величина пропускных способностей, необходимых для обеспечения  передачи информационного потока определенного вида.
 Дуговые потоки разных видов рассматриваются как   \textit{нагрузка} на указанном ребре,  возникающая при одновременном соединении корреспондентов.  
 В~ходе экспериментов осуществляется поиск распределения потоков, при котором  межузловые  нагрузки  равны друг другу. 
 Согласно предложенной  процедуре для каждой пары узлов вычисляется максимальный однопродуктовый поток. Маршруты передачи для каждой 
 пары прокладываются по ребрам соответствующих минимальных разрезов, и~для  всех корреспондентов  подсчитываются  реберные  нагрузки.  
 На основе найденных значений определяются требуемые пропускные способности при одновременной передаче  всех межузловых потоков. 
 Анализ результатов экспериментов  позволяет  оценить  распределение ресурсов при равных межузловых нагрузках и~предельной загрузке в~многопользовательских 
  сетях~[1, 2].

В настоящее время для создания, развития и~эксплуатации  телекоммуникационных систем используются потоковые модели и~разрабатываются специальные методы решения~[3, 4]. 
Математические  модели передачи многопродуктового потока применяются для поиска недискриминирующих правил распределения ресурсов в~системах связи~[5].   
В~русле исследований, изложенных в~указанных работах, лежит   алгоритмическая схема получения\linebreak уравнительных распределений  межузловых нагрузок (разд.~3). 
В~разд.~4 обсуждаются результаты  экспериментов  и~сравниваются  достижимые  межузловые  потоки   при  предельных  распределениях 
\mbox{пропускных} способностей   в~сетях с~различной структурой.  

\vspace*{-6pt}

\section{Математическая модель}

Для описания многопользовательской сетевой системы связи  воспользуемся следующей математической записью модели передачи многопродуктового потока.
Сеть~$G$ задается множествами  $\langle V, R,  U, P \rangle$:
узлов (вершин) сети  $V\hm = \{ v_1, v_2, \ldots\linebreak \ldots, v_n, \ldots, v_N \}$;
неориентированных ребер $R \hm= \{ r_1, r_2,  \ldots, r_k, \ldots, r_E \}$;
ориентированных дуг\linebreak  $U \hm= \{ u_1, u_2,  \ldots, u_k,  \ldots, u_{2E}\}$;
пар уз\-лов-кор\-рес\-пон\-ден\-тов $P\hm = \{ p_1, p_2,  \ldots, p_M\}$.
Предполагается, что в~сети отсутствуют петли и~сдвоенные ребра.
В~многопользовательской сети~$G$ рассматривается $M \hm= N (N\hm-1)$ независимых, невзаимозаменяемых и~равноправных межузловых потоков различных видов.

Ребро $r_k \in R$ соединяет \textit{смежные} вершины $v_{n_k}$ и~$v_{j_k}$.
Каждому ребру~$r_k$ ставятся в~соответствие две ориентированные дуги $u_k$ и~$u_{k+E}$ из множества~$U$.
Дуги $\{u_k, u_{k+E}\}$ определяют прямое и~обратное направления передачи потока по  ребру~$r_k$ между концевыми вершинами $v_{n_k}$  и~$v_{j_k}$.

Каждой паре уз\-лов-кор\-рес\-пон\-ден\-тов~$p_m$ из множества~$P$ ставятся в~соответствие:
вер\-ши\-на-ис\-точ\-ник с~номером~$s_m$,  из~$s_m$  входной поток $m$-го вида поступает в~сеть;
вер\-ши\-на-при\-емн\-ик с~номером~${t_m}$, из~${t_m}$ поток $m$-го вида покидает сеть.
Обозначим через~$z_m$ величину \textit{межузлового} потока $m$-го вида, поступающего в~сеть через узел с~номером~$s_m$ 
и~по\-ки\-да\-юще\-го сеть из узла с~номером~$t_m$;
$x_{mk}$, $x_{m(k + E)}$~---  поток $m$-го вида, который передается по дугам~$u_k$ и~$u_{k + E}$
 согласно направлению передачи, $x_{mk}\hm \ge 0$, $x_{m(k + E)}\hm\ge 0$, $m \hm= \ol{1, M}$, $k \hm= \ol {1, E}$;
$S(v_n)$~--- множество номеров исходящих дуг, по ним поток покидает узел~$v_n$;
$T(v_n)$~--- множество номеров входящих дуг, по ним поток поступает в~узел~$v_n$.
Состав множеств~$S(v_n)$  и~$T(v_n)$ однозначно формируется в~ходе выполнения следующей процедуры. 

Пусть некоторое ребро $r_k \hm\in R$ 
соединяет вершины с~номерами~$n$ и~$j$, такими что $n \hm< j$. Тогда ориентированная дуга $u_k\hm = (v_n, v_j)$, 
направленная из вершины~$v_n$ в~$v_j$, считается \textit{исходящей} из вершины~$v_{n}$ и~ее номер~$k$ заносится в~множество~$S(v_n)$, 
а~дуга~$u_{k+E}$, направленная из~$v_j$ в~$v_n$,~--- \textit{входящей} для~$v_{n}$ и~ее номер~$k\hm+E$ помещается в~список~$T(v_n)$.
Дуга $u_k$ является \textit{входящей} для~$v_j$, и~ее номер~$k$ попадает в~$T(v_j)$, а~дуга~$u_{k+E}$~--- \textit{исходящей}, и~номер $k\hm+E$ 
вносится в~список исходящих дуг~$S(v_j)$.

Во всех узлах сети $v_n\hm \in V$, $n \hm= \ol{1,N}$,  для каж\-до\-го вида потока должны выполняться условия сохранения потоков:
\begin{multline}
\sum\limits_{i \in S(v_n)}{x_{mi}} - \sum\limits_{i \in T(v_n)}{x_{mi}} ={}\\
{}=\begin{cases}
 z_m, & \mbox{если } v_n = v_{s_m}\,; \\
- z_m, & \mbox{если } v_n = v_{t_m}\,; \\
 0 & \mbox{в остальных случаях,}
\end{cases}
\\
n = \ol {1, N}\,, \ m = \ol {1, M}\,, \ x_{mi} \ge 0\,, \ z_m \ge 0\,.
\label{e1-mal}
\end{multline}
Величина $z_m$ равна входному межузловому потоку $m$-го вида, проходящему от источника к~приемнику пары~$p_m$ при распределении  
потоков~$x_{mi}$ по дугам сети.

Каждому ребру $r_k \hm\in R$ приписывается неотрицательное число~$d_k$, определяющее суммарный 
предельно допустимый поток, который можно передать по ребру~$r_k$ в~обоих направлениях. 
В~исходной сети компоненты вектора пропускных способностей   
$\mathbf{d} \hm= (d_1, d_2, \ldots, d_k, \ldots, d_E)$~--- 
наперед заданные положительные числа $d_k\hm > 0$.  Вектор~$\mathbf{d}$ задает следующие ограничения на сумму потоков всех видов, передаваемых по ребру~$r_k$ 
одновременно:
\begin{multline}
\sum\limits_{m=1}^{M} {(x_{mk}+ x_{m(k+E)})} \le d_k,  \enskip x_{mk} \ge 0\,, \\
x_{m(k+E)} \ge 0\,, \enskip  k =\ol{1, E}\,. 
\label{e2-mal}
\end{multline}

Ограничения~(1) и~(2) задают множество  до\-пус\-ти\-мых значений компонент вектора межузловых потоков
$\mathbf{z}\hm = (z_1, z_2, \ldots, z_m, \ldots, z_M)$:
\begin{multline*}
\mathcal{Z}(\mathbf{d}) ={}\\
{}= \{\mathbf{z} \ge 0 \ |\  \e \ \mathbf{x} \ge 0: \ (\mathbf{z}, \mathbf{x})  \mbox{ удовлетворяют } (1), (2)\}.\hspace*{-0.49522pt}
\end{multline*}

В рамках данной модели пропускная способность ребер сети измеряется в~условных единицах потока и~трактуется как \textit{ресурсное} ограничение. 
Суммарное значение пропускной способности сети
$ D(0)\hm =  \sum\nolimits_{k=1}^{E} d_k$ считается заданным.
Для каждой пары уз\-лов-кор\-рес\-пон\-ден\-тов $p_m \hm\in P$, для некоторого допустимого межузлового потока~$\tilde{z}_m$ 
и~соответствующих дуговых потоков~$\tilde{x}_{mk}$, \ $k \hm= \ol{1, 2E}$, величина
$$
\tilde{y}_m = \sum\limits_{i=1}^{2E} \tilde{x}_{mi}, \enskip m = \ol{1, M}\,,
$$
характеризует \textit{нагрузку} на сеть  при передаче  межузлового потока величины~$\tilde{z}_m$ из уз\-ла-ис\-точ\-ни\-ка~$s_m$  
в~узел-при\-ем\-ник~$t_m$. Величина~$\tilde{y}_m$ показывает, какая суммарная пропускная способность сети потребуется для передачи дуговых потоков~$\tilde{x}_{mk}$. 
В~рамках модели отношение реберных и~межузловых потоков
$$ 
\tilde{w}_m = \fr{\tilde{y}_m}{\tilde{z}_m}\,,  \enskip m = \ol{1, M}\,,
$$
можно трактовать как удельные {\it затраты}  ресурсов сети при передаче единичного   потока $m$-го вида между узлами~$s_m$ и~$t_m$ 
при  дуговых потоках~$\tilde{x}_{mi}$.
Величины $\ol z_m\hm = \tilde{z}_m/\tilde{y}_m$,  $\ol{x}_{mi}\hm = \tilde{x}_{mi}/ \tilde{y}_m$, 
 $m \hm= \ol{1, M}$,  $i \hm= \ol{1, E}$, соответствуют межузловому потоку при единичной нагрузке для пары~$p_m$.

\section{Метод оценки распределения ресурсов  при выравнивании  
межузловой нагрузки}

В рамках модели проводились вычислительные эксперименты для оценки распределения пропускной способности сети при равных нагрузках для всех 
пар-кор\-рес\-пон\-ден\-тов. При подготовке  данных формировался вектор исходных пропускных способностей~$\mathbf{d}(0)$  для некоторой заданной сети~$G(0)$, 
в~которой $ D(0) \hm=  \sum\nolimits_{k=1}^{E} d_k(0)$.
В~сети~$G(0)$ при заданных~$d_k(0)$ последовательно решалась задача поиска максимального независимого однопродуктового потока~\cite{Yen} 
для каждой пары узлов $p_m \hm\in P$, $m \hm= \ol{1, M}$.

\smallskip

\noindent
\textbf{Задача~1.} Для некоторой пары узлов~$p_a$ найти 
$$ 
z_a^0 (0) = \max \left\{ z_a | (\mathbf{z}, x)  \in \mathcal{Z}(\mathbf{d}(0))\right\} 
 $$
при дополнительных условиях 
$$
z_m = 0,\ m \not = a,\ m = \ol {1, M}\,. 
$$

При последовательном решении задачи~1 для каждой пары $p_a \hm\in P$ вычисляются максимальный межузловой поток~$z_a^0 (0)$ и~дуговые 
потоки  $(x_{ak}^0(0), x_{a(k+E)}^0(0))$, $k \hm= \ol{1, E}$. Далее подсчитываются  нагрузка
$$ 
y_a^0(0) = \sum\limits_{k=1}^{2E} x_{ak}^0(0), 
$$
при $y_a^0(0)\not= 0$  нормирующий коэффициент
$ \theta_a^0(0) \hm=  1/y_a^0(0)$
и дуговые потоки $x_{ak}^0 \hm= \theta_a^0(0) x_{ak}^0(0)$, $k \hm= \ol{1, 2E}$.
При передаче дуговых потоков~$x_{ak}^0$ соответствующая нагрузка~$y_a^0$ равна единице при передаче межузлового потока
 $z_a^0 \hm= \theta_a^0(0) z_a^0(0) $.

Задача~1 решается последовательно для всех $p_m \hm\in P$, и~для найденных дуговых потоков~$x_{mk}^0$ 
определяется суммарная нагрузка на ребра сети при передаче межузлового потока~$z_m^0$:
$$
\Delta_k^0 = \sum\limits_{m=1}^{M} \left(x_{ak}^0 + x_{a(k+E)}^0\right), \enskip   k= \ol{1, E}\,.
$$

Для заданного значения~$D(0)$ определяется допустимое перераспределение реберных нагрузок:
\begin{multline}
\beta(0) \left[ \sum\limits_{m=1}^{M} \theta_m^0(0) \left(x_{mk}^0(0)+ x_{m(k+E)}^0(0)\right) \right]\!  =\! \Delta_k^0(0),\\
 k= \ol{1, E}\,.
\label{e3-mal}
\end{multline}
В рамках модели суммарная величина нагрузки численно равна требуемой суммарной пропускной способности, которая считается заданной:
\begin{equation}
\sum\limits_{k=1}^{E} \Delta_k^0(0) = D(0). 
\label{e4-mal}
\end{equation}

Из соотношений~(3) и~(4) находится значение~$\beta(0)$:
\begin{multline*}
 \beta(0) \sum\limits_{k=1}^{E} \left[ \sum\limits_{m=1}^{M} \theta_m^0(0) \left(x_{mk}^0(0)+ x_{m(k+E)}^0(0)\right) \right]  ={}\\
 {}= D(0), 
 \end{multline*}
вычисляются значения~$\Delta_k^0(0)$, $k\hm= \ol{1, E}$.

На основе найденных значений формируется сеть~$G(1)$, в~которой $d_k^*(1) := \Delta_k^0(0)$, $k\hm= \ol{1, E}$. 
В~сети~$G(1)$ для всех пар узлов $p_a \hm\in P$, $a \hm= \ol{1, M}$, определяется максимальный однопродуктовый поток.

\smallskip

\noindent
\textbf{Задача~2.} Найти 
$$
 z_a^0 (1) = \max \left\{ z_a | (\mathbf{z}, x)  \hm\in \mathcal{Z}(\mathbf{d^*(1)})\right\} 
 $$
{при дополнительных условиях }
$$z_m = 0\,,\ m \not = a\,,\ m = \ol {1, M}\,. 
$$

На основе последовательного решения задачи 2 вычисляются  коэффициенты нормировки
$$
 \theta_m^0(1) =  \fr{1}{y_m^0(1)} \ \mbox{для всех}  \  z_m^0 (1) > 0\,, \ m = \ol {1,  M}\,,
 $$
и формируется система уравнений для поиска распределения нагрузок
\begin{multline*}
 \beta^*(1)  \sum\limits_{m=1}^{M} \theta_m^0(1) \left(x_{mk}^0(1)+ x_{m(k+E)}^0(1)\right)   ={}\\
 {}=  \Delta_k^0(1),\ k= \ol{1, E}\,, \enskip 
\sum\limits_{k=1}^{E}  \Delta_k^0(1) = D(0). 
\end{multline*}
Вычисляются $\beta^*(1)$ и~$\Delta_k^0(1)$, $k= \ol{1, E}$, и~формируется сеть~$G(2)$, в~которой пропускные способности~$d_k^*(2)$ 
полагаются равными~$\Delta_k^0(1)$, т.\,е.\ $d_k^*(2):= \Delta_k^0(1)$,   $k\hm=\ol{1, E}$.  Для полученных решений находим
$$
 z_m^*(1) = \beta^*(1) \theta_m^0(1) z_m^0(1)\,,
 $$
  $p_m \hm\in P$~--- распределение межузловых потоков при равных значениях нагрузок~$y_m^0(1)$.
Все ребра сети~$G(2)$ загружены пол\-ностью, выделенный ресурс совпадает с~нагрузкой на ребро при равных межузловых нагрузках.

Норма вектора требуемых пропускных способностей в~<<новой>>  сети~$G(2)$:
$$
\left\|
\mathbf{\Delta}_k^0(1)\right\| = \left\| \mathbf{d}^*(2)\right\| = \left[ \sum\limits_{k=1}^{E} d_k^*(2)^2 \right]^{1/2} .
$$

\section{Вычислительный эксперимент}

\vspace*{-4pt}

Вычислительный эксперимент проводился на моделях сетевых систем, представленных на рис.~1. 
В~каждой сети 69~узлов.  Пропускные способности ребер~--- значения~$d_k(0)$~--- 
выбирались случайным образом из отрезка $[900, 999]$ и~совпадали для ребер, общих для обеих сетей. 
В~кольцевой сети пропускная способность каждого из до\-бав\-лен\-ных ребер внутреннего кольца составила~900. 
В~ходе вычислительного эксперимента проводилась нормировка, и~суммарная пропускная способность в~обеих сетях была одинакова:
$ \sum\nolimits_{k=1}^{E} d_k(0) \hm= D(0)\hm= 68\,256.$

\begin{figure*}[b] %fig1
\vspace*{-4pt}
\begin{center}
   \mbox{%
\epsfxsize=153.408mm
\epsfbox{mal-1.eps}
}
\end{center}
\vspace*{-9pt}
    \Caption{Базовая~(\textit{а}) и кольцевая~(\textit{б}) сети }
%\end{figure*}
%
%\begin{figure*} %fig2
\vspace*{9pt}
\begin{center}
   \mbox{%
\epsfxsize=153.476mm
\epsfbox{mal-3.eps}
}
\end{center}
\vspace*{-9pt}
\Caption{Результирующие нагрузки на ребра базовой~(\textit{а}) и~кольцевой~(\textit{б})  сети}
%\end{figure*}
%\begin{figure*} %fig3
\vspace*{9pt}
\begin{center}
   \mbox{%
\epsfxsize=162.263mm
\epsfbox{mal-5.eps}
}
\end{center}
\vspace*{-9pt}
\Caption{Распределение межузловых потоков в~базовой~(\textit{а}) и~кольцевой~(\textit{б}) сети}
\end{figure*}
    

Результаты вычислительных экспериментов представлены на рис.~2 и~3. 
Толщина ребер на рисунках пропорциональна результирующей нагрузке.
%В~состав внутреннего кольца на рис.~2, 4 вошли четыре узла, которые в~базовой сети были  \textit{висячими}, 
%и~соответствующие ребра образовали \textit{мостики}  для транзитных потоков.

В базовой сети основная нагрузка~--- 1400~единиц~--- приходится на радиальные ребра, которые исходят из центрального узла, 
а~на ребрах внешнего кольца не превышает~1250~единиц. При формировании кольцевой сети в~состав внутреннего кольца включаются 4~узла, которые 
в~базовой сети были \textit{висячими}. Соответствующие ребра образуют \textit{мостики}  для передачи транзитных потоков между внутренним и~внешним кольцом. 
Нагрузка на ребра внут\-рен\-не\-го кольца составляет~2050~единиц, на внеш\-нем не превышает~800, а~на реб\-рах-\textit{мос\-ти\-ках}~---~1500. 
В~обеих сетях минимальная на\-груз\-ка достигается на \textit{висячих} ребрах: 300~единиц для базовой и~400 для кольцевой. 
Указанные значения определяются  собственными информационными  потоками, которые передаются из каждого узла. 
При этом транзитные нагрузки, например,  на ребрах внут\-рен\-не\-го кольца, на порядок превышают собственные потоки инцидентных узлов.





Для полученных значений нагрузок вычисляются межузловые потоки~$z_m^*(1)$.  
Результирующие диаграммы  представлены на рис.~3, на котором приведены значения~$z_l$, $l \hm= \ol{1, m}$, 
упорядоченные  по величине от большего к~меньшему. На вертикальной оси  по невозрастанию откладываются~$z_l$, 
а~по горизонтальной указываются порядковые номера в~данной  упорядоченной последовательности
$$ 
\pi(l) = \fr{l}{m} \  \mbox{для}\ l = \ol{1, M}\,,
$$
где $M$~--- общее число элементов в~множестве пар-кор\-рес\-пон\-ден\-тов~$P$. 
Результаты вычислительных экспериментов и~диаграммы, представленные на рис.~3, 
свидетельствуют, что как суммарный поток, так и~медианное значение межузлового потока на~50\% 
больше в~кольцевой сети при равных пропускных  способностях в~обеих сетях. Дело в~том, что при вычислении максимального потока,
 а~в~дальнейшем при поиске равных нагрузок используются все пути, проходящие через минимальные сечения. 
 В~кольцевой сети маршруты соединения оказываются короче, чем в~базовой, для большого числа пар узлов, что уменьшает 
 межузловые нагрузки и~удельные затраты ресурсов. В~базовой сети  только~46~ребер пригодны для транзитной передачи, а~в~кольцевой~--- 58, 
 что только на~25\%  больше, однако позволяет на~30\% увеличить суммарный межузловой поток и~более чем на~50\% 
 медианное значение. Величина суммы межузловых потоков в~базовой сети
$\sum\nolimits_{m=1}^{M} z_m^*(1)\hm= 9\,400$,
а~в~кольцевой  $\sum\nolimits_{m=1}^{M} z_m^*(1)\hm= 12\,400$.
Значение медианы в~базовой сети  $z^{=} \hm= 1{,}5$,
а~в~кольцевой  $z^{=} \hm= 2{,}3$.

Результаты расчетов на рис.~3 показывают, что для~80\%~пар межузловые потоки близки к~медианному значению, а~для~20\%~--- 
в~несколько раз превышают его. Для пар узлов, кратчайший путь между которыми состоит из одного или двух ребер, межузловые потоки превышают медианное значение.
Следует отметить, что при использовании \textit{последовательного максиминного} правила распределения потоков и~нагрузок близко расположенные 
корреспонденты также оказываются в~привилегированном положении и~получают  преимущество при \textit{уравнительных} способах \textit{дележа}~[1, 2]. 

\section{Заключение}

Предложенный метод позволяет проводить оценку  и~сравнение различных способов формирования сети, построенной  
на основе арендованных каналов связи при сохранении их общего числа. В~рассмотренном примере исходные сети, представленные на рис.~1 
имеют на всех ребрах почти равное число каналов. На рис.~2 представлены вторичные сети, схема построения которых позволяет выделить
 всем парам одинаковое число каналов. Кроме того, изменение структуры (графа) сети и~переход от базовой к~кольцевой сети
  позволяет значительно изменить  межузловые потоки, хотя общее число каналов совпадает при одинаковых ресурсных ограничениях. 

{\small\frenchspacing
 {%\baselineskip=10.8pt
 %\addcontentsline{toc}{section}{References}
 \begin{thebibliography}{9}
    


\bibitem{Mal20-5}  %1
\Au{Малашенко~Ю.\,Е., Назарова~И.\,А.} 
Оценка предельных распределений  пропускной способности в~многопользовательской сети при  передаче  межузловых потоков  по кратчайшим  маршрутам~// 
Известия РАН. Тео\-рия и\-сис\-те\-мы управ\-ле\-ния, 2022. №\,5. С.~79--89.

\bibitem{Mal22-3}  %2
\Au{Малашенко~Ю.\,Е., Назарова~И.\,А.} 
Анализ распределения нагрузки и~межузловых потоков при различных стратегиях  маршрутизации  в~многопользовательской сети~// 
Известия РАН. Тео\-рия и~сис\-те\-мы управ\-ле\-ния, 2022. №\,6. С.~112--122.
    
\bibitem{Ogryczak2014} 
\Au{Ogryczak W., Luss~H., Pioro~M., Nace~D., Tomaszewski~A.} Fair optimization and networks: A~survey~// J.~Appl. Math., 2014. Vol.~3. P.~1--25.
    
\bibitem{Salimifard2020} 
\Au{Salimifard K., Bigharaz~S.} The multicommodity network flow problem: State of the art classification, applications, and solution methods~// 
Operational Research, 2020. Vol.~22. Iss.~2.  P.~1--47.
    
\bibitem{Luss2012}  
\Au{Luss H.} Equitable resource allocation: Models, algorithms, and applications.~--- Hoboken, NJ, USA: John Wiley \&~Sons, 2012. 420~p.
    
\bibitem{Yen} 
\Au{Йенсен П., Барнес~Д.} Потоковое программирование~/ Пер. с~англ.~--- М.: Радио и~связь, 1984. 392~с. 
(\Au{Jensen~P.\,A., Barnes~J.\,W.} Network flow programming.~--- New York, NY, USA: Wiley, 1980. 408~p.)
  \end{thebibliography}

 }
 }

\end{multicols}

\vspace*{-6pt}

\hfill{\small\textit{Поступила в~редакцию 13.10.22}}

%\vspace*{8pt}

%\pagebreak

\newpage

\vspace*{-28pt}

%\hrule

%\vspace*{2pt}

%\hrule

%\vspace*{-2pt}

\def\tit{ESTIMATES OF THE RESOURCE DISTRIBUTION IN~THE~MULTIUSER NETWORK WITH~EQUAL INTERNODAL LOADS}


\def\titkol{Estimates of the resource distribution in~the~multiuser network with~equal internodal loads}


\def\aut{Yu.\,E.~Malashenko and~I.\,A.~Nazarova}

\def\autkol{Yu.\,E.~Malashenko and~I.\,A.~Nazarova}

\titel{\tit}{\aut}{\autkol}{\titkol}

\vspace*{-8pt}


\noindent
Federal Research Center ``Computer Science and Control'' of the Russian Academy 
of Sciences, 44-2~Vavilov Str., Moscow 119333, Russian Federation


\def\leftfootline{\small{\textbf{\thepage}
\hfill INFORMATIKA I EE PRIMENENIYA~--- INFORMATICS AND
APPLICATIONS\ \ \ 2023\ \ \ volume~17\ \ \ issue\ 1}
}%
 \def\rightfootline{\small{INFORMATIKA I EE PRIMENENIYA~---
INFORMATICS AND APPLICATIONS\ \ \ 2023\ \ \ volume~17\ \ \ issue\ 1
\hfill \textbf{\thepage}}}

\vspace*{3pt}   



\Abste{A~method for estimating resources with an equalizing distribution of internodal loads in a~multiuser network is proposed. 
Within the framework of a~formal mathematical model, the capacity of edges is considered as components of 
a~vector of resources that are required for the transmission of different types of flows. An algorithmic procedure for the redistribution and 
usage of capacity with equal quota of resources for all pairs is proposed. When searching for the corresponding edge loads, the 
values of the maximum single-product flows for each pair of nodes are determined. In the course of computational experiments, 
the total resource is considered to be set for networks with various structural features.} 

\KWE{multicommodity flow model; network resource distribution and internodal loads; network peak load}

 \DOI{10.14357/19922264230111} 

%\vspace*{-16pt}

%\Ack
%\noindent

  

%\vspace*{4pt}

  \begin{multicols}{2}

\renewcommand{\bibname}{\protect\rmfamily References}
%\renewcommand{\bibname}{\large\protect\rm References}

{\small\frenchspacing
 {%\baselineskip=10.8pt
 \addcontentsline{toc}{section}{References}
 \begin{thebibliography}{9} 

\bibitem{2-mal-1}
\Aue{Malashenko, Yu.\,E., and I.\,A.~Nazarova.} 2022. Estimate of resource distribution with the shortest paths  in the multiuser network. 
 \textit{J.~Comput. Syst. Sci. Int.} 61(4):599--610.    
 
 \bibitem{1-mal-1}
\Aue{Malashenko, Yu.\,E., and I.\,A.~Nazarova.}
 2022. Analysis of load distribution and internodal flows under various routing strategies in the multiuser network. \textit{J.~Comput. Syst. Sci. Int.} 61(6):956--965.  
 
\bibitem{3-mal-1}
\Aue{Ogryczak, W., H.~Luss, and M.~Pioro.} 2014. Fair optimization and networks: A~survey.  \textit{J.~Appl. Math.} 3:1--25.
\bibitem{4-mal-1}
\Aue{Salimifard, K., and S.~Bigharaz.} 
 2020. The multicommodity network flow problem: State of the art classification, applications, and solution methods.  \textit{Operational Research} 22(2):1--47.
\bibitem{5-mal-1}
\Aue{Luss, H.} 2012. \textit{Equitable resource allocation: Models, algorithms, and applications}. Hoboken, NJ: John Wiley \&~Sons. 420~p.
\bibitem{6-mal-1}
\Aue{Jensen, P.\,A., and J.\,W.~Barnes.} 1980. \textit{Network flow programming}. New York, NY: Wiley. 408~p.
 \end{thebibliography}

 }
 }

\end{multicols}

\vspace*{-6pt}

\hfill{\small\textit{Received October 13, 2022}}

\Contr

\noindent
\textbf{Malashenko Yuri E.} (b.\ 1946)~--- 
Doctor of Science in physics and mathematics, principal scientist, Federal Research Center ``Computer Science and Control'' 
of the Russian Academy of Sciences, 44-2~Vavilov Str., Moscow 119333, Russian Federation; \mbox{malash09@ccas.ru} 

\vspace*{4pt}

\noindent
\textbf{Nazarova Irina A.} (b.\ 1966)~--- 
Candidate of Science (PhD) in physics and mathematics, scientist, Federal Research Center ``Computer Science and Control'' 
of the Russian Academy of Sciences, 44-2~Vavilov Str., Moscow 119333, Russian Federation; \mbox{irina-nazar@yandex.ru}


   
\label{end\stat}

\renewcommand{\bibname}{\protect\rm Литература} 