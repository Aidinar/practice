\def\stat{bening}

\def\tit{ОБ ОЦЕНКАХ ФУНКЦИЙ КОНЦЕНТРАЦИИ РЕГУЛЯРНЫХ СТАТИСТИК,
ПОСТРОЕННЫХ ПО~ВЫБОРКАМ СЛУЧАЙНОГО ОБЪЕМА$^*$}

\def\titkol{Об оценках функций концентрации регулярных статистик,
построенных по~выборкам случайного объема}

\def\autkol{В.\,Е.~Бенинг, Н.\,К.~Галиева, В.\,Ю.~Королев}

\def\aut{В.\,Е.~Бенинг$^1$, Н.\,К.~Галиева$^2$, В.\,Ю.~Королев$^3$}

\titel{\tit}{\aut}{\autkol}{\titkol}

{\renewcommand{\thefootnote}{\fnsymbol{footnote}}\footnotetext[1]
{Работа поддержана
Российским фондом фундаментальных исследований (проекты
11-01-00515а, 11-07-00112а, 11-01-12026-офи-м и 12-07-00115а), Министерством
образования и науки (госконтракт 16.740.11.0133).}}

\renewcommand{\thefootnote}{\arabic{footnote}}
\footnotetext[1]{Факультет вычислительной
математики и кибернетики Московского государственного университета
им.\ М.\,В.~Ломоносова; Институт проблем информатики Российской
академии наук, bening@cs.msu.su}
\footnotetext[2]{Казахстанский филиал Московского государственного
университета им.\ М.\,В.~Ломоносова, nurgul\_u@mail.ru}
\footnotetext[3]{Факультет вычислительной
математики и кибернетики Московского государственного университета
им.\ М.\,В.~Ломоносова; Институт проблем информатики Российской
академии наук, vkorolev@cs.msu.su}



\Abst{Приведены оценки  функций концентрации (ф.к.)\ регулярных
статистик, построенных по выборкам случайного объема.}

\KW{функция концентрации; случайная сумма;
асимптотически нормальная статистика; распределение Стьюдента;
распределение Лапласа}

\vskip 14pt plus 9pt minus 6pt

      \thispagestyle{headings}

      \begin{multicols}{2}

            \label{st\stat}



\section{Введение}

Асимптотическим свойствам распределений сумм случайного числа
случайных величин (с.в.)\ посвящено много работ (см., например,~[1--6]).
Такого рода суммы находят широкое применение в страховании,
экономике, биологии и~т.\,п.~\cite{3-ben, 5-ben, 6-ben}. Однако в математической
статистике и ее приложениях часто встречаются статистики, которые не
являются  суммами наблюдений. Примерами могут служить ранговые
статистики, $U$-ста\-ти\-сти\-ки, линейные комбинации порядковых статистик\linebreak
($L$-ста\-ти\-сти\-ки) и~т.\,п. При этом в статистике час\-то возникают
ситуации, когда размер выборки не является заранее определенным и
может рас\-смат\-ри\-вать\-ся как случайный. Например, при испытании на
надежность число отказавших приборов за определенное время является
случайной величиной. Вообще, в подавляющем большинстве ситуаций,
связанных с анализом и обработкой экспериментальных данных, можно
считать, что число случайных факторов, влияющих на наблюдаемые
величины, само является случайным и изменяется от наблюдения к
наблюдению. Поэтому вместо различных вариантов центральной
предельной теоремы, обосновывающих нормальность предельного
распределения в классической статистике, в таких ситуациях следует
опираться на их аналоги для выборок случайного объема. Это делает
естественным изучение асимптотического поведения распределений
статистик достаточно общего вида, основанных на выборках случайного
объема. Примерами могут служить работы~\cite{7-ben, 8-ben}, в которых
рассматривались асимптотические свойства распределений выборочных
квантилей, построенных по выборкам случайного объема.

В данной работе получены оценки для  функций концентрации статистик,
построенных по выборкам случайного объема. Эти оценки
непосредственно зависят от скорости сходимости функций распределения
таких статистик к предельному закону.

В работе приняты следующие обозначения: $\r$~--- множество
вещественных чисел; $\N$~--- множество натуральных чисел; $\Phi(x)$ и
$\varphi(x)$~--- соответственно функция распределения  и плотность
стандартного нормального закона.
%Символ $\Longrightarrow$ будет обозначать сходимость по распределению.

В разд.~2 рассмотрены случаи предельных распределений Стьюдента и
Лапласа, получены аппроксимации для функций концентрации в этих
случаях.

Как известно, ф.к.\ с.в.\ $Z$ называется функция (см., например,~[9, с.~53])
\begin{equation}
Q_Z(\lambda)=\sup_{x\in\r}{\sf P}(x\leqslant Z\leqslant x+\lambda)\,,\enskip \lambda
\geqslant 0\,.
\label{e1.1-ben}
\end{equation}
Очевидно, ф.к.\ $Q_Z(\lambda)$~--- неубывающая функция~$\lambda$,
удовлетворяющая неравенству
$$
0\leqslant Q_Z(\lambda)\leqslant 1
$$
для любого  $\lambda\hm\geqslant0$. Из  ее определения следует оценка
\begin{equation*}
\sup\limits_{x\in\r}{\sf P}(Z= x)=Q_Z(0)\leqslant Q_Z(\lambda)\,,\enskip
\lambda\geqslant0\,,
%\label{e1.2-ben}
\end{equation*}
позволяющая оценить максимальную вероятность отдельного значения
с.в.~$Z$. Из определения~(\ref{e1.1-ben}) ф.к.\ следует, что для любого числа
$a\hm\in\r$    и любого $b\hm>0$  справедливы тождества:
\begin{equation*}
Q_{Z+a}(\lambda)\equiv Q_Z(\lambda)\,;\quad  Q_{bZ}(\lambda)\equiv
Q_Z\left(\fr{\lambda}{b}\right)\,.
%\label{e1.3-ben}
\end{equation*}
Применениям функций концентрации к проблемам слабой сходимости
посвящены гл.~3 и~4 книги~[10].

\medskip

\noindent
\textbf{Лемма 1.1} {\it Пусть $\xi$ и $\eta$~--- две с.в.,
тогда}
$$
\sup\limits_{\lambda\geqslant0}|Q_{\xi}(\lambda)-Q_{\eta}(\lambda)|\leqslant  4
\sup\limits_{x\in\r}|{\sf P}(\xi<x)-{\sf P}(\eta<x)|\,.
$$


\smallskip

\noindent
Д\,о\,к\,а\,з\,а\,т\,е\,л\,ь\,с\,т\,в\,о\,.\ \  Обозначим
$$
\delta=\sup\limits_{x\in\r}\left\vert {\sf P}(\xi<x)-{\sf P}(\eta<x)\right\vert\,.
$$
Тогда для  любого $\lambda\hm\geqslant0$ имеем:
\begin{multline}
Q_{\xi}(\lambda)={}\\
{}=\sup\limits_{x\in\r}\left[{\sf P}(\xi=x+\lambda)+{\sf
P}(\xi<x+\lambda)-{\sf P}(\xi<x)\right]={}\\
{}
=\sup\limits_{x\in\r}\left[\left({\sf P}(\eta=x+ \lambda)+ {\sf P}(\eta<x+
\lambda)-{\sf P}(\eta<x)\right)+{}\right.\\
{}+\left({\sf P}(\xi=x+\lambda)-{\sf
P}(\eta=x+\lambda)\right)+
\left({\sf P}(\xi<x+\lambda)-{}\right.\\
\left.\left.{}-
{\sf P}(\eta<x+\lambda)\right)+ \left({\sf
P}(\eta<x)-{\sf P}(\xi<x)\right)\right]\leqslant{}
\\
{}\leqslant \sup\limits_{x\in\r}\left\vert{\sf P}(\eta=x+\lambda)+{\sf P}(\eta<x+
\lambda)-{\sf P}(\eta<x)\right\vert+{}\\
{}+\sup\limits_ {x\in\r}\left\vert{\sf P}(\xi=x+
\lambda)-{\sf P}(\eta=x+\lambda)\right\vert+{}
\\
{}+\sup\limits_ {x\in\r}\left\vert{\sf P}(\xi<x+\lambda)-{\sf P}(\eta<x+
\lambda)\right\vert+{}\\
{}+\sup_{x\in\r}\left\vert{\sf P}(\eta<x)-{\sf P}(\xi<
x)\right\vert\leqslant{}\\
{}
\leqslant  Q_{\eta}(\lambda)+2\delta+\sup\limits_{y\in\r}\left\vert{\sf P}(\xi=y)-
{\sf P}(\eta=y)\right\vert\,.
\label{e1.4-ben}
\end{multline}
Далее, для любого $y\in\r$ имеем:
\begin{multline*}
\left\vert{\sf P}(\xi=y)-{\sf P}(\eta=y)\right\vert=
\left\vert\left({\sf P}(\xi\leqslant
y)-{}\right.\right.\\
\left.\left.{}-{\sf P}(\xi<y)\right)-\left({\sf P}(\eta\leqslant  y)-{\sf P}(\eta<
y)\right)\right\vert \leqslant{}\\
{}
\leqslant \left\vert{\sf P}(\xi\leqslant  y)-{\sf P}(\eta\leqslant  y)\right\vert+\left\vert{\sf
P}(\xi <y)-{\sf P}(\eta<y)\right\vert={}
\\
{}=\left\vert\lim\limits_{\epsilon\downarrow 0}{\sf P}(\xi<y+ \epsilon)-
\lim\limits_{\epsilon\downarrow 0}{\sf P}(\eta<y+\epsilon)\right\vert+{}\\
{}+\left\vert{\sf
P}(\xi< y)-{\sf P}(\eta<y)\right\vert={}
\\
=\left\vert\lim\limits_{\epsilon\downarrow 0}\left[{\sf P}(\xi<y+\epsilon)-{\sf
P}(\eta<y+\epsilon)\right]\right\vert+{}\\
{}+\left\vert{\sf P}(\xi<y)-{\sf P}(\eta<
y)\right\vert\leqslant{}
\\
{}\leqslant \lim\limits_{\epsilon\downarrow 0}\left\vert{\sf P}(\xi<y+\epsilon)-{\sf P}
(\eta<y+\epsilon)\right\vert+\delta\leqslant{}\\
{}\leqslant \lim\limits_{\epsilon\downarrow 0}
\sup\limits_y\left\vert{\sf P}(\xi<y+\epsilon)-{\sf P}(\eta<y+
\epsilon)\right\vert+\delta\leqslant 2\delta\,.
\end{multline*}
Поэтому
\begin{equation}
\sup_{y\in\r}\left\vert{\sf P}(\xi=y)-{\sf P}(\eta=y)\right\vert\leqslant 2
\delta\,.
\label{e1.5-ben}
\end{equation}
Более того, пример двух случайных величин~$\xi$ и~$\eta$ таких, что
${\sf P}(\xi=0)\hm={\sf P}(\xi=1)\hm=1/2$ и ${\sf P}(\eta\hm=1/2)\hm=1$
показывает, что оценка~(\ref{e1.5-ben}) неулучшаема.

Используя соотношения~(\ref{e1.5-ben}) и~(\ref{e1.4-ben}), получим:
$$
Q_{\xi}(\lambda)\leqslant  Q_{\eta}(\lambda)+4\delta\,.
$$
Точно так же убеждаемся, что справедливо неравенство
$$
Q_{\eta}(\lambda)\leqslant  Q_{\xi}(\lambda)+4\delta\,.
$$
Лемма доказана.

\smallskip

\noindent
\textbf{Замечание~1.1.} Если функцию концентрации определить как
$$
\tilde Q_{\xi}(\lambda)=\sup\limits_{x\in\r}{\sf P}(x\leqslant \xi<x+\lambda)\,,\enskip
\lambda>0,
$$
то, как несложно убедиться, аналог неравенства, устанавливаемого
леммой~1.1, будет иметь вид:
$$
\sup_{\lambda>0}|\tilde Q_{\xi}(\lambda)-\tilde Q_{\eta}(\lambda)|
\leqslant  2\sup_{x\in\r}|{\sf P}(\xi<x)-{\sf P}(\eta<x)|,
$$
т.\,е.\ коэффициент в правой части будет в два раза меньше.

\smallskip

Напомним определение унимодальности распределения вероятности по
Хинчину (см., например,~[11, с.~186] или [10, с.~12], [12, с.~118]).
Согласно этому определению, с.в.~$\xi$ имеет
унимодальное (одновершинное) распределение, если существует точка
$x_0$ такая, что функция распределения (ф.р.) $F_{\xi}(x)$
случайной величины~$\xi$ выпукла при $x\hm<x_0$, а функция
$1-F_{\xi}(x)$ выпукла при $x\hm>x_0$. При этом точка $x_0$ называется
модой случайной величины~$\xi$. Точка~$x_0$ может быть точкой
разрыва ф.р.\ $F_{\xi}(x)$, но вне точки~$x_0$ одновершинность
предполагает существование у ф.р.\ $F_{\xi}(x)$ плотности, которая
монотонна в интервалах $x\hm<x_0$ и $x\hm>x_0$. Несложно убедиться, что
любая унимодальная функция распределения непрерывна всюду, быть
может, за исключением моды.

\smallskip

\noindent
\textbf{Лемма~1.2.} \textit{Пусть $\xi$~--- с.в.\ с
симметричным унимодальным распределением. Тогда для $\lambda\hm>0$}
$$
Q_{\xi}(\lambda)={\sf P}\left(|\xi|<\fr{\lambda}{2}\right)\,.
$$


\smallskip

\noindent
Д\,о\,к\,а\,з\,а\,т\,е\,л\,ь\,с\,т\,в\,о\,.\ \  В~условиях леммы достаточно показать, что
$$
\arg\max\limits_{x\in\r}{\sf P}(x\leqslant \xi\leqslant x+\lambda)=-\fr{\lambda}{2}\,.
$$
Пусть $F_{\xi}(x)$~--- ф.р.\ с.в.~$\xi$. 
В~силу симметричности распределения с.в.~$\xi$
ее мода равна нулю. Пусть $0\hm<a\hm<\lambda/2$. Тогда точки
$\pm{\lambda}/{2}$, $-{\lambda}/{2}+a$ и
${\lambda}/{2}+a$ являются точками непрерывности функции
распределения $F_{\xi}(x)$. Поэтому имеем:
\begin{multline}
{\sf P}\left(-\fr{\lambda}{2}\leqslant \xi\leqslant \fr{\lambda}{2}\right)-{\sf
P}\left(-\fr{\lambda}{2}+a\leqslant \xi\leqslant \fr{\lambda}{2}+a\right)={}
\\
{}=F_{\xi}\left(\fr{\lambda}{2}\right)-
F_{\xi}\left(-\fr{\lambda}{2}\right) - F_{\xi}\left(\fr{\lambda}{2}
+a\right)+{}\\
{}+F_{\xi}\left(-\fr{\lambda}{2}+a\right)=
F_{\xi}\left(-\fr{\lambda}{2}+a\right)-
F_{\xi}\left(-\fr{\lambda}{2}\right) +{}\\
{}+
F_{\xi}\left(\fr{\lambda}{2}\right) - F_{\xi}\left(\fr{\lambda}{2} +
a\right) =
1-F_{\xi}\left(\fr{\lambda}{2}-a\right)- 1+{}\\
{}+
F_{\xi}\left(\fr{\lambda}{2}\right) +
F_{\xi}\left(\fr{\lambda}{2}\right) - F_{\xi}\left(\fr{\lambda}{2} +
a\right) =\left[F_{\xi}\left(\fr{\lambda}{2}\right)-{}\right.\\
\left.{}-
F_{\xi}\left(\fr{\lambda}{2}-a\right)\right]-
\left[F_{\xi}\left(\fr{\lambda}{2}+a\right)-
F_{\xi}\left(\fr{\lambda}{2}\right)\right]={}
\\
{}=2F_{\xi}\left(\fr{\lambda}{2}\right)-
\left[F_{\xi}\left(\fr{\lambda}{2}-a\right)+
F_{\xi}\left(\fr{\lambda}{2}+a\right)\right]\,.
\label{e1.6-ben}
\end{multline}
Здесь третье равенство имеет место в силу сим\-мет\-рич\-ности
распределения с.в.~$\xi$. Поскольку распределение
с.в.~$\xi$ симметрично и унимодально, ф.р.\ $F_{\xi}(z)$ вогнута при $z\hm>0$, т.\,е.\ для любых $0\hm<
x\hm<y\hm<\infty$ и $\alpha\hm\in(0,1)$
\begin{equation}
F_{\xi}\left(\alpha x+(1-\alpha)y\right)\geqslant\alpha F_{\xi}(x)+(1-
\alpha)F_{\xi}(y)\,.
\label{e1.7-ben}
\end{equation}
Положим в  неравенстве~(\ref{e1.7-ben}) $x\hm=\lambda/2\hm-a$, $y\hm=\lambda/2\hm+a$,
$\alpha\hm=1/2$. Получим:
$$
2F_{\xi}\left(\fr{\lambda}{2}\right)\geqslant
F_{\xi}\left(\fr{\lambda}{2}-a\right)+
F_{\xi}\left(\fr{\lambda}{2}+a\right)\,.
$$
Продолжая~(\ref{e1.6-ben}), можно заметить, что для рас\-смат\-ри\-ва\-емых значений
$a$ справедливо неравенство:
\begin{multline}
{\sf P}\left(-\fr{\lambda}{2}\leqslant \xi\leqslant \frac{\lambda}{2}\right)-{\sf
P}\left(-\fr{\lambda}{2}+a\leqslant \xi\leqslant \fr{\lambda}{2}+a\right)\geqslant{}\\
{}\geqslant
0\,.\label{e1.8-ben}
\end{multline}
В силу симметричности распределения случай $-\lambda/2\hm<a\hm<0$
рассматривается аналогично. В~случае $a\hm>\lambda/2$ неравенство~(\ref{e1.8-ben})
имеет место в силу вогнутости $F_{\xi}(x)$ при $x\hm>0$, а в случае $a\hm<-\lambda/2$ 
неравенство~(\ref{e1.8-ben}) выполнено в силу выпуклости
$F_{\xi}(x)$ при $x\hm<0$. Лемма доказана.

\smallskip

\noindent
\textbf{Замечание~1.2.} Заметим, что если  с.в.~$\xi$  имеет
непрерывное симметричное унимодальное распределение, то
$Q_{\xi}(0)\hm=0$.

\smallskip

Пусть $X_1,X_2,\ldots$~--- независимые одинаково распределенные
с.в.\ с ${\sf E}X_1\hm=\mu$, $0\hm<{\sf D}X_1\hm=\sigma^2$ и
${\sf E}|X_1\hm-\mu|^3\hm=\beta^3\hm<\infty$. Для натурального~$n$ обозначим
$$
S_n=X_1+\cdots+X_n\,.
$$

\smallskip

\noindent
\textbf{Теорема~1.1.} \textit{Для любого $n\in\N$ имеет место неравенство}
$$
\sup\limits_{\lambda\geqslant0}\left\vert Q_{S_n}(\lambda)-
2\Phi\left(\fr{\lambda}{2\sigma\sqrt{n}}\right)+1\right\vert\leqslant 1{,}8992
\fr{\beta^3}{\sigma^3\sqrt{n}}\,.
$$

\smallskip

\noindent
Д\,о\,к\,а\,з\,а\,т\,е\,л\,ь\,с\,т\,в\,о\,.\ \  В~лемме~1.1 положим ${\sf P}(\xi\hm<x)\hm= \Phi(x)$,
$$
\eta=\eta_n=\fr{S_n-n\mu}{\sigma\sqrt{n}}\,,
$$
так что
\begin{multline*}
Q_{S_n}(\lambda)=\sup\limits_{x\in\r}{\sf P}(x\leqslant S_n\leqslant x+\lambda)={}\\
{}=
\sup\limits_{x\in\r}{\sf P}\left(\fr{x-n\mu}{\sigma\sqrt{n}}\leqslant \eta_n \leqslant
\fr{x-n\mu+\lambda}{\sigma\sqrt{n}}\right)={}
\\
{}=\sup\limits_{y\in\r}{\sf P}\left(y\leqslant \eta_n\leqslant y+
\fr{\lambda}{\sigma\sqrt{n}}\right)=
Q_{\eta_n}\left(\fr{\lambda}{\sigma\sqrt{n}}\right)\,.
\end{multline*}
При этом в силу неравенства Бер\-ри--Эс\-се\-ена (см.~[13])
\begin{equation*}
\sup\limits_{x\in\r}\big|{\sf P}(\eta_n<x)-\Phi(x)|\leqslant 0{,}4748
\fr{\beta^3}{\sigma^3\sqrt{n}}\,.
%\label{e1.9-ben}
\end{equation*}
Теперь утверждение теоремы следует из лемм~1.1 и~1.2 в силу
симметричности и унимодальности нормального распределения. Теорема
доказана.

\smallskip

\noindent
\textbf{Следствие 1.1.} \textit{Для любых $\lambda\geqslant0$ и $n\in\N$ имеют
место неравенства}
\begin{multline*}
2\Phi\left(\fr{\lambda}{2\sigma\sqrt{n}}\right)-1-1{,}8992
\fr{\beta^3}{\sigma^3\sqrt{n}}\leqslant Q_{S_n}(\lambda)\leqslant{}\\
{}\leqslant
2\Phi\left(\fr{\lambda}{2\sigma\sqrt{n}}\right)-1+1{,}8992
\fr{\beta^3}{\sigma^3\sqrt{n}}\,.
\end{multline*}

\smallskip

Будем говорить, что статистика $T_n$ (т.\,е.\ измеримая функция от
наблюдений $X_1,\ldots,X_n$) асимптотически нормальна, если
существуют $\delta\hm>0$, $\nu\hm\in\r$ и $\mu\hm\in\r$ такие, что для любого
$x\hm\in\r$ справедливо соотношение:
\begin{equation}
{\sf P}\left(\delta n^{\nu}(T_n-\mu)<x\right)\longrightarrow\Phi(x)\,, \enskip
n\to\infty\,. \label{e1.10-ben}
\end{equation}
Предположим, что известна оценка скорости сходимости в~(\ref{e1.10-ben}) вида
\begin{equation}
\sup\limits_{x\in\r}\big|{\sf P}\left(\delta n^{\nu}(T_n-\mu)<x\right)-
\Phi(x)\big|\leqslant \fr{C}{n^{\gamma}}\,,\label{e1.11-ben}
\end{equation}
где $C>0$, $\gamma\hm>0$.

\smallskip

\noindent
\textbf{Теорема~1.2.} \textit{Предположим, что статистика $T_n$
удовлетворяет соотношению}~(\ref{e1.11-ben}). \textit{Тогда для любого $n\hm\in\N$ имеет
место неравенство}:
$$
\sup\limits_{\lambda\geqslant0}\Big|Q_{T_n}(\lambda)-
2\Phi\left(\fr{\lambda\delta n^{\nu}}{2}\right)+1\Big|\leqslant
\fr{4C}{n^{\gamma}}\,.
$$

\smallskip

\noindent
Д\,о\,к\,а\,з\,а\,т\,е\,л\,ь\,с\,т\,в\,о\,.\ \  Положим ${\sf P}(\xi<x)\hm=\Phi(x)$, $\eta\hm=
\eta_n\hm=\delta n^{\nu}(T_n-\theta)$. Тогда
\begin{multline*}
Q_{T_n}(\lambda)=\sup\limits_{x\in\r}{\sf P}(x\leqslant T_n\leqslant x+\lambda)={}\\
{}=
\sup\limits_{x\in\r}{\sf P}\left(\delta n^{\nu}(x-\mu)\leqslant \eta_n\leqslant \delta
n^{\nu}(x+\lambda-\mu)\right)={}
\\
{}=\sup\limits_{y\in\r}{\sf P}\left(y\leqslant  \eta_n\leqslant y+\lambda\delta
n^{\nu}\right)=Q_{\eta_n}\left(\lambda\delta n^{\nu}\right).
\end{multline*}
Теперь утверждение теоремы вытекает из лемм~1.1 и~1.2. Теорема
доказана.

\smallskip

\noindent
\textbf{Следствие 1.2.} \textit{Для любых $\lambda\hm\geqslant0$ и $n\hm\in\N$ имеют
место неравенства}:
\begin{multline*}
2\Phi\left(\fr{\lambda\delta n^{\nu}}{2}\right)-1-
\fr{4C}{n^{\gamma}}\leqslant  Q_{T_n}(\lambda)\leqslant{}\\
{}\leqslant
2\Phi\left(\fr{\lambda\delta n^{\nu}}{2}\right)-1+
\fr{4C}{n^{\gamma}}\,.
\end{multline*}

\smallskip

\section{Статистики, построенные по~выборкам случайного объема}

Рассмотрим с.в.\  $N_1, N_2, \ldots$ и  $X_1, X_2, \ldots$, заданные
на одном и том же вероятностном пространстве $(\Omega, {\cal A},
{\sf P})$. В~статистике с.в.\ $X_1, X_2, \ldots X_n$ имеют смысл
наблюдений, $n$~--- неслучайный объем выборки, а с.в.\ $N_n$~---
случайный объем выборки, зависящий от натурального параметра $n\hm\in\N$. 
Например, если с.в.~$N_n$ имеет гео\-мет\-ри\-че\-ское распределение вида
$$
{\sf P}(N_n=k)=\fr{1}{n}\left(1-\fr{1}{n}\right)^{k-1}\,,\enskip
k\in\N\,,
$$
то ${\sf E}N_n\hm=n$ и, значит, среднее значение случайного объема
выборки равно~$n$.

Предположим, что при каждом  $n\hm\geqslant 1$ с.в.~$N_n$ принимают только
натуральные значения, т.\,е.\ $N_n\hm\in \N$, и независимы от
последовательности с.в.\ $X_1, X_2, \ldots$ Всюду далее считаем с.\,в.\
$X_1, X_2, \ldots$\linebreak  независимыми одинаково распределенными и\linebreak
имеющими ф.р.~$F(x)$. Обозначим через  $T_n\hm=T_n(X_1,\ldots,X_n)$
некоторую статистику, т.\,е.\ действительную измеримую функцию от
наблюдений $X_1,\ldots,X_n$. Для каждого $n\hm\geqslant1$  определим с.в.\
$T_{N_n}$, полагая
$$
T_{N_n}(\omega)\equiv
T_{N_n(\omega)}(X_1(\omega),\ldots,X_{N_n(\omega)}(\omega))\,,\enskip
\omega\in\Omega\,.
$$
Таким образом,  $T_{N_n}$~--- это статистика, построенная на основе
статистики~$T_n$ по выборке случайного объема~$N_n$.

Справедливо следующее утверждение.

\smallskip

\noindent
\textbf{Теорема 2.1.} \textit{Предположим, что для некоторых $\mu\hm\in\r$,
$C\hm>0$, $\sigma\hm>0$, $\nu\hm\in\r$ и симметричной унимодальной ф.р.~$G(x)$ 
статистика $T_{N_n}$ удовлетворяет соотношению}:
$$
\sup\limits_{x\in\r}\big|{\sf P}\left(\sigma n^{\nu}(T_{N_n}-\mu)<x\right)-
G(x)\big|\leqslant \fr{C}{n^{\gamma}}\,.
$$
\textit{Тогда для любого $n\in\N$ имеет место
неравенство}:
$$
\sup\limits_{\lambda\geqslant0}\Big|Q_{T_{N_n}}(\lambda)-
2G\left(\fr{\lambda\sigma n^{\nu}}{2}\right)+1\Big|\leqslant
\fr{4C}{n^{\gamma}}\,.
$$

\smallskip

\noindent
Д\,о\,к\,а\,з\,а\,т\,е\,л\,ь\,с\,т\,в\,о\,.\ \  Непосредственно следует из доказательства
теоремы~1.2.

\smallskip

\noindent
\textbf{Следствие 2.1.} \textit{Для любых $\lambda\hm\geqslant0$ и $n\hm\in N$ имеют место
неравенства}
\begin{multline*}
2G\left(\fr{\lambda\sigma n^\nu}{2}\right)-1-\fr{4C}{n^\gamma} \leqslant
Q_{T_{N_n}}(\lambda)\leqslant{}\\
{}\leqslant 2G\left(\fr{\lambda\sigma n^\nu}{2}\right)
-1+\fr{4C}{n^\gamma}\,.
\end{multline*}

\subsection{Распределение Стьюдента}

В работе~[14] показано, что если случайный объем выборки $N_n$ имеет
отрицательное биномиальное распределение с параметрами  $p\hm=1/n$ и
$r\hm>0$,  т.\,е.\ (при $r\hm=1$ имеем гео\-мет\-ри\-че\-ское распределение)
$$
{\sf P}(N_n=k)=\fr{(k+r-2)\cdots r}{(k-1)!}\,\fr{1}{n^r}\!\left(\!1
-\fr{1}{n}\!\right)^{k-1}\!\!,\ k\in\N\,,
$$
то для асимптотически нормальной статистики $T_n$ справедливо
предельное соотношение~[14, следствие~2.1]:
\begin{equation*}
{\sf P}\left(\sigma\sqrt{n}(T_{N_n} - \mu) < x\right)\longrightarrow
G_{2r}\left(x\sqrt r\right)\,,\enskip n\to\infty\,, 
%\label{e2.1-ben}
\end{equation*}
где  $G_{2r}(x)$~--- функция распределения Стьюдента с параметром
$\gamma\hm=2r$, т.\,е.\ имеющего плотность вида
$$
p_{\gamma}(x)=\fr{\Gamma(\gamma+1/2)}{\sqrt{\pi\gamma}
\Gamma(\gamma/2)}\left( 1+\fr{x^2}{\gamma}\right)^{-(\gamma+1)/2}\,,
\enskip x\in\r\,,
$$
где  $\Gamma(\cdot)$~--- эйлерова гам\-ма-функ\-ция, а  $\gamma\hm>0$~---
параметр формы (если параметр~$\gamma$ натурален, то он называется
числом степеней свободы). В~рас\-смат\-ри\-ва\-емой ситуации он может быть
произвольно мал, что соответствует типичному тяжелохвостому
распределению. Отметим, что распределение Стьюдента является
симметричным унимодальным непрерывным распределением. Если
$\gamma\hm=2$, т.\,е.\ $r\hm=1$, то ф.р.~$G_2(x)$ выражается в явном виде:
$$
G_2(x) = \fr{1}{2}\left( 1+\fr{x}{\sqrt{2+x^2}} \right)\,,\enskip x\in\r\,.
$$
При $\gamma=1$ ($r\hm=1/2$) имеем распределение Коши:
$G_1(x)\hm=1/2\hm+(1/\pi)\arctg x$.

Если для ф.р.\ статистики~$T_n$ справедлива оценка ско\-рости
сходимости вида
\begin{multline}
\sup\limits_{x\in\r}\left\vert{\sf P}\left(\sigma\sqrt n(T_n - \mu) < x\right)
-\Phi(x)  \right\vert\leqslant \fr{C_0}{\sqrt n}\,,\\ 
C_0>0\,, \enskip n\in\N\,,
\label{e2.2-ben}
\end{multline}
где $C_0$~--- величина, не зависящая от~$n$, то, как показано в
работе~[15] (также см.~[16, теорема~6.11]), при $r\hm\in(0,1/2)$
справедлива оценка скорости сходимости статистики $T_{N_n}$ вида
\begin{multline*}
\sup\limits_{x\in\r}\left\vert{\sf P}\left(\sigma\sqrt n(T_{N_n} - \mu) <
x\right)-G_{2r}(x\sqrt r)\right\vert\leqslant  \fr{C_1}{n^{r}}\,, \\
n\in\N\,.
%\label{e2.3-ben}
\end{multline*}
Для случая $r\hm=1/2$ неравенство имеет вид:
\begin{multline*}
\sup\limits_{x\in\r}\left\vert\vphantom{\fr{1}{2}}
{\sf P}\left(\sigma\sqrt n(T_{N_n} - \mu) <
x\right)-{}\right.\\
\left.{}-\fr{1}{\pi}\,\arctg\left(x\sqrt{r}\right)-\fr{1}{2}  \right\vert\leqslant
C_2 \fr{\log n}{\sqrt{n}}\,,\enskip n>1\,.
%\label{e2.4-ben}
\end{multline*}
Если же $r>1/2$, то
\begin{multline*}
\sup\limits_{x\in\r}\Bigl|{\sf P}\left(\sigma\sqrt n(T_{N_n} - \mu) <
x\right)-G_{2r}\left(x\sqrt{r}\right)  \Bigr|\leqslant \fr{C_3}{\sqrt{n}}\,,\\
 n\in\N\,.
\end{multline*}
В частности, если $r\hm=1$, т.\,е.\ если $N_n$ имеет гео\-мет\-ри\-че\-ское
распределение с параметром $1/n$, то
\begin{multline*}
\sup\limits_{x\in\r}\left\vert{\sf P}\left(\sigma\sqrt n(T_{N_n} - \mu) <
x\right)-\fr{x}{2\sqrt{2+x^2}}-\fr{1}{2} \right\vert\leqslant{}\\
{}\leqslant
\fr{C_3}{\sqrt{n}}\,,\quad n\in\N\,.
\end{multline*}
Здесь $C_i$, $i\hm=1,2,3$,~--- величины, не зависящие от~$n$, но,
возможно, зависящие от~$r$ и других параметров задачи. В~работе~[17]
на примере статистики $T_n\hm=(1/n)(X_1+\cdots+X_n)$ показано, что
порядки убывания правых частей в неравенствах, приведенных выше,
неулучшаемы.

Учитывая эти неравенства и теорему~2.1, непосредственно получаем
следующее утверждение.

\smallskip

\noindent
\textbf{Теорема 2.2.} \textit{Предположим, что для некоторых $\mu\hm\in\r$,
$C_0\hm>0$ и $\sigma\hm>0$ ф.р.\ статистики $T_{n}$ удовлетворяет
соотношению}~(\ref{e2.2-ben}). \textit{Тогда при $r\hm\in(0,1/2)$ и любом $n\hm\in\N$
имеет место неравенство}:
$$
\sup_{\lambda\geqslant 0}\Big|\tilde Q_{T_{N_n}}(\lambda)-
2G_{2r}\Big(\frac{\lambda\sigma \sqrt{rn}}{2}\Big)+1\Big|\leqslant
\frac{4C_1}{n^{r}}.
$$
\textit{Если $r=1/2$, то
\begin{multline*}
\sup\limits_{\lambda\geqslant 0}\Big|\tilde Q_{T_{N_n}}(\lambda)-
\fr{2}{\pi}\,\arctg\left(\fr{\lambda\sigma \sqrt
n}{2\sqrt{2}}\right)\Big|\leqslant 4C_2\fr{\log n}{\sqrt{n}}\,,\\ n> 1\,.
\end{multline*}
Если $r>1/2$, то
$$
\sup\limits_{\lambda\geqslant 0}\Big|\tilde Q_{T_{N_n}}(\lambda)-
2G_{2r}\left(\fr{\lambda\sigma \sqrt{rn}}{2}\right)+1\Big|\leqslant
\fr{4C_3}{\sqrt{n}}\,.
$$
В частности, если $r\hm=1$, т.\,е.\ если $N_n$ имеет гео\-мет\-ри\-че\-ское
распределение с параметром $1/n$, то}
$$
\sup\limits_{\lambda\geqslant 0}\Big|\tilde Q_{T_{N_n}}(\lambda)-
\fr{\lambda\sigma \sqrt{n}}{\sqrt{8+\lambda^2\sigma^2 n}}\Big| \leqslant
\fr{4C_3}{\sqrt{n}}\,.
$$

\subsection{Примеры}

Пусть $X_1,X_2,\ldots$~--- независимые одинаково распределенные
случайные величины с ${\sf E}X_1\hm=\mu$ и $0\hm<{\sf D}X_1\hm=\sigma^2$. Для
натурального~$n$ обозначим
$$
T_n=\fr{1}{n}\left(X_1+\cdots+X_n\right)\,.
$$
Тогда, используя неравенство~(\ref{e1.7-ben}) и теорему~2.2, при
$r\hm\in(0,1/2)$ имеем:
$$
\sup\limits_{\lambda\geqslant 0}\Big|\tilde Q_{T_{N_n}}(\lambda)-
2G_{2r}\left(\fr{\lambda\sigma\sqrt{rn}}{2}\right)+1\Big|\leqslant
\fr{4C_1}{n^{r}}\,,\enskip n\in\N\,.
$$
Если $r=1/2$, то
\begin{multline*}
\sup\limits_{\lambda\geqslant 0}\Big|\tilde Q_{T_{N_n}}(\lambda)-
\fr{2}{\pi}\,\arctg\left(\fr{\lambda\sigma\sqrt
n}{2\sqrt{2}}\right)\Big|\leqslant 4C_2\fr{\log n}{\sqrt{n}}\,,\\ n > 1\,.
\end{multline*}
Если $r>1/2$, то
$$
\sup\limits_{\lambda\geqslant 0}\Big|\tilde Q_{T_{N_n}}(\lambda)-
2G_{2r}\left(\fr{\lambda\sigma\sqrt{rn}}{2}\right)+1\Big|\leqslant
\fr{4C_3}{\sqrt{n}}\,,\enskip n\in\N\,.
$$
В частности, если $r=1$, т.\,е.\ если $N_n$ имеет гео\-мет\-ри\-че\-ское
распределение с параметром $1/n$, то
$$
\sup\limits_{\lambda\geqslant 0}\left\vert\tilde Q_{T_{N_n}}(\lambda)-
\fr{\lambda\sigma \sqrt{n}}{\sqrt{8+\lambda^2\sigma^2 n}}\right\vert \leqslant
\fr{4C_3}{\sqrt{n}}\,,\enskip n>1\,.
$$

Рассмотрим теперь $U$-статистики. Пусть $X_1,X_2,\ldots$~---
независимые одинаково распределенные наблюдения и $h(x_1,x_2)$~---
симметричная измеримая функция двух переменных такая, что
$$
{\sf E} h(X_1,X_2) = 0\,,\ \ \ {\sf E} |h(X_1,X_2)|^p <  \infty\,,\ \ \
p > \fr{5}{3}\,.
$$
Определим  $U$-статистику:
$$
U_n = \sum\limits_{1\leqslant  i<j\leqslant  n} h(X_i,X_j)\,.
$$
Предположим также, что
$$
\tau^2\equiv{\sf E}g^2(X_1)>0,\ \ \ {\sf E}|g(X_1)|^3<\infty\,,
$$
где
$$
g(x)={\sf E}(h(X_1,X_2)|X_1=x)\,.
$$
Тогда из теоремы~2.2 и теоремы~2.1 работы~[18] при $r\hm\in(0,1/2)$
получаем неравенство:
\begin{multline*}
\sup\limits_{\lambda\geqslant 0}\left\vert\tilde Q_{U_{N_n}}(\lambda)-
2G_{2r}\left(\fr{\lambda\sqrt{r}}{2(n-1)\sqrt n\tau}\right)+1\right\vert
\leqslant {}\\
{}\leqslant \fr{4C_1}{n^{r}}\,,\enskip 
n\in\N\,.
\end{multline*}
Если $r=1/2$, то
\begin{multline*}
\sup\limits_{\lambda\geqslant 0}\left\vert\tilde Q_{U_{N_n}}(\lambda)-
\fr{2}{\pi}\,\arctg\left(\fr{\lambda}{2\sqrt{2n}(n-1)\tau}\right)\right\vert
\leqslant{}\\
{}\leqslant 4C_2\fr{\log n}{\sqrt{n}}\,,\enskip n>1\,.
\end{multline*}
Если $r>1/2$, то
\begin{multline*}
\sup\limits_{\lambda\geqslant 0}\left\vert\tilde Q_{U_{N_n}}(\lambda)-
2G_{2r}\left(\fr{\lambda\sqrt{r}}{2(n-1)\sqrt n\tau}\right)+1\right\vert
\leqslant{}\\
{}\leqslant \fr{4C_3}{\sqrt{n}}\,,\enskip n\in\N\,.
\end{multline*}
В частности, если $r\hm=1$, т.\,е.\ если $N_n$ имеет гео\-мет\-ри\-че\-ское
распределение с параметром $1/n$, то
\begin{multline*}
\sup\limits_{\lambda\geqslant 0}\left\vert\tilde Q_{T_{N_n}}(\lambda)-
\fr{\lambda}{\sqrt{\lambda^2+8(n-1)^2n\tau^2}}\right\vert\leqslant
\fr{4C_3}{\sqrt{n}}\,,\\  n>1\,.
\end{multline*}

Рассмотрим теперь $L$-статистики. Пусть $X_1,X_2,\ldots$~---
независимые одинаково распределенные наблюдения с ф.р.~$F(x)$ и
пусть $J(s)$~--- измеримая функция, заданная на интервале $(0,1)$  и
удовлетворяющая условию Липшица. Рассмотрим числа вида
$$
c_{in}=J\left(\fr{i}{n+1}\right)\,,\enskip i=1,\ldots,n\,,
$$
и определим  $L$-статистику
$$
L_n=\fr{1}{n}\sum\limits_{i=1}^nc_{in}X_{(i:n)}\,,
$$
где $X_{(1:n)}\leqslant \cdots\leqslant  X_{(n:n)}$~--- вариационный ряд,
построенный по выборке $X_1,\ldots,X_n$. Предположим также, что
\begin{gather*}
{\sf E}|X_1|^3<\infty\,;\\
 \sigma^2_J\equiv
\int\limits_{-\infty}^{\infty}J(F(x))J(F(y))(\min(F(x),F(y))-{}\\
\hspace*{30mm}{}-F(x)F(y))\,dxdy>0
\end{gather*}
Тогда из теоремы~2.2 и теоремы~3.1.2 работы~[19] при
$r\in(0,1/2)$ получаем неравенство:
\begin{multline*}
\sup\limits_{\lambda\geqslant 0}\left\vert\tilde Q_{L_{N_n}}(\lambda)-
2G_{2r}\left(\fr{\lambda\sqrt{rn}}{2\sigma_J}\right)+1\right\vert\leqslant
\frac{4C_1}{n^{r/2}}\,,\\ n\in\N\,.
\end{multline*}
Если $r=1/2$, то
\begin{multline*}
\sup\limits_{\lambda\geqslant 0}\left\vert\tilde Q_{L_{N_n}}(\lambda)-
\fr{2}{\pi}\,\arctg\left(\fr{\lambda\sqrt
n}{2\sqrt{2}\sigma_J}\right)\right\vert\leqslant 4C_2\fr{\log n}{\sqrt{n}}\,,\\
n> 1\,.
\end{multline*}
Если $r>1/2$, то
$$
\sup\limits_{\lambda\geqslant 0}\left\vert\tilde Q_{L_{N_n}}(\lambda)-
2G_{2r}\left(\fr{\lambda\sqrt{rn}}{2\sigma_J}\right)+1\right\vert\leqslant
\fr{4C_3}{\sqrt{n}}\,,\enskip n\in\N\,.
$$
В частности, если $r\hm=1$, т.\,е.\ если $N_n$ имеет гео\-мет\-ри\-че\-ское
распределение с параметром~$1/n$, то
$$
\sup\limits_{\lambda\geqslant 0}\left\vert\tilde Q_{T_{N_n}}(\lambda)-
\fr{\lambda\sqrt{n}}{\sqrt{\lambda^2n+8\sigma^2_J}}\right\vert\leqslant
\fr{4C_3}{\sqrt{n}}\,,\enskip n\geqslant1\,.
$$

\smallskip

\subsection{Распределение Лапласа}

Рассмотрим распределение Лапласа с ф.р.\ $\Lambda_\gamma(x)$ и
плот\-ностью
$$
\lambda_s(x)=\fr{1}{s\sqrt 2}\exp\left\{
-\fr{\sqrt{2}|x|}{s} \right\}\,,\enskip s>0,\  x\in\r\,.
$$
В работе~[20] была построена последовательность с.в.\ $N_n(m)$,
зависящая от параметра $m\hm\in\N$ сле\-ду\-юще\-го вида. Пусть $Y_1, Y_2,\ldots$~--- 
независимые одинаково распределенные с.в., имеющие
непрерывную ф.р. Определим с.в.
$$
N(m) = \min\left\{ i\geqslant 1\!: \max\limits_{1\leqslant q j\leqslant 
q m}\! \!Y_j <\max\limits_{m+1\leqslant q k
\leqslant q m+i}\!\! Y_k \right\}.
$$
Хорошо известно, что так определенные с.в.\ имеют распределение вида
\begin{equation}
{\sf P}(N(m) \geqslant  k) = \fr{m}{m+k-1}\,, \enskip k\geqslant  1\,.
\label{e2.5-ben}
\end{equation}
Пусть теперь  $N^{(1)}(m), N^{(2)}(m),\ldots$~--- независимые
одинаково распределенные с.в., имеющие распределение~(\ref{e2.5-ben}).
Определим с.в.
$$
N_n(m)=\max\limits_{1\leqslant q j\leqslant q n}N^{(j)}(m)\,.
$$
Как показано в работе~[20],
\begin{equation*}
\lim\limits_{n\to\infty}{\sf P}\left( \fr{N_n(m)}{n}<x \right) = e^{-m/x}\,,
\enskip x>0\,,
%\label{e2.6-ben}
\end{equation*}
и для асимптотически нормальной статистики $T_n$ справедливо
соотношение:
$$
{\sf P}\left(\sigma\sqrt{n}(T_{N_n(m)} - \mu) <
x\right)\longrightarrow \Lambda_{1/m}(x)\,,\enskip n\to\infty\,,
$$
где $\Lambda_{1/m}(x)$~--- функция распределения Лапласа с параметром
$s=1/m$. Отметим, что распределение Лапласа является симметричным
унимодальным непрерывным распределением. Если для ф.р.\ статистики
$T_n$ справедлива оценка скорости сходимости типа~(9), то в работе~[21] 
получена оценка скорости сходимости ф.р.\ статистики
$T_{N_n(m)}$
\begin{multline*}
\sup\limits_{x\in\r}\left\vert {\sf P}\Bigl(\sigma\sqrt n(T_{N_n(m)} - \mu) <
x\Bigr)-\Lambda_{1/m}(x)\right\vert\leqslant  \fr{C_4}{\sqrt n},\\
n\in\N\,.
%\label{e2.7-ben}
\end{multline*}
Учитывая это неравенство и теорему~2.1, непосредственно получаем
следующее утверждение.

\smallskip

\noindent
\textbf{Теорема~2.3.} \textit{Предположим, что для некоторых $\mu\hm\in\r$,
$C_0\hm>0$ и $\sigma\hm>0$ ф.р.\ статистики $T_{n}$ удовлетворяет
соотношению}~(\ref{e2.2-ben}). \textit{Тогда при любом $n\hm\in\N$ имеет место
неравенство}:
$$
\sup\limits_{\lambda\geqslant 0}\left\vert\tilde Q_{T_{N_n(m)}}(\lambda)-
2\Lambda_{1/m}\left(\fr{\lambda\sigma \sqrt{n}}{2}\right)+1\right\vert\leqslant
\fr{4C_4}{\sqrt n}\,.
$$

\subsection{Примеры}

Пусть $X_1,X_2,\ldots$~--- независимые одинаково распределенные
случайные величины с ${\sf E}X_1\hm=\mu$ и $0\hm<{\sf D}X_1\hm=\sigma^2$. Для
натурального $n$ обозначим
$$
T_n=\fr{1}{n}\left(X_1+\cdots+X_n\right)\,.
$$
Тогда, используя неравенство~(\ref{e1.7-ben}) и теорему~2.3, имеем  оценку:
$$
\sup\limits_{\lambda\geqslant 0}\left\vert\tilde Q_{T_{N_n(m)}}(\lambda)-
2\Lambda_{1/m}\left(\fr{\lambda\sigma\sqrt n}{2}\right)+1\right\vert\leqslant
\fr{4C_4}{\sqrt n}\,.
$$
Рассмотрим теперь $U$-статистики. Пусть $X_1,X_2,\ldots$~---
независимые одинаково распределенные наблюдения и $h(x_1,x_2)$~---
симметричная измеримая функция двух переменных такая, что
$$
{\sf E}h(X_1,X_2)=0\,,\ \ \ {\sf E}|h(X_1,X_2)|^p< \infty\,,\enskip p >\fr{5}{3}\,.
$$
Определим  $U$-статистику
$$
U_n=\sum\limits_{1\leqslant  i<j\leqslant  n}h(X_i,X_j)\,.
$$
Предположим также, что
$$
\tau^2\equiv{\sf E}g^2(X_1)>0\,,\enskip {\sf E}|g(X_1)|^3<\infty\,,
$$
где
$$
g(x)={\sf E}(h(X_1,X_2)|X_1=x)\,.
$$
Тогда из теоремы~2.3 и теоремы~2.1 работы~[18] получаем неравенство
\begin{multline*}
\sup\limits_{\lambda\geqslant 0}\left\vert\tilde Q_{U_{N_n(m)}}(\lambda)-
2\Lambda_{1/m}\left(\fr{\lambda}{2\tau(n-1)\sqrt n}\right)+1\right\vert \leqslant{}\\
{}\leqslant
\fr{4C_4}{\sqrt n}\,.
\end{multline*}
Рассмотрим теперь $L$-статистики. Пусть $X_1,X_2,\ldots$~---
независимые одинаково распределенные наблюдения с ф.р.\ $F(x)$ и
пусть $J(s)$~--- измеримая функция, заданная на интервале $(0,1)$  и
удовлетворяющая условию Липшица. Рассмотрим числа вида
$$
c_{in}=J\left(\fr{i}{n+1}\right)\,,\enskip i=1,\ldots,n\,,
$$
и определим  $L$-статистику
$$
L_n=\fr{1}{n}\sum\limits_{i=1}^nc_{in}X_{(i:n)}\,,
$$
где $X_{(1:n)}\leqslant \cdots\leqslant  X_{(n:n)}$~--- вариационный ряд,
построенный по выборке $X_1\ldots,X_n$. Предположим также, что
\begin{gather*}
{\sf E}|X_1|^3<\infty\,;\\ 
\sigma^2_J\equiv
\int\limits_{-\infty}^{\infty}J(F(x))J(F(y))(\min(F(x),F(y))-{}\\
\hspace*{30mm}{}-F(x)F(y)) \,dxdy>0\,.
\end{gather*}
Тогда из теоремы~2.3 и теоремы~3.1.2 работы~[19] получаем
неравенство:
$$
\sup\limits_{\lambda\geqslant 0}\left\vert\tilde Q_{L_{N_n(m)}}(\lambda)-
2\Lambda_{1/m}\left(\fr{\lambda\sqrt n}{2\sigma_J}\right)+1\right\vert\leqslant
\fr{4C_4}{\sqrt n}\,.
$$

{\small\frenchspacing
{%\baselineskip=10.8pt
\addcontentsline{toc}{section}{Литература}
\begin{thebibliography}{99}



\bibitem{1-ben}
\Au{Гнеденко Б.\,В., Фахим Х.} Об одной теореме переноса~// Докл.\ АН
СССР, 1969. Т.~187. С.~15--17.

\bibitem{2-ben}
\Au{Von Ghossy R., Rappl~G.} Some approximation methods for the
distribution of random sums~// Insurance: Mathematics and Economics,
1983. Vol.~2. P.~251--270.

\bibitem{3-ben}
\Au{Гнеденко Б.\,В.} Курс теории вероятностей.~--- М.: Наука, 1988.

\bibitem{4-ben}
\Au{Круглов В.\,М., Королев В.\,Ю.} Предельные теоремы для случайных
сумм.~--- М.: Изд-во Московского ун-та, 1990.

\bibitem{5-ben}
\Au{Gnedenko B.\,V., Korolev V.\,Yu.} Random summation. Limit
theorems and applications.~--- Boca Raton: CRC Press, 1996.

\bibitem{6-ben}
\Au{Королев В.\,Ю., Бенинг В.\,Е., Шоргин~С.\,Я.} Математические
основы теории риска.~--- 2-е изд., перераб. и дополн.~--- М.: Физматлит,
2011.

\bibitem{7-ben}
\Au{Гнеденко Б.\,В.} Об оценке неизвестных параметров распределения
при случайном числе независимых наблюдений~// Тр. Тбилисского
матем. ин-та, 1989. Т.~92. С.~146--150.

\bibitem{8-ben}
\Au{Королев В.\,Ю.} Асимптотические свойства выборочных квантилей,
построенных по выборкам случайного объема~// Теория вероятностей и
ее применения, 1999. Т.~44. Вып.~2. С.~440--445.

\bibitem{9-ben}
\Au{Петров В.\,В.} Суммы независимых случайных величин.~--- М.:
Наука, 1972.

\bibitem{10-ben}
\Au{Хенгартнер В., Теодореску~Р.} Функции концентрации.~--- М.:
Наука, 1980.

\bibitem{11-ben}
\Au{Феллер В.} Введение в теорию вероятностей и ее приложения.
Т.~2.~--- М.: Мир, 1984.

\bibitem{12-ben}
\Au{Лукач Е.} Характеристические функции.~--- М.: Наука, 1979.

\bibitem{13-ben}
\Au{Shevtsova I.} On the absolute constants in the
Berry--Esseen inequalities for identically distributed summands.
arXiv: 1111.6554v1, 2011.

\bibitem{14-ben}
\Au{Бенинг В.\,Е., Королев В.\,Ю.} Об использовании распределения
Стьюдента в задачах теории вероятностей и математической статистики~// 
Теория вероятностей и ее применения, 2004. Т.~49. Вып.~3. С.~417--435.

\bibitem{15-ben}
\Au{Гавриленко С.\,В., Зубов В.\,Н., Королев~В.\,Ю.} Оценка скорости
сходимости распределений регулярных статистик, построенных по
выборкам случайного объема с отрицательным биномиальным
распределением, к распределению Стьюдента~// Статистические методы
оценивания и проверки гипотез.~--- Пермь: Изд-во Пермского
гос. ун-та, 2006. C.~118--134.

\bibitem{16-ben}
\Au{Бенинг В.\,Е., Королев В.\,Ю., Соколов~И.\,А., Шоргин~С.\,Я.}
Рандомизированные модели и методы теории надежности информационных и
технических сис\-тем.~--- М.: ТОРУС ПРЕСС, 2007.

\bibitem{17-ben}
\Au{Нефедова Ю.\,С.} Оценки скорости сходимости в предельной теореме
для отрицательных биномиальных случайных сумм~// Статистические
методы оценивания и проверки гипотез.~--- Пермь: Изд-во Пермского
гос. ун-та, 2011. C.~46--61.

\bibitem{18-ben}
\Au{Helmers R., Van Zwet~W.\,R.} The Berry--Esseen bound for
$U$-statistics~// Statistical Decision Theory and Related Topics~3,
1982. Vol.~1. P.~497--512.

\bibitem{19-ben}
\Au{Helmers R.} Edgeworth expansions for linear combinations of
order statistics.~--- Amsterdam: Mathematical Center Tracts~105,
1984. 137~p.

\bibitem{20-ben}
\Au{Бенинг В.\,Е., Королев В.\,Ю.} Некоторые статистические задачи,
связанные с распределением Лапласа~// Информатика и её применения,
2008. Т.~2. Вып.~2. С.~19--34.

\label{end\stat}


\bibitem{21-ben}
\Au{Лямин О.\,О.} О~ско\-рости сходимости распределений некоторых
статистик к распределению Лапласа и Стьюдента~// Вестник Московского
ун-та. Сер.~15.  Вычислительная  математика и кибернетика,
2011. Вып.~1. С.~39--47.
\end{thebibliography}
}
}

\end{multicols}