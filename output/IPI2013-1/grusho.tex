\def\stat{grusho}

\def\tit{СТАТИСТИЧЕСКИЕ МЕТОДЫ ОПРЕДЕЛЕНИЯ ЗАПРЕТОВ 
ВЕРОЯТНОСТНЫХ МЕР НА~ДИСКРЕТНЫХ ПРОСТРАНСТВАХ$^*$}

\def\titkol{Статистические методы определения запретов 
вероятностных мер на~дискретных пространствах}

\def\autkol{А.\,А.~Грушо, Н.\,А.~Грушо, Е.\,Е.~Тимонина}

\def\aut{А.\,А.~Грушо$^1$, Н.\,А.~Грушо$^2$, Е.\,Е.~Тимонина$^3$}

\titel{\tit}{\aut}{\autkol}{\titkol}

{\renewcommand{\thefootnote}{\fnsymbol{footnote}}\footnotetext[1]
{Работа выполнена при поддержке РФФИ, гранты №\,10-01-00480 и №\,11-07-00112.}}

\renewcommand{\thefootnote}{\arabic{footnote}}
\footnotetext[1]{Институт проблем информатики Российской академии наук; Московский государственный 
университет им.\ М.\,В.~Ломоносова, факультет вычислительной математики и кибернетики, 
grusho@yandex.ru}
\footnotetext[2]{Институт проблем информатики Российской академии наук, info@itake.ru}
\footnotetext[3]{Институт проблем информатики Российской академии наук, eltimon@yandex.ru}

\vspace*{6pt}
  
\Abst{Предложен метод статистического определения запретов вероятностных 
мер на дискретных пространствах. Показана состоятельность сделанных оценок. 
Построена схема применения полученных оценок для проверки статистических 
гипотез в дискретных пространствах. Показано, что в некотором смысле оценки 
запретов могут порождать состоятельные последовательности критериев (СПК).}

\vspace*{2pt}

\KW{состоятельные последовательности критериев; запреты вероятностных мер в 
дискретных пространствах; состоятельность оценок}

\vspace*{6pt}

\vskip 14pt plus 9pt minus 6pt

      \thispagestyle{headings}

      \begin{multicols}{2}

            \label{st\stat}

\section{Введение}

  Конечные вероятностные пространства играют существенную роль при 
решении различных проблем сетевой и компьютерной безопасности, 
криптографии и~др. Рассмотрим задачу проверки последовательности простых 
гипотез $H_{0,n}$ против сложных альтернатив $H_{1,n}$ в конечных 
пространствах. Пусть $X\hm = \{x_1, \ldots ,x_m\}$~--- конечное множество, 
$X^n$~--- декартово произведение множества~$X$,\linebreak
 $X^\infty$~--- множество 
бесконечных последователь\-ностей с элементами из~$X$, $\mathcal{A}$~--- 
минимальная $\sigma$-ал\-гебра, порожденная всеми цилиндрическими 
множествами~[1--3]. Если на $X$ рассматривается\linebreak дискретная топология, то 
тихоновское произведение~$X^\infty$ является компактным топологическим 
пространством со счетной базой~[2, 4]. В~этом случае борелевская 
  $\sigma$-ал\-геб\-ра $\mathcal{B}$ совпадает с $\sigma$-ал\-геб\-рой~$\mathcal{A}$. 
  
  Пусть $P_0$~--- вероятностная мера на ($X^\infty,\mathcal{A}$). Для каждого 
$n$ $P_{0,n}$~--- проекция~$P_0$ на первые $n$ координат 
последовательностей из~$X^\infty$.
  
  Ранее было определено~\cite{5-gr, 6-gr} понятие запрета конечной 
вероятностной меры. Под запретами понимаются начальные участки 
последовательностей из~$X^\infty$, вероятности которых равны нулю. 
В~статистических задачах существуют критерии, критические множества 
которых полностью определяются через запреты. В~работах~\cite{5-gr, 6-gr} 
доказаны необходимые и достаточные условия существования СПК, критические множества которых 
полностью определяются запретами. 
  
  Статистические критерии, определяемые с помощью запретов, обладают тем 
свойством, что при нулевой гипотезе $P_{0,n}(S_n)\hm=0$, где $S_n$, $n\hm=1, 
2,\ldots,$~--- критические множества критериев. В~случае явного задания 
вероятностных мер $P_{0,n}$ иногда удается довольно легко определить 
запреты этих мер. 
  
  Если последовательность критериев определяется с помощью запретов, то 
множество запретов является бесконечным. Поэтому необходимо определять 
алгоритмы, порождающие бесконечное множество запретов. Это удается 
сделать для класса мер, у которых все запреты порождаются минимальными 
запретами. Неформально минимальные запреты определяются следующим 
образом. 
  
  Пусть $\mathcal{N}$~--- конечное множество векторов и пусть все запреты меры 
$P_0$ содержат хотя бы один вектор из~$\mathcal{N}$. В~этом случае можно 
говорить, что $\mathcal{N}$ порождает множество запретов. Для этого класса 
вероятностных мер множество~$\mathcal{N}$ можно определить статистически. 
  
  Предположим, что можно сколь угодно раз независимо реализовать векторы 
из $X^n$ с вероятностными мерами $P_{0,n}$ для любых конечных~$n$. Тогда 
можно построить состоятельную оценку множества~$\mathcal{N}$. 
  
  В разд.~2 построены оценки множества~$\mathcal{N}$ и доказана их 
состоятельность. В~разд.~3 рассмотрен алгоритм применения метода запретов 
для проверки гипотез $H_{0,n}$ против $H_{1,n}$ с учетом возможности 
неточного определения множества~$\mathcal{N}$. В~заключении обсуждаются 
полученные результаты.

\section{Оценки множества минимальных запретов}

  Пусть на ($X^\infty, \mathcal{A}$) определена вероятностная мера~$P_0$. 
Очевидно, что для любого $B_n\hm\in X^n$
  $$ 
  P_{0,n}(B_n)=P_0\left( B_n\times X^\infty\right)\,.$$
  Пусть $D_{0,n}$~--- носитель меры~$P_{0,n}$:
  $$
  D_{0,n}=\left\{ \vec{x}_n\in X^\infty,\ P_{0,n}\left( \vec{x}_n\right)>0\right\}\,.
  $$
  
  Обозначим $\Delta_{0,n}\hm=D_{0,n}\times X^\infty$. Последовательность 
$\Delta_{0,n}$, $n\hm=1,2,\ldots$, невозрастающая и 
  $$
  \Delta_0 =\lim\limits_{n\rightarrow \infty} \Delta_{0,n}=\mathop{\bigcap}\limits_{n=1}^\infty 
\Delta_{0,n}\,.
  $$
  
  Множество $\Delta_0$ является замкнутым и носителем меры~$P_0$.
  
  Рассматривается также множество вероятностных мер $\left\{ P_\theta,\ 
\theta\hm\in\Theta\right\}$ на ($X^\infty, \mathcal{A}$), для которых определены 
$P_{\theta,n}$, $D_{\theta,n}$, $\Delta_{\theta,n}$ $\Delta_\theta$.
  
  Если $\overline{\omega}_k\in X^k$, то $\tilde{\omega}_{k-1}$ получена из 
$\overline{\omega}_k$ отбрасыванием последней координаты.
  
  \smallskip
  
  \noindent
  \textbf{Определение 1.} Запретом в мере~$P_{0,n}$ называется вектор 
$\overline{\omega}_k\hm\in X^k$, $k\hm\leq n$, такой, что 
  $$
  P_{0,n}\left( \overline{\omega}_k\times X^{n-k}\right)=0\,.
  $$
  
  Если $P_{0,k-1}\left( \tilde{\omega}_{k-1}\right)\hm>0$, то 
$\overline{\omega}_k$ называется наименьшим запретом. 
  
  Если $\overline{\omega}_k$ является запретом в мере $P_{0,n}$, то для 
любых $k\hm\leq s\hm\leq n$ и для любых векторов $\overline{\omega}_s$, 
начинающихся с $\overline{\omega}_k$, имеем: 
  $$
  P_{0,s}\left( \overline{\omega}_s\right) =0\,.
  $$
  
  Пусть $\overline{\omega}_s\hm\in X^s$ и для любого $n\hm\geq s$ и любого 
$\overline{\omega}_n\hm\in X^n$, если $\overline{\omega}_s$ является частью 
$\overline{\omega}_n$, то $P_{0,n}\left( \overline{\omega}_n\right)\hm=0$. Если 
существует $n$ такое, что $\overline{\omega}_n$ содержит $\tilde{\omega}_s$ 
или содержит вектор, отличающийся от $\overline{\omega}_s$ только первой 
координатой, и $P_{0,n}\left( \overline{\omega}_n\right)>0$, то вектор 
$\overline{\omega}_s$ назовем \textit{минимальным запретом}. Далее 
под~$\mathcal{N}$ будем понимать множество минимальных запретов. 
  
  Пусть все запреты меры $P_0$ определяются минимальными запретами, 
т.\,е.\ любой запрет содержит хотя бы один элемент из множества~$\mathcal{N}$. 
Предположим, что множество~$\mathcal{N}$ конечно. Тогда существует~$s_0$ такое, 
что длины всех минимальных запретов не больше~$s_0$. Допустим, что 
известна оценка~$s_0$, хотя не известно, какие элементы входят в 
множество~$\mathcal{N}$ и мощность множества~$\mathcal{N}$. Построим состоятельную 
оценку множества~$\mathcal{N}$ в указанных предположениях. 
  
  Пусть $\overline{\omega}_n^{(1)}, \overline{\omega}_n^{(2)}, \ldots , 
\overline{\omega}_n^{(N)}$~--- выборка из распределения $P_{0,n}$ 
объема~$N$. Для простоты изложения считаем, что $n$~--- фиксированный 
параметр, \mbox{хотя} дальнейшие оценки можно провести для случайного набора 
$n_1, \ldots , n_N$. В~полученной выборке считаем частоты $\nu_n^{(i)}$, $i\hm=1, 
2, \ldots , m$, встречаемости $i$-го символа в конце каждого вектора из 
$\overline{\omega}_n^{(1)}, \overline{\omega}_n^{(2)}, \ldots , 
\overline{\omega}_n^{(N)}$. Выделяем символы из~$X$, час\-то\-ты которых 
равны нулю. Выбранные символы являются потенциальными минимальными 
запретами длины~1. Считаем частоты $\nu_n^{(i,r)}$, $i,r\hm=1, 2, \ldots , m$, 
последних биграмм в векторах выборки $\overline{\omega}_n^{(1)}, 
\overline{\omega}_n^{(2)}, \ldots , \overline{\omega}_n^{(N)}$. Биграммы, 
имеющие нулевые частоты,~--- это кандидаты на минимальные запреты 
длины~2. Аналогично строим частоты $\nu_n^{(i_1,\ldots , i_s)}$, $i_1,\ldots , 
i_s\hm=1, 2, \ldots , m$, $s$-це\-по\-чек. $s$-це\-поч\-ки, частоты которых 
равны нулю, являются кандидатами на минимальные запреты длины~$s$. 
Число шагов в приведенном алгоритме не превосходит~$s_0$. 
  
  Построенное в результате указанного алгоритма множество потенциальных 
запретов $\mathcal{N}_1$ может отличаться от истинного множества~$\mathcal{N}$. 
Множество~$\mathcal{N}_1$ является оценкой множества~$\mathcal{N}$. Возможны 
следующие ошибки. $s$-це\-поч\-ка, которая не является минимальным 
запретом, случайно ни разу не встретилась в просмотренной выборке и 
попадает в множество~$\mathcal{N}_1$. Тогда при замене~$\mathcal{N}$ 
оценкой~$\mathcal{N}_1$ получаются лишние минимальные запреты. Возможно, 
что мера~$P_0$ такова, что истинный минимальный запрет может встретиться 
в цепочках, длины которых больше~$n$. В~этом случае частота встречаемости 
соответствующей $s$-це\-поч\-ки также равна нулю и она входит в~$\mathcal{N}_1$. 
Это следует из определения минимального запрета (любая встреча 
минимального запрета обращает вероятность содержащей ее цепочки в~0). 
Таким образом, $\mathcal{N}_1$ содержит~$\mathcal{N}$. 
  
  Рассмотрим вероятность того, что $s$-це\-поч\-ка не встретилась в выборке 
случайно. Обозначим вероятности появления всех $s$-це\-по\-чек в концах 
элементов выборки $\overline{\omega}_n^{(1)}, \overline{\omega}_n^{(2)}, \ldots 
, \overline{\omega}_n^{(N)}$ через $p_1^{(s)},p_2^{(s)}, \ldots , p^{(s)}_{m^s}$. 
Вероятность того, что $i$-я цепочка имеет нулевую частоту встречаемости, 
равна $(1-p_i^{(s)})^N$. Среднее число не встретившихся цепочек равно 
$\sum\limits_{i=1}^{m^s}\left(1-p_i^{(s)}\right)^N$. Тогда математическое ожидание 
числа элементов~$\mathcal{N}_1$ равняется 
   $
   \sum\limits_{s=1}^{s_0} \sum\limits_{i=1}^{m^s} \left( 1-p_i^{(s)}\right)^N   $.
  
  Обозначим 
  $$
  \varepsilon \hm=\min\limits_{1\leq i\leq m^s,\ 1\leq s\leq s_0,\ 
p_i^{(s)}>0} p_i^{(s)}\,.$$ 
Тогда математическое ожидание мощности множества 
$\mathcal{N}_1$ равняется $\vert \mathcal{N}\vert +{\sf E}\xi$, где $\xi$~--- случайная 
величина, равная числу случайно не встретившихся $s$-це\-по\-чек, 
$s\hm=1,\ldots , s_0$. Очевидно, что
  $$
{\sf  E}\xi =\sum\limits_{s=1;\ p_i^{(s)}>0}^{s_0} \sum\limits_{i=1}^{m^s} \left( 1-
p_i^{(s)}\right)^N\,.
  $$
Для ${\sf E}\xi$ справедлива следующая оценка:
$$
{\sf E}\xi \leq s_0 m^{s_0}\left( 1-\varepsilon\right)^N\,.
$$
${\sf E}\xi$ стремится к~0. Обозначим через $\lambda_N$ вероятность того, что 
$\vert \mathcal{N}_1\vert > \vert \mathcal{N}\vert$. Используя неравенство 
Маркова, получаем, что при $N\hm\rightarrow\infty$\ $\lambda_N\hm\rightarrow 
0$. 
\section{Состоятельные последовательности критериев 
с~использованием минимальных запретов}

  Перейдем к построению критериев, зависящих от запретов. Ранее 
предполагалось, что существует семейство распределений $\left\{ P_\theta, \ 
\theta\hm\in\Theta\right\}$. Согласно~\cite{5-gr} необходимым и достаточным 
условием существования СПК, определяемой запретами, является условие: для 
всех $\theta\hm\in\Theta$
  $$
  P_\theta(\Delta_0)=0\,.
  $$
    Для того чтобы проверить это условие, достаточно доказать, что для всех 
$\theta\hm\in\Theta$ существует минимальный запрет из~$\mathcal{N}$ такой, что 
вероятность в мере $P_\theta$ появления этой цепочки из~$\mathcal{N}$ стремится 
к~1. 
  
  Пусть условие существования СПК, опреде\-ля\-емой запретами $\mathcal{N}_1$, 
выполняется. Тогда последовательность критериев с критическими 
множествами, содержащими все последовательности, оканчивающиеся 
цепочками из~$\mathcal{N}_1$, является состоятельной последовательностью критериев 
для проверки гипотез $H_{0,n}^\prime:P_{0,n}^\prime$, $n\hm=1,2,\ldots$, 
против альтернатив $H_{1,n}$, где $P^\prime_{0,n}$, $n\hm=1,2,\ldots$,~--- 
некоторые меры с запретами~$\mathcal{N}_1$. 
  
  В работе~\cite{6-gr} было доказано выполнение следующего соотношения 
между мощностями носителей~$P_{0,n}$, $n\hm=1,2\ldots,$ и числами 
наименьших запретов в этих мерах:
  $$
  v_1 m^{n-1}+\cdots +v_{n-1}m+v_n+\vert D_{0,n}\vert =m^n\,,
  $$
где $v_i$~--- число наименьших запретов длины~$i$. Из этого соотношения 
видно, что если увеличивать число наименьших запретов (увеличивая число 
минимальных запретов), то, вообще говоря, уменьшаются размеры носителей 
$D_{0,n}$. Это означает, что вместо исходной меры~$P_0$ с множеством 
минимальных запретов~$\mathcal{N}$ для определения нулевых гипотез можно 
рассматривать некоторую меру~$P^\prime_0$. Для этой меры~$P_0^\prime$ 
носитель $\Delta_0^\prime$, вообще говоря, меньше, чем носитель исходной 
меры~$P_0$, т.\,е.\ $\Delta^\prime_0\subseteq \Delta_0$. Поэтому проверка 
условия существования СПК, зависящей от запретов, для множества 
$\mathcal{N}_1$ (т.\,е.\ для меры~$P^\prime_0$) не гарантирует выполнения этого 
условия для меры~$P_0$ с множеством запретов~$\mathcal{N}$. 
  
  Определим корректно меру $P_0^\prime$ и ее носитель~$\Delta_0^\prime$. 
Рассмотрим множество $B_{0,n}\subseteq D_{0,n}$, которое получается 
исключением из $D_{0,n}$ всех цепочек, содержащих запреты из 
$\mathcal{N}_1\backslash \mathcal{N}$. Определим замкнутое множество 
$\Delta^\prime_{0,n}\hm=B_{0,n}\times X^\infty$. Очевидно, что 
$\Delta^\prime_{0,n}\hm\supseteq \Delta^\prime_{0,n+1}$. Тогда существует 
предел
  $$
  \Delta_0^\prime= \mathop{\bigcap}\limits_{n=1}^\infty \Delta^\prime_{0,n}\,.
  $$
  
  Очевидно, что $\Delta_0\backslash \Delta^\prime_0$~--- измеримое множество. 
Обозначим $P_0(\Delta_0\backslash \Delta^\prime_0)\hm=\mu$. Тогда 
$P_0^\prime$ определяется следующим образом: 
  $$
  P_0^\prime(A) =\fr{1}{1-\mu}\,P_0\left( A\bigcap \Delta^\prime_0\right)\,.
  $$
  
  Для проекций $P^\prime_0$ множество минимальных запретов совпадает с $\mathcal{N}_1$.
  Пусть СПК, зависящая от запретов, для проверки $H^\prime_{0,n}$ против 
$H_{1,n}$ (выше предполагалось, что такая последовательность существует) 
определяется критическими множествами~$S_n$, $n\hm=1,2,\ldots$ Тогда 

\noindent 
  \begin{multline*}
  P_{0,n}(S_n) =P_0 \left(S_n\times X^\infty\right)={}\\
  {}= P_0\left( \left( S_n\times X^\infty\right)\bigcap \left( \Delta_0\backslash 
\Delta^\prime_0\right)\right) +{}\\
{}+P_0\left(\left( S_n\times X^\infty\right) \bigcap 
\left( \Delta_0\bigcap\Delta_0^\prime\right)\right)\,.
  \end{multline*}
Второе слагаемое в последней сумме равно нулю, так как критическое 
множество $S_n$ в мере~$P^\prime_{0,n}$ имеет вероятность~0. 
  
  Используя определение~$\mu$, получаем оценку:
  $$
  P_0\left(\left( S_n\times X^\infty\right) \bigcap \left( \Delta_0\backslash 
\Delta^\prime_0 \right)\right) \leq \mu\,.
  $$
    Отсюда следует, что последовательность критериев с критическими 
множествами $S_n$ при проверке $H_{0,n}$ против $H_{1,n}$ имеет уровень 
значимости~$\mu$. При этом для всех $\theta\hm\in \Theta$ 
$P_{\theta,n}(S_n)\rightarrow 1$. 
  
  Вспомним, что $\mu$ является случайной величиной в вероятностной схеме, 
связанной со статистическим определением минимальных запретов. Было 
показано, что при $N\rightarrow\infty$ с вероятностью, стремящейся к~1 (в 
схеме оценки минимальных запретов), $\mathcal{N}_1\hm=\mathcal{N}$. 
В~этом случае $\mu\hm=0$ и $P_0\hm=P_0^\prime$. Таким образом, 
получается, что при $N\rightarrow \infty$ СПК, постро-\linebreak\vspace*{-12pt}

\pagebreak

\noindent
енная для проверки 
$H^\prime_{0,n}$ против $H_{1,n}$, будет также состоятельной для проверки 
$H_{0,n}$ против~$H_{1,n}$.
  
\section{Заключение}

  Знание того факта, что число минимальных запретов конечно, достаточно 
для того, чтобы определить последовательность критериев, зависящую от 
запретов, которая асимптотически (в определенном смысле) будет 
состоятельной. Этот результат позволяет строить алгоритмы статистического 
выявления скрытых каналов, в которых вставки осуществляются с помощью 
некоторых функциональных соотношений. 
Кроме того, в задачах контроля 
последовательностей на предмет реализации некоторых событий подобный 
подход может использоваться при непараметрическом определении нулевой 
ги\-по\-тезы. 
{ %\looseness=1

}
  
  
{\small\frenchspacing
{%\baselineskip=10.8pt
\addcontentsline{toc}{section}{Литература}
\begin{thebibliography}{9}
  
\bibitem{1-gr}
\Au{Леман Э.} Проверка статистических гипотез.~--- М.: Наука, 1964.

\bibitem{3-gr}
\Au{Бурбаки Н.} Общая топология. Основные структуры.~--- М.: Наука, 1968.

\bibitem{2-gr}
\Au{Неве Ж.} Математические основы теории вероятностей.~--- М.: Мир, 1969.


\bibitem{4-gr}
\Au{Прохоров Ю.\,В., Розанов Ю.\,А.} Теория вероятностей.~--- М.: Наука, 1993.

\bibitem{5-gr}
\Au{Грушо А.\,А., Тимонина Е.\,Е.} Запреты в дискретных ве\-ро\-ят\-но\-ст\-но-ста\-ти\-сти\-че\-ских 
задачах~// Дискретная математика, 2011. Т.~23. №\,2. С.~53--58.

\label{end\stat}

\bibitem{6-gr}
\Au{Grusho А., Timonina E.} Statistical tests based on bans~// 1st Symposium 
(International) and 10th Balkan Conference on Operational Research 
Proceedings.~---  Thessaloniki, Greece, 2011. Vol.~1. P.~234--241.  
\end{thebibliography}
}
}

\end{multicols}