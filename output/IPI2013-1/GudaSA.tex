\def\stat{guda}

\def\tit{ОПЕРАЦИИ НАД ПРЕДСТАВЛЕНИЯМИ КУСОЧНО-КВАЗИАФФИННЫХ ФУНКЦИЙ В~ВИДЕ ДЕРЕВЬЕВ$^*$}

\def\titkol{Операции над представлениями кусочно-квазиаффинных функций в~виде деревьев}

\def\autkol{С.\,А. Гуда}

\def\aut{С.\,А. Гуда$^1$}

\titel{\tit}{\aut}{\autkol}{\titkol}

{\renewcommand{\thefootnote}{\fnsymbol{footnote}}\footnotetext[1]
{Работа выполнена при поддержке программы 
<<Научные и научно-педагогические кадры инновационной России>> на 2009--2013~годы, 
госконтракты №\,02.740.11.0208 от 7~июля 2009~г.; 
№\,14.740.11.0006 от 1~сентября 2010~г.}}

\renewcommand{\thefootnote}{\arabic{footnote}}
\footnotetext[1]{Южный федеральный университет, gudasergey@gmail.com}

\vspace*{-3pt}


\Abst{Введены понятия кусочно заданной квазиаффинной (ККА) функции и ККА-мно\-же\-ст\-ва. 
Определены представления ККА-множеств и ККА-функций в виде деревьев и понятие сложности 
представления. Описаны алгоритмы выполнения операций над древовидными представлениями, в 
частности объединение, пересечение, проверка ККА-мно\-же\-ст\-ва на пустоту, сумма, вычисление 
образа/прообраза, обращение, суперпозиция, сравнение ККА-функ\-ций. Даны оценки сложности 
получающихся в результате объектов. Доказана теорема о виде и сложности лексикографического 
экстремума в ККА-мно\-же\-ст\-ве, зависящем от параметров.}

\vspace*{-1pt}

\KW{кусочно-квазиаффинная функция; выпуклый Z-многогранник; лексикографический экстремум}

\vspace*{-3pt}

\vskip 12pt plus 9pt minus 6pt

      \thispagestyle{headings}

      \begin{multicols}{2}

            \label{st\stat}


\section{Введение}

При анализе программ на этапе компиляции, например во время
автоматического распараллеливания, важную роль играют информационные
зависимости. Решетчатый граф информационных зависимостей гнезда
циклов (или граф алгоритма, см.~\cite{Voevodin, ShteinbBYa0})
связывает итерации, на которых происходят обращения к одной и той же
ячейке памяти. Для хранения решетчатого графа не подходят такие
традиционные методы, как матрица смежностей или список дуг. Это
связано с тем, что множество вершин графа совпадает с итерационным
пространством программы. Затраты памяти на хранение такого
количества вершин или дуг для реальных программ не приемлемы. Кроме
того, время просмотра такого графа будет сопоставимо со временем
исполнения программы, что также недопустимо. В~[1, 3--5]
было предложено описывать решетчатый граф набором функций. Для
каждой дуги $I\hm\to J$ графа найдется функция~$\Phi$ из набора такая,
что $I\hm=\Phi(J)$ или $J\hm=\Phi(I)$. Функции, описывающие информационные
зависимости в гнезде циклов, являются кусочно заданными и на каждой
части области определения представляют собой суперпозицию некоторого
числа аффинных функций и операций целочисленного деления на
константу. Анализ ККА-функ\-ций позволяет применять оптимизирующие или
распараллеливающие преобразования~\cite{ShteinbBYa0, Shulj2, Shulj1}, размещать
данные в параллельной памяти~\cite{ShteinbBYa}, получать детальное
описание потока данных в программе~\cite{Feautrier1, Klimov}, оценивать количество
кеш-про\-ма\-хов~\cite{Verdoolaege3}, вычислять объем динамически
выделяемой памяти, генерировать код для многоконвейерных архитектур~\cite{ShteinbR}, 
делать прогноз возможности эффективного
распараллеливания программы~\cite{KritichPut}. Важно, чтобы время
данного анализа не зависело от времени работы программы или
количества итераций в гнезде циклов. Все алгоритмы работают и для
случая ККА-функ\-ций, зависящих от параметров.

Существует два основных подхода к представлению ККА-функ\-ций в
компьютере: предложенные П.~Фотрье деревья вложенных друг в друга
операторов if~\cite{Feautrier2} и описание каждой части области
определения системой неравенств и возвращаемым значением функции для
данной части~\cite{Voevodin, Pugh, Maslov}.
Так как оба представления математически связаны с одним и тем же
объектом, у них много общего. Деревья позволяют сгенерировать более
быстрый программный код, тогда как системы неравенств допускают
обобщение на случай нелинейных зависимостей~\cite{Maslov}
(остальные недостатки деревьев легко преодолимы). В~данной работе
рассматривается линейный класс программ~\cite{Feautrier2}, от
первого пред\-став\-ле\-ния ККА-функ\-ции легко перейти ко второму, к тому
же автор считает, что нелинейные выражения следует анализировать
сводя к линейным. По этим причинам принято решение использовать
деревья.

\begin{figure*}[b] %fig1
\vspace*{-4pt}
 \begin{center}
 \mbox{%
 \epsfxsize=118.522mm
 \epsfbox{gud-1.eps}
 }
 \end{center}
 \vspace*{-11pt}
\Caption{Зависимости между операциями} 
\label{zavisimosti}
\end{figure*}

Для исследования ККА-функ\-ций используются различные библиотеки,
предназначенные для работы с многогранниками. В~Страсбургском
университете создана библиотека PolyLib~\cite{PolyLib}, выполняющая
операции с отдельными многогранниками, их объединениями и решетками;
в проекте автораспараллеливателя GCC Graphite~\cite{Graphite}
используется Parma Polyhedra Library~\cite{PPL}; недавно начала
развиваться библиотека Integer Set Library~\cite{ISL}, позволяющая
работать с целочисленными отношениями. В~данной работе вводится
обобщающее понятие ККА-мно\-же\-ст\-ва, определяемого своей
характеристической ККА-функ\-ци\-ей. Пользователи библиотек PolyLib\linebreak и
PPL сталкиваются с множеством трудностей:\linebreak объединение или разность
выпуклых многогранников не является выпуклым многогранником; если в
соотношениях, описывающих множество, встречаются операции деления
нацело или взятия остатка, приходится рассматривать решетки или
расширять пространство, вводя новые переменные; при вы\-чис\-ле\-нии
образа многогранника при линейном отображении (например, проекции)
также возникают соотношения с операциями деления нацело, описывающие
<<дырявые>> многогранники (содержащие не все целые точки внутри
своей выпуклой оболочки). 

В~данной работе показано, что класс
ККА-мно\-жеств является замкнутым: объединение, разность, образ
ККА-мно\-же\-ст\-ва при ККА-отобра\-же\-нии принадлежат тому же классу.
Приведены несложные алгоритмы построения ККА-мно\-жеств, получающихся
в результате данных операций.

Тема ККА-функций не нова. Все исследователи,  занимающиеся
автоматическим анализом программ, так или иначе с ними сталкиваются,
преобразуют их древовидные представления или\linebreak семейства
соответствующих систем неравенств. К~сожалению, описание алгоритмов
выполнения операций зачастую остается <<за кадром>>, быть может,
ввиду своей громоздкости и рутинности. Цель \mbox{статьи}~--- представить
известные операции в унифицированной, удобной для программирования\linebreak
форме.

Работа имеет следующую структуру. В разд.~2 определены основные
понятия: ККА-мно\-жеств и ККА-функ\-ций. Раздел~3 посвящен представлению
ККА-функ\-ций в виде деревьев, вводится понятие сложности
представления ККА-функ\-ций. В~разд.~4\linebreak описаны алгоритмы
выполнения основных опе\-раций над представлениями ККА-функ\-ций и\linebreak
ККА-мно\-жеств. В~част\-ности, для древовидных пред\-став\-ле\-ний
ККА-мно\-жеств даны алгоритмы по\-стро\-ения пересечения, объединения,
разности, дополнения и выведены оценки сложности получающихся в
результате объектов. Описан алгоритм, определяющий, является ли
заданное ККА-мно\-жест\-во пустым; алгоритм поиска лексикографического
максимума в ККА-мно\-жест\-ве, зависящем от\linebreak параметров. Для
представлений ККА-функ\-ций\linebreak построены необходимые для автоматического
распараллеливания и конвейеризации операции вы\-чис\-ле\-ния
образа/прообраза ККА-мно\-же\-ст\-ва, выведена оценка сложности
получающегося в результате объекта, приведен алгоритм сравнения двух
ККА-функ\-ций. Заметим, что всюду в статье все мно-\linebreak жества
подразумеваются состоящими из це\-ло\-чис\-лен\-ных точек. Большинство
вышеперечисленных алгоритмов реализовано в Открытой
распараллеливающей системе и Диалоговом высокоуровневом
оптимизирующем распараллеливателе, разрабатываемых на факультете
математики, механики и компьютерных наук Южного федерального
университета под руководством Б.\,Я.~Штейнберга~[19--23].

В качестве базисных можно взять следующие четыре операции:
\begin{enumerate}[(1)]
\item суперпозицию ККА-функ\-ций, \\[-15pt]
\item составление из нескольких координатных
ККА-функ\-ций одной, 
\item поиск лексикографического экстремума в
ККА-мно\-жест\-ве как функции па\-ра\-мет\-ров, 
\item редукцию древовидного
пред\-став\-ле\-ния (удаление недостижимых веток).
\end{enumerate}
 Остальные операции
выражаются через базисные. Схема зависимостей между операциями
изображена на рис.~\ref{zavisimosti}.
Здесь $M$~--- ККА-множество; $\Phi$~--- ККА-функ\-ция; символом~$\equiv$
обозначено сравнение ККА-функ\-ций и ККА-мно\-жеств для всех значений
па\-ра\-мет\-ров; $_B|\Phi$~--- ограничение функции на множество в об\-ласти
значений; $k$~--- набор па\-ра\-мет\-ров, от которых зависят ККА-мно\-жест\-ва
и ККА-функ\-ции; $(\Phi_1,\Phi_2)$~--- составленная из двух
координатных ККА-функ\-ция; <<поточечные операции>>~--- операции над
значениями ККА-функ\-ций в точках (сложение, вычитание, умно\-же\-ние 
и~т.\,п.).


\section{Основные понятия}


\smallskip

\noindent
\textbf{Определение 2.1.}
Функцию $\Phi:\mathbb{Z}^m\to\mathbb{Z}^n$ будем называть аффинной, если она представима в виде
$\Phi(I)\hm=AI\hm+b$ с целочисленными матрицей~$A$ и вектором~$b$.

\smallskip

Правая обратная к аффинной функции $\Phi$, вообще говоря, не
принадлежит к классу аффинных функций в смысле
определения~2.1, так как может содержать операции деления.
Дадим индуктивное определение класса квазиаффинных функций,
свободного от данного недостатка.

\smallskip

\noindent
\textbf{Определение 2.2.}
Квазиаффинными функциями являются:
\begin{enumerate}[1)]
\item аффинная функция;
\item функция вида
\begin{equation}
\Phi(I)=I\div d\,, \label{Idivd}
\end{equation}
где $I\in\mathbb{Z}^n$, $d\hm\in\mathbb{N}^n$, <<$\div$>>~--- 
операция цело\-чис\-лен\-но\-го деления ($0\hm\leq a-(a\div b)b\hm<b$);
\item сумма и суперпозиция квазиаффинных функций.
\end{enumerate}

\smallskip

\noindent
\textbf{Теорема~2.1.}
\textit{Функция $\Phi:\mathbb{Z}^m\to\mathbb{Z}^n$ является квазиаффинной тогда и только тогда, когда ее можно представить в виде
\begin{equation}
\Phi(I)=AI+b+Bp(I)\,, \label{defKwaziAff}
\end{equation}
где $A$ и $B$~--- целочисленные матрицы размера $n\times m$ и $n\times k$
соответственно; $k$~--- некоторое число; $b\hm\in\mathbb{Z}^n$; 
$p\hm=p(I)\hm=(p_1,p_2,\ldots,p_k)^\tau$~--- набор нелинейных функций вида}

\noindent
\begin{equation}
\left.
\!\!\!\!\begin{array}{l}
 p_1 = \left(c^{(1)}\cdot I + e_1\right) \div \ell_1\,;\\
 p_2 = \left(c^{(2)}\cdot I + d_{21}p_1 + e_2\right) \div \ell_2\,;\\
 p_3 = \left(c^{(3)}\cdot I + d_{31}p_1 + d_{32}p_2 + e_3\right) \div \ell_3\,;\\
 \ldots \ldots \ldots \ldots \ldots \ldots \ldots \ldots \ldots \ldots\ldots\ldots\ldots\\
 p_k = (c^{(k)}\cdot I + d_{k1}p_1 + \cdots\\
 \hspace*{50pt}\cdots + d_{k,k-1}p_{k-1} + e_k) \div \ell_k\,,
\end{array}\!
\right\}\!\!
\label{defp}
\end{equation}
$c^{(i)}\in\mathbb{Z}^m$, $\ell_i\hm\in\mathbb{N}$, $d_{i,j},e_i\hm\in\mathbb{Z}$, 
\textit{точка} <<$\cdot$>> \textit{означает скалярное произведение.}

\smallskip

Вектор-функ\-ция $p$ есть суперпозиция сумм аффинных и функций
вида~\eqref{Idivd}, следовательно, является квазиаффинной.
Необходимость~\eqref{defKwaziAff} доказывается по индукции,
представлением суперпозиции и суммы функций вида
\eqref{defKwaziAff}--\eqref{defp} в аналогичной форме.

\smallskip

\noindent
\textbf{Определение~2.3.}
Выпуклым многогранником в $\mathbb{Z}^n$ (выпуклым
$\mathbb{Z}$-много\-гран\-ником) будем называть множество
\textit{целочисленных} решений~$I$ системы неравенств
\begin{equation}
RI+t\geq 0 \label{defVip}
\end{equation}
с целочисленными матрицей~$R$ и вектором~$t$.

\smallskip

Выпуклый $\mathbb{Z}$-многогранник может быть неограниченным
множеством, например: полупространство $I_1\hm\leq0$ или все
пространство $\mathbb{Z}^n$.

\smallskip

\noindent
\textbf{Определение~2.4.}
Квазивыпуклым $\mathbb{Z}$-мно\-го\-гран\-ни\-ком $K$ в $m$-мер\-ном
пространстве назовем прообраз некоторого выпуклого
$\mathbb{Z}$-мно\-го\-гран\-ни\-ка $M\subset\mathbb{Z}^n$ ($n$ может не
совпадать с~$m$) при некотором \textit{квазиаффинном} отображении
$\Phi:\mathbb{Z}^m\to\mathbb{Z}^n$
$$
K = \Phi^{-1}(M)\,.
$$


\noindent
\textbf{Теорема~2.2.}
\textit{Множество $K\subset \mathbb{Z}^m$ является квазивыпуклым
$\mathbb{Z}$-много\-гран\-ником тогда и только тогда, когда
существует определяющая его система неравенств вида}
\begin{equation}
I\in K  \Leftrightarrow RI+t+Qp(I) \geq 0 \label{defnerkwazivip}
\end{equation}
\textit{с целочисленными матрицами $R$, $Q$, вектором~$t$. Функции $p\hm=p(I)$ имеют вид}~\eqref{defp}.

\smallskip

Для доказательства достаточно подставить $\Phi(I)$ в систему неравенств, 
определяющую выпуклый  $\mathbb{Z}$-мно\-го\-гран\-ник $M\hm=\Phi(K)$.

Класс квазивыпуклых $\mathbb{Z}$-мно\-го\-гран\-ни\-ков гораздо шире, чем
это может показаться на первый взгляд. Можно доказать, что любое
конечное\linebreak множество целых точек является квазивыпуклым
$\mathbb{Z}$-много\-гран\-ни\-ком. Однако, чтобы представить
некоторые множества в виде~\eqref{defp}--\eqref{defnerkwazivip},
потребуется гораздо больший объем памяти компьютера, чем при
хранении многогранника в виде списка его точек.

Каждый квазивыпуклый $\mathbb{Z}$-мно\-го\-гран\-ник является проекцией
выпуклого $\mathbb{Z}$-мно\-го\-гран\-ника в $(m+k)$-мер\-ном
пространстве переменных $(I,p)$. Последний описывается системой
неравенств~\eqref{defnerkwazivip} с добавочными неравенствами,
определяющими параметры $p_1$, $p_2$, \dots, $p_k$:
\begin{multline}
\hspace*{-5.52576pt}0 \leq \left(c^{(i)}\cdot I + d_{i1}p_1 + \cdots + d_{i,i-1}p_{i-1} + e_i\right)- 
\ell_i p_i \leq {}\\
{}\leq\ell_i -1,\quad 
i=1,2,\ldots,k\,. 
\label{defpInequality0}
\end{multline}
Каждое $i$-е неравенство данной системы эквивалентно $i$-му
равенству~\eqref{defp}.  Является ли проекция квазивыпуклым
$\mathbb{Z}$-мно\-го\-гран\-ни\-ком в общем случае? В~библиотеке операций
над многогранниками PolyLib $\mathbb{Z}$-мно\-го\-гран\-ником называют
пересечение выпуклого многогранника и решетки. Аффинный образ этих
объектов не принадлежит тому же классу множеств~\cite{ZPolylib}. Это
заставило авторов библиотеки ISL~\cite{ISL} определить
$\mathbb{Z}$-мно\-го\-гран\-ник как множество $\{ x\in\mathbb{Z}^n \mid
\exists z\hm\in\mathbb{Z}^m\ Rx\hm+Qz\hm+t\hm\geq0 \}$. Автор считает
вероятным, что для квазивыпуклых $\mathbb{Z}$-мно\-го\-гран\-ни\-ков из
определения~2.4 или эквивалентной ему
теоремы~2.2 справедливо утверждение: образ выпуклого
$\mathbb{Z}$-мно\-го\-гран\-ника, а также образ квазивыпуклого
$\mathbb{Z}$-мно\-го\-гран\-ни\-ка при аффинном отоб\-ра\-же\-нии $\Phi(I)\hm=AI\hm+b$,
$\Phi:\mathbb{Z}^m\hm\to\mathbb{Z}^n$ являются квазивыпуклыми
$\mathbb{Z}$-мно\-го\-гран\-ни\-ка\-ми. Если доказать данную гипотезу, то
окажется, что образы вы\-пук\-лых $\mathbb{Z}$-мно\-го\-гран\-ни\-ков при
\textit{аффинных} отоб\-ра\-же\-ни\-ях и только они являются квазивыпуклыми
$\mathbb{Z}$-мно\-го\-гран\-ни\-ка\-ми. Сложность доказательства можно
ощутить, попытавшись представить в виде~\eqref{defnerkwazivip}
образы нескольких простых многогранников:
\begin{enumerate}[1)]
  \item $M=\{(I_1,I_2)\in\mathbb{Z}^2 \mid I_{1,2}\geq0\}$, 
  $\Phi:\mathbb{Z}^2\to\mathbb{Z}^1$, 
  $\Phi(I_1,I_2)\hm=10I_1\hm+11I_2$;
  \item $M=\{(I_1,I_2,I_3)\in\mathbb{Z}^3 \mid I_{1,2,3}\hm\geq0$, 
  $I_1\hm\geq10I_3$, $I_2\hm\geq11I_3$,  $13I_3\hm\geq I_1+I_2 \}$, 
  $\Phi:\mathbb{Z}^3\hm\to\mathbb{Z}^2$, $\Phi(I_1,I_2,I_3)\hm=(I_1,I_2)$.
\end{enumerate}
Автору удалось доказать квазивыпуклость следующих типов множеств:
конечное множество, периодическое множество (т.\,е.\ с периодической
по всем переменным характеристической функцией), разность
квазивыпуклого и конечного множества, разность квазивыпуклого и
периодического множества. Задача поиска образа легко решается, если
расширить класс рассматриваемых множеств (см.\ ниже).

\smallskip

\noindent
\textbf{Определение~2.5.}
Функцию $\Phi:\mathbb{Z}^m\to\mathbb{Z}^n$ с областью определения
${\cal D}(\Phi)\subset\mathbb{Z}^m$ будем называть ку\-соч\-но-аф\-фин\-ной,
если область ${\cal D}(\Phi)$ представима в виде объединения
конечного числа непересекающихся выпуклых
$\mathbb{Z}$-мно\-го\-гран\-ников, на каждом из которых функция~$\Phi$
аффинная.


Произвольная функция с конечной областью определения ${\cal D}(\Phi)$ является 
ку\-соч\-но-аф\-фин\-ной в смысле данного определения.
Однако число частей ${\cal D}(\Phi)$ может оказаться очень большим,
сравнимым с количеством элементов в~${\cal D}(\Phi)$.

\smallskip

\noindent
\textbf{Определение~2.6.}
Функцию $\Phi:\mathbb{Z}^m\to\mathbb{Z}^n$ с областью определения
${\cal D}(\Phi)\subset\mathbb{Z}^m$ будем называть
\textit{кусочно-ква\-зи\-аф\-фин\-ной}, если область ${\cal D}(\Phi)$
представима в виде объединения конечного числа не\-пе\-ре\-се\-ка\-ющих\-ся
квазивыпуклых $\mathbb{Z}$-мно\-го\-гран\-ников, на каж\-дом из которых
функция~$\Phi$ квазиаффинная.

\smallskip

\noindent
\textbf{Замечание.} Пусть область определения ку\-соч\-но-ква\-зи\-аф\-фин\-ной функции
$\Phi:\mathbb{Z}^m\to\mathbb{Z}^n$ состоит из $\nu$ кусочков;
$p^{(1)}$, $p^{(2)}$, \dots, $p^{(\nu)}$~--- наборы па\-ра\-мет\-ров
\eqref{defp}, соответствующие каждой квазиаффинной функции, заданной
на своем кусочке. Тогда $\Phi$ можно рассматривать как
ку\-соч\-но-аф\-фин\-ную функцию аргумента
$(I,p^{(1)},\ldots,p^{(\nu)})\hm\in\mathbb{Z}^{m+k_1+\cdots+k_\nu}$.

В приложениях многогранники обычно зависят от параметров. В~связи с
этим дадим еще два определения. Здесь и далее все па\-ра\-мет\-ры
считаются целочисленными.

\smallskip

\noindent
\textbf{Определение~2.7.}
Будем говорить, что вы\-пук\-лый/ква\-зи\-вы\-пук\-лый
$\mathbb{Z}$-мно\-го\-гранник $D(r)\subset\mathbb{Z}^n$ линейно
зависит от вектора параметров $r\hm\in\mathbb{Z}^s$, ес\-ли~$D$ для
каждого $r\hm=r_0$ является сечением некоторого фиксированного
выпуклого/квазивыпуклого $\mathbb{Z}$-мно\-го\-гран\-ни\-ка в расширенном
пространстве $\mathbb{Z}^{n+s}$ переменных $(I,r)$ гиперплоскостью
$r\hm=r_0$.


\smallskip

\noindent
\textbf{Теорема~2.3.}
\textit{Выпуклый $\mathbb{Z}$-многогранник $D$ линейно зависит от вектора
параметров $r\hm\in\mathbb{Z}^s$ тогда и только тогда, когда он
описывается системой неравенств, в которой от параметров зависят
только свободные члены, причем зависят линейно:}
\begin{equation*}
I\in D  \Leftrightarrow  AI+t+Br \geq 0\,,
%\label{kwaziDparam}
\end{equation*}
\textit{где $A$ и $B$~--- постоянные матрицы; $t$~--- постоянный вектор.}


\smallskip

\noindent
\textbf{Теорема~2.4.}
\textit{Квазивыпуклый $\mathbb{Z}$-многогранник $D$ линейно зависит от
вектора параметров $r\in\mathbb{Z}^s$ тогда и только тогда, когда он
описывается системой неравенств}
\begin{equation*}
I\in D  \Leftrightarrow  AI+t+Br+Qp \geq 0\,, 
%\label{kwaziDparam2}
\end{equation*}
\textit{где $A$, $B$ и $Q$~--- постоянные матрицы, $t$~--- постоянный вектор,
$p$~-- нелинейные функции вектора переменных~$I$ и вектора параметров~$r$ следующего вида:}
\begin{equation}
\!\left.
\begin{array}{l}
 p_1 = \left(c^{(1)}\cdot I + g^{(1)}\cdot r + e_1\right) \div \ell_1;\\
 p_2 = \left(c^{(2)}\cdot I + g^{(2)}\cdot r + d_{21}p_1 + e_2\right) \div \ell_2,\\
 p_3 = \left(c^{(3)}\cdot I + g^{(3)}\cdot r + d_{31}p_1 + d_{32}p_2 + e_3\right) \div\\
\hspace*{55mm} {}\div \ell_3;\\
 \ldots \ldots \ldots \ldots \ldots \ldots \ldots\ldots \ldots \ldots \ldots\ldots \ldots \ldots\\
 p_k = \left(c^{(k)}\cdot I + g^{(k)}\cdot r + d_{k1}p_1 + \cdots\right. \\
\hspace*{25mm}\left.\cdots + d_{k,k-1}p_{k-1} + e_k\right) \div \ell_k,
\end{array}\!
\right\}\!\!
\label{defpParam}
\end{equation}
$I,c^{(i)}\in\mathbb{Z}^n$, $r,g^{(i)}\hm\in\mathbb{Z}^s$, 
$\ell_i\hm\in\mathbb{N}$, $d_{i,j},e_i\hm\in\mathbb{Z}$, $s$~--- \textit{число параметров.}


\smallskip

\noindent
\textbf{Определение~2.8.}
Будем говорить, что ККА-функ\-ция $\Phi$ линейно зависит от вектора
па\-ра\-мет\-ров~$r$, если от $r$ линейно зависят значения функции и части
области определения ${\cal D}(\Phi)$.

\smallskip

Приведем теорему, которая следует из алгоритма~\cite{Feautrier2}.

\smallskip

\noindent
\textbf{Теорема~2.5}
(\textit{П.~Фотрье}). \textit{Пусть $D$~--- выпуклый $\mathbb{Z}$-многогранник,
линейно зависящий от вектора параметров $r$. Тогда функции
$\Phi_{\max}(r)=\underset{I\in D(r)}{\mathrm{lex.max}}\;I$  и
$\Phi_{\min}(r)=\underset{I\in D(r)}{\mathrm{lex.min}}\;I$ являются
кусочно-квазиаф\-фин\-ными в смысле определения}~2.6.
\textit{Области определения ${\cal D}(\Phi_{\max})$ и ${\cal
D}(\Phi_{\min})$ состоят из тех точек, для которых многогранник
$D(r)$ не пуст и ограничен с соответствующей стороны.}

\smallskip

При работе с ККА-функциями часто возникают структуры в виде
объединений конечного числа непересекающихся квазивыпуклых
$\mathbb{Z}$-мно\-го\-гран\-ни\-ков. Например, к числу таких структур
относятся области определения ККА-функ\-ций.

\smallskip

\noindent
\textbf{Определение~2.9.}
Кусочно-квазиаффинным множеством будем называть объединение конечного числа квазивыпуклых $\mathbb{Z}$-многогранников.

\smallskip

\noindent
\textbf{Пример~2.1.}
Рассмотрим цикл 

\noindent
\begin{verbatim}
for (i = 0; i < N; i++)
{
    x[i+a] = ...
    x[2*i] = ...
    ... = ... x[i] ...
}
\end{verbatim}

Пространство итераций цикла зависит
от параметра~$N$. В~одно из индексных выражений входит другой
параметр~$a$. Предполагается, что $a\hm\geq0$ и $N\hm\geq1$. Граф истинной
информационной за\-ви\-си\-мости~\cite{Voevodin,ShteinbBYa0},
соответствующий вхождениям массива~$x$, состоит из дуг $j\to i$,
соединяющих итерации цик\-ла так, что на итерации~$i$ произошло чтение
из ячейки массива $x[i]$, а на итерации $j<=i$~--- последняя перед
итерацией $i$ запись в эту же ячейку, т.\,е. $j\hm=
\mathop{\rm{lex.max}}\limits_{\substack{{j^\prime<=i,}\\{j^\prime+a=i \text{ или } 2j^\prime=i}}}
j^\prime$.
Область, в которой ищется лексикографический максимум, является
ККА-мно\-же\-ст\-вом (объединением двух выпуклых
$\mathbb{Z}$-мно\-го\-гран\-ни\-ков), линейно зависящим от параметров~$a$ и~$i$. 
Ниже будет доказано обобщение теоремы~2.5,
согласно которому дуги такого графа можно задать ККА-функ\-ци\-ей
$j\hm=\Phi(i)$. В~данном случае

\noindent
$$
\Phi(i) =
\begin{cases}
i-a\,, &\!\text{если}\  i\geq a\,,\  i\%2=0\,, \ i-a > i\div 2, \\
& \hspace*{30mm}0\leq i < N;\\
i\div2\,, &\!\text{если}\  i\geq a\,, \  i\%2=0\,, \ i-a \leq i\div 2, \\
&\hspace*{30mm} 0\leq i < N;\\
i-a\,, &\!\text{если}\  i\geq a\,, \  i\%2\neq0\,, \ 0\leq i < N;\\
i\div2\,, &\!\text{если}\  i<a\,, \  i\%2=0\,, \ 0\leq i < N.
\end{cases}
$$
Символом <<$\%$>> обозначена операция взятия остатка от деления
($0\hm\leq k\% m\hm<m$). Область опре\-деления функции $\Phi$: 
${\cal D}(\Phi)\hm=\{ i \mid 0\hm\leq i \hm< N$,\linebreak $i\hm\geq a \text{ или } i\%2=0 \}$
представима в виде объединения четырех непересекающихся
квазивыпуклых $\mathbb{Z}$-многогранников: ${\cal D}(\Phi) \hm= {\cal
D}_1 \cup {\cal D}_2\cup{\cal D}_3\cup {\cal D}_4$, где
\begin{align*}
{\cal D}_1 &= \{ i \mid i\geq a\,, \,  i\%2=0\,,\ i-a > i\div2\,,\, 0\leq i < N \};\\
{\cal D}_2 &= \{ i \mid i\geq a\,, \,  i\%2=0\,,\ i-a \leq i\div2\,,\, 0\leq i < N \};\\
{\cal D}_3 &= \{ i \mid i\geq a\,,\,   i\%2\neq0\,,\, 0\leq i < N \}\,;\\
{\cal D}_4 &= \{ i \mid i < a\,, \,  i\%2=0\,,\, 0\leq i < N \}\,.
\end{align*}
Выпишем неравенства~\eqref{defnerkwazivip} для $\mathbb{Z}$-мно\-го\-гран\-ни\-ков~${\cal D}_1$, 
${\cal D}_2$, ${\cal D}_3$ и ${\cal D}_4$: 
\begin{alignat*}{4}
{\cal D}_1:\ &&
\begin{array}{rl}
i-a &\geq 0\,;\\
-i+2p &\geq 0\,;\\
i-a-p-1 &\geq 0\,;\\
i &\geq 0\,;\\
-i+N-1 &\geq 0\,;
\end{array}\qquad
&&
{\cal D}_2:\ &&
\begin{array}{rl}
i-a &\geq 0\,;\\
-i+2p &\geq 0\,;\\
-i+a+p&\geq 0\,;\\
i &\geq 0\,;\\
-i+N-1 &\geq 0\,;
\end{array}
\\[6pt]
{\cal D}_3:\ &&
\begin{array}{rl}
i-a &\geq 0\,;\\
i-2p-1 &\geq 0\,;
\end{array}  \qquad
&&
{\cal D}_4:\ &&
\begin{array}{rl}
-i+a-1 &\geq 0\,;\\
-i+2p &\geq 0\,,
\end{array}
\end{alignat*}
где $p=i\div2$.

\vspace*{-3pt}

\section{Формы представления кусочно-квазиаффинных функций}

На практике кусочно-квазиаффинные функции удобно хранить в виде
бинарных деревьев, которые получаются на выходе параметризованного
метода Гомори (см.\ документацию к~\cite{Feautrier3}). Такая форма
представления ККА-функ\-ций позволяет быст\-ро вычислять значения и
проводить операции над функциями. По сути, это дерево бинарного
поиска нужного кусочка области определения ККА-функ\-ции.

\begin{figure*} %fig2
\vspace*{1pt}
 \begin{center}
 \mbox{%
 \epsfxsize=98.58mm
 \epsfbox{gud-2.eps}
 }
 \end{center}
 \vspace*{-9pt}
\Caption{Схема программы из примера~2.1}
\end{figure*}



\smallskip

\noindent
\textbf{Определение~3.1.}
Будем называть формой Фотрье~\cite{Feautrier2} представления
ККА-функции $\Phi:\mathbb{Z}^m\to\mathbb{Z}^n$, квазилинейно
зависящей от вектора параметров~$r$, бинарное дерево вложенных друг
в друга операторов if, удовлетворяющее условиям:
\begin{enumerate}
\item Каждый узел дерева, не являющийся листом, состоит из
    \begin{itemize}
    \item набора определений локальных па\-ра\-мет\-ров~$p$ вида~\eqref{defpParam} (возможно пустого);
    \item оператора ветвления \\
    if \ \ $(U)$ \ \  $\Phi_{\mathrm{true}}$\\
    else \ \  $\Phi_{\mathrm{false}}$,\\
    где $U$~--- условие неотрицательности аффинной формы от аргумента функции, вектора 
    па\-ра\-мет\-ров~$r$ и набора локальных па\-ра\-мет\-ров~$p$ данного и 
    вышестоящих узлов до корня; $\Phi_{\mathrm{true}}$ и $\Phi_{\mathrm{false}}$~--- 
    поддеревья по веткам true и false.\\[-15pt]
    \end{itemize}
\item Листья состоят из
    \begin{itemize}
    \item набора определений локальных па\-ра\-мет\-ров~$p$ вида~\eqref{defpParam} 
    (возможно пус\-то\-го);\\[-15pt]
    \item $n$-мер\-но\-го вектора~--- значения функции,~--- 
    возвращаемого оператором return (аффинная век\-тор\--функ\-ция аргумента,  
    па\-ра\-мет\-ров~$r$ и набора локальных па\-ра\-мет\-ров~$p$\linebreak данного и вышестоящих узлов 
    до корня) или NULL, если на данной подобласти функция не определена.\\[-15pt]
    \end{itemize}

\item Областью определения такой функции будем считать множество, на котором функция 
возвращает значения, отличные от NULL (NULL и возвращаемое значение~0 предполагаются 
различными).
\end{enumerate}

\smallskip

В документации к~\cite{Feautrier3} данному определению соответствует грамматика 
для~\verb+Quast_group+ (QUAST~--- QUasi Affine Selection Tree).

Каждое такое дерево естественно представляется в виде программы на 
каком-либо языке программирования. Например, функции из примера~2.1
соответствует программа на языке Си:
\begin{verbatim}
ValueType  Ф(int i)  {
  if (i>=0)  {
     if (i<N)  {
        int p = i/2;
        if  ( i-a>=0 )  {
           if  ( -i+2*p>=0 )  {
               if ( i-a-p>0 )  {return i-a;}
               else  {return p;}
           }
           else  {return i-a;}
        }
        else   {
            if  ( -i+2*p>=0 )  {return p;}
            else {return NULL;}
        }
     }
     else  {return NULL;}
  }
  else  {return NULL;}
}
\end{verbatim}
Она описывается двоичным деревом, показанным на рис.~2.


\smallskip

\noindent
\textbf{Лемма~3.1.}
\textit{Форма Фотрье корректно задает функцию. Эта функция является 
кусочно-квазиаффинной в смысле определения}~2.6.

\smallskip

Одной ККА-функции может соответствовать несколько различных представлений в виде де\-ревь\-ев. 
Введем характеристики дерева.

\noindent
\textbf{Обозначения.} Через $\mathrm{NodeNum}(\Phi)$ и
$\mathrm{LeafNum}(\Phi)$ будем обозначать число узлов <<if>> и число
листь\-ев дерева ККА-функ\-ции, которые не возвращают NULL.

\smallskip

\noindent
\textbf{Определение~3.2.}
Сложностью $S_\infty$ представления ККА-функции в виде дерева будем
называть максимальное количество вычислений неравенств в условиях
операторов if, необходимое для того, чтобы посчитать значение
функции в точке. Таким образом, сложность $S_\infty$~--- это
максимальная глубина дерева.

\smallskip

Сложность приближенно показывает, насколько быстрым будет
соответствующий дереву код на ка\-ком-ли\-бо языке программирования.
Наряду с $S_\infty$ для ККА-функции $\Phi$ с ограниченной областью
определения можно рассматривать сложность 
$$
S_p=\left(\fr{1}{|{\cal D}(\Phi)|} \sum\limits_{I\in{\cal D}(\Phi)} \mu(I)^p
\right)^{1/p}\,,
$$ 
где $\mu(I)$~--- количество неравенств,
проверяемых при вычислении $\Phi(I)$, $p\hm\in\mathbb{R}$, $p\hm\geq1$.

При выполнении операций с ККА-функциями удобно пользоваться
расширенным представлением, которое отличается от формы Фотрье лишь
тем, что в условиях операторов if разрешено использовать
<<short-circuit>> операции логического <<и>>, <<или>> и отрицания в
произвольных комбинациях с неравенствами.

\smallskip

\noindent
\textbf{Определение~3.3.}
Будем называть такую форму представления ККА-функции расширенной формой.

\smallskip

Например, у представления на рис.~2 можно два верхних оператора if объединить в один:
\begin{verbatim}
if ((i>=0) && (i<N))
{ ...  }
else return NULL.
\end{verbatim}

\smallskip

\noindent
\textbf{Определение~3.4.}
Сложностью $S_\infty$ расширенного представления ККА-функции также
будем называть максимальное количество вычислений неравенств в
условиях операторов if, необходимое для того, чтобы посчитать
значения функции в точке. При этом предполагается, что логические
операции производят вычисление своих операндов слева направо и
вычисляют минимальное число операндов, необходимое для определения
результата выражения (т.\,е.\ являются <<short-circuit>>).

\smallskip

\smallskip

\noindent
\textbf{Лемма~3.2.}
\textit{Для каждой расширенной формы существует форма Фотрье, задающая ту же 
функцию с такой же сложностью.}


\smallskip


\noindent
\textbf{Лемма~3.3.}
\textit{Любую ККА-функцию можно представить в расширенной форме и, в соответствии с леммой}~3.2,
\textit{в форме Фотрье}.

\smallskip

\noindent
\textbf{Теорема~3.1.}
\textit{Класс функций, которые можно задать при помощи расширенного представления, совпадает с классом функций, задаваемых формой
Фотрье и равен классу ККА-функций.}


\smallskip

\noindent
Д\,о\,к\,а\,з\,а\,т\,е\,л\,ь\,с\,т\,в\,о\ теоремы следует из лемм~3.1--3.3.

\smallskip

Кусочно-квазиффинные множества удобно хранить в виде древовидных структур,
подобных представлениям ККА-функ\-ций. Свяжем с произвольным таким
множеством $M$ характеристическую функцию $P_M$, которая равна
единице на $M$ и не определена (return NULL) во всех остальных
точках. Функция $P_M$~--- ку\-соч\-но-ква\-зи\-аф\-фин\-ная\footnote{Приставку
<<квази>> здесь опустить нельзя, так как кусочки области определения
$P_M$ являются \textit{квази}выпуклыми
$\mathbb{Z}$-мно\-го\-гран\-ни\-ками.}.

\smallskip

\noindent
\textbf{Определение~3.5.}
Представлением ККА-мно\-жест\-ва $M$ в виде дерева операторов if
будем называть представление соответствующей характеристической
ККА-функции $P_M$.

\smallskip

\noindent
\textbf{Определение~3.6.}
Сложностью пред\-став\-ле\-ния ККА-мно\-же\-ст\-ва $M$ будем называть слож\-ность пред\-став\-ле\-ния 
ККА-функ\-ции~$P_M$.

\smallskip

Если $M$~---  квазивыпуклый $\mathbb{Z}$-мно\-го\-гран\-ник, описываемый
системой из $m$ неравенств, то соответствующее функции $P_M$ дерево
линейно: оно состоит из вложенных друг в друга операторов if с
возвращаемым значением NULL по ветке else. Значит, сложность
$S_\infty$ такого представления $P_M$ равна~$m$.


\section{Операции с~кусочно-квазиаффинными функциями}

Все ККА-множества и ККА-функции в данном разделе могут квазилинейно
зависеть от внешних параметров. Заметим, что ККА-функцию
$\Phi_r:\mathbb{Z}^m\to\mathbb{Z}^n$, зависящую от набора
$r\hm=(r_1,r_2,\ldots,r_s)$ целочисленных параметров, можно
рассматривать как обычную ККА-функ\-цию $m+s$ целых переменных
$(I,r)$ с областью определения ${\cal D}(\Phi_r)\hm=\{ (I,r)\hm\in
\mathbb{Z}^{m+s} \mid \Phi_r(I)\hm\neq \mathrm{NULL} \}$. Так же и
ККА-множество $M_r$, зависящее от набора параметров $r$, можно
рассматривать как обычное ККА-мно\-же\-ст\-во в расширенном пространстве
пе\-ре\-мен\-ных-па\-ра\-мет\-ров. Для многих операций смысловое отличие
части переменных $r$ от $I$ не имеет значения. Случаи, где наличие
параметров играет существенную роль, будут оговариваться отдельно.

Чтобы посчитать функцию $\Phi_r$ для некоторого фиксированного
набора параметров $r\hm=r_0$, нужно положить в древовидном
представлении~$\Phi_r$ $r\hm=r_0$. Область определения полученной
функции ${\cal D}(\Phi_{r_0})$ является сечением ${\cal
D}(\Phi_r)\subset \mathbb{Z}^{m+s}$ гиперплоскостью $r\hm=r_0$. Чтобы
определить, для каких значений параметров~$r$ область определения
$\Phi_r$ непуста, нужно вычислить проекцию ${\cal D}(\Phi_r)\subset
\mathbb{Z}^{m+s}$ на пространство параметров $\mathbb{Z}^s$
(вычисление проекции сводится к вычислению образа и рассматривается
в п.~4.2.6). Точно так же вычисляется множество значений
параметров, для которых непусто произвольное ККА-мно\-же\-ство.

\subsection{Базисные операции} %4.1


\subsubsection{Суперпозиция кусочно-квазиаффинных функций} %4.1.1 

Суперпозиция $\Phi_1\circ\Phi_2=\Phi_1(\Phi_2)$ строится следующим
образом. Пусть функция $\Phi_1$ описывается деревом $T_1$, а функция
$\Phi_2$~--- $T_2$. Каждый оператор <<return $q_2$>> дерева $T_2$
заменим на конструкцию $T'_1$, которая получается из $T_1$ заменой
аргумента во всех условиях и возвращаемых выражениях на $q_2$.
Сложность суперпозиции подчиняется неравенству
$S_\infty(\Phi_1\circ\Phi_2)\hm\leq S_\infty(\Phi_1) \hm+
S_\infty(\Phi_2)$.

\subsubsection{Составление из~нескольких координатных кусочно-квазиаффинных 
функций одной} %4.1.2 

Пусть $\Phi_1$, $\Phi_2$~--- ККА-функции, действующие из
$\mathbb{Z}^m$ в $\mathbb{Z}^{n_1}$ и $\mathbb{Z}^{n_2}$ с областями
определения ${\cal D}(\Phi_1)$, ${\cal D}(\Phi_2)$ соответственно.
Составим из них функцию $\Phi \hm=(\Phi_1,\Phi_2):\mathbb{Z}^m\to\mathbb{Z}^{n_1+n_2}$ 
с областью
определения ${\cal D}(\Phi)\hm={\cal D}(\Phi_1)\cap{\cal D}(\Phi_2)$.
Она является ККА, так как на частях области
определения~--- всевозможных попарных пересечениях частей ${\cal
D}(\Phi_1)$ и ${\cal D}(\Phi_2)$~--- функция $\Phi\hm=(\Phi_1,\Phi_2)$
квазиаффинная. Построим соответствующее ей дерево. Пусть $T_1$ и
$T_2$~--- древовидные представления функций $\Phi_1$ и $\Phi_2$.
Каждый оператор <<return $q_1$>> дерева $T_1$ заменим на конструкцию
$T_2$, в которой вмес\-то операторов <<return $q_2$>>, возвращающих
значение функции $\Phi_2$, стоят операторы <<return $(q_1,q_2)$>>,
возвращающие значение функции~$\Phi$. Сложность $S_\infty$
построенного таким образом дерева не превышает суммы сложностей
$S_\infty(\Phi_1)$ и $S_\infty(\Phi_2)$.


\subsubsection{Удаление недостижимых веток} %4.1.3 

В процессе выполнения операций над ККА-функ\-ци\-ями возникают
древовидные представления со значениями на листьях, которые никогда
не возвращаются функцией, потому что соответствующие им части
области определения не содержат целых точек. Опишем преобразование,
которое удаляет все такие недостижимые ветви. Если ККА-функция
зависит от параметров линейным образом, то будем считать эти
параметры дополнительными аргументами. Таким образом, будут удалены
ветви дерева, являющиеся недостижимыми для всех значений параметров.

\textit{Алгоритм.} В цикле, пока в дереве $\Phi$ существует 
поддерево $\phi$, удовлетворяющее одному из сле\-ду\-ющих условий, выполняем:
\begin{enumerate}
\item Если поддерево $\phi$ не является листом и все его листья возвращают NULL, заменим $\phi$ на лист return~NULL.
\item Пусть $\phi$ не является листом:
$$ 
\phi = 
\begin{array}{ll}
\mbox{if}\  (U) &\ \phi_{\mathrm{true}}\,,\\ 
\mbox{else} &\ \phi_{\mathrm{false}}\,.
\end{array}
$$
Рассмотрим путь от корня дерева~$\Phi$ к поддереву~$\phi$. Объединим
в систему условия операторов if вдоль пути. При этом если после
очередной вершины вдоль пути движение происходит по ветке <<true>>,
то включаем условие <<как есть>>, а если по ветке <<false>>~--- то
включаем противоположное условие. Данная система вместе с
неравенствами~\eqref{defpInequality0}, определяющими локальные
параметры $p$, описывает некоторое ККА-мно\-же\-ст\-во. Обозначим его~$B$.

Если условие $U$ или его отрицание $\neg U$ не имеют целых решений
внутри $B$, то в первом случае заменяем $\phi$ на
$\phi_{\mathrm{false}}$, а во втором~--- на $\phi_{\mathrm{true}}$.
Проверку совместимости условия внутри $B$ можно осуществлять любым
алгоритмом проверки разрешимости в целых числах системы неравенств,
например тем же методом отсечений Гомори, который используется при
поиске лексикографического экстремума в многограннике; можно
применить Омега-тест \cite{Pugh}.
\end{enumerate}
Удаление недостижимых веток и лишних условий, выполненное в п.\,2
алгоритма, приводит к эквивалентному дереву, все листья которого
соответствуют непустым в $\mathbb{Z}^n$ частям области определения
ККА-функ\-ции~$F$. Более того, благодаря п.\,1 из дерева удаляются
ветви, у которых все листья возвращают NULL. Таким образом, после
применения данного преобразования пустые ККА-множества и только они
сводятся к единственному узлу <<return NULL>>.

Чтобы выполнить это преобразование в худшем случае потребуется
проверить разрешимость в целых числах $2\mathrm{NodeNum(F')}$ систем
не более чем $S_\infty(F')$ неравенств, где $F'$~--- форма Фотрье
функции~$F$.


\subsubsection{Лексикографические экстремумы множеств, зависящих от~параметров} %4.1.4 

\noindent
\textbf{Обозначение.} $S_\infty(\mathrm{lex.max}\; D)$~--- оценка сложности дерева 
ККА-функции из теоремы~2.5 П.~Фотрье~\cite{Feautrier2}.

\medskip

\noindent
\textbf{Теорема~4.1}
(\textit{обобщение теоремы Фотрье на квазивыпуклые
$\mathbb{Z}$-много\-гран\-ники}). \textit{Пусть $D$~-- квазивыпуклый
$\mathbb{Z}$-многогранник, квазилинейно зависящий от вектора
параметров $r=(r_1,\ldots,r_s)^\tau$. Тогда функции
$\Phi_{\max}(r)=\mathop{\mathrm{lex.max}}\limits_{I\in D(r)}I$ и
$\Phi_{\min}(r)=\mathop{\mathrm{lex.max}}\limits_{I\in D(r)} I$ являются
ККА в смысле определения}~2.6.
\textit{Существуют их древовидные представления со сложностью, не больше чем
$S_\infty(\mathrm{lex.max} \widetilde D)$ и
$S_\infty(\mathrm{lex.min} \widetilde D)$, где $\widetilde D$~---
соответствующий $D$ выпуклый $\mathbb{Z}$-многогранник в
пространстве переменных $(I,p)$} (\textit{см.}~\eqref{defpParam}). 

\medskip

Задача сводится к поиску лексикографического экстремума в обычном
выпуклом $\mathbb{Z}$-мно\-го\-гран\-ни\-ке~$\widetilde D$ в расширенном
пространстве переменных $(I,p)$ размерности $n+k$ и взятию первых
$n$ компонент результата.

\smallskip

\noindent
\textbf{Лемма~4.1.}
\textit{Функция $\mathrm{LM}:\mathbb{Z}^{2n}\to\mathbb{Z}^{n}$,
$\mathrm{LM}(I,J)\hm=\mathrm{lex.max}\;\{I,J\}$ является кусочно-аффинной, и для
нее существует представление со сложностью} $S_\infty(\mathrm{LM})=2n-1$.

\medskip


\noindent
\textbf{Теорема~4.2}
(\textit{обобщение теоремы Фотрье на ККА-мно\-же\-ст\-ва}). \textit{Пусть
$M\subset\mathbb{Z}^n$~-- ККА-мно\-же\-ст\-во, квазилинейно зависящее от
вектора параметров $r\hm=(r_1,\ldots,r_s)^\tau$. Тогда функции
$\Phi_{\max}(r)\hm=\mathop{\mathrm{lex.max}}\limits_{I\in M(r)}I$ и
$\Phi_{\min}(r)=\mathop{\mathrm{lex.max}}\limits_{I\in M(r)}I$ являются
ККА в смысле определения}~2.6. \textit{Сложность
$S_\infty$ представления функции $\Phi_{\max}(r)$ подчиняется
неравенству}:
\begin{multline}
S_\infty(\Phi_{\max})\leq{}\\
{}\leq (\nu-1)(2n-1) + \sum\limits_{i=1}^\nu S_\infty(\mathrm{lex.max}\;D_i)\,, \label{SestimateLexMaxKKA}
\end{multline}
\textit{где $D_i$~--- квазивыпуклые $\mathbb{Z}$-многогранники, со\-став\-ля\-ющие
ККА-мно\-жест\-во $M$, $i\hm=1,2,\ldots,\nu$, $\nu\hm=\mathrm{LeafNum}(M)$,
$S_\infty(\mathrm{lex.max}\;D_i)$~--- слож\-ность пред\-став\-ле\-ния
ККА-функции из теоремы}~4.1. \textit{Для $\Phi_{\min}$ в неравенстве}~\eqref{SestimateLexMaxKKA} 
\textit{нужно $\mathrm{lex.max}$ поменять на}
$\mathrm{lex.min}$. 

\medskip

\noindent
Д\,о\,к\,а\,з\,а\,т\,е\,л\,ь\,с\,т\,в\,о\ \ проведем для максимума. Согласно теореме~4.1
вычислим лексикографические максимумы внутри всех многогранников
$D_i$. Получим $\nu$ ККА-функ\-ций $f_1$, $f_2$, \dots, $f_\nu$
вектора параметров~$r$. Искомая функция
$$
\Phi_{\max}(r) = \mathrm{lex.max}\{f_1(r),f_2(r),\ldots,f_\nu(r)\}$$
является суперпозицией функций $\mathrm{LM}(I_1,I_2,\ldots$\linebreak $\ldots,I_\nu)\hm=
\mathrm{lex.max}\{I_1,I_2,\ldots,I_\nu\}$ и 
$F(r) \hm= \left (f_1(r),f_2(r),\ldots,f_\nu(r)\right)$. 
Первая является ку\-соч\-но-аф\-фин\-ной в силу леммы~4.1, 
вторая~--- поскольку составлена из ККА-функ\-ций. Суперпозиция ККА-функций тоже является 
ККА. Оценим сложность $S_\infty$ ее пред\-став\-ле\-ния. Слож\-ность 
суперпозиции не больше суммы сложностей ее компонент. Функцию $\mathrm{LM}(I_1,\ldots,I_\nu)$ 
можно представить как суперпозицию $(\nu-1)$ функций $\mathrm{lex.max}$ от двух аргументов, 
поэтому $S_\infty(\mathrm{LM}(I_1,I_2,\ldots,I_\nu))\hm\leq (\nu-1)(2n-1)$. В~итоге получим
\begin{multline*}
S_\infty(\Phi_{\max})\leq S_\infty(\mathrm{LM}) + S_\infty(F) \leq{}\\
{}\leq (\nu-1)(2n-1) 
+ \sum\limits_{i=1}^\nu S_\infty(\mathrm{lex.max}\,D_i)\,.
\end{multline*}


\subsection{Небазисные операции} %4.2

\subsubsection{Поточечные операции над кусочно-квазиаффинными функциями} 

Поточечные операции над ККА-функциями, например сложение 
$\Phi_1:\mathbb{Z}^m\to\mathbb{Z}^n$ и
$\Phi_2:\mathbb{Z}^m\to\mathbb{Z}^n$, являются суперпозицией
составной функции $\Phi\hm=(\Phi_1, \Phi_2)$ и соответствующей
операции: $\Phi_1+\Phi_2\hm= \mathrm{Sum} \circ \Phi$, где функция
$\mathrm{Sum}:\mathbb{Z}^{2n}\to\mathbb{Z}^n$ действует по правилу
$\mathrm{Sum}(J_1,J_2)\hm=J_1+J_2$. Сложность представления результата
не превосходит суммы сложностей деревьев $\Phi_1$ и $\Phi_2$.

\subsubsection{Объединение, пересечение, разность и~дополнение
кусочно-квазиаффинных множеств} %4.2.2

Характеристические функции объединения, пересечения и дополнения
ККА-множеств получаются в результате поточечных операций над
характеристическими функциями аргументов: $P_{M_1\cup M_2} \hm= P_{M_1}
\vee P_{M_2}$, $P_{M_1\cap M_2} \hm= P_{M_1} P_{M_2}$, $P_{\overline M}
\hm= \neg P_{M}$. Разность выражается через остальные операции.

Если множество зависит от параметров, то считается, что оно пустое
для тех параметров, для которых не определено. В некоторых задачах
это не так, и тогда приходится еще применять операцию пересечения со
множеством допустимых значений параметров.

\subsubsection{Прообраз кусочно-квазиаффинного 
множества при кусочно-квазиаффинном отображении} %4.2.3

Характеристическая функция прообраза ККА-мно\-же\-ст\-ва $B\subset
\mathbb{Z}^n$ при ККА отображении
$\Phi:\mathbb{Z}^m\hm\to\mathbb{Z}^n$ выражается через известные
операции по формуле $P_{\Phi^{-1}(B)} \hm= P_B\circ \Phi$. Значит,
$S_\infty(\Phi^{-1}(B)) \hm\leq S_\infty(B)\hm+S_\infty(\Phi)$.


\subsubsection{Ограничение кусочно-квазиаффинной функции на~множество в~области значений
аргумента и~в~области значений функции} %4.2.4

Ограничение $\Phi|_A$ ККА-функции $\Phi:\mathbb{Z}^m\to\mathbb{Z}^n$
на ККА-множество $A\subset \mathbb{Z}^m$ вычисляется по формуле
$\Phi|_A \hm= E\circ \Psi$, где
$\Psi:\mathbb{Z}^m\to\mathbb{Z}^{n+1}$~--- составная функция
$\Psi\hm=(\Phi,P_A)$; $E:\mathbb{Z}^{n+1}\hm\to\mathbb{Z}^{n}$~--- оператор
проектирования на $\mathbb{Z}^n$: $E(J,k)\hm=J$, $J\hm\in\mathbb{Z}^n$,
$k\hm\in\mathbb{Z}$. Отсюда $S_\infty(\Phi|_A) \hm\leq
S_\infty(A)\hm+S_\infty(\Phi)$.

\smallskip

\noindent
\textbf{Определение 4.1.}
Ограничением функции $\Phi$ на множество $B$ в области значений
будем называть функцию $\Phi_1$, действующую так же, как и $\Phi$ с
областью определения ${\cal D}(\Phi_1)\hm=\{ I\in {\cal D}(\Phi) \mid
\Phi(I)\hm\in B \}$.  Будем обозначать такое ограничение: $_B|\Phi$.

\smallskip

Ясно, что $_B|\Phi = \Phi|_{\Phi^{-1}(B)} \hm= E\circ (\Phi, P_B\circ
\Phi)$. Заметим, что функцию $(\Phi, P_B\circ \Phi)$ можно составить
так, чтобы ее сложность не превышала суммы
$S_\infty(B)\hm+S_\infty(\Phi)$. Следовательно, $S_\infty(_B|\Phi) \hm\leq
S_\infty(B)\hm+S_\infty(\Phi)$.


\subsubsection{Обращение кусочно-квазиаффинной функции} %4.2.5

Произвольно взятая ККА-функция $\Phi:\mathbb{Z}^m\hm\to\mathbb{Z}^n$
может оказаться неинъективной и, следовательно, необратимой. Однако
даже в этом случае у нее существуют правые обратные функции
$\Phi^{-1}$ такие, что для всех значений параметров и всех точек $J$
из области значений функции $\Phi$ (т.\,е.\ образа области
определения) выполняется равенство:
$$
\Phi(\Phi^{-1}(J))=J\,.
$$
У точки $J$ из области значений $\Phi$ может быть несколько
прообразов, следовательно, и правых обратных функций существует
множество. Многие из них имеют очень большую сложность. Рассмотрим
две правых обратных функции: лексикографически минимальную и
лексикографически максимальную:
\begin{equation}
  \Phi^{-1}_{\min}(J) = 
\mathop{\mathrm{lex.min}}\limits_{ \substack{{K\in{\cal D}(\Phi);}\\ {\Phi(K)=J} }} K\,; 
\enskip \Phi^{-1}_{\max}(J) =\mathop{\mathrm{lex.max}}\limits_{\substack{{K\in{\cal D}(\Phi)};\\ 
{\Phi(K)=J}} }K. \!\!
\label{invLexMinMax}
\end{equation}
Предполагается, что область определения функции~$\Phi$ ограничена
для всех значений параметров. Таким образом, лексикографические
экстремумы существуют для всех $J$ из образа $\Phi$.

\smallskip

\noindent
\textbf{Теорема~4.3}
\textit{Функции $\Phi^{-1}_{\min}$ и $\Phi^{-1}_{\max}$ являются
ККА. Сложность представления $\Phi^{-1}_{\max}$
подчиняется неравенству}:
\begin{multline}
S_\infty(\Phi^{-1}_{\max})\leq{}\\
{}\leq (\nu-1)(2m-1) + 
\sum\limits_{i=1}^\nu S_\infty(\mathrm{lex.max}\,D_i)\,, 
\label{SestimateInverse}
\end{multline}
\textit{где $D_i$~-- квазивыпуклые $\mathbb{Z}$-многогранники, со\-став\-ля\-ющие
ККА-множество $M=\{K\in{\cal D}(\Phi)\mid\Phi(K)=J\}$, зависящее от
вектора параметров $J$, $i=1,2,\ldots,\nu$,
$\nu=\mathrm{LeafNum}(M)=\mathrm{LeafNum}(\Phi)$,
$S_\infty(\mathrm{lex.max}\;D_i)$~--- сложность представления
ККА-функ\-ции из теоремы}~4.1. \textit{Для $\Phi_{\min}$ в неравенстве}~\eqref{SestimateInverse} 
\textit{нужно $\mathrm{lex.max}$ поменять на
$\mathrm{lex.min}$.}

\smallskip

Данное утверждение непосредственно получается, если применить
теорему~4.2 к формулам~\eqref{invLexMinMax}.

Если область ${\cal D}(\Phi)$ не ограничена, то
ку\-соч\-но-ква\-зи\-аф\-фин\-ную правую обратную функцию можно найти, сделав
замену $I\hm=R(r)\hm=(r_1\hm-r_2,r_2\hm-r_3,\ldots,r_n\hm-r_{n+1})$ и перейдя к
неотрицательным переменным~$r$ (условия неотрицательности должны
быть записаны в древовидном пред\-став\-ле\-нии функции~$R$). Тогда
искомая правая обратная к~$\Phi$ функция получается в результате
суперпозиции $R\circ(\Phi\circ R)^{-1}$, где $(\Phi\circ R)^{-1}$~---
лексикографически минимальная правая обрат\-ная к $\Phi\circ R$.

\subsubsection{Образ кусочно-квазиаффинного 
множества при~кусочно-квазиаффинном отображении} %4.2.6

\noindent
\textbf{Определение~4.2.}
Образом $\Phi(B)$ множества $B\subset \mathbb{Z}^m$ при отобра\-же\-нии
$\Phi:\mathbb{Z}^m\to\mathbb{Z}^n$ будем называть множество всех
целых точек $\Phi(I)\hm\in \mathbb{Z}^n$ таких, что $I\hm\in B$.

\smallskip

Заметим, что образом отрезка $[1;5]_\mathbb{Z}$ при отобра\-же\-нии
$\Phi(x)\hm=2x$ является мно\-жест\-во $\{2,4,6,8,10\}$, а не отрезок
$[2,10]$. Отображения, не дейст\-ву\-ющие в $\mathbb{Z}^n$, например,
$\Phi(x)\hm={x}/2$, $x\hm\in\mathbb{Z}$, здесь не рас\-смат\-ри\-ва\-ют\-ся,
хотя для многих остальных операций такое обобщение возможно.

Сведем задачу к поиску прообраза множества~$B$ при правом обратном
отображении $(\Phi|_B)^{-1}$ к ограничению $\Phi|_B$. Тогда
$\Phi(B)\hm=((\Phi|_B)^{-1})^{-1}(B)$ и $S_\infty(\Phi(B)) \hm=
S_\infty((\Phi|_B)^{-1})$. В~качестве $(\Phi|_B)^{-1}$ можно взять
любую ККА правую обратную функцию к $\Phi|_B$,
например лексикографически максимальную (минимальную).

\subsubsection{Проверка на~пустоту и~равенство для~всех~значений
параметров} %4.2.7

Кусочно-квазиаффинное множество пусто при всех значениях параметров тогда и только
тогда, когда после преобразования удаления недостижимых веток его
пред\-став\-ле\-ние превращается в единственный узел <<return NULL>>.
Доказательство следует из свойств алгоритма преобразования.

Кусочно-квазиаффинные множества $M_1$ и $M_2$ равны при всех значениях параметров
тогда и только тогда, когда $(M_1\smallsetminus
M_2)\cup(M_2\smallsetminus M_1)\hm=\varnothing$.

Кусочно-квазиаффинная функция $\Phi$ тождественно равна нулю на своей области
определения для всех значений параметров тогда и только тогда, когда
ККА-мно\-жест\-во $\{ I\in\mathbb{Z}^m \mid \Phi(I) \neq 0\}$ пустое.

Кусочно-квазиаффинные функции $\Phi_1$ и $\Phi_2$ равны при всех значениях па\-ра\-мет\-ров
тогда и только тогда, когда совпадают их области определения ${\cal D}(\Phi_1)\hm=
{\cal D}(\Phi_2)$ и $\Phi_1\hm-\Phi_2\hm\equiv0$.


\subsubsection{Поиск параметров, при~которых выполняется проверка 
на~пустоту и~равенство} %4.2.8

Для определения множества значений па\-ра\-мет\-ров, для которых
ККА-множество непусто, нужно вычислить его проекцию в расширенном
пространстве пе\-ре\-мен\-ных-па\-ра\-мет\-ров на подпространство па\-ра\-мет\-ров
(см.\ п.~4.2.6). В~результате получится древовидное пред\-став\-ле\-ние
множества значений параметров, для которых существует хотя бы одно
значение переменной, принадлежащее множеству.

Задача о поиске па\-ра\-мет\-ров, для которых совпадают два множества,
сводится к предыдущей, так как $M_1=M_2$  $\Leftrightarrow$ 
$(M_1 \smallsetminus M_2)\cup(M_2 \smallsetminus
M_1)=\varnothing$. При этом считается, что множество пустое для
тех па\-ра\-мет\-ров, для которых оно не определено. Если оба множества
пус\-ты, то они совпадают.

Для поиска параметров, при которых ККА-функ\-ция~$\Phi$ тождественно
равна нулю на своей области определения, нужно спроектировать
ККА-мно\-жест\-во $\{ I\in\mathbb{Z}^m \mid \Phi(I) \hm\neq 0\}$ на
подпространство па\-ра\-мет\-ров. При этом считается, что $\Phi\hm=0$ для тех
значений па\-ра\-мет\-ров, для которых ${\cal D}(\Phi)\hm=\varnothing$. Если
в контексте задачи это не так, то нужно пересечь результат со
множеством значений па\-ра\-мет\-ров, для которых ${\cal D}(\Phi)\hm\neq\varnothing$.

Задача о поиске параметров, для которых совпадают две функции,
сводится к предыдущей, так как $\Phi_1\hm=\Phi_2\Leftrightarrow
{\cal D}(\Phi_1)\hm={\cal D}(\Phi_2)$ и $\Phi_1\hm-\Phi_2\hm=0$.


\section{Заключение}

Разработанная библиотека операций пред\-остав\-ля\-ет унифицированный
интерфейс для исследования ККА-функ\-ций и ККА-мно\-жеств. 
В~ОРС~\cite{OPS} они применяются для уточнения зависимостей, при\linebreak генерации
кода на общую и распределенную память, отображении гнезд циклов на
многоконвейерную архитектуру, проверке применимости преобразований
циклов (перестановка, слияние,\linebreak разрезание). Идет работа над
реализацией алгоритма вычисления целых точек в произвольном
ККА-мно\-жест\-ве, зависящем от параметров, с использованием порождающих
функций Барвинка (см.~\cite{Verdoolaege3} и ссылки там). Это
позволит на этапе компиляции определять чис\-ло различных ячеек
памяти, к которым происходит обращение в данном операторе программы;
количество итераций в цикле; чис\-ло кеш-про\-ма\-хов; количество
процессорных элементов, необходимых для запуска цикла на ПЛИС; объем
динамически выделяемой памяти и~т.\,п.

\bigskip

В заключение автор приносит искреннюю благодарность Аркадию Валентиновичу Климову за
ряд критических замечаний и помощь в работе над статьей.

{\small\frenchspacing
{%\baselineskip=10.8pt
\addcontentsline{toc}{section}{Литература}
\begin{thebibliography}{99}


\bibitem{Voevodin} %1
\Au{Воеводин В.\,В., Воеводин Вл.\,В.}  Параллельные вы\-чис\-ле\-ния.~--- СПб.: БХВ-Петербург, 2002.

\bibitem{ShteinbBYa0} %2 
\Au{Штейнберг Б.\,Я.} Информационные зависимости и высокоуровневые распараллеливающие 
преобразования программ: Электронный учебник, 2007. 167~с. 
{\sf http://ops.rsu.ru/works.shtml}.

\bibitem{Feautrier2} %3
\Au{Feautrier P.} Parametric Integer Programming. --- Laboratoire MASI, Institut Blaise Pascal, 
Universite  de Versailles St-Quentin, 1988. P.~25.

\bibitem{Feautrier1} %4
\Au{Feautrier P.} Dataflow analysis of scalar and array references~// 
Int. J.~Parallel  Programming, 1991. Vol.~20. No.\,1. P.~23--52.


\bibitem{Feautrier3}  %5
The Parametric Integer Programming's Home. {\sf http:// www.PipLib.org}.



\bibitem{Shulj2} %6
\Au{Шульженко А.\,М.} 
Расщепление многомерных циклов для эффективного распараллеливания~// 
Параллельные вычисления в задачах математической физики: Труды Всеросс. науч.-технич. конф.~--- 
Ростов-на-Дону, 2004. С.~186--194.

\bibitem{Shulj1} %7
\Au{Шульженко А.\,М.} 
Автоматическое определение цик\-лов ParDo в программе~// Известия вузов. Северокавказский
    регион. Естественные науки. Приложение, 2011. Вып.~5. С.~77--88.

\bibitem{ShteinbBYa} \Au{Штейнберг Б.\,Я.} Оптимизация размещения данных в параллельной 
памяти.~--- Ростов-на-Дону: ЮФУ, 2010. 255 с.

\bibitem{Klimov} \Au{Klimov A.\,V.} 
Transforming affine nested loop programs to dataflow computation model~// 
PSI'11: Ershof Informatic Conference. --- Novosibirsk, 2011. P.~274--285.

\bibitem{Verdoolaege3}  %10
\Au{Seghir R., Verdoolaege S., Beyls~K., Loechner~V.} 
Analytical computation of Ehrhart
    polynomials and its application in compile-time generated cache hints. 
    Research Report of the Universite Louis Pasteur icps-2004-118, 2004.

\bibitem{ShteinbR} \Au{Штейнберг Р.\,Б.} 
Использование решетчатых графов для исследования многоконвейерной модели вы\-чис\-ле\-ний~// 
Известия вузов. Северокавказский регион. Естественные науки, 2009. Вып.~2. С. 16--18.

\bibitem{KritichPut} %12
\Au{Гуда С.\,А.} Оценки длины критического пути в решетчатом графе~// 
Параллельные вы\-чис\-ле\-ния и задачи управления: Труды IV Междунар. конф.~--- 
М., 2008. С.~1253--1267.

\bibitem{Pugh} %13
\Au{Pugh W.} The Omega test: A fast and practical integer programming 
algorithm for dependence analysis~// 1991 ACM/IEEE Conference on Supercomputing Proceedings.~-- 
New York, NY, USA: ACM, 1991.

\bibitem{Maslov} %14
\Au{Maslov~V.} 
Lazy array data-flow dependence analysis~// 21st Annual ACM SIGPLAN-SIGACT Symposium on 
Principles of Programming Languages Proceedings, 1994. P.~311--325.

\bibitem{PolyLib} PolyLib~--- a library of polyhedral functions. 
{\sf http://icps.\linebreak u-strasbg.fr/polylib}.

\bibitem{Graphite} Graphite: Gimple Represented as Polyhedra.
{\sf http://gcc.\linebreak gnu.org/wiki/Graphite}.

\bibitem{PPL} Parma Polyhedra Library.
{\sf http://bugseng.com/\linebreak products/ppl}.

\bibitem{ISL} Integer Set Library.
{\sf http://freecode.com/projects/isl};
{\sf http://www.kotnet.org/$\sim$skimo/isl}.

%\bibitem{Shreiver} \Au{Схрейвер А.}  Теория линейного и целочисленного программирования~/ 
%Пер. с англ.~--- М.: Мир, 1991. 342~с.

\bibitem{OpsProg} \Au{Штейнберг Б.\,Я., Нис З.\,Я., Петренко~В.\,В., 
Черданцев~Д.\,Н., Штейнберг~Р.\,Б., Шульженко~А.\,М.} 
Состояние и возможности Открытой распараллеливающей системы (лето 2006~г.)~// 
Перспективы систем информатики: Труды VI Междунар. конф., рабочего семинара 
<<Наукоемкое программное обеспечение>>.~--- Новосибирск, 2006. С.~122--125.

\bibitem{OPS} Открытая распараллеливающая система.
{\sf http:// ops.rsu.ru}.

\bibitem{Dvor0} \Au{Штейнберг~Б.\,Я., Алымова~Е.\,В., Баглий~А.\,П.\ и~др.} 
Особенности реализации распараллеливающих преобразований программ в ДВОР~// Параллельные 
вы\-чис\-ле\-ния и задачи управления: Труды Междунар. конф. --- М.: ИПУ РАН, 2010. С.~787--854.

\bibitem{Dvor1} \Au{Штейнберг~Б.\,Я., Абрамов А.\,А., Баглий~А.\,П. и~др.} 
Уточнение зависимостей программы в ДВОР~// Параллельные вычисления и задачи управления: Труды 
Междунар. конф. --- М.: ИПУ РАН, 2010. С.~855--864.

\bibitem{Dvor} \Au{Штейнберг Б.\,Я., Абрамов А.\,А., Алымова~Е.\,В. и~др.} 
Диалоговый высокоуровневый оптимизирующий распараллеливатель (ДВОР)~// 
Научный сервис в сети Интернет: суперкомпьютерные центры и задачи: Труды Междунар. 
суперкомпьютерной конф.~--- Новоросcийск, 2010.~--- М.: МГУ, 2010. C.~71--75.

\label{end\stat}

\bibitem{ZPolylib} \Au{Nookala S., Risset~T.} 
A~Library for Z-polyhedral operations: Technical report PI-1330.~--- Rennes, France: IRISA, 2000.
\end{thebibliography}
}
}

\end{multicols}