\def\stat{chertok}

\def\tit{ВЕРОЯТНОСТНО-СТАТИСТИЧЕСКОЕ МОДЕЛИРОВАНИЕ ИНФОРМАЦИОННЫХ ПОТОКОВ
В~СЛОЖНЫХ ФИНАНСОВЫХ СИСТЕМАХ НА ОСНОВЕ ВЫСОКОЧАСТОТНЫХ
ДАННЫХ$^*$}

\def\titkol{Вероятностно-статистическое моделирование информационных потоков
в сложных финансовых системах} % на основе высокочастотных данных}

\def\autkol{В.\,Ю.~Королев, А.\,В.~Черток, А.\,Ю.~Корчагин, А.\,К.~Горшенин}

\def\aut{В.\,Ю.~Королев$^1$, А.\,В.~Черток$^2$, А.\,Ю.~Корчагин$^3$, А.\,К.~Горшенин$^4$}

\titel{\tit}{\aut}{\autkol}{\titkol}

{\renewcommand{\thefootnote}{\fnsymbol{footnote}}\footnotetext[1]
{Работа поддержана Российским фондом фундаментальных
исследований (проекты 11-01-00515а, 11-07-00112а, 11-01-12026-офи-м,
12-07-00115а).}}

\renewcommand{\thefootnote}{\arabic{footnote}}
\footnotetext[1]{Факультет вычислительной
математики и кибернетики Московского государственного университета
им.\ М.\,В.~Ломоносова; Институт проблем информатики Российской
академии наук, vkorolev@cs.msu.su}
\footnotetext[2]{Факультет
вычислительной математики и кибернетики Московского государственного
университета им.\ М.\,В.~Ломоносова; Euphoria Group LLC,
a.v.chertok@gmail.com}
\footnotetext[3]{Факультет вычислительной математики и кибернетики Московского государственного
университета им.\ М.\,В.~Ломоносова, sasha.korchagin@gmail.com}
\footnotetext[4]{Институт проблем информатики Российской академии наук, a.k.gorshenin@gmail.com}


\Abst{Предложена микроструктурная модель, описывающая
информационные потоки в сложных финансовых системах и случайную
природу интенсивностей потоков заявок, определяющих механизм
ценообразования финансовых инструментов. При их моделировании поток
внешнего информационного фона со случайной интенсивностью
рассматривается и аппроксимируется отдельно в рамках предложенной и
статистически обоснованной мультипликативной модели. Эта модель
позволяет анализировать характеристики, связанные с интенсивностями
потоков заявок, а также мгновенное соотношение сил покупателей и
продавцов без моделирования внешнего информационного фона,
практически не поддающегося прогнозированию. Также предложена модель
обобщенного процесса цены, учитывающая всю доступную информацию о
потоках заявок и допускающая дальнейшую аналитическую интерпретацию.}

\KW{финансовые рынки; информационные потоки;
ценообразование; интенсивности потоков заявок; книга заявок; смесь
распределений; обобщенная цена}

\vskip 14pt plus 9pt minus 6pt

      \thispagestyle{headings}

      \begin{multicols}{2}

            \label{st\stat}

\section{Введение}

Финансовый рынок является открытой информационной системой с очень
сложной структурой. В~последнее десятилетие с развитием электронной
торговли на финансовых рынках изучение биржевых высокочастотных
данных стало ключевым для более глубокого понимания закономерностей
динамики подобных сложных систем и, в частности, для описания
механизмов формирования цены.

Как известно, статистические распределения приращений (логарифмов)
финансовых индексов и, в частности, биржевых цен имеют более тяжелые
хвосты, чем нормальное (гауссово) распределение. В~работе~\cite{Korolev2011-1} 
этот феномен объяснен с помощью предельных теорем
для обобщенных дважды стохастических пуассоновских процессов
(обобщенных процессов Кокса). В~соответствии с подходом,
использованным в~\cite{Korolev2011-1}, указанные распределения должны
иметь вид смесей нормальных законов. Обоснованием адекватности таких
моделей предложено считать стохастический характер интенсивностей
хаотических информационных потоков в сложных информационных
финансовых сис\-темах.

Использование высокочастотных статистических данных, ставших
доступными благодаря широ\-ко\-му распространению систем электронной
торгов\-ли на бирже, позволяет верифицировать указанные выше модели и
более детально описать процесс ценообразования.

Финансовые рынки представляют собой пример сложных открытых
стохастических информационных систем, в которых можно выделить два
основных источника случайности: внутренний и внешний. Внутренний
источник случайности порождает неопределенность, обусловленную
различием стратегий очень большого числа участников рынка.
<<Физическим>> аналогом такой слу\-чай\-ности может служить хаотическое
тепловое движение час\-тиц в замкнутых сис\-те\-мах. Внешний источник
случайности~--- это плохо поддающийся более или менее полному
прогнозированию поток новостей политического и экономического
характера (в том числе потоки информации с внешних рынков и
инструментов), в соответствии с которыми изменяются интересы и
стратегии участников рынка. Эти два источника случайности будут
учитываться при описании модели ценообразования.


\section{Описание модели}

На классических электронных рынках, которые в данный момент
принадлежат к числу наиболее активно развивающихся типов рынка,
биржевая цена финансового инструмента в ее классическом понимании
является результирующей, интегральной характеристикой системы
торгов, которая описывается динамикой так называемой {\it книги
заявок} (limit order book), представляющей собой список всех
актуальных на данный момент предложений о покупке и продаже
инструмента по различным ценам. Динамику книги заявок на электронном
рынке определяют три типа заявок:
\begin{enumerate}[(1)]
\item {\it лимитная заявка} обозначает намерение купить или продать
фиксированный объем инструмента по определенной цене (купить по цене
не выше заданной или продать по цене не ниже заданной), заявка немедленно
добавляется в книгу заявок;
\item {\it рыночная заявка} обозначает намерение купить или продать
фиксированный объем инструмента по лучшей предложенной цене, после
чего немедленно происходит ее сведение с одной из лимитных заявок
(при их наличии);
\item {\it заявка на отмену} обозначает намерение отменить существующий
лимитный ордер, после чего он удаляется из книги заявок.
\end{enumerate}

Разумеется, лимитная заявка может оказаться рыночной, если
заявленная в ней цена позволяет немедленно произвести сведение с
одной из лимитных заявок на противоположной стороне книги заявок.
Участники рынка, присылающие лимитные ордера, являются {\it
поставщиками ликвидности} (liquidity providers), а те, кто присылают
рыночные заявки,~--- {\it потребителями ликвидности} (liquidity
takers).

Итак, рассмотрим прежде динамику книги заявок на дискретной сетке цен $\Pi \hm= \{
1, 2, \ldots, M \}$ как процесс с непрерывным временем
\begin{multline*}
\mathbf{x}(t) \equiv \left(\mathbf{V^a}(t); \mathbf{V^b}(t)\right) \equiv{}\\
\hspace*{-3pt}{}\equiv
\left(V^a_1(t), V^a_2(t), \ldots , V^a_M(t); V^b_1(t), V^b_2(t), \ldots ,
V^b_M(t)\right)\!,\hspace*{-8.22688pt}
\end{multline*}
где $V^a_p(t)$ ($V^b_p(t))$ обозначает количество лимитных заявок на
продажу (покупку) с ценой $p \hm\in \Pi$. Так как в один момент не
может существовать заявок на покупку и продажу по одной цене (иначе
они будут сведены), необходимо потребовать $V^a_p(t) \vee V^b_p(t)\hm =
0$ для всех~$p$ и~$t$.

Лучшая цена на продажу~$a(t)$ определяется как
$$
a(t) = \inf \{p: V^a_p(t) > 0\} \wedge (M + 1)\,,
$$
лучшая цена на покупку $b(t)$ определяется как
$$
b(t) = \sup \{p: V^b_p(t) > 0\} \vee 0\,.
$$
При этом процесс цены можно, например, определить как
$$
P(t) = \fr{a(t) + b(t)}{2}\,.
$$
Таким образом, процесс цены $P(t)$ является результатом процесса
эволюции книги заявок, инициированного потоком заявок трех типов.

Предположим вначале, что поток информации,\linebreak поступающей извне,
фиксирован. Тогда при фиксированной информации можно считать, что
внут\-рен\-няя случайность является установившимся\linebreak хаосом. Как показано,
например, в~\cite{Korolev2011-1, KorolevBeningShorgin2011-1},
естественными математическими моделями хаотических потоков являются
пуассоновские процессы, характеризуемые тем, что интервалы времени
между информативными событиями являются независимыми одинаково
распределенными случайными величинами с экспоненциальным
распределением. Поэтому на первом этапе при построении
рас\-смат\-ри\-ва\-емой модели потоки заявок моделируются с использованием
независимых процессов с экспоненциальными распределениями (как это
сделано, например, в работах~\cite{ContRamaStoikov2010b, ContLarrard2011}):
\begin{itemize}
\item лимитные заявки на покупку (продажу) приходят на ценовой
уровень, расположенный на расстоянии~$i$ от лучшей котировки
противоположного типа, в независимые моменты времени, имеющие
экспоненциальное распределение с па\-ра\-мет\-ром $\lambda_i^{+} (\lambda_i^{-})$ 
(эмпирические исследования~\cite{Bouchaud2002, ZovkoFarmer2002} 
показывают, что степенный закон
$\lambda_i^{\pm} = {k}/{i^\alpha}$
является хорошей аппроксимацией);
\item рыночные заявки на покупку (продажу) приходят в независимые
моменты времени, име\-ющие экспоненциальное распределение с параметром $\mu^{+} (\mu^{-})$;
\item заявки на отмену лимитного ордера на покупку (на продажу),
находящегося на дистанции~$i$ от лучшей котировки того же типа,
приходят с частотой $\theta_i^{+} (\theta_i^{-})$.
\end{itemize}

В данной статье для простоты изложения рассматриваются потоки заявок
единичного объема, однако все рассуждения могут быть распространены
на более общий случай (см.\ также~\cite{Huang2012}). Таким образом,
$\mathbf{x}(t)$ является цепью Маркова с непрерывным временем в
пространстве состояний $(\mathbb{Z}^{+})^{2M}$ и следующими
переходами:

\noindent
\begin{align*}
V^a_i(t) \to V^a_i(t) + 1 &\ \mbox{ с интенсивностью } \lambda^{-}_{i - b(t)} & \\
&\hspace*{14mm}\mbox{ для } i > b(t)\,; &\\
V^a_i(t) \to V^a_i(t) - 1 &\ \mbox{ с интенсивностью } \theta^{-}_{i - a(t)} & \\
&\hspace*{14mm}\mbox{ для } i \ge a(t)\,; &\\
V^a_i(t) \to V^a_i(t) - 1 &\ \mbox{ с интенсивностью } \mu^{+} &\\
& \hspace*{14mm}\mbox{ для } i = a(t) > 0\,. &\\
V^b_i(t) \to V^b_i(t) + 1 &\ \mbox{ с интенсивностью } \lambda^{+}_{a(t) - i} &\\
& \hspace*{14mm}\mbox{ для } i < a(t)\,; &\\
V^b_i(t) \to V^b_i(t) - 1 &\ \mbox{ с интенсивностью } \theta^{+}_{b(t) - i} &\\
& \hspace*{14mm}\mbox{ для } i \le b(t)\,; &\\
V^b_i(t) \to V^b_i(t) - 1 &\ \mbox{ с интенсивностью } \mu^{-} & \\
&
\hspace*{14.5mm}\mbox{для } i = b(t) < M + 1\,. &
\end{align*}
С учетом вышесказанного можно определить следующие независимые
пуассоновские процессы:
\begin{itemize}
\item $L^{\pm}_{i}(t):$ потоки лимитных ордеров с интенсивностями $\lambda_{i}^{\pm}$;
\item $M^{\pm}(t):$ потоки рыночных ордеров с интенсивностями
$\mu^{+}\mathbb{I}(\mathbf{V^a} \hm\ne 0)$ и $\mu^{-}\mathbb{I}(\mathbf{V^b} \hm\ne 0)$;
\item $C^{\pm}_{i}(t):$ потоки заявок на отмену лимитных ордеров с интенсивностями 
$\theta_{i}^{\pm}$;
\item пуассоновский процесс
\begin{multline*}
N(t) = M^{+}(t) + M^{-}(t) + \sum\limits_{i = 1}^{M} \left(L_i^{+}(t) +
L_i^{-}(t)\right) +{}
\\
{}+ \sum\limits_{i = 1}^{M} \left(C_i^{+}(t) + C_i^{-}(t)\right)\,,
\end{multline*}
описывающий поток всех заявок, поступающих на рынок.
\end{itemize}

Процессы $L^{\pm}_{i}(t), M^{\pm}(t), C^{\pm}_{i}(t)$ полностью
определяют процесс цены $P(t)$ и для него, вообще говоря, могут быть
выписаны соответствующие стохастические дифференциальные 
уравнения~(см.~\cite{AbergelJedidi2011}), %\linebreak 
однако дальнейшая его аналитическая
интерпретация представляется очень сложной или вообще невозможной
даже при довольно сильных допущениях о постоянных и независимых
интенсивностях потоков заявок разных типов, что никак не
соответствует действительности.

\section{Обобщенный процесс цены}

Справедливо рассмотреть обобщенный процесс цены, в котором учтены не
только изменения лучших котировок, но и по\-ста\-нов\-ка/сня\-тие заявок в
глубине книги заявок, поскольку каждое такое действие оказывает
влияние на текущее распределение сил покупателей и продавцов.

Напомним, что внешний информационный фон пока предполагается
неизменным.

Зафиксируем прежде достаточно небольшой интервал времени $[0, t]$,
позволяющий считать, что на таком интервале интенсивности описанных
событий постоянны. Пусть по-преж\-не\-му $N(t)$~--- пуассоновский
процесс, соответствующий всем событиям в книге заявок и имеющий
интенсивность
$$
\lambda = \mu^{+} + \mu^{-} + \sum\limits_{i = 1}^{M} (\lambda_i^{+} +
\lambda_i^{-}) + \sum\limits_{i = 1}^{M} (\theta_i^{+} + \theta_i^{-})\,.
$$
Расщепим его на два пуассоновских процесса $N_+(t)$ и $N_-(t)$ с
интенсивностями соответственно:
$$
\lambda_+ = \mu^{+} + \sum\limits_{i=1}^M \lambda_i^{+} + \sum\limits_{i=1}^M
\theta_i^{-}\,;
$$
$$
\lambda_- = \mu^{-} + \sum\limits_{i=1}^M \lambda_i^{-} + \sum\limits_{i=1}^M
\theta_i^{+}\,.
$$
Таким образом, $\lambda = \lambda_+ \hm+ \lambda_-$, а процессы
$N_+(t)$ и $N_-(t)$ характеризуют накопленную силу покупателей и
продавцов соответственно (при этом заметим, что снятие заявок на
стороне продавцов в данном случае увеличивает силу покупателей и
наоборот) и являются условно независимыми при фиксированном потоке
информации, поступающем извне за время $[0, t]$.

Теперь рассмотрим процесс обобщенной цены $Q(t)$, приращение
которого на интервале $[0, t]$ имеет вид:
$$
Q(t) - Q(0) = \sum\limits_{j=1}^{N(t)} X_j\,,
$$
где $X_1, X_2, \ldots$~--- независимые одинаково распределенные
величины, такие что
\begin{gather*}
X_j=\begin{cases}
+1 & \mbox{с вероятностью } {\displaystyle
\fr{\lambda_+}{\lambda_+ + \lambda_-}} \,; \\
-1 & \mbox{с вероятностью } {\displaystyle \fr{\lambda_-}{\lambda_+
+\lambda_-}}\,, 
\end{cases}\\
\hspace*{41mm}j =1, 2, \ldots\,,
\end{gather*}
причем случайные величины $X_1,X_2,\ldots$ стохастически независимы
от процесса $N(t)$ (в этом можно убедиться, непосредственно выписав
характеристическую функцию случайной величины~$N(t)$).
При этом
\begin{align*}
{\sf E}X_j&=\fr{\lambda_ + -\lambda_-}{\lambda_+ + \lambda_-}\,;\\
{\sf D}X_j&=1-\left(\fr{\lambda_+-\lambda_-}{\lambda_++\lambda_-}\right)^2=
\fr{4\lambda_+\lambda_-}{\left(\lambda_++\lambda_-\right)^2}\,,
\end{align*}
так что
\begin{align*}
{\sf E}\sum\limits_{j=1}^{N(t)}X_j&=t\left(\lambda_+-\lambda_-\right)\,;
\\
{\sf D}\sum\limits_{j=1}^{N(t)}X_j&=t\left(\lambda_++\lambda_-\right)
\left[\left(\fr{\lambda_+-\lambda_-}{\lambda_++\lambda_-}\right)^2+{}\right.\\
&\hspace*{16mm}\left.{}+
\fr{4\lambda_+\lambda_-}{\left(\lambda_++\lambda_-\right)^2}\right]=
t\left(\lambda_++\lambda_-\right)\,.
\end{align*}

В дальнейшем без ограничения общности будем считать, что $Q(0) \hm= 0$.
Для удобства временно будем считать, что $t\hm=1$ (т.\,е.\
рассматривается приращение обобщенной цены за единицу времени).

Если $\lambda=\lambda_++\lambda_-$ очень велико, т.\,е.\ в единицу
времени происходит очень много информативных событий, то по
центральной предельной теореме для пуассоновских случайных сумм
справедливо приближенное соотношение:
\begin{equation}
{\sf P} \left( Q(1) < x \right) \approx \Phi \left( \fr{x- \lambda_+
+ \lambda_-}{\sqrt{\lambda_+ + \lambda_-}} \right)\,,\enskip
x\in\mathbb{R}\,,\label{e1-ch}
\end{equation}
где $\Phi(x)$~--- стандартная нормальная функция распределения. При
этом, используя результаты работы~\cite{KorolevShevtsova2012}, можно
выписать довольно аккуратные оценки точности приближения~(\ref{e1-ch}).

Теперь вспомним, что выше внешний поток информации считался
фиксированным. Это предположение, в частности, широко используется в
большинстве работ по моделированию динамики книги заявок и дает
возможность использовать аппарат марковских цепей с непрерывным
временем, для которых условие марковости в определенном смысле
эквивалентно тому, что распределение вероятностей интервалов времени
между информативными событиями является экспоненциальным. В~реальной
практике это условие не выполняется, как видно из рис.~1. 
На этом рисунке приведены гистограмма
интервалов времени между событиями, произошедшими в течение всего
рабочего дня, и график плот\-ности гам\-ма-рас\-пре\-де\-ле\-ния с па\-ра\-мет\-ром
формы~0,2637 и па\-ра\-мет\-ром мас\-шта\-ба~1,2421. Это распределение хорошо
согласуется с гистограммой и заметно отличается от
экспоненциального.



С другой стороны, хорошее согласие распределения вероятностей
интервалов времени между событиями, заметное на рис.~1, 
с указанным выше гам\-ма-рас\-пре\-де\-ле\-ни\-ем
подтверждает правильность рассуждений об {\it условной} марковости
рассматриваемых процессов, поскольку, как известно, получение
безусловного распределения из условного сводится к смешиванию
условного распределения\linebreak\vspace*{-12pt}

\begin{center}  %fig1


\vspace*{1pt}

\mbox{%
 \epsfxsize=72.282mm
 \epsfbox{che-1.eps}
 }
 \end{center}
% \vspace*{6pt}
{{\figurename~1}\ \ \small{Гистограмма и распределение вероятностей интервалов времени 
  между информативными событиями: \textit{1}~--- данные; \textit{2}~--- 
  гамма-распределение (0,2637, 1,2421)}}



%\pagebreak

\vspace*{12pt}

\addtocounter{figure}{1}


\noindent
 по распределению вероятностей,
соответствующему закону распределения параметра, описывающего
фиксированное условие. В~то же время гам\-ма-рас\-пре\-де\-ле\-ние может быть
представлено в виде смеси экспоненциальных распределений, только
если его параметр формы не превосходит единицы (см.~\cite{Gleser1987}). 
В~той же работе показано, что если па\-ра\-метр
формы~$r$ гам\-ма-рас\-пре\-де\-ле\-ния, соответствующего плот\-ности
$g(x;r,\mu)$, удовлетворяет условию $0\hm<r\hm\leq 1$, то плот\-ность
$g(x;r,\mu)$ может быть представлена в виде:
$$
g(x;r,\mu)=\int\limits_{0}^{\infty}p_\mu(z) z e^{-zx}\,dz\,,
$$
где $p_\mu(z)$~--- плотность распределения Сне\-де\-ко\-ра--Фи\-ше\-ра:
$$
p_\mu(z) =\fr{(z-\mu)^{-r}\mu^r}{z\Gamma(1-r)\Gamma(r)} \mathbb{I}\left(\mu\leq
z\right)\,.
$$
В~работе~\cite{GorsheninDoynikovKorolevKuzmin2012} можно найти также
результаты статистического анализа эволюции па\-ра\-мет\-ра~$\mu$ 
модели в течение дня:
\begin{multline}
{\sf
P}\left(Q(1)<x\right)\approx{}\\
{}\approx\!\!\!\!\!\int\limits_{\mathbb{R^+}\times\mathbb{R^+}}\!\!
\Phi\left(\fr{x-\lambda_+ +\lambda_-}{\sqrt{\lambda_++\lambda_-}}\right)d
{\sf P}\left(\Lambda_+<\lambda_+,\,\Lambda_-<\lambda_-\right)\,,\\
x\in\mathbb{R}\,.
\label{e2-ch}
\end{multline}

Тем не менее, рабочий день, за который накапливались исходные данные
для рис.~1,~--- это слишком большой интервал
времени, так что практическая ценность этой модели сродни ценности
информации о <<средней температуре по больнице>>.

Поэтому для получения более тонких моделей {\it безусловного}
распределения величины $Q(1)$, в силу непредсказуемости потока
внешней информации, следует считать, что $\lambda_+$ и $\lambda_-$~--- 
это некоторые конкретные значения {\it случайных величин}
$\Lambda_+$ и $\Lambda_-$. Таким образом, для безусловного
распределения приращения $Q(1)$ <<форсированно>> получается модель~(\ref{e2-ch}).
Эту модель можно статистически исследовать методом скользящего
разделения смесей (СРС-ме\-то\-дом), используя конечные аппроксимации
для смеси~(\ref{e2-ch})
\begin{equation}
{\sf
P}\left(Q(1)<x\right)\approx\sum\limits_{j=1}^kp_j\Phi\left(\fr{x-a_j}{\sigma_j}\right)
\label{e3-ch}
\end{equation}
и оценивая параметры
$k,\,p_1,\ldots,p_k,\,a_1,\ldots,a_k$, $\sigma_1,\ldots,\sigma_k$
модели~(\ref{e3-ch}) (см., например, книгу~\cite{Korolev2011-1}, где
довольно подробно изложен и сам СРС-ме\-тод, и его применение к
декомпозиции волатильности). В~этом заключается первый,
<<квазинепараметрический>> способ анализа. Этот способ {\it
непараметрический}, потому что методически аналогичен
непараметрическим процедурам ядерного оценивания распределений. Он
{\it квази}непараметрический, потому что выбор нормальных ядер здесь
форсирован и обусловлен центральной предельной теоремой для
пуассоновских случайных сумм. Но как любой непараметрический метод,
этот метод плох тем, что годится только для {\it ретроспективного}
анализа. Для {\it перспективного} анализа (например,
прогнозирования) намного удобнее параметрические модели, к
построению которых и перейдем.

Предположим, что
$$
\Lambda_+=\Lambda \alpha_+\,;\enskip
\Lambda_-=\Lambda\alpha_-\,,
$$
где $\Lambda$~--- неотрицательная случайная величина, имеющая смысл
внешнего новостного фона на бирже, а $\alpha_+$ и~$\alpha_-$~---
параметры, описывающие тенденции торгов (пока для простоты изложения
будем считать параметры $\alpha_+$ и~$\alpha_-$ неслучайными). Тогда
модель~(\ref{e2-ch}) примет вид
\begin{multline}
{\sf P}\left(Q(1)<x\right)\approx{}\\
\!{}\approx\!\int\limits_{0}^{\infty}\!
\Phi\left(\fr{x-\lambda \left(\alpha_+-\alpha_-\right)}{\sqrt{\lambda
\left(\alpha_++\alpha_-\right)}}\right)d{\sf P}\left(\Lambda<\lambda\right)\,,\enskip
x\in\mathbb{R}.\!\label{e4-ch}
\end{multline}
Заметим, что модель~(\ref{e4-ch})~--- это хорошо известная {\it
дис\-пер\-си\-он\-но-сдви\-го\-вая} смесь (variance-mean mixture) нормальных
законов, в которой смешивание производится как бы и по параметру
сдвига, и по параметру масштаба, но фактически смесь является
однопараметрической. К~такому типу смесей относятся, в частности,
обобщенные гиперболические законы, включая дисперсионные
гам\-ма-рас\-пре\-де\-ле\-ния (variance gamma distributions), скошенные
распределения Стьюдента, нормальные$\backslash\!\backslash$обратные
гауссовские распределения, некоторые устойчивые законы, а также
многие другие. Методы исследования и использования таких моделей
хорошо известны.

Задача исследования смесей типа~(\ref{e4-ch}) также может быть решена
непараметрическими методами. Важность этих методов подчеркивается
нестационарным характером потока новостей и, стало быть,
зависимостью параметров модели~(\ref{e4-ch}) от астрономического времени (т.\,е.\ 
от фактического положения окна). По аналогии с моделью~(\ref{e3-ch}) можно
предложить квазинепараметрический подход к {\it приближенному}
оцениванию параметров модели~(\ref{e4-ch}).

Чтобы модель~(\ref{e4-ch}) можно было использовать для перспективного анализа,
ее нужно настроить, т.\,е.\ определить распределение случайной
величины~$\Lambda$. С~этой целью можно применить уже опробованный и
продемонстрировавший хорошие результаты метод аппроксимации к
распределению {\it длительностей} интервалов между событиями потока
изменения цен модели типа конечной смеси гам\-ма-рас\-пре\-де\-ле\-ния. А~именно: 
если $T$~--- случайная величина, равная длине интервала
времени между изменениями цены, то ее распределение хорошо
согласуется с моделью типа
\begin{equation}
\fr{d}{dx}{\sf P}(T<x)\approx\sum\limits_{j=1}^kp_j
g(x;\,\theta_j,\mu_j)\,,\enskip x\ge0\,,
\label{e4prime-ch}
\end{equation}
где $p_j\hm\ge0$, $j=1,\ldots,k$, $p_1+\cdots+p_k\hm=1$;
$g(x;\,\theta,\mu)$~--- плот\-ность гам\-ма-рас\-пре\-де\-ле\-ния с па\-ра\-мет\-ром
формы $\theta\hm>0$ и па\-ра\-мет\-ром мас\-шта\-ба $\mu\hm>0$:
$$
g(x;\,\theta,\mu)=\fr{\mu^\theta}{\Gamma(\theta)}x^{\theta-1}e^{-\mu
x}\,,\enskip x\ge0\,;
$$
$\Gamma(\theta)$~--- эйлерова гам\-ма-функ\-ция. Если $T$~--- длительность
интервала времени между событиями потока, то интенсивность потока~---
величина, обратно пропорциональная средней длительности. Модель~(\ref{e4prime-ch})
допускает следующую интерпретацию. Имеется $k$ потоков событий,
каждый из которых соответствует своей компоненте в модели~(\ref{e4prime-ch}).
Всякий раз случайно в соответствии с вероятностями $p_j$ выбирается
один из потоков и реализуется случайная\linebreak
 величина, распределение
которой является соответствующей компонентой модели~(\ref{e4prime-ch}). Как
известно, среднее значение величины, име\-ющей гам\-ма-плот\-ность
$g(x;\,\theta,\mu)$, равно $\theta/\mu$. Следовательно,\linebreak
интенсивность соответствующего потока равна $\mu/\theta$. Таким
образом, в рамках модели~(\ref{e4prime-ch}) распределение случайной интенсивности~$\Lambda$ 
(см.\ модель~(\ref{e3-ch})) имеет вид:
$$
{\sf P}\left(\Lambda=\fr{\mu_j}{\theta_j}\right)= p_j\,,\enskip
j=1,\ldots,k\,.
$$

\end{multicols}

\begin{figure} %fig2
\vspace*{1pt}
 \begin{center}
 \mbox{%
 \epsfxsize=111.516mm
 \epsfbox{che-2.eps}
 }
 \end{center}
 \vspace*{-6pt}
\Caption{Эволюция весов и параметров формы компонент
модели~(\ref{e4prime-ch})}
\label{fig:1}
%\end{figure}
%\begin{figure*} %fig3
\vspace*{24pt}
 \begin{center}
 \mbox{%
 \epsfxsize=108.651mm
 \epsfbox{che-3.eps}
 }
 \end{center}
 \vspace*{-6pt}
\Caption{Эволюция дискретного распределения
интенсивности~$\Lambda$} 
\label{fig:2}
\vspace*{24pt}
\end{figure}

\begin{multicols}{2}

\noindent
Это распределение (и его эволюцию во времени при скольжении окна)
несложно построить, имея статистически оцененные параметры модели~(\ref{e4prime-ch}). 
Примеры применения такого метода приближенного восстановления
распределения случайной величины $\Lambda$ содержатся на рис.~\ref{fig:1} и~\ref{fig:2}.



\section{Исследование нестационарности распределения случайной интенсивности}

В данном разделе модель, принципиальное устройство которой описано
выше, будет адаптирована с учетом нестационарного и стохастического
характера интенсивности внешнего информационного потока~$\Lambda$ и
параметров $\alpha_+$ и $\alpha_-$, описывающих степень реакции на
него покупателей и продавцов.

В действительности интенсивности потоков \mbox{заявок} являются
нестационарными, поскольку внешний поток информации, определяющий
интенсивность этих событий, сам по себе является нестационарным. 
В~таком случае интенсивности потоков заявок разных типов, во-пер\-вых,
не могут являться независимыми, а во-вто\-рых, определенным образом
зависят от некоторого случайного процесса~$\Lambda(t)$,
определяющего внешний новостной фон. Поэтому можно определить
следующие соотношения для мгновенных интенсивностей:
$$
\mu^{\pm} = \mu^{\pm}(t) = \alpha^{M^{\pm}}(t) \lambda(t) \mbox{
(рыночные заявки)}\,;
$$
$$
\lambda_i^{\pm} = \lambda_i^{\pm}(t) = \alpha_i^{L^{\pm}}(t)
\lambda(t) \mbox{ (лимитные заявки)}\,;
$$
$$
\theta_i^{\pm} = \theta_i^{\pm}(t) = \alpha_i^{C^{\pm}}(t)
\lambda(t) \mbox{ (заявки на отмену)}\,,
$$
где $\lambda(t)$~--- мгновенная интенсивность случайного процесса
$\Lambda(t)$, определяющего внешний информационный фон (ажиотаж),
$\alpha^{M^{\pm}}(t)$, $\alpha_i^{L^{\pm}}(t)$,
$\alpha_i^{C^{\pm}}(t)$~--- также случайные процессы, характеризующие\linebreak
степень реакции на информационный фон при выставлении заявок данного
типа.

В таком случае $N_+(t)$ и $N_-(t)$ являются неоднородными
пуассоновскими процессами с мгновенными интенсивностями
\begin{equation}
\lambda_+(t) = \alpha_+(t) \lambda(t)\,; \label{e5-ch}
\end{equation}
\begin{equation}
\lambda_-(t) = \alpha_-(t) \lambda(t)\,, \label{e6-ch}
\end{equation}
где
$$
\alpha_+(t) = \alpha^{M^+}(t) + \sum\limits_{i=1}^M \alpha_i^{L^+}(t) +
\sum\limits_{i=1}^M \alpha_i^{C^-}(t)\,;
$$
$$
\alpha_-(t) = \alpha^{M^-}(t) + \sum\limits_{i=1}^M \alpha_i^{L^-}(t) +
\sum\limits_{i=1}^M \alpha_i^{C^+}(t)\,.
$$
При этом по-прежнему обобщенный процесс цены имеет вид:
$$
Q(t) = \sum\limits_{i=1}^{N(t)} X_i\,,
$$
где $X_i$~--- независимые случайные величины, заданные следующим
образом:
\begin{multline}
{\sf P} \left( X_i = 1 \right) = \fr{\lambda_+(T_i)}{\lambda_+(T_i)
+ \lambda_-(T_i)} ={}\\
{}= \fr{\alpha_+(T_i) \lambda(T_i)}{\alpha_+(T_i)
\lambda(T_i) + \alpha_-(T_i) \lambda(T_i)} = {}\\
{}=\fr{\alpha_+(T_i)
}{\alpha_+(T_i) + \alpha_-(T_i)}\,; \label{e7-ch}
\end{multline}

\vspace*{-12pt}

\noindent
\begin{multline}
{\sf P} \left( X_i = -1 \right) = \fr{\lambda_-(T_i)}{\lambda_+(T_i)
+ \lambda_-(T_i)} = {}\\
{}=\fr{\alpha_-(T_i) \lambda(T_i)}{\alpha_+(T_i)
\lambda(T_i) + \alpha_-(T_i) \lambda(T_i)} ={}\\
{}= \fr{\alpha_-(T_i)
}{\alpha_+(T_i) + \alpha_-(T_i)}\,; \label{e8-ch}
\end{multline}
$T_i$~--- последовательные моменты скачков процесса $N(t)$.

Таким образом, статистические свойства процесса $Q(t)$ определяются
свойствами процессов $\alpha_+(t)$, $\alpha_-(t)$ и $\Lambda(t)$
(характеризующего моменты $T_i$ скачков процесса $N(t)$).

Также следует отметить, что мультипликативные представления
мгновенных интенсивностей~(\ref{e5-ch}) и~(\ref{e6-ch}) хорошо согласуются с реальными
данными и дают возможность написать очень важное соотношение:
$$
r(t) = \fr{\lambda_+(t)}{\lambda_-(t)} = \fr{\alpha_+(t)}{\alpha_-(t)}\,,
$$
что избавляет в дальнейшем от необходимости моделирования внешнего
информационного потока $\Lambda(t)$, которое практически
нереализуемо с до\-ста\-точ\-ной степенью адекватности в силу его
непред\-ска\-зу\-емости и при этом дает возможность на основе наблюдаемых
значений процесса $r(t)\hm=\lambda_+(t)/\lambda_-(t)$ исследовать
процесс относительной реакции покупателей и продавцов на новостной
фон $\alpha_+(t)/\alpha_-(t)$, являющийся не чем иным, как мерой их
дисбаланса~--- основным механизмом ценообразования.

\section{Анализ реальных данных}

Для тестирования некоторых из вышеизложенных концепций были выбраны
высокочастотные данные для самого ликвидного инструмента фьючерсного
рынка биржи ММВБ-РТС~--- фьючерса на индекс РТС. Биржа распространяет
информацию о полном потоке обезличенных заявок участников рынка,
что, в частности, позволяет провести анализ процессов $N_+(t)$ и
$N_-(t)$ в рамках описанной модели обобщенного процесса цены.

\end{multicols}

\begin{figure} %fig4
\vspace*{1pt}
 \begin{center}
 \mbox{%
 \epsfxsize=114.955mm
 \epsfbox{che-4.eps}
 }
 \end{center}
 \vspace*{-9pt}
\Caption{График цены}
 \label{fig:price}
%\end{figure*}
%\begin{figure*} %fig5
\vspace*{12pt}
 \begin{center}
 \mbox{%
 \epsfxsize=112.575mm
 \epsfbox{che-5.eps}
 }
 \end{center}
 \vspace*{-9pt}
\Caption{Графики мгновенных интенсивностей потоков заявок 
покупателей $\lambda_+(t)$ (сплошная линия) и продав\-цов~$-\lambda_-(t)$ 
(пунктирная линия), размер скользящего окна $\Delta t \hm= 60$~с} 
\label{fig:intens2sides}
\end{figure}


\begin{multicols}{2}

Для анализа были выбраны данные за первые три часа торгов данным
инструментом за дневную сессию 11~ноября 2012~г.; график цены,
постро\-енный по сделкам, изображен на рис.~\ref{fig:price}.



На рис.~\ref{fig:intens2sides} представлены графики мгновенных
интенсивностей процессов $N_+(t)$ и $N_-(t)$ (количество заявок
каждого типа за последнюю минуту). Данный рисунок хорошо
подтверждает мультипликативное представление мгновенных
интенсивностей в виде $\lambda_+(t) \hm= \alpha_+(t) \lambda(t)$ и
$\lambda_-(t) \hm= \alpha_-(t) \lambda(t)$.


На рис.~\ref{fig:intensities} изображены графики тех же процессов,
но совмещенные в положительной полуплоскости. Наблюдаемые
расхождения графиков означают локальное преобладание покупателей над
продавцами ($\lambda_+(t) \hm> \lambda_-(t)$) или продавцов над
покупателями ($\lambda_-(t) \hm> \lambda_+(t)$), согласованные же их
падение или рост соответствуют общему падению или росту
интенсивности торгов без особенной борьбы между покупателями и
продавцами и, как следствие, без колебаний цены.



На рис.~\ref{fig:intratio} изображен график процесса $r(t)$, из
которого хорошо видно преобладание покупателей над продавцами, что
отразилось в поведении цены на протяжении наблюдаемого периода.
Другой особенностью графика является наличие уровня поддержки $r\hm = 0{,}6$, 
что может означать наличие крупного покупателя, который
сдерживал натиск продавцов при достижении данного уровня дисбаланса
сил (это также согласуется с поведением цены в эти моменты времени).

\end{multicols}

\begin{figure} %fig6
\vspace*{1pt}
 \begin{center}
 \mbox{%
 \epsfxsize=112.55mm
 \epsfbox{che-6.eps}
 }
 \end{center}
 \vspace*{-9pt}
\Caption{Графики мгновенных интенсивностей потоков заявок покупателей 
$\lambda_+(t)$ (сплошная линия) и продав\-цов~$\lambda_-(t)$ (пунктирная 
линия), размер скользящего окна $\Delta t \hm= 60$~с} 
\label{fig:intensities}
%\end{figure}
%\begin{figure*} %fig7
\vspace*{12pt}
 \begin{center}
 \mbox{%
 \epsfxsize=111.303mm
 \epsfbox{che-7.eps}
 }
 \end{center}
 \vspace*{-9pt}
\Caption{Отношение мгновенных интенсивностей
покупателей и продавцов $r(t) \hm= {\lambda_+(t)}/{\lambda_-(t)}$,
размер скользящего окна $\Delta t \hm= 60$~с} 
\label{fig:intratio}
%\vspace*{6pt}
\end{figure}

\begin{multicols}{2}

\section{Заключение}

В данной статье была построена микроструктурная модель, описывающая
информационные потоки в сложных финансовых системах, и смоделирована
случайная природа интенсивностей потоков заявок, определяющих
механизм ценообразования финансовых инструментов. При их
моделировании отдельно был рассмотрен поток внешнего информационного
фона со случайной интенсивностью и произведена его аппроксимация.

Также была предложена модель обобщенного процесса цены, учитывающая
всю доступную информацию о потоках заявок и допускающая дальнейшую
аналитическую интерпретацию. Важным результатом являются также
предложенные мультипликативные пред\-став\-ле\-ния интенсивностей потоков
заявок~(\ref{e5-ch}) и~(\ref{e6-ch}), находящие подтверждение на реальных данных и
позволяющие анализировать характеристики, связанные с интенсивностями потоков заявок, 
а также мгновенное соотношение сил покупателей и продавцов без моделирования внешнего
информационного фона, практически не поддающегося прогнозированию.

В заключение авторы статьи выражают искреннюю признательность
профессору Чикагского университета Юрию Георгиевичу Баласанову за
полезное обсуждение вопросов, затронутых в статье.

{\small\frenchspacing
{%\baselineskip=10.8pt
\addcontentsline{toc}{section}{Литература}
\begin{thebibliography}{99}



\bibitem{Korolev2011-1} 
\Au{Королев В.\,Ю.} Ве\-ро\-ят\-ност\-но-ста\-ти\-сти\-че\-ские методы декомпозиции
волатильности хаотических процессов.~--- М.: Изд-во Моск. ун-та, 2011.

\bibitem{KorolevBeningShorgin2011-1} 
\Au{Королев В.\,Ю., Бенинг В.\,Е., Шоргин~С.\,Я.} 
Математические основы теории риска.~--- 2-е изд., перераб. и доп.~--- М.: Физматлит, 2011. 620~с.

\bibitem{ContRamaStoikov2010b} 
\Au{Cont R., Stoikov S., Talreja~R.} A~stochastic model for order book dynamics~//
Operations Res., 2010. Vol.~58. No.\,3. P.~549--563.

\bibitem{ContLarrard2011} %4
\Au{Cont R., de Larrard~ A.} 
Price dynamics in a Markovian limit order
market. Working Paper, 2011. Available: {\sf http://ssrn.com/abstract=1735338}.

\bibitem{Bouchaud2002} %5
\Au{Bouchaud J.-P., Mezard M., Potters~M.}
Statistical properties of stock order books: Empirical results and
models~// Quantitative Finance, 2002. Vol.~2. P.~251--256.

\bibitem{ZovkoFarmer2002} %6
\Au{Zovko I., Farmer J.\,D.} The power of patience;
A~behavioral regularity in limit order placement~// Quantitative
Finance, 2002. Vol.~2. P.~387--392.

\bibitem{Huang2012} %7
\Au{Huang H., Kercheval A.\,N.} 
A~generalized birth--death stochastic model for high-frequency order book dynamics~// 
Quantitative Finance, 2012. Vol.~12. No.\,4. P.~547--557.

\bibitem{AbergelJedidi2011} %8
\Au{Abergel F., Jedidi~A.} A~mathematical approach to order book modelling~//
Econophysics of order-driven markets~/ Eds.
F.~Abergel, B.\,K.~Chakrabarti, A.~Chakraborti, M.~Mitra.~--- New York: Springer, 2011.
P.~93--108.

\bibitem{KorolevShevtsova2012} %9
\Au{Korolev V., Shevtsova I.} An improvement of the Berry--Esseen
inequality with applications to Poisson and mixed Poisson random
sums~// Scandinavian Actuarial~J., 2012. No.\,2. P.~81--105.
Available online since June 4, 2010.

\bibitem{Gleser1987} %10
\Au{Gleser L.\,J.} 
The gamma distribution as a mixture of exponential
distributions: Technical Report \#\,87-28.~--- West Lafayette: Purdue
University, 1987. 6~p.




\label{end\stat}

\bibitem{GorsheninDoynikovKorolevKuzmin2012}  %11
\Au{Gorshenin A., Doynikov A., Korolev~V., Kuzmin~V.} Statistical
properties of the dynamics of order books: Empirical results~//
Applied Problems in Theory of Probabilities and Mathematical
Statistics Related to Modeling of Information Systems: Abstracts of
4th International Workshop.~--- Moscow: IPI RAS, 2012. P.~31--51.

\end{thebibliography}
}
}

\end{multicols}