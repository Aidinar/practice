
%\renewcommand{\r}{\mathbb R}
%\newcommand{\N}{\mathbb N}
%\renewcommand{\P}{{\sf P}}
%\newcommand{\E}{{\sf E}}
%\newcommand{\D}{{\sf D}}
%\newcommand{\sign}{{\rm sign}}


%\newcommand{\I}{\mathbb{I}}
%\newcommand{\betm}{{\beta_{m+1+\delta}}}
%\newcommand{\bet}{\beta_{2+\delta}}
%\renewcommand{\endproof}{\hfill$\Box$}

%\newcommand{\la}{\lambda}
%\newcommand{\si}{{\rm Si}\:}
%\renewcommand{\Re}{{\rm Re}\:}
%\newcommand{\eqd}{\stackrel{d}{=}}

\def\stat{korolev}

\def\tit{ОБОБЩЕННЫЕ ДИСПЕРСИОННЫЕ ГАММА-РАСПРЕДЕЛЕНИЯ КАК
ПРЕДЕЛЬНЫЕ ДЛЯ СЛУЧАЙНЫХ СУММ$^*$}

\def\titkol{Обобщенные дисперсионные гамма-распределения как
предельные для случайных сумм}

\def\autkol{Л.\,М.~Закс,  В.\,Ю.~Королев}

\def\aut{Л.\,М.~Закс$^1$,  В.\,Ю.~Королев$^2$}

\titel{\tit}{\aut}{\autkol}{\titkol}

{\renewcommand{\thefootnote}{\fnsymbol{footnote}}\footnotetext[1]
{Работа поддержана Российским
фондом фундаментальных исследований (проекты 11-01-00515а,
11-07-00112а, 11-01-12026-офи-м и 12-07-00115а), Министерством образования и науки
(госконтракт 16.740.11.0133).}}

\renewcommand{\thefootnote}{\arabic{footnote}}
\footnotetext[1]{Альфа-банк, отдел моделирования и математической
статистики, lily.zaks@gmail.com}
\footnotetext[2]{Факультет вычислительной математики и кибернетики Московского государственного
университета им.\ М.\,В.~Ломоносова; Институт проблем информатики
Российской академии наук, vkorolev@cs.msu.su}


\Abst{Доказана общая теорема о необходимых и достаточных
условиях сходимости распределений сумм случайного числа независимых
одинаково распределенных случайных величин к однопараметрическим
сдвиг-мас\-штаб\-ным смесям нормальных законов. В~качестве следствия
получены необходимые и достаточные условия сходимости распределений
случайных сумм независимых одинаково распределенных случайных
величин к обобщенным дисперсионным гам\-ма-рас\-пре\-де\-ле\-ни\-ям. Для
частного случая~--- специальных случайных блужданий с непрерывным
временем, порожденных обобщенными дважды стохастическими
пуассоновскими процессами,~--- приведены оценки скорости этой
сходимости.}

\KW{случайная сумма; обобщенное гиперболическое
распределение; обобщенное обратное гауссовское распределение;
обобщенное гам\-ма-рас\-пре\-де\-ле\-ние; обобщенное дисперсионное
гамма-распределение; смесь распределений вероятностей;
идентифицируемые смеси; аддитивно замкнутое семейство; оценка
ско\-рости сходимости}


\vskip 14pt plus 9pt minus 6pt

      \thispagestyle{headings}

      \begin{multicols}{2}

            \label{st\stat}


\section{Введение}

В~данной работе в качестве более гибкой альтернативы часто (и
успешно) применяемым в практических исследованиях обобщенным
гипер\-бо\-лическим распределениям рассматривается класс обобщенных
дисперсионных гам\-ма-рас\-пре\-де\-ле\-ний и дается теоретическое обоснование
использованию его представителей в качестве асимптотических
аппроксимаций при решении практических задач.

\subsection{Обобщенные гиперболические распределения}

Плотность {\it обобщенного обратного гауссовского распределения}
обозначим $p_{\mathrm{GIG}}(x;\nu,\mu,\lambda)$:
\begin{multline*}
p_{\mathrm{GIG}}(x;\nu,\mu,\lambda)={}\\
{}=
\fr{\lambda^{\nu/2}}{2\mu^{\nu/2}K_{\nu}\left(\sqrt{\mu\lambda}\right)}\,
x^{\nu-1}\exp\left\{-\fr{1}{2}\left(\fr{\mu}{x}+\lambda
x\right)\right\}\,,\\  x>0\,. %\label{e1-kor}
\end{multline*}
Здесь $\nu\in\r$,
$$
\begin{array}{lll}
\mu>0, & \lambda\geqslant 0, & \text{если }\nu<0\,;\vspace{1mm}\cr \mu>0, &
\lambda>0, & \text{если }\nu=0\,;\vspace{1mm}\cr \mu\geqslant0, & \lambda>0,
& \text{если }\nu>0\,,
\end{array}
$$
$K_{\nu}(z)$~--- модифицированная бесселева функция третьего рода
порядка~$\nu$:
\begin{multline*}
K_{\nu}(z)=\fr{1}{2}\int\limits_{0}^{\infty}y^{\nu-1}\exp
\left\{-\fr{z}{2}\left(y+\fr{1}{y}\right)\right\}dy\,,\\
 z\in\mathbb{C}\,,\quad \mathrm{Re}\,z>0\,.
\end{multline*}
Соответствующую функцию распределения обозначим
$P_{\mathrm{GIG}}(x;\nu,\mu,\lambda)$:
$$
P_{\mathrm{GIG}}(x;\nu,\mu,\lambda)=
\begin{cases}0,& x<0\,;\\
{\displaystyle\int\limits_{0}^{x}p_{\mathrm{GIG}}(z;\nu,\mu,\lambda)\,dz,}&
x\geqslant0\,.
\end{cases}
$$
Как отмечено в статье~\cite{Seshadri1997}, обобщенное обратное
гауссовское распределение введено в 1946~г.\ Этьеном Альфеном
({\'E}tienne Halphen), который использовал его для описания объема
воды, проходящего ежемесячно через гидроэлектростанции. В~\cite{Seshadri1997} обобщенное обратное гауссовское распределение
названо {\it распределением Альфена}. 
В~1973~г.\ это распределение было переоткрыто Гербертом Зихелем~\cite{Sichel1973}, 
который использовал его в качестве смешивающего
закона при рассмотрении специальных смешанных пуассоновских
распределений ({\it распределений Зихеля}, см., например,~\cite{KorolevBeningShorgin2011}) 
как дискретных распределений с
тяжелыми хвостами. В~1977~г.\ эти распреде-\linebreak\vspace*{-12pt}

\pagebreak

\noindent
ления снова переоткрыл
О.-Э.~Барн\-дорфф-Ниль\-сен~\cite{BN1977, BN1978}, который, в частности,
использовал их для описания распределения размеров частиц.

Класс обобщенных обратных гауссовских распределений довольно обширен
и содержит, в част\-ности, как распределения с экспоненциально
убывающими хвостами (гам\-ма-рас\-пре\-де\-ле\-ние ($\mu\hm=0$, $\nu\hm>0$)), так и
распределения с хвостами, убывающими степенным образом (обратное
гам\-ма-рас\-пре\-де\-ле\-ние ($\lambda\hm=0$, $\nu\hm<0$), обратное гауссовское
распределение ($\nu\hm=-1/2$) и его предельный при $\lambda\to0$
случай~--- распределение Леви (устойчивое распределение с
характеристическим показателем, равным $1/2$, сосредоточенное на
неотрицательной полуоси~--- распределение времени достижения
стандартным винеровским процессом единичного уровня)).

Стандартную нормальную функцию распределения будем обозначать
$\Phi(x)$:
\begin{multline*}
\Phi(x)=\int\limits_{-\infty}^{x}\varphi(z)\,dz\,,\enskip
\varphi(x)=\fr{1}{\sqrt{2\pi}}\exp\left\{-\fr{x^2}{2}\right\}\,,\\
x\in\r\,.
\end{multline*}

В 1977--1978~гг.\ О.-Э.~Барндорфф-Ниль\-сен~\cite{BN1977, BN1978} ввел
класс {\it обобщенных гиперболических распределений} как класс
специальных сдвиг-мас\-штаб\-ных смесей нормальных законов. Пусть
$\alpha\hm\in\r$, $\beta\hm\in\r$. Если функцию обобщенного
гиперболического распределения с параметрами $\alpha$, $\beta$,
$\nu$, $\mu$, $\lambda$ обозначить
$P_{GH}(x;\alpha,\beta,\nu,\mu,\lambda)$, то по определению
\begin{multline}
P_{GH}(x;\alpha,\beta,\nu,\mu,\lambda)={}\\
{}=
\int\limits_{0}^{\infty}\!\Phi\!\left(\fr{x-\beta-\alpha
z}{\sqrt{z}}\right)p_{\mathrm{GIG}}(z;\nu,\mu,\lambda)\,dz,\
x\in\r.\!
\label{e2-kor}
\end{multline}
Обратим внимание, что в~(\ref{e2-kor}) смешивание происходит одновременно и по
параметру сдвига, и по параметру масштаба, но так как эти параметры
в~(\ref{e2-kor})  связаны жесткой зависимостью, то фактически смесь~(\ref{e2-kor})
является {\it однопараметрической}.

Несложно убедиться, что плотность
$p_{GH}(x;\alpha,\beta,\nu,\mu,\lambda)$ обобщенного
гиперболического распределения имеет вид:
\begin{multline*}
p_{GH}(x;\alpha,\beta,\nu,\mu,\lambda)={}\\
{}=
\int\limits_{0}^{\infty}\fr{1}{\sqrt{z}}\varphi\left(\fr{x-\beta-\alpha
z}{\sqrt{z}}\right)p_{\mathrm{GIG}}(z;\nu,\mu,\lambda)\,dz={}\\
{}
=\int\limits_{0}^{\infty}\fr{1}{\sqrt{2\pi
z}}\exp\left\{-\fr{(x-\beta-\alpha
z)^2}{2z}\right\}\times{}
\end{multline*}

\noindent
\begin{multline*}
{}\times \fr{\lambda^{\nu/2}z^{\nu-1}}{2\mu^{\nu/2}K_{\nu}
\left(\sqrt{\mu\lambda}\right)} \exp\left\{-\fr{1}{2}\left(\fr{\mu
}{z}+\lambda  z\right)\right\}dz={}
\\
{}=\fr{\lambda^{\nu/2}(\lambda
+\alpha^2)^{\nu/2-1/4}}{2\sqrt{2\pi}\mu^{\nu/2}K_{\nu}\left(\sqrt{\mu\lambda}\right)}
\left(\mu +(x-\beta)^2\right)^{1/4-\nu/2}\times{}\\
{}\times\exp\{\alpha
(x-\beta)\}K_{\nu-1/2}\!\left(\!\sqrt{\left(\mu
+(x-\beta)^2\right)\left(\lambda  +\alpha^2\right)}\right).\hspace*{-3.10736pt}
\end{multline*}

Класс обобщенных гиперболических распределений очень широк и
содержит, в частности, (а)~симметричные и скошенные (skew)
распределения Стьюдента (в том числе распределение Коши), которым в
представлении~(\ref{e2-kor}) соответствуют смешивающие обратные
гам\-ма-рас\-пре\-де\-ле\-ния; (б)~дис\-пер\-си\-он\-ные гам\-ма-рас\-пре\-де\-ле\-ния
(Variance Gamma (VG) distributions) (в том чис\-ле симметричные и
несимметричные распределения Лапласа), которым в представлении~(\ref{e2-kor})
соответствуют смешивающие гам\-ма-рас\-пре\-де\-ле\-ния; 
(в)~нормальные$\backslash\!\backslash$обратные нормальные (NIG)
распределения, которым в представлении~(\ref{e2-kor}) соответствуют смешивающие
обратные нормальные распределения, и многие другие типы.

Обобщенные гиперболические распределения продемонстрировали
высочайшую адекватность при их применении для описания
статистических\linebreak закономерностей поведения различных харак\-те\-ристик
сложных открытых сис\-тем, в част\-ности тур\-булентных сис\-тем и
финансовых рынков. Пуб\-ли\-кации, посвященные моделям, основанным на\linebreak
обобщенных гиперболических распределениях, ис\-чис\-ля\-ют\-ся сотнями.
Достаточно упомянуть лишь канонические работы~[5--16]. %\cite{BN1978} -- %,
%BN1979, MadanSeneta1990, EberleinKeller1995, Prause1997,
%CarrMadanChang1998, EberleinKellerPrause1998, BarndorffNielsen1998,
%EberleinPrause1998, Shiryaev1998, Eberlein1999, \cite{BNBlaesildSchmiegel2004}.
Согласно расхожему мнению, столь
высокая адекватность обобщенных гиперболических моделей может быть
формально объяснена большим числом настраиваемых параметров,
позволяющим подогнать какую угодно модель к каким угодно данным.

Среди статистиков хорошо известно высказывание Ж.~Берт\-ра\-на <<Give
me four parameters and I shall describe an elephant; with five, it
will wave its trunk>> (цитируется по статье Л.~ЛеКама~\cite{LeCam1990}). 
Это обстоятельство, конечно же, играет свою роль,
однако на самом деле модели типа~(\ref{e2-kor}) в большинстве случаев адекватны
по гораздо более естественным глубоким причинам.

В прикладной теории вероятностей принято считать, что ту или иную
модель можно считать в достаточной мере обоснованной (адекватной)
только тогда, когда она является {\it асимптотической
аппроксимацией}, т.\,е.\ когда существует довольно простая
предельная схема (например, схема суммирования) и соответствующая
предельная теорема, в которой рассматриваемая модель выступает в
качестве предельного распределения~\cite{GnedenkoKolmogorov1949}. 
В~первоисточниках упомянутые выше обобщенные гиперболические модели
вводились чисто умозрительно как распределения процесса броуновского
движения со случайным временем, в каждый момент имеющим то или иное
обобщенное обратное гауссовское распределение. Лишь в статье~\cite{BN1982} 
со ссылкой на работу А.~Реньи~\cite{Renyi1960} имеется
довольно расплывчатый намек на то, что смеси нормальных законов
могут быть предельными для сумм случайного числа случайных величин.

Однако, как ни удивительно, несмотря на то что свойства обобщенных
гиперболических распределений изучены довольно полно, до недавнего
времени не было дано корректного доказательства того факта, что
обобщенные гиперболические распределения выступают в качестве
предельных в прос\-тей\-шей схеме случайно остановленных случайных
блужданий. И, стало быть, приводимая в некоторых работах
аргументация, связывающая смешивание в модели~(\ref{e2-kor}) со случайным
характером во\-ла\-тиль\-ности при применении обобщенных гиперболических
распределений в финансовой математике, не имела строгого формального
обоснования. Возможно, причина этого лежит в том, что в схеме
<<нарастающих>> сумм, рассматривавшейся в~\cite{Renyi1960}, полное
решение указанной задачи невозможно. Его можно получить, лишь
рассматривая случайные суммы в рамках асимптотической схемы серий.
Основополагающей работой в этом направлении стала работа
Б.\,В.~Гнеденко и Х.~Фахима~\cite{GnedenkoFahim1969}.

<<Асимптотическое>> обоснование некоторых из упомянутых выше моделей
было дано лишь недавно в статьях~\cite{KorolevSokolov2012, Korolev2012}, 
где показано, что скошенные распределения Стьюдента и
дисперсионные гам\-ма-рас\-пре\-де\-ле\-ния могут выступать в качестве
предельных в довольно простых предельных теоремах для регулярных
статистик, построенных по выборкам случайного объема, в частности в
схеме случайного суммирования случайных величин, и, следовательно,
могут считаться {\it естественными} асимптотическими аппроксимациями
для распределений многих процессов, например, сходных с
неоднородными случайными блужданиями.

В статье~\cite{Korolev2012b} приведена общая теорема о необходимых и
достаточных условиях сходимости распределений сумм случайного числа
независимых одинаково распределенных случайных величин к
однопараметрическим сдвиг-мас\-штаб\-ным смесям нормальных законов и в
качестве следствия из нее\linebreak получены необходимые и достаточные условия\linebreak
сходимости распределений случайных сумм независимых одинаково
распределенных случайных величин к обобщенным гиперболическим
распределениям. На примере довольно общего и просто
интерпретируемого частного случая~--- специальных случайных блужданий
с непрерывным временем, порожденных обобщенными дважды
стохастическими пуассоновскими процессами,~--- там же приведены
оценки скорости этой сходимости. В~данной работе результаты статьи~\cite{Korolev2012b} 
переносятся на обобщенные дисперсионные
гам\-ма-рас\-пре\-де\-ления.

\subsection{Обобщенные дисперсионные гамма-распределения}

Гамма-распределение и обратное гам\-ма-рас\-пре\-де\-ле\-ние являются частными
представителями класса обобщенных гам\-ма-рас\-пре\-де\-ле\-ний (ОГ-рас\-пре\-де\-ле\-ния), важная роль
которых в моделировании и анализе стохастической структуры
информационных потоков описана в книге~\cite{KorolevShorgin2011}.
Обобщенные гам\-ма-рас\-пре\-де\-ле\-ния были впервые
описаны как единое семейство в 1962~г.\ в работе~\cite{Stacy1962} в
качестве семейства вероятностных моделей, включающего в себя
одновременно гам\-ма-рас\-пре\-де\-ле\-ния и распределения Вейбулла.

Обобщенным гам\-ма-рас\-пре\-де\-ле\-ни\-ем называется распределение,
определяемое плотностью вероятностей вида
\begin{multline}
f(x;\nu,\kappa,\delta)={}\\
\hspace*{-5mm}{}=
\begin{cases}
\fr{|\nu|}{\delta\Gamma(\kappa)}\left(\fr{x}{\delta}\right)^{\kappa\nu-1}\exp
\left\{-\left(\fr{x}{\delta}\right)^{\nu}\right\}\,,
& x\geqslant0\,;\\
0\,, & x<0\,,
\end{cases}\!
\label{e16-1-kor}
\end{multline}
с параметрами $\nu\in\mathbb{R},\,\kappa,\,\delta\in{\mathbb R}^+$,
отвечающими соответственно за {\it степень, форму и масштаб}, где
$\Gamma(\kappa)$~--- эйлерова гам\-ма-функ\-ция:
$$
\Gamma(\kappa)=\int\limits_{0}^{\infty}x^{\kappa-1}e^{-x}\,dx\,.
$$
 Семейство ОГ-рас\-пре\-де\-ле\-ний включает в
себя практически все наиболее популярные абсолютно непрерывные
распределения. В~частности, семейство ОГ-рас\-пре\-де\-ле\-ний содержит
следующие распределения.
\begin{enumerate}[1.]
\item Гамма-распределение ($\nu=1)$:
$$
\hspace*{-5.31412pt}f(x;\kappa,\theta)=\fr{1}{\Gamma(\kappa)}\theta^{\kappa}x^{\kappa-1}e^{-\theta
x},\ \  x\geqslant0,\ \kappa>0,\ \theta>0.
$$
\begin{enumerate}[{1.}1.]
\item Показательное (экспоненциальное) распределение ($\nu=1,\, \kappa=1)$:
$$%\begin{equation}\label{exp}
f(x;\theta)=\theta e^{-\theta x},\ \ \ x\geqslant0,\ \theta>0\,.
$$
\item Распределение Эрланга ($\nu=1,\, \kappa\in\mathbb{N})$:
\begin{multline*}
f(x;\kappa,\theta)=\fr{1}{\Gamma(\kappa)}\theta^{\kappa}x^{\kappa-1}e^{-\theta
x}\,,\\
 x\geqslant0\,,\ \ \kappa>0\,,\ \ \theta>0\,.
\end{multline*}

\item Распределение хи-квадрат ($\nu=1$, $\delta=2)$:
$$%\begin{equation}\label{chisq}
\hspace*{-28pt}f(x;n)=\fr{1}{2\Gamma({n}/{2})}\left(\fr{x}{2}\right)^{n/2-1}\!\!e^{-x/2}.
\ \ x\geqslant0,\ n\in\mathbb{N}.
$$
\end{enumerate}
\item Распределение Накагами ($\nu=2$):
\begin{multline*}
f(x;\mu,\lambda)=\fr{2(\lambda\mu)^{\mu}}{\Gamma(\mu)}\,x^{2\mu-1}e^{-\lambda\mu
x^2}\,,\\
 x\geqslant0\,, \ \mu>0\,,\ \lambda>0\,.
\end{multline*}
\begin{enumerate}[{2.}1.]
\item Полунормальное распределение (распределение максимума
винеровского процесса на отрезке $[0,1]$)
($\nu=2,\,\kappa=1/2$):
$$%\begin{equation}\label{maxwin}
\hspace*{-1.38493pt}f(x;\delta)=\sqrt{\fr{2}{\pi\delta}}\exp\left\{-\fr{x^2}{2\delta^2}\right\},\
\ \ x\geqslant0, \ \delta>0.
$$
\item Распределение Рэлея ($\nu=2,\,\kappa=1$):
$$%\begin{equation}\label{ray}
f(x;\delta)=\fr{x}{\delta^2}\exp\left\{-\fr{x^2}{2\delta^2}\right\},\
\ \ x\geqslant0, \ \delta>0.
$$

\item Хи-распределение ($\nu=2,\,\delta=\sqrt{2}$):
\begin{multline*}
\hspace*{-13.01563pt}f(x;n)=\fr{1}{2^{n/2-1}\,\Gamma\left({n}/{2}\right)}x^{n-1}\exp\left\{-\fr{x^2}{2}\right\}\,,\\
 x\geqslant0\,, \ \ n\in\mathbb{N}\,.
\end{multline*}
\item Распределение Максвелла (распределение модулей скоростей движения
молекул в разреженном газе) ($\nu\hm=2,\,\kappa\hm=3/2$):
\begin{equation*}
\hspace*{-13.16995pt}f(x;\delta)=\sqrt{\fr{2}{\pi}}\,\fr{x^2}{\delta^3}\exp\left\{-\fr{x^2}{2\delta^2}\right\}\,,\
 x\geqslant0\,, \ \delta>0\,.
\end{equation*}
\end{enumerate}
\item Распределение Вейбулла--Гнеденко ($\kappa\hm=1$):
\begin{multline*}
f(x;\eta,\mu)=\fr{\eta
x^{\eta-1}}{\mu^{\eta}}\exp\left\{-\left(\fr{x}{\mu}\right)^{\eta}\right\}\,,\\\
x\geqslant0\,, \ \eta>0\,,\ \mu>0\,.
\end{multline*}

\item Обратное гамма-распределение ($\nu\hm=-1$):
\begin{multline*}
f(x;\mu,\lambda)=\fr{1}{\mu\lambda\Gamma(\lambda)}\left(\fr{\mu\lambda}{x}\right)^{\lambda+1}\exp
\left\{-\fr{\mu\lambda}{x}\right\}\,,\\
x\geqslant0\,, \ \lambda>0\,,\ \mu>0\,.
\end{multline*}
\begin{enumerate}[{4.}1.]
\item Распределение Леви ($\nu=-1,\, \kappa=1/2$):
$$%\begin{equation}\label{Levy}
\hspace*{-27.0681pt}f(x;\mu)=\sqrt{\fr{\mu}{2\pi}}\fr{1}{x^{3/2}}\exp\left\{-\fr{\mu}{2x}\right\}\,,\enskip
 x\geqslant0\,, \ \mu>0\,.
$$
\end{enumerate}
\item Логнормальное распределение ($\kappa\to\infty$):
\begin{multline*}
f(x;\mu,\delta)=\fr{1}{\delta x\sqrt{2\pi}}\exp\left\{-\fr{(\log
x-\mu)^2}{2\delta^2}\right\}\,,\\ x\geqslant0\,, \enskip \mu\in\mathbb{R}\,,\enskip
\delta>0\,.
\end{multline*}

\end{enumerate}

Широкая применимость ОГ-распределений обусловлена возможностью их
использования в качестве адекватных асимптотических аппроксимаций,
поскольку практически все они выступают в качестве предельных в
различных предельных теоремах теории вероятностей, а именно:
\begin{itemize}
\item показательное распределение выступает в качестве предельного
как в схеме максимума (минимума) (см., например,~\cite{Gumbel1965}),
так и в схеме геометрического суммирования, описывая распределение
времени восстановления в прореженных процессах восстановления,
вы\-сту\-па\-ющих моделями потоков редких событий (см., например,~\cite{Kalashnikov1997});
\item гамма-распределение является безгранично делимым и потому
выступает в качестве предельного для распределений сумм независимых
равномер\-но предельно малых случайных величин. При этом распределение
Эрланга возникает как %\linebreak
 допредельное распределение суммы независимых
экспоненциально распределенных случайных величин, что в терминах
случайной интенсивности может означать, что если случайная
интенсивность потока поступления запросов имеет гамма-распределение
со значимым параметром формы, то при обработке этих запросов в
основном задействованы механизмы последовательной обработки
информации;
\item распределение Вейбулла--Гне\-ден\-ко принадлежит к так
называемому первому типу предельных распределений экстремальных
порядко-\linebreak вых статистик (минимума или максимума) (см.,\linebreak  например,~\cite{Gumbel1965}), 
что в терминах случайной интенсивности может
означать, что если случайная интенсивность потока поступления
запросов имеет распределение Вей\-бул\-ла--Гне\-ден\-ко со значимым
параметром степени, то при обработке этих запросов в основном
задействованы механизмы параллельной обработки информации;
\item полунормальное распределение (распределение модуля
стандартной нормальной случайной величины) возникает как предельное
для максимальных частичных сумм независимых случайных величин (см.,
например,~\cite{KorolevSokolov2008});
\item распределение Леви принадлежит к классу устойчивых законов и
потому является предельным для сумм независимых одинаково
распределенных случайных величин. Оно также является распределением
времени достижения стандартным винеровским процессом (процессом
броуновского движения) фиксированного уровня;
\item логнормальное распределение выступает в качестве предельного
для распределения размера частиц при дроблении (см., например,~\cite{Korolev2009}).
\end{itemize}

Эти свойства ОГ-распре\-де\-ле\-ний обосновывают, в частности,
целесообразность моделирования с их помощью распределения случайной
интенсивности потока запросов в информационных системах. Это
семейство также широко используется в других прикладных задачах в
самых разных областях (см., например,~\cite{KorolevShorgin2011}).

Упомянутые выше четырехпараметрические семейства скошенных
распределений Стьюдента и дисперсионных гам\-ма-рас\-пре\-де\-ле\-ний являются
подклассами введенного в работе~\cite{KorolevSokolov2012}
пятипа\-ра\-мет\-ри\-че\-ско\-го семейства распределений
\begin{equation}
W(x;a,\sigma,\nu,\kappa,\delta)=\!\int\limits_{0}^{\infty}
\!\Phi\left(\!\fr{x-au}{\sigma\sqrt{u}}\!\right)
f(u;\nu,\kappa,\delta)\,du,\!\!
\label{e17-1-kor}
\end{equation}
где $f(u;\nu,\kappa,\delta)$~--- плотность ОГ-рас\-пре\-де\-ле\-ния~(\ref{e16-1-kor}). 
В~статье \cite{KorolevSokolov2012} распределения вида~(\ref{e17-1-kor}) названы
{\it обобщенными дисперсионными гамма-распределениями}.

Задача поиска универсальной модели статистических закономерностей во
многих областях, в частности в финансовой математике или в физике
плазмы, подобна задаче отыскания <<философского камня>> в алхимии и
поэтому не имеет точного решения. Однако, основываясь на
вышеперечисленных аналитических и асимптотических свойствах
представителей семейства ОГ-рас\-пре\-де\-ле\-ний и предельных теоремах для
сумм независимых случайных величин как тео\-ре\-ти\-ко-ве\-ро\-ят\-но\-ст\-ной
формализации принципа неубывания неопределенности в сложных системах~\cite{GnedenkoKorolev1996}, 
можно утверждать, что семейство
обобщенных дисперсионных гам\-ма-рас\-пре\-де\-ле\-ний является {\it
практически} универсальным для многих задач. Оно представляется еще
более гибкой моделью, нежели обобщенные гиперболические
распределения, так как класс обобщенных гам\-ма-рас\-пре\-де\-ле\-ний в
определенном смысле шире класса обобщенных обратных гауссовских
распределений, поскольку, в отличие от последнего, он содержит
распределения вейбулловского (экс\-по\-нен\-ци\-аль\-но-сте\-пен\-но\-го) типа с
произвольным показателем степени в экспоненте.

\section{Критерий сходимости распределений случайных сумм к~обобщенным
дисперсионным гамма-распределениям}

Будем считать, что все случайные величины, о которых пойдет речь
ниже, заданы на одном вероятностном пространстве $(\Omega,\,
\mathfrak{A},\,{\sf P})$. Пусть $\{X_{n,j}\}_{j\geqslant1},$
$n=1,2,\ldots,$~--- семейство последовательностей одинаково
распределенных в каждой последовательности (при каждом фиксированном
$n$) случайных величин. Пусть $\{N_n\}_{n\geqslant1}$~---
последовательность целочисленных неотрицательных случайных величин
таких, что пpи каждом $n\hm\geqslant1$ случайные величины
$N_n,X_{n,1},X_{n,2},\cdots$ независимы. Положим
$$
S_{n,k}=X_{n,1}+\cdots +X_{n,k}\,.
$$
Во избежание недоразумений полагаем $\sum\limits_{j=1}^0\hm=0$. Символ
$\Longrightarrow$ будет обозначать слабую сходимость (сходимость по
распределению).

Расстояние Леви, которое, как известно, метризует слабую сходимость
в пространстве функций распределения, будем обозначать
$L(\,\cdot\,,\,\cdot\,)$,
\begin{multline*}
L(F,\,G)=\inf\{\epsilon:\,G(x-\epsilon)-\epsilon\leqslant F(x)\leqslant{}\\
{}\leqslant
G(x+\epsilon)+\epsilon\ \forall\ x\in\r\}\,.
\end{multline*}
%Под слабой сходимостью последовательности с.в. мы подразумеваем
%слабую сходимость их ф.p.. Аналогично, под расстоянием Леви между
%с.в. мы подразумеваем расстояние Леви между их ф.p..

Каждой паре функций распределения $(F,\,H)$ поста\-вим в соответствие
множество $\mathcal{M}(F|H)$, содержащее все функции распределения
$Q(x)$ с $Q(0)\hm=0$, обеспечивающие представление характеристической
функции, соответствующей функции распределения~$F$, в виде
степенн$\acute{\mbox{о}}$й смеси характеристической функции, соответствующей
функции распределения~$H$:
$$
\int\limits_{-\infty}^{\infty}e^{itx}\,dF(x)=\int\limits_{0}^{\infty}h^x(t)\,dQ(x)\,,\enskip
 t\in\r\,,
$$
где
$$
h(t)=\int\limits_{-\infty}^{\infty}e^{itx}\,dH(x)\,,\enskip t\in\r\,.
$$
Везде далее сходимость будет подразумеваться при $n\hm\to\infty$.

\smallskip

\noindent
\textbf{Лемма 1}. \textit{Предположим, что существуют последовательность
натуральных чисел $\{k_n\}_{n\geqslant1}$ и функция распределения $H(x)$
такие, что}
$$
{\sf P}\left(S_{n,k_n}<x\right)\Longrightarrow H(x)\,.
$$
\textit{Предположим, что $N_n\to\infty$ по вероятности. Для того чтобы имела
место сходимость}
$$
{\sf P}\left(S_{n,N_n}<x\right)\Longrightarrow F(x)
$$
\textit{распределений случайных сумм к некоторой функции распределения
$F(x)$, необходимо и достаточно, чтобы существовала слабо компактная
последовательность функций распределения $\{Q_n^*(x)\}_{n\geqslant1}$
такая, что выполняются условия}:
\begin{enumerate}
\item[$\mathrm{(i)}$] $Q_n^*(x)\in\mathcal{M}(F|H)$, $n=1,2,\ldots$;
\item[$\mathrm{(ii)}$] $L(Q_n^*,Q_n)\longrightarrow 0$,
\end{enumerate}
{\it где $Q_n(x)={\sf P}(N_n<k_nx)$, $x\in\r$.}

\smallskip

\noindent
Д\,о\,к\,а\,з\,а\,т\,е\,л\,ь\,с\,т\,в\,о\,.\ \  Данное утверждение является частным
случаем теоремы~4.2.1~в \cite{GnedenkoKorolev1996}.

\smallskip

Напомним определение идентифицируемости смесей распределений
вероятностей, предложенное в работе~\cite{Teicher1961}. Для целей
данной статьи достаточно рассмотреть смеси распределений из
однопараметрических семейств. Пусть функция $H(x;y)$ определена на
плоскости $\mathbb{R}\times{\mathbb{R}}$. Предположим, что функция
$H(x;y)$ измерима по~$y$ при каждом фиксированном $x\hm\in\mathbb{R}$ и
является функцией распределения как функция аргумента~$x$ при каждом
фиксированном $y\hm\in\mathbb{R}$. Пусть $\mathcal{Q}$~--- некоторое
семейство функций распределения. Обозначим
\begin{multline}
\mathcal{F}={}\\
\hspace*{-2.5mm}{}=\left\{F(x)=\!\int\limits_{-\infty}^{\infty}\!
H(x;y)\,dQ(y)\,,\ x\in\mathbb{R}:\,Q\in\mathcal{Q}\right\}.\!\!
\label{e3-kor}
\end{multline}
Семейство $\mathcal{F}$ называется {\it идентифицируемым}, если из
равенства
$$
\int\limits_{-\infty}^{\infty}
H(x;y)\,dQ_1(y)=\int\limits_{-\infty}^{\infty}
H(x;y)\,dQ_2(y),\qquad x\in\mathbb{R}\,,
$$
с $Q_1\in\mathcal{Q}$, $Q_2\hm\in\mathcal{Q}$ вытекает, что
$Q_1(y)\hm\equiv Q_2(y)$.

Рассмотрим некоторые достаточные условия идентифицируемости смесей
рас\-пре\-де\-лений из од\-но\-па\-ра\-мет\-ри\-че\-ских семейств. Хорошо
\mbox{известно}, что в общем случае сдвиг-мас\-штаб\-ные смеси нормальных
законов не являются идентифицируемыми. Однако однопараметрические
сдвиг-мас\-штаб\-ные смеси нормальных законов типа~(\ref{e2-kor}) обладают этим
свойством, как будет показано при доказательстве теоремы~1.

Семейство функций распределения $\{H(x;y):\,y\hm>0\}$ называется {\it
аддитивно зам\-к\-ну\-тым}, если для любых $y_1\hm>0$, $y_2\hm>0$
справедливо соотношение:
\begin{equation}
H(x;y_1)\ast H(x;y_2)\equiv H(x;y_1+y_2)\,. \label{e4-kor}
\end{equation}
Здесь символ $\ast$ обозначает свертку. Иногда свойство~(\ref{e4-kor}) семейств
распределений вероятностей называется {\it воспроизводимостью} по
параметру~$y$.

Следующий результат принадлежит Г.~Тейчеру~\cite{Teicher1961}.

\smallskip

\noindent
\textbf{Лемма~2}. \textit{Предположим, что множество $\mathcal{Q}$ состоит
из всех функций распределения $Q(y)$ с $Q(0)\hm=0$. Пусть семейство
функций распределения $\{H(x;y):\,y\hm>0\}$ ад\-ди\-тив\-но замкнуто.
Тогда семейство смесей~$(\ref{e3-kor})$ является идентифицируемым.}

\smallskip

Основным результатом данного раздела является следующее утверждение.

\smallskip

\noindent
\textbf{Теорема~1}. \textit{Предположим, что существуют последовательность
натуральных чисел $\{k_n\}_{n\geqslant1}$ и число $\alpha\hm\in\mathbb{R}$
такие, что}
\begin{equation}
{\sf P}\left(S_{n,k_n}<x\right)\Longrightarrow \Phi(x-\alpha)\,.\label{e5-kor}
\end{equation}
\textit{Предположим, что $N_n\to\infty$ по вероятности. Для того чтобы имела
место сходимость распределений случайных сумм к некоторой функции
распределения $F(x)$:}
\begin{equation*}
{\sf P}\left(S_{n,N_n}<x\right)\Longrightarrow F(x)\,,
%\label{e6-kor}
\end{equation*}
\textit{необходимо и достаточно, чтобы существовала функция распределения
$Q(x)$ такая, что $Q(0)\hm=0$,}
\begin{gather}
F(x)=\int\limits_{0}^{\infty}\Phi\left(\fr{x-\alpha
z}{\sqrt{z}}\right)dQ(z)\,;\label{e7-kor}\\
{\sf P}(N_n<xk_n)\Longrightarrow Q(x)\,.
\label{e8-kor}
\end{gather}


\smallskip

\noindent
Д\,о\,к\,а\,з\,а\,т\,е\,л\,ь\,с\,т\,в\,о\,.\ \  Зафиксируем вещественное чис\-ло~$\alpha$
и положим
\begin{equation}
H_{\alpha}(x;y)=\Phi\left(\fr{x-\alpha y}{\sqrt{y}}\right)\,.
\label{e9-kor}
\end{equation}
Убедимся, что семейство так определенных функций распределения
аддитивно замкнуто по $y\hm>0$. Действительно, функции распределения~(\ref{e9-kor}) 
соответствует характеристическая функция
$$
\int\limits_{-\infty}^{\infty}e^{itx}\,d_xH_{\alpha}(x;y)=\exp
\left\{y\left(it\alpha-\fr{t^2}{2}\right)\right\}\,.
$$
Следовательно, свертке функций распределения $H_{\alpha}(x;y_1)$ и
$H_{\alpha}(x;y_2)$ соответствует характеристическая функция
\begin{multline*}
\exp\left\{y_1\left(it\alpha-\fr{t^2}{2}\right)\right\}\exp
\left\{y_2\left(it\alpha-\fr{t^2}{2}\right)\right\}={}\\
{}=
\exp\left\{\left(y_1+y_2\right)\left(it\alpha-\fr{t^2}{2}\right)\right\}
={}\\
{}=\int\limits_{-\infty}^{\infty}e^{itx}\,d_xH_{\alpha}(x;y_1+y_2)\,,
\end{multline*}
что и означает аддитивную замкнутость по $y\hm>0$ семейства функций
распределения~(\ref{e9-kor}). Таким образом, согласно лемме~2 семейство
одно\-па\-ра\-мет\-ри\-че\-ских сдвиг-мас\-штаб\-ных смесей нормальных законов
$$
\mathcal{F}_{\alpha}=\left\{\int\limits_{0}^{\infty}
H_{\alpha}(x;y)\,dQ(y)\,,\ x\in\mathbb{R}:\,Q\in\mathcal{Q}\right\}\,,
$$
где $\mathcal{Q}$~--- множество всех функций распределения $Q(y)$ с
$Q(0)\hm=0$, идентифицируемо. В~свою очередь, это означает, что для
любой функции распределения $F(x)$ множество
$\mathcal{M}(F|H_{\alpha})$ содержит не более одного элемента.
Действительно, обозначим $h_{\alpha}(t)\hm=\exp\{it\alpha-t^2/2\}$
(характе\-ри\-стическая функция $h_{\alpha}(t)$ соответствует функции\linebreak
распределения $\Phi(x\hm-\alpha)$). Тогда
$h_{\alpha}^y(t)\hm=\exp\{y(it\alpha\hm-t^2/2)\}$, и, стало быть,
характеристические функции
$\int\limits_{0}^{\infty}h_{\alpha}^y(t)\,dQ(y)$,
фигурирующие в определении множества $\mathcal{M}(F|H_{\alpha})$,
соответствуют функциям распределения
$\int\limits_{0}^{\infty}H_{\alpha}(x;y)\,dQ(y)$,
со\-став\-ля\-ющим семейство $\mathcal{F}$. При этом условие~(\ref{e7-kor}) означает,
что $Q\hm\in\mathcal{M}\left(F(x)|\Phi(x\hm-\alpha)\right)$. Теперь остается
сослаться на лемму~1, в которой роль условий~(i) и~(ii) играют
соответственно~(\ref{e7-kor}) и~(\ref{e8-kor}). Теорема доказана.

\smallskip

\noindent
\textbf{Замечание~1.} Условие~(\ref{e5-kor}) выполняется в следующей довольно
общей ситуации. Предположим, что случайные величины $X_{n,j}$ имеют
конечные дисперсии. Также предположим, что величины $X_{n,j}$ могут
быть представлены в виде
$$
X_{n,j}=X_{n,j}^*+\alpha_n\,,
$$
где $\alpha_n\in\mathbb{R}$, a $X_{n,j}^*$~--- случайная величина с
${\sf E} X_{n,j}^*\hm=0$, ${\sf D} X_{n,j}^*\hm=\sigma_n^2<\infty$, так
что ${\sf E} X_{n,1}\hm=\alpha_n$ и ${\sf D} X_{n,1}\hm=\sigma_n^2$.
Предположим, что $\alpha_nk_n\hm\to a$ и $k_n\sigma_n^2\hm\to 1$ при
$n\hm\to\infty$. Тогда вследствие хорошо известного результата о
необходимых и достаточных условиях сходимости к нормальному закону
распределений сумм независимых случайных величин с конечными
дисперсиями в схеме серий (см., например,~\cite{GnedenkoKolmogorov1949}) 
можно заметить, что соотношение~(\ref{e5-kor})
имеет место тогда и только тогда, когда выполнено условие
Линдеберга: для любого $\varepsilon\hm>0$
$$
\lim\limits_{n\to\infty}k_n{\sf E}(X_{n,1}^*)^2\mathbb{I}(|X_{n,1}^*|\geqslant\varepsilon)=0
$$
(здесь $\mathbb{I}(A)$~--- индикаторная функция множества (события)~$A$), 
т.\,е.\ квадратичные хвосты распределений слагаемых должны убывать достаточно \mbox{быстро}.

\smallskip

\noindent
\textbf{Следствие 1}. \textit{Предположим, что существуют
последовательность натуральных чисел $\{k_n\}_{n\geqslant1}$ и число
$\alpha\hm\in\mathbb{R}$ такие, что имеет место сходимость~$(\ref{e5-kor})$.
Предположим, что $N_n\hm\to\infty$ по вероятности. Для того чтобы имела
место сходимость распределений случайных сумм к обобщенным
дисперсионным гам\-ма-рас\-пре\-де\-ле\-ни\-ям$:$
\begin{equation}
{\sf P}\big(S_{n,N_n}<x\big)\Longrightarrow
W(x;\alpha,\sigma,\nu,\kappa,\delta)\,,\label{e10-kor}
\end{equation}
необходимо и достаточно, чтобы
\begin{equation}
{\sf P}(N_n<xk_n)\Longrightarrow F(x;\nu,\kappa,\delta)\,,\label{e11-kor}
\end{equation}
где $F(x;\nu,\kappa,\delta)$~--- функция распределения обобщенного
гам\-ма-распределения, соответствующая плотности
$f(x;\nu,\kappa,\delta)$ $($см.~$(\ref{e16-kor}))$.}

\smallskip

\noindent
Д\,о\,к\,а\,з\,а\,т\,е\,л\,ь\,с\,т\,в\,о\,.\ \  Из определения обобщенного
дисперсионного гам\-ма-рас\-пре\-де\-ле\-ния вытекает, что множество
$\mathcal{M}\left(W(x;\alpha,\sigma,\nu,\kappa,\delta)|\,\Phi(x\hm-\alpha)\right)$
состоит из {\it единственного} элемента $F(x;\nu,\kappa,\delta)$.
Поэтому, применяя теорему~1 с
$F(x)\hm=W(x;\alpha,\sigma,\nu,\kappa,\delta)$, замечаем, что в
рас\-смат\-ри\-ва\-емом случае условие~(\ref{e8-kor}) принимает вид~(\ref{e11-kor}). Теорема
доказана.

\smallskip

\noindent
\textbf{Замечание~2.} В~соотношениях~(\ref{e5-kor}), (\ref{e10-kor}) и~(\ref{e11-kor}) предельные
функции распределения непрерывны, поэтому в этих соотношениях
сходимость по распределению эквивалентна равномерной схо\-ди\-мости
функций распределения.

\section{Оценки скорости сходимости распределений случайных сумм к~обобщенным дисперсионным 
гамма-распределениям}

В данном разделе будет использоваться специальная и довольно
естественная конструкция случайных блужданий, удовлетворяющая
комплексу условий, указанному в замечании~1.

Пусть $\xi_1,\xi_2,\ldots$~--- независимые одинаково распределенные
случайные величины с ${\sf E}\xi_1\hm=0$, ${\sf D}\xi_1\hm=1$,
$\beta^3\hm={\sf E}|\xi_1|^3\hm<\infty$, $a\hm\in\r$, $n$~--- натуральное
число.

Положим
\begin{equation}
X_{n,j}=\fr{\xi_j}{\sqrt{n}}+\fr \alpha n\,.\label{e12-kor}
\end{equation}
В терминах случайных блужданий случайные величины $X_{n,j}$,
определенные соотношением~(\ref{e12-kor}), могут быть интерпретированы как
элементарные приращения процесса, при этом рассматриваемая их
конструкция~(\ref{e12-kor}) предполагает {\it одинаковый порядок малости}
элементарных трендов и {\it дисперсий}, что характерно, например,
для приращений винеровского процесса со сносом. Обозначим
$$
S_n=\sum\limits_{j=1}^n X_{n,j}\
\left(=\fr{1}{\sqrt{n}}\sum\limits_{j=1}^n\xi_j+\alpha\right)\,.
$$
В~силу классической центральной предельной теоремы имеем:
$$
\lim\limits_{n\to\infty}\sup\limits_{x\in\r}\big|\,{\sf
P}(S_n<x)-\Phi(x-\alpha)\big|=0\,,
$$
т.\,е.\ так определенные случайные величины $X_{n,j}$ удовлетворяют
условию~(\ref{e5-kor}) с $k_n\hm=n$.

В книге~\cite{GnedenkoKorolev1996} и статьях~\cite{Korolev1997, Korolev2000} 
предложено моделировать эволюцию неоднородных
хаотических стохастических процессов, в частности динамику цен
финансовых активов, с помощью обобщенных дважды стохастических
пуассоновских процессов (обобщенных процессов Кокса). Этот подход
получил дополнительное обоснование и развитие в~[3, 29, 35, 36]. 
В~работах~\cite{Korolev2011, KorolevSkvortsova2006} этот подход
успешно применен к моделированию процессов плазменной
турбулентности. В~соответствии с указанным подходом поток
информативных событий, в результате каждого из которых появляется
очередное <<наблюденное>> значение рассматриваемой характеристики,
описывается с помощью точечного случайного процесса вида
$M(\Lambda(t))$, где $M(t)$, $t\geq0$,~--- однородный пуассоновский
процесс с единичной интенсивностью, а $\Lambda(t)$, $t\hm\geq0$,~---
независимый от $M(t)$ случайный процесс, обладающий следующими
свойствами: $\Lambda(0)\hm=0$, ${\sf P}(\Lambda(t)\hm<\infty)\hm=1$ для
любого $t\hm>0$, траектории $\Lambda(t)$ не убывают и непрерывны
справа. Процесс $M(\Lambda(t))$, $t\hm\geq0$, называется дважды
стохастическим пуассоновским процессом (процессом Кокса). 
В~частности, если процесс $\Lambda(t)$ допускает представление
$$
\Lambda(t)=\int\limits_{0}^{t}\lambda(\tau)\,d\tau\,,\quad t\geqslant0\,,
$$
в котором $\lambda(t)$~--- положительный случайный процесс с
интегрируемыми траекториями, то $\lambda(t)$ можно интерпретировать
как мгновенную стохастическую интенсивность процесса Кокса.

В соответствии с такой моделью в каждый момент времени~$t$
распределение случайной величины $M(\Lambda(t))$ является смешанным
пуассонов-\linebreak ским. С~практической точки зрения для описания\linebreak
статистических закономерностей поведения интенсивности потока
информативных событий удобно использо\-вать такую гибкую модель, как
обобщенное гам\-ма-рас\-пре\-де\-ле\-ние. Для большей наглядности рассмотрим
случай, когда в рас\-смат\-ри\-ва\-емой модели время~$t$ остается
фиксированным, а $\Lambda(t)\hm=nU_{\nu,\kappa,\delta}$, где $n$~---
вспомогательный параметр, $U_{\nu,\kappa,\delta}$~--- случайная
величина, имеющая обобщенное гам\-ма-рас\-пре\-де\-ле\-ние
$F(x;\nu,\kappa,\delta)$ и независимая от стандартного
пуассоновского процесса $M(t)$, $t\hm\geqslant0$. При этом асимптотика
$n\hm\to\infty$ может интерпретироваться как то, что (случайная)
интенсивность потока информативных событий считается очень большой,
а при использовании подобных\linebreak
 моделей в задачах финансовой математики
рас\-пределение случайной величины $U_{\nu,\kappa,\delta}$ довольно\linebreak
естественно отождествляется со статистическими закономерностями
поведения (случайной) волатильности. Для каждого натурального~$n$
положим
$$
N_n=M\left(nU_{\nu,\kappa,\delta}\right)\,.
$$
Очевидно, что так определенная случайная величина $N_n$ имеет
смешанное пуассоновское распределение:
\begin{multline}
{\sf P}(N_n=k)={\sf
P}\left(M(nU_{\nu,\kappa,\delta})=k\right)={}\\
{}=
\int\limits_0^{\infty}e^{-nz}\fr{(nz)^k}{k!}\,f(z;\nu,\kappa,\delta)\,dz\enskip
k=0,1,\ldots\label{e13-kor}
\end{multline}

Обозначим $A_n(z)\equiv A_n(z;\nu,\kappa,\delta)\hm={\sf P}(N_n\hm<nz)$,
$z\hm\geqslant0$ ($A_n(z)\hm=0$ при $z\hm<0$). Несложно видеть, что
$$
A_n(z)\Longrightarrow F(x;\nu,\kappa,\delta)\enskip (n\to\infty)\,.
$$
Действительно, как известно, если $\Pi(x;\ell)$~--- функция
распределения Пуассона с параметром $\ell\hm>0$ и $E(x;c)$~--- функция
распределения с единственным единичным скачком в точке $c\hm\in\r$, то
$$
\Pi(\ell x;\ell)\Longrightarrow E(x;1)\enskip (\ell\to\infty)\,.
$$
Так как для $x\hm\in\r$
$$
A_n(x)=\int\limits_{0}^{\infty}\Pi(n x; n z)\,dF(z;\nu,\kappa,\delta)\,,
$$
то по теореме Лебега о мажорируемой сходимости при $n\hm\to\infty$

\pagebreak

\noindent
\begin{multline*}
A_n(x)\Longrightarrow\int\limits_{0}^{\infty}E(xz^{-1};1)\,dF(z;\nu,\kappa,\delta)={}\\
{}=
\int\limits_{0}^{x}dF(z;\nu,\kappa,\delta)=F(x;\nu,\kappa,\delta)\,,
\end{multline*}
т.\,е.\ так определенные случайные величины $N_n$ удовле\-тво\-ря\-ют
условию~(\ref{e11-kor}) с $k_n\hm=n$. Впредь будем считать, что при каждом $n\hm\geqslant1$
случайная величина $N_n$ независима от последовательности
$\{\xi_j\}_{j\geqslant1}$, что гарантирует независимость случайных величин
$N_n,X_{n,1},X_{n,2},\ldots$

Таким образом, в силу непрерывности функции обобщенного
дисперсионного гам\-ма-рас\-пре\-де\-ле\-ния
$W(x;\alpha,\sigma,\nu,\kappa,\delta)$ из следствия~1 вытекает, что

\noindent
\begin{multline*}
D_n\equiv{}\\
{}\equiv \sup\limits_{x\in\r}\left\vert
{\sf P}\left(\sum\limits_{j=1}^{N_n}X_{n,j}<x\right)-
W(x;\alpha,\sigma,\nu,\kappa,\delta)
\right\vert\longrightarrow{}\\
{}\longrightarrow 0\quad (n\to\infty)
\end{multline*}
(см.\ замечание~2).

Скорость стремления $D_n$ к нулю описывается следующим утверждением.

\smallskip

\noindent
\textbf{Теорема~2.} \textit{Для любого $n\hm\geqslant1$ справедлива оценка}
$$
D_n\leqslant 0{,}4532\fr{\beta^3}{\sqrt{n}}\,{\sf E}U_{\nu,\kappa,\delta}^{-1/2} +
0{,}1210\fr{\alpha^2}{n}\,.
$$

\smallskip

\noindent
Д\,о\,к\,а\,з\,а\,т\,е\,л\,ь\,с\,т\,в\,о\,.\ \  Как уже было показано, распределение
случайной величины $N_n$ является смешанным пуассоновским (см.~(\ref{e13-kor})). 
Следовательно, по теореме Фубини

\noindent
\begin{multline*}
{\sf P}\left(\sum\limits_{j=1}^{N_n}X_{n,j}<x\right)=
{\sf P}\left(\sum\limits_{j=1}^{M(nU_{\nu,\kappa,\delta})}X_{n,j}<x\right)={}\\
{}=
\int\limits_0^{\infty}{\sf P}\left(\sum\limits_{j=1}^{M(nz)}X_{n,j}<x\right)f(z;\nu,\kappa,\delta)\,dz\,.
\end{multline*}
При этом

\noindent
$$
{\sf E}X_{n,j}=\fr{\alpha}{n}\,;\enskip {\sf D}X_{n,j}=\fr{1}{n}\,;\enskip
{\sf E}|X_{n,j}-{\sf E}X_{n,j}|^3=\fr{\beta^3}{n^{3/2}}\,.
$$
Таким образом, при каждом $z\hm\in(0,\infty)$

\noindent
\begin{align*}
{\sf E}\sum\limits_{j=1}^{M(nz)}X_{n,j}&=\alpha z\,;
\\
{\sf D}\sum\limits_{j=1}^{M(nz)}X_{n,j}&=
nz\left(\fr{\alpha^2}{n^2}+\fr{1}{n}\right)=z\left(1+\fr{\alpha^2}{n}\right)\,.
\end{align*}
Из~(\ref{e2-kor}) вытекает, что
\begin{multline}
D_n=\sup\limits_x\Bigg|\int\limits_{0}^{\infty}f(z;\nu,\kappa,\delta)\Bigg[{\sf
P}\left(\sum\limits_{j=1}^{M(nz)}X_{n,j}<x\right)-{}\\
{}-
\Phi\Bigg(\fr{x-\alpha
z}{\sqrt{z(1+{\alpha^2}/{n})}}\Bigg)+
\Phi\Bigg(\fr{x-\alpha z}{\sqrt{z(1+{\alpha^2}/{n})}}\Bigg)-{}\\
{}-
\Phi\Bigg(\fr{x-\alpha z}{\sqrt{z}}\Bigg)\Bigg]dz\Bigg| \leqslant I_1+I_2\,,
\label{e14-kor}
\end{multline}
где
\begin{align*}
I_1&=\int\limits_{0}^{\infty}f(z;\nu,\kappa,\delta)\sup_x\Bigg|\,{\sf P}
\left(\sum\limits_{j=1}^{M(nz)}X_{n,j}<x\right)-{}\\
&\hspace*{25mm}{}-
\Phi\Bigg(\fr{x-\alpha z}{\sqrt{z(1+{\alpha^2}/{n})}}\Bigg)\Bigg|\,dz\,;
\\
I_2&=\int\limits_{0}^{\infty}f(z;\nu,\kappa,\delta)
\sup\limits_x\Bigg|\,\Phi\Bigg(\fr{x-\alpha z}{\sqrt{z(1+{\alpha^2}/{n})}}\Bigg)-{}\\
&\hspace*{25mm}{}-
\Phi\Bigg(\fr{x-az}{\sqrt{z}}\Bigg)\Bigg|\,dz\,.
\end{align*}
В дальнейшем понадобится следующее утверждение.

\smallskip

\noindent
\textbf{Лемма~3.} \textit{Пусть случайные величины $X_1,X_2,\ldots$
одинаково распределены. Пусть $N_{\ell}$~--- пуассоновская случайная
величина с параметром $\ell\hm>0$. Предположим, что случайные величины
$N_{\ell},X_1,X_2,\ldots$ независимы в совокупности. Обозначим
$$
S_{\ell}=X_1+\cdots+X_{N_{\ell}}\,.
$$
Тогда}
\begin{multline*}
\sup\limits_x\bigg|\,{\sf P}(S_{\ell}<x)-\Phi\left(\fr{x-{\sf
E}S_{\ell}}{\sqrt{{\sf
D}S_{\ell}}}\right)\bigg|\leqslant{}\\
{}\leqslant\fr{0{,}4532}{\sqrt{\ell}}\,\fr{{\sf E}
\left\vert X_1-{\sf E}X_1\right\vert^3}{({\sf D}X_1)^{3/2}}\,.
\end{multline*}

\noindent
Д\,о\,к\,а\,з\,а\,т\,е\,л\,ь\,с\,т\,в\,о\ \  леммы~3 приведено в 
работе~\cite{KorolevShevtsovaShorgin2011} (см.\ так\-же~[3,  с.~144]).

\smallskip

Продолжим доказательство теоремы~2. Рас\-смот\-рим $I_1$. Применяя лемму~3, получаем
\begin{multline}
I_1\leqslant 0{,}4532\fr{\beta^3}{\sqrt{n}}\int\limits_{0}^{\infty}
\fr{f(z;\nu,\kappa,\delta)}{\sqrt{z}}\,dz={}\\
{}=0{,}4532\fr{\beta^3}{\sqrt{n}}\,{\sf E}U_{\nu,\kappa,\delta}^{-1/2}\,.\label{e15-kor}
\end{multline}

Рассмотрим $I_2$. Сформулируем еще одно вспомогательное утверждение.

\smallskip

\noindent
\textbf{Лемма~4.} \textit{Пусть $b\hm\in\r$, $0\hm<c\hm<\infty$, $0\hm<d\hm<\infty$.
Тогда}
\begin{equation}
\sup\limits_y|\Phi(y)-\Phi(cy)|\leqslant\fr{1}{\sqrt{2\pi e}}
\left\vert\max\left\{c,\,\fr{1}{c}\right\}-1\right\vert\,;\label{e16-kor}
\end{equation}
\begin{equation}
\sqrt{1+d}-1\leqslant\fr{d}{2}\,.\label{e17-kor}
\end{equation}

\smallskip

\noindent
Элементарное д\,о\,к\,а\,з\,а\,т\,е\,л\,ь\,с\,т\,в\,о\ неравенств~(\ref{e16-kor}) и~(\ref{e17-kor})
можно получить, например, с помощью формулы Лагранжа.

\smallskip

Продолжим доказательство теоремы~2. В~лемме~4 положим
$$
y=\fr{x-\alpha z}{\sqrt{z(1+{\alpha^2}/{n})}}\,,\enskip
c=\sqrt{1+\fr{\alpha^2}{n}}\,.
$$
Тогда $c\hm\geqslant1$ и в силу утверждения~(\ref{e16-kor}) леммы~4 имеем:
$$
I_2\leqslant\fr{1}{\sqrt{2\pi e}}\left(\sqrt{1+\fr{\alpha^2}{n}}-1\right)\,.
$$
При этом в силу утверждения~(\ref{e17-kor}) леммы~4
$$
\sqrt{1+\fr{\alpha^2}{n}}-1\leqslant\fr{\alpha^2}{2n}\,.
$$
Окончательно получаем:
\begin{equation}
I_2\leqslant\fr{\alpha^2}{2\sqrt{2\pi e}n}\,.\label{e18-kor}
\end{equation}
Подставляя~(\ref{e15-kor}) и~(\ref{e18-kor}) в~(\ref{e14-kor}), получаем утверждение теоремы. 
Теорема доказана.

\smallskip

Для симметричных обобщенных дисперсионных гам\-ма-рас\-пре\-де\-ле\-ний (т.\,е.\ 
$\alpha\hm=0$) оценку, представленную в теореме~2, можно уточнить
не только за счет того, что в таком случае обнуляется второе
слагаемое в правой части, но и за счет уменьшения коэффициента при
первом слагаемом. Аналогом леммы~3 здесь может служить следующее
утверждение.

\smallskip

\noindent
\textbf{Лемма~5.} \textit{В~условиях леммы~3 справедливо неравенство}:
$$
\sup\limits_x\left\vert\,{\sf P}(S_{\ell}<x)-\Phi\left(\!\fr{x-{\sf E}S_{\ell}}
{\sqrt{{\sf D}S_{\ell}}}\!\right)\right\vert\leqslant\fr{0{,}3041}{\sqrt{\ell}}\,
\fr{{\sf E}|X_1|^3}{({\sf E}X_1^2)^{3/2}}.
$$

\smallskip

\noindent
Д\,о\,к\,а\,з\,а\,т\,е\,л\,ь\,с\,т\,в\,о\ \  этого утверждения приведено 
в~\cite{KorolevShevtsova2010} (см.\ также~[3, теорема~2.4.3]).

\smallskip

Если в случае $\alpha\hm=0$ в доказательстве теоремы~2 вместо леммы~3
воспользоваться леммой~5, то в результате получится следующее
утверждение.

\smallskip

\noindent
\textbf{Следствие~2.} \textit{Пусть в дополнение к условиям теоремы~$2$
$\alpha\hm=0$. Тогда для любого $n\hm\geqslant1$ справедлива оценка}
$$
D_n\leqslant0{,}3041\fr{\beta^3}{\sqrt{n}}\,{\sf E}U_{\nu,\kappa,\delta}^{-1/2}\,.
$$


{\small\frenchspacing
{%\baselineskip=10.8pt
\addcontentsline{toc}{section}{Литература}
\begin{thebibliography}{99}


\bibitem{Seshadri1997}
\Au{Seshadri V.} Halphen's laws~//
Encyclopedia of Statistical Sciences, Update Vol.~1.~/ 
Eds. S.~Kotz, C.\,B.~Read, D.\,L.~Banks.~---
New York: Wiley, 1997. P.~302--306.

\bibitem{Sichel1973}
\Au{Sichel H.\,S.} Statistical evaluation of diamondiferous deposits~// J.~South Afr. Inst. 
Min. Metall., 1973. Vol.~76. P.~235--243.

\bibitem{KorolevBeningShorgin2011}
\Au{Королев В.\,Ю., Бенинг В.\,Е., Шоргин~С.\,Я.} Математические
основы теории риска.~--- 2-е изд., перераб. и дополн.~--- М.: Физматлит,
2011. 620~с.

\bibitem{BN1977}
\Au{Barndorff-Nielsen O.\,E.} Exponentially decreasing distributions
for the logarithm of particle size~// Proc. R. Soc. L.
Ser.~A, 1977. Vol.~A(353). P.~401--419.

\bibitem{BN1978}
\Au{Barndorff-Nielsen O.\,E.} Hyperbolic distributions and
distributions of hyperbolae~// Scand. J.~Statist., 1978. Vol.~5. P.~151--157.

\bibitem{BN1979}
\Au{Barndorff-Nielsen O.\,E.} Models for non-Gaussian variation,
with applications to turbulence~// Proc. R. Soc. L. Ser.~A,
1979. Vol.~A(368). P.~501--520.

\bibitem{MadanSeneta1990}
\Au{Madan D.\,B., Seneta~E.} The variance gamma $($V.G.$)$ model for
share market return~// J.~Business, 1990. Vol.~63. P.~511--524.

\bibitem{EberleinKeller1995}
\Au{Eberlein E., Keller U.} Hyperbolic distributions in finance~//
Bernoulli, 1995. Vol.~1. No.\,3. P.~281--299.

\bibitem{Prause1997}
\Au{Prause K.} Modeling financial data using generalized hyperbolic
distri\-butions.~--- Freiburg: Universit$\ddot{\mbox{a}}$t Freiburg, Institut
f$\ddot{\mbox{u}}$r Mathematische Stochastic, 1997. Preprint No.\,48.

\bibitem{CarrMadanChang1998}
\Au{Carr P.\,P., Madan D.\,B., Chang~E.\,C.} The Variance Gamma
process and option pricing~// Eur. Finance Rev., 1998. Vol.~2. P.~79--105.

\bibitem{EberleinKellerPrause1998}
\Au{Eberlein E., Keller U., Prause~K.} New insights into smile,
mispricing and value at risk: The hyperbolic model~// J.~Business, 1998. Vol.~71. P.~371--405.

\bibitem{BarndorffNielsen1998}
\Au{Barndorff-Nielsen O.\,E.} Processes of normal inverse Gaussian
type~// Finance and Stochastics, 1998. Vol. 2. P. 41--18.

\bibitem{EberleinPrause1998}
\Au{Eberlein E., Prause K.} The generalized hyperbolic model:
Financial derivatives and risk measures.~--- Freiburg:
Universit$\ddot{\mbox{a}}$t Freiburg, Institut f$\ddot{\mbox{u}}$r Mathematische Stochastic,
1998. Preprint No.\,56.

\bibitem{Shiryaev1998}
\Au{Ширяев А.\,Н.} Основы стохастической финансовой математики. Т.~1. 
Факты. Модели.~--- М.: Фазис, 1998.

\bibitem{Eberlein1999}
\Au{Eberlein E.} Application of generalized hyperbolic L$\acute{\mbox{e}}$vy
motions to finance.~--- Freiburg: Universit$\ddot{\mbox{a}}$t Freiburg, Institut
f$\ddot{\mbox{u}}$r Mathematische Stochastic, 1999. Preprint No.\,64.

\bibitem{BNBlaesildSchmiegel2004}
\Au{Barndorff-Nielsen O.\,E., Bl{$\mbox{\ae}$}sild~P., Schmiegel~J.} 
A~parsimonious and universal description of turbulent velocity
increments~// Eur. Phys.~J., 2004. Vol.~B. 41.
P.~345--363.

\bibitem{LeCam1990}
\Au{LeCam L.} Maximum likelihood; an introduction~// Intern.
Stat. Rev., 1990. Vol.~58. P.~153--171.

\bibitem{GnedenkoKolmogorov1949}
\Au{Гнеденко Б.\,В., Колмогоров А.\,Н.} Предельные распределения для
сумм независимых случайных величин.~--- М.--Л.: ГИТТЛ, 1949.

\bibitem{BN1982}
\Au{Barndorff-Nielsen O.\,E., Kent~J., S\!\!\!{\ptb{\o}}\,rensen M.} Normal
variance-mean mixtures and $z$-distributions // Intern.
Stat. Rev., 1982. Vol.~50. No.\,2. P.~145--159.

\bibitem{Renyi1960}
{\it R$\acute{\mbox{e}}$nyi~A.} On the central limit theorem for the sum of a
random number of independent random variables~// Acta Math. Acad.
Sci. Hung., 1960. Vol.~11. P.~97--102.

\bibitem{GnedenkoFahim1969}
\Au{Гнеденко Б.\,В., Фахим Х.} Об одной теореме переноса~// 
Докл.\ АН СССР, 1969. Т.~187. №\,1. С.~15--17.

\bibitem{KorolevSokolov2012}
\Au{Королев В.\,Ю., Соколов И.\,А.} Скошенные распределения
Стьюдента, дисперсионные гам\-ма-рас\-пре\-де\-ле\-ния и их обобщения как
асимптотические аппроксимации~// Информатика и её применения, 2012.
Т.~6. Вып.~1. С.~2--10.

\bibitem{Korolev2012}
\Au{Королев В.\,Ю.} О~взаимосвязи обобщенного распределения
Стьюдента и дисперсионного гам\-ма-рас\-пре\-де\-ле\-ния при статистическом
анализе выборок случайного объема~// Докл.\ РАН, 2012 (в печати).

\bibitem{Korolev2012b}
\Au{Королев В.\,Ю.} Обобщенные гиперболические распределения как
предельные для случайных сумм~// Теория вероятностей и ее
применения, 2013. Т.~58. Вып.~1.

\bibitem{KorolevShorgin2011} 
\Au{Королев В.\,Ю., Шоргин~С.\,Я.}
Математические методы анализа стохастической структуры
информационных потоков.~--- М.: ИПИ РАН, 2011. 130~с.

\bibitem{Stacy1962} 
\Au{Stacy  E.\,W.} A~generalization of the gamma
distribution~// Annals Math. Statistics, 1962. Vol.~33. P.~1187--1192.

\bibitem{Gumbel1965} 
\Au{Гумбель Э.} Статистика экстремальных значений.~--- М.: Мир, 1965.

\bibitem{Kalashnikov1997} %28
\Au{Kalashnikov V.\,V.} Geometric sums: Bounds for rare events with
applications.~--- Dordrecht: Kluwer Academic Publs., 1997.

\bibitem{KorolevSokolov2008} %29
\Au{Королев В.\,Ю., Соколов И.\,А.} Математические модели
неоднородных потоков экстремальных событий.~--- М.: ТОРУС ПРЕСС, 2008.

\bibitem{Korolev2009} %30
\Au{Королев В.\,Ю.} О~распределении размеров частиц при
дроблении~// Информатика и её применения, 2009. Т.~3. Вып.~3. С.~60--68.

\bibitem{GnedenkoKorolev1996} %31
\Au{Gnedenko B.\,V., Korolev~V.\,Yu.} Random summation: Limit
theorems and applications.~--- Boca Raton: CRC Press, 1996.

\bibitem{Teicher1961} %32
\Au{Teicher H.} Identifiability of mixtures~// Ann. Math. Stat.,
1961. Vol.~32. P.~244--248.

\bibitem{Korolev1997} %33
\Au{Королев В.\,Ю.} Постpоение моделей pаспpеделений биpжевых цен
пpи помощи методов асимптотической теоpии случайного суммиpования~// 
Обозpение пpомышленной и пpикладной математики. Сеp. Финансовая и стpаховая математика, 1997. 
Т.~4. Вып.~1. С.~86--102.

\bibitem{Korolev2000} %34
\Au{Королев В.\,Ю.} Асимптотические свойства экстpемумов обобщенных
пpоцессов Кокса и их пpименение к некотоpым задачам финансовой
математики~// Теоpия веpоятностей и ее пpименения, 2000. Т.~45. Вып.~1. С.~182--194.

\bibitem{BeningKorolev2002} %35
\Au{Bening V., Korolev V.} Generalized Poisson models and their
applications in insurance and finance.~--- Utrecht: VSP, 2002.


\bibitem{Korolev2011}
\Au{Королев В.\,Ю.} Веро\-ят\-но\-ст\-но-ста\-ти\-сти\-че\-ские методы декомпозиции
волатильности хаотических процессов.~--- М.: Изд-во Московского ун-та, 2011.

\bibitem{KorolevSkvortsova2006}
Stochastic models of structural plasma turbulence~/
Eds. V.~Korolev, N.~Skvortsova.~--- Utrecht: VSP, 2006.

\bibitem{KorolevShevtsovaShorgin2011}
\Au{Королев В.\,Ю., Шевцова И.\,Г., Шоргин~С.\,Я.} О~неравенствах типа
Бер\-ри--Эс\-се\-ена для пуассоновских случайных сумм~// Информатика и её
применения, 2011. Т.~5. Вып.~3. С.~64--66.

\label{end\stat}

\bibitem{KorolevShevtsova2010}
\Au{Korolev V., Shevtsova I.} An improvement of the Berry--Esseen
inequality with applications to Poisson and mixed Poisson random
sums~// Scandinavian Actuarial J., 2012. No.\,2. P.~81--105.
DOI:10.1080/03461238.2010.485370.
\end{thebibliography}
}
}

\end{multicols}