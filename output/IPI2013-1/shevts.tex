\def\stat{shevtsova}

\def\tit{ОБ АБСОЛЮТНЫХ КОНСТАНТАХ В НЕРАВЕНСТВЕ БЕРРИ--ЭССЕЕНА И ЕГО
СТРУКТУРНЫХ И~НЕРАВНОМЕРНЫХ~УТОЧНЕНИЯХ$^*$}

\def\titkol{Об абсолютных константах в неравенстве Берри--Эссеена и его
структурных и неравномерных уточнениях}

\def\autkol{И.\,Г.~Шевцова}

\def\aut{И.\,Г.~Шевцова$^1$}

\titel{\tit}{\aut}{\autkol}{\titkol}

{\renewcommand{\thefootnote}{\fnsymbol{footnote}}\footnotetext[1]
{Работа поддержана
грантом МК--2256.2012.1, а также Российским фондом фундаментальных
исследований (проекты 12-01-31125-а, 11-01-00515а, 11-07-00112а,
12-07-00115а).}}

\renewcommand{\thefootnote}{\arabic{footnote}}
\footnotetext[1]{Факультет вычислительной математики и
кибернетики Московского государственного университета им.\
М.\,В.\,Ломоносова; Институт проблем информатики Российской академии
наук, ishevtsova@cs.msu.su}


\Abst{Для равномерного расстояния $\Delta_n$ между
функцией распределения (ф.р.)\ стандартного нормального закона и
ф.р.\ нормированной суммы~$n$ независимых случайных величин (с.в.)\
$X_1,\ldots,X_n$ с $\e X_j=0$, $\e X_j^2=\sigma_j^2$,
${j=1,\ldots,n}$, при всех $n\ge1$ приведены оценки
$$
\ud\le \min\{0{,}5583 \ell_n,\, 0{,}3723(\ell_n+0{,}5\tau_n),
\,0{,}3057(\ell_n+\tau_n)\},
$$
$$
\ud\le \min\{0{,}4690\ell_n,\, 0{,}3322(\ell_n+0{,}429\tau_n),
\,0{,}3031(\ell_n+0{,}646\tau_n)\}, \text{ если } X_1\eqd\cdots\eqd X_n,
$$
где $\ell_n\hm=\sum\limits_{j=1}^n\e|X_j|^3$, $\tau_n\hm=\sum\limits_{j=1}^n\sigma_j^3$,
$\sum\limits_{j=1}^n\sigma_j^2\hm=1$. Получены уточненные результаты для
случая симметричного распределения слагаемых. Также показано, что в
неравенстве На\-га\-ева--Би\-кя\-ли\-са (неравномерном аналоге неравенства
Бер\-ри--Эс\-се\-ена) абсолютная константа не превосходит 21,82 в общем
случае и 17,36 в случае одинаково распределенных слагаемых.}

\vspace*{3pt}

\KW{центральная предельная теорема; оценка
скорости сходимости; нормальная аппроксимация; неравенство
Бер\-ри--Эс\-се\-ена; неравенство На\-га\-ева--Би\-кя\-ли\-са; абсолютная константа}

\vspace*{4pt}

\vskip 14pt plus 9pt minus 6pt

      \thispagestyle{headings}

      \begin{multicols}{2}

            \label{st\stat}


Обозначим $\F_3$ множество всех ф.р.~$F$
с.в.~$X$ с ${\e X\hm=0}$ и ${\e|X|^3\hm< \infty}$.
Через $\F_{3,\,s}$ обозначим множество всех симметричных ф.р.\ из
$\F_3$. Пусть $X_1,X_2,\ldots,X_n$~--- независимые с.в.\ с ф.р.\
$F_1,\ldots,F_n\in\F_3$ соответственно. Положим $\sigma_j^2\hm= \e
X_j^2,$ $\beta_{3,\,j}\hm=\e|X_j|^3,$ $j\hm=1,\ldots,n,$
$\sum\limits_{j=1}^n\sigma_j^2\hm=1$, $\ell_n\hm=\sum\limits_{j=1}^n\beta_{3,\,j}$,
$\tau_n\hm=\sum\limits_{j=1}^n\sigma_j^3$,
$\overline F_n(x)\hm=\p(X_1+\cdots+X_n\hm<xB_n)$. Пусть $\phi(x)$ и
$\Phi(x)$~--- соответственно плотность и ф.р.\ стандартного
нормального закона,
\begin{align*}
\nud&=|\overline F_n(x)-\Phi(x)|\,,\quad x\in\R\,;\\
\ud &=\sup\limits_x|\overline F_n(x)-\Phi(x)|\,,\quad n=1,2,\ldots\,
\end{align*}

В работе~\cite{Shevtsova2012ISSPSM3} с использованием техники
преобразования смещения формы и новой оценки точности аппроксимации
характеристической функции первыми членами ее разложения в ряд
Тейлора для всех $n\hm\ge1$ были получены оценки:
\begin{multline*}
\ud\le \min\{0{,}5584\ell_n, 0{,}36266(\ell_n+0{,}54\tau_n),\\
0{,}3129(\ell_n+0{,}922\tau_n)\}\,,\quad F_1,\ldots,F_n\in\F_3\,;
\end{multline*}


\noindent
\begin{multline*}
\ud\le \min\{ 0{,}4693\ell_n, 0{,}3322(\ell_n+0{,}429\tau_n),\\
0{,}3031(\ell_n+0{,}646\tau_n)\}\,,\quad  F_1=\cdots=F_n\in\F_3\,;
\end{multline*}

\vspace*{-9pt}

\noindent
\begin{multline*}
\ud\le \min\{ 0{,}3425(\ell_n+0{,}63\tau_n),\\
0{,}2895(\ell_n+\tau_n) \}\,,\quad  F_1,\ldots,F_n\in\F_{3,\,s}\,;
\end{multline*}

\vspace*{-9pt}

\noindent
\begin{multline*}
\ud\le \min\{ 0{,}29489(\ell_n+0{,}587\tau_n)\,,\\
0{,}2730(\ell_n+0{,}732\tau_n)\}\,, \quad F_1=\cdots=F_n\in\F_{3,\,s}\,.
\end{multline*}

С помощью алгоритма, использованного в~\cite{Shevtsova2012ISSPSM3},
и оценки
$$
|f'(t)|\le \sigma\sin\left(\sigma|t| \wedge\fr{\pi}{2}\right)\,,\quad t\in\R\,,
$$
для производной характеристической функции $f(t)\hm=\e e^{itX}$ с $\e
X\hm=0$, $\sigma^2\hm\equiv\e X^2\hm<\infty$, вытекающей из результатов
работ~\cite{Rossberg1991, MatysiakSzablowski2001}, указанные
результаты можно уточнить и получить следующие оценки, справедливые
при всех $n\hm\ge1$:
\begin{multline*}
\ud\le \min\{0{,}5583\ell_n,\, 0{,}3723(\ell_n+0{,}5\tau_n),\\
\,0{,}3057(\ell_n+\tau_n)\}\,,\quad F_1,\ldots,F_n\in\F_3\,;
\end{multline*}

\vspace*{-12pt}

\noindent
\begin{multline*}
\ud\le \min\{0{,}4690\ell_n,\, 0{,}3322(\ell_n+0{,}429\tau_n),\\
\,0{,}3031(\ell_n+0{,}646\tau_n)\}\,,\quad F_1=\cdots=F_n\in\F_3\,;
\end{multline*}

\vspace*{-12pt}

\noindent
\begin{multline*}
\ud\le \min\{ 0{,}34245(\ell_n+0{,}63\tau_n),\\
\,0{,}2873(\ell_n+\tau_n)\}\,,\quad F_1,\ldots,F_n\in\F_{3,\,s}\,;
\end{multline*}

\vspace*{-12pt}

\noindent
\begin{multline*}
\ud\le \min\{ 0{,}29353(\ell_n+0{,}593\tau_n),\\
\,0{,}2730(\ell_n+0{,}729\tau_n)\}\,,\quad F_1=\cdots=F_n\in\F_{3,\,s}\,.
\end{multline*}

Кроме того, исправляя неточность, допущенную
в~\cite{NefedovaShevtsova2012}, и используя приведенные выше
неравенства, можно показать, что справедливы следующие неравномерные
оценки:
\begin{multline*}
\sup\limits_{x\in\R}(1+|x|^3)\nud\le \min\{21{,}82\ell_n,\,
18{,}19(\ell_n+\tau_n)\}\,,\\ F_1,\ldots,F_n\in\F_3\,;
\end{multline*}

\vspace*{-12pt}

\noindent
\begin{multline*}
\sup\limits_{x\in\R}(1+|x|^3)\nud\le \min\{ 17{,}36\ell_n,\\
15{,}70(\ell_n+0{,}646\tau_n)\}\,,\quad F_1=\cdots=F_n\in\F_3\,.
\end{multline*}
Более того, исправленный метод работы~\cite{NefedovaShevtsova2012}
позволяет построить монотонно убывающую функцию $C(t)$ с
$\lim\limits_{t\to\infty}C(t)\hm=1\hm+e\hm=3{,}71\ldots$ и такую, что
\begin{multline*}
\sup\limits_{|x|\ge t}|x|^3\nud\le C(t)\ell_n,\quad t\ge0,\quad n\ge1,\\
F_1,\ldots,F_n\in\F_3\,.
\end{multline*}
В частности, для $C(t)$ справедливы верхние оценки
\\
$C(0)\le21{,}26$, $C(4)\le17{,}19$, $C(5)\le12{,}35$, $C(10)\hm\le7{,}36$ для
любых $F_1,\ldots,F_n\in\F_3$;
\\
$C(0)\le16{,}90$, $C(4)\le14{,}58$, $C(5)\le11{,}56$, $C(10)\hm\le5{,}85$, если
$F_1\hm=\cdots=F_n\hm\in\F_3$.

С помощью методов, использованных в данной работе, можно получить
аналогичные равномерные и неравномерные оценки для случая, когда
слагаемые обладают моментами порядка лишь $2+\delta$ с некоторым
$0<\delta\le1$.

{\small\frenchspacing
{%\baselineskip=10.8pt
\addcontentsline{toc}{section}{Литература}
\begin{thebibliography}{9}

\bibitem{Shevtsova2012ISSPSM3}
\Au{Shevtsova I.} On the absolute constants in the
Berry--Esseen-type inequalities~// 30th  Seminar (International) on
Stability Problems for Stochastic Models (Svetlogorsk, 2012): Book
of Abstracts.~--- М.: ИПИ РАН, 2012. С.~71--72.



\bibitem{Rossberg1991}
\Au{Ro\!\!\!{\ptb{\ss}}\,berg~H.-J.} Positiv definite Verteilungsdichten~//
Appendix to:  \Au{Gnedenko B.\,W.} Einf$\ddot{\mbox{u}}$hrung in die
Wahrscheinlichkeitstheorie.~--- 9th ed.~--- Berlin:
Akademie--Verlag, 1991.

\bibitem{MatysiakSzablowski2001}
\Au{Matysiak~W., Szab\!\!{\ptb{\l}}owski~P.~J.} Some inequalities for
characteristic functions~// J.~Math. Sci., 2001. Vol.~105. No.\,6.
P.~2594--2598.

\label{end\stat}

\bibitem{NefedovaShevtsova2012}
\Au{Нефедова Ю.\,С., Шевцова И.\,Г.} О~неравномерных оценках
скорости сходимости в центральной предельной теореме~// Теория
вероятн. и ее примен., 2012, Т.~57. Вып.~1. С.~62--97.
\end{thebibliography}
}
}

\end{multicols}