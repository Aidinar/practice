\def\ld{\ldots}
\def\d{\overline d}
\def\oa{\overline\alpha}

\def\stat{milov}

\def\tit{СТАЦИОНАРНЫЕ ХАРАКТЕРИСТИКИ СИСТЕМЫ ОБСЛУЖИВАНИЯ С~ИНВЕРСИОННЫМ
ПОРЯДКОМ ОБСЛУЖИВАНИЯ, ВЕРОЯТНОСТНЫМ ПРИОРИТЕТОМ И~ГИСТЕРЕЗИСНОЙ
ПОЛИТИКОЙ$^*$}

\def\titkol{Стационарные характеристики системы обслуживания с инверсионным
порядком обслуживания} %, вероятностным приоритетом и гистерезисной политикой}

\def\autkol{Т.\,А.~Милованова, А.\,В.~Печинкин}

\def\aut{Т.\,А.~Милованова$^1$, А.\,В.~Печинкин$^2$}

\titel{\tit}{\aut}{\autkol}{\titkol}

{\renewcommand{\thefootnote}{\fnsymbol{footnote}}\footnotetext[1]
{Работа выполнена при поддержке РФФИ (проекты
№~11-07-00112 и №~12-07-00108).}}

\renewcommand{\thefootnote}{\arabic{footnote}}
\footnotetext[1]{Российский университет дружбы народов, tmilovanova77@mail.ru}
\footnotetext[2]{Институт проблем
информатики Российской академии наук, apechinkin@ipiran.ru}

\vspace*{6pt}

\Abst{Рассматривается однолинейная система массового
обслуживания (СМО) с инверсионным порядком обслуживания,
вероятностным приоритетом и простейшим вариантом
гистерезисной политики.
Найдены основные стационарные показатели функционирования
этой сис\-темы.}

\vspace*{4pt}

\KW{система массового обслуживания; инверсионный порядок
обслуживания; вероятностный приоритет; гистерезисная
политика}

\vspace*{14pt}


\vskip 14pt plus 9pt minus 6pt

      \thispagestyle{headings}

      \begin{multicols}{2}

            \label{st\stat}
            
\section{Введение}

Одним из важнейших направлений исследований в теории
массового обслуживания является изучение СМО с дисциплинами
обслуживания, отличны\-ми от обслуживания заявок в
порядке поступления, поскольку такие дисциплины
часто позволяют практически без каких-либо
усовершенствований повысить качество функционирования
самых разнообразных технических систем, например
ин\-фор\-ма\-ци\-он\-но-те\-ле\-ком\-му\-ни\-ка\-ци\-он\-ных сис\-тем (ИТС).
В~частности, дисциплиной такого рода является инверсионный
порядок обслуживания с вероятностным приоритетом,
введенный в~\cite{1-m} для решения задачи А.\,Д.~Соловьева
об оптимальных распределениях для некоторых типов
дисциплин обслуживания.
Подробное изложение полученных в этом направлении
результатов можно найти в~\cite{2-m}.

В последнее время значительное внимание уделяется также СМО с
гистерезисным управлением, являющимся одним из возможных механизмов
пред\-от\-вра\-ще\-ния различного рода перегрузок в ИТС (см., например,~\cite{3-m}). 
Разновидности гистерезисной политики используются при
обнаружении перегрузок как в сетях общеканальной системы
сигнализации №\,7, так и в сетях, где основой сигнализации является
протокол инициации сеансов связи.

В настоящей работе делается попытка связать эти два
направления исследования с по\-мощью СМО с инверсионным
порядком обслуживания, вероятностным приоритетом и
простейшим ва\-риантом гистерезисной политики, для
которой\linebreak находятся основные стационарные показатели
функционирования.
Отметим, что некоторые типы системы $M/G/1$ с
простейшим вариантом гистерезисной политики при
дисциплине обслуживания заявок в порядке поступления
изучались в~[4--7].

\section{Описание системы}

Рассмотрим однолинейную СМО с накопителем бесконечной
емкости, инверсионным
порядком обслуживания, вероятностным приоритетом и
простейшим вариантом гистерезисной политики.
Опишем функционирование этой СМО.

\begin{figure*} %fig1
\vspace*{1pt}
 \begin{center}
 \mbox{%
 \epsfxsize=114.642mm
 \epsfbox{mil-1.eps}
 }
% \vspace*{-9pt}
\end{center}
\begin{center}
{\small Схематическое изображение функционирования системы: \textit{1}~--- поступление; 
\textit{2}~---
обслуживание}
 \end{center}
\end{figure*}



Вариант гистерезисной политики заключается в следующем
(см.\ рисунок).
Имеется два порога $n_0$ и $n_1$, причем $n_1\hm<n_0$.
Пока число заявок в системе меньше $n_0$, система
функционирует в режиме~0.
Это означает, что заявки поступают с
интенсивностью $\lambda_0$ и имеют длину, распределенную
по закону $B_0(x)$ с плот\-ностью $b_0(x)\hm=B'_0(x)$
и средним значением
$\beta_0\hm=\int\limits_0^\infty x b_0(x)\, dx\hm<\infty$.
Но как только число заявок в системе становится равным
$n_0$, система переходит в режим~1.
В~этом режиме заявки поступают с интенсивностью~$\lambda_1$ и имеют длину, 
распределенную по закону $B_1(x)$
с плотностью $b_1(x)=B'_1(x)$ и средним значением
$\beta_1\hm=\int\limits_0^\infty x b_1(x)\, dx\hm<\infty$.
Так продолжа-\linebreak\vspace*{-12pt}

\pagebreak

\noindent
ется до тех пор, пока чис\-ло заявок в сис\-те\-ме не
станет равным~$n_1$.
Тогда система снова переходит в режим~0 и~т.\,д.


В~системе также реализован инверсионный порядок
обслуживания с вероятностным приоритетом.
Предполагается, что в любой момент времени известны
(остаточные) длины всех заявок в системе.
В момент поступления в систему новой заявки ее длина~$x$
сравнивается с (остаточной) длиной~$y$ заявки на приборе.
При этом если система функционирует в режиме~0, то с
вероятностью $d_0(x,y)$ на прибор становится вновь
поступившая заявка, а находившаяся ранее на приборе
занимает первое место в очереди, и наоборот, с
вероятностью $\d_0(x,y)\hm=1\hm-d_0(x,y)$ старая заявка
продолжает обслуживаться, а новая становится на первое
место в очереди.
Если же система функционирует в режиме~1, то
вероятность постановки на прибор вновь поступившей
заявки равна $d_1(x,y)$, а на первое мес\-то в очереди~---
$\d_1(x,y)\hm=1\hm-d_1(x,y)$.

Будем предполагать, что выполнено условие $\lambda b_1\hm<1$,
необходимое и достаточное для существования
стационарного режима функционирования рассматриваемой
системы.

Будем считать также, что $n_0\hm-n_1\hm\ge 2$.
Это предположение вводится только для того, чтобы не
рассматривать случаи, которые по записи расчетных
формул несколько отличаются от общего вида, и нисколько
не умаляет общности полученных результатов.

\section{Вспомогательные функции}

Пусть в некоторый момент система функционирует
в режиме~0, в системе находится $n$, $n_1\hm<n\hm<n_0$,
заявок и в этот момент поступает в сис\-те\-му и становится
на прибор новая заявка длины~$x$.
Обозначим через $\alpha_n(x)$ вероятность того, что в тот
момент, когда в системе впервые снова останется~$n$
заявок, она по-преж\-не\-му будет пребывать в режиме~0.


Функции $\alpha_n(x)$, $n_1\hm<n\hm<n_0$, удовлетворяют системе
уравнений

\columnbreak

\noindent
\begin{equation}
\label{2.1-m}
\alpha_{n_0-1}(x) \equiv 0\,;                    
\end{equation}

\vspace*{-12pt}

\noindent
\begin{multline}
\label{2.2-m}
\alpha_{n}(x)
= e^{-\lambda_0 x} 
+\int\limits_0^x \lambda_0 e^{-\lambda_0 y}\,dy\times{}\\[2pt]
{}\times
\int\limits_0^\infty b_0(z) \left[
d_0(z,x-y) \alpha_{n+1}(z) \alpha_{n}(x-y) + {}\right.\\[2pt]
\left.{}+ \d_0(z,x-y) \alpha_{n+1}(x-y) \alpha_{n}(z)
\right]  dz\,,\\[4pt]
n=\overline{n_1+1,n_0-2}\,.            % \eqno(2)
\end{multline}
Система уравнений~(\ref{2.1-m}), (\ref{2.2-m}) решается
последовательно, начиная с $n\hm=n_0\hm-1$ и кончая $n\hm=n_1+1$.

При решении уравнения~(\ref{2.2-m}) удобно привести его
к более простому виду. Вводя обозначение
\begin{equation*}
%\label{2.3-m}
a_{n}(x) = e^{\lambda_0 x} \alpha_{n}(x)\,,
\enskip  n=\overline{n_1+1,n_0-2}\,,
\end{equation*}
и производя тривиальные преобразования, получаем из~(\ref{2.2-m}):
\begin{multline}
\label{2.4-m}
a_{n}(x) = 1 +{}\\[2pt]
{}+ \int\limits_0^x \left(
\lambda_0 \int\limits_0^\infty b_0(z) d_0(z,y) \alpha_{n+1}(z)\, dz \right)
a_{n}(y)\, dy + {}\\[2pt]
{}+ \int\limits_0^\infty \left(
\lambda_0 b_0(y) e^{-\lambda_0 y} \int\limits_0^x e^{\lambda_0\, z} \d_0(y,z) \alpha_{n+1}(z)\,dz
\right)\times{}\\[2pt]
{}\times  a_{n}(y) \, dy\,,
\quad n=\overline{n_1+1,n_0-2}\,.
\end{multline}
Последнее соотношение представляет собой интегральное уравнение
\begin{multline}
\label{2.5-m}
a_n(x) = 1 + \int\limits_0^\infty K_n(x,y) a_{n}(y) \, dy\,,
\\[2pt]
 n=\overline{n_1+1,n_0-2}\,, 
\end{multline}
ядро которого имеет вид:

\noindent
\begin{multline*}
K_n(x,y) ={}\\
\hspace*{-7.92743pt}{}=
\begin{cases}
\displaystyle\lambda_0 \left( \int\limits_0^\infty
b_0(z) d_0(z,y) \alpha_{n+1}(z)\, dz+
b_0(y) e^{-\lambda_0 y}\times{}\right.\\
\left.\displaystyle{}\times \int\limits_0^x e^{\lambda_0 z}\, \d_0(y,z)
\alpha_{n+1}(z) \,dz \vphantom{\int\limits^\infty_0}\right),                     &\hspace*{-38pt}y<x\,;     \\
\displaystyle\lambda_0 b_0(y) e^{-\lambda_0 y}
\int\limits_0^x e^{\lambda_0 z} \,\d_0(y,z)
\alpha_{n+1}(z) \,dz\,,                         &\hspace*{-38pt}y>x\,.
\end{cases}
\end{multline*}
Численное решение уравнения~(\ref{2.5-m}) можно произ\-вес\-ти итерационным методом.
При этом в качестве нулевой итерации удобно выбрать тождественно равную нулю функцию.
Тогда итерации будут возрастающими, что позволит контролировать сходимость 
итерационного процесса.

В заключение этого раздела приведем условие на функцию $\d_0(x,y)$, при
котором интегральное уравнение~(\ref{2.4-m}) можно
свести к системе линейных алгебраических уравнений.
А~именно: будем предполагать, что 
\begin{equation}
\label{2.5-1}
\hspace*{-2mm}\d_0(x,y) = \sum\limits_{i=1}^I \d^{(1)}_{0,i}(x) \d^{(2)}_{0,i}(y),
\
 n=\overline{n_1+1,n_0-2}.\!\!
\end{equation}
Тогда, вводя обозначения
\begin{multline*}
c_n(y)= \lambda_0 \int\limits_0^\infty b_0(z) d_0(z,y) \alpha_{n+1}(z)\, dz\,,
\\
 n=\overline{n_1+1,n_0-2}\,;
\end{multline*}

\vspace*{-12pt}

\noindent
\begin{multline*}
c_{n,i}(x) = e^{\lambda_0 x} \d^{(2)}_{0,i}(x) \alpha_{n+1}(x)\,,\\
n=\overline{n_1+1,n_0-2}\,,\enskip
i=\overline{1,I}\,;
\end{multline*}

\vspace*{-12pt}

\noindent
\begin{multline*}
a_{n,i} = \int\limits_0^\infty \lambda_0 b_0(y) e^{-\lambda_0\, y} \d^{(1)}_{0,i}(y)
a_{n}(y) \, dy\,, \\ n=\overline{n_1+1,n_0-2}\,,
\ i=\overline{1,I}\,,
\end{multline*}
получаем из~(\ref{2.4-m}):
\begin{multline}
a_{n}(x)=1+\int\limits_0^x c_n(y) a_{n}(y)\, dy +
\sum\limits_{i=1}^I a_{n,i} \int\limits_0^x c_{n,i}(z)\, dz\,,
\\ n=\overline{n_1+1,n_0-2}\,.
\label{2.6-m}
\end{multline}
%%%%%%%%%%%%%%%%%%%%%%%%%%%%%
Дифференцируя теперь равенство~(\ref{2.6-m}), приходим
к дифференциальному уравнению
\begin{multline}
a'_{n}(x)= c_n(x)\, a_{n}(x) + \sum\limits_{i=1}^I a_{n,i} c_{n,i}(x)\,,
\\ n=\overline{n_1+1,n_0-2}\,,
\label{2.7-m}
\end{multline}
начальное условие для которого задается выражением:
\begin{equation}
\label{2.8-m}
a_n(0) = 1\,,
\enskip n=\overline{n_1+1,n_0-2}\,.
\end{equation}
Решение уравнения (\ref{2.7-m}) с начальным условием~(\ref{2.8-m}) имеет вид:
\begin{multline}
a_{n}(x) = \left(
1+ \sum\limits_{i=1}^I\! a_{n,i} \int\limits_0^x c_{n,i}(y) e^{- C_n(y)} \,dy
\right)
e^{C_n(x)} ,\\  n=\overline{n_1+1,n_0-2},
\label{2.9-m}
\end{multline}
где
\begin{equation*}
%\label{2.9}
C_{n}(x) = \int\limits_0^x c_n(y)\, dy\,,
\enskip n=\overline{n_1+1,n_0-2}\,.
\end{equation*}

Для того чтобы найти коэффициенты
$a_{n,i}$, $i\hm=\overline{1,I}$, умножим равенство~(\ref{2.9-m}) на
$\lambda_0 b_0(x) e^{-\lambda_0 x} \d^{(1)}_{0,j}(x)$
и проинтегрируем в пределах от~0 до~$\infty$. Тогда
\begin{multline*}
%\label{2.10-m}
a_{n,j} = \int\limits_0^\infty \lambda_0 b_0(x) e^{-\lambda_0\, x} \d^{(1)}_{0,j}(x)
e^{C_n(x)} dx + {}\\
{}+ \sum\limits_{i=1}^I a_{n,i} \int\limits_0^\infty \lambda_0 b_0(x) e^{-\lambda_0\, x} 
\d^{(1)}_{0,j}(x) e^{C_n(x)} \,dx\times{}\\
{}\times
\int\limits_0^x c_{n,i}(y) e^{- C_n(y)}  \,dy\,,
\enskip n=\overline{n_1+1,n_0-2}\,.
\end{multline*}
%%%%%%%%%%%%%%%%%%%%%%
Производя эту процедуру при всех
$j$, $j\hm=\overline{1,I}$, получаем систему линейных
ал\-геб\-ра\-и\-че\-ских уравнений, решая которую,
находим коэффициенты $a_{n,i}$ и соответственно
функции $a_{n}(x)$ и $\alpha_{n}(x)$.

В дальнейшем будем пользоваться обозначением
$\oa_n(x) \hm= 1 - \alpha_n(x)$.


Отметим, что, используя приближение $\d_0(x,y)$ с
помощью представления~(\ref{2.5-1}), 
можно найти функцию $a_{n}(x)$ с любой степенью точности.
Однако повышение точности влечет за собой существенное
увеличение числа $I$ коэффициентов $a_{n,i}$ и, 
значит, размерности системы линейных алгебраических уравнений.

\section{Стационарные вероятности состояний}

Обозначим через $p_0$ стационарную вероятность того,
что система свободна.
При $n\hm=\overline{1,n_1}$ или $n\hm\ge n_0$ обозначим через
$p_n(x_1,\ld,x_n)$ стационарную плот\-ность вероятностей того,
что в системе находится $n$ заявок, причем заявка на
приборе\linebreak имеет длину~$x_1$, первая заявка в очереди~---
длину~$x_2$ и~т.\,д.
Наконец, при $n\hm=\overline{n_1+1,n_0-1}$\linebreak через
$p_n(0;x_1,\ld,x_n)$ обозначим стационарную плотность
вероятностей того, что система функционирует в режиме~0 и
в системе находится~$n$ заявок, причем заявка на приборе
имеет длину~$x_1$, первая заявка в очереди~--- длину~$x_2$
и~т.\,д., а через $p_n(1;x_1,\ld,x_n)$~--- аналогичную
вероятность, но при этом система функционирует в режиме~1.

Используя метод исключения состояний (см., например,~\cite{22-m}), 
можно получить для $p_n(x_1,\ld,x_n)$,
$n\hm=\overline{1,n_1}$, уравнения
%%%%%%%%%%%%%%%%%%%%%%%%%%%%%%%
\begin{multline}
\label{3-0-m}
-p'_1(x)=-\lambda_0 p_1(x)+\lambda_0 p_1(x)\int\limits_0^\infty b_0(y) d_0(y,x)\, dy
+{}
\\
{}+\lambda_0 b_0(x)\int\limits_0^\infty p_1(y) \d_0(x,y)\, dy+
\lambda_0 p_0 b_0(x)\,;
\end{multline}
%%%%%%%%%%%%%%%%%%%%%%%%%%%%%%

\vspace*{-12pt}

\noindent
\begin{multline*}
%\label{3.3}
-p'_n(x_1,\ld,x_n) = - \lambda_0 p_{n}(x_1,\ld,x_n) +{}\\
{}+ \lambda_0 p_n(x_1,\ld,x_n)
\int\limits_0^\infty b_0(y) d_0(y,x_1)\, dy
+{}
\\
+ \lambda_0 b_0(x_1)\int\limits_0^\infty p_n(y,x_2,\ld,x_n) \d_0(x_1,y)\, dy
+ {}\\
{}+\lambda_0 p_{n-1}(x_2,\ld,x_n) b_0(x_1) d_0(x_1,x_2) +{}
\\
{}+ \lambda_0 b_0(x_2) p_{n-1}(x_1,x_3,\ld,x_n) \d_0(x_2,x_1),
\ n=\overline{2,n_1},\hspace*{-0.52872pt}
\end{multline*}
%%%%%%%%%%%%%%%%%%%%%%%%%%%%%%
с начальным условием
%%%%%%%%%%%%%%%%%%%%%%%%%%%%%%
$$
\lim_{x\to\infty} p_n(x,x_2,\ld,x_n) = 0\,, \enskip n=\overline{1,n_1}\,.
$$
%%%%%%%%%%%%%%%%%%%%%%%%%%%%%%
Можно выписать аналогичные уравнения для остальных
функций $p_n(x_1,\ld,x_n)$, $n\hm\ge n_0$, и
$p_n(i;x_1,\ld,x_n)$, $n\hm=\overline{n_1+1,n_0-1}$,
$i\hm=1,2$,
но они ввиду громоздкости здесь не приводятся.
Вычисления по этим формулам, хотя теоретически и можно
производить на основе решения интегральных уравнений,
практически не реализуемы уже при совсем небольших
значениях~$n$ даже на современной вычислительной технике,
поскольку размерность уравнений растет пропорционально~$n$.

Однако для практических расчетов, как правило,
достаточно знать только маргинальные стационарные
плотности $p_1(x)$,
\begin{multline*}
p_n(x) = \int\limits_0^\infty \cdots \int\limits_0^\infty
p_n(x,x_2,\ld,x_n)\, dx_2\cdots dx_n\,,
\\ n=\overline{2,n_1}
\enskip \hbox{или}
\enskip n\ge n_0\,,
\end{multline*}
и
\begin{multline*}
p_n(i;x)= \int\limits_0^\infty \cdots\int\limits_0^\infty
p_n(i;x,x_2,\ld,x_n)\, dx_2\cdots dx_n\,,
\\ i=0,1\,,\enskip n=\overline{n_1+1,n_0-1}\,.
\end{multline*}
Для них справедливы соотношения
%%%%%%%%%%%%%%%%%%%%%%%%%%%%%%
\begin{multline}
\label{3-1-m} -p'_n(x) = - f_0(x)\, p_n(x) + \int\limits_0^\infty
k_0(x,y)\, p_n(y)\, dy +{}
\\ 
{}+g_{0,n}(x) \,,\enskip
n=\overline{2,n_1}\,;
\end{multline}
%%%%%%%%%%%%%%%%%%%%%%%%%%%%%%

\vspace*{-12pt}

\noindent
\begin{multline}
\label{3-2-m}
-p'_n(0;x) = - f_{0,n}(x)\, p_n(0,x) +{}\\
{}+ \int\limits_0^\infty k_{0,n}(x,y)\, p_n(0,y)\, dy
+ g_{0,n}(x) \,,
\\  n=\overline{n_1+1,n_0-1}\,;
\end{multline}

%%%%%%%%%%%%%%%%%%%%%%%%%%%%%%
\vspace*{-12pt}

\noindent
\begin{multline}
\label{3-3-m}
-p'_n(1;x) = - f_{1}(x) p_n(1,x) + {}\\
{}+\int\limits_0^\infty k_{1}(x,y) p_n(1,y)\, dy
+ g_{1,n}(x)\,, \\ 
n=\overline{n_1+1,n_0-1}\,;
\end{multline}
%%%%%%%%%%%%%%%%%%%%%%%%%%%%%%

\vspace*{-12pt}

\noindent
\begin{multline}
\label{3-4-m}
-p'_n(x) = - f_{1}(x) p_n(x) +{}\\
{}+\int\limits_0^\infty k_{1}(x,y) p_n(y)\, dy + g_{1,n}(x)\,,
\  n\ge n_0\,,
\end{multline}
с начальными условиями
%%%%%%%%%%%%%%%%%%%%%%%%%%%%%%
\begin{equation}
\label{3-beg-1-m}
\lim_{x\to\infty} p_n(x) = 0\,,
\ \ n=\overline{2,n_1}\ \ \hbox{или}\ \ n\ge n_0\,,
\end{equation}
%%%%%%%%%%%%%%%%%%%%%%%%%%%%%%
\begin{equation}
\label{3-beg-2-m}
\lim_{x\to\infty} p_n(i;x) = 0\,,
\ n=\overline{n_1+1,n_0-1}\,,\ \ i=0,1\,,
\end{equation}
%%%%%%%%%%%%%%%%%%%%%%%%%%%%%%
в которых для сокращения записи введены сле\-ду\-ющие
обозначения:
%%%%%%%%%%%%%%%%%%%%%%%%%%%%%%%
\begin{align*}
f_0(x) &= \lambda_0 \left(
1 - \int\limits_0^\infty b_0(y) d_0(y,x)\, dy \right)\,;
\\
k_0(x,y) &= \lambda_0 b_0(x) \d_0(x,y) \,;
\\
g_{0,1}(x) &= \lambda_0 p_0 b_0(x) \,;\\
g_{0,n}(x) &= \lambda_0 b_0(x) \int\limits_0^\infty
p_{n-1}(y) d_0(x,y)\, dy +{}\\
&\hspace*{2mm}{}+
\lambda_0 p_{n-1}(x) \int\limits_0^\infty \d_0(y,x) b_0(y)\, dy\,,
\ \ n=\overline{2,n_1}\,;
\end{align*}
%%%%%%%%%%%%%%%%%%%%%%%%%%%%%%

\vspace*{-24pt}

\noindent
\begin{multline*}
f_{0,n}(x)= \lambda_0 \left(
1 - \int\limits_0^\infty b_0(y) d_0(y,x) \alpha_n(y)\, dy
\right)\,,
\\ n=\overline{n_1+1,n_0-1}\,;
\end{multline*}
%%%%%%%%

\vspace*{-12pt}

\noindent
\begin{multline*}
k_{0,n}(x,y) = \lambda_0 b_0(x) \d_0(x,y) \alpha_n(y) \,,
\\ n=\overline{n_1+1,n_0-1}\,;
\end{multline*}
%%%%%%%%%

\vspace*{-12pt}

\noindent
\begin{multline*}
g_{0,n_1+1}(x) = \lambda_0 b_0(x) \int\limits_0^\infty p_{n_1}(y) d_0(x,y)\, dy
+{}\\
{}+
\lambda_0 p_{n_1}(x) \int\limits_0^\infty b_0(y) \d_0(y,x)\, dy\,;
\end{multline*}
%%%%%%%%%

\vspace*{-12pt}

\noindent
\begin{multline*}
g_{0,n}(x) = \lambda_0 b_0(x) \int\limits_0^\infty p_{n-1}(0;y) d_0(x,y)\, dy
+{}\\
{}+ \lambda_0 p_{n-1}(0;x) \int\limits_0^\infty b_0(y) \d_0(y,x)\, dy\,,
\\ 
n=\overline{n_1+2,n_0-1}\,;
\end{multline*}
%%%%%%%%%%%%%%%%%%%%%%%%%%%%%%
\begin{align}
\label{3-gf-1-m}
f_1(x)&= \lambda_1 \left(
1 - \int\limits_0^\infty b_1(y) d_1(y,x)\, dy\right)\,;
\\
\label{3-gf-2-m}
k_1(x,y) &= \lambda_1 b_1(x) \d_1(x,y) \,,
\end{align}
%%%%%%%%%%%

\vspace*{-12pt}

\noindent
\begin{multline*}
g_{1,n_1+1}(x)={}\\
 {}=\lambda_0 p_{n_1+1}(0;x)\int\limits_0^\infty b_0(y) d_0(y,x) 
\oa_{n_1+1}(y)\, dy
+{}
\\
{}+ \lambda_0 b_0(x) \int\limits_0^\infty p_{n_1+1}(0;y) \d_0(x,y) \oa_{n_1+1}(y)\, dy\,;
\end{multline*}
%%%%%%%%%%


\vspace*{-12pt}

\noindent
\begin{multline*}
g_{1,n}(x) = \lambda_0 p_{n}(0;x) \int\limits_0^\infty b_0(y) d_0(y,x) \oa_n(y)\, dy
+{}\\
{}+ \lambda_0 b_0(x) \int\limits_0^\infty p_{n}(0;y) \d_0(x,y) \oa_n(y)\, dy
+ {}\\
{}+ \lambda_1 b_1(x) \int\limits_0^\infty p_{n-1}(1;y) d_1(x,y)\, dy+{}\\
{}+
\lambda_1 p_{n-1}(1;x) \int\limits_0^\infty b_1(y) \d_1(y,x)\, dy\,,
\\ n=\overline{n_1+2,n_0-1}\,;
\end{multline*}
%%%%%%%%%%%%%%%%%%%%%%%%%%%%%%

\vspace*{-24pt}

\noindent
\begin{multline*}
g_{1,n_0}(x) = \lambda_0 b_0(x)\int\limits_0^\infty p_{n_0-1}(0;y) d_0(x,y)\, dy
+{}\\
{}+
\lambda_0 p_{n_0-1}(0;x) \int\limits_0^\infty b_0(y) \d_0(y,x)\, dy+{}
\\
{}+ \lambda_1 b_1(x) \int\limits_0^\infty p_{n_0-1}(1;y) d_1(x,y)\, dy+{}\\
{}+
\lambda_1 p_{n_0-1}(1;x) \int\limits_0^\infty b_1(y) \d_1(y,x)\, dy \,;
\end{multline*}
%%%%%%%%%%%%%%%%%%%%%%%%%%

\vspace*{-12pt}

\noindent
\begin{multline}
\label{3-gf-3-m}
g_{1,n}(x)= \lambda_1 b_1(x) \int\limits_0^\infty p_{n-1}(y) d_1(x,y)\, dy
+{}\\
{}+ \lambda_1 p_{n-1}(x) \int\limits_0^\infty b_1(y) \d_1(y,x)\, dy\,.
\ \ n>n_0\,,
\end{multline}
%%%%%%%%%%%%%%%%%%%%%%%%%%%%%%%

Вероятность $p_0$ вычисляется из условия нормировки
$$
p_0 + \sum\limits_{n=1}^{n_1} p_n+\sum\limits_{n=n_1+1}^{n_0-1} \left[p_{n,0} + p_{n,1}\right]
+
\sum\limits_{n=n_0}^{\infty} p_n = 1\,,
$$
где $p_n = \int\limits_0^\infty p_n(x)\, dx$,
$n\hm=\overline{1,n_1}$ или $n\hm\ge n_0$,~---
стационарная вероятность того, что в системе находится
$n$ заявок, а $p_{n,i} \hm= \int\limits_0^\infty p_n(i;x)\, dx$,
$n\hm=\overline{n_1+1,n_0-1}$, $i\hm=0,1$,~--- стационарная
вероятность того, что система функционирует в режиме~$i$ и
в системе находится $n$~заявок.

Уравнения~(\ref{3-0-m})--(\ref{3-4-m}) легко приводятся к
интегральным. Действительно, вводя новые обозначения
%%%%%%%%%%%%%%%%%%%%%%%%%%%%%%%
\begin{align*}
F_0(x) &= \int\limits_0^x f_0(y)\, dy\,,
\ \ n=\overline{1,n_1}\,;
\\
%%%%%%%%%%%%%%%%%%%%%%%%%%%%%%
F_{0,n}(x) &= \int\limits_0^x f_{0,n}(y)\, dy\,,
\ \ n=\overline{n_1+1,n_0-1}\,;
\\
%%%%%%%%%%%%%%%%%%%%%%%%%%%%%%
F_{1}(x)&= \int\limits_0^x f_{1}(y)\, dy\,,
\ \ n\ge n_1+1\,;
\\
%%%%%%%%%%%%%%%%%%%%%%%%%%%%%%
p_n(x) &= \pi_n(x) e^{F_0(x)} \,, \ \ n=\overline{1,n_1}\,;
\\
%%%%%%%%%%%%%%%%%%%%%%%%%%%%%%
p_n(0;x) &= \pi_n(0;x) e^{F_{0,n}(x)} \,, \ \ n=\overline{n_1+1,n_0-1}\,;
\\
%%%%%%%%%%%%%%%%%%%%%%%%%%%%%%
p_n(1;x) &= \pi_n(1;x) e^{F_{1}(x)} \,, \ \ n=\overline{n_1+1,n_0-1}\,;
\\
p_n(x) &= \pi_n(x) e^{F_{1}(x)} \,, \ \ n\ge n_0\,,
\end{align*}

%%%%%%%%%%%%%%%%%%%%%%%%%%%%%%


\noindent
из \eqref{3-0-m}--\eqref{3-4-m} получаем соотношения

\noindent
%%%%%%%%%%%%%%%%%%%%%%%%%%%%%%
\begin{multline}
\label{pi-1-m}
-\pi'_n(x) = e^{-F_0(x)} \int\limits_0^\infty e^{F_0(y)} k_0(x,y) \pi_n(y)\, dy
+{}\\
{}+
e^{-F_0(x)} g_{0,n}(x) \,,
\ \ n=\overline{1,n_1}\,;
\end{multline}
%%%%%%%%%%%%%%%%%%%%%%%%%%%%%%

\vspace*{-12pt}

\noindent
\begin{multline}
\label{pi-2-m}
-\pi'_n(0;x) = {}\\
{}=e^{-F_{0,n}(x)} \int\limits_0^\infty e^{F_{0,n}(y)} 
k_{0,n}(x,y) \pi_n(0;y)\, dy +{}\\
{}+
e^{-F_{0,n}(x)} g_{0,n}(x) \,, \ \ n=\overline{n_1+1,n_0-1}\,;
\end{multline}
%%%%%%%%%%%%%%%%%%%%%%%%%%%%%%

\vspace*{-12pt}

\noindent
\begin{multline}
\label{pi-3-m}
-\pi'_n(1;x) = e^{-F_{1}(x)} \int\limits_0^\infty e^{F_{1}(y)} k_{1}(x,y) \pi_n(1;y)\, dy
+{}\\
{}+
e^{-F_{1}(x)} g_{1,n}(x) \,, \ \ n=\overline{n_1+1,n_0-1}\,;
\end{multline}
%%%%%%%%%%%%%%%%%%%%%%%%%%%%%%

\vspace*{-12pt}

\noindent
\begin{multline}
\label{pi-4-m}
-\pi'_n(x) = e^{-F_{1}(x)} \int\limits_0^\infty e^{F_{1}(y)} k_{1}(x,y)\pi_n(y)\, dy
+{}\\
{}+
e^{-F_{1}(x)} g_{1,n}(x) \,, \ \ n\ge n_0\,,
\end{multline}
%%%%%%%%%%%%%%%%%%%%%%%%%%%%%%
интегрируя которые в пределах от~$x$ до $\infty$
и учитывая начальные условия~(\ref{3-beg-1-m}),
(\ref{3-beg-2-m}), имеем
\begin{multline}
\label{int-1-m}
\pi_n(x) ={}\\
{}= \int\limits_0^\infty e^{F_0(y)} \left(
\int\limits_x^\infty e^{-F_0(u)} k_0(u,y)\, du
\right) \pi_n(y)\, dy
+{}\\
{}+
\int\limits_x^\infty e^{-F_0(u)} g_{0,n}(u)\, du\,;
\ \ n=\overline{1,n_1}\,,
\end{multline}

\vspace*{-12pt}

\noindent
\begin{multline}
\label{int-2-m}
\pi_n(0;x) ={}\\
{}= \int\limits_0^\infty\! e^{F_{0,n}(y)} \left(
\int\limits_x^\infty\! e^{-F_{0,n}(u)} k_{0,n}(u,y)\, du \right)
\pi_n(0;y)\, dy+
\\
{}+
\int\limits_x^\infty e^{-F_{0,n}(u)} g_{0,n}(u)\, du\,,
\enskip n=\overline{n_1+1,n_0-1}\,;
\end{multline}
%%%%%%%%%%%%%%%%%%%%%%%%%%%%%%


\vspace*{-12pt}

\noindent
\begin{multline}
\label{int-3-m}
\pi_n(1;x) ={}\\
{}= \int\limits_0^\infty e^{F_{1}(y)} \left(
\int\limits_x^\infty e^{-F_{1}(u)} k_{1}(u,y)\, du\right)
\pi_n(1;y)\, dy
+{}
\\
{}+
\int\limits_x^\infty e^{-F_{1}(u)} g_{1,n}(u)\, du\,,
\ \ n=\overline{n_1+1,n_0-1}\,;
\end{multline}
%%%%%%%%%%%%%%%%%%%%%%%%%%%%%%
\begin{multline}
\label{int-4-m}
\hspace*{-5mm}\pi_n(x) = \int\limits_0^\infty e^{F_{1}(y)} \left(
\int\limits_x^\infty e^{-F_{1}(u)} k_{1}(u,y)\, du \right)
\pi_n(y)\, dy +{}\\
{}+
\int\limits_x^\infty e^{-F_{1}(u)} g_{1,n}(u)\, du\,,
\ \ n\ge n_0\,.
\end{multline}
%%%%%%%%%%%%%%%%%%%%%%%%%

Соотношения \eqref{int-1-m}--\eqref{int-4-m} являются
интегральными уравнениями такого же вида, что и~\eqref{2.5-m},
и к ним применимы те же методы решения, что и
к уравнению~\eqref{2.5-m}.

Так же как для функций $\alpha_n(x)$, приведем условия
для функций $\d_0(x,y)$ и $\d_1(x,y)$, которые поз\-во\-ляют
получить решения ин\-тег\-ро\-диф\-фе\-рен\-ци\-аль\-ных уравнений~\eqref{pi-1-m}--\eqref{pi-4-m}
с помощью приведения к системе линейных алгебраических уравнений.
А~именно: будем предполагать, что выполнены условия~\eqref{2.5-1} и
\begin{equation}
\label{2.5-2-m}
\d_1(x,y) = \sum\limits_{i=1}^{I_1} \d^{(1)}_{1,i}(x) \d^{(2)}_{1,i}(y) \,.
\end{equation}
Тогда, вводя обозначения
$$
c_i(x) = \lambda_0 b_0(x) \d^{(1)}_{0,i}(x) e^{-F_0(x)}\,,\ \ i=\overline{1,I}\,;
$$
$$
q_{n,i} = \int\limits_0^\infty e^{F_0(y)} \d^{(2)}_{0,i}(y) \pi_n(y)\, dy \,,\ \ 
n=\overline{1,n_1}\,,
\ \ i=\overline{1,I}\,;
$$
$$
q_{n}(x)= e^{-F_0(x)} g_{0,n}(x)\,,
\ \ n=\overline{1,n_1}\,;
$$
%%%%%%%%%%%%%%%%%%%%%%%%%%%%%%%%%%%%%%%

\vspace*{-12pt}

\noindent
\begin{multline*}
c_{0;n,i}(x)= \lambda_0 b_0(x) \d^{(1)}_{0,i}(x) e^{-F_{0,n}(x)}\,,\\
n=\overline{n_1+1,n_0-1}\, ,\ \ i=\overline{1,I}\,;
\end{multline*}

\vspace*{-12pt}

\noindent
\begin{multline*}
q_{0;n,i}= \int\limits_0^\infty e^{F_{0,n}(y)} \d^{(2)}_{0,i}(y) \alpha_n(y) \pi_n(0;y)\, dy\,,
\\ n=\overline{n_1+1,n_0-1}\,,
\ \ i=\overline{1,I}\,;
\end{multline*}
$$ 
q_{0;n}(x) = e^{-F_{0,n}(x)} g_{0,n}(x) \,, \ \ n=\overline{n_1+1,n_0-1}\,;
$$
%%%%%%%%%%%%%%%%%%%%%%%%%%%%%%%%%%%%%%%
$$
c_{1;i}(x)= \lambda_1 b_1(x) \d^{(1)}_{1,i}(x) e^{-F_{1}(x)}\,,\ \ i=\overline{1,I_1}\,;
$$

\vspace*{-12pt}

\noindent
\begin{multline*}
q_{1;n,i}= \int\limits_0^\infty e^{F_{1}(y)} \d^{(2)}_{1,i}(y) \pi_n(1;y)\, dy\,,\\ 
n=\overline{n_1+1,n_0-1}\,,
\ \ i=\overline{1,I_1}\,    ;
\end{multline*}
$$
q_{1;n}(x)= e^{-F_{1}(x)} g_{1,n}(x)\,,
\ \ n\ge n_1+1\,;
$$
%%%%%%%%%%%%%%%%%%%%%%%%%%%%%%%%%%%%%%%%%%

\vspace*{-12pt}

\noindent
\begin{equation*}
q_{1;n,i} = \int\limits_0^\infty e^{F_{1}(y)}  \d^{(2)}_{1,i}(y) \pi_n(y)\, dy\,,\\ 
n\ge n_0\,,\  i=\overline{1,I_1}\,,
\end{equation*}
%%%%%%%%%%%%%%%%%%%%%%%%%%%%%%%%%%%%%%%%%%%
получаем из (\ref{pi-1-m})--(\ref{pi-4-m}) после
интегрирования в пределах от $x$ до $\infty$
с учетом начальных условий~(\ref{3-beg-1-m}),
(\ref{3-beg-2-m}):
%%%%%%%%%%%%%%%%%%%%%%%%%%%%%%%%%%%%%%%%%%%
\begin{equation}
\label{ipi-1-m}
\pi_n(x)= \sum\limits_{i=1}^{I} C_i(x) q_{n,i}+Q_{n}(x) \,,
\ \ n=\overline{1,n_1}\,;
\end{equation}
%%%%%%%%%%%%%%%%%%%%%%%%%%%%%%

\vspace*{-24pt}

\noindent
\begin{multline}
\label{ipi-2-m}
\pi_n(0;x) = \sum\limits_{i=1}^{I} C_{0;n,i}(x) q_{0;n,i} +
Q_{0;n}(x) \,, \\[1pt]
 n=\overline{n_1+1,n_0-1}\,;
\end{multline}
%%%%%%%%%%%%%%%%%%%%%%%%%%%%%%

\vspace*{-12pt}

\noindent
\begin{multline}
\label{ipi-3-m}
\pi_n(1;x) = \sum\limits_{i=1}^{I_1} C_{1;i}(x) q_{1;n,i} +
Q_{1;n}(x)\,, \\[1pt] 
n=\overline{n_1+1,n_0-1}\,;
\end{multline}
%%%%%%%%%%%%%%%%%%%%%%%%%%%%%%
\begin{equation}
\label{ipi-4-m}
\pi_n(x)= \sum\limits_{i=1}^{I_1} C_{1;i}(x) q_{1;n,i} +
Q_{1;n}(x)\,, \ \ n\ge n_0\,,
\end{equation}
%%%%%%%%%%%%%
где
%%%%%%%%%%%%%
$$
C_{i}(x)= \int\limits_x^\infty c_{i}(y)\, dy\,, \ \ n=\overline{1,n_1}\,,\ \ i=\overline{1,I}\,;
$$

\vspace*{-12pt}

\noindent
\begin{multline*}
C_{0;n,i}(x)= \int\limits_x^\infty c_{0;n,i}(y)\, dy\,, \\ 
n=\overline{n_1+1,n_0-1}\,,\ \ i=\overline{1,I}\,;
\end{multline*}
$$
C_{1,i}(x) = \int\limits_x^\infty c_{1,i}(y)\, dy\,,
\ \ n\ge n_1\,,\ \ i=\overline{1,I_1}\,;
$$
%%%%%%%%%%%%%%%%%%%%%%%%%%%%%%
$$
Q_{n}(x) = \int\limits_x^\infty q_{n}(y)\, dy\,,
\ \ n=\overline{1,n_1}\,;
$$
$$
Q_{0;n}(x)= \int\limits_x^\infty q_{0;n}(y)\, dy\,,
\ \ n=\overline{n_1+1,n_0-1}\,;
$$
$$
Q_{1;n}(x)= \int\limits_x^\infty q_{1;n}(y)\, dy\,,
\ \ n\ge n_1\,.
$$

Для определения постоянных
$q_{n,i}$, $q_{0;n,i}$ и $q_{1;n,i}$
умножим равенства~\eqref{ipi-1-m}--\eqref{ipi-4-m} на
$\d^{(2)}_{0,j}(y) e^{F_0(y)}$,
$\d^{(2)}_{0,j}(y) \alpha_n(y) e^{F_{0,n}(y)}$
и $\d^{(2)}_{1,j}(y) e^{F_{1}(y)}$
соответственно и проинтегрируем в пределах от~0 до~$\infty$.
Тогда
%%%%%%%%%%%%%%%%%%%%%%%%%%%%%%%%%%%%%%%%%%%
\begin{multline}
\label{cons-1-m}
q_{n,j}= \sum\limits_{i=1}^{I} \int\limits_0^\infty \d^{(2)}_{0,j}(y) 
e^{F_0(y)} C_i(y)\, dy\, q_{n,i}
+{}\\[2pt]
\hspace*{-2mm}{}+
\int\limits_0^\infty \d^{(2)}_{0,j}(y) e^{F_0(y)}  Q_{n}(y)\, dy\,,
\  n=\overline{1,n_1}\,,
\  j=\overline{1,I}\,;\!\!
\end{multline}
%%%%%%%%%%%%%%%%%%%%%%%%%%%%%%

\vspace*{-12pt}

\noindent
\begin{multline*}
q_{0;n,j}= {}\\[2pt]
{}=\sum\limits_{i=1}^{I} \int\limits_0^\infty \d^{(2)}_{0,j}(y) \alpha_n(y)
e^{F_{0,n}(y)} C_{0;n,i}(y)\, dy\, q_{0;n,i}
+{}
\end{multline*}

\noindent
\begin{multline}
\label{cons-2-m}
{}+
\int\limits_0^\infty \d^{(2)}_{0,j}(y) \alpha_n(y) e^{F_{0,n}(y)} Q_{0;n}(y)\, dy\,,\\ 
n=\overline{n_1+1,n_0-1}\,, \enskip j=\overline{1,I}\,;
\end{multline}
%%%%%%%%%%%%%%%%%%%%%%%%%%%%%%

\vspace*{-12pt}

\noindent
\begin{multline}
\label{cons-3-m}
q_{1;n,j} = \sum\limits_{i=1}^{I_1} \int\limits_0^\infty \d^{(2)}_{1,j}(y) 
e^{F_{1}(y)}\, C_{1;i}(y)\, dy\, q_{1;n,i}
+{}\\
{}+
\int\limits_0^\infty \d^{(2)}_{1,j}(y) e^{F_{1}(y)} Q_{1;n}(y)\, dy\,,\\ 
n\ge n_1+1\,, \enskip j=\overline{1,I_1}\,.
\end{multline}
%%%%%%%%%%%%%

Каждое из соотношений~\eqref{cons-1-m}--\eqref{cons-3-m}
представляет собой систему линейных алгебраических
уравнений, что позволяет легко находить коэффициенты
$q_{n,i}$, $q_{0;n,i}$ и $q_{1;n,i}$ и в конечном
счете плотности $p_n(x)$, $p_n(0;x)$ и $p_n(1;x)$.


В~заключение этого раздела приведем выражение для
суммарной стационарной интенсивности~$\lambda$ входящего
потока:
\begin{multline}
\label{inten-1-m}
\lambda = \lambda_0 p_0 + \lambda_0 \sum\limits_{n=1}^{n_1}
p_n +{}\\
\!\!{}+ \lambda_0 \sum\limits_{n=n_1+1}^{n_0-1} p_{n,0}
+ \lambda_1 \sum\limits_{n=n_1+1}^{n_0-1} p_{n,1}
+ \lambda_1 \sum\limits_{n=n_0}^{\infty} p_n .\!\!
\end{multline}

\section{Применение производящих функций}

Для вычисления моментов стационарного распределения
числа заявок в системе можно воспользоваться производящей функцией (ПФ):
$$
p(z,x) = \sum\limits_{n=n_0}^\infty z^n p_n(x)\,.
$$
Правда, для того чтобы определить ПФ $p(z,x)$,
необходимо знать плотности вероятностей $p_{n_0-1}(0;x)$
и $p_{n_0-1}(1;x)$, а для этого предварительно вы\-чис\-лить
$p_{n}(x)$, $n\hm=\overline{1,n_1}$, $p_{n}(0;x)$, $n\hm=\overline{n_1+1,n_0-2}$, и
$p_{n}(1;x)$, $n\hm=\overline{n_1+1,n_0-2}$.

Умножая соотношения~\eqref{3-4-m} на $z^n$ и суммируя по~$n$, 
получаем после простейших преобразований с
учетом~\eqref{3-gf-1-m}--\eqref{3-gf-3-m}
\begin{multline}
\label{3.pf-m}
- p'_{x}(z,x) = - (1-z) f_1(x)\, p(z,x) +{}\\
{}+ \lambda_1 b_1(x)
\int\limits_0^\infty p(z,y) \left[\d_1(x,y) + z d_1(x,y)\right]\, dy
+{}\\
{}+
z^{n_0} g_{1,n_0}(x) 
\end{multline}
с начальным условием
\begin{equation}
\label{3.pf-b-m}
\lim_{x\to\infty} p(z,x) = 0\,.
\end{equation}

Уравнение~\eqref{3.pf-m} с начальным условием~\eqref{3.pf-b-m} легко приводится к интегральному
уравнению
\begin{equation*}
%\label{3.pf-2-m}
q(z,x) = \int\limits_0^\infty K(x,y) q(z,y)\, dy
+ z^{n_0} R(x)\,,
\end{equation*}
где
$$
q(z,x) = e^{-(1-z) F_1(x)} p(z,x)\,;
$$

\vspace*{-12pt}

\noindent
\begin{multline*}
K(x,y) = \lambda_1 \int\limits_x^\infty\! e^{(1-z) \left[F_1(y) - F_1(u)\right]}
b_1(u) \left[\,\d_1(u,y) +{}\right.\\
\left.{}+ z d_1(u,y)\right]\, du\,;
\end{multline*}
%%%%%%%%%%%%%%%
$$
R(x) = \int\limits_x^\infty e^{-(1-z) F_1(u)} g_{1,n_0}(u)\, du\,.
$$
Последнее уравнение имеет такой же вид, как и~\eqref{2.5-m},
с теми же замечаниями относительно решения, что и раньше.
Кроме того, если выполнено условие~\eqref{2.5-2-m},
то решение этого уравнения, как и прежде, сводится
к решению системы линейных алгебраических уравнений.

Производящая функция $P(z)$ стационарного распределения числа заявок в
системе без учета их длин и режима функционирования
определяется формулой:
\begin{multline*}
P(z) = p_0 + \sum\limits_{n=1}^{n_1} z^n p_{n} +
\sum\limits_{n=n_1+1}^{n_0-1} z^n \left[p_{n,0} + p_{n,1}\right]
+{}\\
{}+ \int\limits_0^\infty e^{(1-z) F_1(x)} q(z,x)\, dx\,.
\end{multline*}

Моменты стационарного распределения числа заявок
в системе вычисляются с помощью дифференцирования
ПФ $P(z)$ в точке $z\hm=1$ и по\-сле\-ду\-юще\-го решения
получившихся уравнений.

\section{Стационарное распределение времени пребывания
заявки в~системе}

Обозначим через $u(s;x)$ преобразование Лап\-ла\-са--Стилть\-еса
(ПЛС) для открываемого заявкой длины
$x$ периода занятости (ПЗ) обычной СМО $M/G/1/\infty$
с интенсивностью $\lambda_1$ входящего потока и функцией распределения $B_1(x)$
времени обслуживания заявки,
а через $u(s)$ --- то же самое ПЛС, но для ПЗ, открываемого
заявкой произвольной длины.
Тогда
\begin{align*}
u(s;x)&= e^{-[s + \lambda_1 - \lambda_1 u(s)]\,x}\,;
\\
u(s) &= \beta_1(s + \lambda_1 - \lambda_1 u(s))\,.
\end{align*}
%%%%%%%%%%%%%%%%%%%%%%%%%%

Предположим теперь, что в начальный момент
рассматриваемая СМО функционирует в режиме~0 и
в ней находится~$n$, $n\hm=\overline{1,n_0-1}$,
заявок.
Обозначим через $u_n(s;x)$, $n\hm=\overline{1,n_0-1}$, ПЛС времени до того момента,
когда в системе впервые останется $n-1$ заявок
и при этом система по-преж\-не\-му будет функционировать
в режиме~0, при условии что на приборе начала
обслуживаться заявка длины~$x$, а через $u^*_n(s;x)$, $n\hm=\overline{n_1+2,n_0-1}$,~--- 
функцию, подобную $u_n(s;x)$, но при этом система перейдет в режим~1.

Справедливы уравнения
%%%%%%%%%%%
\begin{equation}
\label{5-1-m}
u'_{n_0-1}(s;x) = - \left[s + \lambda_0\right] u_{n_0-1}(s;x)\,;
\end{equation}
%%%%%%%%%

\vspace*{-12pt}

\noindent
\begin{multline*}
u'_n(s;x) = - (s + \lambda_0) u_{n}(s;x) +{}
\\
{}+
\lambda_0 \int\limits_0^\infty b_0(y) \left[d_0(y,x) u_{n+1}(s;y) u_{n}(s;x)
+{}\right.\\
\left.{}+ \d_0(y,x) u_{n+1}(s;x)\, u_{n}(s;y)\right] \, dy\,,
\\ 
n=\overline{n_1+2,n_0-2}\,,
\end{multline*}
%%%%%%%%%
с начальным условием
$$
u_n(s;0)= 1 \,, \ \ n=\overline{n_1+2,n_0-1}\,,
$$
%%%%%%%%%%%
уравнения
\begin{multline}
\label{5-2-m}
u^{*\,\prime}_{n_0-1}(s;x) = - [s + \lambda_0] u^*_{n_0-1}(s;x)+
\\
{}+
\lambda_0 \int\limits_0^\infty b_0(y) \left[d_0(y,x)  u(s;y)\, u(s;x)+{}\right.\\
\left.{}+
\d_0(y,x) u(s;x)\, u(s;y)\right] \, dy\,;
\end{multline}
%%%%%%%%%%%

\vspace*{-12pt}

\noindent
\begin{multline}
\label{5-2-2-m}
u^{*\,\prime}_n(s;x) = - \left[s + \lambda_0\right] u^*_{n}(s;x) +{}
\\
{}+
\lambda_0 \int\limits_0^\infty b_0(y) \left[d_0(y,x) u^*_{n+1}(s;y) u(s;x)
+ {}\right.\\
\left.{}+\d_0(y,x) u^*_{n+1}(s;x) u(s;y)\right] \, dy
+{}\\
{}+ \lambda_0 \int\limits_0^\infty b_0(y) \left[d_0(y,x) u_{n+1}(s;y) u^*_{n}(s;x)
+ {}\right.\\
\left.{}+d_0(y,x) u_{n+1}(s;x) u^*_{n}(s;y)\right] \, dy\,,
\\  n=\overline{n_1+2,n_0-2}\,,
\end{multline}
%%%%%%%%%
с начальным условием
$$
u_n(s;0) = 0\,, \ \ n=\overline{n_1+2,n_0-1}\,,
$$
и уравнения
\begin{multline}
\label{5-2-3-m}
u'_{n_1+1}(s;x) = - \left[s + \lambda_0\right] u_{n_1+1}(s;x) +{}
\\
{}+
\lambda_0 \int\limits_0^\infty b_0(y) \left[d_0(y,x) u^*_{n_1+2}(s;y) u(s;x)+{}\right.\\
\left.{}+
\d_0(y,x) u^*_{n_1+2}(s;x) u(s;y)\right] \, dy+{}
\\
{}+ 
\lambda_0 \int\limits_0^\infty b_0(y) \left[d_0(y,x) u_{n_1+2}(s;y) u_{n_1+1}(s;x) +{}\right.\\
\left.{}+
\d_0(y,x) u_{n_1+2}(s;x) u_{n_1+1}(s;y)\right]\, dy\,;
\end{multline}
%%%%%%%%%

\vspace*{-12pt}

\noindent
\begin{multline*}
u'_n(s;x) = - (s + \lambda_0) u_{n}(s;x) +{}\\
{}+
\lambda_0 \int\limits_0^\infty b_0(y) \left[d_0(y,x) u_{n+1}(s;y) u_{n}(s;x)
+{}\right.\\
\left.{}+\d_0(y,x) u_{n+1}(s;x) u_{n}(s;y)\right] \, dy\,,
\ \ n=\overline{1,n_1}\,,
\end{multline*}
%%%%%%%%%
с начальным условием
$$
u_n(s;0)= 1 \,,
\ \ n=\overline{1,n_1}\,.
$$

Решения уравнений~\eqref{5-1-m} и~\eqref{5-2-m} имеют вид:
%%%%%%%%%%%
$$
u_{n_0-1}(s;x)= e^{-(s + \lambda_0) x}\,;
$$
%%%%%%%%%%%%%%%%%%%%%%%%%%

\vspace*{-12pt}

\noindent
\begin{multline}
\label{5-2-4-m}
u^*_{n_0-1}(s;x) = \lambda_0 \int\limits_0^x e^{(s + \lambda_0) (z-x)}\, dz\times{}\\
{}\times \int\limits_0^\infty b_0(y) \left[d_0(y,z) u(s;y) u(s;z)+{}\right.\\
\left.{}+
\d_0(y,z) u(s;z) u(s;y)\right] \, dy\,.
\end{multline}
Остальные уравнения являются интегродифференциальными
и подобны уравнениям, полученным в предыдущих разделах.

Пусть в начальный момент в системе находится
$n$, $n\hm\ge n_1\hm+1$, заявок, система функционирует
в режиме~1, на приборе обслуживается заявка
длины~$y$ и в этот момент в систему поступает
заявка длины~$x$. Обозначим через $w(s;x,y)$ ПЛС времени ожидания
начала обслуживания этой заявки.
Тогда
$$
w(s;x,y) = d_1(x,y) + \d_1(x,y) u(s;y) \,.
$$

Пусть в начальный момент в системе находится
$n$, $n\hm=\overline{1,n_0-1}$, заявок, система
функционирует в режиме~0, на приборе обслуживается
заявка длины~$y$ и в этот момент в систему поступает
заявка длины~$x$.
Обозначим через $w_n(s;x,y)$ ПЛС времени ожидания
начала обслуживания этой заявки, причем в момент
начала обслуживания система по-преж\-не\-му будет функционировать в режиме~0.
Имеем:
$$
w_{n_0-1}(s;x,y) = 0\,;
$$
%%%%%%%%%%

\vspace*{-24pt}

\noindent
\begin{multline*}
w_n(s;x,y) = d_0(x,y) + \d_0(x,y) u_{n+1}(s;y)\, ,\\  
n=\overline{1,n_0-2}\,.
\end{multline*}

Наконец, пусть в начальный момент в системе находится~$n$, 
$n\hm=\overline{n_1+1,n_0-1}$, заявок, система
функционирует в режиме~0, на приборе обслуживается
заявка длины~$y$ и в этот момент в систему поступает
заявка длины~$x$.
Обозначим через $w^*_n(s;x,y)$ ПЛС времени ожидания
начала обслуживания этой заявки, причем в момент
начала обслуживания сис\-те\-ма окажется в режиме~1.
В~этом случае
\begin{equation}
\label{5-3-3-m}
w^*_{n_0-1}(s;x,y) = d_0(x,y) + \d_0(x,y) u(s;y)\,;
\end{equation}
%%%%%%%%%
$$
w^*_n(s;x,y) = \d_0(x,y) u^*_{n+1}(s;y)\,,\ \ n=\overline{n_1+1,n_0-2} \,.
$$


Стационарное распределение времени ожидания начала
обслуживания имеет ПЛС
\begin{multline*}
%\label{5-3-4}
w(s) = \fr{1}{ \lambda} \left[ \vphantom{\int\limits_0^\infty}
\lambda_0 p_0+{}\right.\\
{}+ \lambda_0 \int\limits_0^\infty \sum\limits_{n=1}^{n_1}
p_n(y) \, dy \int\limits_0^\infty b_0(x) w_n(s;x,y) \, dx
+{}
\\
{}+
\lambda_0 \int\limits_0^\infty \sum\limits_{n=n_1+1}^{n_0-1} p_n(0;y)\, dy\times{}\\
{}\times
\int\limits_0^\infty b_0(x) \left[w_n(s;x,y) + w^*_n(s;x,y)\right]\, dx
+{}
\\
{}+
\lambda_1 \int\limits_0^\infty \sum\limits_{n=n_1+1}^{n_0-1} p_n(1;y) \, dy
\int\limits_0^\infty b_1(x) w(s;x,y) \, dx
+{}\\
\left.{}+
\lambda_1 \int\limits_0^\infty \sum\limits_{n=n_0}^{\infty} p_n(y) \, dy
\int\limits_0^\infty b_1(x) w(s;x,y) \, dx
\right]\,.
\end{multline*}

Обозначим через $t(s;x)$ ПЛС времени от момента
первого попадания заявки длины~$x$
на прибор до момента ухода ее из системы при условии,
что в момент первого попадания на прибор система
функционировала в режиме~1.
Для $t(s;x)$ справедливо дифференциальное уравнение
\begin{multline*}
t'(s;x)= - t(s;x) \left( \vphantom{\int\limits_0^\infty}
s + {}\right.\\
\!\!\left.{}+\lambda_1 \!\left[
1 - \int\limits_0^\infty\! b_1(y)\left[\d_1(y,x) +
d_1(y,x)\, u(s;y)\vphantom{\overline{d}}\right] \, dy
\right]\!
\right)\hspace*{-1.717pt}
\end{multline*}
с начальным условием
$$
t(s;0)= 1 \,,
$$
решение которого имеет вид:
\begin{multline*}
t(s;x) = \exp\left\{ \vphantom{\int\limits_0^\infty}
-(s + \lambda_1) x +{}\right.\\
\!\!\left.{}+ \lambda_1 \!\int\limits_0^x\,\! dz\!
\int\limits_0^\infty b_1(y) \left[\,\d_1(y,z) +  d_1(y,z) u(s;y)\right] \, dy
\right\}.
\end{multline*}

Обозначим через $t_n(s;x)$, $n\hm=\overline{0,n_0-2}$,
ПЛС времени от момента первого попадания заявки длины~$x$
на прибор до момента ухода ее из системы при условии,
что в момент первого попадания на прибор в очереди
было еще $n$~заявок и система функционировала в режиме~0.
Тогда из дифференциальных уравнений
%%%%%%%%%%%%%%%%%%%
\begin{multline*}
t'_{n_0-2}(s;x) = - (s + \lambda_0) t_{n_0-2}(s;x) +{}\\
{}+
\lambda_0 \int\limits_0^\infty b_0(y) \left[d_0(y,x) u(s;y) + \d_0(y,x)\right] \, dy\, t(s;x)\,;
\end{multline*}
%%%%%%%%%%%%%%%%%%%

\vspace*{-12pt}

\noindent
\begin{multline*}
t'_n(s;x) = -\left( \vphantom{\int\limits_0^\infty}
s + \lambda_0 - {}\right.\\
\left.{}-\lambda_0 \int\limits_0^\infty b_0(y) d_0(y,x) u_{n+2}(s;y) \, dy
\right)
t_n(s;x) +{}
\\
{}+
\lambda_0 \int\limits_0^\infty b_0(y) \left[d_0(y,x) u_{n+2}^*(s;y) t(s;x) +{}\right.\\
\left.{}+
\d_0(y,x) t_{n+1}(s;x)\right] \, dy\,,
\ \ n=\overline{n_1,n_0-3}\,;
\end{multline*}
%%%%%%%%%%%%%%%%%%%%%

\vspace*{-12pt}

\noindent
\begin{multline*}
t'_n(s;x)= - \left( \vphantom{\int\limits_0^\infty}
s + \lambda_0 - {}\right.\\
\left.{}-\lambda_0 \int\limits_0^\infty b_0(y) d_0(y,x) u_{n+2}(s;y)\, dy
\right) t_n(s;x) +{}
\\
{}+
\lambda_0 t_{n+1}(s;x) \int\limits_0^\infty b_0(y) \d_0(y,x) \, dy\,,
\ \ n=\overline{0,n_1-1}\,,
\end{multline*}
%%%%%%%%%%%%%%%%%%%%%%%%%%%%%%%%%
с начальным условием
$$
t_{n}(s;0)= 1\,,\ \ n=\overline{0,n_0-2}\,,
$$
%%%%%%%%%%%%%%%%%%%%%%%%%%%%%%%
имеем:

\noindent
\begin{multline}
\label{5-4-1-m}
t_{n_0-2}(s;x)= {}\\
{}=e^{-(s + \lambda_0) x}\left(
1+ \lambda_0 \int\limits_0^x e^{(s + \lambda_0) z} t(s;z)\, dz\times{}\right.\\
\left.{}\times
\int\limits_0^\infty b_0(y) \left[d_0(y,z) u(s;y) + \d_0(y,z)\right] \, dy
\right)\,;
\end{multline}
%%%%%%%%%%%%%%%%%%%

\vspace*{-12pt}

\noindent
\begin{multline}
\label{5-4-2-m}
t_n(s;x)={}\\
{}= e^{-\int\limits_0^x \left( s + \lambda_0 -
\lambda_0 \int\limits_0^\infty b_0(y)\, d_0(y,z)\, u_{n+2}(s;y) \, dy
\right)\,dz} 
\left( \vphantom{\int\limits_0^\infty}
1 +{}\right.\\
{}+ \lambda_0 \int\limits_0^x e^{\int\limits_0^v \left(
s + \lambda_0 - \lambda_0 \int\limits_0^\infty b_0(y) d_0(y,z) u_{n+2}(s;y) \, dy
\right)\,dz} dv \times{}
\\
{}\times
\int\limits_0^\infty b_0(y) \left[d_0(y,v) u_{n+2}^*(s;y) t(s;v)+{}\right.\\
\left.\left.{}+ \d_0(y,v) t_{n+1}(s;v)
\vphantom{\int\limits_0^\infty}
\right] \, dy
\right) \,, \quad 
n=\overline{n_1,n_0-3}\,;
\end{multline}
%%%%%%%%%%%%%%%%%%%

\vspace*{-12pt}

\noindent
\begin{multline*}
t_n(s;x) = {}\\
{}=e^{-\int\limits_0^x\left( \vphantom{\int\limits_0^\infty}
s + \lambda_0 - \lambda_0 \int\limits_0^\infty b_0(y) d_0(y,z) u_{n+2}(s;y) \, dy \right)
dz} \left(  \vphantom{\int\limits_0^x}
1 +{}\right.\\
{}+ \lambda_0 \int\limits_0^x e^{\int\limits_0^v\left(
s + \lambda_0 -\lambda_0 \int\limits_0^\infty b_0(y) d_0(y,z) u_{n+2}(s;y) \, dy
\right) dz } dv \times{}\\
\left.{}\times
\int\limits_0^\infty b_0(y) \d_0(y,v) t_{n+1}(s;v) \, dy
\right)\,,
\ \ n=\overline{0,n_1-1}\,.
\end{multline*}

Обратимся к общему времени пребывания заявки в системе.

Обозначим через $v(s;x,y)$
ПЛС времени пребывания в системе заявки длины~$x$ при
условии, что эта заявка застала систему в режиме~1,
причем заявка на приборе имела длину~$y$.
Тогда
$$
v(s;x,y)= w(s;x,y) t(s;x)\,.
$$

Обозначим через
$v_{n}(s;x,y)$, $n\hm=\overline{1,n_0-1}$,
ПЛС времени пребывания в системе заявки длины~$x$ при
условии, что эта заявка застала в системе $n$~других
заявок, причем заявка на приборе имела длину~$y$, а
система пребывала в режиме~0.
Имеем:
\begin{multline}
\label{5-5-1-m}
v_{n_0-1}(s;x,y) = d_0(x,y) t(s;x) + {}\\
{}+\d_0(x,y) u(s;x,y) t(s;x)\,;
\end{multline}
%%%%%%%%%%%%%

\vspace*{-12pt}

\noindent
\begin{multline}
\label{5-5-2-m}
v_{n}(s;x,y) = d_0(x,y) t_{n}(s;x) +{}\\
{}+
\d_0(x,y) \left[ u_{n+1}(s;x,y) t_{n-1}(s;x) +{}\right.\\
\left.{}+ u^*_{n+1}(s;x,y) t(s;x)\right]\,,
\ \ n=\overline{n_1+1,n_0-2}\,;
\end{multline}
%%%%%%%%%%%%%

\vspace*{-20pt}

\noindent
\begin{multline*}
v_{n}(s;x,y)= d_0(x,y) t_{n}(s;x) + {}\\
{}+\d_0(x,y) u_{n+1}(s;x,y) t_{n-1}(s;x)\,,
\ \ n=\overline{1,n_1}\,.
\end{multline*}
%%%%%%%%%%%%%

Стационарное распределение общего времени пребывания
заявки в системе имеет ПЛС
\begin{multline*}
v(s) = \fr{1}{\lambda} \left[
\lambda_0 p_0 \int\limits_0^\infty b_0(x) t_0(s;x) \, dx
+ {}\right.\\
{}+\lambda_0 \int\limits_0^\infty \sum\limits_{n=1}^{n_1} p_n(y) \, dy
\int\limits_0^\infty b_0(x) v_n(s;x,y) \, dx +{}
\\
{}+
\lambda_0 \int\limits_0^\infty \sum\limits_{n=n_1+1}^{n_0-1}
p_n(0;y)\, dy \int\limits_0^\infty b_0(x) v_n(s;x,y) \, dx
+{}
\\
{}+
\lambda_1 \int\limits_0^\infty \sum\limits_{n=n_1+1}^{n_0-1}
p_n(1;y)\, dy \int\limits_0^\infty b_1(x) v(s;x,y) \, dx
+{}\\
\left. \lambda_1 \int\limits_0^\infty \sum\limits_{n=n_0}^{\infty}
p_n(y)\, dy \int\limits_0^\infty b_1(x) v(s;x,y)\, dx
\right]\,.
\end{multline*}

Дифференцируя $w(s)$ и $v(s)$ в точке $s\hm=0$,
можно найти моменты стационарных распределений времен
ожидания начала обслуживания и пребывания заявки в
сис\-теме.

\section{Накопитель конечной емкости}

В этом разделе будет показано, какие изменения нужно
произвести в полученных формулах для случая накопителя
конечной емкости.
Заметим, что к формулам, остающимся без изменений,
комментарии приводиться не будут.


Итак, будем предполагать, что максимальное число заявок,
находящихся в системе, равно $n^*$, $n^*\hm\ge n_0$,
(емкость накопителя $n^*\hm-1$).

Для конечного накопителя необходимо также задать
дисциплину принятия заявок в систему при отсутствии в нем
свободных мест.
В~соответствии с рассматриваемой СМО естественно такую
дисциплину определить с помощью функции $d^*(x,y)$
следующим образом: поступающая заявка длины~$x$,\linebreak
застающая на приборе заявку длины~$y$, с ве\-ро\-ят\-ностью
$d^*(x,y)$ сразу же покидает сис\-те\-му, не оказывая на нее
никакого воздействия, и с дополнительной вероятностью
$\d^*(x,y)\hm=1\hm-d^*(x,y)$ становит\-ся на прибор, вытесняя
заявку на приборе из сис\-те\-мы.
Для всех СМО с такой дисциплиной принятия заявок в сис\-те\-му
при отсутствии в накопителе свободных мест стационарные
вероятности $p_n(x_1,\ldots,x_n)$ при $n\hm<n^*$ совпадают
с точ\-ностью до постоянной с аналогичными вероятностями
для сис\-те\-мы с бесконечным накопителем, различие заключается
только в вероятностях $p_{n^*}(x_1,\ldots,x_{n^*})$.
Однако несколько более сложно вычисляются стационарные
распределения, связанные с временем пребывания заявки в
системе.
Более того, заявки, принятые в систему, могут покидать
ее недообслуженными.

Здесь для простоты изложения будет рассмотрен только
случай $d^*(x,y)\hm=1$, т.\,е.\ тот случай, когда поступающая
в заполненную систему заявка теряется.
Заметим, что в этом случае принятая в систему заявка
будет обязательно обслужена полностью.
Общий случай нетрудно исследовать с помощью результатов,
полученных в~[9, 10].

Далее будем предполагать, что $n^*\hm\ge n_0 \hm+ 2$, поскольку
при $n^*\hm=n_0$ и $n^*\hm=n_0 \hm+ 1$ расчетные формулы будут
несколько отличаться от приведенных выше.

Как уже говорилось, стационарные вероятности
$p_n(x)$, $n\hm=\overline{1,n_1}$ или
$n\hm=\overline{n_0,n^*-1}$, и
$p_n(i;x)$, $n\hm=\overline{n_1+1,n_0-1}$, $i\hm=1,2$,
с точностью до вероятности $p_0$ можно определить
из тех же самых уравнений~(\ref{3-0-m})--(\ref{3-4-m}),
что и раньше.
Вероятность $p_{n^*}(x)$ удовлетворяет дифференциальному
уравнению
\begin{equation*}
-p'_{n^*}(x)= g_{1,n^*}(x) 
\end{equation*}
с начальным условием
%%%%%%%%%%%%%%%%%%%%%%%%%%%%%%
\begin{equation*}
%\label{3-beg-1}
\lim\limits_{x\to\infty} p_{n^*}(x) = 0\,,
\end{equation*}
%%%%%%%%%%%%%%%%%%%%%%%%%%%%%%
где
%%%%%%%%%%%%%%%%%%%%%%%%%%
\begin{multline*}
%\label{3-gf-3}
g_{1,n^*}(x) = \lambda_1 b_1(x) \int\limits_0^\infty p_{n^*-1}(y) d_1(x,y)\, dy
+{}\\
{}+ \lambda_1 p_{n^*-1}(x) \int\limits_0^\infty b_1(y) \d_1(y,x)\, dy\,.
\end{multline*}
%%%%%%%%%%%%%%%%%%%%%%%%%%%%%%%
Решение этого уравнения определяется вы\-ра\-же\-нием:
\begin{equation*}
%\label{3-4}
p_{n^*}(x)= \int\limits_x^\infty g_{1,n^*}(y)\, dy\,.
\end{equation*}
%%%%%%%%%%%%%%%%%%%%%%%%%
Вероятность $p_0$ вычисляется из условия нормировки,
которое в данном случае имеет вид:
$$
p_0 + \sum\limits_{n=1}^{n_1} p_n + \sum\limits_{n=n_1+1}^{n_0-1} \left[p_{n,0} + p_{n,1}\right]+
\sum\limits_{n=n_0}^{n^*} p_n = 1\,.
$$

Стационарная интенсивность~$\lambda$ входящего в сис\-те\-му
потока задается формулой~(\ref{inten-1-m}), в которой,
естественно, верхний индекс~$\infty$ в последней сумме
заменен на~$n^*$.

В системах с конечным накопителем важной характеристикой
является стационарная вероятность $\pi_{\mathrm{loss}}$
потери заявки, определяемая формулой:
$$
\pi_{\mathrm{loss}} = \fr{\lambda_1 }{\lambda}\, p_{n^*}\,.
$$

Для того чтобы найти показатели функционирования СМО,
связанные с временем пребывания в системе, нужно прежде
всего изменить некоторые формулы для ПЛС~ПЗ.

Предположим, что в начальный момент рас\-смат\-ри\-ва\-емая
СМО функционирует в режиме~1 и в ней находится
$n$, $n\hm=\overline{n_1+1,n^*}$, заявок.
Обозначим через $\tilde{u}_n(s;x)$, $n\hm=\overline{n_1+1,n^*}$, ПЛС времени до того момента,
когда в системе впервые останется $(n-1)$ заявок,
при условии что на приборе начала
обслуживаться заявка длины~$x$ (очевидно, что в этот
момент система по-преж\-не\-му будет функционировать в режиме~1).
Преобразования Лап\-ла\-са--Стил\-тье\-са $\tilde{u}_n(s;x)$ удовлетворяют уравнениям 
\begin{align*}
\tilde{u}_{n^*}(s;x)&= e^{-sx} \,;\\
\\
\tilde{u}'_n(s;x) &= - (s + \lambda_1) \tilde{u}_{n}(s;x) +{}
\\
&{}+
\lambda_1 \int\limits_0^\infty b_1(y) \left[d_1(y,x) \tilde{u}_{n+1}(s;y) \tilde{u}_{n}(s;x)
+ {}\right.\\
&\hspace*{5mm}\left.{}+\d_1(y,x) \tilde{u}_{n+1}(s;x) \tilde{u}_{n}(s;y)\right] \, dy\,,
\\  
&\hspace*{35mm}n=\overline{n_{1}+1,n^*-1}\,,
\end{align*}
%%%%%%%%%
с начальным условием
$$
\tilde{u}_n(s;0)=1\,,
\ \ n=\overline{n_1+1,n^*-1}\,.
$$
Уравнения~(\ref{5-2-m})--(\ref{5-2-3-m}) принимают следующий вид:
%%%%%%%%%%%
\begin{multline*}
%\label{5-2}
u^{*\,\prime}_{n_0-1}(s;x) = - \left[s + \lambda_0\right] u^*_{n_0-1}(s;x)
+{}
\\
{}+
\lambda_0 \int\limits_0^\infty b_0(y) \left[d_0(y,x) \tilde{u}_{n_0}(s;y) \tilde{u}_{n_0-1}(s;x)
+{}\right.\\
\left.{}+ \d_0(y,x) \tilde{u}_{n_0}(s;x) \tilde{u}_{n_0-1}(s;y)\right]\, dy \,;
\end{multline*}
%%%%%%%%%%%

\vspace*{-12pt}

\noindent
\begin{multline*}
u^{*\,\prime}_n(s;x) = - \left[s + \lambda_0\right] u^*_{n}(s;x)
+{}
\\
{}+ \lambda_0 \int\limits_0^\infty b_0(y) \left[d_0(y,x) u^*_{n+1}(s;y) \tilde{u}_n(s;x)
+ {}\right.\\
\left.{}+\d_0(y,x) u^*_{n+1}(s;x) \tilde{u}_n(s;y)\right] \, dy +{}\\
{}+
\lambda_0 \int\limits_0^\infty b_0(y) \left[d_0(y,x) u_{n+1}(s;y) u^*_{n}(s;x)
+{}\right.\\
\left.{}+\d_0(y,x) u_{n+1}(s;x) u^*_{n}(s;y)\right]\, dy\,,
\\ n=\overline{n_1+2,n_0-2}\,;
\end{multline*}
%%%%%%%%%

\vspace*{-24pt}

\noindent
\begin{multline*}
u'_{n_1+1}(s;x) = - \left[s + \lambda_0\right] u_{n_1+1}(s;x) +{}
\\
{}+
\lambda_0 \int\limits_0^\infty b_0(y) \left[d_0(y,x) u^*_{n_1+2}(s;y) \tilde{u}_{n_1+1}(s;x)
+ {}\right.\\
\left.{}+\d_0(y,x) u^*_{n_1+2}(s;x) \tilde{u}_{n_1+1}(s;y)\right] \, dy+{}
\\
{}+
\lambda_0 \int\limits_0^\infty b_0(y) \left[d_0(y,x) u_{n_1+2}(s;y) u_{n_1+1}(s;x)
+ {}\right.\\
\left.{}+\d_0(y,x) u_{n_1+2}(s;x) u_{n_1+1}(s;y)\right] \, dy\,.
\end{multline*}
%%%%%%%%%
Соответственно изменится и формула~(\ref{5-2-4-m}).

Пусть в начальный момент в системе находится
$n$, $n\hm=\overline{n_1+1,n^*-1}$, заявок, система
функционирует в режиме~1, на приборе обслуживается
заявка длины~$y$ и в этот момент в систему поступает
заявка длины~$x$.
Обозначим через $\tilde{w}_n(s;x,y)$ ПЛС времени ожидания
начала обслуживания этой заявки.
Имеет место равенство:
\begin{multline*}
\tilde{w}_n(s;x,y) = d_1(x,y) + \d_1(x,y) \tilde{u}_{n+1}(s;y)\,,\\ 
n=\overline{n_1+1,n^*-1}\,.
\end{multline*}
Формула~(\ref{5-3-3-m}) %и (\ref{5-3-4})
принимает вид:
\begin{equation*}
%\label{5-3-3}
w^*_{n_0-1}(s;x,y) = d_0(x,y) + \d_0(x,y) \tilde{u}_{n_0}(s;y) \,,
\end{equation*}
%%%%%%%%%
а ПЛС стационарного распределения времени ожидания начала
обслуживания принятой в систему заявки определяется
формулой:
\begin{multline*}
%\label{5-3-4}
w(s) = \fr{1}{\lambda \left( 
1-\pi_{\mathrm{loss}}\right)}
\left[ \vphantom{\int\limits_0^\infty}
\lambda_0 p_0 + {}\right.\\
{}+\lambda_0 \int\limits_0^\infty \sum\limits_{n=1}^{n_1}
p_n(y) \, dy \int\limits_0^\infty b_0(x) w_n(s;x,y) \, dx +{}
\\
{}+
\lambda_0 \int\limits_0^\infty \sum\limits_{n=n_1+1}^{n_0-1} p_n(0;y) \, dy
\int\limits_0^\infty b_0(x) \left[w_n(s;x,y) +{}\right.\\
\left.{}+ w^*_n(s;x,y)
\right]
\, dx +{}\\
{}+ \lambda_1 \int\limits_0^\infty \sum\limits_{n=n_1+1}^{n_0-1}
p_n(1;y) \, dy \int\limits_0^\infty b_1(x) \tilde{w}_n(s;x,y) \, dx +{}\\
\left.{}+
\lambda_1 \int\limits_0^\infty \sum\limits_{n=n_0}^{n^*-1} p_n(y) \, dy
\int\limits_0^\infty b_1(x) \tilde{w}_n(s;x,y) \, dx\right]\,.
\end{multline*}


Обозначим через $\tilde{t}_n(s;x)$,\  $n\hm=\overline{n_1,n^*-1}$,
ПЛС времени от момента первого попадания заявки длины~$x$
на прибор до момента ухода ее из системы при условии,
что в момент первого попадания на прибор в очереди
было еще $n$~заявок и система функционировала в
режиме~1.
Тогда
%%%%%%%%%%%%%%%%%%%
$$
\tilde{t}_{n^*-1}(s;x) = e^{-sx}\,;
$$
%%%%%%%%%%%%%%%%

\vspace*{-12pt}

\noindent
\begin{multline*}
\tilde{t}_n(s;x) = \exp\left\{ \vphantom{\int\limits_0^x}
- (s + \lambda_1) x + {}\right.\\
{}+\lambda_1 \int\limits_0^x \,dz
\int\limits_0^\infty b_1(y) \left[\,\d_1(y,z) +{}\right.\\
\left.\left.{}+ d_1(y,z) \tilde{u}_{n+2}(s;y)\right] \, dy
\vphantom{\int\limits_0^\infty}
\right\}\,,\
n=\overline{n_1,n^*-2}\,.
\end{multline*}
При этом формулы~(\ref{5-4-1-m}) и~(\ref{5-4-2-m}) записываются
в виде:
\begin{multline*}
t_{n_0-2}(s;x)= e^{-(s + \lambda_0) x} \left( 
1 +
\lambda_0 \int\limits_0^x e^{(s + \lambda_0) z}\, dz\times{} \right.\\
{}\times
\int\limits_0^\infty b_0(y) \left[d_0(y,z) \tilde{u}_{n_0}(s;y) \tilde{t}_{n_0-2}(s;z)
+ {}\right.\\
\left.\left.{}+\d_0(y,z) \tilde{t}_{n_0-1}(s;z)\right] \, dy
\vphantom{\int\limits_0^x}
\right)\,;
\end{multline*}
%%%%%%%%%%%%%%%%%%%

\vspace*{-12pt}

\noindent
\begin{multline*}
%\label{5-4-2}
t_n(s;x) = {}\\
{}=e^{- \int\limits_0^x \left(
s + \lambda_0 - \lambda_0 \int\limits_0^\infty b_0(y) d_0(y,z) u_{n+2}(s;y) \, dy\right)dz}
\left(  \vphantom{\int\limits_0^x}
1 +{}\right.\\
{}+ \lambda_0 \int\limits_0^x e^{ \int\limits_0^v \left(
s + \lambda_0 - \lambda_0 \int\limits_0^\infty b_0(y) d_0(y,z) u_{n+2}(s;y) \, dy \right)dz}
dv \times{}\\
{}\times
\int\limits_0^\infty b_0(y) \left[d_0(y,v) u_{n+2}^*(s;y) \tilde{t}_n(s;v)
+{}\right.\\
\left.\left.{}+ \d_0(y,v) t_{n+1}(s;v)\right] \, dy 
\vphantom{\int\limits_0^x}
\right)\,,
\ \ n=\overline{n_1,n_0-3}\,.
\end{multline*}
%%%%%%%%%%%%%%%%%%%

Наконец, перейдем к общему времени пребывания заявки в
системе.
Обозначим через
$\tilde{v}_{n}(s;x,y)$, $n\hm=\overline{n_1+1,n^*-1}$,
ПЛС времени пребывания в сис\-те\-ме заявки длины~$x$ при
условии, что эта заявка застала в системе $n$~других
заявок, причем заявка на приборе имела длину~$y$, а
система пребывала в режиме~1.
Справедливо соотношение:
\begin{multline*}
\tilde{v}_{n}(s;x,y) = d_1(x,y) \tilde{t}_{n}(s;x) +{}\\[2pt]
{}+\d_1(x,y) \tilde{u}_{n+1}(s;x,y) \tilde{t}_{n-1}(s;x)\,,\\[2pt]
  n=\overline{n_1+1,n^*-1}\,.
\end{multline*}
%%%%%%%%%%%%%
Формулы~(\ref{5-5-1-m}) и~(\ref{5-5-2-m}) преобразуются
следующим образом:
\begin{multline*}
%\label{5-5-1}
v_{n_0-1}(s;x,y)= d_0(x,y) \tilde{t}_{n_0-1}(s;x) +{}\\
{}+
\d_0(x,y) \tilde{u}_{n_0}(s;x,y) \tilde{t}_{n_0-2}(s;x) \,;
\end{multline*}
%%%%%%%%%%%%%
\vspace*{-12pt}

\noindent
\begin{multline*}
%\label{5-5-2}
v_{n}(s;x,y) = d_0(x,y) t_{n}(s;x) +{}
\\
{}+
\d_0(x,y) \left[ u_{n+1}(s;x,y) t_{n-1}(s;x) +{}\right.\\
\left.{}+
u^*_{n+1}(s;x,y) \tilde{t}_{n-1}(s;x) \right] \,,
\ \ n=\overline{n_1+1,n_0-2}\,,
\end{multline*}
а ПЛС стационарного распределения общего времени пребывания
в системе принятой к обслуживанию заявки имеет вид:
\begin{multline*}
v(s)= \fr{1}{ \lambda \left(1-\pi_{\mathrm{loss}}\right)}
\left[
\lambda_0 p_0 \int\limits_0^\infty b_0(x) t_0(s;x) \, dx
+{}\right.\\
\left.{}+ \lambda_0 \int\limits_0^\infty \sum\limits_{n=1}^{n_1}
p_n(y) \, dy \int\limits_0^\infty b_0(x) v_n(s;x,y) \, dx +{}\right.
\\
{}+
\lambda_0 \int\limits_0^\infty \sum\limits_{n=n_1+1}^{n_0-1}
p_n(0;y)\, dy \int\limits_0^\infty b_0(x) v_n(s;x,y) \, dx +{}\\
{}+ \lambda_1 \int\limits_0^\infty \sum\limits_{n=n_1+1}^{n_0-1}
p_n(1;y)\, dy \int\limits_0^\infty b_1(x) \tilde{v}_n(s;x,y) \, dx+{}\\
\left.{}+ \lambda_1 \int\limits_0^\infty \sum\limits_{n=n_0}^{n^*-1} p_n(y)\, dy
\int\limits_0^\infty b_1(x) \tilde{v}_n(s;x,y)\, dx
\right]\,.
\end{multline*}

\section{Заключение}


В настоящей статье рассмотрена возможность\linebreak применения
аналитических методов для вы\-чис\-ле\-ния основных стационарных
характеристик функ\-ци\-о\-ни\-ро\-ва\-ния СМО, в которых
одновременно имеется два отличия от
классических СМО:\linebreak инверсионный порядок обслуживания с
вероятностным приоритетом и гистерезисная политика.
На примере однолинейной СМО с простейшим вариантом
гистерезисной политики показано, что полученные
вычислительные алгоритмы основаны на интегральных
и дифференциальных уравнениях, которые могут быть решены
на современной вы\-чис\-ли\-тель\-ной технике.
Приведены условия, при которых интегральные уравнения
могут быть сведены к системам линейных алгебраических
уравнений.

Полученные результаты могут служить основой для
продолжения работ в части математического моделирования
технических систем, ис\-поль\-зу\-ющих как элементы
<<нестандартных>> дисциплин обслуживания, так и сложные
варианты гистерезисного механизма предотвращения
различного рода перегрузок в ИТС.


{\small\frenchspacing
{%\baselineskip=10.8pt
\addcontentsline{toc}{section}{Литература}
\begin{thebibliography}{99}

\bibitem{1-m}
\Au{Печинкин А.\,В.} Об одной инвариантной системе массового
обслуживания~// Math.\ Operationsforsch.\ und Statist. Ser.\
Optimization, 1983. Vol.~14. No.\,3. S.~433--444.

\bibitem{2-m}
\Au{Печинкин А.\,В., Стальченко И.\,В.} Система $MAP/G/1/\infty$ с
инверсионным порядком обслуживания и вероятностным приоритетом,
функционирующая в дискретном времени~// Вестник Российского
ун-та дружбы народов. Сер.\ Математика. Информатика. Физика,
2010. №\,2. С.~26--36.

\bibitem{3-m}
\Au{Абаев П.\,О., Гайдамака Ю.\,В., Самуйлов~К.\,Е.} Гистерезисное
управление сигнальной нагрузкой в сети SIP-сер\-ве\-ров~// Вестник
Российского ун-та дружбы народов. Сер.\ Математика.
Информатика. Физика, 2011. №\,4. С.~54--71.

\bibitem{7-m}
\Au{Nishimura S., Jiang~Y.}
An $M/G/1$ vacation model with two service modes~//
Prob.\ Eng. Informational Sci., 1995. Vol.~9. No.\,3. P.~355--374.

\bibitem{8-m}
\Au{Dudin A.}
Optimal control for an $M^x/G/1$ queue with two operation
modes~// Prob. Eng.  Informational Sci.,
1997. Vol.~11. No.\,2. P.~255--265.

\bibitem{9-m}
\Au{Nobel R.\,D., Tijms H.\,C.} Optimal control for an $M^X/G/1$
queue with two service mo\-des~// Eur. J.~Operational
Res., 1999. Vol.~113. No.\,3. P.~610--619.

\bibitem{10-m}
\Au{Жерновый К.\,Ю., Жерновый Ю.\,В.} Система $M^\theta/G/1/m$ c
двухпороговой гистерезисной стратегией переключения интенсивности
обслуживания~// Информационные процессы, 2012. Т.~12. №\,2. С.~127--140.

\bibitem{22-m}
\Au{Bocharov~P.\,P., D'Apice~C., Pechinkin~A.\,V., Salerno~S.}
Queueing theory.~--- Ut\-recht, Boston: VSP, 2004.

\bibitem{5-m}
\Au{Нагоненко В.\,А.} О~характеристиках одной нестандартной сис\-те\-мы
массового обслуживания. I~// Изв.\ АН СССР. Технич.\ кибернет.,
1981. №\,1. С.~187--195.

\label{end\stat}

\bibitem{6-m}
\Au{Нагоненко В.\,А.} О~характеристиках одной нестандартной сис\-те\-мы
массового обслуживания. II~// Изв.\ АН СССР. Технич.\ кибернет.,
1981. №\,3. С.~91--99.
\end{thebibliography}
}
}

\end{multicols}