\documentclass[10pt]{book}
\usepackage[utf8]{inputenc}

\usepackage{latexsym,amssymb,amsfonts,amsmath,indentfirst,shapepar,%fleqn,%
picinpar,shadow,floatflt,enumerate,multicol,colortbl,ipi}

\usepackage{rotating}
\usepackage{mathrsfs}
\usepackage[noend]{algorithmic}
\usepackage{ulem}

\input{epsf}

%\nofiles

%\includeonly{obchak,avtor,avtor-eng} %+pdf
%\includeonly{obchak,avtor}
%\includeonly{pred}      %+pdf
%\includeonly{podgot-1str}  %+
%\includeonly{ocherk} %+

%\includeonly{sinitsin}  %1Abst+pdf
%\includeonly{bening}   %Abst+pdf  
%\includeonly{korolev}  %+pdf
%\includeonly{zatsman} %+pdf
%\includeonly{grusho}  %pdf
%\includeonly{morozova} %pdf
%\includeonly{kozerenko} %непечатные символ+pdf
%\includeonly{lukmor}  %Abst+pdf
%\includeonly{rudoi}   %ABst+pdf
%\includeonly{gudasa}  %+pdf
%\includeonly{chertok}  %Abst+pdf
%\includeonly{kuzn}    %11+pdf
%\includeonly{milovanova}  %3Abst+pdf
%\includeonly{shevts}  %+pdf
 


%\includeonly{toc-rus, toc-en}
%\includeonly{obchak} %,toc-en}

%\includeonly{obchak}
%\includeonly{reshal}  %pdf
%\includeonly{eng-index}
%\includeonly{cover3}

\usepackage{acad}
%\usepackage{courier}
\usepackage{decor}
\usepackage{newton}
\usepackage{pragmatica}
\usepackage{zapfchan}
\usepackage{petrotex}
\usepackage{bm}                     % полужирные греческие буквы
\usepackage{upgreek}                % прямые греческие буквы
\usepackage{eufrak}
%\usepackage{verbatim}

\renewcommand{\bottomfraction}{0.99}
\renewcommand{\topfraction}{0.99}
\renewcommand{\textfraction}{0.01}

\setcounter{secnumdepth}{1} %здесь - 3 + chapter = 4

\arraycolsep=1.5pt

%\usepackage[pdftex]{graphicx}

%\usepackage{oz}

%NEW COMMANDS


\renewcommand*{\hm}[1]{#1\nobreak\discretionary{}%
            {\hbox{$\mathsurround=0pt #1$}}{}} %% Дублирует знаки операций
                               %при переносе в формуле (перед знаком, который 
                               %надо продублировать ставится команда \hm)

%\newcommand{\endproof}{\hfill$\Box$}
\renewcommand{\r}{\mathbb{R}}
\newcommand{\I}{{\rm I\hspace{-0.7mm}I}}
%\newcommand{\Ikl}{{\tt{1}}\hspace*{-1.44mm}\mathtt{1}}
\newcommand{\Ik}{\mbox{{\small \tt {1}}\hspace{-1.5mm}{\tt 1}}}
\newcommand{\argmin}{\mathop{\mathrm{arg}\,\mathrm{min}}}
\newcommand{\argmax}{\mathop{\mathrm{arg}\,\mathrm{max}}}
%\newcommand{\capr}{\mathop{\cap\,}}
%\newcommand{\cupr}{\mathop{\cup\,}}
%\def\argmin{\mathop{arg\,min}}

\def\vrp{\varphi}
\def\prt{\partial}
\def\mm{{\rm M}}

\newcommand{\il}[2]{\int\limits_{#1}^{#2}}%интеграл с пределами #1 и #2


\def\sss{\sum\limits}
\def\tr{,\,\ldots\,,\,}
\def\rk{\right]}
\def\lk{\left[}
\def\rf{\right\}}
\def\lf{\left\{}

\def\ee{{\cal E}}
\def\ww{{\cal W}}
\def\yy{{\cal Y}}
\def\vv{{\cal V}}

\newcommand{\R}{\mathbb R}
\newcommand{\N}{\mathbb N}

\newcommand{\h}{{\bf H}}
\newcommand{\p}{{\sf P}}  % вероятность

\newcommand{\e}{{\sf E}}  % мат. ожидание
\newcommand{\D}{{\sf D}}  % дисперсия
\newcommand{\eps}{\varepsilon}
\newcommand{\vp}{{\mathbf p}}
\newcommand{\vz}{{\mathbf z}}
\newcommand{\vx}{{\mathbf x}}
\newcommand{\vf}{{\mathbf f}}
%\newcommand{\vp}{\mathrm{v.p.}}
\newcommand{\F}{{\mathcal F}}
\def\ap{{\mathrm{ЭР}}}
\newcommand{\ud}{\Delta_n} %uniform ditance
\newcommand{\nud}{\Delta_n(x)}

\newcommand{\abs}[1]{\left\vert#1\right\vert}
\def\w{\omega}
\def\W{\Omega}
\def\iii{\int\limits}
\def\iin{\int\limits_{-\infty}^\infty}


\DeclareMathOperator{\sign}{sign}

%\newcommand{\gr}{{\geqslant}}

\newcommand{\g}{\mbox{\textit{g}}}

\renewcommand{\la}{\lambda}
\newcommand{\si}{\sigma}
\newcommand{\alp}{\alpha}

%\newcommand{\pto}{\stackrel{P}{\longrightarrow}} % сходимость по веpоятности

\newcommand{\eqd}{\stackrel{d}{=}} % равенство по pаспpеделению

%\newcommand{\kp}{\kappa}
%\def\Q{{\cal Q}} \def\H{{\cal H}}
%\newcommand{\bet}{\beta_{2+\delta}}


%\newtheorem{definition}{Определение}
%\renewcommand{\thedefinition}{\arabic{definition}.}
%END NEW COMMANDS

%\renewcommand{\baselinestretch}{1.2}

%\pagestyle{myheadings}

\setlength{\textwidth}{167mm}      % 122mm
\setlength{\textheight}{658pt}
%\setlength{\textheight}{635.6pt}
\setlength{\columnsep}{4.5mm}

\setcounter{secnumdepth}{4}

%\addtolength{\headheight}{2pt}
%\addtolength{\headsep}{-2mm}

%\addtolength{\topmargin}{-20mm}  % for printing


\hoffset=-30mm  % From Yap
%\hoffset=-20mm  % From Acrobat

%\voffset=0mm % From Yap
%\voffset=-15mm   % From Acrobat

\addtolength{\evensidemargin}{-9.5mm} % for printing
\addtolength{\oddsidemargin}{9.5mm}  % for printing

%\renewcommand{\thefootnote}{\fnsymbol{footnote}}
%\renewcommand{\thefootnote}{\arabic{footnote}}
\renewcommand{\figurename}{\protect\bf Рис.}
\renewcommand{\tablename}{\protect\bf Таблица}

\newcommand{\Caption}[1]{\caption{\protect\small %\baselineskip=2.5ex
#1}}

\renewcommand{\thefigure}{\arabic{figure}}
\renewcommand{\thetable}{\arabic{table}}
\renewcommand{\theequation}{\arabic{equation}}
\renewcommand{\thesection}{\arabic{section}}

\renewcommand{\contentsname}{СОДЕРЖАНИЕ}
\newcommand{\fr}[2]{\displaystyle\frac{\displaystyle #1\mathstrut}{\displaystyle #2\mathstrut}}

%\renewcommand{\thefootnote}{\fnsymbol{footnote}}
%\newcommand{\g}{\mbox{\textit{g}}}

%\newcommand{\Caption}[1]{\caption{\protect\small\baselineskip=2ex #1}}
\newcounter{razdel}
\setcounter{razdel}{0}


\newcommand{\titel}[4]{%
\

\vspace*{5pt}

\ifodd\therazdel {\raggedright\noindent\Large\textrm\textbf
 \lineskip .75em
  \baselineskip=3.2ex #1 \par}
\vskip 1em {\noindent\large\textrm\textbf #2 \par}
\addcontentsline{toc}{subsection}{{\textrm\textbf #3}\protect\newline #1}
\def\rightheadline{\underline{\noindent\hbox to \textwidth{\hfill\small\textrm{#4}
%\hfill \large\bf\thepage
}}}
\def\leftheadline{\underline{\noindent\parbox{\textwidth}{
%\raggedleft\large\bf\thepage \hfill
\small\textit{#3}\hfill}}}
\def\leftfootline{\small{\textbf{\thepage}
\hfill ИНФОРМАТИКА И ЕЁ ПРИМЕНЕНИЯ\ \ \ том~7\ \ \ выпуск 1\ \ \ 2013}
}%
 \def\rightfootline{\small{ИНФОРМАТИКА И ЕЁ ПРИМЕНЕНИЯ\ \ \ том~7\ \ \ выпуск~1\ \ \ 2013
\hfill \textbf{\thepage}}} 
\vskip 2em \setcounter{figure}{0}
\setcounter{table}{0} 
\setcounter{equation}{0} 
\setcounter{section}{0}
\setcounter{subsection}{0} 
\setcounter{subsubsection}{0}
\setcounter{footnote}{0} 
\setcounter{razdel}{0}
%\end{flushleft}
\else {
 \raggedright\noindent\Large\textrm\textbf
 \lineskip .75em
\baselineskip=3.2ex #1 \par} \vskip 1em
%\begin{flushleft}
{\noindent\large\textrm\textbf #2 \par}
\addcontentsline{toc}{subsection}{{\textrm\textbf #3}\protect\newline #1}
\def\rightheadline{\underline{\noindent\hbox to \textwidth{\hfill\small\textrm{#4}
%\hfill \large\bf\thepage
}}}
\def\leftheadline{\underline{\noindent\parbox{\textwidth}{%\raggedleft\large\bf\thepage \hfill
\small\textit{#3}\hfill}}}
\def\leftfootline{\small{\textbf{\thepage}
\hfill ИНФОРМАТИКА И ЕЁ ПРИМЕНЕНИЯ\ \ \ том~7\ \ \ выпуск~1\ \ \ 2013}
}%
 \def\rightfootline{\small{ИНФОРМАТИКА И ЕЁ ПРИМЕНЕНИЯ\ \ \ том~7\ \ \ выпуск~1\ \ \ 2013
\hfill \textbf{\thepage}}} \vskip 2em \setcounter{figure}{0}
\setcounter{table}{0} \setcounter{equation}{0} \setcounter{section}{0}
\setcounter{subsection}{0} \setcounter{subsubsection}{0}
\setcounter{footnote}{0}
%\end{flushleft}
\fi}

\newcommand{\titelr}[2]{%
\

\vspace*{5pt}

\ifodd\therazdel {\raggedright\noindent\large\textrm\textbf
 \lineskip .75em
  \baselineskip=3.2ex #1 \par}
\vskip 1em {\noindent\normalsize\textrm\textbf #2 \par}
\else {
 \raggedright\noindent\large\textrm\textbf
 \lineskip .75em
\baselineskip=3.2ex #1 \par} \vskip 1em
%\begin{flushleft}
{\noindent\normalsize\textrm\textbf #2 \par}
\fi}

\newcommand{\titele}[5]{%
\

%\vspace*{5pt}

\ifodd\therazdel {\raggedright\noindent%\large
\textrm\textbf
 \lineskip .75em
%  \baselineskip=3.2ex
#1 \par}
\vskip .5em {\noindent\large\textrm\textbf #2 \par}
\vskip .5em
 {\noindent\textrm #3 \par}
\addcontentsline{toc}{subsection}{{\textrm\textbf #1}\protect\newline #2}
\def\rightheadline{\underline{\noindent\hbox to \textwidth{\hfill\small\textrm{#4}
%\hfill \large\bf\thepage
}}}
\def\leftheadline{\underline{\noindent\parbox{\textwidth}{
%\raggedleft\large\bf\thepage \hfill
\small\textrm{#5}\hfill}}}
\def\leftfootline{\small{\textbf{\thepage}
\hfill ИНФОРМАТИКА И ЕЁ ПРИМЕНЕНИЯ\ \ \ том~7\ \ \ выпуск~1\ \ \ 2013}
}%
 \def\rightfootline{\small{ИНФОРМАТИКА И ЕЁ ПРИМЕНЕНИЯ\ \ \ том~7\ \ \ выпуск~1\ \ \ 2013
\hfill \textbf{\thepage}}} \vskip 1em \setcounter{figure}{0}
\setcounter{table}{0} \setcounter{equation}{0} \setcounter{section}{0}
\setcounter{subsection}{0} \setcounter{subsubsection}{0}
\setcounter{footnote}{0} \setcounter{razdel}{0}
%\end{flushleft}
\else {
 \raggedright\noindent%\large
 \textrm\textbf
 \lineskip .75em
%\baselineskip=3.2ex
#1 \par} \vskip .5em
%\begin{flushleft}
{\noindent\large\textrm\textbf #2 \par} \vskip .5em
 {\noindent\textrm #3 \par}
\addcontentsline{toc}{subsection}{{\textrm\textbf #1}\protect\newline #2}
\def\rightheadline{\underline{\noindent\hbox to \textwidth{\hfill\small\textrm{#4}
%\hfill \large\bf\thepage
}}}
\def\leftheadline{\underline{\noindent\parbox{\textwidth}{%\raggedleft\large\bf\thepage \hfill
\small\textrm{#5}\hfill}}}
\def\leftfootline{\small{\textbf{\thepage}
\hfill ИНФОРМАТИКА И ЕЁ ПРИМЕНЕНИЯ\ \ \ том~7\ \ \ выпуск~1\ \ \ 2013}
}%
 \def\rightfootline{\small{ИНФОРМАТИКА И ЕЁ ПРИМЕНЕНИЯ\ \ \ том~7\ \ \ выпуск~1\ \ \ 2013
\hfill \textbf{\thepage}}} \vskip 1em \setcounter{figure}{0}
\setcounter{table}{0} \setcounter{equation}{0} \setcounter{section}{0}
\setcounter{subsection}{0} \setcounter{subsubsection}{0}
\setcounter{footnote}{0}
%\end{flushleft}
\fi}

\def\Abst#1{
\begin{center}\small\nwt
\parbox{150mm}{%\baselineskip=2.5ex
\textbf{Аннотация:}\ \
%\hspace*{\parindent}
#1}
\end{center}}
\def\Abste#1{
\begin{center}\small\nwt
\parbox{150mm}{%\baselineskip=2.5ex
\textbf{Abstract:}\ \
%\hspace*{\parindent}
#1}
\end{center}}

\def\KW#1{
\begin{center}\small\nwt
\parbox{150mm}{%\baselineskip=2.5ex
\textbf{Ключевые слова:}\ \ #1}
\end{center}}

\def\KWE#1{
\begin{center}\small\nwt
\parbox{150mm}{%\baselineskip=2.5ex
\textbf{Keywords:}\ \ #1}
\end{center}}


\def\KWN#1{
%\begin{center}
%\small
%\parbox{150mm}\end{center}
}

\renewcommand{\thesubsection}{\thesection.\arabic{subsection}\hspace*{-5pt}}
\renewcommand{\thesubsubsection}{\thesubsection\hspace*{5pt}.\arabic{subsubsection}\hspace*{-3pt}}

\begin{document}
\Rus

\nwt
%\ptb

%\renewcommand{\contentsname}{\protect\Large\bf Содержание}

\setcounter{tocdepth}{2}

%\tableofcontents

\renewcommand{\bibname}{\protect\rmfamily Литература}
  \def\Au#1{{\it #1}}

%\newcommand{\No}{№}
  \newcommand{\tg}{\,\mathrm{tg}\,}
    \newcommand{\ctg}{\,\mathrm{ctg}\,}
  \newcommand{\arctg}{\,\mathrm{arctg}\,}
  
\def\forallb{\mathop{\forall}}
\def\cupb{\mathop{\cup}}
\def\existsb{\mathop{\exists}}

\setcounter{page}{1}

\newpage
\addtocounter{razdel}{1}
%\def\razd{РЕГУЛИРУЕМЫЙ ЭЛЕКТРОПРИВОД ДЛЯ ЭЛЕКТРОЭНЕРГЕТИКИ}


\setcounter{page}{3}

%{ %\Large  
{ %\baselineskip=16.6pt

\vspace*{-48pt}
\begin{center}\LARGE
\textit{Уважаемый читатель!}
\end{center}

%\vspace*{2.5mm}

\vspace*{4mm}

\thispagestyle{empty}

{\small

 
В~2017~г.\ исполняется 10~лет со времени выхода в~свет первого 
номера журнала <<Информатика и~её применения>>~--- 
научного журнала Российской академии наук, издающегося под 
на\-уч\-но-ме\-то\-ди\-че\-ским руководством Отделения нанотехнологий 
и~информационных технологий Российской академии наук. Учредителем журнала 
является Федеральный исследовательский центр <<Информатика и~управ\-ле\-ние>> 
Российской академии наук (ФИЦ ИУ РАН) (до~2015~г.~--- 
Институт проб\-лем информатики РАН).

Необходимость издания такого журнала была вызвана активным развитием 
информатики и~информационных технологий, большой важностью этого научного 
направления для развития страны, проникновением информационных технологий 
во все сферы жизни современного общества.

Тематику журнала определяет тот факт, что информатика~--- это комплексная 
фундаментальная научная дисциплина, опирающаяся на достижения 
ряда других наук, в~том числе математики, физики, лингвистики и~др. 
Одновременно журнал уделяет большое внимание современным информационным технологиям, 
являющимся приложениями результатов информатики как фундаментальной науки.

За прошедшие 10~лет (2007--2016~гг.)\ издано~38~выпусков журнала. В~них 
размещено~452~публикации, в~том числе~430~научных статей и~22~информационных 
публикации (обзоры, рецензии и~др.). Среди авторов журнала представители ведущих 
научных организаций и~университетов страны, в~том числе Московского государственного 
университета им.\ М.\,В.~Ломоносова, ФИЦ ИУ РАН (в~том числе ИПИ РАН, ВЦ 
им.\ А.\,А.~Дородницына РАН, ИСА РАН), Института точной механики и~вычислительной 
техники им.\ С.\,А.~Лебедева РАН, Института космических исследований РАН, 
Института астрономии РАН, ряда институтов Сибирского отделения РАН, МФТИ, МИФИ, 
Высшей школы экономики, Санкт-Пе\-тер\-бург\-ско\-го государственного университета, 
Санкт-Пе\-тер\-бург\-ско\-го государственного политехнического университета 
Петра Великого, Санкт-Пе\-тер\-бург\-ско\-го государственного университета 
телекоммуникаций им.\ проф.\ М.\,А.~Бонч-Бруе\-ви\-ча, 
Российского университета дружбы народов, Балтийского федерального университета 
имени Иммануила Канта, Вологодского государственного университета и~др. 
Публиковались статьи зарубежных авторов, в~том числе ученых из Израиля, 
США, Финляндии, Франции, Швейцарии, Швеции и~других стран. 

В конце настоящего выпуска журнала помещен указатель статей, 
опуб\-ли\-ко\-ван\-ных в~томах~1--10 (2007--2016~гг.).

Журнал включен в~Российский индекс научного цитирования и~в~базу 
данных RSCI Web of Science, перечень ВАК, базу данных CrossRef 
и~информационную систему <<Общероссийский математический портал MathNet>>. 
С~2015~г.\ журнал индексируется в~библиографической и~реферативной базе 
данных SCOPUS.

Мы всегда будем помнить ушедших из жизни членов редакционного совета 
и~редакционной коллегии журнала: академика С.\,К.~Коровина, профессоров 
А.\,В.~Печинкина и~И.\,А.~Ушакова, которые внесли неоценимый вклад в~становление 
и~развитие журнала.

После объединения в~2015~г.\ трех учреждений Российской академии наук~--- 
Института проблем информатики, Вычислительного центра им.\ А.\,А.~Дородницына 
и~Института системного анализа~--- в~Федеральное государственное учреждение 
<<Федеральный исследовательский центр <<Информатика и~управ\-ле\-ние>> 
Российской академии наук>> (ФИЦ ИУ РАН) именно этот Центр стал базовой организацией 
для издания журнала, что существенно расширило как тематику журнала, 
так и~его возможности по привлечению новых авторов, в~том числе и~зарубежных.

В настоящее время тематику журнала в~первую очередь составляют:
\begin{itemize}
\item    теоретические основы информатики;\\[-14.5pt] 
\item    математические методы исследования сложных систем и~процессов;\\[-14.5pt]
\item    информационные системы и~сети;\\[-14.5pt]
\item    информационные технологии;\\[-14.5pt]
\item    архитектура и~программное обеспечение вычислительных комплексов и~сетей. 
\end{itemize}

Эти направления особенно важны в~связи с необходимостью решения задач 
формирования технологической базы инновационного развития, обеспечения 
на\-уч\-но-тех\-но\-ло\-ги\-че\-ско\-го прорыва в~области создания и~развития 
отечественных информационных и~коммуникационных технологий в~интересах 
достижения высокого качества и~стабильности систем управления и~предоставления 
услуг в~экономической и~социальной сферах. 

Мы, как и~ранее, приглашаем авторов представлять для публикации в~журнале 
статьи как с достижениями в~области теоретических проблем информатики, так 
и~с~изложением результатов ее практического приложения, а~также 
рецензии на наиболее интересные книжные новинки в~области информатики 
и~информационных технологий, объявления о~крупнейших международных 
и~всероссийских конференциях, различных научных мероприятиях 
по этой тематике и~другие информационные материалы.

Надеемся, что и~в~дальнейшем содержание статей, помещаемых в~журнале, 
будет вызывать интерес научной общественности. Редакционный совет, редколлегия 
и~редакция журнала, со своей стороны, сделают все для того, 
чтобы журнал и~впредь своевременно и~подробно информировал читателей 
о~новейших достижениях информатики и~ее актуальных практических приложениях.

                

      
\vfill
\noindent
Главный редактор журнала <<Информатика и~её применения>>,\\
академик  РАН\hfill
\textit{И.\,А.~Соколов}\\[-6pt]

%\noindent
%Редактор-составитель тематического выпуска, профессор кафедры математической статистики\\
%факультета вычислительной математики и~кибернетики МГУ им.~М.\,В.~Ломоносова,\\
%ведущий научный сотрудник ИПИ РАН, доктор физико-математических наук\hfill
% \textit{В.\,Ю.~Королев}


} }
}
      

\def\ed{\mathop{1}}

\def\stat{sinits}

\def\tit{МАТЕМАТИЧЕСКОЕ ОБЕСПЕЧЕНИЕ ДЛЯ АНАЛИЗА
НЕЛИНЕЙНЫХ МНОГОКАНАЛЬНЫХ КРУГОВЫХ СТОХАСТИЧЕСКИХ
СИСТЕМ, ОСНОВАННОЕ НА~ПАРАМЕТРИЗАЦИИ РАСПРЕДЕЛЕНИЙ$^*$}

\def\titkol{Математическое обеспечение для анализа
нелинейных многоканальных круговых стохастических
систем} %, основанное на параметризации распределений}

\def\autkol{И.\,Н. Синицын}
\def\aut{И.\,Н. Синицын$^1$}

\titel{\tit}{\aut}{\autkol}{\titkol}

{\renewcommand{\thefootnote}{\fnsymbol{footnote}}\footnotetext[1]
{Работа выполнена при финансовой поддержке РФФИ
(проект №\,10-07-00021).}}


\renewcommand{\thefootnote}{\arabic{footnote}}
\footnotetext[1]{Институт проблем информатики Российской академии наук, sinitsin@dol.ru}

\vspace*{-4pt}

\Abst{Статья посвящена теории и математическому обеспечению для 
анализа одно- и многомерных распределений процессов в многоканальных 
нелинейных круговых стохастических системах (КСтС) 
на базе методов параметризации распределений. Рассматриваются 
круговые ортогональные разложения (КОР) плот\-ностей круговых случайных величин (КСВ) и 
процессов, стохастические уравнения многоканальных нелинейных КСтС, 
интегродифференциальные уравнения для одно- и многомерных плотностей, общий метод КОР, 
а также методы <<намотанной>> нормальной аппроксимации (МННА), начальных и центральных моментов. 
Описаны модули инструментального программного обеспечения <<CStS-ANALYSIS>> в среде MATLAB. 
Результаты проиллюстрированы примером.}

\vspace*{-3pt}

\KW{аналитическое моделирование;
коэффициенты кругового ортогонального разложения;
круговая случайная величина;
круговой стохастических процесс;
<<намотанная>> нормальная плотность;
нелинейная многоканальная стохастическая сис\-те\-ма;
одно- и многомерные плотности распределения;
ортогональное разложение плотности;
эталонная плотность;
MATLAB; <<CStS-ANALYSIS>>}

\vspace*{-3pt}

 \vskip 14pt plus 9pt minus 6pt

      \thispagestyle{headings}

      \begin{multicols}{2}
      
            \label{st\stat}


\section{Введение}

В~[1--4] рассмотрено математическое обеспечение для 
спектрально-корреляционного анализа процессов в нелинейных КСтС, 
основанное на методе круговой статистической 
линеаризации посредством <<намотанного>> (\textit{wrapped}) нормального распределения.

Для анализа одно- и многомерных распределений в нелинейных 
КСтС невысокой размерности прямые методы решения интегродифференциальных уравнений Пугачёва 
для характеристических функций приводят к алгоритмам, требующим применения только средств 
суперкомпьютерной техники~\cite{5s}.

Как известно [6--8], для анализа стохастических процессов в многомерных нелинейных 
КСтС в евкли\-довом пространстве широкое применение получили методы параметризации 
распределений\linebreak (нормальной аппроксимации, начальных и центральных моментов, квазимоментов, 
а также ортогональ\-ных разложений и их модификаций). В~настоящее время создано инструментальное 
программное обеспечение в среде MATLAB~[3, 7, 8]. Применительно к нелинейным 
КСтС это математическое обеспечение требует развития.

Статья включает девять разделов. В~разд.~2 и~3 рассматриваются 
ортогональные разложения плотностей КСВ и стохастических процессов. 
В~разд.~4 приводятся стохастические уравнения нелинейной многоканальной стохастической сис\-те\-мы, 
а также интегродифференциальные уравнения для одно- и многомерных плотностей. 
Общее математические обеспечение <<CStS-ANALYSIS>> описано в разд.~5. В~разд.~6 
рассмотрено математическое обеспечение для метода круговой <<намотанной>> нормальной 
аппроксимации. В~разд.~7 приведены уравнения методов круговых начальных и центральных моментов. 
В~разд.~8 дан иллюстративный пример. Заключение содержит основные выводы.

\vspace*{-6pt}

\section{Ортогональное разложение плотности круговой случайной величины}

В задачах стохастической информатики широкое распространение получили модели 
распределений КСВ, основанные на следующих основных подходах:
\begin{itemize}
\item <<наматывание>> (\textit{wrapping}) распределений линейных СВ~$X$ на круг единичного радиуса 
$\Theta =X(\mathrm{mod}\,2\pi)$;
\item
преобразования к полярным координатам двумерного линейного распределения 
(так называемые  \textit{offset distributions}) или использование стереографической проекции;
\item
характеризации (параметризации) кругового распределения круговыми моментами, семиинвариантами 
и другими характеристиками, имеющими важное предметное значение, на основе ортогонального 
разложения плот\-ности по некоторой биортонормальной сис\-те\-ме функций с весом, 
задаваемым некоторой эталонной плотностью распределения.
\end{itemize}

Рассмотрим ортогональное разложение плот\-ности $f_\theta (\theta)$ для КСВ $\Theta$, 
обладающей конечными круговыми моментами, по некоторой известной биортогональной сис\-те\-ме 
функций  $\{ p_\nu (\theta), q_\nu (\theta)\}$ с весом  $w_\theta  (\theta)$ таким, что
    $$
    \iin w_\theta(\theta) p_\nu (\theta) q_\nu(\theta)\, d\theta= \delta_{\nu\mu}=
    \begin{cases}
    0\ \mbox{при\ }  &\nu\ne\mu\,;\\
    1\ \hbox{при\ }  &\nu=\mu\,,
    \end{cases}
    $$
следующего вида [6--8]:
\begin{equation}
    f_\theta(\theta) = w_\theta(\theta) \sss_\nu c_\nu p_\nu(\theta)\,,
    \label{e1s}
    \end{equation}
где
\begin{multline}
c_\nu =\iin f_\theta(\theta) q_\nu(\theta)\, d\theta =
\left[ q_\nu \left(\fr{\prt}{i\prt \lambda}\right) g(\lambda)\right]_{\lambda=0} 
\\ (i=\sqrt{-1})\,.\label{e2s}
\end{multline}
Здесь величины  $c_\nu$ называются коэффициентами КОР; 
$g_\theta(\theta)$~--- характеристическая функция, соответствующая плот\-ности 
$f_\theta(\theta)$; $w_\theta(\theta)$~--- плотность эталонного распределения.

Если в~(1) потребовать совпадения круговых моментов первого и второго порядка плот\-ностей 
$f_\theta (\theta)$ и $w_\theta(\theta)$, то КОР примет следующий вид:
   \begin{equation}
f_\theta(\theta) = w_\theta(\theta) \left[ 1 +\sss_{\nu=3}^\infty c_\nu p_\nu(\theta)\right]\,.
\label{e3s}
\end{equation}

\medskip

\noindent
\textbf{Замечание~1.} 
Функции $p_\nu(\theta)$ и $q_\nu (\theta)$
необязательно должны быть полиномами. Они могут быть любыми
функциями, удовлетворяющими условию биортонормальности и условиям
существования интегралов~(2). Все сказанное о разложении~(3)
справедливо и в более общем случае. Однако, если функции
$q_\nu(\theta)$ не являются полиномами, то, несмотря на совпадение
моментов первого и второго порядка распределений $f_\theta(\theta)$
и $w_\theta(\theta)$, круговые коэффициенты $c_\nu$ не будут равны
нулю при  $\nu=1,2$, вследствие чего суммирование по  $\nu$ будет
начинаться с $\nu\hm=1$.

\medskip

\noindent
\textbf{Замечание~2.} Иногда применяют разложение по производным некоторой плот\-ности 
$w_\theta(\theta)$, име\-ющей производные и моменты всех порядков~[6--8]:
    $$
    f_\theta(\theta) =\sss_{\nu=0}^\infty c_\nu w_\theta^{(\nu)} (\theta)\,.
    $$
В этом случае $p_\nu (\theta) \hm= w_\theta^{(\nu)}(\theta)/w_\theta(\theta)$, 
а  функции  $q_\nu(\theta)$ представляют собой полиномы.

В общем случае в зависимости от того, какие величины приняты за
параметры конечного отрезка КОР, различают моментные КОР,
семиинвариантные КОР, квазимоментные КОР и~др. Поэтому, обозначая
эти параметры через $u$, будем записывать КОР~(3) в виде:
\begin{align*}
f_\theta(\theta;u)&= w_\theta(\theta;u) \left[ 1+ \sss_{\nu=3}^\infty c_\nu 
p_\nu(\theta)\right]\,;\\
c_\nu &=\left[ q_\nu \left( \fr{\prt}{i\prt \lambda}\right) 
g(\lambda)\right]_{\lambda=0}\,,
%\label{e4s}
\end{align*}
где $c_\nu\hm= c_\nu (u)$; $p_\nu (\theta) \hm=p_\nu (\theta,u)$; 
$q_\nu \hm= q_\nu (\theta, u)$.


\section{Ортогональные разложения одно- 
и~многомерных плотностей круговых стохастических процессов}

Для действительного КСтП одномерная плот\-ность для 
момента времени $t$ в силу~(2) и~(3) определяется следующим КОР:
\begin{equation}
f_1 (\theta_t; t,u) = w_1(\theta_t, t,u) \left[ 
1 + \sss_{\nu=3}^\infty c_{\nu t} p_\nu (\theta_t,u) \right]\,,\label{e5s}
\end{equation}
где
\begin{multline*}
c_{\nu t}=c_{\nu t} (t,u)= \iin f_1(\theta_t;t,u) q_\nu(\theta_t) = {}\\
{}=
\left[ q_\nu \left(\fr{\prt}{i \prt \lambda}\right) g_\theta (\lambda;t,u)\right]\,.
%\label{e6s}
\end{multline*}
Для действительных КСтП $n$-мер\-ные плот\-ности для моментов времени  $t_1\tr t_n$ 
определяются как совокупность согласованных КОР плот\-ностей КСВ  
$\theta_{t_1} \tr \theta_{t_n}$:
\begin{multline*}
 f_n (\theta_{t_1}\tr \theta_{t_n}; t_1\tr t_n,u) ={}\\
 {}=
 w_n (\theta_{t_1}\tr \theta_{t_n}; t_1\tr t_n,u)\times{}
 \end{multline*}
 
 \noindent
 \begin{multline}
{}\times \left[ 1+ \sss_{k=3}^\infty \sss_{\nu_1+\cdots+\nu_n =k} 
c_{\nu_1 \tr \nu_n} (t_1\tr t_n,u) \times{}\right.\\
\left.{}\times p_{\nu_1\tr \nu_n} (\theta_{t_1} \tr \theta_{t_n}; t_1\tr t_n,u) 
\vphantom{\sum\limits_{k=3}^\infty}\right]\,,
\label{e7s}
\end{multline}
где
\begin{multline*}
c_{\nu_1\tr \nu_n} (t_1\tr t_n,u) = {}\\
{}=\iin\cdots \iin f_n (\theta_{t_1}\tr \theta_{t_n}; t_1\tr t_n,u)\times{}\\
{}\times q_{\nu_1\tr \nu_n}(\theta_{t_1}\tr \theta_{t_n}; t_1\tr t_n,u)\times{}\\
{}\times d\theta_{t_1}\cdots  d\theta_{t_n}={}\\
{}= \biggl[ q_{\nu_1\tr \nu_n} \left( \fr{\prt}{ i\prt \lambda_1}\tr \fr{\prt}{i\prt \lambda_n}; 
t_1\tr t_n,u\right)\times{}\\
{}\times g_n (\lambda_1\tr \lambda_n; t_1\tr t_n,u)\biggr]_{\lambda_1 =\cdots = \lambda_n=0}\,.
%\label{e8s}
\end{multline*}
Здесь $\{ p_{\nu_1\tr \nu_n} (\theta_{t_1}\tr \theta_{t_n},u)$, 
$q_{\nu_1\tr \nu_n} (\theta_{t_1} , \cdots$\linebreak 
$\cdots , \theta_{t_n},u)\}$~--- согласованные биортонормальные 
сис\-те\-мы полиномов, соответствующие согласованным многомерным плотностям 
$w_n (\theta_{t_1}\tr \theta_{t_n},u)$, имеющим те же круговые моменты первого и второго порядка, 
что и КСтП $\Theta(t)\hm=\Theta_t$.

Ограничиваясь в~(\ref{e5s}) и~(\ref{e7s}) полиномами  не выше $N$-й степени, 
получим согласованное приближенное представление всех многомерных плотностей КСтП $\Theta_t$. 
Этим приближенным представлением можно практически пользоваться, если КСтП 
имеет конечные круговые моменты до $N$-го порядка включительно, независимо от того, 
существуют или не существуют его моменты высших порядков.

При рассмотрении круговых марковских СтП соответствующие КОР 
используют для двух плотностей: одномерной и переходной.

В рамках спектрально-корреляционной теории принимают $n\hm=1,2$ и ограничиваются 
рассмотрением круговых дисперсий и ковариационных функций~[2].

\section{Многоканальные нелинейные круговые стохастические системы}

Введем векторный КСтП $Y(t) \hm= Y_t$, составленный из КСтП 
$\Theta_j(t) \hm=\Theta_{jt}$ $(j\hm=1\tr r)$, $Y_t \hm=\lk \Theta_{1t}\ldots \Theta_{rt}\rk$, 
где $r_y \hm=r$~--- число каналов в КСтС. 
Пусть векторное нелинейное стохастическое дифференциальное уравнение (понимаемое в смысле Ито), 
описывающее эволюцию $Y_t$, имеет вид~[6--8]:
\begin{equation}
\dot Y_t =\varphi (Y_t, t) +\psi (Y_t, t) V_t\,;\quad Y(t_0) = Y_0\,,\label{e9s}
\end{equation}
где $\varphi (Y_t, t)$ и $\psi (Y_t, t)$~--- $(r_y \hm\times 1)$- и $(r_y\hm\times r_V)$-мер\-ные в 
общем случае  известные нелинейные функции; $V_t$~--- $(r_V \hm\times 1)$-мер\-ный 
круговой белый шум в строгом смысле, представляющий собой среднюю квадратическую 
производную по времени от кругового процесса с независимыми приращениями $W_t, V_t \hm= \dot W_t$.
Начальное значение $Y_0$ будем считать независимым от приращений~$W_t$.

Обозначим через  $\chi (\mu;t)$ логарифмическую производную по времени 
от одномерной характеристической функции $h_1(\mu, t)$ кругового белого шума~$V_t$,  
$\chi(\mu;t)\hm= (\prt/\prt t) h_1 (\mu ;t)$. Тогда при известных условиях~[6, 7] 
одно- и $n$-мер\-ные плотности будут определяться следующей системой интегродифференциальных 
уравнений:
\begin{multline*}
    \fr{\prt}{\prt t_n}\, f_n (y_{t_1} \tr y_{t_n}) ={}\\
    {}= \fr{1}{(2\pi)^{nr} }
    \iin \cdots\iin \left[ i\lambda_n^{\mathrm{T}} \varphi (\eta_n, t_n) +{}\right.\\
\left.    {}+\chi 
    (\psi (\eta_n, t_n)^{\mathrm{T}} \lambda_n ; t_n)\right] %\times{}\\
%{}\times 
\exp \lf i \sss_{k=1}^n \lambda_k^{\mathrm{T}} (\eta_k - y_{t_k}) \rf \times{}\\
{}\times
f_n (\eta_1\tr \eta_n; t_1\tr t_n)\times{}\\
{}\times d\eta_1 \cdots d\eta_n d \lambda_1\ldots d\lambda_n
%\label{e10s}
\end{multline*}
$(n=1,2,\ldots)$ при условиях:
\begin{multline*}
f_n (y_{t_1} \tr y_{t_{n-1}}; t_1\tr t_{n-1}, t_{n-1})={}\\
\!{}= 
f_{n-1} ( y_{t_1}\tr y_{t_{n-1}}; t_1\tr t_{n-1}) \delta (y_{t_n} - y_{t_{n-1}});\hspace*{-1.55534pt} %\label{e11s}
\end{multline*}

\vspace*{-12pt}

\noindent
\begin{equation*}
f_1 (y_t; t_0) = f_0 (y_t)\,. %\label{e12s}
\end{equation*}

\noindent
\textbf{Замечание~3.} Для дискретных КСтС соответствующие интегроразностные уравнения 
получаются путем замены оператора дифференцирования по времени на оператор сдвига по 
времени~[6--8].

\section{Математическое обеспечение, основанное на ортогональных разложениях}

При аппроксимации круговой одномерной плотности конечным отрезком КОР 
естественно за параметры $u$ принять: $m_t$~--- вектор круговых средних; 
$K_t$~--- круговую ковариационную мат\-ри\-цу (от которых зависит эталонная плот\-ность, 
а следовательно, и полиномы $\lf p_\nu (y_t), q_\nu (y_t)\rf$) и коэффициенты $c_{\kappa, t}$. 
Поэтому из~(\ref{e5s}) для отрезка КОР имеем:



\noindent
\begin{multline}
\tilde f_1 (y_t, u) ={}\\
{}= w_1 (y_t,u) \lk 1+ \sss_{l=3}^N \sss_{\left\vert\kappa\right\vert=l} 
c_{\kappa t} p_\kappa (y_t,u)\rk\,.\label{e13s}
\end{multline}
При этом, согласно~[6, 7], $m_t, K_t$ и $c_{\nu t}$ будут определяться 
следующей системой обыкновенных дифференциальных уравнений:
\begin{equation}
\left.
\begin{array}{rl}
\!\!\!\!\!\dot m_t &= \displaystyle
A_0 (m_t, K_t, t) + \sss_{l=3}^N \sss_{\left\vert\nu\right\vert=l} 
A_{1\nu}(m_t, K_t, t)  c_{\nu t};\\[9pt] 
\!\!\!\!\!m_0 &= m(t_0);
\end{array}\!\!
\right\}\!\!\!\label{e14s}
\end{equation}

\noindent
\begin{equation}
\left.
\begin{array}{rl}
\!\!\!\!\!\dot K_t &= B_0 (m_t, K_t, t) +\displaystyle \sss_{l=3}^N \sss_{\left\vert\nu\right\vert=l} 
B_{1\nu}(m_t, K_t, t)  c_{\nu t};\\[9pt]
\!\!\!\!\!K_0 &= K(t_0);
\end{array}\!\!
\right\}\!\!\!
\label{e15s}
\end{equation}

\noindent
\begin{equation}
\left.
\begin{array}{rl}
\!\!\!\!\dot c_{\kappa t} &= C_{\kappa 0}  (m_t, K_t, t)+ A_0 (m_t, K_t, t)^{\mathrm{T}} 
q_\kappa^m (\alpha ) +{}\\[9pt]
&\hspace*{2mm}{}+
\mathrm{tr}\, \lk B_0 (m_t, K_t, t) q_\kappa^K (\alpha)\rk+{}\\[8pt]
&\hspace*{5mm}{}+ \displaystyle \sss_{l=3}^N \sss_{\left\vert\nu\right\vert=l} 
\left\{ \vphantom{B_{1v}k_\kappa^K}
C_{\kappa\nu} (m_t, K_t, t)+{}\right.\\[9pt]
&\hspace*{7mm}\left.{}+A_{1\nu}(m_t, K_t, t)^{\mathrm{T}} q_\kappa^m (\alpha) + {}\right.\\[9pt]
&\hspace*{10mm}\left.{}+
\mathrm{tr}\, \lk B_{1\nu} (m_t, K_t, t)q_\kappa^K(\alpha)\rk\right\} c_{\nu t}\,;\\[9pt]
\!\!\!\!c_{\kappa 0}&= c_\kappa(t_0)\,.
\end{array}\!
\right\}\!\!
\label{e16s}
\end{equation}
Здесь введены следующие обозначения:
    \begin{equation}
    \left.
    \begin{array}{rl}
    A_0 (m_t, K_t, t)&= \displaystyle\iin \varphi(y_t, t) w_1 (y_t,u)\, dy_t\,;\\[9pt]
 A_{1\nu} (m_t, K_t, t) &=\displaystyle\iin \varphi (y_t, t) p_\nu (y_t,u)\, d y_t)\,;
 \end{array}
 \right\}
 \label{e16-1s}
 \end{equation}

\noindent
\begin{equation}
\left.
\begin{array}{rl}
B_0 (m_t, K_t, t)&= B_{01} (m_t, K_t, t)+{}\\[9pt]
&\hspace*{-15mm}{}+B_{01} (m_t, K_t, t)^{\mathrm{T}} +B_{02} (m_t, K_t, t)={}\\[9pt]
&\hspace*{-15mm}{}=\displaystyle\iin \left[ \varphi(y_t, t) (y_t - m_t)^{\mathrm{T}} +{}\right.\\
&\left.{}+(y_t - m_t) 
\varphi(y_t, t)^{\mathrm{T}} +{}\right.\\[9pt]
&\hspace*{-15mm}\left.{}+\psi(y_t, t) G_t \psi_(y_t, t)^{\mathrm{T}} \right]  w_1 (y_t,u)\, dy_t\,;\\[9pt]
B_{1\nu} (m_t, K_t, t)&=\displaystyle\iin \left[ 
\varphi(y_t, t) (y_t- m_t)^{\mathrm{T}} +{}\right.\\[9pt] 
&\hspace*{-23mm}\left.{}+(y_t-m_t) \varphi(y_t, t)^{\mathrm{T}} \psi (y_t, t) G_t 
\psi (y_t, t)^{\mathrm{T}} \right]\times{}\\[9pt]
&{}\times w_1 (y_t,u)\, dy_t\,;
\end{array}\!
\right\}\!
\label{e17s}
\end{equation}

\noindent
\begin{equation}
\left.
\begin{array}{rl}
C_{\kappa 0}(m_t, K_t, t) &={}\\[9pt]
&\hspace*{-8mm}{}=\displaystyle \iin \left\{ q_\kappa \left( \fr{\prt}{i\prt \lambda}\right) 
\left[ i\lambda^{\mathrm{T}} \varphi(y_t, t) +{}\right.\right.\\[9pt]
&\hspace*{-25mm}\left.\left.{}+\chi(\psi(y_t, t)^{\mathrm{T}} \lambda;t)\right]\exp 
\lk i\lambda^{\mathrm{T}} y_t\rk \vphantom{\fr{\partial}{\partial}}
\right\}_{\lambda=0}\times{}\\[9pt]
&\hspace*{17mm}{}\times w_1(y_t,u)\, dy_t\,;\\[9pt]
C_{\kappa \nu}(m_t, K_t, t) &={}\\[9pt]
&\hspace*{-8mm}{}=\displaystyle \iin \left\{ q_\kappa \left( \fr{\prt}{i\prt \lambda}\right) 
\left[ i\lambda^{\mathrm{T}} \varphi(y_t, t) +{}\right.\right.\\[9pt]
&\hspace*{-27mm}\left.\left.{}+\chi(\psi(y_t, t)^{\mathrm{T}} \lambda;t)\right]\exp 
\lk i\lambda^{\mathrm{T}} y_t\rk p_\nu (y_t,u)
\vphantom{\fr{\partial}{\partial}}
\right\}_{\lambda=0}\!\!\!\times{}\\[9pt]
&\hspace*{20mm}{}\times w_1(y_t,u)\, dy_t\,,
\end{array}\!\!
\right\}\!\!
\label{e17-1s}
\end{equation}
где $q_\kappa^m (y_t)$~--- мат\-ри\-ца-стол\-бец производных полинома  $q_\kappa (y_t)$ 
по компонентам вектора  $m_t$; $q_\kappa^K (y_t)$~--- квадратная мат\-ри\-ца 
производных полинома  $q_\kappa (y_t)$ по элементам мат\-ри\-цы  $K_t$; 
$G_t \hm= \lk G_{lj} (t)\rk$~--- мат\-ри\-ца интенсивностей векторного кругового белого шума~$V$, 
причем
    $$
    G_{lj} (t) = \lk \fr{\prt^2 \chi (\mu;t)}{\prt (i\mu_l) \prt (i\mu_j)}\rk_{\mu=0}\,;
    $$
$q_\kappa^m(\alpha)$  и $q_\kappa^K(\alpha)$~--- 
результат замены одночленов вида $y_{1t}^{\rho_1}\cdots y_{rt}^{\rho_r}$ 
соответствующими  начальными моментами $\alpha_{\rho_1 \tr \rho_r}$ которые, 
как известно, зависят от~$c_{\kappa t}$.

\medskip
\noindent
\textbf{Замечание~4.} Уравнения~(\ref{e16s}) нелинейны относительно  $c_{\kappa t}$. 
Если отказаться от требования совпадения первых  моментов и задать эти моменты для 
эталонной плотности априори, то для  $c_{\kappa t}$ получатся линейные уравнения. 
Эти уравнения проще чем~(\ref{e16s}), однако при этом придется взять большее~$N$ в~(\ref{e13s}).

\smallskip

Таким образом, при использовании полиномиальной биортонормальной сис\-те\-мы  
$\lf p_\nu (y_t,u), q_\mu (y_t,u)\rf$ в основе математического обеспечения 
кругового ортогонального~(\ref{e13s}) разложения одномерной плотности КСтП  
$Y_t$ в КСтС~(\ref{e9s}) лежат уравнения~(\ref{e14s})--(\ref{e16s}) при условиях~(\ref{e16-1s})--(\ref{e17-1s}).

Аналогично на основе результатов~[6, 7] составляются уравнения для $n$-мер\-ных 
плот\-но\-стей~(\ref{e7s}) $(n\hm\ge 2)$.

В состав математического обеспечения  <<CStS-ANALYSIS>> входят:
\begin{itemize}
\item
уравнения для различных типов систем;
\item
типовые системы биортонормальных сис\-тем функций для одно- и многомерных плотностей;
\item
ортогональные разложения плотностей одно- и многомерных распределений;
\item
обыкновенные дифференциальные (разностные) уравнения для параметров одно- и 
многомерных распределений с соответствующими начальными условиями;
\item
выражения для вычисления типовых функционалов, определяющих задачи вероятностного 
кругового анализа многоканальной КСтС;
\item
набор тестовых задач.
\end{itemize}

\section{Математическое обеспечение на~основе метода <<намотанной>> нормальной аппроксимации}

Обобщая результаты~[2] на случай, когда коэффициент при круговом белом шуме в 
уравнении~(\ref{e9s}) 
зависит от состояния, приведем основные уравнения кругового МННА:
\begin{align}
\dot m_t &=A_0 (m_t, K_t, t)\,,\enskip m_0 = m(t_0)\,;\label{e18s}\\
\dot K_t &=B_0 (m_t, K_t, t)\,,\enskip K_0 = K(t_0)\,;\label{e19s}
\end{align}

\noindent
\begin{equation}
\left.
\begin{array}{rl}
\fr{\prt K(t_1, t_2)}{\prt t_2} &=\\[9pt] 
&\hspace*{-15mm}{}=\fr{K(t_1, t_2)}{K(t_2)}\, B_{01}(m_{t_2}, K_{t_2}, t_2)\quad 
(t_1< t_2)\,;\\[9pt]
 K(t_1, t_1) &= K_{t_1}\,.
 \end{array}
 \right\}
 \label{e20s}
\end{equation}
Здесь  $A_0(m_t, K_t, t)$, $B_0(m_t, K_t, t)$ и $B_{01}(m_t, K_t, t)$  
определены в~(11)--(12) для 
<<намотанного>> нормального распределения.

В состав МННА включено математическое обеспечение~[2] для сис\-те\-мы~(\ref{e9s}) 
при $\psi (y_t, t) \hm= I_r$.

\section{Математическое обеспечение на~основе методов круговых начальных 
и~центральных моментов}

Из уравнений разд.~5 для круговых <<намотанных>> начальных моментов  
$\alpha_\rho \hm=\alpha_\rho (t)$ порядка  $\rho$, обобщая~[6, 7], имеем следующие уравнения:
\begin{equation}
\dot\alpha_\rho = A_{0,\rho} +\sss_{k=3}^N \sss_{\left\vert\nu\right\vert = 
k} A_{\nu,\rho} c_\nu (\alpha) \enskip ( c_\nu = q_\nu (\alpha))\,,\label{e21s}
\end{equation}
где
\begin{multline*}
A_{0,\rho} = \iin \biggl\{ \fr{\prt^{|\rho |}}{\prt ( i\la_1)^{\rho_1} \cdots 
\prt ( i \la_r)^{\rho_r}}
\left[ i\lambda^{\mathrm{T}} \varphi(y_t, t) +{}\right.\\ 
\left.{}+\chi (\psi(y_t, t)^{\mathrm{T}}\lambda;t)\right] \exp \lk i\lambda^{\mathrm{T}}
y_t\rk\biggr\}_{\lambda=0} 
w_1(y_t,u)\, dy_t\,;
\end{multline*}

\noindent
\begin{multline*}
A_{\nu,\rho} = \iin \biggl\{ \fr{\prt^{|\rho |}}{\prt ( i\lambda_1)^{\rho_1} \cdots 
\prt ( i \lambda_r)^{\rho_r}}
 \left[ i\lambda^{\mathrm{T}} \varphi(y_t, t) +{}\right.\\
\left. {}+ 
\chi (\psi(y_t, t)^{\mathrm{T}}\la;t)\right] \exp \lk i\lambda^{\mathrm{T}} 
y_t\rk\biggr\}_{\lambda=0} \times{}\\[4pt]
{}\times
p_\nu (y_t,u) w_1(y_t,u)\, dy_t %\label{e22s}
\end{multline*}


\vspace*{-8pt}

\noindent
\begin{equation*}
\left(\left\vert \rho\right\vert = \rho_1 +\cdots + \rho_r , 
\rho_1 \tr \rho_r = 0,1\tr N\right)\,.
\end{equation*}
Напомним, что $q_\nu (\alpha)$ представляет собой результат замены всех 
одночленов $y_{1t}^{\rho_1} \cdots y_{rt}^{\rho_r}$ в выражении полинома  
$q_\nu (\alpha)$ соответствующими моментами $\alpha_{\rho_1\tr \rho_r}$.

\medskip

\noindent
\textbf{Замечание~5.} При составлении уравнений~(\ref{e13s}) 
в конкретных задачах следует иметь в виду, что число $N_r^\rho$ 
моментов $\rho$-го порядка $r$-мер\-но\-го вектора определяется формулой:
    $$
    N_r^\rho = C_{r+\rho-1}^\rho = \fr{(r+\rho-1)!}{\rho! (r-1)!}\,,
    $$
а полное число моментов, не превосходящих $N$, для $r$-мер\-но\-го
вектора равно:
    $$
    P_r^N =\sss_{l=1}^N N_r^l = \fr{(N+r)!}{N! r!}-1\,.$$
    
    \medskip

\noindent
\textbf{Замечание~6.} Уравнения~(\ref{e21s}) линейны относительно  
$\alpha_\rho$ $(\left\vert \rho\right\vert = 3\tr N)$ и 
нелинейны относительно моментов первого и второго порядка, поскольку эталонная плот\-ность 
и полиномы  $p_\nu (y_t,u)$ и $q_\nu (y_t,u)$ зависят от моментов первого и второго порядка.

\smallskip

Обобщая [6, 7] для математических ожиданий $m_h$ $(h\hm=1\tr r)$ и центральных 
моментов $\mu_\rho$ порядка~$\rho$, представим уравнения в виде:
\begin{gather}
\dot m_h = A_{0, h} +\sss_{k=3}^\infty \sss_{|\nu |=k} A_{\nu r} c_\nu\,;\enskip 
c_\nu = q_\nu (\alpha)\,;\label{e23s}\\
\dot\mu_\rho = B_{0,\rho} - \sss_{h=1}^r \rho_h B_{0,h} \mu_{\rho- e_h} + {}\notag\\
{}+
\sss_{k=3}^N \sss_{|\nu |=k} \lk B_{\nu,\rho} - 
\sss_{h=1}^r \rho_h B_{\nu,h} \mu_{\rho - e_h}\rk c_\nu\notag \\
(\rho_1\tr \rho_r = 0,1,\tr N\,; \notag\\ 
\left\vert \rho\right\vert = 2\tr N)\,.\label{e24s}
\end{gather}
Здесь введены следующие обозначения:
\begin{equation*}
e_h = \lk 0\cdots 0 \ {\ed\limits_h}\  0\cdots 0\rk^{\mathrm{T}}\,;
\end{equation*}

\noindent
\begin{align*}
A_{0,h} &=\iin \varphi_h (y_t, t) w_1(y_t,u) \,d y_t\,;\\
A_{\nu,h} &=\iin \varphi_h (y_t, t)p_\nu (y_t,u) w_1(y_t,u)\, d y_t\,;\\
A_{0,\rho} &=\iin \biggl\{ \fr{\prt^{|\rho |}}{\prt (i\la_1)^{\rho_1} \cdots 
\prt ( i\lambda_r)^{\rho_r}} \left[ i\lambda^{\mathrm{T}} \varphi(y_t, t)+{}\right.\\
&\hspace*{15mm}\left.{}+
\chi(\psi (y_t, t)^{\mathrm{T}}\lambda;t)\right] \times{}\\
&{}\times \exp \lk i\lambda^{\mathrm{T}} (y_t-m)\rk \biggr\}_{\lambda=0}  w_1(y_t,u)\, d y_t\,;\\
B_{\nu,\rho} &=\iin \biggl\{ \fr{\prt^{|\rho |}}
{\prt (i\lambda_1)^{\rho_1} \cdots \prt ( i\lambda_r)^{\rho_r}} \left[ i\lambda^{\mathrm{T}} 
\varphi(y_t, t)+{}\right.\\
&\left.{}+\chi(\psi (y_t, t)^{\mathrm{T}}\lambda;t)\right ] 
\exp \lk i\lambda^{\mathrm{T}} (y_t-m)\rk \biggr\}_{\lambda=0}\times{}\\
&\hspace*{10mm}{}\times p_\nu (y_t,u)  
w_1(y_t,u)\, d y_t\,, %\label{e25s}
\end{align*}
а  $q_\nu (\alp)$ должны быть выражены через центральные моменты.

\medskip

\noindent
\textbf{Замечание~7.} Уравнения~(\ref{e23s}) и~(\ref{e24s}) 
всегда нелинейны из-за наличия слагаемых вида $\mu_{\rho-e_h} q_\nu (\alpha)$.

\medskip

\noindent
\textbf{Замечание~8.} В~практических задачах для 
полиномиальных функций  $\varphi (y_t, t) , \psi(y_t, t)$ и  $p_\nu (y_t), q_\nu(y_t)$ 
непосредственно используются символьные вы\-чис\-ле\-ния~[3].

%\vspace*{-12pt}

\section{Пример}

Для одномерной нелинейной круговой сис\-те\-мы
\begin{equation}
\dot Y +\varphi (Y)=V\,,\label{e26s}
\end{equation}
($\varphi (Y)$~--- скалярная нелинейная функция, $V$~--- круговой белый шум интенсивности~$G_t$), 
уравнения МННА~(\ref{e18s})--(\ref{e20s}) имеют вид:
\begin{gather*}
\dot m_t = - A_0 (m_t, D_t);\enskip \dot D_t =- 2 k_1 (m_t, D_t) + G_t\,;\\
\fr{\prt K(t_1, t_2)}{ \prt t_2} =- k_1 (m_t, D_t) K(t_1, t_2)\,,
\end{gather*}
где $A_0 (m_t, D_t)$ и $k_1 (m_t, D_t)$~--- 
коэффициенты статистической линеаризации нелинейной функции  $\varphi(Y)$ в~(\ref{e26s}) 
для <<намотанного>> нормального распределения с параметрами  $m_t$ и~$D_t$.

При $N=4$ уравнения~(\ref{e23s}) и~(\ref{e24s}) имеют вид:
\begin{multline*}
\dot m_t ={}\\
{}=  A_0 (m_t, D_t)+A_{13} (m_t, D_t)\mu_{3t} +A_{14} (m_t, D_t)\mu_{4t}\,;
\end{multline*}

\noindent
\begin{align*}
\dot D_t &={}\\[2pt]
&\hspace{-9pt}{}=B_0 (m_t, D_t)+B_{13} (m_t, D_t)\mu_{3t}+B_{14} (m_t, D_t) \mu_{4t}\,;\\[9pt]
\dot \mu_{3t} &={}\\[2pt]
&\hspace{-10pt}{}= C_{30}(m_t, D_t)+ C_{33}(m_t, D_t)\mu_{3t}+C_{34}(m_t, D_t) \mu_{4t}\,;\\[9pt]
\dot \mu_{4t} &={}\\[2pt]
&\hspace*{-16pt}{}= C_{40}(m_t, D_t)+ C_{41}(m_t, D_t)\mu_{3t}+C_{44}(m_t, D_t) \mu_{4t} -{}\\[2pt]
&\hspace*{36pt}{}- 4 A_{13} (m_t, D_t) \mu_{3t}^2 -A_{14} (m_t, D_t) \mu_{3t}\mu_{4t}\,.
\end{align*}
%
Для различных нелинейных функций $\varphi(y)$ эти урав\-не\-ния
позволяют оценить точность МННА.

\section{Заключение}

Разработана прикладная тео\-рия анализа одно- и многомерных распределений в 
нелинейных многоканальных круговых стохастических системах на основе ортогональных 
разложений плотностей.

Для кругового эталонного <<намотанного>> нормаль\-но\-го распределения 
описанное математическое обеспечение положено в основу инструментального 
программного обеспечения <<CStS-ANALYSIS>> в среде MATLAB.

В настоящее время в ИПИ РАН ведутся работы по созданию математического обеспечения для 
других эталонных распределений.

{\small\frenchspacing
{%\baselineskip=10.8pt
\addcontentsline{toc}{section}{Литература}
\begin{thebibliography}{9}
\bibitem{1s}
\Au{Синицын И.\,Н.}
Канонические разложения случайных функций и их применение в стохастических информационных 
технологиях научных исследований: Курс лекций~// Распознавание образов и анализ изображений: 
новые информационные технологии. РОАИ-10-2010: Мат-лы 1-й Междунар. конф.~--- СПб., 2010.

\bibitem{2s}
\Au{Синицын И.\,Н.}
Стохастические информационные технологии для исследования 
нелинейных круговых стохастических систем~// Информатика и её применения, 2011. Т.~5. Вып.~4. 
С.~2--5.

\bibitem{3s}
\Au{Синицын И.\,Н., Корепанов Э.\,Р., Белоусов В.\,В. и~др.}
Развитие компьютерной поддержки статистических научных исследований сис\-тем 
высокой точности и доступности~// Cистемы и средства информатики, 2011. Вып.~21. №\,1. С.~7--37.

\bibitem{4s}
\Au{Sinitsyn I.\,N., Belousov~V.\,V., Konashenkova~T.\,D.}
Software tools for circular stochastic systems analysis~// 29th
Seminar (International) on Stability Problems for Stochastic\linebreak\vspace*{-12pt}
\pagebreak

\noindent
 Models
and 5th Workshop <<Applied Problems in Theory of Probabilities and
Mathematical Statistics Related to Modeling of Information Systems>>
(APTP\;+\;MS'2011): Book on Abstracts.~--- M.: IPIRAS, 2011. P.~86--87.

\bibitem{5s}
\Au{Босов А.\,В., Будзко В.\,И., Захаров В.\,Н., Козмидиади~В.\,А., Корепанов~Э.\,Р., 
Синицын~И.\,Н., Шоргин~С.\,Я., Уш\-ма\-ев~О.\,С.}  
Информатика: состояние, проблемы, перспективы~/ Под. ред. И.\,А.~Соколова.~--- М.: ИПИ
РАН, 2009.

\bibitem{6s}
\Au{Пугачев В.\,С., Синицын И.\,Н. }
Стохастические дифференциальные системы. Анализ и фильтрация.~--- 2-е изд. доп.~--- М.: Наука, 1990.

\bibitem{7s}
\Au{Пугачев В.\,С., Синицын И.\,Н.}
Теория стохастических систем.~--- 2-е изд.~--- М.: Логос, 2004.

\label{end\stat}

\bibitem{8s}
\Au{Синицын И.\,Н.}
Канонические представления случайных функций и их применения 
в задачах компьютерной поддержки научных исследований.~--- М.: ТОРУС ПРЕСС, 2009.
 \end{thebibliography}
}
}


\end{multicols}           %1Abst+avt
\newcommand{\Esf}{{\sf E}}
\newcommand{\Psf}{{\sf P}}



\def\stat{chertok}



\def\tit{О ФОРМАЛИЗАЦИИ ПОНЯТИЯ ТОКСИЧНОСТИ ПОТОКА ЗАЯВОК НА ФИНАНСОВЫХ РЫНКАХ$^*$}

\def\titkol{О формализации понятия токсичности потока заявок на финансовых рынках}

\def\aut{А.\,В.~Черток$^1$}

\def\autkol{А.\,В.~Черток}

\titel{\tit}{\aut}{\autkol}{\titkol}

{\renewcommand{\thefootnote}{\fnsymbol{footnote}} \footnotetext[1]
{Работа выполнена при частичной поддержке РФФИ (проект 14-07-00041а).}}


\renewcommand{\thefootnote}{\arabic{footnote}}
\footnotetext[1]{Факультет вычислительной математики и кибернетики Московского
государственного университета им.\ М.\,В.~Ломоносова; Euphoria Group LLC;
a.v.chertok@gmail.com}

\vspace*{6pt}

\Abst{Рассматривается микроструктурная модель
потоков заявок на финансовых рынках. В~качестве интегрального
индикатора текущего состояния книги заявок используется дисбаланс
потока заявок. Для анализа свойств текущего состояния книги заявок
используется модель дисбаланса потока заявок, имеющая вид
двустороннего процесса риска, известного в~актуарной математике как
процесс риска со случайными премиями. Исследуется понятие
токсичности потока заявок на финансовых рынках. Понятие токсичности
потока заявок на
финансовых рынках формализуется с~по\-мощью вероятностей пересечения
процессом дисбаланса потоков заявок фиксированных уровней. Вводятся
понятия мгновенного профиля токсичности и байесовского
и~квантильного показателей токсичности. Эти показатели рассчитываются
для двух модельных типов потоков заявок, в~первом из которых заявки
имеют единичный объем, во втором~--- объем заявок является случайным
и~имеющим показательное распределение.}

%\vspace*{3pt}

\KW{финансовые рынки; книга заявок; поток заявок;
дисбаланс потока заявок; неблагоприятный отбор; токсичность; пуассоновский процесс;
обобщенный пуассоновский процесс;
двусторонний процесс риска; процесс риска со случайными премиями;
вероятность разорения}

\vspace*{3pt}

\DOI{10.14357/19922264140403}

\vspace*{6pt}


\vskip 14pt plus 9pt minus 6pt

\thispagestyle{headings}

\begin{multicols}{2}

\label{st\stat}

\section{Введение}

Активное развитие электронной торговли на финансовых рынках выявило
необходимость анализа биржевых высокочастотных данных для \mbox{более}
глубокого понимания рыночной микроструктуры, на которую оказали
огромное влияние компании, занимающиеся автоматизированным
высо\-кочастотным трейдингом (они формируют\linebreak до 70\%--80\% дневного
оборота на ведущих мировых площадках). Эти высокочастотные системы,\linebreak
как правило, являются мар\-кет-мей\-ке\-ра\-ми~--- поставщиками ликвидности
посредством размещения пассивных (лимитных) заявок на различных
уровнях электронной книги заявок. Поставщик ликвидности, выставивший
пассивную заявку, не имеет возможности влиять на время ее исполнения
(разуме\-ет\-ся, кроме как снять заявку). Мар\-кет-мей\-ке\-ры
зачастую не прогнозируют в~явном виде динамику рынка, а~используют
шумовую составляющую рыночных движений.

Степень эффективности
деятельности мар\-кет-мей\-ке\-ров
связана с~контролем риска оказаться с~большим количеством
купленных или проданных контрактов, что напрямую
зависит от их способности контролировать эффект неблагоприятного
отбора (adverse selection) в~отношении пассивных заявок.

Практики, как правило, описывают принцип неблагоприятного отбора как
<<естественную тенденцию слишком быстрого исполнения пассивных
заявок в~тех ситуациях, когда они должны исполняться медленно,
и~наоборот: исполняться слишком медленно в~тех ситуациях, когда они
должны исполниться быст\-ро>>~\cite{Jeria2008}. Эта интуитивная
формулировка согласуется с~ранними микроструктурными моделями
рынка~[2--4], в~которых информированные
трейдеры получают преимущество над
неинформированными участниками рынка. Поток заявок считается
токсичным, когда происходит эффект неблагоприятного отбора
мар\-кет-мей\-ке\-ров, поставляющих ликвидность.

В работе~\cite{Easley2012} предложена эмпирическая процедура оценки
токсичности потока заявок на основе анализа информации о~\textit{сделках}.
В~предлагаемой статье рассматривается более точный подход
к~измерению токсичности рынка, использующий всю доступную информацию
о~\textit{потоке заявок} (не только сами сделки, но также
и~по\-ста\-нов\-ки/сня\-тия заявок) на основе аналитической модели процесса
дисбаланса потока заявок, рассмотренной ранее в~работах~\cite{Korolev_2013, Chertok2014}.

\section{Модель потока заявок}

\subsection{Терминология}

На электронных рынках биржевая цена финансового инструмента в~ее
классическом понимании является результирующей, интегральной
характеристикой системы торгов, которая описывается динамикой так
называемой {\it книги заявок $($limit order book$)$}, представляющей
собой информацию о~всех актуальных на данный момент предложениях
о~покупке и продаже инструмента по различным ценам (рис.~1).

\begin{center}  %fig1
\vspace*{8pt}
\mbox{%
\epsfxsize=79.096mm
\epsfbox{che-1.eps}
}
\end{center}

\noindent
{{\figurename~1}\ \ \small{Книга заявок в~некоторый момент времени. Высота столбиков равна
  суммарному объему лимитных заявок на соответствующем ценовом уровне:
\textit{1}~--- покупки; \textit{2}~--- продажи}}


\vspace*{9pt}


\addtocounter{figure}{1}



Динамика книги заявок определяется тремя типами
заявок, которые участники рынка могут отправить на рынок:
\begin{enumerate}[(1)]
\item {\it лимитная} заявка обозначает желание купить (продать)
заданное количество акций по цене не выше (не ниже) заданной, при
этом такая заявка немедленно добавляется в~книгу заявок;
\item {\it рыночная} заявка обозначает желание купить или продать
заданное количество акций по лучшей цене, представленной в~книге
заявок, после чего немедленно происходит сделка;
\item заявка {\it на отмену} обозначает намерение отменить
существующую лимитную заявку, после чего она удаляется из книги заявок.
\end{enumerate}



Более формально, в~каждый момент времени информация о~первых $d \hm= 5$
уровнях книги заявок представляет собой массив
\begin{equation*}
\mathrm{book} = \left( b_1, a_1, v^b_1, v^b_2, \ldots, v^b_{10}, v^a_1, v^a_2,
\ldots, v^a_{10} \right)\,, %\label{e1-che}
\end{equation*}
где
$b_1$~--- лучшая цена на покупку (бид) на текущий момент
(кратная минимальному шагу цены~$\delta$);
$a_1$~--- лучшая цена на продажу (аск) на текущий момент
(кратная минимальному шагу цены~$\delta$);
$v^b_i \hm\geqslant 0$~--- суммарный объем заявок по цене~$b_i$ (при этом
автоматически $b_i \hm= b_1\hm - (i\hm - 1) \delta$);
$v^a_i \hm\geqslant 0$~--- суммарный объем заявок по цене~$a_i$ (при этом
автоматически $a_i \hm= a_1 \hm+ (i \hm- 1) \delta$).

Всегда выполняется условие $b_1 \hm< a_1$, так как иначе
соответствующие заявки должны быть сведены в~сделку, величина $p \hm=
(1/2)(b_1 \hm+ a_1)$ обычно называется {\it мидпрайсом}, а~величина
$s \hm= a_1 \hm- b_1$ называется {\it спредом}.

\subsection{Динамика книги заявок}

Потоки заявок моделируются с~использованием независимых
пуассоновских процессов~--- процессов восстановления
с~экспоненциальными распределе\-ниями интервалов между
восстановлениями (как это сделано, например,
в~работах~\cite{ContRamaStoikov2010b, ContLarrard2011}):
\begin{itemize}
\item лимитные заявки на покупку (продажу) приходят на ценовой уровень,
расположенный на\linebreak
 расстоянии~$i$ от лучшей котировки противоположного
типа, в~независимые моменты времени, имеющие экспоненциальное
распределение с~параметром $\lambda_i^{+} (\lambda_i^{-})$
(эмпирические исследова\-ния~\cite{Bouchaud2002, ZovkoFarmer2002} показывают, что степенный закон
$
\lambda_i^{\pm} = k/i^\alpha$
является хорошей аппроксимацией);
\item рыночные заявки на покупку (продажу) приходят в~независимые
моменты времени, име\-ющие экспоненциальное распределение с~параметром
$\mu^{+} (\mu^{-})$;
\item заявки на отмену лимитного ордера на покупку (продажу), находящегося
на дистанции~$i$ от лучшей котировки того же типа, приходят с~час\-то\-той $\theta_i^{+} (\theta_i^{-})$.
\end{itemize}

Рассмотрим два пуассоновских процесса $N^+(t)$ и $N^-(t)$
с~интенсивностями соответственно
\begin{align*}
\lambda^+ &= \mu^{+} + \sum\limits_{i} \lambda_i^{+} + \sum\limits_{i} \theta_i^{-}\,;
\\
\lambda^- &= \mu^{-} + \sum\limits_i \lambda_i^{-} + \sum\limits_i \theta_i^{+}\,.
\end{align*}
По своей сути процессы $N^+(t)$ и $N^-(t)$ соответствуют информации
о~числе заявок от покупателей и~продавцов соответственно, пришедших
к~моменту времени~$t$. Будем также считать, что объемы заявок от
покупателей и продавцов~--- независимые одинаково распределенные
величины $X_i^+$ и~$X_i^-$ с~функциями распределения $G(t)$ и~$F(t)$
соответственно и~не зависят от процессов $N^+(t)$ и~$N^-(t)$.

\begin{figure*}[b] %fig2
\vspace*{1pt}
 \begin{center}
 \mbox{%
 \epsfxsize=141.02mm
 \epsfbox{che-2.eps}
 }
 \end{center}
 \vspace*{-9pt}
  \Caption{Динамика лучшей цены покупки~(\textit{1}),
  лучшей цены продажи~(\textit{2}) и~процесса
  дисбаланса потока заявок $Q(t)$ в~течение 1~с~(\textit{3}) с~момента
  10:00:12,730 01.07.2014 (фьючерс на индекс РТС)
  \label{fig:ofibidask_mono}}
\end{figure*}

\section{Процесс дисбаланса потока заявок и~его связь с~ценой}

Понятие дисбаланса потока заявок введено в~работе~\cite{Cont2011},
окончательный вариант которой~\cite{Cont2014} опуб\-ли\-ко\-ван в~2014~г.\
В~работах~\cite{Korolev_2013, Chertok2014} независимо этот же
процесс исследовался под названием {\it процесс обобщенной цены}.

В работах~\cite{Korolev_2013, Chertok2014} в~качестве
математической модели эволюции процесса дисбаланса потока \mbox{заявок}
было предложено использовать двусторонний процесс риска~---
специальный обобщенный (compound) пуассоновский процесс. Следуя
этому подходу, зафиксируем малый интервал времени~$[0; T]$,\linebreak в~течение
которого параметры распределений, описывающих объемы заявок,
и~интенсивности потоков заявок одного типа остаются постоянными и~известными.
Для $t\hm\in[0,T]$ пусть $N^+(t)$ и~$N^-(t)$~---\linebreak количества
заявок, пришедших от покупателей и~продавцов соответственно
в~течение интервала времени $[0,t]$~--- независимые пуассоновские
процессы с~интенсивностями $\lambda^+ \hm> 0$ и~$\lambda^- \hm> 0$ ($\Esf
N^+(t) = \lambda^+ t$, $N^+(0) = 0$, $\Esf N^-(t) = \lambda^- t$,
$N^-(0) = 0$). Пусть $X^+_i$ и~$X^-_i$, $i\hm=1,2,\ldots, $~--- объемы
заявок, поступающих от покупателей и~продавцов соответственно -- две
независимые последовательности независимых и~одинаково в~каждой
последовательности распределенных случайных величин с~функциями
распределения $G(x)$ и~$F(x)$ соответственно, независимых от
пуассоновских процессов $N^+(t)$ и~$N^-(t)$. Положим
$$
Q^+(t)=\sum\limits_{i=1}^{N^+(t)}X_i^+\,;\enskip
Q^-(t)=\sum\limits_{j=1}^{N^-(t)}X_j^-
$$
и определим процесс {\it дисбаланса потока заявок} $Q(t)$ как
$$
Q(t)=Q^+(t) - Q^-(t)\,.
$$

Этот процесс является намного более чувствительным индикатором
(показателем) текущего состо\-яния книги заявок, поскольку интервалы
времени между последовательными изменениями со\-сто\-яний книги заявок
обычно так малы, что изменения цены (мидпрайса) по сравнению с~ними
являются редкими событиями. Поэтому процесс цены является намного
более грубым показателем, характеризующим книгу заявок, и~дает грубое
и~весьма ограниченное описание динамики рынка. Вместе с~тем процесс
дисбаланса потоков заявок учитывает не только текущие значения
наилучших цен покупки и~продажи, но и~влияние событий <<в глубине>>
книги заявок и~потому меняется существенно быстрее и~позволяет
интерполировать динамику рынка между изменениями цены, в~част\-ности
отслеживать ситуации, связанные с~токсичностью потоков заявок, т.\,е.\
чреватые необоснованными трендами в~поведении цены (рис.~\ref{fig:ofibidask_mono}).


В работе~\cite{Cont2014} с~помощью линейной модели
$$
\fr{S(t + \Delta) - S(t)}{\delta} = c \fr{Q(t, t + \Delta)}{D(t)} + \epsilon(t)
$$
было показано, что процесс дисбаланса потока заявок $Q(t)$ имеет
сильную связь с~высокочастотными изменениями цены финансового актива
$S(t)$, построенной по ценам сделок, где~$\delta$~--- минимальный шаг
цены (тик цены), $\epsilon(t)$~--- белый шум и~$D(t)$~--- мера глубины
книги заявок (количество заявок на лучшем биде/аске). Эмпирический
анализ высокочастотных данных для американских акций подтверждает
наличие линейной связи: коэффициент~$c$ оценивается между~0,1 и~1
и~оказывается статистически значим в~98\% случаев. Наличие
такого рода связи позволяет напрямую исследовать свойства процесса
дисбаланса потока заявок и~соотносить их со свойствами процесса цены
$S(t)$.

В работах~\cite{Korolev_2013, Chertok2014, Korolev_2014}
с~помощью предельных \mbox{теорем} для двусторонних
процессов риска были получены асимптотические аппроксимации
для процесса дисбаланса потока заявок $Q(t)$ и~его распределений.

\section{Профиль мгновенной токсичности потока заявок}

Как уже было сказано выше, поток заявок считается токсичным, когда
он оказывается неблагоприятным для мар\-кет-мей\-ке\-ров, предоставляющих
ликвидность в~книге заявок. В~работе~\cite{Easley2012} предложена
процедура оценки токсичности потока заявок на основе анализа
информации об интенсивности и~направлении {\it сделок} (направление
сделки определяется в~зависимости от того, кто являлся ее
инициатором~--- покупатель или продавец).
В~данной работе будет рассмотрен более точный подход
к~измерению токсичности потока
заявок, использующий всю доступную информацию о~заявках (не только
сами сделки, но также и~по\-ста\-нов\-ки/сня\-тия заявок).

Чтобы формализовать понятие токсичности потока заявок, для начала
рассмотрим процесс дисбаланса потока заявок $Q(t)\hm=Q^+(t) \hm- Q^-(t)$
в предположении, что $\Esf Q(t) > 0$, т.\,е.\
$$
\lambda^+ \Esf X_1^+ > \lambda^- \Esf X_1^-,
$$
что означает преимущество покупателей над продавцами в~рамках
интервала $[0, T]$. Предположим, что $Q(0)\hm=0$.

Для $u > 0$ рассмотрим вероятность
$$
\varphi_{\pm}(u, T) = \Psf\left(\inf\limits_{0 < t \leqslant T} Q(t) \geqslant -u\right)\,,
$$
т.\,е.\ вероятность того, что траектория процесса $Q(t)$ в~течение
интервала времени $[0, T]$ целиком будет находиться не ниже уровня
$-u$, а~также аналогичную предельную вероятность на бесконечном
интервале времени:
\begin{multline*}
\varphi_{\pm}(u) = \Psf\left(\inf\limits_{t > 0} \left( Q^+(t) - Q^-(t) \right) \geqslant -u\right)
= {}\\
{}=\lim\limits_{T \to \infty} \varphi_{\pm}(u, T)\,.
\end{multline*}
Вероятность $\varphi_{\pm}(u)$ описывает вероятность того, что при {\it
положительном} тренде процесс дисбаланса никогда не достигнет {\it
отрицательного} уровня $-u$ при условии, что параметры потока заявок
($\lambda^+$, $\lambda^-$, $G(x)$ и~$F(x)$) остаются неизменными.


\smallskip

\noindent
\textbf{Определение 1.}\
Функцию $\varphi_{\pm}(u)$ будем называть \textit{профилем мгновенной токсичности}
потока заявок.

\smallskip

Введенная таким образом характеристика~--- профиль мгновенной
токсичности потока заявок~--- формально совпадает с~{\it вероятностью
неразорения} в~классической модели коллективного риска со случайными
премиями, рассматривавшейся, например, в~работах~[15--17].
В~некоторых источниках (см., в~частности,~\cite{KorolevBeningShorgin2011})
справедливо отмечено, что
интерпретация этого показателя именно как вероятности физического
разорения страховой компании некорректна, поскольку изначальное
предположение о~неизменности основных параметров потоков страховых
премий и~страховых выплат в~течение бесконечного интервала времени
заведомо не выполняется. Тем не менее эта характеристика является
удобным показателем текущей эффективности работы страховой компании
и~имеет смысл некоей оценки качества текущего состояния параметров
страховой деятельности. Точно так же профиль мгновенной токсичности
потока заявок является удобно интерпретируемым показателем
неустойчивости текущего состояния потоков заявок.

Из работ~\cite{Boykov2002, Boykov2003} следует

\smallskip

\noindent
\textbf{Лемма~1.}\ \textit{Функция профиля мгновенной токсичности потока заявок
$\varphi_{\pm}(u)$ удовлетворяет интегральному уравнению
\begin{multline*}
\left(\lambda^+ + \lambda^-\right) \varphi_{\pm}(u) =
\lambda^- \int\limits_0^u \varphi_{\pm}(u - v)\,d F(v) + {}\\
{}+\lambda^+
\int\limits_0^{\infty} \varphi_{\pm}(u + v) \,d G(v)\,.
\end{multline*}
Если $R$~--- решение характеристического уравнения
$$
\lambda^+ \left(\Esf e^{-R X^+_1} - 1\right) + \lambda^-
\left(\Esf e^{R X^-_1} - 1\right) = 0\,,
$$
то
$$
\varphi_{\pm}(u) = \fr{e^{-R u}}{\Esf \{\left. e^{-R Q(t)}\right\vert \tau < \infty\}}\,,
$$
при этом} $\varphi_{\pm}(u) \hm\geqslant 1 - e^{-Ru}$.


\section{Токсичность потока заявок}

Профиль токсичности представляет собой {\it функцию}, аргументом
которой является уровень~$u$. Это затрудняет сравнение токсичности
потоков заявок на разных участков рынка, поскольку, вообще говоря,
в~множестве функций
нельзя ввести отношение порядка. Поэтому хотелось бы иметь некий
интегральный показатель токсичности, выражаемый одним числом. Для
построения такого показателя можно воспользоваться одним из двух подходов.

\subsection{Байесовский подход}

Выделим некий <<характеристический>> уровень~$u_0$, пересечение
которого может иметь серьезные последствия. Пусть $w(x)$~---
некоторая плотность распределения вероятностей, обла\-да\-ющая
свойствами
\begin{equation}
\label{eq:evcond}
\int\limits_{0}^{\infty}w(x)\,dx=1\,;\enskip
\int\limits_{0}^{\infty}xw(x)\,dx=u_0\,.
\end{equation}

\noindent
\textbf{Определение~2.}\
Байесовским показателем
\textit{мгновенной токсичности} потока заявок $\theta_{\pm}^{(w)}$
называется величина
$$
\theta_{\pm}^{(w)}=\theta_{\pm}^{(w)}(u_0)=
\int\limits_{0}^{\infty}\varphi_{\pm}(u)w(u)\,du\,.
$$


По сути показатель мгновенной токсичности потока заявок~$\theta_{\pm}$
есть математическое ожидание <<случайного>> профиля
мгновенной токсичности $\varphi_{\pm}(U)$, где~$U$~--- неотрицательная
случайная величина с~плотностью распределения $w(x)$ и~имеющая
математическое ожидание~$u_0$.

В случае, когда $\Esf Q(t) \hm< 0$, т.\,е.\ $\lambda^+ \Esf X_1^+ \hm< \lambda^-
\Esf X_1^-$, что означает преимущество продавцов над покупателями на
интервале $[0, T]$, вместо $\varphi_{\pm}(u)$ будем рассматривать
вероятность
\begin{multline*}
\varphi_{\mp}(u) = \Psf(\sup\limits_{t > 0} \left( Q^+(t) - Q^-(t) \right) \leqslant u)
= {}\\
{}=\lim\limits_{T \to \infty} \varphi_{\mp}(u, T)\,,
\end{multline*}
которая описывает вероятность того, что при {\it отрицательном}
тренде траектория процесса $Q(t)$ не превысит {\it положительный}
уровень~$u$ при условии, что параметры потока заявок ($\lambda^+$,
$\lambda^-$, $G(x)$ и~$F(x)$) остаются неизменными.

В таком случае в~качестве байесовского показателя \textbf{мгновенной
токсичности} потока заявок~$\theta$ возьмем величину
$$
\theta_{\mp}^{(w)}=\theta_{\mp}^{(w)}(u_0)=\int\limits_{0}^{\infty}\varphi_{\mp}(u)w(u)\,du\,.
$$

\subsection{Квантильный подход}

При условии $\Esf Q(t) > 0$ на промежутке $[0, T]$ зафиксируем
некоторое $0 \hm< \alpha\hm < 1$.

\smallskip

\noindent
\textbf{Определение~3.}\
Квантильным {$\alpha$-по\-ка\-за\-те\-лем мгновенной токсичности} потока заявок
называется такое минимальное значение~$q_{\pm}$, при котором
$\varphi_{\pm}(q_{\pm}) \geqslant \alpha$.


Таким образом, при наличии положительного тренда у~процесса $Q(t)$
квантильный $\alpha$-по\-ка\-за\-тель мгновенной токсичности~---
это настолько минимальное значение~$q_{\pm}$, что вероятность того,
что траектория
процесса $Q(t)$ на интервале $[0, T]$ целиком пройдет выше уровня~$-q_{\pm}$,
больше или равна~$\alpha$. Чем больше значение~$q_{\pm}$, тем более
токсичен поток заявок от покупателей.

По аналогии с~предыдущим пунктом при наличии у процесса $Q(t)$
отрицательного тренда (т.\,е.\ при условии $\lambda^+ \Esf X_1^+ \hm<
\lambda^- \Esf X_1^-$) $\alpha$-кван\-тиль\-ный показатель мгновенной
токсичности~$q_{\mp}$ определяется из уравнения
$\varphi_{\mp}(q_{\mp}) \geqslant \alpha.$
Чем больше значение~$q_{\mp}$, тем более токсичен поток заявок от
продавцов на интервале $[0, T$].

\section{Модели потоков заявок}

В некоторых случаях удается напрямую вы\-чис\-лить профиль мгновенной
токсичности потока \mbox{заявок}. Аналоги моделей, приведенных ниже,
рассмотрены в~работе~\cite{Boykov2002} в~рамках модели
Кра\-ме\-ра--Лунд\-бер\-га со стохастическими премиями.

\subsection{Модель рынка с~заявками единичного объема}

Рассмотрим простейшую модель рынка, где потоки заявок имеют
единичный объем, т.\,е.\
$$
\Psf (X_i^+ = 1) = \Psf (X_i^- = 1) = 1\,.
$$
В этом случае
$$
Q(t) = \sum\limits_{i=1}^{N^+(t)} 1 - \sum\limits_{i=1}^{N^-(t)} 1 = N^+(t) - N^-(t)\,.
$$
Несмотря на очевидно идеальный характер такого примера, он имеет
реальный практический смысл, поскольку при этом становится возможным
учитывать чистые интенсивности потоков заявок и~отслеживать влияние
их соотношения (дисбаланса интенсивностей потоков заявок) на
токсичность ситуации. Более того, в~таком случае рассматриваемый
процесс дисбаланса потоков заявок $Q(t)$ является простейшим
процессом рождения и~гибели, различные характеристики которого можно
исследовать специально разработанными для этого методами.

Если $\lambda_+ > \lambda_-$, то покупатели преобладают над
продавцами и~характеристическое уравнение имеет вид:
$$
\lambda^+ \left[ e^{-R} - 1 \right] + \lambda^- \left[ e^{R} - 1 \right] = 0\,,
$$
откуда $e^R = {\lambda^+}/{\lambda^-}$ или $e^R \hm= 1$. По лемме~1
для $u \hm> 0$ имеем
$$
\varphi_{\pm}(u) \geqslant 1 - \left( \fr{\lambda^-}{\lambda^+} \right)^ {u}\,;\enskip
\varphi_{\pm}(\infty) = 1\,.
$$
Равенство $\varphi_{\pm}(u) \hm= \varphi_{\pm}([u])$ очевидно. Для целых~$u$
интегральное уравнение переходит в~разностное:
\begin{equation}
\lambda_1 \varphi_{\pm}(u + 1) - \left(\lambda^- + \lambda^+\right) \varphi_{\pm}(u)
+ \lambda \varphi_{\pm}(u - 1) = 0\,,
\label{eq:phidiff}
\end{equation}
откуда
$$
\varphi_{\pm}(u) = C_1 + C_2 \left (
\fr{\lambda^-}{\lambda^+} \right )^u\,,\enskip C_1 \hm= \varphi(\infty) = 1\,.
$$
Константу~$C_2$ найдем при подстановке в~уравнение~(\ref{eq:phidiff}) $u \hm= 0$:
$$
\left( \lambda^- + \lambda^+ \right) \varphi(0) = \lambda \varphi(1)\,,\enskip
C_2 = -\fr{\lambda^-}{\lambda^+}\,,
$$
откуда получаем, что для $u \hm> 0$ профиль мгновенной токсичности имеет вид:
$$
\varphi_{\pm}(u) = \varphi_{\pm}([u]) = 1 -  \left( \fr{\lambda^-}{\lambda^+} \right)^{[u] + 1}\,.
$$

\subsubsection{Байесовский показатель мгновенной токсичности}



Коль скоро исследователь может сам назначать уровень~$u_0$,
относительно которого будут рассчитываться характеристики
токсичности потока заявок, будем рассматривать~$u_0$ на множестве
натуральных чисел, а~в~качестве функции $w(u)$ можно выбрать функцию
плотности вероятности распределения Пуассона (относительно считающей
меры), также определенную на множестве натуральных чисел.
Для удобства обозначим $r \hm= {\lambda^-}/{\lambda^+}$, при этом
$r \hm< 1$ (рис.~3). В~таком случае


\begin{center}  %fig3
\vspace*{1pt}
 \mbox{%
 \epsfxsize=77.205mm
 \epsfbox{che-3.eps}
 }
\end{center}

\noindent
{{\figurename~3}\ \ \small{Функция профиля токсичности потока заявок $\varphi_{\pm}(u)$
в~модели рынка с~единичными потоками заявок для разных значений
$r \hm= {\lambda^-}/{\lambda^+}$: \textit{1}~--- $r=2/5$;
\textit{2}~--- 1/2; \textit{3}~--- $r=2/3$.
Штрихпунктирная кривая: функция $w(u)$~---
    плотность (относительно считающей меры) пуассоновского
    распределения со средним $u_0 \hm= 3$}}


\vspace*{9pt}


\addtocounter{figure}{1}



\noindent
\begin{multline*}
\theta_{\pm}^{(w)}(u_0) = \int\limits_{\mathbb{N}} \varphi_{\pm}(u) w(u)\, du ={}\\
{}= \sum\limits_{k=0}^{\infty} (1 - r^{k + 1}) \fr{u_0^k e^{-u_0}}{k!} =
1 - r e^{-u_0} \sum\limits_{k=0}^{\infty} \fr{(ru_0)^k}{k!} = {}\\
{}=1 - r e^{u_0(r - 1)}\,.
\end{multline*}
На рис.~\ref{fig:toxunitsize},\,\textit{a} изображен
график токсичности в~зависимости от значений $r \hm= {\lambda^-}/{\lambda^+}$ для
фиксированного значения $u_0 \hm= 3$ в~условиях положительного тренда
($\lambda^+ \hm> \lambda^-$). Чем меньше значение~$r$, тем токсичнее
рынок. И~напротив: рынок нетоксичен, когда $\lambda^+ \hm= \lambda^-$,
т.\,е.\ наблюдается баланс между покупателями и~продавцами.

\subsubsection{Квантильный показатель мгновенной токсичности}

Для заданного $\alpha \hm\in (0, 1)$ квантильный показатель мгновенной
токсичности~--- это такое минимальное $q_{\pm}\hm \in \mathbb{N}$, при котором
$$
\varphi_{\pm}(q_{\pm}) = 1 - r^{q_{\pm} + 1} \geqslant \alpha\,,
$$
откуда
$$
q_{\pm}(\alpha) = \left \lceil \fr{\ln(1 - \alpha)}{\ln r} - 1 \right \rceil\,.
$$
Заметим, что при $r \hm= 1$ $\alpha$-кван\-тиль\-ный показатель мгновенной
токсичности не определен и~в~таком случае полагается равным нулю.

\begin{figure*} %fig4
\vspace*{1pt}
 \begin{center}
 \mbox{%
 \epsfxsize=160.51mm
 \epsfbox{che-4.eps}
 }
 \end{center}
 \vspace*{-9pt}
\Caption{Графики показателей токсичности в~зависимости от значения
$r \hm= {\lambda^-}/{\lambda^+}$: (\textit{а})~байесовский подход;
(\textit{б})~квантильный подход}
    \label{fig:toxunitsize}
\end{figure*}

На рис.~\ref{fig:toxunitsize},\,\textit{б} на график нанесены различные значения
квантильного показателя мгновенной токсичности потока заявок в~зависимости от
значения $r \hm= {\lambda^-}/{\lambda^+}$ на
промежутке $[0, T]$. Токсичность покупателей максимальна при малых
значениях~$r$ и~близка к нулю при наличии баланса между покупателями
и~продавцами.
Монотонность обоих графиков по $r$ подтверждает обоснованность
использования $\theta_{\pm}$ и~$q_{\pm}$ в~качестве показателей
токсичности потока заявок в~случае модели рынка с~заявками
единичного объема.

\vspace*{-3pt}

\subsection{Модель рынка с~экспоненциальными объемами заявок}

Пусть объемы заявок покупателей и~продавцов имеют экспоненциальное распределение
(рис.~5),
т.\,е.\
$$
G(t) = 1 - e^{-bt}\,; \quad F(t) = 1 - e^{-at}\,.
$$

\begin{center}  %fig5
\vspace*{6pt}
  \mbox{%
 \epsfxsize=78.264mm
 \epsfbox{che-5.eps}
 }
\end{center}

\vspace*{-5pt}

\noindent
{{\figurename~5}\ \ \small{Функция профиля токсичности $\varphi_{\pm}(u)$ в~модели рынка
    с~экспоненциальными объемами заявок для разных наборов
    $(\lambda^+, \lambda^-, b, a)$:
    \textit{1}~--- (3, 1, 2, 1); \textit{2}~--- (5, 2, 2, 1);
    \textit{3}~--- (1, 1, 1, 3).
Штрихпунктирная кривая: весовая функция $w(u)$~---
    плот\-ность гам\-ма-рас\-пре\-де\-ле\-ния $\Gamma(u_0^2, u_0^{-1})$ при
    $u_0 \hm= 2$ }}

    \columnbreak


%\vspace*{9pt}




\addtocounter{figure}{1}



\noindent
В случае, когда покупатели преобладают над продавцами, т.\,е.\
$\lambda^+ / b \hm> \lambda^- / a$, характеристическое уравнение имеет
вид:
$$
\lambda^+ \left[ \fr{b}{b + R} - 1 \right] + \lambda^- \left[ \fr{a}{a - R} -
1 \right] = 0\,,
$$
откуда $R = (\lambda^+ a \hm- \lambda^- b) / (\lambda^+ \hm+ \lambda^-)$ или~0,
а~профиль мгновенной токсичности потока заявок~\cite{Boykov2002}
$$
\varphi_{\pm}(u) = \fr{(a + b) \lambda^-}{(\lambda^+ + \lambda^-) a} \exp
\left( -\fr{\lambda^+ a - \lambda^- b}{\lambda^+ + \lambda^-}\, u \right)\,.
$$
В~случае, когда продавцы преобладают над покупателями,
$$
\varphi_{\mp}(u) = \fr{(a + b) \lambda^+}{(\lambda^+ + \lambda^-) b}
\exp \left( -\fr{\lambda^- b - \lambda^+ a}{\lambda^+ + \lambda^-}
\,u \right)\,.
$$







\subsubsection{Байесовский показатель мгновенной токсичности}

На множестве функций $w(u)$, удовле\-тво\-ря\-ющих условиям~(\ref{eq:evcond}),
рассмотрим функции, удовлетво\-ря\-ющие также условию
\begin{equation}
\label{eq:stdcond}
\int\limits_{0}^{\infty}x^2w(x)\,dx - \left(
\int\limits_{0}^{\infty}xw(x)\,dx \right)^2 = 1\,,
\end{equation}
т.\,е.\ обеспечивающие единичную дисперсию соответствующей случайной
величины, имеющей функцию $w(u)$ в~качестве плотности своего
распределения.

Для вычисления байесовского показателя мгновенной токсичности
возьмем в~качестве $w(u)$ плотность гам\-ма-рас\-пре\-де\-ле\-ния

\pagebreak

\noindent
$$
w(u) = u^{k-1} \fr{e^{-u / \theta}}{\theta^k \, \Gamma(k)}\,,
$$
где $\Gamma(k)$~--- гам\-ма-функ\-ция Эйлера:
$$
\Gamma(k) = \int\limits_0^{+\infty} t^{k-1}e^{-t}\,dt\,.
$$
Поскольку математическое ожидание и~дисперсия случайной величины~$U$,
имеющей гам\-ма-рас\-пре\-де\-ле\-ние, равны $k \theta$ и~$k \theta^2$
соответственно, то с~учетом условий~(\ref{eq:evcond}) и~(\ref{eq:stdcond})
значения~$k$ и~$\theta$ определяются из уравнений
$k \theta \hm= u_0$ и~$k \theta^2 \hm= 1$, откуда $k \hm= u_0^2$ и~$\theta \hm=
u_0^{-1}$.

\begin{figure*}[b] %fig6
\vspace*{8pt}
 \begin{center}
 \mbox{%
 \epsfxsize=164.366mm
 \epsfbox{che-6.eps}
 }
 \end{center}
 \vspace*{-9pt}
  \Caption{Оценка параметров $\lambda^+$, $\lambda^-$, $b$ и~$a$ в~режиме реального
  времени (серый цвет~--- интервалы доверия),
  ось~$x$~--- номер соответствующего $\tau$-ин\-тер\-ва\-ла
  (фьючерс на индекс РТС, дневная сессия 01.07.2014)}
  \label{fig:muhat}
%  \vspace*{-2pt}
\end{figure*}


Для удобства обозначим
\begin{equation}
\label{eq:alphabeta}
\beta =  \fr{(a + b) \lambda^-}{(\lambda^+ + \lambda^-) a} \mbox{ и~}
\gamma = \fr{\lambda^+ a - \lambda^- b}{\lambda^+ + \lambda^-} > 0\,.
\end{equation}
Байесовский показатель мгновенной токсичности равен
\begin{multline*}
\theta_{\pm}^{(w)}(u_0) = \int\limits_{0}^{\infty}\varphi_{\pm}(u)w(u)\,du ={}\\
{}=
\int\limits_{0}^{\infty} \left(1 - \beta e^{-\gamma u}\right) u^{k-1}
\fr{e^{-u / \theta}}{\theta^k \, \Gamma(k)}\, du = {}\\
{}    = 1 - \fr{\beta}{\theta^k \Gamma(k)} \int\limits_0^\infty
e^{-(\gamma + \theta^{-1}) u} u^{k - 1}\, du ={}\\
{}=
\left[ t = \fr{\theta \gamma + 1}{\theta} u \right] =
 1 -{}\\
 {}- \fr{\beta}{(\theta \gamma + 1)^k \Gamma(k)}
\int\limits_0^\infty e^{-t} \fr{\theta^{k - 1}}{(\theta \gamma + 1)^{k - 1}}\,
t^{k - 1} \fr{\theta}{\theta \gamma + 1}\, dt ={} \\
{}    = 1 - \fr{\beta}{(\theta \gamma + 1)^k}\,.
\end{multline*}
После подстановки~$k$ и~$\theta$ получаем значение показателя
$$
\theta_{\pm}(u_0) = 1 - \fr{\beta}{ \left( \gamma u_0^{-1} + 1 \right)^{u_0^2}}\,.
$$

\subsubsection{Квантильный показатель мгновенной токсичности}

Для заданного $\alpha \hm\in (0, 1)$ квантильный показатель мгновенной
токсичности~--- это такое минимальное~$q_{\pm}$, при котором
$$
\varphi_{\pm}(q_{\pm}) = 1 - \beta e^{-\gamma q_{\pm}} \geqslant \alpha\,.
$$
Так как функция~$\varphi_{\pm}$ является непрерывной по~$q_{\pm}$, то
данное неравенство может быть обращено в~равенство, откуда получаем
$$
q_{\pm}(\alpha) = \fr{\ln \beta - \ln (1 - \alpha)}{\gamma}\,.
$$

Заметим, что байесовский и~квантильный показатели токсичности
являются монотонными по каждой из величин~$\beta$ и~$\gamma$.

\section{Реальные данные}

В данном разделе описывается структура данных о~потоках заявок, на
базе которых можно провести валидацию модели,
предложенной в~параграфе~6.2.
{\looseness=1

}

Далее оценим параметры потока заявок
$\lambda^+$, $\lambda^-$, $b$ и~$a$ в~режиме скользящего окна и~рассчитаем
функции профиля мгновенной токсичности, а~так\-же показатели
мгновенной токсичности потока \mbox{заявок} $\theta(t)$ и~$q(t)$ в~режиме
реального времени и~проанализируем адекватность полученных
характеристик.
{\looseness=1

}

\begin{table*}\small
\begin{center}


\tabcolsep=3.5pt
\begin{tabular}{|l|*{15}{c|}r|}
\multicolumn{17}{c}{Пример данных о потоке заявок для фьючерса на индекс РТС}\\[6pt]
\hline
&&&&&&&&&&&&&&&&\\[-9pt]
\multicolumn{1}{|c|}{Время} & Тип & Опрерация & Цена, у.е. & Объем & $b_1$, у.е. & $a_1$, у.е. & $v^b_1$ & $v^a_1$ & $v^b_2$ & $v^a_2$ & $v^b_3$ & $v^a_3$ & $v^b_4$ & $v^a_4$ & $v^b_5$ & $v^a_5$  \\
\hline
10:02:36,444 & L & Покупка & 130\,020 & 2 & 130040 & 130050 & 2 & 4 & 22 & 23 & {\bf 54} & 22 & 81 & 31 & 759 & 20 \\
10:02:36,445 & L & Продажа & 130\,070 & 1 & 130040 & 130050 & 2 & 4 & 22 & 23 & {\bf 55} & 22 & 81 & 31 & 759 & 20 \\
10:02:36,465 & C & Покупка & 130\,040 & 1 & 130040 & 130050 & {\bf 1} & 4 & 22 & 23 & 55 & 22 & 81 & 31 & 759 & 20 \\
10:02:36,473 & L & Покупка & 130\,050 & 3 & 130040 & 130050 & 1 & {\bf 1} & 22 & 23 & 55 & 22 & 81 & 31 & 759 & 20 \\
\hline
\end{tabular}
\end{center}
\end{table*}

\subsection{Описание данных}

Рассматриваются данные о~потоках всех заявок
(лимитных, рыночных и~заявок на отмену) на первые $d \hm= 5$
уровней книги заявок фьючерса на
индекс РТС (Российской торговой системы)
за период с~1 по~30~июля 2014~г. Эти данные дают доступ
к~самой детальной информации о~рыночных торгах в~отличие от данных
о~сделках и~котировках (TAQ, Trades and Quotes), которые часто
используются для анализа высокочастотных данных и~состоящих из цен
и~объемов сделок (что соответствует только рыночным заявкам в~потоке
всех заявок), а~так\-же информации о~цене и~объеме лучших котировок на
покупку и~продажу (т.\,е.\ только первый уровень книги заявок)
с~проставленными моментами времени.

В таблице приведен пример данных о~потоке заявок для фьючерса на
индекс РТС и~о~том, как выглядел срез книги заявок после прихода
соответствующей заявки. Заметим, что на рынке FORTS присутствует
всего два типа заявок: лимитные (L) и~заявки на отмену (C),
а~механизм рыночных заявок участники рынка реализуют самостоятельно
(отправляя лимитные заявки с~ценами, гарантирующими их моментальное
исполнение). Тем не менее имеется возможность оценить параметры
потоков рыночных заявок в~рамках предлагаемой модели, рассматривая для
этого потоки лимитных заявок, которые приводили к сделкам.

%\vspace*{-7pt}

\subsection{Процедура оценки параметров}



Разобьем один из рассматриваемых торговых дней (1~июля 2014~г.)\ на
временн$\acute{\mbox{ы}}$е интервалы с~шагом $\tau \hm= 15$~с.
При этом исключим интервалы времени в~5~мин торгов (с~10:00
до~10:05), а~также в~последние 5~мин торгов (с~18:40 до~18:45),
поскольку они характеризуются аномальными всплесками волатильности,
слабо поддающейся анализу в~рамках представленной модели. Внутри
каждого $\tau$-ин\-тер\-ва\-ла проведем оценку параметров $\lambda^+,
\lambda^-, b$ и~$a$ согласно модели рынка с~экспоненциальными
объемами заявок, предложенной в~параграфе~6.2. Результат оценки
параметров в~режиме реального времени изображен на рис.~\ref{fig:muhat}.

%\vspace*{-7pt}

\subsection{Показатели токсичности}



На основе оценок для $\lambda^+$, $\lambda^-$, $b$ и~$a$ можно вы\-чис\-лить~$\beta$
и~$\gamma$, а~затем построить графики показателей мгновенной токсичности
потока заявок
$\theta(u_0)$ и~$q(\alpha)$ для фиксированных~$u_0$ и~$\alpha$
в~режиме реального времени (рис.~7)
и~идентифицировать участки, на которых деятельность покупателей или
продавцов была токсичной. Прикладные исследования демонстрируют
достаточную значимость данного показателя для своевременной
идентификации участков неблагоприятного отбора мар\-кет-мей\-керов.

%\vspace*{-7pt}

\section{Заключение}

В~данной работе рассмотрена микроструктурная модель рынка,
в~которой потоки заявок моделируются пуассоновскими процессами
с~постоянными интенсивностями (такая аппроксимация возможна на
небольших временн$\acute{\mbox{ы}}$х интервалах). В~качестве интегрального индикатора
текущего состояния книги заявок применялся дисбаланс потока заявок
(order flow imbalance), который использует не только текущие значения
наилучших цен покупки и~продажи, но и~влияние событий <<в~глубине>>
книги заявок и~потому меняется существенно быст\-рее и~позволяет
интерполировать динамику рынка между изменениями цены, в~частности
отслеживать ситуации, связанные с~токсичностью потока заявок.
В~рамках рассмотренной модели были введены такие понятия,
как мгновенный профиль токсичности, а~так\-же байесовский и~квантильний
показатели токсичности, рассчитываемые на основе параметров, описывающих
потоки всех заявок. Эти показатели рассчитываются для двух модельных типов
потоков заявок, в~первом из которых заявки имеют единичный объем, во втором~---
объем заявок является случайным и~имеющим показательное распределение. Для
последней из двух моделей была проведена валидация на реальных данных
(фьючерс на индекс РТС) и~были построены показатели токсичности в~режиме
реального\linebreak\vspace*{-12pt}
\begin{center}  %fig7
\vspace*{1pt}
 \mbox{%
 \epsfxsize=79.735mm
 \epsfbox{che-7.eps}
 }
\end{center}

%\vspace*{-3pt}

\noindent
{{\figurename~7}\ \ \small{Графики $\beta$ и $\gamma$, рассчитанных по формулам~(\ref{eq:alphabeta}),
 байесовского и~квантильного показателей токсичности в~режиме реального времени,
 ось~$x$~--- номер сооветствующего $\tau$-ин\-тер\-ва\-ла
 (фьючерс на индекс РТС, дневная сессия 01.07.2014)}}

 \vspace*{13pt}




\noindent
 времени. Предложенная методика расчета показателей токсичности
на основе информации о~потоках всех заявок является перспективной и~может
быть распространена на модели рынка с~неоднородными интенсивностями потоков
заявок.


\smallskip

Автор статьи выражает огромную благодарность своему научному
руководителю профессору Виктору Юрьевичу Королеву за ценные идеи и~замечания,
а~также студентам факультета вычислительной
математики и~кибернетики МГУ
им.\ М.\,В.~Ломоносова Дарье Николайчук и~Гелане
Хазеевой за помощь в~подготовке материала статьи.

%\vspace*{-9pt}

{\small\frenchspacing
 {%\baselineskip=10.8pt
 \addcontentsline{toc}{section}{References}
 \begin{thebibliography}{99}
\bibitem{Jeria2008} %1
\Au{Jeria, D., Sofianos~G.}
Passive orders and natural adverse selection~//
Street Smart, September~4, 2008. No.\,33.

\bibitem{Glosten1985} %2
\Au{Glosten L.\,R., Milgrom~P.} Bid, ask and transaction prices in a specialist market
with heterogeneously informed traders~// J.~Financ. Econ., 1985.
Vol.~14. P.~71--100.

\bibitem{Kyle1985} %3
\Au{Kyle A.\,S.} Continuous auctions and insider trading~// Econometrica, 1985.
Vol.~53. P.~1315--1335.

\bibitem{Easley1992} %4
\Au{Easley D.,  O'Hara~M}. Time and the process of security price adjustment~//
J.~Financ., 1992. Vol.~47. P.~576--605.

\bibitem{Easley2012} %5
\Au{Easley D., Lopez de Prado~M., O'Hara~M.}
Flow toxicity and liquidity in a high frequency world.
\textit{Rev. Financ. Stud.}, 2012. Vol.~25. No.\,5. P.~1457--1493.

\bibitem{Korolev_2013} %6
\Au{Королев В.\,Ю., Черток А.\,В.,  Корчагин~А.\,Ю., Горшенин~А.\,К.}
Ве\-ро\-ят\-ност\-но-ста\-ти\-сти\-че\-ское моделирование информационных потоков
   в~сложных финансовых системах на основе высокочастотных данных~//
  Информатика и её применения, 2013. Т.~7. Вып.~1. С.~12--21.

\bibitem{Chertok2014} %7
\Au{Chertok A., Korolev V., Korchagin~A., Shorgin~S.}
Modeling high-frequency non-homogeneous order flows by compound Cox processes.
January~14, 2014. %Available at SSRN:
{\sf http://ssrn.com/ abstract=2378975}.


\bibitem{ContRamaStoikov2010b} %8
\Au{Cont R., Stoikov~S., Talreja~R.} A~stochastic model for order book dynamics~//
Oper. Res., 2010. Vol.~58. No.\,3. P.~549--563.

\bibitem{ContLarrard2011} %9
\Au{Cont R., de Larrard~A.} Price dynamics in a~Mar\-ko\-vian limit order market.
Working paper.
{\sf http://ssrn.com/\linebreak abstract=1735338}.

\bibitem{Bouchaud2002} %10
\Au{Bouchaud J.-P.,  Mezard M., Potters~M.}  Statistical properties of
stock order books: Empirical results and models~// Quant. Financ., 2002.
Vol.~2. P.~251--256.

\bibitem{ZovkoFarmer2002} %11
\Au{Zovko I., Farmer~J.\,D.}  The power of patience; A behavioral regularity
in limit order placement~// Quant. Financ., 2002. Vol.~2. P.~387--392.



\bibitem{Cont2011} %12
\Au{Cont R.,  Kukanov A., Stoikov~S.}
The price impact of order book
events. March 01, 2011.
{\sf http://ssrn.com/ abstract=1712822}.

\bibitem{Cont2014}
\Au{Cont R., Kukanov A., Stoikov~S.} The price impact of order book
events~// J.~Financ. Economet., 2014. Vol.~12. No.\,1. P.~47--88.

\bibitem{Korolev_2014}
\Au{Korolev V., Chertok A., Zeifman~A.}
Functional limit theorems for order flow imbalance process.
{\sf http://ssrn. com/abstract=1735338}.


\bibitem{Boykov2002}
\Au{Boykov A.} Cramer--Lundberg model with stochastic premiums~//
Theor. Probab. Appl., 2002. Vol.~47. No.\,3. P.~549--553.

\bibitem{Boykov2003}
\Au{Бойков А.\,В.} Стохастические модели капитала страховой
компании и~оценивание вероятности неразорения.  Дисс.\ \ldots\ канд.
физ.-мат. наук.~--- М.: Математический институт им.\ В.\,А.~Стеклова
РАН, 2003. 83~с.

\bibitem{Temnov2004} %17
\Au{Темнов Г.\,О.} Математические модели риска и~случайного
притока взносов в~страховании. Дисс.\ \ldots\ канд.\
 физ.-мат. наук.~---
С.-Пе\-тер\-бург: Санкт-Пе\-тер\-бург\-ский государственный
ар\-хи\-тек\-тур\-но-стро\-и\-тель\-ный университет, 2004. 102~с.

\bibitem{KorolevBeningShorgin2011}
\Au{Королев В.\,Ю., Бенинг В.\,Е., Шоргин~С.\,Я.}
Математические основы теории риска.~--- 2-е изд., перераб. и~доп.~---
М.: Физматлит, 2011. 620~с.
 \end{thebibliography}

 }
 }

\end{multicols}

\vspace*{-6pt}

\hfill{\small\textit{Поступила в редакцию 08.10.14}}

%\newpage

\vspace*{12pt}

\hrule

\vspace*{2pt}

\hrule

%\vspace*{12pt}

\def\tit{ON THE FORMALIZATION OF~ORDER FLOW TOXICITY ON~FINANCIAL MARKETS}

\def\titkol{On the formalization of order flow toxicity on financial markets}

\def\aut{A.\,V.~Chertok$^{1,2}$}

\def\autkol{A.\,V.~Chertok}

\titel{\tit}{\aut}{\autkol}{\titkol}

\vspace*{-9pt}

 \noindent
 $^1$Faculty of Computational Mathematics and Cybernetics,
M.\,V.~Lomonosov Moscow State University;\linebreak
$\hphantom{^1}$1-52  Leninskiye Gory, GSP-1, Moscow 119991, Russian Federation

\noindent
$^2$Euphoria Group LLC,
 9, bld.~1, of.~6 Arkhangelsky Lane, Moscow 101000, Russian Federation




\def\leftfootline{\small{\textbf{\thepage}
\hfill INFORMATIKA I EE PRIMENENIYA~--- INFORMATICS AND
APPLICATIONS\ \ \ 2014\ \ \ volume~8\ \ \ issue\ 4}
}%
 \def\rightfootline{\small{INFORMATIKA I EE PRIMENENIYA~---
INFORMATICS AND APPLICATIONS\ \ \ 2014\ \ \ volume~8\ \ \ issue\ 4
\hfill \textbf{\thepage}}}

\vspace*{3pt}


\Abste{The paper considers the microstructural order flow model for
financial markets. The order flow imbalance process is used as an integral
indicator of the current state of the limit-order book. The model of order flow
imbalance is used to analyze the properties of the current limit-order
book state, which is considered as two-sided risk process with stochastic premiums.
The concept of order flow toxicity on financial markets is studied.
This concept is formalized with probabilities of crossing fixed levels by the
order flow imbalance process. The paper introduces the concepts of the
instantaneous toxicity profile and Bayesian and quantile indicators of toxicity.
These indicators are calculated for two model types of order flows: the first
one has unit volume orders and the second one consists of orders with random volume
which has exponential distribution.}


\KWE{financial markets; limit-order book; order flow;
order flow imbalance; adverse selection; order flow toxicity;
Poisson process; compound Poisson process; two-side risk process;
risk process with stochastic premiums; ruin probability}

  \DOI{10.14357/19922264140403}


\Ack
\noindent
The research was partly financially supported by the Russian Foundation
for Basic Research (project 14-07-00041а).



%\vspace*{3pt}

  \begin{multicols}{2}

\renewcommand{\bibname}{\protect\rmfamily References}
%\renewcommand{\bibname}{\large\protect\rm References}



{\small\frenchspacing
 {%\baselineskip=10.8pt
 \addcontentsline{toc}{section}{References}
 \begin{thebibliography}{99}

\bibitem{Jeria2008-1}
\Aue{Jeria, D., and G. Sofianos}. September~4, 2008.
Passive orders and natural adverse selection. {\it Street Smart} 33.

\bibitem{Glosten1985-1}
\Aue{Glosten L.\,R., and P. Milgrom}. 1985.
Bid, ask and transaction prices in a specialist market
with heterogeneously informed traders.
{\it J.~Financ. Econ.} 14:71--100.

\bibitem{Kyle1985-1}
\Aue{Kyle, A.\,S.} 1985.
Continuous auctions and insider trading. {\it Econometrica} 53:1315--1335.

\bibitem{Easley1992-1}
\Aue{Easley, D., and M.~O'Hara}. 1992.
Time and the process of security price adjustment.
{\it J.~Financ.} 47:576--605.

\bibitem{Easley2012-1}
\Aue{Easley, D., M.~Lopez de~Prado, and M.~O'Hara}. 2012.
Flow toxicity and liquidity in a high frequency world.
{\it Rev. Financ. Stud.} 25(5):1457--1493.

\bibitem{Korolev_2013-1}
\Aue{Korolev, V., A. Chertok, A.~Korchagin, and A.~Gorshenin}. 2013.
Veroyatnostno-statisticheskoe modelirovanie informatsionnykh
potokov v~slozhnykh finansovykh sistemakh na osnove vysokochastotnykh dannykh
[Probability and statistical modeling of information flows in complex financial
systems from high-frequency data]. \textit{Informatika i ee Primeneniya}~---
\textit{Inform. Appl.} 7(1):12--21.

\bibitem{Chertok2014-1}
\Aue{Chertok, A., V. Korolev , A.~Korchagin, and S.~Shorgin}. 2014.
Modeling high-frequency non-homogeneous order flows by compound Cox processes.
Available at: {\sf http://ssrn.com/abstract=2378975} (accessed January~14, 2014).

\bibitem{ContRamaStoikov2010b-1}
\Aue{Cont, R., S. Stoikov, and R.~Talreja}. 2010.
A~stochastic model for order book dynamics. {\it Oper. Res.} 58(3):549--563.

\bibitem{ContLarrard2011-1}
\Aue{Cont, R., and A.~de~Larrard}.
Price dynamics in a Markovian limit order market.
Working paper. Available at: {\sf http://ssrn.com/abstract=1735338}
 (accessed February 2012).

 \bibitem{Bouchaud2002-1}
\Aue{Bouchaud, J.-P., M. Mezard, and M.~Potters}. 2002.
Statistical properties of stock order books: Empirical results and models.
{\it Quant. Financ.} 2:251--256.

\bibitem{ZovkoFarmer2002-1}
\Aue{Zovko, I., and J.\,D.~Farmer}. 2002. The power of patience;
a~behavioral regularity in limit order placement. {\it Quant. Financ.} 2:387--392.



\bibitem{Cont2011-1}
\Aue{Cont, R., A. Kukanov, and S.~Stoikov}.
The price impact of order book events. Available at:
{\sf http:// ssrn.com/ abstract=1735338} (accessed March~01, 2011).

\bibitem{Cont2014-1}
\Aue{Cont, R., A. Kukanov, and S.~Stoikov}.
2014. The price impact of order book events. {\it J.~Financ. Economet.} 12(1):47--88.

\bibitem{Korolev_2014-1}
\Aue{Korolev, V., A. Chertok, and A.~Zeifman}.
Functional limit theorems for order flow imbalance process.
Available at:
{\sf http://ssrn.com/abstract=1735338} (accessed October~6, 2014).

\bibitem{Boykov2002-1}
\Aue{Boykov, A.} 2002. Cramer--Lundberg model with stochastic premiums.
{\it Theor. Probab.  Appl.} 47(3):549--553.

\bibitem{Boykov2003-1}
\Aue{Boykov, A.} 2003. Stokhasticheskie modeli kapitala strakhovoy
kompanii i~otsenivanie veroyatnosti nerazoreniya [Stochastic
models of the capital of the insurance company and the evaluation of the
 probability of non-bankruptcy]. Ph.D. Diss. Moscow. 83~p.

\bibitem{Temnov2004-1}
\Aue{Temnov, G.\,O.}
2004. Matematicheskie modeli riska i~sluchaynogo pritoka vznosov
v~strakhovanii [Mathematical models of risk and random inflow of
contributions to insurance]. Ph.D. Diss. St.\ Petersburg. 102~p.

\bibitem{KorolevBeningShorgin2011-1}
\Aue{Korolev, V.\,Yu., V.\,E.~Bening, and S.\,Ya.~Shorgin}.
2011. {\it Matematicheskie osnovy teorii riska} [Mathematical
foundations of the risk theory]. Moscow: Fizmatlit. 620~p.

\end{thebibliography}

 }
 }

\end{multicols}

\vspace*{-6pt}

\hfill{\small\textit{Received October 8, 2014}}

\vspace*{-18pt}

\Contrl

\noindent
\textbf{Chertok Andrey V.} (b.\ 1987)~---
junior scientist, Faculty of Computational Mathematics and Cybernetics,
M.\,V.~Lomonosov Moscow State University;
1-52  Leninskiye Gory, GSP-1, Moscow 119991, Russian Federation;
Director General,  Euphoria Group LLC,
 9, bld.~1, of.~6 Arkhangelsky Lane, Moscow 101000,
 Russian Federation;  a.v.chertok@gmail.com
\label{end\stat}

\renewcommand{\bibname}{\protect\rm Литература}
     %2Abst+avt
\def\ld{\ldots}
\def\d{\overline d}
\def\oa{\overline\alpha}

\def\stat{milov}

\def\tit{СТАЦИОНАРНЫЕ ХАРАКТЕРИСТИКИ СИСТЕМЫ ОБСЛУЖИВАНИЯ С~ИНВЕРСИОННЫМ
ПОРЯДКОМ ОБСЛУЖИВАНИЯ, ВЕРОЯТНОСТНЫМ ПРИОРИТЕТОМ И~ГИСТЕРЕЗИСНОЙ
ПОЛИТИКОЙ$^*$}

\def\titkol{Стационарные характеристики системы обслуживания с инверсионным
порядком обслуживания} %, вероятностным приоритетом и гистерезисной политикой}

\def\autkol{Т.\,А.~Милованова, А.\,В.~Печинкин}

\def\aut{Т.\,А.~Милованова$^1$, А.\,В.~Печинкин$^2$}

\titel{\tit}{\aut}{\autkol}{\titkol}

{\renewcommand{\thefootnote}{\fnsymbol{footnote}}\footnotetext[1]
{Работа выполнена при поддержке РФФИ (проекты
№~11-07-00112 и №~12-07-00108).}}

\renewcommand{\thefootnote}{\arabic{footnote}}
\footnotetext[1]{Российский университет дружбы народов, tmilovanova77@mail.ru}
\footnotetext[2]{Институт проблем
информатики Российской академии наук, apechinkin@ipiran.ru}

\vspace*{6pt}

\Abst{Рассматривается однолинейная система массового
обслуживания (СМО) с инверсионным порядком обслуживания,
вероятностным приоритетом и простейшим вариантом
гистерезисной политики.
Найдены основные стационарные показатели функционирования
этой сис\-темы.}

\vspace*{4pt}

\KW{система массового обслуживания; инверсионный порядок
обслуживания; вероятностный приоритет; гистерезисная
политика}

\vspace*{14pt}


\vskip 14pt plus 9pt minus 6pt

      \thispagestyle{headings}

      \begin{multicols}{2}

            \label{st\stat}
            
\section{Введение}

Одним из важнейших направлений исследований в теории
массового обслуживания является изучение СМО с дисциплинами
обслуживания, отличны\-ми от обслуживания заявок в
порядке поступления, поскольку такие дисциплины
часто позволяют практически без каких-либо
усовершенствований повысить качество функционирования
самых разнообразных технических систем, например
ин\-фор\-ма\-ци\-он\-но-те\-ле\-ком\-му\-ни\-ка\-ци\-он\-ных сис\-тем (ИТС).
В~частности, дисциплиной такого рода является инверсионный
порядок обслуживания с вероятностным приоритетом,
введенный в~\cite{1-m} для решения задачи А.\,Д.~Соловьева
об оптимальных распределениях для некоторых типов
дисциплин обслуживания.
Подробное изложение полученных в этом направлении
результатов можно найти в~\cite{2-m}.

В последнее время значительное внимание уделяется также СМО с
гистерезисным управлением, являющимся одним из возможных механизмов
пред\-от\-вра\-ще\-ния различного рода перегрузок в ИТС (см., например,~\cite{3-m}). 
Разновидности гистерезисной политики используются при
обнаружении перегрузок как в сетях общеканальной системы
сигнализации №\,7, так и в сетях, где основой сигнализации является
протокол инициации сеансов связи.

В настоящей работе делается попытка связать эти два
направления исследования с по\-мощью СМО с инверсионным
порядком обслуживания, вероятностным приоритетом и
простейшим ва\-риантом гистерезисной политики, для
которой\linebreak находятся основные стационарные показатели
функционирования.
Отметим, что некоторые типы системы $M/G/1$ с
простейшим вариантом гистерезисной политики при
дисциплине обслуживания заявок в порядке поступления
изучались в~[4--7].

\section{Описание системы}

Рассмотрим однолинейную СМО с накопителем бесконечной
емкости, инверсионным
порядком обслуживания, вероятностным приоритетом и
простейшим вариантом гистерезисной политики.
Опишем функционирование этой СМО.

\begin{figure*} %fig1
\vspace*{1pt}
 \begin{center}
 \mbox{%
 \epsfxsize=114.642mm
 \epsfbox{mil-1.eps}
 }
% \vspace*{-9pt}
\end{center}
\begin{center}
{\small Схематическое изображение функционирования системы: \textit{1}~--- поступление; 
\textit{2}~---
обслуживание}
 \end{center}
\end{figure*}



Вариант гистерезисной политики заключается в следующем
(см.\ рисунок).
Имеется два порога $n_0$ и $n_1$, причем $n_1\hm<n_0$.
Пока число заявок в системе меньше $n_0$, система
функционирует в режиме~0.
Это означает, что заявки поступают с
интенсивностью $\lambda_0$ и имеют длину, распределенную
по закону $B_0(x)$ с плот\-ностью $b_0(x)\hm=B'_0(x)$
и средним значением
$\beta_0\hm=\int\limits_0^\infty x b_0(x)\, dx\hm<\infty$.
Но как только число заявок в системе становится равным
$n_0$, система переходит в режим~1.
В~этом режиме заявки поступают с интенсивностью~$\lambda_1$ и имеют длину, 
распределенную по закону $B_1(x)$
с плотностью $b_1(x)=B'_1(x)$ и средним значением
$\beta_1\hm=\int\limits_0^\infty x b_1(x)\, dx\hm<\infty$.
Так продолжа-\linebreak\vspace*{-12pt}

\pagebreak

\noindent
ется до тех пор, пока чис\-ло заявок в сис\-те\-ме не
станет равным~$n_1$.
Тогда система снова переходит в режим~0 и~т.\,д.


В~системе также реализован инверсионный порядок
обслуживания с вероятностным приоритетом.
Предполагается, что в любой момент времени известны
(остаточные) длины всех заявок в системе.
В момент поступления в систему новой заявки ее длина~$x$
сравнивается с (остаточной) длиной~$y$ заявки на приборе.
При этом если система функционирует в режиме~0, то с
вероятностью $d_0(x,y)$ на прибор становится вновь
поступившая заявка, а находившаяся ранее на приборе
занимает первое место в очереди, и наоборот, с
вероятностью $\d_0(x,y)\hm=1\hm-d_0(x,y)$ старая заявка
продолжает обслуживаться, а новая становится на первое
место в очереди.
Если же система функционирует в режиме~1, то
вероятность постановки на прибор вновь поступившей
заявки равна $d_1(x,y)$, а на первое мес\-то в очереди~---
$\d_1(x,y)\hm=1\hm-d_1(x,y)$.

Будем предполагать, что выполнено условие $\lambda b_1\hm<1$,
необходимое и достаточное для существования
стационарного режима функционирования рассматриваемой
системы.

Будем считать также, что $n_0\hm-n_1\hm\ge 2$.
Это предположение вводится только для того, чтобы не
рассматривать случаи, которые по записи расчетных
формул несколько отличаются от общего вида, и нисколько
не умаляет общности полученных результатов.

\section{Вспомогательные функции}

Пусть в некоторый момент система функционирует
в режиме~0, в системе находится $n$, $n_1\hm<n\hm<n_0$,
заявок и в этот момент поступает в сис\-те\-му и становится
на прибор новая заявка длины~$x$.
Обозначим через $\alpha_n(x)$ вероятность того, что в тот
момент, когда в системе впервые снова останется~$n$
заявок, она по-преж\-не\-му будет пребывать в режиме~0.


Функции $\alpha_n(x)$, $n_1\hm<n\hm<n_0$, удовлетворяют системе
уравнений

\columnbreak

\noindent
\begin{equation}
\label{2.1-m}
\alpha_{n_0-1}(x) \equiv 0\,;                    
\end{equation}

\vspace*{-12pt}

\noindent
\begin{multline}
\label{2.2-m}
\alpha_{n}(x)
= e^{-\lambda_0 x} 
+\int\limits_0^x \lambda_0 e^{-\lambda_0 y}\,dy\times{}\\[2pt]
{}\times
\int\limits_0^\infty b_0(z) \left[
d_0(z,x-y) \alpha_{n+1}(z) \alpha_{n}(x-y) + {}\right.\\[2pt]
\left.{}+ \d_0(z,x-y) \alpha_{n+1}(x-y) \alpha_{n}(z)
\right]  dz\,,\\[4pt]
n=\overline{n_1+1,n_0-2}\,.            % \eqno(2)
\end{multline}
Система уравнений~(\ref{2.1-m}), (\ref{2.2-m}) решается
последовательно, начиная с $n\hm=n_0\hm-1$ и кончая $n\hm=n_1+1$.

При решении уравнения~(\ref{2.2-m}) удобно привести его
к более простому виду. Вводя обозначение
\begin{equation*}
%\label{2.3-m}
a_{n}(x) = e^{\lambda_0 x} \alpha_{n}(x)\,,
\enskip  n=\overline{n_1+1,n_0-2}\,,
\end{equation*}
и производя тривиальные преобразования, получаем из~(\ref{2.2-m}):
\begin{multline}
\label{2.4-m}
a_{n}(x) = 1 +{}\\[2pt]
{}+ \int\limits_0^x \left(
\lambda_0 \int\limits_0^\infty b_0(z) d_0(z,y) \alpha_{n+1}(z)\, dz \right)
a_{n}(y)\, dy + {}\\[2pt]
{}+ \int\limits_0^\infty \left(
\lambda_0 b_0(y) e^{-\lambda_0 y} \int\limits_0^x e^{\lambda_0\, z} \d_0(y,z) \alpha_{n+1}(z)\,dz
\right)\times{}\\[2pt]
{}\times  a_{n}(y) \, dy\,,
\quad n=\overline{n_1+1,n_0-2}\,.
\end{multline}
Последнее соотношение представляет собой интегральное уравнение
\begin{multline}
\label{2.5-m}
a_n(x) = 1 + \int\limits_0^\infty K_n(x,y) a_{n}(y) \, dy\,,
\\[2pt]
 n=\overline{n_1+1,n_0-2}\,, 
\end{multline}
ядро которого имеет вид:

\noindent
\begin{multline*}
K_n(x,y) ={}\\
\hspace*{-7.92743pt}{}=
\begin{cases}
\displaystyle\lambda_0 \left( \int\limits_0^\infty
b_0(z) d_0(z,y) \alpha_{n+1}(z)\, dz+
b_0(y) e^{-\lambda_0 y}\times{}\right.\\
\left.\displaystyle{}\times \int\limits_0^x e^{\lambda_0 z}\, \d_0(y,z)
\alpha_{n+1}(z) \,dz \vphantom{\int\limits^\infty_0}\right),                     &\hspace*{-38pt}y<x\,;     \\
\displaystyle\lambda_0 b_0(y) e^{-\lambda_0 y}
\int\limits_0^x e^{\lambda_0 z} \,\d_0(y,z)
\alpha_{n+1}(z) \,dz\,,                         &\hspace*{-38pt}y>x\,.
\end{cases}
\end{multline*}
Численное решение уравнения~(\ref{2.5-m}) можно произ\-вес\-ти итерационным методом.
При этом в качестве нулевой итерации удобно выбрать тождественно равную нулю функцию.
Тогда итерации будут возрастающими, что позволит контролировать сходимость 
итерационного процесса.

В заключение этого раздела приведем условие на функцию $\d_0(x,y)$, при
котором интегральное уравнение~(\ref{2.4-m}) можно
свести к системе линейных алгебраических уравнений.
А~именно: будем предполагать, что 
\begin{equation}
\label{2.5-1}
\hspace*{-2mm}\d_0(x,y) = \sum\limits_{i=1}^I \d^{(1)}_{0,i}(x) \d^{(2)}_{0,i}(y),
\
 n=\overline{n_1+1,n_0-2}.\!\!
\end{equation}
Тогда, вводя обозначения
\begin{multline*}
c_n(y)= \lambda_0 \int\limits_0^\infty b_0(z) d_0(z,y) \alpha_{n+1}(z)\, dz\,,
\\
 n=\overline{n_1+1,n_0-2}\,;
\end{multline*}

\vspace*{-12pt}

\noindent
\begin{multline*}
c_{n,i}(x) = e^{\lambda_0 x} \d^{(2)}_{0,i}(x) \alpha_{n+1}(x)\,,\\
n=\overline{n_1+1,n_0-2}\,,\enskip
i=\overline{1,I}\,;
\end{multline*}

\vspace*{-12pt}

\noindent
\begin{multline*}
a_{n,i} = \int\limits_0^\infty \lambda_0 b_0(y) e^{-\lambda_0\, y} \d^{(1)}_{0,i}(y)
a_{n}(y) \, dy\,, \\ n=\overline{n_1+1,n_0-2}\,,
\ i=\overline{1,I}\,,
\end{multline*}
получаем из~(\ref{2.4-m}):
\begin{multline}
a_{n}(x)=1+\int\limits_0^x c_n(y) a_{n}(y)\, dy +
\sum\limits_{i=1}^I a_{n,i} \int\limits_0^x c_{n,i}(z)\, dz\,,
\\ n=\overline{n_1+1,n_0-2}\,.
\label{2.6-m}
\end{multline}
%%%%%%%%%%%%%%%%%%%%%%%%%%%%%
Дифференцируя теперь равенство~(\ref{2.6-m}), приходим
к дифференциальному уравнению
\begin{multline}
a'_{n}(x)= c_n(x)\, a_{n}(x) + \sum\limits_{i=1}^I a_{n,i} c_{n,i}(x)\,,
\\ n=\overline{n_1+1,n_0-2}\,,
\label{2.7-m}
\end{multline}
начальное условие для которого задается выражением:
\begin{equation}
\label{2.8-m}
a_n(0) = 1\,,
\enskip n=\overline{n_1+1,n_0-2}\,.
\end{equation}
Решение уравнения (\ref{2.7-m}) с начальным условием~(\ref{2.8-m}) имеет вид:
\begin{multline}
a_{n}(x) = \left(
1+ \sum\limits_{i=1}^I\! a_{n,i} \int\limits_0^x c_{n,i}(y) e^{- C_n(y)} \,dy
\right)
e^{C_n(x)} ,\\  n=\overline{n_1+1,n_0-2},
\label{2.9-m}
\end{multline}
где
\begin{equation*}
%\label{2.9}
C_{n}(x) = \int\limits_0^x c_n(y)\, dy\,,
\enskip n=\overline{n_1+1,n_0-2}\,.
\end{equation*}

Для того чтобы найти коэффициенты
$a_{n,i}$, $i\hm=\overline{1,I}$, умножим равенство~(\ref{2.9-m}) на
$\lambda_0 b_0(x) e^{-\lambda_0 x} \d^{(1)}_{0,j}(x)$
и проинтегрируем в пределах от~0 до~$\infty$. Тогда
\begin{multline*}
%\label{2.10-m}
a_{n,j} = \int\limits_0^\infty \lambda_0 b_0(x) e^{-\lambda_0\, x} \d^{(1)}_{0,j}(x)
e^{C_n(x)} dx + {}\\
{}+ \sum\limits_{i=1}^I a_{n,i} \int\limits_0^\infty \lambda_0 b_0(x) e^{-\lambda_0\, x} 
\d^{(1)}_{0,j}(x) e^{C_n(x)} \,dx\times{}\\
{}\times
\int\limits_0^x c_{n,i}(y) e^{- C_n(y)}  \,dy\,,
\enskip n=\overline{n_1+1,n_0-2}\,.
\end{multline*}
%%%%%%%%%%%%%%%%%%%%%%
Производя эту процедуру при всех
$j$, $j\hm=\overline{1,I}$, получаем систему линейных
ал\-геб\-ра\-и\-че\-ских уравнений, решая которую,
находим коэффициенты $a_{n,i}$ и соответственно
функции $a_{n}(x)$ и $\alpha_{n}(x)$.

В дальнейшем будем пользоваться обозначением
$\oa_n(x) \hm= 1 - \alpha_n(x)$.


Отметим, что, используя приближение $\d_0(x,y)$ с
помощью представления~(\ref{2.5-1}), 
можно найти функцию $a_{n}(x)$ с любой степенью точности.
Однако повышение точности влечет за собой существенное
увеличение числа $I$ коэффициентов $a_{n,i}$ и, 
значит, размерности системы линейных алгебраических уравнений.

\section{Стационарные вероятности состояний}

Обозначим через $p_0$ стационарную вероятность того,
что система свободна.
При $n\hm=\overline{1,n_1}$ или $n\hm\ge n_0$ обозначим через
$p_n(x_1,\ld,x_n)$ стационарную плот\-ность вероятностей того,
что в системе находится $n$ заявок, причем заявка на
приборе\linebreak имеет длину~$x_1$, первая заявка в очереди~---
длину~$x_2$ и~т.\,д.
Наконец, при $n\hm=\overline{n_1+1,n_0-1}$\linebreak через
$p_n(0;x_1,\ld,x_n)$ обозначим стационарную плотность
вероятностей того, что система функционирует в режиме~0 и
в системе находится~$n$ заявок, причем заявка на приборе
имеет длину~$x_1$, первая заявка в очереди~--- длину~$x_2$
и~т.\,д., а через $p_n(1;x_1,\ld,x_n)$~--- аналогичную
вероятность, но при этом система функционирует в режиме~1.

Используя метод исключения состояний (см., например,~\cite{22-m}), 
можно получить для $p_n(x_1,\ld,x_n)$,
$n\hm=\overline{1,n_1}$, уравнения
%%%%%%%%%%%%%%%%%%%%%%%%%%%%%%%
\begin{multline}
\label{3-0-m}
-p'_1(x)=-\lambda_0 p_1(x)+\lambda_0 p_1(x)\int\limits_0^\infty b_0(y) d_0(y,x)\, dy
+{}
\\
{}+\lambda_0 b_0(x)\int\limits_0^\infty p_1(y) \d_0(x,y)\, dy+
\lambda_0 p_0 b_0(x)\,;
\end{multline}
%%%%%%%%%%%%%%%%%%%%%%%%%%%%%%

\vspace*{-12pt}

\noindent
\begin{multline*}
%\label{3.3}
-p'_n(x_1,\ld,x_n) = - \lambda_0 p_{n}(x_1,\ld,x_n) +{}\\
{}+ \lambda_0 p_n(x_1,\ld,x_n)
\int\limits_0^\infty b_0(y) d_0(y,x_1)\, dy
+{}
\\
+ \lambda_0 b_0(x_1)\int\limits_0^\infty p_n(y,x_2,\ld,x_n) \d_0(x_1,y)\, dy
+ {}\\
{}+\lambda_0 p_{n-1}(x_2,\ld,x_n) b_0(x_1) d_0(x_1,x_2) +{}
\\
{}+ \lambda_0 b_0(x_2) p_{n-1}(x_1,x_3,\ld,x_n) \d_0(x_2,x_1),
\ n=\overline{2,n_1},\hspace*{-0.52872pt}
\end{multline*}
%%%%%%%%%%%%%%%%%%%%%%%%%%%%%%
с начальным условием
%%%%%%%%%%%%%%%%%%%%%%%%%%%%%%
$$
\lim_{x\to\infty} p_n(x,x_2,\ld,x_n) = 0\,, \enskip n=\overline{1,n_1}\,.
$$
%%%%%%%%%%%%%%%%%%%%%%%%%%%%%%
Можно выписать аналогичные уравнения для остальных
функций $p_n(x_1,\ld,x_n)$, $n\hm\ge n_0$, и
$p_n(i;x_1,\ld,x_n)$, $n\hm=\overline{n_1+1,n_0-1}$,
$i\hm=1,2$,
но они ввиду громоздкости здесь не приводятся.
Вычисления по этим формулам, хотя теоретически и можно
производить на основе решения интегральных уравнений,
практически не реализуемы уже при совсем небольших
значениях~$n$ даже на современной вычислительной технике,
поскольку размерность уравнений растет пропорционально~$n$.

Однако для практических расчетов, как правило,
достаточно знать только маргинальные стационарные
плотности $p_1(x)$,
\begin{multline*}
p_n(x) = \int\limits_0^\infty \cdots \int\limits_0^\infty
p_n(x,x_2,\ld,x_n)\, dx_2\cdots dx_n\,,
\\ n=\overline{2,n_1}
\enskip \hbox{или}
\enskip n\ge n_0\,,
\end{multline*}
и
\begin{multline*}
p_n(i;x)= \int\limits_0^\infty \cdots\int\limits_0^\infty
p_n(i;x,x_2,\ld,x_n)\, dx_2\cdots dx_n\,,
\\ i=0,1\,,\enskip n=\overline{n_1+1,n_0-1}\,.
\end{multline*}
Для них справедливы соотношения
%%%%%%%%%%%%%%%%%%%%%%%%%%%%%%
\begin{multline}
\label{3-1-m} -p'_n(x) = - f_0(x)\, p_n(x) + \int\limits_0^\infty
k_0(x,y)\, p_n(y)\, dy +{}
\\ 
{}+g_{0,n}(x) \,,\enskip
n=\overline{2,n_1}\,;
\end{multline}
%%%%%%%%%%%%%%%%%%%%%%%%%%%%%%

\vspace*{-12pt}

\noindent
\begin{multline}
\label{3-2-m}
-p'_n(0;x) = - f_{0,n}(x)\, p_n(0,x) +{}\\
{}+ \int\limits_0^\infty k_{0,n}(x,y)\, p_n(0,y)\, dy
+ g_{0,n}(x) \,,
\\  n=\overline{n_1+1,n_0-1}\,;
\end{multline}

%%%%%%%%%%%%%%%%%%%%%%%%%%%%%%
\vspace*{-12pt}

\noindent
\begin{multline}
\label{3-3-m}
-p'_n(1;x) = - f_{1}(x) p_n(1,x) + {}\\
{}+\int\limits_0^\infty k_{1}(x,y) p_n(1,y)\, dy
+ g_{1,n}(x)\,, \\ 
n=\overline{n_1+1,n_0-1}\,;
\end{multline}
%%%%%%%%%%%%%%%%%%%%%%%%%%%%%%

\vspace*{-12pt}

\noindent
\begin{multline}
\label{3-4-m}
-p'_n(x) = - f_{1}(x) p_n(x) +{}\\
{}+\int\limits_0^\infty k_{1}(x,y) p_n(y)\, dy + g_{1,n}(x)\,,
\  n\ge n_0\,,
\end{multline}
с начальными условиями
%%%%%%%%%%%%%%%%%%%%%%%%%%%%%%
\begin{equation}
\label{3-beg-1-m}
\lim_{x\to\infty} p_n(x) = 0\,,
\ \ n=\overline{2,n_1}\ \ \hbox{или}\ \ n\ge n_0\,,
\end{equation}
%%%%%%%%%%%%%%%%%%%%%%%%%%%%%%
\begin{equation}
\label{3-beg-2-m}
\lim_{x\to\infty} p_n(i;x) = 0\,,
\ n=\overline{n_1+1,n_0-1}\,,\ \ i=0,1\,,
\end{equation}
%%%%%%%%%%%%%%%%%%%%%%%%%%%%%%
в которых для сокращения записи введены сле\-ду\-ющие
обозначения:
%%%%%%%%%%%%%%%%%%%%%%%%%%%%%%%
\begin{align*}
f_0(x) &= \lambda_0 \left(
1 - \int\limits_0^\infty b_0(y) d_0(y,x)\, dy \right)\,;
\\
k_0(x,y) &= \lambda_0 b_0(x) \d_0(x,y) \,;
\\
g_{0,1}(x) &= \lambda_0 p_0 b_0(x) \,;\\
g_{0,n}(x) &= \lambda_0 b_0(x) \int\limits_0^\infty
p_{n-1}(y) d_0(x,y)\, dy +{}\\
&\hspace*{2mm}{}+
\lambda_0 p_{n-1}(x) \int\limits_0^\infty \d_0(y,x) b_0(y)\, dy\,,
\ \ n=\overline{2,n_1}\,;
\end{align*}
%%%%%%%%%%%%%%%%%%%%%%%%%%%%%%

\vspace*{-24pt}

\noindent
\begin{multline*}
f_{0,n}(x)= \lambda_0 \left(
1 - \int\limits_0^\infty b_0(y) d_0(y,x) \alpha_n(y)\, dy
\right)\,,
\\ n=\overline{n_1+1,n_0-1}\,;
\end{multline*}
%%%%%%%%

\vspace*{-12pt}

\noindent
\begin{multline*}
k_{0,n}(x,y) = \lambda_0 b_0(x) \d_0(x,y) \alpha_n(y) \,,
\\ n=\overline{n_1+1,n_0-1}\,;
\end{multline*}
%%%%%%%%%

\vspace*{-12pt}

\noindent
\begin{multline*}
g_{0,n_1+1}(x) = \lambda_0 b_0(x) \int\limits_0^\infty p_{n_1}(y) d_0(x,y)\, dy
+{}\\
{}+
\lambda_0 p_{n_1}(x) \int\limits_0^\infty b_0(y) \d_0(y,x)\, dy\,;
\end{multline*}
%%%%%%%%%

\vspace*{-12pt}

\noindent
\begin{multline*}
g_{0,n}(x) = \lambda_0 b_0(x) \int\limits_0^\infty p_{n-1}(0;y) d_0(x,y)\, dy
+{}\\
{}+ \lambda_0 p_{n-1}(0;x) \int\limits_0^\infty b_0(y) \d_0(y,x)\, dy\,,
\\ 
n=\overline{n_1+2,n_0-1}\,;
\end{multline*}
%%%%%%%%%%%%%%%%%%%%%%%%%%%%%%
\begin{align}
\label{3-gf-1-m}
f_1(x)&= \lambda_1 \left(
1 - \int\limits_0^\infty b_1(y) d_1(y,x)\, dy\right)\,;
\\
\label{3-gf-2-m}
k_1(x,y) &= \lambda_1 b_1(x) \d_1(x,y) \,,
\end{align}
%%%%%%%%%%%

\vspace*{-12pt}

\noindent
\begin{multline*}
g_{1,n_1+1}(x)={}\\
 {}=\lambda_0 p_{n_1+1}(0;x)\int\limits_0^\infty b_0(y) d_0(y,x) 
\oa_{n_1+1}(y)\, dy
+{}
\\
{}+ \lambda_0 b_0(x) \int\limits_0^\infty p_{n_1+1}(0;y) \d_0(x,y) \oa_{n_1+1}(y)\, dy\,;
\end{multline*}
%%%%%%%%%%


\vspace*{-12pt}

\noindent
\begin{multline*}
g_{1,n}(x) = \lambda_0 p_{n}(0;x) \int\limits_0^\infty b_0(y) d_0(y,x) \oa_n(y)\, dy
+{}\\
{}+ \lambda_0 b_0(x) \int\limits_0^\infty p_{n}(0;y) \d_0(x,y) \oa_n(y)\, dy
+ {}\\
{}+ \lambda_1 b_1(x) \int\limits_0^\infty p_{n-1}(1;y) d_1(x,y)\, dy+{}\\
{}+
\lambda_1 p_{n-1}(1;x) \int\limits_0^\infty b_1(y) \d_1(y,x)\, dy\,,
\\ n=\overline{n_1+2,n_0-1}\,;
\end{multline*}
%%%%%%%%%%%%%%%%%%%%%%%%%%%%%%

\vspace*{-24pt}

\noindent
\begin{multline*}
g_{1,n_0}(x) = \lambda_0 b_0(x)\int\limits_0^\infty p_{n_0-1}(0;y) d_0(x,y)\, dy
+{}\\
{}+
\lambda_0 p_{n_0-1}(0;x) \int\limits_0^\infty b_0(y) \d_0(y,x)\, dy+{}
\\
{}+ \lambda_1 b_1(x) \int\limits_0^\infty p_{n_0-1}(1;y) d_1(x,y)\, dy+{}\\
{}+
\lambda_1 p_{n_0-1}(1;x) \int\limits_0^\infty b_1(y) \d_1(y,x)\, dy \,;
\end{multline*}
%%%%%%%%%%%%%%%%%%%%%%%%%%

\vspace*{-12pt}

\noindent
\begin{multline}
\label{3-gf-3-m}
g_{1,n}(x)= \lambda_1 b_1(x) \int\limits_0^\infty p_{n-1}(y) d_1(x,y)\, dy
+{}\\
{}+ \lambda_1 p_{n-1}(x) \int\limits_0^\infty b_1(y) \d_1(y,x)\, dy\,.
\ \ n>n_0\,,
\end{multline}
%%%%%%%%%%%%%%%%%%%%%%%%%%%%%%%

Вероятность $p_0$ вычисляется из условия нормировки
$$
p_0 + \sum\limits_{n=1}^{n_1} p_n+\sum\limits_{n=n_1+1}^{n_0-1} \left[p_{n,0} + p_{n,1}\right]
+
\sum\limits_{n=n_0}^{\infty} p_n = 1\,,
$$
где $p_n = \int\limits_0^\infty p_n(x)\, dx$,
$n\hm=\overline{1,n_1}$ или $n\hm\ge n_0$,~---
стационарная вероятность того, что в системе находится
$n$ заявок, а $p_{n,i} \hm= \int\limits_0^\infty p_n(i;x)\, dx$,
$n\hm=\overline{n_1+1,n_0-1}$, $i\hm=0,1$,~--- стационарная
вероятность того, что система функционирует в режиме~$i$ и
в системе находится $n$~заявок.

Уравнения~(\ref{3-0-m})--(\ref{3-4-m}) легко приводятся к
интегральным. Действительно, вводя новые обозначения
%%%%%%%%%%%%%%%%%%%%%%%%%%%%%%%
\begin{align*}
F_0(x) &= \int\limits_0^x f_0(y)\, dy\,,
\ \ n=\overline{1,n_1}\,;
\\
%%%%%%%%%%%%%%%%%%%%%%%%%%%%%%
F_{0,n}(x) &= \int\limits_0^x f_{0,n}(y)\, dy\,,
\ \ n=\overline{n_1+1,n_0-1}\,;
\\
%%%%%%%%%%%%%%%%%%%%%%%%%%%%%%
F_{1}(x)&= \int\limits_0^x f_{1}(y)\, dy\,,
\ \ n\ge n_1+1\,;
\\
%%%%%%%%%%%%%%%%%%%%%%%%%%%%%%
p_n(x) &= \pi_n(x) e^{F_0(x)} \,, \ \ n=\overline{1,n_1}\,;
\\
%%%%%%%%%%%%%%%%%%%%%%%%%%%%%%
p_n(0;x) &= \pi_n(0;x) e^{F_{0,n}(x)} \,, \ \ n=\overline{n_1+1,n_0-1}\,;
\\
%%%%%%%%%%%%%%%%%%%%%%%%%%%%%%
p_n(1;x) &= \pi_n(1;x) e^{F_{1}(x)} \,, \ \ n=\overline{n_1+1,n_0-1}\,;
\\
p_n(x) &= \pi_n(x) e^{F_{1}(x)} \,, \ \ n\ge n_0\,,
\end{align*}

%%%%%%%%%%%%%%%%%%%%%%%%%%%%%%


\noindent
из \eqref{3-0-m}--\eqref{3-4-m} получаем соотношения

\noindent
%%%%%%%%%%%%%%%%%%%%%%%%%%%%%%
\begin{multline}
\label{pi-1-m}
-\pi'_n(x) = e^{-F_0(x)} \int\limits_0^\infty e^{F_0(y)} k_0(x,y) \pi_n(y)\, dy
+{}\\
{}+
e^{-F_0(x)} g_{0,n}(x) \,,
\ \ n=\overline{1,n_1}\,;
\end{multline}
%%%%%%%%%%%%%%%%%%%%%%%%%%%%%%

\vspace*{-12pt}

\noindent
\begin{multline}
\label{pi-2-m}
-\pi'_n(0;x) = {}\\
{}=e^{-F_{0,n}(x)} \int\limits_0^\infty e^{F_{0,n}(y)} 
k_{0,n}(x,y) \pi_n(0;y)\, dy +{}\\
{}+
e^{-F_{0,n}(x)} g_{0,n}(x) \,, \ \ n=\overline{n_1+1,n_0-1}\,;
\end{multline}
%%%%%%%%%%%%%%%%%%%%%%%%%%%%%%

\vspace*{-12pt}

\noindent
\begin{multline}
\label{pi-3-m}
-\pi'_n(1;x) = e^{-F_{1}(x)} \int\limits_0^\infty e^{F_{1}(y)} k_{1}(x,y) \pi_n(1;y)\, dy
+{}\\
{}+
e^{-F_{1}(x)} g_{1,n}(x) \,, \ \ n=\overline{n_1+1,n_0-1}\,;
\end{multline}
%%%%%%%%%%%%%%%%%%%%%%%%%%%%%%

\vspace*{-12pt}

\noindent
\begin{multline}
\label{pi-4-m}
-\pi'_n(x) = e^{-F_{1}(x)} \int\limits_0^\infty e^{F_{1}(y)} k_{1}(x,y)\pi_n(y)\, dy
+{}\\
{}+
e^{-F_{1}(x)} g_{1,n}(x) \,, \ \ n\ge n_0\,,
\end{multline}
%%%%%%%%%%%%%%%%%%%%%%%%%%%%%%
интегрируя которые в пределах от~$x$ до $\infty$
и учитывая начальные условия~(\ref{3-beg-1-m}),
(\ref{3-beg-2-m}), имеем
\begin{multline}
\label{int-1-m}
\pi_n(x) ={}\\
{}= \int\limits_0^\infty e^{F_0(y)} \left(
\int\limits_x^\infty e^{-F_0(u)} k_0(u,y)\, du
\right) \pi_n(y)\, dy
+{}\\
{}+
\int\limits_x^\infty e^{-F_0(u)} g_{0,n}(u)\, du\,;
\ \ n=\overline{1,n_1}\,,
\end{multline}

\vspace*{-12pt}

\noindent
\begin{multline}
\label{int-2-m}
\pi_n(0;x) ={}\\
{}= \int\limits_0^\infty\! e^{F_{0,n}(y)} \left(
\int\limits_x^\infty\! e^{-F_{0,n}(u)} k_{0,n}(u,y)\, du \right)
\pi_n(0;y)\, dy+
\\
{}+
\int\limits_x^\infty e^{-F_{0,n}(u)} g_{0,n}(u)\, du\,,
\enskip n=\overline{n_1+1,n_0-1}\,;
\end{multline}
%%%%%%%%%%%%%%%%%%%%%%%%%%%%%%


\vspace*{-12pt}

\noindent
\begin{multline}
\label{int-3-m}
\pi_n(1;x) ={}\\
{}= \int\limits_0^\infty e^{F_{1}(y)} \left(
\int\limits_x^\infty e^{-F_{1}(u)} k_{1}(u,y)\, du\right)
\pi_n(1;y)\, dy
+{}
\\
{}+
\int\limits_x^\infty e^{-F_{1}(u)} g_{1,n}(u)\, du\,,
\ \ n=\overline{n_1+1,n_0-1}\,;
\end{multline}
%%%%%%%%%%%%%%%%%%%%%%%%%%%%%%
\begin{multline}
\label{int-4-m}
\hspace*{-5mm}\pi_n(x) = \int\limits_0^\infty e^{F_{1}(y)} \left(
\int\limits_x^\infty e^{-F_{1}(u)} k_{1}(u,y)\, du \right)
\pi_n(y)\, dy +{}\\
{}+
\int\limits_x^\infty e^{-F_{1}(u)} g_{1,n}(u)\, du\,,
\ \ n\ge n_0\,.
\end{multline}
%%%%%%%%%%%%%%%%%%%%%%%%%

Соотношения \eqref{int-1-m}--\eqref{int-4-m} являются
интегральными уравнениями такого же вида, что и~\eqref{2.5-m},
и к ним применимы те же методы решения, что и
к уравнению~\eqref{2.5-m}.

Так же как для функций $\alpha_n(x)$, приведем условия
для функций $\d_0(x,y)$ и $\d_1(x,y)$, которые поз\-во\-ляют
получить решения ин\-тег\-ро\-диф\-фе\-рен\-ци\-аль\-ных уравнений~\eqref{pi-1-m}--\eqref{pi-4-m}
с помощью приведения к системе линейных алгебраических уравнений.
А~именно: будем предполагать, что выполнены условия~\eqref{2.5-1} и
\begin{equation}
\label{2.5-2-m}
\d_1(x,y) = \sum\limits_{i=1}^{I_1} \d^{(1)}_{1,i}(x) \d^{(2)}_{1,i}(y) \,.
\end{equation}
Тогда, вводя обозначения
$$
c_i(x) = \lambda_0 b_0(x) \d^{(1)}_{0,i}(x) e^{-F_0(x)}\,,\ \ i=\overline{1,I}\,;
$$
$$
q_{n,i} = \int\limits_0^\infty e^{F_0(y)} \d^{(2)}_{0,i}(y) \pi_n(y)\, dy \,,\ \ 
n=\overline{1,n_1}\,,
\ \ i=\overline{1,I}\,;
$$
$$
q_{n}(x)= e^{-F_0(x)} g_{0,n}(x)\,,
\ \ n=\overline{1,n_1}\,;
$$
%%%%%%%%%%%%%%%%%%%%%%%%%%%%%%%%%%%%%%%

\vspace*{-12pt}

\noindent
\begin{multline*}
c_{0;n,i}(x)= \lambda_0 b_0(x) \d^{(1)}_{0,i}(x) e^{-F_{0,n}(x)}\,,\\
n=\overline{n_1+1,n_0-1}\, ,\ \ i=\overline{1,I}\,;
\end{multline*}

\vspace*{-12pt}

\noindent
\begin{multline*}
q_{0;n,i}= \int\limits_0^\infty e^{F_{0,n}(y)} \d^{(2)}_{0,i}(y) \alpha_n(y) \pi_n(0;y)\, dy\,,
\\ n=\overline{n_1+1,n_0-1}\,,
\ \ i=\overline{1,I}\,;
\end{multline*}
$$ 
q_{0;n}(x) = e^{-F_{0,n}(x)} g_{0,n}(x) \,, \ \ n=\overline{n_1+1,n_0-1}\,;
$$
%%%%%%%%%%%%%%%%%%%%%%%%%%%%%%%%%%%%%%%
$$
c_{1;i}(x)= \lambda_1 b_1(x) \d^{(1)}_{1,i}(x) e^{-F_{1}(x)}\,,\ \ i=\overline{1,I_1}\,;
$$

\vspace*{-12pt}

\noindent
\begin{multline*}
q_{1;n,i}= \int\limits_0^\infty e^{F_{1}(y)} \d^{(2)}_{1,i}(y) \pi_n(1;y)\, dy\,,\\ 
n=\overline{n_1+1,n_0-1}\,,
\ \ i=\overline{1,I_1}\,    ;
\end{multline*}
$$
q_{1;n}(x)= e^{-F_{1}(x)} g_{1,n}(x)\,,
\ \ n\ge n_1+1\,;
$$
%%%%%%%%%%%%%%%%%%%%%%%%%%%%%%%%%%%%%%%%%%

\vspace*{-12pt}

\noindent
\begin{equation*}
q_{1;n,i} = \int\limits_0^\infty e^{F_{1}(y)}  \d^{(2)}_{1,i}(y) \pi_n(y)\, dy\,,\\ 
n\ge n_0\,,\  i=\overline{1,I_1}\,,
\end{equation*}
%%%%%%%%%%%%%%%%%%%%%%%%%%%%%%%%%%%%%%%%%%%
получаем из (\ref{pi-1-m})--(\ref{pi-4-m}) после
интегрирования в пределах от $x$ до $\infty$
с учетом начальных условий~(\ref{3-beg-1-m}),
(\ref{3-beg-2-m}):
%%%%%%%%%%%%%%%%%%%%%%%%%%%%%%%%%%%%%%%%%%%
\begin{equation}
\label{ipi-1-m}
\pi_n(x)= \sum\limits_{i=1}^{I} C_i(x) q_{n,i}+Q_{n}(x) \,,
\ \ n=\overline{1,n_1}\,;
\end{equation}
%%%%%%%%%%%%%%%%%%%%%%%%%%%%%%

\vspace*{-24pt}

\noindent
\begin{multline}
\label{ipi-2-m}
\pi_n(0;x) = \sum\limits_{i=1}^{I} C_{0;n,i}(x) q_{0;n,i} +
Q_{0;n}(x) \,, \\[1pt]
 n=\overline{n_1+1,n_0-1}\,;
\end{multline}
%%%%%%%%%%%%%%%%%%%%%%%%%%%%%%

\vspace*{-12pt}

\noindent
\begin{multline}
\label{ipi-3-m}
\pi_n(1;x) = \sum\limits_{i=1}^{I_1} C_{1;i}(x) q_{1;n,i} +
Q_{1;n}(x)\,, \\[1pt] 
n=\overline{n_1+1,n_0-1}\,;
\end{multline}
%%%%%%%%%%%%%%%%%%%%%%%%%%%%%%
\begin{equation}
\label{ipi-4-m}
\pi_n(x)= \sum\limits_{i=1}^{I_1} C_{1;i}(x) q_{1;n,i} +
Q_{1;n}(x)\,, \ \ n\ge n_0\,,
\end{equation}
%%%%%%%%%%%%%
где
%%%%%%%%%%%%%
$$
C_{i}(x)= \int\limits_x^\infty c_{i}(y)\, dy\,, \ \ n=\overline{1,n_1}\,,\ \ i=\overline{1,I}\,;
$$

\vspace*{-12pt}

\noindent
\begin{multline*}
C_{0;n,i}(x)= \int\limits_x^\infty c_{0;n,i}(y)\, dy\,, \\ 
n=\overline{n_1+1,n_0-1}\,,\ \ i=\overline{1,I}\,;
\end{multline*}
$$
C_{1,i}(x) = \int\limits_x^\infty c_{1,i}(y)\, dy\,,
\ \ n\ge n_1\,,\ \ i=\overline{1,I_1}\,;
$$
%%%%%%%%%%%%%%%%%%%%%%%%%%%%%%
$$
Q_{n}(x) = \int\limits_x^\infty q_{n}(y)\, dy\,,
\ \ n=\overline{1,n_1}\,;
$$
$$
Q_{0;n}(x)= \int\limits_x^\infty q_{0;n}(y)\, dy\,,
\ \ n=\overline{n_1+1,n_0-1}\,;
$$
$$
Q_{1;n}(x)= \int\limits_x^\infty q_{1;n}(y)\, dy\,,
\ \ n\ge n_1\,.
$$

Для определения постоянных
$q_{n,i}$, $q_{0;n,i}$ и $q_{1;n,i}$
умножим равенства~\eqref{ipi-1-m}--\eqref{ipi-4-m} на
$\d^{(2)}_{0,j}(y) e^{F_0(y)}$,
$\d^{(2)}_{0,j}(y) \alpha_n(y) e^{F_{0,n}(y)}$
и $\d^{(2)}_{1,j}(y) e^{F_{1}(y)}$
соответственно и проинтегрируем в пределах от~0 до~$\infty$.
Тогда
%%%%%%%%%%%%%%%%%%%%%%%%%%%%%%%%%%%%%%%%%%%
\begin{multline}
\label{cons-1-m}
q_{n,j}= \sum\limits_{i=1}^{I} \int\limits_0^\infty \d^{(2)}_{0,j}(y) 
e^{F_0(y)} C_i(y)\, dy\, q_{n,i}
+{}\\[2pt]
\hspace*{-2mm}{}+
\int\limits_0^\infty \d^{(2)}_{0,j}(y) e^{F_0(y)}  Q_{n}(y)\, dy\,,
\  n=\overline{1,n_1}\,,
\  j=\overline{1,I}\,;\!\!
\end{multline}
%%%%%%%%%%%%%%%%%%%%%%%%%%%%%%

\vspace*{-12pt}

\noindent
\begin{multline*}
q_{0;n,j}= {}\\[2pt]
{}=\sum\limits_{i=1}^{I} \int\limits_0^\infty \d^{(2)}_{0,j}(y) \alpha_n(y)
e^{F_{0,n}(y)} C_{0;n,i}(y)\, dy\, q_{0;n,i}
+{}
\end{multline*}

\noindent
\begin{multline}
\label{cons-2-m}
{}+
\int\limits_0^\infty \d^{(2)}_{0,j}(y) \alpha_n(y) e^{F_{0,n}(y)} Q_{0;n}(y)\, dy\,,\\ 
n=\overline{n_1+1,n_0-1}\,, \enskip j=\overline{1,I}\,;
\end{multline}
%%%%%%%%%%%%%%%%%%%%%%%%%%%%%%

\vspace*{-12pt}

\noindent
\begin{multline}
\label{cons-3-m}
q_{1;n,j} = \sum\limits_{i=1}^{I_1} \int\limits_0^\infty \d^{(2)}_{1,j}(y) 
e^{F_{1}(y)}\, C_{1;i}(y)\, dy\, q_{1;n,i}
+{}\\
{}+
\int\limits_0^\infty \d^{(2)}_{1,j}(y) e^{F_{1}(y)} Q_{1;n}(y)\, dy\,,\\ 
n\ge n_1+1\,, \enskip j=\overline{1,I_1}\,.
\end{multline}
%%%%%%%%%%%%%

Каждое из соотношений~\eqref{cons-1-m}--\eqref{cons-3-m}
представляет собой систему линейных алгебраических
уравнений, что позволяет легко находить коэффициенты
$q_{n,i}$, $q_{0;n,i}$ и $q_{1;n,i}$ и в конечном
счете плотности $p_n(x)$, $p_n(0;x)$ и $p_n(1;x)$.


В~заключение этого раздела приведем выражение для
суммарной стационарной интенсивности~$\lambda$ входящего
потока:
\begin{multline}
\label{inten-1-m}
\lambda = \lambda_0 p_0 + \lambda_0 \sum\limits_{n=1}^{n_1}
p_n +{}\\
\!\!{}+ \lambda_0 \sum\limits_{n=n_1+1}^{n_0-1} p_{n,0}
+ \lambda_1 \sum\limits_{n=n_1+1}^{n_0-1} p_{n,1}
+ \lambda_1 \sum\limits_{n=n_0}^{\infty} p_n .\!\!
\end{multline}

\section{Применение производящих функций}

Для вычисления моментов стационарного распределения
числа заявок в системе можно воспользоваться производящей функцией (ПФ):
$$
p(z,x) = \sum\limits_{n=n_0}^\infty z^n p_n(x)\,.
$$
Правда, для того чтобы определить ПФ $p(z,x)$,
необходимо знать плотности вероятностей $p_{n_0-1}(0;x)$
и $p_{n_0-1}(1;x)$, а для этого предварительно вы\-чис\-лить
$p_{n}(x)$, $n\hm=\overline{1,n_1}$, $p_{n}(0;x)$, $n\hm=\overline{n_1+1,n_0-2}$, и
$p_{n}(1;x)$, $n\hm=\overline{n_1+1,n_0-2}$.

Умножая соотношения~\eqref{3-4-m} на $z^n$ и суммируя по~$n$, 
получаем после простейших преобразований с
учетом~\eqref{3-gf-1-m}--\eqref{3-gf-3-m}
\begin{multline}
\label{3.pf-m}
- p'_{x}(z,x) = - (1-z) f_1(x)\, p(z,x) +{}\\
{}+ \lambda_1 b_1(x)
\int\limits_0^\infty p(z,y) \left[\d_1(x,y) + z d_1(x,y)\right]\, dy
+{}\\
{}+
z^{n_0} g_{1,n_0}(x) 
\end{multline}
с начальным условием
\begin{equation}
\label{3.pf-b-m}
\lim_{x\to\infty} p(z,x) = 0\,.
\end{equation}

Уравнение~\eqref{3.pf-m} с начальным условием~\eqref{3.pf-b-m} легко приводится к интегральному
уравнению
\begin{equation*}
%\label{3.pf-2-m}
q(z,x) = \int\limits_0^\infty K(x,y) q(z,y)\, dy
+ z^{n_0} R(x)\,,
\end{equation*}
где
$$
q(z,x) = e^{-(1-z) F_1(x)} p(z,x)\,;
$$

\vspace*{-12pt}

\noindent
\begin{multline*}
K(x,y) = \lambda_1 \int\limits_x^\infty\! e^{(1-z) \left[F_1(y) - F_1(u)\right]}
b_1(u) \left[\,\d_1(u,y) +{}\right.\\
\left.{}+ z d_1(u,y)\right]\, du\,;
\end{multline*}
%%%%%%%%%%%%%%%
$$
R(x) = \int\limits_x^\infty e^{-(1-z) F_1(u)} g_{1,n_0}(u)\, du\,.
$$
Последнее уравнение имеет такой же вид, как и~\eqref{2.5-m},
с теми же замечаниями относительно решения, что и раньше.
Кроме того, если выполнено условие~\eqref{2.5-2-m},
то решение этого уравнения, как и прежде, сводится
к решению системы линейных алгебраических уравнений.

Производящая функция $P(z)$ стационарного распределения числа заявок в
системе без учета их длин и режима функционирования
определяется формулой:
\begin{multline*}
P(z) = p_0 + \sum\limits_{n=1}^{n_1} z^n p_{n} +
\sum\limits_{n=n_1+1}^{n_0-1} z^n \left[p_{n,0} + p_{n,1}\right]
+{}\\
{}+ \int\limits_0^\infty e^{(1-z) F_1(x)} q(z,x)\, dx\,.
\end{multline*}

Моменты стационарного распределения числа заявок
в системе вычисляются с помощью дифференцирования
ПФ $P(z)$ в точке $z\hm=1$ и по\-сле\-ду\-юще\-го решения
получившихся уравнений.

\section{Стационарное распределение времени пребывания
заявки в~системе}

Обозначим через $u(s;x)$ преобразование Лап\-ла\-са--Стилть\-еса
(ПЛС) для открываемого заявкой длины
$x$ периода занятости (ПЗ) обычной СМО $M/G/1/\infty$
с интенсивностью $\lambda_1$ входящего потока и функцией распределения $B_1(x)$
времени обслуживания заявки,
а через $u(s)$ --- то же самое ПЛС, но для ПЗ, открываемого
заявкой произвольной длины.
Тогда
\begin{align*}
u(s;x)&= e^{-[s + \lambda_1 - \lambda_1 u(s)]\,x}\,;
\\
u(s) &= \beta_1(s + \lambda_1 - \lambda_1 u(s))\,.
\end{align*}
%%%%%%%%%%%%%%%%%%%%%%%%%%

Предположим теперь, что в начальный момент
рассматриваемая СМО функционирует в режиме~0 и
в ней находится~$n$, $n\hm=\overline{1,n_0-1}$,
заявок.
Обозначим через $u_n(s;x)$, $n\hm=\overline{1,n_0-1}$, ПЛС времени до того момента,
когда в системе впервые останется $n-1$ заявок
и при этом система по-преж\-не\-му будет функционировать
в режиме~0, при условии что на приборе начала
обслуживаться заявка длины~$x$, а через $u^*_n(s;x)$, $n\hm=\overline{n_1+2,n_0-1}$,~--- 
функцию, подобную $u_n(s;x)$, но при этом система перейдет в режим~1.

Справедливы уравнения
%%%%%%%%%%%
\begin{equation}
\label{5-1-m}
u'_{n_0-1}(s;x) = - \left[s + \lambda_0\right] u_{n_0-1}(s;x)\,;
\end{equation}
%%%%%%%%%

\vspace*{-12pt}

\noindent
\begin{multline*}
u'_n(s;x) = - (s + \lambda_0) u_{n}(s;x) +{}
\\
{}+
\lambda_0 \int\limits_0^\infty b_0(y) \left[d_0(y,x) u_{n+1}(s;y) u_{n}(s;x)
+{}\right.\\
\left.{}+ \d_0(y,x) u_{n+1}(s;x)\, u_{n}(s;y)\right] \, dy\,,
\\ 
n=\overline{n_1+2,n_0-2}\,,
\end{multline*}
%%%%%%%%%
с начальным условием
$$
u_n(s;0)= 1 \,, \ \ n=\overline{n_1+2,n_0-1}\,,
$$
%%%%%%%%%%%
уравнения
\begin{multline}
\label{5-2-m}
u^{*\,\prime}_{n_0-1}(s;x) = - [s + \lambda_0] u^*_{n_0-1}(s;x)+
\\
{}+
\lambda_0 \int\limits_0^\infty b_0(y) \left[d_0(y,x)  u(s;y)\, u(s;x)+{}\right.\\
\left.{}+
\d_0(y,x) u(s;x)\, u(s;y)\right] \, dy\,;
\end{multline}
%%%%%%%%%%%

\vspace*{-12pt}

\noindent
\begin{multline}
\label{5-2-2-m}
u^{*\,\prime}_n(s;x) = - \left[s + \lambda_0\right] u^*_{n}(s;x) +{}
\\
{}+
\lambda_0 \int\limits_0^\infty b_0(y) \left[d_0(y,x) u^*_{n+1}(s;y) u(s;x)
+ {}\right.\\
\left.{}+\d_0(y,x) u^*_{n+1}(s;x) u(s;y)\right] \, dy
+{}\\
{}+ \lambda_0 \int\limits_0^\infty b_0(y) \left[d_0(y,x) u_{n+1}(s;y) u^*_{n}(s;x)
+ {}\right.\\
\left.{}+d_0(y,x) u_{n+1}(s;x) u^*_{n}(s;y)\right] \, dy\,,
\\  n=\overline{n_1+2,n_0-2}\,,
\end{multline}
%%%%%%%%%
с начальным условием
$$
u_n(s;0) = 0\,, \ \ n=\overline{n_1+2,n_0-1}\,,
$$
и уравнения
\begin{multline}
\label{5-2-3-m}
u'_{n_1+1}(s;x) = - \left[s + \lambda_0\right] u_{n_1+1}(s;x) +{}
\\
{}+
\lambda_0 \int\limits_0^\infty b_0(y) \left[d_0(y,x) u^*_{n_1+2}(s;y) u(s;x)+{}\right.\\
\left.{}+
\d_0(y,x) u^*_{n_1+2}(s;x) u(s;y)\right] \, dy+{}
\\
{}+ 
\lambda_0 \int\limits_0^\infty b_0(y) \left[d_0(y,x) u_{n_1+2}(s;y) u_{n_1+1}(s;x) +{}\right.\\
\left.{}+
\d_0(y,x) u_{n_1+2}(s;x) u_{n_1+1}(s;y)\right]\, dy\,;
\end{multline}
%%%%%%%%%

\vspace*{-12pt}

\noindent
\begin{multline*}
u'_n(s;x) = - (s + \lambda_0) u_{n}(s;x) +{}\\
{}+
\lambda_0 \int\limits_0^\infty b_0(y) \left[d_0(y,x) u_{n+1}(s;y) u_{n}(s;x)
+{}\right.\\
\left.{}+\d_0(y,x) u_{n+1}(s;x) u_{n}(s;y)\right] \, dy\,,
\ \ n=\overline{1,n_1}\,,
\end{multline*}
%%%%%%%%%
с начальным условием
$$
u_n(s;0)= 1 \,,
\ \ n=\overline{1,n_1}\,.
$$

Решения уравнений~\eqref{5-1-m} и~\eqref{5-2-m} имеют вид:
%%%%%%%%%%%
$$
u_{n_0-1}(s;x)= e^{-(s + \lambda_0) x}\,;
$$
%%%%%%%%%%%%%%%%%%%%%%%%%%

\vspace*{-12pt}

\noindent
\begin{multline}
\label{5-2-4-m}
u^*_{n_0-1}(s;x) = \lambda_0 \int\limits_0^x e^{(s + \lambda_0) (z-x)}\, dz\times{}\\
{}\times \int\limits_0^\infty b_0(y) \left[d_0(y,z) u(s;y) u(s;z)+{}\right.\\
\left.{}+
\d_0(y,z) u(s;z) u(s;y)\right] \, dy\,.
\end{multline}
Остальные уравнения являются интегродифференциальными
и подобны уравнениям, полученным в предыдущих разделах.

Пусть в начальный момент в системе находится
$n$, $n\hm\ge n_1\hm+1$, заявок, система функционирует
в режиме~1, на приборе обслуживается заявка
длины~$y$ и в этот момент в систему поступает
заявка длины~$x$. Обозначим через $w(s;x,y)$ ПЛС времени ожидания
начала обслуживания этой заявки.
Тогда
$$
w(s;x,y) = d_1(x,y) + \d_1(x,y) u(s;y) \,.
$$

Пусть в начальный момент в системе находится
$n$, $n\hm=\overline{1,n_0-1}$, заявок, система
функционирует в режиме~0, на приборе обслуживается
заявка длины~$y$ и в этот момент в систему поступает
заявка длины~$x$.
Обозначим через $w_n(s;x,y)$ ПЛС времени ожидания
начала обслуживания этой заявки, причем в момент
начала обслуживания система по-преж\-не\-му будет функционировать в режиме~0.
Имеем:
$$
w_{n_0-1}(s;x,y) = 0\,;
$$
%%%%%%%%%%

\vspace*{-24pt}

\noindent
\begin{multline*}
w_n(s;x,y) = d_0(x,y) + \d_0(x,y) u_{n+1}(s;y)\, ,\\  
n=\overline{1,n_0-2}\,.
\end{multline*}

Наконец, пусть в начальный момент в системе находится~$n$, 
$n\hm=\overline{n_1+1,n_0-1}$, заявок, система
функционирует в режиме~0, на приборе обслуживается
заявка длины~$y$ и в этот момент в систему поступает
заявка длины~$x$.
Обозначим через $w^*_n(s;x,y)$ ПЛС времени ожидания
начала обслуживания этой заявки, причем в момент
начала обслуживания сис\-те\-ма окажется в режиме~1.
В~этом случае
\begin{equation}
\label{5-3-3-m}
w^*_{n_0-1}(s;x,y) = d_0(x,y) + \d_0(x,y) u(s;y)\,;
\end{equation}
%%%%%%%%%
$$
w^*_n(s;x,y) = \d_0(x,y) u^*_{n+1}(s;y)\,,\ \ n=\overline{n_1+1,n_0-2} \,.
$$


Стационарное распределение времени ожидания начала
обслуживания имеет ПЛС
\begin{multline*}
%\label{5-3-4}
w(s) = \fr{1}{ \lambda} \left[ \vphantom{\int\limits_0^\infty}
\lambda_0 p_0+{}\right.\\
{}+ \lambda_0 \int\limits_0^\infty \sum\limits_{n=1}^{n_1}
p_n(y) \, dy \int\limits_0^\infty b_0(x) w_n(s;x,y) \, dx
+{}
\\
{}+
\lambda_0 \int\limits_0^\infty \sum\limits_{n=n_1+1}^{n_0-1} p_n(0;y)\, dy\times{}\\
{}\times
\int\limits_0^\infty b_0(x) \left[w_n(s;x,y) + w^*_n(s;x,y)\right]\, dx
+{}
\\
{}+
\lambda_1 \int\limits_0^\infty \sum\limits_{n=n_1+1}^{n_0-1} p_n(1;y) \, dy
\int\limits_0^\infty b_1(x) w(s;x,y) \, dx
+{}\\
\left.{}+
\lambda_1 \int\limits_0^\infty \sum\limits_{n=n_0}^{\infty} p_n(y) \, dy
\int\limits_0^\infty b_1(x) w(s;x,y) \, dx
\right]\,.
\end{multline*}

Обозначим через $t(s;x)$ ПЛС времени от момента
первого попадания заявки длины~$x$
на прибор до момента ухода ее из системы при условии,
что в момент первого попадания на прибор система
функционировала в режиме~1.
Для $t(s;x)$ справедливо дифференциальное уравнение
\begin{multline*}
t'(s;x)= - t(s;x) \left( \vphantom{\int\limits_0^\infty}
s + {}\right.\\
\!\!\left.{}+\lambda_1 \!\left[
1 - \int\limits_0^\infty\! b_1(y)\left[\d_1(y,x) +
d_1(y,x)\, u(s;y)\vphantom{\overline{d}}\right] \, dy
\right]\!
\right)\hspace*{-1.717pt}
\end{multline*}
с начальным условием
$$
t(s;0)= 1 \,,
$$
решение которого имеет вид:
\begin{multline*}
t(s;x) = \exp\left\{ \vphantom{\int\limits_0^\infty}
-(s + \lambda_1) x +{}\right.\\
\!\!\left.{}+ \lambda_1 \!\int\limits_0^x\,\! dz\!
\int\limits_0^\infty b_1(y) \left[\,\d_1(y,z) +  d_1(y,z) u(s;y)\right] \, dy
\right\}.
\end{multline*}

Обозначим через $t_n(s;x)$, $n\hm=\overline{0,n_0-2}$,
ПЛС времени от момента первого попадания заявки длины~$x$
на прибор до момента ухода ее из системы при условии,
что в момент первого попадания на прибор в очереди
было еще $n$~заявок и система функционировала в режиме~0.
Тогда из дифференциальных уравнений
%%%%%%%%%%%%%%%%%%%
\begin{multline*}
t'_{n_0-2}(s;x) = - (s + \lambda_0) t_{n_0-2}(s;x) +{}\\
{}+
\lambda_0 \int\limits_0^\infty b_0(y) \left[d_0(y,x) u(s;y) + \d_0(y,x)\right] \, dy\, t(s;x)\,;
\end{multline*}
%%%%%%%%%%%%%%%%%%%

\vspace*{-12pt}

\noindent
\begin{multline*}
t'_n(s;x) = -\left( \vphantom{\int\limits_0^\infty}
s + \lambda_0 - {}\right.\\
\left.{}-\lambda_0 \int\limits_0^\infty b_0(y) d_0(y,x) u_{n+2}(s;y) \, dy
\right)
t_n(s;x) +{}
\\
{}+
\lambda_0 \int\limits_0^\infty b_0(y) \left[d_0(y,x) u_{n+2}^*(s;y) t(s;x) +{}\right.\\
\left.{}+
\d_0(y,x) t_{n+1}(s;x)\right] \, dy\,,
\ \ n=\overline{n_1,n_0-3}\,;
\end{multline*}
%%%%%%%%%%%%%%%%%%%%%

\vspace*{-12pt}

\noindent
\begin{multline*}
t'_n(s;x)= - \left( \vphantom{\int\limits_0^\infty}
s + \lambda_0 - {}\right.\\
\left.{}-\lambda_0 \int\limits_0^\infty b_0(y) d_0(y,x) u_{n+2}(s;y)\, dy
\right) t_n(s;x) +{}
\\
{}+
\lambda_0 t_{n+1}(s;x) \int\limits_0^\infty b_0(y) \d_0(y,x) \, dy\,,
\ \ n=\overline{0,n_1-1}\,,
\end{multline*}
%%%%%%%%%%%%%%%%%%%%%%%%%%%%%%%%%
с начальным условием
$$
t_{n}(s;0)= 1\,,\ \ n=\overline{0,n_0-2}\,,
$$
%%%%%%%%%%%%%%%%%%%%%%%%%%%%%%%
имеем:

\noindent
\begin{multline}
\label{5-4-1-m}
t_{n_0-2}(s;x)= {}\\
{}=e^{-(s + \lambda_0) x}\left(
1+ \lambda_0 \int\limits_0^x e^{(s + \lambda_0) z} t(s;z)\, dz\times{}\right.\\
\left.{}\times
\int\limits_0^\infty b_0(y) \left[d_0(y,z) u(s;y) + \d_0(y,z)\right] \, dy
\right)\,;
\end{multline}
%%%%%%%%%%%%%%%%%%%

\vspace*{-12pt}

\noindent
\begin{multline}
\label{5-4-2-m}
t_n(s;x)={}\\
{}= e^{-\int\limits_0^x \left( s + \lambda_0 -
\lambda_0 \int\limits_0^\infty b_0(y)\, d_0(y,z)\, u_{n+2}(s;y) \, dy
\right)\,dz} 
\left( \vphantom{\int\limits_0^\infty}
1 +{}\right.\\
{}+ \lambda_0 \int\limits_0^x e^{\int\limits_0^v \left(
s + \lambda_0 - \lambda_0 \int\limits_0^\infty b_0(y) d_0(y,z) u_{n+2}(s;y) \, dy
\right)\,dz} dv \times{}
\\
{}\times
\int\limits_0^\infty b_0(y) \left[d_0(y,v) u_{n+2}^*(s;y) t(s;v)+{}\right.\\
\left.\left.{}+ \d_0(y,v) t_{n+1}(s;v)
\vphantom{\int\limits_0^\infty}
\right] \, dy
\right) \,, \quad 
n=\overline{n_1,n_0-3}\,;
\end{multline}
%%%%%%%%%%%%%%%%%%%

\vspace*{-12pt}

\noindent
\begin{multline*}
t_n(s;x) = {}\\
{}=e^{-\int\limits_0^x\left( \vphantom{\int\limits_0^\infty}
s + \lambda_0 - \lambda_0 \int\limits_0^\infty b_0(y) d_0(y,z) u_{n+2}(s;y) \, dy \right)
dz} \left(  \vphantom{\int\limits_0^x}
1 +{}\right.\\
{}+ \lambda_0 \int\limits_0^x e^{\int\limits_0^v\left(
s + \lambda_0 -\lambda_0 \int\limits_0^\infty b_0(y) d_0(y,z) u_{n+2}(s;y) \, dy
\right) dz } dv \times{}\\
\left.{}\times
\int\limits_0^\infty b_0(y) \d_0(y,v) t_{n+1}(s;v) \, dy
\right)\,,
\ \ n=\overline{0,n_1-1}\,.
\end{multline*}

Обратимся к общему времени пребывания заявки в системе.

Обозначим через $v(s;x,y)$
ПЛС времени пребывания в системе заявки длины~$x$ при
условии, что эта заявка застала систему в режиме~1,
причем заявка на приборе имела длину~$y$.
Тогда
$$
v(s;x,y)= w(s;x,y) t(s;x)\,.
$$

Обозначим через
$v_{n}(s;x,y)$, $n\hm=\overline{1,n_0-1}$,
ПЛС времени пребывания в системе заявки длины~$x$ при
условии, что эта заявка застала в системе $n$~других
заявок, причем заявка на приборе имела длину~$y$, а
система пребывала в режиме~0.
Имеем:
\begin{multline}
\label{5-5-1-m}
v_{n_0-1}(s;x,y) = d_0(x,y) t(s;x) + {}\\
{}+\d_0(x,y) u(s;x,y) t(s;x)\,;
\end{multline}
%%%%%%%%%%%%%

\vspace*{-12pt}

\noindent
\begin{multline}
\label{5-5-2-m}
v_{n}(s;x,y) = d_0(x,y) t_{n}(s;x) +{}\\
{}+
\d_0(x,y) \left[ u_{n+1}(s;x,y) t_{n-1}(s;x) +{}\right.\\
\left.{}+ u^*_{n+1}(s;x,y) t(s;x)\right]\,,
\ \ n=\overline{n_1+1,n_0-2}\,;
\end{multline}
%%%%%%%%%%%%%

\vspace*{-20pt}

\noindent
\begin{multline*}
v_{n}(s;x,y)= d_0(x,y) t_{n}(s;x) + {}\\
{}+\d_0(x,y) u_{n+1}(s;x,y) t_{n-1}(s;x)\,,
\ \ n=\overline{1,n_1}\,.
\end{multline*}
%%%%%%%%%%%%%

Стационарное распределение общего времени пребывания
заявки в системе имеет ПЛС
\begin{multline*}
v(s) = \fr{1}{\lambda} \left[
\lambda_0 p_0 \int\limits_0^\infty b_0(x) t_0(s;x) \, dx
+ {}\right.\\
{}+\lambda_0 \int\limits_0^\infty \sum\limits_{n=1}^{n_1} p_n(y) \, dy
\int\limits_0^\infty b_0(x) v_n(s;x,y) \, dx +{}
\\
{}+
\lambda_0 \int\limits_0^\infty \sum\limits_{n=n_1+1}^{n_0-1}
p_n(0;y)\, dy \int\limits_0^\infty b_0(x) v_n(s;x,y) \, dx
+{}
\\
{}+
\lambda_1 \int\limits_0^\infty \sum\limits_{n=n_1+1}^{n_0-1}
p_n(1;y)\, dy \int\limits_0^\infty b_1(x) v(s;x,y) \, dx
+{}\\
\left. \lambda_1 \int\limits_0^\infty \sum\limits_{n=n_0}^{\infty}
p_n(y)\, dy \int\limits_0^\infty b_1(x) v(s;x,y)\, dx
\right]\,.
\end{multline*}

Дифференцируя $w(s)$ и $v(s)$ в точке $s\hm=0$,
можно найти моменты стационарных распределений времен
ожидания начала обслуживания и пребывания заявки в
сис\-теме.

\section{Накопитель конечной емкости}

В этом разделе будет показано, какие изменения нужно
произвести в полученных формулах для случая накопителя
конечной емкости.
Заметим, что к формулам, остающимся без изменений,
комментарии приводиться не будут.


Итак, будем предполагать, что максимальное число заявок,
находящихся в системе, равно $n^*$, $n^*\hm\ge n_0$,
(емкость накопителя $n^*\hm-1$).

Для конечного накопителя необходимо также задать
дисциплину принятия заявок в систему при отсутствии в нем
свободных мест.
В~соответствии с рассматриваемой СМО естественно такую
дисциплину определить с помощью функции $d^*(x,y)$
следующим образом: поступающая заявка длины~$x$,\linebreak
застающая на приборе заявку длины~$y$, с ве\-ро\-ят\-ностью
$d^*(x,y)$ сразу же покидает сис\-те\-му, не оказывая на нее
никакого воздействия, и с дополнительной вероятностью
$\d^*(x,y)\hm=1\hm-d^*(x,y)$ становит\-ся на прибор, вытесняя
заявку на приборе из сис\-те\-мы.
Для всех СМО с такой дисциплиной принятия заявок в сис\-те\-му
при отсутствии в накопителе свободных мест стационарные
вероятности $p_n(x_1,\ldots,x_n)$ при $n\hm<n^*$ совпадают
с точ\-ностью до постоянной с аналогичными вероятностями
для сис\-те\-мы с бесконечным накопителем, различие заключается
только в вероятностях $p_{n^*}(x_1,\ldots,x_{n^*})$.
Однако несколько более сложно вычисляются стационарные
распределения, связанные с временем пребывания заявки в
системе.
Более того, заявки, принятые в систему, могут покидать
ее недообслуженными.

Здесь для простоты изложения будет рассмотрен только
случай $d^*(x,y)\hm=1$, т.\,е.\ тот случай, когда поступающая
в заполненную систему заявка теряется.
Заметим, что в этом случае принятая в систему заявка
будет обязательно обслужена полностью.
Общий случай нетрудно исследовать с помощью результатов,
полученных в~[9, 10].

Далее будем предполагать, что $n^*\hm\ge n_0 \hm+ 2$, поскольку
при $n^*\hm=n_0$ и $n^*\hm=n_0 \hm+ 1$ расчетные формулы будут
несколько отличаться от приведенных выше.

Как уже говорилось, стационарные вероятности
$p_n(x)$, $n\hm=\overline{1,n_1}$ или
$n\hm=\overline{n_0,n^*-1}$, и
$p_n(i;x)$, $n\hm=\overline{n_1+1,n_0-1}$, $i\hm=1,2$,
с точностью до вероятности $p_0$ можно определить
из тех же самых уравнений~(\ref{3-0-m})--(\ref{3-4-m}),
что и раньше.
Вероятность $p_{n^*}(x)$ удовлетворяет дифференциальному
уравнению
\begin{equation*}
-p'_{n^*}(x)= g_{1,n^*}(x) 
\end{equation*}
с начальным условием
%%%%%%%%%%%%%%%%%%%%%%%%%%%%%%
\begin{equation*}
%\label{3-beg-1}
\lim\limits_{x\to\infty} p_{n^*}(x) = 0\,,
\end{equation*}
%%%%%%%%%%%%%%%%%%%%%%%%%%%%%%
где
%%%%%%%%%%%%%%%%%%%%%%%%%%
\begin{multline*}
%\label{3-gf-3}
g_{1,n^*}(x) = \lambda_1 b_1(x) \int\limits_0^\infty p_{n^*-1}(y) d_1(x,y)\, dy
+{}\\
{}+ \lambda_1 p_{n^*-1}(x) \int\limits_0^\infty b_1(y) \d_1(y,x)\, dy\,.
\end{multline*}
%%%%%%%%%%%%%%%%%%%%%%%%%%%%%%%
Решение этого уравнения определяется вы\-ра\-же\-нием:
\begin{equation*}
%\label{3-4}
p_{n^*}(x)= \int\limits_x^\infty g_{1,n^*}(y)\, dy\,.
\end{equation*}
%%%%%%%%%%%%%%%%%%%%%%%%%
Вероятность $p_0$ вычисляется из условия нормировки,
которое в данном случае имеет вид:
$$
p_0 + \sum\limits_{n=1}^{n_1} p_n + \sum\limits_{n=n_1+1}^{n_0-1} \left[p_{n,0} + p_{n,1}\right]+
\sum\limits_{n=n_0}^{n^*} p_n = 1\,.
$$

Стационарная интенсивность~$\lambda$ входящего в сис\-те\-му
потока задается формулой~(\ref{inten-1-m}), в которой,
естественно, верхний индекс~$\infty$ в последней сумме
заменен на~$n^*$.

В системах с конечным накопителем важной характеристикой
является стационарная вероятность $\pi_{\mathrm{loss}}$
потери заявки, определяемая формулой:
$$
\pi_{\mathrm{loss}} = \fr{\lambda_1 }{\lambda}\, p_{n^*}\,.
$$

Для того чтобы найти показатели функционирования СМО,
связанные с временем пребывания в системе, нужно прежде
всего изменить некоторые формулы для ПЛС~ПЗ.

Предположим, что в начальный момент рас\-смат\-ри\-ва\-емая
СМО функционирует в режиме~1 и в ней находится
$n$, $n\hm=\overline{n_1+1,n^*}$, заявок.
Обозначим через $\tilde{u}_n(s;x)$, $n\hm=\overline{n_1+1,n^*}$, ПЛС времени до того момента,
когда в системе впервые останется $(n-1)$ заявок,
при условии что на приборе начала
обслуживаться заявка длины~$x$ (очевидно, что в этот
момент система по-преж\-не\-му будет функционировать в режиме~1).
Преобразования Лап\-ла\-са--Стил\-тье\-са $\tilde{u}_n(s;x)$ удовлетворяют уравнениям 
\begin{align*}
\tilde{u}_{n^*}(s;x)&= e^{-sx} \,;\\
\\
\tilde{u}'_n(s;x) &= - (s + \lambda_1) \tilde{u}_{n}(s;x) +{}
\\
&{}+
\lambda_1 \int\limits_0^\infty b_1(y) \left[d_1(y,x) \tilde{u}_{n+1}(s;y) \tilde{u}_{n}(s;x)
+ {}\right.\\
&\hspace*{5mm}\left.{}+\d_1(y,x) \tilde{u}_{n+1}(s;x) \tilde{u}_{n}(s;y)\right] \, dy\,,
\\  
&\hspace*{35mm}n=\overline{n_{1}+1,n^*-1}\,,
\end{align*}
%%%%%%%%%
с начальным условием
$$
\tilde{u}_n(s;0)=1\,,
\ \ n=\overline{n_1+1,n^*-1}\,.
$$
Уравнения~(\ref{5-2-m})--(\ref{5-2-3-m}) принимают следующий вид:
%%%%%%%%%%%
\begin{multline*}
%\label{5-2}
u^{*\,\prime}_{n_0-1}(s;x) = - \left[s + \lambda_0\right] u^*_{n_0-1}(s;x)
+{}
\\
{}+
\lambda_0 \int\limits_0^\infty b_0(y) \left[d_0(y,x) \tilde{u}_{n_0}(s;y) \tilde{u}_{n_0-1}(s;x)
+{}\right.\\
\left.{}+ \d_0(y,x) \tilde{u}_{n_0}(s;x) \tilde{u}_{n_0-1}(s;y)\right]\, dy \,;
\end{multline*}
%%%%%%%%%%%

\vspace*{-12pt}

\noindent
\begin{multline*}
u^{*\,\prime}_n(s;x) = - \left[s + \lambda_0\right] u^*_{n}(s;x)
+{}
\\
{}+ \lambda_0 \int\limits_0^\infty b_0(y) \left[d_0(y,x) u^*_{n+1}(s;y) \tilde{u}_n(s;x)
+ {}\right.\\
\left.{}+\d_0(y,x) u^*_{n+1}(s;x) \tilde{u}_n(s;y)\right] \, dy +{}\\
{}+
\lambda_0 \int\limits_0^\infty b_0(y) \left[d_0(y,x) u_{n+1}(s;y) u^*_{n}(s;x)
+{}\right.\\
\left.{}+\d_0(y,x) u_{n+1}(s;x) u^*_{n}(s;y)\right]\, dy\,,
\\ n=\overline{n_1+2,n_0-2}\,;
\end{multline*}
%%%%%%%%%

\vspace*{-24pt}

\noindent
\begin{multline*}
u'_{n_1+1}(s;x) = - \left[s + \lambda_0\right] u_{n_1+1}(s;x) +{}
\\
{}+
\lambda_0 \int\limits_0^\infty b_0(y) \left[d_0(y,x) u^*_{n_1+2}(s;y) \tilde{u}_{n_1+1}(s;x)
+ {}\right.\\
\left.{}+\d_0(y,x) u^*_{n_1+2}(s;x) \tilde{u}_{n_1+1}(s;y)\right] \, dy+{}
\\
{}+
\lambda_0 \int\limits_0^\infty b_0(y) \left[d_0(y,x) u_{n_1+2}(s;y) u_{n_1+1}(s;x)
+ {}\right.\\
\left.{}+\d_0(y,x) u_{n_1+2}(s;x) u_{n_1+1}(s;y)\right] \, dy\,.
\end{multline*}
%%%%%%%%%
Соответственно изменится и формула~(\ref{5-2-4-m}).

Пусть в начальный момент в системе находится
$n$, $n\hm=\overline{n_1+1,n^*-1}$, заявок, система
функционирует в режиме~1, на приборе обслуживается
заявка длины~$y$ и в этот момент в систему поступает
заявка длины~$x$.
Обозначим через $\tilde{w}_n(s;x,y)$ ПЛС времени ожидания
начала обслуживания этой заявки.
Имеет место равенство:
\begin{multline*}
\tilde{w}_n(s;x,y) = d_1(x,y) + \d_1(x,y) \tilde{u}_{n+1}(s;y)\,,\\ 
n=\overline{n_1+1,n^*-1}\,.
\end{multline*}
Формула~(\ref{5-3-3-m}) %и (\ref{5-3-4})
принимает вид:
\begin{equation*}
%\label{5-3-3}
w^*_{n_0-1}(s;x,y) = d_0(x,y) + \d_0(x,y) \tilde{u}_{n_0}(s;y) \,,
\end{equation*}
%%%%%%%%%
а ПЛС стационарного распределения времени ожидания начала
обслуживания принятой в систему заявки определяется
формулой:
\begin{multline*}
%\label{5-3-4}
w(s) = \fr{1}{\lambda \left( 
1-\pi_{\mathrm{loss}}\right)}
\left[ \vphantom{\int\limits_0^\infty}
\lambda_0 p_0 + {}\right.\\
{}+\lambda_0 \int\limits_0^\infty \sum\limits_{n=1}^{n_1}
p_n(y) \, dy \int\limits_0^\infty b_0(x) w_n(s;x,y) \, dx +{}
\\
{}+
\lambda_0 \int\limits_0^\infty \sum\limits_{n=n_1+1}^{n_0-1} p_n(0;y) \, dy
\int\limits_0^\infty b_0(x) \left[w_n(s;x,y) +{}\right.\\
\left.{}+ w^*_n(s;x,y)
\right]
\, dx +{}\\
{}+ \lambda_1 \int\limits_0^\infty \sum\limits_{n=n_1+1}^{n_0-1}
p_n(1;y) \, dy \int\limits_0^\infty b_1(x) \tilde{w}_n(s;x,y) \, dx +{}\\
\left.{}+
\lambda_1 \int\limits_0^\infty \sum\limits_{n=n_0}^{n^*-1} p_n(y) \, dy
\int\limits_0^\infty b_1(x) \tilde{w}_n(s;x,y) \, dx\right]\,.
\end{multline*}


Обозначим через $\tilde{t}_n(s;x)$,\  $n\hm=\overline{n_1,n^*-1}$,
ПЛС времени от момента первого попадания заявки длины~$x$
на прибор до момента ухода ее из системы при условии,
что в момент первого попадания на прибор в очереди
было еще $n$~заявок и система функционировала в
режиме~1.
Тогда
%%%%%%%%%%%%%%%%%%%
$$
\tilde{t}_{n^*-1}(s;x) = e^{-sx}\,;
$$
%%%%%%%%%%%%%%%%

\vspace*{-12pt}

\noindent
\begin{multline*}
\tilde{t}_n(s;x) = \exp\left\{ \vphantom{\int\limits_0^x}
- (s + \lambda_1) x + {}\right.\\
{}+\lambda_1 \int\limits_0^x \,dz
\int\limits_0^\infty b_1(y) \left[\,\d_1(y,z) +{}\right.\\
\left.\left.{}+ d_1(y,z) \tilde{u}_{n+2}(s;y)\right] \, dy
\vphantom{\int\limits_0^\infty}
\right\}\,,\
n=\overline{n_1,n^*-2}\,.
\end{multline*}
При этом формулы~(\ref{5-4-1-m}) и~(\ref{5-4-2-m}) записываются
в виде:
\begin{multline*}
t_{n_0-2}(s;x)= e^{-(s + \lambda_0) x} \left( 
1 +
\lambda_0 \int\limits_0^x e^{(s + \lambda_0) z}\, dz\times{} \right.\\
{}\times
\int\limits_0^\infty b_0(y) \left[d_0(y,z) \tilde{u}_{n_0}(s;y) \tilde{t}_{n_0-2}(s;z)
+ {}\right.\\
\left.\left.{}+\d_0(y,z) \tilde{t}_{n_0-1}(s;z)\right] \, dy
\vphantom{\int\limits_0^x}
\right)\,;
\end{multline*}
%%%%%%%%%%%%%%%%%%%

\vspace*{-12pt}

\noindent
\begin{multline*}
%\label{5-4-2}
t_n(s;x) = {}\\
{}=e^{- \int\limits_0^x \left(
s + \lambda_0 - \lambda_0 \int\limits_0^\infty b_0(y) d_0(y,z) u_{n+2}(s;y) \, dy\right)dz}
\left(  \vphantom{\int\limits_0^x}
1 +{}\right.\\
{}+ \lambda_0 \int\limits_0^x e^{ \int\limits_0^v \left(
s + \lambda_0 - \lambda_0 \int\limits_0^\infty b_0(y) d_0(y,z) u_{n+2}(s;y) \, dy \right)dz}
dv \times{}\\
{}\times
\int\limits_0^\infty b_0(y) \left[d_0(y,v) u_{n+2}^*(s;y) \tilde{t}_n(s;v)
+{}\right.\\
\left.\left.{}+ \d_0(y,v) t_{n+1}(s;v)\right] \, dy 
\vphantom{\int\limits_0^x}
\right)\,,
\ \ n=\overline{n_1,n_0-3}\,.
\end{multline*}
%%%%%%%%%%%%%%%%%%%

Наконец, перейдем к общему времени пребывания заявки в
системе.
Обозначим через
$\tilde{v}_{n}(s;x,y)$, $n\hm=\overline{n_1+1,n^*-1}$,
ПЛС времени пребывания в сис\-те\-ме заявки длины~$x$ при
условии, что эта заявка застала в системе $n$~других
заявок, причем заявка на приборе имела длину~$y$, а
система пребывала в режиме~1.
Справедливо соотношение:
\begin{multline*}
\tilde{v}_{n}(s;x,y) = d_1(x,y) \tilde{t}_{n}(s;x) +{}\\[2pt]
{}+\d_1(x,y) \tilde{u}_{n+1}(s;x,y) \tilde{t}_{n-1}(s;x)\,,\\[2pt]
  n=\overline{n_1+1,n^*-1}\,.
\end{multline*}
%%%%%%%%%%%%%
Формулы~(\ref{5-5-1-m}) и~(\ref{5-5-2-m}) преобразуются
следующим образом:
\begin{multline*}
%\label{5-5-1}
v_{n_0-1}(s;x,y)= d_0(x,y) \tilde{t}_{n_0-1}(s;x) +{}\\
{}+
\d_0(x,y) \tilde{u}_{n_0}(s;x,y) \tilde{t}_{n_0-2}(s;x) \,;
\end{multline*}
%%%%%%%%%%%%%
\vspace*{-12pt}

\noindent
\begin{multline*}
%\label{5-5-2}
v_{n}(s;x,y) = d_0(x,y) t_{n}(s;x) +{}
\\
{}+
\d_0(x,y) \left[ u_{n+1}(s;x,y) t_{n-1}(s;x) +{}\right.\\
\left.{}+
u^*_{n+1}(s;x,y) \tilde{t}_{n-1}(s;x) \right] \,,
\ \ n=\overline{n_1+1,n_0-2}\,,
\end{multline*}
а ПЛС стационарного распределения общего времени пребывания
в системе принятой к обслуживанию заявки имеет вид:
\begin{multline*}
v(s)= \fr{1}{ \lambda \left(1-\pi_{\mathrm{loss}}\right)}
\left[
\lambda_0 p_0 \int\limits_0^\infty b_0(x) t_0(s;x) \, dx
+{}\right.\\
\left.{}+ \lambda_0 \int\limits_0^\infty \sum\limits_{n=1}^{n_1}
p_n(y) \, dy \int\limits_0^\infty b_0(x) v_n(s;x,y) \, dx +{}\right.
\\
{}+
\lambda_0 \int\limits_0^\infty \sum\limits_{n=n_1+1}^{n_0-1}
p_n(0;y)\, dy \int\limits_0^\infty b_0(x) v_n(s;x,y) \, dx +{}\\
{}+ \lambda_1 \int\limits_0^\infty \sum\limits_{n=n_1+1}^{n_0-1}
p_n(1;y)\, dy \int\limits_0^\infty b_1(x) \tilde{v}_n(s;x,y) \, dx+{}\\
\left.{}+ \lambda_1 \int\limits_0^\infty \sum\limits_{n=n_0}^{n^*-1} p_n(y)\, dy
\int\limits_0^\infty b_1(x) \tilde{v}_n(s;x,y)\, dx
\right]\,.
\end{multline*}

\section{Заключение}


В настоящей статье рассмотрена возможность\linebreak применения
аналитических методов для вы\-чис\-ле\-ния основных стационарных
характеристик функ\-ци\-о\-ни\-ро\-ва\-ния СМО, в которых
одновременно имеется два отличия от
классических СМО:\linebreak инверсионный порядок обслуживания с
вероятностным приоритетом и гистерезисная политика.
На примере однолинейной СМО с простейшим вариантом
гистерезисной политики показано, что полученные
вычислительные алгоритмы основаны на интегральных
и дифференциальных уравнениях, которые могут быть решены
на современной вы\-чис\-ли\-тель\-ной технике.
Приведены условия, при которых интегральные уравнения
могут быть сведены к системам линейных алгебраических
уравнений.

Полученные результаты могут служить основой для
продолжения работ в части математического моделирования
технических систем, ис\-поль\-зу\-ющих как элементы
<<нестандартных>> дисциплин обслуживания, так и сложные
варианты гистерезисного механизма предотвращения
различного рода перегрузок в ИТС.


{\small\frenchspacing
{%\baselineskip=10.8pt
\addcontentsline{toc}{section}{Литература}
\begin{thebibliography}{99}

\bibitem{1-m}
\Au{Печинкин А.\,В.} Об одной инвариантной системе массового
обслуживания~// Math.\ Operationsforsch.\ und Statist. Ser.\
Optimization, 1983. Vol.~14. No.\,3. S.~433--444.

\bibitem{2-m}
\Au{Печинкин А.\,В., Стальченко И.\,В.} Система $MAP/G/1/\infty$ с
инверсионным порядком обслуживания и вероятностным приоритетом,
функционирующая в дискретном времени~// Вестник Российского
ун-та дружбы народов. Сер.\ Математика. Информатика. Физика,
2010. №\,2. С.~26--36.

\bibitem{3-m}
\Au{Абаев П.\,О., Гайдамака Ю.\,В., Самуйлов~К.\,Е.} Гистерезисное
управление сигнальной нагрузкой в сети SIP-сер\-ве\-ров~// Вестник
Российского ун-та дружбы народов. Сер.\ Математика.
Информатика. Физика, 2011. №\,4. С.~54--71.

\bibitem{7-m}
\Au{Nishimura S., Jiang~Y.}
An $M/G/1$ vacation model with two service modes~//
Prob.\ Eng. Informational Sci., 1995. Vol.~9. No.\,3. P.~355--374.

\bibitem{8-m}
\Au{Dudin A.}
Optimal control for an $M^x/G/1$ queue with two operation
modes~// Prob. Eng.  Informational Sci.,
1997. Vol.~11. No.\,2. P.~255--265.

\bibitem{9-m}
\Au{Nobel R.\,D., Tijms H.\,C.} Optimal control for an $M^X/G/1$
queue with two service mo\-des~// Eur. J.~Operational
Res., 1999. Vol.~113. No.\,3. P.~610--619.

\bibitem{10-m}
\Au{Жерновый К.\,Ю., Жерновый Ю.\,В.} Система $M^\theta/G/1/m$ c
двухпороговой гистерезисной стратегией переключения интенсивности
обслуживания~// Информационные процессы, 2012. Т.~12. №\,2. С.~127--140.

\bibitem{22-m}
\Au{Bocharov~P.\,P., D'Apice~C., Pechinkin~A.\,V., Salerno~S.}
Queueing theory.~--- Ut\-recht, Boston: VSP, 2004.

\bibitem{5-m}
\Au{Нагоненко В.\,А.} О~характеристиках одной нестандартной сис\-те\-мы
массового обслуживания. I~// Изв.\ АН СССР. Технич.\ кибернет.,
1981. №\,1. С.~187--195.

\label{end\stat}

\bibitem{6-m}
\Au{Нагоненко В.\,А.} О~характеристиках одной нестандартной сис\-те\-мы
массового обслуживания. II~// Изв.\ АН СССР. Технич.\ кибернет.,
1981. №\,3. С.~91--99.
\end{thebibliography}
}
}

\end{multicols}  %3Abst+avt
\newcommand{\Cov}{\mathrm{Cov}}
\newcommand{\Nor}{\ensuremath{\mathcal{N}}}
\newcommand{\Pu}{\mathbb{P}}


\def\stat{luk-mor}

\def\tit{О СХОДИМОСТИ В ПРОСТРАНСТВЕ $L_p$ %{\boldmath{$L_p$}}  
МАКСИМУМА ПРОЦЕССА НАГРУЗКИ ДЛЯ ОДНОГО КЛАССА
ГАУССОВСКИХ~СИСТЕМ ОБСЛУЖИВАНИЯ$^*$}

\def\titkol{О сходимости в пространстве $L_p$  максимума процесса нагрузки для одного класса
гауссовских систем обслуживания}

\def\autkol{О.\,В.~Лукашенко, Е.\,В.~Морозов}

\def\aut{О.\,В.~Лукашенко$^1$, Е.\,В.~Морозов$^2$}

\titel{\tit}{\aut}{\autkol}{\titkol}

{\renewcommand{\thefootnote}{\fnsymbol{footnote}}\footnotetext[1]
{Работа выполняется при финансовой поддержке Программы стратегического 
развития ПетрГУ   в рамках реализации комплекса мероприятий  по развитию 
научно-исследовательской деятельности.}}

\renewcommand{\thefootnote}{\arabic{footnote}}
\footnotetext[1]{Институт прикладных математических исследований
Карельского научного центра Российской академии наук; Петрозаводский
государственный университет, lukashenko-oleg@mail.ru}
\footnotetext[2]{Институт прикладных математических исследований
Карельского научного центра Российской академии наук; Петрозаводский
государственный университет, emorozov@karelia.ru}

\Abst{Рассматривается класс систем обслуживания, на вход которых
поступает поток, содержащий линейную детерминированную компоненту и
случайную компоненту, описываемую центрированным гауссовским
процессом. Дисперсия входного процесса  является правильно
меняющейся функцией  с показателем  $V\hm\in (0,\,2)$. Найдены условия,
при которых  максимум  стационарного процесса нагрузки
(незавершенной работы) на интервале $[0,\,t]$ сходится при $t\hm\to
\infty$ (и при соответствующей нормировке) в пространстве $L_p$   к
явно выписанной константе. Также найдена асимптотика максимума
процесса нагрузки в нестационарном режиме. Получена асимптотика
минимального времени достижения процессом нагрузки растущего
значения~$b$.}


\KW{гауссовская система обслуживания; максимум
процесса нагрузки; дробное броуновское движение; асимптотический
анализ;  правильное изменение}

\vskip 14pt plus 9pt minus 6pt

      \thispagestyle{headings}

      \begin{multicols}{2}

            \label{st\stat}

\section{Введение}

В работе~\cite{Lukashenko} был осуществлен асимптотический анализ
процесса нагрузки в жидкостной  системе  с постоянной скоростью
обслуживания~$C$  и входным  гауссовским процессом, дисперсия
которого правильно меняется на бесконечности с показателем $V\hm\in
(0,\,2)$. Рассмотренный  класс входных процессов включает, в
частности, сумму независимых дробных броуновских движений (ДБД) с
соответствующими значениями индекса Херста. В~\cite{Lukashenko}
показано, что при соответствующей нормировке максимум  процесса нагрузки
  на интервале $[0,\,t]$ сходится по вероятности при  $t\hm\to \infty$  к  явно выписанной
константе.  Этот результат существенно обобщает результат
из работы~\cite{Zeevi}, в которой рассмотрена  жидкостная система с
единственным входным процессом ДБД.

В данной статье, которая  опирается на методы работы~\cite{Zeevi}, а
также на результаты  статьи~\cite{Lukashenko}, продолжен
асимптотический анализ описанной жидкостной системы. Основной
результат данной работы состоит в том, что при некоторых
дополнительных условиях на асимптотическое поведение дисперсии
входного гауссовского процесса доказанная  в~\cite{Lukashenko}
сходимость усилена до сходимости (к той же константе) в пространстве~$L_p$ 
при любом  $p\hm\ge 1$. Более того, при соответствующей
нормировке найдена асимптотика максимума процесса нагрузки в
нестационарном режиме. Кроме того, с использованием полученной
асимптотики максимума процесса нагрузки найдена асимптотика  времени
достижения процессом нагрузки растущего уровня~$b$.

Опишем рассматриваемую систему и используемые ниже результаты из~\cite{Lukashenko} 
более подробно.  Рассмотрим жидкостную систему с
одним обслуживающим устройством и постоянной скоростью обслуживания~$C$, 
на вход которой поступает процесс $A(t)$, заданный  в следующем виде:
\begin{equation}
A(t)=mt+X(t)\,, 
\label{asymp-l1}
\end{equation}
где  $m$~--- средняя интенсивность входного потока, а
$X:\hm=\{X(t)$, $t \hm\geq 0\}$~---  центрированный гауссовский процесс со
стационарными приращениями,  $X(0)\hm=0$. Такая система обслуживания
называется гауссовской~\cite{Mandjes}.  Будем считать, что выполнено
условие $r:=C\hm-m\hm>0$. Обозначим также $W(t)\hm=X(t)\hm-rt$, и пусть
 $Q(t)$  есть
величина нагрузки (незавершенная работа в системе) в момент времени~$t$. 
Будем предполагать, что  $Q(0)\hm=0$. Тогда имеет место соотношение~ \cite{Reich}:
\begin{equation}
Q(t)=\sup\limits_{0 \leq s \leq t}(W(t)-W(s))\,.\label{6a}
\end{equation}
Условие  $r\hm>0$ обеспечивает существование стационарного процесса
нагрузки,  который определяется следующим образом~\cite{Mandjes}:
\begin{equation}
Q= \sup\limits_{t \geq 0} W(t)\,.\label{6}
\end{equation}
 Для пояснения заметим, что величина $-r\hm<0$ есть средний снос
процесса~$W$, являющегося аналогом случайного блуждания (с
независимыми приращениями), максимум которого, по аналогии с~(\ref{6}), 
определяет стационарное время ожидания в классических системах обслуживания~\cite{Asmus}.

Основное предположение, принятое в работе~\cite{Lukashenko}, а также
в данной статье, состоит в том, что функция  $v(t)$ {\it правильно
меняется на бесконечности c индексом} $0\hm<V\hm<2$, т.\,е.\ представима в виде
\begin{equation}
v(t)=t^V L(t)\,,\label{4}
\end{equation}
где функция $L$  медленно меняется на бесконеч\-ности~\cite{Seneta}.
Обозначим 
$$
\beta=\fr{1}{2-V}\,,
$$ 
а также выберем и зафиксируем
любое $\varepsilon \hm\in (0,2-V)$. Будем  считать, что функция~$L$
является {\it дважды дифференцируемой} на $\mathbb{R}_+$ и выполнены
следующие условия (при $t \hm\to \infty$):
\begin{align}
L(tL^\beta(t)) &\sim L(t)\,;\label{10}\\
L''(t)&=o\left( \fr{1}{t^{V+\varepsilon}} \right)\,.\label{11a}
\end{align}
Нетрудно проверить, что из условия~\eqref{11a} следует сходимость
\begin{equation}
v''(t)\ln t \to 0\,,\enskip t \to \infty\,.\label{2.2.l20}
\end{equation}
Как показано в~\cite{Lukashenko},  условия~(\ref{4})--(\ref{11a}) на
самом деле выполнены для широкого класса гауссовских сис\-тем
обслуживания, включающего, например, сис\-те\-мы, на вход которых
поступает сумма нескольких независимых ДБД.
 В~работе~\cite{Konstantopoulos} показано, что на одном вероятностном пространстве можно
 задать процесс $W(t)\hm=X(t)\hm-rt$ и стационарный процесс $Q^*:=\{Q^*(t),\ t \hm\geq 0\}$
таким образом, что одновременно выполнены условия:
\begin{align*}
Q^*(t)&=_d Q \ \mbox{ для всех } \ t \geq 0\,; %\label{15}
\\
Q^*(t)&=W(t)+\max\left\{Q^*(0), L^*(t)\right\}\,,\,\,t \geq 0\,, %\label{16}
\end{align*}
где $=_d$ означает равенство по распределению, а
$L^*(t):=-\min\limits_{0\le s\le t}\{W(s)\}$. Обозначим
\begin{equation*}
M(t)=\max\limits_{0 \leq s \leq t}Q(s)\,,;\enskip M^*(t)=\max_{0 \leq s \leq
t}Q^*(s)\,.
%\label{13}
\end{equation*}
Для удобства обозначим далее
\begin{equation*}
\gamma(t)=L\left[\left(\ln t \right)^\beta\right]  \ln t \,.
\end{equation*}
Ниже будем опираться на результаты следующей теоремы, доказанной
в работе~\cite{Lukashenko}.

\medskip

\noindent
\textbf{Теорема~1.1.}\ \ 
\textit{Пусть дисперсия гауссовской компоненты $X$ входного
процесса}~(\ref{asymp-l1}) \textit{удовлетворяет условиям}~(\ref{10}) 
\textit{и}~(\ref{11a}), \textit{а также} $r\hm>0$. \textit{Тогда}
\begin{align}
\fr{M^*(t)}{\gamma^\beta(t)} &\Rightarrow
\left(\fr{1}{\theta}\right)^\beta\,,\enskip t \to \infty\,;
\label{asymp1-l8}
\\
\fr{M(t)}{\gamma^\beta(t)} &\Rightarrow
\left(\fr{1}{\theta}\right)^\beta\,,\enskip t \to \infty\,,
\label{asymp1-l9}
\end{align}
\textit{где} $\Rightarrow$ \textit{означает сходимость по вероятности, а параметр}~$\theta$ 
\textit{удовлетворяет соотношению}:
\begin{equation}
\theta=\fr{2}{(2-V)^{2-V}}\left( \fr{r}{V}
\right)^V\,.\label{logbuff-l6}
\end{equation}

\smallskip


Как отмечено выше, этот результат  обобщает работу~\cite{Zeevi}, где
процесс $X\hm=B_H$ является ДБД  c параметром Херста  $H\hm\in (1/2,\,1)$.

\section{Сходимость в пространстве $L_p$ %{\boldmath{$L_p$}}
}

В данном разделе будет доказан основной результат, состоящий в том,
что при  дополнительных условиях  на функцию~$L$ из~(\ref{4})
сходимость по вероятности в~(\ref{asymp1-l8}), (\ref{asymp1-l9})
можно усилить до сходимости в пространстве~$L_p$, где
$p\hm\in[1,\,\infty)$.


\medskip

\noindent
\textbf{Теорема~2.1.}\ \ 
\textit{Пусть дополнительно к условиям теоремы}~1.1
\begin{equation}
\liminf\limits_{t\to \infty} L(t)>0\,;\enskip
\limsup\limits_{t\to \infty} L(t)<\infty\,.
%\leq A_2.
\label{10a}
\end{equation}

\textit{Тогда в}~(\ref{asymp1-l8}), (\ref{asymp1-l9}) \textit{имеет место
сходимость в пространстве $L_p$, $p \hm\in [1,\infty)$}.


\smallskip


\noindent
Д\,о\,к\,а\,з\,а\,т\,е\,л\,ь\,с\,т\,в\,о\,.\ \ 
Зафиксируем $p \hm\in [1,\infty)$. Для доказательства теоремы
достаточно доказать равномерную интегрируемость семейства случайных величин
$$
\left\{\left(\fr{ M^*(t)}{\gamma^\beta(t)}\right)^p,\enskip t\ge
T\right\}\,,
$$
где $T$~--- некоторое (конечное) положительное чис\-ло. Для этого, в
свою очередь, достаточно, чтобы было выполнено (см., например,~\cite{Billingsley}) условие
\begin{equation}
\label{2.2.l1} 
\sup\limits_{t \geq T} \mathbb{E} \left[
\fr{M^*(t)}{\gamma^\beta(t)} \right]^{p+1} < \infty\,.
\end{equation}
 (Значение $\mathbb{E}(\cdot)$ при $t\hm<T$
может быть произвольным.) Выберем далее некоторое число $K\hm>0$.
Значения величин~$K$ и~$T$ будут уточняться в процессе
доказательства. Кроме того, всюду далее предполагается, что $t\hm\ge
T$. Имеют место  соотношения:
\begin{multline}
\mathbb{E} \left[ \fr{M^*(t)}{\gamma^\beta(t)}\right]^{p+1} = {}\\
{}=
\int\limits_{0}^\infty (p+1)y^p \Pu\left(M^*(t)>y \gamma^\beta(t)\right)dy={}\\
{}= \int\limits_{0}^K (p+1)y^p \Pu\left(M^*(t)>y \gamma^\beta(t)\right)dy+{}\\
{}+\int\limits_{K}^\infty (p+1)y^p \Pu\left(M^*(t)>y \gamma^\beta(t)\right)dy \leq{} \\
{}\leq K^{p+1} + (p+1)(I_t+R_t)\,,\label{2.2.l9}
\end{multline}
где использованы обозначения:
\begin{align*}
I_t&=\int\limits_{K}^\infty y^p \,\lceil t\rceil\Pu\left(Q^*(0)>\fr{y
\gamma^\beta(t)}{2}\right)dy\,;\\
 R_t&=\int\limits_{K}^\infty y^p\, \lceil t \rceil \times{}\\
& {}\times \Pu\left(
 \max\limits_{0\leq s \leq 1}W(s)-\min\limits_{0 \leq s \leq 1}W(s)>\fr{y
\gamma^\beta(t)}{2}\right)dy\,.
\end{align*}
Отметим, что при получении выражения~(\ref{2.2.l9})
использовано неравенство:
\begin{multline*}
\Pu(M^*(t)>x) \le {}\\
{}\le \lceil t\rceil\Pu\left(Q^*(0)+
\max\limits_{0 \leq s \leq 1}W(s)-\min\limits_{0 \leq s \leq 1} W(s)>x\right).
\end{multline*}
(Cм.\ доказательство теоремы~1.1 в~\cite{Lukashenko}.) Оценим вначале
 интеграл $I_t$.
 В~работе~\cite{Duffy} показано, что если дисперсия $v(t)$
 центрированного гауссовского процесса со стационарными
приращениями  правильно меняется на бесконечности  с индексом
$0\hm<V\hm<2$, то справедлива такая (логарифмическая) асимптотика:
\begin{equation}
\lim\limits_{b \to \infty} \fr{v(b)}{b^2} \ln \Pu(Q^*>b)=-\theta\,,
\label{asymp1-l13}
\end{equation}
где параметр $\theta$ удовлетворяет соотношению~\eqref{logbuff-l6}.
Определим  число $K_1$ следующим образом:
\begin{multline*}
K_1=\inf\left\{ y>0:\,\,\fr{L(x)\ln\Pu(Q^*(0)>x)}{x^{1/\beta}}\leq -
\fr{\theta}{2}\,,\right.\\ 
\left.\forall\ x>y\vphantom{\fr{L(x)Q^*}{x^{1/\beta}}}\right\}\,.
\end{multline*}
Напомним, что $\beta\hm={1}/({2-V})$. Поэтому ввиду~(\ref{4}) 
из~(\ref{asymp1-l13}) следует, что $K_1\hm<\infty$. Тогда при $x\hm>K_1$
справедливо неравенство:
\begin{equation}
\label{2.2.l2}
\Pu(Q^*(0)>x) \leq \exp\left( -\fr{\theta}{2}\, \fr{x^{1/\beta}}{L(x)} \right)\,.
\end{equation}
Заметим, что $\gamma^\beta(t) \to \infty$, $t \hm\to \infty$.
Следовательно, существует такое число $t_0$, что
$\gamma^\beta(t)/2\hm>1$ при $t \hm\geq t_0$. Поэтому, если $K\hm>K_1$, $t
\hm\geq t_0$, то из~(\ref{2.2.l2}) вытекает неравенство:
\begin{multline}
I_t = \int\limits_{K}^\infty y^p \lceil t\rceil\Pu\left(Q^*(0)>\fr{y 
\gamma^\beta(t)}{2}\right)dy \leq \int\limits_{K}^\infty y^p \lceil t\rceil\times{}\\
{}\times \exp\left[
-\fr{\theta}{2^{1/\beta+1}} \, \fr{\ln t \cdot L\left[ (\ln
t)^\beta\right]}{L(y\gamma^\beta(t)/2)}\, y^{1/\beta}
\right]dy\,.\label{18}
\end{multline}
Из условия~(\ref{10a}) следует, что существуют такие числа $0\hm<A_1\hm\le
A_2\hm<\infty $ и $K_2$, $t_1\hm\ge 0$, что при $y\hm>K_2$, $t\hm>\max(t_0,t_1)$
выполняются неравенства:
\begin{align}
\label{2.2.l5} 
L\left(\fr{y\gamma^\beta(t)}{2}\right) &\le A_2\,;
\\
\label{2.2.l6} L\left( (\ln t)^\beta \right)&\ge A_1\,.
\end{align}
Выберем теперь и временно зафиксируем в~(\ref{18})  некоторое
$K\hm>\max\{K_1,K_2\}$, и пусть  пока $T:=$\linebreak $:=\;\max(t_0,t_1)$. Обозначим также 
$$
\alpha=\fr{\theta A_1}{2^{1/\beta+1}A_2}\,.
$$
Последовательно применяя~(\ref{2.2.l5}), (\ref{2.2.l6}), а также
принимая во внимание, что $\lceil t\rceil \hm\leq 2t$, можно получить
из~(\ref{18})  (при $t\hm\geq T$) следующую оценку сверху интеграла
$I_t$:
\begin{multline}
I_t \leq 2 \int\limits_{K}^\infty y^p\, t\exp\left( - \alpha
\ln t \cdot y^{1/\beta}\right)dy={}\\
{}= 2\int\limits_{K}^\infty y^p\exp\left[ \ln t
\left(1-\alpha y^{1/\beta} \right)
\right]dy\,. \label{2.2.l12}
\end{multline}
Заметим, что при $y\hm>\left({2}/{\alpha}\right)^\beta :=K_3$
справедливо неравенство:
\begin{equation}
\label{2.2.l7}
1-\alpha y^{1/\beta} < -\fr{\alpha}{2}y^{1/\beta}\,.
\end{equation}
Если теперь выбрать в~(\ref{2.2.l12}) $K\hm>K_3$, то ввиду~(\ref{2.2.l7}) получим:
\begin{equation}
I_t \leq  2\int\limits_{K}^\infty y^p\exp\left( -\fr{\alpha}{2} \ln t
\cdot y^{1/\beta}\right)dy\,. \label{2.2.l13}
\end{equation}
Заметим, что $\ln t \hm\geq  \ln T$ при $t \hm\geq T$, и обозначим
$\gamma_1\hm=({\alpha}/{2})\ln T$. Тогда  из~(\ref{2.2.l13}) получим:
\begin{multline}
I_t \leq  2\int\limits_{K}^\infty y^p\exp\left( -\gamma_1 y^{1/\beta}\right)dy={}\\
{}= \fr{2\beta}{\gamma_1^{\beta+\beta p}}\int\limits_{\gamma_1
K^{1/\beta}}^\infty u^{\beta p+\beta-1}e^{-u}du={}\\
{}=\fr{2\beta}{\gamma_1^{\beta+\beta p}}\, \Gamma\left(\gamma_1
K^{1/\beta},\beta p+\beta \right)\,,\label{2.2.l10}
\end{multline}
где $\Gamma(w,z)$~--- неполная гам\-ма-функ\-ция:
$$
\Gamma(w,z):=\int\limits_w^\infty u^{z-1}e^{-u}\,du\,, \enskip w\ge 0\,,\ z \geq 0\,.
$$


Теперь оценим интеграл $R_t$ в~\eqref{2.2.l9}. Напомним соотношение
$W(t)=X(t)-rt$ и  заметим, что
\begin{equation}
\max\limits_{0 \leq s \leq 1}W(s)=\max\limits_{0 \leq s \leq 1}( X(s)-rs)
\leq \max\limits_{0 \leq s \leq 1} X(s)\,. \label{asymp1-l11}
\end{equation}
Кроме того,
\begin{multline}
-\min\limits_{0 \leq s \leq 1} W(s)=\max\limits_{0 \leq s \leq 1} (
rs-X(s))=_d{}\\
{}=_d\max\limits_{0 \leq s \leq 1} (rs+ X(s))
\leq r+ \max\limits_{0 \leq s \leq 1} X(s)\,.\label{2.2.l18}
\end{multline}
Неравенства~\eqref{asymp1-l11} и \eqref{2.2.l18} после несложных
преобразований приводят, в свою очередь, к неравенству:
\begin{multline}
\Pu\left(\max\limits_{0\leq s \leq 1}
W(s)-\min\limits_{0 \leq s \leq 1}W(s)>\fr{y \gamma^\beta(t)}{2}\right) \leq {}\\
{}\leq
2 \Pu\left(\max\limits_{0\leq s \leq 1}X(s)>
\fr{y\gamma^\beta(t)}{4}-r\right)\,. \label{2.2.l19}
\end{multline}
Применяя соотношения~\eqref{2.2.l19} и (\ref{2.2.l6}), можно получить
следующую цепочку неравенств:
\begin{multline}
R_t = \int\limits_{K}^\infty y^p \lceil t
\rceil\times{}\\
{}\times \Pu\left(\vphantom{\fr{\gamma^\beta}{2}}
\max_{0\leq s \leq 1}
W(s)-\min\limits_{0 \leq s \leq 1}W(s)>
\fr{y \gamma^\beta(t)}{2}\right)\,dy \leq{} \\
{}\leq 2 \int\limits_{K}^\infty y^p \lceil
t \rceil\Pu\left(\max\limits_{0\leq s \leq 1}X(s)>\fr{y\gamma^\beta(t)}{4}-r\right)dy \leq{} \\
\!\!{}\leq 4 \int\limits_{K}^\infty y^p  t
\Pu\left(\max\limits_{0\leq s \leq 1}X(s)>
\fr{y A_1^\beta(\ln t)^\beta}{4}-r\right)dy.\!\! \label{2.2.l11}
\end{multline}
Теперь   используем  следующее неравенство
Бо\-ре\-ля--Су\-да\-ко\-ва--Ци\-рель\-со\-на для максимума центрированного
гауссовского процесса со стационарными приращениями на конечном
интервале~\cite{Adler, Lifshits}:
\begin{equation}
\Pu \left( \max\limits_{0 \leq s \leq 1} X(s)> x \right) \leq
e^{-({1}/{(2v)})(x-a)^2}\,, \enskip x>a\,,
\label{2.2.l14}
\end{equation}
где $v:=\mathbb{D} X(1)$, $a:=\mathbb{E} \max\limits_{0 \leq s \leq 1} X(s)<\infty$.
Положим
$$
K_4=\inf\left\{y:\,\,\fr{x A_1^\beta(\ln T)^\beta}{4}-r>a\,,\ \forall\  x \geq y\right\}\,.
$$
Тогда при  $x\ge K_4$ неравенство~(\ref{2.2.l14}) выполнено для всех
$z:={x A_1^\beta(\ln T)^\beta}/{4}-r$,   причем
$ z\hm>a$.

  Введем обозначения:
$$
c_1=\fr{4(r+a)}{A_1^\beta}\,;\quad c_2=\fr{32 v }{A_1^{2\beta}}\,.
$$
Пусть теперь  $K\hm>K_4$ в~(\ref{2.2.l11}).  Тогда с учетом~(\ref{2.2.l14}) 
после несложных преобразований можно получить, что
\begin{equation}
R_t \leq 4 \int\limits_{K}^\infty y^p \exp\left[ \ln t -
\fr{\left( y(\ln t)^\beta-c_1\right)^2}{c_2} \right]dy\,.
\label{2.2.l15}
\end{equation}
Рассмотрим функцию
$$
f(t,y):=\ln t -\fr{\left(y(\ln t)^\beta-c_1\right)^2}{c_2}\,.
$$
 Нетрудно убедиться, что
\begin{equation}
\label{2.2.l3} 
\fr{\partial f(t,\,y)}{\partial t}=\fr{1}{t}\left[ 1-\fr{2\beta}{c_2}\left(y(\ln
t)^\beta-c_1\right)(\ln t)^{\beta-1} \right]
\end{equation}
и что при любом $y\hm>K$
$$
\fr{\partial
f}{\partial t}\left(t,y\right)<\fr{\partial f}{\partial t}(t,K)\,.
$$
Анализ правой части выражения (\ref{2.2.l3}) показывает, что
существует такое число $t_2$, что функция $f(t,y)$ (при каждом
$y\hm>K$) убывает (по~$t$) при $t\hm\geq t_2$. Это, в свою очередь, означает, что
\begin{equation}
\label{2.2.l4}
f(t,y) \leq f(t_2,y)\,,\enskip t\geq t_2\,,\enskip y>K\,.
\end{equation}
Обозначим $T\hm=\max\{t_0,\,t_1,\,t_2\}$. Заметим, что
$f(T,\,y) \hm\to -\infty$, $y \hm\to \infty$, и, как нетрудно проверить,
$$
\lim\limits_{y \to \infty} y^{p+2}e^{f(T,\,y)}=0\,.
$$
Поэтому найдется такое число $K_5\hm>0$, что
\begin{equation}
\label{2.2.l8} 
e^{f(T,\,y)}<y^{-p-2}\,,\enskip y>K_5\,.
\end{equation}
Теперь, используя соотношения~(\ref{2.2.l4}) и (\ref{2.2.l8}),
получим из~(\ref{2.2.l15})  при  $K\hm>K_5$ (и $t \hm\geq T$)
\begin{equation}
R_t  \leq 4 \int\limits_{K}^\infty y^p e^{f(T,y)}dy \leq 
4 \int\limits_{K}^\infty \fr{dy}{y^2}=\fr{4}{K}\,.
\label{2.2.l16}
\end{equation}
Выберем окончательно в~(\ref{2.2.l9}) (и далее, где требуется)
$K\hm=\max \left\{K_1,K_2,K_3,K_4,K_5\right\}$. Тогда из~(\ref{2.2.l10}) и~(\ref{2.2.l16}) 
следует неравенство
\begin{multline}
\mathbb{E} \left[ \fr{M^*(t)}{\gamma^\beta(t)}\right]^{p+1} \leq 
K^{p+1} + {}\\
{}+\fr{2\beta(p+1)}{\gamma_1^{\beta+\beta p}}\, \Gamma\left(
\gamma_1 K^{1/\beta},\beta p+\beta \right) + \fr{4(p+1)}{K}\,, 
\label{2.2.l17}
\end{multline}
 правая часть которого не зависит от~$t$. Беря  в левой части
неравенства~\eqref{2.2.l17}  супремум по  $t \hm\geq T$, получаем требуемое 
условие~(\ref{2.2.l1}). \hfill$\square$


\smallskip

\noindent
\textbf{Замечание.}\ Если существует предел
\begin{equation}
\lim\limits_{t\to \infty} L(t)= A\in (0,\,\infty)\,,
\label{38}
\end{equation}
то
условие~\eqref{10a} автоматически вы\-пол\-нено.

\smallskip

Приведем важные для практических применений примеры, когда условия
теоремы~2.1 вы\-пол\-нены.

\smallskip

\noindent
\textbf{Пример 1.}\ Пусть стохастическая компонента входного
процесса является суммой независимых ДБД, т.\,е.
\begin{equation*}
X(t)=\sum\limits_{i=1}^n B_{H_i}(t)\,, \enskip t\ge 0\,,
%\label{42}
\end{equation*}
где параметры Херста $H_i\hm\in (0,\,1)$. Подробная мотивировка такого
входного потока обсуждается  в работе~\cite{Taqqu}.  Без ограничения
общности будем считать, что $H_1\hm>\max\limits_{i>1}H_i$. Тогда
 дисперсия $v(t)$  процесса $\{X(t)\}$  имеет вид:
$$
v(t)=\sum\limits_{i=1}^n t^{2H_i}=t^{2H_1}L(t)\,,
$$
где медленно меняющаяся функция $ L(t)\hm=1\hm+\sum\limits_{i>1}
t^{2(H_i-H_1)}\hm\to 1$, $t \hm\to \infty. $ Таким образом, условия
теоремы~2.1 выполнены.

\smallskip

\noindent
\textbf{Пример 2.} Пусть стохастическая компонента входного процесса
есть так называемый интегральный гауссовский процесс, т.\,е.
\begin{equation}
X(t)=\int\limits_0^t Z(s)\,ds\,,
\label{41}
\end{equation}
где $Z$~--- центрированный стационарный гауссовский процесс с
ковариационной функцией $R(u):=$\linebreak 
$:=\;\Cov\left(Z(0),Z(u)\right)$. Входные
потоки такого типа рассматрива\-лись в работах~\cite{Debicki1,Kulkarni}. Нетрудно проверить, что для дисперсии
$v(t)$ процесса~$X$ справедливо соотношение:
\begin{equation}
v(t)=2 \int\limits_0^t\!\! \int\limits_0^s R(u)\,duds\,.\label{42a}
\end{equation}
Отсюда следует, что $v''(t)\hm=2 R(t)$, а значит условие~\eqref{2.2.l20} влечет сходимость
\begin{equation*}
R(t)\ln t \to 0\,,\enskip t \to \infty\,. 
%\label{2.2.l21}
\end{equation*}
Если дополнительно к условию~\eqref{2.2.l20} потребовать
существования таких  $A \hm\in (0,\infty)$, $V \hm\in (0,2)$, что
\begin{equation}
\fr{\int_0^t \int_0^s R(u)\,duds}{t^V} \to A\,,\enskip
t \to \infty\,,
\label{2.2.l22}
\end{equation}
то условия теоремы~2.1 оказываются выполненными. Например,
пусть  $Z$~--- процесс Орн\-штей\-на--Улен\-бе\-ка, который по определению
является центрированным стационарным гауссовским процессом.
Поскольку его  ковариационная функция  имеет вид $R(t)\hm=\lambda
e^{-\alpha t}$, $\lambda,\alpha\hm>0$, то из~(\ref{42a}) несложно получить, что
$$
v(t)=\fr{2 \lambda}{\alpha}t+\fr{2\lambda}{\alpha^2}\left(e^{-\alpha t}-1\right)\,.
$$
Следовательно, условие~\eqref{2.2.l22} выполнено для $V\hm=1$,
$A={\lambda}/{\alpha}$. Отметим, что  если   $Z$~--- процесс
Орн\-штей\-на--Улен\-бе\-ка,  то формула~(\ref{41}) определяет   {\it
интегральный процесс Орн\-штей\-на--Улен\-бе\-ка}.  Заметим, что этот
процесс является гауссовским аналогом (т.\,е.\ гауссовским процессом с
соответствующей корреляционной структурой) модели
Ани\-ка--Мит\-ра--Сон\-ди~\cite{Anick}, описывающей динамику некоторых
сетевых ресурсов (см.\ так\-же~\cite{Addie}).

На самом деле в обоих примерах выше  выполнено   более сильное, чем~\eqref{10a}, 
условие~(\ref{38}).
Однако утверждение теоремы~2.1  верно и в случае, когда функция~$L$ не имеет предела при 
$t\hm\to\infty$.

\section{Дополнительные асимптотические результаты}

В данном разделе получены два важных асимптотических результата для
максимума процесса нагрузки $M(t)$, дополняющие анализ, проведенный
в разд.~2. Вначале рассмотрим случай, когда параметр
$r\hm<0$. В~соответствии с замечанием, сделанным  после формулы~(\ref{6}), 
в этом случае система находится в   нестационарном режиме
и величина процесса нагрузки должна неограниченно воз\-рас\-тать.


\medskip

\noindent
\textbf{Теорема~3.1.}\ \textit{Если $r<0$, то имеет место следующая сходимость 
по распределению:}
\begin{equation}
\fr{M(t)+rt}{\sqrt{v(t)}}  \xrightarrow{d} \Nor
(0,1)\,, \enskip t \to \infty\,.
\label{max-l1}
\end{equation}

%\smallskip

\noindent
Д\,о\,к\,а\,з\,а\,т\,е\,л\,ь\,с\,т\,в\,о\,.\ \ Напомним, что 
процесс нагрузки в момент времени~$s$ определяется соотношением
$$
Q(s)=W(s)-\min\limits_{0\leq u \leq s}W(u)\,,
$$
где $W(u)= X(u)\hm-ru$.
Поскольку с вероятностью~1 (с~в.~1)
$$
\fr{X(t)}{t} \to 0\,,
$$
то также $W(t) \to +\infty$ с~в.~1. Пусть $\Psi:=\min\limits_{t \geq
0}{W(t)}$. Тогда существует случайный момент  $T_0\hm<\infty$  (с~в.~1)
такой, что $\min\limits_{t \geq 0}W(t)\hm=\min\limits_{0 \leq t \leq
T_0}W(t)$. Поэтому справедлива следующая цепочка соотношений:
\begin{multline*}
M(t)=\displaystyle\max\limits_{0 \leq s \leq t}\left[ W(s)-\min\limits_{0 \leq u \leq s}W(u)\right]\leq{}\\
{}\leq\displaystyle \max\limits_{0 \leq s \leq t} W(s)- \Psi={}\\
{}= \displaystyle\max\limits_{0 \leq s \leq t} W(s)- \min\limits_{0 \leq s \leq T_0} W(s)\,.
\end{multline*}
Отсюда следует  неравенство:
\begin{multline}
M(t)-W(t) \leq{}\\
{}\leq  \max\limits_{0 \leq s \leq t} W(s)-W(t)- 
\min\limits_{0 \leq s \leq T_0} W(s)\,. 
\label{max-l4}
\end{multline}
Поскольку
\begin{align*}
M(t)+rt&=M(t)+X(t)-W(t)\,;\\
\fr{X(t)}{\sqrt{v(t)}}&=_d \Nor(0,1)\,,
\end{align*}
то сходимость~(\ref{max-l1}) эквивалентна сходимости
\begin{equation}
\fr{M(t)-W(t)}{\sqrt{v(t)}}  \Rightarrow 0\,, \enskip t \to
\infty\,.
\label{max-l2}
\end{equation}
Докажем справедливость~\eqref{max-l2}.
Для этого достаточно показать, что для любого $\varepsilon\hm>0$ справедливо соотношение:
\begin{equation}
\Pu \left(\fr{M(t)-W(t)}{\sqrt{v(t)}}> \varepsilon  \right) \to 0\,,
\enskip t \to \infty\,. 
\label{max-l3}
\end{equation}
В силу~(\ref{max-l4})
\begin{multline*}
\Pu \left(\fr{M(t)-W(t)}{\sqrt{v(t)}}> \varepsilon  \right) \leq{}\\
{}\leq 
\Pu \left(\fr{\max\limits_{0 \leq s \leq t}W(s)-W(t)}{\sqrt{v(t)}}> 
\fr{\varepsilon}{2}  \right)+{}
\\{} + \Pu \left(\fr{-\min\limits_{0 \leq s \leq T_0}W(s)}{\sqrt{v(t)}}> 
\fr{\varepsilon}{2}  \right):=\Pu_1(t)+\Pu_2(t)\,.
\end{multline*}
 В силу стацонарости приращений процесса $X$
\begin{multline}
\max\limits_{0 \leq s \leq t} W(s) - W(t)= {}\\
{}=
\max\limits_{0 \leq s \leq t} \left[  X(s)- X(t)+r(t-s)\right]=_d\\
{}=_d \max\limits_{0 \leq s \leq t}\left[  X(t-s)+r(t-s)\right]=\\
= \max\limits_{0 \leq u \leq t}\left[
X(u)+ru\right]:=\widetilde{Q}(t)\,.
\label{48}
\end{multline}
Поскольку  $r\hm<0$, то существует стационарный предел
$\widetilde{Q}(t) \xrightarrow{d} \widetilde {Q}$ (при $t \hm\to \infty$),
причем  $ \widetilde{Q}\hm<\infty$ с в.~1 (см.\ замечание после формулы~\eqref{6a}). 
Поскольку $v(t) \hm\to \infty$, то из~(\ref{48}) следует,
что
\begin{multline*}
\Pu_1(t)=\Pu\left( \max\limits_{0 \leq s \leq t}W(s)-W(t)>
\fr{\varepsilon}{2}\sqrt{v(t)}  \right) \to 0\,,\\
 t \to \infty\,.
\end{multline*}
Рассмотрим  вероятность $\Pu_2(t)$ и заметим, что
\begin{multline*}
-\min\limits_{0 \leq s \leq T_0} W(s)=\max\limits_{0 \leq s \leq T_0 }[rs- X(s)]\leq{}\\
{}\leq \max\limits_{0 \leq s \leq T_0}[- X(s)]=_d  \max\limits_{0 \leq s \leq T_0}X(s)\,.
\end{multline*}
Поэтому
\begin{multline*}
\Pu_2(t) = \Pu\left( -\min\limits_{0 \leq s \leq T_0}W(s)>
\fr{\varepsilon}{2}\sqrt{v(t)} \right)\leq{}\\
{}\leq \Pu\left( \max\limits_{0 \leq s \leq T_0}X(s)>
\fr{\varepsilon}{2}\sqrt{v(t)}\right) \to 0\,,\enskip t \to \infty\,,
\end{multline*}
где учитывается, что  случайная величина  $T_0$, а значит и $\max\limits_{0\leq s \leq T_0}X(s)$, 
конечны с~в.~1. Таким образом, соотношение~(\ref{max-l3}), а значит и~(\ref{max-l2}), выполнено.\hfill$\square$

\smallskip

Следующий результат касается асимптотики времени достижения
стационарным процессом нагрузки $Q^*(t)$ растущего порога~$b$, т.\,е.\
асимптотики величины
$$
T(b)=\inf\{t \geq 0:\,Q^*(t)\geq b\}
$$
при $b\to \infty$.  Распределение максимума стационарного процесса
нагрузки $M^*(t)$ определяет распределение случайной величины $T(b)$
в силу  очевидного соотношения
\begin{equation}
\{ T(b) \leq t\}=\{M^*(t) \geq b \}\,,\enskip t\ge 0\,.
\label{time-l0}
\end{equation}
Напомним обозначение $\gamma(t)\hm=\ln t \cdot L((\ln t)^\beta).$

\medskip

\noindent
\textbf{Теорема~3.2.}\ \textit{Пусть в дополнение  к условиям теоремы}~1.1 \textit{функция
$\gamma(t)$ монотонно возрастает на некотором луче $[t_0,\infty)$.
Тогда имеет место сходимость}
\begin{equation}
\fr{\gamma(T(b))}{b^{1/\beta}} \Rightarrow \theta\,,\enskip b \to
\infty\,, 
\label{time-l1}
\end{equation}
\textit{где параметр $\theta$ удовлетворяет соотношению}~(\ref{logbuff-l6}).

\smallskip

\noindent
Д\,о\,к\,а\,з\,а\,т\,е\,л\,ь\,с\,т\,в\,о\,.\ 
 В~силу теоремы~1.1 для любого $\delta \hm>0$
\begin{equation}
\Pu\left( M^*(t)
> \left( \fr{1+\delta}{\theta}\,\gamma(t)\right)^\beta \right) \to 0\,,
\enskip t \to \infty\,.
\label{time-l2}
\end{equation}
Ввиду монотонного возрастания функции~$\gamma$, обратная ей функция
также монотонно возрастает. В~частности, при любом $\delta\hm>0$ функция
\begin{equation}
t(b):=\gamma^{-1}\left( b^{1/\beta} \fr{\theta}{1+\delta} \right) \to \infty\,,\enskip
b \to \infty\,.
\label{time-l3}
\end{equation}
Подставляя~(\ref{time-l3}) в~(\ref{time-l2}) и учитывая~(\ref{time-l0}), получим
\begin{multline}
\Pu\left( M^*(t(b)) \ge \left(
\fr{1+\delta}{\theta}\,\gamma(t(b))\right)^\beta \right)
={}\\
{}=\Pu\left(
M^*(t(b)) \ge  b\right)=\Pu\left( T(b) \le  t(b)\right)={}\\
{}=\Pu\left( T(b)\leq \gamma^{-1}\left( b^{1/\beta}
\fr{\theta}{1+\delta}
\right)\right)={}\\
{}=\Pu\left( \fr{\gamma(T(b))}{b^{1/\beta }}
 \leq \fr{\theta}{1+\delta} \right) \to 0\,,\enskip b \to  \infty\,.
 \label{50}
\end{multline}
Снова используя монотонность функции~$\gamma$, получим (для любого
фиксированного $\delta\hm>0$):
$$
\hat t(b):=\gamma^{-1}\left( b^{1/\beta} \fr{\theta}{1-\delta}
\right) \to \infty\,,\enskip b \to \infty\,.
$$
С учетом того, что по теореме~1.1
$$
\Pu\left( M^*(t) > \left(
\fr{1-\delta}{\theta}\,\gamma(t)\right)^\beta \right) \to 1\,,\enskip 
t \to \infty\,,
$$
как и выше, получим:
\begin{multline*}
\Pu\left( M^*(\hat t(b)) > \left(
\fr{1-\delta}{\theta}\,\gamma(\hat t(b))\right)^\beta
\right)={}\\
{}=\Pu\left( \fr{\gamma(T(b))}{b^{1/\beta }} \leq
\fr{\theta}{1-\delta} \right) \to 1\,,\ b \to \infty\,.
\end{multline*}
Ввиду произвольности~$\delta$ отсюда и из~(\ref{50}) следует~\eqref{time-l1}.\hfill$\square$

\section{Заключение}

В данной статье  продолжен (начатый в работе~\cite{Lukashenko})
асимптотический анализ максимума процесса нагрузки в системе
обслуживания, в которой дисперсия гауссовской компоненты входного
процесса  правильно меняется  на бесконечности с показателем
$V\hm\in(0,\,2)$.
  В~частности, показано, что при некотором дополнительном  условии
 доказанная в~\cite{Lukashenko} сходимость по вероятности
 указанного процесса  имеет место и
в пространстве~$L_p$  при любом $p\hm\in [1,\,\infty)$. 

Также найдена
асимптотика максимума процесса нагрузки в нестационарном режиме (при
соответствующей нормировке). 
Кроме того, с использованием полученной
асимптотики максимума  найдена асимптотика  времени достижения
стационарным процессом нагрузки растущего   порога~$b$.

{\small\frenchspacing
{%\baselineskip=10.8pt
\addcontentsline{toc}{section}{Литература}
\begin{thebibliography}{99}

\bibitem{Lukashenko}
\Au{Лукашенко~О.\,В., Морозов~Е.\,В.} Асимптотика максимума
процесса нагрузки для некоторого класса гауссовских очередей~//
Информатика и её применения, 2012. Т.~6. Вып.~3. С.~81--89.

\bibitem{Zeevi}
\Au{Zeevi~A., Glynn~P.} On the maximum workload in a queue fed
by fractional Brownian motion~// Ann. Appl. Prob., 2000. Vol.~10.
P.~1084--1099.

\bibitem{Mandjes}
\Au{Mandjes~M.} Large deviations of Gaussian queues.~---
Chichester: Wiley, 2007. 339~p.



\bibitem{Reich}
\Au{Reich~E.} On the integrodifferential equation of Takacs~I~// 
Ann. Math. Stat., 1958. Vol.~29. P.~563--570.

\bibitem{Asmus}
\Au{Asmussen S.}  Applied probability and queues.~--- New York: Springer, 2002. 440~p.

\bibitem{Seneta}
\Au{Сенета~Е.} Правильно меняющиеся функции.~--- М.: Наука, 1985.
143~с.

\bibitem{Konstantopoulos}
\Au{Konstantopoulos~T., Zazanis~M., De Veciana~G.}
Conservation laws and reflection mappings with application to
multiclass mean value analysis for stochastic fluid queues~// 
Stochastic Processes and Their Applications, 1996. Vol.~65.
P.~139--146.

\bibitem{Billingsley}
\Au{Биллингсли~П.} Сходимость вероятностных мер.~--- М.: Наука, 1977. 352~с.

\bibitem{Duffy}
\Au{Duffy~K., Lewis~J.~T., Sullivan~W.~G.} Logarithmic
asymptotics for the supremum of a stochastic process~// 
Ann. Appl. Prob., 2003. Vol.~13. No.\,2. P.~430--445.

\bibitem{Adler}
\Au{Adler~R.\,J.} An introduction to continuity, extrema, and
related topics for general Gaussian processes.~--- Hayward, CA: Institute of 
Mathematical Statistics, 1990. 170~p.

\bibitem{Lifshits}
\Au{Лифшиц~М.\,А.} Гауссовские случайные функции.~--- Киев: ТвиМС, 1995. 248~с.


\bibitem{Taqqu}
\Au{Taqqu~M.~S., Willinger~W., Sherman~R.} Proof of a
fundamental result in self-similar traffic modeling~// Computer
Comm. Rev., 1997. Vol.~27. P.~5--23.


\bibitem{Kulkarni}
\Au{Kulkarni~V., Rolski~T.} Fluid model driven by an
Ornstein--Uhlenbeck process~// Probability  Engineering 
Informational Sci., 1994. Vol.~8. P.~403--417.

\bibitem{Debicki1}
\Au{Debicki~K., Rolski~T.} A Gaussian fluid model~// Queueing
Syst., 1995. Vol.~20. P.~433--452.

\bibitem{Anick}
\Au{Anick~D., Mitra~D., Sondhi~M.~M.} Stochastic theory of a
data handling system with multiple resources~// Bell Syst.
Techn.~J., 1982. Vol.~61. P.~1871--1894.

\label{end\stat}

\bibitem{Addie}
\Au{Addie~R., Mannersalo~P., Norros~I.} Most probable paths and
performance formulae for buffers with Gaussian input traffic~//
Eur. Trans. Telecommunications, 2002. Vol.~13.
P.~183--196.
\end{thebibliography}
}
}

\end{multicols}      %4Abst+avt
\def\stat{rudoi}

\def\tit{АЛГОРИТМЫ ИНДУКТИВНОГО ПОРОЖДЕНИЯ СУПЕРПОЗИЦИЙ ДЛЯ~АППРОКСИМАЦИИ ИЗМЕРЯЕМЫХ ДАННЫХ$^*$}

\def\titkol{Алгоритмы индуктивного порождения суперпозиций для аппроксимации измеряемых данных}

\def\autkol{Г.\,И.~Рудой, В.\,В.~Стрижов}

\def\aut{Г.\,И.~Рудой$^1$, В.\,В.~Стрижов$^2$}

\titel{\tit}{\aut}{\autkol}{\titkol}

{\renewcommand{\thefootnote}{\fnsymbol{footnote}}\footnotetext[1]
{Работа выполнена при поддержке РФФИ, грант №\,12-07-13118.}}

\renewcommand{\thefootnote}{\arabic{footnote}}
\footnotetext[1]{Московский физико-технический институт, rudoy@forecsys.ru}
\footnotetext[2]{Вычислительный центр Российской академии наук 
им.~А.\,А. Дородницына, strijov@ccas.ru}

\vspace*{-3pt}

\Abst{Исследуется алгоритм индуктивного порождения допустимых существенно
  нелинейных моделей. Предлагается алгоритм, порождающий все возможные
  суперпозиции заданной сложности за конечное число шагов. Приводятся
  результаты вычислительного
  эксперимента по выбору оптимальной модели, аппроксимирующей синтетический
  набор данных.}
  
  \vspace*{-1pt}

\KW{символьная регрессия; нелинейные модели; индуктивное порождение;
    сложность моделей}
    
    \vspace*{-3pt}
    
    \vskip 12pt plus 9pt minus 6pt

      \thispagestyle{headings}

      \begin{multicols}{2}

            \label{st\stat}

\section{Введение}

В~ряде приложений~\cite{duffy:1999:srised, Barmpalexis201175}
возникает задача восстановления некоторой функциональной зависимости
по набору известных данных. При этом предполагается, что эксперт
должен иметь возможность проинтерпретировать полученную модель
в контексте предметной области.

Одним из методов, позволяющих получать интерпретируемые модели, является
символьная регрессия~[3--7],
согласно которой известные данные приближаются некоторой математической
формулой, например $\sin x^2 \hm+ 2x $ или $\log x \hm- {e^x}/{x} $.
Эти формулы являются произвольными суперпозициями функций из некоторого
заданного набора. Одна из возможных реализаций описываемого метода
предложена Джоном Коза~\cite{Koza1998GP, Koza1998Intro}, использовавшим
эволюционные алгоритмы для реализации символьной регрессии. Зелинка с соавторами
предложили дальнейшее развитие этой идеи~\cite{Zelinka2008}, получившее
название аналитического программирования.

Алгоритм построения требуемой математической модели в аналитическом
программировании выглядит следующим образом:
дан набор примитивных функций, из которых можно строить различные формулы
(например, степенная функция, $+$, $\sin$, $\tan$). Начальный набор формул
строится либо произвольным образом, либо на базе некоторых предположений
эксперта. Затем на каждом шаге производится оценка каждой из формул согласно
функции ошибки либо другого функционала качества~\cite{Tirsin2005}. На базе
этой оценки у некоторой части формул случайным образом заменяется одна
элементарная функция на другую (например, $\sin$ на $\cos$ или $+$ на
$\times$), а у некоторой другой части происходит взаимный попарный обмен
подвыражениями.

Получаемая формула является математической моделью исследуемого
процесса или явления, т.\,е.\ представляет собой математическое
отношение, описывающее основные закономерности, присущие этому
явлению~\cite{Pavlovsky2000}.

Алгоритм индуктивного порождения моделей, предложенный в~настоящей работе,
свободен от некоторых типичных проблем известных методов, упомянутых,
например, в~\cite{Zelinka2008}. Вот главные из них:
\begin{itemize}
  \item порождение рекурсивных суперпозиций, суперпозиций, содержащих
    несоответствующее используемым функциям число аргументов, и~т.\,д.
    (в~предложенном алгоритме эти проб\-ле\-мы не возникают по построению);
  \item несовпадение области определения некоторой примитивной функции и области
    значений ее аргументов (возможно, тоже некоторых суперпозиций);
  \item порождение слишком сложных суперпозиций.
\end{itemize}

Для любой выборки можно построить такой многочлен, который пройдет через
все точки выборки, но при этом число параметров такого многочлена линейно
растет с объемом выборки. Кроме того, такой многочлен неинтерпретируем
экспертами. Предложенный в настоящей работе алгоритм решает проблему
порождения слишком сложных суперпозиций введением дополнительного штрафа
за сложность. Кроме того, так как ис\-поль\-зу\-емые признаки объектов выборки
учитываются при расчете сложности, применение подобного штрафа обеспечивает
выбор суперпозиций, использующих меньшее число признаков, т.\,е.\ проводит
отбор признаков.

Во~второй части работы формально поставлена задача построения алгоритма
индуктивного по\-рож\-де\-ния моделей. Затем, в~третьей час\-ти, строится искомый
алгоритм для частного случая беспараметрических моделей и~доказывается его
корректность, а затем алгоритм обобщается на случай моделей, имеющих параметры.
В~четвертой час\-ти оценивается количество порожденных предложенным алгоритмом
моделей на каждой итерации. В~пятой час\-ти предлагается метод выбора
допустимых моделей из множества всех порожденных моделей. В~седьмой час\-ти
описывается адаптированный стохастический алгоритм порождения моделей,
результаты работы которого на синтетических данных приведены в~восьмой
части настоящей работы.

\section{Постановка задачи}

Пусть дана выборка
\begin{multline*}
D = \left\{ (\mathbf{x}_i, y_i) \mid i \in \{1, \dots, N\},\right.\\
           \left. \mathbf{x}_i \in \mathbb{X} \subset \mathbb{R}^n,
            y_i \in \mathbb{Y} \subset \mathbb{R} \right\},
\end{multline*}
где $N$~--- число элементов выборки, $\mathbf{x}_i$~--- вектор значений
свободных переменных для $i$-го элемента выборки, $y_i$~--- значение зависимой
переменной для $i$-го элемента выборки,
$\mathbb{X}$~--- множество значений независимых переменных, лежащее в
$\mathbb{R}^n$, $\mathbb{Y}$~--- множество значений зависимой переменной.

Требуется выбрать параметрическую функцию
$f : \Omega \times \mathbb{X} \rightarrow \mathbb{R}$ из
порождаемого множества $\mathcal{F} \hm= \{ f_r \}$, где $\Omega$~--- пространство
параметров, до\-став\-ля\-ющую минимум некоторому заданному функционалу качества~$Q$,
зависящему от функционала ошибки~$S$ на данной выборке~$D$ и сложности суперпозиции~$C(f)$.

Таким образом, для множества всех суперпозиций
$$
\mathcal{F} = \{ f_r \mid
            f_r : (\boldsymbol{\omega}, \mathbf{x}) \mapsto y \in \mathbb{Y},
            r \in \mathbb{N} \}
$$
требуется найти такой индекс $\hat{r}$, при котором функция $f_r$ среди всех
$f \hm\in \mathcal{F}$ доставляет минимум функционалу качества~$Q$ при данной
выборке~$D$:
\begin{equation*}
  \label{eq:hat_r}
  \hat{r} = \arg \min\limits_{r \in \mathbb{N}} Q (f_r \mid \boldsymbol{\hat{\omega}_r}, D)\,,
\end{equation*}
где $\boldsymbol{\hat{\omega}}_r$~--- оптимальный вектор параметров функции
$f_r$ для каждой $f \hm\in \mathcal{F}$ при данной выборке~$D$:
\begin{equation*}
%  \label{eq:hat_omega}
  \boldsymbol{\hat{\omega}_r} = 
  \arg \min\limits_{\boldsymbol{\omega} \in \Omega} S(\boldsymbol{\omega} \mid f_r, D)\,.
\end{equation*}

Сформулируем также постановку теоретической задачи. Для этого сначала
введем понятие суперпозиции функций.

Если множество значений $\mathbb{Y}_i$ функции $f_i$ содержится в области
определения $\mathbb{X}_{i+1}$ функции $f_{i+1}$, т.\,е.\
$$
f_i : \mathbb{X}_i \to \mathbb{Y}_i \subset \mathbb{X}_{i+1}\,,\enskip i = 1, 2, \dots, \theta - 1\,,
$$
то функция
$$
f_\theta \circ f_{\theta-1} \circ \dots \circ f_1\,, \enskip \theta \geq 2\,,
$$
определяемая равенством
$$
(f_\theta \circ f_{\theta-1} \circ \dots \circ f_1) (\mathbf{x}) =
  f_{\theta} (f_{\theta-1} (\cdots (f_1 (\mathbf{x}))))\,, 
  \ x \in \mathbb{X}_1,
$$
называется \textit{сложной функцией}~\cite{MathEnc1984_4} или
\textit{суперпозицией функций} $f_1, f_2, \dots, f_\theta$.

Таким образом, получаем

\smallskip

\noindent
\textbf{Определение~1.}
\textit{Суперпозиция функций~--- функция, представленная как композиция нескольких
  функций.}
  
  \smallskip


Пусть $G = \{ g_1, \dots, g_l \}$~--- множество данных порождающих
функций, а именно: для каждой $g_i \hm\in G$ заданы
\begin{itemize}
  \item сама функция $g_i$ (например, $\sin$, $\cos$, $\times$);
  \item арность функции и~порядок следования аргументов;
  \item домен ($\text{dom}\, g_i$) и кодомен ($\text{cod}\, g_i$) функции;
  \item область определения $\mathcal{D} g_i \subset \text{dom}\, g_i$ и~область
    значений $\mathcal{E} g_i \subset \text{cod}\, g_i$.
\end{itemize}
Требуется построить упомянутую функцию~$f$ как суперпозицию порождающих
функций из заданного множества~$G$.

Поясним различие между последними двумя пунктами. Например, $\text{dom}\, f$
показывает, значения из какого множества принимает функция~$f$ (целые чис\-ла,
действительные чис\-ла, декартово произведение целых чи\-сел и $\{0, 1\}$,
и~т.\,п.). Область определения же показывает, на каких значениях из
$\text{dom}\, f$ функция $f$ определена и имеет смысл. Так, для функции
$f(x_1, x_2) \hm= \log_{x_1} x_2$:
\begin{gather*}
  \text{dom}\, f = \mathbb{R} \times \mathbb{R}\,,\quad 
  \text{cod}\, f = \mathbb{R}\,;
\\
  \mathcal{D} f = \left\{ (x_1, x_2) \vert x_1 \in (0; 1) \cup (1; +\infty), x_2 \in (0; +\infty) \right\};
\\
  \mathcal{E} f = (-\infty; +\infty)\,.
\end{gather*}

Требуется также:
\begin{itemize}
  \item построить алгоритм $\mathfrak{A}$, за конечное число итераций
    порождающий любую конечную суперпозицию данных примитивных функций;
  \item указать способ проверки изоморфности двух суперпозиций.
\end{itemize}

Заметим, что для примитивных функций  не требуются свойства их непорождаемости
в наиболее общей формулировке типа принципиальной невозможности породить
в ходе работы искомого алгоритма суперпозицию, изоморфную некоторой функции из~$G$. 
Такое требование является слишком ограничивающим. В~частности, невозможно
было бы иметь в~$G$ одновременно, например, функции $\text{id}$, $\exp$
и~$\log$, так как $\text{id}\,\equiv \log \circ \exp$.

В~дальнейшем будем также считать, что суперпозиция, соответствующая
единственной свободной переменной ($f(\mathbf{x}) \hm= x_i$), эквивалентна
функции вида $\text{id}\,x_i$.

\section{Алгоритм индуктивного порождения допустимых суперпозиций}

Условимся считать, что каждой суперпозиции~$f$ сопоставлено дерево~$\Gamma_f$,
эквивалентное этой суперпозиции и строящееся следующим образом:
\begin{itemize}
  \item в~вершинах $V_i$ дерева~$\Gamma_f$ находятся соответствующие
    порождающие функции $g_s, s \hm= s(i)$;
  \item число дочерних вершин у некоторой вершины~$V_i$ равно арности
    соответствующей функции~$g_s$;
  \item порядок смежных некоторой вершине~$V_i$ вершин соответствует порядку
    аргументов соответствующей функции~$g_{s(i)}$;
  \item в~листьях дерева~$\Gamma_f$ находятся свободные переменные~$x_i$
    либо числовые па\-ра\-мет\-ры~$\omega_i$;
  \item порядок вершин~$V_i$ в~смысле уровня вершин определяет порядок
    вычисления примитивных функций: дерево вычисляется снизу вверх,
    т.\,е.\ сначала подставляются конкретные значения свободных переменных,
    затем вычисляются значения в~вершинах, все дочерние вершины которых~---
    свободные переменные, и так далее до тех пор, пока не останется
    единственная вершина, бывшая корнем дерева. Она и содержит результат
    соответствующего выражения.
\end{itemize}

Таким образом, вычисление значения вы\-ра\-жения~$f$ в некоторой точке с данным
вектором\linebreak параметров $\boldsymbol{\omega} \hm= \{ \omega_1, \omega_2, \dots, \omega_\eta\}$
эквивалентно подстановке соответствующих значений свободных перемен\-ных~$x_i$
и параметров $\omega_i$ в дерево~$\Gamma_f$, где $x_i$ --- компоненты
вектора признакового описания объекта~$\mathbf{x}$.

Заметим важное свойство таких деревьев: каж\-дое поддерево $\Gamma_f^i$
дерева $\Gamma_f$, соответствующее вершине~$V_i$, также соответствует
некоторой суперпози-\linebreak\vspace*{-12pt}
\begin{center}  %fig1
\vspace*{1pt}
\mbox{%
 \epsfxsize=48.539mm
 \epsfbox{rud-1.eps}
 }
% \end{center}

 \vspace*{6pt}
{{\figurename~1}\ \ \small{Дерево выражения $\sin (\ln x_1) + {x_2^3}/{2}$}}
\end{center}


%\pagebreak

\vspace*{12pt}

\addtocounter{figure}{1}

\noindent
ции, являющейся составляющей исходной суперпозиции~$f$.

Для примера рассмотрим дерево, со\-от\-вет\-ст\-ву\-ющее суперпозиции $f \hm= \sin (\ln x_1) + 
{x_2^3}/{2}$ (рис.~1).
Здесь точками обозначены аргументы функций. Как видно, корнем дерева является
вершина, соответствующая операции сложения, которая должна быть выполнена
в последнюю очередь. Операция сложения имеет два различных поддерева,
соответствующих двум аргументам этой операции. Заметим также, что здесь не
использованы операции типа <<разделить на два>> или <<возвести в~куб>>.
Вместо этого используются операции деления и возведения в степень в~общем
виде, а в данном конкретном дереве соответствующие аргументы зафиксированы
соответствующими константами.

\smallskip


\noindent
\textbf{Алгоритм порождения суперпозиций.} Сначала определим понятие
\textit{глубины суперпозиции}:

\smallskip

\noindent
\textbf{Определение~2.}
\textit{Глубина суперпозиции $f$~--- максимальная глубина дерева $\Gamma_f$.}

\smallskip

Теперь опишем итеративный алгоритм $\mathfrak{A^*}$, порождающий суперпозиции,
не содержащие параметров. Описанный алгоритм породит любую суперпозицию
конечной глубины за конечное число шагов.

Пусть дано множество примитивных функций $G \hm= \{ g_1, \dots, g_l \}$ и
множество свободных переменных $X \hm= \{ x_1, \dots, x_n \}$. Для удобства будем
исходить из предположения, что множество $G$ состоит только из унарных
и~бинарных функций, и~разделим его соответствующим образом на два подмножества:
$G \hm= G_b \cup G_u \mid G_b \hm= \{ g_{b_1}, \dots, g_{b_k} \}$, 
$G_u \hm= \{ g_{u_1}, \dots, g_{u_l} \}$,
где $G_b$~--- множество всех бинарных функций, а $G_u$~--- множество всех
унарных функций из~$G$. Потребуем также наличия $\text{id}$ в~$G_b$.

\smallskip

\noindent
\textbf{Алгоритм~1.}
  Алгоритм $\mathfrak{A^*}$ итеративного порождения суперпозиций.
\begin{enumerate}[1.]
  \item Перед первым шагом зададим начальные значения множества
    $\mathcal{F}_0$ и вспомогательного индексного множества~$\mathcal{I}$,
    служащего для запоминания, на какой итерации впервые встречена
    каждая суперпозиция:
    \begin{equation*}
      \mathcal{F}_0 = X\,;\quad
      \mathcal{I} = \left\{ (x, 0) \mid x \in X \right\}\,.
\end{equation*}
  \item Для множества $\mathcal{F}_i$ построим вспомогательное множество~$U_i$,
    состоящее из суперпозиций, полученных в результате применения функций
    $g_u \hm\in G_u$ к элементам~$\mathcal{F}_i$:
    $$
      U_i = \left\{ g_u \circ f \mid g_u \in G_u, f \in \mathcal{F}_i \right\}\,.
$$
  \item Аналогичным образом построим вспомогательное множество~$B_i$ для
    бинарных функций $g_b \hm\in G_b$:
    $$
      B_i = \left\{ g_b \circ (f, h) \mid g_b \in G_b, f, h \in \mathcal{F}_i \right\}\,.
$$
  \item Обозначим $\mathcal{F}_{i+1} = \mathcal{F}_i \cup U_i \cup B_i$.
  \item Для каждой суперпозиции~$f$ из~$\mathcal{F}_{i+1}$ добавим пару
    $(f, i+1)$ в~множество $\mathcal{I}_f$, если суперпозиция~$f$ еще там
    не присутствует.
  \item Перейдем к~следующей итерации, п.~2.
\end{enumerate}

Тогда $\mathcal{F} \hm= \cup_{i=0}^\infty \mathcal{F}_i$~--- множество всех
возможных суперпозиций конечной длины, которые можно построить из
данного множества примитивных функций.

Вспомогательное множество~$\mathcal{I}$ позволяет запоминать, на какой
итерации была впервые встречена каждая суперпозиция. Это необходимо, так
как каждая суперпозиция, впервые порожденная на $i$-й итерации, будет
порождена так\-же и на любой итерации после~$i$. Одной из возможностей
избежать необходимости в этом множестве является построение
$\mathcal{F}_{i+1}$ как $\mathcal{F}_{i+1} \hm= U_i \cup B_i$ (без
$\mathcal{F}_i$), а множества~$U_i$ и~$B_i$ строить следующим образом:
\begin{align*}
  U_i &= \left\{ g_u \circ f \mid g_u \in G_u, f \in \cup_{j=0}^{i} \mathcal{F}_j \right\}\,;
\\
  B_i &= \left\{ g_b \circ (f, h) \mid g_b \in G_b, f, h \in \cup_{j=0}^{i} \mathcal{F}_j \right\}\,.
\end{align*}

Алгоритм $\mathfrak{A^*}$ очевидным образом обобщается на случай, когда
множество~$G$ содержит функции произвольной (но конечной) арности.
Действительно, для такого обобщения достаточно строить аналогичным образом
вспомогательные множества для этих функций, а именно: для множества функций~$G_n$ 
арности~$n$ построить вспомогательное множество $H_i^n$ вида
$$
H_i^n = \left\{ g \circ (f_1, f_2, \dots, f_n) \mid g \in G_n, f_j \in \mathcal{F}_i \right\}\,.
$$

В~этих обозначениях $U_i \hm\equiv H_i^1$, а $B_i \hm\equiv H_i^2$.

Тогда множество $\mathcal{F}_{i+1} \hm= \mathcal{F}_i \cup_{n=0}^{n_{\max}} H_i^n$,
где $n_{\max}$~--- максимальное значение арности функций из~$G$.

\smallskip

\noindent
\textbf{Теорема~1.}
\textit{Алгоритм $\mathfrak{A^*}$ действительно породит любую конечную суперпозицию
  за конечное число шагов.}

\smallskip

\noindent
Д\,\,о\,к\,а\,з\,а\,т\,е\,л\,ь\,с\,т\,в\,о\,.\ \ 
  Чтобы убедиться в~этом, \mbox{найдем} номер итерации, на которой будет по\-рож\-де\-на
  некоторая произвольная конечная суперпозиция~$f$. Чтобы найти этот номер,
  пронумеруем вершины графа~$\Gamma_f$ по следующим правилам:
  \begin{itemize}
    \item если это вершина со свободной переменной, то она имеет номер~0;
    \item если вершина $V$ соответствует унарной функции, то она имеет номер
      $i+1$, где $i$~--- номер дочерней для этой функции вершины;
    \item если вершина $V$ соответствует бинарной функции, то она имеет номер
      $i+1$, где $i = \max (l, r)$, а $l$ и $r$~--- номера соответственно
      первой и второй дочерней вершины.
  \end{itemize}

  Нумеруя вершины графа $\Gamma_f$ таким образом, можно получить номер вершины,
  соответ\-ст\-ву\-ющей корню графа. Это и будет номером итерации, на которой получена
  суперпозиция~$f$.

  Иными словами, для любой суперпозиции можно указать конкретный номер
  итерации, на которой она будет получена, что и~требовалось.


\smallskip

В~предложенных ранее методах построения суперпозиций~\cite{Zelinka2008}
необходимо было самостоятельно следить за тем, чтобы в~ходе работы алгоритма
не возникало <<зацикленных>> суперпозиций типа $f(x, y) \hm= g (f(x, y), x, y)$.
Заметим, что в предложенном алгоритме $\mathfrak{A^*}$ такие суперпозиции
не могут возникнуть по построению.

\smallskip

\noindent
\textbf{Порождение моделей с параметрами.}
Алгоритм $\mathfrak{A^*}$, описанный выше, не позволяет получать выражения, содержащие численные
параметры $\boldsymbol{\omega}$ суперпозиции $f(\boldsymbol{\omega}, \mathbf{x})$.
Покажем, однако, на примере конструирования множеств $U_i$ и~$B_i$, как
исходный алгоритм $\mathfrak{A^*}$ может быть расширен путем введения параметров
\begin{align*}
U_i &= g_u \circ (\alpha f + \beta) \,;
\\
B_i &=  g_b \circ (\alpha f + \beta, \psi h + \phi) \,.
\end{align*}
Будем обозначать этот расширенный алгоритм как~$\mathfrak{A}$. Здесь
параметры $\alpha$, $\beta$ зависят только от комбинации $g_u, f$ (или
$g_b, f, h$ для $\alpha$, $\beta$, $\psi$, $\phi$). Соответственно, для
упрощения их индексы опущены. Иными словами, предполагается, что
каждая суперпозиция, полученная на предыдущих итерациях, входит
в порождаемую на следующей итерации, будучи умноженной на некоторый
коэффициент и с константной поправкой.

Очевидно, при таком добавлении параметров $\alpha$, $\beta$, $\psi$,
$\phi$ не происходит изменения мощности получившегося множества
суперпозиций, поэтому алгоритм и~выводы из него остаются
корректными. В~частности, исходный алгоритм является частным случаем
данного при $\alpha \hm\equiv \psi \hm\equiv 1$, $\beta \hm\equiv \phi \hm\equiv 0$.

Переменные $\alpha, \beta, \psi, \phi$ являются параметрами модели. 
В~практических приложениях можно оптимизировать значения этих параметров у
получившихся суперпозиций, например, алгоритмом 
Ле\-вен\-бер\-га--Марк\-вард\-та~\cite{Marquardt1963Algorithm, more:78}.

Заметим также, что такая модификация алгоритма позволяет получить единицу,
например, для построения суперпозиций типа ${1}/{x}$:
$1 \hm= \alpha\ id\ x + \beta \mid \alpha = 0, \beta \hm= 1$.

Отдельно подчеркнем, что параметры $\boldsymbol{\omega}$ у разных
суперпозиций различны. Однако, так как каж\-дый из па\-ра\-мет\-ров зависит только
от со\-от\-вет\-ст\-ву\-ющей комбинации функций, к которым он относит\-ся, конкретные
значения параметров не учитываются при поиске одинаковых суперпозиций.
Иными словами, при тестировании суперпозиций на равенство сравниваются лишь
структуры соответствующих им деревьев и значения в узлах, соответствующих
функциям и свободным переменным.

Заметим, что и~этот алгоритм очевидным образом обобщается на случай
множества~$G$, содержащего функции произвольной арности.

\section{Число возможных суперпозиций}

Оценим число суперпозиций, получаемых после каждой итерации
алгоритма~$\mathfrak{A}$. Очевидно, с учетом вышеупомянутых оговорок
касательно сравнения параметризованных суперпозиций, это число равно
аналогичному числу для алгоритма~$\mathfrak{A^*}$.

Итак, пусть дано $n$ независимых переменных: $| X | \hm= n$, а мощность
множества~$G$ распишем через мощности его подмножеств функций соответствующей
арности: $| G_1 | \hm= l_1$, $| G_2 | \hm= l_2, \dots$, $| G_p | \hm= l_p$. На нулевой
итерации имеем $P_0 \hm= n$ суперпозиций.

На первой итерации дополнительно порож\-да\-ется

\noindent
$$
P_1 = l_1 n + l_2 n^2 + \dots + l_n n^p = \sum\limits_{i=1}^p l_i P_0^i\,,
$$
и суммарное число суперпозиций после первой итерации
$$
\hat{P}_1 = P_1 + P_0 = \sum\limits_{i=1}^p l_i P_0^i + P_0\,.
$$

Как было замечено ранее, суперпозиции, по\-рож\-ден\-ные на $k$-й итерации, будут
также порождены и на любой следующей после $k$ итерации, поэтому суммарное
число суперпозиций после второй итерации будет равно
$$
\hat{P}_2 = \sum\limits_{i=1}^p l_i \hat{P}_1^i\,.
$$

И вообще, после $k$-й итерации будет порождено
$$
\hat{P}_k = \sum\limits_{i=1}^p l_i \hat{P}_{k-1}^i\,.
$$

Оценим порядок роста количества функций, порожденных после $k$-й итерации.

\smallskip

\noindent
\textbf{Теорема~2.}
\textit{Пусть в множестве примитивных функций $G$ содержится $l_p$ функций арности
  $p \hm> 1$ и ни одной функции арности $p \hm+ k \mid k \hm> 0$ и имеется $n \hm> 1$
  независимых переменных. Тогда справедлива следующая оценка чис\-ла
  суперпозиций, порожденных алгоритмом $\mathfrak{A}$ после $k$-й итерации:}
  $$
  \left\vert \mathcal{F}_k \right\vert = \mathcal{O} 
  \left(l_p^{\sum_{i=0}^{k-1} p^i} n^{p^k}\right)\,.
  $$


\smallskip

\noindent
Д\,\,о\,к\,а\,з\,а\,т\,е\,л\,ь\,с\,т\,в\,о\,.\ \ 
  Оценим сначала порядок рос\-та для случая, когда есть лишь одна $m$-ар\-ная
  функция и $n$ свободных переменных.

  После первой итерации алгоритма будет по\-рож\-де\-но $n^m \hm+ n$ суперпозиций.
  После второй~--- $(n^m + n)^m \hm+ n^m + n$, что можно оценить как\linebreak
  $(n^m)^m \hm= n^{m^2}$. И~вообще, после $k$-й итерации чис\-ло
  суперпозиций можно оценить как $n^{m^k}$.

  Видно, что для оценки скорости роста количества порожденных суперпозиций
  можно учитывать только функции с наибольшей арностью.

  Рассмотрим теперь случай, когда имеется не одна функция арности~$m$, а
  $l_m$ таких функций. Тогда на первой итерации порождается $l_m n^m \hm+ n$
  суперпозиций, на второй:
  $$
  l_m (l_m n^m + n)^m + l_m n^m + n \approx l_m^{m+1} n^{m^2}\,,
$$
  на третьей, с учетом этого приближения:
  $$
  l_m (l_m^{m+1} n^{m^2})^m = l_m l_m^{m(m+1)} n^{m^3} = l_m^{m^2 + m + 1} n^{m^3}\,.
$$
  И~вообще, скорость роста количества порожденных суперпозиций можно оценить
  как:
  $$
  \left\vert \mathcal{F}_k \right\vert = \mathcal{O} 
  \left(l_m^{\sum_{i=0}^{k-1} m^i} n^{m^k}\right)\,.
  $$
  Таким образом, получаем оценку в общем случае, когда в множестве $G$ содержится
  $l_p$ функций ар\-ности~$p$ и ни одной функции ар\-ности $p \hm+ k \mid k\hm > 0$:
  $$
\left\vert \mathcal{F}_k \right\vert = \mathcal{O} 
\left(l_p^{\sum_{i=0}^{k-1} p^i} n^{p^k}\right)\,.
  $$


%\smallskip

\section{Множество допустимых суперпозиций}

Предложенный выше алгоритм позволяет получить действительно все возможные
суперпозиции, однако не все они будут пригодны в~практических приложениях:
например, $\ln x$ имеет смысл только при $x \hm> 0$, а ${x}/{0}$ не имеет
смысла вообще никогда. Выражения типа ${x}/{\sin x}$ имеют смысл только
при $x \hm\neq \pi k$.

Таким образом, необходимо введение понятия множества \textit{допустимых}
суперпозиций, т.\,е.\ таких суперпозиций, которые в условиях данной
задачи корректны.

\smallskip

\noindent
\textbf{Определение~3.}
\textit{Допустимая суперпозиция $f$~--- такая суперпозиция, значение которой
  определено для любой комбинации значений свободных переменных, область
  значений $\mathbb{X}$ которых определяется конкретной задачей,
  $\mathbb{X} \subset \mathbb{R}^n$, где $n$~--- число свободных переменных.}


\smallskip

Одним из способов построения только допустимых суперпозиций является
модификация предложенного алгоритма таким образом, чтобы отслеживать
совместность областей определения и \mbox{областей} значений соответствующих
функций в ходе построения суперпозиций. Для свободных переменных это,
в свою очередь, означает необходимость задания областей значений
$\mathbb{X}$ пользователем при решении конкретных задач.

Таким образом, можно сформулировать очевидное \textit{достаточное условие
недопустимости} суперпозиции:

\smallskip

\noindent
\textbf{Определение~4.}
  Достаточное условие недопустимости суперпозиции~$f$: в соответствующем дереве
  $\Gamma_f$ хотя бы одна вершина~$V_i$ имеет хотя бы одну дочернюю вершину~$V_j$ 
  такую, что область значений функции $g_{s(j)}$ шире, чем область
  определения функции $g_{s(i)}$:
  $$
  \exists i, j : V_i \in \Gamma_f, V_j \in \Gamma_f \wedge \exists \kappa :
    \kappa \in \mathcal{E} g_{s(j)} \wedge \kappa \notin \mathcal{D} g_{s(i)}\,.
$$


\smallskip

Говоря, что область значений функции~$f$ шире области определения функции~$g$, 
имеем в~виду, что существует, по крайней мере, одно значение функции~$f$, 
не входящее в область определения функции~$g$.

Подчеркнем, что, хотя свободные переменные могут принимать, например, все
значения из~$\mathbb{R}$, выбором множества~$\mathbb{X}$ можно обеспечить
возможность использования их в качестве аргументов функций с более узкой,
чем $\mathbb{R}$, но не менее узкой, чем $\mathbb{X}$, областью определения,
если это не противоречит данной выборке.

Для построения множества допустимых суперпозиций достаточно построить
множество всех возможных суперпозиций при помощи алгоритма~$\mathfrak{A}$,
а затем удалить из этого множества все суперпозиции, не удовлетворяющие
сформулированному признаку.

\section{Алгоритм итеративного стохастического порождения суперпозиций}

Несмотря на то что построенный ранее итеративный алгоритм~$\mathfrak{A}$ по\-рож\-де\-ния
суперпозиций позволяет получить за конечное число шагов произвольную
суперпозицию, для практических применений он непригоден в~связи с чрезмерной
вычислительной сложностью, как и~любой алгоритм, реализующий полный перебор.
Вместо него предлагается использовать стохастические алгоритмы и~ряд эвристик,
позволяющих на практике получать за приемлемое время результаты,
удовлетворяющие заранее заданным условиям. В~данном разделе описывается
практически реализуемый вариант алгоритма~$\mathfrak{A}$, который и был использован
в~вычислительном эксперименте. Опишем вспомогательный алгоритм 
случайного порождения суперпозиции.

\smallskip

\noindent
\textbf{Алгоритм~2.} 
  Алгоритм случайного порождения суперпозиции $\mathcal{RF}$.

  Вход:
  \begin{itemize}
    \item набор пороговых значений $0 < \xi_1 < \xi_2 \hm< \xi_3 \hm< 1$;
    \item максимальная глубина порождаемой суперпозиции Td.
  \end{itemize}


\smallskip

Алгоритм работает следующим образом. Генерируется случайное чис\-ло~$\xi$ на
интервале $(0; 1)$ и рассматриваются следующие случаи:
\begin{itemize}
  \item $\xi \leq \xi_1$: результатом алгоритма является некоторая случайно
    выбранная свободная переменная;
  \item $\xi_1 < \xi \leq \xi_2$: результатом алгоритма является    числовой
    параметр;
  \item $\xi_2 < \xi \leq \xi_3$: результатом алгоритма является некоторая
    случайно выбранная унарная функция, для определения аргумента которой
    данный алгоритм рекурсивно запускается еще раз;
  \item $\xi_3 < \xi$: результатом алгоритма является некоторая случайно
    выбранная бинарная функция, аргументы которой порождаются аналогичным
    образом.
\end{itemize}
При этом порождение тривиальных суперпозиций (свободных переменных и
параметров) запрещено: на самом первом шаге пороговые значения масштабируются
таким образом, чтобы всегда порождалась унарная или бинарная функция.
Аналогично при превышении значения~Td пороговые значения масштабируются
таким образом, чтобы был порожден узел, соответствующий свободной переменной
или параметру, и алгоритм за\-вер\-шился.

Каждой порожденной суперпозиции~$f$ ставится в
соответствие ее \textit{качество}~$Q_f$, рассчитываемое исходя из функционала ошибки~$S_f$ 
этой суперпозиции на выборке~$D$ и ее сложности $C_f$~---
числа узлов в соответствующем графе~$\Gamma_f$. Функционал~$Q_f$ выбирается эвристически
с учетом следующих естественных соображений:
\begin{itemize}
  \item из двух суперпозиций одинаковой слож\-ности~$C_f$ выбирается обеспечивающая
    более оптимальное значение функционала ошибки~$S_f$;
  \item из двух суперпозиций, имеющих одно и то же значение функционала ошибки~$S_f$,
    выбирается суперпозиция, обладающая меньшей слож\-ностью~$C_f$.
\end{itemize}

\noindent
\textbf{Алгоритм~3.}
  Итеративный алгоритм стохастического порождения суперпозиций.

  Вход:
  \begin{itemize}
    \item множество порождающих функций~$G$, со\-сто\-ящее только из унарных
      и бинарных функций;
    \item регрессионная выборка~$D$;
    \item $N_{\max}$~--- максимальное число одновременно рассматриваемых
      суперпозиций;
    \item $I_{\max}$~--- максимальное число итераций алгоритма;
    \item $\hat{Q}$~--- минимальное значение функционала~$Q_f$:
    \begin{equation}
  \label{eq:q_f}
  Q_f = \fr{1}{1 + S_f} \left(\alpha + \fr{1 - \alpha}
  {1 + \exp \left({C_f}/{\beta} - \tau\right)}\right)\,,
\end{equation}
где $\alpha$~--- некоторый коэффициент влияния штрафа за сложность, $0 \hm\ll \alpha \hm< 1$,
$\beta$~--- коэффициент строгости штрафа за сложность, $\beta \hm> 0$, а
$\tau$~--- коэффициент, характеризующий желаемую сложность модели;
    \item $\gamma_{\mathrm{mut}}$~--- доля суперпозиций, подверженных случайной
      замене узлов их деревьев;
    \item $\gamma_{\mathrm{cross}}$~--- доля суперпозиций, для которых выполняется
      случайный обмен поддеревьями;
    \item прочие параметры, используемые в~\eqref{eq:q_f} и алгоритме~2.
  \end{itemize}


\noindent
\begin{enumerate}
  \item Инициализируется упорядоченный набор $\mathcal{X}_f$ суперпозиций,
    а~именно: порождается $N_{\max}$ суперпозиций алгоритмом~2.
  \item Оптимизируются параметры~$\boldsymbol{\omega}$ суперпозиций
    из~$\mathcal{X}_f$ алгоритмом Ле\-вен\-бер\-га--Марк\-вард\-та.
  \item Выполняются простейшие преобразования, упрощающие суперпозицию:
    например, выражения вида $0 \cdot x$ заменяются на~0.
  \item Вычисляется значение~$Q_f$ для каждой еще не оцененной суперпозиции~$f$ 
  из~$\mathcal{X}_f$: для нее рассчитывается значение функционала ошибки~$S_f$ 
  на выборке~$D$ и ставится в соответствие значение~$Q_f$. Для
    суперпозиций, при вычислении~$Q_f$ которых была хотя бы раз получена
    ошибка вычислений из-за несовпадения областей определений и значений,
    принимается $Q_f \hm= -\infty$.
  \item Набор суперпозиций~$\mathcal{X}_f$ сортируется согласно значениям
    функционала~$Q_f$.
  \item Суперпозиции с наименьшими значениями~$Q_f$ удаляются из массива~$\mathcal{X}_f$ 
  до тех пор, пока его размер не станет равен~$N_{\max}$.
  \item Отбирается некоторая часть~$\gamma_{\mathrm{mut}}$ суперпозиций с наименьшими
    значениями~$Q_f$ из~$\mathcal{X}_f$. У~этой час\-ти происходит случайная замена
    одной функции или свободной переменной на другую: генерируются две случайные величины,
    одна из которых служит для выбора вершины дерева~$\Gamma_f$, которую
    предстоит изменить, а другая~--- для выбора нового элемента для этой вершины.
    Замена такова, что сохраняется структура суперпозиции, т.\,е.\
    в случае замены функции сохраняется арность, а свободная переменная
    заменяется только на другую свободную переменную. Исходные
    суперпозиции сохраняются в массиве~$\mathcal{X}_f$.
  \item Повторяются шаги 4--5.
  
  \begin{figure*} %fig2
\vspace*{1pt}
 \begin{center}
 \mbox{%
 \epsfxsize=112.519mm
 \epsfbox{rud-2.eps}
 }
 \end{center}
 \vspace*{-9pt}
  \Caption{Поверхности функции $Q_f$ для некоторых $\beta$ (\textit{1}~--- $\beta\hm=0{,}1$; 
  \textit{2}~--- 1; \textit{3}~--- $\beta\hm=5$)
    и фиксированного $\tau = 5$}
  \label{fig:fitness_surph}
\end{figure*}

 
  
  
  \item Производится случайный обмен поддеревьями у $\gamma_{\mathrm{cross}}$ суперпозиций
    с наибольшими значениями $Q_f$. Вершины, соответствующие этим поддеревьям,
    выбираются случайным образом. При этом исходные суперпозиции сохраняются
    в~$\mathcal{X}_f$.
    

Таким образом, чем лучше результаты суперпозиции и чем она проще, тем ближе
значение функционала~$Q_f$ к~$1$.


  \item Повторяются шаги 2--5.
  \item Проверяются условия останова: если либо чис\-ло итераций превышает
    $I_{\max}$, либо в~массиве $\mathcal{X}_f$ находится суперпозиция со значением~$Q_f$, 
    большим $\hat{Q}$, то алгоритм останавливается
    и результатом считается суперпозиция с наибольшим значением~$Q_f$, иначе
    осуществляется переход к шагу~2.
\end{enumerate}

\section{Вычислительный эксперимент}

В~вычислительном эксперименте вос\-ста\-нав\-ли\-ва\-ет\-ся функциональная зависимость
$y \hm= 2 \cosh \sqrt{(x_1^2\hm + x_2^2)/2}$, соответствующая фигуре вращения
цепной линии. При этом значения зависимой
переменной~$y$ были искусственно зашумлены аддитивной добавкой из
распределения $\mathcal{N} (0, 0{,}1)$ и соответствующая ей переменная
присутствовала во множестве используемых свободных переменных.

В качестве функционала ошибки~$S$ используется сумма квадратов
регрессионных остатков для данной суперпозиции~$f$ с вектором параметров
$\boldsymbol{\omega}$ при регрессионной выборке~$D$:
\begin{equation}
  \label{eq:sse_expr}
  S(\boldsymbol{\omega}, f, D) = \sum\limits_{i=1}^N (y_i - f (\boldsymbol{\omega}, 
  \mathbf{x}_i))^2\,.
\end{equation}

Значение функционала ошибки~$S$ при подстановке исходной незашумленной
функциональной зависимости составляет $\approx 4{,}29$, сложность исходной
суперпозиции~--- 14.

\begin{table*}\small
\begin{center}
\Caption{Результаты вычислительного эксперимента для предложенного алгоритма}
  \label{tabl:results}
  \vspace*{2ex}

\begin{tabular}{| c | c | l | c | c | c |} 
  \hline
    $N$ & $i$   & \multicolumn{1}{c|}{Суперпозиция}  & $S_f$                & $C_f$ & $Q_f$             \\ 
    \hline
    &&&&&\\[-9pt]
    1   & 13    & $ 1{,}0002 \left(2{,}72^{\sqrt{x \cdot x + y \cdot y}/2} + 
    2{,}56^{\sqrt{x \cdot x + y \cdot y}/{-1{,}93}}\right)$ & 
    $\approx 4{,}10$     & 29    & $\approx 0{,}010$    \\ 
    \hline
    &&&&&\\[-9pt]
    2   & \hphantom{9}9     & $ 2{,}001 \cosh \fr{\sqrt{x \cdot x + y \cdot y}}{1{,}999}$& 
    $\approx 4{,}25$   & 14    & $\approx 0{,}188$   \\ 
    \hline
  \end{tabular}
  \end{center}
\end{table*}

\begin{table*}\small
\begin{center}
  \Caption{Результаты вычислительного эксперимента для алгоритма~\cite{Zelinka2008}}
  \label{tabl:results_Z}
  \vspace{2ex}
  
  \begin{tabular}{| c | c | c | c | c |} 
  \hline
    $i$ & Суперпозиция  & $S_f$                & $C_f$ & $Q_f$             \\ 
    \hline
    &&&&\\[-9pt]
    29  & $ 2{,}66^{\sqrt{x^2 + y^2}/{2{,}23}} - 
    \fr{x^2 + y^2}{3{,}03} + \fr{x^2 \cdot x^2 + y^2\cdot  y^2}{6{,}3} + 0{,}93$
             & $\approx 6{,}2$     & 43    & $ \approx 0{,}007 $ \\ 
             \hline
  \end{tabular}
  \end{center}
\end{table*}

\begin{figure*} %fig3
\vspace*{1pt}
 \begin{center}
 \mbox{%
 \epsfxsize=160.163mm
 \epsfbox{rud-3.eps}
 }
 \end{center}
 \vspace*{-9pt}
\Caption{Первая порожденная суперпозиция~(\textit{1})
и зашумленные точки выборки~(\textit{2})~(\textit{a}) 
и исходная зависимость~(\textit{3})~(\textit{б})}
\end{figure*}
\begin{figure*} %fig4
\vspace*{-3pt}
 \begin{center}
 \mbox{%
 \epsfxsize=160.163mm
 \epsfbox{rud-4.eps}
 }
 \end{center}
 \vspace*{-11pt}
  \Caption{Вторая порожденная суперпозиция~(\textit{1}) и зашумленные точки выборки~(\textit{2})~(\textit{a}) 
  и исходная зависимость~(\textit{3})~(\textit{б})}
\end{figure*}

В данной работе используется функционал~$Q_f$ вида~(\ref{eq:q_f}).
Значения параметров~$\alpha$, $\beta$ и~$\tau$
выбираются экспертно исходя из предположений
о виде искомой суперпозиции и моделируемом явлении.

Второй множитель в~\eqref{eq:q_f} выполняет роль штрафа за слишком
большую сложность суперпозиции, что позволяет выбирать более простые модели,
избегая эффекта переобучения и экстремальных случаев вроде порождения
интерполяционных полиномов. На рис.~\ref{fig:fitness_surph}
приведены поверхности~$Q_f$ для различных значений~$\beta$ при фиксированном
$\tau \hm= 5$.



Использованные параметры алгоритма~3: $N_{\max} \hm= 200, I_{\max} \hm= 50,
\hat{Q} \hm= 0{,}95, \tau \hm= 20, \alpha \hm= 0{,}05$, $\beta \hm= 1$, 
$\gamma_{\mathrm{mut}} \hm= {1}/{3}$,
$\gamma_{\mathrm{cross}} \hm= {1}/{3}$. При отсутствии улучшения результатов в~течение
нескольких итераций подряд алгоритм~3 также завершался.

Результаты вычислительного эксперимента приведены в табл.~\ref{tabl:results}.
Указан номер итерации~$i$, на которой суперпозиция была впервые
получена, сама суперпозиция, среднеквадратичная ошибка~\eqref{eq:sse_expr} и сложность в
смысле числа узлов в соответствующем графе выражения. Числовые коэффициенты
в приведенных формулах и значения функционала~$S_f$ искусственно округлены до
нескольких значащих цифр.

Алгоритм запускался для двух разных наборов элементарных функций.
В обоих случаях элементарные функции включали 
в себя стандартные арифметические операции и операцию возведения в степень. 
Для удобства возведение в степень~${1}/{2}$ (и близкие ей) заменено в таблице 
на операцию извлечения корня.


В первом случае в наборе отсутствовала функция~$\cosh$. При этом по результатам
10 запусков наилучшей суперпозицией, полученной предложенным алгоритмом,
оказалась функция за номером~1 из табл.~\ref{tabl:results}. Видно, что выражение
в скобках близко определению $\cosh x = ({e^x + e^{-x}})/2$,
однако разные значения оснований степенных функций могут затруднить экспертный
анализ полученного выражения, которое само по себе является достаточно громоздким.

Во втором случае набор элементарных функций также включал в себя функцию~$\cosh$, 
результату этого выражения соответствует суперпозиция за номером~2.
Включение $\cosh$ в~$G$ позволило существенно быстрее подобрать искомую функцию, и сложность
получившейся суперпозиции также существенно меньше.

Кроме того, предложенный алгоритм сравнивался с алгоритмом~\cite{Zelinka2008},
в котором суперпозиции кодировались бинарной строкой и применялись стандартные
генетические алгоритмы на получавшихся строках; во множестве используемых
функций также отсутствовала функция~$\cosh$.



Наилучшая суперпозиция, полученная алгоритмом~\cite{Zelinka2008} по результатам
10~запусков, приведена в табл.~\ref{tabl:results_Z}. Полученная суперпозиция
имеет существенно более высокую сложность, чем суперпозиции, перечисленные в
табл.~\ref{tabl:results}.

На рис.~3 отображены изометрические
проекции первой из приведенных в табл.~\ref{tabl:results} суперпозиций. На
рис.~3\textit{а} данная суперпозиция сравнивается с точками синтезированной
зашумленной выборки, на рис.~3\textit{б}  она же приведена вместе
с исходной незашумленной зависимостью. Аналогичные проекции
приведены для второй суперпозиции на рис.~4.

\vspace*{-6pt}

\section{Заключение}

В~работе исследованы индуктивные алгоритмы порождения допустимых существенно
нелинейных суперпозиций. Предложен переборный алгоритм, порождающий все
возможные суперпозиции заданной сложности за конечное число шагов.
Сформулированный алгоритм решает некоторые типичные проблемы предложенных ранее методов.
Описан стохастический алгоритм индуктивного порождения существенно нелинейных
суперпозиций и приведены результаты вычислительного эксперимента на синтетических
данных. Описанный алгоритм выбирает менее точные, но более простые
модели, что позволяет избежать переобучения и выполнить простейший отбор признаков.

\vspace*{-6pt}

{\small\frenchspacing
{%\baselineskip=10.8pt
\addcontentsline{toc}{section}{Литература}
\begin{thebibliography}{99}

\bibitem{duffy:1999:srised} 
\Au{Duffy~J., Engle-Warnick~J.} Using symbolic regression to infer strategies 
from experimental data~// Evolutionary\linebreak\vspace*{-12pt}\columnbreak

\noindent
 Computation in Economics and Finance, 2002. 
Vol.~100.  P.~61--84.

\bibitem{Barmpalexis201175} %2
\Au{Barmpalexis~P., Kachrimanis~K., Tsakonas~A., Georgarakis~E.} 
Symbolic regression via genetic programming in the optimization of 
a controlled release pharmaceutical formulation~// Chemometrics and 
Intelligent Laboratory Systems, 2011. Vol.~107. No.~1. P.~75--82.

\bibitem {davidson:2000:snrea} %3 
\Au{Davidson J.\,W., Savic D.\,A., Walters G.\,A.} Symbolic and numerical 
regression: Experiments and applications~// Developments in Soft Computing, 2001. 
Vol.~6. P.~175--182.

\bibitem {strijov07poisk} %4
\Au{Стрижов В.\,В.} 
Поиск параметрической регрессионной модели в индуктивно заданном множестве~// 
Вычислительные технологии, 2007. T.~1. C.~93--102.

\bibitem {Strijov08InductMethods}  %5
\Au{Стрижов В.\,В.} Методы индуктивного порождения регрессионных моделей.~--- М.:~ВЦ~РАН, 2008.


\bibitem {reference/ml/X10vc} %6
\Au{Sammut C., Webb  G.\,I.} Symbolic regression~// Encyclopedia of Machine Learning.~---  
Berlin: Springer, 2010.


\bibitem {StrijovW10}  %7
\Au{Strijov V.\,V., Weber G.\,W.} Nonlinear regression model generation using 
hyperparameter optimization~// Computers and Mathematics with Applications, 2010. Vol.~60. No.\,4. P.~981--988.

\bibitem {Koza1998GP} 
\Au{Koza J.\,R.} 
Genetic programming~// Encyclopedia of Computer Science and Technology, 1998. Vol.~39. No.\,24. P~29--43.

\bibitem {Koza1998Intro} 
\Au{Koza J.\,R.} Introduction to genetic algorithms.~--- Cambridge: MIT Press, 1998.

\bibitem {Zelinka2008} 
\Au{Zelinka I., Oplatkova Z., Nolle L.} Analytic programming and symbolic 
regression by means of arbitrary evolutionary algorithms~// Int.\ 
J.~Simulation Syst. Sci. Technol., 2005. Vol.~6. No.\,9. P~44--56.

\bibitem {Tirsin2005} 
\Au{Тырсин А.\,Н.} 
Об эквивалентности знакового и наименьших модулей методов построения линейных моделей~// 
Обозрение прикладной и промышленной математики, 2005. Т.~12. №\,4. C.~879--880.

\bibitem {Pavlovsky2000} \Au{Павловский Ю.\,Н.} Имитационные 
модели и системы.~--- М.:~Фазис, 2000.

\bibitem {MathEnc1984_4} \Au{Битюцков В.\,И., Войцеховский М.\,И., Иванов А.\,Б.} 
Математическая энциклопедия. Т.~4.~--- М.:~Советская энциклопедия, 1984.

\bibitem {Marquardt1963Algorithm} 
\Au{Marquardt D.\,W.} An algorithm for least squares estimation of nonlinear parameters~// 
J.~Soc. Ind. Appl. Math., 1963. Vol.~11. No.\,2. P.~431--441.

\label{end\stat}


\bibitem {more:78} \Au{More J.\,J.} 
The Levenberg--Marquardt algorithm: Implementation and theory~// 
Lecture Notes in Mathematics 630: Numerical Analysis.~--- Berlin: Springer-Verlag,
1978. P.~105--116.
\end{thebibliography}
}
}

\end{multicols}       %5Abst+avt
\def\stat{grusho}

\def\tit{МОДЕЛЬ СЛУЧАЙНЫХ ГРАФОВ ДЛЯ~ОПИСАНИЯ~ВЗАИМОДЕЙСТВИЙ В СЕТИ$^*$}

\def\titkol{Модель случайных графов для описания взаимодействий в сети}

\def\autkol{А.\,А.~Грушо, Е.\,Е.~Тимонина}

\def\aut{А.\,А.~Грушо$^1$, Е.\,Е.~Тимонина$^2$}

\titel{\tit}{\aut}{\autkol}{\titkol}

{\renewcommand{\thefootnote}{\fnsymbol{footnote}}\footnotetext[1]
{Работа выполнена при поддержке РФФИ (проекты №\,10-01-00480, №\,11-07-00112).}}

\renewcommand{\thefootnote}{\arabic{footnote}}
\footnotetext[1]{Институт проблем информатики Российской академии наук; Московский государственный 
университет им.\ М.\,В.~Ломоносова, факультет вычислительной математики и кибернетики, 
grusho@yandex.ru}
\footnotetext[2]{Институт проблем информатики Российской академии наук, eltimon@yandex.ru}

\Abst{Рассматривается новый класс случайных графов, призванный 
моделировать функционирование сети во времени. Предполагается, что наблюдения за 
сетью ведутся с помощью <<оконного>> метода. С~целью выявления аномалий 
исследуется нормальное поведение степеней, которые можно наблюдать в <<окнах>> 
рассматриваемой модели. Исследована асимптотика максимальной степени вершин в 
графе, который порожден <<окном>> данного размера.}

\KW{случайные графы; моделирование глобальных сетей; информационная 
безопасность; аномальное поведение}

\vskip 14pt plus 9pt minus 6pt

      \thispagestyle{headings}

      \begin{multicols}{2}

            \label{st\stat}

\section{Введение}

     Возможны различные способы распространения информации в сети. Наиболее 
известным способом является обращение к информационным ресурсам, выложенным на 
тематических сайтах. Поиск таких сайтов по нужной тематике~--- задача 
     ин\-тер\-нет-по\-иско\-ви\-ков. Другое направление~---\linebreak распространение информации 
по электронной поч\-те или через налаженные и специальным образом формируемые связи. 
К~таким способам относятся спам и управление бот-се\-тями. 
     
     Формирование собрания единомышленников или предупреждение о каком-либо 
событии могут передаваться любым из перечисленных выше методов. В~связи с этим 
представляет интерес моделирование процессов последовательной передачи информации. 
Будем рассматривать второй способ передачи информации. 
     
     В предположении о дискретности времени будем описывать состояние связей хостов 
между\linebreak собой неориентированным случайным графом. Реб\-рам графа отвечают логические 
связи хостов. Вмес\-те с тем последовательности связей вершин не обязательно связаны с 
распространением инфор\-мации в рамках некоторой корпорации. В~этом \mbox{случае} 
последовательности связей формируются случайно и независимо друг от друга. 
Выявление корпоратив\-ных связей в сети или центров управления связано с анализом 
случайных связей в рамках большого случайного графа сети. 
     
     Будем рассматривать изменения в графе сети с течением времени с помощью 
<<скользящего окна>>. Граф, который получается при таком рассмотрении, формируется 
фиксацией логических связей, захватываемых <<окном>> и отображенных на одном 
графе. Такой граф получается наложением всех графов в моменты времени, 
принадлежащие <<окну>>, и объединением параллельных ребер.
     
     В работе определена математическая модель таких случайных графов и исследованы 
характеристики, связанные с указанными выше прикладными задачами. Полученные 
результаты носят асимптотический характер при условии, что число хостов стремится к 
бесконечности. 
     
     Работа имеет следующую структуру. В~разд.~2 приведены некоторые близкие 
модели случайных графов. В~разд.~3 определена основная динамическая модель. 
В~разд.~4 исследована асимптотика максимальной степени в случайном графе, связанном 
с <<окнами>>. В~разд.~5 подведены итоги и намечены дальнейшие задачи. 
     
\section{Модели случайных графов}
     
     В научной литературе рассматривались модели случайных графов, связанных с 
Интернетом. В~работе~[1] в качестве одного из примеров приводится классическая 
модель случайного графа $G_{N,p}$ с независимыми ребрами, появляющимися с одной и 
той же вероятностью~$p$. Этой модели посвящено много работ и книг~[2--13]. И хотя 
осново\-по\-ла\-га\-ющая \mbox{статья} Эрдеша и Реньи~\cite{2-gr} связана с несколько другим классом 
случайных графов, большинство аналогичных результатов было также доказано для 
графов $G_{N,p}$ в работах~[3--6, 8] и~др. Фазовые перехо-\linebreak\vspace*{-12pt}

\pagebreak

\noindent
ды в структуре таких графов 
впервые исследованы в\linebreak работах~[3--5]. Первые модели с неравновероятными ребрами 
исследовались в работах~\cite{7-gr}. В~пе\-ре\-чис\-лен\-ных моделях появление ребра не 
допускало его дальнейшего исчезновения.
     
     Изменение случайного графа во времени в связи с задачей роста сети Интернет 
рассматривалось в работах~[7, 9, 10]. 
     
     Специальный класс случайных графов, посвященный исследованию связей в 
Интернете, объединяет модели графов ин\-тер\-нет-типа~[11--13]. Они определяются 
степенями вершин, являющихся независимыми случайными величинами. При этом 
свободные концы ребер замыкаются друг на друга случайно и равновероятно. 
     
\section{Динамическая модель сетевого взаимодействия}
     
     Определим детально модель сетевого взаимодействия, кратко изложенную во 
введении. 
     
     Рассмотрим дискретное время $t \hm= 0, 1, 2, \ldots$ Множество хостов сети 
обозначим $A\hm \{a_1, \ldots , a_N\}$. Логическая связь хостов~$a_i$ и~$a_j$ в момент 
времени~$t$ означает либо наличие в этот момент времени сеанса связи по протоколу 
ТСР между~$a_i$ и~$a_j$, либо передачу одиночного пакета от одного хоста к другому в 
этот момент времени по любому протоколу без установления соединения. Для простоты 
считаем, что время прохождения пакета по сети равно~1. При этом выбрасываются из 
рассмотрения все промежуточные поддерживающие сетевые службы (провайдеры, 
маршрутизаторы, адресные службы и~т.\,д.). Из этих допущений получаем модель графа 
сети. В~каждый момент времени~$t$ определен неориентированный граф~$G_t$, 
вершины которого совпадают с множеством~$A$, а ребра соответствуют существующим 
в момент времени~$t$ логическим связям. Из определения логической связи следует, что 
в соседние моменты времени~$t$ и $t\hm+1$ существование данного ребра в графе~$G_t$ 
и графе $G_{t+1}$ являются зависимыми событиями. Однако процессы появления и 
исчезновения разных ребер можно считать независимыми. 
     
     В простейшем случае полагаем, что процесс, описывающий возникновение и 
исчезновение одного ребра, является стационарной однородной \mbox{цепью} Маркова с двумя 
состояниями: 1~--- есть ребро, 0~--- нет ребра. 
     
     Наблюдения за графами $\{G_t\}$ происходят с помощью <<оконной>> системы. 
Пусть задано натуральное число~$r$ и для любого момента времени~$t$ рассматриваются 
графы $G_t, \ldots , G_{t+r}$, появляющиеся в моменты времени $[t,\,t+r]$. Определим 
операцию объединения этих графов
     $$
     G_{t,r}=\bigcup\limits_{i=0}^r G_{t+i}\,,
     $$
где из нескольких параллельных ребер оставляется одно ребро. Граф $G_{t.r}$ несет 
информацию об активности любой вершины в заданный промежуток времени. Эти графы 
представляют интерес в задачах информационной безопасности. Например, если вершина 
$a_i$ является центром управления бот-сетью, то использование бот-сети для организации 
DDoS атаки должно порождать в некоторый промежуток времени $[t,\,t+r]$ резкое 
повышение степени вершины~$a_i$ в графе~$G_{t,r}$. При малых~$r$ и очень 
больших~$r$ при неизвестном~$t$ этот всплеск активности может оказаться незаметным 
в масштабах всей сети. Поэтому исследование модели случайных графов может позволить 
оценить возможности по выявлению неслучайных всплесков активности отдельных 
вершин и даже дать оценку для центра управления бот-сетью. 

     Пусть поведение каждого ребра описывается стационарной однородной цепью 
Маркова с мат\-ри\-цей переходных вероятностей
     \begin{multline*}
     P=\begin{pmatrix}
     p &\ \  1-p\\
     q &\ \  1-q
     \end{pmatrix}\,,\\ 1>p=p(N)>0\,,\quad 1>q=q(N)>0\,,
%     \label{e1-gr}
     \end{multline*}
и стационарным распределением 
$\left( p_0\ \  1-p_0\right)$.

     
     Тогда вероятность непоявления данного ребра за промежуток времени $[t,\,t+r]$ 
равна
     \begin{equation}
     1-p_r=(1-p_0)(1-q)^r\,.
     \label{e3-gr}
     \end{equation}
 Эта вероятность не зависит от~$t$, поэтому будем обозначать ее~$p_r$. Из~(\ref{e3-gr}) 
     получаем вероятность появления данного ребра в промежуток времени $[t,\,t+r]$ в 
графе~$G_{t,r}$:
     \begin{equation*}
     p_r=1-(1-p_0)(1-q)^r\,.
%     \label{e4-gr}
     \end{equation*}
     
\section{Асимптотические оценки максимальной степени в~графах~{\boldmath{$G_{t,r}$}}}

     Для $v\in A$ обозначим через $d(v)$ степень вершины~$v$. Определим 
индикаторную функцию события~$B$:
     $$
     I(B)=\begin{cases}
     1\,, &\ \mbox{если событие $B$ призошло};\\
     0 &\ \mbox{в противном случае.}
     \end{cases}
     $$
     
     Ожидаемое число соединений у фиксированной вершины в ограниченный 
промежуток времени мало по сравнению с общим числом вершин. Асимптотически это 
отвечает условию $p_0 N\hm\rightarrow \mu\hm>0$, $N\hm\rightarrow\infty$. Серии единиц 
связаны с режимом установления соединения. Поэтому величина ($1-p$) может не 
стремиться к~1. Пусть $q\hm\rightarrow 0$, $N\hm\rightarrow\infty$, так что 
$qN\hm\rightarrow\lambda$. Из условия стационарности следует соотношение:
     $$
     q=\fr{p_0}{1-p_0}\left( 1-p\right)\,.
     $$
Таким образом, получаем
$$
p_r=\fr{\mu}{N}+\fr{\lambda r}{N}+O\left( \fr{\lambda r}{N^2}\right)\,.
$$
Положим $\alpha_r=\mu+\lambda r$ и будем считать, что
$$
p_r=\fr{\alpha_r}{N}\,.
$$
Обозначим 
$$
X=\sum\limits_{v\in A} I(d(v)>d)\,.
$$
Тогда 
$$
\left\{ \max\limits_{v\in A} d(v)\right\} >d =\left\{ X\geq 1\right\}\,.
$$
Используя неравенство Маркова, получаем:
\begin{multline}
P\left\{ \max\limits_{v\in A} d(v)>d\right\} \leq{}\\
{}\leq N\sum\limits_{k>d}\begin{pmatrix}
N-1\\ k\end{pmatrix} p_r^k(1-p_r)^{N-1-k}\,.
\label{e5-gr}
\end{multline}
     
     Пусть $B(N-1, k, p_r)$~--- функция распределения биномиального закона, 
$\overline{B}(N-1, k, p_r)\hm=1 \hm- B(N-1, k, p_r)$. Заметим, что в формуле~(\ref{e5-gr}) 
справа стоит $N\overline{B} (N-1, d, p_r)$. 
     
     Асимптотические оценки проводим в условиях
     $$
     N\rightarrow\infty\,, \ d=\fr{C\ln N}{\ln\ln N}\,,\ C>0\,,\ p_r=\fr{\alpha_r}{N}\,.
     $$
     
     Для оценки функции $\overline{B} (N-1, d, p_r)$ воспользуемся представлением для 
неполной бе\-та-функ\-ции~\cite{14-gr}:
     $$
     \overline{B}(N-1,k,p_r)=N\begin{pmatrix}
     N-2\\ k \end{pmatrix} \int\limits_0^{p_r} z^k (1-z)^{N-k-2}dz\,.
     $$
     
     Используя формулу Тейлора, получим при некотором $0\hm<\theta\hm<1$ 
следующее представление для математического ожидания~$X$:
     $$
     {\sf E}X=N(N-1)\begin{pmatrix}
     N-2\\ d\end{pmatrix} (p_r\theta)^d p_r(1-p_r\theta)^{N-2-d}\,.
     $$
Отсюда получаем следующую асимптотическую формулу:
\begin{equation}
{\sf E}X=N^{(1-C)(1+o(1))}\alpha_r(1-e^{-\alpha_r\theta})\,.
\label{e6-gr}
     \end{equation}
     
     \noindent
     \textbf{Теорема.} \textit{При} $C\hm>1$, $N\hm\rightarrow\infty$, $d=C\ln N/(\ln \ln 
N)$, $p_r=\alpha_r/N$
     $$
     P\left\{ \max\limits_{v\in A} d(v)>d\right\}\rightarrow 0\,.
     $$
     
     \noindent
     Д\,о\,к\,а\,з\,а\,т\,е\,л\,ь\,с\,т\,в\,о\ следует из~(\ref{e6-gr}). 
     
     \smallskip
     
     Таким образом, установлена граница для максимальной степени вершины в графе 
$G_{t,r}$. На основании этого результата можно построить оценку центра управления 
бот-сетью. Если существует вершина, степень которой превосходит заданную границу, то 
с вероятностью, близкой к~1, высокая степень этой вершины получена вне условий 
стационарности и других допущений, которые были сделаны для нормального поведения 
сети. 
     
     Предположим теперь, что $C\hm<1$ и математическое ожидание 
${\sf E}X\hm\rightarrow\infty$. Построим оценку числа вершин, имеющих степень больше~$d$, 
при условии, что ${\sf E}X\hm\rightarrow\infty$. С~этой целью оценим и сравним дисперсию 
$DX$ случайной величины~$X$ и $({\sf E}X)^2$. Очевидно, что 
     \begin{multline*}
     ({\sf E}X)^2=N^2(1-B(N-1,d,p_r))^2=\\
     {}=N^2\overline{B}^2(N-1,d,p_r)\,.
     \end{multline*}
Случайную величину~$X$ можно представить в виде:
\begin{equation}
X=\sum\limits_{i=1}^N I_i\,,
\label{e7-gr}
\end{equation}
где $I_i$~--- индикатор события $d(i)\hm>d$. Тогда из~(\ref{e7-gr}) следует:
\begin{multline*}
{\sf E}X^2={\sf E}\left( \sum\limits_{i=1}^N I_i\right) +{\sf E}
\left( 2\sum\limits_{i<j}I_iI_j\right) ={}\\
{}=
N\overline{B}(N-1,d,p_r) +2\sum\limits_{i<j}P(I_iI_j=1)\,.
\end{multline*}
По формуле полной вероятности
\begin{multline*}
P(I_1I_2=1)=p_r\overline{B}^2(N-2,d-1, p_r)+{}\\
{}+(1-p_r)\overline{B}^2(N-2,d,p_r)\,.
\end{multline*}
     
     Рассмотрим разность ${\sf E}X^2$ и $({\sf E}X)^2$. Несложные вычисления приводят к 
выражению:

\pagebreak

\noindent
     \begin{multline*}
{\sf E}X^2-({\sf E}X)^2={}\\
     {}=N\overline{B}(N-1,d,p_r)(1-\overline{B}(N-1,d,p_r))+{}\\
     {}+N(N-1)(1-p_r)p_r b^2(N-2,d,p_r)\,,
     \end{multline*}
где 
$$
b(N-2,d,p_r)=\begin{pmatrix}
N-2 \\ d\end{pmatrix} p_r^d(1-p_r)^{N-2-d}\,.
$$
     
     Предполагалось, что ${\sf E}X\rightarrow\infty$. Тогда
     \begin{multline*}
     \fr{{\sf D}X}{({\sf E}X)^2}=\fr{1}{N\overline{B}(N-1,d,p_r)}\left(
     B(N-1,d,p_r)+{}\right.\\
\left.     {}+\alpha_r(1-p_r)\fr{(N-1) b^2(N-2,d,p_r)}{N\overline{B}(N-1,d,p_r)}\right)\,.
     \end{multline*}
Из предыдущих оценок имеем, что 
\begin{align*}
Nb(N-2,d,p_r) &=O\left(N^{1-C}\right)\,;\\
N\overline{B} (N-1,d,p_r) &= N^{(1-C)(1+o(1))}\alpha_r\left( 1-e^{-\alpha_r\theta}\right)\,.
\end{align*}
Отсюда следует, что
$$
\fr{{\sf D}X}{({\sf E}X)^2} =O\left(N^{C-1}\right)\,,\enskip C<1\,.
$$
     
     Воспользуемся следствиями~4.32 и~4.33 из работы~\cite{15-gr}. Получаем, что с 
вероятностью, стремящейся к~1, $0<X$ и отношение $X/({\sf E}X)\rightarrow 1$. 
     Это означает, что при $C<1$ с вероятностью, близкой к~1, существует вершина степени 
больше~$d$ и число таких вершин совпадает с математическим ожиданием~$X$. 
     

\section{Заключение}
     
     В ходе решения поставленных в данной работе задач появилось много новых 
направлений, которые заслуживают отдельного внимания. В~данной работе исследовано 
поведение больших степеней в графе, соответствующем фиксированному <<окну>>. 
Естественно, желательно обобщить эти результаты на случай скользящего <<окна>>. 
     
     Поведение графов ин\-тер\-нет-ти\-па часто нельзя считать стационарным. Возникает 
задача анализа <<оконных>> графов в условиях нестационарного поведения сети. 
Перечень проблем, возникших при данном исследовании, не исчерпывается данными 
двумя задачами.

{\small\frenchspacing
{%\baselineskip=10.8pt
\addcontentsline{toc}{section}{Литература}
\begin{thebibliography}{99}
     
\bibitem{1-gr}
\Au{Kolaczyk E.\,D.} Statistical analysis of network data: Methods and models.~--- Springer 
Science\;+\;Business Media, LLC, 2009. 386~p. 
\bibitem{2-gr}
\Au{Erd$\ddot{\mbox{o}}$s P., R$\acute{\mbox{e}}$nyi~A.} On the evolution of random 
graphs~// Publ. Math. Inst. Hungarian Acad. Sci., Ser.~A, 1960. Vol.~5. P.~17--61.
\bibitem{3-gr}
\Au{Степанов В.\,Е.} О вероятности связности случайного графа $g_m(t)$~// Теория 
вероятностей и ее применения, 1970. Т.~15. №\,1. С.~55--67.
\bibitem{4-gr}
\Au{Степанов В.\,Е.} Фазовый переход в случайных графах~// Теория вероятностей и ее 
применения, 1970. Т.~15. №\,2. С.~187--203.
\bibitem{5-gr}
\Au{Степанов В.\,Е.} Структура случайных графов $g_n(x\vert h)$~// Теория вероятностей 
и ее применения, 1972. Т.~17. №\,3. С.~227--242.

\bibitem{6-gr}
\Au{Bollobas B.} Random graphs.~--- London: Academic Press, 1985.

\bibitem{9-gr} %7
\Au{Kleinberg J., Kumar S., Raghavan~P., Rajagopalan~S., Tomkins~A.} The web as a graph: 
measurements, models, and methods~// Conference (International) on Combinatorics and 
Computing Proceedings ~--- Berlin: Springer, 1999. Lecture Notes in Computer Science. 
Vol.~1627. P.~1--18.

\bibitem{7-gr} %8
\Au{Колчин В.\,Ф.} Случайные графы.~--- М.: Физматлит, 2000. 256~с.



\bibitem{10-gr} %9
\Au{Kumar R., Raghavan P., Rajagopalan~S., Sivakumar~D., Tomkins~A., Upfal~E.}
Stochastic models for the web graph~// 42nd Annual IEEE Symposium on the Foundations of 
Computer Science Proceedings, 2000. Vol.~41. P.~57--65. 

\bibitem{8-gr} %10
\Au{Chung F., Lu L., Dewey~T., Galas~D.} Duplication models for biological networks~// 
J.~Comput. Biology, 2003. Vol.~10. No.\,5. P.~677--687. 

\bibitem{11-gr}
\Au{Павлов Ю.\,Л., Степанов М.\,М.} Об асимптотических свойствах случайных графов 
<<ин\-тер\-нет-типа>>~// Обозрение прикладной и промышленной математики, 2005. 
Т.~12. №\,3. С.~677.
\bibitem{12-gr}
\Au{Степанов М.\,М.} О~предельных распределениях степеней узлов в случайных графах 
ин\-тер\-нет-типа~// Методы математического моделирования и информационные 
технологии: Тр. Института прикладных математических исследований Карельского 
научного центра РАН.~--- Петрозаводск: КарНЦ РАН, 2005. Вып.~6. С.~235--242.
\bibitem{13-gr}
\Au{Павлов Ю.\,Л.} Предельное распределение объема гигантской компоненты в 
случайном графе ин\-тер\-нет-типа~// Дискретная математика, 2007. Т.~19. №\,3. 
С.~22--34.
\bibitem{14-gr}
\Au{Феллер В.} Введение в теорию вероятностей и ее приложения.~--- 2-е изд.~--- 
М.: Мир, 1967. Т.~1.

\label{end\stat}

\bibitem{15-gr}
\Au{Alon N., Spencer~J.} The probabilistic method.~--- 2nd ed.~--- New York: Jonh Wiley \& Sons, 
2000.
\end{thebibliography}
}
}

\end{multicols}      %6Abst+avt
\include{gudasa}      %7Abst+tavt
\def\stat{zatsman}

\def\tit{ПРОЦЕССЫ ЦЕЛЕНАПРАВЛЕННОЙ ГЕНЕРАЦИИ И РАЗВИТИЯ КРОСС-ЯЗЫКОВЫХ 
ЭКСПЕРТНЫХ ЗНАНИЙ: СЕМИОТИЧЕСКИЕ~ОСНОВАНИЯ~МОДЕЛИРОВАНИЯ$^*$}

\def\titkol{Процессы целенаправленной генерации и развития кросс-языковых 
экспертных знаний: семиотические основания} % моделирования}

\def\aut{И.\,М.~Зацман$^1$}

\def\autkol{И.\,М.~Зацман}

\titel{\tit}{\aut}{\autkol}{\titkol}

{\renewcommand{\thefootnote}{\fnsymbol{footnote}} \footnotetext[1]
{Работа выполнена при поддержке РФФИ 
(проекты 14-07-00785, 13-06-00403) и РГНФ (проект 15-04-00507).}


\renewcommand{\thefootnote}{\arabic{footnote}}
\footnotetext[1]{Институт проблем информатики Федерального исследовательского
центра <<Информатика и~управление>> Российской академии наук,
iz\_ipi@a170.ipi.ac.ru}

 
   \Abst{Представлены результаты разработки семиотических оснований для создания 
моделей процессов целенаправленной генерации и~развития новых экспертных знаний 
и~разработки технологий, обеспечивающих эти процессы. Необходимость разработки таких 
технологий проявляется наиболее наглядно в~ситуациях, когда имеющиеся системы 
экспертных знаний не удовлетворяют новым социально или технологически значимым 
целям, отражающим новые или изменившиеся потребности общества. В~статье речь идет 
не о~хорошо известных в~области искусственного интеллекта методах и~моделях 
представления знаний, процессах управления формами их представления, а~о~разработке 
новых моделей процессов целенаправленной генерации знаний, отражающих динамику их 
формирования. Рассматриваемый подход к~моделированию этих процессов и~разработке 
обеспечивающих их технологий ориентирован на те прикладные области, где знания 
генерируются экспертами в~процессе анализа текстов или других объектов интерпретации, 
которые могут изменяться во времени, с~последующим пред\-став\-ле\-ни\-ем экспертных знаний 
в~надкорпусных базах данных (НБД). Отличительная черта предлагаемого подхода 
к~моделированию заключается в~явном описании отношений между новыми экспертными 
знаниями и~теми объектами интерпретации, на основе анализа которых были 
сгенерированы элементы новых знаний. Другая отличительная черта заключается в~явном 
описании изменяемых во времени элементов знаний, соответствующих объектам 
интерпретации. Реализуемость такого подхода демонстрируется на примере 
экспериментальной информационной технологии, которая поддерживает 
целенаправленную генерацию экспертами кросс-язы\-ко\-вых знаний о~переводах глагольных 
конструкций русского языка на французский. Эти кросс-язы\-ко\-вые знания формируются 
в~процессе анализа параллельных текстов на русском и~французском языках, пары 
выровненных предложений которых являются объектами интерпретации.}
   
   \KW{кросс-языковые экспертные знания; компьютерное моделирование; генерация 
знаний; объекты интерпретации; семиотические основания; модели процессов генерации 
знаний; надкорпусные базы данных}

\DOI{10.14357/19922264150311 }


\vspace*{-6pt}

\vskip 12pt plus 9pt minus 6pt

\thispagestyle{headings}

\begin{multicols}{2}

\label{st\stat}

\section{Введение}

      Модели процессов генерации и развития новых знаний (далее~--- модели генерации) 
стали активно разрабатываться в~последнем десятилетии прошлого века. Наиболее 
известную модель генерации, названную автором спиральной, предложил Икуджиро 
Нонака~[1, 2]. В~процессе ее построения Нонака рассматривал личностные знания человека 
и~коллективные (согласованные) знания группы людей, которые были разделены на 
выражаемые (explicit) и~невыражаемые знания (tacit). Таким образом, спиральная 
модель генерации включает в~рассмотрение следующие четыре понятия и~соответствующие 
им четыре множества знаний (рис.~1):
      \begin{enumerate}[(1)]
\item личностные невыражаемые знания (individual tacit knowledge);\\[-15pt]
\item коллективные невыражаемые знания (group tacit knowledge);\\[-15pt]
\item личностные выражаемые знания (individual explicit knowledge);\\[-15pt]
\item коллективные выражаемые знания (group explicit knowledge).
\end{enumerate}

\begin{figure*} %fig1
       \vspace*{1pt}
 \begin{center}
 \mbox{%
 \epsfxsize=87.663mm 
 \epsfbox{zac-1.eps}
 }
 \end{center}
 \vspace*{-12pt}
\Caption{Спиральная модель генерации знаний Икуджиро Нонака~\cite[с.~69]{3-zat}}
\vspace*{-4pt}
\end{figure*}


      Наряду с~этими четырьмя понятиями были определены следующие четыре вида 
процессов:
      \begin{enumerate}[(1)]
\item социализация личностных невыражаемых знаний;\\[-15pt]
\item экстернализация коллективных невыража\-емых знаний;\\[-15pt]
\item синтез личностных выража\-емых знаний;
\item интернализация личностных выражаемых знаний.
\end{enumerate}



      Используя эти четыре процесса, Нонака ввел метафорическое понятие спирали 
генерации знаний, каждый виток которой включает следующую последовательность:  
со\-циа\-ли\-за\-ция\;$\to$\;экстер\-на\-ли-\linebreak за\-ция\;$\to$\;син\-тез\;$\to$\;ин\-тер\-на\-ли\-за\-ция\;$\to$\;со\-циа\-ли\-за\-ция 
(как начало следующего витка спирали). 
Было показано на примерах, что эта спираль может служить качественной моделью 
итерационного процесса генерации новых знаний во время <<мозгового штурма>>.
      
      Обобщение и существенное развитие спиральной модели генерации знаний было 
предложено в~работах Йошитеру Накамори и~Анджея Вежбицкого в~рамках создаваемой 
ими научной дисциплины, которую они называют <<Наука о знаниях>>~[3--8]. В~этих 
работах знания разделены на личностные знания человека, коллективные 
и~конвенциональные знания. Это деление они называют социальным аспектом, или 
измерением, так как в~результате обобщения было определено три уровня социализации 
знаний (от первого личностного уровня до третьего конвенционального). С~учетом деления 
на выражаемые и~невыражаемые знания в~результате обобщения были определены еще два 
новых множества знаний, которых нет в~спиральной модели: конвенциональные 
невыражаемые и~конвенциональные выражаемые знания.
      
      Вежбицкий и Накамори определили систему\linebreak
       отношений между множествами 
знаний. Свою модель, включающую шесть множеств знаний и~сис\-те\-му отношений между 
ними, они в~совокупности назвали \textit{креативным пространством} (далее в~\mbox{статье} 
термины <<креативное пространство>> и~<<модель Веж\-биц\-ко\-го--На\-ка\-мо\-ри>> 
будут использоваться как синонимы). Кроме шести множеств знаний ими были также 
определены множества эмоций (личностные, коллективные и конвенциональные), которые 
в статье не рассматриваются. 

В~предлагаемых далее в~статье моделях не используются три 
множества невыражаемых знаний человека, которые по определению непосредст\-венно не 
поддаются экспликации. Однако в~процессе компьютерного моделирования эти три 
множества могут использоваться опосредованно.\linebreak Например, в~технологии генерации новых 
кросс-язы\-ко\-вых знаний, рас\-смат\-ри\-ва\-емой далее в~статье, анализируются объекты 
интерпретации, которые сформированы в~процессе перевода текстов на русском языке 
и~являются результатом применения переводчиком одновременно как конвенциональных 
знаний, так и его личностных невыражаемых знаний\footnote{В статье по смысловому 
содержанию разделяются понятия невыражаемых (tacit) и подразумеваемых (implicit) знаний. Невыражаемые 
и незакрепленные в~знаковой форме знания используются переводчиком часто неявно и могут быть известны 
только ему, т.\,е.\ такие знания могут являться личностными. Подразумеваемые знания косвенно выражаются 
в знаковой форме. Например, фраза <<фирма закрыла отдел разработки прикладных программ>> 
подразумевает, что раньше в~этой фирме существовал отдел прикладного программирования. Такие знания 
могут быть коллективными или конвенциональными.}. Этот анализ является примером извле\-чения и 
опосредованного использования послед-\linebreak них.

\begin{figure*} %fig2
       \vspace*{1pt}
 \begin{center}
 \mbox{%
 \epsfxsize=160.234mm
 \epsfbox{zac-2.eps}
 }
 \end{center}
 \vspace*{-9pt}
\Caption{Система терминов для описания объектов трех сред предметной области информатики и 
интерфейсов между ними~\cite{10-zat, 12-zat} (\textit{денотат по определению является компонентом, 
отношением или свойством объекта интерпретации}$^1$)}
\end{figure*}


      
      
      Отметим, что в~модели Веж\-биц\-ко\-го--На\-ка\-мо\-ри и спиральной модели, 
которые относятся к~категории качественных, нет явно определенной оси времени. Это не 
дает возможности фиксировать моменты времени генерации каждого нового структурного 
элемента экспертных знаний. Кроме того, в~этих моделях не рассматриваются объекты 
интерпретации, служащие источниками новых знаний.

\pagebreak

\renewcommand{\thefootnote}{\arabic{footnote}}
\footnotetext[1]{Отличие денотата от объекта интерпретации будет описано далее 
в~примере генерации  кросс-язы\-ко\-вых знаний.}


      Основная цель статьи заключается в~описании разработанных семиотических 
оснований для создания количественных моделей процессов целенаправленной генерации 
и~развития экспертных знаний, а также для разработки обеспечивающих их технологий и~баз 
данных. С~использованием этих оснований в~статье предлагается развитие модели 
Веж\-биц\-ко\-го--На\-ка\-мо\-ри в~следующих двух на\-прав\-ле\-ниях:
      \begin{enumerate}[(1)]
\item вводится ось времени, на которой фиксируют\-ся дискретные моменты времени, 
в~которые по\-рож\-да\-ют\-ся новые элементы вы\-ра\-жа\-емых экспертных знаний, а~также 
фиксируются моменты времени их изменения;
\item в~явном виде специфицируется связь каждого нового элемента знаний с~тем 
объектом интерпретации, в~результате анализа которого формируется этот элемент 
и~дается формализованное его описание.
\end{enumerate}




\section{Семиотические основания}

      В процессе развития модели Веж\-биц\-ко\-го--На\-ка\-мо\-ри использовалась ранее 
разработанная система терминов~\cite{9-zat, 10-zat}. Она включает, в~част\-ности,\linebreak
термины и~дефиниции для таких понятий, как\linebreak
 кон\-цеп\-ты знаний человека (личностные, коллективные, 
конвенциональные), семантическая информация, сен\-сор\-но вос\-при\-ни\-ма\-емые данные, 
циф\-ро\-вая информация, циф\-ро\-вые данные, циф\-ро\-вые коды нескольких категорий и~ряд 
других терминов, а~также задает их распределение по трем\linebreak средам предметной области 
информатики (ментальной, со\-ци\-аль\-но-ком\-му\-ни\-ка\-ци\-он\-ной и цифровой) 
и~описание отношений между этими терминами. Первая отличительная черта этой системы\linebreak 
терминов заключается в~том, что значения терминов <<знания>>, <<семантическая 
информация>>, <<сен\-сор\-но вос\-при\-ни\-ма\-емые данные>>, <<цифровая инфор\-мация>> 
и~<<цифровые данные>> четко разграничены и они по определению не пересекаются по их 
смысловому содержанию (рис.~2). Вторая ее отличительная черта состоит в~том, что она 
является <<масштабируемой>> по числу сред, так как определен принцип добавления 
новых сред в~предметную область информатики как ин\-фор\-ма\-ци\-он\-но-ком\-пьютерной науки, 
а~также соответствующего расширения системы терминов за счет именования объектов 
каждой новой среды и их интерфейсов с~объектами уже существующих сред. Этот принцип 
получил название аксиомы герметичности сред предметной области 
информатики~\cite{12-zat, 11-zat}.
      
      Основная идея развития модели Веж\-биц\-ко\-го--На\-ка\-мо\-ри заключается 
      в~установлении связи каж\-до\-го нового концепта с~объектом интерпретации. При этом анализ 
объектов интерпретации выполняется в~соответствии с~явно определенными целями 
формирования новых знаний, так как статья посвящена именно целенаправленным 
процессам формирования новых выражаемых знаний на основе извлечения и экспликации 
невыражаемых.





      Развитие модели Веж\-биц\-ко\-го--На\-ка\-мо\-ри было выполнено в~несколько 
этапов. Сначала на первом этапе в~модель были добавлены следующие шесть множеств:
      \begin{enumerate}[(1)]
\item формы представления личностных концептов (структурированные тексты, 
изображения или другие виды семантической информации);
\item формы представления коллективных концептов;
\item формы представления конвенциональных концептов;
\item цифровые коды личностных концептов и~форм их представления;
\item цифровые коды коллективных концептов и~форм их представления;
\item цифровые коды конвенциональных концептов и~форм их представления.
\end{enumerate}

      В последних трех множествах разделяются коды концептов и~их имен (как частного случая
      форм представления концептов), например 
коды для пред\-став\-ле\-ния смыслового содержания слов и~последовательностей их литер 
в~цифровой среде относятся к~разным категориям (см.\ рис.~2).
      
      Затем для количественного описания динамики социализации, определяемой 
процессами согласования личностных концептов в~группе экспертов, была введена ось 
с~числами от нуля до бесконечности. Единица на оси социализации обозначает\linebreak
 личностный 
концепт, $N\hm>1$~--- коллективный,\linebreak который согласован группой из~$N$~экспертов, 
а~бесконечность~--- конвенциональный концепт. Необходимость в~нуле на этой оси 
возникла в~процессе проведения эксперимента для обозначения тех <<бывших>> 
личностных концептов, от которых со временем отказались их авторы. Ось социализации 
позволяет кодировать степень согласованности между экспертами результатов анализа 
динамически изменяемых объектов интерпретации и изменение степени согласованности во 
времени.
      
      Таким образом, на первом этапе развития модели Веж\-биц\-ко\-го--На\-ка\-мо\-ри 
были определены шесть новых множеств знаний и введена ось социализации. Эта ось 
служит для обозначения уровня социализации не только выражаемых знаний, но также 
форм их представления и их цифровых кодов. Отметим, что система отношений между 
множествами знаний, определенная Вежбицким и Накамори, не охватывает шесть новых 
множеств знаний. Поэтому далее потребуется доопределить или построить новую систему 
отношений.
      
      На втором этапе была добавлена ось времени, на которой фиксируются моменты 
порождения новых личностных, коллективных и конвенциональных концептов, а~также 
моменты времени их изменений. Следствием этого этапа является то, что появляется 
возможность фиксировать на оси времени не только моменты порождения и изменения 
концептов, но также форм их представления и~их цифровых кодов.
      
      Третий этап заключается в~добавлении множества объектов интерпретации, каждый 
из которых имеет уникальный цифровой код. Следствием этого этапа является то, что 
появляется потенциальная возможность фиксировать на оси времени не только моменты 
порождения и изменения концептов, форм их представления и их цифровых кодов, но 
также динамику изменения соответствующих им объектов интерпретации.
      
      Результаты трех перечисленных этапов развития модели 
Веж\-биц\-ко\-го--На\-ка\-мо\-ри позволяют фиксировать на оси времени:
      \begin{itemize}
\item моменты изменения экспертами объектов интерпретации, являющихся 
источниками новых знаний;\\[-14pt]
\item моменты порождения новых концептов в~процессе интерпретации объектов 
и интроспекции результатов интерпретации;\\[-14pt]
\item моменты изменения формируемых концептов, форм их представления (их 
имен) и их цифровых кодов.
\end{itemize}

      Определим новую систему отношений между объектами интерпретации, концептами 
как эле\-мен\-тами множеств знаний, их именами и~цифро\-вы\-ми кодами, являющимися их 
идентифика\-торами. Она необходима потому, что система отношений в~модели  
Веж\-биц\-ко\-го--На\-ка\-мо\-ри не охватывает шесть новых множеств знаний. Для 
описания отношений между объектами интерпретации, концептами и~именами предлагается 
использовать треугольник Фреге~[13--15], что представляет собой 
\textit{первое семиотическое основание} моделирования процессов генерации знаний. 
Семиотический треугольник Фреге по определению связывает сам объект интерпретации 
(точнее, некоторый денотат, определяемый в~процессе анализа объекта интерпретации; как 
правило, это его компонент, отношение или свойство), понимание денотата (его смысловое 
содержание), т.\,е.\ его концепт, а также некоторое имя как текстовую, или невербальную, 
форму обозначения денотата и~его концепта. Для циф\-ро\-вого кодирования денотатов, 
концептов и~имен предлагается использовать циф\-ро\-вой семиотический треугольник, что 
представляет собой \textit{второе семиотическое основание} моделирования процессов 
генерации знаний. Его определение дано в~работе~\cite{16-zat}, в~которой рис.~6 
иллюстрирует взаимосвязи этих двух треугольников.
   
   Основная идея цифрового семиотического треугольника заключается в~том, что для 
каждой из трех вершин треугольника Фреге используется своя категория цифровых кодов, 
в~том числе и~для концептов. Предлагаемое введение отдельной кодировки для концептов 
дает возможность строить взаимно однозначные отношения между концептами 
и~циф\-ро\-вы\-ми кодами, являющимися их идентификаторами. Построение таких отношений 
является основой компьютерного моделирования процессов генерации знаний. Важно 
отметить, что циф\-ро\-вой семиотический треугольник дает возможность строить взаимно 
однозначные отношения, но остав\-ля\-ет открытым вопрос о~конкретных методах 
приписывания цифровых кодов концептам. В~ряде задач компьютерного моделирования 
процессов генерации знаний существует свобода выбора метода назначения кодов. Однако 
в~задачах оценивания релевантности новых знаний явно определенным целям их генерации 
этот метод может быть во многом обусловлен заданными целями генерации. Примеры 
таких целей рассматриваются далее в~статье.
      
      Перечислим кратко основные результаты, которые были получены в~процессе 
развития модели Веж\-биц\-ко\-го--На\-ка\-мо\-ри. С~помощью цифр~1 и~2 в~списке 
отмечены положения, взятые из модели Веж\-биц\-ко\-го--На\-ка\-мо\-ри~(1) и~полученные 
в результате развития этой модели~(2):
      \begin{itemize}
\item три множества выражаемых знаний (личностные, коллективные и~
конвенциональные)~(1), которые состоят из концептов соответству\-ющих 
категорий~(2);
\item три множества имен (форм представления концептов)~(2);
\item множество объектов интерпретации~(2);
\item три множества цифровых кодов концептов~(2) (личностные, коллективные 
и конвенциональные);
\item три множества цифровых кодов форм пред\-став\-ле\-ния концептов~(2);
\item множество цифровых кодов, построенное на основе уникальных 
идентификаторов объектов интерпретации~(2);
\item ось времени для отражения динамики процессов генерации знаний~(2);
\item ось социализации выражаемых экспертных знаний (1 и~2, так как в~модели 
Веж\-биц\-ко\-го--На\-ка\-мо\-ри нет детализации коллективных знаний в~
зависимости от числа экспертов в~группе, а в~результате развития этой модели 
добавлена их детализация);
\item система отношений между объектами интерпретации, денотатами, 
концептами вы\-ра\-жа\-емых знаний, именами и~их цифровыми кодами (1 и~2, так 
как система отношений задана в~модели Веж\-биц\-ко\-го--На\-ка\-мо\-ри только 
между знаниями с~учетом трех уровней их социализации; в~результате развития 
модели система отношений дополнена треугольником Фреге и~цифровым 
семиотическим треугольником).
\end{itemize}

      Для описания еще одного, третьего, семиотического основания необходимо 
вернуться к~определению классического треугольника Фреге. В~семиотике он определяется 
как треугольник с~тремя вершинами (денотат, концепт как идеальная вершина 
треугольника, имя как форма представления концепта), находящимися в~отношениях 
устойчивой связи, опосредованной сознанием, представляет собой устойчивое единство, 
которое посредством сенсорно воспринимаемой формы \textit{конвенционально} 
репрезентирует концепт и~денотат. Строго говоря, в~новой системе отношений 
классический треугольник Фреге применим только для случая конвенциональной 
репрезентации концепта сенсорно воспринимаемой формой. Следовательно, дополнительно 
необходим некоторый способ для построения личностного и~коллективного семиотических 
треугольников Фреге, аналогичных классическому треугольнику Фреге и~применимых в~
случае генерации личностных и~коллективных концептов. При этом необходимо учитывать 
то обстоятельство, что личностные и~коллективные концепты понимают только их авторы, 
так как для них отсутствует конвенциональная репрезентация их формой. Следовательно, 
для экспертов, участвующих в~процессе формирования новых знаний, но не являющихся 
авторами генерируемых новых концептов, они будут недоступны, если их авторы не 
эксплицируют свою личностную или коллективную репрезентацию в~форме, доступной 
другим экспертам тем или иным способом.
      
      Отметим, что в~процессе личностной репрезентации кроме объекта интерпретации, 
денотата, концепта и~имени <<задействовано>> персональное авторское сознание. 
В~приведенном определении классического треугольника Фреге говорится об устойчивой 
связи, \textit{опосредованной сознанием}, но ничего не говорится о том, как и~где (в какой 
среде или средах) эта связь закреплена материально. Остановимся кратко на истории этого 
вопроса, чтобы затем предложить способ экспликации личностной репрезентации концепта 
в форме, доступной другим экспертам.

\begin{figure*} %fig3
       \vspace*{1pt}
 \begin{center}
 \mbox{%
 \epsfxsize=138.786mm
 \epsfbox{zac-3.eps}
 }
 \end{center}
 \vspace*{-9pt}
\Caption{Семиотический нейротетраэдр и~нейроквадрат~\cite{12-zat}}
\end{figure*}
      
      В 1988~г.\ была сформирована рабочая группа <<FRamework of Information System 
COncepts~--- FRISCO>> в~рамках Международной федерации по обработке информации 
(International Federation for Information Processing~--- IFIP). Основной целью этой группы 
было создание системы определений для базовых терминов, которую затем можно было бы 
предложить использовать как терминологическую основу разработки и~описания 
информационных систем. Итоги ее работы опубликованы в~виде отчета 
      в~1998~г.~\cite{17-zat}. В~результате работы группы FRISCO, в~частности, было 
определено понятие семиотического тетраэдра с~вершинами <<объект, концепт, имя 
объекта и~интерпретатор>>. Отметим, что в~определении этого понятия есть субъект, 
который \textit{интерпретирует} объект, генерирует концепт и~имя объекта, а также 
\textit{устанавливает} связь между ними~\cite{18-zat}. Идея 
      субъ\-ек\-та-ин\-тер\-пре\-та\-то\-ра или интерпретанта\footnote{Интерпретатор~--- это 
только человек, который анализирует предмет, генерирует концепт и~его имя, а интерпретант~--- это не 
обязательно человек. В~общей семиотике функция абстрактного интерпретанта заключается в~интерпретации 
предметов, генерации концептов и~присвоении им имен~\cite[с.~15, 16]{19-zat}.}, который \textit{является 
носителем} этой связи, зародилась еще раньше в~общей семиотике и~в~рамках этой 
научной дисциплины оказалась весьма продуктивной~\cite{19-zat, 20-zat}.
      
      Однако с~точки зрения разработки и~описания информационных систем и~технологий 
включение субъектов в~дефиниции терминов вместо объектов обладает рядом недостатков. 
Основной из них заключа\-ется в~том, что в~задачах когнитивной информатики 
и~нейроинформатики, например при разработке нейрокоммуникаторов, необходимо 
различать объекты \textit{ментальной}, \textit{нейрофизиологической}, 
со\-ци\-аль\-но-ком\-му\-ни\-ка\-ци\-он\-ной и~цифровой сред. В~традиционных моделях 
с~субъек\-та\-ми-ин\-тер\-пре\-та\-то\-ра\-ми и~интерпретантами это различие между объектами трудно 
провести, так как в~них ментальная среда и~нейрофизиологическая среда (далее~--- 
нейросреда), а~также их объекты, как правило, не различаются. Поэтому 
      в~работе~\cite{12-zat} эти две среды было предложено рассматривать в~информатике 
раздельно, а систему терминов дополнить следующими понятиями (рис.~3):
      \begin{itemize}
\item <<нейроинформация>>, в~частности фик\-си\-ру\-ющая постоянные или 
временные связи между объектом интерпретации, концептом (личностным, 
коллективным или конвенциональным) и~именем объекта;
\item <<нейросемиотический тетраэдр>> с~вершинами объект, концепт, имя 
и~нейроинформация;
\item <<нейроквадрат>> как четверка кодов следующих категорий:
\begin{description}
\item[\,] К1~--- для концептов ментальных знаний человека;
\item[\,] К2~--- для слов как имен объектов и~других знаковых форм для 
обозначения ментальных знаний;
\item[\,] К3~--- для кодирования предметов матери\-альной сферы (в~общем 
случае~--- для кодирования любых объектов, в~результате семантиче\-ской 
интерпретации которых человеком определяются денотаты и~генерируются 
концепты);
\item[\,] К4~--- для кодирования в~цифровой среде нейроинформации,  
с~по\-мощью которой фиксируются связи между объектом интерпретации, 
концептом и~именем;
\end{description}
\item <<нейрокод>> как аналог компьютерных таблиц кодировки символов для 
нейроинформации.
\end{itemize}

      Определение нейросемиотического тетраэдра по своему содержанию во многом 
совпадает с~терми\-ном <<психосемиотический тетраэдр>> Ф.\,Е.~Ва\-силюка с~вершинами 
<<предмет, личностный\linebreak концепт, имя предмета и~чувственная ткань (которая
свя\-зы\-ва\-ет 
воедино первые три вершины)>>~\cite{21-zat}. Различие этих двух тетраэдров заключается 
в том, что первый описывает любые кон\-цеп\-ты (личностный, коллективный 
и~конвенциональный), а~второй был определен Ф.\,Е.~Василюком только для личностных 
концептов. В~нейросемиотическом тет\-ра\-эд\-ре три из четырех его вершин являются 
сущностями трех разных сред: ментальной, со\-ци\-аль\-но-ком\-му\-ни\-ка\-ци\-он\-ной и~нейросреды, к~
которой принадлежит нейроинформация. Таким образом, в~определении 
нейросемиотического тетраэдра использовался альтернативный подход по сравнению 
с~традиционной идеей субъ\-ек\-та-ин\-тер\-пре\-та\-то\-ра или интерпретанта.
      
      Сопоставление рис.~3 и~рис.~6 из работы~\cite{16-zat}, в~которой даны определения 
понятий <<формокод>> и~<<семокод>>, наглядно иллюстрирует то, что нейроквадрат 
является обобщением цифрового се-\linebreak миотического треугольника. Итак, нейротетраэдр 
и~нейро\-квадрат являются \textit{третьим семиотическим основанием} моделирования 
процессов генерации знаний.



\section{Семиотические модели}

      В этом разделе три семиотических основания, рассмотренные в~предыдущем 
разделе, будут использованы в~процессе построения двух моделей генерации знаний: 
модели фиксированного состояния и~нестационарной модели. Они являются результатом 
обобщения двух ранее разработанных моделей, основанных на треугольнике Фреге 
и~циф\-ро\-вом семиотическом треугольнике~\cite{16-zat, 22-zat}.
      
      Кроме трех семиотических оснований исходными данными для обобщения является 
следующее описание этих двух моделей, разработанных для итерационных процессов 
генерации знаний экспертами, которые используют информационную систему для 
фиксации эволюции объектов интерпретации, определения денотатов, сгенерированных 
ими концептов, описания которых эксперты формируют в~процессе интроспекции, имен 
концептов и~денотатов. По определению из работы~\cite{16-zat} первая модель фиксирует 
\textit{состояние процесса генерации в~момент времени}, соответствующий окончанию 
некоторой итерации этого процесса, и~состоит из:
      \begin{itemize}
\item трех сред предметной области информатики: ментальной,  
со\-ци\-аль\-но-ком\-му\-ни\-ка\-ци\-он\-ной и~цифровой;
\item треугольника Фреге, включающего объект интерпретации, концепт, 
сгенерированный или измененный на этой итерации некоторым экспертом, и~имя;
\item цифрового семиотического треугольника, включающего коды объекта 
интерпретации, концепта и~имени, сгенерированные в~этот же момент времени 
информационной системой, обеспечивающей работу экспертов.
\end{itemize}

      Согласно первой модели три вершины треугольника Фреге кодируются тремя 
цифровыми кодами разных категорий, которые генерируются в~конце каждой итерации:
      \begin{itemize}
\item семантическим кодом концепта (К1);
\item информационным кодом его имени, если оно создано экспертом 
(в~противном случае этот код равен нулю) (К2);
\item объектным кодом объекта интерпретации (К3).
\end{itemize}

      Таким образом, после завершения каждой итерации в~информационной системе по 
некоторому заданному алгоритму генерируются три цифровых\linebreak кода разных категорий. Если 
на некоторой итерации принял участие один эксперт, то создается\linebreak только одна запись 
с~результатами личностной семантической интерпретации рассматриваемого объекта и~три 
кода, а если несколько экспертов, то для каж\-до\-го из них~--- одна запись и~три кода.
      
      В процессе целенаправленной генерации знаний существует отдельный вид 
итераций для согласования личностных концептов и~имен, которые могут меняться 
в~пределах итераций, но не между ними. На этих итерациях эксперты ставят своей целью 
согласовать между собой свои личностные интерпретации и~сформировать коллективные 
концепты и~имена. Если это удается сделать, то в~информационной системе создается еще 
одна запись с~результатами коллективной семантической интерпретации и~еще три кода 
с~указанием идентификаторов всех экспертов, которые приняли учас\-тие в~процессе 
согласования и~выработали единую позицию. В~общем случае генерация и~согласование 
концептов могут быть совмещены в~рамках одной комплексной итерации. Отметим, что 
между двумя любыми итерациями объекты интерпретации могут меняться экспертами, но 
не в~пределах итераций. Регистрация в~информационной системе изменений объектов 
интерпретации между итерациями, описаний новых концептов и~результатов их 
согласования дает возможность восстановить ретроспективно все этапы процесса генерации 
знаний.
      
      Вторая модель предназначена для описания динамики процесса 
      генерации~\cite{22-zat}. Концепты, име-\linebreak на и~объекты интерпретации могут 
изменяться\linebreak в~широком диапазоне в~процессе согласования экспертами их личностных 
концептов и~имен. По определению вторая модель описывает динамику процесса генерации 
концептов одним экспертом и~состоит из:
      \begin{itemize}
\item трех сред предметной области информатики: ментальной,  
со\-ци\-аль\-но-ком\-му\-ни\-ка\-ци\-он\-ной и~цифровой;
\item треугольников Фреге, построенных экспертом в~моменты времени 
окончания итераций~$t_i$, $i \hm= 1, 2,\ldots$;
\item цифровых семиотических треугольников, построенных информационной 
системой в~эти же моменты времени~$t_i$.
\end{itemize}

      На основе этой модели динамики процесса генерации концептов 
      в~работе~\cite{22-zat} было дано определение пространства Фреге как 4-мерного 
множества точек для трех кодов разных категорий $\{t_i$, семантический код ($t_i$), 
информационный код ($t_i$), объектный код ($t_i$) при $i \hm= 1, 2,\ldots\}$, 
сгенерированных информационной системой в~процессе работы одного эксперта. 
Аналогичную модель и~соответствующее пространство Фреге можно определить для случая 
генерации знаний коллективом экспертов, пример которого рассматривается 
      в~работе~\cite{23-zat}.
      
      Пространство Фреге имеет три оси координат цифровых кодов: семантическую, 
информационную и~объектную, а также четвертую~--- ось времени, содержащую 
дискретный набор точек начала и~окончания итераций генерации концептов,\linebreak их 
согласования или комплексных итераций. Пространст\-во Фреге дает возможность 
представить графически динамику процесса, используя последовательности значений 
семантических, информационных и~объектных кодов, сгенерированные информационной 
системой в~дискретные моменты времени окончания итераций.
      
      Отметим, что вид дискретных траекторий точек будет определяться алгоритмами 
назначения семантических, информационных и~объектных кодов. Можно ли задать 
некоторую метрику в~пространстве Фреге? В~настоящее время этот вопрос остается 
открытым. Пока этот вопрос находится в~стадии изучения, термин <<пространство Фреге>> 
определен только как семиотическое понятие, но не математическое. При этом если в~
информационной системе фиксируется содержательная эволюция объектов интерпретации, 
сгенерированных концептов, описания которых эксперты формируют в~процессе 
интроспекции, и~имен, то пространство Фреге служит для количественного описания 
процесса генерации концептов.
      
      Прежде чем приступить к~построению модели фиксированного состояния 
      и~нестационар-\linebreak ной модели на основе обобщения двух ранее 
      раз\-работанных моделей, отметим, 
что в~последних использова\-лись только три среды предметной об\-ласти информати\-ки 
(ментальная, со\-ци\-аль\-но-ком\-му\-ни\-ка\-ци\-он\-ная и~цифровая среды) и~первые два 
семиотических основания для моделирования процессов генерации знаний (треугольник 
Фреге\linebreak и~цифровой семиотический треугольник). Пере\-чис\-лим исходные данные построения 
модели фиксированного состояния и~нестационарной модели на основе обобщения двух 
ранее разработанных моделей:
      \begin{itemize}
\item определение семиотического нейротетраэдра и~нейроквадрата, которые 
служат третьим семиотическим основанием;
\item первая исходная модель, которая фиксирует состояние процесса генерации 
концептов в~момент времени, соответствующий окончанию некоторой итерации 
этого процесса;
\item вторая исходная модель, которая описывает динамику процесса генерации 
концептов.
\end{itemize}

      Для обобщения перечисленных двух моделей рассмотрим четыре среды предметной 
области информатики как ин\-фор\-ма\-ци\-он\-но-компью\-тер\-ной науки: ментальную, 
со\-ци\-аль\-но-ком\-му\-ни\-ка\-ци\-он\-ную, нейро- и~цифровую среды~\cite{12-zat}. 
С~формальной 
точки зрения это обобщение представляет собой замену треугольника Фреге на 
семиотический тет\-раэдр, а цифрового семиотического треугольника~--- на нейроквадрат кодов 
(см.\ рис.~3). Сделав такую замену, получаем два следующих обобщения.

      \begin{figure*}[b] %fig4
\vspace*{9pt}
      \begin{center}
      {\small
      \begin{tabular}{|c|p{65mm}|p{65mm}|}
      \hline
\multicolumn{1}{|c|}{Номер пары}&
\multicolumn{1}{c|}{Оригинальный текст}&
\multicolumn{1}{c|}{Перевод}\\
\hline
\hphantom{9}9&Цвет лица у Ильи Ильича не был ни румяный, ни смуглый, ни положительно бледный, а 
безразличный или казался таким, может быть, потому, что Обломов как-то обрюзг не по 
летам: от недостатка ли движения или воздуха, а может быть, того и~другого.&Le teint d'Ilia 
Ilitch n'$\acute{\mbox{e}}$tait ni rose, ni h$\hat{\mbox{a}}$l$\acute{\mbox{e}}$, 
ni carr$\acute{\mbox{e}}$ment p$\hat{\mbox{a}}$le, mais indiff$\acute{\mbox{e}}$rent ou, du 
moins, il le paraissait. Peut-$\hat{\mbox{e}}$tre parce que la chair d'Oblomov 
$\acute{\mbox{e}}$tait pr$\acute{\mbox{e}}$matur$\acute{\mbox{e}}$ment flasque: faute 
d'exercice ou manque d'air, peut-$\hat{\mbox{e}}$tre l'un et l'autre.\\
\hline
18&Халат имел в~глазах Обломова тьму неоцененных достоинств: он мягок, гибок; тело не 
чувствует его на себе; он, как послушный раб, покоряется самомалейшему движению 
тела.&Aux yeux d'Oblomov cette robe de chambre avait une foule de qualit$\acute{\mbox{e}}$s 
inappr$\acute{\mbox{e}}$ciables: elle $\acute{\mbox{e}}$tait douce, souple, ne pesait pas sur le 
corps; telle une esclave docile, elle se pliait au moindre mouvement.\\
\hline
21&Лежанье у Ильи Ильича не было ни необходимостью, \textbf{как} у~больного или как 
\textbf{у~человека, который хочет спать}, ни случайностью, как у~того, кто устал, ни 
наслаждением, как у лентяя: это было его нормальным состоянием.&La position 
allong$\acute{\mbox{e}}$e n'$\acute{\mbox{e}}$tait pour Ilia Ilitch ni 
n$\acute{\mbox{e}}$cessaire, \textbf{comme} pour un malade ou \textbf{pour un homme qui 
veut dormir}, ni accidentelle, comme pour une personne fatigu$\acute{\mbox{e}}$e, ni 
voluptueuse comme chez le fain$\acute{\mbox{e}}$ant; c'$\acute{\mbox{e}}$tait son 
$\acute{\mbox{e}}$tat normal.\\
\hline
\end{tabular}
}
\end{center}

\vspace*{-3pt}

\Caption{Три предложения параллельных текстов на русском языке и~их переводы ({полужирным 
шрифтом выделен контекст, используемый далее на рис.}~6)}
\end{figure*}
      
      Для случая четырех сред модель фиксированного состояния процесса генерации 
знаний со\-сто\-ит~из:
      \begin{itemize}
\item ментальной, со\-ци\-аль\-но-ком\-му\-ни\-ка\-ци\-он\-ной, цифровой сред и~
нейросреды;
\item семиотического тетраэдра, включающего объект интерпретации, концепт, 
сгенерированный или измененный на этой итерации некоторым экспертом, имя 
объекта интерпретации, которое одновременно является и~именем концепта 
в~этот момент времени, а также нейроинформацию о связях между объектом, 
концептом и~их именем;
\item нейроквадрата, включающего цифровые коды объекта, концепта, имени 
и~связывающей их нейроинформации, сгенерированные в~этот же момент времени 
информационной системой, обеспечивающей работу экспертов.
\end{itemize}

      Нестационарная модель динамики процесса генерации знаний одним экспертом 
состоит из:
      \begin{itemize}
\item тех же самых четырех сред предметной области информатики;
\item семиотических тетраэдров, построенных экспертом в~моменты времени 
окончания итераций~$t_i$, $i \hm= 1, 2,\ldots$;
\item нейроквадратов, построенных в~эти же моменты времени~$t_i$.
\end{itemize}

      На основе этой модели динамики процесса аналогично можно определить 
обобщенное пространство Фреге как 5-мер\-ное множество точек для четырех кодов разных 
категорий $\{t_i$, семантический код ($t_i$), информационный код ($t_i$), объектный код 
($t_i$), код нейроинформации ($t_i$) при $i \hm= 1, 2,\ldots\}$, сгенерированных 
информационной системой на $i$-й итерации работы эксперта.
      


      Главное содержание приведенного формального обобщения состоит в~замене 
треугольника Фреге на семиотический тетраэдр, основное отличие которого от 
треугольника Фреге заключается в~наличии нейроинформации. По определению она 
фиксирует связи между объектом, концептом и~именем. Однако остается открытым вопрос 
о практических способах получения нейроинформации об этих связях в~процессе решения 
прикладных задач. Раньше, когда такие задачи решались с~использованием моделей, 
которые охватывали объекты только трех сред, объекты интерпретации были доступны 
экспертам для изменений и~анализа, так как они представляли собой изменяемые во 
времени:
      \begin{itemize}
\item компьютерные программы и~данные, используемые для вычисления 
значений новых индикаторов~\cite{9-zat, 23-zat};
\item фрагменты параллельных текстов на русском и~французском языке (рис.~4), 
в результате контрастивного анализа которых определялись денотаты и~
формировались их кросс-язы\-ко\-вые концепты в~процессе анализа 
параллельных фрагментов~\cite{24-zat, 25-zat}.
      \end{itemize}
      
      Новые концепты знаний, принадлежащие ментальной среде, описывались 
экспертами в~результате субъективной интроспекции с~последующим присвоением имен 
сформированным ими концептам. Иначе говоря, в~моделях, которые охватывали объекты 
трех сред, эксперты сами анализировали объекты интерпретации, описывали концепты 
и~давали имена. После расширения числа сред до четырех в~обобщенных моделях появляется 
нейроинформация о связях между объектом интерпретации, концептом и~именем, которая 
экспертам недоступна. Поэтому и~возникает вопрос о~способах получения 
нейроинформации в~процессе решения практических задач.
      
      Сегодня есть возможность отобразить в~компьютерной форме уровень активности 
разных участ\-ков мозга экспертов в~режиме реального времени, используя метод 
функциональной маг\-нит-\linebreak но-ре\-зо\-нанс\-ной томографии (functional Magnetic Resonance 
Imaging~--- fMRI)~\cite{26-zat}. Этот метод позволяет использовать объективные 
индикаторы уровня активности, наблюдая количественные измерения мозговой 
деятельности, одновременно фиксируя и~описывая концепты как результаты личностного 
анализа экспертами объектов интерпретации в~процессе субъективной интроспекции.
      
      Но и~здесь возникают вполне закономерные вопросы. Можно ли использовать 
объективные индикаторы уровня активности в~процессе генерации новых концептов для 
описания связей между объектом интерпретации, концептом и~именем, а также для 
сопоставления с~результатами личностного анализа? Можно ли их использовать для 
описания процесса согласования новых концептов между экспертами?
      
      В настоящее время действительно есть возможность наблюдать одновременно и~
количественные\linebreak данные измерения мозговой деятельности, и~результаты субъективного 
мышления в~процессе субъективной интроспекции, но из первых сегодня трудно получить 
именно ту нейроинформацию, которая соответствует личностным или согласованным 
концептам экспертов, чтобы провести ее сопоставление с~результатами личностного 
субъективного анализа. Кроме того, есть гипотеза и~подтверждающие ее 
экспериментальные данные, что у экспертов часть нейроинформации, соответствующей 
устоявшимся конвенциональным концептам знаний, носит структурный характер, скорее 
всего, на уровне связей между нейронами долговременной памяти, что не фиксируется 
fMRI и~другими современными методами. Стремительное развитие когнитивной 
нейронауки и~ее инструментальных средств позволяет надеяться, что в~будущем станет 
возможным соотнести структурную нейроинформацию и~количественные нейроданные 
измерений мозговой деятельности с~устоявшимися и~новыми концептами экспертных 
знаний~[26--29].

      
      Однако в~настоящее время при разработке информационных технологий и~решении 
практических задач, когда недоступна \textit{объективная нейроинформация} о~связях 
между объектом интерпретации, концептом и~именем, предлагается по-преж\-не\-му 
использовать результаты \textit{субъективной интроспекции}. Иначе говоря, в~процессе 
итерационной генерации новых знаний эксперты в~информационной системе должны 
описывать не только свои кон\-цеп\-ты и~присваивать им имена, но также устанавливать 
и~фиксировать их связи с~объектами интерпретации, определенными ими денотатами 
и~присвоенными именами.
      
      Предлагаемый подход позволяет уже сегодня использовать обобщенные модели при 
разработке информационных технологий, поддерживающих генерацию новых знаний, 
и~решении практических задач. Кроме того, использование цифровой среды информационной 
системы как носителя этих связей обеспечит доступ всех экспертов к~описаниям 
личностных и~коллективных концептов. Другими словами, анализировать и~обсуждать 
описания таких концептов смогут все эксперты, а не только их авторы, если в~процессе 
итерационной генерации новых знаний эксперты в~информационной системе описывают 
свои концепты, устанавливают и~фиксируют их связи с~объектами интерпретации, 
определенными ими денотатами и~присвоенными именами.


\section{Технология, обеспечивающая генерацию знаний}


      Модели фиксированного состояния и~динамики процесса генерации знаний были 
использованы при разработке информационной технологии,\linebreak
 обеспечивающей 
целенаправленную генерацию и~развитие кросс-язы\-ко\-вых знаний коллективом 
экспертов. Необходимость разработки подобной технологии проявляется наиболее 
наглядно в~ситуации, когда необходимо повысить качество машинного перевода и~для этого 
требуется существенное развитие контрастивных грамматик на основе формирования 
новых кросс-язы\-ко\-вых знаний. При этом направления развития контрастивных 
грамматик должны определяться явно эксплицированными целями, достижение которых 
и~должно непосредственно способствовать повышению качества машинного перевода. 

\begin{figure*}[b] %fig5
\vspace*{1pt}
 \begin{center}
 \mbox{%
 \epsfxsize=165.062mm
 \epsfbox{zac-5.eps}
 }
 \end{center}
 \vspace*{-9pt}
\Caption{Основные этапы технологии (нейросреда не показана;  
со\-ци\-аль\-но-ком\-му\-ни\-ка\-ци\-он\-ная среда для краткости обозначена как 
<<Информационная среда>>)}
\end{figure*}

%\begin{multicols}{2}

      
      При таком подходе кроме моделей состояния и~динамики процесса генерации 
знаний необходимо использовать некоторый способ описания целей. Как было уже 
отмечено, рассматрива\-емый подход к~моделированию процесса генерации знаний при 
явном описании целей и~разработке обеспечивающей технологии ориентирован на те 
прикладные области, где генерируемые экспертные знания являются результатом анализа 
объектов\linebreak интерпретации. В~рассматриваемом примере це\-ле\-на\-прав\-лен\-но\-го формирования 
кросс-язы\-ко\-вых знаний объектами интерпретации являются предложения параллельных 
текстов на рус\-ском и~французском языках, а~денотатами~--- пары тех параллельных 
фрагментов, которые выделяются\linebreak экспертами согласно рассматриваемому ими 
направлению развития контрастивной грамматики (выделенные полужирным шрифтом на 
рис.~4 параллельные фрагменты в~паре №\,21 станут далее одним из объектов анализа).

%\pagebreak

%\end{multicols}




      
      
      Отличительная черта предлагаемого подхода к~моделированию заключается в~явном 
описании
 отношений между новыми экспертными знаниями, объектами интерпретации 
и~денотатами, на основе анализа которых могут быть сгенерированы элементы новых знаний (т.\,е.\ не 
каждый анализируемый объект и~определенный в~процессе анализа денотат всегда 
порождают новый концепт). Реализуемость такого подхода была продемонстрирована 
в~процессе выполнения контрастивных исследований, включающих задачи целенаправленной 
генерации кросс-язы\-ко\-вых знаний:
      \begin{itemize}
\item о переводах глагольных конструкций русского языка на французский;
\item о возможных вариантах перевода лингвоспецифичных слов русского языка 
на французский.
\end{itemize}

      При проведении этих контрастивных исследований кросс-язы\-ко\-вые знания 
формировались экспертами в~процессе анализа параллельных текстов на русском 
и~французском языках с~использованием НДБ~\cite{30-zat, 31-zat}. 
Отметим, что переводной текст является результатом применения переводчиком как 
конвенциональных знаний (в~этом случае анализ соответствующих параллельных текстов 
не приводит к~генерации новых концептов), так и~его невыражаемых знаний, что может 
привести к~генерации новых концептов. Невыражаемые знания могут использоваться 
переводчиками неявно, при этом быть новыми и~неописанными в~контрастивных 
грамматиках в~явной (эксплицитной) форме. В~приведенных далее примерах 
рас\-смат\-ри\-ва\-ют\-ся оригинальные тексты на русском языке, при переводе которых на 
французский язык невыражаемые знания использовались переводчиками, что и~нашло свое 
отражение в~результатах перевода. Поэтому результаты сопоставления оригинальных 
текстов на русском языке и~их переводов могут помочь сформировать и~описать новые 
знания.

     
      
      Разработанная технология~\cite{24-zat, 32-zat, 33-zat}, обеспечивающая генерацию и~
целенаправленное формирование кросс-язы\-ко\-вых знаний, основана на методике, 
созданной Анной А.~Зализняк~[32--34], и~включает следующие 
основные этапы (рис.~5):
      \begin{itemize}
\item из корпуса параллельных текстов отбираются пары предложений как 
объекты интерпретации, содержащие исследуемые языковые объекты (см.\ пару 
выделенных фрагментов на рис.~4, которая является примером 
денотата)\footnote{На рис.~4 приведена пара предложений №\,21 с~глаголом настоящего 
времени русского языка, который в~приведенном далее примере станет исследуемым языковым 
объектом. На момент проведения эксперимента, описанного в~статье, в~рус\-ско-фран\-цуз\-ском 
подкорпусе Национального корпуса русского языка было около 4~тыс.\ пар с~глаголами 
настоящего времени (сейчас их около 20~тыс.).};
\item в~каждой из отобранных пар предложений эксперты анализируют перевод 
исследуемого языково\-го объекта на французский язык и~определяют его 
функционально эквивалентный фрагмент (ФЭФ), в~терминологии 
Д.\,О.~Добровольского;
\item языковой объект текста оригинала сопоставляется с~его ФЭФ согласно 
заданному направлению развития контрастивной грамматики\footnote[2]{Например, 
целью развития контрастивной грамматики может быть формирование расширенного списка 
вариантов перевода глагольных конструкций, включая низкочастотные варианты, 
в том числе такие 
варианты, которые могут отсутствовать в~суще\-ст\-ву\-ющих описаниях контрастивной 
грамматики~\cite{35-zat, 36-zat}, но которые могут использоваться переводчиками. Тогда их 
можно извлекать в~процессе сопоставления текстов оригинала и~перевода.};
\item результат сопоставления описывается экспертами в~формализованном виде 
с использованием методики Анны А.~Зализняк;
\item если вариант перевода исследуемого языкового объекта из текста оригинала 
уже включен в~существующие контрастивные грамматики (это случай пары 
выделенных фрагментов на рис.~4), то формализованное описание 
анализируемого экземпляра переводного соответствия вводится в~НБД 
без дополнения (развития) контрастивной грамматики;
\item если вариант перевода исследуемого языкового объекта из текста оригинала 
экспертами считается новым, то формализованное описание этого экземпляра 
переводного соответствия вводится в~НБД, а в~типологию включается новый вид 
соответствия исследуемого языкового объекта и~его ФЭФ;
\item одновременно НБД генерирует четыре цифровых идентификатора (см.\ рис.~3) 
для обозначения:
\begin{itemize}
\item объекта интерпретации с~выделенным в~нем денотатом, который 
представляет собой пару параллельных текстовых фрагментов;
\item концепта денотата как исследуемого языкового объекта или явления, 
который эксперты описывают в~формализованном виде;
\item имени концепта, которое одновременно является и~именем денотата;
\item связей между объектом интерпретации, включающим денотат, 
концептом и~именем.
\end{itemize}
\end{itemize}

 \begin{figure*} %fig6
      \begin{center}
      {\small
      \begin{tabular}{|p{40mm}| p{30mm}| p{40mm}| p{30mm}|}
      \hline
\multicolumn{1}{|c|}{\tabcolsep=0pt\begin{tabular}{c}Контекст\\ глагольной\\ конструкции\end{tabular}}&
\multicolumn{1}{c|}{\tabcolsep=0pt\begin{tabular}{c}Вид глагольной\\ конструкции\\ и~грамматические\\ 
признаки ее контекста\end{tabular}}&\multicolumn{1}{c|}{Контекст ФЭФ}&
\multicolumn{1}{c|}{\tabcolsep=0pt\begin{tabular}{c}Вид  конструкции ФЭФ\\
и~грамматические\\ признаки его контекста\end{tabular}}\\
\hline
как $[$\ldots$]$ у человека, который \textbf{хочет спать}, 
&\multicolumn{1}{l|}{\tabcolsep=0pt\begin{tabular}{l}\textbf{НастВ}\\
$\langle$~SubInf-IPF~$\rangle$\\
$\langle$~SubAttr~$\rangle$\end{tabular}}&
comme $[$\ldots$]$ pour un homme qui \textbf{veut dormir}, &
\multicolumn{1}{l|}{\tabcolsep=0pt\begin{tabular}{l}\textbf{Present}\\ $\langle$~SubInf~$\rangle$\\ 
$\langle$~SubAttr~$\rangle$\end{tabular}}\\
\hline
 \end{tabular}}
\end{center}
\Caption{Формализованное описание глагольной конструкции и~ее ФЭФ с~известным типологическим видом 
соответствия (НастВ, pr$\acute{\mbox{e}}$sent) ({контекст извлечен из пары №\,$21$ на рис.}~4)}
\end{figure*}


      
      Реализуемость разработанной технологии была проверена в~процессе проведения 
эксперимента по анализу переводов глагольных конструкций русского языка на 
французский~\cite{25-zat}. Целью анализа было развитие типологии видов соответствия 
исследуемых глагольных конструкций и~их ФЭФ. Из рус\-ско-фран\-цуз\-ско\-го 
подкорпуса Национального корпуса русского языка (НКРЯ) были отобраны\linebreak 
около~4000~пар предложений, содержащих глагольные конструкции настоящего времени 
русского языка и~их переводы на французский. Согласно работам~\cite{35-zat, 36-zat} они 
могут быть переведены с~помощью следующих~9~грамматических конструкций: 
pr$\acute{\mbox{e}}$sent, imparfait, infinitif, pass$\acute{\mbox{e}}$ 
compos$\acute{\mbox{e}}$, futur simple, subjonctif pr$\acute{\mbox{e}}$sent, 
g$\acute{\mbox{e}}$rondif, futur imm$\acute{\mbox{e}}$diat и~imp$\acute{\mbox{e}}$ratif.
      
      Таким образом, до начала эксперимента типология видов включала девять записей 
для русского настоящего времени (НастВ). Первая запись имела вид (НастВ, pr$\acute{\mbox{e}}$sent), 
вторая~--- (НастВ, imparfait) и~т.\,д.\ до (НастВ, imp$\acute{\mbox{e}}$ratif). Во время эксперимента эксперты 
сравнивали оригинальный и~переведенный текст в~отобранных парах предложений, выделяя 
глагольную конструкцию НастВ в~тексте оригинала и~ее ФЭФ в~тексте перевода, которые 
в~совокупности являются денотатом.
      
      Если вариант перевода глагольной конструкции НастВ уже был изначально включен 
в типологию (видов соответствия), то формализованное описание этой конструкции и~ее 
ФЭФ (рис.~6) до\-бав\-ля\-ют\-ся в~НБД без изменения типологии видов. Если вариант 
перевода глагольной конструкции НастВ эксперты считают новым, то формализованное 
описание этой конструкции и~ее ФЭФ добавляются в~НБД, а~в~типологию включается 
новый вид (см.\ рис.~5). Формализованное описание создается экспертами на основе 
смыслового содержания соответствия глагольной конструкции и~ее ФЭФ. Это смысловое 
содержание вида соответствия и~является тем концептом, который формируется в~процессе 
анализа этого соответствия. Если концепт оказывается новым, то типология дополняется 
его именем в~формате (вид глагольной конструкции, вид ее ФЭФ).

\begin{table*}\small
\begin{center}
\Caption{Четыре новых типологических вида}
\vspace*{2ex}

\begin{tabular}{|c|l|c|c|}
\hline
№ п/п &\multicolumn{1}{|c|}
{\tabcolsep=0pt\begin{tabular}{c}Типологический\\ вид соответствия\end{tabular}}&
\tabcolsep=0pt\begin{tabular}{c}Число\\ экземпляров\\ вида в~НБД\end{tabular}&
\tabcolsep=0pt\begin{tabular}{c}Статус вида\\ 
(\textit{известный} до начала эксперимента\\ или \textit{новый})\end{tabular}\\
\hline
1&(НастВ, pr$\acute{\mbox{e}}$sent)&311\hphantom{99}&\textit{известный}\\
2&(НастВ, imparfait)&53\hphantom{9}&\textit{известный}\\
3&(НастВ, infinitif)&15\hphantom{9}&\textit{известный}\\
4&(НастВ, pass$\acute{\mbox{e}}$ compos$\acute{\mbox{e}}$)&8&\textit{известный}\\
5&(НастВ, conditionnel pr$\acute{\mbox{e}}$sent)&7&\textit{новый}\\
6&(НастВ, futur simple)&5&\textit{известный}\\
7&(НастВ, subjonctif pr$\acute{\mbox{e}}$sent)&4&\textit{известный}\\
8&(НастВ, participe pass$\acute{\mbox{e}}$)&2&\textit{новый}\\
9&(НастВ, g$\acute{\mbox{e}}$rondif)&2&\textit{известный}\\
10\hphantom{9}&(НастВ, subjonctif imparfait)&1&\textit{новый}\\
11\hphantom{9}&(НастВ, plus-que-parfait)&1&\textit{новый}\\
12\hphantom{9}&(НастВ, futur imm$\acute{\mbox{e}}$diat)&0&\textit{известный}\\
13\hphantom{9}&(НастВ, imp$\acute{\mbox{e}}$ratif)&0&\textit{известный}\\
\hline
\multicolumn{2}{|l|}{Всего записей}&409\hphantom{99}&\\
\hline
\end{tabular}
\end{center}
\end{table*}


      
      Данные эксперимента по извлечению и~описанию новых концептов, полученные на 
первом и~втором его этапах, приведены в~табл.~1 и~2 соответственно.
      

\begin{table*}\small
\begin{center}
\Caption{Восемь новых типологических видов}
\vspace*{2ex}

\begin{tabular}{|c|l|c|c|}
\hline
№ п/п&\multicolumn{1}{|c|}
{\tabcolsep=0pt\begin{tabular}{c}Типологический\\ вид соответствия\end{tabular}}&
\tabcolsep=0pt\begin{tabular}{c}Число\\ экземпляров\\ вида в~НБД\end{tabular}&
\tabcolsep=0pt\begin{tabular}{c}Статус вида\\ 
(\textit{известный} до начала эксперимента\\ или \textit{новый})\end{tabular}\\
\hline
1&(НастВ, pr$\acute{\mbox{e}}$sent)&1587\hphantom{99}&\textit{известный}\\
2&(НастВ, imparfait)&328\hphantom{9}&\textit{известный}\\
3&(НастВ, infinitif)&71&\textit{известный}\\
4&(НастВ, pass$\acute{\mbox{e}}$ compos$\acute{\mbox{e}}$)&30&\textit{известный}\\
5&(НастВ, conditionnel pr$\acute{\mbox{e}}$sent)&23&\textit{новый}\\
6&(НастВ, participe pass$\acute{\mbox{e}}$)&22&\textit{новый}\\
7&(НастВ, subjonctif pr$\acute{\mbox{e}}$sent)&19&\textit{известный}\\
8&(НастВ, futur simple)&19&\textit{известный}\\
9&(НастВ, participe pr$\acute{\mbox{e}}$sent)&19&\textit{новый}\\
10\hphantom{9}&(НастВ, g$\acute{\mbox{e}}$rondif)&15&\textit{известный}\\
11\hphantom{9}&(НастВ, futur imm$\acute{\mbox{e}}$diat)&10&\textit{известный}\\
12\hphantom{9}&(НастВ, pass$\acute{\mbox{e}}$ simple)&10&\textit{новый}\\
13\hphantom{9}&(НастВ, plus-que-parfait)&\hphantom{9}8&\textit{новый}\\
14\hphantom{9}&(НастВ, subjonctif imparfait)&\hphantom{9}6&\textit{новый}\\
15\hphantom{9}&(НастВ, imp$\acute{\mbox{e}}$ratif)&\hphantom{9}5&\textit{известный}\\
16\hphantom{9}&(НастВ, infinitif pass$\acute{\mbox{e}}$)&\hphantom{9}3&\textit{новый}\\
17\hphantom{9}&(НастВ, pass$\acute{\mbox{e}}$ imm$\acute{\mbox{e}}$diat)&\hphantom{9}1&\textit{новый}\\
\hline
\multicolumn{2}{|l|}{Всего записей}&2176\hphantom{99}&\\
\hline
\end{tabular}
\end{center}
\end{table*}

      
      Таблица~1 содержит результаты первого этапа анализа 409~пар предложений 
из~4000, т.\,е.\ приблизительно 10\% от общего их числа. На этом этапе эксперты выявили и~
описали четыре новых типологических вида перевода русского настоящего времени, 
которым присвоили следующие имена: (НастВ, conditionnel pr$\acute{\mbox{e}}$sent), 
(НастВ, participe pass$\acute{\mbox{e}}$), (НастВ, subjonctif imparfait) и~(НастВ, 
      plus-que-parfait). В~то же самое время они не нашли примеры вариантов перевода с~
глагольными конструкциями futur imm$\acute{\mbox{e}}$diat и~imp$\acute{\mbox{e}}$ratif.
      
      Таким образом, первый вариант описания цели развития этой типологии видов мог 
бы состоять в~том, чтобы найти примеры для всех изначально известных девяти 
типологических видов и~описать найденные новые виды (как минимум найти и~описать 
один новый вид). В~этом случае обработка 409~пар недостаточна, так как для достижения 
такой цели эксперты должны продолжать искать примеры французских переводов с~
глагольными конструкциями futur imm$\acute{\mbox{e}}$diat и~imp$\acute{\mbox{e}}$ratif 
(см.\ табл.~1).
      
      Таблица~2 содержит результаты второго этапа анализа 2176~пар предложений 
из~4000, т.\,е.\ приблизительно 54\% от общего их числа. Эксперты выявили и~описали еще 
четыре новых типологических вида перевода русского настоящего времени: (НастВ, 
participe pr$\acute{\mbox{e}}$sent), (НастВ, pass$\acute{\mbox{e}}$ simple), (НастВ, infinitif 
pass$\acute{\mbox{e}}$) и~(НастВ, pass$\acute{\mbox{e}}$ imm$\acute{\mbox{e}}$diat) 
(см.\ табл.~2).
      

      
      Одновременно они нашли французские переводы с~глагольными конструкциями 
futur imm$\acute{\mbox{e}}$diat и~imp$\acute{\mbox{e}}$ratif. В~итоге проведенного 
эксперимента все восемь новых типологических видов были добавлены экспертами 
к~девяти уже имеющимся в~типологии. Дополненная типология включала 17~видов после 
обработки 2176~пар предложений из~4000 (см.\ табл.~2).
      
      Второй вариант описания цели развития этой типологии видов мог бы состоять 
      в~том, чтобы \mbox{найти} примеры для всех изначально известных девяти видов, 
      описать 
найденные новые типологические виды (как минимум найти и~описать один новый вид) при 
условии, что эксперты должны обработать не менее чем 55\% от всех пар предложений 
с~глагольной конструкцией НастВ, имеющихся в~рус\-ско-фран\-цуз\-ском подкорпусе НКРЯ, 
т.\,е.\ 55\% от~4000 на момент проведения эксперимента. В~этом случае анализ 2176~пар 
был недостаточен и~эксперты должны были бы продолжить свою работу, пока не будет 
обработано 2200~пар.
      
      Кроме развития типологии видов соответствия глагольной конструкции и~ее ФЭФ 
разработанная технология в~настоящее время используется для формирования списков 
возможных вариантов перевода лингвоспецифичных слов русского языка на французский 
язык~\cite{37-zat}, а также для формирования методологии контрастивного корпусного 
исследования категории безличности в~русском языке. Таким образом, компьютерная 
поддержка процессов генерации и~целенаправленного развития экспертами  
кросс-язы\-ко\-вых знаний была опробована в~процессе развития контрастивной  
рус\-ско-фран\-цуз\-ской грамматики для глагольных конструкций и~продолжает 
применяться для контрастивных исследований лингвоспецифичных слов русского языка.

\section{Заключение}

      Двуязычные параллельные корпуса, в~которых каждому тексту на русском языке 
соответствует один или несколько его переводов на другой язык, являются потенциальным 
и неисчерпаемым источником генерации новых, но трудно извлека\-емых кросс-язы\-ко\-вых 
знаний. Являясь уникальным и~постоянно пополняемым, он может быть использован для 
существенного повышения качества машинного перевода, актуализации моно- 
и~двуязычных грамматик, а также для обновления широкого спектра образовательных курсов 
по лингвистике, теории и~практике перевода.
      
      Однако функциональность традиционных электронных корпусов не обеспечивает 
извлечения тех невыражаемых знаний переводчиков, которые применялись ими в~процессе 
перевода. Эти знания могут быть личностными или коллективными и~передаваться 
в~процессе их социализации (см.\ рис.~1), например в~процессе демонстрации образцов 
перевода в~процессе обучения, но при этом они могут продолжать оставаться 
невыражаемыми и~неэксплицированными. Наблюдается парадокс: с~одной\linebreak стороны, 
в~электронных корпусах есть образцы переводов, полученные с~применением невыра-\linebreak жа\-емых 
знаний переводчиков; с~другой стороны, традиционные параллельные корпуса не могут 
поддержать процессы извлечения и~экспликации этих знаний. Поэтому понадобилось 
существенное дополнение функциональности традиционных корпусов за счет разработки 
новой информационной технологии целенаправленной генерации знаний.
      
      Разработка этой технологии была связана с~развитием семиотических оснований 
информатики как ин\-фор\-ма\-ци\-он\-но-компью\-тер\-ной науки и~созданием новых 
моделей целенаправленной генерации и~развития новых знаний с~использованием четырех 
сред ее предметной области. К~ментальной, со\-ци\-аль\-но-ком\-му\-ни\-ка\-ци\-он\-ной и~цифровой 
средам была добавлена нейросреда. Ее добавление стало основой для определения 
нейросемиотического тет\-ра\-эдра, что представляет собой качественно новое развитие 
понятия семиотического тет\-ра\-эдра, предложенного группой FRISCO в~конце прошлого 
века.
      
      Суть этого развития в~том, что предложено разделять объекты ментальной среды 
      и~нейросреды в~предметной области информатики. На практике такое деление уже 
используется в~процессе разработки ряда когнитивных и~нейрокоммуникационных 
технологий. Следовательно, это должно найти свое отражение и~в теоретических 
основаниях информатики, а также в~образовательных курсах по ее изучению в~системе 
среднего и~высшего профессионального образования. Отметим, что разделение объектов 
ментальной среды и~нейросреды существенно увеличивает спектр интерфейсов, которые 
являются новыми объектами исследований в~информатике~\cite{12-zat}.
      
      Адаптация разработанной технологии для проведения контрастивных исследований 
повлекла за собой необходимость в~новой категории инфор\-мационных лингвистических 
ресурсов, получивших название НБД, методы формирования которых 
разработаны М.\,Г.~Кружковым~\cite{30-zat, 31-zat, 33-zat, 34-zat}. С~прикладной точки 
зрения реализация в~этих базах данных моделей и~технологии целена\-прав\-лен\-ной генерации 
и развития новых знаний дала возможность существенно дополнить функциональность 
электронных корпусов текстов и~тем самым обеспечить извлечение тех невыражаемых 
и~труд\-нодоступных знаний переводчиков, которые применялись ими, являясь 
неэксплицированными и~новыми в~контрастивной лингвистике.
      
{\small\frenchspacing
 {%\baselineskip=10.8pt
 \addcontentsline{toc}{section}{References}
 \begin{thebibliography}{99}
\bibitem{1-zat}
\Au{Nonaka I.} The knowledge-creating company~// Harvard Bus. Rev., 1991. 
Vol.~69. No.\,6. P.~96--104.
\bibitem{2-zat}
\Au{Nonaka I.} A dynamic theory of organizational knowledge creation~// Organ. 
Sci., 1994. Vol.~5. No.\,1. P.~14--37.
\bibitem{3-zat}
\Au{Wierzbicki A.\,P., Nakamori~Y.} Basic dimensions of creative space // Creative space: 
Models of creative processes for knowledge civilization age~/ Eds. A.\,P.~Wierzbicki, 
Y.~Nakamori.~--- Berlin--Heidelberg: Springer Verlag, 2006. P.~59--90.
\bibitem{4-zat}
\Au{Wierzbicki A.\,P., Nakamori Y.} Knowledge sciences: Some new developments~// 
Zeitschrift f$\ddot{\mbox{u}}$r Betriebswirtschaft, 2007. Vol.~77. No.\,3. P.~271--295.
\bibitem{5-zat}
\Au{Wierzbicki A.\,P., Nakamori Y.} The importance of multimedia principle and emergence 
principle for the knowledge civilisation age~// J.~Syst. Sci. Syst. Eng., 2008. 
Vol.~17. No.\,3. P.~297--318.
\bibitem{6-zat}
\Au{Nakamori Y.} Methodology for knowledge synthesis~// 
Cutting-edge research topics on  multiple criteria decision making~/ 
Eds.\ Y.~Shi, S.~Wang, Y.~Peng, J.~Li, Y.~Zeng.~---
Communications in computer and information science ser.~--- 
Berlin: Springer, 2009. Vol.~35. P.~311--317.
\bibitem{7-zat}
Knowledge science~--- modeling the knowledge creation process~/
Ed. Y.~Nakamori.~--- London\,--\,New York: CRC Press, 2011. 
177~p.
\bibitem{8-zat}
\Au{Nakamori Y.} Knowledge and systems science~--- enabling systemic knowledge 
synthesis.~--- London\,--\,New York: CRC Press, 2013. 234~p.
\bibitem{9-zat}
\Au{Zatsman I., Buntman P.} Theoretical framework and denotatum-based models of 
knowledge creation for monitoring and evaluating R\&D program implementation~// 
Int. J.~Softw. Sci. Comput. Intell., 2013. Vol.~5. No.\,1. 
P.~15--31.
\bibitem{10-zat}
\Au{Зацман И.\,М.} Построение системы терминов 
ин\-фор\-ма\-ци\-он\-но-компью\-тер\-ной науки: 
проб\-лем\-но-ори\-ен\-ти\-ро\-ван\-ный подход~// Теория и~практика общественной 
научной информации.~--- М.: \mbox{ИНИОН} РАН, 2013. Вып.~21. С.~120--159.

\bibitem{12-zat} %11
\Au{Зацман И.\,М.} Таблица интерфейсов информатики как  
ин\-фор\-ма\-ци\-он\-но-компью\-тер\-ной науки~// 
На\-уч.-тех\-нич. информация. Сер.~1: Организация и~методика информационной 
работы, 2014. №\,11. С.~1--15.

\bibitem{11-zat} %12
\Au{Зацман И.\,М.} Ин\-фор\-ма\-ци\-он\-но-компью\-тер\-ная наука: технологические 
предпосылки становления~// Информационные технологии, 2014. №\,3. С.~3--12.

\bibitem{13-zat}
\Au{Успенский В.\,А.} К~публикации статьи Г.~Фреге <<Смысл и~денотат>>~// 
Семиотика и~информатика, 1997. Вып.~35. 
С.~351--352.
\bibitem{14-zat}
\Au{Фреге Г.} Смысл и~денотат~// Семиотика и~информатика, 1997. Вып.~35. С.~352--379.
\bibitem{15-zat}
\Au{Фреге Г.} Понятие и~вещь~// Семиотика и~информатика, 1997. Вып.~35. С.~380--396.
\bibitem{16-zat}
\Au{Зацман И.\,М.} Семиотическая модель взаимосвязей концептов, информационных 
объектов и~компьютерных кодов~// Информатика и~её применения, 2009. Т.~3. Вып.~2. 
С.~65--81.
\bibitem{17-zat}
A~framework of information system concepts (Web edition): The FRISCO Report.~--- IFIP, 
1998. {\sf http://www.mathematik.uni-marburg.de/$\sim$hesse/\linebreak papers/fri-full.pdf}.
\bibitem{18-zat}
\Au{Hesse W., Verrijn-Stuart A.\,A.} Towards a~theory of information systems: The FRISCO 
approach~// Information modelling and knowledge bases~XII~/ Eds. H.~Kangassalo, 
H.~Jaakkola, E.~Kawaguchi.~--- Amsterdam: IOS Press, 2001. P.~81--91.
\bibitem{19-zat}
\Au{Eco U.} A~theory of semiotics.~--- Bloomington: Indiana University Press, 1976. 356~p.
\bibitem{20-zat}
\Au{Пирс Ч.} Логические основания теории знаков~/
Пер. с~англ.~---  СПб.: Алетейя, 2000. 352~с.

\bibitem{21-zat}
\Au{Василюк Ф.\,Е.} Структура образа~// Вопросы психологии, 1993. №\,5. С.~5--19.
\bibitem{22-zat}
\Au{Зацман И.\,М.} Нестационарная семиотическая модель компьютерного кодирования 
концептов, информационных объектов и~денотатов~// Информатика и~её применения, 
2009. Т.~3. Вып.~4. С.~87--101.
\bibitem{23-zat}
\Au{Зацман И.\,М., Бунтман П.\,С.} Проектирование индикаторов мониторинга в~сфере 
науки: теоретические основания и~модели~// Онтология проектирования, 2014. №\,3(13). 
С.~32--51.
\bibitem{24-zat}
\Au{Zatsman I., Buntman N., Kruzhkov M., Nuriev~V., Zalizniak Anna~A.} Conceptual 
framework for development of computer technology supporting cross-linguistic knowledge 
discovery~// 15th European Conference on Knowledge Management Proceedings.~---  
Reading: Academic Publishing International Ltd., 2014. Vol.~3. P.~1063--1071.
\bibitem{25-zat}
\Au{Zatsman I., Buntman N.} Outlining goals for discovering new knowledge and 
computerised tracing of emerging meanings discovery~// 16th European Conference on 
Knowledge Management Proceedings.~--- Reading: Academic Publishing International 
Ltd., 2015. P.~851--860.
\bibitem{26-zat}
\Au{Баарс~Б., Гейдж~Н.} Мозг, познание, разум: введение в~когнитивные 
нейронауки~/ Пер.\ с~англ.~--- М.: БИНОМ. Лаборатория знаний, 2014. Ч.~1. 544~с.; Ч.~2. 464~с.
(\Au{Baars~B., Gage~N.} Cognition, brain, and consciousness: Introduction to cognitive 
neuroscience.~--- Burlington, MA, USA: Academic Press/Elsevier, 2010. 677~p.)
\bibitem{27-zat}
\Au{Секерина И.\,А.} Метод вызванных потенциалов мозга в~экспериментальной 
психолингвистике~// Вопросы языкознания, 2006. №\,3. С.~22--45.
\bibitem{28-zat}
\Au{De Charms R.\,C.} Applications of real-time fMRI~// Nat. Rev. Neurosci., 
2008. Vol.~9. No.\,9. P.~720--729.
\bibitem{29-zat}
\Au{Kumaran D., Summereld J.\,J., Hassabis~D., Maguire~E.\,A.} Tracking the emergence of 
conceptual knowledge during human decision-making~// Neuron, 2009. Vol.~63. No.\,6. 
P.~889--901.
\bibitem{30-zat}
\Au{Зализняк А.\,А., Зацман И.\,М., Инькова~О.\,Ю., Кружков~М.\,Г.} Надкорпусные 
базы данных как лингвистический ресурс~// Корпусная лингвистика-2015: Тр. 7-й 
Междунар. конф.~--- СПб.: СПбГУ, 2015. С.~211--218.
\bibitem{31-zat}
\Au{Кружков М.\,Г.} Информационные ресурсы контрастивных лингвистических 
исследований: электронные корпуса текстов~// Системы и~средства информатики, 2015. 
Т.~25. Вып.~2. С.~140--159.
\bibitem{32-zat}
\Au{Loiseau~S., Sitchinava~D.\,V., Zalizniak~Anna~A., Zatsman~I.\,M.} Information 
technologies for creating the database of equivalent verbal forms in the Russian-French 
multivariant parallel corpus~// Информатика и~её применения, 2013. Т.~7. №\,2. 
С.~100--109.
\bibitem{33-zat}
\Au{Kruzhkov M.\,G., Buntman N.\,V., Loshchilova~E.\,Ju., Sitchinava~D.\,V., Zalizniak 
Anna~A., Zatsman~I.\,M.} A~database of Russian verbal forms and their French translation 
equivalents~// Компьютерная лингвистика и~интеллектуальные технологии: По мат-лам 
ежегодной Междунар. конф. <<Диалог>>.~--- М.: РГГУ, 2014. Вып.~13(20).
С.~284--297.
\bibitem{34-zat}
\Au{Бунтман Н.\,В., Зализняк Анна~A., Зацман~И.\,M., Кружков~М.\,Г., 
Лощилова~Е.\,Ю., Сичинава~Д.\,В.} Инфор\-ма\-ционные технологии корпусных 
исследований:\linebreak принципы построения кросслингвистических баз данных~// Информатика 
и её применения, 2014. Т.~8. Вып.~2. С.~98--110.
\bibitem{35-zat}
\Au{Гак В.\,Г.} Русский язык в~сопоставлении с~французским.~--- М.: УРСС, 2006. 264~с.
\bibitem{36-zat}
\Au{Kouznetsova I.\,N.} Grammaire contrastive du francais et du russe.~--- Moscow: Nestor 
Academic Publs., 2009. 272~p.
\bibitem{37-zat}
\Au{Зализняк Анна А.} Лингвоспецифичные единицы русского языка в~свете 
контрастивного корпусного анализа~// Компьютерная лингвистика и~интеллектуальные 
технологии: По мат-лам ежегодной Междунар. конф. <<Диалог>>.~--- М.: РГГУ, 2015. 
Вып.~14(21). Т.~1. С.~683--695.

 \end{thebibliography}

 }
 }

\end{multicols}

\vspace*{-3pt}

\hfill{\small\textit{Поступила в~редакцию 22.07.15}}

%\newpage

\vspace*{12pt}

\hrule

\vspace*{2pt}

\hrule

%\vspace*{12pt}

\def\tit{GOAL-ORIENTED PROCESSES OF~CROSS-LINGUAL EXPERT KNOWLEDGE CREATION:
SEMIOTIC FOUNDATIONS FOR~MODELING}

\def\titkol{Goal-oriented processes of cross-lingual expert knowledge creation:
Semiotic foundations for modeling}

\def\aut{I.\,M.~Zatsman}

\def\autkol{I.\,M.~Zatsman}

\titel{\tit}{\aut}{\autkol}{\titkol}

\vspace*{-9pt}


\noindent
Institute of Informatics Problems, Federal Research Center ``Computer Science and Control'' of 
the Russian Academy of Sciences, 44-2 Vavilov Str., Moscow 119333, Russian Federation


\def\leftfootline{\small{\textbf{\thepage}
\hfill INFORMATIKA I EE PRIMENENIYA~--- INFORMATICS AND
APPLICATIONS\ \ \ 2015\ \ \ volume~9\ \ \ issue\ 3}
}%
 \def\rightfootline{\small{INFORMATIKA I EE PRIMENENIYA~---
INFORMATICS AND APPLICATIONS\ \ \ 2015\ \ \ volume~9\ \ \ issue\ 3
\hfill \textbf{\thepage}}}

\vspace*{3pt}

\Abste{The results of development of semiotic foundations for modeling goal-oriented processes 
of cross-lingual expert knowledge creation are described. The technology supporting these 
processes is outlined. The demand for such technologies is obvious in situations where present 
systems of expert knowledge do not answer to new socially or technologically significant 
purposes, corresponding to new or changed requirements of modern society. Instead of centering 
on the well-known artificial intelligence methods and models of information processing for 
knowledge representation, this paper focuses on development of new models of goal-oriented 
processes of expert knowledge creation reflecting dynamics of its formation. The suggested 
approach to modeling these processes and to development of technologies supporting them is 
focused on those applied areas where expert knowledge is elicited from domain experts. The 
experts analyze texts or other interpretation objects which can vary over time and enter\linebreak\vspace*{-12pt}}

\Abstend{the 
results of analysis into supracorpus databases. The distinguishing feature of the semiotic approach 
to modeling is the explicit description of relations between new expert knowledge and those 
interpretation objects, from which parts of new knowledge were generated. Other important 
feature is the explicit description of parts of knowledge corresponding to interpretation objects that 
may vary over time. Feasibility of the approach is demonstrated on
the example of information 
technology, which supports the processes of creation of cross-lingual expert knowledge based on 
French translations of Russian verbal constructions. Cross-lingual knowledge is generated in the 
course of analysis of parallel texts in Russian and French languages.}

\KWE{cross-lingual expert knowledge; computer modeling; knowledge creation; interpretation 
objects; semiotic foundations; models of knowledge creation processes}

\DOI{10.14357/19922264150311}

\Ack
\noindent
The work was supported by the Russian Foundation for Basic Research 
(projects 14-07-00785, 13-06-00403) and the Rusian Foundation for Humanities 
(project 15-04-00507).

%\vspace*{3pt}

  \begin{multicols}{2}

\renewcommand{\bibname}{\protect\rmfamily References}
%\renewcommand{\bibname}{\large\protect\rm References}

{\small\frenchspacing
 {%\baselineskip=10.8pt
 \addcontentsline{toc}{section}{References}
 \begin{thebibliography}{99}
\bibitem{1-zat-1}
\Aue{Nonaka, I.} 1991. The knowledge-creating company. \textit{Harvard Bus. Rev.} 
69(6):96--104.
\bibitem{2-zat-1}
\Aue{Nonaka, I.} 1994. A~dynamic theory of organizational knowledge creation. 
\textit{Organ. Sci.} 5(1):14--37.
\bibitem{3-zat-1}
\Aue{Wierzbicki, A.\,P., and Y.~Nakamori}. 2006. Basic dimensions of creative space. 
\textit{Creative space: Models of creative processes for knowledge civilization age}. 
Eds. A.\,P.~Wierzbicki,  and
Y.~Nakamori. Berlin--Heidelberg: Springer Verlag. 59--90.
\bibitem{4-zat-1}
\Aue{Wierzbicki, A.\,P., and Y.~Nakamori}. 2007. Knowledge sciences: Some new 
developments. \textit{Zeitschrift f$\ddot{\mbox{u}}$r Betriebswirtschaft} 77(3):271--295.
\bibitem{5-zat-1}
\Aue{Wierzbicki, A.\,P., and Y. Nakamori}. 2008. The importance of multimedia principle 
and emergence principle for the knowledge civilisation age. \textit{J.~Syst. Sci. Syst.
Eng.} 17(3):297--318.
\bibitem{6-zat-1}
\Aue{Nakamori, Y.} 2009. Methodology for knowledge synthesis. \textit{Cutting-edge 
research topics on multiple criteria decision making}. 
Eds.\ Y.~Shi, S.~Wang, Y.~Peng, J.~Li, and Y.~Zeng. Communications in computer and 
information science ser. Berlin: Springer. 35:311--317.
\bibitem{7-zat-1}
Nakamori, Y., ed. 2011. \textit{Knowledge science~--- modeling the knowledge creation 
process}. London\,--\,New York: CRC Press. 177~p.
\bibitem{8-zat-1}
\Aue{Nakamori, Y.} 2013. \textit{Knowledge and systems science~--- enabling systemic 
knowledge synthesis}. London\,--\,New York: CRC Press. 234~p.
\bibitem{9-zat-1}
\Aue{Zatsman, I., and P. Buntman}. 2013. Theoretical framework and denotatum-based 
models of knowledge creation for monitoring and evaluating R\&D program implementation. 
\textit{Int. J.~Softw. Sci. Comput. Intell.} 5(1):15--31.
\bibitem{10-zat-1}
\Aue{Zatsman, I.} 2013. Postroenie sistemy terminov informatsionno-komp'yuternoy nauki: 
problemno-orientirovannyy podkhod [Construction of the system of terms of information and 
computer science: A~problem-oriented approach]. \textit{Teoriya i~praktika obshchestvennoy 
nauchnoy informatsii} [Theory and practice of scientific information for social sciences]. 
Moscow: INION RAS. 120--159.

\bibitem{12-zat-1} %11
\Aue{Zatsman, I.} 2014. Tablitsa interfeysov informatiki kak informatsionno-komp'yuternoy 
nauki [A~table of interfaces of informatics as computer and information science]. 
\textit{Nauchno-tekhnicheskaya informatsiya. Ser.~1:\linebreak Organizatsiya i~metodika 
informatsionnoy raboty} [Scientific and Technical Information. Ser.~1: Management and 
methodology of information work] (11):1--15.

\bibitem{11-zat-1} %12
\Aue{Zatsman, I.} 2014. Informatsionno-komp'yuternaya nauka: Tekhnologicheskie 
predposylki stanovleniya [Information and computer science: Technological prerequisites of 
formation]. \textit{Informatsionnye Tekhnologii} [Information Technologies] (3):3--12.

\bibitem{13-zat-1}
\Aue{Uspenskiy, V.\,A.} 1997. K~publikatsii stat'i G.~Frege ``Smysl i~denotat'' [To the 
publication of the paper of G.~Frege ``Sense and reference'']. \textit{Semiotika i Informatika} 
[Semiotics and Informatics] 35:351--352.
\bibitem{14-zat-1}
\Aue{Frege, G.} 1997. Smysl i denotat [Sense and reference]. \textit{Semiotika i~Informatika} 
[Semiotics and Informatics] 35:352--379.
\bibitem{15-zat-1}
\Aue{Frege, G.} 1997. Ponyatie i~veshch' [Concept and thing]. \textit{Semiotika 
i~Informatika} [Semiotics and Informatics] 35:380--396.
\bibitem{16-zat-1}
\Aue{Zatsman, I.} 2009. Semioticheskaya model' vzaimosvyazey kontseptov, 
informatsionnykh ob"ektov i komp'yuternykh kodov [Semiotic model of relationships of 
concepts, information objects, and computer codes]. \textit{Informatika i~ee Primeneniya}~--- 
\textit{Inform. Appl.} 3(2):65--81.
\bibitem{17-zat-1}
FRISCO.  1998. A framework of information system concepts. 
Report. Available at: {\sf 
http://www.mathematik.uni-marburg.de/$\sim$hesse/papers/fri-full.pdf} (accessed July~29, 
2015).
\bibitem{18-zat-1}
\Aue{Hesse, W., and A.\,A.~Verrijn-Stuart}. 2001. {Towards a~theory of information 
systems: The FRISCO approach}. \textit{Information modelling and knowledge bases~XII}. 
Eds. H.~Kangassalo,  H.~Jaakkola, and E.~Kawaguchi.
Amsterdam: IOS Press. 81--91.
\bibitem{19-zat-1}
\Aue{Eco, U.} 1976. \textit{A~theory of semiotics}. Bloomington: Indiana University Press. 
356~p.
\bibitem{20-zat-1}
\Aue{Peirce, Ch.\,S.} 1931--1958. \textit{Collected papers of Charles S.~Peirce}. 
Cambridge: Harvard University Press. 8~vols.
\bibitem{21-zat-1}
\Aue{Vasilyuk, F.\,E.} 1993. Struktura obraza [Structure of image]. \textit{Voprosy 
Psikhologii} [Questions of Psychology] (5):5--19.
\bibitem{22-zat-1}
\Aue{Zatsman, I.} 2009. Nestatsionarnaya semioticheskaya model' komp'yuternogo 
kodirovaniya kontseptov, informatsionnykh ob"ektov i~denotatov [Nonstationary semiotic 
model of computer coding of concepts, information objects and denotata]. \textit{Informatika 
i~ee Primeneniya}~--- \textit{Inform. Appl.} 3(4):87--101.
\bibitem{23-zat-1}
\Aue{Zatsman, I., and P. Buntman}. 2014. Proektirovanie indikatorov monitoringa v~sfere 
nauki: Teoreticheskie osnovaniya i~modeli [Design of indicators for monitoring in science: 
Theoretical foundations and models]. \textit{Ontologiya Proektirovaniya} [Ontology of 
Design] (3):32--51.
\bibitem{24-zat-1}
\Aue{Zatsman, I., N. Buntman, M.~Kruzhkov, V.~Nuriev, and Anna A.~Zalizniak}. 2014. 
Conceptual framework for development of computer technology supporting cross-linguistic 
knowledge discovery. \textit{15th European Conference on Knowledge Management 
Proceedings}. Reading: Academic Publishing International Ltd. 3:1063--1071.
\bibitem{25-zat-1}
\Aue{Zatsman, I., and N. Buntman}. 2015. Outlining goals for discovering new knowledge 
and computerised tracing of emerging meanings discovery. \textit{16th European Conference 
on Knowledge Management Proceedings}. Reading: Academic Publishing International 
Ltd. 851--860.
\bibitem{26-zat-1}
\Aue{Baars, B., and N. Gage}. 2010. \textit{Cognition, brain, and consciousness: Introduction 
to cognitive neuroscience}. Burlington, MA: Academic Press/Elsevier. 677~p.
\bibitem{27-zat-1}
\Aue{Sekerina, I.} 2006. Metod vyzvannykh potentsialov mozga v~eksperimental'noy 
psikholingvistike [Method of evoked potentials of brain in experimental psycholinguistics]. 
\textit{Voprosy Yazykoznaniya} [Topics in the Study of Language] 3:22--45.
\bibitem{28-zat-1}
\Aue{De Charms, R.\,C.} 2008. Applications of real-time fMRI. \textit{Nat. Rev. 
Neurosci.} 9(9):720--729.
\bibitem{29-zat-1}
\Aue{Kumaran, D., J.\,J. Summereld, D.~Hassabis, and E.\,A.~Maguire}. 2009. Tracking the 
emergence of conceptual knowledge during human decision-making. \textit{Neuron} 
63(6):889--901.
\bibitem{30-zat-1}
\Aue{Zalizniak, Anna A., I.~Zatsman, O.~Inkova, and M.~Kruzhkov}. 2015. Nadkorpusnye 
bazy dannykh kak lingvisticheskiy resurs [Supracorpus database as linguistic resource]. 
\textit{Tr. 7-y konf. po Korpusnoy Lingvistike} [7th Conference on Corpus Linguistics 
Proceedings]. St.\ Petersburg. 211--218.
\bibitem{31-zat-1}
\Aue{Kruzhkov, M.} 2015. Informatsionnye resursy kontrastivnykh lingvisticheskikh 
issledovaniy: Elektronnye korpusa tekstov [Information resources for contrastive studies: 
Digital text corpora]. \textit{Sistemy i~Sredstva Informatiki}~--- \textit{Systems and Means of 
Informatics} 25(2):140--159.
\bibitem{32-zat-1}
\Aue{Loiseau, S., D.\,V. Sitchinava, Anna A.~Zalizniak, and I.\,M.~Zatsman}. 2013. 
Information technologies for creating the database of equivalent verbal forms in the 
Russian-French multivariant parallel corpus. \textit{Informatika i~ee Primeneniya}~--- 
\textit{Inform.s Appl.} 7(2):100--109.
\bibitem{33-zat-1}
\Aue{Kruzhkov, M.\,G., N.\,V. Buntman, E.\,Ju.~Loshchilova, D.\,V.~Sitchinava, Anna 
A.~Zalizniak, and I.\,M.~Zatsman}. 2014. A~database of Russian verbal forms and their 
French translation equivalents. \textit{Komp'yuternaya Lingvistika i~Intellektual'nye 
Tekhnologii. Po mat-lam ezhegodnoy Mezhdunar. konf. ``Dialog-2014''} [Computational 
Linguistics and Intellectual Technologies: Conference (International) ``Dialog-2014'' 
Proceedings]. Moscow. 13(20):284--297.
\bibitem{34-zat-1}
\Aue{Buntman, N.\,V., Anna A.~Zaliznyak, I.\,M.~Zatsman, M.\,G.~Kruzhkov, 
E.\,Yu.~Loshchilova, and D.\,V.~Sitchinava}. 2014. Informatsionnye tekhnologii korpusnykh 
issledovaniy: Printsipy postroeniya krosslingvisticheskikh baz dannykh [Information 
technologies for corpus studies: Underpinnings for cross-linguistic database creation]. 
\textit{Informatika i~ee Primeneniya}~--- \textit{Inform. Appl.} 8(2):98--110.
\bibitem{35-zat-1}
\Aue{Gak, V.\,G.} 2006. \textit{Russkiy yazyk v~sopostavlenii s~fran\-tsuz\-skim} [Russian in 
comparison to French]. Moscow: URSS. 264~p.
\bibitem{36-zat-1}
\Aue{Kouznetsova, I.\,N.} 2009. \textit{Grammaire contrastive du francais et du russe}. 
Moscow: Nestor Academic Publs. 272~p.
\bibitem{37-zat-1}
\Aue{Zalizniak, Anna~A.} Lingvospetsifichnye edinitsy russkogo yazyka v~svete 
kontrastivnogo korpusnogo analiza [Russian language-specific words in light of the contrastive 
corpus analysis]. \textit{Komp'yuternaya Lingvistika i~Intellektual'nye Tekhnologii. Po 
mat-lam ezhegodnoy Mezhdunar. konf. ``Dialog-2015''} [Computational Linguistics and 
Intellectual Technologies: Conference (International) ``Dialog-2015'' Proceedings]. Moscow.  
14(21):683--695.
\end{thebibliography}

 }
 }

\end{multicols}

\vspace*{-3pt}

\hfill{\small\textit{Received July 22, 2015}}

\Contrl

\noindent
\textbf{Zatsman Igor M.} (b.\ 1952)~--- 
Doctor of Science in technology, Head of Department, Institute of Informatics Problems, Federal Research Center 
``Computer Science and Control'' of the Russian Academy of Sciences, 44-2 Vavilov Str., Moscow 119333, 
Russian Federation;  iz\_ipi@a170.ipi.ac.ru
\label{end\stat}


\renewcommand{\bibname}{\protect\rm Литература}     %8Abst+avt

\def\stat{kozerenko}

\def\tit{СЕМАНТИЧЕСКАЯ ОБРАБОТКА НЕСТРУКТУРИРОВАННЫХ ТЕКСТОВЫХ 
ДАННЫХ НА~ОСНОВЕ ЛИНГВИСТИЧЕСКОГО ПРОЦЕССОРА PullEnti}

\def\titkol{Семантическая обработка неструктурированных текстовых 
данных на~основе лингвистического процессора PullEnti}

\def\aut{Е.\,Б.~Козеренко$^1$, К.\,И.~Кузнецов$^2$, Д.\,А.~Романов$^3$}

\def\autkol{Е.\,Б.~Козеренко, К.\,И.~Кузнецов, Д.\,А.~Романов}

\titel{\tit}{\aut}{\autkol}{\titkol}

\index{Козеренко Е.\,Б.}
\index{Кузнецов К.\,И.}
\index{Романов Д.\,А.}
\index{Kozerenko E.\,B.}
\index{Kuznetsov K.\,I.}
\index{Romanov D.\,A.}




%{\renewcommand{\thefootnote}{\fnsymbol{footnote}} \footnotetext[1]
%{Работа выполнена при частичной поддержке РФФИ (проект 16-07-00677).}}


\renewcommand{\thefootnote}{\arabic{footnote}}
\footnotetext[1]{Институт проблем информатики Федерального исследовательского центра <<Информатика и~управ\-ле\-
ние>> Российской академии наук, \mbox{kozerenko@mail.ru}}
\footnotetext[2]{Институт проблем информатики Федерального исследовательского центра <<Информатика 
и~управление>> Российской академии наук, \mbox{k.smith@mail.ru}}
\footnotetext[3]{Национальный исследовательский университет <<Высшая школа экономики>>, DRomanov@it.ru}

%\vspace*{8pt}
  

  
   
     \Abst{Представлена методика создания систем извлечения 
знаний, основанная на подходе, главным инструментом которого является 
программный пакет PullEnti, включающий алгоритмы морфологического 
и~се\-ман\-ти\-ко-син\-так\-си\-че\-ско\-го анализа для выделения сущностей 
определенных типов из текстов естественного языка 
(персоны, организации, локации и~другие целевые семантические объекты). 
В~сис\-те\-ме PullEnti используются динамически подключаемые компоненты 
(плагины), что позволяет без перекомпилирования активировать различные 
функциональные воз\-мож\-ности. Именно таким образом запускается блок 
семантического анализа. В~процессе анализа выделяются семантические 
единицы (токены), которые представляют собой типизированные фразы: 
текстовые, чис\-ло\-вые и~др. Приводятся примеры реализованных проектов 
для различных предметных областей.}
     
     \KW{семантическое моделирование; извлечение именованных 
сущностей; области с~интенсивным использованием данных; 
автоматизированные сис\-те\-мы извлечения знаний; семантический поиск; 
интеллектуальные ин\-тер\-нет-тех\-но\-логии}

\DOI{10.14357/19922264180313}
  
\vspace*{-1pt}


\vskip 10pt plus 9pt minus 6pt

\thispagestyle{headings}

\begin{multicols}{2}

\label{st\stat}
     
     \section{Введение}
     
     \vspace*{-4pt}
     
     Задача автоматического анализа текстовой информации, 
представленной в~Интернете, является актуальной во всем мире. В~данной 
статье пред\-ставле\-ны результаты исследований и~разработок, на\-прав\-лен\-ных 
на решение научной проб\-ле\-мы со\-зда\-ния оптимальной методики  
ло\-ги\-ко-ста\-ти\-сти\-че\-ско\-го моделирования механизмов целевого 
семантического анализа в~информационных сис\-те\-мах с~интенсивным 
использованием знаний, выполняющих функции извлечения знаний, 
поддержки аналитических решений, в~том чис\-ле в~среде нескольких 
естественных языков. 

Для решения полного спектра задач обработки 
естественного языка создан се\-ман\-ти\-че\-ски-ори\-ен\-ти\-ро\-ван\-ный 
лингвистический процессор (СОЛП). Центральным компонентом СОЛП 
является инструментальный пакет (SDK-мо\-дуль) PullEnti. Этот процессор 
в~рамках проводимых соревнований конференции <<Диа\-лог-2016>> занял 
несколько первых мест при анализе текс\-тов в~рамках решения задач 
извлечения именованных сущностей. Разработчик PullEnti~--- Кузнецов 
Константин Игоревич. В~сис\-те\-ме PullEnti используются динамически 
подключаемые компоненты (плагины), что позволяет без 
перекомпилирования запускать различные функциональные возможности. 
Именно таким образом активируется блок семантического анализа. 
     
     В процессе анализа выделяются семантические единицы (токены), 
которые представляют собой типизированные фразы, такие как текс\-то\-вые, 
чис\-ло\-вые и~др. Например, в~результате анализа фразы <<В~2017~году>> 
будут выделены три токена: <<В>>~--- текс\-то\-вый; <<году>>~--- текстовый; 
<<2017>>~--- чис\-ло\-вой. Такие токены можно назвать прос\-ты\-ми. Кроме того, 
выделяются \textit{метатокены}~--- слож\-ные токены, которые объединяют 
несколько прос\-тых токенов, например существительные с~определителями, 
скобки, кавычки и~т.\,п.
     
     В системе существует пополняемый статический словарь терминов. 
В~него можно добавлять термины и~затем проверять их наличие в~тексте. 
Кроме того, в~сис\-те\-ме можно формировать динамически подобные словари 
на основе анализа текста.
     
     При анализе текста создается аналитический контейнер, в~который 
помещаются вы\-де\-ля\-емые сущности, токены в~определенной 
последовательности, статистические данные и~др.
     
    \section{Лингвистическое моделирование в~системах 
обработки знаний в~многоязычной среде}
     
     Способы представления информации, знаний многообразны. Огромный 
объем данных пред\-став\-лен в~виде текс\-тов естественного языка, что делает 
задачу извлечения и~структурирования информации из текстов весьма 
важной. Это относится к~различным предметным областям. Для 
оперирования данными на компьютере необходимо выделить из текста 
объекты, их атрибуты, связи между объектами, процессы, в~которых эти 
объекты задействованы, другую важ\-ную информацию, которая бы позволяла 
не только описать ситуацию, но и~строить выводы, характерные для 
конкретной предметной об\-ласти, прогнозировать развитие ситуации.
     
     Для решения поставленных задач проведены эксперименты 
с~различными грамматическими формализмами, в~том чис\-ле с~грамматикой 
категориального типа~[1]. Проведено сравнительное\linebreak
 исследование методов 
классификации применительно к~лингвистическим задачам; выработан\linebreak 
эффективный метод отображения вектора ес\-те\-ст\-вен\-но-язы\-ко\-вых 
структур в~расширенное пространство признаков для классификации новых 
языковых объектов и~структур; сформирована фокусная выборка 
параллельных текстов деловых и~научных документов на русском 
и~английском языках по различным отраслям науки и~техники;\linebreak 
сформирована расширенная система новых категорий для повышения 
изобразительных возможностей двуязычной грамматики; выработаны пути\linebreak 
расширения базовых представлений на основе аппарата расширенных 
семантических сетей~[2]\linebreak и~результатов применения метода векторных 
пространств, направленного на разрешение не\-од\-но\-знач\-ности языковых 
структур для синтаксического разбора при распознавании текста в~процессе 
извлечения знаний из текстов на разных естественных языках. Разработаны 
алгоритмы автоматического выравнивания параллельных текстов для 
развития грамматических компонент сис\-тем обработки знаний 
в~многоязычном режиме. 

%o
Основной результат исследований~--- модель 
лингвистической со\-став\-ля\-ющей интеллектуальных информационных сис\-тем, 
работающих в~многоязычном пространстве для поиска информации, 
обеспечения оптимальных аналитических и~управ\-лен\-че\-ских решений 
в~сферах деятельности с~интенсивным использованием данных. Результаты 
исследований применяются в~ло\-ги\-ко-се\-ман\-ти\-че\-ских 
и~статистических процедурах обработки слабоструктурированной текс\-то\-вой 
информации, при разработке технологии и~инструментальных средств 
построения лингвистических компонент интеллектуальных сис\-тем и~сис\-тем 
машинного перевода.
     
    \section{Представление лингвистических знаний на~основе 
векторных пространств }
    
     Процедуры анализа и~синтеза ес\-те\-ст\-вен\-но-язы\-ко\-вых высказываний 
отражают динамический характер языка как деятельности; соответственно, 
в~модели, которая кладется в~основу проекта сис\-те\-мы обработки  
ес\-те\-ст\-вен\-но-язы\-ко\-вых высказываний, дол\-жен быть заложен 
механизм, позволяющий строить пред\-став\-ле\-ния движения. 
     
     Методы машинного обучения на основе векторных моделей 
развиваются и~используются в~различных областях знаний, применительно 
к~лингвистическим задачам эти методы вполне эффективны для разрешения 
лексической мно\-го\-знач\-ности~[3--8]. 
     
     Более сложной задачей и~новым направлением исследований 
возможности применения векторных моделей для пред\-став\-ле\-ния и~обработки 
лингвистических данных является моделирование грамматических 
преобразований на основе векторных пространств и~тензоров. Тензор (от 
лат.\ \textit{tensus}, напряженный)~--- объект линейной алгебры, пре\-об\-ра\-зу\-ющий 
элементы одного линейного пространства в~элементы другого. Часто тензор 
представляют как многомерную таб\-ли\-цу, заполненную чис\-ла\-ми~--- 
компонентами тензора $d \cdot d \cdots d$, где $d$~--- раз\-мер\-ность, 
над которой задан тензор, а~чис\-ло сомножителей совпадает с~так называемой 
валентностью, или рангом тензора. Важно, что такое представление (кроме 
скаляров, т.\,е.\ тензоров валентности ноль) возможно только после выбора 
базиса (или системы координат): при смене базиса компоненты тензора 
меняются определенным образом. Сам тензор как <<геометрическая 
сущность>> от выбора базиса не зависит, компоненты вектора меняются при 
смене координатных осей, но сам вектор, образом которого может быть 
прос\-то нарисованная стрелка, от этого не изменяется. Тензор обычно 
обозначают некоторой буквой с~совокупностью верх\-них (контрвариантных) и~ниж\-них (ковариантных) индексов: $X_{j_1 j_2\ldots j_s}^{i_1i_2\ldots i_r}$. 
При смене базиса ковариантные компоненты меняются так же, как и~базис 
(с~по\-мощью того же преобразования), а~контрвариантные~--- обратно 
изменению базиса (обратным преобразованием). Тензор является сущностью 
любой системы реального мира и~сохраняется, несмотря на происходящие 
изменения в~этой системе~\cite{9-koz}. Эта особенность тензора чрезвычайно 
актуальна для моделирования языковых преобразований в~лингвистических 
процессорах, когда необходимо выявлять сходные значения, выраженные 
многочисленными способами, сис\-те\-мой разнородных языковых средств. 
     
     В работе, представленной в~данной статье, используются два основных 
подхода к~пред\-став\-ле\-нию смысла в~вы\-чис\-ли\-тель\-ной лингвистике: 
символьный подход~\cite{10-koz, 11-koz} и~подход на основе 
дистрибутивной семантики~\cite{5-koz, 7-koz, 8-koz, 9-koz}; решение 
заключается\linebreak в~сочетании методов компьютерной лингвистики и~когнитивной 
науки, в~котором символьное и~<<\textit{коннекционистское}>> (от англ.\ 
\textit{connectionist}, т.\,е.\ основанное на нейронных сетях как модели 
машинного обучения) представления объединяются с~по\-мощью тензорных 
произведений. Исследованы возможные применения данного метода для 
обработки синтаксических структур и~контекстов в~русском и~английском 
языках и~проведены межъязыковые сопоставления.
\vspace*{-3pt}
     
\section{Семантико-ориентированный лингвистический процессор}
     
     Методы, описанные выше, используются в~процедурах семантической 
обработки текстовых знаний в СОЛП, который решает задачу извлечения 
структурированной информации из текс\-тов на русском и~английском языках. 
Ядром СОЛП является программный пакет PullEnti, вклю\-ча\-ющий алгоритмы 
морфологического и~синтаксического анализа, который позволяет выделять 
сущности определенных типов из текс\-тов естественного языка (персоны, 
организации, локации и~другие семантические объекты). <<Именованная 
сущность>>~--- это объект, содержащий набор значений атрибутов, 
отличающий его от других объектов этого же типа. В~тексте находятся 
именованные сущности и~устанавливаются семантические связи между 
ними, при этом учитывается воз\-мож\-ность обозначения одной сущ\-ности 
несколькими способами (синонимия). Все множество сущностей, 
выделенных из текста или нескольких текс\-тов, представляет собой 
ориентированный граф.
     
     Предварительный этап обработки текстов включает в~себя 
морфологический и~синтаксический анализ. При морфологическом анализе 
текст разбивается на словоформы, так\-же называемые токенами (от англ.\ 
\textit{token}~--- пример использования лингвистической единицы в~тексте). 
Основными наследными классами базового класса Token являются TextToken и~MetaToken. 
TextToken~--- это исходный фрагмент текста, содержащий результат 
морфологического анализа. TextToken ссылается\linebreak на MorphToken, 
содержащий все морфологические варианты разбора. MetaToken~--- это 
группа токенов, соответствующих одной синтаксической или семантической 
группе. К~классу метатокенов относятся NumberToken, пред\-став\-ля\-ющий 
чис\-ло, и~\mbox{ReferentToken}, представляющий сущ\-ность.
     
     Приведем пример морфологического анализа предложения.
     
     Исходный текст: 
     
     <<По словам директора департамента экономического развития 
автономного округа Павла Сидорова, на эти цели планируется при\-влечь 
200~млн рублей из федерального бюджета и~еще 450~млн 
рублей из внебюджетных источников>>.
     
     В результате морфологического разбора текст был разбит на 
словоформы, для каждой словоформы указана начальная форма, часть речи 
и~морфологические характеристики. В~случае мно\-го\-знач\-ности или 
омонимии указываются все варианты морфологического разбора. 
     
     Также были выделены сле\-ду\-ющие метатокены:
     
      Павел Сидоров~--- текстовый фрагмент <<директора департамента 
экономического развития автономного округа Павла Сидорова>>
      
      200.000.000 RUB~--- текстовый фрагмент <<200~миллионов рублей>>
      
      450.000.000 RUB~--- текстовый фрагмент <<450~миллионов рублей>>
     
     Вслед за морфологическим анализом проводится выделение 
именованных сущностей различных типов. Се\-ман\-ти\-ко-ори\-ен\-ти\-ро\-ван\-ный 
лингвистический процессор извлекает из текстов 
объекты следующих типов: дата, временной период, территориальное 
образование, денежная сумма, телефон, URL, адрес, организация, транспорт, 
свойство персоны, персона, декрет, часть декрета. Каждому типу 
соответствуют свои свойства и~связи с~объектами других типов. 

\begin{figure*}[b] %fig1
  \vspace*{6pt}
 \begin{center}
 \mbox{%
 \epsfxsize=163mm 
 \epsfbox{koz-1.eps}
 }
 \end{center}
\vspace*{-9pt}
\Caption{Результаты работы программы <<Доктор Ватсон>>. Выделение сущностей}
\end{figure*}
     
     Базовым классом для сущностей является класс Referent. Тип 
сущностей задается классом Referent Class, на\-след\-ным от Referent, который 
содержит набор атрибутов. Значение может быть как прос\-тым (строка, 
число), так и~ссылкой на другую сущ\-ность. Помимо значений атрибутов 
сущность содержит список ссылок на участки исходного текс\-та, в~которых 
эта сущность располагается. Для задач, в~которых требуется обрабатывать 
множество текстов и~хранить по\-лу\-ча\-емые сущности, в~СОЛП используется 
базовый класс Repository Base, об\-лег\-ча\-ющий реализацию хранилища 
сущностей. Мес\-то хранения сериализованных данных от сущностей 
определяется в~наследном классе (например, это может быть реляционная 
система управления базами данных или файловая сис\-те\-ма). 
Repository Base берет на себя функции 
отож\-де\-ст\-вле\-ния новых данных со старыми данными и~поддержки 
не\-про\-ти\-во\-ре\-чи\-вости семантической сети.
     
Извлекаемая из текс\-та информация должна быть адресной, поэтому из 
одного и~того же текста можно извлекать совершенно различные виды 
информации, характерные для конкретной предметной об\-ласти. 
В~результате анализа текстовой информации выделяются типизированные 
объекты предметной об\-ласти. Программа PullEnti стала основой для 
построения множества сис\-тем, таких как программа <<Доктор Ватсон>>, 
система поиска экспертов, процессор BRef и~др. Программа <<Доктор 
Ватсон>> предназначена для исследования массивов текс\-то\-вой информации с~целью выявления сущностей и~связей между ними. При этом пользователь 
может добавить недостающие сущности и~связи (которые не были выделены 
программой), настроить выдаваемую информацию, сформировать отчет 
о~результатах работы программы. Данная программа может использоваться в~таких сферах деятельности, как криминалистика, конкурентная разведка, 
маркетинг, реклама, безопас\-ность. Результат работы программы~--- отчет об 
исследуемом объекте, диаграммы сущностей и~связей~--- пред\-став\-лен на 
рис.~1. Из текущего текста выделены организации, персоны, их связи.
   
   На рис.~2 представлены выделенные объекты, их связи. Для каждой связи 
выделяется тип связи и~название (например, тип связи~--- <<родственные>>, 
заголовок связи~--- <<отец>>; тип связи~--- <<владение>>, заголовок  
связи~--- <<особняк в~центре Вашингтона>> и~т.\,д.), определяются попарно 
объекты-участники связи. Для более полного определения ситуации 
выделяется не только время, характеризующее текущую ситуацию, но 
и~интервалы времени. Дополнительные параметры поз\-во\-ля\-ют выяснить, 
является ли связь симметричной для данной пары выделенных объектов 
(субъектов). Также для каж\-до\-го выделенного объекта (субъекта) выделяются 
атрибуты. Например, для типа объекта <<персона>> выделяются имя, 
фамилия, отчество, дата рождения и~др. 


   Результаты работы программы могут быть пред\-став\-ле\-ны в~виде графа 
(вкладка <<Диаграммы>>) (см.\ рис.~3). В~отчете выводятся обнаруженные 
объекты (персоны, организации, локации, атрибуты), их связи в~удоб\-ном для 
анализа виде.


   
   Программа <<Логика ECM. Правовая экспертиза>> предназначена для 
автоматизации процесса\linebreak проведения экспертизы проектов  
нор\-ма\-тив\-но-пра\-во\-вых актов,  
ор\-га\-ни\-за\-ци\-он\-но-рас\-по\-ря\-ди\-тель\-ных документов, договоров 
и~других документов. Сис\-те\-ма значительно упрощает процесс проведения 
правовой экспертизы и~сокращает его сроки, выполняя рутинные операции 
и~кардинально снижая за\-тра\-ты рабочего времени квалифицированных 
юристов. Сис\-те\-ма <<Логика ECM. Правовая экспер-\linebreak\vspace*{-12pt}

\pagebreak

\end{multicols}

\begin{figure*} %fig2
\vspace*{1pt}
 \begin{center}
 \mbox{%
 \epsfxsize=163mm 
 \epsfbox{koz-2.eps}
 }
 \end{center}
\vspace*{-11pt}
\Caption{Выделение связей, периодов}
%\end{figure*}
%\begin{figure*} %fig3
\vspace*{6pt}
 \begin{center}
 \mbox{%
 \epsfxsize=163mm 
 \epsfbox{koz-3.eps}
 }
 \end{center}
\vspace*{-11pt}
\Caption{Графическое представление результатов работы программы <<Доктор 
Ватсон>>}
\vspace*{-2pt}
\end{figure*}
   

\begin{multicols}{2}

\noindent
тиза>> автоматически за 
несколько секунд поможет, например, установить:
   \begin{itemize}
   \item не содержатся ли в~проверяемом документе ссылки на нормативные 
правовые акты, которые утратили силу;
   \item нет ли в~проверяемом документе фрагментов других документов, не 
возникает ли избыточное дублирование нормативной документации;
   \item соответствует ли оформление и~структура документа уста\-нов\-лен\-ным 
   в~организации правилам;\\[-13pt]
   \item нет ли ошибок в~оформлении цифровой информации в~договоре, 
соответствуют ли друг другу суммы, указанные цифрами и~про\-писью, 
правильно ли рассчитан налог на добавленную стоимость и~т.\,п. На основе лингвистического процессора 
PullEnti был реализован процессор обработки ссылок и~списка литературы 
BREF,\linebreak
\vspace*{-12pt}

\pagebreak

\noindent
который позволяет по выделенной информации по\-строить \textit{Граф 
Цитирования} и~\textit{Граф Соавторов}, отражающие формальные связи 
в~коллекции документов.
   \end{itemize}
    
   Лингвистический процессор PullEnti под псевдонимом Pink на 
соревновании FactRuEval конференции <<Диа\-лог-2016>> занял первые места 
на большинстве дорожек~\cite{12-koz}.
   
   Соревнование проводилось на следующих дорожках:
   \begin{itemize}
   \item определение в~тексте границ именованных сущностей, таких как 
персона, организация, локация;
   \item выделение именованных сущностей с~определением атрибутов 
в~нормализованном виде. Для персон это фамилия, имя и~отчество. Для 
организаций и~локаций~--- нормализованное название;
   \item извлечение фактов (например: <<встреча>>, <<покупка>>, $\ldots$) 
и~наборов строковых полей (например: <<участник встречи~1>>, 
<<участник встречи~2>>, <<место встречи>>, <<да\-та/вре\-мя начала 
встречи>>, $\ldots$).
   \end{itemize}
   
   \vspace*{-13pt}
   
  \section{Заключение}
  \vspace*{-3pt}
  
   Лингвистические процессоры на основе программы PullEnti могут быть 
использованы в~различных областях, в~которых информация пред\-став\-ле\-на 
в~текс\-то\-вом виде. Особенно это важно в~тех случаях, когда необходимо 
выделять важ\-ную информацию из большого потока документов на 
естественном языке. Очень хорошо данная технология работает в~задачах 
кластеризации текстов по определенным признакам. При этом существует 
воз\-мож\-ность автоматической настройки программы на требования 
пользователя.
   
   Описанные выше системы, созданные на основе технологии PullEnti, 
доказывают ее эффективность в~самых различных областях. Сле\-ду\-ющи\-ми 
шагами исследований станут: методы уточнения границ мо\-де\-ли\-ру\-емых 
предметных областей за счет построения семантического ядра каж\-дой 
области (в~том чис\-ле с~использованием методов вероятностного 
тематического моделирования); выделение массива неявных ссылок 
(упоминаний персон и~идей, выраженных ключевыми фра\-за\-ми/зна\-чи\-мы\-ми 
словосочетаниями); расчет корреляции между явными и~неявными ссылками 
в~рамках созданной коллекции, формирование  
функ\-цио\-на\-ль\-но-грам\-ма\-ти\-че\-ских модулей естественных языков, 
включаемых в~лингвистический процессор.

    
{\small\frenchspacing
 {%\baselineskip=10.8pt
 \addcontentsline{toc}{section}{References}
 \begin{thebibliography}{99}

 \vspace*{-4pt}

\bibitem{1-koz}
\Au{Shaumyan S.} Categorial grammar and semiotic universal grammar~// 
Conference (International) on Artificial Intelligence Proceedings.~--- 
Las Vegas, NV, USA: CSREA Press, 2003. 
P.~623--629.

\bibitem{2-koz}
\Au{Kuznetsov I.\,P., Kozerenko~E.\,B., Matskevich~A.\,G.} 
Intelligent extraction of knowledge structures from natural language texts~// 
IEEE/WIC/ACM Joint Conferences (International) on Web Intelligence and 
Intelligent Agent Technology Proceedings.~---
Washington, DC, USA: IEEE Computer Society, 2011. Vol.~3. P.~269--272.

\bibitem{3-koz}
\Au{Dempster A.\,P., Laird N.\,M., Rubin~D.\,B.} 
Maximum likelihood from incomplete data via the EM algorithm~// 
J.~Roy. Stat. Soc.~B, 1977. Vol.~39. Iss.~1. P.~1--22.

\bibitem{7-koz} %4
\Au{Lund K., Burgess~C.} Producing high-dimensional semantic spaces from 
lexical co-occurrence~// 
Behav. Res. Meth. Ins. C., 1996. Vol.~28. Iss.~2. 
P.~203--208.

\bibitem{5-koz} %5
\Au{Curran J.\,R.} From distributional to semantic similarity.~--- 
Edinburgh: University of 
Edinburgh, 2004. PhD Thesis. 177~p.
{\sf https://www.inf.ed.ac.uk/publications/thesis/ online/IP030023.pdf}

\bibitem{8-koz} %6
\Au{McCarthy D., Koeling R., Weeds~J., Carroll~J.} Finding predominant 
senses in untagged text~// 42nd Annual Meeting of the Association for 
Computational Linguistics Proceedings.~---
Stroudsburg, PA, USA: Association for 
Computational Linguistics, 2004. P.~280--287. doi: 10.3115/1218955.1218991.

\bibitem{4-koz} %7
\Au{Clark S., Pulman S.} Combining symbolic and distributional models of 
meaning~// AAAI Spring Symposium on Quantum Interaction Proceedings.~--- 
Palo Alto, CA, USA: AAAI Press, 2007. 4~p. {\sf 
http://www.cl.cam.ac.uk/ $\sim$sc609/pubs/aaai07.pdf.}

\bibitem{6-koz} %8
\Au{Kozerenko E.\,B.} Parallel texts alignment strategies~// 
\textit{Conference (International) on Artificial Intelligence Proceedings}.~--- 
Las Vegas, NV, USA: CSREA Press, 
2012. Vol.~2. P.~945--951.

\bibitem{9-koz}
\Au{Danielson D.\,A.} Vectors and tensors in engineering and physics.~--- 
2nd ed.~--- Boulder, CO, USA: Westview (Perseus), 2003. 287~p.

\bibitem{11-koz} %10
\Au{Montague R.} Universal grammar~// Theoria, 1970. Vol.~36. P.~373--398. 
(Reprinted in: Formal philosophy: Selected 
papers of Richard Montague~/ 
Ed. R.\,H.~Thomason.~--- 
New Haven, CT, USA: Yale University Press, 1974. P.~7--27.)

\bibitem{10-koz}
\Au{Pang B., Knight K., Marcu~D.} Syntax-based alignment of multiple translations: 
Extracting paraphrases and generating new sentences~// 
Conference of the North 
American Chapter of the Association for Computational Linguistics on Human Language 
Technology Proceedings.~--- 
Stroudsburg, PA, USA: Association for Computational Linguistics.
2003. Vol.~1. P.~102--109. doi: 10.3115/1073445.1073469.

\bibitem{12-koz}
FACRUEVAL. Evaluation of named entity recognition and fact extraction systems for 
Russian, 2016. {\sf http:// ww.dialog-21.ru/media/3430/starostinaetal.pdf.}
 \end{thebibliography}

 }
 }

\end{multicols}

%\vspace*{-12pt}

\hfill{\small\textit{Поступила в~редакцию 13.07.18}}

%\vspace*{-36pt}

\pagebreak

\vspace*{-36pt}

%\hrule

%\vspace*{2pt}

%\hrule

\vspace*{-2pt}


\def\tit{SEMANTIC PROCESSING OF~UNSTRUCTURED TEXTUAL DATA BASED 
ON~THE~LINGUISTIC PROCESSOR PullEnti}


\def\titkol{Semantic processing of~unstructured textual data based 
on~the~linguistic processor PullEnti}


\def\aut{E.\,B.~Kozerenko$^1$, K.\,I.~Kuznetsov$^1$, and~D.\,A.~Romanov$^2$}

\def\autkol{E.\,B.~Kozerenko, K.\,I.~Kuznetsov, and~D.\,A.~Romanov}

\titel{\tit}{\aut}{\autkol}{\titkol}

\vspace*{-11pt}


\noindent
$^1$Institute of Informatics Problems, Federal Research Center ``Computer Science 
and Control'' of the Russian 
$\hphantom{^1}$Academy of Sciences,  44-2~Vavilov Str., Moscow 119333, 
Russian Federation

\noindent
$^2$National Research University ``Higher School of Economics,'' 
20~Myasnitskaya Str., Moscow 101000, Russian 
$\hphantom{^1}$Federation


\def\leftfootline{\small{\textbf{\thepage}
\hfill INFORMATIKA I EE PRIMENENIYA~--- INFORMATICS AND
APPLICATIONS\ \ \ 2018\ \ \ volume~12\ \ \ issue\ 3}
}%
 \def\rightfootline{\small{INFORMATIKA I EE PRIMENENIYA~---
INFORMATICS AND APPLICATIONS\ \ \ 2018\ \ \ volume~12\ \ \ issue\ 3
\hfill \textbf{\thepage}}}

\vspace*{3pt}



\Abste{The paper presents the method for creation of knowledge extraction 
systems based on the approach employing the software tool system 
PullEnti comprising the algorithms for morphological and semantic-syntactical 
analysis which makes it possible to extract entities of certain types 
from natural language texts (persons, organizations, locations, and other 
target semantic objects). The PullEnti system uses dynamically connected 
components (plugins) which makes it possible to activate various functions 
without recompiling. This is how the semantic analysis unit is incorporated. 
During the analysis, the semantic units (tokens) are established, which are 
typed phrases: text, numerical data, etc. 
Examples of implemented projects for different subject areas are given.}

\KWE{semantic modeling; named entities recognition, data intensive domains; 
automated systems of knowledge extraction; semantic search; intelligent Internet 
technologies}




\DOI{10.14357/19922264180313} %

%\vspace*{-14pt}

%\Ack
%\noindent



%\vspace*{6pt}

  \begin{multicols}{2}

\renewcommand{\bibname}{\protect\rmfamily References}
%\renewcommand{\bibname}{\large\protect\rm References}

{\small\frenchspacing
 {%\baselineskip=10.8pt
 \addcontentsline{toc}{section}{References}
 \begin{thebibliography}{99}
     
\bibitem{1-koz-1}
 \Aue{Shaumyan, S.} 2003. Categorial grammar and semiotic universal grammar.  
\textit{Conference (International) on Artificial Intelligence Proceedings}. 
Las Vegas, NV: CSREA Press. 623--629.

\bibitem{2-koz-1}
\Aue{Kuznetsov, I.\,P., E.\,B.~Kozerenko, and A.\,G.~Matskevich.} 
2011. Intelligent extraction of knowledge structures from natural language texts. 
\textit{IEEE/WIC/ACM Conferences (International) on Web Intelligence and 
Intelligent Agent Technology Proceeding}. 
Washington, DC: IEEE Computer Society. 3:269--272. doi: 10.1109/WI-IAT.2011.235.

\bibitem{3-koz-1}
\Aue{Dempster, A.\,P., N.\,M.~Laird, and D.\,B.~Rubin.} 1977. Maximum likelihood 
from incomplete data via the EM algorithm. 
\textit{J.~Roy. Stat. Soc.~B} 39(1):1--22.

 \bibitem{7-koz-1} %4
\Aue{Lund, K., and C.~Burgess.} 1996. Producing high-dimensional semantic spaces 
from lexical co-occurrence. \textit{Behav. Res. Meth. Ins. C.}  
28(2):203--208.

\bibitem{5-koz-1} %5
\Aue{Curran, J.\,R.} 2004. From distributional to semantic similarity. Edinburgh: 
University of Edinburgh. PhD Thesis. 177~p.  Available at: 
{\sf https://www.inf.ed.ac.uk/\linebreak publications/thesis/online/IP030023.pdf} 
(accessed July~19, 2018).

 \bibitem{8-koz-1} %6
\Aue{McCarthy, D., R.~Koeling, J.~Weeds, and J.~Carroll.} 2004. 
Finding predominant senses in untagged text. 
\textit{42nd Annual Meeting of Association for Computational Linguistics Proceedings}. 
Stroudsburg, PA: Association for 
Computational Linguistics. 280--287. doi: 10.3115/1218955.1218991.

\bibitem{4-koz-1}%7
\Aue{Clark, S., and S.~Pulman.} 2007. Combining symbolic and distributional 
models of meaning. \textit{AAAI Spring Symposium on Quantum Interaction Proceedings}. 
Palo Alto, CA: AAAI Press. 4~p. Available at: 
{\sf http://www.cl.cam.ac.uk/ $\sim$sc609/pubs/aaai07.pdf} (accessed July~19, 2018).

 \bibitem{6-koz-1} %8
 \Aue{Kozerenko, E.\,B.} 2012. Parallel texts alignment strategies. 
\textit{Conference (International) on Artificial Intelligence Proceedings}. 
Las Vegas, NV: CSREA Press. 2:945--951.

\bibitem{9-koz-1}
\Aue{Danielson, D.\,A.} 2003. \textit{Vectors and tensors in engineering and 
physics.} 2nd ed. Boulder, CO: Westview Press. 287~p.

\bibitem{11-koz-} %10
\Aue{Montague, R.} 1970. Universal grammar. \textit{Theoria} 36:373--398. 
(Reprinted in: 1974.
\textit{Formal philosophy: Selected papers of Richard Montague}. 
Ed. R.\,H.~Thomason. New Haven, CT: Yale University Press. 7--27.)

\bibitem{10-koz-1} %11
\Aue{Pang, B., K.~Knight, and D.~Marcu.} 2003. Syntax-based alignment of 
multiple translations: Extracting paraphrases and generating new sentences. 
\textit{Conference of the North American Chapter of the Association for 
Computational Linguistics on Human Language Technology Proceedings}. 
Stroudsburg, PA: Association 
for Computational Linguistics. 1:102--109. doi: 10.3115/1073445.1073469.

\bibitem{12-koz-1}
FACRUEVAL. 2016. Evaluation of named entity recognition and fact extraction 
systems for Russian. Available at: {\sf http://www.dialog-21.ru/media/3430/\linebreak starostinaetal.pdf} 
(accessed July~19, 2018).

\end{thebibliography}

 }
 }

\end{multicols}

\vspace*{-6pt}

\hfill{\small\textit{Received July 13, 2018}}

\pagebreak

%\vspace*{-18pt}
     
     \Contr
     
     \noindent
     \textbf{Kozerenko Elena B.} (b.\ 1959)~--- Candidate of Science (PhD) in linguistics, leading scientist, 
Institute of Informatics Problems, Federal Research Center ``Computer Science and Control'' of the Russian 
Academy of Sciences,  44-2 Vavilov Str., Moscow 119333, Russian Federation; \mbox{kozerenko@mail.ru} 
      
       \vspace*{6pt}
      
     \noindent
       \textbf{Kuznetsov Konstantin I.} (b.\ 1968)~--- leading engineer, Institute 
of Informatics Problems, Federal Research Center ``Computer Science and 
Control'' of the Russian Academy of Sciences,  44-2~Vavilov Str., Moscow 
119333, Russian Federation; \mbox{k.smith@mail.ru} 
       
       \vspace*{6pt}
       
     \noindent
       \textbf{Romanov Dmitri A.} (b.\ 1967)~--- Candidate of Science (PhD) in 
technology, associate professor, National Research University ``Higher School of 
Economics,'' 20~Myasnitskaya Str., Moscow 101000, Russian Federation; 
\mbox{DRomanov@it.ru} 

\label{end\stat}

\renewcommand{\bibname}{\protect\rm Литература}       
          %9Abst+avt
\def\stat{morozova}

\def\tit{ПОСТРОЕНИЕ СЕМАНТИЧЕСКИХ ВЕКТОРНЫХ 
ПРОСТРАНСТВ РАЗЛИЧНЫХ ПРЕДМЕТНЫХ 
ОБЛАСТЕЙ$^*$}

\def\titkol{Построение семантических векторных 
пространств различных предметных 
областей}

\def\autkol{Ю.\,И.~Морозова}

\def\aut{Ю.\,И.~Морозова$^1$}

\titel{\tit}{\aut}{\autkol}{\titkol}

{\renewcommand{\thefootnote}{\fnsymbol{footnote}}\footnotetext[1]
{Работа выполнена при частичной поддержке РФФИ (проект 11-06-00476-а).}}

\renewcommand{\thefootnote}{\arabic{footnote}}
\footnotetext[1]{Институт проблем информатики Российской академии наук, yulia-ipi@yandex.ru}


\Abst{Данная работа посвящена актуальным проблемам исследования семантики 
лингвистических единиц с использованием корпусных методов. В~работе дается 
описание нового направления лингвистических исследований~--- дистрибутивной 
семантики. Предлагается расширение существующих моделей дистрибутивной 
семантики за счет перехода от описания лексем к описанию значимых словосочетаний. 
Описывается методика построения семантических векторных пространств (СВП) для 
различных предметных областей.}

\KW{дистрибутивная семантика; векторные пространства; значимые словосочетания; 
коллокации}

\vskip 14pt plus 9pt minus 6pt

      \thispagestyle{headings}

      \begin{multicols}{2}

            \label{st\stat}

\section{Обзор моделей дистрибутивной семантики}

Дистрибутивная семантика~--- об\-ласть научных исследований, 
занимающаяся вычислением степени семантической близости между 
лингвистическими единицами на основании их дистрибутивных 
(контекстных) признаков в больших массивах лингвистических данных. 
Модели векторных пространств находят все более широкое применение в 
исследованиях, связанных с семантическими моделями естественного 
языка, и имеют разнообразный спектр потенциальных и действующих 
приложений. Основными сферами применения дистрибутивных моделей 
являются: разрешение лексической неоднозначности, информационный 
поиск,\linebreak кластеризация документов, автоматическое формирование словарей 
(словарей семантических отноше\-ний, двуязычных словарей), создание 
семантических карт, моделирование перифраз, определение тематики 
документа, определение тональности высказывания, биоинформатика. 

Теоретические основы данного направления восходят к дистрибутивной 
методологии З.~Харриса~[1, 2]. Близкие идеи выдвигали 
основоположники структурной лингвистики Ф.~де~Сос\-сюр и 
Л.~Витгенштейн. Дистрибутивная семантика основывается на 
дистрибутивной гипотезе о том, что лингвистические элементы со схожей 
дистрибуцией имеют близкие значения~[3, 4]. 

В качестве вычислительного инструмента и спосо\-ба представления 
моделей используется линейная алгебра. Информация о дистрибуции 
лингви\-сти\-че\-ских единиц представляется в виде многоразрядных векторов, 
а семантическая близость между лингви\-сти\-че\-ски\-ми единицами 
вы\-чис\-ля\-ет\-ся как расстояние между векторами. Много\-разрядные векторы 
образуют матрицу, где каждый\linebreak вектор соответствует лингвистической 
единице (слово или словосочетание), а каждое измерение вектора 
соответствует контексту (документ, параграф, предложение, 
словосочетание, слово).

Для вычисления меры близости между векторами могут использоваться 
различные формулы: расстояние Минковского, расстояние Манхеттена, 
евклидово расстояние, расстояние Чебышёва, скалярное произведение, 
косинусная мера. Наиболее популярной является косинусная мера:
$$
\fr{x\bullet y}{\vert x\vert\bullet\vert y\vert }= \fr{\sum\limits_{i=1}^n 
x_i\bullet y_i}{\sqrt{\sum\limits_{i=1}^n 
x_i^2}\bullet\sqrt{\sum\limits_{i=1}^n y_i^2}}\,.
$$

Существует множество разновидностей моделей дистрибутивной 
семантики, которые различаются по следующим параметрам:
\begin{itemize}
\item тип контекста (размер контекста, правый или левый контекст, 
ранжирование);
\item количественная оценка частоты встречаемости слова в данном 
контексте (абсолютная частота, энтропия, совместная информация и~пр.); 
\item метод вычисления расстояния между векторами (косинус, скалярное 
произведение, расстояние Минковского и~пр.);
\item метод уменьшения размерности матрицы (случайная проекция, 
сингулярное разложение и~пр.).
\end{itemize}

Наиболее известными моделями дистрибутивной семантики являются 
латентный семантический анализ, разработанный для решения проблемы 
синонимии при информационном поиске~[5], и модель языка как 
гиперпространства, разработанная как модель семантической памяти 
человека~[6].

Концепция СВП впервые была 
реализована в ин\-фор\-ма\-ци\-он\-но-поиско\-вой системе SMART~[7].\linebreak 
Идея СВП состоит в представлении каждого документа из коллекции в 
виде точки в пространстве, т.\,е.\ вектора в векторном пространстве. Точки,\linebreak 
расположенные ближе друг к другу в этом пространстве, считаются более 
близкими по смыслу. Пользовательский запрос рассматривается как 
псевдодокумент и тоже представляется как точка в этом же пространстве. 
Документы сортируются в порядке возрастания расстояния, т.\,е.\ в 
порядке уменьшения семантической близости от запроса, и в таком виде 
предоставляются пользователю. 

Впоследствии концепция СВП была успешно применена для других 
семантических задач. Например, в работе~[8] контекстное векторное 
пространство было использовано для оценки семантической близости 
слов. Данная сис\-те\-ма достигла результата 92,5\% на тесте по выбору 
наиболее подходящего синонима из стандартного теста английского языка 
TOEFL, в то время как средний результат при прохождении теста 
человеком был 64,5\%.

В настоящее время ведутся активные исследования по унификации модели 
СВП и выработке общего подхода к различным задачам выявления 
семантических связей из корпусов текстов~[9].

\section{Выделение значимых словосочетаний}

Целью данной работы является применение модели СВП для построения 
концептуальных моделей различных предметных областей. Развитие 
существующих подходов к построению СВП заключается в использовании 
значимых словосочетаний (ЗС) вместо отдельных лексем. Под ЗС
 понимаются лексические последовательности, 
имеющие тенденцию к совместной встречаемости. 
     
     В лингвистике для обозначения ЗС
используется также термин <<коллокация>>. Этот термин был впервые 
введен в <<Словаре лингвистических терминов>> О.\,С.~Ахмановой~[10]. 
Исследованиям коллока\-ций русского языка посвящено большое 
количество литературы, например монография Е.\,Г.~Борисовой~[11]. 
В~теоретической лингвистике под коллокациями понимают 
словосочетания из двух или более слов, которые обусловливают друг друга 
семантически и грамматически~[12]. В~корпусной лингвистике 
коллокациями называют статистически устойчивые словосочетания, 
причем они могут быть как фразеологизированными, так и свободными.
     
     Для выделения значимых словосочетаний в компьютерной 
лингвистике используются различные статистические меры (меры 
ассоциации, меры ассоциативной связанности, \textit{англ.}\ association measures), 
вычисляющие силу связи между элементами в составе коллокации. 
В~литературе упоминается несколько десятков мер ассоциативной 
свя\-зан\-ности. Чаще других используются MI, t-score и log-likelihood~[13].

Мера MI (mutual information), введенная в работе~\cite{14-mor}, сравнивает 
зависимые кон\-текст\-но-свя\-зан\-ные час\-то\-ты с независимыми 
частотами слов в тексте. Если значение MI превосходит определенное 
пороговое значение, то словосочетание считают статистически значимым. 
Мера MI вычисляется по следующей формуле:
$$
\mathrm{MI}=\log_2\fr{f(n,c)\times N}{f(n)\times f)c)}\,,
$$
где $n$~--- первое слово словосочетания; $c$~--- второе слово 
словосочетания; $f(n,c)$~--- частота совместной встречаемости двух слов; 
$f(n)$, $f(c)$~--- абсолютные частоты встречаемости каждого слова по 
отдельности; $N$~--- общее число словоупотреблений в корпусе.

Мера t-score также используется при ответе на вопрос, насколько не 
случайным является сочетание двух или более слов в тексте. Для 
вычисления t-score используется следующая формула:
$$
\mathrm{t}\mbox{-}\mathrm{score}=\fr{f(n,c)-f(n)\times 
f(c)/N}{\sqrt{f(n,c)}}\,.
$$

     Также достаточно часто применяется мера, известная под названием 
log-likelihood, или логарифмическая функция правдоподобия, введенная в\linebreak 
работе~\cite{15-mor}. Для вычисления log-likelihood применяется 
следующая формула:
     $$
     \mathrm{log}\mbox{-}\mathrm{likelihood}=2\sum f(n,c)\times 
\log_2\fr{f(n,c)\times N}{f(n)\times f(c)}\,.
     $$

Применив различные меры ассоциативной связанности слов к материалам 
научных патентов, авторы составили частотный словарь значимых 
словосочетаний для предметной области научных\linebreak патентов. Примеры 
выделенных значимых словосочетаний: \textit{благородный металл}, 
\textit{вспомогательное устройство}, \textit{жесткий элемент}, 
\textit{измерительная ячейка}, \textit{опорный карниз}, \textit{оптический 
луч}, \textit{система охлаж\-де\-ния}, \textit{тяжелая фракция}.

\section{Построение семантического векторного пространства}

Модель семантического векторного пространства, которую планируется 
построить в ходе данного исследования, обладает следующими 
характеристиками:
\begin{itemize}
\item тип изучаемых единиц: значимые словосочетания;
\item тип контекста: лексемы и словосочетания, размер контекста~--- 
предложение, ранжирование контекста~--- нет;
\item количественная оценка частоты встречаемости изучаемой единицы в 
данном контексте: абсолютная частота;
\item метод вычисления расстояния между векторами: косинусная мера.
\end{itemize}

Приведем пример использования методики построения СКП на основе 
следующего текстового фрагмента:

\noindent
\textit{Искусственный интеллект~--- наука и технология создания 
интеллектуальных машин, особенно интеллектуальных компьютерных 
программ.}

\noindent
\textit{Компьютерная лингвистика~--- направление искусственного 
интеллекта, которое ставит своей целью использование математических 
моделей для описания естественных языков}. 

\noindent
\textit{Дискретная математика~--- область математики, занимающаяся 
изучением дискретных структур, которые возникают как в пределах 
самой математики, так и в ее приложениях.}
\noindent
\textit{Конструктивная математика~--- близкое к интуиционизму течение 
в математике, изучающее конструктивные построения.}

Построим контекстные векторы для ЗС <<\textit{искусственный 
интеллект}>>, <<\textit{компьютерная лингвистика}>>, 
<<\textit{дискретная математика}>>, <<\textit{конструктивная 
математика}>> и слов, встречающихся в текстовом фрагменте более 
одного раза. В~таблице используются сокращенные обозначения: 
$c_1$~--- искусственный интеллект;  $c_2$~--- компьютерная 
лингвистика; $c_3$~--- дискретная математика; $c_4$~--- конструктивная 
математика; $c_5$~--- интеллект, интеллектуальный; $c_6$~--- 
математика, математический; $c_7$~--- изучать, изучение.

    Применив формулу вычисления косинусной меры между 
контекстными векторами, получим\linebreak\vspace*{-12pt}


\begin{center}
\vspace*{1pt}
\begin{tabular}{|c|c|c|c|c|c|}
\hline
&$c_1$&$c_2$&$c_3$&$c_4$&$\ldots$\\
\hline
$c_1$&0&1&0&0&$\ldots$\\
$c_2$&1&0&0&0&$\ldots$\\
$c_3$&0&0&0&0&$\ldots$\\
$c_4$&0&0&0&0&$\ldots$\\
$\ldots$&$\ldots$&$\ldots$&$\ldots$&$\ldots$&$\ldots$\\
\hline
\end{tabular}
\vspace*{6pt}
\end{center}

\noindent
 следующие коэффициенты 
семантической бли\-зости между рассматриваемыми ЗС:

<<\textit{дискретная математика}>> и <<\textit{конструктивная 
математика}>>~--- 0,95;

<<\textit{искусственный интеллект}>> и <<\textit{компьютерная 
лингвистика}>>~--- 0,7;

<<\textit{компьютерная лингвистика}>> и <<\textit{дискретная 
математика}>>~--- 0,52;

<<\textit{компьютерная лингвистика}>> и <<\textit{конструктивная 
математика}>>~--- 0,4;

<<\textit{искусственный интеллект}>> и <<\textit{дискретная 
математика}>>~--- 0,36;

<<\textit{искусственный интеллект}>> и <<\textit{конструктивная 
математика}>>~--- 0,29.

\section{Заключение}

В работе были рассмотрены основные направления и модели нового 
направления исследований в компьютерной лингвистике~--- 
дистрибутивной семантики. На основании автоматической обработки 
больших массивов лингвистических данных возможно создавать 
различные лингвистические ресурсы: семантические словари, 
многоязычные словари, семантические карты предметных областей. 
В~качестве математической модели используются многоразрядные 
векторы и мат\-ри\-цы линейной ал\-геб\-ры, что представляет собой 
удобный формализм для компьютерной реализации. В~рамках данного 
направления предлагается разработать методику построения 
семантических векторных пространств для различных предметных 
областей, где в качестве изучаемых единиц будут выступать значимые 
словосочетания, выделенные из текстов с использованием мер 
ассоциативной связанности слов. 

{\small\frenchspacing
{%\baselineskip=10.8pt
\addcontentsline{toc}{section}{Литература}
\begin{thebibliography}{99}

\bibitem{1-mor}
\Au{Harris Z.\,S.} Papers in structural and transformational linguistics.~--- 
Dordrecht: Reidel, 1954.
\bibitem{2-mor}
\Au{Harris Z.\,S.} Mathematical structures of language.~--- New York: John 
Wiley \& Sons, 1968.
\bibitem{3-mor}
\Au{Sahlgren M.} The distributional hypothesis~// From context to meaning: 
Distributional models of the lexicon in linguistics and cognitive science (Special 
issue of the Italian Journal of Linguistics).~--- Pisa: Pacini Editore, 2008. 
Vol.~20. No.\,1. P.~33--53.
\bibitem{4-mor}
\Au{Turney P.\,D., Pantel P.} From frequency to meaning: Vector space models 
of semantics~// J.~Artificial Intelligence Research.~--- Menlo Park, California: 
AAAI Press, 2010. No.\,37. P.~141--188.
\bibitem{5-mor}
\Au{Landauer Th.\,K., McNamara D.\,S., Dennis~S., Kintsch~W.} Handbook of 
Latent Semantic Analysis.~--- Mahwah, NJ: Lawrence Erlbaum, 2007.
\bibitem{6-mor}
\Au{Lund K., Burgess C.} Producing high-dimensional semantic spaces from 
lexical co-occurrence~// Behavior Research Methods, Instruments \& 
Computers.~--- New York: Psychonomic Society, 1996. Vol.~28. No.\,2. 
P.~203--208.
\bibitem{7-mor}
\Au{Salton G.\,M.} The SMART retrieval system: Experiments in automatic 
document processing.~--- Eaglewood Cliffs, NJ: Prentice-Hall, 1971.
\bibitem{8-mor}
\Au{Rapp R.} Word sense discovery based on sense descriptor dissimilarity~// 
9th MT Summit Proceedings.~--- New Orleans, LA, 2003. P.~315--322. {\sf 
http://www.amtaweb.org/ summit/MTSummit/FinalPapers/19-Rapp-final.pdf}.
\bibitem{9-mor}
\Au{Turney P.} A~uniform approach to analogies, synonyms, antonyms and 
associations~// 22nd Conference (International) on Computational Linguistics 
(\mbox{COLING}) Proceedings, 2008. P.~905--912. {\sf 
http://www.aclweb. org/anthology-new/C/C08/C08-1114.pdf}.
\bibitem{10-mor}
\Au{Ахманова О.\,С.} Словарь лингвистических терминов.~--- М.: 
Советская энциклопедия, 1966.
\bibitem{11-mor}
\Au{Борисова Е.\,Г.} Коллокации. Что это такое и как их изучать.~--- 2-е 
изд., стер.~--- М.: Филология, 1995.
\bibitem{12-mor}
\Au{Иорданская Л.\,Н., Мельчук И.\,А.} Смысл и сочетаемость в 
словаре.~--- М.: Языки славянских культур, 2007.
\bibitem{13-mor}
\Au{Захаров В.\,П., Хохлова М.\,В.} Анализ эффективности статистических 
методов выявления коллокаций в текстах на русском языке~// 
Компьютерная лингвистика и интеллектуальные технологии: Труды 
Междунар. конф. Диалог'2010.~--- М.: РГГУ, 2010.
\bibitem{14-mor}
\Au{Church K., Hanks P.} Word association norms, mutual information, and 
lexicography~// Computational Linguistics,
1996. Vol.~16. No.\,1. P.~22--29.

\label{end\stat}

\bibitem{15-mor}
\Au{Dunning T.} Accurate methods for the statistics of surprise and 
coincidence~// Computational Linguistics,
1993. Vol.~19. No.\,1. P.~61--74.
\end{thebibliography}
}
}

\end{multicols}    %10Abst+avt

\def\stat{kuzn}

\def\tit{СВЯЗЬ МЕЖДУ ВРЕМЕННЫМИ 
И~СТРУКТУРНО-ТОПОЛОГИЧЕСКИМИ ХАРАКТЕРИСТИКАМИ 
ДИАГРАММ РИТМА СЕРДЦА ЗДОРОВЫХ ЛЮДЕЙ}

\def\titkol{Связь между временными 
и~структурно-топологическими характеристиками 
диаграмм ритма сердца здоровых людей}

\def\autkol{А.\,А.~Кузнецов}
\def\aut{А.\,А.~Кузнецов$^1$}

\titel{\tit}{\aut}{\autkol}{\titkol}

%{\renewcommand{\thefootnote}{\fnsymbol{footnote}}\footnotetext[1]
%{Исследование поддержано грантами РФФИ 08-07-00152 и 09-07-12032.
%Статья написана на основе материалов доклада, представленного на IV 
%Международном семинаре  <<Прикладные задачи теории вероятностей и математической статистики, 
%связанные с моделированием информационных систем>> (зимняя сессия, Аоста, Италия, январь--февраль 2010~г.).}}

\renewcommand{\thefootnote}{\arabic{footnote}}
\footnotetext[1]{Владимирский государственный университет, artemi-k@mail.ru}

\vspace*{12pt}

\Abst{По данным 628~регистраций электрокардиограмм (ЭКГ) у~177~здоровых и больных людей проведен 
сравнительный анализ параметров реальной и виртуальной диаграмм ритма сердца для 
оценки влияния системы регуляции на ритм сердца. Между параметрами диаграмм и 
информационной энтропией в условиях дискретной сезонной адаптации определены 
функциональные связи. Предложены <<формулы функционального состояния организма>>, 
связывающие параметры макроструктуры диаграммы ритма сердца с параметрами ее 
ярусной микроструктуры. Обнаружено, что режим ритма сердца здорового человека вне 
зависимости от пола имеет цикл календарного года, в течение которого трижды дискретно 
меняется.}

\KW{диаграмма ритма сердца; функциональное состояние организма; ярусная структура; 
информационная энтропия; количество информации}

       \vskip 14pt plus 9pt minus 6pt

      \thispagestyle{headings}

      \begin{multicols}{2}

      \label{st\stat}
  
\section{Введение }

  При исследовании процессов, характеризуемых большим набором 
параметров, возникает вопрос\linebreak о <<цене>> каждого из них. Поначалу все 
па\-ра\-мет\-ры равноценны, поэтому обычно проводят многофакторный 
параметрический анализ, одним из\linebreak возможных инструментом которого служит 
дискриминантный анализ. По величине вероятности влияния на процесс 
анализируемых параметров, удовлетворяющих предложенной гипотезе, все они 
выстраиваются в вариационный ряд по статистической значимости, теряя при 
этом свою равноценность. С~одной стороны, это является хорошей подсказкой 
в выборе параметров для исследования процесса. С~другой стороны, 
статистическая независимость выделяемых параметров вызывает большие 
сомнения~[1--3]. При исследовании сис\-тем\-ных процессов в стороне остаются и 
редко реализуемые параметры, ответственные за функции управления, 
регуляции и контроля, и нелинейные связи между этими функциями, 
количественно характеризуемые указанными параметрами. Вероятно, 
дискриминантный анализ (ДА) может иметь успех на начальной стадии 
исследования реализаций результирующих сигналов <<искусственных 
сис\-тем\-ных процессов>>.
  
  Известны попытки применения технологии ДА для исследования 
естественных системных процессов, к которым относится ритм сердца 
человека, например, по набору параметров вариабельности ритма сердца 
(ВРС)~[4] или по морфологическим\linebreak параметрам ЭКГ~[5]. Результаты таких исследований являются техническими и не дают 
никаких предпосылок к пониманию универсальных и индивидуальных 
физиологических процессов и механизмов управления ими при организации 
результирующего системного процесса ритма сердца. Вероят\-но, исследование в 
указанном направлении следует начинать с определения <<критериев нормы>> 
и поиска нормальных физиологических закономерностей ритма сердца как 
сис\-тем\-но\-го процесса, характера и причин искажения или нарушения этих 
закономерностей в онтогенезе. 
  
  В~данной работе используется физическая модель исследования, 
включающая сравнительный качественный и количественный анализ поведения 
группы параметров ВРС реальной и виртуальной~[6] диаграммы ритма сердца 
(ДРС). Целью его применения является анализ статистических зависимостей 
общепринятых параметров при поиске общих закономерностей в динамике 
ритма. 

\vspace*{-12pt}
  
\section{Экспериментальная часть }
  
  Регистрация ЭКГ проводилась монитором\linebreak Холтера комплекса амбулаторной 
регистрации электрокардиосигнала <<\textit{AnnA Flash}~2000>>~[7] с 
использованием накожных электродов для электрокардиографии. При 
регистрации биопотенциалов применялись двухполюсные отведения по Небу: 
первый электрод располагался во втором межреберном положении у правого 
края грудины (соответствует~$V_5^2$), второй электрод располагался в области\linebreak 
верхушки сердца. Такое расположение электродов позволяет записать переднее 
грудное отведение\linebreak ($A$-\textit{anterior}), соответствующее стандартному 
отведению~II с максимальной амплитудой зубцов на ЭКГ. Данные каждой ЭКГ 
в лицензированной программе \textit{EScreen}~[8] конвертировались в 
ритмограммы в форме последовательности значений $R$--$R$\linebreak интервалов и 
далее посредством встроенной процедуры \textit{Heart rate variability} в 
программе \textit{EScreen} определялись выборочные значения всех параметров 
ВСР для каждой ритмограммы.
  
  Проведено 628~регистраций ЭКГ у 177 здоровых людей и больных~--- 
пациентов реанимационных и кардиореанимационных отделений. Серийные и 
групповые регистрации здоровых молодых людей выделены отдельно и 
представляют основной экспериментальный материал данной работы. 
  
  Серийные двадцатиминутные посуточные ре\-гист\-ра\-ции ЭКГ проводились 
тремя сезонными сериями в течение 5--7~недель каждая в одинаковых 
условиях покоя в одно время суток (двумя мониторами) для двух молодых 
людей 21~года: юноши~К.\ и девушки~Ш. Все серийные регистрации ЭКГ в 
количестве $N_{\mathrm{рег}} = 176$ проводились в домашних условиях при 
температуре 20--22~$^\circ$C в положении лежа на спине с периодом адаптации 
5--10~мин. Серия из 45~регистраций ЭКГ юноши~Р.\ (21~год) проводилась 
отдельно несколько раз в сутки в течение первых двух недель февраля в разных 
условиях покоя и движения. 
  
  Групповые двадцатиминутные регистрации ЭКГ проводились в течение 
9~недель (февраль--март 2008~г.) для группы из 32~молодых людей 19--
24~лет: 20 юношей и 12 девушек. Групповые регистрации проводились в 
лаборатории университета один раз в неделю в интервале времени 14:00--19:00 
в положении покоя: сидя, без адаптации к условиям регистрации. 

\section{Методика обработки и анализа данных}
  
  Индивидуальные особенности ритма сердца найти несложно даже при 
коротких записях ЭКГ~[9]. Сложнее найти общие закономерности ритма 
сердца одного человека в разные последовательные интервалы времени. Еще 
более сложно найти общие динамические закономерности ритма сердца разных 
людей, особенно если записи ЭКГ имеют разную длительность. Сравнение 
временных величин, характеризующих общую вариабельность ритма и 
вычисленных на основе записей различной длительности, является 
некорректным. Методы оценки общей вариабельности сердечного ритма и ее 
компонентов с коротким и длинным периодом не могут заменить друг 
друга~\cite{1ku, 3ku}. Поэтому при анализе ВСР возникают непреодолимые 
трудности при сопоставлении данных записей ЭКГ разной длины с нормой~[1] 
для фиксированной короткой или длинной записи. Более того, определение 
самой нормы функционального состояния организма посредством 
количественных показателей ВСР становится неоднозначным.
  
  Для решения этой проблемы предлагается перейти от анализа 
вариабельности ритма сердца по совокупности группы соответствующих 
показателей, представляющих по отдельности тот или иной информативный 
признак вариабельности, к функционально-параметрическому анализу их 
связей. При этом предлагается использовать <<во благо>> другой проблемный 
момент метода оценки ВСР: статистическую зависимость и дублирование 
информации разными параметрами ВСР~[1--3]. Тесная корреляционная связь 
между параметрами ВСР нивелирует их индивидуальные зависимости от длины 
записи на фазовой плоскости. При этом параметр ритма сердца, выбранный 
общим аргументом, не должен быть ограничен теми или иными 
характеристиками ритмограммы. В~качестве такого параметра была выбрана 
информационная энтропия ярусной диаграммы ритма 
  сердца~\cite{3ku, 6ku, 10ku}. 
    
    Существуют разные подходы к понятию <<энтропия>>, связанные с 
разными объектами и задачами исследования. К~наиболее известным 
относятся~[11--14] подходы: (1)~Клаузиуса, определяющий энтропию функцией 
состояния газовой системы при исследовании тепловых потоков; (2)~Бриллюэна, 
определяющий энтропию мерой <<деградации>>\linebreak энергии; 
(3)~Пригожина, определяющий энтропию мерой <<связанной 
энергии>>; (4)~Больцмана, определяющий энтропию мерой 
интенсивности молекулярного хаоса; (5)~Шеннона, связывающий\linebreak 
энтропию с количеством информации в информационном сообщении и 
определяющий ее как меру степени неопределенности состояния физической 
системы.
    
    Проблема оценки количества информации, содержащегося в сообщении, 
была решена в~1949~г.~\cite{14ku}. В~качестве единицы (бит) информации 
$I=-\log_2 p$ принимают количество информации в достоверном сообщении о 
событии, априорная вероятность~$p$ которого равна~1/2. 
  
  Известно~\cite{10ku, 15ku}, что количество информации~$I_X$, 
приобретаемое физической системой~$X$ при полном выяснении ее состояния, 
равно энтропии~$H(X)$ системы $I_X=H(X)$. Если непрерывную систему 
свести к~дискретной, установив предел\linebreak точности измерения (шаг 
дискретизации~$\Delta x$), это будет равносильно замене плавной кривой на 
графике функции плотности вероятности~$f(x)$ на ступенчатую~--- в форме 
гистограммы. При такой замене вероятности попадания~$p_i$ в 
соответствующие разряды определены в форме $p_i=f(x_i)\Delta x$. В~таком 
случае 
  \begin{equation}
  I_X =-\sum\limits_{i=1} p_i \log_2 p_i\,.
  \label{e1ku}
  \end{equation}
  
  При формировании диаграммы ритма сердца количество информации 
набирается дискретно: от систолы к систоле, поэтому принципиально 
невозможно использовать понятие скорости набора информации, являющееся 
основным параметром для технических устройств связи~\cite{15ku}. В~этом 
случае масштабной единицей становится переменный интервал времени 
события ($R$--$R$ интервал). Ритмограмма представляет собой номерной ряд 
последовательности $n$~таких событий. Постоянная\linebreak частота считывания 
монитором значений биопотенциалов при формировании ЭКГ задает 
постоянным шаг дискретизации~$\Delta x$ значений $R$--$R$ интервала на 
ДРС. Поэтому значения $R$--$R$\linebreak интервала на ДРС формируют <<ярусы 
микросостояний>>. Это позволяет трактовать ДРС как реализацию 
макросостояния системы ритма и применить к ней 
  струк\-тур\-но-то\-по\-ло\-ги\-че\-ский анализ неупорядоченности 
распределения значений $R$--$R$ интервала по микросостояниям с 
использованием информационной энтропии~\cite{6ku, 10ku}. 
  
  В качестве меры фрактальной размерности странных аттракторов в фазовом 
пространстве применяют информационную размерность~$D_I$. Мерой 
непредсказуемости в системе служит информационная энтропия~[10, 16--19]:
  \begin{equation}
  I(\varepsilon ) =-\sum\limits_{i=1} p_i\log_2 p_i\,.
  \label{e2ku}
  \end{equation}
  
  Перенося определение этой меры на ярусную ДРС, получим вертикальный 
размер~$\varepsilon$ ячейки\linebreak  покрытия, равный шагу дискретизации~$\Delta x$. 
В~таком представлении категории количества ин\-формации~(\ref{e1ku}) и 
информационной энтропии~(\ref{e2ku})\linebreak становятся тождественными. Чтобы не 
допускать путаницы и для конкретного объекта исследования (ярусной 
структуры ДРС) в обозначении информационной энтропии будем использовать 
символ~$I^*$. 
  
  Степень неопределенности состояния системы ритма может определяться и 
вероятностями ($p_i$) ее возможных состояний, и их количеством~\cite{15ku}, 
поэтому возникает возможность перехода от вероятностных категорий к 
макропараметрам ярусной ДРС. После несложных алгебраических 
преобразований формулы~(\ref{e1ku}) и~(\ref{e2ku}), примененные к ярусной 
структуре ДРС, можно представить в виде~\cite{6ku}: 
  \begin{equation}
  I^*=\fr{A}{n}\left[ \ln \Gamma +B\right ]\,,
  \label{e2-2ku}
  \end{equation}
где $A=1/\ln 2$~--- полиномиальный коэффициент, $\Gamma 
=N!/\prod\limits_{i=1}^n N_i!$, $N$~--- число дискретных значений~$R$--$R$ 
интервала в анализируемой выборке ритмограммы, $N_i$~--- число дискретных 
значений $R$--$R$ интервала на $i$-м ярусе ДРС.

  При принятой точности расчета (до двух значащих цифр) величиной~$B$ 
можно пренебречь уже при $n > 100$, так как величина абсолютной 
погрешности $\Delta I_X=AB_{\max}/n$ с ростом~$n$ асимптотически 
стремится к нулю~\cite{6ku}. С~учетом этого формула~(\ref{e2-2ku}) 
принимает окончательный расчетный вид: 
  \begin{equation}
  I^*=\fr{A}{n}\,\ln\Gamma\,.
  \label{e3ku}
  \end{equation}
  
  В формуле~(\ref{e3ku}) полиномиальный коэффициент~$\Gamma$ 
приобретает смысл термодинамической ве\-ро\-ят\-ности и определяет число 
микросостояний (ком\-бинаций), посредством которых реализуется\linebreak 
макросостояние системы~$X$. Величина~$I^*$, определенная с точностью до 
величины~$AB_{\max}/n$, определяет среднее количество информации, 
недостающее до полного описания одного отсчета. 
  
  С одной стороны, информационная энтропия~$I^*$ обладает основными 
свойствами физической энтропии~--- при фиксированных внешних условиях 
растет с ростом~$n$, принимая максимальное со\-вмес\-ти\-мое с внешними 
условиями значение. С~другой стороны, она определена отношением 
количества информации~$I_\Sigma$, недостающего для полного описания ДРС 
к объему выборки~$n$ и по смыс\-лу является средним количеством 
информации, недостающим для описания одного микроперехода на 
ДРС~\cite{6ku}. По сравнению с другими параметрами вариабельности 
сердечного ритма (ВСР), информационная энтропия~$I^*$ не теряет 
адекватного физического смыс\-ла для многомодального распределения и имеет 
постоянную, четко выраженную <<правую границу условного 
здоровья>>~\cite{6ku}. Очевидно, что разные заболевания могут дать один и 
тот же результат по величине~$I^*$ для ДРС. Это может означать, что разные 
стадии разных заболеваний подобны по результирующему параметру~$I^*$, 
т.\,е.\ по неупорядоченности ритма сердца обследуемых людей. В~таком случае 
параметр~$I^*$ может служить количественной оценкой общего 
функционального состояния человека (ФСО)~\cite{6ku}.
\end{multicols}

  \begin{figure} %fig1
  \vspace*{1pt}
\begin{center}
\mbox{%
\epsfxsize=162.85mm
\epsfbox{kuz-1.eps}
}
\end{center}
\vspace*{-6pt}
\Caption{Графики функциональных связей параметров ВСР и информационной 
энтропии для пяти серий ($N = 253$) сезонных регистраций ЭКГ здоровых 
молодых людей~(\textit{а})--(\textit{в}) и для 375~регистраций ЭКГ здоровых 
людей и пациентов отделений реанимации за 9~лет~(\textit{г}) 
  \label{f1ku}}
  \vspace*{6pt}
  \end{figure}
  
  \begin{multicols}{2}
  
  На рис.~\ref{f1ku},\,\textit{а}--\textit{г} в полулогарифмическом 
масштабе приведены точечные графики зависимости основных расчетных 
выборочных параметров ВСР от соответствующих значений информационной 
энтропии по всем $N$~ритмограммам. В~качестве основных параметров ВСР 
представлены: из временной области анализа ДРС~--- стандартное отклонение 
($\sigma$, мс), из частной области анализа~--- полная спектральная мощность 
(\textit{Total Power}, или~TP, мс$^2$), из набора производных показателей 
  Баевского~--- индекс напряжения (ИН), или стресс-ин\-декс (SI), 
характеризующий степень централизации управления ритмом~\cite{20ku}. 
  
  Графики на рис.~\ref{f1ku},\,\textit{а}--\textit{в} представлены 
услов\-но-се\-зон\-ны\-ми линиями одинакового наклона. Это указывает на 
наличие прочных функциональных связей между параметрами ВСР и 
информационной энтропией в условиях дискретной сезонной адаптации. 
В~зависимости от уровня ФСО в интервалах времени услов\-но-се\-зон\-но\-го 
исследования точки на графиках соответствующих параметров перемещаются 
вдоль линий функциональных кривых <<как по монорельсу>>. При смене 
сезона заполняются новые <<функциональные уровни>>, соответствующие 
обретению ритмом качественно новых стационарных режимов 
(рис.~\ref{f1ku},\,\textit{а}--\textit{г}). Данные юношей ($\times$,~$+$) и 
девушек ($\bullet$,~$\circ$) для каждого услов\-но-се\-зон\-но\-го интервала времени 
принадлежат соответствующему <<функциональному уровню>> 
(см.\ рис.~\ref{f1ku},\,\textit{а}--\textit{в}) без разделения по полу. На 
основании этого результата при одинаковом возрасте здоровых молодых 
людей, обследуемых и в группе, и серийно, оказалось возможным 
предположить, что ритм сердца меняет режим дискретно при изменении 
сезонных внешних условий. Механизм адаптации является стабилизирующим 
каждое новое качество ритма сердца.

  
  При изменении длины записи функциональные кривые лишь меняют свою 
длину изменением координат правой или левой границы. Величина $I^* = 
6$~бит характеризует режимы ритма здорового молодого человека на всех 
ступенях сезонной адаптации при двадцатиминутной записи ЭКГ. Обращает 
внимание, что определенная ранее нелинейная динамика выборочных 
коэффициента асимметрии и эксцесса распределений на ДРС при серийных 
регистрациях~\cite{6ku} не сказывается на форме графиков~$\sigma (I^*, N)$ в 
интервале времени любого услов\-но-се\-зон\-но\-го исследования.
  
  При серийных и групповых исследованиях обнаружен нелинейный характер 
связи стандартного отклонения~$\sigma$ и информационной энтропии~$I^*$ 
со сред\-не-вы\-бо\-роч\-ным значением $R$--$R$ интервала ($\langle X\rangle$) 
и моды (Mo) ритмограмм. Внешне форма функций $\langle X\rangle(I^*, N)$ и 
$\langle X\rangle(\sigma, N)$ на соответствующих графиках напоминает <<полет 
мухи под люстрой>>. При внутривыборочных исследованиях ритмограмм, 
проведенных с использованием <<метода скользящих средних>>~\cite{6ku}, 
оказалось, что на малые флуктуации среднего уровня ритма сердца 
ни~$\sigma$, ни~$I^*$ практически не откликаются. Если средний уровень 
ритма на некотором интервале времени ($\Delta n < 100$) постоянен, то и 
указанные параметры не меняются. Однако если постоянство $\langle 
X\rangle(n)$ во времени затягивается, то величины обоих параметров начинают 
медленно монотонно падать. На любое относительно резкое и длительное 
изменение $\langle X\rangle$ функция~$\sigma(n)$ откликается импульсно 
таким образом, что $\sigma_{\max}$ всегда приходится на точку перегиба 
графика~$\langle X\rangle(n)$. В~таком случае~$\sigma(n)$ определяется 
скоростью изменения функции $\langle X\rangle (n)$. При монотонном 
росте~$\langle X\rangle$ значения~$I^*$ и~$\sigma$ обретают тенденцию к 
росту за счет набора нерабочих (<<пустых>>) ярусов. При этом процентный 
состав рабочих ярусов падает, а неупорядоченность ДРС слабо растет. Таким 
образом, функция~$I^*(n)$ <<следит>> за средним уровнем~$\langle X\rangle 
(n)$ комплексно: за величиной, за длительностью его характерных 
динамических участков и за скоростью изменения. 
  

\section{Формулы функционального состояния организма}

    Подавляющее количество моделей и приемов исследования временных 
рядов относится к стационарным в широком смысле рядам, т.\,е.\ к рядам, для 
которых первые четыре момента не зависят от времени~\cite{21ku, 22ku}. 
В~общем случае даже в интервале одной двадцатиминутной регистрации ЭКГ 
не удается исключить тренд в дисперсии, асимметрии и эксцессе разностным 
дифференцированием рядов ДРС~\cite{6ku, 21ku}. Таким образом, реальные 
процессы ритма сердца не являются стационарными. Для того чтобы в первом 
приближении для задач краткосрочного прогноза считать их таковыми, 
необходимо создать определенные экспериментальные условия, а именно в 
состав анализа включать выборки только молодых здоровых людей, 
находящихся в состоянии стабильного психоэмоционального покоя. В~рамках 
такого приближения к стационарным рядам можно применить теорему Вольда 
о разложении~\cite{23ku}, согласно которой всякий стационарный процесс 
может быть единственным образом представлен в виде суммы двух не 
кор\-ре\-ли\-ру\-ющих между собой процессов: детерминированного (сингулярного) 
и случайного (регулярного белого шума). 

Если принять в качестве гипотезы утверждение, что ритм сердца является 
нестационарным процессом в той мере, в которой в него включены 
составляющие внешнего влияния механизмов управления и 
регуляции~\cite{2ku, 6ku}, то стационарным ритмическим процессом может 
характеризоваться состояние <<неуправляемого сердца>>. Обычно этот 
физиологический термин применяется для описания работы изолированного от 
организма сердца с перфузией. В~применении к описанию работы сердца, 
изолированного только от детерминистской информации внешнего влияния, 
жесткий ритм, обеспечиваемый функцией автоматии, дополняется на заданном 
уровне регулярным белым шумом, определяющим лишь присутствие и 
функциональную готовность разных механизмов управления и регуляции. 
Приближением к такому режиму ритма может быть ритм сердца молодого 
здорового человека, находящегося в условиях адаптации к условиям 
регистрации и в состоянии устойчивого психоэмоционального покоя. При этом 
распределение $R$--$R$ интервалов на ДРС приближается к нормальному 
распределению~[6, 24--28]. Крайней идеализацией такого режима ритма может 
служить <<виртуальный ритм>> с реализацией в форме цифрового ряда, 
полученного генерацией случайных чисел по нормальному закону и по 
заданным значениям стандартного отклонения и шага 
дискретизации~\cite{6ku,  28ku}. Для построения виртуальной диаграммы 
ритма сердца используется нормальный генератор случайный чисел в 
программе~Excel. Оказалось, что величина информационной 
энтропии~$I^*_{\mathrm{г}}$ виртуальной ДРС (ВДРС) 
при $n\rightarrow \infty$ монотонно приближается к значению энтропии 
$$
H(X) = \log_2\fr{(2\pi e)^{1/2}\sigma}{\Delta x}
$$ 
в форме математического ожидания для непрерывного 
множества случайных чисел, распределенных по нормальному 
закону~\cite{6ku, 15ku} (рис.~\ref{f2ku}). 
\begin{figure*} %fig2
  \vspace*{1pt}
\begin{center}
\mbox{%
\epsfxsize=164.59mm
\epsfbox{kuz-2.eps}
}
\end{center}
\vspace*{-6pt}
\begin{minipage}[t]{79.5mm}
\Caption{Графики зависимости $I_{\mathrm{г}}^*(\sigma, n)$ в сравнении с 
функцией~$H(X)$
\label{f2ku}}
%\end{figure*}
\end{minipage}
\hfill
\vspace*{-6pt}
\begin{minipage}[t]{79.5mm}
%  \begin{figure*} %fig3
%    \vspace*{1pt}
\Caption{Графики $N_{\mathrm{МЯП}}(n)$~(\textit{1}) 
и~$\langle\Delta_{\mathrm{ЯП}}\rangle (n)$~(\textit{2}) виртуальных диаграмм ритма 
сердца при заданных $\sigma = 40$~($\times$), 70~($\bullet$) и 
100~мс~($\circ$)
  \label{f3ku}}
  \end{minipage}
  \vspace*{9pt}
  \end{figure*}

  Очевидно, что величины энтропии~$H(X)$ и информационной 
энтропии~$I^*_{\mathrm{г}}$ сближаются при размерах $n$~цифровых рядов, 
стремящихся к бесконечности (см.\ рис.~\ref{f2ku}). Если величина выборки 
конечна, то результаты расчета по этим двум величинам расходятся. Они 
расходятся тем больше, чем меньше величина~$n$. Например, при $n = 1000$ 
отсчетов расхождение достигает 20\%. Такое расхождение объясняется тем, что 
связующим звеном между двумя формами записи энтропии является формула 
Стирлинга для случая $n\rightarrow \infty$~\cite{6ku, 29ku}. При уменьшении 
величины~$n$ расчетная погрешность этой формулы нарастает.
  
  Уравнения трендовых линий функциональных кривых (см.\ рис.~1,\,\textit{а},~\textit{г}) могут 
быть представлены в общем виде $3\sigma = 2^{I^*+i}$ при коэффициенте 
достоверности аппроксимации $R^2> 0{,}9$ (см.\ рис.~\ref{f1ku},\,\textit{а}) для 
ряда значений $i = 0$, 1, 2, 3. Эта формула, с одной стороны, иллюстрирует 
<<правило~$3\sigma$>> для ДРС в интерпретации, отличной от 
общепринятой~\cite{15ku, 23ku}: величина~$3\sigma$ представляет полное 
число комбинаций (число кодонов) при переходе от <<алфавита>> с двумя 
буквами к алфавиту с $k$~буквами ($k < 3\sigma$) и переменным размером 
кодона ($I^*+ i$). С~другой стороны, она может быть представлена в виде: $I^* 
= \log_23\sigma -i$. Такую форму записи автор назвал <<формулой ФСО>>, так 
как по величине~$i$ определяются и уровень ФСО здорового человека, и 
степень тяжести заболевания больного (см.\ далее). 
  
  При сопоставлении теоретической и экспериментальной форм записи 
энтропии~$H(X)$ и~$I^*$ величина~$\sigma$ оказывается одинаковой, 
поэтому в качестве переменных параметров могут служить~$\Delta x$ и~$i$. 
Если в формуле математического ожидания энтропии цифрового ряда 
случайных величин, распределенных по нормальному закону 
\begin{equation}
H(X) \approx  \log_2\left(\fr{4{,}13\sigma}{\Delta x}\right)\,,
\label{aa}
\end{equation}
принять шаг дискретизации исходной непрерывной 
функции $\Delta x = 1$~мс, то 
$$H(X) \approx \log_2(4{,}13\sigma)\,,
$$ 
где единица 
измерения $[\sigma ] = 1$~мс. При $\Delta x = 2$~мс получим 
$$
H(X) \approx  \log_2(4{,}13\sigma) - 1\,;$$
при $\Delta x = 4$~мс~--- 
$$H(X) \approx  \log_2(4{,}13\sigma)- 2$$ 
и~т.\,д. Точка на графике~$H(\sigma, \Delta x)$ 
сдвигается влево на единицу при изменении величины~$\Delta x$ кратно~2. 
Если $\Delta x = 0{,}5$~мс, точка на указанном графике (см.\ рис.~\ref{f2ku}) 
сместится на единицу вправо. При фиксированном значении~$\sigma$ 
график~$H(\Delta x)$ будет линейным в полулогарифмическом масштабе. 
  
  Величина шага дискретизации $\Delta x = 1$, определенная приборной 
частотой, не равна выборочному среднему расстоянию между ярусами, т.\,е.\ 
средней величине межъярусного промежутка $\langle 
\Delta_{\mathrm{ЯП}}\rangle$. Результаты анализа, проведенные для 
виртуальной ДРС, показали, что с ростом объема выборки~$n$ число 
межъярусных промежутков~$N_{\mathrm{МЯП}}$ на диаграмме монотонно 
растет, а величина $\langle \Delta_{\mathrm{ЯП}}\rangle \rightarrow \Delta x$ при 
$n\rightarrow\infty$ (рис.~\ref{f3ku}). 
  

  
  Если для виртуальных цифровых рядов формуле~(\ref{aa})
  поставить в адекватное соответствие форму 
записи 
\begin{equation}
I^*_{\mathrm{г}} \approx 
\log_2 \fr{4{,}13\sigma}{\langle\Delta_{\mathrm{ЯП}}\rangle}\,,
\label{aaa}
\end{equation} 
то при 
$n\rightarrow\infty$ они совпадут. Можно оценить, что $4{,}13\sigma /\langle 
\Delta_{\mathrm{ЯП}}\rangle \approx 3\sigma/\Delta x$ (см.\ рис.~\ref{f3ku}), и 
формула~(\ref{aaa}) описывает предельный режим ритма, 
информационная энтропия ДРС которого определена формулой 
$$I_{\max}^* = 
\log_2\fr{3\sigma}{\Delta x}
$$ (см.\ выше). Ясно, что всегда выполняется тройное 
неравенство: $H(X) > I_{\mathrm{г}}^* \geq I_{\max}^* \geq I^*$. Верхняя 
<<математическая граница>>, с~одной стороны, является идеализацией, а 
с~другой~--- эталоном для ритма сердца как маркер <<правой границы нормы 
условного здоровья>>. 
  
  Следуя той же логике, для реальных ДРС 
  $$
  I^* \approx \log_2\fr{4{,}13\sigma}{\langle \Delta_{\mathrm{ЯП}}\rangle }\,,
  $$ или с учетом экспериментальных 
данных (см.\ рис.~\ref{f1ku},\,\textit{а}) 
$$I^* \approx \log_2\fr{3\sigma}{2^i\Delta x}\,,
$$ 
где $\Delta x = 1$~мс. Эти формулы связывают параметры макроструктуры 
ДРС ($I^*$ и~$\sigma$) с параметрами ее ярусной микроструктуры ($\langle 
\Delta_{\mathrm{ЯП}}\rangle $ и~$i$). При $i = 0$ данные условно 
соответствуют весенним, при $i = 1$~--- зимним, при $i = 2$~--- осенним, при 
$i = 3$ и выше~--- болезни. Для людей одного возраста получается 
возможность организации шкалы ФСО по функции $\langle 
\Delta_{\mathrm{ЯП}}\rangle (N)$ или по величине~$i$ как по группе, так и по 
серии $N$~опытов. Предлагается величину~$i$ определять показателем ФСО: 
$i = 0$~--- норма, $i=1$~--- обратимое угнетение в рамках сезонной адаптации, 
$i = 2$~--- обратимое донозологическое состояние в рамках сезонной 
адаптации, $i = 3$~--- необратимое состояние вне рамок сезонной адаптации 
(патогенез). 


  
  Уравнением, связывающим среднюю величину микроперехода на ДРС и 
информационную энтропию ДРС, может служить $\langle 
\Delta_{\mathrm{ЯП}}\rangle  = - 2I^* + (13\div 17)$ (рис.~4). Для 
здорового молодого человека гради-\linebreak
%\noindent
\begin{center} %fig4
%\vspace*{6pt}
\mbox{%
\epsfxsize=70.456mm
\epsfbox{kuz-4.eps}
}
\end{center}
\vspace*{3pt}
%\begin{center}
{{\figurename~4}\ \ \small{Графики $\langle\Delta_{\mathrm{ЯП}}\rangle (I^*, N)$ по данным 
регистраций ЭКГ здоровых людей}}
%\end{center}
%\vspace*{3pt}

%\bigskip
\addtocounter{figure}{1}


\noindent
ент средней величины микроперехода 
направлен в сторону убывания~$I^*$ и определен величиной~--- $2$~мс/бит. 
При $\langle \Delta_{\mathrm{ЯП}}\rangle \rightarrow (\Delta x = 1)$ показатель 
ФСО $i = 0$ и величине~$I^*$ разрешено варьировать в пределах 6--8~бит; 
при $\langle \Delta_{\mathrm{ЯП}}\rangle \rightarrow 2$ ($i = 1$)~--- $I^* = 
5{,}5$--7,5~бит; при $\langle \Delta_{\mathrm{ЯП}}\rangle  \rightarrow 4$ 
($i=2$) $I^* = 4{,}5$--6,5~бит. Переход <<весна--осень>> соответствует 
дискретному четырехкратному увеличению величины~$\langle 
\Delta_{\mathrm{ЯП}}\rangle $ (см.\ рис.~4). 

  
\section{Относительная информационная энтропия}
  
  Оба энтропийных параметра: $I^*$ для ДРС и~$I_{\mathrm{г}}^*$ для 
виртуальной ДРС~--- зависят от~$n$~\cite{3ku, 6ku}. Это доставляет неудобства 
при сравнении цифровых рядов разной длины. Следовательно, необходим 
параметр, который бы сохранял информацию о неупорядоченности ярусной 
структуры ДРС и не зависел от~$n$. 
  
  При сравнении ДРС с ВДРС информационная энтропия~$I_{\mathrm{г}}^*$ 
имеет ту же функциональную погрешность по~$n$, что и~$I^*$ для реальной 
ДРС~\cite{6ku}. Поэтому их отношение $i_r = I^/I_{\mathrm{г}}^*$ становится 
свободным от функциональной погрешности, связанной с конечностью числа 
измерений. Оба энтропийных параметра~$I^*$ и~$I_{\mathrm{г}}^*$ зависят и 
от~$n$, и от~$\sigma$, но в первом из них содержится информация о 
функции~$\sigma(n)$, т.\,е.\ о детерминистской составляющей сигнала. 
Поэтому их отношение $I^*/I_{\mathrm{г}}^*$, оценивая это влияние, является 
мерой неупорядоченности ярусной структуры ДРС по отношению к 
максимально возможной неупорядоченности, ограниченной задаваемыми 
параметрами, сводя влияние объема выборки к пренебрежимо малому.
  
  Таким образом, относительную информационную энтропию 
($I^*/I_{\mathrm{г}}^*$) можно использовать как индикатор уровня 
ФСО по фактору регуляции ритма в 
смысле отклонения от нормального закона распределения. При этом можно 
исходить из утверждения, что отклонение гомеостатической функции от 
стационарного уровня вызвано внешним управляющим влиянием и в любом 
проявлении приводит к появлению и интенсификации работы механизмов 
регуляции. Механизмы адаптации тормозят процессы отклонения, а механизмы 
регуляции возвращают его в норму. 
  \begin{table*}\small
  \begin{center}
  \Caption{Показатели ФСО по всем регистрациям
  \label{t1ku}}
  \vspace*{2ex}
  
  \begin{tabular}{|l|c|c|c|c|c|l|}
  \hline
\multicolumn{1}{|c|}{\tabcolsep=0pt\begin{tabular}{c}Время\\ Серия или группа\end{tabular}}&
\tabcolsep=0pt\begin{tabular}{c} $N_{\mathrm{рег}}$\end{tabular}&
\tabcolsep=0pt\begin{tabular}{c}ЧСС,\\ уд./мин\end{tabular}&
\tabcolsep=0pt\begin{tabular}{c}Рабочие ярусы,\\ \%\end{tabular}&
\tabcolsep=0pt\begin{tabular}{c}$\langle  \Delta_{\mathrm{ЯП}}\rangle$,\\ мс\end{tabular}&
\tabcolsep=0pt\begin{tabular}{c}$I^*/I_{\mathrm{г}}^*$,\\ \%\end{tabular}&
\tabcolsep=0pt\begin{tabular}{c}Формула ФСО \\(см.\ рис.~\ref{f1ku},\,\textit{а}--\textit{г})\end{tabular}\\
\hline
Декабрь--январь 2008~г. &&&&&&\\
\tabcolsep=0pt\begin{tabular}{l}К.\ (21~г.) \\ Ш.\ (21~г.)\end{tabular}&
\tabcolsep=0pt\begin{tabular}{c}  34\\  48\end{tabular}&
\tabcolsep=0pt\begin{tabular}{c} $74 \pm 2$\\ $75 \pm 2$\end{tabular}&
\tabcolsep=0pt\begin{tabular}{c} $34{,}4 \pm 2{,}7$\\ $40{,}4 \pm 1{,}6$\end{tabular}&
\tabcolsep=0pt\begin{tabular}{c} $2{,}5 \pm 0,1$\\ $3{,}2 \pm 0,3$\end{tabular}&
\tabcolsep=0pt\begin{tabular}{c} $89{,}8 \pm 0{,}7$\\ $87{,}2 \pm 1,8$\end{tabular}&
\tabcolsep=0pt\begin{tabular}{l} $I^* = \log_23\sigma - 1$\\ $I^* = \log_23\sigma - 1$\end{tabular}\\
\hline
Январь--февраль 2008~г. &&&&&&\\
Р.\ (21~г.)&
45& $63 \pm 2$&  $36{,}7 \pm 1{,}4$&  $2{,}8 \pm 0{,}1$& $92{,}1 \pm 0{,}6$&  
$I^* = \log_23\sigma - 1$\\
\hline
Февраль--март  2008~г. &&&&&&\\
\tabcolsep=0pt\begin{tabular}{l}Гр.~32 (19--24~г.) \\Юноши\\ Девушки\end{tabular}&
 \tabcolsep=0pt\begin{tabular}{c}  32\\  20\\  12\end{tabular}&
\tabcolsep=0pt\begin{tabular}{c} $81 \pm 5$\\ $82 \pm 7$\\ $79 \pm 5$\end{tabular}&
\tabcolsep=0pt\begin{tabular}{c}
$72{,}6 \pm 2{,}6$\\ $72{,}3 \pm 3{,}3$\\ $73{,}2 \pm 4{,}2$\end{tabular}&
\tabcolsep=0pt\begin{tabular}{c} \ \\ $1{,}4 \pm 0{,}1$\\ $1{,}4 \pm 0{,}1$\end{tabular}&
\tabcolsep=0pt\begin{tabular}{c} $97{,}9 \pm 2$\\ $99{,}3 \pm 0{,}3$\\ $95{,}5 \pm 
4{,}5$\end{tabular}&
\tabcolsep=0pt\begin{tabular}{l} $I^* = \log_23\sigma$\\ $I^* = \log_23\sigma$\\ $I^* = 
\log_23\sigma$\end{tabular}\\
\hline
Октябрь--ноябрь 2008~г. &&&&&&\\
\tabcolsep=0pt\begin{tabular}{l}К.\ (22~г.)\\ Ш.\ (22~г.)\end{tabular}&
\tabcolsep=0pt\begin{tabular}{c} 34\\  33\end{tabular}&
\tabcolsep=0pt\begin{tabular}{c}
$68 \pm 3$\\ $81 \pm 2 $\end{tabular}&
\tabcolsep=0pt\begin{tabular}{c} $21{,}9 \pm 1{,}3$\\ $41{,}7 \pm 1{,}4$\end{tabular}&
\tabcolsep=0pt\begin{tabular}{c}
$4{,}6 \pm 0{,}2$\\ $2{,}4 \pm 0{,}1$\end{tabular}&
\tabcolsep=0pt\begin{tabular}{c} $79{,}0 \pm 1{,}1$\\ $88{,}5 \pm 0{,}4$\end{tabular}&
\tabcolsep=0pt\begin{tabular}{l}
$I^* = \log_23\sigma - 2$\\  $I^* = \log_23\sigma - 1$ \end{tabular}\\
\hline
Апрель--май 2009~г. &&&&&&\\
Ш.\ (22~г.)& 27& $79 \pm 2$& $73{,}2 \pm 2{,}6$& $1{,}4 \pm 0{,}1$& $99{,}4 \pm 0{,}2$& $I^* = 
\log_23\sigma$\\
\hline
1999--2009~гг. &&&&&&\\
Группа 330 (17--75~лет)&375&---&2--75&
\tabcolsep=0pt\begin{tabular}{c}
 (1,4)\\ (2,8)\\ (4,2)\\ (5,6)\end{tabular}&
20--99&
\tabcolsep=0pt\begin{tabular}{l}
$I^* = \log_23\sigma$\\ $I^* = \log_23\sigma - 1$\\ $I^* = \log_23\sigma - 2$\\ $I^* = \log_23\sigma- 3$\\ 
\ldots\end{tabular}\\
\hline
\multicolumn{7}{l}{\footnotesize(\ )~--- предполагаемые расчетные значения сертификации больных по 
ФСО.}
\end{tabular}
\end{center}
\vspace*{-12pt}
\end{table*}
Механизмы контроля удерживают его 
около нормы в разрешенных пределах флуктуаций. Например, для состояния 
<<весна>> механизмы контроля удерживают отношение~$I^*/I_{\mathrm{г}}^*$ 
около индивидуального значения, близкого к единице (или~100\%) и должны 
быть определены процессами непрерывного действия. По величине этого 
отношения и его динамике человек принимает решение: переходить к 
нозологическим процедурам или в этом нет необходимости. Иными словами, 
величина~$I^*/I_{\mathrm{г}}^*$ и ее динамика позволяют сопоставить 
субъективные оценки состояния человека с относительным уровнем 
хаотичности ритма сердца.
  
  В табл.~\ref{t1ku} представлены данные по сериям регистраций ЭКГ 
обследуемых~К., Ш., Р.\ и группы здоровых молодых людей (32~человека), а 
также данные массовых нерегулярных во времени ре\-гист\-ра\-ций по группе из 
330~человек. В~таблице приведены: число регистраций~$N_{\mathrm{рег}}$ и 
расчетные средние значения частоты сердечных сокращений (ЧСС), 
относительного количества рабочих ярусов, величины межъярусного 
промежутка и относительной информационной энтропии. Все расчеты 
проведены с уровнем значимости $\alpha = 0{,}05$. В~последнем столбце 
таблицы приведены формулы ФСО, представляющие зависимости~$I^*(\sigma, 
N)$ раздельно по ДРС выделенных в строках серий и групп ре\-гист\-раций. 
  


  
  Данные по группе из 330~человек разного воз\-рас\-та (140~здоровых людей, 
235~пациентов отделений реанимации за период 1999--2009~гг.) выделены в 
отдельную нижнюю строку таблицы. Соответствующие графики представлены 
ранее (см.\ рис.~\ref{f1ku},\,\textit{г}). В~сравнении с графиком на 
рис.~\ref{f1ku},\,\textit{а} и данными табл.~1 для них характерными являются 
следующие отличия: 

\noindent
  \begin{enumerate}[(1)]
  \item
   слабо выражены данные, определенные по формуле ФСО <<весна>>; 
  \item  отчетливо проявляются данные, соответст\-ву\-ющие формуле ФСО $I^* = 
\log_2 3\sigma - 3$; 
  \item заметно рассеяние данных, связанное в основном с большим 
расхождением возраста обследуемых.
  \end{enumerate}
  
  Визуальный анализ графиков (см.\ рис.~\ref{f1ku},\,\textit{а}--\textit{в})\linebreak 
и расчетных данных табл.~\ref{t1ku} приводит к следующим общим 
результатам: 
  \begin{enumerate}[1.]
  \item В~течение одного года режим ритма меняется скачкообразно трижды. 
  \item С~позиции~($I^*/I_{\mathrm{г}}^*)_{\min}$ условный год начинается в 
начале октября. Первый триместр, <<осень>> (октябрь--декабрь), и второй 
триместр, <<зима>>\linebreak (декабрь--февраль), характеризуются относительным 
постоянством индивидуальных величин~$I^*/I_{\mathrm{г}}^*$, но 
заканчиваются их скачко\-об\-разным ростом, характеризующим изменение\linebreak 
качества режима ритма и структуры ДРС смещением в сторону превалирования 
хаотической составляющей. Третий триместр, <<весна>> (март--сентябрь), 
характеризуется слабым рассеянием величины~$I^*/I_{\mathrm{г}}^*$ около 
единицы. В~ритме сердца превалирует хаотическая со\-став\-ля\-ющая со слабыми 
признаками проявления механизмов управления и регуляции. 
  \item  В~начале октября режим ритма сердца столь значительно и 
скачкообразно меняет качество, что переходный процесс претендует на 
категорию <<катастрофы>>~\cite{30ku}.
  \end{enumerate}
  
  Предварительные исследования показали, что вне зависимости от того, 
болеет человек или нет, с возрастом~$I^*$ падает и все дальше отходит от 
эталонного значения. Отношение~$I^*/I_{\mathrm{г}}^*$ с возрастом 
уменьшается, но не монотонно, а ступенчато, как бы задерживаясь на 
определенных функциональных уровнях, которые могут служить индикаторами 
биологического возраста человека. По колебанию величины этого отношения 
около индивидуального уровня соответствующего возраста можно определить 
уровень ФСО, поэтому данное отношение предлагается в качестве индикатора 
донозологической диагностики вне зависимости от возраста человека.
  
\section{Выводы}
  
  \noindent
  \begin{enumerate}[1.]
  \item Между параметрами ВСР и информационной энтропией ДРС в 
условиях дискретной сезонной адаптации существуют прочные 
функциональные связи. При смене сезона заполняются новые 
<<функциональные уровни>>, соответствующие обретению ритмом 
качественно новых стационарных режимов. 
  \item При изменении длины записи ЭКГ графики функциональных кривых 
сохраняются, меняя длину изменением координат правой или левой границы.
  \item Сравнительный анализ реальной ДРС и соответствующей ей 
виртуальной ДРС позволяет оценивать влияние системы регуляции на ритм 
сердца в форме отклонения распределения значений $R$--$R$ интервалов на 
ДРС от нормального закона.
  \item Всегда выполняется тройное неравенство: $H(X) > I_{\mathrm{г}}^* \geq 
I_{\max}^* \geq I^*$. Верхняя <<математическая граница>>, с одной стороны, 
является идеализацией, а с другой~--- эталоном для ритма сердца как маркер 
<<правой границы нормы условного здоровья>>. 
  \item Формулы ФСО адекватно связывают па\-ра\-мет\-ры макроструктуры ДРС 
($I^*$ и~$\sigma$) с па\-ра\-мет\-ра\-ми ее ярусной микроструктуры 
($\langle\Delta_{\mathrm{ЯП}}\rangle$, $\Delta x$ и~$i$).
  \item В формуле ФСО $I^* \approx \log_2[(3\sigma)/(2^i\Delta x)]$ 
величину~$i$ предлагается определять показателем ФСО: $ i= 0$~--- норма, $i = 
1$~--- обратимое угнетение в рамках сезонной адаптации, $i = 2$~--- обратимое 
донозологическое состояние в рамках сезонной адаптации, $i = 3$~--- 
необратимое состояние вне рамок сезонной адаптации (патогенез). 
  \item Уравнением, связывающим среднюю величину микроперехода и 
информационную энтропию ДРС, может служить $\langle 
\Delta_{\mathrm{ЯП}}\rangle  = - 2I^* + (13\div17)$. 
  \item  Относительную информационную энтропию ($I^*/I_{\mathrm{г}}^*$) 
можно использовать как индикатор уровня функционального состояния 
организма по фактору регуляции ритма. Режим ритма сердца здорового 
молодого человека вне зависимости от пола в течение календарного года 
трижды дискретно меняет свое качество: от $(I^*/I_{\mathrm{г}}^*)_{\min} \approx  
0{,}8$ в интервале октябрь--ноябрь к $(I^*/I_{\mathrm{г}}^*) \approx 0{,}9$ в 
интервале декабрь--февраль и до $I^*/I_{\mathrm{г}}^* \leq 1$ в интервале 
  март--сентябрь. Замыкает годовой цикл изменений режима ритма наиболее 
резкое изменение качества при возврате к значению 
$(I^*/I_{\mathrm{г}}^*)_{\min} \approx 0{,}8$ в начале октября.
  \item Обнаружено сильное и направленное влияние возраста человека на 
качество режима ритма сердца. Отношение~$I^*/I_{\mathrm{г}}^*$ с возрастом 
уменьшается, но не монотонно, а ступенчато, как бы задерживаясь на 
определенных функциональных уровнях, которые могут служить индикаторами 
биологического возраста человека.
  \end{enumerate}
  

{\small\frenchspacing
{%\baselineskip=10.8pt
\addcontentsline{toc}{section}{Литература}
\begin{thebibliography}{99}

\bibitem{1ku}
Heart rate variability. Standards of measurement, physiological interpretation, and 
clinical use. Task Force of The Europian Society of Cardiology and The North 
American Society of Pacing and Electrophysiology~// European Heart J., 1996. 
Vol.~17. P.~354--381. 

\bibitem{2ku}
\Au{Амиров Н.\,Б., Чухнин Е.\,В.}
Применение метода изуче\-ния вариабельности сердечного ритма при различных 
состояниях (Обзор литературы)~// Диагностика и лечение нарушений 
регуляции сер\-деч\-но-со\-су\-ди\-стой сис\-те\-мы.~--- М.: ГКГ МВД России, 
2008.~С. 63--75.

\bibitem{3ku}
\Au{Кузнецов А.\,А.}
Методы анализа и обработки электрокардиографических сигналов: Новые 
подходы к выделению информации.~--- Владимир: ВлГУ, 2008.~--- 140~с. 

\bibitem{4ku}
\Au{Малиновский Л.\,Г.}
Классификация объектов средствами дискриминантного анализа.~--- М.: Наука, 
1979.~--- 260~с. 

\bibitem{5ku}
\Au{Зозуля Е.\,П.}
Геометрический анализ нелинейных хаотических колебаний кардиоритма как 
новый метод для автоматического обнаружения фибрилляции предсердий~// 
Физика и радиоэлектроника в медицине и экологии. Кн.~1.~--- 
Владимир--Суздаль: ВлГУ, 2008. С.~172--175.

\bibitem{6ku}
\Au{Кузнецов А.\,А.}
Энтропия ритма сердца.~--- Владимир: ВлГУ, 2009.~--- 172~с. 

\bibitem{7ku}
\Au{Прилуцкий Д.\,А., Кузнецов А.\,А., Плеханов~А.\,А., Чепенко~В.\,В.}
Накопитель ЭКГ <<\textit{AnnA Flash}~2000>>~// Методы и средства 
измерений физических величин.~--- Н.~Новгород: НГТУ, 2006. С.~31.

\bibitem{8ku}
Medical Computer Systems, Zelenograd, Moscow. {\sf http://www.mks.ru}. 

\bibitem{9ku}
\Au{Кушаковский М.\,С., Журавлева Н.\,Б.}
Аритмии и блокады сердца (атлас электрокардиограмм).~--- Л.: Медицина, 
1981.~--- 340~с. 

\bibitem{10ku}
\Au{Мун Ф.}
Хаотические колебания: Вводный курс для научных сотрудников и инженеров~/
Пер. с англ. Ю.\,А.~Данилова и А.\,М.~Шукурова.~--- М.: Мир, 1990.~--- 312~с.

\bibitem{14ku} %11
\Au{Shannon C.\,E., Weaver~W.}
The mathematical theory of communication.~--- Urbana, IL: The University of 
Illinois Press, 1949.

\bibitem{13ku} %12
\Au{Матвеев А.\,Н.}
Молекулярная физика: Учеб. пособие для физ. спец. вузов.~--- М.: Высшая 
школа, 1987.~--- 360~с. 

\bibitem{12ku} %13
Биофизика: Учебник~/ Под ред. акад. П.\,Г.~Костюка.~--- Киев: Выща школа, 
1988.~--- 504~с.

\bibitem{11ku} %14
\Au{Блюменфельд Л.\,А.}
Информация, термодинамика и конструкция биологических систем~// СОЖ, 
1996. №\,7. С.~88--92. 

\bibitem{15ku}
\Au{Вентцель Е.\,С.}
Теория вероятностей: Учебник для вузов.~--- М.: Высшая школа, 1999.~--- 
576~с. 

\bibitem{19ku} %16
\Au{Shaw R.}
Strange attractors, chaotic behavior and information flow~// Z. Naturforsh., 1981. 
Vol.~A36. P.~80--112.

\bibitem{17ku}
\Au{Farmer J.\,D., Ott E., Yorke~J.\,A.}
The dimension of chaotic attractors~// Physica, 1983. Vol.~7D. P.~153--170. 

\bibitem{18ku}
\Au{Grassberger P., Proccacia~I.}
Characterization of strange attractors~// Phys. Rev. Lett., 1983. Vol.~50. 
P.~346--349.

\bibitem{16ku} %19
\Au{Кузнецов А.\,А.}
Энтропия, количество информации и информационная размерность 
$RR$-ин\-тер\-ва\-ло\-грам\-мы~// Биомедицинские технологии и 
радиоэлектроника, 2008. №\,6. С.~15--19.


\bibitem{20ku}
\Au{Баевский Р.\,М., Берсенева А.\,П.}
Введение в донозологическую диагностику.~--- М.: Слово, 2008.~--- 176~с. 

\bibitem{21ku}
\Au{Кобзарь А.\,И.}
Прикладная математическая статистика.~--- М.: Физматлит, 2006.~--- 816~с.

\bibitem{22ku}
\Au{Орлов Ю.\,Н., Осминин К.\,П.}
Методика определения оптимального объема выборки для прогнозирования 
нестационарного временного ряда~// Информационные технологии и 
вычислительные системы, 2008. №\,3. С.~3--13. 

\bibitem{23ku}
\Au{Королюк В.\,С., Портенко Н.\,И., Скороход~А.\,В., Турбин~А.\,Ф.} 
Справочник по теории вероятностей и математической статистике.~--- М.: 
Наука, 1985.~--- 640~с.

\bibitem{25ku} %24
\Au{Babloyantz A., Destexhe A.}
Is the normal heart a periodic oscillator~// Biol. Cybern., 1988. Vol.~58. P.~203.

\bibitem{24ku} %25
\Au{Pool R.}
Is it healthy to be chaotic?~// Science, 1989. Vol.~243. P.~604.


\bibitem{26ku}
\Au{Эйдукайтис А., Варонецкас Г., Жемайтите~Д.}
Применение теории хаоса для анализа сердечного ритма в различных стадиях 
сна у здоровых лиц~// Физиология человека, 2004. Т.~30. №\,5. С.~56--61.

\bibitem{27ku}
\Au{Кузнецов А.\,А.}
Проверка возможности применения функциональных уравнений для оценки 
состава и распределения ритма сердца~// Биомедицинские технологии и 
радиоэлектроника.~--- М.: Радиотехника, 2008. №\,3. С.~17--20.

\bibitem{28ku}
\Au{Кузнецов А.\,А.}
Структурно-топологические особенности диаграмм ритма сердца~// 
Инфокоммуникационные технологии, 2009. Т.~7. №\,3. С.~80--85. 

\bibitem{29ku}
Математическая энциклопедия~/ Гл. ред. И.\,М.~Виноградов. Т.~5.~--- М.: Сов. 
энциклопедия, 1984.~--- 1248~с.

 \label{end\stat}

\bibitem{30ku}
\Au{Арнольд В.\,И.}
Теория катастроф.~--- М.: Наука, 1990.~--- 128~с.
 \end{thebibliography}
}
}


\end{multicols}        %11Abst+avt

%\newcommand{\norm}[1]{\left\Vert#1\right\Vert}
%\newcommand{\abs}[1]{\left\vert#1\right\vert}
%\newcommand{\eps}{\varepsilon}
%\renewcommand{\r}{\mathbb R}
%\newcommand{\N}{\mathbb N}
%\renewcommand{\P}{{\sf P}}
%\newcommand{\E}{{\sf E}}
%\newcommand{\D}{{\sf D}}
%\newcommand{\sign}{{\rm sign}}
%\renewcommand{\le}{\leqslant}
%\renewcommand{\ge}{\geqslant}
%\newcommand{\I}{\mathbb{I}}
%\newcommand{\betm}{{\beta_{m+1+\delta}}}
%\newcommand{\bet}{\beta_{2+\delta}}
%\renewcommand{\endproof}{\hfill$\Box$}
%\renewcommand{\phi}{\varphi}
%\newcommand{\la}{\lambda}
%\newcommand{\si}{{\rm Si}\:}
%\renewcommand{\Re}{{\rm Re}\:}
%\newcommand{\eqd}{\stackrel{d}{=}}

\def\stat{korolev}

\def\tit{ОЦЕНКИ СКОРОСТИ СХОДИМОСТИ РАСПРЕДЕЛЕНИЙ НЕКОТОРЫХ СЛУЧАЙНЫХ СУММ К~УСТОЙЧИВЫМ 
ЗАКОНАМ$^*$}

\def\titkol{Оценки скорости сходимости распределений некоторых случайных сумм к устойчивым 
законам}

\def\autkol{В.\,Ю.~Королев,  Л.\,М.~Закс}

\def\aut{В.\,Ю.~Королев$^1$,  Л.\,М.~Закс$^2$}

\titel{\tit}{\aut}{\autkol}{\titkol}

{\renewcommand{\thefootnote}{\fnsymbol{footnote}}
\footnotetext[1] {Работа поддержана Российским фондом фундаментальных исследований (проекты 
12-07-00115а, 12-07-00109а,  11-01-00515а и 11-07-00112а).}}

\renewcommand{\thefootnote}{\arabic{footnote}}
\footnotetext[1]{Факультет вычислительной математики и кибернетики 
Московского государственного университета им.\ М.\,В.~Ломоносова; Институт 
проблем информатики РАН, vkorolev@cs.msu.su}
\footnotetext[2]{Альфа-банк, 
отдел моделирования и математической статистики, lily.zaks@gmail.com}

\vspace*{6pt}

\Abst{Приведены оценки скорости сходимости распределений
специальных сумм случайного числа независимых одинаково
распределенных случайных величин с конечными дисперсиями к
симметричным строго устойчивым законам. Предполагается, что
случайный индекс имеет смешанное пуассоновское распределение, в
котором смешивающее распределение является устойчивым законом,
сосредоточенным на положительной полуоси. Абсолютные константы
выписаны в явном виде.}

\vspace*{1pt}

\KW{устойчивое распределение; неравенство
Бер\-ри--Эс\-се\-ена; случайная сумма; дважды стохастический пуассоновский
процесс (процесс Кокса); смешанное пуассоновское распределение}


\vspace*{4pt}

 \vskip 14pt plus 9pt minus 6pt

      \thispagestyle{headings}

      \begin{multicols}{2}

            \label{st\stat}



Функцию распределения и плотность строго устойчивого распределения с
характеристическим показателем $\alpha$ и параметром~$\theta$,
задаваемого характеристической функцией
\begin{multline}
\mathfrak{g}_{\alpha,\theta}(t)={}\\
{}=\exp\left\{-|t|^{\alpha}\exp\left\{
-\fr{i\pi\theta\alpha}{2}\mathrm{sign}t\right\}\right\}\,,\enskip
t\in\r\,,\label{e1-kor}
\end{multline}
где $0<\alpha\hm\le2$,
$|\theta|\hm\le\theta_{\alpha}\hm=\min\{1,{2}/{\alpha}-1\}$, будем обозначать 
соответственно $G_{\alpha,\theta}(x)$ и $g_{\alpha,\theta}(x)$. Симметричным 
строго устойчивым распределениям соответствует значение $\theta\hm=0$. 
Односторонним устойчивым распределениям соответствуют значения $\theta\hm=1$ и 
$0\hm<\alpha\hm\le1$. Функцию распределения и плотность стандартного 
нормального закона ($\alpha\hm=2$, $\theta\hm=0$) будем обозначать 
соответственно $\Phi(x)$ и~$\phi(x)$:
$$
\phi(x)=\fr{1}{\sqrt{2\pi}}\,e^{-x^2/2}\,;\quad
\Phi(x)=\int\limits_{-\infty}^x\phi(z)\,dz\,.
$$

Рассмотрим последовательность независимых одинаково распределенных
случайных величин $X_1,X_2,\ldots$, заданных на некотором
вероятностном пространстве $(\Omega,\, \mathfrak{A},\,{\sf P})$.
Будем предполагать, что
\begin{equation*}
{\sf E}X_1=0\,, \enskip 0<\sigma^2={\sf D}X_1<\infty\,. %\label{e2-kor}
\end{equation*}
Для натурального $n\hm\ge1$ положим
$$
S_n=X_1+\cdots+X_n\,.
$$
Пусть $N_1,N_2,\ldots$~--- последовательность це\-ло\-чис\-лен\-ных 
неотрицательных случайных величин, заданных на том же самом вероятностном 
пространстве так, что при каждом $n\hm\ge1$ случайная величина $N_n$ независима 
от последовательности $X_1,X_2,\ldots$ Всюду далее для определенности будем 
считать, что $\sum\limits_{j=1}^0\hm=0$.

Принято считать, что случайная последовательность $N_1,N_2,\ldots$
неограниченно возрастает ($N_n\hm\longrightarrow\infty$) по
вероятности, если для любого $m\hm\in(0,\infty)$ ${\sf P}(N_n\hm\le
m)\longrightarrow 0$ при $n\hm\to\infty$. Всюду далее символы
$\Longrightarrow$ и $\eqd$ обозначают соответственно сходимость по
распределению и совпадение распределений.

В статье~\cite{Korolev1997} доказан следующий критерий схо\-димости
сумм случайного числа независимых одинако\-во распределенных случайных
величин \textit{с конечными дисперсиями} к симметричным строго
устойчивым законам.

\smallskip

\noindent
\textbf{Лемма 1.} \textit{Предположим, что случайные величины
$X_1,X_2,\ldots$ и $N_1,N_2,\ldots$ удовлетворяют указанным выше
условиям, причем $N_n\longrightarrow\infty$ по вероятности при
$n\hm\to\infty$. Для того чтобы при} $n\hm\to\infty$
$$
{\sf P}\left(\fr{S_{N_n}}{\sigma\sqrt{n}}<x\right) \Longrightarrow
G_{\alpha,0}(x)\,,
$$
\textit{необходимо и достаточно, чтобы}
$$
{\sf P}(N_n<nx)\Longrightarrow G_{\alpha/2,1}(x)\,.
$$



В лемме~1 главным условием является сходимость распределений
нормированных индексов $N_n$ к одностороннему строго устойчивому
распределению $G_{\alpha/2,1}(x)$. Далее будет рассматриваться
довольно полезная с точки зрения практических приложе\-ний специальная
ситуация, в которой это условие выполнено.

В книге~\cite{GnedenkoKorolev1996} предложено моделировать эволюцию\linebreak
неоднородных хаотических стохастических процессов, в частности
динамику цен финансовых активов, с помощью обобщенных дважды
стохастических пуассоновских процессов (обобщенных\linebreak процессов Кокса).
Этот подход получил дополнительное обоснование и развитие в книгах~[3--6]. 
В~книгах~\cite{Korolev2011, KorolevSkvortsova2006} этот
подход успешно применен к моделированию процессов плазменной
турбулентности. В~соответствии с указанным подходом поток
информативных событий, в результате каждого из которых появляется
очередное <<наблюденное>> значение рассматриваемой характеристики,
описывается с помощью точечного случайного процесса вида
$M(\Lambda(t))$, где $M(t)$, $t\hm\geq0$,~--- однородный пуассоновский
процесс с единичной ин\-тен\-сив\-ностью, а $\Lambda(t)$, $t\hm\geq0$,~---
независимый от $M(t)$ случайный процесс, обладающий следующими
свойствами: $\Lambda(0)\hm=0$, ${\sf P}(\Lambda(t)\hm<\infty)\hm=1$ для
любого $t\hm>0$, траектории $\Lambda(t)$ не убывают и непрерывны
справа. Процесс $M(\Lambda(t))$, $t\hm\geq0$, называется дважды
стохастическим пуассоновским процессом (процессом Кокса). 
В~частности, если процесс $\Lambda(t)$ допускает представление
$$
\Lambda(t)=\int\limits_{0}^{t}\lambda(\tau)\,d\tau\,,\enskip t\ge0\,,
$$
в котором $\lambda(t)$~--- положительный случайный процесс с
интегрируемыми траекториями, то $\lambda(t)$ можно интерпретировать
как мгновенную стохастическую интенсивность процесса Кокса.

В соответствии с такой моделью в каждый момент времени~$t$
распределение случайной величины $M(\Lambda(t))$ является смешанным
пуассоновским. Для большей наглядности рассмотрим случай, когда в
рассматриваемой модели время~$t$ остается фиксированным, а
$\Lambda(t)\hm=nU_{\alpha/2,1}$, где $n$~--- вспомогательный параметр,
$U_{\alpha/2,1}$~--- случайная величина c функцией распределения
$G_{\alpha/2,1}(x)$, независимая от стандартного пуассоновского
процесса $M(t)$, $t\hm\ge0$. При этом асимптотика $n\hm\to\infty$ может
интерпретироваться как то, что (случайная) интенсивность потока
информативных событий считается очень большой. Для каждого
натурального~$n$ положим
$$
N_n=M(nU_{\alpha/2,1})\,.
$$
Очевидно, что так определенная случайная величина $N_n$ имеет
смешанное пуассоновское распределение:

\noindent
\begin{multline}
{\sf P}(N_n=k)={\sf P}
\left(M(nU_{\alpha/2,1})=k\right)={}\\
{}=
\int\limits_0^{\infty}e^{-nz}\fr{(nz)^k}{k!}\,g_{\alpha/2,1}(z)\,dz\,,\enskip
k=0,1,\ldots
\label{e3-kor}
\end{multline}
Случайная величина $N_n$ может быть интер\-пре\-ти\-рована как число
событий, зарегистрированных к моменту времени~$n$ в пуассоновском
процессе со случайной интенсивностью, имеющей строго устойчивую
плотность $g_{\alpha/2,1}(z)$. Высокая адекватность устойчивых
распределений как моделей статистических закономерностей динамики
цен финансовых активов отмечается во многих работах (см., например,~\cite{McCulloch1996}).

Предположим, что случайная величина $U_{\alpha/2,1}$ и пуассоновский
процесс $M(t)$ независимы от последовательности $X_1,X_2,\ldots$
Тогда, очевидно, при каждом~$n$ случайная величина~$N_n$ также будет
независима от этой последовательности.

Обозначим $A_n(z)={\sf P}(N_n<nz)$, $z\hm\ge0$ ($A_n(z)\hm=0$ при $z\hm<0$).
Несложно видеть, что
$$
A_n(x)\Longrightarrow G_{\alpha/2,1}(x)\enskip (n\to\infty)\,.
$$
Действительно, как известно, если $\Pi(x;\ell)$~--- функция
распределения Пуассона с параметром $\ell\hm>0$ и $E(x;c)$~--- функция
распределения с единственным единичным скачком в точке $c\hm\in\r$, то
$$
\Pi(\ell x;\ell)\Longrightarrow E(x;1)\enskip (\ell\to\infty)\,.
$$
Так как для $x\in\r$
$$
A_n(x)=\int\limits_{0}^{\infty}\Pi(n x; n z)\,dG_{\alpha/2,1}(z)\,,
$$
то по теореме Лебега о мажорируемой сходимости при $n\hm\to\infty$
\begin{multline*}
A_n(x)\Longrightarrow\int\limits_{0}^{\infty}E(x/z;1)\,dG_{\alpha/2,1}(z)={}\\
{}=
\int\limits_{0}^{x}\,\,dG_{\alpha/2,1}(z)=G_{\alpha/2,1}(x)\,,
\end{multline*}
т.\,е.\ так определенные случайные величины $N_n$\linebreak удовлетворяют
условию, фигурирующему в леммe~1.

В дополнение к сформулированным выше условиям на случайные величины
$X_1,X_2,\ldots$ предположим, что
\begin{equation}
\beta^3={\sf E}|X_1|^3<\infty\,.\label{e4-kor}
\end{equation}
Обозначим
$$
D_{n,\alpha}=\sup_x\left\vert {\sf
P}\left(S_{N_n}<x\sigma\sqrt{n}\right)-G_{\alpha,0}(x)\right\vert\,.
$$


\smallskip

\noindent
\textbf{Теорема~1.} \textit{Пусть выполнены условия}~(\ref{e3-kor}) и~(\ref{e4-kor}). \textit{Для
любого $n\ge1$ справедлива оценка}
$$
D_{n,\alpha}\le0{,}2428
\fr{\Gamma\left({1}/{\alpha}\right)\beta^3}{\alpha\sigma^3\sqrt{n}}\,.
$$

\smallskip

\noindent
Д\,о\,к\,а\,з\,а\,т\,е\,л\,ь\,с\,т\,в\,о\,.\ \  Распределение случайной величины $N_n$
является смешанным пуассоновским (см.~(\ref{e3-kor})). Следовательно, по теореме
Фубини
\begin{multline}
{\sf P}\left(S_{N_n}<x\sigma\sqrt{n}\right)=
{\sf P}\left(S_{M(nU_{\alpha/2,1})}<x\sigma\sqrt{n}\right)={}\\
{}=\int\limits_0^{\infty}{\sf P}\left(S_{M(nz)}<x\sigma\sqrt{n}\right)
g_{\alpha/2,1}(z)\,dz\,.\label{e5-kor}
\end{multline}
Далее, как известно, симметричное строго устойчивое распределение с
параметром~$\alpha$ является масштабной смесью нормальных законов, в
которой смешивающим распределением является односторонний устойчивый
закон ($\theta\hm=1$) с параметром $\alpha/2$:
\begin{equation}
G_{\alpha,0}(x)=\int\limits_{0}^{\infty}\Phi\left(\fr{x}{\sqrt{z}}\right)\,dG_{\alpha/2,1}(z)\,,\enskip
x\in\r\label{e6-kor}
\end{equation}
(см., например, теорему~3.3.1 в~\cite{Zolotarev1983}). Из~(\ref{e5-kor}) и~(\ref{e6-kor})
следует, что
\begin{multline}
D_{n,\alpha}\le{}\\
\!{}\le\int\limits_0^{\infty}\!\!\sup\limits_x\left\vert{\sf P}\!
\left(\fr{S_{M(nz)}}{\sigma\sqrt{n}}<x\right)-
\Phi\left(\fr{x}{\sqrt{z}}\right)\right\vert\,dG_{\alpha/2,1}(z)={}\\
\!\!\!{}= \int\limits_0^{\infty}\sup\limits_x \left\vert{\sf P}
\left(\fr{S_{M(nz)}}{\sigma\sqrt{nz}}<x\right)-\Phi(x)\right\vert\,dG_{\alpha/2,1}(z).\!\!
\label{e7-kor}
\end{multline}
Подынтегральное выражение в~(\ref{e7-kor}) оценим с по\-мощью следующего аналога
неравенства Бер\-ри--Эс\-се\-ена для пуассоновских случайных сумм.

\medskip

\noindent
\textbf{Лемма 2.} \textit{Пусть случайные величины $X_1,X_2,\ldots$
одинаково распределены, причем ${\sf E}X_1\hm=0$ и ${\sf E}|X_1|^3\hm<\infty$. 
Пусть $N_{\lambda}$~--- пуассоновская случайная
величина с параметром $\lambda\hm>0$. Предположим, что случайные
величины $N_{\lambda},X_1,X_2,\ldots$ независимы в совокупности.
Обозначим}
$$
Z_{\lambda}=X_1+\cdots+X_{N_{\lambda}}\,.
$$
\textit{Тогда}
$$
\sup\limits_x\left\vert {\sf P}\left(\fr{Z_{\lambda}}{\sqrt{{\sf D}Z_{\lambda}}}<x \right)-
\Phi(x)\right\vert\le\fr{0{,}3041}{\sqrt{\lambda}}\,\fr{{\sf E}|X_1|^3}{({\sf E}X_1^2)^{3/2}}\,.
$$

\smallskip

\noindent
Д\,о\,к\,а\,з\,а\,т\,е\,л\,ь\,с\,т\,в\,о\ этого утверждения приведено 
в~\cite{KorolevShevtsova2010}, также см.\ теорему~2.4.3 в~\cite{KorolevBeningShorgin2011}.

\smallskip

Далее понадобится следующее утверждение, позволяющее вычислить ${\sf E}U_{\alpha/2,1}^{-1/2}$, 
несмотря на то что плот\-ность
$g_{\alpha/2,1}(z)$, вообще говоря, нельзя выписать в явном виде в
терминах элементарных функций.

\medskip

\noindent
\textbf{Лемма 3.}
\begin{equation*}
{\sf E}U_{\alpha/2,1}^{-1/2}=\fr{\sqrt{2}\Gamma\left({1}/{\alpha}\right)}{\alpha\sqrt{\pi}}\,.
%\label{e8-kor}
\end{equation*}

\smallskip

\noindent
Д\,о\,к\,а\,з\,а\,т\,е\,л\,ь\,с\,т\,в\,о\,.\ Из~(\ref{e1-kor}) вытекает, что характеристическая
функция симметричного ($\theta\hm=0$) строго устойчивого распределения
имеет вид:
\begin{equation}
\mathfrak{f}_{\alpha,0}(t)=e^{-|t|^{\alpha}}\,,\enskip t\in\r\,. \label{e9-kor}
\end{equation}
С другой стороны, записав соотношение~(\ref{e6-kor}) в терминах
характеристических функций с учетом~(\ref{e9-kor}), получим
\begin{equation}
e^{-|t|^{\alpha}}=\int\limits_0^{\infty}\exp\left\{-\fr{t^2z}{2}\right\}
g_{\alpha/2,1}(z)\,dz\,.\label{e10-kor}
\end{equation}
Обозначим
$$
h_{\alpha/2}(z)=\fr{\alpha}{\Gamma({1}/{\alpha})}\sqrt{\fr{\pi}{2}}\,
\fr{g_{\alpha/2,1}(z)}{\sqrt{z}}\,,\enskip
z\ge0\,.
$$
Обобщенным распределением Лапласа принято называть абсолютно
непрерывное распределение вероятностей, задаваемое плотностью
$$
\ell_{\alpha}(x)=\fr{\alpha}{2\Gamma({1}/{\alpha})} \,
e^{-|x|^{\alpha}}\,,\enskip -\infty< x<\infty\,.
$$
Переобозначив аргумент $t\mapsto x$ и выполнив несколько формальных
тождественных преобразований равенства~(\ref{e10-kor}), будем иметь:
\begin{multline}
\ell_{\alpha}(x)=\fr{\alpha}{2\Gamma({1}/{\alpha})}e^{-|x|^{\alpha}}={}\\
{}=
\fr{\alpha}{\Gamma({1}/{\alpha})}\sqrt{\fr{\pi}{2}}\,\int\limits_0^{\infty}
\fr{\sqrt{z}}{\sqrt{2\pi}}\,\exp\left\{-\fr{x^2z}{2}\right\}
\fr{g_{\alpha/2,1}(z)}{\sqrt{z}}\,dz={}\\
{}=
\int\limits_0^{\infty}\sqrt{z}\phi(x\sqrt{z})h_{\alpha/2}(z)\,dz\,.\label{e11-kor}
\end{multline}
Можно убедиться, что $h_{\alpha/2}(z)$~--- плотность распределения
неотрицательной случайной величины. Действительно, при каждом $z\hm>0$
$$
\int\limits_{-\infty}^{\infty}\sqrt{z}\phi(x\sqrt{z})\,dx=1\,.
$$
Поэтому из~(\ref{e11-kor}) вытекает, что
\begin{multline*}
1=\int\limits_{-\infty}^{\infty}\!\ell_{\alpha}(x)\,dx=
\int\limits_{-\infty}^{\infty}\!\int\limits_{0}^{\infty}\!\sqrt{z}\phi(x\sqrt{z})
h_{\alpha/2}(z)\,dz dx={}
\\
{}=\int\limits_{0}^{\infty}\!h_{\alpha/2}(z)\left(\,
\int\limits_{-\infty}^{\infty}\sqrt{z}\phi(x\sqrt{z})\,dx\right)\,dz={}\\
{}=
\int\limits_{0}^{\infty}\!h_{\alpha/2}(z)\,dz.
\end{multline*}
Лемма доказана.

\smallskip

Продолжив~(\ref{e7-kor}) с учетом лемм~2 и~3, получим
\begin{multline*}
D_{n,\alpha}\le0{,}3041\,\fr{\beta^3}{\sigma^3\sqrt{n}}\,
{\sf E}U_{\alpha/2,1}^{-1/2}={}\\
{}=0{,}3041
\fr{\sqrt{2}\Gamma\left({1}/{\alpha}\right)}{\alpha\sqrt{\pi}}\,
\fr{\beta^3}{\sigma^3\sqrt{n}}\,.
\end{multline*}
Теорема доказана.

\vspace*{-6pt}

{\small\frenchspacing
{%\baselineskip=10.8pt
\addcontentsline{toc}{section}{Литература}
\begin{thebibliography}{99}

\bibitem{Korolev1997} 
\Au{Королев В.\,Ю.} О~сходимости pаспpеделений случайных сумм независимых
случайных величин к устойчивым законам~// Теоpия веpоятностей и ее
пpименения, 1997. Т.~42. Вып.~4. С.~818--820.

\bibitem{GnedenkoKorolev1996} 
\Au{Gnedenko B.\,V., Korolev~V.\,Yu.} Random summation:
Limit theorems and applications.~--- Boca Raton: CRC Press, 1996.

\bibitem{BeningKorolev2002} 
\Au{Bening V., Korolev V.} Generalized Poisson models and their applications in
insurance and finance.~--- Utrecht: VSP, 2002. 434~p.

\bibitem{KorolevSokolov2008} 
\Au{Королев В.\,Ю., Соколов И.\,А.} Математические модели
неоднородных потоков экстремальных событий.~--- М.: ТОРУС ПРЕСС,
2008.

\bibitem{KorolevBeningShorgin2011} 
\Au{Королев В.\,Ю., Бенинг В.\,Е., Шоргин~С.\,Я.}
Математические основы теории риска.~--- 2-е изд., перераб. и доп.~--- М.:
Физматлит, 2011. 620~с.

\bibitem{Korolev2011} 
\Au{Королев В.\,Ю.} Вероят\-но\-ст\-но-ста\-ти\-сти\-че\-ские методы
декомпозиции волатильности хаотических процессов.~--- М.: Изд-во
Московского ун-та, 2011. 510~с.

\bibitem{KorolevSkvortsova2006} 
Stochastic models of structural plasma turbulence~/
Eds. V.~Korolev, N.~Skvortsova.~--- Utrecht: VSP, 2006. 400~p.

\bibitem{McCulloch1996} 
\Au{McCulloch J.\,H.} Financial applications of stable
distributions~// Handbook of statistics.~--- Amsterdam:
Elsevier Science, 1996.  Vol.~14. P.~393--425.

\bibitem{Zolotarev1983} 
\Au{Золотарев В.\,М.} Одномерные устойчивые
распределения.~--- М.: Наука, 1983.

\label{end\stat}

\bibitem{KorolevShevtsova2010} 
\Au{Korolev V., Shevtsova~I.} An improvement of the Berry--Esseen
inequality with applications to Poisson and mixed Poisson random
sums~// Scandinavian Actuarial~J., 2012. Vol.~2012. No.\,2.
P.~81--105. Available online since June~4, 2010.

\end{thebibliography} } }

\end{multicols}     %12 Abst+avt      
\def\stat{bening}

\def\tit{АСИМПТОТИЧЕСКИЕ РАЗЛОЖЕНИЯ ДЛЯ ФУНКЦИЙ РАСПРЕДЕЛЕНИЯ
СТАТИСТИК, ПОСТРОЕННЫХ\\ ПО ВЫБОРКАМ СЛУЧАЙНОГО ОБЪЕМА$^*$}

\def\titkol{Асимптотические разложения для функций распределения
статистик, построенных по выборкам случайного объема}

\def\autkol{В.\,Е.~Бенинг, Н.\,К.~Галиева, В.\,Ю.~Королев}

\def\aut{В.\,Е.~Бенинг$^1$, Н.\,К.~Галиева$^2$, В.\,Ю.~Королев$^3$}

\titel{\tit}{\aut}{\autkol}{\titkol}

{\renewcommand{\thefootnote}{\fnsymbol{footnote}}\footnotetext[1]
{Работа
поддержана Российским фондом фундаментальных исследований (проекты
11-01-00515а, 11-07-00112а, 11-01-12026-офи-м), Министерством
образования и науки РФ (госконтракт 16.740.11.0133).}}

\renewcommand{\thefootnote}{\arabic{footnote}}
\footnotetext[1]{Факультет вычислительной
математики и кибернетики Московского государственного университета
им.\ М.\,В.~Ломоносова; Институт проблем информатики Российской
академии наук, bening@yandex.ru}
\footnotetext[2]{Казахстанский филиал Московского государственного
университета им.\ М.\,В.~Ломоносова, nurgul\_u@mail.ru}
\footnotetext[3]{Факультет вычислительной
математики и кибернетики Московского государственного университета
им.\ М.\,В.~Ломоносова; Институт проблем информатики Российской
академии наук, vkorolev@cs.msu.su}

\vspace*{4pt}

\Abst{Доказана общая теорема переноса,
позволяющая получать асимптотические разложения (а.р.)\ для функций
распределения (ф.р.)\ статистик, основанных на выборках случайного объема,
из а.р.\  для ф.р.\ случайного
объема выборки и а.р.\ для  ф.р.\ статистик, построенных по выборкам неслучайного
объема.}

\vspace*{2pt}

\KW{выборка случайного объема;
асимптотическое разложение; теорема переноса; смесь вероятностных
законов; распределение Лапласа; распределение Стьюдента}

\vspace*{4pt}

\vskip 14pt plus 9pt minus 6pt

      \thispagestyle{headings}

      \begin{multicols}{2}

            \label{st\stat}


\section{Введение}

В классических задачах математической статистики объем выборки,
доступной исследователю, традиционно считается детерминированным и в
асимптотических постановках играет роль (как правило, неограниченно
возрастающего) {\it известного} параметра. В~то же время на практике
час\-то возникают ситуации, когда размер выборки не является заранее
определенным и может рас\-смат\-ри\-вать\-ся как случайный. Такие ситуации,
как правило, связаны с тем, что статистические данные накапливаются
в течение фиксированного времени. Это \mbox{имеет} место, в частности, в
страховании, когда в течение разных отчетных периодов одинаковой
длины (скажем, месяцев) происходит разное число страховых событий
(страховых выплат и/или заключений страховых контрактов); в
медицине, когда число пациентов с тем или иным заболеванием
варьируется от года к году; в технике, когда при испытании на
надежность (скажем, при определении наработки на отказ) разных
партий приборов (изделий), чис\-ло отказавших приборов в разных
партиях оказывается разным. В~таких ситуациях заранее не известное
число наблюдений, которые будут доступны исследователю, разумно
считать случайной величиной (с.в.). Другими словами, в таких ситуациях
объем выборки является не (известным) параметром, а сам становится
{\it наблюдением}, т.\,е.\ статистикой. В~силу указанных
обстоятельств вполне естественным становится изуче\-ние
асимптотического поведения распределений статистик достаточно общего
вида, основанных на выборках случайного объема.

На естественность такого подхода, в частности, обратил внимание
Б.\,В.~Гнеденко в работе~\cite{2-ben}, в которой рассматривались
асимптотические свойства\linebreak распределений выборочных квантилей,
построенных по выборкам случайного объема, и было
продемонстрировано, что при замене неслучайного \mbox{объема} выборки
случайной величиной асимптотические свойства статистик могут
радикально измениться. К~примеру, если объем выборки является
геометрически распределенной с.в., то вместо
ожидаемого в соответствии с классической теорией нормального закона
в качестве асимптотического распределения выборочной медианы
возникает распределение Стьюдента с двумя степенями свободы, хвосты
которого столь тяжелы, что у него отсутствуют моменты порядков,
б$\acute{\mbox{о}}$льших второго. <<Тяжесть>> хвостов асимптотических распределений
имеет же  критически важное значение, в частности, в задачах проверки
гипотез.

Простейшей статистикой является сумма наблюдений. Для выборок
случайного объема число слагаемых в таких суммах само становится
случайным. Асимптотическим свойствам распределений сумм случайного
числа с.в.\ посвящено много работ (см., например,~[1--7]). 
Такого рода суммы находят широкое применение в страховании,
экономике, биологии и~т.\,п.~[2, 5, 7, 8]. В~классической статистике
суммирование наблюдений, как правило, возникает при определении
выборочных средних. При статистическом анализе, основанном на
моделях, в которых объем выборки считается неслучайным,
асимптотическое поведение статистик типа сумм и статистик типа
средних арифметических одинаково~--- эти статистики после нормировки,
обязательной для получения нетривиальных предельных распределений,
становятся неразличимыми. Однако, как уже говорилось, в реальной
практике очень часто объем выборки сам является статистикой и, как
недавно показано, например, в работе~\cite{24-ben}, асимптотическое
поведение статистик типа сумм и статистик типа средних
арифметических при их неслучайной нормировке оказывается различным.
Заметим, что, конечно же, формально допустима и случайная
нормировка, но для построения разумных асимптотических аппроксимаций
для распределений статистик (а именно это и является целью
асимптотической статистики) она не применима. Именно использованием
неслучайной нормировки и объясняется возникновение не <<чис\-то\-го>>
нормального закона, а (разных!) смешанных нормальных предельных
распределений у статистик типа сумм и типа средних арифметических.
При этом различие этих предельных законов может дать дополнительную
информацию о структуре исходных данных.

Более того, в математической статистике и ее приложениях часто
встречаются статистики, которые не являются суммами наблюдений.
Примерами могут служить ранговые статистики, $U$-ста\-ти\-сти\-ки,
линейные комбинации порядковых статистик\linebreak ($L$-ста\-ти\-сти\-ки) и~т.\,п.

В данной работе получены а.р.\ для
ф.р.\ статистик, построенных по выборкам
случайного объема. Эти а.р.\ непосредственно зависят от а.р.\ ф.р.\
случайного объема выборки и а.р.\ ф.р.\ статистики, основанной на
неслучайной выборке. Подобного рода утверждения принято называть
тео\-ре\-ма\-ми переноса. Таким образом, в данной работе доказаны теоремы
переноса для а.р.\ статистик, построенных по выборкам случайного
объема.

В работе приняты следующие обозначения: $\R$~--- множество
вещественных чисел; $\N$~--- множество натуральных чисел; $\Phi(x)$ и
$\varphi(x)$~--- соответственно ф.р.\ и плот\-ность стандартного
нормального закона.

В разд.~2 приведен эвристический вывод основного результата, в
разд.~3--5 содержатся строгая формулировка основной теоремы, ее
доказательство и примеры.

Рассмотрим с.в.\ $N_1, N_2, \ldots$ и  $X_1, X_2,\ldots$, 
заданные на одном и том же вероятностном пространстве
$(\Omega, {\cal A}, {\p})$. В~статистике с.в.\ $X_1, X_2, \ldots X_n$
имеют смысл наблюдений, $n$~--- неслучайный объем выборки, а с.в.\
$N_n$~--- случайный объем выборки, зависящий от натурального
параметра $n\hm\in \N$. Например, если с.в.~$N_n$ имеет геометрическое
распределение вида
$$
{\p}(N_n = k) =  \fr{1}{n} \left(1 - \fr{1}{n}\right)^{k-1}\,,\enskip
 k\in\N\,,
$$
то
$$
\e N_n = n\,,
$$
т.\,е.\ среднее значение случайного объема выборки равно~$n$.

Предположим, что при каждом $n\hm\geq1$ с.в.~$N_n$ принимают только
натуральные значения (т.\,е.\ $N_n\hm\in \N$) и независимы от
последовательности с.в.\ $X_1, X_2, \ldots$ Всюду далее считаем с.в.\
$X_1, X_2, \ldots$ независимыми и одинаково распределенными.

Обозначим через  $T_n=T_n(X_1,\ldots ,X_n)$ некоторую статистику, т.\,е.\
действительную измеримую функцию от наблюдений $X_1,\ldots ,X_n$.
Для каждого  $n\hm\geq1$ определим с.в.\ $T_{N_n}$, полагая
$$
T_{N_n}(\omega)\equiv
T_{N_n(\omega)}(X_1(\omega),\ldots ,X_{N_n(\omega)}(\omega))\,, \enskip
\omega\in\Omega\,.
$$
Можно сказать, что $T_{N_n}$~--- это статистика, построенная на
основе статистики $T_n$ по выборке случайного объема $N_n$.

Сформулируем условие, определяющее а.р.\ для ф.р.\ статистики $T_n$
при неслучайном объеме вы\-борки.

\smallskip

\noindent
\textbf{Условие 1.1.} \textit{Существуют константы  $l\hm\in\N$, $\mu\hm\in\R$,
$\sigma\hm>0$, $\alpha\hm>l/2$, $\gamma\hm\ge0$, $C_1\hm>0$, дифференцируемая
ф.р.\ $F(x)$ и дифференцируемые ограниченные функции $f_j(x)$,
$j=1,\ldots ,l$, такие что}
\begin{multline*}
\sup\limits_x \left|\vphantom{\sum\limits_{j=1}^l}
{\p}\left(\sigma n^\gamma(T_n - \mu) < x\right) -\ F(x)
-{}\right.\\
\left.{}- \sum\limits_{j=1}^l n^{-j/2} f_j(x) \right|  \leq \fr{C_1}{n^{\alpha}}\,,
  \ \ \ n\in\N\,.
\end{multline*}

\smallskip

Следующее условие определяет а.р.\ ф.р.\ нормированного случайного
индекса $N_n$.

\smallskip

\noindent
\textbf{Условие 1.2.} \textit{Существуют константы  $m\hm\in\N$, $\beta\hm>m/2$,
$C_2\hm>0$, функция $0\hm<g(n)\uparrow \infty$, $n\hm\to\infty$, ф.р.~$H(x),
H(0+)\hm=0$, и функции ограниченной вариации $h_i(x)$, $i\hm=1,\ldots ,m$, такие
что}
\begin{multline*}
\sup\limits_{x\ge0} \left|{\p}\left(\fr{N_n}{g(n)} < x\right) - H(x)\ -
\sum\limits_{i=1}^m n^{-i/2} h_i(x) \right|  \leq{}\\
{}\leq \fr{C_2}{n^{\beta}}\,, \ \
\  n\in\N\,.
\end{multline*}

\smallskip

В данной работе с помощью а.р.\ для ф.р.\ нормированной статистики
$T_{N_n}$, основанной на выборке случайного объема, получена
аппроксимация вида
\begin{equation*}
{\p}\left(\sigma g^\gamma(n)(T_{N_n}  -  \mu)  <  x\right)\ \approx
G_{n}(x)\,, \ \  n \to \infty\,,
%\label{e1.1-ben}
\end{equation*}
где функция $G_n(x)$ имеет вид (см.\ условия~1.1, 1.2):
\begin{multline}
G_n(x) = {}\\
{}=\!\!\int\limits_{1/g(n)}^\infty\!\!\!\! F(x y^\gamma)\, dH(y) +
\sum\limits_{i=1}^m n^{-i/2}\! \int\limits_{1/g(n)}^\infty \!\!F(xy^\gamma)\,
dh_i(y)\ +{}
\\
{}+ \sum\limits_{j=1}^l g^{-j/2}(n)\! \int\limits_{1/g(n)}^\infty \!
y^{-j/2}f_j(xy^\gamma)\, dH(y) +{}\\
\hspace*{-3mm}{}+
\sum\limits_{j=1}^l \sum\limits_{i=1}^m n^{-i/2}g^{-j/2}(n)
\!\!\!\int\limits_{1/g(n)}^\infty y^{-j/2}\!\!f_j(xy^\gamma)
\,dh_i(y).\!
\label{e1.2-ben}
\end{multline}
Для пояснения этой формулы, идеи доказательства и удобства
дальнейших ссылок приведем ее эвристический вывод.

\section{Эвристический вывод основного результата}

В идейном плане доказательство основного результата данной работы~---
теоремы~3.1~--- близко к доказательству теорем 6.6.1 и 6.7.3 для
случайных сумм из работы~\cite{6-ben} и оценкам скорости сходимости
распределений случайно индексированных последовательностей из работы~\cite{14-ben} 
($\S$~1.3, с.~63).

По формуле полной вероятности имеем:
\begin{multline}
{\p}\left(\sigma g^\gamma(n)(T_{N_n} - \mu) < x\right)  ={}\\
{}=
{\p}\left(\sigma N_n^\gamma(T_{N_n} - \mu) <
\left(\fr{N_n}{g(n)}\right)^\gamma x\right)  ={}\\
{}
= {\e} {\p}\left(\sigma N_n^\gamma(T_{N_n} - \mu) <
\left(\fr{N_n}{g(n)}\right)^\gamma x \Big| N_n\right)  ={}\\
\!{}=
\sum\limits_{k=1}^{\infty} {\p}\left(\!\sigma k^\gamma(T_{k}  -  \mu) <
\left(\fr{k}{g(n)}\right)^\gamma\!\! x\right){\p}(N_n=k).\!\!
\label{e2.1-ben}
\end{multline}
Используя условие~1.1, вероятность под знаком ряда аппроксимируем
следующим образом:
\begin{multline*}
\!\!\!\!{\p}\left(\sigma g^\gamma(n)(T_{N_n} - \mu) < x\right) \approx
\sum\limits_{k=1}^{\infty} \!\left(\!
\vphantom{\sum\limits_{j=1}^l\
k^{-j/2}f_j\left(x\left(\fr{k}{g(n)}\right)^\gamma\right)}
F\left(x
\left(\fr{k}{g(n)}\right)^\gamma\right) + {}\right.\hspace*{-1.8927pt}\\
\left.{}+\sum\limits_{j=1}^l\
k^{-j/2}f_j\left(x\left(\fr{k}{g(n)}\right)^\gamma\right)\right)
{\p}(N_n=k) ={}
\end{multline*}

\noindent
\begin{multline}
{}
= {\e} \left( \vphantom{\sum\limits_{j=1}^l\
k^{-j/2}f_j\left(x\left(\fr{k}{g(n)}\right)^\gamma\right)}
F\left(x \left(\fr{N_n}{g(n)}\right)^\gamma\right) +{}\right.\\
\left.{}+
\sum\limits_{j=1}^l 
N_n^{-j/2}f_j\left(x\left(\fr{N_n}{g(n)}\right)^\gamma\right)\right)
= \!\int\limits_{1/g(n)}^\infty \!\left(
\vphantom{\sum\limits_{j=1}^l}
F(x y^\gamma) +{}\right.\\
\left.{}+ \sum\limits_{j=1}^l
\left(yg(n)\right)^{-j/2}f_j(xy^\gamma)\right)\, d
{\p}\left(\fr{N_n}{g(n)} < y\right)\,.\label{e2.2}
\end{multline}
Теперь, аппроксимируя вероятность под знаком последнего интеграла с
помощью условия~1.2, получим формулу~(\ref{e1.2-ben}):
\begin{multline}
\!\!\!{\p}\left(\sigma g^\gamma(n)(T_{N_n} - \mu) < x\right) \approx G_n(x)
=\!\!\! \int\limits_{1/g(n)}^\infty\! \!\!\left(\!
\vphantom{\sum\limits_{j=1}^l}
F(x y^\gamma) + {}\right.\hspace*{-3.19522pt}\\
\left.{}+\sum\limits_{j=1}^l
\left(yg(n)\right)^{-j/2}f_j(xy^\gamma)\right)\, d \left(
\vphantom{\sum\limits_{i=1}^m}
H(y) +{}\right.\\
\left.{}+
\sum\limits_{i=1}^m n^{-i/2} h_i(y)\right) ={}\\
{}
 = \!\!\int\limits_{1/g(n)}^\infty\!\! F(x y^\gamma)\, dH(y) + 
 \sum\limits_{i=1}^m n^{-i/2} \!\!\int\limits_{1/g(n)}^\infty \!\!F(xy^\gamma) \,dh_i(y) +{}
\\
{}+ \sum\limits_{j=1}^l g^{-j/2}(n) \!\int\limits_{1/g(n)}^\infty\!
y^{-j/2}f_j(xy^\gamma) \,dH(y) +{}\\
\!\!\!{}+
\sum\limits_{j=1}^l \sum\limits_{i=1}^m n^{-i/2}g^{-j/2}(n)\!\!
\int\limits_{1/g(n)}^\infty \!\!y^{-j/2}f_j(xy^\gamma)
\,dh_i(y).\!\!
\label{e2.3-ben}
\end{multline}
Если в условии~1.1 статистика $T_n$ не нормирована, т.\,е.\
$\gamma\hm=0$, то полученное а.р.\ приобретает вид:
\begin{multline*}
G_n(x) =   F(x)\left(1 - H\left(\fr{1}{g(n)}\right)\right) +
 \sum\limits_{j=1}^l \left(g(n)\right)^{-j/2}\times{}\\
{}\times f_j(x)
\int\limits_{1/g(n)}^\infty y^{-j/2} d \left(H(y) + \sum\limits_{i=1}^m
n^{-i/2} h_i(y)\right)\,. %\label{e2.4-ben}
\end{multline*}
Таким образом, у ненормированной статистики $T_n$ исходное а.~р.
$$
{\p}\left(\sigma (T_n  -  \mu)  <  x\right) \approx  F(x) +
\sum\limits_{j=1}^l n^{-j/2} f_j(x)
$$
при переходе к выборке случайного объема $N_n$  заменяется на а.р.\
вида:

\noindent
\begin{multline*}
{\p}\left(\sigma (T_{N_n}  -  \mu)  <  x\right) \approx {}\\
{}\approx  F(x)\left(1 -
H\left(\fr{1}{g(n)}\right)\right) + \sum\limits_{j=1}^l c_{jn} f_j(x)\,,
\end{multline*}
где
\begin{multline*}
c_{jn} = {}\\
\hspace*{-4pt}{}=\left(g(n)\right)^{-j/2} \!\!\!\int\limits_{1/g(n)}^\infty\!\!\!
y^{-j/2} \,d \left(H(y) + \sum\limits_{i=1}^m n^{-i/2} h_i(y)\right).
\end{multline*}

\section {Формулировка основного результата}

\noindent
\textbf{Теорема~3.1.} \textit{Пусть  статистика
$T_n\hm=T_n(X_1,\ldots$\linebreak $\ldots ,X_n)$ удовлетворяет условию~$1.1,$ а случайный
объем выборки $N_n$~--- условию~$1.2$. Тогда  существует константа
$C_3\hm>0$ такая, что справедливо неравенство:
\begin{multline*}
\sup\limits_x \left| \p\left(\sigma g^\gamma(n)(T_{N_n}  -  \mu) < x\right)
-  G_n(x)\right| \leq {}\\
{}\leq C_1 \e N_n^{-\alpha} + \fr{C_3 + C_2
M_n}{n^\beta}\,,
\end{multline*}
где
\begin{multline*}
M_n = \sup_x \int\limits_{1/g(n)}^\infty
\Bigl|\fr{\partial}{\partial y} \bigl(F(xy^\gamma) +{}\\
{}+ \sum_{j=1}^l
(yg(n))^{-j/2}\ f_j(xy^\gamma)\bigr)\Bigr|\, dy
\end{multline*}
и а.р.\ $G_n(x)$ определено по формуле}~(\ref{e1.2-ben}).

\smallskip

\noindent
\textbf{Следствие 3.1.} \textit{Если моменты $\e (N_n/g(n))^{-\alpha}$
равномерно по $n$ ограничены, т.\,е.\
$$
\e \left(\fr{N_n}{g(n)}\right)^{-\alpha} \leq C_4\,,  \enskip C_4 >
0\,,\ \ n \in \N\,,
$$
то правая часть неравенства в формулировке теоремы~$3.1$ приобретает
вид:}
$
{C_1 C_4}/{g^\alpha(n)} \hm+  (C_3 \hm+ C_2 M_n)/{n^\beta}.
$


\smallskip

\noindent
\textbf{Следствие~3.2.} \textit{В силу неравенства Гёльдера при
$0\hm<\alpha\hm\leq1$ справедлива оценка
$$
\e N_n^{-\alpha} \leq \left(\e N_n^{-1}\right)^\alpha\,,
$$
которая может быть полезной при практическом применении теоремы. 
В~этом случае правая часть неравенства из формулировки теоремы может
быть записана в виде:}
$
C_1 \left(\e N_n^{-1}\right)^\alpha  +  ({C_3 + C_2
M_n})/{n^\beta}.
$


\smallskip

Для вычисления  $\e N_n^{-1}$ можно использовать следующую формулу
(см., например,~\cite{25-ben},  с.~93, задача~40,\,б). Если
неотрицательная целочисленная с.в.~$N$  имеет производящую функцию
$$
\Psi(s) =  \e s^{N}\,,   \enskip  |s| \le 1\,,
$$
то из теоремы Фубини непосредственно следует, что
\begin{equation}
\e N^{-1} = \int\limits_0^1 \fr{\Psi(s)}{s}\, ds\,. \label{e3.1-ben}
\end{equation}
Используя это соотношение, оценку из формулировки теоремы можно
представить  в виде:
\begin{equation}
C_1 \left(\int\limits_0^1 \fr{\Psi_n(s)}{s}\, ds\right)^\alpha  +
\fr{C_3 + C_2 M_n}{n^\beta}\,,\label{e3.2-ben}
\end{equation}
где $\Psi_n(s)$~--- производящая функция с.в.~$N_n$.

Приведем пример использования формулы~(\ref{e3.1-ben}). Пусть с.в.~$N_n$ имеет
геометрическое распределение:
$$
{\p}(N_n = k) =  \fr{1}{n} \left(1 - \fr{1}{n}\right)^{k-1}\,,\enskip
 k\in\N\,.
$$
В этом случае производящая функция имеет вид:
\begin{multline*}
\Psi_n(s) =  \e s^{N_n} =  \fr{s}{n}\left[1 -  s\left(1 -
\fr{1}{n}\right)\right]^{-1}={}\\
{}=\fr{s}{n-s(n-1)}\,, \enskip   |s| \le 1\,,
\end{multline*}
поэтому
\begin{equation}
\e N_n^{-1} = \int\limits_0^1 \fr{\Psi_n(s)}{s}\, ds = \fr{1}{n -
1} \,\log n\,,  \enskip  n > 1\,. \label{e3.3-ben}
\end{equation}
С учетом формулы~(\ref{e3.3-ben}) оценка~(\ref{e3.2-ben}) принимает вид:
$(C_1 \log^{\alpha} n)/(n - 1)^\alpha  +  (C_3 + C_2
M_n)/n^\beta$, $n \hm> 1$.

\smallskip

\textbf{Замечание~3.1.} Заметим, что из условия~1.2, в частности,
вытекает, что с.в.\ $N_n/g(n)$ слабо сходится к с.в.~$V$,  имеющей
ф.р.~$H(x)$. Из определения слабой сходимости с функцией
$x^{-\alpha}$, $ x \hm\ge 1$, в случае, если $N_n \hm\ge g(n)$, $n\hm\in\N$,
следует, что
$$
\e \left(\fr{N_n}{g(n)}\right)^{-\alpha} \longrightarrow   \e
\fr{1}{V^\alpha}\,,  \enskip  n\to\infty\,,
$$
т.\,е.\  моменты  $\e (N_n/g(n))^{-\alpha}$ равномерно ограничены по~$n$  
и справедливо утверждение из следствия~3.1.

\smallskip

Случай, когда $N_n\hm\ge g(n)$, возникает, например, если с.в.\ $N_n$
принимает значения $g(n), 2g(n), \ldots , kg(n)$  с равными
вероятностями    $1/k$ при любом фиксированном $k\hm\in\N$. В~этом
случае с.в.\ $N_n/g(n)$  вообще не зависит от~$n$ и, значит, слабо
сходится к с.в.~$V$,  которая принимает значения $1, 2, \ldots , k$ с
равными вероятностями~$1/k$.

\section{Доказательство теоремы~3.1}

Используя формулы~(\ref{e2.1-ben})--(\ref{e2.3-ben}),  получаем оценку:
\begin{multline}
\sup\limits_{x}\left| \p\left(\sigma g^\gamma(n)(T_{N_n}  -  \mu)  <
x\right) -  G_n(x)\right| \leq {}\\
{}\leq I_{1n} + I_{2n}\,, \label{e4.1-ben}
\end{multline}
где
\begin{multline}
I_{1n} =
 \sup\limits_{x}\left| \, \int\limits_{1/g(n)}^\infty \left(
 \vphantom{\sum\limits_{j=1}^l}
 F(x y^\gamma) +{}\right.\right.\\
\left. {}+
\sum\limits_{j=1}^l \left(yg(n)\right)^{-j/2}f_j(xy^\gamma)\right)  d
\left({\p}\left(\fr{N_n}{g(n)} < y\right) -{}\right.\\
\left.\left.{}- H(y) - \sum\limits_{i=1}^m
n^{-i/2} h_i(y)\right)
\vphantom{\int\limits_{1/g(n)}^\infty}\right|\,; \label{e4.2-ben}
\end{multline}

\vspace*{-12pt}

\noindent
\begin{multline}
I_{2n}  =  \sum\limits_{k=1}^{\infty} \sup\limits_{x}  \left|
\vphantom{\sum\limits_{j=1}^l}
 {\p}\left(\sigma
k^\gamma(T_{k}  -  \mu)  < x\left(\fr{k}{g(n)}\right)^\gamma\right)\right.
-{}
\\
- F\left(x \left(\fr{k}{g(n)}\right)^\gamma\right) - {}\\
\left.{}-\sum\limits_{j=1}^l
k^{-j/2}f_j\left(x\left(\fr{k}{g(n)}\right)^\gamma\right)\right|
{\p}(N_n=k)\,.    \label{e4.3-ben}
\end{multline}
Для оценки величины  $I_{1n}$ используем равенство~(\ref{e4.2-ben}),  
условие~1.2, формулу интегрирования по частям (см., например,~\cite{5-ben},
теорема~2.6.11, с.~222 или~\cite{15-ben}, теорема~18.4, с.~236) и
ограниченность функций $f_j(z)$, $j=1,\ldots ,l$, получим, что существует
константа $C_3\hm>0$ такая, что
\begin{multline*}
I_{1n} \le \fr{C_3}{n^\beta} + \sup\limits_{x}\left|\,
\int\limits_{1/g(n)}^\infty \left({\p}\left(\fr{N_n}{g(n)} <
y\right) -{}\right.\right.\\
\left.{}- H(y) - \sum\limits_{i=1}^m n^{-i/2} h_i(y)\right) 
d \left(\vphantom{\sum\limits_{j=1}^l}
F(x y^\gamma) + {} \right.\\
\left.\left.{}+\sum\limits_{j=1}^l
\left(yg(n)\right)^{-j/2}f_j(xy^\gamma)\right)
\vphantom{\int\limits_{1/g(n)}^\infty} \right| \le{}
\\
{}
\le \fr{C_3}{n^\beta} +   \sup\limits_{x}  \int\limits_{1/g(n)}^\infty
\left| \vphantom{\sum\limits_{j=1}^l}
{\p}\left(\fr{N_n}{g(n)} < y\right) -{}\right.\\
\left.{}- H(y) - \sum\limits_{i=1}^m
n^{-i/2} h_i(y)\right| \times{}
\end{multline*}

\noindent
\begin{multline}
{}\times  \left|\fr{\partial}{\partial y} \left(F(xy^\gamma) +
\sum\limits_{j=1}^l (yg(n))^{-j/2} f_j(xy^\gamma)\right)\right|\, dy \le{}\\
{}\le
\fr{C_3}{n^\beta} + \fr{C_1M_n}{n^\beta}\,.  \label{e4.4-ben}
\end{multline}
Ряд  в определении $I_{2n}$   (см.~(\ref{e4.3-ben})) оценим с помощью условия~1.1 и получим:
\begin{equation}
I_{2n}  \leq C_1  \sum\limits_{k=1}^{\infty} \fr{1}{k^\alpha}\, {\p}(N_n=k)
= C_1 \e N_n^{-\alpha}\,.    \label{e4.5-ben}
\end{equation}
Теперь утверждение теоремы следует из неравенств~(\ref{e4.1-ben}), (\ref{e4.4-ben}) и~(\ref{e4.5-ben}).
Теорема доказана.


\section{Примеры}

Приведем два примера применения теоремы~3.1 с вполне конкретными
предельными функциями распределения статистик, построенных по
выборкам случайного объема. Рассмотрим а.р.\
для ф.р.\ выборочных средних, построенных по выборкам случайного
объема. Аналогичные результаты могут быть получены для статистик,
допускающих а.р.\ типа Эдж\-вор\-та для ф.р.\ при
неслучайном объеме выборки. Например, используя результаты работ~[13--18], 
можно получить а.р.\ для  ф.р.\
ранговых статистик, $L$-ста\-ти\-стик и $U$-ста\-ти\-стик.

Пусть $X_1,X_2,\ldots$~--- независимые одинаково распределенные
случайные величины с ${\sf E}X_1 \hm= \mu$, $0\hm<{\sf D}X_1
\hm=\sigma^{-2}$, $\e |X_1|^{3+2\delta} \hm< \infty$,
$\delta\hm\in(0,1/2)$ и ${\sf E}(X_1 - \mu)^3 \hm= \mu_3$. Для
натурального~$n$ обозначим
\begin{equation}
T_n = \fr{X_1 + \cdots + X_n}{n}\,.\label{e5.1-ben}
\end{equation}
Предположим также, что случайная величина $X_1$ удовлетворяет
условию Крам$\acute{\mbox{е}}$ра ($C$)
$$
\limsup\limits_{|t|\to\infty} |\e \exp\{itX_1\}| < 1\,,
$$
тогда с учетом  теоремы~6.3.2 из~\cite{22-ben} получаем, что
\begin{multline}
\sup\limits_x \left|\vphantom{\fr{\mu_3 \sigma^3}{6 \sqrt n}}
{\p}\left(\sigma \sqrt n(T_n - \mu) < x\right) - \Phi(x)
- {}\right.\\
\left.{}-\fr{\mu_3 \sigma^3}{6 \sqrt n} \left(1 - x^2\right) \varphi(x) \right| \leq
\fr{C_1}{n^{1/2+\delta}}\,,  \\  C_1 > 0\,, \  \delta \in
\left(0,\fr{1}{2}\right)\,,
\   n \in \N\,. \label{e5.2-ben}
\end{multline}
Таким образом, статистика~(\ref{e5.1-ben}) удовлетворяет условию~1.1~с

\noindent
\begin{gather}
\gamma = \fr{1}{2}\,; \ \   \alpha = \fr{1}{2} +\delta\,; \ \ \   l = 1\,;
\label{e5.3-ben}
\\
F(x) = \Phi(x)\,;\ \    f_1(x) = \fr{\mu_3 \sigma^3}{6} (1 -\ x^2)
\varphi(x)\,. \label{e5.4-ben}
\end{gather}
Справедлива следующая лемма.

\smallskip

\noindent
\textbf{Лемма 5.1.} \textit{Пусть $l \hm= 1$,  $0 \hm< g(n) \uparrow \infty$,
$F(x) \hm= \Phi(x)$,
$$
f_1(x) = \fr{\mu_3 \sigma^3}{6} (1 - x^2) \varphi(x)\,.
$$
Тогда для величины $M_n$ $($см.\ теорему~$3.1)$ справедливо
неравенство
$$
M_n \le  2 + \widetilde C|\mu_3|\sigma^3\,,
$$
где}
\begin{multline*}
\widetilde C = \fr{1}{3}\,\sup\limits_{u\ge0}
\left\{\varphi(u)(u^4+2u^2+1)\right\}=\fr{16}{3\sqrt{2\pi
e^3}}\approx {}\\
{}\approx 0{,}474752293191785\ldots
\end{multline*}


\smallskip

\noindent
Д\,о\,к\,а\,з\,а\,т\,е\,л\,ь\,с\,т\,в\,о\,.\ \  С~учетом формул~(\ref{e5.1-ben})--(\ref{e5.4-ben}) 
имеем (см.\ теорему~3.1):
\begin{multline*}
M_n = \sup\limits_x \int\limits_{(g(n))^{-1}}^\infty
\left|
\fr{\partial}{\partial y} \left( \vphantom{\sum\limits_{j=1}^l}
F(xy^\gamma) + {}\right.\right.\\
\left.\left.{}+\sum\limits_{j=1}^l
(yg(n))^{-j/2} f_j(xy^\gamma)\right)\right|\, dy  ={}
\\
{}= \sup\limits_x \int\limits_{(g(n))^{-1}}^\infty
\left|\fr{\partial}{\partial y} \left(
\vphantom{\fr{\mu_3\sigma^3(1-x^2 y)\varphi(x\sqrt y)}{6
\sqrt{yg(n)}}}
\Phi(x\sqrt y) +{}\right.\right.\\
\left.\left.{}+
\fr{\mu_3\sigma^3(1-x^2 y)\varphi(x\sqrt y)}{6
\sqrt{yg(n)}}\right)\right|\, dy \le{}
\\
{}\le \sup\limits_{x\ge0} \int\limits_{x(g(n))^{-1/2}}^\infty
\left|\fr{\partial}{\partial u} \left(
\vphantom{\fr{x\mu_3\sigma^3(1-u^2)\varphi(u)}{6u \sqrt{g(n)}}}
\Phi(u) +{}\right.\right.\\
\left.\left.{}+\fr{x\mu_3\sigma^3(1-u^2)\varphi(u)}{6u \sqrt{g(n)}}\right)\right|
\,du +{}
\\
{}+ \sup\limits_{x<0} \int\limits_{-\infty}^{x(g(n))^{-1/2}}
\left|\fr{\partial}{\partial u} \left(
\vphantom{\fr{x\mu_3\sigma^3(1-u^2)\varphi(u)}{6u \sqrt{g(n)}}}
\Phi(u) +{}\right.\right.\\
\left.\left.{}+
\fr{x\mu_3\sigma^3(1-u^2)\varphi(u)}{6u \sqrt{g(n)}}\right)\right|
\,du ={}
\\
{}= \sup\limits_{x\ge0} \int\limits_{x(g(n))^{-1/2}}^\infty \!\!\!\!\!\!\!\!\!\!\varphi(u)
\left|1 + \fr{x\mu_3\sigma^3}{6
\sqrt{g(n)}}\fr{(u^4-2u^2-1)}{u^2}\right|\, du +{}\hspace*{-5.40286pt}
\end{multline*}

\noindent
\begin{multline}
{}+ \sup\limits_{x<0} \!\!\!\int\limits_{-\infty}^{x(g(n))^{-1/2}} \!\!\!\!\!\!\!\!\!\!\varphi(u)
\left|1 + \fr{x\mu_3\sigma^3}{6
\sqrt{g(n)}}\fr{(u^4-2u^2-1)}{u^2}\right|\, du \le{}\hspace*{-0.40298pt}
\\
\hspace*{-1.8mm}{}\le 2 + \fr{|\mu_3|\sigma^3}{3\sqrt{g(n)}} \sup\limits_{x\ge0} x
\!\!\!\int\limits_{x(g(n))^{-1/2}}^\infty\!\!\!\!\!\!\!\!\!\!\varphi(u)
\fr{u^4+2u^2+1}{u^2} \,du.\!\!\label{e5.5-ben}
\end{multline}
Далее заметим, что справедливо соотношение
$$
\sup\limits_{u\ge0}
\left\{\varphi(u)(u^4+2u^2+1)\right\}=\fr{1}{\sqrt{2\pi}}\,\sup\limits_{u\ge0}e^{-u}(2u+1)^2\,.
$$
Легко видеть, что $\left(e^{-u}(2u+1)^2\right)^\prime\hm=e^{-u}(2u\hm+1)(3\hm-2u)\hm=0$
при $u={3}/{2}$. Таким образом,
\begin{multline*}
\hspace*{-1.36589pt}\fr{1}{\sqrt{2\pi}}\,\sup\limits_{u\ge0}e^{-u}(2u+1)^2
=\fr{1}{\sqrt{2\pi}}\,e^{-u}(2u+1)^2\Big|_{u=3/2}={}\\
{}=\fr{16}{\sqrt{2\pi
e^3}}\approx 1{,}42425687951535\ldots ,
\end{multline*}
так что
\begin{multline}
\widetilde C=\fr{1}{3}\sup\limits_{u\ge0}
\left\{\varphi(u)(u^4+2u^2+1)\right\}=\fr{16}{3\sqrt{2\pi
e^3}}\approx{}\\
{}\approx 0{,}474752293191785\ldots 
\label{e5.6-ben}
\end{multline}
При этом из неравенства~(\ref{e5.5-ben}) следует, что
\begin{multline*}
M_n \le 2 + \widetilde C \fr{|\mu_3|\sigma^3}{\sqrt{g(n)}}
\sup\limits_{x\ge0} x \!\int\limits_{x(g(n))^{-1/2}}^\infty \!\!u^{-2}\, du
={}\\
{}= 2 + \widetilde C|\mu_3|\sigma^3\,.
\end{multline*}
Таким образом, справедливо неравенство:
\begin{equation}
M_n \le  2 + \widetilde C|\mu_3|\sigma^3\,,\label{e5.7-ben}
\end{equation}
где постоянная $\widetilde C$ определена в соотношении~(\ref{e5.6-ben}). Из
неравенства~(\ref{e5.7-ben}) следует утверждение леммы. Лемма доказана.

\subsection{Распределение Стьюдента}

В работе~\cite{3-ben} показано, что если случайный объем выборки $N_n$
имеет отрицательно биномиальное распределение с параметрами $p \hm=
1/n$ и $r \hm> 0$,  т.\,е.\
$$
{\p}(N_n = k) = \fr{(k+r-2)\cdots r}{(k-1)!}\, \fr{1}{n^r} \left(1
- \fr{1}{n}\right)^{k-1}\!\!, \   k\in\N
$$
(при $r=1$ имеем геометрическое распределение), то для
асимптотически нормальной статистики $T_n$ справедливо предельное
соотношение~(\cite{3-ben}, следствие~2.1)
\begin{equation}
{\p}(\sigma\sqrt{n} (T_{N_n} - \mu) < x)\longrightarrow G_{2r}(x
\sqrt r), \ \  n\to\infty,\!\! \label{e5.8-ben}
\end{equation}
где $G_{f}(x)$~--- функция распределения Стьюдента с параметром $f \hm= 2r$, соответствующая 
плотности вида

\noindent
$$
p_{f}(x) = \fr{\Gamma(f+1/2)}{\sqrt{\pi f}\, \Gamma(f/2)}\left(
1+\fr{x^2}{f}\right)^{-(\gamma+1)/2}\,,\enskip   x\in\R\,,
$$
где $\Gamma(\cdot)$~--- эйлерова гам\-ма-функ\-ция, а $f\hm>0$~--- параметр
формы (если параметр $f$ натурален, то он называется числом степеней
свободы). В~рас\-смат\-ри\-ва\-емой ситуации он может быть произвольно мал,
т.\,е.\ может иметь место типичное распределение с тяжелыми
хвостами. Если $f\hm=2$, т.\,е.\ $r\hm=1$, то ф.р.\ $G_2(x)$ выражается в
явном виде:

\noindent
$$
G_2(x) = \fr{1}{2}\left( 1+\fr{x}{\sqrt{2+x^2}} \right)\,, \enskip  x\in\R\,.
$$
При $r=1/2$ имеем распределение Коши.

В книге~\cite{22-ben} (формула~(6.112)) приведена следующая оценка
скорости сходимости:

\noindent
\begin{multline}
\sup_{x\ge0} \left|{\p}\left(\fr{N_n}{\e N_n} <  x\right) -
H_r(x)\right| \leq 
\begin{cases}
\displaystyle\fr{C_r}{n}\,, &  r \ge 1;\\
\displaystyle\fr{C_r}{n^r}\,, &  r \in (0,1),
\end{cases}\\  C_r > 0\,,\ \
  n\in\N\,,\label{e5.9-ben}
\end{multline}
где $H_r(x)$~--- функция гам\-ма-рас\-пре\-де\-ле\-ния с параметром $r \hm> 0$:

\noindent
\begin{equation}
H_r(x) = \fr{r^r}{\Gamma(r)} \int\limits_0^x e^{-ry} y^{r-1} \,dy\,, 
\enskip
 x\ge 0\,,\label{e5.10-ben}
\end{equation}
 При этом
\begin{equation}
\e N_n = r(n - 1) + 1\,. \label{e5.11-ben}
\end{equation}
Таким образом, из соотношений~(\ref{e5.9-ben})--(\ref{e5.11-ben}) следует, что случайный
индекс $N_n$ удовлетворяет условию~1.2 с
\begin{gather*}
g(n) = r(n - 1) + 1\,; \enskip  H(x) = H_r(x)\,; \enskip  m = 1\,; % \label{e5.12-ben}
\\
h_1(x) \equiv 0\,;  \enskip   C_2 = C_r > 0\,; %  \label{e5.13-ben}
\\
\beta = \begin{cases} 1\,, &  r \ge 1\,;\\ 
r\,, &  r \in (0,1)\,.
\end{cases}
%\label{e5.14-ben}
\end{gather*}
Далее, используя равенство

\noindent
\begin{multline*}
(1 + x)^\gamma = \sum\limits_{k=0}^\infty \fr{\gamma(\gamma -
1)\cdots(\gamma - k + 1 )}{k!}\,  x^k\,, \\
    |x| < 1\,,\enskip  \gamma
 \in \R\,,
\end{multline*}
нетрудно получить, что
\columnbreak


\noindent
\begin{multline}
\e N_n^{-1} = \fr{1}{(n - 1) \left(1 - r\right)}\left(\fr{1}{n^{r-1}} -
1\right) ={}\\
{}= {O}(n^{-r})\,, \ \ \   r > 0\,, \ \   r \ne 1\,, \ \  n \in \N\,.
\label{e5.15-ben}
\end{multline}
Для случая $r=1$, используя формулу~(\ref{e3.3-ben}), имеем:
\begin{equation}
\e N_n^{-1} = \fr{1}{n - 1} \log n\,, \enskip  n > 1\,. \label{e5.16-ben}
\end{equation}
Таким образом, учитывая теорему~3.1, следствие~3.2, формулы~(\ref{e5.2-ben})--(\ref{e5.4-ben}),
а также лемму~5.1, соотношения~(\ref{e5.15-ben}), (\ref{e5.16-ben}) и
равенства (справедливые равномерно по~$x$)
\begin{gather}
\int\limits_{(r(n-1)+1)^{-1}}^\infty \!\!\!\Phi(x\sqrt y)\, dH_r(y) =
 \int\limits_{0}^\infty \Phi(x\sqrt y)\, dH_r(y) +{}\notag\\
 \hspace*{15mm}{}+
{O}\left(\fr{1}{n}\right) =
 G_{2r}(x) + O\left(\fr{1}{n}\right)\,;\label{e5.17-ben}
\\
\int\limits_{(r(n-1)+1)^{-1}}^\infty \!\!\!\varphi(x\sqrt y)
\fr{1-x^2y}{\sqrt y}\, dH_r(y) ={}\notag\\
{}=
 \int\limits_{0}^\infty \varphi(x\sqrt y) \fr{1-x^2y}{\sqrt y}\,
dH_r(y) +  o(1) \equiv\notag\\
{}\hspace*{30mm}\equiv g_{r}(x) + {\it o}(1)\,,\label{e5.18-ben}
\end{gather}
получаем следующее утверждение.

\smallskip

\noindent
\textbf{Теорема 5.1.} \textit{Пусть статистика $T_n$ имеет вид $(\ref{e5.1-ben})$,
где  $X_1,X_2,\ldots$~--- независимые одинаково распределенные с.в.\ с
${\sf E}X_1 \hm= \mu$, $0\hm<{\sf D}X_1 \hm=\sigma^{-2}$, $\e
|X_1|^{3+2\delta} \hm< \infty$, $\delta\hm\in(0,1/2)$ и ${\sf E}(X_1 -
\mu)^3 \hm= \mu_3$, причем с.в.\ $X_1$ удовлетворяет условию
Крам$\acute{\mbox{е}}$ра $(C)$. Предположим, что при некотором $r\hm>0$
случайная величина $N_n$ имеет распределение вида:
\begin{multline*}
{\p}(N_n = k) = {}\\
{}=\fr{(k+r-2)\cdots r}{(k-1)!} \,\fr{1}{n^r} \left(1
- \fr{1}{n}\right)^{k-1}\,, \quad   k\in\N.
\end{multline*}
Тогда при $r > 1/(1+2\delta)$ для ф.р.\ нормированной статистики
$T_{N_n}$ при $n\to\infty$ справедливо а.р.\ вида
\begin{multline*}
\hspace*{-7.7pt}\sup\limits_x \left|{\p}\left(\sigma\sqrt {r(n-1)+1} (T_{N_n} - \mu) <
x\right) - G_{2r}(x) -{}\right.\\
\left.{}- \fr{\mu_3\sigma^3g_r(x)}{6\sqrt{r(n-1)+1}}
\right| ={}
\\
{}= \begin{cases} 
{O}\left(\left(\fr{\log
n}{n}\right)^{1/2+\delta}\right)\,, & \quad \hspace*{2pt} r = 1\,;\\[4pt]
{O}\left(\fr{1}{n^{\min(1, r(1/2+\delta))}}\right)\,,
&\quad \hspace*{2pt}  r > 1\,;\\[4pt]
{O}\left(\fr{1}{n^{r(1/2+\delta)}}\right)\,, &
\hspace*{-11mm}\fr{1}{1+2\delta} < r < 1\,,
\end{cases}
\end{multline*}
где функции $G_{2r}(x)$ и $g_r(x)$ определены в соотношениях
$(\ref{e5.17-ben})$ и $(\ref{e5.18-ben})$.}

\subsection{Распределение Лапласа}

Рассмотрим распределение Лапласа с ф.р.\ $\Lambda_\theta(x)$ и
плотностью
$$
\lambda_\theta(x)=\fr{1}{\theta\sqrt 2}\exp\left\{
-\fr{\sqrt{2}|x|}{\theta} \right\}\,, \ \ \   \theta > 0\,,\  x\in\R\,.
$$
В работе~\cite{9-ben} была построена последовательность с.в. $N_n(s)$,
зависящая от параметра $s \in \N$, сле\-ду\-юще\-го вида. Пусть $Y_1, Y_2,
\ldots$~--- независимые одинаково распределенные с.в., имеющие
непрерывную ф.р. Определим с.в.
$$
N(s) = \min\left\{ i\geq1: \max\limits_{1\leq j\leq s} Y_j < \max\limits_{s+1\leq k
\leq s+i} Y_k \right\}\,.
$$
Хорошо известно, что так определенные с.в.\ имеют распределение вида
\begin{equation}
{\p}(N(s) \geq k) = \fr{s}{s+k-1}\,, \enskip   k \geq \ 1\label{e5.19-ben}
\end{equation}
(см., например,~\cite{26-ben, 27-ben}). Пусть теперь  $N^{(1)}(s),
N^{(2)}(s),\ldots$~--- независимые одинаково распределенные с.в.,
имеющие распределение~(\ref{e5.19-ben}). Определим с.в.\
$$
N_n(s) = \max\limits_{1\leq j\leq n} N^{(j)}(s)\,,
$$
тогда, как показано в работе~\cite{9-ben},
\begin{equation}
\lim\limits_{n\to\infty} {\p}\left( \fr{N_n(s)}{n} < x \right) = e^{-s/x}\,,
\enskip  x>0\,,\label{e5.20-ben}
\end{equation}
и для асимптотически нормальной статистики $T_n$ справедливо соотношение:
\begin{multline*}
{\p}\left(\sigma\sqrt{n}(T_{N_n(s)} - \mu) < x\right) \longrightarrow{}\\
{}\longrightarrow
\Lambda_{1/s}(x)\,,  \quad n\to\infty\,,\enskip  x\in\R\,,
\end{multline*}
где $\Lambda_{1/s}(x)$~--- функция распределения Лапласа с параметром
$\theta\hm=1/s$.

В работе~\cite{11-ben} была получена следующая оценка скорости
сходимости в соотношении~(\ref{e5.20-ben}):
\begin{multline}
\sup\limits_{x\ge0} \left|\:{\p}\left(\fr{N_n(s)}{n} <  x\right) -
e^{-s/x} \right| \leq \fr{C_s}{n}\,, \\    C_s > 0\,, \quad
n\in\N\,.\label{e5.21-ben}
\end{multline}
Таким образом, из соотношения~(\ref{e5.21-ben}) следует, что случайный индекс
$N_n(s)$ удовлетворяет усло\-вию~1.2~с
\begin{equation}
g(n) = n\,; \ \  H(x) = e^{-s/x}\,; \ \  m = 1\,; \label{e5.22-ben}
\end{equation}
\begin{equation}
h_1(x) \equiv 0\,; \ \    C_2 = C_s > 0\,;  \ \  \beta = 1\,.\label{e5.23-ben}
\end{equation}
Рассмотрим более подробно величину $ \e N_n^{-1}(s)$. Из определения
с.в.\ $N_n(s)$ и равенства~(\ref{e5.19-ben}) имеем
\begin{multline*}
{\p}(N_n(s) = k) = \left( \fr{k}{s+k}\right)^n - \left( \fr{k -
1}{s+k-1}\right)^n ={}\\
{}=
 sn \int\limits_{k-1}^ k \fr{x^{n-1}}{(s + x)^{n+1}}\, dx\,,
\end{multline*}
поэтому
\begin{multline*}
\e N_n^{-1}(s) = \sum\limits_{k=1}^\infty \fr{1}{k} \,{\p}(N_n(s) = k) ={}\\
{}=
 sn \sum\limits_{k=1}^\infty \fr{1}{k} \int\limits_{k-1}^ k
\fr{x^{n-1}}{(s + x)^{n+1}}\, dx \leq{}
\\
{}\leq sn \sum\limits_{k=1}^\infty \int\limits_{k-1}^k \fr{x^{n-2}}{(s +
x)^{n+1}}\, dx =
 sn \int\limits_{0}^\infty \fr{x^{n-2}}{(s + x)^{n+1}} \,dx\,.\hspace*{-0.76227pt}
\end{multline*}
Для вычисления последнего интеграла используем формулу (см.~\cite{13-ben} формула~856.12, с.~184):
$$
\int\limits_{0}^\infty \fr{x^{s-1}}{(a + bx)^{s+n}} \,dx =
\fr{\Gamma(s)\Gamma(n)}{a^n b^s\Gamma(s+n)}\,,  \ \ \  a, b, s, n>0\,.
$$
Получим
\begin{equation*}
\e N_n^{-1}(s) \leq sn \fr{\Gamma(n-1)\Gamma(2)}{s^2\Gamma(n+1)} =
\fr{1}{s(n - 1)} = {O}(n^{-1})\,. \label{e5.24-ben}
\end{equation*}
Таким образом, учитывая теорему~3.1, следствие~3.2, формулы~(\ref{e5.2-ben})--(\ref{e5.4-ben}), 
а также лемму~5.1, соотношения~(\ref{e5.22-ben}), (\ref{e5.23-ben}) и
равенства (справедливые равномерно по~$x$)
\begin{gather}
\int\limits_{n^{-1}}^\infty \Phi(x\sqrt y)\, de^{-s/y} =
\int\limits_{0}^\infty \Phi(x\sqrt y) \,de^{-s/y} +
{O}\left(\fr{1}{n}\right) ={}\notag\\
\hspace*{30mm}{}= \Lambda_{1/s}(x) +
{O}\left(\fr{1}{n}\right)\,;\label{e5.25-ben}
\\
\hspace*{-20mm}\int\limits_{n^{-1}}^\infty \varphi(x\sqrt y) \fr{1-x^2y}{\sqrt
y}\, de^{-s/y} = {}\notag\\
{}=\int\limits_{0}^\infty \varphi(x\sqrt y)
\fr{1-x^2y}{\sqrt y}\, de^{-s/y} + {\it o}(1) \equiv{}\notag\\
\hspace*{40mm}{}\equiv l_{s}(x) +
{\it o}(1)\,,\label{e5.26-ben}
\end{gather}
непосредственно получаем следующую теорему.

\smallskip

\noindent
\textbf{Теорема 5.2.} \textit{Пусть статистика $T_n$ имеет вид $(\ref{e5.1-ben})$,
где $X_1,X_2,\ldots$~--- независимые одинаково распределенные с.в.\ с
${\sf E}X_1 \hm= \mu$, $0<{\sf D}X_1 \hm=\sigma^{-2}$, $\e
|X_1|^{3+2\delta} \hm< \infty$, $\delta\hm\in(0,1/2)$ и ${\sf E}(X_1 -
\mu)^3 \hm= \mu_3$, причем с.в.\ $X_1$ удовлетворяет условию
Крам$\acute{\mbox{е}}$ра $(C)$. Предположим, что при некотором $s\hm\in\N$ 
с.в.\ $N_n(s)$ имеет распределение вида:
$$
{\p}(N_n(s) = k) = \left( \fr{k}{s+k}\right)^n - \left( \fr{k -
1}{s+k-1}\right)^n\,, \ \ \   k\in\N\,.
$$
Тогда для ф.р.\ нормированной статистики $T_{N_n(s)}$ справедливо а.р.\ вида:
\begin{multline*}
\sup\limits_x \left| \vphantom{\fr{\mu_3\sigma^3l_s(x)}{6\sqrt{n}}}
{\p}\left(\sigma\sqrt {n} (T_{N_n(s)} - \mu) < x\right)
- \Lambda_{1/s}(x) - {}\right.\\
\left.{}-\fr{\mu_3\sigma^3l_s(x)}{6\sqrt{n}} \right| =
{O}\left(\fr{1}{n^{1/2+\delta}}\right)\,, \ \ n \to \infty\,,
\end{multline*}
где функции $\Lambda_{1/s}(x)$ и $l_s(x)$ определены соответственно
в соотношениях $(\ref{e5.25-ben})$ и~$(\ref{e5.26-ben})$.}

{\small\frenchspacing
{%\baselineskip=10.8pt
\addcontentsline{toc}{section}{Литература}
\begin{thebibliography}{99}

\bibitem{2-ben} %1
\Au{Гнеденко Б.\,В.} Об оценке неизвестных параметров
распределения при случайном числе независимых наблюдений~// Тр.
Тбилисского математического института, 1989. Т.~92. С.~146--150.

\bibitem{1-ben} %2
\Au{Гнеденко Б.\,В., Фахим Х.} Об одной теореме переноса~// Докл. АН
СССР, 1969. Т.~187. С.~15--17.

\bibitem{12-ben} %3
\Au{Von Ghossy R., Rappl G.} Some approximation methods for the
distri\-bution of random sums~// Insurance: Mathematics and
Economics, 1983. Vol.~2. P.~251--270.

\bibitem{6-ben}  %4
\Au{Круглов В.\,М., Королев~В.\,Ю.} Предельные теоремы для случайных
сумм.~--- М.: Изд-во Московского ун-та, 1990.

\bibitem{14-ben} %5
\Au{Королев В.\,Ю.} Предельные распределения для случайно
индексированных   последовательностей и их применения: Дисс. \ldots\
докт. физ.-мат. наук.~--- М.: МГУ, 1993.

\bibitem{7-ben} %6
\Au{Gnedenko B.\,V., Korolev V.\,Yu.} Random summation. Limit
theorems and applications.~--- Boca Raton: CRC Press, 1996.

\bibitem{8-ben}  %7
\Au{Bening V.\,E., Korolev V.\,Yu.} Generalized Poisson models and
their applications in insurance and finance.~--- Utrecht: VSP, 2002.

\bibitem{4-ben} %8
\Au{Гнеденко Б.\,В.} Курс теории вероятностей.~--- М.: Наука, 1988.

\bibitem{24-ben} %9
\Au{Королев В.\,Ю.} О~взаимосвязи обобщенного распределения
Стьюдента и дисперсионного гам\-ма-рас\-пре\-де\-ле\-ния при статистическом
анализе выборок случайного объема~// Докл. РАН, 2012. Т.~445.
Вып.~6. С.~622--627.

\bibitem{25-ben} %10
\Au{Климов Г.\,П.} Теория вероятностей и
математическая статистика.~--- М.: Изд-во Московского ун-та,
1983.

\bibitem{5-ben} 
\Au{Ширяев А.\,Н.} Вероятность.~--- М.: Наука, 1989.

\bibitem{15-ben} 
\Au{Billingsley P.} Probability and measure.~--- John Wiley \&
Sons, 1995.

\bibitem{16-ben} 
\Au{Bickel P.\,G.} Edgeworth expansions in nonparametric
statistics~// Ann. Stat., 1974. Vol.~2. P.~1--21.

\bibitem{17-ben} %14
\Au{Albers W.} Asymptotic  expansions and the deficiency
concept in statistics~// Mathematical Centre Tracts 58.~---
Amsterdam: Mathematisch Centrum, 1974.

\bibitem{19-ben} %15
\Au{Albers W., Bickel P.\,G., Van Zwet~W.\,R.} Asymptotic
expansions for the power of distribution free tests in the
one-sample problem~// Ann. Stat., 1976. Vol.~4. P.~108--156.

\bibitem{20-ben} %16
\Au{Bickel P.\,G., Van Zwet~W.\,R.} Asymptotic expansions for the
power of distribution free tests in the two-sample problem~// Ann.
Stat., 1978. Vol.~6. P.~947--1004.

\bibitem{18-ben} %17
\Au{Helmers R.} Edgeworth   expansions for linear combinations
of order statistics~// Mathematical Centre Tracts 105.~--- Amsterdam:
Mathematisch Centrum, 1984.

\bibitem{21-ben} %18
\Au{Bentkus V., Gotze F., Van Zwet~W.\,R.} An Edgeworth
expansions for symmetric statistics~// Ann. Stat., 1997. Vol.~25.
P.~851--896.

\bibitem{22-ben} 
\Au{Бенинг В.\,Е., Королев~В.\,Ю., Соколов~И.\,А., Шоргин~С.\,Я.}
Рандомизированные модели и методы теории надежности информационных и
технических систем.~--- М.: ТОРУС ПРЕСС, 2007.

\bibitem{3-ben} 
\Au{Бенинг В.\,Е., Королев В.\,Ю.} Об использовании распределения
Стьюдента в задачах теории вероятностей и математической статистики~// 
Теория вероятностей и ее применения, 2004. Т.~49. Вып.~3. С.~417--435.

\bibitem{9-ben} 
\Au{Бенинг В.\,Е., Королев В.\,Ю.} Некоторые статистические задачи,
связанные с распределением Лапласа~// Информатика и её применения,
2008. Т.~2. Вып.~2. С.~19--34.

\bibitem{26-ben} 
\Au{Wilks S.\,S.} Recurrence of extreme observations~// J.~Amer. Math. Soc., 
1959. Vol.~1. No.~1. P.~106--112.

\bibitem{27-ben} 
\Au{Невзоров В.\,Б.} Рекорды. Математическая теория.~--- М.: Фазис, 2000.

\bibitem{11-ben} 
\Au{Лямин О.\,О.} О~скорости сходимости распределений некоторых
статистик к распределению Лапласа и Стьюдента~// Вестник Московского
ун-та. Сер.~15: Вычислительная  математика и кибернетика,
2011. Вып.~1. С.~39--47.

\label{end\stat}


\bibitem{13-ben} 
\Au{Двайт Г.\,Б.} 
Таблицы интегралов и другие математические формулы.~--- М.: Наука, 1977.
\end{thebibliography}
}
}

\end{multicols}      %13Abst+avt

%\renewcommand{\r}{\mathbb{R}}

\newcommand{\abs}[1]{\left|#1\right|}
\newcommand{\ex}{C_0}
%\newcommand{\lowex}{C_1}
\newcommand{\exlow}{C_1}
\newcommand{\exlowk}{C_k}
\renewcommand{\le}{\leqslant}
\renewcommand{\ge}{\geqslant}
\renewcommand{\d}{\delta}
\newcommand{\bet}{\beta_{2+\delta}}

\newcommand{\gd}{\gamma(\delta)}
\newcommand{\kd}{\varkappa(\delta)}
\newcommand{\sign}{\mathrm{sign}\,}
\newcommand{\R}{\mathbb R}
\newcommand{\C}{\mathbb C}
\newcommand{\To}{\longrightarrow}

\newcommand{\ud}{\rho(F_n,\Phi)} %uniform distance


\def\stat{sevts}

\def\tit{УТОЧНЕНИЕ НЕРАВЕНСТВА КАЦА--БЕРРИ--ЭССЕЕНА$^*$}

\def\titkol{Уточнение неравенства Каца--Берри--Эссеена}

\def\autkol{М.\,Е.\,Григорьева, И.\,Г.~Шевцова}
\def\aut{М.\,Е.\,Григорьева$^1$, И.\,Г.~Шевцова$^2$}

\titel{\tit}{\aut}{\autkol}{\titkol}

{\renewcommand{\thefootnote}{\fnsymbol{footnote}}\footnotetext[1]
{Работа выполнена при поддержке Российского фонда фундаментальных
исследований (проекты 08-01-00563, 08-01-00567, 08-07-00152 и
09-07-12032-офи-м), а также Министерства образования и науки
(государственные контракты П1181 и П958, грант МК-581.2010.1).}}

\renewcommand{\thefootnote}{\arabic{footnote}}
\footnotetext[1]{Московский
государственный университет имени М.\,В.~Ломоносова, факультет
вычислительной математики и кибернетики, maria-grigoryeva@yandex.ru}
\footnotetext[2]{Московский государственный университет
имени М.\,В.~Ломоносова, факультет вычислительной математики и
кибернетики, ishevtsova@cs.msu.su}

\Abst{Уточнены верхние оценки абсолютной константы в
неравенстве Каца--Берри--Эссеена для сумм независимых одинаково
распределенных случайных величин с конечными абсолютными моментами
порядка $2+\d$, $0<\d<1$. Предложена альтернатива неравенству
Каца--Берри--Эссеена, имеющая более тонкую структуру, и построены
верхние оценки входящих в уточненное неравенство констант.}

\KW{центральная предельная теорема; неравенство
Каца--Берри--Эссеена; дробь Ляпунова}

       \vskip 18pt plus 9pt minus 6pt

      \thispagestyle{headings}

      \begin{multicols}{2}

      \label{st\stat}
  

\section{Введение}

При решении многих прикладных задач приходится учитывать эффекты,
возникающие в результате суммарного воздействия большого числа
случайных факторов, отдельный вклад каждого из которых в сумму
пренебрежимо мал. Чаще всего в таких ситуациях статистические
закономерности поведения суммы в силу центральной предельной теоремы
аппроксимируются нормальным распределением вероятностей. При этом
точность нормальной аппроксимации зависит от наличия у случайных
слагаемых моментов достаточно высокого порядка или, другими
словами, тяжестью (или легкостью) их <<хвостов>>. Известно, что при
некоторых достаточно общих условиях нормальная аппроксимация
адекватна, если случайные слагаемые имеют моменты хотя бы второго
порядка, причем чем выше порядок момента, тем, как правило, выше
точность нормальной аппроксимации. При этом большой интерес
представляет ситуация, когда случайные слагаемые имеют моменты,
порядок которых заключен между двумя и тремя: с одной стороны, для
распределений, имеющих моменты порядка, большего трех, скорость
сходимости в центральной предельной теореме остается в общем случае
такой же, как для распределений с третьими моментами; с другой
стороны, во многих прикладных задачах важно оценивать точность
нормальной аппроксимации, когда центральная предельная теорема все
еще выполняется, но слагаемые имеют распределения со столь тяжелыми
<<хвостами>>, что третьего момента уже не существует. Такие задачи
возникают, например, в страховании, когда речь заходит о
маловероятных, но экстремально больших выплатах по тому или иному
страховому случаю. Другие примеры связаны с практическим применением
моделей типа распределения Парето с <<хвостами>>, убывающими степенным
образом, при анализе трафика в телекоммуникационных сис\-те\-мах. Часто
статистический анализ таких моделей позволяет сделать вывод, что
показатель степени заключен между тремя и четырьмя, т.\,е.\
дисперсия существует, а третий момент отсутствует. Улучшению оценок
точности нормальной аппроксимации именно для таких ситуаций и
посвящена данная работа.

Для $0\le\d\le 1$ обозначим через $\F_{2+\d}$ множество функций
распределения с нулевым средним, единичной дисперсией и конечным
абсолютным моментом $\bet$ порядка ${2+\d}$. При $\d=0$ полагаем
$\beta_2=1$ и $\F_2$~--- класс всех распределений с нулевым средним и
единичной дисперсией. Пусть $X_1,\ldots,X_n$~--- независимые
одинаково распределенные случайные величины с общей функцией
распределения $F\in\F_{2+\d}$, заданные на некотором вероятностном
пространстве $(\Omega,\mathcal{A},\p)$. Обозначим
\begin{align*}
F_n(x)&=F^{*n}(x\sqrt{n})=
\p\left(\fr{X_1+\ldots+X_n}{\sqrt{n}}<x\right)\,;
\\
\Phi(x)&=\fr{1}{\sqrt{2\pi}}\int\limits_{-\infty}^xe^{-t^2/2}\,dt\,,\quad
x\in\R\,.
\end{align*}

Классическое неравенство Каца--Берри--Эс\-се\-ена устанавливает
существование конечной положительной постоянной $\ex=\ex(\d)$,
зависящей только от~$\d$, которая гарантирует справедливость
неравенства

\noindent
\begin{equation}
\!\left.
\begin{array}{rl}
\ud &\equiv\sup_x|F_n(x)-\Phi(x)|\le \ex(\delta)
L_n^{2+\delta}\,;\\[6pt]
 L_n^{2+\delta}&=\fr{\bet}{n^{\delta/2}}
 \end{array}\!
 \right \}\!\!\!\!\!
\label{Bikelis}
\end{equation}
для всех $n\ge1$ и $F\in\F_3$. 

Для  $\d=1$
неравенство~(\ref{Bikelis}) было доказано независимо и одновременно
Э.~Берри~\cite{Berry1941} и К.-Г.~Эссе-\linebreak еном~\cite{Esseen1942}. В~1960-е~гг.\
разными авторами были предприняты успешные попытки обобщить\linebreak
результат Берри--Эссеена. Так, в 1963~г.\ М.~Кац~\cite{Katz1963}
доказал аналог~(\ref{Bikelis}) для независимых одинаково
распределенных случайных величин с $\e X_1^2g(X_1)<\infty$ для
функций $g$ из некоторого класса, включающего $g(x)=|x|^\d$. 
В~1965~г.\ В.\,В.~Пет\-ров~\cite{Petrov1965} обобщил неравенство Каца
на разнораспределенные слагаемые. В~1966~г.\
А.~Бикялис~\cite{Bikelis1966} доказал неравномерную оценку для
разнораспределенных случайных величин, имеющих конечные абсолютные
моменты порядка $2+\d$, $0<\d\le1$, из которой также вытекает
неравенство~(\ref{Bikelis}). Точные формулировки  упомянутых
результатов вместе с их доказательствами можно найти, например, в
монографии В.\,В.~Петрова~\cite{Petrov1972}.

Относительно константы $\ex(1)$ известно, что
$$
\fr{\sqrt{10}+3}{6\sqrt{2\pi}}\le\ex(1)\le 0{,}4784
$$
(нижняя оценка найдена К.-Г.~Эссееном~\cite{Esseen1956}, верхняя~--- В.\,Ю.~Королевым и 
И.\,Г.~Шевцовой~\cite{KorolevShevtsova2010}).

Верхние оценки величины $\ex(\d)$ при некоторых $0<\d<1$ впервые
были получены в 1983~г.\ В.~Тысиаком~\cite{Tysiak1983} 
(см.\ также~\cite{Paditz1996}) и недавно были уточнены в 
работе~\cite{GaponovaKorchaginShevtsova2009}. В 1986~г.\
Г.~Падитц~\cite{Paditz1986} показал, что для всех $F\in\F_2$ и
$n\ge1$ имеет место неравенство
$$
\rho(F_n,\Phi)\le 3{,}51{\e}\left(X_1^2\min\left\{1,
\fr{|X_1|}{\sqrt{n}}\right\}\right)\,,
$$
откуда вытекает равномерная по $\delta\in[0,1]$ оценка
$\ex(\delta)\le 3{,}51$, так как при любом $\delta\in[0,1]$ выражение
в правой части последнего неравенства не превосходит $3{,}51\cdot
L_n^{2+\delta}$.

Несмотря на то, что со времени первой публикации верхних оценок
прошло более 25~лет, нижние оценки для величины $\ex(\d)$ получены
совсем недавно (см.~\cite{Shevtsova2010}). Перечисленные
оценки указаны в табл.~1: во втором
столбце~--- верхние оценки Тысиака~\cite{Tysiak1983}, в третьем~---
верхние оценки из работы~\cite{GaponovaKorchaginShevtsova2009}, в
пятом~--- нижние оценки из~\cite{Shevtsova2010}; в четвертом же
столбце указаны новые оценки, доказанные в данной работе. Для
удобства в первой строке таблицы приведен год соответствующей
публикации.


Из табл.~1 видно, что представленные в данной работе верхние оценки
константы $\ex(\d)$ не очень далеки от неулучшаемых: зазор между
найденной\linebreak\vspace*{-12pt}
\columnbreak

%\bigskip

%\begin{center} %tabl1
\noindent
{{\tablename~1}\ \ \small{История верхних оценок, а также нижние оценки
константы $\ex(\d)$}}
%\end{center}
\vspace*{2pt}

{\small \begin{center}
\tabcolsep=11.8pt
\begin{tabular}{|c|c|c|c|c|}
\hline
Год & 1983  & 2009 & 2010 &   2010  \\
\hline
$\d$ & $\ex \le$ & $\ex\le$ & $\ex\le$ & $\ex\ge$ \\
\hline
0,9 & 0,802  & 0,7671 & 0,5383 &   0,2133  \\
0,8 & 0,812  & 0,7720 & 0,5723 &   0,2245  \\
0,7 & 0,833  & 0,7876 & 0,6026 &   0,2376  \\
0,6 & 0,863  & 0,8126 & 0,6276 &   0,2530  \\
0,5 & 0,902  & 0,8454 & 0,6413 &   0,2715  \\
0,4 & 0,950  & 0,8876 & 0,6342 &   0,2939  \\
0,3 & 1,008  & 0,9407 & 0,6195 &   0,3220  \\
0,2 & 1,076  & 1,0001 & 0,6094 &   0,3585  \\
0,1 & 1,102  & 1,0739 & 0,6028 &   0,4092  \\
\hline
\end{tabular}
\end{center}
}
%\vspace*{6pt}


\bigskip

\begin{center} %tabl2
\noindent
\parbox{56mm}{{\tablename~2}\ \ \small{Двусторонние оценки
константы $\exlow(\d)$}}
%\end{center}
\vspace*{4pt}

{\small 
\tabcolsep=16pt
\begin{tabular}{|c|c|c|c|c|}
\hline \vphantom{$\frac{\displaystyle R}2$}
  $\d$ & $\exlow\le$ & $\exlow\ge$ \\
\hline
0,9 &0,3089 &0,0323 \\
0,8 &0,3187 &0,0356 \\
0,7 &0,3334 &0,0396 \\
0,6 &0,3528 &0,0444 \\
0,5 &0,3775 &0,0503 \\
0,4 &0,4080 &0,0575 \\
0,3 &0,4450 &0,0665 \\
0,2 &0,4901 &0,0780 \\
0,1 &0,5451 &0,0939 \\
  \hline
\end{tabular}
}
\end{center}

\addtocounter{table}{2}

\bigskip

\noindent
 мажорантой и соответствующей минорантой со\-став\-ля\-ет всего
0,2--0,3, а их отношение колеблется в пределах 1,5--2,5 в
зависимости от~$\d$.


Наряду с уточнением константы $\ex(\d)$ в~(\ref{Bikelis}) данная
работа ставит своей целью уточнение и самой структуры
неравенства~(\ref{Bikelis}). А~именно, в качестве альтернативы
предлагается рассмотреть неравенство
\begin{equation}
\label{K-B-E-sharpened}
\ud\le \exlow(\d)\fr{\bet+1}{n^{\d/2}}\,,\enskip n\ge1\,,\
F\in\F_{2+\d}\,,
\end{equation}
справедливость которого с некоторым положительным и конечным
$\exlow(\d)$ вытекает тривиальным образом из~(\ref{Bikelis})
(например, с $\exlow(\d)=2\ex(\d)$ в силу условия $\bet\ge1$).
Однако константа $\exlow(\d)$ в~(\ref{K-B-E-sharpened})\linebreak
 оказывается
гораздо меньше, чем $\ex(\d)$ в~(\ref{Bikelis}), поэтому
неравенство~(\ref{K-B-E-sharpened}) при достаточно больших~$\bet$
дает оценку, заведомо лучшую, чем~(\ref{Bikelis}), несмотря на то
что для его справедливости необходима та же априорная информация о
распределении~$F$ (а именно, только значение абсолютного момента~$\bet$). 
Кроме того, более оптимистичными оказываются и нижние
оценки~$\exlow(\d)$, построенные в работе~\cite{Shevtsova2010}.
Например, для $\d=1$ двусторонняя оценка имеет вид
$0{,}2659\le\exlow(1)\le 0{,}3041$~[13--16]. Для $0<\d<1$ верхние
оценки константы~$\exlowk(\d)$, устанавливаемые в данной статье, и
нижние, полученные в
работе~\cite{Shevtsova2010}, приведены в
табл.~2 во втором и третьем
столбцах соответст\-венно.


Рассуждения, приводящие к форме неравенства~(\ref{K-B-E-sharpened}),
основаны на используемых оценках для характеристических функций и
подробно описаны в~\cite{KorolevShevtsova2010, KorolevShevtsova2009}.

\section{Главный результат и~основные идеи его доказательства}

\noindent
\textbf{Теорема 1.}
\textit{Для константы $\exlowk(\d)$ в неравенстве}
$$
\ud\le
\exlowk(\d)\fr{\bet+k}{n^{\d/2}}\,,\quad n\ge1\,,\ F\in\F_{2+\d}\,,
$$
\textit{при $k=0$ и $k=1$ имеют место оценки, приведенные в
табл.}~3.

\medskip

\begin{center} %tabl1
\noindent
\parbox{56mm}{{\tablename~3}\ \ \small{Верхние оценки
констант $\ex(\d)$ и $\exlow(\d)$}}
%\end{center}
\vspace*{4pt}

{\small 
\tabcolsep=16pt
\begin{tabular}{|c|c|c|}
  \hline $\d$ & $\ex\le$ & $\exlow\le$ \\
\hline
0,9 & 0,5383 &  0,3089  \\
0,8 & 0,5723 &  0,3187  \\
0,7 & 0,6026 &  0,3334  \\
0,6 & 0,6276 &  0,3528  \\
0,5 & 0,6413 &  0,3775  \\
0,4 & 0,6342 &  0,4080  \\
0,3 & 0,6195 &  0,4450  \\
0,2 & 0,6094 &  0,4901  \\
0,1 & 0,6028 &  0,5451  \\
  \hline
\end{tabular}
}
\end{center}

\addtocounter{table}{1}

\bigskip


%\medskip


При доказательстве теоремы~1 будем придерживаться
подхода, предложенного и развитого В.\,М.~Золотарёвым в его
работах~[17--19]. Этот
подход основан на применении неравенств сглаживания, которые
позволяют оценить расстояние между функциями распределения через
расстояние между соответствующими характеристическими функциями. 
В~рамках этого подхода ключевыми моментами являются: ($i$)~выбор
надлежащего неравенства сглаживания; ($ii$) выбор в нем сглаживающего
ядра; ($iii$)~конструирование оценок для характеристических функций и
($i\nu$) выбор вычислительной оптимизационной процедуры. Опишем эти
моменты в той последовательности, в которой они появляются при
доказательстве неравенств~(\ref{Bikelis}) и~(\ref{K-B-E-sharpened}).
Соответствующие утверждения сформулируем в виде лемм.
{\looseness=1

}

\medskip

Обозначим $f(t)$ и $f_n(t)$ характеристические функции случайных
величин~$X_1$ и стандартизованной суммы $(X_1+\ldots+X_n)/\sqrt{n}$
соответственно:

\noindent
\begin{align*}
f(t)&=\int\limits_{-\infty}^\infty e^{itx}\,dF(x)\,;\\
f_n(t)&=\int\limits_{-\infty}^{\infty}e^{itx}\,dF_n(x)=
\left(f\left(\fr{t}{\sqrt n}\right)\right)^n\,,\quad t\in\R\,.
\end{align*}
Пусть
$$
r_n(t)=\left|f_n(t)-e^{-t^2/2}\right|\,.
$$

\medskip

\noindent
\textbf{Лемма 1} (см.~\cite{Prawitz1972}). %\label{LemPrawitzSmoothIneq}
\textit{Для произвольной функции распределения~$F$ при всех $n\ge1$,
$0<t_0\le1$ и $U>0$ имеет место неравенство}
\begin{multline*}
\ud\le 2\int\limits_0^{t_0}\left|K(t)\right|r_n(Ut)\,dt+{}\\
{}+
2\int\limits_{t_0}^{1}|K(t)\left|\cdot|f_n(Ut)\right|\,dt+{}\\
{}+
2\int\limits_0^{t_0}\left|K(t)-\fr i{2\pi t}\right|e^{-U^2t^2/2}\,dt +
\fr{1}{\pi}\int\limits_{t_0}^\infty e^{-U^2t^2/2}\,\fr{dt}t\,,\hspace*{-0.98pt}
\end{multline*}
\textit{где}
\begin{multline*}
K(t)=\fr{1}{2}\left(1-|t|\right)+\fr {i}{2}\left[(1-|t|)\cot\pi
t+\frac{\sign t}\pi\right]\,,\\
-1\le t\le1\,. %\eqno(8)
\end{multline*}

\smallskip

Перейдем теперь к оцениванию характеристических функций,
фигурирующих в лемме~1. Пусть
$\theta_0(\d)$~--- единственный корень уравнения
$$
\fr{\d\theta^2}2+ \theta\sin \theta + (2+\d)(\cos \theta - 1)=0\,,
\quad \pi\le\theta\le2\pi\,;
$$

\vspace*{-12pt}

\noindent
\begin{multline*}
\kd \equiv \sup_{x>0}\fr{\left|\cos
x-1+x^2/2\right|}{x^{2+\d}}={}\\
{}=\fr{\cos
x-1+x^2/2}{x^{2+\d}}\Bigg|_{x=\theta_0(\d)}\,;
\end{multline*}

\vspace*{-12pt}

\noindent
\begin{multline*}
\!\!\gd= \sup_{x>0}\sqrt{\left(\fr{\cos x-1+x^2/2}{x^{2+\d}}\right)^2\! +\!
\left(\fr{\sin x-x}{x^{2+\d}}\right)^2 }\,,\\ 0<\d\le 1\,.
\end{multline*}
Для $\eps>0$ положим
$$
\psi_\d(t,\eps)= 
\begin{cases}
  t^2/2-\kd\eps|t|^{2+\d}\,, & |t|<\theta_0(\d)\eps^{-1/\d}\,;\\[6pt]
  \fr{1-\cos\big(\eps^{1/\d} t\big)}{\eps^{2/\d}}\,,
      &\!\!\!\!\! \theta_0(\d)\le\eps^{1/\d}|t|\le 2\pi\,;\\[6pt]
  0\,, & |t|>2\pi\eps^{-1/\d}\,.
\end{cases}
$$
Несложно убедиться, что функция~$\psi_\delta(t,\eps)$ монотонно убывает 
по~$\eps$ при каждом фиксированном $t\in\R$.

Ляпуновская дробь будет обозначаться $\ell=$\linebreak $=\bet n^{-\d/2}$.
Дополнительно обозначим
$$
\ell_n=\ell+n^{-\d/2}\,.
$$

\smallskip

\noindent
\textbf{Лемма 2}.
\textit{При всех $F\in\F_{2+\d}$, $0<\d\le1$, $n\ge1$ и $t\in\R$ (если не
оговорено иное) справедливы оценки}
$$
|f_n(t)|\le  \left[1-\fr{2}{n}\psi_\d(t,\ell_n)\right]^{n/2} \equiv
f_1(t,\ell_n,n)\,;
$$
$$
|f_n(t)|\le \exp\{-\psi_\d(t,\ell_n)\}\equiv  f_2(t,\ell_n)\,;
$$
$$
|f_n(t)|\le \exp\left\{-\fr{t^2}{2}+\kd\ell_n|t|^{2+\d} \right\}\equiv
f_3(t,\ell_n)\,;
$$

\vspace*{-6pt}

\noindent
\begin{multline*}
\!r_n(t)\le e^{-t^2/2}\left[\exp\left\{\! \gd\ell|t|^{2+\d}-
n\ln\left(\!1-\fr{t^2}{2n}\!\right)-{}\right.\right.\hspace*{-0.67pt}\\
{}-\left.\left. \fr{t^2}2\right\}-1\right]\equiv
r_1(t,\ell,n), \enskip |t|<\sqrt{2n};
\end{multline*}

\vspace*{-9pt}

\noindent
\begin{multline*}
r_n(t)\le  \left(\gd{\ell|t|^{2+\d}}+\fr{|t|^4}{8n}\right)
\times{}\\
{}\times 
\left(\max\left\{f_1(t,\ell_n,n),\,e^{-t^2/2}\right\}\right)^{(n-1)/n}
\equiv r_2(t,\ell,n)\,;
\end{multline*}

\vspace*{-9pt}

\noindent
\begin{multline*}
r_n(t)\le  \fr{1}{2}\left(\gd{\ell|t|^{2+\d}}+\fr{|t|^4}{8n}\right)
\left(
e^{-t^2/2}+{}\right.\\
{}+\left.\max\left\{f_1(t,\ell_n,n),\,e^{-t^2/2}\right\}\right)
e^{t^2/(2n)}\equiv r_3(t,\ell,n)\,;
\end{multline*}

\vspace*{-3pt}

\noindent
$$
r_n(t)\le f_1(t,\ell_n,n)+e^{-t^2/2}\equiv r_4(t,\ell,n)\,.
$$

\medskip

\noindent
\textbf{Замечание 1.}
Очевидно, $f_1(t,\eps,n)\le f_2(t,\eps)$ при всех $n\ge1$, $\eps>0$
и $t\in\R$. Более того, из результата работы~\cite{Shevtsova2009}
вытекает, что $f_2(t,\eps)\le f_3(t,\eps)$ для всех $\eps>0$ и
$t\in\R$, так что самую точную оценку для~$|f_n(t)|$ дает 
$f_1(t,\ell_n,n)$, тогда как функции $f_j(t,\ell_n)$, $j=2,3$,
обладают полезным свойством монотонности по~$\ell_n$, играющим
важную роль в оптимизационной процедуре.

\medskip

\noindent
Д\,о\,к\,а\,з\,а\,т\,е\,л\,ь\,с\,т\,в\,о\,.\
Первые три оценки ($f_j$, $j=$\linebreak $=1,2,3$) являются тривиальными
следствиями оценок
$$
|f(t)|^2\le 1-2\psi_\d(t,\bet+1)\,,\quad t\in\R\,;
$$
$$
|f(t)|^2\le 1-t^2 + 2\kd(\bet+1)|t|^{2+\d}\,,\quad t\in\R\,,
$$
полученных Шевцовой в~\cite{Shevtsova2009}. Четвертая оценка ($r_1$)
впервые объявлена В.\,М.~Зо\-ло\-та\-рё\-вым~\cite{Zolotarev1966}
для $\d=1$ (без доказательства), ниже будет приведено полное
доказательство для всех $0<\d\le1$. Пятая ($r_2$) и шестая ($r_3$)
оценки являются несложной комбинацией методов и результатов работ
Правитца~\cite{Prawitz1975}, Шевцовой~\cite{Shevtsova2009},
Гапоновой и Шевцовой~\cite{GaponovaShevtsova2009}. Последняя 
оценка~($r_4$) тривиальна.

Докажем неравенство $r_n(t)\le r_1(t,\ell,n)$, $|t|<$\linebreak $<\sqrt{2n}$. Из
неравенств $|e^{ix}-1-ix|\le|x|^2/2$ и $|e^{ix}-1-ix-(ix)^2/2|\le
\gd|x|^{2+\d}$, $x\in\R$, с учетом моментных условий для
распределения $F\in\F_{2+\d}$ вытекают соотношения
\begin{gather*}
|f(t)-1|\le\fr{t^2}{2}\,,\quad|t|\le\sqrt2\,,\\
f(t)=1-\fr{t^2}2 + \theta_1 \gd \bet|t|^{2+\d}\,,\quad t\in\R\,,
%\label{ch_f_expansion}
\end{gather*}
с некоторым $\theta_1\in\C$, $|\theta_1|\le1$. Следовательно, для
всех $|t|<\sqrt{2n}$ определен логарифм~$\ln f(t)$ (условимся всегда
выбирать главную ветвь логарифма) и
\begin{multline*}
\abs{\ln f(t)+\fr{t^2}2}=\abs{\ln[1-(1-f(t))]+\fr{t^2}2}={}\\
{}=
\left|-\sum_{k=1}^\infty\fr{(1-f(t))^k}k+\fr{t^2}2\right|\le{}\\
{}\le
\sum_{k=2}^\infty\fr{1}{k}\left(\fr{t^2}2\right)^k+
\abs{f(t)-1+\fr{t^2}2}\le{}\\
{}\le  -\left[\ln\left(1-\fr{t^2}{2}\right)
+\fr{t^2}2\right]+\gd\bet|t|^{2+\d}\,,\\ |t|<\sqrt{2n}\,,
\end{multline*}
откуда с учетом неравенства $\abs{e^z-1}\le e^{|z|}-1$, $z\in\C$,
получаем
\begin{multline*}
r_n(t)=\abs{f_n(t)-e^{-t^2/2}}={}\\
{}=e^{-t^2/2} \left|\exp\left\{n\ln
f\left(\fr{t}{\sqrt n}\right)+\fr{t^2}2\right\}- 1\right|\le{}
\\
{}\le e^{-t^2/2} \left(\exp\left\{n\left|\ln f\left(\fr{t}{\sqrt
n}\right)+\fr{t^2}{2n}\right|\right\}-1\right)\le{}
\\
{}\le e^{-t^2/2} \left(\exp\left\{\gd \fr{\bet|t|^{2+\d}}{n^{\d/2}}
-n\ln \left(1-\fr{t^2}{2n}\right)-{}\right.\right.\\
{}-\left.\left.\fr{t^2}{2}\right\}-1\right)\equiv
r_1(t,\ell,n)\,,
\end{multline*}
что и требовалось доказать.

\medskip

Следующая лемма позволяет ограничить сверху множество
рассматриваемых значений~$n$ при оценивании констант~$\exlowk(\d)$ в
неравенстве~(\ref{K-B-E-sharpened}) с $0\le k\le 1$.

\medskip

\noindent
\textbf{Лемма 3.} %\begin{lemma}\label{LemMonotone}
\textit{Для любых положительных $k\le1$, $T$, $\eps$,
\begin{multline*}
N_1 \ge N_1(T) \equiv{}\\
{}\equiv T^2 \left( \fr{1}{8k\d\gd} + \sqrt{1 + \left(
\fr{1}{8k\d\gd}\right)^2}\ \right)^2\,;
\end{multline*}
$$
N_3 \ge N_3(T,\eps) \equiv \left( \fr{ T^{2-\delta}}{4\d\gd} +
\fr{\eps T^2}{\delta}\right)^{2/(2-\delta)}
$$
и таких, что $N_j\ge((1+k)/\eps)^{2/\d}$, при всех $|t| \le T$
справедливы оценки}
\begin{multline*}
\sup\limits_{n\ge N_1}r_1 \left(t, \eps - kn^{-\d/2}, n \right) \le{}\\
{}\le e^{-t^2/2}
\left( \exp \left\{ \gd \eps |t|^{2+\d} \right\} - 1
\right)\equiv\widetilde r_1(t, \eps)\,;
\end{multline*}

\vspace*{-9pt}

\noindent
\begin{multline*}
\sup\limits_{n\ge N_3}r_3\left(t,\eps-n^{-\delta/2},n\right)\le{}\\
{}\le \fr{\gd \eps
|t|^{2+\delta}}{2} \left( e^{\kd\eps |t|^{2+\delta}} + 1 \right)
e^{-t^2/2}\equiv \widetilde r_3(t,\eps)\,.
\end{multline*}

\medskip

\noindent
Д\,о\,к\,а\,з\,а\,т\,е\,л\,ь\,с\,т\,в\,о\,.\
Для доказательства первой оценки запишем~$r_1$ в виде
\begin{multline*}
r_1 \left(t, \eps - kn^{-\d/2}, n \right) = {}\\
{}=e^{-t^2/2} \left( \exp
\left\{ \gd \eps |t|^{2+\d} + g(n, |t|) \right\} - 1 \right)\,,
\end{multline*}
где
$$
g(n, |t|) = -\fr{k\gd}{n^{\d/2}} |t|^{2+\d} - n \ln \left(1 -
\fr{t^2}{2n} \right) - \fr{t^2}{2}\,.
$$
Тогда достаточно показать, что $g(x, t) \le 0$ для всех $0 \le t \le
T$ и $x \ge N_1(T,\eps)$.

Используя разложение логарифма в степенной ряд, для всех $0 \le t
\le \sqrt{2x}$ и $x > 0$ получаем
\begin{multline*}
g(x, t) = -\fr{k\gd}{x^{\d/2}}|t|^{2+\d} + x
\sum\limits_{j=2}^{\infty} \fr{1}{j} \left(\fr{t^2}{2x} \right)^j
\le{}\\
{}\le
 -\fr{k\gd}{x^{\d/2}}|t|^{2+\d} + \fr{x}{2}
\sum\limits_{j=2}^{\infty} \left(\fr{t^2}{2x} \right)^j ={}
\\
{}
= -\fr{k\gd}{x^{\d/2}}\,|t|^{2+\d} + \fr{t^4}{4(2x - t^2)} \equiv
\widetilde{g}(x, t)\,.
\end{multline*}
Заметим, что для $0 \le t \le \sqrt{2x}$ и $x > 0$
\begin{multline*}
\fr{\partial \widetilde{g}(x, t)}{\partial x} = \fr{k\d\gd
t^{2+\d}}{2 x^{1+\d/2}} - \fr{t^4}{2(2x - t^2)^2} > 0
\Longleftrightarrow{} \\
{}\Longleftrightarrow h(x, t) \equiv k\d\gd(2x - t^2)^2 -
t^{2-\d} x^{1+\d/2} > 0\,.
\end{multline*}

Пусть $T$~--- произвольное число из интервала $(0, \sqrt{2x})$.
Несложно видеть, что функция~$h(x, t)$ монотонно убывает по~$t$,
поэтому для всех $0 \le t \le T$
\begin{multline*}
h(x, t) \ge h(x, T) = x \left( 4k\d\gd x - 4k\d\gd T^2 -{}\right.\\
{}-\left.
x^{\d/2}T^{2-\d}\right) + k\d\gd T^4 > \\
> x \left( 4k\d\gd x - 4k\d\gd T^2 - x^{\d/2}T^{2-\d}\right)\,.
\end{multline*}
Для неотрицательности последнего выражения достаточно, чтобы
$$
H(x)\equiv 4k\d\gd (x - T^2) - x^{\d/2}T^{2-\d}\ge0\,.
$$
Очевидно, для всех достаточно больших~$x$ функция~$H(x)$ монотонно
возрастает и не ограничена при $x\to\infty$. Следовательно, найдется
такая точка $x_0>0$, что $H(x)\ge0$ для всех $x\ge x_0$. Будем
искать эту точку в виде $x_0=zT^2$. Не ограничивая общности, можно
считать, что $z>1$, поскольку $H(T^2)=-T^2<0$. Имеем
\begin{multline*}
H(zT^2) = 4k\d\gd T^2(z-1) - T^2z^{\d/2} > 0 
\Longleftrightarrow{}\\
{}\Longleftrightarrow{} 4k\d\gd(z-1) - z^{\d/2} > 0\,.
\end{multline*}
Поскольку $z^{\d/2} \le \sqrt{z}$ при $z>1$, для справедли\-вости
последнего условия достаточно, чтобы
$$
4k\d\gd z - \sqrt{z} - 4k\d\gd > 0\,,
$$
откуда, решив квадратное уравнение, получаем
$$
\sqrt{z} >\fr{1 + \sqrt{1 + 64(\d\gd k)^2}}{8\d\gd k}\equiv z_0\,.
$$
Следовательно, $x_0=z_0^2T^2$ и $H(x) > 0$ при
$$
x\ge  T^2 \left( \fr{1}{8\d\gd k} + \sqrt{1 + \left( \fr{1}{8\d\gd
k}\right)^2} \right)^2\,.
$$

Таким образом, для всех $0 \le t \le T$ и $x \ge N_1(T,\eps)$ имеем
$h(x, t)>xH(x)>0$, а значит~$\widetilde{g}(x, t)$ монотонно
возрастает по $x \ge N_1(T,\eps)$ при каждом фиксированном $0<t\le
T$ и для всех $N_1 \ge N_1(T,\eps)$
\begin{multline*}
\sup\limits_{0 \le t \le T} \sup_{x \ge N_1} g(x, t) \le \sup\limits_{0 \le t \le
T} \sup\limits_{x \ge N_1} \widetilde{g}(x, t) ={}\\
{}= \sup\limits_{0 \le t \le T}
\lim_{x\rightarrow\infty} \widetilde{g}(x, t) = 0\,,
\end{multline*}
что и требовалось доказать.

Далее заметим, что в силу неравенства $f_1(t,\ell_n,n)\le
f_3(t,\ell_n)$, $t\in\R$, величину~$r_3$ можно оценить следующим
образом:
\begin{multline*}
r_3\left(t,\eps-n^{-\delta/2},n\right)\le{}\\
{}\le \fr{1}{2} \left( \gd
\eps|t|^{2+\delta} - \fr{\gd |t|^{2+\delta}}{n^{\delta/2}} +
\fr{t^4}{8n}\right) \left(
\vphantom{\fr{t^2}{2}} e^{-{t^2}/{2}} +{}\right.\\
{}\left. \exp \left\{-\fr{t^2}{2} +
\varkappa(\delta) \eps |t|^{2+\delta} - \fr{\varkappa(\delta)
|t|^{2+\delta}}{n^{\delta/2}}\right\} \right) e^{t^2/(2n)} \le{}
\\
{}\le \fr{1}{2} \left( \gd \eps|t|^{2+\delta} - \fr{\gd
|t|^{2+\delta}}{n^{\delta/2}} + \fr{t^4}{8n}\right)\times{}\\
{}\times \left( 1 +
e^{\varkappa(\delta) \eps |t|^{2+\delta}}\right)e^{t^2/(2n)-t^2/2} ={}
\\
{}= \fr{|t|^{2+\delta}}{2} \left( \gd \eps - \fr{\gd}{n^{\delta/2}}
+ \fr{|t|^{2 - \delta}}{8n}\right)\times{}\\
{}\times \left( 1 + e^{\varkappa(\delta)
\eps |t|^{2+\delta}}\right) e^{t^2/(2n)-t^2/2} \equiv{}
\\
{}\equiv \fr{|t|^{2+\delta}}{2} \left( 1 + e^{\varkappa(\delta) \eps
|t|^{2+\delta}}\right) \exp \left( -\fr{t^2}{2} + g(n, |t|)
\right)\,,
\end{multline*}
где
\begin{multline*}
g(x, t) = \fr{t^2}{2x} + \ln \left( \gd\eps  -
\fr{\gd}{x^{{\delta}/{2}}} + \fr{t^{2 - \delta}}{8x}
\right)\,,\\
 x>\left(\fr{2}{\eps}\right)^{2/\delta}\,,\quad  \eps,\ t > 0\,.
\end{multline*}
Заметим, что выражение под знаком логарифма положительно.

Покажем, что при всех фиксированных положительных~$\eps$ и $t\le T$
функция~$g(x, t)$ монотонно возрастает по~$x$ при $x \ge
N_3(T,\eps)$. Вычислим производную
\begin{multline*}
\!\!\!\!g_x' (x, t) = -\fr{t^2}{2x^2} + \frac{(\delta/2) \gd
x^{-1-\d/2} - t^{2-\delta}/(8x^2)}{\gd\eps - {\gd}{x^{-\d/2}}
+ {t^{2-\delta}}/(8x)}=
\\
{}= \left (-8\gd\eps x t^2 + 8\gd x^{1-\d/2}t^2 - t^{4-\delta} +{}\right.\\
\left.{}+ 8\d\gd
x^{2-\d/2} - 2x t^{2-\delta}\right) \Big/
\left (
\vphantom{8\gd x^{1-\d/2} +
t^{2-\delta}}
2x^2 \left(
\vphantom{8\gd x^{1-\d/2} +
t^{2-\delta}}
8\gd\eps x -{}\right.\right.\\
\left.\left.{}- 8\gd x^{1-\d/2} +
t^{2-\delta}\right)\right)\,.
\end{multline*}
Знаменатель в последнем выражении совпадает с точностью до множителя
$2x^3$ с выражением под знаком логарифма в определении $g(x,t)$, а
следовательно, он положителен. В таком случае условие $g_x'(x,t)>0$
равносильно неравенству
\begin{multline*}
x\left(8\d\gd x^{1-\d/2} - \left(2t^{2-\delta} + 8\gd\eps t^2\right)\right) +{}\\
{}+ 8\gd
x^{1-\d/2} t^2 - t^{4-\delta} \ge 0\,,
\end{multline*}
для чего достаточно, чтобы

\noindent
\begin{multline*}
x^{1-\d/2} \ge \max\left\{\fr{t^{2-\delta}}{4\d\gd} + \fr{\eps
t^2}{\delta},\, \frac{t^{2-\delta}}{8\gd}\right\}={}\\
{}=
\fr{t^{2-\delta}}{4\d\gd} + \fr{\eps t^2}{\delta}\,.
\end{multline*}

Таким образом, при всех $0\le t\le T$ и
\begin{multline*}
x \ge  N_3(T,\eps)={}\\
{}= \max \left\{
\left(\fr{2}{\eps}\right)^{2/\delta}, \left( \fr{
T^{2-\delta}}{4\d\gd} + \fr{\eps
T^2}{\delta}\right)^{2/(2-\delta)} \right\}
\end{multline*}
функция $g(x,t)$ монотонно возрастает по~$x$, а следовательно, при
всех $N_3\ge N_3(T,\eps)$
$$
\sup\limits_{0\le t\le T}\sup\limits_{n \ge N_3} g(n,t) = \sup\limits_{0\le t\le
T}\lim_{x\rightarrow\infty} g(x,t) = \ln(\gd\eps)\,,
$$
что и требовалось доказать. Лемма доказана.

\medskip

Наконец, правильно организовать процесс вычислительной оптимизации
позволяют следующие утверждения.

\noindent
\textbf{Лемма 4} (см.~[24]). %\begin{lemma} [см. \cite{BhatRangaRao1982}] \label{LemBhRRao}
\textit{Для любого распределения~$F$ с нулевым средним и единичной
дисперсией справедливо неравенство}
\begin{multline*}
\sup\limits_{x}|F(x)-\Phi(x)|\le
\sup\limits_{x>0}\left(\Phi(x)-\fr{x^2}{1+x^2}\right)={}\\
{}= 0{,}54093654\ldots
\end{multline*}

\medskip

\noindent
\textbf{Лемма 5.} (см.~[23]). %\begin{lemma}[см. \cite{GaponovaShevtsova2009}]
%\label{LemEps_le_0.3}
\textit{Для любой функции распределения $F\in\F_{2+\d}$ и всех $n\ge2$
таких, что $(\bet+1)/n^{\d/2}\le0{,}6$, справедливо неравенство
$$
\rho(F_n,\Phi) \le
C'(\d)\fr{\bet}{n^{\d/2}}+\fr{C''(\d)}{n^{\d/2}}
$$
с $C'(\d)$ и $C''(\d)$, указанными в
табл.}~4.

\smallskip

\begin{center} %tabl4
\noindent
\parbox{56mm}{{\tablename~4}\ \ \small{Значения $C'(\d)$ и
$C''(\d)$ из леммы~5 при некоторых $\d$}}
%\end{center}
\vspace*{2pt}

{\small 
\tabcolsep=16.1pt
\begin{tabular}{|c|c|c|}
\hline
$\d$ & $C'(\d)$ & $C''(\d)$ \\
\hline
0,9 & 0,3085 & 0,2399  \\
0,8 & 0,2987 & 0,2166  \\
0,7 & 0,2912 & 0,1921  \\
0,6 & 0,2852 & 0,1655  \\
0,5 & 0,2800 & 0,1382  \\
0,4 & 0,2765 & 0,1044  \\
0,3 & 0,2776 & 0,0714  \\
0,2 & 0,2915 & 0,0327  \\
0,1 & 0,1500 & 0,0021  \\
  \hline
\end{tabular}
}
\end{center}

\addtocounter{table}{1}

%\bigskip

Из леммы~5 вытекает, что при всех~$n$ и~$\bet$
таких, что ${(\bet+k)/n^{\d/2}<0.3(1+k)}$,
неравенство~(\ref{K-B-E-sharpened}) имеет место при любых
$k\in[0,\,1]$ и $\exlowk(\d)>0$, удовлетворяющих условию
$(k+1)\exlowk(\d) \ge C'(\d) + C''(\d)$.

Подставляя оценки для характеристических функций из леммы~2 в правую часть 
неравенства сглаживания Правитца из леммы~1, получаем некоторую функцию 
$D(\ell,n,t_0,U)$, мажорирующую равномерное расстояние 
$\rho(F_n,\Phi)$ при всех $U>0$, $t_0\in(0,1]$, $n\geqslant1$ и~$F$ 
с фиксированной ляпуновской дробью $\beta_{2+\delta}n^{-\delta/2}=\ell$.
Приведенные леммы дают основание ограничить область рассматриваемых
значений величины $\eps=(\bet+k)/n^{\d/2}$ некоторым конечным
отрезком, отделенным от нуля (подробнее об этом будет сказано ниже),
и искать константу~$C_k$ при каждом $k\in[0,\,1]$ в виде
\begin{equation}
\left.
\begin{array}{rl}
\exlowk(\d)&=\max\limits_{\eps}C_\d(\eps)\,;\\[6pt]
C_\d(\eps)&=\fr{D_\d(\eps)}{\eps}\,;\\[6pt]
D_\d(\eps)&=\sup\left\{D_\d(\eps,n)\colon n\ge n_*\right\}\,,
\end{array}
\right \}
\label{FormMax}
\end{equation}
где
\begin{gather*}
D_\d(\eps,n) =\inf\limits_{t_0,\,U>0} D\left(\eps-\fr{k}{n^{\d/2}},n,t_0,U\right)\,,\\
n_*=\max\left\{1,\,\left\lceil\left(\fr{1+k}{\ell}\right)^{2/\d}\right\rceil\right\}\,.
\end{gather*}
Здесь $\lceil x\rceil$~--- минимальное целое, не меньшее~$x$. Условие
$n\ge n_*$ является следствием неравенства $\bet\ge1$. При этом для
оценивания супремума по~$n$ вместо входящих в $D_\d(\eps,n)$ величин
$r_j$, $j=1,2,3,4$, для достаточно больших~$n$ используются их
монотонные мажоранты. Вычисление максимума по~$\eps$ существенно
опирается на свойство монотонного возрастания по~$\eps$ всех
используемых оценок для функций~$|f_n(t)|$ и~$r_n(t)$, а
следовательно, и величины $D_\d(\eps)=\eps C_\d(\eps)$. Это свойство
позволяет оценить $\max\limits_{\eps}C_\d(\eps)$ по значениям~$C_\d(\eps)$
лишь в конечном числе точек. А~именно имеет место

\medskip

\noindent
\textbf{Лемма 6.} %\begin{lemma}\label{LemMonDeps}
\textit{Для всех $\eps_2>\eps_1>0$ имеет место неравенство}
$$
\max\limits_{\eps_1\le\eps\le\eps_2}C_\d(\eps)\le
C_\d(\eps_2)\fr{\eps_2}{\eps_1}\,.
$$

\medskip

Минимизация функции $D(\eps-k{n}^{-\d/2},\,n,\,t_0,\,U)$ по~$t_0$ и
$U$ проводится численно c использованием стандартных процедур в
системе Matlab~7.3 (R2006b).

\medskip

Перейдем теперь к описанию алгоритма вычисления константы~$\ex(\d)$
в неравенстве~(\ref{Bikelis}). Положим
$$
\eps = \fr{\bet}{n^{\d/2}}\,.
$$
%\vspace*{-12pt}


\noindent
\begin{center} %tabl5
\vspace*{-8pt}

\noindent
\parbox{79mm}{{\tablename~5}\ \ \small{Экстремальные значения
$n^*$, $\eps^*=(n^*)^{-\d/2}$ и оптимальные $t_0,U$ при вычислении
константы $\ex(\d)$}}
%\end{center}

\vspace*{2ex}

{\small 
\tabcolsep=11pt
\begin{tabular}{|c|c|c|c|c|c|}
  \hline
  $\d$ & $\eps_{\max}$ & $n^*$ & $\eps^*$ & $t_0$ & $U$\\
  \hline 
  0,9& 1,006& 3& 0,610& 0,35& 4,94\\
  0,8& 0,946& 3& 0,644& 0,39& 4,40\\
  0,7& 0,898& 3& 0,681& 0,46& 3,81\\
  0,6& 0,862& 4& 0,660& 0,53& 3,71\\
  0,5& 0,844& 5& 0,669& 0,66& 3,35\\
  0,4& 0,853& 6& 0,699& 0,82& 3,09\\
  0,3& 0,874& 5& 0,786& 1,00& 2,69\\
  0,2& 0,888& 4& 0,871& 0,78& 2,42\\
  0,1& 0,898& 9& 0,896& 0,87& 2,53\\
  \hline
\end{tabular}
}
\end{center}

\addtocounter{table}{1}

\vspace*{12pt}


\noindent
Тогда при $\eps\le0.3$ неравенство~(\ref{Bikelis}) с~$\ex(\d)$,
указанной в теореме~1, вытекает из
леммы~5. C~другой стороны, лемма~4
позволяет ограничить сверху область рассматриваемых значений~$\eps$
величиной $0{,}5409\ldots/\ex(\d)\equiv\eps_{\max}(\d)$, фиксированной
при каждом~$\d$. Таким образом, при вычислении~$\ex(\d)$
максимизация по~$\eps$ в формулах~(\ref{FormMax}) проводится на
конечном отрезке $0{,}3\le\eps\le\eps_{\max}(\d)$. Значения правой
границы~$\eps_{\max}(\d)$ приведены в
табл.~5. Для оценки
характеристических функций при $n<100$ используется~$f_1$, а при
$n\ge100$~--- функция~$f_2$, монотонно убывающая по~$n$, что
позволяет при каждом~$\eps$ оценивать супремум $D_\d(\eps,n)$ по
значениям~$n$ лишь в конечном числе точек:
$n_*,\ldots,\max\{n_*,100\}$, где $n_*=\max\{1,\eps^{-2/\d}\}$.
Максимум $C_\d(\eps)=D_\d(\eps)/\eps$ по
$0{,}3\le\eps\le\eps_{\max}(\d)$ оценивается  с помощью
леммы~6 и не превосходит тех значений~$\ex(\d)$,
которые указаны в формулировке теоремы~1.
Экстремальные значения $n=n^*$ и $\eps=(\ell^*)^{-\d/2}$ указаны в
табл.~5 в третьем и четвертом
столбцах, а соответствующие оптимальные значения параметров~$t_0$ и~$U$~--- 
в пятом и шестом столбцах. Отметим, что точке экстремума соответствует $\bet=1$.


Пусть теперь $k=1$. Обозначим
$$
\eps=\fr{\bet+1}{n^{\d/2}}\,.
$$
Тогда при $\eps\le0{,}3$ неравенство~(\ref{K-B-E-sharpened}) является
следствием леммы~5, а при $\eps\ge
0{,}5409\ldots/\exlow(\d)\equiv$\linebreak $\equiv\eps_{\max}(\d)$~--- следствием
леммы~4. Таким образом, при вычислении~$\exlow(\d)$
максимизацию по~$\eps$ в формулах~(\ref{FormMax}) достаточно
проводить на отрезке $0{,}3\le\eps\le\eps_{\max}(\d)$. Значения
$\eps_{\max}(\d)$ приведены в
табл.~6. Из этой таблицы видно,
что максимальное рассматриваемое значение~$\eps$ не превосходит~1,76. 
Для вычисления супремума по $n\ge n_*$ используется
лемма~3 с $T=2{,}2$, соответствующие значения
$N_1=$\linebreak $=N_1(2{,}2)$ и $N_3=N_3(2{,}2,\,1{,}76)$ приведены в
табл.~6 (для $N_3(T,\eps)$ взято
<<с запасом>> значение $\eps=1{,}76$). Как видно, уже при $n\ge184$
для всех рассматриваемых значений~$\d$ можно использовать оценки
$\widetilde r_1(t,\eps)$, \linebreak\vspace*{-12pt}
\pagebreak

%\vspace*{1pt}

\noindent
\begin{center} %tabl6
\vspace*{-8pt}

\noindent
\parbox{79mm}{{\tablename~6}\ \ \small{Экстремальные значения
$\eps^*$ и оптимальные $t_0,U$ при вычислении константы
$\exlow(\d)$}}
%\end{center}
\vspace*{2ex}

{\small 
\tabcolsep=8pt
\begin{tabular}{|c|c|c|c|c|c|c|}
  \hline
$\d$ & $\eps_{\max}$ & $N_1$ & $N_3$ & $\eps^*$ & $t_0$ & $U$\\
\hline 0,9& 1,752& 38& 142& 1,061& 0,38& 2,14\\
0,8& 1,698& 37& 110& 1,108& 0,40& 2,12\\
0,7& 1,623& 36& 91& 1,135& 0,43& 2,09\\
0,6& 1,534& 36& 79& 1,143& 0,44& 2,06\\
0,5& 1,434& 37& 73& 1,136& 0,46& 2,02\\
0,4& 1,326& 40& 71& 1,114& 0,47& 1,99\\
0,3& 1,213& 49& 75& 1,079& 0,48& 1,96\\
0,2& 1,099& 72& 90& 1,032& 0,49& 1,93\\
0,1& 0,985& 184& 144& 0,976& 0,49& 1,90\\
\hline
\end{tabular}
}
\end{center}

\addtocounter{table}{1}

\vspace*{12pt}


\noindent
$\widetilde r_3(t,\eps)$ из
леммы~3. Таким  образом, супремум по~$n$ достаточно оценивать
по значениям~$n$ лишь в конечном числе точек:
$n_*,\ldots,\max\{n_*,184\}$, где
 $n_*=\max\{1,(2/\eps)^{2/\d}\}$.
При этом экстремум целевой функции не превосходит значений,
указанных  в теореме~1 и достигается при $n\to\infty$
и $\eps=\eps^*(\d)$~--- указано в пятом столбце
табл.~6. Соответствующие
оптимальные значения~$t_0$ и~$U$ приведены в шестом и седьмом
столбцах табл.~6.



\bigskip

В заключение авторы выражают свою признательность В.\,Ю.~Королеву
за поддержку и постоянное внимание к работе.

{\small\frenchspacing
{%\baselineskip=10.8pt
\addcontentsline{toc}{section}{Литература}
\begin{thebibliography}{99}

\bibitem{Berry1941} %1
\Au{Berry A.\,C.} The accuracy of the Gaussian approximation to the
sum of independent variates~// Trans. Amer. Math. Soc., 1941.
Vol.~49. P.~122--139.

\bibitem{Esseen1942} %2
\Au{Esseen C.-G.} On the Liapunoff limit of error in the theory of
probability~// Ark. Mat. Astron. Fys., 1942. Vol.~A28. No.~9.
P.~1--19.

\bibitem{Katz1963} %3
\Au{Katz M.} A note on the Berry--Esseen theorem~// Ann. Math.
Statist., 1963. Vol.~34. P.~1107--1108.

\bibitem{Petrov1965} %4
\Au{Петров В.\,В.} Одна оценка отклонения распределения суммы
независимых случайных величин от нормального закона~// ДАН СССР,
1965. Т.~160. Вып.~5. С.~1013--1015.

\bibitem{Bikelis1966} %5
\Au{Бикялис А.} Оценки остаточного члена в центральной предельной
теореме~// Литовский математический сб., 1966. Т.~6. Вып.~3.
С.~323--346.

\bibitem{Petrov1972} %6
\Au{Петров В.\,В.} Суммы независимых случайных величин.~--- М.: Наука, 1972.

\bibitem{Esseen1956} %7
\Au{Esseen~C.-G.} A moment inequality with an application to the
central limit theorem~// Skand. Aktuarietidskr., 1956. Vol.~39.
P.~160--170.

\bibitem{KorolevShevtsova2010} %8
\Au{Королев В.\,Ю., Шевцова И.\,Г.} Уточнение неравенства
Берри--Эссеена с приложениями к пуассоновским и смешанным
пуассоновским случайным суммам~// Обозрение прикладной и
промышленной математики, 2010. Т.~17. Вып.~1. С.~25--56.

\bibitem{Tysiak1983} %9
\Au{Tysiak W.} Gleichm$\ddot{\mbox{a}}${\!\!\ptb\ss}ige und
nicht-gleichm$\ddot{\mbox{a}}${\!\!\ptb\ss}ige Berry--\linebreak Esseen--Absch{\"a}tzungen.
Dissertation. --- Wuppertal, 1983.

\bibitem{Paditz1996} %10
\Au{Paditz H.} On the error-bound in the nonuniform version of
Esseen's inequality in the $L_p$-metric~// Statistics, 1996.
Vol.~27. P.~379--394.

\bibitem{GaponovaKorchaginShevtsova2009} %11
\Au{Гапонова М.\,О., Корчагин А.\,Ю., Шевцова~И.\,Г.} Об абсолютных
константах в равномерной оценке точности нормальной аппроксимации
для распределений, не имеющих третьего момента~// Сб.\ статей
молодых ученых факультета ВМК МГУ. Вып.~6.~--- М.: Макс Пресс, 2009.
С.~81--89.

\bibitem{Paditz1986} %12
\Au{Paditz H.} $\ddot{\mbox{U}}$ber eine Fehlerabsch$\ddot{\mbox{a}}$tzung im zentralen
Grenzwertsatz~// Wiss. Z. Hochschule f$\ddot{\mbox{u}}$r Verkehswesen
``Friedrich List.''~--- Dresden, 1986. Bd.~33. H.~2. S.~399--404.

\bibitem{Shevtsova2010} %13
\Au{Шевцова И.\,Г.} Об асимптотически варл правильных постоянных в
центральной предельной теореме~// Тео\-рия вероятностей и ее
применения, 2010 (в пе\-ча\-ти). Т.~55. Вып.~2.

\bibitem{KorolevShevtsova2009} %14
\Au{Королев В.\,Ю., Шевцова И.\,Г.} О верхней оценке абсолютной
постоянной в неравенстве Берри--Эссеена~// Теория вероятностей и ее
применения, 2009. Т.~54. Вып.~4. С.~671--695.

\bibitem{Shevtsova2010a} %15
\Au{Шевцова И.\,Г.} Нижняя асимптотически правильная постоянная в
центральной предельной теореме~// Докл. РАН, 2010.
Т.~430. Вып.~4. С.~466--469.

\bibitem{KorolevShevtsova2010a} %16
\Au{Королев В.\,Ю., Шевцова И.\,Г.} Уточнение неравенства
Берри--Эссеена~// Докл. РАН, 2010. Т.~430. Вып.~6.
С.~738--742.

\bibitem{Zolotarev1966} %17
\Au{Золотарёв В.\,М.} Абсолютная оценка остаточного члена в
центральной предельной теореме~// Теория вероятностей и ее
применения, 1966. Т.~11. Вып.~1. С.~108--119.

\bibitem{Zolotarev1967a} %18
\Au{Золотарёв В.\,М.} Некоторые неравенства теории вероятностей и их
применение к уточнению теоремы А.\,М.~Ляпунова~// ДАН СССР, 1967.
Т.~177. №\,3. С.~501--504.

\bibitem{Zolotarev1967b} %19
\Au{Zolotarev V.\,M.} A sharpening of the inequality of
Berry--Esseen~// Z. Wahrsch. verw. Geb., 1967. Bd.~8. P.~332--342.

\bibitem{Prawitz1972} %20
\Au{Prawitz H.} Limits for a distribution, if the characteristic
function is given in a finite domain~// Scand. Aktuar Tidskr., 1972.
P.~138--154.

\bibitem{Shevtsova2009} %21
\Au{Шевцова И.\,Г.} Некоторые оценки для характеристических функций
с применением к уточнению неравенства Мизеса~// Информатика и её
применения, 2009. Т.~3. Вып.~3. С.~69--78.


\bibitem{Prawitz1975} %22
\Au{Prawitz H.} On the remainder in the central limit theorem.~I.
Onedimensional independent variables with finite absolute moments of
third order~// Scand. Actuarial J., 1975. No.~3. P.~145--156.

\bibitem{GaponovaShevtsova2009} %23
\Au{Гапонова М.\,О., Шевцова  И.\,Г.} Асимптотические оценки
абсолютной постоянной в неравенстве Берри--Эссеена для
распределений, не имеющих третьего момента~// Информатика и её
применения, 2009. Т.~3. Вып.~4. С.~41--56.

\label{end\stat}

\bibitem{BhatRangaRao1982} %24
\Au{Бхаттачария Р.\,Н., Ранга~Р.\,Р.} 
Аппроксимация нормальным распределением.~--- М.: Наука, 1982.
 \end{thebibliography}
}
}

\end{multicols}      %Abst


%\end{document}

%{ %\Large  
{ %\baselineskip=16.6pt

\vspace*{-48pt}
\begin{center}\LARGE
\textit{Уважаемый читатель!}
\end{center}

%\vspace*{2.5mm}

\vspace*{4mm}

\thispagestyle{empty}

{\small

 
В~2017~г.\ исполняется 10~лет со времени выхода в~свет первого 
номера журнала <<Информатика и~её применения>>~--- 
научного журнала Российской академии наук, издающегося под 
на\-уч\-но-ме\-то\-ди\-че\-ским руководством Отделения нанотехнологий 
и~информационных технологий Российской академии наук. Учредителем журнала 
является Федеральный исследовательский центр <<Информатика и~управ\-ле\-ние>> 
Российской академии наук (ФИЦ ИУ РАН) (до~2015~г.~--- 
Институт проб\-лем информатики РАН).

Необходимость издания такого журнала была вызвана активным развитием 
информатики и~информационных технологий, большой важностью этого научного 
направления для развития страны, проникновением информационных технологий 
во все сферы жизни современного общества.

Тематику журнала определяет тот факт, что информатика~--- это комплексная 
фундаментальная научная дисциплина, опирающаяся на достижения 
ряда других наук, в~том числе математики, физики, лингвистики и~др. 
Одновременно журнал уделяет большое внимание современным информационным технологиям, 
являющимся приложениями результатов информатики как фундаментальной науки.

За прошедшие 10~лет (2007--2016~гг.)\ издано~38~выпусков журнала. В~них 
размещено~452~публикации, в~том числе~430~научных статей и~22~информационных 
публикации (обзоры, рецензии и~др.). Среди авторов журнала представители ведущих 
научных организаций и~университетов страны, в~том числе Московского государственного 
университета им.\ М.\,В.~Ломоносова, ФИЦ ИУ РАН (в~том числе ИПИ РАН, ВЦ 
им.\ А.\,А.~Дородницына РАН, ИСА РАН), Института точной механики и~вычислительной 
техники им.\ С.\,А.~Лебедева РАН, Института космических исследований РАН, 
Института астрономии РАН, ряда институтов Сибирского отделения РАН, МФТИ, МИФИ, 
Высшей школы экономики, Санкт-Пе\-тер\-бург\-ско\-го государственного университета, 
Санкт-Пе\-тер\-бург\-ско\-го государственного политехнического университета 
Петра Великого, Санкт-Пе\-тер\-бург\-ско\-го государственного университета 
телекоммуникаций им.\ проф.\ М.\,А.~Бонч-Бруе\-ви\-ча, 
Российского университета дружбы народов, Балтийского федерального университета 
имени Иммануила Канта, Вологодского государственного университета и~др. 
Публиковались статьи зарубежных авторов, в~том числе ученых из Израиля, 
США, Финляндии, Франции, Швейцарии, Швеции и~других стран. 

В конце настоящего выпуска журнала помещен указатель статей, 
опуб\-ли\-ко\-ван\-ных в~томах~1--10 (2007--2016~гг.).

Журнал включен в~Российский индекс научного цитирования и~в~базу 
данных RSCI Web of Science, перечень ВАК, базу данных CrossRef 
и~информационную систему <<Общероссийский математический портал MathNet>>. 
С~2015~г.\ журнал индексируется в~библиографической и~реферативной базе 
данных SCOPUS.

Мы всегда будем помнить ушедших из жизни членов редакционного совета 
и~редакционной коллегии журнала: академика С.\,К.~Коровина, профессоров 
А.\,В.~Печинкина и~И.\,А.~Ушакова, которые внесли неоценимый вклад в~становление 
и~развитие журнала.

После объединения в~2015~г.\ трех учреждений Российской академии наук~--- 
Института проблем информатики, Вычислительного центра им.\ А.\,А.~Дородницына 
и~Института системного анализа~--- в~Федеральное государственное учреждение 
<<Федеральный исследовательский центр <<Информатика и~управ\-ле\-ние>> 
Российской академии наук>> (ФИЦ ИУ РАН) именно этот Центр стал базовой организацией 
для издания журнала, что существенно расширило как тематику журнала, 
так и~его возможности по привлечению новых авторов, в~том числе и~зарубежных.

В настоящее время тематику журнала в~первую очередь составляют:
\begin{itemize}
\item    теоретические основы информатики;\\[-14.5pt] 
\item    математические методы исследования сложных систем и~процессов;\\[-14.5pt]
\item    информационные системы и~сети;\\[-14.5pt]
\item    информационные технологии;\\[-14.5pt]
\item    архитектура и~программное обеспечение вычислительных комплексов и~сетей. 
\end{itemize}

Эти направления особенно важны в~связи с необходимостью решения задач 
формирования технологической базы инновационного развития, обеспечения 
на\-уч\-но-тех\-но\-ло\-ги\-че\-ско\-го прорыва в~области создания и~развития 
отечественных информационных и~коммуникационных технологий в~интересах 
достижения высокого качества и~стабильности систем управления и~предоставления 
услуг в~экономической и~социальной сферах. 

Мы, как и~ранее, приглашаем авторов представлять для публикации в~журнале 
статьи как с достижениями в~области теоретических проблем информатики, так 
и~с~изложением результатов ее практического приложения, а~также 
рецензии на наиболее интересные книжные новинки в~области информатики 
и~информационных технологий, объявления о~крупнейших международных 
и~всероссийских конференциях, различных научных мероприятиях 
по этой тематике и~другие информационные материалы.

Надеемся, что и~в~дальнейшем содержание статей, помещаемых в~журнале, 
будет вызывать интерес научной общественности. Редакционный совет, редколлегия 
и~редакция журнала, со своей стороны, сделают все для того, 
чтобы журнал и~впредь своевременно и~подробно информировал читателей 
о~новейших достижениях информатики и~ее актуальных практических приложениях.

                

      
\vfill
\noindent
Главный редактор журнала <<Информатика и~её применения>>,\\
академик  РАН\hfill
\textit{И.\,А.~Соколов}\\[-6pt]

%\noindent
%Редактор-составитель тематического выпуска, профессор кафедры математической статистики\\
%факультета вычислительной математики и~кибернетики МГУ им.~М.\,В.~Ломоносова,\\
%ведущий научный сотрудник ИПИ РАН, доктор физико-математических наук\hfill
% \textit{В.\,Ю.~Королев}


} }
}
      

%%%%%%%%%%%%%%%%%%%%%%%%%%%%%%%%%%%%%%%%%%%%%%%


%\def\stat{rez}
{%\hrule\par
%\vskip 7pt % 7pt
\raggedleft\Large \bf%\baselineskip=3.2ex
Р\,Е\,Ц\,Е\,Н\,З\,И\,И \vskip 17pt
    \hrule
    \par
\vskip 6pt plus 6pt minus 3pt }

%\thispagestyle{headings} %с верхним колонтитулом
%\thispagestyle{myheadings} %с нижним колонтитулом, но в верхнем РЕЦЕНЗИИ

\def\tit{НОВАЯ КНИГА И.\,Н.~СИНИЦЫНА, А.\,С.~ШАЛАМОВА <<ЛЕКЦИИ ПО ТЕОРИИ 
ИНТЕГРИРОВАННОЙ ЛОГИСТИЧЕСКОЙ ПОДДЕРЖКИ>> (М.: ТОРУС ПРЕСС, 2012. 624~с.)}

%1
\def\aut{Д.ф.-м.н., профессор С.\,Я.~Шоргин}

\def\auf{\ }

\def\leftkol{\ % РЕЦЕНЗИИ
}

\def\rightkol{ \ } 

%\def\leftkol{\ } % ENGLISH ABSTRACTS}

%\def\rightkol{\ } %ENGLISH ABSTRACTS}

%\def\leftkol{РЕЦЕНЗИИ}

%\def\rightkol{РЕЦЕНЗИИ}

\titele{\tit}{\aut}{\auf}{\leftkol}{\rightkol}
\vspace*{-18pt}


     \label{st\stat}

     \begin{multicols}{2}
     {\small
     {\baselineskip=10.1pt
     

      В книге представлено системное изложение теоретических основ одного из новейших 
направлений в \mbox{об\-ласти} экономики послепродажного обслуживания изделий наукоемкой 
продукции (ИНП) длительного пользования~--- интегрированной логистической поддержки
(ИЛП). 
{\looseness=1

}

Приведены также результаты новых работ, выполненных в Институте проблем информатики 
Российской академии наук в рамках научного направления <<Информационные технологии и 
анализ сложных сис\-тем>>.
 {%\looseness=1

}
     
      Излагаемые в книге научные подходы позво\-ляют карди\-наль\-но реформировать 
существующие системы производства и эксплуатации ИНП путем создания и внед\-ре\-ния 
методов рационального и оптимального управ\-ле\-ния процессами расходования 
вре\-мен\-н$\acute{\mbox{ы}}$х, 
мате\-ри\-аль\-ных, трудовых и других ресурсов на всех стадиях жизненного цикла изделий (ЖЦИ) по 
критериям экономической целесообразности и эф\-фек\-тив\-ности.
  {\looseness=1

}
    
      В книге приведен краткий обзор причин возник\-новения и
      развития CALS-методологии как основы 
современных международных стандартов по созданию и функционированию глобальных 
ин\-фор\-ма\-ци\-он\-но-ком\-му\-ни\-ка\-ци\-он\-ных систем, ее ключевых возможностей и эффективности 
результатов ее использования. 
Авторы %\linebreak 
предлагают ряд научных обоснований для разработки 
единой теории проектирования и управления систем ИЛП для полноценного использования 
преимуществ %\linebreak
 суще\-ст\-ву\-ющей методологии, определяют \mbox{общую} структурную схему 
комплексной системы <<ИНП-СППО>> и необходимость разработки для ее описания 
гибридных стохастических моделей.
{%\looseness=1

}

%\columnbreak
      
      Книга состоит из пяти частей, где последовательно излагается материал по каждой из 
следующих тем: <<Интегрированная логистическая поддержка>>, <<Теория гибридных 
стохастических систем и компьютерная поддержка исследований и разработок>>, <<Основы 
математического моделирования, анализа и синтеза систем послепродажного обслуживания>>, 
<<Определение и анализ показателей экспортного потенциала ИНП при проектировании>>, 
<<Задачи управления поддержкой послепродажного обслуживания>>, а также 
<<Моделирование инвестиционных процессов ИЛП в условиях неравновесных финансовых 
рынков>>. 
   
      В конце каждой главы приведены выводы и даны вопросы и задания для 
самоконтроля. В~приложениях содержатся основные определения по программам работ по 
анализу ИЛП, логистическим базам данных и компьютерным решениям, эквивалентной статистической 
линеаризации нелинейных преобразований ИЛП, справочный материал, а также развернутые 
уравнения для вероятностных характеристик.


      \def\leftkol{РЕЦЕНЗИИ}

\def\rightkol{РЕЦЕНЗИИ} 

      
      Книга заинтересует широкий круг специалистов и может быть использована научными 
проектными организациями в сфере промышленного производства ИНП. Большое количество 
иллюстраций, примеров и вопросов, обращенных к читателю, позволяет использовать книгу 
также в качестве учебного пособия для студентов и аспирантов машиностроительных, 
транспортных и~других специальностей, а также для самостоятельного изучения. 
{%\looseness=-1

}

Книга 
представляет несомненный интерес для специалистов и студентов в области прикладной 
математики и информатики.
    

}

}
\end{multicols}

%\newpage

%\end{document}

\include{obchak}


\def\stat{authorsrus}
{%\hrule\par
%\vskip 7pt % 7pt
\raggedleft\Large \bf%\baselineskip=3.2ex
О\,Б\ \ А\,В\,Т\,О\,Р\,А\,Х \vskip 17pt
    \hrule
    \par
\vskip 21pt plus 8pt minus 6pt }


\def\tit{\ }

\def\aut{\ }

\def\auf{\ }

\def\leftkol{ОБ АВТОРАХ}

\def\rightkol{\ }

\titele{\tit}{\aut}{\auf}{\leftkol}{\rightkol}
\addcontentsline{toc}{subsection}{\textrm\textbf ОБ АВТОРАХ}
\label{st\stat}



\vspace*{-38pt}

\begin{multicols}{2}

\noindent
\textbf{Агаларов Явер Мирзабекович} (р.\ 1952)~--- 
кандидат технических наук, доцент, ведущий научный сотрудник 
Института проб\-лем информатики Федерального исследовательского центра 
<<Информатика и~управ\-ле\-ние>> Российской академии наук

\vspace*{3pt}

\noindent
\textbf{Битюков Юрий Иванович} (р.\ 1972)~---
доктор технических наук, доцент Московского авиационного института 
(национального исследовательского университета) 

\vspace*{3pt}

\noindent
\textbf{Буянов Михаил Владимирович} (р.\ 1994)~--- 
аспирант Московского авиационного института (национального исследовательского 
университета)

\vspace*{3pt}

\noindent
\textbf{Вихрова Ольга Геннадиевна} (р.\ 1990)~---
 аспирант Российского университета дружбы народов
 
 \vspace*{3pt}
 

\noindent
\textbf{Гайдамака Юлия Васильевна} (р.\ 1971)~--- 
кандидат фи\-зи\-ко-ма\-те\-ма\-ти\-че\-ских наук, доцент Российского университета 
дружбы народов; старший научный сотрудник Института проб\-лем информатики 
Федерального исследовательского центра <<Информатика и~управ\-ле\-ние>> 
Российской академии наук 

\vspace*{3pt}

\noindent
\textbf{Горшенин Андрей Константинович} (р.\ 1986)~--- 
кандидат фи\-зи\-ко-ма\-те\-ма\-ти\-че\-ских наук, доцент, 
ведущий научный сотрудник Института проб\-лем информатики Федерального 
исследовательского\linebreak
 центра <<Информатика и~управ\-ле\-ние>> Российской академии наук;
 старший научный сотрудник
 Института океанологии им.\ П.\,П.~Ширшова Российской академии наук

\vspace*{3pt}


\noindent
\textbf{Гребешков Александр Юрьевич} (р.\ 1967)~--- 
кандидат технических наук, старший научный сотрудник Поволжского 
государственного университета телекоммуникаций и информатики

\vspace*{3pt}

\noindent
\textbf{Грушо Александр Александрович} (р.\ 1946)~--- доктор 
фи\-зи\-ко-ма\-те\-ма\-ти\-че\-ских наук, профессор, заведующий лабораторией 
Института проб\-лем информатики Федерального исследовательского центра 
<<Информатика и~управ\-ле\-ние>> Российской академии наук 

\vspace*{3pt}

\noindent
\textbf{Забежайло Михаил Иванович} (р.\ 1956)~--- 
кандидат фи\-зи\-ко-ма\-те\-ма\-ти\-че\-ских наук, доцент, заведующий лабораторией 
Института проб\-лем информатики Федерального исследовательского центра 
<<Информатика и~управ\-ле\-ние>> Российской академии наук 

%\vspace*{3pt}
\columnbreak

\noindent
\textbf{Зарипова Эльвира Ринатовна} (р.\ 1979)~--- 
кандидат фи\-зи\-ко-ма\-те\-ма\-ти\-че\-ских наук, доцент Российского университета 
дружбы народов

\vspace*{3pt}

\noindent
\textbf{Иванов Сергей Валерьевич} (р.\ 1989)~--- 
кандидат фи\-зи\-ко-ма\-те\-ма\-ти\-че\-ских наук, доцент Московского 
авиационного института (национального исследовательского университета)

\vspace*{3pt}

\noindent
\textbf{Кибзун Андрей Иванович}  (р.\ 1951)~--- 
доктор фи\-зи\-ко-ма\-те\-ма\-ти\-че\-ских наук, профессор, 
заведующий кафедрой Московского авиационного института 
(национального исследовательского университета)

\vspace*{3pt}

\noindent
\textbf{Королев Виктор Юрьевич} (р.\ 1954)~--- доктор 
фи\-зи\-ко-ма\-те\-ма\-ти\-че\-ских наук, профессор, 
заведующий кафедрой математической статистики факультета вычислительной 
математики и~кибернетики МГУ им.\ М.\,В.~Ломоносова; 
ведущий научный сотрудник Института проб\-лем информатики 
Федерального исследовательского центра <<Информатика и~управ\-ле\-ние>> 
Российской академии наук; профессор Университета Дианьзи города Ханчжоу (Китай)

\vspace*{3pt}


\noindent
\textbf{Кружков Михаил Григорьевич} (р.\ 1975)~--- 
старший научный сотрудник Института проб\-лем 
информатики Федерального исследовательского центра 
<<Информатика и~управ\-ле\-ние>> Российской академии наук

\vspace*{3pt}

\noindent
\textbf{Кудрявцев Алексей Андреевич} (p.\ 1978)~--- кандидат 
фи\-зи\-ко-ма\-те\-ма\-ти\-че\-ских наук, 
доцент кафедры математической статистики факультета вычислительной математики 
и~кибернетики Московского государственного университета им.\ М.\,В.~Ломоносова

\vspace*{3pt}

\noindent
\textbf{Лисовская Екатерина Юрьевна} (р.\ 1992)~--- 
аспирант Национального исследовательского 
Томского государственного университета 

\vspace*{3pt}

\noindent
\textbf{Малашенко Юрий Евгеньевич} (р.\ 1946)~---
доктор фи\-зи\-ко-ма\-те\-ма\-ти\-че\-ских наук, заведующий сектором 
Вычислительного центра им.\ А.\,А.~Дородницына Федерального исследовательского центра 
<<Информатика и~управ\-ле\-ние>> Российской академии \mbox{наук}

\vspace*{3pt}


\noindent
\textbf{Моисеева Светлана Петровна} (р.\ 1971)~--- 
доктор фи\-зи\-ко-ма\-те\-ма\-ти\-че\-ских наук, доцент; 
профессор Национального исследовательского Томского государственного 
университета  

%\vspace*{3pt}
\pagebreak

\noindent
\textbf{Мокров Евгений Владимирович} (р.\ 1988)~--- 
аспирант Российского университета дружбы народов 

\vspace*{3pt}

\noindent
\textbf{Назарова Ирина Александровна} (р.\ 1966)~---
 кандидат фи\-зи\-ко-ма\-те\-ма\-ти\-че\-ских наук, научный сотрудник 
 Вычислительного центра им.\ А.\,А.~Дородницына Федерального исследовательского центра 
 <<Информатика и~управ\-ле\-ние>> Российской академии наук

\vspace*{3pt}

\noindent
\textbf{Наумов Андрей Викторович} (р.\ 1966)~--- 
доктор фи\-зи\-ко-ма\-те\-ма\-ти\-че\-ских наук, доцент, 
профессор\linebreak Московского авиационного института (национального исследовательского 
университета)

\vspace*{3pt}

\noindent
\textbf{Наумов Валерий Арсентьевич} (р.\ 1950)~--- 
кандидат фи\-зи\-ко-ма\-те\-ма\-ти\-че\-ских наук, 
научный руководитель Исследовательского института инноваций, 
г.~Хельсинки, Финляндия

\vspace*{3pt}

\noindent
\textbf{Новикова Наталья Михайловна} (р.\ 1953)~--- 
доктор фи\-зи\-ко-ма\-те\-ма\-ти\-че\-ских наук, профессор, ведущий научный сотрудник 
Вычислительного центра им.\ А.\,А.~Дородницына Федерального исследовательского центра 
<<Информатика и~управ\-ле\-ние>> Российской академии наук

\vspace*{3pt}

\noindent
\textbf{Пагано Микеле} (р.\ 1968)~---
PhD по информационным технологиям, профессор Университета 
г.\ Пиза (Италия) 

\vspace*{3pt}

\noindent
\textbf{Платонов Евгений Николаевич} (р.\ 1976)~---  
кандидат фи\-зи\-ко-ма\-те\-ма\-ти\-че\-ских наук, 
доцент Московского авиационного института (национального исследовательского 
университета)

\vspace*{3pt}

\noindent
\textbf{Потатуева Виктория Владимировна} (р.\ 1993)~---  
студентка магистратуры Национального исследовательского 
Томского государственного университета

\vspace*{3pt}


\noindent
\textbf{Разумчик Ростислав Валерьевич} (р.\ 1984)~--- 
кандидат фи\-зи\-ко-ма\-те\-ма\-ти\-че\-ских наук, 
ведущий научный сотрудник Института проб\-лем 
информатики Федерального исследовательского центра <<Информатика и~управ\-ле\-ние>>
Российской академии наук;  доцент Российского университета дружбы народов

\vspace*{3pt}

\noindent
\textbf{Самуйлов Константин Евгеньевич} (р.\ 1955)~---
доктор технических наук, профессор, заведующий ка\-фед\-рой Российского 
университета дружбы наро-\linebreak дов, директор Института прикладной математики\linebreak 
и~телекоммуникаций Российского университета дружбы народов; 
старший научный сотрудник Института проб\-лем информатики Федерального 
исследовательского центра <<Информатика и~управ\-ле\-ние>> 
Российской академии наук

\vspace*{3pt}

\noindent
\textbf{Смирнов Дмитрий Владимирович} (р.\ 1984)~--- 
биз\-нес-парт\-нер по информационным технологиям Департамента безопасности ПАО 
<<Сбербанк России>>

\vspace*{3pt}

\noindent
\textbf{Тимонина Елена Евгеньевна} (р.\ 1952)~--- 
доктор технических наук, профессор, ведущий научный\linebreak сотрудник 
Института проб\-лем информатики Федерального исследовательского центра 
<<Информатика и~управ\-ле\-ние>> Российской академии наук 

\vspace*{3pt}

\noindent
\textbf{Титова Анастасия Игоревна} (p.\ 1995)~--- 
студентка кафедры математической статистики факультета вычисли\-тельной математики 
и~кибернетики Московского государственного университета им.\ М.\,В.~Ломоносова

\vspace*{3pt}

\noindent
\textbf{Шоргин Всеволод Сергеевич} (р.\ 1978)~---
кандидат технических наук, старший научный сотрудник Института проб\-лем 
информатики Федерального исследовательского центра <<Информатика и~управ\-ле\-ние>> 
Российской академии наук

\vspace*{3pt}

\noindent
\textbf{Шоргин Сергей Яковлевич} (р.\ 1952)~--- 
доктор фи\-зи\-ко-ма\-те\-ма\-ти\-че\-ских наук, профессор, заместитель директора 
Федерального исследовательского цент\-ра <<Информатика и~управ\-ле\-ние>> 
Российской академии наук (ФИЦ ИУ РАН); главный научный сотрудник Института проб\-лем 
информатики ФИЦ ИУ РАН
 



 \label{end\stat}

%\def\leftfootline{\small{\textbf{\thepage}
%\hfill ИНФОРМАТИКА И ЕЁ ПРИМЕНЕНИЯ\ \ \ том~11\ \ \ выпуск~4\ \ \ 2017}
%}%
% \def\rightfootline{\small{ИНФОРМАТИКА И ЕЁ ПРИМЕНЕНИЯ\ \ \ том~11\ \ \ выпуск~4\ \ \ 2017
%\hfill \textbf{\thepage}}}


%\thispagestyle{myheadings}



\end{multicols}

\newpage


\def\stat{authors}
{%\hrule\par
%\vskip 7pt % 7pt
\raggedleft\Large \bf%\baselineskip=3.2ex
A\,B\,O\,U\,T\  \  A\,U\,T\,H\,O\,R\,S \vskip 17pt
    \hrule
    \par
\vskip 21pt plus 8pt minus 3pt }

\label{st\stat}


\def\leftkol{ABOUT AUTHORS} % 
\def\rightkol{\ } %ABOUT AUTHORS} 


\vspace*{36pt}

\begin{multicols}{2}

%\vspace*{4pt}

\noindent \textbf{Andreev Arkady M.} (b.\ 1943)~--- Candidate of Science (PhD) in 
technology, assistant professor, Bauman Moscow State Technical University

\vspace*{4pt}

\vspace*{4pt}

\noindent
\textbf{Belyaev Mikhail G.} (b.\ 1987)~--- PhD student, Institute for Information 
Transmission Problems, Russian Academy of Sciences; junior scientist, Moscow 
Institute of Physics and Technology; scientist, Datadvance LLC

\vspace*{4pt}

\noindent
\textbf{Berezkin Dmitry V.} (b.\ 1966)~--- Candidate of Science (PhD) in technology, senior 
scientist, Bauman Moscow State Technical University


\vspace*{4pt}

\noindent
\textbf{Burnaev Evgeny V.} (b.\ 1983)~--- Candidate of Science (PhD) in physics and 
mathematics, associate professor; Head of Laboratory, Institute for Information 
Transmission Problems, Russian Academy of Sciences; senior scientist, Moscow 
Institute of Physics and Technology; Head of Laboratory, Datadvance LLC

\vspace*{4pt}

\noindent
\textbf{Glushanovskiy Alexey V.} (b.\ 1944)~--- senior scientist, Library for Natural 
Sciences, Russian Academy of Sciences

\vspace*{4pt}

\noindent
\textbf{Kaganov Vladislav Yu.} (b.\ 1993)~--- student, Faculty of Computational Mathematics 
and Cybernetics, M.\,V.~Lomonosov Moscow State University


\vspace*{4pt}

\noindent
\textbf{Kalenov Nikolay E.} (b.\ 1945)~--- Doctor of Science in technology, professor, 
Director, Library for Natural Sciences, Russian Academy of Sciences 

\vspace*{4pt}

\noindent
\textbf{Kapnin Alexey V.} (b.\ 1986)~--- PhD student, assistant professor of Lipetsk State 
Technical University


\vspace*{4pt}

\noindent \textbf{Kirikov Igor  A.} (b.\ 1955)~--- Candidate of Science (PhD) 
in technology, Director, Kaliningrad Branch of Institute of Informatics 
Problems, Russian Academy of Sciences 

\vspace*{4pt}

\noindent
\textbf{Klemenkov Pavel A.} (b.\ 1986)~--- PhD student, Department of  System Programming, 
Faculty of Computational Mathematics and Cybernetics, M.\,V.~Lomonosov Moscow State 
University


\vspace*{4pt}

\noindent
\textbf{Kolesnikov Alexander V.} (b.\ 1948)~--- Doctor of Science in technology; professor, 
Immanuel Kant Baltic Federal University; senior scientist, Kaliningrad Branch 
of Institute of Informatics Problems, Russian Academy of Sciences 

\columnbreak

\noindent
\textbf{Korenkov Vladimir V.} (b.\ 1953)~--- Candidate of Science (PhD) in physics and 
mathematics; Director, Laboratory of Information Technologies, Joint 
Institute for Nuclear Research (JINR); Head of Department, International 
University of Nature, Society and Man ``Dubna'' 

\vspace*{5.5pt}


\noindent
\textbf{Korolev Victor Yu.} (b.\ 1954)~--- Doctor of Science in physics and mathematics, 
professor, Department of Mathematical Statistics, Faculty of Computational 
Mathematics and Cybernetics, M.\,V.~Lomonosov Moscow State University; leading 
scientist, Institute of Informatics Problems, Russian Academy of Sciences

\vspace*{5.5pt}

\noindent
\textbf{Korolyov Andrey K.} (b.\ 1992)~--- student, Faculty of Computational Mathematics 
and Cybernetics, M.\,V.~Lomonosov Moscow State University

\vspace*{5.5pt}

\noindent
\textbf{Kovalyov Sergey P.} (b.\ 1972)~--- Candidate of Science (PhD) in physics and 
mathematics, senior scientist, Institute of Control Problems, Russian Academy 
of Sciences

\vspace*{5.5pt}

\noindent
\textbf{Kozhunova Olga S.} (b.\ 1982)~--- Candidate of Science (PhD) in technology, Head of 
Laboratory, Institute of Informatics Problems, Russian Academy of Sciences

\vspace*{5.5pt}

\noindent
\textbf{Kozlov Ilya A.} (b.\ 1989)~--- MD student, Department of 
Informatics and Control Systems, Bauman Moscow State 
Technical University

\vspace*{5.5pt}

\noindent
\textbf{Krylov Michael N.} (b.\ 1992)~--- student, Faculty of Computational Mathematics and 
Cybernetics, M.\,V.~Lomonosov Moscow State University

\vspace*{5.5pt}

\noindent
\textbf{Kuznetsov Leonid A.} (b.\ 1942)~--- Doctor of Science in technology, professor, 
Head of Department, Russian Presidential Academy of National Economy and Public 
Administration  (Lipetsk Branch)

\vspace*{5.5pt}

\noindent
\textbf{Kuznetsova Vera F.} (b.\ 1948)~--- Candidate of Science (PhD) in technology, 
associate professor of the Russian Presidential Academy of National Economy and 
Public Administration  (Lipetsk Branch)


\vspace*{5.5pt}

\noindent
\textbf{Listopad Sergey V.} (b.\ 1984)~--- Candidate of Science (PhD) in technology, 
scientist, Kaliningrad Branch of Institute of Informatics Problems, Russian 
Academy of Sciences

\vspace*{5.5pt}

\noindent
\textbf{Mashechkin Igor V.} (b.\ 1956)~--- Doctor of Science in physics and mathematics, 
professor, Faculty of Computational Mathematics and Cybernetics, M.\,V.~Lomonosov Moscow State University

%\pagebreak

\vspace*{2pt}



\noindent
\textbf{Nechaevskiy Andrey V.} (b.\ 1982)~--- programmer, Laboratory of Information 
Technologies, Joint Institute for 
nuclear research (JINR) 

\vspace*{2pt}

\noindent
\textbf{Petrovskiy Michael I.} (b.\ 1975)~--- Candidate of Science (PhD) in physics and 
mathematics,  associate professor, Faculty of Computational Mathematics and 
Cybernetics, M.\,V.~Lomonosov Moscow State University

\def\leftkol{ABOUT AUTHORS} %
\def\rightkol{ABOUT AUTHORS} 

\vspace*{2pt}

\noindent
\textbf{Shkotin Alexander V.} (b.\ 1952)~--- software engineer, GIS Department,  State 
Geological Museum of Russian Academy of Sciences

\columnbreak

%\vspace*{4pt}

\noindent
\textbf{Simakov Konstantin V.} (b.\ 1980)~--- Candidate of Science 
(PhD) in technology, senior scientist, Bauman Moscow State Technical 
University

%\columnbreak

\def\leftkol{ABOUT AUTHORS} %
\def\rightkol{ABOUT AUTHORS} 

\vspace*{7pt}

\noindent
\textbf{Stupnikov Sergey A.} (b.\ 1978)~--- Candidate of Science (PhD) in technology, 
senior scientist, Institute of Informatics Problems, Russian Academy of 
Sciences

\vspace*{7pt}

\noindent
\textbf{Trofimov Vladimir V.} (b.\ 1955)~--- leading programmer, 
Laboratory of Information Technologies, Joint Institute for nuclear research 
(JINR)


\vspace*{7pt}

\noindent
\textbf{Zaks Lily M.} (b.\ 1989)~--- principal officer, Department of Modeling and 
Mathematical Statistics, Alpha-Bank

\def\leftkol{ABOUT AUTHORS} %
\def\rightkol{ABOUT AUTHORS} 


\end{multicols}
\newpage

%   \vspace*{-36pt}

\begin{center}
\vspace*{6pt}
\mbox{%
\epsfxsize=79.5mm
\epsfbox{korov-tg.eps}
}
\end{center} 

\vspace*{12pt} %Академик


   \begin{center}
\fbox{\Large\textbf{Академик Сергей Константинович Коровин}}\\[12pt]
\textbf{\large 24.05.1945--7.12.2011}
   \end{center}
   
   %\vspace*{2.5mm}
   
   \vspace*{5mm}
   
   \thispagestyle{empty}

%\

%\vspace*{-12pt}


Редакционная коллегия журнала <<Информатика и её применения>> с глубоким 
прискорбием извещает, что 7 декабря~2011~года на 67-м году жизни скоропостижно 
скончался выдающийся российский ученый в области теории управления сложными 
динамическими системами, член редколлегии журнала <<Информатика и её применения>> 
академик КОРОВИН Сергей Константинович.

Коровин Сергей Константинович окончил факультет радиотехники и 
кибернетики Московского физико-технического института в 1969~г. 
С~1969~г.\ по 1975~г.\ работал в Институте проблем управления АН СССР. 
Здесь же без отрыва от производства учился в аспирантуре (1971--1974), 
защитил диссертацию на степень кандидата технических наук по теме 
<<Алгоритмы оптимизации на скользящих режимах>> (1975). С~1975 по 2011~гг.\ 
работал в Институте системного анализа Российской академии наук 
в должностях от ведущего инженера 
до главного научного сотрудника, заведующего лабораторией. В~1985~г.\ защитил 
диссертацию на степень доктора технических наук по теме <<Системы управления 
с автоматически регулируемыми связями>>, в 1990~г.\ ему присвоено ученое звание профессора.

С 1989~г.\ С.\,К.~Коровин работал в МГУ им.\ М.\,В.~Ломоносова, с 1996~г.\ 
являлся профессором кафедры нелинейных динамических систем и процессов 
управ\-ле\-ния факультета вычислительной математики и кибернетики. 

В 1994~г.\ избран членом-кор\-рес\-пон\-ден\-том РАН, в 2000~г.~--- действительным членом 
РАН (2000). Лауреат Государственной премии РФ (1994), премии Совета Министров СССР (1981), 
премии Правительства РФ (2009), премии РАН им.\ А.\,А.~Андронова (2000), Ломоносовской 
премии МГУ I~степени в области науки (2002). С.\,К.~Коровин~--- автор 260~научных работ, в том 
чис\-ле 15~книг, 50~авторских свидетельств. 

Сергей Константинович~Коровин являлся членом редколлегии журнала <<Информатика и её применения>> с 
момента основания журнала и принимал активное участие в формировании редакционной 
политики журнала. 

\include{podgot-2str}


%\end{document}

%\include{IPPM-25}

%\def\stat{cont}
{%\hrule\par
%\vskip 7pt % 7pt
\raggedleft\Large \bf%\baselineskip=3.2ex
А\,В\,Т\,О\,Р\,С\,К\,И\,Й\ \ У\,К\,А\,З\,А\,Т\,Е\,Л\,Ь\ \ З\,А\ \ 2\,0\,1\,0 г. \vskip 17pt
    \hrule
    \par
\vskip 21pt plus 6pt minus 3pt }

\label{st\stat}

\def\tit{\ }

\def\aut{\ }
\def\auf{\ }

\def\leftkol{\ } % ENGLISH ABSTRACTS}

\def\rightkol{\ } %АВТОРСКИЙ УКАЗАТЕЛЬ ЗА 2010 г.} %ENGLISH ABSTRACTS}

\titele{\tit}{\aut}{\auf}{\leftkol}{\rightkol}

\vspace*{-12pt}

{\tabcolsep=3pt
\begin{tabular}{p{388pt}rr}
&\textbf{Выпуск} & \textbf{Стр.}\\[6pt]
\hangindent=23pt\noindent\textbf{Арутюнян~А.\,Р.} Моделирование влияния деформаций отпечатков пальцев на 
точность\linebreak
\vspace*{-12pt}\\
\hspace*{23pt}дактилоскопической идентификации$\dotfill$&1&51\\
\hangindent=23pt\noindent\textbf{Архипов~О.\,П., Зыкова~З.\,П.} Интеграция гетерогенной информации о цветных 
пикселях\linebreak
\vspace*{-12pt}\\
\hspace*{23pt}и их цветовосприятии$\dotfill$&4&15\\
\hangindent=23pt\noindent\textbf{Баранов~С.\,И., Френкель~С.\,Л., Захаров~В.\,Н.} Полуформальная верификация 
цифрового устройства с конвейером, основанная на использовании алгоритмических машин\linebreak
\vspace*{-12pt}\\
\hspace*{23pt}состояния$\dotfill$&4&49\\
\textbf{Бекетова~И.\,В.} см.~Каратеев~С.\,Л.&&\\
\textbf{Белоусов~В.\,В.} см.~Синицын~И.\,Н.&&\\
\hangindent=23pt\noindent\textbf{Бенинг~В.\,Е., Королев~Р.\,А.} О предельном поведении мощностей критериев в 
случае\linebreak
\vspace*{-12pt}\\
\hspace*{23pt}распределения Лапласа$\dotfill$&2&63\\
\hangindent=23pt\noindent\textbf{Бенинг~В.\,Е., Сипина~А.\,В.} Асимптотическое разложение для мощности 
критерия,\linebreak
\vspace*{-12pt}\\
\hspace*{23pt}основанного на выборочной медиане, в случае распределения Лапласа$\dotfill$&1&18\\
\textbf{Бондаренко~А.\,В.} см.~Каратеев~С.\,Л.&&\\
\hangindent=23pt\noindent\textbf{Бородина~А.\,В., Морозов~Е.\,В.} Об оценивании асимптотики вероятности 
большого\linebreak
\vspace*{-12pt}\\
\hspace*{23pt}уклонения стационарной регенеративной очереди с одним прибором$\dotfill$&3&29\\
\hangindent=23pt\noindent\textbf{Бунтман~Н.\,В., Минель~Ж.-Л., Ле~Пезан~Д., Зацман~И.\,М.} Типология и 
компьютерное\linebreak
\vspace*{-12pt}\\
\hspace*{23pt}моделирование трудностей перевода$\dotfill$&3&77\\
\textbf{Визильтер~Ю.\,В.} см.~Каратеев~С.\,Л.&&\\
\hangindent=23pt\noindent\textbf{Гавриленко~С.\,В.} Оценки скорости сходимости распределений случайных сумм с 
безгранично делимыми индексами к нормальному закону$\dotfill$&4&81\\
\hangindent=23pt\noindent\textbf{Григорьева~М.\,Е., Шевцова~И.\,Г.} Уточнение неравенства 
Каца--Берри--Эссеена$\dotfill$&2&75\\
\hangindent=23pt\noindent\textbf{Грушо~А.\,А., Грушо~Н.\,А., Тимонина~Е.\,Е.} Поиск конфликтов в политиках 
безопасности: модель случайных графов$\dotfill$&3&38\\
\textbf{Грушо~Н.\,А.} см.~Грушо~А.\,А.&&\\
\hangindent=23pt\noindent\textbf{Гудков~В.\,Ю.} Математические модели изображения отпечатка пальца на основе 
описания линий$\dotfill$&1&58\\
\textbf{Гуртов~А.\,В.} см.~Лукьяненко~А.\,С.&&\\
\textbf{Желтов~С.\,Ю.} см.~Каратеев~С.\,Л.&&\\
\hangindent=23pt\noindent\textbf{Захаров~А.\,А., Серебряков~В.\,А.} Система управления электронной библиотекой 
LibMeta$\dotfill$&4&2\\
\textbf{Захаров~В.\,Н.} см.~Баранов~С.\,И.&&\\
\textbf{Захарова~Т.\,В.} см.~Матвеева~С.\,С.&&\\
\hangindent=23pt\noindent\textbf{Зацаринный~А.\,А., Чупраков~К.\,Г.} Некоторые аспекты выбора технологии для 
постро-\linebreak
\vspace*{-12pt}\\
\hspace*{23pt}ения систем отображения информации ситуационного центра$\dotfill$&3&59\\
\textbf{Зацман~И.\,М.} см.~Бунтман~Н.\,В.&&\\
\hangindent=23pt\noindent\textbf{Зейфман~А.\,И., Коротышева~А.\,В., Сатин~Я.\,А., Шоргин~С.\,Я.} Об 
устойчивости нестаци-\linebreak
\vspace*{-12pt}\\
\hspace*{23pt}онарных систем обслуживания с катастрофами$\dotfill$&3&9\\
\textbf{Зыкова~З.\,П.} см.~Архипов~О.\,П.&&\\
\hangindent=23pt\noindent\textbf{Илюшин~Г.\,Я., Соколов~И.\,А.} Организация управляемого доступа пользователей 
к\linebreak
\vspace*{-12pt}\\
\hspace*{23pt}разнородным ведомственным информационным ресурсам$\dotfill$&1&24\\
\hangindent=23pt\noindent\textbf{Кавагучи~Ю., Ульянов~В.\,В., Фуджикоши~Я.} Приближения для статистик, 
описывающих\linebreak
\vspace*{-12pt}\\
\hspace*{23pt}геометрические свойства данных большой размерности, с оценками 
ошибок$\dotfill$&1&12\\
\hangindent=23pt\noindent\textbf{Каратеев~С.\,Л., Бекетова~И.\,В., Ососков~М.\,В., Князь~В.\,А., 
Визильтер~Ю.\,В., Бондаренко~А.\,В., Желтов~С.\,Ю.} Автоматизированный контроль 
качества цифровых\linebreak
\vspace*{-12pt}\\
\hspace*{23pt}изображений для персональных документов$\dotfill$&1&65\\
\end{tabular}
}

\pagebreak

\def\leftkol{АВТОРСКИЙ УКАЗАТЕЛЬ ЗА 2010 г.} % ENGLISH ABSTRACTS}

\def\rightkol{АВТОРСКИЙ УКАЗАТЕЛЬ ЗА 2010 г.} %ENGLISH ABSTRACTS}

{\tabcolsep=3pt
\begin{tabular}{p{388pt}rr}
&\textbf{Выпуск} & \textbf{Стр.}\\[3pt]
\hangindent=23pt\noindent\textbf{Козеренко~Е.\,Б.} Лингвистические фильтры в статистических моделях машинного\linebreak
\vspace*{-12pt}\\
\hspace*{23pt}перевода$\dotfill$&2&83\\
\hangindent=23pt\noindent\textbf{Козеренко~Е.\,Б., Кузнецов~И.\,П.} Когнитивно-лингвистические представления в 
систе-\linebreak
\vspace*{-12pt}\\
\hspace*{23pt}мах обработки текстов$\dotfill$&3&69\\
\textbf{Князь~В.\,А.} см.~Каратеев~С.\,Л.&&\\
\hangindent=23pt\noindent\textbf{Колесников~А.\,В., Солдатов~С.\,А.} Алгоритм координации для гибридной 
интеллектуальной системы решения сложной задачи оперативно-производственного\linebreak
\vspace*{-12pt}\\
\hspace*{23pt}планирования$\dotfill$&4&61\\
\hangindent=23pt\noindent\textbf{Коновалов~М.\,Г.} О планировании потоков в системах вычислительных 
ресурсов$\dotfill$&2&3\\
\textbf{Конушин~А.\,С.} см.~Конушин~В.\,С.&&\\
\hangindent=23pt\noindent\textbf{Конушин~В.\,С., Кривовязь~Г.\,Р., Конушин~А.\,С.} Алгоритм распознавания людей 
в видео-\linebreak
\vspace*{-12pt}\\
\hspace*{23pt}последовательности по одежде$\dotfill$&1&74\\
\textbf{Корепанов~Э.\, Р.} см.~Синицын~И.\,Н.&&\\
\textbf{Королев~В.\,Ю.} см.~Соколов~И.\,А.&&\\
\textbf{Королев~Р.\,А.} см.~Бенинг~В.\,Е.&&\\
\textbf{Коротышева~А.\,В.} см.~Зейфман~А.\,И.&&\\
\hangindent=23pt\noindent\textbf{Кривенко~М.\,П.} Непараметрическое оценивание элементов байесовского 
клас\-си-\linebreak
\vspace*{-12pt}\\
\hspace*{23pt}фикатора$\dotfill$&2&13\\
\textbf{Кривовязь~Г.\,Р.} см.~Конушин~В.\,С.&&\\
\textbf{Крылов~А.\,С.} см.~Павельева~Е.\,А.&&\\
\hangindent=23pt\noindent\textbf{Крылов~В.\,А.} Моделирование и классификация многоканальных дистанционных\linebreak
\vspace*{-12pt}\\
\hspace*{23pt}изображений с использованием копул$\dotfill$&4&34\\
\hangindent=23pt\noindent\textbf{Крючин~О.\,В.} Разработка параллельных эвристических алгоритмов подбора 
весовых\linebreak
\vspace*{-12pt}\\
\hspace*{23pt}коэффициентов искусственной нейтронной сети$\dotfill$&2&53\\
\hangindent=23pt\noindent\textbf{Кудрявцев~А.\,А., Шоргин~С.\,Я.} Байесовские модели массового обслуживания и 
надеж-\linebreak
\vspace*{-12pt}\\
\hspace*{23pt}ности: характеристики среднего числа заявок в системе $M\vert M \vert 1\vert 
\infty$$\dotfill$&3&16\\
\hangindent=23pt\noindent\textbf{Кузнецов~А.\,А.} Связь между временными и структурно-топологическими 
характери-\linebreak
\vspace*{-12pt}\\
\hspace*{23pt}стиками диаграмм ритма сердца здоровых людей$\dotfill$&4&39\\
\textbf{Кузнецов~И.\,П.} см.~Козеренко~Е.\,Б.&&\\
\textbf{Ле~Пезан~Д.} см.~Бунтман~Н.\,В.&&\\
\hangindent=23pt\noindent\textbf{Лукьяненко~А.\,С., Морозов~Е.\,В., Гуртов~А.\,В.} Анализ сетевого протокола с общей 
функ-\linebreak
\vspace*{-12pt}\\
\hspace*{23pt}цией расширения окна передачи сообщения при конфликтах$\dotfill$&2&46\\
\hangindent=23pt\noindent\textbf{Лямин~О.\,О.} О предельном поведении мощностей критериев в случае обобщенного\linebreak
\vspace*{-12pt}\\
\hspace*{23pt}распределения Лапласа$\dotfill$&3&47\\
\hangindent=23pt\noindent\textbf{Маркин~А.\,В., Шестаков~О.\,В.} Асимптотики оценки риска при пороговой 
обработке\linebreak
\vspace*{-12pt}\\
\hspace*{23pt}вейвлет-вейглет коэффициентов в задаче томографии$\dotfill$&2&36\\
\hangindent=23pt\noindent\textbf{Матвеева~С.\,С., Захарова~Т.\,В.} Сети массового обслуживания с наименьшей 
длиной\linebreak
\vspace*{-12pt}\\
\hspace*{23pt}очереди$\dotfill$&3&22\\
\hangindent=23pt\noindent\textbf{Матюшенко~С.\,И.} Стационарные характеристики двухканальной системы 
обслужива-\linebreak
\vspace*{-12pt}\\
\hspace*{23pt}ния с переупорядочиванием заявок и распределениями фазового типа$\dotfill$&4&68\\
\textbf{Минель~Ж.-Л.} см.~Бунтман~Н.\,В.&&\\
\textbf{Морозов~Е.\,В.} см.~Бородина~А.\,В.&&\\
\textbf{Морозов~Е.\,В.} см.~Лукьяненко~А.\,С.&&\\
\textbf{Ососков~М.\,В.} см.~Каратеев~С.\,Л.&&\\
\hangindent=23pt\noindent\textbf{Павельева~Е.\,А., Крылов~А.\,С.} Поиск и анализ ключевых точек радужной 
оболочки\linebreak
\vspace*{-12pt}\\
\hspace*{23pt}глаза методом преобразования Эрмита$\dotfill$&1&79\\
\textbf{Печинкин~А.\,В.} см.~Френкель~С.\,Л.,&&\\
\hangindent=23pt\noindent\textbf{Протасов~В.\,И.} Составление субъективного портрета с использованием 
эволюционно-\linebreak
\vspace*{-12pt}\\
\hspace*{23pt}го морфинга и квалиметрия метода$\dotfill$&1&83\\
\hangindent=23pt\noindent\textbf{Рудаков~К.\,В., Торшин~И.\,Ю.} Вопросы разрешимости задачи распознавания 
вторичной\linebreak
\vspace*{-12pt}\\
\hspace*{23pt}структуры белка$\dotfill$&2&25\\
\textbf{Сатин~Я.\,А.} см.~Зейфман~А.\,И.&&\\
\hangindent=23pt\noindent\textbf{Сейфуль-Мулюков~Р.\,Б.} Нефть как носитель информации о своем 
происхождении,\linebreak
\vspace*{-12pt}\\
\hspace*{23pt}структуре и эволюции$\dotfill$&1&41\\
\end{tabular}
}

{\tabcolsep=3pt
\begin{tabular}{p{388pt}rr}
&\textbf{Выпуск} & \textbf{Стр.}\\[6pt]
\textbf{Семендяев~Н.\,Н.} см.~Синицын~И.\,Н.&&\\
\textbf{Серебряков~В.\,А.} см.~Захаров~А.\,А.&&\\
\textbf{Синицын~В.\,И.} см.~Синицын~И.\,Н.&&\\
\hangindent=23pt\noindent\textbf{Синицын~И.\,Н., Синицын~В.\,И., Корепанов~Э.\, Р., Белоусов~В.\,В., 
Семендяев~Н.\,Н.} Оперативное построение информационных моделей движения полюса 
Земли\linebreak
\vspace*{-12pt}\\
\hspace*{23pt}методами линейных и линеаризованных фильтров$\dotfill$&1&2\\
\textbf{Сипина~А.\,В.} см.~Бенинг~В.\,Е.&&\\
\hangindent=23pt\noindent\textbf{Соколов~И.\,А.} О работах заслуженного деятеля науки Российской Федерации 
И.\,Н.~Синицына в области информационных технологий и автоматизации (к 70-летию\linebreak
\vspace*{-12pt}\\
\hspace*{23pt}со дня рождения)$\dotfill$&3&84\\
\textbf{Соколов~И.\,А.} см.~Илюшин~Г.\,Я.&&\\
\hangindent=23pt\noindent\textbf{Соколов~И.\,А., Королев~В.\,Ю.} Предисловие$\dotfill$&2&2\\
\textbf{Солдатов~С.\,А.} см.~Колесников~А.\,В.&&\\
\hangindent=23pt\noindent\textbf{Степанов~С.\,Ю.} Использование координатного метода фрагментации 
коммутаторной\linebreak
\vspace*{-12pt}\\
\hspace*{23pt}нейронной сети для сокращения трафика$\dotfill$&2&57\\
\textbf{Тимонина~Е.\,Е.} см.~Грушо~А.\,А.&&\\
\textbf{Торшин~И.\,Ю.} см.~Рудаков~К.\,В.&&\\
\textbf{Ульянов~В.\,В.} см.~Кавагучи~Ю.&&\\
\textbf{Фазекаш~И.} см.~Чупрунов~А.\,Н.&&\\
\textbf{Френкель~С.\,Л.} см.~Баранов~С.\,И.&&\\
\hangindent=23pt\noindent\textbf{Френкель~С.\,Л., Печинкин~А.\,В.} Оценка времени самовосстановления в 
цифровых\linebreak
\vspace*{-12pt}\\
\hspace*{23pt}системах после сбоев, вызываемых переходными помехами$\dotfill$&3&2\\
\textbf{Фуджикоши~Я.} см.~Кавагучи~Ю.&&\\
\hangindent=23pt\noindent\textbf{Цискаридзе~А.\,К.} Математическая модель и метод восстановления позы человека 
по\linebreak
\vspace*{-12pt}\\
\hspace*{23pt}стереопаре силуэтных изображений$\dotfill$&4&27\\
\hangindent=23pt\noindent\textbf{Чупраков~К.\,Г.} К вопросу о размещении коллективных средств отображения в 
ситуа-\linebreak
\vspace*{-12pt}\\
\hspace*{23pt}ционном зале с заданными параметрами$\dotfill$&4&89\\
\textbf{Чупраков~К.\,Г.} см.~Зацаринный~А.\,А.&&\\
\hangindent=23pt\noindent\textbf{Чупрунов~А.\,Н., Фазекаш~И.} Законы повторного логарифма для числа 
безошибочных\linebreak
\vspace*{-12pt}\\
\hspace*{23pt}блоков при помехоустойчивом кодировании$\dotfill$&3&42\\
\textbf{Шевцова~И.\,Г.} см.~Григорьева~М.\,Е.&&\\
\hangindent=23pt\noindent\textbf{Шестаков~О.\,В.} Аппроксимация распределения оценки риска пороговой 
обработки вейвлет-коэффициентов нормальным распределением при использовании 
выбо-\linebreak
\vspace*{-12pt}\\
\hspace*{23pt}рочной дисперсии$\dotfill$&4&73\\
\textbf{Шестаков~О.\,В.} см.~Маркин~А.\,В.&&\\
\textbf{Шоргин~С.\,Я.} см.~Зейфман~А.\,И.&&\\
\textbf{Шоргин~С.\,Я.} см.~Кудрявцев~А.\,А.&&\\
\end{tabular}
}

%\thispagestyle{myheadings}
\def\leftfootline{\small{\textbf{\thepage}
\hfill ИНФОРМАТИКА И ЕЁ ПРИМЕНЕНИЯ\ \ \ том~4\ \ \ выпуск~4\ \ \ 2010}
}%
 \def\rightfootline{\small{ИНФОРМАТИКА И ЕЁ ПРИМЕНЕНИЯ\ \ \ том~4\ \ \ выпуск~4\ \ \ 2010
 \hfill \textbf{\thepage}}}
 \label{end\stat}

%
%Том 10 Выпуск 1-4 Год 2016

\def\stat{cont-e}
{%\hrule\par
%\vskip 7pt % 7pt
\raggedleft\Large \bf%\baselineskip=3.2ex
2\,0\,1\,6\ \ A\,U\,T\,H\,O\,R\ \ I\,N\,D\,E\,X \vskip 17pt
 \hrule
 \par
\vskip 21pt plus 6pt minus 3pt }

\label{st\stat}

\def\tit{\ }

\def\aut{\ }
\def\auf{\ }

\def\leftkol{\ } %2016 AUTHOR INDEX} % ENGLISH ABSTRACTS}

\def\rightkol{\ } %2016 AUTHOR INDEX} %ENGLISH ABSTRACTS}

\titele{\tit}{\aut}{\auf}{\leftkol}{\rightkol}

\def\leftfootline{\small{\textbf{\thepage}
\hfill INFORMATIKA I EE PRIMENENIYA~--- INFORMATICS AND APPLICATIONS\ \ \ 2016\
\ \ volume~10\ \ \ issue\ 4}
}%
 \def\rightfootline{\small{INFORMATIKA I EE PRIMENENIYA~--- INFORMATICS AND APPLICATIONS\ \ \ 2016\ \ \ volume~10\ \ \ issue\ 4
\hfill \textbf{\thepage}}}

\vspace*{-12pt}
\vspace*{-18pt}

{\tabcolsep=2.8pt
\begin{tabular}{p{382pt}cc}
&\textbf{Issue} & \textbf{Page}\\[6pt]
\Avtors{Agalarov~M.\,Ya.} see~Agalarov~Ya.\,M.&&\\
\Avtors{Agalarov~Ya.\,M., Agalarov~M.\,Ya., and
Shorgin~V.\,S.} About the optimal threshold of queue\linebreak
\\[-12pt]
\hspace*{23pt}length in a~particular problem of profit maximization
in the $M/G/1$ queuing system&2&70--79\\
\Avtors{Alexeyevsky~D.\,A.} BioNLP ontology extraction from 
a~restricted language corpus with\linebreak
\\[-12pt]
\hspace*{23pt}context-free grammars&1&119--128\\
\Avtors{Andreev~S.\,D.} see~Gaidamaka~Yu.\,V.&&\\
\Avtors{Andreev~S.\,D.} see~Ometov~A.\,Ya.&&\\
\Avtors{Arkhipov~O.\,P., Arkhipov~P.\,O., and Sidorkin~I.\,I.} The
option to create a~local coordinate\linebreak
\\[-12pt]
\hspace*{23pt}system for synchronization of selected images&3&91--97\\
\Avtors{Arkhipov~P.\,O.} see~Arkhipov~O.\,P.&&\\
\Avtors{Belousov~V.\,V.} see~Shnurkov~P.\,V.&&\\
\Avtors{Belousov~V.\,V.} see~Shnurkov~P.\,V.&&\\
\Avtors{Bening~V.\,E.} Calculation of~the~asymptotic deficiency
of~some statistical procedures based\linebreak
\\[-12pt]
\hspace*{23pt}on~samples with~random sizes&4&34--45\\
\Avtors{Borisov~A.\,V., Bosov~A.\,V., and Miller~G.\,B.} Modeling and
monitoring of VoIP connection&2&\hphantom{1}2--13\\
\Avtors{Bosov~A.\,V.} see~Borisov~A.\,V.&&\\
\Avtors{Briukhov~D.\,O.} see~Stupnikov~S.\,A.&&\\
\Avtors{Callaos~N.\,K.\ and Seyful-Mulyukov~R.\,B.} Complexity and
its information content&1&129--139\\
\Avtors{Chertok~A.\,V., Kadaner~A.\,I., Khazeeva~G.\,T., and
Sokolov~I.\,A.} Regime switching detection\linebreak
\\[-12pt]
\hspace*{23pt}for~the~Levy driven
Ornstein--Uhlenbeck process using CUSUM methods&4&46--56\\
\Avtors{Chichagov~V.\,V.} Asymptotic expansions of mean absolute
error of uniformly minimum variance unbiased and maximum likelihood
estimators on the one-parameter exponential\linebreak
\\[-12pt]
\hspace*{23pt}family model of lattice distributions&3&66--76\\
\Avtors{Danishevsky~V.\,I.} see~Kolesnikov A.\,V.&&\\
\Avtors{Fazliev~A.\,Z.} see~Kalinichenko~L.\,A.&&\\
\Avtors{Fedoseev~A.\,A.} What is behind the concept of ``knowledge in
small packages''&3&105--110\\
\Avtors{Gaidamaka~Yu.\,V., Andreev~S.\,D., Sopin~E.\,S.,
Samouylov~K.\,E., and Shorgin~S.\,Ya.} Interference analysis
of~the~device-to-device communications model with~regard to~a~signal\linebreak
\\[-12pt]
\hspace*{23pt}propagation environment&4&\hphantom{1}2--10\\
\Avtors{Gasilov~A.\,V.} see~Yakovlev~O.\,A.&&\\
\Avtors{Goncharov~A.\,V.\ and Strijov~V.\,V.} Metric time series
classification using weighted dynamic\linebreak
\\[-12pt]
\hspace*{23pt}warping relative to centroids of classes&2&36--47\\
\Avtors{Gordov~E.\,P.} see~Kalinichenko~L.\,A.&&\\
\Avtors{Gorshenin~A.\,K.} Concept of online service for stochastic
modeling of real processes&1&72--81\\
\Avtors{Gorshenin~A.\,K.} see~Shnurkov~P.\,V.&&\\
\Avtors{Gorshenin~A.\,K.} see~Shnurkov~P.\,V.&&\\
\Avtors{Grusho~A.\,A., Grusho~N.\,A., Zabezhailo~M.\,I., and
Timonina~E.\,E.} Integration of statistical and\linebreak
\\[-12pt]
\hspace*{23pt}deterministic methods for
analysis of information security&3&2--8\\
\Avtors{Grusho~A.\,A., Zabezhailo~M.\,I., and Zatsarinny~A.\,A.} On
the advanced procedure to reduce\linebreak
\\[-12pt]
\hspace*{23pt}calculation of Galois closures&4&\hphantom{1}96--104\\
\Avtors{Grusho~N.\,A.} see~Grusho~A.\,A.&&\\
\Avtors{Havanskov~V.\,A.} see~Minin~V.\,A.&&\\
\Avtors{Inkova~O.\,Yu.} see~Zatsman~I.\,M.&&\\
\Avtors{Isachenko~R.\,V.\ and Strijov~V.\,V.} Metric learning in
multiclass time series classification\linebreak
\\[-12pt]
\hspace*{23pt}problem&2&48--57\\
\end{tabular}
}
\pagebreak

\def\leftfootline{\small{\textbf{\thepage}
\hfill INFORMATIKA I EE PRIMENENIYA~--- INFORMATICS AND APPLICATIONS\ \ \ 2016\
\ \ volume~10\ \ \ issue\ 4}
}%
 \def\rightfootline{\small{INFORMATIKA I EE PRIMENENIYA~---
INFORMATICS AND APPLICATIONS\ \ \ 2016\ \ \ volume~10\ \ \ issue\ 4
\hfill \textbf{\thepage}}}

\def\leftkol{2016 AUTHOR INDEX} % ENGLISH ABSTRACTS}

\def\rightkol{2016 AUTHOR INDEX} %ENGLISH ABSTRACTS}


{\tabcolsep=2.83pt
\begin{tabular}{p{382pt}cc}
&\textbf{Issue} & \textbf{Page}\\[6pt]
\Avtors{Kadaner~A.\,I.} see~Chertok~A.\,V.&&\\[.255pt]
\Avtors{Kalinichenko~L.\,A., Volnova~A.\,A., Gordov~E.\,P.,
Kiselyova~N.\,N., Kovaleva~D.\,A., Malkov~O.\,Yu., Okladnikov~I.\,G.,
Podkolodnyy~N.\,L., Pozanenko~A.\,S., Ponomareva~N.\,V.,
Stupnikov~S.\,A.,} \textbf{and Fazliev~A.\,Z.} Data access challenges for data
intensive\linebreak
\\[-12pt]
\hspace*{23pt}research in Russia&1& 2--22\\[.255pt]
\Avtors{Karasikov~M.\,E.\ and Strijov~V.\,V.} Feature-based
time-series classification&4&121--131\\[.255pt]
\Avtors{Khazeeva~G.\,T.} see~Chertok~A.\,V.&&\\[.255pt]
\Avtors{Khokhlov~Yu.\,S.} Multivariate fractional Levy motion and its
applications&2&\hphantom{1}98--106\\[.255pt]
\Avtors{Kirikov~I.\,A., Kolesnikov~A.\,V., Listopad~S.\,V., and
Rumovskaya~S.\,B.} Fine-grained hybrid\linebreak
\\[-12pt]
\hspace*{23pt}intelligent systems. Part 2:
Bidirectional hybridization&1&\hphantom{1}96--105\\[.255pt]
\Avtors{Kirikov~I.\,A., Kolesnikov~A.\,V., Listopad~S.\,V., and
Rumovskaya~S.\,B.} ``Virtual council''~---\linebreak
\\[-12pt]
\hspace*{23pt}source environment
supporting complex diagnostic decision making&3&81--90\\[.255pt]
\Avtors{Kiselyova~N.\,N.} see~Kalinichenko~L.\,A.&&\\[.255pt]
\Avtors{Kolesnikov A.\,V., Listopad~S.\,V., Rumovskaya~S.\,B., and
Danishevsky~V.\,I.} Informal axiomatic\linebreak
\\[-12pt]
\hspace*{23pt}theory of~the~role visual models&4&114--120\\[.255pt]
\Avtors{Kolesnikov~A.\,V.} see~Kirikov~I.\,A.&&\\[.255pt]
\Avtors{Kolesnikov~A.\,V.} see~Kirikov~I.\,A.&&\\[.255pt]
\Avtors{Kolin~K.\,K.} Humanitarian aspects of information
security&3&111--121\\[.255pt]
\Avtors{Konovalov~M.\,G.\ and Razumchik~R.\,V.} Dispatching
to~two parallel nonobservable queues using\linebreak
\\[-12pt]
\hspace*{23pt}only static
information&4&57--67\\[.255pt]
\Avtors{Korchagin~A.\,Yu.} see~Korolev~V.\,Yu.&&\\[.255pt]
\Avtors{Korchagin~A.\,Yu.} see~Korolev~V.\,Yu.&&\\[.255pt]
\Avtors{Korepanov~E.\,R.} see~Sinitsyn~I.\,N.&&\\[.255pt]
\Avtors{Korepanov~E.\,R.} see~Sinitsyn~I.\,N.&&\\[.255pt]
\Avtors{Korolev~V.\,Yu., Korchagin~A.\,Yu., and Zeifman~A.\,I.} The
Poisson theorem for Bernoulli trials\linebreak
\\[-12pt]
\hspace*{23pt}with~a~random probability
of~success and~a~discrete analog of~the~Weibull distribution&4&11--20\\[.255pt]
\Avtors{Korolev~V.\,Yu., Zeifman~A.\,I., and Korchagin~A.\,Yu.}
Asymmetric Linnik distributions as~limit\linebreak
\\[-12pt]
\hspace*{23pt}laws for~random sums
of~independent random variables with~finite variances&4&21--33\\[.255pt]
\Avtors{Koucheryavy~E.\,A.} see~Ometov~A.\,Ya.&&\\[.255pt]
\Avtors{Kovaleva~D.\,A.} see~Kalinichenko~L.\,A.&&\\[.255pt]
\Avtors{Kovalyov~S.\,P.} Metaprogramming to increase
manufacturability of large-scale software-\linebreak
\\[-12pt]
\hspace*{23pt}intensive systems&1&56--66\\[.255pt]
\Avtors{Krivenko~M.\,P.} Significance tests of feature selection for
classification&3&32--40\\[.255pt]
\Avtors{Kruzhkov~M.\,G.} see~Zalizniak~Anna~A.&&\\[.255pt]
\Avtors{Kruzhkov~M.\,G.} see~Zatsman~I.\,M.&&\\[.255pt]
\Avtors{Kudryavtsev~A.\,A.} Bayesian queueing and reliability models:
\textit{A~priori} distributions with\linebreak
\\[-12pt]
\hspace*{23pt}compact support&1&67--71\\[.255pt]
\Avtors{Kudryavtsev~A.\,A.} Characteristics dependent on the balance
coefficient in Bayesian models\linebreak
\\[-12pt]
\hspace*{23pt}with compact support of \textit{a priori}
distributions&3&77--80\\[.255pt]
\Avtors{Kudryavtsev~A.\,A.\ and Palionnaia~S.\,I.} Bayesian recurrent
model of reliability growth:\linebreak
\\[-12pt]
\hspace*{23pt}Parabolic distribution of parameters&2&80--83\\[.255pt]
\Avtors{Kudryavtsev~A.\,A.\ and Titova~A.\,I.} Bayesian queuing
and~reliability models: Degenerate-\linebreak
\\[-12pt]
\hspace*{23pt}Weibull case&4&68--71\\[.255pt]
\Avtors{Leontyev~N.\,D.\ and Ushakov~V.\,G.} Analysis of a queueing
system with autoregressive arrivals\linebreak
\\[-12pt]
\hspace*{23pt}and nonpreemptive priority&3&15--22\\[.255pt]
\Avtors{Listopad~S.\,V.} see~Kirikov~I.\,A.&&\\[.255pt]
\Avtors{Listopad~S.\,V.} see~Kirikov~I.\,A.&&\\[.255pt]
\Avtors{Listopad~S.\,V.} see~Kolesnikov A.\,V.&&\\[.255pt]
\Avtors{Malkov~O.\,Yu.} see~Kalinichenko~L.\,A.&&\\[.255pt]
\Avtors{Markov~A.\,S., Monakhov~M.\,M., and
Ulyanov~V.\,V.} Generalized Cornish--Fisher expansions\linebreak
\\[-12pt]
\hspace*{23pt}for distributions of statistics based on samples
of random size&2&84--91\\[.255pt]
\Avtors{Melnikov~A.\,K.\ and Ronzhin~A.\,F.} Generalized statistical
method of~text analysis based\linebreak
\\[-12pt]
\hspace*{23pt}on~calculation of~probability distributions
of~statistical values&4&89--95\\
\end{tabular}
}
\pagebreak

\def\leftfootline{\small{\textbf{\thepage}
\hfill INFORMATIKA I EE PRIMENENIYA~--- INFORMATICS AND APPLICATIONS\ \ \ 2016\
\ \ volume~10\ \ \ issue\ 4}
}%
 \def\rightfootline{\small{INFORMATIKA I EE PRIMENENIYA~---
INFORMATICS AND APPLICATIONS\ \ \ 2016\ \ \ volume~10\ \ \ issue\ 4
\hfill \textbf{\thepage}}}

\def\leftkol{2016 AUTHOR INDEX} % ENGLISH ABSTRACTS}

\def\rightkol{2016 AUTHOR INDEX} %ENGLISH ABSTRACTS}


{\tabcolsep=3pt
\begin{tabular}{p{381pt}cc}
&\textbf{Issue} & \textbf{Page}\\[6pt]
\Avtors{Meykhanadzhyan~L.\,A.} Stationary characteristics of the finite
capacity queueing system with\linebreak
\\[-12pt]
\hspace*{23pt}inverse service order and generalized
probabilistic priority&2&123--131\\[.23pt]
\Avtors{Miller~G.\,B.} see~Borisov~A.\,V.&&\\[.23pt]
\Avtors{Minin~V.\,A., Zatsman~I.\,M., Havanskov~V.\,A., and
Shubnikov~S.\,K.} Intensity of citation of scientific publications in
inventions on information and computer technologies patented\linebreak
\\[-12pt]
\hspace*{23pt}in Russia by domestic and foreign applicants&2&107--122\\[.23pt]
\Avtors{Monakhov~M.\,M.} see~Markov~A.\,S.&&\\[.23pt]
\Avtors{Naumov~V.\,A.\ and Samouylov~K.\,E.} On relationship
between queuing systems with resources\linebreak
\\[-12pt]
\hspace*{23pt}and Erlang networks&3&\hphantom{1}9--14\\[.23pt]
\Avtors{Okladnikov~I.\,G.} see~Kalinichenko~L.\,A.&&\\[.23pt]
\Avtors{Ometov~A.\,Ya., Andreev~S.\,D., Turlikov~A.\,M., and
Koucheryavy~E.\,A.} Performance analysis of\linebreak
\\[-12pt]
\hspace*{23pt}a wireless data
aggregation system with contention for contemporary sensor
networks&3&23--31\\[.23pt]
\Avtors{Palionnaia~S.\,I.} see~Kudryavtsev~A.\,A.&&\\[.23pt]
\Avtors{Podkolodnyy~N.\,L.} see~Kalinichenko~L.\,A.&&\\[.23pt]
\Avtors{Ponomareva~N.\,V.} see~Kalinichenko~L.\,A.&&\\[.23pt]
\Avtors{Popkova~N.\,A.} see~Zatsman~I.\,M.&&\\[.23pt]
\Avtors{Pozanenko~A.\,S.} see~Kalinichenko~L.\,A.&&\\[.23pt]
\Avtors{Razumchik~R.\,V.} see~Konovalov~M.\,G.&&\\[.23pt]
\Avtors{Ronzhin~A.\,F.} see~Melnikov~A.\,K.&&\\[.23pt]
\Avtors{Rumovskaya~S.\,B.} see~Kirikov~I.\,A.&&\\[.23pt]
\Avtors{Rumovskaya~S.\,B.} see~Kirikov~I.\,A.&&\\[.23pt]
\Avtors{Rumovskaya~S.\,B.} see~Kolesnikov A.\,V.&&\\[.23pt]
\Avtors{Samouylov~K.\,E.} see~Gaidamaka~Yu.\,V.&&\\[.23pt]
\Avtors{Samouylov~K.\,E.} see~Naumov~V.\,A.&&\\[.23pt]
\Avtors{Serebryanskii~S.\,M.} see~Tyrsin~A.\,N.&&\\[.23pt]
\Avtors{Seyful-Mulyukov~R.\,B.} see~Callaos~N.\,K.&&\\[.23pt]
\Avtors{Shestakov~O.\,V.} Statistical properties of the denoising method
based on the stabilized hard\linebreak
\\[-12pt]
\hspace*{23pt}thresholding&2&65--69\\[.23pt]
\Avtors{Shestakov~O.\,V.} The strong law of large numbers for the risk
estimate in the problem of\linebreak
\\[-12pt]
\hspace*{23pt}tomographic image reconstruction from
projections with a correlated noise&3&41--45\\[.23pt]
\Avtors{Shestakov~O.\,V.} see~Zakharova~T.\,V.&&\\[.23pt]
\Avtors{Shnurkov~P.\,V., Gorshenin~A.\,K., and Belousov~V.\,V.}
Analytical solution of~the~optimal control\linebreak
\\[-12pt]
\hspace*{23pt}task of~a~semi-Markov
process with~finite set of~states&4&72--88\\[.23pt]
\Avtors{Shnurkov~P.\,V., Zasypko~V.\,V., Belousov~V.\,V., and
Gorshenin~A.\,K.} Development of the algorithm of numerical solution
of the optimal investment control problem\linebreak
\\[-12pt]
\hspace*{23pt}in the closed dynamical model of three-sector economy&1&82--95\\[.23pt]
\Avtors{Shorgin~S.\,Ya.} see~Gaidamaka~Yu.\,V.&&\\[.23pt]
\Avtors{Shorgin~V.\,S.} see~Agalarov~Ya.\,M.&&\\[.23pt]
\Avtors{Shubnikov~S.\,K.} see~Minin~V.\,A.&&\\[.23pt]
\Avtors{Sidorkin~I.\,I.} see~Arkhipov~O.\,P.&&\\[.23pt]
\Avtors{Sinitsyn~I.\,N.} Analytical modeling of processes in stochastic
systems with complex fractional\linebreak
\\[-12pt]
\hspace*{23pt}order Bessel nonlinearities&3&55--65\\[.23pt]
\Avtors{Sinitsyn~I.\,N.} Orthogonal supoptimal filters for nonlinear
stochastic systems on manifolds&1&34--44\\[.23pt]
\Avtors{Sinitsyn~I.\,N.\ and Korepanov~E.\,R.} Normal Pugachev
conditionally-optimal filters and extra-\linebreak
\\[-12pt]
\hspace*{23pt}polators for state linear stochastic systems&2&14--23\\[.23pt]
\Avtors{Sinitsyn~I.\,N.\ and Sinitsyn~V.\,I.} Analytical modeling of
distributions in stochastic systems on\linebreak
\\[-12pt]
\hspace*{23pt}manifolds based on ellipsoidal approximation&1&45--55\\[.23pt]
\Avtors{Sinitsyn~I.\,N., Sinitsyn~V.\,I., and
Korepanov~E.\,R.} Ellipsoidal suboptimal filters for nonlinear\linebreak
\\[-12pt]
\hspace*{23pt}stochastic systems on manifolds&2&24--35\\[.23pt]
\Avtors{Sinitsyn~V.\,I.} see~Sinitsyn~I.\,N.&&\\[.23pt]
\Avtors{Sinitsyn~V.\,I.} see~Sinitsyn~I.\,N.&&\\[.23pt]
\Avtors{Skvortsov~N.\,A.} see~Stupnikov~S.\,A.&&\\[.23pt]
\Avtors{Sokolov~I.\,A.} see~Chertok~A.\,V.&&\\
\end{tabular}
}
\pagebreak

\def\leftfootline{\small{\textbf{\thepage}
\hfill INFORMATIKA I EE PRIMENENIYA~--- INFORMATICS AND APPLICATIONS\ \ \ 2016\
\ \ volume~10\ \ \ issue\ 4}
}%
 \def\rightfootline{\small{INFORMATIKA I EE PRIMENENIYA~---
INFORMATICS AND APPLICATIONS\ \ \ 2016\ \ \ volume~10\ \ \ issue\ 4
\hfill \textbf{\thepage}}}

\def\leftkol{2016 AUTHOR INDEX} % ENGLISH ABSTRACTS}

\def\rightkol{2016 AUTHOR INDEX} %ENGLISH ABSTRACTS}


{\tabcolsep=3pt
\begin{tabular}{p{382pt}cc}
&\textbf{Issue} & \textbf{Page}\\[6pt]
\Avtors{Sopin~E.\,S.} see~Gaidamaka~Yu.\,V.&&\\
\Avtors{Strijov~V.\,V.} see~Goncharov~A.\,V.&&\\
\Avtors{Strijov~V.\,V.} see~Isachenko~R.\,V.&&\\
\Avtors{Strijov~V.\,V.} see~Karasikov~M.\,E.&&\\
\Avtors{Stupnikov~S.\,A., Briukhov~D.\,O., and Skvortsov~N.\,A.}
Co-lending systemic risk analysis over\linebreak
\\[-12pt]
\hspace*{23pt}heterogeneous data collections&1&23--33\\
\Avtors{Stupnikov~S.\,A.} see~Kalinichenko~L.\,A.&&\\
\Avtors{Suchkov~A.\,P.} see~Zatsarinny~A.\,A.&&\\
\Avtors{Timonina~E.\,E.} see~Grusho~A.\,A.&&\\
\Avtors{Titova~A.\,I.} see~Kudryavtsev~A.\,A.&&\\
\Avtors{Turlikov~A.\,M.} see~Ometov~A.\,Ya.&&\\
\Avtors{Tyrsin~A.\,N.\ and Serebryanskii~S.\,M.} Recognition of
dependences on the basis of inverse\linebreak
\\[-12pt]
\hspace*{23pt}mapping&2&58--64\\
\Avtors{Ulyanov~V.\,V.} see~Markov~A.\,S.&&\\
\Avtors{Ushakov~V.\,G.} Queueing system with working vacations and
hyperexponential input stream&2&92--97\\
\Avtors{Ushakov~V.\,G.} see~Leontyev~N.\,D.&&\\
\Avtors{Volnova~A.\,A.} see~Kalinichenko~L.\,A.&&\\
\Avtors{Yakovlev~O.\,A.\ and Gasilov~A.\,V.} Speeded-up stereo
matching using geodesic support weights&3&\hphantom{1}98--104\\
\Avtors{Zabezhailo~M.\,I.} see~Grusho~A.\,A.&&\\
\Avtors{Zabezhailo~M.\,I.} see~Grusho~A.\,A.&&\\
\Avtors{Zakharova~T.\,V.\ and Shestakov~O.\,V.} Precision analysis of
wavelet processing of aerodynamic\linebreak
\\[-12pt]
\hspace*{23pt}flow patterns&3&46--54\\
\Avtors{Zalizniak~Anna~A.\ and Kruzhkov~M.\,G.} Database
of~Russian impersonal verbal constructions&4&132--141\\
\Avtors{Zasypko~V.\,V.} see~Shnurkov~P.\,V.&&\\
\Avtors{Zatsarinny~A.\,A.\ and Suchkov~A.\,P.} Systems engineering
approaches to~the~establishment of\linebreak
\\[-12pt]
\hspace*{23pt}a~system for~decision support based
on~situational analysis&4&105--113\\
\Avtors{Zatsarinny~A.\,A.} see~Grusho~A.\,A.&&\\
\Avtors{Zatsman~I.\,M., Inkova~O.\,Yu., Kruzhkov~M.\,G., and
Popkova~N.\,A.} Representation of cross-\linebreak
\\[-12pt]
\hspace*{23pt}lingual knowledge about
connectors in supracorpora databases&1&106--118\\
\Avtors{Zatsman~I.\,M.} see~Minin~V.\,A.&&\\
\Avtors{Zeifman~A.\,I.} see~Korolev~V.\,Yu.&&\\
\Avtors{Zeifman~A.\,I.} see~Korolev~V.\,Yu.&&\\
\end{tabular}
}

%\thispagestyle{myheadings}
\def\leftfootline{\small{\textbf{\thepage}
\hfill INFORMATIKA I EE PRIMENENIYA~--- INFORMATICS AND APPLICATIONS\ \ \ 2016\
\ \ volume~10\ \ \ issue\ 4}
}%
 \def\rightfootline{\small{INFORMATIKA I EE PRIMENENIYA~---
INFORMATICS AND APPLICATIONS\ \ \ 2016\ \ \ volume~10\ \ \ issue\ 4
\hfill \textbf{\thepage}}}

 \label{end\stat}

\newpage


%\vspace*{-60pt} {\small
{\baselineskip=9.1pt
\section*{Правила подготовки рукописей статей для публикации в журнале
<<Информатика и её применения>>}

\thispagestyle{empty}

 Журнал <<Информатика и её применения>> публикует
теоретические, обзорные и дискуссионные статьи, посвященные научным
исследованиям и разработкам в области информатики и ее приложений. Журнал
издается на русском языке. По специальному решению редколлегии отдельные статьи,
в виде исключения, могут печататься на английском языке.
Тематика журнала охватывает следующие направления:
\begin{itemize}
\item теоретические основы информатики; %\\[-13.5pt]
\item математические методы исследования сложных систем и процессов; %\\[-13.5pt]
\item информационные системы и сети; %\\[-13.5pt]
\item информационные технологии; %\\[-13.5pt]
\item архитектура и программное
обеспечение вычислительных комплексов и сетей.
\end{itemize}
\begin{enumerate}
\item В журнале печатаются результаты, ранее не
опубликованные и не предназначенные к одновременной публикации в других
изданиях. Публикация не должна нарушать закон об авторских правах. Направляя
свою рукопись в редакцию, авторы автоматически передают учредителям и
редколлегии неисключительные права на издание данной статьи на русском языке и
на ее распространение в России и за рубежом. При этом за авторами сохраняются
все права как собственников данной рукописи. В связи с этим авторами должно
быть представлено в редакцию письмо в следующей форме:
Соглашение о передаче права на публикацию:

\textit{<<Мы, нижеподписавшиеся, авторы рукописи <<$\qquad\qquad$>>, передаем
учредителям и редколлегии журнала <<Информатика и её применения>>
неисключительное право опубликовать данную рукопись статьи на русском языке как
в печатной, так и в электронной версиях журнала. Мы подтверждаем, что данная
публикация не нарушает авторского права других лиц или организаций. Подписи
авторов: (ф.\,и.\,о., дата, адрес)>>.}

Указанное соглашение может быть представлено 
как в бумажном виде, так и в виде отсканированной копии (с подписями авторов).


Редколлегия вправе запросить у авторов экспертное заключение о возможности
опубликования представленной статьи в открытой печати. %\\[-13.5pt]
\item Статья
подписывается всеми авторами. На отдельном листе представляются данные автора
(или всех авторов): фамилия, полные имя и отчество, телефон, факс, e-mail,
почтовый адрес. Если работа выполнена несколькими авторами, указывается фамилия
одного из них, ответственного за переписку с редакцией. %\\[-13.5pt]
\item Редакция журнала
осуществляет самостоятельную экспертизу присланных статей. Возвращение рукописи
на доработку не означает, что статья уже принята к печати. Доработанный вариант
с ответом на замечания рецензента необходимо прислать в редакцию. %\\[-13.5pt]
\item Решение
редакционной коллегии о принятии статьи к печати или ее отклонении сообщается
авторам. Редколлегия не обязуется направлять рецензию авторам отклоненной
статьи; дискуссия с авторами по поводу отклоненных статей не ведется. %\\[-13.5pt]
\item Корректура статей высылается авторам для просмотра. Редакция
просит авторов присылать свои замечания в кратчайшие сроки. %\\[-13.5pt]
\item При
подготовке рукописи в MS Word рекомендуется использовать следующие настройки.
Параметры страницы: формат~--- А4; ориентация~--- книжная; поля (см): внутри~---
2,5, снаружи~--- 1,5, сверху~--- 2, снизу~--- 2, от края до нижнего
колонтитула~--- 1,3. Основной текст: стиль~--- <<Обычный>>: шрифт Times New
Roman, размер 14~пунктов, абзацный отступ~--- 0,5~см, 1,5 интервала,
выравнивание~--- по ширине. Рекомендуемый объем рукописи~--- не свыше
25~страниц указанного формата. Ознакомиться с шаблонами, содержащими примеры
оформления, можно по адресу в Интернете:
\textsf{http://www.ipiran.ru/journal/template.doc}.
\item К рукописи, предоставляемой в 2-х
экземплярах, обязательно прилагается электронная версия статьи (как правило, в
форматах MS WORD (.doc) или \LaTeX\ (.tex), а также~--- дополнительно~--- в
формате .pdf) на дискете, лазерном диске или по электронной почте. Сокращения
слов, кроме стандартных, не применяются. Все страницы рукописи должны быть
пронумерованы. %\\[-13.5pt]
\item Статья должна содержать следующую информацию на русском и
английском языках: название, Ф.И.О. авторов, места работы авторов и их
электронные адреса, подробные сведения об авторах, оформленные в соответствии с форматом, 
определяемым файлами {\sf http://www.ipiran.ru/journal/issues/2011\_05\_01/authors.asp} и 
{\sf http://www.ipiran.ru/journal/issues/2011\_01\_eng/authors.asp},
аннотация (не более 100~слов), ключевые слова. Ссылки на
литературу в тексте статьи нумеруются (в квадратных скобках) и располагаются в
порядке их первого упоминания. В~списке литературы не должно быть позиций, на которые нет ссылки в тексте статьи.
Все фамилии авторов, заглавия статей, названия
книг, конференций и~т.\,п.\ даются на языке оригинала, если этот язык
использует кириллический или латинский алфавит. %\\[-13.5pt]
\item Присланные в редакцию материалы авторам не возвращаются.
\item При отправке файлов по электронной
почте просим придерживаться следующих правил:
\begin{itemize}
\item указывать в поле subject (тема) название журнала и фамилию автора; %\\[-13.5pt]
\item использовать attach (присоединение); %\\[-13.5pt]
\item в случае больших объемов информации возможно
использование общеизвестных архиваторов (ZIP, RAR); %\\[-13.5pt]
\item в состав электронной версии статьи должны входить: файл, содержащий текст статьи, и файл(ы),
содержащий(е) иллюстрации. %\\[-13.5pt]
\end{itemize}
\item Журнал <<Информатика и её применения>> является некоммерческим изданием. 
Плата за публикацию с авторов не взимается, гонорар авторам не выплачивается.
\end{enumerate}
\thispagestyle{empty}
\textbf{Адрес редакции:} Москва 119333,
ул.~Вавилова, д.~44, корп.~2, ИПИ РАН\\
\hphantom{\textbf{Адрес редакции:} }Тел.: +7 (499) 135-86-92\ \
Факс:  +7 (495) 930-45-05\ \  E-mail:   rust@ipiran.ru }
}

\end{document}


%\tableofcontents

%\end{document}





%\def\stat{cont}
{%\hrule\par
%\vskip 7pt % 7pt
\raggedleft\Large \bf%\baselineskip=3.2ex
А\,В\,Т\,О\,Р\,С\,К\,И\,Й\ \ У\,К\,А\,З\,А\,Т\,Е\,Л\,Ь\ \ З\,А\ \ 2\,0\,0\,7 г. \vskip 17pt
    \hrule
    \par
\vskip 21pt plus 6pt minus 3pt }

\label{st\stat}

\def\tit{\ }

\def\aut{\ }
\def\auf{\ }

\def\leftkol{\ } % ENGLISH ABSTRACTS}

\def\rightkol{\ } %ENGLISH ABSTRACTS}

\titele{\tit}{\aut}{\auf}{\leftkol}{\rightkol}


\contentsline {chapter}{\ }{Выпуск \quad Стр.} 
\contentsline {section}{\textbf{Батракова Д.\,А., Королев В.\,Ю., Шоргин С.\,Я.}\ \ Новый метод вероятностно-ста\-ти\-сти\-че\-ско\-го анализа информационных потоков в\nobreakspace {}телекоммуникационных сетях}{\qquad 1 \qquad 40} 
\contentsline {section}{\textbf{Борисов А.\,В.}\ \ Байесовское оценивание в системах наблюдения с\nobreakspace {}марковскими скачкообразными процессами: игровой подход}{\qquad 2 \qquad 65}
\contentsline {section}{\textbf{Босов А.\,В., Иванов А.\,В.}\ \ Программная инфраструктура информационного Web-пор\-тала}{\qquad 2 \qquad 50}
\contentsline {section}{\textbf{Захаров В.\,Н., Калиниченко Л.\,А., Соколов И.\,А., Ступников С.\,А.}\ \ Конструирование канонических информационных моделей для интегрированных информационных систем}{\qquad 2 \qquad 15}
\contentsline {section}{\textbf{Захаров В.\,Н., Козмидиади В.\,А.}\ \ Средства обеспечения отказоустойчивости при\-ло\-жений}{\qquad 1 \qquad 14} 
\contentsline {section}{\textbf{Иванов А.\,В.}\ \ см. Босов А.\,В.\hfill\hfill\hfill\hfill\hfill\hfill\hfill\hfill\hfill\hfill\hfill\hfill\hfill\hfill\hfill\hfill\hfill\hfill\hfill\hfill\hfill\hfill\hfill\hfill\hfill\hfill\hfill\hfill\hfill\hfill\hfill\hfill\hfill\hfill\hfill}{\ }
\contentsline {section}{\textbf{Ильин В.\,Д., Соколов И.\,А.}\ \ Символьная модель системы знаний информатики в\nobreakspace {}че\-ло\-ве\-ко-автоматной среде}{\qquad 1 \qquad 66} 
\contentsline {section}{\textbf{Калиниченко Л.\,А.}\ \ см. Захаров В.\,Н.\hfill\hfill\hfill\hfill\hfill\hfill\hfill\hfill\hfill\hfill\hfill\hfill\hfill\hfill\hfill\hfill\hfill\hfill\hfill\hfill\hfill\hfill\hfill\hfill\hfill\hfill\hfill\hfill\hfill\hfill\hfill\hfill\hfill\hfill\hfill}{\ }
\contentsline {section}{\textbf{Козеренко Е.\,Б.}\ \ Лингвистическое моделирование для систем машинного перевода и обработки знаний}{\qquad 1 \qquad 54} 
\contentsline {section}{\textbf{Козмидиади В.\,А.}\ \ см. Захаров В.\,Н.\hfill\hfill\hfill\hfill\hfill\hfill\hfill\hfill\hfill\hfill\hfill\hfill\hfill\hfill\hfill\hfill\hfill\hfill\hfill\hfill\hfill\hfill\hfill\hfill\hfill\hfill\hfill\hfill\hfill\hfill\hfill\hfill\hfill\hfill\hfill }{\ } 
\contentsline {section}{\textbf{Королев В.\,Ю.}\ \ см. Батракова Д.\,А.\hfill\hfill\hfill\hfill\hfill\hfill\hfill\hfill\hfill\hfill\hfill\hfill\hfill\hfill\hfill\hfill\hfill\hfill\hfill\hfill\hfill\hfill\hfill\hfill\hfill\hfill\hfill\hfill\hfill\hfill\hfill\hfill\hfill\hfill\hfill}{\ } 
\contentsline {section}{\textbf{Кудрявцев А.\,А., Шоргин С.\,Я.}\ \ Байесовский подход к\nobreakspace {}анализу систем массового обслуживания и\nobreakspace {}показателей надежности}{\qquad 2 \qquad 76}
\contentsline {section}{\textbf{Печинкин А.\,В., Соколов И.\,А., Чаплыгин В.\,В.}\ \ Многолинейная система массового обслуживания с конечным накопителем и ненадежными приборами}{\qquad 1 \qquad 27} 
\contentsline {section}{\textbf{Печинкин А.\,В., Соколов И.\,А., Чаплыгин В.\,В.}\ \ Стационарные характеристики многолинейной\nobreakspace {}системы массового обслуживания с\nobreakspace {}одновременными отказами приборов}{\qquad 2 \qquad 39}
\contentsline {section}{\textbf{Синицын И.\,Н.}\ \ Корреляционные методы построения аналитических информационных моделей флуктуаций полюса Земли по априорным данным}{\qquad 2 \qquad \hphantom{9}2}
\contentsline {section}{\textbf{Синицын И.\,Н.}\ \ Развитие теории фильтров Пугачева для оперативной обработки информации в стохастических системах}{{\qquad 1 \qquad \hphantom{9}3}} 
\contentsline {section}{\textbf{Соколов И.\,А.}\ \ см. Захаров В.\,Н.\hfill\hfill\hfill\hfill\hfill\hfill\hfill\hfill\hfill\hfill\hfill\hfill\hfill\hfill\hfill\hfill\hfill\hfill\hfill\hfill\hfill\hfill\hfill\hfill\hfill\hfill\hfill\hfill\hfill\hfill\hfill\hfill\hfill\hfill\hfill}{\ }
\contentsline {section}{\textbf{Соколов И.\,А.}\ \ см. Ильин В.\,Д.\hfill\hfill\hfill\hfill\hfill\hfill\hfill\hfill\hfill\hfill\hfill\hfill\hfill\hfill\hfill\hfill\hfill\hfill\hfill\hfill\hfill\hfill\hfill\hfill\hfill\hfill\hfill\hfill\hfill\hfill\hfill\hfill\hfill\hfill\hfill}{\ } 
\contentsline {section}{\textbf{Соколов И.\,А.}\ \ см. Печинкин А.\,В.\hfill\hfill\hfill\hfill\hfill\hfill\hfill\hfill\hfill\hfill\hfill\hfill\hfill\hfill\hfill\hfill\hfill\hfill\hfill\hfill\hfill\hfill\hfill\hfill\hfill\hfill\hfill\hfill\hfill\hfill\hfill\hfill\hfill\hfill\hfill}{\ } 
\contentsline {section}{\textbf{Соколов И.\,А.}\ \ см. Печинкин А.\,В.\hfill\hfill\hfill\hfill\hfill\hfill\hfill\hfill\hfill\hfill\hfill\hfill\hfill\hfill\hfill\hfill\hfill\hfill\hfill\hfill\hfill\hfill\hfill\hfill\hfill\hfill\hfill\hfill\hfill\hfill\hfill\hfill\hfill\hfill\hfill}{\ }
\contentsline {section}{\textbf{Ступников С.\,А.}\ \ см. Захаров В.\,Н.\hfill\hfill\hfill\hfill\hfill\hfill\hfill\hfill\hfill\hfill\hfill\hfill\hfill\hfill\hfill\hfill\hfill\hfill\hfill\hfill\hfill\hfill\hfill\hfill\hfill\hfill\hfill\hfill\hfill\hfill\hfill\hfill\hfill\hfill\hfill}{\ }
\contentsline {section}{\textbf{Чаплыгин В.\,В.}\ \ см. Печинкин А.\,В.\hfill\hfill\hfill\hfill\hfill\hfill\hfill\hfill\hfill\hfill\hfill\hfill\hfill\hfill\hfill\hfill\hfill\hfill\hfill\hfill\hfill\hfill\hfill\hfill\hfill\hfill\hfill\hfill\hfill\hfill\hfill\hfill\hfill\hfill\hfill}{\ } 
\contentsline {section}{\textbf{Чаплыгин В.\,В.}\ \ см. Печинкин А.\,В.\hfill\hfill\hfill\hfill\hfill\hfill\hfill\hfill\hfill\hfill\hfill\hfill\hfill\hfill\hfill\hfill\hfill\hfill\hfill\hfill\hfill\hfill\hfill\hfill\hfill\hfill\hfill\hfill\hfill\hfill\hfill\hfill\hfill\hfill\hfill}{\ }
\contentsline {section}{\textbf{Шоргин С.\,Я.}\ \ см. Батракова Д.\,А.\hfill\hfill\hfill\hfill\hfill\hfill\hfill\hfill\hfill\hfill\hfill\hfill\hfill\hfill\hfill\hfill\hfill\hfill\hfill\hfill\hfill\hfill\hfill\hfill\hfill\hfill\hfill\hfill\hfill\hfill\hfill\hfill\hfill\hfill\hfill}{\ } 
\contentsline {section}{\textbf{Шоргин С.\,Я.}\ \ см. Кудрявцев А.\,А.\hfill\hfill\hfill\hfill\hfill\hfill\hfill\hfill\hfill\hfill\hfill\hfill\hfill\hfill\hfill\hfill\hfill\hfill\hfill\hfill\hfill\hfill\hfill\hfill\hfill\hfill\hfill\hfill\hfill\hfill\hfill\hfill\hfill\hfill\hfill}{\ }
%\thispagestyle{myheadings}
\def\leftfootline{\small{\textbf{\thepage}
\hfill ИНФОРМАТИКА И ЕЁ ПРИМЕНЕНИЯ\ \ \ том~1\ \ \ выпуск~2\ \ \ 2007}
}%
 \def\rightfootline{\small{ИНФОРМАТИКА И ЕЁ ПРИМЕНЕНИЯ\ \ \ том~1\ \ \ выпуск~2\ \ \ 2007
 \hfill \textbf{\thepage}}}
 \label{end\stat}

%\def\stat{cont-e}
{%\hrule\par
%\vskip 7pt % 7pt
\raggedleft\Large \bf%\baselineskip=3.2ex
2\,0\,0\,7\ \ A\,U\,T\,H\,O\,R\ \ I\,N\,D\,E\,X \vskip 17pt
    \hrule
    \par
\vskip 21pt plus 6pt minus 3pt }

\label{st\stat}

\def\tit{\ }

\def\aut{\ }
\def\auf{\ }

\def\leftkol{\ } % ENGLISH ABSTRACTS}

\def\rightkol{\ } %ENGLISH ABSTRACTS}

\titele{\tit}{\aut}{\auf}{\leftkol}{\rightkol}


\contentsline {chapter}{\ }{Issue \quad Page} 
\contentsline {subsection}{\textbf{Batrakova D.\,A., Korolev V.\,Yu., Shorgin S.\,Ya.}\ \ A New Method for the Probabilistic and Statistical Analysis of Information Flows in Telecommunication Networks}{\qquad 1 \qquad 40} 
\contentsline {subsection}{\textbf{Borisov A.\,V.}\ \ Bayesian Estimation in\nobreakspace {}Observation Systems with\nobreakspace {}Markov Jump Processes: Game-Theoretic Approach}{\qquad 2 \qquad 65} 
\contentsline {subsection}{\textbf{Bosov A.\,V., Ivanov A.\,V.}\ \ Linguistic Simulation for Machine Translation and Knowledge Management Systems}{\qquad 2 \qquad 50} 
\contentsline {subsection}{\textbf{Chaplygin V.\,V.} see Pechinkin A.\,V.\hfill\hfill\hfill\hfill\hfill\hfill\hfill\hfill\hfill\hfill\hfill\hfill\hfill\hfill\hfill\hfill\hfill\hfill\hfill\hfill\hfill\hfill\hfill\hfill\hfill\hfill\hfill\hfill\hfill\hfill\hfill\hfill\hfill\hfill\hfill}{\ }
\contentsline {subsection}{\textbf{Chaplygin V.\,V.} see Pechinkin A.\,V.\hfill\hfill\hfill\hfill\hfill\hfill\hfill\hfill\hfill\hfill\hfill\hfill\hfill\hfill\hfill\hfill\hfill\hfill\hfill\hfill\hfill\hfill\hfill\hfill\hfill\hfill\hfill\hfill\hfill\hfill\hfill\hfill\hfill\hfill\hfill}{\ }
\contentsline {subsection}{\textbf{Ilyin V.\,D., Sokolov I.\,A.}\ \ The Symbol Model of Informatics Knowledge System in Human-Automaton Environment}{\qquad 1 \qquad 66} 
\contentsline {subsection}{\textbf{Ivanov A.\,V.} see Bosov A.\,V.\hfill\hfill\hfill\hfill\hfill\hfill\hfill\hfill\hfill\hfill\hfill\hfill\hfill\hfill\hfill\hfill\hfill\hfill\hfill\hfill\hfill\hfill\hfill\hfill\hfill\hfill\hfill\hfill\hfill\hfill\hfill\hfill\hfill\hfill\hfill}{\ }
\contentsline {subsection}{\textbf{Kalinichenko L.\,A.} see Zakharov V.\,N.\hfill\hfill\hfill\hfill\hfill\hfill\hfill\hfill\hfill\hfill\hfill\hfill\hfill\hfill\hfill\hfill\hfill\hfill\hfill\hfill\hfill\hfill\hfill\hfill\hfill\hfill\hfill\hfill\hfill\hfill\hfill\hfill\hfill\hfill\hfill}{\ }
\contentsline {subsection}{\textbf{Korolev V.\,Yu.} see Batrakova D.\,A.\hfill\hfill\hfill\hfill\hfill\hfill\hfill\hfill\hfill\hfill\hfill\hfill\hfill\hfill\hfill\hfill\hfill\hfill\hfill\hfill\hfill\hfill\hfill\hfill\hfill\hfill\hfill\hfill\hfill\hfill\hfill\hfill\hfill\hfill\hfill}{\ }
\contentsline {subsection}{\textbf{Kozerenko E.\,B.}\ \ Linguistic Simulation for Machine Translation and Knowledge Management Systems}{\qquad 1 \qquad 54} 
\contentsline {subsection}{\textbf{Kozmidiady V.\,A.} see Zakharov V.\,N.\hfill\hfill\hfill\hfill\hfill\hfill\hfill\hfill\hfill\hfill\hfill\hfill\hfill\hfill\hfill\hfill\hfill\hfill\hfill\hfill\hfill\hfill\hfill\hfill\hfill\hfill\hfill\hfill\hfill\hfill\hfill\hfill\hfill\hfill\hfill}{\ }
\contentsline {subsection}{\textbf{Kudryavtsev A.\,A., Shorgin S.\,Ya.}\ \ Bayesian Approach to Queueing Systems and Reliability Characteristics}{\qquad 2 \qquad 76} 
\contentsline {subsection}{\textbf{Pechinkin A.\,V., Sokolov I.\,A., Chaplygin V.\,V.}\ \ Multichannel Queuing System with Finite Buffer and Unreliable Servers}{\qquad 1 \qquad 27} 
\contentsline {subsection}{\textbf{Pechinkin A.\,V., Sokolov I.\,A., Chaplygin V.\,V.}\ \ Stationary Characteristics of a Multichannel Queueing System with\nobreakspace {}Simultaneous Refusals of Servers}{\qquad 2 \qquad 39} 
\contentsline {subsection}{\textbf{Shorgin S.\,Ya.} see Batrakova D.\,A.\hfill\hfill\hfill\hfill\hfill\hfill\hfill\hfill\hfill\hfill\hfill\hfill\hfill\hfill\hfill\hfill\hfill\hfill\hfill\hfill\hfill\hfill\hfill\hfill\hfill\hfill\hfill\hfill\hfill\hfill\hfill\hfill\hfill\hfill\hfill}{\ }
\contentsline {subsection}{\textbf{Shorgin S.\,Ya.} see Kudryavtsev A.\,A.\hfill\hfill\hfill\hfill\hfill\hfill\hfill\hfill\hfill\hfill\hfill\hfill\hfill\hfill\hfill\hfill\hfill\hfill\hfill\hfill\hfill\hfill\hfill\hfill\hfill\hfill\hfill\hfill\hfill\hfill\hfill\hfill\hfill\hfill\hfill}{\ }
\contentsline {subsection}{\textbf{Sinitsyn I.\,N.}\ \ Correlational Methods for Analytical Informational Models of the Earth Pole Fluctuations Design Based on a priori Data}{\qquad 2 \qquad \hphantom{9}2}
\contentsline {subsection}{\textbf{Sinitsyn I.\,N.}\ \ Development of Pugachev Filtering for Stochastic Systems}{\qquad 1 \qquad \hphantom{9}3}
\contentsline {subsection}{\textbf{Sokolov I.\,A.} see Ilyin V.\,D.\hfill\hfill\hfill\hfill\hfill\hfill\hfill\hfill\hfill\hfill\hfill\hfill\hfill\hfill\hfill\hfill\hfill\hfill\hfill\hfill\hfill\hfill\hfill\hfill\hfill\hfill\hfill\hfill\hfill\hfill\hfill\hfill\hfill\hfill\hfill}{\ }
\contentsline {subsection}{\textbf{Sokolov I.\,A.} see Pechinkin A.\,V.\hfill\hfill\hfill\hfill\hfill\hfill\hfill\hfill\hfill\hfill\hfill\hfill\hfill\hfill\hfill\hfill\hfill\hfill\hfill\hfill\hfill\hfill\hfill\hfill\hfill\hfill\hfill\hfill\hfill\hfill\hfill\hfill\hfill\hfill\hfill}{\ }
\contentsline {subsection}{\textbf{Sokolov I.\,A.} see Pechinkin A.\,V.\hfill\hfill\hfill\hfill\hfill\hfill\hfill\hfill\hfill\hfill\hfill\hfill\hfill\hfill\hfill\hfill\hfill\hfill\hfill\hfill\hfill\hfill\hfill\hfill\hfill\hfill\hfill\hfill\hfill\hfill\hfill\hfill\hfill\hfill\hfill}{\ }
\contentsline {subsection}{\textbf{Sokolov I.\,A.} see Zakharov V.\,N.\hfill\hfill\hfill\hfill\hfill\hfill\hfill\hfill\hfill\hfill\hfill\hfill\hfill\hfill\hfill\hfill\hfill\hfill\hfill\hfill\hfill\hfill\hfill\hfill\hfill\hfill\hfill\hfill\hfill\hfill\hfill\hfill\hfill\hfill\hfill}{\ }
\contentsline {subsection}{\textbf{Stupnikov S.\,A.} see Zakharov V.\,N.\hfill\hfill\hfill\hfill\hfill\hfill\hfill\hfill\hfill\hfill\hfill\hfill\hfill\hfill\hfill\hfill\hfill\hfill\hfill\hfill\hfill\hfill\hfill\hfill\hfill\hfill\hfill\hfill\hfill\hfill\hfill\hfill\hfill\hfill\hfill}{\ }
\contentsline {subsection}{\textbf{Zakharov V.\,N., Kalinichenko L.\,A., Sokolov I.\,A., Stupnikov S.\,A.}\ \ Development of Canonical Information Models for Integrated Information Systems}{\qquad 2 \qquad 15} 
\contentsline {subsection}{\textbf{Zakharov V.\,N., Kozmidiady V.\,A.}\ \ Means Providing Applications Fault Tolerance}{\qquad 1 \qquad 14} 
\def\leftfootline{\small{\textbf{\thepage}
\hfill ИНФОРМАТИКА И ЕЁ ПРИМЕНЕНИЯ\ \ \ том~1\ \ \ выпуск~2\ \ \ 2007}
}%
 \def\rightfootline{\small{ИНФОРМАТИКА И ЕЁ ПРИМЕНЕНИЯ\ \ \ том~1\ \ \ выпуск~2\ \ \ 2007
 \hfill \textbf{\thepage}}}
 \label{end\stat}


%\tableofcontents


\end{document}