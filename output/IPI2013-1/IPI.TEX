\documentclass[10pt]{book}
\usepackage[utf8]{inputenc}

\usepackage{latexsym,amssymb,amsfonts,amsmath,indentfirst,shapepar,%fleqn,%
picinpar,shadow,floatflt,enumerate,multicol,colortbl,ipi}

\usepackage{rotating}
\usepackage{mathrsfs}
\usepackage[noend]{algorithmic}
\usepackage{ulem}

\input{epsf}

%\nofiles

%\includeonly{obchak,avtor,avtor-eng} %+pdf
%\includeonly{obchak,avtor}
%\includeonly{pred}      %+pdf
%\includeonly{podgot-1str}  %+
%\includeonly{ocherk} %+

%\includeonly{sinitsin}  %1Abst+pdf
%\includeonly{bening}   %Abst+pdf  
%\includeonly{korolev}  %+pdf
%\includeonly{zatsman} %+pdf
%\includeonly{grusho}  %pdf
%\includeonly{morozova} %pdf
%\includeonly{kozerenko} %непечатные символ+pdf
%\includeonly{lukmor}  %Abst+pdf
%\includeonly{rudoi}   %ABst+pdf
%\includeonly{gudasa}  %+pdf
%\includeonly{chertok}  %Abst+pdf
%\includeonly{kuzn}    %11+pdf
%\includeonly{milovanova}  %3Abst+pdf
%\includeonly{shevts}  %+pdf
 


%\includeonly{toc-rus, toc-en}
%\includeonly{obchak} %,toc-en}

%\includeonly{obchak}
%\includeonly{reshal}  %pdf
%\includeonly{eng-index}
%\includeonly{cover3}

\usepackage{acad}
%\usepackage{courier}
\usepackage{decor}
\usepackage{newton}
\usepackage{pragmatica}
\usepackage{zapfchan}
\usepackage{petrotex}
\usepackage{bm}                     % полужирные греческие буквы
\usepackage{upgreek}                % прямые греческие буквы
\usepackage{eufrak}
%\usepackage{verbatim}

\renewcommand{\bottomfraction}{0.99}
\renewcommand{\topfraction}{0.99}
\renewcommand{\textfraction}{0.01}

\setcounter{secnumdepth}{1} %здесь - 3 + chapter = 4

\arraycolsep=1.5pt

%\usepackage[pdftex]{graphicx}

%\usepackage{oz}

%NEW COMMANDS


\renewcommand*{\hm}[1]{#1\nobreak\discretionary{}%
            {\hbox{$\mathsurround=0pt #1$}}{}} %% Дублирует знаки операций
                               %при переносе в формуле (перед знаком, который 
                               %надо продублировать ставится команда \hm)

%\newcommand{\endproof}{\hfill$\Box$}
\renewcommand{\r}{\mathbb{R}}
\newcommand{\I}{{\rm I\hspace{-0.7mm}I}}
%\newcommand{\Ikl}{{\tt{1}}\hspace*{-1.44mm}\mathtt{1}}
\newcommand{\Ik}{\mbox{{\small \tt {1}}\hspace{-1.5mm}{\tt 1}}}
\newcommand{\argmin}{\mathop{\mathrm{arg}\,\mathrm{min}}}
\newcommand{\argmax}{\mathop{\mathrm{arg}\,\mathrm{max}}}
%\newcommand{\capr}{\mathop{\cap\,}}
%\newcommand{\cupr}{\mathop{\cup\,}}
%\def\argmin{\mathop{arg\,min}}

\def\vrp{\varphi}
\def\prt{\partial}
\def\mm{{\rm M}}

\newcommand{\il}[2]{\int\limits_{#1}^{#2}}%интеграл с пределами #1 и #2


\def\sss{\sum\limits}
\def\tr{,\,\ldots\,,\,}
\def\rk{\right]}
\def\lk{\left[}
\def\rf{\right\}}
\def\lf{\left\{}

\def\ee{{\cal E}}
\def\ww{{\cal W}}
\def\yy{{\cal Y}}
\def\vv{{\cal V}}

\newcommand{\R}{\mathbb R}
\newcommand{\N}{\mathbb N}

\newcommand{\h}{{\bf H}}
\newcommand{\p}{{\sf P}}  % вероятность

\newcommand{\e}{{\sf E}}  % мат. ожидание
\newcommand{\D}{{\sf D}}  % дисперсия
\newcommand{\eps}{\varepsilon}
\newcommand{\vp}{{\mathbf p}}
\newcommand{\vz}{{\mathbf z}}
\newcommand{\vx}{{\mathbf x}}
\newcommand{\vf}{{\mathbf f}}
%\newcommand{\vp}{\mathrm{v.p.}}
\newcommand{\F}{{\mathcal F}}
\def\ap{{\mathrm{ЭР}}}
\newcommand{\ud}{\Delta_n} %uniform ditance
\newcommand{\nud}{\Delta_n(x)}

\newcommand{\abs}[1]{\left\vert#1\right\vert}
\def\w{\omega}
\def\W{\Omega}
\def\iii{\int\limits}
\def\iin{\int\limits_{-\infty}^\infty}


\DeclareMathOperator{\sign}{sign}

%\newcommand{\gr}{{\geqslant}}

\newcommand{\g}{\mbox{\textit{g}}}

\renewcommand{\la}{\lambda}
\newcommand{\si}{\sigma}
\newcommand{\alp}{\alpha}

%\newcommand{\pto}{\stackrel{P}{\longrightarrow}} % сходимость по веpоятности

\newcommand{\eqd}{\stackrel{d}{=}} % равенство по pаспpеделению

%\newcommand{\kp}{\kappa}
%\def\Q{{\cal Q}} \def\H{{\cal H}}
%\newcommand{\bet}{\beta_{2+\delta}}


%\newtheorem{definition}{Определение}
%\renewcommand{\thedefinition}{\arabic{definition}.}
%END NEW COMMANDS

%\renewcommand{\baselinestretch}{1.2}

%\pagestyle{myheadings}

\setlength{\textwidth}{167mm}      % 122mm
\setlength{\textheight}{658pt}
%\setlength{\textheight}{635.6pt}
\setlength{\columnsep}{4.5mm}

\setcounter{secnumdepth}{4}

%\addtolength{\headheight}{2pt}
%\addtolength{\headsep}{-2mm}

%\addtolength{\topmargin}{-20mm}  % for printing


\hoffset=-30mm  % From Yap
%\hoffset=-20mm  % From Acrobat

%\voffset=0mm % From Yap
%\voffset=-15mm   % From Acrobat

\addtolength{\evensidemargin}{-9.5mm} % for printing
\addtolength{\oddsidemargin}{9.5mm}  % for printing

%\renewcommand{\thefootnote}{\fnsymbol{footnote}}
%\renewcommand{\thefootnote}{\arabic{footnote}}
\renewcommand{\figurename}{\protect\bf Рис.}
\renewcommand{\tablename}{\protect\bf Таблица}

\newcommand{\Caption}[1]{\caption{\protect\small %\baselineskip=2.5ex
#1}}

\renewcommand{\thefigure}{\arabic{figure}}
\renewcommand{\thetable}{\arabic{table}}
\renewcommand{\theequation}{\arabic{equation}}
\renewcommand{\thesection}{\arabic{section}}

\renewcommand{\contentsname}{СОДЕРЖАНИЕ}
\newcommand{\fr}[2]{\displaystyle\frac{\displaystyle #1\mathstrut}{\displaystyle #2\mathstrut}}

%\renewcommand{\thefootnote}{\fnsymbol{footnote}}
%\newcommand{\g}{\mbox{\textit{g}}}

%\newcommand{\Caption}[1]{\caption{\protect\small\baselineskip=2ex #1}}
\newcounter{razdel}
\setcounter{razdel}{0}


\newcommand{\titel}[4]{%
\

\vspace*{5pt}

\ifodd\therazdel {\raggedright\noindent\Large\textrm\textbf
 \lineskip .75em
  \baselineskip=3.2ex #1 \par}
\vskip 1em {\noindent\large\textrm\textbf #2 \par}
\addcontentsline{toc}{subsection}{{\textrm\textbf #3}\protect\newline #1}
\def\rightheadline{\underline{\noindent\hbox to \textwidth{\hfill\small\textrm{#4}
%\hfill \large\bf\thepage
}}}
\def\leftheadline{\underline{\noindent\parbox{\textwidth}{
%\raggedleft\large\bf\thepage \hfill
\small\textit{#3}\hfill}}}
\def\leftfootline{\small{\textbf{\thepage}
\hfill ИНФОРМАТИКА И ЕЁ ПРИМЕНЕНИЯ\ \ \ том~7\ \ \ выпуск 1\ \ \ 2013}
}%
 \def\rightfootline{\small{ИНФОРМАТИКА И ЕЁ ПРИМЕНЕНИЯ\ \ \ том~7\ \ \ выпуск~1\ \ \ 2013
\hfill \textbf{\thepage}}} 
\vskip 2em \setcounter{figure}{0}
\setcounter{table}{0} 
\setcounter{equation}{0} 
\setcounter{section}{0}
\setcounter{subsection}{0} 
\setcounter{subsubsection}{0}
\setcounter{footnote}{0} 
\setcounter{razdel}{0}
%\end{flushleft}
\else {
 \raggedright\noindent\Large\textrm\textbf
 \lineskip .75em
\baselineskip=3.2ex #1 \par} \vskip 1em
%\begin{flushleft}
{\noindent\large\textrm\textbf #2 \par}
\addcontentsline{toc}{subsection}{{\textrm\textbf #3}\protect\newline #1}
\def\rightheadline{\underline{\noindent\hbox to \textwidth{\hfill\small\textrm{#4}
%\hfill \large\bf\thepage
}}}
\def\leftheadline{\underline{\noindent\parbox{\textwidth}{%\raggedleft\large\bf\thepage \hfill
\small\textit{#3}\hfill}}}
\def\leftfootline{\small{\textbf{\thepage}
\hfill ИНФОРМАТИКА И ЕЁ ПРИМЕНЕНИЯ\ \ \ том~7\ \ \ выпуск~1\ \ \ 2013}
}%
 \def\rightfootline{\small{ИНФОРМАТИКА И ЕЁ ПРИМЕНЕНИЯ\ \ \ том~7\ \ \ выпуск~1\ \ \ 2013
\hfill \textbf{\thepage}}} \vskip 2em \setcounter{figure}{0}
\setcounter{table}{0} \setcounter{equation}{0} \setcounter{section}{0}
\setcounter{subsection}{0} \setcounter{subsubsection}{0}
\setcounter{footnote}{0}
%\end{flushleft}
\fi}

\newcommand{\titelr}[2]{%
\

\vspace*{5pt}

\ifodd\therazdel {\raggedright\noindent\large\textrm\textbf
 \lineskip .75em
  \baselineskip=3.2ex #1 \par}
\vskip 1em {\noindent\normalsize\textrm\textbf #2 \par}
\else {
 \raggedright\noindent\large\textrm\textbf
 \lineskip .75em
\baselineskip=3.2ex #1 \par} \vskip 1em
%\begin{flushleft}
{\noindent\normalsize\textrm\textbf #2 \par}
\fi}

\newcommand{\titele}[5]{%
\

%\vspace*{5pt}

\ifodd\therazdel {\raggedright\noindent%\large
\textrm\textbf
 \lineskip .75em
%  \baselineskip=3.2ex
#1 \par}
\vskip .5em {\noindent\large\textrm\textbf #2 \par}
\vskip .5em
 {\noindent\textrm #3 \par}
\addcontentsline{toc}{subsection}{{\textrm\textbf #1}\protect\newline #2}
\def\rightheadline{\underline{\noindent\hbox to \textwidth{\hfill\small\textrm{#4}
%\hfill \large\bf\thepage
}}}
\def\leftheadline{\underline{\noindent\parbox{\textwidth}{
%\raggedleft\large\bf\thepage \hfill
\small\textrm{#5}\hfill}}}
\def\leftfootline{\small{\textbf{\thepage}
\hfill ИНФОРМАТИКА И ЕЁ ПРИМЕНЕНИЯ\ \ \ том~7\ \ \ выпуск~1\ \ \ 2013}
}%
 \def\rightfootline{\small{ИНФОРМАТИКА И ЕЁ ПРИМЕНЕНИЯ\ \ \ том~7\ \ \ выпуск~1\ \ \ 2013
\hfill \textbf{\thepage}}} \vskip 1em \setcounter{figure}{0}
\setcounter{table}{0} \setcounter{equation}{0} \setcounter{section}{0}
\setcounter{subsection}{0} \setcounter{subsubsection}{0}
\setcounter{footnote}{0} \setcounter{razdel}{0}
%\end{flushleft}
\else {
 \raggedright\noindent%\large
 \textrm\textbf
 \lineskip .75em
%\baselineskip=3.2ex
#1 \par} \vskip .5em
%\begin{flushleft}
{\noindent\large\textrm\textbf #2 \par} \vskip .5em
 {\noindent\textrm #3 \par}
\addcontentsline{toc}{subsection}{{\textrm\textbf #1}\protect\newline #2}
\def\rightheadline{\underline{\noindent\hbox to \textwidth{\hfill\small\textrm{#4}
%\hfill \large\bf\thepage
}}}
\def\leftheadline{\underline{\noindent\parbox{\textwidth}{%\raggedleft\large\bf\thepage \hfill
\small\textrm{#5}\hfill}}}
\def\leftfootline{\small{\textbf{\thepage}
\hfill ИНФОРМАТИКА И ЕЁ ПРИМЕНЕНИЯ\ \ \ том~7\ \ \ выпуск~1\ \ \ 2013}
}%
 \def\rightfootline{\small{ИНФОРМАТИКА И ЕЁ ПРИМЕНЕНИЯ\ \ \ том~7\ \ \ выпуск~1\ \ \ 2013
\hfill \textbf{\thepage}}} \vskip 1em \setcounter{figure}{0}
\setcounter{table}{0} \setcounter{equation}{0} \setcounter{section}{0}
\setcounter{subsection}{0} \setcounter{subsubsection}{0}
\setcounter{footnote}{0}
%\end{flushleft}
\fi}

\def\Abst#1{
\begin{center}\small\nwt
\parbox{150mm}{%\baselineskip=2.5ex
\textbf{Аннотация:}\ \
%\hspace*{\parindent}
#1}
\end{center}}
\def\Abste#1{
\begin{center}\small\nwt
\parbox{150mm}{%\baselineskip=2.5ex
\textbf{Abstract:}\ \
%\hspace*{\parindent}
#1}
\end{center}}

\def\KW#1{
\begin{center}\small\nwt
\parbox{150mm}{%\baselineskip=2.5ex
\textbf{Ключевые слова:}\ \ #1}
\end{center}}

\def\KWE#1{
\begin{center}\small\nwt
\parbox{150mm}{%\baselineskip=2.5ex
\textbf{Keywords:}\ \ #1}
\end{center}}


\def\KWN#1{
%\begin{center}
%\small
%\parbox{150mm}\end{center}
}

\renewcommand{\thesubsection}{\thesection.\arabic{subsection}\hspace*{-5pt}}
\renewcommand{\thesubsubsection}{\thesubsection\hspace*{5pt}.\arabic{subsubsection}\hspace*{-3pt}}

\begin{document}
\Rus

\nwt
%\ptb

%\renewcommand{\contentsname}{\protect\Large\bf Содержание}

\setcounter{tocdepth}{2}

%\tableofcontents

\renewcommand{\bibname}{\protect\rmfamily Литература}
  \def\Au#1{{\it #1}}

%\newcommand{\No}{№}
  \newcommand{\tg}{\,\mathrm{tg}\,}
    \newcommand{\ctg}{\,\mathrm{ctg}\,}
  \newcommand{\arctg}{\,\mathrm{arctg}\,}
  
\def\forallb{\mathop{\forall}}
\def\cupb{\mathop{\cup}}
\def\existsb{\mathop{\exists}}

\setcounter{page}{1}

\newpage
\addtocounter{razdel}{1}
%\def\razd{РЕГУЛИРУЕМЫЙ ЭЛЕКТРОПРИВОД ДЛЯ ЭЛЕКТРОЭНЕРГЕТИКИ}


\setcounter{page}{3}

%   { %\Large  
   { %\baselineskip=16.6pt
   
   \vspace*{-48pt}
   \begin{center}\LARGE
   \textit{Предисловие}
   \end{center}
   
   %\vspace*{2.5mm}
   
   \vspace*{25mm}
   
   \thispagestyle{empty}
   
   { %\small 

    
Вниманию читателей журнала <<Информатика и её применения>> предлагается 
очередной тематический выпуск <<Вероятностно-статистические методы и 
задачи информатики и информационных технологий>>. Предыдущие тематические 
выпуски журнала по данному направлению вышли в 2008~г.\ (т.~2, вып.~2), 
в 2009~г.\ (т.~3, вып.~3) и в 2010~г.\ (т.~4, вып.~2). 

Статьи, собранные в данном журнале, посвящены разработке новых вероятностно-статистических 
методов, ориентированных на применение к решению конкретных задач информатики и информационных 
технологий, а также~--- в ряде случаев~--- и других прикладных задач. Проблематика, охватываемая 
публикуемыми работами, развивается в рамках научного сотрудничества между Институтом проблем 
информатики Российской академии наук (ИПИ РАН) и Факультетом вычислительной математики и 
кибернетики Московского государственного университета им.\ М.\,В.~Ломоносова в ходе работ 
над совместными научными проектами (в том числе в рамках функционирования 
Научно-образовательного центра <<Вероятностно-статистические методы анализа рисков>>). 
Многие из авторов статей, включенных в данный номер журнала, являются активными участниками 
традиционного международного семинара по проблемам устойчивости стохастических моделей, 
руководимого В.\,М.~Золотаревым и В.\,Ю.~Королевым; регулярные сессии этого семинара 
проводятся под эгидой МГУ и ИПИ РАН (в 2011~г.\ указанный семинар проводится в октябре 
в Калининградской области РФ). 

Наряду с представителями ИПИ РАН и МГУ в число авторов данного выпуска журнала входят 
ученые из Научно-исследовательского института системных исследований РАН, Института 
проблем технологии микроэлектроники и особочистых материалов РАН, Института 
прикладных математических исследований Карельского НЦ РАН, Московского 
авиационного института, Вологодского государственного педагогического университета, 
НИИММ им.\ Н.\,Г.~Чеботарева, Казанского государственного университета, Дебреценского 
университета (Венгрия).

Несколько статей выпуска посвящено разработке и применению стохастических методов и 
информационных технологий для решения различных прикладных задач. В~работе В.\,Г.~Ушакова 
и О.\,В.~Шестакова рассмотрена задача определения вероятностных характеристик случайных 
функций по распределениям интегральных преобразований, возникающих в задачах эмиссионной 
томографии. В~статье Д.\,О.~Яковенко и М.\,А.~Целищева рассмотрены некоторые вопросы 
математической теории риска и предложен новый подход к диверсификации инвестиционных 
портфелей. Работа И.\,А.~Кудрявцевой и А.\,В.~Пантелеева посвящена построению и 
исследованию математической модели, описывающей динамику сильноионизованной плазмы. 
В~статье П.\,П.~Кольцова изучается качество работы ряда алгоритмов сегментации изображений. 
Статья А.\,Н.~Чупрунова и И.~Фазекаша посвящена вероятностному анализу числа без\-оши\-бочных 
блоков при помехоустойчивом кодировании; получены усиленные законы больших чисел для указанных 
величин.

В данном выпуске традиционно присутствует тематика, весьма активно разрабатываемая в течение 
многих лет специалистами ИПИ РАН и МГУ,~--- методы моделирования и управления для 
информационно-телекоммуникационных и вычислительных систем, в частности методы 
теории массового обслуживания. В~статье А.\,И.~Зейфмана с соавторами рассматриваются 
модели обслуживания, описываемые марковскими цепями с непрерывным временем в случае 
наличия катастроф. В~работе М.\,М.~Лери и И.\,А.~Чеплюковой рассматриваются случайные 
графы Интернет-типа, т.\,е.\ графы, степени вершин которых имеют степенные распределения; 
такие задачи находят применение при исследовании глобальных сетей передачи данных. 
Работа Р.\,В.~Разумчика посвящена исследованию систем массового обслуживания специального 
вида~--- с отрицательными заявками и хранением вытесненных заявок.

Ряд статей посвящен развитию перспективных теоретических 
вероятностно-статистических методов, которые находят широкое применение в различных 
задачах информатики и информационных технологий. В~работе В.\,Е.~Бенинга, А.\,К.~Горшенина 
и В.\,Ю.~Королева рассмотрена задача статистической проверки гипотез о числе компонент 
смеси вероятностных распределений, приводится конструкция асимптотически наиболее мощного 
критерия. Результаты этой работы найдут применение в ряде прикладных задач, использующих 
математическую модель смеси вероятностных распределений (в информатике, моделировании 
финансовых рынков, физике турбулентной плазмы и~т.\,д.). В~статье В.\,Ю.~Королева, 
И.\,Г.~Шевцовой и С.\,Я.~Шоргина строится новая, улучшенная оценка точности нормальной 
аппроксимации для пуассоновских случайных сумм; как известно, указанные случайные суммы 
широко используются в качестве моделей многих реальных объектов, в том числе в информатике, 
физике и других прикладных областях. Работа В.\,Г.~Ушакова и Н.\,Г.~Ушакова посвящена 
исследованию ядерной оценки плотности распределения; эти результаты могут применяться, 
в част\-ности, при анализе трафика в телекоммуникационных системах. Серьезные приложения 
в статистике могут получить результаты работы О.\,В.~Шестакова, в которой доказаны оценки 
скорости сходимости распределения выборочного абсолютного медианного отклонения к нормальному 
закону. 

\smallskip

Редакционная коллегия журнала выражает надежду, что данный тематический  выпуск 
будет интересен специалистам в области теории вероятностей и математической статистики 
и их применения к решению задач информатики и информационных технологий.
     
     %\vfill 
     \vspace*{20mm}
     \noindent
     Заместитель главного редактора журнала <<Информатика и её 
применения>>,\\
     директор ИПИ РАН, академик  \hfill
     \textit{И.\,А.~Соколов}\\
     
     \noindent
     Редактор-составитель тематического выпуска,\\
     профессор кафедры математической статистики факультета\\
      вычислительной математики и кибернетики МГУ им.\ М.\,В.~Ломоносова,\\
     ведущий научный сотрудник ИПИ РАН,\\ 
доктор физико-математических наук \hfill
      \textit{В.\,Ю.~Королев}
     
     } }
     }

\def\stat{sinits}

\def\tit{АНАЛИТИЧЕСКОЕ МОДЕЛИРОВАНИЕ  РАСПРЕДЕЛЕНИЙ С~ИНВАРИАНТНОЙ
МЕРОЙ В~СТОХАСТИЧЕСКИХ СИСТЕМАХ С~РАЗРЫВНЫМИ ХАРАКТЕРИСТИКАМИ$^*$}

\def\titkol{Аналитическое моделирование  распределений с~инвариантной
мерой в~стохастических системах} % с~разрывными характеристиками}

\def\autkol{И.\,Н.~Синицын}

\def\aut{И.\,Н.~Синицын$^1$}

\titel{\tit}{\aut}{\autkol}{\titkol}

{\renewcommand{\thefootnote}{\fnsymbol{footnote}}\footnotetext[1]
{Работа выполнена при финансовой поддержке РФФИ
(проект №\,13-07-00036) и программой <<Интеллектуальные информационные 
технологии, системный анализ и автоматизация>> (проект~1.7).}}

\renewcommand{\thefootnote}{\arabic{footnote}}
\footnotetext[1]{Институт проблем информатики Российской академии наук, sinitsin@dol.ru}



\Abst{На базе методов нормальной аппроксимации и статистической линеаризации разработаны 
точные и приближенные алгоритмы аналитического моделирования плотностей стохастических 
режимов с инвариантной мерой в гауссовых и негауссовых стохастических системах (СтС)
с разрывными 
характеристиками. Рассмотрены особенности моделирования в СтС с 
пуассоновскими шумами. На тестовых примерах показана достаточная для многих приложений 
точность алгоритмов.}

\KW{автокоррелированная помеха; аналитическое моделирование;
интегродифференциальные уравнения Пугачёва; метод нормальной аппроксимации;
метод статистической линеаризации; нелинейная гауссовская и негауссовская стохастическая система в смысле Ито;
пуассоновская стохастическая сис\-те\-ма; распределение с инвариантной мерой;
стохастический режим}

\vskip 14pt plus 9pt minus 6pt

      \thispagestyle{headings}

      \begin{multicols}{2}

            \label{st\stat}



\section{Введение}

Следуя [1, 2], будем рассматривать нестационарный стохастических режим $Z\hm=Z(t)$ 
в нелинейной дифференциальной СтС, понимаемой в смысле Ито:
    \begin{equation}
    \dot Z = a(Z,t) + b (Z,t) V\,, \enskip Z(t_0) = Z_0\,.
    \label{e1.1-sin}
    \end{equation}
Здесь $Z$~--- $k$-мер\-ный вектор состояния СтС, $Z\hm\in \Delta$ ($\Delta$~--- 
многообразие состояний); $a\hm=a(Z,t)$ и $b\hm= b(Z,t)$~--- детерминированные  
$(k\times 1)$- и $(k\times m)$-мер\-ные  функции  отмеченных аргументов; 
$V\hm=V(t)$~--- $m$-мер\-ный вектор негауссовских (в общем случае) белых шумов 
с нулевыми математическими ожиданиями и представляющий собой среднеквадратичную 
(с.к.)\ производную процесса с независимыми приращениями  $W\hm=W(t)$, 
$V\hm=\dot W$. Обозначим через $\chi\hm=\chi(\mu;t)$ логарифмическую производную 
одномерной характеристической функции $h_1\hm=h_1(\mu;t)$ процесса $W\hm=W(t)$, определяемую формулой
    \begin{equation}
    \chi(\mu;t)=\fr{\prt \ln h_1 (\mu;t)}{\prt t}=
    \fr{1}{h_1(\mu;t)}\,\fr{\prt h_1(\mu;t)}{\prt t}\,.
    \label{e1.2-sin}
    \end{equation}

Начальное состояние $Z_0$ будем считать случайной величиной (СВ), не зависящей 
от $W(t)$ для $t\hm>t_0$. Предположим, что стохастический режим $Z(t)$ является 
сильным решением~(\ref{e1.1-sin}), а функции $a,b$ и $\chi$ удовлетворяют известным 
условиям существования и единственности~[1, 2].

Пусть существуют одно- и $n$-мерные плот\-ности\linebreak $f_1\hm=f_1(z;t)$ и 
$f_n\hm= f_n(z_1\tr z_n; t_1 \tr t_n)$ и характеристические функции $g_1\hm=g_1(\la;t)$ и 
$g_n\hm=g_n(\la_1\tr \la_n; t_1\tr t_n)$ $(n\hm\ge 2)$, удовлетворяющие 
интегродифференциальным уравнениям\linebreak Пугачева~[1, 2]:
    \begin{multline}
    \fr{\prt f_1(z;t)}{\prt t}+\fr{\prt^{\mathrm{T}}}{\prt z}\lk a(z,t)f_1(z;t)\rk = 
\fr{1}{(2 \pi)^k} \times{}\\
{}\times \iin\iin \chi(b(\xi,t)^{\mathrm{T}}\la;t) e^{i\la^{\mathrm{T}}(\xi-z)} f_1(z;t) \,
    d\xi d\la\,;
    \label{e1.3-sin}
    \end{multline}
   \begin{equation}
    f_1(z;t_0)=f_0(z)\,;\label{e1.4-sin}
    \end{equation}
    
    \vspace*{-12pt}
    
    \begin{multline*}
\fr{\prt f_n(z_1\tr z_n;t_1\tr t_n)}{\prt t_n}+{}\\
{}+\fr{\prt^{\mathrm{T}}}{\prt z_n}\left[
a(z_n, t_n) f_n (z_1\tr z_n; t_1\tr t_n)\right]={}\\
{}= \fr{1}{(2\pi)^{kn}} \iin\iin \chi(b(\xi_n, t_n)^{\mathrm{T}} \la_n;t_n) \times{}\\
{}\times \exp\lf i \sss_{l=1}^n \la_l^{\mathrm{T}} (\xi_l-z_l)\rf \times{}\\
{}\times
f_n (\xi_1\tr \xi_n; t_1\tr t_n)\,d\xi_1\cdots d\xi_n d\la_1\cdots d\la_n\,;
%\label{e1.5-sin}
\end{multline*}

\vspace*{-12pt}

\begin{multline*}
f_n(z_1\tr z_{n-1},z_n;t_1\tr t_{n-1},t_{n})={}\\
{}= f_{n-1} (z_1\tr z_{n-1};t_1\tr t_{n-1})\delta (z_n - z_{n-1})\,;
%\label{e1.6-sin}
\end{multline*}
       
        
\noindent        
\begin{multline}
\fr{\prt g_1 (\la;t)}{\prt t} -{}\\
{}-\fr{1}{(2\pi)^k} \iin \iin i\la^{\mathrm{T}} a (z,t) 
e^{i(\la^{\mathrm{T}} -\mu^{\mathrm{T}})z} g_1 (\mu;t)\, d\mu dz={}\\
{}=\fr{1}{(2\pi)^k} \iin \iin \chi(b(z,t)^{\mathrm{T}} \la^{\mathrm{T}};t) 
e^{i(\la^{\mathrm{T}} -\mu^{\mathrm{T}})z} \times{}\\
{}\times
g_1 (\mu;t)\, d\mu dz\,;
\label{e1.7-sin}
\end{multline}
\begin{equation}
g_1(\la;t_0) = g_0(\la)\,\,; \label{e1.8-sin}
\end{equation}

\vspace*{-12pt}

\begin{multline*}
\fr{\prt g_n (\la_1\tr \la_n; t_1\tr t_n)}{\prt t_n} -{}\\
{}-
\fr{1}{(2\pi)^{kn}} \iin \cdots \iin i\la^{\mathrm{T}} a (z_n,t_n) \times{}\\
{}\times \exp \lk i \sss\limits_{k=1}^n (\la_k^{\mathrm{T}} - \mu_k^{\mathrm{T}}) z_k\rk \times{}\\
{}\times g_n 
(\mu_1\tr \mu_n; t_1\tr t_n)\, d\mu_1 \cdots d \mu_n dz_1\cdots dz_n={}\\
{}= \fr{1}{(2\pi)^{kn}} \iin\cdots \iin \chi (b(z_n;t)^{\mathrm{T}} \la_n;t_n)\times{}\\
{}\times\exp \lk i \sss_{k=1}^n (\la_k^{\mathrm{T}} - \mu_k^{\mathrm{T}}) z_k\rk \times{}\\
{}\times g_n 
(\mu_1\tr \mu_n; t_1\tr t_n) \,d\mu_1 \cdots d \mu_n dz_1\cdots dz_n;
%\label{e1.9-sin}
\end{multline*}

\vspace*{-12pt}

\noindent
\begin{multline*}
g_n (\la_1\tr \la_n; t_1\tr t_{n-1},t_{n-1})= {}\\
{}=
g_{n-1} (\la_1\tr \la_{n-2},\la_{n-1}+\la_n; t_1\tr t_{n-1})\,, %\label{e1.10-sin}
\end{multline*}
 $$
        t_1\le t_2 \le \cdots \le t_n,\enskip n=2,3,\ldots
        $$

При этом одно- и $n$-мер\-ные плотности и характеристические функции связаны 
между собой известными соотношениями:
\begin{equation*}
f_1(z;t) = \fr{1}{(2\pi)^{k}} \iin e^{-i\mu^{\mathrm{T}} z} g_1(\mu;t) d\mu\,; %\label{e1.11-sin}
    \end{equation*}
      \begin{equation*}
   g_1(\la;t) = \iin e^{i\la^{\mathrm{T}} z} f_1(z;t)\, dz\,; %\label{e1.12-sin}
   \end{equation*}
   
   \vspace*{-12pt}

\noindent
\begin{multline}
f_n( z_1\tr z_n; t_1\tr t_n) ={}\\
{}=
\fr{1}{(2\pi)^{kn}} 
\iin\cdots \iin \exp \lf - i \sss_{l=1}^n \la_l^{\mathrm{T}} z_l\rf \times{}\\
{}\times g_n (\la_1\tr \la_n; t_1\tr t_n)\, d\la_1\cdots d\la_n\,;\label{e1.13-sin}
\end{multline}


\vspace*{-12pt}

\noindent
\begin{multline*}
g_n (\la_1\tr \la_n; t_1\tr t_n) ={}\\
{}=\iin\cdots \iin \exp\lf i \sss_{l=1}^n \la_l^{\mathrm{T}} z_l\rf \times{}\\
{}\times f_n (z_1\tr z_n; t_1\tr t_n)\, dz_1\cdots dz_n\,. %\label{e1.14-sin}
\end{multline*}

Для нахождения одномерных плотностей $f_1(z,t) \hm= f_1^* (z)$ и характеристических функций 
$g_1(\la;t) \hm= g_1^* (\la)$ стохастических режимов в стационарных СтС~(\ref{e1.1-sin}) при
    \begin{equation}
    a(z,t) = a^*(z)\,;\ b(z,t)=b^*(z)\,;\ \chi(\mu;t)= \chi^*(\mu)
    \label{e1.15-sin}
    \end{equation}
следует в~(\ref{e1.3-sin}) и~(\ref{e1.7-sin}) положить 
$\prt f_1/\prt t \hm= 0$ и $\prt g_1/ \prt t \hm=0$. В~результате получим соответственно
\begin{multline*}
\fr{\prt^{\mathrm{T}}}{\prt z}\lk a^* (z) f_1^* (z)\rk = {}\\
{}=
\fr{1}{(2\pi)^k} \iin \iin \chi^* (b^*(\xi)^{\mathrm{T}} \la) e^{i\la^{\mathrm{T}}(\xi-z)} f_1^* (\xi)\, d\xi d\la\,;
%\label{e1.16-sin}
\end{multline*}

\vspace*{-12pt}

\noindent
\begin{multline*}
-\fr{1}{(2\pi)^k} \iin  \iin i\la^{\mathrm{T}} a^*(z) e^{i(\la^{\mathrm{T}}-\mu^{\mathrm{T}})z} g_1^*(\mu)\, d\mu dz={}\\
{}=\fr{1}{(2\pi)^k} \iin  \iin \chi^*(b^*(z)^{\mathrm{T}}\la) e^{i(\la^{\mathrm{T}}-\mu^{\mathrm{T}})z} g_1^*(\mu)\, d\mu dz.
%\label{e1.17-sin}
\end{multline*}
Поставим задачу разработки точных и приближенных  алгоритмов
аналитического моделирования распределений (плотностей и
характеристических функций) стохастических режимов  $Z\hm=Z(t)$ в
нелинейных гауссовских и негауссовских СтС~(\ref{e1.1-sin})  с разрывными
характеристиками $a\hm=a(z,t)$ и $b\hm=b(z,t)$, обладающих свойством
сохранения инвариантной меры, т.\,е.\ удовлетворяющих уравнениям~(\ref{e1.3-sin})
и~(\ref{e1.7-sin}) при $\chi\hm=0$.

Условия сохранения инвариантной меры можно представить в следующем развернутом виде:
\begin{equation}
\left.
\begin{array}{c}
\displaystyle\fr{\prt f_1 (z;t)}{\prt t} + A_a f_1 (z;t) =0\,;\\[9pt] 
\hspace*{-4.5mm}\displaystyle A_a f_1(z;t) = 
    \fr{\prt^{\mathrm{T}}}{\prt z} \lk a(z,t) f_1(z;t)\rk = \mathrm{div}\, \pi(z;t)\,;
    \end{array}
    \right\}
    \label{e1.18-sin}
    \end{equation}
\begin{equation}
\left.
\begin{array}{c}
A_a^* f_1^*(z) =0\,;\\[9pt]
\displaystyle A_a^* f_1^* (z) = \fr{\prt^{\mathrm{T}}}{ \prt z} \lk a^* 
(z) f_1^* (z)\rk =\mathrm{div}\, \pi^* (z)\,;
\end{array}
\right\}
\label{e1.19-sin}
\end{equation}
$$
\fr{\prt g_1 (\la;t)}{\prt t} - B_a g_1(\la;t) =0\,;
$$

\vspace*{-14pt}

\noindent
\begin{multline}
B_a g_1(\la;t) ={}\\[2pt]
{}=\fr{1}{(2\pi)^k} \iin\iin i\la^{\mathrm{T}} a(z,t) e^{i(\la^{\mathrm{T}}-\mu^{\mathrm{T}})z}
 g_1(\mu;t)\, d\mu dz={}\\[2pt]
{}= \iin i\la^{\mathrm{T}} a(z,t) e^{i\la^{\mathrm{T}}z} f_1(z;t)\, dz={}\\[2pt]
{}= \iin e^{i\la^{\mathrm{T}} z} i\la^{\mathrm{T}} \pi(z;t)\, dz\,;
\label{e1.20-sin}
\end{multline}

\vspace*{-9pt}

\noindent
\begin{equation}
\left.
\begin{array}{c}
\hspace*{-45mm}B_a^* g_1^* (\la)=0\,;\\[12pt]
\hspace*{-48mm}B_a^* g_1^* (\la) = {}\\[10pt]
\hspace*{-3mm}{}=\fr{1}{(2\pi)^k} \iin\! i\la^{\mathrm{T}} a^* (z) e^{i(\la^{\mathrm{T}} -\mu^{\mathrm{T}})z} g_1^* (\mu)\, d\mu dz={}\\[10pt]
{}=\displaystyle\iin\! i\la^{\mathrm{T}} a^*(z) e^{i\la^{\mathrm{T}}z} f_1^* (z)\, dz = {}\\[10pt]
\displaystyle{}=
\iin e^{i\la^{\mathrm{T}} z} i\la^{\mathrm{T}} \pi^* (z)\, dz\,.
\end{array}
\right\}
\label{e1.21-sin}
\end{equation}
Для гауссовских (нормальных) СтС с гладкими характери\-стиками точные и приближенные 
методы  и алгоритмы аналитического моделирования рассмотрены в~[1--15]. 

Особое внимание 
уделим приближенным методам, основанным на методах нор\-маль\-ной аппроксимации и статистической 
линеаризации. Подробно рассмотрим их применение к пуассоновским СтС.



\section{Точные методы и~алгоритмы аналитического моделирования распределений 
с~инвариантной мерой}

Пусть функция~$a$ в СтС~(\ref{e1.1-sin}) допускает пред\-став\-ле\-ние
\begin{equation}
a= a(z,t) = a_1(z,t) +a_2 (z,t) \label{e2.1-sin}
\end{equation}
такое, что функция  $f_1\hm=f_1(z;t)$ является плот\-ностью инвариантной меры 
невозмущенной шумами системы, описываемой векторным обыкновенным дифференциальным 
уравнением вида
   \begin{equation}
   \dot z = a_1 (z,t)\,,\label{e2.2-sin}
   \end{equation}
т.\,е.\ удовлетворяет условию~(\ref{e1.18-sin}):
\begin{equation}
\fr{\prt f_1 (z;t)}{\prt t}+ \fr{\prt^{\mathrm{T}}}{\prt z} \lk a_1 (z,t) f_1(z;t)\rk =0\,.
\label{e2.3-sin}
\end{equation}

Для гладких функций $a_1\hm=a_1(z,t)$ вопросы существования и основные свойства 
интегральных 
 инвариантов изучены в~\cite{16-sin, 17-sin}. При этом в~(\ref{e2.1-sin}) 
функция $a_2 \hm= a_2(z,t)$ определяется путем решения следующего интегродифференциального 
уравнения:
\begin{multline}
\fr{\prt^{\mathrm{T}}}{\prt z}\lk a_2 (z,t) f_1(z;t) \rk 
=
\fr{1}{(2\pi)^k}\times{}\\
{}\times \iin\iin \chi(b(\xi,t)^{\mathrm{T}} \la;t) 
e^{i\la^{\mathrm{T}}(\xi-z)} f_1(\xi;t)\, d\xi d\la\,.\label{e2.4-sin}
\end{multline}
В общем случае нахождение функций $a_1$ и~$a_2$ в~(\ref{e2.1-sin})~--- такая же
трудная задача, как решение уравнений~(\ref{e1.3-sin}) и~(\ref{e1.4-sin}).

Для стационарных СтС, когда выполнены условия~(\ref{e1.15-sin}), 
уравнения~(\ref{e2.1-sin})--(\ref{e2.4-sin}) имеют вид:
\begin{align}
a(z)&= a_1(z) + a_2(z)\,;\label{e2.5-sin}
\\
\dot z &= a_1(z)\,,\label{e2.6-sin}
\\
\fr{\prt^{\mathrm{T}}}{\prt z}\lk a_2^*(z) f_1^*(z)\rk &= {}\notag\\
&\hspace*{-28mm}{}=
\fr{1}{(2\pi)^k} \!\!\iin \iin\!\! \chi^* (b^*(\xi)^{\mathrm{T}} \la) 
e^{i\la^{\mathrm{T}}(\xi-z)} f_1^*(\xi)\, d\xi d\la\,.\!\!\!\label{e2.7-sin}
\end{align}
В этом случае можно выбирать невозмущенную сис\-те\-му~(\ref{e2.6-sin}) так, чтобы
она имела первые интегралы.

В терминах характеристических функций соотношения~(\ref{e2.3-sin}), (\ref{e2.4-sin})
и~(\ref{e2.7-sin}) могут быть записаны следующим образом:

\noindent
\begin{equation}
\fr{\prt g_1 (\la;t)}{\prt t} - B_{a_1} g_1(\la;t) =0\,;\label{e2.8-sin}
\end{equation}
\begin{equation*}
B_{a_1}^* g_1^*(\la) =0\,. %\label{e2.9-sin}
\end{equation*}
Для составляющих $a_2(z,t)$ и $a_2^*(z)$ имеют место уравнения
\begin{multline}
B_{a_2} g_1(\la;t) 
= \fr{1}{(2\pi)^k} \times{}\\
\hspace*{-2.5mm}{}\times\iin\iin \!\chi(b(z,t)^{\mathrm{T}} \la;t) 
e^{i(\la^{\mathrm{T}}-\mu^{\mathrm{T}})z} g_1(\mu;t) \,d\mu dz;\label{e2.10-sin}
\end{multline}

\vspace*{-16pt}

\noindent
\begin{multline}
B_{a_2}^* g_1^*(\la) 
= \fr{1}{(2\pi)^k} \times{}\\
{}\times\iin\iin 
\chi^*(b^*(z)^{\mathrm{T}} \la) e^{i(\la^{\mathrm{T}}-\mu^{\mathrm{T}})z} g_1^*(\mu)\, d\mu dz\,.
\label{e2.11-sin}
\end{multline}

Отсюда вытекают конструктивные точные алгоритмы аналитического
моделирования распределений с инвариантной мерой. В~их основе лежат
следующие теоремы.

%\pagebreak

\medskip

\noindent
\textbf{Теорема~2.1.} \textit{Функция $f_1\hm=f_1(z;t)$ будет решением}~(\ref{e1.3-sin})
\textit{и}~(\ref{e1.4-sin}) \textit{тогда и только тогда, когда $a\hm=a(z,t)$ допускает
представление}~(\ref{e2.1-sin}) \textit{такое, что $f_1\hm=f_1(z;t)$ является плотностью
инвариантной меры обыкновенного дифференциального уравнения}~(\ref{e2.2-sin}),
\textit{т.\,е.\ удовле\-тво\-ря\-ет условию}~(\ref{e2.3-sin}). \textit{При этом со\-став\-ля\-ющая $a_2$
определяется из решения интегродифференциального уравнения}~(\ref{e2.4-sin}).

\medskip

\noindent
\textbf{Теорема~2.2.} \textit{Функция $f_1^*\hm=f_1^*(z)$ будет решением}~(\ref{e1.3-sin}) 
\textit{тогда и только тогда, когда $a^*\hm=a^*(z)$ допускает
представление}~(\ref{e2.5-sin}) \textit{такое, что $f_1^*\hm=f_1^*(z)$ является плотностью
инвариантной меры}~(\ref{e2.6-sin}). \textit{При этом составляющая $a_2^{*}$
определяется из решения  уравнения}~(\ref{e2.7-sin}).

\medskip

\noindent
\textbf{Теорема~2.3.} \textit{Функция $g_1\hm=g_1(\la;t)$ будет ре\-ше\-нием}~(\ref{e1.7-sin}), 
(\ref{e1.8-sin}) \textit{тогда и только тогда, когда недиф\-фе\-ренцируемая функция
$a\hm=a(z,t)$  допускает пред\-став\-ление}~(\ref{e2.1-sin}) \textit{такое, что
$g_1\hm=g_1(\la;t)$ является ха\-рак\-теристической функцией инвариантной
меры \mbox{уравнения}}~(\ref{e2.2-sin}), \textit{т.\,е.\ удовлетворяет условию}~(\ref{e2.8-sin}). 
\textit{При этом составляющая $a_2$ определяется из уравнения}~(\ref{e2.10-sin}).

\medskip

\noindent
\textbf{Теорема 2.4.} \textit{Функция $g_1^*\hm=g_1^*(\la)$  будет решением}~(\ref{e1.13-sin}) 
\textit{тогда и только тогда, когда недифференцируемая функция $a^*\hm=a^*(z)$  
допускает представление}~(\ref{e2.5-sin}) \textit{такое, что $g_1^*$ является  
характеристической функцией инвариантной меры}~(\ref{e2.2-sin}). 
\textit{При этом $a_2^*$ определяется из решения}~(\ref{e2.11-sin}).

\smallskip

Теоремы~2.1--2.4 легко обобщаются на случай многомерных распределений с инвариантной мерой.

\section{Приближенные методы и~алгоритмы аналитического моделирования распределений 
с~инвариантной мерой, основанные на~нормальной аппроксимации и статистической линеаризации}

Пусть нелинейная СтС~(\ref{e1.1-sin}) допускает применение метода нормальной аппроксимации 
(МНА)~[1, 2]. Тогда одно- и двумерные нормальные плот\-ности $f_1^{\mathrm{МНА}}$,
 $f_2^{\mathrm{МНА}}$ и характеристические функции  $g_1^{\mathrm{МНА}}$,  
 $g_2^{\mathrm{МНА}}$, а также вектор математического ожидания $m_t = M^{\mathrm{МНА}} Z(t)$, 
 ковариационная мат\-ри\-ца $K_t \hm= M^{\mathrm{МНА}} Z^{0\mathrm{T}} Z^0 (t)$ 
 $(Z^0 (t) \hm= Z(t) \hm- m_t)$ и матрица ковариационных функций 
 $K(t_1, t_2) \hm= M^{\mathrm{МНА}} Z^{0\mathrm{T}} (t_1) Z^0 (t_2)$ $(t_1\hm< t_2)$ определяются 
 следующими уравнениями:
    \begin{multline}
    f_1^{\mathrm{МНА}} = f_1^{\mathrm{МНА}} (z;t, m_t, K_t) =
    \lk (2\pi)^k |K_t|\rk^{-1/2}\times{}\\
    {}\times \exp \lf -  \fr{1}{ 2} 
    \left(z^{\mathrm{T}} - m_t^{\mathrm{T}}\right) K_t^{-1}(z-m_t)\rf\,;\label{e3.1-sin}
    \end{multline}
    
    \vspace*{-12pt}
    
    \noindent
\begin{multline}
f_2^{\mathrm{МНА}} ={}\\
= f_2^{\mathrm{МНА}} (z_1, z_2;t_1, t_2, m_{t_1}, m_{t_2}, K_{t_1}, K_{t_2}, K(t_1, t_2))=\\
{}=\lk (2\pi)^k |\bar K_2|\rk^{-1/2}\times{}\\
\hspace*{-2mm}{}\times \exp \lf - 
([z_1^{\mathrm{T}} z_2^{\mathrm{T}}] - \bar m_2^{\mathrm{T}}) 
\bar K_2^{-1}([z_1^{\mathrm{T}} z_2^{\mathrm{T}}]^{\mathrm{T}}-\bar m_2)\rf;
\!\!\label{e3.2-sin}
\end{multline}
\begin{equation}
g_1^{\mathrm{МНА}} (\la;t)=
\exp\lf i\la^{\mathrm{T}} m- \fr{1}{2}\,\la^{\mathrm{T}} K_t \la\rf\,;\label{e3.3-sin}
\end{equation}

\vspace*{-12pt}

\noindent
\begin{multline}
g_2^{\mathrm{МНА}} (\la_1, \la_2; t_1,t_2) ={}\\
{}= \exp \lf i \bar \la^{\mathrm{T}} \bar m_2 - 
    \fr{1}{2} \,\bar \la^{\mathrm{T}} \bar K_2 \bar \la\rf\,;\label{e3.4-sin}
    \end{multline}
$$
    \bar \la =\lk \la_1^{\mathrm{T}} \la_2^{\mathrm{T}}\rk^{\mathrm{T}}\,;\enskip 
    \bar m_2 =\lk m_{t_1}^{\mathrm{T}} m_{t_2}^{\mathrm{T}}\rk^{\mathrm{T}}\,;
    $$
    $$
    \bar K_2 =\begin{bmatrix}
        K(t_1, t_1)&K(t_1, t_2)\\[3pt]
        K(t_2, t_1)& K(t_2, t_2)
        \end{bmatrix}\,;
        $$
  \begin{multline}
  \dot m_t = a_1 (t, m_t, K_t) ={}\\
  {}=\iin a(z,t) f_1^{\mathrm{МНА}} (z; t, m_t, K_t) \,dz\,;
  \label{e3.5-sin}
  \end{multline}

\vspace*{-12pt}

\noindent
\begin{multline}
\dot K_t = a_2(t, m_t, K_t) = a_{21} + a_{12}+a_{22}={}\\
{}=\left[ \iin a(z,t) (z^{\mathrm{T}}-m_t^{\mathrm{T}}) + (z-m_t) a^{\mathrm{T}} (z,t) +{}\right.\\
\left.{}+ \sigma (z,t)
\vphantom{\iin}\right] f_1^{\mathrm{МНА}} (z;t, m_t, K_t)\, dz\,;
\label{e3.6-sin}
\end{multline}

\vspace*{-12pt}

\noindent
\begin{multline}
\fr{\prt K(t_1, t_2)}{\prt t_2} ={}\\
{}= a_3 (t_1, t_2, m_{t_1},m_{t_2}, K_{t_1}, K_{t_2}, K(t_1,t_2))={}\\
{}=\lk (2\pi)^{2k} |\bar K_2|\rk^{-1/2}\times{}\\
{} \times\iin\iin (z_1-m_{t_1}) a(z_2, t_2)
\exp\left\{ - ([z_1^{\mathrm{T}} z_2^{\mathrm{T}}]-\bar m_2^{\mathrm{T}})\times{}\right.\\
\left.{}\times\bar K_2^{-1} 
([z_1^{\mathrm{T}} z_2^{\mathrm{T}}]-\bar m_2)\right\} dz_1 dz_2\,.
\label{e3.7-sin}
\end{multline}
Здесь введены следующие обозначения:
\begin{equation}
\left.
\begin{array}{c}
z_1=z_{t_1}\,;\enskip  z_2=z_{t_2}\,;\enskip \bar m_2 =\lk m_{t_1}^{\mathrm{T}} m_{t_2}^{\mathrm{T}}\rk^{\mathrm{T}}\,;\\[9pt]
\displaystyle \bar K_2 =\begin{bmatrix}
        K(t_1,t_1)&K(t_1, t_2)\\[3pt]
        K(t_2, t_1)& K(t_2, t_2)
        \end{bmatrix}\,,
        \end{array}
        \right\}
        \label{e3.8-sin}
        \end{equation}
\begin{equation}
\sigma(z,t) = b(z,t) \nu(t) b(z,t)^{\mathrm{T}}\,,\label{e3.9-sin}
\end{equation}
где $\nu=\nu(t)$~--- интенсивность негауссовского белого шума $V\hm=V(t)$.

Для стационарных СтС  при $\dot m^* \hm=0$, $\dot K^* \hm=0$, 
$K(t_1, t_2)\hm= k(\tau)$ $(\tau\hm=t_1-t_2)$  соотношения~(\ref{e3.5-sin})--(\ref{e3.9-sin}) 
принимают вид:
\begin{equation}
a_1^* (m^*, K^*) =0\,;\label{e3.10-sin}
\end{equation}
\begin{equation}
    a_2^*(m^*, K^*) =0\,;\label{e3.11-sin}
    \end{equation}
    \begin{equation}
    \fr{dk(\tau) }{d\tau} = a_{11}^{\mathrm{МНА}} (m^*, K^*) k(\tau)\,;\label{e3.12-sin}
    \end{equation}
$$
k(\tau) = k(-\tau^{\mathrm{T}})\,;\enskip k(0)=K\,.
$$
Из уравнения~(\ref{e3.12-sin}) следует, что алгоритм МНА будет устойчивым, если матрица 
$a_{11}^{\mathrm{МНА}} (m_t, K_t, t)$ будет асимптотически устойчива.

Для $m$ и $K$ уравнения метода статистической линеаризации (МСЛ) в 
нелинейных СтС  при аддитивных шумах, когда $b(z,t) \hm= b_0(t)$, $b^*(z)\hm=b_0^*$ 
получаются из~(\ref{e3.5-sin})--(\ref{e3.7-sin}) и (\ref{e3.10-sin})--(\ref{e3.12-sin}) 
как частный случай.

Условия наличия нормального распределения с инвариантной мерой~(\ref{e1.18-sin}) 
и~(\ref{e1.19-sin}), если заменить $a(z,t)$ статистически
линеаризованным выраже\-нием
\begin{equation*}
    a(Z,t)\approx a_{10}^{\mathrm{МНА}} (t, m_t, K_t) + a_{11}^{\mathrm{МНА}} (t, m_t, K_t) 
    (Z-m_t)\,, %\label{e3.13-sin}
    \end{equation*}
где
\begin{equation*}
a_{10}^{\mathrm{МНА}} =a_{10}^{\mathrm{МНА}} (t, m_t, K_t)\equiv a_1\,; %\label{e3.14-sin}
\end{equation*}
    
    
   
    \noindent
    \begin{multline*}
    a_{11}^{\mathrm{МНА}}=a_{11}^{\mathrm{МНА}} (t, m_t, K_t) = {}\\
    {}=\lk \iin a(z,t) (z^{\mathrm{T}}-m_t^{\mathrm{T}}) 
        f_1^{\mathrm{МНА}} (z; t , m_t, K_t)\, dz\rk\times{}\\
        {}\times K_t^{-1} 
=\left(\fr{\prt}{\prt m_t} a_1^{\mathrm{T}}\right)^{\mathrm{T}}\,, %\label{e3.15-sin}
\end{multline*}
приводят к следующим соотношениям:
        \begin{multline}
\fr{\prt f_1^{\mathrm{МНА}} (z; t, m_t, K_t)}{\prt t} +\fr{\prt^{\mathrm{T}}}{ \prt z} 
\left\{ \left[ a_{10}^{\mathrm{МНА}} (t, m_t, K_t) 
+{}\right.\right.\\
\left.{}+ a_{11}^{\mathrm{МНА}} (t, m_t, K_t) (z-m_t) \vphantom{a_{10}^{\mathrm{МНА}}}
\right]\times{}\\
\left.{}\times 
     f_1^{\mathrm{МНА}} ( z; t , m_t, K_t)\right\} =0\,;
     \label{e3.16-sin}
     \end{multline}
     
     
     \noindent
\begin{multline}
\hspace*{-9.81628pt}\fr{\prt^{\mathrm{T}}}{\prt z} \left\{ \left[ a_{10}^{*{\mathrm{МНА}}}(m^*, K^*) + 
 a_{11}^{*{\mathrm{МНА}}}(m^*, K^*) (z-m^*)\right] \times{}\right.\\
\left.{}\times f_1^{*{\mathrm{МНА}}}(z; m^*, K^*)\right\} =0\,,\label{e3.17-sin}
 \end{multline}
где
\begin{multline*}
f_1^{*{\mathrm{МНА}}} (z; m^*, K^*) = \lk (2\pi)^k |K^*|\rk^{-1/2}\times{}\\
{}\times \exp \lf -
    \fr{1}{2} (z^{\mathrm{T}}-m^{*\mathrm{T}})(K^*)^{-1} (z-m^*)\rf\,.
    \end{multline*}

Аналогично в развернутом виде выписываются условия~(\ref{e1.20-sin}) и~(\ref{e1.21-sin}):
\begin{multline}
\fr{\prt g_1^{\mathrm{МНА}} (\la;t)}{\prt t} -\iin i\la^{\mathrm{T}} \left[ a_{10}^{\mathrm{МНА}} 
    (m_t, K_t, t) +{}\right.\\[2pt]
\left.    {}+ a_{11}^{\mathrm{МНА}} (m_t, K_t, t) (z- m_t) \right]\times{}\\[2pt]
{}\times e^{i\la^{\mathrm{T}} z} f_1^{\mathrm{МНА}} (z; m_t, K_t, t)\, dz=0\,;\label{e3.18-sin}
\end{multline}


\noindent
\begin{multline}
\iin i\la^{\mathrm{T}} \left[ a_{10}^{*{\mathrm{МНА}} } (m^*, K^*) 
+{}\right.\\[2pt]
\left.{}+a_{11}^{*{\mathrm{МНА}} } 
    (m^*, K^*) (z-m^*)\right]\times{}\\[2pt]
    {}\times
     e^{i\la^{\mathrm{T}}z} f_1^{*{\mathrm{МНА}} } (z; m^*, K^*)\, dz =0\,.
    \label{e3.19-sin}
    \end{multline}

Отсюда вытекают следующие теоремы.

\bigskip

\noindent
\textbf{Теорема~3.1.}\ \textit{Если существуют одно- и двумерные  плотности
стохастического режима, а  матрица $a_{11}^{\mathrm{МНА}}$ коэффициентов
статистической (нормальной) линеаризации асимптотически устойчива,
то приближенный алгоритм аналитического моделирования МНА
нестационарных стохастических режимов в СтС}~(\ref{e1.1-sin}) \textit{с инвариантной
мерой определяется выражениями}~(\ref{e3.1-sin})--(\ref{e3.7-sin}) и~(\ref{e3.16-sin}).

\bigskip

\noindent
\textbf{Теорема 3.2.}\ \textit{Если существуют стационарные одно- и
двумерные плотности стохастического режима, а матрица
$a_{11}^{*{\mathrm{МНА}}}$  коэффициентов статистической (нормальной)
линеаризации асимптотически устойчива, то приближенный алгоритм
аналитического моделирования стационарных стохастических режимов с
инвариантной мерой в стационарной СтС}~(\ref{e1.1-sin}) \textit{определяется 
выражениями}~(\ref{e3.10-sin})--(\ref{e3.12-sin}) и~(\ref{e3.17-sin}).

\bigskip

Как известно~[1, 2], одно- и двумерные нормальные распределения
определяют и все  $n$-мер\-ные распределения $(n\hm\ge 3)$, поэтому МНА и
МСЛ дают приближенные алгоритмы для любых многомерных плотностей
стохастических режимов, если они существуют. Аналогично
формулируются теоремы~3.3 и~3.4 на основе условий~(\ref{e3.18-sin}) и~(\ref{e3.19-sin}).


\section{О других приближенных методах и~алгоритмах аналитического моделирования 
распределений с~инвариантной мерой}

\vspace*{-2pt}

 Обобщением МНА являются различные
приближенные методы, основанные на параметризации распределений~[1, 2].
Аппроксимируя одномерную характеристическую функцию $g_1 (\la;t)$
и соответствующую плотность $f_1 (z,t)$ известными функциями
 $g_1^* (\la;\theta)$, $f_1^* (z;\theta)$,  зависящими от
конечномерного векторного параметра~$\theta$, можно свести задачу
приближенного определения одномерного распределения к выводу из
уравнения для характеристических функций обыкновенных
дифференциальных уравнений, определяющих~$\theta$ как функцию
времени. Это относится и к остальным многомерным распределениям.
При аппроксимации многомерных распределений целесообразно выбирать
последовательности функций $\{ f_n^* (z_1,\ldots,z_n;\theta_n)\}$ и 
$\{g_n^* (\la_1\tr \la_n;\theta_n)\}$, каждая пара
которых находилась бы в такой  зависимости от векторного параметра~$\theta_n$, 
чтобы при любом~$n$ множество параметров, образующих
вектор~$\theta_n$, включало в качестве подмножества множество
параметров, образующих вектор~$\theta_{n-1}$. Тогда при
аппроксимации $n$-мер\-но\-го распределения придется определять только
те координаты вектора~$\theta_n$, которые не были определены ранее
при аппроксимации функций $f_1, g_1\tr f_{n-1}$, $g_{n-1}$.

В зависимости от того, что представляют собой параметры, от
которых зависят функции $f_n^* (z_1\tr z_n;\theta_n)$ и 
$g_n^* (\la_1\tr \la_n;\theta_n)$, аппрок-\linebreak симирующие неизвестные
многомерные плотности $f_n (z_1,  \ldots,z_n; t_1 \tr t_n)$ и
характеристические функции $g_n (\la_1\tr \la_n; t_1,\ldots,t_n)$,
используются различные приближенные методы решения
 уравнений при условиях~(9)--(12), определяющих\linebreak многомерные
распределения вектора состояния сис\-те\-мы~$X_t$, в частности методы
моментов (ММ), семиинвариантов (МСИ), ортогональных разложений
(МОР), квазимоментов (МКМ) и~др.~[1, 2].

\vspace*{-6pt}


\section{Обобщение на~случай стохастических систем с~автокоррелированными шумами}

\vspace*{-2pt}

Пусть  СтС описывается нелинейным, в общем случае векторным дифференциальным 
стохастическим уравнением Ито~\cite{1-sin, 2-sin, 15-sin, 18-sin}

\noindent
\begin{equation}
\left.
\begin{array}{c}
    \dot Z = a(Z,t) + b_U(Z,t) U\,;\\[6pt] 
\displaystyle    \sss_{i=0}^l \alpha_i U^{(i)} =
\displaystyle\sss_{j=0}^h \beta_j V^{(j)}\enskip (h<l)\,.
\end{array}
\right\}
    \label{e5.1-sin}
    \end{equation}
    Здесь $U=U(t)$~--- векторная помеха размерности  $m\times 1$; $V\hm=V(t)$~--- 
    негауссовский белый шум с нулевым математическим ожиданием и известной функцией  
    $\chi\hm=\chi(\mu;t)$. В~таком случае в за\-ви\-си\-мости от степени <<гладкости>> 
    стохастического режима $Z\hm=Z(t)$ и помехи $U\hm=U(t)$ уравнения~(\ref{e5.1-sin})  
    путем расширения вектора состояния согласно~[1, 2] приводятся к виду~(\ref{e1.1-sin}) 
    для расширенного вектора состояния~$\bar Z$. Тогда, но уже для расширенного вектора 
    состояния СтС, при решении уравнений~(\ref{e5.1-sin}) могут быть использованы точные 
    (разд.~2) и приближенные (разд.~3) методы и алгоритмы аналитического моделирования 
    нестационарных и стационарных распределений с инвариантной мерой.

\section{Особенности аналитического моделирования распределений с~инвариантной мерой 
в~пуассоновских стохастических системах}

Рассмотрим СтС~(\ref{e1.1-sin}) при $b(z,t) \hm=I_m$ для обобщенного пуассоновского 
белого шума  $V^{\mathrm{OP}}\hm=  V^{\mathrm{OP}}(t)$, когда функция~(\ref{e1.2-sin}) 
определяется формулой
\begin{equation*}
\chi^{\mathrm{OP}} (\mu;t) =\lk g_c^{\mathrm{OP}} (\mu) -
1\rk \nu^{\mathrm{OP}} (t)\,, %\label{e6.1-sin}
\end{equation*}
где $g_c^{\mathrm{OP}} \hm=g_c^{\mathrm{OP}} (\mu)$~--- характеристическая 
функция скачков; $\nu^{\mathrm{OP}} \hm= \nu^{\mathrm{OP}} (t)$~--- 
интенсивность пуассоновского белого шума 
$V^{\mathrm{OP}}\hm=V^{\mathrm{OP}} (t)$. Обозначим через $f_c^{\mathrm{OP}} \hm=
 f_c^{\mathrm{OP}} (z)$ плотность скачков обобщенного пуассоновского процесса. 
 Тогда~(\ref{e1.3-sin}) будет представлять собой известное уравнение Фел\-ле\-ра--Кол\-мо\-го\-ро\-ва
\begin{multline}
\fr{\prt f_1(z;t)}{\prt t} + \fr{\prt^{\mathrm{T}}}{\prt z} 
    \lk a(z,t) f_1(z;t)\rk ={}\\
    \hspace*{-3mm}{}= \nu^{\mathrm{OP}} (t) \lk \iin f_c^{\mathrm{OP}} (z-\xi) f_1 (\xi;t)\, d\xi - f_1(z;t)\rk
    \label{e6.2-sin}
    \end{multline}
с начальным условием~(\ref{e1.4-sin}). В~случае простого пуассоновского белого шума 
с единичными скачками $g_c (\mu) \hm= e^{i\mu}$.

Для  стационарной пуассоновской СтС~(\ref{e1.1-sin}) уравнение~(\ref{e6.2-sin}) имеет следующий вид:
\begin{multline}
\fr{\prt^{\mathrm{T}}}{\prt z} \lk a^* (z) f_1^* (z)\rk = {}\\
{}=
\nu^{\mathrm{OP} *} \lk \iin f_c^{\mathrm{OP}} (z-\xi) f_1^* (\xi)\, d\xi- 
f_1^* (z)\rk\,.\label{e6.3-sin}
\end{multline}

Пользуясь уравнениями~(\ref{e6.2-sin}), (\ref{e6.3-sin})  
и результатами разд.~1 и~2, нетрудно сформулировать следующие утверждения.

\medskip

\noindent
\textbf{Теорема 6.1.}\ \textit{Функция $f_1 \hm= f_1(z;t)$ будет
нестационарным решением}~(\ref{e6.2-sin}), (\ref{e1.4-sin}) \textit{тогда и только тогда, 
когда $a$ допускает представление}~(\ref{e2.1-sin}) \textit{такое, что $f_1$ является плот\-ностью
инвариантной меры обыкновенного дифференциального уравнения}~(\ref{e2.2-sin}),
\textit{т.\,е.\ удовле\-тво\-ря\-ет условию}~(\ref{e2.3-sin}), \textit{а составляющая $a_2$ определяется
из решения следующего уравнения}:
\begin{multline*}
    \fr{\prt^{\mathrm{T}}}{\prt z} \lk a_2 (z,t) f_1 (z;t)\rk =
     \fr{1}{(2\pi)^k}\times{}\\
     {}\times \iin\iin \chi^{\mathrm{OP}} 
    \left(b(\xi,t)^{\mathrm{T}} \la;t\right) e^{i\la^{\mathrm{T}}(\xi-z)} f_1(\xi,t)\,d\xi d\la\,.
%    \label{e6.4-sin}
    \end{multline*}

%\smallskip

\noindent
\textbf{Теорема 6.2.}\ \textit{Функция $f_1^* \hm= f_1^* (z)$ будет стационарным 
решением}~(\ref{e6.3-sin}) \textit{тогда и только тогда, когда $a_2^*$ допускает 
представление}~(\ref{e2.5-sin}) \textit{такое, что  $f_1^*$ является плот\-ностью 
инвариантной меры}~(\ref{e2.6-sin}), \textit{а составляющая $a_2^{*}$ определяется 
из решения следующего уравнения}:
\begin{multline*}
\fr{\prt^{\mathrm{T}} }{\prt z} \lk a_2^{*} (z) f_1^* (z)\rk ={}\\
{}=
    \fr{1}{(2\pi)^k} \iin\iin \chi^{\mathrm{OP} *} (b(\xi)^{\mathrm{T}} \la) 
    e^{i\la^{\mathrm{T}}(\xi-z)} f_1^*(\xi)\,d\xi d\la\,.
%    \label{e6.5-sin}
    \end{multline*}

При использовании МНА и МСЛ для пуассоновских СтС непосредственно применяются теоремы~3.1--3.4, 
причем в формулу~(\ref{e3.9-sin}) для  
$\sigma(z,t)$ входит интенсивность 
$\nu^{\mathrm{OP}} (t)$ обобщенного пуассоновского белого шума.

\section{Тестовые примеры}

\noindent
\textbf{Пример~1}. Рассмотрим осциллятор Дуффинга в обобщенной пуассоновской 
стохастической среде:
\begin{equation}
\ddot X +\w^2 X -\mu X^3 =-\delta^{\mathrm{OP}} \dot X + V^{\mathrm{OP}} (t)\,.\label{e7.1-sin}
\end{equation}
Уравнения МСЛ для~(\ref{e7.1-sin}) имеют следующий вид:
\begin{equation}
\dot m_X = m_{\dot X}\,;\enskip 
\dot m_{\dot X} =- \w_{\mathrm{э}}^2 m_X -\delta^{\mathrm{OP}} m_{\dot X}\,;
\label{e7.2-sin}
\end{equation}
    \begin{equation}
    \left.
    \begin{array}{rl}
    \dot D_{X} &= 2 K_{X\dot X}\,;\\[6pt] 
    \dot D_{\dot X} &=\nu^{\mathrm{OP}} - 2 (\w_{1 \mathrm{э}}^2 K_{X\dot X} + 
    \delta^{\mathrm{OP}} D_{\dot X})\,;\\[6pt]
\dot K_{X\dot X} &= D_{\dot X} -\w_{1 \mathrm{э}}^2 D_X - 
\delta^{\mathrm{OP}} K_{X\dot X}\,.
\end{array}
\right\}
 \label{e7.3-sin}
\end{equation}
Здесь кубическая функция $X^3$ была заменена на статистически линеаризованную при 
гауссовом распределении с дисперсией  $D_X$ согласно~[1, 2]:
\begin{equation*}
X^3 \approx k_0 (m_X, D_X) m_X + k_1 (m_X, D_X) X^0\,,\label{e7.4-sin}
\end{equation*}
где
\begin{align*}
k_0 (m_X, D_X) &= m_X^2 + 3 D_X\,;\\ 
k_1 (m_X, D_X) &= 3 (m_X^2 + D_X)\,;\\
%\label{e7.5-sin}
\w_{\mathrm{э}}^2 &=\w^2 \lk 1- \fr{\mu (m_X^2 + 3D_X)}{\w^2}\rk\,;\\
\w_{1 \mathrm{э}}^2 &=\w^2 \lk 1-  \fr{3\mu (m_X^2 + D_X)}{\w^2}\rk \enskip 
(\w_{\mathrm{э}}>\w_{1 \mathrm{э}})\,.
\end{align*}
%\label{e7.6-sin}
Из~(\ref{e7.2-sin}) и~(\ref{e7.3-sin}) в стационарном режиме имеем:
\begin{gather*}
m_X^* =0\,;\enskip 
m_{\dot X}^* =0\,;\enskip 
K_{X\dot X}^* =0\,;\\
D_{\dot X}^* =\vartheta\,;\enskip 
\vartheta =  \fr{\nu^{\mathrm{OP}}}{ 2\delta^{\mathrm{OP}}}\,,
\end{gather*}
%\label{e7.7-sin}
а $D_X^*$ определяется из уравнения:
    \begin{equation*}
    \w_{1 \mathrm{э}}^2 (D_X^*) D_X^* =\vartheta\,. %\label{e7.8-sin}
    \end{equation*}
Условие наличия стационарного распределения с инвариантной мерой~(\ref{e3.17-sin}) 
требует консерватизма линеаризованной левой части~(\ref{e7.1-sin}). 
Процесс установления стационарных стохастических колебаний происходит 
в два этапа: сначала устанавливается $D_{\dot X}^*$, а затем $D_X^*$.

Интересно отметить, что уравнения МСЛ~(\ref{e7.2-sin}) и~(\ref{e7.3-sin}) сохраняют свой
вид и для любого белого шума интенсивности  $\nu(t)$,
представляющего собой с.к., производную от произвольного процесса с
независимыми приращениями~$W(t)$. Для гауссовского белого шума
$\nu\hm=\nu^G$ соответствующие результаты получены в~\cite{1-sin, 2-sin, 15-sin}. Как
показали вычислительные эксперименты для значений~$\mu$, отвечающих
стохастическим колебаниям, точность составляет около 10\%~\cite{15-sin}.

\medskip

\noindent
\textbf{Пример~2}.\  Для осциллятора Дуффинга в автокоррелированной  пуассоновской среде, когда
\begin{equation*}
\ddot X+ \w^2 X -\mu X^3 =-\delta^{\mathrm{OP}} \dot X + U\,;\enskip 
\dot U +\gamma U =V^{\mathrm{OP}} (t)\,, %\label{e7.9-sin}
\end{equation*}
уравнения МСЛ для  $Z\hm= [X\dot X U]^{\mathrm{T}}$ имеют вид~(\ref{e3.5-sin}) и~(\ref{e3.6-sin}) при
    \begin{gather*}
   a_1 = \begin{bmatrix}
        m_{\dot X}\\
        -\w_{ \mathrm{э}}^2 m_X-\delta^{\mathrm{OP}} m_{\dot X}\\
        -m_U\end{bmatrix}\,;\\
    \alpha=  \begin{bmatrix}
            0&1&0\\
            -\w_{1 \mathrm{э}}^2&-\delta^{\mathrm{OP}}&0\\
            0&0&-\gamma\end{bmatrix}\,;\enskip
    \beta= \begin{bmatrix}
        0&0&0\\
        0&0&0\\
        0&0&1\end{bmatrix}\,;
%        \label{e7.10-sin}
\\
a_2 =\alpha K_t+ K_t \alpha^{\mathrm{T}} +\beta \nu^{\mathrm{OP}} \beta^{\mathrm{T}}\,.
        \end{gather*}
Здесь $\nu^{\mathrm{OP}} =\nu^{\mathrm{OP}}(t)$~--- интенсивность белого шума 
$V^{\mathrm{OP}}(t)$. 
Отсюда аналитическим мо\-де\-ли\-ро\-ванием определяются стационарные
режимы, а также режимы их установления. Так же, как в\linebreak случае
автокоррелированных гауссовских белых шумов~\cite{1-sin, 2-sin, 15-sin}, точность МСЛ
за счет <<профильтрованности>> помех значительно повышается и
достигает 2\%--4\%. Результат справедлив и для произвольных
негауссовских белых шумов.

\medskip

\noindent
\textbf{Пример 3}.\  Для релейного осциллятора в гауссовской стохастической среде
\begin{equation}
\ddot X + \w^2 {\mathrm{sgn}} X = -\delta^G \dot X + V^G + U_0\label{e7.11-sin}
\end{equation}
плотность распределения стационарного режима стохастических колебаний при $U_0\hm=0$ 
определяется формулой Гиббса~[1, 2]:
\begin{equation}
f^* (x,\dot x) = c \exp \lf - 
    \fr{H(x,\dot x)}{\vartheta^G}\rf\,,\enskip \vartheta^G = 
    \fr{\nu^G}{ 2\delta^G}\,.\label{e7.12-sin}
    \end{equation}
Здесь через
\begin{equation*}
H(x,\dot x) = \fr{\dot x^2}{2} +\Pi(x)\,,\enskip \Pi (x) =\w^2 |x|\,, %\label{e7.13-sin}
\end{equation*}
обозначена полная энергия осциллятора.

Для~(\ref{e7.11-sin}) при  $U_0\hm\ne 0$, если заменить релейную характеристику 
статистически линеаризованной, согласно~[1, 2]
\begin{equation*}
\mathrm{sgn}\, X = k_0 (m_X, D_X) m_X + k_1 (m_X, D_X) (X^0 - m_X)\,; %\label{e7.14-sin}
\end{equation*}
    $$
    k_0(m_X, D_X) =\fr{2}{ m_X} \Phi \left( \fr{m_X}{\sqrt{D_X}}\right)\,;
    $$
    $$ 
    k_1 (m_X,D_X) = \fr{1}{\sqrt{D_X}} \sqrt{\fr{2}{\pi}}\, \exp \left( -\fr{m_X^2}{2D_X}\right)\,;
    $$
\begin{equation}
\Phi (\tau) = \fr{1}{2\pi} \int\limits_0^\tau e^{-t^2/2} dt\,.\label{e7.15-sin}
\end{equation}
Тогда уравнения МСЛ будут иметь вид:
\begin{equation}
\left.
\begin{array}{rl}
\dot m_X &= m_{\dot X}\,;\\[9pt]
\dot m_X &= U_0 - \w^2 k_0 (m_X, D_X) m_X -\delta m_{\dot X}\,;
\end{array}
\right\}
\label{e7.16-sin}
\end{equation}
    \begin{equation}
\left.
\hspace*{-3.5mm}\begin{array}{c}
    \dot D_X = 2 K_{X\dot X}\,;
\\
    \dot D_{\dot X} = \nu^G - 2\lk \delta D_{\dot X} + \w^2 k_1(m_X,D_X) K_{X\dot X}\rk\,;\\[9pt]
    \dot K_{X\dot X} = D_{\dot X} - \w^2 k_1 (m_X, D_X) D_X - \delta K_{X\dot X}\,,
    \end{array}
    \right\}\!\!
    \label{e7.17-sin}
    \end{equation}
где $\delta \hm= \delta^G$, $\nu\hm=\nu^G$.
Отсюда для стационарных стохастических колебаний имеем связанную систему уравнений:
\begin{equation}
m_{\dot X}^* =0\,;\enskip \w^2 k_0 (m_X^*, D_X^*) = U_0\,;\label{e7.18-sin}
\end{equation}
\begin{equation}
\left.
\begin{array}{c}
K_{X\dot X}^* =0\,;\enskip 
D_X^* =\vartheta=\displaystyle \fr{\nu}{ 2\delta}\,;\\[9pt]
k_1(m_X^*, D_X^*) D_X^* =\rho= \displaystyle \fr{\vartheta}{\w^2} =\fr{\nu}{ 2\delta \w^2}\,.
\end{array}
\right\}
\label{e7.19-sin}
\end{equation}

При $U_0 =0$ из~(\ref{e7.15-sin}), (\ref{e7.18-sin}) и~(\ref{e7.19-sin}) находим:
\begin{equation*}
m_X^* =0\,;\enskip 
m_{\dot X}^* =0\,; \enskip 
D_{\dot X}^* =\vartheta\,;\enskip 
D_X^* =  \fr{\pi}{2}\,\rho^2\,. %\label{e7.20-sin}
\end{equation*}
Отсюда видно, что стационарная дисперсия скорости совпадает с точным
решением~(\ref{e7.12-sin}). Стационарная дисперсия координаты, найденная
согласно МСЛ, отличается от следующего точного решения, полученного
согласно~(\ref{e7.12-sin}). При $\rho\hm \le 1$ относительная ошибка составляет
10\%. Стационарные колебания по~$X$ и $\dot X$ не коррелированы.

Уравнения~(\ref{e7.16-sin}) и~(\ref{e7.17-sin}) показывают, что процесс установления 
режима стохастических колебаний происходит в две стадии: сначала устанавливается 
стационарное распределение по ско\-рости~$\dot X$, а затем по координате~$X$.

\medskip

\noindent
\textbf{Пример 4}.  В~условиях примера~3, но для пуассоновской среды, когда
    \begin{equation*}
    \ddot X +\w^2 {\mathrm{sgn}} X =-\delta^{\mathrm{OP}} \dot X + 
    V^{\mathrm{OP}} + U_0\,,
%    \label{e7.21-sin}
    \end{equation*}
уравнения МСЛ имеют вид~(\ref{e7.16-sin}), (\ref{e7.17-sin}), если принять 
$\delta\hm= \delta^{\mathrm{OP}}$, $ \nu\hm=\nu^{\mathrm{OP}}$, 
$\vartheta\hm=\vartheta^{\mathrm{OP}}\hm=\nu^{\mathrm{OP}}/(2\delta^{\mathrm{OP}})$, 
$\rho \hm=\vartheta^{\mathrm{OP}}/\w^2$. Точного аналитического уравнения 
Фел\-ле\-ра--Кол\-мо\-го\-ро\-ва не обнаружено.

Другие тестовые примеры можно найти в~[10, 12--14].

\section{Заключение}

Дано обобщение точных и приближенных (основанных на параметризации распределений)\linebreak 
методов и алгоритмов теории распределений с инвари\-антной мерой на случай нелинейных 
дифференциальных гауссовых и негауссовых стохастических систем с гладкими и разрывными 
характеристиками.

Особое внимание уделено пуассоновским стохастическим системам с разрывными характеристиками.

На тестовых примерах показана достаточная точность для практических приложений в стохастической 
информатике.

{\small\frenchspacing
{%\baselineskip=10.8pt
\addcontentsline{toc}{section}{Литература}
\begin{thebibliography}{99}
\bibitem{1-sin}
\Au{Пугачёв В.\,С., Синицын И.\,Н.} Стохастические дифференциальные системы. 
Анализ и фильтрация.~--- 2-е изд., доп.~--- М.: Наука, 1990.

\bibitem{2-sin}
\Au{Пугачёв В.\,С., Синицын И.\,Н.} Теория стохастических систем.~--- 2-е изд.~--- М.: Логос,  2004.

\bibitem{3-sin}
\Au{Moshchuk N.\,K., Sinitsyn I.\,N.} On stationary distributions in nonlinear 
stochastic differential systems: Preprint.~--- Coventry, UK: 
University of Warwick, Mathematics Institute, 1989. 15~p.

\bibitem{4-sin}
\Au{Moshchuk N.\,K., Sinitsyn I.\,N.} On stochastic nonholonomic systems: Preprint.~--- 
Coventry, UK: University of Warwick, Mathematics Institute, 1989. 32~p.

\bibitem{5-sin}
\Au{Мощук Н.\,К., Синицын И.\,Н.} О~стохастических неголономных системах~// 
Прикладная механика и математика, 1990. Т.~54. Вып.~2. С.~213--223.

\bibitem{6-sin}
\Au{Moshchuk N.\,K., Sinitsyn I.\,N.} On stationary distributions in 
nonlinear stochastic differential systems~// Quart. J. Mech. Appl. Math., 1991. Vol.~44.  
Pt.~4.  P.~571--579.

\bibitem{7-sin}
\Au{Мощук Н.\,К., Синицын И.\,Н.} О~стационарных и приводимых к стационарным 
режимах в нормальных стохастических системах~// 
Прикладная механика и математика, 1991. Т.~55. Вып.~6. С.~895--903.

\bibitem{8-sin}
\Au{Мощук Н.\,К., Синицын И.\,Н.} Распределения с инвариантной мерой в механических 
стохастических нормальных сис\-те\-мах~// Докл. АН СССР, 1992. Т.~322. №\,4. С.~662--667.

\bibitem{9-sin}
\Au{Синицын И.\,Н.} Конечномерные распределения с инвариантной мерой в стохастических 
механических сис\-те\-мах~// Докл. РАН, 1993. Т.~328. №\,3. С.~308--310.

\bibitem{13-sin} %10
\Au{Soize C.} The Fokker--Plank equation for stochastic dynamical systems 
and its explicit steady state solutions.~--- Singapore: World Scientific,  1994.

\bibitem{10-sin} %11
\Au{Синицын И.\,Н.} Конечномерные распределения с инвариантной мерой в 
стохастических нелинейных дифференциальных системах.~--- М.: Диалог--МГУ, 1997. С.~129--140.

\bibitem{11-sin} %12
\Au{Синицын И.\,Н., Корепанов Э.\,Р., Белоусов~В.\,В.} 
Точные методы расчета стационарных режимов с инвариантной мерой в стохастических 
сис\-те\-мах управ\-ле\-ния~// Кибернетика и технологии XXI~ве\-ка: Тр.\ II Междунар. 
науч.-техн. конф. C\&T'2002.~--- Воронеж: Саквое, 2002. С.~124--131.

\bibitem{12-sin} %13
\Au{Синицын И.\,Н., Корепанов Э.\,Р., Белоусов~В.\,В.} 
Точные аналитические методы в статистической динамике нелинейных 
ин\-фор\-ма\-ци\-он\-но-управ\-ля\-ющих сис\-тем~// Сис\-те\-мы и средства информатики. 
Спец. вып. Математическое и алгоритмическое обеспечение 
ин\-фор\-ма\-ци\-он\-но-те\-ле\-ком\-му\-ни\-ка\-ци\-он\-ных сис\-тем.~--- М.: Наука, 2002. С.~112--121.

\bibitem{14-sin}
\Au{Синицын И.\,Н.} Развитие методов аналитического моделирования распределений с 
инвариантной мерой в стохастических сис\-те\-мах~// Современные проб\-ле\-мы 
прикладной математики, информатики и автоматизации: Тр. Междунар. науч.-техн. семинара.~--- 
Севастополь, 2012. С.~24--35.

\bibitem{15-sin}
\Au{Синицын И.\,Н.} Аналитическое моделирование распределений с инвариантной мерой 
в стохастических сис\-те\-мах с автокоррелированными шумами~// 
Информатика и её применения, 2012. Т.~6. Вып.~4. С.~4--8.

\bibitem{16-sin}
\Au{Немыцкий В.\,В., Степанов В.\,В.} Качественная теория дифференциальных уравнений.~--- 
М.--Л.: Гостехиздат, 1949.


\bibitem{17-sin}
\Au{Козлов В.\,В.} О~существовании интегрального инварианта гладких динамических систем~// 
ПММ, 1987. №\,1. С.~538--545.

\label{end\stat}

\bibitem{18-sin}
\Au{Синицын И.\,Н.} Фильтры Калмана и Пугачёва.~--- 2-е изд.~--- М.: Логос, 2007.
\end{thebibliography}
}
}

\end{multicols}    %1Abst+avt
\def\stat{chertok}

\def\tit{МЕТОД КУМУЛЯТИВНЫХ СУММ ДЛЯ~ПОИСКА СМЕНЫ РЕЖИМА В~ПРОЦЕССЕ 
ОРНШТЕЙНА--УЛЕНБЕКА\\ НА~ОСНОВЕ ПРОЦЕССА ЛЕВИ$^*$}

\def\titkol{Метод кумулятивных сумм для поиска смены режима в~процессе 
Орнштейна--Уленбека на основе процесса Леви}

\def\aut{А.\,В.~Черток$^1$, А.\,И.~Каданер$^2$, Г.\,Т.~Хазеева$^3$, И.\,А.~Соколов$^4$}

\def\autkol{А.\,В.~Черток, А.\,И.~Каданер, Г.\,Т.~Хазеева, И.\,А.~Соколов}

\titel{\tit}{\aut}{\autkol}{\titkol}

\index{Черток А.\,В.}
\index{Каданер А.\,И.}
\index{Хазеева Г.\,Т.}
\index{Соколов И.\,А.}
\index{Chertok A.\,V.}
\index{Kadaner A.\,I.}
\index{Khazeeva G.\,T.} 
\index{Sokolov I.\,A.}


{\renewcommand{\thefootnote}{\fnsymbol{footnote}} \footnotetext[1]
{Работа выполнена при частичной 
финансовой поддержке РФФИ (проект 14-07-00041).}}


\renewcommand{\thefootnote}{\arabic{footnote}}
\footnotetext[1]{Факультет вычислительной математики и~кибернетики 
Московского государственного университета им.\ М.\,В.~Ломоносова; Сбербанк России, 
\mbox{avchertok.sbt@sberbank.ru}}
\footnotetext[2]{Механико-математический 
факультет Московского государственного университета им.\ М.\,В.~Ломоносова; 
Сбербанк России, \mbox{aikadaner.sbt@sberbank.ru}}
\footnotetext[3]{Факультет вычислительной математики и~кибернетики 
Московского государственного университета им.~М.\,В.~Ломоносова, 
\mbox{gelana.khazeyeva@gmail.com}}
\footnotetext[4]{Институт проб\-лем информатики Федерального 
исследовательского центра <<Информатика и~управ\-ле\-ние>> Российской академии наук, 
\mbox{isokolov@ipiran.ru}}

\vspace*{-3pt}

\Abst{Рассматривается процесс Орн\-штей\-на--Улен\-бе\-ка (ОУ) с~трендом 
на основе процесса Леви для моделирования финансовых временных рядов. 
Продемонстрировано, что использование процесса Леви в~основе процесса 
ОУ дает больше гибкости для описания финансовых 
временных рядов по 
сравнению с~классической гауссовой моделью. В~частности, процесс Леви позволяет 
моделировать остатки с~тяжелыми хвостами, что является  распространенным 
свойством реальных временных рядов. Приводятся эффективные решения для 
оценивания параметров модели с~использованием таких методов, как OLS (ordinary least squares)
и~RLS (regularized least squares). 
Решается задача поиска моментов смены режима в~модели при условии поступления 
данных в~режиме реального времени. Приведен алгоритм, основанный на  
CUSUM (CUmulative SUM) ме\-то\-дах,  способный последовательно обрабатывать смены режима и~поддерживать 
параметры модели актуальными для каждого момента времени.  Решение задачи поиска 
разладки модели и~соответствующих смен режима имеет важное прикладное значение, 
поскольку в~большинстве случаев параметры моделей, описывающих динамику реальных 
систем, меняются во времени под действием внешних факторов.}

\KW{случайные процессы; процессы со свойством возвратности к~среднему; 
процесс Орн\-штей\-на--Улен\-бе\-ка, управляемый процессом Леви; процесс 
Орн\-штей\-на--Улен\-бе\-ка с~трендом; смена режима; CUSUM-ал\-го\-ритмы}

\DOI{10.14357/19922264160405}

\vspace*{-3pt} 


\vskip 10pt plus 9pt minus 6pt

\thispagestyle{headings}

\begin{multicols}{2}

\label{st\stat}

\section{Введение}

Процессы со свойством возвратности к~среднему играют важную роль в~моделировании 
динамики явлений из различных областей человеческой деятельности.  В~частности, 
эти процессы привлекательны для моделирования различных явлений в~эконометрике, 
таких как процентные ставки, курсы обмена валют и~цены на сырьевые товары, где 
свойство возвратности к~среднему имеет фундаментальную природу. 

В~работе~\cite{brigo2007} рассмотрено несколько видов случайных процессов со свойством 
возвратности к~среднему и~описаны их основные характеристики.
В~настоящей статье в~качестве такого процесса рассматривается процесс 
ОУ, управляемый процессом Леви. 

Классическая версия 
процесса была 
впервые представлена в~совместной работе голландских физиков Л.\,С.~Орнштейна 
и~Дж.\,Е.~Уленбека~\cite{ou1930} в~качестве модели, которая способна описать данные 
с~гауссовской и~диффузионной структурой. В~экономике же классический процесс 
ОУ известен как модель Васичека благодаря фундаментальной 
работе~\cite{vasicek1977}, где автор предлагает использовать ее для 
моделирования временн$\acute{\mbox{о}}$го ряда процентной ставки. Ее основной недостаток 
заключается в~том, что существует ненулевая вероятность появления отрицательных 
значений, нереалистичных для экономических процессов. Для решения данной 
проблемы позднее была разработана экспоненциальная модель Васичека, а~также 
модель процесса Кок\-са--Ин\-гер\-сол\-ла--Рос\-са, также называемая <<мо\-делью 
с~квад\-рат\-ным корнем>>, в~которой процентная ставка принимает только 
неотрицательные значения и~имеет гам\-ма-рас\-пре\-де\-ле\-ние~\cite{cox1985}.

        \begin{figure*} %fig1
        \vspace*{1pt}
\begin{center}
\mbox{%
\epsfxsize=109.749mm
\epsfbox{che-1.eps}
}
\end{center}
\vspace*{-9pt}
 \Caption{График соотношения цен для фьючерсов компаний 
<<Лукойл>> и~<<Роснефть>>}
                \label{rtsmixpic}
        \end{figure*}


В настоящей статье подтверждается тот факт, что предположение нормальности 
в~классической версии процесса ОУ не описывает реальную структуру 
данных, и~поэтому рассматривается обоб\-щение классического процесса~--- процесс 
ОУ,\linebreak управ\-ля\-емый процессом Леви. Некоторые его модификации 
изучены в~работе~\cite{GarOlk2000}. Предложено рас\-смат\-ри\-вать нормальный обратный 
гауссовский и~дис\-пер\-сионный гам\-ма-про\-цесс для описания динамики его остатков. 
Распределения прираще\-ний этих процессов имеют хвосты тяжелее, чем у~нормального 
распределения, что часто встречается в~реальных данных. 

Дополнительная мотивация 
в~использовании именно этих распределений исходит из приложений в~финансах. 
Например, дисперсионное гам\-ма-рас\-пре\-де\-ле\-ние используется для моделирования цен 
акций, как это делается в~работе~\cite{Fin2009}, а~нормальное обратное 
гауссовское распределение хорошо описывает логарифмические приращения цен 
активов, например в~работе А.\,В.~Кузьминой~\cite{Kuzmina2011} это подтверждается 
на примере данных о~цене фьючерса RTS.

Для более общего механизма построения моделей к~классической модели ОУ 
добавляется линейная составляющая, или тренд. Такой подход позволяет 
моделировать большее число явлений, не выходя за рамки одной модели.

Как известно, финансовые рынки являются динамическими и~нестационарными 
системами. Поэтому отношения, связывающие различные факторы рынка, склонны 
меняться во времени. Пример данного явления продемонстрирован на рис.~\ref{rtsmixpic}. 
По оси~$x$ отложены цены фьючерса на акции компании 
<<Роснефть>> (ROSN), а~по оси~$y$~--- цены фьючерса на акции компании <<Лукойл>> 
(LKOH) с~08.01.2013 по~28.10.2016. Видно, что параметры этой зависимости 
являются также изменяющимися во времени на протяжении дня, так как можно 
отчетливо выделить области, где точки группируются в~окрестностях прямых 
с~разными параметрами.

Все это ставит задачу определения моментов, в~которые предложенная для описания 
данных модель с~определенными параметрами перестает работать, после чего процесс 
начинает следовать той же самой модели, но уже с~другими параметрами. В~данной 
статье эта проблема решена для модели\linebreak
 ОУ. Более того, предлагается процедура 
оценивания параметров  и~обнаружения смен режима в~реаль\-ном времени 
с~использованием RLS или рекурсивного метода наименьших квадратов для оценивания 
параметров, а~также алгоритм, основанный на CUSUM-про\-це\-ду\-рах для обнаружения 
смен режима. В~конце статьи предложенная процедура применяется на различных 
данных.


\section{Моделирование временного ряда}

        \subsection {Одномерный процесс Орнштейна--Уленбека}

       Процесс ОУ с~трендом, управляемый процессом Леви, 
определяется как решение стохастического дифференциального уравнения  (СДУ):       
\begin{align*}
d\left(X_t -\mu -  \nu t\right)& = -\alpha\left(X_t - \mu - \nu 
t\right) dt +  dL_{\lambda t}\,,\\ 
&\hspace{45mm}\forall\  t>0\,;  \\
X(0) &= X_0\,,  
       \end{align*}
         где $\alpha, \mu \in \mathbb{R}$; $L_t$~--- процесс Леви; $X_0$~--- 
некоторая случайная величина, независимая от~$\{L_t\}$; $\nu$ определяет 
постоянный на всем промежутке времени линейный тренд. Параметр~$\mu$ здесь 
означает долгосрочное среднее, а~$\alpha$ определяет скорость стремления 
процесса возвращаться к~своему среднему~--- тренду.

Как показано в~\cite{Protter}, данное СДУ имеет сле\-ду\-ющее решение:
\begin{multline*}
X(t) =\nu t + \mu + \exp\left(-\alpha t\right) \times{}\\
{}\times
\left(
\left(X_0 - \mu \right)+ \int\limits_0^t\exp(\alpha  s)\,dL_{\lambda s}\right)\,, 
\quad X_0 = X(0)\,,
\end{multline*}
или
\begin{multline}
X(t + \tau) =\mu +\nu  (t + \tau) +   \exp
\left(-\alpha \tau\right)\times{}\\
\!\!{}\times\! \left(\!\!
(X(t) - \mu - \nu  t) + \exp(- \alpha t)\!\int\limits_t^{t + \tau}\!\!
\exp(\alpha s)\,dL_{\lambda s}\!\right).\!\!\!\!
\label{explicit_ou}
\end{multline}
Отсюда, в~частности, следует, что $X_t$~--- марковский процесс. Еще стоит 
заметить, что данное решение единственно с~точностью до неотличимости~\cite{sato}. 
Для более подробного рассмотрения процессов Леви см.~[8, 10].

Для удобства обозначим через $Y_t\hm = X_t \hm-\mu\hm- \nu t$ соответствующий приведенный 
процесс ОУ без тренда и~имеющий нулевое среднее.

Свойство возвратности  процесса~$Y_t$ к~нулевому уровню  при $\alpha \hm> 0$ может 
быть получено из~(\ref{explicit_ou}):  если~$Y_t$ стал больше~0 в~момент 
времени~$t$, то коэффициент при~$dt$ отрицательный и~$Y_t$ будет стремиться 
немедленно вернуться к~0; аналогично происходит, если случайный процесс 
становится меньше~0.

\subsubsection{Авторегрессия и~оценка параметров}

Пусть $ X ^ * = {\left(X^*_{t_i}\right)}_ {i = 1, \ldots, N} $~--- 
наблюдения с~интервалом 
$\Delta \hm=  1$ процесса, описываемого определенной выше моделью ОУ с~трендом. 
В~дискретном случае уравнение процесса~(\ref{explicit_ou}) выглядит следующим 
образом:
        \begin{multline}
         \label{OUtrend_d}
        X_{i+1} = \mu + \mu_0\left(1 - e^{-\lambda}\right) + 
        \mu \left(1 - e^{-\lambda}\right)i+ {}\\
        {}+e^{-\lambda } X_{i} + l_i\,,
        \end{multline}
        где $l_i $~--- некоторая случайная величина с~нулевым средним.

        Соотношение~(\ref{OUtrend_d}) описывается регрессионной моделью. Запишем 
его в~виде:
        \begin{equation*} 
%        \label{OUtrend_regr}
        X_{i+1} = c + b t_i + a X_{i} + l_i\,.
        \end{equation*}

        Чтобы оценить параметры $a$, $b$ и~$c$ регрессии, можно воспользоваться 
методом наименьших квад\-ра\-тов и~получить оценки~$\hat{a}$, $\hat{b}$ и~$\hat{c}.$ 
Тогда параметры исходного процесса ОУ с~трендом можно получить 
из соотношений:
        \begin{equation*}
                \hat{\lambda} = -\fr{\ln\hat{a}}{\tau}\,;\quad
                \hat{\mu} = \fr{\hat{b}}{1-\hat{a}}\,; \quad
                \hat{\mu}_0 = \fr{\hat{c} - \hat{\mu} \tau}{1 - \hat{a}}\,.
        \end{equation*}

        Из независимости приращений также можно явно посчитать логарифмическую 
функцию правдоподобия: 
\begin{multline*}
L\left(X^*, \theta\right) =  \sum\limits_{k = 2}^n \ln 
f_{Y_i |Y_{i -1}}(X_i , \theta) = {}\\
{}=  \sum\limits_{k = 2}^n \ln 
f\left(Y_i - a Y_{i - 1} - c,\theta\right)\,,
\end{multline*}
где $\theta$~--- параметры модели.

\subsubsection{Симуляция}

     Используя соотношения авторегрессионного вида процесса ОУ, можно 
смоделировать процесс ОУ итеративно, задав некоторую начальную 
точку~$X_0$. На рис.~2 проиллюстрирован построенный 
итеративно  процесс ОУ с~положительным трендом.

{ \begin{center}  %fig2
 \vspace*{6pt}
 \mbox{%
\epsfxsize=77.781mm
\epsfbox{che-2.eps}
}
\end{center}

%\vspace*{-3pt}


\noindent
{{\figurename~2}\ \ \small{Пример процесса ОУ с~трендом ($\alpha\hm=0{,}5$, 
$\nu\hm=0{,}1$, $\mu_0\hm=0$ и~$\sigma\hm=1$)}}
}

\addtocounter{figure}{1}

\begin{table*}[b]\small
\begin{center}
\Caption{Характеристики дисперсионного гам\-ма-про\-цес\-са $V \hm= (V_t)_{t \geqslant 0}$}
\label{table1}
\vspace*{2ex}

                \begin{tabular}{|c|c|c|c|}
                        \hline
                       \tabcolsep=0pt\begin{tabular}{c}
                        Математическое\\ ожидание\end{tabular} &                         Дисперсия &
                                               Асимметрия&                         Эксцесс\\
                                               \hline
                                               &&&\\[-9pt]
                        $\theta t$  & $\left(\sigma^2 + \nu \theta^2\right) t$ 
 & $\displaystyle
                        \fr{\theta \nu \left(3 \sigma^ 2 + 2 \nu 
\theta^2\right)}{t^{{1}/{2}}} \left(\sigma^2 + \nu \theta ^2\right)^{{3/}{2}}$ 
 & $\displaystyle 3 \left( 1 + \fr{2 \nu}{t} - \nu \theta^4 
t \left(\sigma^2 + \nu \theta^2\right)^{-2} \right) $ \\
                        \hline
                \end{tabular}
        \end{center}
%\end{table*}
%\begin{table*}\small
\begin{center}
\Caption{Характеристики нормального обратного гауссовского распределения}
\label{table2}
\vspace*{2ex}

                \begin{tabular}{|c|c|c|c|}
                        \hline
                        Математическое ожидание & 
                                                Дисперсия &
                                                                        Асимметрия &
                                    Эксцесс \\
 \hline
 &&&\\[-9pt]
 $\mu + \delta \tau$ &
 $\displaystyle\fr{\delta^2(1 + \tau^2)}{\xi}$ 
& $\displaystyle\fr{3}{\tau \sqrt{\xi (1 + \tau^2)}}$ 
& $\displaystyle\fr{3}{\xi} \left(  1 + 4 \fr{\tau^2}
                        {1 + \tau^2}  \right)$  \\[8pt]
                        \hline
                \end{tabular}
        \end{center}
\end{table*}
        

\subsection{Частные случаи моделирования остатков процесса Орнштейна--Уленбека}

В работе~\cite{taufer} авторы приводят быстрые и~эффективные методики для 
симуляции различных ОУ-про\-цес\-сов, управляемых процессом Леви, а~также  
описание множества различных частных его случаев. Будем рассматривать три типа 
процессов Леви: винеровский процесс, дисперсионный гам\-ма-про\-цесс (VG), а~также 
нормальный обратный гауссовский процесс (NIG).  Оба последних процесса 
моделируют тяжелые хвосты и~принадлежат классу обобщенных гиперболических 
распределений. Они часто применяются в~финансах и~эконометрике (для 
VG см.~[12,  13]), для NIG см.~[10, 14--16]).

\subsubsection {Дисперсионный гамма-процесс}

\noindent
\textbf{Определение 2.1.}\
Случайная величина~$\xi$ имеет дисперсионное гам\-ма-рас\-пре\-де\-ле\-ние, 
если ее плотность распределения имеет вид:
\begin{multline} \label{vgpdf}
 f_\xi(x) = \int\limits_{0}^{\infty} \fr{1}{\sigma  \sqrt{2 \pi g}} 
 \exp \left( - \fr{(x - \theta g)^2}{2 \sigma^2 g}  \right) \times{}\\
 {}\times
\fr{g ^{{1}/{\nu} - 1} \exp \left( - {g}/{\nu} \right)}
{\nu  ^{{1}/{\nu}} \Gamma \left({1}/{\nu}\right) }\, dg\,,\enskip x \in \mathbb{R},
                \end{multline}
                где $\Gamma (x)$, $x\hm>0$,~--- гам\-ма-функ\-ция, 
                а~$\sigma \hm> 0$, $\nu \hm> 0$,  
$\theta \hm\in \mathbb{R}$.


        Обозначение: $\xi \sim V(\sigma, \nu, \theta)$.

\smallskip

\noindent
\textbf{Определение~2.2.}\
        Случайный  процесс  $V \hm= (V_t)_{t \hm\geqslant 0} $  с~ параметрами        
$\sigma\hm >0$, $\nu \hm> 0$, $\theta \hm\in \mathbb{R} $, заданный на вероятностном 
пространстве $ (\Omega, F, \mathbb{P}) $ со
        значениями в~$ \mathbb{R}$, называется дисперсионным гам\-ма-про\-цес\-сом, 
если $V_0\overset{\mathrm{p.n.}}{=} 0$, $V$ имеет независимые приращения и~для любых 
$s \hm\geqslant 0$, $t \hm\geqslant 0$ 
$V$ имеет стационарные приращения с~дисперсионным 
гам\-ма-рас\-пре\-де\-ле\-ни\-ем~(\ref{vgpdf}) с~параметрами $\sigma \sqrt{t}\hm > 0$, 
$\nu/t \hm> 0$ и~$t \theta\hm > 0$,~т.\,е.\
$$
V_{t+s} - V_s \overset{\mathrm{d}}{=} V_t - V_0 \sim 
V\left(\sigma \sqrt{t}, \fr{\nu}{t}, t \theta\right) \,.
$$

\smallskip

        Характеристики дисперсионного гам\-ма-про\-цес\-са $V \hm= (V_t)_{t \geqslant 0}$ 
с параметрами~$\sigma$, $\nu$ и~$\theta$ представлены в~табл.~1.


       
        В работе~\cite{madancarr} показано, что плотность дисперсионного 
        гамма-процесса  $V \hm= (V_t)_{t \geqslant 0}$ выражается аналитически с~использованием 
модифицированной функции Бесселя второго рода с~индексом~$\nu$.

        \subsubsection {Нормальный обратный гауссовский процесс}

\noindent
\textbf{Определение 2.3.}\  Случайная величина~$\eta$ имеет 
нормальное обратное гауссовское распределение с~параметрами~$\alpha$, $\beta$, 
$\delta$ и~$\mu$ ($\eta \hm\sim \mathrm{NIG}\,(\alpha, \beta, \delta, \mu)$), если ее плотность 
распределения имеет вид:
                \begin{multline*} 
%                \label{nigpdf}
                   \hspace*{-3mm}f_\eta(x, \alpha, \beta, \delta, \mu) = \fr{\alpha 
\delta}{\pi} \exp \left(\delta \sqrt{\alpha^2 - \beta^2} + \beta (x - \mu)\right) 
\times{}\\
{}\times \fr{K_1\left(\alpha \sqrt{\delta^2 + (x - \mu)^2}\right)}{\sqrt{\delta ^2 + (x - 
\mu)^2}}\,,
                \end{multline*}
                где $K_1(z) = (1/2) \int\nolimits_{0}^{\infty} \exp 
                (-({1}/{2}) z (u \hm+ u^{-1}))\,du$, $z\hm>0$,~--- 
                модифицированная функция Бесселя 
второго рода с~индексом~1, $\alpha \hm> 0$, $-\alpha \hm< \beta \hm< \alpha$, 
$\delta \hm> 0$,  $\mu \hm\in  \mathbb{R}$, $x\hm>0$.

                Параметры $\alpha$, $\beta$, $\delta$ и~$\mu$ являются параметрами 
формы, асимметрии, масштаба и~расположения соответственно. 

\smallskip

\noindent
\textbf{Определение 2.4.}\
                Случайный процесс $N \hm= (N_t)_{t \geqslant 0}$ с~параметрами 
$\alpha$, $\beta$, $\delta$ и~$\mu$, заданный на вероятностном пространстве $(\Omega, 
F, P)$ со значениями  в~$\mathbb{R}$,\linebreak
 называется нормальным обратным гауссовским 
процессом, если $N_0 \overset{\mathrm{p.n.}}{=} 0$, $N$ имеет независимые приращения 
и~для любых $s \hm\geqslant 0$, $t \hm\geqslant 0$ $N$ имеет стационарные приращения 
с~нормальным обратным гауссовским распределением:
$$
    N_{t+s} - N_s  \overset{\mathrm{d}}{=} N_t - N_0 \sim 
\mathrm{NIG}\,( \alpha, \beta, \delta t, \mu t) 
  $$
        с~параметрами $\alpha \hm> 0$, 
        $-\alpha \hm< \beta \hm< \alpha$, $\delta t\hm > 0$ 
и~$\mu t \hm\in  \mathbb{R}$.

\smallskip

    Плотность нормального обратного гауссовского распределения может быть 
представлена в~аналитической форме.

        Характеристики нормального обратного гауссовского распределения 
представлены в~табл.~\ref{table2}.



\section {Оценивание параметров и~поиск смен режима в~реальном времени}

В~данной разделе рассмотрено моделирование и~описание данных при условии их 
поступления в~режиме реального времени, когда значения выборки данных поступают 
одно за другим. Специфика данной задачи заключается в~высокой скорости 
поступления данных в~ее приложениях и~их большом объеме, поэтому любые 
приводимые алгоритмы должны быть достаточно быстрыми и~эффективно использовать 
компьютерную память.

\subsection{Оценивание параметров}

Без каких-ли\-бо ограничений на компьютерные мощности самым очевидным решением для 
оценивания параметров было бы на каждом шаге использовать метод наименьших 
квадратов (OLS). Чтобы удовлетворить необходимость обрабатывать потоковые 
данные, воспользуемся рекурсивным методом наименьших квадратов (RLS). Данный 
алгоритм на каждом шаге обновляет рекурсивно оценку параметра~$\theta$ 
и~ковариационную матрицу~$X^{\mathrm{T}} X$  вмес\-то того, чтобы насчитываться с~нуля каждый 
раз. Данный алгоритм и~его реализация хорошо известны и~могут быть найдены 
в~\cite{haykin}.

\subsection{Постановка простейшей смены режима}

        В реальной жизни некоторые явления могут быть связаны отношениями, 
например линейными, параметры которых изменяются во времени. Самым простым 
подобным примером является процесс, описываемый следующим образом:      
 \begin{equation*}
        M_t= 
                        \begin{cases}
                M_t^1\,, &\ t \leqslant t^*\,; \\
                M_t^2\,, &\ t > t^*\,,
                \end{cases}
        \end{equation*}
        где $t \in [1,\ldots,T]$ обозначает время; $ t^*$~--- критическое 
значение внешней переменной~$t$, или момент смены режима (change point, regime 
switch);  $M^{1,2}$~--- это две различные модели, соответствующие разным 
временным промежуткам: до и~после.
        В общем случае нельзя точно определить значение~$t^*$. Задача состоит 
        в~том, чтобы наилучшим образом оценить ее значение, имея на входе выборку 
наблюдений, при условии что на данном временн$\acute{\mbox{о}}$м промежутке произошла ровно одна 
смена режима (рис.~3), а~также оценить параметры старой и~новой модели.   
В~данном случае 
рассматривается модель ОУ и~исследуемый процесс выглядит 
следующим образом:

\columnbreak

\noindent
 \begin{center}  %fig1
 \vspace*{-2pt}
 \mbox{%
\epsfxsize=77.781mm
\epsfbox{che-3.eps}
}

\vspace*{3pt}

\noindent
{{\figurename~3}\ \ \small{Пример смены режима}}
\end{center}



 \vspace*{6pt}



\addtocounter{figure}{1}


\noindent
\begin {equation*}
        X_t =                \begin{cases}
                \mathrm{OU}_t^1\,, &\ t \leqslant t^*\,; \\
                \mathrm{OU}_t^2\,, &\ t > t^*\,,
                \end{cases}
        \end {equation*}
        где $\mathrm{OU}^i$~--- процессы ОУ, описанные выше.

        

\subsection{Постановка задачи для~потоковых данных}

            В случае потока данных значения выборки $X_1,X_2,\ldots,X_n, \ldots$ 
поступают последовательно. В~этом случае нет предпосылок для того, чтобы 
в~ка\-кой-то момент произошла смена режима, а~сами смены могут происходить 
последовательно много раз. Задача состоит в~том, чтобы последовательно их 
обнаруживать и~предоставлять оценку для параметров новой модели. Эффективность 
методов определяется тем, как часто метод ошибочно определяет режимы и~как 
быстро он способен обнаруживать смену режима.

\subsection{Решение задачи}

В современной литературе можно найти множество методов  для определения смен 
режима. Основным применяемым подходом является по\-стро\-ение по наблюдаемой системе  
некоторого детектора (change-detector), который сигнализирует, когда параметры 
модели перестают соответствовать выборке наблюдений и~предположительно сменился 
режим. Одним из таких методов  является CUSUM-тест, или метод кумулятивных сумм, 
который рассматривается в~данной работе для обнаружения смен режима. Стоит 
отметить другие известные методы, такие как метод GLT (generalized likehood 
test)~\cite{appel} и~MLT (marginalized likelihood text)~\cite{gustafsson}.

\subsubsection{Краткое описание CUSUM-методов}

В данной статье будут рассмотрены два базовых CUSUM-ме\-то\-да: CUSUM-ме\-тод, 
основанный на максимизации правдоподобия выборки наблюдений, и~CUSUM-ме\-тод, 
определяющий смену среднего значения выборки наблюдений.

\subsubsection*{CUSUM-метод максимизации правдоподобия}

Пусть есть некоторый поток данных $X^* \hm= X^*_1,\ldots X^*_n,\ldots $, 
элементы которого 
являются выборкой независимых одинаково распределенных случайных величин. 
Обозначим плотность соответствующей случайной величины через $p_\theta(x).$ 
Предполагается, что в~ка\-кой-то момент времени~$r^*$ может произойти смена режима 
модели. Это означает, что до~$r^*$ в~модели действует параметр~$\theta_0$, 
а~после~---~$\theta_1$. Введем соответствующие гипотезы: гипотезу <<неизменности>> 
модели~$H_0$ (разладки не произошло) и~гипотезу~$H_1$ об <<одноразовой 
разладке>>. Одним из самым известных и~простых методов для нахождения разладки 
модели является тест отношения правдоподобий~\cite{kay}.

\noindent
\textbf{Алгоритм 3.1.}\ %\begin{algorithm}
 Определим логарифмическое отношение правдоподобий для моделей~$H_0$ и~$H_1$: 
\begin{equation*}
\mathrm{LLR} \left(X, r^*\right) = 
\ln\fr{p_{H_1}(X)}{p_{H_0}(X)}\,.
\end{equation*} 
Тогда  принимается гипотеза~$H_1$, если $\mathrm{LLR} \hm>h$, где параметр~$h$  
отвечает за чувствительность 
алгоритма: чем меньше~$h$, тем быстрее будет происходить обнаружение разладки, 
но при этом тем больше будет срабатывать ложных сигналов.

\smallskip

К~сожалению,  в~данном случае неизвестно значение~$r^*$ и~поэтому явно посчитать 
$p_{H_1}(X)$ не представляется возможным. Данную проблему можно решить, перебрав 
все значения~$r^*$ и~взяв то, которому соответствует максимальное значение~$\mathrm{LLR}$. 
Данный метод называется обобщенным методом максимального 
правдоподобия (GLT).


\smallskip

\noindent
\textbf{Алгоритм 3.2.} %\begin{algorithm}\label{algo2}
Определим обобщенное логарифмическое отношение 
правдоподобий для выборки размера~$N$: 
\begin{multline*}
\mathrm{GLLR}(X) =\max\limits_{1   \leqslant r^* \leqslant N } 
\mathrm{LLR}(X, r^*)  ={}\\
{}=\max\limits_{1  \leqslant r^* \leqslant N } 
\ln\fr{p_{H_1}(X)}{p_{H_0}(X)}  =\max\limits_{1  \leqslant r^* \leqslant N } 
\sum\limits_{i = r^*}^N \ln \fr{p_{\theta_1}(X_i)}{p_{\theta_0}(X_i)}\,. 
%\label{gllr} 
\end{multline*}
Тогда на каждом шаге~$n$ принимается гипотеза~$H_1$  против гипотезы~$H_0$, если 
$\mathrm{GLLR}(X)\hm>h.$

\smallskip

Введем  кумулятивную сумму точечных отношений правдоподобий:
\begin{equation*}
S(n) = \sum\limits_{i = 1}^n \ln  
\fr{p_{\theta_1}(X_i)}{p_{\theta_0}(X_i)}\,. 
\end{equation*}
Тогда 
\begin{align*}
\mathrm{LLR} (X, r^*) &= S(N) - S(r^*)\,; \\
\mathrm{GLLR}(X, N) &= S(N) - \min\limits_{1  \leqslant r^* \leqslant N  }S(r^*)\,,
\end{align*}
где
$$
\hat{r}^* = \mathop{\mathrm{argmin}}\limits_{1  \leqslant r^* \leqslant N}S(r^*)\,.
$$
Заметим, что для того чтобы использовать алгоритм~3.2, достаточно 
считать кумулятивную сумму~$S$. Более того, так как уровень~$h$ берется 
положительным, вместо того, чтобы явно насчитывать значение~$\mathrm{GLLR}$ на 
каждом шаге, достаточно рекурсивно считать функцию 
\begin{equation*}
 G(N)  =\max\left(G(N- 1) +\ln \fr{p_{\theta_1}(X_N)}{p_{\theta_0}(X_N)}\,, 
\;0\right),
\end{equation*}
которая совпадает с~$\mathrm{GLLR}$ там, где последняя положительна. В~этом 
и~заключается метод кумулятивных сумм. Из вышесказанного вытекает следующий 
алгоритм, эквивалентный алгоритму~3.2:

\smallskip

\noindent
\textbf{Алгоритм 3.3.} %\begin{algorithm}
На каждом шаге~$N$ принимается гипотеза~$H_1$ против гипотезы~$H_0$, если 
$G(N)\hm>h.$


\smallskip

На практике значение~$\theta_0$ можно оценить, а~значение~$\theta_1$ неизвестно. 
Поэтому  берут $\theta_1 \hm= \theta_0 \hm+ \delta$, где~$\delta$~--- минимальная 
величина, изменение которой хотят детектировать.

\subsubsection*{CUSUM-метод определения смены среднего выборки}

%\smallskip

\noindent
\textbf{Алгоритм 3.4.} %\begin{algorithm}
Определим рекурсивно 
\begin{equation*} 
S_n^+=\max\left (S_{n- 1}^+ + 
\fr{X_N - \mu_0}{\sigma} - k, \;0\right)\,, 
\end{equation*}
где $\mu_0$~--- среднее текущей модели; $\sigma$~--- среднее текущей выборки; $k$~--- 
уровень чувствительности метода к~разбросам. Тогда считается, что принимается 
гипотеза~$H_1$, если $S^+_n \hm> h.$

\smallskip

Подробнее с~алгоритмом можно 
ознакомиться, например, в~\cite{page61}.

\begin{figure*}[b] %fig4
 \vspace*{1pt}
 \begin{minipage}[t]{79mm}
\begin{center}
\mbox{%
\epsfxsize=77.835mm
\epsfbox{che-5.eps}
}
\end{center}
\vspace*{-9pt}
  \Caption{Частичная автокорреляционная функция}\label{pacfpic}
  \end{minipage}
%\end{figure*}
\hfill
%\begin{figure*} %fig5
        \vspace*{1pt}
         \begin{minipage}[t]{79mm}
\begin{center}
\mbox{%
\epsfxsize=78.035mm
\epsfbox{che-4.eps}
}
\end{center}
\vspace*{-9pt}
  \Caption{Автокорреляционная функция}\label{acfpic}
    \end{minipage}
\end{figure*}

\subsection{Общий алгоритм для процесса Орнштейна--Уленбека}

Как было замечено ранее, приведенный процесс ОУ с~трендом обладает 
авторегрессионным свойством AR(1). Поэтому можно применять CUSUM-ме\-тод 
максимального правдоподобия для приращений~$l_i$ данного процесса. Данный метод 
будем\linebreak
 применять для детектирования смены во\-ла\-тиль\-ности модели, в~то время как 
для обнаружения смены среднего, или тренда, будем применять CUSUM-ме\-тод поиска 
смены среднего. Применить первый алгоритм для детектирования среднего 
оказывается сложно из-за большого числа параметров, которые нельзя адекватно 
оценить, в~частности па\-ра\-мет\-ров~$\mu_0$ и~$\nu$.

Таким образом, общий алгоритм следующий:

\smallskip

\noindent
\textbf{Алгоритм 3.5.} %%rithm}

\noindent
\begin{enumerate}[1.]
\item На каждом шаге оцениваем наиболее вероятные параметры выборки~$\theta_0$ 
для выбранной модели с~помощью метода RLS.
\item На каждом шаге считаем значения детекторов CUSUM смены волатильности 
и~смены тренда.
\item В случае, когда детекторы сигнализируют\linebreak о~смене режима, проходимся общим 
методом обобщенного отношения правдоподобий (GLT) по выборке и~находим наиболее 
вероятную точку смены режима~$r^*$ с~оценкой параметров~$\theta_1$. Далее 
исключаем из выборки все ее элементы до~$r^*$ и~продолжаем процедуру алгоритма 
с~$\theta_0:=\theta_1$.
\end{enumerate}


\section{Анализ данных}

       В этом разделе описывается моделирование процессом ОУ 
реальных финансовых данных. В~качестве данных были выбраны секундные данные по 
ценам фьючерсов на индекс RTS ($x_t$) и~на акции компании <<Газпром>> ($y_t$) 
с~MOEX за~07.10.2014. Если наблюдения сделаны через равные промежутки времени, то 
можно рассматривать их как временной ряд.
Предполагается, что разность $z_t \hm= x_t \hm - 6  y_t$ обладает свойством 
стационарности и~может быть описана с~помощью процесса ОУ. Для 
того чтобы ряд имел свойство авторегрессии, вычитаем из ряда его скользящее 
среднее с~периодом 5~мин.

Тест Дики--Фуллера (с уровнем зна\-чи\-мости 0,05) подтверждает предположение 
о~стационар\-ности: значение статистики Ди\-ки--Фул\-ле\-ра: $-25{,}374$; $p$-зна\-че\-ние: 
0,001. Тест  отвергает нулевую гипотезу о существовании единичного корня 
с~уровнем значимости~0,05, что подтверждает стационарность данного ряда.  Для 
проверки наличия свойства AR(1) проанализируем вид автокорреляционной (ACF) 
и~частичной автокорреляционной (PACF) функций.

         Для модели AR(1) характерен следующий вид автокорреляционных 
         и~частичных автокорреляционных функций: график ACF экспоненциально убывает, 
         а~график PACF имеет пик при значении сдвига, равном~1, и~практически равен~0 при 
значениях сдвига более высокого порядка.

       
       На рис.~\ref{pacfpic} изображен график PACF для рас\-смат\-ри\-ва\-емых данных. 
Заметно, что он имеет пик при сдвиге~1 и~практически равен нулю для сдвигов 
более высокого порядка.
        На рис.~\ref{acfpic} ACF для исходных данных убывает экспоненциально.
        Такое поведение графиков ACF и~PACF соответствует модели AR(1).

        Теперь можно говорить о том, что данные представляют собой процесс 
ОУ, и~перейти к~оценке его параметров.

        Для начала оценим параметры~$\theta$ и~$\alpha$ с~по\-мощью метода 
наименьших квадратов, получаем оценки параметров $\hat \theta \hm= 0$ и~$\hat \alpha 
\hm= 0{,}7506$.


        Для того чтобы оценить качество полученных оценок для данного процесса, 
построим QQ-плот для оцененных параметров нормального распределения для остатков 
(рис.~\ref{qqplot}). Это график, где по оси~$x$~--- квантили теоретического 
распределения, а~по оси~$y$~--- эмпирические квантили данных. Если теоретическое 
распределение хорошо описывает\linebreak\vspace*{-12pt}

\pagebreak

\end{multicols}

\begin{figure*} %fig6
        \vspace*{1pt}
\begin{center}
\mbox{%
\epsfxsize=161.601mm
\epsfbox{che-6.eps}
}
\end{center}
\vspace*{-11pt}
\Caption{Графики QQ-plot для оценивания параметров}\label{qqplot}
\vspace*{-3pt}
\end{figure*}

\begin{multicols}{2}

\noindent
 реальные данные, то график <<кван\-тиль--кван\-тиль>> 
близок к~прямой $y \hm= x$.


        Видно, что нормальное распределение не очень\linebreak хорошо описывает 
распределение остатков. По\-пробуем вместо нормального распределение\linebreak использовать 
распределение с~более тяжелыми хвостами, например дисперсионное 
гам\-ма-рас\-пре\-де\-ле\-ние и~нормальное обратное гауссовское распределение. Для оценки 
параметров этих распределений используем метод максимального правдоподобия, 
описанный в~п.~3.2.3.

       

        Анализируя графики QQ-plot (см.\ рис.~\ref{qqplot}) для оцененных параметров 
дисперсионного гамма- и~нормального обратного гауссовского распределений для 
остатков, можно прийти к~выводу, что дисперсионное гамма- и~нормальное обратное 
гауссовское распределение лучше описывают структуру независимых приращений 
в~процессе~ОУ.

        Для того чтобы оценить качество полученных результатов, применим критерий 
согласия Колмогорова для новой выборки данных. Результат подсчета статистики 
представлен в~табл.~3.



        По результатам, представленным в~табл.~3, \mbox{можно} заключить, 
что гипотеза о нормальном рас\-пре\-делении остатков отвергнута при уровне 
зна\-чи\-мости~0,01, гипотеза о~дисперсионном гам\-ма-рас\-пре\-де\-ле\-нии остатков 
и~нормальном обратном гауссовском распределении остатков принята при уровне 
значимости~0,01.

\vspace*{12pt}

\noindent
{{\tablename~3}\ \ \small{Оценки параметров с~результатом критерия Колмогорова}}

\vspace*{1pt}

{\small
 \begin{center}  %
\tabcolsep=3pt
                        \begin{tabular}{|c|c|c|c|c|}
                                \hline
                                \multicolumn{3}{|c|} {Оценка параметра} &
\tabcolsep=0pt\begin{tabular}{c} Значение\\ статистики \end{tabular}&  
$p$-значение  \\
                                \hline
\multicolumn{1}{|c|}{\raisebox{-6pt}[0pt][0pt]{$N(\mu, \sigma^2)$}}
                                & $\hat \mu$ & 0 &  &   \\ 
                              %  \cline{2-3}
                                & $\hat \sigma$ & 
3,6885 &
\multicolumn{1}{c|}{\raisebox{6pt}[0pt][0pt]{ 0,087731}} & 
\multicolumn{1}{c|}{\raisebox{6pt}[0pt][0pt]{$1{,}6679\cdot 10^{-12}$}}\\
                                \hline
                                & $\hat \sigma$ & 3,6712 &  &   \\ 
                                %\cline{2-3}
$\mathrm{VG}\,(\sigma, \nu, \theta)$& $\hat \nu$ & 1,5226 & 0,026806 &   0,14799  \\ 
%\cline{2-3}
                                & $\hat \theta$ & 0,0379 &  & 
\\
                                \hline
\multicolumn{1}{|c|}{\raisebox{-18pt}[0pt][0pt]{$\mathrm{NIG}\,(\theta, \xi, \delta, \mu)$}}
                                & $\hat \theta$ & 0,0592 &  &   \\ 
                                %\cline{2-3}
                                & $\hat \xi$ & 0,3714 &  &   \\ 
                                %\cline{2-3}
                                & $\hat \delta$ & 2,2690 &  &   \\ 
                                %\cline{2-3}
                                & $\hat \mu$ & $-$0,0963 & 
 \multicolumn{1}{c|}{\raisebox{18pt}[0pt][0pt]{0,034654}} & 
 \multicolumn{1}{c|}{\raisebox{18pt}[0pt][0pt]{0,025953}}  
\\
                                \hline
                        \end{tabular}
                        \vspace*{3pt}
\end{center}
}

\pagebreak

\end{multicols}

 \begin{figure*} %fig7
         \vspace*{1pt}
         \begin{minipage}[t]{80mm}
\begin{center}
\mbox{%
\epsfxsize=78.067mm
\epsfbox{che-7.eps}
}
\end{center}
\vspace*{-9pt}
        \Caption{Пример применения CUSUM-алгоритма}
        \label{cusum_vola}
%        \end{figure*}
\end{minipage}
\hfill
%        \begin{figure*} %fig8[H]
                 \vspace*{1pt}
                          \begin{minipage}[t]{80mm}
\begin{center}
\mbox{%
\epsfxsize=78.067mm
\epsfbox{che-8.eps}
}
\end{center}
\vspace*{-9pt}
               \Caption{Пример применения CUSUM-алгоритма}
        \label{cusum_mean}
        \end{minipage}
        \end{figure*}

\begin{multicols}{2}

        Полученный результат говорит о том, что структура реальных данных 
сложнее, чем может описать классический процесс ОУ. 
Целесообразнее использовать обобщенный процесс ОУ, где процесс 
броуновского движения заменен на процесс Леви.

        \subsection{Применение алгоритма для~детектирования изменения 
волатильности}

       Будем рассматривать гауссовский процесс ОУ. Для этого 
были сгенерированы две выборки процесса размера~100 с~$\sigma_1\hm=1$ и~$\sigma_2\hm=3.$ 
Для CUSUM-тес\-та будем брать $\theta_1\hm=2$, т.\,е.\ $\delta\hm=1$. Уровень $h\hm=45.$ 
Результат применения алгоритма проиллюстрирован на рис.~7. На 
рис.~7,\,\textit{а} изображена выборка сгенерированного процесса ОУ со сменой 
режима, в~то время как на рис.~7,\,\textit{б} построено значение CUSUM-де\-тек\-то\-ра. Сплошная 
вертикальная линяя обозначает фактическую смену режима, а~пунктирная~--- время ее 
обнаружения. Заметим, что смена режима могла бы быть обнаружена быстрее при 
другом выборе уровня~$h$.

        \subsection{Применение алгоритма для~обнаружения тренда}

        Для обнаружения тренда также были сгенерированы две выборки гауссовского 
процесса ОУ, которые потом были склеены. Параметры процессов 
следующие: $\alpha_0\hm=0{,}5$, $\nu_0\hm=0$, $\mu_0^0\hm=0$, 
$\sigma_0\hm=0{,}6$, $\alpha_1\hm=0{,}5$, 
$\nu_1\hm=0{,}05$, $\mu_0^1\hm=0$ и~$\sigma_1\hm=0{,}6$. 
Аналогично построена выборка и~значения 
CUSUM-де\-тек\-то\-ра. Уровень $h\hm=13.$ Алгоритм успешно определил смену режима 
(рис.~\ref{cusum_mean}).
       



\section {Заключение}

В статье рассмотрен процесс ОУ с~трендом, управ\-ля\-емый процессом 
Леви, для описания финансовых временн$\acute{\mbox{ы}}$х рядов.
На реальных данных было показано, что дисперсионный гамма- и~нормальный обратный 
гауссовский процессы в~качестве процесса Леви способны гораздо точнее описывать 
реальные явления. Также были рас\-смот\-ре\-ны проб\-ле\-мы разладки модели и~поиска смены 
режима в~реальном времени. Была представлена процедура обнаружения разладки 
модели, а~также определения параметров новой модели. Данный алгоритм способен 
детектировать многократные смены режима последовательно, сохраняя текущую модель 
актуальной для текущего потока данных.

{\small\frenchspacing
 {%\baselineskip=10.8pt
 \addcontentsline{toc}{section}{References}
 \begin{thebibliography}{99}
\bibitem{brigo2007}
    \Au{Brigo D., Dalessandro~A., Neugebauer~M., Triki~F.} A~stochastic 
processes toolkit for risk management.~--- London: King's College 
London, November 2007.  Working paper. 48~p.


\bibitem{ou1930}
\Au{Ornstein L.\,S., Uhlenbeck~G.\,E.} On the theory of the Brownian motion~// 
Phys. Rev., 1930. Vol.~36. No.\,5. P.~823.
    
    \bibitem{vasicek1977}
    \Au{Vasicek O.} An equilibrium characterization of the term structure~// 
J.~Financ. Econ., 1977. Vol.~5. P.~177.

\bibitem{cox1985}
        \Au{Cox J.\,C., Ingersoll E., Jr., Ross~S.\,A.} A~theory of the term 
structure of interest rates~//  Econometrica, 1985. Vol.~53. No.\,2.  P.~385--407.



\bibitem{GarOlk2000}
    \Au{Garbaczewski P., Olkiewicz~R.} Ornstein--Uhlenbeck--Cauchy process~// 
J.~Math. Phys., 2000. Vol.~41. P.~6843.

\bibitem{Fin2009}
    \Au{Finlay R.} The variance gamma (VG) model with long range dependence: 
A~model for financial data incorporating long range dependence in squared 
returns.~--- Sydney, Australia: University of Sydney, School of 
Mathematics and Statistics, 2009. PhD Thesis. 144~p.

\bibitem{Kuzmina2011}
    \Au{Кузьмина А.\,В.} Моделирование нормального обратного гауссовского 
процесса и~оценивание его параметров~// Информатика, 2011. №\,2. С.~133--136.

    \bibitem{Protter}
    \Au{Protter P.} Stochastic integration and differential equations.~--- 
Heidelberg: Springer-Verlag, 1990. 415~p.
    
           \bibitem{sato}
    \Au{Sato K.\,I.} L$\acute{\mbox{e}}$vy processes and 
    infinitely divisible distributions.~--- Cambridge: Cambridge University Press, 1999.
    500~p.
    
        \bibitem{nielsen}
\Au{Barndorff-Nielsen O.\,E., Shephard~N.} Non-Gaussian Ornstein--Uhlenbeck-based 
models and some of their uses in financial economics~// 
J.~Roy. Stat. Soc. B, 2001. Vol.~63. P.~167--241.
    
    \bibitem{taufer}
\Au{Taufer E., Leonenko~N.} Simulation of L$\acute{\mbox{e}}$vy-driven 
Ornstein--Uhlenbeck processes with given marginal distribution~// 
Comput. Stat. Data An., 2008. Vol.~53.  P.~2427--2437.

\bibitem{madanseneta}
    \Au{Madan D.\,B., Seneta~E.} The VG model for share market returns~// 
    J.~Bus., 1990. Vol.~63. P.~511--524.

\bibitem{madancarr}
\Au{Madan D.\,B., Carr P.\,P., Chang~E.\,C.} The variance gamma 
process and option pricing~// Eur. Finance Rev., 1998. Vol.~2. P.~79--105.

\bibitem{nielsen2}
 \Au{Barndorff-Nielsen O.\,E.} Normal inverse Gaussian distributions and 
stochastic volatility modelling~// Scand. J.~Stat., 1997. Vol.~24. No.\,1. P.~1--13.
    
   
    
    \bibitem{rydberg}
\Au{Rydberg H.} The Normal inverse Gaussian L$\acute{\mbox{e}}$vy process: Simulation 
and approximation~// Commun. Stat. Stochastic Models, 1997. 
Vol.~13. No.\,4. P.~887--910.

 \bibitem{nielsen3}
   \Au{Barndorff-Nielsen O.\,E.} Processes of normal inverse Gaussian type~// 
Financ. Stoch., 1998. Vol.~2. P.~41--68.

\bibitem{haykin}
    \Au{Haykin S.} Adaptive filter theory.~--- 3rd ed.~--- Upper Saddle River, NJ, USA: 
Prentice Hall, 1996. 989~p.

\bibitem{appel}
\Au{Appel U., Brandt~A.\,V.} Adaptive sequential segmentation of 
piecewise stationary time series~// Inform. Sci., 1983. Vol.~29. P.~27--56.

\bibitem{gustafsson}
\Au{Gustafsson F.} The marginalized likelihood ratio test for detecting 
abrupt changes~// IEEE Trans. Automat. Contr., 1996. Vol.~41. P.~66--78.
    
    \bibitem{kay}
    \Au{Kay S.} Fundamentals of statistical signal processing. Vol.~I. 
Estimation theory.~--- Upper Saddle River, NJ, USA: Prentice Hall, 1993. 625~p.

\bibitem{page61}
   \Au{Page E.\,S.} Cumulative sum control chart~// Technometrics, 1961. 
Vol.~3. P.~1--9.
 \end{thebibliography}

 }
 }

\end{multicols}

\vspace*{-6pt}

\hfill{\small\textit{Поступила в~редакцию 20.10.16}}

\vspace*{8pt}

%\newpage

%\vspace*{-24pt}

\hrule

\vspace*{2pt}

\hrule

\vspace*{8pt}


\def\tit{REGIME SWITCHING DETECTION FOR~THE~LEVY DRIVEN ORNSTEIN--UHLENBECK PROCESS 
USING CUSUM METHODS}

\def\titkol{Regime switching detection for the Levy driven Ornstein--Uhlenbeck process 
using CUSUM methods}

\def\aut{A.\,V.~Chertok$^{1,2}$, A.\,I.~Kadaner$^{2,3}$, G.\,T.~Khazeeva$^1$, 
and~I.\,A.~Sokolov$^4$}

\def\autkol{A.\,V.~Chertok, A.\,I.~Kadaner, G.\,T.~Khazeyeva, 
and~I.\,A.~Sokolov}

\titel{\tit}{\aut}{\autkol}{\titkol}

\vspace*{-9pt}

\noindent
$^1$Faculty of Computational Mathematics and Cybernetics, 
M.\,V.~Lomonosov Moscow State University, 1-52~Lenin-\linebreak
$\hphantom{^1}$skiye Gory, GSP-1, 
Moscow 119991, Russian Federation

\noindent
$^2$Sberbank of Russia, 19~Vavilov Str., Moscow 117999, Russian Federation

\noindent
$^3$Faculty of Mechanics and Mathematics, 
M.\,V.~Lomonosov Moscow State University, Main Building, Leninskiye\linebreak
$\hphantom{^1}$Gory, 
GSP-1, Moscow 119991, Russian Federation

\noindent
$^4$Federal Research Center ``Computer Science and Control'' 
of the Russian Academy of Sciences, 44-2~Vavilov\linebreak 
$\hphantom{^1}$Str., Moscow 119333, 
Russian Federation


\def\leftfootline{\small{\textbf{\thepage}
\hfill INFORMATIKA I EE PRIMENENIYA~--- INFORMATICS AND
APPLICATIONS\ \ \ 2016\ \ \ volume~10\ \ \ issue\ 4}
}%
 \def\rightfootline{\small{INFORMATIKA I EE PRIMENENIYA~---
INFORMATICS AND APPLICATIONS\ \ \ 2016\ \ \ volume~10\ \ \ issue\ 4
\hfill \textbf{\thepage}}}

\vspace*{3pt}



\Abste{The article considers using a trending Ornstein--Uhlenbeck process, driven 
by a~Levy process, for modeling financial time series. The authors demonstrate 
that the Levy driven model gives more flexibility to describe financial time series 
than the simple classical model. In particular, the Levy driven model allows 
modeling distributions with heavy tails, which is a~common property of time series 
in real applications. The authors describe efficient methods for estimating model 
parameters using such methods as OLS (ordinary least squares)
and RLS (regularized least squares). The article also solves the regime 
switching problem in a~real time data stream. The authors built an algorithm based 
on CUSUM (CUmulative SUM) methods that is capable of determining regime switches consecutively as 
they happen online and keep model parameters up to date. Solution of the regime 
switching problem is important in real applications, since the dynamics of real 
systems tend to change over time under the influence of external factors.} 

\KWE{random process; mean-reverting process; Ornstein--Uhlenbeck process driven 
by Levy process; trending Ornstein--Uhlenbeck process; regime switch; 
change point detection; CUSUM algorithm}



\DOI{10.14357/19922264160405} 

\vspace*{-16pt}

\Ack
\noindent
The research was partially supported by the Russian Foundation for Basic Research 
(project 14-07-00041).



%\vspace*{3pt}

  \begin{multicols}{2}

\renewcommand{\bibname}{\protect\rmfamily References}
%\renewcommand{\bibname}{\large\protect\rm References}

{\small\frenchspacing
 {%\baselineskip=10.8pt
 \addcontentsline{toc}{section}{References}
 \begin{thebibliography}{99}

\bibitem{1-ch-1}
\Aue{Brigo, D., A.~Dalessandro, M.~Neugebauer, and F.~Triki}. 2007. 
{A~stochastic processes toolkit for risk management}. 
London: King's College London.  Working paper. 48~p.
\bibitem{2-ch-1}
\Aue{Ornstein, L.\,S., and G.\,E.~Uhlenbeck}. 1930. On the theory of the Brownian motion. 
\textit{Phys. Rev.} 36(5):823.
\bibitem{3-ch-1}
\Aue{Vasicek, O.} 1977. An equilibrium characterization of the term structure. 
\textit{J.~Financ. Econ.} 5(2):177--188.
\bibitem{4-ch-1}
\Aue{Cox, J.\,C., E.~Ingersoll, Jr., and S.\,A.~Ross}. 1985. 
A~theory of the term structure of interest rates. \textit{Econometrica} 53(2):385--407.
\bibitem{5-ch-1}
\Aue{Garbaczewski, P., and R.~Olkiewicz}. 2000. Ornstein--Uhlenbeck--Cauchy process. 
\textit{J.~Math. Phys.} 41:6843--6860.
\bibitem{6-ch-1}
\Aue{Finlay, R.} 2009. The variance gamma (VG) model with long range dependence: 
A~model for financial data incorporating long range dependence in squared returns.
Sydney, Australia: University of Sydney, School of Mathematics and Statistics. 
 PhD Thesis. 144~p.
\bibitem{7-ch-1}
\Aue{Kuzmina, A.\,V.} 2011. Modelirovanie normal'nogo obratnogo gaussovskogo 
protsessa i~otsenivanie ego papametrov [Normal inverse Gaussian distribution 
modeling and its parameters estimation]. 
\textit{Vestnik Belorusskogo gosudarstvennogo universiteta. Ser.~1: Fizika. Matematika. 
Informatika} [Herald of the Belarusian State University. Ser.~1: 
Physics. Mathematics. Informatics] 2:133--136. 
\bibitem{8-ch-1}
\Aue{Protter, P.} 1990. 
\textit{Stochastic integration and differential equations.}  
Heidelberg: Springer-Verlag. 415~p.
\bibitem{9-ch-1}
\Aue{Sato, K.\,I.} 1999. \textit{L$\acute{\mbox{e}}$vy processes and infinitely divisible 
distributions.}  Cambridge: Cambridge University Press. 500~p.
\bibitem{10-ch-1}
\Aue{Barndorff-Nielsen, O.\,E., and N.~Shephard}. 2001. Non-Gaussian 
Ornstein--Uhlenbeck-based models and some of their uses in financial economics. 
\textit{J.~Roy. Stat. Soc.~B} 63:167--241.
\bibitem{11-ch-1}
\Aue{Taufer, E., and N.~Leonenko.} 2008. Simulation of L$\acute{\mbox{e}}$vy-driven 
Ornstein--Uhlenbeck processes with given marginal distribution. 
\textit{Comput. Stat. Data An.} 53:2427--2437.
\bibitem{12-ch-1}
\Aue{Madan, D.\,B., and E.~Seneta}. 1990. 
The VG model for share market returns. \textit{J.~Bus.} 63:511--524.
\bibitem{13-ch-1}
\Aue{Madan, D.\,B, P.\,P.~Carr, and E.\,C.~Chang}. 1998. 
The variance gamma process and option pricing. \textit{Eur. Finance Rev.} 2:79--105.
\bibitem{14-ch-1}
\Aue{Barndorff-Nielsen, O.\,E.} 1997. 
Normal inverse Gaussian distributions and stochastic volatility modeling. 
\textit{Scand. J.~Stat.} 24(1):1--13.

\bibitem{16-ch-1}
\Aue{Rydberg, H.} 1997. The normal inverse Gaussian L$\acute{\mbox{e}}$vy process: 
Simulation and approximation. \textit{Commun. Stat. Stochastic Models} 
13(4):887--910.
\bibitem{15-ch-1}
\Aue{Barndorff-Nielsen, O.\,E.} 1998. Processes of normal inverse Gaussian type. 
\textit{Financ.  Stoch.} 2:41--68.
\bibitem{17-ch-1}
\Aue{Haykin, S.} 1996. \textit{Adaptive filter theory}. 3rd ed. 
Upper Saddle River, NJ: Prentice Hall. 989~p.
\bibitem{18-ch-1}
\Aue{Appel, U., and A.\,V.~Brandt}. 1983. 
Adaptive sequential segmentation of piecewise stationary time series.  
\textit{Inform. Sci.} 29:27--56.
\bibitem{19-ch-1}
\Aue{Gustafsson, F.} 1996. The marginalized likelihood ratio test for 
detecting abrupt changes. \textit{IEEE Trans. Automat. Contr.} 41:66--78.
\bibitem{20-ch-1}
\Aue{Kay, S.} 1993. \textit{Fundamentals of statistical signal processing. Vol.~I. 
Estimation theory}.  Upper Saddle River, NJ: Prentice Hall. 625~p.
\bibitem{21-ch-1}
\Aue{Page, E.\,S.} 1961. Cumulative sum control chart. 
\textit{Technometrics} 3:1--9.
\end{thebibliography}

 }
 }

\end{multicols}

\vspace*{-6pt}

\hfill{\small\textit{Received October 20, 2016}}

\vspace*{-18pt}

\Contr

\vspace*{-2pt}

\noindent
\textbf{Chertok Andrey V.} (b.\ 1987)~--- 
junior scientist, Faculty of Computational Mathematics and Cybernetics, 
M.\,V.~Lo\-monosov Moscow State University, 1-52~Leninskiye Gory, GSP-1, Moscow 119991, 
Russian Federation; Head of R\&D, Data Science, Sberbank of Russia, 19~Vavilov Str.,
Moscow 117999, Russian Federation; \mbox{avchertok.sbt@sberbank.ru}

 \vspace*{1pt}
 
\noindent
\textbf{Kadaner Arsenii I.} (b.\ 1995)~--- 
student, Faculty of Mechanics and Mathematics, 
M.\,V.~Lomonosov Moscow State University, 
Main Building, Leninskiye Gory, GSP-1, Moscow 119991, Russian Federation; 
data scientist, Sberbank of Russia, 19~Vavilov Str., Moscow 117999, 
Russian Federation; \mbox{aikadaner.sbt@sberbank.ru}

  \vspace*{1pt}
 
\noindent
\textbf{Khazeeva Gelana T.} (b.\ 1993)~---
 student,  Faculty of Computational Mathematics and Cybernetics, M.\,V.~Lo\-monosov 
 Moscow State University, 1-52~Leninskiye Gory, GSP-1, Moscow 119991, 
 Russian Federation; \mbox{gelana.khazeyeva@gmail.com} 

 
 \vspace*{1pt}
 
\noindent
\textbf{Sokolov Igor A.} (b.\ 1954)~---
Academician of the Russian Academy of Sciences, Doctor of Science in technology, 
Director, Federal Research Center ``Computer Science and Control'' of 
the Russian Academy of Sciences, 44-2~Vavilov Str., Moscow 119333, Russian Federation; 
\mbox{isokolov@ipiran.ru}
\label{end\stat}


\renewcommand{\bibname}{\protect\rm Литература}      %2Abst+avt
\def\ld{\ldots}
\def\d{\overline d}
\def\oa{\overline\alpha}

\def\stat{milov}

\def\tit{СТАЦИОНАРНЫЕ ХАРАКТЕРИСТИКИ СИСТЕМЫ ОБСЛУЖИВАНИЯ С~ИНВЕРСИОННЫМ
ПОРЯДКОМ ОБСЛУЖИВАНИЯ, ВЕРОЯТНОСТНЫМ ПРИОРИТЕТОМ И~ГИСТЕРЕЗИСНОЙ
ПОЛИТИКОЙ$^*$}

\def\titkol{Стационарные характеристики системы обслуживания с инверсионным
порядком обслуживания} %, вероятностным приоритетом и гистерезисной политикой}

\def\autkol{Т.\,А.~Милованова, А.\,В.~Печинкин}

\def\aut{Т.\,А.~Милованова$^1$, А.\,В.~Печинкин$^2$}

\titel{\tit}{\aut}{\autkol}{\titkol}

{\renewcommand{\thefootnote}{\fnsymbol{footnote}}\footnotetext[1]
{Работа выполнена при поддержке РФФИ (проекты
№~11-07-00112 и №~12-07-00108).}}

\renewcommand{\thefootnote}{\arabic{footnote}}
\footnotetext[1]{Российский университет дружбы народов, tmilovanova77@mail.ru}
\footnotetext[2]{Институт проблем
информатики Российской академии наук, apechinkin@ipiran.ru}

\vspace*{6pt}

\Abst{Рассматривается однолинейная система массового
обслуживания (СМО) с инверсионным порядком обслуживания,
вероятностным приоритетом и простейшим вариантом
гистерезисной политики.
Найдены основные стационарные показатели функционирования
этой сис\-темы.}

\vspace*{4pt}

\KW{система массового обслуживания; инверсионный порядок
обслуживания; вероятностный приоритет; гистерезисная
политика}

\vspace*{14pt}


\vskip 14pt plus 9pt minus 6pt

      \thispagestyle{headings}

      \begin{multicols}{2}

            \label{st\stat}
            
\section{Введение}

Одним из важнейших направлений исследований в теории
массового обслуживания является изучение СМО с дисциплинами
обслуживания, отличны\-ми от обслуживания заявок в
порядке поступления, поскольку такие дисциплины
часто позволяют практически без каких-либо
усовершенствований повысить качество функционирования
самых разнообразных технических систем, например
ин\-фор\-ма\-ци\-он\-но-те\-ле\-ком\-му\-ни\-ка\-ци\-он\-ных сис\-тем (ИТС).
В~частности, дисциплиной такого рода является инверсионный
порядок обслуживания с вероятностным приоритетом,
введенный в~\cite{1-m} для решения задачи А.\,Д.~Соловьева
об оптимальных распределениях для некоторых типов
дисциплин обслуживания.
Подробное изложение полученных в этом направлении
результатов можно найти в~\cite{2-m}.

В последнее время значительное внимание уделяется также СМО с
гистерезисным управлением, являющимся одним из возможных механизмов
пред\-от\-вра\-ще\-ния различного рода перегрузок в ИТС (см., например,~\cite{3-m}). 
Разновидности гистерезисной политики используются при
обнаружении перегрузок как в сетях общеканальной системы
сигнализации №\,7, так и в сетях, где основой сигнализации является
протокол инициации сеансов связи.

В настоящей работе делается попытка связать эти два
направления исследования с по\-мощью СМО с инверсионным
порядком обслуживания, вероятностным приоритетом и
простейшим ва\-риантом гистерезисной политики, для
которой\linebreak находятся основные стационарные показатели
функционирования.
Отметим, что некоторые типы системы $M/G/1$ с
простейшим вариантом гистерезисной политики при
дисциплине обслуживания заявок в порядке поступления
изучались в~[4--7].

\section{Описание системы}

Рассмотрим однолинейную СМО с накопителем бесконечной
емкости, инверсионным
порядком обслуживания, вероятностным приоритетом и
простейшим вариантом гистерезисной политики.
Опишем функционирование этой СМО.

\begin{figure*} %fig1
\vspace*{1pt}
 \begin{center}
 \mbox{%
 \epsfxsize=114.642mm
 \epsfbox{mil-1.eps}
 }
% \vspace*{-9pt}
\end{center}
\begin{center}
{\small Схематическое изображение функционирования системы: \textit{1}~--- поступление; 
\textit{2}~---
обслуживание}
 \end{center}
\end{figure*}



Вариант гистерезисной политики заключается в следующем
(см.\ рисунок).
Имеется два порога $n_0$ и $n_1$, причем $n_1\hm<n_0$.
Пока число заявок в системе меньше $n_0$, система
функционирует в режиме~0.
Это означает, что заявки поступают с
интенсивностью $\lambda_0$ и имеют длину, распределенную
по закону $B_0(x)$ с плот\-ностью $b_0(x)\hm=B'_0(x)$
и средним значением
$\beta_0\hm=\int\limits_0^\infty x b_0(x)\, dx\hm<\infty$.
Но как только число заявок в системе становится равным
$n_0$, система переходит в режим~1.
В~этом режиме заявки поступают с интенсивностью~$\lambda_1$ и имеют длину, 
распределенную по закону $B_1(x)$
с плотностью $b_1(x)=B'_1(x)$ и средним значением
$\beta_1\hm=\int\limits_0^\infty x b_1(x)\, dx\hm<\infty$.
Так продолжа-\linebreak\vspace*{-12pt}

\pagebreak

\noindent
ется до тех пор, пока чис\-ло заявок в сис\-те\-ме не
станет равным~$n_1$.
Тогда система снова переходит в режим~0 и~т.\,д.


В~системе также реализован инверсионный порядок
обслуживания с вероятностным приоритетом.
Предполагается, что в любой момент времени известны
(остаточные) длины всех заявок в системе.
В момент поступления в систему новой заявки ее длина~$x$
сравнивается с (остаточной) длиной~$y$ заявки на приборе.
При этом если система функционирует в режиме~0, то с
вероятностью $d_0(x,y)$ на прибор становится вновь
поступившая заявка, а находившаяся ранее на приборе
занимает первое место в очереди, и наоборот, с
вероятностью $\d_0(x,y)\hm=1\hm-d_0(x,y)$ старая заявка
продолжает обслуживаться, а новая становится на первое
место в очереди.
Если же система функционирует в режиме~1, то
вероятность постановки на прибор вновь поступившей
заявки равна $d_1(x,y)$, а на первое мес\-то в очереди~---
$\d_1(x,y)\hm=1\hm-d_1(x,y)$.

Будем предполагать, что выполнено условие $\lambda b_1\hm<1$,
необходимое и достаточное для существования
стационарного режима функционирования рассматриваемой
системы.

Будем считать также, что $n_0\hm-n_1\hm\ge 2$.
Это предположение вводится только для того, чтобы не
рассматривать случаи, которые по записи расчетных
формул несколько отличаются от общего вида, и нисколько
не умаляет общности полученных результатов.

\section{Вспомогательные функции}

Пусть в некоторый момент система функционирует
в режиме~0, в системе находится $n$, $n_1\hm<n\hm<n_0$,
заявок и в этот момент поступает в сис\-те\-му и становится
на прибор новая заявка длины~$x$.
Обозначим через $\alpha_n(x)$ вероятность того, что в тот
момент, когда в системе впервые снова останется~$n$
заявок, она по-преж\-не\-му будет пребывать в режиме~0.


Функции $\alpha_n(x)$, $n_1\hm<n\hm<n_0$, удовлетворяют системе
уравнений

\columnbreak

\noindent
\begin{equation}
\label{2.1-m}
\alpha_{n_0-1}(x) \equiv 0\,;                    
\end{equation}

\vspace*{-12pt}

\noindent
\begin{multline}
\label{2.2-m}
\alpha_{n}(x)
= e^{-\lambda_0 x} 
+\int\limits_0^x \lambda_0 e^{-\lambda_0 y}\,dy\times{}\\[2pt]
{}\times
\int\limits_0^\infty b_0(z) \left[
d_0(z,x-y) \alpha_{n+1}(z) \alpha_{n}(x-y) + {}\right.\\[2pt]
\left.{}+ \d_0(z,x-y) \alpha_{n+1}(x-y) \alpha_{n}(z)
\right]  dz\,,\\[4pt]
n=\overline{n_1+1,n_0-2}\,.            % \eqno(2)
\end{multline}
Система уравнений~(\ref{2.1-m}), (\ref{2.2-m}) решается
последовательно, начиная с $n\hm=n_0\hm-1$ и кончая $n\hm=n_1+1$.

При решении уравнения~(\ref{2.2-m}) удобно привести его
к более простому виду. Вводя обозначение
\begin{equation*}
%\label{2.3-m}
a_{n}(x) = e^{\lambda_0 x} \alpha_{n}(x)\,,
\enskip  n=\overline{n_1+1,n_0-2}\,,
\end{equation*}
и производя тривиальные преобразования, получаем из~(\ref{2.2-m}):
\begin{multline}
\label{2.4-m}
a_{n}(x) = 1 +{}\\[2pt]
{}+ \int\limits_0^x \left(
\lambda_0 \int\limits_0^\infty b_0(z) d_0(z,y) \alpha_{n+1}(z)\, dz \right)
a_{n}(y)\, dy + {}\\[2pt]
{}+ \int\limits_0^\infty \left(
\lambda_0 b_0(y) e^{-\lambda_0 y} \int\limits_0^x e^{\lambda_0\, z} \d_0(y,z) \alpha_{n+1}(z)\,dz
\right)\times{}\\[2pt]
{}\times  a_{n}(y) \, dy\,,
\quad n=\overline{n_1+1,n_0-2}\,.
\end{multline}
Последнее соотношение представляет собой интегральное уравнение
\begin{multline}
\label{2.5-m}
a_n(x) = 1 + \int\limits_0^\infty K_n(x,y) a_{n}(y) \, dy\,,
\\[2pt]
 n=\overline{n_1+1,n_0-2}\,, 
\end{multline}
ядро которого имеет вид:

\noindent
\begin{multline*}
K_n(x,y) ={}\\
\hspace*{-7.92743pt}{}=
\begin{cases}
\displaystyle\lambda_0 \left( \int\limits_0^\infty
b_0(z) d_0(z,y) \alpha_{n+1}(z)\, dz+
b_0(y) e^{-\lambda_0 y}\times{}\right.\\
\left.\displaystyle{}\times \int\limits_0^x e^{\lambda_0 z}\, \d_0(y,z)
\alpha_{n+1}(z) \,dz \vphantom{\int\limits^\infty_0}\right),                     &\hspace*{-38pt}y<x\,;     \\
\displaystyle\lambda_0 b_0(y) e^{-\lambda_0 y}
\int\limits_0^x e^{\lambda_0 z} \,\d_0(y,z)
\alpha_{n+1}(z) \,dz\,,                         &\hspace*{-38pt}y>x\,.
\end{cases}
\end{multline*}
Численное решение уравнения~(\ref{2.5-m}) можно произ\-вес\-ти итерационным методом.
При этом в качестве нулевой итерации удобно выбрать тождественно равную нулю функцию.
Тогда итерации будут возрастающими, что позволит контролировать сходимость 
итерационного процесса.

В заключение этого раздела приведем условие на функцию $\d_0(x,y)$, при
котором интегральное уравнение~(\ref{2.4-m}) можно
свести к системе линейных алгебраических уравнений.
А~именно: будем предполагать, что 
\begin{equation}
\label{2.5-1}
\hspace*{-2mm}\d_0(x,y) = \sum\limits_{i=1}^I \d^{(1)}_{0,i}(x) \d^{(2)}_{0,i}(y),
\
 n=\overline{n_1+1,n_0-2}.\!\!
\end{equation}
Тогда, вводя обозначения
\begin{multline*}
c_n(y)= \lambda_0 \int\limits_0^\infty b_0(z) d_0(z,y) \alpha_{n+1}(z)\, dz\,,
\\
 n=\overline{n_1+1,n_0-2}\,;
\end{multline*}

\vspace*{-12pt}

\noindent
\begin{multline*}
c_{n,i}(x) = e^{\lambda_0 x} \d^{(2)}_{0,i}(x) \alpha_{n+1}(x)\,,\\
n=\overline{n_1+1,n_0-2}\,,\enskip
i=\overline{1,I}\,;
\end{multline*}

\vspace*{-12pt}

\noindent
\begin{multline*}
a_{n,i} = \int\limits_0^\infty \lambda_0 b_0(y) e^{-\lambda_0\, y} \d^{(1)}_{0,i}(y)
a_{n}(y) \, dy\,, \\ n=\overline{n_1+1,n_0-2}\,,
\ i=\overline{1,I}\,,
\end{multline*}
получаем из~(\ref{2.4-m}):
\begin{multline}
a_{n}(x)=1+\int\limits_0^x c_n(y) a_{n}(y)\, dy +
\sum\limits_{i=1}^I a_{n,i} \int\limits_0^x c_{n,i}(z)\, dz\,,
\\ n=\overline{n_1+1,n_0-2}\,.
\label{2.6-m}
\end{multline}
%%%%%%%%%%%%%%%%%%%%%%%%%%%%%
Дифференцируя теперь равенство~(\ref{2.6-m}), приходим
к дифференциальному уравнению
\begin{multline}
a'_{n}(x)= c_n(x)\, a_{n}(x) + \sum\limits_{i=1}^I a_{n,i} c_{n,i}(x)\,,
\\ n=\overline{n_1+1,n_0-2}\,,
\label{2.7-m}
\end{multline}
начальное условие для которого задается выражением:
\begin{equation}
\label{2.8-m}
a_n(0) = 1\,,
\enskip n=\overline{n_1+1,n_0-2}\,.
\end{equation}
Решение уравнения (\ref{2.7-m}) с начальным условием~(\ref{2.8-m}) имеет вид:
\begin{multline}
a_{n}(x) = \left(
1+ \sum\limits_{i=1}^I\! a_{n,i} \int\limits_0^x c_{n,i}(y) e^{- C_n(y)} \,dy
\right)
e^{C_n(x)} ,\\  n=\overline{n_1+1,n_0-2},
\label{2.9-m}
\end{multline}
где
\begin{equation*}
%\label{2.9}
C_{n}(x) = \int\limits_0^x c_n(y)\, dy\,,
\enskip n=\overline{n_1+1,n_0-2}\,.
\end{equation*}

Для того чтобы найти коэффициенты
$a_{n,i}$, $i\hm=\overline{1,I}$, умножим равенство~(\ref{2.9-m}) на
$\lambda_0 b_0(x) e^{-\lambda_0 x} \d^{(1)}_{0,j}(x)$
и проинтегрируем в пределах от~0 до~$\infty$. Тогда
\begin{multline*}
%\label{2.10-m}
a_{n,j} = \int\limits_0^\infty \lambda_0 b_0(x) e^{-\lambda_0\, x} \d^{(1)}_{0,j}(x)
e^{C_n(x)} dx + {}\\
{}+ \sum\limits_{i=1}^I a_{n,i} \int\limits_0^\infty \lambda_0 b_0(x) e^{-\lambda_0\, x} 
\d^{(1)}_{0,j}(x) e^{C_n(x)} \,dx\times{}\\
{}\times
\int\limits_0^x c_{n,i}(y) e^{- C_n(y)}  \,dy\,,
\enskip n=\overline{n_1+1,n_0-2}\,.
\end{multline*}
%%%%%%%%%%%%%%%%%%%%%%
Производя эту процедуру при всех
$j$, $j\hm=\overline{1,I}$, получаем систему линейных
ал\-геб\-ра\-и\-че\-ских уравнений, решая которую,
находим коэффициенты $a_{n,i}$ и соответственно
функции $a_{n}(x)$ и $\alpha_{n}(x)$.

В дальнейшем будем пользоваться обозначением
$\oa_n(x) \hm= 1 - \alpha_n(x)$.


Отметим, что, используя приближение $\d_0(x,y)$ с
помощью представления~(\ref{2.5-1}), 
можно найти функцию $a_{n}(x)$ с любой степенью точности.
Однако повышение точности влечет за собой существенное
увеличение числа $I$ коэффициентов $a_{n,i}$ и, 
значит, размерности системы линейных алгебраических уравнений.

\section{Стационарные вероятности состояний}

Обозначим через $p_0$ стационарную вероятность того,
что система свободна.
При $n\hm=\overline{1,n_1}$ или $n\hm\ge n_0$ обозначим через
$p_n(x_1,\ld,x_n)$ стационарную плот\-ность вероятностей того,
что в системе находится $n$ заявок, причем заявка на
приборе\linebreak имеет длину~$x_1$, первая заявка в очереди~---
длину~$x_2$ и~т.\,д.
Наконец, при $n\hm=\overline{n_1+1,n_0-1}$\linebreak через
$p_n(0;x_1,\ld,x_n)$ обозначим стационарную плотность
вероятностей того, что система функционирует в режиме~0 и
в системе находится~$n$ заявок, причем заявка на приборе
имеет длину~$x_1$, первая заявка в очереди~--- длину~$x_2$
и~т.\,д., а через $p_n(1;x_1,\ld,x_n)$~--- аналогичную
вероятность, но при этом система функционирует в режиме~1.

Используя метод исключения состояний (см., например,~\cite{22-m}), 
можно получить для $p_n(x_1,\ld,x_n)$,
$n\hm=\overline{1,n_1}$, уравнения
%%%%%%%%%%%%%%%%%%%%%%%%%%%%%%%
\begin{multline}
\label{3-0-m}
-p'_1(x)=-\lambda_0 p_1(x)+\lambda_0 p_1(x)\int\limits_0^\infty b_0(y) d_0(y,x)\, dy
+{}
\\
{}+\lambda_0 b_0(x)\int\limits_0^\infty p_1(y) \d_0(x,y)\, dy+
\lambda_0 p_0 b_0(x)\,;
\end{multline}
%%%%%%%%%%%%%%%%%%%%%%%%%%%%%%

\vspace*{-12pt}

\noindent
\begin{multline*}
%\label{3.3}
-p'_n(x_1,\ld,x_n) = - \lambda_0 p_{n}(x_1,\ld,x_n) +{}\\
{}+ \lambda_0 p_n(x_1,\ld,x_n)
\int\limits_0^\infty b_0(y) d_0(y,x_1)\, dy
+{}
\\
+ \lambda_0 b_0(x_1)\int\limits_0^\infty p_n(y,x_2,\ld,x_n) \d_0(x_1,y)\, dy
+ {}\\
{}+\lambda_0 p_{n-1}(x_2,\ld,x_n) b_0(x_1) d_0(x_1,x_2) +{}
\\
{}+ \lambda_0 b_0(x_2) p_{n-1}(x_1,x_3,\ld,x_n) \d_0(x_2,x_1),
\ n=\overline{2,n_1},\hspace*{-0.52872pt}
\end{multline*}
%%%%%%%%%%%%%%%%%%%%%%%%%%%%%%
с начальным условием
%%%%%%%%%%%%%%%%%%%%%%%%%%%%%%
$$
\lim_{x\to\infty} p_n(x,x_2,\ld,x_n) = 0\,, \enskip n=\overline{1,n_1}\,.
$$
%%%%%%%%%%%%%%%%%%%%%%%%%%%%%%
Можно выписать аналогичные уравнения для остальных
функций $p_n(x_1,\ld,x_n)$, $n\hm\ge n_0$, и
$p_n(i;x_1,\ld,x_n)$, $n\hm=\overline{n_1+1,n_0-1}$,
$i\hm=1,2$,
но они ввиду громоздкости здесь не приводятся.
Вычисления по этим формулам, хотя теоретически и можно
производить на основе решения интегральных уравнений,
практически не реализуемы уже при совсем небольших
значениях~$n$ даже на современной вычислительной технике,
поскольку размерность уравнений растет пропорционально~$n$.

Однако для практических расчетов, как правило,
достаточно знать только маргинальные стационарные
плотности $p_1(x)$,
\begin{multline*}
p_n(x) = \int\limits_0^\infty \cdots \int\limits_0^\infty
p_n(x,x_2,\ld,x_n)\, dx_2\cdots dx_n\,,
\\ n=\overline{2,n_1}
\enskip \hbox{или}
\enskip n\ge n_0\,,
\end{multline*}
и
\begin{multline*}
p_n(i;x)= \int\limits_0^\infty \cdots\int\limits_0^\infty
p_n(i;x,x_2,\ld,x_n)\, dx_2\cdots dx_n\,,
\\ i=0,1\,,\enskip n=\overline{n_1+1,n_0-1}\,.
\end{multline*}
Для них справедливы соотношения
%%%%%%%%%%%%%%%%%%%%%%%%%%%%%%
\begin{multline}
\label{3-1-m} -p'_n(x) = - f_0(x)\, p_n(x) + \int\limits_0^\infty
k_0(x,y)\, p_n(y)\, dy +{}
\\ 
{}+g_{0,n}(x) \,,\enskip
n=\overline{2,n_1}\,;
\end{multline}
%%%%%%%%%%%%%%%%%%%%%%%%%%%%%%

\vspace*{-12pt}

\noindent
\begin{multline}
\label{3-2-m}
-p'_n(0;x) = - f_{0,n}(x)\, p_n(0,x) +{}\\
{}+ \int\limits_0^\infty k_{0,n}(x,y)\, p_n(0,y)\, dy
+ g_{0,n}(x) \,,
\\  n=\overline{n_1+1,n_0-1}\,;
\end{multline}

%%%%%%%%%%%%%%%%%%%%%%%%%%%%%%
\vspace*{-12pt}

\noindent
\begin{multline}
\label{3-3-m}
-p'_n(1;x) = - f_{1}(x) p_n(1,x) + {}\\
{}+\int\limits_0^\infty k_{1}(x,y) p_n(1,y)\, dy
+ g_{1,n}(x)\,, \\ 
n=\overline{n_1+1,n_0-1}\,;
\end{multline}
%%%%%%%%%%%%%%%%%%%%%%%%%%%%%%

\vspace*{-12pt}

\noindent
\begin{multline}
\label{3-4-m}
-p'_n(x) = - f_{1}(x) p_n(x) +{}\\
{}+\int\limits_0^\infty k_{1}(x,y) p_n(y)\, dy + g_{1,n}(x)\,,
\  n\ge n_0\,,
\end{multline}
с начальными условиями
%%%%%%%%%%%%%%%%%%%%%%%%%%%%%%
\begin{equation}
\label{3-beg-1-m}
\lim_{x\to\infty} p_n(x) = 0\,,
\ \ n=\overline{2,n_1}\ \ \hbox{или}\ \ n\ge n_0\,,
\end{equation}
%%%%%%%%%%%%%%%%%%%%%%%%%%%%%%
\begin{equation}
\label{3-beg-2-m}
\lim_{x\to\infty} p_n(i;x) = 0\,,
\ n=\overline{n_1+1,n_0-1}\,,\ \ i=0,1\,,
\end{equation}
%%%%%%%%%%%%%%%%%%%%%%%%%%%%%%
в которых для сокращения записи введены сле\-ду\-ющие
обозначения:
%%%%%%%%%%%%%%%%%%%%%%%%%%%%%%%
\begin{align*}
f_0(x) &= \lambda_0 \left(
1 - \int\limits_0^\infty b_0(y) d_0(y,x)\, dy \right)\,;
\\
k_0(x,y) &= \lambda_0 b_0(x) \d_0(x,y) \,;
\\
g_{0,1}(x) &= \lambda_0 p_0 b_0(x) \,;\\
g_{0,n}(x) &= \lambda_0 b_0(x) \int\limits_0^\infty
p_{n-1}(y) d_0(x,y)\, dy +{}\\
&\hspace*{2mm}{}+
\lambda_0 p_{n-1}(x) \int\limits_0^\infty \d_0(y,x) b_0(y)\, dy\,,
\ \ n=\overline{2,n_1}\,;
\end{align*}
%%%%%%%%%%%%%%%%%%%%%%%%%%%%%%

\vspace*{-24pt}

\noindent
\begin{multline*}
f_{0,n}(x)= \lambda_0 \left(
1 - \int\limits_0^\infty b_0(y) d_0(y,x) \alpha_n(y)\, dy
\right)\,,
\\ n=\overline{n_1+1,n_0-1}\,;
\end{multline*}
%%%%%%%%

\vspace*{-12pt}

\noindent
\begin{multline*}
k_{0,n}(x,y) = \lambda_0 b_0(x) \d_0(x,y) \alpha_n(y) \,,
\\ n=\overline{n_1+1,n_0-1}\,;
\end{multline*}
%%%%%%%%%

\vspace*{-12pt}

\noindent
\begin{multline*}
g_{0,n_1+1}(x) = \lambda_0 b_0(x) \int\limits_0^\infty p_{n_1}(y) d_0(x,y)\, dy
+{}\\
{}+
\lambda_0 p_{n_1}(x) \int\limits_0^\infty b_0(y) \d_0(y,x)\, dy\,;
\end{multline*}
%%%%%%%%%

\vspace*{-12pt}

\noindent
\begin{multline*}
g_{0,n}(x) = \lambda_0 b_0(x) \int\limits_0^\infty p_{n-1}(0;y) d_0(x,y)\, dy
+{}\\
{}+ \lambda_0 p_{n-1}(0;x) \int\limits_0^\infty b_0(y) \d_0(y,x)\, dy\,,
\\ 
n=\overline{n_1+2,n_0-1}\,;
\end{multline*}
%%%%%%%%%%%%%%%%%%%%%%%%%%%%%%
\begin{align}
\label{3-gf-1-m}
f_1(x)&= \lambda_1 \left(
1 - \int\limits_0^\infty b_1(y) d_1(y,x)\, dy\right)\,;
\\
\label{3-gf-2-m}
k_1(x,y) &= \lambda_1 b_1(x) \d_1(x,y) \,,
\end{align}
%%%%%%%%%%%

\vspace*{-12pt}

\noindent
\begin{multline*}
g_{1,n_1+1}(x)={}\\
 {}=\lambda_0 p_{n_1+1}(0;x)\int\limits_0^\infty b_0(y) d_0(y,x) 
\oa_{n_1+1}(y)\, dy
+{}
\\
{}+ \lambda_0 b_0(x) \int\limits_0^\infty p_{n_1+1}(0;y) \d_0(x,y) \oa_{n_1+1}(y)\, dy\,;
\end{multline*}
%%%%%%%%%%


\vspace*{-12pt}

\noindent
\begin{multline*}
g_{1,n}(x) = \lambda_0 p_{n}(0;x) \int\limits_0^\infty b_0(y) d_0(y,x) \oa_n(y)\, dy
+{}\\
{}+ \lambda_0 b_0(x) \int\limits_0^\infty p_{n}(0;y) \d_0(x,y) \oa_n(y)\, dy
+ {}\\
{}+ \lambda_1 b_1(x) \int\limits_0^\infty p_{n-1}(1;y) d_1(x,y)\, dy+{}\\
{}+
\lambda_1 p_{n-1}(1;x) \int\limits_0^\infty b_1(y) \d_1(y,x)\, dy\,,
\\ n=\overline{n_1+2,n_0-1}\,;
\end{multline*}
%%%%%%%%%%%%%%%%%%%%%%%%%%%%%%

\vspace*{-24pt}

\noindent
\begin{multline*}
g_{1,n_0}(x) = \lambda_0 b_0(x)\int\limits_0^\infty p_{n_0-1}(0;y) d_0(x,y)\, dy
+{}\\
{}+
\lambda_0 p_{n_0-1}(0;x) \int\limits_0^\infty b_0(y) \d_0(y,x)\, dy+{}
\\
{}+ \lambda_1 b_1(x) \int\limits_0^\infty p_{n_0-1}(1;y) d_1(x,y)\, dy+{}\\
{}+
\lambda_1 p_{n_0-1}(1;x) \int\limits_0^\infty b_1(y) \d_1(y,x)\, dy \,;
\end{multline*}
%%%%%%%%%%%%%%%%%%%%%%%%%%

\vspace*{-12pt}

\noindent
\begin{multline}
\label{3-gf-3-m}
g_{1,n}(x)= \lambda_1 b_1(x) \int\limits_0^\infty p_{n-1}(y) d_1(x,y)\, dy
+{}\\
{}+ \lambda_1 p_{n-1}(x) \int\limits_0^\infty b_1(y) \d_1(y,x)\, dy\,.
\ \ n>n_0\,,
\end{multline}
%%%%%%%%%%%%%%%%%%%%%%%%%%%%%%%

Вероятность $p_0$ вычисляется из условия нормировки
$$
p_0 + \sum\limits_{n=1}^{n_1} p_n+\sum\limits_{n=n_1+1}^{n_0-1} \left[p_{n,0} + p_{n,1}\right]
+
\sum\limits_{n=n_0}^{\infty} p_n = 1\,,
$$
где $p_n = \int\limits_0^\infty p_n(x)\, dx$,
$n\hm=\overline{1,n_1}$ или $n\hm\ge n_0$,~---
стационарная вероятность того, что в системе находится
$n$ заявок, а $p_{n,i} \hm= \int\limits_0^\infty p_n(i;x)\, dx$,
$n\hm=\overline{n_1+1,n_0-1}$, $i\hm=0,1$,~--- стационарная
вероятность того, что система функционирует в режиме~$i$ и
в системе находится $n$~заявок.

Уравнения~(\ref{3-0-m})--(\ref{3-4-m}) легко приводятся к
интегральным. Действительно, вводя новые обозначения
%%%%%%%%%%%%%%%%%%%%%%%%%%%%%%%
\begin{align*}
F_0(x) &= \int\limits_0^x f_0(y)\, dy\,,
\ \ n=\overline{1,n_1}\,;
\\
%%%%%%%%%%%%%%%%%%%%%%%%%%%%%%
F_{0,n}(x) &= \int\limits_0^x f_{0,n}(y)\, dy\,,
\ \ n=\overline{n_1+1,n_0-1}\,;
\\
%%%%%%%%%%%%%%%%%%%%%%%%%%%%%%
F_{1}(x)&= \int\limits_0^x f_{1}(y)\, dy\,,
\ \ n\ge n_1+1\,;
\\
%%%%%%%%%%%%%%%%%%%%%%%%%%%%%%
p_n(x) &= \pi_n(x) e^{F_0(x)} \,, \ \ n=\overline{1,n_1}\,;
\\
%%%%%%%%%%%%%%%%%%%%%%%%%%%%%%
p_n(0;x) &= \pi_n(0;x) e^{F_{0,n}(x)} \,, \ \ n=\overline{n_1+1,n_0-1}\,;
\\
%%%%%%%%%%%%%%%%%%%%%%%%%%%%%%
p_n(1;x) &= \pi_n(1;x) e^{F_{1}(x)} \,, \ \ n=\overline{n_1+1,n_0-1}\,;
\\
p_n(x) &= \pi_n(x) e^{F_{1}(x)} \,, \ \ n\ge n_0\,,
\end{align*}

%%%%%%%%%%%%%%%%%%%%%%%%%%%%%%


\noindent
из \eqref{3-0-m}--\eqref{3-4-m} получаем соотношения

\noindent
%%%%%%%%%%%%%%%%%%%%%%%%%%%%%%
\begin{multline}
\label{pi-1-m}
-\pi'_n(x) = e^{-F_0(x)} \int\limits_0^\infty e^{F_0(y)} k_0(x,y) \pi_n(y)\, dy
+{}\\
{}+
e^{-F_0(x)} g_{0,n}(x) \,,
\ \ n=\overline{1,n_1}\,;
\end{multline}
%%%%%%%%%%%%%%%%%%%%%%%%%%%%%%

\vspace*{-12pt}

\noindent
\begin{multline}
\label{pi-2-m}
-\pi'_n(0;x) = {}\\
{}=e^{-F_{0,n}(x)} \int\limits_0^\infty e^{F_{0,n}(y)} 
k_{0,n}(x,y) \pi_n(0;y)\, dy +{}\\
{}+
e^{-F_{0,n}(x)} g_{0,n}(x) \,, \ \ n=\overline{n_1+1,n_0-1}\,;
\end{multline}
%%%%%%%%%%%%%%%%%%%%%%%%%%%%%%

\vspace*{-12pt}

\noindent
\begin{multline}
\label{pi-3-m}
-\pi'_n(1;x) = e^{-F_{1}(x)} \int\limits_0^\infty e^{F_{1}(y)} k_{1}(x,y) \pi_n(1;y)\, dy
+{}\\
{}+
e^{-F_{1}(x)} g_{1,n}(x) \,, \ \ n=\overline{n_1+1,n_0-1}\,;
\end{multline}
%%%%%%%%%%%%%%%%%%%%%%%%%%%%%%

\vspace*{-12pt}

\noindent
\begin{multline}
\label{pi-4-m}
-\pi'_n(x) = e^{-F_{1}(x)} \int\limits_0^\infty e^{F_{1}(y)} k_{1}(x,y)\pi_n(y)\, dy
+{}\\
{}+
e^{-F_{1}(x)} g_{1,n}(x) \,, \ \ n\ge n_0\,,
\end{multline}
%%%%%%%%%%%%%%%%%%%%%%%%%%%%%%
интегрируя которые в пределах от~$x$ до $\infty$
и учитывая начальные условия~(\ref{3-beg-1-m}),
(\ref{3-beg-2-m}), имеем
\begin{multline}
\label{int-1-m}
\pi_n(x) ={}\\
{}= \int\limits_0^\infty e^{F_0(y)} \left(
\int\limits_x^\infty e^{-F_0(u)} k_0(u,y)\, du
\right) \pi_n(y)\, dy
+{}\\
{}+
\int\limits_x^\infty e^{-F_0(u)} g_{0,n}(u)\, du\,;
\ \ n=\overline{1,n_1}\,,
\end{multline}

\vspace*{-12pt}

\noindent
\begin{multline}
\label{int-2-m}
\pi_n(0;x) ={}\\
{}= \int\limits_0^\infty\! e^{F_{0,n}(y)} \left(
\int\limits_x^\infty\! e^{-F_{0,n}(u)} k_{0,n}(u,y)\, du \right)
\pi_n(0;y)\, dy+
\\
{}+
\int\limits_x^\infty e^{-F_{0,n}(u)} g_{0,n}(u)\, du\,,
\enskip n=\overline{n_1+1,n_0-1}\,;
\end{multline}
%%%%%%%%%%%%%%%%%%%%%%%%%%%%%%


\vspace*{-12pt}

\noindent
\begin{multline}
\label{int-3-m}
\pi_n(1;x) ={}\\
{}= \int\limits_0^\infty e^{F_{1}(y)} \left(
\int\limits_x^\infty e^{-F_{1}(u)} k_{1}(u,y)\, du\right)
\pi_n(1;y)\, dy
+{}
\\
{}+
\int\limits_x^\infty e^{-F_{1}(u)} g_{1,n}(u)\, du\,,
\ \ n=\overline{n_1+1,n_0-1}\,;
\end{multline}
%%%%%%%%%%%%%%%%%%%%%%%%%%%%%%
\begin{multline}
\label{int-4-m}
\hspace*{-5mm}\pi_n(x) = \int\limits_0^\infty e^{F_{1}(y)} \left(
\int\limits_x^\infty e^{-F_{1}(u)} k_{1}(u,y)\, du \right)
\pi_n(y)\, dy +{}\\
{}+
\int\limits_x^\infty e^{-F_{1}(u)} g_{1,n}(u)\, du\,,
\ \ n\ge n_0\,.
\end{multline}
%%%%%%%%%%%%%%%%%%%%%%%%%

Соотношения \eqref{int-1-m}--\eqref{int-4-m} являются
интегральными уравнениями такого же вида, что и~\eqref{2.5-m},
и к ним применимы те же методы решения, что и
к уравнению~\eqref{2.5-m}.

Так же как для функций $\alpha_n(x)$, приведем условия
для функций $\d_0(x,y)$ и $\d_1(x,y)$, которые поз\-во\-ляют
получить решения ин\-тег\-ро\-диф\-фе\-рен\-ци\-аль\-ных уравнений~\eqref{pi-1-m}--\eqref{pi-4-m}
с помощью приведения к системе линейных алгебраических уравнений.
А~именно: будем предполагать, что выполнены условия~\eqref{2.5-1} и
\begin{equation}
\label{2.5-2-m}
\d_1(x,y) = \sum\limits_{i=1}^{I_1} \d^{(1)}_{1,i}(x) \d^{(2)}_{1,i}(y) \,.
\end{equation}
Тогда, вводя обозначения
$$
c_i(x) = \lambda_0 b_0(x) \d^{(1)}_{0,i}(x) e^{-F_0(x)}\,,\ \ i=\overline{1,I}\,;
$$
$$
q_{n,i} = \int\limits_0^\infty e^{F_0(y)} \d^{(2)}_{0,i}(y) \pi_n(y)\, dy \,,\ \ 
n=\overline{1,n_1}\,,
\ \ i=\overline{1,I}\,;
$$
$$
q_{n}(x)= e^{-F_0(x)} g_{0,n}(x)\,,
\ \ n=\overline{1,n_1}\,;
$$
%%%%%%%%%%%%%%%%%%%%%%%%%%%%%%%%%%%%%%%

\vspace*{-12pt}

\noindent
\begin{multline*}
c_{0;n,i}(x)= \lambda_0 b_0(x) \d^{(1)}_{0,i}(x) e^{-F_{0,n}(x)}\,,\\
n=\overline{n_1+1,n_0-1}\, ,\ \ i=\overline{1,I}\,;
\end{multline*}

\vspace*{-12pt}

\noindent
\begin{multline*}
q_{0;n,i}= \int\limits_0^\infty e^{F_{0,n}(y)} \d^{(2)}_{0,i}(y) \alpha_n(y) \pi_n(0;y)\, dy\,,
\\ n=\overline{n_1+1,n_0-1}\,,
\ \ i=\overline{1,I}\,;
\end{multline*}
$$ 
q_{0;n}(x) = e^{-F_{0,n}(x)} g_{0,n}(x) \,, \ \ n=\overline{n_1+1,n_0-1}\,;
$$
%%%%%%%%%%%%%%%%%%%%%%%%%%%%%%%%%%%%%%%
$$
c_{1;i}(x)= \lambda_1 b_1(x) \d^{(1)}_{1,i}(x) e^{-F_{1}(x)}\,,\ \ i=\overline{1,I_1}\,;
$$

\vspace*{-12pt}

\noindent
\begin{multline*}
q_{1;n,i}= \int\limits_0^\infty e^{F_{1}(y)} \d^{(2)}_{1,i}(y) \pi_n(1;y)\, dy\,,\\ 
n=\overline{n_1+1,n_0-1}\,,
\ \ i=\overline{1,I_1}\,    ;
\end{multline*}
$$
q_{1;n}(x)= e^{-F_{1}(x)} g_{1,n}(x)\,,
\ \ n\ge n_1+1\,;
$$
%%%%%%%%%%%%%%%%%%%%%%%%%%%%%%%%%%%%%%%%%%

\vspace*{-12pt}

\noindent
\begin{equation*}
q_{1;n,i} = \int\limits_0^\infty e^{F_{1}(y)}  \d^{(2)}_{1,i}(y) \pi_n(y)\, dy\,,\\ 
n\ge n_0\,,\  i=\overline{1,I_1}\,,
\end{equation*}
%%%%%%%%%%%%%%%%%%%%%%%%%%%%%%%%%%%%%%%%%%%
получаем из (\ref{pi-1-m})--(\ref{pi-4-m}) после
интегрирования в пределах от $x$ до $\infty$
с учетом начальных условий~(\ref{3-beg-1-m}),
(\ref{3-beg-2-m}):
%%%%%%%%%%%%%%%%%%%%%%%%%%%%%%%%%%%%%%%%%%%
\begin{equation}
\label{ipi-1-m}
\pi_n(x)= \sum\limits_{i=1}^{I} C_i(x) q_{n,i}+Q_{n}(x) \,,
\ \ n=\overline{1,n_1}\,;
\end{equation}
%%%%%%%%%%%%%%%%%%%%%%%%%%%%%%

\vspace*{-24pt}

\noindent
\begin{multline}
\label{ipi-2-m}
\pi_n(0;x) = \sum\limits_{i=1}^{I} C_{0;n,i}(x) q_{0;n,i} +
Q_{0;n}(x) \,, \\[1pt]
 n=\overline{n_1+1,n_0-1}\,;
\end{multline}
%%%%%%%%%%%%%%%%%%%%%%%%%%%%%%

\vspace*{-12pt}

\noindent
\begin{multline}
\label{ipi-3-m}
\pi_n(1;x) = \sum\limits_{i=1}^{I_1} C_{1;i}(x) q_{1;n,i} +
Q_{1;n}(x)\,, \\[1pt] 
n=\overline{n_1+1,n_0-1}\,;
\end{multline}
%%%%%%%%%%%%%%%%%%%%%%%%%%%%%%
\begin{equation}
\label{ipi-4-m}
\pi_n(x)= \sum\limits_{i=1}^{I_1} C_{1;i}(x) q_{1;n,i} +
Q_{1;n}(x)\,, \ \ n\ge n_0\,,
\end{equation}
%%%%%%%%%%%%%
где
%%%%%%%%%%%%%
$$
C_{i}(x)= \int\limits_x^\infty c_{i}(y)\, dy\,, \ \ n=\overline{1,n_1}\,,\ \ i=\overline{1,I}\,;
$$

\vspace*{-12pt}

\noindent
\begin{multline*}
C_{0;n,i}(x)= \int\limits_x^\infty c_{0;n,i}(y)\, dy\,, \\ 
n=\overline{n_1+1,n_0-1}\,,\ \ i=\overline{1,I}\,;
\end{multline*}
$$
C_{1,i}(x) = \int\limits_x^\infty c_{1,i}(y)\, dy\,,
\ \ n\ge n_1\,,\ \ i=\overline{1,I_1}\,;
$$
%%%%%%%%%%%%%%%%%%%%%%%%%%%%%%
$$
Q_{n}(x) = \int\limits_x^\infty q_{n}(y)\, dy\,,
\ \ n=\overline{1,n_1}\,;
$$
$$
Q_{0;n}(x)= \int\limits_x^\infty q_{0;n}(y)\, dy\,,
\ \ n=\overline{n_1+1,n_0-1}\,;
$$
$$
Q_{1;n}(x)= \int\limits_x^\infty q_{1;n}(y)\, dy\,,
\ \ n\ge n_1\,.
$$

Для определения постоянных
$q_{n,i}$, $q_{0;n,i}$ и $q_{1;n,i}$
умножим равенства~\eqref{ipi-1-m}--\eqref{ipi-4-m} на
$\d^{(2)}_{0,j}(y) e^{F_0(y)}$,
$\d^{(2)}_{0,j}(y) \alpha_n(y) e^{F_{0,n}(y)}$
и $\d^{(2)}_{1,j}(y) e^{F_{1}(y)}$
соответственно и проинтегрируем в пределах от~0 до~$\infty$.
Тогда
%%%%%%%%%%%%%%%%%%%%%%%%%%%%%%%%%%%%%%%%%%%
\begin{multline}
\label{cons-1-m}
q_{n,j}= \sum\limits_{i=1}^{I} \int\limits_0^\infty \d^{(2)}_{0,j}(y) 
e^{F_0(y)} C_i(y)\, dy\, q_{n,i}
+{}\\[2pt]
\hspace*{-2mm}{}+
\int\limits_0^\infty \d^{(2)}_{0,j}(y) e^{F_0(y)}  Q_{n}(y)\, dy\,,
\  n=\overline{1,n_1}\,,
\  j=\overline{1,I}\,;\!\!
\end{multline}
%%%%%%%%%%%%%%%%%%%%%%%%%%%%%%

\vspace*{-12pt}

\noindent
\begin{multline*}
q_{0;n,j}= {}\\[2pt]
{}=\sum\limits_{i=1}^{I} \int\limits_0^\infty \d^{(2)}_{0,j}(y) \alpha_n(y)
e^{F_{0,n}(y)} C_{0;n,i}(y)\, dy\, q_{0;n,i}
+{}
\end{multline*}

\noindent
\begin{multline}
\label{cons-2-m}
{}+
\int\limits_0^\infty \d^{(2)}_{0,j}(y) \alpha_n(y) e^{F_{0,n}(y)} Q_{0;n}(y)\, dy\,,\\ 
n=\overline{n_1+1,n_0-1}\,, \enskip j=\overline{1,I}\,;
\end{multline}
%%%%%%%%%%%%%%%%%%%%%%%%%%%%%%

\vspace*{-12pt}

\noindent
\begin{multline}
\label{cons-3-m}
q_{1;n,j} = \sum\limits_{i=1}^{I_1} \int\limits_0^\infty \d^{(2)}_{1,j}(y) 
e^{F_{1}(y)}\, C_{1;i}(y)\, dy\, q_{1;n,i}
+{}\\
{}+
\int\limits_0^\infty \d^{(2)}_{1,j}(y) e^{F_{1}(y)} Q_{1;n}(y)\, dy\,,\\ 
n\ge n_1+1\,, \enskip j=\overline{1,I_1}\,.
\end{multline}
%%%%%%%%%%%%%

Каждое из соотношений~\eqref{cons-1-m}--\eqref{cons-3-m}
представляет собой систему линейных алгебраических
уравнений, что позволяет легко находить коэффициенты
$q_{n,i}$, $q_{0;n,i}$ и $q_{1;n,i}$ и в конечном
счете плотности $p_n(x)$, $p_n(0;x)$ и $p_n(1;x)$.


В~заключение этого раздела приведем выражение для
суммарной стационарной интенсивности~$\lambda$ входящего
потока:
\begin{multline}
\label{inten-1-m}
\lambda = \lambda_0 p_0 + \lambda_0 \sum\limits_{n=1}^{n_1}
p_n +{}\\
\!\!{}+ \lambda_0 \sum\limits_{n=n_1+1}^{n_0-1} p_{n,0}
+ \lambda_1 \sum\limits_{n=n_1+1}^{n_0-1} p_{n,1}
+ \lambda_1 \sum\limits_{n=n_0}^{\infty} p_n .\!\!
\end{multline}

\section{Применение производящих функций}

Для вычисления моментов стационарного распределения
числа заявок в системе можно воспользоваться производящей функцией (ПФ):
$$
p(z,x) = \sum\limits_{n=n_0}^\infty z^n p_n(x)\,.
$$
Правда, для того чтобы определить ПФ $p(z,x)$,
необходимо знать плотности вероятностей $p_{n_0-1}(0;x)$
и $p_{n_0-1}(1;x)$, а для этого предварительно вы\-чис\-лить
$p_{n}(x)$, $n\hm=\overline{1,n_1}$, $p_{n}(0;x)$, $n\hm=\overline{n_1+1,n_0-2}$, и
$p_{n}(1;x)$, $n\hm=\overline{n_1+1,n_0-2}$.

Умножая соотношения~\eqref{3-4-m} на $z^n$ и суммируя по~$n$, 
получаем после простейших преобразований с
учетом~\eqref{3-gf-1-m}--\eqref{3-gf-3-m}
\begin{multline}
\label{3.pf-m}
- p'_{x}(z,x) = - (1-z) f_1(x)\, p(z,x) +{}\\
{}+ \lambda_1 b_1(x)
\int\limits_0^\infty p(z,y) \left[\d_1(x,y) + z d_1(x,y)\right]\, dy
+{}\\
{}+
z^{n_0} g_{1,n_0}(x) 
\end{multline}
с начальным условием
\begin{equation}
\label{3.pf-b-m}
\lim_{x\to\infty} p(z,x) = 0\,.
\end{equation}

Уравнение~\eqref{3.pf-m} с начальным условием~\eqref{3.pf-b-m} легко приводится к интегральному
уравнению
\begin{equation*}
%\label{3.pf-2-m}
q(z,x) = \int\limits_0^\infty K(x,y) q(z,y)\, dy
+ z^{n_0} R(x)\,,
\end{equation*}
где
$$
q(z,x) = e^{-(1-z) F_1(x)} p(z,x)\,;
$$

\vspace*{-12pt}

\noindent
\begin{multline*}
K(x,y) = \lambda_1 \int\limits_x^\infty\! e^{(1-z) \left[F_1(y) - F_1(u)\right]}
b_1(u) \left[\,\d_1(u,y) +{}\right.\\
\left.{}+ z d_1(u,y)\right]\, du\,;
\end{multline*}
%%%%%%%%%%%%%%%
$$
R(x) = \int\limits_x^\infty e^{-(1-z) F_1(u)} g_{1,n_0}(u)\, du\,.
$$
Последнее уравнение имеет такой же вид, как и~\eqref{2.5-m},
с теми же замечаниями относительно решения, что и раньше.
Кроме того, если выполнено условие~\eqref{2.5-2-m},
то решение этого уравнения, как и прежде, сводится
к решению системы линейных алгебраических уравнений.

Производящая функция $P(z)$ стационарного распределения числа заявок в
системе без учета их длин и режима функционирования
определяется формулой:
\begin{multline*}
P(z) = p_0 + \sum\limits_{n=1}^{n_1} z^n p_{n} +
\sum\limits_{n=n_1+1}^{n_0-1} z^n \left[p_{n,0} + p_{n,1}\right]
+{}\\
{}+ \int\limits_0^\infty e^{(1-z) F_1(x)} q(z,x)\, dx\,.
\end{multline*}

Моменты стационарного распределения числа заявок
в системе вычисляются с помощью дифференцирования
ПФ $P(z)$ в точке $z\hm=1$ и по\-сле\-ду\-юще\-го решения
получившихся уравнений.

\section{Стационарное распределение времени пребывания
заявки в~системе}

Обозначим через $u(s;x)$ преобразование Лап\-ла\-са--Стилть\-еса
(ПЛС) для открываемого заявкой длины
$x$ периода занятости (ПЗ) обычной СМО $M/G/1/\infty$
с интенсивностью $\lambda_1$ входящего потока и функцией распределения $B_1(x)$
времени обслуживания заявки,
а через $u(s)$ --- то же самое ПЛС, но для ПЗ, открываемого
заявкой произвольной длины.
Тогда
\begin{align*}
u(s;x)&= e^{-[s + \lambda_1 - \lambda_1 u(s)]\,x}\,;
\\
u(s) &= \beta_1(s + \lambda_1 - \lambda_1 u(s))\,.
\end{align*}
%%%%%%%%%%%%%%%%%%%%%%%%%%

Предположим теперь, что в начальный момент
рассматриваемая СМО функционирует в режиме~0 и
в ней находится~$n$, $n\hm=\overline{1,n_0-1}$,
заявок.
Обозначим через $u_n(s;x)$, $n\hm=\overline{1,n_0-1}$, ПЛС времени до того момента,
когда в системе впервые останется $n-1$ заявок
и при этом система по-преж\-не\-му будет функционировать
в режиме~0, при условии что на приборе начала
обслуживаться заявка длины~$x$, а через $u^*_n(s;x)$, $n\hm=\overline{n_1+2,n_0-1}$,~--- 
функцию, подобную $u_n(s;x)$, но при этом система перейдет в режим~1.

Справедливы уравнения
%%%%%%%%%%%
\begin{equation}
\label{5-1-m}
u'_{n_0-1}(s;x) = - \left[s + \lambda_0\right] u_{n_0-1}(s;x)\,;
\end{equation}
%%%%%%%%%

\vspace*{-12pt}

\noindent
\begin{multline*}
u'_n(s;x) = - (s + \lambda_0) u_{n}(s;x) +{}
\\
{}+
\lambda_0 \int\limits_0^\infty b_0(y) \left[d_0(y,x) u_{n+1}(s;y) u_{n}(s;x)
+{}\right.\\
\left.{}+ \d_0(y,x) u_{n+1}(s;x)\, u_{n}(s;y)\right] \, dy\,,
\\ 
n=\overline{n_1+2,n_0-2}\,,
\end{multline*}
%%%%%%%%%
с начальным условием
$$
u_n(s;0)= 1 \,, \ \ n=\overline{n_1+2,n_0-1}\,,
$$
%%%%%%%%%%%
уравнения
\begin{multline}
\label{5-2-m}
u^{*\,\prime}_{n_0-1}(s;x) = - [s + \lambda_0] u^*_{n_0-1}(s;x)+
\\
{}+
\lambda_0 \int\limits_0^\infty b_0(y) \left[d_0(y,x)  u(s;y)\, u(s;x)+{}\right.\\
\left.{}+
\d_0(y,x) u(s;x)\, u(s;y)\right] \, dy\,;
\end{multline}
%%%%%%%%%%%

\vspace*{-12pt}

\noindent
\begin{multline}
\label{5-2-2-m}
u^{*\,\prime}_n(s;x) = - \left[s + \lambda_0\right] u^*_{n}(s;x) +{}
\\
{}+
\lambda_0 \int\limits_0^\infty b_0(y) \left[d_0(y,x) u^*_{n+1}(s;y) u(s;x)
+ {}\right.\\
\left.{}+\d_0(y,x) u^*_{n+1}(s;x) u(s;y)\right] \, dy
+{}\\
{}+ \lambda_0 \int\limits_0^\infty b_0(y) \left[d_0(y,x) u_{n+1}(s;y) u^*_{n}(s;x)
+ {}\right.\\
\left.{}+d_0(y,x) u_{n+1}(s;x) u^*_{n}(s;y)\right] \, dy\,,
\\  n=\overline{n_1+2,n_0-2}\,,
\end{multline}
%%%%%%%%%
с начальным условием
$$
u_n(s;0) = 0\,, \ \ n=\overline{n_1+2,n_0-1}\,,
$$
и уравнения
\begin{multline}
\label{5-2-3-m}
u'_{n_1+1}(s;x) = - \left[s + \lambda_0\right] u_{n_1+1}(s;x) +{}
\\
{}+
\lambda_0 \int\limits_0^\infty b_0(y) \left[d_0(y,x) u^*_{n_1+2}(s;y) u(s;x)+{}\right.\\
\left.{}+
\d_0(y,x) u^*_{n_1+2}(s;x) u(s;y)\right] \, dy+{}
\\
{}+ 
\lambda_0 \int\limits_0^\infty b_0(y) \left[d_0(y,x) u_{n_1+2}(s;y) u_{n_1+1}(s;x) +{}\right.\\
\left.{}+
\d_0(y,x) u_{n_1+2}(s;x) u_{n_1+1}(s;y)\right]\, dy\,;
\end{multline}
%%%%%%%%%

\vspace*{-12pt}

\noindent
\begin{multline*}
u'_n(s;x) = - (s + \lambda_0) u_{n}(s;x) +{}\\
{}+
\lambda_0 \int\limits_0^\infty b_0(y) \left[d_0(y,x) u_{n+1}(s;y) u_{n}(s;x)
+{}\right.\\
\left.{}+\d_0(y,x) u_{n+1}(s;x) u_{n}(s;y)\right] \, dy\,,
\ \ n=\overline{1,n_1}\,,
\end{multline*}
%%%%%%%%%
с начальным условием
$$
u_n(s;0)= 1 \,,
\ \ n=\overline{1,n_1}\,.
$$

Решения уравнений~\eqref{5-1-m} и~\eqref{5-2-m} имеют вид:
%%%%%%%%%%%
$$
u_{n_0-1}(s;x)= e^{-(s + \lambda_0) x}\,;
$$
%%%%%%%%%%%%%%%%%%%%%%%%%%

\vspace*{-12pt}

\noindent
\begin{multline}
\label{5-2-4-m}
u^*_{n_0-1}(s;x) = \lambda_0 \int\limits_0^x e^{(s + \lambda_0) (z-x)}\, dz\times{}\\
{}\times \int\limits_0^\infty b_0(y) \left[d_0(y,z) u(s;y) u(s;z)+{}\right.\\
\left.{}+
\d_0(y,z) u(s;z) u(s;y)\right] \, dy\,.
\end{multline}
Остальные уравнения являются интегродифференциальными
и подобны уравнениям, полученным в предыдущих разделах.

Пусть в начальный момент в системе находится
$n$, $n\hm\ge n_1\hm+1$, заявок, система функционирует
в режиме~1, на приборе обслуживается заявка
длины~$y$ и в этот момент в систему поступает
заявка длины~$x$. Обозначим через $w(s;x,y)$ ПЛС времени ожидания
начала обслуживания этой заявки.
Тогда
$$
w(s;x,y) = d_1(x,y) + \d_1(x,y) u(s;y) \,.
$$

Пусть в начальный момент в системе находится
$n$, $n\hm=\overline{1,n_0-1}$, заявок, система
функционирует в режиме~0, на приборе обслуживается
заявка длины~$y$ и в этот момент в систему поступает
заявка длины~$x$.
Обозначим через $w_n(s;x,y)$ ПЛС времени ожидания
начала обслуживания этой заявки, причем в момент
начала обслуживания система по-преж\-не\-му будет функционировать в режиме~0.
Имеем:
$$
w_{n_0-1}(s;x,y) = 0\,;
$$
%%%%%%%%%%

\vspace*{-24pt}

\noindent
\begin{multline*}
w_n(s;x,y) = d_0(x,y) + \d_0(x,y) u_{n+1}(s;y)\, ,\\  
n=\overline{1,n_0-2}\,.
\end{multline*}

Наконец, пусть в начальный момент в системе находится~$n$, 
$n\hm=\overline{n_1+1,n_0-1}$, заявок, система
функционирует в режиме~0, на приборе обслуживается
заявка длины~$y$ и в этот момент в систему поступает
заявка длины~$x$.
Обозначим через $w^*_n(s;x,y)$ ПЛС времени ожидания
начала обслуживания этой заявки, причем в момент
начала обслуживания сис\-те\-ма окажется в режиме~1.
В~этом случае
\begin{equation}
\label{5-3-3-m}
w^*_{n_0-1}(s;x,y) = d_0(x,y) + \d_0(x,y) u(s;y)\,;
\end{equation}
%%%%%%%%%
$$
w^*_n(s;x,y) = \d_0(x,y) u^*_{n+1}(s;y)\,,\ \ n=\overline{n_1+1,n_0-2} \,.
$$


Стационарное распределение времени ожидания начала
обслуживания имеет ПЛС
\begin{multline*}
%\label{5-3-4}
w(s) = \fr{1}{ \lambda} \left[ \vphantom{\int\limits_0^\infty}
\lambda_0 p_0+{}\right.\\
{}+ \lambda_0 \int\limits_0^\infty \sum\limits_{n=1}^{n_1}
p_n(y) \, dy \int\limits_0^\infty b_0(x) w_n(s;x,y) \, dx
+{}
\\
{}+
\lambda_0 \int\limits_0^\infty \sum\limits_{n=n_1+1}^{n_0-1} p_n(0;y)\, dy\times{}\\
{}\times
\int\limits_0^\infty b_0(x) \left[w_n(s;x,y) + w^*_n(s;x,y)\right]\, dx
+{}
\\
{}+
\lambda_1 \int\limits_0^\infty \sum\limits_{n=n_1+1}^{n_0-1} p_n(1;y) \, dy
\int\limits_0^\infty b_1(x) w(s;x,y) \, dx
+{}\\
\left.{}+
\lambda_1 \int\limits_0^\infty \sum\limits_{n=n_0}^{\infty} p_n(y) \, dy
\int\limits_0^\infty b_1(x) w(s;x,y) \, dx
\right]\,.
\end{multline*}

Обозначим через $t(s;x)$ ПЛС времени от момента
первого попадания заявки длины~$x$
на прибор до момента ухода ее из системы при условии,
что в момент первого попадания на прибор система
функционировала в режиме~1.
Для $t(s;x)$ справедливо дифференциальное уравнение
\begin{multline*}
t'(s;x)= - t(s;x) \left( \vphantom{\int\limits_0^\infty}
s + {}\right.\\
\!\!\left.{}+\lambda_1 \!\left[
1 - \int\limits_0^\infty\! b_1(y)\left[\d_1(y,x) +
d_1(y,x)\, u(s;y)\vphantom{\overline{d}}\right] \, dy
\right]\!
\right)\hspace*{-1.717pt}
\end{multline*}
с начальным условием
$$
t(s;0)= 1 \,,
$$
решение которого имеет вид:
\begin{multline*}
t(s;x) = \exp\left\{ \vphantom{\int\limits_0^\infty}
-(s + \lambda_1) x +{}\right.\\
\!\!\left.{}+ \lambda_1 \!\int\limits_0^x\,\! dz\!
\int\limits_0^\infty b_1(y) \left[\,\d_1(y,z) +  d_1(y,z) u(s;y)\right] \, dy
\right\}.
\end{multline*}

Обозначим через $t_n(s;x)$, $n\hm=\overline{0,n_0-2}$,
ПЛС времени от момента первого попадания заявки длины~$x$
на прибор до момента ухода ее из системы при условии,
что в момент первого попадания на прибор в очереди
было еще $n$~заявок и система функционировала в режиме~0.
Тогда из дифференциальных уравнений
%%%%%%%%%%%%%%%%%%%
\begin{multline*}
t'_{n_0-2}(s;x) = - (s + \lambda_0) t_{n_0-2}(s;x) +{}\\
{}+
\lambda_0 \int\limits_0^\infty b_0(y) \left[d_0(y,x) u(s;y) + \d_0(y,x)\right] \, dy\, t(s;x)\,;
\end{multline*}
%%%%%%%%%%%%%%%%%%%

\vspace*{-12pt}

\noindent
\begin{multline*}
t'_n(s;x) = -\left( \vphantom{\int\limits_0^\infty}
s + \lambda_0 - {}\right.\\
\left.{}-\lambda_0 \int\limits_0^\infty b_0(y) d_0(y,x) u_{n+2}(s;y) \, dy
\right)
t_n(s;x) +{}
\\
{}+
\lambda_0 \int\limits_0^\infty b_0(y) \left[d_0(y,x) u_{n+2}^*(s;y) t(s;x) +{}\right.\\
\left.{}+
\d_0(y,x) t_{n+1}(s;x)\right] \, dy\,,
\ \ n=\overline{n_1,n_0-3}\,;
\end{multline*}
%%%%%%%%%%%%%%%%%%%%%

\vspace*{-12pt}

\noindent
\begin{multline*}
t'_n(s;x)= - \left( \vphantom{\int\limits_0^\infty}
s + \lambda_0 - {}\right.\\
\left.{}-\lambda_0 \int\limits_0^\infty b_0(y) d_0(y,x) u_{n+2}(s;y)\, dy
\right) t_n(s;x) +{}
\\
{}+
\lambda_0 t_{n+1}(s;x) \int\limits_0^\infty b_0(y) \d_0(y,x) \, dy\,,
\ \ n=\overline{0,n_1-1}\,,
\end{multline*}
%%%%%%%%%%%%%%%%%%%%%%%%%%%%%%%%%
с начальным условием
$$
t_{n}(s;0)= 1\,,\ \ n=\overline{0,n_0-2}\,,
$$
%%%%%%%%%%%%%%%%%%%%%%%%%%%%%%%
имеем:

\noindent
\begin{multline}
\label{5-4-1-m}
t_{n_0-2}(s;x)= {}\\
{}=e^{-(s + \lambda_0) x}\left(
1+ \lambda_0 \int\limits_0^x e^{(s + \lambda_0) z} t(s;z)\, dz\times{}\right.\\
\left.{}\times
\int\limits_0^\infty b_0(y) \left[d_0(y,z) u(s;y) + \d_0(y,z)\right] \, dy
\right)\,;
\end{multline}
%%%%%%%%%%%%%%%%%%%

\vspace*{-12pt}

\noindent
\begin{multline}
\label{5-4-2-m}
t_n(s;x)={}\\
{}= e^{-\int\limits_0^x \left( s + \lambda_0 -
\lambda_0 \int\limits_0^\infty b_0(y)\, d_0(y,z)\, u_{n+2}(s;y) \, dy
\right)\,dz} 
\left( \vphantom{\int\limits_0^\infty}
1 +{}\right.\\
{}+ \lambda_0 \int\limits_0^x e^{\int\limits_0^v \left(
s + \lambda_0 - \lambda_0 \int\limits_0^\infty b_0(y) d_0(y,z) u_{n+2}(s;y) \, dy
\right)\,dz} dv \times{}
\\
{}\times
\int\limits_0^\infty b_0(y) \left[d_0(y,v) u_{n+2}^*(s;y) t(s;v)+{}\right.\\
\left.\left.{}+ \d_0(y,v) t_{n+1}(s;v)
\vphantom{\int\limits_0^\infty}
\right] \, dy
\right) \,, \quad 
n=\overline{n_1,n_0-3}\,;
\end{multline}
%%%%%%%%%%%%%%%%%%%

\vspace*{-12pt}

\noindent
\begin{multline*}
t_n(s;x) = {}\\
{}=e^{-\int\limits_0^x\left( \vphantom{\int\limits_0^\infty}
s + \lambda_0 - \lambda_0 \int\limits_0^\infty b_0(y) d_0(y,z) u_{n+2}(s;y) \, dy \right)
dz} \left(  \vphantom{\int\limits_0^x}
1 +{}\right.\\
{}+ \lambda_0 \int\limits_0^x e^{\int\limits_0^v\left(
s + \lambda_0 -\lambda_0 \int\limits_0^\infty b_0(y) d_0(y,z) u_{n+2}(s;y) \, dy
\right) dz } dv \times{}\\
\left.{}\times
\int\limits_0^\infty b_0(y) \d_0(y,v) t_{n+1}(s;v) \, dy
\right)\,,
\ \ n=\overline{0,n_1-1}\,.
\end{multline*}

Обратимся к общему времени пребывания заявки в системе.

Обозначим через $v(s;x,y)$
ПЛС времени пребывания в системе заявки длины~$x$ при
условии, что эта заявка застала систему в режиме~1,
причем заявка на приборе имела длину~$y$.
Тогда
$$
v(s;x,y)= w(s;x,y) t(s;x)\,.
$$

Обозначим через
$v_{n}(s;x,y)$, $n\hm=\overline{1,n_0-1}$,
ПЛС времени пребывания в системе заявки длины~$x$ при
условии, что эта заявка застала в системе $n$~других
заявок, причем заявка на приборе имела длину~$y$, а
система пребывала в режиме~0.
Имеем:
\begin{multline}
\label{5-5-1-m}
v_{n_0-1}(s;x,y) = d_0(x,y) t(s;x) + {}\\
{}+\d_0(x,y) u(s;x,y) t(s;x)\,;
\end{multline}
%%%%%%%%%%%%%

\vspace*{-12pt}

\noindent
\begin{multline}
\label{5-5-2-m}
v_{n}(s;x,y) = d_0(x,y) t_{n}(s;x) +{}\\
{}+
\d_0(x,y) \left[ u_{n+1}(s;x,y) t_{n-1}(s;x) +{}\right.\\
\left.{}+ u^*_{n+1}(s;x,y) t(s;x)\right]\,,
\ \ n=\overline{n_1+1,n_0-2}\,;
\end{multline}
%%%%%%%%%%%%%

\vspace*{-20pt}

\noindent
\begin{multline*}
v_{n}(s;x,y)= d_0(x,y) t_{n}(s;x) + {}\\
{}+\d_0(x,y) u_{n+1}(s;x,y) t_{n-1}(s;x)\,,
\ \ n=\overline{1,n_1}\,.
\end{multline*}
%%%%%%%%%%%%%

Стационарное распределение общего времени пребывания
заявки в системе имеет ПЛС
\begin{multline*}
v(s) = \fr{1}{\lambda} \left[
\lambda_0 p_0 \int\limits_0^\infty b_0(x) t_0(s;x) \, dx
+ {}\right.\\
{}+\lambda_0 \int\limits_0^\infty \sum\limits_{n=1}^{n_1} p_n(y) \, dy
\int\limits_0^\infty b_0(x) v_n(s;x,y) \, dx +{}
\\
{}+
\lambda_0 \int\limits_0^\infty \sum\limits_{n=n_1+1}^{n_0-1}
p_n(0;y)\, dy \int\limits_0^\infty b_0(x) v_n(s;x,y) \, dx
+{}
\\
{}+
\lambda_1 \int\limits_0^\infty \sum\limits_{n=n_1+1}^{n_0-1}
p_n(1;y)\, dy \int\limits_0^\infty b_1(x) v(s;x,y) \, dx
+{}\\
\left. \lambda_1 \int\limits_0^\infty \sum\limits_{n=n_0}^{\infty}
p_n(y)\, dy \int\limits_0^\infty b_1(x) v(s;x,y)\, dx
\right]\,.
\end{multline*}

Дифференцируя $w(s)$ и $v(s)$ в точке $s\hm=0$,
можно найти моменты стационарных распределений времен
ожидания начала обслуживания и пребывания заявки в
сис\-теме.

\section{Накопитель конечной емкости}

В этом разделе будет показано, какие изменения нужно
произвести в полученных формулах для случая накопителя
конечной емкости.
Заметим, что к формулам, остающимся без изменений,
комментарии приводиться не будут.


Итак, будем предполагать, что максимальное число заявок,
находящихся в системе, равно $n^*$, $n^*\hm\ge n_0$,
(емкость накопителя $n^*\hm-1$).

Для конечного накопителя необходимо также задать
дисциплину принятия заявок в систему при отсутствии в нем
свободных мест.
В~соответствии с рассматриваемой СМО естественно такую
дисциплину определить с помощью функции $d^*(x,y)$
следующим образом: поступающая заявка длины~$x$,\linebreak
застающая на приборе заявку длины~$y$, с ве\-ро\-ят\-ностью
$d^*(x,y)$ сразу же покидает сис\-те\-му, не оказывая на нее
никакого воздействия, и с дополнительной вероятностью
$\d^*(x,y)\hm=1\hm-d^*(x,y)$ становит\-ся на прибор, вытесняя
заявку на приборе из сис\-те\-мы.
Для всех СМО с такой дисциплиной принятия заявок в сис\-те\-му
при отсутствии в накопителе свободных мест стационарные
вероятности $p_n(x_1,\ldots,x_n)$ при $n\hm<n^*$ совпадают
с точ\-ностью до постоянной с аналогичными вероятностями
для сис\-те\-мы с бесконечным накопителем, различие заключается
только в вероятностях $p_{n^*}(x_1,\ldots,x_{n^*})$.
Однако несколько более сложно вычисляются стационарные
распределения, связанные с временем пребывания заявки в
системе.
Более того, заявки, принятые в систему, могут покидать
ее недообслуженными.

Здесь для простоты изложения будет рассмотрен только
случай $d^*(x,y)\hm=1$, т.\,е.\ тот случай, когда поступающая
в заполненную систему заявка теряется.
Заметим, что в этом случае принятая в систему заявка
будет обязательно обслужена полностью.
Общий случай нетрудно исследовать с помощью результатов,
полученных в~[9, 10].

Далее будем предполагать, что $n^*\hm\ge n_0 \hm+ 2$, поскольку
при $n^*\hm=n_0$ и $n^*\hm=n_0 \hm+ 1$ расчетные формулы будут
несколько отличаться от приведенных выше.

Как уже говорилось, стационарные вероятности
$p_n(x)$, $n\hm=\overline{1,n_1}$ или
$n\hm=\overline{n_0,n^*-1}$, и
$p_n(i;x)$, $n\hm=\overline{n_1+1,n_0-1}$, $i\hm=1,2$,
с точностью до вероятности $p_0$ можно определить
из тех же самых уравнений~(\ref{3-0-m})--(\ref{3-4-m}),
что и раньше.
Вероятность $p_{n^*}(x)$ удовлетворяет дифференциальному
уравнению
\begin{equation*}
-p'_{n^*}(x)= g_{1,n^*}(x) 
\end{equation*}
с начальным условием
%%%%%%%%%%%%%%%%%%%%%%%%%%%%%%
\begin{equation*}
%\label{3-beg-1}
\lim\limits_{x\to\infty} p_{n^*}(x) = 0\,,
\end{equation*}
%%%%%%%%%%%%%%%%%%%%%%%%%%%%%%
где
%%%%%%%%%%%%%%%%%%%%%%%%%%
\begin{multline*}
%\label{3-gf-3}
g_{1,n^*}(x) = \lambda_1 b_1(x) \int\limits_0^\infty p_{n^*-1}(y) d_1(x,y)\, dy
+{}\\
{}+ \lambda_1 p_{n^*-1}(x) \int\limits_0^\infty b_1(y) \d_1(y,x)\, dy\,.
\end{multline*}
%%%%%%%%%%%%%%%%%%%%%%%%%%%%%%%
Решение этого уравнения определяется вы\-ра\-же\-нием:
\begin{equation*}
%\label{3-4}
p_{n^*}(x)= \int\limits_x^\infty g_{1,n^*}(y)\, dy\,.
\end{equation*}
%%%%%%%%%%%%%%%%%%%%%%%%%
Вероятность $p_0$ вычисляется из условия нормировки,
которое в данном случае имеет вид:
$$
p_0 + \sum\limits_{n=1}^{n_1} p_n + \sum\limits_{n=n_1+1}^{n_0-1} \left[p_{n,0} + p_{n,1}\right]+
\sum\limits_{n=n_0}^{n^*} p_n = 1\,.
$$

Стационарная интенсивность~$\lambda$ входящего в сис\-те\-му
потока задается формулой~(\ref{inten-1-m}), в которой,
естественно, верхний индекс~$\infty$ в последней сумме
заменен на~$n^*$.

В системах с конечным накопителем важной характеристикой
является стационарная вероятность $\pi_{\mathrm{loss}}$
потери заявки, определяемая формулой:
$$
\pi_{\mathrm{loss}} = \fr{\lambda_1 }{\lambda}\, p_{n^*}\,.
$$

Для того чтобы найти показатели функционирования СМО,
связанные с временем пребывания в системе, нужно прежде
всего изменить некоторые формулы для ПЛС~ПЗ.

Предположим, что в начальный момент рас\-смат\-ри\-ва\-емая
СМО функционирует в режиме~1 и в ней находится
$n$, $n\hm=\overline{n_1+1,n^*}$, заявок.
Обозначим через $\tilde{u}_n(s;x)$, $n\hm=\overline{n_1+1,n^*}$, ПЛС времени до того момента,
когда в системе впервые останется $(n-1)$ заявок,
при условии что на приборе начала
обслуживаться заявка длины~$x$ (очевидно, что в этот
момент система по-преж\-не\-му будет функционировать в режиме~1).
Преобразования Лап\-ла\-са--Стил\-тье\-са $\tilde{u}_n(s;x)$ удовлетворяют уравнениям 
\begin{align*}
\tilde{u}_{n^*}(s;x)&= e^{-sx} \,;\\
\\
\tilde{u}'_n(s;x) &= - (s + \lambda_1) \tilde{u}_{n}(s;x) +{}
\\
&{}+
\lambda_1 \int\limits_0^\infty b_1(y) \left[d_1(y,x) \tilde{u}_{n+1}(s;y) \tilde{u}_{n}(s;x)
+ {}\right.\\
&\hspace*{5mm}\left.{}+\d_1(y,x) \tilde{u}_{n+1}(s;x) \tilde{u}_{n}(s;y)\right] \, dy\,,
\\  
&\hspace*{35mm}n=\overline{n_{1}+1,n^*-1}\,,
\end{align*}
%%%%%%%%%
с начальным условием
$$
\tilde{u}_n(s;0)=1\,,
\ \ n=\overline{n_1+1,n^*-1}\,.
$$
Уравнения~(\ref{5-2-m})--(\ref{5-2-3-m}) принимают следующий вид:
%%%%%%%%%%%
\begin{multline*}
%\label{5-2}
u^{*\,\prime}_{n_0-1}(s;x) = - \left[s + \lambda_0\right] u^*_{n_0-1}(s;x)
+{}
\\
{}+
\lambda_0 \int\limits_0^\infty b_0(y) \left[d_0(y,x) \tilde{u}_{n_0}(s;y) \tilde{u}_{n_0-1}(s;x)
+{}\right.\\
\left.{}+ \d_0(y,x) \tilde{u}_{n_0}(s;x) \tilde{u}_{n_0-1}(s;y)\right]\, dy \,;
\end{multline*}
%%%%%%%%%%%

\vspace*{-12pt}

\noindent
\begin{multline*}
u^{*\,\prime}_n(s;x) = - \left[s + \lambda_0\right] u^*_{n}(s;x)
+{}
\\
{}+ \lambda_0 \int\limits_0^\infty b_0(y) \left[d_0(y,x) u^*_{n+1}(s;y) \tilde{u}_n(s;x)
+ {}\right.\\
\left.{}+\d_0(y,x) u^*_{n+1}(s;x) \tilde{u}_n(s;y)\right] \, dy +{}\\
{}+
\lambda_0 \int\limits_0^\infty b_0(y) \left[d_0(y,x) u_{n+1}(s;y) u^*_{n}(s;x)
+{}\right.\\
\left.{}+\d_0(y,x) u_{n+1}(s;x) u^*_{n}(s;y)\right]\, dy\,,
\\ n=\overline{n_1+2,n_0-2}\,;
\end{multline*}
%%%%%%%%%

\vspace*{-24pt}

\noindent
\begin{multline*}
u'_{n_1+1}(s;x) = - \left[s + \lambda_0\right] u_{n_1+1}(s;x) +{}
\\
{}+
\lambda_0 \int\limits_0^\infty b_0(y) \left[d_0(y,x) u^*_{n_1+2}(s;y) \tilde{u}_{n_1+1}(s;x)
+ {}\right.\\
\left.{}+\d_0(y,x) u^*_{n_1+2}(s;x) \tilde{u}_{n_1+1}(s;y)\right] \, dy+{}
\\
{}+
\lambda_0 \int\limits_0^\infty b_0(y) \left[d_0(y,x) u_{n_1+2}(s;y) u_{n_1+1}(s;x)
+ {}\right.\\
\left.{}+\d_0(y,x) u_{n_1+2}(s;x) u_{n_1+1}(s;y)\right] \, dy\,.
\end{multline*}
%%%%%%%%%
Соответственно изменится и формула~(\ref{5-2-4-m}).

Пусть в начальный момент в системе находится
$n$, $n\hm=\overline{n_1+1,n^*-1}$, заявок, система
функционирует в режиме~1, на приборе обслуживается
заявка длины~$y$ и в этот момент в систему поступает
заявка длины~$x$.
Обозначим через $\tilde{w}_n(s;x,y)$ ПЛС времени ожидания
начала обслуживания этой заявки.
Имеет место равенство:
\begin{multline*}
\tilde{w}_n(s;x,y) = d_1(x,y) + \d_1(x,y) \tilde{u}_{n+1}(s;y)\,,\\ 
n=\overline{n_1+1,n^*-1}\,.
\end{multline*}
Формула~(\ref{5-3-3-m}) %и (\ref{5-3-4})
принимает вид:
\begin{equation*}
%\label{5-3-3}
w^*_{n_0-1}(s;x,y) = d_0(x,y) + \d_0(x,y) \tilde{u}_{n_0}(s;y) \,,
\end{equation*}
%%%%%%%%%
а ПЛС стационарного распределения времени ожидания начала
обслуживания принятой в систему заявки определяется
формулой:
\begin{multline*}
%\label{5-3-4}
w(s) = \fr{1}{\lambda \left( 
1-\pi_{\mathrm{loss}}\right)}
\left[ \vphantom{\int\limits_0^\infty}
\lambda_0 p_0 + {}\right.\\
{}+\lambda_0 \int\limits_0^\infty \sum\limits_{n=1}^{n_1}
p_n(y) \, dy \int\limits_0^\infty b_0(x) w_n(s;x,y) \, dx +{}
\\
{}+
\lambda_0 \int\limits_0^\infty \sum\limits_{n=n_1+1}^{n_0-1} p_n(0;y) \, dy
\int\limits_0^\infty b_0(x) \left[w_n(s;x,y) +{}\right.\\
\left.{}+ w^*_n(s;x,y)
\right]
\, dx +{}\\
{}+ \lambda_1 \int\limits_0^\infty \sum\limits_{n=n_1+1}^{n_0-1}
p_n(1;y) \, dy \int\limits_0^\infty b_1(x) \tilde{w}_n(s;x,y) \, dx +{}\\
\left.{}+
\lambda_1 \int\limits_0^\infty \sum\limits_{n=n_0}^{n^*-1} p_n(y) \, dy
\int\limits_0^\infty b_1(x) \tilde{w}_n(s;x,y) \, dx\right]\,.
\end{multline*}


Обозначим через $\tilde{t}_n(s;x)$,\  $n\hm=\overline{n_1,n^*-1}$,
ПЛС времени от момента первого попадания заявки длины~$x$
на прибор до момента ухода ее из системы при условии,
что в момент первого попадания на прибор в очереди
было еще $n$~заявок и система функционировала в
режиме~1.
Тогда
%%%%%%%%%%%%%%%%%%%
$$
\tilde{t}_{n^*-1}(s;x) = e^{-sx}\,;
$$
%%%%%%%%%%%%%%%%

\vspace*{-12pt}

\noindent
\begin{multline*}
\tilde{t}_n(s;x) = \exp\left\{ \vphantom{\int\limits_0^x}
- (s + \lambda_1) x + {}\right.\\
{}+\lambda_1 \int\limits_0^x \,dz
\int\limits_0^\infty b_1(y) \left[\,\d_1(y,z) +{}\right.\\
\left.\left.{}+ d_1(y,z) \tilde{u}_{n+2}(s;y)\right] \, dy
\vphantom{\int\limits_0^\infty}
\right\}\,,\
n=\overline{n_1,n^*-2}\,.
\end{multline*}
При этом формулы~(\ref{5-4-1-m}) и~(\ref{5-4-2-m}) записываются
в виде:
\begin{multline*}
t_{n_0-2}(s;x)= e^{-(s + \lambda_0) x} \left( 
1 +
\lambda_0 \int\limits_0^x e^{(s + \lambda_0) z}\, dz\times{} \right.\\
{}\times
\int\limits_0^\infty b_0(y) \left[d_0(y,z) \tilde{u}_{n_0}(s;y) \tilde{t}_{n_0-2}(s;z)
+ {}\right.\\
\left.\left.{}+\d_0(y,z) \tilde{t}_{n_0-1}(s;z)\right] \, dy
\vphantom{\int\limits_0^x}
\right)\,;
\end{multline*}
%%%%%%%%%%%%%%%%%%%

\vspace*{-12pt}

\noindent
\begin{multline*}
%\label{5-4-2}
t_n(s;x) = {}\\
{}=e^{- \int\limits_0^x \left(
s + \lambda_0 - \lambda_0 \int\limits_0^\infty b_0(y) d_0(y,z) u_{n+2}(s;y) \, dy\right)dz}
\left(  \vphantom{\int\limits_0^x}
1 +{}\right.\\
{}+ \lambda_0 \int\limits_0^x e^{ \int\limits_0^v \left(
s + \lambda_0 - \lambda_0 \int\limits_0^\infty b_0(y) d_0(y,z) u_{n+2}(s;y) \, dy \right)dz}
dv \times{}\\
{}\times
\int\limits_0^\infty b_0(y) \left[d_0(y,v) u_{n+2}^*(s;y) \tilde{t}_n(s;v)
+{}\right.\\
\left.\left.{}+ \d_0(y,v) t_{n+1}(s;v)\right] \, dy 
\vphantom{\int\limits_0^x}
\right)\,,
\ \ n=\overline{n_1,n_0-3}\,.
\end{multline*}
%%%%%%%%%%%%%%%%%%%

Наконец, перейдем к общему времени пребывания заявки в
системе.
Обозначим через
$\tilde{v}_{n}(s;x,y)$, $n\hm=\overline{n_1+1,n^*-1}$,
ПЛС времени пребывания в сис\-те\-ме заявки длины~$x$ при
условии, что эта заявка застала в системе $n$~других
заявок, причем заявка на приборе имела длину~$y$, а
система пребывала в режиме~1.
Справедливо соотношение:
\begin{multline*}
\tilde{v}_{n}(s;x,y) = d_1(x,y) \tilde{t}_{n}(s;x) +{}\\[2pt]
{}+\d_1(x,y) \tilde{u}_{n+1}(s;x,y) \tilde{t}_{n-1}(s;x)\,,\\[2pt]
  n=\overline{n_1+1,n^*-1}\,.
\end{multline*}
%%%%%%%%%%%%%
Формулы~(\ref{5-5-1-m}) и~(\ref{5-5-2-m}) преобразуются
следующим образом:
\begin{multline*}
%\label{5-5-1}
v_{n_0-1}(s;x,y)= d_0(x,y) \tilde{t}_{n_0-1}(s;x) +{}\\
{}+
\d_0(x,y) \tilde{u}_{n_0}(s;x,y) \tilde{t}_{n_0-2}(s;x) \,;
\end{multline*}
%%%%%%%%%%%%%
\vspace*{-12pt}

\noindent
\begin{multline*}
%\label{5-5-2}
v_{n}(s;x,y) = d_0(x,y) t_{n}(s;x) +{}
\\
{}+
\d_0(x,y) \left[ u_{n+1}(s;x,y) t_{n-1}(s;x) +{}\right.\\
\left.{}+
u^*_{n+1}(s;x,y) \tilde{t}_{n-1}(s;x) \right] \,,
\ \ n=\overline{n_1+1,n_0-2}\,,
\end{multline*}
а ПЛС стационарного распределения общего времени пребывания
в системе принятой к обслуживанию заявки имеет вид:
\begin{multline*}
v(s)= \fr{1}{ \lambda \left(1-\pi_{\mathrm{loss}}\right)}
\left[
\lambda_0 p_0 \int\limits_0^\infty b_0(x) t_0(s;x) \, dx
+{}\right.\\
\left.{}+ \lambda_0 \int\limits_0^\infty \sum\limits_{n=1}^{n_1}
p_n(y) \, dy \int\limits_0^\infty b_0(x) v_n(s;x,y) \, dx +{}\right.
\\
{}+
\lambda_0 \int\limits_0^\infty \sum\limits_{n=n_1+1}^{n_0-1}
p_n(0;y)\, dy \int\limits_0^\infty b_0(x) v_n(s;x,y) \, dx +{}\\
{}+ \lambda_1 \int\limits_0^\infty \sum\limits_{n=n_1+1}^{n_0-1}
p_n(1;y)\, dy \int\limits_0^\infty b_1(x) \tilde{v}_n(s;x,y) \, dx+{}\\
\left.{}+ \lambda_1 \int\limits_0^\infty \sum\limits_{n=n_0}^{n^*-1} p_n(y)\, dy
\int\limits_0^\infty b_1(x) \tilde{v}_n(s;x,y)\, dx
\right]\,.
\end{multline*}

\section{Заключение}


В настоящей статье рассмотрена возможность\linebreak применения
аналитических методов для вы\-чис\-ле\-ния основных стационарных
характеристик функ\-ци\-о\-ни\-ро\-ва\-ния СМО, в которых
одновременно имеется два отличия от
классических СМО:\linebreak инверсионный порядок обслуживания с
вероятностным приоритетом и гистерезисная политика.
На примере однолинейной СМО с простейшим вариантом
гистерезисной политики показано, что полученные
вычислительные алгоритмы основаны на интегральных
и дифференциальных уравнениях, которые могут быть решены
на современной вы\-чис\-ли\-тель\-ной технике.
Приведены условия, при которых интегральные уравнения
могут быть сведены к системам линейных алгебраических
уравнений.

Полученные результаты могут служить основой для
продолжения работ в части математического моделирования
технических систем, ис\-поль\-зу\-ющих как элементы
<<нестандартных>> дисциплин обслуживания, так и сложные
варианты гистерезисного механизма предотвращения
различного рода перегрузок в ИТС.


{\small\frenchspacing
{%\baselineskip=10.8pt
\addcontentsline{toc}{section}{Литература}
\begin{thebibliography}{99}

\bibitem{1-m}
\Au{Печинкин А.\,В.} Об одной инвариантной системе массового
обслуживания~// Math.\ Operationsforsch.\ und Statist. Ser.\
Optimization, 1983. Vol.~14. No.\,3. S.~433--444.

\bibitem{2-m}
\Au{Печинкин А.\,В., Стальченко И.\,В.} Система $MAP/G/1/\infty$ с
инверсионным порядком обслуживания и вероятностным приоритетом,
функционирующая в дискретном времени~// Вестник Российского
ун-та дружбы народов. Сер.\ Математика. Информатика. Физика,
2010. №\,2. С.~26--36.

\bibitem{3-m}
\Au{Абаев П.\,О., Гайдамака Ю.\,В., Самуйлов~К.\,Е.} Гистерезисное
управление сигнальной нагрузкой в сети SIP-сер\-ве\-ров~// Вестник
Российского ун-та дружбы народов. Сер.\ Математика.
Информатика. Физика, 2011. №\,4. С.~54--71.

\bibitem{7-m}
\Au{Nishimura S., Jiang~Y.}
An $M/G/1$ vacation model with two service modes~//
Prob.\ Eng. Informational Sci., 1995. Vol.~9. No.\,3. P.~355--374.

\bibitem{8-m}
\Au{Dudin A.}
Optimal control for an $M^x/G/1$ queue with two operation
modes~// Prob. Eng.  Informational Sci.,
1997. Vol.~11. No.\,2. P.~255--265.

\bibitem{9-m}
\Au{Nobel R.\,D., Tijms H.\,C.} Optimal control for an $M^X/G/1$
queue with two service mo\-des~// Eur. J.~Operational
Res., 1999. Vol.~113. No.\,3. P.~610--619.

\bibitem{10-m}
\Au{Жерновый К.\,Ю., Жерновый Ю.\,В.} Система $M^\theta/G/1/m$ c
двухпороговой гистерезисной стратегией переключения интенсивности
обслуживания~// Информационные процессы, 2012. Т.~12. №\,2. С.~127--140.

\bibitem{22-m}
\Au{Bocharov~P.\,P., D'Apice~C., Pechinkin~A.\,V., Salerno~S.}
Queueing theory.~--- Ut\-recht, Boston: VSP, 2004.

\bibitem{5-m}
\Au{Нагоненко В.\,А.} О~характеристиках одной нестандартной сис\-те\-мы
массового обслуживания. I~// Изв.\ АН СССР. Технич.\ кибернет.,
1981. №\,1. С.~187--195.

\label{end\stat}

\bibitem{6-m}
\Au{Нагоненко В.\,А.} О~характеристиках одной нестандартной сис\-те\-мы
массового обслуживания. II~// Изв.\ АН СССР. Технич.\ кибернет.,
1981. №\,3. С.~91--99.
\end{thebibliography}
}
}

\end{multicols}  %3Abst+avt
\newcommand{\Cov}{\mathrm{Cov}}
\newcommand{\Nor}{\ensuremath{\mathcal{N}}}
\newcommand{\Pu}{\mathbb{P}}


\def\stat{luk-mor}

\def\tit{О СХОДИМОСТИ В ПРОСТРАНСТВЕ $L_p$ %{\boldmath{$L_p$}}  
МАКСИМУМА ПРОЦЕССА НАГРУЗКИ ДЛЯ ОДНОГО КЛАССА
ГАУССОВСКИХ~СИСТЕМ ОБСЛУЖИВАНИЯ$^*$}

\def\titkol{О сходимости в пространстве $L_p$  максимума процесса нагрузки для одного класса
гауссовских систем обслуживания}

\def\autkol{О.\,В.~Лукашенко, Е.\,В.~Морозов}

\def\aut{О.\,В.~Лукашенко$^1$, Е.\,В.~Морозов$^2$}

\titel{\tit}{\aut}{\autkol}{\titkol}

{\renewcommand{\thefootnote}{\fnsymbol{footnote}}\footnotetext[1]
{Работа выполняется при финансовой поддержке Программы стратегического 
развития ПетрГУ   в рамках реализации комплекса мероприятий  по развитию 
научно-исследовательской деятельности.}}

\renewcommand{\thefootnote}{\arabic{footnote}}
\footnotetext[1]{Институт прикладных математических исследований
Карельского научного центра Российской академии наук; Петрозаводский
государственный университет, lukashenko-oleg@mail.ru}
\footnotetext[2]{Институт прикладных математических исследований
Карельского научного центра Российской академии наук; Петрозаводский
государственный университет, emorozov@karelia.ru}

\Abst{Рассматривается класс систем обслуживания, на вход которых
поступает поток, содержащий линейную детерминированную компоненту и
случайную компоненту, описываемую центрированным гауссовским
процессом. Дисперсия входного процесса  является правильно
меняющейся функцией  с показателем  $V\hm\in (0,\,2)$. Найдены условия,
при которых  максимум  стационарного процесса нагрузки
(незавершенной работы) на интервале $[0,\,t]$ сходится при $t\hm\to
\infty$ (и при соответствующей нормировке) в пространстве $L_p$   к
явно выписанной константе. Также найдена асимптотика максимума
процесса нагрузки в нестационарном режиме. Получена асимптотика
минимального времени достижения процессом нагрузки растущего
значения~$b$.}


\KW{гауссовская система обслуживания; максимум
процесса нагрузки; дробное броуновское движение; асимптотический
анализ;  правильное изменение}

\vskip 14pt plus 9pt minus 6pt

      \thispagestyle{headings}

      \begin{multicols}{2}

            \label{st\stat}

\section{Введение}

В работе~\cite{Lukashenko} был осуществлен асимптотический анализ
процесса нагрузки в жидкостной  системе  с постоянной скоростью
обслуживания~$C$  и входным  гауссовским процессом, дисперсия
которого правильно меняется на бесконечности с показателем $V\hm\in
(0,\,2)$. Рассмотренный  класс входных процессов включает, в
частности, сумму независимых дробных броуновских движений (ДБД) с
соответствующими значениями индекса Херста. В~\cite{Lukashenko}
показано, что при соответствующей нормировке максимум  процесса нагрузки
  на интервале $[0,\,t]$ сходится по вероятности при  $t\hm\to \infty$  к  явно выписанной
константе.  Этот результат существенно обобщает результат
из работы~\cite{Zeevi}, в которой рассмотрена  жидкостная система с
единственным входным процессом ДБД.

В данной статье, которая  опирается на методы работы~\cite{Zeevi}, а
также на результаты  статьи~\cite{Lukashenko}, продолжен
асимптотический анализ описанной жидкостной системы. Основной
результат данной работы состоит в том, что при некоторых
дополнительных условиях на асимптотическое поведение дисперсии
входного гауссовского процесса доказанная  в~\cite{Lukashenko}
сходимость усилена до сходимости (к той же константе) в пространстве~$L_p$ 
при любом  $p\hm\ge 1$. Более того, при соответствующей
нормировке найдена асимптотика максимума процесса нагрузки в
нестационарном режиме. Кроме того, с использованием полученной
асимптотики максимума процесса нагрузки найдена асимптотика  времени
достижения процессом нагрузки растущего уровня~$b$.

Опишем рассматриваемую систему и используемые ниже результаты из~\cite{Lukashenko} 
более подробно.  Рассмотрим жидкостную систему с
одним обслуживающим устройством и постоянной скоростью обслуживания~$C$, 
на вход которой поступает процесс $A(t)$, заданный  в следующем виде:
\begin{equation}
A(t)=mt+X(t)\,, 
\label{asymp-l1}
\end{equation}
где  $m$~--- средняя интенсивность входного потока, а
$X:\hm=\{X(t)$, $t \hm\geq 0\}$~---  центрированный гауссовский процесс со
стационарными приращениями,  $X(0)\hm=0$. Такая система обслуживания
называется гауссовской~\cite{Mandjes}.  Будем считать, что выполнено
условие $r:=C\hm-m\hm>0$. Обозначим также $W(t)\hm=X(t)\hm-rt$, и пусть
 $Q(t)$  есть
величина нагрузки (незавершенная работа в системе) в момент времени~$t$. 
Будем предполагать, что  $Q(0)\hm=0$. Тогда имеет место соотношение~ \cite{Reich}:
\begin{equation}
Q(t)=\sup\limits_{0 \leq s \leq t}(W(t)-W(s))\,.\label{6a}
\end{equation}
Условие  $r\hm>0$ обеспечивает существование стационарного процесса
нагрузки,  который определяется следующим образом~\cite{Mandjes}:
\begin{equation}
Q= \sup\limits_{t \geq 0} W(t)\,.\label{6}
\end{equation}
 Для пояснения заметим, что величина $-r\hm<0$ есть средний снос
процесса~$W$, являющегося аналогом случайного блуждания (с
независимыми приращениями), максимум которого, по аналогии с~(\ref{6}), 
определяет стационарное время ожидания в классических системах обслуживания~\cite{Asmus}.

Основное предположение, принятое в работе~\cite{Lukashenko}, а также
в данной статье, состоит в том, что функция  $v(t)$ {\it правильно
меняется на бесконечности c индексом} $0\hm<V\hm<2$, т.\,е.\ представима в виде
\begin{equation}
v(t)=t^V L(t)\,,\label{4}
\end{equation}
где функция $L$  медленно меняется на бесконеч\-ности~\cite{Seneta}.
Обозначим 
$$
\beta=\fr{1}{2-V}\,,
$$ 
а также выберем и зафиксируем
любое $\varepsilon \hm\in (0,2-V)$. Будем  считать, что функция~$L$
является {\it дважды дифференцируемой} на $\mathbb{R}_+$ и выполнены
следующие условия (при $t \hm\to \infty$):
\begin{align}
L(tL^\beta(t)) &\sim L(t)\,;\label{10}\\
L''(t)&=o\left( \fr{1}{t^{V+\varepsilon}} \right)\,.\label{11a}
\end{align}
Нетрудно проверить, что из условия~\eqref{11a} следует сходимость
\begin{equation}
v''(t)\ln t \to 0\,,\enskip t \to \infty\,.\label{2.2.l20}
\end{equation}
Как показано в~\cite{Lukashenko},  условия~(\ref{4})--(\ref{11a}) на
самом деле выполнены для широкого класса гауссовских сис\-тем
обслуживания, включающего, например, сис\-те\-мы, на вход которых
поступает сумма нескольких независимых ДБД.
 В~работе~\cite{Konstantopoulos} показано, что на одном вероятностном пространстве можно
 задать процесс $W(t)\hm=X(t)\hm-rt$ и стационарный процесс $Q^*:=\{Q^*(t),\ t \hm\geq 0\}$
таким образом, что одновременно выполнены условия:
\begin{align*}
Q^*(t)&=_d Q \ \mbox{ для всех } \ t \geq 0\,; %\label{15}
\\
Q^*(t)&=W(t)+\max\left\{Q^*(0), L^*(t)\right\}\,,\,\,t \geq 0\,, %\label{16}
\end{align*}
где $=_d$ означает равенство по распределению, а
$L^*(t):=-\min\limits_{0\le s\le t}\{W(s)\}$. Обозначим
\begin{equation*}
M(t)=\max\limits_{0 \leq s \leq t}Q(s)\,,;\enskip M^*(t)=\max_{0 \leq s \leq
t}Q^*(s)\,.
%\label{13}
\end{equation*}
Для удобства обозначим далее
\begin{equation*}
\gamma(t)=L\left[\left(\ln t \right)^\beta\right]  \ln t \,.
\end{equation*}
Ниже будем опираться на результаты следующей теоремы, доказанной
в работе~\cite{Lukashenko}.

\medskip

\noindent
\textbf{Теорема~1.1.}\ \ 
\textit{Пусть дисперсия гауссовской компоненты $X$ входного
процесса}~(\ref{asymp-l1}) \textit{удовлетворяет условиям}~(\ref{10}) 
\textit{и}~(\ref{11a}), \textit{а также} $r\hm>0$. \textit{Тогда}
\begin{align}
\fr{M^*(t)}{\gamma^\beta(t)} &\Rightarrow
\left(\fr{1}{\theta}\right)^\beta\,,\enskip t \to \infty\,;
\label{asymp1-l8}
\\
\fr{M(t)}{\gamma^\beta(t)} &\Rightarrow
\left(\fr{1}{\theta}\right)^\beta\,,\enskip t \to \infty\,,
\label{asymp1-l9}
\end{align}
\textit{где} $\Rightarrow$ \textit{означает сходимость по вероятности, а параметр}~$\theta$ 
\textit{удовлетворяет соотношению}:
\begin{equation}
\theta=\fr{2}{(2-V)^{2-V}}\left( \fr{r}{V}
\right)^V\,.\label{logbuff-l6}
\end{equation}

\smallskip


Как отмечено выше, этот результат  обобщает работу~\cite{Zeevi}, где
процесс $X\hm=B_H$ является ДБД  c параметром Херста  $H\hm\in (1/2,\,1)$.

\section{Сходимость в пространстве $L_p$ %{\boldmath{$L_p$}}
}

В данном разделе будет доказан основной результат, состоящий в том,
что при  дополнительных условиях  на функцию~$L$ из~(\ref{4})
сходимость по вероятности в~(\ref{asymp1-l8}), (\ref{asymp1-l9})
можно усилить до сходимости в пространстве~$L_p$, где
$p\hm\in[1,\,\infty)$.


\medskip

\noindent
\textbf{Теорема~2.1.}\ \ 
\textit{Пусть дополнительно к условиям теоремы}~1.1
\begin{equation}
\liminf\limits_{t\to \infty} L(t)>0\,;\enskip
\limsup\limits_{t\to \infty} L(t)<\infty\,.
%\leq A_2.
\label{10a}
\end{equation}

\textit{Тогда в}~(\ref{asymp1-l8}), (\ref{asymp1-l9}) \textit{имеет место
сходимость в пространстве $L_p$, $p \hm\in [1,\infty)$}.


\smallskip


\noindent
Д\,о\,к\,а\,з\,а\,т\,е\,л\,ь\,с\,т\,в\,о\,.\ \ 
Зафиксируем $p \hm\in [1,\infty)$. Для доказательства теоремы
достаточно доказать равномерную интегрируемость семейства случайных величин
$$
\left\{\left(\fr{ M^*(t)}{\gamma^\beta(t)}\right)^p,\enskip t\ge
T\right\}\,,
$$
где $T$~--- некоторое (конечное) положительное чис\-ло. Для этого, в
свою очередь, достаточно, чтобы было выполнено (см., например,~\cite{Billingsley}) условие
\begin{equation}
\label{2.2.l1} 
\sup\limits_{t \geq T} \mathbb{E} \left[
\fr{M^*(t)}{\gamma^\beta(t)} \right]^{p+1} < \infty\,.
\end{equation}
 (Значение $\mathbb{E}(\cdot)$ при $t\hm<T$
может быть произвольным.) Выберем далее некоторое число $K\hm>0$.
Значения величин~$K$ и~$T$ будут уточняться в процессе
доказательства. Кроме того, всюду далее предполагается, что $t\hm\ge
T$. Имеют место  соотношения:
\begin{multline}
\mathbb{E} \left[ \fr{M^*(t)}{\gamma^\beta(t)}\right]^{p+1} = {}\\
{}=
\int\limits_{0}^\infty (p+1)y^p \Pu\left(M^*(t)>y \gamma^\beta(t)\right)dy={}\\
{}= \int\limits_{0}^K (p+1)y^p \Pu\left(M^*(t)>y \gamma^\beta(t)\right)dy+{}\\
{}+\int\limits_{K}^\infty (p+1)y^p \Pu\left(M^*(t)>y \gamma^\beta(t)\right)dy \leq{} \\
{}\leq K^{p+1} + (p+1)(I_t+R_t)\,,\label{2.2.l9}
\end{multline}
где использованы обозначения:
\begin{align*}
I_t&=\int\limits_{K}^\infty y^p \,\lceil t\rceil\Pu\left(Q^*(0)>\fr{y
\gamma^\beta(t)}{2}\right)dy\,;\\
 R_t&=\int\limits_{K}^\infty y^p\, \lceil t \rceil \times{}\\
& {}\times \Pu\left(
 \max\limits_{0\leq s \leq 1}W(s)-\min\limits_{0 \leq s \leq 1}W(s)>\fr{y
\gamma^\beta(t)}{2}\right)dy\,.
\end{align*}
Отметим, что при получении выражения~(\ref{2.2.l9})
использовано неравенство:
\begin{multline*}
\Pu(M^*(t)>x) \le {}\\
{}\le \lceil t\rceil\Pu\left(Q^*(0)+
\max\limits_{0 \leq s \leq 1}W(s)-\min\limits_{0 \leq s \leq 1} W(s)>x\right).
\end{multline*}
(Cм.\ доказательство теоремы~1.1 в~\cite{Lukashenko}.) Оценим вначале
 интеграл $I_t$.
 В~работе~\cite{Duffy} показано, что если дисперсия $v(t)$
 центрированного гауссовского процесса со стационарными
приращениями  правильно меняется на бесконечности  с индексом
$0\hm<V\hm<2$, то справедлива такая (логарифмическая) асимптотика:
\begin{equation}
\lim\limits_{b \to \infty} \fr{v(b)}{b^2} \ln \Pu(Q^*>b)=-\theta\,,
\label{asymp1-l13}
\end{equation}
где параметр $\theta$ удовлетворяет соотношению~\eqref{logbuff-l6}.
Определим  число $K_1$ следующим образом:
\begin{multline*}
K_1=\inf\left\{ y>0:\,\,\fr{L(x)\ln\Pu(Q^*(0)>x)}{x^{1/\beta}}\leq -
\fr{\theta}{2}\,,\right.\\ 
\left.\forall\ x>y\vphantom{\fr{L(x)Q^*}{x^{1/\beta}}}\right\}\,.
\end{multline*}
Напомним, что $\beta\hm={1}/({2-V})$. Поэтому ввиду~(\ref{4}) 
из~(\ref{asymp1-l13}) следует, что $K_1\hm<\infty$. Тогда при $x\hm>K_1$
справедливо неравенство:
\begin{equation}
\label{2.2.l2}
\Pu(Q^*(0)>x) \leq \exp\left( -\fr{\theta}{2}\, \fr{x^{1/\beta}}{L(x)} \right)\,.
\end{equation}
Заметим, что $\gamma^\beta(t) \to \infty$, $t \hm\to \infty$.
Следовательно, существует такое число $t_0$, что
$\gamma^\beta(t)/2\hm>1$ при $t \hm\geq t_0$. Поэтому, если $K\hm>K_1$, $t
\hm\geq t_0$, то из~(\ref{2.2.l2}) вытекает неравенство:
\begin{multline}
I_t = \int\limits_{K}^\infty y^p \lceil t\rceil\Pu\left(Q^*(0)>\fr{y 
\gamma^\beta(t)}{2}\right)dy \leq \int\limits_{K}^\infty y^p \lceil t\rceil\times{}\\
{}\times \exp\left[
-\fr{\theta}{2^{1/\beta+1}} \, \fr{\ln t \cdot L\left[ (\ln
t)^\beta\right]}{L(y\gamma^\beta(t)/2)}\, y^{1/\beta}
\right]dy\,.\label{18}
\end{multline}
Из условия~(\ref{10a}) следует, что существуют такие числа $0\hm<A_1\hm\le
A_2\hm<\infty $ и $K_2$, $t_1\hm\ge 0$, что при $y\hm>K_2$, $t\hm>\max(t_0,t_1)$
выполняются неравенства:
\begin{align}
\label{2.2.l5} 
L\left(\fr{y\gamma^\beta(t)}{2}\right) &\le A_2\,;
\\
\label{2.2.l6} L\left( (\ln t)^\beta \right)&\ge A_1\,.
\end{align}
Выберем теперь и временно зафиксируем в~(\ref{18})  некоторое
$K\hm>\max\{K_1,K_2\}$, и пусть  пока $T:=$\linebreak $:=\;\max(t_0,t_1)$. Обозначим также 
$$
\alpha=\fr{\theta A_1}{2^{1/\beta+1}A_2}\,.
$$
Последовательно применяя~(\ref{2.2.l5}), (\ref{2.2.l6}), а также
принимая во внимание, что $\lceil t\rceil \hm\leq 2t$, можно получить
из~(\ref{18})  (при $t\hm\geq T$) следующую оценку сверху интеграла
$I_t$:
\begin{multline}
I_t \leq 2 \int\limits_{K}^\infty y^p\, t\exp\left( - \alpha
\ln t \cdot y^{1/\beta}\right)dy={}\\
{}= 2\int\limits_{K}^\infty y^p\exp\left[ \ln t
\left(1-\alpha y^{1/\beta} \right)
\right]dy\,. \label{2.2.l12}
\end{multline}
Заметим, что при $y\hm>\left({2}/{\alpha}\right)^\beta :=K_3$
справедливо неравенство:
\begin{equation}
\label{2.2.l7}
1-\alpha y^{1/\beta} < -\fr{\alpha}{2}y^{1/\beta}\,.
\end{equation}
Если теперь выбрать в~(\ref{2.2.l12}) $K\hm>K_3$, то ввиду~(\ref{2.2.l7}) получим:
\begin{equation}
I_t \leq  2\int\limits_{K}^\infty y^p\exp\left( -\fr{\alpha}{2} \ln t
\cdot y^{1/\beta}\right)dy\,. \label{2.2.l13}
\end{equation}
Заметим, что $\ln t \hm\geq  \ln T$ при $t \hm\geq T$, и обозначим
$\gamma_1\hm=({\alpha}/{2})\ln T$. Тогда  из~(\ref{2.2.l13}) получим:
\begin{multline}
I_t \leq  2\int\limits_{K}^\infty y^p\exp\left( -\gamma_1 y^{1/\beta}\right)dy={}\\
{}= \fr{2\beta}{\gamma_1^{\beta+\beta p}}\int\limits_{\gamma_1
K^{1/\beta}}^\infty u^{\beta p+\beta-1}e^{-u}du={}\\
{}=\fr{2\beta}{\gamma_1^{\beta+\beta p}}\, \Gamma\left(\gamma_1
K^{1/\beta},\beta p+\beta \right)\,,\label{2.2.l10}
\end{multline}
где $\Gamma(w,z)$~--- неполная гам\-ма-функ\-ция:
$$
\Gamma(w,z):=\int\limits_w^\infty u^{z-1}e^{-u}\,du\,, \enskip w\ge 0\,,\ z \geq 0\,.
$$


Теперь оценим интеграл $R_t$ в~\eqref{2.2.l9}. Напомним соотношение
$W(t)=X(t)-rt$ и  заметим, что
\begin{equation}
\max\limits_{0 \leq s \leq 1}W(s)=\max\limits_{0 \leq s \leq 1}( X(s)-rs)
\leq \max\limits_{0 \leq s \leq 1} X(s)\,. \label{asymp1-l11}
\end{equation}
Кроме того,
\begin{multline}
-\min\limits_{0 \leq s \leq 1} W(s)=\max\limits_{0 \leq s \leq 1} (
rs-X(s))=_d{}\\
{}=_d\max\limits_{0 \leq s \leq 1} (rs+ X(s))
\leq r+ \max\limits_{0 \leq s \leq 1} X(s)\,.\label{2.2.l18}
\end{multline}
Неравенства~\eqref{asymp1-l11} и \eqref{2.2.l18} после несложных
преобразований приводят, в свою очередь, к неравенству:
\begin{multline}
\Pu\left(\max\limits_{0\leq s \leq 1}
W(s)-\min\limits_{0 \leq s \leq 1}W(s)>\fr{y \gamma^\beta(t)}{2}\right) \leq {}\\
{}\leq
2 \Pu\left(\max\limits_{0\leq s \leq 1}X(s)>
\fr{y\gamma^\beta(t)}{4}-r\right)\,. \label{2.2.l19}
\end{multline}
Применяя соотношения~\eqref{2.2.l19} и (\ref{2.2.l6}), можно получить
следующую цепочку неравенств:
\begin{multline}
R_t = \int\limits_{K}^\infty y^p \lceil t
\rceil\times{}\\
{}\times \Pu\left(\vphantom{\fr{\gamma^\beta}{2}}
\max_{0\leq s \leq 1}
W(s)-\min\limits_{0 \leq s \leq 1}W(s)>
\fr{y \gamma^\beta(t)}{2}\right)\,dy \leq{} \\
{}\leq 2 \int\limits_{K}^\infty y^p \lceil
t \rceil\Pu\left(\max\limits_{0\leq s \leq 1}X(s)>\fr{y\gamma^\beta(t)}{4}-r\right)dy \leq{} \\
\!\!{}\leq 4 \int\limits_{K}^\infty y^p  t
\Pu\left(\max\limits_{0\leq s \leq 1}X(s)>
\fr{y A_1^\beta(\ln t)^\beta}{4}-r\right)dy.\!\! \label{2.2.l11}
\end{multline}
Теперь   используем  следующее неравенство
Бо\-ре\-ля--Су\-да\-ко\-ва--Ци\-рель\-со\-на для максимума центрированного
гауссовского процесса со стационарными приращениями на конечном
интервале~\cite{Adler, Lifshits}:
\begin{equation}
\Pu \left( \max\limits_{0 \leq s \leq 1} X(s)> x \right) \leq
e^{-({1}/{(2v)})(x-a)^2}\,, \enskip x>a\,,
\label{2.2.l14}
\end{equation}
где $v:=\mathbb{D} X(1)$, $a:=\mathbb{E} \max\limits_{0 \leq s \leq 1} X(s)<\infty$.
Положим
$$
K_4=\inf\left\{y:\,\,\fr{x A_1^\beta(\ln T)^\beta}{4}-r>a\,,\ \forall\  x \geq y\right\}\,.
$$
Тогда при  $x\ge K_4$ неравенство~(\ref{2.2.l14}) выполнено для всех
$z:={x A_1^\beta(\ln T)^\beta}/{4}-r$,   причем
$ z\hm>a$.

  Введем обозначения:
$$
c_1=\fr{4(r+a)}{A_1^\beta}\,;\quad c_2=\fr{32 v }{A_1^{2\beta}}\,.
$$
Пусть теперь  $K\hm>K_4$ в~(\ref{2.2.l11}).  Тогда с учетом~(\ref{2.2.l14}) 
после несложных преобразований можно получить, что
\begin{equation}
R_t \leq 4 \int\limits_{K}^\infty y^p \exp\left[ \ln t -
\fr{\left( y(\ln t)^\beta-c_1\right)^2}{c_2} \right]dy\,.
\label{2.2.l15}
\end{equation}
Рассмотрим функцию
$$
f(t,y):=\ln t -\fr{\left(y(\ln t)^\beta-c_1\right)^2}{c_2}\,.
$$
 Нетрудно убедиться, что
\begin{equation}
\label{2.2.l3} 
\fr{\partial f(t,\,y)}{\partial t}=\fr{1}{t}\left[ 1-\fr{2\beta}{c_2}\left(y(\ln
t)^\beta-c_1\right)(\ln t)^{\beta-1} \right]
\end{equation}
и что при любом $y\hm>K$
$$
\fr{\partial
f}{\partial t}\left(t,y\right)<\fr{\partial f}{\partial t}(t,K)\,.
$$
Анализ правой части выражения (\ref{2.2.l3}) показывает, что
существует такое число $t_2$, что функция $f(t,y)$ (при каждом
$y\hm>K$) убывает (по~$t$) при $t\hm\geq t_2$. Это, в свою очередь, означает, что
\begin{equation}
\label{2.2.l4}
f(t,y) \leq f(t_2,y)\,,\enskip t\geq t_2\,,\enskip y>K\,.
\end{equation}
Обозначим $T\hm=\max\{t_0,\,t_1,\,t_2\}$. Заметим, что
$f(T,\,y) \hm\to -\infty$, $y \hm\to \infty$, и, как нетрудно проверить,
$$
\lim\limits_{y \to \infty} y^{p+2}e^{f(T,\,y)}=0\,.
$$
Поэтому найдется такое число $K_5\hm>0$, что
\begin{equation}
\label{2.2.l8} 
e^{f(T,\,y)}<y^{-p-2}\,,\enskip y>K_5\,.
\end{equation}
Теперь, используя соотношения~(\ref{2.2.l4}) и (\ref{2.2.l8}),
получим из~(\ref{2.2.l15})  при  $K\hm>K_5$ (и $t \hm\geq T$)
\begin{equation}
R_t  \leq 4 \int\limits_{K}^\infty y^p e^{f(T,y)}dy \leq 
4 \int\limits_{K}^\infty \fr{dy}{y^2}=\fr{4}{K}\,.
\label{2.2.l16}
\end{equation}
Выберем окончательно в~(\ref{2.2.l9}) (и далее, где требуется)
$K\hm=\max \left\{K_1,K_2,K_3,K_4,K_5\right\}$. Тогда из~(\ref{2.2.l10}) и~(\ref{2.2.l16}) 
следует неравенство
\begin{multline}
\mathbb{E} \left[ \fr{M^*(t)}{\gamma^\beta(t)}\right]^{p+1} \leq 
K^{p+1} + {}\\
{}+\fr{2\beta(p+1)}{\gamma_1^{\beta+\beta p}}\, \Gamma\left(
\gamma_1 K^{1/\beta},\beta p+\beta \right) + \fr{4(p+1)}{K}\,, 
\label{2.2.l17}
\end{multline}
 правая часть которого не зависит от~$t$. Беря  в левой части
неравенства~\eqref{2.2.l17}  супремум по  $t \hm\geq T$, получаем требуемое 
условие~(\ref{2.2.l1}). \hfill$\square$


\smallskip

\noindent
\textbf{Замечание.}\ Если существует предел
\begin{equation}
\lim\limits_{t\to \infty} L(t)= A\in (0,\,\infty)\,,
\label{38}
\end{equation}
то
условие~\eqref{10a} автоматически вы\-пол\-нено.

\smallskip

Приведем важные для практических применений примеры, когда условия
теоремы~2.1 вы\-пол\-нены.

\smallskip

\noindent
\textbf{Пример 1.}\ Пусть стохастическая компонента входного
процесса является суммой независимых ДБД, т.\,е.
\begin{equation*}
X(t)=\sum\limits_{i=1}^n B_{H_i}(t)\,, \enskip t\ge 0\,,
%\label{42}
\end{equation*}
где параметры Херста $H_i\hm\in (0,\,1)$. Подробная мотивировка такого
входного потока обсуждается  в работе~\cite{Taqqu}.  Без ограничения
общности будем считать, что $H_1\hm>\max\limits_{i>1}H_i$. Тогда
 дисперсия $v(t)$  процесса $\{X(t)\}$  имеет вид:
$$
v(t)=\sum\limits_{i=1}^n t^{2H_i}=t^{2H_1}L(t)\,,
$$
где медленно меняющаяся функция $ L(t)\hm=1\hm+\sum\limits_{i>1}
t^{2(H_i-H_1)}\hm\to 1$, $t \hm\to \infty. $ Таким образом, условия
теоремы~2.1 выполнены.

\smallskip

\noindent
\textbf{Пример 2.} Пусть стохастическая компонента входного процесса
есть так называемый интегральный гауссовский процесс, т.\,е.
\begin{equation}
X(t)=\int\limits_0^t Z(s)\,ds\,,
\label{41}
\end{equation}
где $Z$~--- центрированный стационарный гауссовский процесс с
ковариационной функцией $R(u):=$\linebreak 
$:=\;\Cov\left(Z(0),Z(u)\right)$. Входные
потоки такого типа рассматрива\-лись в работах~\cite{Debicki1,Kulkarni}. Нетрудно проверить, что для дисперсии
$v(t)$ процесса~$X$ справедливо соотношение:
\begin{equation}
v(t)=2 \int\limits_0^t\!\! \int\limits_0^s R(u)\,duds\,.\label{42a}
\end{equation}
Отсюда следует, что $v''(t)\hm=2 R(t)$, а значит условие~\eqref{2.2.l20} влечет сходимость
\begin{equation*}
R(t)\ln t \to 0\,,\enskip t \to \infty\,. 
%\label{2.2.l21}
\end{equation*}
Если дополнительно к условию~\eqref{2.2.l20} потребовать
существования таких  $A \hm\in (0,\infty)$, $V \hm\in (0,2)$, что
\begin{equation}
\fr{\int_0^t \int_0^s R(u)\,duds}{t^V} \to A\,,\enskip
t \to \infty\,,
\label{2.2.l22}
\end{equation}
то условия теоремы~2.1 оказываются выполненными. Например,
пусть  $Z$~--- процесс Орн\-штей\-на--Улен\-бе\-ка, который по определению
является центрированным стационарным гауссовским процессом.
Поскольку его  ковариационная функция  имеет вид $R(t)\hm=\lambda
e^{-\alpha t}$, $\lambda,\alpha\hm>0$, то из~(\ref{42a}) несложно получить, что
$$
v(t)=\fr{2 \lambda}{\alpha}t+\fr{2\lambda}{\alpha^2}\left(e^{-\alpha t}-1\right)\,.
$$
Следовательно, условие~\eqref{2.2.l22} выполнено для $V\hm=1$,
$A={\lambda}/{\alpha}$. Отметим, что  если   $Z$~--- процесс
Орн\-штей\-на--Улен\-бе\-ка,  то формула~(\ref{41}) определяет   {\it
интегральный процесс Орн\-штей\-на--Улен\-бе\-ка}.  Заметим, что этот
процесс является гауссовским аналогом (т.\,е.\ гауссовским процессом с
соответствующей корреляционной структурой) модели
Ани\-ка--Мит\-ра--Сон\-ди~\cite{Anick}, описывающей динамику некоторых
сетевых ресурсов (см.\ так\-же~\cite{Addie}).

На самом деле в обоих примерах выше  выполнено   более сильное, чем~\eqref{10a}, 
условие~(\ref{38}).
Однако утверждение теоремы~2.1  верно и в случае, когда функция~$L$ не имеет предела при 
$t\hm\to\infty$.

\section{Дополнительные асимптотические результаты}

В данном разделе получены два важных асимптотических результата для
максимума процесса нагрузки $M(t)$, дополняющие анализ, проведенный
в разд.~2. Вначале рассмотрим случай, когда параметр
$r\hm<0$. В~соответствии с замечанием, сделанным  после формулы~(\ref{6}), 
в этом случае система находится в   нестационарном режиме
и величина процесса нагрузки должна неограниченно воз\-рас\-тать.


\medskip

\noindent
\textbf{Теорема~3.1.}\ \textit{Если $r<0$, то имеет место следующая сходимость 
по распределению:}
\begin{equation}
\fr{M(t)+rt}{\sqrt{v(t)}}  \xrightarrow{d} \Nor
(0,1)\,, \enskip t \to \infty\,.
\label{max-l1}
\end{equation}

%\smallskip

\noindent
Д\,о\,к\,а\,з\,а\,т\,е\,л\,ь\,с\,т\,в\,о\,.\ \ Напомним, что 
процесс нагрузки в момент времени~$s$ определяется соотношением
$$
Q(s)=W(s)-\min\limits_{0\leq u \leq s}W(u)\,,
$$
где $W(u)= X(u)\hm-ru$.
Поскольку с вероятностью~1 (с~в.~1)
$$
\fr{X(t)}{t} \to 0\,,
$$
то также $W(t) \to +\infty$ с~в.~1. Пусть $\Psi:=\min\limits_{t \geq
0}{W(t)}$. Тогда существует случайный момент  $T_0\hm<\infty$  (с~в.~1)
такой, что $\min\limits_{t \geq 0}W(t)\hm=\min\limits_{0 \leq t \leq
T_0}W(t)$. Поэтому справедлива следующая цепочка соотношений:
\begin{multline*}
M(t)=\displaystyle\max\limits_{0 \leq s \leq t}\left[ W(s)-\min\limits_{0 \leq u \leq s}W(u)\right]\leq{}\\
{}\leq\displaystyle \max\limits_{0 \leq s \leq t} W(s)- \Psi={}\\
{}= \displaystyle\max\limits_{0 \leq s \leq t} W(s)- \min\limits_{0 \leq s \leq T_0} W(s)\,.
\end{multline*}
Отсюда следует  неравенство:
\begin{multline}
M(t)-W(t) \leq{}\\
{}\leq  \max\limits_{0 \leq s \leq t} W(s)-W(t)- 
\min\limits_{0 \leq s \leq T_0} W(s)\,. 
\label{max-l4}
\end{multline}
Поскольку
\begin{align*}
M(t)+rt&=M(t)+X(t)-W(t)\,;\\
\fr{X(t)}{\sqrt{v(t)}}&=_d \Nor(0,1)\,,
\end{align*}
то сходимость~(\ref{max-l1}) эквивалентна сходимости
\begin{equation}
\fr{M(t)-W(t)}{\sqrt{v(t)}}  \Rightarrow 0\,, \enskip t \to
\infty\,.
\label{max-l2}
\end{equation}
Докажем справедливость~\eqref{max-l2}.
Для этого достаточно показать, что для любого $\varepsilon\hm>0$ справедливо соотношение:
\begin{equation}
\Pu \left(\fr{M(t)-W(t)}{\sqrt{v(t)}}> \varepsilon  \right) \to 0\,,
\enskip t \to \infty\,. 
\label{max-l3}
\end{equation}
В силу~(\ref{max-l4})
\begin{multline*}
\Pu \left(\fr{M(t)-W(t)}{\sqrt{v(t)}}> \varepsilon  \right) \leq{}\\
{}\leq 
\Pu \left(\fr{\max\limits_{0 \leq s \leq t}W(s)-W(t)}{\sqrt{v(t)}}> 
\fr{\varepsilon}{2}  \right)+{}
\\{} + \Pu \left(\fr{-\min\limits_{0 \leq s \leq T_0}W(s)}{\sqrt{v(t)}}> 
\fr{\varepsilon}{2}  \right):=\Pu_1(t)+\Pu_2(t)\,.
\end{multline*}
 В силу стацонарости приращений процесса $X$
\begin{multline}
\max\limits_{0 \leq s \leq t} W(s) - W(t)= {}\\
{}=
\max\limits_{0 \leq s \leq t} \left[  X(s)- X(t)+r(t-s)\right]=_d\\
{}=_d \max\limits_{0 \leq s \leq t}\left[  X(t-s)+r(t-s)\right]=\\
= \max\limits_{0 \leq u \leq t}\left[
X(u)+ru\right]:=\widetilde{Q}(t)\,.
\label{48}
\end{multline}
Поскольку  $r\hm<0$, то существует стационарный предел
$\widetilde{Q}(t) \xrightarrow{d} \widetilde {Q}$ (при $t \hm\to \infty$),
причем  $ \widetilde{Q}\hm<\infty$ с в.~1 (см.\ замечание после формулы~\eqref{6a}). 
Поскольку $v(t) \hm\to \infty$, то из~(\ref{48}) следует,
что
\begin{multline*}
\Pu_1(t)=\Pu\left( \max\limits_{0 \leq s \leq t}W(s)-W(t)>
\fr{\varepsilon}{2}\sqrt{v(t)}  \right) \to 0\,,\\
 t \to \infty\,.
\end{multline*}
Рассмотрим  вероятность $\Pu_2(t)$ и заметим, что
\begin{multline*}
-\min\limits_{0 \leq s \leq T_0} W(s)=\max\limits_{0 \leq s \leq T_0 }[rs- X(s)]\leq{}\\
{}\leq \max\limits_{0 \leq s \leq T_0}[- X(s)]=_d  \max\limits_{0 \leq s \leq T_0}X(s)\,.
\end{multline*}
Поэтому
\begin{multline*}
\Pu_2(t) = \Pu\left( -\min\limits_{0 \leq s \leq T_0}W(s)>
\fr{\varepsilon}{2}\sqrt{v(t)} \right)\leq{}\\
{}\leq \Pu\left( \max\limits_{0 \leq s \leq T_0}X(s)>
\fr{\varepsilon}{2}\sqrt{v(t)}\right) \to 0\,,\enskip t \to \infty\,,
\end{multline*}
где учитывается, что  случайная величина  $T_0$, а значит и $\max\limits_{0\leq s \leq T_0}X(s)$, 
конечны с~в.~1. Таким образом, соотношение~(\ref{max-l3}), а значит и~(\ref{max-l2}), выполнено.\hfill$\square$

\smallskip

Следующий результат касается асимптотики времени достижения
стационарным процессом нагрузки $Q^*(t)$ растущего порога~$b$, т.\,е.\
асимптотики величины
$$
T(b)=\inf\{t \geq 0:\,Q^*(t)\geq b\}
$$
при $b\to \infty$.  Распределение максимума стационарного процесса
нагрузки $M^*(t)$ определяет распределение случайной величины $T(b)$
в силу  очевидного соотношения
\begin{equation}
\{ T(b) \leq t\}=\{M^*(t) \geq b \}\,,\enskip t\ge 0\,.
\label{time-l0}
\end{equation}
Напомним обозначение $\gamma(t)\hm=\ln t \cdot L((\ln t)^\beta).$

\medskip

\noindent
\textbf{Теорема~3.2.}\ \textit{Пусть в дополнение  к условиям теоремы}~1.1 \textit{функция
$\gamma(t)$ монотонно возрастает на некотором луче $[t_0,\infty)$.
Тогда имеет место сходимость}
\begin{equation}
\fr{\gamma(T(b))}{b^{1/\beta}} \Rightarrow \theta\,,\enskip b \to
\infty\,, 
\label{time-l1}
\end{equation}
\textit{где параметр $\theta$ удовлетворяет соотношению}~(\ref{logbuff-l6}).

\smallskip

\noindent
Д\,о\,к\,а\,з\,а\,т\,е\,л\,ь\,с\,т\,в\,о\,.\ 
 В~силу теоремы~1.1 для любого $\delta \hm>0$
\begin{equation}
\Pu\left( M^*(t)
> \left( \fr{1+\delta}{\theta}\,\gamma(t)\right)^\beta \right) \to 0\,,
\enskip t \to \infty\,.
\label{time-l2}
\end{equation}
Ввиду монотонного возрастания функции~$\gamma$, обратная ей функция
также монотонно возрастает. В~частности, при любом $\delta\hm>0$ функция
\begin{equation}
t(b):=\gamma^{-1}\left( b^{1/\beta} \fr{\theta}{1+\delta} \right) \to \infty\,,\enskip
b \to \infty\,.
\label{time-l3}
\end{equation}
Подставляя~(\ref{time-l3}) в~(\ref{time-l2}) и учитывая~(\ref{time-l0}), получим
\begin{multline}
\Pu\left( M^*(t(b)) \ge \left(
\fr{1+\delta}{\theta}\,\gamma(t(b))\right)^\beta \right)
={}\\
{}=\Pu\left(
M^*(t(b)) \ge  b\right)=\Pu\left( T(b) \le  t(b)\right)={}\\
{}=\Pu\left( T(b)\leq \gamma^{-1}\left( b^{1/\beta}
\fr{\theta}{1+\delta}
\right)\right)={}\\
{}=\Pu\left( \fr{\gamma(T(b))}{b^{1/\beta }}
 \leq \fr{\theta}{1+\delta} \right) \to 0\,,\enskip b \to  \infty\,.
 \label{50}
\end{multline}
Снова используя монотонность функции~$\gamma$, получим (для любого
фиксированного $\delta\hm>0$):
$$
\hat t(b):=\gamma^{-1}\left( b^{1/\beta} \fr{\theta}{1-\delta}
\right) \to \infty\,,\enskip b \to \infty\,.
$$
С учетом того, что по теореме~1.1
$$
\Pu\left( M^*(t) > \left(
\fr{1-\delta}{\theta}\,\gamma(t)\right)^\beta \right) \to 1\,,\enskip 
t \to \infty\,,
$$
как и выше, получим:
\begin{multline*}
\Pu\left( M^*(\hat t(b)) > \left(
\fr{1-\delta}{\theta}\,\gamma(\hat t(b))\right)^\beta
\right)={}\\
{}=\Pu\left( \fr{\gamma(T(b))}{b^{1/\beta }} \leq
\fr{\theta}{1-\delta} \right) \to 1\,,\ b \to \infty\,.
\end{multline*}
Ввиду произвольности~$\delta$ отсюда и из~(\ref{50}) следует~\eqref{time-l1}.\hfill$\square$

\section{Заключение}

В данной статье  продолжен (начатый в работе~\cite{Lukashenko})
асимптотический анализ максимума процесса нагрузки в системе
обслуживания, в которой дисперсия гауссовской компоненты входного
процесса  правильно меняется  на бесконечности с показателем
$V\hm\in(0,\,2)$.
  В~частности, показано, что при некотором дополнительном  условии
 доказанная в~\cite{Lukashenko} сходимость по вероятности
 указанного процесса  имеет место и
в пространстве~$L_p$  при любом $p\hm\in [1,\,\infty)$. 

Также найдена
асимптотика максимума процесса нагрузки в нестационарном режиме (при
соответствующей нормировке). 
Кроме того, с использованием полученной
асимптотики максимума  найдена асимптотика  времени достижения
стационарным процессом нагрузки растущего   порога~$b$.

{\small\frenchspacing
{%\baselineskip=10.8pt
\addcontentsline{toc}{section}{Литература}
\begin{thebibliography}{99}

\bibitem{Lukashenko}
\Au{Лукашенко~О.\,В., Морозов~Е.\,В.} Асимптотика максимума
процесса нагрузки для некоторого класса гауссовских очередей~//
Информатика и её применения, 2012. Т.~6. Вып.~3. С.~81--89.

\bibitem{Zeevi}
\Au{Zeevi~A., Glynn~P.} On the maximum workload in a queue fed
by fractional Brownian motion~// Ann. Appl. Prob., 2000. Vol.~10.
P.~1084--1099.

\bibitem{Mandjes}
\Au{Mandjes~M.} Large deviations of Gaussian queues.~---
Chichester: Wiley, 2007. 339~p.



\bibitem{Reich}
\Au{Reich~E.} On the integrodifferential equation of Takacs~I~// 
Ann. Math. Stat., 1958. Vol.~29. P.~563--570.

\bibitem{Asmus}
\Au{Asmussen S.}  Applied probability and queues.~--- New York: Springer, 2002. 440~p.

\bibitem{Seneta}
\Au{Сенета~Е.} Правильно меняющиеся функции.~--- М.: Наука, 1985.
143~с.

\bibitem{Konstantopoulos}
\Au{Konstantopoulos~T., Zazanis~M., De Veciana~G.}
Conservation laws and reflection mappings with application to
multiclass mean value analysis for stochastic fluid queues~// 
Stochastic Processes and Their Applications, 1996. Vol.~65.
P.~139--146.

\bibitem{Billingsley}
\Au{Биллингсли~П.} Сходимость вероятностных мер.~--- М.: Наука, 1977. 352~с.

\bibitem{Duffy}
\Au{Duffy~K., Lewis~J.~T., Sullivan~W.~G.} Logarithmic
asymptotics for the supremum of a stochastic process~// 
Ann. Appl. Prob., 2003. Vol.~13. No.\,2. P.~430--445.

\bibitem{Adler}
\Au{Adler~R.\,J.} An introduction to continuity, extrema, and
related topics for general Gaussian processes.~--- Hayward, CA: Institute of 
Mathematical Statistics, 1990. 170~p.

\bibitem{Lifshits}
\Au{Лифшиц~М.\,А.} Гауссовские случайные функции.~--- Киев: ТвиМС, 1995. 248~с.


\bibitem{Taqqu}
\Au{Taqqu~M.~S., Willinger~W., Sherman~R.} Proof of a
fundamental result in self-similar traffic modeling~// Computer
Comm. Rev., 1997. Vol.~27. P.~5--23.


\bibitem{Kulkarni}
\Au{Kulkarni~V., Rolski~T.} Fluid model driven by an
Ornstein--Uhlenbeck process~// Probability  Engineering 
Informational Sci., 1994. Vol.~8. P.~403--417.

\bibitem{Debicki1}
\Au{Debicki~K., Rolski~T.} A Gaussian fluid model~// Queueing
Syst., 1995. Vol.~20. P.~433--452.

\bibitem{Anick}
\Au{Anick~D., Mitra~D., Sondhi~M.~M.} Stochastic theory of a
data handling system with multiple resources~// Bell Syst.
Techn.~J., 1982. Vol.~61. P.~1871--1894.

\label{end\stat}

\bibitem{Addie}
\Au{Addie~R., Mannersalo~P., Norros~I.} Most probable paths and
performance formulae for buffers with Gaussian input traffic~//
Eur. Trans. Telecommunications, 2002. Vol.~13.
P.~183--196.
\end{thebibliography}
}
}

\end{multicols}      %4Abst+avt
\def\stat{rudoi}

\def\tit{АЛГОРИТМЫ ИНДУКТИВНОГО ПОРОЖДЕНИЯ СУПЕРПОЗИЦИЙ ДЛЯ~АППРОКСИМАЦИИ ИЗМЕРЯЕМЫХ ДАННЫХ$^*$}

\def\titkol{Алгоритмы индуктивного порождения суперпозиций для аппроксимации измеряемых данных}

\def\autkol{Г.\,И.~Рудой, В.\,В.~Стрижов}

\def\aut{Г.\,И.~Рудой$^1$, В.\,В.~Стрижов$^2$}

\titel{\tit}{\aut}{\autkol}{\titkol}

{\renewcommand{\thefootnote}{\fnsymbol{footnote}}\footnotetext[1]
{Работа выполнена при поддержке РФФИ, грант №\,12-07-13118.}}

\renewcommand{\thefootnote}{\arabic{footnote}}
\footnotetext[1]{Московский физико-технический институт, rudoy@forecsys.ru}
\footnotetext[2]{Вычислительный центр Российской академии наук 
им.~А.\,А. Дородницына, strijov@ccas.ru}

\vspace*{-3pt}

\Abst{Исследуется алгоритм индуктивного порождения допустимых существенно
  нелинейных моделей. Предлагается алгоритм, порождающий все возможные
  суперпозиции заданной сложности за конечное число шагов. Приводятся
  результаты вычислительного
  эксперимента по выбору оптимальной модели, аппроксимирующей синтетический
  набор данных.}
  
  \vspace*{-1pt}

\KW{символьная регрессия; нелинейные модели; индуктивное порождение;
    сложность моделей}
    
    \vspace*{-3pt}
    
    \vskip 12pt plus 9pt minus 6pt

      \thispagestyle{headings}

      \begin{multicols}{2}

            \label{st\stat}

\section{Введение}

В~ряде приложений~\cite{duffy:1999:srised, Barmpalexis201175}
возникает задача восстановления некоторой функциональной зависимости
по набору известных данных. При этом предполагается, что эксперт
должен иметь возможность проинтерпретировать полученную модель
в контексте предметной области.

Одним из методов, позволяющих получать интерпретируемые модели, является
символьная регрессия~[3--7],
согласно которой известные данные приближаются некоторой математической
формулой, например $\sin x^2 \hm+ 2x $ или $\log x \hm- {e^x}/{x} $.
Эти формулы являются произвольными суперпозициями функций из некоторого
заданного набора. Одна из возможных реализаций описываемого метода
предложена Джоном Коза~\cite{Koza1998GP, Koza1998Intro}, использовавшим
эволюционные алгоритмы для реализации символьной регрессии. Зелинка с соавторами
предложили дальнейшее развитие этой идеи~\cite{Zelinka2008}, получившее
название аналитического программирования.

Алгоритм построения требуемой математической модели в аналитическом
программировании выглядит следующим образом:
дан набор примитивных функций, из которых можно строить различные формулы
(например, степенная функция, $+$, $\sin$, $\tan$). Начальный набор формул
строится либо произвольным образом, либо на базе некоторых предположений
эксперта. Затем на каждом шаге производится оценка каждой из формул согласно
функции ошибки либо другого функционала качества~\cite{Tirsin2005}. На базе
этой оценки у некоторой части формул случайным образом заменяется одна
элементарная функция на другую (например, $\sin$ на $\cos$ или $+$ на
$\times$), а у некоторой другой части происходит взаимный попарный обмен
подвыражениями.

Получаемая формула является математической моделью исследуемого
процесса или явления, т.\,е.\ представляет собой математическое
отношение, описывающее основные закономерности, присущие этому
явлению~\cite{Pavlovsky2000}.

Алгоритм индуктивного порождения моделей, предложенный в~настоящей работе,
свободен от некоторых типичных проблем известных методов, упомянутых,
например, в~\cite{Zelinka2008}. Вот главные из них:
\begin{itemize}
  \item порождение рекурсивных суперпозиций, суперпозиций, содержащих
    несоответствующее используемым функциям число аргументов, и~т.\,д.
    (в~предложенном алгоритме эти проб\-ле\-мы не возникают по построению);
  \item несовпадение области определения некоторой примитивной функции и области
    значений ее аргументов (возможно, тоже некоторых суперпозиций);
  \item порождение слишком сложных суперпозиций.
\end{itemize}

Для любой выборки можно построить такой многочлен, который пройдет через
все точки выборки, но при этом число параметров такого многочлена линейно
растет с объемом выборки. Кроме того, такой многочлен неинтерпретируем
экспертами. Предложенный в настоящей работе алгоритм решает проблему
порождения слишком сложных суперпозиций введением дополнительного штрафа
за сложность. Кроме того, так как ис\-поль\-зу\-емые признаки объектов выборки
учитываются при расчете сложности, применение подобного штрафа обеспечивает
выбор суперпозиций, использующих меньшее число признаков, т.\,е.\ проводит
отбор признаков.

Во~второй части работы формально поставлена задача построения алгоритма
индуктивного по\-рож\-де\-ния моделей. Затем, в~третьей час\-ти, строится искомый
алгоритм для частного случая беспараметрических моделей и~доказывается его
корректность, а затем алгоритм обобщается на случай моделей, имеющих параметры.
В~четвертой час\-ти оценивается количество порожденных предложенным алгоритмом
моделей на каждой итерации. В~пятой час\-ти предлагается метод выбора
допустимых моделей из множества всех порожденных моделей. В~седьмой час\-ти
описывается адаптированный стохастический алгоритм порождения моделей,
результаты работы которого на синтетических данных приведены в~восьмой
части настоящей работы.

\section{Постановка задачи}

Пусть дана выборка
\begin{multline*}
D = \left\{ (\mathbf{x}_i, y_i) \mid i \in \{1, \dots, N\},\right.\\
           \left. \mathbf{x}_i \in \mathbb{X} \subset \mathbb{R}^n,
            y_i \in \mathbb{Y} \subset \mathbb{R} \right\},
\end{multline*}
где $N$~--- число элементов выборки, $\mathbf{x}_i$~--- вектор значений
свободных переменных для $i$-го элемента выборки, $y_i$~--- значение зависимой
переменной для $i$-го элемента выборки,
$\mathbb{X}$~--- множество значений независимых переменных, лежащее в
$\mathbb{R}^n$, $\mathbb{Y}$~--- множество значений зависимой переменной.

Требуется выбрать параметрическую функцию
$f : \Omega \times \mathbb{X} \rightarrow \mathbb{R}$ из
порождаемого множества $\mathcal{F} \hm= \{ f_r \}$, где $\Omega$~--- пространство
параметров, до\-став\-ля\-ющую минимум некоторому заданному функционалу качества~$Q$,
зависящему от функционала ошибки~$S$ на данной выборке~$D$ и сложности суперпозиции~$C(f)$.

Таким образом, для множества всех суперпозиций
$$
\mathcal{F} = \{ f_r \mid
            f_r : (\boldsymbol{\omega}, \mathbf{x}) \mapsto y \in \mathbb{Y},
            r \in \mathbb{N} \}
$$
требуется найти такой индекс $\hat{r}$, при котором функция $f_r$ среди всех
$f \hm\in \mathcal{F}$ доставляет минимум функционалу качества~$Q$ при данной
выборке~$D$:
\begin{equation*}
  \label{eq:hat_r}
  \hat{r} = \arg \min\limits_{r \in \mathbb{N}} Q (f_r \mid \boldsymbol{\hat{\omega}_r}, D)\,,
\end{equation*}
где $\boldsymbol{\hat{\omega}}_r$~--- оптимальный вектор параметров функции
$f_r$ для каждой $f \hm\in \mathcal{F}$ при данной выборке~$D$:
\begin{equation*}
%  \label{eq:hat_omega}
  \boldsymbol{\hat{\omega}_r} = 
  \arg \min\limits_{\boldsymbol{\omega} \in \Omega} S(\boldsymbol{\omega} \mid f_r, D)\,.
\end{equation*}

Сформулируем также постановку теоретической задачи. Для этого сначала
введем понятие суперпозиции функций.

Если множество значений $\mathbb{Y}_i$ функции $f_i$ содержится в области
определения $\mathbb{X}_{i+1}$ функции $f_{i+1}$, т.\,е.\
$$
f_i : \mathbb{X}_i \to \mathbb{Y}_i \subset \mathbb{X}_{i+1}\,,\enskip i = 1, 2, \dots, \theta - 1\,,
$$
то функция
$$
f_\theta \circ f_{\theta-1} \circ \dots \circ f_1\,, \enskip \theta \geq 2\,,
$$
определяемая равенством
$$
(f_\theta \circ f_{\theta-1} \circ \dots \circ f_1) (\mathbf{x}) =
  f_{\theta} (f_{\theta-1} (\cdots (f_1 (\mathbf{x}))))\,, 
  \ x \in \mathbb{X}_1,
$$
называется \textit{сложной функцией}~\cite{MathEnc1984_4} или
\textit{суперпозицией функций} $f_1, f_2, \dots, f_\theta$.

Таким образом, получаем

\smallskip

\noindent
\textbf{Определение~1.}
\textit{Суперпозиция функций~--- функция, представленная как композиция нескольких
  функций.}
  
  \smallskip


Пусть $G = \{ g_1, \dots, g_l \}$~--- множество данных порождающих
функций, а именно: для каждой $g_i \hm\in G$ заданы
\begin{itemize}
  \item сама функция $g_i$ (например, $\sin$, $\cos$, $\times$);
  \item арность функции и~порядок следования аргументов;
  \item домен ($\text{dom}\, g_i$) и кодомен ($\text{cod}\, g_i$) функции;
  \item область определения $\mathcal{D} g_i \subset \text{dom}\, g_i$ и~область
    значений $\mathcal{E} g_i \subset \text{cod}\, g_i$.
\end{itemize}
Требуется построить упомянутую функцию~$f$ как суперпозицию порождающих
функций из заданного множества~$G$.

Поясним различие между последними двумя пунктами. Например, $\text{dom}\, f$
показывает, значения из какого множества принимает функция~$f$ (целые чис\-ла,
действительные чис\-ла, декартово произведение целых чи\-сел и $\{0, 1\}$,
и~т.\,п.). Область определения же показывает, на каких значениях из
$\text{dom}\, f$ функция $f$ определена и имеет смысл. Так, для функции
$f(x_1, x_2) \hm= \log_{x_1} x_2$:
\begin{gather*}
  \text{dom}\, f = \mathbb{R} \times \mathbb{R}\,,\quad 
  \text{cod}\, f = \mathbb{R}\,;
\\
  \mathcal{D} f = \left\{ (x_1, x_2) \vert x_1 \in (0; 1) \cup (1; +\infty), x_2 \in (0; +\infty) \right\};
\\
  \mathcal{E} f = (-\infty; +\infty)\,.
\end{gather*}

Требуется также:
\begin{itemize}
  \item построить алгоритм $\mathfrak{A}$, за конечное число итераций
    порождающий любую конечную суперпозицию данных примитивных функций;
  \item указать способ проверки изоморфности двух суперпозиций.
\end{itemize}

Заметим, что для примитивных функций  не требуются свойства их непорождаемости
в наиболее общей формулировке типа принципиальной невозможности породить
в ходе работы искомого алгоритма суперпозицию, изоморфную некоторой функции из~$G$. 
Такое требование является слишком ограничивающим. В~частности, невозможно
было бы иметь в~$G$ одновременно, например, функции $\text{id}$, $\exp$
и~$\log$, так как $\text{id}\,\equiv \log \circ \exp$.

В~дальнейшем будем также считать, что суперпозиция, соответствующая
единственной свободной переменной ($f(\mathbf{x}) \hm= x_i$), эквивалентна
функции вида $\text{id}\,x_i$.

\section{Алгоритм индуктивного порождения допустимых суперпозиций}

Условимся считать, что каждой суперпозиции~$f$ сопоставлено дерево~$\Gamma_f$,
эквивалентное этой суперпозиции и строящееся следующим образом:
\begin{itemize}
  \item в~вершинах $V_i$ дерева~$\Gamma_f$ находятся соответствующие
    порождающие функции $g_s, s \hm= s(i)$;
  \item число дочерних вершин у некоторой вершины~$V_i$ равно арности
    соответствующей функции~$g_s$;
  \item порядок смежных некоторой вершине~$V_i$ вершин соответствует порядку
    аргументов соответствующей функции~$g_{s(i)}$;
  \item в~листьях дерева~$\Gamma_f$ находятся свободные переменные~$x_i$
    либо числовые па\-ра\-мет\-ры~$\omega_i$;
  \item порядок вершин~$V_i$ в~смысле уровня вершин определяет порядок
    вычисления примитивных функций: дерево вычисляется снизу вверх,
    т.\,е.\ сначала подставляются конкретные значения свободных переменных,
    затем вычисляются значения в~вершинах, все дочерние вершины которых~---
    свободные переменные, и так далее до тех пор, пока не останется
    единственная вершина, бывшая корнем дерева. Она и содержит результат
    соответствующего выражения.
\end{itemize}

Таким образом, вычисление значения вы\-ра\-жения~$f$ в некоторой точке с данным
вектором\linebreak параметров $\boldsymbol{\omega} \hm= \{ \omega_1, \omega_2, \dots, \omega_\eta\}$
эквивалентно подстановке соответствующих значений свободных перемен\-ных~$x_i$
и параметров $\omega_i$ в дерево~$\Gamma_f$, где $x_i$ --- компоненты
вектора признакового описания объекта~$\mathbf{x}$.

Заметим важное свойство таких деревьев: каж\-дое поддерево $\Gamma_f^i$
дерева $\Gamma_f$, соответствующее вершине~$V_i$, также соответствует
некоторой суперпози-\linebreak\vspace*{-12pt}
\begin{center}  %fig1
\vspace*{1pt}
\mbox{%
 \epsfxsize=48.539mm
 \epsfbox{rud-1.eps}
 }
% \end{center}

 \vspace*{6pt}
{{\figurename~1}\ \ \small{Дерево выражения $\sin (\ln x_1) + {x_2^3}/{2}$}}
\end{center}


%\pagebreak

\vspace*{12pt}

\addtocounter{figure}{1}

\noindent
ции, являющейся составляющей исходной суперпозиции~$f$.

Для примера рассмотрим дерево, со\-от\-вет\-ст\-ву\-ющее суперпозиции $f \hm= \sin (\ln x_1) + 
{x_2^3}/{2}$ (рис.~1).
Здесь точками обозначены аргументы функций. Как видно, корнем дерева является
вершина, соответствующая операции сложения, которая должна быть выполнена
в последнюю очередь. Операция сложения имеет два различных поддерева,
соответствующих двум аргументам этой операции. Заметим также, что здесь не
использованы операции типа <<разделить на два>> или <<возвести в~куб>>.
Вместо этого используются операции деления и возведения в степень в~общем
виде, а в данном конкретном дереве соответствующие аргументы зафиксированы
соответствующими константами.

\smallskip


\noindent
\textbf{Алгоритм порождения суперпозиций.} Сначала определим понятие
\textit{глубины суперпозиции}:

\smallskip

\noindent
\textbf{Определение~2.}
\textit{Глубина суперпозиции $f$~--- максимальная глубина дерева $\Gamma_f$.}

\smallskip

Теперь опишем итеративный алгоритм $\mathfrak{A^*}$, порождающий суперпозиции,
не содержащие параметров. Описанный алгоритм породит любую суперпозицию
конечной глубины за конечное число шагов.

Пусть дано множество примитивных функций $G \hm= \{ g_1, \dots, g_l \}$ и
множество свободных переменных $X \hm= \{ x_1, \dots, x_n \}$. Для удобства будем
исходить из предположения, что множество $G$ состоит только из унарных
и~бинарных функций, и~разделим его соответствующим образом на два подмножества:
$G \hm= G_b \cup G_u \mid G_b \hm= \{ g_{b_1}, \dots, g_{b_k} \}$, 
$G_u \hm= \{ g_{u_1}, \dots, g_{u_l} \}$,
где $G_b$~--- множество всех бинарных функций, а $G_u$~--- множество всех
унарных функций из~$G$. Потребуем также наличия $\text{id}$ в~$G_b$.

\smallskip

\noindent
\textbf{Алгоритм~1.}
  Алгоритм $\mathfrak{A^*}$ итеративного порождения суперпозиций.
\begin{enumerate}[1.]
  \item Перед первым шагом зададим начальные значения множества
    $\mathcal{F}_0$ и вспомогательного индексного множества~$\mathcal{I}$,
    служащего для запоминания, на какой итерации впервые встречена
    каждая суперпозиция:
    \begin{equation*}
      \mathcal{F}_0 = X\,;\quad
      \mathcal{I} = \left\{ (x, 0) \mid x \in X \right\}\,.
\end{equation*}
  \item Для множества $\mathcal{F}_i$ построим вспомогательное множество~$U_i$,
    состоящее из суперпозиций, полученных в результате применения функций
    $g_u \hm\in G_u$ к элементам~$\mathcal{F}_i$:
    $$
      U_i = \left\{ g_u \circ f \mid g_u \in G_u, f \in \mathcal{F}_i \right\}\,.
$$
  \item Аналогичным образом построим вспомогательное множество~$B_i$ для
    бинарных функций $g_b \hm\in G_b$:
    $$
      B_i = \left\{ g_b \circ (f, h) \mid g_b \in G_b, f, h \in \mathcal{F}_i \right\}\,.
$$
  \item Обозначим $\mathcal{F}_{i+1} = \mathcal{F}_i \cup U_i \cup B_i$.
  \item Для каждой суперпозиции~$f$ из~$\mathcal{F}_{i+1}$ добавим пару
    $(f, i+1)$ в~множество $\mathcal{I}_f$, если суперпозиция~$f$ еще там
    не присутствует.
  \item Перейдем к~следующей итерации, п.~2.
\end{enumerate}

Тогда $\mathcal{F} \hm= \cup_{i=0}^\infty \mathcal{F}_i$~--- множество всех
возможных суперпозиций конечной длины, которые можно построить из
данного множества примитивных функций.

Вспомогательное множество~$\mathcal{I}$ позволяет запоминать, на какой
итерации была впервые встречена каждая суперпозиция. Это необходимо, так
как каждая суперпозиция, впервые порожденная на $i$-й итерации, будет
порождена так\-же и на любой итерации после~$i$. Одной из возможностей
избежать необходимости в этом множестве является построение
$\mathcal{F}_{i+1}$ как $\mathcal{F}_{i+1} \hm= U_i \cup B_i$ (без
$\mathcal{F}_i$), а множества~$U_i$ и~$B_i$ строить следующим образом:
\begin{align*}
  U_i &= \left\{ g_u \circ f \mid g_u \in G_u, f \in \cup_{j=0}^{i} \mathcal{F}_j \right\}\,;
\\
  B_i &= \left\{ g_b \circ (f, h) \mid g_b \in G_b, f, h \in \cup_{j=0}^{i} \mathcal{F}_j \right\}\,.
\end{align*}

Алгоритм $\mathfrak{A^*}$ очевидным образом обобщается на случай, когда
множество~$G$ содержит функции произвольной (но конечной) арности.
Действительно, для такого обобщения достаточно строить аналогичным образом
вспомогательные множества для этих функций, а именно: для множества функций~$G_n$ 
арности~$n$ построить вспомогательное множество $H_i^n$ вида
$$
H_i^n = \left\{ g \circ (f_1, f_2, \dots, f_n) \mid g \in G_n, f_j \in \mathcal{F}_i \right\}\,.
$$

В~этих обозначениях $U_i \hm\equiv H_i^1$, а $B_i \hm\equiv H_i^2$.

Тогда множество $\mathcal{F}_{i+1} \hm= \mathcal{F}_i \cup_{n=0}^{n_{\max}} H_i^n$,
где $n_{\max}$~--- максимальное значение арности функций из~$G$.

\smallskip

\noindent
\textbf{Теорема~1.}
\textit{Алгоритм $\mathfrak{A^*}$ действительно породит любую конечную суперпозицию
  за конечное число шагов.}

\smallskip

\noindent
Д\,\,о\,к\,а\,з\,а\,т\,е\,л\,ь\,с\,т\,в\,о\,.\ \ 
  Чтобы убедиться в~этом, \mbox{найдем} номер итерации, на которой будет по\-рож\-де\-на
  некоторая произвольная конечная суперпозиция~$f$. Чтобы найти этот номер,
  пронумеруем вершины графа~$\Gamma_f$ по следующим правилам:
  \begin{itemize}
    \item если это вершина со свободной переменной, то она имеет номер~0;
    \item если вершина $V$ соответствует унарной функции, то она имеет номер
      $i+1$, где $i$~--- номер дочерней для этой функции вершины;
    \item если вершина $V$ соответствует бинарной функции, то она имеет номер
      $i+1$, где $i = \max (l, r)$, а $l$ и $r$~--- номера соответственно
      первой и второй дочерней вершины.
  \end{itemize}

  Нумеруя вершины графа $\Gamma_f$ таким образом, можно получить номер вершины,
  соответ\-ст\-ву\-ющей корню графа. Это и будет номером итерации, на которой получена
  суперпозиция~$f$.

  Иными словами, для любой суперпозиции можно указать конкретный номер
  итерации, на которой она будет получена, что и~требовалось.


\smallskip

В~предложенных ранее методах построения суперпозиций~\cite{Zelinka2008}
необходимо было самостоятельно следить за тем, чтобы в~ходе работы алгоритма
не возникало <<зацикленных>> суперпозиций типа $f(x, y) \hm= g (f(x, y), x, y)$.
Заметим, что в предложенном алгоритме $\mathfrak{A^*}$ такие суперпозиции
не могут возникнуть по построению.

\smallskip

\noindent
\textbf{Порождение моделей с параметрами.}
Алгоритм $\mathfrak{A^*}$, описанный выше, не позволяет получать выражения, содержащие численные
параметры $\boldsymbol{\omega}$ суперпозиции $f(\boldsymbol{\omega}, \mathbf{x})$.
Покажем, однако, на примере конструирования множеств $U_i$ и~$B_i$, как
исходный алгоритм $\mathfrak{A^*}$ может быть расширен путем введения параметров
\begin{align*}
U_i &= g_u \circ (\alpha f + \beta) \,;
\\
B_i &=  g_b \circ (\alpha f + \beta, \psi h + \phi) \,.
\end{align*}
Будем обозначать этот расширенный алгоритм как~$\mathfrak{A}$. Здесь
параметры $\alpha$, $\beta$ зависят только от комбинации $g_u, f$ (или
$g_b, f, h$ для $\alpha$, $\beta$, $\psi$, $\phi$). Соответственно, для
упрощения их индексы опущены. Иными словами, предполагается, что
каждая суперпозиция, полученная на предыдущих итерациях, входит
в порождаемую на следующей итерации, будучи умноженной на некоторый
коэффициент и с константной поправкой.

Очевидно, при таком добавлении параметров $\alpha$, $\beta$, $\psi$,
$\phi$ не происходит изменения мощности получившегося множества
суперпозиций, поэтому алгоритм и~выводы из него остаются
корректными. В~частности, исходный алгоритм является частным случаем
данного при $\alpha \hm\equiv \psi \hm\equiv 1$, $\beta \hm\equiv \phi \hm\equiv 0$.

Переменные $\alpha, \beta, \psi, \phi$ являются параметрами модели. 
В~практических приложениях можно оптимизировать значения этих параметров у
получившихся суперпозиций, например, алгоритмом 
Ле\-вен\-бер\-га--Марк\-вард\-та~\cite{Marquardt1963Algorithm, more:78}.

Заметим также, что такая модификация алгоритма позволяет получить единицу,
например, для построения суперпозиций типа ${1}/{x}$:
$1 \hm= \alpha\ id\ x + \beta \mid \alpha = 0, \beta \hm= 1$.

Отдельно подчеркнем, что параметры $\boldsymbol{\omega}$ у разных
суперпозиций различны. Однако, так как каж\-дый из па\-ра\-мет\-ров зависит только
от со\-от\-вет\-ст\-ву\-ющей комбинации функций, к которым он относит\-ся, конкретные
значения параметров не учитываются при поиске одинаковых суперпозиций.
Иными словами, при тестировании суперпозиций на равенство сравниваются лишь
структуры соответствующих им деревьев и значения в узлах, соответствующих
функциям и свободным переменным.

Заметим, что и~этот алгоритм очевидным образом обобщается на случай
множества~$G$, содержащего функции произвольной арности.

\section{Число возможных суперпозиций}

Оценим число суперпозиций, получаемых после каждой итерации
алгоритма~$\mathfrak{A}$. Очевидно, с учетом вышеупомянутых оговорок
касательно сравнения параметризованных суперпозиций, это число равно
аналогичному числу для алгоритма~$\mathfrak{A^*}$.

Итак, пусть дано $n$ независимых переменных: $| X | \hm= n$, а мощность
множества~$G$ распишем через мощности его подмножеств функций соответствующей
арности: $| G_1 | \hm= l_1$, $| G_2 | \hm= l_2, \dots$, $| G_p | \hm= l_p$. На нулевой
итерации имеем $P_0 \hm= n$ суперпозиций.

На первой итерации дополнительно порож\-да\-ется

\noindent
$$
P_1 = l_1 n + l_2 n^2 + \dots + l_n n^p = \sum\limits_{i=1}^p l_i P_0^i\,,
$$
и суммарное число суперпозиций после первой итерации
$$
\hat{P}_1 = P_1 + P_0 = \sum\limits_{i=1}^p l_i P_0^i + P_0\,.
$$

Как было замечено ранее, суперпозиции, по\-рож\-ден\-ные на $k$-й итерации, будут
также порождены и на любой следующей после $k$ итерации, поэтому суммарное
число суперпозиций после второй итерации будет равно
$$
\hat{P}_2 = \sum\limits_{i=1}^p l_i \hat{P}_1^i\,.
$$

И вообще, после $k$-й итерации будет порождено
$$
\hat{P}_k = \sum\limits_{i=1}^p l_i \hat{P}_{k-1}^i\,.
$$

Оценим порядок роста количества функций, порожденных после $k$-й итерации.

\smallskip

\noindent
\textbf{Теорема~2.}
\textit{Пусть в множестве примитивных функций $G$ содержится $l_p$ функций арности
  $p \hm> 1$ и ни одной функции арности $p \hm+ k \mid k \hm> 0$ и имеется $n \hm> 1$
  независимых переменных. Тогда справедлива следующая оценка чис\-ла
  суперпозиций, порожденных алгоритмом $\mathfrak{A}$ после $k$-й итерации:}
  $$
  \left\vert \mathcal{F}_k \right\vert = \mathcal{O} 
  \left(l_p^{\sum_{i=0}^{k-1} p^i} n^{p^k}\right)\,.
  $$


\smallskip

\noindent
Д\,\,о\,к\,а\,з\,а\,т\,е\,л\,ь\,с\,т\,в\,о\,.\ \ 
  Оценим сначала порядок рос\-та для случая, когда есть лишь одна $m$-ар\-ная
  функция и $n$ свободных переменных.

  После первой итерации алгоритма будет по\-рож\-де\-но $n^m \hm+ n$ суперпозиций.
  После второй~--- $(n^m + n)^m \hm+ n^m + n$, что можно оценить как\linebreak
  $(n^m)^m \hm= n^{m^2}$. И~вообще, после $k$-й итерации чис\-ло
  суперпозиций можно оценить как $n^{m^k}$.

  Видно, что для оценки скорости роста количества порожденных суперпозиций
  можно учитывать только функции с наибольшей арностью.

  Рассмотрим теперь случай, когда имеется не одна функция арности~$m$, а
  $l_m$ таких функций. Тогда на первой итерации порождается $l_m n^m \hm+ n$
  суперпозиций, на второй:
  $$
  l_m (l_m n^m + n)^m + l_m n^m + n \approx l_m^{m+1} n^{m^2}\,,
$$
  на третьей, с учетом этого приближения:
  $$
  l_m (l_m^{m+1} n^{m^2})^m = l_m l_m^{m(m+1)} n^{m^3} = l_m^{m^2 + m + 1} n^{m^3}\,.
$$
  И~вообще, скорость роста количества порожденных суперпозиций можно оценить
  как:
  $$
  \left\vert \mathcal{F}_k \right\vert = \mathcal{O} 
  \left(l_m^{\sum_{i=0}^{k-1} m^i} n^{m^k}\right)\,.
  $$
  Таким образом, получаем оценку в общем случае, когда в множестве $G$ содержится
  $l_p$ функций ар\-ности~$p$ и ни одной функции ар\-ности $p \hm+ k \mid k\hm > 0$:
  $$
\left\vert \mathcal{F}_k \right\vert = \mathcal{O} 
\left(l_p^{\sum_{i=0}^{k-1} p^i} n^{p^k}\right)\,.
  $$


%\smallskip

\section{Множество допустимых суперпозиций}

Предложенный выше алгоритм позволяет получить действительно все возможные
суперпозиции, однако не все они будут пригодны в~практических приложениях:
например, $\ln x$ имеет смысл только при $x \hm> 0$, а ${x}/{0}$ не имеет
смысла вообще никогда. Выражения типа ${x}/{\sin x}$ имеют смысл только
при $x \hm\neq \pi k$.

Таким образом, необходимо введение понятия множества \textit{допустимых}
суперпозиций, т.\,е.\ таких суперпозиций, которые в условиях данной
задачи корректны.

\smallskip

\noindent
\textbf{Определение~3.}
\textit{Допустимая суперпозиция $f$~--- такая суперпозиция, значение которой
  определено для любой комбинации значений свободных переменных, область
  значений $\mathbb{X}$ которых определяется конкретной задачей,
  $\mathbb{X} \subset \mathbb{R}^n$, где $n$~--- число свободных переменных.}


\smallskip

Одним из способов построения только допустимых суперпозиций является
модификация предложенного алгоритма таким образом, чтобы отслеживать
совместность областей определения и \mbox{областей} значений соответствующих
функций в ходе построения суперпозиций. Для свободных переменных это,
в свою очередь, означает необходимость задания областей значений
$\mathbb{X}$ пользователем при решении конкретных задач.

Таким образом, можно сформулировать очевидное \textit{достаточное условие
недопустимости} суперпозиции:

\smallskip

\noindent
\textbf{Определение~4.}
  Достаточное условие недопустимости суперпозиции~$f$: в соответствующем дереве
  $\Gamma_f$ хотя бы одна вершина~$V_i$ имеет хотя бы одну дочернюю вершину~$V_j$ 
  такую, что область значений функции $g_{s(j)}$ шире, чем область
  определения функции $g_{s(i)}$:
  $$
  \exists i, j : V_i \in \Gamma_f, V_j \in \Gamma_f \wedge \exists \kappa :
    \kappa \in \mathcal{E} g_{s(j)} \wedge \kappa \notin \mathcal{D} g_{s(i)}\,.
$$


\smallskip

Говоря, что область значений функции~$f$ шире области определения функции~$g$, 
имеем в~виду, что существует, по крайней мере, одно значение функции~$f$, 
не входящее в область определения функции~$g$.

Подчеркнем, что, хотя свободные переменные могут принимать, например, все
значения из~$\mathbb{R}$, выбором множества~$\mathbb{X}$ можно обеспечить
возможность использования их в качестве аргументов функций с более узкой,
чем $\mathbb{R}$, но не менее узкой, чем $\mathbb{X}$, областью определения,
если это не противоречит данной выборке.

Для построения множества допустимых суперпозиций достаточно построить
множество всех возможных суперпозиций при помощи алгоритма~$\mathfrak{A}$,
а затем удалить из этого множества все суперпозиции, не удовлетворяющие
сформулированному признаку.

\section{Алгоритм итеративного стохастического порождения суперпозиций}

Несмотря на то что построенный ранее итеративный алгоритм~$\mathfrak{A}$ по\-рож\-де\-ния
суперпозиций позволяет получить за конечное число шагов произвольную
суперпозицию, для практических применений он непригоден в~связи с чрезмерной
вычислительной сложностью, как и~любой алгоритм, реализующий полный перебор.
Вместо него предлагается использовать стохастические алгоритмы и~ряд эвристик,
позволяющих на практике получать за приемлемое время результаты,
удовлетворяющие заранее заданным условиям. В~данном разделе описывается
практически реализуемый вариант алгоритма~$\mathfrak{A}$, который и был использован
в~вычислительном эксперименте. Опишем вспомогательный алгоритм 
случайного порождения суперпозиции.

\smallskip

\noindent
\textbf{Алгоритм~2.} 
  Алгоритм случайного порождения суперпозиции $\mathcal{RF}$.

  Вход:
  \begin{itemize}
    \item набор пороговых значений $0 < \xi_1 < \xi_2 \hm< \xi_3 \hm< 1$;
    \item максимальная глубина порождаемой суперпозиции Td.
  \end{itemize}


\smallskip

Алгоритм работает следующим образом. Генерируется случайное чис\-ло~$\xi$ на
интервале $(0; 1)$ и рассматриваются следующие случаи:
\begin{itemize}
  \item $\xi \leq \xi_1$: результатом алгоритма является некоторая случайно
    выбранная свободная переменная;
  \item $\xi_1 < \xi \leq \xi_2$: результатом алгоритма является    числовой
    параметр;
  \item $\xi_2 < \xi \leq \xi_3$: результатом алгоритма является некоторая
    случайно выбранная унарная функция, для определения аргумента которой
    данный алгоритм рекурсивно запускается еще раз;
  \item $\xi_3 < \xi$: результатом алгоритма является некоторая случайно
    выбранная бинарная функция, аргументы которой порождаются аналогичным
    образом.
\end{itemize}
При этом порождение тривиальных суперпозиций (свободных переменных и
параметров) запрещено: на самом первом шаге пороговые значения масштабируются
таким образом, чтобы всегда порождалась унарная или бинарная функция.
Аналогично при превышении значения~Td пороговые значения масштабируются
таким образом, чтобы был порожден узел, соответствующий свободной переменной
или параметру, и алгоритм за\-вер\-шился.

Каждой порожденной суперпозиции~$f$ ставится в
соответствие ее \textit{качество}~$Q_f$, рассчитываемое исходя из функционала ошибки~$S_f$ 
этой суперпозиции на выборке~$D$ и ее сложности $C_f$~---
числа узлов в соответствующем графе~$\Gamma_f$. Функционал~$Q_f$ выбирается эвристически
с учетом следующих естественных соображений:
\begin{itemize}
  \item из двух суперпозиций одинаковой слож\-ности~$C_f$ выбирается обеспечивающая
    более оптимальное значение функционала ошибки~$S_f$;
  \item из двух суперпозиций, имеющих одно и то же значение функционала ошибки~$S_f$,
    выбирается суперпозиция, обладающая меньшей слож\-ностью~$C_f$.
\end{itemize}

\noindent
\textbf{Алгоритм~3.}
  Итеративный алгоритм стохастического порождения суперпозиций.

  Вход:
  \begin{itemize}
    \item множество порождающих функций~$G$, со\-сто\-ящее только из унарных
      и бинарных функций;
    \item регрессионная выборка~$D$;
    \item $N_{\max}$~--- максимальное число одновременно рассматриваемых
      суперпозиций;
    \item $I_{\max}$~--- максимальное число итераций алгоритма;
    \item $\hat{Q}$~--- минимальное значение функционала~$Q_f$:
    \begin{equation}
  \label{eq:q_f}
  Q_f = \fr{1}{1 + S_f} \left(\alpha + \fr{1 - \alpha}
  {1 + \exp \left({C_f}/{\beta} - \tau\right)}\right)\,,
\end{equation}
где $\alpha$~--- некоторый коэффициент влияния штрафа за сложность, $0 \hm\ll \alpha \hm< 1$,
$\beta$~--- коэффициент строгости штрафа за сложность, $\beta \hm> 0$, а
$\tau$~--- коэффициент, характеризующий желаемую сложность модели;
    \item $\gamma_{\mathrm{mut}}$~--- доля суперпозиций, подверженных случайной
      замене узлов их деревьев;
    \item $\gamma_{\mathrm{cross}}$~--- доля суперпозиций, для которых выполняется
      случайный обмен поддеревьями;
    \item прочие параметры, используемые в~\eqref{eq:q_f} и алгоритме~2.
  \end{itemize}


\noindent
\begin{enumerate}
  \item Инициализируется упорядоченный набор $\mathcal{X}_f$ суперпозиций,
    а~именно: порождается $N_{\max}$ суперпозиций алгоритмом~2.
  \item Оптимизируются параметры~$\boldsymbol{\omega}$ суперпозиций
    из~$\mathcal{X}_f$ алгоритмом Ле\-вен\-бер\-га--Марк\-вард\-та.
  \item Выполняются простейшие преобразования, упрощающие суперпозицию:
    например, выражения вида $0 \cdot x$ заменяются на~0.
  \item Вычисляется значение~$Q_f$ для каждой еще не оцененной суперпозиции~$f$ 
  из~$\mathcal{X}_f$: для нее рассчитывается значение функционала ошибки~$S_f$ 
  на выборке~$D$ и ставится в соответствие значение~$Q_f$. Для
    суперпозиций, при вычислении~$Q_f$ которых была хотя бы раз получена
    ошибка вычислений из-за несовпадения областей определений и значений,
    принимается $Q_f \hm= -\infty$.
  \item Набор суперпозиций~$\mathcal{X}_f$ сортируется согласно значениям
    функционала~$Q_f$.
  \item Суперпозиции с наименьшими значениями~$Q_f$ удаляются из массива~$\mathcal{X}_f$ 
  до тех пор, пока его размер не станет равен~$N_{\max}$.
  \item Отбирается некоторая часть~$\gamma_{\mathrm{mut}}$ суперпозиций с наименьшими
    значениями~$Q_f$ из~$\mathcal{X}_f$. У~этой час\-ти происходит случайная замена
    одной функции или свободной переменной на другую: генерируются две случайные величины,
    одна из которых служит для выбора вершины дерева~$\Gamma_f$, которую
    предстоит изменить, а другая~--- для выбора нового элемента для этой вершины.
    Замена такова, что сохраняется структура суперпозиции, т.\,е.\
    в случае замены функции сохраняется арность, а свободная переменная
    заменяется только на другую свободную переменную. Исходные
    суперпозиции сохраняются в массиве~$\mathcal{X}_f$.
  \item Повторяются шаги 4--5.
  
  \begin{figure*} %fig2
\vspace*{1pt}
 \begin{center}
 \mbox{%
 \epsfxsize=112.519mm
 \epsfbox{rud-2.eps}
 }
 \end{center}
 \vspace*{-9pt}
  \Caption{Поверхности функции $Q_f$ для некоторых $\beta$ (\textit{1}~--- $\beta\hm=0{,}1$; 
  \textit{2}~--- 1; \textit{3}~--- $\beta\hm=5$)
    и фиксированного $\tau = 5$}
  \label{fig:fitness_surph}
\end{figure*}

 
  
  
  \item Производится случайный обмен поддеревьями у $\gamma_{\mathrm{cross}}$ суперпозиций
    с наибольшими значениями $Q_f$. Вершины, соответствующие этим поддеревьям,
    выбираются случайным образом. При этом исходные суперпозиции сохраняются
    в~$\mathcal{X}_f$.
    

Таким образом, чем лучше результаты суперпозиции и чем она проще, тем ближе
значение функционала~$Q_f$ к~$1$.


  \item Повторяются шаги 2--5.
  \item Проверяются условия останова: если либо чис\-ло итераций превышает
    $I_{\max}$, либо в~массиве $\mathcal{X}_f$ находится суперпозиция со значением~$Q_f$, 
    большим $\hat{Q}$, то алгоритм останавливается
    и результатом считается суперпозиция с наибольшим значением~$Q_f$, иначе
    осуществляется переход к шагу~2.
\end{enumerate}

\section{Вычислительный эксперимент}

В~вычислительном эксперименте вос\-ста\-нав\-ли\-ва\-ет\-ся функциональная зависимость
$y \hm= 2 \cosh \sqrt{(x_1^2\hm + x_2^2)/2}$, соответствующая фигуре вращения
цепной линии. При этом значения зависимой
переменной~$y$ были искусственно зашумлены аддитивной добавкой из
распределения $\mathcal{N} (0, 0{,}1)$ и соответствующая ей переменная
присутствовала во множестве используемых свободных переменных.

В качестве функционала ошибки~$S$ используется сумма квадратов
регрессионных остатков для данной суперпозиции~$f$ с вектором параметров
$\boldsymbol{\omega}$ при регрессионной выборке~$D$:
\begin{equation}
  \label{eq:sse_expr}
  S(\boldsymbol{\omega}, f, D) = \sum\limits_{i=1}^N (y_i - f (\boldsymbol{\omega}, 
  \mathbf{x}_i))^2\,.
\end{equation}

Значение функционала ошибки~$S$ при подстановке исходной незашумленной
функциональной зависимости составляет $\approx 4{,}29$, сложность исходной
суперпозиции~--- 14.

\begin{table*}\small
\begin{center}
\Caption{Результаты вычислительного эксперимента для предложенного алгоритма}
  \label{tabl:results}
  \vspace*{2ex}

\begin{tabular}{| c | c | l | c | c | c |} 
  \hline
    $N$ & $i$   & \multicolumn{1}{c|}{Суперпозиция}  & $S_f$                & $C_f$ & $Q_f$             \\ 
    \hline
    &&&&&\\[-9pt]
    1   & 13    & $ 1{,}0002 \left(2{,}72^{\sqrt{x \cdot x + y \cdot y}/2} + 
    2{,}56^{\sqrt{x \cdot x + y \cdot y}/{-1{,}93}}\right)$ & 
    $\approx 4{,}10$     & 29    & $\approx 0{,}010$    \\ 
    \hline
    &&&&&\\[-9pt]
    2   & \hphantom{9}9     & $ 2{,}001 \cosh \fr{\sqrt{x \cdot x + y \cdot y}}{1{,}999}$& 
    $\approx 4{,}25$   & 14    & $\approx 0{,}188$   \\ 
    \hline
  \end{tabular}
  \end{center}
\end{table*}

\begin{table*}\small
\begin{center}
  \Caption{Результаты вычислительного эксперимента для алгоритма~\cite{Zelinka2008}}
  \label{tabl:results_Z}
  \vspace{2ex}
  
  \begin{tabular}{| c | c | c | c | c |} 
  \hline
    $i$ & Суперпозиция  & $S_f$                & $C_f$ & $Q_f$             \\ 
    \hline
    &&&&\\[-9pt]
    29  & $ 2{,}66^{\sqrt{x^2 + y^2}/{2{,}23}} - 
    \fr{x^2 + y^2}{3{,}03} + \fr{x^2 \cdot x^2 + y^2\cdot  y^2}{6{,}3} + 0{,}93$
             & $\approx 6{,}2$     & 43    & $ \approx 0{,}007 $ \\ 
             \hline
  \end{tabular}
  \end{center}
\end{table*}

\begin{figure*} %fig3
\vspace*{1pt}
 \begin{center}
 \mbox{%
 \epsfxsize=160.163mm
 \epsfbox{rud-3.eps}
 }
 \end{center}
 \vspace*{-9pt}
\Caption{Первая порожденная суперпозиция~(\textit{1})
и зашумленные точки выборки~(\textit{2})~(\textit{a}) 
и исходная зависимость~(\textit{3})~(\textit{б})}
\end{figure*}
\begin{figure*} %fig4
\vspace*{-3pt}
 \begin{center}
 \mbox{%
 \epsfxsize=160.163mm
 \epsfbox{rud-4.eps}
 }
 \end{center}
 \vspace*{-11pt}
  \Caption{Вторая порожденная суперпозиция~(\textit{1}) и зашумленные точки выборки~(\textit{2})~(\textit{a}) 
  и исходная зависимость~(\textit{3})~(\textit{б})}
\end{figure*}

В данной работе используется функционал~$Q_f$ вида~(\ref{eq:q_f}).
Значения параметров~$\alpha$, $\beta$ и~$\tau$
выбираются экспертно исходя из предположений
о виде искомой суперпозиции и моделируемом явлении.

Второй множитель в~\eqref{eq:q_f} выполняет роль штрафа за слишком
большую сложность суперпозиции, что позволяет выбирать более простые модели,
избегая эффекта переобучения и экстремальных случаев вроде порождения
интерполяционных полиномов. На рис.~\ref{fig:fitness_surph}
приведены поверхности~$Q_f$ для различных значений~$\beta$ при фиксированном
$\tau \hm= 5$.



Использованные параметры алгоритма~3: $N_{\max} \hm= 200, I_{\max} \hm= 50,
\hat{Q} \hm= 0{,}95, \tau \hm= 20, \alpha \hm= 0{,}05$, $\beta \hm= 1$, 
$\gamma_{\mathrm{mut}} \hm= {1}/{3}$,
$\gamma_{\mathrm{cross}} \hm= {1}/{3}$. При отсутствии улучшения результатов в~течение
нескольких итераций подряд алгоритм~3 также завершался.

Результаты вычислительного эксперимента приведены в табл.~\ref{tabl:results}.
Указан номер итерации~$i$, на которой суперпозиция была впервые
получена, сама суперпозиция, среднеквадратичная ошибка~\eqref{eq:sse_expr} и сложность в
смысле числа узлов в соответствующем графе выражения. Числовые коэффициенты
в приведенных формулах и значения функционала~$S_f$ искусственно округлены до
нескольких значащих цифр.

Алгоритм запускался для двух разных наборов элементарных функций.
В обоих случаях элементарные функции включали 
в себя стандартные арифметические операции и операцию возведения в степень. 
Для удобства возведение в степень~${1}/{2}$ (и близкие ей) заменено в таблице 
на операцию извлечения корня.


В первом случае в наборе отсутствовала функция~$\cosh$. При этом по результатам
10 запусков наилучшей суперпозицией, полученной предложенным алгоритмом,
оказалась функция за номером~1 из табл.~\ref{tabl:results}. Видно, что выражение
в скобках близко определению $\cosh x = ({e^x + e^{-x}})/2$,
однако разные значения оснований степенных функций могут затруднить экспертный
анализ полученного выражения, которое само по себе является достаточно громоздким.

Во втором случае набор элементарных функций также включал в себя функцию~$\cosh$, 
результату этого выражения соответствует суперпозиция за номером~2.
Включение $\cosh$ в~$G$ позволило существенно быстрее подобрать искомую функцию, и сложность
получившейся суперпозиции также существенно меньше.

Кроме того, предложенный алгоритм сравнивался с алгоритмом~\cite{Zelinka2008},
в котором суперпозиции кодировались бинарной строкой и применялись стандартные
генетические алгоритмы на получавшихся строках; во множестве используемых
функций также отсутствовала функция~$\cosh$.



Наилучшая суперпозиция, полученная алгоритмом~\cite{Zelinka2008} по результатам
10~запусков, приведена в табл.~\ref{tabl:results_Z}. Полученная суперпозиция
имеет существенно более высокую сложность, чем суперпозиции, перечисленные в
табл.~\ref{tabl:results}.

На рис.~3 отображены изометрические
проекции первой из приведенных в табл.~\ref{tabl:results} суперпозиций. На
рис.~3\textit{а} данная суперпозиция сравнивается с точками синтезированной
зашумленной выборки, на рис.~3\textit{б}  она же приведена вместе
с исходной незашумленной зависимостью. Аналогичные проекции
приведены для второй суперпозиции на рис.~4.

\vspace*{-6pt}

\section{Заключение}

В~работе исследованы индуктивные алгоритмы порождения допустимых существенно
нелинейных суперпозиций. Предложен переборный алгоритм, порождающий все
возможные суперпозиции заданной сложности за конечное число шагов.
Сформулированный алгоритм решает некоторые типичные проблемы предложенных ранее методов.
Описан стохастический алгоритм индуктивного порождения существенно нелинейных
суперпозиций и приведены результаты вычислительного эксперимента на синтетических
данных. Описанный алгоритм выбирает менее точные, но более простые
модели, что позволяет избежать переобучения и выполнить простейший отбор признаков.

\vspace*{-6pt}

{\small\frenchspacing
{%\baselineskip=10.8pt
\addcontentsline{toc}{section}{Литература}
\begin{thebibliography}{99}

\bibitem{duffy:1999:srised} 
\Au{Duffy~J., Engle-Warnick~J.} Using symbolic regression to infer strategies 
from experimental data~// Evolutionary\linebreak\vspace*{-12pt}\columnbreak

\noindent
 Computation in Economics and Finance, 2002. 
Vol.~100.  P.~61--84.

\bibitem{Barmpalexis201175} %2
\Au{Barmpalexis~P., Kachrimanis~K., Tsakonas~A., Georgarakis~E.} 
Symbolic regression via genetic programming in the optimization of 
a controlled release pharmaceutical formulation~// Chemometrics and 
Intelligent Laboratory Systems, 2011. Vol.~107. No.~1. P.~75--82.

\bibitem {davidson:2000:snrea} %3 
\Au{Davidson J.\,W., Savic D.\,A., Walters G.\,A.} Symbolic and numerical 
regression: Experiments and applications~// Developments in Soft Computing, 2001. 
Vol.~6. P.~175--182.

\bibitem {strijov07poisk} %4
\Au{Стрижов В.\,В.} 
Поиск параметрической регрессионной модели в индуктивно заданном множестве~// 
Вычислительные технологии, 2007. T.~1. C.~93--102.

\bibitem {Strijov08InductMethods}  %5
\Au{Стрижов В.\,В.} Методы индуктивного порождения регрессионных моделей.~--- М.:~ВЦ~РАН, 2008.


\bibitem {reference/ml/X10vc} %6
\Au{Sammut C., Webb  G.\,I.} Symbolic regression~// Encyclopedia of Machine Learning.~---  
Berlin: Springer, 2010.


\bibitem {StrijovW10}  %7
\Au{Strijov V.\,V., Weber G.\,W.} Nonlinear regression model generation using 
hyperparameter optimization~// Computers and Mathematics with Applications, 2010. Vol.~60. No.\,4. P.~981--988.

\bibitem {Koza1998GP} 
\Au{Koza J.\,R.} 
Genetic programming~// Encyclopedia of Computer Science and Technology, 1998. Vol.~39. No.\,24. P~29--43.

\bibitem {Koza1998Intro} 
\Au{Koza J.\,R.} Introduction to genetic algorithms.~--- Cambridge: MIT Press, 1998.

\bibitem {Zelinka2008} 
\Au{Zelinka I., Oplatkova Z., Nolle L.} Analytic programming and symbolic 
regression by means of arbitrary evolutionary algorithms~// Int.\ 
J.~Simulation Syst. Sci. Technol., 2005. Vol.~6. No.\,9. P~44--56.

\bibitem {Tirsin2005} 
\Au{Тырсин А.\,Н.} 
Об эквивалентности знакового и наименьших модулей методов построения линейных моделей~// 
Обозрение прикладной и промышленной математики, 2005. Т.~12. №\,4. C.~879--880.

\bibitem {Pavlovsky2000} \Au{Павловский Ю.\,Н.} Имитационные 
модели и системы.~--- М.:~Фазис, 2000.

\bibitem {MathEnc1984_4} \Au{Битюцков В.\,И., Войцеховский М.\,И., Иванов А.\,Б.} 
Математическая энциклопедия. Т.~4.~--- М.:~Советская энциклопедия, 1984.

\bibitem {Marquardt1963Algorithm} 
\Au{Marquardt D.\,W.} An algorithm for least squares estimation of nonlinear parameters~// 
J.~Soc. Ind. Appl. Math., 1963. Vol.~11. No.\,2. P.~431--441.

\label{end\stat}


\bibitem {more:78} \Au{More J.\,J.} 
The Levenberg--Marquardt algorithm: Implementation and theory~// 
Lecture Notes in Mathematics 630: Numerical Analysis.~--- Berlin: Springer-Verlag,
1978. P.~105--116.
\end{thebibliography}
}
}

\end{multicols}       %5Abst+avt
\def\stat{grusho}

\def\tit{АРХИТЕКТУРНЫЕ РЕШЕНИЯ В~ЗАДАЧЕ ВЫЯВЛЕНИЯ МОШЕННИЧЕСТВА ПРИ~АНАЛИЗЕ 
ИНФОРМАЦИОННЫХ ПОТОКОВ В~ЦИФРОВОЙ ЭКОНОМИКЕ$^*$}

\def\titkol{Архитектурные решения в~задаче выявления мошенничества при~анализе 
информационных потоков в
%~цифровой 
экономике}

\def\aut{А.\,А.~Грушо$^1$, М.\,И.~Забежайло$^2$, Н.\,А.~Грушо$^3$, 
Е.\,Е.~Тимонина$^4$}

\def\autkol{А.\,А.~Грушо, М.\,И.~Забежайло, Н.\,А.~Грушо, 
Е.\,Е.~Тимонина}

\titel{\tit}{\aut}{\autkol}{\titkol}

\index{Грушо А.\,А.}
\index{Забежайло М.\,И.}
\index{Грушо Н.\,А.}
\index{Тимонина Е.\,Е.}
\index{Grusho A.\,A.}
\index{Zabezhailo M.\,I.}
\index{Grusho N.\,A.}
\index{Timonina E.\,E.}


{\renewcommand{\thefootnote}{\fnsymbol{footnote}} \footnotetext[1]
{Работа частично поддержана РФФИ (проекты 18-29-03081 и~18-07-00274).}}


\renewcommand{\thefootnote}{\arabic{footnote}}
\footnotetext[1]{Институт проблем информатики Федерального исследовательского центра <<Информатика и~управление>> 
Российской академии наук, grusho@yandex.ru}
\footnotetext[2]{Институт проблем информатики Федерального исследовательского центра <<Информатика и~управление>> 
Российской академии наук, m.zabezhailo@yandex.ru}
\footnotetext[3]{Институт проблем информатики Федерального исследовательского центра <<Информатика и~управление>> 
Российской академии наук, info@itake.ru}
\footnotetext[4]{Институт проблем информатики Федерального исследовательского центра <<Информатика и~управление>> 
Российской академии наук, eltimon@yandex.ru}

\vspace*{-12pt}
   

 
  
  \Abst{Cформулирован подход к~исследованию некоторых видов мошенничества в~цифровой 
экономике с~использованием причинно-следственных связей. Во всех видах рассматриваемых 
мошенничеств должно наблюдаться несоответствие между целями финансовых транзакций 
и~реальной стоимостью достижения этих целей. Данные о транзакциях можно собирать, 
наблюдая информационные потоки, в~которых отражаются эти транзакции. Архитектура сбора 
данных и~их анализа может быть организована с~помощью распределенных реестров 
с~централизованным консенсусом, что позволяет создать аналог электронной бухгалтерской 
книги, фиксирующей финансово-экономическую деятельность субъектов цифровой экономики в~регионе. 
  Рассматриваемые методы выявления мошенничества основаны на противоречиях 
между действиями, описанными в~транзакциях, и~информацией, содержащейся в~планах, 
стандартах, прецедентах и~др. Рассмотрен метод, основанный на некоторой упрощенной схеме 
реализации абстрактного проекта. Для выявления противоречий необходимо проводить анализ 
от следствия к~причине, т.\,е.\ искать аномалии в~информации, описывающей порождение 
наблюдаемых следствий. 
  Показано, как в~реализации проекта можно выделять простые <<необходимые условия>> 
нарушения при\-чин\-но-след\-ст\-вен\-ных связей, т.\,е.\ множество <<необходимых условий>>, 
нарушение которых свидетельствует о наличии мошенничества. Это множество <<необходимых 
условий>> можно назвать метаданными для контроля проекта на выявление мошенничества.} 
 
 
  \KW{цифровая экономика; информационные потоки; при\-чин\-но-след\-ст\-вен\-ные связи; 
выявление мошеннических схем} 

\DOI{10.14357/19922264190204}
  
\vspace*{-4pt}


\vskip 10pt plus 9pt minus 6pt

\thispagestyle{headings}

\begin{multicols}{2}

\label{st\stat}

\section{Введение}

\vspace*{3pt}

  В работе сформулирован подход к~исследованию некоторых видов 
мошенничества в~цифровой экономике с~использованием  
при\-чин\-но-след\-ст\-вен\-ных связей. Рассматриваются три вида мошенничества, 
а именно:
  \begin{enumerate}[(1)]
\item отмыв денег; 
\item обман при выполнении договорных обязательств при реализации 
технических проектов (строительные проекты и~др.); 
\item незаконный вывод денег. 
\end{enumerate}

  Названные виды мошенничества могут быть сведены к~решению одного типа 
задач. Для отмывания денег источник должен заключать фиктивные контракты, 
в~соответствии с~которыми будут переводиться средства за заведомо ненужную 
работу и~материалы. 
  
  Мошенничество, связанное с~невыполнением договорных обязательств, связано 
со снижением качества услуг, качества и~количества закупаемых 
материалов, выполнением работ с~ненадлежащим качеством. 
  
  Вывод денег связан с~переводом средств фир\-мам-од\-но\-днев\-кам, которые 
заведомо не могут выполнить обязательства по контрактам, за которые им 
переводятся средства. 
  
  Таким образом, во всех трех видах рассматриваемых мошенничеств должно 
наблюдаться несоответствие между целями финансовых транзакций и~реальной 
стоимостью достижения этих целей. Данные о транзакциях можно собирать, 
наблюдая информационные потоки, в~которых отражаются эти транзакции. 
  
  Однако для наблюдения таких информационных потоков необходимо создавать 
архитектуру\linebreak телекоммуникационной системы, позволяющей перехватывать 
и~собирать данные о всех транзакциях. Например, такая архитектура может быть 
организована с~помощью распределенных реестров с~централизованным 
консенсусом, т.\,е.\ все информационные потоки, сформированные в~цифровой 
экономике и~несущие информацию о транзакциях, проходят через некоторый 
центральный узел, запоминающий их в~форме распределенного реестра. Такие 
реестры могут дублироваться в~аналогичных центрах различных регионов, что 
позволяет создать аналог электронной бухгалтерской книги, фиксирующей 
фи\-нан\-со\-во-эко\-но\-ми\-че\-скую деятельность субъектов цифровой экономики. Такой 
подход предложено реализовать на базе системы ситуационных центров, что 
отражено в~работах~[1, 2].
  
  Собранная из информационных потоков информация о~транзакциях, т.\,е.\ 
о~контрактах, договорах, платежах, отчетах, закупленных материалах, 
характеристиках исполнителей работ и~др., собирается в~базе данных в~указанном 
центре. Согласно теории интеллектуальных сис\-тем~[3], эту базу данных можно 
называть базой фактов (БФ). Базу фактов можно представить как бинарную мат\-ри\-цу, 
строки которой описывают характеристики, входящие в~транзакции, а столбцы 
нумеруются характеристиками. Строки матрицы будем называть 
\textit{объектами}~[4, 5]. 
  
  Рассматриваемые в~работе методы выявления мошенничества будут основаны 
на противоречиях между действиями, описанными в~транзакциях, и~информацией, 
содержащейся в~планах, стандартах, прецедентах и~др. Для нахождения 
противоречий в~архитектуре центра предусмотрена другая база данных~--- база 
знаний (БЗ)~\cite{3-gr, 6-gr}, которая устроена так же, как БФ. 
  
  Информация в~БЗ собирается на основе положительного опыта или расчетов. 
Используя БЗ, можно выводить факты нарушения при\-чин\-но-след\-ст\-вен\-ных 
связей. Нарушения при\-чин\-но-след\-ст\-вен\-ных связей будем называть 
\textit{аномалиями}. 
  
  Для упрощения дальнейшее изложение будет вестись в~рамках поиска 
противоречий при выполнении некоторого абстрактного проекта. Выявление 
аномалий будет происходить на основе фактов из БФ с~помощью знаний из БЗ 
методами искусственного интеллекта и~интеллектуального анализа 
данных~\cite{6-gr}. 

\vspace*{-10pt}
  
  \section{Модели}
  
  \vspace*{-3pt}
  
  Наиболее сложная из рассмотренных выше задач~--- выявление противоречий, 
т.\,е.\ использование БЗ для получения новых знаний и~выявление аномалий из 
полученных фактов. 
  
  Все способы выявления противоречий основаны на определении 
  причинно-следственных связей. При этом противоречия в~параметрах транзакций по 
отношению к~требуемым в~БЗ составляют сущность аномалий. 
  
   Далее будет рассмотрен метод, основанный на некоторой упрощенной схеме 
реализации абстрактного проекта. 
  
  Каждый проект имеет цель: например, цель представляет собой построение 
некоторой системы. Воспользуемся структурным подходом, который позволяет 
строить проект на основе разбиения системы на подсистемы и~определения 
взаимодействий подсистем~\cite{7-gr}. При этом каждая подсистема также 
представима структурной моделью. 
  
  Как сама система, так и~каждая ее подсистема имеют свой функционал 
и~спецификацию, па\-ра\-мет\-ры настройки и~домены параметров настройки. Кроме 
этих характеристик существует множество характеристик, связанных 
с~<<жизненным циклом>> создания системы. Сюда входят работы, ресурсы, 
сроки выполнения работ по созданию подсистем и~самой системы, стоимости 
компонентов и~материалов, стоимости работ, схемы поставок, договорные 
обязательства и~др. Все характеристики связаны между собой, поэтому можно 
говорить о стоимости и~времени изготовления структурных компонентов системы. 
  
  Одной из важнейших характеристик является смета (система смет для 
подсистем). Смета сопоставляет каждому компоненту системы стоимость его 
изготовления и~настройки. 
  
  Схема построения системы может быть пред\-став\-ле\-на диаграммой, 
изображенной на рис.~1. 

{ \begin{center}  %fig1
 \vspace*{9pt}
   \mbox{%
 \epsfxsize=79mm 
 \epsfbox{gru-1.eps}
 }


\vspace*{9pt}


\noindent
{{\figurename~1}\ \ \small{Диаграмма достижения цели}}
\end{center}
}

\vspace*{9pt}

\addtocounter{figure}{1}
  
  


  Представленная на рис.~1 диаграмма позволяет описать основные классы 
возможных противоречий при достижении цели. Противоречия возникают, когда 
данные БФ не соответствуют требуемым характеристикам. 
  
  
  \section{Потенциальные классы аномалий при~достижении цели}
  
  Выделим четыре потенциальных класса противоречий, которые показывают, 
каким образом нужно искать эти противоречия.
  
 
  Противоречие цели и~проекта (рис.~2) возникает при отсутствии обоснования 
или в~случае логического противоречия между возможностями проектируемого 
функционала и~целью системы. Отметим, что в~проект входят сроки, перечень 
работ, материалы, настройки, которые описываются соответствующими 
параметрами и~допустимыми значениями этих параметров. Проект формируется 
на основе БЗ и~расчетов, исходя из информации, полученной по аналогии 
с~другими проектами и~решениями, которые считаются апробированными. 
  
  Отметим, что цель порождает проект и~в этом смысле является причиной 
проекта. Однако для анализа противоречий необходимо двигаться по штриховой 
стрелке диаграммы (см.\ рис.~2) от проекта к~цели. В~самом деле, любой компонент 
проекта направлен на теоретическое достижение цели. Цель~--- сложный объект, 
поэтому в~проекте могут возникнуть характеристики, противоречащие хотя бы 
некоторым характеристикам цели. Это делает проект противоречивым, но вывод 
об этом может быть сделан только на уровне описания цели. 
  

  Противоречия между проектом и~его реализацией, исключая настройки 
(рис.~3), могут возникать, например, при закупке исполнителем материалов более 
низкого качества по более низким ценам, при попытках достижения требуемых 
сроков работы за счет снижения качества выполнения работ, за счет нахождения 
<<объективных>> причин для увеличения сроков работы и,~следовательно, 
увеличения цены реализации проекта. 


  Для выявления указанных противоречий необходимо двигаться по диаграмме 
(см.\ рис.~3) в~обратную сторону в~соответствии со~штриховыми стрелками. 
Действительно, выявить противоречия между характеристиками закупленных 
материалов и~требуемыми по проекту можно только при обращении к~проекту 
и~его спецификациям. Манипуляции со сроками работы также можно выявить 
только при обращении к~соответствующим расчетам в~проекте. Задержки в~сроках 
работы, связанные с~поставками материалов, можно определить только на 
предыдущем этапе диаграммы (см.\ рис.~3) в~описании проекта. 


  


  Противоречия между реализацией проекта и~его настройкой (рис.~4) возникает, 
когда не удается добиться требуемых значений параметров функционала, не 
удается обеспечить необходимый уровень\linebreak\vspace*{-12pt}

{ \begin{center}  %fig2
 \vspace*{-6pt}
   \mbox{%
 \epsfxsize=16mm 
 \epsfbox{gru-2.eps}
 }


\vspace*{6pt}


\noindent
{{\figurename~2}\ \ \small{Противоречия цели и~проекта}}
\end{center}
}

%\vspace*{9pt}

\addtocounter{figure}{1}

{ \begin{center}  %fig3
 \vspace*{6pt}
    \mbox{%
 \epsfxsize=79mm 
 \epsfbox{gru-3.eps}
 }


\end{center}

\vspace*{-2pt}


\noindent
{{\figurename~3}\ \ \small{Противоречия проекта и~его реализации (без настройки)}}
}

\vspace*{6pt}

\addtocounter{figure}{1}

{ \begin{center}  %fig4
 \vspace*{1pt}
   \mbox{%
 \epsfxsize=54.5mm 
 \epsfbox{gru-4.eps}
 }


\end{center}


\noindent
{{\figurename~4}\ \ \small{Противоречия реализации проекта и~его на\-стройки}}
}

%\vspace*{9pt}

\addtocounter{figure}{1}

{ \begin{center}  %fig5
 \vspace*{5pt}
    \mbox{%
 \epsfxsize=79mm 
 \epsfbox{gru-5.eps}
 }


\end{center}



\noindent
{{\figurename~5}\ \ \small{Противоречия цели и~достигнутой реализации проекта}}
}

\vspace*{6pt}

\addtocounter{figure}{1}

\noindent
 качества реализации проекта. Для 
определения противоречия в~настройках надо опять же двигаться по диаграмме 
(см.\ рис.~4) в~обратную сторону по штриховым стрелкам, так как для выявления 
характеристик результатов работы, которые не дают возможности реализации 
определенного функционала, необходимо иметь информацию о результатах этой 
работы. 


  



  Противоречие между целью и~достигнутой реализацией проекта (рис.~5) 
возникает, когда реализованная система не позволяет достичь цели. В~этом случае 
опять противоречие нужно искать, двигаясь от цели к~реальному достигнутому 
функционалу по штриховой стрелке (см.\ рис.~5).
  
  Суммируя положения, изложенные в~данном разделе, приходим к~выводу, что 
для выявления противоречий необходимо проводить анализ от следствия 
к~причине, т.\,е.\ искать аномалии в~информации, описывающей порождение 
наблюдаемых следствий. 
  
  
  \section{Связь противоречий и~причин}
  
  Прежде чем построить связь между причинами и~противоречиями, кратко 
опишем простейшую модель связи этих понятий. Причины и~противоречия будут 
сформулированы для представления компонентов системы как объектов, 
обладающих наборами известных характеристик~\cite{4-gr, 5-gr}. 
  
  Пусть $U\hm=\{\alpha, \beta, \ldots\}$~--- совокупность характеристик 
(пространство характеристик). Согласно~\cite{4-gr} \textit{объектом}~$O$ 
называется любое подмножество характеристик $O\hm\subseteq U$. Рассмотрим 
последовательность объектов, возможно в~различных пространствах 
характеристик. 
  
  \smallskip
  
  \noindent
  \textbf{Определение~1.}\ Объект~$P$ с~числом характеристик, большим или 
равным~2, является \textit{причиной} объекта (\textit{свойства})~$B$ в~цепочке 
наблюдаемых объектов тогда и~только тогда, когда выполнены следующие 
условия:
  \begin{enumerate}[(1)]
\item для каждого объекта~$C$, если $P\hm\subseteq C$, то $C\mapsto B$, где 
$C\mapsto B$ означает, что объект~$B$ присутствует в~объекте, следующем за 
объектом~$C$;
\item объект~$P$ является минимальным объектом, удовлетворяющим 
условию~1, а~именно: $\forall \alpha\hm\in P$ объект~$P\backslash \{\alpha\}$ 
не является причиной, т.\,е.\ $\exists C:\ \alpha\not\in C$, $P\backslash 
\{\alpha\}\hm\subseteq C$ и~$C\not\mapsto B$, где $C\not\mapsto B$ означает, 
что~$B$ не может содержаться в~объекте, следующем за объектом~$C$. 
\end{enumerate}

  Приведенное определение причины является упрощением причин, 
возникающих в~реальном мире. Например, реальные причины могут возникать\linebreak 
как совокупность характеристик из разных пространств. Одно следствие может 
порождаться разными причинами или возникать из внешних\linebreak и~ненаблюдаемых 
характеристик. Однако пред\-став\-лен\-ная далее формализация позволяет доступно 
изложить при\-чин\-но-след\-ст\-вен\-ные истоки противоречий, которые 
инициируют в~дальнейшем глубокое исследование рассматриваемых процессов.
  
  Будем считать, что для любого интересующего нас свойства~$B$ существует 
причина. Тогда справедлива следующая теорема.
  
  \smallskip
  
  \noindent
  \textbf{Теорема~1.}\ \textit{Для любого свойства~$B$ существует 
единственная причина}. 
  
  \smallskip
  
  \noindent
  Д\,о\,к\,а\,з\,а\,т\,е\,л\,ь\,с\,т\,в\,о\,.\ \ Доказательство будем вести от противного, 
т.\,е.\ предположим, что существуют две причины свойства~$B$: $P$ 
и~$P^\prime$, $P\hm\not= P^\prime$. Тогда существует $\alpha\hm\in U$, которое 
удовлетворяет одному из двух условий:
  \begin{itemize}
\item[(а)] $\alpha\in P$, $\alpha\notin P^\prime$;
\item[(б)] $\alpha\notin P$, $\alpha \in P^\prime$.
\end{itemize}

  Пусть выполняется условие~(б). Тогда $P^\prime\backslash \{\alpha\}$ не 
является причиной по условию~2 определения~1, т.\,е.\ $\exists C$ такое, что 
$\alpha\notin C$, $P^\prime\backslash \{\alpha\}\hm\subseteq C$ и~$C\not\mapsto B$. 
Но если~$B$ произошло и~$P$ его причина, то $C\mapsto B$, что противоречит 
предположению. Теорема~1 доказана.
  
  \smallskip
  
  \noindent
  \textbf{Лемма.} \textit{Если $P$~--- причина появления свойства~$B$, то 
объект~$B$ определяет существование свойства~$P$ в~объекте, 
предшествующем~$B$. }
  
  \smallskip
  
  \noindent
  Д\,о\,к\,а\,з\,а\,т\,е\,л\,ь\,с\,т\,в\,о\,.\ \ Из предположения, что у~каж\-до\-го 
свойства~$B$ есть причина, и~условия, что~$P$ является причиной~$B$, следует, 
что при появлении в~данных свойства~$B$ объект~$C$, предшествующий 
появлению~$B$, содержит как часть объект~$P$. Это следует из теоремы~1 
и~определения причины. 
  
  Докажем принцип <<необходимого условия>>, который, несмотря на простоту 
доказательства, будет играть в~дальнейшем существенную роль.
  
  \smallskip
  
  \noindent
  \textbf{Теорема~2.} \textit{Если~$P$~--- причина появления свойства~$B$ 
и~$A\hm\subseteq P$, то объект~$B$ определяет наличие свойства~$A$ 
в~объекте, предшествующем~$B$}. 
  
  \smallskip
  
  \noindent
  Д\,о\,к\,а\,з\,а\,т\,е\,л\,ь\,с\,т\,в\,о\,.\ \ Пусть в~данных имеется объект~$B$ 
и~$P\mapsto B$, тогда в~силу существования и~единственности причины~$B$ 
в~данных должен существовать объект~$C$, предшествующий~$B$ 
и~содержащий причину~$P$. Поскольку $A\hm\subseteq P$ и~$B$ содержит 
причину~$P$, то $B\mapsto A$. С~учетом леммы теорема~2 доказана.
  
  \smallskip
  
  Пусть даны пространства $U_1, U_2,\ldots$ и~имеется последовательность 
данных (процесс выполнения этапов проекта в~соответствии с~рис.~1) $A, B, 
\ldots$, где каждый объект является подмножеством некоторого 
пространства~$U_i$, $i\hm=1,\ldots$ Тогда в~объекте~$A$ присутствует 
причина~$P$ появления интересующего нас свойства~$C$ в~объекте~$B$. Пусть 
$P\hm\subseteq A$, тогда по теореме~2 $\forall \alpha\hm\in P$:  
$C\mapsto \{\alpha\}$, т.\,е.\ из появления~$C$ следует появление 
характеристики~$\alpha$ в~предшествующем объекте. Это необходимое условие 
того, что~$C$ удовлетворяет причинно-следственным связям развития процесса 
выполнения проекта. Если для~$C$ нет характеристики~$\alpha$, которую можно 
отнести к~причине~$C$, то можно считать, что нарушена  
при\-чин\-но-след\-ст\-вен\-ная связь и~$C$~--- аномальный объект. 
  
  \smallskip
  
  \noindent
  \textbf{Пример.} Если объект~$C$ состоит в~получении суммы~$a$ 
фирмой~$K$, то согласно теореме~2 в~пред\-шест\-ву\-ющем объекте должна 
существовать причина перевода суммы~$a$ на фирму~$K$. Если эта причина 
в~проекте отсутствует, то это можно считать признаком мошеннической схемы. 
Все проекты по предположению собираются из <<кубиков>>, содержащихся в~БЗ. 
Тогда можно сравнить цену объекта~$C$, породившего получение суммы~$a$, 
и~сумму, присутствующую в~смете проекта. Если разница велика, то это либо 
ошибка проекта, либо признак мошеннической схемы.
  
  \section{Поиск противоречий на~основе~принципа <<необходимых~условий>>}
   
  Как было показано в~разд.~3, нахождение противоречий соответствуют 
движению от следствия к~причине. Для каждого объекта в~наблюдаемых данных 
выявление причин его появления является трудоемкой задачей. Кроме того, при 
реализации контроля соблюдения при\-чин\-но-след\-ст\-вен\-ных связей на 
большом множестве участников экономической деятельности задача анализа 
причин становится трудоемкой. Поэтому процедуру контроля необходимо разбить 
на два этапа, где первый этап состоит в~анализе простых <<необходимых 
условий>> проявления мошенничества, когда используется хотя бы одна 
известная характеристика причины. Второй этап (в~режиме офлайн) состоит 
в~выявлении причин, позволяющих провести анализ источников мошеннических 
схем. 
  
  Один из подходов к~выбору <<необходимых условий>> состоит в~построении 
множества подцелей исходной цели проекта (структурный метод построения 
проекта~\cite{7-gr}). Каждая подцель описывается диаграммой на рис.~1, 
и~реализации подцелей должны образовывать полный функционал цели. Это 
является необходимым, но не достаточным условием достижения цели, так как 
при таком подходе отсутствует компонент согласования всех подцелей в~единую 
систему. Однако такой подход значительно упрощает анализ выполнения проекта 
на предмет поиска мошенничества. Если признаки мошенничества будут 
обнаружены в~реализации хотя бы одной из подцелей, то это значит, что 
мошенничество присутствует в~реализации всего проекта. 
  
  Аналогично в~реализации каждого этапа в~любой из подцелей можно выделять 
простые <<необходимые условия>> нарушения при\-чин\-но-след\-ст\-венн\-ых 
связей. 
  
  Таким образом, получается множество <<необходимых условий>>, нарушение 
которых свидетельствует о наличии мошенничества. Это множество 
<<необходимых условий>> можно назвать метаданными~[8, 9] для контроля 
проекта на выявление мошенничества. 
  
  
  \section{Заключение }
  
  В поиске противоречий необходимо от транзакций, соответствующих 
следствиям при\-чин\-но-след\-ст\-вен\-ных связей, переходить к~анализу причин 
наблюдаемых следствий. Это сложная задача, которая связана с~описанием причин 
определенных свойств. 
  
  В работе представлена модель, позволяющая строить множество необходимых 
условий соответствия наблюдаемого следствия вызвавшей его причине. Этот 
подход делает поиск противоречий вполне вычислимой задачей, но не гарантирует 
успех. 
  
  {\small\frenchspacing
 {%\baselineskip=10.8pt
 \addcontentsline{toc}{section}{References}
 \begin{thebibliography}{9}
\bibitem{1-gr}
\Au{Грушо А.\,А., Зацаринный~А.\,А., Тимонина~Е.\,Е.} Блокчейны цифровой экономики на базе 
системы ситуационных центров и~централизованного консенсуса~// Радиолокация, навигация, 
связь: Мат-лы XXV Междунар. научн.-технич. конф.~---
Воронеж: Издательский дом ВГУ, 2019. Т.~6. С.~183--191. 
\bibitem{2-gr}
\Au{Grusho A., Zatsarinny~A., Timonina~E.} A~system approach to information security in 
distributed ledgers on the situational centers platform.~---
Lecture notes in computer science ser.~--- Springer, 2019 
(in press).
\bibitem{3-gr}
\Au{Финн В.\,К.} Искусственный интеллект: Методология, применения, философия.~--- М.: 
Красанд, 2011. 448~с.

\bibitem{5-gr} %4
\Au{Аншаков~О.\,М., Фабрикантова~Е.\,Ф.} ДСМ-ме\-тод автоматического порождения 
гипотез: Логические и~эпистемологические основания.~--- М.: Либроком, 2009. 432~с.

\bibitem{4-gr} %5
\Au{Poelmans J., Elzinga~P., Viaene~S., Dedene~G.} Formal concept analysis in knowledge 
discovery: A~survey~// Conceptual structures: From information to intelligence~/ Eds.\ M.~Croitoru, 
S.~Ferr$\acute{\mbox{e}}$, and D.~Lukose.~--- Lecture notes in computer science 
ser.~--- Berlin--Heidelberg: Springer, 2010. Vol.~6208.  P.~139--153.

\bibitem{6-gr}
\Au{Панкратова~Е.\,С., Финн~В.\,К.} Автоматическое по\-рож\-де\-ние гипотез в~интеллектуальных 
системах.~--- М.: Либроком, 2009. 528~с. 
\bibitem{7-gr}
\Au{Денисов А.\,А., Колесников~Д.\,Н.} Теория больших систем управления.~--- Л.: Энергоиздат, 1982. 488~с.

\bibitem{9-gr}
\Au{Грушо А.\,А., Грушо Н.\,А., Забежайло~М.\,И., Смирнов~Д.\,В., Тимонина~Е.\,Е.} 
Параметризация в~прикладных задачах поиска эмпирических причин~// Информатика и~её 
применения, 2018. Т.~12. Вып.~3. С.~62--66.

\bibitem{8-gr}
\Au{Грушо А.\,А., Грушо Н.\,А., Левыкин~М.\,В., Тимонина~Е.\,Е.} Методы идентификации 
захвата хоста в~распределенной ин\-фор\-ма\-ци\-он\-но-вы\-чис\-ли\-тель\-ной сис\-те\-ме, 
защищенной с~помощью метаданных~// Информатика и~её применения, 2018. Т.~12. Вып.~4. 
С.~41--45.

 \end{thebibliography}

 }
 }

\end{multicols}

\vspace*{-3pt}

\hfill{\small\textit{Поступила в~редакцию 03.04.19}}

%\vspace*{8pt}

%\pagebreak

\newpage

\vspace*{-28pt}

%\hrule

%\vspace*{2pt}

%\hrule

%\vspace*{-2pt}

\def\tit{ARCHITECTURAL DECISIONS IN~THE~PROBLEM 
OF~IDENTIFICATION OF~FRAUD IN~THE~ANALYSIS 
OF~INFORMATION FLOWS IN~DIGITAL ECONOMY\\[-5pt]}


\def\titkol{Architectural decisions in~the~problem 
of~identification of~fraud in~the~analysis 
of~information flows in~digital economy}

\def\aut{A.\,A.~Grusho, M.\,I.~Zabezhailo, N.\,A.~Grusho, and~E.\,E.~Timonina}

\def\autkol{A.\,A.~Grusho, M.\,I.~Zabezhailo, N.\,A.~Grusho, and~E.\,E.~Timonina}

\titel{\tit}{\aut}{\autkol}{\titkol}

\vspace*{-13pt}


 \noindent
   Institute of Informatics Problems, Federal Research Center ``Computer Sciences and 
Control'' of the Russian Academy of Sciences; 44-2~Vavilov Str., Moscow 119133, 
Russian Federation

\def\leftfootline{\small{\textbf{\thepage}
\hfill INFORMATIKA I EE PRIMENENIYA~--- INFORMATICS AND
APPLICATIONS\ \ \ 2019\ \ \ volume~13\ \ \ issue\ 2}
}%
 \def\rightfootline{\small{INFORMATIKA I EE PRIMENENIYA~---
INFORMATICS AND APPLICATIONS\ \ \ 2019\ \ \ volume~13\ \ \ issue\ 2
\hfill \textbf{\thepage}}}

\vspace*{3pt}


   
     
   \Abste{An approach to a~research of some types of fraud in digital economy with the usage of relationships of 
cause and effect is formulated. In all types of the considered frauds, the discrepancy between the 
purposes of financial transactions and actual cost of achievement of these purposes
has to be observed. Data on 
transactions can be collected by observing information flows in which these transactions are reflected. 
The architecture of data collection and their analysis can be organized by means of the distributed 
ledgers with the centralized consensus that allows creating an analog of the electronic account book 
fixing financial and economic activity of subjects of digital economy in the region. 
   The methods of fraud identification considered are based on the contradictions 
between actions described in transactions and information, which is contained in plans, standards, 
precedents, etc. 
   The method based on a~simplified scheme of implementation of the abstract project is considered. 
For identification of contradictions, it is necessary to carry out the analysis from the effect to the cause, 
i.\,e., to look for anomalies in information describing the generation of the observed effects. 
   It is shown how in implementation of the project it is possible to allocate simple ``necessary 
conditions'' of violation of cause and effect relationships, i.\,e., a~set of ``necessary conditions'' 
violation of which demonstrates fraud existence. It is possible to call this set of "necessary conditions" 
by metadata for control of the project for fraud identification.} 
   
   \KWE{digital economy; information flows; relationships of reason and effect; detection of 
fraudulent schemes}
   
  

 \DOI{10.14357/19922264190204}

\vspace*{-20pt}

 \Ack
   \noindent
   The work was partially supported by the Russian Foundation for Basic Research (projects  
18-29-03081 and 18-07-00274).



%\vspace*{6pt}

  \begin{multicols}{2}

\renewcommand{\bibname}{\protect\rmfamily References}
%\renewcommand{\bibname}{\large\protect\rm References}

{\small\frenchspacing
 {\baselineskip=10.5pt
 \addcontentsline{toc}{section}{References}
 \begin{thebibliography}{9}
\bibitem{1-gr-1}
\Aue{Grusho, A.\,A., A.\,A.~Zatsarinny, and E.\,E.~Timonina.} 2019. Blokcheyny tsifrovoy ekonomiki 
na baze sistemy situatsionnykh tsentrov i~tsentralizovannogo konsensusa [Blockchains of digital 
economy on the basis of the system of the situational centres and the centralized consensus]. 
\textit{25th Scientific and Technical Conference (International) ``Radar-Location, Navigation, 
Communication'' Proceedings}. Voronezh: VSU Publs. 6:183--191.
\bibitem{2-gr-1}
\Aue{Grusho, A., A.~Zatsarinny, and E.~Timonina.} 2019 (in press). 
A~system approach to information security 
in distributed ledgers on the situational centers platform. 
Lecture notes in computer science ser. Springer.
\bibitem{3-gr-1}
\Aue{Finn, V.\,K.} 2011. \textit{Iskusstvennyy intellekt: Metodologiya, primeneniya, filosofiya} 
[Artificial intelligence: Methodology, applications, philosophy]. Moscow: KRASAND. 448~p.

\bibitem{5-gr-1}
\Aue{Anshakov, O.\,M., and E.\,F.~Fabrikantova}. 2009. \textit{DSM-metod avtomaticheskogo porozhdeniya gipotez: Logicheskie 
i~epistemologicheskie osnovaniya} [JSM-method of automatic hypothesis generation: Logical and 
epistemological]. Moscow: KD LIBROKOM. 432~p.
\bibitem{4-gr-1} %5
\Aue{Poelmans, J., P.~Elzinga, S.~Viaene, and G.~Dedene.} 2010. Formal concept analysis in 
knowledge discovery: A~survey. \textit{Conceptual structures: From information to intelligence}. 
Eds.\ M.~Croitoru, S.~Ferr$\acute{\mbox{e}}$, and D.~Lukose. Lecture notes in 
computer science ser. Berlin--Heidelberg: Springer. 6208:139--153.

\bibitem{6-gr-1}
\Aue{Pankratov, E.\,S., and V.\,K.~Finn}. 
2009. \textit{Avtomaticheskoe porozhdenie gipotez v~intellektual'nykh 
sistemakh} [Automatic hypotheses generation in intelligent systems]. Moscow: KD 
\mbox{LIBROKOM}.  528~p. 
\bibitem{7-gr-1}
\Aue{Denisov, A.\,A., and D.\,N.~Kolesnikov.} 1982. \textit{Teoriya bol'shikh 
sistem upravleniya} [Theory of big control systems]. Leningrad: Energoizdat. 488~p.

\bibitem{9-gr-1}
\Aue{Grusho, A.\,A., N.\,A.~Grusho, M.\,I.~Zabezhailo, D.\,V.~Smirnov, and 
E.\,E.~Timonina.} 2018. 
Parametrizatsiya v~prikladnykh zadachakh poiska empiricheskikh prichin 
[Parametrization in applied 
problems of search of the empirical reasons]. 
\textit{Informatika i~ee Primeneniya~--- 
Inform. Appl.} 12(3):62--66.

\bibitem{8-gr-1}
\Aue{Grusho, A.\,A., N.\,A.~Grusho, M.\,V.~Levykin, and E.\,E.~Timonina.} 2018. Metody 
identifikatsii zakhvata khosta v~raspredelennoy informatsionno-vychislitel'noy sisteme, 
zashchishchennoy s~pomoshch'yu metadannykh [Methods of identification of host capture 
in the  distributed information system which is protected on the base of meta data].
\textit{Informatika i~ee 
Primeneniya~--- Inform. Appl.} 12(4):41--45.
{ %\looseness=1

}

\end{thebibliography}

 }
 }

\end{multicols}

\vspace*{-12pt}

\hfill{\small\textit{Received April 3, 2019}}

%\pagebreak

%\vspace*{-18pt}

\Contr

\noindent
\textbf{Grusho Alexander A.} (b.\ 1946)~--- Doctor of Science in physics and 
mathematics, professor, principal scientist, Institute of Informatics Problems, 
Federal Research Center ``Computer Sciences and Control'' of the Russian 
Academy of Sciences; 44-2~Vavilov Str., Moscow 119133, Russian Federation; 
\mbox{grusho@yandex.ru} 

\vspace*{3pt}

\noindent
\textbf{Zabezhailo Michael I.} (b.\ 1956)~--- Doctor of Science in physics and 
mathematics, principal scientist, Institute of Informatics Problems, Federal Research 
Center ``Computer Sciences and Control'' of the Russian Academy of Sciences;  
44-2~Vavilov Str., Moscow 119133, Russian Federation; 
\mbox{m.zabezhailo@yandex.ru} 

\vspace*{3pt}


\noindent
\textbf{Grusho Nikolai A.} (b.\ 1982)~--- Candidate of Science (PhD) in physics 
and mathematics, senior scientist, Institute of Informatics Problems, Federal 
Research Center ``Computer Sciences and Control'' of the Russian Academy of 
Sciences; 44-2~Vavilov Str., Moscow 119133, Russian Federation; 
\mbox{info@itake.ru} 

\vspace*{3pt}


\noindent
\textbf{Timonina Elena E.} (b.\ 1952)~--- Doctor of Science in technology, 
professor, leading scientist, Institute of Informatics Problems, Federal Research 
Center ``Computer Sciences and Control'' of the Russian Academy of Sciences;  
44-2~Vavilov Str., Moscow 119133, Russian Federation; 
\mbox{eltimon@yandex.ru} 

\label{end\stat}

\renewcommand{\bibname}{\protect\rm Литература}        %6Abst+avt
\include{gudasa}      %7Abst+tavt
\def\stat{zatsman}

\def\tit{ТРАНСФОРМАЦИИ ОБЪЕКТОВ ПЕРВОГО И~ВТОРОГО ПОРЯДКА 
В~ЛЕКСИКОГРАФИЧЕСКОЙ ИНФОРМАЦИОННОЙ СИСТЕМЕ$^*$}

\def\titkol{Трансформации объектов первого и~второго порядка 
в~лексикографической информационной системе}

\def\aut{И.\,М.~Зацман$^1$}

\def\autkol{И.\,М.~Зацман}

\titel{\tit}{\aut}{\autkol}{\titkol}

\index{Зацман И.\,М.}
\index{Zatsman I.\,M.}


{\renewcommand{\thefootnote}{\fnsymbol{footnote}} \footnotetext[1]
{Исследование выполнено в~ФИЦ ИУ РАН за счет гранта Российского научного фонда №\,24-18-00155, {\sf 
https://rscf.ru/project/24-18-00155}. Работа выполнялась с~использованием инфраструктуры Центра 
коллективного пользования <<Высокопроизводительные вычисления и~большие данные>> (ЦКП 
<<Информатика>>) ФИЦ ИУ РАН (г.\ Москва).}}


\renewcommand{\thefootnote}{\arabic{footnote}}
\footnotetext[1]{ Федеральный исследовательский центр <<Информатика и~управление>> Российской академии наук, 
\mbox{izatsman@yandex.ru}}

\vspace*{-12pt}


  
  \Abst{Рассматриваются теоретические основания проектирования информационных 
технологий (ИТ) интеграции двуязычных словарей и~параллельных корпусов. Дано описание 
первых результатов создания третьего уровня классификации трансформаций объектов 
предметной области информатики, которую предполагается использовать при создании 
концепции лексикографической информационной системы, обеспечивающей интеграцию. 
Все сущности информатики в~статье разделены на два глобальных класса: объекты и~их 
трансформации. Для каждого такого класса конструируется своя классификация. Ранее были 
описаны два верхних уровня классификации трансформаций объектов предметной области. 
В~данной статье рассматривается третий уровень этой классификации. Основанием для 
построения самого верхнего ее уровня служило деление предметной области информатики 
на среды (ментальная, сенсорно воспринимаемая, цифровая и~ряд других сред), каждая из 
которых по определению включает объекты одной природы. Основанием для построения 
второго уровня классификации трансформаций объектов служила типология знаковых  
сис\-тем А.~Соломоника. Цель статьи состоит в~систематизации трансформаций первого 
и~второго порядка объектов предметной области на третьем уровне этой классификации. 
Основанием для систематизации служит средовая версия иерархии Акоффа.}
  
  \KW{объекты предметной области; трансформации объектов; классификация; данные; 
информация; знание; лексикографическая информационная сис\-тема}

\DOI{10.14357/19922264240211}{VZTGVV}
  
\vspace*{3pt}


\vskip 10pt plus 9pt minus 6pt

\thispagestyle{headings}

\begin{multicols}{2}

\label{st\stat}
  
\section{Введение}

\vspace*{-9pt}

  Возникновение параллельных корпусов, в~которых предложениям 
оригинального текста со\-по\-став\-ле\-ны предложения его перевода, обеспечило 
возможность контрастивного лингвистического\linebreak \mbox{анализа} на принципиально 
новом уровне полноты и~точности, недостижимом в~докорпусную эпоху. 
Пионерскими в~этой области стали работы \mbox{1990-х~гг}. Стига Йоханссона  
с~анг\-ло-нор\-веж\-ским корпусом~[1]. В России параллельные корпусы стали 
формироваться в~начале XXI~века в~рамках Национального корпуса русского 
языка~[2].
  
  Создатели двуязычных словарей используют параллельные корпусы для 
сбора материала и~эмпирической проверки своих гипотез, касающихся 
межъязы\-ко\-вой эквивалентности. Ценность параллельных корпусов 
определяется тем, что в~лингвистике этап сбора исходного материала считается 
наиболее трудоемким и~наименее творческим, а~параллельные корпусы 
позволяют значительно сэкономить время и~силы для творческого этапа 
создания словарей~[3].
 % 
  При этом двуязычные словари, создаваемые на основе исходного материала, 
извлеченного из параллельных корпусов, сейчас формируются без связей с~их 
текстами. Другими словами, онлайновые связи созданных словарей 
с~параллельными корпусами, которые служили источниками исходного 
материала, отсутствуют. 

Параллельные корпусы постоянно пополняются 
новыми текстами, в~предложениях которых можно обнаружить новые значения 
слов и~устойчивых словосочетаний. Однако при этом отсутствуют методы 
и~средства оперативного обновления словарей по корпусным данным. 
В~настоящее время проблема установления связей между двуязычными 
словарями и~параллельными корпусами (далее~--- проблема интеграции) 
находится на стадии поиска концептуальных подходов к~их интеграции на 
уровне значений.
  
  Подход к~решению проблемы интеграции, предлагаемый в~статье, учитывает 
  и~появление новых значений слов и~устойчивых словосочетаний, и~динамику 
смысловых значений, которая обусловлена развитием и~пополнением знания 
лингвистов, фиксирующих эти значения в~результате семантического анализа 
пополняемых корпусных данных. Проведенные эксперименты показали, что 
обнаружение нового лингвистического знания обусловливает и~формирование 
дефиниций новых значений, и~пересмотр уже существующих дефиниций~[4, 5].
  
  Например, в~проведенных экспериментах с~использованием ЦКП 
<<Информатика>> ФИЦ ИУ РАН фиксировалась эволюция значений немецких 
модальных глаголов, исходное состояние значений которых было описано 
в~не\-мец\-ко-рус\-ском словаре. В~экспериментальном массиве текстов как 
потенциальных источниках нового знания 16\,268 предложений содержали 
немецкие модальные глаголы и~в~2041 из них встречался глагол sollen. 
В~начале эксперимента в~словаре были описаны~12~значений этого модального 
глагола. По окончании эксперимента лингвисты обнаружили два новых его 
значения, согласовали их дефиниции и~описали эволюцию дефиниций~[6, 7].
  
  Таким образом, для решения проблемы интеграции требуется фиксировать 
новое знание, обнаруженное лингвистами в~текстовых данных параллельных 
корпусов, отслеживать эволюцию знания, представленного в~виде дефиниций 
значений слов и~устойчивых словосочетаний, и,~соответственно, 
актуализировать электронные двуязычные словари. Предлагаемый 
концептуальный подход к~интеграции, который планируется реализовать 
в~процессе проектирования лексикографической информационной сис\-те\-мы, 
фиксирующей эволюцию лингвистического знания, основан на решении 
следующих задач:\\[-14pt]
  \begin{itemize}
  \item категоризация трех базовых понятий информатики, включенных 
  в~иерархию Акоффа~[8] (данные, информация, знание), на объекты 
проектируемой сис\-те\-мы, которая необходима, чтобы фиксировать 
<<кванты>> нового знания и~отслеживать его эволюцию в~этой сис\-теме;\\[-15pt]
  \item  систематизация трансформаций объектов этой сис\-темы.\\[-14pt]
  \end{itemize}
  
  Цель статьи и~состоит в~решении двух задач: категоризации трех базовых 
понятий информатики на объекты лексикографической информационной  
сис\-те\-мы и~сис\-те\-ма\-ти\-за\-ции трансформаций первого и~второго порядка 
ее объектов.
  
  Трансформациями первого порядка, о которых сказано в~формулировке цели 
статьи, называются взаимные преобразования между двумя объектами  
сис\-те\-мы одной природы. Например, перевод в~сис\-те\-ме текста с~русского 
языка на английский относится к~ним. Трансформациями второго порядка 
и~выше называются взаимные преобразования между двумя и~более объектами 
разной природы. Например, кодирование символов текс\-та компьютерными 
кодами и~их декодирование относятся по определению к~трансформациям 
второго порядка.

%\vspace*{-9pt}
  
\section{Процессы трансформаций в~информатике}

%\vspace*{-3pt}

Процессы трансформаций, рассматриваемые в~статье, относятся к~теоретическому ядру информатики, а~не 
только к~проектированию лексикографической информационной сис\-те\-мы. Например, из трех основных 
подходов к~описанию предметной об\-ласти информатики\footnote{В статье предметная область информатики 
трактуется согласно концепции полиадического компьютинга Пола Розенблума~\cite{9-zac}.} (объектный, 
трансформационный и~синтетический) сис\-те\-ма\-ти\-за\-ция трансформаций ближе всего ко второму 
подходу. Примерами первого подхода, в~рамках которого основное внимание уделяется объектам предметной 
области информатики и~в~меньшей степени отношениям\linebreak между ними, могут служить  
работы~\cite{8-zac, 10-zac, 11-zac}; \mbox{примерами} второго подхода, в~рамках которого основное внимание 
уделяется трансформациям и~в~меньшей степени трансформируемым объектам,~---  
работы~\cite{12-zac, 13-zac}; примерами третьего, синтетического подхода, в~котором уделяется внимание 
и~объектам предметной об\-ласти информатики, и~отношениям между ними, могут служить работы~\cite{14-zac, 
15-zac, 16-zac, 17-zac, 18-zac}.

  Таким образом, для описания трансформаций объектов лексикографической 
информационной\linebreak системы предпочтительнее всего трансформационный 
подход, который упоминается и~в определениях информатики. Например, 
в~2009~г.\ П.~Деннинг и~П.~Розенблум сформулировали суть \mbox{информатики} как 
компьютинга следующим образом: <<$\ldots$информатика~--- это не просто 
алгоритмы и~структуры данных; это преобразования [трансформации] 
представлений>>~\cite{12-zac}. Чуть позже, в~контексте краткого описания 
парадигмы информатики как компьютинга, П.~Деннинг и~П.~Фриман изменили 
эту формулировку на такую: <<Центральный объект внимания в~информатике 
можно определить как информационные процессы~--- \textit{естественные или 
искусственные процессы, преобразующие информацию} (курсив мой~--- 
И.\,З.)>>~\cite{13-zac}. Согласно парадигме, предлагаемой авторами этой 
статьи, на начальном этапе проектирования автоматизированных систем 
базовыми элементами моделей их функционирования служат 
\textit{информационные про\-цессы}.
  
  Однако если 15~лет назад в~формулировке из работы~\cite{13-zac} шла речь 
о~процессах, преобразующих информацию, то в~последние~10~лет в~спектр 
процессов трансформаций все чаще стали включать процессы, преобразующие 
не только информацию, но также и~другие объекты автоматизированных 
систем, в~первую очередь данные и~знания~[19--21]. Например, Виктория 
Стодден, позиционируя науку о~данных как одну из дисциплин информатики, 
говорит, что центральный объект исследований в~науке о~данных~--- это 
<<изучение обобщаемого извлечения знания из данных>>~\cite{21-zac}. 
Увеличение и~чис\-ла объектов, и~спект\-ра процессов их трансформаций 
в~автоматизированных сис\-те\-мах обуслов\-ли\-ва\-ет не\-об\-хо\-ди\-мость 
систематизации и~объектов, и~процессов их трансформаций на начальном этапе 
проектирования сис\-тем.
  
  Для создания концепции лексикографической информационной сис\-те\-мы 
и~проектирования ИТ, обеспечивающих интеграцию 
двуязычных словарей и~параллельных корпусов, сначала выполним 
категоризацию на объекты этой сис\-те\-мы трех базовых понятий информатики 
(данные, информация, знание) в~контексте построения классификаций 
сущностей ее предметной об\-ласти.
  
  Необходимость использования классификаций информатики в~процессе 
создания концепции проиллюстрируем, используя иерархию  
Акоффа~\cite{8-zac}. Он использовал принцип их вертикального размещения 
в~иерархии снизу вверх: данные, информация и~знание. Еще в~ней есть термин 
<<мудрость>>, который в~статье не рассматривается. Такое размещение Акофф 
прокомментировал так: <<Каждое из пе\-ре\-чис\-лен\-ных понятий [кроме данных] 
содержит в~себе нижестоящие$\ldots$>>~\cite{8-zac}.
  
  Этому принципу размещения и~комментарию Акоффа свойственны 
недостатки, проанализированные, в~частности, в~работе~\cite{10-zac}. Главный 
вывод, к~которому пришла Роули после изучения иерархии Акоффа, 
заключается в~следующем: <<$\ldots$информация определяется в~терминах 
данных, знание~--- в~терминах информации$\ldots$ но существует меньше 
консенсуса в~описании трансформаций, которые преобразуют сущности, 
расположенные ниже в~иерархии, в~те, которые находятся над ними, что 
приводит к~их терминологической неопределенности>>~\cite{10-zac}. Причина 
этой неопределенности, скорее всего, в~том, что базовые понятия информатики 
включены в~иерархию Акоффа изолированно от общего контекста 
классификаций сущностей ее предметной об\-ласти.

%\vspace*{-9pt}
  
\section{Классификации сущностей информатики}


%\vspace*{-2pt}

  Все сущности предметной области информатики в~работах~[22, 23] 
разделены на два глобальных класса: ее объекты и~их трансформации. Для 
каждого такого класса была предложена своя классификация. 
В~работе~\cite{22-zac} дано описание классификации объектов предметной 
области информатики, первый уровень которой содержит базовые понятия ее 
предметной области (данные, информация, знания и~др.).  
В~работе~\cite{23-zac} дано описание двух верхних уровней классификации 
трансформаций объектов предметной об\-ласти (см.\ рисунок 
в~работе~\cite{23-zac}). Основанием для построения самого верхнего ее уровня послужило деление 
предметной области информатики на среды\footnote{В~работе~\cite{24-zac} дано описание пяти сред 
предметной области информатики (ментальная; сенсорно воспринимаемая, или информационная; 
цифровая; нейро- и~ДНК-среда), каждая из которых по определению включает объекты одной и~той же 
природы.} и~степень разнообразия природы объектов, вовлеченных в~трансформации:
\begin{itemize}
\item  первый класс верхнего уровня классификации включает 
трансформации объектов в~пределах среды только одной природы 
(трансформации первого порядка);
\item  второй класс включает трансформации объектов, относящихся 
к~двум средам разной природы (трансформации второго порядка);
\item третий и~последующие классы включают трансформации объектов, 
относящихся к~трем и~более средам разной природы (трансформации 
третьего и~более высоких порядков).
\end{itemize}

  В работе~\cite{23-zac} были приведены примеры для трех первых классов 
трансформаций, включая пример трансформаций объектов, относящихся 
к~двум средам разной природы (компьютерное кодирование символов текстов 
с~по\-мощью таб\-лиц Unicode).
  
Основанием для построения второго уровня классификации трансформаций объектов послужила типология 
знаковых сис\-тем А.~Соломоника~\cite[c.~131]{25-zac}: естественные знаковые сис\-те\-мы, образные,  
ес\-тест\-вен\-но-язы\-ко\-в$\acute{\mbox{ы}}$е,  
вер\-баль\-но-не\-сло\-вес\-ные сис\-те\-мы записи\footnote{Под системой записи понимается знаковая 
система, сочетающая вербальные знаки с~несловесными (языки нотной записи, карт, таблиц и~др.).} 
и~формализованные знаковые сис\-те\-мы, включая математические. Введем понятие обобщенного текста~--- 
это текст, который может быть создан в~любой из перечисленных знаковых систем. Тогда обобщенные тексты 
могут быть естественными, образными, ес\-тест\-вен\-но-язы\-ко\-в$\acute{\mbox{ы}}$\-ми,  
вер\-баль\-но-не\-сло\-вес\-ны\-ми и~формализованными. Второй уровень классификации трансформаций 
охватывает не все виды объектов предметной  
об\-ласти информатики, а~только перечисленные~5~видов текс\-тов и~их представления, вовлеченные 
в~процессы трансформаций в~одной или более средах вместе с~данными, знанием и~его концептами.

\begin{figure*}[b] %fig1
\vspace*{6pt}
      \begin{center}
     \mbox{%
\epsfxsize=121.191mm 
\epsfbox{zac-1.eps}
}
\end{center}
\vspace*{-6pt}
\Caption{Средовая версия иерархии Акоффа}
\end{figure*}

\section{Классификация трансформаций: построение~третьего 
уровня}

  Основанием для систематизации трансформаций первого и~второго порядка 
на третьем уровне этой классификации служит иерархия Акоффа~\cite{8-zac}, 
на основе которой и~была создана ее средов$\acute{\mbox{а}}$я версия~[26, 
27]. Для создания средов$\acute{\mbox{о}}$й версии была выполнена 
категоризация трех базовых понятий информатики (данные, информация, 
знания) на объекты лексикографической информационной сис\-те\-мы 
в~процессе создания ее концепции\linebreak (рис.~1).
  


  В отличие от классической иерархии Акоффа, в~ее 
средов$\acute{\mbox{о}}$й версии различаются три вида данных: сенсорно 
воспринимаемые, цифровые и~те данные, которые генерируются 
искусственными нейронными сетями (ИНС) в~системах искусственного интеллекта 
(далее~--- ИИ-дан\-ные). Последний вид данных необходим, например, для 
различения входа и~выхода процесса применения обученной 
ИНС в~цифровой модели генерации знания, описанию которой 
посвящена работа~\cite{27-zac}.
  
  Также предлагается различать два вида информации: сенсорно 
воспринимаемая и~цифровая. Кроме знания в~средов$\acute{\mbox{у}}$ю 
версию добавлены концепты и~ментальные образы сенсорно воспринимаемых 
данных. Последние служат промежуточной сущностью между сенсорно 
воспринимаемыми данными и~генерируемым знанием при описании процессов 
извлечения знания из текстовых данных лексикографической информационной 
системы. Описание объектов средов$\acute{\mbox{о}}$й версии иерархии 
Акоффа (см.\ рис.~1) и~отношений между ними дано в~работах~\cite{26-zac, 28-zac}.
  
  В средов$\acute{\mbox{о}}$й версии число объектов равно восьми. Если 
учитывать направления трансформаций, то между восемью объектами на 
рис.~1 она включает~16 их видов (трансформации на границе между сенсорно 
воспринимаемыми данными и~информацией, обозначенные символом~<<?>>, 
в~статье не рас\-смат\-ри\-ва\-ют\-ся). В~будущем число объектов 
в~средов$\acute{\mbox{о}}$й версии, которая выбрана как основание для 
сис\-те\-ма\-ти\-за\-ции трансформаций первого и~второго порядка, может быть 
увеличено. Для построения классификации трансформаций 
важ\-но не возможное увеличение числа объектов 
и~трансформаций между ними, а то, что их виды в~средов$\acute{\mbox{о}}$й 
версии распределены между трансформациями первого и~второго порядка. Из 
16~видов на рис.~1 шесть относятся к~трансформациям первого порядка, это\linebreak 
виды с~номерами~7, 8, 13--16 (далее~--- типология трансформаций первого 
порядка), а~десять~--- к~трансформациям второго порядка, это виды 
с~\mbox{номерами}~1--6 и~9--12 (далее~--- типология трансформаций второго 
порядка). Разместим обе типологии на третьем уровне классификации (см.\ ее 
схему на рис.~2). Перечислим виды трансформаций первой типологии, вводя 
в~скобках их краткие названия, используемые ниже на рис.~3:
  \begin{description}
  \item[\,] 7~--- членение знания на концепты с~помощью одной или нескольких 
знаковых систем (далее~--- членение знания);
  \item[\,] 8~--- формирование знания на основе концептов (формирование 
знания);
  \item[\,] 13~--- обучение ИНС;
  \end{description}
  
  \vspace*{-6pt}
  
  \pagebreak
  
  \end{multicols}
  
  \begin{figure*} %fig2
\vspace*{1pt}
      \begin{center}
     \mbox{%
\epsfxsize=127.513mm 
\epsfbox{zac-2.eps}
}
\end{center}
\vspace*{-9pt}
\Caption{Схема трех верхних уровней классификации трансформаций объектов (объединены 
по три слоя и~для второго, и~для третьего уровней этой классификации)}
\end{figure*}
  
  \begin{multicols}{2}
  
  \noindent
  \begin{description}
  \item[\,] 14~--- восстановление обучающей информации на основе 
содержания обученной ИНС (обращение ИНС);
  \item[\,] 15~--- использование обученной ИНС (использование ИНС);



  \item[\,] 16~--- восстановление исходных данных, соответствующих 
полученным результатам работы обучен\-ной ИНС (восстановление исходных данных 
по результатам ИНС).
  \end{description}
  
  
  Не все виды трансформаций 13--16 поддерживаются в~конкретных системах 
искусственного интеллекта, но с~теоретической точки зрения все их 
предлагается включить в~первую типологию для полноты спектра видов 
трансформаций.
  
  Перечислим виды трансформаций второй типологии:
  \begin{description}
  \item[\,] 1~--- декодирование цифровых данных в~компьютерных системах 
(декодирование данных);
  \item[\,]  2~--- кодирование сенсорно воспринимаемых данных (кодирование 
данных);
  \item[\,] 3~--- ментальное копирование сенсорно воспринимаемых данных 
(ментальное копирование);
  \item[\,] 4~--- восстановление сенсорно воспринимаемых данных по 
ментальным образам (восстановление по образам);
  \item[\,] 5~--- смысловая интерпретация без деления на концепты ментальных 
образов сенсорно воспринимаемых данных (смысловая интерпретация);
  \item[\,] 6~--- восстановление ментальных образов (восстановление образов);
  \item[\,] 9~--- представление концептов в~виде сенсорно воспринимаемой 
информации, например текс\-та\-ми, формулами, таблицами, рисунками и~т.\,д.\ 
(представление концептов);
  \item[\,] 10~--- понимание смысла сенсорно воспринимаемой информации 
(понимание смысла);
  \item[\,] 11~--- кодирование сенсорно воспринимаемой информации 
(кодирование информации);
\end{description}

\vspace*{-6pt}

\pagebreak

\end{multicols}

\begin{figure*} %fig3
\vspace*{1pt}
      \begin{center}
     \mbox{%
\epsfxsize=163mm 
\epsfbox{zac-3.eps}
}
\end{center}
\vspace*{-9pt}
\Caption{Схема частного случая классификации трансформаций объектов (трансформации 
пронумерованы согласно рис.~1)}
\end{figure*}

\begin{multicols}{2}

\noindent
\begin{description}

  \item[\,] 12~--- декодирование цифровой информации (декодирование 
информации).
  \end{description}
  
  Отметим, что в~существующих ИТ
  и~компьютерных системах наиболее часто используются виды 
трансформаций~13 и~15 типологии первого порядка и~1, 2, 11 и~12 типологии 
второго порядка. На рис.~2 в~первом слое третьего уровня классификации 
показаны типологии первого порядка без указания числа трансформаций в~них 
и~без детализации трансформируемых объектов.
  
  Во втором слое третьего уровня классификации условно (без названий) 
показаны типологии второго порядка. Также на рис.~2 в~третьем слое третьего 
уровня классификации условно (также без названий) показаны типологии 
третьего порядка, которые планируется рассмотреть в~отдельной статье. По 
определению они должны включать трансформации между тремя объектами 
разной природы, но средов$\acute{\mbox{а}}$я версия иерархии Акоффа 
включает трансформации только между двумя объектами разной природы. 
Поэтому потребуется другое основание для их систематизации (ранее были 
рассмотрены отдельные примеры трансформаций третьего 
порядка\footnote{Далеко не всегда трансформации третьего и~более высоких порядков можно 
рассматривать как последовательность трансформаций второго порядка. Примером этого могут 
служить трансформации в~процессе обучения пациента пользованию роботизированной рукой, 
охватывающие личностные концепты пациента, релевантные его намерениям, сигналы активности 
мозга как объекты нейросреды и~компьютерные коды~\cite{29-zac}.}~\cite{29-zac}).

\section{Классификация трансформаций: частный~случай}

  Выше было отмечено, что в~будущем число объектов 
в~средов$\acute{\mbox{о}}$й версии иерархии Акоффа может быть увеличено. 
Это означает, что увеличатся и~чис\-ло объектов, и~чис\-ло трансформаций между 
ними в~классификации трансформаций, так как эта средов$\acute{\mbox{а}}$я 
версия служит по определению основанием для систематизации 
трансформаций первого и~второго порядка. Поэтому на третьем уровне рис.~2 
указаны типологии без детализации объектов и~без указания числа 
трансформаций в~каждой из них. С~одной стороны, при таком подходе 
получаем достаточно общий вид этой классификации, так как она не зависит от 
числа объектов в~том или ином варианте средов$\acute{\mbox{о}}$й версии 
(и~это существенно упрощает рис.~2). С~другой стороны, на третьем уровне 
такой общей классификации подразумевается, но не эксплицируется природа 
трансформируемых объектов и~их возможные сочетания в~трансформациях. 

При проектировании лексикографической информационной системы важно 
эксплицировать природу трансформируемых объектов и~их возможные 
сочетания.
  %
  Поэтому в~парадигму информатики~\cite{30-zac} кроме общей 
классификации трансформаций предлагается включать и~ее частные случаи, 
эксплицирующие природу трансформируемых объектов. 

В~этом разделе 
рассмотрим один частный случай, когда используются только естественные 
знаковые сис\-те\-мы из типологии А.~Соломоника~\cite{25-zac} вместе 
с~данными, знанием и~его концептами. Чис\-ло естественных языков при этом не 
ограничено. И~этот частный случай классификации включает только три 
класса природных трансформаций (первого, второго и~третьего порядка, см.\ 
схему классификации на рис.~3).
  
  Первый и~второй уровни схемы общей классификации (см.\ рис.~2) можно 
объединить в~один уровень в~этом частном случае. Ниже этого уровня 
приведено содержание типологий первого и~второго порядка без содержания 
типологий третьего по\-рядка.




  Наполнение типологий первого и~второго порядка соответствует 
средов$\acute{\mbox{о}}$й версии иерархии Акоффа на рис.~1, содержащей 
6~видов трансформаций типологии первого порядка и~10~видов 
трансформаций типологии второго порядка (на рис.~3 стрелки указывают 
направления трансформаций согласно средов$\acute{\mbox{о}}$й версии на рис.~1).
  
  Таким образом, частный случай классификации содержит для этих двух 
типологий 16~теоретически возможных трансформаций, 6 из которых 
в~настоящее время в~существующих ИТ применяются наиболее часто: виды 
трансформаций~1, 2, 11 и~12 типологии второго порядка реализуются 
с~помощью тех или иных методов ко\-ди\-ро\-ва\-ния/де\-ко\-ди\-ро\-ва\-ния 
(например, с~использованием таблиц Unicode), а~виды трансформаций~13 и~15
 в~типологии первого порядка реализуются полностью с~по\-мощью процессов 
цифровой обработки компьютерами.
  
  Остальные виды трансформаций или применяются намного реже (это 
виды~3, 5, 7, 9 и~10), или находятся в~стадии поиска и~разработки (14 и~16) или 
в~настоящее время носят только теоретический характер, обеспечивая полноту 
первой и~второй типологий (4, 6 и~8). Знаком~<<?>> обозначены те виды 
трансформаций, которые по определению не существуют в~используемой 
парадигме информатики~\cite{30-zac}. Однако возможно, что в~других 
будущих подходах к~построению ее парадигмы эти виды трансформаций будут 
существовать.
  
\section{Заключение}

  На сегодняшний день процесс построения классификаций объектов 
предметной области информатики~\cite{22-zac} и~их  
трансформаций~\cite{23-zac} еще не завершен. Однако первые результаты их 
построения уже используются для создания концепции лексикографической 
информационной сис\-те\-мы, обеспечивающей интеграцию двуязычных 
словарей и~параллельных корпусов.
  
  \bigskip
  
  
  Автор признателен рецензентам за помощь в~улучшении статьи.
  
{\small\frenchspacing
 { %\baselineskip=10.6pt
 %\addcontentsline{toc}{section}{References}
 \begin{thebibliography}{99}
\bibitem{1-zac}
\Au{Aijmer K., Altenberg~B.} Advances in corpus-based contrastive linguistics. Studies in honour 
of Stig Johansson.~--- Amsterdam: John Benjamins, 2013. 295~p.  doi: 10.1075/scl.54.
\bibitem{2-zac}
\Au{Добровольский Д.\,О., Кретов~А.\, А., Шаров~С.\,А.} Корпус параллельных текстов~// 
Научная и~техническая информация. Сер.~2: Информационные процессы и~сис\-те\-мы, 2005. 
№\,6. С.~16--27.
\bibitem{3-zac}
\Au{Добровольский Д.\,О.} Корпус параллельных текстов и~сопоставительная 
лексикология~// Труды Института русского языка им.\ В.\,В.~Виноградова, 2015. №\,6. 
С.~413--449. EDN: VJQBHP.
\bibitem{4-zac}
\Au{Гончаров А.\,А., Зацман~И.\,М., Кружков~М.\,Г.} Эволюция классификаций 
в~надкорпусных базах данных~// Информатика и~её применения, 2020. Т.~14. Вып.~4. 
С.~108--116. doi: 10.14357/19922264200415.  
EDN: \mbox{GKWBZT}.
\bibitem{5-zac}
\Au{Гончаров А.\, А., Зацман И. \,М., Кружков~М.\, Г}. Представление новых 
лексикографических знаний в~динамических классификационных сис\-те\-мах~// 
Информатика и~её применения, 2021. Т.~15. Вып.~1. С.~86--93.  doi: 10.14357/19922264210112. EDN: OPEFXW.
\bibitem{6-zac}
\Au{Zatsman I.} Finding and filling lacunas in linguistic typologies~// 15th Forum (International) 
on Knowledge Asset Dynamics Proceedings.~--- Matera, Italy: Institute of Knowledge Asset 
Management, 2020. P.~780--793.
\bibitem{7-zac}
\Au{Zatsman I.} Three-dimensional encoding of emerging meanings in AI-systems~// 21st 
European Conference on Knowledge Management Proceedings.~--- Reading, U.K.: Academic 
Publishing International Ltd., 2020. P.~878--887.
\bibitem{8-zac}
\Au{Ackoff R.} From data to wisdom~// J.~Applied Systems Analysis, 1989. Vol.~16. No.\,1. P.~3--9.
\bibitem{9-zac}
\Au{Rosenbloom P.\,S.} On computing: The fourth great scientific domain.~--- Cambridge, MA, 
USA: MIT Press, 2013. 307~p.
\bibitem{10-zac}
\Au{Rowley J.} The wisdom hierarchy: Representations of the DIKW hierarchy~// J.~Inf. 
Sci., 2007. Vol.~33. Iss.~2. P.~163--180. doi: 10.1177/0165551506070706.
\bibitem{11-zac} 
\Au{Frick$\acute{\mbox{e}}$~M.\,H.} Data--Information--Knowledge--Wisdom (DIKW) pyramid, 
framework, continuum~// Encyclopedia of big data~/ Eds. L.~Schintler, C.~McNeely.~--- Cham: 
Springer, 2018. 4~p. doi: 10.1007/978-3-319-32001-4\_331-1.
\bibitem{12-zac}
\Au{Denning P., Rosenbloom~P.} Computing: The fourth great domain of science~// Commun. 
ACM, 2009. Vol.~52. Iss.~9. P.~27--29.
\bibitem{13-zac}
\Au{Denning P., Freeman~P.} Computing's paradigm~// Commun.  ACM, 2009. Vol.~52. 
Iss.~12. P.~28--30. doi: 10.1145/ 1610252.1610265.
\bibitem{17-zac} %14
\Au{Farradane J.} Knowledge, information, and information science~// J.~Inf. Sci., 
1980. Vol.~2. Iss.~2. P.~75--80. doi: 10.1177/01655515800020020.

\bibitem{15-zac}
\Au{Шрейдер Ю.\,А.} Информация и~знание~// Сис\-тем\-ная концепция информационных 
процессов.~--- М.: ВНИИСИ, 1988. С.~47--52.
\bibitem{16-zac}
\Au{Ingwersen P.} Information and information science~// Enclyclopaedie of library and 
information science~/ Eds. J.\,D.~McDonald, 
M.~Levine-Clark.~--- New York, NY, USA: Marcel Dekker Inc., 1992. Vol.~56. Sup.~19. 
P.~137--174.

\bibitem{14-zac} %17
Информатика как наука об информации: Информационный, документальный, 
технологический, экономический, социальный и~организационный аспекты~/ Под ред. 
Р.\,С.~Гиляревского.~--- М.: Фаир-Пресс, 2006. 592~с.

\bibitem{18-zac}
\Au{Hjorland B.} Library and information science: practice, theory, and philosophical basis~// 
Inform. Process. Manag., 2000. Vol.~36. Iss.~3. P.~501--531. doi:  
10.1016/S0306-\mbox{4573(99)00038-2}.
\bibitem{19-zac}
Deep shift~--- technology tipping points and societal impact.~--- Geneva: WE Forum, 2015. 44~p. 
{\sf http://www3.weforum.org/docs/WEF\_GAC15\_ Technological\_Tipping\_Points\_report\_2015.pdf}.
\bibitem{20-zac}
\Au{Berman F., Rutenbar~R., Hailpern~B., Christensen~H., Davidson~S., Estrin~D., 
Franklin~M., Martonosi~M., Raghavan~P., Stodden~V., Szalay~A.\,S.} Realizing the potential of 
data science~// Commun.  ACM, 2018. Vol.~61. Iss.~4. P.~67--72. doi: 10.1145/3188721.

\bibitem{21-zac}
\Au{Stodden V.} The data science life cycle: A~disciplined approach to advancing data science as 
a~science~// Commun.  ACM, 2020. Vol.~63. Iss.~7. P.~58--66. doi: 10.1145/ 3360646.


\bibitem{23-zac} %22
\Au{Зацман И.\,М.} Научная парадигма информатики: классификация трансформаций 
объектов предметной об\-ласти~// Системы и~средства информатики, 2023. Т.~33. №\,4. 
С.~126--138. doi: 10.14357/08696527230412. EDN: ZIKUWO.

\bibitem{22-zac} %23
\Au{Зацман И.\,М.} Научная парадигма информатики: классификация объектов предметной  
об\-ласти~// Информатика и~её применения, 2023. Т.~17. Вып.~4. С.~96--103. doi: 
10.14357/19922264230413. EDN: FIUQAT.

\bibitem{24-zac}
\Au{Зацман И.\,М.} О~научной парадигме информатики: верхний уровень классификации 
объектов ее предметной об\-ласти~// Информатика и~её применения, 2022. Т.~16. Вып.~4. 
С.~73--79. doi: 10.14357/ 19922264220411. EDN: XZNKVI.

\bibitem{25-zac}
\Au{Соломоник А.\,Б.} Философия знаковых систем и~язык.~--- М.: ЛКИ, 2011. 408~с.
\bibitem{26-zac}
\Au{Зацман И.\,М.} Трансформация иерархии Акоффа в~научной парадигме информатики~// 
Информатика и~её применения, 2023. Т.~17. Вып.~3. С.~107--113. doi: 
10.14357/19922264230315. EDN: UMVRRV.

\bibitem{27-zac}
\Au{Zatsman I.} Building digital spiral models of knowledge generation~// 19th Forum 
(International) on Knowledge Asset Dynamics Proceedings.~--- Matera, Italy: Arts for Business 
Institute, 2024. P.~2185--2196.
\bibitem{28-zac}
\Au{Zatsman I.} Digital spiral model of knowledge creation and encoding its dynamics~// 18th 
Forum (International) on Knowledge Asset Dynamics Proceedings.~--- Matera, Italy: Arts for 
Business Institute, 2023. P.~581--596.
\bibitem{29-zac}
\Au{Зацман И.\,М.} Интерфейсы третьего порядка в~информатике~// Информатика и~её 
применения, 2019. Т.~13. Вып.~3. С.~82--89. doi: 10.14357/19922264190312. EDN: 
EHRQLF.

\bibitem{30-zac}
\Au{Зацман И.\,М.} Научная парадигма информатики как третьей культуры~//  
На\-уч\-но-тех\-ни\-че\-ская информация. Сер.~1: Организация и~методика информационной 
работы, 2023. №\,11. С.~1--14.

\end{thebibliography}

 }
 }

\end{multicols}

\vspace*{-9pt}

\hfill{\small\textit{Поступила в~редакцию 14.04.24}}

\vspace*{4pt}

%\pagebreak

%\newpage

%\vspace*{-28pt}

\hrule

\vspace*{2pt}

\hrule



\def\tit{OBJECT TRANSFORMATIONS OF~THE~FIRST AND~SECOND ORDER
IN~A~LEXICOGRAPHIC INFORMATION SYSTEM\\[-5pt]}


\def\titkol{Object transformations of~the~first and~second order
in~a~lexicographic information system}


\def\aut{I.\,M.~Zatsman}

\def\autkol{I.\,M.~Zatsman}

\titel{\tit}{\aut}{\autkol}{\titkol}

\vspace*{-13pt}


\noindent
Federal Research Center ``Computer Science and Control'' of the Russian Academy of Sciences, 
44-2~Vavilov Str., Moscow 119133, Russian Federation


\def\leftfootline{\small{\textbf{\thepage}
\hfill INFORMATIKA I EE PRIMENENIYA~--- INFORMATICS AND
APPLICATIONS\ \ \ 2024\ \ \ volume~18\ \ \ issue\ 2}
}%
 \def\rightfootline{\small{INFORMATIKA I EE PRIMENENIYA~---
INFORMATICS AND APPLICATIONS\ \ \ 2024\ \ \ volume~18\ \ \ issue\ 2
\hfill \textbf{\thepage}}}

\vspace*{2pt}



\Abste{The theoretical foundations of the design of information technologies used for 
the integration of bilingual dictionaries and parallel corpora are considered. The 
description of the first outcomes of the creation of the third\linebreak\vspace*{-12pt}}

\Abstend{ level of object 
transformations classification in the subject domain of informatics, which is supposed 
to be used
in creating the lexicographic information system providing integration, is 
given. All the entities of informatics are divided into two global classes: objects and 
their transformations. For each such class, its own classification is constructed. 
Previously, the two upper levels of the object transformation classification in the subject 
domain have been described. The present paper discusses the third level of this classification. The 
basis for the construction of its highest level was the division of the subject domain of 
informatics into media (mental, sensory, digital, and a~number of other media), each 
of which by definition includes objects of the same nature. The Solomonick's 
typology of sign systems served as the basis for constructing the second level of the 
object transformation classification. The aim of the paper is to systematize object 
transformations of the first and second orders at the third level of this classification. 
The basis for systematization is the medium version of the Ackoff's hierarchy.}

\KWE{subject domain objects; object transformations; classification; data; 
information; knowledge; lexicographic information system}


\DOI{10.14357/19922264240211}{VZTGVV}

\vspace*{-12pt}

\Ack

\vspace*{-3pt}


\noindent
The reported study was funded by the Russian Science Foundation, project  
No.\,24-18-00155, {\sf 
https://rscf.ru/project/24-18-00155}. The research was carried out using the infrastructure of the Shared 
Research Facilities ``High Performance Computing and Big Data'' (CKP 
``Informatics'') of FRC CSC RAS (Moscow) .
   


  \begin{multicols}{2}

\renewcommand{\bibname}{\protect\rmfamily References}
%\renewcommand{\bibname}{\large\protect\rm References}

{\small\frenchspacing
 {%\baselineskip=10.8pt
 \addcontentsline{toc}{section}{References}
 \begin{thebibliography}{99} 
\bibitem{1-zac-1}
\Aue{Aijmer, K., and B.~Altenberg.} 2013. \textit{Advances in corpus-based 
contrastive linguistics. Studies in honour of Stig Johansson}. Amsterdam: John 
Benjamins. 295~p. doi: 10.1075/scl.54.
\bibitem{2-zac-1}
\Aue{Dobrovolskiy, D.\,O., A.\,A.~Kretov, and S.\,A.~Sharov.} 2005. Korpus 
parallel'nykh tekstov [Corpus of parallel texts]. \textit{Nauchnaya i~tekhnicheskaya 
informatsiya. Ser. 2. Informatsionnye protsessy i~sistemy} [Scientific and Technical 
Information. Ser.~2: Information Processes and Systems] 6:16--27.
\bibitem{3-zac-1}
\Aue{Dobrovolskiy, D.\,O.} 2015. Korpus parallel'nykh tekstov i~sopostavitel'naya 
leksikologiya [The corpus of parallel texts and contrastive lexicology]. \textit{Trudy 
Instituta russkogo yazyka im. V.\,V.~Vinogradova} [Proceedings of the 
V.\,V.~Vinogradov Russian Language Institute] 6:413--449. EDN: VJQBHP.
\bibitem{4-zac-1}
\Aue{Goncharov, A.\,A., I.\,M.~Zatsman, and M.\,G.~Kruzhkov.} 2020. Evolyutsiya 
klassifikatsiy v~nadkorpusnykh ba\-zakh dannykh [Evolution of classifications in 
supracorpora databases]. \textit{Informatika i~ee Primeneniya~--- Inform. \mbox{Appl.}}  
14(4):108--116. doi: 10.14357/19922264200415.  
EDN: GKWBZT.
\bibitem{5-zac-1}
\Aue{Goncharov, A.\,A., I.\,M.~Zatsman, and M.\,G.~Kruzhkov.} 2021. 
Predstavlenie novykh leksikograficheskikh znaniy v~dinamicheskikh 
klassifikatsionnykh sistemakh [Representation of new lexicographical knowledge in 
dynamic classification systems]. \textit{Informatika i~ee Primeneniya~--- Inform. 
Appl.} 15(1):86--93. doi: 10.14357/19922264210112. EDN: OPEFXW.
\bibitem{6-zac-1}
\Aue{Zatsman, I.} 2020. Finding and filling lacunas in linguistic typologies. 
\textit{15th Forum (International) on Knowledge Asset Dynamics Proceedings}. 
Matera, Italy: Institute of Knowledge Asset Management. 780--793.
\bibitem{7-zac-1}
\Aue{Zatsman, I.} 2020. Three-dimensional encoding of emerging meanings in  
AI-systems. \textit{21st European Conference on Knowledge Management 
Proceedings}. Reading, U.K.: Academic Publishing International Ltd. 878--887.
\bibitem{8-zac-1}
\Aue{Ackoff, R.} 1989. From data to wisdom. \textit{J.~Applied Systems Analysis} 
16(1):3--9.
\bibitem{9-zac-1}
\Aue{Rosenbloom, P.\,S.} 2013. \textit{On computing: The fourth great scientific 
domain}. Cambridge, MA: MIT Press. 307~p.
\bibitem{10-zac-1}
\Aue{Rowley, J.} 2007. The wisdom hierarchy: Representations of the DIKW 
hierarchy. \textit{J.~Inf. Sci.} 33(2):163--180. doi: 10.1177/0165551506070706.
\bibitem{11-zac-1}
\Aue{Frick$\acute{\mbox{e}}$, M.\,H.} 2018.  
Data-Information-Knowledge-Wisdom (DIKW) pyramid, framework, continuum. 
\textit{Encyclopedia of big data}. Eds. L.~Schintler and C.~McNeely. Cham: 
Springer. 4~p. doi: 10.1007/978-3-319-32001- 4\_331-1.
\bibitem{12-zac-1}
\Aue{Denning, P., and P.~Rosenbloom.} 2009. Computing: The fourth great domain 
of science. \textit{Commun. ACM} 52(9):27--29.
\bibitem{13-zac-1}
\Aue{Denning, P., and P.~Freeman.} 2009. Computing's paradigm. \textit{Commun. 
ACM} 52(12):28--30. doi: 10.1145/ 1610252.1610265.

\bibitem{17-zac-1} %14
\Aue{Farradane, J.} 1980. Knowledge, information, and information science. 
\textit{J.~Inf. Sci.} 2(2):75--80. doi: 10.1177/ 01655515800020020.

\bibitem{15-zac-1}
\Aue{Shreyder, Yu.\,A.} 1988. Informatsiya i~znanie [Information and knowledge]. 
\textit{Sistemnaya kontseptsiya in\-for\-ma\-tsi\-on\-nykh protsessov} [System concept of 
information processes]. Moscow: VNIISI. 47--52.
\bibitem{16-zac-1}
\Aue{Ingwersen, P.} 1995. Information and information science. 
\textit{Encyclopedia of library and information science}. Eds. J.\,D.~McDonald and 
M.~Levine-Clark. New York, NY: Marcel Dekker Inc. 56(19):137--174.

\bibitem{14-zac-1} %17
Gilyarevskiy, R.\,S., ed. 2006. \textit{Informatika kak nauka ob informatsii: 
informatsionnyy, dokumental'nyy, tekh\-no\-lo\-gi\-che\-skiy, ekonomicheskiy, sotsial'nyy 
i~organizatsionnyy aspekty} [Informatics as information science: Informational, 
documentary, technological, economic, social, and organizational dimensions]. 
Moscow: FAIR-PRESS. 592~p.

\bibitem{18-zac-1}
\Aue{Hjorland, B.} 2000. Library and information science: Practice, theory, and 
philosophical basis. \textit{Inform. Process. Manag.} 36(3):501--531. doi:  
10.1016/S0306-\mbox{4573(99)00038-2}.
\bibitem{19-zac-1}
Deep shift~--- technology tipping points and societal impact. 2015. \textit{World Economic 
Forum}. Geneva. 44~p. Available at: {\sf 
http://www3.weforum.org/docs/WEF\_ GAC15\_Technological\_Tipping\_Points\_report\_2015.pdf} (accessed May~20, 
2024).
\bibitem{20-zac-1}
\Aue{Berman, F., R.~Rutenbar, B.~Hailpern, H.~Christensen, S.~Davidson, 
D.~Estrin, M.~Franklin, M.~Martonosi, P.~Raghavan, V.~Stodden, and 
A.\,S.~Szalay.} 2018. Realizing the potential of data science. \textit{Commun. ACM} 
61(4):67--72. doi: 10.1145/3188721.
\bibitem{21-zac-1}
\Aue{Stodden, V.} 2020. The data science life cycle: A~disciplined approach to 
advancing data science as a~science. \textit{Commun. ACM} 
 63(7):58--66. doi: 10.1145/3360646.

\bibitem{23-zac-1} %22
\Aue{Zatsman, I.\,M.} 2023. Nauchnaya paradigma informatiki: klassifikatsiya 
transformatsiy ob''ektov predmetnoy oblasti [Scientific paradigm of informatics: 
Transformation classification of domain objects]. \textit{Sistemy i~Sredstva 
Informatiki~--- Systems and Means of Informatics} 33(4):126--138. doi: 
10.14357/08696527230412. EDN: ZIKUWO.

\bibitem{22-zac-1} %23
\Aue{Zatsman, I.\,M.} 2023. Nauchnaya paradigma informatiki: klassifikatsiya 
ob''ektov predmetnoy oblasti [Scientific paradigm of informatics: Classification of 
domain objects]. \textit{Informatika i~ee Primeneniya~--- Inform. Appl.} 
 17(4):96--103. doi: 10.14357/19922264230413. EDN: FIUQAT.
 
\bibitem{24-zac-1}
\Aue{   Zatsman, I.\,M.} 2022. O nauchnoy paradigme informatiki: verkhniy uroven' 
klassifikatsii ob''ektov ee predmetnoy oblasti [On the scientific paradigm of 
informatics: The classification high level of its objects]. \textit{Informatika i~ee 
Primeneniya~--- Inform. Appl.} 16(4):73--79. doi: 10.14357/19922264220411. EDN: 
XZNKVI.
\bibitem{25-zac-1}
\Aue{Solomonick, A.\,B.} 2011. \textit{Filosofiya znakovykh system i~yazyk} 
[Philosophy of sign systems and language]. Moscow: LKI. 408~p.
\bibitem{26-zac-1}
\Aue{Zatsman, I.\,M.} 2023. Transformatsiya ierarkhii Akoffa v~nauchnoy 
paradigme informatiki [Transformation of the Ackoff's hierarchy in the scientific 
paradigm of informatics]. \textit{Informatika i~ee Primeneniya~--- Inform. \mbox{Appl.}} 
17(3):107--113. doi: 10.14357/19922264230315. EDN: UMVRRV.
\bibitem{27-zac-1}
\Aue{Zatsman, I.} 2024. Building digital spiral models of knowledge 
generation. \textit{19th Forum (International) on Knowledge Asset Dynamics 
Proceedings}. Matera, Italy: Arts for Business Institute. 2185--2196.
\bibitem{28-zac-1}
\Aue{Zatsman, I.} 2023. Digital spiral model of knowledge creation and encoding its 
dynamics. \textit{18th Forum (International) on Knowledge Asset Dynamics 
Proceedings}. Matera, Italy: Arts for Business Institute. 581--596.
\bibitem{29-zac-1}
\Aue{Zatsman, I.\,M.} 2019. Interfeysy tret'ego poryadka v~informatike 
 [Third-order interfaces in informatics]. \textit{Informatika i~ee Primeneniya~--- 
Inform. Appl.} 13(3):82--89. doi: 10.14357/19922264190312. EDN: EHRQLF.
\bibitem{30-zac-1}
\Aue{Zatsman, I.} 2023. Scientific paradigm of informatics as a~third culture. 
\textit{Scientific Technical Information Processing} 50(4):246--258. doi: 
10.3103/S0147688223040111. EDN: CKHMYS.

\end{thebibliography}

 }
 }

\end{multicols}

\vspace*{-6pt}

\hfill{\small\textit{Received April 14, 2024}} 


\vspace*{-12pt}


\Contrl

\vspace*{-3pt}

\noindent
\textbf{Zatsman Igor M.} (b.\ 1952)~--- Doctor of Science in technology, head of 
department, Federal Research Center ``Computer Science and Control'' of the 
Russian Academy of Sciences, 44-2~Vavilov Str., Moscow 119333, Russian 
Federation; \mbox{izatsman@yandex.ru}





\label{end\stat}

\renewcommand{\bibname}{\protect\rm Литература}      %8Abst+avt
\def\stat{kozerenko}

\def\tit{КОГНИТИВНО-ЛИНГВИСТИЧЕСКИЕ ПРЕДСТАВЛЕНИЯ 
В~СИСТЕМАХ ОБРАБОТКИ ТЕКСТОВ}

\def\titkol{Когнитивно-лингвистические представления 
в~системах обработки текстов}

\def\autkol{Е.\,Б.~Козеренко, И.\,П.~Кузнецов}
\def\aut{Е.\,Б.~Козеренко$^1$, И.\,П.~Кузнецов$^2$}

\titel{\tit}{\aut}{\autkol}{\titkol}

%{\renewcommand{\thefootnote}{\fnsymbol{footnote}}\footnotetext[1]
%{Работа выполнена при поддержке Российского фонда фундаментальных
%исследований, проект~10-01-00480. Статья написана на основе материалов доклада, 
%представленного на IV Международном семинаре <<Прикладные задачи теории вероятностей 
%и математической статистики, связанные с моделированием информационных систем>> 
%(зимняя сессия, Аоста, Италия, январь--февраль 2010 г.).}}

\renewcommand{\thefootnote}{\arabic{footnote}}
\footnotetext[1]{Институт проблем информатики Российской академии наук, kozerenko@mail.ru}
\footnotetext[2]{Институт проблем информатики Российской академии наук, igor-kuz@mtu-net.ru}


\Abst{Рассмотрены вопросы проектирования и развития 
семантико-синтаксических и лексико-семантических представлений в 
лингвистических процессорах ряда систем, основанных на аппарате расширенных 
семантических сетей (РСС). Системы этого класса создаются для извлечения знаний из 
текстов на естественных языках, отображения извлеченных сущностей и связей в 
структуры базы знаний (БЗ) и использования знаний для поддержки экспертных 
аналитических решений в различных сферах приложения. В~фокусе внимания 
находятся ин\-же\-нер\-но-линг\-ви\-сти\-че\-ские представления, позволяющие 
построить целостную работающую лингвистическую модель, которая 
модифицируется в зависимости от конкретной задачи: от <<тяжелой>> формы на 
основе детальных глубинных представлений до фокусных редуцированных 
оболочек, настроенных на узкую предметную область (ПО) и ограниченный язык 
общения. Особое внимание уделяется способам описания 
дис\-три\-бу\-тив\-но-транс\-фор\-ма\-ци\-он\-ных признаков языковых объектов.}

\KW{интеллектуальные системы; семантические представления; лингвистические 
процессоры; обработка естественного языка; извлечение знаний}

       \vskip 14pt plus 9pt minus 6pt

      \thispagestyle{headings}

      \begin{multicols}{2}

      \label{st\stat}

\section{Введение}

     Данная работа посвящена проблемам создания\linebreak 
     когни\-тив\-но-линг\-ви\-сти\-че\-ских моделей естественного языка для 
различных классов информационных систем и описанию опыта создания 
линг\-ви\-сти\-че\-ских представлений для интеллектуальных\linebreak технологий 
обработки текстов. Вопросы извлечения знаний из текстов и создания модели 
естественного языка рассматриваются в единстве. В центре внимания будут 
находиться лингвистические процессоры интеллектуальных систем, 
разработанных на основе аппарата \textit{расширенных семантических 
сетей}~[1--5]. %\cite{1koz}--\cite{3koz}, \cite{18koz}--\cite{19koz}. 
Будем 
называть их \textit{РСС-сис\-те\-мы}. Эти системы создавались коллективом 
разработчиков, включая авторов данной статьи в Институте проб\-лем 
информатики РАН на протяжении целого ряда лет в рамках 
исследовательских проектов и прикладных систем, ориентированных на 
конкретные ПО заказчиков. Можно выделить четыре 
поколения РСС-систем. Ко\-гни\-тив\-но-линг\-ви\-сти\-че\-ские 
представления, заложенные в основу систем этого класса, прошли 
определенный эволюционный путь. 
     
     Интеллектуальные РСС-сис\-те\-мы содержат развитые \textit{базы 
знаний}, при этом знания представлены в виде записей на языке 
РСС, называемых 
     \textit{РСС-струк\-ту\-ра\-ми}. Лингвистические знания, таким 
образом, являются частным случаем <<знаний>> и также представлены в 
виде записей на языке РСС. Основным 
конструктивным элементом РСС\linebreak является именованный $N$-мест\-ный 
предикат, на\-зы\-ва\-емый <<\textit{фрагментом}>>. Все множество языковых 
объектов задается в виде системы пре\-ди\-кат\-но-ак\-тант\-ных структур, при этом 
поддерживаются механизмы представления вложенных структур, что дает 
очень мощные изобразительные возможности для описания объектов 
различных языковых уровней. Очень важными факторами являются 
однородность и единообразие лингвистических представлений. 
     
     В процессе анализа и синтеза предложений естественного языка 
используется фор\-маль\-но-грам\-ма\-ти\-че\-ский аппарат, сходный с 
грамматиками зависимостей. При этом подходе опорными элементами 
служат слова и конструкции, выполняющие роль предикатов в предложении, 
и результатом анализа предложения должен стать один предикат, 
соответствующий сказуемому рассматриваемого предложения (т.\,е.\ 
основному глаголу в личной форме или другому основному предикатному 
выражению). Таким образом, в процессе анализа происходит выявление 
\textit{когнитивных опор} предложения: <<слов-дейст\-вий>> и 
     <<слов-от\-но\-ше\-ний>>, т.\,е.\ глаголов и других слов, имеющих 
синтактико-семантические валентности. Примером <<слов-от\-но\-ше\-ний>> 
могут служить, например, слова <<отец>>, <<друг>> и~т.\,п., т.\,е.\ в данном 
случае <<отношения>> (или \textit{функции}~--- в терминах языка логики 
предикатов 1-го порядка)~--- это слова, которые задают сильные, четко 
выраженные син\-так\-ти\-ко-се\-ман\-ти\-че\-ские ожидания. 
     
     Семантический анализ в ин\-же\-нер\-но-линг\-ви\-сти\-че\-ском 
понимании~--- это процесс перевода ес\-тест\-вен\-но-язы\-ко\-вых 
выражений во <<внутренние>> структуры БЗ, в 
рассматриваемой ситуации этими <<внутренними>> структурами являются 
записи на языке РСС. Таким образом, структуры БЗ~--- это код смысла в 
интеллектуальных информационных системах подобного рода. 
     
     В работе рассматриваются ин\-же\-нер\-но-линг\-ви\-сти\-че\-ские 
решения в системах с <<пол\-ным>> линг\-ви\-сти\-че\-ским анализом~--- это 
     сис\-те\-мы 1-го и 2-го поколения: ДИЕС1, ДИЕС2, 
     Логос-Д~\cite{2koz, 3koz}~--- и сис\-те\-мах с <<фактографическим>> 
подходом: интеллектуальных системах поддержки аналитических решений 
(ИСПАР)~\cite{18koz, 19koz}, где целью анализа является выделение 
сущностей и связей из текстов,~--- это системы 3-го и 4-го поколения. 

\section{Процесс концептуально-лингвистического моделирования 
в системах, основанных на аппарате расширенных семантических сетей}
     
\subsection{Центральные вопросы семантического моделирования} %2.1
     
     Концептуально-лингвистическое моделирование (КЛМ)~--- это 
процесс построения ес\-тест\-вен\-но-язы\-ко\-вой модели ПО (рис.~1), синтезирующий в себе подходы 
концептуального и лингвистического моделирования~[1--3]. 
По\-стро\-ение концептуально-лингвистической модели некоторой 
ПО подразделяется на следующие этапы:
     \begin{itemize}
     \item построение собственно концептуальной модели, т.\,е.\ вычленение 
базовых понятий, организация их в ро\-до-ви\-до\-вые деревья и определение 
связей между ними;
     \item разработка идеографического словаря ПО, т.\,е.\ 
лексическое наполнение концептуальной модели;
     \item ввод базовых правил, описывающих на естественном языке 
<<модель мира>>, релевантную данной ПО.
     \end{itemize}
     
     
     Методика КЛМ на 
основе аппарата РСС базируется на следующих принципах:
     \begin{itemize}
\item модель должна быть <<открытой>>, т.\,е.\ поддерживать эффективный 
механизм расширения и обновления информации;
\begin{center} %fig1
%\vspace*{3pt}
\hspace*{-10.7158pt}\mbox{%
\epsfxsize=77.871mm
\epsfbox{koz-1.eps}
}\hspace{10.7158pt}
%\end{center}
\vspace*{4pt}
%\begin{center}
{{\figurename~1}\ \ \small{Процесс КЛМ}}
\end{center}
\vspace*{3pt}

%\bigskip
\addtocounter{figure}{1}
\item модель представления <<смысла>> должна учитывать факты 
экстралингвистической реаль\-ности, которые в виде правил и отношений 
составляют некоторую базовую <<модель мира>>, достраиваемую 
конкретными моделями ПО;
\item модель должна быть практичной, т.\,е.\ не перегруженной детальными 
описаниями связей и отношений между понятиями, чтобы обеспечить 
возможность ее реализации, но в то же время отражать всю релевантную 
конкретной задаче информацию.
\end{itemize}

     \begin{figure*} %fig2
%     \begin{center}
\hspace*{23mm}\{(ВЫРАБАТЫВА895\_\_)(DICSEM)\\
\hspace*{23mm}COORD(PROGNOZ1,RUS,ВЫРАБАТЫВА895\_\_,S50\_31\_51\_20,\%)\\
\hspace*{23mm}SUB(UNIV,0+)~SUB(UNIV,1+)~SUB(UNIV,2+)\\
\hspace*{23mm}ВЫРАБАТЫВ(0-,1-,2-/3+)~INFI(3-)~ПРИДЕТСЯ(3-)~ПРИДЕТСЯ(3$-$/4+) \\
\hspace*{23mm}FUT1(4$-$)~SUB(СРЕД,5+)
%\end{center}
%\vspace*{2pt}
\Caption{Пример записи представления глагола <<вырабатывать>> в семантическом 
словаре
\label{f2koz}}
%\vspace*{6pt}
\end{figure*}

     Реалистичный подход к постановке задачи диктует необходимость 
ограничения моделируемого подмножества естественного языка. Суть 
ограничений сводится к следующему:
     \begin{enumerate}[(1)]
     \item анализируемые текстовые материалы содержат 
экспертные знания из конкретных ПО (в разработанных 
авторами системах это были такие ПО, как диагностика 
брака при изготовлении микросхем, социальное прогнозирование, 
криминалистика и другие);
     \item в целях максимально возможного устранения 
неоднозначности словарь строится по модульному принципу: есть некоторая 
наиболее общая часть (1--2~уровня), которая достраивается специальными 
словарями для каж\-дой отдельной~ПО.
     \end{enumerate}
     
     Предлагаемая модель лексической семантики основана на принципе 
<<ядерного>> значения, реализуемого в контексте данной 
ПО, с последующим индуктивным наращиванием других значений (если 
они актуализируются в рас\-смат\-ри\-ва\-емых контекстах). Также используется 
таксономия, которая реализуется в виде иерархических деревьев классов 
слов. 
     
     Общая <<модель мира>> системы является основой для моделей ПО. 
Элементами этой модели служат классы слов, которые подразделяются на 
понятия/имена, отношения, действия, свойства, характеристики действий, 
временные и пространственные характеристики.
     
     Самым общим понятием является \textit{концепт}, или 
\textit{универсальный класс}, который подразделяется на объект, ситуацию, 
процесс и~др. 
     
     Слова, относящиеся к классам действий и отношений, представлены 
как се\-ман\-ти\-ко-син\-так\-си\-че\-ские фреймы, задающие 
     пре\-ди\-кат\-но-ак\-тант\-ные структуры (модель управления). Однако 
в описываемом подходе (назовем его РСС-под\-хо\-дом) существенно 
расширена область значений актантов. Суть расширения состоит, во-первых, 
в том, что в роли актантов могут выступать не только простые объекты, 
соответствующие отдельным словам, но и структурные объекты, 
представляющие словосочетания и фразы, а во-вторых, в том, что понятие 
падежа включает в себя не только семантические, но и синтаксические 
признаки.
     
     Подход, основанный на РСС, позволяет отражать произвольный 
уровень вложенности структур за счет пропозициональных вершин 
семантической сети. Это обеспечивает представление\linebreak сложных 
синтаксических конструкций фраз\linebreak естественного языка, а также позволяет 
отразить\linebreak структурный характер лексической семантики,\linebreak которая в 
предлагаемой модели имеет иерар\-хи\-че\-ски-се\-те\-вую структуру. 
Линг\-ви\-сти\-че\-ские зна-\linebreak ния пред\-став\-ле\-ны в системном словаре и 
декла\-ра\-тивных модулях линг\-ви\-сти\-че\-ско\-го процессора.\linebreak В РСС-сис\-те\-мах 
так\-же реализована функция динамически форми\-ру\-емо\-го семантического 
словаря, который на основе исходной лингвистической информации 
достраивается системой автоматически в процессе об\-ра\-бот\-ки конкретных 
текстов. На рис.~\ref{f2koz} пред\-став\-ле\-но \mbox{такое} <<внутреннее>> описание 
глагола в семантическом словаре. Этот словарь автоматически генерируется 
РСС-системами ДИЕС2, ЛОГОС-Д, ИКС в процессе обработки 
     естест\-вен\-но-язы\-ко\-вых \mbox{текстов}. 
     {\looseness=1
     
     }
     
     
\subsection{Особенности применения аппарата расширенных семантических сетей 
в~когнитивно-лингвистическом моделировании} %2.2
     
     Дадим краткое описание аппарата РСС и  
обос\-ну\-ем выбор именно этого метода представления для моделирования 
естественного языка. Классическое понятие семантической сети сводится к 
следующему: задаются некоторые вершины, соответствующие объектам,  
вершины связываются дугами, которые помечаются именами отношений. 
Однако с помощью подобных сетей оказывается трудно представлять 
сложные виды информации, например, когда объекты, связанные 
отношениями, образуют агрегаты и когда отношения связываются между 
собой отношениями и~др. Поэтому в сети вводятся вершины, 
соответствующие именам отношений, а также специальный композиционный 
элемент, называемый вершиной связи. Вершина связи как бы <<разрывает>> 
дугу и подсоединяется одним ребром к вершине-отношению, а другими 
ребрами~--- к вершинам-объектам. Расширенная семантическая сеть является развитием такого сорта 
сетей в направлении повышения изобразительных возможностей при 
сохранении свойства однородности.
     
     Основой РСС является множество вершин ($V$), из которых 
составляются элементарные фрагменты (ЭФ) вида
     $
     V_0(V_1,V_2,\ldots ,V_k/V_{k+1})
     $, 
     где
$V_0, V_1, V_2,\ldots , V_k, V_{k+1}>0$.
     
     
     Такой фрагмент представляет $k$-местное отношение. Позиции 
вершин в ЭФ определяют их роли. 
Вершина~$V_0$ ставится в соответствие имени отношения, 
вершины~$V_1$, $V_2$, \ldots , $V_k$~--- объектам, участ\-ву\-ющим в 
отношении, а вершина~$V_{k+1}$, отделенная косой линией,~--- всей 
совокупности упомянутых объектов с учетом их отношения. В~дальнейшем 
будем $V_{k+1}$ называть $C$-вершиной ЭФ.\linebreak 
Множество ЭФ образует РСС. 
С~помощью РСС представляются наборы отношений, различные ситуации, 
сце\-нарии. Сильной стороной РСС-под\-хо\-да является возможность 
однородного пред\-став\-ле\-ния как предметной (концептуальной), так и 
лингвистической информации, что обеспечивает эффективную обработку 
знаний и поддержание непротиворечи\-вости~БЗ.
          \begin{figure*} %fig3
     \vspace*{1pt}
\begin{center}
\mbox{%
\epsfxsize=125.039mm
\epsfbox{koz-3.eps}
}
\end{center}
\vspace*{-9pt}
     \Caption{Семантико-синтаксический анализ без выявления глагольных 
словоформ
      \label{f3koz}}
\vspace*{12pt}
 %     \end{figure*}
%            \begin{figure*} %fig4
           \vspace*{1pt}
\begin{center}
\mbox{%
\epsfxsize=103.129mm
\epsfbox{koz-4.eps}
}
\end{center}
\vspace*{-9pt}
      \Caption{Целостная семантическая структура предложения
      \label{f4koz}}
      \end{figure*}

     
     Посредством РСС в БЗ представлены лингвистические  и 
предметные знания. Обработка этих знаний осуществляется 
продукциями языка ДЕКЛ, на котором реализованы сле\-ду\-ющие шесть 
блоков: морфологического анализа, семанти\-ческого анализа слов, 
син\-так\-ти\-ко-се\-ман\-ти\-че\-ско\-го анализа форм, 
прагматических функций, организации системной активности и 
обратный лингвистический процессор. С~помощью продукций 
осущест\-вля\-ет\-ся последовательное преобразование сети~--- РСС. При этом 
проходятся фазы, соответствующие уровню понимания входного текста. 
Рас\-смот\-рим~их.
     \begin{enumerate}[1.]
     \item На первом шаге анализа строится 
пространственная структура предложения с морфологической информацией 
для каждого слова.\linebreak Каж\-дый член предложения представляется вершиной 
семантической сети. Вместо слова генерируется код (если слово 
многозначно, т.\,е.\ принадлежит к нескольким классам,~--- то более одного 
кода). Основой кода служит корень слова. На этом этапе предложение 
представляется в виде набора фрагментов типа LRR (специальных меток 
результатов 1-го этапа анализа), объединяемых в целостную структуру 
посредством вершины связи. Результат 1-го этапа постоянно обращается к 
словарю: <<Что значит данное слово?>>
     \item На втором этапе каждой вершине сопоставляется семантический 
класс и присваивается новый код. За словами (т.\,е.\ конкретными вершинами 
РСС) система видит объекты, действия, свойства, т.\,е.\ строит 
классификации. Производится се\-ман\-ти\-ко-син\-так\-си\-че\-ский анализ 
без выявления глагольных словоформ, при этом предложение представляется 
в виде совокупности фрагментов типа SEM и SEMD~--- специальных меток 
результатов 2-го этапа анализа (рис.~\ref{f3koz}).
     \item На третьем этапе происходит частичное <<сворачивание>> 
синтаксических структур в более компактные (например, свойство объекта и 
сам объект) с присваиванием нового кода и строится фрагмент для объекта, 
обладающего этим свойством.
     \begin{figure*}[b] %fig5
          \vspace*{12pt}
\begin{center}
\mbox{%
\epsfxsize=147.485mm
\epsfbox{koz-5.eps}
}
\end{center}
\vspace*{-9pt}
     \Caption{Глубинная структура предложений
      \label{f5koz}}
      \end{figure*}      
     \item На четвертом этапе выявляются отношения и действия и 
производится анализ непосредственного контекста на соответствие заданным 
семантическим падежам. Система проверяет, подходят ли объекты 
(концепты, понятия) на аргументные места данного действия или отношения. 
При этом отглагольные существительные (<<делатель>>, т.\,е.\ агент 
действия, или <<делание>>~--- процесс~--- анализируются как слова с 
двойной природой: вначале как действия, а затем как объекты). Результатом 
этого этапа является целостная семантическая структура предложения, 
которая представляется фрагментом типа SEMSTR~--- метки результата 4-го 
этапа анализа (рис.~\ref{f4koz}).
     \item На пятом этапе происходит анализ прагматики: установление 
кореференциальных отношений, частичное восстановление эллиптических 
конструкций, система производит дальнейшие действия с построенными 
фрагментами.
     \end{enumerate}

     
Система ДИЕС допускает ввод полисемичных форм глаголов. Для этого следует 
воспользоваться формальной записью лингвистических знаний. 
     В~сис\-те\-мах, основанных на РСС, все функции реализованы на 
единой основе~--- в рамках языков РСС и ДЕКЛ, которые были разработаны 
с ориентацией на задачи обработки естественного языка.

%\vspace*{-6pt}

\section{Представление семантики глаголов, глубинные 
и~поверхностные структуры}
     
     В процессе анализа выявляются семантические вершины предложения: 
происходит выявление <<слов-дей\-ст\-вий>>, т.\,е.\ глаголов, и 
     <<слов-от\-но\-ше\-ний>>. Что же является конструктивной основой\linebreak 
задания семантических представлений предикатных слов и выражений? Как 
убедительно показано в работе~\cite{4koz}, семантика глагола 
определяется его дис\-три\-бу\-тив\-но-транс\-фор\-ма\-ци\-он\-ны\-ми\linebreak 
свойствами. Поэтому смысл предикатных выражений должен кодироваться с 
учетом их дистрибутивных и трансформационных признаков. 
     
     Выдвинутая рядом лингвистов (Хомский, Филлмор) гипотеза о том, что 
все предложения имеют глубинные и поверхностные 
     структуры~[7--10], явилась очень продуктивным 
источником проектных решений при создании первых РСС-сис\-тем и 
развивалась в дальнейшем. 

В~тео\-ре\-ти\-ко-линг\-ви\-сти\-че\-ском 
понимании глубинная структура~--- это абстракция, содержащая все 
элементы, необходимые для образования поверхностных структур 
предложений со сходной семантикой. 

     В~ин\-же\-нер\-но-линг\-ви\-сти\-че\-ском понимании\linebreak глубинная 
структура~--- это запись на языке БЗ, например на языке РСС, 
которая может быть представлена в <<поверхностном>> виде на одном из 
естественных языков в результате конечного числа определенных 
преобразований. Например, предложения

\noindent
\begin{align*}    
(1)\ &\mbox{\textit{The programmer writes the code}}\\
(2)\ &\mbox{\textit{The code is written by the programmer}}
\end{align*}
имеют истоком одну глубинную структуру:

\medskip

\noindent
     \begin{verbatim}
  Programmer <---- write ----> Code
      agent                   object,
\end{verbatim}

\medskip

\noindent
хотя и отличаются своими поверхностными структурами. В~каждом из них 
имеется агент (the programmer), объект (the code) и действие (write).\linebreak Согласно 
концепции \textit{падежной грамматики} Филлмора~\cite{5koz} глубинная 
структура для обоих предложений инвариантна. Эту структуру можно 
представить в виде скобочной записи $V(\mathrm{AGENT}, \mathrm{OBJECT})$. В~графическом 
виде глубинная структура предложения также может быть представлена 
диаграммой в виде дерева, где отражены инвариантные отношения 
зависимости между предикатной вершиной и актантами (рис.~\ref{f5koz}), 
причем в таком представлении явным образом разграничиваются 
\textit{модальность} (MOD) и \textit{пропозиция} (PROP).
     

     В исходном варианте~\cite{5koz} теория признавала шесть падежей: 
агентив, инструменталис, датив, объектив, локатив и фактитив. По мере 
развития теории~\cite{8koz} происходило увеличение числа падежей, однако 
<<умножение>> количества падежей утяжеляет первоначальную 
конфигурацию, поэтому при построении инженерных семантических 
представлений требуется некоторый <<компромиссный>> вариант, 
сочетающий в себе необходимую полноту, с одной стороны, и простоту и 
гибкость, с другой.

\begin{figure*}[b] %fig6
\vspace*{24pt}
\begin{center}
\mbox{%
\epsfxsize=156.873mm
\epsfbox{koz-6.eps}
}
\end{center}
%\vspace*{-9pt}
\Caption{Обобщенное функциональное представление систем ИСПАР
\label{f6koz}}
\end{figure*}
     
%\vspace*{-6pt}

\section{Некоторые базовые аспекты построения многоязычных 
систем}
     
     Одним из приоритетных направлений развития РСС-сис\-тем является 
обеспечение обработки текстов на нескольких языках, прежде всего для 
рус\-ско-анг\-лий\-ской языковой пары. В системах 2-го поколения~--- ДИЕС2, 
ИКС, ЛОГОС-Д были реализованы лингвистические процессоры и словари 
для русского и английского языка, позволявшие обрабатывать тексты для 
ряда ПО. При этом поддерживался как режим ввода 
лингвистических знаний линг\-вис\-том-ана\-ли\-ти\-ком, так и 
автоматический режим самообучения системы по вводимым \mbox{текстам}. 
{\looseness=1

}

Проводились также эксперименты с итальянским и французским языком. 
При создании многоязычных систем авторы обращались к европейским 
языкам. Очевидно, что европейские языки обладают большим числом общих 
правил, чем любой из них с языками других групп. Но при этом все 
естественные языки обладают общей структурой на самом глубинном 
уровне. На этом уровне располагаются главные элементы естественного 
языка: \textit{предложение}, \textit{модальность}, \textit{пропозиция}.
     
     Моделирование смысловых представлений~--- это процесс, 
развивающийся в направлении от поверхностных семантических структур к 
глубинным. Поиск такого внутреннего представления смысла в условиях 
многоязычной ситуации является на\-прав\-ле\-ни\-ем развития методов 
     КЛМ на базе  РСС. 
     
%     \vspace*{-48pt}
     
\section{Интеллектуальные системы поддержки аналитических 
решений}
     
Системы РСС 3-го и 4-го поколения на\-прав\-ле\-ны на извлечение знаний 
в виде \textit{объектов}, или \textit{сущностей}, и связей между ними из 
пред\-мет\-но-ориен\-ти\-ро\-ван\-ных текстов на русском и английском 
языке~\cite{18koz, 19koz}.

    
В настоящее время во всем мире активно ведутся работы по созданию 
систем извлечения фактов из текстов на естественных языках~[11--14], создаются развитые тезаурусы и 
онтологии~\cite{17koz}. Сис\-те\-мы РСС функционально шире, поскольку 
имеют возможность не только извлекать факты, но и поддерживать 
механизмы логического анализа и экспертного вывода на основе 
извлеченных знаний. Сис\-те\-ма\-ми такого рода являются ИСПАР. В~целом это 
направление исследований требует дальнейшей проработки 
     лек\-си\-ко-се\-ман\-ти\-че\-ских представлений, создания 
     пред\-мет\-но-ориен\-ти\-ро\-ван\-ных семантических словарей. 

Обобщенное функциональное представление систем ИСПАР дано на 
рис.~\ref{f6koz}. 
     
     В рамках ИСПАР на основе РСС 
(\mbox{ИСПАР}--РСС) были реализованы полномасштабные и\linebreak пилотные 
проекты для ряда ПО: криминалистики, управления 
кадрами, мониторинга финансово-экономического кризиса и 
др.~\cite{18koz, 19koz}.

\section{Применение аппарата расширенных семантических сетей в~лингвистических 
исследованиях}
     
     В настоящее время в рамках проектов, на\-прав\-лен\-ных на создание 
открытых лингвистических ресурсов~\cite{20koz} для 
     на\-уч\-но-прак\-ти\-че\-ских целей, ведутся работы по выравниванию 
параллельных текстов научных статей, патентов и 
     фи\-нан\-со\-во-эко\-но\-ми\-че\-ских текстов. В~качестве одного из 
методов выравнивания используется РСС-под\-ход, поскольку он позволяет 
отразить глу\-бин\-но-се\-ман\-ти\-че\-ский уровень языковых структур. 

На  рис.~7 представлен фрагмент первого этапа лингвистического 
анализа в многоязычных системах. Для <<идеальной>> ситуации, когда 
структуры исходного текста и текста перевода практически совпадают, такая 
ситуация имеет место в меньшинстве случаев. Основные трудности 
возникают при наличии переводческих трансформаций в параллельных 
текстах. Особое внимание следует уделять гла\-голь\-но-имен\-ным 
трансформациям, например явлению \textit{номинализации}, поскольку она 
очень продуктивна для всех исследовавшихся языков.

     
     Ключевой задачей при разработке методов сопоставления 
параллельных текстов является выявление и детальное описание тех 
языковых трансформаций, которые имеют место при переводе 
     естест\-вен\-но-язы\-ко\-вых конструкций с одного языка на 
другой~\cite{9koz}, потому что далеко не всегда некое содержание 
передается струк\-тур\-но-по\-доб\-ны\-ми средствами в текстах на разных 
языках. Сравнительное исследование употребления различных частей речи в 
параллельных текстах на разных языках создает основу для выявления и 
описания языковых транс-\linebreak

\begin{center} %fig7
\vspace*{3pt}
\mbox{%
\epsfxsize=79.726mm
\epsfbox{koz-7.eps}
}
\end{center}
\vspace*{4pt}
%\begin{center}
{{\figurename~7}\ \ \small{Первый этап анализа параллельных текстов ($W_n$
обозначает словоформу с номером~$n$, $1\leq n\geq 5$)}}
%\end{center}
%\vspace*{9pt}

%\bigskip
\addtocounter{figure}{1}
      

\noindent 
формаций, при этом центральной трансформацией
является \textit{номинализация}. Явление номинализации
было исследовано в 
ряде работ отечественных и зарубежных лингвистов~[17--20]. 
Ближе всего к правильному, по мнению авторов данной статьи, 
пониманию этого явления следующие определения номинализации: 
<<конструкции\ldots называются номинализованными~--- в том смысле, что 
их естественно рассматривать как результат номинализации конструкций с 
предикативным употреблением глаголов и прилагательных>>; 
<<номинализация~--- это синтаксический процесс, который соотносит 
предложения с именными группами>>~\cite{9koz, 10koz}. Выявление 
номинализованных конструкций в параллельных научных и патентных 
текстах на русском, английском, французском и немецком языках в научных 
и патентных текстах и сопоставительное описание гла\-голь\-но-имен\-ных 
межъязыковых трансформаций~--- одна из центральных задач 
     ин\-же\-нер\-но-линг\-ви\-сти\-че\-ских исследований. 
     
     Следующей базовой трансформацией в исследуемых текстах на 
нескольких европейских языках является адъек\-тив\-но-ад\-вер\-би\-аль\-ное 
преобразование. Это означает, что при переводе с одного языка на другой 
происходит синтаксическое преобразование имен прилагательных в наречия 
и обратное преобразование~--- наречий в прилагательные. Установление 
семантических соответствий между этими языковыми объектами также 
возможно осуществить посредством аппарата~РСС. 
     
     При семантическом выравнивании непараллельных текстов, имеющих 
одну и ту же денотативную составляющую, аппарат РСС позволяет выявить в 
текстах когнитивные опоры (слова с сильной валентностью~--- 
     <<сло\-ва-дейст\-вия>> и <<сло\-ва-от\-но\-ше\-ния>>) и установить 
между ними семантические соответствия.

\section{Заключение}

     В данной работе представлен опыт создания и развития 
     когни\-тив\-но-линг\-ви\-сти\-че\-ских пред\-став\-ле\-ний в 
интеллектуальных информационных сис\-те\-мах, разработанных на основе 
аппарата РСС. Аппарат РСС 
обеспечивает мощные изобразительные возможности для описания всех 
уровней естественного языка, включая уровень 
     глу\-бин\-но-се\-ман\-ти\-че\-ских представлений и межъязыковых 
соответствий. Конкретные лингвистические процессоры, которые были 
созданы на основе этого подхода, прошли определенный путь развития и 
позволили выработать проектные решения для основных задач текущего 
этапа~--- извлечения и обработки содержательных знаний из текстов на 
естественных языках и сопоставления языковых структур в текстах на 
различных языках с учетом базовых трансформаций.
     
     Проблема извлечения и обработки знаний открывает перспективы 
развития интеллектуальных направлений компьютерной лингвистики, 
поскольку ее основной акцент смещен в сторону\linebreak глубинных представлений 
языка, в которых используются как грамматические (морфологические и 
синтаксические), так и семантические атрибуты для описания языковых 
объектов. Проводи-\linebreak мые авторами исследования параллельных текстов 
направлены также на рассмотрение этой проблемы~\cite{20koz}. 
Центральное место в проводящихся линг\-ви\-сти\-че\-ских исследованиях 
занимает изучение и формализация процессов трансформации языковых 
структур, особенно все варианты глагольно-но\-ми\-на\-тив\-ных трансформаций, 
создание развитых дис\-три\-бу\-тив\-но-транс\-фор\-ма\-ци\-он\-ных 
описаний предикатых структур для рассматриваемых языков. 
     
     Для задач извлечения знаний и создания \mbox{ИСПАР} 
     дис\-три\-бу\-тив\-но-транс\-фор\-ма\-ци\-он\-ные описания имеют 
особое значение, поскольку таким образом задаются все возможные способы 
перевода языковых структур в пре\-ди\-кат\-но-ар\-гу\-мент\-ные 
представления, которые затем используются в процедурах обработки знаний.

{\small\frenchspacing
{%\baselineskip=10.8pt
%\addcontentsline{toc}{section}{Литература}
\begin{thebibliography}{99}

     \bibitem{1koz}
     \Au{Кузнецов~И.\,П.}
     Семантические представления.~--- М.: Наука, 1986. 290~с.
     
     \bibitem{2koz}
     \Au{Козеренко~Е.\,Б.}
     Кон\-цеп\-ту\-аль\-но-линг\-вис\-ти\-че\-ское моделирование в среде 
интеллектуального редактора знаний ИКС~// Проблемы проектирования и 
использования баз знаний.~--- Киев: Ин-т кибернетики им.\ В.\,М.~Глушкова, 
1992. C.~73--79.
     
     \bibitem{3koz}
     \Au{Kozerenko~E.\,B.}
     Multilingual processors: A unified approach to semantic and syntactic 
knowledge presentation~// Conference (International ) on Artificial Intelligence 
IC-AI'2001 Proceedings. Las Vegas, Nevada, USA. June 25--28, 2001.~--- Las 
Vegas: CSREA Press, 2001. P.~1277--1282.

     \bibitem{18koz} %4
     \Au{Kuznetsov~I.\,P., Efimov~D.\,A., Kozerenko~E.\,B.}
     Tools for tuning the semantic processor to application areas~// ICAI'09 
Proceedings, WORLDCOMP'09. July 13--16, 2009. Las Vegas, Nevada, USA. 
Vol.~I.~--- Las Vegas: CRSEA Press, 2009. P.~467--472.
     
     \bibitem{19koz} %5
     \Au{Kuznetsov~I.\,P., Kozerenko~E.\,B., Kuznetsov~K.\,I., 
Timonina~N.\,O.}
     Intelligent system for entities extraction (ISEE) from natural language 
texts~// Workshop (International) on Conceptual Structures for Extracting Natural 
Language Semantics (Sense'09) at the 17th Conference 
(International ) on Conceptual Structures (ICCS'09) Proceedings. University Higher School of 
Economics. Moscow, Russia, 2009. P.~17--25.
     
     \bibitem{4koz} %6
     \Au{Апресян~Ю.\,Д.}
     Экспериментальное исследование семантики русского глагола.~--- М.: 
Наука, 1967.  252~с.
     
     \bibitem{5koz} %7
     \Au{Филлмор~Ч.}
     Дело о падеже~// Новое в зарубежной линг\-вистике, 1968. Вып.~X. С.~369--495.
     
     \bibitem{6koz} %8
     \Au{Хомский~Н.}
     Аспекты теории синтаксиса.~--- М.: МГУ, 1972.
     
     \bibitem{7koz} %9
     \Au{Хомский Н.}
     Язык и мышление.~--- М.: МГУ, 1972.
     
     
     \bibitem{8koz} %10
     \Au{Fillmore~C.}
     The case for case reopened~// Syntax and Semantics. Vol.~8.~--- N.Y.: 
Academic Press, 1977. 
     

          \bibitem{15koz} %11
     FASTUS: A cascaded finite-state trasducer for extracting information from 
natural-language text~// AIC, SRI International, Menlo Park, California, 1996. 
     
     \bibitem{16koz} %12
     \Au{Han~J., Pei~Y., Mao~R.}
     Mining frequent patterns without candidate generation: A frequent-pattern 
tree approach~// Data Mining and Knowledge Discovery, 2004. Vol.~8. No.\,1. 
P.~53--87.
     
     
     \bibitem{13koz} %13
     \Au{Cunningham~H.}
     Automatic information extraction~// Encyclopedia of Language and 
Linguistics. 2nd ed.~--- Elsevier, 2005.
     
     \bibitem{14koz} %14
     \Au{Han~J., Kamber~M.}
     Data mining: Concepts and techniques.~--- Morgan Kaufmann, 2006.
     
     
     \bibitem{17koz} %15
     \Au{Добров~Б.\,В., Лукашевич~Н.\,В.}
     Онтологии для автоматической обработки текстов: Описание понятий 
и лексических значений~// Компьютерная лингвистика и интеллектуальные 
технологии: Тр. межд. конф. <<Диалог'06>>. Бекасово, 31~мая\,--\,4~июня 
2006. С.~138--142.

     \bibitem{20koz} %16
     \Au{Kozerenko~E.\,B.}
     INTERTEXT: A multilingual knowledge base for machine translation~// 
Conference (International) on Machine Learning, Models, Technologies and 
Applications Proceedings. June 25--28, 2007. Las Vegas, USA.~--- Las Vegas: 
CSREA Press, 2007. P.~238--243.

     \bibitem{9koz} %17
     \Au{Жолковский~А.\,К., Мельчук~И.\,А.}
     О семантическом синтезе~// Проблемы кибернетики, 1967. Вып.~19.
     
         
     \bibitem{11koz} %18
     \Au{Jacobs~R.\,A., Rosenbaum P.\,S.}
     English transformational grammar.~--- Blaisdell, 1968.
     

\label{end\stat}
     
          \bibitem{12koz} %19
     \Au{Балли~Ш.}
     Общая лингвистика и вопросы французского языка. 2-е изд.~--- М.: 
УРСС, 2001.

\bibitem{10koz} %20
     \Au{Падучева~Е.\,В.}
     О~семантике синтаксиса: Мат-лы к трансформационной 
грамматике русского языка. 2-е изд.~--- М: КомКнига, 2007.  296~с. 
     
 \end{thebibliography}
}
}


\end{multicols}   %9Abst+avt
\def\stat{morozova}

\def\tit{ТРАНСФОРМАЦИОННЫЕ МОДЕЛИ ЯЗЫКОВЫХ СТРУКТУР 
ДЛЯ~ФРАНЦУЗСКО-РУССКОГО МАШИННОГО ПЕРЕВОДА}

\def\titkol{Трансформационные модели языковых структур 
для~французско-русского машинного перевода}

\def\autkol{Ю.\,И.~Морозова}
\def\aut{Ю.\,И.~Морозова$^1$}

\titel{\tit}{\aut}{\autkol}{\titkol}

%{\renewcommand{\thefootnote}{\fnsymbol{footnote}}\footnotetext[1]
%{Работа поддержана Российским фондом фундаментальных исследований
%(проекты 11-01-00515а и 11-07-00112а), а также Министерством
%образования и науки РФ в рамках ФЦП <<Научные и
%научно-педагогические кадры инновационной России на 2009--2013~годы>>.}}


\renewcommand{\thefootnote}{\arabic{footnote}}
\footnotetext[1]{Институт проблем информатики Российской академии наук, yulia-ipi@yandex.ru}


\Abst{Данная работа посвящена актуальным проблемам исследования 
трансформационных свойств языковых объектов при переводе предикативных 
структур с французского языка на русский. Основное внимание уделено 
изменению категориальной принадлежности и изменению грамматических 
характеристик предикатных слов при переводе. Материалом исследования 
послужили патентные тексты на французском языке и их переводы на русский 
язык, выполненные спе\-ци\-а\-ли\-ста\-ми-пе\-ре\-вод\-чи\-ками. }

\KW{французско-русский автоматический перевод; функциональная семантика; 
языковые трансформации; вершинные грамматики}

 \vskip 14pt plus 9pt minus 6pt

      \thispagestyle{headings}

      \begin{multicols}{2}
      
            \label{st\stat}

\section{Введение}

Данное исследование направлено на исследование предикатных фразовых 
структур на основе вершинных грамматик применительно к задачам 
моделирования машинного перевода и извлечения знаний из текста для 
французско-русского на\-прав\-ле\-ния. Основной задачей ставилось создание 
унифицированной модели функциональных значений синтаксем, в которой 
бы учитывались сдвиги значений, производимые переводческими 
трансформациями. Предикатные слова являются вершинами синтаксической 
структуры предложения, а также вершинами внутреннего представления 
знаний в структурах баз знаний, поэтому описание их дистрибутивных и 
трансформационных свойств имеет первостепенное значение.

Исследования ведутся в рамках проекта по созданию многоязычного 
лингвистического процессора для задач машинного перевода и извлечения 
знаний из текстов, разрабатываемого на основе 
функ\-ци\-о\-наль\-но-се\-ман\-ти\-че\-ско\-го подхода~[1]. В~качестве 
материала исследования были использованы фрагменты параллельных 
текстов патентов, содержащие предикатные выражения. Модель перевода с 
учетом трансформаций для рус\-ско-фран\-цуз\-ской\linebreak языковой пары основана 
на многовариантной когнитивной трансферной грамматике (МКТГ), 
разработанной Е.\,Б.~Козеренко~[1--4]. Данный формализм имеет 
определенные черты грамматики\linebreak составляющих и вершинной грамматики 
HPSG (Head-driven phrase structure grammar)~[5]. 
Преимущество данного формализма заключается в том, что он 
позволяет описывать как отношения линейного порядка, так и отношения 
зависимости в рамках одной и той же фразовой структуры. Формализмы, 
основанные на порождающей грамматике Хомского и на вершинной 
грамматике HPSG, широко применяются для создания сис\-тем 
автоматической обработки текстов на английском языке и других 
европейских языках (французском, испанском, немецком, чешском). Однако 
возможности применения данных формализмов для автоматической 
обработки русского языка изучены недостаточно. 
     
\section{Грамматические формализмы, используемые для~создания 
лингвистических процессоров}
      
     Для формального описания синтаксиса естест\-вен\-ных языков 
применительно к задачам автоматической обработки языка наиболее часто 
использу\-ются следующие виды формализмов: регулярные грамматики, 
     кон\-текст\-но-сво\-бод\-ные грамматики,\linebreak мягко 
     кон\-текст\-но-за\-ви\-си\-мые грамматики. Регулярные грамматики не 
могут быть использованы для полноценного описания синтаксиса. Данный\linebreak 
формализм используется только для частичного синтаксического анализа 
предложений (shallow parsing). С~помощью кон\-текст\-но-сво\-бод\-ных 
грамматик можно описать большинство предложений естественного языка, 
однако грамматики данного класса не позволяют описывать предложения с 
разрывными структурами. Наконец, мягко кон\-текст\-но-за\-ви\-си\-мые 
грамматики являются наиболее мощным формализмом и позволяют 
описывать любые виды предложений естественных языков, однако их 
применение в сис\-те\-мах автоматической обработки естественного языка 
связано с большими\linebreak
 вычислительными затратами. Для описания явлений 
естественных языков применяются, в основном, кон\-текст\-но-сво\-бод\-ные 
грамматики с некоторыми расширениями. Однако вопрос о выборе наиболее 
адекватного формализма для описания синтаксиса естественных языков и 
создания линг\-ви\-сти\-че\-ских процессоров остается открытым. Во многих 
сис\-те\-мах автоматической обработки текс\-тов на английском языке 
используются модернизированные грамматики Н.~Хомского~\cite{6-mor}. 
     
     Для создания сис\-тем автоматической обработки текстов на языках с 
богатой морфологией и относительно свободным порядком слов часто 
используется вершинная грамматика HPSG, разработанная Карлом Поллардом и Иваном Сагом~\cite{5-mor}. 
Согласно данной теории описание грамматики языка должно состоять из 
очень подробного словаря и очень небольшого количества грамматических 
правил, носящих универсальный характер. Словарь имеет хорошо 
проработанную иерархическую структуру, которая характеризуется 
наследованием свойств по умолчанию. В~словарном описании слов, которые 
могут являться вершинами синтаксических групп (существительных, 
глаголов, предлогов, прилагательных) есть поле HEAD, в котором 
описываются такие важные с точки зрения синтаксического поведения слова 
свойства, как часть речи, признаки согласования, форма, предикативность и~др. 
Данные свойства передаются группам, порождаемым данными 
вершинами, в соответствии с правилами грамматики. Таким образом, 
процесс порождения правильно построенных предложений определяется 
свойствами вершин (отсюда название Head-driven phrase structure grammar). 
Одним из основных понятий HPSG является структура свойств (feature 
structure). Это набор атрибутов с их значениями, например словарное 
описание лексемы задается следующей структурой свойств: [PHON$\ldots$ 
SYN$\ldots$ SEM$\ldots$ ARG-ST$\ldots$].
     
     Значение каждого из свойств может пред\-став\-лять собой как единый 
элемент, так и структуру свойств, например свойство AGR (согласование) 
имеет следующую структуру: AGR [PER$\ldots$, NUM$\ldots$, 
GEND$\ldots$].
     
     Унификационный механизм, использующийся в грамматике HPSG, 
позволяет объединить два описания структур свойств. Результатом данной 
операции является структура свойств, содержащая информацию из обеих 
структур. Механизм унификации используется при проверке согласования 
морфологических характеристик слов, необходимой для включения их в одну 
синтаксическую группу. Вершинная грамматика HPSG была с успехом 
применена при создании сис\-тем автоматической обработки текстов на 
разных языках (английском, французском, чешском). Один из примеров такой 
сис\-те\-мы~--- грамматика английского языка \mbox{LinGO} English Resource Grammar 
(ERG) и синтаксически аннотированный корпус Redwoods, размеченный 
автоматически с использованием грамматики ERG~\cite{7-mor}.
     
     В работе~\cite{8-mor} предлагается формализм, име\-ющий некоторые 
черты универсальной грамматики Хомского, вершинной грамматики HPSG и 
лек\-сико-функ\-ци\-о\-наль\-ной грамматики LFG
(Lexical functional grammar). С~использованием 
данного\linebreak формализма была реализована программа синтаксического анализа 
русского языка, которая строит структуру предложения в двух аспектах: как 
структуру составляющих и как функциональную структуру. Для описания 
согласования морфологических характеристик применяется аппарат 
унификации, используемый в вершинной грамматике HPSG. В~данной 
работе обосновывается возможность применения грамматик составляющих с 
различными модификациями для автоматической обработки текстов на 
русском языке (в частности, для перевода с русского языка на другой язык).

\section{Современные подходы к~проблеме машинного перевода}
     
     В области машинного перевода существуют два основных направления 
исследований~--- подход на основе правил и статистический подход. 
Системы, созданные в рамках подхода на основе правил, включают в себя 
компоненты, отвечающие за последовательный морфологический, 
синтаксический и семантический анализ предложений исходного языка и 
синтез предложений целевого языка (с прохождением тех же уровней). 
Создание таких сис\-тем требует многолетней кропотливой работы 
лингвистов, так как для функционирования сис\-те\-мы необходим словарь с 
подробными син\-так\-ти\-ко-се\-ман\-ти\-че\-ски\-ми описаниями словарных 
единиц и правила анализа и синтеза предложений (морфологического, 
синтаксического и семантического уров\-ней). Достоинствами данного 
подхода являются высокое качество перевода, соответствие теоретическим 
концепциям и возможность удобного внесения изменений. Недостатками 
являются большие трудозатраты для создания словарей и сис\-тем правил 
перевода. 
     
     Статистический подход заключается в выявлении закономерностей 
перевода путем автоматического анализа параллельных текстов с 
использованием методов математической статистики и без использования 
лингвистических знаний. Достоинством статистического подхода является 
быстрота создания подобных сис\-тем. Для того чтобы сис\-те\-ма начала 
работать, необходим лишь текстовый корпус, переводческий словарь 
(возможно, неполный) и словарь основ. По данным из~\cite{9-mor}, 
требуется всего несколько часов для того, чтобы сис\-те\-ма начала работать, и 
1--2~недели, чтобы настроить ее и получать приемлемые результаты. 
Недостатком данного подхода является необходимость использования 
больших параллельных корпусов (от~1~млн слов) для получения 
удовлетворительных результатов перевода. Не для всех языковых пар 
существуют такие большие текстовые коллекции. Если же использовать не 
только статистические методы, но добавить и частичную лингвистическую 
разметку, размер корпуса можно существенно уменьшить (с~1~млн до 
300~тыс.\ слов)~\cite{10-mor}.
     
     Современный период развития исследований и разработок в области 
машинного перевода и сис\-тем извлечения знаний из текстов характеризует-\linebreak ся 
интенсивным процессом <<гибридизации>> подходов и моделей. Создатели 
сис\-тем, основанных на правилах, вводят в правила различные стохастические 
модели, которые позволяют отобразить\linebreak
 динамику и разнообразие языковых 
форм и значений, порождаемых в процессе речевой дея\-тель\-ности, а 
сторонники статистических методов построения лингвистических моделей 
все чаще\linebreak обращаются к подходам, основанным на лингвистических знаниях, 
рассматривая их как средства <<интеллектуализации>> сис\-тем. В~настоящее 
время появляется все больше исследований в рамках синергетического 
подхода, использующего лингвистические знания, статистические методы и 
механизмы машинного обучения~\cite{3-mor}. Как пишут авторы~\cite{11-mor}, 
<<наше убеждение состоит в том, что в долгосрочной перспективе 
самые эффективные технологии машинного перевода объединят в себе 
преимущества обоих подходов>>. По мнению авто-\linebreak ров~\cite{11-mor}, подход на основе 
правил следует применять для анализа тех уровней языка, для которых\linebreak 
существуют детальные лингвистические теории, описывающие по\-дав\-ля\-ющее 
большинство случаев, в то время как статистический подход следует 
применять для извлечения лексической и предметно-ори\-ен\-ти\-ро\-ван\-ной 
лингвистической информации, для которой пока что не существует 
разработанной теории. 
     
     Наиболее перспективными направлениями в области статистического 
машинного перевода являются перевод цепочек слов (phrase-based translation) 
и синтаксический перевод (syntax-based translation). При использовании 
метода перевода цепочек слов сопоставлению и переводу подвергаются 
цепочки слов (обычно не длиннее трех слов), выделенные путем применения 
статистических методик. Они не всегда совпадают со словосочетаниями в 
традиционном лингвистическом понимании (группа слов, взаимосвязанных 
синтаксически и семантически). Например, группа слов <<\textit{in 
accordance with the}>> является цепочкой слов, подлежащей переводу, в 
рамках статистического машинного перевода, но не является 
словосочетанием в лингвистическом смысле. При синтаксическом переводе 
сопоставлению и переводу подвергаются синтаксические поддеревья, а не 
конкретные слова или словосочетания.
     
\section{Создание системы правил для~русско-французского 
машинного перевода}

     В работе~\cite{12-mor} описывается сис\-те\-ма перевода с английского 
на французский язык, сочетающая в себе традиционный правиловый 
подход и статистический подход~--- перевод цепочек слов с использованием 
соответствий, извлеченных из параллельного текстового корпуса. В~качестве 
цепочек слов авторы предлагают использовать не любые последовательности 
слов, а только синтаксически мотивированные, другими словами, из текста 
извлекаются именные, глагольные группы, группы прилагательных и 
наречий. При выборе наилучшего варианта перевода цепочки слов 
предпочтение отдается цепочке слов, имеющей ту же самую синтаксическую 
категорию, т.\,е.\ в качестве перевода для именных групп используются 
именные группы и~т.\,д. Французско-русское направление машинного 
перевода в нашей стране развивается с самого начала исследований по 
машинному переводу. Первыми появились экспериментальные сис\-те\-мы 
фран\-цуз\-ско-рус\-ско\-го автоматического перевода ФРАП 
     (1976--1986~гг.)~\cite{13-mor} и \mbox{ЭТАП-1} (1985~г.)~\cite{14-mor}. Эти 
сис\-те\-мы были основаны на последовательном морфологическом, 
синтаксическом и семантическом анализе предложений исходного языка с 
последующим синтезом предложений целевого языка (с прохождением тех 
же уровней). В~сис\-те\-мах использовались словари с подробными 
     син\-так\-ти\-ко-се\-ман\-ти\-че\-ски\-ми описаниями слов и сис\-те\-мы 
правил анализа и синтеза предложений естественного языка.
     
     В 1990-е~гг.\ появилась первая коммерческая сис\-те\-ма автоматического 
перевода фран\-цуз\-ско-рус\-ско\-го направления \mbox{ПРОМТ}. В~основу 
архитектуры сис\-тем было положено представление процесса перевода как 
процесса с объект\-но-ориен\-ти\-ро\-ван\-ной организацией, основанной на 
иерархии обрабатываемых компонентов предложения. В~сис\-те\-мах работают 
сетевые грамматики, близкие по типу к расширенным сетям переходов, а 
также процедурные алгоритмы заполнения и трансформаций фреймовых 
структур для анализа сложных предикатов~\cite{15-mor}.
     
     Качество перевода в современных сис\-те\-мах машинного перевода 
фран\-цуз\-ско-рус\-ско\-го направления достигло высокого уровня, однако 
многие особенности синтаксиса русского языка, а также \mbox{многие} типы 
регулярных трансформаций, происходящих при переводе с русского языка на 
французский, остаются неучтенными в этих сис\-темах. 
     
     В рамках описываемых проектов разрабатывается сис\-те\-ма правил 
трансфера синтаксических структур, учитывающая возможность 
синтаксических трансформаций при переводе и многовариантность перевода. 
Козеренко была разработана многоязычная семантическая грамматика 
русского и английского языков для задач автоматической обработки 
текстов~--- МКТГ~[1--4]. Данная грамматика является разновидностью 
уни\-фи\-ка\-ци\-он\-но-по\-рож\-да\-ющей грамматики. Многовариантные правила 
функ\-ци\-о\-наль\-но-се\-ман\-ти\-че\-ско\-го переноса фразовых структур 
задают алгоритм перевода с одного языка на другой, причем учитывается 
вероятность каждого из вариантов перевода. Функциональные\linebreak значения 
языковых единиц отражены в рас\-ши\-ренной сис\-те\-ме ка\-те\-го\-ри\-аль\-но-функ\-ци\-о\-наль\-ных 
ат\-рибутов. Структуры атрибутов и значений и правила
их преобразования задаются в виде кон\-текст\-но-сво\-бод\-ных и мягко 
     кон\-текст\-но-за\-ви\-си\-мых продукционных правил. Отношения 
зависимости реализуются через механизм головных вершин фразовых 
структур, а сами фразовые структуры задают линейные последовательности 
языковых объектов. Лингвистический процессор сегментирует входные 
предложения на фразовые структуры и осуществляет трансфер этих структур 
в соответствующие им структуры целевого языка. Сегментация фразовых 
структур входного предложения проводится с учетом смысла структур, 
который при переводе должен быть передан средствами целевого языка. 
Задачей проведенных исследований ставилось создание сис\-те\-мы правил 
многовариантного трансфера для перевода с русского языка на французский.
     
     С точки зрения синтаксической структуры предложения русский и 
французский языки очень сильно отличаются друг от друга. Во французском 
языке большинство предложений двусоставны, т.\,е.\ и подлежащие, и 
сказуемые выражены на поверхностном уров\-не, причем сказуемое всегда 
выражается личной формой глагола. В~рус\-ском языке кроме канонической 
структуры <<Подлежащее (выраженное существительным в именительном 
падеже)\;+\;сказуемое (выраженное личным глаголом)>> возможны также 
другие синтаксические струк-\linebreak туры:
       
       $\bullet$~В предложении отсутствует сказуемое, выраженное глаголом в личной 
форме. Сказуемое выражено кратким причастием, кратким или полным прилагательным, 
существительным, предложной группой, инфинитивом и~др.
     
     \smallskip
     
     \noindent
     \textbf{Примеры}:
       
     \textit{Дом красив} (краткое прилагательное). \textit{Дом построен} 
(краткое причастие). \textit{Пьер~--- учащийся} (существительное).
       
       \smallskip
     
     Также к данному классу относятся случаи назывных предложений, 
состоящих из одного подлежащего, выраженного существительным в 
именительном падеже (например, заголовки) и случаи эллипсиса, когда 
глагол в личной форме <<подразумевается>>, но не выражен в 
поверхностной структуре предложения. Приведем пример эллипсиса из 
текста научного патента:
       
     \textit{Рисунки~1--4 ИЗОБРАЖАЮТ продольный разрез половины 
детали различных вариантов, соответствующих выполнению зубного 
штифта согласно первому варианту осуществления изобретения} (полная 
структура).
       
     \textit{Рисунки 5 и 6~--- продольный разрез половины детали 
моноблочного компонента протеза} (структура с эллипсисом).
     
     Подобные структуры являются трудными для синтаксического анализа 
и перевода на французский язык, так как при переводе требуется вставить 
пропущенный глагол в личной форме (глагол \textit{быть} или другой 
глагол). Многие из структур данного вида в существующих сис\-те\-мах 
автоматического перевода с русского языка на французский язык 
обрабатываются некорректно.

$\bullet$~В предложении отсутствует подлежащее, выраженное 
существительным в именительном падеже. К~данному типу предложений 
относятся безличные предложения, не\-опре\-де\-лен\-но-лич\-ные 
предложения, опре\-де\-лен\-но-лич\-ные предложения, инфинитивные 
предложения. 

\smallskip

\noindent
\textbf{Примеры:}
       
     \textit{Мне нравится работать} (безличное предложение).
       
     \textit{Маше подарили книгу} (не\-опре\-де\-лен\-но-лич\-ное 
предложение).
       
     \textit{Еду в кино} (опре\-де\-лен\-но-лич\-ное предложение).
       
     \textit{Нам бы сессию сдать} (инфинитивное предложение).
       
       \smallskip
       
     Также к данному классу относятся предложения, в которых 
подлежащее выражено инфинитивом. 
     
     \smallskip
     
     \noindent
     \textbf{Пример:}
       
     \textit{Курить~--- здоровью вредить.}
       
       \smallskip
       
     Предложения с подобной синтаксической структурой также являются 
источником значительных трудностей при автоматическом анализе и 
переводе, так как французский язык требует, чтобы в каждом предложении 
было подлежащее, выраженное существительным или местоимением без 
предлогов (функционально соответствует существительному в именительном 
падеже в русском языке). Следовательно, при переводе предложения 
русского языка, в котором нет подлежащего, выраженного существительным 
в именительном падеже, требуется восстановить подлежащее, используя 
<<формальное>> подлежащее (безличное местоимение \textit{il}) или личное 
местоимение. 
     
     Чтобы обосновать описание всех типов предложений в создаваемой 
сис\-те\-ме правил многовариантного перевода, было проведено исследование 
частоты встречаемости предложений различных типов в текстах научных 
патентов. Предложения были разделены на 3~класса.
       \begin{description}
     \item[Класс 1.] Подлежащее (выраженное существительным в 
именительном падеже)\;+\;сказуемое (выраженное глаголом в личной форме).
     \item[Класс 2.] Сказуемое, выраженное глаголом в личной форме, 
отсутствует. 
     \item[Класс 3.] Подлежащее, выраженное существительным в 
именительном падеже, отсутствует.
     \end{description}
     
     Некоторые предложения относятся одновременно и к классу~2, и к 
классу~3. Такие предложения были отнесены к классу~2.
     
     В результате распределения предложений по груп\-пам и подсчета 
относительной частоты\linebreak встречаемости предложений каждого вида в текс\-тах 
на\-уч\-ных патентов были получены следующие\linebreak результаты:
     класс~1~--- 48\%; класс~2~--- 38\%;\linebreak класс~3~--- 14\%.
     
     Как видно, предложения, относящиеся к каж\-до\-му из трех классов, 
встречаются в текстах с достаточно большой частотой, и существует 
необходимость включения правил для всех трех классов в сис\-те\-му 
многовариантного трансфера.
       
     Кроме различия структурных типов предложений французский и 
русский языки также очень существенно различаются между собой в аспекте 
порядка слов в предложении. Во французском языке большинство 
предложений имеют канонический порядок слов:
       
\begin{center}
 \textit{Подлежащее\,--\,сказуемое\,--\,прямое дополнение\,--\,косвенные 
дополнения}.
\end{center}
     
     В русском языке данный порядок регулярно нарушается как в устной, 
так и в письменной речи. Несоответствие порядка слов в предложениях на 
русском и французском языке создает необходимость изменения порядка 
слов при переводе. 
     
     При создании сис\-те\-мы правил перевода будем использовать 
классификации предложений русского языка, изложенные в учебниках по 
русскому синтаксису~\cite{16-mor, 17-mor}, а также типичные переводческие 
трансформации, описанные в учебниках по переводу с французского языка 
на русский~\cite{18-mor}.
     
     Рассмотрим наиболее частотные типы предложений русского языка, 
которые создают трудности при переводе, на материале патентных текстов.

$\bullet$~Безличные предложения. Будем понимать под безличным 
предложением такое предложение, которое содержит глагол в личной форме, 
но не содержит существительного в именительном падеже, которое 
выполняло бы роль подлежащего. Безличным предложениям русского языка 
соответствуют предложения с безличным местоимением \textit{il} 
французского языка. Пример перевода:

     \textit{Однако} {\bfseries\textit{оказалось}}, \textit{что эта прочность 
может в конечном счете привести к нарушению надежности 
соединения}.\;$\rightarrow$\;\textit{Toutefois}, {\bfseries\textit{il est apparu}} 
\textit{que cette robustesse pouvait finalement porter atteinte}  
$\grave{\mbox{\textit{a}}}$ \textit{la fiabilit}$\acute{\mbox{\textit{e}}}$ \textit{de 
la liaison}.
{\looseness=1

}
     
     Правило переноса выглядит следующим образом:
     \begin{multline*}
     \mathrm{V[Person~3, Number SG, Gender NEUTR]} \&{}\\
     {}\&  \mathrm{NO  NP[Case NOM]} \rightarrow \mathrm{Il} +{}\\
     {}+\mathrm{ V[Person 3, Number SG, Gender  MASC]}\,.
     \end{multline*}

$\bullet$~Неопределенно-личные предложения.

\medskip

\noindent
\textbf{Пример:}
     
\begin{center}
     \textit{Указанное раструбное соединение осуществляют}~[$\ldots$]. 
     \end{center}
     
     При переводе на французский язык чаще всего используется пассивная 
конструкция:

\begin{center}
          \textit{Cet embo}$\hat{\iota}$\textit{tement est 
effectu}$\acute{\mbox{\textit{e}}}$ [$\ldots$].
\end{center}
     
     Возможен и другой вариант перевода, с использованием безличного 
местоимения \textit{on}:
     
\begin{center}
     \textit{On effectue cet embo}$\hat{\iota}$\textit{tement} [$\ldots$].
     \end{center}
     
     Правило переноса выглядит следующим образом:
     \begin{multline*}
     \mathrm{NP[Case Acc] + V[Person 3, Number PL]}\rightarrow {}\\
     {}\rightarrow  \left \{\mathrm{NP^* + V(be)^*} +{}\right.\\
\left.{}+ \mathrm{V[FORM PART, TENSE PAST]^*}\right\} \\
\mathrm{OR} \left\{\mathrm{On + V[Person 3, Number SG] + NP}\right\}\,.
\end{multline*}
     
     Знак $^*$ означает согласование морфологических признаков.
     
\begin{figure*}[b] %fig1
\vspace*{6pt}
\begin{center}
\mbox{%
\epsfxsize=159.941mm
\epsfbox{mor-1.eps}
}
\end{center}
\vspace*{-9pt}
\Caption{Распределение по частям речи в русских и французских научных текстах 
патентных рефератов: (\textit{а})~реферат патента WO2004009333; 
(\textit{б})~реферат патента WO2004017987;
\textit{1}~--- предложения; \textit{2}~--- строки; \textit{3}~--- 
существительные; \textit{4}~--- глаголы; \textit{5}~--- причастия;
\textit{6}~--- деепричастия (герундии)}
%\end{figure*}
%\begin{figure*} %fig2
\vspace*{12pt}
     \begin{center}
     {\tabcolsep=3pt
     \begin{tabular}{lcl}
     \textbf{Цель, назначение} &&\\
     \textit{Русский язык} &&\textit{Французский язык}\\
     \textbf{Существительное (98)} &&\textbf{Инфинитив (72)}\\
     Инфинитив (2)  &$\leftarrow\,\rightarrow$ &Существительное (26)\\
     Придаточное предложение (0)&&Придаточное предложение (2)
     \end{tabular}
     }
          \end{center}
          \vspace*{-6pt}
\Caption{Правила когнитивного переноса для функциональных значений цели и 
назначения}
\end{figure*}


     
\section{Трансфер пропозиционального ядра в~русско-французской 
языковой паре}
       
     Основу семантико-синтаксической структуры предложения составляет 
пропозициональное ядро, прежде всего языковые средства предикации. Были 
изучены структуры когнитивного переноса в \mbox{рамках} поля функционального 
переноса (ПФП) первичной и вторичной предикации для 
     рус\-ско-фран\-цуз\-ской языковой пары по аналогии с 
     рус\-ско-анг\-лий\-ской языковой парой. Были выделены базовые 
правила когнитивного переноса для различных функциональных значений 
(частотные характеристики были выделены на основании анализа патентных 
текстов). Материалом анализа послужили параллельные тексты патентов 
и/или рефератов патентов на русском и французском языках, взятые из базы 
данных Роспатента.
     
     Сравнение русских и французских текстов рефератов научных 
патентов показало, что доля дей\-ст\-ви\-тель\-но параллельных текстов в них 
составляет примерно 30\%. Остальные тексты можно \mbox{назвать}\linebreak 
     ког\-ни\-тив\-но-со\-по\-ста\-ви\-мы\-ми, причем объем русского текста 
может превышать объем французского на две трети. Однако распределение 
по частям речи в русских и французских научных текстах патентных 
рефератов (и самих патентов) очень близко по составу и объему, что 
отражено на рис.~1. Русский текст в целом на 30\%--35\% более 
номинативен, чем французский, в котором в поле вторичной предикации 
предпочтение отдается инфинитиву (в русском~--- отглагольным 
существительным).
     
     В первом примере тексты рефератов не параллельные, а 
     когни\-тив\-но-со\-по\-ста\-ви\-мые, во втором тексты русского и 
французского реферата параллельны: перевод выполнен точно, почти 
дословно. В~любом случае, как видно из примеров, и в русских, и во 
французских патентных текстах очень высока доля именных групп, что 
вообще всегда характерно для на\-уч\-но-тех\-ни\-че\-ских текстов.
     
     Правила когнитивного переноса для функциональных значений цели и 
назначения представлены на рис.~2.
     

     Таким образом, набор структур, используемых для выражения цели 
действия, одинаков для русского и французского языка, однако французский 
тяготеет к инфинитивной структуре, а русский~--- к именной (\textit{Для 
увеличения способности сети к обобщению}$\ldots$~/ \textit{Afin d'augmenter 
la capacit$\acute{\mbox{\textit{e}}}$ du r$\acute{\mbox{\textit{e}}}$seau de 
g$\acute{\mbox{\textit{e}}}$n$\acute{\mbox{\textit{e}}}$raliser}\ldots).
     
     \smallskip
     
     \noindent
     \textbf{Примеры.}
     \begin{enumerate}
     \item $[$Cat~: VerbNoun$]$ \{для распознавания\} \{pour la reconnaissance\}~--- 
предложная группа: предлог\;+\;существительное.
     \item  $[$Cat~: VerbInf$]$ \{\textit{чтобы распознать}\} \{\textit{afin de 
reconn$\hat{\mbox{\textit{a}}}$itre}\}~--- союз\;+\;инфинитив.
     \item $[$Cat : Sentence$]$ \{\textit{чтобы распознавание было 
эффективным}\} \{\textit{pour que la reconnaissance soit\linebreak
efficace}\}~--- 
придаточное предложение, присоединяемое подчинительной связью (союзом\linebreak 
цели). При трансформации русского отглагольного существительного во 
французский инфинитив необходимо сделать выбор между его активной и 
пассивной формой. Видимо, в рамках сис\-те\-мы автоматического перевода 
данный выбор лучше всего осуществляется с применением статистических 
данных (активный инфинитив встречается в текстах намного чаще 
пассивного; в анализируемых текстах французский пассивный инфинитив в 
качестве перевода русского отглагольного существительного встретился в 
13\% случаев).
     \end{enumerate}
     
\section{Заключение}
     
     Были описаны предикативные синтаксические структуры русского 
языка, принадлежащие к функционально-семантическому полю первичной и 
вторичной предикации и соответствующие\linebreak им синтаксические структуры 
французского языка. Была составлена подробная классификация ти-\linebreak пов 
предложений русского языка с точки зрения синтаксиса и соответствующих 
им синтаксических типов во французском языке, которая может быть 
использована при написании правил переноса синтаксических структур, 
происходящего при переводе с русского языка на французский. 
В~классификации учтено синтаксическое многообразие русского языка: 
назывные предложения, безглагольная предикация (в случае невыраженного 
глагола <<быть>> в настоящем времени), безличные предложения, 
     опре\-де\-лен\-но-лич\-ные предложения, не\-опре\-де\-лен\-но-лич\-ные 
предложения, двусоставные предложения с различным типом ска\-зу\-емых 
(глагольные, именные и~пр.). 
     
     Были изучены категориальные трансформации предикативных 
структур, происходящие при переводе с русского языка на французский и в 
обратном\linebreak
направлении. Моделирование трансформаций\linebreak
предикативных 
структур для задачи машинного перевода является актуальной задачей, так 
как это явление мало исследовано с точки зрения компьютерной реализации 
и недостаточно учтено в дей\-ст\-ву\-ющих сис\-те\-мах машинного перевода. Кроме того, правила, 
задающие функциональную синонимию языковых конструкций, могут 
использоваться также при машинном обучении на корпусе параллельных 
текстов, позволяя избежать формирования избыточных правил и <<шумов>>. 
     
     Дальнейшие исследования будут направлены на уточнение сис\-те\-мы 
синтаксических соответствий с помощью параллельного корпуса текстов 
научных патентов, а также на расширение числа типов трансформаций при 
рус\-ско-фран\-цуз\-ском машинном переводе, дальнейшее изучение 
     дис\-три\-бу\-тив\-но-транс\-фор\-ма\-ци\-он\-ных характеристик 
языковых структур и сбор статистической информации.

{\small\frenchspacing
{%\baselineskip=10.8pt
\addcontentsline{toc}{section}{Литература}
\begin{thebibliography}{99}

      
     \bibitem{1-mor}
     \Au{Козеренко Е.\,Б.}
     Моделирование переноса функциональных значений для 
     анг\-ло-рус\-ско\-го машинного перевода~// Компьютерная лингвистика 
и интеллектуальные технологии: Труды Междунар. конф. Диалог'2004.~--- 
М.: Наука, 2004.
     
     \bibitem{2-mor}
     \Au{Kozerenko E.\,B.}
     Cognitive approach to language structure segmentation for machine 
translation algorithms~// Conference (International ) on Machine Learning, 
Models, Technologies and Applications Proceedings.~--- Las Vegas, USA, 2003. 
     P.~49--55.
     
     \bibitem{4-mor} %3
     \Au{Козеренко Е.\,Б.}
     Функционально-семантические инварианты для алгоритмов 
синтаксического анализа и разметки полнотекстового научного документа~// 
Системы и средства информатики.~--- М.:\ Наука, 2003. Вып.~13. 
     С.~298--312.
     
     \bibitem{3-mor} %4
     \Au{Козеренко Е.\,Б.}
     Лингвистическое моделирование для сис\-тем машинного перевода и 
обработки знаний~// Информатика и её применения, 2007. Т.~1. Вып.~1. 
     С.~54--66.
          
          \bibitem{5-mor}
     \Au{Sag I., Wasow Th., Bender E.\,M.}
     Syntactic theory: A formal introduction.~--- Stanford: CSLI Publications, 
2003.
     
     \bibitem{6-mor}
     \Au{Chomsky N., Lasnik H.}
     The theory of principles and parameters~// The minimalist program.~--- 
Cambridge: MIT Press, 1995.
     
     \bibitem{7-mor}
     \Au{Oepen S., Toutanova K., Shieber~S., Manning~C., Flickinger~D., 
Brants~T.}
     The LinGO Redwoods Treebank: Motivation and preliminary 
applications~// 19th Conference (International) on Computational Linguistics 
Proceedings.~--- Taipei, Taiwan, 2002. P.~1253--1257.
     
     \bibitem{8-mor}
     \Au{Перекрестенко А.}
     Разработка и программная реализация сис\-те\-мы автоматического 
выделения синтаксических групп для естественных языков~// Системы и 
средства информатики.~--- М.: Наука, 2007. Вып.~17. С.~273--291. 
      
      \bibitem{9-mor}
      \Au{Brown R.\,D.}
      Example-based machine translation in the Pangloss system~// 16th 
Conference (International) on Computational Linguistics (COLING-96) 
Proceedings.~--- Copenhagen, Denmark, 1996. P.~169--174.
      
      \bibitem{10-mor}
      \Au{Brown R.\,D.}
      Adding linguistic knowledge to a lexical example-based translation 
system~// 8th Conference (International) on Theoretical and Methodological Issues 
in Machine Translation Proceedings.~--- Chester, UK, 1999. P.~22--32.
     
     \bibitem{11-mor}
     \Au{Grishman R., Kosaka~M.}
     Combining rationalist and empiricist approaches to machine translation~// 
4th Conference (International) on Theoretical and Methodological Issues in 
Machine Translation Proceedings.~--- Montreal, Canada, 1992. P.~263--274.
     
     \bibitem{12-mor}
     \Au{Dugast L., Senellart J., Koehn~P.}
     Selective addition of corpus-extracted phrasal lexical rules to a rule-based 
machine translation system~// 12th Machine Translation Summit Proceedings.~--- 
Ottawa, ON, Canada, 2009. P.~222--229.
     
     \bibitem{13-mor}
     \Au{Леонтьева~Н.\,Н., Никогосов~С.\,Л.}
     Система ФРАП как информационная сис\-те\-ма~// Актуальные вопросы 
практической реализации сис\-тем автоматического перевода.~--- М.: МГУ, 
1982. С.~134--166.
     
     \bibitem{14-mor}
     \Au{Апресян~Ю.\,Д., Богуславский~И.\,М., Иомдин~Л.\,Л.\ и~др.}
     Лингвистическое обеспечение сис\-те\-мы фран\-цуз\-ско-рус\-ско\-го 
автоматического перевода ЭТАП-1. 1.~Общая характеристика сис\-те\-мы~// 
Теория и модели знаний (Теория и практика создания сис\-тем 
искусственного интеллекта): Труды по искусственному интеллекту. Ученые 
записки Тартуского гос. ун-та.~--- Тарту, 1985. 
Вып.~714. С.~20--39.
     
     \bibitem{15-mor}
     \Au{Соколова~С.}
     Как переводит компьютер. {\sf 
http:// www.translationmemory.ru/technology/articles/article\_\linebreak Sokolova.php}.
     
     \bibitem{16-mor}
     \Au{Валгина Н.\,С.}
     Синтаксис современного русского языка.~--- М.: Агар, 2000.  416~с.
     
     \bibitem{17-mor}
     \Au{Шелякин М.\,А.}
     Справочник по русской грамматике.~--- М.: Дрофа, 2006.  355~с.
     
     \label{end\stat}
     
     \bibitem{18-mor}
     \Au{Гак В.\,Г., Григорьев Б.\,Б.}
     Теория и практика перевода: Французский язык.~--- СПб.: 
Интердиалект+, 2000. 456~с.
 \end{thebibliography}
}
}


\end{multicols}           %10Abst+avt
\def\stat{kuzn}

\def\tit{ВЕРОЯТНОСТНО-СТАТИСТИЧЕСКАЯ ОЦЕНКА 
АДЕКВАТНОСТИ ИНФОРМАЦИОННЫХ ОБЪЕКТОВ}

\def\titkol{Вероятностно-статистическая оценка 
адекватности информационных объектов}

\def\autkol{Л.\,А.~Кузнецов}
\def\aut{Л.\,А.~Кузнецов$^1$}

\titel{\tit}{\aut}{\autkol}{\titkol}

%{\renewcommand{\thefootnote}{\fnsymbol{footnote}}\footnotetext[1]
%{Работа поддержана Российским фондом фундаментальных исследований
%(проекты 11-01-00515а и 11-07-00112а), а также Министерством
%образования и науки РФ в рамках ФЦП <<Научные и
%научно-педагогические кадры инновационной России на 2009--2013~годы>>.}}


\renewcommand{\thefootnote}{\arabic{footnote}}
\footnotetext[1]{Липецкий государственный технический университет, kuznetsov@stu.lipetsk.ru}


\Abst{Приведены математические основы и оригинальная методология разработки 
систем оценки семантической близости информационных объектов (ИО), представленных на 
естественном языке. Вводится ве\-ро\-ят\-но\-ст\-но-ста\-ти\-сти\-че\-ское представление 
сопоставляемых ИО. Используется теория информации для 
оценки уровня семантической близости ИО. Методология 
доведена до алгоритмов ее реализации в виде соответствующей автоматизированной 
системы. Представлены результаты практической проверки эффективности методологии. 
}

\vspace*{2pt}

\KW{информационные объекты; естественный язык; семантическая адекватность; 
вероятностная модель; теория информации}

\vspace*{6pt}

 \vskip 14pt plus 9pt minus 6pt

      \thispagestyle{headings}

      \begin{multicols}{2}
      
            \label{st\stat}

\section{Введение. Информация, знания, семантический анализ}
   
   Основным мотивом перехода от индустриальной к постиндустриальной 
модели развития в промышленно развитых странах, начавшегося в конце 
прошлого века, является стремительное увеличение скорости развития науки 
и знания. В~постиндустриальной модели развития, по мнению ведущих 
западных ученых в области социального развития и управления, 
принципиальным является изменение статуса и значения информации, науки 
и знания, которые становятся важнейшими факторами, определяющими 
эволюцию общества. 
   
   Питер Ф.~Друкер, признанный специалист в области организационного 
управления, пишет: <<Изменение значения знания, начавшееся 250~лет тому 
назад, преобразовало общество и экономику. Знание стало сегодня основным 
условием производства. Традиционные <<факторы производства>>~--- земля 
(природные ресурсы), рабочая сила и капитал~--- не исчезли, но приобрели 
второстепенное значение. Эти ресурсы можно получить, причем без особого 
труда, если есть необходимые знания>>~\cite{1-k}. 
   
   Информация и знания становятся главной движущей силой 
экономического развития и перехо-\linebreak дят из категории бесплатного 
общественного бла-\linebreak га в категорию товара. В~промышленно развитых\linebreak \mbox{странах} 
разработка и внедрение технологических инноваций~--- решающий фактор 
социального и экономического развития, залог экономической безопасности. 
В~США, по оценкам американских специалистов, прирост душевого 
национального дохода благодаря этому фактору составляет 90\%. 
   
   Беспрецедентный рост потока информации и знаний, скорости их 
передачи и возможностей доступа на первый план научных проблем 
выдвигает разработку технологий их автоматической обработки. Б$\acute{\mbox{о}}$льшая 
часть существующих и вновь формируемых знаний и информации 
представлена на естественном языке. 

Одной из актуальных, 
фундаментальных проблем в области обработки информации становится 
обеспечение возможности формального семантического сравнения, оценки 
семантической \mbox{бли\-зости} ИО, представленных 
на естественном языке. Разработка формализованных технологий оценки 
семантической близости ИО позволила бы перейти к практической 
реализации важных задач в сфере обработки информации, распространения 
знаний и образования. 
   
   В настоящее время в литературе задача семантического сравнения двух 
текстов, в основном, рас\-смат\-ри\-ва\-ется в контексте дубликатов в 
   веб-докумен\-тах и в системах автоматизированного перевода. При 
поиске дубликатов опираются на число слов, совпавших в двух текстах. 
Алгоритм сравнения на основе шинглов является наиболее простым и 
распространенным. Такой подход используется для нахождения копий 
текстов, полученных копированием и перестановкой слов, но он не позволяет 
оценить семантическую близость текстов.
   
   При более сложном анализе текстов учитывается структура входящих в 
них предложений. В~предложениях выделяются элементы (слова или группы 
слов) и сопоставляются определенные шаблоны для этих элементов. Данный 
подход описан в книге~\cite{2-k} и используется в ряде кандидатских 
диссертаций. 
   
   Однако задача оценки информационной бли\-зости двух текстов в 
обнаруженных автором работах не затрагивается. Используемые там 
концепции не ориентированы на ее решение и не могут быть использованы в 
качестве основы для ее решения.
   
   В данной статье предлагается оригинальная методология оценки 
информационной близости текстов на основании вероятностно-ста\-ти\-сти\-че\-ско\-го 
подхода и теории информации. Реализация методологии 
позволит перейти к практическому решению разнообразных задач, в которых 
требуется определять меру информационной адекватности\linebreak документов, 
представленных на естественном языке. В~част\-ности, разработанная 
концепция будет использована для синтеза автоматизированных сис\-тем 
оценки уровня знаний. 

\section{Формализация анализа текстов}

   Развитие информационных технологий, предо\-став\-ля\-ющих широкие 
возможности ав\-то\-ма\-ти\-зи\-рован\-но\-го анализа и обработки вербально 
пред\-став\-лен\-ной информации, существенно повысило интерес к разработке 
формальных методов исследования и сопоставления текстов. Современные 
компьютерные системы позволяют хранить и обрабатывать практически 
неограниченные объемы текстовой информации. Это стимулирует 
разработку формальных методов для поддержки выполнения постоянно 
расширяющихся и углубляющихся исследований информации, 
представленной на естественном языке. В~настоящее время интенсивно 
разрабатываются формальные методы, позволяющие автоматизировать 
решение задач в области морфологического, синтаксического и 
семантического анализа текстовой информации. 
   
   Из имеющихся публикаций следует, что методологии различных видов 
анализа базируются на сходных концепциях и достаточно близки по своему 
содержанию. Формализация морфологического анализа направлена на 
алгоритмическое пред\-став\-ле\-ние грамматики русского языка. Час\-ти речи 
русского языка определены, однозначно определены формы, в которых они 
могут быть, определены правила, следуя которым должно осуществляться 
изменение слов, принадлежащих к различным час\-тям речи при образовании 
соответствующих форм. Значительная часть правил изменения частей речи 
уже отражена в словарях. Все правила могут быть представлены в виде 
соответствующих процедур, функций, подпрограмм и~т.\,п. В~соответствии 
с имеющимися правилами может быть идентифицировано, какой частью 
речи является конкретное слово, в какой форме оно находится. Поэтому 
может быть написана программа, обеспечивающая автоматизированное 
выполнение морфологического анализа, так что на ее вход будет поступать 
слово предложения, а на выходе она выдаст результат анализа: какой частью 
речи является данное слово и в какой форме оно находится. 
   
   Формализация синтаксического анализа~--- задача также понятная: 
существует синтаксис русского языка, представляющий свод правил, следуя 
которым могут быть достаточно четко определены члены предложения. Раз 
правила существуют, то их можно представить в виде набора процедур, 
обеспечивающих выявление состояния (роли) каждого слова в предложении. 
Правила и процедуры могут быть более или менее сложными, но 
принципиально то, что правила имеются, а следовательно, и процедуры 
могут быть синтезированы по ним. На основании этих процедур может быть 
разработана система синтаксического анализа, которая, получая на свой вход 
предложение, выполнит его разбор и анализ и на выходе, как примерный 
ученик выдаст о каждом слове, входящем в предложение, информацию: 
каким его членом оно является. 
   
   Проблема семантического анализа текстов интенсивно исследуется в 
различных аспектах: разрабатываются правила и алгоритмы анализа 
предложений, выявления их структуры, установления соответствия между 
разноязычными текстами, поиска информации и~т.\,д. При этом, однако, 
формализация семантического анализа ка\-ко\-го-ли\-бо одноязычного текста 
или даже одного предложения представляется задачей весьма малопонятной. 
   
   Под формализацией обычно понимается однозначное математическое 
представление существующих правил, которые, возможно, в текстовом, 
вербальном виде содержат определение способа извлечения нужных 
сведений из начальных, исходно заданных данных. В~контексте 
формализации семантического анализа математическому оформлению 
должны подлежать правила, позволяющие извлечь из слова его смысл. Но, в 
отличие от морфологии и синтаксиса, не существует ка\-ких-ли\-бо 
формальных семантических правил, следуя которым можно было бы 
установить смысл, вложенный в предложение или в каждое отдельное его 
слово. Поэтому невозможно представить систему семантического анализа в 
виде упорядоченного набора правил или предписаний, которая (по аналогии 
с системами морфологического или синтаксического анализа) получала бы 
на входе предложение или слово, а на выходе выдавала бы его смысл. Ибо 
слово и есть его смысл. В~толковом словаре, конечно, разъясняется смысл 
отдельных слов, но, в конечном итоге, это разъяснение представляет 
сопоставление одному слову других слов, близких по смыслу, и следует из 
словаря, а не из каких-либо правил, которые можно было бы формализовать.
   
Следовательно, если морфологический и синтаксический анализ действительно 
представляют анализ в соответствии со смыслом этого слова, т.\,е.\ разбор 
предложения на составляющие его элементы и выяснение их роли и 
состояния, то семантический анализ может пониматься только в смысле 
сравнения и выяснения смысловой близости разных слов и текстов. Только 
при наличии эталона анализируемого предложения, смысл которого 
известен, опираясь на словари, в которых отражена семантическая близость 
отдельных слов и словосочетаний, может быть получен ответ, что 
анализируемое предложение находится в некотором соответствии с эталоном 
и, следовательно, имеет определенный смысл. Например, при переводе 
смысл на язы\-ке-ори\-ги\-на\-ле принимается за известный эталон. С~помощью 
словаря, в котором имеется соответствие между словами и 
словосочетаниями, находится соответствующее выражение на другом языке. 
Важно понимать, что соответствие при этом следует не из слов, а из словаря. 
   
Таким образом, представляется, что при сопоставлении одноязычных текстов 
более правильно говорить не об их семантическом анализе, а об 
уста\-нов\-ле\-нии уровня их информационной адекватности, об определении 
взаимного количества информации, общего для сравниваемых текстов, из 
общего объема информации, содержащегося в одном из них, принимаемом за 
эталонный текст. 
   
   Автоматизированная технология должна обеспечивать реализацию 
функций определения пересечения, общей части дубля и эталона. Для этого в 
автоматизированной технологии должны быть разработаны формально-математические 
инструменты для представления текстов дубля и эталона в 
виде, позволяющем оценить количество информации, содержащейся в них, 
определить долю информации в дубле, отражающую содержание эталона, и 
на этой основе сформировать оценку их семантической близости. 
   
\section{Ограниченность возможностей детерминированного 
подхода}
   
   Имеется принципиальное базовое отличие семантического анализа от 
морфологического и синтак\-си\-че\-ско\-го. Отмеченное кратко выше показывает, 
что объектом морфологического и синтаксического анализа является 
фиксированный, \mbox{полностью} однозначно определенный текстовый фрагмент. 
Определение роли и состояния оборотов или отдельных слов, составляющих 
предложения анализируемого фрагмента, производится по четко 
определенным, детерминированным правилам, которые могут быть 
представлены в виде более или менее сложных алгоритмов, формирующих 
морфологические или синтаксические характеристики предложений, 
оборотов и слов. Процесс и правила формирования морфологических и 
синтаксических характеристик и сами характеристики определенны и 
закономерны. Поэтому эти виды анализа закономерны или 
детерминированы.
   
   На первый взгляд, кажется заманчивой идея представить текст эталона и 
дубля в виде предложений, предложения в виде деревьев или иных 
детерминированных структур и затем сравнить структурированное таким 
образом представление эталона и дубля. Однако русский язык, а здесь 
подразумевается, что именно он используется для вербального 
представления информации, совершенно игнорирует какие-либо 
структурные ограничения по расположению членов предложения, по виду 
предложений, формированию фрагментов предложений из групп слов, 
изобилует бесконечным многообразием форм управления отдельными 
словами и группами слов. По этой причине можно ожидать, что в эталоне и 
дубле не окажется тождественно равных структурных единиц, а чис\-ло 
альтернатив, подлежащих сравнению, может быть бесконечным. В~такой 
ситуации становится принципиальной проблема определения альтернатив и 
логических правил их разрешения. Поэтому представляется, что 
детерминированный семантический анализ не вполне соответствует 
содержательному существу проб\-лемы.
   
   По мнению автора, семантический анализ как термин не вполне удачен. 
Речь может идти об установлении уровня информационного соответствия 
содержания одного, анализируемого текста, который здесь именуется 
дублем, содержанию другого, именуемого эталоном, текста. Представляется, 
что реализация сравнения, выявления уровня соответствия нескольких 
русскоязычных текстов использованием детерминированного 
структурирования и детерминированных правил оценки близости не 
представляется практически возможной. С~учетом интеллектуальной 
специфики одна и та же информация может случайным образом облекаться в 
различную текстовую оболочку. Основная проб\-ле\-ма оценки степени 
близости информационных объектов, представленных на естественном 
языке, следует из семантической многозначности слов и наличия синонимов. 
Эти обстоятельства приводят к неоднозначности лексического представления 
семантического содержания текстов. Проблемы неоднозначности текстовой 
информации известны и активно исследуются специалистами в области 
русского языка. Детерминированный подход сравнения текстов оказывается 
нацеленным фактически на формальное описание тонкостей образования 
синтаксических форм русского языка, многообразие которых представляется 
бесконечным. 
   
   Очевидно, что случай полного совпадения\linebreak текстов, используемый при 
формировании рег\-ла\-мента доступа к информации, здесь не рас\-смат\-ри\-ва\-ет\-ся. 
Должна присутствовать возможность выделения общности 
информационного содержания со\-по\-став\-ля\-емых информационных объектов 
из их случайным образом выбранной формы представления на естественном 
языке. Решение такой задачи может быть получено только при описании 
взаимосвязи семантического (информационного) содержания и лексического 
оформления обоих срав\-ни\-ва\-емых текстов с вероятностно-статистических 
позиций.

\section{Элементы теории информации}

   Более полувека существует в виде научной дисциплины теория 
информации. Ее основоположником является американский специалист в 
об\-ласти передачи информации в технических линиях связи Клод 
   Шен\-нон~\cite{3-k}. Значительный вклад в теорию информации, 
особенно в строгое доказательство ее основных принципов, внесли советские\linebreak 
ученые школы А.\,Н.~Колмогорова~\cite{4-k}. В~теории информа\-ции 
исследуются проблемы передачи и преобразования информации, при этом 
вводится количественная оценка информации. Применительно к проблеме 
сравнения близости ИО, которой посвящена 
данная статья, является важным, что в теории информации разработаны 
теоретические основы исследования бли\-зости сообщений, переданного 
передатчиком и принятого приемником на другом конце линии связи. При 
этом вводится количественная мера информации, которая позволяет 
осуществить сопоставление информационной емкости переданного и 
принятого сообщений и на этой основе оценить искажение (потери) 
информации в линии связи при ее пе\-ре\-даче. 
   
   Теория информации может быть использована для решения проблемы 
оценки близости ИО, представленных на естественном языке. Ничто не 
мешает вместо сообщений, принятого и переданного, рассматривать 
ИО, трактуя один из них~--- аналог переданного 
сообщения~--- как эталонный информационный объект (ЭИО), а другой~--- 
аналог принятого сообщения~--- как дубль ЭИО (ДИО). 
Как будет видно дальше, мера количества информации в 
одном объекте о другом симметрична, поэтому при количественной оценке 
их близости не важно, какой объект считать эталоном, а какой~--- дублем. 
Понятно, что сравнение ЭИО и ДИО на абсолютное их совпадение 
неприемлемо. В~теории информации исследуется количественная, а не 
содержательная сторона информации. В~связи с наличием случайных помех 
в системах формирования и линиях передачи информации сообщения, 
переданное и принятое, интерпретируются случайными величинами~$\xi$. 
Шенноном было предложено использовать энтропию\footnote{Энтропия (от 
греческого \textit{entropia}~--- превращение) введена в 1865~г.\ немецким физиком Р.~Клаузиусом как 
функция состояния термодинамической системы, изменение которой $dS$ в равновесном процессе равно 
отношению количества теплоты $dQ$, подведенного к системе или отведенного от нее, к 
термодинамической температуре системы~$T$: $dS=dQ/T$. Л.~Больцман, один из основателей 
статистической термодинамики, предложил использовать энтропию как меру вероятности пребывания 
системы в данном состоянии. Шеннон ввел энтропию в теорию информации в качестве меры количества 
информации, которое выражается через распределение вероятностей.} как вероятностную меру 
количества информации~\cite{3-k}.
   
   Энтропия исхода определяется в виде логарифма вероятности этого 
исхода:
   \begin{equation}
   H(\xi_i)=- \log p(\xi_i)\,,
   \label{e1-k}
   \end{equation}
а усредненная энтропия случайной величины~$\xi$ выражается через 
функцию распределения ее вероятностей в виде:
\begin{equation}
H_\xi =- \sum\limits_\xi p(\xi)\log p(\xi)\,,
\label{e2-k}
\end{equation}
где $\xi$~--- случайная величина; $p(\xi)\leq 1$~--- распределение ее 
вероятностей. 
   
   В рассматриваемом здесь случае анализа текстов случайными 
величинами могут быть слова или другие конструкции. Под количеством 
информации в теории информации понимается неопределенность, 
устраняемая в результате выяснения исхода, т.\,е.\ значения, принимаемого 
случайной величиной. 
   
   Простейший содержательный пример в контексте статьи может быть 
следующим. Слова: \textit{пример, образец, экспонат} в некотором контексте 
являются синонимами и могут использоваться для обозначения 
объекта~$\xi$ с вероятностями: $p(\xi=\;\mbox{\textit{пример}})\hm=0{,}2$; 
$p(\xi=\;\mbox{\textit{образец}})\hm=0{,}4$; 
$р(\xi=\;\mbox{\textit{экспонат}})\hm=0{,}6$. В~этом случае количество 
информации, получаемое при реализации конкретного исхода, допустим, при 
использовании слова экспонат, т.\,е.\ $\xi\hm=\;\mbox{\textit{экспонат}}$, будет 
в соответствии с~(\ref{e1-k}) равно $\log p(0{,}6)$, а усредненная 
неопределенность объекта~$\xi$ будет по~(\ref{e2-k}) равна: $H_\xi\hm=- 
p(0{,}2)\log p(0{,}2)\hm- p(0{,}4)\log p(0{,}4)\hm- p(0{,}6)\log p(0{,}6)$. 
В~теории информации наиболее часто используются логарифмы по 
основанию~2, в этом случае количество информации определяется в битах.
   
   В реальности чаще интерес представляет сравнение информационной 
емкости сообщений, оценка имеющейся в них совместной информации. Для 
этого по распределениям вероятностей сообщений определяется энтропия 
каждого из них (количество информации в каждом из них), а по совместному 
распределению вероятностей~--- совместная энтропия. По энтропиям 
оценивается количество взаимной информации. 
   
   При использовании теории информации для описания закономерностей 
передачи информации энтропия переданного сообщения определяет 
количество переданной информации, а энтропия принятого сообщения~--- 
количество принятой информации. Общая часть в переданном и принятом 
сообщениях определяет количество взаимной информации. Обозначив через 
$\xi$ переданное сообщение, а через~$\eta$~--- принятое, количество 
взаимной информации $I (\xi\eta)$, или количество информации, 
содержащееся в принятом сообщении~$\eta$ из переданного 
сообщения~$\xi$, можно определить, следуя теории информации~\cite{3-k}, 
в виде: 
   \begin{equation}
   I_{\xi\eta} =\int\limits_X\! \int\limits_Y p_{\xi\eta} (x,y)\log \fr{p_{\xi\eta} 
(x,y)}{p_\xi(x)p_\eta(y)}\,dxdy\,,
   \label{e3-k}
   \end{equation}
где $p_\xi(x)$~--- плотность распределения переданного сообщения;
   $p_\eta(y)$~--- плотность распределения принятого сообщения;
   $p_{\xi\eta}(x,y)$~--- плотность совместного распределения; 
   $X$ и $Y$~--- области определения~$x$ и~$y$ соответственно.
   
   В~(\ref{e3-k}) имеет место равноправное симметричное вхождение~$\xi$ 
и~$\eta$, поэтому взаимная информация симметрична относительно~$\xi$ 
и~$\eta$. Отсюда следует, что при количественной оценке взаимной 
информации не важно, какое сообщение выступает в роли переданного, а 
какое~--- в роли принятого. 
   
   Именно взаимная информация может использоваться в качестве меры 
подобия ИО. Нетрудно видеть, что содержательное существо теории 
информации, направленное на оценку потерь информации при ее передаче, 
адекватно содержательной сущности многих задач в области исследования 
семантической близости ИО на естественном языке. Применение теории 
информации для решения проблемы оценки близости ИО может 
основываться на замене сообщений исследуемыми ИО. 
Один объект (эталон) может интерпретироваться переданным 
сообщением, а второй (дубль)~--- принятым сообщением. Вследствие 
симметрии от перемены ролей количество взаимной информации не 
изменится. Формула~(\ref{e3-k}) отражает количество взаимной информации 
непрерывных сообщений, точнее сообщений, случайный характер которых 
определяется непрерывными функциями распределения вероятностей вида 
$p_\xi(x)$, $x\in X \hm= [x^\prime, x^{\prime\prime}]$, где $x^\prime$ и 
$x^{\prime\prime}$~--- предельные значения~$x$. 
   
   При анализе текстов в качестве случайных величин будут выступать 
синтаксические или морфологические компоненты, которые являются 
дискретными величинами. Их случайными лексическими значениями будут 
выступать слова или словосочетания, характеризуемые дискретными 
ве\-ро\-ят\-но\-стя\-ми, как это показано в приведенном выше кратком примере. 
В~примере под случайным объектом или компонентом~$\xi$ может 
пониматься подлежащее при использовании синтаксической структуризации 
или существительное при использовании морфологической структуризации. 
Компонент~$\xi$ в обоих случаях может принимать случайные значения 
\textit{пример}, \textit{образец}, \textit{экспонат} с вероятностями 
$p(\xi=\;\mbox{\textit{пример}})\hm=0{,}2$; 
$p(\xi=\;\mbox{\textit{образец}})\hm=0{,}4$; 
$p(\xi=\;\mbox{\textit{экспонат}})\hm=0{,}6$. Дискретные значения 
вероятностей могут суммироваться, и поэтому вмес\-то интегралов, 
присутствующих в~(\ref{e3-k}), будут исполь-\linebreak зоваться суммы по всем 
возможным значениям\linebreak случайных величин. Количество взаимной 
информации в дискретном случае будет определяться следующим образом:
   \begin{equation}
   I_{\xi\eta} =\sum\limits_{\xi\in X} \sum\limits_{\eta\in Y} p(\xi,\eta) \log 
\fr{p(\xi,\eta)}{p(\xi)p(\eta)}\,,
   \label{e4-k}
   \end{equation}
где $X = \{x_1, x_2, \ldots , x_n\}$, $Y \hm= \{y_1, y_2, \ldots , y_m\}$~--- 
множества значений случайных величин~$\xi$ и~$\eta$, 
   $p(\xi)$, $p(\eta)$ и $p(\xi,\eta)$~--- распределения их вероятностей. 
   
   Данная статья посвящена изложению общей концепции 
автоматизированной технологии оценки степени близости ИО, 
представленных на естественном языке. Поэтому здесь ограничимся этим 
кратким представлением основного существа теории информации и ее 
дальнейшее использование объясним <<на словах>>. Достаточно полные и 
строгие сведения по энтропии и взаимной информации интересующийся 
читатель сможет найти в оригинальной литературе, например~[3--5], а их 
применение в далекой от передачи информации области управления 
качеством и технологиями~--- в работах автора~\cite{6-k, 7-k} и~др.

\section{Понятие вероятностной модели}

   Структуризация текста с целью извлечения заключенного в нем смысла 
представляется не вполне определенной задачей. Неопределенность следует 
из сложности представления ее содержательного существа, откуда вытекают 
и проблемы с определением методов решения. При оценке степени подобия 
содержания двух ИО смысл каждого из них, вообще 
говоря, интереса не представляет, так как целью является не выяснение 
семантики, а оценка степени их содержательного подобия. Для этого 
необходимо оценить меру совпадения в текстах того, о чем (ком) идет речь, 
что, как, где, когда и~т.\,п.\ с ними происходит или они делают. 
   
   Синтез формального подхода к оценке бли\-зости ИО, представленных на 
естественном языке, тре\-бу\-ет формального пред\-став\-ления самих 
срав\-ни\-ва\-емых объектов, т.\,е.\ разработки модели представления ИО. 
Математические модели объектов\linebreak являются основой для разработки систем 
управления этими объектами, а также решения задач анализа объектов, 
исследования взаимосвязей между компонентами, образующими объект, 
синтеза суж\-дений о состоянии и эволюции объекта. Поэтому структура и 
содержание модели должны разрабатываться с учетом четкого представления 
целей, для достижения которых она будет использоваться. Модель должна 
адекватно отражать все наиболее важные для правильного решения 
поставленной задачи содержательные аспекты объекта и игнорировать те, 
которые, усложняя модель, не способствуют повышению качества решения. 
Модели одного и того же объекта, предназначенные для решения различных 
задач, могут значительно различаться глубиной учета отдельных деталей. 
   
   Применительно к проблеме формального представления 
ИО в задачах оценки их семантической близости 
модель должна обеспечивать возможность сопоставления эквивалентных 
компонентов объектов, отражающих содержание сопоставляемых ИО, и 
игнорировать стилистические тонкости, влия\-ющие на форму представления 
содержания, но не на его смысл. 
   
   Обычно первым шагом при построении модели является структуризация 
объекта, выделение его компонентов, которые в совокупности определяют 
рассматриваемый объект. 

При разработке структуры модели ИО можно было 
бы, следуя имеющимся в литературе примерам, исходить из структуры 
простых предложений, в виде совокупности которых тем или иным способом 
может быть пред\-став\-лен ИО. Простое предложение русская грамматика 
определяет центральной грамматической единицей. <<Это определяется тем, 
что простое предложение представляет собой элементарную 
предназначенную для передачи относительно законченной информации 
единицу\ldots>>~\cite[с.~405]{8-k}. Но далее следуют 154~параграфа, в 
которых излагаются типы и формы простых предложений. Их многообразие 
и присутствие неполной четкости деления по типам и формам делает 
нереальной задачу формального описания даже простых предложений, не 
говоря о более сложных типах предложений. Именно по этой причине 
детерминированный подход, опирающийся на представления русской 
грамматики, как отмечалось выше, представляется малопригодным для 
анализа семантической бли\-зости ИО. 
   
   Вследствие того, что целью разрабатываемой методологии является не 
анализ текстов с позиций грамматики русского языка, а сопоставление их 
семантического содержания, которое может\linebreak случайным образом облекаться в 
лексическую оболочку, к определению структуры модели пред\-став\-ля\-ет\-ся 
целесообразным подойти с вероятностно-ста\-ти\-сти\-че\-ских позиций. 
   
   В теории вероятностей~\cite{9-k} существует вероятностная модель, 
которая позволяет дать формальное, максимально полное описание 
   ве\-ро\-ят\-но\-ст\-но-ста\-ти\-сти\-че\-ско\-го объекта. Она определяется 
на множестве элементарных событий $\{\omega_1, \omega_2, \ldots , 
\omega_n\}$, которое образует пространство элементарных событий, или 
исходов $\Omega =\{\omega_1, \omega_2, \ldots , \omega_n\}$. Известны 
вероятности элементарных событий $p(\omega_i)$, $i=1, 2, \ldots , n$. На 
множестве элементарных событий задается алгебра $\aleph=(A_j \vert 
A_j\subseteq\Omega$) или, иначе, система случайных событий, составленных 
каким-ли\-бо определенным образом из элементарных событий $\omega_i 
\in\Omega$. Для каждого из случайных событий $A_j=\{\omega_i\in \Omega\}$, 
образующих алгебру, по вероятностям элементарных исходов $p(\omega_i)$, 
$\omega_i\in A_j$, определяется его вероятность $P(A_j)$. 
   
   Набор: множество элементарных событий $\Omega \hm=\{\omega_1, \ldots , 
\omega_n\}$, система случайных событий (ал\-геб\-ра) $\aleph=(A_j\vert  
A_j\subseteq\Omega$) и вероятности случайных событий $P(A_j)$~--- образует 
вероятностную модель случайного объекта. Она содержит всю информацию, 
которой может быть охарактеризован случайный объект. Формально 
вероятностная модель (или вероятностное пространство эксперимента с 
конечным пространством исходов~$\Omega$ и алгеброй событий~$\aleph$) 
может быть представлена в виде:
   \begin{equation}
   M_\Omega =\{ \Omega, \aleph, P(A)\}\,,
   \label{e5-k}
   \end{equation}
где $\Omega= \{\omega_1, \omega_2, \ldots , \omega_n\}$, $\aleph= (A_j\vert A_j 
\subseteq \Omega)$, $P(A) \hm= (P(A_j)\vert A_j\in \aleph)$.

\begin{table*}[b]\small
\begin{center}
\Caption{Представление ВСММ ИО (фрагмент)}
\vspace*{2ex}

\tabcolsep=4.5pt
\begin{tabular}{|c|c|c|c|c|c|c|c|c|}
\hline
\multicolumn{2}{|c|}{Существительные}&\multicolumn{2}{c|}{Прилагательные}&
\multicolumn{2}{c|}{Числительные}&\ldots&\multicolumn{2}{c|}{Глаголы}\\
\hline
Слова&Характеристики&Слова&Характеристики&Слова&Характеристики&\ldots&Слова&Характеристики\\
\hline
1 Дом&$p$(дом)&Серый&$p$(сер.)&Три&$p$ (три)&\ldots&Стоит&$P$(стоит)\\
2 Стол&$p$(стол)&Белый&$p$ (бел.)&Два&$p$ (два)&\ldots&Идет&$P$ (идет)\\
\ldots&\ldots&\ldots&\ldots&\ldots&\ldots&\ldots&\ldots&\ldots\\
\hline
\end{tabular}
\end{center}
\end{table*}
   
   Определить вероятностную модель конкретного случайного объекта 
значит определить все ее элементы~--- множество элементарных исходов, 
сис\-те\-му случайных событий и их вероятности~--- для этого конкретного 
объекта.

\section{Вероятностно-статистическая морфологическая модель 
информационного объекта}
   
   Информационный объект может быть пред\-став\-лен в виде вероятностной 
модели. В нем множество элементарных исходов $\Omega = \{\w_1, \w_2, 
\ldots , \w_n\}$ представляют слова $\w_i$, $i=1, 2, \ldots , n$, со\-став\-ля\-ющие 
текст ИО. Существуют системы структуризации лексического материала. 
Достаточно общими и пригодными для использования при разработке 
ве\-ро\-ят\-но\-ст\-но-ста\-ти\-сти\-че\-ской модели ИО являются синтаксическая и 
морфологическая структуризации русского языка. Морфологическая 
структуризация задается определением частей речи русского языка, которые 
разделяют язык на самые крупные грамматические классы слов~\cite{8-k}. 
Различают десять частей речи, среди которых шесть знаменательных: 
существительные, прилагательные, чис\-ли\-тель\-ные, 
   мес\-то\-име\-ния-су\-ще\-ст\-ви\-тель\-ные, наречия, глаголы и три 
служебные: предлоги, союзы, частицы. Десятой частью являются 
междометия. Части речи, к которым относятся отдельные слова, могут 
трактоваться случайными событиями~$A_j$, $j = 1, 2, \ldots , 10$. Каждое 
отдельное слово (реализация, элементарный исход) $\omega_i$, $i=1, 2, \ldots 
, n$, входит в текст с определенной вероятностью~$p(\omega_i)$. В~тексте 
роль вероятности играет относительная частота $p(\omega_ii)=n_i/n$, где 
$n_i$~--- число употреблений в ИО слова~$i$, $n$~--- общее количество слов 
в ИО. Относительная частота получается экспериментально и называется в 
теории вероятностей эмпирической вероятностью. По вероятностям 
$p(\omega_i)$ отдельных слов вычисляются вероятности событий~$A_j$~--- 
час\-тей речи. Вероятностно-статистическая модель ИО~(\ref{e5-k}), в которой 
алгебра (способ структуризации) слов определяется морфологией, может 
быть названа вероятностно-статистической морфологической моделью 
(ВСММ) ИО, которая может быть по аналогии с~(\ref{e5-k}) записана в виде: 
   \begin{equation}
   M_M =\{\Omega, \aleph_M, P(A)\}\,,
   \label{e6-k}
   \end{equation}
где индекс $M$ подчеркивает морфологический характер модели, который 
отражается через определение алгебры~$\aleph_M$.
   
   Конкретный ИО представляется в виде соответствующего 
   ве\-ро\-ят\-но\-ст\-но-ста\-ти\-сти\-че\-ско\-го морфологического образа 
ИО. Он синтезируется на основа\-нии модели~(\ref{e6-k}) введением 
конкретного множества элементарных исходов $\W_O= (\w_1, \w_2, \ldots$\linebreak $\ldots , 
\w_n)$ -- слов. Обозначение $\W_O$ подчеркивает, что это множество слов 
конкретного ИО. Множество структурируется в соответствии с введенной 
ал\-геб\-рой $\aleph_M=(A_1, A_2, \ldots , A_J)$, где $J$~--- число случайных 
событий (частей речи), используемых в образе, $J\leq  10$, т.\,е.\ некоторые 
части речи, например междометия, предлоги, могут не использоваться при 
формировании образа. В~результате пол\-ностью определяется 
ве\-ро\-ят\-но\-ст\-но-ста\-ти\-сти\-че\-ский морфологический образ (ВСМО) ИО ВСММ~(\ref{e5-k}) 
конкретного ИО, который может быть представлен в виде:
   \begin{equation}
   O_M=\{ \W_O,\aleph_M,P_O(A)\}\,,
   \label{e7-k}
   \end{equation}
где обозначения следуют из~(\ref{e5-k}), (\ref{e6-k}) и текста. 
   
   По множеству элементарных исходов $\W_O$ вычисляются 
количественные характеристики образа: вероятности $p(\w_i)$ и вероятности 
случайных событий $P(A_j)$. По вероятностям может быть в соответствии 
с~(\ref{e2-k}) определена энтропия $H_O$ образа ИО, характеризующая 
количество информации в об\-разе.
   
   Для пояснения, возможно, непривычного для исследований в области 
русского языка подхода и терминологии воспользуемся наглядной 
иллюстрацией. Вероятностно-ста\-ти\-сти\-че\-ская морфологическая
модель ИО может быть представлена в виде табл.~1.
   
   В шапке таблицы для сокращения размеров примера указаны в явном 
виде только 4~части речи из десяти, имеющихся в языке. В~модели будет 
использоваться таблица с полным набором частей речи. Шапка таблицы 
является атрибутом модели. Она отражает структуру ВСММ и является 
общей для представления образов всех ИО. Части речи, указанные в шапке, 
трактуются случайными величинами. В~вероятностном смысле шапка 
таблицы содержит все рассматриваемые при морфологическом подходе 
случайные события или полное поле событий. Смысл полного поля событий 
в данном контексте в том, что любое встреченное в тексте (в ИО) слово 
относится к одному из них (является ка\-кой-либо частью речи). 
   
   Все то, что находится в таблице под шапкой, отражает конкретный ИО, 
т.\,е.\ определяет ВСМО ИО~(\ref{e7-k}). Слова \textit{дом}, \textit{стол} 
являются в примере случайными значениями, которые приняла часть речи 
<<существительное>>, аналогично \textit{серый}, \textit{белый}~--- 
случайные значения <<прилагательного>>, \textit{стоит}, \textit{идет}~--- 
<<глагола>>. Кроме собственно значений, которые принимают части речи в 
ИО, в модели отражаются их случайные характеристики, например 
относительные частоты.
   
   Вероятностно-ста\-ти\-сти\-че\-ский морфологический
образ~(\ref{e7-k}) может быть сформирован для любого произвольного 
ИО, представленного на русском языке, да и не на 
русском тоже. При его формировании могут быть использованы имеющиеся 
достаточно эффективные инструменты морфологического анализа текстов, 
которые позволяют автоматизировать процедуры отнесения слов к частям 
речи. 
   
   При решении задачи оценки семантической близости 
ИО ВСМО синтезируется для обоих сравниваемых объектов. Для 
определенности один из них называется эталоном (ИОЭ), его 
морфологический образ обозначается $O_{\mathrm{МЭ}}$, а второй~--- дублем 
(ИОД), его образ~--- $O_{\mathrm{МД}}$. Ве\-ро\-ят\-но\-ст\-но-ста\-ти\-сти\-че\-ский 
морфологический образ содержит все слова ИО, 
структурированные по частям речи, ве\-ро\-ят\-ности отдельных слов и частей 
речи в ИО. Ве\-ро\-ят\-ности количественно характеризуют ВСМО ИОЭ и ВСМО 
ИОД. На их основе может быть осуществлена оценка количества 
информации в каждом из объектов и оценено количество взаимной 
информации~(\ref{e4-k}) в объектах. Выражение для оценки количества 
взаимной информации~(\ref{e4-k}) может быть приведено к виду:
   \begin{equation}
   I_{\mathrm{МЭД}} = H(O_{\mathrm{МЭ}}) + H(O_{\mathrm{МД}})- 
H(O_{\mathrm{МЭ}}, O_{\mathrm{МД}})\,,
   \label{e8-k}
   \end{equation}
где обозначения совпадают с введенными раньше. 
   
   Можно утверждать, что полное совпадение ВСМО объектов будет иметь 
место при полной идентичности ИОД и ИОЭ. Наличие отклонений ВСМО 
дубля от ВСМО эталона будет указывать на несовпадение содержания ИО, 
количественной оценкой которого является количество совместной 
информации. 
   
   Формирование ВСМО объектов может опираться на имеющиеся 
фундаментальные исследования в области морфологии русского языка и 
разработки\linebreak мощных инструментов морфологической структуризации. 
В~частности, выполнение морфологической структуризации в данном 
исследовании опирает\-ся на электронную версию словаря 
А.\,А.~Зализняка~\cite{10-k}, для практического использования которой 
разработаны оригинальные программные продукты. 

\section{Вероятностно-статистическая синтаксическая модель 
информационного объекта}
   
   С другой стороны, в русском языке классифицированы члены 
предложения. Определение членов предложения задает синтаксическую 
структуру русского языка. Важно подчеркнуть, что части речи обладают 
общностью синтаксических функций, так что эти два способа 
структуризации лексического состава русского языка взаимосвязаны и 
дополняют друг друга. Вероятностно-статистическая модель, синтезируемая 
на синтаксической основе, будет отличаться только системой случайных 
событий, образующих полное поле событий, т.\,е.\ шапкой таблицы, 
отражающей модель. 
   
   Между морфологической и синтаксической структуризацией имеется 
значительное отличие, сле\-ду\-ющее из того, что морфологическая 
структури\-за\-ция является фиксированной, так как имеется всего 10~час\-тей 
речи. Синтаксическая структуризация допускает введение более детальной 
структуры членов предложения. Она определяется разработчиком системы 
сравнения объектов и допускает определенный волюнтаризм в выборе 
системы случайных событий. Для возможности отражения семантических 
оттенков в систему случайных событий могут быть введены разнообразные 
синтаксические конструкции, связанные со спецификой содержания 
сравниваемых ИО. Понятно, что увеличение отражаемого в модели 
разнообразия конструкций, с одной стороны, будет способствовать 
повышению качества сравнения текстов, а с другой~--- усложнению модели. 
Однако если учесть табличное представление модели, стандартное для 
реляционных баз данных, то увеличение таблиц не приведет к 
принципиальным затруднениям при реализации систем оценки близости ИО. 
   
   Вероятностно-статистическая синтаксическая модель (ВССМ) ИО 
аналогична ВСММ ИО и может быть представлена в виде таблицы, подобной 
табл.~1. Шапка таблицы будет отражать принцип 
структурирования по случайным событиям, которые в ней связываются уже с 
членами предложения, т.\,е.\ с синтаксическими конструкциями языка. 
Алгебра вероятностной модели в этом случае будет определяться типами 
членов предложения, которые используются для представления ИО: 
$\aleph_C \hm= \{B_1, B_2, \ldots , B_L)$, где $B_1$~--- подлежащее; $B_2$~--- 
сказуемое; $B_3$~--- определение и~т.\,д.;
   $L$~--- общее число типов членов предложения, используемых в 
синтаксической модели и образующих в ней полное поле событий; 
   $B_l$, $l \hm= 1, 2, \ldots , L$, как и $A_j$, трактуются как случайные 
синтаксические события. Таким образом, ВССМ будет иметь вид: 
\begin{equation}
M_C = \{\Omega, \aleph_C,P(B)\}\,,
\label{e9-k}
\end{equation}
отличающийся от~(\ref{e6-k}) только алгеброй~$\aleph_C$.

Исследуемый ИО посредством какого-либо синтаксического анализатора 
разделяется на члены предложения. На основе этого разделения 
формируются случайные события и синтезируется вероятностно-статистический 
синтаксический образ (ВССО) ИО:
\begin{equation}
O_C=\{\W_O,\aleph_C,P_O(B)\}\,.
\label{e10-k}
\end{equation}
   
   Для оценки семантической близости ИО ВССО 
синтезируется для обоих сравниваемых объектов. На их основе может быть 
оценено количество взаимной информации в сравниваемых 
объектах~(\ref{e4-k}). Выражение для оценки количества взаимной 
информации получается из~(\ref{e8-k}) заменой морфологических образов 
син\-так\-си\-че\-скими:
   \begin{equation}
I_{\mathrm{СЭД}} = H(O_{\mathrm{СЭ}}) + H(O_{\mathrm{СД}})- 
H(O_{\mathrm{СЭ}}, O_{\mathrm{СД}})\,,
\label{e11-k}
\end{equation}
где обозначения совпадают с введенными раньше. 
   
   Синтаксический анализ текста представляет отдельную проблему, 
отличающуюся от рассматриваемой здесь. Поэтому для определения 
инструментов формирования синтаксических образов были 
проанализированы имеющиеся в литературе наработки в этом направлении и 
практически использовался <<Синтаксический анализатор Cognitive 
Dwarf 2.0>>~[11--13].

\vspace*{-0.9pt}

\section{Методология оценки семантической близости информационных объектов}
   
   В морфологической и в синтаксической структуре структурные 
компоненты несут достаточно определенную и близкую семантическую 
нагрузку. Поэтому могут быть установлены отношения эквивалентности 
между компонентами двух структур. Вследствие того, что эти структуры 
охватывают весь лексический состав и грамматический строй русского 
языка, они обеспечивают отражение семантического содержания текстов и, 
следовательно, являются достаточными для сопоставления этого 
семантического содержания. 
   
   Таким образом, ИО может быть представлен в виде 
морфологического\addtolength{\footnotesep}{1.351pt}\footnote{Минимальное количество грамматических терминов, 
используемых в статье, заимствовано из~\cite{8-k} с единственной целью: приблизить 
изложение терминологически к области русского языка, хотя содержание статьи, как 
представляется, достаточно далеко от вопросов собственно языка.}\addtolength{\footnotesep}{-1.351pt} и/или 
синтаксического образа. Формальное представление ИО в виде 
математических объектов~--- ве\-ро\-ят\-но\-ст\-но-ста\-ти\-сти\-че\-ских 
образов~--- позволяет использовать математический аппарат для получения 
количественных оценок их близости. Можно утверждать, что полное 
совпадение как ВСМО, так и ВССО со\-по\-став\-ля\-емых объектов будет 
соответствовать равенству представляемых ими ИО. 

Оценка степени близости ИО, представленных на естественном языке, может 
быть реализована на основе применения вероятностно-статистической 
морфологической и/или синтаксической модели. 
   
   Мерой степени близости служит энтропия и взаимная информация, 
количественные значения которых вычисляются по~(\ref{e2-k}), (\ref{e8-k}) 
и~(\ref{e11-k}). Выше отмечалось, что под количеством информации в теории
информации понимается количество неопределенности случайного объекта, 
которое исчезает при выяснении этой неопределенности. Неопределенность 
объекта характеризуется распределением его вероятностей. 
В~использованном выше примере объекта~$\xi$ с синонимами было задано 
распределение вероятностей: $p(\xi=\;\mbox{\textit{пример}}) \hm= 0{,}2$; 
$p(\xi\hm=\;\mbox{\textit{образец}}) \hm= 0{,}4$; $p(\xi\hm=\;\mbox{\textit{экспонат}}) 
\hm= 0{,}6$. Так что в этом случае энтропия отдельных исходов будет: 
$H(\xi=\;\mbox{\textit{пример}}) \hm=- \log 0{,}2$, 
$H(\xi=\;\mbox{\textit{образец}}) \hm=- \log 0{,}4$, а 
$H(\xi=\;\mbox{\textit{экспонат}}) \hm=- \log 0{,}6$, а\linebreak усредненная энтропия 
$H_\xi \hm=- 0{,}2 \log 0{,}2\hm- 0{,}4 \log 0{,}4\hm - 0{,}6 \log 0{,}6$. Таким образом, 
энтропия будет некоторым числом, зависящим от распределения 
вероятностей случайных событий, но не от их содержания. 
   
   Из выражений~(\ref{e8-k}) и (\ref{e11-k}) для взаимной информации 
можно видеть, что, во-пер\-вых, она тоже является числом, так как 
выражается через числа~--- значения соответствующих энтропий. 
   Во-вто\-рых, выражения~(\ref{e8-k}) и~(\ref{e11-k}) отражают смысл 
взаимной информации как меры неопределенности. 

Пусть используются 
синтаксические образы со\-по\-став\-ля\-емых объектов, а их взаимная информация 
оценивается выражением~(\ref{e11-k}). Рассмотрим два предельных случая. 
   
   В первом пусть ВССО эталона $O_{\mathrm{СЭ}}$ не имеет ничего общего 
с ИССО дубля~$O_{\mathrm{СД}}$. Отсутствие общего означает, что 
вероятность совместного распределения $P(O_{\mathrm{СЭ}},O_{\mathrm{СД}}) 
\hm= 0$. В~теории информации принято считать $0 \log 0=0$, поэтому 
$H(O_{\mathrm{СЭ}}, O_{\mathrm{СД}}) \hm= 0$
 и из~(\ref{e11-k}) следует, что 
количество совместной информации, содержащееся в двух со\-по\-став\-ля\-емых 
объектах, равно их общей неопределенности: $I_{\mathrm{СЭД}} \hm= 
H(O_{\mathrm{СЭ}}) \hm+ H(O_{\mathrm{СД}})$.
   
   Во втором случае пусть дубль полностью совпадает с эталоном, т.\,е.\ 
$O_{\mathrm{СЭ}}\hm=O_{\mathrm{СД}}$. Тогда совпадут энтропии 
$H(O_{\mathrm{СЭ}}) \hm= H(O_{\mathrm{СД}})$, более того, и совместная 
энтропия $H(O_{\mathrm{СЭ}}, O_{\mathrm{СД}})$ будет равна энтропии 
эталона или дубля. Так что количество совместной информации будет равно 
$I_{\mathrm{СЭД}} = H(O_{\mathrm{СЭ}})$, т.\,е.\ неопределенность дубля 
отсутствует, вся неопределенность связана только с неопределенностью 
эталона, только с количеством заключенной в нем информации. Отсюда 
можно заключить, что количество информации~(\ref{e11-k}) изменяется от 
значения\linebreak
$I_{\mathrm{СЭД}} \hm= H(O_{\mathrm{СЭ}})$ до значения 
$I_{\mathrm{СЭД}}\hm = H(O_{\mathrm{СЭ}}) \hm+ H(O_{\mathrm{СД}})$. При 
этом взаимная информация~---\linebreak
 величина положительная. Это следует из 
положительности энтропий: 
$
p(\xi) \leq 1$, $\log p(\xi)\hm\leq  0
$ и, следовательно, 
$$
H(\xi) =- \log p(\xi) \geq 0
$$ 
и факта 
$$
H(O+{\mathrm{СЭ}})  + H(O_{\mathrm{СД}})  \geq H(O_{\mathrm{СЭ}}, O_{\mathrm{СД}})\,.
$$ 

Разумеется, такой же результат может быть получен и для~(\ref{e8-k}). 
Вследствие свойств логарифмической функции количество информации 
изменяется монотонно в определенных выше пределах, что и позволяет 
использовать его в качестве меры семантической близости ИО. 
   
   Для практического применения абстрактные значения энтропии и 
взаимной информации необходимо проградуировать в некоторых понятных и 
связанных с содержательным существом задачи мерах оценки семантической 
близости ИО. 
%
Такая градуировка (тарирование) их значений может 
осуществляться разными способами и, в частности, обеспечивать реализацию 
функций адаптации сис\-те\-мы к различным задачам и типам ИО. Например, в 
простейшем случае может быть взят реальный эталонный ИО такого типа, 
для работы с которым предполагается использовать систему. 
%
Искажением 
эталонного ИО случайным образом и в заданных объемах может быть 
получена серия дублей с известной степенью семантического несовпадения. 
Для каждой пары <<эта\-лон--дубль>> находится значение взаимной 
информации, которое сопоставляется с известной степенью семантического 
несовпадения. На основании сопоставления определяется линейное 
преобразование перевода количества информации в удобную меру оценки 
степени семантического соответствия ИО.
   
   Методология реализуется в виде последовательности следующих этапов: 
   \begin{itemize}
\item выбор вида и формирование вероятностно-ста\-ти\-сти\-че\-ской модели 
конкретизацией ал\-геб\-ры (системы случайных событий) $\aleph_M$ или 
$\aleph_C$;
\item введение содержания (текстов) образов ИО;
\item формирование образа эталонного ИО в виде $O_{\mathrm{МЭ}}$ или 
$O_{\mathrm{СЭ}}$;
\item формирование образа второго ИО (дубля) в виде $O_{\mathrm{МД}}$ или 
$O_{\mathrm{СД}}$; 
\item определение энтропии эталонного ИО в виде $H(O_{\mathrm{МЭ}})$ или 
$H(O_{\mathrm{СЭ}})$;
\item определение энтропии второго ИО (дубля) в виде $H(O_{\mathrm{МД}})$ 
или $H(O_{\mathrm{СД}})$;
\item определение совместной энтропии эталонного ИО и дубля в виде 
$H(O_{\mathrm{МЭ}}, O_{\mathrm{МД}})$ или $H(O_{\mathrm{СЭ}}, 
O_{\mathrm{СД}})$; 
\item определение совместной информации в соответствии с~(\ref{e8-k}) 
или/и в соответствии с~(\ref{e11-k});
\item перевод совместной информации в выбранную меру информационной 
близости ИО.
\end{itemize}

   Разработанная методология оценки семантической близости ИО, 
кажущаяся, на первый взгляд, совершенно от семантики оторванной, имеет 
глубокую содержательную основу. Если следовать более общему 
представлению языка, чем детальное грамматическое, то множества 
элементарных исходов (слов), образующих в вероятностно-ста\-ти\-сти\-че\-ских 
образах случайные события (час\-ти речи в ВСМО и члены предложения в 
ВССО)\linebreak могут трактоваться как соответствующие обобщенные час\-ти речи и 
члены предложения, образующие эти образы. Такое представление позволяет 
выделить главное содержание в сравниваемых ИО.\linebreak Содержательным 
примером, подтверждающим реальность разработанного подхода, является 
достаточно час\-то встречающееся в реальности продуктивное общение людей 
на плохо знакомом им \mbox{языке}. Они не владеют склонениями, спряжениями, 
формами времени и другими элементами грамматики, но, зная две--три сотни 
слов, достаточно успешно общаются, вполне понимая друг друга. 
   
   Здесь излагается ядро методологии и не рассматриваются возможности 
привлечения дополнительных инструментов, повышающих адекватность 
оценки, таких как использование синонимов, введение весовых 
коэффициентов, детализация и комбинация событий и~т.\,п. 
   
   Заметим еще, что такой подход может быть использован и для оценки 
близости ИО, реализованных на других языках и с использованием иных 
алфавитов. Другие языки, например английский или немецкий, отличаются 
от русского в сторону уменьшения свободы в порядке слов и разнообразия 
способов управления, что упрощает задачи их структурирования и 
построения вероятностно-ста\-ти\-сти\-че\-ских моделей ИО, не требуя изменения 
методологии. 
   
   В ряде случаев ИО могут быть представлены с использованием не 
естественного языка, а, например, формального математического языка 
формул. В~этом случае изменяется входной алфавит и, возможно, принцип 
синтеза алгебры случайных событий. Но эти изменения не касаются 
представленной здесь собственно методологии оценки подобия ИО. 

\vspace*{-6pt}
   
\section{Оценка знаний}

\vspace*{-2pt}
   
   Одной из проблем, для решения которой предпринята данная разработка, 
является автоматизированный контроль знаний. Использование для этой 
цели системы тестов представляется автору неприемлемым по множеству 
причин. На кафедре АСУ Липецкого государственного технического 
университета разрабатывается <<Автоматизированная система поддержки 
образовательной программы обучения>> (АСПОП)~\cite{14-k}, одним из 
важнейших компонентов которой является подсистема автоматизированного 
контроля знаний. Концепция подсистемы базируется на изложенной 
методологии. 
   
   Практическая проверка в минимально возможном объеме 
работоспособности принципиальных положений концепции осуществлена 
проверкой знаний студентов. При реализации проверки студентам на экране 
демонстрировался эталонный ответ из АСПОП, который они воспроизводили 
на память и заносили в компьютер. По эталонным ответам из АСПОП 
формировались ВСМО $O_{\mathrm{МЭ}}$ и ВССО $O_{\mathrm{СЭ}}$. По 
ответам студентов формировались соответствующие ВСМО 
$O_{\mathrm{МД}}$ и ВССО $O_{\mathrm{СД}}$. По морфологическим образам 
$O_{\mathrm{МЭ}}$ и $O_{\mathrm{МД}}$ определялись энтропии 
$H(O_{\mathrm{МЭ}})$, $H(O_{\mathrm{МД}})$ и $H(O_{\mathrm{МЭ}}, 
O_{\mathrm{МД}})$, а по ним оценивалось количество взаимной информации. 
Также обрабатывались синтаксические образы эталонных ответов и их 
дублей~--- ответов студентов. 
   
   Распечатанные эталонные ответы и ответы студентов анонимно 
сопоставлялись группой преподавателей, которые выставляли оценки 
студентам по существующей методике по 100-балль\-ной шкале. Оценки 
преподавателей надлежащим образом усреднялись. По оценкам 
преподавателей и количествам взаимной информации, определенным 
автоматизированной системой, определялись параметры масштабного 
преобразования количества информации в принятые в университете 
   100-балль\-ные оценки. После введения коэффициентов масштабного 
преобразования система, как и преподаватели, выдавала 100-балль\-ные 
оценки. 
   
   Оценки, автоматически сформированные сис\-те\-мой, были сопоставлены с 
оценками, вы\-став\-лен\-ны\-ми преподавателями. В~итоге было получено, что 
среднее квадратическое отклонение оценок, вычисленных системой на 
основании сопоставлений вероятностно-статистических образов эталона и 
ответа по изложенной методологии, от оценок, выставленных 
преподавателями, по 100-балль\-ной шкале составило 10\%--15\%. 
Результаты со\-по\-став\-ле\-ния ВСМО и ВССО эталона и ответа оказались 
достаточно близкими. Отметим, что это была пробная проверка, 
предпринятая исключительно для обретения уверенности в практической 
эффективности оригинальной концепции.
   
   Углубление и детализация вероятностно-ста\-ти\-сти\-че\-ских моделей ИО на 
естественном языке, их исследование и применение представляют 
неограниченное, научно новое и практически полезное поле деятельности, в 
освоении которого автор может оказать посильную помощь. 

\vspace*{-6pt}
   
\section{Заключение}

\vspace*{-2pt}

   Разработана оригинальная методология оценки степени семантической 
близости инфор\-ма\-ционных объектов. Методология может служить 
   фор\-маль\-но-ма\-те\-ма\-ти\-че\-ской основой в сфере современных 
информационных технологий для решения разнообразных задач сравнения и 
оценки подобия информационных объектов, представленных на 
естественном языке. Методология вводит ве\-ро\-ят\-но\-ст\-но-ста\-ти\-сти\-че\-скую 
модель представления русскоязычного текста и определяет способы 
представления текстов в виде ве\-ро\-ят\-но\-ст\-но-ста\-ти\-сти\-че\-ских 
морфологических и синтаксических образов, которые позволяют оценить 
количественно и объем информации в информационных объектах, и степень 
их семантического совпадения. Экспериментальная прикидочная проверка 
показала эффективность применения методологии для разработки 
автоматизированных систем оценки знаний. Практическое применение 
методологии только в этой сфере может привести к принципиальным 
изменениям в сфере образования. 

\vspace*{-6pt}

{\small\frenchspacing
{%\baselineskip=10.8pt
\addcontentsline{toc}{section}{Литература}
\begin{thebibliography}{99}

\bibitem{1-k}
\Au{Друкер П.}
Посткапиталистическое общество. Новая постиндустриальная волна на Западе: 
Антология~/ Под ред. В.\,Л.~Иноземцева.~--- М.: Academia, 1990. 

\bibitem{2-k}
\Au{Мельчук И.\,А.}
Опыт теории лингвистических моделей <<Смысл\;$\leftrightarrow$\;Текст>>.~--- 
2-е изд.~--- М.: Школа <<Языки русской культуры>>, 1999.

\bibitem{3-k}
\Au{Шеннон К.}
Математическая теория связи. 1948~// Работы по теории информации и 
кибернетике~/ Пер. с англ. под ред. Р.\,Л.~Добрушина и О.\,Б.~Лупанова.~--- 
М.: ИЛ, 1963.

\bibitem{4-k}
\Au{Колмогоров А.\,Н.}
Теория информации и теория алгоритмов.~--- М.: Наука, 1987.

\bibitem{5-k}
\Au{Стратонович Р.\,Л.} Теория информации.~--- М.: Сов. радио, 1975.

\bibitem{6-k}
\Au{Кузнецов Л.\,А.}
Введение в САПР производства проката.~--- М.: Металлургия, 1991.

\bibitem{7-k}
\Au{Kuznetsov L.\,A.}
The entropy and information application to identify fuzzy sets~//  ICSC Symposium 
(International) on Fuzzy Logic Proceedings.~---  
Academic Press, 1995. P.~A109--A111.

\bibitem{8-k}
\Au{Белоусов В.\,Н., Ковтунова И.\,И., Кручинина~И.\,Н. и~др.}
Краткая русская грамматика~/ Под ред. Н.\,Ю.~Шведовой и 
В.\,В.~Лопатина~--- М.: Рус. яз., 1989.

\bibitem{9-k}
\Au{Гнеденко Б.\,В.}
Курс теории вероятностей: Учебник.~--- 9-е изд., испр.~--- М.: ЛКИ, 2007.

\bibitem{10-k}
\Au{Зализняк А.\,А.}
Грамматический словарь русского языка: Словоизменение.~--- 3-е изд., 
стер.~--- М.: Рус. яз.,1987. 880~с.
\bibitem{11-k}
Синтаксический анализатор Cognitive Dwarf 2.0. {\sf 
http://cs.isa.ru:10000/dwarf/d2/dw2.html}.

\bibitem{12-k}
\Au{Антонова А.\,А., Мисюрев~А.\,В.}
Реализация синтаксического разбора для русского и английского языков~// 
Системный анализ и информационные технологии (САИТ 2005): Мат-лы 
I~Междунар.\ конф.~--- Пе\-ре\-славль-За\-лес\-ский, 2005.~--- 
Переславль-Залесский, 2005. С.~245--249.

\bibitem{13-k}
\Au{Антонова А.\,А., Мисюрев А.\,В.}
Синтаксический анализатор для русского и английского языков~// Сб. трудов 
ИСА РАН~/ Под ред. В.\,Л.~Арлазарова и Н.\,Е.~Емельянова.~--- М.: УРСС, 
2007.

\label{end\stat}

\bibitem{14-k}
\Au{Кузнецов Л.\,А., Фарафонов А.\,С., Тищенко~А.\,Д., Капнин~А.\,В.}
Автоматизированная система поддержки образовательной программы 
обучения~// Качество. Инновации. Образование, 2010. №\,9. С.~12--20. 
 \end{thebibliography}
}
}


\end{multicols}               %11Abst+avt
\def\stat{kor-kor}



\def\tit{МОДИФИЦИРОВАННЫЙ СЕТОЧНЫЙ МЕТОД РАЗДЕЛЕНИЯ ДИСПЕРСИОННО-СДВИГОВЫХ
СМЕСЕЙ НОРМАЛЬНЫХ ЗАКОНОВ$^*$}



\def\titkol{Модифицированный сеточный метод разделения дисперсионно-сдвиговых
смесей нормальных законов}

\def\aut{В.\,Ю.~Королев$^1$,  А.\,Ю.~Корчагин$^2$}

\def\autkol{В.\,Ю.~Королев,  А.\,Ю.~Корчагин}

\titel{\tit}{\aut}{\autkol}{\titkol}

{\renewcommand{\thefootnote}{\fnsymbol{footnote}} \footnotetext[1]
{Работа поддержана Российским научным фондом (проект 14-11-00364).}}


\renewcommand{\thefootnote}{\arabic{footnote}}
\footnotetext[1]{Факультет
вычислительной математики и кибернетики Московского государственного
университета им.\ М.\,В.~Ломоносова; Институт проблем информатики
Российской академии наук; victoryukorolev@yandex.ru}
\footnotetext[2]{Факультет вычислительной математики и кибернетики
Московского государственного университета им.\ М.\,В.~Ломоносова;
sasha.korchagin@gmail.com}

%\vspace*{2pt}



\Abst{Описывается модифицированный двухэтапный
сеточный метод разделения дис\-пер\-си\-он\-но-сдви\-го\-вых смесей нормальных
законов, представляющий собой альтернативу чистому ЕМ (expectation-maximization)
ал\-го\-рит\-му. На
первом этапе этого алгоритма строится дискретная аппроксимация для
смешивающего распределения, на втором этапе подбирается абсолютно
непрерывное распределение из заранее заданного семейства, например,
обобщенных обратных гауссовских законов, ближайшее к~дискретному
распределению, полученному на первом этапе. Обсуждаются вопросы
сходимости этого двухэтапного алгоритма. Доказана монотонность
сеточного итерационного метода, используемого на первом этапе.
Подробно обсуждается вопрос оптимального выбора параметров метода,
прежде всего сетки, накидываемой на носитель смешивающего
распределения. С~этой целью предложены статистические оценки
квантилей смешивающего распределения. Эффективность метода
иллюстрируется примерами конкретных вычислений оценок параметров
обобщенных гиперболических распределений.}

\KW{смесь распределений вероятностей;
дис\-пер\-си\-он\-но-сдви\-го\-вая смесь нормальных законов; обобщенное
гиперболическое распределение; ЕМ-ал\-го\-ритм; сеточный метод
разделения смесей}

\vspace*{1pt}

%\vspace*{2pt}

\DOI{10.14357/19922264140402}


\vskip 12pt plus 9pt minus 6pt

\thispagestyle{headings}

\begin{multicols}{2}

\label{st\stat}

\section{Введение}

При {\it практическом} решении задачи моделирования и исследования
волатильности (изменчивости) хаотических стохастических процессов
ключевым этапом является статистическое разделение смесей
вероятностных распределений. Задача разделения смесей~---
статистического оценивания параметров смесей вероятностных
распределений~--- в~деталях разобрана, например, в~книге~\cite{k2011}.

Для решения задачи разделения смесей вероятностных распределений
традиционно используются итерационные процедуры типа ЕМ-ал\-го\-рит\-ма.
К~сожалению, классический ЕМ-ал\-го\-ритм обладает рядом серьезных
недостатков при его применении к~смесям нормальных законов, а~именно:
он демонстрирует крайнюю неустойчивость по отношению к~исходным
данным и~начальным приближениям.

Для преодоления этих недостатков
предложено много модификаций ЕМ-ал\-го\-рит\-ма (см., например,~\cite{k2011}).
Вместе с тем в~указанной книге предложен и~исследован
принципиально новый~--- сеточный~--- метод приближенного решения
задачи разделения смесей. В~работе~\cite{n2013} подробно исследованы
вопросы сходимости сеточных методов разделения смесей.

В соответствии с подходом к~статистическому анализу хаотических
стохастических процессов, в~частности к~решению задачи декомпозиции
волатильности таких процессов, развитом в~книге~\cite{k2011},
в~общем случае на практике приходится решать задачу разделения
конечных смесей нормальных законов с~произвольно большим числом
неизвестных параметров (параметров компонент и~их весов).
И~хотя в~большинстве приложений возникают смеси не более чем с~пятью--семью
компонентами, даже при использовании таких смесей, скажем, в~задачах
анализа и~прогнозирования финансовых рисков приходится моделировать
траекторию движения точки в~пространствах, размерность которых
соответственно лежит в~пределах от~14 (для пятикомпонентных смесей)
до~20 (для семикомпонентных смесей), что существенно увеличивает
вычислительные и~временн$\acute{\mbox{ы}}$е ресурсы, необходимые для практического
решения указанных задач.

Поскольку во многих ситуациях (например,
при прогнозировании на основе высокочастотных данных) эти задачи
необходимо решать в~режиме, близком к~реальному времени, для
создания эффективных методов статистического анализа на основе
смешанных моделей на первый план выходит проб\-ле\-ма снижения
размерности решаемой задачи, т.\,е.\ параметрического пространства.

Одним из возможных подходов к~снижению размерности является
априорное сужение классов допусти\-мых смесей. К~примеру, при решении
многих задач, связанных с~анализом процессов атмосферной или
плазменной турбулентности, а~так\-же процессов, описывающих эволюцию
различных финансовых индексов, высочайшую адекватность
продемонстрировали модели, основанные на дис\-пер\-си\-он\-но-сдви\-го\-вых
смесях нормальных законов. Класс таких смесей очень обширен
и,~в~част\-ности, включает в~себя обобщенные гиперболические распределения,
которые были введены О.-Е.~Барн\-дорфф-Ниль\-се\-ном в~1977--1978~гг.\ как
класс специальных сдвиг-мас\-штаб\-ных смесей нормальных законов~\cite{BN1977, BN1978}.
Пусть $\alpha\hm\in\r$, $\beta\hm\in\r$. Если
функцию распределения обобщенного гиперболического закона
с~параметрами~$\alpha$, $\beta$, $\nu$, $\mu$, $\lambda$ обозначить
$P_{GH}(x;\alpha,\beta,\nu,\mu,\lambda)$, то по определению
\begin{multline}
P_{GH}(x;\alpha,\beta,\nu,\mu,\lambda)={}\\
{}=
\int\limits_{0}^{\infty}\Phi\left(\fr{x-\beta-\alpha
z}{\sqrt{z}}\right)\,p_{GIG}(z;\nu,\mu,\lambda)\,dz\,,\\
x\in\r\,,
\label{e1-kor}
\end{multline}
где $\Phi(x)$~--- стандартная нормальная функция распределения:
$$
\Phi(x)=\int\limits_{-\infty}^{x}\varphi(z)\,dz\,,\enskip
\varphi(x)=\fr{1}{\sqrt{2\pi}}e^{-x^2/2}\,,\enskip  x\in\mathbb{R}\,;
$$
$p_{GIG}(x;\nu,\mu,\lambda)$~--- плот\-ность обобщенного обратного
гауссовского распределения:
\begin{multline*}
p_{GIG}(x;\nu,\mu,\lambda)={}\\
{}=\fr{\lambda^{\nu/2}}{2\mu^{\nu/2}
K_{\nu}\left(\sqrt{\mu\lambda}\right)}\,
x^{\nu-1}\exp\left\{-\fr{1}{2}\left(\fr{\mu}{x}+\lambda
x\right)\right\}\,,\\ x>0\,.
\end{multline*}
Здесь $\nu\in\r$;
$$
\begin{array}{lll}
\mu>0\,, & \lambda\geqslant0\,, & \mbox{если }\nu<0\,;\\[6pt]
\mu>0\,, & \lambda>0\,, & \mbox{если }\nu=0\,;\\[6pt]
\mu\geqslant0\,, & \lambda>0\,, & \mbox{если }\nu>0\,;
\end{array}
$$
$K_{\nu}(z)$~--- модифицированная бесселева функция третьего рода
порядка~$\nu$:

\noindent
\begin{multline*}
K_{\nu}(z)=\fr{1}{2}\int\limits_{0}^{\infty}y^{\nu-1}\exp
\left\{-\fr{z}{2}\left(y+\fr{1}{y}\right)\right\}\,dy\,,\\
z\in\mathbb{C}\,,\enskip \mathrm{Re}\,z>0\,.
\end{multline*}
Обратим внимание, что в~(1) смешивание происходит одновременно и~по
параметру сдвига, и~по параметру масштаба, но так как эти параметры
в~(1)  связаны жесткой зависимостью, так что параметр сдвига
смешиваемого распределения пропорционален его дисперсии, то
фактически смесь~(1) является {\it однопараметрической} и~поэтому
называется {\it дис\-пер\-си\-он\-но-сдви\-го\-вой} (см., например,~\cite{BN1982}).

Другим примером дис\-пер\-си\-он\-но-сдви\-го\-вых смесей нормальных законов
являются обобщенные дисперсионные гам\-ма-рас\-пре\-де\-ле\-ния, в~которых
смешивающими являются обобщенные гам\-ма-рас\-пре\-де\-ле\-ния~\cite{ks2012, zk2013}.

В указанных семействах смесей число неизвестных параметров равно
пяти или шести (если\linebreak учитывать неслучайный сдвиг). Вместе
с~тем у~подоб\-ных моделей имеются довольно серьезные тео\-ре\-ти\-че\-ские
обоснования: в~работах~\cite{zk2013, k2013} показано, что указанные
модели являются асимптотическими аппроксимациями в~простой
предельной схеме случайного суммирования и~потому могут успешно
применяться для анализа процессов типа остановленных случайных
блужданий. Эти выводы подтверждены статистическим анализом
вы\-со\-ко\-час\-тот\-ных финансовых данных, в~результате которого выявлен
синхронизированный характер изменения интенсивностей потоков заявок
в~сис\-те\-мах электронных торгов, что естественно приводит к~синхронизированному
поведению па\-ра\-мет\-ров сдвига и~диффузии в~соответствующих моделях вида смесей
нормальных законов~\cite{kckg2013}.

\section{Описание моди\-фи\-ци\-ро\-ван\-но\-го
сеточного ме\-то\-да разделения дисперсионно-сдвиговых смесей
нормальных законов и~его свойства}

Оказывается, что сеточные методы разделения смесей довольно
эффективны не только при разделении конечных смесей нормальных
законов, но и~при разделении произвольных дис\-пер\-си\-он\-но-сдви\-го\-вых
смесей нормальных законов. Поясним сказанное на примере задачи
оценивания па\-ра\-мет\-ров обобщенных гиперболических распределений.

Для решения задачи оценивания параметров обобщенных гиперболических
распределений традиционно используется метод, предложенный в~статье~\cite{p2004}
и~по сути являющийся классическим ЕМ-ал\-го\-рит\-мом,
приспособленным к~конкретной задаче, и,~соответственно, наследующий
присущие ЕМ-ал\-го\-рит\-мам недостатки.

Рассмотрим следующий альтернативный двухэтапный метод. На первом
этапе на поло\-жи\-тельной полупрямой выделим основную часть носителя
смешивающего распределения, т.\,е.\ \mbox{ограниченный} интервал,
вероятность которого, вычисленная в~соответствии со смешивающим
распределением, практически равна единице. На этот интервал накинем
конечную сетку, содержащую, возможно, очень много {\it известных}
узлов $u_1,\ldots,u_K$. Считая параметр сдвига~$\beta$ равным нулю,
приблизим искомое обобщенное гиперболическое распределение конечной
смесью нормальных законов:

\noindent
\begin{multline}
P_{GH}(x;\,\alpha,0,\nu,\mu,\lambda)\approx{}\\
{}\approx \sum\limits_{i=1}^K
p_i\Phi\left(\fr{x-\alpha u_i}{\sqrt{u_i}}\right)\,,\enskip
x\in\mathbb{R}\,.\label{e2-kor}
\end{multline}
В смеси, стоящей в~правой части соотношения~(2), неизвестными
являются только параметры $p_1,\ldots,p_{K-1}$ и~$\alpha$. Пусть
$x_1,\ldots,x_n$~--- анализируемая выборка значений случайной
величины с~оцениваемым обобщенным гиперболическим распределением.
Итерационный процесс, определяющий сеточный ЕМ-ал\-го\-ритм для данной
задачи, задается следующим образом. Пусть
$p_1^{(m)},\ldots,p_{K-1}^{(m)}$ и~$\alpha^{(m)}$~--- оценки параметров
$p_1,\ldots,p_{K-1}$ и~$\alpha$ на $m$-й итерации,
$p_K^{(m)}\hm=1\hm-p_1^{(m)}-\cdots-p_{K-1}^{(m)}$. Обозначим

\noindent
\begin{align*}
\varphi_{ij}^{(m)}&=\fr{1}{\sqrt{u_i}}\varphi\left(\fr{x_j-\alpha^{(m)}u_i}{\sqrt{u_i}}\right)\,;
\\
g_{ij}^{(m)}&=\fr{p_i^{(m)}\varphi_{ij}^{(m)}}{\sum\limits_{r=1}^K
p_r^{(m)}\varphi_{rj}^{(m)}}\,,\\
&\hspace*{14mm}i=1,\ldots,K\,;\enskip j=1,\ldots,n\,.
\end{align*}
Тогда, используя стандартные рассуждения, определяющие
вычислительные формулы EM-ал\-го\-рит\-ма для параметров конечной смеси
нормальных законов (см, например,~[1, разд.~5.3.7--5.3.8]),
следует положить

\noindent
\begin{equation}
p_i^{(m+1)}=\fr{1}{n}\sum\limits_{j=1}^n g_{ij}^{(m)}\,, \enskip
i=1,\ldots,K\,.\label{e3-kor}
\end{equation}
Обозначим $\overline{x}=(1/n)\sum\limits_{j=1}^nx_j$. Используя
соотношение~(5.3.24) в~\cite{k2011}, с~учетом очевидного равенства
$\sum\limits_{i=1}^K g_{ij}^{(m)}\hm=1$ можно заметить, что уточненная
оценка параметра~$\alpha$ имеет вид:

\columnbreak

\noindent
\begin{equation}
\alpha^{(m+1)}=\fr{\overline{x}}{\sum\limits_{i=1}^K u_ip_i^{(m+1)}}\,,
\label{e4-kor}
\end{equation}
т.\,е.\ равна отношению генерального выборочного среднего и~текущего
эмпирического среднего смешивающего распределения, что вполне
согласуется с~тем, что в~соответствии с~приводимым ниже соотношением~(\ref{e5-kor})
в~данном случае ${\sf E}X\hm=\alpha{\sf E}U$.

В силу монотонности классического ЕМ-ал\-го\-рит\-ма справедливо следующее
утверждение.

\smallskip

\noindent
\textbf{Теорема~1.} {\it Пусть узлы $u_1,\ldots,u_K$ сетки различны,
неотрицательны и~известны. Тогда итерационный процесс $(3)$--$(4)$
является монотонным, т.\,е.\ каждая его итерация не уменьшает
целевую сеточную функцию правдоподобия}
\begin{multline*}
L(p_1,\ldots,p_K,\alpha;x_1,\ldots,x_n)={}\\
{}=
\prod\nolimits_{j=1}^n\left[\sum\nolimits_{i=1}^K
\fr{p_i}{\sqrt{u_i}}\,\varphi\left(\fr{x_j-\alpha^{(m)}u_i}{\sqrt{u_i}}\right)\right].
\end{multline*}

\smallskip

\noindent
\textbf{Замечание~1.} В~разд.~5.7.4 книги~\cite{k2011} показано, что
при каждом фиксированном значении параметра~$\alpha$ сеточная
функция правдоподобия\linebreak
$L(p_1,\ldots,p_{K-1},\alpha;\,x_1,\ldots,x_n)$ вогнута по
аргументам $p_1,\ldots,p_{K-1}$. Поэтому на каждом шаге
итерационного процесса вместо соотношения~(3) можно\linebreak использо\-вать
любой более быстрый алгоритм максимизации функции
$L(p_1,\ldots,p_{K-1},\alpha^{(m)};\,x_1,\ldots$\linebreak $\ldots,x_n)$ по переменным
$p_1,\ldots,p_{K-1}$. Например, оценки весов $p_1,\ldots,p_K$ можно
искать методом условного градиента~\cite{k2011, kn2010}.

\smallskip

Таким образом, на первом этапе получаются оценки параметра~$\alpha$
и~весов всех узлов~$u_i$ конечной сетки, накинутой на носитель
смешивающего обобщенного обратного гауссовского распределения
$P_{\mathrm{GIG}}(z;\,\nu,\mu,\lambda)$.

На втором этапе остается применить ка\-кой-ли\-бо стандартный метод
подгонки обобщенного обратного гауссовского распределения
$P_{\mathrm{GIG}}(z;\,\nu,\mu,\lambda)$ к~эмпирическим данным типа
гистограммы $(u_1, p_1),\ldots, (u_K, p_K)$. Например, параметры~$\nu$,
$\mu$ и~$\lambda$ можно оценить, минимизируя соответствующую
статистику хи-квад\-рат. Или же, например, можно решить задачу
наименьших квад\-ратов:
\begin{multline*}
(\nu^*,\mu^*,\lambda^*)={}\\
{}=\arg\min\limits_{\nu,\mu,\lambda}\sum\limits_{i=1}^K
\left[p_i- \!\!\!\!\!
\int\limits_{(1/2)\left(u_{i-1}+u_i\right)}^{(1/2)(u_i+u_{i+1})}\!\!\!\!\!\!\!\!\!\!\!\!\!\!\!
p_{GIG}(u;\,\nu,\mu,\lambda)\,du\right]^2,
\end{multline*}
где $u_0=0$; $u_{K+1}\hm=\infty$.

На практике хорошие результаты показал подход с решением задачи
наименьших квадратов. Для поиска параметров использовался алгоритм
ns2sol, описанный в~книге~\cite{DSch1983}. Указанный алгоритм
доступен во многих статистических пакетах, отличается высоким
быстродействием и~возможностью при желании задавать разумные
интервалы для поиска параметров.

%\vspace*{-9pt}

\section{О практическом выборе сетки
на~первом этапе моди\-фи\-ци\-ро\-ван\-но\-го
сеточного метода разделения дисперсионно-сдвиговых смесей нормальных
законов}

Естественно, что при использовании указанного двухэтапного метода
в~динамическом режиме крайне важным становится вопрос о~выборе
наиболее эффективных и~быстродействующих численных процедур и~их
параметров. В~частности, исключительную важность приобретает
правильный выбор сетки на первом этапе. Рассмотрим этот вопрос
подробнее.

Формально рассматриваемая задача выглядит так: по наблюдаемым
значениям $x_1,\ldots,x_n$ требуется построить статистическую оценку
верхней границы квантилей заданного порядка сме\-ши\-ва\-юще\-го закона так,
чтобы как можно точнее оценить носитель смешивающего распределения.

В дальнейшем будем считать, что $x_1,\ldots,x_n$~--- независимые
реализации случайной величины $X\hm=Y\sqrt{U}+\alpha U$, где $Y$~---
случайная величина со стандартным нормальным распределением, а~$U$~---
независимая от нее случайная величина с~обобщенным обратным
гауссовским распределением. Тогда, очевидно, распределение случайной
величины~$X$ имеет вид~(1). Предположим, что у~случайной величины~$U$
существуют моменты первых двух порядков. Тогда, как несложно видеть,
\begin{equation}
{\sf E}X={\sf E}Y\cdot{\sf E}\sqrt{U}+\alpha{\sf E}U=\alpha{\sf
E}U\,.\label{e5-kor}
\end{equation}
При этом по усиленному закону больших чисел с~вероятностью единица
$\overline x\hm\longrightarrow {\sf E}X$ $(n\hm\to\infty)$, так что при
больших~$n$ справедливо приближенное равенство ${\sf E}X\hm\approx\overline x$
и~с учетом~(\ref{e5-kor})
\begin{equation}
{\sf E}U\approx\fr{\overline x}{\alpha}\,.\label{e6-kor}
\end{equation}
Далее, очевидно,

\columnbreak

\noindent
\begin{multline}
{\sf E}X^2={\sf E}Y^2\cdot{\sf E}U+2\alpha{\sf E}X\cdot{\sf E}U^{3/2}+{}\\
{}+
\alpha^2{\sf E}U^2={\sf E}U+\alpha^2{\sf E}U^2\,.
\label{e7-kor}
\end{multline}

\noindent
Поэтому, обозначив
$$
m^2=\fr{1}{n}\sum\limits_{i=1}^nx_i^2\,,
$$
получаем приближенное равенство ${\sf E}X^2\hm\approx m^2$, так что
с~учетом~(\ref{e6-kor}) и~(\ref{e7-kor}) имеем:
\begin{equation}
{\sf E}U^2\approx\fr{1}{\alpha^2}\left(m^2-\fr{\overline
x}{\alpha}\right)\,.\label{e8-kor}
\end{equation}
Если параметр~$\alpha$ известен, то для определения верхней границы~$u^*$
сетки, накидываемой на носитель распределения случайной
величины~$U$, можно задать малое положительное число~$\varepsilon$
и~воспользоваться требованием
\begin{equation}
{\sf P}(U\geqslant u^*)\leqslant\varepsilon\,.\label{e9-kor}
\end{equation}
А~для гарантированного выполнения требования~(\ref{e9-kor}) можно использовать
неравенство Маркова:
$$
{\sf P}(U\geqslant u^*)\leqslant\fr{{\sf E}U^2}{(u^*)^2}\leqslant \varepsilon\,,
$$
откуда с учетом~(\ref{e8-kor})
$$
(u^*)^2\geqslant\fr{{\sf E}U^2}{\varepsilon}\approx
\fr{1}{\alpha^2\varepsilon}\left( m^2-\fr{\overline x}{\alpha}\right)
$$
или
\begin{equation}
u^*\approx\fr{1}{\alpha\sqrt{\varepsilon}}\sqrt{m^2-
\fr{\overline x}{\alpha}}\,.\label{e10-kor}
\end{equation}

\begin{figure*}[b] %fig1
\vspace*{1pt}
 \begin{center}
 \mbox{%
 \epsfxsize=161.718mm
 \epsfbox{kor-1.eps}
 }
 \end{center}
 \vspace*{-9pt}
\Caption{Примеры применения модифицированного двухэтапного сеточного
ЕМ-ал\-го\-рит\-ма для подгонки обобщенного гиперболического распределения
к искусственным данным, $\beta\hm=0$: (\textit{a})~$n\hm=1000$, $\alpha\hm=0{,}3$,
$\nu\hm=1{,}3$, $\mu\hm=1{,}6$, $\lambda\hm=0{,}2$;
(\textit{б})~$n\hm=1000$, $\alpha\hm=0{,}5$, $\nu\hm=1$, $\mu\hm=1$,
$\lambda\hm=3$;
(\textit{в})~$n\hm=1000$, $\alpha\hm=3$,
 $\nu\hm=1{,}3$, $\mu\hm=1{,}6$, $\lambda\hm=2$;
(\textit{г})~$n\hm=10\,000$,
$\alpha\hm=0{,}3$, $\nu\hm=1{,}3$, $\mu\hm=1{,}6$, $\lambda\hm=0{,}2$}
\end{figure*}


Если же параметр~$\alpha$, определяющий асим\-мет\-рию распределения
случайной величины~$X$, неизвестен, то можно воспользоваться
следующими рассуждениями. Обозначим
$$
q_n=\fr{1}{n}\sum\limits_{i=1}^n{\bf 1}(x_i<0)\,,
$$
где ${\bf 1}(A)$~--- индикаторная функция множества (события)~$A$.
При этом по усиленному закону больших чисел с~вероятностью единица
$q_n\hm\longrightarrow {\sf P}(X\hm<0)$ $(n\hm\to\infty)$, так что при
больших~$n$ справедливо приближенное равенство
\begin{equation}
q_n\approx{\sf P}(X<0)\,.\label{e11-kor}
\end{equation}
Но
\begin{multline}
{\sf P}(X<0)=\int\limits_{0}^{\infty}\Phi
\left(-\alpha\sqrt{u}\right) p_{\mathrm{GIG}}(u;\nu,\mu,\lambda)\,du={}\\
{}=
{\sf E}\Phi\left(-\alpha\sqrt{U}\right)\,.\label{e12-kor}
\end{multline}

\pagebreak

\noindent
Предположим сначала, что $q_n\hm<1/2$. Если~$n$ достаточно велико,
то можно с~большой степенью
 уверенности утверж\-дать, что тогда
$\overline x\hm>0$ и~$-\alpha\hm<0$, т.\,е.
 $\alpha\hm>0$ и,~стало быть, на
положительной полуоси значений аргумента~$u$ функция $\Phi(\alpha u)$
вогнута, т.\,е.\ выпукла вверх. Тогда из~(\ref{e11-kor}) и~(\ref{e12-kor}), дважды
применяя неравенство Иенсена, в~силу монотонности функции~$\Phi$
получаем:
\begin{multline}
1-q_n\approx 1-{\sf E}\Phi\left(-\alpha\sqrt{U}\right)=
          {\sf E}\Phi\left(\alpha\sqrt{U}\right)\leqslant{}\\
          {}\leqslant\Phi
          \left(\alpha{\sf E}\sqrt{U}\right)\leqslant
          \Phi\left(\alpha\sqrt{{\sf E}U}\right)\,.\label{e13-kor}
\end{multline}
Если теперь для $t\hm\in(0,1)$ символом~$v_t$ обозначить $t$-кван\-тиль
стандартного нормального закона, то из~(\ref{e13-kor}) и~(\ref{e6-kor}) вытекает
<<приближенное неравенство>>
$$
v_{1-q_n}\hm\leqslant \alpha\sqrt{{\sf E}U}\,,
$$
т.\,е.
$$
\alpha\geqslant\fr{v_{1-q_n}}{\sqrt{{\sf E}U}}\approx
\fr{v_{1-q_n}\sqrt{\alpha}}{\sqrt{\overline x}}\,,
$$
откуда получаем, что при достаточно больших~$n$
\begin{equation}
\alpha\geqslant\fr{v_{1-q_n}^2}{\overline x}\,.\label{e14-kor}
\end{equation}
Если теперь задать малое положительное число~$\varepsilon$, то
для определения верхней границы~$u^*$ сетки, накидываемой на
носитель распределения случайной величины~$U$, можно воспользоваться
требованием~(\ref{e9-kor}), для гарантированного выполнения которого
с~учетом~(\ref{e6-kor}) и~(\ref{e14-kor}) можно использовать неравенство Маркова:
$$
{\sf P}(U\geqslant u^*)\leqslant \fr{{\sf E}U}{u^*}\approx\fr{\overline
x}{\alpha u^*}\leqslant \fr{(\overline x)^2}{v_{1-q_n}^2 u^*}\leqslant
\varepsilon\,,
$$
откуда окончательно вытекает оценка
\begin{equation}
u^*\approx\fr{(\overline x)^2}{v_{1-q_n}^2 \varepsilon}\,.\label{e15-kor}
\end{equation}

\begin{figure*}[b] %fig2
\vspace*{18pt}
 \begin{center}
 \mbox{%
 \epsfxsize=162.433mm
 \epsfbox{kor-3.eps}
 }
 \end{center}
 \vspace*{-9pt}
\Caption{Примеры применения модифицированного двухэтапного
сеточного ЕМ-ал\-го\-рит\-ма для подгонки обобщенного гиперболического
распределения к~искусственным данным, $n=10\,000$, $\beta\hm=0$:
(\textit{а})~$\alpha\hm=0{,}3$,
$\nu\hm=2$, $\mu\hm=2$, $\lambda\hm=2{,}5$;
(\textit{б})~$\alpha\hm=0{,}5$,  $\nu\hm=1$, $\mu\hm=1$, $\lambda\hm=3$;
(\textit{в})~$\alpha\hm=0{,}8$,
$\nu\hm=1{,}3$, $\mu\hm=1{,}6$, $\lambda\hm=2$;
(\textit{г})~$\alpha\hm=1{,}3$, $\nu\hm=2$, $\mu\hm=2$, $\lambda\hm=2{,}5$}
\end{figure*}



В случае $q_n\hm\geqslant1/2$, если $n$ достаточно велико, то можно
с~большой степенью уверенности утверж\-дать, что $\overline x\hm\leqslant 0$
и~$-\alpha\hm\geqslant 0$, т.\,е.\ на положительной\linebreak\vspace*{-12pt}

\pagebreak

%\end{multicols}


%\begin{multicols}{2}

\noindent
 полуоси значений аргумента~$u$
функция $\Phi(-\alpha u)$ вогнута, т.\,е.\ выпукла вверх. Тогда
из~(\ref{e11-kor}) и~(\ref{e12-kor}), дважды применяя неравенство Иенсена, в~силу
монотонности функции~$\Phi$ получаем
$$
q_n\approx {\sf E}\Phi\left(-\alpha\sqrt{U}\right)\leqslant
\Phi\left(-\alpha\sqrt{{\sf E}U}\right)\,,
$$
откуда вытекает <<приближенное неравенство>> $v_{q_n}\hm \leqslant
-\alpha\sqrt{{\sf E}U}$,
т.\,е.
$$
-\alpha\geqslant\fr{v_{q_n}}{\sqrt{{\sf E}U}}\approx
\fr{v_{q_n}\sqrt{|\alpha|}}{\sqrt{|\overline x|}}
$$
и при достаточно больших~$n$
\begin{equation}
|\alpha|\geqslant\fr{v_{q_n}^2}{|\overline x|}\,.\label{e16-kor}
\end{equation}
Для определения верхней границы~$u^*$ сетки, накидываемой на
носитель распределения случайной величины~$U$, снова зададим малое
положительное число~$\varepsilon$ и~потребуем, чтобы было
справедливо условие~(\ref{e9-kor}), для гарантированного выполнения которого
с~учетом~(\ref{e6-kor}) и~(\ref{e16-kor}) используем неравенство Маркова и~тот факт, что
$\mathrm{sign}\, \overline x\hm=\mathrm{sign}\,\alpha$ при достаточно
больших~$n$:
\begin{multline}
{\sf P}(U\geqslant u^*)\leqslant \fr{{\sf E}U}{u^*}\approx
\fr{\overline x}{\alpha u^*}=
\fr{|\overline x|}{|\alpha| u^*} \leqslant{}\\
{}\leqslant
\fr{(\overline x)^2}{v_{q_n}^2 u^*}\leqslant
\varepsilon\,.\label{e17-kor}
\end{multline}
В силу симметричности нормального распределения $v_{t}\hm=-v_{1-t}$ для
любого $t\hm\in(0,1)$, поэтому $v_{q_n}^2\hm=v_{1-q_n}^2$ и~в~случае
$q_n\hm\geqslant1/2$ соотношение~(\ref{e17-kor}) снова приводит к~оценке~(\ref{e15-kor}).

Справедливости ради необходимо отметить, что оценки~(\ref{e10-kor}) и~(\ref{e15-kor})
являются завышенными, но они гарантируют, что
$(1-\varepsilon)$-почти-весь носитель распределения случайной
величины~$U$ будет лежать внутри интервала $[0, u^*]$.

\section{Результаты численных экспериментов}

Приводимые в~данном разделе графики иллюстрируют качество работы
модифицированного сеточного метода разделения дис\-пер\-си\-он\-но-сдви\-го\-вых
смесей нормальных законов на примере его\linebreak применения к~оцениванию
параметров обоб\-щенных гиперболических распределений с~ис\-поль\-зованием
указанного алгоритма выбора сетки\linebreak с~умеренным чис\-лом узлов $K\hm=40$.
Для вы\-чис\-ле\-ний использовались искусственно сгенерированные выборки
объемов $n\hm=1000$ и~$n\hm=10\,000$ с~разными наборами параметров, значения
которых указаны на рисунках. На рис.~1 и~2 изображены гистограммы
(серые столбики) и~графики
истинной плот\-ности (штриховые линии), промежуточной
оценки, полученной сеточным ЕМ-ал\-го\-рит\-мом (пунктирные линии)
и~итоговой оценки (непрерывные линии). На рис.~1 и~2 так\-же указаны
значения полученных оценок параметров. Как видно из приводимых
рисунков, параметры~$\alpha$ оцениваются очень точно. Точность
оценок остальных параметров удовлетворительная и~может быть повышена
за счет использования более частых сеток и~более чувствительных
критериев остановки ЕМ-ал\-го\-рит\-ма на первом этапе. Следует отметить,
что даже в~тех случаях, в~которых наблюдаются заметные расхождения
оценок параметров и~их точных значений, оценки самих плотностей
довольно \mbox{точны}.




{\small\frenchspacing
 {%\baselineskip=10.8pt
 \addcontentsline{toc}{section}{References}
 \begin{thebibliography}{99}
\bibitem{k2011}
\Au{Королев В.\,Ю.} Ве\-ро\-ят\-но\-ст\-но-ста\-ти\-сти\-че\-ские методы
декомпозиции волатильности хаотических процессов.~--- М.: Изд-во
Московского ун-та, 2011.

\bibitem{n2013}
\Au{Назаров А.\,Л.} Приближенные методы разделения смесей
вероятностных распределений: Дисс.\ \ldots\  канд. физ.-мат. наук.~--- М.:
МГУ им.\ М.\,В.~Ломоносова, 2013.

\bibitem{BN1977}
\Au{Barndorff-Nielsen~O.-E.} Exponentially decreasing distributions
for the logarithm of particle size~// Proc. Roy. Soc. Lond.~A,
1977. Vol.~353. P.~401--419.

\bibitem{BN1978}
\Au{Barndorff-Nielsen~O.-E.} Hyperbolic distributions and
distributions of hyperbolae~// Scand. J. Statist., 1978. Vol.~5.
P.~151--157.

\bibitem{BN1982}
\Au{Barndorff-Nielsen~O.-E., Kent~J., S\!{\!\ptb{\!\o}}\,rensen~M.} Normal
variance-mean mixtures and $z$-distributions~// Int. Statist. Rev.,
1982. Vol.~50. No.\,2. P.~145--159.

\bibitem{ks2012}
\Aue{Королев В.\,Ю., Соколов И.\,А.} Скошенные распределения
Стьюдента, дисперсионные гам\-ма-рас\-пре\-де\-ле\-ния и~их обобщения как
асимптотические аппроксимации~// Информатика и~её применения, 2012.
Т.~6. Вып.~1. С.~2--10.

\bibitem{zk2013}
\Au{Закс Л.\,М., Королев В.\,Ю.} Обобщенные дисперсионные
гам\-ма-рас\-пре\-де\-ле\-ния как предельные для случайных сумм~// Информатика
и её применения, 2013. Т.~7. Вып.~1. С.~105--115.

\bibitem{k2013}
\Au{Королев В.\,Ю.} Обобщенные гиперболические
распределения как предельные для случайных сумм~// Тео\-рия
вероятностей и~ее применения, 2013. Т.~58. Вып.~1. С.~117--132.

\bibitem{kckg2013}
\Au{Королев В.\,Ю., Черток А.\,В., Корчагин~А.\,Ю.,
Горшенин~А.\,К.} Ве\-ро\-ят\-но\-ст\-но-ста\-ти\-сти\-че\-ское моделирование
информационных потоков в~сложных финансовых системах на основе
высокочастотных данных~// Информатика и~её применения, 2013. Т.~7.
Вып.~1. С.~12--21.

\bibitem{p2004}
\Au{Protassov R.\,S.} EM-based maximum likelihood parameter
estimation for a~multivariate generalized hyperbolic distribution
with fixed~$\lambda$~// Statistics Computing, 2004. Vol.~14.
P.~67--77.

\bibitem{kn2010}
\Au{Королев В.\,Ю., Назаров А.\,Л.} Разделение смесей
вероятностных распределений при помощи сеточных методов моментов и~максимального правдоподобия~//
Автоматика и~телемеханика, 2010. Вып.~3. С.~98--116.

\bibitem{DSch1983}
\Au{Dennis J.\,E., Schnabel R.\,B.} Numerical methods for
unconstrained optimization and nonlinear equations.~--- Englewood
Cliffs: Prentice-Hall, 1983. 378~p.
 \end{thebibliography}

 }
 }

\end{multicols}

\vspace*{-6pt}

\hfill{\small\textit{Поступила в редакцию 01.10.14}}

\newpage

%\vspace*{12pt}

%\hrule

%\vspace*{2pt}

%\hrule

%\vspace*{12pt}

\def\tit{A MODIFIED GRID METHOD FOR~STATISTICAL SEPARATION
OF~NORMAL VARIANCE-MEAN MIXTURES}

\def\titkol{A modified grid method for statistical separation
of~normal variance-mean mixtures}

\def\aut{V.\,Yu.~Korolev$^{1,2}$ and~A.\,Yu.~Korchagin$^1$}

\def\autkol{V.\,Yu.~Korolev and~A.\,Yu.~Korchagin}

\titel{\tit}{\aut}{\autkol}{\titkol}

\vspace*{-9pt}


\noindent
$^1$Faculty of Computational Mathematics and Cybernetics,
M.\,V.~Lomonosov Moscow State University,\linebreak
$\hphantom{^1}$1-52 Leninskiye Gory, GSP-1, Moscow 119991, Russian Federation


\noindent
$^2$Institute of Informatics Problems, Russian Academy of Sciences,
44-2~Vavilov Str., Moscow 119333, Russian\linebreak
$\hphantom{^1}$Federation

\def\leftfootline{\small{\textbf{\thepage}
\hfill INFORMATIKA I EE PRIMENENIYA~--- INFORMATICS AND
APPLICATIONS\ \ \ 2014\ \ \ volume~8\ \ \ issue\ 4}
}%
 \def\rightfootline{\small{INFORMATIKA I EE PRIMENENIYA~---
INFORMATICS AND APPLICATIONS\ \ \ 2014\ \ \ volume~8\ \ \ issue\ 4
\hfill \textbf{\thepage}}}

\vspace*{3pt}

\Abste{A~modified two-stage grid method for
statistical separation of normal variance-mean mixtures is described
as an alternative to a pure EM (expectation-maximization) algorithm.
At the first stage of this
algorithm, a~discrete approximation is constructed to the mixing
distribution. At the second stage, the obtained discrete
distribution is approximated by an absolutely continuous
distribution from a~predetermined family, say, by a generalized
inverse Gaussian distribution. The convergence of this two-stage
procedure is discussed. The monotonicity of the grid procedure used
at the first stage is proved. The problem of the optimal choice of
the parameters of the method is discussed in detail. First of all,
the problem of the optimal choice of the grid thrown on the support
of the mixing distribution is considered. Statistical estimators are
proposed for the quantiles of the mixing law. The efficiency of the
method is illustrated by examples of its application to the
estimation of the parameters of generalized hyperbolic
distributions.}

\smallskip

\KWE{mixture of probability distributions; normal
variance-mean mixture; generalized hyperbolic distribution;
EM-algorithm; grid method of separation of mixtures}

\DOI{10.14357/19922264140402}

\Ack
\noindent
The research was supported by the Russian Science Foundation (project 14-11-00364).

%\vspace*{3pt}

  \begin{multicols}{2}

\renewcommand{\bibname}{\protect\rmfamily References}
%\renewcommand{\bibname}{\large\protect\rm References}



{\small\frenchspacing
 {%\baselineskip=10.8pt
 \addcontentsline{toc}{section}{References}
 \begin{thebibliography}{99}
 \bibitem{k2011eng}
 \Aue{Korolev, V.\,Yu.} 2011.
\textit{Veroyatnostno-statisticheskie metody dekompozitsii
volatil'nosti khaoticheskikh protsessov}
[Probabilistic and statistical methods for the decomposition of volatility
of chaotic processes].
Moscow: Moscow University Press. 510~p.

\bibitem{n2013eng}
\Aue{Nazarov, A.\,L.} 2013.
{Priblizhennye metody razdeleniya smesey veroyatnostnykh raspredeleniy}
[Approximate methods for the decomposition of volatility of chaotic processes].
Ph.D. Thesis. Moscow: Moscow State University.

\bibitem{BN1977eng}
\Aue{Barndorff-Nielsen, O.\,E.} 1977.
Exponentially decreasing distributions for the logarithm of particle size.
\textit{Proc. Roy. Soc. Lond. A} 353:401--419.

\bibitem{BN1978eng}
\Aue{Barndorff-Nielsen, O.\,E.} 1978.
Hyperbolic distributions and distributions of hyperbolae.
\textit{Scand. J. Statist.} 5:151--157.

\bibitem{BN1982eng}
\Aue{Barndorff-Nielsen, O.\,E., J.~Kent, and M.~S\!{\ptb{\o}}rensen}. 1982.
Normal variance-mean mixtures and $z$-distributions.
\textit{Int. Statist. Rev.} 50(2):145--159.

\bibitem{ks2012eng}
\Aue{Korolev, V.\,Yu., and I.\,A. Sokolov}. 2012.
{Skoshennye raspredeleniya St'yudenta, dispersionnye
gam\-ma-ras\-pre\-de\-le\-niya i~ikh obobshcheniya kak asimptoticheskie
approksimatsii}
[Skewed Student's distributions, variance gamma distributions, and their
generalizations as asymptotic approximations].
\textit{Informatika i ee Primeneniya}~--- \textit{Inform. Appl.} 6(1):2--10.

\bibitem{zk2013eng}
\Aue{Korolev, V.\,Yu., and L.\,M.~Zaks}. 2013.
{Obobshchennye dispersionnye gam\-ma-ras\-pre\-de\-le\-niya kak
predel'nye dlya sluchaynykh summ}
[Generalized variance gamma distributions as limiting for random sums].
\textit{Informatika i ee Primeneniya}~--- \textit{Inform. Appl.} 7(1):105--115.

\bibitem{k2013eng} \Aue{Korolev, V.\,Yu.} 2013.
{Obobshchennye giperbolicheskie raspredeleniya kak predel'nye dlya sluchaynykh summ}
[Generalized hyperbolic distributions as limiting for random sums]
\textit{Theory Probab. Appl.} 58(1):117--132.

\bibitem{kckg2013eng}
\Aue{Korolev, V.\,Yu., A.\,V. Chertok, A.\,Yu.~Korchagin, and A.\,K.~Gorshenin}.
2013. {Ve\-ro\-yat\-no\-st\-no-sta\-ti\-sti\-che\-skoe
mo\-de\-li\-ro\-va\-nie informatsionnykh potokov v~slozhnykh finansovykh sistemakh
na osnove vysokochastotnykh dannykh}
[Probability and statistical modeling of information flows in complex
financial systems from high-frequency data].
\textit{Informatika i~ee Primeneniya}~--- \textit{Inform.  Appl.} 7(1):12--21.

\bibitem{p2004eng-1}
\Aue{Protassov, R.\,S.} 2004.
EM-based maximum likelihood parameter estimation for a multivariate
generalized hyperbolic distribution with fixed~$\lambda$.
\textit{Statistics Computing} 14:67--77.

\bibitem{kn2010eng-1}
\Aue{Korolev, V.\,Yu., and A.\,L.~Nazarov}. 2010.
{Razdelenie smesey veroyatnostnykh raspredeleniy pri pomoshchi
setochnykh metodov momentov i~maksimal'nogo pravdopodobiya}
[Separation of mixtures using grid moment-based methods and maximum likelihood].
\textit{Avtomatika i~Telemekhanika} [Automatics and Telemechanics] 3:98--116.

\bibitem{DSch1983eng}
\Aue{Dennis, J.\,E., and R.\,B.~Schnabel}. 1983.
\textit{Numerical methods for unconstrained optimization and nonlinear equations}.
Englewood Cliffs: Prentice-Hall. 378~p.


\end{thebibliography}

 }
 }

\end{multicols}

\vspace*{-6pt}

\hfill{\small\textit{Received October 01, 2014}}

\vspace*{-18pt}

\Contr

\noindent
\textbf{Korolev Victor Yu.} (b.\ 1954)~---
Doctor of Science in physics and mathematics, professor,
Department of Mathematical Statistics, Faculty of Computational Mathematics
and Cybernetics, M.\,V.~Lomonosov Moscow State University,
1-52 Leninskiye Gory, GSP-1, Moscow 119991, Russian Federation;
leading scientist, Institute of Informatics Problems,
Russian Academy of Sciences, 44-2~Vavilov Str., Moscow 119333, Russian
Federation; victoryukorolev@yandex.ru

\vspace*{3pt}

\noindent
\textbf{Korchagin Alexander Yu.} (b.\ 1989)~---
PhD student, Faculty of Computational Mathematics and Cybernetics,
M.\,V.~Lomonosov Moscow State University,
1-52 Leninskiye Gory, GSP-1, Moscow 119991, Russian Federation;
sasha.korchagin@gmail.com


\label{end\stat}

\renewcommand{\bibname}{\protect\rm Литература}     %12 Abst+avt      
\def\stat{bening}


\def\tit{АСИМПТОТИЧЕСКОЕ
РАЗЛОЖЕНИЕ ДЛЯ МОЩНОСТИ КРИТЕРИЯ, ОСНОВАННОГО НА ВЫБОРОЧНОЙ
МЕДИАНЕ, В~СЛУЧАЕ РАСПРЕДЕЛЕНИЯ ЛАПЛАСА$^*$}
\def\titkol{Асимптотическое
разложение для мощности критерия, основанного на выборочной
медиане} %, в случае распределения Лапласа}

\def\autkol{В.\,Е.~Бенинг, А.\,В.~Сипина}
\def\aut{В.\,Е.~Бенинг$^1$, А.\,В.~Сипина$^2$}

\titel{\tit}{\aut}{\autkol}{\titkol}

{\renewcommand{\thefootnote}{\fnsymbol{footnote}}\footnotetext[1]
{Работа выполнена
при финансовой поддержке РФФИ, проекты 08-01-00567 и
08-07-00152.}}

\renewcommand{\thefootnote}{\arabic{footnote}}
\footnotetext[1]{Московский государственный университет им.\
М.\,В.~Ломоносова, факультет вычислительной математики и
кибернетики; Институт проблем информатики Российской академии наук, bening@yandex.ru}
\footnotetext[2]{Московский государственный университет им.\
М.\,В.~Ломоносова, факультет вычислительной математики и
кибернетики, anna@sipin.ru}



\Abst{В работе прямыми методами, использующими асимптотические разложения,
получена формула для предельного отклонения мощности критерия, 
основанного на выборочной медиане, от мощности наилучшего критерия в случае распределения Лапласа.}

\KW{выборочная медиана; асимптотичсекое разложение; функция мощности; распределение Лапласа}

      \vskip 18pt plus 9pt minus 6pt

      \thispagestyle{headings}

      \begin{multicols}{2}

      \label{st\stat}


\section{Введение}

Следуя работе~\cite{3ben}, рассмотрим задачу проверки гипотезы
\begin{equation*}
{\sf H}_0: \theta = 0     
%\label{e1.1b}
\end{equation*}
против последовательности сложных близких альтернатив вида
\begin{equation*}
{\sf H}_{n,1}: \theta = \fr{t}{\sqrt{n}}\,,\quad 0<t<C\,,\quad
 C > 0
% \label{e1.2b}
\end{equation*}
на основе выборки $(X_1, \ldots , X_n)$~--- независимых одинаково распределенных наблюдений, имеющих распределение Лапласа 
с плотностью
\begin{equation}
p(x, \theta) = \fr{1}{2}e^{-|x-\theta|}\,, \quad x,\:
\theta \in{\sf R}^1\,. 
\label{e1.3b}
\end{equation}
Распределение Лапласа широко применяется в прикладной статистике, например
в задачах вы\-де\-ле\-ния полезного сигнала на фоне помех~[2--4].
Естественность возникновения этого распределения обоснована в
работе~\cite{6ben}.

Для каждого фиксированного $t\in (0,C]$
обозначим через~$\beta_n^*(t)$ мощность наилучшего критерия размера
$\alpha\in (0,1)$. По лемме Неймана--Пирсона %\linebreak 
[6, с.~94]
такой критерий всегда существует и  основан на логарифме отношения правдоподобия
\begin{equation}
\Lambda_n(t) = 
\sum_{i=1}^{n}\left( \left|X_i\right|-\left|X_i-tn^{-1/2}\right|\right)\,.
 \label{e1.4b}
\end{equation}
В работах~\cite{3ben, 2ben} рассмотрен критерий, основанный на знаковой статистике,
и получена формула для предельного отклонения мощности данного
критерия от мощности наилучшего критерия, основанного на~$\Lambda_n(t)$.
Поскольку у плотности~$p(x,\theta)$ не существует производной по~$\theta$ в 
точке $\theta = 0$, то это семейство не является регулярным.
Это выражается в нарушении естественного порядка~$n^{-1}$ разности мощностей
этих критериев и приводит к порядку~$n^{-1/2}$.

В  работе рассматривается статистика
\begin{equation*}
T_n = \sqrt{2k}\, \zeta_n\,,\quad k=\left[\fr{n}{2}\right]\,, 
%\label{e1.5b}
\end{equation*}
где $\zeta_n$~--- выборочная медиана:
\begin{equation*}
\zeta_n= 
\begin{cases}
X_{(k+1)}\,, & n=2k+1\,; \\
\fr{X_{(k)}+X_{(k+1)}}{2}\,, &  n=2k\,.
\end{cases}
%\label{e1.6b}
\end{equation*}
Заметим, что в случае распределения Лапласа выборочная медиана
совпадает с оценкой максимального правдоподобия (см.~\cite{1ben}).

Обозначим через~$\beta_n(t)$ мощность критерия размера $\alpha\in (0,1)$,
основанного на статистике~$T_n$. В работе получено асимптотическое
разложение для~$\beta_n(t)$ и вычислен предел разности мощностей~$\beta_n^*(t)$ и~$\beta_n(t)$
$$
r(t)\equiv\lim_{n\to\infty}\sqrt n\left(\beta_n^*(t)-\beta_n(t)\right)
$$
критериев (см.~(\ref{e2.14b})),
основанных соответственно на статистиках~$\Lambda_n$ и~$T_n$.

В работе также приведено полное доказательство  (см.~\cite{5ben})
представления выборочной медианы в виде случайной суммы
независимых экспоненциально распределенных  случайных величин.


\section{Асимптотическое разложение для мощности критерия,
основанного на выборочной медиане}

В этом разделе будет построено  асимптотическое разложение  для мощности~$\beta_n(t)$.
Основой для его получения служит  работа~\cite{1ben} (см.\ теорему~2.1),
в которой получено разложение для функции распределения выборочной медианы.
Члены порядка~$n^{-1/2}$ в разложении для функции распределения выборочной медианы
без доказательства приведены  также в работе~\cite{9ben}.

\medskip
\noindent
\textbf{Теорема 1.} {\it Для мощности~$\beta_n(t)$ равномерно по
$t\in(0,C]$, $C>0$,
справедливо следующее асимптотическое разложение:
\begin{equation*}
\beta_n(t)=
\begin{cases}
\Phi(t-u_\alpha)-\fr{t(2u_\alpha-t)}{2\sqrt{n}}\,\varphi(u_\alpha-t)+{} \\
\hspace*{8mm}{}+o\left(n^{-1/2}\right)\,,  \quad t \le u_\alpha\,,\enskip  \alpha <\fr{1}{2}\,;\\
\Phi(t-u_\alpha)-\fr{2u_\alpha^2+t^2-2u_\alpha t}{2\sqrt{n}}\,\varphi(u_\alpha -t)+{}\\
\hspace*{8mm}{}+o\left(n^{-1/2}\right)\,, \quad t>u_\alpha\,, \enskip \alpha <\fr{1}{2}\,;\\
\Phi(t-u_\alpha)+\fr{t(2u_\alpha-t)}{2\sqrt{n}}\,\varphi(u_\alpha -t)+{}\\
\hspace*{22mm}{}+{} o\left(n^{-1/2}\right)\,, \quad 
\alpha \ge \fr{1}{2}\,,
\end{cases}\hspace*{-6pt}
%\label{e2.1b}
\end{equation*}
где  $\Phi(x)$  и~ $\varphi(x)$~---  функция распределения и
плотность стандартного нормального закона и $\Phi(u_\alpha)=1-\alpha$.}

\medskip

\noindent
Д\,о\,к\,а\,з\,а\,т\,е\,л\,ь\,с\,т\,в\,о\,.\
Для доказательства теоремы воспользуемся асимптотическим разложением
для функции распределения выборочной медианы в случае
распределения Лапласа из работы~\cite{1ben} (см.\ формулу~(1.3)):
\begin{multline}
\p_{n,\theta} \left( \sqrt{2k}(\zeta_n - \theta) < x \right) = 
\Phi(x)-\fr{x|x|}{2\sqrt{2k}}\,\varphi(x)+{}\\
{}+
\fr{x(18+10x^2-3x^4)}{48k}\,\varphi(x)+ o(n^{-1})\,.
\label{e2.2b}
\end{multline}
Подберем критическое значение~$d_n$, исходя из условия
\begin{equation*}
\p_{n,0}(T_n>d_n)=\alpha+ o(n^{-1})\,.
%\label{e2.3b}
\end{equation*}
Будем искать $d_n$ в виде
\begin{equation*}
d_n = u_\alpha +\fr{a}{\sqrt{2k}}+\fr{b}{2k}\,.
%\label{e2.4b}
\end{equation*}
Из формулы~(\ref{e2.2b}) следует, что

\noindent
\begin{multline}
\p_{n, 0} \left( T_n> d_n \right) = 1 -
\Phi(d_n)+\fr{d_n|d_n|}{2\sqrt{2k}}\varphi(d_n)-{}\\
{}-
\fr{d_n(18+10d_n^2-3d_n^4)}{48k}\,\varphi(d_n)+ o(n^{-1})\,.
\label{e2.5b}
\end{multline}
Чтобы раскрыть модуль в выражении~(\ref{e2.5b}),  рас\-смот\-рим два случая:
$\alpha<1/2$ и $\alpha \ge 1/2$.

Рассмотрим случай $\alpha < 1/2$. Это означает, что при достаточно
больших $n$ справедливо неравенство $d_n > 0$.
Подставляя выражение для~$d_n$ в формулу~(\ref{e2.5b}) и применяя следующие разложения:
\begin{multline*}
\Phi(d_n)=\Phi\left(u_\alpha+\fr{a}{\sqrt{2k}}+\fr{b}{2k}\right)=
\Phi(u_\alpha)+{}\\
{}+
\left(\fr{a}{\sqrt{2k}}+\fr{b}{2k}\right)\varphi(u_\alpha)-
\fr{u_\alpha a^2}{4k}\varphi(u_\alpha)+ o(n^{-1})\,;
\end{multline*}
\vspace*{-12pt}

\noindent
\begin{multline*}
\varphi(d_n)=\varphi\left(u_\alpha+\fr{a}{\sqrt{2k}}
+\fr{b}{2k}\right)= {}\\
{}=
\varphi(u_\alpha)-\left(\fr{a}{\sqrt{2k}}+\fr{b}{2k}\right)u_\alpha
\varphi(u_\alpha)+ o(n^{-1})\,,
\end{multline*}
получаем
\begin{multline*}
1-\Phi(u_\alpha)-\left(\fr{a}{\sqrt{2k}}+
\fr{b}{2k}\right)\varphi(u_\alpha)+\fr{u_\alpha a^2}{4k}\,\phi(u_\alpha)
+{}\\
{}+\fr{(u_\alpha+(a/\sqrt{2k})+b/(2k))^2}
{2\sqrt{2k}}\times{}\\
{}\times \left(\varphi(u_\alpha) - \fr{a}{\sqrt{2k}}\,u_\alpha
\varphi(u_\alpha)\right)-{}\\
{}-
\fr{u_\alpha(18+10u_\alpha^2-3u_\alpha^4)}{48k}\,\varphi(u_\alpha)=
\alpha + o(n^{-1})\,.
\end{multline*}
Приравнивая коэффициенты при~$1/\sqrt{2k}$ и~$1/(2k)$ к нулю,
находим выражения для~$a$ и~$b$:
\begin{gather*}
a=\fr{u_\alpha^2}{2}\,;
\\
b=-\fr{3}{4}\,u_\alpha+\fr{1}{12}\,u_\alpha^3\,;
\\
d_n = u_\alpha+\fr{u_\alpha^2}{2\sqrt{2k}}-\fr{3}{8k}\,
u_\alpha+\fr{1}{24k}\,u_\alpha^3\,.
\end{gather*}
Теперь для получения асимптотического разложения мощности критерия используем
разложение
\begin{multline*}
\p_{n,tn^{-1/2}}(T_n<x)= \Phi\left(x-t\sqrt{2k}n^{-1/2}\right) -{}\\
{}-
\fr{\left(x-t\sqrt{2k}n^{-1/2}\right)\left| x\:-\:t\sqrt{2k}\,n^{-1/2}\right|}{2\sqrt{2k}}\,
{}\times{}\\
{}\times\varphi(x-t\sqrt{2k}\,n^{-1/2})+ {}
\end{multline*}
\begin{multline*}
{}+
\fr{ x-t\sqrt{2k}\,n^{-1/2}}{48k}
\left(18+10(x-
t\sqrt{2k}\,n^{-1/2})^2-{}\right.\\
\left.{}-3(x-t\sqrt{2k}\,n^{-1/2})^4\right)\times{}
\\
{}\times\varphi\left(x-t\sqrt{2k}\,n^{-1/2}\right)+ o\left(n^{-1}\right)\,,
%\label{e2.6b}
\end{multline*}
которое  получается при подстановке $\theta=tn^{-1/2}$ в
формулу~(\ref{e2.2b}).

Имеем
\begin{multline*}
\beta_n(t)=\p_{n,tn^{-1/2}}\left(T_n>d_n\right) ={}\\
{}=
1-\Phi\left(d_n-t\right) +
\fr{\left(d_n-t\right)\left|d_n-t\right|}{2\sqrt{2k}}\,\varphi\left(d_n-t\right)-{}
\\\!
{}-\fr{d_n-t}{48k}\left(18+10\left(d_n-t\right)^2
-3(d_n-t)^4\right)\, \varphi\left(d_n-t\right)+{}\\
{}+ o\left(n^{-1}\right)\,.
%\label{e2.7b}
\end{multline*}
Аналогично предыдущему, рассмотрим  два случая: $t\le u_\alpha$ и
$t>u_\alpha$.

Пусть сначала $t \le u_\alpha$.
Используя разложения
\begin{multline*}
\Phi\left(d_n-t\right)={}\\
{}=\Phi\left(u_\alpha-t+
\fr{u_\alpha^2}{2\sqrt{2k}}-\fr{3}{8k}\,u_\alpha+
\fr{1}{24k}\,u_\alpha^3\right)={}\\
{}=\Phi\left(u_\alpha-t\right)+
\left(\fr{u_\alpha^2}{2\sqrt{2k}}-\fr{3}{8k}\,u_\alpha+
\fr{1}{24k}\,u_\alpha^3\right)\times{}\\
{}\times\varphi\left(u_\alpha-t\sqrt{2k}\,n^{-1/2}\right)-{}
\\
{}-
\fr{\left(u_\alpha-t\sqrt{2k}\,n^{-1/2}\right)\varphi\left(u_\alpha-
t\sqrt{2k}\,n^{-1/2}\right)u_\alpha^4}{16k}+{}\\
{}+ o\left(n^{-1}\right)\,; 
%\label{e2.8b}
\end{multline*}

\vspace*{-12pt}

\noindent
\begin{multline*}
\varphi\left(d_n-t\right)={}\\
{}= \varphi\left(u_\alpha-t+
\fr{u_\alpha^2}{2\sqrt{2k}}-\fr{3}{8k}\,u_\alpha+
\fr{1}{24k}\,u_\alpha^3\right)={}\\
{}=
\varphi\left(u_\alpha-t\right)-\left(u_\alpha-t\right)
\varphi\left(u_\alpha-t\right)\fr{u_\alpha^2}{2\sqrt{2k}}+{}\\
{}+
o\left(n^{-1/2}\right)\,,
%\label{e2.9}
\end{multline*}
получаем, что
\begin{multline*}
\beta_n(t)=1-\Phi\left(u_\alpha-t\right)-
\fr{u_\alpha^2}{2\sqrt{2k}}\,\varphi\left(u_\alpha-t\right)+{}\\
{}+\fr{u_\alpha^2}{2\sqrt{2k}}\,\varphi(u_\alpha-t)-
\fr{2u_\alpha t - t^2}{2\sqrt{2k}}\,\varphi(u_\alpha-t)+{}\\
{}+
o\left(n^{-1/2}\right)=
\Phi\left(t-u_\alpha\right)-\fr{t\left(2u_\alpha - t\right)}{2\sqrt{2k}}\,
\varphi\left(u_\alpha - t\right)+{}\\
{}+ o\left(n^{-1/2}\right)\,.
%\label{e2.10b}
\end{multline*}
Во втором случае при $t > u_\alpha$  выражение
для мощности приобретает вид:

\noindent
\begin{multline*}
\beta_n(t)=\Phi\left(t-u_\alpha\right)-{}\\
{}-
\fr{t\left(2u_\alpha^2+t^2 -2u_\alpha t\right)}{2\sqrt{n}}\,
\varphi\left(u_\alpha-t\right)+ o\left(n^{-1/2}\right)\,.
%\label{e2.11b}
\end{multline*}
При $\alpha \ge 1/2$  аналогичным образом имеем
\begin{multline*}
\beta_n(t)={}\\
{}=
 \Phi\left(t-u_\alpha\right)+
\fr{t\left(2u_\alpha - t\right)}{2\sqrt{n}}\,\varphi\left(u_\alpha - t\right)+
o\left(n^{-1/2}\right)\,.
%\label{e2.12b}
\end{multline*}
Из этих формул следует утверждение теоремы.~$\Box$

\medskip

В работе~\cite{2ben} было показано, что для мощ\-ности~$\beta_n^*(t)$ 
критерия размера $\alpha\in (0,1)$, осно\-ван\-но\-го на
логарифме отношения прав\-до\-подобия~$\Lambda_n(t)$~(\ref{e1.4b}),
справедливо  асимптотическое\linebreak разложение
\begin{equation*}
\beta_n^*(t)=\Phi(t-u_\alpha) - \fr{t^2}{6\sqrt{n}}\,
\varphi(t-u_\alpha)+ o(n^{-1/2})\,.
%\label{e2.13b}
\end{equation*}
Используя это разложение и теорему~1, получаем формулу
для предельного отклонения нормированной разности мощностей
рассматриваемых критериев:
\begin{multline}
r(t)= \lim_{n \to \infty}\sqrt{n}(\beta_n^*(t)-\beta_n(t))
={}\\
{}=
\begin{cases}
\left(t u_\alpha-\fr{2t^2}{3}\right)
\varphi(u_\alpha-t)\,,\\
\hspace*{30mm} t \le u_\alpha\,,\enskip \alpha < \fr{1}{2}\,; \\
\left(u_\alpha^2+\fr{t^2}{3}-u_\alpha t \right)
\varphi(u_\alpha - t)\,,\\
\hspace*{30mm}  t>u_\alpha\,,\enskip \alpha<\fr{1}{2}\,; \\
\left(\fr{t^2}{3}-t u_\alpha\right)\varphi(u_\alpha-t)\,, \quad\quad\ \  \alpha \ge \fr{1}{2}\,. 
\end{cases}
\label{e2.14b}
\end{multline}

\section{Представление выборочной медианы в~виде случайной суммы}

В этом разделе докажем лемму о представлении выборочной медианы
в случае распределения Лапласа в виде суммы случайного числа
независимых экспоненциально распределенных случайных величин.
Формулы для представления порядковых статистик в случае распределения
Лапласа в виде подобной суммы приведены в работе~[4, с.~63],
но без строгого доказательства.

\bigskip

\noindent
\textbf{Лемма 1.}
{\it В случае распределения Лапласа выборочную медиану
можно представить в следующем виде (здесь равенства по распределению):
\begin{align}
\zeta_{2k+1} &\stackrel{d}{=}\delta_{2k+1}
\sum\limits_{j=k+1}^{K_{2k+1}}{\fr{W_j}{j}}\,;
\label{e3.1b}\\
\zeta_{2k}&\stackrel{d}{=}\fr{W_1-W_2}{2k}\,\mathbf{1}(B_{2k+1}=k)+{}\notag\\[1pt]
&\!\!\!\!\!\!\!\!\!\!\!\!\!\!{}+
\left(\delta_{2k}\sum\limits_{j=k+1}^{K_{2k+1}}\fr{W_j}{j}+
\delta_{2k}\fr{W_k}{2k}\right)\mathbf{1}\left(B_{2k+1} \ne k\right)\,,
\label{e3.2b}
\end{align}
где
$$
\delta_n=\mathrm{sign}\left(B_n-k-\fr{1}{2}\right)\,,
$$
$W_j$~--- независимые экспоненциально (с параметром~1) распределенные
случайные величины; $B_n$~--- бернуллиевские случайные величины с параметрами
$p=1/2$ и~$n$, независимые от~$W_j$;
\begin{equation*}
K_n = \max\left(B_n, \bar{B_n}\right)\,,\quad
\bar{B_n}= n - B_n\,.
\end{equation*}
}

\smallskip

\noindent
Д\,о\,к\,а\,з\,а\,т\,е\,л\,ь\,с\,т\,в\,о\,.

Вначале докажем две вспомогательные формулы, справедливые для любого
действительного чис\-ла~$s$
и любых натуральных чисел~$a$ и~ $b$:
\begin{gather}
\prod\limits_{j=a}^{a+b}{\fr{1}{j+is}}=
\sum\limits_{j=0}^b \fr{(-1)^j}{(a+j+is)(b-j)!j!}\,;
\label{e3.3b}
\\[3pt]
\!\!\!\!\!\!\!\!\sum\limits_{l=0}^k\fr{k!}{l!} \prod\limits_{j=a}^{a+k-l}\fr{1}{j+is}=
\sum\limits_{l=0}^k \begin{pmatrix}
k\\ l\end{pmatrix}
\fr{(-1)^l 2^{k-l}}{a+l+is}\,.
\label{e3.4b}
\end{gather}
Формулу~(\ref{e3.3b}) докажем методом математической индукции.

При $b=1$ формула верна. Предполагая ее верной при $b\ge1$,
докажем что она  верна и  при~$b+1$:
\begin{multline*}
\prod\limits_{j=a}^{a+b+1}\fr{1}{j+is}=\fr{1}{a+b+1+is}\prod\limits_{j=a}^{a+b}
\fr{1}{j+is}={}\\[2pt]
{}=
\fr{1}{a+b+1+is}\left(\sum\limits_{l=0}^k 
\begin{pmatrix}
k\\ l
\end{pmatrix}
\fr{(-1)^l 2^{k-l}}
{a+l+is}\right)={}\\[2pt]
{}=
\sum\limits_{j=0}^{b}\fr{(-1)^j}{(b-j)!j!} \left(\fr{1}{(b+1-j)(a+j+is)}
- {}\right.\\[2pt]
\left.{}-\fr{1}{(b+1-j)(a+b+1+is)} \right)={}
\end{multline*}
\begin{multline*}
{}=
\sum\limits_{j=0}^{b}\fr{(-1)^j}{(a+j+is)(b+1-j)!j!}-{}\\
{}-
\fr{1}{a+b+1+is}\sum\limits_{j=0}^{b}\fr{(-1)^j}{(b-j+1)!j!}\,.
\end{multline*}
Заметим, что
\begin{multline*}
\!\!\sum\limits_{j=0}^b\fr{(-1)^j}{(b-j+1)!j!}=
\sum\limits_{j=0}^{b+1}\fr{(-1)^j}{(b-j+1)!j!}
-\fr{(-1)^{b+1}}{(b+1)!}={}\\
{}=
\fr{1}{(b+1)!}(1-1)^{b+1}-\fr{(-1)^{b+1}}{(b+1)!}=
-\fr{(-1)^{b+1}}{(b+1)!}\,.
\end{multline*}
И следовательно, формула~(\ref{e3.3b}) доказана.
Формула~(\ref{e3.4b}) следует  из доказанной формулы~(\ref{e3.3b}), по\-скольку
\begin{multline*}
\sum_{l=0}^k{\fr{k!}{l!}}\prod\limits_{j=a}^{a+k-l}\fr{1}{j+is}={}\\
{}=
\sum\limits_{l=0}^{k}\fr{k!}{l!}\sum\limits_{j=0}^{k-l}
\fr{(-1)^j}{(a+j+is)(k-l-j)! j!}={}\\
{}
=\sum\limits_{j=0}^{k}\fr{(-1)^j}{a+j+is}\sum\limits_{l=0}^{k-j}
\begin{pmatrix}
k\\ j
\end{pmatrix}
\begin{pmatrix}
k-j\\  l
\end{pmatrix}={}\\
{}=
\sum\limits_{j=0}^k\fr{(-1)^j}{a+j+is}
\begin{pmatrix}
k\\ j
\end{pmatrix}
2^{k-j}\,.
\end{multline*}
Теперь приступим к доказательству основного утверждения леммы.
Рассмотрим сначала случай $n=2k+1$.
Плотность $(k+1)$-й порядковой статистики, как известно,
выражается формулой (см.~\cite{4ben})
\begin{equation*}
p_{2k+1}(x) = (2k+1)
\begin{pmatrix}
2k\\  k\end{pmatrix}
f(x)(F(x)(1-F(x))^k\,,
%\label{3.5b}
\end{equation*}
где $f(x)$ и  $F(x)$~--- соответственно плотность и
функция распределения исходных случайных величин.

Найдем характеристическую функцию~$\phi_{2k+1}(s)$ выборочной
медианы~$\zeta_{2k+1}$:
\begin{multline*}
\phi_{2k+1}(s)=\e e^{is\zeta_{2k+1}}=
\int\limits_{-\infty}^{\infty}e^{isx}f(x)\,dx={}\\
{}=
(2k+1)
\begin{pmatrix}
2k\\  k\end{pmatrix}
2^{-(k+1)}\times{}\\
{}\times
\sum\limits_{j=0}^k (-1)^j 2^{-j}
\begin{pmatrix}
k\\ e j\end{pmatrix}
\fr{2(k+1+j)}{(k+1+j)^2+s^2}\,.
%\label{e3.6b}
\end{multline*}
Теперь найдем характеристическую функцию~$f_{2k+1}(s)$ случайной величины, определенной\linebreak\vspace*{-12pt}\pagebreak

\noindent
в правой части  формулы~(\ref{e3.1b}).
С учетом того, что
 характеристическая функция стандартной экспоненциальной
случайной величины равна $1/(1-is)$, имеем
\begin{multline*}
f_{2k+1}(s)={}\\
{}=
\sum\limits_{l=0}^{2k+1}\e \exp \left(is\delta_{2k+1}
\sum\limits_{j=k+1}^{K_{2k+1}}\fr{W_j}{j}\right)\mathbf{1}(B_{2k+1}=l)={}
\\
=2^{-(2k+1)}\left(\sum\limits_{l=0}^k \begin{pmatrix}
2k+1\\  l\end{pmatrix}
\prod\limits_{j=k+1}^{2k+1-l}\fr{j}{j+is}+{}\right.\\
\left.{}+
\sum\limits_{l=k+1}^{2k+1}
\begin{pmatrix}
2k+1\\ l\end{pmatrix}
\prod\limits_{j=k+1}^{l}\fr{j}{j-is}
\right)={}\\
{}
=2^{-(2k+1)}(2k+1)
\begin{pmatrix}
2k\\ k\end{pmatrix}
\sum\limits_{l=0}^k\fr{k!}{l!}
\left(\prod\limits_{j=k+1}^{2k+1-l}\fr{1}{j+is} +{}\right.\\
\left.{}+
\prod\limits_{j=k+1}^{2k+1-l}\fr{1}{j-is}\right)\,.
%\label{e3.7b}
\end{multline*}
Применяя формулу~(\ref{e3.4b}), получаем
\begin{multline*}
f_{2k+1}(s)=(2k+1)
\begin{pmatrix}
2k\\  k
\end{pmatrix}
2^{-(k+1)}\times{}\\
{}\times
\sum\limits_{j=0}^k(-1)^j 2^{-j}
\begin{pmatrix}
k\\  j\end{pmatrix}
\fr{2(k+1+j)}{(k+1+j)^2+s^2}\,.
%\label{e3.8b}
\end{multline*}
Значит, $f_{2k+1}(s)\equiv\phi_{2k+1}(s)$ и представление~(\ref{e3.1b}) доказано.
\medskip

Перейдем теперь к рассмотрению случая четного $n=2k$.
Совместная плотность двух порядковых статистик~$X_{(k)}$ и~$X_{(k+1)}$
определяется формулой (см.~\cite{4ben})
\begin{equation*}
p(x,y)=\fr{(2k)!}{((k-1)!)^2}\,(F(x)(1-F(y)))^{k-1}f(x)f(y)\,.
%\label{e3.9b}
\end{equation*}
Из этой формулы нетрудно получить, что плотность случайной величины
$$
\zeta_{2k}=\fr{X_{(k)}+X_{(k+1)}}{2}
$$
равна
\begin{multline*}
p_{2k}(x) = \fr{(2k)!}{2^k ((k-1)!)^2}\times{}\\
{}\times
\left(\sum_{j=0}^{k-2}\fr{(-1)^j
\begin{pmatrix}
k-1\\ j
\end{pmatrix}
2^{-j}}{k-1-j}
e^{-(k+1+j)|x|}\times{}\right.
\end{multline*}
\begin{multline}
\left.{}\times \left(1-e^{-(k-1-j)|x|}\right)- \right.{}\\
{}\left.
- \fr{(-1)^k}{2^{k-1}}|x|e^{-2k|x|} + \fr{1}{k2^k}e^{-2k|x|}
\vphantom{\fr{(-1)^j
\begin{pmatrix}
k-1\\ j
\end{pmatrix}
2^{-j}}{k-1-j}}\right)\,.
\label{e3.10b}
\end{multline}
Подробный вывод этой формулы приведен в работе~\cite{8ben}.
Исходя их формулы~(\ref{e3.10b}), найдем характеристическую функцию~$\phi_{2k}(s)$
выборочной медианы~$\zeta_{2k}$:
\begin{multline*}
\!\phi_{2k}(s)=
\fr{(2k)!}{2^k ((k-1)!)^2}
\left( \sum\limits_{j=0}^{k-2}
\fr{(-1)^j
\begin{pmatrix}
k-1\\ j
\end{pmatrix}
2^{-j}}{k-1-j}\times{}\right.
\\
\left.{}\times
\left(
\fr{2(k+1+j)}{(k+1+j)^2+s^2}  -
 \fr{4k}{4k^2+s^2} \right)-{}\right.\\
\left. {}- 
 \fr{(-1)^k}{2^{k-2}(4k^2+s^2)} + \fr{1}{2^{k-2}(4k^2+s^2)}
 \vphantom{\sum\limits_{j=0}^{k-2}
\fr{(-1)^j
\begin{pmatrix}
k-1\\ j
\end{pmatrix}
2^{-j}}{k-1-j}}
\right)\,.
%\label{e3.11b}
\end{multline*}
Найдем теперь характеристическую функ-\linebreak цию~$f_{2k}(s)$ случайной величины,
определенной
 в правой части формулы~(\ref{e3.2b}). Учитывая формулу~(\ref{e3.4b}), получим
\begin{multline*}
f_{2k}(s)=\sum\limits_{l=0}^{k-1}{\p(B_{2k}=l)
\fr{2k}{2k+is}\prod\limits_{j=k+1}^{2k-l}{\fr{j}{j+is}}}+{}\\
{}+
\sum\limits_{j=k+1}^{2k}{\p(B_{2k}=l)\fr{2k}{2k-is}\prod\limits_{j=k+1}^{2k-l}\fr{j}{j-is}}+{}\\
{}+
\p(B_{2k}=k)\fr{4k^2}{4k^2+s^2}={}\\
{}=
\fr{(2k)!}{2^k ((k-1)!)^2} \left( \fr{1}{2^{k-2}(4k^2+s^2)}
+{}\right.\\
\left.{}+2^{1-k}\sum\limits_{l=0}^{k-1}(-1)^l 2^{k-l-1}
\begin{pmatrix}
k-1\\ l\end{pmatrix}\times\right.{}\\
{}\times
\left( \fr{1}{(2k+is)(k+1+l-is)}+{}\right.\\
\left.\left.{}+ 
\fr{1}{(2k-is)(k+1+l-is)}\right) \right)\,.
\end{multline*}
Применяя при $l \ne k-1$ следующее соотношение:
\begin{multline*}
\fr{1}{(2k+is)(k+1+l+is)}={}\\
{}=
\fr{1}{k-1-l}\left( \fr{1}{k+1+l+is} - \fr{1}{2k+is}\right)\,,
\end{multline*}
получаем равенство

\noindent
\begin{multline*}
f_{2k}(s)=
\fr{(2k)!}{2^k ((k-1)!)^2}
\left( \sum\limits_{j=0}^{k-2}
\fr{(-1)^j 
\begin{pmatrix}
k-1\\ j
\end{pmatrix}
2^{-j}}{k-1-j}\times{}\right.\\
\left.{}\times
\left(
\fr{2\left(k+1+j\right)}{(k+1+j)^2+s^2} 
-  \fr{4k}{4k^2+s^2} \right)
-{}\right. \\
\left.{}- \fr{\left(-1\right)^k}{2^{k-2}\left(4k^2+s^2\right)} + \fr{1}{2^{k-2}(4k^2+s^2)}
\vphantom{\sum_{j=0}^{k-2}
\fr{(-1)^j 
\begin{pmatrix}
k-1\\ j
\end{pmatrix}
2^{-j}}{k-1-j}}
\right)\,.
%\label{e3.12b}
\end{multline*}
Таким образом,  $\phi_{2k}(s)\equiv f_{2k}(s)$ и утверждение~(\ref{e3.2b})
доказано.~$\Box$

{\small\frenchspacing
{%\baselineskip=10.8pt
\addcontentsline{toc}{section}{Литература}
\begin{thebibliography}{9}

\bibitem{3ben} %1
\Au{Королев Р.\,А., Тестова  А.\,В., Бенинг~В.\,Е.} 
О мощ\-ности асимптотически оптимального критерия в случае 
распределения Лапласа~// Вестник Тверского Государственного Университета, 
серия Прикладная математика, 2008. Вып.~8. №\,4(64). С.~5--23.

\bibitem{9ben} %2
\Au{Takeuchi K.} 
Asymptotic theory of statistical estimation.~---  Tokyo, 1974. (In Japanese.)

\bibitem{1ben} %3
\Au{Бурнашев М.\,В.} 
Асимптотические разложения для 
медианной оценки параметра~// Теор. вероятн. и ее
прим., 1996. Т.~41. Вып.~4. С.~738--753.

\bibitem{5ben}  %4
\Au{Kotz S., Kozubowski~T.\,J., Podgorski~K.}
The Laplace distribution and generalizations: 
A revisit with applications to communications, economics, engineering, 
and finance.~--- Birkhauser, 2001.  P.~349.

\bibitem{6ben}  %5
\Au{Бенинг В.\,Е., Королев В.\,Ю.}
Некоторые статистические  задачи, связанные с распределением Лапласа~// 
Информатика и её применения, 2008. Т.~2.  Вып.~2. С.~19--34.

\bibitem{7ben}  %6
\Au{Леман Э.} 
Проверка статистических гипотез.~--- М.: Наука, 1964. 498~с.

\bibitem{2ben} %7
\Au{Королев Р.\,А., Бенинг В.\,Е.}
Асимптотические 
разложения для мощностей критериев в случае распределения Лапласа~//
Вестник Тверского Государственного Университета, серия 
Прикладная математика, 2008. Вып.~3(10). №\,26(86). С.~97--107.

\bibitem{4ben} %8
\Au{David H.\,A., Nagaraja H.\,N.}
Order Statistics.  3rd ed.~--- New Jersey: Wiley, 2003.  P.~458.

\label{end\stat}

\bibitem{8ben} %9
\Au{Asrabadi B.\,R.} 
The exact confidence interval for 
the scale parameter and the MVUE of the Laplace distribution~// 
Communications in statistics. Theory and methods, 1985. Vol.~14. No.\,3. 
P.~713--733.

 \end{thebibliography}
}
}
\end{multicols}      %13Abst+avt
\def\stat{shevtsova}

\def\tit{ОБ АБСОЛЮТНЫХ КОНСТАНТАХ В НЕРАВЕНСТВЕ БЕРРИ--ЭССЕЕНА И ЕГО
СТРУКТУРНЫХ И~НЕРАВНОМЕРНЫХ~УТОЧНЕНИЯХ$^*$}

\def\titkol{Об абсолютных константах в неравенстве Берри--Эссеена и его
структурных и неравномерных уточнениях}

\def\autkol{И.\,Г.~Шевцова}

\def\aut{И.\,Г.~Шевцова$^1$}

\titel{\tit}{\aut}{\autkol}{\titkol}

{\renewcommand{\thefootnote}{\fnsymbol{footnote}}\footnotetext[1]
{Работа поддержана
грантом МК--2256.2012.1, а также Российским фондом фундаментальных
исследований (проекты 12-01-31125-а, 11-01-00515а, 11-07-00112а,
12-07-00115а).}}

\renewcommand{\thefootnote}{\arabic{footnote}}
\footnotetext[1]{Факультет вычислительной математики и
кибернетики Московского государственного университета им.\
М.\,В.\,Ломоносова; Институт проблем информатики Российской академии
наук, ishevtsova@cs.msu.su}


\Abst{Для равномерного расстояния $\Delta_n$ между
функцией распределения (ф.р.)\ стандартного нормального закона и
ф.р.\ нормированной суммы~$n$ независимых случайных величин (с.в.)\
$X_1,\ldots,X_n$ с $\e X_j=0$, $\e X_j^2=\sigma_j^2$,
${j=1,\ldots,n}$, при всех $n\ge1$ приведены оценки
$$
\ud\le \min\{0{,}5583 \ell_n,\, 0{,}3723(\ell_n+0{,}5\tau_n),
\,0{,}3057(\ell_n+\tau_n)\},
$$
$$
\ud\le \min\{0{,}4690\ell_n,\, 0{,}3322(\ell_n+0{,}429\tau_n),
\,0{,}3031(\ell_n+0{,}646\tau_n)\}, \text{ если } X_1\eqd\cdots\eqd X_n,
$$
где $\ell_n\hm=\sum\limits_{j=1}^n\e|X_j|^3$, $\tau_n\hm=\sum\limits_{j=1}^n\sigma_j^3$,
$\sum\limits_{j=1}^n\sigma_j^2\hm=1$. Получены уточненные результаты для
случая симметричного распределения слагаемых. Также показано, что в
неравенстве На\-га\-ева--Би\-кя\-ли\-са (неравномерном аналоге неравенства
Бер\-ри--Эс\-се\-ена) абсолютная константа не превосходит 21,82 в общем
случае и 17,36 в случае одинаково распределенных слагаемых.}

\vspace*{3pt}

\KW{центральная предельная теорема; оценка
скорости сходимости; нормальная аппроксимация; неравенство
Бер\-ри--Эс\-се\-ена; неравенство На\-га\-ева--Би\-кя\-ли\-са; абсолютная константа}

\vspace*{4pt}

\vskip 14pt plus 9pt minus 6pt

      \thispagestyle{headings}

      \begin{multicols}{2}

            \label{st\stat}


Обозначим $\F_3$ множество всех ф.р.~$F$
с.в.~$X$ с ${\e X\hm=0}$ и ${\e|X|^3\hm< \infty}$.
Через $\F_{3,\,s}$ обозначим множество всех симметричных ф.р.\ из
$\F_3$. Пусть $X_1,X_2,\ldots,X_n$~--- независимые с.в.\ с ф.р.\
$F_1,\ldots,F_n\in\F_3$ соответственно. Положим $\sigma_j^2\hm= \e
X_j^2,$ $\beta_{3,\,j}\hm=\e|X_j|^3,$ $j\hm=1,\ldots,n,$
$\sum\limits_{j=1}^n\sigma_j^2\hm=1$, $\ell_n\hm=\sum\limits_{j=1}^n\beta_{3,\,j}$,
$\tau_n\hm=\sum\limits_{j=1}^n\sigma_j^3$,
$\overline F_n(x)\hm=\p(X_1+\cdots+X_n\hm<xB_n)$. Пусть $\phi(x)$ и
$\Phi(x)$~--- соответственно плотность и ф.р.\ стандартного
нормального закона,
\begin{align*}
\nud&=|\overline F_n(x)-\Phi(x)|\,,\quad x\in\R\,;\\
\ud &=\sup\limits_x|\overline F_n(x)-\Phi(x)|\,,\quad n=1,2,\ldots\,
\end{align*}

В работе~\cite{Shevtsova2012ISSPSM3} с использованием техники
преобразования смещения формы и новой оценки точности аппроксимации
характеристической функции первыми членами ее разложения в ряд
Тейлора для всех $n\hm\ge1$ были получены оценки:
\begin{multline*}
\ud\le \min\{0{,}5584\ell_n, 0{,}36266(\ell_n+0{,}54\tau_n),\\
0{,}3129(\ell_n+0{,}922\tau_n)\}\,,\quad F_1,\ldots,F_n\in\F_3\,;
\end{multline*}


\noindent
\begin{multline*}
\ud\le \min\{ 0{,}4693\ell_n, 0{,}3322(\ell_n+0{,}429\tau_n),\\
0{,}3031(\ell_n+0{,}646\tau_n)\}\,,\quad  F_1=\cdots=F_n\in\F_3\,;
\end{multline*}

\vspace*{-9pt}

\noindent
\begin{multline*}
\ud\le \min\{ 0{,}3425(\ell_n+0{,}63\tau_n),\\
0{,}2895(\ell_n+\tau_n) \}\,,\quad  F_1,\ldots,F_n\in\F_{3,\,s}\,;
\end{multline*}

\vspace*{-9pt}

\noindent
\begin{multline*}
\ud\le \min\{ 0{,}29489(\ell_n+0{,}587\tau_n)\,,\\
0{,}2730(\ell_n+0{,}732\tau_n)\}\,, \quad F_1=\cdots=F_n\in\F_{3,\,s}\,.
\end{multline*}

С помощью алгоритма, использованного в~\cite{Shevtsova2012ISSPSM3},
и оценки
$$
|f'(t)|\le \sigma\sin\left(\sigma|t| \wedge\fr{\pi}{2}\right)\,,\quad t\in\R\,,
$$
для производной характеристической функции $f(t)\hm=\e e^{itX}$ с $\e
X\hm=0$, $\sigma^2\hm\equiv\e X^2\hm<\infty$, вытекающей из результатов
работ~\cite{Rossberg1991, MatysiakSzablowski2001}, указанные
результаты можно уточнить и получить следующие оценки, справедливые
при всех $n\hm\ge1$:
\begin{multline*}
\ud\le \min\{0{,}5583\ell_n,\, 0{,}3723(\ell_n+0{,}5\tau_n),\\
\,0{,}3057(\ell_n+\tau_n)\}\,,\quad F_1,\ldots,F_n\in\F_3\,;
\end{multline*}

\vspace*{-12pt}

\noindent
\begin{multline*}
\ud\le \min\{0{,}4690\ell_n,\, 0{,}3322(\ell_n+0{,}429\tau_n),\\
\,0{,}3031(\ell_n+0{,}646\tau_n)\}\,,\quad F_1=\cdots=F_n\in\F_3\,;
\end{multline*}

\vspace*{-12pt}

\noindent
\begin{multline*}
\ud\le \min\{ 0{,}34245(\ell_n+0{,}63\tau_n),\\
\,0{,}2873(\ell_n+\tau_n)\}\,,\quad F_1,\ldots,F_n\in\F_{3,\,s}\,;
\end{multline*}

\vspace*{-12pt}

\noindent
\begin{multline*}
\ud\le \min\{ 0{,}29353(\ell_n+0{,}593\tau_n),\\
\,0{,}2730(\ell_n+0{,}729\tau_n)\}\,,\quad F_1=\cdots=F_n\in\F_{3,\,s}\,.
\end{multline*}

Кроме того, исправляя неточность, допущенную
в~\cite{NefedovaShevtsova2012}, и используя приведенные выше
неравенства, можно показать, что справедливы следующие неравномерные
оценки:
\begin{multline*}
\sup\limits_{x\in\R}(1+|x|^3)\nud\le \min\{21{,}82\ell_n,\,
18{,}19(\ell_n+\tau_n)\}\,,\\ F_1,\ldots,F_n\in\F_3\,;
\end{multline*}

\vspace*{-12pt}

\noindent
\begin{multline*}
\sup\limits_{x\in\R}(1+|x|^3)\nud\le \min\{ 17{,}36\ell_n,\\
15{,}70(\ell_n+0{,}646\tau_n)\}\,,\quad F_1=\cdots=F_n\in\F_3\,.
\end{multline*}
Более того, исправленный метод работы~\cite{NefedovaShevtsova2012}
позволяет построить монотонно убывающую функцию $C(t)$ с
$\lim\limits_{t\to\infty}C(t)\hm=1\hm+e\hm=3{,}71\ldots$ и такую, что
\begin{multline*}
\sup\limits_{|x|\ge t}|x|^3\nud\le C(t)\ell_n,\quad t\ge0,\quad n\ge1,\\
F_1,\ldots,F_n\in\F_3\,.
\end{multline*}
В частности, для $C(t)$ справедливы верхние оценки
\\
$C(0)\le21{,}26$, $C(4)\le17{,}19$, $C(5)\le12{,}35$, $C(10)\hm\le7{,}36$ для
любых $F_1,\ldots,F_n\in\F_3$;
\\
$C(0)\le16{,}90$, $C(4)\le14{,}58$, $C(5)\le11{,}56$, $C(10)\hm\le5{,}85$, если
$F_1\hm=\cdots=F_n\hm\in\F_3$.

С помощью методов, использованных в данной работе, можно получить
аналогичные равномерные и неравномерные оценки для случая, когда
слагаемые обладают моментами порядка лишь $2+\delta$ с некоторым
$0<\delta\le1$.

{\small\frenchspacing
{%\baselineskip=10.8pt
\addcontentsline{toc}{section}{Литература}
\begin{thebibliography}{9}

\bibitem{Shevtsova2012ISSPSM3}
\Au{Shevtsova I.} On the absolute constants in the
Berry--Esseen-type inequalities~// 30th  Seminar (International) on
Stability Problems for Stochastic Models (Svetlogorsk, 2012): Book
of Abstracts.~--- М.: ИПИ РАН, 2012. С.~71--72.



\bibitem{Rossberg1991}
\Au{Ro\!\!\!{\ptb{\ss}}\,berg~H.-J.} Positiv definite Verteilungsdichten~//
Appendix to:  \Au{Gnedenko B.\,W.} Einf$\ddot{\mbox{u}}$hrung in die
Wahrscheinlichkeitstheorie.~--- 9th ed.~--- Berlin:
Akademie--Verlag, 1991.

\bibitem{MatysiakSzablowski2001}
\Au{Matysiak~W., Szab\!\!{\ptb{\l}}owski~P.~J.} Some inequalities for
characteristic functions~// J.~Math. Sci., 2001. Vol.~105. No.\,6.
P.~2594--2598.

\label{end\stat}

\bibitem{NefedovaShevtsova2012}
\Au{Нефедова Ю.\,С., Шевцова И.\,Г.} О~неравномерных оценках
скорости сходимости в центральной предельной теореме~// Теория
вероятн. и ее примен., 2012, Т.~57. Вып.~1. С.~62--97.
\end{thebibliography}
}
}

\end{multicols}      %Abst


%\end{document}

%   { %\Large  
   { %\baselineskip=16.6pt
   
   \vspace*{-48pt}
   \begin{center}\LARGE
   \textit{Предисловие}
   \end{center}
   
   %\vspace*{2.5mm}
   
   \vspace*{25mm}
   
   \thispagestyle{empty}
   
   { %\small 

    
Вниманию читателей журнала <<Информатика и её применения>> предлагается 
очередной тематический выпуск <<Вероятностно-статистические методы и 
задачи информатики и информационных технологий>>. Предыдущие тематические 
выпуски журнала по данному направлению вышли в 2008~г.\ (т.~2, вып.~2), 
в 2009~г.\ (т.~3, вып.~3) и в 2010~г.\ (т.~4, вып.~2). 

Статьи, собранные в данном журнале, посвящены разработке новых вероятностно-статистических 
методов, ориентированных на применение к решению конкретных задач информатики и информационных 
технологий, а также~--- в ряде случаев~--- и других прикладных задач. Проблематика, охватываемая 
публикуемыми работами, развивается в рамках научного сотрудничества между Институтом проблем 
информатики Российской академии наук (ИПИ РАН) и Факультетом вычислительной математики и 
кибернетики Московского государственного университета им.\ М.\,В.~Ломоносова в ходе работ 
над совместными научными проектами (в том числе в рамках функционирования 
Научно-образовательного центра <<Вероятностно-статистические методы анализа рисков>>). 
Многие из авторов статей, включенных в данный номер журнала, являются активными участниками 
традиционного международного семинара по проблемам устойчивости стохастических моделей, 
руководимого В.\,М.~Золотаревым и В.\,Ю.~Королевым; регулярные сессии этого семинара 
проводятся под эгидой МГУ и ИПИ РАН (в 2011~г.\ указанный семинар проводится в октябре 
в Калининградской области РФ). 

Наряду с представителями ИПИ РАН и МГУ в число авторов данного выпуска журнала входят 
ученые из Научно-исследовательского института системных исследований РАН, Института 
проблем технологии микроэлектроники и особочистых материалов РАН, Института 
прикладных математических исследований Карельского НЦ РАН, Московского 
авиационного института, Вологодского государственного педагогического университета, 
НИИММ им.\ Н.\,Г.~Чеботарева, Казанского государственного университета, Дебреценского 
университета (Венгрия).

Несколько статей выпуска посвящено разработке и применению стохастических методов и 
информационных технологий для решения различных прикладных задач. В~работе В.\,Г.~Ушакова 
и О.\,В.~Шестакова рассмотрена задача определения вероятностных характеристик случайных 
функций по распределениям интегральных преобразований, возникающих в задачах эмиссионной 
томографии. В~статье Д.\,О.~Яковенко и М.\,А.~Целищева рассмотрены некоторые вопросы 
математической теории риска и предложен новый подход к диверсификации инвестиционных 
портфелей. Работа И.\,А.~Кудрявцевой и А.\,В.~Пантелеева посвящена построению и 
исследованию математической модели, описывающей динамику сильноионизованной плазмы. 
В~статье П.\,П.~Кольцова изучается качество работы ряда алгоритмов сегментации изображений. 
Статья А.\,Н.~Чупрунова и И.~Фазекаша посвящена вероятностному анализу числа без\-оши\-бочных 
блоков при помехоустойчивом кодировании; получены усиленные законы больших чисел для указанных 
величин.

В данном выпуске традиционно присутствует тематика, весьма активно разрабатываемая в течение 
многих лет специалистами ИПИ РАН и МГУ,~--- методы моделирования и управления для 
информационно-телекоммуникационных и вычислительных систем, в частности методы 
теории массового обслуживания. В~статье А.\,И.~Зейфмана с соавторами рассматриваются 
модели обслуживания, описываемые марковскими цепями с непрерывным временем в случае 
наличия катастроф. В~работе М.\,М.~Лери и И.\,А.~Чеплюковой рассматриваются случайные 
графы Интернет-типа, т.\,е.\ графы, степени вершин которых имеют степенные распределения; 
такие задачи находят применение при исследовании глобальных сетей передачи данных. 
Работа Р.\,В.~Разумчика посвящена исследованию систем массового обслуживания специального 
вида~--- с отрицательными заявками и хранением вытесненных заявок.

Ряд статей посвящен развитию перспективных теоретических 
вероятностно-статистических методов, которые находят широкое применение в различных 
задачах информатики и информационных технологий. В~работе В.\,Е.~Бенинга, А.\,К.~Горшенина 
и В.\,Ю.~Королева рассмотрена задача статистической проверки гипотез о числе компонент 
смеси вероятностных распределений, приводится конструкция асимптотически наиболее мощного 
критерия. Результаты этой работы найдут применение в ряде прикладных задач, использующих 
математическую модель смеси вероятностных распределений (в информатике, моделировании 
финансовых рынков, физике турбулентной плазмы и~т.\,д.). В~статье В.\,Ю.~Королева, 
И.\,Г.~Шевцовой и С.\,Я.~Шоргина строится новая, улучшенная оценка точности нормальной 
аппроксимации для пуассоновских случайных сумм; как известно, указанные случайные суммы 
широко используются в качестве моделей многих реальных объектов, в том числе в информатике, 
физике и других прикладных областях. Работа В.\,Г.~Ушакова и Н.\,Г.~Ушакова посвящена 
исследованию ядерной оценки плотности распределения; эти результаты могут применяться, 
в част\-ности, при анализе трафика в телекоммуникационных системах. Серьезные приложения 
в статистике могут получить результаты работы О.\,В.~Шестакова, в которой доказаны оценки 
скорости сходимости распределения выборочного абсолютного медианного отклонения к нормальному 
закону. 

\smallskip

Редакционная коллегия журнала выражает надежду, что данный тематический  выпуск 
будет интересен специалистам в области теории вероятностей и математической статистики 
и их применения к решению задач информатики и информационных технологий.
     
     %\vfill 
     \vspace*{20mm}
     \noindent
     Заместитель главного редактора журнала <<Информатика и её 
применения>>,\\
     директор ИПИ РАН, академик  \hfill
     \textit{И.\,А.~Соколов}\\
     
     \noindent
     Редактор-составитель тематического выпуска,\\
     профессор кафедры математической статистики факультета\\
      вычислительной математики и кибернетики МГУ им.\ М.\,В.~Ломоносова,\\
     ведущий научный сотрудник ИПИ РАН,\\ 
доктор физико-математических наук \hfill
      \textit{В.\,Ю.~Королев}
     
     } }
     }

%%%%%%%%%%%%%%%%%%%%%%%%%%%%%%%%%%%%%%%%%%%%%%%


%\def\stat{rez}
{%\hrule\par
%\vskip 7pt % 7pt
\raggedleft\Large \bf%\baselineskip=3.2ex
Р\,Е\,Ц\,Е\,Н\,З\,И\,И \vskip 17pt
    \hrule
    \par
\vskip 6pt plus 6pt minus 3pt }

%\thispagestyle{headings} %с верхним колонтитулом
%\thispagestyle{myheadings} %с нижним колонтитулом, но в верхнем РЕЦЕНЗИИ

\def\tit{НОВАЯ КНИГА И.\,Н.~СИНИЦЫНА, А.\,С.~ШАЛАМОВА <<ЛЕКЦИИ ПО ТЕОРИИ 
ИНТЕГРИРОВАННОЙ ЛОГИСТИЧЕСКОЙ ПОДДЕРЖКИ>> (М.: ТОРУС ПРЕСС, 2012. 624~с.)}

%1
\def\aut{Д.ф.-м.н., профессор С.\,Я.~Шоргин}

\def\auf{\ }

\def\leftkol{\ % РЕЦЕНЗИИ
}

\def\rightkol{ \ } 

%\def\leftkol{\ } % ENGLISH ABSTRACTS}

%\def\rightkol{\ } %ENGLISH ABSTRACTS}

%\def\leftkol{РЕЦЕНЗИИ}

%\def\rightkol{РЕЦЕНЗИИ}

\titele{\tit}{\aut}{\auf}{\leftkol}{\rightkol}
\vspace*{-18pt}


     \label{st\stat}

     \begin{multicols}{2}
     {\small
     {\baselineskip=10.1pt
     

      В книге представлено системное изложение теоретических основ одного из новейших 
направлений в \mbox{об\-ласти} экономики послепродажного обслуживания изделий наукоемкой 
продукции (ИНП) длительного пользования~--- интегрированной логистической поддержки
(ИЛП). 
{\looseness=1

}

Приведены также результаты новых работ, выполненных в Институте проблем информатики 
Российской академии наук в рамках научного направления <<Информационные технологии и 
анализ сложных сис\-тем>>.
 {%\looseness=1

}
     
      Излагаемые в книге научные подходы позво\-ляют карди\-наль\-но реформировать 
существующие системы производства и эксплуатации ИНП путем создания и внед\-ре\-ния 
методов рационального и оптимального управ\-ле\-ния процессами расходования 
вре\-мен\-н$\acute{\mbox{ы}}$х, 
мате\-ри\-аль\-ных, трудовых и других ресурсов на всех стадиях жизненного цикла изделий (ЖЦИ) по 
критериям экономической целесообразности и эф\-фек\-тив\-ности.
  {\looseness=1

}
    
      В книге приведен краткий обзор причин возник\-новения и
      развития CALS-методологии как основы 
современных международных стандартов по созданию и функционированию глобальных 
ин\-фор\-ма\-ци\-он\-но-ком\-му\-ни\-ка\-ци\-он\-ных систем, ее ключевых возможностей и эффективности 
результатов ее использования. 
Авторы %\linebreak 
предлагают ряд научных обоснований для разработки 
единой теории проектирования и управления систем ИЛП для полноценного использования 
преимуществ %\linebreak
 суще\-ст\-ву\-ющей методологии, определяют \mbox{общую} структурную схему 
комплексной системы <<ИНП-СППО>> и необходимость разработки для ее описания 
гибридных стохастических моделей.
{%\looseness=1

}

%\columnbreak
      
      Книга состоит из пяти частей, где последовательно излагается материал по каждой из 
следующих тем: <<Интегрированная логистическая поддержка>>, <<Теория гибридных 
стохастических систем и компьютерная поддержка исследований и разработок>>, <<Основы 
математического моделирования, анализа и синтеза систем послепродажного обслуживания>>, 
<<Определение и анализ показателей экспортного потенциала ИНП при проектировании>>, 
<<Задачи управления поддержкой послепродажного обслуживания>>, а также 
<<Моделирование инвестиционных процессов ИЛП в условиях неравновесных финансовых 
рынков>>. 
   
      В конце каждой главы приведены выводы и даны вопросы и задания для 
самоконтроля. В~приложениях содержатся основные определения по программам работ по 
анализу ИЛП, логистическим базам данных и компьютерным решениям, эквивалентной статистической 
линеаризации нелинейных преобразований ИЛП, справочный материал, а также развернутые 
уравнения для вероятностных характеристик.


      \def\leftkol{РЕЦЕНЗИИ}

\def\rightkol{РЕЦЕНЗИИ} 

      
      Книга заинтересует широкий круг специалистов и может быть использована научными 
проектными организациями в сфере промышленного производства ИНП. Большое количество 
иллюстраций, примеров и вопросов, обращенных к читателю, позволяет использовать книгу 
также в качестве учебного пособия для студентов и аспирантов машиностроительных, 
транспортных и~других специальностей, а также для самостоятельного изучения. 
{%\looseness=-1

}

Книга 
представляет несомненный интерес для специалистов и студентов в области прикладной 
математики и информатики.
    

}

}
\end{multicols}

%\newpage

%\end{document}

\include{obchak}


\def\stat{authorsrus}
{%\hrule\par
%\vskip 7pt % 7pt
\raggedleft\Large \bf%\baselineskip=3.2ex
О\,Б\ \ А\,В\,Т\,О\,Р\,А\,Х \vskip 17pt
    \hrule
    \par
\vskip 21pt plus 8pt minus 4pt }


\def\tit{\ }

\def\aut{\ }

\def\auf{\ }

\def\leftkol{\ } % ENGLISH ABSTRACTS}

\def\rightkol{ОБ АВТОРАХ} %ENGLISH ABSTRACTS}

\titele{\tit}{\aut}{\auf}{\leftkol}{\rightkol}
      
            \label{st\stat}



\vspace*{24pt}

\begin{multicols}{2}




\noindent
\textbf{Архипов Олег Петрович} (р.\ 1948)~---
кандидат технических наук, директор Орловского филиала Института проб\-лем информатики
Российской академии наук
%302025, г.Орел, Московское шоссе, д.137

\vspace*{3pt}

\noindent
\textbf{Бирюкова Татьяна Константиновна} (р.\ 1968)~---
кандидат фи\-зи\-ко-ма\-те\-ма\-ти\-че\-ских наук, старший научный сотрудник Института проб\-лем информатики
Российской академии наук

\vspace*{3pt}

\noindent 
\textbf{Бобков  Сергей Геннадьевич} (р.\ 1955)~---
доктор технических наук,  заведующий отделением На\-уч\-но-ис\-сле\-до\-ва\-тель\-ско\-го 
института системных исследований Российской академии наук
%117218, Москва, Нахимовский просп., 36, к.1 

\vspace*{3pt}

\noindent \textbf{Васильев Николай Семенович} (р.\ 1952)~--- доктор 
фи\-зи\-ко-ма\-те\-ма\-ти\-че\-ских наук, профессор, 
МГТУ им.\ Н.\,Э.~Баумана 
%, Москва 105005, 2-я Бауманская ул., д.~5,

\vspace*{3pt}

\noindent
\textbf{Гершкович Максим Михайлович} (р.\ 1968)~---
старший научный сотрудник Института проб\-лем информатики
Российской академии наук

\vspace*{3pt}

\noindent 
\textbf{Дьяченко Юрий Георгиевич} (р.\ 1958)~--- кандидат технических наук, 
старший научный сотрудник Института проб\-лем информатики
Российской академии наук

\vspace*{3pt}

\noindent 
\textbf{Ерошенко Александр Андреевич} (р.\ 1989)~--- аспирант кафедры 
математической статистики факультета вычисли\-тельной математики и кибернетики 
Московского государственного университета им.\ М.\,В.~Ломоносова
%119991, Москва ГСП-1, Ленинские горы, д.\ 1, стр. 52

\vspace*{3pt}
 
\noindent 
\textbf{Захаров Виктор Николаевич} (р.\ 1948)~--- 
доктор технических наук, доцент, ученый секретарь Института проб\-лем информатики
Российской академии наук

\vspace*{3pt}

\noindent
\textbf{Зейфман Александр Израилевич} (р.\ 1954)~---
доктор фи\-зи\-ко-ма\-те\-ма\-ти\-че\-ских наук, профессор, 
заведующий кафедрой Вологодского государственного университета; 
старший научный сотрудник Института проб\-лем информатики
Российской академии наук; главный научный сотрудник ИСЭРТ Российской академии наук

\vspace*{3pt}

\noindent
\textbf{Зыкин Сергей Владимирович} (р.\ 1959)~--- 
доктор технических наук, профессор, заведующий лабораторией Института математики 
им.\ С.\,Л.~Соболева Сибирского отделения Российской академии наук, Новосибирск 
%630090, пр.\ ак.\ Коптюга, 4 

\vspace*{4pt}

\noindent
\textbf{Киреев Владимир Иванович} (р.\ 1938)~---
доктор фи\-зи\-ко-ма\-те\-ма\-ти\-че\-ских наук, профессор Московского 
государственного горного университета
%Адрес: Россия, 119991, г. Москва, Ленинский проспект, д. 6

%\columnbreak

\vspace*{4pt}

\noindent
\textbf{Козеренко Елена Борисовна} (р.\ 1959)~---
кандидат филологических наук, заведующая лабораторией Института проб\-лем информатики
Российской академии наук

\vspace*{4pt}

\noindent
\textbf{Королев Виктор Юрьевич} (р.\ 1954)~--- доктор
фи\-зи\-ко-ма\-те\-ма\-ти\-че\-ских наук, профессор кафедры математической 
статистики факультета вычисли\-тельной математики и кибернетики 
Московского государственного университета; 
ведущий научный сотрудник Института проб\-лем информатики
Российской академии наук

\vspace*{4pt}

\noindent
\textbf{Коротышева Анна Владимировна} (р.\ 1988)~---
старший преподаватель Вологодского государственного университета

\vspace*{4pt}

\noindent 
\textbf{Кун Де Турк} (р.\ 1981)~--- научный сотрудник 
исследовательской группы SMACS факультета телекоммуникаций и обработки информации
Университета Гента, Бельгия
%В-9000 Гент, Бельгия

\vspace*{4pt}

\noindent
\textbf{Лупенцов Олег Сергеевич} (р.\ 1986)~---
аспирант Омского государственного института сервиса
%Омск 644043, ул.\ Певцова 13

\vspace*{4pt}

\noindent
\textbf{Лучко Олег Николаевич} (р.\ 1961)~---
кандидат педагогических наук, профессор, заведующий кафедрой 
Омского государственного института сервиса
%Омск 644043, ул.\ Певцова 13

\vspace*{4pt}

\noindent
\textbf{Малашенко Юрий Евгеньевич} (р.\ 1946)~---
доктор фи\-зи\-ко-ма\-те\-ма\-ти\-че\-ских наук, заведующий сектором 
Вычислительного центра им.\ А.\,А.~Дородницына Российской академии наук
%Адрес: 119333, Москва, ул. Вавилова, 40,

\vspace*{4pt}

\noindent
\textbf{Маньяков Юрий Анатольевич} (р.\ 1984)~---
кандидат технических наук, научный сотрудник Орловского филиала Института проб\-лем информатики
Российской академии наук
%302025, г.Орел, Московское шоссе, д.137

\vspace*{4pt}

\noindent
\textbf{Маренко Валентина Афанасьевна} (р.\ 1951)~---
кандидат технических наук, доцент, старший научный сотрудник 
Института математики им.\ С.\,Л.~Соболева Сибирского отделения Российской академии наук
%Новосибирск 630090, пр. ак. Коптюга, 4 

\vspace*{3pt}

\noindent 
\textbf{Морозов Евсей Викторович} (р.\ 1947)~--- доктор 
фи\-зи\-ко-ма\-те\-ма\-ти\-че\-ских, профессор, ведущий научный сотрудник 
Института прикладных математических исследований Карельского научного центра Российской
академии наук; 
%%185910 Россия, Республика Карелия, г.\ Петрозаводск, ул.\ Пушкинская, 11
профессор Петрозаводского государственного университета, Петрозаводск
%185910 Россия, Республика Карелия, г.\ Петрозаводск, пр.\ Ленина, 33

%\pagebreak

\vspace*{3pt}

\noindent
\textbf{Назарова Ирина Александровна} (р.\ 1966)~---
кандидат фи\-зи\-ко-ма\-те\-ма\-ти\-че\-ских наук, 
научный сотрудник Вычислительного центра им.\ А.\,А.~Дородницына Российской академии наук 
%Адрес: 119333, Москва, ул. Вавилова, 40

\vspace*{3pt}

\noindent
\textbf{Павлов Игорь Валерианович} (р.\ 1945)~--- 
доктор фи\-зи\-ко-ма\-те\-ма\-ти\-че\-ских наук, профессор МГТУ им.\ Н.\,Э.~Баумана 
%Москва 105005, 2-я Бауманская ул., д.~5 

%\pagebreak

\vspace*{3pt}

\noindent 
\textbf{Потахина Любовь Викторовна} (р.\ 1989)~--- аспирантка
Института прикладных математических исследований Карельского научного центра
Российской академии наук; 
%%185910 Россия, Республика Карелия, г.\ Петрозаводск, ул.\ Пушкинская, 11
инженер Петрозаводского государственного университета, Петрозаводск
%185910 Россия, Республика Карелия, г.\ Петрозаводск, пр.\ Ленина, 33

\vspace*{3pt}

\noindent 
\textbf{Рождественский Юрий Владимирович} (р.\ 1952)~--- 
кандидат технических наук, заведующий сектором Института проб\-лем информатики
Российской академии наук

\vspace*{3pt}

\noindent 
\textbf{Синицын Игорь Николаевич} (р.\ 1940)~--- доктор технических наук,
профессор, заслуженный деятель\linebreak\vspace*{-12pt}

\columnbreak

\noindent
 науки РФ, заведующий отделом Института проб\-лем информатики
Российской академии наук

\vspace*{7pt}


\noindent
\textbf{Сиротинин Денис Олегович} (р.\ 1984)~---
кандидат технических наук, научный сотрудник Орловского филиала Института проб\-лем информатики
Российской академии наук
%302025, г.Орел, Московское шоссе, д.137

\vspace*{7pt}

%\columnbreak

\noindent 
\textbf{Соколов  Игорь Анатольевич} (р.\ 1954)~--- академик (действительный член) Российской 
академии наук, доктор технических наук, директор Института проб\-лем информатики
Российской академии наук

\vspace*{7pt}

\noindent
\textbf{Степченков Юрий Афанасьевич} (р.\ 1951)~---
кандидат технических наук, заведующий отделом Института проб\-лем информатики
Российской академии наук

\vspace*{7pt}

\noindent
\textbf{Сурков Алексей Викторович} (р.\ 1978)~--- 
старший научный сотрудник На\-уч\-но-ис\-сле\-до\-ва\-тель\-ско\-го 
института системных исследований Российской академии наук
%117218, Москва, Нахимовский просп., 36, к.1 

\vspace*{7pt}

\noindent 
\textbf{Шестаков Олег Владимирович} (р.\ 1976)~--- доктор 
фи\-зи\-ко-ма\-те\-ма\-ти\-че\-ских, доцент кафедры математической статистики 
факультета вычисли\-тельной математики и кибернетики Московского 
государственного университета им.\ М.\,В.~Ломоносова; 
%119991, Москва ГСП-1, Ленинские горы, д.\ 1, стр. 52
старший научный сотрудник Института проб\-лем информатики
Российской академии наук
%, Москва 119333, ул. Вавилова, д.~44, корп.~2

\vspace*{7pt}

\noindent 
\textbf{Шоргин Сергей Яковлевич} (р.\ 1952.)~--- доктор
фи\-зи\-ко-ма\-те\-ма\-ти\-че\-ских наук, профессор, заместитель директора Института 
проб\-лем информатики Российской академии наук





%%%%%%%%%%%%%%%%%%%%%%%%%%%%%%%%%%%%%%%%%%%%%%%%%%%%%%%%%%%%%%%%%%%%%%%%%%%%%%%




%\def\rightkol{ОБ АВТОРАХ}
%\def\leftkol{ОБ АВТОРАХ}

 \label{end\stat}





%\def\leftfootline{\small{\textbf{\thepage}
%\hfill ИНФОРМАТИКА И ЕЁ ПРИМЕНЕНИЯ\ \ \ том~7\ \ \ выпуск~1\ \ \ 2013}
%}%
% \def\rightfootline{\small{ИНФОРМАТИКА И ЕЁ ПРИМЕНЕНИЯ\ \ \ том~7\ \ \ выпуск~1\ \ \ 2013
%\hfill \textbf{\thepage}}}


%\thispagestyle{myheadings}



\end{multicols}

\newpage


%\vspace*{-48pt}
\begin{center}\LARGE
\textit{About Authors}
\end{center}

\thispagestyle{empty}
\def\tit{\ }

\def\aut{\ }

\def\auf{\ }


\def\leftkol{ABOUT AUTHORS}

\def\rightkol{ABOUT AUTHORS}

\vspace*{-18pt}

\titele{\tit}{\aut}{\auf}{\leftkol}{\rightkol}

%\vspace*{36pt}

\def\rightmark{{\noindent\hbox to \textwidth{\hfill\small ABOUT AUTHORS
%\hfill \large\bf\thepage
}}}
\def\leftmark{{\noindent\parbox{\textwidth}{
%\raggedleft\large\bf\thepage \hfill
\small\textrm{ABOUT AUTHORS}\hfill}}}


\def\leftfootline{\small{\textbf{\thepage}
\hfill ИНФОРМАТИКА И ЕЁ ПРИМЕНЕНИЯ\ \ \ том~6\ \ \ выпуск~2\ \ \ 2012}
}%
 \def\rightfootline{\small{ИНФОРМАТИКА И ЕЁ ПРИМЕНЕНИЯ\ \ \ том~6\ \ \ выпуск~2\ \ \ 2012
\hfill \textbf{\thepage}}}


\begin{multicols}{2}

\noindent
\textbf{Agalarov Yaver M.} (b.\ 1952)~--- Candidate of Science (PhD)
in technology, 
leading scientist, Institute of Informatics Problems, Russian Academy of Sciences

\vspace*{5pt}


  \noindent
\textbf{Bosov Alexey V.} (b.\ 1969)~--- Doctor of Science in technology, Head of
Laboratory, Institute of Informatics Problems, Russian Academy of Sciences

\vspace*{5pt}


\noindent
\textbf{Dulin Sergey K.} (b.\ 1950)~--- Doctor of Science in technology, 
professor, senior scientist, Institute of Informatics Problems, Russian Academy of Sciences

\vspace*{5pt}

\noindent
\textbf{Gorshenin Andrey K.}~--- (b.\ 1986)~--- Candidate of Science (PhD)
in physics and mathematics,
senior scientist, Institute of Informatics Problems, Russian Academy of Sciences

\vspace*{5pt}

\noindent
\textbf{Kalenov Nikolay E.}  (b.\ 1945)~--- Doctor of Science in technology,
professor, Director, Library for Natural Sciences,  Russian Academy of Sciences 

\vspace*{5pt}

\noindent
\textbf{Kalinichenko Leonid A.} (b.\ 1937)~--- Doctor of Science in physics and mathematics, 
professor, Honored scientist of RF, 
Head of Laboratory, Institute of Informatics Problems, Russian Academy of Sciences 

\vspace*{5pt}

\noindent
\textbf{Karpov Alexey A.} (b.\ 1978)~--- Candidate of Science (PhD) in technology, 
senior scientist, St.\ Petersburg Institute for
Informatics and Automation,  Russian Academy of Sciences

\vspace*{5pt}

\noindent
\textbf{Kuznetsov Igor P.} (b.\ 1938)~--- Doctor of Science in technology, 
professor, principal scientist, Institute of Informatics Problems, Russian Academy of Sciences

\vspace*{5pt}


\noindent
\textbf{Markova Natalia A.} (b.\ 1950)~--- Candidate of Science (PhD) in
physics and mathematics, leading scientist,  
Institute of Informatics Problems, Russian Academy of Sciences

\vspace*{5pt}

\noindent
\textbf{Nikolaev Andrey V.} (b.\ 1985)~--- Candidate of Science (PhD) in technology, 
senior lecturer, Tchaikovsky Technological Institute, Branch of the Izhevsk State Technical 
University

\vspace*{6pt}

\noindent
\textbf{Pavlov Igor V.} (b.\ 1945)~---  Doctor of Science in physics and mathematics,
professor, Bauman Moscow State Technical University

\vspace*{6pt}

%\columnbreak

\noindent
\textbf{Rozenberg Igor N.} (b.\ 1965)~--- Doctor of Science in technology, 
First Deputy Director General, Research \& Design Institute for Information 
Technology, Signalling and Telecommunications on Railway Transport (JSC NIIAS)

\vspace*{6pt}


\noindent
\textbf{Semenov Konstantin K.} (b.\ 1986)~--- MPhil, 
associate professor, St.\ Petersburg State Polytechnical University

\vspace*{6pt}

\noindent
\textbf{Sharnin Mikhail M.} (b.\ 1959)~--- Candidate of Science (PhD) 
in technology, senior scientist, Institute of Informatics Problems, Russian Academy of Sciences

\vspace*{6pt}

\noindent 
\textbf{Shestakov Oleg V.} (b.\ 1976)~--- Candidate of Science (PhD) in physics and mathematics,
associate professor, Department of Mathematical Statistics, Faculty of Computational Mathematics and Cybernetics,
M.\,V.~Lomonosov Moscow State University; senior scientist, Institute of Informatics Problems, 
Russian Academy of Sciences

\vspace*{6pt}

\noindent
\textbf{Stupnikov Sergey A.} (b.\ 1978)~--- Candidate of Science (PhD) in technology, 
senior scientist, Institute of Informatics Problems, Russian Academy of Sciences 

\vspace*{6pt}

\noindent
\textbf{Umansky Vladimir I.} (b.\ 1954)~--- Candidate of Science (PhD) in technology, 
Director General, ``IntechGeoTrans'' Closed Joint Stock Company

\vspace*{6pt}

\noindent
\textbf{Zhevnerchuk Dmitry V.} (b.\ 1978)~--- Candidate of Science (PhD) in technology, 
associate professor, Tchaikovsky Technological Institute, Branch of the Izhevsk State 
Technical University

%\vspace*{6pt}

\def\leftfootline{\small{\textbf{\thepage}
\hfill ИНФОРМАТИКА И ЕЁ ПРИМЕНЕНИЯ\ \ \ том~6\ \ \ выпуск~2\ \ \ 2012}
}%
 \def\rightfootline{\small{ИНФОРМАТИКА И ЕЁ ПРИМЕНЕНИЯ\ \ \ том~6\ \ \ выпуск~2\ \ \ 2012
\hfill \textbf{\thepage}}}



%\thispagestyle{myheadings}

\end{multicols}
\newpage

%   \vspace*{-48pt}

\begin{center}
\vspace*{6pt}
\mbox{%
\epsfxsize=53.502mm
\epsfbox{foto-1.eps}
}
\end{center}

\vspace*{6pt} %Академик


   \begin{center}
\fbox{\Large\textbf{Профессор Игорь Алексеевич Ушаков}}\\[12pt]
\textbf{\large 22.01.1935--27.02.2015}
   \end{center}


   %\vspace*{2.5mm}

   \vspace*{5mm}

   \thispagestyle{empty}

%\

%\vspace*{-12pt}


Редакционный совет и редакционная коллегия журнала <<Информатика и~её применения>> с~глубоким прискорбием извещают, что 27~февраля 2015~г.\ после тяжелой
и~продолжительной болезни скончался Игорь Алексеевич Ушаков~--- доктор технических наук, профессор, член редколлегии журнала <<Информатика и ее применения>>.

Игорь Алексеевич Ушаков окончил Московский авиационный институт, в~1963~г.\ защитил кандидатскую, а~в~1968~г.~--- докторскую диссертацию. С~1958 по 1989~гг.\ работал в~ряде научно-исследовательских организаций СССР, в~том числе руководил отделами в~НИИ АА и~ВЦ АН СССР; с 1969 по 1989 гг. преподавал в~МФТИ (был профессором, а~затем заведующим кафедрой) и~в~МЭИ. С~1989~г.~---- в~США: являлся профессором университета Дж.\ Вашингтона, университета Дж.\ Мэйсона и~Калифорнийского университета, сотрудником компаний MCI, Qualcomm и Hughes.

И.\,А.~Ушаков с момента основания журнала <<Надежность и~контроль качества>> был заместителем ответственного редактора, а~затем на протяжении многих лет членом редколлегии. В~2006~г.\ основал электронный международный журнал ``Reliability: Theory \& Application'', главным редактором которого оставался до конца жизни.

Учебниками и справочниками по теории надежности, написанными И.\,А.~Ушаковым, пользовались и~пользуются несколько поколений ученых и~специалистов в~разных странах мира.

Игорь Алексеевич всегда уделял огромное внимание работе с~молодежью; более~50 его учеников защитили докторские и~кандидатские диссертации.

И.\,А.~Ушаков вел активную научно-про\-све\-ти\-тель\-скую деятельность. В~частности, он был одним из организаторов и~руководителей Московского кабинета качества и~надежности при Политехническом музее (целью этого Кабинета было оказание консультаций работникам промышленных предприятий и~чтение курсов лекций для инженеров, занимающихся проблемой надежности). Находясь в~США, И.\,А.~Ушаков создал международный ин\-тер\-нет-фо\-рум им.\ Б.\,В.~Гнеденко, объединивший около~400~видных специалистов по приложениям теории вероятностей и~математической статистики, преимущественно в~об\-ласти теории надежности и~анализа риска, из десятков стран мира; коллективным членов этого Форума является и~наш журнал. Цели Форума~--- содействие контактам между специалистами из разных стран, организация обмена профессиональными 
новостями и~информацией (новые публикации, предстоящие события и~др.). Также необходимо отметить большое число на\-уч\-но-по\-пу\-ляр\-ных работ, опубликованных И.\,А.~Ушаковым.

И.\,А.~Ушаков обладал большим личным обаянием, имел широкий круг интересов. Все знавшие И.\,А.~Ушакова всегда будут помнить его как замечательного ученого и~прекрасного человека.

\bigskip

Редакционный совет и редакционная коллегия журнала <<Информатика и~её применения>> 
выражают глубокие соболезнования родным и близким покойного, всем, кто его знал и~работал с~ним.


\vspace*{-60pt} {%\small 
{%\baselineskip=10.65pt
\section*{Правила подготовки рукописей статей для публикации в журнале
<<Информатика и её применения>>}

\thispagestyle{empty}

 Журнал <<Информатика и её применения>> публикует
теоретические, обзорные и дискуссионные статьи, посвященные научным
исследованиям и разработкам в области информатики и ее приложений. Журнал
издается на русском языке. По специальному решению редколлегии отдельные статьи,
в виде исключения, могут печататься на английском языке.
Тематика журнала охватывает следующие направления:
\begin{itemize}
\item теоретические основы информатики;
\item математические методы исследования сложных систем и процессов;
\item информационные системы и сети;
\item информационные технологии;
\item архитектура и программное
обеспечение вычислительных комплексов и сетей.
\end{itemize}
\begin{enumerate}
\item В журнале печатаются результаты, ранее не
опубликованные и не предназначенные к одновременной публикации в других
изданиях. Публикация не должна нарушать закон об авторских правах. Направляя
свою рукопись в редакцию, авторы автоматически передают учредителям и
редколлегии неисключительные права на издание данной статьи на русском языке и
на ее распространение в России и за рубежом. При этом за авторами сохраняются
все права как собственников данной рукописи. В связи с этим авторами должно
быть представлено в редакцию письмо в следующей форме:
Соглашение о передаче права на публикацию:

\textit{<<Мы, нижеподписавшиеся, авторы рукописи <<$\qquad\qquad$>>, передаем
учредителям и редколлегии журнала <<Информатика и её применения>>
неисключительное право опубликовать данную рукопись статьи на русском языке как
в печатной, так и в электронной версиях журнала. Мы подтверждаем, что данная
публикация не нарушает авторского права других лиц или организаций. Подписи
авторов: (ф.\,и.\,о., дата, адрес)>>.}

Указанное соглашение может быть представлено как в бумажном виде, так и в виде 
отсканированной копии (с подписями авторов).

Редколлегия вправе запросить у авторов экспертное заключение о возможности
опубликования представленной статьи в открытой печати.
\item Статья
подписывается всеми авторами. На отдельном листе представляются данные автора
(или всех авторов): фамилия, полные имя и отчество, телефон, факс, e-mail,
почтовый адрес. Если работа выполнена несколькими авторами, указывается фамилия
одного из них, ответственного за переписку с редакцией.
\item Редакция журнала
осуществляет самостоятельную экспертизу присланных статей. Возвращение рукописи
на доработку не означает, что статья уже принята к печати. Доработанный вариант
с ответом на замечания рецензента необходимо прислать в редакцию.
\item Решение
редакционной коллегии о принятии статьи к печати или ее отклонении сообщается
авторам. Редколлегия не обязуется направлять рецензию авторам отклоненной
статьи.
\item Корректура статей высылается авторам для просмотра. Редакция
просит авторов присылать свои замечания в кратчайшие сроки.
\item При
подготовке рукописи в MS Word рекомендуется использовать следующие настройки.
Параметры страницы: формат~--- А4; ориентация~--- книжная; поля (см): внутри~---
2,5, снаружи~--- 1,5, сверху~--- 2, снизу~--- 2, от края до нижнего
колонтитула~--- 1,3. Основной текст: стиль~--- <<Обычный>>: шрифт Times New
Roman, размер 14~пунктов, абзацный отступ~--- 0,5~см, 1,5 интервала,
выравнивание~--- по ширине. Рекомендуемый объем рукописи~--- не свыше
25~страниц указанного формата. Ознакомиться с шаблонами, содержащими примеры
оформления, можно по адресу в Интернете:
\textsf{http://www.ipiran.ru/journal/template.doc}.
\item К рукописи, предоставляемой в 2-х
экземплярах, обязательно прилагается электронная версия статьи (как правило, в
форматах MS WORD (.doc) или \LaTeX\  (.tex), а также~--- дополнительно~--- в
формате .pdf) на дискете, лазерном диске или по электронной почте. Сокращения
слов, кроме стандартных, не применяются. Все страницы рукописи должны быть
пронумерованы.
\item Статья должна содержать следующую информацию на русском и
английском языках: название, Ф.И.О.\ авторов, места работы авторов и их
электронные адреса,
подробные сведения об авторах, оформленные в соответствии с форматом, определяемым файлами

\noindent
{\sf http://www.ipiran.ru/journal/issues/2011\_05\_01/authors.asp} и 

\noindent
{\sf http://www.ipiran.ru/journal/issues/2011\_01\_eng/authors.asp},

\noindent
аннотация (не более 100~слов), ключевые слова. Ссылки на
литературу в тексте статьи нумеруются (в квадратных скобках) и располагаются в
порядке их первого упоминания. В~списке литературы не должно быть позиций, на 
которые нет ссылки в тексте статьи.
Все фамилии авторов, заглавия статей, названия
книг, конференций и~т.\,п.\ даются на языке оригинала, если этот язык
использует кириллический или латинский алфавит.
\item Присланные в редакцию
материалы авторам не возвращаются.
\item При отправке файлов по электронной
почте просим придерживаться следующих правил:
\begin{itemize}
\item указывать в поле subject (тема) название журнала и фамилию автора;
\item использовать attach (присоединение);
\item в случае больших объемов информации возможно
использование общеизвестных архиваторов (ZIP, RAR);
\item в состав электронной версии статьи должны входить: файл, содержащий текст статьи, и файл(ы),
содержащий(е) иллюстрации.
\end{itemize}
\item Журнал <<Информатика и её применения>> является некоммерческим изданием. 
Плата за публикацию с авторов не взимается, гонорар авторам не выплачивается.
\end{enumerate}
\thispagestyle{empty}

\medskip
\noindent
\textbf{Адрес редакции:} Москва 119333,
ул.~Вавилова, д.~44, корп.~2, ИПИ РАН\\
\hphantom{\textbf{Адрес редакции:} }Тел.: +7 (499) 135-86-92\ \
Факс:  +7 (495) 930-45-05\ \  E-mail:   rust@ipiran.ru }

\vfill
\begin{center}


Технический редактор Л. Кокушкина\\
Выпускающий редактор Т. Торжкова\\
Художественный редактор М. Седакова\\
Сдано в набор 11.01.12. Подписано в печать 02.03.12. Формат 60 х 84 / 8\\
Бумага офсетная. Печать цифровая. Усл.-печ. л. 19,0. Уч.-изд. л. 24,2. Тираж 100 экз.\\
\ \\
Заказ №\,279\\
\ \\
Издательство <<ТОРУС ПРЕСС>>, Москва 119991, ул. Косыгина, д.~4\\
torus@torus-press.ru; http://www.torus-press.ru\\
\ \\
Отпечатано в Академиздатцентре <<Наука>> РАН с готовых файлов\\
Москва 121099, Шубинский пер., д.~6\\
\end{center}


%\end{document}

%\include{IPPM-25}

%\def\stat{cont}
{%\hrule\par
%\vskip 7pt % 7pt
\raggedleft\Large \bf%\baselineskip=3.2ex
А\,В\,Т\,О\,Р\,С\,К\,И\,Й\ \ У\,К\,А\,З\,А\,Т\,Е\,Л\,Ь\ \ З\,А\ \ 2\,0\,1\,0 г. \vskip 17pt
    \hrule
    \par
\vskip 21pt plus 6pt minus 3pt }

\label{st\stat}

\def\tit{\ }

\def\aut{\ }
\def\auf{\ }

\def\leftkol{\ } % ENGLISH ABSTRACTS}

\def\rightkol{\ } %АВТОРСКИЙ УКАЗАТЕЛЬ ЗА 2010 г.} %ENGLISH ABSTRACTS}

\titele{\tit}{\aut}{\auf}{\leftkol}{\rightkol}

\vspace*{-12pt}

{\tabcolsep=3pt
\begin{tabular}{p{388pt}rr}
&\textbf{Выпуск} & \textbf{Стр.}\\[6pt]
\hangindent=23pt\noindent\textbf{Арутюнян~А.\,Р.} Моделирование влияния деформаций отпечатков пальцев на 
точность\linebreak
\vspace*{-12pt}\\
\hspace*{23pt}дактилоскопической идентификации$\dotfill$&1&51\\
\hangindent=23pt\noindent\textbf{Архипов~О.\,П., Зыкова~З.\,П.} Интеграция гетерогенной информации о цветных 
пикселях\linebreak
\vspace*{-12pt}\\
\hspace*{23pt}и их цветовосприятии$\dotfill$&4&15\\
\hangindent=23pt\noindent\textbf{Баранов~С.\,И., Френкель~С.\,Л., Захаров~В.\,Н.} Полуформальная верификация 
цифрового устройства с конвейером, основанная на использовании алгоритмических машин\linebreak
\vspace*{-12pt}\\
\hspace*{23pt}состояния$\dotfill$&4&49\\
\textbf{Бекетова~И.\,В.} см.~Каратеев~С.\,Л.&&\\
\textbf{Белоусов~В.\,В.} см.~Синицын~И.\,Н.&&\\
\hangindent=23pt\noindent\textbf{Бенинг~В.\,Е., Королев~Р.\,А.} О предельном поведении мощностей критериев в 
случае\linebreak
\vspace*{-12pt}\\
\hspace*{23pt}распределения Лапласа$\dotfill$&2&63\\
\hangindent=23pt\noindent\textbf{Бенинг~В.\,Е., Сипина~А.\,В.} Асимптотическое разложение для мощности 
критерия,\linebreak
\vspace*{-12pt}\\
\hspace*{23pt}основанного на выборочной медиане, в случае распределения Лапласа$\dotfill$&1&18\\
\textbf{Бондаренко~А.\,В.} см.~Каратеев~С.\,Л.&&\\
\hangindent=23pt\noindent\textbf{Бородина~А.\,В., Морозов~Е.\,В.} Об оценивании асимптотики вероятности 
большого\linebreak
\vspace*{-12pt}\\
\hspace*{23pt}уклонения стационарной регенеративной очереди с одним прибором$\dotfill$&3&29\\
\hangindent=23pt\noindent\textbf{Бунтман~Н.\,В., Минель~Ж.-Л., Ле~Пезан~Д., Зацман~И.\,М.} Типология и 
компьютерное\linebreak
\vspace*{-12pt}\\
\hspace*{23pt}моделирование трудностей перевода$\dotfill$&3&77\\
\textbf{Визильтер~Ю.\,В.} см.~Каратеев~С.\,Л.&&\\
\hangindent=23pt\noindent\textbf{Гавриленко~С.\,В.} Оценки скорости сходимости распределений случайных сумм с 
безгранично делимыми индексами к нормальному закону$\dotfill$&4&81\\
\hangindent=23pt\noindent\textbf{Григорьева~М.\,Е., Шевцова~И.\,Г.} Уточнение неравенства 
Каца--Берри--Эссеена$\dotfill$&2&75\\
\hangindent=23pt\noindent\textbf{Грушо~А.\,А., Грушо~Н.\,А., Тимонина~Е.\,Е.} Поиск конфликтов в политиках 
безопасности: модель случайных графов$\dotfill$&3&38\\
\textbf{Грушо~Н.\,А.} см.~Грушо~А.\,А.&&\\
\hangindent=23pt\noindent\textbf{Гудков~В.\,Ю.} Математические модели изображения отпечатка пальца на основе 
описания линий$\dotfill$&1&58\\
\textbf{Гуртов~А.\,В.} см.~Лукьяненко~А.\,С.&&\\
\textbf{Желтов~С.\,Ю.} см.~Каратеев~С.\,Л.&&\\
\hangindent=23pt\noindent\textbf{Захаров~А.\,А., Серебряков~В.\,А.} Система управления электронной библиотекой 
LibMeta$\dotfill$&4&2\\
\textbf{Захаров~В.\,Н.} см.~Баранов~С.\,И.&&\\
\textbf{Захарова~Т.\,В.} см.~Матвеева~С.\,С.&&\\
\hangindent=23pt\noindent\textbf{Зацаринный~А.\,А., Чупраков~К.\,Г.} Некоторые аспекты выбора технологии для 
постро-\linebreak
\vspace*{-12pt}\\
\hspace*{23pt}ения систем отображения информации ситуационного центра$\dotfill$&3&59\\
\textbf{Зацман~И.\,М.} см.~Бунтман~Н.\,В.&&\\
\hangindent=23pt\noindent\textbf{Зейфман~А.\,И., Коротышева~А.\,В., Сатин~Я.\,А., Шоргин~С.\,Я.} Об 
устойчивости нестаци-\linebreak
\vspace*{-12pt}\\
\hspace*{23pt}онарных систем обслуживания с катастрофами$\dotfill$&3&9\\
\textbf{Зыкова~З.\,П.} см.~Архипов~О.\,П.&&\\
\hangindent=23pt\noindent\textbf{Илюшин~Г.\,Я., Соколов~И.\,А.} Организация управляемого доступа пользователей 
к\linebreak
\vspace*{-12pt}\\
\hspace*{23pt}разнородным ведомственным информационным ресурсам$\dotfill$&1&24\\
\hangindent=23pt\noindent\textbf{Кавагучи~Ю., Ульянов~В.\,В., Фуджикоши~Я.} Приближения для статистик, 
описывающих\linebreak
\vspace*{-12pt}\\
\hspace*{23pt}геометрические свойства данных большой размерности, с оценками 
ошибок$\dotfill$&1&12\\
\hangindent=23pt\noindent\textbf{Каратеев~С.\,Л., Бекетова~И.\,В., Ососков~М.\,В., Князь~В.\,А., 
Визильтер~Ю.\,В., Бондаренко~А.\,В., Желтов~С.\,Ю.} Автоматизированный контроль 
качества цифровых\linebreak
\vspace*{-12pt}\\
\hspace*{23pt}изображений для персональных документов$\dotfill$&1&65\\
\end{tabular}
}

\pagebreak

\def\leftkol{АВТОРСКИЙ УКАЗАТЕЛЬ ЗА 2010 г.} % ENGLISH ABSTRACTS}

\def\rightkol{АВТОРСКИЙ УКАЗАТЕЛЬ ЗА 2010 г.} %ENGLISH ABSTRACTS}

{\tabcolsep=3pt
\begin{tabular}{p{388pt}rr}
&\textbf{Выпуск} & \textbf{Стр.}\\[3pt]
\hangindent=23pt\noindent\textbf{Козеренко~Е.\,Б.} Лингвистические фильтры в статистических моделях машинного\linebreak
\vspace*{-12pt}\\
\hspace*{23pt}перевода$\dotfill$&2&83\\
\hangindent=23pt\noindent\textbf{Козеренко~Е.\,Б., Кузнецов~И.\,П.} Когнитивно-лингвистические представления в 
систе-\linebreak
\vspace*{-12pt}\\
\hspace*{23pt}мах обработки текстов$\dotfill$&3&69\\
\textbf{Князь~В.\,А.} см.~Каратеев~С.\,Л.&&\\
\hangindent=23pt\noindent\textbf{Колесников~А.\,В., Солдатов~С.\,А.} Алгоритм координации для гибридной 
интеллектуальной системы решения сложной задачи оперативно-производственного\linebreak
\vspace*{-12pt}\\
\hspace*{23pt}планирования$\dotfill$&4&61\\
\hangindent=23pt\noindent\textbf{Коновалов~М.\,Г.} О планировании потоков в системах вычислительных 
ресурсов$\dotfill$&2&3\\
\textbf{Конушин~А.\,С.} см.~Конушин~В.\,С.&&\\
\hangindent=23pt\noindent\textbf{Конушин~В.\,С., Кривовязь~Г.\,Р., Конушин~А.\,С.} Алгоритм распознавания людей 
в видео-\linebreak
\vspace*{-12pt}\\
\hspace*{23pt}последовательности по одежде$\dotfill$&1&74\\
\textbf{Корепанов~Э.\, Р.} см.~Синицын~И.\,Н.&&\\
\textbf{Королев~В.\,Ю.} см.~Соколов~И.\,А.&&\\
\textbf{Королев~Р.\,А.} см.~Бенинг~В.\,Е.&&\\
\textbf{Коротышева~А.\,В.} см.~Зейфман~А.\,И.&&\\
\hangindent=23pt\noindent\textbf{Кривенко~М.\,П.} Непараметрическое оценивание элементов байесовского 
клас\-си-\linebreak
\vspace*{-12pt}\\
\hspace*{23pt}фикатора$\dotfill$&2&13\\
\textbf{Кривовязь~Г.\,Р.} см.~Конушин~В.\,С.&&\\
\textbf{Крылов~А.\,С.} см.~Павельева~Е.\,А.&&\\
\hangindent=23pt\noindent\textbf{Крылов~В.\,А.} Моделирование и классификация многоканальных дистанционных\linebreak
\vspace*{-12pt}\\
\hspace*{23pt}изображений с использованием копул$\dotfill$&4&34\\
\hangindent=23pt\noindent\textbf{Крючин~О.\,В.} Разработка параллельных эвристических алгоритмов подбора 
весовых\linebreak
\vspace*{-12pt}\\
\hspace*{23pt}коэффициентов искусственной нейтронной сети$\dotfill$&2&53\\
\hangindent=23pt\noindent\textbf{Кудрявцев~А.\,А., Шоргин~С.\,Я.} Байесовские модели массового обслуживания и 
надеж-\linebreak
\vspace*{-12pt}\\
\hspace*{23pt}ности: характеристики среднего числа заявок в системе $M\vert M \vert 1\vert 
\infty$$\dotfill$&3&16\\
\hangindent=23pt\noindent\textbf{Кузнецов~А.\,А.} Связь между временными и структурно-топологическими 
характери-\linebreak
\vspace*{-12pt}\\
\hspace*{23pt}стиками диаграмм ритма сердца здоровых людей$\dotfill$&4&39\\
\textbf{Кузнецов~И.\,П.} см.~Козеренко~Е.\,Б.&&\\
\textbf{Ле~Пезан~Д.} см.~Бунтман~Н.\,В.&&\\
\hangindent=23pt\noindent\textbf{Лукьяненко~А.\,С., Морозов~Е.\,В., Гуртов~А.\,В.} Анализ сетевого протокола с общей 
функ-\linebreak
\vspace*{-12pt}\\
\hspace*{23pt}цией расширения окна передачи сообщения при конфликтах$\dotfill$&2&46\\
\hangindent=23pt\noindent\textbf{Лямин~О.\,О.} О предельном поведении мощностей критериев в случае обобщенного\linebreak
\vspace*{-12pt}\\
\hspace*{23pt}распределения Лапласа$\dotfill$&3&47\\
\hangindent=23pt\noindent\textbf{Маркин~А.\,В., Шестаков~О.\,В.} Асимптотики оценки риска при пороговой 
обработке\linebreak
\vspace*{-12pt}\\
\hspace*{23pt}вейвлет-вейглет коэффициентов в задаче томографии$\dotfill$&2&36\\
\hangindent=23pt\noindent\textbf{Матвеева~С.\,С., Захарова~Т.\,В.} Сети массового обслуживания с наименьшей 
длиной\linebreak
\vspace*{-12pt}\\
\hspace*{23pt}очереди$\dotfill$&3&22\\
\hangindent=23pt\noindent\textbf{Матюшенко~С.\,И.} Стационарные характеристики двухканальной системы 
обслужива-\linebreak
\vspace*{-12pt}\\
\hspace*{23pt}ния с переупорядочиванием заявок и распределениями фазового типа$\dotfill$&4&68\\
\textbf{Минель~Ж.-Л.} см.~Бунтман~Н.\,В.&&\\
\textbf{Морозов~Е.\,В.} см.~Бородина~А.\,В.&&\\
\textbf{Морозов~Е.\,В.} см.~Лукьяненко~А.\,С.&&\\
\textbf{Ососков~М.\,В.} см.~Каратеев~С.\,Л.&&\\
\hangindent=23pt\noindent\textbf{Павельева~Е.\,А., Крылов~А.\,С.} Поиск и анализ ключевых точек радужной 
оболочки\linebreak
\vspace*{-12pt}\\
\hspace*{23pt}глаза методом преобразования Эрмита$\dotfill$&1&79\\
\textbf{Печинкин~А.\,В.} см.~Френкель~С.\,Л.,&&\\
\hangindent=23pt\noindent\textbf{Протасов~В.\,И.} Составление субъективного портрета с использованием 
эволюционно-\linebreak
\vspace*{-12pt}\\
\hspace*{23pt}го морфинга и квалиметрия метода$\dotfill$&1&83\\
\hangindent=23pt\noindent\textbf{Рудаков~К.\,В., Торшин~И.\,Ю.} Вопросы разрешимости задачи распознавания 
вторичной\linebreak
\vspace*{-12pt}\\
\hspace*{23pt}структуры белка$\dotfill$&2&25\\
\textbf{Сатин~Я.\,А.} см.~Зейфман~А.\,И.&&\\
\hangindent=23pt\noindent\textbf{Сейфуль-Мулюков~Р.\,Б.} Нефть как носитель информации о своем 
происхождении,\linebreak
\vspace*{-12pt}\\
\hspace*{23pt}структуре и эволюции$\dotfill$&1&41\\
\end{tabular}
}

{\tabcolsep=3pt
\begin{tabular}{p{388pt}rr}
&\textbf{Выпуск} & \textbf{Стр.}\\[6pt]
\textbf{Семендяев~Н.\,Н.} см.~Синицын~И.\,Н.&&\\
\textbf{Серебряков~В.\,А.} см.~Захаров~А.\,А.&&\\
\textbf{Синицын~В.\,И.} см.~Синицын~И.\,Н.&&\\
\hangindent=23pt\noindent\textbf{Синицын~И.\,Н., Синицын~В.\,И., Корепанов~Э.\, Р., Белоусов~В.\,В., 
Семендяев~Н.\,Н.} Оперативное построение информационных моделей движения полюса 
Земли\linebreak
\vspace*{-12pt}\\
\hspace*{23pt}методами линейных и линеаризованных фильтров$\dotfill$&1&2\\
\textbf{Сипина~А.\,В.} см.~Бенинг~В.\,Е.&&\\
\hangindent=23pt\noindent\textbf{Соколов~И.\,А.} О работах заслуженного деятеля науки Российской Федерации 
И.\,Н.~Синицына в области информационных технологий и автоматизации (к 70-летию\linebreak
\vspace*{-12pt}\\
\hspace*{23pt}со дня рождения)$\dotfill$&3&84\\
\textbf{Соколов~И.\,А.} см.~Илюшин~Г.\,Я.&&\\
\hangindent=23pt\noindent\textbf{Соколов~И.\,А., Королев~В.\,Ю.} Предисловие$\dotfill$&2&2\\
\textbf{Солдатов~С.\,А.} см.~Колесников~А.\,В.&&\\
\hangindent=23pt\noindent\textbf{Степанов~С.\,Ю.} Использование координатного метода фрагментации 
коммутаторной\linebreak
\vspace*{-12pt}\\
\hspace*{23pt}нейронной сети для сокращения трафика$\dotfill$&2&57\\
\textbf{Тимонина~Е.\,Е.} см.~Грушо~А.\,А.&&\\
\textbf{Торшин~И.\,Ю.} см.~Рудаков~К.\,В.&&\\
\textbf{Ульянов~В.\,В.} см.~Кавагучи~Ю.&&\\
\textbf{Фазекаш~И.} см.~Чупрунов~А.\,Н.&&\\
\textbf{Френкель~С.\,Л.} см.~Баранов~С.\,И.&&\\
\hangindent=23pt\noindent\textbf{Френкель~С.\,Л., Печинкин~А.\,В.} Оценка времени самовосстановления в 
цифровых\linebreak
\vspace*{-12pt}\\
\hspace*{23pt}системах после сбоев, вызываемых переходными помехами$\dotfill$&3&2\\
\textbf{Фуджикоши~Я.} см.~Кавагучи~Ю.&&\\
\hangindent=23pt\noindent\textbf{Цискаридзе~А.\,К.} Математическая модель и метод восстановления позы человека 
по\linebreak
\vspace*{-12pt}\\
\hspace*{23pt}стереопаре силуэтных изображений$\dotfill$&4&27\\
\hangindent=23pt\noindent\textbf{Чупраков~К.\,Г.} К вопросу о размещении коллективных средств отображения в 
ситуа-\linebreak
\vspace*{-12pt}\\
\hspace*{23pt}ционном зале с заданными параметрами$\dotfill$&4&89\\
\textbf{Чупраков~К.\,Г.} см.~Зацаринный~А.\,А.&&\\
\hangindent=23pt\noindent\textbf{Чупрунов~А.\,Н., Фазекаш~И.} Законы повторного логарифма для числа 
безошибочных\linebreak
\vspace*{-12pt}\\
\hspace*{23pt}блоков при помехоустойчивом кодировании$\dotfill$&3&42\\
\textbf{Шевцова~И.\,Г.} см.~Григорьева~М.\,Е.&&\\
\hangindent=23pt\noindent\textbf{Шестаков~О.\,В.} Аппроксимация распределения оценки риска пороговой 
обработки вейвлет-коэффициентов нормальным распределением при использовании 
выбо-\linebreak
\vspace*{-12pt}\\
\hspace*{23pt}рочной дисперсии$\dotfill$&4&73\\
\textbf{Шестаков~О.\,В.} см.~Маркин~А.\,В.&&\\
\textbf{Шоргин~С.\,Я.} см.~Зейфман~А.\,И.&&\\
\textbf{Шоргин~С.\,Я.} см.~Кудрявцев~А.\,А.&&\\
\end{tabular}
}

%\thispagestyle{myheadings}
\def\leftfootline{\small{\textbf{\thepage}
\hfill ИНФОРМАТИКА И ЕЁ ПРИМЕНЕНИЯ\ \ \ том~4\ \ \ выпуск~4\ \ \ 2010}
}%
 \def\rightfootline{\small{ИНФОРМАТИКА И ЕЁ ПРИМЕНЕНИЯ\ \ \ том~4\ \ \ выпуск~4\ \ \ 2010
 \hfill \textbf{\thepage}}}
 \label{end\stat}

%
%Том 10 Выпуск 1-4 Год 2016

\def\stat{cont-e}
{%\hrule\par
%\vskip 7pt % 7pt
\raggedleft\Large \bf%\baselineskip=3.2ex
2\,0\,1\,6\ \ A\,U\,T\,H\,O\,R\ \ I\,N\,D\,E\,X \vskip 17pt
 \hrule
 \par
\vskip 21pt plus 6pt minus 3pt }

\label{st\stat}

\def\tit{\ }

\def\aut{\ }
\def\auf{\ }

\def\leftkol{\ } %2016 AUTHOR INDEX} % ENGLISH ABSTRACTS}

\def\rightkol{\ } %2016 AUTHOR INDEX} %ENGLISH ABSTRACTS}

\titele{\tit}{\aut}{\auf}{\leftkol}{\rightkol}

\def\leftfootline{\small{\textbf{\thepage}
\hfill INFORMATIKA I EE PRIMENENIYA~--- INFORMATICS AND APPLICATIONS\ \ \ 2016\
\ \ volume~10\ \ \ issue\ 4}
}%
 \def\rightfootline{\small{INFORMATIKA I EE PRIMENENIYA~--- INFORMATICS AND APPLICATIONS\ \ \ 2016\ \ \ volume~10\ \ \ issue\ 4
\hfill \textbf{\thepage}}}

\vspace*{-12pt}
\vspace*{-18pt}

{\tabcolsep=2.8pt
\begin{tabular}{p{382pt}cc}
&\textbf{Issue} & \textbf{Page}\\[6pt]
\Avtors{Agalarov~M.\,Ya.} see~Agalarov~Ya.\,M.&&\\
\Avtors{Agalarov~Ya.\,M., Agalarov~M.\,Ya., and
Shorgin~V.\,S.} About the optimal threshold of queue\linebreak
\\[-12pt]
\hspace*{23pt}length in a~particular problem of profit maximization
in the $M/G/1$ queuing system&2&70--79\\
\Avtors{Alexeyevsky~D.\,A.} BioNLP ontology extraction from 
a~restricted language corpus with\linebreak
\\[-12pt]
\hspace*{23pt}context-free grammars&1&119--128\\
\Avtors{Andreev~S.\,D.} see~Gaidamaka~Yu.\,V.&&\\
\Avtors{Andreev~S.\,D.} see~Ometov~A.\,Ya.&&\\
\Avtors{Arkhipov~O.\,P., Arkhipov~P.\,O., and Sidorkin~I.\,I.} The
option to create a~local coordinate\linebreak
\\[-12pt]
\hspace*{23pt}system for synchronization of selected images&3&91--97\\
\Avtors{Arkhipov~P.\,O.} see~Arkhipov~O.\,P.&&\\
\Avtors{Belousov~V.\,V.} see~Shnurkov~P.\,V.&&\\
\Avtors{Belousov~V.\,V.} see~Shnurkov~P.\,V.&&\\
\Avtors{Bening~V.\,E.} Calculation of~the~asymptotic deficiency
of~some statistical procedures based\linebreak
\\[-12pt]
\hspace*{23pt}on~samples with~random sizes&4&34--45\\
\Avtors{Borisov~A.\,V., Bosov~A.\,V., and Miller~G.\,B.} Modeling and
monitoring of VoIP connection&2&\hphantom{1}2--13\\
\Avtors{Bosov~A.\,V.} see~Borisov~A.\,V.&&\\
\Avtors{Briukhov~D.\,O.} see~Stupnikov~S.\,A.&&\\
\Avtors{Callaos~N.\,K.\ and Seyful-Mulyukov~R.\,B.} Complexity and
its information content&1&129--139\\
\Avtors{Chertok~A.\,V., Kadaner~A.\,I., Khazeeva~G.\,T., and
Sokolov~I.\,A.} Regime switching detection\linebreak
\\[-12pt]
\hspace*{23pt}for~the~Levy driven
Ornstein--Uhlenbeck process using CUSUM methods&4&46--56\\
\Avtors{Chichagov~V.\,V.} Asymptotic expansions of mean absolute
error of uniformly minimum variance unbiased and maximum likelihood
estimators on the one-parameter exponential\linebreak
\\[-12pt]
\hspace*{23pt}family model of lattice distributions&3&66--76\\
\Avtors{Danishevsky~V.\,I.} see~Kolesnikov A.\,V.&&\\
\Avtors{Fazliev~A.\,Z.} see~Kalinichenko~L.\,A.&&\\
\Avtors{Fedoseev~A.\,A.} What is behind the concept of ``knowledge in
small packages''&3&105--110\\
\Avtors{Gaidamaka~Yu.\,V., Andreev~S.\,D., Sopin~E.\,S.,
Samouylov~K.\,E., and Shorgin~S.\,Ya.} Interference analysis
of~the~device-to-device communications model with~regard to~a~signal\linebreak
\\[-12pt]
\hspace*{23pt}propagation environment&4&\hphantom{1}2--10\\
\Avtors{Gasilov~A.\,V.} see~Yakovlev~O.\,A.&&\\
\Avtors{Goncharov~A.\,V.\ and Strijov~V.\,V.} Metric time series
classification using weighted dynamic\linebreak
\\[-12pt]
\hspace*{23pt}warping relative to centroids of classes&2&36--47\\
\Avtors{Gordov~E.\,P.} see~Kalinichenko~L.\,A.&&\\
\Avtors{Gorshenin~A.\,K.} Concept of online service for stochastic
modeling of real processes&1&72--81\\
\Avtors{Gorshenin~A.\,K.} see~Shnurkov~P.\,V.&&\\
\Avtors{Gorshenin~A.\,K.} see~Shnurkov~P.\,V.&&\\
\Avtors{Grusho~A.\,A., Grusho~N.\,A., Zabezhailo~M.\,I., and
Timonina~E.\,E.} Integration of statistical and\linebreak
\\[-12pt]
\hspace*{23pt}deterministic methods for
analysis of information security&3&2--8\\
\Avtors{Grusho~A.\,A., Zabezhailo~M.\,I., and Zatsarinny~A.\,A.} On
the advanced procedure to reduce\linebreak
\\[-12pt]
\hspace*{23pt}calculation of Galois closures&4&\hphantom{1}96--104\\
\Avtors{Grusho~N.\,A.} see~Grusho~A.\,A.&&\\
\Avtors{Havanskov~V.\,A.} see~Minin~V.\,A.&&\\
\Avtors{Inkova~O.\,Yu.} see~Zatsman~I.\,M.&&\\
\Avtors{Isachenko~R.\,V.\ and Strijov~V.\,V.} Metric learning in
multiclass time series classification\linebreak
\\[-12pt]
\hspace*{23pt}problem&2&48--57\\
\end{tabular}
}
\pagebreak

\def\leftfootline{\small{\textbf{\thepage}
\hfill INFORMATIKA I EE PRIMENENIYA~--- INFORMATICS AND APPLICATIONS\ \ \ 2016\
\ \ volume~10\ \ \ issue\ 4}
}%
 \def\rightfootline{\small{INFORMATIKA I EE PRIMENENIYA~---
INFORMATICS AND APPLICATIONS\ \ \ 2016\ \ \ volume~10\ \ \ issue\ 4
\hfill \textbf{\thepage}}}

\def\leftkol{2016 AUTHOR INDEX} % ENGLISH ABSTRACTS}

\def\rightkol{2016 AUTHOR INDEX} %ENGLISH ABSTRACTS}


{\tabcolsep=2.83pt
\begin{tabular}{p{382pt}cc}
&\textbf{Issue} & \textbf{Page}\\[6pt]
\Avtors{Kadaner~A.\,I.} see~Chertok~A.\,V.&&\\[.255pt]
\Avtors{Kalinichenko~L.\,A., Volnova~A.\,A., Gordov~E.\,P.,
Kiselyova~N.\,N., Kovaleva~D.\,A., Malkov~O.\,Yu., Okladnikov~I.\,G.,
Podkolodnyy~N.\,L., Pozanenko~A.\,S., Ponomareva~N.\,V.,
Stupnikov~S.\,A.,} \textbf{and Fazliev~A.\,Z.} Data access challenges for data
intensive\linebreak
\\[-12pt]
\hspace*{23pt}research in Russia&1& 2--22\\[.255pt]
\Avtors{Karasikov~M.\,E.\ and Strijov~V.\,V.} Feature-based
time-series classification&4&121--131\\[.255pt]
\Avtors{Khazeeva~G.\,T.} see~Chertok~A.\,V.&&\\[.255pt]
\Avtors{Khokhlov~Yu.\,S.} Multivariate fractional Levy motion and its
applications&2&\hphantom{1}98--106\\[.255pt]
\Avtors{Kirikov~I.\,A., Kolesnikov~A.\,V., Listopad~S.\,V., and
Rumovskaya~S.\,B.} Fine-grained hybrid\linebreak
\\[-12pt]
\hspace*{23pt}intelligent systems. Part 2:
Bidirectional hybridization&1&\hphantom{1}96--105\\[.255pt]
\Avtors{Kirikov~I.\,A., Kolesnikov~A.\,V., Listopad~S.\,V., and
Rumovskaya~S.\,B.} ``Virtual council''~---\linebreak
\\[-12pt]
\hspace*{23pt}source environment
supporting complex diagnostic decision making&3&81--90\\[.255pt]
\Avtors{Kiselyova~N.\,N.} see~Kalinichenko~L.\,A.&&\\[.255pt]
\Avtors{Kolesnikov A.\,V., Listopad~S.\,V., Rumovskaya~S.\,B., and
Danishevsky~V.\,I.} Informal axiomatic\linebreak
\\[-12pt]
\hspace*{23pt}theory of~the~role visual models&4&114--120\\[.255pt]
\Avtors{Kolesnikov~A.\,V.} see~Kirikov~I.\,A.&&\\[.255pt]
\Avtors{Kolesnikov~A.\,V.} see~Kirikov~I.\,A.&&\\[.255pt]
\Avtors{Kolin~K.\,K.} Humanitarian aspects of information
security&3&111--121\\[.255pt]
\Avtors{Konovalov~M.\,G.\ and Razumchik~R.\,V.} Dispatching
to~two parallel nonobservable queues using\linebreak
\\[-12pt]
\hspace*{23pt}only static
information&4&57--67\\[.255pt]
\Avtors{Korchagin~A.\,Yu.} see~Korolev~V.\,Yu.&&\\[.255pt]
\Avtors{Korchagin~A.\,Yu.} see~Korolev~V.\,Yu.&&\\[.255pt]
\Avtors{Korepanov~E.\,R.} see~Sinitsyn~I.\,N.&&\\[.255pt]
\Avtors{Korepanov~E.\,R.} see~Sinitsyn~I.\,N.&&\\[.255pt]
\Avtors{Korolev~V.\,Yu., Korchagin~A.\,Yu., and Zeifman~A.\,I.} The
Poisson theorem for Bernoulli trials\linebreak
\\[-12pt]
\hspace*{23pt}with~a~random probability
of~success and~a~discrete analog of~the~Weibull distribution&4&11--20\\[.255pt]
\Avtors{Korolev~V.\,Yu., Zeifman~A.\,I., and Korchagin~A.\,Yu.}
Asymmetric Linnik distributions as~limit\linebreak
\\[-12pt]
\hspace*{23pt}laws for~random sums
of~independent random variables with~finite variances&4&21--33\\[.255pt]
\Avtors{Koucheryavy~E.\,A.} see~Ometov~A.\,Ya.&&\\[.255pt]
\Avtors{Kovaleva~D.\,A.} see~Kalinichenko~L.\,A.&&\\[.255pt]
\Avtors{Kovalyov~S.\,P.} Metaprogramming to increase
manufacturability of large-scale software-\linebreak
\\[-12pt]
\hspace*{23pt}intensive systems&1&56--66\\[.255pt]
\Avtors{Krivenko~M.\,P.} Significance tests of feature selection for
classification&3&32--40\\[.255pt]
\Avtors{Kruzhkov~M.\,G.} see~Zalizniak~Anna~A.&&\\[.255pt]
\Avtors{Kruzhkov~M.\,G.} see~Zatsman~I.\,M.&&\\[.255pt]
\Avtors{Kudryavtsev~A.\,A.} Bayesian queueing and reliability models:
\textit{A~priori} distributions with\linebreak
\\[-12pt]
\hspace*{23pt}compact support&1&67--71\\[.255pt]
\Avtors{Kudryavtsev~A.\,A.} Characteristics dependent on the balance
coefficient in Bayesian models\linebreak
\\[-12pt]
\hspace*{23pt}with compact support of \textit{a priori}
distributions&3&77--80\\[.255pt]
\Avtors{Kudryavtsev~A.\,A.\ and Palionnaia~S.\,I.} Bayesian recurrent
model of reliability growth:\linebreak
\\[-12pt]
\hspace*{23pt}Parabolic distribution of parameters&2&80--83\\[.255pt]
\Avtors{Kudryavtsev~A.\,A.\ and Titova~A.\,I.} Bayesian queuing
and~reliability models: Degenerate-\linebreak
\\[-12pt]
\hspace*{23pt}Weibull case&4&68--71\\[.255pt]
\Avtors{Leontyev~N.\,D.\ and Ushakov~V.\,G.} Analysis of a queueing
system with autoregressive arrivals\linebreak
\\[-12pt]
\hspace*{23pt}and nonpreemptive priority&3&15--22\\[.255pt]
\Avtors{Listopad~S.\,V.} see~Kirikov~I.\,A.&&\\[.255pt]
\Avtors{Listopad~S.\,V.} see~Kirikov~I.\,A.&&\\[.255pt]
\Avtors{Listopad~S.\,V.} see~Kolesnikov A.\,V.&&\\[.255pt]
\Avtors{Malkov~O.\,Yu.} see~Kalinichenko~L.\,A.&&\\[.255pt]
\Avtors{Markov~A.\,S., Monakhov~M.\,M., and
Ulyanov~V.\,V.} Generalized Cornish--Fisher expansions\linebreak
\\[-12pt]
\hspace*{23pt}for distributions of statistics based on samples
of random size&2&84--91\\[.255pt]
\Avtors{Melnikov~A.\,K.\ and Ronzhin~A.\,F.} Generalized statistical
method of~text analysis based\linebreak
\\[-12pt]
\hspace*{23pt}on~calculation of~probability distributions
of~statistical values&4&89--95\\
\end{tabular}
}
\pagebreak

\def\leftfootline{\small{\textbf{\thepage}
\hfill INFORMATIKA I EE PRIMENENIYA~--- INFORMATICS AND APPLICATIONS\ \ \ 2016\
\ \ volume~10\ \ \ issue\ 4}
}%
 \def\rightfootline{\small{INFORMATIKA I EE PRIMENENIYA~---
INFORMATICS AND APPLICATIONS\ \ \ 2016\ \ \ volume~10\ \ \ issue\ 4
\hfill \textbf{\thepage}}}

\def\leftkol{2016 AUTHOR INDEX} % ENGLISH ABSTRACTS}

\def\rightkol{2016 AUTHOR INDEX} %ENGLISH ABSTRACTS}


{\tabcolsep=3pt
\begin{tabular}{p{381pt}cc}
&\textbf{Issue} & \textbf{Page}\\[6pt]
\Avtors{Meykhanadzhyan~L.\,A.} Stationary characteristics of the finite
capacity queueing system with\linebreak
\\[-12pt]
\hspace*{23pt}inverse service order and generalized
probabilistic priority&2&123--131\\[.23pt]
\Avtors{Miller~G.\,B.} see~Borisov~A.\,V.&&\\[.23pt]
\Avtors{Minin~V.\,A., Zatsman~I.\,M., Havanskov~V.\,A., and
Shubnikov~S.\,K.} Intensity of citation of scientific publications in
inventions on information and computer technologies patented\linebreak
\\[-12pt]
\hspace*{23pt}in Russia by domestic and foreign applicants&2&107--122\\[.23pt]
\Avtors{Monakhov~M.\,M.} see~Markov~A.\,S.&&\\[.23pt]
\Avtors{Naumov~V.\,A.\ and Samouylov~K.\,E.} On relationship
between queuing systems with resources\linebreak
\\[-12pt]
\hspace*{23pt}and Erlang networks&3&\hphantom{1}9--14\\[.23pt]
\Avtors{Okladnikov~I.\,G.} see~Kalinichenko~L.\,A.&&\\[.23pt]
\Avtors{Ometov~A.\,Ya., Andreev~S.\,D., Turlikov~A.\,M., and
Koucheryavy~E.\,A.} Performance analysis of\linebreak
\\[-12pt]
\hspace*{23pt}a wireless data
aggregation system with contention for contemporary sensor
networks&3&23--31\\[.23pt]
\Avtors{Palionnaia~S.\,I.} see~Kudryavtsev~A.\,A.&&\\[.23pt]
\Avtors{Podkolodnyy~N.\,L.} see~Kalinichenko~L.\,A.&&\\[.23pt]
\Avtors{Ponomareva~N.\,V.} see~Kalinichenko~L.\,A.&&\\[.23pt]
\Avtors{Popkova~N.\,A.} see~Zatsman~I.\,M.&&\\[.23pt]
\Avtors{Pozanenko~A.\,S.} see~Kalinichenko~L.\,A.&&\\[.23pt]
\Avtors{Razumchik~R.\,V.} see~Konovalov~M.\,G.&&\\[.23pt]
\Avtors{Ronzhin~A.\,F.} see~Melnikov~A.\,K.&&\\[.23pt]
\Avtors{Rumovskaya~S.\,B.} see~Kirikov~I.\,A.&&\\[.23pt]
\Avtors{Rumovskaya~S.\,B.} see~Kirikov~I.\,A.&&\\[.23pt]
\Avtors{Rumovskaya~S.\,B.} see~Kolesnikov A.\,V.&&\\[.23pt]
\Avtors{Samouylov~K.\,E.} see~Gaidamaka~Yu.\,V.&&\\[.23pt]
\Avtors{Samouylov~K.\,E.} see~Naumov~V.\,A.&&\\[.23pt]
\Avtors{Serebryanskii~S.\,M.} see~Tyrsin~A.\,N.&&\\[.23pt]
\Avtors{Seyful-Mulyukov~R.\,B.} see~Callaos~N.\,K.&&\\[.23pt]
\Avtors{Shestakov~O.\,V.} Statistical properties of the denoising method
based on the stabilized hard\linebreak
\\[-12pt]
\hspace*{23pt}thresholding&2&65--69\\[.23pt]
\Avtors{Shestakov~O.\,V.} The strong law of large numbers for the risk
estimate in the problem of\linebreak
\\[-12pt]
\hspace*{23pt}tomographic image reconstruction from
projections with a correlated noise&3&41--45\\[.23pt]
\Avtors{Shestakov~O.\,V.} see~Zakharova~T.\,V.&&\\[.23pt]
\Avtors{Shnurkov~P.\,V., Gorshenin~A.\,K., and Belousov~V.\,V.}
Analytical solution of~the~optimal control\linebreak
\\[-12pt]
\hspace*{23pt}task of~a~semi-Markov
process with~finite set of~states&4&72--88\\[.23pt]
\Avtors{Shnurkov~P.\,V., Zasypko~V.\,V., Belousov~V.\,V., and
Gorshenin~A.\,K.} Development of the algorithm of numerical solution
of the optimal investment control problem\linebreak
\\[-12pt]
\hspace*{23pt}in the closed dynamical model of three-sector economy&1&82--95\\[.23pt]
\Avtors{Shorgin~S.\,Ya.} see~Gaidamaka~Yu.\,V.&&\\[.23pt]
\Avtors{Shorgin~V.\,S.} see~Agalarov~Ya.\,M.&&\\[.23pt]
\Avtors{Shubnikov~S.\,K.} see~Minin~V.\,A.&&\\[.23pt]
\Avtors{Sidorkin~I.\,I.} see~Arkhipov~O.\,P.&&\\[.23pt]
\Avtors{Sinitsyn~I.\,N.} Analytical modeling of processes in stochastic
systems with complex fractional\linebreak
\\[-12pt]
\hspace*{23pt}order Bessel nonlinearities&3&55--65\\[.23pt]
\Avtors{Sinitsyn~I.\,N.} Orthogonal supoptimal filters for nonlinear
stochastic systems on manifolds&1&34--44\\[.23pt]
\Avtors{Sinitsyn~I.\,N.\ and Korepanov~E.\,R.} Normal Pugachev
conditionally-optimal filters and extra-\linebreak
\\[-12pt]
\hspace*{23pt}polators for state linear stochastic systems&2&14--23\\[.23pt]
\Avtors{Sinitsyn~I.\,N.\ and Sinitsyn~V.\,I.} Analytical modeling of
distributions in stochastic systems on\linebreak
\\[-12pt]
\hspace*{23pt}manifolds based on ellipsoidal approximation&1&45--55\\[.23pt]
\Avtors{Sinitsyn~I.\,N., Sinitsyn~V.\,I., and
Korepanov~E.\,R.} Ellipsoidal suboptimal filters for nonlinear\linebreak
\\[-12pt]
\hspace*{23pt}stochastic systems on manifolds&2&24--35\\[.23pt]
\Avtors{Sinitsyn~V.\,I.} see~Sinitsyn~I.\,N.&&\\[.23pt]
\Avtors{Sinitsyn~V.\,I.} see~Sinitsyn~I.\,N.&&\\[.23pt]
\Avtors{Skvortsov~N.\,A.} see~Stupnikov~S.\,A.&&\\[.23pt]
\Avtors{Sokolov~I.\,A.} see~Chertok~A.\,V.&&\\
\end{tabular}
}
\pagebreak

\def\leftfootline{\small{\textbf{\thepage}
\hfill INFORMATIKA I EE PRIMENENIYA~--- INFORMATICS AND APPLICATIONS\ \ \ 2016\
\ \ volume~10\ \ \ issue\ 4}
}%
 \def\rightfootline{\small{INFORMATIKA I EE PRIMENENIYA~---
INFORMATICS AND APPLICATIONS\ \ \ 2016\ \ \ volume~10\ \ \ issue\ 4
\hfill \textbf{\thepage}}}

\def\leftkol{2016 AUTHOR INDEX} % ENGLISH ABSTRACTS}

\def\rightkol{2016 AUTHOR INDEX} %ENGLISH ABSTRACTS}


{\tabcolsep=3pt
\begin{tabular}{p{382pt}cc}
&\textbf{Issue} & \textbf{Page}\\[6pt]
\Avtors{Sopin~E.\,S.} see~Gaidamaka~Yu.\,V.&&\\
\Avtors{Strijov~V.\,V.} see~Goncharov~A.\,V.&&\\
\Avtors{Strijov~V.\,V.} see~Isachenko~R.\,V.&&\\
\Avtors{Strijov~V.\,V.} see~Karasikov~M.\,E.&&\\
\Avtors{Stupnikov~S.\,A., Briukhov~D.\,O., and Skvortsov~N.\,A.}
Co-lending systemic risk analysis over\linebreak
\\[-12pt]
\hspace*{23pt}heterogeneous data collections&1&23--33\\
\Avtors{Stupnikov~S.\,A.} see~Kalinichenko~L.\,A.&&\\
\Avtors{Suchkov~A.\,P.} see~Zatsarinny~A.\,A.&&\\
\Avtors{Timonina~E.\,E.} see~Grusho~A.\,A.&&\\
\Avtors{Titova~A.\,I.} see~Kudryavtsev~A.\,A.&&\\
\Avtors{Turlikov~A.\,M.} see~Ometov~A.\,Ya.&&\\
\Avtors{Tyrsin~A.\,N.\ and Serebryanskii~S.\,M.} Recognition of
dependences on the basis of inverse\linebreak
\\[-12pt]
\hspace*{23pt}mapping&2&58--64\\
\Avtors{Ulyanov~V.\,V.} see~Markov~A.\,S.&&\\
\Avtors{Ushakov~V.\,G.} Queueing system with working vacations and
hyperexponential input stream&2&92--97\\
\Avtors{Ushakov~V.\,G.} see~Leontyev~N.\,D.&&\\
\Avtors{Volnova~A.\,A.} see~Kalinichenko~L.\,A.&&\\
\Avtors{Yakovlev~O.\,A.\ and Gasilov~A.\,V.} Speeded-up stereo
matching using geodesic support weights&3&\hphantom{1}98--104\\
\Avtors{Zabezhailo~M.\,I.} see~Grusho~A.\,A.&&\\
\Avtors{Zabezhailo~M.\,I.} see~Grusho~A.\,A.&&\\
\Avtors{Zakharova~T.\,V.\ and Shestakov~O.\,V.} Precision analysis of
wavelet processing of aerodynamic\linebreak
\\[-12pt]
\hspace*{23pt}flow patterns&3&46--54\\
\Avtors{Zalizniak~Anna~A.\ and Kruzhkov~M.\,G.} Database
of~Russian impersonal verbal constructions&4&132--141\\
\Avtors{Zasypko~V.\,V.} see~Shnurkov~P.\,V.&&\\
\Avtors{Zatsarinny~A.\,A.\ and Suchkov~A.\,P.} Systems engineering
approaches to~the~establishment of\linebreak
\\[-12pt]
\hspace*{23pt}a~system for~decision support based
on~situational analysis&4&105--113\\
\Avtors{Zatsarinny~A.\,A.} see~Grusho~A.\,A.&&\\
\Avtors{Zatsman~I.\,M., Inkova~O.\,Yu., Kruzhkov~M.\,G., and
Popkova~N.\,A.} Representation of cross-\linebreak
\\[-12pt]
\hspace*{23pt}lingual knowledge about
connectors in supracorpora databases&1&106--118\\
\Avtors{Zatsman~I.\,M.} see~Minin~V.\,A.&&\\
\Avtors{Zeifman~A.\,I.} see~Korolev~V.\,Yu.&&\\
\Avtors{Zeifman~A.\,I.} see~Korolev~V.\,Yu.&&\\
\end{tabular}
}

%\thispagestyle{myheadings}
\def\leftfootline{\small{\textbf{\thepage}
\hfill INFORMATIKA I EE PRIMENENIYA~--- INFORMATICS AND APPLICATIONS\ \ \ 2016\
\ \ volume~10\ \ \ issue\ 4}
}%
 \def\rightfootline{\small{INFORMATIKA I EE PRIMENENIYA~---
INFORMATICS AND APPLICATIONS\ \ \ 2016\ \ \ volume~10\ \ \ issue\ 4
\hfill \textbf{\thepage}}}

 \label{end\stat}

\newpage


%\vspace*{-60pt} {\small
{\baselineskip=9.1pt
\section*{Правила подготовки рукописей статей для публикации в журнале
<<Информатика и её применения>>}

\thispagestyle{empty}

 Журнал <<Информатика и её применения>> публикует
теоретические, обзорные и дискуссионные статьи, посвященные научным
исследованиям и разработкам в области информатики и ее приложений. Журнал
издается на русском языке. По специальному решению редколлегии отдельные статьи,
в виде исключения, могут печататься на английском языке.
Тематика журнала охватывает следующие направления:
\begin{itemize}
\item теоретические основы информатики; %\\[-13.5pt]
\item математические методы исследования сложных систем и процессов; %\\[-13.5pt]
\item информационные системы и сети; %\\[-13.5pt]
\item информационные технологии; %\\[-13.5pt]
\item архитектура и программное
обеспечение вычислительных комплексов и сетей.
\end{itemize}
\begin{enumerate}
\item В журнале печатаются результаты, ранее не
опубликованные и не предназначенные к одновременной публикации в других
изданиях. Публикация не должна нарушать закон об авторских правах. Направляя
свою рукопись в редакцию, авторы автоматически передают учредителям и
редколлегии неисключительные права на издание данной статьи на русском языке и
на ее распространение в России и за рубежом. При этом за авторами сохраняются
все права как собственников данной рукописи. В связи с этим авторами должно
быть представлено в редакцию письмо в следующей форме:
Соглашение о передаче права на публикацию:

\textit{<<Мы, нижеподписавшиеся, авторы рукописи <<$\qquad\qquad$>>, передаем
учредителям и редколлегии журнала <<Информатика и её применения>>
неисключительное право опубликовать данную рукопись статьи на русском языке как
в печатной, так и в электронной версиях журнала. Мы подтверждаем, что данная
публикация не нарушает авторского права других лиц или организаций. Подписи
авторов: (ф.\,и.\,о., дата, адрес)>>.}

Указанное соглашение может быть представлено 
как в бумажном виде, так и в виде отсканированной копии (с подписями авторов).


Редколлегия вправе запросить у авторов экспертное заключение о возможности
опубликования представленной статьи в открытой печати. %\\[-13.5pt]
\item Статья
подписывается всеми авторами. На отдельном листе представляются данные автора
(или всех авторов): фамилия, полные имя и отчество, телефон, факс, e-mail,
почтовый адрес. Если работа выполнена несколькими авторами, указывается фамилия
одного из них, ответственного за переписку с редакцией. %\\[-13.5pt]
\item Редакция журнала
осуществляет самостоятельную экспертизу присланных статей. Возвращение рукописи
на доработку не означает, что статья уже принята к печати. Доработанный вариант
с ответом на замечания рецензента необходимо прислать в редакцию. %\\[-13.5pt]
\item Решение
редакционной коллегии о принятии статьи к печати или ее отклонении сообщается
авторам. Редколлегия не обязуется направлять рецензию авторам отклоненной
статьи. %\\[-13.5pt]
\item Корректура статей высылается авторам для просмотра. Редакция
просит авторов присылать свои замечания в кратчайшие сроки. %\\[-13.5pt]
\item При
подготовке рукописи в MS Word рекомендуется использовать следующие настройки.
Параметры страницы: формат~--- А4; ориентация~--- книжная; поля (см): внутри~---
2,5, снаружи~--- 1,5, сверху~--- 2, снизу~--- 2, от края до нижнего
колонтитула~--- 1,3. Основной текст: стиль~--- <<Обычный>>: шрифт Times New
Roman, размер 14~пунктов, абзацный отступ~--- 0,5~см, 1,5 интервала,
выравнивание~--- по ширине. Рекомендуемый объем рукописи~--- не свыше
25~страниц указанного формата. Ознакомиться с шаблонами, содержащими примеры
оформления, можно по адресу в Интернете:
\textsf{http://www.ipiran.ru/journal/template.doc}.
\item К рукописи, предоставляемой в 2-х
экземплярах, обязательно прилагается электронная версия статьи (как правило, в
форматах MS WORD (.doc) или \LaTeX\ (.tex), а также~--- дополнительно~--- в
формате .pdf) на дискете, лазерном диске или по электронной почте. Сокращения
слов, кроме стандартных, не применяются. Все страницы рукописи должны быть
пронумерованы. %\\[-13.5pt]
\item Статья должна содержать следующую информацию на русском и
английском языках: название, Ф.И.О. авторов, места работы авторов и их
электронные адреса, подробные сведения об авторах, оформленные в соответствии с форматом, 
определяемым файлами {\sf http://www.ipiran.ru/journal/issues/2011\_05\_01/authors.asp} и 
{\sf http://www.ipiran.ru/journal/issues/2011\_01\_eng/authors.asp},
аннотация (не более 100~слов), ключевые слова. Ссылки на
литературу в тексте статьи нумеруются (в квадратных скобках) и располагаются в
порядке их первого упоминания. В~списке литературы не должно быть позиций, на которые нет ссылки в тексте статьи.
Все фамилии авторов, заглавия статей, названия
книг, конференций и~т.\,п.\ даются на языке оригинала, если этот язык
использует кириллический или латинский алфавит. %\\[-13.5pt]
\item Присланные в редакцию материалы авторам не возвращаются.
\item При отправке файлов по электронной
почте просим придерживаться следующих правил:
\begin{itemize}
\item указывать в поле subject (тема) название журнала и фамилию автора; %\\[-13.5pt]
\item использовать attach (присоединение); %\\[-13.5pt]
\item в случае больших объемов информации возможно
использование общеизвестных архиваторов (ZIP, RAR); %\\[-13.5pt]
\item в состав электронной версии статьи должны входить: файл, содержащий текст статьи, и файл(ы),
содержащий(е) иллюстрации. %\\[-13.5pt]
\end{itemize}
\item Журнал <<Информатика и её применения>> является некоммерческим изданием. 
Плата за публикацию с авторов не взимается, гонорар авторам не выплачивается.
\end{enumerate}
\thispagestyle{empty}
\textbf{Адрес редакции:} Москва 119333,
ул.~Вавилова, д.~44, корп.~2, ИПИ РАН\\
\hphantom{\textbf{Адрес редакции:} }Тел.: +7 (499) 135-86-92\ \
Факс:  +7 (495) 930-45-05\ \  E-mail:   rust@ipiran.ru }
}

\end{document}


%\tableofcontents

%\end{document}





%\def\stat{cont}
{%\hrule\par
%\vskip 7pt % 7pt
\raggedleft\Large \bf%\baselineskip=3.2ex
А\,В\,Т\,О\,Р\,С\,К\,И\,Й\ \ У\,К\,А\,З\,А\,Т\,Е\,Л\,Ь\ \ З\,А\ \ 2\,0\,0\,7 г. \vskip 17pt
    \hrule
    \par
\vskip 21pt plus 6pt minus 3pt }

\label{st\stat}

\def\tit{\ }

\def\aut{\ }
\def\auf{\ }

\def\leftkol{\ } % ENGLISH ABSTRACTS}

\def\rightkol{\ } %ENGLISH ABSTRACTS}

\titele{\tit}{\aut}{\auf}{\leftkol}{\rightkol}


\contentsline {chapter}{\ }{Выпуск \quad Стр.} 
\contentsline {section}{\textbf{Батракова Д.\,А., Королев В.\,Ю., Шоргин С.\,Я.}\ \ Новый метод вероятностно-ста\-ти\-сти\-че\-ско\-го анализа информационных потоков в\nobreakspace {}телекоммуникационных сетях}{\qquad 1 \qquad 40} 
\contentsline {section}{\textbf{Борисов А.\,В.}\ \ Байесовское оценивание в системах наблюдения с\nobreakspace {}марковскими скачкообразными процессами: игровой подход}{\qquad 2 \qquad 65}
\contentsline {section}{\textbf{Босов А.\,В., Иванов А.\,В.}\ \ Программная инфраструктура информационного Web-пор\-тала}{\qquad 2 \qquad 50}
\contentsline {section}{\textbf{Захаров В.\,Н., Калиниченко Л.\,А., Соколов И.\,А., Ступников С.\,А.}\ \ Конструирование канонических информационных моделей для интегрированных информационных систем}{\qquad 2 \qquad 15}
\contentsline {section}{\textbf{Захаров В.\,Н., Козмидиади В.\,А.}\ \ Средства обеспечения отказоустойчивости при\-ло\-жений}{\qquad 1 \qquad 14} 
\contentsline {section}{\textbf{Иванов А.\,В.}\ \ см. Босов А.\,В.\hfill\hfill\hfill\hfill\hfill\hfill\hfill\hfill\hfill\hfill\hfill\hfill\hfill\hfill\hfill\hfill\hfill\hfill\hfill\hfill\hfill\hfill\hfill\hfill\hfill\hfill\hfill\hfill\hfill\hfill\hfill\hfill\hfill\hfill\hfill}{\ }
\contentsline {section}{\textbf{Ильин В.\,Д., Соколов И.\,А.}\ \ Символьная модель системы знаний информатики в\nobreakspace {}че\-ло\-ве\-ко-автоматной среде}{\qquad 1 \qquad 66} 
\contentsline {section}{\textbf{Калиниченко Л.\,А.}\ \ см. Захаров В.\,Н.\hfill\hfill\hfill\hfill\hfill\hfill\hfill\hfill\hfill\hfill\hfill\hfill\hfill\hfill\hfill\hfill\hfill\hfill\hfill\hfill\hfill\hfill\hfill\hfill\hfill\hfill\hfill\hfill\hfill\hfill\hfill\hfill\hfill\hfill\hfill}{\ }
\contentsline {section}{\textbf{Козеренко Е.\,Б.}\ \ Лингвистическое моделирование для систем машинного перевода и обработки знаний}{\qquad 1 \qquad 54} 
\contentsline {section}{\textbf{Козмидиади В.\,А.}\ \ см. Захаров В.\,Н.\hfill\hfill\hfill\hfill\hfill\hfill\hfill\hfill\hfill\hfill\hfill\hfill\hfill\hfill\hfill\hfill\hfill\hfill\hfill\hfill\hfill\hfill\hfill\hfill\hfill\hfill\hfill\hfill\hfill\hfill\hfill\hfill\hfill\hfill\hfill }{\ } 
\contentsline {section}{\textbf{Королев В.\,Ю.}\ \ см. Батракова Д.\,А.\hfill\hfill\hfill\hfill\hfill\hfill\hfill\hfill\hfill\hfill\hfill\hfill\hfill\hfill\hfill\hfill\hfill\hfill\hfill\hfill\hfill\hfill\hfill\hfill\hfill\hfill\hfill\hfill\hfill\hfill\hfill\hfill\hfill\hfill\hfill}{\ } 
\contentsline {section}{\textbf{Кудрявцев А.\,А., Шоргин С.\,Я.}\ \ Байесовский подход к\nobreakspace {}анализу систем массового обслуживания и\nobreakspace {}показателей надежности}{\qquad 2 \qquad 76}
\contentsline {section}{\textbf{Печинкин А.\,В., Соколов И.\,А., Чаплыгин В.\,В.}\ \ Многолинейная система массового обслуживания с конечным накопителем и ненадежными приборами}{\qquad 1 \qquad 27} 
\contentsline {section}{\textbf{Печинкин А.\,В., Соколов И.\,А., Чаплыгин В.\,В.}\ \ Стационарные характеристики многолинейной\nobreakspace {}системы массового обслуживания с\nobreakspace {}одновременными отказами приборов}{\qquad 2 \qquad 39}
\contentsline {section}{\textbf{Синицын И.\,Н.}\ \ Корреляционные методы построения аналитических информационных моделей флуктуаций полюса Земли по априорным данным}{\qquad 2 \qquad \hphantom{9}2}
\contentsline {section}{\textbf{Синицын И.\,Н.}\ \ Развитие теории фильтров Пугачева для оперативной обработки информации в стохастических системах}{{\qquad 1 \qquad \hphantom{9}3}} 
\contentsline {section}{\textbf{Соколов И.\,А.}\ \ см. Захаров В.\,Н.\hfill\hfill\hfill\hfill\hfill\hfill\hfill\hfill\hfill\hfill\hfill\hfill\hfill\hfill\hfill\hfill\hfill\hfill\hfill\hfill\hfill\hfill\hfill\hfill\hfill\hfill\hfill\hfill\hfill\hfill\hfill\hfill\hfill\hfill\hfill}{\ }
\contentsline {section}{\textbf{Соколов И.\,А.}\ \ см. Ильин В.\,Д.\hfill\hfill\hfill\hfill\hfill\hfill\hfill\hfill\hfill\hfill\hfill\hfill\hfill\hfill\hfill\hfill\hfill\hfill\hfill\hfill\hfill\hfill\hfill\hfill\hfill\hfill\hfill\hfill\hfill\hfill\hfill\hfill\hfill\hfill\hfill}{\ } 
\contentsline {section}{\textbf{Соколов И.\,А.}\ \ см. Печинкин А.\,В.\hfill\hfill\hfill\hfill\hfill\hfill\hfill\hfill\hfill\hfill\hfill\hfill\hfill\hfill\hfill\hfill\hfill\hfill\hfill\hfill\hfill\hfill\hfill\hfill\hfill\hfill\hfill\hfill\hfill\hfill\hfill\hfill\hfill\hfill\hfill}{\ } 
\contentsline {section}{\textbf{Соколов И.\,А.}\ \ см. Печинкин А.\,В.\hfill\hfill\hfill\hfill\hfill\hfill\hfill\hfill\hfill\hfill\hfill\hfill\hfill\hfill\hfill\hfill\hfill\hfill\hfill\hfill\hfill\hfill\hfill\hfill\hfill\hfill\hfill\hfill\hfill\hfill\hfill\hfill\hfill\hfill\hfill}{\ }
\contentsline {section}{\textbf{Ступников С.\,А.}\ \ см. Захаров В.\,Н.\hfill\hfill\hfill\hfill\hfill\hfill\hfill\hfill\hfill\hfill\hfill\hfill\hfill\hfill\hfill\hfill\hfill\hfill\hfill\hfill\hfill\hfill\hfill\hfill\hfill\hfill\hfill\hfill\hfill\hfill\hfill\hfill\hfill\hfill\hfill}{\ }
\contentsline {section}{\textbf{Чаплыгин В.\,В.}\ \ см. Печинкин А.\,В.\hfill\hfill\hfill\hfill\hfill\hfill\hfill\hfill\hfill\hfill\hfill\hfill\hfill\hfill\hfill\hfill\hfill\hfill\hfill\hfill\hfill\hfill\hfill\hfill\hfill\hfill\hfill\hfill\hfill\hfill\hfill\hfill\hfill\hfill\hfill}{\ } 
\contentsline {section}{\textbf{Чаплыгин В.\,В.}\ \ см. Печинкин А.\,В.\hfill\hfill\hfill\hfill\hfill\hfill\hfill\hfill\hfill\hfill\hfill\hfill\hfill\hfill\hfill\hfill\hfill\hfill\hfill\hfill\hfill\hfill\hfill\hfill\hfill\hfill\hfill\hfill\hfill\hfill\hfill\hfill\hfill\hfill\hfill}{\ }
\contentsline {section}{\textbf{Шоргин С.\,Я.}\ \ см. Батракова Д.\,А.\hfill\hfill\hfill\hfill\hfill\hfill\hfill\hfill\hfill\hfill\hfill\hfill\hfill\hfill\hfill\hfill\hfill\hfill\hfill\hfill\hfill\hfill\hfill\hfill\hfill\hfill\hfill\hfill\hfill\hfill\hfill\hfill\hfill\hfill\hfill}{\ } 
\contentsline {section}{\textbf{Шоргин С.\,Я.}\ \ см. Кудрявцев А.\,А.\hfill\hfill\hfill\hfill\hfill\hfill\hfill\hfill\hfill\hfill\hfill\hfill\hfill\hfill\hfill\hfill\hfill\hfill\hfill\hfill\hfill\hfill\hfill\hfill\hfill\hfill\hfill\hfill\hfill\hfill\hfill\hfill\hfill\hfill\hfill}{\ }
%\thispagestyle{myheadings}
\def\leftfootline{\small{\textbf{\thepage}
\hfill ИНФОРМАТИКА И ЕЁ ПРИМЕНЕНИЯ\ \ \ том~1\ \ \ выпуск~2\ \ \ 2007}
}%
 \def\rightfootline{\small{ИНФОРМАТИКА И ЕЁ ПРИМЕНЕНИЯ\ \ \ том~1\ \ \ выпуск~2\ \ \ 2007
 \hfill \textbf{\thepage}}}
 \label{end\stat}

%\def\stat{cont-e}
{%\hrule\par
%\vskip 7pt % 7pt
\raggedleft\Large \bf%\baselineskip=3.2ex
2\,0\,0\,7\ \ A\,U\,T\,H\,O\,R\ \ I\,N\,D\,E\,X \vskip 17pt
    \hrule
    \par
\vskip 21pt plus 6pt minus 3pt }

\label{st\stat}

\def\tit{\ }

\def\aut{\ }
\def\auf{\ }

\def\leftkol{\ } % ENGLISH ABSTRACTS}

\def\rightkol{\ } %ENGLISH ABSTRACTS}

\titele{\tit}{\aut}{\auf}{\leftkol}{\rightkol}


\contentsline {chapter}{\ }{Issue \quad Page} 
\contentsline {subsection}{\textbf{Batrakova D.\,A., Korolev V.\,Yu., Shorgin S.\,Ya.}\ \ A New Method for the Probabilistic and Statistical Analysis of Information Flows in Telecommunication Networks}{\qquad 1 \qquad 40} 
\contentsline {subsection}{\textbf{Borisov A.\,V.}\ \ Bayesian Estimation in\nobreakspace {}Observation Systems with\nobreakspace {}Markov Jump Processes: Game-Theoretic Approach}{\qquad 2 \qquad 65} 
\contentsline {subsection}{\textbf{Bosov A.\,V., Ivanov A.\,V.}\ \ Linguistic Simulation for Machine Translation and Knowledge Management Systems}{\qquad 2 \qquad 50} 
\contentsline {subsection}{\textbf{Chaplygin V.\,V.} see Pechinkin A.\,V.\hfill\hfill\hfill\hfill\hfill\hfill\hfill\hfill\hfill\hfill\hfill\hfill\hfill\hfill\hfill\hfill\hfill\hfill\hfill\hfill\hfill\hfill\hfill\hfill\hfill\hfill\hfill\hfill\hfill\hfill\hfill\hfill\hfill\hfill\hfill}{\ }
\contentsline {subsection}{\textbf{Chaplygin V.\,V.} see Pechinkin A.\,V.\hfill\hfill\hfill\hfill\hfill\hfill\hfill\hfill\hfill\hfill\hfill\hfill\hfill\hfill\hfill\hfill\hfill\hfill\hfill\hfill\hfill\hfill\hfill\hfill\hfill\hfill\hfill\hfill\hfill\hfill\hfill\hfill\hfill\hfill\hfill}{\ }
\contentsline {subsection}{\textbf{Ilyin V.\,D., Sokolov I.\,A.}\ \ The Symbol Model of Informatics Knowledge System in Human-Automaton Environment}{\qquad 1 \qquad 66} 
\contentsline {subsection}{\textbf{Ivanov A.\,V.} see Bosov A.\,V.\hfill\hfill\hfill\hfill\hfill\hfill\hfill\hfill\hfill\hfill\hfill\hfill\hfill\hfill\hfill\hfill\hfill\hfill\hfill\hfill\hfill\hfill\hfill\hfill\hfill\hfill\hfill\hfill\hfill\hfill\hfill\hfill\hfill\hfill\hfill}{\ }
\contentsline {subsection}{\textbf{Kalinichenko L.\,A.} see Zakharov V.\,N.\hfill\hfill\hfill\hfill\hfill\hfill\hfill\hfill\hfill\hfill\hfill\hfill\hfill\hfill\hfill\hfill\hfill\hfill\hfill\hfill\hfill\hfill\hfill\hfill\hfill\hfill\hfill\hfill\hfill\hfill\hfill\hfill\hfill\hfill\hfill}{\ }
\contentsline {subsection}{\textbf{Korolev V.\,Yu.} see Batrakova D.\,A.\hfill\hfill\hfill\hfill\hfill\hfill\hfill\hfill\hfill\hfill\hfill\hfill\hfill\hfill\hfill\hfill\hfill\hfill\hfill\hfill\hfill\hfill\hfill\hfill\hfill\hfill\hfill\hfill\hfill\hfill\hfill\hfill\hfill\hfill\hfill}{\ }
\contentsline {subsection}{\textbf{Kozerenko E.\,B.}\ \ Linguistic Simulation for Machine Translation and Knowledge Management Systems}{\qquad 1 \qquad 54} 
\contentsline {subsection}{\textbf{Kozmidiady V.\,A.} see Zakharov V.\,N.\hfill\hfill\hfill\hfill\hfill\hfill\hfill\hfill\hfill\hfill\hfill\hfill\hfill\hfill\hfill\hfill\hfill\hfill\hfill\hfill\hfill\hfill\hfill\hfill\hfill\hfill\hfill\hfill\hfill\hfill\hfill\hfill\hfill\hfill\hfill}{\ }
\contentsline {subsection}{\textbf{Kudryavtsev A.\,A., Shorgin S.\,Ya.}\ \ Bayesian Approach to Queueing Systems and Reliability Characteristics}{\qquad 2 \qquad 76} 
\contentsline {subsection}{\textbf{Pechinkin A.\,V., Sokolov I.\,A., Chaplygin V.\,V.}\ \ Multichannel Queuing System with Finite Buffer and Unreliable Servers}{\qquad 1 \qquad 27} 
\contentsline {subsection}{\textbf{Pechinkin A.\,V., Sokolov I.\,A., Chaplygin V.\,V.}\ \ Stationary Characteristics of a Multichannel Queueing System with\nobreakspace {}Simultaneous Refusals of Servers}{\qquad 2 \qquad 39} 
\contentsline {subsection}{\textbf{Shorgin S.\,Ya.} see Batrakova D.\,A.\hfill\hfill\hfill\hfill\hfill\hfill\hfill\hfill\hfill\hfill\hfill\hfill\hfill\hfill\hfill\hfill\hfill\hfill\hfill\hfill\hfill\hfill\hfill\hfill\hfill\hfill\hfill\hfill\hfill\hfill\hfill\hfill\hfill\hfill\hfill}{\ }
\contentsline {subsection}{\textbf{Shorgin S.\,Ya.} see Kudryavtsev A.\,A.\hfill\hfill\hfill\hfill\hfill\hfill\hfill\hfill\hfill\hfill\hfill\hfill\hfill\hfill\hfill\hfill\hfill\hfill\hfill\hfill\hfill\hfill\hfill\hfill\hfill\hfill\hfill\hfill\hfill\hfill\hfill\hfill\hfill\hfill\hfill}{\ }
\contentsline {subsection}{\textbf{Sinitsyn I.\,N.}\ \ Correlational Methods for Analytical Informational Models of the Earth Pole Fluctuations Design Based on a priori Data}{\qquad 2 \qquad \hphantom{9}2}
\contentsline {subsection}{\textbf{Sinitsyn I.\,N.}\ \ Development of Pugachev Filtering for Stochastic Systems}{\qquad 1 \qquad \hphantom{9}3}
\contentsline {subsection}{\textbf{Sokolov I.\,A.} see Ilyin V.\,D.\hfill\hfill\hfill\hfill\hfill\hfill\hfill\hfill\hfill\hfill\hfill\hfill\hfill\hfill\hfill\hfill\hfill\hfill\hfill\hfill\hfill\hfill\hfill\hfill\hfill\hfill\hfill\hfill\hfill\hfill\hfill\hfill\hfill\hfill\hfill}{\ }
\contentsline {subsection}{\textbf{Sokolov I.\,A.} see Pechinkin A.\,V.\hfill\hfill\hfill\hfill\hfill\hfill\hfill\hfill\hfill\hfill\hfill\hfill\hfill\hfill\hfill\hfill\hfill\hfill\hfill\hfill\hfill\hfill\hfill\hfill\hfill\hfill\hfill\hfill\hfill\hfill\hfill\hfill\hfill\hfill\hfill}{\ }
\contentsline {subsection}{\textbf{Sokolov I.\,A.} see Pechinkin A.\,V.\hfill\hfill\hfill\hfill\hfill\hfill\hfill\hfill\hfill\hfill\hfill\hfill\hfill\hfill\hfill\hfill\hfill\hfill\hfill\hfill\hfill\hfill\hfill\hfill\hfill\hfill\hfill\hfill\hfill\hfill\hfill\hfill\hfill\hfill\hfill}{\ }
\contentsline {subsection}{\textbf{Sokolov I.\,A.} see Zakharov V.\,N.\hfill\hfill\hfill\hfill\hfill\hfill\hfill\hfill\hfill\hfill\hfill\hfill\hfill\hfill\hfill\hfill\hfill\hfill\hfill\hfill\hfill\hfill\hfill\hfill\hfill\hfill\hfill\hfill\hfill\hfill\hfill\hfill\hfill\hfill\hfill}{\ }
\contentsline {subsection}{\textbf{Stupnikov S.\,A.} see Zakharov V.\,N.\hfill\hfill\hfill\hfill\hfill\hfill\hfill\hfill\hfill\hfill\hfill\hfill\hfill\hfill\hfill\hfill\hfill\hfill\hfill\hfill\hfill\hfill\hfill\hfill\hfill\hfill\hfill\hfill\hfill\hfill\hfill\hfill\hfill\hfill\hfill}{\ }
\contentsline {subsection}{\textbf{Zakharov V.\,N., Kalinichenko L.\,A., Sokolov I.\,A., Stupnikov S.\,A.}\ \ Development of Canonical Information Models for Integrated Information Systems}{\qquad 2 \qquad 15} 
\contentsline {subsection}{\textbf{Zakharov V.\,N., Kozmidiady V.\,A.}\ \ Means Providing Applications Fault Tolerance}{\qquad 1 \qquad 14} 
\def\leftfootline{\small{\textbf{\thepage}
\hfill ИНФОРМАТИКА И ЕЁ ПРИМЕНЕНИЯ\ \ \ том~1\ \ \ выпуск~2\ \ \ 2007}
}%
 \def\rightfootline{\small{ИНФОРМАТИКА И ЕЁ ПРИМЕНЕНИЯ\ \ \ том~1\ \ \ выпуск~2\ \ \ 2007
 \hfill \textbf{\thepage}}}
 \label{end\stat}


%\tableofcontents


\end{document}