\newcommand{\Cov}{\mathrm{Cov}}
\newcommand{\Nor}{\ensuremath{\mathcal{N}}}
\newcommand{\Pu}{\mathbb{P}}


\def\stat{luk-mor}

\def\tit{О СХОДИМОСТИ В ПРОСТРАНСТВЕ $L_p$ %{\boldmath{$L_p$}}  
МАКСИМУМА ПРОЦЕССА НАГРУЗКИ ДЛЯ ОДНОГО КЛАССА
ГАУССОВСКИХ~СИСТЕМ ОБСЛУЖИВАНИЯ$^*$}

\def\titkol{О сходимости в пространстве $L_p$  максимума процесса нагрузки для одного класса
гауссовских систем обслуживания}

\def\autkol{О.\,В.~Лукашенко, Е.\,В.~Морозов}

\def\aut{О.\,В.~Лукашенко$^1$, Е.\,В.~Морозов$^2$}

\titel{\tit}{\aut}{\autkol}{\titkol}

{\renewcommand{\thefootnote}{\fnsymbol{footnote}}\footnotetext[1]
{Работа выполняется при финансовой поддержке Программы стратегического 
развития ПетрГУ   в рамках реализации комплекса мероприятий  по развитию 
научно-исследовательской деятельности.}}

\renewcommand{\thefootnote}{\arabic{footnote}}
\footnotetext[1]{Институт прикладных математических исследований
Карельского научного центра Российской академии наук; Петрозаводский
государственный университет, lukashenko-oleg@mail.ru}
\footnotetext[2]{Институт прикладных математических исследований
Карельского научного центра Российской академии наук; Петрозаводский
государственный университет, emorozov@karelia.ru}

\Abst{Рассматривается класс систем обслуживания, на вход которых
поступает поток, содержащий линейную детерминированную компоненту и
случайную компоненту, описываемую центрированным гауссовским
процессом. Дисперсия входного процесса  является правильно
меняющейся функцией  с показателем  $V\hm\in (0,\,2)$. Найдены условия,
при которых  максимум  стационарного процесса нагрузки
(незавершенной работы) на интервале $[0,\,t]$ сходится при $t\hm\to
\infty$ (и при соответствующей нормировке) в пространстве $L_p$   к
явно выписанной константе. Также найдена асимптотика максимума
процесса нагрузки в нестационарном режиме. Получена асимптотика
минимального времени достижения процессом нагрузки растущего
значения~$b$.}


\KW{гауссовская система обслуживания; максимум
процесса нагрузки; дробное броуновское движение; асимптотический
анализ;  правильное изменение}

\vskip 14pt plus 9pt minus 6pt

      \thispagestyle{headings}

      \begin{multicols}{2}

            \label{st\stat}

\section{Введение}

В работе~\cite{Lukashenko} был осуществлен асимптотический анализ
процесса нагрузки в жидкостной  системе  с постоянной скоростью
обслуживания~$C$  и входным  гауссовским процессом, дисперсия
которого правильно меняется на бесконечности с показателем $V\hm\in
(0,\,2)$. Рассмотренный  класс входных процессов включает, в
частности, сумму независимых дробных броуновских движений (ДБД) с
соответствующими значениями индекса Херста. В~\cite{Lukashenko}
показано, что при соответствующей нормировке максимум  процесса нагрузки
  на интервале $[0,\,t]$ сходится по вероятности при  $t\hm\to \infty$  к  явно выписанной
константе.  Этот результат существенно обобщает результат
из работы~\cite{Zeevi}, в которой рассмотрена  жидкостная система с
единственным входным процессом ДБД.

В данной статье, которая  опирается на методы работы~\cite{Zeevi}, а
также на результаты  статьи~\cite{Lukashenko}, продолжен
асимптотический анализ описанной жидкостной системы. Основной
результат данной работы состоит в том, что при некоторых
дополнительных условиях на асимптотическое поведение дисперсии
входного гауссовского процесса доказанная  в~\cite{Lukashenko}
сходимость усилена до сходимости (к той же константе) в пространстве~$L_p$ 
при любом  $p\hm\ge 1$. Более того, при соответствующей
нормировке найдена асимптотика максимума процесса нагрузки в
нестационарном режиме. Кроме того, с использованием полученной
асимптотики максимума процесса нагрузки найдена асимптотика  времени
достижения процессом нагрузки растущего уровня~$b$.

Опишем рассматриваемую систему и используемые ниже результаты из~\cite{Lukashenko} 
более подробно.  Рассмотрим жидкостную систему с
одним обслуживающим устройством и постоянной скоростью обслуживания~$C$, 
на вход которой поступает процесс $A(t)$, заданный  в следующем виде:
\begin{equation}
A(t)=mt+X(t)\,, 
\label{asymp-l1}
\end{equation}
где  $m$~--- средняя интенсивность входного потока, а
$X:\hm=\{X(t)$, $t \hm\geq 0\}$~---  центрированный гауссовский процесс со
стационарными приращениями,  $X(0)\hm=0$. Такая система обслуживания
называется гауссовской~\cite{Mandjes}.  Будем считать, что выполнено
условие $r:=C\hm-m\hm>0$. Обозначим также $W(t)\hm=X(t)\hm-rt$, и пусть
 $Q(t)$  есть
величина нагрузки (незавершенная работа в системе) в момент времени~$t$. 
Будем предполагать, что  $Q(0)\hm=0$. Тогда имеет место соотношение~ \cite{Reich}:
\begin{equation}
Q(t)=\sup\limits_{0 \leq s \leq t}(W(t)-W(s))\,.\label{6a}
\end{equation}
Условие  $r\hm>0$ обеспечивает существование стационарного процесса
нагрузки,  который определяется следующим образом~\cite{Mandjes}:
\begin{equation}
Q= \sup\limits_{t \geq 0} W(t)\,.\label{6}
\end{equation}
 Для пояснения заметим, что величина $-r\hm<0$ есть средний снос
процесса~$W$, являющегося аналогом случайного блуждания (с
независимыми приращениями), максимум которого, по аналогии с~(\ref{6}), 
определяет стационарное время ожидания в классических системах обслуживания~\cite{Asmus}.

Основное предположение, принятое в работе~\cite{Lukashenko}, а также
в данной статье, состоит в том, что функция  $v(t)$ {\it правильно
меняется на бесконечности c индексом} $0\hm<V\hm<2$, т.\,е.\ представима в виде
\begin{equation}
v(t)=t^V L(t)\,,\label{4}
\end{equation}
где функция $L$  медленно меняется на бесконеч\-ности~\cite{Seneta}.
Обозначим 
$$
\beta=\fr{1}{2-V}\,,
$$ 
а также выберем и зафиксируем
любое $\varepsilon \hm\in (0,2-V)$. Будем  считать, что функция~$L$
является {\it дважды дифференцируемой} на $\mathbb{R}_+$ и выполнены
следующие условия (при $t \hm\to \infty$):
\begin{align}
L(tL^\beta(t)) &\sim L(t)\,;\label{10}\\
L''(t)&=o\left( \fr{1}{t^{V+\varepsilon}} \right)\,.\label{11a}
\end{align}
Нетрудно проверить, что из условия~\eqref{11a} следует сходимость
\begin{equation}
v''(t)\ln t \to 0\,,\enskip t \to \infty\,.\label{2.2.l20}
\end{equation}
Как показано в~\cite{Lukashenko},  условия~(\ref{4})--(\ref{11a}) на
самом деле выполнены для широкого класса гауссовских сис\-тем
обслуживания, включающего, например, сис\-те\-мы, на вход которых
поступает сумма нескольких независимых ДБД.
 В~работе~\cite{Konstantopoulos} показано, что на одном вероятностном пространстве можно
 задать процесс $W(t)\hm=X(t)\hm-rt$ и стационарный процесс $Q^*:=\{Q^*(t),\ t \hm\geq 0\}$
таким образом, что одновременно выполнены условия:
\begin{align*}
Q^*(t)&=_d Q \ \mbox{ для всех } \ t \geq 0\,; %\label{15}
\\
Q^*(t)&=W(t)+\max\left\{Q^*(0), L^*(t)\right\}\,,\,\,t \geq 0\,, %\label{16}
\end{align*}
где $=_d$ означает равенство по распределению, а
$L^*(t):=-\min\limits_{0\le s\le t}\{W(s)\}$. Обозначим
\begin{equation*}
M(t)=\max\limits_{0 \leq s \leq t}Q(s)\,,;\enskip M^*(t)=\max_{0 \leq s \leq
t}Q^*(s)\,.
%\label{13}
\end{equation*}
Для удобства обозначим далее
\begin{equation*}
\gamma(t)=L\left[\left(\ln t \right)^\beta\right]  \ln t \,.
\end{equation*}
Ниже будем опираться на результаты следующей теоремы, доказанной
в работе~\cite{Lukashenko}.

\medskip

\noindent
\textbf{Теорема~1.1.}\ \ 
\textit{Пусть дисперсия гауссовской компоненты $X$ входного
процесса}~(\ref{asymp-l1}) \textit{удовлетворяет условиям}~(\ref{10}) 
\textit{и}~(\ref{11a}), \textit{а также} $r\hm>0$. \textit{Тогда}
\begin{align}
\fr{M^*(t)}{\gamma^\beta(t)} &\Rightarrow
\left(\fr{1}{\theta}\right)^\beta\,,\enskip t \to \infty\,;
\label{asymp1-l8}
\\
\fr{M(t)}{\gamma^\beta(t)} &\Rightarrow
\left(\fr{1}{\theta}\right)^\beta\,,\enskip t \to \infty\,,
\label{asymp1-l9}
\end{align}
\textit{где} $\Rightarrow$ \textit{означает сходимость по вероятности, а параметр}~$\theta$ 
\textit{удовлетворяет соотношению}:
\begin{equation}
\theta=\fr{2}{(2-V)^{2-V}}\left( \fr{r}{V}
\right)^V\,.\label{logbuff-l6}
\end{equation}

\smallskip


Как отмечено выше, этот результат  обобщает работу~\cite{Zeevi}, где
процесс $X\hm=B_H$ является ДБД  c параметром Херста  $H\hm\in (1/2,\,1)$.

\section{Сходимость в пространстве $L_p$ %{\boldmath{$L_p$}}
}

В данном разделе будет доказан основной результат, состоящий в том,
что при  дополнительных условиях  на функцию~$L$ из~(\ref{4})
сходимость по вероятности в~(\ref{asymp1-l8}), (\ref{asymp1-l9})
можно усилить до сходимости в пространстве~$L_p$, где
$p\hm\in[1,\,\infty)$.


\medskip

\noindent
\textbf{Теорема~2.1.}\ \ 
\textit{Пусть дополнительно к условиям теоремы}~1.1
\begin{equation}
\liminf\limits_{t\to \infty} L(t)>0\,;\enskip
\limsup\limits_{t\to \infty} L(t)<\infty\,.
%\leq A_2.
\label{10a}
\end{equation}

\textit{Тогда в}~(\ref{asymp1-l8}), (\ref{asymp1-l9}) \textit{имеет место
сходимость в пространстве $L_p$, $p \hm\in [1,\infty)$}.


\smallskip


\noindent
Д\,о\,к\,а\,з\,а\,т\,е\,л\,ь\,с\,т\,в\,о\,.\ \ 
Зафиксируем $p \hm\in [1,\infty)$. Для доказательства теоремы
достаточно доказать равномерную интегрируемость семейства случайных величин
$$
\left\{\left(\fr{ M^*(t)}{\gamma^\beta(t)}\right)^p,\enskip t\ge
T\right\}\,,
$$
где $T$~--- некоторое (конечное) положительное чис\-ло. Для этого, в
свою очередь, достаточно, чтобы было выполнено (см., например,~\cite{Billingsley}) условие
\begin{equation}
\label{2.2.l1} 
\sup\limits_{t \geq T} \mathbb{E} \left[
\fr{M^*(t)}{\gamma^\beta(t)} \right]^{p+1} < \infty\,.
\end{equation}
 (Значение $\mathbb{E}(\cdot)$ при $t\hm<T$
может быть произвольным.) Выберем далее некоторое число $K\hm>0$.
Значения величин~$K$ и~$T$ будут уточняться в процессе
доказательства. Кроме того, всюду далее предполагается, что $t\hm\ge
T$. Имеют место  соотношения:
\begin{multline}
\mathbb{E} \left[ \fr{M^*(t)}{\gamma^\beta(t)}\right]^{p+1} = {}\\
{}=
\int\limits_{0}^\infty (p+1)y^p \Pu\left(M^*(t)>y \gamma^\beta(t)\right)dy={}\\
{}= \int\limits_{0}^K (p+1)y^p \Pu\left(M^*(t)>y \gamma^\beta(t)\right)dy+{}\\
{}+\int\limits_{K}^\infty (p+1)y^p \Pu\left(M^*(t)>y \gamma^\beta(t)\right)dy \leq{} \\
{}\leq K^{p+1} + (p+1)(I_t+R_t)\,,\label{2.2.l9}
\end{multline}
где использованы обозначения:
\begin{align*}
I_t&=\int\limits_{K}^\infty y^p \,\lceil t\rceil\Pu\left(Q^*(0)>\fr{y
\gamma^\beta(t)}{2}\right)dy\,;\\
 R_t&=\int\limits_{K}^\infty y^p\, \lceil t \rceil \times{}\\
& {}\times \Pu\left(
 \max\limits_{0\leq s \leq 1}W(s)-\min\limits_{0 \leq s \leq 1}W(s)>\fr{y
\gamma^\beta(t)}{2}\right)dy\,.
\end{align*}
Отметим, что при получении выражения~(\ref{2.2.l9})
использовано неравенство:
\begin{multline*}
\Pu(M^*(t)>x) \le {}\\
{}\le \lceil t\rceil\Pu\left(Q^*(0)+
\max\limits_{0 \leq s \leq 1}W(s)-\min\limits_{0 \leq s \leq 1} W(s)>x\right).
\end{multline*}
(Cм.\ доказательство теоремы~1.1 в~\cite{Lukashenko}.) Оценим вначале
 интеграл $I_t$.
 В~работе~\cite{Duffy} показано, что если дисперсия $v(t)$
 центрированного гауссовского процесса со стационарными
приращениями  правильно меняется на бесконечности  с индексом
$0\hm<V\hm<2$, то справедлива такая (логарифмическая) асимптотика:
\begin{equation}
\lim\limits_{b \to \infty} \fr{v(b)}{b^2} \ln \Pu(Q^*>b)=-\theta\,,
\label{asymp1-l13}
\end{equation}
где параметр $\theta$ удовлетворяет соотношению~\eqref{logbuff-l6}.
Определим  число $K_1$ следующим образом:
\begin{multline*}
K_1=\inf\left\{ y>0:\,\,\fr{L(x)\ln\Pu(Q^*(0)>x)}{x^{1/\beta}}\leq -
\fr{\theta}{2}\,,\right.\\ 
\left.\forall\ x>y\vphantom{\fr{L(x)Q^*}{x^{1/\beta}}}\right\}\,.
\end{multline*}
Напомним, что $\beta\hm={1}/({2-V})$. Поэтому ввиду~(\ref{4}) 
из~(\ref{asymp1-l13}) следует, что $K_1\hm<\infty$. Тогда при $x\hm>K_1$
справедливо неравенство:
\begin{equation}
\label{2.2.l2}
\Pu(Q^*(0)>x) \leq \exp\left( -\fr{\theta}{2}\, \fr{x^{1/\beta}}{L(x)} \right)\,.
\end{equation}
Заметим, что $\gamma^\beta(t) \to \infty$, $t \hm\to \infty$.
Следовательно, существует такое число $t_0$, что
$\gamma^\beta(t)/2\hm>1$ при $t \hm\geq t_0$. Поэтому, если $K\hm>K_1$, $t
\hm\geq t_0$, то из~(\ref{2.2.l2}) вытекает неравенство:
\begin{multline}
I_t = \int\limits_{K}^\infty y^p \lceil t\rceil\Pu\left(Q^*(0)>\fr{y 
\gamma^\beta(t)}{2}\right)dy \leq \int\limits_{K}^\infty y^p \lceil t\rceil\times{}\\
{}\times \exp\left[
-\fr{\theta}{2^{1/\beta+1}} \, \fr{\ln t \cdot L\left[ (\ln
t)^\beta\right]}{L(y\gamma^\beta(t)/2)}\, y^{1/\beta}
\right]dy\,.\label{18}
\end{multline}
Из условия~(\ref{10a}) следует, что существуют такие числа $0\hm<A_1\hm\le
A_2\hm<\infty $ и $K_2$, $t_1\hm\ge 0$, что при $y\hm>K_2$, $t\hm>\max(t_0,t_1)$
выполняются неравенства:
\begin{align}
\label{2.2.l5} 
L\left(\fr{y\gamma^\beta(t)}{2}\right) &\le A_2\,;
\\
\label{2.2.l6} L\left( (\ln t)^\beta \right)&\ge A_1\,.
\end{align}
Выберем теперь и временно зафиксируем в~(\ref{18})  некоторое
$K\hm>\max\{K_1,K_2\}$, и пусть  пока $T:=$\linebreak $:=\;\max(t_0,t_1)$. Обозначим также 
$$
\alpha=\fr{\theta A_1}{2^{1/\beta+1}A_2}\,.
$$
Последовательно применяя~(\ref{2.2.l5}), (\ref{2.2.l6}), а также
принимая во внимание, что $\lceil t\rceil \hm\leq 2t$, можно получить
из~(\ref{18})  (при $t\hm\geq T$) следующую оценку сверху интеграла
$I_t$:
\begin{multline}
I_t \leq 2 \int\limits_{K}^\infty y^p\, t\exp\left( - \alpha
\ln t \cdot y^{1/\beta}\right)dy={}\\
{}= 2\int\limits_{K}^\infty y^p\exp\left[ \ln t
\left(1-\alpha y^{1/\beta} \right)
\right]dy\,. \label{2.2.l12}
\end{multline}
Заметим, что при $y\hm>\left({2}/{\alpha}\right)^\beta :=K_3$
справедливо неравенство:
\begin{equation}
\label{2.2.l7}
1-\alpha y^{1/\beta} < -\fr{\alpha}{2}y^{1/\beta}\,.
\end{equation}
Если теперь выбрать в~(\ref{2.2.l12}) $K\hm>K_3$, то ввиду~(\ref{2.2.l7}) получим:
\begin{equation}
I_t \leq  2\int\limits_{K}^\infty y^p\exp\left( -\fr{\alpha}{2} \ln t
\cdot y^{1/\beta}\right)dy\,. \label{2.2.l13}
\end{equation}
Заметим, что $\ln t \hm\geq  \ln T$ при $t \hm\geq T$, и обозначим
$\gamma_1\hm=({\alpha}/{2})\ln T$. Тогда  из~(\ref{2.2.l13}) получим:
\begin{multline}
I_t \leq  2\int\limits_{K}^\infty y^p\exp\left( -\gamma_1 y^{1/\beta}\right)dy={}\\
{}= \fr{2\beta}{\gamma_1^{\beta+\beta p}}\int\limits_{\gamma_1
K^{1/\beta}}^\infty u^{\beta p+\beta-1}e^{-u}du={}\\
{}=\fr{2\beta}{\gamma_1^{\beta+\beta p}}\, \Gamma\left(\gamma_1
K^{1/\beta},\beta p+\beta \right)\,,\label{2.2.l10}
\end{multline}
где $\Gamma(w,z)$~--- неполная гам\-ма-функ\-ция:
$$
\Gamma(w,z):=\int\limits_w^\infty u^{z-1}e^{-u}\,du\,, \enskip w\ge 0\,,\ z \geq 0\,.
$$


Теперь оценим интеграл $R_t$ в~\eqref{2.2.l9}. Напомним соотношение
$W(t)=X(t)-rt$ и  заметим, что
\begin{equation}
\max\limits_{0 \leq s \leq 1}W(s)=\max\limits_{0 \leq s \leq 1}( X(s)-rs)
\leq \max\limits_{0 \leq s \leq 1} X(s)\,. \label{asymp1-l11}
\end{equation}
Кроме того,
\begin{multline}
-\min\limits_{0 \leq s \leq 1} W(s)=\max\limits_{0 \leq s \leq 1} (
rs-X(s))=_d{}\\
{}=_d\max\limits_{0 \leq s \leq 1} (rs+ X(s))
\leq r+ \max\limits_{0 \leq s \leq 1} X(s)\,.\label{2.2.l18}
\end{multline}
Неравенства~\eqref{asymp1-l11} и \eqref{2.2.l18} после несложных
преобразований приводят, в свою очередь, к неравенству:
\begin{multline}
\Pu\left(\max\limits_{0\leq s \leq 1}
W(s)-\min\limits_{0 \leq s \leq 1}W(s)>\fr{y \gamma^\beta(t)}{2}\right) \leq {}\\
{}\leq
2 \Pu\left(\max\limits_{0\leq s \leq 1}X(s)>
\fr{y\gamma^\beta(t)}{4}-r\right)\,. \label{2.2.l19}
\end{multline}
Применяя соотношения~\eqref{2.2.l19} и (\ref{2.2.l6}), можно получить
следующую цепочку неравенств:
\begin{multline}
R_t = \int\limits_{K}^\infty y^p \lceil t
\rceil\times{}\\
{}\times \Pu\left(\vphantom{\fr{\gamma^\beta}{2}}
\max_{0\leq s \leq 1}
W(s)-\min\limits_{0 \leq s \leq 1}W(s)>
\fr{y \gamma^\beta(t)}{2}\right)\,dy \leq{} \\
{}\leq 2 \int\limits_{K}^\infty y^p \lceil
t \rceil\Pu\left(\max\limits_{0\leq s \leq 1}X(s)>\fr{y\gamma^\beta(t)}{4}-r\right)dy \leq{} \\
\!\!{}\leq 4 \int\limits_{K}^\infty y^p  t
\Pu\left(\max\limits_{0\leq s \leq 1}X(s)>
\fr{y A_1^\beta(\ln t)^\beta}{4}-r\right)dy.\!\! \label{2.2.l11}
\end{multline}
Теперь   используем  следующее неравенство
Бо\-ре\-ля--Су\-да\-ко\-ва--Ци\-рель\-со\-на для максимума центрированного
гауссовского процесса со стационарными приращениями на конечном
интервале~\cite{Adler, Lifshits}:
\begin{equation}
\Pu \left( \max\limits_{0 \leq s \leq 1} X(s)> x \right) \leq
e^{-({1}/{(2v)})(x-a)^2}\,, \enskip x>a\,,
\label{2.2.l14}
\end{equation}
где $v:=\mathbb{D} X(1)$, $a:=\mathbb{E} \max\limits_{0 \leq s \leq 1} X(s)<\infty$.
Положим
$$
K_4=\inf\left\{y:\,\,\fr{x A_1^\beta(\ln T)^\beta}{4}-r>a\,,\ \forall\  x \geq y\right\}\,.
$$
Тогда при  $x\ge K_4$ неравенство~(\ref{2.2.l14}) выполнено для всех
$z:={x A_1^\beta(\ln T)^\beta}/{4}-r$,   причем
$ z\hm>a$.

  Введем обозначения:
$$
c_1=\fr{4(r+a)}{A_1^\beta}\,;\quad c_2=\fr{32 v }{A_1^{2\beta}}\,.
$$
Пусть теперь  $K\hm>K_4$ в~(\ref{2.2.l11}).  Тогда с учетом~(\ref{2.2.l14}) 
после несложных преобразований можно получить, что
\begin{equation}
R_t \leq 4 \int\limits_{K}^\infty y^p \exp\left[ \ln t -
\fr{\left( y(\ln t)^\beta-c_1\right)^2}{c_2} \right]dy\,.
\label{2.2.l15}
\end{equation}
Рассмотрим функцию
$$
f(t,y):=\ln t -\fr{\left(y(\ln t)^\beta-c_1\right)^2}{c_2}\,.
$$
 Нетрудно убедиться, что
\begin{equation}
\label{2.2.l3} 
\fr{\partial f(t,\,y)}{\partial t}=\fr{1}{t}\left[ 1-\fr{2\beta}{c_2}\left(y(\ln
t)^\beta-c_1\right)(\ln t)^{\beta-1} \right]
\end{equation}
и что при любом $y\hm>K$
$$
\fr{\partial
f}{\partial t}\left(t,y\right)<\fr{\partial f}{\partial t}(t,K)\,.
$$
Анализ правой части выражения (\ref{2.2.l3}) показывает, что
существует такое число $t_2$, что функция $f(t,y)$ (при каждом
$y\hm>K$) убывает (по~$t$) при $t\hm\geq t_2$. Это, в свою очередь, означает, что
\begin{equation}
\label{2.2.l4}
f(t,y) \leq f(t_2,y)\,,\enskip t\geq t_2\,,\enskip y>K\,.
\end{equation}
Обозначим $T\hm=\max\{t_0,\,t_1,\,t_2\}$. Заметим, что
$f(T,\,y) \hm\to -\infty$, $y \hm\to \infty$, и, как нетрудно проверить,
$$
\lim\limits_{y \to \infty} y^{p+2}e^{f(T,\,y)}=0\,.
$$
Поэтому найдется такое число $K_5\hm>0$, что
\begin{equation}
\label{2.2.l8} 
e^{f(T,\,y)}<y^{-p-2}\,,\enskip y>K_5\,.
\end{equation}
Теперь, используя соотношения~(\ref{2.2.l4}) и (\ref{2.2.l8}),
получим из~(\ref{2.2.l15})  при  $K\hm>K_5$ (и $t \hm\geq T$)
\begin{equation}
R_t  \leq 4 \int\limits_{K}^\infty y^p e^{f(T,y)}dy \leq 
4 \int\limits_{K}^\infty \fr{dy}{y^2}=\fr{4}{K}\,.
\label{2.2.l16}
\end{equation}
Выберем окончательно в~(\ref{2.2.l9}) (и далее, где требуется)
$K\hm=\max \left\{K_1,K_2,K_3,K_4,K_5\right\}$. Тогда из~(\ref{2.2.l10}) и~(\ref{2.2.l16}) 
следует неравенство
\begin{multline}
\mathbb{E} \left[ \fr{M^*(t)}{\gamma^\beta(t)}\right]^{p+1} \leq 
K^{p+1} + {}\\
{}+\fr{2\beta(p+1)}{\gamma_1^{\beta+\beta p}}\, \Gamma\left(
\gamma_1 K^{1/\beta},\beta p+\beta \right) + \fr{4(p+1)}{K}\,, 
\label{2.2.l17}
\end{multline}
 правая часть которого не зависит от~$t$. Беря  в левой части
неравенства~\eqref{2.2.l17}  супремум по  $t \hm\geq T$, получаем требуемое 
условие~(\ref{2.2.l1}). \hfill$\square$


\smallskip

\noindent
\textbf{Замечание.}\ Если существует предел
\begin{equation}
\lim\limits_{t\to \infty} L(t)= A\in (0,\,\infty)\,,
\label{38}
\end{equation}
то
условие~\eqref{10a} автоматически вы\-пол\-нено.

\smallskip

Приведем важные для практических применений примеры, когда условия
теоремы~2.1 вы\-пол\-нены.

\smallskip

\noindent
\textbf{Пример 1.}\ Пусть стохастическая компонента входного
процесса является суммой независимых ДБД, т.\,е.
\begin{equation*}
X(t)=\sum\limits_{i=1}^n B_{H_i}(t)\,, \enskip t\ge 0\,,
%\label{42}
\end{equation*}
где параметры Херста $H_i\hm\in (0,\,1)$. Подробная мотивировка такого
входного потока обсуждается  в работе~\cite{Taqqu}.  Без ограничения
общности будем считать, что $H_1\hm>\max\limits_{i>1}H_i$. Тогда
 дисперсия $v(t)$  процесса $\{X(t)\}$  имеет вид:
$$
v(t)=\sum\limits_{i=1}^n t^{2H_i}=t^{2H_1}L(t)\,,
$$
где медленно меняющаяся функция $ L(t)\hm=1\hm+\sum\limits_{i>1}
t^{2(H_i-H_1)}\hm\to 1$, $t \hm\to \infty. $ Таким образом, условия
теоремы~2.1 выполнены.

\smallskip

\noindent
\textbf{Пример 2.} Пусть стохастическая компонента входного процесса
есть так называемый интегральный гауссовский процесс, т.\,е.
\begin{equation}
X(t)=\int\limits_0^t Z(s)\,ds\,,
\label{41}
\end{equation}
где $Z$~--- центрированный стационарный гауссовский процесс с
ковариационной функцией $R(u):=$\linebreak 
$:=\;\Cov\left(Z(0),Z(u)\right)$. Входные
потоки такого типа рассматрива\-лись в работах~\cite{Debicki1,Kulkarni}. Нетрудно проверить, что для дисперсии
$v(t)$ процесса~$X$ справедливо соотношение:
\begin{equation}
v(t)=2 \int\limits_0^t\!\! \int\limits_0^s R(u)\,duds\,.\label{42a}
\end{equation}
Отсюда следует, что $v''(t)\hm=2 R(t)$, а значит условие~\eqref{2.2.l20} влечет сходимость
\begin{equation*}
R(t)\ln t \to 0\,,\enskip t \to \infty\,. 
%\label{2.2.l21}
\end{equation*}
Если дополнительно к условию~\eqref{2.2.l20} потребовать
существования таких  $A \hm\in (0,\infty)$, $V \hm\in (0,2)$, что
\begin{equation}
\fr{\int_0^t \int_0^s R(u)\,duds}{t^V} \to A\,,\enskip
t \to \infty\,,
\label{2.2.l22}
\end{equation}
то условия теоремы~2.1 оказываются выполненными. Например,
пусть  $Z$~--- процесс Орн\-штей\-на--Улен\-бе\-ка, который по определению
является центрированным стационарным гауссовским процессом.
Поскольку его  ковариационная функция  имеет вид $R(t)\hm=\lambda
e^{-\alpha t}$, $\lambda,\alpha\hm>0$, то из~(\ref{42a}) несложно получить, что
$$
v(t)=\fr{2 \lambda}{\alpha}t+\fr{2\lambda}{\alpha^2}\left(e^{-\alpha t}-1\right)\,.
$$
Следовательно, условие~\eqref{2.2.l22} выполнено для $V\hm=1$,
$A={\lambda}/{\alpha}$. Отметим, что  если   $Z$~--- процесс
Орн\-штей\-на--Улен\-бе\-ка,  то формула~(\ref{41}) определяет   {\it
интегральный процесс Орн\-штей\-на--Улен\-бе\-ка}.  Заметим, что этот
процесс является гауссовским аналогом (т.\,е.\ гауссовским процессом с
соответствующей корреляционной структурой) модели
Ани\-ка--Мит\-ра--Сон\-ди~\cite{Anick}, описывающей динамику некоторых
сетевых ресурсов (см.\ так\-же~\cite{Addie}).

На самом деле в обоих примерах выше  выполнено   более сильное, чем~\eqref{10a}, 
условие~(\ref{38}).
Однако утверждение теоремы~2.1  верно и в случае, когда функция~$L$ не имеет предела при 
$t\hm\to\infty$.

\section{Дополнительные асимптотические результаты}

В данном разделе получены два важных асимптотических результата для
максимума процесса нагрузки $M(t)$, дополняющие анализ, проведенный
в разд.~2. Вначале рассмотрим случай, когда параметр
$r\hm<0$. В~соответствии с замечанием, сделанным  после формулы~(\ref{6}), 
в этом случае система находится в   нестационарном режиме
и величина процесса нагрузки должна неограниченно воз\-рас\-тать.


\medskip

\noindent
\textbf{Теорема~3.1.}\ \textit{Если $r<0$, то имеет место следующая сходимость 
по распределению:}
\begin{equation}
\fr{M(t)+rt}{\sqrt{v(t)}}  \xrightarrow{d} \Nor
(0,1)\,, \enskip t \to \infty\,.
\label{max-l1}
\end{equation}

%\smallskip

\noindent
Д\,о\,к\,а\,з\,а\,т\,е\,л\,ь\,с\,т\,в\,о\,.\ \ Напомним, что 
процесс нагрузки в момент времени~$s$ определяется соотношением
$$
Q(s)=W(s)-\min\limits_{0\leq u \leq s}W(u)\,,
$$
где $W(u)= X(u)\hm-ru$.
Поскольку с вероятностью~1 (с~в.~1)
$$
\fr{X(t)}{t} \to 0\,,
$$
то также $W(t) \to +\infty$ с~в.~1. Пусть $\Psi:=\min\limits_{t \geq
0}{W(t)}$. Тогда существует случайный момент  $T_0\hm<\infty$  (с~в.~1)
такой, что $\min\limits_{t \geq 0}W(t)\hm=\min\limits_{0 \leq t \leq
T_0}W(t)$. Поэтому справедлива следующая цепочка соотношений:
\begin{multline*}
M(t)=\displaystyle\max\limits_{0 \leq s \leq t}\left[ W(s)-\min\limits_{0 \leq u \leq s}W(u)\right]\leq{}\\
{}\leq\displaystyle \max\limits_{0 \leq s \leq t} W(s)- \Psi={}\\
{}= \displaystyle\max\limits_{0 \leq s \leq t} W(s)- \min\limits_{0 \leq s \leq T_0} W(s)\,.
\end{multline*}
Отсюда следует  неравенство:
\begin{multline}
M(t)-W(t) \leq{}\\
{}\leq  \max\limits_{0 \leq s \leq t} W(s)-W(t)- 
\min\limits_{0 \leq s \leq T_0} W(s)\,. 
\label{max-l4}
\end{multline}
Поскольку
\begin{align*}
M(t)+rt&=M(t)+X(t)-W(t)\,;\\
\fr{X(t)}{\sqrt{v(t)}}&=_d \Nor(0,1)\,,
\end{align*}
то сходимость~(\ref{max-l1}) эквивалентна сходимости
\begin{equation}
\fr{M(t)-W(t)}{\sqrt{v(t)}}  \Rightarrow 0\,, \enskip t \to
\infty\,.
\label{max-l2}
\end{equation}
Докажем справедливость~\eqref{max-l2}.
Для этого достаточно показать, что для любого $\varepsilon\hm>0$ справедливо соотношение:
\begin{equation}
\Pu \left(\fr{M(t)-W(t)}{\sqrt{v(t)}}> \varepsilon  \right) \to 0\,,
\enskip t \to \infty\,. 
\label{max-l3}
\end{equation}
В силу~(\ref{max-l4})
\begin{multline*}
\Pu \left(\fr{M(t)-W(t)}{\sqrt{v(t)}}> \varepsilon  \right) \leq{}\\
{}\leq 
\Pu \left(\fr{\max\limits_{0 \leq s \leq t}W(s)-W(t)}{\sqrt{v(t)}}> 
\fr{\varepsilon}{2}  \right)+{}
\\{} + \Pu \left(\fr{-\min\limits_{0 \leq s \leq T_0}W(s)}{\sqrt{v(t)}}> 
\fr{\varepsilon}{2}  \right):=\Pu_1(t)+\Pu_2(t)\,.
\end{multline*}
 В силу стацонарости приращений процесса $X$
\begin{multline}
\max\limits_{0 \leq s \leq t} W(s) - W(t)= {}\\
{}=
\max\limits_{0 \leq s \leq t} \left[  X(s)- X(t)+r(t-s)\right]=_d\\
{}=_d \max\limits_{0 \leq s \leq t}\left[  X(t-s)+r(t-s)\right]=\\
= \max\limits_{0 \leq u \leq t}\left[
X(u)+ru\right]:=\widetilde{Q}(t)\,.
\label{48}
\end{multline}
Поскольку  $r\hm<0$, то существует стационарный предел
$\widetilde{Q}(t) \xrightarrow{d} \widetilde {Q}$ (при $t \hm\to \infty$),
причем  $ \widetilde{Q}\hm<\infty$ с в.~1 (см.\ замечание после формулы~\eqref{6a}). 
Поскольку $v(t) \hm\to \infty$, то из~(\ref{48}) следует,
что
\begin{multline*}
\Pu_1(t)=\Pu\left( \max\limits_{0 \leq s \leq t}W(s)-W(t)>
\fr{\varepsilon}{2}\sqrt{v(t)}  \right) \to 0\,,\\
 t \to \infty\,.
\end{multline*}
Рассмотрим  вероятность $\Pu_2(t)$ и заметим, что
\begin{multline*}
-\min\limits_{0 \leq s \leq T_0} W(s)=\max\limits_{0 \leq s \leq T_0 }[rs- X(s)]\leq{}\\
{}\leq \max\limits_{0 \leq s \leq T_0}[- X(s)]=_d  \max\limits_{0 \leq s \leq T_0}X(s)\,.
\end{multline*}
Поэтому
\begin{multline*}
\Pu_2(t) = \Pu\left( -\min\limits_{0 \leq s \leq T_0}W(s)>
\fr{\varepsilon}{2}\sqrt{v(t)} \right)\leq{}\\
{}\leq \Pu\left( \max\limits_{0 \leq s \leq T_0}X(s)>
\fr{\varepsilon}{2}\sqrt{v(t)}\right) \to 0\,,\enskip t \to \infty\,,
\end{multline*}
где учитывается, что  случайная величина  $T_0$, а значит и $\max\limits_{0\leq s \leq T_0}X(s)$, 
конечны с~в.~1. Таким образом, соотношение~(\ref{max-l3}), а значит и~(\ref{max-l2}), выполнено.\hfill$\square$

\smallskip

Следующий результат касается асимптотики времени достижения
стационарным процессом нагрузки $Q^*(t)$ растущего порога~$b$, т.\,е.\
асимптотики величины
$$
T(b)=\inf\{t \geq 0:\,Q^*(t)\geq b\}
$$
при $b\to \infty$.  Распределение максимума стационарного процесса
нагрузки $M^*(t)$ определяет распределение случайной величины $T(b)$
в силу  очевидного соотношения
\begin{equation}
\{ T(b) \leq t\}=\{M^*(t) \geq b \}\,,\enskip t\ge 0\,.
\label{time-l0}
\end{equation}
Напомним обозначение $\gamma(t)\hm=\ln t \cdot L((\ln t)^\beta).$

\medskip

\noindent
\textbf{Теорема~3.2.}\ \textit{Пусть в дополнение  к условиям теоремы}~1.1 \textit{функция
$\gamma(t)$ монотонно возрастает на некотором луче $[t_0,\infty)$.
Тогда имеет место сходимость}
\begin{equation}
\fr{\gamma(T(b))}{b^{1/\beta}} \Rightarrow \theta\,,\enskip b \to
\infty\,, 
\label{time-l1}
\end{equation}
\textit{где параметр $\theta$ удовлетворяет соотношению}~(\ref{logbuff-l6}).

\smallskip

\noindent
Д\,о\,к\,а\,з\,а\,т\,е\,л\,ь\,с\,т\,в\,о\,.\ 
 В~силу теоремы~1.1 для любого $\delta \hm>0$
\begin{equation}
\Pu\left( M^*(t)
> \left( \fr{1+\delta}{\theta}\,\gamma(t)\right)^\beta \right) \to 0\,,
\enskip t \to \infty\,.
\label{time-l2}
\end{equation}
Ввиду монотонного возрастания функции~$\gamma$, обратная ей функция
также монотонно возрастает. В~частности, при любом $\delta\hm>0$ функция
\begin{equation}
t(b):=\gamma^{-1}\left( b^{1/\beta} \fr{\theta}{1+\delta} \right) \to \infty\,,\enskip
b \to \infty\,.
\label{time-l3}
\end{equation}
Подставляя~(\ref{time-l3}) в~(\ref{time-l2}) и учитывая~(\ref{time-l0}), получим
\begin{multline}
\Pu\left( M^*(t(b)) \ge \left(
\fr{1+\delta}{\theta}\,\gamma(t(b))\right)^\beta \right)
={}\\
{}=\Pu\left(
M^*(t(b)) \ge  b\right)=\Pu\left( T(b) \le  t(b)\right)={}\\
{}=\Pu\left( T(b)\leq \gamma^{-1}\left( b^{1/\beta}
\fr{\theta}{1+\delta}
\right)\right)={}\\
{}=\Pu\left( \fr{\gamma(T(b))}{b^{1/\beta }}
 \leq \fr{\theta}{1+\delta} \right) \to 0\,,\enskip b \to  \infty\,.
 \label{50}
\end{multline}
Снова используя монотонность функции~$\gamma$, получим (для любого
фиксированного $\delta\hm>0$):
$$
\hat t(b):=\gamma^{-1}\left( b^{1/\beta} \fr{\theta}{1-\delta}
\right) \to \infty\,,\enskip b \to \infty\,.
$$
С учетом того, что по теореме~1.1
$$
\Pu\left( M^*(t) > \left(
\fr{1-\delta}{\theta}\,\gamma(t)\right)^\beta \right) \to 1\,,\enskip 
t \to \infty\,,
$$
как и выше, получим:
\begin{multline*}
\Pu\left( M^*(\hat t(b)) > \left(
\fr{1-\delta}{\theta}\,\gamma(\hat t(b))\right)^\beta
\right)={}\\
{}=\Pu\left( \fr{\gamma(T(b))}{b^{1/\beta }} \leq
\fr{\theta}{1-\delta} \right) \to 1\,,\ b \to \infty\,.
\end{multline*}
Ввиду произвольности~$\delta$ отсюда и из~(\ref{50}) следует~\eqref{time-l1}.\hfill$\square$

\section{Заключение}

В данной статье  продолжен (начатый в работе~\cite{Lukashenko})
асимптотический анализ максимума процесса нагрузки в системе
обслуживания, в которой дисперсия гауссовской компоненты входного
процесса  правильно меняется  на бесконечности с показателем
$V\hm\in(0,\,2)$.
  В~частности, показано, что при некотором дополнительном  условии
 доказанная в~\cite{Lukashenko} сходимость по вероятности
 указанного процесса  имеет место и
в пространстве~$L_p$  при любом $p\hm\in [1,\,\infty)$. 

Также найдена
асимптотика максимума процесса нагрузки в нестационарном режиме (при
соответствующей нормировке). 
Кроме того, с использованием полученной
асимптотики максимума  найдена асимптотика  времени достижения
стационарным процессом нагрузки растущего   порога~$b$.

{\small\frenchspacing
{%\baselineskip=10.8pt
\addcontentsline{toc}{section}{Литература}
\begin{thebibliography}{99}

\bibitem{Lukashenko}
\Au{Лукашенко~О.\,В., Морозов~Е.\,В.} Асимптотика максимума
процесса нагрузки для некоторого класса гауссовских очередей~//
Информатика и её применения, 2012. Т.~6. Вып.~3. С.~81--89.

\bibitem{Zeevi}
\Au{Zeevi~A., Glynn~P.} On the maximum workload in a queue fed
by fractional Brownian motion~// Ann. Appl. Prob., 2000. Vol.~10.
P.~1084--1099.

\bibitem{Mandjes}
\Au{Mandjes~M.} Large deviations of Gaussian queues.~---
Chichester: Wiley, 2007. 339~p.



\bibitem{Reich}
\Au{Reich~E.} On the integrodifferential equation of Takacs~I~// 
Ann. Math. Stat., 1958. Vol.~29. P.~563--570.

\bibitem{Asmus}
\Au{Asmussen S.}  Applied probability and queues.~--- New York: Springer, 2002. 440~p.

\bibitem{Seneta}
\Au{Сенета~Е.} Правильно меняющиеся функции.~--- М.: Наука, 1985.
143~с.

\bibitem{Konstantopoulos}
\Au{Konstantopoulos~T., Zazanis~M., De Veciana~G.}
Conservation laws and reflection mappings with application to
multiclass mean value analysis for stochastic fluid queues~// 
Stochastic Processes and Their Applications, 1996. Vol.~65.
P.~139--146.

\bibitem{Billingsley}
\Au{Биллингсли~П.} Сходимость вероятностных мер.~--- М.: Наука, 1977. 352~с.

\bibitem{Duffy}
\Au{Duffy~K., Lewis~J.~T., Sullivan~W.~G.} Logarithmic
asymptotics for the supremum of a stochastic process~// 
Ann. Appl. Prob., 2003. Vol.~13. No.\,2. P.~430--445.

\bibitem{Adler}
\Au{Adler~R.\,J.} An introduction to continuity, extrema, and
related topics for general Gaussian processes.~--- Hayward, CA: Institute of 
Mathematical Statistics, 1990. 170~p.

\bibitem{Lifshits}
\Au{Лифшиц~М.\,А.} Гауссовские случайные функции.~--- Киев: ТвиМС, 1995. 248~с.


\bibitem{Taqqu}
\Au{Taqqu~M.~S., Willinger~W., Sherman~R.} Proof of a
fundamental result in self-similar traffic modeling~// Computer
Comm. Rev., 1997. Vol.~27. P.~5--23.


\bibitem{Kulkarni}
\Au{Kulkarni~V., Rolski~T.} Fluid model driven by an
Ornstein--Uhlenbeck process~// Probability  Engineering 
Informational Sci., 1994. Vol.~8. P.~403--417.

\bibitem{Debicki1}
\Au{Debicki~K., Rolski~T.} A Gaussian fluid model~// Queueing
Syst., 1995. Vol.~20. P.~433--452.

\bibitem{Anick}
\Au{Anick~D., Mitra~D., Sondhi~M.~M.} Stochastic theory of a
data handling system with multiple resources~// Bell Syst.
Techn.~J., 1982. Vol.~61. P.~1871--1894.

\label{end\stat}

\bibitem{Addie}
\Au{Addie~R., Mannersalo~P., Norros~I.} Most probable paths and
performance formulae for buffers with Gaussian input traffic~//
Eur. Trans. Telecommunications, 2002. Vol.~13.
P.~183--196.
\end{thebibliography}
}
}

\end{multicols}