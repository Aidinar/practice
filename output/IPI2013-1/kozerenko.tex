\def\stat{kozerenko}

\def\tit{СТРАТЕГИИ ВЫРАВНИВАНИЯ ПАРАЛЛЕЛЬНЫХ ТЕКСТОВ:
СЕМАНТИЧЕСКИЕ АСПЕКТЫ$^*$}

\def\titkol{Стратегии выравнивания параллельных текстов:
семантические аспекты}

\def\autkol{Е.\,Б.~Козеренко}

\def\aut{Е.\,Б.~Козеренко$^1$}

\titel{\tit}{\aut}{\autkol}{\titkol}

{\renewcommand{\thefootnote}{\fnsymbol{footnote}}\footnotetext[1]
{Работа выполнена при частичной поддержке РФФИ (проект 11-06-00476-а).}}

\renewcommand{\thefootnote}{\arabic{footnote}}
\footnotetext[1]{Институт проблем информатики Российской академии наук, kozerenko@mail.ru}



      \Abst{Данная статья посвящена проблемам проектирования и разработки лингвистически
мотивированных механизмов выравнивания параллельных текстов и выявления грамматических
(функ\-ци\-о\-наль\-но-се\-ман\-ти\-че\-ских) соответствий для формирования статистических
портретов языковых употреблений, которые в дальнейшем будут встроены в гибридные модели
машинного перевода. Гибридными называются такие модели, в которых для обработки
естественного языка применяются как статистические механизмы, так и механизмы, основанные
на правилах. Представленный в данной работе подход заключается в использовании исходной
расширяемой грамматики, которая в процессе развития дополняется соответствиями,
извлеченными из параллельных текстов. В~качестве исходной грамматики используется
\textit{когнитивная трансферная грамматика} (КТГ), основанная на \textit{трансфемах}
(двуязычных фразовых структурах), в которой представлены когнитивные и функциональные
характеристики фразовых структур.}

      \KW{выравнивание; параллельные тексты; синтаксис; семантика; фразовые структуры;
гибридные модели; машинный перевод}

\vskip 14pt plus 9pt minus 6pt

      \thispagestyle{headings}

      \begin{multicols}{2}

            \label{st\stat}

    \section{Введение}

     В данной работе проблема выравнивания параллельных текстов
рассматривается в связи с задачей создания лингвистически достоверных
механизмов для построения модулей обучения в системах машинного перевода и
извлечения знаний из текстов. Современный этап исследований в области
обработки естественного языка характеризуется стремительной <<гибридизацией>>
подходов и моделей. Включение статистических характеристик в системы,
основанные на правилах, позволяет отразить динамику и разнообразие языковых
форм и значений, порождаемых в процессе речевой де\-я\-тель\-ности.

     Основная цель исследований, проводимых автором данной работы,~---
извлечение из параллельных текстов на разных языках таких фразовых структур,
которые выражают одинаковые значения, и включение их в систему правил
расширяемой грамматики для решения задач машинного перевода и извлечения
знаний из многоязычных пред\-мет\-но-ори\-ен\-ти\-ро\-ван\-ных текстов.
Расширяемая грамматика, которая при этом используется, содержит когнитивные
и функциональные характеристики фразовых структур и базируется на
формализме КТГ~[1]. Основным
конструктивным элементом КТГ являются \textit{трансфемы} (двуязычные
фразовые структуры), которые задаются лексиконом нетерминальных символов.
Этот лексикон динамически пополняется новыми структурами в процессе
выравнивания параллельных текстов. Формализм КТГ отражает
парадигматические свойства рас\-смат\-ри\-ва\-емых языков, т.\,е.\ язык как сис\-те\-му,
тогда как выравниваемые параллельные тексты являются реализацией речи, т.\,е.\
языка как деятельности. Ключевыми аспектами этой деятельности являются
динамика и синтагматика. Основной единицей анализа синтагматических свойств
параллельных текстов является предложение.

   Выравнивание по предложениям и словам проводится с использованием
статистических методов. На первом этапе получаем параллельные корпуса,
выравненные по предложениям, на следующем этапе производится пословное
выравнивание. Далее проводится выравнивание по \textit{фразам}.

   В статистической исследовательской парадигме термин \textit{фраза} означает
произвольный сегмент предложения, выделенный статистическим инстру\-ментом
и вовсе не являющийся нетерминальным символом (синтаксической единицей)
ка\-кой-ли\-бо формальной грамматики. В~отличие от такого подхода, в
представленных исследованиях \textit{фраза}~--- это синтаксически значимая
единица в составе предложения, которая рассматривается в парадигматическом и
синтагматическом аспектах. При этом сочетаются подходы КГТ
и категориальной грамматики SUG (\textit{Semiotic
Universal Grammar})~[2, 3].

   Основные проблемы выравнивания параллельных текстов и, соответственно,
обучения статистических процессоров естественного языка обуслов\-ле\-ны
наличием значительных трансформаций предложений исходного и целевого
текстов, поскольку каждый язык использует свои специфические механизмы
описания референтной ситуации. Переводческие трансформации исследуются в
теории и практике перевода, структурной и прикладной лингвистике. Усилия
исследователей направлены на выявление наиболее типичных (час\-тот\-ных) и
значимых трансформаций в изучаемых параллельных текстах и развитии
типологии трансформаций, представленных в исходном формализме КТГ. Таким
образом, необходимо разработать и применить такие стратегии и инструменты,
которые обеспечивают наиболее адекватные средства для описания и выявления
сопоставимых языковых структур.

   Важным проектным решением, которое вытекает из опыта исследований и
разработок, является различие между задачами машинного перевода\linebreak методом
трансфера, исполняемого на уровне \textit{трансфем} (синтаксических структур с
функ\-ци\-о\-наль\-но-се\-ман\-ти\-че\-ской мотивацией)~[1], и задачей извлечения
знаний из текстов на разных языках. Для извлечения знаний создается метод
многоязычной обработки, который основан на принципе \textit{интерлингвы}
(язы\-ка-посредника). В~рассматриваемых проектах в роли интерлингвы
используется язык расширенных семантических сетей (РСС)~[4]. Соответственно,
семантически-ориентированное выравнивание параллельных текстов проводится в
двух режимах: (1)~сопоставления на уровне \textit{трансфем} и (2)~сопоставления
на уровне \textit{концептов} и \textit{отношений}. Обе стратегии рассматриваются
в разд.~2 и~3. В~разд.~4 обсуждаются современные подходы к проблеме
выравнивания и анализа параллельных текстов. Функционально-семантические
методы выравнивания с примерами и статистическими данными приводятся в
разд.~5. В~заключительном разделе представлены направления дальнейших
исследований.

    \section{Выравнивание на основе трансфем}

   \textit{Трансфема}~--- это единица когнитивного переноса, устанавливающая
функционально-се\-ман\-ти\-че\-ское соответствие между структурами исходного языка
$L_s$ и структурами целевого языка~$L_T$. Для выравнивания параллельных
текстов трансфемы задаются как правила переписывания, в которых в левой части
стоит нетерминальный символ, а в правой~--- выравненные пары цепочек
терминальных и нетерминальных символов, принадлежащих исходному и
целевому языкам (например, русскому и английскому):
   \begin{equation*}
   T\rightarrow \langle \rho,\alpha \sim \rangle\,,
%   \label{e1-koz}
   \end{equation*}
     где $T$~--- нетерминальный символ, $\rho$ и $\alpha$~--- цепочки
терминальных и нетерминальных символов, принадлежащих русскому и
английскому языкам, $\sim$~--- символ соответствия между нетерминальными
символами, входящими в~$\rho$, и нетерминальными символами, входящими
в~$\alpha$. При выравнивании параллельных текстов на основе когнитивной
трансферной грамматики процесс деривации начинается с пары связанных
исходных символов $S_\rho$ и $S_\alpha$, далее на каждом шаге связанные
нетерминальные символы попарно переписываются с использованием двух
компонентов единого правила.

     Для автоматического извлечения правил из параллельных текстов на основе
когнитивной трансферной грамматики необходимо предварительно\linebreak выравнять
тексты по предложениям и словам. Извлека\-емые правила опираются на пословные
выравнивания таким образом, что вначале идентифицируются исходные фразовые
пары с использованием такого же критерия, как и большинство статистических
моделей перевода, основанных на фразовом подходе~\cite{14-koz}, а именно:
должно быть хотя бы одно слово внутри фразы на одном языке, выравненное с
некоторым словом внутри фразы на другом языке, но никакое слово внутри фразы
на одном языке не может быть выравнено ни с каким словом за пределами парной
ей фразы на другом языке.

   При выравнивании параллельных текстов ставятся следующие цели:
   \begin{itemize}
\item извлечь фразовые структуры, выражающие одинаковые значения в разных
языках~--- \textit{трансфемы};
\item усилить и расширить исходную грамматику;
\item разработать типологию \textit{соответствий} $M$ (от английского слова
\textit{match}~--- \textit{пара}, \textit{соответствие});
\item исследовать контекстные характеристики \textit{трансфем}~$T$ и
межъязыковых \textit{соответствий}~$M$.
\end{itemize}

   Первоочередной задачей при выявлении переводческих трансформаций в
параллельных текстах является выявление наиболее предпочтительных для
каждого языка (русского, английского, французского) способов
конфигурирования фраз (последовательностей категориальных вершин). Эти
предпочтения задаются в виде частотных характеристик для каждого типа
соответствий, которые затем в виде гибридных правил включаются в
расширяемую грамматику. Особое внимание уделяется ситуациям
категориального сдвига (конверсиям) при переводе фраз с одного языка на другой.
Категория фразовой структуры определяется категорией ее головной вершины.
Таким образом, когда в процессе трансфера происходит конверсия и категория
головной вершины изменяется, новое категориальное значение присваивается
всей фразовой структуре. Так, имя существительное, модифицирующее другое
существительное в английском языке, становится в целом ряде случаев
прилагательным при переводе на русский язык (например, \textit{stone wall}~---
\textit{каменная стена}); неличная форма глагола, игра\-ющая роль глагольного
модификатора, становится личной глагольной формой в придаточном
обстоятельственном предложении и~т.\,д.

   В ходе исследований было выявлено 34~типа основных трансформаций,
наиболее статистически значимыми являются гла\-голь\-но-имен\-ные трансформации,
пассивизация (при этом в русском,\linebreak английском и французском языках существует
несколько способов оформления пассива), топикализация, падежный сдвиг
(например, \textit{именительный}~--- \textit{дательный} и~т.\,п.), преобразования
личных\linebreak форм глагола в неличные и наоборот. Были ис\-следованы синонимичные
способы выражения\linebreak одинаковых функционально-семантических значений~--- это
явление в литературе часто обозначают термином \textit{перифраза}
(\textit{paraphrasing})~--- как в пределах одного языка, так и в языковых парах.

   Динамика выравнивания реализуется в \textit{соответствиях}~$M$, которые
фиксируются посредством механизма категориальной грамматики SUG~[2, 3].

   В целом процесс выравнивания организуется в два этапа: предварительный и
семантически-ори\-ен\-ти\-ро\-ван\-ный. На этапе предварительного выравнивания
выполняются следующие процедуры с\linebreak использованием специального набора
инструментальных систем: выравнивание по предложениям~--- ABBYY
aligner~\cite{4-koz}; выравнивание по словам~--- Verbalizator (tokenizer), Berkeley
aligner~\cite{5-koz};\linebreak выравнивание поддеревьев~--- Cognitive Dwarf~\cite{6-koz};
выравнивание по структурам и формирование конкордансов: Sketch
Engine~\cite{7-koz}.

   Семантически-ориентированный этап выравнивания параллельных текстов
проводится в двух режимах: (1)~сопоставления на уровне \textit{трансфем}, при
этом выявляются языковые структуры, выражающие сходные функциональные
значения в параллельных текстах, и (2)~сопоставления на уровне
\textit{концептов} и \textit{отношений}.

   Режим выравнивания первого типа базируется на понятиях
\textit{трансфемы}~$T$ (\textit{transfeme}) и \textit{соответствия}~$M$
(\textit{match}).

   \medskip

   \noindent
   \textbf{Определение.} Трансфема~$T$~--- это единица парадигматического
плана, относящаяся к \textit{языку} как системе, соответствие~$M$~--- единица
синтагматического плана, относящаяся к \textit{речи} (\textit{дискурсу}); таким
образом, трансфемы~$T$ реализуются в соответствиях~$M$.

   \smallskip

   Соответствие~$M$ может быть шире, чем трансфема~$T$, и часто включает
контекст.

   Функционально-семантическое выравнивание на основе трансфем включает:
   \begin{itemize}
     \item выявление фразовых структур, управляемых головными вершинами;\\[-15pt]
     \item выравнивание головных вершин с учетом регулярных соответствий и
трансформаций.
     \end{itemize}
   Функционально-семантическое выравнивание осуществляется на основе
формализма КТГ~[1]:

\noindent
   \begin{multline*}
   G_{CT}={}\\
   {}=\left\{ T_{L_1}, T_{L_2}, N_{L_1},
N_{L_2},P_{CA}, P_{CT}, S_{L_1},S_{L_2}, M, D\right\},\hspace*{-2.78597pt}
%   \label{e2-koz}
   \end{multline*}
где $T_{L_1}$ и $T_{L_2}$~--- множества терминальных символов языков~$L_1$
и~$L_2$; $N_{L_1}$ и $N_{l_2}$~--- множества нетерминальных символов языков
$L_1$ и~$L_2$; $P_{CA}$ и $P_{CT}$~--- правила анализа и синтеза на основе
когнитивного трансфера; $S_{L_1}$ и $S_{L_2}$~--- пара исходных символов
языков~$L_1$ и~$L_2$, с которых начинается процесс анализа и выравнивания
предложений; $M$~--- функция установления соответствия между языковыми
структурами~$L_1$ и~$L_2$; $D$~--- функция, приписывающая значения
вероятности каждому правилу из множеств $P_{CA}$, $P_{CT}$. Ядром
формализма КТГ является лексикон трансфем, который относится к множеству
нетерминальных символов.

\vspace*{-6pt}

    \section{Выравнивание на основе концептов и~отношений}

   Выравнивание на основе концептов (сущностей) и отношений (связей)
параллельных и концептуально-со\-по\-ста\-ви\-мых текстов на различных языках
направлено на выявление языковых реализаций структур знаний и формирования
многоязычных баз знаний, которые затем будут применяться в интеллектуальных
аналитических системах. Такой режим выравнивания будем называть
кон\-цеп\-ту\-ально-ори\-ен\-ти\-ро\-ван\-ным, инструментом для него служит
лингвистический процессор Semantix. При этом выравнивание производится по
языковым объектам, выражающим сущности (концепты), действия (предикаты) и
связи. Для выравнивания по сущностям сопоставляются фразы с именными
головными вершинами, для выравнивания по действиям~--- глагольные вершины
(при этом рассматриваются как личные, так и неличные формы глаголов), для
выравнивания по связям сопоставляются генитивные предложные и
беспредложные фразы.

   Механизмы концептуально-ориентированного выравнивания основаны на
аппарате РСС~\cite{8-koz}, обладающих
достаточной выразительной мощностью для представления естественно-языковых
структур с высокой степенью вложенности и выполняющих роль
\textit{интерлингвы} (языка-посредника). Базовым структурным элементом РСС
является именованный $N$-местный предикат, называемый \textit{фрагментом}.
В~основе РСС лежит множество вершин ($V$), из которых состоят элементарные
фрагменты: $V_0(V_1, V_2, \ldots , V_k/V_{k+1})$, где $V_0, V_1, V_2, \ldots , V_k,
V_{k+1}$, $V, k \hm> 0$. Данный фрагмент представляет $k$-арное отношение.
Фрагментам приписываются роли, вершина~$V_0$ соответствует имени
отношения, вершины $V_1, V_2, \ldots , V_k$ соответствуют объектам, которые
объединены в отношение, а вершина $V_{k+1}$, отделенная косой чертой (/) от всей
структуры в целом, соответствует вершине связи. $V_{k+1}$ называется
   $C$-вершиной, и все эти элементы образуют РСС~\cite{8-koz}

   Все множество языковых объектов задается в виде предикатно-аргументных
структур. Анализ предложения производится на основе унификационной
грамматики. Слова и конструкции, игра\-ющие роль предикатов, в предложении
служат <<опорными>> элементами, и результат анализа предложения образует один
<<расширенный>> предикат, соответствующий главному предикату данного
предложения. Модели управления и трансформационные свойства задаются в
словаре в рамках словарных статей глаголов. В~результате трансформаций
происходит сдвиг моделей управления.

   Особое внимание в исследованиях уделяется номинализации и изменению
предложного управления на беспредложное, например: \textit{стрелять по
уткам}~--- \textit{стрелять уток}.

   Таким образом, кон\-цеп\-ту\-аль\-но-ори\-ен\-ти\-ро\-ван\-ное выравнивание~--- это
процесс извлечения знаний в многоязычном режиме и наполнения базы знаний
для создания пред\-мет\-но-ори\-ен\-ти\-ро\-ван\-ных интеллектуальных экспертных систем.

\vspace*{-12pt}

    \section{Современные подходы к~выравниванию параллельных
текстов: статистические и~лингвистические аспекты}

     Статистические подходы к выравниванию параллельных текстов имеют
целью установление\linebreak\vspace*{-12pt}
\columnbreak

\noindent
 наиболее вероятного выравнивания~$A$ для двух заданных
параллельных текстов~$S$ и~$T$:
     \begin{equation*}
     \argmax\limits_A  P\left( A\vert S,T\right) =\argmax\limits_A P(A,S,T)\,.
%     \label{e3-koz}
     \end{equation*}

     В статистическом машинном переводе делается попытка построить модель
вероятности перевода $P\left( r_1^J\vert e_1^I\right)$, которая описывает
соотношения между некоторой цепочкой~$r_1^J$ на исходном языке и
цепочкой~$e_1^I$ на целевом языке. В~статистической модели выравнивания
текстов $P\left( r_1^J,\,a_1^J\vert e_1^I\right)$ вводится <<скрытое>>
выравнивание~$a_1^J$, задающее отображение из исходной позиции~$j$ в
целевую позицию~$a_j$. Соответствие между моделью перевода и моделью
выравнивания задается следующим образом:
     \begin{equation*}
     P\left( r_1^J\vert e_1^I\right)=\sum\limits_{a_1^J}P\left( r_1^J,\, a_1^J\vert
e_1^I\right)\,.
%     \label{e4-koz}
     \end{equation*}

     Выравнивание $a_1^J$ может содержать выравнивания $a_j\hm=0$ с пустым
словом~$e_0$ для тех слов исходного языка, которые не нашли ни одного
соответствия среди слов целевого языка. В~целом статистическая модель зависит
от множества неизвестных параметров~$\theta$, которые извлекаются из
обучающей выборки. Следующее выражение представляет зависимость модели от
набора параметров:
     \begin{equation*}
     P\left( r_1^J,\,a_1^J\vert e_1^I\right) = p_\theta \left( r_1^J,\,a_1^J\vert
e_1^I\right)\,.
%     \label{e5-koz}
     \end{equation*}

     Для выявления неизвестных параметров~$\theta$ задается обучающий
корпус параллельных текстов, содержащий $S$ пар предложений $\left\{
({r}_s,{e}_s):\ s=1, \ldots ,S\right\}$. Для каждой пары
$({r}_s,{e}_s)$ переменная выравнивания обозначается как
$a\hm=a_1^J$. Неизвестные па\-ра\-мет\-ры устанавливаются посредством
максимизации сходства параллельных текстов в корпусе:
     \begin{equation*}
     \hat{\theta} = \argmax\limits_\theta \prod\limits_{s=1}^S
\sum\limits_{{a}} p_\theta \left({r}_s, {a}\vert
{e}_s\right)\,.
%     \label{e6-koz}
     \end{equation*}

     Как правило, максимизация для таких моделей выполняется на основе
алгоритма максимизации ожидания~\cite{13-koz} или ему подобных. Такой
алгоритм полезен для решения проблемы оценки па\-ра\-мет\-ров, но не является
необходимым для статистического подхода. Следовательно, несмотря на то что
существует большое число выравниваний для данной пары предложений, всегда
можно найти наилучшее выравнивание
     \begin{equation*}
     \hat{a}_1^J=\argmax\limits_{a_1^J} p_{\hat{\theta}}\left(  r_1^J,\,a_1^J\vert
e_1^I\right)\,.
%     \label{e7-koz}
     \end{equation*}

     Выравнивание $\hat{a}_1^J$ также называется выравниванием Витерби для
пары предложений $(r_1^J,\,e_1^I)$.\linebreak Оценка качества выравнивания Витерби по
сравнению с некоторым эталонным выравниванием\linebreak осуществляется вручную.
Параметры моделей статистического выравнивания оптимизируются с\linebreak учетом
критерия максимального сходства, который далеко не всегда отражает качество
выравнивания.

   В этой связи чрезвычайную важность приобретает проблема
перифразирования, и ей уделяется все больше внимания в работах ведущих
исследовательских групп в области компьютерной
   лингвистики~[11--21]. Модель
перифразы, описанная в работе~\cite{13-koz}, обучалась на основе корпуса
Europarl. Авторы использовали десять параллельных корпусов между английским
и каждым из следующих языков: датским, голландским, финским, французским,
немецким, греческим, италь\-ян\-ским, португальским, испанским и шведским,
приблизительно с 30~млн слов для каждого из этих языков и 315~млн английских
слов. Автоматические пословные выравнивания были получены с по\-мощью
Giza++~\cite{15-koz}. Английская часть каждого параллельного корпуса
анализировалась с применением парсера Bikel~\cite{16-koz}. Были разобраны в
общей сложности 1,6~млн уникальных предложений. Языковая модель на основе
триграмм была обучена на этих английских предложениях с использованием
инструментария языкового моделирования SRI~\cite{17-koz}. Этот
инструментарий поддерживает создание и оценку разнообразных типов языковых
моделей на основе статистики $N$-грамм, а также ряд сопутствующих задач,
таких как статистическая разметка (тегирование) и манипуляции списками
   $N$-лучших и решетками слов.

   Извлечение перифраз из двуязычных параллельных корпусов было описано в
работах Bannard и Callison-Burch~\cite{18-koz}, которые выводили перифразы с
использованием методов статистического машинного перевода на основе
фраз~\cite{19-koz}. Затем путем введения комплексных синтаксических меток
вместо использования только нетерминальных символов из деревьев разбора
авторы смогли добиться существенного улучшения по сравнению с основным
методом. Синтаксические ограничения значительно улучшают качество этого
метода перифразы. В~работе~\cite{20-koz} представлен новый подход к
перифразе на основе двуязычного корпуса. Автор демонстрирует, что более
высокое качество может быть достигнуто, если ввести такое ограничение:
перифраза должна иметь тот же синтаксический тип, что и исходная фраза.
В~работе предложены ограничения на перифразы на двух этапах: когда они
выводятся на основе разобранных параллельных корпусов и когда они
подставляются в разобранные тестовые предложения. Автор ввел синтаксические
ограничения, пометив все фразы и перифразы (даже не входящие в число
составляющих) с помощью слеш-ка\-те\-го\-рий, используемых в категориальной
грамматике CCG (комбинаторной категориальной грамматике). Однако автор не
дает ни формального определения некоторой синхронной грамматики, ни
предлагает систему декодирования, поскольку его система представлений не
содержит иерархических правил (деревьев разбора). Правила синхронной
грамматики для перевода извлекаются из пар предложений, автоматически
разобранных и выравненных по словам. Методы извлечения варь\-и\-ру\-ют\-ся в
зависимости от того, извлекают ли они только минимальные правила для фраз, у
которых есть доминирующие узлы в дереве разбора, или более сложные правила,
включающие фразы, не входящие в число составляющих~\cite{21-koz}.

   Главной мотивацией для использования синтаксических перифраз наряду с их
исходными фразовыми соответствиями является их потенциальная возможность в
более общем виде отразить лингвистические трансформации, сохраняющие
значение. Система, обладающая механизмом синтаксических перифраз, должна
быть способна обучаться хорошо оформленным обобщенным фразовым
структурам, которые можно будет применять для анализа невидимых данных.

    \section{Типология наиболее частотных языковых трансформаций
в~параллельных текстах}

   Для создания \textit{типологии трансформаций} были рассмотрены
следующие приемы синхронного перевода:
   \begin{itemize}
     \item полный перевод лексико-грамматических форм (ЛГФ), когда имеет
место полное соответствие между структурами исходного и целевого языков по
форме, функции и значению;
     \item нулевой перевод (ЛГФ используются по-раз\-ному);
     \item частичный перевод (совпадают несколько содержательных функций
ЛГФ);
     \item функциональная замена (имеет место функциональная идентичность
ЛГФ);
     \item конверсия (замена одной грамматической категории на другую).
     \end{itemize}

     Фокусная выборка трансформаций формируется на основе их частотности в
параллельных текстах:
\begin{itemize}
\item     \textit{номинализация} (35\% в англо-русской языковой паре);
\item
     \textit{пассивизация} (18\%--24\% в русско-английском направлении
перевода);
   \item
     \textit{адъективно-адвербиальные трансформации} (12\%--14\% в обоих
направлениях анг\-ло-рус\-ской языковой пары);
     \item
     \textit{субъектно-объектные трансформации} (28\% в обоих направлениях
анг\-ло-рус\-ской языковой пары).
\end{itemize}

     Рассмотрим некоторые примеры:

     Косвенный объект\;$\rightarrow$\;Субъект

     \textit{Серьезными разногласиями была отмечена встреча сторон.}~---
\textit{Serious disagreements arose during the meeting of the sides}.

     Прямой объект\;$\rightarrow$\;Субъект

     \textit{Новую планету назвали в честь ее открывателя}.~--- \textit{The new
planet was named after its discoverer}.

     Предложная фраза\;$\rightarrow$\;Субъект

     \textit{На встрече договорились}.~--- \textit{The meeting reached the
conclusion}.

     Функционально-семантическая мотивация фразовых структур представлена
множеством нетерминальных символов с их частотными характеристиками,
например:

     (``Объектный инфинитив'', 400,0)

     (``Объектный \textit{be}-инфинитив'', 600,0)

     (``Субъектный инфинитив с причастием прошедшим'', 580,0)

     (``Субъектный инфинитив с прилагательным'', 490,0)

     (``Субъект с причастием'', 570,0)

     Будем использовать типизацию соответствий и статистические портреты
трансформаций. Так, русский язык примерно на 35\% более номинативен, чем
английский:

     \textit{In vacuum molecules have large space in which to move} ($V$);

     \textit{В вакууме молекулы имеют большое пространство для движения}
($N$);

     \textit{in which to move} ($V$)\;$\rightarrow$\;\textit{для движения} ($N$).

     Наиболее продуктивные типы глагольно-но\-ми\-на\-тив\-ных трансформаций
коррелируют со следующими функциональными значениями:
     \begin{itemize}
\item обстоятельство цели и следствия, выраженное инфинитивом (58\%);
\item сложный предикат с инфинитивом (be\;+\;infinitive) (51\%).
\end{itemize}

   Эта статистика учитывается в многовариантных правилах когнитивного
трансфера. Соответствия, или \textit{события}, составляют множество хорошо
оформленных нетерминалов, а динамически выявляемые контекстные структуры
представляются в нотации категориальной грамматики (SUG)~[2,~3].

   Для разрешения неоднозначности языковых объектов и структур используются
векторные пространства. Основной прием выравнивания параллельных тестов по
трансфемам~--- установление соответствий между головными вершинами
фразовых структур, в настоящее время осуществляется в полуавтоматическом
режиме, автоматическое сопоставление находится в стадии разработки:
выравнивание по головным элементам фраз и сравнение фразовых структур.

   Трансфемы~$T$ реализуются в соответствиях~$M$. Будем различать два типа
соответствий: прямые соответствия без трансформаций $M^d$  и соответствия, в
которых имеют место трансформации~$M^t$.  Соответствия~$M^d$ демонстрируют прямое соответствие
категориальных значений головных элементов сопоставляемых структур ($S_s$  and
$S_0$), например имя существительное в исходном тексте будет соответствовать
имени существительному в целевом тексте, и фразовые структуры будут
подобными: NP (именная фраза) в исходном и целевом текстах.
      M~--- это соответствия, содержащие трансформации, например имя
существительное в исходном тексте будет соответствовать глагольному элементу
в целевом тексте, и фразовые структуры будут различаться, например:
NP\;$\rightarrow$\;VP.

   Стратегии выравнивания параллельных текстов и установления соответствий,
представленные в данной работе, учитывают возможные трансформации  $M^t_{s\rightarrow o}$,
где $M$~--- это соответствие, $t$~--- трансформация, $s$~--- исходный
(\textit{source}) язык, $o$~--- целевой (\textit{objective}) язык;
$S_M^{CT}\hm\sum M^t_{s\rightarrow o}$ образует пространство соответствий когнитивного
трансфера ($S_M^{CT}$).

   Таким образом, основная цель формирования пространства двуязычных
соответствий~--- обеспечение обучающей выборки для машинного обучения
систем автоматического перевода, основанных на сим\-воль\-но-ста\-ти\-сти\-че\-ских
(гибридных) под\-ходах.

Рассмотрим примеры представления анг\-ло-рус\-ских
выравниваний параллельных текстов.

     \textit{The second step of the above described process is conducted in the
presence of a metal catalyst.}

     NP (Subj) PP VP (Passive) PP (English).

     \textit{Вторую (Adj, F, Sg, Acc) стадию (N, F, Sg, Acc) вышеописанного (N, F,
Sg, Acc) способа (N, M, Sg, Gen) проводят (V, Pl, 3-d Person, Pres) в (P)
присутствии (N, Neutr, Sg, Prep) металлического (Adj, M, Sg, Gen) катализатора
(N, M, Sg, Gen).}

     NP (Accusative) NP (Gen) VP (Person Undefined~--- Plural) PP (Russian).

     Формальное представление трансформации выглядит следующим образом:
     \begin{multline*}
     \mathrm{NP\ (Subj)\ PP\ VP\ (Passive)\ PP}\rightarrow{}\\
{}     \rightarrow\mathrm{NP\ (Accusative)\ NP\ (Gen)\ VP}\\
     (\mathrm{Person\ Undefined}\mbox{~---}
     \mathrm{Plural})\ \mathrm{PP}
     \end{multline*}

     Было выявлено 134~типа соответствий~$M$, которые принадлежат
пространствам когнитивного трансфера.

   Методы машинного обучения~\cite{23-koz, 22-koz} обеспечивают системе
возможность извлекать значения, которые ожидаются в рамках определенных
контекстов, следовательно, вполне естественно использовать статистические
данные, извлеченные из больших массивов текстовой информации и использовать
их для вычисления вероятности реализации определенного значения в
соответствующем контексте.

   Успех вероятностной модели зависит, помимо прочего, от адекватного
определения \textit{события}. Обычный тип события в вероятностной обработке
естественного языка~--- совместное появление одного или нескольких слов в
определенном контексте. В~данном случае \textit{слово} означает не словоформу
(т.\,е.\ цепочку символов в том виде, как она появилась в тексте), а его основу, или
\textit{лемму}. Контекст, в свою очередь, определяется как упорядоченное
множество прилегающих слов с обеих сторон каждого словоупотребления, при
этом ширина такого контекстного окна задается произвольным целым числом.
В~то время как основа лексической единицы (лексемы) может быть разумным
(если не идеальным) способом представления значений, определение контекста,
обычно упо\-треб\-ля\-емое в статистической обработке естественного языка,~--- это
явное упрощение, которое затемняет важные лингвистические отношения. Вполне
разумно предположить, что вероятность слова, имеющего определенное значение,
зависит от других слов, расположенных поблизости. Однако этот <<мешок слов>>
неизбежно скрывает сложные аспекты, для которых требуются более точные
процедуры выявления.

    \section{Заключение}

   В данной статье рассматривается языковая пара рус\-ский--английский, однако
проведенные эксперименты показывают, что основные выводы справедливы для
языков с близкой структурой, например белорусского и украинского. Процессы
синтакси\-ческих трансформаций также очень сходны в ряде других европейских
языков. Включение статистических данных в системы, основанные на правилах,
позволяет отразить динамику и разнообразие языковых форм и значений,
порождаемых в процессе речевой деятельности. Функциональная и когнитивная
мотивация правил исходной грамматики позволяет увеличить точность
соответствий на 23\%--42\% (по оценке экспертов) в за\-ви\-си\-мости от типа
сопоставляемых текстов. Дальнейшие исследования и разработки направлены на
расширение числа сопоставляемых языковых пар и формирование
лингвистического ресурса размеченных параллельных текстов по научной и
экономической тематике для обучения систем машинного перевода и обработки
знаний.


{\small\frenchspacing
{%\baselineskip=10.8pt
\addcontentsline{toc}{section}{Литература}
\begin{thebibliography}{99}

\bibitem{1-koz}
\Au{Kozerenko E.\,B.} Cognitive approach to language structure segmentation for machine translation
algorithms~// Conference (International) on Machine Learning, Models, Technologies and
Applications Proceedings.~--- Las Vegas, USA: CSREA Press, 2003. P.~49--55.
\bibitem{2-koz}
\Au{Shaumyan S.} Categorial grammar and semiotic universal grammar~// IC-AI'03: Conference
(International) on Artificial Intelligence Proceedings.~--- Las Vegas, USA: CSREA Press, 2003.
P.~623--629.
\bibitem{3-koz}
\Au{Kozerenko E.\,B., Shaumyan S.} Discourse projections of semiotic universal grammar~//
Conference (International) on Machine Learning, Models, Technologies and Applications
Proceedings.~--- Las Vegas, USA: CSREA Press, 2005. P.~3--9.

\bibitem{8-koz} %4
\Au{Kuznetsov I.\,P., Kozerenko E.\,B., Matskevich~A.\,G.} Intelligent extraction of knowledge
structures from natural language texts~// 2011 IEEE/WIC/ACM Joint Conferences (International) on
Web Intelligence and Intelligent Agent Technology~--- Workshops WI-IAT 2011: Proceedings.
P.~269--272.

\bibitem{14-koz} %5
\Au{Och F.\,J., Ney H.} A~systematic comparison of various statistical alignment models~//
Computational Linguistics, 2003. Vol.~29. No.\,1. P.~19--51.


\bibitem{4-koz} %6
The web site for ABBYY Aligner. {\sf http://aligner.\linebreak abbyyonline.com/ru}.
\bibitem{5-koz} %7
The web site for Berkeley Aligner. {\sf http://snap.cs.\linebreak berkeley.edu}.
\bibitem{6-koz} %8
The description of Cognitive Dwarf.
{\sf http://www.isa.ru/ proceedings/images/documents/2008-38/91-109.pdf}.
\bibitem{7-koz} %9
The web site for Sketch Engine. {\sf http://www.\linebreak sketchengine.co.uk}.

%%%%%%%%%%%%%%%%%

\bibitem{9-koz} %10n
\Au{Dempster A.\,P., Laird N.\,M., Rubin~D.\,B.} Maximum likelihood from incomplete data via the
EM algorithm~// J.~R. Stat. Soc. Ser.~B, 1977. Vol.~39. No.\,1. P.~1--22.

\bibitem{16-koz} %11n
\Au{Bikel D.} Design of a multi-lingual, parallel processing statistical parsing engine~// HLT-2002
Proceedings, 2002. P.~178--182.

\bibitem{17-koz} %12n
\Au{Stolcke A.} SRILM~---  an extensible language modeling toolkit~//
Conference (International) on Spoken Language Processing Proceedings.~---
Denver, Colorado, 2002. P.~901--904.

\bibitem{10-koz} %13n
\Au{Pang B., Knight K., Marcu~D.} Syntax-based alignment of multiple translations: Extracting
paraphrases and generating new sentences~// NAACL'03: Conference of the North American Chapter
of the Association for Computational Linguistics on Human Language Technology Proceedings, 2003.
Vol.~1. P.~102--109.

\bibitem{19-koz} %14
\Au{Koehn P., Och F.\,J., Marcu~D.} Statistical phrase-based translation~// NAACL'03:  Conference
of the North American Chapter of the Association for Computational Linguistics on Human Language
Technology Proceedings, 2003. Vol.~1. P.~48--54.

\bibitem{11-koz} %15n
\Au{Galley M., Hopkins M., Knight~K., Marcu~D.}
What's in a translation rule?~// HLT/NAACL Proceedings, 2004. P.~273--280.

\bibitem{12-koz} %16n
\Au{Koehn P.} A~parallel corpus for statistical machine translation~//
MT-Summit Proceedings.~--- Phuket, Thailand, 2005. P.~79--86.

\bibitem{18-koz} %15 17n
\Au{Bannard C., Callison-Burch C.} Paraphrasing with bilingual parallel corpora~// ACL Proceedings,
2005. P.~597--604.

\bibitem{20-koz} %16 18n
\Au{Callison-Burch C.} Syntactic constraints on paraphrases extracted from parallel corpora~//
EMNLP-2008 Proceedings, 2008. P.~196--205.

\bibitem{13-koz} %19n
\Au{Koehn P.} Statistical machine translation.~---
Cambridge: University Press, 2009.

\bibitem{15-koz} %20n
The web site for GIZA++. {\sf http://www.statmt.org/ moses/giza/GIZA++.html}.

\bibitem{21-koz}
\Au{Ganitkevitch Ju., Callison-Burch~C., Napoles~C., Van~Durme~B.} Learning sentential
paraphrases from bilingual parallel corpora for text-to-text generation~// Conference on Empirical
Methods in Natural Language Processing Proceedings, 2011. P.~1168--1179.

\bibitem{23-koz}
\Au{Malkov K.\,V. Tunitsky D.\,V.} On extreme principles of machine learning in anomaly and
vulnerability assessment~// MLMTA'06: Conference (International) on Machine Learning, Models,
Technologies and Applications Proceedings.~--- Las Vegas, USA, 2006. P.~24--29.

\label{end\stat}

\bibitem{22-koz}
\Au{Bogatyrev K.} In defense of symbolic NLP~// MLMTA'06: Conference (International) on
Machine Learning, Models, Technologies and Applications Proceedings.~--- Las Vegas, USA, 2006.
P.~63--68.

\end{thebibliography}
}
}

\end{multicols}