\def\stat{abstr}
{%\hrule\par
%\vskip 7pt % 7pt
\raggedleft\Large \bf%\baselineskip=3.2ex
A\,B\,S\,T\,R\,A\,C\,T\,S \vskip 17pt
    \hrule
    \par
\vskip 21pt plus 6pt minus 3pt }

\label{st\stat}

%\def\rightmark{\ }

%1
\def\tit{ANALYTICAL MODELING OF~INVARIANT
MEASURE DISTRIBUTIONS IN~STOCHASTIC 
SYSTEMS WITH~DISCONTINUOUS CHARACTERISTICS}

\def\aut{I.\,N. Sinitsyn}

\def\auf{IPI RAN, sinitsin@dol.ru}

\def\leftkol{\ } % ENGLISH ABSTRACTS}
\def\rightkol{\ } %ENGLISH ABSTRACTS}

\titele{\tit}{\aut}{\auf}{\leftkol}{\rightkol}

%\vspace*{12pt}

\noindent
Based on normal approximation and statistical linearization,   exact and approximate
algorithms for distibutions with invariant measure analytical modeling in nongaussian 
stochastic systems (StS) with discontinuous characteristics are developed. Peculiarities 
in Poisson StS are considered. Test examples confirm the practical accuracy of algorithms.



\vspace*{-2pt}

\KWN{analytical modeling; autocorrelated noise; distribution with invariant measure;
method of normal approximation; method of statistical linearization; Poisson stochastic system;
Pugachev integrodifferential equations, stochastic regime; stochastic system in Ito sense}

%\thispagestyle{myheadings}



\vskip 12pt plus 6pt minus 3pt

%\pagebreak


%2
\def\tit{PROBABILITY AND STATISTICAL MODELING OF~INFORMATION FLOWS 
IN~COMPLEX FINANCIAL SYSTEMS BASED ON~HIGH-FREQUENCY DATA}

\def\aut{V.\,Yu.~Korolev$^1$, A.\,V.~Chertok$^2$, A.\,Yu.~Korchagin$^3$, 
and~A.\,K.~Gorshenin$^4$}


\def\auf{$^1$Faculty of Computational Mathematics and Cybernetics, 
   M.\,V.~Lomonosov Moscow State University; IPI RAN,\\
   $\hphantom{^1}$ vkorolev@cs.msu.su\\[1pt]
$^2$Faculty of Computational Mathematics and Cybernetics, 
   M.\,V.~Lomonosov Moscow State University; Euphoria\\
      $\hphantom{^1}$Group LLC, a.v.chertok@gmail.com\\[2pt]
$^3$Faculty of Computational Mathematics and Cybernetics, 
   M.\,V.~Lomonosov Moscow State University,\\
      $\hphantom{^1}$sasha.korchagin@gmail.com\\[1pt]
$^4$IPI RAN, a.k.gorshenin@gmail.com}


\def\leftkol{\ } % ENGLISH ABSTRACTS}

\def\rightkol{\ } %ENGLISH ABSTRACTS}

\titele{\tit}{\aut}{\auf}{\leftkol}{\rightkol}

%\vspace*{-2pt}

\noindent
A microstructure model is proposed for information flows in complex financial systems 
and stochastic nature of intensities of flows of claims which determines market price 
formation. The outer information flow with random intensity is considered separately within 
the proposed and statistically verified multiplicative model. This model makes it possible to 
analyze characteristics related to the intensities of the flows of claims and instant relation 
between the forces of buyers and sellers without modeling of exogenous information background 
which is practically impossible to predict. Also, the generalized price process model is 
proposed which makes account of all the available information on flows of claims and admits 
further analytical interpretation.

\vspace*{-2pt}

\KWN{financial markets; information flows; market price formation; 
intensity of the flow of claims; limit order book; mixture of probability distributions; 
generalized price}


%\pagebreak

 \vskip 12pt plus 6pt minus 3pt

%\pagebreak

\def\leftkol{\ } % ENGLISH ABSTRACTS}
\def\rightkol{\ } %ENGLISH ABSTRACTS}

 %3
\def\tit{STATIONARY CHARACTERISTICS OF~THE~QUEUEING SYSTEM
WITH~LIFO SERVICE, PROBABILISTIC PRIORITY, AND~HYSTERIC POLICY}

\def\aut{T.\,A.~Milovanova$^1$ and A.\,V.~Pechinkin$^2$}

\def\auf{$^1$People's Friendship University of
Russia, tmilovanova77@mail.ru\\[1pt]
$^2$IPI RAN, apechinkin@ipiran.ru}



\titele{\tit}{\aut}{\auf}{\leftkol}{\rightkol}

%\vspace*{-4pt}
 
\noindent
Consideration is given to single server queueing system with LIFO service discipline,
probabilistic priority, and single-threshold hysteric policy.
Expressions for main stationary characteristics, including stationary
waiting and sojourn times distributions in terms of Laplace--Stieltjes transform, are obtained.


\vspace*{-2pt}

\KWN{queueing system; last-in-first-out; probabilistic priority; hysteretic policy}

 \vskip 12pt plus 6pt minus 3pt

%4
\def\tit{ON CONVERGENCE IN  THE SPACE $L_p$ OF~THE~WORKLOAD MAXIMUM 
FOR~A~CLASS OF~GAUSSIAN QUEUEING  SYSTEMS}

\def\aut{O.\,V.~Lukashenko$^1$ and~E.\,V.~Morozov$^2$}

\def\auf{$^1$Institute of Applied Mathematical Research of Karelian Research Center, 
Russian Academy of Sciences;\\
$\hphantom{^1}$Petrozavodsk State University, lukashenko-oleg@mail.ru\\[1pt]
$^2$Institute of Applied Mathematical Research of Karelian Research Center, 
Russian Academy of Sciences;\\
$\hphantom{^1}$Petrozavodsk State University, emorozov@karelia.ru}


%\def\leftkol{ENGLISH ABSTRACTS}
%\def\rightkol{ENGLISH ABSTRACTS}

\titele{\tit}{\aut}{\auf}{\leftkol}{\rightkol}

\vspace*{-2pt}

\def\leftkol{ENGLISH ABSTRACTS}

\def\rightkol{ENGLISH ABSTRACTS}


\noindent
A class of queueing systems fed by an input containing linear 
deterministic component and a random component described by a centered Gaussian process  
is considered.   The variance of the input is a regularly varying at infinity function 
with exponent $0<V<2$.  The conditions are found  under which the maximum of stationary  
workload  (remaining  work)  over time interval  $[0,\,t]$ 
converges in the space $L_p$  as $t\rightarrow\infty$ (and under an appropriate scaling) 
to  an explicitly given constant~$a$. Asymptotics of the workload maximum in 
nonstationary regime is also given.  The asymptotics of the hitting time  
of an increasing value~$b$ by the workload process is  obtained.

\vspace*{-2pt}

\KWN{Gaussian queue;  workload maximum; fractional Brownian motion; 
asymptotical analysis;  regular varying}


 \vskip 12pt plus 6pt minus 3pt

%5
\def\tit{ALGORITHMS FOR INDUCTIVE GENERATION OF~SUPERPOSITIONS FOR~APPROXIMATION 
OF~EXPERIMENTAL DATA}

\def\aut{G.\,I.~Rudoy$^1$ and V.\,V.~Strijov$^2$}

\def\auf{$^1$Moscow Institute of Physics and Technology, rudoy@forecsys.ru\\[1pt]
$^2$Dorodnicyn Computing Centre of RAS, 
strijov@ccas.ru}

%\def\leftkol{ENGLISH ABSTRACTS}
%\def\rightkol{ENGLISH ABSTRACTS}

\titele{\tit}{\aut}{\auf}{\leftkol}{\rightkol}

\vspace*{-2pt}

\def\leftkol{ENGLISH ABSTRACTS}

\def\rightkol{ENGLISH ABSTRACTS}

\noindent
The paper presents an algorithm which inductively generates admissible 
nonlinear models. An algorithm to generate all admissible superpositions 
of given complexity in finite number of iterations is proposed. The proof 
of its correctness is stated. The proposed approach is illustrated by a 
computational experiment on synthetic data.


\vspace*{-2pt}

\KWN{symbolic regression; nonlinear models; inductive generation; models complexity}

 \vskip 12pt plus 6pt minus 3pt

%6
\def\tit{STATISTICAL TECHNIQUES OF~BANS DETERMINATION OF~PROBABILITY MEASURES IN~DISCRETE SPACES}


\def\aut{A.\,A.~Grusho$^1$, N.\,A.~Grusho$^2$, and~E.\,E.~Timonina$^3$}

\def\auf{$^1$IPI RAN; Faculty of Computational Mathematics and Cybernetics, 
   M.\,V.~Lomonosov Moscow State University,  grusho@yandex.ru\\[1pt]
$^2$IPI RAN, info@itake.ru\\[1pt]
$^3$IPI RAN, eltimon@yandex.ru}

\def\auf{IPI RAN, mkrivenko@ipiran.ru}

%\def\leftkol{ENGLISH ABSTRACTS}
%\def\rightkol{ENGLISH ABSTRACTS}

\titele{\tit}{\aut}{\auf}{\leftkol}{\rightkol}

\vspace*{-2pt}

\def\leftkol{ENGLISH ABSTRACTS}

\def\rightkol{ENGLISH ABSTRACTS}

\noindent
The method of statistical bans determination of probability measures in 
the discrete spaces is suggested. Consistency of the constructed estimates 
is shown. The application of the received estimates for testing the statistical 
hypotheses in the discrete spaces is constructed. It is shown that estimates of 
bans can generate consistent sequences of criteria in some sense. 
 

\vspace*{-2pt}

\KWN{consistent sequences of criteria; bans of probability measures in the discrete spaces; 
consistency of estimates}

%\pagebreak

 \vskip 12pt plus 6pt minus 3pt

%7
\def\tit{OPERATIONS ON THE TREE REPRESENTATIONS OF~PIECEWISE QUASI-AFFINE FUNCTIONS}

\def\aut{S.\,A.~Guda} 


\def\auf{Faculty of Mathematics, Mechanics and Computer Science, 
Southern Federal University, gudasergey@gmail.com}

%\def\leftkol{ENGLISH ABSTRACTS}
%\def\rightkol{ENGLISH ABSTRACTS}

\titele{\tit}{\aut}{\auf}{\leftkol}{\rightkol}

\vspace*{-2pt}

\def\leftkol{ENGLISH ABSTRACTS}

\def\rightkol{ENGLISH ABSTRACTS}

\noindent
The piecewise quasi-affine (PQAF) functions  and PQAF-sets are considered. 
Tree representation of PQAF-function and PQAF-set and its complexity are introduced. 
The algorithms of union, intersection, empty check, sum, image/preimage calculation, 
inversion, composition, and comparison are given. The algorithms are provided with complexity 
estimates of the resulting objects. A~theorem about the type and complexity of 
lexicographical extrema in the PQAF-set, depending on the parameters, has been proved.


%\vspace*{-2pt}

\KWN{piecewise quasi-affine function; quasi-affine selection tree; lexicographical extremum; 
Z-polytope}


 \vskip 14pt plus 6pt minus 3pt


%8
\def\tit{METHODOLOGICAL BASIS FOR~THE~CREATION OF~INFORMATION SYSTEMS FOR~THE~CALCULATION 
OF~INDICATORS 
OF~THEMATIC LINKAGES BETWEEN~SCIENCE~AND~TECHNOLOGY}

\def\aut{V.\,A.~Minin$^1$, I.\,M.~Zatsman$^2$, M.\,G.~Kruzhkov$^3$, and~T.\,P.~Norekyan$^4$} 

\def\auf{$^1$RFBR, minin@rfbr.ru\\[1pt]
$^2$IPI RAN, iz\_ipi@a170.ipi.ac.ru\\[1pt]
$^3$IPI RAN, magnit75@yandex.ru\\[1pt]
$^4$IPI RAN, izzittami@gmail.com}

%\def\leftkol{ENGLISH ABSTRACTS}
%\def\rightkol{ENGLISH ABSTRACTS}

\titele{\tit}{\aut}{\auf}{\leftkol}{\rightkol}

%\vspace*{-2pt}

\def\leftkol{ENGLISH ABSTRACTS}

\def\rightkol{ENGLISH ABSTRACTS}

%\vspace*{-2pt}

\noindent
International experience of indicators' calculation of thematic 
linkages between science and technology is analyzed. The purpose 
of the analysis is to develop creation principles of information systems 
for calculating indicators of linkages with an allowance for structure of 
inherited scientific and patent information resources, which have been historically 
formed in Russia. This type of information systems is new to the Russian 
scientific and technical sphere. Their creation is necessary for evaluation of R\&D 
programs and decision-making at all stages of program activities. In the paper, the 
methodology of indicators' calculation of thematic linkages in the Russian scientific 
and technical sphere is outlined to develop creation principles of these information 
systems.

%\vspace*{-2pt}

\KWN{linkages between science and technology; 
classification of the scientific domains; international patent classification; 
classifying scientific documents}

\vskip 14pt plus 6pt minus 3pt



%9
\def\tit{PARALLEL TEXTS ALIGNMENT STRATEGIES: THE~SEMANTIC ASPECTS}

\def\aut{E.\,B.~Kozerenko}

\def\auf{$^1$IPI RAN, kozerenko@mail.ru}


\def\leftkol{ENGLISH ABSTRACTS}

\def\rightkol{ENGLISH ABSTRACTS}

\titele{\tit}{\aut}{\auf}{\leftkol}{\rightkol}

%\vspace*{-2pt}

\noindent
The paper deals with the problems of design and development of 
the linguistically motivated mechanisms for parallel texts alignment and grammatical 
(functional semantic) matches extraction for design of statistical language portraits 
to be further incorporated into the hybrid models of machine translation. By hybrid models 
are meant those employing both statistical and rule-based mechanisms within one framework for 
natural language processing. The presented approach consists in the use of the starting (``seed'') 
grammar to be further enriched by the matches discovered in parallel texts. The seed grammar 
is used featuring the cognitive and functional characteristics of phrase structures. The 
grammar formalism employed is the Cognitive Transfer Grammar based on the transfemes 
(bilingual phrase structures). The interpreters' transformations existing in parallel 
texts are of primary importance to the research efforts.

%\vspace*{-2pt}

\KWN{alignment; parallel texts; syntax; semantics; phrase structures; 
hybrid models; machine translation}
%\pagebreak

\vskip 14pt plus 6pt minus 3pt

%10
\def\tit{SEMANTIC VECTOR SPACES FOR~DIFFERENT KNOWLEDGE DOMAINS}

\def\aut{Yu.\,I.~Morozova}

\def\auf{IPI RAN, yulia-ipi@yandex.ru}


\def\leftkol{ENGLISH ABSTRACTS}

\def\rightkol{ENGLISH ABSTRACTS}

\titele{\tit}{\aut}{\auf}{\leftkol}{\rightkol}

%\vspace*{-2pt}

\noindent
The paper focuses on the actual problems of studying semantics of linguistic units 
using the methods of corpus linguistics. It gives a review of distributional semantics 
which is a new area of linguistic research. The paper proposes an enhancement to the 
existing distributional models by switching from lexemes to word collocations. The 
paper describes the methodology used to build semantic vector spaces for different 
knowledge domains.
 
%\vspace*{-2pt}

\KWN{distributional semantics; vector spaces; meaningful word combinations; collocations}

%\pagebreak

\vskip 12pt plus 6pt minus 3pt

% \vskip 12pt plus 6pt minus 3pt

%11
\def\tit{INFORMATION METHOD FOR ASSESSMENT SEMANTIC ADEQUACY OF~TEXTS}

\def\aut{L.\,A.~Kuznetsov$^1$ and V.\,F.~Kuznetsova$^2$}

\def\auf{$^1$Lipetsk State Technical University, kuznetsov.leonid48@gmail.com\\[1pt]
$^2$Lipetsk State Technical University, kuznetsov@stu.lipetsk.ru}


\def\leftkol{ENGLISH ABSTRACTS}

\def\rightkol{ENGLISH ABSTRACTS}

\titele{\tit}{\aut}{\auf}{\leftkol}{\rightkol}

\vspace*{-2pt}

\noindent 
A problem of automated knowledge examination is considered based 
on intelligent comparison of student's answers with reference knowledge chunks. 
In order to increase relevance of automatically drawn conclusions, the 
information theory ground is used to develop the original methodology of textual content 
closeness evaluation. As a part of this work, this approach is applied to evaluate 
how close students' essays match reference textual blocks, relating to a subject.


\vspace*{-2pt}

\KWN{semantic similarity texts; probabilistic model of text; information theory; entropy; 
transmitted information of texts; knowledge estimate automatization}

\vskip 12pt plus 6pt minus 3pt


%12
\def\tit{VARIANCE-GENERALIZED-GAMMA-DISTRIBUTIONS AS~LIMIT LAWS FOR~RANDOM SUMS}

\def\aut{L.\,M.~Zaks$^1$ and V.\,Yu.~Korolev$^2$}

\def\auf{$^1$Department of Modeling and 
Mathematical Statistics, Alpha-Bank, lily.zaks@gmail.com\\[1pt]
$^2$Faculty of Computational Mathematics and Cybernetics, 
   M.\,V.~Lomonosov Moscow State University; IPI RAN,\\
   $\hphantom{^1}$vkorolev@cs.msu.su}


\def\leftkol{ENGLISH ABSTRACTS}

\def\rightkol{ENGLISH ABSTRACTS}

\titele{\tit}{\aut}{\auf}{\leftkol}{\rightkol}

\vspace*{-2pt}

\noindent 
A general theorem is proved establishing necessary and sufficient conditions 
for the convergence of the distributions of sums of a random number of independent 
identically distributed random variables to variance-mean mixtures of normal laws. 
As a corollary, necessary and sufficient conditions for the convergence of the 
distributions of sums of a random number of independent identically distributed 
random variables to variance-generalized-gamma-distributions are obtained. For a 
special case of continuous-time random walks generated by compound doubly stochastic 
Poisson processes, convergence rate estimates are presented.


\vspace*{-2pt}

\KWN{random sum; generalized hyperbolic distribution; generalized inverse Gaussian distribution; 
generalized gamma-distribution; variance-generalized-gamma-distribution; mixture of probability 
distributions; identifiable mixtures; additively closed family; convergence rate estimate}


\vskip 12pt plus 6pt minus 3pt

%13
\def\tit{ON BOUNDS FOR THE CONCENTRATION FUNCTIONS OF~REGULAR STATISTICS CONSTRUCTED 
FROM~SAMPLES WITH~RANDOM SIZES}

\def\aut{V.\,E.~Bening$^1$, N.\,K.~Galieva$^2$, and~V.\,Yu.~Korolev$^3$}

\def\auf{$^1$Faculty of Computational Mathematics and Cybernetics, 
   M.\,V.~Lomonosov Moscow State University; IPI RAN,\\
   $\hphantom{^1}$bening@cs.msu.su\\[1pt]
$^2$Kazakhstan Branch,  M.\,V.~Lomonosov Moscow State University, nurgul\_u@mail.ru\\[1pt]
$^3$Faculty of Computational Mathematics and Cybernetics, 
   M.\,V.~Lomonosov Moscow State University; IPI RAN,\\
      $\hphantom{^1}$vkorolev@cs.msu.su}


\def\leftkol{ENGLISH ABSTRACTS} % ENGLISH ABSTRACTS}

\def\rightkol{ENGLISH ABSTRACTS}

\titele{\tit}{\aut}{\auf}{\leftkol}{\rightkol}

\vspace*{-2pt}

\noindent
Two-sided bounds for the concentration functions 
of regular statistics constructed from samples with random sizes are constructed.

\vspace*{-2pt}


\KWN{concentration function; random sum; asymptotically normal statistic; 
Student distribution; Laplace distribution}


\vskip 14pt plus 6pt minus 3pt

%14
\def\tit{ON THE ABSOLUTE CONSTANTS IN~THE~BERRY--ESSEEN INEQUALITY AND~ITS~STRUCTURAL 
AND~NONUNIFORM IMPROVEMENTS}

\def\aut{I.\,G.~Shevtsova}

\def\auf{Department of Mathematical Statistics, Faculty of Computational Mathematics 
and Cybernetics, 
   M.\,V.~Lomonosov Moscow State University; IPI RAN, ishevtsova@cs.msu.su}



\def\leftkol{ENGLISH ABSTRACTS} % ENGLISH ABSTRACTS}

\def\rightkol{ENGLISH ABSTRACTS}

\titele{\tit}{\aut}{\auf}{\leftkol}{\rightkol}

%\vspace*{12pt}

\noindent
New improved upper bounds are presented for the absolute constants in the 
Berry--Esseen inequality and its structural and nonuniform improvements. 
In particular, it is shown that the absolute constant in the classical Berry--Esseen 
inequality does not exceed 0.5514 in general case and 0.4690 for the case of 
identically distributed summands. The corresponding bounds in the Nagaev--Bikelis inequality
are 21.82 and~17.32.


%\vspace*{-5pt}

 \label{end\stat}

\KWN{central limit theorem; convergence rate estimate; normal approximation; 
Berry--Esseen inequality; Nagaev--Bikelis inequality; absolute constant}

%\vspace*{-2pt}




\newpage