
\def\stat{adigeev}

\def\tit{О ПОЛИНОМИАЛЬНОЙ РАЗРЕШИМОСТИ УЛЬ\-ТРА\-МЕТ\-РИ\-ЧЕС\-КИХ ВЕРСИЙ\\
  НЕ\-КО\-ТО\-РЫХ NP-ТРУД\-НЫХ ЗАДАЧ}

\def\titkol{О полиномиальной разрешимости уль\-тра\-мет\-ри\-чес\-ких версий
  не\-ко\-то\-рых NP-трудных задач}

\def\autkol{М.\,Г.~Адигеев}

\def\aut{М.\,Г.~Адигеев$^1$}

\titel{\tit}{\aut}{\autkol}{\titkol}

%{\renewcommand{\thefootnote}{\fnsymbol{footnote}} \footnotetext[1]{Работа
%выполнена при финансовой поддержке РФФИ (проект 11-01-00515а).}}

\renewcommand{\thefootnote}{\arabic{footnote}}
\footnotetext[1]{Южный федеральный университет, madi@math.sfedu.ru}


\Abst{Статья посвящена анализу важных частных случаев
задачи коммивояжера и задачи Штейнера. Обе эти задачи являются
NP-трудными даже в метрическом случае, т.\,е.\ для графов, у которых
функция стои\-мости ребер удовлетворяет неравенству треугольника.
Более строгим ограничением является \textbf{усиленное} неравенство
треугольника: $ \forall x,y,z \hm\in X \quad c(x,z) \hm\leq \max\{c(x,y),
c(y,z)\} $. Метрические функции, удовлетворяющие такому условию,
называются \textbf{ультраметрическими}. В~статье на основе анализа
графов с ультраметрической функцией стои\-мости ребер разработан
алгоритм, позволяющий построить для такого графа гамильтонов цикл
минимальной стои\-мости за время $O(n^2)$, где $n$~--- число вершин
графа. Для задачи Штейнера показано, что при ультраметрической
функции стои\-мости ребер минимальное дерево Штейнера содержит только
терминальные вершины и поэтому также может быть построено за
полиномиальное время как минимальное остовное дерево на подграфе
исходного графа.}

\KW{ультраметрическая функция; усиленное неравенство треугольника; задача коммивояжера;
дерево Штейнера; полиномиальные алгоритмы}

\DOI{10.14357/19922264140207}

\vskip 10pt plus 9pt minus 6pt

      \thispagestyle{headings}

      \begin{multicols}{2}

            \label{st\stat}



\section{Введение}

В статье рассматривается вопрос о полиномиальной разрешимости
частных случаев известных вычислительно сложных задач~--- задачи
коммивояжера и задачи Штейнера. Известно~\cite{Gary_Johnson}, что в
общем случае обе эти задачи являются NP-труд\-ны\-ми, т.\,е.\ в
настоящее время не известны алгоритмы, находящие точное решение этих
задач за полиномиальное время (и, более того, есть основания
полагать, что таких алгоритмов не существует в принципе).\linebreak Эти задачи
остаются NP-трудными даже при наложении значительных ограничений на
исходные данные~--- в том числе при условии, что функция стои\-мости ребер
удовлетворяет неравенству треугольника. Более строгим ограничением
является \textbf{усиленное} неравенство треугольника: $ \forall x,y,z
\hm\in X \quad c(x,z) \hm\leq \max\{c(x,y), c(y,z)\} $. Метрические
функции, удовлетворяющие такому условию, называются \textbf{ульт\-ра\-мет\-ри\-че\-ски\-ми}. Например, подобные функции возникают при
решении одной из ключевых задач вычислительной биологии~---
вычислении филогенетического дерева, отража\-юще\-го эволюционные связи
между современными видами~[2--5],
а также в задачах управления кэшированием данных~\cite{Li},
разработки и анализа нейроподобных сетей~\cite{Kintsel}.

В работе Д.~Гасфилда~\cite{Gusfield} приведен алгоритм, имеющий временн$\acute{\mbox{у}}$ю сложность
$O(n^2)$ ($n$~--- чис\-ло вершин графа) и строящий цепь на графе с ульт\-ра\-мет\-рической функцией стои\-мости.
Эта цепь оказывается минимальной гамильтоновой цепью, хотя данный факт не используется в дальнейших построениях Гасфилда.
Таким образом, из полученных в~\cite{Gusfield} результатов следует
полиномиальная разрешимость \textbf{незамкнутой} ультраметрической задачи коммивояжера,
однако приведенный алгоритм не обобщается на ультраметрические версии других NP-трудных задач.

В~данной работе, отталкиваясь от результата~\cite{Gusfield}, предлагается метод,
позволяющий строить полиномиальные по времени алгоритмы для замкнутого варианта задачи коммивояжера и
для задачи Штейнера.

\section{Определения}

Дадим необходимые определения.
Пусть $G(V,E)$~--- неориентированный связный граф и $c: E\hm\rightarrow R_{+}$~---
функция стоимости, заданная на ребрах графа~$G$. Стоимость цепи, цикла или дерева на графе
определяется как сумма стоимостей ребер, входящих в эту цепь, цикл или дерево.

Цепь (цикл) на графе называется \textbf{га\-миль\-то\-но\-вой}, если она ровно по одному разу
проходит через каждую вершину графа.

\textbf{Незамкнутая задача коммивояжера}: для заданного графа~$G$ и
функции стоимости~$c$
найти гамильтонову {\it цепь} минимальной стоимости.

\textbf{Замкнутая задача коммивояжера}: для заданного графа~$G$ и функции стоимости~$c$
найти гамильтонов {\it цикл} минимальной стоимости.

Известно~\cite{Gary_Johnson, Christofides}, что как замкнутая, так и незамкнутая
задачи коммивояжера являются NP-труд\-ны\-ми, т.\,е.\ для этих задач в общем виде в
настоящее время не существует полиномиальных по времени точных алгоритмов решения.

Задача коммивояжера остается NP-трудной даже при рассмотрении важного частного случая ---
метрической задачи коммивояжера, т.\,е.\ варианта, в котором функция стоимости
ребер удовлетворяет требованиям к метрическим функциям:
\begin{itemize}
\item
неотрицательность:

\noindent
\begin{gather*}
\forall x,y \in X \quad c(x,y)\geq 0\,;
\\
c(x,y)\hm=0 \Leftrightarrow  x=y\,;
\end{gather*}
\item
симметричность:

\vspace*{2pt}

\noindent
$$
\forall x,y \in X \quad c(x,y)= c(y,x)\,;
$$
\item
неравенство треугольника:

\vspace*{2pt}

\noindent
$$
\forall x,y,z \in X \quad c(x,z) \leq c(x,y)\hm + c(y,z)\,.
$$
\end{itemize}

Иногда рассматривают более общий вариант задачи, допуская прохождение через каждую
вершину более одного раза. В~\cite{Meinika} показано, что в случае мет\-ри\-че\-ской задачи это
обобщение не является существенным: оптимальный цикл (или цепь) проходит через
каждую вершину ровно один раз даже в том случае, если разрешено проходить
более одного \mbox{раза}.

Другим частным случаем задачи коммивояжера является задача с \textbf{ульт\-ра\-мет\-рической}
функцией стоимости. Метрическая функция называется ульт\-ра\-мет\-ри\-че\-ской, если помимо
приведенных выше условий неотрицательности и симметричности она удовлетворяет также
\textbf{усиленному неравенству треугольника}:
$$
\forall x,y,z \in X \quad c(x,z) \leq \max\left\{c(x,y), c(y,z)\right\}.
$$

В работе~\cite{Gusfield} приведен алгоритм, который за полиномиальное
($O(n^2)$, где $n$~--- число вершин графа) время строит на графе гамильтонову цепь
(алгоритм {\sf НайтиГамильтоновуЦепь} в данной \mbox{статье}).
Таким образом, незамкнутая задача коммивояжера на графах с
ультраметрической функцией сто\-и\-мости
является полиномиально разрешимой.
В~данной работе на основе результатов~\cite{Gusfield} показано, что полиномиально разрешимым
является также и за\-мк\-ну\-тый вариант задачи коммивояжера с ульт\-ра\-мет\-ри\-че\-ской функцией стоимости ребер.

Обобщением задачи коммивояжера является задача нахождения
минимального остова ограниченной степени~\cite{Gary_Johnson, Bui}:
для заданного графа~$G$, функции стоимости~$c$ и натурального числа~$k$
найти остовное дерево, у которого степени всех вершин не
превосходят~$k$ и которое имеет минимальную стоимость среди деревьев
такого вида. При $k\hm=2$ эта задача преобразуется в незамкнутую задачу
коммивояжера и, таким образом, является NP-труд\-ной в случае
произвольной функции~$c$. Из~\cite{Gusfield} немедленно следует, что
эта задача также полиномиально разрешима для ультраметрических
графов.

Еще одной известной NP-труд\-ной (в общем случае) задачей является
построение минимального дерева Штейнера~\cite{Gary_Johnson}. Деревом
Штейнера для заданного графа $G(V,E)$ и множества \textbf{терминальных}
вершин $X \hm\subseteq V$ называется подграф графа~$G$, являющийся
деревом и содержащий все терминальные вершины. Требуется построить
дерево Штейнера, имеющее минимальную стоимость (т.\,е.\ минимальное
дерево Штейнера). В~данной работе показано, что в случае
ультраметрической функции стоимости минимальное дерево Штейнера
содержит только терминальные вершины и поэтому является минимальным
остовным деревом на подграфе, порожденном множеством терминальных
вершин. Из этого следует, что такое дерево может быть построено за
полиномиальное время (например, алгоритмом Краскала или Прима).

\section{Алгоритмы для~ультраметрических графов}

Пусть $G(V,E)$~--- неориентированный связный граф и $c: E\hm\rightarrow
R_{+}$~--- функция стоимости, заданная на ребрах графа~$G$ и
удовлетворяющая требованиям к ультраметрическим функциям. В~этом
случае граф~$G$ без потери общности можно считать полным. Положим
также $c(v,v)\hm=0$ для любой вершины~$v$. Для упрощения формулировок
всюду в данной статье \textbf{треугольником} $(u,v,w)$ будем называть
подграф графа~$G$, порожденный множеством вершин $\{ u, v, w\}$,
т.\,е.\ состоящий из этих вершин и из ребер, соединяющих эти вершины
между собой.

\subsection{Построение минимальной гамильтоновой цепи}

Полученные в данной работе результаты основаны на алгоритме Гас\-фил\-да~\cite{Gusfield}.
Этот алгоритм приведен (с изменением обозначений на используемые в данной статье) ниже в виде процедуры {\sf НайтиГамильтоновуЦепь}.
Процедура получает на входе граф $G(V,E)$ с $n$ вершинами и за время $O(n^2)$
строит гамильтонову цепь $P$ минимальной стои\-мости.

\medskip

{\centering
\framebox{
\centering
\begin{minipage}{77 mm}
Процедура {\sf НайтиГамильтоновуЦепь}$(G)$
\begin{enumerate}
\item
$U := V$.
\item
Положить $P$ равным пустому пути.
\item
Произвольно выбрать вершину $v \in V$.
\item
Повторить $n-1$ раз:
\begin{enumerate}
\item[(а)]
Удалить $v$ из $U$.
\item[(б)]
Найти вершину $w \in U$ такую, что $c(v,w) \hm\leq c(v,u)$ для всех $u \in U$.
\item[(в)]
Добавить дугу $(v,w)$ в $P$.
\item[(г)]
$v := w$.
\end{enumerate}
\end{enumerate}
\end{minipage}
}
}


%\begin{multicols}{2}

Поскольку гамильтонова цепь является частным случаем остовного дерева
и удовлетворяет ограничениям на степень вершины при любом $k\hm>1$, из
результата~\cite{Gusfield} немедленно следует

\smallskip

\noindent
\textbf{Утверждение.}
Для неориентированного графа $G(V,E)$ с ультраметрической функцией стоимости и
произвольного натурального числа $k\hm>1$ минимальное остовное дерево со степенями вершин,
меньшими или равными~$k$, может быть найдено за время $O(|V|^2)$.

\subsection{Преобразование треугольника}
Алгоритм {\sf НайтиГамильтоновуЦепь} за полиномиальное время решает незамкнутую задачу коммивояжера.
Однако он не адаптируется напрямую для решения других задач, рассматриваемых в данной статье.
Их решение начнем с операции, которую будем называть <<преобразование треугольника>>.

Пусть $H$~--- подграф графа~$G$ и для тройки вершин $r$, $v$ и~$w$ ребра $(r,v)$ и $(r,w)$ принадлежат~$H$,
а ребро $(v,w)$ не принадлежит. Преобразование треугольника $(r,v,w)$ заключается в следующем:

\medskip

{\centering
\framebox{
\centering
\begin{minipage}{77 mm}
Процедура {\sf ПреобразоватьТреугольник}$(H, r,v,w)$
\begin{enumerate}
\item
Из ребер $(r,v)$ и $(r,w)$ выбрать такое, стоимость которого больше или равна $c(v , w)$.
Если этому требованию удовлетворяют оба ребра, то выбрать любое из них.
%\par
%Пусть выбрано ребро $(r,x)$, где $x$ --- это либо $v$, либо $w$.
%Через $y$ обозначим оставшуюся вершину из $\{v,w\}$.
\item
Удалить из $H$ выбранное в п.~1 ребро.
\item
Добавить к $H$ ребро $(v,w)$.
%\item
%Вернуть вершину $y$.
\end{enumerate}
\end{minipage}
}
}

%\begin{multicols}{2}

Рисунок~1 иллюстрирует преобразование треугольника для случая $c(r,v) \hm\geq c(v,w)$.
Пунктиром показаны ребра, не принадлежащие~$H$.

Заметим, что на шаге~1 алгоритма {\sf ПреобразоватьТреугольник} требуется выбрать
ребро, удов-\linebreak\vspace*{-12pt}
\noindent
\begin{center}  %fig1
\mbox{%
\epsfxsize=72.364mm
\epsfbox{adi-1.eps}
}
  \vspace*{5pt}

{{\figurename~1}\ \ \small{Преобразование треугольника}}
  \end{center}

\vspace*{8pt}

\addtocounter{figure}{1}

\noindent
ле\-творяющее определенному
условию. Поэтому необходимо обосновать допустимость этого преобразования, т.\,е.\ показать,
что такое ребро всегда \mbox{найдется}.

\smallskip

\noindent
\textbf{Теорема 1.} \textit{Если
функция стоимости является ультра\-мет\-ри\-че\-ской, то для любых $(r,v,w)$
преобразование треугольника является допустимым и в результате его выполнения
стоимость подграфа~$H$ не увеличивается.}

\smallskip

\noindent
Д\,о\,к\,а\,з\,а\,т\,е\,л\,ь\,с\,т\,в\,о\,.\ \
Для обоснования допустимости преобразования
треугольника необходимо и достаточно показать, что на шаге~1 всегда
найдется ребро, принадлежащее~$H$ (т.\,е.\ $(r,v)$ или $(r,w)$),
стоимость которого больше или равна стоимости ребра $(v,w)$. Но это
следует из усиленного неравенства треугольника

\noindent
$$
c(v,w)  \leq  \max\{c(r,v), c(r,w)\}.
$$
В~силу правила выбора ребра на шаге~1 при замене этого ребра на ребро $(v,w)$
общая стоимость подграфа~$H$ не увеличится.
Теорема~1 доказана.

%\vspace*{-4pt}

\subsection{Задача коммивояжера}

Алгоритм решения замкнутой задачи коммивояжера основан на применении
полиномиального по времени алгоритма для построения гамильтоновой цепи~---
алгоритма {\sf НайтиГамильтоновуЦепь}.

Для построения и обоснования алгоритма решения замкнутой задачи коммивояжера
введем дополнительные обозначения и докажем два вспомогательных утверждения.

Пусть $G(V,E)$~--- неориентированный граф, на ребрах которого задана
ультраметрическая функция стоимости~$c$.
Обозначим:

\noindent
$$
c_{\max} = \max \limits_{e \in E} c(e)\,;
\
E_{\max}= \{ e \in E : c(e) = c_{\max} \}\,.
$$

\smallskip

\noindent
\textbf{Лемма 1.}\
\textit{Любая вершина $v \hm\in V$ инцидентна ребру из $E_{\max}$.
}

\smallskip

\noindent
Д\,о\,к\,а\,з\,а\,т\,е\,л\,ь\,с\,т\,в\,о\,.\ \
Выберем произвольную вершину $v \hm\in V$ и произвольное ребро
$e\hm=(u,w)\in$\linebreak\vspace*{-12pt}

\pagebreak

\noindent
$\in  E_{\max}$.
Рассмотрим треугольник $(u, v, w)$. В~этом треугольнике ребро $e\hm=(u,w)$
имеет стоимость~$c_{\max}$.
Поэтому, по усиленному неравенству треугольника, как минимум одно из
ребер $(v,u)$ или $(v,w)$
также имеет стоимость $c_{\max}$. Лемма~1 доказана.

\smallskip

\noindent
\textbf{Лемма 2. }
\textit{Множество~$V$ можно разбить на непересекающиеся подмножества
(\textbf{кластеры}) $V_1,\ldots , V_k$ таким образом, что}
\begin{enumerate}
\item
$\forall i \neq j, \forall u \hm\in V_i, \forall v\hm \in V_j$
\textit{выполняется} $c(u,v)\hm=c_{\max}$.
\item
$\forall i$ и $ \forall u,v \hm\in V_i$ \textit{выполняется} $c(u,v)\hm < c_{\max}$.
\end{enumerate}

\smallskip

\noindent
Д\,о\,к\,а\,з\,а\,т\,е\,л\,ь\,с\,т\,в\,о\,.\ \
Для любой вершины~$v$ через $R(v)$ обозначим множество вершин, соединенных с
$v$ ребрами со стоимостью, меньшей~$c_{\max}$:
$R(v) \hm= \{u \in V : c(u,v) < c_{\max}\}$.
Заметим, что каждое множество $R(v)$ не пусто,
поскольку $v \hm\in R(v)$.
Для доказательства леммы достаточно показать, что для различных вершин
$u,v \hm\in V$ множества $R(u)$ и $R(v)$ либо не пересекаются, либо совпадают.
Покажем это методом от противного.

Предположим, что существуют две вершины~$u$ и~$v$ ($u \hm\neq v$), для
которых $R(u)$ и $R(v)$ пересекаются, но не совпадают. Тогда
существуют вершины~$x$ и~$y$, не совпадающие ни с~$u$, ни с~$v$,
такие что: $x \hm\in R(u) \cap R(v)$ и $y \hm\in R(u) \setminus R(v)$.
Рассмотрим подграф, образованный вершинами~$v$, $x$ и~$y$. Так как $y
\not\in R(v)$, то $c(v,y)\hm=c_{\max}$. В~соответствии с усиленным
неравенством треугольника как минимум одно из ребер $(v,x)$ или
$(x,y)$ также должно иметь стоимость~$c_{\max}$. Но $x \hm\in R(v)$,
поэтому $c(v,x) \hm< c_{\max}$. Следовательно, $c(x,y) \hm= c_{\max}$. Но,
с другой стороны, в подграфе, образованном вершинами~$u$, $x$ и~$y$,
стоимость каждого из ребер $(u,x)$ и $(u,y)$ меньше~$c_{\max}$,
поскольку $x,y \hm\in R(u)$. Это в сочетании с $c(x,y) \hm= c_{\max}$
противоречит усиленному неравенству треугольника.

Таким образом, $R(u)$ и $R(v)$ либо не пересекаются, либо совпадают.
Поэтому в качестве кластеров можно взять различные множества вида $R(v)$, $v \hm\in V$.
Лемма~2 доказана.


\smallskip

\noindent
\textbf{Лемма 3. }
\textit{Пусть $V \hm= \bigcup \limits_{i=1}^{k} V_i$~--- разбиение множества
вершин на кластеры и $G_i$ $(i=1,\ldots,k)$~---
подграфы, порожденные этими кластерами. Тогда любой минимальный
гамильтонов цикл~$Z^{*}$ на графе~$G$ может быть представлен в виде}
\begin{equation}
\label{min_ham_cycle}
P_1, e_1, P_2, e_2, \dots, e_{k-1}, P_k, e_k\,,
\end{equation}
где $P_i$ $(i=1,\dots,k)$~--- минимальные гамильтоновы цепи на~$G_i$ и
ребра~$e_i$ принадлежат~$E_{\max}$ (рис.~2).
И~наоборот, любой
гамильтонов цикл, имеющий такой вид, является минимальным.

\columnbreak

\noindent
\begin{center}  %fig2
\mbox{%
\epsfxsize=73.449mm
\epsfbox{adi-2.eps}
}
%  \vspace*{6pt}
 \end{center}

\vspace*{3pt}

 \noindent
{{\figurename~2}\ \ \small{Структура минимального гамильтонова цикла на
ультраметрическом графе}}


\vspace*{12pt}

\addtocounter{figure}{1}






\smallskip

\noindent
Д\,о\,к\,а\,з\,а\,т\,е\,л\,ь\,с\,т\,в\,о\.\ \
Для того чтобы доказать лемму, достаточно показать,
что любой минимальный гамильтонов цикл~$Z^{*}$ имеет следующую структуру:
\begin{enumerate}[($i$)]
\item заходя в ка\-кой-ли\-бо кластер~$G_i$,
выходит из него только после полного обхода всех вершин клас\-те\-ра. Иными словами,
$Z^{*}$ заходит и выходит из каждого кластера ровно по одному разу;
\item внутри кластера~$Z^{*}$ проходит по минимальной гамильтоновой цепи.
\end{enumerate}

Рассмотрим гамильтонов цикл~$Z$, не удовлетворяющий условию~($i$).
Тогда $Z$ имеет вид, изображенный на рис.~3,\,\textit{а}, где ребра
$e_1,\dots,e_4$ имеют стоимость~$c_{\max}$, так как соединяют вершины
разных кластеров (кластер изображен овалом).

\begin{figure*} %fig3
\vspace*{1pt}
\begin{center}
\mbox{%
\epsfxsize=158.106mm
\epsfbox{adi-3.eps}
}
\end{center}
\vspace*{-9pt}
\Caption{Преобразование цикла, не удовлетворяющего условию~($i$)}
\end{figure*}

Преобразуем данный цикл так, как изображено на рис.~3,\,\textit{б}. Здесь ребра
$e_1$, $e_5$ и $e_6$ также имеют стоимость~$c_{\max}$, а стоимость
ребра~$e_7$ меньше~$c_{\max}$, так как оно находится внутри кластера.
Допустимость преобразования следует из полноты графа~$G$ (т.\,е.\
ребро~$e_7$ обязательно существует) и леммы~1 (существуют требуемые
для преобразования ребра~$e_5$ и~$e_6$). В~результате получим
гамильтонов цикл меньшей стоимости. Это означает, что исходный цикл
не был минимальным.

Если для гамильтонова цикла~$Z$ выполняется~($i$), но нарушается
условие~($ii$), то $Z$ не минимален, поскольку можно заменить его
фрагмент внутри кластера на минимальную гамильтонову цепь и получить
гамильтонов цикл меньшей стоимости.

Верно и обратное утверждение. Действительно, если гамильтонов цикл~$Z$
имеет вид~(\ref{min_ham_cycle}), то он
совпадает по стоимости с одним из минимальных гамильтоновых циклов и,
следовательно, сам является минимальным.
Лемма~3 доказана.

\smallskip

\noindent
\textbf{Теорема 2.}\
\textit{Для неориентированного графа $G(V,E)$ с ультраметрической функцией
стоимости минимальный гамильтонов цикл может быть найден за время $O(|V|^2)$.}


\smallskip

\noindent
Д\,о\,к\,а\,з\,а\,т\,е\,л\,ь\,с\,т\,в\,о\,.\ \
Для нахождения минимального гамильтонова
цикла применим алгоритм {\sf НайтиГамильтоновЦикл}, приведенный
ниже.

\medskip

{\centering
\framebox{
\centering
\begin{minipage}{77 mm}
Процедура {\sf НайтиГамильтоновЦикл}($G$)
\begin{enumerate}
\item
Построить кластеры $\{V_1,\dots,V_k\}$.
\par
Пусть $G_i (i=1,\dots,k)$~--- подграфы, порожденные кластерами~$V_i$.
\item
Для каждого $i$ от~1 до~$k$ на графе~$G_i$ построить минимальную гамильтонову
цепь~$P_i$ с по\-мощью процедуры {\sf НайтиГамильтоновуЦепь}.
\item
Построить гамильтонов цикл~$Z$, последовательно соединив для каждого~$i$
конец цепи~$P_i$ с началом цепи $P_{i+1}$ ребром из~$E_{\max}$ (конец $P_k$
соединяется с началом~$P_1$), т.\,е.\ цикл~$Z$ должен выглядеть как на рис.~2.
\end{enumerate}
\end{minipage}
}
}

В~силу леммы~2 всегда существует требуемое на шаге~1 разбиение
множества вершин на кластеры. В силу леммы~1 на шаге~3 существует
возможность соединить концы и начала построенных цепей требуемым
образом. Из леммы~3 следует, что цикл~$Z$ совпадает по виду с
минимальным гамильтоновым циклом и, следовательно, сам является
минимальным.

Оценим временн$\acute{\mbox{у}}$ю сложность алгоритма.
Шаг~1 (разбиение на кластеры) можно выполнить, обходя граф поиском в
глубину по ребрам из $E \setminus E_{\max}$. Каждая компонента связности,
выделяемая при таком обходе, соответствует кластеру. Поэтому временн$\acute{\mbox{а}}$я сложность
шага~1 совпадает со сложностью поиска в глубину и не превышает $O(|V|^2)$.

На шаге~2 процедура {\sf НайтиГамильтоновуЦепь} вызывается $k$~раз.
Пусть $n_i \hm= |V_i|$ $(i\hm=1,\ldots,k)$.
Тогда временн$\acute{\mbox{а}}$я сложность шага~2 оценивается как
$O\left(\sum\limits_{i=1}^{k} {n_i}^2\right)$, что не превышает
$O\left(\left(\sum\limits_{i=1}^{k} {n_i}\right)^2\right)$, т.\,е.\ $O\left(|V|^2\right)$.
Очевидно, что шаг~3 также может быть выполнен за время $O(|V|^2)$.
Теорема~2 доказана.


%\vspace*{-6pt}

\subsection{Задача Штейнера}

%\vspace*{-2pt}

Покажем, что в случае ультраметрической функции стоимости минимальное дерево
Штейнера совпадает
с минимальным остовным деревом подграфа, порожденного множеством терминальных вершин
(т.\,е.\ графа, состоящего из множества терминальных вершин и множества всех ребер, соединяющих
терминальные вершины друг с другом). Таким образом, минимальное дерево Штейнера для графа с
ультраметрической функцией стои\-мости можно строить за полиномиальное время уже известными алгоритмами
(например, алгоритмом Прима).

\smallskip

\noindent
\textbf{Теорема 3. }
\textit{Пусть $G(V,E)$~--- связный неориентированный граф,
$X \subseteq V$~--- множество терминальных вершин,
$c: E \rightarrow R_{+}$~---  ультраметрическая функция стоимости.
Тогда на~$G$ существует минимальное дерево Штейнера, состоящее только
из терминальных вершин.
Если стоимости всех ребер графа строго положительны, то любое
минимальное дерево Штейнера на~$G$ содержит только терминальные вершины.}


\smallskip

\noindent
Д\,о\,к\,а\,з\,а\,т\,е\,л\,ь\,с\,т\,в\,о\,.\ \
Пусть $T(V_T,E_T)$~--- минимальное дерево Штейнера для $G$, $c$ и~$X$.

Предположим, что $T$ содержит нетерминальную вершину~$s$.
Если $\mathrm{deg}_T(s) \hm= 1$, то эту вершину вместе с инцидентным ей ребром
$(s,v) \hm\in E_T$ можно удалить из~$T$,
получив штейнеровское дерево меньшей или равной (при $c(s,v)\hm=0$) стоимости.

Рассмотрим случай $\mathrm{deg}_T(s) \hm\geq 2$. Пусть $v$ и $w$~---
две вершины, инцидентные $s$ на~$T$.
Применим процедуру {\sf Пре\-об\-ра\-зо\-вать\-Тре\-у\-голь\-ник}$(T,s,v,w)$. %\linebreak
В~ре\-зультате получится новое дерево Штейнера~$T'$,\linebreak также являющееся минимальным
(в силу теоремы~1). В~$T'$ степень вершины~$s$ на~1 меньше, чем~в~$T$.

Последовательным применением подобных преобразований получим дерево Штейнера~$T''$,
в котором степень~$s$ равна~1 и $c(T'') \hm\leq c(T)$. Но, удалив из~$T''$ вершину~$s$ вместе с
инцидентным ей ребром $(s,v)$, получим дерево Штейнера~$T^*$, для которого
$c(T^*) \hm\leq c(T'') \hm\leq c(T)$. Таким образом, $T^*$~---
минимальное дерево Штейнера для того же
множества терминальных вершин и $T^*$ содержит
на одну нетерминальную вершину (вершина~$s$) меньше,
чем  исходное дерево~$T$. Очевидно, что, применив подобное преобразование несколько раз,
можно получить минимальное дерево Штейнера, содержащее только терминальные вершины.

Рассмотрим случай, когда стоимости всех ребер на графе~$G$ строго положительны.
Тогда после удаления вершины~$s$ и ребра $(s,v)$ в соответствии с описанной выше процедурой
получим дерево Штейнера~$T^*$ такое, что $c(T^*) \hm< c(T)$.
А~это противоречит тому, что исходное дерево~$T$ является минимальным по стоимости деревом Штейнера.
Таким образом, методом от противного доказано, что $T$ не может содержать нетерминальные вершины.
Теорема~3 доказана.


\smallskip

Если все вершины графа являются терминальными, то минимальное дерево Штейнера
совпадает с
минимальным остовным деревом. Поэтому справедливо следующее утверждение.


\smallskip

\noindent
\textbf{Следствие.} Для графа с ультраметрической функцией стоимости
минимальное дерево Штейнера
является минимальным остовным деревом подграфа, порожденного множеством терминальных вершин.

%\vspace*{-6pt}

\section{Заключение}

%\vspace*{-2pt}

В данной работе проведен анализ ультрамет\-рических
версий нескольких задач, являющихся
NP-труд\-ны\-ми в общем случае. Для замкнутой за-\linebreak да\-чи коммивояжера приведен
полиномиальный \mbox{алгоритм} решения. Для задачи построения минимального дерева Штейнера показано, что в
ультрамет\-ри\-че\-ском случае решение совпадает с минимальным остовным деревом для подграфа и,
следова\-тельно, может быть построено за полиномиальное время одним из ранее известных алгоритмов.

\smallskip
Автор благодарит Б.\,Я.~Штейнберга за ценные замечания и предложения.

{\small\frenchspacing
{%\baselineskip=10.8pt
\addcontentsline{toc}{section}{References}
\begin{thebibliography}{99}


\bibitem{Gary_Johnson} %1
\Au{Гэри М., Джонсон Д. } Вычислительные машины и труднорешаемые
задачи~/ Пер. с англ.~--- М.: Мир, 1982. 416~с. (\Au{Garey~M.\,R.,
Johnson~D.\,S.} Computers and intractability: A~guide to the theory
of NP-completeness.~--- New York: W.\,H.~Freeman \& Co, 1979. 338~p.)


\bibitem{Farach} %2
\Au{Farach M., Kannan~S., Warnow~T.} A~robust model for finding
optimal evolutionary trees~// Algorithmica. Special Issue on
Computational Biology, 1995. Vol.~13. No.\,1. P.~155--179.

\bibitem{Gusfield} %3
\Au{Гасфилд Д. } Строки, деревья и последовательности в алгоритмах.
Информатика и вычислительная биология~/ Пер. с англ.~--- СПб.:
Невский Диалект, БХВ-Петербург, 2003. 654~с. (\Au{Gusfield~D.}
Algorithms on strings, trees and sequences: Computer \mbox{science} and
computational biology.~--- Cambridge: Cambridge University Press,
1997. 556~p.
{\sf http://www.cs. ucdavis.edu/$\sim$gusfield/ultraerrat/ultraerrat.html}.)

\bibitem{Moore} %4
\Au{Moore N.\,C.\,A., Proseer~P. } The ultrametric constraint and its
application to phylogenetics~// J.~Artif. Intell.
Res., 2008. Vol.~32. P.~901--938.
\bibitem{Haubold_Wiehe} %5
\Au{Хаубольд Б., Вие~Т. } Введение в вычислительную биоло\-гию.
Эволюционный подход~/ Пер. с англ.~--- М.--Ижевск: НИЦ <<Регулярная
и хаотическая динамика>>, Ижевский институт компьютерных
исследований, 2011. 424~с. (\Au{Haubold~B., Wiehe~T. } Introduction
to computational biology: An evolutionary approach.~--- 3rd ed.~---
Basel: Birkh$\ddot{\mbox{a}}$user, 2006. 328~p.)

\bibitem{Li} %6
\Au{Li X., Plaxton C.\,G., Tiwari~M., Venkataramani~A. } Online
hierarchical cooperative caching~// SPAA'04: 16th
Annual ACM Symposium on Parallelism in Algorithms and
Architectures Proceedings.~--- New York: ACM, 2004. P.~74--83.

\bibitem{Kintsel} %7
\Au{Кинцель~В. } Спиновые стекла как модельные системы для
нейронных сетей~// Успехи физических наук, 1987. Т.~152.
C.~123--131.

\bibitem{Christofides} %8
\Au{Кристофидес~Н. } Теория графов. Алгоритмический подход~/ Пер. с
англ.~--- М.: Мир, 1978. 432~с. (\Au{Christofides~N. } Graph
theory: An algorithmic approach (Computer science and applied
mathematics).~--- New York: Academic Press, 1975. 400~p.)

\bibitem{Meinika} %9
\Au{Майника~Э. } Алгоритмы оптимизации на сетях и графах~/ Пер. с
англ.~--- М.: Мир, 1981. 323~с. (\Au{Meinika~E.} Optimization
algorithms on networks and graphs.~--- New York\,--\,Basel: Dekker,
1978. 356~p.)

\bibitem{Bui}
\Au{Bui T.\,N., Zrncic~C.\,M. } An ant-based algorithm for finding
degree-constrained minimum spanning tree~// \mbox{GECCO'06}:
8th Annual Conference on Genetic and Evolutionary
Computation Proceedings.~--- New York: ACM, 2006. P.~11--18.

\end{thebibliography}
} }

\end{multicols}

\vspace*{-6pt}

\hfill{\small\textit{Поступила в редакцию 30.07.13}}

\newpage


%\vspace*{12pt}

%\hrule

%\vspace*{2pt}

%\hrule


\def\tit{ON POLYNOMIAL TIME COMPLEXITY OF~ULTRAMETRIC VERSIONS OF~CERTAIN NP-HARD PROBLEMS}

\def\titkol{On polynomial time complexity of~ultrametric versions of~certain
NP-HARD problems}

\def\aut{M.\,G.~Adigeev}
\def\autkol{M.\,G.~Adigeev}


\titel{\tit}{\aut}{\autkol}{\titkol}

\vspace*{-9pt}

\noindent
Southern Federal University, 105/42 Bol'shaya Sadovaya Str., Rostov-on-Don 344006,
Russian Federation



\def\leftfootline{\small{\textbf{\thepage}
\hfill INFORMATIKA I EE PRIMENENIYA~--- INFORMATICS AND APPLICATIONS\ \ \ 2014\ \ \ volume~8\ \ \ issue\ 2}
}%
 \def\rightfootline{\small{INFORMATIKA I EE PRIMENENIYA~--- INFORMATICS AND APPLICATIONS\ \ \ 2014\ \ \ volume~8\ \ \ issue\ 2
\hfill \textbf{\thepage}}}

\vspace*{6pt}


\Abste{The paper deals with important special cases of the travelling
salesman problem and the Steiner tree problem.
Both of these problems are NP-hard even in the metric case, i.\,e., for graphs whose
edge cost function meets the triangle inequality.
Even more severe restriction is imposed by the \textbf{strong } triangle inequality:
$
\forall x, y, z$\linebreak $\in X \quad c (x, z) \leq \max \{c (x, y), c (y, z) \}$.
The function which meets this inequality is called \textbf{ultrametric}.
The analysis of graphs with an ultrametric edge cost function is presented.
This analysis leads to an algorithm for building the minimal cost
Hamiltonian cycle
in time $ O \left(n^ 2 \right)$ where $n$ is the number of vertices.
For the Steiner tree problem, it is proven that in the
case of an ultrametric edge function,
 the minimum Steiner tree
includes only terminal vertices and thus may also be constructed in polynomial time,
as a minimum spanning tree on a subgraph of the original graph.}

\KWE{ultrametric function; strong triangle inequality;
travelling sales\-man problem; Steiner tree; polynomial-time algorithms}

\DOI{10.14357/19922264140207}

\Ack
\noindent
The author thanks B.\,Ya.~Shteinberg for his valuable comments and suggestions.


  \begin{multicols}{2}

\renewcommand{\bibname}{\protect\rmfamily References}
%\renewcommand{\bibname}{\large\protect\rm References}

{\small\frenchspacing
{%\baselineskip=10.8pt
\addcontentsline{toc}{section}{References}
\begin{thebibliography}{99}


\bibitem{Gary_Johnson-1} %1
\Aue{Garey, M.\,R., and D.\,S.~Johnson}. 1979.
\textit{Computers and intractability: A~guide to the theory of NP-completeness. }
New York: W.\,H.~Freeman \& Co.\ New York. 338~p.

\bibitem{Farach-1} %2
\Aue{Farach, M., S. Kannan, and T.~Warnow}. 1995.
A~robust model for finding optimal evolutionary trees.
\textit{Algorithmica}. Special Issue on Computational Biology.
13(1):155--179.

\bibitem{Gusfield-1} %3
\Aue{Gusfield, D.} 1997.
\textit{Algorithms on strings, trees and sequences: Computer science and
computational biology. } Cambridge:
Cambridge University Press. 556~p.
{\sf http:// www.cs.ucdavis.edu/$\sim$gusfield/ultraerrat/ultraerrat.\linebreak html}.


\bibitem{Moore-1} %4
\Aue{Moore, N.\,C.\,A., and P.~Proseer}. 2008.
The ultrametric constraint and its application to phylogenetics.
\textit{J.~Artif. Intell. Res.} 32:901--938.

\bibitem{Haubold_Wiehe-1} %5
\Aue{Haubold, B., and T.~Wiehe}. 2006.
\textit{Introduction to computational biology: An evolutionary approach.}
3rd ed. Basel: Birkh$\ddot{\mbox{a}}$user. 328~p.

\bibitem{Li-1} %6
\Aue{Li, X., C.\,G.~Plaxton, M.~Tiwari, and A.~Venkataramani}. 2004.
Online hierarchical cooperative caching.
\textit{16th Annual ACM Symposium on Parallelism in Algorithms and
Architectures (SPAA'04) Proceedings.}
New York. 74--83.

\bibitem{Kintsel-1}
\Aue{Kintsel' V.} 1987.
Spinovye stekla kak model'nye sistemy dlya neyronnykh setey.
[Spin glasses as model systems for neural networks].
\textit{Uspekhi Fizicheskikh Nauk}
[Advances in Physical Sciences] 152:123--131.

\bibitem{Christofides-1}
\Aue{Christofides, N.} 1975.
\textit{Graph theory: An algorithmic approach
(Computer science and applied mathematics).}
New York: Academic Press. 400~p.

\bibitem{Meinika-1}
\Aue{Meinika, E.} 1978.
\textit{Optimization algorithms on networks and graphs.}
New York\,--\,Basel: Dekker. 356~p.

\bibitem{Bui-1}
\Aue{Bui, T.\,N., and C.\,M.~Zrncic}. 2006.
An ant-based algorithm for finding degree-constrained minimum spanning tree.
\textit{GECCO'06: 8th Annual Conference on Genetic and Evolutionary Computation Proceedings.}
New York. 11--18.
{\looseness=1

}

\end{thebibliography}
} }


\end{multicols}

\vspace*{-6pt}

\hfill{\small\textit{Received July 30, 2013}}

\vspace*{-18pt}

\Contrl

\noindent
\textbf{Adigeev Mikhail G.} (b.\ 1973)~---
Candidate of Science (PhD) in technology, associate professor,
Southern Federal University, 105/42 Bol'shaya Sadovaya Str., Rostov-on-Don 344006,
Russian Federation; madi@math.sfedu.ru

 \label{end\stat}

\renewcommand{\bibname}{\protect\rm Литература}