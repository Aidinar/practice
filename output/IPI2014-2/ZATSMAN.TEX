\def\stat{zatsman}

\def\tit{ИНФОРМАЦИОННЫЕ ТЕХНОЛОГИИ КОРПУСНЫХ ИССЛЕДОВАНИЙ:
ПРИНЦИПЫ ПОСТРОЕНИЯ КРОССЛИНГВИСТИЧЕСКИХ БАЗ~ДАННЫХ$^*$}

\def\titkol{Информационные технологии корпусных исследований: принципы
построения кросслингвистических баз данных}

\def\aut{Н.\,В.~Бунтман$^1$, Анна A.~Зализняк$^2$, И.\,M.~Зацман$^3$,
М.\,Г.~Кружков$^4$, Е.\,Ю.~Лощилова$^5$, Д.\,В.~Сичинава$^6$}

\def\autkol{Н.\,В.~Бунтман, Анна A. Зализняк, И.\,M.~Зацман и др.} %, М.\,Г.~Кружков, Е.\,Ю.~Лощилова, Д.\,В.~Сичинава}

\titel{\tit}{\aut}{\autkol}{\titkol}

{\renewcommand{\thefootnote}{\fnsymbol{footnote}}
\footnotetext[1]{Работа выполнена в ИПИ РАН при поддержке фонда
<<Династия>> (грант NG13-036) и РФФИ (грант №\,13-06-00403).}}

\renewcommand{\thefootnote}{\arabic{footnote}}
\footnotetext[1]{Московский государственный университет
им.~М.\,В.~Ломоносова, факультет иностранных языков и регионоведения,
nabunt@hotmail.com}
\footnotetext[2]{Институт языкознания Российской академии наук; Институт
проблем информатики Российской академии наук,\\
anna.zalizniak@gmail.com}
\footnotetext[3]{Институт проблем информатики Российской академии наук,
izatsman@yandex.ru}
\footnotetext[4]{Институт проблем информатики Российской академии наук,
magnit75@yandex.ru}
\footnotetext[5]{Институт проблем информатики Российской академии наук,
lena0911@mail.ru}
\footnotetext[6]{Институт русского языка Российской академии наук,
mitrius@gmail.com}

\vspace*{-12pt}

\Abst{Рассматривается информационная технология создания
кросслингвистических баз данных текстов на русском языке и их переводов на
французский язык, называемых параллельными текстами. Разработанные
принципы построения этой базы данных обеспечивают реализацию
уникального сочетания трех видов двуязычного поиска: лексического,
грамматического и лексико-грамматического. Отличительной чертой
рассматриваемой технологии является одновременное формирование
рус\-ско-фран\-цуз\-ско\-го параллельного подкорпуса Национального корпуса русского
языка (НКРЯ) и кросслингвистической базы данных глагольных лек\-си\-ко-грам\-ма\-ти\-че\-ских форм русского языка и их функциональных эквивалентов во
французских переводах. Подкорпус и база данных обладают разной глубиной
выравнивания: в первом случае оно выполняется на уровне предложений, а во
втором~--- на уровне конструкций. Теоретическое значение создания
этой базы данных заключается в обеспечении исследований как в области
двуязычной контрастивной грамматики, так и в направлении создания
грамматики русского языка, опирающейся на современную эмпирическую базу
и информационные технологии корпусной лингвистики. Ее основное
прикладное назначение заключается в повышении качества машинного
перевода.}

\KW{параллельный корпус; информационная технология;
кросслингвистические базы данных; двуязычный лексико-грамматический
поиск; корпусная лингвистика; контрастивная грамматика}

\vspace*{-12pt}

\DOI{10.14357/19922264140210}

\vskip 10pt plus 9pt minus 6pt

\thispagestyle{headings}

\begin{multicols}{2}

\label{st\stat}

\section{Введение} %1

\vspace*{-3pt}

  Возникновение параллельных электронных корпусов ознаменовало начало
новой эры контрастивных лингвистических исследований; пионерскими в этой
области стали работы 1990-х гг.\ Стига Йоханссона с анг\-ло-нор\-веж\-ским
корпусом. Сочетание методов современной компьютерной лингвистики с
возможностями сопоставления текстов на двух и более языках,
предоставляемыми параллельными корпусами, обеспечило возможность
осуществления контрастивного лингвистического анализа на принципиально
новом уровне точности (ср.~\cite{zat-1}). Благодаря таким корпусам за
прошедшие два десятилетия в этой области были достигнуты значительные
успехи как в плане разработки методик анализа, так и в плане создания
оригинальных лексикографических описаний. О~перспективах контрастивных
грамматических исследований на базе параллельных корпусов см.\
работы~[1--7].
\columnbreak

  В работе~\cite{zat-8} была описана технология фор\-ми\-рования
  русско-французского параллельного\linebreak подкорпуса НКРЯ, содержащего литературные произведения на русском языке и их
переводы на французский язык.

В~настоящее время рус\-ско-фран\-цуз\-ский
подкорпус НКРЯ содержит тексты произведений совокупным объемом в 2~млн
словоупотреблений. При этом часть параллельных текстов представлена в
\textit{поливариантном формате}, т.\,е.\ одно произведение на русском языке
выровнено в корпусе по предложениям с \textit{несколькими вариантами его
перевода} на французский язык. Совокупный объем поливариантных
текстов~--- 700~тыс.\ словоупотреблений, т.\,е.\ больше трети всего
параллельного рус\-ско-фран\-цуз\-ско-рус\-ско\-го подкорпуса НКРЯ.

 Параллельные корпуса стали включаться в состав НКРЯ с 2005~г.~\cite{zat-2,
zat-4, zat-9, zat-10}. Сейчас он включает восемь двуязычных параллельных
подкорпусов\linebreak\vspace*{-12pt}

\pagebreak

\noindent
 с русским языком оригинала или перевода
(английский, немецкий, французский, испанский, италь\-ян\-ский, польский,
украинский и белорусский) и один многоязычный параллельный подкорпус.
Подкорпус параллельных текстов на русском и французском языках появился в
составе НКРЯ в декабре 2012~г. Технологию, используемую для формирования
этого подкорпуса и описанную ранее в работе~\cite{zat-8}, обозначим как
Parallel Corpus technology или ParCor-технология.

  В 2013~г.\ ParCor-технология была дополнена новыми операциями: была
создана база данных глагольных форм русского языка и вариантов их перевода
на французский язык (далее~--- БД) и сформирован поливариантный подкорпус
параллельных текстов на русском и французском языках (далее~--- подкорпус).
Новая технология дала возможность формировать БД одновременно с
пополнением подкорпуса и реализовать три вида двуязычного поиска
глагольных форм и их переводов: лексического, грамматического и
лек\-си\-ко-грам\-ма\-ти\-че\-ско\-го. Например, в этой БД можно задать и
выполнить запрос на поиск параллельных выровненных текстовых фрагментов,
в которых в русском оригинале употреблена глагольная форма прошедшего
времени несовершенного вида, а в параллельных французских фрагментах~---
pass$\acute{\mbox{e}}$ compos$\acute{\mbox{e}}$. Технологию,
ориентированную на одновременное формирование подкорпуса и БД с
двуязычным поиском параллельных глагольных форм в оригинальном и
переведенных текстах обозначим как Database Parallel Corpus technology или
DBParCor-технология.

  Цель статьи состоит в том, чтобы описать назначение, задачи и принципы
построения БД, формируемой на основе параллельных текстов НКРЯ, а также
функции двуязычного поиска, реализованные в этой БД.

\vspace*{-9pt}

\section{Назначение базы данных и~принципы ее~построения} %2

  Принципы построения БД во многом диктовались ее назначением. Она
создавалась как инст\-румент описания русской грамматической семантики <<в
зеркале французского языка>>, а также с целью уточнения положений
рус\-ско-фран\-цуз\-ской контрастивной грамматики. При выработке принципов
построения БД использовались работы Гака~[11, 12], Кузнецовой~[13],
Гиро-Вебер~[14] и~др. Эти работы, однако, появились в докорпусную
эпоху; теперь, когда созданы и регулярно пополняются
рус\-ско-фран\-цуз\-ские корпуса, стали доступны параллельные тексты в
цифровой форме, их со\-по\-ставление и анализ дает возможность уточнить
описание рус\-ско-фран\-цуз\-ской контрастивной грам\-ма\-тики.

  В~ходе разработки БД учитывалось то, что объектом анализа являются
соответствия глагольных категорий русского и французского языка в
параллельных текстах. Было определено несколько новых терминов, которые
отражают существо принципов построения БД.

  Ключевыми являются понятия <<лек\-си\-ко-грам\-ма\-ти\-че\-ская форма>>,
или ЛГФ, и <<базовый вид ЛГФ>>, определения которых даны ниже.

  \medskip

  \noindent
  \textbf{Определение~1.} Под \textit{лек\-си\-ко-грам\-ма\-ти\-че\-ской
формой} (ЛГФ) понимается совокупность элементов конкретного предложения,
обладающая набором признаков, задаваемых базовым видом ЛГФ.

  \medskip

  \noindent
  \textbf{Определение~2.} Под \textit{базовым видом ЛГФ} понимается
определенная комбинация значений следующих параметров:
  \begin{itemize}
\item категориальная принадлежность языковой единицы (в данной статье
рассматриваются только глагольные ЛГФ, т.\,е.\ значение этого параметра
фиксировано);
\item набор значений грамматических категорий, релевантных для
выбранного класса единиц;
\item (факультативно) определенные элементы структуры предложения,
задающие <<конструкцию>>; например: <<PastPF\;+\;\textit{если бы}>>.
\end{itemize}

  Другими словами, базовый вид ЛГФ пред\-став\-ля\-ет собой некоторую
комбинацию значений глагольных категорий, в совокупности с определенными
элементами структуры предложения задающую некоторую
<<конструкцию>>\footnote{В значении, которое придается этому понятию в
Грамматике конструкций~[15--17].}.

  В~процессе формирования БД было выделено 15 базовых видов ЛГФ
русского языка; это так называемое мно\-же\-ст\-во-ис\-точ\-ник
(табл.~1)\footnote{В текущую версию БД включены только те ЛГФ
русского языка, которые содержат глагол в финитной форме (т.\,е.\
исключались безличные глаголы, слова категории состояния, причастия,
деепричастия, а также перифразы с глаголом \textit{быть}). В~дальнейшем
состав рассматриваемых видов глагольных форм будет расширяться.}.
Количество базовых видов ЛГФ французского языка (мно\-же\-ст\-во-цель) не
фиксировано, поскольку оно возрастает по мере пополнения БД; в текущем
варианте БД оно составляет 25 единиц (табл.~2).



  Помимо базовых видов ЛГФ для каждого из двух языков сформировано
множество дополнительных признаков, которые позволяют специфицировать
тип конструкции. А~именно: дополнительные признаки характеризуют либо
состав глагольной груп-\linebreak\vspace*{-12pt}

\pagebreak

{\small %tabl1
%\vspace*{-12pt}

\noindent
{{\tablename~1}\ \ \small{Множество-ис\-точ\-ник базовых видов
глагольных ЛГФ русского языка}}
\vspace*{1pt}

\tabcolsep=8.5pt
\begin{center}
\begin{tabular}{|rl|l|}
\hline
 \multicolumn{2}{|c|}{\tabcolsep=0pt\begin{tabular}{c}Полное название \\
базового вида ЛГФ\end{tabular}}&
\multicolumn{1}{|c|}{\tabcolsep=0pt\begin{tabular}{c}
Сокращенное\\ обозначение\\ базового вида ЛГФ\end{tabular}}\\
 \hline
1.&Настоящее&Pres\\
2.&Прошедшее НСВ&Past-IPF\\
3.&Прошедшее СВ&Past-PF\\
4.&Простое будущее&Fut-PF\\
5.&Сложное будущее&Fut-IPF\\
6.&Императив СВ&Imperat-PF\\
7.&Императив НСВ&Imperat-IPF\\
8.&Форма с \textit{бы} СВ&Past-PF+\textit{бы}\\
9.&Форма с \textit{бы} НСВ&Past-IPF+\textit{бы}\\
10.&Форма с \textit{если бы} СВ&Past-PF+\textit{если бы}\\
11.&Форма с \textit{если бы} НСВ&Past-IPF+\textit{если бы}\\
12.&Форма с \textit{чтобы} СВ&Past-PF+\textit{чтобы}\\
13.&Форма с \textit{чтобы} НСВ&Past-IPF+\textit{чтобы}\\
14.&Форма с \textit{было} СВ&Past-PF+\textit{было}\\
15.&Форма с \textit{было} НСВ&Past-IPF+\textit{было}\\
  \hline
  \end{tabular}
  \end{center}
}

\vspace*{12pt}



  \addtocounter{table}{1}

\noindent
пы (например, наличие при глаголе подчиненного
инфинитива, модального детерминанта или отрицания), либо тип предложения,
в котором упо\-треб\-ле\-на данная ЛГФ (например, придаточное, вопросительное
предложение, диалогическая реплика) (табл.~3 и~4).
Каждый признак приложим или ко всем, или к некоторым из базовых видов
ЛГФ. На всех рисунках статьи дополнительные признаки указаны в квадратных
скобках после базового вида ЛГФ.

  \smallskip

  \noindent
  \textbf{Определение~3.} Комбинацию базового вида ЛГФ с одним или
несколькими из дополнительных признаков назовем \textit{видом ЛГФ}.



  Принципы установления соответствия в параллельных выровненных текстах
между русскими и французскими ЛГФ состоят в следующем. Сначала из фразы
русского оригинала вычленяется фрагмент, включающий ЛГФ, базовый вид
которой принадлежит множеству-источнику (см.\ табл.~1). Далее
ищется ее <<функционально эквивалентный фрагмент>>
(ФЭФ)\footnote{Термин <<функционально эквивалентный фрагмент>> введен в
работе~\cite{zat-2}, см. также~[9].} во французском переводе, из которого
извлекается ЛГФ, базовый вид которой принадлежит мно\-же\-ст\-ву-цели (см.\
табл.~2).



  Лексико-грамматическая форма русского языка и соответствующая ей ЛГФ французского языка
образуют \textit{мо\-но\-эк\-ви\-ва\-лен\-цию} (см.\ определение~4 и
табл.~5). Если в процессе анализа ФЭФ оказывается, что нужный
базовый вид французской ЛГФ в табл.~2 отсутствует, то мно\-же\-ст\-во-цель
может быть\linebreak\vspace*{-12pt}

{\small %tabl2
%\vspace*{-12pt}

\noindent
{{\tablename~2}\ \ \small{Множество-цель базовых видов ЛГФ французского языка}}
\vspace*{1pt}

\begin{center}
\begin{tabular}{|rl|l|}
\hline
 \multicolumn{2}{|c|}{\tabcolsep=0pt\begin{tabular}{c}Полное название \\
базового вида ЛГФ\end{tabular}}&
\multicolumn{1}{|c|}{\tabcolsep=0pt\begin{tabular}{c}Сокращенное\\
обозначение\\ базового вида ЛГФ\end{tabular}}\\
 \hline
1.&Pr$\acute{\mbox{e}}$sent&Pr\\
2.&Pass$\acute{\mbox{e}}$ compos$\acute{\mbox{e}}$&PasCom\\
3.&Pass$\acute{\mbox{e}}$ simple&PasSim\\
4.&Imparfait&Imparf\\
5.&Plus-que-parfait&PqParf\\
6.&Pass$\acute{\mbox{e}}$ ant$\acute{\mbox{e}}$rieur&PasAnt\\
7.&Pass$\acute{\mbox{e}}$ imm$\acute{\mbox{e}}$diat&PasIm\\
8.&Futur simple&Fut\\
9.&Futur ant$\acute{\mbox{e}}$rieur&FutAnt\\
10.&Futur imm$\acute{\mbox{e}}$diat&FutIm\\
11.&Imp$\acute{\mbox{e}}$ratif&Imperat\\
12.&Subjonctif pr$\acute{\mbox{e}}$sent&SubjPres\\
13.&Subjonctif pass$\acute{\mbox{e}}$&SubjPas\\
14.&Subjonctif imparfait&SubjImparf\\
15.&Subjonctif plus-que-parfait&SubjPqParf\\
16.&Conditionnel pr$\acute{\mbox{e}}$sent&CondPr\\
17.&Conditionnel pass$\acute{\mbox{e}}$&CondPas\\
18.&Participe pr$\acute{\mbox{e}}$sent&PartPr\\
19.&Participe pass$\acute{\mbox{e}}$&PartPas\\
20.&Participe pass$\acute{\mbox{e}}$ compos$\acute{\mbox{e}}$&PartPasComp\\
21.&G$\acute{\mbox{e}}$rondif&en PartPr\\
22.&Infinitif&Inf\\
23.&Pr$\acute{\mbox{e}}$position+infinitif&Prep+Inf\\
24.&Pr$\acute{\mbox{e}}$position+infinitif
pass$\acute{\mbox{e}}$&Prep+InfPas\\
25.&Substantif&Subst\\
  \hline
  \end{tabular}
  \end{center}
% \vspace*{9pt}
}
%\vspace*{4pt}

\addtocounter{table}{1}

\noindent
 пополнено. Случаи, когда для русской ЛГФ французский
эквивалент не найден, отмечаются в БД специальной пометой (Nondetermined),
и в процессе обработки данных они пока не учитываются\footnote{Речь идет о
таких случаях, когда семантическое содержание, заключенное в выбранной ЛГФ
оригинала, передано в переводе столь существенно иными лексическими средствами,
что установление соответствия между ЛГФ при
помощи того аппарата, который имеется на сегодня, оказывается невозможно.
Например: \textit{ты [\ldots] так теребишь за носы, что} {\bfseries\textit{еле
держатся}}~\textit{--- tu tirais tellement sur leur nez [\ldots] que}
{\bfseries\textit{tu as failli le leur arracher}}.}.



  Поиск ФЭФ и выявление содержащейся в нем ЛГФ французского языка
являются первой задачей, решение которой обеспечивается разработанным
вариантом БД. Для описания других задач БД, рассмотренных в следующем
разделе, определим еще пять терминов: <<моноэквиваленция>>, <<тип
моноэквиваленции>>, <<полиэквиваленция>>, <<тип полиэквиваленции>> и
<<гиперэквиваленция>>.

  \smallskip

  \noindent
  \textbf{Определение 4.} \textit{Моноэквиваленция} (МЭ)~--- это двухместный
кортеж вида $\langle R_n(i); F_m(j)\rangle$, где первую позицию занимает $i$-е
вхождение ЛГФ базового вида $R_n$ русского языка (см.\ табл.~1) в
оригинальном тексте. Вторую позицию занимает $j$-е вхождение ЛГФ
базового вида $F_m$ французского языка (см.\ табл.~2) в одном из
вариантов перевода $i$-го вхождения русской ЛГФ. Все МЭ, входящие в БД,
имеют идентификационный номер.

\end{multicols}

\begin{table}\small %tabl3
%\vspace*{-12pt}
\begin{center}
\Caption{Дополнительные признаки для базовых видов ЛГФ русского языка
\label{zat-t3}}
\vspace*{2ex}

\tabcolsep=8.5pt
\begin{tabular}{|p{200pt}|l|}
\hline
\multicolumn{1}{|c|}{\tabcolsep=0pt\begin{tabular}{c}Полное название\\
дополнительного признака\end{tabular}}&
\multicolumn{1}{c|}{\tabcolsep=0pt\begin{tabular}{c} Сокращенное
обозначение\\ дополнительного признака\end{tabular}}\\
\hline
Подчиненный инфинитив СВ&$[$SubInf-PF$]$\\
Подчиненный инфинитив НСВ&$[$SubInf-IPF$]$\\
Модальный детерминант&$[$ModDet$]$\\
Отрицание&$[$Neg$]$\\
Вопросительное предложение&$[$Interrog$]$\\
Восклицательное предложение&$[$Exclam$]$\\
Глагол, вводящий прямую речь&$[$VerbDirSp$]$\\
Глагол в составе диалогической реплики&$[$DialRepl$]$\\
Глагол в придаточном предложении&$[$Sub$]$\\
Глагол в изъяснительном придаточном&$[$SubCompl$]$\\
Глагол в определительном придаточном&$[$SubAttr$]$\\
  \hline
  \end{tabular}
  \end{center}
\vspace*{-11pt}
 %\end{table*}
%\begin{table*}\small %tabl4
%\vspace*{-6pt}
\begin{center}
\Caption{Дополнительные признаки для базовых видов ЛГФ французского
языка
\label{zat-t4}}
\vspace*{2ex}

\tabcolsep=8.5pt
\begin{tabular}{|p{200pt}|l|}
\hline
\multicolumn{1}{|c|}{\tabcolsep=0pt\begin{tabular}{c} Полное название\\
дополнительного признака\end{tabular}}&
\multicolumn{1}{c|}{\tabcolsep=0pt\begin{tabular}{c}Сокращенное
обозначение\\ дополнительного признака\end{tabular}}\\
\hline
Подчиненный инфинитив&$[$SubInf$]$\\
Подчиненный инфинитив прошедшего времени&$[$SubInfPas$]$\\
Добавление подчиняющего предиката&$[$+SuperPred$]$\\
Модальный детерминант&$[$ModDet$]$\\
Отрицание&$[$Neg$]$\\
Вопросительное предложение&$[$Interrog$]$\\
Восклицательное предложение&$[$Exclam$]$\\
Глагол, вводящий прямую речь&$[$VerbDirSp$]$\\
Глагол в составе диалогической реплики&$[$DialRepl$]$\\
Глагол в придаточном предложении &$[$Sub$]$\\
Глагол в изъяснительном придаточном&$[$SubCompl$]$\\
Глагол в определительном придаточном&$[$SubAttr$]$\\
Глагол в условном придаточном&$[$SubCond$]$\\
Accusativus cum infinitivo &$[$Acc.c.Inf$]$\\
Faire\;+\;Infinitif&[\textit{faire}\;+\;Inf]\\
Laisser\;+\;Infinitif&[\textit{laisser}\;+\;Inf]\\
Sembler\;+\;Infinitif&[\textit{sembler}\;+\;Inf]\\
Para$\hat{\mbox{\!\ptb{\i}}}$tre\;+\;Infinitif&
$[$$para\hat{\mbox{\!\!\ptb{\textit{\i}}}}tre $\;+\;Inf$]$\\
  \hline
  \end{tabular}
  \end{center}
  \vspace*{-11pt}
%  \end{table*}
 %  \begin{table*}\small %tabl5
\begin{center}
\Caption{Моноэквиваленция, зарегистрированная в БД под номером~4711}
\vspace*{2ex}

\begin{tabular}{|l|l|c|l|c|}
\hline
№ МЭ &\tabcolsep=0pt\begin{tabular}{c}ЛГФ\\ русского языка\end{tabular}&
\tabcolsep=0pt\begin{tabular}{c}Вид ЛГФ\\ русского языка\end{tabular}&
\multicolumn{1}{c|}{\tabcolsep=0pt\begin{tabular}{c}ЛГФ\\ перевода\end{tabular}}&
\tabcolsep=0pt\begin{tabular}{c}Вид ЛГФ\\ перевода\end{tabular}\\
\hline
4711 & \tabcolsep=0pt\begin{tabular}{l}потом [\ldots] плотно\\
\textbf{запер} все двери \end{tabular}&\tabcolsep=0pt\begin{tabular}{c}
 Past-PF\\ $[$ModDet$]$\end{tabular} &
\tabcolsep=0pt\begin{tabular}{l}apr{\!\ptb\`{e}}s \textbf{avoir} bien\\
\textbf{ferm}$\acute{\mbox{\textbf{e}}}$ toutes les\\ portes\end{tabular} &
\tabcolsep=0pt\begin{tabular}{c} Prep\;+\;InfPas\\ $[$Sub$]$\end{tabular}\\
\hline
\end{tabular}
\end{center}
\vspace*{-8pt}
\end{table}

\begin{multicols}{2}


%  \smallskip

  \noindent
  \textbf{Определение 5.} \textit{Типом моноэквиваленции} называется кортеж
базовых видов ЛГФ русского и французского языка $\langle R_n; F_m\rangle$,
например $\langle$Past-PF; Prep\;+\;InfPas$\rangle$ (см.\ 3-й и 5-й столбцы в
табл.~5).

  \medskip

  \noindent
  \textbf{Определение 6.} \textit{Полиэквиваленция}~--- это двухместный
кортеж вида $\langle R_n(i); \{F_m(j), F_k (r), \ldots\}\rangle$, представляющий
собой объединение нескольких моноэквиваленций с идентичной первой
позицией ($\langle R_n(i); F_m(j)\rangle$, $\langle R_n(i); F_k(r)\rangle$ и т.\,д.),
отражающих разные варианты перевода одного и того же $i$-го вхождения
ЛГФ базового вида $R_n$ в русском оригинальном тексте: $F_m(j)$~--- это ЛГФ
французского языка, идентифицированная в первом переводе и
соответствующая $i$-му вхождению русской ЛГФ, $F_k(r)$~--- во втором
переводе и т.\,д.\ (табл.~6).

  \medskip

  \noindent
  \textbf{Определение 7.} \textit{Типом полиэквиваленции} называется кортеж
базовых видов ЛГФ русского и французского языка $\langle R_n; \{F_m, F_k,
\ldots\}\rangle$, например $\langle$Pres-IPF; \{Pr,Pr\}$\rangle$ (см.\ 2-й и 5-й столбцы
в  табл.~6).



\begin{table*}\small % tabl6
\begin{center}
\Caption{Две моноэквиваленции (№№\,596, 5927), составляющие
полиэквиваленцию$^*$}
\vspace*{2ex}

\begin{tabular}{|l|l|c|l|c|}
\hline
\multicolumn{1}{|c|}{\raisebox{-18pt}[0pt][0pt]
{\tabcolsep=0pt\begin{tabular}{c}ЛГФ\\ русского\\ языка\end{tabular}}}&
\multicolumn{1}{c|}{\raisebox{-18pt}[0pt][0pt]
{\tabcolsep=0pt\begin{tabular}{c}Вид ЛГФ\\ русского языка\end{tabular}}}&
\multicolumn{3}{c|}{ЛГФ в текстах французских переводов и их виды}\\
\cline{3-5}
& & \tabcolsep=0pt\begin{tabular}{c}Номер\\ моноэкви-\\валенции \end{tabular}&
\multicolumn{1}{c|}{\tabcolsep=0pt\begin{tabular}{c}ЛГФ в текстах\\ французских\\ переводов\end{tabular}} &
\tabcolsep=0pt\begin{tabular}{c}Вид ЛГФ\\ французского\\ языка\end{tabular}\\
\hline
\multicolumn{1}{|l|}{\raisebox{-18pt}[0pt][0pt]{
\tabcolsep=0pt\begin{tabular}{l}Я иногда в театр\\ \textbf{хожу}\end{tabular}}} &
\multicolumn{1}{c|}{\raisebox{-18pt}[0pt][0pt]{\tabcolsep=0pt\begin{tabular}{l}Pres-IPF\\ $[$ModDet$]$\\ $[$DialRepl$]$\end{tabular} }}&
596 & \tabcolsep=0pt\begin{tabular}{c}\textbf{Il m'arrive d'aller} au\\
th$\acute{\mbox{e}}$$\hat{\mbox{a}}$tre,\end{tabular} &
\tabcolsep=0pt\begin{tabular}{l}Pr\\ $[$SubInf$]$\\ $[$+SuperPred$]$\\
$[$DialRepl$]$\end{tabular}\\
 \cline{3-5}
&&5927&
\tabcolsep=0pt\begin{tabular}{l}Non, je \textbf{vais parfois} au\\  th$\acute{\mbox{e}}$$\hat{\mbox{a}}$tre,
et en visite.\end{tabular}
& \tabcolsep=0pt\begin{tabular}{l} Pr\\ $[$ModDet$]$\\ $[$DialRepl$]$\end{tabular}
\\
\hline
\multicolumn{5}{p{126mm}}{\footnotesize \hspace*{4mm}$^*$Французские ЛГФ, входящие в данную
полиэквиваленцию, имеют одинаковый базовый вид, но различаются на уровне
дополнительных признаков, указанных в квадратных скобках.}
\end{tabular}
\end{center}
\end{table*}


\pagebreak



  \noindent
  \textbf{Определение 8.} \textit{Гиперэквиваленция}~--- это двухместный
кортеж вида $\langle R_n; \{F\}\rangle$, репрезентирующий соответствие между
базовым видом ЛГФ русского языка $R_n$ и множеством базовых видов
эквивалентных ЛГФ французского языка, входящих во вторую позицию
моноэквиваленций БД с ЛГФ базового вида $R_n$.

  Другими словами, каждая гиперэквиваленция включает один базовый вид
ЛГФ русского языка $R_n$ и список базовых видов ЛГФ французского языка~---
при условии, что хотя бы одна ЛГФ базового вида из этого списка образовала в
БД моноэквиваленцию с русской ЛГФ базового вида $R_n$.

  Используя определенные выше термины, перечислим те задачи, для решения
которых предназначена спроектированная БД:
  \begin{itemize}
\item построение моно-, поли- и гиперэквиваленций;
\item двуязычный лексический, грамматический и лексико-грамматический
поиск моно- и полиэквиваленций;
\item вычисление частотности для каждого типа моно- или
полиэквиваленций.
  \end{itemize}

  Для решения этих задач был разработан веб-интерфейс, который позволяет
поль\-зо\-ва\-те\-лям-линг\-вис\-там взаимодействовать с БД в он\-лайн-ре\-жи\-ме с
помощью распространенных веб-брау\-зе\-ров (Internet Explorer, Mozilla Firefox,
Google Chrome). Для создания и ведения БД используется СУБД Microsoft SQL
Server.

  Функции БД можно разделить на две основные группы:
\begin{enumerate}[(1)]
\item первая группа
функций служит для по\-стро\-ения и редактирования моноэквиваленций (см.\
рис.~1 для функции редактирования);
\item вторая группа функций~---
для поиска уже по\-стро\-ен\-ных моно- и полиэквиваленций (см.\ рис.~2 с
интерфейсом поиска и просмотра полиэквиваленций).
\end{enumerate}
 Группа функций
построения и редактирования моноэквиваленций в БД позволяет
отфильтровывать выровненные фрагменты оригинального и переводных
текстов по названию книги, автору перевода и присутствующих в этих
фрагментах видам ЛГФ. Используя эти функции, поль\-зо\-ва\-тель-лин\-г\-вист может
просматривать выровненные фрагменты параллельных текстов с целью
формирования моноэквиваленций.

  На начало 2014~г.\ построено 10\,527 моноэквиваленций и на их основе
автоматически было сгенерировано 4128 полиэквиваленций (т.\,е.\ объединений
моноэквиваленций из разных переводов \mbox{одного} оригинального текста с одной и
той же ЛГФ русского языка в первой позиции кортежа).

\vspace*{-6pt}

\section{Двуязычный поиск} %3

  На странице поиска и просмотра полиэквиваленций пользователи БД могут
видеть подборки полиэквиваленций (см.\ рис.~\ref{zat-f4}), которые генерируются в
соответствии с поисковым запросом. Пользователи БД могут осуществлять
поиск моно- и полиэквиваленций, используя следующие поисковые признаки:
название русского произведения, французский перевод, базовые виды и
признаки ЛГФ русского и французского языка, лексемы оригинала и переводов,
искомые тексты как последовательности знаков, включая знаки препинания
(ср.\ опцию <<поиск точных форм>> в НКРЯ).

\begin{figure*} %fig1
\vspace*{1pt}
\begin{center}
\mbox{%
\epsfxsize=140mm
\epsfbox{zac-3.eps}
}
\end{center}
\vspace*{-9pt}
\Caption{Интерфейс для редактирования моноэквиваленций
\label{zat-f3}
}
\end{figure*}

\begin{figure*} %fig2
\vspace*{1pt}
\begin{center}
\mbox{%
\epsfxsize=140mm
\epsfbox{zac-4.eps}
}
\end{center}
\vspace*{-9pt}
\Caption{Интерфейс для поиска и просмотра полиэквиваленций
\label{zat-f4}
}
\end{figure*}

\begin{table*}\small %fig7
\begin{center}
\Caption{Две полиэквиваленции, найденные в БД по поливариантному
двуязычному грамматическому запросу
\label{zat-f5}
}
\vspace*{2ex}

\tabcolsep=10pt
\begin{tabular}{|c|c|c|l|c|}
\hline
\multicolumn{2}{|c|}{Полиэквиваленция} &\multicolumn{1}{c|}{Номер МЭ} &
\multicolumn{1}{c|}{ЛГФ перевода}&\tabcolsep=0pt\begin{tabular}{c}Базовый вид\\
ЛГФ перевода\end{tabular}\\
\hline
\multicolumn{1}{|c|}{\raisebox{-12pt}[0pt][0pt]{
\tabcolsep=0pt\begin{tabular}{l}он \textbf{решил} \textbf{оставить} $[$\ldots$]$\\
липовые и дубовые деревья \end{tabular}}}&
\multicolumn{1}{c|}{\raisebox{-12pt}[0pt][0pt]{\tabcolsep=0pt\begin{tabular}{c}Past-PF\\ $[$SubInf$]$ \end{tabular}}}& 2931 &
\tabcolsep=0pt\begin{tabular}{l}alors qu'\textbf{il garderait}
les $[$\ldots$]$\\ tilleuls et ch$\hat{\mbox{e}}$nes, \end{tabular} &
CondPr\\
 \cline{3-5}
&& 8011 &\tabcolsep=0pt\begin{tabular}{l}
Il \textbf{d$\acute{\mbox{\textbf{e}}}$cida de laisser}
tels quels les\\ $[$\ldots$]$ tilleuls et les ch$\hat{\mbox{e}}$nes,\end{tabular} &
\tabcolsep=0pt\begin{tabular}{c}PasSim\\ $[$SubInf$]$\end{tabular}\\
\hline
\multicolumn{1}{|c|}{\raisebox{-12pt}[0pt][0pt]{
\tabcolsep=0pt\begin{tabular}{l}он \textbf{решил} $[$\ldots$]$
яблони\\ и груши \textbf{уничтожить}\end{tabular} }}&
\multicolumn{1}{c|}{\raisebox{-12pt}[0pt][0pt]{
\tabcolsep=0pt\begin{tabular}{c}Past-PF\\ $[$SubInf$]$\end{tabular}}}&
2932 &
\tabcolsep=0pt\begin{tabular}{l} il \textbf{se d}$\acute{\mbox{\textbf{e}}}$\textbf{barrasserait}
 des pommiers\\ et des poiriers\end{tabular} & CondPr\\
 \cline{3-5}
 &&8013 & \tabcolsep=0pt\begin{tabular}{l} Il
 \textbf{d}$\acute{\mbox{\textbf{e}}}$\textbf{cida} $[$\ldots$]$
 \textbf{de supprimer} les\\ pommiers et les poitiers \end{tabular} &
 \tabcolsep=0pt\begin{tabular}{c} PasSim\\ $[$SubInf$]$\end{tabular}\\
\hline
\end{tabular}
\end{center}
\vspace*{-6pt}
\end{table*}

\begin{table*}[b]\small %tabl8
\vspace*{-12pt}
\begin{center}
\Caption{ Три полиэквиваленции, найденные в БД по видам ЛГФ
\label{zat-f6}
}
\vspace*{2ex}

\begin{tabular}{|c|c|c|l|l|}
\hline
\multicolumn{2}{|c|}{Полиэквиваленция}&\multicolumn{1}{c|}{Номер МЭ} &
\multicolumn{1}{c|}{ЛГФ переводов}&\tabcolsep=0pt\begin{tabular}{c}
Вид ЛГФ\\ переводов\end{tabular}\\
\hline
\multicolumn{1}{|l|}{\raisebox{-18pt}[0pt][0pt]{\tabcolsep=0pt\begin{tabular}{l}
\textbf{Не может} \textbf{постараться}\\ для барина!\end{tabular} }}&
\multicolumn{1}{c|}{\raisebox{-18pt}[0pt][0pt]{\tabcolsep=0pt\begin{tabular}{l}
Pres \\ $[$SubInf-PF$]$\\ $[$Neg$]$\end{tabular}}} &
661 &
\tabcolsep=0pt\begin{tabular}{l}
\textbf{Tu pourrais} tout de m$\hat{\mbox{e}}$me \textbf{faire un}\\
\textbf{effort} pour ton ma$\hat{\mbox{\!\ptb{\i}}}$tre!
\end{tabular} &
\tabcolsep=0pt\begin{tabular}{l} CondPr\\ $[$SubInf$]$\\ $[$Exclam$]$\end{tabular}\\
\cline{3-5}
&& 5897 & \tabcolsep=0pt\begin{tabular}{l}Il \textbf{ne peut} m$\hat{\mbox{e}}$me
\textbf{pas faire} un petit effort\\ pour son ma$\hat{\mbox{\!\ptb{\i}}}$tre!\end{tabular} &
\tabcolsep=0pt\begin{tabular}{l} Pr$\acute{\mbox{e}}$sent \\ $[$SubInf$]$\\ $[$Exclam$]$\end{tabular}\\
\hline
\multicolumn{1}{|l|}{\raisebox{-18pt}[0pt][0pt]{
\tabcolsep=0pt\begin{tabular}{l}теперь \textbf{можете} \textbf{отдать }
\end{tabular}}} &
\multicolumn{1}{c|}{\raisebox{-18pt}[0pt][0pt]{
\tabcolsep=0pt\begin{tabular}{l} Pres\\ $[$SubInf-PF$]$\end{tabular}}} &945 &
\tabcolsep=0pt\begin{tabular}{l}maintenant vous \textbf{pouvez} me\\
\textbf{rembourser}.\end{tabular}  &
\tabcolsep=0pt\begin{tabular}{l}Pr$\acute{\mbox{e}}$sent\\ $[$SubInf$]$\end{tabular}
\\
\cline{3-5}
&& 7584 & \tabcolsep=0pt\begin{tabular}{l}alors vous \textbf{pourriez}
peut-$\hat{\mbox{e}}$tre \textbf{me}\\
\textbf{rembourser}?\end{tabular}  &
\tabcolsep=0pt\begin{tabular}{l}
CondPr\\ $[$SubInf$]$\\ $[$ModDet$]$ \\ $[$Interrog$]$\end{tabular}
\\
\hline
\multicolumn{1}{|l|}{\raisebox{-18pt}[0pt][0pt]{\tabcolsep=0pt\begin{tabular}{l}Разве я \textbf{могу}
все это $[$\ldots$]$\\ \textbf{перенести}?\end{tabular}}}  &
\multicolumn{1}{l|}{\raisebox{-18pt}[0pt][0pt]{\tabcolsep=0pt\begin{tabular}{l}
Pres\\ $[$SubInf-PF$]$\\ $[$Interrog$]$\end{tabular} }}&
8939 & Est-ce que je \textbf{puis} $[$\ldots$]$
le \textbf{supporter?}&
\tabcolsep=0pt\begin{tabular}{l}Pr$\acute{\mbox{e}}$sent\\ $[$SubInf$]$\\
$[$Interrog$]$\end{tabular}\\
\cline{3-5}
&& 8940 & Je \textbf{pourrais} $[$\ldots$]$ \textbf{supporter} tout
$\mbox{\ptb{\c{c}}}$a? &
\tabcolsep=0pt\begin{tabular}{l} CondPr\\ $[$SubInf$]$\\ $[$Interrog$]$
\end{tabular}
\\
\hline
\end{tabular}
\end{center}
\end{table*}

  Поисковые признаки можно задавать как по отдельности, так и в сочетании.
В~результате выполнения поискового запроса можно узнать число найденных
полиэквиваленций, удовлетворяющих заданным поисковым признакам, и
посмотреть их.

  Принципиально новой является функция двуязычного грамматического
поиска, который применим как к одному, так и одновременно к нескольким
переводам (поливариантный двуязычный запрос).
%
Например, если задать один
базовый вид ЛГФ русского языка <<Past-PF>> и соответстующие в двух
вариантах перевода два базовых вида ЛГФ французского языка CondPr и
PasSim, то в БД будут найдены две полиэквиваленции с заданными
поисковыми грамматическими признаками, отраженные в табл.~7.



  Кроме базового вида ЛГФ в поисковом запросе могут задаваться также
дополнительные признаки для ЛГФ русского языка (из табл.~\ref{zat-t3}) и для
ЛГФ французского языка (из табл.~\ref{zat-t4}).

Например, если задать
поисковый запрос <<Pres [SubInf-PF]>> в русской фразе и <<CondPr [SubInf]>>
хотя бы в одном из двух ее переводов, то в БД будут найдены три
полиэквиваленции с заданными видами ЛГФ (при этом в найденных
полиэквиваленциях могут присутствовать и другие дополнительные признаки)
(табл.~8).

  Приведенные примеры двуязычного грамматического поиска говорят о том,
что разработанная БД является на сегодняшний день уникальным
лингвистическим ресурсом, который может быть использован для
исследования не только глагольных форм, но и более широкого спектра
языковых единиц (см.\ разд.~4).

\vspace*{-12pt}

\section{Исследование лингвоспецифичной лексики с~помощью базы данных} %4

  В настоящее время DBParCor-технология и БД адаптируются к
исследованию лингвоспецифичных единиц (ЛСЕ) русского языка <<в зеркале
иностранных языков>>, включая французский. Для решения задач проекта
<<Контрастивное корпусное исследование специфических черт семантической
системы русского языка>>, финансируемого по гранту РФФИ, была
разработана оригинальная методология контрастивного корпусного анализа,
осуществляемого с помощью БД, которая опирается на концептуальный
аппарат контрастивного анализа русских лек\-си\-ко-грам\-ма\-ти\-че\-ских
форм, описанный выше. Данная методология предусматривает:
  \begin{itemize}
\item статистическое и/или экспертное обоснование гипотез
лингвоспецифичности лексических единиц русского языка на основе анализа
текстов двуязычных корпусов, БД и других текстовых источников;
\item статистическое обоснование с помощью БД гипотез
лингвоспецифичности лексических единиц русского языка,
сформулированных в ходе предшествующих исследований на основании
семантического анализа;
\item статистическую и экспертную верификацию гипотез с использованием
БД.
  \end{itemize}

  Если для статистического обоснования гипотез могут использоваться разные
информационные ресурсы (книги, корпуса или БД), то для верификации
гипотез используется только БД, так как ключевым этапом верификации
является по\-стро\-ение моно- и полиэквиваленций и вычисление час\-тот\-ности их
типов. Построение моно- и полиэквиваленций позволяет документировать
процесс статистической верификации гипотез, а также согласовывать и
документировать результаты верификации, выполненной лингвис\-та\-ми-экс\-пер\-та\-ми (экспертная верификация гипотез).

  Проведенные эксперименты по формированию, обоснованию,
статистической и экспертной верификации гипотез с помощью БД показали,
что для их верификации потребуется увеличить ее \mbox{объем}.

  Разработанная методика построения статистической и экспертной
верификации гипотез основана на использовании количественного
статистического и качественного экспертного методов. Суть статистического
метода заключается в следующем. Для каждой языковой единицы из списка
потенциально ЛСЕ русского языка определяется чис\-ло
ее переводных эквивалентов в тексте переводов, имеющихся в параллельном
корпусе, вычисляются час\-тот\-ности переводных эквивалентов и их разброс. Для
определения чис\-ла переводных эквивалентов могут использоваться книжные
источники, корпуса и БД: линг\-вист-экс\-перт анализирует причины разброса
частотности переводных эквивалентов и отбрасывает те случаи, когда разброс
не связан с лингвоспецифичностью (в частности, он может быть обусловлен
различием в способе лексикализации, например русскому \textit{плавать} в
английском языке соответствует три разных глагола, обозначающих три
различных вида плавания: \textit{swim}, \textit{sail}, \textit{float}). Оставшиеся
статистически выявленные лексические единицы считаются гипотетически
лингвоспецифичными.

  Разработанная методика включает стадию уточнения степени
лингвоспецифичности языковой единицы. На этой стадии, в частности,
проводится анализ условий появления рассматриваемой единицы русского
языка в обратных переводах (т.\,е.\ множество <<стимулов>> перевода на
русский язык). Чем больше таких стимулов, тем больше вероятность
лингвоспецифичности рассматриваемой единицы.

  Категория лингвоспецифичных слов находится в отношении пересечения с
категорией безэквивалентной лексики, т.\,е.\ имеется множество языковых
единиц, относящихся к обеим категориям, но есть и непересекающиеся
области. Статистические методы применимы только к тем лингвоспецифичным
словам, которые относятся также к категории безэквивалентной лексики. Если
слово не принадлежит к этой категории, то применяется качественный
экспертный метод построения гипотезы, который включает детальный
сопоставительный семантический анализ рассматриваемой лексической
единицы и его переводного эквивалента и выявление возможных расхождений
в составе компонентов, формирующих их семантическую структуру.

  Верификация гипотез лингвоспецифичности языковых единиц выполняется
лингвистами-экс\-пер\-та\-ми с использованием БД, сформированной на основе
параллельных текстов двуязычного корпуса. Каждая построенная в БД
моноэквиваленция включает гипотетически лингвоспецифичную лексическую
единицу русского языка и один из ее переводных эквивалентов, найденных в
параллельных текстах. (На данном этапе ограничим исследуемый материал, с
одной стороны, переводами на французский язык и с другой~---
лингвоспецифичными личными глагольными формами; в дальнейшем
исследовательская база будет расширена в обоих направлениях: по числу
языков и по спектру конструкций).

  Если сопоставительный статистический и семантический анализ
гипотетически лингвоспецифичной лексической единицы и ее эквивалентов
позволяет линг\-ви\-сту-экс\-пер\-ту выявить специфический смысловой
компонент, присутствующий в русском слове и отсутствующий в переводе, то
лингвоспецифичность этой единицы признается им верифицированной.
Особую ценность для нужд семантического анализа представляют
полиэквиваленции, которые предоставляют в распоряжение эксперта данные о
границах вариативности перевода интересующей его единицы в контексте,
зафиксированном в полиэквиваленции.

  Так, в двух переводах фразы \textit{Давно собирался к тебе} из
БД (рис.~\ref{zat-f7}), оба французских глагола
\textit{s'appr}$\hat{\mbox{\textit{e}}}$\textit{ter} и
\textit{se pr$\acute{\mbox{e}}$parer} (буквально
`готовиться') не содержат специфического смыслового компонента
неконтролируемости, заключенного в русском глаголе \textit{собираться}
(см.~\cite{zat-18}) и, наоборот, усиливают, по сравнению с оригиналом, семы
приготовления и прилагаемых усилий.

\begin{figure*} %fig3
\begin{center}
\tabcolsep=9.5pt
\begin{tabular}{|l|l|}
\hline
\multicolumn{1}{|c|}{\raisebox{-6pt}[0pt][0pt]
{Давно \textbf{собирался} к тебе,~--- }}& Depuis longtemps je
\textbf{m'appr}$\hat{\mbox{\textbf{e}}}$\textbf{tais} {\ptb{\`{a}}} te rendre visite.\\
\cline{2-2}
&Il y a d$\acute{\mbox{e}}$j{\!\ptb{\`{a}}} longtemps que je
\textbf{me pr}$\acute{\mbox{\textbf{e}}}$\textbf{parais} {\ptb{\`{a}}} venir te voir,\\
\hline
\end{tabular}
\end{center}
\vspace*{-6pt}
\Caption{Лингвоспецифичная единица \textit{собираться}
\label{zat-f7}
}
\end{figure*}

  База данных предоставляет в распоряжение пользователя практически полный спектр
семантических компонентов, составляющих ту сложную концептуальную
конфигурацию, которая заключена, например, в русском глаголе \textit{успеть}
(ср.~[18--20]), что позволяет уточнить проведенный ранее анализ этого
лингвоспецифичного слова (рис.~\ref{zat-f8}). Сопоставление с двумя
французскими переводами, где использовано выражение со словом `время',
особенно ясно выявляет эти дополнительные семы, определяющие
лингвоспецифичный характер данного русского глагола. В русском оригинале
речь идет не столько о возможной нехватке времени, сколько о наклонности и
способности, с одной стороны, и о случайности и удаче~--- с другой; кроме
того, в русском глаголе имеется отсутствующая во французском переводном
эквиваленте оценочная сема (ср.\ существительное \textit{успех}).

\begin{figure*} %fig4
\begin{center}
\begin{tabular}{|l|l|}
\hline
\multicolumn{1}{|c|}{\raisebox{-6pt}[0pt][0pt]
{Когда это он \textbf{успел} опять лечь-то}} &
Quand est-ce qu'il  \textbf{a trouv}$\acute{\mbox{\textbf{e}}}$ \textbf{le temps} de se recoucher
\\
\cline{2-2}
&Mais, comment \textbf{a}-t-il \textbf{eu le temps} de se recoucher?\\
\hline
\end{tabular}
\end{center}
\vspace*{-6pt}
\Caption{Глагол \textit{успеть}
\label{zat-f8}
}
\end{figure*}

  К каждой единице предварительного списка ЛСЕ
русского языка применяется процедура анализа, включающая следующие шаги:
  \begin{itemize}
  \item формирование и выполнение поискового запроса по данной лексеме,
который позволяет выявить в БД все включающие ее моно- и
полиэквиваленции;
\item анализ грамматической составляющей полученных моно- и
полиэквиваленций, в том числе статистическое распределение реально
встречающихся в узусе грамматических форм;
\item анализ лексической составляющей полученных моно- и
полиэквиваленций, в том числе статистические параметры выбора
переводного эквивалента;
\item интерпретация полученных результатов с точки зрения семантического
анализа исходной единицы русского языка и оценки степени ее
лингвоспецифичности.
  \end{itemize}

  Разработанная методология контрастивного корпусного анализа
специфических черт семантической системы русского языка предполагает
использование уже имеющейся БД глагольных форм, а также построение
лексических моноэквиваленций, что планируется осуществить в дальнейшем.
При этом базовые виды лексических моноэквиваленций будут определяться
включенными в них ЛСЕ.

\section{Заключение} %5

  Сформированная БД позволила уточнить ряд положений рус\-ско-фран\-цуз\-ской
  контрастивной грамматики. В частности, список соответствий,
описанных в работах~\cite{zat-11, zat-12} и частично суммированных в
работе~\cite{zat-13}:
  \begin{itemize}
  \item инвертирован (в работах Гака и Кузнецовой материал
рассматривается в направлении от французского к русскому, так как конечной
целью там является интерпретация значения и функции форм французского
языка);
  \item существенно расширен, т.\,е.\ установлены новые типы переводных
соответствий;
  \item подвергнут статистической оценке.
  \end{itemize}

  Особый интерес представляют полученные результаты частотного анализа
переводных соответствий. В частности, корреляция между оппозициями <<совершенный
\textit{vs.}\ несовершенный вид>> в русском языке и <<passe
compos$\acute{\mbox{e}}$/passe simple \textit{vs.}\ imparfait>> во французском
может быть уточнена на основе количественных показателей: базовому виду
русской ЛГФ Past-IPF лишь в 49,4\% случаев соответствует базовый вид
французской ЛГФ Imparf и в 21\% случаев~--- PasCom/PasSim; особенно
значимой представляется последняя цифра, отражающая широту
семантического диапазона русского несовершенного вида.

  Разработанная методология, DBParCor-тех\-но\-ло\-гия и созданная БД,
сформированная на основе выровненных текстов поливариантного
параллельного корпуса, позволили также уточнить семантику русских
глагольных форм: варианты перевода на французский язык, обладающий более
детализированной сеткой грамматических противопоставлений в области
темпорально-модальных значений, выявляют определенные семантические
компоненты, заключенные в значении русских глагольных форм.

  В заключение отметим, что разработанная DBParCor-технология может быть
адаптирована для использования в других кросслингвистических проектах,
целью которых является приведение в соответствие знаний о русском языке
современному состоянию лингвистической теории и эмпирической базе,
представленной современными электронными корпусами, с одной стороны, и, с
другой стороны, потребностям современной системы образования, а также
требованиям, предъявляемым новыми информационными технологиями
машинного перевода. Необходимость использования кросслингвистических
моделей для разработки технологий машинного перевода была обоснована в
работах~[21--23].

  DBParCor-технология может быть использована в проектах, посвященных
изучению на базе параллельных выровненных текстов лексико-грам\-ма\-ти\-че\-ских
форм других категорий без изменения структуры БД или с небольшими ее
изменениями. Для адаптации DBParCor-технологии нужно сформировать
перечень используемых языков, определить списки базовых видов ЛГФ и их
дополнительных признаков для языков оригинала и перевода в соответствии с
целями конкретных проектов.

  {\small\frenchspacing
  {%\baselineskip=10.8pt
  \addcontentsline{toc}{section}{Литература}
  \begin{thebibliography}{99}
\bibitem{zat-1}
\Au{Aijmer~K., Altenberg~B.}
 Advances in corpus-based contrastive linguistics. Studies in honour of Stig
Johansson.~--- Amsterdam: John Benjamins, 2013. 295~p.
\bibitem{zat-2}
\Au{Добровольский~Д.\,О., Кретов~А.\,А., Шаров~С.\,А.}
 Корпус параллельных текстов~// Научная и техническая информация. Сер.~2.
Информационные процессы и системы, 2005. №\,6. С.~16--27.
\bibitem{zat-3}
Корпусные исследования по русской грамматике~/
Под ред.\ К.\,Л.~Киселевой, Е.\,В.~Рахилиной,
В.\,А.~Плунгяна, С.\,Г.~Татевосова.~--- М.: Пробел-2000,
2009. 516~с.
\bibitem{zat-4}
\Au{Добровольский~Д.\,О.}
 Корпус параллельных текстов в исследовании культурно-специфичной
лексики~// Национальный корпус русского языка: 2006--2008. Новые
результаты и перспективы.~--- СПб.: Нестор-Ис\-то\-рия, 2009. С.~383--401.

\bibitem{zat-6} %5
\Au{Сичинава~Д.\,В., Шведова~М.\,А.}
 Параллельные корпуса в составе Национального корпуса русского языка:
технологии и решаемые задачи~// Компьютерная лингвистика: научное
направление и учебная дисциплина.~--- Гомель: ГГУ им.\ Ф.~Скорины, 2010.
С.~30--34.

\bibitem{zat-5} %6
\Au{Сичинава~Д.\,В.}
 Комплексное исследование одноязычного и параллельного корпусов в
грамматических исследованиях~// Корпусная лингвистика-2011: Труды
Междунар. конф.~--- СПб.: СПбГУ, 2011. С.~316--322.

\bibitem{zat-7}
\Au{Сичинава~Д.\,В., Архангельский~Т.\,А.}
 Параллельные бе\-ло\-рус\-ско-рус\-ский и русско-белорусский корпуса: совместный
проект Национального корпуса русского языка~// Труды шко\-лы-се\-ми\-на\-ра
TEL-2012.~--- Казань: КФУ, 2012. С.~54--60.
\bibitem{zat-8}
\Au{Loiseau~S., Sitchinava~D.\,V., Zalizniak~A.\,A., Zatsman~I.\,M.}
 Information technologies for creating the database of equivalent verbal forms in the
Russian-French multivariant parallel corpus~// Информатика и её применения,
2013. Т.~7. Вып.~2. С.~100--109.
\bibitem{zat-9}
\Au{Добровольский~Д.\,О., Кретов~А.\,А., Шаров~С.\,А.}
 Корпус параллельных текстов: архитектура и возможности использования~//
Национальный корпус русского языка: 2003--2005.~--- М.: Индрик, 2005.
С.~263--296.
\bibitem{zat-10}
\Au{Андреева~Е.\,Г., Касевич~В.\,Б.}
 Грамматика и лексика (на материале анг\-ло-рус\-ско\-го корпуса параллельных
текстов)~// Национальный корпус русского языка: 2003--2005.~--- М.: Индрик,
2005. С.~297--307.
\bibitem{zat-11}
\Au{Гак~В.\,Г.}
 Русский язык в сопоставлении с французским.~--- М.: УРСС, 2006. 264~с.
\bibitem{zat-12}
\Au{Гак~В.\,Г.}
 Сравнительная типология французского и русского языков.~--- М.: УРСС,
2009. 288~с.
\bibitem{zat-13}
\Au{Kouznetsova~I.\,N.}
 Grammaire contrastive du francais et du russe.~--- M.: Nestor Academic Publs.,
2009. 272~p.
\bibitem{zat-14}
\Au{Guiraud-Weber~M.}
Essais de syntaxe russe et contrastive.~--- Aix: Universit$\acute{\mbox{e}}$ de
Provence, 2011. 337~p.


\bibitem{zat-16} %15
\Au{Goldberg~A.}
Constructions: A~Construction Grammar approach to argument structure.~---
Chicago: Univ. of Chicago Press, 1995. 265~p.

\bibitem{zat-17} %16
\Au{Goldberg~A.}
Constructions at work. The nature of generalization in grammar.~---
 Oxford: Oxford Univ. Press, 2006. 290~p.

 \bibitem{zat-15} %17
Лингвистика конструкций~/ Под ред. Е.\,В.~Рахилиной.~--- М.: Азбуковник,
2010. 584~с.

\bibitem{zat-18}
\Au{Зализняк~Анна\,А., Левонтина~И.\,Б.}
Отражение <<национального характера>> в лексике русского языка
(размышления по поводу книги: \Au{Wierzbicka~Anna}. Semantics, culture, and
cognition. Universal human concepts in culture-specific configurations.~--- N.Y.,
Oxford: Oxford Univ. Press, 1992)~// Russian Linguistics, 1996. Vol.~20. No.\,2/3.
P.~237--264.
\bibitem{zat-19}
\Au{Виноградов~В.\,В.}
 История слов.~--- М.: Толк, 1994. 1138~с.
\bibitem{zat-20}
\Au{Плунгян~В.\,А.}
Конструкция с \textit{успеть} и \textit{не успеть} в русском языке
XIX--XX~вв.: корпусное исследование~// Русский язык XIX~века: Проб\-ле\-мы
изучения и лексикографического описания.~--- СПб.: Наука, 2004. С.~112--115.
\bibitem{zat-21}
\Au{Kozerenko~E.\,B.}
 Cognitive approach to language structure segmentation for machine translation
algorithms~// MLMTA'03:  Conference (International) on Machine
Learning; Models, Technologies and Applications Proceedings.~--- Las Vegas: CSREA Press,
2003. P.~49--55.
\bibitem{zat-22}
\Au{Козеренко~E.\,Б.}
Лингвистические фильтры в статистических моделях машинного перевода~//
Информатика и её применения, 2010. Т.~4. Вып.~2. С.~83--92.
\bibitem{zat-23}
\Au{Kozerenko~E.\,B.}
 Syntactic transformations modelling for hybrid machine translation~//
ICAI'11, WORLDCOMP'11 Proceedings.~--- Las Vegas: CSREA Press, 2011.
P.~875--881.

\end{thebibliography}
} }

\end{multicols}

\vspace*{-6pt}

\hfill{\small\textit{Поступила в редакцию 29.03.14}}

%\newpage


\vspace*{12pt}

\hrule

\vspace*{2pt}

\hrule

\def\tit{INFORMATION TECHNOLOGIES FOR~CORPUS STUDIES:
UNDERPINNINGS FOR~CROSS-LINGUISTIC DATABASE CREATION }

\def\titkol{Information technologies for corpus studies: Underpinnings for
cross-linguistic database creation}

\def\aut{N.\,V.~Buntman$^1$, Anna A.~Zaliznyak$^{2,3}$, I.\,M.~Zatsman$^3$,
 M.\,G.~Kruzhkov$^3$, E.\,Yu.~Loshchilova$^3$, and~D.\,V.~Sitchinava$^4$}

\def\autkol{N.\,V.~Buntman, A.\,A.~Zaliznyak, I.\,M.~Zatsman, \textit{et al.}}

\titel{\tit}{\aut}{\autkol}{\titkol}

\vspace*{-9pt}

\noindent
$^1$Faculty of Foreign Languages and Area Studies,
M.\,V.~Lomonosov Moscow State University,
31-a Lomonosov\\
$\hphantom{^1}$Str., Moscow 119192, Russian Federation

\noindent
$^2$Institute of Linguistics, Russian Academy of Sciences,
1-1 Bolshyi Kislovskiy pereulok, Moscow 125009, Russian\\
$\hphantom{^1}$Federation

\noindent
$^3$Institute of Informatics Problems, Russian Academy of Sciences,
44-2 Vavilov Str., Moscow 119333, Russian\\
$\hphantom{^1}$Federation

\noindent
$^4$Institute of Russian Language, Russian Academy of Sciences,
18/2 Volkhonka Str., Moscow 119019, Russian\\
$\hphantom{^1}$Federation

\def\leftfootline{\small{\textbf{\thepage}
\hfill INFORMATIKA I EE PRIMENENIYA~--- INFORMATICS AND
APPLICATIONS\ \ \ 2014\ \ \ volume~8\ \ \ issue\ 2}
}%
 \def\rightfootline{\small{INFORMATIKA I EE PRIMENENIYA~---
INFORMATICS AND APPLICATIONS\ \ \ 2014\ \ \ volume~8\ \ \ issue\ 2
\hfill \textbf{\thepage}}}

\vspace*{6pt}


\Abste{Information technology for creation of cross-linguistic databases of Russian
texts with French translations (also known as parallel texts) is considered. The
underlying principles of the developed database provide a unique combination of
three types of bilingual search: lexical, grammatical, and lexico-grammatical.
A~distinctive feature of the considered technology is simultaneous creation of
Russian-French parallel subcorpus within the National Russian Corpus and of the
cross-linguistic database of Russian verbal lexico-grammatical forms and their
French functional equivalents. The subcorpus and the database have different levels
of alignment: the former is aligned at the level of sentences, and the later
at the level
of constructions. The academic relevance of the developed database is due to its
support of bilingual contrastive grammar development, as well as to
its role in creation
of Russian grammar based on the modern empirical base and information technologies of
corpus linguistics. The main practical application of the database consists in
improvement of quality of machine translation.}

\KWE{parallel corpus; information technology; cross-linguistic databases; bilingual
lexical grammar search; corpus linguistics; contrastive grammar}

\DOI{10.14357/19922264140210}

\Ack

\noindent
The work was performed in the Institute of Informatics Problems of the Russian Academy
of Sciences with financial support of Foundation ``Dynasty'' (grant NG13-036)
and Russian Foundation for Basic Research (grant No.\,13-06-00403).

  \begin{multicols}{2}

\renewcommand{\bibname}{\protect\rmfamily References}
%\renewcommand{\bibname}{\large\protect\rm References}

{\small\frenchspacing
{%\baselineskip=10.8pt
\addcontentsline{toc}{section}{References}
\begin{thebibliography}{99}

\bibitem{zate-1}
\Aue{Aijmer,~K., and B.~Altenberg.}
 2013. \textit{Advances in corpus-based contrastive linguistics.
 Studies in honour of Stig
Johansson}. Amsterdam: John Benjamins. 295~p.
\bibitem{zate-2}
\Aue{Dobrovolsky,~D.\,O., A.\,A.~Kretov, and S.\,A.~Sharoff.}
 2005. Korpus parallel'nykh tekstov [Corpus of parallel texts]. \textit{Nauchnaya i
Tekhnicheskaya Informatsiya. Ser.~2.
Informatsionnye protsessy i sistemy} [Scientific
and technical information. Ser.~2: Information processes and systems] 6:16--27.
\bibitem{zate-3}
Kiseleva,~K.\,L., E.\,V.~Rahilina, V.\,A.~Plungjan, and S.\,G.~Tatevosov, eds.
2009. \textit{Korpusnye issledovaniya po russkoy grammatike}
[Corpus studies on Russian grammar]. Мoscow: Probel-2000. 516~p.
\bibitem{zate-4}
\Aue{Dobrovolsky,~D.\,O.}
 2009. Korpus parallel'nykh tekstov v issledovanii kul'turno-spetsifichnoy
 leksiki [A~corpus of parallel texts and studying culture-specific lexicon].
 \textit{Natsional'nyy korpus
russkogo yazyka: 2006--2008. Novye rezul'taty i perspektivy} [Russian National
Corpus: 2006--2008. New results and prospects]. St.\ Petersburg: Nestor-Istoriya.
383--401.

\bibitem{zate-6} %5
\Aue{Sitchinava,~D.\,V., and M.\,A.~Shvedova.}
 2010. Parallel'nye korpusa v sostave Natsional'nogo korpusa russkogo yazyka:
Tekhnologii i reshaemye zadachi [Parallel corpora of the Russian National Corpus:
Technologies and problems].
\textit{Komp'yuternaya lingvistika: Nauchnoe napravlenie i
uchebnaya distsiplina} [Computational linguistics: Scientific field and academic
discipline]. Gomel': Gomel' University. 30--34.

\bibitem{zate-5} %6
\Aue{Sitchinava,~D.\,V.}
 2011. Kompleksnoe issledovanie od\-no\-yazych\-no\-go i parallel'nogo korpusov v
grammaticheskikh issledovaniyakh [Comprehensive study of monolingual and
parallel corpora in grammatical studies].
\textit{Korpusnaya Lingvistika-2011: Trudy
Konferentsii} [Corpus-Based Linguistics-2011 Proceedings]. St.\ Petersburg. 316--322.

\bibitem{zate-7}
\Aue{Sitchinava,~D.\,V., and T.\,A.~Arhangel'skij.}
 2012. Pa\-ral\-lel'nye belorussko-russkiy i russko-belorusskiy kor\-pu\-sa: Sovmestnyy
proekt Natsional'nogo kor\-pu\-sa\linebreak russkogo yazyka [Parallel Belarusian-Russian and
Russian-Belarusian corpora: Joint project of the Russian National Corpus].
School-Seminar TEL-2012 Proceedings.
Kazan': Kazan' University. 54--60.
\bibitem{zate-8}
\Aue{Loiseau,~S., D.\,V.~Sitchinava, A.\,A.~Zalizniak, and I.\,M.~Zatsman.}
 2013. Information technologies for creating the database of equivalent verbal
 forms in the Russian-French multivariant parallel corpus.
 \textit{Informatika i ee Primeneniya}~--- \textit{Inform. Appl.}
 7(2):100--109.
\bibitem{zate-9}
\Aue{Dobrovolsky,~D.\,O., A.\,A.~Kretov, and S.\,A.~Sharoff.}
 2005. Korpus parallel'nykh tekstov: Arkhitektura i vozmozhnosti ispol'zovaniya
[Corpus of parallel texts: Architecture and usage].
\textit{Natsional'nyy korpus russkogo
yazyka: 2003--2005} [Russian National Corpus 2003--2005]. Moscow: Indrik.
263--296.
\bibitem{zate-10}
\Aue{Andreeva,~E.\,G., and V.\,B.~Kasevich.}
 2005. Grammatika i leksika (na materiale anglo-russkogo korpusa parallel'nykh
tekstov) [Grammar and lexicon in the English-Russian corpus of parallel texts].
\textit{Natsional'nyy korpus russkogo yazyka: 2003--2005} [Russian National Corpus
2003--2005]. Moscow: Indrik. 297--307.
\bibitem{zate-11}
\Aue{Gak,~V.\,G.}
 2006. \textit{Russkiy yazyk v sopostavlenii s frantsuzskim}
 [Russian language compared to French]. Moscow: URSS. 264~p.
 
\columnbreak

\bibitem{zate-12}
\Aue{Gak,~V.\,G.}
 2009. \textit{Sravnitel'naya tipologiya frantsuzskogo i russkogo yazykov}
 [Comparative typology of French and Russian]. Moscow: URSS. 288~p.
\bibitem{zate-13}
\Aue{Kouznetsova,~I.\,N.} 2009.
\textit{Grammaire contrastive du francais et du russe}.
Moscow: Nestor Academic Publs. 272~p.
\bibitem{zate-14}
\Aue{Guiraud-Weber,~M.}
 2011. \textit{Essais de syntaxe russe et contrastive}.
 Aix: Universit$\acute{\mbox{e}}$ de Provence. 337~p.


\bibitem{zate-16}
\Aue{Goldberg,~A.} 1995. \textit{Constructions: A~construction grammar approach to
argument structure}. Chicago: Univ. of Chicago Press. 265~p.
\bibitem{zate-17}
\Aue{Goldberg,~A.}
2006. \textit{Constructions at work. The nature of generalization in language}.
Oxford: Oxford Univ. Press. 290~p.

\bibitem{zate-15} %17
Rakhilina,~E.\,V., ed.
2010. \textit{Lingvistika konstruktsiy} [Construction linguistics]. Moscow: Azbukovnik.
584~p.

\bibitem{zate-18}
\Aue{Zaliznjak,~Anna A., and I.\,B.~Levontina.}
 1996. Otrazhenie ``natsional'nogo kharaktera'' v leksike russkogo yazyka
(razmyshleniya po povodu knigi: Anna Wierzbicka. 1992. \textit{Semantics, culture, and
cognition. Universal human concepts in culture-specific configurations}.~--- New York,
Oxford: Oxford Univ. Press) [Representation of ``national character'' in the
Russian lexicon (reflections on the book: Anna Wierzbicka. 1992. \textit{Semantics,
culture, and cognition. Universal human concepts in culture-specific configurations}.
New York, Oxford: Oxford Univ. Press]. \textit{Russian Linguistics} 20:237--264.
\bibitem{zate-19}
\Aue{Vinogradov,~V.\,V.}
 1994. \textit{Istoriya slov} [History of words]. Moscow: Tolk. 1138~p.
\bibitem{zate-20}
\Aue{Plungjan,~V.\,A.}
 2004. Konstruktsiya s uspet' i ne uspet' v russkom yazyke XIX--XX~vv.: Korpusnoe
issledovanie [Constructions with ``uspet''' and ``ne uspet''' in Russian language
in XIX--XX centuries: Corpus-based studies]. \textit{Russkiy yazyk XIX~veka: Problemy
izucheniya i leksikograficheskogo opisaniya} [Russian language in XIX~century:
Studies and lexicographical description]. St.\ Petersburg: Nauka. 112--115.
\bibitem{zate-21}
\Aue{Kozerenko,~E.\,B.}
2003. Cognitive approach to language structure segmentation for machine
translation algorithms. \textit{MLMTA'03:  Conference (International) on
Machine Learning; Models, Technologies and Applications Proceedings}. Las Vegas. 49--55.
\bibitem{zate-22}
\Aue{Kozerenko,~E.\,B.}
 2010. Lingvisticheskie fil'try v stati\-sti\-che\-skikh modelyakh mashinnogo perevoda
[Linguistic filters for statistical machine translation models].
\textit{Informatika i ee Primeneniya}~--- \textit{Inform. Appl.}] 4(2):83--92.
\bibitem{zate-23}
\Aue{Kozerenko,~E.\,B.}
 2011. Syntactic transformations modelling for hybrid machine translation.
\textit{ICAI'11, WORLDCOMP'11 Proceedings}. Las Vegas. 875--881.

\end{thebibliography}
} }


\end{multicols}

\vspace*{-9pt}

\hfill{\small\textit{Received March 29, 2014}}

\vspace*{-12pt}

\Contr

\noindent
\textbf{Buntman Nadezhda V.}~(b.~1957)~--- Candidate of Science (PhD)
in philology, associated professor, Faculty of Foreign Languages and Area
Studies, M.\,V.~Lomonosov Moscow State University,
31-a Lomonosov Str., Moscow 119192, Russian Federation;
nabunt@hotmail.com

\pagebreak

\noindent
\textbf{Zalizniak Anna A.}~(b.~1959)~--- Doctor of Science in philology,
leading scientist, Institute of Linguistics, Russian Academy of Sciences,
1-1 Bolshoy Kislovskiy pereulok, Moscow 125009, Russian Federation;
Institute of Informatics Problems, Russian Academy of Sciences,
44-2 Vavilov Str., Moscow 119333, Russian Federation;
anna.zalizniak@gmail.com

\vspace*{3pt}

\noindent
\textbf{Zatsman Igor M.}~(b.~1952)~--- Doctor of Science in technology,
Head of Department, Institute of Informatics Problems, Russian Academy of
Sciences, 44-2 Vavilov Str., Moscow 119333, Russian Federation;
izatsman@yandex.ru

\vspace*{3pt}

\noindent
\textbf{Kruzhkov Mikhail G.}~(b.~1975)~--- leading programmer, Institute
of Informatics Problems, Russian Academy of Sciences,
44-2 Vavilov Str., Moscow 119333, Russian Federation; magnit75@yandex.ru

\vspace*{3pt}

\noindent
\textbf{Loshchilova Elena J.}~(b.~1960)~--- scientist, Institute of
Informatics Problems, Russian Academy of Sciences,
44-2 Vavilov Str., Moscow 119333, Russian Federation;
lena0911@mail.ru

\vspace*{2pt}

\noindent
\textbf{Sitchinava Dmitri V.}~(b.~1980)~--- Candidate of Science (PhD) in
philology, senior scientist, Institute of the Russian Language, Russian
Academy of Sciences, 18/2 Volkhonka Str., Moscow 119019, Russian Federation;
mitrius@gmail.com

 \label{end\stat}

\renewcommand{\bibname}{\protect\rm Литература}