\def\stat{zaharova}

\def\tit{РЕШЕНИЕ ОБРАТНОЙ ЗАДАЧИ В МНОГОДИПОЛЬНОЙ МОДЕЛИ ИСТОЧНИКОВ МАГНИТОЭНЦЕФАЛОГРАММ МЕТОДОМ НЕЗАВИСИМЫХ КОМПОНЕНТ$^*$}

\def\titkol{Решение обратной задачи в многодипольной модели источников
МЭГ %магнитоэнцефалограмм
методом независимых компонент}

\def\autkol{В.\,Е.~Бенинг,  М.\,А.~Драницына, Т.\,В.~Захарова, П.\,И.~Карпов}

\def\aut{В.\,Е.~Бенинг$^1$,  М.\,А.~Драницына$^2$, Т.\,В.~Захарова$^3$, П.\,И.~Карпов$^4$}

\titel{\tit}{\aut}{\autkol}{\titkol}

{\renewcommand{\thefootnote}{\fnsymbol{footnote}}
\footnotetext[1]{Работа выполнена при поддержке РНФ (проект 14-11-00364).}}

\renewcommand{\thefootnote}{\arabic{footnote}}
\footnotetext[1]{Московский государственный университет им.\ М.\,В.~Ломоносова,
факультет вычислительной математики и кибернетики;
Институт проблем информатики Российской академии наук, bening@yandex.ru}
\footnotetext[2]{Московский государственный университет им.\ М.\,В.~Ломоносова,
факультет вычислительной математики и кибернетики,
margarita13april@mail.ru}
\footnotetext[3]{Московский государственный университет им.\ М.\,В.~Ломоносова,
факультет вычислительной математики и кибернетики, lsa@cs.msu.ru}
\footnotetext[4]{Национальный исследовательский технологический университет <<МИСиС>>, karpov.petr@gmail.com}

\vspace*{-9pt}


\Abst{Настоящая работа посвящена изучению функциональных зон головного
мозга человека.
Функциональное картирование коры головного мозга является чрезвычайно сложной
задачей, возникновение которой обусловлено современным уровнем развития методов
 неинвазивного исследования головного мозга. Магнитоэнцефалография (МЭГ), один из
 таких современных неинвазивных методов,~--- очень мощный инструмент, обладающий
 научным и прикладным медицинским потенциалом. Результатом проведения МЭГ
 являются большие массивы данных, несущие информацию о процессах, происходящих в головном
 мозге. В~ходе обработки этих данных перед исследователем ставится некорректная обратная
 задача, заключающаяся в пространственной реконструкции источников МЭГ-сиг\-на\-лов
 в коре головного мозга человека с заданной точностью.
На настоящий момент не существует универсальных инструментов для
точного в достаточной степени решения такой обратной задачи при
анализе МЭГ-сиг\-на\-лов. Одному и тому же распределению потенциалов на
поверхности головы могут соответствовать различные зоны активности
коры головного мозга. Однако при некоторых предположениях: источники
потенциала дискретные, относятся к различным функциональным областям
мозга, располагаются относительно неглубоко,~--- задача имеет
однозначное решение.
В данной работе предполагается, что МЭГ-сиг\-нал представляет собой
суперпозицию сигналов мультидиполей. Решение обратной задачи в таком
случае называется многодипольным приближением. Нахождение источников
активности проходит в два этапа: на первом методом независимых
компонент производится декомпозиция исходных МЭГ-сиг\-на\-лов на
конечное число независимых компонент; на втором по аналитической
формуле рассчитываются координаты однодипольного источника
активности для каждой отдельной независимой компоненты.}

\KW{метод независимых компонент; нормальное распределение; токовый диполь;
многодипольная модель; магнитоэнцефалограмма}

\DOI{10.14357/19922264140208}

\vskip 10pt plus 9pt minus 6pt

      \thispagestyle{headings}

      \begin{multicols}{2}

            \label{st\stat}

\section{Введение}
Головной мозг человека~--- это орган центральной нервной системы. Он
состоит из большого числа (до 200~млрд) нейронов, связанных
между собой особыми связями, превращающими наш мозг во
взаимосвязанную сеть. Взаимодействуя посредством этих связей,
нейроны формируют электрические импульсы, которые управляют
деятельностью всего организма. Ввиду высокой сложности организации
мозга его работа до сих пор является недостаточно изученной
областью.

Магнитоэнцефалография~--- это новый неинвазивный метод исследования
активности головного мозга~[1, 2]. Интерес к МЭГ в
мире очень высок.

Начиная с 1992~г.\ финская компания Elekta Neuromag Oy занимается
разработкой программного обеспечения в области МЭГ.
В~2000~г.\ в США был образован Martinos Center for
Biomedical Imaging при Массачусетском технологическом институте.
В~июле 2005~г.\ был запущен проект по компьютерному моделированию
коры головного мозга человека BlueBrainProject. Над ним совместно
работают компания IBM и Швейцарский федеральный технический институт
Лозанны. С~2007~г.\ в Кембридже в одном из крупнейших центров по
изучению психологии Medical Research Council Cognitionand Brain
Sciences Unit (MRC) начинает действовать МЭГ-ла\-бо\-ра\-то\-рия.
Исследовательская работа по изучению головного мозга ведется на
медицинских факультетах: в старейшем университете Швейцарии в
Базеле, в Швеции в Гетеборгском университете, в Хельсинкском
технологическом университете.

В России впервые МЭГ-центр <<На\-уч\-но-обра\-зо\-ва\-тель\-ный центр
нейрокогнитивных исследований>> был создан в 2008~г.\ в Москве.
И~с~2011~г.\ на базе этого центра на факультете ВМК МГУ им.\
М.\,В.~Ломоносова стали проводиться исследования по обработке
МЭГ-сиг\-на\-лов. При этом одной из важнейших ставилась задача точной
локализации областей активности нейронов~\cite{friston}.

Наиболее сложной является проблема повышения точности локализации
первичной моторной коры (М1) и конкретно области представительства
руки в зоне~М1. Эта задача решалась при помощи метода вызванных
потенциалов и построения ассоциативного фильтра~[4, 5]. Был
рассмотрен и иной статистический подход, основанный на различных
способах кластеризации мозга~[6, 7].

В данной работе решается обратная задача по локализации источников
следующим образом.\linebreak В~случае однодипольного источника имеется
аналитическое решение обратной задачи нахождения координат диполя.
Авторами рассматривается многодипольная модель с конечным числом
областей активности нейронов. Предлагается метод сведения
многодипольной модели к решению обратной задачи для некоторого числа
разных функциональных однодипольных источников. Это стало возможным
в связи с применением метода независимых компонент (ICA~--- Independent
Component Analysis)~\cite{ICA},
который разделил смешанный МЭГ-сиг\-нал на разные функциональные
компоненты. Далее для каждой такой компоненты рассчитываются
координаты источника мозговой активности.

\section{Электромагнитное поле, создаваемое нейронной активностью}

С физической точки зрения мозговая активность описывается с помощью
классической электродинамики сплошных сред. Динамика
электромагнитного поля определяется уравнениями Максвел\-ла в среде~\cite{landau},
которые в системе СИ записываются следующим образом:
\begin{equation}
\left.
\begin{array}{rl}
\nabla \times \mathbf{ H} &= \mathbf {j} + \fr{\partial D}{\partial t}\, ;\\[6pt]
\nabla \times \mathbf{ E} &= -\fr{\partial B}{\partial t} \,; \\[6pt]
\nabla \cdot\mathbf{B} &= 0 \,;\\[6pt]
\nabla \cdot\mathbf{D} &=  \rho\,,
\end{array}
\right\}
\label{maxwell}
\end{equation}
где ${\mathbf H}$~--- напряженность магнитного поля;
${\mathbf E}$~--- напряженность электрического поля;  ${\mathbf B}$~---
магнитная индукция;  ${\mathbf D}$~--- электрическая индукция;
а также материальными уравнениями, в которых заложены свойства среды:
\begin{equation}
\left.
\begin{array}{rl}
 \mathbf {D} &= \epsilon \mathbf{E} \,;\\[6pt]
 \mathbf {B} &= \mu \mathbf{ H}  \,;\\[6pt]
 \mathbf {j} &= \sigma \mathbf{E} \,,
 \end{array}
 \right\}
 \label{material}
\end{equation}
\noindent где $\epsilon$ и $\mu$~--- диэлектрическая и магнитная проницаемость
среды соответственно; $\sigma$~--- проводимость среды; ${\mathbf j}$~---
плот\-ность электрического тока.



Для исследования мозговой активности используется два стандартных
приближения~\cite{hamalainen93}. Во-пер\-вых, считается, что магнитная
проницаемость всех тканей головы совпадает с магнитной
про\-ни\-ца\-емостью вакуума: $\mu \hm= \mu_0$. Во-вто\-рых, используется
приближение квазистатического магнитного поля, при котором в
уравнениях Максвелла~(\ref{maxwell})  можно пренебречь всеми
производными по времени, т.\,е.\ в любой момент времени
электрическое и магнитное поля определяются мгновенным
распределением всех зарядов и токов в системе, как если бы они были
стационарными.

В квазистатическом приближении поле~${\mathbf E}$ оказывается безвихревым,
поэтому можно ввести скалярный электрический потенциал
\begin{equation}
\label{varphi}
{\mathbf E} = - \nabla \varphi
\end{equation}
и для расчета магнитного поля использовать закон
Био--Савара--Лапласа:
\begin{equation}
\label{bsl}
\mathbf{B(r)} = \fr{\mu_0}{4 \pi} \int
\fr{\mathbf{j(r')} \times \mathbf{R}}{R^3}\, d^3 r' \,,
\end{equation}
где $\mathbf{r}$~--- ра\-ди\-ус-век\-тор точки, в которой
вычисляется магнитное поле, интегрирование ведется по~$\mathbf{r'}$~---
всем точкам источника; $\mathbf{R} \hm= \mathbf{r}\hm - \mathbf{r'}$.

Плотность тока ${\mathbf j}$, создаваемого нейронной активностью,
можно разделить на две компоненты: первичный ток и объемный
(пассивный) ток:
\begin{equation}
\label{current}
\mathbf{j} = \mathbf{j}^p + \mathbf{j}^v \,.
\end{equation}

Первичный ток~--- это ток, создаваемый непосредственно нейронной
активностью (т.\,е.\ градиентами химического потенциала), и его
распределение сильно локализовано в небольшой \mbox{области} головного
мозга. Такое локализованное распределение плотности тока удобно
моделировать с по\-мощью понятия токового диполя, для которого
плотность тока задается $\delta$-функ\-ци\-ей Дирака:
\begin{equation}
\label{dipole}
\mathbf{j}^p = \mathbf{Q}  \delta( \mathbf{r} - \mathbf{r_Q}) \,,
\end{equation}
где $\mathbf{Q}$~--- дипольный момент токового диполя,
расположенного в точке~$\mathbf{r_Q}$. В~данной статье
рассматриваются только токовые диполи (в отличие, например, от
магнитных диполей), поэтому далее токовые диполи будут называться
просто диполями. Заметим, что на клеточном уровне один диполь может
порождаться  большим количеством микроскопических первичных токов
(называемых в этом случае возбужденными), которые вызываются
десятками тысяч синхронно активируемых больших пирамидальных
нейронов коры головного мозга~\cite{baillet}; несмотря на это, такие
возбуждения хорошо аппроксимируются однодипольной моделью. Если
одновременно проявляется активность в нескольких хорошо
локализованных зонах головного мозга, то можно пользоваться
многодипольным приближением~\cite{mosher}:
\begin{equation*}
%\label{dipoles}
\mathbf{j}^p = \sum\limits_{i=1}^N \mathbf{Q}_i  \delta( \mathbf{r} - \mathbf{r_Q}_i) \,,
\end{equation*}
где $N$~--- число диполей, каждый из которых имеет дипольный момент~$\mathbf{Q}_i$
и расположен в точке~$\mathbf{r_Q}_i$.

Объемный ток создается макроскопическим электрическим полем и
обеспечивает локальную электронейтральность, которую стремится
нарушить первичный ток. Таким образом, объемный ток определяется
материальным уравнением~(\ref{material}):
\begin{equation}
\label{volume-ohm}
\mathbf{j}^v (\mathbf{r}) =  \sigma(\mathbf{r}) \mathbf{E(r)} \,.
\end{equation}
В отличие от первичного, объемный ток (порожденный
некоторыми первичными токами) распределен по всему объему головы.

Применив разделение тока на первичный и вторичный~(\ref{current}) к
закону Био--Са\-ва\-ра--Лап\-ла\-са~(\ref{bsl}), а также воспользовавшись
уравнениями~(\ref{varphi}) и~(\ref{volume-ohm}), получим
\begin{equation*}
%\label{B-varphi}
\mathbf{B(r)} = \fr{\mu_0}{4 \pi} \int \left(
\mathbf{j}^p \mathbf{(r')}  - \sigma(\mathbf{r'})
\nabla \varphi(\mathbf{r'})\right) \times \fr{\mathbf{R}}{R^3}\,  d^3 r' \,,
\end{equation*}
что после преобразований можно записать сле\-ду\-ющим образом:
\begin{equation}
\label{B2-varphi}
\mathbf{B(r)} = \fr{\mu_0}{4 \pi} \int \left(
\mathbf{j}^p \mathbf{(r')}  + \varphi(\mathbf{r'}) \nabla'
\sigma(\mathbf{r'}) \right) \times \fr{\mathbf{R}}{R^3} \, d^3 r' \,,
\end{equation}
где оператор $\nabla'$ относится к переменной~$\mathbf{r'}$.

В следующем разделе применяется общая тео\-рия электромагнитного поля,
создаваемого нейронной активностью, для случая однодипольной
сферической модели.

\section{Обратная задача для~однодипольной сферической модели головы}

В данной работе рассматривается сферическая модель головы. Эта
модель является самой прос\-той и позволяет получить аналитические
решения, но в то же время она улавливает большинство принципиальных
эффектов. Эта модель дает не только качественные, но и даже неплохие
количественные\linebreak
 предсказания, если источник сигнала не расположен в
непосредственной близости от центра аппроксимирующей сферы~\cite{whysphermodel}.
С~небольшими уточнениями эту модель можно
использовать для\linebreak постро\-ения более реалистичных моделей головы~\cite{hamalainen85}.

В данной работе голова моделируется сферой радиуса~$R$ с
цент\-раль\-но-сим\-мет\-рич\-ным распределением проводимости $\sigma(r)$.
В~такой сис\-те\-ме магнитное поле вне головы можно вычислить, используя
только первичные токи и не рассматривая вторичные~\cite{hamalainen93}.

Рассмотрим радиальную компоненту магнитного поля $B_r \hm= \mathbf{B}
\cdot \mathbf{e}_r$ ($\mathbf{e}_r \hm= \mathbf{r}/r$~--- единичный
вектор, сонаправленный с ра\-ди\-ус-век\-то\-ром). Из формулы~(\ref{B2-varphi})
следует, что вклад объемных токов в~$B_r$ равен
нулю, так как векторы $\nabla' \sigma \hm\sim \mathbf{r'}$, $\mathbf{R}
\hm= \mathbf{r} \hm- \mathbf{r'}$ и  $\mathbf{e}_r$ компланарны, поэтому
их смешанное произведение равно нулю. Следовательно, в проекции на
ра\-ди\-ус-век\-тор в формулу~(\ref{B2-varphi}) дает вклад только
первичный ток:
\begin{equation}
\label{B2}
\mathbf{B(r)} = \fr{\mu_0}{4 \pi} \int  \mathbf{j}^p \mathbf{(r')}  \times
\fr{\mathbf{R}}{R^3} \, d^3 r' \,.
\end{equation}

Теперь рассмотрим однодипольную модель. Если источником является один диполь~(\ref{dipole}),
то формула~(\ref{B2}) превращается в
\begin{equation}
\label{Bdipole}
\mathbf{B_r}=-\fr{\mu_0}{4\pi}\,
\fr{ \mathbf{Q} \times \mathbf{r_Q}}
{ |\mathbf{r} - \mathbf{r_Q}|^3} \,\mathbf{e}_r \,.
\end{equation}



Так как радиальная компонента диполя не дает вклада в~$B_r$ вне
головы, то, не теряя общности, можно считать, что диполь имеет
ориентацию $\mathbf{Q} \hm= Q \mathbf{e}_x$ и расположен в точке
$\mathbf{r_Q}\hm = r_Q \mathbf{e}_z$ (рис.~1). Тогда
формула~(\ref{Bdipole}) запишется следующим образом~\cite{hamalainen85}:
\begin{equation*}
%\label{Br}
B_r = - \fr{\mu_0}{4 \pi}\,\fr{Q r_Q r \sin\theta \cos\phi}
{(r^2 + r_Q^2 -2 r r_Q \cos\theta)^{3/2}} \,.
\end{equation*}
Отсюда видно, что $B_r \hm= 0$ во всей плоскости $\phi \hm= \pm
\pi/2$. Найдем точки, в которых~$B_r$ достигает локальных
экстремумов. В~этих точках $\cos\phi \hm= \pm 1$.\linebreak\vspace*{-12pt}

\pagebreak

\noindent
\begin{center}  %fig1
\mbox{%
\epsfxsize=73.457mm
\epsfbox{zah-1.eps}
}
  \vspace*{5pt}

{{\figurename~1}\ \ \small{Геометрия задачи}}
  \end{center}

\vspace*{4pt}

\addtocounter{figure}{1}

\noindent
\begin{center}  %fig2
\mbox{%
\epsfxsize=75mm %76.738mm
\epsfbox{zah-2.eps}
}
\end{center}

%\vspace*{6pt}

\noindent
{{\figurename~2}\ \ \small{Обратная задача в однодипольной сферической модели. Красным цветом показаны области максимума $|B_r|$. Диполь, создающий данное магнитное поле, должен лежать в плоскости симметрии системы}}

\vspace*{16pt}

\addtocounter{figure}{1}



\noindent
 Находя локальный
экстремум по углу~$\theta$, получим экстремальные значения угла:

\noindent
\begin{equation}
\hspace*{-3mm}\cos\theta = \fr{-(R^2 + r_Q^2) + \sqrt{(R^2 + r_Q^2)^2 + 12 R^2 r_Q^2}}{2 R r_Q} \,.\!
\label{theta-max}
\end{equation}
Этот результат для прямой задачи дает возможность решить обратную
задачу в предположении, что источником сигнала является один диполь.
Если разрешить~(\ref{theta-max}) относительно~$r_Q$, получим
\begin{equation}
\hspace*{-3mm}r_Q = R \fr{(3-\cos^2 \theta) - \sqrt{(3-\cos^2 \theta)^2 - 4 \cos^2 \theta}}{2 \cos\theta}\,.\!
\label{rQ-max}
\end{equation}

Таким образом, в однодипольной модели обратная задача решается
следующим способом. Нужно найти на поверхности сферы две точки с
максимальным значением радиальной компоненты магнитного поля $B_r \hm=
\pm B_{\max}$ (рис.~2). Тогда источник будет лежать в
плоскости симметрии этих точек на расстоянии~$r_Q$ от центра сферы,
которое задается формулой~(\ref{rQ-max}).



В следующих разделах полученный результат будет применен к
многодипольной модели, которую методом ICA можно разделить  на
независимые однодипольные источники.

\section{Метод независимых компонент}

Метод независимых компонент
является методом декомпозиции смеси случайных функций \cite{ICA}.
В~рассматриваемой задаче метод ICA раскладывает регистрируемые
МЭГ-сиг\-на\-лы в линейную комбинацию независимых случайных компонент.

\subsection{Математическая модель метода независимых компонент}

Пусть существует $n$ случайных величин $x_{1}, \ldots, x_{n}$,
каждая из которых представляет собой линейную комбинацию~$n$~случайных
величин $s_{1}, \ldots, s_{n}$:
$$
x_{i} = a_{i1}s_{1} + a_{i2}s_{2} + \cdots + a_{in}s_{n}, \enskip
i=1, \ldots, n\,,
$$
где $a_{ij}$~--- некоторые действительные числа при $i,j\hm=1, \ldots, n$.

По определению $s_i$ взаимно независимы (независимые компоненты).
Модель ICA описывает, как данные наблюдений могут генерироваться в
процессе смешивания независимых компонент~$s_{i}$, и в этом смысле
данная модель является порождающей. При этом, независимые компоненты~$s_i$
не могут наблюдаться непосредственно в ходе эксперимента,
также неизвестными являются и смешивающие коэффициенты~$a_{ij}$.
Следовательно, для решения задачи есть только случайный вектор
наблюдений, при этом, применяя ICA, требуется определить как
независимые компоненты, так и смешивающие коэффициенты исходя из
максимально общих предположений.

Зачастую удобнее использовать матричное представление модели. Пусть
$x$~--- случайный вектор (век\-тор-стол\-бец), $x \hm= (x_1, \ldots, x_n)^{\mathrm{T}}$
и $s$~--- случайный вектор (век\-тор-стол\-бец), $s \hm= (s_{1}, \ldots,
s_{n})^{\mathrm{T}}$. Определим матрицу~$A$ с элементами $\{a_{ij}\}$. Тогда,
используя векторные обозначения, можно записать модель ICA в виде
$$
x = A s \,.
$$

\subsection{Ограничения метода}

Введем некоторые ограничения для обеспечения возможности оценки
независимых компонент:
\begin{itemize}
\item  статистическая независимость компонент;

\item распределение независимых компонент обязано быть отличным от гауссовского.
\end{itemize}

Заметим, что в исходной модели не предполагается ка\-кое-ли\-бо
известное распределение независимых компонент. Но в том случае, если
это распределение известно, задача может быть значительно упрощена.

Для простоты изложения метода будем считать квадратной и обратимой
смешивающую матрицу~$A$.

Оценив матрицу $A$, можно вычислить обратную к ней разделяющую
мат\-ри\-цу~$B$ и искомые независимые компоненты~$s$:

\noindent
$$
s \hm= B x \,.
$$

Без потери общности также будем полагать, что наблюдения~$x$
и независимые компоненты~$s$ имеют нулевое среднее.

\vspace*{-4pt}

\subsection{Неопределенность метода}

Отметим следующие особенности метода ICA:
\begin{itemize}
\item в представленной модели невозможно оценить дисперсии независимых
компонент, причиной является одновременная неопределенность
смешивающей матрицы~$А$ и независимых компонент~$s$ (модель не
изменится, если поделить и умножить на скалярную величину столбец~$a_{i}$
матрицы~$А$ и $s_{i}$ соответственно);

\item в представленной модели невозможно оценить порядок независимых
компонент, т.\,е.\ любую из найденных независимых компонент можно
обозначить как первой, так и $n$-й.

\end{itemize}

\vspace*{-8pt}

\subsection{Основная идея метода независимых компонент}

Пусть имеется вектор наблюдений~$х$ и согласно общей модели ICA он
представляет собой линейную комбинацию независимых компонент:
$$
x = A s \,.
$$

Будем считать, что независимые компоненты одинаково распределены.
Для вычисления независимых компонент необходимо, с учетом
обрати\-мости смешивающей матрицы, разрешить уравнение:

\noindent
$$
s = A^{-1}x\,.
$$

Определим векторы $y \hm= b^{\mathrm{T}}x \hm= \sum\limits_{i}b_{i}x_{i} \hm= b^{\mathrm{T}}As$
(вектор~$b$ рассчитывается специальным образом и будет определен
ниже) и $q \hm= A^{\mathrm{T}} b$. Тогда можно записать

\noindent
$$
y = b^{\mathrm{T}}x = q^{\mathrm{T}}s = \sum\limits_{i}q_{i}s_{i}\,.
$$

Если вектор $b^{\mathrm{T}}$ будет совпадать с одной из строк обратной матрицы~$A^{-1}$
(допустим, $k$-й), тогда скалярное произведение $b^{\mathrm{T}}x$
совпадет с $k$-й независимой компонентой~$s_k$. Понятно, что вектор~$q$
тогда будет иметь только одну ненулевую компоненту, $k$-ю и $q_k\hm=1$.

Но матрица $A$ неизвестна, и поэтому точно рассчитать вектор~$b$
невозможно. Попробуем оценить вектор~$b$, руководствуясь следующими
рассуждениями.

Примем во внимание тот факт, что сумма независимых одинаково
распределенных случайных величин имеет распределение, более близкое к
гауссовскому, чем каждая из этих случайных величин сама по себе.
Тогда случайная величина $y \hm= b^{\mathrm{T}}x \hm= q^{\mathrm{T}}s$ имеет
распределение,
максимально далекое от гауссовского в том случае, если случайная
величина равна одной из независимых компонент~$s_i$. Можно выбрать в
качестве~$b$ вектор, который максимизирует негауссовость $y \hm=
b^{\mathrm{T}}x$. Этот вектор определяет вектор $q \hm= A^{\mathrm{T}}b$ с единственной
ненулевой компонентой, а вектор $y \hm= b^{\mathrm{T}}x\hm = q^{\mathrm{T}}s$ соответствует
одной из независимых компонент. Таким образом, максимизация меры
негауссовости $b^{\mathrm{T}}x$ позволяет получить одну из независимых
компонент.

Решая задачу максимизации негауссовости по $n$-мер\-но\-му вектору~$b$,
получают $2n$ локальных максимумов, по 2~максимума на каждую
независимую компоненту: со знаком плюс и минус ($s_i$ и $-s_i$).

\vspace*{-4pt}

\subsection{Меры негауссовости}

В качестве меры негауссовости наиболее часто при расчетах используют
коэффициент эксцесса и негэнтропию~\cite{ICA}. Рассмотрим каждую из
них в отдельности.

\vspace*{-6pt}

\subsubsection{Эксцесс}

Эксцесс случайной величины~$y$ с учетом нулевого среднего можно
рассчитать по формуле:
$$
\mathrm{kurt}(y) = \mathrm{E}\lbrace y^{4}\rbrace -
3(\mathrm{E}\lbrace y^{2}\rbrace )^{2}.
$$

Основные преимущества использования эксцесса как меры негауссовости~---
это простота вычислений и теоретических выкладок, однако
коэффициент эксцесса не устойчив к выбросам.

Одним из методов нахождения локального максимума является
градиентный метод. Это метод нахождения локального экстремума
функции с помощью движения вдоль градиента. В~данном случае
градиентный алгоритм с использованием эксцесса может быть записан в
виде:
\begin{equation}
\label{G1}
\vartriangle w \thicksim \mathrm{sign}
( \mathrm{kurt}(w^{\mathrm{T}}z)) \mathrm{E}\lbrace z(w^{\mathrm{T}}z)^3\rbrace \,.
\end{equation}

Для максимизации  эксцесса $\mathrm{kurt}(y)$ с заданным вектором
наблюдений~$z$ выбирается некоторый начальный вектор~$w$,
рассчитывается направление, по которому абсолютное значение эксцесса
случайной величины $y \hm= w^{\mathrm{T}}z$ растет наиболее быстро (по формуле~(\ref{G1})),
и далее вектор~$w$ сдвигается в этом направлении.
Решение состоит в построении последовательности векторов~$w$,
увеличивающих эксцесс $\mathrm{kurt}(y)$, т.\,е.\ меру негауссовости.

\vspace*{-6pt}

\subsubsection {Негэнтропия}

Рассмотрим еще одну меру негауссовости. Для этого введем следующее
определение.

Отбеливание~--- это процесс получения случайного вектора с нулевым
математическим ожиданием, компоненты которого не коррелированы и
имеют единичные дисперсии.

Отбеливание часто используется в качестве предварительной обработки
данных и состоит в линейном преобразовании случайного вектора $x$
следующим образом: $z \hm= Vx \hm= VAs ,$ при этом $\mathrm{E}\lbrace
zz^{\mathrm{T}}\rbrace  \hm= \mathrm{I}$. В~данном случае также предполагаем,
что наблюдаемый сигнал~$x$ прошел процедуру отбеливания.

Понятие негэнтропии (от \textit{англ}.\  negative entropy) основывается
на ин\-фор\-ма\-ци\-он\-но-тео\-ре\-ти\-че\-ских свойствах дифференциальной энтропии
(далее~--- энтропии).

Одним из основных результатов теории информации является
экстремальность гауссовской случайной величины в том смысле, что она
имеет наибольшую энтропию среди всех случайных величин с одинаковыми
дисперсиями. Отметим, что малой энтропией обладают случайные
величины с островершинной плотностью распределения. Следовательно,
энтропия может служить мерой негаус\-со\-вости.

Наиболее подробно понятие и свойства энтропии изложены в~\cite{korolev}.

Чтобы получить меру негауссовости, неотрицательную и равную нулю для
гауссовских случайных величин, вводят понятие негэнтропии. Обозначим
негэнтропию как~$J$ и определим ее по формуле
$$
J(y) = H(y_{\mathrm{gauss}}) - H(y) \,,
$$
где $y_{\mathrm{gauss}}$~--- нормальный случайный вектор с такой же матрицей
корреляции, как и у вектора~$y$, а энтропия~$H$ случайного вектора~$y$
с плот\-ностью $p_y(x)$ определяется как
$$
H = - \int p_{y}(x)\log p_y(x)\,dx \,.
$$

Использование негэнтропии как показателя негауссовости оправдано
теоретическими соображениями, но, в отличие от эксцесса, негэнтропия
обладает высокой сложностью вычислений.

Обычно в вычислениях используют некую аппроксимацию негэнтропии.
Используя моменты высоких порядков, можно получить аппроксимацию вида
$$
J(y)\approx \fr{1}{12}\left(\mathrm{E} \{y^3\}\right)^2 +
\fr{1}{48} \left(\mathrm{kurt} (y)\right)^2 \,.
$$
Но такая оценка не устойчива к выбросам и характеризует в основном
хвосты распределения, не отражая особенности распределения около его центра.

В случаях, когда известна некоторая информация о характере плот\-ности
распределения, аппроксимацию негэнтропии получают при помощи метода
максимума энтропии.

Например, при помощи только одной неквадратичной функции~$G$
можно построить следующее приближение:
\begin{equation}
\label{G2}
J(y) \approx \left[\mathrm{E}\lbrace G(y)\rbrace - \mathrm{E}\lbrace G(z)\rbrace
\right]^{2}\,,
\end{equation}
где $z$~--- нормальная случайная величина с нулевым математическим
ожиданием и единичной дис\-пер\-си\-ей. При этом, выбирая не слишком
быстро воз\-рас\-та\-ющую функцию~$G$, можно получить надежную оценку
негэнтропии.

На практике при выборе функции~$G$ руководствуются следующими требованиями.
\begin{enumerate}[1.]
\item Оценивание $\mathrm{E}{G(X)}$ не должно быть сложным статистически и
оценка должна быть устойчивой к выборосам.

\item Функция $G(x)$ не должна расти быстрее, чем $|x|^2$.

\item Функция $G(x)$ должна отражать особенности распределения~$X$.
\end{enumerate}

Далее максимизируют негэнтропию, воспользовавшись ее аппроксимацией~(\ref{G2}).
Для этого применяется градиентный метод. В~этом случае
алгоритм может быть записан в виде:
\begin{align*}
\vartriangle w &\sim \left (\mathrm{E}\lbrace G(w^{\mathrm{T}}z)\rbrace -
\mathrm{E}\lbrace G(v)\rbrace \right ) \mathrm{E}\lbrace zg(w^{\mathrm{T}}z)\rbrace\,; \\
w &:= \fr{w}{||w||}\,,
\end{align*}
где $v$~--- случайная величина со стандартным нормальным распределением;
функция  $g$~--- производная функции~$G$.

\section{Заключение}

Магнитоэнцефалография~--- это неинвазивный метод исследования
функционирования головного мозга. Этот метод,  при условии внедрения
высокоточных математических методов обработки и интерпретации
полученных сигналов, в перспективе может стать ключевым инструментом
исследования в нейронауках. С~по\-мощью магнитоэнцефалографа на
поверхности головы фиксируется магнитная активность нейронов, а
затем на основе этих данных решается обратная задача по локализации
самих источников активности.

Очевидна ценность метода как в научных исследованиях, так и в
реальной клинической практике. Так, в ходе нейрохирургических
вмешательств могут быть повреждены невосполнимые зоны головного
мозга, что ведет к развитию необратимого нарушения различных функций
(например, речевых, двигательных). Так как расположение
функциональных зон в мозге человека индивидуально, для врача крайне
важно иметь инструмент по локализации с высокой точностью этих
областей.

Метод локализации, представленный в данной работе, разработан для
простой модели: количест\-во источников конечно, источники
фиксированы, источники относятся к разным функциональным областям.
В~дальнейшем предполагается, усложняя модель исследования и привлекая
суперкомпьютерную технику, приблизиться в смысле модели к реальному
самому сложному органу центральной нервной системы -- головному
мозгу и решить задачу в максимально реальных условиях.

{\small\frenchspacing
{%\baselineskip=10.8pt
\addcontentsline{toc}{section}{References}
\begin{thebibliography}{99}

\bibitem{sarvas} %1
\Au{Sarvas~J.} Basic mathematical and electromagnetic concepts of
the biomagnetic inverse problem~// Physics in Medicine and Biology,
1987. Vol.~32. No.\,1. P.~11--22.

\bibitem{baillet} %2
\Au{Baillet~S., Mosher J.\,C., Leahy~R.\,M.} Electromagnetic brain
mapping~// IEEE Signal Processing Magazine, 2001. Vol.~7. No.\,2.
P.~14--30.

\bibitem{friston} %3
\Au{Friston K., Harrison~L., Daunizeau~J., Kiebel~S., Phillips~C.,
Trujillo-Barreto~N., Henson~R., Flandin~G., Mattoutf~J.} Multiple
sparse priors for the M/EEG inverse problem~// NeuroImage, 2008.
Vol.~39. No.\,4. P.~1104--1120.

    \bibitem{article} %4
\Au{Захарова Т.\,В., Никифоров~С.\,Ю., Гончаренко~М.\,Б., Драницына~М.\,А.,
Климов~Г.\,А., Хазиахметов~М.\,Ш., Чаянов~Н.\,В.} Методы
обработки сигналов для локализации невосполнимых областей головного
мозга~// Системы и средства информатики, 2012. Т.~22. Вып.~2.
С.~157--176.

    \bibitem{opor} %5
\Au{Хазиахметов М.\,Ш., Захарова~Т.\,В.}  Об алгоритмах нахождения
опорных точек миограммы для использования в локализации
невосполнимых областей головного мозга~// Статистические методы
оценивания и проверки гипотез: Межвузовский сборник научных трудов,
2013. Т.~25. С.~56--63.

    \bibitem{ben} %6
\Au{Бенинг В.\,Е., Горшенин А.\,К., Королев В.\,Ю.} Асимптотически
оптимальный критерий проверки гипотез о числе компонент смеси
вероятностных распределений~// Информатика и её применения, 2011.
Т.~5. Вып.~3. С.~4--16.

    \bibitem{klaster} %7
\Au{Захарова Т.\,В., Гончаренко М.\,Б., Никифоров~С.\,Ю.} Метод
решения обратной задачи магнитоэнцефалографии, основанный на
кластеризации поверхности мозга~// Статистические методы оценивания
и проверки гипотез: Межвузовский сборник научных трудов, 2013.
Т.~25. С.~120--125.

\bibitem{ICA} %8
\Au{Hyv$\ddot{\mbox{a}}$rinen A., Karhunen~J., Oja~E.}  Independent
component analysis.~--- New York: John Wiley \& Sons, 2001. 504~p.

\bibitem{landau}
\Au{Landau  L.\,D., Pitaevskii~L.\,P., Lifshitz~E.\,M.}
Electrodynamics of continuous media.~--- New York: Pergamon, 1984. 432~p.

\bibitem{hamalainen93} %10
\Au{H$\ddot{\mbox{a}}$m$\ddot{\mbox{a}}$l$\ddot{\mbox{a}}$inen~M., Hari~R.,
Ilmoniemi~R.\,J.\, Knuutila~J., Lounasmaa~O.\,V.} Magnetoencephalography~---
theory, instrumentation, and applications to noninvasive studies of
the working human brain~// Rev. Modern Phys., 1993. Vol.~65.
No.\,1. P.~413--497.

\bibitem{mosher} %11
\Au{Mosher J.\,C., Lewis P.\,S., Leahy~R.\,M.}  Multiple dipole
modeling and localization from spatio-temporal MEG data~// IEEE
Trans. Biomedical Eng., 1992. Vol.~39. No.\,6. P.~541.

\bibitem{whysphermodel} %11
\Au{Uitert R., Weinstein D., Johnson~C.} Can a spherical model
substitute for a realistic head model in forward and inverse MEG
simulations?~// Biomag 2002: 13th Conference (International) on Biomagnetism
Proceedings.~--- Berlin, Offenbach: VDE Verlag, 2002. P.~798--800.

\bibitem{hamalainen85} %12
\Au{Ilmoniemi R.\,J., Hamalainen~M.\,S., Knuutila~J.} The forward
and inverse problems in the spherical model~// Biomagnetism:
Applications and theory~/
Eds. H.~Weinberg, G.~Stroink,  T.~Katila.~--- New York: Pergamon, 1985. P.~278--282.

\bibitem{korolev} %13
\Au{Королев В.\,Ю., Бенинг В.\,Е., Шоргин~С.\,Я.} Математические
основы теории риска.~--- М.: Физматлит, 2011. 620~с.

\end{thebibliography}
} }

\end{multicols}

\vspace*{-6pt}

\hfill{\small\textit{Поступила в редакцию 3.05.14}}

\newpage


%\vspace*{12pt}

%\hrule

%\vspace*{2pt}

%\hrule


\def\tit{INDEPENDENT COMPONENT ANALYSIS FOR~THE~INVERSE PROBLEM IN~THE~MULTIDIPOLE MODEL
OF~MAGNETOENCEPHALOGRAM'S SOURCES}

\def\titkol{Independent component analysis for the
inverse problem in the multidipole model magnetoencephalogram's sources}

\def\aut{V.\,E.~Bening$^{1,2}$, M.\,A.~Dranitsyna$^1$, T.\,V.~Zakharova$^1$,
and~P.\,I.~Karpov$^3$}
\def\autkol{V.\,E.~Bening, M.\,A.~Dranitsyna, T.\,V.~Zakharova,
and~P.\,I.~Karpov}


\titel{\tit}{\aut}{\autkol}{\titkol}

\vspace*{-9pt}

\noindent
$^1$Department of Mathematical Statistics,
Faculty of Computational Mathematics and Cybernetics,\\
$\hphantom{^1}$M.\,V.~Lomonosov
Moscow State University,  1-52 Leninskiye Gory, GSP-1, Moscow 119991, Russian\\
$\hphantom{^1}$Federation

\noindent
$^2$Institute of Informatics Problems, Russian Academy of Sciences,
44-2 Vavilov Str., Moscow 119333, Russian\\
$\hphantom{^1}$Federation

\noindent
$^3$Department of Theoretical Physics and Quantum Technologies,
College of New Materials and Nanotechnology,\\
$\hphantom{^1}$National University of Science and
Technology ``MISiS,''  4~Leninskiy Prosp., Moscow, Russian Federation


\def\leftfootline{\small{\textbf{\thepage}
\hfill INFORMATIKA I EE PRIMENENIYA~--- INFORMATICS AND APPLICATIONS\ \ \ 2014\ \ \ volume~8\ \ \ issue\ 2}
}%
 \def\rightfootline{\small{INFORMATIKA I EE PRIMENENIYA~--- INFORMATICS AND APPLICATIONS\ \ \ 2014\ \ \ volume~8\ \ \ issue\ 2
\hfill \textbf{\thepage}}}

\vspace*{6pt}


\Abste{This paper is devoted to a challenging task of brain functional mapping
which is posed due to the current techniques of noninvasive human brain investigation.
One of such techniques is magnetoencephalography (MEG) which is very potent in the
scientific and practical contexts. Large data retrieved from the MEG
procedure comprise information about brain processes.
Magnetoencephalography data processing sets a
 highly ill-posed problem consisting in spatial reconstruction of MEG-signal
 sources with a given accuracy. At the present moment, there is no universal tool for
 accurate solution of such inverse problem.
The same distribution of potentials on the surface of a human head may be
caused by activity of different areas within cerebral cortex. Nevertheless,
under certain assumptions, this task can be solved unambiguously. The assumptions
are the following: signal sources are discrete, belong to distinct functional areas of
 the brain, and have superficial location. The MEG-signal obtained is
 assumed to be a superposition of multidipole signals. In this case, the solution
 of the inverse problem is a multidipole approximation.
The algorithm proposed assumes two main steps.
The first step includes application of independent component analysis to
primary/basic MEG-signals and obtaining independent components, the second
step consists of treating these independent components separately and
employing an analytical formula to them as for monodipole model to get the
isolated signal source location for each component.}

\KWE{independent component analysis; normal distribution; current dipole;
multidipole model; magnetoencephalogram}

\DOI{10.14357/19922264140208}

\Ack
\noindent
The work was supported by the Russian Scientific Foundation (project 14-11-00364).


  \begin{multicols}{2}

\renewcommand{\bibname}{\protect\rmfamily References}
%\renewcommand{\bibname}{\large\protect\rm References}

{\small\frenchspacing
{%\baselineskip=10.8pt
\addcontentsline{toc}{section}{References}
\begin{thebibliography}{99}

\bibitem{2-z-1} %1
\Aue{Sarvas, J.} 1987. Basic mathematical and electromagnetic concepts of the
biomagnetic inverse problem. \textit{Phys. Medicine Biol.} 32(1):11--22.

\bibitem{1-z-1} %2
\Aue{Baillet, S., J.\,C. Mosher, R.\,M.~Leahy}.
2001. Electromagnetic brain mapping. \textit{IEEE Signal Processing Magazine}.
7(2):14--30.



\bibitem{3-z-1}
\Aue{Friston,~K., L.~Harrison, J.~Daunizeau, S.~Kiebel, Ch.~Phillips,
N.~Trujillo-Barreto, R.~Henson, G.~Flandin, and J.~Mattoutf}.
2008. Multiple sparse priors for the M/EEG inverse problem.
\textit{NeuroImage} 39:1104--1120.

\bibitem{4-z-1}
\Aue{Zakharova, T.\,V., S.\,Yu.~Nikiforov, M.\,B.~Goncharenko,
M.\,A.~Dranitsyna, G.\,A.~Klimov,  M.\,S.~Khaziakhmetov, and
N.\,V.~Chayanov}. 2012. Metody obrabotki signalov dlya lokalizatsii
nevospolnimykh oblastey golovnogo mozga
[Signal processing methods for the localization of nonrenewable brain regions].
\textit{Sistemy i Sredstva Informatiki}~--- \textit{Systems and
Means of Informatics} 22(2):157--176.

\bibitem{5-z-1}
\Aue{Khaziakhmetov, M.\,S., and T.\,V.~Zakharova}.
2013.  Ob algoritmakh nakhozhdeniya opornykh tochek miogrammy dlya ispol'zovaniya
v lokalizatsii nevospolnimykh oblastey golovnogo mozga
[Algorithms for myogram reference points search with the aim of irrecoverable
brain regions localization]. \textit{Statisticheskie Metody Otsenivaniya i
Proverki Gipotez. Mezhvuzovskiy Sbornik Nauchnykh Trudov}
[Statistical methods for estimating and hypothesis testing.
Interuniversity Collection of Research Papers]. Perm. 25:56--63.

\bibitem{7-z-1} %6
\Aue{Bening, V.\,Ye., A.\,K.~Gorshenin, and V.\,Yu.~Korolev}. 2011.
Asimptoticheski optimal'nyy kriteriy proverki gipotez o chisle komponent
smesi veroyatnostnykh raspredeleniy [Asymptotically optimum hypothesis
test for number of components in mixture of probability distribution].
\textit{Informatika i ee Primeneniya}~--- \textit{Inform. Appl.} 5(3):4--16.

\bibitem{6-z-1} %7
\Aue{Zakharova, T.\,V., M.\,B.~Goncharenko, and S.\,Yu.~Nikiforov}.
2013. Metod resheniya obratnoy zadachi mag\-ni\-to\-en\-tse\-falografii, osnovannyy na
klasterizatsii poverkhnosti mozga [Inverse problem solving method based on
clustering of brain surface]. \textit{Statisticheskie Metody Otsenivaniya i
Proverki Gipotez. Mezhvuzovskiy Sbornik Nauchnykh Trudov}
[Statistical methods for estimating and hypothesis testing. Interuniversity
Collection of Research Papers]. Perm. 25:120--125.

\bibitem{8-z-1}
\Aue{Hyv$\ddot{\mbox{a}}$rinen,~A., J.~Karhunen, and E.~Oja}.
2001. \textit{Independent component analysis}. New-York: John Wiley \& Sons.
504~p.

\bibitem{9-z-1}
\Aue{Landau,  L.\,D., L.\,P.~Pitaevskii, and E.\,M.~Lifshitz}.
1984. \textit{Electrodynamics of continuous media}.  New York: Pergamon.
432~p.


\bibitem{10-z-1}
\Aue{H$\ddot{\mbox{a}}$m$\ddot{\mbox{a}}$l$\ddot{\mbox{a}}$inen,~M., R.~Hari,  R.\,J.~Ilmoniemi,
J.~Knuutila, and O.\,V.~Lounasmaa}.
1993. Magnetoencephalography~--- theory, instrumentation, and applications to
noninvasive studies of the working human brain.
\textit{Rev. Modern Phys.} 65:413--497.

\bibitem{11-z-1}
\Aue{Mosher, J.\,C., P.\,S.~Lewis, and R.\,M.~Leahy}. 1992.
Multiple dipole modeling and localization from spatio-temporal MEG Data.
\textit{IEEE Trans. Biomedical Eng.} 39(6):541.

\bibitem{12-z-1} %11
\Aue{Uitert, R., D.~Weinstein, and C.~Johnson}.  2002.
Can a spherical model substitute for a realistic head model in forward
and inverse MEG simulations? \textit{13th  Conference (International)
on Biomagnetism Proceedings.} Jena, Germany. 798--800.

\bibitem{13-z-1}
\Aue{Ilmoniemi, R.\,J., M.\,S.~Hamalainen, and  J.~Knuutila}.
1985. The forward and inverse problems in the spherical model.
\textit{Biomagnetism: Applications and theory}.
Eds.~Weinberg,~H., G.~Stroink, and T.~Katila. New York: Pergamon. 278--282.

\bibitem{14-z-1}
\Aue{Korolev, V.\,Yu., V.\,E.~Bening, and S.\,Ya.~Shorgin}. 2011.
Matematicheskie osnovy teorii riska [Mathematical basics of risk theory].
Moscow: Fizmatlit. 620~p.

\end{thebibliography}
} }


\end{multicols}


\vspace*{-6pt}

\hfill{\small\textit{Received May 3, 2014}}

\vspace*{-18pt}




\Contr

\noindent
\textbf{Bening Vladimir E.} (b.\ 1954)~---
Doctor of Science in physics and mathematics; professor, Department of
Mathematical Statistics, Faculty of Computational Mathematics and Cybernetics,
M.\,V.~Lomonosov Moscow State University,
1-52 Leninskiye Gory, GSP-1, Moscow 119991, Russian Federation;
senior scientist, Institute of Informatics Problems, Russian Academy of Sciences,
44-2 Vavilov Str., Moscow 119333, Russian Federation;  bening@yandex.ru

\vspace*{3pt}

\noindent
\textbf{Dranitsyna Margarita A.} (b.\ 1983)~---
PhD student, Department of Mathematical Statistics,
Faculty of Computational Mathematics and Cybernetics, M.\,V.~Lomonosov
Moscow State University,  1-52 Leninskiye Gory, GSP-1, Moscow 119991, Russian Federation;
margarita13april@mail.ru

\vspace*{3pt}

\noindent
\textbf{Zakharova Tatyana V.} (b.\ 1962)~---
Candidate of Science (PhD) in physics and mathematics, senior lecturer,
Department of Mathematical Statistics, Faculty of Computational Mathematics
and Cybernetics, M.\,V.~Lomonosov Moscow State University,
Moscow State University,  1-52 Leninskiye Gory, GSP-1, Moscow 119991,
Russian Federation; lsa@cs.msu.ru

\vspace*{3pt}

\noindent
\textbf{Karpov Peter I.} (b.\ 1990)~---
PhD student, Department of Theoretical Physics and Quantum Technologies,
College of New Materials and Nanotechnology, National University of Science and
Technology ``MISiS,''  4 Leninskiy Prosp., Moscow, Russian Federation;
karpov.petr@gmail.com

 \label{end\stat}

\renewcommand{\bibname}{\protect\rm Литература}