\def\stat{rekl}
%\label{preobr}

%\def\tit{АКАДЕМИК ПУГАЧЁВ  ВЛАДИМИР СЕМЁНОВИЧ\\
%25.03.1911--25.03.1998}


%   \vspace*{-48pt}
%   \begin{center}\LARGE
%Академик Пугачёв  Владимир Семёнович\\ (25.03.1911--25.03.1998)
%   \end{center}

   %\vspace*{2.5mm}

   \begin{center}

{\prgsh\LARGE
ЮБИЛЕИ}

\end{center}
%\hrule

\vspace*{6pt}


   \vspace*{8mm}

   \thispagestyle{empty}


%\def\stat{emel}


{\prg{\Large\textbf{К 85-летию научного руководителя Федерального}}}\\[-4pt]

{\prg{\Large\textbf{государственного бюджетного учреждения науки}}}\\[-4pt]

{\prg{\Large\textbf{Института системного анализа}}}\\[-4pt]

{\prg{\Large\textbf{Российской академии наук,}}}\\[-4pt]

{\prg{\Large\textbf{главного редактора журнала <<Информатика}}}\\[-4pt]

{\prg{\Large\textbf{и её применения>>  академика}}}\\[-4pt]

 {\prg{\Large\textbf{Российской академии наук С.\,В.~Емельянова}}}

%\def\titkol{Юбилеи}

%\def\autkol{Юбилеи} %И.\,А.~Соколов, С.\,Я.~Шоргин}
%\def\aut{\ } %И.\,А.~Соколов$^1$, С.\,Я.~Шоргин$^2$}

%\titel{\tit}{\aut}{\autkol}{\titkol}

%{\renewcommand{\thefootnote}{\fnsymbol{footnote}}\footnotetext[1]
%{Работа поддержана Российским фондом
%фундаментальных исследований (проекты 11-01-00515а и 11-07-00112а),
%Федеральной целевой программой <<Научные и научно-педагогические
%кадры инновационной России на 2009--2013~годы>> и грантом Президента
%РФ МК--581.2010.1.}}

%\renewcommand{\thefootnote}{\arabic{footnote}}
%\footnotetext[1]{Институт проблем информатики Российской академии наук, %isokolov@ipiran.ru}
%\footnotetext[2]{Институт проблем информатики Российской академии наук,
%sshorgin@ipiran.ru}

%\vskip 14pt plus 9pt minus 6pt

 %     \thispagestyle{headings}

\vspace*{10mm}


          \begin{multicols}{2}

%            \label{st\stat}

            \begin{center}
\mbox{%
\epsfxsize=57.143mm
\epsfbox{eme-1.eps}
}
\end{center}

\vspace*{18pt}



  18 мая 2014~года исполнилось 85~лет академику РАН С.\,В.~Емельянову.

     Станислав Васильевич Емельянов родился 18~мая 1929~г.\ в г.~Воронеже. Учился в Московском авиационном
институте на факультете приборостроения и систем управления летательных\linebreak аппаратов (1947--1952), затем
(1953--1957)~--- в аспирантуре (без отрыва от производства) при Институте автоматики и телемеханики АН СССР (ныне
Институт проблем управления РАН).

     В 1952~г.\ С.\,В.~Емельянов поступил на работу в Институт автоматики и телемеханики, где прошел путь от
инженера до заместителя директора по науке. С~1976~г.\ работает в Институте системных исследований АН СССР (в
настоящее время Институт системного анализа РАН~--- ИСА РАН). С~1993 по 2003~гг.\ С.\,В.~Емельянов~--- директор
ИСА РАН; в настоящее время~--- научный руководитель ИСА РАН.

     Член-корреспондент АН СССР (1970), действительный член РАН (1984), ака\-де\-мик-сек\-ре\-тарь Отделения
информатики, вычислительной техники и автоматизации РАН (1992--2002). В~настоящее время~---
заместитель
ака\-де\-ми\-ка-сек\-ре\-та\-ря Отделения нанотехнологий и информационных технологий (ОНИТ РАН), руководитель
секции информационных технологий и автоматизации.

     Основные научные результаты С.\,В.~Емельянова относятся к теории систем переменной структуры; теории
бинарного управления и новых типов обратной связи; глобальной управляемости и стабилизации нелинейных систем;
технологии системного моделирования и системного проектирования средств автоматизации; геометрическим методам
анализа нелинейных систем; робастной устой\-чи\-вости и стабилизации неопределенных систем.

     С.\,В.~Емельянов~--- основатель известной научной школы. Он подготовил более 30~докторов и 70~кандидатов
наук; среди его учеников~--- академики и чле\-ны-кор\-рес\-пон\-ден\-ты РАН, члены других академий, руководители институтов, фирм.

     Является автором 25~книг и свыше 278 статей. Получил 72~патента на изобретения.

     Заведующий кафедрой нелинейных динамических систем и процессов управления факультета вычислительной
математики и кибернетики МГУ (с 1989~г.). Почетный профессор МГУ (1998), заслуженный профессор МГУ (1999).

     Лауреат Ленинской премии (1972), Государственной премии СССР (1980), премии Совета министров СССР (1981),
Государственной премии Российской Федерации (1994), Премии Президиума РАН им.\ акад.\ А.\,А.~Андронова (2000),
лауреат Ломоносовской премии МГУ по науке I~степени (2002), Премии Правительства РФ в области науки и технологий
(2009), Премии Правительства РФ в области образования (2012).

%\pagebreak

     С.\,В.~Емельянов награжден орденами Октябрьской Революции (1974), Дружбы народов (1979),
<<За заслуги перед
Отечеством>> III~степени (1999, 2004), Почета (2010), а также орденами Кирилла и Мефодия (Болгария), <<За заслуги>>
(Польша).

     С.\,В.~Емельянов является главным редактором журнала РАН <<Информатика и ее применения>>, осуществляя
общее руководство выработкой редакционной политики и процессом издания нашего журнала.

     Он является также главным редактором журналов РАН <<Информационные технологии и вы\-чис\-ли\-тель\-ные
системы>>, <<Искусственный интеллект и принятие решений>> и членом редакционных коллегий журналов РАН
<<Автоматика и телемеханика>>, <<Дифференциальные уравнения>>, <<Доклады РАН>>.

     С.\,В.~Емельянов~--- Председатель Совета по математике при Министерстве образования РФ; является членом
ученых и специализированных советов в МГУ и ИСА РАН.

     \bigskip

     Редакционный совет и Редакционная коллегия журнала <<Информатика и ее применения>> сердечно поздравляют
Станислава Васильевича Емельянова с юбилеем и желают ему крепкого здоровья и новых научных достижений.


%\label{end\stat}


\end{multicols}

\hrule

\vspace*{2pt}

\hrule

\vspace*{24pt}


%\def\stat{sokolov-u}

{\prg{\Large\textbf{К 60-летию директора Федерального государственного}}}\\[-4pt]

{\prg{\Large\textbf{бюджетного учреждения науки Института проблем}}}\\[-4pt]

{\prg{\Large\textbf{информатики Российской академии наук,}}}\\[-4pt]

{\prg{\Large\textbf{заместителя главного редактора журнала}}}\\[-4pt]

{\prg{\Large\textbf{<<Информатика и её применения>> академика}}}\\[-4pt]

{\prg{\Large\textbf{Российской академии наук И.\,А.~Соколова}}}

\vspace*{10mm}

%\def\titkol{К 60-летию академика Российской академии наук И.\,А.~Соколова}

%\def\autkol{К 60-летию академика Российской академии наук И.\,А.~Соколова} %И.\,А.~Соколов, С.\,Я.~Шоргин}
%\def\aut{\ } %И.\,А.~Соколов$^1$, С.\,Я.~Шоргин$^2$}

%\titel{\tit}{\aut}{\autkol}{\titkol}

%\vskip 14pt plus 9pt minus 6pt

%      \thispagestyle{headings}

%      \vspace*{6pt}

      \begin{multicols}{2}

%            \label{st\stat}

            \begin{center}
\mbox{%
\epsfxsize=78mm
\epsfbox{foto-sokol.eps}
}
\end{center}

\vspace*{6pt}




 27~марта 2014 года исполнилось 60~лет академику РАН И.\,А.~Соколову.

Игорь Анатольевич Соколов~--- известный ученый в области теоретической и
прикладной информатики, основатель научной школы в области информационных
технологий для распределенных\linebreak
 автоматизи\-рованных ин\-фор\-ма\-ци\-он\-но-управ\-ля\-ющих сис\-тем.

И.\,А.~Соколов окончил Московский государственный университет
им.\ М.\,В.~Ломоносова (факультет вычислительной математики и кибернетики) в
1976~г., аспирантуру там же~--- в 1979~г., работал в НИИ систем связи и
управления \mbox{ЦНПО} <<Каскад>>, с 1992~г.\ работает в Институте
проблем информатики
Российской академии наук (ИПИ РАН), с 1999~г.~--- директор ИПИ РАН.

В 2003~г.\ избран членом-корреспондентом РАН, в 2008~г.~--- академиком РАН.

В июне 2013~г.\ избран главным ученым секретарем Президиума РАН.

И.\,А.~Соколов опубликовал более 150~научных трудов, в том числе 7~монографий, он является автором 23~авторских свидетельств и патентов.

Основные научные результаты И.\,А.~Соколова связаны с разработкой инструментальных комплексов программных средств анализа и расчета
вероятностно-временн$\acute{\mbox{ы}}$х характеристик систем в рамках моделей с дискретным и непрерывным временем, обоснованием и разработкой принципов построения и системотехнических решений по архитектуре крупномасштабных информационных систем двойного применения, базовым информационным и телекоммуникационным технологиям, обеспечению информационной безопасности.

Научные результаты И.\,А.~Соколова позволили разработать, под его руководством и при его участии, специализированные информационные технологии, аппаратные и программные средства, комплексы, на основе которых создан ряд информационных систем национального масштаба.

В качестве Генерального конструктора руководит разработкой и развитием
системы информа\-ци\-онного обеспечения управления государством,
автоматизированной сис\-те\-мы управ\-ле\-ния и информационного обеспечения
принятия управленческих решений органов безопасности и системы распределенных
ситуационных центров, работа\-ющих по единому регламенту взаимодействия.
Член Научного совета при Совете Безопасности РФ, член президиума
На\-уч\-но-тех\-ни\-че\-ско\-го совета ВПК, председатель Совета РАН по
исследованиям в области обороны. Председатель диссертационных советов в ИПИ РАН
и в НИИ АА. Научные достижения И.\,А.~Соколова в области создания сис\-тем
информационного обеспечения безопасности мегаполиса отмечены Премией правительства
РФ (2004~г.). Награжден ведомственными наградами Совета Безопасности РФ и ГУСП
 Президента РФ.

Является заведующим кафедрой информационной безопасности факультета Вычислительной математики и кибернетики МГУ им.\ М.\,В.~Ломоносова и кафедрой проблем информатики МИРЭА.

И.\,А.~Соколов уделяет большое внимание организационной работе по
редактированию и изданию научных журналов. Он является заместителем главного
редактора журнала РАН <<Информатика и её применения>> и на этом посту
выполняет основную текущую работу по отбору статей в журнал, организации их
редактирования и публикации. И.\,А.~Соколов является также главным редактором
журнала РАН <<Системы и средства информатики>>, членом редакционной
коллегии журналов <<Информационные технологии и вычислительные системы>>,
<<Сис\-те\-мы высокой доступности>>, <<Право и Кибер\-без\-опас\-ность>>, членом
редакционного совета журнала <<Проб\-ле\-мы информатики>>.

\bigskip

Редакционный совет и Редакционная коллегия журнала <<Информатика и её применения>> сердечно поздравляют Игоря Анатольевича Соколова с 60-ле\-ти\-ем и желают крепкого здоровья и новых научных достижений.


%\label{end\stat}


\end{multicols}

%\newpage