\def\stat{sokolov-u}

\def\tit{К 60-летию директора Федерального государственного бюджетного учреждения науки Института проблем информатики Российской академии наук,
заместителя главного редактора журнала <<Информатика и её применения>>
академика Российской академии наук И.\,А.~Соколова}

\def\titkol{\textit{Юбилеи}}

\def\autkol{Юбилеи}
\def\aut{\ } %И.\,А.~Соколов$^1$, С.\,Я.~Шоргин$^2$}

\titel{\tit}{\aut}{\autkol}{\titkol}

\vskip 14pt plus 9pt minus 6pt

      \thispagestyle{headings}

%      \vspace*{6pt}

      \begin{multicols}{2}

            \label{st\stat}

            \begin{center}
\mbox{%
\epsfxsize=78mm
\epsfbox{foto-sokol.eps}
}
\end{center}

\vspace*{18pt}




 27~марта 2014 года исполнилось 60~лет академику РАН И.\,А.~Соколову.

Игорь Анатольевич Соколов~--- известный ученый в области теоретической и
прикладной информатики, основатель научной школы в области информационных
технологий для распределенных\linebreak
 автоматизи\-рованных ин\-фор\-ма\-ци\-он\-но-управ\-ля\-ющих сис\-тем.

И.\,А.~Соколов окончил Московский государственный университет
им.\ М.\,В.~Ломоносова (факультет вычислительной математики и кибернетики) в
1976~г., аспирантуру там же~--- в 1979~г., работал в НИИ систем связи и
управления \mbox{ЦНПО} <<Каскад>>, с 1992~г.\ работает в Институте
проблем информатики
Российской академии наук (ИПИ РАН), с 1999~г.\~--- директор ИПИ РАН.

В 2003~г.\ избран членом-корреспондентом РАН, в 2008~г.~--- академиком РАН.

В июне 2013~г.\ избран главным ученым секретарем Президиума РАН.

И.\,А.~Соколов опубликовал более 150~научных трудов, в том числе 7~монографий, он является автором 23~авторских свидетельств и патентов.

Основные научные результаты И.\,А.~Соколова связаны с разработкой инструментальных комплексов программных средств анализа и расчета
вероятностно-временн$\acute{\mbox{ы}}$х характеристик систем в рамках моделей с дискретным и непрерывным временем, обоснованием и разработкой принципов построения и системотехнических решений по архитектуре крупномасштабных информационных систем двойного применения, базовым информационным и телекоммуникационным технологиям, обеспечению информационной безопасности.

Научные результаты И.\,А.~Соколова позволили разработать, под его руководством и при его участии, специализированные информационные технологии, аппаратные и программные средства, комплексы, на основе которых создан ряд информационных систем национального масштаба.

В качестве Генерального конструктора руководит разработкой и развитием
системы информа\-ци\-онного обеспечения управления государством,
автоматизированной сис\-те\-мы управ\-ле\-ния и информационного обеспечения
принятия управленческих решений органов безопасности и системы распределенных
ситуационных центров, работа\-ющих по единому регламенту взаимодействия.
Член Научного совета при Совете Безопасности РФ, член президиума
На\-уч\-но-тех\-ни\-че\-ско\-го совета ВПК, председатель Совета РАН по
исследованиям в области обороны. Председатель диссертационных советов в ИПИ РАН
и в НИИ АА. Научные достижения И.\,А.~Соколова в области создания сис\-тем
информационного обеспечения безопасности мегаполиса отмечены Премией правительства
РФ (2004~г.). Награжден ведомственными наградами Совета Безопасности РФ и ГУСП
 Президента РФ.

Является заведующим кафедрой информационной безопасности факультета Вычислительной математики и кибернетики МГУ им.\ М.\,В.~Ломоносова и кафедрой проблем информатики МИРЭА.

И.\,А.~Соколов уделяет большое внимание организационной работе по
редактированию и изданию научных журналов. Он является заместителем главного
редактора журнала РАН <<Информатика и её применения>> и на этом посту
выполняет основную текущую работу по отбору статей в журнал, организации их
редактирования и публикации. И.\,А.~Соколов является также главным редактором
журнала РАН <<Системы и средства информатики>>, членом редакционной
коллегии журналов <<Информационные технологии и вычислительные системы>>,
<<Сис\-те\-мы высокой доступности>>, <<Право и Кибербезопасность>>, членом
редсовета журнала <<Проб\-ле\-мы информатики>>.

\bigskip

Редакционный совет и Редакционная коллегия журнала <<Информатика и её применения>> сердечно поздравляют Игоря Анатольевича Соколова с 60-ле\-ти\-ем и желают крепкого здоровья и новых научных достижений.


\label{end\stat}


\end{multicols}

%\newpage