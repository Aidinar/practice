
%\def\ss2{\mathop {\sum\limits^p\sum\limits^p}}
\def\mna{\mathrm{МНА}}
\def\mop{\mathrm{МОР}}
\def\op{\mathrm{ОР}}

\def\stat{sinits}

\def\tit{АНАЛИТИЧЕСКОЕ МОДЕЛИРОВАНИЕ
РАСПРЕДЕЛЕНИЙ С~ИНВАРИАНТНОЙ МЕРОЙ В~НЕГАУССОВСКИХ ДИФФЕРЕНЦИАЛЬНЫХ
И~ПРИВОДИМЫХ К НИМ ЭРЕДИТАРНЫХ СТОХАСТИЧЕСКИХ СИСТЕМАХ$^*$}

\def\titkol{Аналитическое моделирование
распределений с инвариантной мерой в~негауссовских дифференциальных
%и~приводимых к ним эредитарных       стохастических
системах}

\def\autkol{И.\,Н.~Синицын}

\def\aut{И.\,Н.~Синицын$^1$}

\titel{\tit}{\aut}{\autkol}{\titkol}

{\renewcommand{\thefootnote}{\fnsymbol{footnote}}
\footnotetext[1]{Работа выполнена при финансовой поддержке  программы
       <<Интеллектуальные информационные технологии, системный анализ
       и автоматизация>> (проект~1.7).}}

\renewcommand{\thefootnote}{\arabic{footnote}}
\footnotetext[1]{Институт проблем
информатики Российской академии наук, sinitsin@dol.ru}

\vspace*{-12pt}

\Abst{Представлены методы и алгоритмы аналитического моделирования одно- и
многомерных распределений с инвариантной мерой в дифференциальных и
интегродифференциальных (эредитарных)  стохастических системах (СтС),
описываемых уравнениями Ито в конечномерных пространствах с винеровскими
и пуассоновскими шумами. Сначала в разд.~2 рассматриваются интегродифференциальные
уравнения Пугачева для распределений процессов в дифференциальных СтС (ДСтС).
Применительно к ДСтС с гладкими и разрывными регулярными правыми частями найдены
условия сохранения инвариантной меры для нестационарных и стационарных процессов.
Сформулированы 4~теоремы, определяющие точные алгоритмы аналитического моделирования
распределений с инвариантной мерой в ДСтС общего вида. В~разд.~3 дан краткий обзор
приближенных методов аналитического моделирования в ДСтС,
основанных на параметризации распределений. Особое внимание
уделено методам нормальной аппроксимации и статистической линеаризации
для приближенного определения одно- и двумерных распределений с инвариантной
мерой. Получены условия устойчивости алгоритмов. Сформулированы две теоремы,
определяющие приближенные алгоритмы аналитического моделирования в ДСтС. Раздел~4
посвящен методам и алгоритмам аналитического моделирования распределений с
инвариантной мерой в интегродифференциальных эредитарных СтС (ЭСтС), приводимых
к дифференциальным. Представлены нелинейные стохастические интегродифференциальные
уравнения  Ито с винеровскими и пуассоновскими шумами. Для затухающих физически
возможных эредитарных ядер рассматривается два способа их аппроксимации
(на основе линейных операторных уравнений и вырожденных ядер). Рассмотрены
три теоремы, определяющие точные и приближенные алгоритмы приведения ЭСтС к ДСтС
для гладких и разрывных регулярных правых частей. В~приложении даны тестовые
примеры для разрабатываемого в ИПИ РАН инструментального программного обеспечения
<<ID StS>> в среде MATLAB. Заключение содержит основные выводы и возможные
обобщения. Рассмотрено применение результатов разд.~2--4 к задачам эквивалентности
гауссовских и негауссовских ДСтС и ЭСтС.}

\KW{аналитическое моделирование; гауссовская (нормальная) стохастическая
система; дифференциальная стохастическая система; инструментальное
программное обеспечение <<ID StS>>; метод нормальной аппроксимации;
метод статистической линеаризации; негауссовская (с винеровскими и
пуассоновскими шумами) стохастическая система; распределение с
инвариантной мерой; сингулярное (вырожденное) ядро; стохастическое
дифференциальное уравнение Ито; система, приводимая к
дифференциальной; эредитарное ядро}

\DOI{10.14357/19922264140201}

\vskip 14pt plus 9pt minus 6pt

      \thispagestyle{headings}

      \begin{multicols}{2}

            \label{st\stat}


\section{Введение}

%\vspace*{-4pt}

Вопросам разработки методов, алгоритмов и инструментальных
программных средств для анализа и моделирования  распределений
процессов в гауссовских (нормальных) ДСтС с инвариантной мерой
посвящена обширная литература (см., например,~[1--18]).
Методы анализа и моделирования распределений процессов с
инвариантной мерой в интегродифференциальных ЭСтС
подробно изложены в~[1, 11, 19--22].

Для ДСтС и ЭСтС, приводимых к ДСтС, с винеров\-скими и пуассоновскими
шумами соответствующие точные и приближенные методы аналитического
моделирования распределений с инвариантной мерой не разработаны.
Особое внимание уделяется приближенным методам аналитического
моделирования распределений с инвариантной мерой, основанным на
нормальной аппроксимации одно- и двумерных распределений.
Приводятся тестовые примеры.

\section{Уравнения для распределений процессов с~инвариантной мерой в~дифференциальных
стохастических системах}

Как известно~\cite{1-s, 2-s}, для ДСтС в
конечномерных пространствах используется дифференциальное стохастическое
уравнение Ито следующего \mbox{вида}:
    \begin{multline}
    dY= a (Y,t) \,dt + b(Y,t)\, d W_0 +{}\\
    {}+ \int\limits_{R_0^q} c(Y,t,v)\, dP^0 (t,dv)\,.
    \label{e2.1-s}
    \end{multline}
Здесь $Y$~--- $p$-мер\-ный вектор состояния, $Y\hm\in \Delta^y$
($\Delta^y$~--- многообразие состояний);
$a\hm=a (y,t)$ и $b\hm= b(y,t)$~--- известные $(p\times 1)$-мер\-ная и
$(p\times r)$-мер\-ная функции вектора~$Y$ и времени~$t$; $W_0\hm= W_0(t)$~---\linebreak
$r$-мер\-ный винеровский случайный процесс интенсивности $\nu_0\hm=
\nu_0(t)$; $c(y, t,v)$~--- $(p\times 1)$-мер\-ная функция~$y$, $t$ и
вспомогательного $(q\times 1)$-мер\-но\-го параметра~$v$;
$\int\limits_{\Delta_t}\, d P^0 (t,A)$~---
центрированная пуассоновская мера, удовлетворяющая условию:
 \begin{equation}
 \int\limits_{\Delta_t} d P^0 (t,A)= \int\limits_{\Delta_t} d P (t,A)-
    \int\limits_{\Delta_t} \nu_P (t,A) \,dt\,,
    \label{e2.2-s}
    \end{equation}
где $\int\limits_{\Delta_t} d P (t,A)$~--- число скачков пуассоновского
процесса $P (t,A)$ в интервале времени~$\Delta$; $\nu_P (t,A)$~--- интенсивность
пуассоновского процесса $P(t,A)$; $A$~--- некоторое борелевское
множество пространства~$R^q_0$ с выколотым началом координат.

Интеграл~(\ref{e2.1-s}) в общем случае распространяется на все пространство~$R_0^q$
с выколотым началом координат.
Начальное значение~$Y_0$ вектора~$Y$ пред\-став\-ля\-ет
собой случайную величину, не зависящую от приращений винеровского
процесса $W_0(t)$ и пуассоновского процесса $P(t,A)$ на интервалах
времени $\Delta_t \hm= (t_1, t_2]$, следующих за~$t_0$, $t_0\hm\le t_1\hm\le t_2$,
для любого множества~$A$.

В случае, когда подынтегральная функция $c(y,t,v)$ в уравнении~(\ref{e2.1-s})
допускает представление $c(y,t,v)\hm= b(y,t) c'(v)$,
уравнение~(\ref{e2.1-s}) приводится к  виду:
\begin{equation*}
{\dot Y} = a (Y, t) +b(Y, t)V\,, %\label{e2.3-s}
\end{equation*}
если принять
\begin{equation*}
V=\dot W\,;\enskip W(t)= W_0(t) +\int\limits_{R_0^q} c'(v) P^0 (t, dv)\,.
%\label{e2.4-s}
\end{equation*}

Пусть существуют одно- и $n$-мер\-ные плот\-но\-сти $f_1\hm=f_1(z;t)$ и
$f_n\hm= f_n(z_1\tr z_n; t_1 \tr t_n)$ и характеристические функции
$g_1\hm=g_1(\la;t)$ и $g_n\hm=g_n(\la_1\tr \la_n; t_1\tr t_n)$ $(n\hm\ge 2)$,
удовлетворяющие интегродифференциальным уравнениям Пугачева~\cite{1-s, 2-s}:
    \begin{multline}
    \fr{\prt f_1(z;t)}{\prt t}+\fr{\prt^T}{\prt z}\lk a(z,t)f_1(z;t)\rk ={}\\
    \!\!{}=
    \fr{1}{(2 \pi)^p}
    \iin\iin \!\!\!\chi^f(\la,\zeta,t) e^{i\la^{\mathrm{T}}(\zeta-z)} f_1(z;t) \,d\zeta d\la\,;\!\!
    \label{e2.5-s}
    \end{multline}

    \vspace*{-12pt}

    \noindent
\begin{equation}
f_1(z;t_0)=f_0(z)\,;\label{e2.6-s}
\end{equation}

\vspace*{-12pt}

    \noindent
\begin{multline*}
 \fr{\prt f_n(z_1\tr z_n;t_1\tr t_n)}{\prt t_n}+{}\\
 {}+
    \fr{\prt^{\mathrm{T}}}{\prt z_n}\left[a(z_n, t_n) f_n (z_1\tr z_n; t_1\tr t_n)\right]={}\\
{}= \fr{1}{(2\pi)^{pn}} \iin\iin \chi_n^f (\la_n, \zeta_n, t_n)
\times{}\\
{}\times \exp\!\lf i \sss_{l=1}^n \la_l^{\mathrm{T}} (\zeta_l-z_l)\rf \! f_n
    (\zeta_1\tr \zeta_n; t_1\tr t_n)\times{}\\
    {}\times d\zeta_1\cdots d\zeta_n d\la_1\cdots d\la_n\,;
%    \label{e2.7-s}
    \end{multline*}

    \vspace*{-12pt}

    \noindent
    \begin{multline*}
f_n(z_1\tr z_{n-1},z_n;t_1\tr t_{n-1},t_{n-1})={}\\
{}= f_{n-1} (z_1\tr z_{n-1};t_1\tr t_{n-1})\delta (z_n - z_{n-1})\,,\\
t_1\le t_2 \le \cdots \le t_n\,,\enskip n=2,3,\ldots;
%        \label{e2.8-s}
        \end{multline*}


\vspace*{-12pt}

    \noindent
\begin{multline}
\fr{\prt g_1 (\la;t)}{\prt t} -{}\\
{}-\fr{1}{(2\pi)^p}
    \iin \iin i\la^{\mathrm{T}} a (z,t) e^{i(\la^{\mathrm{T}} -
    \mu^{\mathrm{T}})z} g_1 (\mu;t) \,d\mu dz={}\\
{}=\fr{1}{(2\pi)^k}\! \iin \iin \! \!\chi^g(\la, z,t) e^{i(\la^{\mathrm{T}} -
\mu^{\mathrm{T}})z} \times{}\\
{}\times g_1 (\mu;t)\, d\mu dz\,;
    \label{e2.9-s}
    \end{multline}
\begin{equation}
g_1(\la;t_0) = g_0(\la)\,;
    \label{e2.10-s}
    \end{equation}

    \vspace*{-12pt}

    \noindent
\begin{multline*}
 \fr{\prt g_n (\la_1\tr \la_n; t_1\tr t_n)}{\prt t_n} -
    \fr{1}{(2\pi)^{pn}} \times{}\\
    {}\times \iin\! \cdots \!\iin i\la^{\mathrm{T}} a (z_n,t_n)
    \exp \lk i \sss_{k=1}^n (\la_k^{\mathrm{T}} - \mu_k^{\mathrm{T}}) z_k\rk\times{}\\
    {}\times g_n
    (\mu_1\tr \mu_n; t_1\tr t_n) d\mu_1 \ldots d \mu_n dz_1\ldots dz_n={}\\
{}= \fr{1}{(2\pi)^{pn}} \iin\ldots \iin \chi^n (\la_n, z_n,t_n)\times{}\\
{}\times\exp \lk i \sss_{k=1}^n (\la_k^{\mathrm{T}} - \mu_k^{\mathrm{T}}) z_k\rk \times{}\\
{}\times g_n
     (\mu_1\tr \mu_n; t_1\tr t_n)\, d\mu_1 \cdots d \mu_n dz_1\cdots dz_n\,;
%     \label{e2.11-s}
     \end{multline*}


\vspace*{-12pt}

    \noindent
\begin{multline*}
 g_n (\la_1\tr \la_n; t_1\tr t_{n-1},t_{n-1})={}\\
 {}=
    g_{n-1} (\la_1\tr \la_{n-2},\la_{n-1}+\la_n; t_1\tr t_{n-1})\\
t_1\le t_2 \le\cdots \le t_n\,, \enskip n=2,3,\ldots
%    \label{e2.12-s}
    \end{multline*}
Здесь приняты следующие обозначения:
\begin{multline*}
    \chi^f (\la,\zeta, t) =-\fr{1}{2}\, \la^{\mathrm{T}}
    b(\zeta,t) \nu_0(t) b(\zeta,t)^{\mathrm{T}}\lambda +{}\\
{}+ \iii_{R_0^q}\left\{ \exp \lk i\la^{\mathrm{T}} c(\zeta,t,v) \rk
    -1 -{}\right.\\
\left.    {}-i\la^{\mathrm{T}} c(\zeta, t,v)\right\} \nu_P (t,dv)\,;
%    \label{e2.13-s}
    \end{multline*}

\vspace*{-14pt}

    \noindent
\begin{multline*}
    \chi^f_n (\la_n,\zeta_n, t_n) =-\fr{1}{2}\,
    \la^{\mathrm{T}}_n b(\zeta_n,t) \nu_0(t) b(\zeta_n,t)^{\mathrm{T}}\lambda +{}\\
{}+ \iii_{R_0^q} \left\{ \exp \lk i\la^{\mathrm{T}}_n c(\zeta_n,t_n,v) \rk -
    1 -{}\right.\\
\left.    {}-i\la^{\mathrm{T}}_n c(\zeta_n, t_n,v)\right\} \nu_P (t_n,dv)\,;
%    \label{e2.14-s}
    \end{multline*}

    \vspace*{-14pt}

    \noindent
\begin{multline}
        \chi^g (\la,z, t) =-\fr{1}{2}\, \la^{\mathrm{T}} b(z,t) \nu_0(t) b(z,t)^{\mathrm{T}}\lambda +{}\\
{}+ \iii_{R_0^q} \left\{ \exp \lk i\la^{\mathrm{T}} c(z,t,v) \rk
    -1 -{}\right.\\
\left.    {}-i\la^{\mathrm{T}} c(z, t,v)\right\} \nu_P (t,dv)\,;
    \label{e2.15-s}
    \end{multline}


    \vspace*{-14pt}

    \noindent
\begin{multline*}
\chi^g_n (\la_n,z_n, t_n) =-\fr{1}{2}\, \la^{\mathrm{T}}_n b(z_n,t) \nu_0(t) b(z_n,t)^{\mathrm{T}}\lambda +{}\\
{}+ \iii_{R_0^q} \left\{ \exp \lk i\la^{\mathrm{T}}_n c(z_n,t_n,v) \rk
    -1 -{}\right.\\
\left.    {}-i\la^{\mathrm{T}}_n c(z_n, t_n,v)\right\} \nu_P (t_n,dv)\,.
%    \label{e2.16-s}
    \end{multline*}
При этом одно- и $n$-мер\-ные плотности и характеристические функции связаны
между собой известными соотношениями:
\begin{align*}
f_1(z;t) &=\displaystyle \fr{1}{(2\pi)^{p}} \iin e^{-i\mu^{\mathrm{T}} z} g_1(\mu;t)\, d\mu\,;\\
    g_1(\la;t) &= \iin e^{i\la^{\mathrm{T}} z} f_1(z;t) \,dz\,;
%    \label{e2.17-s}
    \end{align*}

    \vspace*{-14pt}

    \noindent
\begin{multline*}
f_n( z_1\tr z_n; t_1\tr t_n) ={}\\
{}=\fr{1}{(2\pi)^{pn}} \iin\cdots \iin \exp
    \lf - i \sss_{l=1}^n \la_l^{\mathrm{T}} z_l\rf \times{}\\
{}\times g_n (\la_1\tr \la_n; t_1\tr t_n)\, d\la_1\cdots \la_n\,;
\end{multline*}

\vspace*{-12pt}

\noindent
\begin{multline*}
g_n (\la_1\tr \la_n; t_1\tr t_n) ={}\\
{}=\iin\cdots \iin \exp
    \lf i \sss_{l=1}^n \la_l^{\mathrm{T}} z_l\rf \times{}\\
{}\times f_n (z_1\tr z_n; t_1\tr t_n) \,dz_1\cdots dz_n\,.
%    \label{e2.18-s}
    \end{multline*}

Для нахождения одномерных плотностей $f_1(z,t) \hm= f_1^* (z)$ и характеристических
функций $g_1(\la;t) \hm= g_1^* (\la)$ стохастических процессов в стационарных
ДСтС~(\ref{e2.1-s}), когда
    \begin{equation}
    \left.
    \begin{array}{c}
    a(z,t) = a^*(z)\,;\quad b(z,t)=b^*(z)\,;\\[9pt]
    \chi(\mu;t)= \chi^f(\mu,\zeta, t)={\chi*}^f (\mu, \zeta)\,,
    \end{array}
    \right\}
    \label{e2.19-s}
    \end{equation}
    в~(\ref{e2.5-s}) и (\ref{e2.9-s}) следует положить
$\prt f_1/\prt t \hm= 0$ и $\prt g_1/ \prt t\hm =0$.
В~результате получим соответственно сле\-ду\-ющие интегродифференциальные уравнения:
    \begin{multline*}
    \fr{\prt^{\mathrm{T}}}{\prt z}\lk a^* (z) f_1^* (z)\rk ={}\\
    {}=
    \fr{1}{(2\pi)^p} \iin \iin {\chi^*}^f (\la,\zeta) e^{i\la^{\mathrm{T}}(\zeta-z)} f_1^*
    (\zeta)\, d\zeta d\la\,; %\label{e2.20-s}
    \end{multline*}

    \vspace*{-12pt}

    \noindent
\begin{multline*}
-\fr{1}{(2\pi)^p} \iin  \iin i\la^{\mathrm{T}} a^*(z) e^{i(\la^{\mathrm{T}}-\mu^{\mathrm{T}})z} g_1^*(\mu)
   \, d\mu dz={}\\
{}=\fr{1}{(2\pi)^p} \iin  \iin \chi^{*g} (\la, z) e^{i(\la^{\mathrm{T}}-\mu^{\mathrm{T}})z}
    g_1^*(\mu) \,d\mu dz\,. %\label{e2.21-s}
    \end{multline*}

Пусть функция $a$ в ДСтС~(\ref{e2.1-s}) допускает пред\-став\-ле\-ние
    \begin{equation}
    a= a(z,t) = a_1(z,t) +a_2 (z,t)
    \label{e2.22-s}
\end{equation}
такое, что функция  $f_1\hm=f_1(z;t)$ является плот\-ностью инвариантной
меры не возмущенной шумами сис\-те\-мы, описываемой векторным
обыкновенным дифференциальным уравнением вида
\begin{equation}
\dot z = a_1 (z,t)\,,
\label{e2.23-s}
\end{equation}
т.\,е.\ удовлетворяет следующему условию сохранения инвариантной меры:
    \begin{equation}
   \fr{\prt f_1 (z;t)}{\prt t}+ \fr{\prt^{\mathrm{T}}}{\prt z} \lk
   a_1 (z,t) f_1(z;t)\rk =0\,.\label{e2.24-s}
   \end{equation}

Для гладких функций $a_1\hm=a_1(z,t)$ вопросы существования и основные
свойства интегральных инвариантов и инвариантных мер изучены в~\cite{23-s, 24-s}.
При этом  функция $a_2 \hm= a_2(z,t)$ в~(\ref{e2.22-s})
определяется путем решения следующего интегродифференциального уравнения:
    \begin{multline}
    \fr{\prt^{\mathrm{T}}}{\prt z}\lk a_2 (z,t) f_1(z;t) \rk ={}\\
    \hspace*{-4mm}{}=
    \fr{1}{(2\pi)^k} \iin\iin \chi^f
    (\la,\zeta,t) e^{i\la^{\mathrm{T}}(\zeta-z)} f_1(\zeta;t) \,d\zeta d\la\,.\!
    \label{e2.25-s}
    \end{multline}

Для стационарных ДСтС, когда выполнены условия~(\ref{e2.19-s}),
уравнения~(\ref{e2.22-s})--(\ref{e2.24-s}) имеют вид:
    \begin{gather}
    a(z)= a_1(z) + a_2(z)\,;\label{e2.26-s}
\\
    \dot z = a_1(z)\,;\label{e2.27-s}
\\
    \fr{\prt^{\mathrm{T}}}{\prt z}\lk a_2^*(z) f_1^*(z)\rk = 0\,,
    \label{e2.28-s}
\end{gather}

\vspace*{-12pt}

\noindent
\begin{multline}
\fr{\prt^{\mathrm{T}}}{\prt z}\lk a_2^*(z) f_1^*(z)\rk ={}\\
{}= \fr{1}{(2\pi)^p}
    \iin \iin {\chi^*}^f (\la,\zeta) e^{i\la^{\mathrm{T}}(\zeta-z)} f_1^*(\zeta)\, d\zeta d\la\,.
    \label{e2.29-s}
\end{multline}

Условия сохранения инвариантной меры можно представить в следующем
развернутом виде:
    \begin{equation*}
    \fr{\prt f_1 (z;t)}{\prt t} + A_a f_1 (z;t) =0\,;
    \end{equation*}

\vspace*{-12pt}

\noindent
    \begin{align*}
    A_a f_1(z;t)&=\fr{\prt^{\mathrm{T}}}{\prt z} \lk a_1(z,t) f_1(z;t)\rk =
    \mathrm{div}\, \pi(z;t)\,;
%    \label{e2.30-s}
\\
    A_a^* f_1^* (z) &=0\,;\\
    A_a^* f_1(z)&=\fr{\prt^{\mathrm{T}}}{\prt z} \lk a_1^* (z) f_1^* (z)\rk =
    \mathrm{div}\, \pi^* (z)\,;
%    \label{e2.31-s}
    \end{align*}
    \begin{equation}
    \left.
    \begin{array}{rl}
   \displaystyle \fr{\prt g_1 (\la;t)}{\prt t} - B_a g_1(\la;t) &=0\,;\\[9pt]
    B_a g_1(\la;t) &={}\\[9pt]
    &\hspace*{-30mm}{}=   \displaystyle\fr{1}{(2\pi)^p} \iin\iin\! i\la^{\mathrm{T}} a_1(z,t)
    e^{i(\la^{\mathrm{T}}-\mu^{\mathrm{T}})z}\times{}\\[9pt]
    &\hspace*{1mm}{}\times g_1(\mu;t)\, d\mu dz={}\\[9pt]
&\hspace*{-23mm}{}=    \displaystyle\iin i\la^{\mathrm{T}} a(z,t) e^{i\la^{\mathrm{T}}z} f_1(z;t)\, dz={}\\[9pt]
&\hspace*{-12mm}{}=
        \displaystyle\iin e^{i\la^{\mathrm{T}} z} i\la^{\mathrm{T}} \pi(z;t)\, dz\,;
     \end{array}
     \right\}
    \label{e2.32-s}
\end{equation}

    \vspace*{-12pt}

    \noindent
\begin{multline}
B_a^* g_1^* (\la)=0\,;\quad
    B_a^* g_1^* (\la) = {}\\
    {}=\fr{1}{(2\pi)^p} \iin i\la^{\mathrm{T}} a_1^* (z)
    e^{i(\la^{\mathrm{T}} -\mu^{\mathrm{T}})z} g_1^* (\mu) \,d\mu dz={}\\
\!{}=\iin\! i\la^{\mathrm{T}} a_1^*(z) e^{i\la^{\mathrm{T}}z} f_1^* (z)\, dz ={}\\
{}=
    \iin \!e^{i\la^{\mathrm{T}} z} i\la^{\mathrm{T}} \pi^* (z)\, dz\,.\!\!
    \label{e2.33-s}
\end{multline}

Для разрывных функций $a_1 (z,t)$ в терминах характеристических функций
соотношения~(\ref{e2.24-s}) и~(\ref{e2.28-s}) могут быть записаны в
виде~(\ref{e2.32-s}) и~(\ref{e2.33-s}). При этом для
составляющих $a_2(z,t)$ и $a_2^*(z)$ имеют место уравнения:
    \begin{align}
    B_{a_2} g_1(\la;t) &=\notag\\[6pt]
    &\hspace*{-18mm}{}=\fr{1}{(2\pi)^p}\! \!\iin\iin\!\! \chi^g(\la,z,t)
    e^{i(\la^{\mathrm{T}}-\mu^{\mathrm{T}})z} g_1(\mu;t)\, d\mu dz;\!\!\!
    \label{e2.34-s}\\[9pt]
    B_{a_2}^* g_1^*(\la) &={}\notag\\[6pt]
    &\hspace*{-18mm}{}=\fr{1}{(2\pi)^p} \!\!\iin\iin \!\!{\chi^*}^g(\la,z)
    e^{i(\la^{\mathrm{T}}-\mu^{\mathrm{T}})z} g_1^*(\mu)\, d\mu dz.\!\!\!
    \label{e2.35-s}
\end{align}

Отсюда вытекают  точные алгоритмы аналитического
моделирования распределений с инвариантной мерой. В~их основе лежат
следующие тео\-ремы.
{\looseness=1

}

\smallskip

\noindent
\textbf{Теорема 1.}\ \textit{Функция $f_1\hm=f_1(z;t)$ будет решением~$(\ref{e2.5-s})$
и~$(\ref{e2.6-s})$ тогда и только тогда, когда $a\hm=a(z,t)$ допускает
представление~$(\ref{e2.22-s})$ такое, что $f_1\hm=f_1(z;t)$ \mbox{является} плотностью
инвариантной меры обыкновенного дифференциального уравнения~$(\ref{e2.23-s})$,
т.\,е.\ удовле\-тво\-ря\-ет условию~$(\ref{e2.24-s})$. При этом со\-став\-ля\-ющая~$a_2$
определяется из решения интегродифференциального уравнения}~(\ref{e2.25-s}).

\smallskip

\noindent
\textbf{Теорема 2.}\ \textit{Функция $f_1^*\hm=f_1^*(z)$ будет решением~$(\ref{e2.5-s})$
тогда и только тогда, когда $a^*\hm=a^*(z)$ допускает
представление~$(\ref{e2.26-s})$ такое, что $f_1^*\hm=f_1^*(z)$ является плотностью
инвариантной меры~$(\ref{e2.27-s})$. При этом составляющая~$a_2^{*}$
определяется из решения  уравнения}~(\ref{e2.29-s}).

\smallskip

\noindent
\textbf{Теорема 3.} \textit{Функция $g_1\hm=g_1(\la;t)$ будет решением~$(\ref{e2.9-s})$,
$(\ref{e2.10-s})$ тогда и только тогда, когда недифференцируемая функция
$a\hm=a(z,t)$  допускает пред\-став\-ле\-ние~$(\ref{e2.22-s})$ такое, что
$g_1\hm=g_1(\la;t)$ является характеристической функцией инвариантной
меры уравнения~$(\ref{e2.23-s})$, т.\,е.\
удовлетворяет условию~$(\ref{e2.29-s})$. При этом
со\-став\-ля\-ющая~$a_2$ определяется из уравнения}~(\ref{e2.34-s}).

\smallskip

\noindent
\textbf{Теорема 4.} \textit{Функция $g_1^*\hm=g_1^*(\la)$  будет
решением~$(\ref{e2.28-s})$ тогда и только тогда, когда недифференцируемая
функция $a^*\hm=a^*(z)$  допускает представление~$(\ref{e2.26-s})$
такое, что $g_1^*$ является  характеристической функцией инвариантной меры
уравнения~$(\ref{e2.23-s})$. При этом~$a_2^*$ определяется из решения}~(\ref{e2.35-s}).

\smallskip

Теоремы~1--4 легко обобщаются на случай многомерных распределений
с инвариантной мерой.

\section{Приближенные методы и~алгоритмы аналитического моделирования
распределений процессов с~инвариантной мерой в~дифференциальных стохастических
системах}

Пусть нелинейная ДСтС~(\ref{e2.1-s})
допускает применение метода нормальной аппроксимации (МНА)~\cite{1-s, 2-s}.
Тогда одно- и двумерные нормальные плотности $f_1^\mna$,
 $f_2^\mna$ и характеристические функции  $g_1^\mna$,  $g_2^\mna$,
 а также вектор математического ожидания $m_t \hm= {\sf M}^\mna Z(t)$, ковариационная
 мат\-ри\-ца $K_t \hm= {\sf M}^\mna Z^{0T} Z^0 (t)$ $(Z^0 (t) \hm= Z(t) \hm- m_t)$
 и мат\-ри\-ца ковариационных функций
 $K(t_1, t_2) \hm= {\sf M}^\mna Z^{0T} (t_1) Z^0 (t_2)$ $(t_1< t_2)$
 определяются следующими уравнениями:
    \begin{multline}
    f_1^\mna = f_1^\mna (z;t, m_t, K_t) =
    \left[ (2\pi)^p |K_t|\right]^{-1/2}\times{}\\
    {}\times
     \exp \lf -\fr{1}{2} (z^{\mathrm{T}} - m_t^{\mathrm{T}}) K_t^{-1}
    (z-m_t)\rf\,;
    \label{e3.1-s}
    \end{multline}


\vspace*{-12pt}

    \noindent
    \begin{multline*}
    f_2^\mna = {}\\
    {}=f_2^\mna (z_1, z_2;t_1, t_2, m_{t_1}, m_{t_2}, K_{t_1},
    K_{t_2}, K(t_1, t_2))={}\hspace*{-0.62715pt}\\
{}=\left[ \left(2\pi\right)^p |\bar K_2|\right]^{-1/2}  \exp
\left( - \left( \left[ z_1^{\mathrm{T}} z_2^{\mathrm{T}} \right] -
    \bar m_2^{\mathrm{T}}\right)\right.\times{}\\
    \left.{}\times \bar K_2^{-1}\left(\left[
    z_1^{\mathrm{T}} z_2^{\mathrm{T}}\right]^{\mathrm{T}}-\bar m_2\right)\right)\,;
    \label{e3.2-s}
    \end{multline*}
    \begin{equation}
    g_1^\mna (\la;t)=\exp\lf i\la^{\mathrm{T}} m- \fr{1}{2}\, \la^{\mathrm{T}} K_t \la\rf\,;
    \label{e3.3-s}
    \end{equation}

    \vspace*{-12pt}

    \noindent
\begin{multline}
    g_2^\mna (\la_1, \la_2; t_1,t_2) = {}\\
    {}=\exp \lf i \bar \la^{\mathrm{T}} \bar m_2 -
    \fr{1}{2}\, \bar \la^{\mathrm{T}} \bar K_2 \bar \la\rf\,,\enskip
\bar \la =\lk \la_1^{\mathrm{T}} \la_2^{\mathrm{T}}\rk^{\mathrm{T}}\,;
    \label{e3.4-s}
    \end{multline}


\vspace*{-12pt}

\noindent
\begin{multline}
\dot m_t = \Phi_1 (t, m_t, K_t) ={}\\
{}=\iin a(z,t) f_1^\mna (z; t, m_t, K_t) \,dz\,;
    \label{e3.5-s}
    \end{multline}

\vspace*{-12pt}

\noindent
\begin{multline}
\dot K_t = \Phi_2(t, m_t, K_t) = \Phi_{21} + \Phi_{12}+\Phi_{22}={}\\
{}=\left[ \iin a(z,t) (z^{\mathrm{T}}-m_t^{\mathrm{T}}) + (z-m_t)
a^{\mathrm{T}} (z,t) +{}\right.\\
\left.{}+ \vphantom{\iin}
    \bar \si (z,t)\right] f_1^\mna (z;t, m_t, K_t)\, dz\,,\label{e3.6-s}
    \end{multline}

    \vspace*{-12pt}

    \noindent
\begin{multline*}
\fr{\prt K(t_1, t_2)}{\prt t_2} ={}\\
{}= \Phi_3 (t_1, t_2, m_{t_1},m_{t_2},
    K_{t_1}, K_{t_2}, K(t_1,t_2))={}
    \end{multline*}

    \noindent
    \begin{multline}
    {}=
    \lk (2\pi)^{2p} |\bar K_2|\rk^{-1/2}
    \iin\iin \left(z_1-m_{t_1}\right) a\left(z_2, t_2\right)\times{}\\
    {}\times
    \exp\left\{ - \left(\left[z_1^{\mathrm{T}} z_2^{\mathrm{T}}\right]-
    \bar m_2^{\mathrm{T}}\right)\bar K_2^{-1} \times{}\right.\\
\left.    {}\times
    \left(\left[z_1^{\mathrm{T}} z_2^{\mathrm{T}}\right]-
    \bar m_2\right)\right\}\, dz_1 dz_2={}\\
{}= K(t_1, t_2) K(t_2)^{-1} \Phi_{21} (m(t_2), K(t_2), t_2)^{\mathrm{T}}.
    \label{e3.7-s}
    \end{multline}
Здесь введены следующие обозначения:
    \begin{equation}
    \left.
    \begin{array}{c}
    z_1=z_{t_1}\,;\enskip
    z_2=z_{t_2}\,;\enskip \bar m_2 =\lk m_{t_1}^{\mathrm{T}} m_{t_2}^{\mathrm{T}}\rk^{\mathrm{T}}\,;\\[9pt]
    \bar K_2 =\begin{bmatrix}
        K(t_1,t_1)&K(t_1, t_2)\\[3pt]
        K(t_2, t_1)& K(t_2, t_2)\end{bmatrix}\,;
        \end{array}
        \right\}
        \label{e3.8-s}
\end{equation}
    \begin{equation}
    \left.
    \begin{array}{rl}
    \bar \si (z,t) &={}\\
&\hspace*{-10mm}{}=\displaystyle
    \si(z,t) +\iii_{R_0^q} c(z,t,v) c(z,t,v)^{\mathrm{T}} \nu_P (t,dv)\,;
    \\[9pt]
     \si(z,t) &= b(z,t) \nu_0(t) b(z,t)^{\mathrm{T}}\,.
     \end{array}
     \right\}
    \label{e3.9-s}
    \end{equation}

Для стационарных ДСтС  при $\dot m^* \hm=0$, $\dot K^* \hm=0$,
$K(t_1, t_2)\hm= k(\tau)$
$(\tau\hm=t_1-t_2)$  соотношения~(\ref{e3.5-s})--(\ref{e3.9-s}) принимают вид:
    \begin{equation}
    \Phi_1^* (m^*, K^*) =0\,;\label{e3.10-s}
    \end{equation}
\begin{equation}
\Phi_2^*(m^*, K^*) =0\,;
\label{e3.11-s}
\end{equation}
\begin{equation}
\fr{dk(\tau)}{ d\tau} = \Lambda k(\tau)\,;\enskip
    \Lambda =\Phi_{21}(m^*, K^*) K^{*-1} k(\tau)\,;
    \label{e3.12-s}
    \end{equation}
    $$k(\tau) = k(-\tau)^{\mathrm{T}},\enskip k(0)=K\,.$$
Из~(\ref{e3.12-s}) следует, что алгоритм МНА будет устойчивым,
если матрица $\Lambda^* \hm= \Lambda (m^*, K^*)\hm=\Phi_{21}(m^*, K^*) K^{* -1}$
будет асимптотически устойчива.

Уравнения метода статистической линеаризации (МСЛ) в нелинейных ДСтС  при
аддитивных шумах, когда $b(z,t) \hm= b_0(t)$, $b^*(z)\hm=b_0^*$
получаются из~(\ref{e3.5-s})--(\ref{e3.7-s}) и~(\ref{e3.10-s})--(\ref{e3.12-s})
как частный случай.

Условия наличия нормального распределения с инвариантной мерой,
если заменить $a(z,t)$ статистически линеаризованным выражением вида:
    \begin{equation*}
    a(Z,t)\approx a_{10}^\mna (t, m_t, K_t) + a_{11}^\mna (t, m_t, K_t) (Z-m_t)\,,
%    \label{e3.13-s}
    \end{equation*}
где
\begin{equation*}
a_{10}^\mna =a_{10}^\mna (t, m_t, K_t)\,; %\label{e3.14-s}
\end{equation*}

%\vspace*{-12pt}

    \noindent
\begin{multline*}
a_{11}^\mna=a_{11}^\mna (t, m_t, K_t) ={}\\
\hspace*{-3.71278pt}{}=  \!  \left[ \iin\! \!a(z,t) (z^{\mathrm{T}}-m_t^{\mathrm{T}}) f_1^\mna (z; t , m_t, K_t)\, dz\right]\! K_t^{-1} ={}\\
{}=\lk \fr{\prt}{\prt m_t}\left( a_{10}^\mna\right)^{\mathrm{T}}\rk^{\mathrm{T}}\,,
%\label{e3.15-s}
\end{multline*}
приводят к следующим соотношениям:
\begin{multline}
\fr{\prt f_1^\mna (z; t, m_t, K_t)}{\prt t} +{}\\
{}+\fr{\prt^{\mathrm{T}}}{\prt z} \left\{\! \left[ a_{10}^\mna (t, m_t, K_t) +
     a_{11}^\mna (t, m_t, K_t)\times{}\right.\right.\\
\left.\left.     {}\times (z-m_t)
\vphantom{a_{10}^\mna}
\right]
      f_1^\mna ( z; t , m_t, K_t)\right\} =0\,;
     \label{e3.16-s}
     \end{multline}

\vspace*{-12pt}

\noindent
\begin{multline}
\fr{\prt^{\mathrm{T}}}{\prt z} \left\{ \left[ a_{10}^{*\mna}(m^*, K^*) +
    a_{11}^{*\mna}(m^*, K^*)\times{}\right.\right.\\
\left.\left.    {}\times (z-m^*)
\vphantom{a_{10}^\mna}
\right] f_1^{*\mna}(z; m^*, K^*)\right\} =0\,,
    \label{e3.17-s}
    \end{multline}
где

\noindent
\begin{multline*}
    f_1^{*\mna} (z; m^*, K^*) = \lk (2\pi)^p |K^*|\rk^{-1/2}\times{}\\
    {}\times
    \exp \left\{ -\fr{1}{2} \left(z^{\mathrm{T}}-m^{*T}\right)(K^*)^{-1} (z-m^*)\right\}.
    \end{multline*}

Аналогично в развернутом виде выписываются условия~(\ref{e2.32-s})
и~(\ref{e2.33-s}):

\noindent
    \begin{multline}
    \fr{\prt g_1^\mna (\la;t)}{\prt t} -\iin i\la^{\mathrm{T}}
    \left[ a_{10}^\mna (m_t, K_t, t) +{}\right.\\
\left.    {}+ a_{11}^\mna (m_t, K_t, t) (z- m_t) \right]\times{}\\
{}\times e^{i\la^{\mathrm{T}} z} f_1^\mna (z; m_t, K_t, t) dz=0\,,
    \label{e3.18-s}
    \end{multline}

    \vspace*{-14pt}

    \noindent
\begin{multline}
\!\!\!\!\!\!\iin \!\!\!i\la^{\mathrm{T}}\! \left[ a_{10}^{*\mna } (m^*, \!K^*) +  a_{11}^{*\mna } (m^*, K^*)
(z-m^*)\right]\times{}\\
{}\times
e^{i\la^{\mathrm{T}}z} f_1^{*\mna } (z; m^*, K^*) dz = 0\,.
    \label{e3.19-s}
    \end{multline}

Отсюда вытекают следующие теоремы, лежащие в основе приближенных
нелинейных методов.

\smallskip

\noindent
\textbf{Теорема 5.} \textit{Если существуют одно- и двумерные  плотности
стохастического процесса, а  матрица $a_{11}^\mna$ коэффициентов
статистической линеаризации асимптотически устойчива,
то приближенный \mbox{алгоритм} аналитического моделирования МНА
нестационарных стохастических процессов в ДСтС~$(\ref{e2.1-s})$ с инвариантной
мерой определяется выражениями~$(\ref{e3.1-s})$--$(\ref{e3.7-s})$ и}~(\ref{e3.16-s}).

\smallskip

\noindent
\textbf{Теорема 6.} \textit{Если существуют стационарные одно- и
двумерные плотности стохастического процесса, а матрица
$a_{11}^{*\mna}$ коэффициентов статистической линеаризации
асимптотически устойчива, то приближенный алгоритм аналитического
моделирования стационарных стохастических процессов с инвариантной
мерой в стационарной ДСтС~$(\ref{e2.1-s})$ определяется
выражениями}~$(\ref{e3.10-s})$--$(\ref{e3.12-s})$ \textit{и}~(\ref{e3.17-s}).

\smallskip

Как известно~\cite{1-s, 2-s}, одно- и двумерные нормальные распределения
определяют и все  $n$-мер\-ные распределения $(n\hm> 3)$. Поэтому МНА и
МСЛ  при $b(Y,t)\hm=b_0(t)$, $c(Y,t,z) \hm=c_0 (t,v)$ дают приближенные
алгоритмы для любых многомерных плот-\linebreak\vspace*{-12pt}
\columnbreak

\noindent
ностей стохастических процессов,
если они существуют. Аналогично формулируются теоремы~3.3 и~3.4 в
терминах характеристических функций на основе условий~(\ref{e3.18-s}) и~(\ref{e3.19-s}).

 Обобщением МНА являются различные
приближенные методы, основанные на параметризации распределений~\cite{1-s, 2-s}.
Аппроксимируя одномерную характеристическую функцию $g_1 (\la;t)$
и соответствующую плотность $f_1 (z,t)$ известными функциями
 $g_1^* (\la;\theta)$ и $f_1^* (z;\theta)$,  зависящими от
конечномерного векторного параметра~$\theta$, можно свести задачу
приближенного определения одномерного распределения к выводу из
уравнения для характеристических функций обыкновенных
дифференциальных уравнений, определяющих~$\theta$ как функцию
времени. Это относится и к остальным многомерным распределениям.

При аппроксимации многомерных распределений целесообразно выбирать
последовательности функций $\{ f_n^* (z_1,\ldots,z_n;\theta_n)\}$ и
$\{g_n^* (\la_1\tr \la_n;\theta_n)\}$, каждая пара
которых находилась бы в такой  зависимости от векторного параметра~$\theta_n$,
чтобы при любом $n$ множество параметров, образу\-ющих
вектор~$\theta_n$, включало в качестве подмножества множество
параметров, образующих вектор~$\theta_{n-1}$. То-\linebreak гда при
аппроксимации $n$-мер\-но\-го распределения придется определять только
те координаты вектора~$\theta_n$, которые не были определены ранее
при аппроксимации функций $f_1, g_1\tr f_{n-1}$, $g_{n-1}$. %\linebreak
В зависимости от того, что представ\-ляют собой параметры, от
которых зависят функции $f_n^* (z_1\tr z_n;\theta_n)$ и $g_n^*
(\la_1\tr \la_n;\theta_n)$, аппроксими\-ру\-ющие неизвестные
многомерные плотности $f_n (z_1,  \ldots,z_n; t_1 \tr t_n)$ и
характеристические функции $g_n (\la_1\tr \la_n; t_1,\ldots,t_n)$,
используются различные приближенные методы решения
 уравнений, определяющих многомерные
распределения вектора состояния системы~$X_t$, в частности методы
моментов, семиинвариантов, ортогональных разложений, квазимоментов
и~др.~\cite{1-s, 2-s} и в условиях сохранения инвариантной меры.

\vspace*{-20pt}

\section{Анализ и~моделирование распределений с~инвариантной мерой в~эредитарных
стохастических системах,
приводимых к~дифференциальным}

\vspace*{-9pt}
Рассмотрим ЭСтС, описываемую интегродифференциальным уравнением Ито
следующего вида~\cite{21-s}:

\noindent
    \begin{multline}
    d X =\lk a (X,t) +\iii_{t_0}^t a_1 (X(\tau),\tau, t)\, d \tau\rk dt+{}\\
{}+\lk b(X, t) +\iii_{t_0}^t b_1 (X(\tau), \tau, t) \,d \tau\rk dW_0+{}\\
\hspace*{-5mm}{}+\iii_{R_0^q} \lk c(X,t,v) +\iii_{t_0}^t c_1 (X(\tau),  \tau, t,v) \rk
    d P^0 (t, dv)\!\label{e4.1-s}
    \end{multline}
с начальным условием $X(t_0)\hm= X_0$.

В~(\ref{e4.1-s}) приняты следующие обозначения и допущения: $X\hm=X(t)$~---
$p$-мер\-ный вектор состояния;
    $W_0$~--- $r$-мер\-ный винеровский процесс интенсивности $\nu_0 \hm= \nu_0 (t)$;
    $ \iii_{\Delta_t} d P^0 (t, A)$~--- центрированная пуассоновская мера,
    удовлетворяющая условию~(\ref{e2.2-s}).

Функции $a=a(X, t)$, $a_1 \hm= a_1(X (\tau),\tau, t)$, $b\hm=b(X, t)$,
$b_1 \hm= b_1(X (\tau),\tau, t)$, $c\hm=c(X,t,v)$ и
$c_1 \hm= c_1(X (\tau),\tau, t,v)$ имеют соответственно размерности
$p\times 1$, $p\times 1$, $p\times r$, $p\times r$, $p\times 1$ и
$p\times 1$ и допускают представления следующего вида:
    \begin{equation}
    \left.
    \begin{array}{rl}
    a_1&=A(t,\tau) \vrp (X(\tau) , \tau)\,;\\[9pt]
    b_1&=B(t,\tau) \psi (X(\tau) ,  \tau)\,;\\[9pt]
    c_1&=C(t,\tau) \chi (X(\tau) ,  \tau, v)\,.
    \end{array}
    \right\}
    \label{e4.2-s}
    \end{equation}
Здесь эредитарные ядра $A\hm=A(t,\tau)\hm=\lk A_{ij}(t,\tau)\rk$
$(i,j\hm=\overline{1,p})$,
$B\hm=B(t,\tau)\hm=\lk B_{i l}(t,\tau)\rk$ $(i\hm=\overline{1,p}$,
$l\hm=\overline{1,r})$ и $C\hm=C(t,\tau)=\lk C_{ij}(t,\tau)\rk$
$(i,j\hm=\overline{1,p})$ имеют соответственно размерности
$p\times p$, $p\times r$ и $p\times p$. Они удовлетворяют следующим условиям
физической реализуемости и асимптотического затухания:
    \begin{equation}
    \left.
    \begin{array}{c}
    A_{ij}(t,\tau)=0\,;\enskip
    B_{i l}(t,\tau)=0\,;\\[9pt]
    C_{ij}(t,\tau)=0\enskip \forall \tau >t\,;
    \end{array}
    \right\}
    \label{e4.3-s}
    \end{equation}
\begin{equation}
\left.
\begin{array}{rl}
\displaystyle\iin \lv A_{ij} (t,\tau) \rv d\tau &<\infty \,;\\[9pt]
\displaystyle\iin \lv B_{i l} (t,\tau) \rv d\tau &<\infty \,;\\[9pt]
\displaystyle \iin \lv C_{ij} (t,\tau) \rv d\tau &<\infty\,.
 \end{array}
 \right\}
 \label{e4.4-s}
 \end{equation}
При этом нелинейные в общем случае функции
$\vrp\hm=\vrp(X(\tau),\tau)$, $\psi \hm=\psi(X(\tau), \tau)$ и $\chi \hm=\chi
(X(\tau),  \tau, v)$ имеют размерности $p\times 1$, $p\times p$ и
$p\times 1$ соответст\-венно.

В случае если эредитарные ядра $A$, $B$, $C$ удовле\-тво\-ря\-ют условиям

\noindent
    \begin{align*}
    A_{ij} (t,\tau) &=\tilde A_{ij} (u)\,;\\
 B_{i l} (t,\tau) &=\tilde B_{i l} (u)\,;\\
    C_{ij} (t,\tau)& =\tilde C_{ij} (u)\enskip (u=t-\tau)\,,
   %    \label{e4.5-s}
    \end{align*}
то говорят об ЭСтС со стационарным затуханием.

Важный класс ядер представляют собой сингулярные (вырожденные) ядра,
когда имеют место представления:
\begin{equation}
\left.
\begin{array}{rl}
A_{ij} (t,\tau) &= A_{ij}^+(t) A_{ij}^-(\tau)\,;\\[9pt]
    B_{i l} (t,\tau) &= B_{il}^+(t) B_{il}^-(\tau)\,;\\[9pt]
    C_{ij} (t,\tau) &= C_{ij}^+ ( t) C_{ij}^- (\tau)
    \end{array}
    \right\}
    \label{e4.6-s}
    \end{equation}
$(i,l= \overline{1,p}$; $j=\overline{1,r}).$

В случае, когда подынтегральные функции  $c(X, t, v)$ и  $c_1(X(\tau), \tau, v)$
в~(\ref{e4.1-s}) допускают пред\-став\-ле\-ния

\noindent
    \begin{align*}
    c(X,t, v)&=b(X, t)c'(v)\,;\\
    c_1(X(\tau), \tau, v)&=b(X(\tau),\tau)c'(v)\,,
%    \label{e4.7-s}
    \end{align*}
ЭСтС~(\ref{e4.1-s}) приводится к виду:

\noindent
\begin{multline}
\dot X =  a(X, t)+\iii_{t_0}^t a_1 (X(\tau),\tau, t)\,d\tau
    +{}\\
    {}+\lk b(X, t)+ \iii_{t_0}^t b_1 (X(\tau),\tau, t)\,d\tau\rk V\,,
    \label{e4.8-s}
    \end{multline}
если принять

\noindent
    $$
    V=\dot W\,;\enskip W(t) = W_0(t) +\iii_{R_0^q} c' (v) P^0 (t, dv)\,.$$

В~\cite{21-s} решена задача приведения ЭСтС~(\ref{e4.1-s}) при
условиях~(\ref{e4.2-s})--(\ref{e4.4-s})
и~(\ref{e4.2-s})--(\ref{e4.4-s}), (\ref{e4.6-s}) к ДСтС~(\ref{e2.1-s}),
а также установлены следующие утверждения.

Рассмотрим сначала ЭСтС~(\ref{e4.1-s}) при условиях~(\ref{e4.2-s})--(\ref{e4.4-s}).
Будем считать, что эредитарные ядра  $A(t,\tau)$, $B(t,\tau)$, $C(t,\tau)$
удовлетворяют следующим нестационарным линейным операторным уравнениям:

\noindent
    \begin{align*}
    F^{At}A(t,\tau) &= H^{At} \delta (t-\tau)\,;\\
    F^{Bt}B(t,\tau) &= H^{Bt} \delta (t-\tau)\,;\\
    F^{Ct}C(t,\tau) &= H^{Ct} \delta (t-\tau)\,;\\
    A(t,\tau)&= A'(t,\tau)^{\mathrm{T}} (H^{A*\tau})^{\mathrm{T}}\,;
\\
    A'(t,\tau)^{\mathrm{T}} (F^{A*\tau})^{\mathrm{T}}&=    I_h^A\delta(t-\tau)\,;\\
B(t,\tau)&= B'(t,\tau)^{\mathrm{T}} (H^{B*\tau})^{\mathrm{T}}\,;\\
    B'(t,\tau)^{\mathrm{T}} (F^{B*\tau})^{\mathrm{T}}&= I_h^B\delta(t-\tau)\,;
\\
    C(t,\tau)&= C'(t,\tau)^{\mathrm{T}} (H^{C*\tau})^{\mathrm{T}}\,;\\
    C'(t,\tau)^{\mathrm{T}} (F^{C*\tau})^{\mathrm{T}}&= I_h^C\delta(t-\tau)\,.
%    \label{e4.10-s}
    \end{align*}
Здесь $F^A$, $H^A$, $F^B$, $H^B$, $F^C$ и~$H^C$~---
известные матричные дифференциальные операторы размерности
$h_A\times h_A$, $h_B\times h_B$, $h_C\times h_C$ порядка $n_A,
m_A$, $n_B, m_B$, $n_C, m_C$, ($n_A\hm>m_A$, $n_B\hm>m_B$, $n_C\hm>m_C$)
соответственно:
\begin{equation}
\hspace*{-3.8mm}\left.
\begin{array}{rl}
F^A = F^A (t,D)&=\displaystyle\sss_{l=0}^{n_A} \alp_l^A (t) D^l\,;\\[9pt]
H^A=H^A(t,D) &=\displaystyle\sss_{l=0}^{m_A} \beta_l^A (t) D^l\,;
\\[9pt]
F^B = F^B (t,D)&=\displaystyle\sss_{l=0}^{n_B} \alp_l^B (t) D^l\,;
\\[9pt]
    H^B=H^B(t,D) &=\displaystyle\sss_{l=0}^{m_B} \beta_l^B (t) D^l\,;
\\[9pt]
F^C = F^C (t,D)&=\displaystyle\sss_{l=0}^{n_C} \alp_l^C (t) D^l\,;
\\[9pt]
    H^C=H^C(t,D) &=\displaystyle\sss_{l=0}^{m_CA} \beta_l^C (t) D^l\,;
    \end{array}
    \right\}\!
     \label{e4.11-s}
    \end{equation}
индекс~$t$ у операторов означает, что оператор
действует на функцию от~$t$ при фиксированном~$\tau$; звездочкой
обозначен символ сопряжения оператора; $I_h^A$, $I_h^B$, $I_h^C$~---
единичные $(h\times h)$-мат\-ри\-цы. Введем $h^A$-, $h^B$-, $h^C$-мер\-ные
векторы посредством соотношений:
\begin{align*}
Z_1' &= U' =\iii_{t_0}^t A(t,\tau) \vrp(X(\tau), \tau)\,d\tau\,;\\
Z_1''&= U'' =\iii_{t_0}^t B(t,\tau) \psi(X(\tau), \tau)\,d\tau\,;\\
Z_1'''&= U''' =\iii_{t_0}^t C(t,\tau) \chi(X(\tau), \tau,v)\,d\tau\,.
%    \label{e4.12-s}
    \end{align*}
Эти переменные  $Z'$, $Z''$, $Z'''$ будут удовлетворять следующим линейным
дифференциальным уравнениям:
\begin{align*}
F^A(t, D) Z_1' &= H^A (t, D) \vrp (X, t)\,;\\
    F^B(t, D) Z_1'' &= H^B (t, D) \psi (X, t)\,;\\
F^C(t, D) Z_1''' &= H^C (t, D) \chi (X, t,v)\,.
% \label{e4.13-s}
 \end{align*}
Тогда ЭСтС~(\ref{e4.1-s}) приводится к искомой ДСтС для расширенного вектора
состояния $Z\hm= \lk X^{\mathrm{T}} {Z_1'}^{\mathrm{T}}\, {Z_1''}^{\mathrm{T}}\,{Z_1'''}^{\mathrm{T}}\rk^{\mathrm{T}}$:

\noindent
    \begin{multline}
    dZ = a_1^z (Z, t)\, dt + b_1^z (Z, t)\, dW_0+{}\\
    {}+
    \iii_{R_0^q} c_1^z (Z,  t, v) \,d P^0 (t, dv)\,.
    \label{e4.14-s}
    \end{multline}
Для случая  $h_A \hm= h_B \hm= h_C=h$, $n_A\hm= n_B\hm=n_C\hm=n$,
$m_A\hm=m_B\hm=m_C\hm=m$ в подробной записи функции $a^z_1 (Z,t)$,
$b^z_1 (Z,t)$, $c^z_1 (Z, t,v)$ имеют следующий вид:
\begin{equation}
\left.
\begin{array}{c}
a_1^z (Z, t)=\begin{bmatrix}
        a(X, t)+ Z_1'\\
        a'(t)Z_1'\\
        a''(t) Z_1''\\
        a'''(t)Z_1'''\end{bmatrix}\,;\\[20pt]
    b_1^z (Z, t)=\begin{bmatrix}
        b(X, t)+ Z_1''\\
        b''(t)Z_1''\\
        0\\
        0\end{bmatrix}\,;\\[20pt]
c_1^z (Z, t,v)=\begin{bmatrix}
        c(X, t,v)+ Z_1'''\\
        c'''(t)Z_1'''\\
        0\\
        0\end{bmatrix}\,.
        \end{array}
        \right\}
        \label{e4.15-s}
        \end{equation}
При условии существования обратных матриц $(\alp_n^A)^{-1}$,
 $(\alp_n^B)^{-1}$, $(\alp_n^C)^{-1}$ входящие в~(\ref{e4.15-s})
 переменные и коэффициенты допускают  следующую запись:
\begin{equation}
\left.
\begin{array}{rl}
Z_{j+1}' &=\dot Z_j' - q_j' \vrp(X,t)\,;\\[9pt]
    Z_{j+1}'' &=\dot Z_j'' - q_j'' \psi(X,t)\,;\\[9pt]
 Z_{j+1}''' &=\dot Z_j'''- q_j''' \chi(X,t,v)\enskip (j=\overline{1,(n-1)})\,;
\end{array}
\right\}
 \label{e4.16-s}
 \end{equation}

 \vspace*{-13pt}

\noindent
\begin{multline}
a'(t) ={}\\[-3pt]
\hspace*{-2mm}{}=\!\!\begin{bmatrix}
    I_h&0&\cdots&0\\
    0&I_h&\ddots&0\\
    \vdots&\ddots&\ddots&\cdots\\
    0&0&\cdots&I_h\\
    -(\alp_n^A)^{-1}\alp_0^A& -(\alp_n^A)^{-1}\alp_1^A&\cdots&-(\alp_n^A)^{-1} \alp_{n-1}^A
    \end{bmatrix}\!\!;\!\!\!\!
    \end{multline}

 \vspace*{-13pt}

\noindent
\begin{multline}
a''(t) ={}\\[-3pt]
\hspace*{-1.5mm}{}=\!\!\begin{bmatrix}
    I_h&0&\cdots&0\\
    0&I_h&\ddots&0\\
    \vdots&\ddots&\ddots&\cdots\\
    0&0&\cdots&I_h\\
    -(\alp_n^B)^{-1}\alp_0^B& -(\alp_n^B)^{-1}\alp_1^B&\cdots&-(\alp_n^B)^{-1}\alp_{n-1}^B
    \end{bmatrix}\!\!;\!\!\!
    \end{multline}

 \vspace*{-13pt}

\noindent
\begin{multline}
a'''(t) ={}\\[-3pt]
\hspace*{-2mm}{}=\!\!\begin{bmatrix}
    I_h&0&\cdots&0\\
    0&I_h&\ddots&0\\
    \vdots&\ddots&\ddots&\cdots\\
    0&0&\cdots&I_h\\
    -(\alp_n^C)^{-1}\alp_0^C& -(\alp_n^C)^{-1}\alp_1^C&\cdots&-(\alp_n^C)^{-1}\alp_{n-1}^C
    \end{bmatrix}\!\!;\!\!\!
    \label{e4.17-s}
    \end{multline}

 \vspace*{-12pt}

\noindent
\begin{multline}
    q_j' = (\alp_n^A)^{-1} \left[
    \vphantom{    \sss_{l=0}^{j-i}}
    \beta_{n-j}^A -{}\right.\\
\left.    {}-\sss_{i=0}^{j-1}
    \sss_{l=0}^{j-i} {\cal C}_{n-j-l}^{n-j} \alp_{n-j+i+l}^A {q_i'}^{(l)}\right]\,;
\end{multline}

%\vspace*{-6pt}

    \noindent
    \begin{equation}
    q_n' = (\alp_n^A)^{-1} \lk \beta_{0}^A -\sss_{i=0}^{n-1} \sss_{l=0}^{n-i}
    \alp_{i+l}^A {q_i'}^{(l)}\rk\,;
    \end{equation}

     \vspace*{-12pt}

\noindent
\begin{multline}
    q_j'' = (\alp_n^B)^{-1} \left[
        \vphantom{    \sss_{l=0}^{j-i}}
        \beta_{n-j}^B -{}\right.\\[3pt]
\left.    {}-\sss_{i=0}^{j-1}
    \sss_{l=0}^{j-i} {\cal C}_{n-j-l}^{n-j} \alp_{n-j+i+l}^A {q_i''}^{(l)}\right]\,;
    \end{multline}

    \vspace*{-6pt}

    \begin{equation}
    q_n'' = (\alp_n^B)^{-1} \lk \beta_{0}^B -\sss_{i=0}^{n-1}
    \sss_{l=0}^{n-i}  \alp_{i+l}^A {q_i''}^{(l)}\rk\,;
    \end{equation}

    \vspace*{-12pt}

    \noindent
    \begin{multline}
    q_j''' = (\alp_n^C)^{-1} \left[
        \vphantom{    \sss_{l=0}^{j-i}}
         \beta_{n-j}^C -{}\right.\\[3pt]
\left.    {}-\sss_{i=0}^{j-1}
    \sss_{l=0}^{j-i} {\cal C}_{n-j-l}^{n-j} \alp_{n-j+i+l}^C {q_i'''}^{(l)}\right]\,;
    \end{multline}

\noindent
\begin{equation}
    q_j''' = (\alp_n^C)^{-1} \lk \beta_{0}^C -\sss_{i=0}^{n-1} \sss_{l=0}^{n-i}
    \alp_{i+l}^C {q_i'''}^{(l)}\rk\,.
    \label{e4.18-s}
    \end{equation}
Здесь $C_m^n=n!/(m!(n-m)!)$; индекс~$l$ означает, что суммирование
проводится по всем индексам, исключая~$l$.

Таким образом, справедливо следующее утверждение.

\medskip

\noindent
\textbf{Теорема 7.} \textit{Пусть  ядра $A(t,\tau)$, $B(t,\tau)$,
$C(t,\tau)$ в ЭСтС~$(\ref{e2.1-s})$ удовлетворяют
условиям~$(\ref{e4.2-s})$--$(\ref{e4.4-s})$
или~$(\ref{e4.15-s})$, причем матрицы  $\alp_n^A$, $\alp_n^B$, $\alp_n^C$
в~$(\ref{e4.11-s})$ обратимы, а функции  $\vrp$, $\psi$, $\chi$
дифференцируемы по переменным расширенного вектора состояния достаточное число раз.
Тогда ЭСтС~$(\ref{e4.1-s})$ приводится к ДСтС~$(\ref{e4.14-s})$
на основе}~(\ref{e4.15-s})--(\ref{e4.18-s}).

\medskip

\noindent
\textbf{Замечание 1.} Векторное уравнение~(\ref{e4.14-s})
всегда линейно относительно $Z_1'$, $Z_1''$, $Z_1'''$, но в общем
случае нелинейно относительно~$X$.

\medskip

В том случае, когда выполнены условия~(\ref{e4.2-s})--(\ref{e4.4-s}),
а функции $\vrp$, $\psi$, $\chi$ не дифференцируемы по переменным расширенного
вектора состояния, целесообразна аппроксимация вырожденными ядрами~(\ref{e4.6-s}).
В~этом случае имеют место следующие соотношения:
   \begin{multline}
    dZ= a_2^z (Z,t)\, dt+ b_2^z(Z,t)\, dW_0 +{}\\
    {}+
    \iii_{R_0^q} c_2^z (Z,t,v)\,dP^0 (t, dv)\,;
    \label{e4.19-s}
    \end{multline}

\vspace*{-9pt}

\noindent
\begin{equation}
Z=\lk X^{\mathrm{T}} {Y'}^{\mathrm{T}} {Y''}^{\mathrm{T}} {Y'''}^{\mathrm{T}}\rk^{\mathrm{T}}\,;
    \label{e4.20-s}
    \end{equation}
\begin{equation}
\left.
\begin{array}{rl}
\displaystyle\iii_{t_0}^t A(t,\tau) \vrp(X(\tau),  \tau) d\tau &= A^+ Y';\\[9pt]
\displaystyle \iii_{t_0}^t B(t,\tau) \psi(X(\tau),  \tau) d\tau &= B^+ Y'';\\[9pt]
\displaystyle\iii_{t_0}^t C(t,\tau) \chi(X(\tau),  \tau,v) d\tau &= C^+ Y''';
\end{array}
\right\}
    \label{e4.21-s}
    \end{equation}

    \vspace*{-6pt}


        \begin{gather*}
\dot Y'=A^-\vrp\,;\enskip  Y'(t_0)=0\,;\enskip
        \dot Y''=B^-\psi\,;\\[2pt]
        Y''(t_0)=0\,;\enskip
        \dot Y'''=C^-\chi\,;\enskip Y'''(t_0)=0\,;
%        \label{e4.22-s}
        \end{gather*}

        \vspace*{-3pt}

        \begin{equation}
        \left.
        \begin{array}{rl}
        a_2^z (Z,t)&=\displaystyle\begin{bmatrix}
        a(X,t)+A^+\vrp\\
        A^-\vrp\\
        B^- \psi\\
        C^-\chi\end{bmatrix}\,;\\[21pt]
    b_2^z (Z,t)&=\displaystyle\begin{bmatrix}
        b(X,t)+B^+\psi\\
        0\\
        0\\
        0\end{bmatrix}\,; \\[21pt]
      c_2^z (Z,t,v)&\displaystyle=\begin{bmatrix}
        c(X,t,v)+C^+\chi\\
        0\\
        0\\
        0\end{bmatrix}\,.
        \end{array}
        \right\}
        \label{e4.23-s}
        \end{equation}

Таким образом, имеем следующий результат.

\smallskip

\noindent
\textbf{Теорема 8.} \textit{Пусть эредитарные ядра $A(t,\tau)$,
$B(t,\tau)$, $C(t,\tau)$ в ЭСтС~$(\ref{e4.1-s})$ удовлетворяют условиям~$(\ref{e4.3-s})$,
$(\ref{e4.4-s})$ и~$(\ref{e4.6-s})$, а функции $\vrp$, $\psi$, $\chi$
не дифференцируемы по переменным расширенного вектора состояния.
Тогда ЭСтС~$(\ref{e4.1-s})$ приводится к ДСтС~$(\ref{e4.19-s})$
на основе~$(\ref{e4.20-s})$--$(\ref{e4.23-s})$.}

\smallskip

\noindent
\textbf{Замечание 2.}
Векторное уравнение~(\ref{e4.19-s}) для $Y'$, $Y''$, $Y'''$ относится к
числу так называемых приводимых к линейным уравнениям~\cite{2-s}.

Аналогичные теоремы устанавливаются для ЭСтС~(\ref{e4.8-s}).

Следовательно, если выполнены условия теорем~7 и~8, то ЭСтС~(\ref{e4.1-s})
приводится к ДСтС~(\ref{e4.14-s}) или~(\ref{e4.19-s}) и могут быть использованы
точные и приближенные методы анализа и моделирования распределений с
инвариантной мерой (см.\ разд.~2 и~3).

\smallskip

Таким образом, получены следующие утверждения, лежащие в основе точных и
приближенных методов для ЭСтС, приводимых к ДСтС.

\smallskip

\noindent
\textbf{Теорема 9.} \textit{В~условиях теоремы~$7$ для гладких функций
$a$, $a_1$, $b$, $b_1$, $c$, $c_1$ одномерные нестационарные и стационарные
распределения с инвариантной мерой определяются уравнениями теорем}~1 \textit{и}~2.

\smallskip

\noindent
\textbf{Теорема 10.} \textit{В~условиях теоремы~$8$ для разрывных функций~$a$,
$a_1$, $b$, $b_1$, $c$, $c_1$ одномерные нестационарные и стационарные
распределения с инвариантной мерой определяются уравнениями теорем~$3$ и~$4$.}

\smallskip

\noindent
\textbf{Теорема 11.} \textit{В~условиях теорем~$7$ и~$8$
приближенный алгоритм аналитического моделирования нестационарных
процессов с инвариантной мерой по МНА определяется теоремой~$5$,
а стационарных процессов~--- теоремой~$6$.}

\vspace*{-12pt}

\section{Заключение}

Получено обобщение точных и приближенных (основанных на
параметризации распределений) методов и алгоритмов моделирования
стационарных и нестационарных процессов с инвариантной мерой в
негауссовских ДСтС и ЭСтС с винеровскими и пуассоновскими шумами,
приводимых к ДСтС, для случаев гладких и разрывных регулярных правых
частей уравнений.

Особое внимание уделено приближенным
МНА и МСЛ для нахождения
распределений процессов с инвариантной мерой в ДСтС и ЭСтС,
приводимых к ДСтС.

Аналогично~\cite{2-s, 15-s, 25-s}, результаты допускают обобщение на случай
ЭСтС, приводимых к ДСтС, с автокоррелированными шумами.

Разработан комплекс тестовых примеров для инструментального
программного обеспечения в   <<ID StS>> в среде  MATLAB (см.\
приложение).

Аналогично~[2--8] может быть рас\-смот\-ре\-но применение представленных
методов в задачах эквивалентности гауссовских и негауссовских ДСтС и
ЭСтС. В~частности, соотношения~(\ref{e2.15-s}) и~(\ref{e3.9-s})
позволяют заменять в
ДСтС и ЭСтС стационарные и нестационарные негауссовские шумы
гауссовскими. Часто оказывается полезным заменить  $p$-мер\-ную
негауссовскую ДСтС или ЭСтС эквивалентной системой из  $p_1$
независимых ДСтС меньшей размерности  $(p_1\hm<p)$. В~этом случае
следует учесть дополнительные связи на  $K_{ij}(t)$, вытекающие из
аналитической природы рассматриваемой задачи.

\vspace*{-12pt}

{\small

\setcounter{equation}{0}

 \section*{\raggedleft Приложение}

 \vspace*{-6pt}

 \section*{Тестовые примеры}

\renewcommand{\theequation}{П\arabic{equation}}

\noindent
\textbf{Пример 1.}
 В условиях примера~6~\cite{22-s}, когда
\begin{multline*}
\ddot{X}+ \w^2 X -\mu X^3 =
    -\delta \dot X +\gamma + V^{\mathrm{ОР}}-{}\\
  {}-
    \int\limits_{t_0}^{t} \lk \w_1 X(\tau) -\delta_1 \dot X (\tau) +
    \mu_1 X^3 (\tau)\rk e^{-\la | t-\tau|}\, d\tau \,,
\\
  X(t_0) = X_0\,,\enskip \dot X (t_0) =\dot X_0\,,
%      \label{p1.1}
\end{multline*}
для обобщенного пуассоновского белого шума интенсивности $\nu
\hm=\nu^{\mathrm{OP}}$ уравнения для математических ожиданий, дисперсий и
ковариаций в силу теоремы~11 имеют следующий вид:
\begin{equation}
\left.
\begin{array}{rl}
\dot m_1 &=m_2;\\[9pt]
\dot m_2 &=-\w^2_{\mathrm{0э}} m_1 -\delta m_2 -\la^{-1} m_3 +\gamma\,;\\[9pt]
\dot m_3 &=\la (\w_{\mathrm{1э}} m_1 +\delta_1 m_2 -m_3)\,;
\end{array}
\right\}
\label{p1.2}
\end{equation}
\begin{equation}
\left.
\begin{array}{rl}
  \dot K_{11} &= 2 K_{12}\,;\\[9pt]
    \dot K_{12}& = K_{22} -(\w_{\mathrm{0э}}^2 K_{11} + \delta K_{12}
    +\la^{-1} K_{13})\,;\\[9pt]
    \dot K_{13} &= K_{23} +\la \w_{\mathrm{1э}}' K_{11} +\la \delta_1 K_{12} -
    \la K_{13}\,;\\[9pt]
    \dot K_{22} &=-2 (\w_{\mathrm{0э}}^2 K_{12} +\delta K_{22} +
    \la^{-1} K_{23})+\nu^{\mathrm{ОР}}\,;\\[9pt]
    \dot K_{23} &=-(\w_{\mathrm{0э}}^2 K_{13} +\la^{-1} K_{33})+
    \la\w_{\mathrm{1э}}' K_{12} +{}\\[9pt]
    &\hspace*{20mm}{}+\la \delta_1 K_{22} -(\delta+\la) K_{23}\,;\\[9pt]
    \dot K_{33} &= 2 ( \la \w_{\mathrm{1э}}' K_{13} +\la \delta_1 K_{23} -
    \la K_{33} )\,.
    \end{array}
    \right\}
    \label{p1.3}
    \end{equation}
Здесь приняты следующие обозначения:
\begin{gather*}
Z_1 = X\,;\enskip Z_2 = \dot X_3\,;
\\
     Z_3 =\int\limits_{t_0}^t \lk \w_1 Z_1^{(\tau)} +
    \delta_1 Z_2 (\tau) -\mu_1 Z_1^3 (\tau)\rk e^{-\la |t-\tau|} \,d\tau\,;
  \\
    m_i = {\sf M} Z_i \ (i=1,2,3)\,;\enskip K_{ij} = {\sf M} Z_i^0 Z_j^0\
    (i,j=1,2,3)\,;
    \end{gather*}
\begin{align*}
    &\w_{\mathrm{0э}}^2=\w_{\mathrm{0э}}^2(m_1, D_1) =\w^2 \lk 1-
    \mu \fr{(m_1^2 +3 D_1)}{\w^2}\rk\,;\\
    &\w_{\mathrm{1э}}=\w_{\mathrm{1э}}(m_1, D_1) =
    \w_1 \lk 1- \fr{\mu_1 (m_1^2 +3 D_1)}{\w_1}\rk\,;\\
    &\w_{\mathrm{1э}}'=\w_{\mathrm{1э}}'(m_1, D_1) =
    \w_1 \lk 1- \fr{3\mu (m_1^2 +3 D_1)}{\w_1}\rk\,.
    \end{align*}

Приравнивая в~(\ref{p1.2}) и~(\ref{p1.3}) правые части нулю, получим уравнения
для стационарных значений  $m_i^*$ и $K_{ij}^*$. Для устойчивости (в
среднем квадратическом) стационарных колебаний необходима
асимптотическая устойчивость матрицы
    \begin{equation*}
    \Lambda =\begin{bmatrix}
    0&1&0\\
    -\w_{\mathrm{0э}}^2 (m_1, D_1)& -\delta& -\la^{-1}\\
    \la \w_{\mathrm{1э}}'(m_1, D_1)& \la \delta_1 & -\la
    \end{bmatrix}
%\label{p1.4}
\end{equation*}
в~(\ref{e3.12-s}).

\smallskip

\noindent
\textbf{Пример~2.}
Для релейного осциллятора
    \begin{multline*}
    \ddot X +\alpha \,\mathrm{sgn}\, X =
    - \delta \dot X +\gamma +V^{\mathrm{ОР}}-{}\\
    {}-\int\limits_{t_0}^t \lk \alpha_1
    \mathrm{sgn}, X(\tau) +\delta_1 \dot X (\tau)\rk
    e^{-\la |t-\tau|}\,d\tau\,;\\
%\label{p2.1}
X(t_0) = X_0\,;\enskip \dot X (t_0) =\dot X \enskip (\alpha>0)
   \end{multline*}
в случае обобщенного пуассоновского белого шума~$V^{\mathrm{ОР}}$ интенсивности
$\nu\hm=\nu^{\mathrm{ОР}}$ в силу теоремы~11 имеем:

\noindent
\begin{equation}
\left.
\begin{array}{rl}
\dot m_1 &=m_2\,;\\[6pt]
\dot m_2 &=\alpha k_0 m_1 -\delta m_2 -\la^{-1} m_3 +\gamma\,;\\[6pt]
\dot m_3 &=\la (\alpha_1 k_0 m_1 +\delta_1 m_2 -m_3)\,;
\end{array}
\right\}
\label{p2.2}
\end{equation}
\begin{equation}
\left.
\begin{array}{rl}
\dot K_{11} &= 2 K_{12}\,;\\[5pt]
\dot K_{12} &= K_{22} -\alpha k_1 K_{11} - \delta K_{12} -\la^{-1} K_{13}\,;\\[5pt]
\dot K_{13} &= K_{23} +\la \alpha_1 k_1 K_{12} +\la \delta_1 K_{22} -\la K_{13}\,;\\[5pt]
\dot K_{22} &=-2 \alpha k_1 K_{12} -2\delta K_{22} -2\la^{-1} K_{23}+
 \nu^{\mathrm{ОР}}\,;\\[5pt]
\dot K_{23} &=-\alpha k_1 K_{13}  -(\delta+\la) K_{23}-\la^{-1} K_{33}+{}\\[5pt]
&\hspace*{24mm}{}+
    \la\alpha_1 k_1 K_{12} +\la \delta_1 K_{22}\,;\\[5pt]
\dot K_{33} &= 2  \la (\alpha_1 k_1 K_{13} +\delta_1 K_{23} -
    K_{33})\,.
    \end{array}
    \right\}
\label{p2.3}
\end{equation}
Здесь введены следующие обозначения:
    \begin{gather*}
    Z_1 = X\,;\quad Z_2 =\dot X\,;
    \\[-1pt]
    Z_3 =\la \int\limits_{t_0}^t \lk \alpha_1 \mathrm{sgn}\, X(\tau) +
    \delta_1 \dot X(\tau)\rk e^{-\la |t-\tau |} \,d\tau\,;
\\
    k_0 = k_0(m_1, D_1) =2\Phi\left(\fr{m_1}{\sqrt{ D_1}}\right)\,;\\
    k_1 = k_1(m_1, D_1) =\fr{2}{\sqrt{2\pi D_1}} \exp \lk -\fr{1}{2}
    \left(\fr{m_1}{\sqrt{ D_1}}\right)^2\rk\,.
    \end{gather*}

Приравнивая правые части~(\ref{p2.2}) и~(\ref{p2.3})
 нулю, получим уравнения для стационарных значений  $m_i^*$ и~$K_{ij}^*$.
Устойчивость стационарных колебаний определяется асимптотической устойчивостью
матрицы
    \begin{equation*}
    \Lambda =\begin{bmatrix}
    0&1&0\\
    -\alpha k_1 (m_1, D_1)& -\delta& -\la^{-1}\\
    \la \alpha_1 k_1(m_1, D_1)& \la \delta_1 & -\la
    \end{bmatrix}
%    \label{p2.4}
    \end{equation*}
    в~(\ref{e3.12-s}).


}

\vspace*{-8pt}


{\small\frenchspacing
{%\baselineskip=10.8pt
%\addcontentsline{toc}{section}{References}
\begin{thebibliography}{99}

\vspace*{-4pt}

\bibitem{1-s}
\Au{Пугачев В.\,С., Синицын И.\,Н.}
Стохастические дифференциальные системы. Анализ и фильтрация.~--- М.:
Наука,  1990.  632~с. [Англ. пер. Stochastic differential systems.
Analysis and filtering.~--- Chichester, N.Y.: Jonh Wiley, 1987.
549~p.].

\bibitem{3-s} %2
\Au{Moshchuk N.\,K., Sinitsyn I.\,N.}
On stationary distributions in nonlinear stochastic differential systems.~---
Coventry, UK: University of Warwick, Mathematics Institute, 1989. Preprint. 15~p.

\bibitem{4-s} %3
\Au{Moshchuk N.\,K., Sinitsyn I.\,N.} On stochastic nonholonomic systems.~---
Coventry, UK: University of Warwick, Mathematics Institute,
1989. Preprint. 32~p.

\bibitem{5-s} %4
\Au{Мощук Н.\,К., Синицын И.\,Н.}
О~стохастических неголономных системах~// Прикладная механика и математика, 1990.
Т.~54. Вып.~2. С.~213--223.

\bibitem{6-s} %5
\Au{Moshchuk N.\,K., Sinitsyn I.\,N.}
On stationary distributions in nonlinear stochastic differential systems~//
Quart. J.~Mech. Appl. Math., 1991. Vol.~44.  Pt.~4.  P.~571--579.

\bibitem{7-s} %6
\Au{Мощук Н.\,К., Синицын И.\,Н.} О~стационарных и приводимых к стационарным
режимах в нормальных стохастических системах~// Прикладная механика и математика,
1991. Т.~55. Вып.~6. С.~895--903.

\bibitem{8-s} %7
\Au{Мощук Н.\,К., Синицын И.\,Н.}
Распределения с инвариантной мерой в механических стохастических нор-\linebreak\vspace*{-12pt}

\columnbreak

\noindent
мальных
системах~// Докл. АН СССР, 1992. Т.~322. №\,4. С.~662--667.

\bibitem{9-s} %8
\Au{Синицын И.\,Н.} Конечномерные распределения с инвариантной мерой в
стохастических механических системах~// Докл. РАН, 1993. Т.~328. №\,3. С.~308--310.

\bibitem{13-s} %9
\Au{Soize C.} The Fokker--Plank equation for stochastic dynamical
systems and its explicit steady state solutions.~--- Singapore: World Scientific,
1994. 321~p.

\bibitem{10-s} %10
\Au{Синицын И.\,Н.} Конечномерные распределения с инвариантной мерой в
стохастических нелинейных дифференциальных системах.~--- М.:
Диа\-лог--МГУ, 1997. С.~129--140.

\bibitem{2-s} %11
\Au{Пугачев В.\,С., Синицын И.\,Н.}
Теория стохастических систем.~--- М.: Логос, 2000; 2004. 1000~с.
[Англ. пер. Stochastic systems. Theory and  applications.~---
Singapore: World Scientific, 2001. 908~p.].


\bibitem{11-s} %12
\Au{Синицын И.\,Н., Корепанов Э.\,Р., Белоусов~В.\,В.}
Точные методы расчета стационарных режимов с инвариантной мерой в
стохастических системах управления~// Кибернетика и высокие
технологии XXI века: Сб. докл.  II~Междунар.
на\-уч.-тех\-нич. конф.~--- Воронеж: Саквоее, 2002.
С.~124--131.

\bibitem{12-s} %13
\Au{Синицын И.\,Н., Корепанов Э.\,Р., Белоусов~В.\,В.}
Точные аналитические методы в статистической динамике нелинейных
ин\-фор\-ма\-ци\-он\-но-управ\-ля\-ющих сис\-тем~//
Системы и средства информатики.
Спец. вып. Математическое и алгоритмическое обеспечение
ин\-фор\-ма\-ци\-он\-но-те\-ле\-ком\-му\-ни\-ка\-ци\-он\-ных сис\-тем.~---
М.: Наука, 2002. С.~112--121.


\bibitem{14-s}
\Au{Синицын И.\,Н.} Развитие методов аналитического моделирования
распределений с инвариантной мерой в стохастических системах~//
Современные проблемы прикладной математики, информатики, автоматизации,
управления: Мат-лы Междунар. семинара.~--- Севастополь:  СевНТУ, 2012.
С.~24--35.

\bibitem{15-s}
\Au{Синицын И.\,Н.} Аналитическое моделирование распределений с
инвариантной мерой в стохастических системах с автокоррелированными шумами~//
Информатика и её применения, 2012. Т.~6. Вып.~4. С.~4--8.

\bibitem{16-s}
\Au{Синицын И.\,Н. }
Аналитическое моделирование распределений с инвариантной мерой в
стохастических системах с разрывными характеристиками~// Информатика
и её применения, 2013. Т.~7. Вып.~1.  С.~3--11.

\bibitem{17-s}
\Au{Синицын И.\,Н. }
Параметрическое статистическое и аналитическое моделирование распределений
в нелинейных стохастических системах на многообразиях~//
Информатика и её применения, 2013. Т.~7. Вып.~2. С.~4--16.

\bibitem{18-s}
\Au{Синицын И.\,Н.,  Синицын В.\,И. }
Лекции по нормальной и эллипсоидальной аппроксимации распределений в
стохастических сис\-те\-мах.~--- М.: ТОРУС ПРЕСС, 2013. 488~с.

\bibitem{20-s} %19
\Au{Синицын И.\,Н. }
Stochastic hereditary control systems~// Проблемы управления и
теории информации, 1986. Т.~15. №\,4. С.~287--298.

\bibitem{19-s} %20
\Au{Синицын И.\,Н. }
Конечномерные распределения процессов в стохастических интегральных
и интегродифференциальных системах~// Preprints of the 2nd IFAC
Symposium on Stochastic Control. Vol.~1.~--- Zurich: Pergamon Press,
1987. P.~144--153.


\bibitem{22-s} %21
\Au{Синицын И.\,Н., Синицын~В.\,И., Корепанов~Э.\,Р., Белоусов~В.\,В.,
Сергеев~И.\,В., Басилашвили~Д.\,А.}
Опыт моделирования эредитарных стохастических систем~//
Кибернетика и высокие технологии XXI~века: Сб. докл.  XIII~Междунар.
на\-уч.-тех\-нич. конф.~--- Воронеж: Саквоее, 2012. Т.~2. C.~346--357.

%\columnbreak

\bibitem{21-s} %22
\Au{Синицын И.\,Н. }
Анализ и моделирование распределений в эредитарных стохастических
системах~// Информатика и её применения, 2014. Т.~8. Вып.~1.
С.~2--11.


\bibitem{23-s}
\Au{Немыцкий В.\,В., Степанов В.\,В.}
Качественная теория дифференциальных уравнений.~--- М.--Л.: Гостехиздат,
1949. 448~с.

\bibitem{24-s}
\Au{Козлов В.\,В.} О~существовании интегрального инварианта гладких
динамических систем~// ПММ, 1987. №\,1. С.~538--545.

\bibitem{25-s}
\Au{Синицын И.\,Н.} Фильтры Калмана и Пугачева.~--- 2-е изд.~--- М.: Логос,
2007. 776~с.

%\bibitem{26-s}
%\Au{Синицын И.\,Н. }
%Канонические представления случайных функций и их применение в
%задачах компьютерной поддержки научных исследований.~--- М.: ТОРУС
%ПРЕСС, 2009. 768~с.

\end{thebibliography}
} }

\end{multicols}

\vspace*{-9pt}

\hfill{\small\textit{Поступила в редакцию 16.01.14}}

%\newpage


\vspace*{6pt}

\hrule

\vspace*{2pt}

\hrule

\vspace*{-6pt}


\def\tit{ANALYTICAL MODELING OF~DISTRIBUTIONS
 WITH  INVARIANT MEASURE IN~NON-GAUSSIAN
DIFFERENTIAL  AND~REDUCABLE TO~DIFFERENTIAL HEREDITARY
 STOCHASTIC SYSTEMS}

\def\titkol{Analytical modeling of distributions
 with  invariant measure in~non-Gaussian
differential  and~reducable to~differential HStS} %hereditary stochastic systems}

\def\aut{I.\,N.~Sinitsyn}
\def\autkol{I.\,N.~Sinitsyn}


\titel{\tit}{\aut}{\autkol}{\titkol}

\vspace*{-12pt}

\noindent
Institute of Informatics Problems, Russian Academy of Sciences,
44-2 Vavilov Str., Moscow 119333, Russian
Federation



\def\leftfootline{\small{\textbf{\thepage}
\hfill INFORMATIKA I EE PRIMENENIYA~--- INFORMATICS AND APPLICATIONS\ \ \ 2014\ \ \ volume~8\ \ \ issue\ 2}
}%
 \def\rightfootline{\small{INFORMATIKA I EE PRIMENENIYA~--- INFORMATICS AND APPLICATIONS\ \ \ 2014\ \ \ volume~8\ \ \ issue\ 2
\hfill \textbf{\thepage}}}

%\vspace*{6pt}



\Abste{Exact and approximate methods and algorithms of one- and multidimensional
distributions with invariant measure for analytical modeling in differential
non-Gaussian (with Wiener and Poisson noises) stochastic systems (StS) and
hereditary StS (HStS) reducible to differential are presented.
    Four theorems giving exact methods of analysis modeling in differential StS (DStS)
    of general type are proved. Approximate methods based on distributions
    parametrization in DStS are disscused. Special attention is paid to the methods
    of normal approximation (MNA) and statistical linearization (MSL) for
    one-    and dimensional distributions in DStS. Stability conditions are presented.
    three theorems giving exact and approximate analytical modeling in HStS resucible
    to DStS with asymptotically dying kernels are given.
    Some equavalency applications  of DStS and HStS are considered. Test examples
    for software tools ``ID StS'' are given.}

\KWE{analytical modeling; differential stochastic system;
distribution with invariant measure; Gaussian (normal) stochastic system;
hereditary kernel; hereditary stochastic system;
hereditary system reducible to differential;
Ito stochastic differential equation; method of statistical linearization;
non-Gaussian (with Wiener and Poisson noises) stochastic system;
normal approximation method;
singular kernel; software tools  ``ID  StS'' }

\DOI{10.14357/19922264140201}

\vspace*{-24pt}

\Ack
\noindent
The work was financially supported by the Program ``Intelligent information
technology, system analysis, and automation'' (project~1.7).

\vspace*{-2pt}


  \begin{multicols}{2}

\renewcommand{\bibname}{\protect\rmfamily References}
%\renewcommand{\bibname}{\large\protect\rm References}

{\small\frenchspacing
{%\baselineskip=10.8pt
\addcontentsline{toc}{section}{References}
\begin{thebibliography}{99}

\vspace*{-5pt}

\bibitem{1-s-1}
\Aue{Pugachev, V.\,S., and  I.\,N.~Sinitsyn}.  1987.
\textit{Stochastic differential systems. Analysis and filtering.}
Chichester, New York: Jonh Wiley. 549~p.


\bibitem{3-s-1} %2
\Aue{Moshchuk, N.\,K., and I.\,N.~Sinitsyn}. 1989.
\textit{On stationary distributions in nonlinear stochastic differential systems}.
Coventry, UK: University of Warwick, Mathematics Institute. Preprint.
15~p.

\bibitem{4-s-1} %3
\Aue{Moshchuk, N.\,K., and I.\,N.~Sinitsyn}.  1989.
\textit{On stochastic nonholonomic systems}.
Coventry, UK: University of Warwick, Mathematics Institute. Preprint. 32~p.

\bibitem{5-s-1} %4
\Aue{Moshchuk, N.\,K., and I.\,N.~Sinitsyn}. 1990.
 O stokhasti\-che\-skikh negolonomnykh sistemakh
 [About stochastic nonholonomial systems].
 \textit{Prikladnaya Mekhanika i Matematika}
 [\textit{Appl. Mech. Math.}]. 54(2):213--223.

\bibitem{6-s-1} %5
\Aue{Moshchuk, N.\,K., and I.\,N.~Sinitsyn}. 1991.
On stationary distributions in nonlinear stochastic differential systems.
\textit{Quart. J.~Mech. Appl. Math.} 44(4):571--579.

\bibitem{7-s-1} %6
\Aue{Moshchuk, N.\,K., and I.\,N.~Sinitsyn}. 1991.
 O statsi\-o\-nar\-nykh i privodimykh k statsionarnym rezhimakh v
 normal'nykh stokhasticheskikh sistemakh
 [About stationary and reducible to stationary regimes in normal
 stochastic systems].
 \textit{Prikladnaya Mekhanika i Matematika}
 [\textit{Appl. Mech. Math.}]. 55(6):895--903.

\bibitem{8-s-1} %7
\Aue{Moshchuk, N.\,K., and I.\,N.~Sinitsyn}. 1992.
Raspredeleniya s invariantnoy meroy v mekhanicheskikh sto\-kha\-sti\-che\-skikh
normal'nykh sistemakh [Distributions with invariant measure in mechanical
stochastic normal systems]. \textit{Dokl. AN SSSR} 322(4):662--667.

\bibitem{9-s-1} %8
\Aue{Sinitsyn, I.\,N.} 1993.
 Konechnomernye raspredeleniya s invariantnoy meroy v stokhasticheskikh
 mekhanicheskikh sistemakh [Finite dimensional distributions with invariant
 measure in stochastic mechanical systems]. \textit{Dokl. RAN} 328(3):308--310.

 \bibitem{13-s-1} %9
\Aue{Soize, C.}  1994.
 \textit{The Fokker--Plank equation for stochastic dynamical systems and
 its explicit steady state solutions}. Singapore: World Scientific. 321~p.


\bibitem{10-s-1} %10
\Aue{Sinitsyn, I.\,N.} 1997.
 Konechnomernye raspredeleniya s invariantnoy meroy v stokhasticheskikh
 nelineynykh differentsial'nykh sistemakh [Finite dimensional distributions
 with invariant measure in stochastic nonlinear differential systems].
 Moscow: Dialog--MGU. 129--140.

 \bibitem{2-s-1} %11
\Aue{Pugachev, V.\,S., and I.\,N.~Sinitsyn}. 2001.
\textit{Stochastic systems. Theory and  applications.}
 Singapore: World Scientific. 908~p.


\bibitem{11-s-1} %12
\Aue{Sinitsyn, I.\,N., E.\,R.~Korepanov, and  V.\,V.~Belousov}.  2002.
 Tochnye metody rascheta statsionarnykh rezhimov s invariantnoy
 meroy v stokhasticheskikh sistemakh upravleniya [Exact  analysis of
 stationary with invariant\linebreak
  measure regimes in stochastic control systems].
 \textit{Kibernetika i Tekhnologii XXI~veka: Tr. II Mezhdunar. Nauch.-Tekhnich.
 Konf.} [Cybernetics and High Technologies of XXI~Century.
2nd~International Science and\linebreak Technology Conference Proceedings] C\&T'2002.
 Voronezh: Sakvoe. 124--131.

\bibitem{12-s-1} %13
\Aue{Sinitsyn, I.\,N., E.\,R.~Korepanov, and V.\,V.~Belousov}.  2002.
{Tochnye analiticheskie metody v statisticheskoy dinamike
nelineynykh informatsionno-upravlyayushchikh sistem} [Exact analytical
methods in statistical dynamics of nonlinear informational and control issue].
\textit{Sistemy i Sredstva Informatiki}~---
\textit{Systems and Means of Informatics}. Spets. Vyp. Matematicheskoe i
algoritmicheskoe obespechenie informatsionno-telecommunikatsionnykh sistem
[Mathematical sofware for information and telecommunication systems].
Moscow: Nauka. 112--121.


\bibitem{14-s-1}
\Aue{Sinitsyn, I.\,N.} 2012.
 Razvitie metodov analiticheskogo modelirovaniya raspredeleniy s invariantnoy
 meroy v stokhasticheskikh sistemakh [Development of analytical modeling
 methods for distributions with invariant measure in stochastic systems].
 \textit{Sovremennye Problemy Prikladnoy Matematiki, Informatiki, Avtomatizatsii,
 Upravleniya: Materialy Mezhdunar. Seminara}
 [Modern Problems of Applied Mathematics Informatics, Atomization and Control:
 Seminar (International) Proceedings].
 Sevastopol':  SevNTU. 24--35.

\bibitem{15-s-1}
\Aue{Sinitsyn, I.\,N.} 2012.
 Analiticheskoe modelirovanie raspredeleniy s invariantnoy meroy v
 stokhasticheskikh sistemakh s avtokorrelirovannymi shumami
 [Analytical modeling of distributions with invariant measure in stochastic
 systems with autocorrelated noise].
 \textit{Informatika i ee Primeneniya}~--- \textit{Inform. Appl.} 6(4):4--8.

\bibitem{16-s-1}
\Aue{Sinitsyn, I.\,N.}  2013.
Analiticheskoe modelirovanie raspredeleniy s invariantnoy meroy v
stokhasticheskikh sistemakh s razryvnymi kharakteristikami
[Analytical modeling of distributions with invariant measure in stochastic
systems with discontinuous nonlinearities].
\textit{Informatika i ee Primeneniya}~--- \textit{Inform. Appl.} 7(1):3--11.

\bibitem{17-s-1}
\Aue{Sinitsyn, I.\,N.}  2013.
Parametricheskoe statisticheskoe i analiticheskoe modelirovanie raspredeleniy
v nelineynykh stokhasticheskikh sistemakh na mnogoobraziyakh
[Parametric statistical and analytical modeling of distributions in stochastic
systems on manifolds].
\textit{Informatika i ee Primeneniya}~--- \textit{Inform. Appl.} 7(2):4--16.

\bibitem{18-s-1}
\Aue{Sinitsyn, I.\,N., and  V.\,I.~Sinitsyn}.  2013.
\textit{Lektsii po normal'noy i ellipsoidal'noy approksimatsii raspredeleniy
v stokhasticheskikh sistemakh} [Lectures on normal and ellipsoidal
approximation of distributions in stochastic systems].
Moscow: TORUS PRESS. 488~p.

\bibitem{20-s-1} %19
\Aue{Sinitsyn, I.\,N.}  1986.
{Stochastic hereditary control systems}.
\textit{Problems Control Inform. Theory} 15(4):287--298.


\bibitem{19-s-1} %20
\Aue{Sinitsyn, I.\,N.}  1987.
Konechnomernye raspredeleniya protsessov v stokhasticheskikh integral'nykh i
integ\-ro\-dif\-fe\-ren\-tsi\-al'nykh sistemakh [Finite dimensional distributions
of processes in stochastic integral and integrodifferential systems].
\textit{2nd Symposium (International) IFAC on Stochastic Control}.
Preprints. Vilnius, 1986. Pergamon Press. 1:144--153.

\bibitem{22-s-1} %21
\Aue{Sinitsyn, I.\,N., V.\,I.~Sinitsyn, E.\,R.~Korepanov, V.\,V.~Belousov,
I.\,V.~Sergeev, and D.\,A.~Basilashvili}. 2012.
Opyt modelirovaniya ereditarnykh stokhasticheskikh sistem
[Experience of modeling in hereditary stochastic systems].
\textit{Kibernetika i Vysokie Tekhnologii XXI~Veka: Sbornik dokladov
XIII Mezhdunar. Nauch.-Tekhnich. Konf.}
[Cybernatics and High Technologies of XXI~Century.
13th  Scientific and Technological Conference (International) Proceedings].
Voronezh: Sakvoee. 2:346--357.


\bibitem{21-s-1} %22
\Aue{Sinitsyn, I.\,N.}  2014.
Analiz i modelirovanie raspredeleniy v ereditarnykh stokhasticheskikh sistemakh
[Analysis and modeling of distributions in hereditary stochastic systems].
\textit{Informatika i ee Primeneniya}~--- \textit{Inform. Appl.} 8(1):2--11.



\bibitem{23-s-1}
\Aue{Nemytskiy, V.\,V., and V.\,V.~Stepanov}.  1949.
\textit{Ka\-chest\-ven\-naya teoriya differentsial'nykh uravneniy}
[Analytical theory of differential equations].
Moscow--Leningrad: Gostekhizdat. 448~p.

\bibitem{24-s-1}
\Aue{Kozlov, V.\,V.} 1987.
O~sushchestvovanii integral'nogo invarianta gladkikh dinamicheskikh sistem
[Existence of integral invariants in oblique dynamical systems].
\textit{Prikladnaya Mekhanika i Matematika}
[Appl. Mech. Math.] 1:~538--545.

\bibitem{25-s-1}
\Aue{Sinitsyn, I.\,N.} 2007.
\textit{Fil'try Kalmana i Pugacheva} [Kalman and Pugachev filters].
 2nd ed. Moscow: Logos.  776~p.

%\bibitem{26-s-1}
%\Aue{Sinitsyn, I.\,N.} 2009.
%\textit{Kanonicheskie predstavleniya sluchaynykh funktsiy i ikh primenenie v
%zadachakh komp'yuternoy podderzhki nauchnykh issledovaniy}
%[Canonical expansions of random functions and their application to scientific
%computer-aided support]. Moscow: TORUS PRESS. 768~p.

       \end{thebibliography}
} }


\end{multicols}

\vspace*{-12pt}

\hfill{\small\textit{Received January 16, 2014}}

\vspace*{-24pt}


\Contrl

\vspace*{-2pt}

\noindent
\textbf{Sinitsyn Igor N.} (b.\ 1940)~---
Doctor of Science in technology, professor, Honored scientist of RF, Head of Department, Institute of
Informatics Problems, Russian Academy of Sciences, 44-2 Vavilov Str., Moscow 119333, Russian
Federation; sinitsin@dol.ru

\vspace*{-12pt}

 \label{end\stat}

\renewcommand{\bibname}{\protect\rm Литература}