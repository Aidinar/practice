%\renewcommand{\figurename}{\protect\bf Figure}

\def\stat{lukmorozov}


\def\tit{ON THE  OVERFLOW PROBABILITY ASYMPTOTICS IN~A~GAUSSIAN~QUEUE}

\def\titkol{On the  overflow probability asymptotics in a Gaussian
queue}

\def\autkol{O.\,V.~Lukashenko, E.\,V.~Morozov, and~M.~Pagano}

\def\aut{O.\,V.~Lukashenko$^{1,2}$, E.\,V.~Morozov$^{1,2}$, and~M.~Pagano$^3$}

\titel{\tit}{\aut}{\autkol}{\titkol}

%{\renewcommand{\thefootnote}{\fnsymbol{footnote}}
%\footnotetext[1] {The work of first and second  authors is partially supported by the
%Program of Strategy development of Petrozavodsk State University in
%the framework of the research activity. The third author is a
%postdoctoral fellow with the Research Foundation-Flanders
%(FWO-Vlaanderen).}}

\renewcommand{\thefootnote}{\arabic{footnote}}
\footnotetext[1]{Institute of Applied Mathematical
Research, Karelian Research Center, Russian Academy of Sciences,
11 Pushkinskaya Str., Petrozavodsk 185910,
Russian Federation}
\footnotetext[2]{Petrozavodsk State University, 33 Lenin Str., Petrozavodsk 185910,
Russian Federation}
\footnotetext[3]{University of Pisa, 43 Lungarno Pacinotti, Pisa 56126,
Italy}


\vspace*{-12pt}

\def\leftfootline{\small{\textbf{\thepage}
\hfill INFORMATIKA I EE PRIMENENIYA~--- INFORMATICS AND APPLICATIONS\ \ \ 2014\ \ \ volume~8\ \ \ issue\ 2}
}%
 \def\rightfootline{\small{INFORMATIKA I EE PRIMENENIYA~--- INFORMATICS AND APPLICATIONS\ \ \ 2014\ \ \ volume~8\ \ \ issue\ 2
\hfill \textbf{\thepage}}}




\Abste{Gaussian processes are a powerful tool in network modeling since they permit
to capture the long memory property of actual traffic flows. In more detail,
under realistic assumptions, fractional Brownian motion (FBM) arise as the
limit process when a huge number of on-off sources (with heavy-tailed sojourn
times) are multiplexed in backbone networks. This paper  studies  fluid queuing
systems  with a constant service rate fed by a  sum of independent FBMs,
corresponding to the aggregation of heterogeneous traffic flows. For such
queuing systems, logarithmic asymptotics of the overflow probability, an upper
bound for the loss probability in the corresponding finite-buffer queues, are
derived, highlighting that the FBM with the largest Hurst parameter dominates
in the estimation. Finally, asymptotic results for the workload maximum  in
the more general case of a Gaussian input with slowly varying at infinity
variance are given.}

\KWE{Gaussian fluid system; overflow probability;
logarithmic asymptotics}

\DOI{10.14357/19922264140203}


\vskip 10pt plus 9pt minus 6pt

      \thispagestyle{myheadings}

      \begin{multicols}{2}

                  \label{st\stat}



\section{Introduction}

\noindent
Gaussian processes are well-recognized models to describe the
traffic dynamics of a wide class of modern telecommunication
networks. The main motivation to apply these  models is
their ability of capturing, in a simple and parsimonious way,
the properties of self-similarity and long-range dependence, which are
inherent in multimedia
network traffic~\cite{Leland, Willinger}. Self-similarity means that
the distribution of the process remains unchanged under suitable
scaling of time and space, while long-range dependence implies a
slow decay of the autocorrelation function. These properties make
difficult the probabilistic analysis and, as a consequence, to
obtain  key performance characteristics, crucial  to evaluate   the Quality
of Service (QoS) provided  by the considered networks,
 in an explicit form.

The FBM
is one of the most studied self-similar long-range dependent
Gaussian processes. Its use as a traffic model is supported  by the
following theoretical analysis~\cite{Taqqu}:
the sum of an increasing  number of the so-called on-off inputs,
with either on-times or off-times having a heavy-tailed distribution
with infinite variance, converges weakly to an FBM, after an
appropriate time  scaling. If an FBM is  the  input to a queueing
system, then  let call it  fractional Brownian (FB) input.


One of the main characteristics of the queueing systems is the  {\it
overflow probability}, i.\,e.,
the probability that the workload process exceeds a finite threshold.

In  Gaussian queueing systems with infinite buffer, the
analysis of the overflow probability (closely related  to the
workload maximum) is reduced to the analysis of the extremes of
Gaussian processes~\cite{Reich}.

There are  no explicit expressions for the overflow probability in
queueing  systems with general Gaussian input (including FB input),
while   a few   asymptotic results    are available. In this regard,
let  mention the following key works~[5--7].
It is important to stress that in a general setting, the asymptotic
analysis of the overflow probability is
 based on a number of the assumptions which sometimes are  difficult
to verify. The loss probability in the Gaussian queues with the FB
input  and a finite buffer  is  studied in~[8--12]. Also,  let mention  closely related
 works~[13--17], where the maximum of the
workload process is studied.  Since explicit analysis is
unavailable in general case, the numerical analysis of the
overflow probability  presented in~[18--22]  plays an important
role in the studying of the Gaussian queueing systems. Note that
analysis of the systems with the Brownian input is much easier
because this process has independent increments. This property
allows to obtain the tail probability  for the maximum of the
Brownian motion~\cite{Takacs} in an explicit form. In turn, this
result is directly connected with the overflow probability in the
queueing system fed by the Brownian input.

In this paper, the authors first present the asymptotic analysis of the
overflow probability in the  queueing system where the input is a
sum of the independent FBMs. Thus, they extend the result which has
been proved in a seminal work~\cite{Duffield} for   only FB input.
The present authors follow  mainly the approach developed in~\cite{Duffield} and
discuss in brief  inevitable differences in the proofs. For this
reason, the  proof is straightforward and more transparent than that
can be extracted from  the related works~\cite{Debicki2,Duffy},
where generalizations of the basic model from~\cite{Duffield} are
studied. In particular,  the proof in~\cite{Duffy}
is based on a number of rather complicated  assumptions some of
which, as was mentioned, are not easy to verify for the specific
models.

In summary, the present work  reviews the main  results on the
asymptotics of the overflow probability  in Gaussian  queueing
systems
and also discusses some new results on the workload maximum.

The  paper is organized as follows.  Section~2 contains the description
of Gaussian queueing systems, while section~3 presents the proof of the
overflow probability asymptotics for the
superposition of independent identically distributed  (i.i.d.)\
FB inputs.  Finally, in section~4, the  results concerning the workload
maximum,  when  the input process belongs to  a wide class of the
Gaussian processes, are  analyzed.

\vspace*{-6pt}


\section{Theoretical Background}

\noindent
First of all, the authors motivate their interest to Gaussian queueing  systems.
To this aim, they consider~$N$ i.i.d.\ {\it on-off sources}, modelling
the traffic flows generated by independent connections.  Each  source~$k$ is
described by the process $\{I_k(t)$, $t \geq 0\}$, $k=1,\ldots ,N$,
where
\begin{equation*}
I_k(t)=\begin{cases}
 1\,, &\ t\in \mbox{ on-period}\,; \\
 0\,, &\ t\in \mbox{ off-period.} \\
\end{cases}
%\label{Luk-l1}
\end{equation*}
During an {\it on-period}, a source  is active, while  it keeps
silence (inactive) during the following {\it off-period}.  The
on-off periods are i.i.d.\ and  form an {\it alternating renewal
process}. Furthermore, the processes formed by different sources are
assumed to be independent. As a result, the aggregated traffic
(cumulative workload) generated by all $N$ sources during
time interval $[0,t]$ is given by
\begin{equation*}
A_N (t):=\int\limits_0^{t} \left( \sum\limits_{k=1}^N {I_{k}(u)}
\right)du\,.
\end{equation*}
It is assumed  that there are $M$ types of sources , and  $N_i$ is
the number of the $i$th type sources, $i=1,\ldots ,M$, so  $\sum\limits_{i=1}^M
N_i=N$. The statistical behavior of the cumulative workload
crucially depends on the distribution of on-off periods. Let
$F_{\mathrm{on}}^i,\, F_{\mathrm{off}}^i$ be the distribution of on- and off-period,
respectively. Let assume that the following conditions hold:
\begin{equation}
\left.
\begin{array}{rl}
1-F_{\mathrm{on}}^i(x)& \sim  \ell_{\mathrm{on}}^i x^{-\alpha_{\mathrm{on}}^i}L_{\mathrm{on}}^i(x)\,;
\\[9pt]
1-F^i_{\mathrm{off}}(x)& \sim  \ell_{\mathrm{off}}^i
x^{-\alpha_{\mathrm{off}}^i}L_{\mathrm{off}}^i(x),\,\,\,x\to \infty\,,
\end{array}
\right\}
\label{3}
\end{equation}
where $\ell_{\mathrm{on}}^i$ and $\ell_{\mathrm{off}}^i$ are the
positive constants; exponents
$\alpha_{\mathrm{on}}^i,\alpha_{\mathrm{off}}^i\in (1,\,2)$;
and functions $L_{\mathrm{on}}^i$ and
$L_{\mathrm{off}}^i$ are slowly varying at infinity, i.\,e., for any $t >0$,
$$
\lim\limits_{x \to \infty} \fr{L^i(tx)}{L^i(x)}=1\,,\enskip i=1,\ldots,M\,.
$$
(Relation  $a\sim b$ means that $a/b\to 1$.) Indeed, conditions~(\ref{3})
mean that the distributions $F_{\mathrm{on}}^i$ and  $F_{\mathrm{off}}^i$
are {\it heavy-tailed}. For each~$i$, denote by $\mu_{\mathrm{on}}^i$,
$\mu_{\mathrm{off}}^i$ the mean length of  on-  and off-period, respectively
(note that $\mu_{\mathrm{on}}^i$ and $\mu_{\mathrm{off}}^i<\infty$ because
$\alpha_{\mathrm{on}}^i$ and $\alpha_{\mathrm{off}}^i>1$).
It has been  shown in~\cite{Taqqu}  that the scaled cumulative workload arrived during
period $[0,\,Tt]$ converges weakly to a sum of independent
FBMs provided that:
\begin{enumerate}[($i$)]
\item $N_i\to \infty$ such that
 $\lim\limits_{N\to \infty}N_i/N>0$ for each~$i$;
 and
 \item the scaling factor $T\to \infty$.
 \end{enumerate}
 This functional limit theorem leads to the following approximation:
\begin{multline*}
A(tT)\approx T\left( \sum\limits_{i=1}^M N_i
\fr{\mu_{\mathrm{on}}^i}{\mu_{\mathrm{on}}^i+\mu_{\mathrm{off}}^i} \right)t\\
{} + \sum\limits_{i=1}^M
T^{H_i} \sqrt{L_i(T)N_i}c_i B_{H_i}(t)
\end{multline*}
where $c_i$ are the positive constants; $L_i$ are the slowly varying at
infinity functions (expressed in the terms of given  parameters); and
$B_{H_i}$ are the independent FBMs with the Hurst parameters~$H_i$
 given by
$$
H_i=\fr{3-\min(\alpha_{\mathrm{on}}^i,\,\alpha_{\mathrm{off}}^i)}{2}\in
\left(\fr{1}{2},\,1 \right)\,,\enskip i=1,\ldots, M\,.
$$
Thus, the aggregated traffic generated by a large number of
i.i.d.\ heavy-tailed on-off sources is approximated by a
superposition of  the independent FBMs with a linear drift. This
result gives a motivation to consider  a  queueing system  fed by a
sum of independent FBMs as  a suitable model for a wide class of
modern telecommunication systems.

Now, let describe  a {\it fluid queue} with a constant service rate~$C$
driven by the input process $\{A(t),\,t\ge 0\}$ which is defined
as follows:
 $$
 A(t)=mt+X(t)
 $$
 where $m>0$ is the  mean input rate and the process $X:=\{X(t)\}$
 is the  sum of~$M$ independent
 FB inputs such that  the $i$th summand  has
the   Hurst parameter $H_i \in (1/2,1)$.
 Obviously, $A(t)$ describes the amount of data (workload)
arrived into a communication node within time interval $[0, t]$.
Thus, the variance of the input process in  interval $[0, t]$ is
 $$
 v(t)=\sum\limits_{i=1}^M t^{H_i}\,.
 $$
Introduce the parameter
%{\it traffic intensity}
$r:=C-m$. Denote
$W(t)=X(t)-rt$ and  let  $Q(t)$ be the current workload  at instant~$t$.
If $Q(0)=0$, then the workload $Q(t)$ satisfies the  following
equation~\cite{Reich}:
\begin{multline*}
Q(t)=_d \sup\limits_{0 \leq s \leq t}(A(t)-A(s)-C(t-s))\\
{}= \sup\limits_{0 \leq s \leq t}(X(t)-X(s)-r(t-s))\\
{}=\sup\limits_{0 \leq s \leq t}(W(t)-W(s))
%\label{6a}
\end{multline*}
where symbol $=_d$ stands for the equality in distribution.
 If, moreover, $r>0$, then the
system is stable and a stationary  workload process~$Q$ exists such
that~\cite{Mandjes}:
\begin{multline}
Q=_d \sup\limits_{t \in \T} \left( A(t)-Ct \right)\\
{}= \sup\limits_{t \in \T} W(t)\;\;\; (\T=\Z_+\, \mbox{or}\,\,
\T=\R_+)\,.
\label{asymp-l1}
\end{multline}
Hence, for an  arbitrary threshold $b\in [0,\,\infty)$, the {\it overflow
probability} is defined as
\begin{equation*}
\P(Q>b)=\P\left( \sup\limits_{t \in \T} W(t) >b \right).
%\label{largebuff-l1}
\end{equation*}
It is worth mentioning that some nonasymptotic upper bounds for the
overflow probability have been proposed.
For instance, in case of ordinary FB input ($H$ is the Hurst parameter)
and $\T=\Z_+$, it was shown that~\cite{Fidler1, Fidler2}
$$
\P(Q>b)\le \fr{\Gamma\left( 1/(2\beta) \right)}{2\beta(-\log \eta)^{{1}/(2\beta)}}
$$
where $\Gamma$ denotes the Gamma-function, $\beta \in (0,1-H)$ is the
free parameter and
\begin{multline*}
\eta=\exp\left( -\fr{1}{2} \left( \fr{C-m}{H+\beta} \right)^{2(H+\beta)}\right.\\
\left.{}\times
\left( \fr{b}{1-(H+\beta)} \right)^{2-2(H+\beta)} \right).
\end{multline*}
This result can be extended to general Gaussian inputs, but the value of~$\eta$
can be estimated only by numerical methods (see~\cite{Luk0}).

\vspace*{-6pt}

\section{Asymptotics of~the~Overflow Probability for~a~Superposition of~Fractional
Brownian Inputs}

\noindent
The following result shows that the  FB input  with the largest
Hurst parameter dominates in the asymptotic analysis of the overflow
probability. Recall that the  presented proof is mainly based on
the technique developed in~\cite{Duffield} where
a  system with  single FB input process has been  analyzed.

\smallskip

\noindent
\textbf{Theorem 3.1.}\
\textit{For the  stationary workload~$(\ref{asymp-l1})$, the following
asymptotic holds}:

\noindent
\begin{multline*}
\lim_{b \to \infty}b^{2H-2}\log \P (Q>b)\\
{}=-\fr{r^{2H}}
{ 2H^{2H}(1-H)^{2(1-H)} }:=-\Theta
%\label{ar-l5}
\end{multline*}
\textit{where} $H=\max(H_1,\ldots ,H_M).$

\smallskip

\noindent
P\,r\,o\,o\,f.\ \ Consider the following  relations:

\noindent
\begin{multline}
\P(W(t)/t>x)=\P\left(\Nor(0,1)>
\fr{(x+r)t}{\sqrt{v(t)}} \right)\\
{}=\Psi\left( \fr{(x+r)t}{\sqrt{v(t)}} \right)
\label{ar-l2}
\end{multline}
where

\noindent
$$
\Psi(x):=\fr{1}{\sqrt{2\pi}}\int\limits_x^\infty e^{-y^2/2}dy
$$
is the tail distribution of the standard normal variable
$\Nor(0,1)$.   Function~$\Psi$ satisfies   the following
inequalities~\cite{Mandjes} for   $x>0$:

\noindent
\begin{equation*}
\fr{1-x^{-2}}{x\sqrt{2\pi}}\, e^{-{x^2}/2} \le \Psi(x) \le
\fr{1}{x\sqrt{2\pi}}\, e^{-x^2/2},
%\label{norm-l2}
\end{equation*}
which, in turn, imply  the approximation

\noindent
\begin{equation}
\log \Psi(x) \sim -\fr{x^2}{2}\,,\enskip x \to \infty\,.
\label{ar-l1}
\end{equation}
Denote

\noindent
$$
\nu(t)=\fr{t^2}{v(t)}
$$
and note that $\nu(t) \sim t^{2-2H} \to \infty$ as $t \to \infty$.
It now follows from~\eqref{ar-l1} and~\eqref{ar-l2} that the
following limit exists:

\vspace*{-6pt}

\noindent
\begin{multline}
\lim\limits_{t \to \infty} \fr{1}{\nu(t)}\,\log \P\left( W(t)/t >x
\right){}\\
{}=-\fr{1}{2}\left(x+r\right)^2:=-\lambda(x)\,.
\label{ar-l4}
\end{multline}
It is easy to check that
$$
\inf\limits_{c>0}c^{2H-2}\lambda(c)=\Theta\,.
$$
Let emphasize  that, in contrast to this  straightforward
analysis, the proof of~\eqref{ar-l4} in~\cite{Duffield, Duffy}
(obtained for general non-Gaussian case)  is  based on a large
deviation principle and  some  technical conditions placed on the
logarithmic moment generating function.

To prove the statement of theorem~3.1, it suffices to establish the
following lower and upper bounds:
\begin{align}
\liminf_{b \to \infty}\fr{\log \P (Q>b)}{\nu(b)}&\geq
-\inf\limits_{c>0}c^{2H-2}\lambda(c)\,;\label{asymp-l2}\\
\limsup\limits_{b \to \infty}\fr{\log \P (Q>b)}{\nu(b)}&\leq -\inf_{c>0}
c^{2H-2}\lambda(c) \label{asymp-l3}\,.
\end{align}
First, let note that for each $c>0$,
\begin{multline*}
\liminf\limits_{b \to \infty}\fr{\log \P (Q>b)}{\nu(b)} \geq
\liminf\limits_{b \to \infty} \fr{\log \P \left( W(b/c)>b \right)}{\nu(b)}\\
{}=\lim_{t \to \infty}\fr{ \log \P (W(t)/t >c)}{\nu(t\,c)}
= -c^{2H-2}\lambda(c)
\end{multline*}
and, thus,  the lower bound follows.

The proof of the upper bound is  more challenging. First, let consider the
 discrete time case $\T=\Z_+$ and then  verify some technical conditions
to extend this  result to continuous time case $\T=\R_+$.

Consider an arbitrary $d>0$; then, one obtains

\noindent
\begin{multline}
\P (Q>b) \leq \P \left( \sup_{n<b/d}W(n)>b \right)\\
{}+\P \left(
\sup\limits_{n \geq b/d}W(n)>b \right)\\
{}\leq \fr{b}{d}\sup_{c>d}\P (W(b/c)>b) +\sum\limits_{n \geq b/d}\P
(W(n)>b).
\label{asymp-l7}
\end{multline}
Denote

\noindent
\begin{equation}
\left.
\begin{array}{c}
f_1(b)=\displaystyle\fr{b}{d}\sup\limits_{c>d}\P (W(b/c)>b)\,;\\[9pt]
f_2(b)=\displaystyle\sum\limits_{n \geq b/d}\P (W(n)>b)\,.
\end{array}
\right\}
\label{16}
\end{equation}
It is easy to check that if  $(b+rk)/\sqrt{v(k)}>1$, then
\begin{multline}
\P(W(k) >b) = \Psi \left( \fr{b+rk}{\sqrt{v(k)}} \right) \le
\exp \left( -\fr{(b+rk)^2}{2v(k)} \right)\\
{}\le \exp\left(-\fr{r^2}{2}\nu(k)\right)\,.
\label{ar-l3}
\end{multline}
Also, note   that  for $t\ge1$,
\begin{equation}
\nu(t)=\fr{t^2}{\sum\limits_{i=1}^M t^{H_i}}\geq
\fr{1}{M}t^{2-2H}\,.
\label{logbuff-l13}
\end{equation}
Denote

\noindent
$$
 a=2-2H>0\,,\enskip \gamma=\fr{r^2}{2M}\,.
 $$
It  then follows from~\eqref{ar-l3} and~\eqref{logbuff-l13} that

\columnbreak

\noindent
\begin{multline}
\limsup\limits_{b \to \infty} \fr{\log f_2(b)}{\nu(b)}\\
{} \le M\limsup\limits_{b \to
\infty}\fr{1}{ b^{a}}\,\log \left[\sum\limits_{k=\lfloor b/d
\rfloor}^\infty e^{-\gamma k^a}\right]\,.
\label{asymp-l4}
\end{multline}
 Note that if $k \geq \lfloor b/d \rfloor$ and $k-1\leq x \leq k$,
then the inequality $ e^{-\gamma k^a}\leq e^{-\gamma x^a}$ holds. Hence,

\vspace*{-2pt}

\noindent
\begin{multline}
\sum\limits_{k=\lfloor b/d \rfloor}^\infty e^{-\gamma k^a} \leq
\int\limits_{\lfloor b/d \rfloor-1}^\infty e^{-\gamma x^a}dx\\[-1pt]
{} \leq
\int\limits_{b/d-2}^\infty e^{-\gamma x^a}dx\,.
\label{asymp-l5}
\end{multline}
It follows from~(\ref{asymp-l4}) and~(\ref{asymp-l5}) that

\vspace*{-2pt}

\noindent
\begin{multline}
\limsup\limits_{b \to \infty}\fr{ \log f_2(b)}{\nu(b)}\\
{}\leq M
\limsup\limits_{b \to \infty}\fr{1}{ b^{a}}\,\log
\left[ \int\limits_{b/d-2}^\infty
e^{-\gamma x^a}dx\right]
\label{ar-l6}
\end{multline}
and by applying  the L'H$\hat{\mbox{o}}$pital's rule twice, one obtains the
following limit:
\vspace*{-2pt}

\noindent
\begin{multline*}
\lim\limits_{b \to \infty} \fr{1}{b^a}\log \int\limits_{b/d-2}^\infty e^{-\gamma x^a}dx\\
{} =-\fr{1}{da}\,\lim\limits_{b \to \infty} \left
[e^{-\gamma(b/d-2)^a}b^{1-a} \fr{1}{\int\limits_{b/d-2}^\infty
e^{-\gamma x^a}dx}\right]\\
{}=-\fr{\gamma}{d}\,\lim\limits_{b \to \infty}\left(
\fr{b/d-2}{b} \right)^{a-1} =-\gamma d^{-a}\,.
\end{multline*}
 Now, let  choose  $d \in \left(0,\,((1-H)r)/H \right)$ such  that
$$
-\gamma d^{-a}\le -\inf\limits_{c>0} c^{-a}\lambda(c)\,.
$$
It then follows from~\eqref{ar-l6} that

\noindent
\begin{equation}
\hspace*{-3mm}\limsup\limits_{b \to \infty} \fr{\log\left[ \sum\limits_{n \geq b/d} \P
(W(n)>b)\right]}{\nu(b)}  \leq -\inf\limits_{c>0}\fr{\lambda(c)}{c^{a}}.\!\!
\label{asymp-l8}
\end{equation}
Consider the term~$f_1$ from~(\ref{16}) and note that
\begin{multline*}
\limsup\limits_{b \to \infty}\fr{1}{\nu(b)}\,\log f_1(b)\\
{}=
\limsup\limits_{b \to \infty}\fr{1}{\nu(b)}\,\log \left[
\fr{b}{d}\sup\limits_{c>d} \P
\left(W(b/c)>b\right) \right]
\end{multline*}


\noindent
\begin{multline*}
{}=\limsup\limits_{b \to \infty}\fr{1}{\nu(b)}\,\log \left[\sup\limits_{c>d} \P
\left(W(b/c)>b\right)\right]\\
{}=\limsup\limits_{b \to \infty}\,\sup\limits_{c>d}\fr{1}{\nu(b)}\,\log  \P
\left(W(b/c)>b\right)\\
{}=\limsup\limits_{n \to \infty}\,\sup\limits_{c>d}\fr{1}{\nu(nc)}\,\log \P
(W(n)/n>c)\,.
 \end{multline*}
 By~\eqref{ar-l4}, for   any given $\delta>0$, let
 choose  sufficiently large~$n$ such that
 $$
 \log \P (W(n)/n>x) \leq \nu(n) (\delta -\lambda(x))\,.
 $$
 Using the last inequality,
 \begin{multline}
\limsup\limits_{b \to \infty}\fr{\log f_1(b)}{\nu(b)} \leq
\limsup\limits_{n \to \infty}\sup\limits_{c>d}
\fr{\nu(n)}{\nu(cn)}\left[\delta-\lambda(c)\right]\\
{}=\limsup\limits_{n \to \infty} \sup\limits_{c>d}\left[ \fr{\nu(n)
\delta}{h(cn)}-\fr{\lambda(c)\nu(n)}{\nu(cn)} \right]\\
{}\leq\limsup\limits_{n \to \infty} \left[ \sup\limits_{c>d}\fr{\nu(n)
\delta}{h(cn)}-\inf\limits_{c>d}\fr{\lambda(c)\nu(n)}{\nu(cn)}
\right]\\
{}=-\limsup\limits_{n \to \infty}\inf_{c>d}\fr{\lambda(c)\nu(n)}{\nu(cn)}+
\limsup\limits_{n \to \infty}\fr{\nu(n) \delta}{\nu(dn)}\\[3pt]
{}=-\limsup\limits_{n \to \infty}\inf\limits_{c>d}\fr{\lambda(c)\nu(n)}{\nu(cn)}+
\fr{\delta}{ d^{a}}\,.\label{22}
\end{multline}
Consider the  following function:
$$
f(x):=\fr{1}{2}\left(x+r\right)^2 x^{2H-2}\,,\enskip x \geq 0\,.
$$
It is easy to check that $\min f(x)$ is attained at the point
$x=(1-H)r/H>0$. Thus,  for any $d \in \left(0,\,((1-H)r)/H\right)$,
$$
\inf\limits_{c>0} f(c)=\inf\limits_{c>d} f(c)\,.
$$
Moreover,
\begin{multline*}
\inf\limits_{c>d} \fr{\lambda(c)\nu(t)}{\nu(ct)}=\inf\limits_{c>d}
\fr{(c+r)^2 \sum_{i=1}^M c^{2H_i-2}t^{2H_i}}{2\sum_{i=1}^M t^{2H_i}}
\\
{}\geq \fr{\sum_{i=1}^M \inf_{c>d} t^{2H_i} (c+r)^2  c^{2H_i-2}}{2\sum_{i=1}^M t^{2H_i}}\\
{}= \fr{\sum_{i=1}^M t^{2H_i} f\left( ((1-H_i)r)/H_i\right)}
{\sum_{i=1}^M t^{2H_i}}\,.
\end{multline*}
It now easily follows  as $t\to \infty$ that
\begin{multline}
 \fr{\sum_{i=1}^M t^{2H_i} f\left( ((1-H_i)r)/H_i\right)}
 {\sum_{i=1}^M t^{2H_i}}\to
\fr{1}{2}\,f\left( \fr{(1-H)r}{H} \right)\\
{} =
\inf\limits_{c>d}c^{2H-2}\lambda(c)\,.
\end{multline}
Thus,  let obtain  from~(\ref{22})  that
\begin{equation*}
\limsup\limits_{b \to \infty}\fr{\log f_1(b)}{\nu(b)} \le - \inf\limits_{c>0}
c^{2H-2}\lambda(c)+\fr{\delta}{ d^{a}}\,,
\end{equation*}
and because $\delta$ is arbitrary,
\begin{equation}
\limsup\limits_{b \to \infty}\fr{\log f_1(b)}{\nu(b)} \leq -\inf\limits_{c>0}
c^{2H-2}\lambda(c)\,.
\label{asymp-l11}
\end{equation}
Now, let take  into account the following  inequality
\begin{multline}
\limsup\limits_{b \to \infty}\fr{\log \left(f_1(b)+f_2(b)\right)}{h(b)}\\
{}\leq \limsup\limits_{b \to \infty}\fr{ \log\left(\max(f_1(b),f_2(b))\right)+
\log 2}{h(b)}\,.
\label{asymp-l12}
\end{multline}
Then, let combine~(\ref{asymp-l12})  with~(\ref{asymp-l7}),
(\ref{asymp-l8}),  and~(\ref{asymp-l11}) to obtain~(\ref{asymp-l3}).
In turn, inequalities~(\ref{asymp-l2}) and~(\ref{asymp-l3})  imply
$$
\lim\limits_{b \to \infty}  \fr{\log \P
(Q>b)}{\nu(b)}=-\inf\limits_{c>0}c^{2H-2}\lambda(c)\,,
$$
and the proof for $\T=\Z_+$ is completed.


\smallskip

To consider the case  $\T=\R_+$, let  define the process
$\{W^*(n),\,n \in \Z_+\}$ as
$$
W^*(n)=\sup\limits_{0 \le s \le 1} W(n+s)\,.
$$
Note that
$$
\P \left( \sup\limits_{t \in \R_+}W(t) >b\right)=
\P \left( \sup\limits_{n \in \Z_+} W^*(n) >b \right)\,.
$$
Thus, the asymptotics for the process $\{W^*(n),\, n \in \Z_+\}$
implies  the asymptotics for the continuous-time process $\{W(t),\,
t \in \R_+\}$. The lower bound  follows from the  obvious inequality:
$$
\P\left(\sup\limits_{n \in \Z_+} W^*(n)>b\right) \ge
\P\left(\sup\limits_{n \in \Z_+} W(n)>b\right)\,.
$$
Denote
$$
Y(s)=W(n+s)-W(n)\,, \enskip s \geq 0\,,\ \ n \in \Z_+\,,
$$
 and note that $\E Y(s)=-rs$.  Applying  Borell--Sudakov--Tsirelson
 inequality~\cite{Adler, Debicki}, one obtains
\begin{multline}
\P \left( \sup\limits_{0 \leq s \leq 1} Y(s) \geq u \right) \leq
\P \left( \sup\limits_{0 \leq s \leq 1} \left(Y(s)
+rs \right)\geq u \right)\\
{}\leq 2 \P\left( \Nor(a,\sigma^2)>u \right)
\label{asymp-l13}
\end{multline}
where
\begin{align*}
a&:=\mathrm{med}\, \left( \sup\limits_{0 \leq s \leq 1}\left(Y(s) +rs
\right)\right)\,;\\
\sigma^2&:=\sup\limits_{0 \leq s \leq 1} \mathbb{D} Y(s)\,.
\end{align*}
Denote  $\varphi(n)=\theta\nu(n)/n$. It now follows from~(\ref{asymp-l13})
 that
\begin{multline*}
\E \exp\left[\varphi(n)\left(W^*(n)-W(n)\right)\right]\\
{}=
\E \exp\left[\varphi(n)\sup\limits_{0 \leq s \leq 1}Y(s)\right]
\leq 2 \E \exp\left[\varphi(n) \Nor(a,\sigma^2)\right]\\
{}=\exp\left[\fr{\sigma^2\varphi^2(n)+2\varphi(n) a}{2}\right].
\end{multline*}
Thus, one obtains
\begin{multline}
\limsup\limits_{n \to \infty}\fr{1}{\nu(n)}\,\log \E
e^{\theta\nu(n)\left(W^*(n)-W(n)\right)/n} \\
{}\leq \limsup\limits_{n \to \infty}
\fr{\sigma^2\varphi^2(n)+2\varphi(n) a}{2\nu(n)}\\
{}= \limsup\limits_{n \to \infty} \fr{\sigma^2\theta^2\nu(n)+2\theta an}
{2n^2}=0\,.
\label{Duff2}
\end{multline}
Note that the statement
\begin{equation}
\limsup\limits_{n \to \infty}\fr{1}{\nu(n)}\,\log \E
e^{\theta\nu(n)\left(W^*(n)-W(n)\right)/n}=0
\label{ar-l7}
\end{equation}
 is used  as  an assumption  in  the  paper~\cite{Duffield},
while  above,  a detailed proof of~(\ref{ar-l7}) is given for the system with
Gaussian input.  By the H$\ddot{\mbox{o}}$lder's inequality,  one obtains for
$1<p,g<\infty$, ${1}/{p}+{1}/{q}=1$ that
\begin{multline}
\lambda_1(\theta) := \limsup\limits_{n \to \infty } \fr{1}{\nu(n)}\,
\log \E e^{\theta \nu(n) W^*(n)/n}\\
{}= \limsup\limits_{n \to \infty}\fr{1}{\nu(n)}\\
{}\times \log \E e^{\theta
 \nu(n)(W^*(n)-W(n))/n}e^{\theta \nu(n) W(n)/n}
\\
{}\leq \limsup\limits_{n \to \infty} \fr{1}{\nu(n)}\,\log \left\{ \left[ \E
e^{\theta q \nu(n) (W^*(n)-W(n))/n} \right]^{1/q}\right.\\
\left.{}\times \left[ \E
e^{\theta p \nu(n) W(n)/n } \right]^{1/p} \right\}\\
{}= \fr{1}{q}\,\limsup\limits_{n \to \infty} \fr{1}{\nu(n)}\,\log \E
e^{(\theta q\nu(n) \left(W^*(n)-W(n)\right))/n}\\
{}+
\fr{1}{p}\,\limsup\limits_{n \to \infty} \fr{1}{\nu(n)}\, \log
\E e^{(\theta p \nu(n) W(n))/n}
\label{logbuff-l9}
\end{multline}

\vspace*{-12pt}

\noindent
\begin{equation*}
{}\leq \limsup\limits_{n \to \infty}\fr{1}{\nu(n)} \,\log \E
e^{\theta p \nu(n) W(n)/n}=\fr{1}{2}\left(\theta p\right)^2 - \theta p r
\label{logbuff-l10}
\end{equation*}
where  the first term in~(\ref{logbuff-l9}) equals zero
by~(\ref{Duff2}). Taking $p \to 1$, one gets
$$
\lambda_1(\theta)\leq \fr{1}{2}\,\theta^2 - \theta r\,.
$$
On the other  hand,

\noindent
\begin{multline*}
\liminf\limits_{n \to \infty} \fr{1}{\nu(n)}\, \log \E e^{\theta \nu(n)
W^*(n)/n} \\
{}\geq \liminf\limits_{n \to \infty}\fr{1}{\nu(n)}\, \log \E
e^{\theta \nu(n) W(n)/n}=\fr{1}{2}\theta^2 - \theta r\,.
\end{multline*}
Thus, the following limit exists:

\noindent
$$
\lim\limits_{n \to \infty} [\nu(n)]^{-1} \log \E e^{\theta \nu(n)
W^*(n)/n}=\fr{1}{2}\,\theta^2 - \theta r=\lambda_1(\theta)\,.
$$
It now follows  by the G$\ddot{\mbox{a}}$rtner--Ellis theorem~\cite {Dembo}
that the sequence  $\{W^*(n)/n,$ $\, \nu(n)\}$  satisfies a large
deviation principle with the following rate function $\lambda$ which
is   the Fenchel--Legendre transform of function~$\lambda_1$:

\noindent
\begin{multline*}
\sup\limits_{\theta \in \R}\{\theta x-\lambda_1(\theta)\}=
\sup\limits_{\theta \in \R}\left\{ -\fr{\theta^2}{2}+(x+r)\theta
\right\}\\
{}=\fr{1}{2}\left(x+r\right)^2:=\lambda(x)\,.
%\label{logbuff-l11}
\end{multline*}
(For more details, see~\cite{Dembo}.) As a consequence, one obtains
$$
\lim\limits_{n \to \infty} \fr{\P (W^*(n)/n > x)}{\nu(n)} =-\lambda (x)\,.
$$
Thus,   equation~\eqref{ar-l4} for the process $\{ W^*(n) \}$ is
proved.

\smallskip

To establish   the upper bound,  let  repeat
steps~\eqref{asymp-l7}--\eqref{asymp-l12} with $W(n)$ replaced by
$W^*(n)$. The proof
remains unchanged with exception of the point, where another
arguments are used to come to the  upper bound~\eqref{ar-l3}. More exactly,
at this point, Chernoff's  inequality is applied which gives, for any
$\varepsilon>0$,

\noindent
\begin{multline*}
\P \left(W^*(k)>b \right) \leq e^{-\sup\limits_{\theta}\left( \theta b -\log \E
e^{\theta W^*(k)} \right)}\\
{}\leq  e^{-\theta \nu(k) b/k}e^{(\lambda_1(\theta)+\varepsilon)\nu(k)}
\leq e^{(\lambda_1(\theta)+\varepsilon)\nu(k)}
\end{multline*}
where $\theta$ is chosen in such a way  that
$\lambda_1(\theta)+\varepsilon<0$.~\hfill$\square$

\vspace*{-6pt}


\section{Closely Related Results}

\noindent
In this section,  a number of results for
Gaussian queueing systems are discussed in brief which are closely connected  with the
asymptotics of the overflow probability  analyzed above.
  More exactly,  the   asymptotics   for the workload
 maximum are obtained
in the more general case when the variance $v$ of the input process~$X$
is regu-\linebreak\vspace*{-12pt}

\pagebreak

\noindent
larly varying at infinity function with index $0<V<2$, i.\,e.,
for any $y>0$,
$$
\lim_{t \to \infty} \fr{v(yt) }{v(t)}=y^V.
$$
The asymptotic has the following form~\cite{Debicki2, Duffy}
\begin{equation*}
\lim_{b \to \infty} \fr{v(b)}{b^2} \,\log \P(Q>b)=-\Theta
%\label{asymp1-l13}
\end{equation*}
where
\begin{eqnarray}
\Theta=\fr{2}{(2-V)^{2-V}}\left( \fr{r}{V} \right)^V\,.
\label{reg}
\end{eqnarray}
It is well-known that every regularly varying at infinity function can be
represented~as
\begin{equation}
v(t)=t^V L(t)
\label{MEV:1}
\end{equation}
 where function $L(t)$  is slowly varying (as $t\to\infty$) and
index $V\in (0,\,2)$~\cite{Seneta}. Denote $\beta=(2-V)^{-1}$ and
take (arbitrary) $\varepsilon \in (0,2-V)$. Moreover, it is assumed
that the following conditions hold as $t \to \infty$:
\begin{equation}
L(tL^\beta(t)) \sim L(t)\,;
\label{MEV:2}
\end{equation}
 function $L(t)$ is twice differentiable on~$\mathbb{R}_+$ and
\begin{equation}
L''(t)=o\left( \fr{1}{t^{V+\varepsilon}} \right)\,.
\label{MEV:3}
\end{equation}
It follows from~\eqref{MEV:3}  that
\begin{equation}
v''(t)\log t \to 0\,,\enskip t \to \infty\,.
\label{2.2.l20}
\end{equation}
Also, recall that a stationary version $Q^*(t)$ of the workload
process $Q(t)$ exists~\cite{MEV:Konstantopoulos}. Let
\begin{gather*}
\gamma(t)=L[\left(\log t \right)^\beta]  \log t \,;
\\
M(t):=\max\limits_{0 \leq s \leq t}Q(s)\,;\enskip M^*(t):=\max\limits_{0 \leq s \leq t}Q^*(s)\,.
\end{gather*}
The following asymptotic result for the workload maximum  has been
established in~\cite{Luk1} (see also~\cite{Luk3, Luk2}).

\smallskip

\noindent
\textbf{Theorem~4.1.}\
\textit{If the variance $v(t)$ of the Gaussian component~$X$ satisfies
conditions~$(\ref{MEV:1})$--$(\ref{MEV:3})$ and $r>0$, then
\begin{equation}
\frac{M^*(t)}{\gamma^\beta(t)} \Rightarrow
\left(\fr{1}{\Theta}\right)^\beta;\,\,
\frac{M(t)}{\gamma^\beta(t)} \Rightarrow
\left(\fr{1}{\Theta}\right)^\beta,\,\,\,t \to \infty\,,
\label{MEV:4}
\end{equation}
where the constant $\Theta$ is given by expression \eqref{reg},  and
$\Rightarrow$ stands for convergence in probability}.


\smallskip

\noindent
P\,r\,o\,o\,f.\ \  Let give a sketch of the proof which  mainly follows the
technique developed in~\cite{Zeevi}. It is sufficient to prove that
for any $\delta>0$,
\begin{align}
&\P \left( \fr{M^*(t)}{\gamma^\beta(t)} \geq \left(
\fr{1-\delta}{\theta} \right)^\beta \right ) \to 1\,,\enskip t \to
\infty \,; \label{pr-l1}\\
&\P \left( \fr{M^*(t)}{\gamma^\beta(t)} \geq \left(
\fr{1+\delta}{\theta} \right)^\beta \right) \to 0\,,\enskip t \to
\infty\,. \label{pr-l2}
\end{align}
To prove~\eqref{pr-l1}, let fix $\Delta \in (0,t)$ and note that

\noindent
$$
Q^*(t) \geq W(t)-W(t-\Delta)\,.
$$
Denote

\noindent
$$
Y_k^{(\Delta)}=W(k\Delta)-W((k-1)\Delta)\,,
$$
then for each~$t$, one has

\noindent
$$
M^*(t)\geq \max\limits_{1\leq k \leq \lfloor t/\Delta
\rfloor}Y_k^{(\Delta)}\,.
$$
Thus, the original  problem reduces to the analysis of the extremes
of a stationary normal sequence $\{Z_k\}$ (with $Z_k =_d \Nor(0,1)$) since

\noindent
\begin{multline*}
\P \left( \fr{M^*(t)}{\gamma^\beta(t)}\geq \left(
\fr{1-\delta}{\theta}\right)^\beta \right) \\
{}\geq \P \left(
\max\limits_{i=1,\ldots ,\lfloor t/\Delta \rfloor}Z_i \geq u(t)\right) := \P_1(t)\,,
\end{multline*}
where

\noindent
$$
u(t):= \fr{ \alpha(t)+r\Delta(t)}{\sqrt{v(\Delta(t))}}\,,\enskip
\alpha(t):=\left( \fr{1-\delta}{\theta}\,\,\gamma (t) \right)^\beta.
$$
If $\Delta:=\Delta(t)=A\gamma^\beta(t)$, where  $A>0$ is the
constant, is chosen, then it is possible to show that

\noindent
$$
\P_1(t) \to 1\,,\enskip  t \to \infty\,,
$$
as required. To establish~\eqref{pr-l2}, let  consider another stationary sequence:

\noindent
$$
Y_i:=\sup\limits_{s \in [i-1,i)} Q^*(s)\,,\enskip
i=1,\ldots ,\lceil t \rceil\,,
$$
and, thus,

\noindent
$$
M^*(t) \leq \max\{Y_i:\ \, i=1,\ldots ,\lceil t \rceil\}\,.
$$
This  immediately implies

\noindent
\begin{multline*}
\P \left( M^*(t) \geq \left(\fr{1+\delta}{\theta}\right)^\beta
\gamma^\beta(t) \right)\\
{} \leq \P \left( \max\limits_{i=1,\ldots ,\lceil t
\rceil} Y_i \geq \left(\fr{1+\delta}{\theta}\right)^\beta
\gamma^\beta(t)  \right)\\
{}\leq \lceil t \rceil \P \left(Y \geq
\left(\fr{1+\delta}{\theta}\right)^\beta \gamma^\beta(t)
\right) :=\P_2(t)\,.
\end{multline*}
A careful analysis allows to conclude   that

\noindent
$$
\P_2(t) \to 0\,,\enskip  t \to \infty\,,
$$
that completes  the proof.  (More detailed analysis can be found in~\cite{Luk1}.)~\hfill$\square$

\smallskip

Under  slightly less general  assumptions, the  previous result can
be generalized  to the convergence in the  space~$L_p$ for any $p\ge 1$.


\smallskip

\noindent
\textbf{Theorem 4.2.}\
\textit{Let conditions of  theorem~4.1 hold.  If,
moreover,
\begin{equation}
\liminf_{t\to \infty} L(t)>0\,;\enskip \limsup\limits_{t\to \infty} L(t)<\infty\,,
\label{10a}
\end{equation}
then convergence in~$\eqref{MEV:4}$ holds in the space~$L_p$  for any}
$p \in [1,\infty)$.


\smallskip

To prove this statement, it is sufficient to establish the uniform
integrability of the following sequence:\linebreak
$\left\{\left({ M^*(t)}/{\gamma^\beta(t)}\right)^p,\,t\ge T\right\}$
where $T$ is the finite positive constant. Indeed, it is shown
in~\cite{Luk2} that
\begin{equation*}
%\label{2.2.l1}
\sup\limits_{t \geq T} \E \left[
\fr{M^*(t)}{\gamma^\beta(t)} \right]^{p+1} < \infty\,.
\end{equation*}


\noindent
\textbf{Remark.}\
If the  limit

\noindent
\begin{equation*}
\lim\limits_{t\to \infty} L(t)= A\in (0,\,\infty)
%\label{38}
\end{equation*}
 exists, then conditions~\eqref{10a} are automatically fulfilled.


\smallskip

Theorems~4.1 and~4.2 can be applied to some specific
processes.  First, let  consider the following input:
\begin{equation*}
X(t)=\sum\limits_{i=1}^n B_{H_i}(t)\,,\enskip t\ge 0\,,
\end{equation*}
where $B_{H_i}$ are the independent FBMs with the Hurst parameters
$H_i\in(0,\,1)$ and  $H_1>\max\limits_{i>1}H_i$. Then,  the variance
$v(t)$ of the process $X(t)$ satisfies condition~(\ref{MEV:1}), and
 the statements of theorems~4.1 and~4.2 hold  with $V=2H_1$.
 This is an extension of the results derived in~\cite{Zeevi}.

The second example is the integrated Gaussian process~$X$, that is,

\noindent
\begin{equation*}
X(t)=\int\limits_0^t Z(s)\,ds
%\label{41}
\end{equation*}
where $Z$ is the centered stationary Gaussian process with the
covariance function  $R(u):=\mathbb{C}\mathrm{ov}(Z(0),Z(u))$. Such models
 have been considered in~\cite{Kulkarni, Debicki1}. It is easy to
check that the variance $v(t)$ of  $X(t)$ can be written as
\begin{equation}
v(t)=2 \int\limits_0^t \int\limits_0^s R(u)\,duds\,.
\label{42a}
\end{equation}
Then $v''(t)=2 R(t)$ and condition~\eqref{2.2.l20} is equivalent to
\begin{equation*}
R(t)\log t \to 0\,,\enskip t \to \infty\,.
%\label{2.2.l21}
\end{equation*}
If, in addition, $A \in (0,\ \infty)$ and  exists $V \in (0,2)$
such that
\begin{equation}
\fr{\int\limits_0^t \int\limits_0^s R(u)\,duds}{t^V} \to A\,,\enskip
t \to \infty\,,
\label{2.2.l22}
\end{equation}
then conditions of  theorem~4.2 are
satisfied as well. For example, let~$Z$ be the Ornstein--Uhlenbeck
process with $R(t)=\lambda e^{-\alpha t}$ and parameters
$\lambda,\alpha>0$. It then follows from~(\ref{42a}) that
$$
v(t)=\fr{2 \lambda}{\alpha}\,t+\fr{2\lambda}{\alpha^2}\left(e^{-\alpha t}-1\right)
$$
and, hence, condition~\eqref{2.2.l22} is satisfied with  $V=1$ and
$A={\lambda}/{\alpha}$.

Note that the integrated Ornstein--Uhlenbeck process is the Gaussian
counterpart of the well-known Anick--Mitra--Sondi fluid model~\cite{Anick}
(see also~\cite{Addie}), and its relevance for the
modeling of the  traffic in communications systems is  motivated in~\cite{Kulkarni}.

Theorem~4.1 yields the asymptotics for another important
characteristic, called hitting time,  the time required to reach  a
threshold~$b$,
$$
T(b)=\inf\left\{t \geq 0:\,Q^*(t)\geq b\right\}\,.
$$
The analysis of $T(b)$ is based on the  following relation between
$M^*(t)$ and~$T(b)$:
\begin{equation*}
\left\{ T(b) \leq t\right\}=\left\{M^*(t) \geq b \right\}\,.
%\label{time-l0}
\end{equation*}
Finally, the following result proved in~\cite{Luk2} has been got as well.


\smallskip

\noindent
\textbf{Theorem~4.3.}
\textit{If conditions of  theorem~$4.1$ hold and  function
$\gamma(t)$ monotonically increases in  $[t_0,\infty)$, for some
$t_0<\infty$, then}
\begin{equation*}
\fr{\gamma(T(b))}{b^{1/\beta}} \Rightarrow \Theta\,,\enskip b \to \infty\,.
%\label{time-l1}
\end{equation*}

\vspace*{-6pt}

\Ack
This work is done under financial   support
of  the Program of Strategy Development of  Petrozavodsk State
University  in the framework
of the research activity.



\renewcommand{\bibname}{\protect\rmfamily References}

{\small\frenchspacing
{%\baselineskip=10.8pt
\begin{thebibliography}{99}

\bibitem{Leland} %1
\Aue{Leland,~W.\,E., M.\,S. Taqqu, W.~Willinger, and D.\,V.~Wilson.}  1994.
On the self-similar nature of ethernet traffic (extended version).
\textit{IEEE/ACM Trans. Networking} 2(1):1--15.

\bibitem{Willinger} %2
\Aue{Willinger,~W., M.\,S.~Taqqu, W.\,E.~Leland, and D.~Wilson.}
1995. Self-similarity in high-speed packet traffic: Analysis and
modeling of Ethernet traffic measurements. \textit{Stat. Sci}.
10(1):67--85.

\bibitem{Taqqu} %3
\Aue{Taqqu,~M.\,S., W.~Willinger, and R.~Sherman.} 1997. Proof of a
fundamental result in self-similar traffic modeling. \textit{Comp.
Comm. Rev.} 27:5--23.

\bibitem{Reich} %4
\Aue{Reich,~E.} 1958.
On the integrodifferential equation of Takacs~I.
\textit{Ann. Math. Stat.} 29:563--570.

\bibitem{Duffield} %5
\Aue{Duffield,~N., and N.~O'Connell.}  1995.
Large deviations and overflow
probabilities for the general single server queue, with
applications. \textit{Proc. Cambridge Phil. Soc.} 118:363--374.

\bibitem{Debicki2}  %6
\Aue{Debicki,~K.} 1999.
A~note on LDP for supremum of Gaussian processes over infinite
horizon. \textit{Stat. Probab. Lett.} 44:211--220.

\bibitem{Duffy} %7
\Aue{Duffy,~K.,  J.\,T.~Lewis, and W.\,G.~Sullivan.} 2003.
Logarithmic asymptotics for the supremum of a stochastic process.
\textit{Ann. Appl. Probab.} 13(2):430--445.

\bibitem{Kim1} %8
\Aue{Kim,~H.\,S., and N.\,B.~Shroff.}  2001.
Loss probability calculations and asymptotic analysis for finite buffer
multiplexers. \textit{IEEE/ACM Trans. Networking} 9:755--768.

\bibitem{Kim2} %9
\Aue{Kim,~H.\,S., and N.\,B.~Shroff.}  2001.
On the asymptotic relationship between the overflow
probability and the loss ratio. \textit{Adv. Appl. Probab.}
33:836--863.

\bibitem{Luk6} %10
\Aue{Goricheva,~R.\,S., O.\,V.~Lukashenko, E.\,V.~Morozov, and
M.~Pagano.} 2010. Regenerative
analysis of a finite buffer fluid queue. \textit{2010
Congress (International) on Ultra Modern Telecommunications and Control Systems and
Workshops (ICUMT) Proceedings}. 1132--1136.

\bibitem{Luk4} %11
\Aue{Lukashenko,~O.\,V., E.\,V.~Morozov, and M.~Pagano.} 2011.
Estimation of loss probability
in Gaussian queues. \textit{Conference (International) ``Modern
Probabilistic Methods for Analysis and Optimization of Information and
Telecommunication Networks'' Proceedings}. 142--147.

\bibitem{Luk7} %12
\Aue{Lukashenko,~O.\,V., E.\,V.~Morozov, R.\,S.~Nekrasova, and M.~Pagano.} 2013.
Performance evaluation of finite buffer queues through regenerative simulation.
\textit{Comm. Comp. Inform. Sci. BWWQT 2013.} 356:131--139.

\bibitem{Zeevi} %13
\Aue{Zeevi,~A., and P.~Glynn.} 2000. On the maximum workload in a queue fed
by fractional Brownian motion. \textit{Ann. Appl. Probab}. 10:1084--1099.

\bibitem{Husler1} %14
\Aue{H$\ddot{\mbox{u}}$sler,~J., and V.\,I.~Piterbarg.}
2004. Limit theorem for maximum
of the storage process with fractional Brownian as input.
\textit{Stochastic Proc. Their Appl.} 114:231--250.

\bibitem{Luk1} %15
\Aue{Lukashenko,~O.\,V., and E.\,V.~Morozov.} 2012. Asymptotics of the maximum
workload for a class of Gaussian queues. \textit{Informatics and
Its Applications}~--- \textit{Inform. Appl.}
6(3):81--89.

\bibitem{Luk3} %16
\Aue{Lukashenko~O.~V., and E.\,V.~Morozov.} 2012. On the maximum workload for
a class of Gaussian queues. \textit{Conference (International) ``Probability
Theory and Its Applications'' in Commemoration of the Centennial of
B.\,V.~Gnedenko}. 231--232.

\bibitem{Luk2} %17
\Aue{Lukashenko,~O.\,V., and E.\,V.~Morozov.} 2013.
On convergence in the $L_p$
space of the workload maximum for a class of Gaussian queueing
systems. \textit{Informatics and Its Applications}~---
\textit{Inform. Appl.} 7(1):36--43.


\bibitem{Dieker1} %18
\Aue{Dieker,~A.\,B., and M.~Mandjes.} 2005.
On asymptotically efficient simulation of large deviation probabilities.
 \textit{Adv. Appl. Probab.} 37:539--552.

 \bibitem{Michele1} %19
\Aue{Giordano,~S., M.~Gubinelli, and M.~Pagano.}
2005. Bridge Monte-Carlo: A~novel approach to rare events of Gaussian processes.
\textit{5th St.\ Petersburg Workshop on Simulation Proceedings}.
St.\ Petersburg, Russia. 281--286.

\bibitem{Dieker2} %20
\Aue{Dieker,~A.\,B., and M.~Mandjes.} 2006. Fast simulation of overflow
probabilities in a queue with Gaussian input. \textit{ACM Trans. Model.
Comput. Simul.} 16(2):119--151.

\bibitem{Michele2} %21
\Aue{Giordano,~S., M.~Gubinelli, and M.~Pagano.}
2007. Rare events of Gaussian processes:
A~performance comparison between Bridge Monte-Carlo and Importance Sampling.
\textit{Next generation teletraffic and wired/wireless advanced networking}.
St.\ Petersburg, Russia. 269--280.

\bibitem{Luk5} %22
\Aue{Lukashenko,~O.\,V., E.\,V.~Morozov, and M.~Pagano.} 2012.
Performance analysis of bridge Monte-Carlo estimator.
\textit{Trans. KarRC RAS} 3:54--60.

\bibitem{Takacs} %23
\Aue{Takacs,~L.} 1967. \textit{Combinatorial methods in the theory of stochastic processes}.
John Wiley\&Sons. 262~p.

\bibitem{Mandjes} %24
\Aue{Mandjes,~M.} 2007. \textit{Large deviations of Gaussian queues}.
Chichester: Wiley. 339~p.



\bibitem{Fidler1} %25
\Aue{Rizk,~A., and M.~Fidler.} 2010.
Sample path bounds for long memory fbm traffic.
\textit{29th Conference on Information Communications, INFOCOM'10 Proceedings}.
Piscataway, NJ, USA: IEEE Press. 61--65.

\bibitem{Fidler2} %26
\Aue{Rizk,~A., and M.~Fidler.} 2012.
Non-asymptotic end-to-end performance bounds
for networks with long range dependent FBM cross traffic. \textit{Comp. Networks}.
56(1):127--141.


\bibitem{Luk0} %27
\Aue{Lukashenko,~O., E.~Morozov, and M.~Pagano.}  2014.
On the effective envelopes for fluid queues with Gaussian input.
\textit{Comm. Comp. Inform. Sci. DCCN 2013.} 279:178--189.

\bibitem{Adler} %28
\Aue{Adler,~R.\,J.} 1990. \textit{An introduction to continuity, extrema, and
related topics for general Gaussian processes}. Hayward, CA: Institute of
Mathematical Statistics. 160~p.

\bibitem{Debicki} %29
\Aue{Debicki,~K.} 2004. Gaussian processes. \textit{Encyclopedia of actuarial
sciences} 2:752--757.

\bibitem{Dembo} %30
\Aue{Dembo,~A., and O.~Zeitouni.} 1998.
\textit{Large deviations techniques and applications}. Springer. 396~p.

\bibitem{Seneta} %31
\Aue{Seneta,~E.} 1985. \textit{Regularly varying functions}. Springer. 116~p.

\bibitem{MEV:Konstantopoulos} %32
\Aue{Konstantopoulos,~T., M.~Zazanis, and G.~De Veciana}. 1996.
Conservation
laws and reflection mappings with application to multiclass mean
value analysis for stochastic fluid queues. \textit{Stochastic
Proc. Their Appl.} 65:139--146.

\bibitem{Kulkarni} %33
\Aue{Kulkarni,~V., and T.~Rolski.} 1994.
Fluid model driven by an Ornstein--Uhlenbeck process.
\textit{Prob. Eng. Inform. Sci.} 8:403--417.

\bibitem{Debicki1} %34
\Aue{Debicki,~K., and T.~Rolski.}  1995.
A~Gaussian fluid model. \textit{Queueing Syst.} 20:433--452.

\bibitem{Anick} %35
\Aue{Anick,~D., D.~Mitra, and M.\,M.~Sondhi.}  1982.
Stochastic theory of a data handling system with multiple resources.
\textit{Bell Syst. Techn.~J.} 61:1871--1894.

\bibitem{Addie} %36
\Aue{Addie,~R., P.~Mannersalo, and I.~Norros.} 2002. Most probable paths and performance
formulae for buffers with Gaussian input traffic.
\textit{Eur. Trans. Telecomm.} 13:183--196.



\end{thebibliography} } }

\end{multicols}

\vspace*{-6pt}

\hfill{\small\textit{Received March 8, 2014}}

\vspace*{-6pt}

\Contr

\noindent
\textbf{Lukashenko Oleg V.} (b.\ 1986)~---
Candidate of Science (PhD) in physics and mathematics,
junior scientist, Institute of Applied Mathematical Research of Karelian
Research Center, Russian Academy of Sciences; lecturer,
Petrozavodsk State University; lukashenko-oleg@mail.ru

\vspace*{3pt}

\noindent
\textbf{Morozov Evsei V.} (b.\ 1947)~--- Doctor of Science in physics and
mathematics, professor,
leading scientist, Institute of Applied Mathematical Research of Karelian Research Center,
Russian Academy of Sciences, 11 Pushkinskaya Str., Petrozavodsk 185910,
Republic of Karelia, Russian Federation; professor, Petrozavodsk State University,
33 Lenin Str., Petrozavodsk 185910, Republic of Karelia,
Russian Federation; emorozov@karelia.ru

\vspace*{3pt}

\noindent
\textbf{Pagano Michele} (b.\ 1968)~---
 PhD in electronics engineering, associate professor, University of Pisa,
 43 Lungarno Pacinotti, Pisa 56126, Italy;
 m.pagano@iet.unipi.it

\vspace*{12pt}

\hrule

\vspace*{2pt}

\hrule

\vspace*{6pt}

%\newpage


\def\tit{ОБ АСИМПТОТИКЕ ВЕРОЯТНОСТИ ПЕРЕПОЛНЕНИЯ ГАУССОВСКОЙ ОЧЕРЕДИ$^*$}

\def\aut{О.\,В.~Лукашенко$^1$, Е.\,В.~Морозов$^2$,  М.~Пагано$^3$}


\def\titkol{Об асимптотике вероятности переполнения гауссовской очереди}

\def\autkol{О.\,В.~Лукашенко, Е.\,В.~Морозов,  М.~Пагано}

{\renewcommand{\thefootnote}{\fnsymbol{footnote}}
\footnotetext[1]{Работа проводится при финансовой поддержке Программы
стратегического развития Петрозаводского государственного университета в рамках
на\-уч\-но-ис\-сле\-до\-ва\-тель\-ской деятельности.}}


\titel{\tit}{\aut}{\autkol}{\titkol}

\vspace*{-12pt}

\noindent $^1$Институт прикладных математических исследований КарНЦ РАН,
 Россия, Республика Карелия,\\
 $\hphantom{^1}$г.~Петрозаводск 185910, ул.\ Пушкинская 11;
Петрозаводский государственный университет,\\
$\hphantom{^1}$Россия, Республика Карелия, г.~Петрозаводск 185910,
пр.\ Ленина 33; lukashenko-oleg@mail.ru\\
\noindent $^2$Институт прикладных математических исследований КарНЦ РАН,
 Россия, Республика Карелия,\\
  $\hphantom{^1}$г.~Петрозаводск 185910, ул.\ Пушкинская 11;
Петрозаводский государственный университет,\\
 $\hphantom{^1}$Россия, Республика Карелия, г.~Петрозаводск 185910,
 пр.\ Ленина 33; emorozov@karelia.ru\\
\noindent
$^3$Университет г.\ Пиза, Италия; m.pagano@iet.unipi.it


\vspace*{6pt}

\def\leftfootline{\small{\textbf{\thepage}
\hfill ИНФОРМАТИКА И ЕЁ ПРИМЕНЕНИЯ\ \ \ том\ 8\ \ \ выпуск\ 2\ \ \ 2014}
}%
 \def\rightfootline{\small{ИНФОРМАТИКА И ЕЁ ПРИМЕНЕНИЯ\ \ \ том\ 8\ \ \ выпуск\ 2\ \ \ 2014
\hfill \textbf{\thepage}}}



\Abst{Гауссовские процессы являются мощным инструментом в моделировании сетей,
так как они позволяют описать эффект долгой памяти реальных сетевых потоков.
Более точно, при  реалистичных предположениях, дробное броуновское движение
(ДБД) возникает как предельный процесс, когда огромное число on-off
источников (с тяжелыми хвостами) мультиплексируются в магистральных сетях.
В~данной работе изучается жидкостная  система массового обслуживания с постоянной
скоростью обслуживания, с суммой независимых ДБД на входе, что соответствует
агрегации гетерогенных сетевых потоков. Для таких систем массового обслуживания
получена  логарифмическая асимптотика вероят\-ности переполнения, которая является
верхней границей  вероятности потери в соответствующих очередях с конечным буфером
и которая показывает, что в оценке доминирует ДБД с наибольшим параметром Херста.
Наконец, приведены асимптотические результаты для максимума  нагрузки в более
общем случае гауссовского входного процесса с дисперсией, которая правильно
меняется на бесконечности}

\KW{гауссовские жидкостные системы; вероятность переполнения;
логарифмические асимптотики}

\DOI{10.14357/19922264140203}

\vspace*{6pt}


 \begin{multicols}{2}

\renewcommand{\bibname}{\protect\rmfamily Литература}
%\renewcommand{\bibname}{\large\protect\rm References}

{\small\frenchspacing
{%\baselineskip=10.8pt
\addcontentsline{toc}{section}{References}
\begin{thebibliography}{99}

\bibitem{Leland-1} %1
\Au{Leland~W.\,E., Taqqu M.\,S., Willinger~W., Wilson~D.\,V.}
On the self-similar nature of ethernet traffic (extended version).
{IEEE/ACM Transactions of Networking}, 1994. Vol.~2. No.\,1. P.~1--15.

\bibitem{Willinger-1} %2
\Au{Willinger~W., Taqqu M.\,S., Leland~W.\,E., Wilson~D.}
Self-similarity in high-speed packet traffic: Analysis and
modeling of Ethernet traffic measurements~// {Stat. Sci.}, 1995.
Vol.~10. No.\,1. P.~67--85.

\bibitem{Taqqu-1} %3
\Au{Taqqu~M.\,S., Willinger W., Sherman~R.}  Proof of a
fundamental result in self-similar traffic modeling~// {Comp.
Comm. Rev.}, 1997. Vol.~27. P.~5--23.

\bibitem{Reich-1} %4
\Au{Reich~E.} On the integrodifferential equation of Takacs~I~//
Ann. Math. Stat., 1958. Vol.~29. P.~563--570.

\bibitem{Duffield-1} %5
\Au{Duffield~N., O'Connell N.}
Large deviations and overflow
probabilities for the general single server queue, with
applications~// {Proc. Cambridge Phil.
Soc.}, 1995. Vol.~118. P.~363--374.

\pagebreak

\bibitem{Debicki2-1} %6
\Au{Debicki~K.} A~note on LDP for supremum of Gaussian processes over infinite
horizon~// {Stat. Probab. Lett.}, 1999. Vol.~44. P.~211--220.

\bibitem{Duffy-1} %7
\Au{Duffy~K.,  Lewis J.\,T.,  Sullivan~W.\,G.}
Logarithmic asymptotics for the supremum of a stochastic process~//
{Ann. Appl. Probab.}, 2003. Vol.~13. No.\,2. P.~430--445.

\bibitem{Kim1-1} %8
\Au{Kim~H.\,S., Shroff N.\,B.}
Loss probability calculations and asymptotic analysis for finite buffer
multiplexers~// {IEEE/ACM Trans. Networking}, 2001. Vol.~9. P.~755--768.

\bibitem{Kim2-1} %9
\Au{Kim~H.\,S., Shroff N.\,B.}
On the asymptotic relationship between the overflow
probability and the loss ratio~// {Adv. Appl. Probab.},  2001.
Vol.~33. P.~836--863.

\bibitem{Luk6-1} %10
\Au{Goricheva~R.\,S., Lukashenko O.\,V., Morozov~E.\,V.,
Pagano~M.}  Regenerative
analysis of a finite buffer fluid queue~// \textit{2010
Congress (International) on Ultra Modern Telecommunications and Control Systems and
Workshops (ICUMT) Proceedings}, 2010. P.~1132--1136.

\bibitem{Luk4-1} %11
\Au{Lukashenko~O.\,V., Morozov E.\,V.,  Pagano~M.}
Estimation of loss probability
in Gaussian queues~// {Conference (International) ``Modern
Probabilistic Methods for Analysis and Optimization of Information and
Telecommunication Networks'' Proceedings}, 2011. P.~142--147.

\bibitem{Luk7-1} %12
\Au{Lukashenko~O.\,V., Morozov E.\,V., Nekrasova~R.\,S., Pagano~M.}
Performance evaluation of finite buffer queues through regenerative simulation~//
Comm. Comp. Inform. Sci. BWWQT 2013, 2013.
Vol.~356. P.~131--139.


\bibitem{Zeevi-1} %13
\Au{Zeevi~A., Glynn~P.} On the maximum workload in a queue fed
by fractional Brownian motion~// {Ann. Appl. Probab}., 2000. Vol.~10. P.~1084--1099.

\bibitem{Husler1-1} %14
\Au{H$\ddot{\mbox{u}}$sler~J., Piterbarg~V.\,I.}
 Limit theorem for maximum
of the storage process with fractional Brownian as input~//
{Stochastic Proc. Their Appl.}, 2004. Vol.~114. P.~231--250.


\bibitem{Luk1-1} %15
\Au{Lukashenko~O.\,V., Morozov E.\,V.}  Asymptotics of the maximum
workload for a class of Gaussian queues~// Информатика и её применения, 2012.
Т.~6. Вып.~3. С.~81--89.

\bibitem{Luk3-1} %16
\Au{Lukashenko~O.~V., Morozov~E.~V.} On the maximum workload for
a class of Gaussian queues~// {Conference (International) ``Probability
Theory and Its Applications'' in Commemoration of the Centennial of
B.\,V.~Gnedenko}, 2012. P.~231--232.

\bibitem{Luk2-1} %17
\Au{Lukashenko~O.\,V., Morozov  E.\,V.}
On convergence in the $L_p$
space of the workload maximum for a class of Gaussian queueing
systems~// Информатика и её применения, 2013.
Т.~7. Вып.~1. С.~36--43.

\bibitem{Dieker1-1} %18
\Au{Dieker~A.\,B., Mandjes~M.}
On asymptotically efficient simulation of large deviation probabilities~//
{Adv. Appl. Probab.}, 2005. Vol.~37. P.~539--552.





\bibitem{Michele1-1} %19
\Au{Giordano~S., Gubinelli M., Pagano~M.}
Bridge Monte-Carlo: A~novel approach to rare events of Gaussian processes~//
{5th St.\ Petersburg Workshop on Simulation Proceedings}.
St.\ Petersburg, Russia, 2005. P.~281--286.

\bibitem{Dieker2-1} %20
\Au{Dieker,~A.\,B., Mandjes M.}  Fast simulation of overflow
probabilities in a queue with Gaussian input~// {ACM Trans. Model.
Comput. Simul.}, 2006. Vol.~16. No.\,2. P.~119--151.

\bibitem{Michele2-1} %21
\Au{Giordano~S., Gubinelli M., Pagano~M.}
Rare events of Gaussian processes:
A~performance comparison between Bridge Monte-Carlo and Importance Sampling~//
{Next Generation Teletraffic and Wired/Wireless Advanced Networking}.
St.\ Petersburg, Russia, 2007. P.~269--280.


\bibitem{Luk5-1} %22
\Au{Lukashenko~O.\,V., Morozov E.\,V., Pagano~M.}
Performance analysis of bridge Monte-Carlo estimator~//
Trans. KarRC RAS, 2012. Vol.~3. P.~54--60.


\bibitem{Takacs-1} %23
\Au{Takacs~L.}
{Combinatorial methods in the theory of stochastic processes}.~---
John Wiley\&Sons, 1967. 262~p.

\bibitem{Mandjes-1} %24
\Au{Mandjes~M.}  {Large deviations of Gaussian queues}.
Chichester: Wiley, 2007. 339~p.


\bibitem{Fidler1-1} %25
\Au{Rizk~A., Fidler M.}
Sample path bounds for long memory fbm traffic~//
{29th Conference on Information Communications, INFOCOM'10 Proceedings}.~---
Piscataway, NJ, USA: IEEE Press, 2010. P.~61--65.

\bibitem{Fidler2-1} %26
\Au{Rizk~A., Fidler M.}
Non-asymptotic end-to-end performance bounds
for networks with long range dependent fbm cross traffic~//
{Computer Networks}, 2012. Vol.~56. No.\,1. P.~127--141.

\bibitem{Luk0-1} %27
\Au{Lukashenko~O., Morozov E.,  Pagano~M.}
On the effective envelopes for fluid queues with Gaussian input~//
Comm. Comp. Inform. Sci. DCCN 2013,  2014.
Vol.~279. P.~178--189.

\bibitem{Adler-1} %28
\Au{Adler~R.\,J.} {An introduction to continuity, extrema, and
related topics for general Gaussian processes}.~--- Hayward, CA: Institute of
Mathematical Statistics, 1990. 160~p.


\bibitem{Debicki-1} %29
\Au{Debicki~K.} Gaussian processes~// {Encyclopedia of actuarial
sciences}, 2004. Vol.~2. P.~752--757.

\bibitem{Dembo-1} %30
\Au{Dembo~A., Zeitouni O.}
{Large deviations techniques and applications}. Springer, 1998. 396~p.



\bibitem{Seneta-1} %31
\Au{Seneta~E.}  {Regularly varying functions}.~--- Springer, 1985. 116~p.

\bibitem{MEV:Konstantopoulos-1} %32
\Au{Konstantopoulos~T., Zazanis M., De~Veciana~G.}
Conservation
laws and reflection mappings with application to multiclass mean
value analysis for stochastic fluid queues~// {Stochastic
Proc. Their Appl.}, 1996. Vol.~65. P.~39--146.

\bibitem{Kulkarni-1} %33
\Au{Kulkarni~V., Rolski T.}
Fluid model driven by an Ornstein--Uhlenbeck process~//
{Prob. Eng. Inform. Sci.}, 1994.
Vol.~8. P.~403--417.

\bibitem{Debicki1-1} %34
\Au{Debicki~K., Rolski T.}
A~Gaussian fluid model~// Queueing Syst.,  1995. Vol.~20. P.~433--452.

\bibitem{Anick-1} %35
\Au{Anick~D., Mitra D., Sondhi~M.\,M.}
Stochastic theory of a data handling system with multiple resources~//
Bell Syst. Techn.~J.,  1982. Vol.~61. P.~1871--1894.



\bibitem{Addie-1} %36
\Au{Addie~R., Mannersalo P., Norros I.} Most probable paths and performance
formulae for buffers with Gaussian input traffic~//
Eur. Trans. Telecommunications, 2002. Vol.~13. P.~183--196.


 \label{end\stat}

\end{thebibliography}
} }

\end{multicols}



\hfill{\small\textit{Поступила в редакцию 08.03.2014}}
%\renewcommand{\bibname}{\protect\rm Литература}
\renewcommand{\figurename}{\protect\bf Рис.}