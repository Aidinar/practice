
\def\stat{kudr}

\def\tit{БАЙЕСОВСКАЯ РЕКУРРЕНТНАЯ МОДЕЛЬ РОСТА НАДЕЖНОСТИ:
БЕТА-РАСПРЕДЕЛЕНИЕ ПАРАМЕТРОВ$^*$}

\def\titkol{Байесовская рекуррентная модель роста надежности: бета-распределение параметров}

\def\autkol{Ю.\,В.~Жаворонкова,  А.~А.\ Кудрявцев,  С.\,Я.~Шоргин}

\def\aut{Ю.\,В.~Жаворонкова$^1$,  А.~А.\ Кудрявцев$^2$,  С.\,Я.~Шоргин$^3$}

\titel{\tit}{\aut}{\autkol}{\titkol}

{\renewcommand{\thefootnote}{\fnsymbol{footnote}}
\footnotetext[1]{Работа выполнена при поддержке РФФИ (проекты 12-07-00109-а
и 12-07-00115-а).}}

\renewcommand{\thefootnote}{\arabic{footnote}}
\footnotetext[1]{ООО КМ Медиа, juliana-zh@yandex.ru}
\footnotetext[2]{Московский государственный университет им. М.\,В.~Ломоносова, факультет
вычислительной математики и кибернетики, nubigena@hotmail.com}
\footnotetext[3]{Институт проблем информатики Российской
академии наук, sshorgin@ipiran.ru}

\Abst{Одной из актуальных задач современной прикладной
математики является задача прогнозирования надежности сложных
модифицируемых информационных систем. Любая впервые созданная
сложная система, предназначенная для переработки или передачи
информационных потоков, как правило, не обладает требуемой
надежностью. Такие системы подвергаются модификациям в ходе
разработки, опытной эксплуатации и штатного функционирования. Целью
таких модификаций является увеличение надежности информационных
систем. В~связи с этим возникает необходимость формализации понятия
надежности модифицируемых информационных систем и разработки методов
и алгоритмов оценивания и прогнозирования различных надежностных
характеристик. Одним из подходов к определению надежности системы
является вычисление вероятности того, что на сигнал, поданный на
вход системы в определенный момент времени, система отреагирует
корректно. В~статье рассматривается экспоненциальная рекуррентная
модель роста надежности, в которой вероятность надежности системы
представляется как линейная комбинация параметров <<дефективности>>
и <<эффективности>> средства, исправляющего недостатки системы.
Предполагается, что исследователь не имеет точных сведений об
исследуемой системе, а лишь знаком с характеристиками класса, из
которого берется данная система. В~рамках байесовского подхода
предполагается, что показатели <<дефективности>> и <<эффективности>>
имеют бе\-та-рас\-пре\-де\-ле\-ние. Вычисляется средняя предельная надежность
сис\-те\-мы. Приводятся чис\-лен\-ные результаты для модельных примеров.}



\KW{модифицируемые информационные системы; теория надежности; байесовский подход;
бе\-та-рас\-пре\-де\-ле\-ние}

\DOI{10.14357/19922264140205}

\vskip 14pt plus 9pt minus 6pt

      \thispagestyle{headings}

      \begin{multicols}{2}

            \label{st\stat}


\section{Введение}

Задача прогнозирования надежности сложных модифицируемых
информационных систем была сформулирована в~[1], а в дальнейшем
более подробно рассмотрена в~[2].

В статье~[3] подробно описана байесовская рекуррентная модель роста надежности.
Ниже соответствующая постановка задачи будет сформулирована вкратце.

Любой впервые созданный более или менее сложный агрегат,
предназначенный для переработки или передачи информационных потоков,
например новая программная система для компьютера, новая
ин\-фор\-ма\-ци\-он\-но-вы\-чис\-ли\-тель\-ная сеть или новая
адми\-ни\-ст\-ра\-тив\-но-ин\-фор\-ма\-ци\-он\-ная сис\-те\-ма, как правило, не обладает
требуемой надежностью. Для единства терминологии впредь будет
говориться о сложных информационных сис\-те\-мах. Такие сис\-те\-мы
подвергаются периодическим изменениям (модификациям), целью которых
является увеличение надежности информационных сис\-тем.\linebreak В~связи с этим
возникает необходимость формализации понятия надежности
модифицируемых информационных сис\-тем и разработки методов и
алгорит\-мов оценивания и прогнозирования различных надежностных
характеристик.

В книге~[2]  приводятся общие соображения, являющиеся основой для
построения математических моделей, описывающих изменение надеж\-ности
модифицируемых информационных сис\-тем, а затем обсуждаются конкретные
аналитические модели роста (изменения) надежности мо\-ди\-фи\-ци\-ру\-емых
информационных сис\-тем.

К числу таких  моделей относятся рекуррентные модели роста
надежности. Они могут использоваться в случае, когда удобно иметь
дело непосредственно с параметром, интерпретируемым как надежность
системы.

Рассмотрим произвольную сис\-те\-му, на вход которой подаются
некоторые сигналы (например, команды оператора или внешние
воздействия). Реакция сис\-те\-мы на поданные сигналы может быть либо
правильной (корректной), либо неправильной (некорректной). В~каждый
момент времени~$t$ надежность сис\-те\-мы можно характеризовать
параметром $p(t)$~--- вероятностью того, что на сигнал, поданный на
вход системы в момент~$t$, сис\-те\-ма отреагирует правильно. По смыслу
такая характеристика надежности ближе всего к традиционно
используемому коэффициенту готовности. В~случайные моменты времени
$0\hm=Y_0\hm\le Y_1 \hm\le Y_2 \hm\le\ldots$ сис\-те\-ма подвергается (мгновенной)
модификации, в результате чего изменяется параметр $p(t)$. Следует
обратить внимание на то обстоятельство, что ниже рассматривается
непрерывное время, без привязки напрямую процесса модифицирования
системы к процессу ее тестирования. Предположим, что траектории
процесса $p(t)$ непрерывны справа и ку\-соч\-но-по\-сто\-ян\-ны, так что $p(t)
\hm= p(Y_j)$ при $Y_j \hm\le t\hm< Y_{j+1}$.

Задача прогнозирования  поведения процесса $p(t)$ чрезвычайно важна.
Например, в программировании параметр $p(t)$ можно рас\-смат\-ри\-вать как
надежность программного  обеспечения, в которое по ходу отладки в
моменты $0\hm=Y_0\hm\le Y_1 \hm\le Y_2 \le\ldots$ вносятся изменения для
исправления замеченных ошибок. Оценивание $p(t)$ и прогнозирование
поведения этого параметра здесь важно как для оценивания надежности
всего комплекса, со\-став\-ной \mbox{частью} которого является программное
обеспечение, так и для прогнозирования продолжитель\-ности от\-ладки.
{\looseness=1

}

Обозначим $p_j \hm= p(Y_j)$. Рассмотрим  поведение $p_j$ в зависимости
от изменения~$j$. Другими словами, будем изучать изменение
надежности сис\-те\-мы в зависимости от номера модификации. В~книге~[2]
рассматривается, в частности, следующая рекуррентная модель роста
надежности.  Пусть  $\{(\theta_j, \eta_j)\}$, $j\ge1$,~---
последовательность независимых одинаково распределенных двумерных
случайных векторов таких, что $0 \hm< \eta_1 \hm< 1$; $0 \hm< \theta_1\hm  < 1$
почти наверное.

Задав начальную надежность $p_0$, рассмотрим модель, определяемую рекуррентным
соотноше\-нием
$$
p_{j+1} = \eta_{j+1}p_j + \theta_{j+1}\left(1-p_j\right)\,.
$$
Эта модель названа дискретной  экспоненциальной моделью. В~такой
модели случайные величины~$\eta_j$ описывают возможное уменьшение
надежности из-за некачественных модификаций, в ходе которых вместо
исправления существующих дефектов в сис\-те\-му могут быть внесены
новые, в то время как величины~$\theta_j$ описывают повышение
надежности за счет исправления дефектов.

Обозначим $\lambda = 1 \hm- {\sf E}\theta_1$, $\mu \hm = {\sf E}\eta_1$.
В~[2] доказано, что при условии $\lambda\hm+\mu\hm\neq1$
$$
p=\lim\limits_{j\to\infty}{\sf E} p_j = \fr{\mu}{\lambda+\mu}\,.
$$

\section{Постановка задачи}

Изучение предельного значения  средней величины ${\sf E} p_j$
представляет значительный интерес, посколь\-ку эта величина
характеризует асимптотическое значение надежности системы в рамках
некоторой рекуррентной модели, задаваемой набором $\{(\theta_j,
\eta_j)\}$. Из результатов~[2] следует, что это асимптотическое
значение зависит только от средних значений величин $\{(\theta_j,
\eta_j)\}$, $j\hm\ge1$.

В~[3] исследовалась ситуация,  при которой рассматривается целый
набор однотипных сложных модифицируемых объектов (МО), каждый из
которых обслуживается собственной ремонтной бригадой (РБ).
Исследователю хотелось бы определить усредненное значение~$p$ по
всем МО. Для решения этой задачи  в указанной работе предложена так
называемая байесовская постановка. Предполагается, что
рассматривается целая группа однотипных МО и группа им
соответствующих однотипных РБ. Пусть $m\hm=1,2,\ldots$~--- номера этих
объектов. Для каждого МО (вместе с его РБ) существует собственный
набор $\{(\theta_j^m, \eta_j^m)\}$ ($j\hm\ge1$, $m\hm\ge1$) независимых
одинаково распределенных при каждом фиксированном~$j$ двумерных
случайных векторов таких, что $0 \hm< \eta_1^m \hm< 1$; $0 \hm< \theta_1^m \hm< 1$
почти наверное. Но средние значения величин~$\theta_j^m$,
$\eta_j^m$, $j\hm\ge1$, $m\hm\ge1$, не предполагаются известными; более
того, они не предполагаются даже одинаковыми. Вводится
предположение, что величины $\lambda \hm= 1 \hm- {\sf E}\theta_j^m$, $\mu\hm =
{\sf E}\eta_j^m$ сами по себе являются случайными, т.\,е.\ на
вероятностном пространстве, в которое в качестве элементарных
событий входят все рас\-смат\-ри\-ва\-емые в рамках данной постановки МО
вместе с их РБ, заданы случайные величины~$\lambda$ и~$\mu$ (которые
 полагаем независимыми), имеющие смысл $\lambda \hm= 1 \hm-
{\sf E}\theta_j^m$, $\mu \hm= {\sf E}\eta_j^m$, где $m$~--- случайный номер МО.

Принимаемые исследователем за основу распределения величин~$\lambda$
и~$\mu$ будем называть априорными.
%
При этом подлежащие  вычислению характеристики такой
<<рандомизированной>> группы МО, естественно, являются рандомизацией
аналогичных характеристик <<отдельно взятой>> МО с учетом априорного
распределения параметров~$\lambda$ и~$\mu$, взятого исследователем
за основу. Наиболее естественной и удобной для изучения
характеристикой является усредненное по всем МО значение предельной
вероятности надежности, т.\,е.\

\noindent
$$
p_{\mathrm{сред}} = {\sf E} p = {\sf E} \fr{\mu}{\lambda+\mu}\,,
$$
где усреднение ведется по совместному распредению случайных величин
$(\lambda,\mu)$.

В рассматриваемой  ситуации величины~$\eta_1^m$ и~$\theta_1^m$
удовлетворяют ограничениям $0 \hm< \eta_1^m \hm< 1$, $0 \hm< \theta_1^m  \hm<1$.
Значит, и средние значения~$\lambda$ и~$\mu$ величин
$1 \hm-{\sf E}\theta_j$ и ${\sf E}\eta_j$ также находятся на отрезке $[0,1]$.
Поэтому в качестве априорных распределений па\-ра\-мет\-ров~$\lambda$ и~$\mu$
следует выбирать только распределения, сосредоточенные на $[0,1]$.

В работе~[3] были  рассмотрены независимые случайные параметры~$\lambda$
и~$\mu$, распределенные равномерно на некоторых (вообще
говоря, разных) отрезках, являющихся подмножествами отрезка $[0,1]$.
В~настоящей статье исследования байесовской рекуррентной модели
роста надежности продолжены для ситуации, когда оба параметра имеют
бе\-та-рас\-пре\-де\-ление.

\section{Основные результаты}

Пусть $\lambda$ и $\mu$ имеют соответственно бе\-та-рас\-пре\-де\-ле\-ния
$\beta(m,n)$ и $\beta(k,l)$, $m,n,k,l\hm>0$.
Введем следующие обозначения.

Через $B(m,n)$, $m,n\hm>0$, будем обозначать бе\-та-функ\-цию. Пусть
$$
(\alpha)_i=\alpha \left(\alpha+1\right)\cdots \left(\alpha+i-1\right)\,,\ \ (\alpha)_0=1\,.
$$
Несмотря на то что~$(\alpha)_i$ имеет смысл неполного факториала,
нигде далее не требуется, чтобы~$\alpha$ было положительным.
Рассмотрим классическую гипергеометрическую функцию Гаусса
\begin{equation*}
G(\alpha,\beta,\gamma;x)=
\sum\limits_{i=0}^\infty \fr{(\alpha)_i(\beta)_i}{{(\gamma)_i \,i!}}\,x^i\,.
%\label{e1-kud}
\end{equation*}
По аналогии с~9.180.1 и~9.14 п.\,1 из~[4] введем в рассмотрение
обобщенную гипергеометрическую функцию двух переменных
\begin{multline*}
G^{p,q}_{s,t}(\alpha, \beta_1,\ldots,\beta_p,\beta_1',\ldots,\beta_q';\\
\gamma, \delta_1,\ldots,\delta_s,\delta_1',\ldots, \delta_t';x,y)={}\\
{}=
\sum\limits_{i=0}^\infty\sum\limits_{j=0}^\infty
\fr{(\alpha)_{i+j}(\beta_1)_i\cdots(\beta_p)_i(\beta_1')_j\cdots
(\beta_q')_j}{(\gamma)_{i+j}
(\delta_1)_i\cdots(\delta_s)_i(\delta_1')_j
\cdots(\delta_t')_j}\,\fr{x^iy^j}{{i!j!}}.
\end{multline*}
В дальнейшем изложении будет интересен частный случай последней функции
\begin{multline}
G^{2,1}_{1,0}(\alpha,\beta_1,\beta_2,\beta_1';\gamma,\delta_1;x,y)={}\\
{}=\sum\limits_{i=0}^\infty\sum\limits_{j=0}^\infty
\fr{(\alpha)_{i+j}(\beta_1)_i(\beta_2)_i(\beta_1')_j}{(\gamma)_{i+j}
(\delta_1)_i}\,\fr{x^iy^j}{{i!j!}}\,.\label{e2-kud}
\end{multline}

\noindent
\textbf{Теорема.}\
\textit{Пусть случайные величины~$\lambda$ и~$\mu$ независимы
и имеют соответственно распределения $\beta(m,n)$ и $\beta(k,l)$,
$m,n,k,l\hm>0$. Тогда}
\begin{multline}
p_{\mathrm{сред}} = \fr{B(k+m,n)}{B(k,l)B(m,n)(k+1)}\times{}\\
{}\times
 G^{2,1}_{1,0}
(k+1,1-l,k+m,1;k+2,k+m+n;-1,1)+{}\\
{}+\fr{B(k+m,l)}{B(k,l)B(m,n)m}
G^{2,1}_{1,0}\left(m,1-n,k+m,1;\right.\\
\left.m+1,k+m+l;-1,1\right).
\label{e3-kud}
\end{multline}


\smallskip

\noindent
Д\,о\,к\,а\,з\,а\,т\,е\,л\,ь\,с\,т\,в\,о\,.\ \
Найдем плотность $f_p(x)$ случайной величины~$p$. Имеем
$$
f_p(x)=\int\limits_0^1\fr{y}{(1-x)^2}\,f_\mu\left(
\fr{x}{1-x}\,y\right)f_\lambda(y)\,dy\,.
$$
Положим $f_p(1/2)\hm=0$. Согласно формуле~3.197.3 из~[4] получаем при $0\hm<x\hm<1/2$
\begin{multline*}
f_p(x)=\int\limits_0^1\fr{y}{(1-x)^2}\left(\fr{x}{1-x}\right)^{k-1}\times{}\\
{}\times
\fr{y^{k-1}}{B(k,l)}\left(1-\fr{x}{1-x}\,y\right)^{l-1}
\fr{y^{m-1}(1-y)^{n-1}}{B(m,n)}\,dy={}\\
{}=\fr{B(k+m,n)}{B(k,l)B(m,n)}\,\fr{x^{k-1}}{(1-x)^{k+1}}\times{}\\
{}\times
G\left(1-l,k+m,k+m+n;\fr{x}{1-x}\right)\equiv S_1(x)\,.
\end{multline*}
Аналогично при $1/2\hm<x\hm<1$, используя замену переменной $z\hm=xy/(1\hm-x)$,
получаем:
\begin{multline*}
f_p(x)=\fr{B(k+m,l)}{B(k,l)B(m,n)}\,\fr{(1-x)^{m-1}}{x^{m+1}}\times{}\\
{}\times
G\left(1-n,k+m,k+m+l;\fr{1-x}{x}\right)\equiv S_2(x)\,.
\end{multline*}

Таким образом,
\begin{equation}
{\sf E} p=\int \!\!xf_p(x)\,dx=\int\limits_0^{1/2}\!\!xS_1(x)dx+\int\limits_{1/2}^{1}\!\!
xS_2(x)\,dx.
\label{e4-kud}
\end{equation}

Вычислим отдельно первый интеграл из правой части~(\ref{e4-kud}). Имеем:

\noindent
\begin{multline*}
\int\limits_{0}^{1/2}xS_1(x)\,dx=
\fr{B(k+m,n)}{B(k,l)B(m,n)}\times{}\\
{}\times
\sum\limits_{i=0}^\infty \fr{(1-l)_i(k+m)_i}{(k+m+n)_i\,i!}
\int\limits_{0}^{1/2} \fr{x^{k+i}dx}{(1-x)^{k+i+1}}={}\\
{}=
\fr{B(k+m,n)}{B(k,l)B(m,n)}\times{}\\
{}\times \sum\limits_{i=0}^\infty
\fr{(1-l)_i(k+m)_i}{(k+m+n)_i\,i!}\int\limits_{0}^{1/2}
\fr{(1/2-t)^{k+i}dt}{(1/2+t) ^{k+i+1}}\,.
\end{multline*}
\begin{table*}[b]\small %tabl1
\vspace*{-12pt}
\begin{center}
\Caption{Частные значения средней надежности
(равномерный случай)}
\vspace*{2ex}

\tabcolsep=5pt
\begin{tabular}{|c|c|c|c|c|c|c|c|c|c|c|}
\hline
&\multicolumn{10}{c|}{$[a_\mu,b_\mu]$}\\
\cline{2-11}
\multicolumn{1}{|c|}{\raisebox{6pt}[0pt][0pt]{$[a_\lambda,b_\lambda]$}}&$[0,1/4]$&$[0,1/2]$&$[0,3/4]$&$[0,1]$&$[1/4,1/2]$&$[1/4,3/4]$&$[1/4,1]$&$[1/2,3/4]$&$[1/2,1]$&$[3/4,1]$\\
\hline
$[0,1/4]$&0,50&0,63&0,70&0,74&0,76&0,80&0,83&0,84&0,86&0,88\\
%\hline
$[0,1/2]$&0,37&0,50&0,58&0,63&0,63&0,68&0,72&0,73&0,76&0,79\\
%\hline
$[0,3/4]$&0,30&0,42&0,50&0,56&0,55&0,60&0,64&0,66&0,69&0,72\\
%\hline
$[0,1]$&0,25&0,37&0,44&0,50&0,48&0,54&0,58&0,60&0,63&0,67\\
%\hline
$[1/4,1/2]$&0,24&0,37&0,45&0,52&0,50&0,56&0,61&0,63&0,66&0,70\\
%\hline
$[1/4,3/4]$&0,20&0,32&0,40&0,46&0,44&0,50&0,55&0,56&0,60&0,64\\
%\hline
$[1/4,1]$&0,17&0,28&0,36&0,42&0,39&0,46&0,50&0,51&0,55&0,60\\
%\hline
$[1/2,3/4]$&0,16&0,27&0,34&0,40&0,37&0,44&0,49&0,50&0,54&0,58\\
%\hline
$[1/2,1]$&0,14&0,24&0,31&0,37&0,34&0,40&0,45&0,46&0,50&0,54\\
%\hline
$[3/4,1]$&0,12&0,21&0,28&0,33&0,30&0,36&0,40&0,42&0,46&0,50\\
\hline
\end{tabular}
\end{center}
\end{table*}
\begin{table*}\small %tabl2
\begin{center}
\Caption{Частные значения средней
надежности (случай бета-распределения)}
\vspace*{2ex}

 \begin{tabular}{|c|c|c|c|c|c|c|c|c|c|c|}
 \hline
 &\multicolumn{10}{c|}{$k; l$}\\
 \cline{2-11}
 \multicolumn{1}{|c|}{\raisebox{6pt}[0pt][0pt]{${m;n}$}}  & ${1; 7}$ & ${1; 3}$ & ${3; 5}$ & ${1; 1}$ & ${6; 10}$ & ${2; 2}$ & ${10; 6}$ & ${5; 3}$ & ${3; 1}$ & ${7; 1}$\\
 \hline
${1; 7}$&0,47&0,64&0,74&0,80&0,75&0,80&0,84&0,84&0,87&0,88\\
% \hline
${1; 3}$&0,35&0,50&0,59&0,66&0,60&0,66&0,71&0,72&0,76&0,79\\
 %\hline
${3; 5}$&0,22&0,35&0,44&0,51&0,44&0,52&0,59&0,59&0,65&0,70\\
 %\hline
${1; 1}$&0,23&0,35&0,43&0,49&0,43&0,49&0,53&0,53&0,56&0,60\\
 %\hline
\hphantom{9}${6; 10}$&0,21&0,33&0,42&0,50&0,42&0,50&0,58&0,58&0,65&0,70\\
 %\hline
${2; 2}$&0,20&0,31&0,39&0,45&0,40&0,45&0,50&0,50&0,55&0,60\\
 %\hline
${10; 6}$\hphantom{9}&0,15&0,26&0,32&0,35&0,33&0,35&0,37&0,37&0,43&0,52\\
 %\hline
${5; 3}$&0,15&0,26&0,32&0,36&0,33&0,37&0,40&0,39&0,44&0,51\\
 %\hline
${3; 1}$&0,14&0,24&0,31&0,35&0,30&0,36&0,38&0,38&0,38&0,40\\
 %\hline
${7; 1}$&0,12&0,21&0,28&0,33&0,26&0,34&0,35&0,35&0,33&0,29\\
 \hline
 \end{tabular}
 \end{center}
 \vspace*{-16pt}
 \end{table*}

\noindent
Воспользовавшись формулой~3.196.1 из~[4], получим:

\noindent
\begin{multline}
\int\limits_{0}^{1/2}xS_1(x)\,dx=
\fr{B(k+m,n)}{B(k,l)B(m,n)} \times{}\\
{}\times \sum\limits_{i=0}^\infty
\fr{(1-l)_i(k+m)_i}{(k+m+n)_i\,i!}\times{}\\
{}\times \fr{G(1,k+i+1,k+i+2;-1)}{k+i+1}={}\\
{}=\fr{B(k+m,n)}{B(k,l)B(m,n)}\times{}\\
{}\times \sum\limits_{i=0}^\infty
\fr{(1-l)_i(k+m)_i}{(k+m+n)_i\,i!}\sum\limits_{j=0}^\infty
\fr{(-1)^j}{k+i+j+1}\,.
\label{e5-kud}
\end{multline}

\noindent
Аналогично~(\ref{e5-kud}) получаем:
\begin{multline}
\int\limits_{1/2}^{1}xS_2(x)\,dx=
\fr{B(k+m,l)}{B(k,l)B(m,n)}\times{}\\
{}\times\sum\limits_{i=0}^\infty
\fr{(1-n)_i(k+m)_i}{(k+m+n)_i\,i!}\sum\limits_{j=0}^\infty
\fr{(-1)^j}{m+i+j}\,.\label{e6-kud}
\end{multline}
Заметим, что соотношения~(\ref{e5-kud}) и~(\ref{e6-kud})
можно преобразовать, что дает возможность получить из~(\ref{e4-kud})
для $p_{\mathrm{сред}}$ следующее выражение:

\noindent
\begin{multline*}
p_{\mathrm{сред}}=\fr{B(k+m,n)}{B(k,l)B(m,n)}\times{}\\
{}\times\sum\limits_{i=0}^\infty
\sum\limits_{j=0}^\infty \fr{(1-l)_i(k+m)_i(k+1)_{i+j}(1)_j(-1)^j1^i}
{(k+m+n)_i(k+2)_{i+j}(k+1)i!j!}+{}\\
{}+\fr{B(k+m,l)}{B(k,l)B(m,n)}\times{}\\
{}\times\sum\limits_{i=0}^\infty\sum\limits_{j=0}^\infty
\fr{(1-n)_i(k+m)_i(m)_{i+j}(1)_j(-1)^j1^i}{(k+m+l)_i(m+1)_{i+j}\,m\,i!j!}\,.
\end{multline*}
 Из последнего соотношения
по определению~(\ref{e2-kud}) получаем~(\ref{e3-kud}), что завершает доказательство
тео\-ремы.

\smallskip

\noindent
\textbf{Замечание.} Выражение~(\ref{e3-kud})  служит для компактной записи
$p_{\mathrm{сред}}$. Для практического использования
(непосредственного вычисления) имеет смысл представлять
$p_{\mathrm{сред}}$ в виде рядов типа~(\ref{e5-kud}) и~(\ref{e6-kud}), которые
несложно вычисляются с любой наперед заданной точностью.

\smallskip

В качестве иллюстрации  приведем несколько таблиц со значениями
$p_{\mathrm{сред}}$. Для удобства изложения будем использовать
индексы, соответствующие номерам таблицы, для случайных величин~$\lambda$ и~$\mu$.

Таблица~1 была  опубликована в~[3] для случая, когда~$\lambda_1$ и~$\mu_1$
имеют равномерное распределение на отрезках
$[a_\lambda,b_\lambda]$ и $[a_\mu,b_\mu]$ соответственно.



В табл.~2 приведены  значения $p_{\mathrm{сред}}$ для случая,
когда $\lambda_2$ и~$\mu_2$ имеют бе\-та-рас\-пре\-де\-ле\-ние с па\-ра\-мет\-ра\-ми
$(m;n)$ и $(k;l)$ соответственно, причем имеют место соотношения
${\sf E}\lambda_1\hm={\sf E}\lambda_2\hm=m/(m+n)$ и
${\sf E}\mu_1\hm={\sf E}\mu_2\hm=k/(k+l)$.



В табл.~3 также выполняются соотношения ${\sf E}\lambda_1\hm={\sf E}\lambda_3$ и
${\sf E}\mu_1\hm={\sf E}\mu_3$.

\begin{table*}\small %tabl3
\begin{center}
\Caption{Частные значения средней надежности
(случай бета-распределения)}
\vspace*{2ex}

\tabcolsep=5.5pt
 \begin{tabular}{|l|c|c|c|c|c|c|c|c|c|c|}
 \hline
 &\multicolumn{10}{c|}{${k; l}$}\\
 \cline{2-11}
 \multicolumn{1}{|c|}{\raisebox{6pt}[0pt][0pt]{${m;n}$}}  &
${1; 7}$ & ${1{,}1; 3{,}3}$ & ${1{,}2; 2{,}0}$ & ${1{,}3; 1{,}3}$ &
 ${1{,}4; 2{,}33}$ & ${1{,}5; 1{,}5}$ & ${1{,}6; 0{,}96}$ & ${1{,}7; 1{,}02}$ &
 ${1{,}8; 0{,}6}$ & ${1{,}9; 0{,}27}$\\
 \hline
 \hphantom{9,}${1; 7}$&0,47&0,61&0,70&0,76&0,71&0,77&0,81&0,82&0,85&0,88\\
% \hline
${1{,}1; 3{,}3}$&0,34&0,47&0,56&0,63&0,57&0,64&0,69&0,69&0,74&0,78\\
% \hline
${1{,}2; 2{,}0}$&0,26&0,38&0,46&0,53&0,47&0,53&0,59&0,59&0,64&0,69\\
% \hline
${1{,}3; 1{,}3}$&0,21&0,32&0,39&0,45&0,40&0,45&0,51&0,51&0,56&0,60\\
% \hline
${1{,}4; 2{,}33}$&0,25&0,37&0,45&0,52&0,46&0,53&0,59&0,59&0,64&0,69\\
% \hline
${1{,}5; 1{,}5}$&0,21&0,31&0,38&0,45&0,39&0,45&0,50&0,50&0,55&0,60\\
% \hline
${1{,}6; 0{,}96}$&0,17&0,26&0,33&0,38&0,34&0,39&0,43&0,43&0,47&0,51\\
% \hline
${1{,}7; 1{,}02}$&0,17&0,26&0,33&0,38&0,33&0,38&0,43&0,43&0,47&0,51\\
% \hline
${1{,}8; 0{,}6}$&0,14&0,23&0,29&0,33&0,30&0,34&0,37&0,37&0,38&0,39\\
% \hline
${1{,}9; 0{,}27}$&0,12&0,20&0,26&0,30&0,27&0,31&0,32&0,32&0,30&0,26\\
 \hline
 \end{tabular}
 \end{center}
 \vspace*{-16pt}
 \end{table*}

Для табл.~4 и~5 имеет место ${\sf E}\lambda_4\hm={\sf E}\lambda_5$ и
${\sf E}\mu_4\hm={\sf E}\mu_5$.

\begin{table*}\small %tabl4
\begin{center}
\Caption{Частные значения средней надежности
(случай бета-распределения)}
\vspace*{2ex}

\begin{tabular}{|c|c|c|c|c|c|c|c|c|c|c|}
 \hline
 &\multicolumn{10}{c|}{${k;l}$}\\
 \cline{2-11}
\multicolumn{1}{|c|}{\raisebox{6pt}[0pt][0pt]{ ${m;n}$ }} & ${1; 10}$ & ${2; 9}$ & ${3; 8}$ & ${4; 7}$ & ${5; 6}$ & ${6; 5}$ & ${7; 4}$ & ${8; 3}$ & ${9; 2}$ & ${10; 1}$\\
 \hline
\hphantom{9}${1; 10}$&0,47&0,64&0,74&0,79&0,83&0,86&0,88&0,89&0,91&0,91\\
 %\hline
${2; 9}$&0,30&0,46&0,56&0,64&0,70&0,74&0,78&0,80&0,82&0,84\\
% \hline
${3; 8}$&0,23&0,36&0,45&0,52&0,58&0,64&0,69&0,73&0,75&0,78\\
% \hline
${4; 7}$&0,18&0,30&0,37&0,43&0,49&0,55&0,60&0,65&0,69&0,72\\
% \hline
${5; 6}$&0,16&0,26&0,33&0,38&0,42&0,46&0,52&0,57&0,62&0,66\\
% \hline
${6; 5}$&0,13&0,23&0,30&0,34&0,37&0,40&0,44&0,49&0,55&0,60\\
% \hline
${7; 4}$&0,12&0,21&0,28&0,32&0,35&0,36&0,38&0,42&0,47&0,53\\
% \hline
${8; 3}$&0,11&0,19&0,26&0,31&0,33&0,35&0,35&0,36&0,39&0,44\\
% \hline
${9; 2}$&0,09&0,17&0,24&0,29&0,33&0,34&0,34&0,33&0,32&0,34\\
% \hline
${10; 1}$\hphantom{9}&0,09&0,16&0,22&0,28&0,32&0,35&0,35&0,34&0,29&0,25\\
 \hline
 \end{tabular}
 \end{center}
 \vspace*{-16pt}
 \end{table*}

\begin{table*}\small %tabl5
\begin{center}
\Caption{Частные значения средней надежности
(случай бета-распределения)}
\vspace*{2ex}

\tabcolsep=5pt
\begin{tabular}{|c|c|c|c|c|c|c|c|c|c|c|}
 \hline
 &\multicolumn{10}{c|}{$k;l$}\\
 \cline{2-11}
 \multicolumn{1}{|c|}{\raisebox{6pt}[0pt][0pt]{$m;n$ }}&
  ${0{,}1; 1{,}0}$ & ${0{,}2; 0{,}9}$ & ${0{,}3; 0{,}8}$ &
  ${0{,}4; 0{,}7}$ & ${0{,}5; 0{,}6}$ & ${0{,}6; 0{,}5}$ &
  ${0{,}7; 0{,}4}$ & ${0{,}8; 0{,}3}$ & ${0{,}9; 0{,}2}$ & ${1{,}0; 0{,}1}$\\
 \hline
${0{,}1; 1{,}0}$&0,50&0,56&0,63&0,69&0,74&0,79&0,83&0,86&0,88&0,90\\
% \hline
${0{,}2; 0{,}9}$&0,44&0,49&0,55&0,60&0,65&0,69&0,74&0,77&0,80&0,83\\
% \hline
${0{,}3; 0{,}8}$&0,39&0,43&0,48&0,53&0,58&0,62&0,66&0,70&0,74&0,77\\
% \hline
${0{,}4; 0{,}7}$&0,35&0,39&0,43&0,48&0,52&0,56&0,60&0,64&0,67&0,71\\
% \hline
${0{,}5; 0{,}6}$&0,30&0,34&0,38&0,42&0,46&0,50&0,54&0,57&0,61&0,65\\
% \hline
${0{,}6; 0{,}5}$&0,26&0,30&0,34&0,38&0,41&0,44&0,47&0,51&0,54&0,57\\
% \hline
${0{,}7; 0{,}4}$&0,22&0,26&0,30&0,33&0,37&0,39&0,41&0,44&0,46&0,48\\
% \hline
${0{,}8; 0{,}3}$&0,18&0,22&0,26&0,29&0,32&0,34&0,36&0,37&0,38&0,38\\
% \hline
${0{,}9; 0{,}2}$&0,14&0,18&0,22&0,25&0,28&0,29&0,30&0,30&0,29&0,28\\
% \hline
${1{,}0; 0{,}1}$&0,10&0,14&0,18&0,22&0,24&0,25&0,25&0,24&0,21&0,18\\
 \hline
 \end{tabular}
 \end{center}
 \end{table*}

\section{Заключение}

Полученные результаты могут применяться, например, для вычисления
других моментов и построения доверительных интервалов для характеристики~$p$.

В дальнейшем предполагается  рассмотреть другие модели, в частности
ситуации, когда один из параметров $(\lambda, \mu)$ распределен
равномерно, а другой имеет бе\-та-рас\-пре\-де\-ле\-ние. Предполагается
разработать расчетные алгоритмы для вычисления величины
$p_{\mathrm{сред}}$, соответствующие программные модели и
провести тестовые расчеты.

{\small\frenchspacing
{%\baselineskip=10.8pt
\addcontentsline{toc}{section}{References}
\begin{thebibliography}{9}

\bibitem{1-kud} %GK}
\Au{Gnedenko B.\,V., Korolev V.\,Yu.}
Random summation: Limit theorems and applications.~--- Boca Raton, FL:
CRC Press, 1996. 288~p.

\bibitem{2-kud} %\bibitem{KS}
\Au{Королев В.\,Ю., Соколов И.\,А.}
Основы математической теории надежности модифицируемых сис\-тем.~---
М.: ИПИ РАН, 2006. 108~с.

\bibitem{3-kud} %\bibitem{KuSoSh}
\Au{Кудрявцев А.\,А., Соколов~И.\,А., Шоргин~С.\,Я.}
Байесовская рекуррентная модель роста надежности: равномерное распределение
параметров~// Информатика и её применения, 2013. Т.~7. Вып.~2. С.~55--59.

\bibitem{4-kud} %\bibitem{GR71}
\Au{Градштейн И.\,С., Рыжик~И.\,М.}
Таблицы интегралов, сумм, рядов и произведений.~--- М.: Наука, 1971. 1108~с.

\end{thebibliography}
} }

\end{multicols}

%\vspace*{-6pt}

\hfill{\small\textit{Поступила в редакцию 19.11.13}}

%\newpage


\vspace*{16pt}

\hrule

\vspace*{2pt}

\hrule


\def\tit{BAYESIAN RECURRENT MODEL OF~RELIABILITY GROWTH: BETA-DISTRIBUTION OF PARAMETERS}

\def\titkol{Bayesian recurrent model of reliability growth: Beta-distribution of parameters}

\def\aut{Iu.\,V.~Zhavoronkova$^1$, A.\,A.~Kudryavtsev$^2$,
and~S.\,Ya.~Shorgin$^3$}
\def\autkol{Iu.\,V.~Zhavoronkova, A.\,A.~Kudryavtsev,
and~S.\,Ya.~Shorgin}


\titel{\tit}{\aut}{\autkol}{\titkol}

\vspace*{-9pt}

\noindent
$^1$KM Media Company, 8/2 Prishvina Str., Moscow 127549,
Russian Federation

\noindent
$^2$Department of Mathematical Statistics, Faculty of Computational Mathematics and
Cybernetics,\\
$\hphantom{^1}$M.\,V.~Lomonosov Moscow State University,
1-52 Leninskiye Gory, GSP-1, Moscow 119991, Russian\\
$\hphantom{^1}$Federation

\noindent
$^3$Institute of Informatics Problems, Russian Academy of Sciences,
44-2 Vavilov Str., Moscow 119333, Russian\\
$\hphantom{^1}$Federation




\def\leftfootline{\small{\textbf{\thepage}
\hfill INFORMATIKA I EE PRIMENENIYA~--- INFORMATICS AND APPLICATIONS\ \ \ 2014\ \ \ volume~8\ \ \ issue\ 2}
}%
 \def\rightfootline{\small{INFORMATIKA I EE PRIMENENIYA~--- INFORMATICS AND APPLICATIONS\ \ \ 2014\ \ \ volume~8\ \ \ issue\ 2
\hfill \textbf{\thepage}}}

\vspace*{12pt}

\Abste{One of the topical problems of modern applied mathematics is the task
of forecasting reliability of modifiable complex information systems.
Any first established complex system designed for processing or transmission
information flows, as a rule, does not possess the required reliability.
Such systems are subject to modifications during development, testing, and regular
functioning. The purpose of such modifications is to increase reliability of
information systems. In this connection, it is necessary to formalize
the concept of reliability of modifiable information systems and to develop
methods and algorithms of estimating and forecasting various reliability
characteristics. One approach to determine system reliability is to compute
the probability that a signal fed to the input of a system at a given point of
time will be processed correctly by the system. The article considers the exponential
recurrent growth model of reliability, in which the probability of system reliability
is represented as a linear combination of the ``defectiveness'' and ``efficiency''
parameters of tools correcting deficiencies in a system. It is assumed that
the researcher does not have exact information about the system under study and
is only familiar with the characteristics of the class from which this system is
taken. In the framework of the Bayesian approach, it is assumed that the indicators
of ``defectiveness'' and ``efficiency'' have beta-distribution. Average marginal
system reliability is calculated. Numerical results for model examples are
obtained.}

\KWE{modifiable information systems; theory of reliability; Bayesian approach;
beta-distribution}

\DOI{10.14357/19922264140205}

\Ack
\noindent
The work was supported by the Russian Foundation for Basic Research
(projects 12-07-00109-а and 12-07-00115-а).

\pagebreak


  \begin{multicols}{2}

\renewcommand{\bibname}{\protect\rmfamily References}
%\renewcommand{\bibname}{\large\protect\rm References}

{\small\frenchspacing
{%\baselineskip=10.8pt
\addcontentsline{toc}{section}{References}
\begin{thebibliography}{9}

\bibitem{1-kud-1}
\Aue{Gnedenko, B.\,V., and V.\,Yu.~Korolev}.
1996. \textit{Random summation: Limit theorems and applications}.
Boca Raton, FL: CRC Press, 1996. 288~p.
\bibitem{2-kud-1}
\Aue{Korolev, V.\,Yu., and I.\,A.~Sokolov}.
2006. \textit{Osnovy matemati\-che\-skoy teorii nadezhnosti modifitsiruemykh system}
[Fundamentals of mathematical theory of modified systems reliability].
Moscow.: IPI RAN, 2006. 108~p.
\bibitem{3-kud-1}
\Aue{Kudryavtsev, A.\,A., I.\,A. Sokolov, and S.\,Ya.~Shorgin}.
2013. Bayesovskaya rekurrentnaya model' rosta nadezhnosti:
Ravnomernoe raspredelenie parametrov
[Bayesian recurrent model of reliability growth:
Uniform distribution of parameters].
\textit{Informatika i ee Primeneniya}~--- \textit{Inform. Appl.} 7(2):55--59.
\bibitem{4-kud-1}
\Aue{Gradshteyn, I.\,S., and I.\,M.~Ryzhik}.
1971. \textit{Tablitsy integralov, summ, ryadov i proizvedeniy}
[Tables of integrals, sums, series, and products]. Moscow: Nauka, 1971. 1108~p.

\end{thebibliography}
} }


\end{multicols}

\vspace*{-6pt}

\hfill{\small\textit{Received November 19, 2013}}

\vspace*{-18pt}


\Contr

\noindent
\textbf{Zhavoronkova Iuliia V.} (b.\ 1990)~---
Software Developer, KM Media Company, 8/2 Prishvina Str., Moscow 127549,
Russian Federation; juliana-zh@yandex.ru

\vspace*{3pt}

\noindent
\textbf{Kudryavtsev Alexey A.} (b.\ 1978)~---
Candidate of Science (PhD) in physics and mathematics, associate professor,
Department of Mathematical Statistics, Faculty of Computational Mathematics and
Cybernetics, M.\,V.~Lomonosov Moscow State University,
1-52 Leninskiye Gory, GSP-1, Moscow 119991, Russian Federation;\linebreak
 nubigena@hotmail.com

\vspace*{3pt}

\noindent
\textbf{Shorgin Sergey Ya.} (b.\ 1952)~---
Doctor of Science in physics and mathematics, professor, Deputy Director,
Institute of Informatics Problems, Russian Academy of Sciences,
44-2 Vavilov Str., Moscow 119333, Russian Federation;  sshorgin@ipiran.ru

 \label{end\stat}

\renewcommand{\bibname}{\protect\rm Литература}

