\def\stat{matskevich}

\def\tit{ДЕКЛАРАТИВНЫЕ СТРУКТУРЫ ЗНАНИЙ
В~ПРОБЛЕМНО-ОРИЕНТИРОВАННЫХ СИСТЕМАХ
ИСКУССТВЕННОГО ИНТЕЛЛЕКТА}

\def\titkol{Декларативные структуры знаний в проблемно-ориентированных
системах искусственного интеллекта}

\def\aut{А.\,Г.~Мацкевич$^1$}

\def\autkol{А.\,Г.~Мацкевич}

\titel{\tit}{\aut}{\autkol}{\titkol}

\renewcommand{\thefootnote}{\arabic{footnote}}
\footnotetext[1]{Институт проблем информатики Российской академии наук;
Московский технический университет связи и информатики (МТУСИ),
xmag@mail.ru}

\vspace*{-12pt}

\Abst{Описаны методы и средства представления знаний в
виде декларативных структур на основе расширенных семантических сетей
(РСС) для формирования баз знаний (БЗ) сис\-тем искусственного интеллекта, а
также приведены примеры реализованных интеллектуальных систем
обработки знаний для различных предметных областей. Для обработки
декларативных структур знаний, представленных в виде РСС, разработан
специализированный язык логического программирования ДЕКЛ. На языке
ДЕКЛ были реализованы лингвистические процессоры (ЛП), осуществляющие
перевод предложений естественного языка (ЕЯ) (русского и английского) в
структуры БЗ, а также обратный перевод из внутренних
(глубинных) представлений в поверхностные формы русского или
английского языка.}

\KW{интеллектуальные системы; представление знаний; обработка
естественного языка; семантические сети; логическое программирование}

\DOI{10.14357/19922264140212}


\vskip 10pt plus 9pt minus 6pt

\thispagestyle{headings}

\begin{multicols}{2}

\label{st\stat}


\section{Введение} %1

 Стремительно развивается тенденция к глобализации информации: она
проявляется практически во всех сферах жизни общества. В~то же время
тексты на ЕЯ, в том числе в оцифрованном виде, являются
основным способом хранения и передачи знаний. Во многих случаях человек
не в силах прочитать и осмыслить даже малую часть того, что ему
предлагается. В~настоящее время решением этой глобальной проблемы
занимаются, в частности, системы искусственного интеллекта. Основными
компонентами таких систем являются БЗ, предназначенные для
хранения информации в форме, удобной для ее обработки, а также средства,
необходимые для преобразования текстов на ЕЯ в такую
форму. Кроме этого, в состав таких систем входят компоненты, с помощью
которых обеспечивается эффективный поиск и анализ информации и решаются
другие насущные задачи.

 В ИПИ РАН накоплен большой опыт разработки интеллектуальных
систем обработки текстовых знаний. В~коллективе, научным руководителем
которого на протяжении многих лет являлся доктор технических наук,
профессор Игорь Петрович Кузнецов, создана линия систем искусственного
интеллекта, построенных на аппарате представления знаний в виде
РСС. Еще в 1980-е~гг.\ в своих
монографиях~\cite{ma-l1, ma-l2} и в докторской диссертации И.\,П.~Кузнецов
предложил использовать РСС для описания декларативных знаний.

 Семантическая сеть в этом варианте состоит из множества вершин,
представляющих объекты. Из вершин составляются элементарные фрагменты (ЭФ),
каждый из которых представляет $k$-мест\-ное отношение. Во фрагмент
вводятся две дополнительные вершины: одна соответствует отношению, а
другая~--- всей совокупности упомянутых объектов с учетом их отношения.
Эти вершины, как и любые другие вершины, могут стоять на местах объектов в
других фрагментах, что обеспечивает высокие изоб\-ра\-зи\-тель\-ные возможности и
гибкость: представление отношений между отношениями, между
совокупностями связанных объектов и т.\,п. Из таких фрагментов и
составляются сети, названные расширенными семантическими сетями.
Как показали исследования~[1--3], подобные сети оказываются удобными для
представления различных языковых конструкций, в которых отглагольные
существительные, представляющие определенное действие, сами могут
связываться в рамках глагольных форм. Такие сети могут служить основой для
решения многих ло\-ги\-ко-ана\-ли\-ти\-че\-ских задач.

\vspace*{-6pt}

\section{Языки представления и~обработки знаний} %2

 Для обработки декларативных структур знаний, представленных в виде
РСС, разработан специализированный язык логического программирования
ДЕКЛ. Он обеспечивает представление декларатив-\linebreak\vspace*{-12pt}
\noindent
\begin{center}  %fig1
\mbox{%
\epsfxsize=76.56mm
\epsfbox{mac-1.eps}
}
  \vspace*{5pt}

{{\figurename~1}\ \ \small{Обычная семантическая сеть из вершин и дуги}}
  \end{center}

\vspace*{8pt}



\begin{center}
\mbox{%
\epsfxsize=72.092mm
\epsfbox{mac-2.eps}
}
\end{center}

\noindent
{{\figurename~2}\ \ \small{Расширенная семантическая сеть
из вер\-шин-объ\-ек\-тов, вер\-ши\-ны-от\-но\-ше\-ния и
вер\-ши\-ны-комп\-лек\-сно\-го объ\-ек\-та}}

\addtocounter{figure}{2}

\vspace*{8pt}

\noindent
ных структур знаний в виде
РСС и имеет средства их обработки~\cite{ ma-l4, ma-l5}.

 Обычные семантические сети состоят из вершин (они соответствуют
объектам) и связывающих их дуг, которые соответствуют отношениям.
Например, <<\textit{Иван отец Петра}>> представляется в виде графа,
представленного на рис.~1.


 В РСС используются не слова ЕЯ, например русского
ОТЕЦ, ИВАН, ПЁТР, а конкретные объекты базы знаний~--- ИВАН1, ИВАН2,
\ldots, ПЕТР1, соответствующие референтам, т.\,е.\ людям с именами Иван,
Петр. База знаний может содержать много подобных объектов.

 Такое различие необходимо для более точного представления
информации. Далее, вместо дуги используется специальная вершина связи и~ЭФ.


 На рис.~2 представлено, что \textit{ИВАН1 отец ПЕТР1}.
Цифры 1, 2 возле стрелок указывают, что ИВАН1~--- это первый объект
отношений, а ПЕТР1~--- 2-й. Если поменять их местами, то уже будет
\textit{ПЕТР1 отец ИВАН1}. При этом вер\-ши\-на-от\-но\-ше\-ние (ОТЕЦ)
выделена как самостоятельная. Она тоже может быть связана отношением.
Далее, выделена \textit{комплексная вершина}~G1, которая соответствует
объектам ИВАН1, ПЕТР1 с их отношением. Все это образует \textit{часть
семьи}~--- более сложный объект, которому сопоставлена вершина~G1. На
языке ДЕКЛ такой ЭФ записывается в виде:
 \begin{center}
ОТЕЦ(ИВАН1,ПЕТР1/G1).
 \end{center}
 Это предикатная форма записи ЭФ. Если вер\-ши\-на-от\-но\-ше\-ние (ОТЕЦ) и
комплексная вершина (G1)\linebreak\vspace*{-12pt}

\noindent
\begin{center}  %fig3
\mbox{%
\epsfxsize=76.56mm
\epsfbox{mac-3.eps}
}
  \end{center}
  \vspace*{2pt}

\noindent
{{\figurename~3}\ \ \small{Упрощенная РСС из вер\-шин-объ\-ек\-тов и вер\-ши\-ны-свя\-зи}}


\vspace*{6pt}



\begin{center}
\mbox{%
\epsfxsize=65.372mm
\epsfbox{mac-4.eps}
}


{{\figurename~4}\ \ \small{Упрощенная РСС с вер\-ши\-ной-пе\-ре\-мен\-ной}}
\end{center}

\addtocounter{figure}{2}

\vspace*{6pt}


\noindent
не связаны с какими-либо другими вершинами (не
входят в другие ЭФ в качестве вер\-шин-объ\-ек\-тов), то будем использовать
более простую запись ЭФ (рис.~3).



 Запись сети рис.~3 в предикатном виде: ОТЕЦ(ИВАН1,ПЕТР1).

 В РСС различают 2 типа вершин.
 \begin{enumerate}
\item Вершины, соответствующие \textit{определенным объектам},
отношениям, классам объектов (ИВАН1, ОТЕЦ, \ldots).
\item Вершины, соответствующие \textit{неопределенным объектам} (X1,
X2, \ldots, Xn). Они называются {\bfseries\textit{х-вер\-ши\-на\-ми}} или
{\bfseries\textit{вер\-ши\-на\-ми-пе\-ре\-мен\-ны\-ми}}. Их обозначения
выносятся за рамки вершин, указывая таким образом, что они не означены.
 \end{enumerate}


 Сеть рис.~4 представляет: \textit{ИВАН1 отец неизвестно
кому}~--- ОТЕЦ(ИВАН1, X1)


 С помощью сети рис.~5 представлено, что \textit{ИВАН1 и
ПЕТР1 как-то связаны между собой}, но неизвестно, каким отношением~---
X2(ИВАН1,ПЕТР1).



 В дальнейшем вводится понятие <<отношение в широком смысле>> и
допускается множество объектов отношений (более двух). В~этом случае ЭФ в
общем виде будет выглядеть, как на рис.~6.
%
 Здесь представлено N-\textit{арное отношение} R1 \textit{между объектами}
А1, А2, \ldots, АN, которые образуют комплексный объект G2. Такой
ЭФ записывается в предикатном виде как
 $\mathrm{R1(A1,A2, \ldots ,AN/G2)}.
$

\noindent
\begin{center}
\mbox{%
\epsfxsize=46.06mm
\epsfbox{mac-5.eps}
}

\end{center}

\noindent
{{\figurename~5}\ \ \small{Расширенная семантическая сеть с вер\-ши\-ной-пе\-ре\-мен\-ной}}


\addtocounter{figure}{1}

%\vspace*{8pt}

\noindent
\begin{center}
\mbox{%
\epsfxsize=67.656mm
\epsfbox{mac-6.eps}
}

\end{center}

\noindent
{{\figurename~6}\ \ \small{Расширенная семантическая сеть с N-ар\-ным отношением~R1
между объектами A1, \ldots,
AN, образу\-ющи\-ми комплексный объект~G2}}


\addtocounter{figure}{1}

\vspace*{18pt}


 Посредством РСС вводятся представления отношений, описываются
родовидовые деревья и другие конструкции для комплексных
объектов. В~РСС используется принцип наследования свойств: свойства
каждой вершины более высокого уровня справедливы для всех ее вершин более
низкого уровня. Это позволяет значительно сократить объем знаний, используя
свойства (отношения) классов объектов и автоматически перенося их на
конкретные объекты.



 Такое разнообразие <<изобразительных средств>> делает РСС очень
гибким и практически универсальным средством описания декларативных
знаний.

 Для обработки структур знаний, представленных в виде РСС, разработана
инструментальная среда ДЕКЛ, включающая в себя язык ДЕКЛ, а также
собственную базу данных~--- для хранения РСС и текстов. Последние версии
этой среды написаны на языке Object Pascal в среде программирования Delphi.
Язык ДЕКЛ основан на применении правил ЕСЛИ$\ldots$ТО$\ldots$, которые
содержат правую и левую часть. Для левой части ищутся сопоставимые
структуры в БЗ. Если в БЗ оказалась сеть (РСС), сопоставимая с левой частью
правила, тогда правило делается применимым и выполняются действия,
записанные в правой части этого правила. В~частности, могут быть вызваны
другие правила.

 В языке есть специальные конструкции, обеспечивающие интерфейсы
пользователя, например работу с файлами, построение окон, выдачу на экран,
обработчики событий (например, нажатия клавиш, действия с мышкой) и~др.

 Далее приведен простой пример представления декларативных знаний в
виде БЗ и программы, написанной на языке логического
программирования ДЕКЛ, выдающей студентов группы ГР\_1.

\columnbreak

{\small
\noindent
 \{== {\bfseries\textit{База знаний}} ==\}

\noindent
 СТУДЕНТ(ГР\_1,STUD\_1) ФАМ(IVANOV,STUD\_1)

 ИМЯ(IVAN,STUD\_1)

\noindent
 СТУДЕНТ(ГР\_1,STUD\_2) ФАМ(PETROV,STUD\_2)

 ИМЯ(PETER,STUD\_2)

\noindent
 СТУДЕНТ(ГР\_1,STUD\_3) ФАМ(SIDOROV,STUD\_3)

 ИМЯ(IVAN,STUD\_3)

\noindent
 СТУДЕНТ(ГР\_1,STUD\_4) ФАМ(SMIRNOV,STUD\_4)

 ИМЯ(PETER,STUD\_4)

\noindent
 СТУДЕНТ(ГР\_1,STUD\_5) ФАМ(IVANOV,STUD\_5)

 ИМЯ(ANDREY,STUD\_5)

\noindent
 СТУДЕНТ(ГР\_2,STUD\_6) ФАМ(PETROV,STUD\_6)

 ИМЯ(OLEG,STUD\_6)

\noindent
 \{== {\bfseries\textit{Программа}} ==\}

\noindent
 BEG(/91+) \{= начало программы -- код 91+ =\}

\noindent
 \{= с продукции START начинается применение программы =\}

\noindent
 START:IF THEN

\noindent
 \{B:PAR(1,9) = Включение трассировки.

  B:PAR(1,0) -- выключение =\}

\noindent
 T!:STUD\_OUT(ГР\_1)

\noindent
 B:IN()\ \{= ДЕКЛ встает и ожидает нажатия клавиши =\}

\noindent
 B:HALT(); \{= Выход из ДЕКЛ =\}

\noindent
 \{== Поиск по группе X1 ее студентов X2

  с выдачей ФИО ==\}

\noindent
 STUD\_OUT(X1):IF СТУДЕНТ(X1,X2) ФАМ(X10,X2)

  ИМЯ(X20,X2) THEN

\noindent
 B:BK() \{= Переход на новую строку =\}

\noindent
 B:A(<< Студент гр.\ >>,X1) \{= Выдача на экран =\}

\noindent
 B:A(<<: >>,X10,<<\ >>,X20);

\noindent
 END(/92+) \{= Конец программы -- код 92+ =\}

\noindent
 @BL(91-,92-) \{= указывает на зону, к которой не должны

  применяться продукции =\}

 }

 Если сохранить БЗ и программу в файле TEST\_1.Z и написать в
командной строке ДЕКЛ\_WIN.EXE TEST\_1.Z, то на экран будет выдано
следующее:

 Студент гр.\ ГР\_1: IVANOV IVAN

 Студент гр.\ ГР\_1: PETROV PETER

 Студент гр.\ ГР\_1: SIDOROV IVAN

 Студент гр.\ ГР\_1: SMIRNOV PETER

 Студент гр.\ ГР\_1: IVANOV ANDREY


\vspace*{-6pt}

\section{Лингвистические процессоры на основе расширенных
семантических сетей} %3

 На языке ДЕКЛ были реализованы средства, строящие РСС по текстам
ЕЯ~[6--10], назы\-ва\-емые лингвистическими процессорами.
С~использованием ЛП созданы такие
объ\-ект\-но-ори\-ен\-ти\-ро\-ван\-ные системы, как <<Криминал>>,
<<Аналитик>>, <<LINGVO-MASTER>>, <<Антитеррор>>.

 На рис.~\ref{ma-f7} представлена схема, которую можно считать общей
для этих систем.

\begin{figure*} %fig7
\vspace*{1pt}
\begin{center}
\mbox{%
\epsfxsize=148.465mm
\epsfbox{mac-7.eps}
}
\end{center}
\vspace*{-9pt}
\Caption{Схема функционирования объ\-ект\-но-ори\-ен\-ти\-ро\-ван\-ных
систем, использующих ЛП
\label{ma-f7}
}
\end{figure*}


 В данной схеме с помощью ЛП накапливаются
\textit{предметные знания}~(ПЗ), которые определяют ответы,
вырабатываемые системой на тот или иной запрос. Сам
ЛП настраивается на работу с входными текстами с помощью
\textit{лингвистических знаний}~(ЛЗ). Так как преобразование текстов и их
обработка имеют много общих задач, то перспективным является
использование единого инструментария: представление предметных и
лингвистических знаний с помощью РСС и их обработка программами на языке
ДЕКЛ. Таким образом формируются декларативные знания в виде базы
предметных и базы лингвистических знаний.

 Более того, ЛП может быть использован для
поддержания режима ответа на запросы, выраженные на
ЕЯ. Запросы представляются в виде РСС, и поиск ответа идет на уровне
обработки структур знаний. На это и ориентирован язык ДЕКЛ.


\vspace*{-6pt}

\section{Реализованные логико-аналитические системы} %4

 Уникальной по своим возможностям системой, использующей РСС и
написанной на языке ДЕКЛ, является ло\-ги\-ко-ана\-ли\-ти\-че\-ская система
<<Криминал>>~\cite{ma-l7, ma-l8}. Эта система нашла свое применение в
аналитических отделах ГУВД г.~Москвы и МВД.

 Система <<Криминал>> базируется на документах, поступающих из
различных источников. Это сводки, объяснительные, служебные записки служб
органов внутренних дел; записные книжки фигурантов; отчеты, документы
общего назначения; газетные публикации; словесные портреты фигурантов
и~т.\,д. Эти документы обрабатываются ЛП в
автоматическом режиме, по каждому из них строится РСС. Далее следует
постлингвистическая обработка. Она заключается в логическом анализе и
выделении наиболее значимых характеристик документа с точки зрения
сотрудников правоохранительных органов: орудий преступления, способа его
совершения, способа проникновения и~др. Осуществляется дополнение
документа атрибутами в соответствии с принятыми классификаторами. Вся
выделенная информация образует РСС, которая называется содержательным
портретом документа, где представлены значимые элементы текста и их связи.
Такие портреты (РСС) образуют БЗ, которая является основой для
ло\-ги\-ко-ана\-ли\-ти\-че\-ской обработки.

 На основе содержательных портретов строятся предметные каталоги. Это
списки фигурантов, адресов и других объектов (которые были выявлены из
документов), упорядоченные по алфавиту. Такие списки делают поиск
направленным. Пользователь может выбрать из них любой объект для
последующего анализа.

 На основе декларативных знаний, полученных в ходе анализа
документов, в системе <<Криминал>> реализованы следующие задачи
ло\-ги\-ко-ана\-ли\-ти\-че\-ской обработки:
 \begin{itemize}
 \item поиск похожих происшествий и фигурантов по информации,
извлеченной автоматически из имеющихся источников;
 \item контекстный поиск документов;
 \item поиск фигурантов по словесному портрету;
 \item поиск информации по запросам на естественном (русском) языке;
 \item объяснение результатов поиска;
 \item анализ и отображение связей между фигурантами;
 \item оценка степени причастности фигуранта к происшествию;
 \item упорядочение фигурантов по степени их криминальной и
преступной активности;
 \item выявление организованных преступных формирований (на основе
связей фигурантов);
 \item статистическая обработка информации, выдача усредненных и
оценочных данных, характеризующих динамику изменения криминогенных
процессов во времени.
 \end{itemize}

Приведем типовой пример документа и его содержательного портрета (РСС).

 \medskip

 \noindent
 {\small \prg \textit{1.05.2013г.\ в 7.10 Иванова Ивана Ивановича 1953г.р прож.\ ул. Ивановская 25-1-273,
работает АОЗТ <<ХДУ>>, зам.\ директора, о том, что 1-05-98г.\ неизвестные от д.22
кор.3 по ул.Сенная, похитили а/м ГАЗ 31029, черная, 1995 г/в, дв.402-0019476, кузов
0285927 \ldots}}

 \medskip

 Его содержательный портрет имеет следующий вид:

 \medskip

{\small

\noindent
 ДОК\_(221,$'$TEXT\_98.TXT$'$,$'$S\_CRI.NL$'$)

\noindent
 ДАТА\_(\#1.5.2013,2013,МАЙ,$\sim$1,7.1/4+) 4--(221,ДАТА\_)

\noindent
 FIO(ИВАНОВ,ИВАН,ИВАНОВИЧ,1953/5+)

  5--(221,FIO)

\noindent
 АДР\_(УЛ.,ИВАНОВСКАЯ,25,1,273/6+) 6--(221,АДР\_)

 \noindent
 ПРОЖ.(5--,6--/7+)

 \noindent
 ОРГ\_(АОЗТ,ХДУ/8+) 8--(221,ОРГ\_)

 \noindent
 РАБ\_(5--,8--,ЗАМ.,ДИРЕКТОР/9+) 10--(221,РАБ\_)

 \noindent
 FIO($"$\ $"$,$"$\ $"$,$"$\ $"$,НЕСКОЛЬКО/10+) 10--(221,FIO)

 \noindent
 НЕИЗВЕСТНЫЙ(10--)

 \noindent
 АВТО\_(АВТОМАШИНА,ГАЗ,31029,ЧЕРНЫЙ,

 1995,Г/В,ДВ.,402,19476,
КУЗОВ,285927,УЧЕТ/11+) 11--(221,АВТО\_)

 \noindent
 УГНАТЬ(10--,11--/12+)

 \noindent
 ДАТА\_(\#1.5.2013,2013,МАЙ,$\sim$1/14+) 4--(221,ДАТА\_)

 \noindent
 КОГДА(12--,14--)

 \noindent
 АДР\_(УЛ.,СЕННАЯ,ДОМ,22,КОРП.,3/15+)

  15--(221,АДР\_) ГДЕ(12--,15--)

 \noindent
 ПРЕДЛ\_(221,4--,5--,6--,8--,9--,О,ТОМ,12--,14--,15--)

 }

 \medskip

 Первый фрагмент ДОК\_(221,$'$TEXT\_98.TXT$'$,\linebreak $'$S\_CRI.NL$'$)
 указывает, что
содержательный портрет построен на основе документа 221 из файла
$'$TEXT\_98.TXT$'$. При этом были использованы лингвистические знания
$'$S\_CRI.NL$'$. Второй фрагмент представляет дату. Здесь используются
внут\-ри\-сис\-тем\-ные коды, которые соответствуют определенным вершинам РСС
без мнемонических обозначений. Причем 4+ значит, что создается новая
вершина без имени, а 4-- значит, что вершина без имени используется. Эти
фрагменты необходимы для быстрого поиска нужных фрагментов, когда в
оперативной памяти (БЗ) находится множество содержательных портретов.

 Далее следует фрагмент FIO(ИВАНОВ,ИВАН, ИВАНОВИЧ,1953/5+),
представляющий фигуранта, затем следует фрагмент
АДР\_(УЛ.,\linebreak ИВАНОВСКАЯ, 25,1,273/6+), соответствующий выделенному
адресу, а за ним фрагмент ПРОЖ.\linebreak(5--,6--/7+), в котором с использованием
внутренних кодов записана информация о том, что 5--, т.\,е.\ ИВАНОВ ИВАН
ИВАНОВИЧ, проживает по адресу~6, т.\,е.\ УЛ.,ИВАНОВСКАЯ,25,1,273.
Этой выделенной информации соответствует вершина 7+.

 Фрагменты ПРЕДЛ\_ представляют предложения с аргументами: кодами
фрагментов, которые представляют объекты и действия, и словами, которые
никуда не вошли. За счет фрагментов ПРЕДЛ\_ текст может быть восстановлен
по содержательному портрету документа.

 Содержательные портреты запоминаются в БЗ\linebreak системы. На основе этой
БЗ и построенных предметных каталогов реализованы различные виды
семанти\-ческого поиска: поиск по признакам и связям, поиск связанных
объектов на различных уровнях, поиск похожих фигурантов и происшествий,
поиск по приметам.

 Таким образом, ло\-ги\-ко-ана\-ли\-ти\-че\-ская система <<Криминал>>
стала важным этапом в развитии использования механизмов РСС для
представления декларативных структур знаний.

 Система LINGVO-MASTER обеспечивает автоматическую
формализацию различного рода справок и сообщений (автобиографических
данных, заявок на работу, резюме), представляющих собой тексты на
ЕЯ в достаточно произвольной форме~[10]. Результатами
формализации являются выделенные значимые компоненты (информационные
объекты) и их отображение в поля некоторой базы данных и соответствующей
формы (анкеты) с фиксированными полями. Тогда становится возможным
использование типовых средств для решения пользовательских задач, например
поиска работы по запросу пользователя или поиска специалиста,
удовлетворяющего требованиям работодателя. Общая схема такой системы
отличается от схемы рис.~\ref{ma-f7}, хотя имеет много общих частей.
Методики извлечения семантической информации, разработанные в системе
<<Криминал>>, используются и здесь. Лингвистический процессор системы
LINGVO-MASTER в своей основе имеет тот же ЛП.
Так же строится содержательный портрет документа. Конечно, в БЗ
(см.\ рис.~\ref{ma-f7}) изменяются ЛЗ, а~знания о
предметной области~(ПЗ), представленные в виде РСС, являются исходными
данными для обратного ЛП. Обратный ЛП служит
для преобразования содержательных портретов (РСС) в компоненты ЕЯ и для
их отоб\-ра\-же\-ния в поля формы или сайта. Этот процессор имеет свои
ЛЗ, с помощью которых задается последовательность
выдачи руб\-рик (полей) и то, какими объектами они должны заполняться. Для
выделения таких объектов служат их имена, а также связи, заданные в РСС.
Для каждого выделенного объекта строится его описание~--- из входящих в
него нормализованных слов. Далее ищется сегмент предложения,
со\-от\-вет\-ст\-ву\-ющий объекту. Этот сегмент и выдается в качестве результата.

 Таким образом, в системе LINGVO-MASTER, во-пер\-вых, обработка
идет на уровне структур знаний (РСС) с использованием созданного для этого
инструментария (языка ДЕКЛ). Отсюда возможность вовлечения в процесс
анализа семантических категорий и различного рода связей. И, во-вто\-рых,
основные процессоры в данной системе сделаны как оболочки, которые легко
подстраивать под предметную область и особенности текстов за счет
лингвистических и экспертных знаний. Это очень важно, когда требуется
обработка реальных текстов. На стадии проектирования удается учесть лишь
малую часть их особенностей. Дальнейшее совершенствование (и качество
системы) определяется удобством и возможностями средств подстройки.

 Система отлаживалась на материалах кадрового портала HeadHunter и
обеспечивала следующие результаты: коэффициент шумов в компонентах
(лишних слов в объектах)~--- не более 1\%--2\% и потерь (отсутствие нужных
слов)~--- не более 3\%.

\vspace*{-6pt}

\section{Заключение} %5

 Тот факт, что основные процессоры в реализованных системах сделаны
как оболочки, которые легко подстраивать под предметную область и
особенности текстов за счет лингвистических и экспертных знаний, открывает
большие возможности использования РСС и языка ДЕКЛ для создания
проб\-лем\-но-ори\-ен\-ти\-ро\-ван\-ных систем искусственного интеллекта. Благодаря
этому достоинству была создана англоязычная версия системы обработки
автобиографических данных, а также система <<Антитеррор>>,
ориентированная на выявление информации о террористической деятельности
из материалов СМИ~\cite{ma-l11, ma-l12}.

\vspace*{-6pt}


{\small\frenchspacing
 {%\baselineskip=10.8pt
 \addcontentsline{toc}{section}{References}
 \begin{thebibliography}{99}

\bibitem{ma-l1}
\Au{Кузнецов~И.\,П.}
Механизмы обработки семантической информации.~--- М.: Наука, 1978. 175~с.
\bibitem{ma-l2}
\Au{Кузнецов~И.\,П.}
 Семантические представления.~--- М.: Наука, 1986. 296~с.
\bibitem{ma-l3}
\Au{Кузнецов~И.\,П.} Расширяющиеся системы активного диалога.~--- М.:
Наука, 1982. 309~с.
\bibitem{ma-l4}
\Au{Кузнецов~И.\,П. Пузанов~В.\,В., Шарнин~М.\,М.}
Сис\-те\-ма обработки декларативных структур знаний \mbox{ДЕКЛАР-2}.~--- М.:
ИПИАН, 1989. 106~с.
\bibitem{ma-l5}
\Au{Кузнецов~И.\,П. Шарнин~М.\,М.}
Интеллектуальный редактор знаний на основе расширенных семантических
сетей~// Системы и средства информатики, 1993.  Вып.~5.
С.~14--21.

\bibitem{ma-l7} %6
\Au{Кузнецов~И.\,П.}
 Методы обработки сводок с выявлением особенностей фигурантов и
происшествий~// Диалог-98: Труды Междунар. семинара по
компьютерной лингвистике и ее приложениям.~--- Казань: Хэтер, 1998.  Т.~2.
С.~691--700.

\bibitem{ma-l6} %7
\Au{Кузнецов~И.\,П., Козеренко~Е.\,Б., Шарнин~М.\,М.}
Се\-ман\-ти\-ко-ори\-ен\-ти\-ро\-ван\-ная сис\-те\-ма фактографического поиска со
входом на русском и английском языках~// Диалог-98: Труды Междунар.
семинара по компьютерной лингвистике и ее приложениям.~--- Казань:
Хэтер, 1998.  Т.~2. С.~821--830.

\bibitem{ma-l8}
\Au{Кузнецов~И.\,П., Мацкевич~А.\,Г.}
Лингвистический процессор для автоматического выявления из текстов
значимой информации с ее компоновкой в рамках указанных шаблонов~//
Диалог 2001: Труды Междунар. семинара по компьютерной лингвистике
и ее приложениям.~--- М.: Наука, 2001.  Т.~2. С.~134--137.
\bibitem{ma-l9}
\Au{Kuznetsov~I., Kozerenko~E.}
The system for extracting semantic information from natural language texts~//
MLMTA-03:  Conference (International) on Machine Learning Proceedings.~---
Las Vegas: CSREA, 2003. P.~75--80.
\bibitem{ma-l10}
\Au{Кузнецов~И.\,П., Мацкевич~А.\,Г.}
Се\-ман\-ти\-ко-ори\-ен\-тированный лингвистический процессор для
авто\-матической формализации автобиографических \mbox{данных}~// Компьютерная
лингвистика и интеллектуальные технологии: Труды Междунар.
конференции <<Диалог'2006>>.~--- М.: РГГУ, 2006. С.~317--322.
\bibitem{ma-l11}
\Au{Кузнецов~И.\,П., Мацкевич~А.\,Г.}
Англоязычная версия системы автоматического выявления значимой
информации из текстов естественного языка~// Компьютерная лингвистика и
интеллектуальные технологии: Труды Междунар. конф.
<<Диалог'2005>>.~--- М.: Наука, 2005. С.~303--311.
\bibitem{ma-l12}
\Au{Кузнецов~И.\,П., Козеренко~Е.\,Б., Мацкевич~А.\,Г.}
Принципы организации объ\-ект\-но-ори\-ен\-ти\-ро\-ван\-ных сис\-тем
обработки неформализованной информации~// Искусственный интеллект:
Журнал НАН Украины, 2010. Вып.~3. С.~227--237.

\end{thebibliography}

 }
 }

 \end{multicols}

 \vspace*{-6pt}

 \hfill{\small\textit{Поступила в редакцию 7.05.14}}

 \newpage

% \vspace*{12pt}

% \hrule

% \vspace*{2pt}

% \hrule

 \def\tit{DECLARATIVE KNOWLEDGE STRUCTURES
IN~PROBLEM-ORIENTED SYSTEMS OF~ARTIFICIAL~INTELLIGENCE}

 \def\titkol{Declarative knowledge structures in problem-oriented systems
of artificial intelligence}

 \def\aut{A.\,G.~Matskevich$^{1,2}$}

 \def\autkol{A.\,G.~Matskevich}

 \titel{\tit}{\aut}{\autkol}{\titkol}

 \vspace*{-10pt}

 \noindent
$^1$Institute of Informatics Problems, Russian Academy of Sciences,
 44-2 Vavilov Str., Moscow 119333, Russian\\
 $\hphantom{^1}$Federation

 \noindent
$^2$Moscow Technical University of Communications and Informatics
(MTUCI),  8a Aviamotornaya Str., Moscow
 $\hphantom{^1}$111024, Russian Federation

 \def\leftfootline{\small{\textbf{\thepage}
 \hfill INFORMATIKA I EE PRIMENENIYA~--- INFORMATICS AND
 APPLICATIONS\ \ \ 2014\ \ \ volume~8\ \ \ issue\ 2}
 }%
 \def\rightfootline{\small{INFORMATIKA I EE PRIMENENIYA~---
 INFORMATICS AND APPLICATIONS\ \ \ 2014\ \ \ volume~8\ \ \ issue\ 2
 \hfill \textbf{\thepage}}}

 \vspace*{3pt}

 \Abste{The article describes the techniques and tools of knowledge
representation in the form of declarative structures based on the Extended Semantic
Networks (ESN) formalism to produce intelligent systems knowledge bases and it
provides the examples of intellectual knowledge processing systems for different
domains. For processing the declarative knowledge structures presented in the form of
ESN, a specialized  logical programming language DECL has been developed. In
DECL, the linguistic processors have been implemented executing
translation of natural language (Russian and English) sentences into structures of
a knowledge base, as well as reverse translation of inner representations into the
surface forms of Russian or English.}

 \KWE{intelligent systems; knowledge representation; natural language
processing; semantic networks; logical programming}


 \DOI{10.14357/19922264140212}

 \vspace*{-6pt}

 \begin{multicols}{2}

 \renewcommand{\bibname}{\protect\rmfamily References}
 %\renewcommand{\bibname}{\large\protect\rm References}

 {\small\frenchspacing
 {%\baselineskip=10.8pt
 \addcontentsline{toc}{section}{References}
 \begin{thebibliography}{99}
 \bibitem{mae-l1}
\Aue{Kuznetsov,~I.\,P.}
 1978. \textit{Mekhanizmy obrabotki semanticheskoy informatsii} [Semantic information
processing mechanisms]. Moscow: Nauka. 175~p.
 \bibitem{mae-l2}
\Aue{Kuznetsov,~I.\,P.}
 1986. \textit{Semanticheskie predstavleniya} [Semantic representations]. Moscow:
Nauka. 296~p.
 \bibitem{mae-l3}
\Aue{Kuznetsov,~I.\,P.}
 1982. \textit{Rasshiryayushchiesya sistemy aktivnogo dialoga} [Extending systems of
active dialogue]. Moscow: Nauka. 309~p.
 \bibitem{mae-l4}
\Aue{Kuznetsov,~I.\,P., V.\,V.~Puzanov, and M.\,M.~Sharnin.}
 1989. \textit{Sistema obrabotki deklarativnykh struktur znaniy
 \mbox{DEKLAR-2}} [The system
of declarative knowledge structures processing]. Moscow: IPIAN. 106~p.
 \bibitem{mae-l5}
\Aue{Kuznetsov,~I.\,P., and M.\,M.~Sharnin.}
 1993. Intellektual'nyy redaktor znaniy na osnove rasshirennykh semanticheskikh
setey [Intelligent knowledge editor based on the extended semantic networks].
\textit{Sistemy i Sredstva Informatiki}~--- \textit{Systems and Means of Informatics}
5:14--21.

 \bibitem{mae-l7} %6
\Aue{Kuznetsov,~I.\,P.}
 1998. Metody obrabotki svodok s vyyavleniem osobennostey figurantov i
proisshestviy [The methods of reports processing with the identification of
criminal acts and participants characteristics].
\textit{Dialog-98: Trudy Mezhdunar. Seminara po Komp'yuternoy Lingvistike i
ee Prilozheniyam} [Dialogue-98:  Seminar (International) in Computational Linguistics and Its
Applications Proceedings]. Kazan': Kheter. 2:691--700.

 \bibitem{mae-l6} %7
\Aue{Kuznetsov,~I.\,P., E.\,B.~Kozerenko, and M.\,M.~Sharnin.}
 1998. Semantiko-orientirovannaya sistema fakto\-gra\-fi\-che\-sko\-go poiska so vkhodom
na russkom i angliyskom yazykakh [The semantic-oriented system of factual search
with the input in  Russian and in English]. \textit{Dialog-98: Trudy
Mezhdunar. Seminara po Komp'yuternoy Lingvistike i ee Prilozheniyam}
[Dialogue-98:   Seminar (International) in Computational
Linguistics and Its Applications Proceedings].
Kazan': Kheter. 2:821--830.

 \bibitem{mae-l8}
\Aue{Kuznetsov,~I.\,P., and A.\,G.~Matskevich.}
 2001. Lingvisticheskiy protsessor dlya avtomaticheskogo vyyavleniya iz tekstov
znachimoy informatsii s ee komponovkoy v ramkakh ukazannykh shablonov [The
linguistic processor for automatic identification of the meaningful information
from texts and its arrangement within the framework of language templates].
\textit{Dialog 2001: Trudy Mezhdunar. Seminara po Komp'yuternoy Lingvistike i ee
Prilozheniyam} [Dialogue-2001:  Seminar (International) in
Computational Linguistics and Its Applications Proceedings].
 Moscow: Nauka.
2: 134--137.
 \bibitem{mae-l9}
\Aue{Kuznetsov,~I., and E.~Kozerenko.}
 2003. The system for extracting semantic information from natural language texts.
\textit{MLMTA-03:  Conference (International) on Machine Learning Proceedings}. Las
Vegas: CSREA. 75--80.
 \bibitem{mae-l10}
\Aue{Kuznetsov,~I.\,P., and A.\,G.~Matskevich.}
 2006. Semantiko-orientirovannyy lingvisticheskiy protsessor dlya avtomaticheskoy
formalizatsii avtobiograficheskikh dannykh [The semantic-oriented linguistic
processor for automatic formalization of autobiogrphy data].
\textit{Komp'yuternaya
Lingvistika i Intellektual'nye Tekhnologii: Trudy Mezhdunar. Konf.
``Dialog'2006''} [Computational Linguistics and Intelligent Technologies:
 Conference (International) ``Dialogue'2006'' Proceedings]. Moscow: RGGU.
317--322.
 \bibitem{mae-l11}
\Aue{Kuznetsov,~I.\,P., and A.\,G.~Matskevich.}
 2005. Anglo\-yazych\-naya versiya sistemy avtomaticheskogo vyyavleniya\linebreak\vspace*{-12pt}

 \pagebreak

 \noindent
znachimoy informatsii iz tekstov estestvennogo yazyka [The English
version of the system for automatic extraction of the meaningful information from
natural language texts].
\textit{Komp'yuternaya Lingvistika i Intellektual'nye Tekhnologii:
Trudy Mezhdunar. Konf. ``Dialog'2005''} [Computational Linguistics and
Intelligent Technologies:   Conference (International)
``Dialogue'2005'' Proceedings]. Moscow: Nauka. 303--311.
 \bibitem{mae-l12}
\Aue{Kuznetsov,~I.\,P., E.\,B.~Kozerenko, and A.\,G.~Matskevich.}
 2010. Printsipy organizatsii ob''ektno-orientirovannykh sistem obrabotki
neformalizovannoy informatsii [The principles of organization of the
object-oriented systems of unstructured information processing].
\textit{Iskusstvennyy Intellekt: Zhurnal NAN Ukrainy}
[Artificial Intelligence: The Ukraine National Academy of
Sciences~J.] 3:227--237.

 \end{thebibliography}
 } }


 \end{multicols}

 \vspace*{-6pt}

 \hfill{\small\textit{Received May 7, 2014}}

 \vspace*{-18pt}

 \Contrl

 \noindent
\textbf{Matskevich Andrey G.}~(b.~1953)~---
 senior scientist, Institute of Informatics Problems, Russian Academy of Sciences,
 44-2 Vavilov Str., Moscow 119333, Russian Federation;
 associate professor, Moscow Technical University of Communications and Informatics
(MTUCI),  8a Aviamotornaya Str., Moscow 111024, Russian Federation;  xmag@mail.ru




 \label{end\stat}

 \renewcommand{\bibname}{\protect\rm Литература}