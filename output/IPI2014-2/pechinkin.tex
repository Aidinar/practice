%\def\o{\omega}

\def\stat{pechinkin}

\def\tit{СИСТЕМА Geo/Geo/1/$R$ С ГИСТЕРЕЗИСНОЙ
ПОЛИТИКОЙ$^*$}

\def\titkol{Система Geo/Geo/1/$R$ с гистерезисной
политикой}

\def\autkol{А.\,В.~Печинкин, Р.\,В.~Разумчик}

\def\aut{А.\,В.~Печинкин$^{1}$, Р.\,В.~Разумчик$^2$}

\titel{\tit}{\aut}{\autkol}{\titkol}

{\renewcommand{\thefootnote}{\fnsymbol{footnote}} \footnotetext[1]{Работа
выполнена при поддержке РФФИ (проекты
№№\,12-07-00108, 13-07-00223 и 13-07-00665).}}

\renewcommand{\thefootnote}{\arabic{footnote}}
\footnotetext[1]{Институт проблем информатики Российской академии наук;
Российский университет дружбы народов, apechinkin@ipiran.ru}
\footnotetext[2]{Институт проблем информатики Российской академии наук, rrazumchik@ieee.org}

\Abst{Пороговое управление нагрузкой
является одним из основных
средств предотвращения перегрузок в сетях связи.
Его разновидности применяются при обнаружении
перегрузок как в сетях общеканальной сигнализации №\,7, так и в сетях
связи следующего поколения,
где основой сигнализации служит протокол~установления сессий (SIP~--- session initiation
protocol).
В работе рассматривается функционирующая
в дискретном времени система массового обслуживания (СМО)
 Geo/Geo/1/$R$ с двухпороговой гистерезисной политикой, представляющая
собой одну из возможных математических моделей SIP-сер\-ве\-ра с управлением нагрузкой.
Предложены методы нахождения совместного стационарного
распределения числа заявок в системе и состояния системы, распределения
времен выхода системы из множества состояний нормальной
работы и множества состояний перегрузки и блокировки и
их моментов, стационарного распределения времени ожидания начала
обслуживания заявки. Приведены примеры численных расчетов, проведенных с
помощью полученных аналитических соотношений.}

\KW{система массового обслуживания;
дискретное время;
гистерезисное управление нагрузкой;
показатели функционирования системы}

\DOI{10.14357/19922264140202}

\vskip 14pt plus 9pt minus 6pt

      \thispagestyle{headings}

      \begin{multicols}{2}

            \label{st\stat}



\section{Введение и~описание системы}


Исследованию СМО с гистерезисным
управлением посвящено много работ. Достаточно полный обзор
результатов по гистерезисному управлению можно найти в работах~[1--9].
Как видно из этих работ, большинство исследований относится
к изучению СМО, функционирующих в непрерывном времени.

Однако, как давно было замечено, дискретные СМО
позволяют учитывать дискретность функционирования
телекоммуникационных систем и дискретный характер
передаваемой информации.
В~насто\-ящей работе представлен подробный анализ
стационарных вероятностных и временн$\acute{\mbox{ы}}$х характеристик
сис\-те\-мы Geo/Geo/1 конечной емкости с двухпороговым
гистерезисным управлением.
Некоторые результаты исследования этой модели приведены в~\cite{38-p}.
Однако предлагаемый здесь подход, а также используемые
методы являются несколько отличными от представленных
в~\cite{38-p} и позволяют проводить более глубокий анализ
рассматриваемой системы.


Рассмотрим функционирующую в дискретном времени
однолинейную СМО с накопителем ограниченной емкости,
входящим геометрическим потоком заявок (вероятность
поступления заявки на такте ${a\hm=a_1\hm+a_2}$) и
геометрическим распределением времени обслуживания (вероятность
окончания обслуживания заявки на любом такте~$b$).
Прос\-тей\-ший вариант гистерезисной политики, который
здесь рассматривается, заключается в сле\-ду\-ющем.
Пусть имеется три числа: $R$ (максимальное чис\-ло
находящихся в системе заявок, или емкость системы),
$L$ (нижняя граница) и $H$ (средняя граница), причем ${L \hm< H \hm< R}$.
Каждая поступающая в систему заявка может быть одного из двух типов.
С~вероятностью~$a_1$ поступает заявка первого типа,
а с вероятностью~$a_2$~--- второго типа, причем поступление заявки одного типа исключает
поступление заявки другого типа.
Пусть в начальный момент система свободна.
Тогда до того такта, после которого в системе
станет~$H$~заявок, в нее принимаются все заявки.
Но как только число заявок достигнет~$H$, прекращается
прием заявок второго типа.
Далее заявки второго типа не принимаются до того
такта, после которого число заявок в системе станет
равным или ${L\hm-1}$, или~$R$.
В~первом случае в систему снова начинают
приниматься заявки обоих типов.
Во втором случае прекращается прием заявок первого
типа, прибор занят только обслуживанием заявок, и
так происходит до того такта, после которого число
заявок станет~$H$, после чего возобновляется прием
заявок первого типа.

\begin{figure*} %fig1
\vspace*{1pt}
\begin{center}
\mbox{%
\epsfxsize=114.281mm
\epsfbox{pec-1.eps}
}
\end{center}
\vspace*{-9pt}
\Caption{Схема двухпорогового гистерезисного
         управления перегрузками}
%\label{fig7}
\end{figure*}

Для определенности будем считать, что если на
некотором такте одновременно оканчивается
обслуживание заявки на приборе и в систему
поступает новая заявка, то число заявок в системе
не изменяется.
Кроме того, если к этому такту, кроме заявки на
приборе, в системе не было других заявок, то
вновь поступившая заявка сразу же начинает
обслуживаться.


Описанная процедура обслуживания, носящая название
гистерезисной политики, графически изображена на
рис.~1, где на оси абсцисс отложено число заявок в системе.
Уровень~1,2 соответствует состояниям,
когда в систему принимаются заявки обоих типов,
уровень~1~--- принимаются заявки только первого типа,
уровень~0~--- в систему не принимаются заявки обоих
типов. Возможные переходы отмечены стрелками.



Далее с целью сокращения записи для любой вероятности
дополнительную вероятность будем снабжать чертой сверху.
Например, $\overline{a}\hm=1\hm-a$, $\overline{b}\hm=1\hm-b$.

\section{Цепь Маркова}

Введем цепь Маркова $\{\nu(t)$,  $t\hm\ge0\}$, описывающую функционирование системы.
Множество состояний цепи Маркова $\nu(t)$ имеет вид:
$$
{\cal X} = {\cal X}_1 \cup {\cal X}_{20} \cup {\cal X}_{21} \cup {\cal X}_{31} \cup
{\cal X}_{32}\,.
$$
Здесь подмножество~${{\cal X}_{1}}$ включает
состояния $(n)$, ${n\hm=\overline{0,L-1}}$,
причем пребывание в состоянии $(n)$ означает, что
в системе находится $n$ заявок и принимаются заявки обоих типов.
Подмножество ${\cal X}_{20}$ содержит состояния
$(n,0)$, ${n\hm=\overline{L,H-1}}$,
где пребывание в состоянии $(n,0)$ означает, что
в системе находится $n$~заявок и принимаются заявки обоих типов.
Подмножество ${\cal X}_{21}$ состоит из состояний
$(n,1)$, ${n=\overline{L,H-1}}$,
и при этом пребывание в состоянии $(n,1)$ означает, что
в системе находится $n$~заявок и принимаются заявки только первого типа.
Подмножество ${\cal X}_{31}$ включает состояния
$(n,1)$, ${n\hm=\ov{H,R-1}}$,
причем пребывание в состоянии $(n,1)$ означает, что
в системе находится $n$~заявок и принимаются заявки только первого типа.
Наконец, подмножество ${\cal X}_{32}$ содержит состояния
$(n,2)$, ${n\hm=\ov{H+1,R}}$, а пребывание в
состоянии $(n,2)$ означает, что
в системе находится $n$~заявок и не принимаются заявки любого типа.

Будем считать, что значение цепи Маркова $\nu(t)$
образуют состояния системы непосредственно после такта~$t$.

\section{Вспомогательные функции}

Определим вспомогательные функции, которые
понадобятся в дальнейшем.

Пусть в начальный момент цепь Маркова $\nu(t)$
находится в состоянии
$(n,1) \in {\cal X}_{31}$,
$n\hm=\overline{H+1,R-1}$
(т.\,е.\ в системе имеется $n$~заявок, причем в
систему принимаются заявки только первого типа).
Обозначим через~$c_n$ вероятность того, что $\nu(t)$
в  состояние $(n-1,1)$ попадет раньше, чем в
состояние $(R,2)$ (т.\,е.\ до того момента, когда в
сис\-те\-ме впервые останется ${(n-1)}$ заявок, в системе никогда не будет $R$~заявок).

В состояние $(R-2,1)$ из состояния $(R-1,1)$ с
вероятностью $\overline{a}_1 b$ можно попасть на первом же такте.
Кроме того, из $(R-1,1)$ в $(R-2,1)$ раньше, чем в $(R,2)$,
можно попасть также с вероятностью
${a_1 b \hm+ \overline{a}_1 \overline{b}}$, оставшись на первом такте в
состоянии $(R-1,1)$, а затем уже с вероятностью
$c_{R-1}$ перейти из $(R-1,1)$ в $(R-2,1)$.
Следовательно,
\begin{equation}
\label{3.1}
c_{R-1} = \overline{a}_1 b + (a_1 b + \overline{a}_1 \overline{b}) c_{R-1} \,.
\end{equation}
%%%%%%
Из состояния $(n,1)$, ${n=\overline{H+1,R-2}}$,
в состояние $(n-1,1)$ можно попасть, как и в предыдущем случае,
или после первого же шага, или оставшись после
первого такта в этом состоянии, а затем уже перейти в $(n-1,1)$.
Кроме того, из $(n,1)$ в $(n-1,1)$ раньше, чем в $(R,2)$,
можно попасть, перейдя на первом шаге с
вероятностью $a_1 \overline{b}$ в состояние $(n+1,1)$,
затем, не заходя в $(R,2)$, с вероятностью
$c_{n+1}$ вернуться
в состояние $(n,1)$ и, наконец, снова, не заходя
в $(R,2)$,
с вероятностью~$c_{n}$ попасть в $(n-1,1)$.
Поэтому

\noindent
\begin{multline}
\label{3.2}
c_n= \overline{a}_1 b + (a_1 b + \overline{a}_1 \overline{b}) c_{n}
+ a_1 \overline{b} c_{n+1} c_n \,,\\ n=\overline{H+1,R-2}\,.
\end{multline}

Пусть в начальный момент цепь Маркова $\nu(t)$
находится в состоянии
${(n,0) \hm\in {\cal X}_{20}}$,
${n=\overline{L+1,H-1}}$
(т.\,е.\ в системе имеется $n$~заявок, причем в
систему принимаются заявки обоих типов).
Обозначим через~$c_n$ вероятность того, что $\nu(t)$
попадет в состояние ${(n-1,0)}$ раньше, чем в
состояние $(H,1)$ (т.\,е.\ до того момента, когда в сис\-те\-ме впервые
останется ${(n-1)}$ заявок, в системе никогда не
будет $H$~заявок).
Следующие формулы
получаются аналогично~(\ref{3.1}) и~(\ref{3.2}):

\noindent
\begin{align}
\label{3.3}
c_{H-1} &= \overline{a} b + (a b + \overline{a} \overline{b}) c_{H-1}\,;
\\
c_n &= \overline{a} b + (a b + \overline{a} \overline{b}) c_{n}
+ a \overline{b} c_{n+1} c_n \,,\notag
\\
 &\hspace*{30mm}n=\overline{L+1,H-2}\,. \label{3.4}
\end{align}

Пусть в начальный момент цепь Маркова $\nu(t)$
находится в состоянии ${(n,1) \hm\in {\cal X}_{21}}$,
${n\hm=\overline{L,H-1}}$
(т.\,е.\ в системе имеется $n$~заявок, причем в
систему принимаются заявки только первого типа).
Обозначим через $c^*_n$ вероятность того,
что $\nu(t)$ попадет в состояние $(n+1,1)$ раньше,
чем в состояние ${(L-1)}$
(т.\,е.\ до того момента, когда в системе впервые
окажется ${(n+1)}$ заявок, в системе никогда не
будет меньше $L$~заявок).
Поскольку из состояния $(L,1)$ на каждом такте
можно попасть только в состояния ${(L+1,1)}$,
$(L,1)$ и $(L-1)$, причем с вероятностями
$a_1 \overline{b}$, ${a_1 b \hm+ \overline{a}_1 \overline{b}}$ и
$\overline{a}_1 b$, то
\begin{equation}
\label{3.5}
c^*_{L}= a_1 \overline{b} + (a_1 b + \overline{a}_1 \overline{b}) c^*_{L}\,.
\end{equation}
Из состояния $(n,1)$, ${n\hm=\overline{L+1,H-1}}$,
в состояние $(n+1,1)$, не заходя в ${(L-1)}$, можно попасть
с вероятностью $a_1 \overline{b}$ после первого
шага, а также оставшись после первого шага с
вероятностью ${a_1 b\hm + \overline{a}_1 \overline{b}}$ в состоянии
$(n,1)$, а затем уже с вероятностью $c^*_{n}$
пе\-рейти из $(n,1)$ в ${(n+1,1)}$.
Кроме того, не заходя в $(L-1)$, в это же
состояние можно попасть, пе\-рейдя на первом шаге
с вероятностью $\overline{a}_1 b$
в состояние $(n-1,1)$, затем, не заходя в $(L-1)$,
с вероятностью $c^*_{n-1}$ вернуться в состояние
$(n,1)$ и, наконец, снова, не заходя в $(L-1)$,
с вероятностью $c^*_{n}$ попасть в $(n+1,1)$.
Значит,

\noindent
\begin{multline}
\label{3.6}
c^*_n= a_1 \overline{b} + (a_1 b + \overline{a}_1 \overline{b}) c^*_n
+ \overline{a}_1 b c^*_{n-1} c^*_n \,,
\\
n=\overline{L+1,H-1}\,.
\end{multline}


Соотношения (\ref{3.1})--(\ref{3.6}) позволяют
получить рекуррентный алгоритм вычисления $c_n$ и~$c^*_n$.
Действительно, $c_n$ при ${n\hm=\overline{H+1,R-1}}$
вычисляются последовательно от $n\hm=R\hm-1$ до $H\hm+1$ по
формулам~(\ref{3.1}) и~(\ref{3.2}):
\begin{align*}
c_{R-1} &= \fr{\overline{a}_1 b}{a_1 \overline{b} + \overline{a}_1 b}
\,;
\\
c_{n} &= \fr{\overline{a}_1 b }{a_1 \overline{b} + \overline{a}_1 b -
a_1 \overline{b} c_{n+1}}\,,\enskip n=\overline{H+1,R-2}\,,
\end{align*}
а при $n=\overline{L+1,H-1}$~---
последовательно от $n\hm=H\hm-1$ до $L\hm+1$ по формулам~(\ref{3.3}) и~(\ref{3.4}):
$$
c_{H-1}= \fr{\overline{a} b}{a \overline{b} + \overline{a} b} \,;
$$
$$
c_{n}= \fr{\overline{a} b }
{a \overline{b} + \overline{a} b - a \overline{b} c_{n+1}}\,,\enskip
n=\overline{L+1,H-2}\,.
$$
Аналогично $c^*_n$ вычисляются последовательно
от $n\hm=L$ до ${n\hm=H\hm-1}$ по формулам~(\ref{3.5}) и~(\ref{3.6}):
\begin{align*}
c^*_{L}&= \fr{a_1 \overline{b}}{a_1 \overline{b} + \overline{a}_1 b}\,;
\\
c^*_{n}&= \fr{a_1 \overline{b}}{a_1 \overline{b} + \overline{a}_1 b - \overline{a}_1 b c^*_{n-1}}
\,,\enskip  n=\overline{L+1,H-1}\,.
\end{align*}

\section{Стационарные вероятности состояний}

Рассмотрим стационарный режим функционирования системы.

Введем обозначения (напомним, что состояние системы
определяется после окончания такта):
\begin{description}
\item[\,] $p_n$, $n=\overline{0,L-1}$,~--- стационарная
вероятность того, что сис\-те\-ма находится в
состоянии $(n)$ (в~сис\-те\-ме имеется $n$~заявок);
\item[\,]
$p_n$, $n\hm=\overline{L,H-1}$,~--- стационарная
вероятность того, что сис\-те\-ма находится в
состоянии $(n,0)$
(в~сис\-те\-ме имеется $n$~заявок и к обслуживанию
принимаются все заявки);
\item[\,]
$p'_n$, $n=\overline{L,R-1}$,~--- стационарная
вероятность того, что сис\-те\-ма находится в
состоянии $(n,1)$ (в~сис\-те\-ме имеется $n$~заявок и к обслуживанию
принимаются только заявки первого типа);
\item[\,]
$p''_n$, $n=\overline{H+1,R}$,~--- стационарная
вероятность того, что сис\-те\-ма находится в
состоянии $(n,2)$
(в сис\-те\-ме имеется $n$~заявок и новые заявки к
обслуживанию не принимаются).
\end{description}

Выпишем систему уравнений равновесия (СУР).

Рассматривая потоки вероятностей переходов из
состояний $(n-1)$ в
$(n)$, $n\hm=\overline{1,L-1}$, и обратно, получаем
\begin{equation}
\label{4.1}
p_{n} \overline{a} b = p_{n-1} a \overline{b} \,,
\enskip n=\overline{1,L-1}\,.
\end{equation}

Для нахождения $p_{n}$, $n\hm=\overline{L,H-2}$,
применим метод исключения состояний.
С~этой целью исключим все состояния
$(i,0)$, ${i\hm=\overline{n+1,H-1}}$ (в~сис\-те\-ме
находится~$i$, ${i\hm=\overline{n+1,H-1}}$, заявок
и к обслуживанию принимаются все заявки).
После такого исключения из состояния $(n,0)$ можно
перейти в состояние $(n-1,0)$ (с
вероятностью $\overline{a} b$) и в состояние $(H,1)$
(с вероятностью~$a \overline{b} \overline{c}_{n+1}$), а в
состояние $(n,0)$ можно попасть из состояния $(n-1)$
(с вероятностью $a\overline{b}$) и из самого этого состояния
(с вероятностью $a b \hm+ \overline{a} \overline{b}\hm + a \overline{b} c_{n+1}$).
Из уравнения глобального баланса для
состояния $(n,0)$ имеем:
\begin{multline}
\label{4.2}
p_{n}= p_{n-1} a \overline{b} + p_{n} (a b + \overline{a} \overline{b} +
 a \overline{b} c_{n+1}) \,,
\\ n=\overline{L,H-2}\,.
\end{multline}
Из уравнения глобального баланса для со\-сто\-яния $(H-1,0)$
(в системе находится $(H-1)$ заявок и к обслуживанию принимаются
все заявки) находим
\begin{equation}
\label{4.3}
p_{H-1} = p_{H-2} a \overline{b} + p_{H-1} (a b + \overline{a} \overline{b}) \,.
\end{equation}

Снова обратимся к методу исключения состояний.
Исключим состояния $(i,1)$, ${i=\overline{H+1,R-1}}$
(в системе находится более $H$~заявок и к
обслуживанию принимаются только заявки первого
типа), состояния $(i,1)$, $i\hm=\overline{L,H-1}$
(в системе находится менее $H$~заявок и к
обслуживанию принимаются только заявки первого
типа), и состояния $(i,2)$, $i\hm=\overline{H+1,R}$
(в системе находится более $H$~заявок и заявки к
обслуживанию не принимаются).
После этого исключения из состояния $(H,1)$ можно
будет перейти только в состояние $(L-1)$ (с вероятностью
$\overline{a}_1 b \overline{c}^*_{H-1}$),
а в состояние $(H,1)$ можно попасть из
состояния $(H-1,0)$ (с вероятностью $a \overline{b}$)
и из самого состояния $(H,1)$ (с вероятностью
$a_1 b \hm+ \overline{a}_1 \overline{b}\hm + a_1 \overline{b} +
\overline{a}_1 b c^*_{H-1}$).
Поэтому из уравнения глобального баланса для этого состояния получаем
\begin{equation}
\label{4.4}
p'_{H} = p_{H-1} a \overline{b} + p'_{H} (a_1 b + \overline{a}_1 \overline{b}
+ a_1 \overline{b} + \overline{a}_1 b c^*_{H-1})\,.
\end{equation}

Исключая состояния $(i,1)$, $i\hm=\overline{n+1,R-1}$
(в~сис\-те\-ме находится~$i$, $i\hm=\overline{n+1,R-1}$, заявок
и к обслуживанию принимаются только заявки первого типа),
из уравнения глобального баланса для состояния
$(n,1)$, $n\hm=\overline{H+1,R-1}$, имеем:
\begin{multline}
\label{4.5}
p'_{n} = p'_{n-1} a_1 \overline{b} +
p'_{n} (a_1 b + \overline{a}_1 \overline{b} + a_1 \overline{b} c_{n+1})\,,
\\
n=\overline{H+1,R-2}\,,
\end{multline}
\begin{equation}
\label{4.6}
p'_{R-1} = p'_{R-2} a_1 \overline{b} + p'_{R-1}
(a_1 b + \overline{a}_1 \overline{b}) \,.
\end{equation}
Из уравнения глобального баланса для
состояния $(R,2)$ (в системе $R$~заявок) находим:
\begin{equation}
\label{4.7}
b p''_{R} = p'_{R-1} a_1 \overline{b} \,.
\end{equation}

Аналогично из уравнения глобального баланса для
состояния $(n,2)$, ${n\hm=\overline{H+1,R-1}}$
(в системе находится~$n$, $n=\overline{H+1,R-1}$,
заявок и новые заявки в систему не принимаются),
получаем:
\begin{equation}
\label{4.8}
p''_{n} = p''_{n+1} \,,
\enskip n=\overline{H+1,R-1}\,.
\end{equation}

Наконец, исключая состояния~$i$, $i\hm=\overline{L,n-1}$
(в системе находится~$i$, $i\hm=\overline{L,n-1}$, заявок
и к обслуживанию принимаются только заявки первого типа),
из уравнения глобального баланса для состояния
$(n,1)$, $n\hm=\overline{L,H-1}$, имеем:
\begin{align}
p'_{n} &= p'_{n+1} \overline{a}_1 b +
p'_{n} (a_1 b + \overline{a}_1 \overline{b} + \overline{a}_1 b c^*_{n-1})\,,
\notag\\
 &\hspace*{40mm}n=\overline{L+1,H-1}\,;\label{4.9}
\\
\label{4.10}
p'_{L} &= p'_{L+1} \overline{a}_1 b +
p'_{L} (a_1 b + \overline{a}_1 \overline{b})\,.
\end{align}

Вероятность $p_{0}$ определяется из условия нормировки:
\begin{equation}
\label{4.11}
\sum\limits_{n=0}^{H-1} p_{n} + \sum\limits_{n=L}^{R-1} p'_{n} +
\sum\limits_{n=H+1}^{R} p''_{n} = 1\,.
\end{equation}

Приведем алгоритм решения СУР.

Сначала по формулам~(\ref{4.1})--(\ref{4.3})
последовательно по~$n$ от $n\hm=1$ до $(H\hm-1)$
через~$p_0$ вычисляются вероятности~$p_n$:
\begin{align*}
%\label{4.12}
p_{n} &= \begin{cases}
\fr{a \overline{b}}{\overline{a} b} p_{n-1}\,,
&  n=\overline{1,L-1}\,;
\\
 \fr{a \overline{b}} {a \overline{b} + \overline{a} b - a \overline{b} c_{n+1}} \,p_{n-1}\,,
&
n=\overline{L,H-2}\,;
\end{cases}\\
 p_{H-1} &= \fr{a \overline{b}}{a \overline{b} + \overline{a} b}\, p_{H-2}\,.
\end{align*}
%%%%%%%%%%%%%%%%%%%
Затем по формуле~(\ref{4.4}) находится~$p'_{H}$:
\begin{equation*}
%\label{4.14}
p'_{H} =
\fr{ a \overline{b}}{\overline{a}_1 b - \overline{a}_1 b c^*_{H-1}}\, p_{H-1}\,,
\end{equation*}
%%%%%%%%%%%%%%%%%%%
а по формулам~(\ref{4.5}),~(\ref{4.6}) последовательно
по~$n$ от $(H\hm+1)$ до $(R\hm-1)$ и по формулам~(\ref{4.9}),
(\ref{4.10}) последовательно по~$n$ от $(H\hm-1)$ до~$L$ определяются~$p'_{n}$:

\noindent
\begin{align*}
%\label{4.15}
p'_{n} &=
\fr{a_1 \overline{b}}{a_1 \overline{b} + \overline{a}_1 b - a_1 \overline{b} c_{n+1}}
\,p'_{n-1} \,,\\
&\hspace*{40mm}n=\overline{H+1,R-2}\,;
\\
p'_{R-1} &=
\fr{a_1 \overline{b}}{a_1 \overline{b} + \overline{a}_1 b} p'_{R-2} \,;
\\
%\label{4.16}
p'_{n} &=
\fr{\overline{a}_1 b }{a_1 \overline{b} + \overline{a}_1 b -
\overline{a}_1 b c^*_{n-1}}\,p'_{n+1}\,,
\\
&\hspace*{41mm}n=\overline{L+1,H-1}\,;
\\
p'_{L} & =
\fr{\overline{a}_1 b }{a_1 \overline{b} + \overline{a}_1 b}\, p'_{L+1}.
\end{align*}
Далее по формулам~(\ref{4.7}) и~(\ref{4.8})
последовательно по~$n$ от $(R\hm-1)$ до $(H\hm+1)$ находятся~$p''_{n}$:
\begin{equation*}
%\label{4.17}
p''_{R} = \fr{a_1 \overline{b} }{ b} p'_{R-1}\,,
%$$
%%%%%%%%%%
%$$
\enskip
p''_{n} = p''_{n+1} \,,
\ \ n=\overline{H+1,R-1}\,.
\end{equation*}
%%%%%%%%%%%%
Наконец, вычисляется $p_{0}$ из условия нормировки~(\ref{4.11}).


В заключение этого раздела приведем выражение для
стационарного среднего числа~$N$~заявок в сис\-теме:
\begin{equation*}
%\label{4.18}
N = \sum\limits_{n=0}^{H-1} n p_{n} + \sum\limits_{n=L}^{R-1} n p'_{n}
+ \sum\limits_{n=H+1}^{R} n p''_{n}\,.
\end{equation*}

\section{Время выхода из~множества состояний}

В этом разделе решим следующую задачу:
найти распределения времен выхода цепи Маркова~$\nu(t)$ из некоторых множеств состояний.
Задача будет решена двумя способами.
Первый способ заключается в последовательном вычислении вероятностей.
Удобство этого способа применительно к настоящей системе
заключается в том, что переходы цепи Маркова из каждого состояния происходят не более чем в
три других состояния, что снижает число операций
при вычислении вероятностей выхода на каждом
шаге с~$n^2$~операций до~$3 n$~операций.
Второй способ~--- определение времен выхода в терминах производящих функций (ПФ).
Такой способ обычно применяется при вычислении моментов случайных величин.

В качестве множеств, вероятности выхода из которых
будут найдены, выбраны множество
${{\cal Y}_0\hm={\cal X}_0\cup{\cal X}_{20}}$
нормального функционирования сис\-те\-мы и множество
${{\cal Y}_1\hm= {\cal X}_{21}\cup{\cal X}_{31}\cup{\cal X}_{32}}$,
при пребывании системы в котором либо принимаются
заявки только первого типа, либо вообще не
принимаются никакие заявки, хотя используемые
здесь методы применимы и для вычисления распределения времени до момента
первого достижения из каждого состояния (или множества
состояний) любого другого состояния (или множества состояний).
Выбор множеств~${\cal Y}_0$ и~${\cal Y}_1$ связан с
техническим приложением задачи к анализу качества
управления SIP-сер\-ве\-ром с по\-мощью вариантов гистерезисной политики.

\subsection{Последовательное вычисление вероятностей}

Пусть в начальный момент система находится в
состоянии~$n$, ${n\hm=\overline{0,L-1}}$, или
в состоянии $(n,0)$, ${n\hm=\overline{L,H-1}}$ (в обоих
этих случаях в систему принимаются заявки любых типов).
Обозначим через~$t_{n,i}$, $n\hm=\overline{0,H-1}$, $i\hm\ge1$,
вероятность того, что первый переход из множества
состояний ${{\cal X}_0\cup{\cal X}_{20}}$ в состояние~$H$ (т.\,е.\ в множество
${{\cal X}_{21}\cup{\cal X}_{31}\cup{\cal X}_{32}}$)
произойдет на $i$-м шаге, через ${\vec t_i\hm= (t_{0,i},\ldots,t_{H-1,i})^{\mathrm{T}}}$,
${i\hm\ge1}$,~--- вектор размерности~$H$ с координатами~$t_{n,i}$
и через $P\hm=(p_{n,m})_{n,m=\overline{0,H-1}}$~---
квадратную матрицу порядка~$H$ с ненулевыми элементами:
\begin{align*}
%\label{5.1.1}
p_{n,n} &=
\begin{cases}
\overline{a}\,,            &n=0\,,             \\
\overline{a}\overline{b} + a b\,,    &n=\overline{1,H-1}\,,
\end{cases}
\\
%\label{5.1.2}
p_{n,n+1}
&= \begin{cases}
a\,,             &n=0\,,             \\
a\overline{b}\,,           &n=\overline{1,H-2}\,,
\end{cases}
\\
%\label{5.1.3}
p_{n,n-1} &= \overline{a} b\,,\enskip       n=\overline{1,H-1}\,.
\end{align*}

Заметим теперь, что вектор~$\vec t_1$ имеет вид:
\begin{equation}
\label{5.1.4}
\vec t_1 = (0,\ldots,0,a\overline{b})^{\mathrm{T}}\,,
\end{equation}
а векторы $\vec t_i$ при $i\hm\ge 2$ определяются рекуррентной формулой
\begin{equation}
\label{5.1.5}
\vec t_i = P \vec t_{i-1}\,,  \enskip i\ge 2\,.
\end{equation}

Далее, пусть в начальный момент система находится в
состоянии $(n,1)$, ${n\hm=\overline{L,R-1}}$ (в систему
принимаются заявки только первого типа).
Обозначим через $t'_{n,i}$, $n\hm=\overline{L,R-1}$, ${i\hm\ge1}$,
вероятность того, что первый переход из множества
состояний ${{\cal X}_{21}\cup{\cal X}_{31}}$
в состояние $(L-1)$ (т.\,е.\ в множество ${{\cal X}_0\cup{\cal X}_{20}}$)
произойдет на $i$-м шаге.

Наконец, пусть в начальный момент система находится
в состоянии $(n,2)$,\ $n\hm=\overline{H+1,R}$ (в систему
не принимаются заявки любого типа).
Обозначим через
$t''_{n,i}$, $n\hm=\overline{H+1,R}$, $i\hm\ge1$,
вероятность того, что первый переход из множества
состояний~${{\cal X}_{32}}$ в состояние $(L-1)$ (т.\,е.\ в множество
${{\cal X}_0\cup{\cal X}_{20}}$) произойдет на $i$-м шаге.

Введем теперь вектор
\begin{multline*}
\vec t_i^* = (t^*_{L,i},\ldots,t^*_{2R-H-1,i})^{\mathrm{T}}
={}\\
{}= (t'_{L,i},\ldots,t'_{R-1,i}, t''_{H+1,i},\ldots,t''_{R,i})^{\mathrm{T}}\,,\
i\ge1\,,
\end{multline*}
размерности $R\hm-L\hm+R\hm-H\hm=2R\hm-L\hm-H$, первые
$(R\hm-L)$ координат которого образуют вероятности~$t'_{n,i}$, а
остальные $(R\hm-H)$~--- вероятности~$t''_{n,i}$,
и квадратную матрицу
$$
P^*=(p^*_{n,m})_{n,m=\overline{L,2R-H-1}}
$$
порядка~$2R\hm-L\hm-H$ с ненулевыми элементами:
\begin{align*}
%\label{5.1.6}
p^*_{n,n} &= \begin{cases}
\overline{a}_1 \overline{b} + a_1 b\,,   &n=\overline{L,R-1}\,;   \\
\overline{b}\,,                &n=\overline{R,2R-H-1}\,;
\end{cases}
\\
%\label{5.1.7}
p^*_{n,n+1} &= \begin{cases}
a_1 \overline{b}\,,        &n=\overline{L,R-2}\,;      \\
0\,,             &n=\overline{R-1,2R-H-2}\,;
\end{cases}
\\
%\label{5.1.8}
p^*_{R-1,2R-H-1} &= a_1 \overline{b}\,;
\\
%\label{5.1.9}
p^*_{n,n-1} &= \begin{cases}
\overline{a}_1 b\,,        &n=\overline{L+1,R-1}\,;      \\
b\,,             &n=\overline{R+1,2R-H-1}\,;
\end{cases}
\\
%\label{5.1.10}
p^*_{R,H} &= b\,.
\end{align*}

Заметим теперь, что вектор~$\vec t^*_1$ имеет вид:
\begin{equation}
\label{5.1.10}
\vec t^*_1 = (\overline{a}_1 b,0,\ldots,0)^{\mathrm{T}}\,,
\end{equation}
а векторы $\vec t^*_i$ при $i\hm\ge 2$ определяются рекуррентной формулой
\begin{equation}
\label{5.1.11}
\vec t^*_i = P^* \vec t^*_{i-1}\,, \enskip i\ge 2\,.
\end{equation}

Соотношения~(\ref{5.1.4})--(\ref{5.1.11}) позволяют вычислять вероятности моментов выхода
из рассматриваемого множества состояний.

\subsection{Применение производящих функций}

Введем ПФ
$$
T(z|n) = \sum\limits_{i=1}^{\infty} z^i t_{n,i} \,,\enskip n=\overline{0,H-1}\,;
$$
$$
T'(z|n) = \sum\limits_{i=1}^{\infty} z^i t'_{n,i} \,,\enskip n=\overline{L,R-1}\,;
$$
$$
T''(z|n) = \sum\limits_{i=1}^{\infty} z^i t''_{n,i}\,,\enskip n=\overline{H+1,R}\,.
$$


Для вычисления ПФ времен выхода из состояний
множества ${{\cal X}_0\cup{\cal X}_{20}}$ и
из состояний множества
${{\cal X}_{21}\cup{\cal X}_{31}\cup{\cal X}_{32}}$
удобно воспользоваться одним из\linebreak двух следующих способов.

Первый способ заклю\-чается в решении однородного
разностного уравнения второго порядка с постоянными
коэф\-фициентами.

Второй способ аналогичен методу, примененному в
разд.~4 для нахождения стационарных вероятностей
состояний, и основан на исключении определенных состояний.

Чтобы продемонстрировать оба способа,
решим задачу вычисления ПФ времени выхода из
со\-сто\-яния множества ${{\cal X}_0\cup{\cal X}_{20}}$
первым способом, а из состояния множества
${{\cal X}_{21}\cup{\cal X}_{31}   \cup{\cal X}_{32}}$~--- вторым способом.

Вероятности $t_{n,i}$, $n\hm=\overline{0,H-1}$, $i\hm\ge 1$,
удовлетворяют соотношениям
\begin{align*}
%\label{5.2.1}
t_{n,1} &= 0 \,,\enskip n=\overline{0,H-2}\,;
\\
%\label{5.2.2}
t_{H-1,1} &= a \overline{b} \,;
\\
%\label{5.2.3}
t_{0,i} &= \overline{a}  t_{0,i-1} + a t_{1,i-1} \,,\enskip i\ge 2\,;
\\
%\label{5.2.4}
t_{n,i} &= \overline{a} b t_{n-1,i-1} + (a b + \overline{a} \overline{b}) t_{n,i-1}
+ a \overline{b} t_{n+1,i-1}\,,\\
&\hspace*{35mm}n=\overline{1,H-2}\,,\ \ i\ge 2\,;
\\
%\label{5.2.5}
t_{H-1,i} &= \overline{a} b t_{H-2,i-1} + (a b + \overline{a}
\overline{b}) t_{H-1,i-1} \,,\ \ i\ge 2\,,
\end{align*}
откуда, переходя к ПФ, получаем
\begin{equation}
\label{5.2.6}
T(z|0) = \overline{a} z T(z|0) + a z T(z|1)\,;
\end{equation}

\vspace*{-12pt}

\noindent
\begin{multline}
\label{5.2.7}
T(z|n) = \overline{a} b z T(z|n-1) + (a b + \overline{a} \overline{b}) z T(z|n)
+{}\\
{}+ a \overline{b} z T(z|n+1) \,,\ \ n=\overline{1,H-2}\,;
\end{multline}
\begin{multline}
\label{5.2.8}
T(z|H-1) = a \overline{b} z + \overline{a} b z T(z|H-2) + {}\\
{}+
(a b + \overline{a} \overline{b}) z T(z|H-1)\,.
\end{multline}

Решение системы уравнений~\eqref{5.2.6}--\eqref{5.2.8} имеет вид:
\begin{equation*}
%\label{5.2.9}
T(z|n) = C_1 u_1^n + C_2 u_2^n \,,\ \ n=\ov{0,H-1}\,,
\end{equation*}
где $u_1=u_1(z)$ и $u_2\hm=u_2(z)$~--- решения уравнения
\begin{equation*}
%\label{5.2.10}
u = \overline{a} b z + (a b + \overline{a} \overline{b}) z u +
a \overline{b} z u^2 \,,
\end{equation*}
т.\,е.
\begin{multline*}
%\label{5.2.11}
u_{1,2} ={}\\
\!{}= \fr{ 1 - (a b + \overline{a} \overline{b}) z
\pm \sqrt{[1 - (a b + \overline{a} \overline{b}) z]^2 -
4 a \overline{a} b \overline{b} z^2}}{2 a \overline{b} z }\,.
\end{multline*}

Для вычисления коэффициентов $C_1\hm=C_1(z)$ и
$C_2\hm=C_2(z)$ воспользуемся равенствами~\eqref{5.2.6} и~\eqref{5.2.8}.
Тогда
\begin{equation*}
%\label{5.2.12}
C_1 + C_2 = \overline{a} z (C_1 + C_2) + a z (C_1 u_1 + C_2 u_2)\,;
\end{equation*}

\vspace*{-12pt}

\noindent
\begin{multline*}
%\label{5.2.13}
\!\!\!C_1 u_1^{H-1} + C_2 u_2^{H-1} = a \overline{b} z +
\overline{a} b z (C_1 u_1^{H-2} + C_2 u_2^{H-2}) +{}\\
{}+
(a b + \overline{a} \overline{b}) z (C_1 u_1^{H-1} + C_2 u_2^{H-1})\,.
\end{multline*}
Из этих равенств получаем, что коэффициенты~$C_1$ и~$C_2$ имеют вид:
\begin{multline*}
%\label{5.2.16}
C_1 =
a \overline{b} z (\overline{a} z + a z u_2 - 1)\big /
\left((\overline{a} z + a z u_2 - 1) \left[ \vphantom{\overline{b}}
u_1 -{}\right.\right.\\
\left.\left.{}- \overline{a} b z -
(a b + \overline{a} \overline{b}) z u_1 \right] u_1^{H-2}
+ (\overline{a} z + a z u_1 - 1)\times{}\right.\\
\left.{}\times \left[
\overline{a} b z + (a b + \overline{a} \overline{b}) z u_2
- u_2 \right] u_2^{H-2} \right)\,;
\end{multline*}

%\vspace*{-12pt}

\noindent
\begin{multline*}
%\label{5.2.17}
C_2 =
{ a \overline{b} z (1 - \overline{a} z - a z u_1)}\big /
\left((\overline{a} z + a z u_2 - 1) \left[  \vphantom{\overline{b}}
u_1 -{}\right.\right.\\
\left.{}- \overline{a} b z -
(a b + \overline{a} \overline{b}) z u_1 \right] u_1^{H-2} +{}\\
\!\!\!\left.{}+(\overline{a} z + a z u_1 - 1)\!
\left[ \overline{a} b z
+ (a b + \overline{a} \overline{b}) z u_2 - u_2 \right] u_2^{H-2}\right).\hspace*{-3.44753pt}
\end{multline*}

Вычисление ПФ $T'(z|n)$, $n\hm=\overline{L,R-1}$,
и $T''(z|n)$, $n\hm=\overline{H+1,R}$, начнем с введения вспомогательных ПФ:
\begin{description}
\item[\,]
$G^*(z|n)$, $n\hm=\overline{H+1,R-1}$,~--- ПФ момента
первого достижения состояния $(n-1,1)$ и
вероятность того, что до этого момента система
не попадет в состояние $(R,2)$, при условии,
что в начальный момент система находилась в
состоянии $(n,1)$;\\[-9pt]
\item[\,]
$g^*(z|n)$, $n\hm=\overline{H+1,R-1}$,~--- ПФ момента
первого достижения состояния $(n+1,1)$
(или состояния $(R,2)$, если $n\hm=R\hm-1$) и
вероятность того, что до этого момента система
не попадет в состояние $(H,1)$, при условии,
что в начальный момент система находилась в
состоянии $(n,1)$;\\[-9pt]
\item[\,]
$\tilde{G}(z|n)$, $n\hm=\overline{H+1,R-1}$,~--- ПФ момента
первого достижения состояния $(H,1)$ и
вероятность того, что до этого момента система
не попадет в состояние $(R,2)$, при условии,
что в начальный момент система находилась в
состоянии $(n,1)$;\\[-9pt]
\item[\,]
$\tilde{g}(z|n)$, $n\hm=\overline{H+1,R-1}$,~--- ПФ момента
первого достижения состояния $(R,2)$ и
вероятность того, что до этого момента система
не попадет в состояние $(H,1)$, при условии,
что в начальный момент система находилась в состоянии $(n,1)$;\\[-9pt]
\item[\,]
$G'(z|n)$, $n\hm=\overline{H+1,R-1}$,~--- ПФ момента
первого достижения состояния $(H,1)$ при условии,
что в начальный момент система находилась в состоянии $(n,1)$;\\[-9pt]
\item[\,]
$G''(z|n)$, $n\hm=\overline{H+1,R}$,~--- ПФ момента
первого достижения состояния $(H,1)$ при
условии, что в начальный момент система находилась в состоянии $(n,2)$.
\end{description}

Очевидно,

\noindent
\begin{equation*}
%\label{5.2.18}
G''(z|n) = \left(\fr{b z}{ 1 - \overline{b} z} \right)^{n-H}\,,\enskip n=\ov{H+1,R}\,.
\end{equation*}


Далее,
\begin{multline}
G^*(z|n) = z \left[ \overline{a}_1 b +
(a_1 b + \overline{a}_1 \overline{b}) G^*(z|n)
+{}\right.\\[2pt]
\hspace*{-1mm}\left.{}+ a_1 \overline{b} G^*(z|n+1) G^*(z|n) \right],\ \ n=\overline{H+1,R-1};\!\!\!
\label{5.2.19}
\end{multline}

\vspace*{-12pt}

\noindent
\begin{multline}
g^*(z|n) = z \left[ \overline{a}_1 b g^*(z|n-1) g^*(z|n)
+{}\right.\\
\left.{}+ (a_1 b + \overline{a}_1 \overline{b}) g^*(z|n) +
a_1 \overline{b} \right]\,,\\
n=\overline{H+1,R-1},
\label{5.2.20}
\end{multline}
где положено

\columnbreak

\noindent
\begin{equation}
\label{5.2.21}
G^*(z|R) = 0\,;
\end{equation}
\begin{equation}
\label{5.2.22}
g^*(z|H) = 0\,.
\end{equation}
Решая уравнения~\eqref{5.2.19} и~\eqref{5.2.20}, получаем
\begin{multline}
\label{5.2.23+}
G^*(z|n) = z \fr{ \overline{a}_1 b }{1 - (a_1 b + \overline{a}_1 \overline{b}) z
- a_1 \overline{b} z G^*(z|n+1) }\,,\\
n=\overline{H+1,R-1}\,;
\end{multline}

\vspace*{-12pt}

\noindent
\begin{multline}
\label{5.2.24+}
g^*(z|n) = z\fr{ a_1 \overline{b} }{1 - \overline{a}_1 b z g^*(z|n-1)
- (a_1 b + \overline{a}_1 \overline{b}) z }\,,\\
n=\overline{H+1,R-1}\,.
\end{multline}

Согласно формулам~\eqref{5.2.21} и~\eqref{5.2.23+}
функции $G^*(z|n)$, $n\hm=\overline{H+1,R-1}$, являются
дроб\-но-ра\-ци\-о\-наль\-ны\-ми функциями $G^*(z|n)\hm=z G_n^*(z)/H_n^*(z)$, где
полиномы $G_n^*(z)$ и $H_n^*(z)$ степеней $(R\hm-n\hm-1)$ и $(R\hm-n)$ вычисляются
по рекуррентным соотношениям:
\begin{align*}
%\label{5.2.25}
G_{R-1}^*(z) &= \overline{a}_1 b \,;
\\
%\label{5.2.26}
H_{R-1}^*(z) &= 1 - (a_1 b + \overline{a}_1 \overline{b}) z\,;
\end{align*}
\begin{equation}
\label{5.2.27}
G_n^*(z) = \overline{a}_1 b H_{n+1}^*(z) \,,\ \ n=\ov{H+1,R-2}\,;
\end{equation}

\vspace*{-12pt}

\noindent
\begin{multline*}
%\label{5.2.28}
H_n^*(z) = \left[1 - (a_1 b + \overline{a}_1 \overline{b}) z\right] H_{n+1}^*(z) -{}\\
{}- a_1 \overline{b} z^2 G_{n+1}^*(z) \,,\ \
n=\overline{H+1,R-2}\,.
\end{multline*}


Аналогично в соответствии с
формулами~\eqref{5.2.22} и~\eqref{5.2.24+}
функции $g^*(z|n)$, ${n\hm=\overline{H+1,R-1}}$, также
являются дроб\-но-ра\-ци\-о\-наль\-ны\-ми
функциями ${g^*(z|n)\hm=z g_n^*(z)/h_n^*(z)}$ с
полиномами~$g_n^*(z)$ и~$h_n^*(z)$ степеней
$(n\hm-H\hm-1)$ и $(n\hm-H)$, вычисляемыми рекуррентно следующим образом:
\begin{align*}
%\label{5.2.29}
g_{H+1}^*(z) &= a_1 \overline{b} \,;
\\
%\label{5.2.30}
h_{H+1}^*(z) &= 1 - (a_1 b + \overline{a}_1 \overline{b}) z \,;
\end{align*}
\begin{equation}
\label{5.2.31}
g_n^*(z) = a_1 \overline{b} h_{n-1}^*(z) \,,\ \ n=\overline{H+2,R-1}\,;
\end{equation}

\vspace*{-12pt}

\noindent
\begin{multline*}
%\label{5.2.32}
h_n^*(z) = \left[1 - (a_1 b + \overline{a}_1 \overline{b}) z\right] h_{n+1}^*(z)
- \overline{a}_1 b z^2 g_{n-1}^*(z) \,,\\
n=\overline{H+2,R-1}\,.
\end{multline*}

Из приведенных соотношений получаем выражения для
$\tilde{G}(z|n)$ и $\tilde{g}(z|n)$, с учетом~\eqref{5.2.27}
и~\eqref{5.2.31} имеющие вид:
\begin{equation*}
%\label{5.2.33}
\tilde{G}(z|H+1) = G^*(z|H+1) = z \fr{G^*_{H+1}(z)}{H^*_{H+1}(z)}\,;
\end{equation*}

\vspace*{-12pt}

\noindent
\begin{multline*}
%\label{5.2.34}
\tilde{G}(z|n) = G^*(z|n) \tilde{G}(z|n-1) ={}\\
{}=
(\overline{a}_1 b z)^{n-H-1} z \fr{G^*_n(z)}{H^*_{H+1}(z)}\,,\ \ n=\overline{H+2,R-1}\,;
\end{multline*}

\vspace*{-6pt}

\noindent
\begin{equation*}
%\label{5.2.33}
\tilde{g}(z|R-1) = g^*(z|R-1) = z \fr{g^*_{R-1}(z) }{h^*_{R-1}(z)}\,;
\end{equation*}

%\vspace*{-12pt}

\noindent
\begin{multline*}
%\label{5.2.34}
\tilde{g}(z|n) = g^*(z|n) \tilde{G}(z|n+1)
={}\\
{}=
(a_1 \overline{b} z)^{R-n-1} z \fr{g^*_n(z) }{ h^*_{R-1}(z)}\,,\ \
n=\overline{H+1,R-2}\,.
\end{multline*}

Вспомогательная функция $G'(z|n)$ определяется формулой:
\begin{multline*}
%\label{5.2.35}
G'(z|n) = \tilde{G}(z|n) + \tilde{g}(z|n) G''(z|R) =
z \fr{G'_{n}(z)}{ H'_{H+1}(z)}\,,\\
n=\overline{H+1,R-1}\,,
\end{multline*}
где полиномы $G'_{n}(z)$ и $H'_{H+1}(z)$
степеней $(2R\hm-2H\hm-2)$ и $(2R\hm-2H\hm-1)$ задаются равенствами:
\begin{multline*}
%\label{5.2.36}
G'_n(z) = (1 - \overline{b} z)^{R-H} (\overline{a}_1 b z)^{n-H-1}
G^*_{n}(z) + {}\\
{}+(b z)^{R-H} (a_1 \overline{b} z)^{R-n-1} g^*_{n}(z) \,,\ \ n=\overline{H+1,R-1}\,;
\end{multline*}
\begin{equation*}
%\label{5.2.37}
H'_{H+1}(z) = (1 - \overline{b} z)^{R-H} H^*_{H+1}(z) \,.
\end{equation*}

Последняя необходимая вспомогательная функция
$G'(z|n)$, $n\hm=\overline{L,H}$, представляющая собой
ПФ момента первого достижения состояния $(n-1,1)$
при условии, что в начальный момент сис\-те\-ма
находилась в состоянии $(n,1)$, удовлетворяет уравнению
\begin{multline*}
%\label{5.2.38}
G'(z|n) = z \left[ \overline{a}_1 b +
(a_1 b + \overline{a}_1 \overline{b}) G'(z|n) + {}\right.\\
\left.{}+a_1 \overline{b}
G'(z|n+1) G'(z|n) \right]\,,\ \ n=\overline{L,H}\,,
\end{multline*}
т.\,е.\ определяется равенством:
\begin{multline*}
%\label{5.2.39}
G'(z|n) = \fr{ \overline{a}_1 b z }{1 - (a_1 b + \overline{a}_1 \overline{b}) z
- a_1 \overline{b} z G'(z|n+1) } ={}\\
{}=
z \fr{G'_{n}(z)}{H'_{n}(z)}\,,\ \ n=\overline{L,H}\,,
\end{multline*}
где полиномы $G'_{n}(z)$, $n\hm=\overline{L,H}$,
и $H'_{n}(z)$, $n\hm=\overline{L,H}$, вычисляются по формулам:
\begin{align*}
%\label{5.2.40}
G'_n(z) &= \overline{a}_1 b z  H'_{n+1}(z)\, ,\enskip n=\overline{L,H}\,;
\\
%\label{5.2.41}
H'_{n}(z)
&= \left[ 1 - (a_1 b + \overline{a}_1 \overline{b}) z
\right] H'_{n+1}(z) -{}\\
&\hspace*{13mm}{}- a_1 \overline{b} z^2 G'_{n+1}(z)\,,\ \ n=\overline{L,H}\,.
\end{align*}

Теперь можно привести выражения для
ПФ $T'(z|n)$, $n\hm=\overline{L,R-1}$,
и $T''(z|n)$, ${n\hm=\overline{H+1,R}}$:
\begin{equation*}
%\label{5.2.42}
T'(z|L) = G'(z|L) = z \fr{G'_{L}(z)}{ H'_{L}(z)}\,;
\end{equation*}

\vspace*{-12pt}

\noindent
\begin{equation*}
%\label{5.2.43}
T'(z|n) =
\begin{cases}
G'(z|n) T'(z|n-1) ={}\\
{}= (\overline{a}_1 b z)^{n-L}
z \fr{G'_{n}(z) }{H'_{L}(z)} \,, &\hspace*{-28mm}
n=\overline{L+1,H};
\\
%\label{5.2.44}
G'(z|n) T'(z|H) = (\overline{a}_1 b z)^{H-L+1}
z \fr{G'_{n}(z)}{H'_{L}(z)} \,, &\\
&\hspace*{-32mm} n=\overline{H+1,R-1};
\end{cases}\hspace*{-9.86768pt}
\end{equation*}

%\vspace*{-12pt}

\noindent
\begin{multline*}
%\label{5.2.45}
T''(z|n) = G''(z|n) T'(z|H) = {}\\
{}=(\overline{a}_1 b z)^{H-L+1}
z \fr{G'_{H}(z)}{H'_{L}(z)} \left(
\fr{b z}{1 - \overline{b} z} \right)^{n-H} \,,\\
n=\overline{H+1,R}\,,
\end{multline*}
решающие задачу вычисления распределения моментов
выхода из рассматриваемого множества состояний в терминах ПФ.

\section{Средние времена}

Пусть в начальный момент система находится в
состоянии $(n,1)$, $n\hm=\overline{L,R-1}$
(в системе имеется $n$~заявок, причем к
обслуживанию принимаются заявки первого типа),
или в со\-сто\-янии $(n,2)$, $n\hm=\overline{H+1,R}$
(в системе имеется $n$~заявок и к обслуживанию новые заявки не принимаются).
Вы\-чис\-лим~$M_n$, $n\hm=\overline{L,R-1}$, и~$M^*_n$,\linebreak $n\hm=\overline{H+1,R}$,~--- средние
времена первого достижения состояния $(L-1)$
(в системе впервые останется $L-1$ заявок) для первого и второго вариантов.
При этом не будем пользоваться результатами
предыдущего раздела, дифференцируя соответствующие
ПФ, а выведем для средних простые рекуррентные соотношения.

Введем сначала следующие величины:
\begin{description}
\item[\,]
$m_n$, $n\hm=\overline{L,H}$,~--- среднее время до того момента,
когда система впервые попадет в состояние $(n-1,1)$
(в системе впервые останется $n-1$ заявок), при условии, что в начальный момент
система находилась в состоянии $(n,1)$
(в системе было $n$~заявок и принимались заявки только первого типа);
\item[\,]
$m^*_n$, $n\hm=\overline{H+1,R}$,~--- среднее время
до того момента, когда система впервые попадет в состояние $(n-1,2)$
(в системе впервые останется $n-1$ заявок), при условии, что в начальный момент
система находилась в состоянии $(n,2)$
(в системе было $n$~заявок и новые заявки не принимались);
\item[\,]
$m_n$, $n\hm=\overline{H+1,R-1}$,~--- среднее время
до того момента, когда система впервые попадет
или в состояние $(n-1,1)$ (в системе впервые останется
$n\hm-1$ заявок), причем всегда принимались заявки первого типа,
или в состояние $(H,1)$ (в системе впервые останется
$H$~заявок), причем с некоторого момента заявки в систему не принимались,
при условии, что в начальный момент система
находилась в состоянии$(n,1)$ (в системе было $n$~заявок
и принимались заявки первого типа);
\item[\,]
$\tilde{m}_n$, $n\hm=\overline{H+1,R-1}$,~--- среднее
время до того момента, когда система впервые попадет в со\-сто\-яние $(H,1)$
(в~системе впервые останется $H$~заявок),
при условии, что в начальный момент система находилась в состоянии $(n,1)$
(в~системе было $n$~заявок и принимались заявки первого типа).
\end{description}

Поскольку из состояния
$(n,2)$, $n\hm=\overline{H+1,R}$,
система может попасть только в состояние $(n-1,2)$
(в случае $n\hm=H\hm+1$ только в состояние $(H,1)$),
причем за геометрически распределенное с параметром~$b$ время, то
$$
m^*_{n} = \fr{1}{b}\,,
\enskip n=\overline{H+1,R} \,.
$$

Далее, из состояния $(R-1,1)$ за один шаг можно попасть
с вероятностью $\overline{a}_1 b$ в состояние $(R-2,1)$,
с вероятностью $a_1 \overline{b}$  в состояние $(R,2)$
и, наконец, остаться в этом состоянии с вероятностью
$a_1 b \hm+ \overline{a}_1 \overline{b}$.
При попадании в состояние $(R,2)$ система не может
попасть в состояние $(R-2,1)$ раньше, чем в
состояние $(H,1)$, причем, как нетрудно видеть,
среднее время перехода из состояния $(R,2)$ в
состояние $(H,1)$ равно $(R\hm-H)/b$.
Оставшись в состоянии $(R-1,1)$, система впервые
попадет или в состояние $(n-1,1)$, или в
состояние $(H,1)$ за среднее время $m_{R-1}$.
Поэтому
$$
m_{R-1} = 1 + a_1 \overline{b}
\fr{R-H}{b} + (a_1 b + \overline{a}_1 \overline{b}) m_{R-1} \,,
$$
или
$$
m_{R-1} = \fr{1 }{a_1 \overline{b} + \overline{a}_1 b}
\left( 1 + a_1 \overline{b} \fr{R-H}{b} \right)\,.
$$

Из состояния $(n,1)$, $n\hm=\overline{H+1,R-2}$,
так же, как и прежде, за один шаг можно попасть
с вероятностью $\overline{a}_1 b$  в состояние $(n-1,1)$,
с вероятностью $a_1 \overline{b}$~--- в состояние $(n+1,1)$
и, наконец, остаться в этом состоянии с
вероятностью $a_1 b \hm+ \overline{a}_1 \overline{b}$.
Однако, в отличие от предыдущего случая, при
попадании в состояние $(n+1,1)$ система за среднее
время $m_{n+1}$ может впервые или вернуться в
состояние $(n,1)$, или попасть в состояние $(H,1)$.
При этом она с вероятностью~$c_{n+1}$ возвратится в
состояние $(n,1)$, и тогда до момента первого попадания
или в состояние $(n-1,1)$, или в состояние $(H,1)$
пройдет дополнительно среднее время~$m_{n}$.
Таким образом,
\begin{multline*}
m_{n} = 1 + a_1 \overline{b} (m_{n+1} + c_{n+1} m_n)
+ (a_1 b + \overline{a}_1 \overline{b}) m_{n} \,,
\\ n=\overline{H+1,R-2}\,,
\end{multline*}
и
\begin{multline*}
m_{n} = \fr{1}{a_1 \overline{b} + \overline{a}_1 b - a_1 \overline{b} c_{n+1}}
\left( 1 + a_1 \overline{b} m_{n+1} \right)\,,
\\ n=\overline{H+1,R-2}\,.
\end{multline*}

Заметим теперь, что, попав из состояния
$(n,1)$, $n\hm=\overline{L,H}$, в
состояние ${(n+1,1)}$,
система за среднее время $m_{n+1}$ обязательно
возвратится в состояние $(n,1)$. Поступая, как и раньше, имеем:
$$
m_{n} = 1 + a_1 \overline{b} (m_{n+1} + m_{n}) +
(a_1 b + \overline{a}_1 \overline{b}) m_{n}\,,
\ \ n=\overline{L,H}\,,
$$
т.\,е.
$$
m_{n} = \fr{1}{a_1 \overline{b} + \overline{a}_1 b - a_1 \overline{b}}
\left( 1 + a_1 \overline{b} m_{n+1} \right)\,,
\ \ n=\overline{L,H}\,.
$$

Вычислим $\tilde{m}_{n}$, $n\hm=\overline{H+1,R-1}$. Очевидно,
$$
\tilde{m}_{H+1} = m_{H+1} \,.
$$
%%%%%%
В остальных случаях среднее время первого
достижения состояния $(H,1)$ состоит из среднего
времени первого попадания или в
состояние $(n-1,1)$, или в состояние $(H,1)$.
Кроме того, в случае попадания в состояние $(n-1,1)$
(с~вероятностью $c_n$) нужно добавить еще среднее
время $\tm_{n-1}$ первого попадания в
состояние $(H,1)$ из состояния $(n-1,1)$.
Значит,
%%%%%%%%%
$$
\tilde{m}_{n} = m_n + c_n \tilde{m}_{n-1}\,,
\ \ n=\overline{H+2,R-1}\,.
$$

Формулы для вычисления~$M_n$ и~$M^*_n$ в силу их
очевидности приводятся здесь без пояснений:
\begin{align*}
M_{L} &= m_{L} \,;
\\
M_{n} &= m_n + M_{n-1} \,,
\enskip n=\overline{L+1,H}\,;
\\
M_{n} &= \tilde{m}_{n} + M_H \,,
\enskip  n=\overline{H+1,R-1}\,;
\\
M^*_{n} &= \fr{n-H}{b}+ M_H \,,
\enskip n=\overline{H+1,R}\,.
\end{align*}


Полученные результаты позволяют получить простой
рекуррентный алгоритм вычисления средних времен~$M_n$ и~$M^*_n$,
который в силу его элементарности здесь не приводится.

\section{Некоторые стационарные характеристики}

В этом разделе найдем выражения для некоторых
важных стационарных показателей функционирования системы.
При этом будем предполагать, что заявки обоих типов
образуют общую очередь и обслуживаются в порядке поступления.

Проще всего вычисляются вероятности~$\pi_1$ и~$\pi_2$ потерь
заявок первого и второго типа. Поскольку заявка первого типа теряется только в том
случае, когда после предыдущего такта система находится в одном из состояний
$(n,2)$, $n\hm=\overline{H+1,R}$ (в системе имеется от $H\hm+1$ до~$R$~заявок и новые
заявки не принимаются), то

\noindent
$$
\pi_1 = \sum\limits_{n=H+1}^{R} p''_{n}\,.
$$
Аналогично

\noindent
$$
\pi_2 = \sum\limits_{n=L}^{R-1} p'_{n} + \sum\limits_{n=H+1}^{R} p''_{n}\,.
$$

Обратимся теперь к стационарным вероятностям
состояний по моментам поступления заявок в систему.

Введем обозначения:
\begin{description}
\item[\,]
$p^*_n$, $n\hm=\overline{0,R-1}$,~--- стационарная
вероятность того, что поступившая (принятая к
обслуживанию) заявка первого типа застанет перед
собой сразу же после поступления~$n$~других заявок (любого типа);
\item[\,]
$p^{*\prime}_n$, $n\hm=\overline{0,H-1}$,~---
стационарная вероятность того, что поступившая (принятая к
обслуживанию) заявка второго типа застанет перед
собой сразу же после поступления~$n$~других заявок (любого типа).
\end{description}

Поскольку перед поступившей заявкой первого типа
будут отсутствовать другие заявки, только если она поступает
в свободную систему (с~ве\-ро\-ят\-ностью~$p_{0}$)
или если перед ее поступлением в системе была
заявка (с~вероятностью~$p_{1}$), которая на данном
такте обслужилась (с~вероятностью~$b$), то с
учетом условия принятия заявки в систему
(вероятность $1\hm-\pi_1$) имеем:

\noindent
$$
p^*_0=\fr{1}{1 - \pi_1}(p_{0} + p_1 b)\,.
$$
%%%%%%%

Подобные рассуждения для
случая $n\hm=\overline{1,L-2}$ дают

\noindent
$$
p^*_n=\fr{1}{1 - \pi_1}
(p_{n} \overline{b} + p_{n+1} b)\,,\ \ n=\overline{1,L-2}\,.
$$

Далее, перед поступившей заявкой первого типа
будет $L\hm-1$ других заявок,
если или перед ее поступлением в системе было $L\hm-1$
заявок (с~вероятностью~$p_{L-1}$) и ни одна из них
не обслужилась (с~вероятностью~$\overline{b}$),
или если перед ее поступлением в системе было~$L$
заявок (с~вероятностью $p_{L}\hm+p^*_L$, поскольку
заявки второго типа могут как приниматься, так и не
приниматься в систему)
и одна обслужилась (с~вероятностью~$b$).
Поэтому

\noindent
$$
p^*_{L-1}=\fr{1}{1 - \pi_1}
(p_{L-1} \overline{b} + p_{L} b + p'_{L} b)\,.
$$
%%%%%%%

Незначительные изменения в рассуждениях приводят к формулам:

\noindent
\begin{gather*}
p^*_n=\fr{1}{1 - \pi_1}
(p_{n} \overline{b} + p_{n+1} b
+ p'_{n} \overline{b} + p'_{n+1} b)\,,\\
\hspace*{41mm}n=\overline{L,H-2}\,;
\\
p^*_{H-1} = \fr{1}{1 - \pi_1}
\left(p_{H-1} \overline{b} + p'_{H} b + p'_{H-1} \overline{b}\right)\,;
\\
p^*_{H}=\fr{1}{1 - \pi_1}
\left(p'_{H} \overline{b} + p'_{H+1} b+p''_{H+1} b\right)\,;
\\
p^*_n=\fr{1}{1 - \pi_1}\left(p'_{n} \overline{b} + p'_{n+1} b\right)\,,\ \
n=\overline{H+1,R-2}\,;
\\
p^*_{R-1}=\fr{1}{1 - \pi_1} p'_{R-1} \overline{b}\,.
\end{gather*}

Следующие формулы для $p^{*\prime}_n$ приводятся без пояснений:
\begin{align*}
p^{*\prime}_0 &= \fr{1}{1 - \pi_2}
\left(p_{0} + p_1 b\right)\,;
\\
p^{*\prime}_n &= \fr{1}{1 - \pi_2}
\left(p_{n} \overline{b} + p_{n+1} b\right)\,,
 \ n=\overline{1,H-2}\,,\\
 &\hspace*{50mm}n\ne L-1\,;
\\
p^{*\prime}_{L-1}&=\fr{1}{1 - \pi_2}
\left(p_{L-1} \overline{b} + p_{L} b + p'_{L} b\right)\,;
\\
p^{*\prime}_{H-1}&=\fr{1}{1 - \pi_2}\,
p_{H-1} \overline{b} \,.
\end{align*}

Поскольку распределение времени ожидания начала
обслуживания заявки, перед которой в очереди
находится $n$ других заявок, имеет распределение
Паскаля с параметрами~$b$ и~$n$, то ПФ $\omega_1(z)$
и $\omega_2(z)$ стационарных распределений времен
ожидания начала обслуживания заявок первого и
второго типа определяются выражениями:
\begin{align*}
\omega_1(z)&=\sum\limits_{n=0}^{R-1}
\left(\fr{bz}{1 - \overline{b} z}\right)^n p^*_{n}\,;
\\
\omega_2(z)&=\sum\limits_{n=0}^{H-1}
\left(\fr{bz}{1 - \overline{b} z}\right)^n p^{*\prime}_{n}\,.
\end{align*}
В частности, стационарные средние времена~$w_1$ и~$w_2$
ожидания начала обслуживания заявок первого и второго типа имеют вид:
\begin{align*}
w_1=\omega'_1(1)&=
\fr{1}{b} \sum\limits_{n=0}^{R-1} n  p^*_{n}=
\fr{1}{b}\, N_1^* \,;
\\
\omega_2(z)&= \omega'_2(1) =
\fr{1}{b}\sum\limits_{n=0}^{H-1} n p^{*\prime}_{n}
=\fr{1}{b}\, N_2^* \,,
\end{align*}
где $N_1^*$ и $N_2^*$~--- стационарные средние
числа заявок,
 которых застают в очереди
поступающие в систему заявки первого и второго типа.

\begin{figure*}[b] %fig2
\vspace*{1pt}
\begin{center}
\mbox{%
\epsfxsize=162.499mm
\epsfbox{pec-2.eps}
}
\end{center}
\vspace*{-12pt}
\Caption{Поведение вероятностей $p_n$~(\textit{1}),
$p'_n$~(\textit{2}) и~$p''_n$~(\textit{3}) как функций от~$n$
($b\hm=0{,}2$): (\textit{а})~$a_1\hm=0{,}21$; $a_2\hm=0{,}09$;
(\textit{б})~$a_1\hm=0{,}15$; $a_2\hm=0{,}15$}
%\end{figure*}
%\begin{figure*} %fig3-4
\vspace*{9pt}
\begin{center}
\mbox{%
\epsfxsize=161.119mm
\epsfbox{pec-4.eps}
}
\end{center}
\vspace*{-12pt}
\begin{minipage}[t]{80mm}
\Caption{Распределение времени (числа тактов $i$)
  до первого достижения состояния $(L-1)$ из
  состояния~$H$ (${b\hm=0{,}2}$,
  ${\rho\hm=1{,}5}$, ${a_2\hm=0{,}3\hm-a_1}$):
  \textit{1}~--- $a_1\hm=0{,}21$; \textit{2}~---
  0,15; \textit{3}~--- $a_1\hm=0{,}09$}
  \end{minipage}
  \hfill
  \vspace*{-12pt}
  \begin{minipage}[t]{80mm}
  \Caption{Среднее время (число тактов) до первого
достижения состояния $(L-1)$ из состояния~$H$ в
зависимости от загрузки системы~$\rho$ ($b\hm=0{,}2$):
\textit{1}~--- $a_1\hm=2{,}3a_2$; \textit{2}~--- $a_1\hm=a_2$; \textit{3}~--- $a_1
\hm\approx 0{,}4a_2$}
\end{minipage}
\vspace*{6pt}
\end{figure*}




Наконец, используя формулу Литтла (см., например,~[11]), получаем
выражения для стационарных средних~$N_1$ и~$N_2$
чисел заявок первого и второго типа в очереди
\begin{align*}
N_1&=a_1 (1 - \pi_1) w_1\,;
\\
N_2&=a_2 (1 - \pi_2) w_2\,.
\end{align*}


\vspace*{-12pt}

\section{Примеры расчетов}

На основе полученных результатов
была написана программа, позволяющая вычислять
распределения времен выхода из множества состояний,
совместное стационарное распределение
числа заявок в системе и состояния системы и
связанные с ним характеристики, а также исследовать
поведение рассматриваемой СМО в
зависимости от значений определяющих ее исходных параметров.

Приведем лишь некоторые из результатов расчетов.
Всюду в дальнейшем предполагается,
что ${R\hm=100}$, ${H\hm=70}$, ${L\hm=40}$,
а вероятность обслуживания заявки на такте ${b\hm=0{,}2}$.
Загрузка системы обозначается через
${\rho\hm=(a_1+a_2)/b}$.

Так как простое выписывание всех совместных
стационарных вероятностей
состояний занимает очень много места,
представим их графически (рис.~2).
На рис.~2,\,\textit{а}, при загрузке $\rho\hm=1{,}5$
и вероятности поступления заявки первого типа,
равной ${a_1\hm=0{,}21}$, изображено поведение
вероятностей~$p_n$, $p'_n$ и~$p''_n$ как функций
от числа заявок в системе~$n$.



На рис.~2,\,\textit{б} изображено поведение вероятностей
$p_n$, $p'_n$ и~$p''_n$ как функций от~$n$ при
загрузке ${\rho\hm=1{,}5}$  и вероятности поступления
заявки первого типа, равной ${a_1\hm=0{,}15}$.





С точки зрения практического применения
подобных систем к моделированию SIP-сер\-ве\-ров
отдельный интерес представляют временн$\acute{\mbox{ы}}$е характеристики работы системы.
Они позволяют выявлять эффективность работы
гистерезисного управления и подлежат оптимизации.
Как упоминается в~\cite{8-p}, одной из таких
характеристик является время выхода системы из режима перегрузки.
Моментом входа в режим перегрузки можно считать
момент, когда число заявок в системе
впервые стало равным~$H$.
Поэтому обратимся к
распределению времени (числа тактов~$i$) до
первого достижения состояния $(L\hm-1)$
из состояния~$H$ и его среднему значению
(т.\,е.\ $t^*_{i,H}$, ${i\hm\ge0}$, и $M_H$ в
обозначениях разд.~5 и~6 соответственно).
На рис.~3 представлено распределение времени
(числа тактов~$i$) достижения
состояния $(L-1)$ из $H$ при загрузке ${\rho\hm=1{,}5}$
для различных вероятностей поступления заявки
первого типа
($a_1\hm=0{,}21$, 0,15 и 0,09).



Наконец, на рис.~4 представлено поведение
среднего времени (числа тактов) до первого достижения
состояния $(L-1)$ из состояния~$H$ в зависимости от
загрузки системы $\rho$.
Рассмотрено 3~соотношения для потоков первого и
второго типа, а именно: рассмотрен случай, когда заявки первого
типа поступают в систему чаще, чем заявки второго
типа (${a_1 \hm\approx 2{,}3\, a_2}$),
случай, когда в среднем в систему поступает
одинаковое число заявок обоих типов (${a_1 \hm= a_2}$),
и случай, когда заявки второго поступают в систему
чаще, чем заявки первого типа (${a_1 \hm\approx 0{,}4\, a_2}$).



Полученные аналитические результаты были проверены
путем сравнения с результатами работы имитационной
модели, разработанной на языке GPSS (General Purpose Simulation System).
Сравнения показали высокую точность расчетов,
проведенных на основе аналитических соотношений.

\vspace*{-9pt}


{\small\frenchspacing
{%\baselineskip=10.8pt
\addcontentsline{toc}{section}{References}
\begin{thebibliography}{99}



\bibitem{10-p} %1
\Au{Gebhart R.\,F.}
A~queuing process with bilevel hysteretic service-rate control~//
Nav. Res. Logist. Q., 1967. Vol.~14. No.\,1. P.~55--67.
\columnbreak

\bibitem{14-p} %2
\Au{Golubchik L., Lui~J.\,C.\,S.}
Bounding of performance measures for a
threshold-based queueing system with hysteresis~//
Newsl. ACM SIGMETRICS Performance Evaluation Rev., 1997.
Vol.~25. No.\,1. P.~147--157.



\bibitem{19-p} %3
\Au{Dshalalow J.\,H.} Queueing systems with state dependent
parameters~// Frontiers in queueing: Models and applications in
science and engineering.~--- Boca Raton: CRC Press, 1997.
P.~61--116.

\bibitem{21-p} %4
\Au{Roughan M., Pearce~C.} A~martingale analysis of hysteretic
overload control~// Adv. Performance Anal. J.~Teletraffic Theory
Performance Anal. Comm. Syst.
Networks, 2000. Vol.~3. No.\,1. P.~1--30.

\bibitem{20-p} %5
\Au{Bekker~R.}
Queues with Levy input and hysteretic control~//
Queueing Syst., 2009. Vol.~63. No.\,1. P.~281--299.

\bibitem{8-p} %6
\Au{Abaev P.\,O., Gaidamaka~Yu.\,V., Pechinkin~A.\,V.,
Razumchik~R.\,V., Shorgin~S.\,Ya.}
Simulation of overload control in SIP Server Networks~//
26th European Conference on Modelling and
Simulation (ECMS 2012) Proceedings.~--- Koblenz: Digitaldruck Pirrot GmbH, 2012.
P.~533--539.

\bibitem{16-p} %7
\Au{Жерновый К.\,Ю., Жерновый~Ю.\,В.}
Система $M^\theta/G/1$ c гистерезисным переключением интенсивности обслуживания~//
Информационные процессы, 2012. Т.~12. №\,3. С.~176--190.



\bibitem{26-p} %8
\Au{Abaev P., Pechinkin~A., Razumchik~R.}
Analysis of queueing system with constant service
time for SIP server hop-by-hop overload control~//
Modern Probab. Meth. Anal. Telecommunication Networks
Comm. Comp. Information Sci., 2013. Vol.~356.
P.~1--10. doi:10.1007/978-3-642-35980-4\_1.

\bibitem{30-p} %9
\Au{Shorgin S., Samouylov~K., Gaidamaka~Yu., Etezov~Sh.} Polling
system with threshold control for modeling of SIP server under
overload~// 18th Conference (International) on Systems Science (ICSS
2013) Proceedings. Advances in intelligent systems and computing
ser., 2014. Vol.~240. P.~97--107.



\bibitem{38-p} %10
\Au{Разумчик Р.\,В., Абаев~П.\,О., Корабельников~Д.\,М., Пяткина~Д.\,А.}
Моделирование SIP-сер\-ве\-ра с гистерезисным управлением нагрузкой с помощью системы
массового обслуживания в дискретном времени~//
Научные труды ФГУП ЦНИИС: Сб. статей.~--- М.: ЦНИИС, 2011. C.~119--127.

\bibitem{36-p} %11
\Au{Bocharov P.\,P., D'Apice~C., Pechinkin~A.\,V., Salerno~S.}
Queueing theory.~--- Utrecht, Boston: VSP, 2004.


\end{thebibliography}
} }

\end{multicols}

\vspace*{-6pt}

\hfill{\small\textit{Поступила в редакцию 2.04.14}}

%\newpage


\vspace*{6pt}

\hrule

\vspace*{2pt}

\hrule

\vspace*{-2pt}


\def\tit{PERFORMANCE CHARACTERISTICS OF~Geo/Geo/1/$R$
QUEUE WITH~HYSTERETIC LOAD CONTROL}

\def\titkol{Performance characteristics of~Geo/Geo/1/$R$
queue with hysteretic load control}

\def\aut{A.\,V.~Pechinkin$^{1,2}$ and~R.\,V.~Razumchik$^1$}
\def\autkol{A.\,V.~Pechinkin and~R.\,V.~Razumchik}


\titel{\tit}{\aut}{\autkol}{\titkol}

\vspace*{-9pt}

\noindent
$^1$Institute of Informatics
Problems, Russian Academy of Sciences, 44-2 Vavilov Str.,
Moscow 119333, Russian\\
$\hphantom{^1}$Federation

\noindent
$^2$Peoples' Friendship University of Russia, 6 Miklukho-Maklaya Str.,
Moscow 117198, Russian Federation



\def\leftfootline{\small{\textbf{\thepage}
\hfill INFORMATIKA I EE PRIMENENIYA~--- INFORMATICS AND APPLICATIONS\ \ \ 2014\ \ \ volume~8\ \ \ issue\ 2}
}%
 \def\rightfootline{\small{INFORMATIKA I EE PRIMENENIYA~--- INFORMATICS AND APPLICATIONS\ \ \ 2014\ \ \ volume~8\ \ \ issue\ 2
\hfill \textbf{\thepage}}}

\vspace*{6pt}




\Abste{Threshold load control is one of the key techniques to
prevent overloads in telecommunication networks.
Its variants are used for overload detection
in signalling system No.\,7 as well as in next generation
networks where\linebreak\vspace*{-12pt}}

 \Abstend{session initiation protocol (SIP) is the main signalling protocol.
Consideration is given to discrete-time Geo/Geo/1/$R$
queueing system which is one of the possible mathematical models of the SIP
proxy-server. Bi-level hysteretic load control is implemented in the system.
The methods that allow one to obtain a stationary
joint probability distribution of the number
of customers in a system and system's state,
stationary waiting and sojourn time distributions,
and distribution of first passage times
from different system's states are presented.
A~numerical example based on obtained analytical
expressions is given.}

\KWE{queueing system; discrete time; hysteretic load control; performance
characteristics}

\DOI{10.14357/19922264140202}

\Ack
\noindent
The research was supported by the Russian Foundation for Basic Research
(projects
Nos.\,12-07-00108, 13-07-00223, and 13-07-00665).


  \begin{multicols}{2}

\renewcommand{\bibname}{\protect\rmfamily References}
%\renewcommand{\bibname}{\large\protect\rm References}

{\small\frenchspacing
{%\baselineskip=10.8pt
\addcontentsline{toc}{section}{References}
\begin{thebibliography}{99}




\bibitem{2-p-1} %1
\Aue{Gebhart, R.\,F.} 1967.
A~queuing process with bilevel hysteretic
service-rate control. \textit{Nav. Res. Logist. Q.} 14(1):55--68.

\bibitem{3-p-1} %2
\Aue{Golubchik,~L., and C.\,S.~Lui}. 1997.
Bounding of performance measures for a
threshold-based queueing system with hysteresis.
\textit{Newsl. ACM SIGMETRICS Performance Evaluation Rev.}
25(1):147--157.

\bibitem{7-p-1} %3
\Aue{Dshalalow, J.\,H.} 1997.
Queueing systems with state dependent parameters.
\textit{Frontiers in queueing: Models and applications in
science and engineering}. Boca Raton: CRC Press. 61--116.

\bibitem{9-p-1} %4
\Aue{Roughan, M., and C.~Pearce}. 2000.
A martingale analysis of hysteretic overload
control. \textit{Adv. Performance Anal. J.~Teletraffic Theory
Performance Anal.
Comm. Syst. Networks} 3:1--30.

\bibitem{8-p-1} %5
\Aue{Bekker, R.} 2009
Queues with Levy input and hysteretic control.
\textit{Queueing Syst.} 63(1):281--299.


\bibitem{1-p-1} %6
\Aue{Abaev, P.\,O., Yu.\,V.~Gaidamaka, A.\,V.~Pechinkin, R.\,V.~Razumchik,
and S.\,Ya.~Shorgin}. 2012.
Simulation of overload control in SIP Server
Networks. \textit{26th European Conference on
Modelling and Simulation Proceedings}.
Koblenz: Digitaldruck Pirrot GmbH. 533--539.

\bibitem{6-p-1} %7
\Aue{Zhernovyi, K., and Yu.~Zhernovyi}. 2012.
Sistema $M^\theta/G/1$ s gisterezisnym
pereklyucheniem intensiv\-nosti obsluzhivaniya
[The system $M^\theta/G/1$  with a hysteretic switching intensity of service].
\textit{Informatsionnye Protsessy} [Information Processes]
12(3):176--190.

\bibitem{4-p-1} %8
\Aue{Abaev, P., A.~Pechinkin, and R.~Razumchik}. 2013.
Analysis of queueing system with constant service
time for SIP server hop-by-hop overload control. \textit{Modern
Probab. Meth. Anal. Telecommunication Networks
Comm. Comp. Information Sci.}
356:1--10. doi:10.1007/978-3-642-35980-4\_1.

\bibitem{5-p-1} %9
\Aue{Shorgin, S., K.~Samouylov, Yu.~Gaidamaka, and Sh.~Etezov}. 2013.
Polling system with threshold control for modeling of
SIP server under overload. \textit{18th  Conference (International) on
Systems Science Proceedings}. Wroclaw. 240:97--107.
doi:10.1007/978-3-319-01857-7\_10.



\bibitem{11-p-1} %10
\Aue{Razumchik, R.\,V., P.\,O.~Abaev, D.\,M.~Korabelnikov, and D.\,A.~Pyatkina}. 2012.
Modelirovanie servera s giste\-re\-zis\-nym upravleniem
nagruzkoy s pomoshch'yu sistemy massovogo
obsluzhivaniya v diskretnom vremeni [Modeling of
SIP-server with hysteric overload control as
discrete time queueing system]. \textit{Nauchnie Trudy ZNIIS}
[Proceedings of ZNIIS]. 119--127.

\bibitem{10-p-1}
\Aue{Bocharov, P.\,P., C.~D'Apice, A.\,V.~Pechinkin, and S.~Salerno}. 2004.
\textit{Queueing theory}. Utrecht, Boston: VSP. 446~p.

\end{thebibliography}
} }


\end{multicols}

\vspace*{-6pt}

\hfill{\small\textit{Received April 2, 2014}}

\vspace*{-18pt}





\Contr

\noindent
\textbf{Pechinkin Alexander V.} (b.\ 1946)~--- Doctor
of Science in physics and mathematics; principal
scientist, Institute of Informatics Problems,
Russian Academy of Sciences, 44-2 Vavilov Str.,
Moscow 119333, Russian Federation; professor,
Peoples' Friendship University of Russia, 6 Miklukho-Maklaya Str.,
Moscow 117198, Russian Federation; apechinkin@ipiran.ru

\vspace*{3pt}

\noindent
\textbf{Razumchik Rostislav V.} (b.\ 1984)~--- Candidate
of Science (PhD) in physics and mathematics,
senior research fellow, Institute of Informatics
Problems, Russian Academy of Sciences, 44-2 Vavilov Str.,
Moscow 119333, Russian Federation; rrazumchik@ieee.org

 \label{end\stat}

\renewcommand{\bibname}{\protect\rm Литература} 