\def\stat{spivak}

\def\tit{ПОСТРОЕНИЕ МОДЕЛЕЙ СИСТЕМНОЙ ДИНАМИКИ
В~УСЛОВИЯХ ОГРАНИЧЕННОЙ ЭКСПЕРТНОЙ ИНФОРМАЦИИ$^*$}

\def\titkol{Построение моделей системной динамики в условиях ограниченной
экспертной информации}

\def\aut{О.\,Г.~Кантор$^1$, С.\,И.~Спивак$^2$}

\def\autkol{О.\,Г.~Кантор, С.\,И.~Спивак}

\titel{\tit}{\aut}{\autkol}{\titkol}

{\renewcommand{\thefootnote}{\fnsymbol{footnote}}
\footnotetext[1]{Работа выполнена при финансовой поддержке РФФИ (проект
№\,13-01-00749).}

\renewcommand{\thefootnote}{\arabic{footnote}}
\footnotetext[1]{Институт социально-экономических исследований Уфимского
научного центра Российской академии наук, o\_kantor@mail.ru }
\footnotetext[2]{Башкирский государственный университет,
semen.spivak@mail.ru}


\Abst{Системная динамика~--- методология изучения сложных динамических
систем, ориентированная на проведение компьютерного эксперимента.
Построение моделей системной динамики во многом зависит от имеющейся
экспериментальной информации и квалификации экспертов. Имитационный
эксперимент с <<плохими>> моделями может привести к существенному или
даже полному искажению свойств изучаемой системы. В~настоящей работе
приводится описание метода построения моделей системной динамики,
представляющего собой комплекс математических моделей, в основу которых
положены идеи подхода Л.\,В.~Канторовича к математической обработке
экспериментальных данных, и вычислительных процедур. Важным
преимуществом при этом является возможность включения в модель значимых
с точки зрения исследователя условий, влияющих на ее адекватность.
Апробация разработанного метода осуществлялась на примере моделирования
численности населения Российской Федерации.}

\KW{модели системной динамики; точечные и интервальные оценки
параметров моделей; подход Л.\,В.~Канторовича; предельно допустимые
погрешности измерений}

\DOI{10.14357/19922264140211}



\vskip 10pt plus 9pt minus 6pt

\thispagestyle{headings}

\begin{multicols}{2}

\label{st\stat}


\section{Введение} %1

  Основу концепции системной динамики со\-став\-ляет представление
функционирования изуча\-емой системы в виде совокупности потоков ресурсов,
определяющих ее. При этом подразумевается достаточно высокая степень
агрегирования, в результате чего рассмотрению подлежат лишь наиболее
значимые факторы~[1--4].

  Основой для построения уравнений системной динамики служат
дифференциальные модели. При построении математических моделей
системной динамики используются переменные двух типов: системные уровни
и темпы. Системные уровни полностью описывают состояние системы в
произвольный момент времени. Изменение системных уровней вызвано
соответствующими темпами, которые, в свою очередь, зависят от одного или
нескольких системных уровней (но не от других темпов). В~некоторых случаях
в целях более детального отражения процессов, протекающих в изучаемой
системе, и/или более удобной формы записи уравнений модели системной
динамики могут использоваться вспомогательные переменные.

  В~моделях системной динамики для всех системных уровней пишутся
уравнения одного и того же типа~\cite{spi-4, spi-1}:

\noindent
\begin{equation}
\fr{d\bar{x}}{dt} = f \left( \bar{x}, \bar{a}\right) = \bar{x}^+ - \bar{x}^-\,,
\label{spi-e1}
\end{equation}
где $\left( \bar{x}, \bar{a}\right)$--- век\-тор-функ\-ция, зависящая от переменных
$\bar{x}$ и параметров $\bar{a}$ модели; $\bar{x}^+$ и $\bar{x}^-$~---
положительный и отрицательный темпы скорости системных уровней
$\bar{x}$, каждый из которых включает в себя все факторы, вызывающие
соответственно рост и убывание $\bar{x}$.

  Аналитическое решение систем дифференциальных уравнений~(\ref{spi-e1})
с учетом размерности реальных задач представляет собой практически
неразрешимую задачу. Поэтому традиционным является переход от
дифференциальных уравнений~(\ref{spi-e1}) к их разностным аналогам:
\begin{equation}
\Delta \bar{x} = f \left( \bar{x}, \bar{a}\right)\,,
\label{spi-e2}
\end{equation}
для численного интегрирования которых разработано большое количество
методов. (Далее для определенности будем говорить только о
модели~(\ref{spi-e2}), подразумевая, что если ее параметры известны, то и
модель~( \ref{spi-e1}) также определена.)

  Следует заметить, что темпы в случае использования модели~( \ref{spi-e1})
показывают закон изменения соответствующих системных уровней, а в случае
их разностных аналогов~(\ref{spi-e2})~--- каким образом изменяются
соответствующие системные уровни за временной интервал, равный шагу
моделирования, выбор которого во многом зависит от имеющейся
экспериментальной информации, получаемой из фактических наблюдений за
исследуемой системой.

\noindent
\begin{center}  %fig1
\mbox{%
\epsfxsize=68.266mm
\epsfbox{spi-1.eps}
}
  \end{center}

  \vspace*{6pt}

\noindent
{{\figurename~1}\ \ \small{Потоковая диаграмма модели системной динамики общего вида:
\textit{1}~--- системные уровни;
\textit{2}~--- системные темпы;
\textit{3}~--- неучтенные факторы}}


\vspace*{9pt}

\addtocounter{figure}{1}



  В~некоторых случаях уравнения системной динамики составляются на основе
очевидных логических связей между системными уровнями и темпами.
Ключевая роль при этом отводится экспертам, опыт и знания которых
позволяют полагаться на их мнения и оценки. В~более сложных ситуациях,
например, когда ни исследователю, ни экспертам до конца не ясно, каким
образом выбранные для анализа системные уровни и темпы взаимодействуют
друг с другом, определение точного вида зависимостей~(\ref{spi-e2})
представляет собой самостоятельную задачу.

  Одним из способов наглядного отображения процессов, протекающих в
изучаемой системе, являются потоковые диаграммы (рис.~1), которые
обеспечивают целостное представление структуры уравнений~(\ref{spi-e2}),
включая отображение при\-чин\-но-след\-ст\-вен\-ных связей и петель обратных
связей. Потоковые диаграммы, по сути, представляют собой инструмент
системного анализа проблемы, применение которого способствует ее
детальному пониманию и в ряде случаев позволяет осуществить декомпозицию
задачи на несколько самостоятельных задач меньшей размерности, тем самым
снижая общую трудоемкость процесса получения ре\-шения.
{ %\looseness=1

}

  Отсутствие априори известных закономерностей является характерной
особенностью задач,\linebreak
 связанных с моделированием со\-ци\-аль\-но-эко\-но\-мических
систем, при исследовании которых необходимо учитывать ряд особенностей,
наиболее существен\-ные из которых~--- ограниченность имеющейся
информации и ее неточность. Ограниченность информации обусловливается
изменчивостью самих со\-ци\-аль\-но-эко\-но\-ми\-че\-ских систем, в процессе
развития которых может возникать или исчезать необходимость сбора той или
иной статистической информации, при этом сам по себе процесс сбора и
обработки информации является достаточно длительным и трудоемким.
Неточность исходной информации объясняется многостадийностью процесса
ее сбора, существенной долей\linebreak субъективизма, а в некоторых случаях и
умышлен\-ным искажением информации с целью <<приукрасить>> реальное
положение дел. При изучении  со\-ци\-аль\-но-эко\-но\-ми\-че\-ских
систем не
пред\-став\-ля\-ет\-ся\linebreak возможным оценить точность полученных наблюдений с
помощью применения стандартных подходов, основанных на сравнении
данных с эталон\-ными значениями, известными изначально или получен\-ными
на основе многократного проведения наблюдений в одних и тех же условиях,
по причине их отсутствия или невозможности организации соответствующего
эксперимента. В~таких условиях едва ли разумно полностью полагаться на
экспертные оценки, если даже таковые и будут получены.



  Аналогичные проблемы могут возникать и не только при изучении
 со\-ци\-аль\-но-эко\-но\-ми\-че\-ских систем. Полагаясь на мнение эксперта,
исследователь всегда рискует, в силу того что даже самый квалифицированный
эксперт может оказаться неправым и все усилия исследователя по построению
математической модели и дальнейшие эксперименты с ней окажутся
напрасными.

\section{Постановка задачи} %2

  В~условиях ограниченности экспертных представлений об исследуемой
системе исследователь может обладать информацией лишь самого общего
характера, что не позволяет ему идентифицировать зависимости~(\ref{spi-e2}).
Возникающую на этапе по\-стро\-ения уравнений системной динамики
неопределенность целесообразно подразделять на два типа:
  \begin{enumerate}[(1)]
\item \textit{неопределенность первого типа}, характери\-зу\-ющу\-юся отсутствием
информации о значениях параметров зависимостей заданной функциональной
структуры, что соответствует ситуа\-ции, когда известны закономерности,
связывающие системные уровни и системные темпы, но параметры,
фигурирующие в них, подлежат определению;
\item \textit{неопределенность второго типа}, при которой неизвестна сама
функциональная структура связи между системными уровнями и тем\-пами.
\end{enumerate}

  Неопределенность второго типа характерна для ситуаций, когда
исследователь либо не доверяет мнению экспертов, либо в принципе не
прибегает к их опросу, а его собственных знаний об изучаемом объекте
недостаточно для того, чтобы самостоятельно определить функциональные
связи и соответствующие им параметры.

  Очевидно, что <<раскрывать>> неопределенность второго типа можно
посредством последовательного рассмотрения приемлемых с точки зрения
исследователя структур для уравнений~(\ref{spi-e2}). При этом каждый такой
отдельный случай порождает неопределенность первого типа, что и
обусловливает актуальность методов определения значений параметров
моделей системной динамики в соответствующих условиях.

  Любая задача определения значений параметров моделей заданной
функциональной структуры, по сути, сводится к поиску такого набора
па\-ра\-мет\-ров~$\bar{a}$, которые обеспечивали бы приемлемые с точки зрения
исследователя качественные характеристики. Выбор конкретного
инструментария для определения совокупности параметров модели зависит
лишь от объема знаний и предпочтений самого исследователя. Большое
значение при этом имеет цель, которую ставит перед собой исследователь. Так,
если целью является максимально точное соответствие имеющимся
наблюдениям, то, скорее всего, определяющим критерием будет выступать
близость расчетных и экспериментальных данных; если же целью исследования
ставится адекватность модели каким-либо свойствам реального объекта, то
логично в качестве определяющего критерия выбрать степень соответствия
модели именно таким свойствам. Следует особо отметить, что критерий,
обеспечивающий достижение лучшего значения по одной из качественных
характеристик, вовсе не обязательно будет обеспечивать хорошее значение по
другой характеристике.

  Определение параметров модели~(\ref{spi-e2}) на основании имеющейся
экспериментальной информации сопряжено с двумя существенными
проблемами:
  \begin{enumerate}[(1)]
  \item число наблюдений $n$ в практических задачах превышает (а чаще~---
существенно превышает) число параметров, подлежащих определению,
поэтому решаемые задачи априори являются \textit{переопределенными};
 \item такого рода задачи изначально являются \textit{некорректными}, так как
приближенный характер исходной информации влечет невыполнение
требований, предъявляемых к корректно поставленным задачам
(существование точного решения, его единственность и устойчивость к малым
изменениям исходных данных)~\cite{spi-5}.
\end{enumerate}

  Перечисленные проблемы ограничивают, а иногда и вовсе исключают
возможность применения классических методов, таких как, например, методы
статистического анализа. Если же исследователь будет обладать информацией
о диапазонах вариации каждого из параметров~$\bar{a}$, то определение
непосредственного вида модели~(\ref{spi-e2}) может быть осуществлено
посредством реализации специального численного эксперимента, в ходе
которого из множества допустимых значений параметров~$\bar{a}$ будет
выбран единственный набор~$\bar{a}^*$, доставляющий оптимальное значение
некоторому критерию, отражающему, по мнению исследователя, соответствие
расчетных и экспериментальных данных.

В~некоторых случаях диапазоны
значений па\-ра\-мет\-ров модели могут быть заданы исходя из их\linebreak смысловой
нагрузки, однако в общем случае их определение является самостоятельной
задачей. Решение этой задачи <<на глазок>> может при\-вести либо к слишком
большим диапазонам вариации\linebreak параметров модели, что потребует при
организации численного эксперимента по определению
параметров~$\bar{a}^*$ существенных временн$\acute{\mbox{ы}}$х затрат и повышен\-ных
требований к производительности вычислительной техники, либо к ситуации,
когда в заданном множестве по результатам численного эксперимента
оптимальный набор параметров~$\bar{a}^0$ попросту не будет существовать.

  Исследование в настоящей работе посвящено разработке метода определения
значений па\-ра\-мет\-ров моделей системной динамики в условиях ограниченности
информации относительно значений параметров искомых зависимостей.

\vspace*{-6pt}

\section{Описание метода определения диапазона вариации параметров
моделей системной динамики} %3

\vspace*{-2pt}


  Для решения задачи определения диапазона вариации параметров моделей
системной динамики авторы разработали метод, базирующийся на подходе,
основоположником которого является Л.\,В.~Канторович, впервые
высказавший идеи получения точных двусторонних границ для параметров
моделей и областей расположения искомых и наблюдаемых величин~\cite{spi-6}.
 (Дальнейшее описание предлагаемого метода осуществляется на примере
построения зависимости для одного отдельно взятого системного уровня, для
которого необходимо определить точный вид функциональной зависимости
${dx}/{dt} = f \left( x, \bar{a}\right) = x^+ - x^-$ и, соответственно, ее
разностного аналога $\Delta x = f \left( x, \bar{a}\right)$.)

  Традиционно проверка соответствия расчетных и экспериментальных
данных осуществляется посредством введения в рассмотрение величин
отклонений:
\begin{multline*}
\eta_j = \left. \Delta x^{\mathrm{расч}}\right\vert_j - \left. \Delta x^{\mathrm{эксп}}
\right\vert_j ={}\\
{}= f \left( \left. x^{\mathrm{расч}}\right\vert_j , \bar{a}\right) - \left.
\Delta x^{\mathrm{эксп}}\right\vert_j \,, \quad j = \overline{1, n}\,,
%\label{spi-e3}
\end{multline*}
где $\left. \Delta x^{\mathrm{эксп}}\right\vert_j $~--- известное из наблюдений
изменение переменной модели в $j$-й момент времени;
$\left.x^{\mathrm{расч}}\right\vert_j $~--- рассчитанное согласно модели значение
переменной~$x$ в $j$-й момент времени; $n$~--- общее число имеющихся
наблюдений.

  Стандартный путь решения задачи определения значений параметров
модели~(\ref{spi-e2}) заключается в минимизации отклонений $\left\{ \eta_j, j =
\overline{1, n}\right\}$ в смысле некоторого введенного критерия. В~рамках
математической статистики дается обоснование вида такого критерия в случае
известного закона распределения погрешности измерений. Так, если
погрешности измерений подчиняются нормальному закону распределения или
распределению Лапласа, то критерий соответственно принимает вид
\begin{equation}
\sum\limits_{j =1}^n \fr{1}{\sigma_j^2}\, \eta_j^2\,,
\label{spi-e4}
\end{equation}
или
\begin{equation*}
\sum\limits_{j =1}^n \fr{1}{\sigma_j}\, \left\vert\eta_j\right\vert\,,
%\label{spi-e5}
\end{equation*}
где $\sigma_j^2$~---  дисперсия измерений, $j = \overline{1, n}$.

В~реальных системах, как правило, информация о законе распределения
погрешности измерений отсутствует, в то время как доступной является
информация о предельно допустимой погрешности измерений (именно этот
факт и был взят за основу Канторовичем в работе~\cite{spi-6}). Условие
того, что модель описывает наблюдаемые величины, приводит к системе
неравенств
\begin{equation}
\hspace*{-2mm}\left\vert \eta_j\right\vert  = \left\vert f \left(\left. x^{\mathrm{расч}}\right\vert_j,
\bar{a} \right) - \left. \Delta x^{\mathrm{эксп}}\right\vert_j\right\vert \leq
\varepsilon_j \,, \  j = \overline{1, n}\,,\!\!
\label{spi-e6}
\end{equation}
где $\varepsilon_j$~--- погрешность $j$-го измерения, численное решение
которой предполагает использование в качестве начального приближения хотя
бы одной точки, обеспечивающей справедливость всех
соотношений~(\ref{spi-e6}).

  Основной принцип интервального оценивания в свете подхода
Канторовича состоит в том, что величины отклонений~$\eta_j$ должны
находиться в пределах погрешностей измерений~$\varepsilon_j$. Описывая
предлагаемый им подход к обработке наблюдений~\cite{spi-6},
Канторович считал, что исследователь должен располагать информацией
о величине предельно допустимой погрешности~$\varepsilon_j$. Однако далеко
не всегда это является возможным (например, при исследовании
со\-ци\-аль\-но-эко\-но\-ми\-че\-ских систем в силу упомянутых выше особенностей). Более того,
даже в тех случаях, когда известны величины погрешностей
измерений~$\varepsilon_j$, система~(\ref{spi-e6}) может оказаться
несовместной.

  В~этой связи, по мнению авторов, целесообразно величины~$\varepsilon_j$
рассматривать как неизвестные, что позволит осуществлять поиск точки,
гаранти\-ру\-ющей справедливость всех соотношений~(\ref{spi-e6}), исходя из
условия обеспечения оптимума любого критерия, характеризующего
соответствие расчетных и экспериментальных данных. В~качестве такого
критерия авторы применили функцию
  \begin{equation}
  \max\limits_{1\leq j \leq n} \left\vert \eta_j\right\vert \,,
  \label{spi-e7}
  \end{equation}
используемую в методе выравнивания по Чебы\-шёву. Основная идея метода
выравнивания по\linebreak
 Чебышёву заключается в приближении экспериментальных
данных таким способом, чтобы обеспечивалась равномерная точность описания
во всей исследуемой области, а из всей экспериментальной информации
фактически используется $k+1$ точка, где $k$~--- число искомых параметров.
Очевидно, что оптимальные параметры модели~(\ref{spi-e2}) должны
минимизировать норму~(\ref{spi-e7}), а это эквивалентно решению задачи
  \begin{equation}
  \min\limits_{\Omega}\max\limits_{1\leq j \leq n} \left\vert \eta_j\right\vert \,,
  \label{spi-e8}
  \end{equation}
где $\Omega$~--- множество допустимых значений искомых
параметров~$\bar{a}$, в качестве которого в первом приближении может быть
взят многомерный параллелепипед произвольного размера.

  Задача~(\ref{spi-e8}) сводится~\cite{spi-5} к решению оптимизационной
задачи с помощью введения дополнительного параметра~$\lambda$, такого что
\begin{equation*}
\left\vert \eta_j \right\vert \leq \lambda\,, \quad j = \overline{1, n}\,.
%\label{spi-e9}
\end{equation*}

  В~результате задача~(\ref{spi-e8}) может быть формализована следующим
образом:
\begin{equation}
\hspace*{-3mm}\left.
\begin{array}{l}
\lambda \rightarrow \min\limits_{\Omega, \lambda}\,;\\[9pt]
-\lambda \leq f \left(\left. x^{\mathrm{расч}}\right\vert_j , \bar{a}\right) -
\left. \Delta x^{\mathrm{эксп}}\right\vert_j \leq \lambda \,, \ j = \overline{1, n}\,; \\[9pt]
\lambda \geq 0\,.
\end{array}
\!\right\}\!\!
\label{spi-e10}
\end{equation}
  Данная задача, в отличие от задачи~(\ref{spi-e4}), является совместной
всегда. Ее решением будет набор точечных оценок параметров
модели~(\ref{spi-e2}) $\bar{a}^*$  и значение~$\lambda^*$.

  Определение диапазонов вариации па\-ра\-мет\-ров~$\bar{a}$ может быть
осуществлено посредством решения оптимизационных задач вида

\noindent
\begin{equation}
\left.
\hspace*{-1mm}\begin{array}{l}
a_i \rightarrow \min\limits_{\Omega}(\max\limits_{\Omega})\,;\\[9pt]
\left\vert f \left(\left. x^{\mathrm{расч}}\right\vert_j , \bar{a}\right) -
\left. \Delta x^{\mathrm{эксп}}\right\vert_j \right\vert \leq \lambda^*  \,,
 \enskip j = \overline{1, n}\,,
\end{array}
\!\right\}\!
\label{spi-e11}
\end{equation}
для каждого отдельно взятого компонента $a_i$ вектора $\bar{a}$. При этом в
целях получения большей информации о диапазонах значений параметров
модели~(\ref{spi-e2}) правые части системы ограничений могут варьироваться,
т.\,е.\ вместо модели~(\ref{spi-e11}) могут рас\-смат\-ри\-вать\-ся модели вида
\begin{align*}
&a_i \rightarrow \min\limits_{\Omega}(\max\limits_{\Omega})\,;\\
&\left\vert f \left(\left. x^{\mathrm{расч}}\right\vert_j , \bar{a}\right) -
\left. \Delta x^{\mathrm{эксп}}\right\vert_j \right\vert \leq \lambda^*(1 + \delta) \,,
\enskip j = \overline{1, n}\,,
\end{align*}
где $\delta $~--- числовой параметр, $\delta \geq 0$.

\section{Упрощения и допущения} %4

  Основная сложность практической реализации описанного подхода
применительно к моделям сис\-тем\-ной динамики вытекает из спецификации
\mbox{правых} частей соотношений~(\ref{spi-e2}) и заключается, во-пер\-вых, в
большой размерности задач~(\ref{spi-e10}) и~(\ref{spi-e11}) и, во-вто\-рых, в
нелинейности век\-тор-функ\-ции $f\left(\bar{x}, \bar{a}\right)$, что существенно
усложняет процесс решении. Для решения перечисленных проблем авторы
использовали идеи линеаризации уравнений системной динамики~(\ref{spi-e2})
по параметрам~$\bar{a}$~\cite{spi-7}. Благодаря этому задача определения
диапазонов вариации па\-ра\-мет\-ров модели~(\ref{spi-e2}) (т.\,е.\ решения
задач~(\ref{spi-e10}) и~(\ref{spi-e11})) сводится к классическим задачам
линейного программирования, для решения которых разработано множество
эффективных алгоритмов. Такое упрощение приводит к определенной потере
точности искомых интервалов, однако названный недостаток можно считать
компенсированным за счет экономии временн$\acute{\mbox{ы}}$х и программных ресурсов.

  Важным преимуществом разработанного метода является возможность
включения в модель на этапе численного интегрирования
  системы~(\ref{spi-e2}) различных дополнительных условий, соблюдение
которых продиктовано очевидными соображениями, что позволяет повысить
степень адекватности модели, но не осуществимо в рамках классических
методов. В~качестве примера такого рода условий могут быть названы
ограничения на будущие значения переменных модели~(\ref{spi-e2}) или их
предполагаемые приращения. Совокупность всех таких условий может быть
формализована в виде ограничений $G\left(\bar{x}, \bar{a}\right) \subset S^0$,
подлежащих включению в модели~(\ref{spi-e10}) и~(\ref{spi-e11}).

  К определению параметров моделей системной динамики авторы подошли с
позиций учета сле\-ду\-ющих аспектов: необходимо добиваться, во-пер\-вых,
близости расчетных и экспериментальных данных, во-вто\-рых,~--- минимально
возможной области предельно допустимых погрешностей аппроксимации;
в-третьих,~--- минимального уровня вариации оцениваемых параметров. (Под
минимально возможной областью предельно допустимых погрешностей
аппроксимации подразумевается область значений
величин~$\left\{\eta_j\right\}$ с минимальным диаметром.) Соблюдение
третьего принципа позволяет снизить неопределенность, обусловленную
не\-един\-ст\-вен\-ностью решения поставленной задачи.

  С~этих позиций в рамках предлагаемого подхода формализация показателей
качественных характеристик модели~(\ref{spi-e2}), а именно точности,
адекватности и~пр., может быть обеспечена посредством задания целевой
функции и ограничений, отражающих каждая в отдельности одну из
качественных характеристик. Способ их непосредственного задания должен
определяться исследователем на основе анализа специфики самой задачи и
цели моделирования. Так, возможным вариантом критерия близости расчетных
и экспериментальных данных на стадии численного интегрирования
системы~(\ref{spi-e2}) может служить средняя ошибка аппроксимации, что
позволяет реализовать все названные принципы~\cite{spi-8}. (Средняя ошибка
аппроксимации рассчитывается по формуле
  $$
\bar{A} = \fr{1}{n} \sum\limits_{i =1}^n \left\vert \fr{
 y_i^{\mathrm{расч}} - y_i^{\mathrm{эксп}}}
{y_i^{\mathrm{эксп}}}\right\vert \cdot 100\%\,,
  $$
  где $y_i^{\mathrm{расч}}$ и $y_i^{\mathrm{эксп}}$~--- соответственно
рассчитанные согласно полученной модели и известные из наблюдений
значения исследуемого фактора в $i$-й момент времени; $n$~--- общее число
наблюдений.)

  Заметим, что формализация критерия, характеризующего точность по
совокупности уравнений модели~(\ref{spi-e2}), представляет собой
самостоятельную задачу, решение которой находится в компетенции
исследователя~\cite{spi-9}.

\section{Апробация метода} %5

  Апробация разработанного метода осуществлялась при моделировании
численности населения Российской Федерации. Общий вид исследованной
авторами модели системной динамики в терминах разностных уравнений
следующий:
\begin{equation}
\left.
\begin{array}{rl}
\Delta N & = a_1 N^{\alpha_1} D^{\beta_1} I^{\gamma_1} -
a_2 N^{\alpha_2} D^{\beta_2} I^{\gamma_2}\,;\\[9pt]
\Delta D & = a_3 N^{\alpha_3} D^{\beta_3} I^{\gamma_3} -
a_4 N^{\alpha_4} D^{\beta_4} I^{\gamma_4}\,;\\[9pt]
\Delta I & = a_5 N^{\alpha_5} D^{\beta_5} I^{\gamma_5} -
a_6 N^{\alpha_6} D^{\beta_6} I^{\gamma_6}\,,
\end{array}
\right\}
\label{spi-e12}
\end{equation}

%\begin{table*}
{\small %tabl1
%\vspace*{-12pt}
\begin{center}
\noindent
{{\tablename~1}\ \ \small{Исходные данные для модели (\ref{spi-e12})}}
%\label{spi-t1}}
\vspace*{2ex}

\tabcolsep=3pt
\begin{tabular}{|c|c|c|c|}
\hline
Год&
\tabcolsep=0pt\begin{tabular}{c} Численность\\ населения РФ\\ $N$, чел.
\end{tabular}&
\tabcolsep=0pt\begin{tabular}{c} Душевые\\ доходы $D$,\\ руб./чел.\ в год
\end{tabular}&
\tabcolsep=0pt\begin{tabular}{c} Индекс потреби-\\ тельских цен $I$,\\ доли ед.
\end{tabular}\\
\hline
1998&147\,802\,133&12\,122,4&1,844\\
1999&147\,539\,426&19\,906,8&1,365\\
2000&146\,890\,128&27\,373,2&1,202\\
2001&146\,303\,611&36\,744,0&1,186\\
2002&145\,649\,334&47\,366,4&1,151\\
2003&144\,963\,650&62\,044,8&1,120\\
2004&144\,168\,205&76\,923,6&1,117\\
2005&143\,474\,219&97\,342,8&1,109\\
2006&142\,753\,551&122\,352,0\hphantom{9}&1,090\\
2007&142\,220\,968&151\,232,4\hphantom{9}&1,119\\
2008&142\,008\,800&179\,287,2\hphantom{9}&1,133\\
2009&141\,904\,000&202\,282,8\hphantom{9}&1,088\\
  \hline
  \end{tabular}
  \end{center}
\vspace*{12pt}
  }
%  \end{table*}

\addtocounter{table}{1}

\noindent
где $N$~--- численность населения РФ; $D$~--- душевые доходы за год; $I$~---
индекс потребительских цен. Информационную базу исследования составили
данные официальной статистической отчетности за период с 1998 по 2009~гг.\
(табл.~1).



  В~целях сокращения проблем вычислительного характера была использована
двухэтапная процедура, основанная на применении разработанного метода к
линеаризованным правым частям модели~(\ref{spi-e12}). Целесообразность
двухэтапной процедуры обусловлена необходимостью задания центра
разложения в процедуре линеаризации и спецификой уравнений~(\ref{spi-e12}).
Используя разложение их правых частей в ряд Маклорена по переменным
$\left\{ a_i, \alpha_i, \beta_i, \gamma_i\right\}$, $i = \overline{1, 6}$, получим
систему:
\begin{equation}
\left.
\begin{array}{rl}
\Delta N & \approx a_1 - a_2\,;\\[9pt]
\Delta D & \approx a_3 - a_4\,;\\[9pt]
\Delta I & \approx a_5 - a_6\,,
\end{array}
\right\}
\label{spi-e13}
\end{equation}
в которой отсутствуют параметры, характери\-зу\-ющие показатели степеней всех
переменных модели. Именно по этой причине на первом этапе с помощью
разложения в ряд Маклорена~(\ref{spi-e13}) определялись точечные и
интервальные оценки величин $ \left\{ a_{2i -1} - a_{2i} \right\} $,
$ \left\{\left(  a_{2i -1} - a_{2i}\right)^0 \right\}$ и
$ \left\{\left[\left( a_{2i -1} - a_{2i}\right)^- ; \left( a_{2i -1} - a_{2i}\right)^+ \right]
\right\}$, $i= \overline{1, 3}$.

Разложение правых частей соотношений модели~(\ref{spi-e12}) в ряд Тейлора с
центром в точке $\left\{ a_i^0, \alpha_i= 0, \beta_i =0, \gamma_i = 0 \right\}$, $i =
\overline{1, 6}$, имеют следующий вид (на примере первого уравнения):
\begin{multline}
\Delta N \approx a_1 + a_1^0  \ln N\cdot \alpha_1 +
 a_1^0  \ln D\cdot \beta_1 +
 a_1^0  \ln I\cdot \gamma_1 -{}\\
{}- a_2 - a_2^0  \ln N\cdot \alpha_2 -
 a_2^0  \ln D\cdot \beta_2 -
 a_2^0  \ln I\cdot \gamma_2\,.
\label{spi-e14}
\end{multline}

  Таким образом, используя разложение в ряд Тейлора~(\ref{spi-e14}), можно
осуществлять идентификацию всех параметров первого уравнения
модели~(\ref{spi-e12}). (Строго говоря, вместо точечных и интервальных
оценок параметров $a_1$ и $_2$ будут найдены точечные и интервальные
оценки для выражения $\left( a_1 - a_2\right)$. Однако это обстоятельство
несущественно усложнит процесс численного эксперимента по определению
оптимального набора параметров системы~(\ref{spi-e12}).) Выбор центров
разложения в окрестности нулевых значений искомых параметров был
продиктован стремлением получить для них как можно меньшие значения, что,
в свою очередь, объясняется смыс\-лом каждого из слагаемых исследуемых
уравнений системной динамики: рост и уменьшение каждой из введенных в
рассмотрение переменных должны обладать умеренной интенсивностью.
Следует заметить, что использование только разложения в ряд Тейлора требует
обоснованного подхода к выбору точки, служащей центром разложения.

  Разработанный авторами метод (см.\ разд.~3) применялся для каждого уравнения
модели~(\ref{spi-e12}) в отдельности в рамках каждого из двух описанных
выше этапов процедуры, основанной на линеаризации исследуемых
зависимостей. Далее для определенности все рассуждения приводятся на
примере первого уравнения системы~(\ref{spi-e12}). Для остальных уравнений
этой системы все рассуждения аналогичны.

  Модель~(\ref{spi-e10}) для первого уравнения системы~(\ref{spi-e13}) имеет
вид:
\begin{equation}
\left.
\begin{array}{rl}
\lambda &\rightarrow \min\limits_{a_1, a_2, \lambda}\,;\\[9pt]
-\lambda &\leq a_1 - a_2 - \left. \Delta N^{\mathrm{эксп}}\right\vert_j \leq \lambda \,,
\enskip j = \overline{1, 11}\,; \\[9pt]
a_1 &\geq 0\,;\\[9pt]
a_2 &\geq 0\,;\\[9pt]
\lambda &\geq 0\,,
\end{array}
\right\}
\label{spi-e15}
\end{equation}
где $\left. \Delta N^{\mathrm{эксп}}\right\vert_j$~--- годовые приращения
величины $N$. Требования неотрицательности параметров $a_1$ и $a_2$
следуют из смысла слагаемых уравнений системной динамики. Результатом
решения задачи~(\ref{spi-e15}) являются величины $\left( a_1 -a_2\right)^0$ и
$\lambda^0$.

  Для определения диапазона вариации величины $a_1 - a_2$ решалась задача
вида~(\ref{spi-e11}), которая для первого уравнения системы~(\ref{spi-e13})
имеет вид:
\begin{equation}
\left.
\begin{array}{rl}
a_1 -a_2  &\rightarrow \min\limits_{a_1, a_2}(\max\limits_{a_1, a_2})\,;\\[9pt]
\left\vert a_1 -a_2 - \left. N^{\mathrm{эксп}}\right\vert_j \right\vert &\leq \lambda^0 \,,
 \enskip j = \overline{1, 11}\,;\\[9pt]
a_1 &\geq 0\,;\\[9pt]
a_2 &\geq 0\,.
\end{array}
\right\}
\label{spi-e16}
\end{equation}

\begin{table*}\small %tabl2
%\vspace*{-12pt}
\begin{center}
\Caption{Оптимальные решения задач (\ref{spi-e15}) и (\ref{spi-e16})
\label{spi-t2}}
\vspace*{2ex}

%\tabcolsep=2pt
\begin{tabular}{|c|c|c|c|r|}
\hline
Параметр&\tabcolsep=0pt \begin{tabular}{c} Точечные\\ оценки\\
$\left(a_1 - a_2\right)^0$
\end{tabular}&\tabcolsep=0pt\begin{tabular}{c}Минимальное\\ значение\\
$\left(\left(a_1 - a_2\right)^0\right)^{\min}$
\end{tabular}&\tabcolsep=0pt\begin{tabular}{c}Максимальное\\ значение\\
$\left(\left(a_1 - a_2\right)^0\right)^{\max}$   \end{tabular}&
\multicolumn{1}{c|}{$\lambda^0$}\\
\hline
$a_1 - a_2$&$-$392457,5\hphantom{999$-$}&$-$795445,0\hphantom{9999$-$}&10530,0\hphantom{99}&402987,5\hphantom{99}\\
$a_3 - a_4$&1514,45&622,20&2406,70&892,25\hphantom{9}\\
$a_5 - a_6$&\hphantom{9999}0,227&\hphantom{999}0,088&\hphantom{9999}0,365&0,139\\
  \hline
  \end{tabular}
  \end{center}
  %\vspace*{-12pt}
  \end{table*}

  Результаты численной реализации задач~(\ref{spi-e15}) и~(\ref{spi-e16})
(табл.~\ref{spi-t2}) позволяют осуществлять обоснованный выбор центра
разложения на втором этапе процедуры линеаризации системы~(\ref{spi-e12}),
при этом полученные интервальные оценки в случае необходимости
предоставляют исследователю дополнительную степень свободы при
определении его координат.



  Далее на основании полученных результатов (см.\ табл.~\ref{spi-t2}) и
разложения~(\ref{spi-e14}) решались задачи определения точечных:
\begin{equation}
\hspace*{-3mm}\left.
\begin{array}{rl}
\lambda  &\rightarrow \min\limits_{\left\{ \left. a_i, \alpha_i, \beta_i, \gamma_i
\right\vert j = \overline{1, 2}\right\}, \lambda}\,;\\[9pt]
-\lambda  &\leq a_1^0 + a_1 + a_1^0  \ln N\cdot \alpha_1 +  a_1^0  \ln
D\cdot \beta_1 +{}\\[9pt]
&{}+ a_1^0  \ln I\cdot \gamma_1 -
 a_2^0 -a_2 - a_2^0  \ln N\cdot \alpha_2 - {}\\[9pt]
 &\hspace*{15mm}{}- a_2^0  \ln D\cdot \beta_2 -
a_2^0  \ln I\cdot \gamma_2 -{}\\[9pt]
&\hspace*{15mm}{}-
\left. \Delta N^{\mathrm{эксп}}\right\vert_j \leq \lambda\,;
\enskip j = \overline{1, 11}\,;\\[9pt]
a_1 &\geq 0\,;\\[9pt]
a_2 &\geq 0\,;\\[9pt]
\lambda &\geq 0
\end{array}\!
\right\}\!\!
\label{spi-e17}
\end{equation}
 и интервальных оценок параметров модели~(\ref{spi-e12}):
 \begin{equation}
\hspace*{-2mm}\left.
\begin{array}{l}
a_1- a_2  \rightarrow \min\limits_{\left\{ \left. a_i, \alpha_i, \beta_i, \gamma_i
\right\vert j = \overline{1, 2}\right\}}\,;\\[9pt]
a_1^0 + a_1 + a_1^0  \ln N\cdot \alpha_1 +  a_1^0  \ln D\cdot \beta_1+{}\\[9pt]
{}+ a_1^0  \ln I\cdot \gamma_1 -
a_2^0 -a_2 - a_2^0  \ln N\cdot \alpha_2 -{}\\[9pt]
{}-   a_2^0  \ln D\cdot \beta_2 -
a_2^0  \ln I\cdot \gamma_2 -
 \Delta N^{\mathrm{эксп}}\Big\vert_j
\leq \lambda^*\,,\\[9pt]
\hspace*{150pt}\enskip j = \overline{1, 11}\,;\\[9pt]
a_1 \geq 0\,;\\[9pt]
a_2 \geq 0\,.
\end{array}\!
\right\}\!\!
\label{spi-e18}
\end{equation}



  Результатом решения задачи~(\ref{spi-e17}) являются точечные оценки
$\left( a_1 - a_2 \right)^*$, $\alpha_{1,2}^*$, $\beta_{1,2}^*$, $\gamma_{1,2}^*$,
$\lambda^*$.

  В~(\ref{spi-e18}) приведен вид оптимизационной задачи для определения интервалов
значений величины $\left( a_1 - a_2\right)$. Для остальных параметров модели
сис\-тем\-ной динамики составлялись аналогичные задачи, при этом на диапазон
их значений накладывались ограничения (столбец 2 табл.~\ref{spi-t3}), которые
в сочетании с условиями на неотрицательность параметров $\left\{ a_i, i =
\overline{1, 6}\right\}$ формировали упомянутое выше
множество~$\Omega$~(\ref{spi-e18}).


  Результаты численной реализации моделей~(\ref{spi-e17}) и~(\ref{spi-e18})
(см.\ табл.~\ref{spi-t3}) были использованы для организации специальной
вычислительной процедуры по определению оптимального набора параметров
модели~(\ref{spi-e12}) (рис.~\ref{spi-f2}).

\begin{table*}\small %tabl3
%\vspace*{-12pt}
\begin{center}
\Caption{Оценки параметров модели (\ref{spi-e12})
\label{spi-t3}}
\vspace*{2ex}

%\tabcolsep=2pt
\begin{tabular}{|c|c|c|c|c|}
\hline
Параметр&\tabcolsep=0pt \begin{tabular}{c} Ограничения\\ на
вариацию\end{tabular}&\tabcolsep=0pt \begin{tabular}{c} Точечные\\
оценки\end{tabular}&\tabcolsep=0pt\begin{tabular}{c} Минимальное\\ значение\\
параметра \end{tabular}&\tabcolsep=0pt\begin{tabular}{c} Максимальное\\
значение\\ параметра\end{tabular}\\
\hline
$a_1 - a_2$&---&22,03\hphantom{9}&$-$14\,151\,439,74\hphantom{$-$99999999}&23,4\hphantom{99}\\
$\alpha_1$&$[$0; 5$]$&5,00&0,00&5,00\\
$\beta_1$&$[$0; 5$]$&1,02&1,02&5,00\\
$\gamma_1$&$[$$-$5; 5$]$\hphantom{$-$}&5,00&$-$5,00\hphantom{$-$}&5,00\\
$\alpha_2$&$[$0; 5$]$&1,41&0,11&\hphantom{9}1,412\\
$\beta_2$&$[$0; 5$]$&0,00&0,00&1,03\\
$\gamma_2$&$[$$-$5; 5$]$\hphantom{$-$}&4,06&1,47&4,06\\
$a_3 - a_4$&---&$-$7173,5\hphantom{$-$9999}&$-$7173,5\hphantom{$-$9999}&
$-$3459,3\hphantom{$-$9999}\\
$\alpha_3$&$[$0; 2$]$&0,13&0,00&0,13\\
$\beta_3$&$[$0; 2$]$&0,32&0,32&0,33\\
$\gamma_3$&$[$$-$2; 2$]$\hphantom{$-$}&$-$1,18\hphantom{$-$}&$-$1,21\hphantom{$-$}&$-$1,14\hphantom{$-$}\\
$\alpha_4$&$[$0; 2$]$&2,00&0,00&2,00\\
$\beta_4$&$[$0; 2$]$&0,00&0,00&2,00\\
$\gamma_4$&$[$$-$2; 2$]$\hphantom{$-$}&1,99&$-$2,00\hphantom{$-$}&2,00\\
$a_5 - a_6$&$-$&$-$6,59\hphantom{$-$}&$-$9,60\hphantom{$-$}&7,59\\
$\alpha_5$&$[$0; 3$]$&0,00&0,00&2,99\\
$\beta_5$&$[$0; 3$]$&0,32&0,32&0,33\\
$\gamma_5$&$[$$-$2; 3$]$\hphantom{$-$}&1,70&$-$1,99\hphantom{$-$}&3,00\\
$\alpha_6$&$[$0; 3$]$&3,00&0,00&3,00\\
$\beta_6$&$[$0; 3$]$&0,01&0,01&3,00\\
$\gamma_6$&$[$$-$2; 3$]$\hphantom{$-$}&1,70&$-$2,00\hphantom{$-$}&3,00\\
  \hline
  \end{tabular}
  \end{center}
  %\vspace*{-12pt}
  \end{table*}


\begin{figure*} %fig2
\vspace*{1pt}
\begin{center}
\mbox{%
\epsfxsize=159.578mm
\epsfbox{spi-2.eps}
}
\end{center}
%\vspace*{-9pt}
\Caption{Схема разработанного метода определения оптимального набора
параметров модели системной динамики
\label{spi-f2}}
\end{figure*}

  Как отмечалось ранее, существенным преимуществом разработанного метода
является возможность учета априорных ограничений на значения параметров
искомых зависимостей, известных из очевидных соображений, что позволяет
значительно сократить неопределенность решаемых задач. В~качестве таких
ограничений были использованы условия на приращения переменных:
  \begin{alignat}{2}
  \left\vert \left. \Delta N\right\vert_j\right\vert &\leq  0{,}006N\,, \quad&&\enskip j
= \overline{1, 12}\,; \label{spi-e19}\\
  \left\vert \left. \Delta D\right\vert_j\right\vert &\leq  0{,}7D\,, &&\enskip j =
\overline{1, 12}\,; \label{spi-e20}\\
  \left\vert \left. \Delta I\right\vert_j\right\vert &\leq  0{,}7I\,, &&\enskip j =
\overline{1, 12}\,, \label{spi-e21}
\end{alignat}
и их будущие значения:
\begin{alignat}{2}
  \left\vert N_{13}^{\mathrm{расч}} - N_{12}^{\mathrm{эксп}}\right\vert & \leq
100\,000\,; \label{spi-e22}\\
  \left\vert D_{13}^{\mathrm{расч}} - D_{12}^{\mathrm{эксп}}\right\vert & \leq
120\,000\,. \label{spi-e23}
  \end{alignat}

  Условие~(\ref{spi-e19}) обусловлено максимальным за весь период
  1998--2009~гг.\ изменением показателя чис\-лен\-ности населения: в 2004~г.\
численность населения РФ сократилась на 0,6\% (что соответствует примерно
800~тыс.\,чел.). Условия~(\ref{spi-e20}) и~(\ref{spi-e21}) ограничивают рост
переменных $D$ (душевых доходов за год) и $I$ (индекса потребительских
цен) величиной в 70\%. Условия~(\ref{spi-e22}) и~(\ref{spi-e23}) отражают
тенденцию изменения соответствующих переменных, сложившуюся к 2010~г.\
($N_{13}^{\mathrm{расч}}$ и $D_{13}^{\mathrm{расч}}$~--- рассчитанные
согласно модели~(\ref{spi-e12}) прогнозные значения переменных $N$ и $D$ в
момент времени $j = 13$, т.\,е.\ для 2010~г.).

\begin{figure*} %fig3
\vspace*{1pt}
\begin{center}
\mbox{%
\epsfxsize=161.058mm
\epsfbox{spi-3.eps}
}
\end{center}
\vspace*{-9pt}
\Caption{Графическая иллюстрация результатов численного интегрирования
системы~(\ref{spi-e25}) методом Рунге--Кутты (кривые):
(\textit{a})~$\bar{A}_N = 0{,}13\%$;
(\textit{б})~$\bar{A}_D = 3{,}33\%$;
(\textit{в})~$\bar{A}_I = 5{,}86\%$. Значки~--- экспериментальные данные
\label{spi-f3}
}
\end{figure*}
\begin{table*}[b]\small %tabl4
\vspace*{-12pt}
\begin{center}
\Caption{Сравнение прогнозных и фактических значений численности
населения РФ, тыс.~чел.
\label{spi-t4}}
\vspace*{2ex}

%\tabcolsep=2pt
\begin{tabular}{|l|c|c|c|}
\hline
\multicolumn{1}{|c|}{Источник}&\tabcolsep=0pt\begin{tabular}{c} На 1 января\\ 2010 г. \end{tabular}&
\tabcolsep=0pt\begin{tabular}{c} На 1 января\\ 2011 г. \end{tabular}&
\tabcolsep=0pt\begin{tabular}{c} На 1 января\\ 2012 г. \end{tabular}\\
\hline
По данным Федеральной службы государственной статистики&
142\,833,0&142\,865,0&143\,056,0\\
%\hline
Согласно модели системной динамики~(\ref{spi-e25})& 142\,025,2&142\,649,1&
143\,793,4\\
\hline
Погрешность&807,8 (0,57\%)&215,9 (0,15\%)&737,4 (0,52\%)\\
  \hline
  \end{tabular}
  \end{center}
  %\vspace*{-12pt}
  \end{table*}

  В~качестве числовых критериев, характери\-зу\-ющих точность каждого
уравнения модели~(\ref{spi-e12}), рассматривались средние ошибки
аппроксимации, для которых приемлемым считался уровень, не превышающий
10\%~\cite{spi-10} (т.\,е.\ $\bar{A}_N \hm\leq 10\%$,  $\bar{A}_D \hm\leq 10\%$,
$\bar{A}_I \hm\leq 10\%$).

  Критерий оптимальности набора значений параметров модели~(\ref{spi-e12})
определялся на основе модуля вектора средних ошибок аппроксимации,
компонентами которого являлись рассчитываемые средние ошибки
аппроксимации по каждому уравнению модели~(\ref{spi-e2}):
\begin{equation*}
\sqrt{\bar{A}_N^2 + \bar{A}_D^2 + \bar{A}_I^2} \rightarrow \min\,.
%\label{spi-e24}
\end{equation*}

  В~качестве метода численного интегрирования системы~(\ref{spi-e12})
авторы выбрали метод Рунге--Кутты\linebreak 4-го порядка ввиду его высокой точности
и меньшей склонности к возникновению неустойчивости решения. По
результатам вычислительной процедуры был определен оптимальный с
позиций заданных условий и критерия набор параметров исследуемой модели
численности населения Российской Федерации:
\begin{equation}
\left.
\begin{array}{rl}
\fr{dN}{dt} & = {}\\[6pt]
&\hspace*{-8mm}{}=8{,}139 \cdot 10^{-22} \, \fr{N^{2{,}05} D^2}{I^2} -
64{,}1 \, \fr{N^{0{,}33}  D^{0{,}3}}{I^{0{,}3}}\,;\\[6pt]
\fr{dD}{dt} & = 560  D^{0{,}35} - 9900\, I\,;\\[6pt]
\fr{dI}{dt} & = 0{,}131  I^{-0{,}4} - 0{,}0072 \, \fr{N^{0{,}092}
D^{0{,}092}}{I^{0{,}092}}\,.
\end{array}\!
\right\}\!
\label{spi-e25}
\end{equation}

  Модель~(\ref{spi-e25}) характеризуется хорошими показателями точности
(рис.~\ref{spi-f3}) и адекватности, что позволило, в частности, эффективно
решать задачи получения прогнозных оценок. Так, полученные прогнозные
оценки согласно модели~(\ref{spi-e25}) в дальнейшем были подтверждены
фактическими наблюдениями (табл.~\ref{spi-t4}).


\section{Заключение} %6

  Для успешной реализации метода системной динамики необходимо
использовать математические модели, обладающие <<хорошими>>
качественными характеристиками. Предложенный в работе подход к
построению уравнений системной динамики позволяет определять
оптимальный с точки зрения исследователя вид модели в условиях отсутствия
четких представлений о функциональных связях между переменными.
Полученные в ходе апробации результаты свидетельствуют о том, что
описанный в настоящей работе метод определения точечных и интервальных
оценок параметров моделей системной динамики, основанный на
использовании идей подхода Л.\,В.~Канторовича к обработке наблюдений,
является эффективным инструментом для подготовки и организации
численного эксперимента по определению оптимального набора параметров
моделей заданной структуры. Существенным преимуществом разработанного
подхода является возможность соблюдения ряда дополнительных условий,
значимых на взгляд исследователя, учесть которые классическими
статистическими методами невозможно.

{\small\frenchspacing
 {%\baselineskip=10.8pt
 \addcontentsline{toc}{section}{References}
 \begin{thebibliography}{99}

 \bibitem{spi-4} %1
\Au{Форрестер~Дж.}
  Мировая динамика~/ Пер. с англ.~--- М.: Наука, 1978.
  (\Au{Forrester~J.}
 {World dynamics}.~--- Wright-Allen Press, 1971. 144~p.)

 \bibitem{spi-2} %2
\Au{Wolctenholme~E.}
 System enquiry~--- a system dynamic approach.~--- Chichester,
England: John Wiley and Sons, 1990. 238~p.


\bibitem{spi-1} %3
\Au{Белолипецкий~В.\,М., Шокин~Ю.\,И.}
 Математическое моделирование в задачах окружающей среды.~---
Новосибирск: Инфолио-пресс, 1997.

\bibitem{spi-3} %4
\Au{Sterman~J.\,D.}
 Business dynamics systems thinking and modeling for a complex
world.~--- Irwin: McGrow-Hill, 2000. 1008~p.

\bibitem{spi-5} %5
\Au{Спивак~С.\,И.}
 Информативность кинетических измерений~// Вестник Башкирского
ун-та, 2009. Т.~14. №\,3. С.~1056--1059.
\bibitem{spi-6}
\Au{Канторович~Л.\,В.}
 О~некоторых новых подходах к вычислительным методам и обработке
наблюдений~// Сибирский математический журнал,
1962.  Т.~3. №\,5. С.~701--709.
\bibitem{spi-7}
\Au{Моисеев~Н.\,Н.}
 Математические задачи системного анализа.~--- М.: Наука, 1981. 488~с.
\bibitem{spi-8}
\Au{Спивак~С.\,И., Кантор~О.\,Г.}
Качество моделей математической обработки наблюдений
со\-ци\-аль\-но-эко\-но\-ми\-че\-ских систем~//
Системы управления и информационные
технологии, 2012. №\,2(48). С.~44--49.
\bibitem{spi-9}
\Au{Гайнанов~Д.\,А., Кантор~О.\,Г., Казаков~В.\,В.}
Оценка уровня  со\-ци\-аль\-но-эко\-но\-ми\-че\-ско\-го
развития территориальных систем на основе метрического анализа~//
Вестник Томского гос. ун-та, 2009. №\,322. С.~138--144.

\bibitem{spi-10}
Эконометрика~/
 Под ред.\ Елисеевой~И.\,И.~--- М.: Финансы и статистика, 2008. 576~с.

\end{thebibliography}

 }
 }

\end{multicols}

\vspace*{-6pt}

\hfill{\small\textit{Поступила в редакцию 25.10.13}}

%\newpage

\vspace*{12pt}

\hrule

\vspace*{2pt}

\hrule

\def\tit{CONSTRUCTION OF~SYSTEM DYNAMICS MODELS IN~CONDITIONS OF~LIMITED
EXPERT INFORMATION}

\def\titkol{Construction of~system dynamics models in conditions of limited expert information}

\def\aut{O.\,G.~Kantor$^1$, S.\,I.~Spivak$^2$}

\def\autkol{O.\,G.~Kantor and S.\,I.~Spivak}

\titel{\tit}{\aut}{\autkol}{\titkol}

\vspace*{-9pt}

\noindent
$^1$Institute of Social and Economic Researches of Ufa Scientific Centre RAS; 71 Av.
Oktyabrya, Ufa 450054, Russian\\
$\hphantom{^1}$Federation

\noindent
$^2$Bashkir State University,
32 Validy Str., Ufa 450076, Russian Federation

\def\leftfootline{\small{\textbf{\thepage}
\hfill INFORMATIKA I EE PRIMENENIYA~--- INFORMATICS AND
APPLICATIONS\ \ \ 2014\ \ \ volume~8\ \ \ issue\ 2}
}%
 \def\rightfootline{\small{INFORMATIKA I EE PRIMENENIYA~---
INFORMATICS AND APPLICATIONS\ \ \ 2014\ \ \ volume~8\ \ \ issue\ 2
\hfill \textbf{\thepage}}}

\vspace*{6pt}


\Abste{System dynamics is a methodology for studying of complex dynamic systems
focused on conducting computer simulations. Construction of system dynamics
models is largely dependent on the available experimental information and expert
judgments. A~simulation experiment with ``bad'' models can lead to significant or even
total distortion of the system properties. This paper describes a method of
constructing the system dynamics models, which is a set of mathematical models,
based on the idea of the Kantorovich approach to the mathematical treatment of
experimental data, and computational procedures. An important advantage is the
possibility of including in the model significant conditions which are important to
researcher and affect model adequacy. The developed method was tested on the
example of the Russian population.}

\KWE{system dynamics models; point and interval estimation of model parameters;
the Kantorovich approach; the maximum permissible error of measurement}

\DOI{10.14357/19922264140211}

\vspace*{-12pt}

\Ack

\noindent
The work was supported by the Russian Foundation for Basic Research (project
No.\,13-01-00749).


  \begin{multicols}{2}

\renewcommand{\bibname}{\protect\rmfamily References}
%\renewcommand{\bibname}{\large\protect\rm References}


{\small\frenchspacing
 {%\baselineskip=10.8pt
 \addcontentsline{toc}{section}{References}
 \begin{thebibliography}{99}

 \bibitem{spie-4} %1
\Aue{Forrester,~J.}
 1971. \textit{World dynamics}. Wright-Allen Press. 144~p.

\bibitem{spie-2} %2
\Aue{Wolctenholme,~E.}
 1990. \textit{System enquiry --- a~system dynamic approach}. Chichester, England: John Wiley and Sons. 238~p.

 \bibitem{spie-1} %3
\Aue{Belolipeckij,~V.\,M., and Ju.\,I.~Shokin.}
 1997. \textit{Matema\-ti\-che\-skoe modelirovanie v zadachakh okruzhayushchey sredy}
[Mathematical modeling in environmental problems]. Novosibirsk:
Infolio-Press. 240~p.


\bibitem{spie-3} %4
\Aue{Sterman,~J.\,D.}
 2000. \textit{Business dynamics systems thinking and modeling for a complex world}.
Irwin: McGrow-Hill. 1008~p.


\bibitem{spie-5} %5
\Aue{Spivak,~S.\,I.}
 2009. Informativnost' kineticheskikh izmereniy [Informative kinetic measurements].
\textit{Vestnik Bashkirskogo Un-ta} [Bashkir University Bulletin] 3(14):1056--1059.
\bibitem{spie-6}
\Aue{Kantorovich,~L.\,V.}
 1962. O~nekotorykh novykh podkhodakh k vychislitel'nym metodam i obrabotke
nablyudeniy [Some new approaches to computational methods and treatment of
observations].
\textit{Sibirskiy Matematicheskiy Zhurnal} [Siberian Mathematical~J.]
5(3):701--709.
\bibitem{spie-7}
\Aue{Moiseev,~N.\,N.}
 1981. \textit{Matematicheskie zadachi sistemnogo analiza} [Mathematical problems of
system analysis]. Moscow: Nauka. 488~p.
\bibitem{spie-8}
\Aue{Spivak,~S.\,I., and O.\,G.~Kantor.}
 2012. Kachestvo modeley matematicheskoy obrabotki sotsial'no-ekonomicheskikh
sistem [Quality of mathematical processing observations models of socio-economic
systems]. \textit{Sistemy Upravleniya i Informatsionnye Tekhnologii} [Management and
Information Technology] 2(48):44--49.
\bibitem{spie-9}
\Aue{Gajnanov,~D.\,A., O.\,G.~Kantor, and V.\,V.~Kazakov.}
 2009. Otsenka urovnya sotsial'no-ekonomicheskogo razvitiya territorial'nykh sistem
na osnove metricheskogo analiza [Estimation of level of social and economic
development of territorial systems based on metric analysis].
\textit{Vestnik Tomskogo
Gos. Un-ta} [Bulletin of \mbox{Tomsk} State University] 322:138--144.
\bibitem{spie-10}
Eliseeva,~I.\,I., ed. [Ed. I.\,I.~Eliseeva.]. 2008.
\textit{Ekonometrika} [Econometrics].  Moscow: Finance and Statistics. 576~p.

\end{thebibliography}

 }
 }

\end{multicols}

%\vspace*{-6pt}

\hfill{\small\textit{Received October 25, 2013}}

\vspace*{-18pt}

\Contr

\noindent
\textbf{Kantor Olga G.}~(b.~1971)~--- Candidate of Science (PhD) in physics and
mathematics, associate professor, senior scientist, Institute of Social and Economic
Researches, Ufa Scientific Centre of the Russian Academy of Sciences,
71 Av. Oktyabrya, Ufa 450054, Russian Federation;
o\_kantor@mail.ru

\vspace*{3pt}

\noindent
\textbf{Spivak Semen I.}~(b.~1945)~--- Doctor of Science in physics and
mathematics, professor, Head of Department, Bashkir State University,
32 Validy Str., Ufa 450076, Russian Federation;
semen.spivak@mail.ru

\label{end\stat}

\renewcommand{\bibname}{\protect\rm Литература}