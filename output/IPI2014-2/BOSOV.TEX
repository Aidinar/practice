\def\stat{bosov}

\def\tit{ОБОБЩЕННАЯ ЗАДАЧА РАСПРЕДЕЛЕНИЯ РЕСУРСОВ ПРОГРАММНОЙ СИСТЕМЫ}

\def\titkol{Обобщенная задача распределения ресурсов программной системы}

\def\aut{А.\,В. Босов$^1$}

\def\autkol{А.\,В. Босов}

\titel{\tit}{\aut}{\autkol}{\titkol}

\renewcommand{\thefootnote}{\arabic{footnote}}
\footnotetext[1]{Институт проблем информатики Российской академии наук,
AVBosov@ipiran.ru}

\vspace*{-6pt}

\Abst{Представлены постановка и решение задачи
оптимизации динамической системы с линейным выходом по
квадратичному критерию качества. Неопределенность системы
описывается наблюдаемым случайным процессом второго порядка.
В~качестве практического обоснования рас\-смат\-ри\-ва\-ют\-ся потребности
оптимизации распределения ресурсов некоторой программной
системы. В~такой интерпретации неопределенность системы
описывает пользовательскую активность, а выход~--- число
выполняемых запросов или объем запрашиваемой памяти. Цель
оптимизации формализуется квадратичным критерием качества
общего вида. Критерий, в частности, обобщает две задачи
распределения ресурсов программной системы, рассмотренные ранее.
Целевой функционал позволяет, в частности, ставить задачи
выделения достаточного объема программных ресурсов (нитей,
памяти и~т.\,п.), штрафуя за их неограниченное расходование. Для
решения задачи используется метод динамического
программирования. Оптимальная стратегия находится в виде
линейной комбинации выхода и прогнозов состояния вплоть до
горизонта управления. В~связи с вычислительной трудоемкостью
оптимальной стратегии обсуждается возможность ее упрощения и
применения ло\-каль\-но-оп\-ти\-маль\-ной стратегии.}

\KW{программная система; стохастическая система наблюдения;
квадратичный критерий; динамическое программирование}

\DOI{10.14357/19922264140204}

\vskip 14pt plus 9pt minus 6pt

\thispagestyle{headings}

\begin{multicols}{2}

\label{st\stat}

\section{Введение} %1

 Задачи оптимального распределения ресурсов могут возникать в самых
разных приложениях~--- от индустриальных до финансовых. Наиболее
популярны постановки, имеющие экономический контекст, например,
связанные с инвестированием~[1]. Однако в связи с повсеместным
распространением информационных технологий (ИТ) традиционные постановки
оптимального распределения ресурсов нашли новую область применения.

 Надо отметить, что понятие ресурса благодаря
ИТ приобрело множество специфических оттенков. Так,
 ИТ-спе\-ци\-а\-лист под ресурсом скорее будет понимать веб-сайт или
базу данных, чем сырье, средства производства или финансовые инстру\-менты.
Для таких ресурсов применяется термин <<информационный ресурс>>~[2].
Близким, но не синонимичным ему является термин <<вычислительный
ресурс>>~[3]. Развитие телекоммуникационных сетей, распределенных сис\-тем,
центров обработки данных и технологий виртуализации обогатило
терминологию множеством вариаций на эту тему. Часть из них носит вполне
материальный характер (например, сервер, процессорное время, объем памяти,
пропускная способность сети). Часть~--- виртуальны по своей сути (например,
программа,\linebreak сайт, банк данных, запрос). Задачи оптимально-\linebreak го распределения
ИТ-ре\-сур\-сов применительно к\linebreak ресурсам первого типа выглядят вполне
традиционными. Но представляется перспективным рас\-смат\-ри\-вать такие же
задачи применительно и к ресурсам второго типа. Объединяет виртуальные
ресурсы то, что все они имеют отношение к программным системам (сами
являются программами или ими обслуживаются).

Примеры программных
продуктов, в которых так или иначе алгоритмизируются процедуры управления
ресурсами, хорошо известны. Достаточно упомянуть менеджеры задач,
имеющиеся в любой операционной системе~[4], и оптимизаторы запросов,
функционирующие в любой системе управления базами данных~[5]. Можно
упомянуть еще задачи балансировки нагрузки и управления потоками заданий,
а также постановки, связанные с управляемым протоколом TCP
(transmission control protocol)~[6--8].

 Модель и отчасти критерий оптимальности, рассмотренные в данной
работе, возникли именно из задачи оптимизации функционирования
программной системы. Предложенная постановка, по-ви\-ди\-мо\-му, может
представлять интерес и в ка\-кой-то экономической интерпретации, и в иных
приложениях, но главным приложением рассмотренной далее задачи является
область программирования (точнее, проектирования программных систем). По
крайней мере, именно это дает обоснование предложенному далее виду
целевого функционала.

\section{Постановка задачи}

 Далее используются следующие обозначения:
 \begin{itemize}
 \item $\eqdelta$~--- равенство по определению;
\item $\mathbb{R}^p$~--- $p$-мерное евклидово пространство;
\item $\mathbb{R}^{p\times q}$~--- пространство матриц размерности
$p\times q$;
\item $\mathbb{M}[x], \mathbb{M}[x\vert \mathfrak{J}]$~--- безусловное
математическое ожидание случайного вектора $x\hm\in\mathbb{R}^p$ и
условное математическое ожидание~$x$ относительно
$\sigma$-ал\-геб\-ры $\mathfrak{J}$;
\item $x^{\mathrm{T}}$~--- операция транспонирования вектора (мат\-ри\-цы)~$x$;
\item $\alpha^+$~--- операция псевдообращения матрицы~$\alpha$;
\item $\| x\|_\alpha \eqdelta \sqrt{x^{\mathrm{T}} \alpha x}$~--- вес (норма при $\alpha\hm>0$)
вектора $x\hm\in \mathbb{R}^p$, заданный симметричной
($\alpha\hm=\alpha^{\mathrm{T}}$) неотрицательно определенной ($\alpha\hm\geq0$)
матрицей $\alpha\hm\in \mathbb{R}^{p\times p}$; $\| x\|^2\eqdelta \|x\|^2_E$,
где $E$~--- единичная матрица соответствующей размерности;
\item $\mathrm{col}\left(x_1,\ldots , x_q\right)\eqdelta (x_1^{\mathrm{T}},
\ldots x_q^{\mathrm{T}})^{\mathrm{T}}$~---
век\-тор-стол\-бец, составленный из векторов $x_1, \ldots , x_q$;
\item $\mathfrak{J}_t^y\eqdelta \sigma\{y_\tau, \tau\leq t\}$~---
$\sigma$-ал\-геб\-ра, порожденная случайной
последовательностью~$y_\tau$, $\tau\hm\leq t$.
\end{itemize}

 Пусть задано распределение случайной последовательности
$y_t\hm\in \mathbb{R}^q$, $t=0, 1, \ldots , N+1$, второго порядка:
$\mathbb{M}[\|y_t\|^2]\hm<\infty$. Будем предполагать, что
последовательность~$y_t$ наблюдается, порождает\linebreak
 $\sigma$-ал\-геб\-ру $\mathfrak{J}_t^y$ и выход $z_t\hm\in
\mathbb{R}^p$, описываемый уравнением
 \begin{multline}
 z_{t+1}= a_t y_t +b_t z_t +c_t u_t\,,\\
 t=0,1,\ldots , N\,,\enskip z_0=0\,,
 \label{e1-bos}
 \end{multline}
где $u_t\in \mathbb{R}^r$~--- управ\-ля\-ющее воздействие, формируемое по
входной последовательности~$y_\tau$, $\tau\hm\leq t$, с целью минимизации
следующего целевого функционала:
\begin{equation}
\left.
\begin{array}{rl}
J(U_N)& =\\[9pt]
&\hspace*{-15mm}{}=\displaystyle
\sum\limits_{t=0}^N \mathbb{M}
\Big[
 \| P_t y_{t+1} +Q_t z_{t+1}
 +R_t u_t +S_t\|^2_{\alpha_t} +{} \\
&\hspace*{10mm} {}+\| u_t-u_{t-1}\|^2_{\beta_t}+
\| u_t\|^2_{\gamma_t}\Big]\,;\\[9pt]
V_N &=\mathrm{col} \left( u_0,\ldots , u_t, \ldots , u_N\right)
\in \mathbb{R}^{r\times (N+1)}\,.
\end{array}\!\!
\right\}\!
\label{e2-bos}
\end{equation}

 Матрицы $a_t\hm\in \mathbb{R}^{p\times q}$,
$b_t\hm\in \mathbb{R}^{p\times p}$, $c_t\hm\in \mathbb{R}^{p\times r}$,
$P_t\hm\in \mathbb{R}^{s\times q}$, $Q_t\hm\in \mathbb{R}^{s\times p}$,
$R_t\hm\in \mathbb{R}^{s\times r}$, $\alpha_t\hm\in \mathbb{R}^{s\times s}$,
$\beta_t\hm\in \mathbb{R}^{r\times r}$, $\gamma_t\hm\in \mathbb{R}^{r\times r}$
и векторы
$S_t\hm\in \mathbb{R}^s$, $t\hm=0,1,\ldots N$, предполагаются известными,
$\beta_0\hm=0$. Допустимыми будем считать управляющие воздействия~$u_t$,
являющиеся $\mathfrak{J}_t^y$-из\-ме\-ри\-мы\-ми случайными
последовательностями второго порядка:
$\mathbb{M}\left[ \|u_t\|^2\right]\hm<\infty$. Задачей оптимизации
является поиск управ\-ля\-юще\-го
воздействия~$u_t^*$, минимизирующего целевой функционал $J(U_N)$:
 \begin{equation}
 \left.
 \begin{array}{rl}
 U_N^*&\in \mathop{\mbox{argmin}}\limits_{U_N} J(U_N)\,;\\[9pt]
 U_N^* &=\mathrm{col} \left( u_0^*, \ldots , u_t^*,\ldots , u_N^*\right)\,.
 \end{array}
 \right\}
 \label{e3-bos}
 \end{equation}

 Отметим, что процесс, описываемый уравнением~(\ref{e1-bos}), в общем
случае не предполагается гауссовским и не обладает марковским свойством.

\section{Обсуждение целевого функционала}

 Функционал~(\ref{e2-bos}) сформирован в результате обобщения двух
задач оптимизации функционирования определенной программной системы.
Эта программная система~--- Информационный веб-пор\-тал~--- реализация
одного из вариантов организации распределенных информационных систем,
взаимодействующих на принципах федеративности в среде Интернета~[9].
В~связи с функционированием этой системы возникли две задачи:
оптимизации расходования <<внутренних>>~[10] и <<внешних>>~[11]
ресурсов. Основной функцией портала является обеспечение промежуточного
слоя взаимодействия пользователя, формирующего поисковые запросы к
порталу, и информационных источников портала: пользовательский запрос
преобразуется в набор команд, исполняемых взаимодействующими с порталом
сис\-те\-ма\-ми-ис\-точ\-ни\-ка\-ми, результаты работы которых
консолидируются и возвращаются пользователю в качестве ответа на запрос.
Соответственно, портал функционирует в условиях неопределенности
пользовательской активности, описываемой процессом~$y_t$.

 Для взаимодействия с источниками про\-грам\-мны\-ми средствами портала
поддерживается некоторый пул (множество нитей), используемый для
организа\-ции параллельного исполнения команд источниками. Определение
размера пула и со\-став\-ля\-ет задачу оптимизации расходования <<внутренних>>
ресурсов. При отсутствии ка\-кой-ли\-бо стратегии управления пулом число
поддерживаемых им нитей может, вообще говоря, расти неограниченно. Эта
задача по сути сводится к отслеживанию управ\-ля\-ющей переменной~$u_t$
(числа нитей) выхода~$z_t$ (числа выполняемых команд). В~[10] цель
управления пулом формализуется слагаемым $\| z_{t+1}\hm- u_t\|^2$, а штраф
за его рост~--- слагаемым $\| u_t\|^2$.

 Далее для обеспечения собственного функцио\-ни\-рования портал
использует <<внешние>> (предоставляемые обслуживающими подсистемами)
ресур\-сы, в частности, активно изменяющееся дисковое пространство.
Оптимизация запросов на выделение/освобождение дискового пространства
сводится к задаче отслеживания выходом~$z_t$ (общим объемом портального
хранилища) определенной траектории, <<близкой>> к траектории фазовой
переменной~$y_t$ (размера хранимых пользовательских данных). В~[11] цель
управления, сформулированная как поддержание хранилища достаточного
объема, формализуется слагаемым $\| y_{t+1} \hm- z_{t+1} \hm- S_t\|^2$, где
$S_t$~--- объем свободного места, штраф за применение управляющего
воздействия~--- сла\-га\-емым $\| u_t\|^2$.

 Обобщение целей оптимизации этих двух задач в функционале~(\ref{e2-bos})
 представляет первое слагаемое. Во-пер\-вых, пары слагаемых
$P_ty_{t+1},R_tu_t$ и $Q_tz_{t+1},R_tu_t$ позволяют ставить задачи
отслеживания управляющим воздействием $u_t$ траекторий $y_t$ или $z_t$
соответственно, формализуя таким образом задачу оптимизация распределения
<<внутренних>> ресурсов. Во-вто\-рых, пара слагаемых $P_t y_{t+1}, Q_t
z_{t+1}$ позволяет отслеживать выходом~$z_t$ фазовую переменную~$y_t$.
 В-третьих, слагаемое $S_t$ позволяет ставить задачи формирования
траектории выхода, <<близкой>> к заданной опорной. Кроме того, слагаемое
$Q_t z_{t+1}$ позволяет ставить традиционную задачу минимизации нормы
выходной переменной. Наконец, очевидно, что перечисленные цели
оптимизации могут произвольно комбинироваться.

 Довольно специфическим является второе слагаемое из~(\ref{e2-bos}).
Оно вносит в целевой функционал штраф за изменение управляющего
воздействия. Объяснение этому слагаемому дает заданная предметная область.
Его уместно охарактеризовать как плату за реконфигурацию программной
системы.\linebreak Очевидно, что такие действия, как постоянное\linebreak
 под\-клю\-че\-ние/от\-клю\-че\-ние хостов, процессоров, дисков и~т.\,п.,
служат очевидным поводом к сокращению их времени службы, а
 вы\-де\-ле\-ние/осво\-бож\-де\-ние памяти, создание/удаление нитей,
указателей и~т.\,п.\ в программах~--- источником порождения ошибок как
самой программой, так и обеспечи\-ва\-ющи\-ми подсистемами. Кроме того, частое
реконфигурирование существенно труднее реализовать, и оно само по себе
расходует значительные ресурсы. Соответственно, назначение этого
слагаемого~--- сократить число подобных реконфигураций.

 Последнее, третье, слагаемое вносит в целевой функционал
традиционный штраф за размер управ\-ля\-юще\-го воздействия.

 Легкость в формализации целей оптимизации является несомненным
преимуществом целевого функционала. Понятный физический смысл
перечисленных выше слагаемых~--- это достоинство
 функционала~(\ref{e2-bos}). Однако очевидны и его недостатки.
Впрочем, это недостатки, свойственные любому квадратичному критерию.
В~дискуссии о <<нефизичности>> квадратичного функционала качества
участвовал еще Р.~Беллман, а в попытках решить <<обратную>> задачу, связав
параметры квад\-ра\-тич\-но\-го функционала с динамическими характеристиками
процессов,~--- например, Р.~Калман. В~исследованиях на эту тему участвовали
многие ученые, обозревать эту историю в данной работе неуместно.
В~рассматриваемой же задаче с учетом предметной области проблему выбора
весовых коэффициентов $\alpha_t$, $\beta_t$ и~$\gamma_t$, а также горизонта
управ\-ле\-ния $N$ представляется возможным возложить на этап реализации
программной системы. Тем более, что в большинстве случаев создаваемые
программы вполне допускают возможность многократного проведения
недорогостоящих экспериментов и эмпирический анализ качества
оптимизированной программной системы по другим критериям (например,
мнению эксперта, пользователя, заказчика). В~пользу же квадратичного
критерия говорит его <<технологичность>>~--- развитый аппарат исследования
и возможность получения конечных аналитических решений.

\section{Решение задачи оптимизации}

 \noindent
 \textbf{Теорема.}\ Решение задачи оптимизации~(\ref{e3-bos})
 выхода~(\ref{e1-bos}), формируемого случайной
последова\-тель\-ностью~$y_t$, существует. Оптимальное решение
$U_N^*\hm= \mathrm{col} \left( u_0^*, \ldots , u_N^*\right)$ с минимальной нормой
$\| u_t^*\|^2$ определяется следующими выражениями:
 \begin{equation}
\hspace*{-3mm}u_t^*=L_t^+ \left( \beta_t u^*_{t-1} -K_t^z z_t -K_t\right),\ t=0,1,\ldots , N,\!\!
 \label{e4-bos}
 \end{equation}
где

\noindent
\begin{equation}
\left.
\begin{array}{rl}
K_t &\eqdelta \sum\limits_{j=0}^{N-t+1} K_t^j \overline{y} (t+j,t) +K_t^s\,,\\[6pt]
\overline{y} (t+j,t) &\eqdelta \mathbb{M} \left[ y_{t+j}\vert \mathfrak{J}_t^y \right]\,;
\end{array}
\right\}
\label{e5-bos}
\end{equation}

\vspace*{-6pt}

\noindent
 \begin{multline}
K_t^z = \left(Q_t c_t +R_t\right)^{\mathrm{T}} \alpha_t Q_t b_t +
\beta_{t+1} L^+_{t+1} K^z_{t+1} b_t +{}\\
{} + c_t^{\mathrm{T}}(M^z_{t+1} -\left(K^z_{t+1}\right)^{\mathrm{T}} L^+_{t+1}
K^z_{t+1})b_t\,;
\end{multline}

\vspace*{-12pt}

\noindent
\begin{multline}
 K_t^s = \left(Q_t c_t +R_t\right)^{\mathrm{T}} \alpha_t S_t +\beta_{t+1}L^+_{t+1} K^s_{t+1}
+{}\\
{}+c_t^{\mathrm{T}} (M^s_{t+1} -\left(K^z_{t+1}\right)^{\mathrm{T}} L^+_{t+1} K^s_{t+1})\,;
\end{multline}

\vspace*{-12pt}

\noindent
\begin{multline}
 K_t^0 = \left(Q_t c_t +R_t\right)^{\mathrm{T}} \alpha_t Q_t a_t +\beta_{t+1} L^+_{t+1}
K^z_{t+1} a_t +{}\\[2pt]
{}+c_t^{\mathrm{T}}\left(M^z_{t+1}-\left(K^z_{t+1}\right)^{\mathrm{T}} L^+_{t+1} K^z_{t+1}\right)
a_t\,;
\end{multline}

%\pagebreak

%\vspace*{-12pt}

\noindent
\begin{multline}
 K_t^1 =\left(Q_t c_t +R_t\right)^{\mathrm{T}} \alpha_t P_t +\beta_{t+1}
 L^+_{t+1} K^0_{t+1}+{}\\[2pt]
{}+c_t^{\mathrm{T}}\left(M^0_{t+1}-\left(K^z_{t+1}\right)^{\mathrm{T}} L^+_{t+1} K^0_{t+1}\right)\,,
\end{multline}

\vspace*{-12pt}

\noindent
\begin{multline}
 K_t^j = \beta_{t+1} L^+_{t+1} K_{t+1}^{j-1} +{}\\[2pt]
 {}+c_t^{\mathrm{T}}\left(M_{t+1}^j - \left(K^z_{t+1}\right)^{\mathrm{T}}
 L^+_{t+1} K_{t+1}^{j-1}\right)\,,\\[2pt]
    j=2,\ldots , N-t+1\,,\enskip t\hm=0,1,\ldots , N-1\,;
  \label{e6-bos}
  \end{multline}

  \vspace*{-12pt}

 \begin{equation}
 \left.
 \begin{array}{l}
 K_N^z=\left(Q_N c_N+R_N\right)^{\mathrm{T}}\alpha_N Q_N b_N\,;\\[9pt]
 K_N^s= \left(Q_N c_N +R_N\right)^{\mathrm{T}}\alpha_N S_N\,;\\[9pt]
 K_N^0 =\left(Q_N c_N+ R_N\right)^{\mathrm{T}} \alpha_N Q_N a_N\,;\\[9pt]
 K_N^1=\left(Q_N c_N+R_N\right)^{\mathrm{T}}\alpha_N P_N\,;
 \end{array}
 \right\}
 \label{e7-bos}
 \end{equation}
 \begin{equation}
 \left.
 \hspace*{-1mm}\begin{array}{l}
\!\!\! L_t = \left(Q_t c_t+R_t\right)^{\mathrm{T}} \alpha_t (Q_t c_t+R_t) +\beta_t +\gamma_t+{}\\[9pt]
\!\!\!{}+\beta_{t+1} +c_t^{\mathrm{T}} M^z_{t+1} c_t-
\left(\beta_{t+1} - c_t^{\mathrm{T}}(K^z_{t+1})^{\mathrm{T}}\right)^{\mathrm{T}}\times{}\\[9pt]
\hspace*{18mm}{}\times
L^+_{t+1} \left(\beta_{t+1} -c_t^{\mathrm{T}}(K^z_{t+1})^{\mathrm{T}}\right)\,;\\[9pt]
\!\!\! L_N =\left(Q_N c_N +R_N\right)^{\mathrm{T}} \alpha_N (Q_N c_N +R_N) +{}\\[9pt]
 \!\!\!\hspace*{48mm}{}+\beta_N +\gamma_N\,;
 \end{array}\!\!
 \right\}\!\!
 \label{e8-bos}
 \end{equation}
 \begin{equation}
 \left.
 \begin{array}{l}
 M_t^z=b_t^{\mathrm{T}} Q_t^{\mathrm{T}} \alpha_t Q_t b_t +{}\\[9pt]
 \hspace*{7mm}{}+b_t^{\mathrm{T}} \left(M^z_{t+1}-(K^z_{t+1})^{\mathrm{T}}
L^+_{t+1} K^z_{t+1}\right)b_t\,;\\[9pt]
 M_t^s =b_t^{\mathrm{T}} Q_t^{\mathrm{T}} \alpha_t S_t +{}\\[9pt]
 \hspace*{7mm}{}+b_t^{\mathrm{T}}\left(M^s_{t+1} -(K^z_{t+1})^{\mathrm{T}}
L^+_{t+1} K^s_{t+1}\right)\,;\\[9pt]
 M_t^0 =b_t^{\mathrm{T}} Q_t^{\mathrm{T}} \alpha_t Q_t a_t +{}\\[9pt]
 \hspace*{7mm}{}+b_t^{\mathrm{T}} \left(M^z_{t+1}-(K^z_{t+1})^{\mathrm{T}}
L^+_{t+1} K^z_{t+1}\right)a_t\,;\\[9pt]
 M_t^1 = b_t^{\mathrm{T}} Q_t^{\mathrm{T}} \alpha_t P_t +{}\\[9pt]
 \hspace*{7mm}{}+b_t^{\mathrm{T}} \left(M^0_{t+1}-(K^z_{t+1})^{\mathrm{T}}
L^+_{t+1} K^0_{t+1}\right)\,;\\[9pt]
 M_t^j = b_t^{\mathrm{T}}\left(M_{t+1}^{j-1} -(K^z_{t+1})^{\mathrm{T}} L^+_{t+1}
 K_{t+1}^{j-1}\right)\,,\\
 \hspace*{2mm}j=2, \ldots , N-t+1\,,\enskip  t=0,1,\ldots , N-1\,;
 \end{array}
 \right\}
 \label{e9-bos}
 \end{equation}

 \vspace*{-6pt}

 \begin{equation}
\hspace*{-3mm}\left.
 \begin{array}{l}
 M_N^z= b_N^{\mathrm{T}} Q_N^{\mathrm{T}} \alpha_N Q_N b_N\,;\enskip M_N^s =b_N^{\mathrm{T}} Q_N^{\mathrm{T}}
\alpha_N S_N\,;\\[9pt]
 M_N^0 =b_N^{\mathrm{T}} Q_N^{\mathrm{T}} \alpha_N Q_N a_N\,;\enskip M_N^1 =b_N^{\mathrm{T}} Q_N^{\mathrm{T}}
\alpha_N P_N\,.
 \end{array}
 \right\}\!\!
 \label{e10-bos}
 \end{equation}

 %\smallskip

 \noindent
 Д\,о\,к\,а\,з\,а\,т\,е\,л\,ь\,с\,т\,в\,о\,.\ \ Дополнительно к~(\ref{e5-bos})
обозначим
$$
M_t\eqdelta \sum\limits_{j=0}^{N-t+1} M^j_t\overline{y}(t+j,t)\hm+M_t^s\,.
$$
Для доказательства утверждения
воспользуемся методом динамического программирования~[12]. Обозначим
функцию Беллмана:
\begin{multline*}
 W_t \eqdelta \min\limits_{u_t,\ldots , u_N} \sum\limits^N_{\tau=t}
\mathbb{M} \left[ \| P_\tau y_{\tau+1} +Q_\tau z_{\tau+1} +R_\tau
u_\tau+{}\right.\\
\left.{}+S_\tau\|^2_{\alpha_\tau} + \| u_\tau -u_{\tau-1}\|^2_{\beta_\tau} +\|
u_\tau\|^2_{\gamma_\tau} \vert \mathfrak{J}_t^y\right]\,.
 \end{multline*}

 При $t=N$ с учетом~(\ref{e1-bos}) имеем:
 \begin{multline*}
W_N = \min\limits_{u_N} \mathbb{M}\Big[ \left\Vert P_N y_{N+1} +
Q_N a_N y_N +{}\right.\\
\left.{}+
Q_N b_N z_N +
\left( Q_N c_N + R_N\right) u_N +S_N\right\Vert^2_{\alpha_N} +{}\\
{}+
\left\Vert u_N - u_{N-1}\right\Vert^2_{\beta_N} +
\left\Vert u_N\right\Vert^2_{\gamma_N} \Big\vert \mathfrak{J}_N^y \Big] = {}\\
{}= \min\limits_{u_N} \mathbb{M}
\left[
u_N^{\mathrm{T}} \Big( \left( Q_N c_N + R_N\right)^{\mathrm{T}} \alpha_N
\left( Q_N c_N + R_N\right) +{}\right.\\
{}+
\beta_N + \gamma_N\Big) u_N - 2 u_N^{\mathrm{T}} \Big( \beta_N u_{N-1} -{}\\
{}- \left( Q_N c_N + R_N\right)^{\mathrm{T}} \alpha_N
\left(P_N y_{N+1} + Q_N a_N y_N + {}\right.\\
\left.{}+Q_N b_N z_N + S_N\right)\Big) +
\left\Vert P_N y_{N+1} + Q_N a_N y_N +{}\right.\\
\left.\left.{}+ Q_N b_N z_N
+S_N\right\Vert^2_{\alpha_N} +
\left\Vert u_{N-1} \right\Vert^2_{\beta_N}
\Big\vert
\mathfrak{J}_N^y\right] = {}\\
{} = \min\limits_{u_N}
\left( u_N^{\mathrm{T}} L_N u_N - 2 u_N^{\mathrm{T}} \left(
\beta_N u_{N-1} - K_N^z z_N -{}\right.\right.\\
\left.\left.{}- K_N^0
\overline{y} (N, N) - K_N^1 \overline{y} (N+1, N) - K_N^s\right)\right) +{}\\
{}+ \left\Vert u_{N-1}\right\Vert^2_{\beta_N}
+ \mathbb{M}\Big[ \left\Vert P_N y_{N+1} + Q_N a_N y_N +{}\right.\\
\left.{}+ Q_N b_N z_N +
S_N\right\Vert^2_{\alpha_N}
\Big\vert \mathfrak{J}_N^y \Big]\,.
\end{multline*}
Отсюда с учетом обозначения
$$
K_N\eqdelta K_N^0 \overline{y}(N,N)+ K_N^1\overline{y}\left(N+1,N\right)+K_N^s
$$
и в предположении $L_N\hm\geq 0$
вытекает, что  (см.~(\ref{e4-bos}))
$$
u_N^*= L_N^+\left(\beta_N u_{N-1}- K_N^z z_N- K_N\right)
$$
 и $u_N^*$ имеет минимальную евклидову норму~[13].
Здесь были использованы обозначения, введенные
в~(\ref{e7-bos}) и~(\ref{e8-bos}). Кроме того, здесь и далее учитывается
очевидное равенство $\overline{y}(t,t)\eqdelta y_t$.

 Подставляя $u_N^*$ в полученное выражение для $W_N$, получаем:
 \begin{multline*}
\hspace*{-3mm}W_N= -\left(\beta_N u_{N-1} -K^z_N z_N -K_N\right)^{\mathrm{T}}\! L^+_N
 \left(\beta_N u_{N-1} -{}\right.\\
\left. {}-
K^z_N z_N -K_N\right) +\| u_{N-1}\|^2_{\beta_N}+
\mathbb{M} \Big[ \| P_N y_{N+1} +{}\\
 {}+Q_N a_N y_N +Q_N b_N z_N +S_N
\|^2_{\alpha_N}\Big\vert \mathfrak{J}^y_N\Big]={}\\
 {}=-(\beta_N u_{N-1} -K^z_N z_N -K_N)^{\mathrm{T}} L^+_N \left(\beta_N u_{N-1} -{}\right.\\
 \left. {}-
K^z_N z_N -K_N\right) +\|u_{N-1}\|^2_{\beta_N}+{}\\
{}+
z_N^{\mathrm{T}} b_N^{\mathrm{T}} Q_N^{\mathrm{T}} \alpha_N Q_N b_N z_N +{}\\
{}+2z_N^{\mathrm{T}} b_N^{\mathrm{T}} Q_N^{\mathrm{T}} \alpha_N
(P_N \overline{y} (N+1,N) +Q_N a_N \overline{y} (N,N)+{}\\
{}+S_N)+\mathbb{M}\left[ \| P_N y_{N+1} +Q_N a_N y_N +S_N
\|^2_{\alpha_N}\Big\vert \mathfrak{J}_N^y\right]={}
\end{multline*}

\noindent
\begin{multline*}
 {}=-(\beta_N u_{N-1} -K^z_N z_N -K_N)^{\mathrm{T}} L_N^+ \left(\beta_N u_{N-1} -{}\right.\\[1pt]
\left. {}-
K_N^z z_N -K_N\right)+\|u_{N-1}\|^2_{\beta_N}+
z_N^{\mathrm{T}} M^z_N z_N +{}\\[1pt]
{}+2z_N^{\mathrm{T}}(M_N^0 \overline{y}(N,N) +M^1_N
\overline{y}(N+1,N) +M_N^s)+{}\\[1pt]
 {}+\mathbb{M}\left[ \| P_N y_{N+1} +Q_N a_N y_N +S_N \|^2_{\alpha_N}
\Big\vert \mathfrak{J}^y_N\right]\,.
 \end{multline*}
 Здесь были использованы обозначения, введенные в~(\ref{e10-bos}).
Группируя далее слагаемые при $u_{N-1}$ и $z_N$, окончательно получаем:
 \begin{multline*}
 W_N =u_{N-1}^{\mathrm{T}} (\beta_N -\beta_N L_N^+ \beta_N) u_{N-1} +{}\\[1pt]
 {}+
 2u^{\mathrm{T}}_{N-1} \beta_N L^+_N K^z_N z_N +2u^{\mathrm{T}}_{N-1} \beta_N L^+_N
K_N+{}\\[1pt]
 {} +2 z_N^{\mathrm{T}} (M_N-(K_N^z)^{\mathrm{T}} L_N^+ K_N) +{}\\[1pt]
 {}+z_N^{\mathrm{T}} (M_N^z-(K_N^z)^{\mathrm{T}}
L_N^+ K_N^z)z_N+\mathbb{M}\Big[ \| P_N y_{N+1} +{}\\[1pt]
 {}+Q_N a_N y_N +S_N \|^2_{\alpha_N}
-\| K_N\|^2_{L^+_N}\Big\vert \mathfrak{J}^y_N\Big].
 \end{multline*}
 Здесь учтена $\mathfrak{J}^y_N$-измеримость $K_N$ из~(\ref{e5-bos}).

 Предположим теперь, что полученное выражение для $W_N$ имеет
место и для $W_t$, т.\,е.
 \begin{multline}
 W_t = u^{\mathrm{T}}_{t-1} (\beta_t -\beta_t L_t^+ \beta_t) u_{t-1} +
 2u^{\mathrm{T}}_{t-1} \beta_t L_t^+ K_t^z z_t +{}\\
 {}+2u^{\mathrm{T}}_{t-1} \beta_t L_T^+ K_t+
 2z_t^{\mathrm{T}} \left(M_t -(K_t^z)^{\mathrm{T}} L_t^+ K_t\right) +{}\\
 {}+
 z_t^{\mathrm{T}}\left(M_t^z-\left(K_t^z\right)^{\mathrm{T}} L_t^+
K_t^z\right)z_t +\mathbb{M} \left[ A_t\Big\vert \mathfrak{J}_t^y\right]\,,
 \label{e11-bos}
 \end{multline}
где обозначено:
\begin{align*}
A_t &= \| P_t y_{t+1} +Q_t a_t y_t +S_t\|^2_{\alpha_t} -
\|K_t\|^2_{L_t^+}+{}\\
&\hspace*{1mm}{}+ 2y_t^{\mathrm{T}} a_t^{\mathrm{T}}
\left(M_t-(K_t^z)^{\mathrm{T}} L_t^+ K_t\right) +
y_t^{\mathrm{T}} a_t^{\mathrm{T}} \left(M^z_{t+1} -{}\right.\\
&\hspace*{3mm}\left.{}-\left(K^z_{t+1}\right)^{\mathrm{T}} L^+_{t+1} K^z_{t+1}\right) a_t y_t +A_{t+1}\,;\\
A_N&= \| P_N y_{N+1} +Q_N a_N y_N +S_N \|^2_{\alpha_N} -
\|K_N\|^2_{L^+_N}\,.
%\label{e12-bos}
\end{align*}

 Для доказательства~(\ref{e11-bos}) и~(\ref{e4-bos}) для $t\hm-1$ запишем
уравнение Беллмана для $W_{t-1}$:
 \begin{multline*}
 W_{t-1} = %{}\\
% {}=
\min\limits_{u_{t-1}} \mathbb{M}\Big[ \left\Vert P_{t-1} y_t +
Q_{t-1} z_t + R_{t-1} u_{t-1} +{}\right.\\
\left.{}+ S_{t-1} \right\Vert^2_{\alpha_{t-1}} +
\left\Vert u_{t-1} - u_{t-2} \right\Vert^2_{\beta_{t-1}} +{}\\
{}+
 \left\Vert u_{t-1}\right\Vert^2_{\gamma_{t-1}} +
 W_t \Big\vert \mathfrak{J}_{t-1}^y  \Big]\,.
 \end{multline*}

С учетом~(\ref{e1-bos}) и~(\ref{e11-bos}) имеем:
\begin{multline*}
W_{t-1} = \min\limits_{u_{t-1}} \mathbb{M} \Big[\!
 \left\Vert P_{t-1} y_t + Q_{t-1}a_{t-1} y_{t-1} +{}\right.\\[1pt]
 {}+
 Q_{t-1} b_{t-1} z_{t-1}+ {}+\left(Q_{t-1} c_{t-1} +
R_{t-1}\right) u_{t-1} + {}\\[1pt]
\left.{}+S_{t-1} \right\Vert^2_{\alpha_{t-1}} +
\left\Vert u_{t-1} - u_{t-2} \right\Vert^2_{\beta_{t-1}} +{}\\[1pt]
{}+
\left\Vert u_{t-1}\right\Vert^2_{\gamma_{t-1}} +
u_{t-1}^{\mathrm{T}}\left( \beta_t - \beta_t L_t^+ \beta_t \right)u_{t-1} + {}\\[1pt]
{}+2 u_{t-1}^{\mathrm{T}} \beta_t L_t^+ K_t^z \left( a_{t-1} y_{t-1} +
b_{t-1} z_{t-1} +{}\right.\\[1pt]
\left.{}+
c_{t-1} u_{t-1} \right) + 2 u_{t-1}^{\mathrm{T}} \beta_t L_t^+ K_t +
2\left( a_{t-1} y_{t-1} +{}\right.
\end{multline*}

\noindent
\begin{multline*}
\left.{}+ b_{t-1} z_{t-1} + c_{t-1} u_{t-1} \right)^{\mathrm{T}}
\left(M_t - \left( K_t^z\right)^{\mathrm{T}} L_t^+ K_t \right) +{}\\
{}+ \left( a_{t-1} y_{t-1} + b_{t-1} z_{t-1} + c_{t-1} u_{t-1} \right)^{\mathrm{T}}
\Big(M_t^z -{}\\
{}- \left( K_t^z\right)^{\mathrm{T}} L_t^+ K_t^z \Big)
\left( a_{t-1} y_{t-1} + b_{t-1} z_{t-1} + c_{t-1} u_{t-1} \right) +{}\\
{}+A_t \Big\vert
\mathfrak{J}_{t-1}^y\Big]  ={}\\
= \min\limits_{u_{t-1}} \mathbb{M} \Big[
u_{t-1}^{\mathrm{T}} \Big(
 \left( Q_{t-1} c_{t-1} + R_{t-1}\right)^{\mathrm{T}} \alpha_{t-1}\times\\
 {}\times
\left( Q_{t-1} c_{t-1} + R_{t-1}\right) + \beta_{t-1} + \gamma_{t-1} +
\beta_t -\beta_t L_t^+ \beta_t
+{}\\
{}+ 2 \beta_t L_t^+ K_t^z c_{t-1} + c_{t-1}^{\mathrm{T}}
\left(M_t^z - \left( K_t^z\right)^{\mathrm{T}} L_t^+ K_t^z \right) c_{t-1} %\right)
\Big)\times{}\\
{}\times u_{t-1} - 2 u_{t-1}^{\mathrm{T}} \left(
\beta_{t-1} u_{t-2} - \left(Q_{t-1} c_{t-1} + R_{t-1}\right)^{\mathrm{T}}\times{}\right.\\
{}\times \alpha_{t-1}
\left(P_{t-1} y_t + Q_{t-1} a_{t-1} y_{t-1} + Q_{t-1} b_{t-1} z_{t-1} +{}\right.\\
\left.{}+ S_{t-1} \right) - \beta_t L_t^+ K_t^z \left( a_{t-1} y_{t-1} + b_{t-1} z_{t-1} \right) -{}\\
{}-
\beta_t L_t^+ K_t - c_{t-1}^{\mathrm{T}}
\left(M_t - \left( K_t^z\right)^{\mathrm{T}} L_t^+ K_t \right) -{}\\
\left.{}-c_{t-1}^{\mathrm{T}}\!\left(\!
M_t^z - \left( K_t^z\right)^{\mathrm{T}}\! \!L_t^+ K_t^z \right)\!
\left( a_{t-1} y_{t-1} + b_{t-1} z_{t-1}\right)\!\right)
+{}\\
{}+\left\Vert P_{t-1} y_t + Q_{t-1}a_{t-1} y_{t-1} + Q_{t-1} b_{t-1} z_{t-1}
+ \right.\\
\left.{}+S_{t-1} \right\Vert^2_{\alpha_{t-1}} +
\left\Vert u_{t-2} \right\Vert^2_{\beta_{t-1}} +{}\\
{}+ 2\left( a_{t-1} y_{t-1} + b_{t-1} z_{t-1} \right)^{\mathrm{T}}
\left(M_t - \left( K_t^z\right)^{\mathrm{T}} L_t^+ K_t \right) +{}\\
{}+ \left( a_{t-1} y_{t-1} + b_{t-1} z_{t-1} \right)^{\mathrm{T}}
\left(M_t^z - \left( K_t^z\right)^{\mathrm{T}} L_t^+ K_t^z \right)\times{}\\
{}\times
\left( a_{t-1} y_{t-1} + b_{t-1} z_{t-1} \right) +A_t
\Big\vert
\mathfrak{J}_{t-1}^y \Big] ={}\\
= \min\limits_{u_{t-1}}
\left( u_{t-1}^{\mathrm{T}} L_{t-1} u_{t-1} -
2 u_{t-1}^{\mathrm{T}} \left( \beta_{t-1} u_{t-2} -{}\right.\right.\\
\left.\left.{}- K_{t-1}^z
z_{t-1} - K_{t-1} \right)\right) +
\left\Vert u_{t-2} \right\Vert^2_{\beta_{t-1}} +{}\\
{}+ \mathbb{M} \Big[
\left\Vert P_{t-1} y_t + Q_{t-1}a_{t-1} y_{t-1} +{}\right.\\
\left.{}+ Q_{t-1} b_{t-1} z_{t-1} +
 S_{t-1} \right\Vert^2_{\alpha_{t-1}} +{}\\
{}+ 2\left( a_{t-1} y_{t-1} + b_{t-1} z_{t-1} \right)^{\mathrm{T}}
\left(M_t - \left( K_t^z\right)^{\mathrm{T}} L_t^+ K_t \right) +{}\\
{}+ \left( a_{t-1} y_{t-1} + b_{t-1} z_{t-1} \right)^{\mathrm{T}}
\left(M_t^z - \left( K_t^z\right)^{\mathrm{T}} L_t^+ K_t^z \right)\times{}\\
{}\times
\left( a_{t-1} y_{t-1} + b_{t-1} z_{t-1} \right) +A_t
\Big\vert
\mathfrak{J}_{t-1}^y \Big] \,,
\end{multline*}
откуда в предположении $L_{t-1}\hm\geq 0$ вытекает (см.~(\ref{e4-bos}))
\begin{equation}
u^*_{t-1} = L^+_{t-1} \left( \beta_{t-1} u_{t-2} - K^z_{t-1} z_{t-1} - K_{t-1}\right)
\label{e13-bos}
\end{equation}
и $u_{t-1}^*$ имеет минимальную евклидову норму~[13].

 Выше были использованы обозначения, введенные в~(6)--(\ref{e6-bos})
и~(\ref{e8-bos}), и формулы полного математического ожидания
\begin{align*}
 \mathbb{M}\left[\left. A_t \right\vert \left.\mathfrak{J}_t^y \right\vert
\mathfrak{J}_{t-1}^y \right] &= \mathbb{M}\left[ A_t \left\vert \mathfrak{J}_{t-1}^y
\right. \right]\,;
\\
 \mathbb{M}\left[\overline{y} \left( t +j, t\right) \left\vert
\mathfrak{J}_{t-1}^y \right. \right]& = \overline{y} \left( t+j, t-1\right)\,.
 \end{align*}

 Подставляя $u_{t-1}^*$ в полученное выражение для $W_{t-1}$,
получаем:

\pagebreak

\noindent
\begin{multline*}
W_{t-1} =
-\left( \beta_{t-1} u_{t-2} - K_{t-1}^z z_{t-1} - K_{t-1} \right)^{\mathrm{T}}
L_{t-1}^+\times{}\\
{}\times
\left(\beta_{t-1} u_{t-2} - K_{t-1}^z z_{t-1} - K_{t-1} \right) +
\left\Vert u_{t-2}\right\Vert^2_{\beta_{t-1}} +{}\\
{}+ \mathbb{M} \Big[
 \left\Vert P_{t-1} y_t +{}\right.\\
\left. {}+ Q_{t-1}a_{t-1} y_{t-1} + Q_{t-1} b_{t-1} z_{t-1}
+ S_{t-1} \right\Vert^2_{\alpha_{t-1}} +{}\\
{}+ 2\left( a_{t-1} y_{t-1} + b_{t-1} z_{t-1} \right)^{\mathrm{T}}
\left(M_t - \left( K_t^z\right)^{\mathrm{T}} L_t^+ K_t \right) +{}\\
{}+ \left( a_{t-1} y_{t-1} + b_{t-1} z_{t-1} \right)^{\mathrm{T}}
\left(M_t^z - \left( K_t^z\right)^{\mathrm{T}} L_t^+ K_t^z \right)\times{}\\
{}\times
\left( a_{t-1} y_{t-1} + b_{t-1} z_{t-1}   \right) +A_t
\Big\vert
\mathfrak{J}_{t-1}^y \Big] ={}\\
{}= -\left( \beta_{t-1} u_{t-2} - K_{t-1}^z z_{t-1} - K_{t-1} \right)^{\mathrm{T}}
L_{t-1}^+ \times{}\\
{}\times
\left(\beta_{t-1} u_{t-2} - K_{t-1}^z z_{t-1} - K_{t-1} \right) +
\left\Vert u_{t-2}\right\Vert^2_{\beta_{t-1}} +{}\\
+z_{t-1}^{\mathrm{T}} \left( b_{t-1}^{\mathrm{T}} Q_{t-1}^{\mathrm{T}}
\alpha_{t-1} Q_{t-1} b_{t-1} +{}\right.\\
\left.{}+b_{t-1}^{\mathrm{T}}
\left(M_t^z - \left( K_t^z\right)^{\mathrm{T}} L_t^+ K_t^z \right) b_{t-1}\right) z_{t-1} +{}\\[9pt]
{}+ 2 z_{t-1}^{\mathrm{T}}
\Big(
b_{t-1}^{\mathrm{T}} Q_{t-1}^{\mathrm{T}} \alpha_{t-1} \big(
P_{t-1} \overline{y}(t, t-1) +{}\\
{}+ Q_{t-1} a_{t-1}
\overline{y}(t-1, t-1) + S_{t-1} \big) +{}\\
{}+ b_{t-1}^{\mathrm{T}} \left(\mathbb{M}\left[ M_t \left\vert \mathfrak{J}_{t-1}^y \right. \right]
- \left( K_t^z\right)^{\mathrm{T}} L_t^+ \mathbb{M}\left[ K_t \left\vert \mathfrak{J}_{t-1}^y
\right. \right] \right) +{}\\
{}+ b_{t-1}^{\mathrm{T}} \left(M_t^z - \left( K_t^z\right)^{\mathrm{T}} L_t^+ K_t^z \right) a_{t-1}
 \overline{y} (t-1, t-1)
\Big) +{}\\
{}+ \mathbb{M} \Big[
\left\Vert P_{t-1} y_t + Q_{t-1}a_{t-1} y_{t-1} +
S_{t-1} \right\Vert^2_{\alpha_{t-1}} +{}\\
{}+ 2 y_{t-1}^{\mathrm{T}} a_{t-1}^{\mathrm{T}}
\left(M_t - \left( K_t^z\right)^{\mathrm{T}} L_t^+ K_t \right) +{}\\
{}+
y_{t-1}^{\mathrm{T}} a_{t-1}^{\mathrm{T}} \left(
M_t^z - \left( K_t^z\right)^{\mathrm{T}} L_t^+ K_t^z \right)
a_{t-1} y_{t-1} +{}\\
{}+A_t \Big\vert
\mathfrak{J}_{t-1}^y \Big] ={}\\
{}= -\left( \beta_{t-1} u_{t-2} - K_{t-1}^z z_{t-1} - K_{t-1} \right)^{\mathrm{T}}
L_{t-1}^+\times{}\\
{}\times
\left(\beta_{t-1} u_{t-2} - K_{t-1}^z z_{t-1} - K_{t-1} \right) + {}\\
{}+ \left\Vert u_{t-2}\right\Vert^2_{\beta_{t-1}} +
 z_{t-1}^{\mathrm{T}} M_{t-1}^z z_{t-1} + 2 z_{t-1}^{\mathrm{T}} M_{t-1} +{}\\[9pt]
{}+ \mathbb{M} \Big[
\left\Vert P_{t-1} y_t + Q_{t-1}a_{t-1} y_{t-1} +
S_{t-1} \right\Vert^2_{\alpha_{t-1}} +{}\\
{}+ 2 y_{t-1}^{\mathrm{T}} a_{t-1}^{\mathrm{T}}
\left(M_t - \left( K_t^z\right)^{\mathrm{T}} L_t^+ K_t \right) +{}\\
{}+
y_{t-1}^{\mathrm{T}} a_{t-1}^{\mathrm{T}}
\left(M_t^z - \left( K_t^z\right)^{\mathrm{T}} L_t^+ K_t^z \right)
a_{t-1} y_{t-1} +{}\\
{}+A_t
\Big\vert
\mathfrak{J}_{t-1}^y \Big] \,.
\end{multline*}

%\begin{multicols}{2}
\noindent
 Здесь были использованы обозначения, введенные в~(\ref{e9-bos}).
Группируя далее слагаемые при $u_{t-2}$ и $z_{t-1}$ и учитывая
$\mathfrak{J}^y_{t-1}$-из\-ме\-ри\-мость $K_{t-1}$ из~(\ref{e5-bos}),
окончательно получаем~(\ref{e11-bos}) для $W_{t-1}$.

 Подставляя далее в~(\ref{e13-bos}) $u_{t-2}\hm= u_{t-2}^*$ (в том числе
учитывая в~(\ref{e13-bos}) для $u_0^*$ условие $\beta_0\hm=0$), окончательно
получаем~(\ref{e4-bos}).

 Для завершения доказательства осталось заметить, что выполнение
неравенства $L_t\hm\geq0$ для всех~$t$ очевидно. Действительно, в выкладках,
проделанных для функции Беллмана $W_{t-1}$, показано, что ее можно
представить в виде минимума квадратичной формы с матрицей
 $L_{t-1}$ при квадратичных членах~$u_{t-1}$. Неотрицательная
определенность $L_{t-1}$ вытекает, таким образом, из не\-отри\-ца\-тель\-ности
$W_{t-1}$, имеющей место по условию задачи.

 Теорема доказана.

 \smallskip

 Согласно полученному результату решение $u_t^*$ является линейной
комбинацией предыдущего управления $u_{t-1}^*$, последнего выхода~$z_t$ и
оптимальных в среднем квадратическом прогнозов состояния $K_t$
из~(\ref{e5-bos}) вплоть до горизонта времени $N+1$. Соответственно, с
каждым шагом по времени число сла\-га\-емых
 в~(\ref{e5-bos}) уменьшается на единицу. Коэффициенты для
вычисления~$K_t$ заданы рекуррентными (в обратном времени)
соотношениями~(6)--(10) и~(\ref{e9-bos}): коэффициент~$K_t^j$,
соответствующий прогнозу на $j$~шагов ($K_t^0$~--- множитель при~$y_t$),
вычисляется через коэффициенты $K_{t+1}^{j-1}$,
 $M_{t+1}^{j-1}$, определенные на предыдущем шаге рекурсии. На
каждом шаге, кроме того, вычисляются коэффициенты $K_t^z$, $M_t^z$,
$K_t^s$, $M_t^s$ и $L_t$~--- соответственно множитель при~$z_t$, свободный
член и общий нормировочный коэффициент. Рекуррентное вычисление
коэффициентов стартует при $t\hm=N$ по формулам~(\ref{e7-bos})
и~(\ref{e10-bos}).

 Полученный результат объясняет, почему задача изначально не
формулировалась как классическая задача ли\-ней\-но-квад\-ра\-тич\-но\-го
управления путем расширения вектора состояния~$y_t$. Действительно,
расширив вектор состояния вектором~$u_t$, можно избавиться от
слагаемого~$u_t$ в уравнении входа~$z_t$, сохранив квадратичность критерия
($c_t\hm=0$). Но тогда придется рассматривать зависимость прогнозов~$y_t$
от управ\-ле\-ний~$u_t$, что приведет либо к не\-об\-хо\-ди\-мости учета влияния
управ\-ле\-ний на точ\-ность прогнозирования (дуальное управление), либо
ограничиваться линейностью~$y_t$.

 Кроме того, надо отметить, что невозможны и другие упрощения в
предложенной модели и критерии. Так, не удается за счет расширения вектора
выхода~$z_t$ избавиться от слагаемого~$z_{t-1}$ в уравнении входа $z_t$
($b_t\hm=0$). Также нельзя избавиться от слагаемого
 $\| u_t\hm- u_{t-1}\|^2_{\beta_t}$ в критерии, расширив вектор~$u_t$
(такое расширение вообще искажает смысл задачи).

 Таким образом, предложенную в данной работе постановку можно
рассматривать как вариант обобщения ли\-ней\-но-квад\-ра\-тич\-ной задачи на
случай нелинейного состояния~$y_t$: нелинейность может иметь место, но не
должна влиять на выбор управления.

 Сделанные замечания показывают также и возможность дальнейшего
обобщения критерия. Например, целью оптимизации можно объявить
отслеживание $u_t$ линейной комбинации выходов~$z_\tau$, $t\hm\leq
\tau\hm\leq N+1$, или вместо точечного штрафа за смену управляющего
воздействия (слагаемое $\| u_t\hm- u_{t-1}\|^2_{\beta_t}$) задавать
интегральный. Представляется, что подобные обобщения будут иметь смысл
только в том случае, если найдут обоснование
 ка\-ки\-ми-ли\-бо прикладными постановками.

 \vspace*{-6pt}

\section{Вопросы практической реализуемости}

\vspace*{-2pt}

 Обращаясь к вопросу практического применения полученного результата,
в частности реализуемости оптимального алгоритма, еще раз отметим, что
целевой функционал сформировался в связи с оптимизацией
функционирования программной системы. Логично предположить, что для
разных программ вполне могут возникать аналогичные описанным выше цели
оптимизации функционирования~--- распределения ресурсов. При этом
бо\-лее-ме\-нее общим, по-ви\-ди\-мо\-му, можно считать и физический смысл
неопределенности, описываемой в предложенной модели процессом $y_t$, а
именно: $y_t$ характеризует активность пользователей, обращающихся к
программной системе. Процесс может описывать как число пользователей,
применяющих программную систему, так и число\linebreak формируемых ими запросов
или команд. Кон\-крет\-ные математические модели~$y_t$ могут быть до\-воль\-но
разнообразными. Практически от модели~$y_t$\linebreak требуется только возможность
прогнозирования.\linebreak
 В~любом случае для рассматриваемой задачи оптимиза\-ции
задачи собственно оптимизации и оценивания фазового процесса разделены,
как в традиционной задаче ли\-ней\-но-квад\-ра\-тич\-но\-го управ\-ле\-ния, что является
еще одним важным достоинством квад\-ра\-тич\-но\-го функционала качества.

 Реализуемость алгоритма оптимизации даже применительно к
обозначенной области приложения зависит прежде всего от выбора горизонта
управления $N+1$. Понятно, что, выбрав слишком большой горизонт, придется
проводить значительный объем вычислений в связи с расчетом прогнозов.
Использование малого горизонта не вполне понятно в связи с реальной целью
оптимизации~--- распределять ресурсы эффективно все время, пока программа
работает. Рассуждения здесь зависят единственно от характера априорной
информации об~$y_t$. Если $y_t$ свойственны
 ка\-кие-то прогнозируемые особенности (например, резкое увеличение
пользовательской активности во вполне определенные часы, дни), то их нельзя
не учитывать. На практике такую ситуацию действительно можно видеть, и
характеризуется она временн$\acute{\mbox{о}}$й периодичностью, например суточными или
недельными циклами. Это позволяет вполне обоснованно делить весь
потенциально бесконечный период функционирования программы на
небольшие повторяющиеся час\-ти, равные периодам.

 Если же процесс $y_t$ не имеет таких особых моментов, например
является эргодическим, то можно ограничиться и очень малым горизонтом.
Проведенные в работах~[10, 11] расчеты показывают, что в действительности
можно ограничиться значениями $N\hm=1$ и даже $N\hm=0$. Это означает,
что вместо интегрального квадратичного критерия~(\ref{e2-bos}) можно
использовать ло\-каль\-но-оп\-ти\-маль\-ный подход~[14]. Так, одношаговое
ло\-каль\-но-оп\-ти\-маль\-ное управление $u_t^L$ минимизирует целевой
функционал
 \begin{multline*}
 J^L(u_t) =\mathbb{M}\left[ \| P_t y_{t+1} +Q_t z_{t+1} +R_t u_t
+S_t\|^2_{\alpha_t} +{}\right.\\[9pt]
\left.{}+\| u_t- u^L_{t-1} \|^2_{\beta_t} +
\| u_t\|^2_{\gamma_t}\right]\,,
\end{multline*}
т.\,е.\ одно слагаемое исходного функционала~(\ref{e2-bos}).\linebreak Соответственно, в
двухшаговое ло\-каль\-но-оп\-ти\-маль\-ное управление надо включить два
сла\-га\-емых. В~расчетах надо сравнивать значения~(\ref{e2-bos}), до\-став\-ля\-емые
субоптимальными (ло\-каль\-но-оп\-ти\-маль\-ными) и оптимальным алгоритмами.
Примеры вы\-чис\-лений, выполненные в работах~[10, 11], показывают, что
потери при этом составляют 2\%--5\%, что с учетом ограниченной адекватности
любой модели можно отнести к статистической погрешности.

 Собственно, вывод о целесообразности практического применения
именно ло\-каль\-но-оп\-ти\-маль\-но\-го подхода и является основным
результатом обсуждения практической реализуемости. Значение же
оптимального алгоритма, как чаще всего и бывает, заключается в определении
некоторого эталона, близость к которому и является основанием для
применения субоптимального алгоритма.

\vspace*{-6pt}

\section{Заключение}

\vspace*{-2pt}

 Задача, сформулированная в данной работе как задача оптимизации
выхода стохастической системы, очевидно, может быть обобщена на случай
косвенных наблюдений. Неопределенность, описываемая процессом~$y_t$, при
этом будет считаться состоянием стохастической динамической системы
наблюдения, а выход~$z_t$~--- косвенными наблюдениями. При этом в
уравнение~(1) логичным будет добавить шум, описывающий ошибку
наблюдения. Вторым направлением для дальнейших исследований является
постановка аналогичной задачи для системы с непрерывным временем.
В~общем случае здесь выход~$z_t$ будет описываться стохастическим
дифференциальным уравнением.



{\small\frenchspacing
 {%\baselineskip=10.8pt
 \addcontentsline{toc}{section}{References}
 \begin{thebibliography}{99}

\bibitem{1-bos}
\Au{Гитман Л.\,Дж., Джонк М.\,Д.} Основы инвестирования~/ Пер. с
англ.~--- М.: Дело, 1997. 810~с. (\Au{Gitman~L.\,J., Joehnk~M.\,D.}
Fundamentals of investing.~--- 4th ed.~--- N.Y.: Harper \& Row,
1990. 1008~p.)

\bibitem{2-bos}
ГОСТ 7.70-2003. СИБИД. Описание баз данных и машиночитаемых
информационных ресурсов. Состав и обозначение характеристик.~--- М.:
Изд-во стандартов, 2003. 11~с.
\bibitem{3-bos}
ГОСТ 28195-89. Оценка качества программных средств. Общие положения.~---
М.: Изд-во стандартов, 2001. 39~с.
\bibitem{4-bos}
\Au{Таненбаум Э.\,С., Вудхалл А.\,С.} Операционные системы. Разработка и
реализация~/ Пер. с англ.~--- 3-е изд.~--- СПб.: Питер, 2007. 704~с.
(\Au{Tanenbaum~A.\,S., Woodhull~A.\,S.} Operating systems: Design and
implementation.~--- 3rd ed.~--- Upper Saddle River, NJ: Prentice Hall, 2006.
1080~p.)
\bibitem{5-bos}
\Au{Дейт К.\,Дж.} Введение в системы баз данных~/ Пер. с англ.~--- 8-е изд.~---
М.: Вильямс, 2005. 1328~с. (\Au{Date~C.\,J.} An introduction to database
systems.~--- 8th ed.~--- Reading, MA: Addison-Wesley, 2004. 1024~p.)
\bibitem{6-bos}
\Au{Els$\ddot{\mbox{a}}$sser~R., Monien B., Preis~R.} Diffusion schemes for load
balancing on heterogeneous networks~// Theory Comput. Syst., 2002.
Vol.~35. No.\,3. P.~305--320.
\bibitem{8-bos}
\Au{Low S.\,H., Paganini F., Doyle~J.\,C.} Internet congestion control~// IEEE
Control Syst. Magazine, 2002. Vol.~22. No.\,1. P.~28--43.
\bibitem{7-bos}
\Au{Welzl M.} Network congestion control.~--- N.Y.: Wiley, 2005. 263~p.
\bibitem{9-bos}
\Au{Босов А.\,В., Иванов А.\,В.} Программная инфраструктура
Информационного web-пор\-та\-ла РАН~// Информатика и её применения,
2007. Т.~1. Вып.~2. С.~39--53.
\bibitem{10-bos}
\Au{Босов А.\,В.} Задачи анализа и оптимизации для модели пользовательской
активности. Часть~2. Оптимизация внутренних ресурсов~// Информатика и её
применения, 2012. Т.~6. Вып.~1. С.~18--25.
\bibitem{11-bos}
\Au{Босов А.\,В.} Задачи анализа и оптимизации для модели пользовательской
активности. Часть~3. Оптимизация внешних ресурсов~// Информатика и её
применения, 2012. Т.~6. Вып.~2. С.~15--22.
\bibitem{12-bos}
\Au{Бертсекас Д., Шрив С.} Стохастическое оптимальное управление: случай
дискретного времени~/ Пер. с англ.~--- М.: Наука, 1985. 279~с.
(\Au{Bertsekas~D.\,P., Shreve~S.\,E.} Stochastic optimal control: The
discrete-time case.~--- N.Y.: Academic Press, 1978. 323~p.)
\bibitem{13-bos}
\Au{Алберт А.} Регрессия, псевдоинверсия и рекуррентное оценивание~/ Пер. с
англ.~--- М.: Наука, 1977. 224~с. (\Au{Albert~A.} Regression and the
Moore--Penrose pseudoinverse.~--- N.Y.: Academic Press, 1972. 179~p.)
\bibitem{14-bos}
\Au{Коган М.\,М., Неймарк Ю.\,И.} Адаптивное
ло\-каль\-но-оп\-ти\-маль\-ное управление~// Автоматика и телемеханика, 1987.
№\,8. С.~126--136.

\end{thebibliography}

 }
 }

\end{multicols}

\vspace*{-6pt}

\hfill{\small\textit{Поступила в редакцию 6.03.14}}

%\newpage

\vspace*{12pt}

\hrule

\vspace*{2pt}

\hrule

\def\tit{THE GENERALIZED PROBLEM OF~SOFTWARE SYSTEM RESOURCES DISTRIBUTION}

\def\titkol{The generalized problem of~software system resources distribution}

\def\aut{A.\,V.~Bosov}

\def\autkol{A.\,V.~Bosov}

\titel{\tit}{\aut}{\autkol}{\titkol}

\vspace*{-9pt}

\noindent
Institute of Informatics Problems, Russian Academy of Sciences,
44-2 Vavilov Str., Moscow 119333, Russian Federation


\def\leftfootline{\small{\textbf{\thepage}
\hfill INFORMATIKA I EE PRIMENENIYA~--- INFORMATICS AND
APPLICATIONS\ \ \ 2014\ \ \ volume~8\ \ \ issue\ 2}
}%
 \def\rightfootline{\small{INFORMATIKA I EE PRIMENENIYA~---
INFORMATICS AND APPLICATIONS\ \ \ 2014\ \ \ volume~8\ \ \ issue\ 2
\hfill \textbf{\thepage}}}

\vspace*{6pt}

\Abste{The paper presents the statement and the solution of the optimization
problem for a dynamic system with a linear output and the quadratic
performance criterion. System uncertainty is described by the observed
second-order stochastic process. The need to optimize
resource distribution of software systems gives practical
justification to the problem.
In such interpretation, the uncertainty of a system describes user
activity and the output describes running queries or the volume of the
requested memory. The goals of optimization are formalized by the quadratic
performance criterion of the general form. The criterion, in particular,
summarizes two problems of resource distribution of software systems
discussed earlier. The objective functional makes it possible,
in particular, to state the
problem of adequate program resources allocation (of threads, memory, etc.),
penalizing for unlimited spending. To solve the problem, the
method of dynamic programming is used. The optimal strategy is a linear
combination of the output and state predictions up to the control horizon.
In the context of computational complexity of the optimal strategy, the possibility
of its simplicity and of using the locally-optimal strategy is discussed.}

\KWE{software system; stochastic observation system; quadratic criterion;
dynamic programming}

\DOI{10.14357/19922264140204}

  \begin{multicols}{2}

\renewcommand{\bibname}{\protect\rmfamily References}
%\renewcommand{\bibname}{\large\protect\rm References}

{\small\frenchspacing
 {%\baselineskip=10.8pt
 \addcontentsline{toc}{section}{References}
 \begin{thebibliography}{99}

\bibitem{1-bos-1}
\Aue{Gitman, L.\,J., and M.\,D.~Joehnk}. 1990. \textit{Fundamentals of investing}.
4th ed. N.Y.: Harper \& Row. 1008~p.
\bibitem{2-bos-1}
GOST 7.70-2003 SIBID. 2003. Opisanie baz dannykh i mashinochitaemykh
informatsionnykh resursov. Sostav i oboznachenie kharakteristik [Description of data
bases and information resources. The composition and characteristics of the
designation]. Moscow: Standardinform Publs. 11~p.
\bibitem{3-bos-1}
GOST 28195-89. 2001. Otsenka kachestva programmnykh sredstv. Obshchie
polozheniya [Assessment of the quality of software. General provisions]. Moscow:
Standardinform Publs. 39~p.
\bibitem{4-bos-1}
\Aue{Tanenbaum, A.\,S., and A.\,S.~Woodhull}. 2006. \textit{Operating systems:
Design and implementation}. 3rd ed. Upper Saddle River, NJ: Prentice Hall. 1080~p.
\bibitem{5-bos-1}
\Aue{Date, C.\,J.} 2004. \textit{An introduction to database systems}. 8th ed.
Reading, MA: Addison-Wesley. 1024~p.
\bibitem{6-bos-1}
\Aue{Els$\ddot{\mbox{a}}$sser, R., B.~Monien, and R.~Preis}. 2002. Diffusion
schemes for load balancing on heterogeneous networks. \textit{Theory Comput.
Syst.} 35(3):305--320.

\bibitem{8-bos-1}
\Aue{Low, S.\,H., F.~Paganini, and J.\,C.~Doyle}. 2002. Internet congestion
control. \textit{IEEE Control Syst. Magazine} 22(1):28--43.

\bibitem{7-bos-1}
\Aue{Welzl, M.} 2005. \textit{Network congestion control}. N.Y.: Wiley. 263~p.
\bibitem{9-bos-1}
\Aue{Bosov, A.\,V., and A.\,V.~Ivanov}. 2007. Programmnaya infrastruktura
Informatsionnogo web-portala RAN [RAS Informational web-portal software
infrastructure]. \textit{Informatika i ee Primeneniya}~--- \textit{Inform. Appl.}
2(1):39--53.
\bibitem{10-bos-1}
\Aue{Bosov, A.\,V.} 2012. Zadachi analiza i optimizatsii dlya mo\-de\-li
pol'zovatel'skoy aktivnosti. Chast'~2. Optimizatsiya vnutrennikh resursov [Analysis
and optimization problems for some users activity model. Part~2. Internal resources
optimization]. \textit{Informatika i ee Primeneniya}~--- \textit{Inform. Appl.}
6(1):18--25.
\bibitem{11-bos-1}
\Aue{Bosov, A.\,V.} 2012. Zadachi analiza i optimizatsii dlya modeli
pol'zovatel'skoy aktivnosti. Chast'~3. Optimizatsiya vneshnikh resursov [Analysis and
optimization problems for some users activity model. Part~3. External resources
optimization]. \textit{Informatika i ee Primeneniya}~--- \textit{Inform. Appl.}
6(2):15--22.
\bibitem{12-bos-1}
\Aue{Bertsekas, D.\,P., and S.\,E.~Shreve}. 1978. \textit{Stochastic optimal control:
The discrete-time case}. N.Y.: Academic Press. 323~p.
\bibitem{13-bos-1}
\Aue{Albert, A.} 1972. \textit{Regression, and the Moore--Penrose pseudoinverse}.
N.Y.: Academic Press. 179~p.
\bibitem{14-bos-1}
\Aue{Kogan, M.\,M., and Ju.\,I.~Nejmark}. 1987. Adaptivnoe
lokal'no-optimal'noe upravlenie [Locally-optimal adaptive control].
\textit{Avtomatika i Telemehanika} [Automation and Remote Control]
8:126--136.

\end{thebibliography}

 }
 }

\end{multicols}

\vspace*{-6pt}

\hfill{\small\textit{Received March 6, 2014}}

\vspace*{-18pt}

\Contrl

\noindent
\textbf{Bosov Alexey V.}~(b.~1969)~--- Doctor of Science in technology,
Head of Department, Institute of Informatics Problems, Russian Academy of
Sciences, 44-2 Vavilov Str., Moscow 119333, Russian Federation;
AVBosov@ipiran.ru

\label{end\stat}

\renewcommand{\bibname}{\protect\rm Литература}