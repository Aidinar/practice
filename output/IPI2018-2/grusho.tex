\def\stat{grusho}

\def\tit{ИЕРАРХИЧЕСКИЙ МЕТОД ПОРОЖДЕНИЯ МЕТАДАННЫХ 
ДЛЯ~УПРАВЛЕНИЯ СЕТЕВЫМИ СОЕДИНЕНИЯМИ$^*$}

\def\titkol{Иерархический метод порождения метаданных 
для~управления сетевыми соединениями}

\def\aut{А.\,А.~Грушо$^1$, Е.\,Е.~Тимонина$^2$, С.\,Я.~Шоргин$^3$}

\def\autkol{А.\,А.~Грушо, Е.\,Е.~Тимонина, С.\,Я.~Шоргин}

\titel{\tit}{\aut}{\autkol}{\titkol}

\index{Грушо А.\,А.}
\index{Тимонина Е.\,Е.}
\index{Шоргин С.\,Я.}
\index{Grusho A.\,A.}
\index{Timonina E.\,E.}
\index{Shorgin S.\,Ya.}




{\renewcommand{\thefootnote}{\fnsymbol{footnote}} \footnotetext[1]
{Работа поддержана РНФ (проект 16-11-10227).}}


\renewcommand{\thefootnote}{\arabic{footnote}}
\footnotetext[1]{Институт проблем информатики Федерального исследовательского центра <<Информатика и~управление>> 
Российской академии наук, \mbox{grusho@yandex.ru}}
\footnotetext[2]{Институт проблем информатики Федерального исследовательского центра <<Информатика и~управление>> 
Российской академии наук, \mbox{eltimon@yandex.ru}}
\footnotetext[3]{Институт проблем информатики Федерального исследовательского центра 
<<Информатика и~управление>> 
Российской академии наук, \mbox{sshorgin@ipiran.ru}}

%\vspace*{8pt}

    

  
  \Abst{Важным классом угроз для распределенных 
ин\-фор\-ма\-ци\-он\-но-вы\-чис\-ли\-тель\-ных сис\-тем (РИВС)
является возможность организации нелегальных информационных взаимодействий и,~наоборот, 
запрет на разрешенные информационные взаимодействия. Для пред\-от\-вра\-ще\-ния этого класса 
угроз предложена организация управ\-ле\-ния со\-еди\-не\-ни\-ями с~по\-мощью метаданных. Метаданные 
формируются на основе математических моделей биз\-нес-про\-цес\-сов, 
которые в~РИВС
пред\-став\-ле\-ны со\-во\-куп\-ностью информационных 
технологий (ИТ). Разобраны два метода по\-стро\-ения метаданных для 
ИТ. Предложенный подход основан на иерархической декомпозиции 
ИТ и~задач.}
  
  \KW{информационная безопасность; распределенные ин\-фор\-ма\-ци\-он\-но-вы\-чис\-ли\-тель\-ные 
сис\-те\-мы; метаданные; иерархическая декомпозиция; составные задачи}

\DOI{10.14357/19922264180207}
  
%\vspace*{4pt}


\vskip 10pt plus 9pt minus 6pt

\thispagestyle{headings}

\begin{multicols}{2}

\label{st\stat}

\section{Введение }

  Важным классом угроз для \mbox{РИВС} 
  является воз\-мож\-ность организации нелегальных инфор\-мационных 
взаимодействий и,~на\-обо\-рот, запрет на\linebreak разрешенные информационные 
взаимодейст\-вия~[1--4]. Для пред\-от\-вра\-ще\-ния этого класса угроз  
в~\cite{1-gr, 2-gr, 3-gr} предложена организация управ\-ле\-ния со\-еди\-не\-ни\-ями 
с~по\-мощью метаданных, а~именно: за\-шиф\-ро\-ван\-ное взаимодействие двух хос\-тов 
разрешено тогда, когда данное со\-еди\-не\-ние служит для реализации 
производственной ИТ. 
  
  Вместе с~изоляцией задач и~сис\-те\-мой иденти\-фикаторов ре\-ша\-емых задач данный 
подход обеспечивает достаточно высокую за\-щи\-щен\-ность \mbox{РИВС}~\cite{1-gr, 2-gr}.
  
  Метаданные формируются на основе математических моделей  
биз\-нес-про\-цес\-сов, которые в~РИВС пред\-став\-ле\-ны со\-во\-куп\-ностью ИТ. Часть 
информации, связанная с~функционированием ИТ, используется для организации 
разрешительной сис\-те\-мы со\-еди\-не\-ния хос\-тов в~сети.
  
  Информационная безопасность в~таком подходе основана на сле\-ду\-ющих 
принципах: 
  \begin{enumerate}[(1)]
\item политики без\-опас\-ности (ПБ) формулируются на уровне моделей 
биз\-нес-про\-цес\-сов и~моделей ИТ~[5] 
и~выражаются в~терминах языков, ис\-поль\-зу\-емых на 
этих уровнях; 
\item основная идея управ\-ле\-ния со\-еди\-не\-ни\-ями в~сети с~по\-мощью 
метаданных со\-сто\-ит в~разделении плос\-кости логики и~функционала 
решения задач от плос\-кости передачи данных через сеть~[6]. При таком разделении 
без\-опас\-ность со\-еди\-не\-ний определяется проекцией ПБ 
верх\-ней плос\-кости на нижнюю. 
\end{enumerate}

  Соединения хостов разрешаются, когда на них находятся задачи, 
вза\-имо\-дей\-ст\-ву\-ющие по логике ИТ. Тем самым решается проб\-ле\-ма воз\-мож\-ности 
реализации разрешенных информационных ВЗАИМОДЕЙСТВИЙ 
и~существенного ограничения воз\-мож\-ности существования нелегальных 
информационных потоков. 

\vspace*{-9pt}
  
\section{Моделирование задач}

  Для формирования метаданных необходимо по\-стро\-ить по\-сле\-до\-ва\-тель\-ность 
переходов от моделей биз\-нес-про\-цес\-сов к~моделям ИТ, из которых 
выделяются метаданные.
  
  Основой этих моделей является понятие задачи. Простейшее определение 
задачи~--- это преобразование вход\-ных данных в~выходные. 
Преобразование со\-сто\-ит в~выполнении некоторых функций. Значит, задача должна 
реализовать некоторый функционал. Для задачи~$A$ обозначим функционал 
преобразования $\pi(A)$. Для выполнения преобразования задачи~$A$ 
необходимо обеспечить~$\pi(A)$ исходными данными и~значениями 
конфигурационных па\-ра\-мет\-ров функционала. Эти данные надо получить извне. 
Кроме того, необходимо распределить результаты преобразования входной 
информации или сохранить эти данные в~памяти. Объединение этих функций 
организации решения задачи~$A$ будем обозначать через~$\varphi(A)$. Таким 
образом, задача~$A$ определяется как пара $(\pi (A), \varphi (A))$.
  
  Далее обобщим это определение на случай, когда задача~$A$ со\-сто\-ит из 
нескольких задач $\{B_1, B_2,\ldots , B_s\}$.
  
  Согласно введенному выше определению $A\hm = (\pi (A), \varphi (A))$, 
а~$B_i\hm = (\pi (B_i), \varphi (B_i))$, $i\hm=1,\ldots , s$. 
  
  \smallskip
  
  \noindent
  \textbf{Определение~1.} Можно отож\-де\-ст\-вить со\-став\-ную задачу~$A$ 
с~множеством задач $\{B_1, B_2,\ldots , B_s\}$ тогда и~только тогда, когда 
выполняются сле\-ду\-ющие условия:
  \begin{enumerate}[(1)]
\item функционал множества $\{\pi (B_i)$, $i\hm= 1,\ldots , s\}$ покрывает 
функционал~$\pi (A)$;
\item объект $\varphi(A)$ имеет воз\-мож\-ность взаимодействия с~каждым из 
объектов $\{\varphi (B_i)$, $i\hm=1,\ldots , s\}$:
\begin{itemize}
\item[(а)] для получения и~передачи всех вход\-ных данных для каж\-дой 
задачи~$B_i$, $i\hm= 1,\ldots , s$;
\item[(б)]распределения всех выходных данных задач~$B_i$, $i\hm= 1,\ldots , s$, 
которые должны быть определены в~задаче~$A$;
\item[(в)]определения порядка выполнения задач~$B_i$, $i\hm= 1,\ldots , s$, для 
решения задачи~$A$;
\item[(г)]связи задач~$B_i$, $i\hm = 1,\ldots, s$, меж\-ду собой. 
\end{itemize}
\end{enumerate}
  
  Это определение сходно с~понятием кли\-ент-сер\-вер\-ной архитектуры, 
в~которой клиенты~$B_i$ организуют свою работу под управ\-ле\-ни\-ем 
сервера~$\varphi(A)$.
  
  Введенное определение связано с~иерархической декомпозицией задач как 
множеств подзадач. Развитием этого взгляда на задачу является понимание того, 
что сами подзадачи исходной задачи могут пред\-став\-лять собой множество задач. 
Например, имеется составная задача $A\hm = \{ B_1, B_2, \ldots ,B_s\}$. В~то же 
время каждая $B_i\hm=\{C_{i,j}\}$. Тогда возникает иерархическая декомпозиция 
задачи~$A$ на подзадачи. Такая иерархическая декомпозиция в~тео\-рии 
искусственного интеллекта называется редукцией~[7]. В~[7] приведены 
алгоритмы по\-стро\-ения редукции. При этом необходимо, чтобы все задачи были 
однозначно идентифицированы. 
  
  Отметим, что $\pi(C_{i,j})$ од\-но\-знач\-но определяется на\-зва\-ни\-ем 
задачи~$C_{i,j}$, а~объект~$\varphi(C_{i,j})$ связан с~объектом~$\varphi(B_i)$. 
В~связи с~этим будем использовать сле\-ду\-ющее обо\-зна\-че\-ние: $\varphi_X(C)$~--- 
это объект организации решения задачи~$C$ как подзадачи задачи~$X$. 
  
  Задачи в~иерархической декомпозиции могут повторяться. При этом если 
значения объектов $\pi(\cdot)$ для всех повторений одной задачи одинаковы, то 
значения объектов~$\varphi(\cdot)$ для всех по\-вто\-ре\-ний различны, поэтому 
по\-вто\-ря\-ющи\-еся задачи однозначно идентифицируются. 
  
  Повторное использование задач можно реализовать тремя способами. 

Предположим, что задача является со\-вмест\-но используемым объектом. Тогда 
возможно параллельное обращение к~этой задаче от различных ее повторений. 
Например, сис\-те\-ма управ\-ле\-ния базами данных 
может обрабатывать параллельно несколько запросов от разных 
источников. Тогда воз\-мож\-но, что такая задача~$A$ характеризуется обращением 
к~ней с~помощью разных объектов~$\varphi_X(A)$. 
  
  Вместо параллельного обращения возможно создание очереди использования 
задачи~$A$. Эта очередь будет со\-сто\-ять из цепочки объектов $\varphi_X(A)$ для 
разных~$X$. 
  
  Третий способ связан с~со\-зда\-ни\-ем копий для каждого экземпляра задачи~$A$. 
  
  Таким образом, каждая составная задача~$A$, пред\-став\-лен\-ная в~виде 
множества подзадач, образует дерево подзадач $L(A)$ с~корнем в~$A$. Это дерево 
взаимно однозначно определяет дерево $L(\varphi(A))$. Вершинами дерева 
$L(\varphi(A))$ являются объекты~$\varphi_X(C)$, где~$X$~--- вершина дерева, 
по\-рож\-да\-ющая подзадачу~$C$. 
  
  В этом дереве находится вся информация о~взаимодействиях задач. 
В~принципе, каждое такое дерево можно рассматривать как максимальные 
метаданные, со\-во\-куп\-ность которых определяет порядок решения задачи~$A$, 
поэтому это дерево может использоваться как разрешительная информация для 
со\-еди\-не\-ний в~сети. Однако эти данные слож\-ны для использования в~реальном 
времени. 
  
  Модель РИВС можно представить как со\-став\-ную задачу, включающую все 
задачи, которые можно решать с~по\-мощью этой сис\-те\-мы. Можно\linebreak
 считать, что 
объект $\varphi (\mathrm{РИВС})$ способен выделять и~конфигурировать 
различные наборы задач. Тогда модель ИТ~--- это подмножество задач модели 
РИВС, объединенных моделью биз\-нес-про\-цес\-са и~конкретными 
конфигурациями этих задач. Объект $\varphi (\mathrm{ИТ})$ определяет также 
порядок преобразования исходных данных для реализации этой~ИТ. 
  
  Определим задачу $\mathfrak{M}$ для ИТ сле\-ду\-ющим образом. Эта задача 
связывает физические и~сис\-тем\-ные ресурсы РИВС с~множеством задач, которые 
определены деревом задач, по\-стро\-ен\-ным для данной ИТ, а~именно: 
$\mathfrak{M}$ отоб\-ра\-жа\-ет множество задач дерева задач ИТ во мно\-же\-ст\-во 
компьютеров и~сетевого оборудования. Конфигурирование компьютеров 
и~сетевого оборудования долж\-но обеспечивать связ\-ность решения задач данной 
ИТ. Поскольку дерево задач ИТ~$L$ изоморфно дереву~$L(\varphi)$ этой 
технологии, то для функционирования этой ИТ необходимо вмес\-те 
с~объектами~$\pi$ разместить и~согласовать взаимодействия объектов из 
дерева~$L(\varphi)$. 
  
  Часть этих взаимодействий должна быть реализована через сеть. Традиционно 
расположение одной задачи на другом хос\-те задается в~объекте~$\varphi$. Это 
дает воз\-мож\-ность хос\-ту инициировать соединение с~другим хос\-том, где находится 
необходимая задача. 
  
  Однако реализация соединения от хоста может порождать нелегальные 
информационные взаимодействия~[1--4]. Также такой подход переносит 
биз\-нес-ло\-ги\-ку на уровень низ\-шей плос\-кости. Поэтому подход, связанный с~метаданными, 
предлагает не передавать информацию о~размещении задач  на уровень хостов 
(хотя в~памяти могут храниться следы прошедших реализаций ИТ). 
  
  Для разделения плоскостей необходимо ввес\-ти для ИТ задачу~$\mathfrak{N}$, 
  к~которой задача~$A$ на хосте~$H(A)$ всегда может обратиться
  через защищенный 
канал  с~прось\-бой о~разрешении взаимодействия 
с~задачей~$B$~\cite{1-gr}. Задача~$\mathfrak{N}$ на основании метаданных определяет 
возможность разрешения взаимодействия с~задачей~$B$, исходя из текущей 
стадии реализации ИТ. 
  
  При разрешении взаимодействия задача~$\mathfrak{N}$ по защищенному 
каналу передает на хост~$H(B)$ информацию о~не\-об\-хо\-ди\-мости взаимодействия 
с~задачей~$A$ на хосте~$H(A)$. Так\-же на хост~$H(B)$ передается ключ для 
связи с~хостом~$H(A)$ и~другие па\-ра\-мет\-ры организации соединения. 
Хос\-ту~$H(A)$ также передается ключ для связи с~хостом~$H(B)$ и~другие 
па\-ра\-мет\-ры организации соединения.

\vspace*{-6pt}
  
\section{Информационная безопасность в~распределенных
информационно-вычислительных~системах}

  Рассмотрим случай, когда порядок взаимодействий в~ИТ жест\-ко определен. При 
этом информация~$L(\varphi)$ для ИТ не позволяет привлекать для 
вы\-чис\-ли\-тель\-но\-го процесса экземпляры задач, не связанных с~данным фрагментом 
дерева~$L(\varphi)$.
  
  \smallskip
  
  \noindent
  \textbf{Теорема~1.}\ \textit{В~ИТ, пред\-став\-лен\-ной деревом $L(\varphi)$, для 
того чтобы все разрешенные жест\-ко определенные взаимодействия задач были 
разрешены, а~остальные были запрещены, необходимо и~до\-ста\-точ\-но, чтобы для 
каждой со\-став\-ной задачи~$A$ в~объекте $\varphi(A)$ однозначно (без 
из\-бы\-точ\-ности) был определен порядок всех взаимодействий непосредственных 
подзадач задачи~$A$ и~исключены взаимодействия, не со\-от\-вет\-ст\-ву\-ющие этому 
порядку.}
  
  \smallskip
  
  \noindent
  Д\,о\,к\,а\,з\,а\,т\,е\,л\,ь\,с\,т\,в\,о\,.\ \ Докажем до\-ста\-точ\-ность. 
Каж\-дая со\-став\-ная задача~$A$ в~ИТ определяет поддерево в~$L(\varphi)$ с~корнем 
в~$A$. Если строить дерево вниз, то чис\-ло слоев будем называть глубиной (это 
соответствует высоте при по\-стро\-ении дерева вверх). Доказательство 
до\-ста\-точ\-ности будем вес\-ти по индукции по значениям глубины. Пусть со\-став\-ное 
дерево задачи~$A$ имеет глубину $h\hm = 1$. Множество взаимодействий всех 
подзадач решает задачу~$A$. По условию тео\-ре\-мы для такой задачи, 
пред\-став\-лен\-ной деревом $\varphi (A)$, все разрешенные взаимодействия задач 
разрешены, а~остальные~--- запрещены, т.\,е.\ для ИТ, опре\-де\-ля\-емой этой 
задачей~$A$, до\-ста\-точ\-ность доказана. 
  
  Предположим, что до\-ста\-точ\-ность доказана для любых $\varphi(A)$ глубиной 
меньше~$h$. Пусть глубина~$\varphi(A)$ рав\-на~$h$. Предположим, что задачи  
$A_1, A_2, \ldots ,A_t$ являются со\-став\-ны\-ми и~расположены на глубине $h\hm-1$. 
Тогда по\-рож\-да\-емые ими де\-ревья имеют глубину~1 и~для каж\-до\-го из них до-\linebreak казана 
до\-ста\-точ\-ность. Для дерева глубиной $h\hm-1$\linebreak до\-ста\-точ\-ность выполняется по 
предположению индукции. Тогда в~исходном дереве порядок всех взаимодействий 
однозначно определен и~разрешен. Остальные взаимодействия отсутствуют.
  
  Докажем необходимость. Каждая со\-став\-ная задача~$A$ может рассматриваться 
как ИТ. Тогда по условию для дерева $\varphi(A)$ однозначно определен порядок 
взаимодействий всех подзадач. Этот порядок порождает единственный порядок 
взаимодействий подзадач~$A$. Это следует из того, что любое поддерево, 
за\-мы\-ка\-юще\-еся на подзадачу~$A$, можно считать выполнением этой подзадачи 
и~это выполнение не влияет на порядок взаимодействий непосредственных 
подзадач~$A$. При этом возможно неоднократное возвращение к~любой из 
подзадач~$A$ в~этом однозначном порядке. Тео\-ре\-ма доказана.
  
  \smallskip
  
  \noindent
   \textbf{Следствие~1.}\ Дерево $L(\varphi)$ можно рас\-смат\-ри\-вать как 
метаданные для ИТ в~том смысле, что со\-еди\-не\-ния в~сети могут однозначно 
определяться в~условиях тео\-ре\-мы~1.

\vspace*{-6pt}
   
\section{Модель допустимых взаимодействий}

   Пусть метаданные для каждой со\-став\-ной задачи~$A$~--- это множества 
идентификаторов непосредственных подзадач~$A$, помеченных также 
идентификатором ИТ. Эти множества легко вычисляются по дереву~$L(\varphi)$ 
для всех задач и~подзадач ИТ и~определяют метаданные ИТ. Обозначим для любой 
составной задачи~$A$ множество идентификаторов ее непосредственных 
подзадач из ИТ через $c(A)$, включая идентификатор самой~$A$.
   
  \medskip
  
  \noindent
  \textbf{Определение~2.}\ Назовем простыми допустимыми взаимодействиями 
задачи~$A$ из ИТ любые взаимодействия задач с~идентификаторами из~$c(A)$.
  
  Тогда дерево $L(\varphi)$ порождает семейство множеств прос\-тых допустимых 
взаимодействий. Это семейство определяет метаданные. 
  
  Таким образом, при обращении к~составной задаче~$A$ со\-глас\-но метаданным 
рас\-смат\-ри\-ва\-емой ИТ разрешаются любые прос\-тые до\-пус\-ти\-мые взаимодействия из 
множества~$c(A)$. При этом сама задача~$A$ входит так\-же во множество 
подзадач ка\-кой-то задачи~$C$ верхнего уровня, что позволяет ей 
взаимодействовать с~задачами, опре\-де\-ля\-емы\-ми $c(C)$, но другие задачи 
из~$c(A)$ это делать не могут. Движение по связанным множествам~$c(\cdot)$ 
потенциально позволяет задаче~$A$ снаб\-жать и~получать любые данные от 
любых других задач ИТ. Построенные метаданные значительно проще 
в~реализации и~позволяют объектам~$\varphi(A)$ изменять порядок 
взаимодействий внут\-ри множества, опре\-де\-ля\-емо\-го~$c(A)$.
   
  \medskip
  
  \noindent
  \textbf{Определение~3.}\ Любую связ\-ную по\-сле\-до\-ва\-тель\-ность 
множеств~$c(A)$ для~$A$ из ИТ назовем до\-пус\-ти\-мы\-ми взаимодействиями. 
  
  \medskip
  
  \noindent
  \textbf{Лемма~1.}\ \textit{Все жестко определенные реализации ИТ входят 
в~мно\-же\-ст\-во допустимых взаимодействий.}
  
  \medskip
  
  \noindent
  Д\,о\,к\,а\,з\,а\,т\,е\,л\,ь\,с\,т\,в\,о\,.\ \ Все жест\-ко определенные 
реализации ИТ для каждой со\-став\-ной задачи образуют мно\-же\-ст\-во прос\-тых 
допустимых взаимодействий. Тогда по тео\-ре\-ме~1 все жестко определенные 
реализации ИТ входят в~множество допустимых взаимодействий.
  
  На множествах $c(A)$ как на вершинах построим граф метаданных~$G$. Реб\-ро 
со\-еди\-ня\-ет~$c(A)$ с~$c(B)$ тогда и~только тогда, когда~$B$ является 
подзадачей~$A$.
  
  \medskip
  
  \noindent
  \textbf{Лемма~2.}\ \textit{Граф~$G$ является деревом, изоморфным 
дереву~$L(\varphi)$.}
  
  \medskip
  
  \noindent
  Д\,о\,к\,а\,з\,а\,т\,е\,л\,ь\,с\,т\,в\,о\,.\ \ Множества вершин 
графа~$G$ и~графа~$L(\varphi)$ взаимно однозначно соответствуют друг другу. 
Вершина~$c(A)$ соответствует~$\varphi(A)$.
  
  Ребро $(c(A), c(B))$ взаимно однозначно соответствует связи 
объектов~$\varphi(A)$ и~$\varphi(B)$. Лемма доказана.
  
  \medskip
  
  Пусть ПБ в~каж\-дой вершине~$\varphi(A)$ 
дерева~$L(\varphi)$ определяет воз\-мож\-ность сквоз\-ной передачи информации 
через~$\varphi(A)$. Тогда со\-глас\-но лемме~2 метаданные~$G$ могут не заметить 
нарушение ПБ только для взаимодействий внут\-ри множеств 
$c(A)$ для задач рас\-смат\-ри\-ва\-емой~ИТ.
  
  Отметим, что в~множествах $c(A)$ всегда присутствует идентификатор ИТ. Это 
позволяет не допускать нелегальных взаимодействий задач раз\-ных~ИТ.
{\looseness=1

}
  
  Из лемм~1, 2 и~последних замечаний следует тео\-ре\-ма~2.

  \medskip
  
  \noindent
  \textbf{Теорема~2.}\ \textit{Метаданные~$G$ не разрешают недопустимых 
взаимодействий.}
  
  \medskip
  
  \noindent
  \textbf{Следствие~2.}\ Если соединения в~сети определяются 
метаданными~$G$, то недопустимых взаимодействий хос\-тов нет.
  
  Таким образом, метаданные~$G$ разрешают изменение порядка 
взаимодействий в~сети только в~рамках изменения порядка допустимых 
взаимодействий задач ИТ.

%\vspace*{-6pt}
  
\section{Заключение}

  В работе разобраны два метода по\-стро\-ения метаданных для 
ИТ. 

Первый метод связан с~жест\-кой последовательностью 
взаимодействий задач в~ИТ. 
Второй метод неоднозначно определяет порядок 
взаимодействий задач, но проще для вы\-чис\-ле\-ний~\cite{3-gr} и~может быть 
реализован в~реальном времени в~разрешительной сис\-те\-ме установления 
соединений в~сети~\cite{1-gr}. Оба метода могут реализовать требования ПБ по 
безопасности взаимодействий. 

Предложенный подход основан на иерархической 
декомпозиции ИТ и~задач, где каж\-дая задача~$A$ 
пред\-став\-ля\-ет\-ся парой $(\pi (A), \varphi (A))$. Такое пред\-став\-ле\-ние поз\-во\-ля\-ет 
абстрагироваться от существа алгоритмов и~слож\-ности их взаимодействий. 
Однако остается открытой проб\-ле\-ма практического по\-стро\-ения 
объектов~$\varphi(A)$. Если эта проблема для ИТ может быть решена не\-слож\-но, 
то будет получен эффективный метод по\-стро\-ения метаданных для управ\-ле\-ния 
сетями.
{ %\looseness=1

}

%\vspace*{-6pt}
  
{\small\frenchspacing
 {%\baselineskip=10.8pt
 \addcontentsline{toc}{section}{References}
 \begin{thebibliography}{9}
 
 \bibitem{4-gr} %1
\Au{Грушо А., Забежайло~М., Зацаринный~А.} Контроль и~управ\-ле\-ние информационными 
потоками в~облачной среде~// Информатика и~её применения, 2015. Т.~9. Вып.~4. С.~95--101.
\bibitem{1-gr} %2
\Au{Grusho A.\,A., Timonina~E.\,E., Shorgin~S.\,Ya.} Modelling for ensuring information security of 
the distributed information systems~// 
31th European Conference on Modelling and Simulation 
Proceedings.~--- Dudweiler, Germany: Digitaldruck Pirrot GmbHP, 
2017. P.~656--660.
{\sf http:// www.scs-europe.net/dlib/2017/ecms2017acceptedpapers/0656-probstat\_ECMS2017\_0026.pdf}.
\bibitem{2-gr} %3
\Au{Grusho A., Grusho N., Zabezhailo~M., Zatsarinny~A., Timonina~E.} Information security of SDN 
on the basis of
 meta data~// Computer network security~/
Eds. J.~Rak, J.~Bay, I.\,V.~Kotenko, \textit{et al.}~---
Lecture notes in computer science ser.~--- Springer, 2017. Vol.~10446. P.~339--347.
{\sf https://link.springer.com/chapter/10.1007/978-3-319-65127-9\_27}.
\bibitem{3-gr} %4
\Au{Grusho A., Timonina E., Shorgin~S.} 
Security models based on stochastic meta data~// Analytical and computational
methods in theory probability~/
Eds.\ V.~Rykov, N.~Singpurwalla, A.~Zubkov.~--- Lecture 
notes in computer science ser.~--- Springer, 2017. Vol.~10684. P. 388--400. 
 {\sf 
https://link.springer.com/chapter/10.1007/978-3-319-71504-9\_32}.

\bibitem{5-gr}
\Au{Грушо А.\,А., Применко~Э.\,А., Тимонина~Е.\,Е.} Тео\-ре\-ти\-че\-ские 
основы компьютерной 
без\-опас\-ности.~--- М.: Академия, 2009. 
272~с.

\bibitem{6-gr}
\Au{Grusho A.\,A., Abaev~P.\,O., Shorgin~S.\,Ya., Timonina~E.\,E.} Graphs for information security 
control in software defined networks~// AIP Conf. Proc., 2017. Vol.~1863.  
P.~090002-1--090002-4. {\sf http://aip.scitation.\linebreak org/doi/pdf/10.1063/1.4992267}.
\bibitem{7-gr}
\Au{Нильсон Н.} Искусственный интеллект. Методы поиска решений~/ Пер. с~англ.~--- М.: 
Мир, 1973. 272~с. (\Au{Nilsson~N.\,J.} {Problem-solving methods in artificial intelligence.}~--- 
New York, NY, USA: McGraw-Hill Publ. Co., 1971. 255~p.)
 \end{thebibliography}

 }
 }

\end{multicols}

\vspace*{-6pt}

\hfill{\small\textit{Поступила в~редакцию 21.02.18}}

\vspace*{6pt}

%\newpage

%\vspace*{-24pt}

\hrule

\vspace*{2pt}

\hrule

\vspace*{-2pt}


\def\tit{HIERARCHICAL METHOD OF~META DATA GENERATION 
FOR~CONTROL OF~NETWORK CONNECTIONS}

\def\titkol{Hierarchical method of~meta data generation 
for~control of~network connections}

\def\aut{A.\,A.~Grusho, E.\,E.~Timonina, and~S.\,Ya.~Shorgin}

\def\autkol{A.\,A.~Grusho, E.\,E.~Timonina, and~S.\,Ya.~Shorgin}

\titel{\tit}{\aut}{\autkol}{\titkol}

\vspace*{-11pt}


 \noindent
Institute of Informatics Problems, Federal Research Center ``Computer Sciences and 
Control'' of the Russian Academy of Sciences, 44-2 Vavilov Str., Moscow 119133, 
Russian Federation


\def\leftfootline{\small{\textbf{\thepage}
\hfill INFORMATIKA I EE PRIMENENIYA~--- INFORMATICS AND
APPLICATIONS\ \ \ 2018\ \ \ volume~12\ \ \ issue\ 2}
}%
 \def\rightfootline{\small{INFORMATIKA I EE PRIMENENIYA~---
INFORMATICS AND APPLICATIONS\ \ \ 2018\ \ \ volume~12\ \ \ issue\ 2
\hfill \textbf{\thepage}}}

\vspace*{3pt}

    
     
\Abste{An important class of threats for distributed information systems is the possibility of  
organization of illegal information interactions and, vice versa, prohibition of the allowed information 
interactions. For preventing these threats, creation of connections control with the help of meta data is 
proposed. Meta data are created on the basis of mathematical models 
of business processes which are provided in  
distributed information systems by a~set of information technologies. In the paper, two 
methods of meta data creation for information technologies are considered. The 
suggested approach is 
based on hierarchical decomposition of information technologies and tasks.}
 
\KWE{information security; distributed information systems; meta data; hierarchical decomposition; 
composite tasks}
   
 
   
   
\DOI{10.14357/19922264180207} %

\vspace*{-14pt}

  \Ack
   \noindent
   The paper was supported by the Russian Science Foundation (project 16-11-10227).



%\vspace*{6pt}

  \begin{multicols}{2}

\renewcommand{\bibname}{\protect\rmfamily References}
%\renewcommand{\bibname}{\large\protect\rm References}

{\small\frenchspacing
 {%\baselineskip=10.8pt
 \addcontentsline{toc}{section}{References}
 \begin{thebibliography}{9}
 
 \bibitem{4-gr-1} %1
\Aue{Grusho, A., M.~Zabezhailo, and A.~Zatsarinny.} 2015. Kontrol' i~upravlenie informatsionnymi 
potokami v~oblachnoy srede [Information flow monitoring and control in cloud computing 
environment]. \textit{Informatika i~ee Primeneniya~--- Inform. Appl.} 9(4):95--101.
\bibitem{1-gr-1} %2
\Aue{Grusho, A.\,A., E.\,E.~Timonina, and S.\,Ya.~Shorgin.} 2017. Modelling for ensuring 
information security of the distributed information systems. \textit{31th European Conference on 
Modelling and Simulation Proceedings}. Digitaldruck Pirrot GmbHP Dudweiler, Germany. 656--660. 
Available at: {\sf http://www.scs-europe.net/dlib/2017/ecms2017acceptedpapers/0656-probstat\_ECMS2017\_0026.pdf} (accessed February~20, 
2018).
\bibitem{2-gr-1} %3
\Aue{Grusho, A., N.~Grusho, M.~Zabezhailo, A.~Zatsarinny, and E.~Timonina.} 2017. Information 
security of SDN on the basis of meta data. \textit{Computer network security}.
Eds. J.~Rak, J.~Bay, I.\,V.~Kotenko, \textit{et al}. Lecture notes in computer science  ser.
Springer. 10446:339--347. Available at: {\sf https://link.springer.com/chapter/10.1007/978-3-319-65127-9\_27} 
(accessed February~20, 2018).
\bibitem{3-gr-1} %4
\Aue{Grusho, A., E.~Timonina, and S.~Shorgin.} 2017. Security models based on stochastic meta 
data. \textit{Analytical and computational
methods in theory probability}.
Eds.\ V.~Rykov, N.~Singpurwalla, and A.~Zubkov.
Lecture notes in computer science ser. Springer. 10684:388--400. Available at:
 {\sf 
https://link.springer.com/chapter/10.1007/978-3-319-71504-9\_32} (accessed February~20, 2018). 

\bibitem{5-gr-1}
\Aue{Grusho, A., Ed.~Primenko, and E.~Timonina.} 2009. \textit{Teoreticheskie osnovy 
komp'yuternoy bezopasnosti} [Theoretical 
bases of computer security]. Moscow: 
Academy. 272~р.
\bibitem{6-gr-1}
\Aue{Grusho, A.\,A., P.\,O.~Abaev, S.\,Ya.~Shorgin, and E.\,E.~Ti\-mo\-ni\-na}. 2017. 
Graphs for 
information security control in software defined networks. \textit{AIP Conf. Proc.} 
1863:090002. 4~p. Available at: {\sf http://aip.scitation.org/ doi/pdf/10.1063/1.4992267} (accessed 
February~20, 2018).
\bibitem{7-gr-1}
\Aue{Nilsson, N.\,J.} 1971. \textit{Problem-solving methods in artificial intelligence.} New York, NY: 
McGraw-Hill Publ. Co. 255~p.
{\looseness=1

}

\end{thebibliography}

 }
 }

\end{multicols}

\vspace*{-9pt}

\hfill{\small\textit{Received February 21, 2018}}

%\pagebreak

%\vspace*{-24pt}


\Contr

\noindent
\textbf{Grusho Alexander A.} (b.\ 1946)~--- Doctor of Science in physics and 
mathematics, professor, principal scientist, Institute of Informatics Problems, Federal 
Research Center ``Computer Sciences and Control'' of the Russian Academy of 
Sciences, 44-2~Vavilov Str., Moscow 119133, Russian Federation; 
\mbox{grusho@yandex.ru} 

\vspace*{6pt}

\noindent
\textbf{Timonina Elena E.} (b.\ 1952)~--- Doctor of Science in technology, 
professor, leading scientist, Institute of Informatics Problems, Federal Research 
Center ``Computer Sciences and Control'' of the Russian Academy of Sciences,  
44-2~Vavilov Str., Moscow 119133, Russian Federation; 
\mbox{eltimon@yandex.ru} 

\vspace*{6pt}

\noindent
\textbf{Shorgin Sergey Ya.} (b.\ 1952)~--- Doctor of Science in physics and 
mathematics, professor, principal scientist, Institute of Informatics Problems, Federal 
Research Center ``Computer Sciences and Control'' of the Russian Academy of 
Sciences, 44-2~Vavilov Str., Moscow 119133, Russian Federation; 
\mbox{sshorgin@ipiran.ru} 

\label{end\stat}


\renewcommand{\bibname}{\protect\rm Литература} 