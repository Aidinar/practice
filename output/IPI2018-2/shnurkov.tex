\def\stat{shnurkov}

\def\tit{РЕШЕНИЕ ПРОБЛЕМЫ ОПТИМАЛЬНОГО УПРАВЛЕНИЯ ЗАПАСОМ НЕПРЕРЫВНОГО 
ПРОДУКТА ПРИ~ПОСТОЯННО ПРОИСХОДЯЩЕМ ПОТРЕБЛЕНИИ 
В~СТОХАСТИЧЕСКОЙ ПОЛУМАРКОВСКОЙ МОДЕЛИ}

\def\titkol{Решение проблемы оптимального управления запасом непрерывного 
продукта} % при постоянно происходящем потреблении  в~стохастической полумарковской модели}

\def\aut{П.\,В.~Шнурков$^1$, А.\,Ю.~Егоров$^2$}

\def\autkol{П.\,В.~Шнурков, А.\,Ю.~Егоров}

\titel{\tit}{\aut}{\autkol}{\titkol}

\index{Шнурков П.\,В.}
\index{Егоров А.\,Ю.}
\index{Shnurkov P.\,V.}
\index{Egorov A.\,Y.}


%{\renewcommand{\thefootnote}{\fnsymbol{footnote}} \footnotetext[1]
%{Работа поддержана РНФ (проект 16-11-10227).}}


\renewcommand{\thefootnote}{\arabic{footnote}}
\footnotetext[1]{Национальный исследовательский университет <<Высшая школа экономики>>,
\mbox{pshnurkov@hse.ru}}
\footnotetext[2]{Национальный исследовательский университет <<Высшая школа экономики>>, 
\mbox{ayuegorov@hse.ru}}

%\vspace*{-6pt}

  
  
  \Abst{Тео\-ре\-ти\-че\-ски 
обосновано решение задачи оптимального управ\-ле\-ния запасом в~рас\-смат\-ри\-ва\-емой 
полумарковской модели. Для достижения этой цели произведены формальные 
аналитические преобразования полученных авторами ранее интег\-раль\-ных пред\-став\-ле\-ний для основных 
вероятностных характеристик модели. Данные пре\-обра\-зо\-ва\-ния сделали возможным 
применение тео\-ре\-мы об аналитическом пред\-став\-ле\-нии стационарного стоимостного 
показателя эф\-фек\-тив\-ности управ\-ле\-ния полумарковским процессом в~форме  
дроб\-но-ли\-ней\-но\-го интег\-раль\-но\-го функционала. 
Используется общая теорема об экстремуме дроб\-но-ли\-ней\-но\-го интегрального 
функционала, доказанная П.\,В.~Шнур\-ко\-вым. Эта тео\-ре\-ма позволяет свести поставленную 
задачу оптимального управ\-ле\-ния запасом к~задаче исследования на глобальный экстремум 
некоторой заданной функции от конечного чис\-ла действительных неотрицательных 
переменных, которая может быть эффективно решена на практике с~использованием 
известных чис\-лен\-ных методов.}
  
  \KW{управление запасом; полумарковский случайный процесс; стационарный 
стоимостный показатель; оптимальное управ\-ле\-ние стохастическими сис\-те\-ма\-ми;  
дроб\-но-ли\-ней\-ный интегральный функционал}

\DOI{10.14357/19922264180212}
  
%\vspace*{-6pt}


\vskip 10pt plus 9pt minus 6pt

\thispagestyle{headings}

\begin{multicols}{2}

\label{st\stat}

\section{Введение}

  Настоящая работа является второй и~за\-вер\-шающей частью исследования 
стохастической полумарковской модели управ\-ле\-ния запасом непрерывного 
продукта при по\-сто\-ян\-но происходящем по\-треб\-ле\-нии, пер\-вая часть которого 
была изложена в~\mbox{статье}~[1].
  
   Характеризуя данное исследование в~целом, можно отметить сле\-ду\-ющее. 

В~первой части была построена и~проанализирована стохастическая 
полумарковская модель управ\-ле\-ния запасом непрерывного продукта. 
Основным результатом первой час\-ти является получение явных аналитических 
пред\-став\-ле\-ний для вероятностных характеристик по\-стро\-ен\-ной полумарковской 
модели, необходимых для решения задачи оптимального управ\-ле\-ния. 

Вторая 
часть по\-свя\-ще\-на непосредственному решению задачи оптимального 
управ\-ле\-ния, которая формально пред\-став\-ля\-ет собой задачу на\-хож\-де\-ния 
глобального экстремума стационарного стоимостного показателя 
эффективности управ\-ле\-ния, заданного на множестве наборов вероятностных 
распределений, которые определяют стратегию управ\-ления.
  
  Напомним в~краткой форме читателю описание сис\-те\-мы управ\-ле\-ния запасом 
и~соответствующей стохастической модели, пред\-ло\-жен\-ной в~работе~[1]. 

Исследуемая сис\-те\-ма пред\-став\-ля\-ет собой хранилище максимальной 
вмес\-ти\-мости~$\tau$, из которого производится непрерывное по\-треб\-ле\-ние продукта. 
Рас\-смат\-ри\-ва\-емая стохастическая модель пред\-став\-ля\-ет собой пару случайных 
процессов $(x(t), \zeta(t))$ в~которой основ\-ной процесс~$x(t)$ описывает 
текущий объем запаса в~сис\-те\-ме в~момент времени~$t$. Помимо основного 
случайного процесса~$x(t)$ вводится вспомогательный полумарковский 
случайный процесс~$\zeta(t)$ с~конечным множеством со\-сто\-яний $\{0,1,\ldots, 
N_0\}$, на\-зы\-ва\-емый в~данном исследовании со\-про\-вож\-да\-ющим. 
Такой полумарковский процесс определяется со\-сто\-яни\-ем 
основного процесса в~моменты пополнения запаса~$t_n$, $n\hm=0,1,2,\ldots$
  В~эти моменты происходят изменения со\-сто\-яния со\-провождающе\-го 
процесса и~случайный выбор реше\-ния (управ\-ле\-ния), которое определяется как 
реализация случайной величины~$u_n$ с~распределением~$G_i(u)$ при 
условии $\zeta(t_n)\hm=\zeta_n\hm= i$. В~рас\-смат\-ри\-ва\-емой модели случайная 
величина~$u_n$ пред\-став\-ля\-ет собой время от момента очередного пополнения 
до момента заказа на следующее пополнение запаса. Период времени между 
последовательными моментами пополнения запаса $[t_n, t_{n+1})$ разделяется 
на два этапа. На первом этапе, называемом периодом чис\-то\-го по\-треб\-ле\-ния, 
скорость потребления фиксирована и~рав\-на величине~$\alpha_c$, 
а~дли\-тель\-ность такого периода пред\-став\-ля\-ет собой случайную 
величину~$u_n$, име\-ющую распределение $G_i(u)$, если 
$\zeta(t_n)\hm=\zeta_n\hm=i$. Второй этап по содержанию пред\-став\-ля\-ет собой 
время выполнения заказа или период за\-держ\-ки по\-став\-ки. Дли\-тель\-ность этого 
этапа явятся известной детерминированной величиной~$h$. По\-треб\-ле\-ние 
продукта на втором этапе продолжается с~заданной ско\-ростью~$\alpha_w$, 
причем $\alpha_w\hm\leq \alpha_c$.
  
  Пополнение запаса происходит мгновенно в~моменты~$t_n$, 
$n\hm=0,1,2,\ldots$ Процедура пополнения является стохастической и~имеет 
двух\-этап\-ный характер. Полное формализованное описание процедуры 
пополнения и~всей полумарковской модели в~целом приведено в~работе~[1] 
и~приложении к~ней.
  
\section{Показатель эффективности управления и~общая постановка 
оптимизационной задачи}

  Определим показатель эффективности управ\-ле\-ния. Обозначим через~$V(t)$, 
$t\hm\geq 0$, некоторый аддитивный функционал, име\-ющий смысл прибыли, 
полученной в~результате функционирования сис\-те\-мы за период времени $[0,t]$. 
Данный функционал связан с~исходным случайным процессом~$x(t)$ 
и~сопровождающим полумарковским процессом~$\zeta(t)$. Формальное 
задание такого функционала производится по схеме, описанной в~классических 
работах~[2, 3]. Известно, что при достаточно общих условиях, 
сформулированных в~указанных работах (см.\ так\-же современные исследования 
для общих управ\-ля\-емых полумарковских моделей~[4]), имеет мес\-то сле\-ду\-ющее 
утверж\-де\-ние, называемое эргодической тео\-ре\-мой для аддитивного 
функционала:
  \begin{equation}
  I=\lim\limits_{t\to\infty} \fr{{\sf E}V(t)}{t}=\fr{\sum\nolimits_{i=0}^{N_0} d_i \pi_i} 
{ \sum\nolimits_{i=0}^{N_0} T_i \pi_i}\,.
  \label{e1-sh}
  \end{equation}
  
  Дадим определения величин, входящих в~правую часть соотношения~(\ref{e1-sh}): 
  \begin{description}
  \item[\,]
  $\Delta V_n\hm= V(t_{n+1})\hm- V(t_n)$~--- приращение функ\-ционала 
прибыли за период времени между по\-следовательными моментами изменения 
со\-стояния со\-про\-вож\-да\-юще\-го полумарковского процесса~$\zeta(t)$;
  \item[\,]
   $d_i\hm= {\sf E} [\Delta V_n\vert \zeta_n=i]$~--- условное математическое ожидание прибыли, обра\-зу\-ющей\-ся за период 
между последовательными моментами изменения состояния со\-про\-вож\-да\-юще\-го 
процесса при условии, что в~этот период указанный процесс находился 
в~со\-сто\-янии~$i$;
  \item[\,]
   $T_i={\sf E}\left[t_{n+1}-t_n\vert \zeta_n=i\right]$~--- условное математическое ожидание дли\-тель\-ности периода между 
последовательными моментами изменения со\-сто\-яния со\-про\-вож\-да\-юще\-го 
процесса при условии, что в~этот период указанный процесс находился 
в~со\-сто\-янии~$i$;
  \item[\,]
  $\Pi\hm= \left\{ \pi_0, \ldots ,\pi_{N_0}\right\}$~--- стационарное 
распределение цепи Маркова~$\{\zeta_n\}$, вло\-жен\-ной в~данный 
полумарковский процесс~$\zeta(t)$. 
  \end{description}
  
   Следуя уже упомянутым классическим исследованиям~[2, 3], будем считать 
величину~$I$ показателем эффективности управ\-ле\-ния в~рас\-смат\-ри\-ва\-емой 
модели. Данный показатель имеет смысл сред\-ней удель\-ной прибыли 
в~установившемся (стационарном) режиме функционирования сис\-темы.
  
  Задача оптимального управления в~рас\-смат\-ри\-ва\-емой
   модели формализуется в~виде задачи нахождения без\-услов\-но\-го 
   глобального экст\-ре\-му\-ма функционала 
$I\hm= I(G_0, G_1, \ldots , G_{N_0})$ на множестве наборов вероятностных 
распределений $\{G_0(\cdot), G_1(\cdot), \ldots , G_{N_0}(\cdot)\}$.
  
\section{Интегральные преобразования вероятностных характеристик 
модели}

  В работе~[1] были доказаны три основных утверж\-де\-ния, сформулированных 
в~виде тео\-рем, в~которых были пред\-став\-ле\-ны явные аналитические выражения 
для величин~$d_i$ и~$T_i$, $i\hm=0,1,\ldots , N_0$, а также для вероятностей 
перехода цепи Маркова, вложенной в~со\-про\-вож\-да\-ющий полумарковский 
процесс~$\zeta(t)$. Полученные аналитические выражения име\-ют форму 
двойных интегралов от вероятностных распределений~$G_i(\cdot)$ 
и~$B_i(\cdot)$, $i\hm=0,1,\ldots, N_0$.
  
  Вероятностные распределения~$B_i(\cdot)$, $i\hm=0,1,\ldots ,N_0$, связаны 
с~процедурой пополнения запаса и~пред\-по\-ла\-га\-ют\-ся заданными. 
Рас\-пре\-де\-ле\-ния 
$G_0(\cdot), G_1(\cdot), \ldots , G_{N_0}(\cdot)$ описывают управ\-ле\-ния 
и~должны быть определены в~результате решения задачи оптимизации. При 
этом указанные двойные интегралы устро\-ены таким образом, что внут\-рен\-ний 
из них пред\-став\-ля\-ет собой интеграл по мере, за\-да\-ва\-емой 
распределением~$G_i(\cdot)$, а~внеш\-ний~--- по мере, за\-да\-ва\-емой 
распределением~$B_i(\cdot)$, $i\hm=0,1,\ldots ,N_0$.
  
  Для решения поставленной задачи оптимального управ\-ле\-ния необходимо 
про\-вес\-ти аналитические преобразования упомянутых вероятностных 
характеристик, которые в~целом за\-клю\-ча\-ют\-ся в~изменении порядка 
интегрирования в~со\-от\-вет\-ст\-ву\-ющих двойных интегралах. Проводимые 
пре\-об\-ра\-зо\-ва\-ния основаны на тео\-ре\-ме Фубини~\cite{5-sh} и~связаны 
с~конкретным видом областей интегрирования. 

Приведем полученные 
результаты в~форме утверж\-де\-ний об аналитических пред\-став\-ле\-ни\-ях 
характеристик~$d_i$ и~$T_i$, $i\hm=0,1,\ldots, N_0$, и~вероятностей перехода 
вло\-жен\-ной цепи Маркова~$\{\zeta_n\}$. Пол\-ные аналитические обосно\-ва\-ния 
этих утверж\-де\-ний приведены в~приложении к~на\-сто\-ящей работе~\cite{6-sh}.
  
  \smallskip
  
  \noindent
  \textbf{Утверждение~1.} \textit{Вспомогательные вероятностные 
характеристики, связанные с~вероятностями перехода вложенной цепи 
Маркова $\{\zeta_n\}$, введенные в~работе}~\cite{1-sh}, \textit{пред\-ста\-ви\-мы 
в~сле\-ду\-ющем виде}:
  \begin{multline*}
  p_{ik}^{(+)} =\hspace*{-6mm} 
  \int\limits_{(i-k-1)L/\alpha_c - \alpha_w h/\alpha_c}^{(i-
k)L/\alpha_c - \alpha_w h/\alpha_c}\hspace*{-6mm}
  \left[ B_i\left(\alpha_c u+(k+1)L +\alpha_wh\right) -{}\right.\\
\left.  {}-B_i(iL)\right] dG_i+\int\limits_{(i-k)L/\alpha_c - 
\alpha_wh/\alpha_c}^{(i+1-k)L/\alpha_c - 
\alpha_wh/\alpha_c} 
\hspace*{-4mm}\left[ B_i((i+1)L) -{}\right.\\
\left.{}-B_i\left( \alpha_c u 
+kL+\alpha_w h\right) \right] dG_i(u)\\
  \mbox{\textit{при} } k=0,\ldots ,  k_i^{(+)}-1\,;
  \end{multline*}
  
  \vspace*{-12pt}
  
  \noindent
  \begin{multline*}
  p_{ik_i^{(+)}}^{(+)} = \int\limits_0^{\left(i+1-k_i^{(+)}\right)L/\alpha_c-
\alpha_wh/\alpha_c} \left[ 
\vphantom{k_i^{(+)}}
B_i((i+1)L) -{}\right.\\
\left.{}-B_i\left( \alpha_c u + 
k_i^{(+)}L +\alpha_w h\right)\! \right] dG_i(u)\ \ \ 
  \mbox{\textit{при} } k=k_i^{(+)}\,;\hspace*{-0.62201pt}
  \end{multline*}
  
  \noindent
  $$
  p_{ik}^{(+)} = 0\enskip \mbox{\textit{при} } k> k_i^{(+)}\,;
  $$
  
  \vspace*{-12pt}
  
  \noindent
  \begin{multline*}
  p_{ik}^{(-)} = \hspace*{-6mm}\int\limits_{(i+k)L/\alpha_c-
\alpha_wh/\alpha_c}^{(i+k+1)L/\alpha_c -\alpha_wh/\alpha_c} \hspace*{-6mm}
  \left[ B_i\left(\alpha_c u_0-kL +\alpha_w h\right) -{}\right.\\
\left.  {}-B_i (iL)\right] dG_i(u)+ 
\hspace*{-6mm}\int\limits_{(i+k+1)L/\alpha_c-\alpha_w h/\alpha_c}^{(i+k+2)L/\alpha_c -
\alpha_wh/\alpha_c} \hspace*{-6mm}\left[ B_i((i+1)L) -{}\right.\\
\left.{}-B_i\left( \alpha_c u -
(k+1)L +\alpha_w h\right)\right] dG_i(u)\\
  \mbox{\textit{при} } k= k_i^{(-)}+1,\ldots , N_1-1\,;
  \end{multline*}
  
  %\vspace*{-12pt}
  
  \noindent
  \begin{multline*}
  p^{(-)}_{ik_i^{(-)}} = \hspace*{-4mm}\int\limits_0^{\left(i+k_i^{(-)} +2\right)L/\alpha_c -
\alpha_wh/\alpha_c} \hspace*{-4mm} \left[ 
\vphantom{k_i^{(-)}}
B_i((i+1)L)-{}\right.\\
\left.{}-B_i\left( \alpha_c u- 
Lk_i^{(-)} +\alpha_w h\right)\right] dG_i(u)\\
  \mbox{\textit{при} } k=k_i^{(-)}\,;
  \end{multline*}
  
  \vspace*{-12pt}
  
  \noindent
  \begin{multline*}
  p^{(-)}_{iN_1}= \hspace*{-6mm}
  \int\limits_{(i+N_1)L/\alpha_c -\alpha_wh/\alpha_c}^{(i+N_1+1)L/\alpha_c-
  \alpha_wh/\alpha_c} 
 \hspace*{-6mm} \left[ B_i\left( \alpha_c u- LN_1 +\alpha_w h\right) -{}\right.\\[3pt]
\left.  {}-B_i(iL)\right] 
dG_i(u)+
  \left[ B_i((i+1)L) -B_i(iL)\right] \times{}\\[3pt]
  {}\times\left[ 1-G_i\left( \fr{(i+N_1+1)L}{\alpha_c}-
\fr{\alpha_w h}{\alpha_c}\right) \right]\\[3pt]
  \mbox{\textit{при} } k=N_1\,,\enskip \left(i+N_1\right)L>\alpha_w h\,;
  \end{multline*}
  
  \vspace*{-12pt}
  
  \noindent
  \begin{multline*}
  p^{(-)}_{iN_1}=\int\limits_{iL}^{(i+1)L} \left[ \int\limits_0^\infty dG_i(u)\right] 
dB_i(x)={}\\[3pt]
  {}= \left[ B_i((i+1)L) -B_i(iL)\right] \left[1-g_{0i}\right]\\[3pt]
  \mbox{\textit{при} } k=N_1\,,\enskip \left(i+N_1\right)L\leq \alpha_w h\,;
  \end{multline*}
  
  %\vspace*{-6pt}
  
  \noindent
  $$
  p_{ik}^{(-)}=0 \ \mbox{\textit{при} } k=0,1,\ldots  k_i^{(-)}-1\,,
  $$
\textit{где $k_i^{(\pm)}$~--- граничные значения дискретной величины~$k$, 
которые зависят от известных исходных па\-ра\-мет\-ров сис\-те\-мы} $\alpha_w$, 
$h$  \textit{и}~$L$: 
\begin{gather*}
k =k_i^{(+)}\left(\alpha_w, h, L\right)= \max\left(0, i - \fr{\alpha_w h}{L}\right); 
\\
k=k_i^{(-)}\left( \alpha_w, h, L\right) = \left\vert \min \left( 0, i-\fr{\alpha_w 
h}{L}\right)\right\vert\,.
\end{gather*}

%\vspace*{3pt}

  \noindent
  \textbf{Утверждение~2.} \textit{Выражения для математических ожиданий 
длительностей пребывания со\-про\-вож\-да\-юще\-го процесса~$\zeta(t)$ в~различных 
со\-сто\-яни\-ях имеют вид}:
  \begin{multline*}
  T_i=\int\limits_0^\infty \left( u\left[ B_i\left(( i+1)L\right) -
B_i(iL)\right]\right)\,dG_i(u)+h={}\\
{}=\int\limits_0^\infty u\,dG_i(u)+h\,,\enskip
  i=0,1,\ldots ,N_0\,.
  \end{multline*}
  
 % \vspace*{3pt}
  
  \noindent
  \textbf{Утверждение~3.} \textit{Вспомогательные характеристики, 
связанные с~математическими ожиданиями приращений функционала 
прибыли~$d_{ik}^{(+)}$ и~$d_{ik}^{(-)}$,  введенными в~работе}~\cite{1-sh}, 
\textit{пред\-ста\-ви\-мы в~виде линейных комбинаций двойных интегралов по 
распределениям $G_i(\cdot)$ и~$B_i(\cdot)$, $i\hm=0,1,\ldots ,N_0$, таких, что 
внут\-рен\-ний интеграл представляет собой интеграл по мере, за\-да\-ва\-емой 
распределением~$B_i(\cdot)$, а~внеш\-ний~--- по мере, за\-да\-ва\-емой 
распределением~$G_i(\cdot)$, $i\hm=0,1,\ldots , N_0$. Яв\-ные пред\-став\-ле\-ния для 
характеристик~$d_{ik}^{(+)}$ име\-ют сле\-ду\-ющий вид}:

\noindent
  \begin{multline*}
  d_{ik}^{(+)}= {}\\
  {}=\hspace*{-3mm}\int\limits_{\frac{(i-k-1)L}{\alpha_c} -
  \frac{\alpha_w h}{\alpha_c}}^{\frac{(i-k)L}{\alpha_c} -
  \frac{\alpha_w h}{\alpha_c}} \!\left[ \int\limits_{iL}^{\alpha_c u+(k+1)L 
+\alpha_w h} \hspace*{-10mm} D_{ik}^{(+)} (x,u)\,dB_i(x)\right] dG_i(u)+{}\\
  {}+ \hspace*{-2mm}\int\limits_{\frac{(i-k)L}{\alpha_c} -
  \frac{\alpha_w h}{\alpha_c}}^{\frac{(i+1-k)L}{\alpha_c} -
\frac{\alpha_w h}{\alpha_c}} \left[ \,\int\limits^{(i+1)L}_{\alpha_c u +kL +\alpha_w 
h}\hspace*{-8mm} D_{ik}^{(+)} (x,u)\, dB_i(x)\right] dG_i(u)\\
  \mbox{\textit{при} } k=0,\ldots ,  k_i^{(+)} -1\,;
  \end{multline*}
  
  \vspace*{-12pt}
  
  \noindent
  \begin{multline*}
  d^{(+)}_{ik_i^{(+)}} ={}\\
  {}= \hspace*{-3mm}\hspace*{-7pt}
  \int\limits_0^{\frac{\left(i+1-k_i^{(+)}\right)L}{\alpha_c}- 
\frac{\alpha_wh}{\alpha_c}} \hspace*{-0.8mm}\left[
  \int\limits^{(i+1)L}_{\,\alpha_c u+k_i^{(+)} L+\alpha_w h} \hspace*{-10mm}
  D^{(+)}_{ik_i^{(+)}} (x,u)\,dB_i(x)\right] dG_i(u)\\
  \mbox{\textit{при} } k=k_i^{(+)}\,;
  \end{multline*}
  
  \noindent
  $$ 
  d_{ik}^{(+)}=0\ \mbox{\textit{при} } k> k_i^{(+)}\,;
  $$
  
  \vspace*{-12pt}
  
  \noindent
  \begin{multline*}
  \!D_{ik}^{(+)}(x,u) =  \int\limits_0^u \! \left[ \alpha_c g_1\left(x-\alpha_ct\right) -c_1 
\left( x-\alpha_c t\right) \right] dt+{}\\
  {}+ \int\limits_0^h \!\left[ \alpha_w g_1\left( x-\alpha_c u -\alpha_w t\right) -
c_1\left( x-\alpha_c u-\alpha_w t\right)\right] dt\,.\hspace*{-4.80554pt}
  \end{multline*}
  
  Отметим, что работа по аналитическому преобразованию всех 
вспомогательных вероятностных характеристик, полученных в~тео\-ре\-мах~1--3 
ра\-боты~[1], по\-тре\-бо\-ва\-ла детального исследования\linebreak
 всех воз\-мож\-ных вариантов 
соотношений исходных па\-ра\-мет\-ров модели. При этом выражения для 
характеристик~$d_{ik}^{(-)}$ име\-ют структуру, аналогичную структуре 
характеристик~$d_{ik}^{(+)}$, пред\-став\-лен\-ных в~утверж\-де\-нии~3, однако более 
громоздки по фор\-ме и~связаны с~анализом большего чис\-ла вариантов в~силу 
особенностей областей интегрирования. В~связи с~этим яв\-ные пред\-став\-ле\-ния 
для указанных характеристик не могут быть приведены в~основном текс\-те 
на\-сто\-ящей работы. 

Конкретные формулы для характеристик~$d_{ik}^{(-)}$ 
и~необходимые пояснения к~проведенным аналитическим преобразованиям 
мож\-но найти в~приложении~\cite{6-sh}.
  
\section{Аналитическое представление показателя эффективности 
управления в~форме дробно-линейного интегрального функционала}

  В результате проведения интегральных преобразований, описанных 
в~предыду\-щем разделе, основные вероятностные характеристики 
рас\-смат\-ри\-ва\-емой полумарковской модели пред\-став\-ле\-ны\linebreak
 в~сле\-ду\-ющей форме:
  \begin{align}
  p_{ij} &= \int\limits_0^\infty p_{ij}(u_i)\,dG_i(u_i)\,, \enskip i,j\in \{0,1,\ldots 
,N_0\}\,;
  \label{e2-sh}\\
  T_{i} &= \int\limits_0^\infty T_{i}(u_i)\,dG_i(u_i)\,, \enskip i\in \{0,1,\ldots 
,N_0\}\,;
  \label{e3-sh}\\
  d_{i} &= \int\limits_0^\infty d_{i}(u_i)\,dG_i(u_i)\,, \enskip i\in \{0,1,\ldots 
,N_0\}\,.
  \label{e4-sh}
  \end{align}
    При этом подынтегральные функции $p_{ij}(u_i)$, $d_i(u_i)$ и~$T_i(u_i)$  
заданы явно и~не зависят от вероятностных распределений $G_i(u_i)$, $i\hm= 
0,1,\ldots ,N_0$. 
  
  \noindent
  \textbf{Замечание~1.} Условные математические ожидания~$d_i$ 
непосредственно определяются через вспомогательные вероятностные 
характеристики $d_{ik}^{(+)}$ и~$d_{ik}^{(-)}$, полученные в~разд.~3 
(см.\ Утверж\-де\-ние~3), на основе фор\-му\-лы~(7) работы~\cite{1-sh}. Вероятности 
перехода~$p_{ij}$, $i,j\hm\in \{0,1,\ldots , N_0\}$, определяются аналогичным 
образом при помощи интегральных пред\-став\-ле\-ний для вспомогательных 
вероятностных характеристик $p_{ik}^{(+)}$ и~$p_{ik}^{(-)}$, $i,k\hm\in 
\{0,1,\ldots ,N_0\}$, полученных в~разд.~3 (см.\ Утверж\-де\-ние~1), на осно\-ве 
формулы~(4) работы~\cite{1-sh}. 

  \smallskip
  
  Теперь возникает воз\-мож\-ность использовать тео\-ре\-му об аналитическом 
пред\-став\-ле\-нии стационарного показателя эффективности управ\-ле\-ния 
полумарковским процессом в~форме дроб\-но-ли\-ней\-но\-го интегрального 
функционала, доказанную в~работе~\cite{7-sh}. 
  
  Приведем формулировку указанной тео\-ре\-мы. Заметим предварительно, что 
вероятностная модель со\-про\-вож\-да\-юще\-го управ\-ля\-емо\-го полумарковского 
процесса~$\zeta(t)$, рас\-смат\-ри\-ва\-емо\-го в~на\-сто\-ящем исследовании, пол\-ностью 
соответствует об\-щей модели управ\-ле\-ния полумарковским процессом, 
рас\-смат\-ри\-ва\-емой в~работе~\cite{7-sh}.
  
  \smallskip
  
  \noindent
  \textbf{Теорема~1.}\ \textit{Стационарный стоимостный показатель 
средней удельной прибыли, который изначально задается равенством}~(\ref{e1-sh}), 
\textit{может быть пред\-став\-лен в~виде дроб\-но-ли\-ней\-но\-го интегрального 
функционала от вероятностных распределений~$G_i(u_i)$, $i\hm\in \{0,\ldots 
,N_0\}$. Данный функ\-цио\-нал аналитически определяется сле\-ду\-ющи\-ми 
формулами}:
  \begin{multline}
  I=I\left( G_0(\cdot),\ldots ,G_{N_0}(\cdot)\right) ={}\\
  {}= %\left(
  \int\limits_0^\infty\!\!\cdots\!\!\int\limits_0^\infty 
  \!\!\!A_d(u_0,\ldots , u_{N_0}) 
dG_0(u_0)\cdots%\right.
\\
%\left.
\cdots dG_{N_0}(u_{N_0})%\right)
\Bigg/ \int\limits_0^\infty\!\!\cdots\!\!\int\limits_0^\infty 
  \!\!\!B_d(u_0,\ldots , u_{N_0}) 
dG_0(u_0)\cdots\\
\cdots dG_{N_0}(u_{N_0})\,,
  \label{e5-sh}
  \end{multline}
  \textit{где}
  \begin{equation}
  \left.
  \begin{array}{rl}
  \hspace*{-3mm}A_d\left( u_0,\ldots, u_{N_0}\right) &={}\\[6pt]
  &\hspace*{-26mm}{}=\displaystyle \sum\limits_{i=0}^{N_0} d_i(u_i) 
\hat{D}^i \left( u_0, \ldots , u_{i-1}, u_{i+1},\ldots , u_{N_0}\right)\,;\\[6pt]
  \hspace*{-3mm}B_d\left( u_0,\ldots, u_{N_0}\right) &= {}\\[6pt]
  &\hspace*{-26mm}{}=\displaystyle\sum\limits_{i=0}^{N_0} T_i(u_i) 
\hat{D}^i \left( u_0, \ldots , u_{i-1}, u_{i+1},\ldots , u_{N_0}\right)\,. %\label{e7-sh}
\end{array}
\right\}
\label{e6-sh}
  \end{equation}
    \textit{При этом функции $\hat{D}^i(u_0, \ldots , u_{i-1}, u_{i+1},\ldots\linebreak 
  \ldots, 
u_{N_0})$, $i\hm=0,1,\ldots ,N_0$, входящие в~правые час\-ти 
соотношений}~(\ref{e6-sh}), \textit{пред\-став\-ля\-ют\-ся в~виде}: 
  \begin{multline}
  \hat{D}^i \left( u_0,\ldots , u_{i-1}, u_{i+1}, \ldots , u_{N_0}\right) ={}\\
  {}=
  (-1)^{N_0+i+2} \sum\limits_{\alpha^{N_0,i}} (-1)^{\delta\left(\alpha^{N_0,i}\right)} 
  \times{}\\
  {}\times \hat{D}^i 
\left( \alpha^{N_0,i}, u_0, \ldots , u_{i-1}, u_{i+1},\ldots , u_{N_0}\right)\,,
  \label{e8-sh}
  \end{multline}
  где $\alpha^{N_0,i} = (\alpha_0,\ldots , \alpha_{i-1}, \alpha_{i+1}, \ldots , 
\alpha_{N_0})$~--- \textit{произвольная пе\-ре\-ста\-нов\-ка чисел} $(0,\ldots , i-1, i+1, 
\ldots, N_0)$,
  $\delta(\alpha^{N_0,i})$~--- \textit{чис\-ло инверсий 
  в~перстановке}~$\alpha^{N_0,i}$,
  \begin{multline}
  \hat{D}^i\left( \alpha^{N_0,i}, u_0, \ldots , u_{i-1}, u_{i+1}, \ldots , 
u_{N_0}\right)={}\\
  {}=\tilde{p}_{0,\alpha_0}\left(u_0\right)\cdots \tilde{p}_{i-1,\alpha_{i-1}} \left( 
u_{i-1}\right) \tilde{p}_{i+1, \alpha_{i+1}}\left( u_{i+1}\right)\cdots \\
\cdots
\tilde{p}_{N_0,\alpha_{N_0}}\left( u_{N_0}\right)\,,
  \label{e9-sh}
  \end{multline}
где  
  \begin{multline*}
  \tilde{p}_{k,\alpha_k}(u_k) = \begin{cases}
  p_{kk}(u_k)-1\,, &\ \alpha_k=k\,;\\
  p_{k\alpha_k}(u_k)\,, &\ \alpha_k\not=k\,,
  \end{cases}\\
  k=0,\ldots, i-1, i+1, \ldots , N_0\,.
  \end{multline*}
  
  \noindent
  \textbf{Замечание~2.} Функции $d_i(u_i)$ и~$T_i(u_i)$, $i\hm=0,1,\ldots , N_0$,  
входящие в~правые час\-ти равенств~(\ref{e6-sh}), определяются 
явными пред\-став\-ле\-ни\-ями для характеристик~$d_i$ и~$T_i$, $i\hm=0,1,\ldots , 
N_0$, полученными в~разд.~3. Указанные функ\-ции 
формально совпадают с~подынтегральными выражениями 
в~со\-от\-вет\-ст\-ву\-ющих интегральных пред\-став\-ле\-ни\-ях для $d_i$ и~$T_i$ 
(см.\ формулы~(\ref{e3-sh}) и~(\ref{e4-sh})) и~получаются при под\-ста\-нов\-ке в~эти 
пред\-став\-ле\-ния вы\-рож\-ден\-но\-го вероятностного распределения~$G_i(u)$, 
сосредоточенного в~фиксированной точке $u\hm=u_i$, $i\hm=0,1,\ldots ,N_0$.  
Функции~$p_{ij}(u_i)$, входящие в~правые час\-ти равенств~(\ref{e8-sh}) 
и~(\ref{e9-sh}), определяются явными пред\-став\-ле\-ни\-ями для вероятностей 
перехода~$p_{ij}$, полученными в~разд.~3. Указанные функ\-ции формально 
совпадают с~подынтегральными выражениями в~интегральных 
пред\-став\-ле\-ни\-ях для вероятностей~$p_{ij}$ (см.\ формулу~(\ref{e2-sh})). 
  
  \smallskip
  
  \noindent
  \textbf{Замечание~3.} Функции~$d_i(u_i)$ могут быть также определены 
непосредственно через вспомогательные вероятностные характеристики 
$d_{ik}^{(+)}$ и~$d_{ik}^{(-)}$, полученные в~разд.~3 (см.\ Утверж\-де\-ние~3). Для этого 
необходимо использовать соотношение~(7) работы~\cite{1-sh} и~под\-ста\-вить 
в~со\-от\-вет\-ст\-ву\-ющие выражения для вспомогательных характеристик 
вы\-рож\-ден\-ные вероятностные распределения~$G_i(u)$, как было указано 
в~Замечании~2. Функции~$p_{ij}(u_i)$, $i,j\hm\in \{0,1,\ldots , N_0\}$, 
определяются непосредственно аналогичным образом при помощи 
интегральных пред\-став\-ле\-ний для вспомогательных вероятностных 
характеристик~$p_{ik}^{(+)}$ и~$p_{ik}^{(-)}$, $i,k\hm\in \{0,1,\ldots ,N_0\}$, 
полученных в~разд.~3 (см.\ Утверж\-де\-ние~1). При этом используется 
соотношение~(4) работы~\cite{1-sh} и~применяется формальный прием, 
описанный в~замечании~2.

\vspace*{-6pt}
  
\section{Теоретическое решение задачи оптимального управления}

  Согласно тео\-ре\-ме~1, стационарный стоимостный показатель~(1) мож\-но 
пред\-ста\-вить в~дроб\-но-ли\-ней\-ной интегральной форме~(\ref{e5-sh}), которая 
явно выражает его за\-ви\-си\-мость от вероятностных распределений $G_0(t),\ldots , 
G_{N_0}(t)$, задающих стратегию управ\-ле\-ния. Таким образом, задача поиска 
оптимального управ\-ле\-ния сводится к~экстремальной задаче для  
дроб\-но-ли\-ней\-но\-го интегрального функционала. 
  
   Проблема нахождения безуслов\-но\-го экстремума для  
дроб\-но-ли\-ней\-но\-го интегрального функционала по\-дроб\-но изуче\-на 
в~работах П.\,В.~Шнур\-ко\-ва~\cite{8-sh, 9-sh}. В~част\-ности, доказано, что если 
основная функция дроб\-но-ли\-ней\-но\-го интегрального функционала, 
оп\-ре\-де\-ля\-емая формулой 
  \begin{equation}
  C_d\left( u_0, \ldots , u_{N_0}\right) =\fr{A_d(u_0,\ldots, u_{N_0})} 
{B_d(u_0,\ldots, u_{N_0})}\,,
  \label{e10-sh}
  \end{equation}
достигает глобального экстремума на множестве
$$
U^{N_0} =\left\{ \left( u_0, \ldots , u_{N_0}\right),\ u_i\in [0;\infty)\,,\ i\in 
\{0,\ldots , N_0\}\right\}
$$
в~некоторой фиксированной точ\-ке $u_*\hm=(u_{0*}, \ldots ,u_{N_0*})$, то 
решение экстремальной задачи для функционала $I(G_0(\cdot),\ldots 
,G_{N_0}(\cdot))$ на множестве наборов вероятностных распределений 
$\{G_0(\cdot), \ldots ,G_{N_0}(\cdot)\}$ существует и~достигается на наборе 
вы\-рож\-ден\-ных распределений $G_0^*(\cdot), G_1^*(\cdot), \ldots , 
G^*_{N_0}(\cdot)$, со\-сре\-до\-то\-чен\-ных в~точ\-ках $(u_{0*}, \ldots, u_{N_0*})$ 
соответственно. В~по\-став\-лен\-ной задаче основ\-ная функция  
дроб\-но-ли\-ней\-но\-го интегрального функционала задается аналитически 
формулами~(\ref{e6-sh})--(\ref{e10-sh}). Таким образом, задача оптимального 
управ\-ле\-ния запасом в~рас\-смат\-ри\-ва\-емой полумарковской модели 
получила пол\-ное тео\-ре\-ти\-че\-ское решение.

\vspace*{-6pt}


\section{Итоги теоретического исследования~проблемы 
оптимального~управления}

   Существенная особенность полученных результатов заключается в~том, что 
для заданных исходных па\-ра\-мет\-ров модели может быть найдено 
конкретное решение задачи оптимального управ\-ле\-ния чис\-лен\-ным 
методом. Такая воз\-мож\-ность была впервые реализована в~диссертационной 
работе А.\,В.~Ива\-но\-ва~\cite{10-sh}, в~которой была разработана программа  
чис\-лен\-но\-го на\-хож\-де\-ния оптимальной стратегии управ\-ле\-ния запасом для 
полумарковской модели с~периодическим пре\-кра\-ще\-ни\-ем по\-треб\-ле\-ния, 
тео\-ре\-ти\-че\-ское исследование которой было произведено в~работе~\cite{7-sh}. 
Методика этого чис\-лен\-но\-го решения была в~крат\-кой форме описана 
в~работе~\cite{11-sh}. Отметим так\-же, что в~за\-клю\-чи\-тель\-ной час\-ти 
работы~\cite{9-sh} была указана принципиальная воз\-мож\-ность со\-зда\-ния 
универсальных программных продуктов, пред\-на\-зна\-чен\-ных для чис\-лен\-но\-го 
решения задач оптимального управ\-ле\-ния полумарковскими процессами 
с~конечным множеством со\-сто\-яний. 

\vspace*{-6pt}
  
{\small\frenchspacing
 {%\baselineskip=10.8pt
 \addcontentsline{toc}{section}{References}
 \begin{thebibliography}{99}
  \bibitem{1-sh}
  \Au{Шнурков П.\,В., Егоров~А.\,Ю.} Разработка и~предварительное 
исследование стохастической полумарковской модели управ\-ле\-ния запасом 
непрерывного продукта при по\-сто\-ян\-но происходящем по\-треб\-ле\-нии~// 
Информатика и~её применения, 2018. Т.~12. Вып.~1. С.~109--117.
  \bibitem{2-sh}
  \Au{Джевелл В.} Управ\-ля\-емые полумарковские процессы~// 
Кибернетический сбор\-ник. Новая серия.~--- М.: Мир, 1967. Вып.~4. С.~97--134.
  \bibitem{3-sh}
  \Au{Майн Х., Осаки~С.} Марковские процессы принятия решений~/ Пер. 
с~англ.~--- М.: Наука, 1977. 176~с. (\Au{Mine~H., Osaki~S.} Markovian decision 
processes.~--- New York, NY, USA: Elsevier, 1970. 142~p.).
  \bibitem{4-sh}
  \Au{Luque-Vasquez F., Herndndez-Lerma~O.} Semi-Markov control models with 
average costs~// Appl. Math., 1999. Vol.~26. No.\,3. P.~315--331.
  \bibitem{5-sh}
  \Au{Халмош П.} Теория меры~/ Пер. с~англ.~--- М.: Изд-во ИЛ, 
  1953. 282~с. (\Au{Halmos~P.\,R.} Measure theory.~--- Princeton, NJ, 
USA: Van Nostrand, 1950. 304~p.)
  \bibitem{6-sh}
  \Au{Шнурков П.\,В., Егоров~А.\,Ю.} Приложение к~\mbox{статье} <<Решение 
проб\-ле\-мы оптимального управ\-ле\-ния запасом непрерывного продукта при 
по\-сто\-ян\-но происходящем по\-треб\-ле\-нии в~стохастической полумарковской 
модели>>, 2018. 21~с. {\sf http://www.ipiran.ru/publications/app\_var6.docx}.
  \bibitem{7-sh}
  \Au{Шнурков П.\,В., Иванов~А.\,В.} Анализ дискретной полумарковской 
модели управ\-ле\-ния запасом непрерывного продукта при периодическом 
прекращении по\-треб\-ле\-ния~// Дискретная математика, 2014. Т.~26. №\,1. 
С.~143--154. 
  \bibitem{8-sh}
  \Au{Шнурков П.\,В.} О~решении проб\-ле\-мы без\-услов\-но\-го экстремума для  
дроб\-но-ли\-ней\-но\-го интегрального функционала на множестве 
вероятностных мер~// Докл. Академии наук. Сер. Математика, 2016. Т.~470. 
№\,4. С.~387--392.
  \bibitem{9-sh}
  \Au{Шнурков П.\,В., Горшенин~А.\,К., Белоусов~В.\,В.} Аналитическое 
решение задачи оптимального управ\-ле\-ния полумарковским процессом 
с~конечным множеством со\-сто\-яний~// Информатика и~её применения, 2016. 
Т.~10. Вып.~4. С.~72--88.
  \bibitem{10-sh}
  \Au{Иванов А.\,В.} Анализ дискретной полумарковской модели управ\-ле\-ния 
запасом непрерывного продукта при периодическом прекращении по\-треб\-ле\-ния:\linebreak 
Дис.\ \ldots\ канд. физ.-мат. наук.~--- М.: НИУ ВШЭ, 2014. 120~с.
  \bibitem{11-sh}
  \Au{Gorshenin A.\,K., Belousov~V.\,V., Shnourkoff~P.\,V., Ivanov~A.\,V.} 
Numerical research of the optimal control problem in the semi-Markov inventory 
model~// AIP Conf. Proc., 2015. Vol.~1648. P.~250007-1--250007-4.
   \end{thebibliography}

 }
 }

\end{multicols}

\vspace*{-6pt}

\hfill{\small\textit{Поступила в~редакцию 19.02.18}}

\vspace*{6pt}

\newpage

\vspace*{-28pt}

%\hrule

%\vspace*{2pt}

%\hrule

%\vspace*{8pt}


\def\tit{SOLUTION TO THE PROBLEM OF~OPTIMAL CONTROL 
OF~A~STOCHASTIC SEMI-MARKOV MODEL OF~CONTINUOUS SUPPLY 
OF~PRODUCT MANAGEMENT UNDER~THE~CONDITION OF~CONSTANTLY HAPPENING CONSUMPTION\\[-6pt]}

\def\titkol{Solution to the problem of~optimal control 
of~a~stochastic semi-Markov model of~continuous supply 
of~product management} % under~the~condition of~constantly happening consumption}

\def\aut{P.\,V.~Shnurkov and A.\,Y.~Egorov}

\def\autkol{P.\,V.~Shnurkov and A.\,Y.~Egorov}

\titel{\tit}{\aut}{\autkol}{\titkol}

\vspace*{-16pt}


\noindent
  National Research University Higher School of Economics, 34~Tallinskaya Str., 
Moscow 123458, Russian Federation


\def\leftfootline{\small{\textbf{\thepage}
\hfill INFORMATIKA I EE PRIMENENIYA~--- INFORMATICS AND
APPLICATIONS\ \ \ 2018\ \ \ volume~12\ \ \ issue\ 2}
}%
 \def\rightfootline{\small{INFORMATIKA I EE PRIMENENIYA~---
INFORMATICS AND APPLICATIONS\ \ \ 2018\ \ \ volume~12\ \ \ issue\ 2
\hfill \textbf{\thepage}}}

%\vspace*{3pt}

  
  
  \Abste{The 
  solution of the optimal control problem in the semi-Markov model in question 
  is theoretically justified. To achieve this goal, formal analytic transformations 
  of the integral representations obtained by the authors earlier for the basic probabilistic 
  characteristics of the model were carried out. These transformations made 
  it possible to use the theorem on the analytical representation of the 
  stationary value of the management effectiveness of a~semi-Markov process in 
  the form of a~fractional-linear integral functional. In the sequel, the 
  authors use the general theorem on the extremum of a~fractional-linear integral 
  functional, proved by P.\,V.~Shnurkov. This theorem makes it possible to reduce 
  the problem of optimal reserve management to the problem of investigating the 
  global extremum of a~given function from a~finite number of real nonnegative 
  variables that   can be effectively solved in practice using the
  known numerical methods.}
  
  \KWE{inventory management; semi-Markov stochastic process; stationary value index; 
  optimal control of stochastic systems; fractional-linear integral functional}
  
\DOI{10.14357/19922264180212} %

%\vspace*{-14pt}

 % \Ack
 %  \noindent
 


\vspace*{-6pt}

  \begin{multicols}{2}

\renewcommand{\bibname}{\protect\rmfamily References}
%\renewcommand{\bibname}{\large\protect\rm References}

{\small\frenchspacing
 {\baselineskip=10.4pt
 \addcontentsline{toc}{section}{References}
 \begin{thebibliography}{99}
  \bibitem{1-sh-1}
  \Aue{Shnurkov, P.\,V., and A.\,Y.~Egorov.} 2018. Raz\-ra\-bot\-ka i~predvaritel'noe 
issledovanie stokhasticheskoy polumarkovskoy modeli upravleniya zapasom 
ne\-pre\-ryv\-no\-go pro\-duk\-ta pri po\-sto\-yan\-no pro\-is\-kho\-dya\-shchem 
po\-treb\-le\-nii [Development 
and preliminary study stochastic semi-Markov model of continuous supply of product 
management at constantly happening consumption]. \textit{Informatika i~ee 
Primeneniya~--- Inform. Appl.} 12(1):109--119.
  \bibitem{2-sh-1}
  \Aue{Jewell, W.\,S.} 1963. Markov-renewal programming. \textit{Oper. Res.} 
11:938--971.
  \bibitem{3-sh-1}
  \Aue{Mine,~H., and S.~Osaki.} 1970. \textit{Markovian decision 
processes}. New York, NY: Elsevier. 142~p.
  \bibitem{4-sh-1}
  \Aue{Luque-Vasquez, F., and O.~Herndndez-Lerma.} 1999. Semi-Markov control 
models with average costs. \textit{Appl. Math.} 26(3):315--331.
  \bibitem{5-sh-1}
  \Aue{Halmos, P.\,R.} 1950. \textit{Measure theory}. Princeton, NJ: Van Nostrand.  
304~p.
  \bibitem{6-sh-1}
  \Aue{Shnurkov, P.\,V., and A.\,Y.~Egorov.} 2018. Pri\-lo\-zhe\-nie k~stat'e 
  ``Re\-she\-nie 
prob\-le\-my op\-ti\-mal'\-no\-go uprav\-le\-niya za\-pa\-som ne\-pre\-ryv\-no\-go 
pro\-duk\-ta pri po\-sto\-yan\-no 
pro\-is\-kho\-dya\-shchem po\-treb\-le\-nii v~sto\-ha\-sti\-che\-skoy 
po\-lu\-mar\-kov\-skoy mo\-de\-li'' 
[Appendix to article ``Solution to the problem of optimal control stochastic  
semi-Markov model of continuous supply of product management at constantly 
happening consumption'']. 21~p. Available at: {\sf 
http://www.ipiran.ru/publications/app\_var6.docx} (accessed April~25, 2018).
  \bibitem{7-sh-1}
  \Aue{Shnurkov, P.\,V., and A.\,V.~Ivanov.} 2015. Analysis of a discrete  
semi-Markov model of continuous inventory control with periodic interruptions of 
consumption. \textit{Discrete Math. Appl.} 25(1):59--67.
  \bibitem{8-sh-1}
  \Aue{Shnurkov, P.\,V.} 2016. Solution of the unconditional extremum problem for 
a~linear-fractional integral functional on a~set of probability measures. \textit{Dokl. 
Math.} 94(2):550--554.
  \bibitem{9-sh-1}
  \Aue{Shnurkov, P.\,V., A.\,K.~Gorshenin, and V.\,V.~Belousov.} 2016. 
Ana\-li\-ti\-che\-skoe re\-she\-nie za\-da\-chi op\-ti\-mal'\-no\-go 
uprav\-le\-niya po\-lu\-mar\-kov\-skim 
pro\-tses\-som s~ko\-nech\-nym mno\-zhest\-vom so\-sto\-yaniy 
[Analytical solution of the 
optimal control task of a~semi-Markov process with finite set of states]. 
\textit{Informatika i~ee Primeneniya~--- Inform. \mbox{Appl.}} 10(4):72--88.
  \bibitem{10-sh-1}
  \Aue{Ivanov, A.\,V.} 2014. Analiz dis\-kret\-noy po\-lu\-mar\-kov\-skoy mo\-de\-li 
uprav\-le\-niya za\-pa\-som ne\-pre\-ryv\-no\-go pro\-duk\-ta pri 
pe\-rio\-di\-che\-skom pre\-kra\-shche\-nii 
po\-treb\-le\-niya [Analysis of a~discrete semi-Markov control model of continuous 
product inventory in a~periodic cessation of consumption]. Moscow: HSE Publishing House.
 PhD Thesis. 120~p.
  \bibitem{11-sh-1}
  \Aue{Gorshenin, A.\,K., V.\,V.~Belousov, P.\,V.~Shnourkoff, and 
A.\,V.~Ivanov}. 2015. Numerical research of the optimal control problem in the 
semi-Markov inventory model. \textit{AIP Conf. Proc}.  
1648:250007-1--250007-4.
 \end{thebibliography}

 }
 }

\end{multicols}

\vspace*{-9pt}

\hfill{\small\textit{Received February 19, 2018}}

\vspace*{-24pt}  
  
    \Contr
    
    \vspace*{-3pt}
    
    \noindent
    \textbf{Shnurkov Peter V.} (b.\ 1953)~--- Candidate of Science (PhD) in physics 
and mathematics, associate professor, Moscow Institute of Electronics and
Mathematics, National Research University Higher 
School of Economics, 34~Tallinskaya Str., Moscow 123458, Russian Federation; 
\mbox{pshnurkov@hse.ru}
    
    %\vspace*{3pt}
    
    \noindent
    \textbf{Egorov Artem Y.} (b.\ 1992)~--- Master student, Moscow Institute of Electronics and
Mathematics, National 
Research University Higher School of Economics, 34~Tallinskaya Str., Moscow 
123458, Russian Federation; \mbox{ayuegorov@edu.hse.ru}
  
    
  \label{end\stat}
  
  
  \renewcommand{\bibname}{\protect\rm Литература} 
  