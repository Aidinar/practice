\def\stat{leri}

\def\tit{ОБ УСТОЙЧИВОСТИ КОНФИГУРАЦИОННЫХ ГРАФОВ
В~СЛУЧАЙНОЙ СРЕДЕ$^*$}

\def\titkol{Об устойчивости конфигурационных графов
в~случайной среде}

\def\aut{М.\,М.~Лери$^1$, Ю.\,Л.~Павлов$^2$}

\def\autkol{М.\,М.~Лери, Ю.\,Л.~Павлов}

\titel{\tit}{\aut}{\autkol}{\titkol}

\index{Лери М.\,М.}
\index{Павлов Ю.\,Л.}
\index{Leri M.\,M.}
\index{Pavlov Yu.\,L.}




{\renewcommand{\thefootnote}{\fnsymbol{footnote}} \footnotetext[1]
{Финансовое обеспечение исследований осуществлялось из средств федерального 
бюджета на выполнение государственного задания КарНЦ РАН 
(Институт прикладных математических исследований КарНЦ РАН) 
и~при частичной финансовой поддержке РФФИ (проект №\,16-01-0005a). 
Исследования выполнены на научном оборудовании Центра коллективного пользования 
ФИЦ <<Карельский научный центр РАН>>.}}


\renewcommand{\thefootnote}{\arabic{footnote}}
\footnotetext[1]{Институт прикладных математических исследований КарНЦ РАН, ФИЦ 
<<Карельский научный центр РАН>>, \mbox{leri@krc.karelia.ru}}
\footnotetext[2]{Институт прикладных математических исследований 
КарНЦ РАН, ФИЦ <<Карельский научный центр РАН>>, \mbox{pavlov@krc.karelia.ru}}

\vspace*{-11pt}




\Abst{Рассматриваются случайные конфигурационные графы, степени вер\-шин которых независимы
и~имеют дискретное степенное распределение со случайным па\-ра\-мет\-ром. Законом распределения
па\-ра\-мет\-ра является усеченное гам\-ма-рас\-пре\-де\-ле\-ние. Изучается устой\-чи\-вость таких графов
к двум типам разрушающего воздействия: случайному и~целенаправленному. Функционирование
графов происходит в~случайной среде, когда значения па\-ра\-мет\-ра распределения степеней
вершин выбираются для каж\-дой вершины отдельно. Проведен сравнительный анализ последствий
раз\-ру\-ша\-ющих воздействий в~таких моделях и~в графах с~общим для всех вершин распределением
степеней, индуцированным усреднением по распределению па\-ра\-мет\-ра. Обсуждаются условия,
при которых исследование поведения графов в~случайной среде может быть сведено к~изучению
эволюции графов с~усредненным распределением степеней. Проведен сравнительный анализ
последствий этих двух типов разрушающих воздействий.}

\KW{конфигурационные графы; степенное распределение;
гам\-ма-рас\-пре\-де\-ле\-ние; устойчивость; модель лесного пожара; 
имитационное моделирование}

\DOI{10.14357/19922264180201}
  
\vspace*{-1pt}


\vskip 10pt plus 9pt minus 6pt

\thispagestyle{headings}

\begin{multicols}{2}

\label{st\stat}


\section{Введение}
%\label{SC:1}

%\vspace*{2pt}

Широкое использование случайных графов для моделирования топологии и~динамики
развития таких слож\-ных сетей коммуникаций, как Интернет, социальные 
и~телекоммуникационные сети (см., например,~\cite{Dur,Hof}), привело к~появлению
новых важ\-ных на\-прав\-ле\-ний исследования по\-доб\-ных структур. Одним из 
на\-прав\-ле\-ний
является исследование устой\-чи\-вости случайных графов к~различным раз\-ру\-ша\-ющим
воздействиям \cite{Dur,Coh,Bol2}, име\-ющим как случайный, так и~целенаправленный
характер.
{\looseness=1

}

Наблюдения за реальными сетями показали (см., например,~\cite{Dur,Fa,RN1}), что
степени вершин соответствующих графов мож\-но считать независимыми одинаково
распределенными случайными величинами, име\-ющи\-ми дискретное степенное распределение.
Одними из наиболее подходящих для моделирования сетей видов случайных графов являются
так называемые конфигурационные графы~\cite{Bol1} со случайными независимыми степенями.

Будем рассматривать конфигурационные графы, состоящие из~$N$~вершин, степени которых
являются независимыми одинаково распределенными случайными величинами, име\-ющи\-ми
сле\-ду\-ющее дискретное степенное распределение:

%\vspace*{-3pt}

\noindent
\begin{multline}
\label{EQ:1}
{\sf P}\{\xi = k\} = k^{-\tau} - (k+1)^{-\tau}\,, \\[-2pt] 
k=1,2,\dots; \enskip \tau >0\,,
\end{multline}

\vspace*{-6pt}

\noindent где $\xi$~--- случайная величина, равная степени произвольной вершины.
Случайная величина~$\xi$ принимает целые положительные значения, равные чис\-лу полуребер
вершины, т.\,е.\ ребер, для которых смежные вершины еще не определены. Все полуребра
занумерованы в~произвольном порядке. Поскольку сумма степеней вершин конфигурационного
графа должна быть четной, в~случае нечетной суммы к~равновероятно выбранной вершине
добавляется еще одно ребро. Построение графа завершается путем попарного равновероятного
соединения полуребер друг с~другом для образования ребер. Ясно, что такое
строение графа допускает наличие петель, кратных ребер и~циклов.

Во многих работах отмечается, что для реальных сетей коммуникаций обыч\-но можно
считать, что~$\tau$ фиксировано и~принадлежит интервалу $(1,2)$ (см., 
например,~\cite{Dur,Hof,Fa,RN1}). Но в~некоторых задачах, как показали последние
исследования \cite{Ler1,Ler3}, пред\-став\-ля\-ют интерес и~модели, в~которых $\tau \hm>2$.
Известны также модели, в~которых распределения степеней вершин могут
изменяться при воз\-рас\-та\-нии чис\-ла вершин и~даже быть случайными~\cite{Bia}.

В статье~\cite{Pav} при $N\hm\rightarrow\infty$ рас\-смат\-ри\-ва\-лась динамика степенной
структуры конфигурационного графа, степени вершин которого имеют 
распределение~(\ref{EQ:1}), а~параметр~$\tau$ является случайной величиной, равномерно
распределенной на отрезке $[a,b]$, $0\hm<a\hm<b\hm<\infty$. Однако более естественным
пред\-став\-ля\-ет\-ся предположение, что распределение~$\tau$, заданное на конечном интервале,
является унимодальным. 

В~на\-сто\-ящей работе предполагается, что~$\tau$ имеет
усеченное гам\-ма-рас\-пре\-де\-ле\-ние на отрезке $[a,b]$ с~параметрами $(2,\lambda)$.
Плот\-ность этого распределения имеет сле\-ду\-ющий вид:
\begin{equation}\label{EQ:2}
f_{\lambda}(x)=\fr{\lambda^2xe^{-\lambda x}}{F_{2,\lambda}(b)-
F_{2,\lambda}(a)}\,, \enskip x\in[a,b]\,,
\end{equation}

\noindent где $F_{2,\lambda}(a)$ и~$F_{2,\lambda}(b)$~--- значения функции
гам\-ма-рас\-пре\-де\-ле\-ния с~па\-ра\-мет\-ра\-ми~$2$ и~$\lambda$ в~точ\-ках~$a$ и~$b$
соответственно.

Главной целью данной работы было изучение устой\-чи\-вости таких конфигурационных графов
к~раз\-ру\-ша\-ющим воздействиям при условии, что эволюция графов происходит в~случайной
среде. Это означает, что степени вершин имеют распределение~(\ref{EQ:1}), в~котором
значение параметра~$\tau$ является случайным и~выбирается отдельно для каждой вершины
из распределения с~плот\-ностью~(\ref{EQ:2}). 
%
Заметим, что, используя~(\ref{EQ:1}) 
и~(\ref{EQ:2}), мож\-но усред\-нить распределение случайной величины~$\xi$ и~получить,
что
\begin{multline}
\label{EQ:3}
{\sf P}\{\xi=k\}=\fr{\lambda^2}{F_{2,\lambda}(b)-F_{2,\lambda}(a)}\times{}\\
{}\times \int\limits_a^b
xe^{-\lambda x}(e^{-x\ln k}-e^{-x\ln{(k+1)}})\,dx\,,
\end{multline}
где $k=1,2,\dots$ Понятно, что проще изучать поведение таких графов,
в которых степени всех вершин име\-ют общее распределение~(\ref{EQ:3}). Поэтому была
поставлена задача срав\-нить устой\-чи\-вость графов в~случайной среде и~в~случае
распределения~(\ref{EQ:3}), когда случайной среды не возникает. Это поз\-во\-лит вы\-явить
условия, при которых изучение случайных графов в~случайной среде мож\-но заменить на
исследование графов с~общим распределением степеней вершин. 

Заметим, что одним из
интенсивно раз\-ви\-ва\-ющих\-ся на\-прав\-ле\-ний современной тео\-рии вероятностей является
изучение случайных процессов в~случайной среде, что отражено в~мно\-го\-чис\-лен\-ных
пуб\-ли\-ка\-ци\-ях как в~нашей стране, так и~за рубежом (см., например,~\cite{Afan}). Однако
авторам неизвестны статьи других исследователей, посвященные случайным графам, 
фор\-ми\-ру\-ющим\-ся в~случайной среде. В~данной работе (см.\
 так\-же материалы конференций~\cite{Pav,Ler4})
такие задачи, по-ви\-ди\-мо\-му, рассматриваются впервые.

\columnbreak


\section{Разрушение графов} %\label{SC:2}

\vspace*{-9pt}

Под разрушением графов далее понимается по\-сле\-до\-ва\-тель\-ное удаление вер\-шин вместе
с~инцидентными им реб\-ра\-ми. Известно (см., например,~\cite{RN1}), что если
$\tau\hm\in(1,2)$, то конфигурационный граф имеет единственную гигантскую компоненту
связ\-ности, объем которой при $N\rightarrow\infty$ пропорционален чис\-лу вершин.
Объем любой другой компоненты является бесконечно малой величиной по срав\-не\-нию с~объемом гигантской компоненты. Если $\tau\hm>2$, то в~графе отсутствует гигантская
компонента, но, как показали исследования~\cite{Ler1,Ler3}, в~практически
важ\-ных случаях и~при таких значениях па\-ра\-мет\-ра~$\tau$ наибольшая по объему компонента
связности содержит значительно больше вершин, чем любая другая. Ниже наибольшую
компоненту будем называть максимальной независимо от того, является ли она гигантской
или нет. Обозначим $\eta$ долю вершин графа, входящих в~максимальную компоненту; 
$\eta_2$~--- долю вершин, входящих во вторую по объему компоненту. В~\cite{Ler1}
был предложен критерий разрушения графа, со\-глас\-но которому граф считается разрушенным,
если произошло событие ${A}:\{\eta\hm\leq 2\eta_2\}$. Рас\-смат\-ри\-ва\-лись два вида
разрушения: 
случайное и~целенаправленное. При случайном разрушении вершины, подлежащие
по\-сле\-до\-ва\-тель\-но\-му удалению из графа, выбираются равновероятно, 
тогда как в~случае
целенаправленной атаки удаляются вершины с~максимальными степенями.


Обозначим $p={\sf P}\{{A}\}$ ве\-ро\-ят\-ность события~${A}$, а~$r$~--- доля
удаленных из графа вершин. 
Для указанных двух видов разрушений были найдены зависимости~$p$ 
и~$\eta$ от~$r$ и~$\lambda$. В~этих регрессионных моделях выражение $\xi\sim F[a,b]$
означает, что рас\-смат\-ри\-ва\-лись графы, степени вершин которых имеют распределение~(\ref{EQ:3}),
а~$\tau\sim\mathrm{G}[a,b]$ означает, что процесс разрушения происходит в~условиях случайной
среды. Имитационные эксперименты по разрушению графов проводились для моделей объемом
от~1000 до~10\,000~вершин с~шагом~1000, с~параметром $0{,}3\hm\leq\lambda\hm\leq 2{,}5$
(с~шагом~0,2 и~0,3) для трех интервалов $[a,b]$: $(1,2)$, $(1,3]$ и~$[2,3]$. 
Для каждой
тройки $([a,b],N,\lambda)$ генерировалось по~100~моделей, что в~целом 
со\-ста\-ви\-ло~30\,000
экспериментов. Конфигурационные графы с~па\-ра\-мет\-ром~$\tau$ из интервала $(1,2)$, 
со\-глас\-но~\cite{Fa,RN1}, хорошо подходят для описания сложных телекоммуникационных 
сетей. Графы
с~па\-ра\-мет\-ром $\tau\hm\in[2,3]$ показали~\cite{Ler1,Ler3,Ler2} свою
 устой\-чи\-вость к~<<лесному пожару>>
(см.\ разд.~3), а~интервал $(1,3]$ был рас\-смот\-рен в~качестве обобщения.



Таким образом, в~случае целенаправленного разрушения были получены 
сле\-ду\-ющие регрессионные модели:

\pagebreak

\noindent
\begin{align*}
\tau\sim {G}(1,2):\ \  & \eta = 49,3+2,91\lambda-7,76r-7,21\ln r,\\
& p = -0,09-0,084\lambda+0,082r^{1,64};\\
\xi\sim F(1,2):\ \  & \eta = 48,94+2,74\lambda-7,55r-7,44\ln r,\\
& p = -0,09-0,09\lambda+0,086r^{1,6};\\
\tau\sim {G}[2,3]:\ \  & \eta = -3,94+1,53\lambda+3,3r-5,64\ln r,\\
& p = 1,25-0,12\lambda+0,37\ln r;\\
\xi\sim F[2,3]:\ \  & \eta = -3,77+1,52\lambda+3,13r-5,58\ln r,\hspace*{-2.8pt}\\
& p = 1,25-0,13\lambda+0,36\ln r;\\
\tau\sim {G}(1,3]:\  \ & \eta = 26,94+10,34\lambda-7,1r-8,78\ln r,\\
& p = 0,012-0,29\lambda+0,34r;\\
\xi\sim F(1,3]:\ \  & \eta = 26,7+10,48\lambda-7,05r-8,79\ln r,\\
& p = 0,01-0,3\lambda+0,35r.
\end{align*}

Коэффициенты детерминации всех приведенных выше моделей имеют значения не
ниже~0,94. Значения~$r$ в~моделях за\-ви\-си\-мости ве\-ро\-ят\-ности разрушения графа~$p$ 
от~$\lambda$ и~$r$ име\-ют сле\-ду\-ющие ограничения:

\noindent
\begin{gather*}%{2}
1{,}13+0{,}44\lambda\leq r\leq 4{,}85+0{,}22\lambda\ \mbox{при }
\tau\sim{G}(1,2);\\
1{,}11+0{,}47\lambda\leq r\leq 4{,}9+0{,}24\lambda\  \mbox{при }
\xi\sim F(1,2);\\
0{,}03+0{,}02\lambda\leq r\leq 0{,}54+0{,}16\lambda^{1,4}\
\mbox{при } \tau\sim {G}[2,3];\\
0{,}026+0{,}019\lambda\leq r\leq 0{,}53+0{,}18\lambda^{1,5}\
\mbox{при } \xi\sim F[2,3];\\
-0{,}035+0{,}85\lambda\leq r\leq 2{,}91+0{,}85\lambda\
\mbox{при } \tau\sim {G}(1,3];\\
-0{,}03+0{,}86\lambda\leq r\leq 2{,}83+0{,}86\lambda\
\mbox{при } \xi\sim F(1,3]\,,
\end{gather*}
где $R^2\hm=0,99$ для всех моделей. Ясно, что $p\hm=0$ для значений~$r$, 
меньших ниж\-ней
границы, и~$p\hm=1$ для~$r$, больших верх\-не\-го ограничения. То же самое вер\-но и~для
со\-от\-вет\-ст\-ву\-ющих моделей, приведенных ниже.

Для процесса случайных разрушений были получены сле\-ду\-ющие модели:

\noindent
\begin{align*}
\tau\sim {G}(1,2):\quad & \eta = 83{,}02+2{,}19\lambda-1{,}45r,\\
& p = -0{,}75-0{,}035\lambda+0{,}00037r^2;\\
\xi\sim F(1,2):\quad & \eta = 82{,}98+2{,}21\lambda-1{,}44r,\\
& p = -0{,}72-0{,}026\lambda+0{,}00036r^2;\\
\tau\sim {G}[2,3]:\quad & \eta = 18{,}97+1{,}82\lambda-0{,}28r-2{,}1\ln r,\\
& p = 0{,}13-0{,}077\lambda+0{,}00035r^2;\\
\xi\sim F[2,3]:\quad & \eta = 18{,}95+1{,}82\lambda-0{,}28r-2{,}13\ln r,\\
& p = 0{,}13-0{,}075\lambda+0{,}00035r^2;\\
\tau\sim {G}(1,3]:\quad & \eta = 68{,}63+6{,}24\lambda-1{,}33r,\\
& p = -0{,}54-0,074\lambda+0{,}00034r^2;\\
\xi\sim F(1,3]:\quad & \eta = 68{,}65+6,28\lambda-1{,}33r,\\
& p = -0{,}55-0{,}086\lambda+0{,}00036r^2
\end{align*}

\noindent 
с~коэффициентами детерминации не ниже~0,92 
и~сле\-ду\-ющи\-ми ограничениями на значения~$r$ в~моделях для 
ве\-ро\-ят\-ности разрушения~$p$:

\noindent
\begin{gather*}
45{,}04+1{,}02\lambda\leq r\leq 68{,}8+0{,}68\lambda \mbox{ при }
\tau\sim\mathrm{G}(1,2);\\
44{,}73+0{,}79\lambda\leq r\leq 69{,}12+0{,}52\lambda 
\mbox{ при } \xi\sim F(1,2);\\
-0{,}65+0{,}6\lambda^{3,6}\leq r\leq 49{,}9+2{,}08\lambda
\mbox{ при } \tau\sim {G}[2,3];\\
-0{,}57+0{,}43\lambda^{3,9}\leq r\leq 50+2{,}03\lambda \mbox{ при } \xi\sim F[2,3];\\
40+2{,}5\lambda\leq r\leq 67{,}3+1{,}57\lambda 
\mbox{ при } \tau\sim {G}(1,3];\\ 
39{,}23+2{,}77\lambda\leq r\leq 65{,}65+1{,}75\lambda 
\mbox{ при } \xi\sim F(1,3].
\end{gather*}

Регрессионные модели обычно нивелируют отклонения, поэтому было проведено 
срав\-не\-ние\linebreak
исходного материала, использованного для по\-стро\-ения этих моделей. Для каждой тройки
па\-ра\-мет\-ров $([a,b],\lambda,r)$ вы\-чис\-ля\-лась раз\-ни\-ца между такими основными
характеристиками, как объем максимальной компоненты~$\eta$ и~ве\-ро\-ят\-ность
разрушения графа~$p$, полученными для графов в~случайной среде и~для графов 
с~распределением (\ref{EQ:3}) степеней вершин. На рис.~1 и~2 приведены примеры результатов
вы\-чис\-ле\-ния $\triangle\eta\hm=\eta_G\hm-\eta_F$ (см.\ рис.~1) 
и~$\triangle p\hm=p_G\hm-p_F$ (см.\ рис.~2),
где $\eta_G$~--- объем максимальной компоненты для графов в~случайной среде;
$\eta_F$~--- объем максимальной компоненты для графов с~распределением~(\ref{EQ:3}) 
степеней вершин; $p_G$ и~$p_F$~--- вероятности разрушения графа для графов
в~случайной среде и~с~рас\-пре\-де\-ле\-ни\-ем~(\ref{EQ:3}) соответственно.



Эти и~аналогичные результаты, полученные для других интервалов $[a,b]$, показывают,
что при обоих видах процесса разрушения графа (целенаправленном и~случайном) разница 
в~динамике изменения объема гигантской компоненты не превышает~3\%. Что касается вероятности
разрушения графа, здесь уже в~некоторых случаях (см., например, рис.~2,\,\textit{б}) разница может
превышать~5\%.


Полученные результаты показывают, что регрессионные модели закономерностей, 
опи\-сы\-ва\-ющих
разрушения максимальной компоненты в~случайной среде и~с~распределением~(\ref{EQ:3})
степеней вершин, практически совпадают. Это означает, что исследование таких процессов
разрушения в~случайной среде может быть заменено исследованием более прос\-то\-го случая 
с~усредненным распределением степеней вершин при изучении процесса целенаправленного
разрушения графов. Однако при исследовании вероятностей случайных разрушений в~случае
такой замены следует учесть воз\-мож\-ность заметных отклонений при достаточно больших
значениях па\-ра\-метра.

\vspace*{-9pt}

\section{<<Лесные пожары>>} %\label{SC:3}

Второй вид процесса разрушения известен как <<модель лесного пожара>> (см., 
например,~\cite{Dros,Ber}), которая так\-же может быть использована при изуче\-нии банковских 
кризисов~\cite{Ari}.  В~этом случае вер-\linebreak\vspace*{-12pt}



\pagebreak

\end{multicols}

\begin{figure*} %fig1
  \vspace*{1pt}
 \begin{center}
 \mbox{%
 \epsfxsize=123.097mm 
 \epsfbox{ler-1.eps}
 }
 \end{center}
\vspace*{-9pt}
\Caption{Сравнение динамики объемов максимальных компонент: 
(\textit{а})~при целенаправленном
разрушении для интервала $(1,2)$; (\textit{б})~при случайном разрушении для 
интервала $[2,3]$}
%\vspace*{12pt}
\end{figure*}

\begin{multicols}{2}


\noindent
шины конфигурационного графа интерпретируются как деревья,
растущие на некоторой ограниченной территории, а~реб\-ра графа~--- это воз\-мож\-ные пути перехода
<<огня>> от одной вершины к~другой. 

Рас\-смат\-ри\-ва\-лась задача сравнения устой\-чи\-вости 
к~<<лесному пожару>> двух типов моделей конфигурационных графов:\\[-15pt] 
\begin{enumerate}[(1)]
\item графов, степени вершин
которых имеют распределение~(\ref{EQ:1}) с~па\-ра\-мет\-ром~$\tau$, 
име\-ющим распределение 
с~плот\-ностью~(\ref{EQ:2})  (графы
в~случайной
 среде);
 \item графов, распределение
 степеней вершин
которых имеет вид~(\ref{EQ:3}).\\[-15pt] 
\end{enumerate}

Для согласования топологии конфигурационного графа 
с~моделью <<лес\-но\-го пожара>> в~предыду\-щих работах (см., например,~\cite{Ler1,Ler3,Ler2})
было пред\-ло\-же\-но использовать вспомогательные графы, %\linebreak 
верши\-ны которых располагаются 
в~узлах квад\-рат\-ной це\-ло\-чис\-лен\-ной решетки. 
Реб\-ра вспо\-мо\-га\-тель\-но\-го графа со\-еди\-ня\-ют вер\-ши\-ны по
прин\-ци\-пу <<бли\-жай\-ше\-го соседа>>. Таким образом, степень
 лю\-бой\linebreak вершины, на\-ходящейся внутри
решетки, не пре-\linebreak\vspace*{-12pt}

\pagebreak



\end{multicols}

\begin{figure*} %fig2
 \vspace*{1pt}
 \begin{center}
 \mbox{%
 \epsfxsize=118.748mm 
 \epsfbox{ler-2.eps}
 }
 \end{center}
\vspace*{-9pt}
\Caption{Сравнение вероятностей разрушения графа: 
(\textit{а})~при целенаправленном
разрушении для интервала $(1,3]$;
(\textit{б})~при случайном разрушении для интервала $[2,3]$}
\vspace*{3pt}
\end{figure*}

\begin{multicols}{2}


\noindent
вышает~8. Вспомога\-тельные
 графы были использованы для 
уста\-нов\-ле\-ния за\-ви\-си\-мости
меж\-ду па\-ра\-мет\-ром~$\lambda$ распределения
 степеней вершин и~чис\-лом~$N$~исходных вершин
конфигурационного\linebreak графа. 
{ %\looseness=-1

}

Для нахождения такой за\-ви\-си\-мости были вы\-чис\-ле\-ны 
сред\-ние значения $m\hm=m(N)$ степеней вершин 
(см.\ также~\cite{Ler3}) при различных значениях
 $N\hm\leq 10\,000$ и~различных вариантах размещения~$N$~вершин на решетке 
 размера $100\times 100$ (табл.~1),
при этом, учитывая явный вид математического ожидания~$\xi$, 
использовалось соотношение:
{\looseness=1

}

\noindent
\begin{multline*}
%\label{EQ:4}
m\approx\int\limits_a^b x\fr{\lambda^2 x e^{-\lambda x}\,dx}
{F_{2,\lambda}(b)-F_{2,\lambda}(a)}={}\\[-1pt]
{}=
\fr{1}{\lambda}\, \fr{\Gamma(3)}{F_{2,\lambda}(b)-F_{2,\lambda}(a)}
\left(F_{3,\lambda}(b)-F_{3,\lambda}(a)\right)\,.
\end{multline*}

\vspace*{-2pt}






На основании приведенных в~табл.~1 данных были найдены регрессионные 
зависимости~$\lambda$ 
от~$N$ (рис.~3) для трех интервалов $[a,b]$:

 {   %tabl1

\noindent
{{\tablename~1}\ \ 
\small{Данные для нахождения за\-ви\-си\-мости между~$N$ и~$\lambda$}}



\begin{center}
{\small 
\tabcolsep=12.2pt
\begin{tabular}{|l|c|l|l|l|}
\hline
\multicolumn{1}{|c|}{\raisebox{-6pt}[0pt][0pt]{$m$}}& 
\multicolumn{1}{c|}{\raisebox{-6pt}[0pt][0pt]{$N$}}& \multicolumn{3}{c|}{$\lambda$}\\
\cline{3-5}
 &   & (1,2) & [2,3] & (1,3]\\ 
\hline
1,01 & 3350 & 0,824 & 0,638 & 0,722\\
1,21 & 3600 & 0,904 & 0,699 & 0,795\\
1,33 & 3750 & 0,949 & 0,733 & 0,836\\
1,42 & 3900 & 0,981 & 0,758 & 0,866\\
1,5 & 4000 & 1,009 & 0,779 & 0,892\\
1,6 & 4200 & 1,043 & 0,805 & 0,924\\
2 & 5000 & 1,17 & 0,902 & 1,043\\
2,66 & 4780 & 1,357 & 1,043 & 1,221\\
3 & 4489 & 1,444 & 1,108 & 1,306\\
4 & 5578 & 1,679 & 1,284 & 1,536\\
5 & 6700 & 1,888 & 1,439 & 1,745\\
5,33 & 7500 & 1,952 & 1,487 & 1,811\\
6 & 8350 & 2,078 & 1,58 & 1,939\\
6,93 & 8680 & 2,244 & 1,701 & 2,109\\
7 & 8911 & 2,256 & 1,71 & 2,122\\
8 & 10000\hphantom{9} & 2,422 & 1,831 & 2,294\\ 
\hline
\end{tabular}}
\end{center}
}
%\end{table*}

\vspace*{9pt}

 { \begin{center}  %fig3
 \vspace*{-2pt}
  \mbox{%
 \epsfxsize=77.942mm 
 \epsfbox{ler-3.eps}
 }


\end{center}


\noindent
{{\figurename~3}\ \ \small{Зависимости $N$ от~$\lambda$ на интервалах 
$(1,2)$~(\textit{1}), 
$[2,3]$~(\textit{2}) и~$(1,3]$~(\textit{3})}}
}

\vspace*{14pt}





\noindent
\begin{equation}
\label{EQ:5}
N(\lambda)=\begin{cases}
1986{,}36 e^{0,66\lambda} & \quad \mbox{для}~(1,2);\\
1934{,}98 e^{0,89\lambda} & \quad \mbox{для}~[2,3];\\
2143{,}92 e^{0,67\lambda} & \quad \mbox{для}~(1,3]
\end{cases}
\end{equation}

\noindent 
с~коэффициентами детерминации $R^2\hm=0{,}99$ для всех трех моделей.


Процесс <<лесного пожара>> со\-сто\-ит в~сле\-ду\-ющем. Возгорание начинается с~некоторой
вершины графа, которая может быть либо выбрана равновероятно (случайное возгорание),
либо имеет максимальную степень, т.\,е.\ происходит це\-ле\-на\-прав\-лен\-ный поджог. После того
как первая вершина загорелась, огонь переходит по реб\-рам на инцидентные вершины 
с~ве\-ро\-ят\-ностью $0\hm<p\hm\leq 1$. Ве\-ро\-ят\-ность~$p$ 
либо фиксирована для всех ребер графа, либо
рав\-но\-мер\-но распределена на интервале $(0,1]$ и~генерируется отдельно для 
каж\-до\-го ребра.

Имитационное моделирование процесса <<лесного пожара>> проводилось для моделей 
с~па\-ра\-мет\-ром $0{,}5\hm\leq\lambda\hm\leq 2{,}0$, изменяющимся с~шагом~0,1, 
для трех интервалов
$[a,b]$: $(1,2)$, $(1,3]$ и~$[2,3]$. Объемы графов вы\-чис\-ля\-лись из 
уравнений~(\ref{EQ:5}).

\setcounter{table}{1}

\begin{table*}\small
\hspace*{3mm}\begin{minipage}[t]{75mm}
\begin{center}
\Caption{Сравнение числа оставшихся после пожара вершин ($0<p\hm\leq 1$ фиксировано)}
\vspace*{2ex}



\tabcolsep=7.1pt
\begin{tabular}{|l|c|c|c|}
\hline
\multicolumn{1}{|c|}{Вид} & \multicolumn{3}{c|}{Интервал $[a,b]$}\\ 
\cline{2-4}
\multicolumn{1}{|c|}{возгорания} & (1,2) & [2,3] & (1,3]\\ 
\hline
\tabcolsep=0pt\begin{tabular}{l}Целенаправленный\\ поджог\end{tabular}& 0,25\% & {\bf 6,19\%} & 4,87\%\\ 
\hline
\tabcolsep=0pt\begin{tabular}{l}Случайное\\  возгорание\end{tabular} & 4,06\% & 4,94\% & {\bf 5,87\%}\\ 
\hline
\end{tabular}
\end{center}
\end{minipage}
\hfill
%\end{table*}
%\begin{table*}\small %tabl3
\begin{minipage}[t]{75mm}
\begin{center}
\Caption{Сравнение числа оставшихся после пожара вершин}
\label{Tab3}
\vspace*{2ex}

\tabcolsep=9.4pt
\begin{tabular}{|l|c|c|c|}
\hline
\multicolumn{1}{|c|}{Вид} & \multicolumn{3}{c|}{Интервал $[a,b]$}\\ 
\cline{2-4}
\multicolumn{1}{|c|}{возгорания} & (1,2) & [2,3] & (1,3]\\ 
\hline
\tabcolsep=0pt\begin{tabular}{l}Целенаправленный\\ поджог\end{tabular} & 0 & 1 & 2\\ 
\hline
\tabcolsep=0pt\begin{tabular}{l}Случайное\\  возгорание\end{tabular}& 1 & 1 & 1\\ 
\hline
\end{tabular}
\end{center}
\end{minipage}\hspace*{3mm}
\end{table*}


Рассмотрим сначала результаты экспериментов, полученные в~случае фиксированных значений
вероятности перехода огня по реб\-ру графа $0\hm<p\hm\leq 1$, за\-да\-ва\-емых с~шагом~0,01.
 Для каждой
тройки па\-ра\-мет\-ров $([a,b],\lambda,p)$ было по\-стро\-ено по~100 моделей графов. 
Найдены
за\-ви\-си\-мо\-сти чис\-ла~$n$~оставшихся
 после пожара вершин графа от па\-ра\-мет\-ров~$\lambda$ и~$p$
для трех интервалов $[a,b]$. В~случае на\-прав\-лен\-но\-го поджога были получены следующие
регрессионные модели:

\noindent
\begin{align*}
\tau\sim {G}(1,2):\quad & n = -1477{,}5+2320{,}14\lambda-{}\\
&\hspace*{8mm}{}-1016{,}68\ln \lambda-1199{,}93\ln p;\\
\xi\sim F(1,2):\quad & n = -1479{,}46+2322{,}17\lambda-{}\\
&\hspace*{8mm}{}-1016{,}65\ln\lambda- 1199{,}88\ln p;\\
\tau\sim {G}[2,3]:\quad & n = -4608{,}64+9630{,}65\lambda-{}\\
&\hspace*{12mm}{}-5139{,}23\ln\lambda-1301{,}38p;\\
\xi\sim F[2,3]:\quad & n = -4657{,}43+9671{,}31\lambda-{}\\
&\hspace*{12mm}{}-5184{,}17\ln\lambda-1276{,}98p;\\
\tau\sim {G}(1,3]:\quad & n = -1125{,}4+2473{,}31\lambda-{}\\
&\hspace*{8mm}{}-1087{,}46\ln\lambda-1209{,}57\ln p;\\
\xi\sim F(1,3]:\quad & n = -1180+2521{,}48\lambda-{}\\
&\hspace*{8mm}{}-1146{,}44\ln\lambda-1209{,}79\ln p,
\end{align*}
коэффициенты детерминации которых не ниже~0,95. Аналогично при рас\-смот\-ре\-нии
случайного возгорания модели были сле\-ду\-ющими:

\noindent
\begin{align*}
\tau\sim {G}(1,2):\quad & n = 1395{,}8+3255\lambda-{}\\
&\hspace*{14mm}{}-1444\ln\lambda -3848{,}76p;\\
\xi\sim F(1,2):\quad & n = 1434{,}29+3218{,}54\lambda-{}\\
&\hspace*{10mm}{}-1403{,}52\ln\lambda-3837{,}88p;\\
\tau\sim {G}[2,3]:\quad & n = -5657{,}21+10335{,}37\lambda-{}\\
&\hspace*{12mm}{}-5563{,}53\ln\lambda-236{,}79p;\\
\xi\sim F[2,3]:\quad & n = -5674{,}23+10349{,}4\lambda-{}\\
&\hspace*{14mm}{}-5578{,}1\ln\lambda-230{,}24p;\\
\tau\sim {G}(1,3]:\quad & n = 1612{,}9+3465{,}12\lambda-{}\\
&\hspace*{14mm}{}-1469{,}7\ln\lambda-3539{,}2p;\\
\xi\sim F(1,3]:\quad & n = 1443{,}68+3615{,}87\lambda-{}\\
&\hspace*{10.5mm}{}-1632{,}72\ln\lambda-3541{,}65p
\end{align*}

\noindent со значениями коэффициентов детерминации не ниже~0,89.

Для каждой тройки значений $([a,b],\lambda,p)$ было проведено попарное срав\-н\-ение
выборок, со\-сто\-ящих каж\-дая из~100~значений~$n$~--- чис\-ла остав\-ших\-ся
 после пожара вершин.
Таким образом, всего было сформировано~1600~выборок для каждого интервала. Для всех
пар выборок (в~случайной среде и~с~распределением~(\ref{EQ:3}) степеней вершин)
проверялась гипотеза о~равенстве сред\-них с~уров\-нем зна\-чи\-мости~0,05. 
Вслед\-ст\-вие отсутствия
нор\-маль\-ности был использован критерий сравнения Ман\-на--Уит\-ни. 
В~табл.~2 приведены
доли случаев, в~которых гипотеза отвергалась для целенаправленного поджога 
и~случайного возгорания.





Далее рассматривался процесс случайного распространения огня на срав\-ни\-ва\-емых графах,
т.\,е.\ случай, когда ве\-ро\-ят\-ность~$p$~перехода огня по реб\-ру графа не фиксирована, 
а~является случайной величиной, рав\-но\-мер\-но распределенной на интервале $(0,1]$. Аналогично
были по\-стро\-ены регрессионные модели зависимостей чис\-ла оставшихся в~графе вершин~$n$ от
па\-ра\-мет\-ра~$\lambda$. В~случае целенаправленного поджога были получены 
сле\-ду\-ющие зависимости:

\noindent
\begin{align*}
\tau\sim {G}(1,2):\quad & n = -228{,}81+2161{,}35\lambda-937{,}28\ln\lambda;\\
\xi\sim F(1,2):\quad & n = -235{,}57+2166{,}73\lambda-942{,}78\ln\lambda;\\
\tau\sim {G}[2,3]:\quad & n = -5680{,}61+10162{,}21\lambda-{}\\
&\hspace*{34mm}{}-5458{,}21\ln\lambda;\\
\xi\sim F[2,3]:\quad & n = -5715{,}7+10202{,}6\lambda-{}\\
&\hspace*{34mm}{}-5507{,}79\ln\lambda;\\
\tau\sim {G}(1,3]:\quad & n = 202{,}28+2293{,}47\lambda-999{,}13\ln\lambda;\\
\xi\sim F(1,3]:\quad & n = 151{,}96+2336{,}82\lambda-1054{,}33\ln\lambda,
\end{align*}

\noindent все коэффициенты детерминации которых примерно рав\-ны~0,99. Аналогично при
рас\-смот\-ре\-нии случайного возгорания модели были сле\-ду\-ющими:

\noindent
\begin{align*}
\tau\sim {G}(1,2):\quad & n = -1082{,}11+3896{,}2\lambda-{}\\
&\hspace*{25mm}{}-1956{,}43\ln\lambda;\\
\xi\sim F(1,2):\quad & n = -660{,}53+3545{,}96\lambda-{}\\
&\hspace*{25mm}{}-1531{,}34\ln\lambda;\\
\tau\sim {G}[2,3]:\quad & n = -5680{,}61+10162{,}21\lambda-{}\\
&\hspace*{25mm}{}-5458{,}21\ln\lambda;\\
\xi\sim F[2,3]:\quad & n = -5881{,}93+10467{,}04\lambda-{}\\
&\hspace*{25mm}{}-5649{,}63\ln\lambda;
\end{align*}

\noindent
\begin{align*}
\tau\sim {G}(1,3]:\quad & n = -1118{,}8+4536{,}45\lambda-{}\\
&\hspace*{25mm}{}-2417{,}32\ln\lambda;\\
\xi\sim F(1,3]:\quad & n = -351{,}88+3824{,}35\lambda-{}\\
&\hspace*{25mm}{}-1716{,}98\ln\lambda
\end{align*}

\vspace*{-6pt}

\noindent
со значениями коэффициентов детерминации не ниже~0,89.

По аналогии со случаем фиксированного значения~$p$~для каж\-дой пары значений
$([a,b],\lambda)$ было проведено попарное сравнение чис\-ла оставшихся после 
пожара вершин
на двух типах графов. Вследствие того, что для каждого интервала $[a,b]$ здесь 
имеется
только~16~пар $([a,b],\lambda)$, в~табл.~3 приведена не доля, а~чис\-ло случаев
принятия альтернативной гипотезы.



Таким образом, проведенное исследование процесса разрушения типа <<лес\-но\-го пожара>>
показало, что с~до\-ста\-точ\-но высокой до\-сто\-вер\-ной ве\-ро\-ят\-ностью исследование такого процесса
на графах в~случайной среде может быть заменено изучением более прос\-то\-го случая 
с~усредненным распределением степеней вершин. Понятно, что выводы этой статьи 
о~воз\-мож\-ности
упрощения методов изучения процессов разрушения графов в~случайной среде путем усреднения
распределений степеней вершин относятся только к~рас\-смот\-рен\-ным распределениям степеней 
и~видам разрушения. В~других случаях, особенно в~более общих и~ответственных, следует,
видимо, проводить дополнительные исследования.

\vspace*{-12pt}

{\small\frenchspacing
 {%\baselineskip=10.8pt
 \addcontentsline{toc}{section}{References}
 
 \vspace*{-2pt}
 
 \begin{thebibliography}{99}
\bibitem{Dur}
\Au{Durrett R.} Random graph dynamics.~--- Cambridge: Cambridge University
Press, 2007. 221~p.

\bibitem{Hof}
\Au{Hofstad R.} Random graphs and complex networks.~--- Cambridge:
Cambridge University Press, 2017. Vol.~1. 337~p.

\bibitem{Coh}
\Au{Cohen R., Erez~K., Ben-Avraham~D., Havlin~S.} Resilience of the Internet
to random breakdowns~// Phys. Rev. Lett., 2000. Vol.~85. P.~4626--4628.

\bibitem{Bol2}
\Au{Bollobas B., Riordan~O.} Robustness and vulnerability of scale-free
random graphs~// Internet Math., 2004. Vol.~1. No.\,1. P.~1--35.

\bibitem{Fa}
\Au{Faloutsos C., Faloutsos P., Faloutsos~M.} On power-law relationships of
the internet topology~// Comp. Comm.~R., 1999. Vol.~29. P.~251--262.

\bibitem{RN1}
\Au{Reittu H., Norros~I.} On the power-law random graph model of massive
data networks~// Perform. Evaluation, 2004. Vol.~55. P.~3--23.

\bibitem{Bol1}
\Au{Bollobas B.\,A.} A~probabilistic proof of an asymptotic formula
for the number of labelled regular graphs~// Eur. J.~Combust., 1980. Vol.~1. P.~311--316.

\bibitem{Ler1}
\Au{Leri M., Pavlov~Yu.} Power-law random graphs' robustness: Link saving
and forest fire model~// Austrian J.~Stat., 2014. Vol.~43. No.\,4. P.~229--236.

\bibitem{Ler3}
\Au{Leri M., Pavlov~Y.} Forest fire models on configuration random
graphs~// Fundam. Inform., 2016. Vol.~145. Iss.~3. P.~313--322.

\bibitem{Bia}
\Au{Biaconi G., Barabasi~A.-L.} Bose--Einstein condensation in complex
networks~// Phys. Rev. Lett., 2001. Vol.~86. Iss.~24. P.~5632--5635.

\bibitem{Pav}
\Au{Pavlov Y.} On random graphs in random environment~// Computer data
analysis and modeling: Theoretical and applied stochastics: 
11th Conference (International)
Proceedings.~--- Minsk, 2016. P.~177--180.

\bibitem{Afan}
\Au{Afanasyev V.~I., Boinghoff~C., Kersting~G., Vatutin~V.\,A.} Limit theorems
for weekly branching processes in random environment~// J.~ Theor. Probab.,
2012. Vol.~25. No.~3. P.~703--732.

\bibitem{Ler4}
\Au{Leri M., Pavlov~Y.} On robustness of configuration graphs in random
environment~//  4th Russian Finnish Symposium on Descrete
Mathematics Proceedings~/ Eds. J.~Karhum$\ddot{\mbox{a}}$ki,
A.~Saarela.~--- TUCS lecture notes ser.~--- Turku, Finland: 
Turku Centre for Computer Science, 2017. Vol.~26. P.~112--115.

\bibitem{Ler2}
\Au{Лери М.\,М.} Пожар на конфигурационном графе со случайными переходами
огня по реб\-рам~// Информатика и~её применения, 2015. Т.~9. Вып.~3. С.~67--73.

\bibitem{Dros}
\Au{Drossel B., Schwabl~F.} Self-organized critical forest-fire model~//
Phys. Rev. Lett., 1992. Vol.~69. P.~1629--1632.

\bibitem{Ber}
\Au{Bertoin J.} Fires on trees~// Ann. I.~H.~Poincare Pr., 2012. Vol.~48. No.\,4. P.~909--921.

\bibitem{Ari}
\Au{Arinaminparty N., Kapadia~S., May~R.} Size and complexity in model
financial systems~// P.~Natl. Acad. Sci. USA, 2012. Vol.~109. No.\,45. 
P.~18338--18343.
 \end{thebibliography}

 }
 }

\end{multicols}

\vspace*{-9pt}

\hfill{\small\textit{Поступила в~редакцию 20.10.17}}

\vspace*{6pt}

%\newpage

%\vspace*{-24pt}

\hrule

\vspace*{2pt}

\hrule

\vspace*{-4pt}


\def\tit{ON~THE~ROBUSTNESS OF~CONFIGURATION GRAPHS
IN~A~RANDOM ENVIRONMENT}

\def\titkol{On~the~robustness of~configuration graphs
in~a~random environment}

\def\aut{M.\,M.~Leri and~Yu.\,L.~Pavlov}

\def\autkol{M.\,M.~Leri and~Yu.\,L.~Pavlov}

\titel{\tit}{\aut}{\autkol}{\titkol}

\vspace*{-9pt}


\noindent 
Institute of Applied Mathematical Research of~the~Karelian Research Centre 
of~the~Russian Academy of Sciences, 11~Pushkinskaya Str.,
Petrozavodsk 185910, Russian Federation


\def\leftfootline{\small{\textbf{\thepage}
\hfill INFORMATIKA I EE PRIMENENIYA~--- INFORMATICS AND
APPLICATIONS\ \ \ 2018\ \ \ volume~12\ \ \ issue\ 2}
}%
 \def\rightfootline{\small{INFORMATIKA I EE PRIMENENIYA~---
INFORMATICS AND APPLICATIONS\ \ \ 2018\ \ \ volume~12\ \ \ issue\ 2
\hfill \textbf{\thepage}}}

\vspace*{3pt}




\Abste{The paper considers configuration graphs with vertex degrees being
independent identically distributed random variables following the power-law distribution
with a~random parameter. The parameter of the vertex degree distribution follows the
truncated gamma distribution. The authors study the robustness of such graphs to the two types
of destruction processes: random and targeted. The graphs function in a~random environment
where the values of the vertex degree distribution parameter are chosen separately for
each vertex. A~comparative analysis of destruction effects on these models and on graphs
with the degree distribution common for all vertices and induced by averaging over the
distribution parameter has been performed. The conditions under which the study of the graphs'
behavior in a~random environment can be reduced to the study of the evolution of graphs
with an averaged vertex degree distribution are discussed. A~comparative analysis of
destruction effects of the two types of destruction processes has been performed.}

\KWE{configuration graphs; power-law distribution; gamma
distribution; robustness; forest fire model; simulation}



\DOI{10.14357/19922264180201}

%\vspace*{-24pt}

\Ack
\noindent
The study was carried out under state order to the 
Karelian Research Centre of the Russian Academy of Sciences 
(Institute of Applied Mathematical Research KarRC RAS) 
and partly supported by the Russian Foundation for Basic Research,
grant No.\,16-01-0005a. The research was carried out using the equipment 
of the Core Facility of the Karelian Research Centre of the Russian 
Academy of Sciences.


\pagebreak


%\vspace*{-3pt}

  \begin{multicols}{2}

\renewcommand{\bibname}{\protect\rmfamily References}
%\renewcommand{\bibname}{\large\protect\rm References}

{\small\frenchspacing
 {%\baselineskip=10.8pt
 \addcontentsline{toc}{section}{References}
 \begin{thebibliography}{99}
 
 %\vspace*{3pt}
 
 
\bibitem{1-ler}
\Aue{Durrett, R.} 2007. \textit{Random graph dynamics.} 
Cambridge: Cambridge University Press. 221~p.

\bibitem{2-ler}
\Aue{Hofstad, R.} 2011. 
\textit{Random graphs and complex networks.} Cambridge:
Cambridge University Press.  Vol.~1. 363~p.

\bibitem{3-ler}
\Aue{Cohen, R., K.~Erez, D.~Ben-Avraham, and S.~Havlin.} 
2000. Resilience of the Internet
to random breakdowns. \textit{Phys. Rev. Lett.} 85:4626--4628.

\bibitem{4-ler}
\Aue{Bollobas, B., and O.~Riordan.} 2004. Robustness and vulnerability of scale-free
random graphs. \textit{Internet Math.} 1(1):1--35.

\bibitem{5-ler}
\Aue{Faloutsos, C., P.~Faloutsos, and M.~Faloutsos}. 1999.
 On power-law relationships of
the internet topology. \textit{Comp. Comm.~R.} 29:251--262.

\bibitem{6-ler}
\Aue{Reittu, H., and I.~Norros.} 2004. On the power-law random graph model of massive data
networks. \textit{Perform. Evaluation} 55:3--23.

\bibitem{7-ler}
\Aue{Bollobas, B.\,A.} 1980. A~probabilistic proof of an asymptotic formula for the number
of labelled regular graphs. \textit{Eur. J.~Combust.} 1:311--316.

\bibitem{8-ler}
\Aue{Leri, M., and Yu.~Pavlov.} 2014. 
Power-law random graphs' robustness: Link saving and
forest fire model. \textit{Austrian J.~Stat.} 43(4):229--236.

\bibitem{9-ler}
\Aue{Leri, M., and Yu.~Pavlov.} 2016. Forest fire models on configuration random graphs.
\textit{Fundam. Inform.} 145(3):313--322.

\bibitem{10-ler}
\Aue{Biaconi, G., and A.-L.~Barabasi.} 2001. Bose--Einstein condensation in complex networks.
\textit{Phys. Rev. Lett.} 86(24):5632--5635.

\bibitem{11-ler}
\Aue{Pavlov, Y.} 2016. On random graphs in random environment.
 \textit{11th Conference
(International) ``Computer Data Analysis and Modeling: Theoretical and Applied Stochastics''
Proceedings}. Minsk. 177--180.

\bibitem{12-ler}
\Aue{Afanasyev, V.\,I., C.~Boinghoff, G.~Kersting, and V.\,A.~Vatutin.}
2012. Limit theorems
for weekly branching processes in random environment. \textit{J.~Theor. 
Probab.} 25(3):703--732.

\bibitem{13-ler}
\Aue{Leri, M., and Y.~Pavlov.} 2017. On robustness of configuration graphs in random
environment. \textit{4th Russian Finnish Symposium on Descrete
Mathematics Proceedings}. Eds. J.~Karhum$\ddot{\mbox{a}}$ki
and A.~Saarela. TUCS lecture notes ser. Turku, Finland: 
Turku Centre for Computer Science. 26:112--115.

\bibitem{14-ler}
\Aue{Leri, M.\,M.} 2015. Pozhar na konfiguratsionnom grafe so sluchaynymi 
perekhodami ognya
po rebram [Forest fire on configuration graph with random fire propagation]. 
\textit{Informatika i~ee Primeneniya~--- Inform. Appl.} 9(3):67--73.

\bibitem{15-ler}
\Aue{Drossel, B., and F.~Schwabl.} 1992. Self-organized critical forest-fire model.
\textit{Phys. Rev. Lett.} 69:1629--1632.

\bibitem{16-ler}
\Aue{Bertoin, J.} 2012. Fires on trees. \textit{Ann. I.~H.~Poincare
Pr.} 48(4):909--921.

\bibitem{17-ler}
\Aue{Arinaminparty, N., S.~Kapadia, and R.~May.} 2012. Size and complexity in model
financial systems. \textit{P.~Natl. Acad. Sci. USA} 109:18338--18343.
\end{thebibliography}

 }
 }

\end{multicols}

\vspace*{-3pt}

\hfill{\small\textit{Received October 20, 2017}}

%\vspace*{-24pt}

\Contr

\noindent
\textbf{Leri Marina M.} (b.\ 1969)~--- 
Candidate of Science (PhD) in technology, scientist, Institute 
of Applied Mathematical Research of the Karelian Research Centre 
of the Russian Academy of Sciences, 
11~Pushkinskaya Str., Petrozavodsk 185910, Russian Federation; 
\mbox{leri@krc.karelia.ru}

\vspace*{3pt}



\noindent
\textbf{Pavlov Yuri L.} (b.\ 1949)~--- 
Doctor of Science in physics and mathematics, principal scientist,
Institute of Applied Mathematical Research of the Karelian Research Centre 
of the Russian Academy of Sciences, 11~Pushkinskaya Str., Petrozavodsk 185910, 
Russian Federation; \mbox{pavlov@krc.karelia.ru}


\label{end\stat}


\renewcommand{\bibname}{\protect\rm Литература} 