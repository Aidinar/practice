\def\stat{tirsin}

\def\tit{МОДЕЛИ УПРАВЛЕНИЯ РИСКОМ В~ГАУССОВСКИХ СТОХАСТИЧЕСКИХ 
СИСТЕМАХ$^*$}

\def\titkol{Модели управления риском в~гауссовских стохастических 
системах}

\def\aut{А.\,Н.~Тырсин$^1$, А.\,А.~Сурина$^2$}

\def\autkol{А.\,Н.~Тырсин, А.\,А.~Сурина}

\titel{\tit}{\aut}{\autkol}{\titkol}

\index{Тырсин А.\,Н.}
\index{Сурина А.\,А.}
\index{Tyrsin A.\,N.}
\index{Surina A.\,A.}




{\renewcommand{\thefootnote}{\fnsymbol{footnote}} \footnotetext[1]
{ Работа выполнена при финансовой поддержке РФФИ (проект 17-01-00315а).}}


\renewcommand{\thefootnote}{\arabic{footnote}}
\footnotetext[1]{Уральский федеральный университет имени первого Президента России Б.\,Н. Ельцина; Институт 
экономики Уральского отделения Российской академии наук, \mbox{at2001@yandex.ru}}
\footnotetext[2]{Южно-Уральский государственный университет (национальный 
исследовательский университет),  \mbox{dallila87@mail.ru}}

\vspace*{-2pt}

 
  
  \Abst{Описан новый подход к~исследованию риска многомерных стохастических 
сис\-тем. Он основан на гипотезе о~том, что рис\-ком можно управ\-лять за счет изменения 
вероятностных свойств компонент многомерной стохастической сис\-те\-мы, в~качестве 
которых используют факторы рис\-ка. Исследован случай гауссовских стохастических сис\-тем, 
опи\-сы\-ва\-емых случайными векторами, име\-ющи\-ми многомерное нормальное распределение. 
Как показало моделирование, не учтенные в~яв\-ном виде многомерность сис\-те\-мы и~взаимная 
коррелированность ее компонент могут привести к~существенному занижению фактического 
риска. Приведены результаты расчета ве\-ро\-ят\-ности опасного исхода в~зависимости от 
чис\-ло\-вых характеристик многомерной гауссовской случайной величины~--- ковариационной 
мат\-ри\-цы и~вектора математических ожиданий. Выполнена апро\-ба\-ция предложенной модели 
на примере анализа популяционного рис\-ка сер\-деч\-но-со\-су\-ди\-стых заболеваний. Описаны 
модели управ\-ле\-ния рис\-ком в~виде задач его минимизации или достижения заданного уров\-ня. 
Управляющими переменными являются чис\-ло\-вые характеристики случайного вектора~--- 
ковариационная мат\-ри\-ца и~век\-тор математических ожиданий. Проведена апро\-ба\-ция метода 
управ\-ле\-ния рис\-ком с~по\-мощью статистического моделирования методом Мон\-те Карло.}
  
  \KW{риск; модель; стохастическая сис\-те\-ма; случайный вектор; управ\-ле\-ние; нормальное 
рас\-пре\-де\-ление}

\DOI{10.14357/19922264180208}
  
%\vspace*{-6pt}


\vskip 10pt plus 9pt minus 6pt

\thispagestyle{headings}

\begin{multicols}{2}

\label{st\stat}
  
\section{Введение}

\vspace*{-4pt}

  Уже не вызывает сомнений наличие общемировой тенденции быст\-ро\-го рос\-та 
ущерба от природных катаклизмов, техногенных катастроф, террористических 
актов и~экономических потрясений. Многие авторы отмечают, что тем\-пы рос\-та 
ущерба значительно превосходят темпы рос\-та экономики~[1--3]. Это можно 
объяснить по\-сто\-ян\-ным возрастанием рис\-ка в~условиях на\-уч\-но-тех\-ни\-че\-ской 
революции и~форсированного развития техносферы~[4]. Очевидно, что для 
снижения ущер\-ба от природных катаклизмов, техногенных катастроф, 
террористических актов и~экономических потрясений необходимо повысить 
без\-опас\-ность функционирования со\-от\-вет\-ст\-ву\-ющих сис\-тем, а~значит, снизить 
риск. Для этого необходимы адекватные модели и~эффективные методы 
управ\-ле\-ния риском сис\-тем.
  
  Реальные системы, как правило, являются многомерными, их 
функционирование во многом носит стохастический характер, у~них час\-то 
мож\-но выделить десятки различных факторов риска~\cite{1-t}. При решении 
задачи управления рис\-ком необходимо опираться на модель рис\-ка. 

Обычно 
моделирование рис\-ка сводится к~выделению опасных исходов, 
количественному заданию по\-след\-ст\-вий от их наступления и~оцениванию 
вероятностей этих исходов~\cite[с.~37--43]{5-t}. При этом вклад компонент 
многомерной сис\-те\-мы объединяют и~рас\-смат\-ри\-ва\-ют уже одномерную сис\-те\-му 
как случайную величину~[5, с.~148--156; 6, с.~82--87]. 

Но вопрос взаимного 
влияния опас\-ных ситуаций, вызванных разными элементами многомерной 
сис\-те\-мы, мало исследован, чаще всего им пренебрегают или существенно 
упрощают, считая разные опас\-ные исходы взаимно независимыми, 
и~пренебрегают ве\-ро\-ят\-ностью их одновременного наступления. 

Для 
относительно прос\-тых объектов, когда можно априори указать все опасные 
исходы, при наличии статистической информации или экспертных оценок 
о~шан\-сах их по\-яв\-ле\-ния в~целом данный подход дает приемлемые на практике 
результаты. 
%
Обычно здесь удается накопить достаточную статистику для 
оценивания вероятностей на\-ступ\-ле\-ния опас\-ных исходов, а~форма взаимосвязи 
между элементами сис\-те\-мы является до\-ста\-точ\-но прос\-той и~может быть 
описана, например, с~по\-мощью ло-\linebreak\vspace*{-12pt}

\pagebreak

\noindent
ги\-ко-ве\-ро\-ят\-ност\-ных моделей 
риска~\cite{7-t} в~рам\-ках тео\-рии струк\-тур\-но-слож\-ных сис\-тем~\cite{8-t}.
  
  Однако у сложных систем структуру взаимодействия между элементами 
обычно не удается описать с~по\-мощью ло\-ги\-ко-ве\-ро\-ят\-ност\-ных моделей~--- 
стохастические связи между элементами не позволяют их адекватно 
моделировать с~по\-мощью алгебры логики (AND, OR, NOT), а~изменения 
со\-сто\-яния\linebreak элементов и~самой сис\-те\-мы носят непрерывный харак\-тер. Понятия 
опас\-ных исходов также могут размываться, делая невозможным их конкретное 
выделение. К~таким сис\-те\-мам мож\-но отнес\-ти,\linebreak например, 
со\-ци\-аль\-но-эко\-но\-ми\-че\-ские сис\-те\-мы, 
вклю\-чая территориальные и~региональные сис\-те\-мы, 
живые сис\-те\-мы, например человека с~точ\-ки зрения со\-сто\-яния здоровья.
  
  Таким образом, несмотря на большое число исследований, взаимному 
влиянию элементов и~различных фак\-то\-ров рис\-ка на без\-опас\-ность слож\-ных 
многомерных сис\-тем уделяется недостаточно внимания. Во многих случаях, 
когда нет воз\-мож\-ности явно связать разные факторы рис\-ка в~виде  
ло\-ги\-ко-ве\-ро\-ят\-ност\-ной модели, их корреляция при расчете рис\-ка не 
учитывается, поэтому проб\-ле\-ма\-ти\-ка исследований в~об\-ласти анализа рис\-ка, 
особенно в~час\-ти со\-зда\-ния эффективных моделей описания и~управ\-ле\-ния 
рис\-ком слож\-ных многомерных сис\-тем, в~на\-сто\-ящее время становится одной из 
актуальных.
  
  В~\cite{9-t, 10-t} предложен подход к~моделированию риска, со\-глас\-но 
которому стохастическую сис\-те\-му представляют в~виде случайного вектора со 
взаимно коррелированными компонентами, а~в~качестве управ\-ля\-ющих 
переменных используют его чис\-ло\-вые характеристики. Целью \mbox{статьи} является 
описание моделей управ\-ле\-ния рис\-ком на основе данного подхода.

\section{Модель риска в~гауссовских стохастических системах}

  Пусть $S$~--- некоторая многомерная стохастическая сис\-те\-ма. Выделим 
в~этой сис\-те\-ме фак\-то\-ры рис\-ка $X_1, X_2, \ldots, X_m$. В~результате получим 
пред\-став\-ле\-ние сис\-те\-мы в~виде случайного вектора $\mathbf{X}\hm= (X_1, X_2, 
\ldots, X_m)$ с~некоторой плот\-ностью 
ве\-ро\-ят\-ности~$p_{\mathbf{X}}(\mathbf{x})$.
  
  Вместо общепринятого выделения конкретных опасных ситуаций будем 
задавать гео\-мет\-ри\-че\-ские области неблагоприятных исходов. Они могут 
выглядеть произвольным образом в~за\-ви\-си\-мости от конкретной задачи 
и~определяются на основе име\-ющей\-ся априорной информации. Для 
опре\-де\-лен\-ности опишем пред\-ла\-га\-емый под\-ход на примере распространенной 
концепции нежелательных событий как больших и~маловероятных отклонений 
случайной величины относительно ее математического ожидания. Тогда 
опасными ситуациями будем считать большие и~маловероятные отклонения 
вы\-бо\-роч\-ных значений~$x_{ij}$ любой из компонент~$X_j$ относительно 
математических ожиданий $\mu_j\hm=Х{\sf M}[X_j]$, $j\hm=1, 2,\ldots ,m$. 
Вероятность неблагоприятного исхода для каж\-дой из компонент~$X_j$ 
зададим как
 \begin{multline*}
  {\sf P}\left(D_j\right)={\sf P}\left(X_j\in D_j\right)=
  {\sf P}\left( X_j\notin \overline{D}_j\right)\,,\\
  \overline{D}_j=\left\{ x:\ \mu_j-A_{1j}\sigma_j<x<\mu_j+A_{2j}\sigma_j\right\}\,,
\end{multline*}
где $\sigma_j$~--- среднее квад\-ра\-ти\-че\-ское отклонение случайной 
величины~$X_j$; $A_{1j}$ и~$A_{2j}$~--- заданные ниж\-ний и~верх\-ний 
пороговые уров\-ни (в~единицах~$\sigma_j$), т.\,е.\ об\-ласть благоприятных 
исходов ограничена диапазоном $(\mu_j\hm-A_{1j}\sigma_j; 
\mu_j+A_{2j}\sigma_j)$.

  Теперь необходимо задать многомерную область опас\-ных ситуаций~$D$, 
учтя взаимное вли\-яние компонент на по\-яв\-ле\-ние неблагоприятных исходов. Она 
равна $D\hm= \mathbf{R}^m\backslash \overline{D}$, где $\overline{D}$~--- 
об\-ласть допустимых значений фак\-то\-ров рис\-ка. Опишем 
об\-ласть~$\overline{D}$. Это можно сделать различными способами. Наиболее 
оправданным с~гео\-мет\-ри\-че\-ской точки зрения пред\-став\-ля\-ет\-ся задать ее в~виде 
внут\-рен\-ней об\-ласти $m$-ос\-но\-го эллипсоида
  $$
  \overline{D}= \left\{ \mathbf{x}=\left( x_1, x_2, \ldots , x_m\right): 
\sum\limits_{j=1}^m \fr{(x_j-\mu_j^\prime)^2}{A^2_j \sigma_j^2}<1\right\}
  $$
с~центром в~точке $\boldsymbol{\mu}^\prime \hm= (\mu_1^\prime, \mu_2^\prime, 
\ldots , \mu_m^\prime)$, $\mu_j^\prime\hm= \mu_j\hm+A_j\sigma_j$, 
$A_j\hm=(A_{1j}\hm+ A_{2j})/2$, $j\hm=1, 2,\ldots, m$. Тогда для случайного 
вектора~$\mathbf{X}$ ве\-ро\-ят\-ность неблагоприятного исхода будет равна
\begin{multline}
{\sf P}(D) ={\sf P}(\mathbf{X}\in D)\,,\quad
D={}\\
\!\!{}=\left\{ \mathbf{x}=\left( x_1, x_2, \ldots ,x_m\right): \sum\limits^m_{j=1} 
\fr{(x_j-\mu_j)^2}{A_j^2\sigma_j^2}\geq 1\right\}.\!\!
\label{e1-t}
\end{multline}

\begin{figure*}[b] %fig1
     \vspace*{1pt}
 \begin{center}
 \mbox{%
 \epsfxsize=162.957mm 
 \epsfbox{tyr-1.eps}
 }
 \end{center}
\vspace*{-9pt}
\Caption{Реализации стандартного нормального случайного вектора: 
(\textit{а})~$\rho\hm= 0$; (\textit{б})~$\rho\hm = 0{,}9$}
\end{figure*}
  
  Заметим, что в~(\ref{e1-t}) об\-ласть~$D$ неблагоприятных исходов 
пред\-став\-ля\-ет собой внешнюю об\-ласть \mbox{$m$-ос}\-но\-го эл\-лип\-со\-ида, у~которого 
полуоси по каж\-дой из координат равны~$A_j\sigma_j$ соответственно, т.\,е.\ по 
каж\-дой $j$-й оси эта об\-ласть соответствует одномерному случаю~$D_j$. 
Очевидно, когда исход не лежит на одной из осей, событие~$D$ может 
реализоваться и~при отсутствии рис\-ко\-вых отклонений по всем компонентам 
(воз\-мож\-ны ситуации $\mathbf{X}\hm\in D$ и~$\forall j\ X_j\notin D_j$).
  
  Задав функцию по\-след\-ст\-вий от опасных си\-ту\-аций в~виде $g(\mathbf{x})$, 
получим модель для количественной оцен\-ки риска:
  \begin{equation*}
  r(\mathbf{X})=\idotsint\limits_{\mathbf{R}^m} g(\mathbf{x}) 
p_{\mathbf{X}}(\mathbf{x})\,d\mathbf{x}\,.
 % \label{e2-t}
  \end{equation*}
  Если принять
  \begin{equation}
  g(\mathbf{x})=\begin{cases}
  1\,, &\ \mathbf{x}\in D\,;\\
  0\,, &\ \mathbf{x}\notin D\,,
  \end{cases}
  \label{e3-t}
  \end{equation}
то $r(\mathbf{X})={\sf P}(\mathbf{X}\in D)$, т.\,е.\ риск оцениваем как ве\-ро\-ят\-ность 
неблагоприятного исхода. Если на ранней стадии исследования сис\-те\-мы 
слож\-но до\-ста\-точ\-но точ\-но описать функцию $g(\mathbf{x})$, то 
формула~(\ref{e3-t}) становится оценкой~${\sf P}(D)$ и~является удоб\-ным 
начальным приб\-ли\-же\-ни\-ем модели риска.

  Рассмотрим далее наиболее рас\-про\-стра\-нен\-ный част\-ный случай, 
когда~$\mathbf{X}$ имеет совместное нормальное рас\-пре\-де\-ле\-ние с~плот\-ностью 
ве\-ро\-ят\-ности
\begin{multline*}
  p_{\mathbf{X}}(\mathbf{x})={}\\
  {}=\fr{1}{\sqrt{(2\pi)^m\vert\boldsymbol{\Sigma}\vert}}\,\exp 
\left\{ -\fr{1}{2}\left( \mathbf{x}-\mathbf{a}\right)^{\mathrm{T}} 
\boldsymbol{\Sigma}^{-1}(\mathbf{x}-\mathbf{a})\right\}\,,
  \end{multline*}
где $\mathbf{a}=(a_1, a_2, \ldots, a_m)^{\mathrm{T}}$~--- век\-тор 
математических ожиданий; $\boldsymbol{\Sigma}\hm= \{ \sigma_{ij}\}_{m\times 
m}$~--- ковариационная мат\-рица.
  
  Использование гауссовского случайного век\-то\-ра опирается на цент\-раль\-ную 
предельную тео\-ре\-му~\cite{11-t}. Как показала апробация на ряде примеров, 
такая идеализация не столь критична, и~если есть ка\-кие-ли\-бо основания 
считать, что плот\-но\-сти вероятностей компонент вектора~$\mathbf{X}$ име\-ют 
более вытянутые хвос\-ты, то это можно скорректировать за счет 
со\-от\-вет\-ст\-ву\-юще\-го задания функции~$g(\mathbf{x})$.
  
  Исследуем влияние многомерности и~коррелированности факторов риска на 
ве\-ро\-ят\-ность по\-яв\-ле\-ния неблагоприятных исходов.
  
  \smallskip
  
  \noindent
  \textbf{Пример~1.}\ Для наглядности рас\-смот\-рим двумерный гауссовский 
случайный вектор ($X_1, X_2$) с~плот\-ностью ве\-ро\-ят\-ности
  \begin{equation}
   p_{X_1, X_2}(x_1, x_2) =\fr{e^{-Q(x_1-a_1, x_2-a_2)/2}}{2\pi 
\sigma_1\sigma_2 \sqrt{1-\rho^2}}\,.
 \label{e4-t}
\end{equation}
Здесь
$$
  Q\left(y_1,y_2\right)=\fr{1}{1-\rho^2}\left( \fr{y_1^2}{\sigma_1^2} -\fr{2\rho 
y_1y_2}{\sigma_1\sigma_2}+\fr{y_2^2}{\sigma_2^2}\right)\,,
$$
где $y_i=x_i-a_i$, $i=1, 2$; $\rho\hm= \sigma_{12}/(\sigma_1\sigma_2)$~--- 
коэффициент корреляции между~$X_1$ и~$X_2$.
  
  На рис.~1 показаны примеры реализаций стандартного нормального 
случайного вектора ($X_1, X_2$) для некоррелированных ($\rho\hm=0$) 
и~коррелированных ($\rho\hm = 0{,}9$) компонент. Видим, что увеличение 
тес\-но\-ты корреляционной связи между компонентами приводит к~вытягиванию 
диа\-грам\-мы рас\-се\-яния и~увеличению ве\-ро\-ят\-ности по\-яв\-ле\-ния больших 
укло\-не\-ний случайного век\-тора.
  
  
 % \smallskip
 
  { \begin{center}  %fig2
 \vspace*{-2pt}
  \mbox{%
 \epsfxsize=78.141mm 
 \epsfbox{tyr-2.eps}
 }


\end{center}


\noindent
{{\figurename~2}\ \ \small{Зависимости $\lg P(D)$ от порогового уров\-ня~$A$: 
(\textit{а})~$D_e(\mathbf{X})\hm = 0$; (\textit{б})~$D_e(\mathbf{X}) \hm= 0{,}5$; 
(\textit{в})~$D_e(\mathbf{X}) \hm=1$; \textit{1}~--- $m \hm= 1$; 
\textit{2}~---2; 
\textit{3}~--- 3; \textit{4}~--- 4; \textit{5}~--- $m\hm= 5$}}
}

\vspace*{18pt}

\setcounter{figure}{2}

  
  \noindent
  \textbf{Пример~2.}\ Зададим для определенности раз\-мер\-ность 
вектора~$\mathbf{X}$ от~1 до~5. В~[12] введен коэффициент тес\-но\-ты 
со\-вмест\-ной линейной корреляционной связи компонент случайного 
век\-то\-ра~$\mathbf{X}$, равный $D_e(\mathbf{X})\hm=1\hm- \vert 
\mathbf{R}_{\mathbf{X}}\vert^{1/m}$, где $\mathbf{R}_{\mathbf{X}}$~--- 
корреляционная мат\-ри\-ца случайного век\-то\-ра~$\mathbf{X}$. Очевидно, что 
$0\hm\leq D_e(\mathbf{X})\hm\leq 1$. Случай $D_e(\mathbf{X})\hm=0$ 
соответствует не\-за\-ви\-си\-мости компонент~$X_1, X_2, \ldots , X_m$, а~при 
$D_e(\mathbf{X})\hm=1$ имеем строгую линейную за\-ви\-си\-мость компонент.
  
  Рассмотрим три случая: $D_e(\mathbf{X}) \hm= 0$, $D_e(\mathbf{X}) \hm= 
0{,}5$ и~$D_e(\mathbf{X}) \hm= 1$. Результаты рас\-че\-та ве\-ро\-ят\-ности 
неблагоприятного исхода~(\ref{e1-t}) приведены на рис.~2. Для большей на\-гляд\-ности 
примем  $A_1\hm=A_2= \cdots =A_m\hm=A$.
  
 
  
  Анализ графиков на рис.~2 говорит о~сле\-ду\-ющем. Увеличение 
раз\-мер\-ности~$m$ и~тес\-но\-ты кор\-ре\-ля\-ционной связи меж\-ду компонентами 
случайного вектора~$\mathbf{X}$ приводит к~резкому рос\-ту ве\-ро\-ят\-ности 
не\-бла\-го\-при\-ят\-но\-го исхода.
  
  Особенно важным оказалось то, что даже относительно малая тес\-но\-та 
корреляционной связи ($D_e(\mathbf{X}) \hm= 0{,}5$), которая почти всегда 
наблюдается на прак\-ти\-ке, уже приводит к~значительному рос\-ту 
ве\-ро\-ят\-ности~${\sf P}(D)$. Эффект усиливается с~увеличением значений~$A_j$, что 
соответствует менее вероятным, но более опас\-ным неблагоприятным исходам. 
Например, при $A\hm = 6$ ве\-ро\-ят\-ность неблагоприятного исхода более чем 
в~7000~раз выше у~коррелированной сис\-те\-мы ($D_e(\mathbf{X}) \hm= 1$) по 
срав\-не\-нию с~некоррелированной ($D_e(\mathbf{X}) \hm= 0$). Поэтому при 
моделировании рис\-ка необходимо учитывать как фактор мно\-го\-мер\-ности, так 
и~тес\-но\-ту корреляционных связей.

  

\section{Апробация модели риска на~примере анализа 
популяционного риска сердечно-сосудистых заболеваний}

  Одной из малоизученных проблем в~медицине является комплексная оценка 
популяционного здоровья одновременно по нескольким факторам риска в~их 
взаимосвязи. Это объясняется тем, что не\-яс\-но, как учитывать вклад каждого 
фак\-то\-ра рис\-ка в~общую оценку со\-сто\-яния здоровья. Обычно в~таких случаях 
используются экспертные оценки, которые нельзя считать в~полной мере 
объективными~[13, 14].
  
  Исследуем динамику изменения с~возрастом популяционного риска 
сер\-деч\-но-со\-су\-ди\-стых за\-бо\-леваний по основным биологическим факторам риска, 
к~которым относят артериальную ги\-пер\-тен\-зию, дис\-ли\-пи\-де\-мию, повышенный 
уро\-вень глюкозы в~крови и~избыточную массу тела~[15]. 

В~качестве 
биологических па\-ра\-мет\-ров, ха\-рак\-те\-ри\-зу\-ющих эти факторы рис\-ка, используют 
уровень общего холестерина (ОХС), сис\-то\-ли\-че\-ское артериальное дав\-ле\-ние 
(САД), индекс массы тела (ИМТ), уровень глюкозы (УГ). 

Статистический 
материал получен в~результате комплексного сплош\-но\-го углуб\-лен\-но\-го  
кли\-ни\-ко-эпи\-де\-мио\-ло\-ги\-че\-ско\-го обследования муж\-ской сельской 
популяции с~гнез\-до\-вой выборкой. Всего было обследовано~1402~мужчины 
одного из сел Челябинской об\-ласти, что со\-ста\-ви\-ло~93\% от списочного со\-ста\-ва 
села. Для всех пациентов был проведен необходимый комплекс клинических, 
лабораторных и~инструментальных методов обследования для 
квалифицированного заключения о~со\-сто\-янии здоровья. Работу проводила 
бригада специалистов, со\-сто\-ящая из со\-труд\-ни\-ков ка\-фед\-ры госпитальной 
терапии и~семейной медицины Челябинской государственной медицинской 
академии и~врачей Челябинской об\-ласт\-ной клинической больницы №\,1~[16].
  
  Пороговые значения биологических па\-ра\-мет\-ров, характеризующих основ\-ные 
биологические фак\-то\-ры рис\-ка, при превышении которых риск 
сер\-деч\-но-со\-су\-ди\-стых 
ослож\-не\-ний резко воз\-рас\-та\-ет (ниже этого значения~--- норма), в~соответствии 
с~[15] рав\-ны: САД~--- 140~мм\ рт.\ ст.; ИМТ~--- 25~кг/м$^2$; ОХС~--- 
5~ммоль/л; УГ~--- 5,5~ммоль/л. 
Исследование проводилось сле\-ду\-ющим 
образом. 

Было сформировано четыре группы по воз-\linebreak рас\-там: 18--24~года,  
25--34~года, 35--44~года и~45--54~года. Проверка по критерию со\-гла\-сия 
\mbox{$\chi^2$-Пир}\-со\-на статистической гипотезы о~соответствии\linebreak каж\-дой группы 
наблюдений для всех фак\-то\-ров рис\-ка нормальному распределению на уровне 
зна\-чи\-мости~0,05 не была отклонена. Поэтому считаем, что имеем гауссовскую 
стохастическую сис\-те\-му раз\-мер\-ности $m\hm = 4$.
  
  Затем для каждой группы были определены средние значения 
и~ковариационные мат\-ри\-цы. Вы\-чис\-ле\-ние вероятности ${\sf P}(D)$ можно 
выполнять двумя способами~--- с~по\-мощью чис\-лен\-но\-го интегрирования для 
малых размерностей ($m\hm\leq 4$) или методом статистических испытаний  
Мон\-те Кар\-ло~[17] при раз\-мер\-ности $m\hm>4$. Результаты расчета 
приведены в~таб\-лице.



  
  Видим, что наблюдается тенденция рос\-та риска возникновения 
сердечно-сосудистых ослож\-не\-ний.\linebreak\vspace*{-12pt}

\vspace*{6pt}

%\begin{table*}
{\small
  \begin{center}
  \begin{tabular}{|c|c|}
\multicolumn{2}{p{43mm}}{Значения вероятностей риска возникновения 
сер\-деч\-но-со\-су\-ди\-стых ослож\-не\-ний}\\[-6pt]
\multicolumn{2}{c}{\ }\\
\hline
Возраст, лет&Вероятность\\
\hline
18--24&0,70\\
25--34&0,78\\
35--44&0,95\\
45--54&0,98\\
\hline
\end{tabular}
\vspace*{2pt}
\end{center}
}
%\end{table*}

\columnbreak
  
  
  \noindent
   Полученные в~целом высокие значения 
вероятностей ${\sf P}(D)$ соответствуют фактическому со\-сто\-янию
   здоровья. 
   
   Как 
показали результаты комплексного сплош\-но\-го углуб\-лен\-но\-го  
кли\-ни\-ко-эпи\-де\-мио\-ло\-ги\-че\-ско\-го обследования, в~обследованной 
популяции здоровых лиц в~воз\-рас\-те старше~34~лет практически не оказалось.

 
\section{Модели управления риском}

  Введенная модель риска позволяет на практике осуществлять управ\-ле\-ние 
сто\-ха\-сти\-че\-ской сис\-те\-мой с~целью его снижения.
  
  \bigskip
  
  \noindent
  \textbf{Пример~3.}\ Проиллюстрируем данный подход на прос\-тей\-шем 
примере гауссовской стохастической сис\-те\-мы с~раз\-мер\-ностью $m\hm=2$. На 
рис.~3 показана воз\-мож\-ность уменьшения ве\-ро\-ят\-ности неблагоприятного 
исхода~${\sf P}(D)$, а~значит, и~рис\-ка за счет варьирования па\-ра\-мет\-ров плот\-ности 
$p_{\mathbf{X}}(\mathbf{x})$. Об\-ласть неблагоприятного исхода~$D$ 
рас\-по\-ло\-же\-на выше линии в~правом верх\-нем углу.
  
  Видим, что возможные варианты изменения па\-ра\-мет\-ров случайного вектора 
($X_1, X_2$): уменьшение ковариации (или коэффициента корреляции), 
изменение математических ожиданий случайных величин, уменьшение 
дис\-пер\-сий~$\sigma_1^2$ или~$\sigma_1^2$~--- могут привести к~снижению 
ве\-ро\-ят\-ности~${\sf P}(D)$.
  

  
  Суть управ\-ле\-ния риском гауссовской стохастической сис\-те\-мы со\-сто\-ит 
в~сле\-ду\-ющем. Задав функцию по\-след\-ст\-вий от опасных ситуаций 
$g(\mathbf{x})$ и~введя ограничения на допустимые значения элементов 
ковариационной мат\-ри\-цы $G(\boldsymbol{\Sigma})$ и~сред\-них значений 
компонент сис\-те\-мы $H(\mathbf{a})$, сформулируем задачу минимизации рис\-ка 
с~переменными~$\boldsymbol{\Sigma}$ и~$\mathbf{a}$:
  \begin{multline}
    r(\boldsymbol{\Sigma}, \mathbf{a})=\displaystyle \idotsint\limits_{\mathbf{R}^m} 
g(\mathbf{x}) p_{\mathbf{X}}(\mathbf{x})\,d\mathbf{x} \to 
\min\limits_{\boldsymbol{\Sigma}, \mathbf{a}}\,,\\
\boldsymbol{\Sigma} \in G(\boldsymbol{\Sigma})\,,\enskip \mathbf{a}\in 
H(\mathbf{a})\,.
    \label{e5-t}
  \end{multline}
  
  Задача~(\ref{e5-t}) является задачей нелинейного про\-грам\-ми\-ро\-ва\-ния. Ее 
мож\-но решить разными методами. Одним из них является метод барьерных 
функ\-ций (внут\-рен\-них штраф\-ных функ\-ций)~[18]. Его основная идея со\-сто\-ит 
в~приведении задачи поиска услов\-но\-го экстремума к~по\-сле\-до\-ва\-тель\-ности задач 
на\-хож\-де\-ния без\-услов\-но\-го экстремума вспомогательной функции:
  $$
  F(\mathbf{X}, b_k) =r(\boldsymbol{\Sigma}, \mathbf{a}) 
+{\sf P}\left(\boldsymbol{\Sigma}, \mathbf{a}, b_k\right)\,,
  $$
где ${\sf P}(\boldsymbol{\Sigma}, \mathbf{a}, b_k)$~--- штраф\-ная функ\-ция; $b_k$~--- 
па\-ра\-метр штрафа.

\pagebreak
  
  
 % \smallskip
 \end{multicols}
 
  \begin{figure*} %fig3
  \vspace*{1pt}
 \begin{center}
 \mbox{%
 \epsfxsize=164.954mm 
 \epsfbox{tyr-3.eps}
 }
 \end{center}
\vspace*{-4pt}
\Caption{Снижение риска: (\textit{а})~исходное со\-сто\-яние; (\textit{б})~за счет уменьшения 
корреляции; (\textit{в}) и~(\textit{г})~за счет изменения 
математических ожиданий случайных величин~$X_1$ или~$X_2$
соответственно; 
(\textit{д}) и~(\textit{е})~за счет уменьшения дис\-пер\-сии~$\sigma_1^2$ 
или~$\sigma_2^2$ соответственно}
\vspace*{12pt}
\end{figure*}
  
  
  \begin{multicols}{2}
  
  \noindent
  \textbf{Пример~4.}\ Рас\-смот\-рим двумерный гауссовский случайный век\-тор 
с~плот\-ностью ве\-ро\-ят\-ности~(\ref{e4-t}). Задача минимизации будет выглядеть 
как
\begin{multline*}
  r(\boldsymbol{\Sigma}, \mathbf{a})={}\\
  {}=\iint\limits_{\mathbf{R}^2} 
   \fr{g(x_1, x_2)}{2\pi \sigma_1\sigma_2 \sqrt{1-\rho^2}} %\times{}\\
%{}\times 
e^{-Q(x_1-a_1, x_2-
a_2)/2} \,dx_1 dx_2 \to {}\\
{}\to \min\limits_{\boldsymbol{\Sigma}, \mathbf{a}}
\end{multline*}
с ограничениями
\begin{equation*}
\begin{array}{l}
\sigma_1^2\sigma_2^2> \sigma_{12}^2\,;\\[6pt]
a_i^- <a_i< a_i^+\,,\enskip i=1,2\,;\\[6pt]
\sigma_{ij}^-<\sigma_{ij}<\sigma_{ij}^+\,,\enskip j=1,2\,.
\end{array}
\end{equation*}
  
  Зададим конкретные значения: $a_1^-\hm=a_2^- \hm=-3$; $a_1^+\hm= a_2^+ 
\hm=3$; $\sigma_{12}^- \hm= 0{,}1$; $\sigma_{12}^+\hm=3$. Для 
опре\-де\-лен\-ности считаем, что $\forall\ \mathbf{x}\hm\notin D$ 
$g(\mathbf{x})\hm=0$. Получаем задачу

\noindent
  \begin{multline*}
  r(\boldsymbol{\Sigma}, \mathbf{a})= \iint\limits_{\mathbf{R}^2} 
\fr{  g(x_1, x_2)}{1{,}2\pi}\times{}\\
\!{}\times e^{-\left((x_1-2)^2-1{,}6x_1 x_2+(x_2-2)^2\right)/\left(2\cdot 0{,}6^2\right)}\, 
dx_1 dx_2\to \min\limits_{\boldsymbol{\Sigma}, \mathbf{a}}\hspace*{-5.74925pt}
\end{multline*}
с ограничениями
\begin{equation*}
\begin{array}{l}
\sigma_1^2\sigma_2^2>\sigma_{12}^2\,;\\[6pt]
-3<a_1, a_2<3\,;\\[6pt]
0{,}1<\sigma_1, \sigma_2<3\,;\\[6pt]
0{,}1< \sigma_2<3\,.
\end{array}
\end{equation*}
  
    
    Выберем в~качестве штрафной функ\-ции обратную: 
    $$
    {\sf P}\left(\boldsymbol{\Sigma},\mathbf{a}, b^k\right) =-b^k \sum\limits^m_{j=1} 
\fr{1}{t_j(\Sigma, a)}\,.
$$
    Тогда с~учетом всех ограничений вспомогательная функ\-ция примет вид:
    
    \noindent
    \begin{multline*}
    F\left( \mathbf{X}, b_k\right) ={}\\
    {}=\iint\limits_{\mathbf{R}^2} \!\fr{1}{1{,}2\pi} 
\,e^{- \left((x_1-2)^2 -1{,}6x_2x_2+(x_2-2)^2\right)/\left(2\cdot 0{,}6^2\right)} \,d\mathbf{x} -{}\\
{}-
b^k\left( \fr{1}{\sigma_1} +\fr{1}{\sigma_2}+\fr{1}{\sigma_1^2 \sigma_2^2-
\sigma_{12}^2} +\fr{1}{a_1+3}+ {}\right.\\
{}+\fr{1}{3-a_1}+ \fr{1}{a_2+3} +\fr{1}{3-a_1} 
+\fr{1}{\sigma_1-0{,}1} +{}\\
\left.{}+\fr{1}{3-\sigma_1} +\fr{1}{\sigma_2 -0{,}1} +\fr{1}{3-
\sigma_2}\right) \to \min\limits_{\boldsymbol{\Sigma}, \mathbf{a}}\,.
    \end{multline*}
  
  Поиск минимума вспомогательной функ\-ции находим с~по\-мощью 
покоординатного спус\-ка. Начальные значения па\-ра\-мет\-ров:
  $$
  \boldsymbol{\Sigma}^0=\begin{pmatrix}
  1 & 0{,}8\\
  0{,}8 & 1
  \end{pmatrix}\,;\enskip \mathrm{a}^0=\begin{pmatrix}
  2\\ 2\end{pmatrix}\,;\enskip b^0=10\,.
  $$
  
  Задача минимизации была решена при значениях па\-ра\-мет\-ров: 
$a_1\hm=2{,}6$; $a_2\hm=0{,}6$; $\sigma_1\hm=2{,}9$; $\sigma_2\hm=2{,}9$; 
$\sigma_{12}\hm= 1{,}39\cdot 10^{-16}$. При этом минимум целевой функ\-ции 
с~точ\-ностью до~0,001:  $r(\boldsymbol{\Sigma}^*, \mathbf{a}^*)\hm= 
0{,}041$.
  
  Задача~(\ref{e5-t}) полезна лишь в~качестве первого приб\-ли\-же\-ния модели 
управ\-ле\-ния рис\-ком, так как не учитывает ограничений, связанных с~за\-тра\-та\-ми 
на изменения варь\-и\-ру\-емых па\-ра\-мет\-ров относительно своих начальных 
значений.
  
  Если ввести ограничения на за\-тра\-ты, связанные с~изменением 
переменных~$\boldsymbol{\Sigma}$ и~$\mathbf{a}$, то получим задачу:
  \begin{equation}
  \left.
  \begin{array}{c}
 \displaystyle r(\boldsymbol{\Sigma}, \mathbf{a}) =\idotsint\limits_{\mathbf{R}^m} 
g(\mathbf{x}) p_{\mathbf{X}}(\mathbf{x})\,d\mathbf{x}\to 
\min\limits_{\boldsymbol{\Sigma}, \mathbf{a}}\,,\\[6pt]
  %\hspace*{27mm}
  \boldsymbol{\Sigma}\in G(\boldsymbol{\Sigma})\,,\enskip \mathbf{a}\in 
H(\mathbf{a})\,,\\[6pt]
  a_i=a_i^0+\delta_i\,,\enskip v_i(\delta_i)\leq V_i\,,\enskip i=1,\ldots ,m\,,\\[6pt]
  \sigma_{ij} =\sigma_{ij}^0 +\Delta_{ij}\,,\enskip w_{ij}(\Delta_{ij})\leq 
W_{ij}\,,\\[6pt]
\hspace*{40mm} i, j=1,\ldots ,m\,,
  \end{array}
  \right\}
  \label{e6-t}
  \end{equation}
где $v_i(\delta_i)$~--- функ\-ция за\-трат на изменение среднего значения $i$-й 
компоненты, име\-ющей начальное значение~$a_i^0$; $V_i$~--- предельная 
величина за\-трат на это изменение; $w_{ij}(\Delta_{ij})$~--- функция за\-трат на 
изменение ковариации между $i$-й и~$j$-й компонентами; $\sigma_{ij}^0$~--- 
начальное значение ковариации; $W_{ij}$~--- предельная величина за\-трат на 
это изменение.
  
  Минимизация риска не всегда может быть приемлемым управ\-ле\-ни\-ем. 
Альтернативой является достижение приемлемого рис\-ка~$r^*$ при 
минимальных изменениях чис\-ло\-вых характеристик гауссовской 
сис\-те\-мы~$\mathbf{X}$. Здесь возможны два варианта по\-ста\-нов\-ки задачи.
  
  Во-первых, на основе~(\ref{e6-t}) получаем альтернативный вариант:
  
  \noindent
  \begin{equation*}
  \begin{array}{c}
  \displaystyle \sum\limits_{i=1}^m v_i(\delta_i) +\sum\limits^m_{i-1} 
\sum\limits^m_{j=i} w_{ij}(\Delta_{ij}) \to \min\limits_{\boldsymbol{\Sigma}, 
\mathbf{a}}\,,\\[6pt]
%  \hspace*{27mm}
\boldsymbol{\Sigma} \in G(\boldsymbol{\Sigma})\,,\enskip \mathbf{a}\in 
H(\mathbf{a})\,,\\[6pt]
  \delta_i =a_i-a_i^0\,,\enskip \Delta_{ij}=\sigma_{ij}-\sigma_{ij}^0\,,\enskip 
i,j=1,\ldots ,m\,,\\[6pt]
  r(\boldsymbol{\Sigma}, \mathbf{a}) =r^*\,.
  \end{array}
  \end{equation*}
  
  
  Во-вторых, если сложно задать функции за\-трат~$v_i(\cdot)$ 
и~$w_{ij}(\cdot)$, то мож\-но минимизировать суммарное 
изменение~$\boldsymbol{\Sigma}$ и~$\mathbf{a}$, перейдя к~задаче
  \begin{equation*}
  \begin{array}{c}
  \displaystyle \sum\limits_{i=1}^m \alpha_i \delta_i^2 +\sum\limits^m_{i-1} 
\sum\limits^m_{j=i} \beta_{ij} \Delta_{ij}^2 \to 
\min\limits_{\boldsymbol{\Sigma}, \mathbf{a}}\,,\\[6pt]
 % \hspace*{27mm}
 \boldsymbol{\Sigma} \in G(\boldsymbol{\Sigma})\,,\enskip \mathbf{a}\in 
H(\mathbf{a})\,,\\[6pt]
  \delta_i =a_i-a_i^0\,,\enskip \Delta_{ij}=\sigma_{ij}-\sigma_{ij}^0\,,\enskip 
i,j=1,\ldots ,m\,,\\[6pt]
  r(\boldsymbol{\Sigma}, \mathbf{a}) =r^*\,,
  \end{array}
  \end{equation*}
где $\alpha_i$ и~$\beta_{ij}$~--- весовые коэффициенты.

\section{Заключение}

\noindent
  \begin{enumerate}[1.]
  \item  Предложен новый подход к~исследованию рис\-ка сложных сис\-тем. 
В~его основе лежит моделирование сис\-те\-мы в~виде мно\-го\-мер\-ной случайной 
величины, компоненты которой являются факторами риска.
  \item  Для гауссовских стохастических сис\-тем предложены модели 
управления рис\-ком на основе его минимизации или до\-сти\-же\-ния заданного 
уров\-ня, используя в~качестве управ\-ля\-ющих переменных чис\-ло\-вые 
характеристики случайного век\-то\-ра~--- вектор математических ожиданий 
и~ковариационную мат\-рицу.
  \item  В настоящее время обычно при исследовании риска слож\-ных 
многомерных сис\-тем не выделяют в~явном виде их компоненты и~их 
коррелированность. Как показало моделирование, неучет в~явном виде 
многомерности сис\-те\-мы и~взаимной коррелированности ее компонент может 
привести к~существенному занижению фактического рис\-ка. Усиление тес\-но\-ты 
корреляционной связи между факторами риска приводит к~значительному 
рос\-ту ве\-ро\-ят\-ности одновременного принятия ими опас\-ных значений.
  \item Предложенная гипотеза об управ\-ле\-нии рис\-ком слож\-ной сис\-те\-мы на 
основе изменения чис\-ло\-вых характеристик ее математической модели в~форме 
случайного век\-то\-ра носит предварительный характер. Необходимо выполнить 
апро\-ба\-цию данного подхода на ряде задач.
  \end{enumerate}
  \vspace*{-8pt}
  
{\small\frenchspacing
 {%\baselineskip=10.8pt
 \addcontentsline{toc}{section}{References}
 \begin{thebibliography}{99}
 
 \bibitem{3-t} %1
\Au{Гор А.} Земля на чаше весов. В~по\-ис\-ках новой об\-щей цели~// Новая 
пост\-ин\-ду\-стри\-аль\-ная волна на Западе: Антология~/ Пер. с~англ.~--- М.: 
Academia, 1999. С.~557--571. (\Au{Gore~A.} Earth in the balance. Forging a~new 
common purpose.~--- London: Earthscan Publications Ltd., 1992.)
\bibitem{1-t} %2
\Au{Воробьев Ю.\,Л., Малинецкий~Г.\,Г., Махутов~Н.\,А.} Управ\-ле\-ние рис\-ком 
и~устойчивое развитие: Человеческое измерение~// Известия вузов. 
Прикладная нелинейная динамика, 2000. Т.~8. №\,6. С.~12--26.
\bibitem{2-t} %3
\Au{Порфирьев Б.\,Н.} Снижение природных рис\-ков экономического развития 
России: роль государства~// Актуальные проб\-ле\-мы гражданской защиты:  
Мат-лы 11-й Междунар. науч.-практич. конф. по проб\-ле\-мам защиты 
населения и~территорий от чрезвычайных ситуаций.~--- 
Н.~Новгород: Вектор-ТиС, 2006. С.~44--50. {\sf 
gov.mari.ru/debzn/omgo/46.djvu}.

\bibitem{4-t}
\Au{Порфирьев Б.\,Н.} Управ\-ле\-ние в~чрез\-вы\-чай\-ных ситуациях. Итоги науки 
и~техники. Проб\-ле\-мы без\-опас\-ности: чрез\-вы\-чай\-ные ситуации. Т.~1.~--- М.: 
ВИНИТИ, 1991. 204~с.
\bibitem{5-t}
\Au{Вишняков Я.\,Д., Радаев~Н.\,Н.} Общая теория рисков.~--- 
2-е изд., испр.~--- М.: Академия, 2008. 368~с.
\bibitem{6-t}
\Au{Акимов В.\,А., Лесных~В.\,В., Радаев~Н.\,Н.} Риски в~природе, техносфере, 
обществе и~экономике.~--- М.: Деловой экспресс, 2004. 352~с.
\bibitem{7-t}
\Au{Соложенцев Е.\,Д.} Сценарное ло\-ги\-ко-ве\-ро\-ят\-ност\-ное управ\-ле\-ние 
рис\-ком в~бизнесе и~технике.~--- 2-е изд.~--- СПб.: Биз\-нес-прес\-са, 2006. 560~с.
\bibitem{8-t}
\Au{Рябинин И.\,А.} На\-деж\-ность и~без\-опас\-ность струк\-тур\-но-слож\-ных  
сис\-тем.~--- СПб.: Политехника, 2000. 248~с.
\bibitem{9-t}
\Au{Тырсин А.\,Н.} О~моделировании рис\-ка в~сис\-те\-мах критичных 
инфра\-струк\-тур~// Экономические и~технические аспекты без\-опас\-ности 
строительных критичных инфраструктур: Тезисы Междунар. конф.~--- 
Екатеринбург: УрФУ, 2015. С.~205--208. {\sf 
http://elar.urfu. ru/bitstream/10995/33468/1/safety\_2015.pdf.}
\bibitem{10-t}
\Au{Тырсин А.\,Н., Сурина~А.\,А.} Моделирование рис\-ка в~многомерных 
сто\-ха\-сти\-че\-ских сис\-те\-мах~// Вестн. Томского государственного университета. 
Управ\-ле\-ние, вы\-чис\-ли\-тель\-ная техника и~информатика, 2017. №\,2(39).  
С.~65--72.
\bibitem{11-t}
\Au{Гнеденко Б.\,В.} Курс тео\-рии вероятностей.~--- 8-е изд., испр. и~доп.~--- М.: 
Едиториал УРСС, 2005. 448~с.
\bibitem{12-t}
\Au{Pena D., Rodriguez~J.} Descriptive measures of multivariate scatter and linear 
dependence~// J.~Multivariate Anal., 2003. Vol.~85. P.~361--374.
\bibitem{13-t}
\Au{Кирьянов Б.\,Ф., Токмачёв~М.\,С.} Математические модели 
в~здравоохранении.~--- Великий Новгород: НовГУ им.\ Ярослава Мудрого, 
2009. 279~с.
\bibitem{14-t}
\Au{Цинкер М.\,Ю., Кирьяков~Д.\,А., Камалтдинов~М.\,Р.} Применение 
комплексного ин\-дек\-са нарушения здо\-ровья населения для оцен\-ки 
популяционного здо\-ровья в~Пермском крае~// Из\-вес\-тия Самарского научного 
цент\-ра РАН, 2013. Т.~15. №\,3(6). С.~1988--1992.
\bibitem{15-t}
Профилактика хронических неинфекционных заболеваний. Рекомендации.~--- 
М., 2013. 128~с. {\sf http:// www.webmed.irkutsk.ru/doc/pdf/prevent.pdf}.
\bibitem{16-t}
\Au{Тырсин А.\,Н., Калев~О.\,Ф., Яшин~Д.\,А., Лебедева~О.\,В.} Оцен\-ка 
со\-сто\-яния здо\-ровья популяции на основе энтропийного моделирования~// 
Математическая био\-ло\-гия и~био\-ин\-фор\-ма\-ти\-ка, 2015. Т.~10. Вып.~1.  
С.~206--219. doi: 10.17537/2015.10.206.
\bibitem{17-t}
\Au{Михайлов Г.\,А., Войтишек~А.\,В.} Чис\-лен\-ное ста\-ти\-сти\-че\-ское 
моделирование. Методы Мон\-те-Кар\-ло.~--- М.: Академия, 2006. 368~с.
\bibitem{18-t}
\Au{Пантелеев А.\,В., Летова~Т.\,А.} Методы оптимизации в~примерах 
и~задачах.~--- 3-е изд., стер.~---М.: Выс\-шая школа, 2008. 544~с.

 \end{thebibliography}

 }
 }

\end{multicols}

\vspace*{-6pt}

\hfill{\small\textit{Поступила в~редакцию 21.08.17}}

\vspace*{6pt}

%\newpage

%\vspace*{-24pt}

\hrule

\vspace*{2pt}

\hrule

%\vspace*{8pt}


\def\tit{A~MODEL OF RISK MANAGEMENT IN~GAUSSIAN\\ STOCHASTIC SYSTEMS}

\def\titkol{A~model of risk management in Gaussian stochastic systems}

\def\aut{A.\,N.~Tyrsin$^{1,2}$ and~A.\,A.~Surina$^3$}

\def\autkol{A.\,N.~Tyrsin and~A.\,A.~Surina}

\titel{\tit}{\aut}{\autkol}{\titkol}

\vspace*{-9pt}


\noindent
$^1$Ural Federal University named after first President of Russia B.\,N.~Yeltsin, 
19~Mira Str., Ekaterinburg 620002,\linebreak
$\hphantom{^1}$Russian Federation 

\noindent
$^2$Institute of Economics, Ural Branch of the Russian Academy of Sciences, 
29~Moskovskaya Str., Yekaterinburg\linebreak
$\hphantom{^1}$620014, Russian Federation

\noindent
$^3$Institute of Natural Sciences, South Ural State University, 87~Lenin Ave., 
Chelyabinsk 454080, Russian Federation


\def\leftfootline{\small{\textbf{\thepage}
\hfill INFORMATIKA I EE PRIMENENIYA~--- INFORMATICS AND
APPLICATIONS\ \ \ 2018\ \ \ volume~12\ \ \ issue\ 2}
}%
 \def\rightfootline{\small{INFORMATIKA I EE PRIMENENIYA~---
INFORMATICS AND APPLICATIONS\ \ \ 2018\ \ \ volume~12\ \ \ issue\ 2
\hfill \textbf{\thepage}}}

\vspace*{3pt}




\Abste{A new approach to research of risk of multidimensional 
stochastic systems is described. It is based on a~hypothesis that 
the risk can be managed by changing probabilistic properties of a~component of 
a~multidimensional stochastic system. The case of Gaussian stochastic systems 
described by random vectors having
the multidimensional normal distribution 
is investigated. Modeling has shown that multidimensionality of a~system
and relative\linebreak\vspace*{-12pt}}

\Abstend{
 correlation of components unaccounted in an explicit form, 
can lead to essential understating of risk factors. Results of calculation 
of the probability of a~dangerous outcome depending on numerical characteristics of 
a~multidimensional Gaussian random variable (a~covariance matrix and 
a~vector of mathematical expectations) are given. Approbation of the suggested model 
is executed by the example of the analysis of the risk of cardiovascular 
diseases in population. Models of risk management in the form of 
a~minimization problem or achievement of the given level are described. 
Control variables are the numerical characteristics of a~random vector covariance 
matrix and a~vector of mathematical expectations. Approbation of the method of 
risk management was carried 
out by means of statistical model operation by the Monte-Carlo method.}

\KWE{risk; model; stochastic system;  random vector; control; normal distribution}


\DOI{10.14357/19922264180208} %

%\vspace*{-14pt}

  \Ack
   \noindent
   The work was supported by the Russian Foundation
   for Basic Research (project 17-01-00315а).
   



%\vspace*{-3pt}

  \begin{multicols}{2}

\renewcommand{\bibname}{\protect\rmfamily References}
%\renewcommand{\bibname}{\large\protect\rm References}

{\small\frenchspacing
 {%\baselineskip=10.8pt
 \addcontentsline{toc}{section}{References}
 \begin{thebibliography}{99}
 
 \bibitem{3-t-1} %1
\Aue{Gore, A.} 1992. \textit{Earth in the balance. Forging a~new common purpose.} 
London: Earthscan Publications Ltd. 440~p.
\bibitem{1-t-1} %2
\Aue{Vorob'ev, Yu.\,L., G.\,G.~Malinetskiy, and N.\,A.~Makhutov}. 2000. Uprav\-le\-nie 
ris\-kom i~ustoy\-chi\-voe raz\-vi\-tie: che\-lo\-ve\-che\-skoe iz\-m\-ere\-nie [Management of risk and 
sustainable development: Human measurement]. \textit{Izvestiya vuzov. Pri\-klad\-naya 
nelineynaya dinamika}  [Proceedings of the Universities. Applied Nonlinear 
Dynamics] 8(6):12--26.
\bibitem{2-t-1} %3
\Aue{Porfir'ev, B.\,N.} 2006. Sni\-zhe\-nie pri\-rod\-nykh ris\-kov eko\-no\-mi\-che\-sko\-go 
raz\-vi\-tiya Ros\-sii: rol' go\-su\-dar\-st\-va
 [The reduction of natural risks of economic 
development of Russia: The role of the state]. \textit{Aktu\-al'\-nye prob\-le\-my 
grazh\-dan\-skoy za\-shchi\-ty: Mat-ly 11-y Mezhdunar.  
nauch.-praktich. konf. po problemam zashchity naseleniya i~territoriy ot\linebreak 
chrezvychaynykh situatsiy} [Actual Problems of Civil\linebreak Protection:  11th Scientific and 
Practical Conference (International) on Problems of Protection of the Population and 
Territories from Emergency Situations Proceedings]. 
N.~Novgorod: Vector-TiS. 44--50. Available at: {\sf 
http://gov.mari.ru/debzn/omgo/46.djvu} (accessed  August~7, 2017).

\bibitem{4-t-1}
\Aue{Porfir'ev,  B.\,N.} 1991. \textit{Up\-rav\-le\-nie v~chrez\-vy\-chay\-nykh si\-tu\-a\-tsi\-yakh. 
T.~1. Ito\-gi nau\-ki i~tekh\-ni\-ki. Prob\-le\-my bez\-opas\-nosti: 
chrezvychaynye situatsii} 
[Management in emergency situations. Vol.~1. The results of science and 
technology. Security concerns: Emergency situations]. Moscow: VINITI. 204~p.
\bibitem{5-t-1}
\Aue{Vishnyakov, Ya.\,D., and N.\,N.~Radaev.} 2008. \textit{Ob\-shhaya teo\-riya 
ris\-kov} [Common theory of risks]. 2nd ed. Moscow: Academy. 368~p.
\bibitem{6-t-1}
\Aue{Akimov, V.\,A.,  V.\,V.~Lesnykh, and N.\,N.~Radaev.} 2004. \textit{Riski 
v~pri\-ro\-de, tekh\-no\-sfe\-re, ob\-shchest\-ve i~eko\-no\-mi\-ke} 
[Risks in the nature, technosphere, 
society, and the economy]. Moscow: Business Express. 352~p.
\bibitem{7-t-1}
\Aue{Solozhentsev, E.\,D.} 2006. \textit{Stse\-nar\-noe 
lo\-gi\-ko-ve\-ro\-yat\-nost\-noe 
uprav\-le\-nie ris\-kom v~biz\-ne\-se i~tekh\-ni\-ke} [Scenario logic and probabilistic 
management of risk in business and engineering]. 2nd ed. St.\ Petersburg: Biznes 
pressa. 560~p.
\bibitem{8-t-1}
\Aue{Ryabinin, I.\,A.} 2000. \textit{Na\-dezh\-nost' i~bezopas\-nost'  
struk\-tur\-no-slozh\-nykh sis\-tem} [Reliability and safety of the structural and composite 
systems]. St.\ Petersburg: Polytechnique. 248~p.
\bibitem{9-t-1}
\Aue{Tyrsin, A.\,N.} 2015. O~mo\-de\-li\-ro\-va\-nii ris\-ka v~sis\-te\-makh kri\-tich\-nykh 
infra\-struk\-tur [About model operation of risk in the systems of critical infrastructures]. 
\textit{Economic and Technical Aspects of
Safety of Civil Engineering Critical Infrastructures
Conference (International) Abstracts}.
Ekaterinburg: Ural Federal University. 205--208.
 Available at: {\sf 
http://elar.urfu.ru/bitstream/10995/33468/1/safety\_\linebreak 2015.pdf}  (accessed August~7, 
2017).
\bibitem{10-t-1}
\Aue{Tyrsin, A.\,N., and A.\,A.~Surina.} 2017. Modelirovanie ris\-ka 
v~mno\-go\-mer\-nykh sto\-kha\-sti\-che\-skikh sis\-te\-makh [Modeling of risk in 
multidimensional stochastic systems]. \textit{Vestn. Tomskogo gosudarstvennogo 
universiteta. Upravlenie, vy\-chis\-li\-tel'\-naya tekh\-ni\-ka 
i~in\-for\-ma\-ti\-ka} [Bull. Tomsk State 
University. Management, Computer Facilities, and Informatics] 2(39):65--72.
\bibitem{11-t-1}
\Aue{Gnedenko, B.\,V.} 2005. \textit{Kurs teo\-rii ve\-ro\-yat\-no\-stey} [Course of 
probability theory]. 8th ed. Moscow: Editorial URSS. 448~p.
\bibitem{12-t-1}
\Aue{Pena, D., and J.~Rodriguez.} 2003. Descriptive measures of multivariate 
scatter and linear dependence.  \textit{J.~Multivariate Anal.} 85:361--374.
\bibitem{13-t-1}
\Aue{Kir'yanov, B.\,F., and M.\,S.~Tokmachev.} 2009. \textit{Ma\-te\-ma\-ti\-che\-skie 
modeli v~zdra\-vo\-okh\-ra\-ne\-nii} [Mathematical models in health care]. Veliky 
Novgorod: NovSU. 279~p.
\bibitem{14-t-1}
\Aue{Tsinker, M.\,Yu., D.\,A.~Kir'yakov, and M.\,R.~Kamaltdinov.} 2013. Pri\-me\-ne\-nie 
komp\-leks\-no\-go indek\-sa na\-ru\-she\-niya zdo\-rov'ya na\-se\-le\-niya dlya otsen\-ki 
po\-pu\-lya\-tsi\-on\-no\-go zdo\-rov'ya v~Permskom krae [The integrated index of health 
situation of the population to assess population health in the Perm region.
\textit{Izvestiya Samarskogo nauchnogo tsent\-ra RAN} [Proceedings of the Samara 
Scientific Center of RAS] 15(3(6)):1988--1992.
\bibitem{15-t-1}
Profilaktika khro\-ni\-che\-skikh ne\-in\-fek\-tsi\-on\-nykh za\-bo\-le\-va\-niy. 
Re\-ko\-men\-da\-tsii 
[Prevention of pre-existing conditions. Recommendations]. Moscow. 128~p.
Available at: {\sf 
http://www.webmed.irkutsk.ru/doc/pdf/prevent.pdf} (accessed August~7, 2017).
\bibitem{16-t-1}
\Aue{Tyrsin, A.\,N., O.\,F.~Kalev, D.\,A.~Yashin, and O.\,V.~Lebedeva.} 2015. 
Otsen\-ka so\-sto\-yaniya zdo\-rov'ya po\-pu\-lya\-tsii 
na osnove entropiynogo mo\-de\-li\-ro\-va\-niya 
[Assessment of health status of a~population on the basis of entropy modeling]. 
\textit{Math. Biol. Bioinf.} 10(1):206--219. doi: 10.17537/2015.10.206.
\bibitem{17-t-1}
\Aue{Mihaylov, G.\,A., and A.\,V.~Voytishek}. 2006. \textit{Chis\-len\-noe 
sta\-ti\-sti\-che\-skoe mo\-de\-li\-ro\-va\-nie. Metody Monte-Karlo} [Numerical statistical model 
operation. Monte-Carlo methods]. Moscow: Akademy. 368~p.
\bibitem{18-t-1}
\Aue{Panteleev, A.\,V., and T.\,A.~Letova.} 2008. \textit{Metody op\-ti\-mi\-za\-tsii 
v~pri\-me\-rakh i~za\-da\-chakh} [Optimization methods in examples and tasks]. 3rd ed. 
Moscow: Higher School. 544~p.
\end{thebibliography}

 }
 }

\end{multicols}

\vspace*{-3pt}

\hfill{\small\textit{Received August 21, 2017}}

%\vspace*{-24pt}

\Contr

\noindent
\textbf{Tyrsin Alexander N.} (b.\ 1961)~-- Doctor of Science in technology, Head of 
Department of Applied Mathematics, Ural Federal University named after first 
President of Russia B.\,N.~Yeltsin, 19~Mira Str., Ekaterinburg 620002, Russian 
Federation; senior scientist, Institute of Economics, Ural Branch of the Russian 
Academy of Sciences, 29~Moskovskaya Str., Yekaterinburg 620014, Russian 
Federation; \mbox{at2001@yandex.ru} 

\vspace*{3pt}

\noindent
\textbf{Surina Alfiya A.} (b.\ 1987)~--- PhD student, Institute of Natural Sciences, 
South Ural State University, 87~Lenin Ave., Chelyabinsk 454080, Russian 
Federation; \mbox{dallila87@mail.ru} 



\label{end\stat}


\renewcommand{\bibname}{\protect\rm Литература} 