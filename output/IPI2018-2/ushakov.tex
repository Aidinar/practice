\def\stat{ushakov}

\def\tit{ДОСТАТОЧНЫЕ УСЛОВИЯ ЭРГОДИЧНОСТИ ПРИОРИТЕТНЫХ СИСТЕМ МАССОВОГО 
ОБСЛУЖИВАНИЯ$^*$}

\def\titkol{Достаточные условия эргодичности приоритетных систем массового 
обслуживания}

\def\aut{А.\,В.~Мистрюков$^1$,  В.\,Г.~Ушаков$^2$}

\def\autkol{А.\,В.~Мистрюков,  В.\,Г.~Ушаков}

\titel{\tit}{\aut}{\autkol}{\titkol}

\index{Мистрюков А.\,В.}
\index{Ушаков В.\,Г.}
\index{Mistryukov A.\,V.}
\index{Ushakov V.\,G.}




{\renewcommand{\thefootnote}{\fnsymbol{footnote}} \footnotetext[1]
{Работа выполнена при частичной
финансовой поддержке РФФИ (проект 18-07-00678).}}


\renewcommand{\thefootnote}{\arabic{footnote}}
\footnotetext[1]{Факультет вычислительной математики и~кибернетики 
Московского государственного университета им.\ М.\,В.~Ломоносова, 
\mbox{unf08@rambler.ru}}
\footnotetext[2]{Факультет вычислительной математики и~кибернетики
Московского государственного университета им.\ М.\,В.~Ломоносова;
Институт проб\-лем информатики Федерального исследовательского
центра <<Информатика и~управ\-ле\-ние>> Российской академии наук,
\mbox{vgushakov@mail.ru}}

%\vspace*{-6pt}


\Abst{Известные результаты по эргодичности
приоритетных сис\-тем массового обслуживания получены 
в~предположении, что входящие потоки требований всех приоритетов
являются пуассоновскими. В~данной работе это требование ослаблено,
а~именно: найдены достаточные условия эргодичности сис\-тем массового
обслуживания с~двумя классами приоритетов, в~которых только поток
требований высшего приоритета является пуассоновским. Исследованы
сис\-те\-мы с~относительным приоритетом и~тремя разновидностями
абсолютного приоритета: с~дообслуживанием, обслуживанием заново 
и~потерей прерванного требования низшего приоритета. Для получения
искомых условий для последовательных времен ожидания в~очереди
требований каждого приоритета получены рекуррентные соотношения,
известные как рекурсия Линдли. Полученная цепь Маркова исследуется
методом проб\-ных функций. Найдены достаточные условия, при которых
ис\-сле\-ду\-емая цепь является харрисовой и,~следовательно, имеет
стационарное распределение.}

\KW{относительный приоритет; абсолютный
приоритет; эргодичность; метод пробных функций; время ожидания;
рекурсия Линдли}

\DOI{10.14357/19922264180204}
  
%\vspace*{-6pt}


\vskip 10pt plus 9pt minus 6pt

\thispagestyle{headings}

\begin{multicols}{2}

\label{st\stat}

\section{Введение} 

Проблема нахождения условий эргодичности традиционна
для тео\-рии массового обслуживания.
Эти условия важны для приложений, поскольку они определяют
соотношения между параметрами модели, при выполнении которых не
возникает бесконечно больших очередей. Существует обширная
литература по эргодической тео\-рии случайных процессов (см.,
например,~\cite{key-1,key-2}). Известно большое чис\-ло достаточных
условий эр\-го\-дич\-ности различных классов случайных процессов. Среди
них особое мес\-то занимают марковские процессы. Большая часть
условий эр\-го\-дич\-ности марковских процессов формулируется в~терминах
свойств переходной функции. Для теории массового обслуживания,
однако, нуж\-ны условия, выраженные через па\-ра\-мет\-ры ис\-сле\-ду\-емой
сис\-те\-мы (входящие потоки, длительности обслуживания и~т.\,п.),
получение которых в~качестве следствия из общих результатов
(особенно для слож\-ных, в~част\-ности приоритетных, сис\-тем) является
нетривиальной задачей.

При изучении эргодичности приоритетных сис\-тем обычно накладывались
ограничения либо на время меж\-ду поступлениями требований в~сис\-те\-му,
либо на время их обслу\-жи\-ва\-ния, а~именно: предполагалось, что эти
времена имеют экспоненциальное распределение (см., например,~\cite{key-3}). 
Работ, в~которых исследуется эр\-го\-дич\-ность
приоритетных сис\-тем, не относящихся к~классам $M/G/1$ и~$G/M/1$,
практически нет.

При исследовании различных сис\-тем массового обслуживания час\-то
удается найти случайный процесс~$w_{n}$ с~дискретным временем,
характеризующий работу сис\-те\-мы (длина очереди, время ожидания,
чис\-ло обслуженных требований и~т.\,п.), который удовле\-тво\-ря\-ет
рекуррентным соотношениям вида:

\noindent
$$
w_{n+1}=\max\left(0,\:w_{n}\hm+s_{n}-t_{n}\right).
$$
 В~тео\-рии массового
обслуживания такие соотношения принято называть рекурсией Линдли.
Различные асимптотические свойства этого соотношения, в~том чис\-ле
усло\-вия эргодичности, достаточно хорошо изучены (см., 
например,~\cite{key-4,key-5}); приложения к~теории массового обслуживания
можно найти в~монографиях~\cite{key-6,key-7}.

В данной статье получено рекуррентное соотношение Линдли для времен
ожидания неприоритетных требований и~найдены достаточные условия
эргодичности сис\-тем обслуживания с~относительным и~тремя
разновидностями абсолютного приорите-\linebreak\vspace*{-12pt}

\pagebreak

\noindent
та при условии, что лишь
приоритетный входящий поток является пуассоновским.



\section{Определения и~вспомогательные утверждения}

Пусть на измеримом пространстве $(X,\sigma (X))$ задана марковская
цепь $x_n,$  $n\hm\in N_{0}\hm=\{0,1,2\ldots\},$ с~переходными
вероятностями 
$$
P(x,A)=\mathbf{P}\left(x_{n+1}\in A|x_{n}=x\right),\enskip x\in X\,,\
A\in\sigma(X)\,.
$$


\noindent
\textbf{Определение~1.}\
Инвариантной мерой марковской цепи с~переходными вероятностями
$P(x,A)$ называется вероятностная мера~$\pi$ на~$\sigma(X),$ для
которой справедливо равенство:
$$
\pi(A)=\int\limits_{X}\pi(dx)P(x,A)\ \mbox{для любого}\
A\in\sigma(X)\,.
$$

\smallskip

\noindent
\textbf{Определение~2.}\
Марковская цепь называется эргодической, если она имеет
единственную инвариантную меру.

\smallskip

Для того чтобы сформулировать достаточные условия эргодичности
марковских цепей, введем сле\-ду\-ющие обозначения:
$$
Pf(x)=\mathbf{E}\left(f(x_{n+1})|x_{n}=x\right)=
\int\limits_{X}f(y)P(x,dy)\,,
$$ 
где $f$~--- функция, определенная на~$X$;
$$
\tau_{A}(x)=\inf\left(n:\ x_n\in A|x_0=x\right),\enskip A\in\sigma(X)\,,
$$
есть время первого попадания в~множество~$A$ из точ\-ки~$x,$ а
$$
\mathbf{E}_{x}\left(\tau_{A}\right)=
\mathbf{E}\left(\inf(n:\:x_n\in A|x_0=x)\right)
$$ 
есть среднее время первого достижения  множества~$A$ из точ\-ки~$x.$

\smallskip

\noindent
\textbf{Теорема~1.}\
\textit{Пусть марковская цепь удовлетворяет сле\-ду\-ющим условиям:
существуют множество $A\hm\subset X$, $p\hm>0$ и~вероятностная мера~$\nu$
на~$\sigma(X)$ такие, что}
\begin{itemize}
\item[(а)] $\displaystyle
\mathbf{P}\left(\tau_{A}(x)<\infty\right)=1$ для всех $x\hm\in A^{c}$;
\item[(б)] $\displaystyle\sup\limits_{x\in A} \mathbf{E}_{x}\left(\tau_{A}\right)<\infty$;
\item[(в)] $\displaystyle P_{m}(x,B)\geq p\:\nu(B)$ для некоторого $m\hm\in N$ и~всех 
$B\hm\in\sigma(X)$, $x\hm\in A$.
\end{itemize}

\textit{Тогда она имеет единственную инвариантную меру~$\pi$. Если, кроме
того, цепь непериодична, то переходные вероятности сходятся 
к~инвариантной мере по вариации.}

\smallskip

Марковскую цепь, удовле\-тво\-ря\-ющую условиям тео\-ре\-мы~1, называют
харрисовой.

\smallskip

\noindent
\textbf{Теорема~2.}\ 
\textit{Пусть существуют неотрицательная функция $w(x)$, $A\hm\subset X$ 
и~$\varepsilon\hm>0$ такие, что $Pw(x)\hm\leq w(x)\hm-\varepsilon$ для всех
$x\hm\in A^{c}$ и~$\sup\limits_{x\in A}Pw(x)\hm<\infty.$ Тогда}
$\sup\nolimits_{x\in A}\mathbf{E}_{x}(\tau_{A})\hm<\infty.$


\smallskip

\noindent
Д\,о\,к\,а\,з\,а\,т\,е\,л\,ь\,с\,т\,в\,а\ \ тео\-рем~1 и~2 мож\-но найти, например, 
в~\cite{key-1,key-2}.

\smallskip

\noindent
\textbf{Лемма~1.}\
\textit{Пусть выполнены условия тео\-ре\-мы~$2$. Тогда}
$$
\mathbf{E}_{x}\left(\tau_{A}\right)\leq
\fr{w(x)}{\varepsilon}\ \mbox{для всех}\ x\in A^{c}.
$$

\section{Условия эргодичности систем с~относительным и~абсолютным приоритетами}

Рассматривается одноканальная сис\-те\-ма массового обслуживания 
с~неограниченным чис\-лом мест для ожидания и~двумя потоками
требований. Первый поток~--- пуассоновский, второй~--- рекуррентный
с~абсолютно непрерывной функцией рас\-пределения интервалов между
поступлениями требований. Требования первого потока имеют один из четырех видов
приоритета перед требованиями второго потока:
\begin{enumerate}[(1)]
\item относительный приоритет;

\item абсолютный приоритет с~дообслуживанием
прерванного требования;

\item абсолютный приоритет с~потерей
прерванного требования;

\item абсолютный приоритет с~обслуживанием заново
прерванного требования.
\end{enumerate}

 Требования одного приоритета обслуживаются
в~порядке их поступления в~сис\-тему.

Обозначим через $s_1^{(i)}, s_2^{(i)},\ldots$ 
и~$t_1^{(i)}, t_2^{(i)},\ldots$ последовательные времена
обслуживания и~интервалы между поступлениями требований $i$-го
потока, $s^{(i)}\hm=\mathbf{E} s_1^{(i)}$, 
$t^{(i)}\hm=\mathbf{E} t_1^{(i)}$, $i\hm=1,2$,
$p(s)\hm=\mathbf{E} e^{-st_1^{(2)}}$, $\alpha\hm=(t^{(1)})^{-1}.$

Пусть $w_n^{(2)}$~--- время ожидания до начала обслуживания $n$-м
требованием второго потока (нумерация производится в~порядке
поступления в~сис\-тему).


\smallskip

\noindent
\textbf{Теорема~3.}\
\textit{Для эргодичности последо\-ва\-тель\-ности~$w_n^{(2)}$ достаточно
существования второго момента времени обслуживания приоритетных
требований и~выполнения условий}:
\begin{enumerate}[(1)]
\item
\textit{для систем обслуживания с~относительным и~абсолютным 
с~дообслуживанием прерванного требования приоритетами}:
$$\displaystyle
\fr{s^{(1)}}{t^{(1)}}+\fr{s^{(2)}}{t^{(2)}}<1\,;
$$

\item \textit{для сис\-тем обслуживания с~абсолютным приоритетом и~обслуживанием
заново прерванного требования}:
$$\displaystyle
\fr{(1-p(\alpha))}{p(\alpha)}\,\fr{t^{(1)}}{t^{(2)}}+\fr{s^{(1)}}{t^{(1)}}<1\,;
$$


\item \textit{для сис\-тем обслуживания с~абсолютным приоритетом и~потерей
прерванного требования}:
$$\displaystyle
(1-p(\alpha))\fr{t^{(1)}}{t^{(2)}}+\fr{s^{(1)}}{t^{(1)}}<1\,.
$$

\end{enumerate}

%\smallskip

\noindent
Д\,о\,к\,а\,з\,а\,т\,е\,л\,ь\,с\,т\,в\,о\,.\ \ 
Для всех рассматриваемых приоритетных дисциплин справедливы
сле\-ду\-ющие рекуррентные соотношения:
\begin{multline*}
w_{n+1}^{(2)}=I\left(t_{n+1}^{(2)}<w_{n}^{(2)}+
T\left(s_n^{(2)}\right)\right)\times{}\\
{}\times
\left(w_n^{(2)}+T\left(s_n^{(2)}\right)-t_{n+1}^{(2)}\right)+{}\\
{}+I\left(t_{n+1}^{(2)}>w_{n}^{(2)}+T\left(s_n^{(2)}\right)\right)\times{}\\
{}\times T^{\ast}\left(t_{n+1}^{(2)},
w_n^{(2)},T\left(s_n^{(2)}\right)\right).
\end{multline*}
Здесь $T\left(s_n^{(2)}\right)$~--- интервал времени с~момента
поступления на обслуживание $n$-го требования второго приоритета
до первого после этого момента освобождения сис\-те\-мы от этого
требования и~требований более высокого приоритета; $I(A)$~---
индикатор события~$A$; случайная величина
$T^{\ast}\left(t_{n+1}^{(2)},
w_n^{(2)},T\left(s_n^{(2)}\right)\right)$ имеет такое же
распределение, как время до первого освобождения системы после
момента времени $t_{n+1}^{(2)}\hm-w_n^{(2)}\hm+T\left(s_n^{(2)}\right),$
если в~сис\-те\-му поступает только поток приоритетных требований 
и~в~начальный момент сис\-те\-ма свободна от них. В~литературе,
посвященной анализу приоритетных сис\-тем массового обслуживания,
случайное время $T\left(s_n^{(2)}\right)$ называют циклом
обслуживания~\cite{key-3,key-8}. Если поток приоритетных
требований является пуассоновским, то распределение и~моменты
цикла обслуживания известны. В~част\-ности, математические ожидания
равны:
${s^{(2)}t^{(1)}}/({t^{(1)}\hm-s^{(1)}})$,  
$({(1\hm-p(\alpha))}/{p(\alpha)}){\left(t^{(1)}\right)^2}/({t^{(1)}\hm-s^{(1)}})$ 
и~$(1\hm-p(\alpha)){\left(t^{(1)}\right)^2}/({t^{(1)}\hm-s^{(1)}})$
для сис\-тем с~относительным и~абсолютным с~дообслуживанием,
абсолютным с~обслуживанием заново и~абсолютным с~потерей
приоритетами соответственно. Нетрудно найти и~распределение
случайной величины $T^{\ast}\left(t_{n+1}^{(2)},
w_n^{(2)},T\left(s_n^{(2)}\right)\right)$ при любых фиксированных
$t_{n+1}^{(2)}$, $T\left(s_n^{(2)}\right)$ и~$w_n^{(2)},$ но оно в~данной ситуации
не понадобится. Заметим только, что при любых $u\hm\geqslant 0$, $v\hm\geqslant 0$,
$w\hm\geqslant 0$ и~$u\hm-w\hm+v\hm\geqslant 0$ существует по\-сто\-ян\-ная~$c$ такая, 
что $\mathbf{E} T^{\ast}(u, w, v)\hm<c$.

При сделанных предположениях по\-сле\-до\-ва\-тель\-ность~$w_n^{(2)}$
является неразложимой непериодической однородной цепью Маркова.
Покажем, что она эргодична, если $\mathbf{E} T\left(s_n^{(2)}\right)\hm-\mathbf{E}
t_{n+1}^{(2)}\hm<0.$ Для этого воспользуемся результатами тео\-рем~1 и~2.

Пусть $w_{0}>0$ таково, что
\begin{multline*}
\varepsilon=-E\left(T\left(s_n^{(2)}\right)-t_{n+1}^{(2)}\right)\times{}\\
{}\times
I\left(T\left(s_n^{(2)}\right)-t_{n+1}^{(2)}>-w_0\right)>0\,.
\end{multline*}

Положим $A\hm=\left[0,\,\max\left(w_0,c\right)\right].$ Тогда для
любого $w\hm\in A^{c}$ имеем:
\begin{multline*}
\mathbf{E} \left(w_{n+1}^{(2)}\left|\right.w_{n}^{(2)}=w\right)={}\\
{}=\!
\mathbf{E}\,
I\!\left(t_{n+1}^{(2)}<w+T\left(s_n^{(2)}\right)\!\right)\!
\left(w+T\left(s_n^{(2)}\right)-t_{n+1}^{(2)}\right)+{}\hspace*{-0.64064pt}
\\
{}+\mathbf{E}\,
I\left(t_{n+1}^{(2)}>w+T\left(s_n^{(2)}\right)\!\right)T^{\ast}\!\left(t_{n+1}^{(2)},
w,T\left(s_n^{(2)}\right)\right)<{}\hspace*{-1.91495pt}\\
{}< -\varepsilon+w\,\mathbf{E}\,
I\left(t_{n+1}^{(2)}< w+
T\left(s_n^{(2)}\right)\right)+{}\\
{}+
c\,\mathbf{E}\, I\left(t_{n+1}^{(2)}>w+T\left(s_n^{(2)}\right)\right)< w-\varepsilon\,.
\end{multline*}
Очевидно, что неравенство 
$$
\sup\limits_{w\in A}\mathbf{E}
\left(w_{n+1}^{(2)}\left|\right.w_{n}^{(2)}\hm=w\right)<\infty$$
 также
выполняется. Таким образом, условия тео\-ре\-мы~2, а~значит, и~условия~(а)
и~(б) тео\-ре\-мы~1 выполнены. Покажем, что и~усло\-вие~(в) тео\-ре\-мы~1
выполнено. Обозначим 
$$
\tau_0(x)=\inf\limits_{n\geqslant 1}\left(n\, w_n^{(2)}=0,w_0^{(2)}\hm=x\right).
$$ 
Так как для любого~$x$
$$\mathbf{E}\left(\tau_0(x)\right)\hm<\infty\,,
$$ 
то существует~$n$ такое, что
вероятность~$p$ перехода за~$n$ шагов из со\-сто\-яния~$x$ в~со\-сто\-яние~$0$ 
больше~0. В~качестве меры~$\nu$ из формулировки теоремы~1
мож\-но взять дель\-та-ме\-ру, сосредоточенную в~нуле. Таким образом, все
условия тео\-ре\-мы~1 выполнены.

{\small\frenchspacing
 {%\baselineskip=10.8pt
 \addcontentsline{toc}{section}{References}
 \begin{thebibliography}{9}


\bibitem{key-2}
\Au{ Walters P.} An introduction to ergodic
theory.~--- New York, NY, USA: Springer-Verlag, 1982. 259~p.

\bibitem{key-1}
\Au{Боровков~А.\,А.} Эргодичность и~устой\-чи\-вость случайных
процессов.~--- М.: Эдиториал УРСС, 1999. 450 с.

\bibitem{key-3}
\Au{Гнеденко Б.\,В., Даниелян~Э.\,А., Димитров~Б.\,Н., Климов~Г.\,П., Матвеев~В.\,Ф.}
Приоритетные сис\-те\-мы обслуживания.~--- М.: Изд-во Московского
ун-та, 1973. 448~с.

\bibitem{key-4} 
\Au{Meyn S., Tweedie~R.} Markov chains and stochastic
stability.~--- New York, NY, USA: Springer Verlag, 1993. 562~p.

\bibitem{key-5}
\Au{Asmussen~S.} Subexponential asymptotics
for stochastic processes: Extremal behavior, stationary
distributions and first passage probabilities~// Ann. 
Appl. Probab., 1998. Vol.~8. P.~354--374.




\bibitem{key-7}
\Au{Cohen J.\,W.} The single server queue.~--- Amsterdam: North-Holland
Publishing Co., 1982. 694~p.

\bibitem{key-6}
\Au{Asmussen S.}  Applied probability and queues.~--- 
New York, NY, USA: Springer-Verlag, 2003. 439~p.

\bibitem{key-8}
\Au{Матвеев В.\,Ф., Ушаков~В.\,Г.} Сис\-те\-мы массового
обслуживания.~--- М.: Изд-во Московского ун-та, 1984. 240~с.

 \end{thebibliography}

 }
 }

\end{multicols}

\vspace*{-6pt}

\hfill{\small\textit{Поступила в~редакцию 31.01.18}}

\vspace*{6pt}

%\newpage

%\vspace*{-24pt}

\hrule

\vspace*{2pt}

\hrule

%\vspace*{8pt}


\def\tit{SUFFICIENT ERGODICITY CONDITIONS FOR~PRIORITY QUEUES}

\def\titkol{Sufficient ergodicity conditions for priority queues}

\def\aut{A.\,V.~Mistryukov$^1$ and V.\,G.~Ushakov$^{1,2}$}

\def\autkol{A.\,V.~Mistryukov and V.\,G.~Ushakov}

\titel{\tit}{\aut}{\autkol}{\titkol}

\vspace*{-9pt}


\noindent
$^1$Department of Mathematical Statistics, 
Faculty of Computational Mathematics and Cybernetics,\linebreak 
$\hphantom{^1}$M.\,V.~Lomonosov Moscow State University, 1-52~Leninskiye Gory, Moscow 119991, 
GSP-1, Russian Fed-\linebreak 
$\hphantom{^1}$eration

\noindent
$^2$Institute of Informatics Problems, Federal Research Center ``Computer Science 
and Control'' of the Russian\linebreak
$\hphantom{^1}$Academy of Sciences, 44-2~Vavilov Str., 
Moscow 119333, Russian Federation


\def\leftfootline{\small{\textbf{\thepage}
\hfill INFORMATIKA I EE PRIMENENIYA~--- INFORMATICS AND
APPLICATIONS\ \ \ 2018\ \ \ volume~12\ \ \ issue\ 2}
}%
 \def\rightfootline{\small{INFORMATIKA I EE PRIMENENIYA~---
INFORMATICS AND APPLICATIONS\ \ \ 2018\ \ \ volume~12\ \ \ issue\ 2
\hfill \textbf{\thepage}}}

\vspace*{3pt}




\Abste{Known results in ergodicity of priority queues are based on 
the assumption that interarrival times in each queue have exponential 
distribution. The aim of this paper is to relax this assumption, 
namely, to establish sufficient conditions of ergodicity for queues 
with two priority classes $GI|GI|1$ under assumption that interarrival 
times only in high priority class queue have exponentional distribution. Queues 
with nonpreemptive and different kinds of preemptive priority are considered. 
To formulate desired conditions, the authors use Lindley's recursion for 
waiting times of each priority class queue. Using Lyapunov--Foster criteria, 
the authors obtain sufficient conditions for a~given recursion to be 
a~Harris-ergodic Markov chain, meaning to have a~unique invariant measure, 
to which its transition probabilities converge in total variation.}

\KWE{head of the line priority; preemptive priority; ergodicity; the method of test functions; 
waiting time; Lindley recursion}


 

\DOI{10.14357/19922264180204} %

%\vspace*{-14pt}

 
\Ack
\noindent
This work was partly supported by the Russian Foundation for Basic Research 
(project No.\,18-07-00678).


%\vspace*{-3pt}

  \begin{multicols}{2}

\renewcommand{\bibname}{\protect\rmfamily References}
%\renewcommand{\bibname}{\large\protect\rm References}

{\small\frenchspacing
 {%\baselineskip=10.8pt
 \addcontentsline{toc}{section}{References}
 \begin{thebibliography}{9}


\bibitem{2-us}
\Aue{Walters, P.} 1982.  
\textit{An introduction to ergodic theory}. New York, NY: Springer-Verlag. 259~p.

\bibitem{1-us}
\Aue{Borovkov, A.\,A.} 1999. \textit{Ergodichnost' 
i~ustoichivost' sluchaynykh protsessov} [Ergodicity and stability of 
stochastic processes]. Moscow: Editorial URSS. 450~p.

\bibitem{3-us}
\Aue{Gnedenko, B.\,V., E.\,A.~Danielyan, B.\,N.~Dimitrov, G.\,P.~Klimov, 
and V.\,F.~Matveev}. 
1973. \textit{Prioritetnye sistemy obsluzhivaniya} [Priority queueing systems].
 Moscow: Izd-vo Moskovskogo un-ta. 448~p.
\bibitem{4-us}
\Aue{Meyn, S., and R.~Tweedie}. 1993.  
\textit{Markov chains and stochastic stability}. New York, NY: Springer Verlag. 562~p.
\bibitem{5-us}
\Aue{Asmussen, S.} 1998.  Subexponential asymptotics for stochastic processes: 
Extremal behavior, stationary distributions and first passage probabilities.  
\textit{Ann. Appl. Probab.} 8:354--374.

\bibitem{7-us}
\Aue{Cohen, J.\,W.} 1982.  
\textit{The single server queue}. Amsterdam: North-Holland Publishing Co. 694~p.

\bibitem{6-us}
\Aue{Asmussen, S.} 2003.  
\textit{Applied probability and queues}. New York, NY: Springer-Verlag. 439~p.
\bibitem{8-us}
\Aue{Matveev, V.\,F., and V.\,G.~Ushakov}. 1984.  
\textit{Sistemy massovogo obsluzhivaniya} [Queueing systems]. 
Moscow: Izd-vo Moskovskogo un-ta. 240~p.
\end{thebibliography}

 }
 }

\end{multicols}

\vspace*{-3pt}

\hfill{\small\textit{Received January 31, 2018}}

%\vspace*{-24pt}


\Contr

\noindent
\textbf{Mistryukov Andrey V.} (b.\ 1988)~--- 
PhD student, Department of Mathematical Statistics, 
Faculty of Computational Mathematics and Cybernetics, 
M.\,V.~Lomonosov Moscow State University, 1-52~Leninskiye Gory, Moscow 119991, 
GSP-1, Russian Federation;
\mbox{unf08@rambler.ru}

\vspace*{6pt}

\noindent
\textbf{Ushakov Vladimir G.} (b.\ 1952)~--- 
Doctor of Science in physics and mathematics, professor, Department 
of Mathematical Statistics, Faculty of Computational Mathematics and 
Cybernetics, M.\,V.~Lomonosov Moscow State University, 1-52~Leninskiye Gory, 
Moscow 119991, GSP-1, Russian Federation; senior scientist, 
Institute of Informatics Problems, Federal Research Center ``Computer Science 
and Control'' of the Russian Academy of Sciences, 44-2~Vavilov Str., 
Moscow 119333, Russian Federation; \mbox{vgushakov@mail.ru}
\label{end\stat}


\renewcommand{\bibname}{\protect\rm Литература} 