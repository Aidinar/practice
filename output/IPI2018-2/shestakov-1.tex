

%\newcommand{\cov}{\textbf{cov}}
\newcommand{\I}{\mathbf{1}}
\newcommand{\Variance}{\sf D}

\def\stat{shest-1}

\def\tit{НЕСМЕЩЕННАЯ ОЦЕНКА РИСКА СТАБИЛИЗИРОВАННОЙ ЖЕСТКОЙ
ПОРОГОВОЙ ОБРАБОТКИ В~МОДЕЛИ С~ДОЛГОСРОЧНОЙ ЗАВИСИМОСТЬЮ$^*$}

\def\titkol{Несмещенная оценка риска стабилизированной жесткой
пороговой обработки в~модели с~долгосрочной зависимостью}

\def\aut{О.\,В.~Шестаков$^1$}

\def\autkol{О.\,В.~Шестаков}

\titel{\tit}{\aut}{\autkol}{\titkol}

\index{Шестаков О.\,В.}
\index{Shestakov O.\,V.}




{\renewcommand{\thefootnote}{\fnsymbol{footnote}} \footnotetext[1]
{Работа выполнена при частичной финансовой поддержке РФФИ (проект 16-07-00736).}}


\renewcommand{\thefootnote}{\arabic{footnote}}
\footnotetext[1]{Московский государственный университет им.\ М.\,В.~Ломоносова, 
кафедра математической статистики факультета вычислительной математики и~кибернетики; 
Институт проб\-лем информатики Федерального исследовательского центра 
<<Информатика и~управ\-ле\-ние>> Российской академии наук, \mbox{oshestakov@cs.msu.su}}

%\vspace*{-6pt}



\Abst{Методы подавления шума в~сигналах и~изображениях, основанные на процедуре 
пороговой обработки коэффициентов вейв\-лет-раз\-ло\-же\-ния, стали популярными благодаря 
своей прос\-то\-те, ско\-рости и~возможности адаптации к~функциям сигналов, име\-ющим на 
разных участках различную степень ре\-гу\-ляр\-ности. Анализ погрешностей этих методов 
является важ\-ной практической задачей, поскольку дает воз\-мож\-ность оценивать качество 
как самих методов, так и~используемого для обработки оборудования. 
Рассматривается предложенный недавно стабилизированный метод жесткой пороговой 
обработки, в~котором устранены основные недостатки мягкой и~жесткой пороговой 
обработки, и~исследуются статистические свойства этого метода. В~модели 
данных с~аддитивным гауссовским шумом проводится анализ несмещенной оценки 
среднеквадратичного риска. В~предположении о~том, что шумовые коэффициенты 
обладают долгосрочной за\-ви\-си\-мостью, приводятся условия, при которых имеет 
место сильная состоятельность и~асимптотическая нор\-маль\-ность несмещенной оценки риска. 
Полученные результаты дают воз\-мож\-ность строить асимптотические доверительные 
интервалы для погрешностей пороговой обработки, используя только наблюдаемые данные.}

\KW{вейвлеты; пороговая обработка; несмещенная оценка риска; коррелированный шум; 
асимптотическая нормальность}

\DOI{10.14357/19922264180202}
  
\vspace*{6pt}


\vskip 10pt plus 9pt minus 6pt

\thispagestyle{headings}

\begin{multicols}{2}

\label{st\stat}

\section{Введение}

В задачах статистической обработки данных час\-то предполагается, что наблюдения 
независимы. Однако существует множество физических процессов, де\-мон\-ст\-ри\-ру\-ющих 
долгосрочную за\-ви\-си\-мость, при которой корреляции между наблюдениями убывают 
столь медленно, что ряд из них не сходится. Такая долгосрочная за\-ви\-си\-мость 
час\-то наблюдается, например, при исследовании геофизических процессов, в~которых 
она принимает форму длительных периодов больших или маленьких значений наблюдений. 
Схожие явления демонстрируют помехи в~коммуникационных каналах. 

При 
анализе и~обработке сигналов, ре\-гист\-ри\-ру\-емых при изучении таких процессов, 
широко применяют\-ся методы вейв\-лет-ана\-ли\-за. К~данным применяется 
вейв\-лет-пре\-об\-ра\-зо\-ва\-ние и~осуществляется пороговая обработка получившихся 
вейв\-лет-ко\-эф\-фи\-ци\-ен\-тов~\cite{Mall99}. Наличие шумовой со\-став\-ля\-ющей 
в~сигнале неизбежно приводит к~погрешностям в~оценке сигнала. При использовании 
метода мягкой пороговой обработки мож\-но по\-стро\-ить статистическую оценку 
среднеквадратичной по\-греш\-ности (рис\-ка)~\cite{DonJ95}. Свойства этой оценки в~моделях 
с~независимым и~коррелированным шумом исследовались 
в~работах~\cite{Mar09,MSH10-1,SH12-1,ESH14,SH16}. Показано, что при определенных 
условиях оценка рис\-ка является сильно со\-сто\-ятель\-ной и~асимптотически нормальной.

Однако при мягкой пороговой обработке в~оценке функции сигнала появляется 
дополнительное смещение. 

При жесткой пороговой обработке используется разрывная 
пороговая функция, что приводит к~появлению дополнительных артефактов, 
отсутствию устой\-чи\-вости при выборе порога и~не\-воз\-мож\-ности по\-стро\-ения 
несмещенной оценки сред\-не\-квад\-ра\-тич\-но\-го риска. 
В~работе~\cite{HL10} предложен стабилизированный вариант жесткой пороговой обработки, 
позволяющий обойти указанные недостатки. 

В~данной работе исследуются статистические 
свойства оценки сред\-не\-квад\-ра\-тич\-но\-го рис\-ка стабилизированной жест\-кой пороговой 
обработки в~модели с~коррелированным шумом. Предполагается, что при обработке 
используется <<универсальный>> порог. При определенных условиях на глад\-кость 
функции сигнала показано, что оценка рис\-ка, как и~в~случае мяг\-кой пороговой 
обработки, является асимп\-то\-ти\-чески нормальной и~сильно со\-сто\-ятель\-ной. Данные 
свойства служат обоснованием использования этой оценки при по\-стро\-ении доверительных 
интервалов для тео\-ре\-ти\-че\-ско\-го среднеквадратичного риска.

\section{Модель данных с~долгосрочной зависимостью}

В данной работе предполагается, что функция сигнала~$f$ задана на отрезке $[0,1]$ 
и~равномерно регулярна по Липшицу с~некоторым показателем $\gamma\hm>0$. 
Вейв\-лет-раз\-ло\-же\-ние функции~$f$ пред\-став\-ля\-ет собой ряд
\begin{equation}
\label{Wavelet_Decomp}
f=\sum\limits_{j,k\in Z}\langle f,\psi_{jk}\rangle\psi_{jk}\,,
\end{equation}
где $\psi_{jk}(t)=2^{j/2}\psi(2^jt-k)$, а $\psi(t)$~--- некоторая материнская 
вейв\-лет-функ\-ция. Индекс~$j$ в~\eqref{Wavelet_Decomp} называется масштабом, 
а~индекс $k$~--- сдвигом. Функция~$\psi$ должна удовле\-тво\-рять определенным 
требованиям, однако ее можно выбрать таким образом, чтобы она обладала 
некоторыми полезными свойствами, например была~$M$ раз дифференцируемой, имела
 заданное чис\-ло~$M$ нулевых моментов и~достаточно быст\-ро убывала на бес\-ко\-неч\-ности. 
 Известно~\cite{Mall99}, что если $M\hm\geqslant\gamma$, то найдется такая 
 константа $C_f\hm>0$, что
\begin{equation}
\label{Coeff_Decay}
\langle f,\psi_{jk}\rangle\leqslant\fr{C_f}{2^{j\left(\gamma+1/2\right)}}\,.
\end{equation}
Всюду далее предполагается, что используются вейв\-ле\-ты Мейера~\cite{Mall99}, 
обладающие нужным чис\-лом нулевых моментов и~непрерывных производных.

На практике регистрируются дискретные отсчеты функции сигнала $f_i\hm=
f\left(i/2^J\right)$, $i\hm=1,\ldots, 2^J$ (считается, что чис\-ло этих отсчетов 
равно~$2^J$ для некоторого $J\hm>0$). Дискретное вейв\-лет-пре\-обра\-зо\-ва\-ние 
пред\-став\-ля\-ет собой умножение вектора\linebreak
 из значений~$f_i$  на ортогональную мат\-ри\-цу, 
определяемую вейв\-лет-функ\-ци\-ей~$\psi$~\cite{Mall99}. 
При этом дискретные вейв\-лет-ко\-эф\-фи\-ци\-ен\-ты связаны с~непрерывными 
коэффициентами разложения в~\eqref{Wavelet_Decomp}\linebreak
 сле\-ду\-ющим образом: 
$\mu_{jk}\hm\approx 2^{J/2}\langle f,\psi_{jk}\rangle$~\cite{Mall99}. 
Это приближение тем точ\-нее, чем больше~$J$.

В~реальных наблюдениях всегда присутствует шум. 
В~данной работе рас\-смат\-ри\-ва\-ет\-ся модель коррелированного шума. 
Пусть $\{e_i, i \hm\in \mathbb{Z}\}$~--- ста\-цио\-нар\-ный гауссовский процесс 
с~ковариационной по\-сле\-до\-ва\-тель\-ностью $r_k \hm= \cov (e_i,e_{i+k})$. Будем 
полагать, что~$e_i$ имеют нулевое сред\-нее и~единичную дис\-пер\-сию. 
Также предположим, что автоковариационная функция шума убывает со ско\-ростью 
$r_k \hm\sim Ak^{-\alpha}$, где $0 \hm< \alpha\hm <1$, что соответствует 
долгосрочной за\-ви\-си\-мости между наблюдениями~\cite{JS97}.

Рассмотрим следующую модель данных:
\begin{equation*}
Y_i = f_i + e_i\,, \enskip i = 1, \ldots, 2^J\,.
%\label{Data_Model}
\end{equation*}
Для $t\in [0,1]$ определим наблюдаемый процесс
\begin{equation*}
Y_J(t) = \fr{1}{2^J} \sum\limits_{j=1}^{\left[2^Jt\right]} 
\!Y_i = F_J(t)+ \fr{1}{2^J} \sum\limits_{i=1}^{\left[2^Jt\right]}\! e_i\,,
\end{equation*}
где $F_J(t)=1/2^J \sum\nolimits_{i=1}^{\left[2^Jt\right]} f(i/2^J)$~--- <<суммарный сигнал>>. 
Положим $\tau^2 \hm= 2A/((1\hm-\alpha)(2\hm-\alpha))$ 
(без ограничения общ\-ности далее предполагается, что $\tau\hm=1$) и~$H \hm= 
1\hm- \alpha/2 \hm\in (1/2,1)$.

Определим дробное броуновское движение $\mathbf{B}_H(t)$ как
гауссовский процесс на~$\mathbb{R}$ с~нулевым средним  
и~ковариационной функцией
\begin{equation*}
r(s,t) = \fr{V_H}{2}\left(|s|^{2H} + |t|^{2H} - |t-s|^{2H}\right)\,, \enskip
 s,t \in \mathbb{R}\,,
\end{equation*}
где
\begin{equation*}
V_H = \Variance \left(\mathbf{B}_H(1)\right) = 
\fr{-\Gamma(2-2H)\cos(\pi H)}{\pi H(2H-1)}\,.
\end{equation*}
Лемма~5.1 из~\cite{T75} показывает, что
\begin{equation*}
2^{\alpha J/2}\left(Y_J(t) - F_J(t)\right) 
\Rightarrow \mathbf{B}_H(t)\,, \enskip t\in[0,1].
\end{equation*}
Таким образом, полагая $\epsilon \hm= 2^{-J/2}$, можно аппроксимировать 
наблюдаемый процесс~$Y_J(t)$ с~по\-мощью~$Y(t)$ для $t \hm\in [0,1]$:
\begin{equation}
\label{Scale_proc}
Y(t) = F(t) + \epsilon^{\alpha} \mathbf{B}_H(t)\,.
\end{equation}
Применяя к~(\ref{Scale_proc}) вейв\-лет-раз\-ло\-же\-ние 
и~аппроксимируя его дискретным вейв\-лет-пре\-об\-ра\-зо\-ва\-ни\-ем, 
приходим к~сле\-ду\-ющей модели дискретных вейв\-лет-ко\-эф\-фи\-ци\-ен\-тов~\cite{JS97,J99}:
\begin{equation}
X_{jk} = \mu_{jk} +  2^{{(1-\alpha)(J-j)}/{2}} z_{jk}\,,
\label{Wav_LRD_model}
\end{equation}
где 
$$
z_{jk}=2^{{j(1-\alpha)}/{2}} \int \psi_{jk}\, d\mathbf{B}_H\,. 
$$

Шумовые коэффициенты~$z_{jk}$ имеют стандартное нормальное распределение, 
но не являются независимыми. Дис\-пер\-сия коэффициентов модели~(\ref{Wav_LRD_model}) 
имеет вид:
$$
\sigma_j^2= 2^{(1-\alpha)(J-j)}\,.
$$


\section{Стабилизированная жесткая обработка}

Для подавления шума и~построения оценки функции сигнала к~коэффициентам~$X_{jk}$ 
обычно применяется функция жесткой пороговой обработки $\rho_{H}(x,T_j)
\hm=y\I(\abs{x}>T_j)$ или функция мяг\-кой пороговой обработки $\rho_{S}(x,T_j)
\hm=\mathrm{sgn}(x)\left(\abs{x}-T_j\right)_{+}$ с~порогом~$T_j$, 
который может зависеть от мас\-шта\-ба~$j$, но не зависит от сдвига~$k$. 
Смысл пороговой обработки заключается в~удалении до\-ста\-точ\-но маленьких коэффициентов, 
которые считаются шумом.

Как уже отмечалось, функция~$\rho_{H}$ разрывна, что приводит к~отсутствию 
устой\-чи\-вости, а~функция~$\rho_{S}$ приводит к~по\-яв\-ле\-нию дополнительного 
смещения в~оценке функции сигнала. В~работе~\cite{HL10} предложен альтернативный 
метод пороговой обработки, являющийся сгла\-жен\-ным аналогом жест\-кой пороговой 
обработки. В~этом методе оценки~$\mu_{jk}$ вы\-чис\-ля\-ют\-ся по формулам:
\begin{equation*}
\widehat{\mu}_{jk}=
\Expect \left[\rho_{H}(X_{jk}+\lambda\xi_{jk},T_j)|X_{jk}\right],
\end{equation*}
где случайные величины~$\xi_{jk}$ имеют стандартное нормальное распределение и~не 
зависят от~$X_{jk}$, а~$\lambda\hm>0$~--- 
параметр стабилизации, отвечающий за степень сглаживания. 
Вы\-чис\-ляя математическое ожидание, получаем:
\begin{multline*}
\hspace*{-1.5mm}\widehat{\mu}_{jk}=X_{jk}\left[\Phi\!\left(-\fr{T_j+X_{jk}}
{\lambda}\!\right)+1-
\Phi\left(\!\fr{T_j-X_{jk}}{\lambda}\!\right)\right]+{}\hspace*{0.43806pt}\hspace*{-0.87613pt}\\
{}+\lambda\left[
\phi\left(\fr{T_j-X_{jk}}{\lambda}\!\right)-
\phi\left(\fr{T_j+X_{jk}}{\lambda}\right)\right].
\end{multline*}
Достоинством такого метода является бесконечная диф\-фе\-рен\-ци\-ру\-емость~$\widehat{\mu}_{jk}$ 
по~$X_{jk}$, что приводит к~более устойчивым оценкам~\cite{HL10}. Заметим также, 
что при $\lambda\hm\to0$ получается обычный метод жест\-кой пороговой обработки.

Среднеквадратичная по\-греш\-ность (риск) описанного метода определяется по формуле:
\begin{equation*}
\label{Risk}
R_J(f)=\sum\limits_{j=0}^{J-1}\sum\limits_{k=0}^{2^j-1}\Expect
\left(\widehat{\mu}_{jk}-\mu_{jk}\right)^2.
\end{equation*}
В~\cite{HL10} показано, что
\begin{multline*}
\Expect\left(\widehat{\mu}_{jk}-\mu_{jk}\right)^2={}\\
{}=
\Expect\left[(X_{jk}-\widehat{\mu}_{jk})^2+2\sigma_j^2
\fr{\partial}{\partial X_{jk}}\,\widehat{\mu}_{jk}\right]-\sigma_j^2,
\end{multline*}
где

\noindent
\begin{multline*}
\fr{\partial}{\partial X_{jk}}\,\widehat{\mu}_{jk}={}\\
{}=
\left[\Phi\left(-\fr{T_j+X_{jk}}{\lambda}\right)+1-
\Phi\left(\fr{T_j-X_{jk}}{\lambda}\right)\right]+{}\\
{}+
\fr{T_j}{\lambda}\left[\phi\left(\fr{T_j-X_{jk}}{\lambda}\right)+
\phi\left(\fr{T_j+X_{jk}}{\lambda}\right)\right].
\end{multline*}
Таким образом, величина
\begin{multline}
\label{Risk_Estimate}
\widehat{R}_J(f)={}\\
{}+\sum\limits_{j=0}^{J-1}\sum\limits_{k=0}^{2^j-1}
\left[(X_{jk}-\widehat{\mu}_{jk})^2+2\sigma_j^2
\fr{\partial}{\partial X_{jk}}\,\widehat{\mu}_{jk}-\sigma_j^2\right]
\end{multline}
является несмещенной оценкой среднеквадратичного риска~$R_J(f)$, не зависящей от 
ненаблюдаемых <<чистых>> значений~$\mu_{jk}$.

В данной работе параметр~$\lambda$ предполагается фиксированным, а~в~качестве~$T_j$ 
для каждого мас\-шта\-ба~$j$ выбирается <<универсальный>> порог $T_j\hm=
\sigma_j\sqrt{2\ln 2^J}$, который поз\-во\-ля\-ет достичь хороших 
результатов при подавлении шума и~обеспечивает бли\-зость сред\-не\-квад\-ра\-тич\-но\-го 
рис\-ка к~минимальному как при жесткой, так и~при мягкой пороговой обработке~\cite{DJ94}. В следующем разделе устанавливаются свойства асимптотической нормальности и~сильной состоятельности оценки (\ref{Risk_Estimate}). Эти свойства служат обоснованием использования $\widehat{R}_J(f)$ при построении доверительных интервалов для $R_J(f)$.

\vspace*{-3pt}

\section{Статистические свойства оценки среднеквадратичного риска}

Покажем, что оценка (\ref{Risk_Estimate}) является асимптотически нормальной.

\smallskip

\noindent
\textbf{Теорема~1.}\ \textit{Пусть $\alpha\hm>1/2$, а функция~$f$ 
задана на отрезке $[0,1]$ и~равномерно регулярна по Липшицу с~показателем 
$\gamma\hm > (4\alpha\hm-2)^{-1}$. Тогда при стабилизированной жест\-кой пороговой 
обработке с~<<универсальными>> порогами~$T_j$ имеет место схо\-ди\-мость по 
распределению}
\begin{align}\label{risk_norm}
{\sf P}\left(\fr{\widehat{R}_J(f) - R_J(f)}{ D_J }<x\right) \to \Phi(x) \
\mbox{при } J \rightarrow \infty\,.
\end{align}
\textit{Здесь $\Phi(x)$~--- функция распределения стандартного нормального закона; 
$D_J^2\hm=C_\alpha 2^J$, где  $C_\alpha$~--- константа, за\-ви\-ся\-щая только от~$\alpha$ 
и~выбранного вейв\-лет-ба\-зиса}.

\smallskip

\noindent
\textbf{Замечание.}\ 
На практике, например при по\-стро\-ении асимптотических доверительных интервалов 
для сред\-не\-квад\-ра\-тич\-но\-го рис\-ка, необходимо знать константу~$C_\alpha$. 
В~отличие от случая независимых наблюдений эта константа зависит от выбранного 
вейв\-лет-ба\-зи\-са (напомним, что по предположению базис строится на основе 
вейвлетов Мейера). Константу~$C_\alpha$ мож\-но вы\-чис\-лить достаточно точ\-но,
 пользуясь методикой работы~\cite{E15}.
 
 \smallskip

\noindent
Д\,о\,к\,а\,з\,а\,т\,е\,л\,ь\,с\,т\,в\,о\,. 
По\-сколь\-ку в~силу условий тео\-ре\-мы $(2\gamma+1)^{-1}\hm<1\hm-(2\alpha)^{-1}$, 
мож\-но выбрать такое~$p$, что
$(2\gamma+1)^{-1}\hm<p\hm<1\hm-(2\alpha)^{-1}$. Обозначим сла\-га\-емые 
в~(\ref{Risk_Estimate}) через~$F_{jk}$ и~запишем дробь под ве\-ро\-ят\-ностью 
в~\eqref{risk_norm} в~виде:

\vspace*{-4pt}

\noindent
\begin{multline}
\label{Two_Sums}
\fr{\widehat{R}_J(f)-R_J(f)}{D_J}=\fr{1}{D_J}\sum\limits_{j=0}^{[pJ]}
\sum\limits_{k=0}^{2^j-1}\left[F_{jk}-\Expect F_{jk}\right]+{}\\
{}+
\fr{1}{D_J}\sum\limits_{j=[pJ]+1}^{J-1}\sum\limits_{k=0}^{2^j-1}
\left[F_{jk}-\Expect F_{jk}\right].
\end{multline}
Учитывая вид~$T_j$, мож\-но убедиться, что существует такая константа $C_F\hm>0$, что

\vspace*{3pt}

\noindent
\begin{equation}
\label{Bound}
\abs{F_{jk}-\Expect F_{jk}} \leqslant C_F J2^{(J-j)(1-\alpha)}\enskip \mbox{п.\ в.}
\end{equation}
Таким образом,

\vspace*{-2pt}

\noindent
\begin{multline*}
\abs{\sum\limits_{i=0}^{[pJ]}\sum\limits_{l=0}^{2^i-1} [F_{jk}-\Expect F_{jk}]}
\leqslant{}\\
{}\leqslant C_F\sum\limits_{i=0}^{[pJ]}\sum\limits_{l=0}^{2^i-1}
 С J2^{(J-i)(1-\alpha)} \leqslant C'_F  J 2^{J(1-\alpha+\alpha p)}\ \  
 \mbox{п.\ в.}
\end{multline*}
с некоторой константой $C'_F\hm>0$.
Следовательно, так как $1\hm-\alpha\hm+\alpha p\hm<1/2$, первая сумма 
в~\eqref{Two_Sums} стремится к~нулю п.\ в.\ при $J\hm\to\infty$.

 Повторяя рассуждения работ~\cite{JS97,J99}, мож\-но убедиться, что 
 по\-сле\-до\-ва\-тель\-ность $\bigl\{F_{jk}\bigr\}$ обладает свойством $\rho$-пе\-ре\-ме\-ши\-ва\-ния 
 и,~следовательно, обладает свойством $\alpha$-пе\-ре\-ме\-ши\-ва\-ния~\cite{B05}.

Далее, рассуждая как в~работе~\cite{ESH14} и~используя~\eqref{Coeff_Decay}, 
мож\-но показать, что при выполнении условий тео\-ре\-мы существует такая 
константа $C_\alpha\hm>0$, за\-ви\-ся\-щая только от~$\alpha$ и~вы\-бран\-но\-го 
вейв\-лет-ба\-зи\-са,~что

\vspace*{4pt}

\noindent
\begin{equation*}
%\label{Var_Order}
\lim\limits_{J\to\infty} \fr{1}{D^2_J}\,\Variance\left[
\sum\limits_{j=[pJ]+1}^{J-1}\sum\limits_{k=0}^{2^j-1}   
\left[F_{jk}-\Expect F_{jk}\right]\right] =1\,.
\end{equation*}
Кроме того, легко видеть, что

\vspace*{2pt}

\noindent
\begin{equation*}
\sup_{J>0} \fr{1}{D^2_J}\sum\limits_{j=[pJ]+1}^{J-1}\sum\limits_{k=0}^{2^j-1} 
\Variance  F_{jk} < \infty\,.
\end{equation*}

 Наконец, выполнено условие Линдеберга: для любого $\eps\hm>0$
 
\columnbreak
 
 \noindent
\begin{multline}
\label{Norm_Cond}
\hspace*{-3mm}\fr{1}{D^2_J}\sum\limits_{j=[pJ]+1}^{J-1}\!\sum\limits_{k=0}^{2^j-1} \!\Expect  
\left( F_{jk} - \Expect F_{jk}\right)^2
\mathbf{1}\left( |F_{jk} - \Expect  F_{jk}| >\right.\\
\left.>\eps D_J\right) \rightarrow 0\ \mbox{при }J\rightarrow\infty\,.
\end{multline}
Действительно, так как 
$$
\abs{F_{jk}-\Expect F_{jk}} \leqslant C_F
J2^{(J-j)(1-\alpha)}\ \mbox{п.~в.}\,,
$$ 
а~$D_J^2\hm=C_\alpha 2^J$, то при
$\alpha\hm>1/2$, начиная с~некоторого~$J$, все индикаторы 
в~(\ref{Norm_Cond}) обращаются в~ноль.

 Таким образом, выполнены все условия тео\-ре\-мы~2.1 из работы~\cite{P96} 
 и~справедливо~\eqref{risk_norm}. Тео\-ре\-ма доказана.

\smallskip

Оценка риска~$\widehat{R}_J(f)$ также является силь\-но со\-сто\-ятель\-ной, причем 
при более слабых ограничениях на~$\alpha$ и~ре\-гу\-ляр\-ность~$f$.

\smallskip

\noindent
\textbf{Теорема~2.}\ \textit{Пусть функция $f\hm\in  L^2([0,1])$. Тогда}
\begin{equation*}
\fr{\widehat{R}_J(f)-R_J(f)}{2^{\lambda J}}\rightarrow 0 \ \mbox{п. в.}\ 
\mbox{при } J\rightarrow\infty
\end{equation*}
\textit{при любом $\lambda>1/2$ в~случае $1/2\hm\leqslant\alpha\hm<1$ 
и~любом $\lambda\hm>1\hm-\alpha$ в~случае} $0\hm<\alpha\hm<1/2$.

\smallskip

Принимая во внимание оценку~\eqref{Bound}, доказательство этого утверж\-де\-ния 
практически пол\-ностью повторяет доказательство со\-от\-вет\-ст\-ву\-юще\-го свойства оцен\-ки 
рис\-ка в~работе~\cite{SH16}.

\vspace*{-9pt}

{\small\frenchspacing
 {%\baselineskip=10.8pt
 \addcontentsline{toc}{section}{References}
 \begin{thebibliography}{99}
\bibitem{Mall99}
\Au{Mallat S.} A~wavelet tour of signal processing.~--- New York, NY, USA: 
Academic Press, 1999. 857~p.

\bibitem{DonJ95}
\Au{Donoho D., Johnstone~I.\,M.} Adapting to unknown smoothness via wavelet shrinkage~// 
J.~Am. Stat. Assoc., 1995. Vol.~90. P.~1200--1224.

\bibitem{Mar09}
\Au{Маркин А.\,В.} Предельное распределение оценки рис\-ка при пороговой обработке 
вейв\-лет-ко\-эф\-фи\-ци\-ен\-тов~// Информатика и~её
применения, 2009. Т.~3. Вып.~4. С.~57--63.

\bibitem{MSH10-1}
\Au{Маркин А.\,В., Шестаков~О.\,В.} 
О~со\-сто\-ятель\-ности оценки рис\-ка при пороговой обработке вейв\-лет-ко\-эф\-фи\-ци\-ен\-тов~// 
Вестн. Моск.
 ун-та. Сер.~15: Вычисл. матем. и~киберн., 2010. №\,1. C.~26--34.

\bibitem{SH12-1}
\Au{Шестаков О.\,В.} Асимптотическая нор\-маль\-ность оценки рис\-ка пороговой обработки 
вейв\-лет-ко\-эф\-фи\-ци\-ен\-тов при выборе адап\-тив\-но\-го порога~// 
Докл. РАН, 2012. Т.~445. №\,5. С.~513--515.

\bibitem{ESH14}
\Au{Ерошенко А.\,А., Шестаков~О.\,В.} 
Асимптотические свойства оценки риска при пороговой обработке вейв\-лет-ко\-эф\-фи\-ци\-ен\-тов 
в~модели
с~коррелированным шумом~// Информатика и~её применения, 2014. Т.~8. Вып.~1. С.~36--44.

\bibitem{SH16}
\Au{Шестаков О.\,В.} Сходимость почти всюду оценки рис\-ка пороговой обработки 
вейв\-лет-ко\-эф\-фи\-ци\-ен\-тов в~модели с~коррелированным шумом~// 
Вестн. Моск.
\linebreak\vspace*{-12pt}

\pagebreak

\noindent
 ун-та. Сер.~15: Вычисл. матем. и~киберн., 2016. №\,3. C.~19--22.

\bibitem{HL10}
\Au{Huang H.-C., Lee~T.\,C.\,M.} 
Stabilized thresholding with generalized sure for image denoising~// 
IEEE 17th  Conference (International) on Image Processing Proceedings.~--- 
IEEE, 2010. P.~1881--1884.

\bibitem{JS97}
\Au{Johnstone I.\,M., Silverman~B.\,W.} 
Wavelet threshold estimates for data with correlated noise~// J.~Roy. Stat.
Soc.~B, 1997. Vol.~59. P.~319--351.

%\pagebreak
\bibitem{T75}
\Au{Taqqu M.\,S.} Weak convergence to fractional Brownian motion and to the 
Rosenblatt process~// Z.~Wahrscheinlichkeit., 1975. Vol.~31. P.~287--302.

\bibitem{J99}
\Au{Johnstone I.\,M.} 
Wavelet shrinkage for correlated data and inverse problems adaptivity results~// 
Stat. Sinica, 1999. Vol.~9. P.~51--83.



\bibitem{DJ94}
\Au{Donoho D., Johnstone~I.\,M.} 
Ideal spatial adaptation via wavelet shrinkage~// Biometrika, 1994. Vol.~81. No.\,3.
P.~425--455.

\bibitem{E15}
\Au{Ерошенко А.\,А.} 
Статистические свойства оценок сигналов и~изображений при пороговой обработке 
коэффициентов в~вейв\-лет-раз\-ло\-же\-ни\-ях. Дис.\ \ldots\ канд. физ.-мат. наук.~--- 
М.:~МГУ, 2015. 82~с.

\bibitem{B05}
\Au{Bradley R.\,C.} 
Basic properties of strong mixing conditions. A~survey and some open questions~// 
Probab. Surveys, 2005. Vol.~2. P.~107--144.

\bibitem{P96}
\Au{Peligrad M.} On the asymptotic normality of sequences of weak dependent 
random variables~// J.~Theor. Probab., 1996. Vol.~9. No.\,3. P.~703--715.

 \end{thebibliography}

 }
 }

\end{multicols}

\vspace*{-6pt}

\hfill{\small\textit{Поступила в~редакцию 09.10.17}}

\vspace*{6pt}

%\newpage

%\vspace*{-24pt}

\hrule

\vspace*{2pt}

\hrule

\vspace*{-5pt}


\def\tit{UNBIASED RISK ESTIMATE 
OF~STABILIZED HARD THRESHOLDING IN~THE~MODEL WITH~A~LONG-RANGE DEPENDENCE}

\def\titkol{Unbiased risk estimate 
of~stabilized hard thresholding in~the~model with~a~long-range dependence}

\def\aut{O.\,V.~Shestakov$^{1,2}$}

\def\autkol{O.\,V.~Shestakov}

\titel{\tit}{\aut}{\autkol}{\titkol}

\vspace*{-14pt}


\noindent
$^1$Department of Mathematical Statistics, Faculty of Computational Mathematics 
and Cybernetics,\linebreak
$\hphantom{^1}$M.\,V.~Lomonosov Moscow State University, 1-52~Leninskiye Gory, 
GSP-1, Moscow 119991, Russian\linebreak
$\hphantom{^1}$Federation


\noindent
$^2$Institute of Informatics Problems, Federal Research Center 
``Computer Science and Control'' of the Russian\linebreak
$\hphantom{^1}$Academy of Sciences, 
44-2~Vavilov Str., Moscow 119333, Russian Federation


\def\leftfootline{\small{\textbf{\thepage}
\hfill INFORMATIKA I EE PRIMENENIYA~--- INFORMATICS AND
APPLICATIONS\ \ \ 2018\ \ \ volume~12\ \ \ issue\ 2}
}%
 \def\rightfootline{\small{INFORMATIKA I EE PRIMENENIYA~---
INFORMATICS AND APPLICATIONS\ \ \ 2018\ \ \ volume~12\ \ \ issue\ 2
\hfill \textbf{\thepage}}}

\vspace*{2pt}


\Abste{De-noising methods for processing signals and images, based on 
the thresholding of wavelet decomposition coefficients, have become 
popular due to their simplicity, speed, and the ability to adapt to signal 
functions that have a~different degree of regularity at different locations. 
An analysis of inaccuracies of these methods is an important practical task, 
since it makes it possible to evaluate the quality of both the methods 
themselves and the equipment used for processing. The present author 
considers the recently proposed stabilized hard thresholding method which avoids 
the main disadvantages of the popular soft and hard thresholding techniques. 
The statistical properties of this method are studied. In the model with an 
additive Gaussian noise, the author analyzes the unbiased risk estimate. 
Assuming that the noise coefficients have a~long-range dependence, the author 
formulates the conditions under which strong consistency and asymptotic normality 
of the unbiased risk estimate take place. The results obtained make it possible 
to construct asymptotic confidence intervals 
for the threshold processing errors using only observable data.}

\KWE{wavelets; thresholding; unbiased risk estimate; correlated noise; 
asymptotic normality}




\DOI{10.14357/19922264180202}

\vspace*{-18pt}

 
\Ack
\noindent
The work was partly supported by the Russian Foundation for 
Basic Research (project 16-07-00736).


\vspace*{-1pt}

  \begin{multicols}{2}

\renewcommand{\bibname}{\protect\rmfamily References}
%\renewcommand{\bibname}{\large\protect\rm References}

{\small\frenchspacing
 {%\baselineskip=10.8pt
 \addcontentsline{toc}{section}{References}
 \begin{thebibliography}{99}

\bibitem{1-ss}
\Aue{Mallat, S.} 1999. 
\textit{A~wavelet tour of signal processing}. New York, NY: Academic Press. 857~p.

\bibitem{2-ss}
\Aue{Donoho, D., and I.\,M.~Johnstone.} 1995. Adapting to unknown smoothness via 
wavelet shrinkage. \textit{J.~Am. Stat. Assoc.} 90:1200--1224.

\columnbreak

\bibitem{3-ss}
\Aue{Markin, A.\,V.} 2009. Predel'noe raspredelenie otsenki riska pri porogovoy 
obrabotke veyvlet-koeffitsientov [Limit distribution of risk estimate of wavelet 
coefficient thresholding]. \textit{Informatika i~ee Primeneniya~--- Inform. Appl.}
3(4):57--63.

\vspace*{-2pt}

\bibitem{4-ss}
\Aue{Markin, A.\,V., and O.\,V.~Shestakov}. 2010. Consistency of risk estimation 
with thresholding of wavelet coefficients. \textit{Mosc. Univ. Comput. Math. 
Cybern.} 34(1):22--30.

\pagebreak

\bibitem{5-ss}
\Aue{Shestakov, O.\,V.}
 2012. Asymptotic normality of adaptive wavelet thresholding risk estimation. 
 \textit{Dokl. Math.} 86(1):556--558.

\bibitem{6-ss}
\Aue{Eroshenko, A.\,A., and O.\,V.~Shestakov.} 2014. Asimptoticheskie svoystva 
otsenki riska pri porogovoy obrabotke veyvlet-koeffitsientov v~modeli 
s~korrelirovannym shumom
[Asymptotic properties of wavelet thresholding risk estimate in the model 
of data with correlated noise]. \textit{Informatika i~ee Primeneniya~--- Inform. Appl.}
8(1):36--44.

\bibitem{7-ss}
\Aue{Shestakov, O.\,V.} 2016. Almost everywhere convergence of 
a~wavelet thresholding risk estimate in a model with correlated noise. 
\textit{Mosc. Univ. Comput. Math. Cybern.} 40(3):114--117.

\bibitem{8-ss}
\Aue{Huang, H.-C. and T.\,C.\,M.~Lee.} 2010. 
Stabilized thresholding with generalized sure for image denoising. 
\textit{IEEE 17th Conference (International) on Image Processing Proceedings}. 
IEEE. 1881--1884.

\bibitem{9-ss}
\Aue{Johnstone, I.\,M., and B.\,W.~Silverman.} 1997.  
Wavelet threshold estimates for data with correlated noise. 
\textit{J.~Roy. Stat. Soc.~B} 59:319--351.

\bibitem{11-ss}
\Aue{Taqqu, M.\,S.} 1975.  Weak convergence to fractional Brownian motion and to 
the Rosenblatt process. \textit{Z.~Wahrscheinlichkeit.} 31:287--302.

\bibitem{10-ss}
\Aue{Johnstone, I.\,M.} 1999. Wavelet shrinkage for correlated data and inverse 
problems adaptivity results.  \textit{Stat. Sinica} 9:51--83.



\bibitem{12-ss}
\Aue{Donoho, D., and I.\,M.~Johnstone.} 1994. Ideal spatial adaptation via 
wavelet shrinkage. \textit{Biometrika} 81(3):425--455.

\bibitem{13-ss}
\Aue{Eroshenko, A.\,A.} 2015. Statisticheskie svoystva otsenok signalov 
i~izobrazheniy pri porogovoy obrabotke ko\-ef\-fi\-tsi\-en\-tov v~veyvlet-razlozheniyakh
[Statistical properties of signal and 
image estimates under thresholding of coefficients in wavelet 
decompositions].   Moscow: MSU. PhD Diss. 82~p. 

\bibitem{14-ss}
\Aue{Bradley, R.\,C.} 2005. 
Basic properties of strong mixing conditions. A~survey and some open questions. 
\textit{Probab. Surveys}  2:107--144.

\bibitem{15-ss}
\Aue{Peligrad, M.} 1996. On the asymptotic normality of sequences of 
weak dependent random variables. \textit{J.~Theor. Probab.} 9(3):703--715.
\end{thebibliography}

 }
 }

\end{multicols}

\vspace*{-3pt}

\hfill{\small\textit{Received October 9, 2017}}

%\vspace*{-24pt}



\Contrl

\noindent
\textbf{Shestakov Oleg V.} (b.\ 1976)~--- 
Doctor of Sciences in physics and mathematics, associate professor, 
Department of Mathematical Statistics, Faculty of Computational Mathematics 
and Cybernetics, M.\,V.~Lomonosov Moscow State University, 1-52~Leninskiye Gory, 
GSP-1, Moscow 119991, Russian Federation; senior scientist, Institute of 
Informatics Problems, Federal Research Center ``Computer Science and Control''
of the Russian Academy of Sciences, 44-2~Vavilov Str., Moscow 119333, 
Russian Federation; \mbox{oshestakov@cs.msu.su}
\label{end\stat}


\renewcommand{\bibname}{\protect\rm Литература} 