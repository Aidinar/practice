\def\stat{mirzabekov}

\def\tit{ДИСКРЕТНЫЙ АНАЛИЗ В~СИНТАКСИЧЕСКОМ АНАЛИЗЕ}

\def\titkol{Дискретный анализ в~синтаксическом анализе}

\def\aut{Я.\,М.~Мирзабеков$^1$, Ш.\,Б.~Шихиев$^2$}

\def\autkol{Я.\,М.~Мирзабеков, Ш.\,Б.~Шихиев}

\titel{\tit}{\aut}{\autkol}{\titkol}

\index{Мирзабеков Я.\,М.}
\index{Шихиев Ш.\,Б.}
\index{Mirzabekov Ya.\,M.}
\index{Shihiev Sh.\,B.}


%{\renewcommand{\thefootnote}{\fnsymbol{footnote}} \footnotetext[1]
%{Работа поддержана РНФ (проект 16-11-10227).}}


\renewcommand{\thefootnote}{\arabic{footnote}}
\footnotetext[1]{Дагестанский государственный университет, yash831@mail.ru}
\footnotetext[2]{Дагестанский государственный университет, \mbox{sh\_sh\_b51@mail.ru}}

%\vspace*{-8pt}

    
        
      \Abst{Дано неформальное определение синтаксиса в~терминах 
дискретной математики и~тео\-рии графов. Приведен алгоритм выделения семантически 
замкнутых (самодостаточных по значению) фрагментов предложения. Отсутствие формального 
определения понятия семантики слова и~сочетания словоформ затрудняет компьютерную 
обработку текста, в~частности процесс его сегментации. Показано, каким образом можно 
классифицировать выражения по семантическому признаку, используя их синтаксические 
особенности. Классификация выражений по тому, каким вопросам они отвечают, является 
самым доступным и~упрощенным способом группировки выражений по семантическому 
признаку. Описан алгоритм, распознающий указатели места, т.\,е.\ выражения, 
отвечающие на вопрос <<где?>>. На конкретных примерах рас\-смот\-ре\-ны задача анализа 
выражений и~обратная ей задача синтеза~--- прикладная задача распознавания выражений.}
      
      \KW{естественный язык; дискретная математика; теория графов; синтаксис; 
словоформа; морфологический параметр; согласованные определения; несогласованные 
определения; лексика; семантика}

\DOI{10.14357/19922264180214}
  
\vspace*{3pt}


\vskip 10pt plus 9pt minus 6pt

\thispagestyle{headings}

\begin{multicols}{2}

\label{st\stat}
     
\section{Введение}

\vspace*{-3pt}

     Рассматриваемая задача имеет целью предложить новые подходы 
к~решению задачи \textit{синтаксического анализа}, и,~в~част\-ности, 
предлагается алгоритм решения известных в~литературе задач Text Mining по 
\textit{извлечению понятий из текста}~[1]. Предлагаемый в~статье алгоритм 
отличается от известных авторам работ тем, что он опирается на 
\textit{конструктивную тео\-рию синтаксиса}, которая построена как раздел 
дискретного анализа и~в~которой \textit{синтаксис} и~его элементы 
(\textit{словосочетание, выражение, предложение} и~т.\,д.)\ имеют формальные 
определения~[2].
     
     Синтаксический анализ предложения на компьютере давно является 
предметом исследования различных инициативных групп, лабораторий 
и~академических институтов. Исторически так сложилось, что под 
\textit{синтаксическим анализом} предложения подразумевается его 
и~\textit{морфологический}, и~\textit{синтаксический}, и~\textit{семантический} 
анализ (и~словоформ, и~словосочетаний, и~их значений). \textit{Морфология} 
с~\textit{синтаксисом}, с~одной стороны, и~\textit{семантика}, с~другой стороны, 
изучают предметы несовместимой природы~--- мир материальных вещей (слова 
и~их сочетания, план выражения) и~мир пред\-став\-ле\-ний и~мыслей (план 
содержания). Язык есть только \textit{отображение} (\textit{форма}, по 
выражению Ф.~де Соссюра) этих двух \textit{субстанций}. Попытка сходу объять 
их в~рамках одной тео\-рии, а~тем более в~рам\-ках одного алгоритма изначально 
была рискованным предприятием. 
     
     Хорошо известны и~те, кто стоял у истоков <<\textit{синтаксического 
и~глубинного анализов}>> (Н.~Хомский, Д.~Слобин и~др.), и~труды российских 
ученых по созданию \textit{синтаксических анализаторов} предложений русского 
языка. 

В~трудах В.\,А.~Тузова (СПбГУ), Р.\,Г.~Пиотровского (СПбГПУ им.\ 
А.\,И.~Герцена), сотрудников МГУ 
(А.\,С.~Старостин, М.\,Г.~Мальковский и~др.)\ и~РАН (И.\,А.~Мельчук, 
Ю.\,Д.~Апресян и~др.)\ и~особенно в~работах ежегодного 
сборника <<Диалог-21>> отражены достижения и~тенденции развития 
компьютерных наук. А~в~работах А.\,О.~Казенникова, И.\,П.~Кузнецова 
и~Н.\,В.~Сомина, а~так\-же А.\,С.~Старостина, М.\,Г.~Мальковского 
и~Н.\,В.~Арефь\-ева действенно используются методы дискретного анализа при 
построении анализаторов. И~задача <<извлечения из текс\-та понятий>> 
определенного типа~\cite{3-mir} так\-же известна дав\-но и~решается различными 
эвристическими методами.
     
     На особенностях и~име\-ющих\-ся серьезных достижениях в~построении 
синтаксических анализаторов не будем останавливаться по той прос\-той причине, 
что в~данной статье под \textit{синтаксическим анализом} подразумевается нечто 
отличное от традиционного определения этого понятия. 
     
     На примере урезанного (упрощенного) син\-таксиса русского языка 
приводится описание \textit{формаль\-но\-го синтаксиса} (без участия семантики 
в~конструкциях \textit{синтаксиса}). В~рамках \textit{формального синтаксиса} 
формулируется и~решается немало\linebreak\vspace*{-12pt}

\pagebreak

\noindent
 интересных задач линг\-ви\-сти\-ки. На примере 
построения универсального алгоритма решения задачи \textit{извлечения понятий 
из текста} показана одна из возможностей \textit{формального синтаксиса}.
     
     \textit{Формальный синтаксис}~--- понятие диск\-рет\-ной математики. Как 
структура, определяемая лексикой и~набором словосочетаний (упорядоченных 
пар словоформ), она открыта в~том смысле, что может быть дополнена не только 
новыми словами и~словосочетаниями, но и~тем, что \textit{элементы синтаксиса} 
могут быть инкапсулированы семантикой. Последнее обстоятельство дает 
основание и~воз\-мож\-ность для автономного изучения \textit{синтаксиса} без 
услуг семантики и~ощущение лег\-кости формулировок задач и~алгоритмов их 
решения.
     
     Как известно, <<из текстов извлекаются отдельные факты с~по\-мощью 
лексического анализа>>, а~далее, на втором шаге, <<за этим следует 
синтаксический разбор>>~[1]. Под \textit{фактом}, видимо, следует 
подразумевать категорию семантики (например, \textit{понятие} или 
\textit{фрагмент знания}), выраженную одним словом или выражением из 
нескольких словоформ. Когда носителем языка, т.\,е.\ \textit{слова} и~его 
\textit{значения}, является человек, тогда грамматика и~языкознание опираются на 
\textit{соответствие} между \textit{словом} и~его \textit{значением}, имеющимся 
в~сознании человека. В~задачах компьютерной обработки текс\-та разработчики 
со\-от\-вет\-ст\-ву\-ющих алгоритмов, как правило, пытаются в~самом алгоритме 
реализовать свои интуитивные пред\-став\-ле\-ния о~\textit{значении слова}. 
     
     Основной целью статьи является при\-вле\-че\-ние внимания читателя 
к~сле\-ду\-ющим двум методологическим проб\-ле\-мам, име\-ющим место 
в~компьютерной лингвистике, и~путям их решения. 
     \begin{enumerate}[1.]
     \item Возможность замены эвристических алгоритмов на точ\-ные алгоритмы 
решения задач выявления типа элементов синтаксиса (например, является ли 
предложением данная строка символов?). 
     \item  Приемы построения \textit{отображения} между двумя множествами: 
синтаксически правильных выражений и~их значений (поэтому в~статье 
в~качестве первого множества рас\-смат\-ри\-ва\-ет\-ся множество выражений с~явной 
семантикой; если быть точ\-нее, то выражений, отвечающих на вопрос \textit{где?}).
     \end{enumerate}
     
     Не вдаваясь в~тонкости проб\-ле\-мы семантики языка, отметим: 
значение слова~--- это то, что просыпается в~памяти человека при восприятии 
этого слова. А~значение слова, описанное в~толковом (семантическом) словаре, 
есть со\-по\-став\-ле\-ние одному слову некоторой другой группы слов и~их сочетаний.
     
     Даже тот, кто воспринимает слово \textit{халва}, затрудняется сказать, чего 
больше для него в~значении этого слова: ощущения сла\-дости или пяти звуков, 
об\-ра\-зу\-ющих это слово. Непосредственная связь между словом и~его значением 
давно используется в~решении задач лингвистики.

\vspace*{-2pt}
     
\section{Формулировка задачи}

     В \textit{синтаксическом анализе} предложения самым трудоемким является 
процесс определения \textit{морфологических параметров} словоформы. 
Следующей по тру\-до\-ем\-кости процедурой является \textit{сегментация членов} 
предложения. Понятие \textit{сегментации} нуж\-да\-ет\-ся в~уточ\-нении.
     
     Предполагается, что предложение имеет древовидную структуру. 
Например, \textit{пред\-ло\-же\-нию-по\-сле\-до\-ва\-тель\-ности} (словоформ) 

\vspace*{-3pt}

\noindent
     \begin{multline}
     \mbox{\textbf{маленькая дочка соседа пошла на родник}}\\
     \mbox{\textbf{за холодной водой}}
     \label{e1-mir}
     \end{multline}
     

          \noindent
соответствует \textit{пред\-ло\-же\-ние-де\-ре\-во}, пред\-став\-лен\-ное на рис.~1.

  { \begin{center}  %fig1
 \vspace*{9pt}
   \mbox{%
 \epsfxsize=77.957mm 
 \epsfbox{mir-1.eps}
 }


\vspace*{6pt}


\noindent
{{\figurename~1}\ \ \small{Предложение-дерево $T_1$}}
\end{center}
}

\vspace*{9pt}      
          
     Каждая ветвь \textit{корневого дерева}~\cite{2-mir, 4-mir} на рис.~1 
(обозначим его через~$T_1$) также является корневым деревом. Например, ветвь 
с~кор\-нем в~вершине <<\textbf{пошла}>>~--- ветвь ска\-зу\-емо\-го из 
предложения~(\ref{e1-mir})~--- показана на рис.~2. У~дерева несколько вет\-вей, 
вершины вет\-ви пред\-ло\-же\-ния-де\-ре\-ва образуют \textit{сегмент} в~этом 
дереве.
     



       { \begin{center}  %fig2
       \vspace*{9pt}
       
  \mbox{%
 \epsfxsize=47.652mm 
 \epsfbox{mir-2.eps}
 }


\vspace*{6pt}


\noindent
{{\figurename~2}\ \ \small{Ветвь <<\textbf{пошла}>> дерева~$T_1$}}
\end{center}
}


\vspace*{9pt}

\setcounter{figure}{2}
    
     
     Запишем предложение-де\-ре\-во~$T_1$ в~скобочном виде:
     
     \vspace*{-3pt}

\noindent
     \begin{multline}
     \mbox{\textbf{дочка~(маленькая,\ соседа,}}\\
     \mbox{\textbf{пошла~(на\ родник,\ за\ 
водой~(холодной)))}}\,.
     \label{e2-mir}
     \end{multline}
      
     Трансформация двумерной структуры~(\ref{e2-mir}) в~линейную 
цепь~(\ref{e1-mir}) интуитивно выполняется нашей языковой спо\-соб\-ностью. 
В~конструктивной теории\linebreak\vspace*{-12pt}

\pagebreak

\noindent
 языка, в~част\-ности в~компьютерной лингвистике, этот 
процесс должен быть описан таким образом, чтобы иметь воз\-мож\-ность найти 
соответствие между сегментами пред\-ло\-же\-ния-де\-ре\-ва~(\ref{e2-mir}) 
и~\textit{фрагментами}  
пред\-ло\-же\-ния-по\-сле\-до\-ва\-тель\-ности~(\ref{e1-mir}). 
     
     \textit{Фрагментом} в~последовательности~(\ref{e1-mir}) называется любой 
ее диа\-па\-зон~--- непрерывная часть цепочки словоформ~(\ref{e1-mir}). 
     
     \textit{Фрагментация сегментов}, т.\,е.\ трансформация  
\textit{пред\-ло\-же\-ния-де\-ре\-ва}  
в~\textit{пред\-ло\-же\-ние-по\-сле\-до\-ва\-тель\-ность} таким образом, чтобы 
каж\-дый сегмент первого был фрагментом во втором,~--- принципиальной 
важ\-ности задача в~анализе предложения. Решается эта задача несложно~--- 
простым односторонним обходом дерева~\cite{4-mir}. Однако правила 
трансформации, которыми пользуется наша языковая спо\-соб\-ность, час\-тень\-ко 
отступают от пути обхода дерева. Например, левосторонний обход дерева~$T_1$ 
таков:

\vspace*{-6pt}

\noindent
     \begin{multline*}
     \mbox{\textbf{дочка\ маленькая\ соседа\ пошла}}\\
     \mbox{\textbf{за\ водой\ холодной\ на\ 
родник}}
     %\label{e3-mir}
     \end{multline*}
     
\vspace*{-6pt}
     
     Рассматриваемая задача имеет непосредственное отношение к~правилам 
трансформации вершин пред\-ло\-же\-ния-де\-ре\-ва во фрагменты 
пред\-ло\-же\-ния-по\-сле\-до\-ва\-тель\-ности.
     
     Пусть $(v, w)$~--- некоторая дуга в~пред\-ло\-же\-нии-де\-ре\-ве. Известно, что 
слово~$w$ доопределяет (уточняет) значение слова~$v$. Следовательно, и~вся 
ветвь с~корнем в~$w$ присутствует в~предложении исключительно для уточнения 
определенного семантического признака слова~$v$. Признаком слова~$v$ может 
быть и~\textit{вес}, и~\textit{цвет}, и~\textit{местонахождение} и~т.\,д.\ предмета 
с~именем~$v$. Такие сегменты назовем \textit{описателями признаков} вещи 
с~именем~$v$, или \textit{описателями семантических} признаков слова~$v$. 
     
     Также известно, что \textit{описатель семантического} признака всегда 
образует фрагмент в~предложении. Например, в~предложении <<\textbf{Фермер 
отправил} (\textbf{на пять километров двести метров}) (\textbf{в~двух 
контейнерах и~в~семи ящиках}) (\textbf{три тонны пятьсот килограмм}) 
\textbf{зерна}>> в~круглых скобках выделены три выражения, каж\-дое из которых 
описывает один из трех признаков (глагола \textit{отправил} и~имени 
\textit{зерна}). При любом нарушении фрагментарности этих сегментов их 
значения изменятся. 
     
     Используя это свойство предложения, можно разработать алгоритмы для 
выделения в~предложении отдельных его фрагментов~--- описателей семантических 
признаков имен существительных (ИС) и~глаголов, что существенно упростит 
анализ предложения. (Напомним, что не\-уве\-рен\-ность во \textit{фрагментарности 
сегментов} существенно услож\-ня\-ет работу синтаксических анализаторов.) 
     
     Собирать дерево из нескольких его ветвей намного проще, нежели из 
мно\-го\-чис\-лен\-ных его вер-\linebreak\vspace*{-12pt}

\columnbreak

\noindent
шин. (Напомним также, анализ предложения означает: по 
заданному пред\-ло\-же\-нию-по\-сле\-до\-ва\-тель\-ности построить 
со\-от\-вет\-ст\-ву\-ющее ему пред\-ло\-же\-ние-де\-ре\-во~\cite{4-mir}.) 
     
     Для решения поставленной задачи необходимы некоторые определения 
и~обозначения из синтаксиса естественного языка (ЕЯ). В~данном случае в~качестве 
синтаксиса используется синтаксис русского языка.

\vspace*{-6pt}
     
\section{Синтаксис}

     Задачи компьютерной лингвистики предполагают компьютерный 
морфологический анализ словоформ и~синтаксический анализ предложения на 
\textit{естественном языке}. Морфология и~синтаксис письменного языка 
давно рас\-смат\-ри\-ва\-ют\-ся разработчиками синтаксических анализаторов 
и~про\-грамм-пе\-ре\-вод\-чи\-ков как дискретные сис\-те\-мы с~заданными множествами 
элементов и~операций над ними. 
      
     Компьютерный синтаксический словарь (вернее было бы называть его 
морфологическим словарем) является обязательной \mbox{частью} базы данных\linebreak 
в~синтаксических анализаторах. У~этих словарей такое же предназначение, как 
у~обычных словообразовательных словарей: анализ (распознавание) и~синтез 
(построение) словоформ. Построение компьютерного словаря, разработка 
алгоритмов анализа и~синтеза словоформ по данному словарю является 
при\-клад\-ной задачей дискретного анализа~\cite{2-mir}. 
     
     Под анализом словоформы, например <<\textit{ящику}>>, под\-ра\-зу\-ме\-ва\-ет\-ся 
поиск исходной формы этого слова и~\textit{морфологических параметров} его: 
<<\textit{ящик 01~11~21~33}>>, где 01~--- код ИС; 
11~--- код единственного чис\-ла; 33~--- код дательного падежа.
     
     Далее будем пользоваться следующей кодировкой час\-тей речи 
и~морфологических категорий: 01~--- ИС; 02~--- ИП (полная форма имени 
прилагательного); 10, 20 и~30~--- \textit{категории рода}, \textit{числа} 
и~\textit{падежа} соответственно. Значения категорий так\-же пронумерованы:

\vspace*{-3pt}

\noindent
     \begin{multline*}
\mbox{род} = (\mbox{мужской,\ средний,\ женский}) ={}\!\\
    {}= 10 = 
(11, 12, 13)\,; 
\end{multline*}

\vspace*{-13pt}

\noindent
\begin{multline*}
     \mbox{число} = (\mbox{единственное,\ множественное}) = {}\\
     {}=20=  (21, 22)\,; 
    \end{multline*}
    
    \vspace*{-13pt}

\noindent
    \begin{multline*}
     \mbox{падеж} = (\mbox{именительный},
    \mbox{родительный},\\ 
    \mbox{дательный,\ 
винительный,\ творительный},\\
 \mbox{предложный}) = 30 = (31, 32, 33, 34, 35, 36)\,.
 \end{multline*}
     

     
     Нетрудно заметить, что значения категорий суть унарные операции, 
определенные на ИС и~ИП. Действительно,

\pagebreak

\noindent
     \begin{align*}
22(\mbox{\textbf{большой}}) &= \mbox{\textbf{большие}};\\ 
35(\mbox{\textbf{большой}})& = \mbox{\textbf{большим}};\\ 
33(\mbox{\textbf{ящик}}) &= \mbox{\textbf{ящику}}.
\end{align*}


     
     \noindent
     Композиции этих операций образуют новые операции. Например:
     \begin{align*}
     22\cdot35(\mbox{\textbf{большой}}) &= \mbox{\textbf{большими}};\\ 
22\cdot33(\mbox{\textbf{ящик}}) &= \mbox{\textbf{ящикам}}.
     \end{align*}
     

     
     Таким образом, на ИС и~ИП определены по $3\cdot2\cdot6 \hm= 36$ операций, 
в~виде композиции трех операций типа~10, 20 и~30. (Операции~11, 12 и~13 не 
меняют ИС.) Каж\-дая из них однозначно определяет новое значение своего 
операнда. Например:
     \begin{align*}
     13\cdot21\cdot34(\mbox{\textbf{большие}}) &= \mbox{\textbf{большую}};\\ 
13\cdot22\cdot35(\mbox{\textbf{ящик}})& = \mbox{\textbf{ящиками}}.
     \end{align*}
     
     Каждая из этих операций \textit{образует} (строит, синтезирует) одну 
определенную словоформу.
     
     Каждая из этих операций имеет обратную себе операцию: по заданной 
словоформе~$w$ определяется ее исходная форма~$w_1$ и~операция~$f$ такая, что 
$f(w_1) \hm= w$. Например, для $w \hm= \mbox{\textbf{ящиками}}$ имеем $w_1 
\hm= \mbox{\textbf{ящик}}$, $f \hm= 13\cdot22\cdot35$.
     
     Действие, обратное \textit{синтезу} (построению) словоформы, называется 
\textit{анализом} (распознаванием) словоформы.
     
     Если $f(w_1) = w$, то три операции, обра\-зу\-ющие~$f$, называются 
\textit{морфологическими параметрами} словоформы~$w$. Например, для 
словоформы \textit{ящиками} морфологическими па\-ра\-мет\-ра\-ми являются~13, 22 
и~35. Иногда (для убедительности) за словоформой пишутся ее 
\textit{па\-ра\-мет\-ры}. Например, \textbf{ящиками}~13~22~35.
     
     Область значений операции~$f$ называется \textit{лексической 
группой}~$f$. Например, 01~13~22~35~--- под\-мно\-же\-ст\-во ИС в~форме~13~22~35; 
ИС делится на~12~таких лексических групп. 
     
     Над лексическими группами можно выполнить множественные операции, 
например операцию объединения: 
01~13~20~35\;=\;01~13~21~35\;$\cup$\;01~13~22~35, пред\-став\-ля\-ющую собой 
множество ИС женского рода в~творительном падеже в~единственном 
и~множественном чис\-ле, так как $20 =\hm (21, 22)$. 
     
     Аналогично ИП делятся на~36~лексических групп. 
     
     Введенные обозначения 
можно использовать для формирования более сложных обозначений (формул), 
порождающих определенные группы элементов синтаксиса. Например, 
минимальная неделимая единица синтаксиса~--- \textit{словосоче\-тание}. 
Примером словосочетания является \textit{согла\-со\-ван\-ное определение} 
\textbf{ящиками нижними} или пара\linebreak (\textit{ящиками, нижними}), которая, 
в~свою очередь, является элементом прямого произведения 
$(01~13~22~35)\cdot(02~13~22~35)$. Таких сочетаний ИС с~ИП в~русском языке будут 
десятки тысяч. Резонно обозначить множество этих словосочетаний через 
(01~13~22~35)$\cdot$(02~13~22~35) и~называть его кодом \textit{синтаксического 
отношения} (СО) между группами 01132235 и~02132235. Код лексической группы 
начинается с~нуля, пробелы меж\-ду кодами категорий и~знак~<<$\cdot$>> 
необязательны: 0113223502132235. 
     
     Обозначив значения категорий буквами, получим все~36~СО: 
     \begin{equation}
     01~1x~2y~3z~02~1x~2y~3z\,,
          \label{e4-mir}
          \end{equation}
где  
$x= 1\ldots 3$; $y = 1\ldots 2$;  $z = 1\ldots 6$,
порождающих \textit{согласованные определения}, име\-ющие место 
в~русском языке. 
     
     Элементы множеств~(\ref{e4-mir}) обозначим через СО1. 
Все~36~записей~(\ref{e4-mir}) упорядочим лексикографически по возрастанию 
и~обозначим через СО1001, СО1002, \ldots, СО1036.
     
     Аналогично $3\cdot2\cdot6\cdot3\cdot2 = 216$~пар 
     \begin{equation}
     01~1x~2y~3z~01~1r~2t~32\,,
          \label{e5-mir}
          \end{equation}
где $x, r = 1\ldots 3$; $y, t = 1\ldots 2$;  $z = 1\ldots 6$,
     порождают словосочетания (\textit{несогласованные определения} или СО2) 
вида: \textbf{ящиками фермеров}, \textbf{книгу биб\-лио\-те\-ки}, \textbf{заботы 
врачей} и~т.\,д.
     
     В~(\ref{e5-mir}) присутствуют пять па\-ра\-мет\-ров. Согласно грамматике 
русского языка все первые пять категорий, кроме последней категории~<<32>>, 
могут принимать всевозможные значения независимо друг от друга. Объединение 
всех~216~пар~(\ref{e5-mir}) удобно писать в~виде: 
     \begin{equation}
     01~10~20~30~01~10~20~32\,.
     \label{e6-mir}
     \end{equation}
     
     Заменив в~(\ref{e6-mir}) нули их допустимыми значениями и~упорядочив 
полученные записи в~лексикографическом порядке возрастания, 
получим~216~записей с~кодами СО2001, СО2002, \ldots, СО2216.
     
     В русском языке более~50~тыс.\ ИС и~ИП. Словосочетаний из 
\textit{согласованных определений} будет около миллиона. А~словосочетаний из 
\textit{согласованных и~несогласованных определений} окажется десятки 
миллионов. На ЕЯ самым распространенным 
и~мно\-го\-чис\-лен\-ным видом выражений являются сочетания из ИС и~ИП, 
образованных по правилам~СО1 и~СО2. 
     
     Правила СО1 позволяют привязывать к~ИС несколько ИП, например:
         \begin{equation}
     \mbox{\textbf{хороший\ высокий\ умный\ студент.}}
     \label{e7-mir}
     \end{equation}
     
                 
          \noindent
     (Знаков препинания в~рас\-смат\-ри\-ва\-емом здесь синтаксисе нет.) 
Выражению~(\ref{e7-mir}) соответствует дерево в~форме звезды (рис.~3).

  { \begin{center}  %fig3
 \vspace*{1pt}
  \mbox{%
 \epsfxsize=77.957mm 
 \epsfbox{mir-3.eps}
 }


\vspace*{6pt}


\noindent
{{\figurename~3}\ \ \small{Предложение-дерево. Звезда}}
\end{center}
}

\vspace*{9pt}      




 Или в~ско\-боч\-ной записи:
\begin{equation}
\mbox{\textbf{студент\ (хороший,\ высокий,\ умный)}}\,.
\label{e8-mir}
\end{equation}  
     
     \noindent
\textit{Шириной} выражения с~одним ИС называется чис\-ло ИП в~этом выражении. 
Выражение~(\ref{e8-mir}) имеет ширину три.
     
     
     

     По правилам СО2 образуется линейная цепь связанных ИС, например:
          \begin{equation}
     \mbox{\textbf{отряд\ студентов\ школы\ мужества}}\,.           
     \label{e9-mir}
     \end{equation}
     

     
     Выражению~(\ref{e9-mir}) соответствует цепь (которая на восточных 
языках называется \textit{идафной цепью}; \textit{ид$\acute{\mbox{а}}$фа} 
(\textit{араб.}\hspace*{-1pt}{\raisebox{-4pt}{
\epsfxsize=9mm %8.981mm 
\epsfbox{mir-t.eps}
}}
 \textit{дополнение})~--- способ оформления 
несогласованного определения в~арабском и~ряде других восточных языков), 
пред\-став\-лен\-ная на рис.~4.

  { \begin{center}  %fig4
 \vspace*{9pt}
  \mbox{%
 \epsfxsize=78.036mm 
 \epsfbox{mir-4.eps}
 }


\vspace*{6pt}


\noindent
{{\figurename~4}\ \ \small{Предложение-дерево. Цепь}}
\end{center}
}

\vspace*{9pt} 


     
Или в~скобочной записи: 
\begin{equation*}
     \mbox{\textbf{отряд\ (студентов\ (школы\ (мужества)))}}\,.
     %\label{e10-mir}
     \end{equation*}
     
     
     \textit{Длиной} идафной цепи называется число ИС в~этой цепи.
     
     Далее для определенности и~обозримости выражений вводятся ограничения 
на ширину \textit{согласованного определения}~--- она не более одного, и~на 
длину \textit{несогласованного определения}~--- она не более трех.
     
     В грамматике русского языка кроме~12~\textit{беспредложных форм} ИС 
используются еще \textit{предложные формы} ИС. Что\-бы разнообразить 
рас\-смат\-ри\-ва\-емый синтаксис, в~лексику из ИС и~ИП добавляются сле\-ду\-ющие 
предлоги: \textit{над}, \textit{под}, \textit{на}, \textit{в}~--- и~соответствующие им 
\textit{пред\-лож\-ные формы} ИС: первые два предлога управ\-ля\-ют творительным 
падежом, а~последние два~--- предложным.
     
     Если предположить, что эти предлоги имеют коды~21, 22, 23 и~34, то коды 
\textit{предложных форм}, со\-от\-вет\-ст\-ву\-ющих этим предлогам, будут такими: 
01~10~20~35~21, 01~10~20~35~22, 01~10~20~36~23 и~01~10~20~36~24. 
Например:
  \begin{align*}
     0110~20~35~21(\mbox{\textbf{отряд}}) &= \mbox{\textbf{над\ отрядом}};\\ 
01~10~20~36~24(\mbox{\textbf{отряд}}) &= \mbox{\textbf{в~отряде}}\,.
\end{align*}

     
     Обозначим через~$S$ множество ИС и~ИП в~лексике русского языка и~их 
формы, о которых было сказано выше. Обозначим через~$Q$ множество 
\textit{согласованных} и~\textit{несогласованных определений}, определенных 
на~$S$.
     
     Пара ($S, Q$) представляет собой ориентированный граф, обозначим его 
через~$\mathrm{Sint}$. Имеются сле\-ду\-ющие основания называть этот граф 
\textit{синтаксисом}. 
     
     Представим себе \textit{корневое дерево} в~этом графе: 
     \begin{itemize}
\item[(а)] у которого корнем является ИС; 
\item[(б)] каждое ИС в~нем имеет ширину не более одного; 
\item[(в)] идафная цепь в~нем имеет длину не более трех.
\end{itemize}

     Деревья, удовлетворяющие требованиям~(а)--(в), назовем 
\textit{выражениями} в~синтаксисе $\mathrm{Sint}$. (Требова\-ния~(б) и~(в) здесь 
присутствуют для на\-гляд\-ности деревьев и~\textit{выражений}.) Примерами 
\textit{выражений} являются:
\textbf{дом}; \textbf{под\ домом}; \textbf{дом\ белый}; \textbf{под\ домом\ белым};
\textbf{в\ доме\ соседа}; \textbf{в\ доме\ высоком\ соседа\ хорошего\ врача\ зубного}.
        
              
     Последнее выражение из шести слов является самым длинным в~синтаксисе 
$\mathrm{Sint}$. (Переставляя ИС и~следующее за ним ИП, получим привычные для 
русского языка выражения.)
     
     Данное определение \textit{формального синтаксиса} никак не зависит от 
объема лексики и~числа СО, на которых построен \textit{синтаксис}. 
     
     Этот синтаксис не обременен семантикой. Вопрос инкапсуляции синтаксиса 
семантикой здесь не рассматривается. Хотя известна сила синтаксиса вдохнуть 
смысл даже в~то, как <<Глокая куздра штеко будланула бокра и~курдячит 
бокренка>>. Поэтому выражения (деревья) в~синтаксисе $\mathrm{Sint}$, начинающиеся 
с~\textit{предложной формы} имен, обозначающих предметы, будут отвечать на 
вопрос <<\textit{где?>>}, и~назовем их \textit{указателями места}. 
     
      Элементами синтаксиса $\mathrm{Sint}$ являются обычные (синтаксически 
правильно по\-стро\-ен\-ные) выражения на русском языке, и~задача заключается 
в~поиске \textit{указателей места} в~этом синтаксисе.
     
     Алгоритм извлечения \textit{указателей места} из заданного текста очень 
прост. Имеет смысл рассмотреть работу алгоритма по шагам на прос\-том примере:
     \begin{equation}
         \hspace*{-2.5mm}\mbox{\textbf{в\ высоком\ доме\ хорошего\ соседа\ зубного\ 
врача}}.\!\!
         \label{e11-mir}
         \end{equation}
         
         
         \noindent
         \begin{description}
\item[Шаг~1.] Выделяются словоформы в~(\ref{e11-mir}) и~определяются 
коды лексических групп, к~которым они принадлежат:
\begin{alignat*}{2}
     \mbox{\textbf{в\ доме}}\mbox{~---}&\ 01213624; 
&\enskip \mbox{\textbf{высоком}}\mbox{~---}&\  022136;\\
     \mbox{\textbf{соседа}}\mbox{~---}&\ 012132;  &\enskip 
\mbox{\textbf{хорошего}}\mbox{~---}&\ 022132;\\
     \mbox{\textbf{врача}}\mbox{~---}&\  012132; &\enskip 
\mbox{\textbf{зубного}}\mbox{~---}&\ 022132.
     \end{alignat*}

     

     
    \item[Шаг~2.] Коды лексических групп позволяют группировать попарно 
словоформы из~(\ref{e11-mir}) таким образом, чтобы они были отношениями 
\textit{согласованного} и~\textit{несогласованного определений}; полученные таким 
образом пары словоформ образуют дерево, представленное на рис.~5.
\end{description}

{ \begin{center}  %fig5
 \vspace*{9pt}
  \mbox{%
 \epsfxsize=68.512mm 
 \epsfbox{mir-5.eps}
 }


\end{center}


\noindent
{{\figurename~5}\ \ \small{Дерево, порожденное синтаксисом $\mathrm{Sint}$ на словоформах~(\ref{e11-mir}}}
}

\vspace*{9pt}      

     
     
     
 \begin{description}    
     \item[Шаг~3.] После замены словоформ в~дереве с~рис.~5 на коды лексических 
групп, к~которым они принадлежат, получится дерево, показанное на рис.~6.
     

     
     Дерево на рис.~6 называется \textit{синтаксической формой} (СФ) дерева 
с~рис.~5, а~следовательно, и~выражения~(\ref{e11-mir}). Она является СФ для 
большого чис\-ла выражений; действительно, заменяя в~этой СФ код каждой 
лексической группы некоторой словоформой из этой же группы, можно получить 
новые выражения. Так что СФ~--- \textit{правило}, порождающее выражения 
определенной формы. Из сказанного следует шаг~4. (Сформулирован он в~форме 
утверж\-де\-ния.) 
\end{description}

{ \begin{center}  %fig6
 \vspace*{9pt}
  \mbox{%
 \epsfxsize=68.512mm 
 \epsfbox{mir-6.eps}
 }


\end{center}


\noindent
{{\figurename~6}\ \ \small{Синтаксическая форма дерева, представленного на рис.~5}}
}
    
    \vspace*{6pt}
     
     
\begin{description}
     \item[Шаг~4.] При\-над\-леж\-ность выражения~(\ref{e11-mir}) к~СФ с~рис.~6 
указывает на то, что~(\ref{e11-mir}) есть \textit{указатель места}. 
\end{description}
     
     Рассмотрим еще несколько примеров \textit{указателей места} из шес\-ти 
слов. Все они имеют одинаковую структуру:
\textbf{в~большом ауле нижнего района нашей 
области}; \textbf{под нижним ящиком письменного стола
большого  начальника};
\textbf{на длинном хвосте серого слона бедного индуса}.

      

     
     Анализируя эти выражения по тому же алгоритму из четырех шагов, можно 
убедиться в~том, что и~они~--- \textit{указатели места}. 
     
     Обобщая СФ всевозможных \textit{указателей места}, нетрудно заметить, 
что СФ, по\-рож\-да\-ющие \textit{указатели места}, имеют форму дерева, 
пред\-став\-лен\-но\-го на рис.~7.

{ \begin{center}  %fig7
 \vspace*{9pt}
  \mbox{%
 \epsfxsize=68.512mm 
 \epsfbox{mir-7.eps}
 }


\end{center}


\noindent
{{\figurename~7}\ \ \small{Формы указателей места, объединенные в~дереве $T_{\mathrm{Space}}$}}
}
    
    \vspace*{12pt}

    
     
     Корню этого дерева, обозначенного $T_{\mathrm{Space}}$, соответствует код 
предложной формы ИС $011x2y3z\,\mathrm{MN}$, где $\mathrm{MN}$~--- код предлога, управ\-ля\-юще\-го 
падежной формой~$3z$.
     
     Также нетрудно заметить, что любое поддерево дерева $T_{\mathrm{Space}}$ с~корнем 
в~$011x2y3z\,\mathrm{MN}$ есть СФ, порождающая \textit{указатели места} в~синтаксисе 
$\mathrm{Sint}$. 
     
     Более того, эти СФ порождают все\-воз\-мож\-ные \textit{указатели места} 
в~синтаксисе $\mathrm{Sint}$. 
     
\section{Заключение}

Авторы оперировали синтаксисом $\mathrm{Sint}$~--- вырезкой из синтаксиса 
русского языка~--- для наглядной доступности задачи и~алгоритма ее решения. 
Задача и~алгоритм ее решения не претерпят существенных изменений, если их 
сформулировать для всего синтаксиса русского языка. Изменятся размеры 
словаря и~синтаксиса (графа $\mathrm{Sint}$). Для быстрого поиска слов в~словаре 
(в~упорядоченном массиве строк) можно использовать известные ал\-го\-ритмы.
     
     Приведенное выше описание алгоритма выявления \textit{указателей 
места} в~тексте, написанном в~синтаксисе $\mathrm{Sint}$, указывает на прос\-то\-ту 
и~корректность ал\-го\-рит\-ма. Листинг программы, реализующей этот алгоритм, 
занимает не более одной страницы. Линейная слож\-ность алгоритма обеспечивает 
ему мгновенное исполнение на компь\-ютере.
{\looseness=1

}
     
     Программа, анализирующая предложения в~упрощенном синтаксисе 
и~с~лексикой из~16~слов (синтаксический анализатор), под\-го\-тов\-ле\-на авторами 
и~реализована в~IDE Delphi~7.
     
    {\small\frenchspacing
 {%\baselineskip=10.8pt
 \addcontentsline{toc}{section}{References}
 \begin{thebibliography}{9}
\bibitem{1-mir}
\Au{Барсегян А.\,А., Куприянов~М.\,С., Степаненко~В.\,В., Холод~И.\,И.} Технологии анализа 
данных: Data Mining, Visual Mining, Text Mining, OLAP.~--- СПб.: БХВ-Петербург, 2008. 384~с.
\bibitem{2-mir}
\Au{Шихиев Ф.\,Ш.} Формализация и~сетевая формулировка задачи синтаксического анализа. 
Дис.\ \ldots\ канд. физ.-мат. наук.~--- СПб.: СпбГУ, 2006. 171~с.
\bibitem{3-mir}
\Au{Рубашкин В.\,Ш., Чуприн~Б.\,Ю.} Распознавание количественной информации  
в~ЕЯ-текс\-тах~// Компьютерная лингвистика и~интеллектуальные технологии: Тр. Междунар. 
конф. <<Диалог-2006>>.~--- М.: РГГУ, 2006. С.~456--467.
\bibitem{4-mir}
\Au{Харари~Ф.} Теория графов~/ Пер. с~англ.~--- М.: Едиториал УРСС, 2003. 296~с.
(\Au{Harary~F.} {Graph theory}.~--- Boulder, CO, USA: Westview Press, 1994. 284~p.)
 \end{thebibliography}

 }
 }

\end{multicols}

\vspace*{-6pt}

\hfill{\small\textit{Поступила в~редакцию 13.04.17}}

\vspace*{6pt}

%\newpage

%\vspace*{-24pt}

\hrule

\vspace*{2pt}

\hrule

%\vspace*{8pt}


\def\tit{DISCRETE ANALYSIS IN~PARSING}

\def\titkol{Discrete analysis in~parsing}

\def\aut{Ya.\,M.~Mirzabekov and Sh.\,B.~Shihiev}

\def\autkol{Ya.\,M.~Mirzabekov and Sh.\,B.~Shihiev}

\titel{\tit}{\aut}{\autkol}{\titkol}

\vspace*{-9pt}


 \noindent
Dagestan State University, 43-a~Gadzhiyev Str., Makhachkala 367000, Republic 
of Dagestan, Russian Federation


\def\leftfootline{\small{\textbf{\thepage}
\hfill INFORMATIKA I EE PRIMENENIYA~--- INFORMATICS AND
APPLICATIONS\ \ \ 2018\ \ \ volume~12\ \ \ issue\ 2}
}%
 \def\rightfootline{\small{INFORMATIKA I EE PRIMENENIYA~---
INFORMATICS AND APPLICATIONS\ \ \ 2018\ \ \ volume~12\ \ \ issue\ 2
\hfill \textbf{\thepage}}}

\vspace*{3pt}
      
     
     
     
     \Abste{An informal definition 
     of syntax is given in terms of discrete mathematics and graph theory. 
     The main difficulty for numerous attempts to formalize a~natural language 
     is the semantics of the language. It is shown how it is possible to classify 
     expressions on a~semantic basis, using their syntactic features. 
     Classification of expressions by the kind of questions they answer is the 
     simplest way of grouping expressions on the semantic basis. 
The present authors describe an algorithm that recognizes 
     place pointers, that is, expressions  which answer the question ``where?''. 
     On specific examples, the problem of analysis of expressions and the inverse 
     problem of synthesis, more precisely, 
     the applied problem of recognition of expressions are considered.}
     
     \KWE{natural language; discrete mathematics; graph theory; syntax; 
     word forms; morphological parameters; consistent definitions; 
     inconsistent definitions; vocabulary; semantics}

\DOI{10.14357/19922264180214} %

%\vspace*{-14pt}

 % \Ack
  % \noindent
 



\vspace*{3pt}

  \begin{multicols}{2}

\renewcommand{\bibname}{\protect\rmfamily References}
%\renewcommand{\bibname}{\large\protect\rm References}

{\small\frenchspacing
 {%\baselineskip=10.8pt
 \addcontentsline{toc}{section}{References}
 \begin{thebibliography}{9}
     
    
\bibitem{1-mir-1}
\Aue{Barsegyan, A.\,A., M.\,S.~Kupriyanov, V.\,V.~Stepanenko, and I.\,I.~Cholod}. 
2007. \textit{Tekhnologiya analiza dannykh: Data Mining, Visual Mining, Text 
Mining, OLAP} [Technology of data analysis: Data Mining, Visual Mining, Text 
Mining, OLAP]. St.\ Petersburg: BHV. 384~p.
\bibitem{2-mir-1}
\Aue{Shikhiev, F.\,Sh.} 2006. Formalizatsiya i~setevaya formulirovka zadachi 
sintaksicheskogo analiza [Formalization and network formulation of the task of 
parsing].  
St.\ Petersburg: SpbGU. PhD Diss. 171~p.
\bibitem{3-mir-1}
\Aue{Rubashkin, V.\,Sh., and B.\,Y.~Chuprin.} 2006. Ras\-po\-zna\-va\-nie kolichestvennoy 
informatsii v~EYa-tekstakh [Quantitative data recognition at NLP]. Computational 
Linguistics and Intellectual Technologies: Conference (International) ``Dialogue-2006'' 
Proceedings. Moscow: RGGU. 456--467.
\bibitem{4-mir-1}
\Aue{Harary, F.} 1994. \textit{Graph theory}. Boulder, CO: Westview Press. 284~p.
\end{thebibliography}

 }
 }

\end{multicols}

\vspace*{-3pt}

\hfill{\small\textit{Received April 13, 2017}}

%\vspace*{-24pt}


      \Contr
      
      \noindent
     \textbf{Mirzabekov Yahya M.} (b.\ 1983)~--- senior lecturer, Dagestan State 
University, 43-a Gadzhiyev Str., Makhachkala 367000, Republic of Dagestan, Russian 
Federation; \mbox{yash831@mail.ru}
      
      \vspace*{3pt}
      
      \noindent
      \textbf{Shihiev Shukur B.} (b.\ 1951)~--- Candidate of Sciences (PhD) in physics 
and mathematics, associate professor, Dagestan State University, 43-a~Gadzhiyev Str., 
Makhachkala 367000, Republic of Dagestan, Russian Federation; 
\mbox{sh\_sh\_b51@mail.ru}
\label{end\stat}


\renewcommand{\bibname}{\protect\rm Литература} 