\def\stat{ogaltsov}

\def\tit{АВТОМАТИЧЕСКОЕ ИЗВЛЕЧЕНИЕ МЕТАДАННЫХ\\ ИЗ НАУЧНЫХ PDF-ДОКУМЕНТОВ$^*$}

\def\titkol{Автоматическое извлечение метаданных из научных PDF-документов}

\def\aut{А.\,В.~Огальцов$^{1}$, О.\,Ю. Бахтеев$^{2}$}

\def\autkol{А.\,В.~Огальцов, О.\,Ю. Бахтеев}

\titel{\tit}{\aut}{\autkol}{\titkol}

\index{Огальцов А.\,В.}
\index{Бахтеев О.\,Ю.}
\index{Ogaltsov A.\,V.}
\index{Bakhteev O.\,Y.}




{\renewcommand{\thefootnote}{\fnsymbol{footnote}} \footnotetext[1]
{Работа выполнена при поддержке РФФИ (проект 18-07-01441).}}


\renewcommand{\thefootnote}{\arabic{footnote}}
\footnotetext[1]{Высшая школа экономики; ЗАО <<Антиплагиат>>, 
\mbox{ogaltsov@ap-team.ru}}
\footnotetext[2]{Московский физико-технический институт; 
ЗАО <<Антиплагиат>>, \mbox{bahteev@ap-team.ru}}

%\vspace*{-6pt}



\Abst{Исследуется извлечение метаданных документа. 
Рас\-смат\-ри\-ва\-ют\-ся научные PDF-до\-ку\-мен\-ты на русском языке. 
Особенностью формата PDF 
является раз\-но\-об\-ра\-зие рас\-по\-ло\-же\-ния текста на 
страницах документа. Это создает 
труд\-ности для автоматического извлечения метаданных. Предложенный метод 
извлечения метаданных основан на рас\-смот\-ре\-нии текстовых блоков, полученных при 
помощи PDF-пар\-се\-ра, как объектов в~задаче машинного обучения. Признаковое 
пространство содержит не только текс\-то\-вые признаки, но и~признаки, связанные 
с~форматированием и~расположением блока, которые получены из PDF-пар\-се\-ра. В~работе 
измерено качество классификации предложенного алгоритма и~проведено срав\-не\-ние 
с~базовым алгоритмом.}


\KW{извлечение метаданных; обработка естественного языка; признаки форматирования; 
извлечение информации; метаописания}

\DOI{10.14357/19922264180211}
  
\vspace*{2pt}


\vskip 10pt plus 9pt minus 6pt

\thispagestyle{headings}

\begin{multicols}{2}

\label{st\stat}


\section{Введение}

\vspace*{-2pt}

Метаданные~--- это информация об объекте, рас\-по\-ло\-жен\-ном в~ка\-ком-ли\-бо репозитории. 
Репозитории, такие как электронные биб\-лио\-те\-ки, Semantic Web или Open Archives, 
могут содержать миллионы документов. Авторы достаточно ред\-ко предоставляют 
метаданные, а~электронные документы, со\-бран\-ные в~сети Интернет автоматически, 
име\-ют метаданные еще реже; извлечение метаданных из документов вруч\-ную 
пред\-став\-ля\-ет собой труд\-ную задачу. По этой причине возникает не\-об\-хо\-ди\-мость 
в~сис\-те\-мах автоматического извлечения метаданных из самих документов. В~данной 
статье рас\-смат\-ри\-ва\-ет\-ся извлечение \textit{Заголовка, Автора, Аннотации, 
Содержания, Библиографии.}

Задача извлечения метаданных может быть решена с~помощью правил. Преимущество 
этого подхода в~том, что его достаточно легко реализовать без размеченного 
корпуса. Однако ни правила, ни регулярные выражения не могут обеспечить 
до\-ста\-точ\-но\-го покрытия все\-воз\-мож\-ных стилей форматирования, 
так как сам процесс 
со\-зда\-ния правил содержит в~себе предположение о~схо\-жести форматирования 
ис\-сле\-ду\-емых документов~[1--3]. Недостатки правил за\-клю\-ча\-ют\-ся также 
в~не\-об\-хо\-ди\-мости экспертных знаний и~до\-ста\-точ\-но большого времени, необходимого для 
их создания.

Помимо правил существуют еще несколько подходов к~решению задачи извлечения 
метаданных. Основной из них~--- рас\-смат\-ри\-вать текстовые блоки, полученные из 
pdf-пар\-се\-ра, как объекты. Ставится задача классификации рас\-смат\-ри\-ва\-емо\-го блока 
с~признаковым про\-стран\-ст\-вом, содержащим информацию о~форматировании и~расположении 
блока, а~так\-же текс\-то\-вую и~кон\-текст\-ную информацию.
{\looseness=1

}

Наиболее известные алгоритмы обуче\-ния с~учителем, при\-ме\-ня\-емые в~данной задаче: 
Support Vector Machine (SVM)~\cite{Han, Kovacevic}, Hidden Markov Models (HMMs)~\cite{Seymore} 
и~Conditional Random Fields (CRFs)~\cite{Councill}.

Существует семейство подходов, основанных на нейронных 
сетях~\cite{Rangoni, Rangoni_1}. Здесь объектом является изображение PDF-до\-ку\-мен\-та. 
Для обработки 
изображения применяются OCR (Optical Character Recognition) сис\-те\-мы, но они показывают низкую скорость 
и~точ\-ность работы, поэтому в~данной \mbox{статье} этот подход не используется.

Качество работы алгоритмов извлечения метаданных час\-то измеряется на небольшой 
коллекции достаточно однотипных документов, например из одной об\-ласти знания 
или одного журнала~\cite{Kovacevic, Tao}.

\begin{table*}[b]\small %tabl1
%\vspace*{-12pt}
    \begin{center}
\Caption{Описание признаков}
\label{tab:table1}
\vspace*{2ex}


    \begin{tabular}{|c|l|l|} %\multicolumn{1}{|c|}{\raisebox{-6pt}[0pt][0pt]{
        \hline
\multicolumn{2}{|c|}{Признаки} & \multicolumn{1}{c|}{Описание} \\
\hline

\multicolumn{1}{|c|}{\raisebox{-60pt}[0pt][0pt]
{\tabcolsep=0pt\begin{tabular}{c}Текстовые\\ признаки\end{tabular}}}& isUpper & Написан ли 
весь текст в~верхнем регистре \\
%\cline{2-3}
        & blockLen & Число токенов в~блоке \\
        %\cline{2-3}
        & hasDigit & Есть ли в~блоке  цифра \\
        %\cline{2-3}
        &NumOfPoints & Число точек\\
        %\cline{2-3}
        &NumOfCommas & Число  запятых \\
        %\cline{2-3}
        & NumOfUpCase & Число  символов в~верхнем регистре \\
        %\cline{2-3}
        & UpcaseFraction & Доля  символов в~верхнем регистре \\
        %\cline{2-3}
        & NumOfDigits & Число цифр\\
        %\cline{2-3}
        & NumOfBibSymbols & Число  символов из словаря частых слов библиографии \\
        %\cline{2-3}
        & hasYear & Есть ли год\\
        %\cline{2-3}
        & hasAnno & Есть ли слово аннотация \\
        %\cline{2-3}
        &isEmpty & Является ли пустым  (без текста) \\
        %\cline{2-3}
        & hasKeyW & Есть ли слова  из словаря ключевых слов \\
        %\cline{2-3}
            \hline
   \multicolumn{1}{|c|}{\raisebox{-16pt}[0pt][0pt]
   { \tabcolsep=0pt\begin{tabular}{c}Признаки\\ форматирования\end{tabular} }}& 
Max/Mean/MedianAdv & 
\tabcolsep=0pt\begin{tabular}{l}Максимальное, среднее и~медианное значение толщины\\ 
символов блока\end{tabular} \\
\cline{2-3}
        & Max/Mean/MedianCharH & 
\tabcolsep=0pt\begin{tabular}{l}Максимальное, среднее и~медианное значение высоты\\ 
символов блока\end{tabular} \\
\cline{2-3}
        & fontIndex & Индекс семейства шрифтов \\
        %\cline{2-3}
            \hline
      \multicolumn{1}{|c|}{\raisebox{-22pt}[0pt][0pt]
      { \tabcolsep=0pt\begin{tabular}{c}Признаки\\ расположения\end{tabular} }}& normBlockH/W/A & 
Нормированные высота, ширина и~площадь блока\\
%\cline{2-3}
 & verticalPos & Позиция сверху  страницы \\
% \cline{2-3}
        & leftIndent & Левый отступ\\
   %     \cline{2-3}
        & rightIndent & Правый отступ\\
     %   \cline{2-3}
        &relativeBlockCenter & Позиция  центра блока \\
        %\cline{2-3}
            \hline
                  \multicolumn{1}{|c|}{\raisebox{-38pt}[0pt][0pt]
{    \tabcolsep=0pt\begin{tabular}{c}Контекстные\\ признаки\end{tabular} }}&  
    sameFontAbove/Below &  Блок сверху/снизу имеет такой же индекс шрифта\\
    \cline{2-3}
        & sameCenterAbove/Below & 
Блок сверху/снизу имеет такой же центр \\
\cline{2-3}
        &{sameLeftIndentAbove/Below} & 
Блок сверху/снизу имеет такой же левый отступ\\
\cline{2-3}
        &{hasYearInNeigh} & В фиксированном окне вокруг блока есть год \\
        \cline{2-3}
        &{isEmptyAbove/Below} & Блок  сверху/снизу пуст \\
        \cline{2-3}
        &{hasAnnoInNeigh} & \tabcolsep=0pt\begin{tabular}{l}В~фиксированном окне вокруг блока 
        есть слова из словаря\\ аннотации\end{tabular} \\
        \cline{2-3}
        &{isCharHSmallerThanInNeigh} & 
\tabcolsep=0pt\begin{tabular}{l}Средний размер шрифта блока меньше, чем средний размер\\ 
шрифта в~фиксированном окне\end{tabular} \\
%\cline{2-3}
         \hline
 \end{tabular}
 \end{center}
 \end{table*}   

Научная новизна данной работы заключается в~том, что авторы не ограничиваются 
типом \mbox{\textit{Статья}}, а~работают с~документами различных типов (\textit{Статья, 
Автореферат, Диссертация, Монография, Учебное пособие}) на рус\-ском языке. Оценка\linebreak 
качества алгоритмов производится при помощи\linebreak
 таких мет\-рик, как точ\-ность, полнота 
и~F-ме\-ра, на коллекции PDF-до\-ку\-мен\-тов различных жанров, полученной авторами из 
открытых источников. Проведено сравнение результатов с~базовым SVM-ал\-го\-рит\-мом~\cite{Kovacevic}. 
По сравнению с~базовым алгоритмом использовалось больше 
небинарных при\-зна\-ков и~расширенная информация о~расположении блока.


\section{Формальная постановка задачи}

Задана выборка 
$$
\mathfrak{D} = \left\{({\bf{x}}_i, y_i)\right\}\,, \enskip i = 1,\dots,m\,,
$$ 
состоящая из множества пар <<объ\-ект--класс>>, $\mathbf{x}_i \hm\in  \mathbb{R}^n$. 
Каждый объект~$\mathbf{x}_i$ принадлежит одному из~$Z$~классов с~меткой $y_i \hm\in 
\mathbb{Y} \hm= \{1,\dots,Z\}.$

Пусть также задано множество моделей~$\mathfrak{F}$, среди которых производится 
поиск подходящей модели классификации объектов выборки~$\mathfrak{D}$.
Требуется найти отоб\-ра\-же\-ние $\hat{f} \hm\in \mathfrak{F}: \mathbb{R}^d \hm\to 
\mathbb{Y}$, которое бы минимизировало эмпирический риск на вы\-бор\-ке~$\mathfrak{D}$:
{\looseness=1

}

\noindent
$$
\hat{f} = \underset{f \in \mathfrak{F}}{\argmin}\sum\limits_{\mathbf{x}_i, y_i \in 
\mathfrak{D}}\left[f({\bf{x}}_i) \ne y_i\right].
$$


\section{Описание метода и~признаков}

Для получения блоков документ был обработан парсером. Был выбран парсер Apache 
PDFBox~\cite{PDFBox}, так как он показал наилучшую ско\-рость работы и~воз\-мож\-ность 
на\-строй\-ки. Парсер возвращает текс\-то\-вый блок, который является прямоугольником 
с~текс\-том внут\-ри. Все признаки блока можно раз\-де\-лить на четыре группы (описание 
всех признаков содержится в~табл.~\ref{tab:table1}):
\begin{enumerate}[(1)]
\item \textbf{текстовые признаки.} Данная группа извлекалась из текс\-та 
напрямую. Признаки этой группы проверяют наличие в~строке клю\-че\-вых слов 
и~собирают различные текс\-то\-вые статистики;
\item \textbf{признаки форматирования}. В~эту группу входят такие 
при\-зна\-ки, как размер шриф\-та, семейство шриф\-тов и~др.;

\item \textbf{признаки расположения}. В~этой группе содержится информация 
о~положении блока на странице, такая как правый и~левый отступ, координата 
цент\-ра блока и~др.;
\item \textbf{контекстные признаки}. Эти признаки описывают контекст 
рас\-смат\-ри\-ва\-емо\-го блока, как пред\-ло\-же\-но в~\cite{Tao}.
\end{enumerate}

\vspace*{-12pt}

\section{Эксперимент}

\subsection{Описание коллекции}

Коллекция содержит 550 размеченных PDF-до\-ку\-мен\-тов из различных областей знаний 
и~разных типов (табл.~2).


  
Одна из основных характеристик задачи~--- сильная несбалансированность классов 
(табл.~3). Видно, что общее чис\-ло блоков в~табл.~3 
варьируется для категорий метаданных. Это связано с~эвристиками, примененными на 
этапе обработки документа. Например, для заголовка, автора и~аннотации 
обрабатывалась только первая и~вторая страницы документа, так как обработка 
занимает достаточно много времени и~нет не\-об\-хо\-ди\-мости обрабатывать всю 
диссертацию, чтобы извлечь данные категории. Для библиографии извлекались 
последние $k = \max(0{,}1p, 1)$ стра\-ниц документа, для содержания~--- первые $n\hm = 
\max(0{,}1p, 4)$ стра\-ниц документа, где $p$~--- чис\-ло стра\-ниц в~документе. 
Количество блоков с~автором меньше, чем чис\-ло работ. Это вызвано ошибками работы 
пар\-се\-ра, когда блок с~именем автора не извлекался.

%\begin{table*}
{\small %tabl2
    \begin{center}
    \vspace*{6pt}
    
    \noindent
{{\tablename~2}\ \ \small{Распределение жанров}}

%    \label{tab:table2}  
\vspace*{2ex}

    \begin{tabular}{|l|c|}
        \hline
       \multicolumn{1}{|c|}{Жанр} & Число документов \\ 
        \hline
        Статья & 148 \\ 
       % \hline
        Автореферат & 161 \\ 
       % \hline
        Диссертация & 196 \\ 
        %\hline
        Монография, пособие & \hphantom{9}45 \\
        \hline
    \end{tabular}
    \end{center}}
%\end{table*}
%\begin{table*}
{\small %tabl3
\vspace*{3pt}
\begin{center}
  \noindent
{{\tablename~3}\ \ \small{Распределение блоков по категориям}}

    \vspace*{2ex}
    
    \begin{tabular}{|l|c|c|}
        \hline
        \multicolumn{1}{|c|}{Категория} & 
        \tabcolsep=0pt\begin{tabular}{c}Число блоков\\ данной\\ категории\end{tabular} & 
        \tabcolsep=0pt\begin{tabular}{c}Полное число\\ извлеченных\\ 
блоков\end{tabular} \\ 
\hline
        Заголовок & 957 & 76\,986 \\ %\hline
        Автор & 432 & 76\,986 \\ %\hline
        Аннотация & 1386\hphantom{9} & 76\,986 \\ %\hline
        Библиография & 45\,323\hphantom{\,99} & 316\,015\hphantom{9} \\ %\hline
        Содержание &  5866\hphantom{9} & 309\,681\hphantom{9} \\
        \hline
    \end{tabular}
    \end{center}
    }
%    \end{table*}

\setcounter{table}{3}


\subsection{Описание эксперимента}

Для обоснованного выбора алгоритма была снижена раз\-мер\-ность данных 
с~по\-мощью t-SNE~\cite{Maaten}. Визуализация пред\-став\-ле\-на на рис.~1. 
Видно, что в~про\-стран\-ст\-ве сниженной раз\-мер\-ности данные не 
являются линейно разделимыми, но есть однородные клас\-те\-ры. Этот факт по\-слу\-жил 
одним из обос\-но\-ва\-ний применения ан\-самб\-ля ре\-ша\-ющих деревьев. Также этот алгоритм 
показал наилучшее качество во время экспериментов.


Для проведения экспериментов выборка была разделена на обуча\-ющую и~тес\-то\-вую 
в~пропорции~80/20 стратифицированно в~соответствии с~табл.~3. 
Качество измерялось стандартными мет\-ри\-ка\-ми бинарной классификации: точ\-ность, 
пол\-но\-та и~F-ме\-ра. Результаты приведены в~табл.~\ref{tab:table4}.\linebreak Качество работы 
алгоритма в~данной работе срав\-ни\-ва\-лось с~базовым алгоритмом, взятым из 
\mbox{статьи}~\cite{Kovacevic}. Отличия признакового про\-стран\-ст\-ва базового алгоритма от 
предложенного:
\begin{itemize}
\item
в базовом алгоритме есть признаки, со\-би\-ра\-ющие раз\-лич\-ные html-те\-ги, так как 
использовался другой pdf-пар\-сер. В~проведенном эксперименте такие признаки не 
собирались;

\item
в отличие от базового, в~предложенном алгоритме использовались небинарные 
текстовые статистики, больше признаков расположения и~кон\-текст\-ные признаки.
\end{itemize}


В базовом алгоритме использовался SVM и~извлекалась только первая страница 
статей узкой на\-прав\-лен\-ности. Базовый алгоритм показал низ-\linebreak\vspace*{-12pt}

 { \begin{center}  %fig1
 \vspace*{9pt}
  \mbox{%
 \epsfxsize=74.724mm 
 \epsfbox{oga-1.eps}
 }


\end{center}


\noindent
{{\figurename~1}\ \ \small{Визуализация данных при помощи t-SNE: 
    \textit{1}~--- заголовок;
        \textit{2}~--- автор;  \textit{3}~--- биб\-лио\-гра\-фия;
            \textit{4}~--- содержание;
        \textit{5}~--- другое}}
}

%\vspace*{9pt}

\setcounter{figure}{1}

\begin{table*}\small %tabl4
\begin{center}
\Caption{Результаты эксперимента
    \label{tab:table4}}
\vspace*{2ex}

    \begin{tabular}{|l|l|c|c|c|}
        \hline
\multicolumn{1}{|c|}{Категория}&\multicolumn{1}{c|}{Метод} &
{P} &
{R} &
{F} \\
\hline
      & 
                  {Предложенный} & \textbf{0,74}&
\textbf{0,79} &
        \textbf{0,76} \\
        %\cline{2-5}
     {Заголовок}   & Предложенный только на текстовых 
признаках & 0,67 &
        0,66 &
        0,66 \\
       % \cline{2-5}
        & Базовый & 0,20 &
0,77 &
    0,32 \\
        \hline
 & {Предложенный} & \textbf{0,78}&
0,71 &
    \textbf{0,74} \\
    %\cline{2-5}
       {Автор}  & Предложенный только на текстовых 
признаках & 0,45 &
        0,74 &
        0,56 \\
       % \cline{2-5}
        & Базовый & 0,33 &
        \textbf{0,75} &
        0,46 \\
                \hline
 &  
        {Предложенный} & \textbf{0,76} 
&\textbf{0,85} &
\textbf{0,80} \\
%\cline{2-5}
              Библиография  & Предложенный только на текстовых 
признаках & 0,59 &
        \textbf{0,85} &
0,69 \\
%\cline{2-5}
        & {Базовый} & ---& ---&    --- \\
           \hline
& Предложенный & \textbf{0,72} 
&
\textbf{0,74} &
        \textbf{0,73} \\
       % \cline{2-5}
         {Содержание}    & Предложенный только на текс\-то\-вых 
признаках & 0,71 & 0,69 & 0,70 \\
%\cline{2-5}
        & Базовый & --- & --- &--- \\
        \hline
 &  {Предложенный} & \textbf{0,73}  &
\textbf{0,71} & \textbf{0,72} \\
%\cline{2-5}
     Аннотация   & Предложенный только на текс\-то\-вых 
признаках & 0,22 & 0,54 & 0,31 \\
%\cline{2-5}
        & Базовый & 0,09 & 0,63 & 0,16 \\
                \hline
            \end{tabular}
    \end{center}
    \vspace*{-6pt}
\end{table*}


{ \begin{center}  %fig2
 %\vspace*{0.5pt}
  \mbox{%
 \epsfxsize=78.254mm 
 \epsfbox{oga-2.eps}
 }


\end{center}


\noindent
{{\figurename~2}\ \ \small{Кривые полнота--точ\-ность:
           \textit{1}~--- заголовок;
               \textit{2}~--- автор; 
           \textit{3}~--- биб\-лио\-гра\-фия; \textit{4}~--- содержание;
    \textit{5}~--- аннотация}}
}

\vspace*{9pt}

\setcounter{figure}{2} 



\noindent
кое качество, так как 
в~текущей по\-ста\-нов\-ке 
эксперимента стиль форматирования документов очень сильно 
варьируются.


В табл.~\ref{tab:table4} приведены результаты эксперимента, в~котором 
признаковое пространство со\-сто\-яло только из текстовых статистик. Вид\-но, что 
расширение признакового про\-стран\-ст\-ва информацией о~расположении и~форматировании 
блока заметно повышает качество классификации.

Из-за несбалансированности классов была проведена редукция блоков, относящихся 
к~классу \textit{Прочее}. Если~150~элементов до и~после блока \textit{Прочее} не 
содержали других классов, то такой блок удалялся. Проводились эксперименты 
с~порогом ве\-ро\-ят\-ности классификации и~весами классов. Эксперименты показали, что 
порог классификации~0,5 и~вес редкого класса, равный двум, демонстрирует 
наилучший результат для  \textit{Содержания, Библиографии, Аннотации}. Для 
\textit{Заголовка} и~\textit{Автора} перенос порога классификации в~0,6 и~0,65 
соответственно в~ком\-би\-на\-ции с~весом редкого класса, равным~20, дали 
повышение точ\-ности ($+0{,}05$ и~$+0{,}18$) с~падением пол\-но\-ты ($-0{,}02$ и~$-0{,}16$), 
но с~небольшим повышением F-ме\-ры. 
Кривая пол\-но\-та--точ\-ность изоб\-ра\-же\-на на 
рис.~2.

 







Был проведен анализ значимости признаков, которая была получена стандартным для 
алгоритма Random Forest способом~\cite{Breiman}. Три первых по значимости 
при\-зна\-ка для каж\-до\-го класса приведены в~табл.~5. 
Видно, что почти в~каждом классе\linebreak\vspace*{-12pt}

%\begin{table*}
{\small %tabl5
\vspace*{1pt}
    \begin{center}
    \noindent
{{\tablename~5}\ \ \small{Результаты анализа значимости признаков}}
    \vspace*{2ex}
    
    \begin{tabular}{|l|l|c|}
        \hline
 \multicolumn{1}{|c|}{Категория}& \multicolumn{1}{c|}{Признак} &
        \multicolumn{1}{c|}{Значимость} \\
        \hline
        & {NumOfUpCase} & 0,65 \\
%\cline{2-3}
{Заголовок}        & {verticalPos} & 0,06 
\\
%\cline{2-3}
        & {MeanAdv} & 0,06 \\
          \hline
&  {UpcaseFraction} & 0,34 \\
%\cline{2-3}
    Автор     & {NumOfUpCase} & 0,23\\
%\cline{2-3}
        & {verticalPos} & 0,11\\
        %\cline{2-3}
                \hline
 &  NumOfPoints & 0,41\\
        %\cline{2-3}
        Содержание        & blockLen & 0,18 
\\
%\cline{2-3}
        & {NumOfDigits} & 0,08\\
        %\cline{2-3}
                \hline
&  {blockLen} & 0,14\\
        %\cline{2-3}
        Библиография  & {relativeBlockCenter} & 
0,12 \\
%\cline{2-3}
        & {normBlockW} & 0,10\\
        %\cline{2-3}
                \hline
 & {verticalPos} & 0,10\\
        %\cline{2-3}
               Аннотация & normBlockA & 0,10\\
        %\cline{2-3}
        & MeanAdv & 0,09\\
        %\cline{2-3}
        \hline  
     \end{tabular}
               \vspace*{1pt}   
   \end{center}
   }
%\end{table*}

\setcounter{table}{5}

\end{multicols}

\begin{figure*} %fig3
\vspace*{1pt}
 \begin{center}
 \mbox{%
 \epsfxsize=165.062mm 
 \epsfbox{oga-3.eps}
 }
 \end{center}
\vspace*{-9pt}
    \Caption{Визуализация данных для заголовка
    (\textit{1}~--- заголовок; \textit{2}~--- прочее)
     в~проекции на про\-стран\-ст\-во 
значимых при\-зна\-ков }
    \label{fig:figure3}
\end{figure*}

\begin{multicols}{2}

\noindent
 признаки расположения и~форматирования вошли в~число значимых. 
Данные были визуализированы в~проекции на про\-стран\-ст\-во значимых при\-зна\-ков. На 
рис.~\ref{fig:figure3} и~\ref{fig:figure4} показаны примеры таких визуализаций 
для категорий \textit{Заголовок} и~\textit{Автор}. На рисунках видны сегменты, 
где преимущественно расположен целевой класс.





    
\vspace*{-6pt}


\subsection{Анализ ошибок}

С целью выявления основных ошибок предложенного алгоритма был произведен анализ 
результатов.

\vspace*{-12pt}

\paragraph*{Заголовок.}
В~табл.~5 видно, что для категории \textit{Заголовок} самым 
значимым признаком является число символов в~верхнем регистре. Поэтому 
лож\-но-по\-ло\-жи\-тель\-ные примеры наблюдаются вбли\-зи заголовка по расположению 
и~с~большим 
количеством символов в~верхнем ре\-гистре.

\textbf{Примеры:}

\begin{itemize}
\item[$\bullet$]
\textit{ГОСУДАРСТВЕННЫЙ УНИВЕРСИТЕТ УПРАВ\-ЛЕ\-НИЯ}

\item[$\bullet$]
\textit{05.13.11 МАТЕМАТИЧЕСКОЕ И ПРО\-ГРАМ\-МНОЕ ОБЕСПЕЧЕНИЕ}
\end{itemize}

Основной причиной лож\-но-от\-ри\-ца\-тель\-ных примеров оказались длинные заголовки 
с~форматированием, близким к~обыкновенному тексту. 
Примеры:

\begin{itemize}
\item[$\bullet$]
\textit{Французское органное искусство Барокко: музыка, органостроение, 
исполнительство}\\[-14.5pt]

\item[$\bullet$]
\textit{Нейтральный внешнеполитический курс Нидерландов: от Мюнстерского мира 
1648~г.\ до конца Первой мировой войны}\\[-14.5pt]

\item[$\bullet$]
\textit{Музыкальная жизнь Москвы XIX столетия и~ее отражение в~фортепианной 
практике}
\end{itemize}

\vspace*{-13pt}

\paragraph*{Автор.}

Если извлеченный блок располагался вверху страницы и~содержал имя и~фамилию, то 
наблюдался лож\-но-по\-ло\-жи\-тель\-ный пример. Примеры такого рода наблюдались, когда 
блок содержал комбинацию точек и~символов в~верх\-нем регистре, схожую с~именем 
и~фамилией.

\textbf{Примеры:}

\begin{itemize}
\item[$\bullet$]
\textit{им.\ М.\,В.~Келдыша РАН}\\[-15pt]

\item[$\bullet$]
\textit{(ПЖ). Одним}

Ложно-отрицательные примеры возникали, когда блок с~автором находился в~очень 
нетипичном месте.\\[-15pt]
\end{itemize}

\end{multicols}

\begin{figure*} %fig4
\vspace*{1pt}
 \begin{center}
 \mbox{%
 \epsfxsize=165.459mm 
 \epsfbox{oga-4.eps}
 }
 \end{center}
\vspace*{-9pt}
       \Caption{Визуализация данных для автора 
       (\textit{1}~--- автор; \textit{2}~--- прочее)
       в~проекции на про\-стран\-ст\-во 
значимых при\-зна\-ков
}
    \label{fig:figure4}
\end{figure*}

\begin{multicols}{2}

\begin{itemize}
\item[$\bullet$]
\textit{Огальцов А.\,В.} (если находится где-то в~очень нетипичном месте 
страницы)
\end{itemize}

\vspace*{-10pt}

\paragraph*{Аннотация.}

Самые частые ошибки обоих типов в~этой категории возникали в~связи с~тем, что 
блоки с~аннотацией легко спутать с~обычным текс\-том.

\vspace*{-10pt}

\paragraph*{Содержание.}

Ложно-положительные примеры~--- это в~основном блоки с~большим чис\-лом точек, 
скорее всего вызванных ошибками обработки.

\textbf{Примеры:}

\begin{itemize}

\item[$\bullet$]
\textit{В ТОМСКОЙ ОБЛАСТИ ............}

\item[$\bullet$]
\textit{Случай d(k)=0 ...........................}

Ложно-отрицательные примеры вызваны форматированием, близким к~обычному тексту.

\item[$\bullet$]
\textit{Функции ввода, вывода и~работы с~символами}

\item[$\bullet$]
\textit{Циклы, блоки и~присваивания}
\end{itemize}

\vspace*{-10pt}

\paragraph*{Библиография.}

Ложно-положительные примеры этой категории похожи на биб\-лио\-гра\-фию 
расположением и~форматированием.

\textbf{Примеры:}

\begin{itemize}
\item[$\bullet$]
\textit{Тел.: (3472) 1234567, 1234567}

\item[$\bullet$]
\textit{Печать офсетная. Гарнитура NewtonC. Бумага офсетная}

\item[$\bullet$]
\textit{как отмечают А.\,Я.~Гаев и~др. 2005, нереально.}


Ложно-отрицательные примеры были вызваны сильной схо\-жестью с~обычным текс\-том.
\end{itemize}


Большое число блоков с~обычным текст\-ом делает возможным ситуацию сходства этих 
блоков с~целевыми по ряду признаков, поэтому одним из на\-прав\-ле\-ний улуч\-ше\-ния 
является снижение несбалансированности выборки различными методами.
    
    
\section{Заключение}

В работе была исследована возможность применения методов машинного обучения 
к~задаче извлечения метаданных. Использовано расширенное признаковое про\-стран\-ст\-во, 
вклю\-ча\-ющее не только текс\-то\-вые признаки, но также признаки форматирования 
и~расположения, которые оказались важ\-ны\-ми в~решении проб\-ле\-мы большой вариативности 
стилей форматирования документа. Измерено качество предлагаемого алгоритма на 
коллекции научных PDF-до\-ку\-мен\-тов на рус\-ском языке, взятых из открытых 
источников. Проведен анализ ошибок. Показано, что предложенный алгоритм 
значительно превосходит базовый на документах различных типов.

Будущие исследования будут направлены на расширение признакового пространства 
и~на обзор возможностей новых PDF-пар\-се\-ров. Также будет исследована воз\-мож\-ность 
применения графических моделей, таких как услов\-ные случайные поля.

\bigskip

Авторы выражают свою благодарность к.ф.-м.н.\ Чеховичу~Ю.\,В.\ за ценные советы 
при планировании исследования и~рекомендации по оформлению статьи.


{\small\frenchspacing
 {%\baselineskip=10.8pt
 \addcontentsline{toc}{section}{References}
 \begin{thebibliography}{99}
\bibitem{Bergmark}
  \Au{Bergmark D.} Automatic extraction of reference linking information 
from online documents.~--- Ithaca, NY, USA: Cornell University, 2000. 20~p.

\bibitem{Klink}
\Au{Klink S., Dengel~A., Kieninger~T. } Document structure analysis 
based on layout and textual features~//  Workshop (International) on 
Document Analysis Systems Proceedings.~---
        Boston, MA, USA, 2000. P.~99--111.

\bibitem{Mao}
    \Au{Mao S., Kim J.\,W., Thoma~G.\,R.} A~dynamic feature generation 
system for automated metadata extraction in preservation of digital materials~// 
1st  Workshop (International) on Document Image Analysis for 
Libraries Proceedings.~--- Palo Alto, CA, USA: IEEE Computer Society, 2004. P.~225--232.
    
\bibitem{Han}
\Au{Han H., Giles C.\,L., Manavoglu~E., Zha~H., Zhang~Z., Fox~E.\,A.} 
Automatic document metadata extraction using support vector machines~// 
3rd ACM/IEEE-CS Joint Conference on Digital Libraries Proceedings, 2003. 
P.~37--48.

\bibitem{Kovacevic}
\Au{Kovacevi$\acute{\mbox{c}}$~A., Ivanovi$\acute{\mbox{c}}$~D., Milosavljevi$\acute{\mbox{c}}$~B., 
Konjovi$\acute{\mbox{c}}$~Z., Surla~D.} 
Automatic extraction of metadata from scientific publications for CRIS systems~// 
Program, 2011. Vol.~45. Iss.~4. P.~376--396.

\bibitem{Seymore}
\Au{Seymore K., McCallum~A., Rosenfeld~R.} Learning hidden Markov model 
structure for information extraction~// AAAI~99 Workshop on 
Machine Learning for Information Extraction Proceedings, 1999. P.~37--42.

\bibitem{Councill}
\Au{Councill I., Giles~C.\,L., Kan~M.-Y.} ParsCit: An open-source CRF 
reference string parsing package~// 6th Conference 
(International) on Language Resources and Evaluation Proceedings.~---
Marrakech, Morocco, 2008. P.~661--667.


\bibitem{Rangoni}
\Au{Rangoni Y., \mbox{Bela$\ddot{\mbox{\!\hspace*{-0.2pt}{\ptb{\!\i}}}}$d~A.}} Document logical structure analysis based on 
perceptive cycles~// IAPR Workshop on Document Analysis Systems, 2006. P.~117--128.

\bibitem{Rangoni_1}
\Au{Rangoni Y., Bela$\ddot{\mbox{\!\hspace*{-0.2pt}{\ptb{\!\i}}}}$d~A., Vajda~S.} Labelling logical structures of 
document images using a~dynamic perceptive neural network~// Int. 
J.~Doc. Anal. Recog., 2012. Vol.~15. Iss.~1. P.~45--55.

\bibitem{Tao}
\Au{Tao X., Tang~Z., Xu~C.} Document page structure learning for fixed-layout 
e-books using conditional random fields~// 
 Document Recognition and Retrieval XXI:
 Proc. SPIE, 2014. 
Vol.~9021. P.~1--9.

\bibitem{PDFBox}
Apache PDFBox. {\sf http://pdfbox.apache.org}.

\bibitem{Maaten}
\Au{Van der Maaten L.\,J.\,P., Hinton~G.\,E.} Visualizing 
high-dimensional data using t-SNE~// J.~Mach. Learn. Res., 2008. 
Vol.~9. P.~2579--2605.

\bibitem{Breiman}
\Au{Breiman L.} Random forests~// Mach. Learn., 2001. Vol.~45. P.~5--32.

 \end{thebibliography}

 }
 }

\end{multicols}

\vspace*{-6pt}

\hfill{\small\textit{Поступила в~редакцию 20.12.17}}

\vspace*{6pt}

%\newpage

%\vspace*{-24pt}

\hrule

\vspace*{2pt}

\hrule

%\vspace*{8pt}


\def\tit{AUTOMATIC METADATA EXTRACTION FROM SCIENTIFIC\\ PDF DOCUMENTS}

\def\titkol{Automatic metadata extraction from scientific PDF documents}

\def\aut{A.\,V.~Ogaltsov$^{1,2}$ and~O.\,Y.~Bakhteev$^{2,3}$}

\def\autkol{A.\,V.~Ogaltsov and~O.\,Y.~Bakhteev}

\titel{\tit}{\aut}{\autkol}{\titkol}

\vspace*{-9pt}


\noindent
$^1$National Research University Higher School of Economics, 
20~Myasnitskaya Str., Moscow 101000, Russian\linebreak
$\hphantom{^1}$Federation

\noindent
$^2$Antiplagiat JSC, 33~Varshavskoe Shosse, Moscow 117105, Russian Federation

\noindent
$^3$Moscow Institute of Physics and Technology, 9~Institutskiy Per., Dolgoprudny, 
Moscow Region 141700, Russian\linebreak
$\hphantom{^1}$Federation

\def\leftfootline{\small{\textbf{\thepage}
\hfill INFORMATIKA I EE PRIMENENIYA~--- INFORMATICS AND
APPLICATIONS\ \ \ 2018\ \ \ volume~12\ \ \ issue\ 2}
}%
 \def\rightfootline{\small{INFORMATIKA I EE PRIMENENIYA~---
INFORMATICS AND APPLICATIONS\ \ \ 2018\ \ \ volume~12\ \ \ issue\ 2
\hfill \textbf{\thepage}}}

\vspace*{3pt}




\Abste{The authors investigate the task of metadata extraction. The authors consider 
scientific PDF documents in Russian. One of the features of PDF is a~rich layout. 
It is difficult to extract metadata due to this fact. The authors propose 
a~method based on considering blocks from pdf-parser as objects in 
a~machine learning task. The feature space is constructed not only of text 
statistics but also of formatting and positioning features of the block. 
The authors performed computational experiments and compared their approach 
with the baseline.}

\KWE{metadata extraction; natural language processing; layout features; 
information retrieval; metadescriptions} 

\DOI{10.14357/19922264180211} %

%\vspace*{-14pt}

  \Ack
   \noindent
   The paper was supported by the Russian Foundation 
   for Basic Research (project 18-07-01441).



%\vspace*{-3pt}

  \begin{multicols}{2}

\renewcommand{\bibname}{\protect\rmfamily References}
%\renewcommand{\bibname}{\large\protect\rm References}

{\small\frenchspacing
 {%\baselineskip=10.8pt
 \addcontentsline{toc}{section}{References}
 \begin{thebibliography}{99}
        \bibitem{Bergmark-1}
        \Aue{Bergmark, D.} 2000. Automatic extraction of reference 
        linking information from online documents.  Ithaca, NY: Cornell University.  20~p.
        
        \bibitem{Klink-1}
        \Aue{Klink, S., A.~Dengel, and T.~Kieninger. } 2010. 
        Document structure analysis based on layout and textual features. 
        \textit{Workshop (International) on Document Analysis Systems Proceedings}. 
        Boston, MA. 99--111.
        
        \bibitem{Mao-1}
        \Aue{Mao, S., J.\,W.~Kim, and G.\,R.~Thoma.} 2004. 
        A~dynamic feature generation system for automated metadata extraction in 
        preservation of digital materials. 
        \textit{1st  Workshop (International) on Document Image Analysis for Libraries
        Proceedings}. Palo Alto, CA: IEEE Computer Society. 225-232.
        
        \bibitem{Han-1}
        \Aue{Han,~H., C.\,L.~Giles, E.~Manavoglu, H.~Zha, Z.~Zhang, and E.\,A.~Fox.} 
        2003. Automatic document metadata extraction using support vector machines. 
        \textit{3rd ACM/IEEE-CS Joint Conference on Digital Libraries Proceedings}. 
        37--48.
        
        \bibitem{Kovacevic-1}
        \Aue{Kovacevi$\acute{\mbox{c}}$, A., D.~Ivanovi$\acute{\mbox{c}}$, 
        B.~Milosavljevi$\acute{\mbox{c}}$, Z.~Konjovi$\acute{\mbox{c}}$, and D.~Surla.} 
        2011. Automatic extraction of metadata from scientific publications 
        for CRIS systems. \textit{Program}  45(4):376--396.
        
        \bibitem{Seymore-1}
        \Aue{Seymore, K., A.~McCallum, and R.~Rosenfeld.} 1999. 
        Learning hidden Markov model structure for information extraction. 
        \textit{AAAI~99 Workshop on Machine Learning for Information Extraction
        Proceedings}. 37--42.
        
        \bibitem{Councill-1}
        \Aue{Councill, I., C.\,L.~Giles, and M.-Y.~Kan.} 2008. ParsCit: An 
        open-source CRF reference string parsing package. 
        \textit{6th  Conference (International) on Language Resources and Evaluation 
        Proceedings}. Marrakech, Morocco. 661--667.
        
        
        \bibitem{Rangoni-1}
        \Aue{Rangoni, Y., and A.~\mbox{Bela$\ddot{\mbox{{\ptb{\!\i}}}}$d.}} 2006. 
        Document logical structure analysis based on perceptive cycles.
         \textit{IAPR Workshop on Document Analysis Systems}. 117--128.
        
        \bibitem{Rangoni_1-1}
        \Aue{Rangoni, Y., A.~\mbox{Bela$\ddot{\mbox{\ptb{\!\i}}}$d}, and S.~Vajda.} 
        2012. Labelling logical structures of document images using a~dynamic 
        perceptive neural network \textit{Int. J.~Doc. Anal. Recog.} 
        15(1):45--55.
        
        \bibitem{Tao-1}
        \Aue{Tao, X., Z.~Tang, and C.~Xu.} 2013. Document page structure learning 
        for fixed-layout e-books using conditional random fields. 
        \textit{Document Recognition and Retrieval XXI: Proc. SPIE} 9021:1--9.
        
        \bibitem{PDFBox-1}
        {Apache PDFBox}. Available at: {\sf http://pdfbox.apache.\linebreak org/}
        (accessed June~19, 2018).
        
        \bibitem{Maaten-1}
        \Aue{Van der Maaten, L.\,J.\,P., and G.\,E.~Hinton.} 
        2008. Visualizing high-dimensional data using t-SNE. \textit{J.~Mach. 
        Learn. Res.} 9:2579--2605.
        
        \bibitem{Breiman-1}
        \Aue{Breiman, L.} 2001. Random forests. \textit{Mach. Learn.} 45:5--32.
        
        \end{thebibliography}

 }
 }

\end{multicols}

\vspace*{-3pt}

\hfill{\small\textit{Received December 20, 2017}}

%\vspace*{-24pt}
        
\Contr

\noindent
\textbf{Ogaltsov Alexander V.} (b.\ 1993)~--- 
student, National Research University Higher School of Economics, 
20~Myasnitskaya Str., Moscow 101000, Russian Federation; researcher, 
Antiplagiat JSC, 33~Varshavskoe Shosse, Moscow 117105, Russian Federation; 
\mbox{avogaltsov@edu.hse.ru}

\vspace*{3pt}

\noindent
\textbf{Bakhteev  Oleg Y.} (b.\ 1993)~--- PhD student, 
Moscow Institute of Physics and Technology, 9~Institutskiy Per., Dolgoprudny, 
Moscow Region 141700, Russian Federation; 
researcher, Antiplagiat JSC, 33~Varshavskoe Shosse, Moscow 117105, Russian Federation; 
\mbox{bakhteev@phystech.edu}


\label{end\stat}


\renewcommand{\bibname}{\protect\rm Литература} 