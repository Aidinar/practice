\def\stat{zhukov}

\def\tit{ВЛИЯНИЕ ПЛОТНОСТИ СВЯЗЕЙ НА~КЛАСТЕРИЗАЦИЮ И~ПОРОГ ПЕРКОЛЯЦИИ 
ПРИ~РАСПРОСТРАНЕНИИ ИНФОРМАЦИИ В~СОЦИАЛЬНЫХ СЕТЯХ$^*$}

\def\titkol{Влияние плотности связей на кластеризацию и~порог перколяции 
при~распространении информации в %~социальных 
сетях}

\def\aut{Д.\,О.~Жуков$^1$, Т.\,Ю.~Хватова$^2$, С.\,А.~Лесько$^3$, 
А.\,Д.~Зальцман$^4$}

\def\autkol{Д.\,О.~Жуков, Т.\,Ю.~Хватова, С.\,А.~Лесько, 
А.\,Д.~Зальцман}

\titel{\tit}{\aut}{\autkol}{\titkol}

\index{Жуков Д.\,О.}
\index{Хватова Т.\,Ю.}
\index{Лесько С.\,А.} 
\index{Зальцман А.\,Д.}
\index{Zhukov D.\,O.}
\index{Khvatova T.\,Yu.}
\index{Lesko S.\,A.} 
\index{Zaltsman A.\,D.}




{\renewcommand{\thefootnote}{\fnsymbol{footnote}} \footnotetext[1]
{Работа выполнена при финансовой поддержке РФФИ (проект 16-29-09458~офи\_м).}}


\renewcommand{\thefootnote}{\arabic{footnote}}
\footnotetext[1]{Московский технологический университет (МИРЭА), zhukov\_do@mirea.ru}
\footnotetext[2]{Санкт-Петербургский политехнический университет Петра Великого, 
\mbox{khvatova.ty@spbstu.ru}}
\footnotetext[3]{Московский технологический университет (МИРЭА), sergey@testor.ru}
\footnotetext[4]{Московский технологический университет (МИРЭА), 
\mbox{ad.zaltcman@gmail.com}}





\Abst{Рассматриваются вопросы применения новых тео\-ре\-ти\-че\-ских подходов 
к~описанию процессов передачи и~обработки информации в~социотехнических сис\-те\-мах 
и~сетях социальных связей на основе теории перколяции. Величина порога перколяции 
случайной сети зависит от ее плот\-ности. В~сетях, име\-ющих случайную структуру, пороги 
перколяции как в~задаче узлов, так и~в задаче связей при большой плот\-ности сети достигают 
величины насыщения, причем величина насыщения порога перколяции в~задаче связей 
больше, чем в~задаче узлов. С~точ\-ки зрения информационного влияния сети, имеющей 
случайную структуру, увеличение плот\-ности связей оказывается более предпочтительным, 
чем наличие небольшого чис\-ла отдельных <<центральных>> узлов, имеющих множество 
связей. 
В~практическом плане полученные результаты могут быть применены 
в~междисциплинарных исследованиях, вклю\-чая информатику, математическое 
моделирование и~экономику, с~при\-вле\-че\-ни\-ем социологических данных для прогнозирования 
поведения и~управ\-ле\-ния группами людей в~сетевых сообществах.
Полученные результаты дополняют и~расширяют применение методов и~подходов, принятых 
в~классической информатике, на описание социальных и~социотехнических сис\-тем, что 
может быть полезно для широкого круга исследователей, занимающихся изуче\-ни\-ем 
социальных сетевых структур.}

\KW{теория перколяции; структура социальной сети; плот\-ность связей; кластеризация сети; 
порог перколяции} 

\DOI{10.14357/19922264180213}
  
%\vspace*{-6pt}
\vspace*{6pt}


\vskip 10pt plus 9pt minus 6pt

\thispagestyle{headings}

\begin{multicols}{2}

\label{st\stat}

\section{Введение}

Изначально, в~классическом пред\-став\-ле\-нии, информатика (\textit{фр.}\ informatique, 
\textit{англ.}\ computer science)~--- это наука о~методах и~процессах сбора, хранения, 
обработки, передачи, анализа и~оценки информации с~применением 
компьютерных технологий, обеспечивающих воз\-мож\-ность ее использования 
для принятия решений. Однако стремительное развитие сетевых технологий, 
и~в~первую очередь Интернета, привело к~появлению обширного класса 
социотехнических сис\-тем, ярким пред\-ста\-ви\-те\-лем которых выступают 
социальные сети. Наличие человеческого фактора приводит к~тому, что для 
описания про\-те\-ка\-ющих в~них процессов уже недостаточно классических 
методов и~моделей, принятых в~информатике, для принятия решений требуется 
использование междисциплинарных подходов и~расширение усто\-яв\-ших\-ся 
классических пред\-став\-ле\-ний, особенно при исследовании процессов 
распространения и~обработки информации в~социальных сетях. В~этом плане 
необходимо говорить о том, что применение методов и~моделей информатики 
должно быть расширено на социотехнические и~социальные сис\-те\-мы точ\-но так 
же, как в~свое время Норберт Винер определил кибернетику как науку об 
управ\-ле\-нии в~живом и~неживом~[1]. 

  Исследование процессов распространения информации и~кластеризации 
узлов (клас\-тер~--- группа связанных между собой узлов, выделенных по 
определенным свойствам или типам) в~сетях социальных связей, име\-ющих 
случайную топологию, является очень важной и~актуальной задачей для 
экономики, рекламы, маркетинга, социологии, политологии и~т.\,д., что 
подтверждается достаточно большим чис\-лом работ по данной тематике~[2--4].
  
  Сеть социальных связей можно определить как со\-во\-куп\-ность 
информационных каналов каж\-до\-го человека, свя\-зы\-ва\-ющих его с~другими 
членами сообщества, а~так\-же средств массовой информации (т.\,е.\ не только 
взаимодействие между членами определенной группы в~социальной сети). 
Иными словами, сеть социальных связей является социальной сетью, 
помещенной в~информационную среду (средства массовой информации, книги, 
газеты, журналы и~т.\,д.).
  
  Средства массовой информации (радио, телевидение, ин\-тер\-нет-ре\-сур\-сы, 
социальные сети и~др.), а~так\-же книги, газеты, журналы оказывают 
существенное влияние на со\-сто\-яние отдельных узлов социальной сети (выбор 
предпочтений и~поведенческие реакции людей), которое может меняться 
с~течением времени  за счет как информационной среды, так 
и~взаимодействия пользователей между собой. 
  
  В связи с~этим возникает ряд вопросов. Во-пер\-вых, как может происходить 
изменение в~сети доли узлов, находящихся в~том или ином со\-сто\-янии?  
Во-вто\-рых, как эти узлы свя\-зы\-ва\-ют\-ся между собой в~подгруппы 
(клас\-те\-ри\-за\-ция сети)? В-треть\-их, как информационные процессы зависят от 
всей сети в~целом для различных со\-сто\-яний узлов?
  
  Узлами социальной сети являются отдельные люди, а~реб\-ра\-ми~--- 
коммуникативные связи, чис\-ло которых может иметь произвольное значение. 
  
  В такой сети распространение информации может одновременно 
происходить множеством путей\linebreak
 через разные узлы сети. Отдельный узел сети 
может получать от другого узла некоторую инфор\-мацию (рекламные 
предложения, идеи, политические взгляды, профессиональные сведения, 
\mbox{мнения} и~др.)\ и~передавать ее другим узлам, если имеет со\-глас\-ную 
с~по\-сту\-пив\-шей информацией позицию (является активным проводником), или 
блокирует ее в~противном случае. 
  
  Для моделирования и~анализа информационных процессов, протекающих 
в~социальных сетях со случайной структурой, воз\-мож\-но применение методов 
теории перколяции~[5--7], которая может поз\-во\-лить, например, ответить на 
сле\-ду\-ющие вопросы:
  \begin{enumerate}[1.]
\item Как происходит клас\-те\-ри\-за\-ция сети на группы связанных между собой 
определенными взглядами людей в~за\-ви\-си\-мости, например, от среднего чис\-ла 
связей на узел?
\item При какой доле людей (узлов сети) с~определенными взглядами может 
создаваться условие для беспрепятственного распространения этих взглядов 
между двумя любыми произвольно вы\-бран\-ны\-ми узлами (протекание или 
пер\-ко\-ля\-ция)? 
  \end{enumerate}
  
  Наиболее распространенными задачами тео\-рии перколяции являются 
\textit{решеточные задачи}: задача связей и~задача узлов. В~задаче связей ищут 
ответ на вопрос: какую долю связей нужно удалить (перерезать), чтобы сетка 
рас\-па\-лась на две части? В~задаче узлов блокируют узлы (удаляют узел, 
перерезая все входящие в~узел связи) и~ищут, при какой доле блокированных 
узлов сетка рас\-па\-дется. 
  
  \textbf{Кластеризация социальной сети.} Если переход любого узла 
социальной сети (индивидуума) из одно\-го со\-сто\-яния в~другое рас\-смат\-ри\-вать 
как случайный процесс (с~некоторой вероятностью перехода, определяемой 
множеством случайных факторов, в~том чис\-ле зависящих от вли\-яния средств 
массовой информации), то ве\-ро\-ят\-ность перехода будет влиять на средний 
размер клас\-те\-ра пользователей сети (группы на\-пря\-мую связанных меж\-ду собой 
узлов). 
  
  \textbf{Перколяция в~социальной сети.} В~тео\-рии пер\-ко\-ля\-ции доля 
проводящих (неблокированных) узлов, при которой возникает про\-во\-ди\-мость 
между двумя различными произвольно вы\-бран\-ны\-ми узлами сети, называется 
порогом пер\-ко\-ля\-ции (протекания). 
  
  Доля узлов социальной сети, находящихся в~том или ином со\-сто\-янии, может 
быть выявлена путем социологических опро\-сов, что пред\-остав\-ля\-ет 
воз\-мож\-ность оценить, насколько социальная сеть близка к~порогу пер\-ко\-ля\-ции, 
а~так\-же управ\-лять ее со\-сто\-яни\-ями.

\vspace*{-6pt}
  
\section{Обзор существующих моделей описания характеристик 
и~анализа структуры социальных сетей}
 
  Наиболее часто для проведения исследований операций и~процессов 
в~сетевых структурах сегодня используются готовые средства анализа, 
например инструменты анализа социальных сетей (SNA~--- social network 
analysis), позволяющие получить количественные характеристики па\-ра\-мет\-ров 
графа сети, таких как <<цент\-раль\-ность>>, <<про\-ме\-жу\-точ\-ность>>,  
<<плот\-ность>> (сред\-нее чис\-ло связей, приходящихся на один узел 
сети)~\cite{7-zh}. Цент\-раль\-ность характеризует степень влияния данного узла 
на всю сеть. Про\-ме\-жу\-точ\-ность характеризует степень включенности объекта 
в~марш\-ру\-ты связей между другими участниками сети. Про\-ме\-жу\-точ\-ность 
показывает, насколько час\-то данный узел встречается на кратчайших путях 
меж\-ду другими узлами. 
  
  Использование готовых инструментов имеет как свои преимущества, так 
и~ряд недостатков. К~преимуществам можно отнести то, что готовые 
инструменты социального сетевого анализа поз\-во\-ля\-ют сравнивать между собой 
 однотипные сети
по количественным характеристикам. Основные недостатки 
заключаются в~том, что они не позволяют создавать новые, более 
информативные модели. 
  
  В работе~\cite{8-zh} были изучены статистические свойства реальных 
социальных сетей работников домохозяйств и~их мера фрагментации после 
удаления некоторых долей узлов или ссылок из сети. 
  
  В работе~\cite{9-zh} рас\-смат\-ри\-ва\-ет\-ся взаимосвязь перколяционного перехода 
и~выживаемости узлов в~сложной сети с~3~млн связей, по\-стро\-ен\-ная 
вокруг примерно~300~тыс.\ фирм (узлы). Характер\-ная особенность этих 
реальных сетей за\-клю\-ча\-ет-\linebreak ся в~том, что они являются мас\-шта\-би\-ру\-емы\-ми\linebreak 
и~степень их мас\-шта\-би\-ру\-емости асимптотически следует степенному закону. 
Эта функ\-ция под\-разуме\-ва\-ет, что каждая из таких сетей со\-сто\-ит из нескольких 
крупных цент\-раль\-ных узлов с~тысячами связей, мно\-же\-ст\-вом промежуточных 
узлов и~еще б$\acute{\mbox{о}}$льшим числом очень мелких узлов 
с~несколькими связями~\cite{10-zh}. Масштабируемые сети кластеризуются на 
отдельные блоки при достаточно высокой плот\-ности связей, если узлы 
удаляются в~порядке убывания степени цент\-раль\-ности~\cite{11-zh}. При 
случайном удалении узлов и~связей мас\-шта\-би\-ру\-емые сети плохо 
клас\-те\-ри\-зу\-ют\-ся, даже при очень низкой плот\-ности узлов, и~перколяционные 
переходы не отмечаются. В~данном исследовании с~по\-мощью 
точного чис\-лен\-но\-го расчета~\cite{9-zh} доказано существование перколяционного 
перехода в~сложных сетях при случайном удалении узлов, когда плот\-ность сети 
очень низкая, но не нулевая. 

\vspace*{-6pt}
     
\section{Постановка задачи и~методика проведения исследований}

  Несмотря на существенный прогресс в~исследовании информационных 
процессов, про\-те\-ка\-ющих в~социальных сетях, и~использование для этого 
тео\-рии перколяции, еще очень многие задачи остаются нерешенными. 
Большинство исследователей уделяют большое внимание изучению степени 
про\-ме\-жу\-точ\-ности, плот\-ности сети, средней длине пути (бли\-зости), 
цент\-раль\-ности и~т.\,д. Вместе с~тем никто из исследователей не обращал еще 
це\-ле\-на\-прав\-лен\-но внимание на изуче\-ние таких вопросов, как влияние плот\-ности 
(среднего чис\-ла связей в~расчете на один узел) сети на ее кластеризацию 
и~величину порога перколяции (как в~задаче узлов, так и~в задаче связей) как 
в~масштабируемых, так и~в~случайных сетях. На взгляд авторов, необходимо 
учитывать в~целом всю со\-во\-куп\-ность свойств сети, которые определяют порог 
перколяции и~кластеризацию; значение имеют как узлы, у~которых много 
связей и~они явля\-ют\-ся значимыми или цент\-раль\-ны\-ми, так и~с~малым чис\-лом 
связей. 
  
  Следует отметить, что существующие про\-грам\-мные инструменты 
информационного анализа социальных сетей (SNA) в~данном случае не могут 
быть использованы, поскольку они не позволяют со\-зда\-вать случайные сети 
с~произвольной плот\-ностью связей и~изучать их клас\-те\-ри\-за\-цию и~перколяцию 
при блокировании узлов.
  
  Для изучения случайных сетей с~множеством связей аналитических моделей 
описания перколяционных процессов не существует, и~их исследование 
воз\-мож\-но только методами чис\-лен\-но\-го моделирования~\cite{12-zh, 13-zh} 
с~использованием специально разработанного программного обеспечения. Для 
этого необходимо сначала по\-стро\-ить структурную модель социальной сети, 
со\-сто\-ящую из большого чис\-ла (например, в~данном исследовании~--- 1~млн) узлов. Затем вы\-брать пару произвольных узлов и~с~по\-мощью 
методов чис\-лен\-но\-го моделирования определить, при какой доле 
неблокированных узлов (для задачи узлов) в~рас\-смат\-ри\-ва\-емой сети появляется 
свободный путь между рас\-смат\-ри\-ва\-емы\-ми узлами (или, наоборот, исчезает при 
блокировании). Затем аналогичным образом эта процедура проводится для 
других произвольных пар узлов. После этого со статистическим усреднением 
результатов по отдельным экспериментам необходимо про\-вес\-ти определение 
сред\-не\-го значения порога перколяции по всем рас\-смат\-ри\-ва\-емым парам 
узлов~\cite{13-zh, 14-zh}. В~задаче связей используется практически такой же 
алгоритм исследования, однако блокируются не узлы, а~связи. 

\begin{figure*} %fig1
\vspace*{1pt}
 \begin{center}
 \mbox{%
 \epsfxsize=106.462mm 
 \epsfbox{zhu-1.eps}
 }
 \end{center}
\vspace*{-9pt}
\Caption{Зависимость вероятности не\-воз\-мож\-ности передачи информации между 
двумя произвольно вы\-бран\-ны\-ми узлами случайной сети от ве\-ро\-ят\-ности блокирования 
(разрыва) одной связи}
\end{figure*}

\vspace*{-6pt}

\section{Исследование кластеризации узлов социальных сетей 
и~достижения порога перколяции}

\subsection{Перколяция (протекание) информации в~сетях со~случайной 
структурой}

  Задача связей (разрываются связи между узлами) при определении порогов 
перколяции в~сети, име\-ющей случайную структуру, была решена авторами 
ранее в~работе~\cite{14-zh}. Результаты проведенного чис\-лен\-но\-го 
моделирования~\cite{13-zh, 14-zh} за\-ви\-си\-мости средней ве\-ро\-ят\-ности 
не\-воз\-мож\-ности передачи информации между двумя произвольно выбранными 
узлами случайной сети с~множеством путей между узлами и~различным 
сред\-ним чис\-лом связей на один узел от ве\-ро\-ят\-ности блокирования (разрыва) 
связи пред\-став\-ле\-ны на рис.~1. 
  
  


Кривая~\textit{1} построена для сети, у~которой среднее чис\-ло связей 
в~расчете на один узел со\-став\-ля\-ет~3,99, кривая~\textit{2}~--- 5,99, 
кривая~\textit{3}~--- 7,97, кривая~\textit{4}~--- 9,93, кривая~\textit{5}~--- 13,86 
и~кривая~\textit{6}~--- 15,79~связей. Основной задачей исследования является 
определение порогов перколяции для сетей с~различным средним чис\-лом 
связей в~расчете на один узел, но в~задаче узлов при этом удаляются узлы, 
а~в~задаче связей~--- связи. Можно определить линейные участки в~цент\-ре 
кривых~\textit{1}--\textit{6} и~экстраполировать их до 
пересечения с~осью абсцисс (см.\ рис.~1), значения величин которых можно 
условно принять за величину нижний границы порога перколяции данной сети. 
  
  На рис.~2 пред\-став\-ле\-на зависимость величины нижней границы порога 
перколяции от среднего чис\-ла связей на один узел данной сети, найденная по 
описанной выше методике.

 { \begin{center}  %fig2
 \vspace*{9pt}
  \mbox{%
 \epsfxsize=79mm 
 \epsfbox{zhu-2.eps}
 }


\end{center}

\vspace*{-6pt}


\noindent
{{\figurename~2}\ \ \small{Зависимость порога перколяции в~случайной сети от среднего чис\-ла 
связей (плот\-ности сети) на один ее узел в~задаче связей}}
}

%\vspace*{9pt}


  

  Представленная на рис.~2 зависимость хорошо линеаризуется 
в~координатах $\ln P(x)$ от $z\hm=1/x$ (натуральный логарифм 
порога перколяции~--- величина, обратная среднему чис\-лу связей~$x$, 
приходящихся на один узел) и~поз\-во\-ля\-ет получить линейную за\-ви\-си\-мость 
$y\hm= -6{,}581z\hm-0{,}203$ со значением коэффициента корреляции, 
рав\-ным~0,992. 
  
  Здесь говорится о~пороге перколяции как невозможности передачи 
информации при некоторой ве\-ро\-ят\-ности разрыва связи. Например, если 
в~данном случае для сети с~плот\-ностью связей~3,99 порог перколяции 
равен~0,16, то это значит, что если разорвать меньше~16\% связей, то сеть 
информацию передает, а~если~16\% или больше, то передача информации 
в~сети в~целом прекращается. Но это будет ниж\-няя оценка. Заметим, что 
другие методы будут давать большее значение величины порога (верхнюю 
оценку). Например, если за осно\-ву взять ве\-ро\-ят\-ность невозможности передачи 
информации, равную~0,5, которая показана на рис.~1 горизонтальной штриховой
линией, 
то получим сле\-ду\-ющие значения порогов перколяции: для среднего чис\-ла 
связей~3,99~--- 0,46; для~5,99~--- 0,60; для~7,97~--- 0,66; для~9,93~--- 0,69; 
для~13,86~--- 0,73; для~15,79~--- 0,75.



  
  Задача нахождения порога перколяции не для разорванных связей, а~для 
блокированных узлов в~случайной сети была решена в~работе~\cite{12-zh}. 
Результаты чис\-лен\-но\-го моделирования нахождения порога перколяции для 
случайных сетей с~множеством путей между узлами и~различным средним 
чис\-лом связей на один узел также хорошо линеаризуются в~координатах $\ln 
P(x)$ от $z\hm=1/x$ (натуральный логарифм порога 
пер\-ко\-ля\-ции~--- величина, обратная сред\-не\-му чис\-лу связей~$x$ (плот\-ность 
сети), приходящихся на один узел) и~позволяют получить линейную 
за\-ви\-си\-мость $y\hm=4{,}39z\hm-2{,}41$ со значением коэффициента 
корреляции, рав\-ным~0,95.
  
  Здесь речь идет о~пороге перколяции уже как о~воз\-мож\-ности передачи 
информации при некоторой ве\-ро\-ят\-ности активации узлов (узел становится 
проводящим). Например, если в~дан\-ном случае для сети с~плот\-ностью 
связей~4,70 порог перколяции равен~0,27, то это значит, что если будет 
активировано меньше~27\%~узлов, то сеть информацию не передаст, 
а~если~27\% или больше, то передача информации в~сети в~целом возникает. 
  
  Проанализируем полученные данные. Выберем для задачи узлов и~задачи 
связей в~качестве примера четыре значения плот\-ности сети (среднего чис\-ла 
связей, приходящихся на один узел): 5, 10, 50 и~100. Для задачи узлов 
используем урав\-не\-ние $y\hm=4{,}39z\hm-2{,}41$ и~при $z \hm=1/5\hm=0{,}2$ 
получим $y\hm= -1{,}532$, и~величина порога пер\-ко\-ля\-ции будет рав\-на~0,22 
(в~данном случае доля проводящих узлов, при которой появляется 
про\-во\-ди\-мость). Рас\-чет по урав\-не\-нию $y\hm= -6{,}581z\hm- 0{,}203$ для задачи 
связей при $z\hm=0{,}2$ дает величину доли разорванных связей, при которой 
исчезает про\-во\-ди\-мость всей сети в~целом, рав\-ную~0,22 (т.\,е., если 
разорвем~22\% связей и~более, про\-во\-ди\-мость исчезнет).
  
  Полученные результаты позволяют сделать  ряд важ\-ных выводов
  для перколяционных процессов 
в~случайных сетях. Они приводятся в~разд.~5.
  
\subsection{Кластеризация социальной сети}

  Поскольку проводимость узлов в~большей степени определяет решение 
информационных задач всей сетью в~целом, то будем рас\-смат\-ри\-вать задачу 
узлов. Переход любого узла социальной сети из одного со\-сто\-яния в~другое 
можно рас\-смат\-ри\-вать как случайный процесс (с~некоторой вероятностью 
перехода, опре\-де\-ля\-емой множеством случайных факторов), и~эта ве\-ро\-ят\-ность 
должна влиять на средний раз\-мер клас\-те\-ра (группа напрямую связанных между 
собой узлов). В~проведенных авторами чис\-лен\-ных 
 экспериментах~\cite{12-zh, 14-zh} было исследовано влияние вероятности 
перехода узла из одного со\-сто\-яния в~другое (например, желания голосовать 
<<за>> или <<против>>) на средний раз\-мер клас\-те\-ра пользователей, 
находящихся в~данном со\-сто\-янии (в~долях от общего чис\-ла всех 
индивидуумов) в~случайных сетях с~различным средним чис\-лом связей на один 
узел и~общим чис\-лом узлов, равном~1~млн. Можно рас\-смот\-реть два 
предельных случая: первый~--- небольшое среднее чис\-ло связей на 
один узел социальной сети; второй~--- большое среднее 
чис\-ло связей. Исследования показали, что с~рос\-том среднего чис\-ла связей при 
фиксированной ве\-ро\-ят\-ности воздействия раз\-мер клас\-те\-ра увеличивается. 
Аналогичная ситуация наблю\-да\-ет\-ся и~при большом среднем чис\-ле связей на 
один узел случайной сети.
  
  С ростом среднего числа связей при фиксированной вероятности воздействия 
размер кластера увеличивается, а~ско\-рость рос\-та клас\-те\-ри\-за\-ции узлов, 
находящихся в~данном со\-сто\-янии, наиболее сильно увеличивается в~об\-ласти 
значений ве\-ро\-ят\-ности перехода единичных узлов от~0,4 до~0,6, а~при малых 
и~высоких значениях воз\-рас\-та\-ет не так \mbox{сильно}.

\vspace*{-6pt}
  
  \section{Выводы}
  
  \noindent
  \begin{enumerate}[1.]
\item Как в~задаче связей, так и~в задаче узлов величина порога перколяции 
случайной сети зависит от ее плот\-ности (среднего чис\-ла узлов, приходящихся 
на один узел).\\[-13pt]
\item В сетях, имеющих случайную структуру, пороги перколяции как в~задаче 
узлов, так и~в задаче связей при большой плот\-ности сети (среднее число связей 
на один узел) практически достигают величины насыщения (0,24~для задачи 
связей и~0,10 для задачи узлов) и~затем слабо зависят от нее. Величина 
насыщения порога перколяции в~задаче связей почти в~2,5~раза больше, чем 
в~задаче узлов.\\[-13pt] 
\item С точки зрения создания про\-во\-ди\-мости случайной сети в~целом 
образование про\-во\-дя\-щих связей в~задаче связей менее эффективно, чем 
образование проводящих узлов в~задаче узлов (например, при плотности сети, 
рав\-ной~5, для возникновения про\-во\-ди\-мости необходимо иметь долю 
проводящих узлов, равную~0,22, в~то время как доля проводящих связей 
должна быть равной~0,78).\\[-13pt] 
\item С ростом среднего чис\-ла связей (плот\-ности сети) при фиксированной 
ве\-ро\-ят\-ности воздействия раз\-мер клас\-те\-ра увеличивается, а~ско\-рость роста 
клас\-те\-ри\-за\-ции узлов, находящихся в~данном состоянии, наиболее сильно 
увеличивается в~области значений ве\-ро\-ят\-ности перехода единичных узлов 
от~0,4 до~0,6, а~при малых и~высоких значениях возрастает не так сильно.\\[-13pt] 
\item С точки зрения информационного влияния сети, име\-ющей случайную 
структуру, увеличение плот\-ности связей оказывает большее влияние, чем 
наличие отдельных <<центральных>> узлов, име\-ющих множество связей.\\[-13pt] 
\item В практическом плане полученные результаты могут быть применены 
в~междисциплинар-\linebreak\vspace*{-12pt}

\pagebreak

\noindent
 ных исследованиях, вклю\-чая информатику, ма\-те\-матическое 
моделирование и~экономику, с~при\-вле\-че\-ни\-ем социологических данных для 
прогнозирования поведения и~управ\-ле\-ния группами людей в~сетевых 
сообществах. Кроме того, полученные результаты дополняют и~расширяют 
применение методов и~подходов, принятых в~классической информатике, на 
описание социальных и~социотехнических сис\-тем, что может быть полезно для 
широкого круга исследователей, занимающихся изуче\-ни\-ем социальных 
сетевых струк\-тур.
\end{enumerate}

\vspace*{-6pt}

{\small\frenchspacing
 {%\baselineskip=10.8pt
 \addcontentsline{toc}{section}{References}
 \begin{thebibliography}{99}
 \bibitem{0-zh}
 \Au{Винер Н.} Кибернетика, или Управление и~связь в~животном и~машине~//
 Пер. с~англ.~--- 2-е изд.~--- М.: Наука, 1983. 344~с.
 (\Au{Wiener~N.} Cybernetics: Or control and communication in 
 the animal and the machine.~---
 2nd ed.~--- MIT Press, 1961. 212~p.)
\bibitem{1-zh}
\Au{Баканова С.\,А., Силкина~Г.\,Ю.} Процессы распространения знаний 
в~параметризованной сети информационных обменов~// На\-уч\-но-тех\-ни\-че\-ские 
ведомости Санкт-Пе\-тер\-бург\-ско\-го государственного политехнического 
университета. Экономические науки, 2015. №\,2(216). С.~133--146.
\bibitem{2-zh}
\Au{Сулимов П.\,А.} Методы машинного обучения для предсказания распространения 
инфекции в~сети~// Вестн. НГУЭУ, 2016. №\,1. С.~285--306.
\bibitem{3-zh}
\Au{Торопов Б.\,А.} Модель независимых каскадов распространения репоста 
в~онлайновой социальной сети~// Кибернетика и~программирование, 2016. №\,5.  
С.~61--67.
\bibitem{4-zh}
\Au{Тарасевич Ю.\,Ю.} Перколяции: теория, приложения, алгоритмы.~--- М.: Эдиториал 
УРСС, 2002. 112~с.
\bibitem{5-zh}
\Au{Лесько С.\,А., Жуков~Д.\,О., Самойло~И.\,В.} Математическое моделирование 
перколяционных процессов передачи данных и~потери работоспособности  
в~ин\-фор\-ма\-ци\-он\-но-вы\-чис\-ли\-тель\-ных сетях с~2D и~3D регулярной и~случайной 
структурой~// Качество. Инновации. Образование, 2013. №\,6(97). С.~42--50.
\bibitem{6-zh}
\Au{Лесько С.\,А., Жуков~Д.\,О., Самойло~И.\,В., Брукс~Д.\,У.} Алгоритмы построения 
сетей и~моделирования потери их работоспособности в~результате кластеризации 
блокированных узлов~// Качество. Инновации. Образование, 2013. №\,12(103). С.~82--87. 
\bibitem{7-zh}
\Au{Павлековская И.\,В.} Применение метода анализа социальных сетей в~моделировании 
процессов распространения информации и~знаний в~организации~//\linebreak  
На\-уч\-но-тех\-ни\-че\-ская информация. Сер.~2: Информационные процессы 
и~системы, 2007. №\,3. С.~30--36.
\bibitem{8-zh}
\Au{Chen Y., Paul~G., Cohen~R., Yavlin~S., Borgatti~S.\,P., Liljeros~F., Stanley~H.\,E.} 
Percolation theory and fragmentation measures in social networks~// Physica~A, 2006. Vol.~378. No.\,1. P.~11--19.
\bibitem{9-zh}
\Au{Kawamoto H., Takayasu~H., Jensen~H.\,J., Takayasu~M.} Precise calculation of a~bond 
percolation transition and survival rates of nodes in a~complex network~// 
PLoS One, 2015. {\sf https://doi.org/10.1371/journal.pone.0119979}.
\bibitem{10-zh}
\Au{Barab$\acute{\mbox{a}}$si A.\,L., Albert~R.} Emergence of scaling in random networks~// 
Science, 1999. Vol.~286. P.~509--512.
\bibitem{11-zh}
\Au{Albert R., Jeong~H., Barab$\acute{\mbox{a}}$si~A.\,L.} Error and attack tolerance of 
complex networks~// Nature, 2000. Vol.~406. P.~378--382.
\bibitem{12-zh}
\Au{Zhukov D., Lesko~S.} Percolation models of information dissemination in social 
networks~// IEEE Conference (International) on Smart City/SocialCom/SustainCom together 
with DataCom Proceedings.~--- IEEE, 2015. P.~213--216.
\bibitem{13-zh}
\Au{Block M., Khvatova~T., Zhukov~D., Lesko~S.} Studying the structural topology of the 
knowledge sharing network~// 11th European Conference on Management, Leadership and 
Governance Proceedings.~--- Lisbon, Portugal: Academic Conferences and 
Publishing International Ltd., 2015. P.~20--27.
\bibitem{14-zh}
\Au{Khvatova T., Block~M., Zhukov~D., Lesko~S.} How to measure trust: 
The percolation model 
applied to intra-organisational knowledge sharing networks~// J.~Knowl. Manag., 
2016. Vol.~20. No.\,5. P.~918--935.
%\bibitem{15-zh}
%\Au{Сигов А.\,С., Акимов~Д.\,А., Жуков~Д.\,О., Андрианова~Е.\,Г., Сачков~В.\,Е., 
%Раев~В.\,К.} Психолингвистический анализ русскоязычных текстовых сообщений на 
%основе их фоносемантических статистических характеристик~// Информатика и~её 
%применения, 2017. Т.~11. Вып.~3. С.~77--86.
 \end{thebibliography}

 }
 }

\end{multicols}

\vspace*{-7pt}

\hfill{\small\textit{Поступила в~редакцию 04.07.17}}

\vspace*{6pt}

%\newpage

%\vspace*{-28pt}

\hrule

\vspace*{2pt}

\hrule

\vspace*{-6pt}


\def\tit{THE INFLUENCE OF~THE~CONNECTIONS' DENSITY ON~CLUSTERIZATION AND~PERCOLATION 
THRESHOLD DURING~INFORMATION DISTRIBUTION IN~SOCIAL NETWORKS}

\def\titkol{The influence of~the~connections' density on~clusterization and~percolation 
threshold during~information distribution in %~social 
networks}


\def\aut{D.\,O.~Zhukov$^1$, T.\,Yu.~Khvatova$^2$, S.\,A.~Lesko$^1$, 
and~A.\,D.~Zaltsman$^1$}

\def\autkol{D.\,O.~Zhukov, T.\,Yu.~Khvatova, S.\,A.~Lesko, 
and~A.\,D.~Zaltsman}

\titel{\tit}{\aut}{\autkol}{\titkol}

\vspace*{-11pt}


\noindent
$^1$Moscow Technological University (MIREA), 78~Vernadskogo Ave., 
Moscow 119454, Russian Federation 

\noindent
$^2$Peter the Great St. Petersburg Polytechnic University, 
29~Polytechnicheskaya Str., St.\ Petersburg 195251, Russian\linebreak
$\hphantom{^1}$Federation


\def\leftfootline{\small{\textbf{\thepage}
\hfill INFORMATIKA I EE PRIMENENIYA~--- INFORMATICS AND
APPLICATIONS\ \ \ 2018\ \ \ volume~12\ \ \ issue\ 2}
}%
 \def\rightfootline{\small{INFORMATIKA I EE PRIMENENIYA~---
INFORMATICS AND APPLICATIONS\ \ \ 2018\ \ \ volume~12\ \ \ issue\ 2
\hfill \textbf{\thepage}}}

\vspace*{3pt}

 
  \Abste{The paper is focused on applying new theoretical approaches to describing 
  the processes of information transmission and transformation in sociotechnical 
  systems and in social networks based on the percolation theory.
  Percolation 
  threshold of a~random network depends on its density. In networks with random 
  structure, in both the\linebreak\vspace*{-12pt}}
  
  \Abstend{task of bonds and the task of nodes, percolation 
  thresholds reach saturation when the network's density is high. The saturation 
  value of a~percolation threshold is higher in the task of bonds. From the point 
  of information influence of a~random network, increasing the average connection's 
  density within the network turns out to be more preferable than fostering 
  a~small number of separate `central nodes' with numerous connections. The results 
  obtained in this study can be applied in interdisciplinary research in such 
  areas as informatics, mathematic modeling, and economics involving certain 
  sociological survey data for forecasting behavior and managing groups of individuals 
  in network communities. This research enhances and enlarges the scope of methods 
  and approaches applied in classic informatics for describing social 
  and sociotechnical systems, which can be useful for 
  a~wide range of researchers engaged into studying social network structures.}

\KWE{percolation theory; social network structure; connections' density; 
network clusterisation; percolation threshold}



\DOI{10.14357/19922264180213} %

\vspace*{-19pt}

 \Ack
 
 \vspace*{-3pt}
 
\noindent
This research was performed with financial support of the Russian Foundation for 
Basic Research (project No.\,16-29-09458~ofi\_m) ``Developing percolation 
topological models for describing virtual social systems, their participants' 
clusterization into groups according to their views, stochastic dynamics of influence 
distribution, and for managing transitions.''



%\vspace*{2pt}

  \begin{multicols}{2}

\renewcommand{\bibname}{\protect\rmfamily References}
%\renewcommand{\bibname}{\large\protect\rm References}

{\small\frenchspacing
 {%\baselineskip=10.8pt
 \addcontentsline{toc}{section}{References}
 \begin{thebibliography}{99}
 
\bibitem{0-zh-1}
\Aue{Wiener,~N.} 1961. \textit{Cybernetics: Or control and communication in 
the animal and the machine}.
 2nd ed. MIT Press. 212~p.
\bibitem{1-zh-1}
\Aue{Bakanova, S.\,A., and G.\,Yu.~Silkina.} 2015. 
Knowledge dissemination process in parametrized networks of enterprises. 
\textit{St.\ Petersburg State Polytechnical University~J. Economics}
2(216):133--146.
\bibitem{2-zh-1}
\Aue{Sulimov, P.\,A.} 2016. Metody ma\-shin\-no\-go obucheniya dlya pred\-ska\-za\-niya 
ras\-pro\-stra\-ne\-niya in\-fek\-tsii v~seti [Methods of machine learning for forecasting an 
infection distribution within a~network]. \textit{Vestnik NGUEU} [NGUEU~J.] 
1:285--306.
\bibitem{3-zh-1}
\Aue{Toropov, B.\,A.} 2016. Model' ne\-za\-vi\-si\-mykh kas\-ka\-dov ras\-pro\-stra\-ne\-niya 
re\-pos\-ta v~on\-lay\-no\-voy so\-tsi\-al'\-noy seti [The model of independent cascades of 
a~repost distribution in an online social network]. \textit{Kibernetika 
i~programmirovanie} [Cybernetics and Programming] 5:61--67.

\bibitem{4-zh-1}
\Aue{Tarasevich, Yu.} 2002. \textit{Perkolyatsii: teoriya, prilozheniya, algoritmy} 
[Percolations: Theory, applications, algorithms]. Moscow: Editorial URSS. 112~p.
\bibitem{5-zh-1}
\Aue{Les'ko, S., D.~Zhukov, and I.~Samoylo.} 2013. Ma\-te\-ma\-ti\-che\-skoe 
mo\-de\-li\-ro\-va\-nie per\-ko\-lya\-tsi\-on\-nykh pro\-tses\-sov 
peredachi dan\-nykh i~poteri 
ra\-bo\-to\-spo\-sob\-nosti v~in\-for\-ma\-tsi\-on\-no-vy\-chis\-li\-tel'\-nykh setyakh s~2D 
i~3D re\-gu\-lyar\-noy i~slu\-chay\-noy struk\-tu\-roy [Mathematic modeling of percolation 
processes of data transmission and operability loss in informational-computational 
networks with 2D and 3D regular and random structure]. \textit{Kachestvo. 
Innovatsii. Obrazovanie} [Quality. Innovations. Education] 11(97):42--50.

\bibitem{6-zh-1}
\Aue{Les'ko, S., D.~Zhukov, I.~Samoylo, and D.~Bruks.} 2013b. Algoritmy 
po\-stro\-eniya se\-tey i~mo\-de\-li\-ro\-va\-niya poteri ikh 
ra\-bo\-to\-spo\-sob\-nosti v~re\-zul'\-ta\-te 
klas\-te\-ri\-za\-tsii blo\-ki\-ro\-van\-nykh uz\-lov 
[Algorithms of network construction and 
modeling of their operability loss as a~result of blocked nodes clusterization]. 
\textit{Kachestvo. Innovatsii. Obrazovanie} [Quality. Innovations. Education] 
12(103):25--33. 
\bibitem{7-zh-1}
\Aue{Pavlekovskaya, I.\,V.} 2007. The use
of social network analysis in modeling the organizational
processes of 
 information and knowledge circulation.  
\textit{Automatic Documentation Math. Linguistics} 41(2):65--72.

\bibitem{8-zh-1}
\Aue{Chen, Y., G.~Paul, R.~Cohen, S.~Yavlin, S.\,P.~Borgatti, F.~Liljeros, and 
H.\,E.~Stanley.} 2006. Percolation theory and fragmentation measures in social networks. 
\textit{Physica~A} 378(1):11--19.

\bibitem{9-zh-1}
\Aue{Kawamoto, H., H.~Takayasu, H.\,J.~Jensen, and M.~Ta\-ka\-yasu.} 2015. Precise 
calculation of a~bond percolation transition and survival rates of nodes in a~complex 
network. \textit{PLoS One}. Available at:  {\sf 
http://journals. plos.org/plosone/article?id=10.1371/journal.pone.\linebreak 0119979} (accessed 
December~16, 2017).


\bibitem{10-zh-1}
\Aue{Barab$\acute{\mbox{a}}$si, A.\,L., and R.~Albert.} 1999. Emergence of 
scaling in random networks. \textit{Science} 286:509--512.
\bibitem{11-zh-1}
\Aue{Albert, R., H.~Jeong, and A.\,L.~Barab$\acute{\mbox{a}}$si.} 2000. Error 
and attack tolerance of complex networks. \textit{Nature} 406:378--382.
\bibitem{12-zh-1}
\Aue{Zhukov, D., and S.~Lesko.} 2015. Percolation models of information 
dissemination in social networks. \textit{IEEE Conference (International) on Smart 
City/SocialCom/SustainCom together with DataCom Proceedings}. IEEE.
213--216.
\bibitem{13-zh-1}
\Aue{Block, M., T.~Khvatova, D.~Zhukov, and S.~Lesko.} 2015. Studying the 
structural topology of the knowledge sharing network. \textit{11th European 
Conference on Management, Leadership and Governance Proceedings}. 
Lisbon, Portugal: Academic Conferences and 
Publishing International Ltd. 20--27. 
\bibitem{14-zh-1}
\Aue{Khvatova, T., M.~Block, D.~Zhukov, and S.~Lesko.} 2016. How to measure 
trust: The percolation model applied to intra-organisational knowledge sharing 
networks. \textit{J.~Knowl. Manag.} 20(5):918--935.
%\bibitem{15-zh-1}
%\Aue{Sigov, A.\,S., D.\,A.~Akimov,  D.\,O.~Zhukov, E.\,G.~An\-drianova, 
%V.\,E.~Sachkov, and V.\,K.~Raev.} 2017. Psi\-kholing\-vi\-sti\-che\-skiy analiz 
%russkoyazychnykh tekstovykh soobshche\-niy na osnove ikh fonosemanticheskikh 
%statisticheskikh kharakteristik [Psycho-linguistic analysis of Russian text messages 
%based on their phonosemantic statistical characteristics]. \textit{Informatika i~ee 
%Primeneniya~--- Informatics Appl.} 11(3):77--86.
\end{thebibliography}

 }
 }

\end{multicols}

\vspace*{-9pt}

\hfill{\small\textit{Received July 4, 2017}}

%\pagebreak

%\vspace*{-24pt}


\Contr

\noindent
\textbf{Zhukov Dmitry O.} (b.\ 1965)~--- Doctor of Science in technology, 
professor, Head of Department, Moscow Technological University (MIREA), 
78~Vernadskogo Ave., Moscow 119454, Russian Federation; 
\mbox{zhukov\_do@mirea.ru}

\vspace*{6pt}

\noindent
\textbf{Khvatova Tatiana Yu.} (b.\ 1971)~---  Doctor of Science in economics, 
professor, Peter the Great St.\ Petersburg Polytechnic University, 
29~Polytechnicheskaya Str., St.\ Petersburg 195251, Russian Federation; 
\mbox{khvatova.ty@spbstu.ru}

\vspace*{6pt}

\noindent
\textbf{Lesko Sergey A.} (b.\ 1983)~--- Candidate of Science (PhD) in technology, 
associate professor, Moscow Technological University (MIREA), 78~Vernadskogo 
Ave., Moscow 119454, Russian Federation; \mbox{sergey@testor.ru}

\vspace*{6pt}

\noindent
\textbf{Zaltsman Anastasia D.} (b.\ 1989)~--- lecturer, 
Moscow Technological University (MIREA), 78~Vernadskogo Ave., Moscow 119454, Russian 
Federation; \mbox{ad.zaltcman@gmail.com} 

\label{end\stat}


\renewcommand{\bibname}{\protect\rm Литература} 