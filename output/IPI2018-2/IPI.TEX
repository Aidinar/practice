\documentclass[10pt]{book}
\usepackage[utf8]{inputenc}

\usepackage{latexsym,amssymb,amsfonts,amsmath,amsxtra,indentfirst,shapepar,%fleqn,%
picinpar,shadow,floatflt,enumerate,multicol,colortbl,moreverb,cite,ipi}

\usepackage{rotating}
\usepackage{mathrsfs}
\usepackage[noend]{algorithmic}
\usepackage{ulem}
\usepackage{graphicx}
%\usepackage{algorithm2e}
\usepackage[linesnumbered,boxed,ruled]{algorithm2e}
%\usepackage{xypic}
\usepackage{oldgerm}
\usepackage{epic}
\usepackage{eepic}


\SetAlgorithmName{Алгоритм}{алгоритм}{Список алгоритмов}

%из Дюковой

\newcommand{\algKeyword}[1]{{\bf #1}}
\newcommand{\Proc}[1]{\text{\tt #1}}
\def\CALL{\algKeyword{call}~}

\newenvironment{AlgProcedure}[1]
{
    \small
    \medskip
    %    \hrule
    \medskip
    \algKeyword{PROCEDURE} #1
    \begin{algorithmic}[1]}
    {\end{algorithmic}
    %    \hrule
    \bigskip
}

\def\CALL{\algKeyword{call}~}

%конец для Дюковой

%\RequirePackage[ruled]{algorithm}


\input{epsf}

%\nofiles

%\includeonly{avtor} %+pdf+
%\includeonly{obchak,avtor}
%\includeonly{pred}                 %+
%\includeonly{podgot-rus-site,podgot-eng-site}  
%\includeonly{ocherk} 
%\includeonly{nekrol} 
%\includeonly{ipi-ind} 
%\includeonly{toc-rus}
%\includeonly{toc-en} 


%\includeonly{leri}                  %+1pdf+авт
%\includeonly{shestakov-1}    %2без рис+pdf+авт
%\includeonly{kudr-shest}     %3без рис+pdf+авт
%\includeonly{ushakov}        %4без рис+pdf+авт
%\includeonly{nazarov}        %+5без рисpdf+авт
%\includeonly{lebedev}        %+6без рисpdf
%\includeonly{grusho}         %7без рис+pdf
%\includeonly{tyrsin}                %+8pdf+авт
%\includeonly{vaskan}                %+9pdf+авт
%\includeonly{maniakov}              %+10pdf+авт
%\includeonly{ogaltsov}              %+11pdf+авт
%\includeonly{shnurkov}       %12+без рисpdf+авт
%\includeonly{zhukov}         %+13pdf
%\includeonly{mizarbekov}     %14+арабские буквыpdf+авт
%\includeonly{nuriev}         %15+pdf




%\includeonly{toc-rus, toc-en}
%\includeonly{obchak} %,toc-en}
%\includeonly{rekl}
%\includeonly{rekl-1}
%\includeonly{reshal}  %
%\includeonly{eng-index}
%\includeonly{cover3}

\usepackage{acad}
%\usepackage{courier}
\usepackage{decor}
\usepackage{newton}
\usepackage{pragmatica}
\usepackage{zapfchan}
\usepackage{petrotex}
\usepackage{bm}                     % полужирные греческие буквы
\usepackage{upgreek}                % прямые греческие буквы
\usepackage{eufrak}
\usepackage{verbatim}

\renewcommand{\bottomfraction}{0.99}
\renewcommand{\topfraction}{0.99}
\renewcommand{\textfraction}{0.01}

\setcounter{secnumdepth}{1} %здесь - 3 + chapter = 4

\arraycolsep=1.5pt

%\usepackage[pdftex]{graphicx}

%\usepackage{oz}

%NEW COMMANDS


\renewcommand*{\hm}[1]{#1\nobreak\discretionary{}%
            {\hbox{$\mathsurround=0pt #1$}}{}} %% Дублирует знаки операций
                               %при переносе в формуле (перед знаком, который
                               %надо продублировать ставится команда \hm)

%\newcommand{\endproof}{\hfill$\Box$}
\renewcommand{\r}{\mathbb{R}}
%\newcommand{\I}{{\rm I\hspace{-0.7mm}I}}
%\newcommand{\Ikl}{{\tt{1}}\hspace*{-1.44mm}\mathtt{1}}
\newcommand{\Ik}{\mbox{{\small \tt {1}}\hspace{-1.3mm}{\tt 1}}}
\newcommand{\argmin}{\mathop{\mathrm{arg}\,\mathrm{min}}}
\newcommand{\argmax}{\mathop{\mathrm{arg}\,\mathrm{max}}}
%\newcommand{\capr}{\mathop{\cap\,}}
%\newcommand{\cupr}{\mathop{\cup\,}}
%\def\argmin{\mathop{arg\,min}}

\def\vrp{\varphi}
\def\prt{\partial}
\def\mm{{\sf M}}
\def\modnop#1{\mathop{#1}\limits_{n}}
\def\eam{\mathbin{{\mathop{=}\limits^{\mathrm{def}}}}}
\def\dey#1#2{#1 (#2)}
\def\deyc#1#2{#1 \cdot  #2}
\def\ra#1{\;\mathop{\to}\limits^{#1}\;}
\def\raz#1{\;\mathop{\longrightarrow}\limits^{\!\!\!#1}\;}
\def\ral#1{\;\mathop{\longrightarrow}\limits^{#1}\;}

\newcommand{\Nor}{\mathcal{N}}
\newcommand{\T}{\mathbb{T}}
\newcommand{\Z}{\mathbb{Z}}



\newcommand{\il}[2]{\int\limits_{#1}^{#2}}%интеграл с пределами #1 и #2

\def\sm2{\mathop {\sum\limits^{n^\Theta}\sum\limits^{n^\Theta}}}
\def\sss{\sum\limits}
\def\tr{,\,\ldots\,,\,}
\def\rk{\right]}
\def\lk{\left[}
\def\rf{\right\}}
\def\lf{\left\{}
\def\lv{\,\left\vert}
\def\rv{\right\vert\,}
\def\iii{\int\limits}
\def\iin{\int\limits_{-\infty}^\infty}
\def\rrv{\right\vert}


\def\ee{{\cal E}}
\def\ww{{\cal W}}
\def\yy{{\cal Y}}
\def\vv{{\cal V}}

\newcommand{\R}{\mathbb R}
\newcommand{\E}{\mathbb E}
\newcommand{\N}{\mathbb N}

\renewcommand{\P}{\mathbb{P}}

\newcommand{\h}{{\bf H}}
\newcommand{\p}{{\sf P}}  % вероятность

\newcommand{\e}{{\sf E}}  % мат. ожидание
\newcommand{\D}{{\sf D}}  % дисперсия
\newcommand{\eps}{\varepsilon}
\newcommand{\vp}{{\mathbf p}}
\newcommand{\vz}{{\mathbf z}}
\newcommand{\vx}{{\mathbf x}}
\newcommand{\vf}{{\mathbf f}}
\newcommand{\F}{{\mathcal F}}
\def\ap{{\mathrm{ЭР}}}
\newcommand{\ud}{\Delta_n} %uniform ditance
\newcommand{\nud}{\Delta_n(x)}
%\renewcommand{\Re}{\mathrm{Re}\,}

\newcommand{\abs}[1]{\left\vert#1\right\vert}

\newcommand{\norm}[1]{\left\Vert#1\right\Vert}
\def\da{(\Delta_t,A)}

\newcommand{\corr}{\mathrm{corr}}

\newcommand{\cov}{\mathrm{cov}}
\newcommand{\Expect}{\mathbb{E}}

\def\w{\omega}
\def\W{\Omega}

\def\inh{\int\limits_{nh}^{(n+1)h}}

\def\sumin{\sum_{i=1}^N}


\def\bxt{(Y,t)}
\def\xt{(y,t)}

\def\ovth{{\fr{\tau-nh}{h}}}
\def\ov{\overline}
\def\tm{\tilde m}
\def\tl{\tilde\lambda}
\def\tB{\widetilde B}
\def\tb{\tilde b}
\def\ld{\ldots}
\def\cd{\cdots}


\DeclareMathOperator{\sign}{sign}

%\newcommand{\gr}{{\geqslant}}


\newcommand{\g}{\mbox{\textit{g}}}

\renewcommand{\la}{\lambda}
\newcommand{\si}{\sigma}
\newcommand{\alp}{\alpha}

%\newcommand{\pto}{\stackrel{P}{\longrightarrow}} % сходимость по веpоятности

\newcommand{\eqd}{\stackrel{\mathrm{d}}{=}} % равенство по pаспpеделению
\newcommand{\eqdelta}{\stackrel{\triangle}{=}} % равенство по pаспpеделению

\def\be#1{\begin{equation}\label{#1}}
\def\ee{\end{equation}}
\def\re#1{(\ref{#1})}

\def\bn{\begin{enumerate}}
\def\en{\end{enumerate}}
\def\bi{\begin{itemize}}
\def\ei{\end{itemize}}
%\def\i{\item}

%\newcommand{\kp}{\kappa}
%\def\Q{{\cal Q}} \def\H{{\cal H}}
%\newcommand{\bet}{\beta_{2+\delta}}


%\newtheorem{definition}{Определение}
%\renewcommand{\thedefinition}{\arabic{definition}.}
%END NEW COMMANDS

%\renewcommand{\baselinestretch}{1.2}

%\pagestyle{myheadings}

\setlength{\textwidth}{167mm}      % 122mm
\setlength{\textheight}{658pt}
%\setlength{\textheight}{635.6pt}
\setlength{\columnsep}{4.5mm}

\setcounter{secnumdepth}{4}

%\addtolength{\headheight}{2pt}
%\addtolength{\headsep}{-2mm}

\addtolength{\topmargin}{-7mm}  % for printing


%\hoffset=-30mm  % From Yap
\hoffset=-23mm  % From Acrobat

%\voffset=0mm % From Yap
\voffset=-5mm   % From Acrobat

%\addtolength{\evensidemargin}{-2.5mm} % for printing
%\addtolength{\oddsidemargin}{2.5mm}  % for printing

\addtolength{\evensidemargin}{-12mm} % for printing
\addtolength{\oddsidemargin}{8mm}  % for printing

%\renewcommand{\thefootnote}{\fnsymbol{footnote}}
%\renewcommand{\thefootnote}{\arabic{footnote}}
\renewcommand{\figurename}{\protect\bf Рис.}
\renewcommand{\tablename}{\protect\bf Таблица}

\newcommand{\Caption}[1]{\caption{\protect\small %\baselineskip=2.5ex
#1}}

\renewcommand{\thefigure}{\arabic{figure}}
\renewcommand{\thetable}{\arabic{table}}
\renewcommand{\theequation}{\arabic{equation}}
\renewcommand{\thesection}{\arabic{section}}

\renewcommand{\contentsname}{СОДЕРЖАНИЕ}
\newcommand{\fr}[2]{\displaystyle\frac{\displaystyle #1\mathstrut}{\displaystyle #2\mathstrut}}

%\renewcommand{\thefootnote}{\fnsymbol{footnote}}
%\newcommand{\g}{\mbox{\textit{g}}}

%\newcommand{\Caption}[1]{\caption{\protect\small\baselineskip=2ex #1}}
\newcounter{razdel}
\setcounter{razdel}{0}


\newcommand{\titel}[4]{%
\

\vspace*{5pt}

\ifodd\therazdel {\raggedright\noindent\Large\textrm\textbf
 \lineskip .75em
  \baselineskip=3.2ex #1 \par}
\vskip 1em {\noindent\large\textrm\textbf #2 \par}
\addcontentsline{toc}{subsection}{{\textrm\textbf #1}\protect\newline #2}
\def\rightheadline{\underline{\noindent\hbox to \textwidth{\hfill\small\textrm{#4}
%\hfill \large\bf\thepage
}}}
\def\leftheadline{\underline{\noindent\parbox{\textwidth}{
%\raggedleft\large\bf\thepage \hfill
\small\textit{#3}\hfill}}}
\def\leftfootline{\small{\textbf{\thepage}
\hfill ИНФОРМАТИКА И ЕЁ ПРИМЕНЕНИЯ\ \ \ том~12\ \ \ выпуск 2\ \ \ 2018}
}%
 \def\rightfootline{\small{ИНФОРМАТИКА И ЕЁ ПРИМЕНЕНИЯ\ \ \ том~12\ \ \ выпуск~2\ \ \ 2018
\hfill \textbf{\thepage}}}
\vskip 2em \setcounter{figure}{0}
\setcounter{table}{0}
\setcounter{equation}{0}
\setcounter{section}{0}
\setcounter{subsection}{0}
\setcounter{subsubsection}{0}
\setcounter{footnote}{0}
\setcounter{razdel}{0}
%\end{flushleft}
\else {
 \raggedright\noindent\Large\textrm\textbf
 \lineskip .75em
\baselineskip=3.2ex #1 \par} \vskip 1em
%\begin{flushleft}
{\noindent\large\textrm\textbf #2 \par}
\addcontentsline{toc}{subsection}{{\textrm\textbf #1}\protect\newline #2}
\def\rightheadline{\underline{\noindent\hbox to \textwidth{\hfill\small\textrm{#4}
%\hfill \large\bf\thepage
}}}
\def\leftheadline{\underline{\noindent\parbox{\textwidth}{%\raggedleft\large\bf\thepage \hfill
\small\textit{#3}\hfill}}}
\def\leftfootline{\small{\textbf{\thepage}
\hfill ИНФОРМАТИКА И ЕЁ ПРИМЕНЕНИЯ\ \ \ том~12\ \ \ выпуск~2\ \ \ 2018}
}%
 \def\rightfootline{\small{ИНФОРМАТИКА И ЕЁ ПРИМЕНЕНИЯ\ \ \ том~12\ \ \ выпуск~2\ \ \ 2018
\hfill \textbf{\thepage}}} \vskip 2em \setcounter{figure}{0}
\setcounter{table}{0} \setcounter{equation}{0} \setcounter{section}{0}
\setcounter{subsection}{0} \setcounter{subsubsection}{0}
\setcounter{footnote}{0}
%\end{flushleft}
\fi}

\newcommand{\titelr}[2]{%
\

\vspace*{5pt}

\ifodd\therazdel {\raggedright\noindent%\Large\textrm\textbf
 \lineskip .75em
  \baselineskip=3.2ex #1 \par}
\vskip 1em {\noindent\normalsize\textrm\textbf #2 \par}
\else {
 \raggedright\noindent\Large\textrm\textbf
 \lineskip .75em
\baselineskip=3.2ex #1 \par} \vskip 1em
%\begin{flushleft}
{\noindent\large\textrm\textbf #2 \par
%\noindent\normalsize\textrm\textbf #2 \par
} \fi}

\newcommand{\titele}[5]{%
\

%\vspace*{5pt}

\ifodd\therazdel {\raggedright\noindent\large
\textrm\textbf
 \lineskip .75em
%  \baselineskip=3.2ex
#1 \par}
\vskip .5em {\noindent\large\textrm\textbf #2 \par}
\vskip .5em
 {\noindent\textrm #3 \par}
\addcontentsline{toc}{subsection}{{\textrm\textbf #1}\protect\newline #2}
\def\rightheadline{\underline{\noindent\hbox to \textwidth{\hfill\small\textrm{#4}
%\hfill \large\bf\thepage
}}}
\def\leftheadline{\underline{\noindent\parbox{\textwidth}{
%\raggedleft\large\bf\thepage \hfill
\small\textrm{#5}\hfill}}}
\def\leftfootline{\small{\textbf{\thepage}
\hfill ИНФОРМАТИКА И ЕЁ ПРИМЕНЕНИЯ\ \ \ том~12\ \ \ выпуск~2\ \ \ 2018}
}%
 \def\rightfootline{\small{ИНФОРМАТИКА И ЕЁ ПРИМЕНЕНИЯ\ \ \ том~12\ \ \ выпуск~2\ \ \ 2018
\hfill \textbf{\thepage}}} \vskip 1em \setcounter{figure}{0}
\setcounter{table}{0} \setcounter{equation}{0} \setcounter{section}{0}
\setcounter{subsection}{0} \setcounter{subsubsection}{0}
\setcounter{footnote}{0} \setcounter{razdel}{0}
%\end{flushleft}
\else {
 \raggedright\noindent\large
 \textrm\textbf
 \lineskip .75em
%\baselineskip=3.2ex
#1 \par} \vskip .5em
%\begin{flushleft}
{\noindent\large\textrm\textbf #2 \par} \vskip .5em
 {\noindent\textrm #3 \par}
\addcontentsline{toc}{subsection}{{\textrm\textbf #1}\protect\newline #2}
\def\rightheadline{\underline{\noindent\hbox to \textwidth{\hfill\small\textrm{#4}
%\hfill \large\bf\thepage
}}}
\def\leftheadline{\underline{\noindent\parbox{\textwidth}{%\raggedleft\large\bf\thepage \hfill
\small\textrm{#5}\hfill}}}
\def\leftfootline{\small{\textbf{\thepage}
\hfill ИНФОРМАТИКА И ЕЁ ПРИМЕНЕНИЯ\ \ \ том~12\ \ \ выпуск~2\ \ \ 2018}
}%
 \def\rightfootline{\small{ИНФОРМАТИКА И ЕЁ ПРИМЕНЕНИЯ\ \ \ том~12\ \ \ выпуск~2\ \ \ 2018
\hfill \textbf{\thepage}}} \vskip 1em \setcounter{figure}{0}
\setcounter{table}{0} \setcounter{equation}{0} \setcounter{section}{0}
\setcounter{subsection}{0} \setcounter{subsubsection}{0}
\setcounter{footnote}{0}
%\end{flushleft}
\fi}

\def\Abst#1{
\begin{center}\small\nwt
\parbox{150mm}{%\baselineskip=2.5ex
\textbf{Аннотация:}\ \
%\hspace*{\parindent}
#1}
\end{center}}
\def\Abste#1{
\begin{center}\small\nwt
\parbox{150mm}{%\baselineskip=2.5ex
\textbf{Abstract:}\ \
%\hspace*{\parindent}
#1}
\end{center}}

\def\DOI#1{
\begin{center}\small\nwt
\parbox{150mm}{%\baselineskip=2.5ex
\textbf{DOI:}\ \
%\hspace*{\parindent}
#1}
\end{center}}

\def\Abstend#1{
\begin{center}\small\nwt
\parbox{150mm}{%\baselineskip=2.5ex
%\hspace*{\parindent}
#1}
\end{center}}


\def\KW#1{
\begin{center}\small\nwt
\parbox{150mm}{%\baselineskip=2.5ex
\textbf{Ключевые слова:}\ \ #1}
\end{center}}

\def\KWE#1{
\begin{center}\small\nwt
\parbox{150mm}{%\baselineskip=2.5ex
\textbf{Keywords:}\ \ #1}
\end{center}}


\def\KWN#1{
%\begin{center}
%\small
%\parbox{150mm}\end{center}
}

\newcommand{\Avtors}[1]{%\smallskip
%\vspace*{.5pt}
\hangindent=23pt\noindent
%\nwt
{\bfseries#1}\
}


\renewcommand{\thesubsection}{\thesection.\arabic{subsection}\hspace*{-5pt}}
\renewcommand{\thesubsubsection}{\thesubsection\hspace*{5pt}.\arabic{subsubsection}\hspace*{-3pt}}

\newcommand{\Ack}{\section*{\protect\rmfamily Acknowledgments}\noindent}
\newcommand{\Contr}{\section*{\protect\rmfamily Contributors}\noindent}
\newcommand{\Contrl}{\section*{\protect\rmfamily Contributor}\noindent}

\makeindex


\begin{document}
\Rus

\nwt
%\ptb


%\renewcommand{\contentsname}{\protect\Large\bf Содержание}

\setcounter{tocdepth}{2}

%\tableofcontents

\renewcommand{\bibname}{\protect\rmfamily Литература}
  \def\Au#1{{\it #1}}
    \def\Aue#1{{#1}}

%\newcommand{\No}{№}
  \newcommand{\tg}{\,\mathrm{tg}\,}
    \newcommand{\ctg}{\,\mathrm{ctg}\,}
  \newcommand{\arctg}{\,\mathrm{arctg}\,}

\def\forallb{\mathop{\forall}}
\def\cupb{\mathop{\cup}}
\def\existsb{\mathop{\exists}}


\newpage
\addtocounter{razdel}{1}
%\def\razd{РЕГУЛИРУЕМЫЙ ЭЛЕКТРОПРИВОД ДЛЯ ЭЛЕКТРОЭНЕРГЕТИКИ}


\setcounter{page}{2}

%   { %\Large  
   { %\baselineskip=16.6pt
   
   \vspace*{-48pt}
   \begin{center}\LARGE
   \textit{Предисловие}
   \end{center}
   
   %\vspace*{2.5mm}
   
   \vspace*{25mm}
   
   \thispagestyle{empty}
   
   { %\small 

    
Вниманию читателей журнала <<Информатика и её применения>> предлагается 
очередной тематический выпуск <<Вероятностно-статистические методы и 
задачи информатики и информационных технологий>>. Предыдущие тематические 
выпуски журнала по данному направлению вышли в 2008~г.\ (т.~2, вып.~2), 
в 2009~г.\ (т.~3, вып.~3) и в 2010~г.\ (т.~4, вып.~2). 

Статьи, собранные в данном журнале, посвящены разработке новых вероятностно-статистических 
методов, ориентированных на применение к решению конкретных задач информатики и информационных 
технологий, а также~--- в ряде случаев~--- и других прикладных задач. Проблематика, охватываемая 
публикуемыми работами, развивается в рамках научного сотрудничества между Институтом проблем 
информатики Российской академии наук (ИПИ РАН) и Факультетом вычислительной математики и 
кибернетики Московского государственного университета им.\ М.\,В.~Ломоносова в ходе работ 
над совместными научными проектами (в том числе в рамках функционирования 
Научно-образовательного центра <<Вероятностно-статистические методы анализа рисков>>). 
Многие из авторов статей, включенных в данный номер журнала, являются активными участниками 
традиционного международного семинара по проблемам устойчивости стохастических моделей, 
руководимого В.\,М.~Золотаревым и В.\,Ю.~Королевым; регулярные сессии этого семинара 
проводятся под эгидой МГУ и ИПИ РАН (в 2011~г.\ указанный семинар проводится в октябре 
в Калининградской области РФ). 

Наряду с представителями ИПИ РАН и МГУ в число авторов данного выпуска журнала входят 
ученые из Научно-исследовательского института системных исследований РАН, Института 
проблем технологии микроэлектроники и особочистых материалов РАН, Института 
прикладных математических исследований Карельского НЦ РАН, Московского 
авиационного института, Вологодского государственного педагогического университета, 
НИИММ им.\ Н.\,Г.~Чеботарева, Казанского государственного университета, Дебреценского 
университета (Венгрия).

Несколько статей выпуска посвящено разработке и применению стохастических методов и 
информационных технологий для решения различных прикладных задач. В~работе В.\,Г.~Ушакова 
и О.\,В.~Шестакова рассмотрена задача определения вероятностных характеристик случайных 
функций по распределениям интегральных преобразований, возникающих в задачах эмиссионной 
томографии. В~статье Д.\,О.~Яковенко и М.\,А.~Целищева рассмотрены некоторые вопросы 
математической теории риска и предложен новый подход к диверсификации инвестиционных 
портфелей. Работа И.\,А.~Кудрявцевой и А.\,В.~Пантелеева посвящена построению и 
исследованию математической модели, описывающей динамику сильноионизованной плазмы. 
В~статье П.\,П.~Кольцова изучается качество работы ряда алгоритмов сегментации изображений. 
Статья А.\,Н.~Чупрунова и И.~Фазекаша посвящена вероятностному анализу числа без\-оши\-бочных 
блоков при помехоустойчивом кодировании; получены усиленные законы больших чисел для указанных 
величин.

В данном выпуске традиционно присутствует тематика, весьма активно разрабатываемая в течение 
многих лет специалистами ИПИ РАН и МГУ,~--- методы моделирования и управления для 
информационно-телекоммуникационных и вычислительных систем, в частности методы 
теории массового обслуживания. В~статье А.\,И.~Зейфмана с соавторами рассматриваются 
модели обслуживания, описываемые марковскими цепями с непрерывным временем в случае 
наличия катастроф. В~работе М.\,М.~Лери и И.\,А.~Чеплюковой рассматриваются случайные 
графы Интернет-типа, т.\,е.\ графы, степени вершин которых имеют степенные распределения; 
такие задачи находят применение при исследовании глобальных сетей передачи данных. 
Работа Р.\,В.~Разумчика посвящена исследованию систем массового обслуживания специального 
вида~--- с отрицательными заявками и хранением вытесненных заявок.

Ряд статей посвящен развитию перспективных теоретических 
вероятностно-статистических методов, которые находят широкое применение в различных 
задачах информатики и информационных технологий. В~работе В.\,Е.~Бенинга, А.\,К.~Горшенина 
и В.\,Ю.~Королева рассмотрена задача статистической проверки гипотез о числе компонент 
смеси вероятностных распределений, приводится конструкция асимптотически наиболее мощного 
критерия. Результаты этой работы найдут применение в ряде прикладных задач, использующих 
математическую модель смеси вероятностных распределений (в информатике, моделировании 
финансовых рынков, физике турбулентной плазмы и~т.\,д.). В~статье В.\,Ю.~Королева, 
И.\,Г.~Шевцовой и С.\,Я.~Шоргина строится новая, улучшенная оценка точности нормальной 
аппроксимации для пуассоновских случайных сумм; как известно, указанные случайные суммы 
широко используются в качестве моделей многих реальных объектов, в том числе в информатике, 
физике и других прикладных областях. Работа В.\,Г.~Ушакова и Н.\,Г.~Ушакова посвящена 
исследованию ядерной оценки плотности распределения; эти результаты могут применяться, 
в част\-ности, при анализе трафика в телекоммуникационных системах. Серьезные приложения 
в статистике могут получить результаты работы О.\,В.~Шестакова, в которой доказаны оценки 
скорости сходимости распределения выборочного абсолютного медианного отклонения к нормальному 
закону. 

\smallskip

Редакционная коллегия журнала выражает надежду, что данный тематический  выпуск 
будет интересен специалистам в области теории вероятностей и математической статистики 
и их применения к решению задач информатики и информационных технологий.
     
     %\vfill 
     \vspace*{20mm}
     \noindent
     Заместитель главного редактора журнала <<Информатика и её 
применения>>,\\
     директор ИПИ РАН, академик  \hfill
     \textit{И.\,А.~Соколов}\\
     
     \noindent
     Редактор-составитель тематического выпуска,\\
     профессор кафедры математической статистики факультета\\
      вычислительной математики и кибернетики МГУ им.\ М.\,В.~Ломоносова,\\
     ведущий научный сотрудник ИПИ РАН,\\ 
доктор физико-математических наук \hfill
      \textit{В.\,Ю.~Королев}
     
     } }
     }

\def\stat{leri}

\def\tit{СРЕДНЕЕ РАССТОЯНИЕ В~КОНФИГУРАЦИОННЫХ ГРАФАХ СО~СТЕПЕННЫМ РАСПРЕДЕЛЕНИЕМ$^*$}

\def\titkol{Среднее расстояние в~конфигурационных графах со~степенным распределением}

\def\aut{М.\,М.~Лери$^1$}

\def\autkol{М.\,М.~Лери}

\titel{\tit}{\aut}{\autkol}{\titkol}

\index{Лери М.\,М.}
\index{Leri M.\,M.}


{\renewcommand{\thefootnote}{\fnsymbol{footnote}} \footnotetext[1]
{Финансовое обеспечение исследований осуществлялось из средств федерального
бюджета на выполнение государственного задания Карельского научного центра Российской академии наук
(Институт прикладных математических исследований КарНЦ РАН).}}


\renewcommand{\thefootnote}{\arabic{footnote}}
\footnotetext[1]{Институт прикладных математических исследований Карельского научного центра
Российской академии наук, \mbox{leri@krc.karelia.ru}}

%\vspace*{-2pt}








\Abst{В случайных конфигурационных графах с~дискретным степенным распределением степеней вершин
с фиксированным параметром рассматривается среднее расстояние в~графе, которое вычисляется
как среднее арифметическое расстояний между всеми парами вершин графа.
Эта характеристика оценивается с~по\-мощью методов имитационного моделирования. В~силу вычислительных
ограничений рассматриваются графы в~доасимптотической области (в~настоящей работе это графы объемом
до 7000~вершин). По\-стро\-ены модели зависимостей сред\-не\-го рас\-сто\-яния от объема графа и~па\-ра\-мет\-ра распределения степеней вершин.
Проведено сравнение полученных результатов с~результатами тео\-ре\-ти\-че\-ских исследований типичного расстояния
в графе в~асимп\-то\-ти\-ке (т.\,е.\ когда число вершин графа стремится к~бес\-ко\-неч\-ности), приведенными в~работах
Р.~Хофстада.}

\KW{конфигурационные графы; степенное распределение;
сред\-нее рас\-сто\-яние в~графе; имитационное моделирование}

\DOI{10.14357/19922264230104} 
  
\vspace*{-6pt}


\vskip 10pt plus 9pt minus 6pt

\thispagestyle{headings}

\begin{multicols}{2}

\label{st\stat}

\section{Введение}

\vspace*{-1pt}

Изучение структуры и~функционирования сложных сетей продолжает оставаться одним из важных
направлений исследований в~науке и~технике~\cite{Dur,Hof1}. Примерами таких сетей, окружающих
нас в~повседневной жизни, служат интернет, электрические и~телекоммуникационные сети, сети
социальных отношений, соавторства и~цитирования и~др.
Их быст\-рое и~динамичное развитие и~нарастающая популярность легли в~основу многих фундаментальных
исследований в~об\-ласти топологии таких сетей (см., например,~\cite{Dur,Hof1,Hof2,New1,New2}).
В~качестве моделей слож\-ных сетей широко используются случайные графы, причем их разнообразие
касается как определения степеней вершин графа, так и~уста\-нов\-ле\-ния связей между этими вершинами.
В~частности, было показано (см., например,~\cite{Fa,RN}), что модели случайных графов с~независимыми
одинаково распределенными степенями вершин с~общим дискретным законом распределения подходят для
моделирования сети Интернет в~случае, когда в~качестве узлов сети рассматриваются автономные системы.

Увеличение размеров сетей и~изменчивость сетевой структуры дают понять, что для адекватного отражения
их топологии и~функционирования в~ходе по\-стро\-ения их математических моделей необходимо учитывать не
только распределение степеней вершин в~со\-от\-вет\-ст\-ву\-ющей модели случайного графа, но также
принимать во внимание и~другие не менее важ\-ные характеристики исследуемых сетей~\cite{Hof1, New1}.
В связи с~этим различные структурные характеристики слож\-ных сетей пред\-став\-ля\-ют определенный интерес
как при моделировании их топологии, так и~при изучении динамических процессов, происходящих в~таких сетях
по мере их рос\-та или под внешними воздействиями. Одна из таких характеристик~---
рас\-сто\-яние между двумя произвольными вершинами графа~\cite{Dur, Hof2, New1}.

 Мож\-но рас\-смат\-ри\-вать различные
виды расстояний в~графе: расстояние между двумя заданными вершинами, рас\-сто\-яние между произвольно\linebreak
выбранными вершинами, все расстояния между каж\-дой парой вершин, наименьшее воз\-мож\-ное {рас\-сто\-яние},
наибольшее, или диаметр графа, и~т.\,д. 

В~на\-сто\-ящей работе рассматривается среднее расстояние в~графе,
которое вы\-чис\-ля\-ет\-ся как среднее арифметическое расстояний между всеми парами вершин графа.
Цель работы со\-сто\-яла в~на\-хож\-де\-нии зависимостей среднего расстояния в~графе от числа его вершин и~па\-ра\-мет\-ра 
распределения степеней вершин, а~также в~сравнении результатов работы с~результатами тео\-ре\-ти\-че\-ских
исследований рас\-сто\-яния в~графе, приведенными в~\cite{Hof2}.
Исследование проводилось по\-средст\-вом методов имитационного моделирования с~по\-сле\-ду\-ющей статистической
обработкой данных с~помощью программного обеспечения Statistica 10 и~Wolfram Mathematica 9.0.

\section{Описание модели}

Рассматривается случайный граф, со\-сто\-ящий из~$N$~вершин. Через $\xi_1,\xi_2,\ldots,\xi_N$ обозначим
степени вершин, которые являются независимыми одинаково распределенными случайными величинами
со сле\-ду\-ющим дискретным степенным распределением~\cite{RN}:
\begin{equation}
\label{eq1}
{\bf P}\{\xi = k\} = k^{-\tau} - (k+1)^{-\tau}, \quad k=1,2,\ldots,
\end{equation}
с фиксированным параметром~$\tau\hm>1$. Легко показать, что математическое ожидание
распределения~(\ref{eq1}) рав\-но ${\bf E}\xi\hm=\zeta(\tau)$, где $\zeta(\tau)$~--- значение дзе\-та-функ\-ции
Римана в~точке~$\tau$. Что касается дисперсии, то при $\tau\hm>2$ она конечна, а при
$\tau\hm\in(1,2]$~-- бесконечна.
В~работе рассматриваются конфигурационные графы~\cite{Bol}, построение которых происходит сле\-ду\-ющим
образом.
Для каждой из $N$ вершин графа задаются степени в~соответствии  с~распределением~(\ref{eq1}) с~выбранным
значением па\-ра\-мет\-ра~$\tau$. Степени определяют чис\-ло различимых полуребер~\cite{RN} (под полуребром понимают
ребро, инцидентное данной вершине графа, для которого смежная вершина еще не определена), занумерованных в~произвольном порядке. 
Для формирования ребер все полуребра попарно и~равновероятно соединяют между собой.
Сумма степеней вершин рас\-смат\-ри\-ва\-емо\-го графа является случайной величиной. Очевидно, что она должна быть чет\-ной,
поэтому, если это не так, для построения недостающего ребра к~равновероятно выбранной
вершине добавляется одно полуребро, увеличивая степень этой вершины на~1. Граф, по\-стро\-ен\-ный таким
образом, может иметь пет\-ли, цик\-лы и~кратные \mbox{ребра}.

Известно (см., например,~\cite{Dur,Hof1,RN}), что степенной конфигурационный граф, значение па\-ра\-мет\-ра
распределения степеней вершин которого $\tau\hm>1$, асимптотически почти наверное содержит больше одной
компоненты связности, причем при $\tau\hm\in(1,2)$ в~таком графе существует, и~она единственна, так
называемая гигантская компонента связ\-ности, чис\-ло вершин в~которой пропорционально~$N$
при $N\hm\rightarrow\infty$, а~объем любой другой компоненты такого графа бесконечно мал по
сравнению с~объемом гигантской компоненты.


\vspace*{-6pt}

\section{Среднее расстояние в~графе}

\vspace*{-2pt}

Расстояния между узлами сложной сети служат важными числовыми характеристиками сетевой
топологии (см., например,~\cite{Hof2, Chu}).
Пусть $G \hm= (V,E)$~--- неориентированный граф, в~котором~$V$~--- множество вершин, а~$E$~--- множество ребер.
Обозначим через~$l(v,u)$ чис\-ло ребер простой цепи, со\-еди\-ня\-ющей вершины~$v$ и~$u$ графа~$G$
($v,u\hm\in V$ и~$v\hm\neq u$). Если вершины~$v$ и~$u$ принадлежат разным компонентам связ\-ности, то $l(v,u)$
полагают равным~$\infty$. Длину цепи от вершины~$v$ до вершины~$u$ наименьшей длины называют расстоянием
между этими вершинами:
$$
d(v,u)=\min\limits_{l(v,u)\neq\infty}l(v,u).
$$

Пусть $k$~--- чис\-ло рас\-сто\-яний $d(v,u)\hm\neq\infty$ между всеми парами вершин~$v$ и~$u$ ($v\hm\neq u$).
Среднее расстояние вы\-чис\-ля\-ет\-ся как среднее арифметическое всех рас\-сто\-яний $d(v,u)$ графа~$G$:
\begin{equation*}
\mathrm{dist} = \mathrm{dist}\,(G) = \fr{\sum\nolimits_{v,u\in V(v\neq u)}d(v,u)}{k}.
\end{equation*}

Из теорем~7.2 и~7.1 в~\cite{Hof2} следует, что при $N\hm\rightarrow\infty$ 
\begin{equation}
\label{eq2}
d(v,u)\sim\fr{2\ln\ln N}{|\ln(\tau-1)|}\,,
\end{equation}
если $1\hm<\tau\hm<2$,
и
\begin{equation}
\label{eq3}
d(v,u)\sim\fr{\ln N}{\ln\nu}\,,
\end{equation}
если $\tau>2$, 
где $\nu={\bf E}\xi(\xi-1)/({\bf E}\xi)$ и~$\nu\hm>1$.
Выражения~(\ref{eq2}) и~(\ref{eq3}) носят асимптотический характер и~получены для <<типичных рас\-сто\-яний>>
(где под типичным рас\-сто\-яни\-ем понимается математическое ожидание рас\-сто\-яния)~\cite{Hof2} в~конфигурационных
графах с~бесконечной и~с~конечной дис\-пер\-си\-ями соответственно.
Легко показать, что для конфигурационных графов с~распределением~(\ref{eq1})
$$
\nu=\fr{2\zeta(\tau-1)}{\zeta(\tau)}-2\,,
$$
причем $\nu>1$ при $2\hm<\tau\hm\leq 2,8106\ldots$

Для нахождения зависимости среднего расстояния от числа вершин графа~$N$ и~па\-ра\-мет\-ра распределения
степеней вершин~$\tau$ были получены оценки средних расстояний в~конфигурационных графах различных
размерностей с~разными па\-ра\-мет\-ра\-ми распределения степеней вершин. По полученным результатам были
построены зависимости $\mathrm{dist}$ от чис\-ла вершин графа~$N$ при конкретных значениях па\-ра\-мет\-ра распределения
степеней вершин~$\tau$, а~также зависимости $\mathrm{dist}$ от~$N$ и~$\tau$ на интервалах $\tau\hm\in(1,2)$ и~$\tau\hm\in(2,\,2,8]$.
Рассматривались конфигурационные графы сле\-ду\-ющих размерностей:
$10\hm\leq N\hm\leq 100$ с~шагом~$10$, $100\hm\leq N\hm\leq 1000$ с~шагом~$50$, $1000\hm\leq N\hm\leq 7000$ с~шагом~$500$.
Значения параметра~$\tau$ изменялись с~шагом $0{,}1$ в~двух интервалах: $1{,}1\hm\leq\tau\hm<2$ и~$2\hm<\tau\hm\leq 2{,}8$,
а~так\-же были взяты два дополнительных значения: $\tau\hm=1{,}99$ и~$2{,}01$. Для
каждой пары значений $(N,\tau)$
генерировалось\linebreak\vspace*{-12pt}

\pagebreak

\noindent
  по~$100$~графов, т.\,е.\ $40\,000$ и~$36\,000$ графов на интервалах $1{,}1\hm\leq\tau\hm\leq 1{,}99$
и~$2{,}01\hm\leq\tau\hm\leq 2{,}8$ соответственно.
Расстояния в~графе находились с~применением алгоритма Дейкстры~\cite{Dijk}.

\subsection{Результаты для интервала $\tau\in(1,2)$}

Сначала для рассматриваемых в~настоящей работе графов были по\-стро\-ены зависимости среднего рас\-сто\-яния $\mathrm{dist}$ от
чис\-ла вершин $N$ при фиксированных значениях па\-ра\-мет\-ра~$\tau$, т.\,е.\ для каждого из рас\-смот\-рен\-ных значений
$1{,}1\hm\leq\tau\hm\leq 1{,}99$ были получены регрессионные зависимости вида
\begin{equation}
\label{eq4}
\mathrm{dist} = a \ln\ln N + b\,.
\end{equation}
Здесь и~далее коэффициенты всех регрессионных уравнений находили по\-средст\-вом метода наименьших
квадратов, зна\-чи\-мость коэффициентов проверяли с~помощью критерия Стьюдента. Для оценки степени подгонки
регрессионной модели к~данным вы\-чис\-ля\-ли коэффициент детерминации этой модели и~с~по\-мощью критерия Фишера
проверяли гипотезу $H_0: R^2\hm=0$. Проверка всех статистических гипотез осуществлялась на 5\%-ном уровне
зна\-чи\-мости.

Обозначим $a_f={2}/{|\ln(\tau-1)|}$ и~сравним эти значения с~коэффициентами~$a$ уравнений~(\ref{eq4})
для каждого~$\tau$.
В табл.~1 приведены значения~$a_f$, коэффициенты~$a$ и~$b$ регрессионных урав\-не\-ний вида~(\ref{eq4})
и~соответствующие коэффициенты детерминации~$R^2$ этих уравнений. Все коэффициенты~$a$ и~$b$ в~табл.~1
значимы, а~гипотезы $H_0: R^2\hm=0$ отвергаются для всех уравнений.




Таким образом, при фиксированных~$\tau$ сред\-нее\linebreak рас\-сто\-яние в~графе с~рос\-том чис\-ла его вершин\linebreak
рас\-тет как $\ln\ln N$, так же как и~расстояние в~асимп\-то\-ти\-ке~(\ref{eq2}). Из табл.~1 видно,
что значения коэффициента~$a$ ниже значений~$a_f$ на всем
интервале\linebreak  изменения~$\tau$, однако эта разница не остается неизменной,
а~возрастает с~рос\-том~$\tau$.
Более того, сам угловой коэффициент~$a$ возрастает с~рос\-том
па\-ра-\linebreak\vspace*{-12pt}

%\begin{table*}\small %tabl1
\begin{center}

\vspace*{6pt}

\noindent
\parbox{64mm}{{{\tablename~1}\ \ \small{Значения $a_f$, коэффициенты $a$ и~$b$ зависимостей вида~(\ref{eq4})
и коэффициенты детерминации $R^2$ этих уравнений
}}}


\vspace*{6pt}


{\small 
\begin{tabular}{|c|c|c|c|c|}
\hline
&&&&\\[-10pt]
$\tau$ & $a_f$ & $a$ & $b$ & $R^2$ \\ 
\hline
1,1 & 0,869 &   0,745 & \hphantom{$-$}1,702 & 0,88 \\
1,2 & 1,243 &   0,975 & \hphantom{$-$}1,513 & 0,91 \\
1,3 & 1,661 &   1,255 & \hphantom{$-$}1,244 & 0,94 \\
1,4 & 2,183 &   1,596 & \hphantom{$-$}0,869 & 0,96 \\
1,5 & 2,885 &   1,959 & \hphantom{$-$}0,494 & 0,96 \\
1,6 & 3,915 &   2,421 & $-$0,065 & 0,98 \\
1,7 & 5,607 &   3,048 & $-$0,909 & 0,98 \\
1,8 & 8,963 &   3,668 & $-$1,710 & 0,98 \\
1,9 & 18,982\hphantom{9} & 4,417 & $-$2,806 & 0,98 \\
\hphantom{9}1,99 & 198,998\hphantom{99} & 5,262 & $-$4,045 & 0,96\\
\hline
\end{tabular}
}
\end{center}
%\end{table*}

{ \begin{center}  %fig1
 \vspace*{-2pt}
    \mbox{%
\epsfxsize=78.504mm
\epsfbox{ler-1.eps}
}

\end{center}



\noindent
{{\figurename~1}\ \ \small{График экспериментальных
значений $\mathrm{dist}$ от $N$ и~$1{,}1\hm\leq\tau\hm<2$
}}}

\vspace*{8pt}

\addtocounter{figure}{1}
\addtocounter{table}{1}



\noindent
мет\-ра~$\tau$,
 и~это  на\-гляд\-но видно
на рис.~1, где показана за\-ви\-си\-мость
экспериментальных значений $\mathrm{dist}$ от~$N$ и~$\tau$.


Далее задача состояла в~том, чтобы найти за\-ви\-си\-мость сред\-не\-го рас\-сто\-яния $\mathrm{dist}$ от обеих переменных:
$N$ и~$\tau$.
Сначала была построена за\-ви\-си\-мость в~виде $\mathrm{dist}\hm={2\ln\ln N}/({|\ln(\tau-1)|})\hm+b$. Получено
сле\-ду\-ющее регрессионное уравнение:
\begin{equation*}
\mathrm{dist} = \fr{2\ln\ln N}{|\ln(\tau-1)|}-39{,}604\,.
\end{equation*}
К сожалению, коэффициент детерминации полученной за\-ви\-си\-мости оказался очень низ\-ким
($R^2\hm=0{,}01$). Гипотеза о~равенстве~$R^2$ нулю не отвергается, коэффициент~$b$
статистически не значим, поэтому такую модель использовать для прогноза не имеет смысла.



\begin{figure*} %fig2
\vspace*{1pt}
\begin{minipage}[t]{80mm}
\begin{center}
   \mbox{%
\epsfxsize=79mm
\epsfbox{ler-2-a.eps}
}
\end{center}
\vspace*{-9pt}
\Caption{Регрессионная зависимость~(\ref{eq6}) среднего расстояния $\mathrm{dist}$ от $N$ при фиксированных
значениях $1{,}1\hm\leq\tau\hm<2$: \textit{1}~--- $\tau\hm=1{,}1$;
\textit{2}~--- 1,3; \textit{3}~--- 1,5; \textit{4}~--- 1,7; \textit{5}~--- $\tau\hm= 1{,}99$
}
\end{minipage}
%\end{figure*}
\hfill
%\begin{figure*} %fig3
\vspace*{1pt}
\begin{minipage}[t]{80mm}
\begin{center}
   \mbox{%
\epsfxsize=77.81mm
\epsfbox{ler-2-b.eps}
}
\end{center}
\vspace*{-9pt}
\Caption{Регрессионная зависимость~(\ref{eq6}) среднего расстояния $\mathrm{dist}$
от~$\tau$ при фиксированных значениях $10\hm\leq N\hm\leq 7000$:) \textit{1}~--- $N\hm=10$; \textit{2}~--- 100; \textit{3}~--- 1000;
\textit{4}~--- 5000; \textit{5}~--- $N\hm=7000$}
\end{minipage}
\vspace*{-4pt}
\end{figure*}



Далее была построена регрессия вида $\mathrm{dist}\hm={a\ln\ln N}/({|\ln(\tau-1)|})\hm+b$ и~получена
за\-ви\-си\-мость
\begin{equation}
\label{eq5}
\mathrm{dist} = \fr{0{,}0104\ln\ln N}{|\ln(\tau-1)|}+3{,}922
\end{equation}
с~коэффициентом детерминации $R^2\hm=0{,}21$. В~данном случае гипотеза $H_0: R^2\hm=0$ отвергается,
а~что касается коэффициентов~$a$ и~$b$ регрессионного уравнения~(\ref{eq5}), то коэффициент~$a$
оказался статистически значим, а~коэффициент~$b$ нет.
Поиск наилучшей регрессионной за\-ви\-си\-мости был продолжен и~привел к~получению сле\-ду\-ющей
модели за\-ви\-си\-мости сред\-не\-го рас\-сто\-яния конфигурационного графа $\mathrm{dist}$ от объема графа~$N$ 
и~па\-ра\-мет\-ра распределения степеней вершин~$\tau$:
\begin{equation}\label{eq6}
\mathrm{dist} = \fr{2(4{,}488-3{,}077\tau+0{,}417\tau^2)\ln\ln N}{|\ln(\tau-1)|}
\end{equation}
с коэффициентом детерминации $R^2\hm=0{,}88$ и~значимыми коэффициентами регрессии. Графически
за\-ви\-си\-мость~(\ref{eq6}) пред\-став\-ле\-на на рис.~2 и~3.

Оценка значимости различия между коэффициентами множественной корреляции $r\hm=\sqrt{R^2}$ регрессионных моделей~(\ref{eq5}) и~(\ref{eq6})
 на уровне зна\-чи\-мости~0,05 показала, что нулевая гипотеза $H_0:\linebreak r_{(5)}\hm=r_{(6)}$
($r_{(5)}$ и~$r_{(6)}$~--- коэффициенты множественной корреляции зависимостей~(\ref{eq5}) и~(\ref{eq6})
соответственно) отвергается; следовательно, различие между коэффициентами корреляции значимо.
Остатки обеих моделей распределены нормально, но сравнение сумм квад\-ра\-тов
остатков $\mathrm{SSR}_{(5)}\hm=62951{,}1$ и~$\mathrm{SSR}_{(6)}\hm=24786{,}9$ показывает, что $\mathrm{SSR}_{(5)}\hm>\mathrm{SSR}_{(6)}$, т.\,е.\ модель~(\ref{eq6}) 
<<лучше>> в~смыс\-ле описания изуча\-емо\-го явления и~для прогнозирования.
Таким образом, в~качестве наиболее под\-хо\-дя\-щей модели за\-ви\-си\-мости $\mathrm{dist}$ от~$N$ и~$\tau$ для до\-асимп\-то\-ти\-че\-ской
об\-ласти предлагается за\-ви\-си\-мость, описываемая уравнением~(\ref{eq6}).



На рис.~2 и~3 линии внут\-ри затененных областей соответствуют зависимостям $\mathrm{dist}$ от~$N$ (см.\ рис.~2)
и~от~$\tau$ (см.\ рис.~3) при некоторых (отраженных в~легендах) значениях па\-ра\-мет\-ра~$\tau$
или объема графа $N$ соответственно. Заметим, что кривые зависимостей $\mathrm{dist}$ от~$N$ на рис.~2
расположены одна над другой по мере роста значения параметра $1,1\leq\tau<2$ в~пределах его граничных значений.
Аналогично кривые зависимостей $\mathrm{dist}$ от~$\tau$ на рис.~3 также расположены друг над другом по
мере воз\-рас\-та\-ния чис\-ла вершин графа $10\hm\leq N\hm\leq 7000$.


\vspace*{-6pt}

\subsection{Результаты для интервала $\tau\in(2,\,2{,}8]$}

\vspace*{-2pt}


Исследование зависимости сред\-не\-го рас\-сто\-яния от~$N$ и~$\tau\hm\in(2,\,2{,}8]$ в~до\-асимп\-то\-ти\-че\-ской об\-ласти
было проведено аналогично предыду\-ще\-му исследованию для $\tau\hm\in(1,2)$.
Сначала для фиксированных значений па\-ра\-мет\-ра $2{,}01\hm\leq\tau\hm\leq 2{,}8$ были построены зависимости сред\-не\-го
рас\-сто\-яния $\mathrm{dist}$ от чис\-ла вершин графа~$N$ сле\-ду\-юще\-го вида:
\begin{equation}
\label{eq7}
\mathrm{dist} = a \ln N + b\,.
\end{equation}

Обозначим 
$$
a_f=\fr{1}{\ln\left({2\zeta(\tau-1)}/{\zeta(\tau)}-2\right)}\,.
$$
 Для сравнения этих
значений с~коэффициентами~$a$ уравнений~(\ref{eq7}) для каждого~$\tau$ все они приведены
в~табл.~2 наряду с~коэффициентами~$b$ и~со\-от\-вет\-ст\-ву\-ющи\-ми коэффициентами детерминации
$R^2$ этих уравнений. Все коэффициенты~$a$ и~$b$ значимы, а~гипотезы о~ра\-венст\-ве нулю коэффициентов
детерминации полученных моделей отвергаются.


\setcounter{figure}{4}
\begin{figure*}[b] %fig5
\vspace*{1pt}
\begin{minipage}[t]{81mm}
\begin{center}
   \mbox{%
\epsfxsize=80mm
\epsfbox{ler-4-a.eps}
}
\end{center}
\vspace*{-13pt}
\Caption{Регрессионная зависимость~(\ref{eq9}) среднего рас\-сто\-яния $\mathrm{dist}$ от~$N$ при фиксированных
значениях $2\hm<\tau\hm\leq 2{,}8$:
\textit{1}~--- $\tau\hm= 2{,}01$; \textit{2}~--- 2,2288\ldots; \textit{3}~--- 2,4; \textit{4}~--- 2,6; \textit{5}~--- $\tau\hm= 2{,}8$}
%\end{figure*}
\end{minipage}
\hfill
%\begin{figure*} %fig6
\vspace*{1pt}
\begin{minipage}[t]{79.94mm}
\begin{center}
   \mbox{%
\epsfxsize=78.94mm
\epsfbox{ler-4-b.eps}
}
\end{center}
\vspace*{-13pt}
\Caption{Регрессионная зависимость~(\ref{eq9}) среднего рас\-сто\-яния $\mathrm{dist}$ 
от~$\tau$ при фиксированных значениях $10\hm\leq N\hm\leq 7000$:
\textit{1}~--- $N\hm=10$; \textit{2}~--- 100; \textit{3}~--- 1000;
\textit{4}~--- 5000; \textit{5}~--- $N\hm=7000$}
\end{minipage}
\end{figure*}

Таким образом, при фиксированных~$\tau$ из интервала $(2,\,2{,}8]$ сред\-нее рас\-сто\-яние в~графе
возрастает логарифмически с~рос\-том чис\-ла его вершин $N$, так
же как и~рас\-сто\-яние в~асимп\-то\-ти\-ке
(см.\ выраже-\linebreak\vspace*{-12pt}

%\begin{table*}\small  %tabl2
\begin{center}

\vspace*{6pt}

\noindent
\parbox{62mm}{{{\tablename~2}\ \ \small{Значения $a_f$, коэффициенты~$a$ и~$b$ зависимостей вида~(\ref{eq7})
и~коэффициенты детерминации $R^2$ этих уравнений
}}
}

\vspace*{6pt}


{\small \begin{tabular}{|c|c|c|c|c|}
\hline
&&&&\\[-10pt]
$\tau$ & $a_f$ & $a$ & $b$ & $R^2$ \\
 \hline
\hphantom{9}2,01 & 0,209 & 0,967 & $-$0,684 & 0,96 \\
2,1 &   0,408 & 1,151 & $-$1,661 & 0,98 \\
2,2 &   0,586 & 1,344 & $-$2,777 & 0,98 \\
2,3 &   0,800 & 1,541 & $-$3,994 & 0,95 \\
2,4 &   1,096 & 1,573 & $-$4,428 & 0,91 \\
2,5 &   1,565 & 1,430 & $-$4,078 & 0,88 \\
2,6 &   2,459 & 1,076 & $-$2,673 & 0,89 \\
2,7 &   4,942 & 0,760 & $-$1,387 & 0,91 \\
2,8 &   53,870\hphantom{9} & 0,507 & $-$0,352 & 0,94\\
\hline
\end{tabular}
}
\vspace*{3pt}
\end{center}
%\end{table*}

%\vspace*{3pt}

{ \begin{center}  %fig4
 \vspace*{-2pt}
   \mbox{%
\epsfxsize=78.504mm
\epsfbox{ler-3.eps}
}

\end{center}

\noindent
{{\figurename~4}\ \ \small{График экспериментальных
значений $\mathrm{dist}$ от $N$ и~$2\hm<\tau\hm\leq 2{,}8$
}}}

\vspace*{6pt}

\addtocounter{figure}{1}
\addtocounter{table}{1}


\noindent
 ние~(\ref{eq3})).
Сравнение значений коэффициентов~$a$ с~$a_f$ показывает, что для $2{,}01\leq\tau\leq 2{,}4$ значения~$a$ 
выше значений $a_f$, а~при $2{,}5\hm\leq\tau\hm\leq 2{,}8$ ниже (см.\ табл.~2).
Изменение углового коэффициента~$a$ в~данном случае показывает, что сред\-нее расстояние в~графе
с~рос\-том значения~$\tau$ сначала возрастает, достигая максимума в~промежутке от~2,2 до~2,4, 
а~затем убывает. На\-гляд\-но это мож\-но видеть на рис.~4, где показана за\-ви\-си\-мость
экспериментальных значений $\mathrm{dist}$ от~$N$ и~$\tau$.





Поиск зависимости среднего расстояния $\mathrm{dist}$ от переменных $N$ и~$\tau$ 
проходил по аналогии с~предыду\-щим интервалом изменения параметра распределения степеней вершин.
Сначала была по\-стро\-ена за\-ви\-си\-мость в~виде 
$$
\mathrm{dist}=\fr{\ln N}{\ln\left({2\zeta(\tau-1)}/{\zeta(\tau)}-2\right)}+b
$$
и получено сле\-ду\-ющее регрессионное уравнение:
\begin{equation*}
\mathrm{dist} = \fr{\ln N}{\ln\left({2\zeta(\tau-1)}/{\zeta(\tau)}-2\right)}-40{,}936\,.
\end{equation*}
К сожалению, на этом интервале коэффициент детерминации полученной за\-ви\-си\-мости оказался очень низким
($R^2\hm=0{,}0005$), гипотеза о~равенстве~$R^2$ нулю не отвергается и~коэффициент~$b$ не значим.
Следовательно, такую модель не имеет смыс\-ла использовать для прогноза.
Поэтому была осуществлена попытка по\-стро\-ить регрессию вида
$$
\mathrm{dist}=\fr{a\ln N}{\ln\left({2\zeta(\tau-1)}/{\zeta(\tau)}-2\right)}+b
$$ 
и~была получена зависимость
\begin{equation}
\label{eq8}
\mathrm{dist} = 4{,}962 - \fr{0{,}005\ln N}{\ln\left({2\zeta(\tau-1)}/{\zeta(\tau)}-2\right)}
\end{equation}
с коэффициентом детерминации $R^2\hm=0{,}06$. Несмотря на столь низ\-кое значение~$R^2$, гипотеза о~его
равенстве нулю отвергается, однако оценка зна\-чи\-мости коэффициентов~$a$ и~$b$
регрессионного уравнения~(\ref{eq8}) показала, что коэффициент~$a$ статистически значим, тогда как~$b$~-- нет.
Дальнейший поиск наилучшей регрессии привел к~получению сле\-ду\-ющей за\-ви\-си\-мости
сред\-не\-го рас\-сто\-яния конфигурационного графа $\mathrm{dist}$ от $N$ и~$\tau$:
\begin{equation}
\label{eq9}
\mathrm{dist} = \fr{(31{,}706-22{,}076\tau+3{,}841\tau^2)\ln N}{\ln\left({2\zeta(\tau-1)}/{\zeta(\tau)}-2\right)}\,,
\end{equation}
где все коэффициенты модели значимы, а $R^2\hm=0{,}74$. Зависимость~(\ref{eq9}) отражена графически на рис.~5 и~6.

Для моделей~(\ref{eq8}) и~(\ref{eq9}) была оценена зна\-чи\-мость различия между коэффициентами множественной
корреляции этих моделей при 5\%-ном уровне зна\-чи\-мости. В~результате $H_0:r_{(8)}=r_{(9)}$ была отвергнута, т.\,е.\
различие между коэффициентами корреляции оказалось значимым.
Проверка остатков регрессий~(\ref{eq8}) и~(\ref{eq9}) на нормальность показала, что нормальное распределение
имеют только остатки модели~(\ref{eq9}). Кроме того, остаточная сумма квад\-ра\-тов модели~(\ref{eq8})
$\mathrm{SSR}_{(8)}\hm=245793{,}9$ больше, чем $\mathrm{SSR}_{(9)}\hm=102369{,}9$. Поэтому мож\-но сделать вывод о~том, что модель~(\ref{eq9})
лучше подходит для прогноза, чем модель~(\ref{eq8}).
Таким образом, при значениях па\-ра\-мет\-ра $2\hm<\tau\hm\leq 2{,}8$ в~качестве наиболее подходящей модели за\-ви\-си\-мости
среднего рас\-сто\-яния $\mathrm{dist}$ от~$N$ и~$\tau$ в~до\-асимп\-то\-ти\-че\-ской об\-ласти предлагается за\-ви\-си\-мость, опи\-сы\-ва\-емая
уравнением~(\ref{eq9}).



На рис.~5 и~6 линии, находящиеся внут\-ри затененных областей и~отраженные в~легендах, соответствуют
зависимостям $\mathrm{dist}$ от~$N$ (см.\ рис.~5) и~от~$\tau$ (см.\ рис.~6) при некоторых
значениях па\-ра\-мет\-ра $\tau$ или объема графа~$N$ соответственно.
На рис.~5 ниж\-няя граница об\-ласти соответствует $\tau\hm=2{,}01$, верхняя~--- максимуму функции~(\ref{eq9}) 
по параметру $\tau$: $\tau^*\hm=2{,}2288\ldots$, а~кривые зависимостей $\mathrm{dist}$ от~$N$ внут\-ри затененной
об\-ласти расположены сле\-ду\-ющим образом: по воз\-рас\-та\-нию значений $\mathrm{dist}$ при увеличении значений~$\tau$ от~1,1
до~$\tau^*$ и~по убыванию $\mathrm{dist}$ при рос\-те~$\tau$ от~$\tau^*$ до~2,8. А~на рис.~6 кривые
зависимостей $\mathrm{dist}$ от~$\tau$ расположены друг над другом по мере воз\-рас\-та\-ния чис\-ла вершин графа
$10\hm\leq N\hm\leq 7000$ в~пределах граничных значений.

\vspace*{-9pt}


\subsection{Результаты при $\tau=2$}

\vspace*{-2pt}

Заметим, что модели~(\ref{eq6}) и~(\ref{eq9}) не охватывают значение па\-ра\-мет\-ра распределения степеней
вершин $\tau\hm=2$. Однако по экспериментальным данным была по\-стро\-ена сле\-ду\-ющая регрессионная за\-ви\-си\-мость
$\mathrm{dist}$ от~$N$ при фиксированном $\tau\hm=2$ (все коэффициенты модели значимы):
\begin{equation}
\label{eq10}
\mathrm{dist} = 5{,}262\ln\ln N - 4{,}045 \quad \left(R^2=0{,}73\right).
\end{equation}

\vspace*{-14.5pt}

\section{Выводы}

\vspace*{-2.5pt}

Итак, экспериментальные результаты на степенных конфигурационных графах с~фиксированным
па\-ра\-мет\-ром~$\tau$ распределения~(\ref{eq1}) степеней вершин показывают, что на интервале $(1,2)$ 
с~рос\-том объема~$N$ среднее расстояние $\mathrm{dist}$ в~графе рас\-тет как $\ln\ln N$, а~на интервале $2\hm<\tau\hm\leq 2{,}8$
рас\-тет логарифмически в~доасимптотической об\-ласти (при $N\hm\leq 7000$) так же, как это было показано
Р.~Хофстадом~\cite{Hof2} для типичного расстояния в~графах при $N\hm\rightarrow\infty$.
Однако что касается за\-ви\-си\-мости среднего расстояния от переменных~$N$ и~$\tau$, то при малых
объемах графа предлагается использовать модели~(\ref{eq6}) и~(\ref{eq9}) в~соответствующих интервалах
изменения параметра~$\tau$, так как они лучше описывают данную за\-ви\-си\-мость, что было под\-тверж\-де\-но
в~настоящей работе с~по\-мощью методов статистического анализа
и предложена модель~(\ref{eq10}) за\-ви\-си\-мости сред\-не\-го рас\-сто\-яния от чис\-ла вершин~$N$ при $\tau\hm=2$
так\-же для графов в~до\-асимп\-то\-ти\-че\-ской об\-ласти.

{\small\frenchspacing
 {\baselineskip=10.7pt
 %\addcontentsline{toc}{section}{References}
 \begin{thebibliography}{99}
 
 %\vspace*{-6pt}
 
\bibitem{Dur} %1
\Au{Durrett~R.} Random graph dynamics.~--- Cambridge: Cambridge University
Press, 2007. 221~p. doi: 10.1017/ CBO9780511546594.

\bibitem{Hof1} %2
\Au{Hofstad~R.} Random graphs and complex networks.~--- Cambridge:
Cambridge University Press, 2017.  Vol.~1. 337~p. doi: 10.1017/9781316779422.



\bibitem{New1} %3
\Au{Newman~M.\,E.\,J.} Networks. An introduction.~--- Oxford: Oxford University Press, 2010. 772~p.
doi: 10.1093/ acprof:oso/9780199206650.001.0001.

\bibitem{New2} %4
\textit{Newman~M.\,E.\,J.} Networks.~--- 2nd ed.~--- Oxford: Oxford University Press, 2018. 800~p.
doi: 10.1093/oso/ 9780198805090.001.0001.

\bibitem{Hof2} %5
\textit{Hofstad~R.} Random graphs and complex networks~// Notes RGCNII, 2020.  Vol.~2.
314~p. {\sf https://www.win. tue.nl/$\sim$rhofstad/NotesRGCNII.pdf.}

\bibitem{Fa} %6
\Au{Faloutsos~C., Faloutsos~P., Faloutsos~M.} On power-law relationships of
the internet topology~// Comput. Commun. Rev., 1999. Vol.~29. P.~251--262.
doi: 10.1145/ 316194.316229.

\bibitem{RN} %7
\Au{Reittu~H., Norros~I.} On the power-law random graph model of massive
data networks~// Perform. Evaluation, 2004. Vol.~55. Iss.~1-2. P.~3--23.
doi: 10.1016/S0166-5316(03)00097-X.

\bibitem{Bol}
\Au{Bollobas~B.} A~probabilistic proof of an asymptotic formula for the number
of labelled regular graphs~// Eur. J.~Combin., 1980. Vol.~1.
Iss.~4. P.~311--316. doi: 10.1016/S0195-6698(80)80030-8.



\bibitem{Chu} %9
\Au{Chung~F., Lu~L.} The average distances in random graphs with given expected degrees~// 
P.~Natl. Acad. Sci. USA, 2002. Vol.~99. Iss.~25. P.~15879--15882.
doi: 10.1073/pnas.252631999.

\bibitem{Dijk}
\Au{Dijkstra~E.\,W.} A~note on two problems in connexion with graphs~// 
Numer. Math., 1959. Vol.~1. Iss.~1. P.~269--271. doi: 10.1007/BF01386390.
\end{thebibliography}

 }
 }

\end{multicols}

\vspace*{-7pt}

\hfill{\small\textit{Поступила в~редакцию 21.03.22}}

%\vspace*{8pt}

%\pagebreak

\newpage

\vspace*{-28pt}

%\hrule

%\vspace*{2pt}

%\hrule

%\vspace*{-2pt}

\def\tit{AN AVERAGE DISTANCE IN~THE~POWER-LAW CONFIGURATION GRAPHS}


\def\titkol{An average distance in~the~power-law configuration graphs}


\def\aut{M.\,M.~Leri}

\def\autkol{M.\,M.~Leri}

\titel{\tit}{\aut}{\autkol}{\titkol}

\vspace*{-8pt}


\noindent
Institute of Applied Mathematical Research of the Karelian Research Center of the Russian Academy of Sciences, 
11~Pushkinskaya Str., Petrozavodsk 185910, Russian Federation

\def\leftfootline{\small{\textbf{\thepage}
\hfill INFORMATIKA I EE PRIMENENIYA~--- INFORMATICS AND
APPLICATIONS\ \ \ 2023\ \ \ volume~17\ \ \ issue\ 1}
}%
 \def\rightfootline{\small{INFORMATIKA I EE PRIMENENIYA~---
INFORMATICS AND APPLICATIONS\ \ \ 2023\ \ \ volume~17\ \ \ issue\ 1
\hfill \textbf{\thepage}}}

\vspace*{3pt} 


\Abste{In random configuration graphs with a~discrete power-law vertex degree distribution with a~fixed parameter, 
the average distance in the graph is considered, i.\,e., the arithmetic mean of distances between all pairs of graph nodes. 
This characteristic is estimated using simulation methods. Due to computational constraints, the author considers graphs
 in the pre-asymptotic domain (in this paper, these are the graphs up to 7000~nodes). The models of dependencies of the average distance on 
 the graph size and the parameter of vertex degree distribution are reseived. The obtained results are compared with the results 
 of theoretical studies of the typical distance in a graph in the asymptotics (i.\,e., when the number of graph vertices tends to infinity), 
 given in the works by R.~Hofstad.}

\KWE{configuration graph; power-law distribution;
average distance in a graph; simulation}



\DOI{10.14357/19922264230104} 

\vspace*{-16pt}

\Ack
\noindent
The study was carried out under state order to the Karelian Research Center 
of the Russian Academy of Sciences (Institute of Applied Mathematical Research KarRC RAS).

\vspace*{6pt}

  \begin{multicols}{2}

\renewcommand{\bibname}{\protect\rmfamily References}
%\renewcommand{\bibname}{\large\protect\rm References}

{\small\frenchspacing
 {%\baselineskip=10.8pt
 \addcontentsline{toc}{section}{References}
 \begin{thebibliography}{99} 
\bibitem{1-leri-1}
\Aue{Durrett, R.} 2007. \textit{Random graph dynamics.} Cambridge: Cambridge University
Press. 221~p. doi: 10.1017/ CBO9780511546594.

\bibitem{2-leri-1}
\Aue{Hofstad, R.} 2017. \textit{Random graphs and complex networks.} Cambridge:
Cambridge University Press.   Vol.~1. 337~p. doi: 10.1017/9781316779422.

\bibitem{4-leri-1} %3
\Aue{Newman, M.\,E.\,J.} 2010. \textit{Networks. An introduction.} Oxford: Oxford
University Press. 772~p. doi: 10.1093/ acprof:oso/9780199206650.001.0001.

\bibitem{5-leri-1} %4
\Aue{Newman, M.\,E.\,J.} 2018. \textit{Networks.} 2nd ed. Oxford: Oxford
University Press. 800~p. doi: 10.1093/oso/ 9780198805090.001.0001.

\bibitem{3-leri-1} %5
\Aue{Hofstad, R.} 2020. Random graphs and complex networks.  \textit{Notes RGCNII}. Vol.~2. 314~p.
Available at: {\sf https:// www.win.tue.nl/$\sim$rhofstad/NotesRGCNII.pdf} (accessed January~18, 2023)



\bibitem{6-leri-1}
\Aue{Faloutsos, C., P.~Faloutsos, and M.~Faloutsos.} 1999. On power-law relationships of
the internet topology. \textit{Comput. Commun. Rev.} 29:251--262.
doi: 10.1145/316194.316229.

\bibitem{7-leri-1}
\Aue{Reittu, H., and I.~Norros.} 2004. On the power-law random graph model of massive data
networks. \textit{Perform. Evaluation} 5(1-2)5:3--23.
doi: 10.1016/S0166-5316(03)00097-X.

\bibitem{8-leri-1}
\Aue{Bollobas, B.} 1980. A~probabilistic proof of an asymptotic formula for the number
of labelled regular graphs. \textit{Eur. J.~Combin.} 1(4):311--316.
doi: 10.1016/S0195-6698(80)80030-8.

\bibitem{9-leri-1}
\Aue{Chung, F., and L.~Lu.} 2002. The average distances in random graphs with given expected
degrees. \textit{P.~Natl. Acad. Sci. USA} 99(25):15879--15882.
doi: 10.1073/pnas.252631999.

\bibitem{10-leri-1}
\Aue{Dijkstra, E.\,W.} 1959. A~note on two problems in connexion with graphs.
\textit{Numer. Math.} 1(1):269--271. doi: 10.1007/BF01386390.
\end{thebibliography}

 }
 }

\end{multicols}

\vspace*{-6pt}

\hfill{\small\textit{Received March 21, 2022}}


\Contrl

\noindent
\textbf{Leri Marina M.} (b.\ 1969)~--- Candidate of Science (PhD) in technology, scientist,
Institute of Applied Mathematical Research of the Karelian Research Center of the Russian Academy of Sciences,
11~Pushkinskaya Str., Petrozavodsk 185910, Russian Federation; \mbox{leri@krc.karelia.ru}


\label{end\stat}

\renewcommand{\bibname}{\protect\rm Литература}   %1


%\newcommand{\cov}{\textbf{cov}}
\newcommand{\I}{\mathbf{1}}
\newcommand{\Variance}{\sf D}

\def\stat{shest-1}

\def\tit{НЕСМЕЩЕННАЯ ОЦЕНКА РИСКА СТАБИЛИЗИРОВАННОЙ ЖЕСТКОЙ
ПОРОГОВОЙ ОБРАБОТКИ В~МОДЕЛИ С~ДОЛГОСРОЧНОЙ ЗАВИСИМОСТЬЮ$^*$}

\def\titkol{Несмещенная оценка риска стабилизированной жесткой
пороговой обработки в~модели с~долгосрочной зависимостью}

\def\aut{О.\,В.~Шестаков$^1$}

\def\autkol{О.\,В.~Шестаков}

\titel{\tit}{\aut}{\autkol}{\titkol}

\index{Шестаков О.\,В.}
\index{Shestakov O.\,V.}




{\renewcommand{\thefootnote}{\fnsymbol{footnote}} \footnotetext[1]
{Работа выполнена при частичной финансовой поддержке РФФИ (проект 16-07-00736).}}


\renewcommand{\thefootnote}{\arabic{footnote}}
\footnotetext[1]{Московский государственный университет им.\ М.\,В.~Ломоносова, 
кафедра математической статистики факультета вычислительной математики и~кибернетики; 
Институт проб\-лем информатики Федерального исследовательского центра 
<<Информатика и~управ\-ле\-ние>> Российской академии наук, \mbox{oshestakov@cs.msu.su}}

%\vspace*{-6pt}



\Abst{Методы подавления шума в~сигналах и~изображениях, основанные на процедуре 
пороговой обработки коэффициентов вейв\-лет-раз\-ло\-же\-ния, стали популярными благодаря 
своей прос\-то\-те, ско\-рости и~возможности адаптации к~функциям сигналов, име\-ющим на 
разных участках различную степень ре\-гу\-ляр\-ности. Анализ погрешностей этих методов 
является важ\-ной практической задачей, поскольку дает воз\-мож\-ность оценивать качество 
как самих методов, так и~используемого для обработки оборудования. 
Рассматривается предложенный недавно стабилизированный метод жесткой пороговой 
обработки, в~котором устранены основные недостатки мягкой и~жесткой пороговой 
обработки, и~исследуются статистические свойства этого метода. В~модели 
данных с~аддитивным гауссовским шумом проводится анализ несмещенной оценки 
среднеквадратичного риска. В~предположении о~том, что шумовые коэффициенты 
обладают долгосрочной за\-ви\-си\-мостью, приводятся условия, при которых имеет 
место сильная состоятельность и~асимптотическая нор\-маль\-ность несмещенной оценки риска. 
Полученные результаты дают воз\-мож\-ность строить асимптотические доверительные 
интервалы для погрешностей пороговой обработки, используя только наблюдаемые данные.}

\KW{вейвлеты; пороговая обработка; несмещенная оценка риска; коррелированный шум; 
асимптотическая нормальность}

\DOI{10.14357/19922264180202}
  
\vspace*{6pt}


\vskip 10pt plus 9pt minus 6pt

\thispagestyle{headings}

\begin{multicols}{2}

\label{st\stat}

\section{Введение}

В задачах статистической обработки данных час\-то предполагается, что наблюдения 
независимы. Однако существует множество физических процессов, де\-мон\-ст\-ри\-ру\-ющих 
долгосрочную за\-ви\-си\-мость, при которой корреляции между наблюдениями убывают 
столь медленно, что ряд из них не сходится. Такая долгосрочная за\-ви\-си\-мость 
час\-то наблюдается, например, при исследовании геофизических процессов, в~которых 
она принимает форму длительных периодов больших или маленьких значений наблюдений. 
Схожие явления демонстрируют помехи в~коммуникационных каналах. 

При 
анализе и~обработке сигналов, ре\-гист\-ри\-ру\-емых при изучении таких процессов, 
широко применяют\-ся методы вейв\-лет-ана\-ли\-за. К~данным применяется 
вейв\-лет-пре\-об\-ра\-зо\-ва\-ние и~осуществляется пороговая обработка получившихся 
вейв\-лет-ко\-эф\-фи\-ци\-ен\-тов~\cite{Mall99}. Наличие шумовой со\-став\-ля\-ющей 
в~сигнале неизбежно приводит к~погрешностям в~оценке сигнала. При использовании 
метода мягкой пороговой обработки мож\-но по\-стро\-ить статистическую оценку 
среднеквадратичной по\-греш\-ности (рис\-ка)~\cite{DonJ95}. Свойства этой оценки в~моделях 
с~независимым и~коррелированным шумом исследовались 
в~работах~\cite{Mar09,MSH10-1,SH12-1,ESH14,SH16}. Показано, что при определенных 
условиях оценка рис\-ка является сильно со\-сто\-ятель\-ной и~асимптотически нормальной.

Однако при мягкой пороговой обработке в~оценке функции сигнала появляется 
дополнительное смещение. 

При жесткой пороговой обработке используется разрывная 
пороговая функция, что приводит к~появлению дополнительных артефактов, 
отсутствию устой\-чи\-вости при выборе порога и~не\-воз\-мож\-ности по\-стро\-ения 
несмещенной оценки сред\-не\-квад\-ра\-тич\-но\-го риска. 
В~работе~\cite{HL10} предложен стабилизированный вариант жесткой пороговой обработки, 
позволяющий обойти указанные недостатки. 

В~данной работе исследуются статистические 
свойства оценки сред\-не\-квад\-ра\-тич\-но\-го рис\-ка стабилизированной жест\-кой пороговой 
обработки в~модели с~коррелированным шумом. Предполагается, что при обработке 
используется <<универсальный>> порог. При определенных условиях на глад\-кость 
функции сигнала показано, что оценка рис\-ка, как и~в~случае мяг\-кой пороговой 
обработки, является асимп\-то\-ти\-чески нормальной и~сильно со\-сто\-ятель\-ной. Данные 
свойства служат обоснованием использования этой оценки при по\-стро\-ении доверительных 
интервалов для тео\-ре\-ти\-че\-ско\-го среднеквадратичного риска.

\section{Модель данных с~долгосрочной зависимостью}

В данной работе предполагается, что функция сигнала~$f$ задана на отрезке $[0,1]$ 
и~равномерно регулярна по Липшицу с~некоторым показателем $\gamma\hm>0$. 
Вейв\-лет-раз\-ло\-же\-ние функции~$f$ пред\-став\-ля\-ет собой ряд
\begin{equation}
\label{Wavelet_Decomp}
f=\sum\limits_{j,k\in Z}\langle f,\psi_{jk}\rangle\psi_{jk}\,,
\end{equation}
где $\psi_{jk}(t)=2^{j/2}\psi(2^jt-k)$, а $\psi(t)$~--- некоторая материнская 
вейв\-лет-функ\-ция. Индекс~$j$ в~\eqref{Wavelet_Decomp} называется масштабом, 
а~индекс $k$~--- сдвигом. Функция~$\psi$ должна удовле\-тво\-рять определенным 
требованиям, однако ее можно выбрать таким образом, чтобы она обладала 
некоторыми полезными свойствами, например была~$M$ раз дифференцируемой, имела
 заданное чис\-ло~$M$ нулевых моментов и~достаточно быст\-ро убывала на бес\-ко\-неч\-ности. 
 Известно~\cite{Mall99}, что если $M\hm\geqslant\gamma$, то найдется такая 
 константа $C_f\hm>0$, что
\begin{equation}
\label{Coeff_Decay}
\langle f,\psi_{jk}\rangle\leqslant\fr{C_f}{2^{j\left(\gamma+1/2\right)}}\,.
\end{equation}
Всюду далее предполагается, что используются вейв\-ле\-ты Мейера~\cite{Mall99}, 
обладающие нужным чис\-лом нулевых моментов и~непрерывных производных.

На практике регистрируются дискретные отсчеты функции сигнала $f_i\hm=
f\left(i/2^J\right)$, $i\hm=1,\ldots, 2^J$ (считается, что чис\-ло этих отсчетов 
равно~$2^J$ для некоторого $J\hm>0$). Дискретное вейв\-лет-пре\-обра\-зо\-ва\-ние 
пред\-став\-ля\-ет собой умножение вектора\linebreak
 из значений~$f_i$  на ортогональную мат\-ри\-цу, 
определяемую вейв\-лет-функ\-ци\-ей~$\psi$~\cite{Mall99}. 
При этом дискретные вейв\-лет-ко\-эф\-фи\-ци\-ен\-ты связаны с~непрерывными 
коэффициентами разложения в~\eqref{Wavelet_Decomp}\linebreak
 сле\-ду\-ющим образом: 
$\mu_{jk}\hm\approx 2^{J/2}\langle f,\psi_{jk}\rangle$~\cite{Mall99}. 
Это приближение тем точ\-нее, чем больше~$J$.

В~реальных наблюдениях всегда присутствует шум. 
В~данной работе рас\-смат\-ри\-ва\-ет\-ся модель коррелированного шума. 
Пусть $\{e_i, i \hm\in \mathbb{Z}\}$~--- ста\-цио\-нар\-ный гауссовский процесс 
с~ковариационной по\-сле\-до\-ва\-тель\-ностью $r_k \hm= \cov (e_i,e_{i+k})$. Будем 
полагать, что~$e_i$ имеют нулевое сред\-нее и~единичную дис\-пер\-сию. 
Также предположим, что автоковариационная функция шума убывает со ско\-ростью 
$r_k \hm\sim Ak^{-\alpha}$, где $0 \hm< \alpha\hm <1$, что соответствует 
долгосрочной за\-ви\-си\-мости между наблюдениями~\cite{JS97}.

Рассмотрим следующую модель данных:
\begin{equation*}
Y_i = f_i + e_i\,, \enskip i = 1, \ldots, 2^J\,.
%\label{Data_Model}
\end{equation*}
Для $t\in [0,1]$ определим наблюдаемый процесс
\begin{equation*}
Y_J(t) = \fr{1}{2^J} \sum\limits_{j=1}^{\left[2^Jt\right]} 
\!Y_i = F_J(t)+ \fr{1}{2^J} \sum\limits_{i=1}^{\left[2^Jt\right]}\! e_i\,,
\end{equation*}
где $F_J(t)=1/2^J \sum\nolimits_{i=1}^{\left[2^Jt\right]} f(i/2^J)$~--- <<суммарный сигнал>>. 
Положим $\tau^2 \hm= 2A/((1\hm-\alpha)(2\hm-\alpha))$ 
(без ограничения общ\-ности далее предполагается, что $\tau\hm=1$) и~$H \hm= 
1\hm- \alpha/2 \hm\in (1/2,1)$.

Определим дробное броуновское движение $\mathbf{B}_H(t)$ как
гауссовский процесс на~$\mathbb{R}$ с~нулевым средним  
и~ковариационной функцией
\begin{equation*}
r(s,t) = \fr{V_H}{2}\left(|s|^{2H} + |t|^{2H} - |t-s|^{2H}\right)\,, \enskip
 s,t \in \mathbb{R}\,,
\end{equation*}
где
\begin{equation*}
V_H = \Variance \left(\mathbf{B}_H(1)\right) = 
\fr{-\Gamma(2-2H)\cos(\pi H)}{\pi H(2H-1)}\,.
\end{equation*}
Лемма~5.1 из~\cite{T75} показывает, что
\begin{equation*}
2^{\alpha J/2}\left(Y_J(t) - F_J(t)\right) 
\Rightarrow \mathbf{B}_H(t)\,, \enskip t\in[0,1].
\end{equation*}
Таким образом, полагая $\epsilon \hm= 2^{-J/2}$, можно аппроксимировать 
наблюдаемый процесс~$Y_J(t)$ с~по\-мощью~$Y(t)$ для $t \hm\in [0,1]$:
\begin{equation}
\label{Scale_proc}
Y(t) = F(t) + \epsilon^{\alpha} \mathbf{B}_H(t)\,.
\end{equation}
Применяя к~(\ref{Scale_proc}) вейв\-лет-раз\-ло\-же\-ние 
и~аппроксимируя его дискретным вейв\-лет-пре\-об\-ра\-зо\-ва\-ни\-ем, 
приходим к~сле\-ду\-ющей модели дискретных вейв\-лет-ко\-эф\-фи\-ци\-ен\-тов~\cite{JS97,J99}:
\begin{equation}
X_{jk} = \mu_{jk} +  2^{{(1-\alpha)(J-j)}/{2}} z_{jk}\,,
\label{Wav_LRD_model}
\end{equation}
где 
$$
z_{jk}=2^{{j(1-\alpha)}/{2}} \int \psi_{jk}\, d\mathbf{B}_H\,. 
$$

Шумовые коэффициенты~$z_{jk}$ имеют стандартное нормальное распределение, 
но не являются независимыми. Дис\-пер\-сия коэффициентов модели~(\ref{Wav_LRD_model}) 
имеет вид:
$$
\sigma_j^2= 2^{(1-\alpha)(J-j)}\,.
$$


\section{Стабилизированная жесткая обработка}

Для подавления шума и~построения оценки функции сигнала к~коэффициентам~$X_{jk}$ 
обычно применяется функция жесткой пороговой обработки $\rho_{H}(x,T_j)
\hm=y\I(\abs{x}>T_j)$ или функция мяг\-кой пороговой обработки $\rho_{S}(x,T_j)
\hm=\mathrm{sgn}(x)\left(\abs{x}-T_j\right)_{+}$ с~порогом~$T_j$, 
который может зависеть от мас\-шта\-ба~$j$, но не зависит от сдвига~$k$. 
Смысл пороговой обработки заключается в~удалении до\-ста\-точ\-но маленьких коэффициентов, 
которые считаются шумом.

Как уже отмечалось, функция~$\rho_{H}$ разрывна, что приводит к~отсутствию 
устой\-чи\-вости, а~функция~$\rho_{S}$ приводит к~по\-яв\-ле\-нию дополнительного 
смещения в~оценке функции сигнала. В~работе~\cite{HL10} предложен альтернативный 
метод пороговой обработки, являющийся сгла\-жен\-ным аналогом жест\-кой пороговой 
обработки. В~этом методе оценки~$\mu_{jk}$ вы\-чис\-ля\-ют\-ся по формулам:
\begin{equation*}
\widehat{\mu}_{jk}=
\Expect \left[\rho_{H}(X_{jk}+\lambda\xi_{jk},T_j)|X_{jk}\right],
\end{equation*}
где случайные величины~$\xi_{jk}$ имеют стандартное нормальное распределение и~не 
зависят от~$X_{jk}$, а~$\lambda\hm>0$~--- 
параметр стабилизации, отвечающий за степень сглаживания. 
Вы\-чис\-ляя математическое ожидание, получаем:
\begin{multline*}
\hspace*{-1.5mm}\widehat{\mu}_{jk}=X_{jk}\left[\Phi\!\left(-\fr{T_j+X_{jk}}
{\lambda}\!\right)+1-
\Phi\left(\!\fr{T_j-X_{jk}}{\lambda}\!\right)\right]+{}\hspace*{0.43806pt}\hspace*{-0.87613pt}\\
{}+\lambda\left[
\phi\left(\fr{T_j-X_{jk}}{\lambda}\!\right)-
\phi\left(\fr{T_j+X_{jk}}{\lambda}\right)\right].
\end{multline*}
Достоинством такого метода является бесконечная диф\-фе\-рен\-ци\-ру\-емость~$\widehat{\mu}_{jk}$ 
по~$X_{jk}$, что приводит к~более устойчивым оценкам~\cite{HL10}. Заметим также, 
что при $\lambda\hm\to0$ получается обычный метод жест\-кой пороговой обработки.

Среднеквадратичная по\-греш\-ность (риск) описанного метода определяется по формуле:
\begin{equation*}
\label{Risk}
R_J(f)=\sum\limits_{j=0}^{J-1}\sum\limits_{k=0}^{2^j-1}\Expect
\left(\widehat{\mu}_{jk}-\mu_{jk}\right)^2.
\end{equation*}
В~\cite{HL10} показано, что
\begin{multline*}
\Expect\left(\widehat{\mu}_{jk}-\mu_{jk}\right)^2={}\\
{}=
\Expect\left[(X_{jk}-\widehat{\mu}_{jk})^2+2\sigma_j^2
\fr{\partial}{\partial X_{jk}}\,\widehat{\mu}_{jk}\right]-\sigma_j^2,
\end{multline*}
где

\noindent
\begin{multline*}
\fr{\partial}{\partial X_{jk}}\,\widehat{\mu}_{jk}={}\\
{}=
\left[\Phi\left(-\fr{T_j+X_{jk}}{\lambda}\right)+1-
\Phi\left(\fr{T_j-X_{jk}}{\lambda}\right)\right]+{}\\
{}+
\fr{T_j}{\lambda}\left[\phi\left(\fr{T_j-X_{jk}}{\lambda}\right)+
\phi\left(\fr{T_j+X_{jk}}{\lambda}\right)\right].
\end{multline*}
Таким образом, величина
\begin{multline}
\label{Risk_Estimate}
\widehat{R}_J(f)={}\\
{}+\sum\limits_{j=0}^{J-1}\sum\limits_{k=0}^{2^j-1}
\left[(X_{jk}-\widehat{\mu}_{jk})^2+2\sigma_j^2
\fr{\partial}{\partial X_{jk}}\,\widehat{\mu}_{jk}-\sigma_j^2\right]
\end{multline}
является несмещенной оценкой среднеквадратичного риска~$R_J(f)$, не зависящей от 
ненаблюдаемых <<чистых>> значений~$\mu_{jk}$.

В данной работе параметр~$\lambda$ предполагается фиксированным, а~в~качестве~$T_j$ 
для каждого мас\-шта\-ба~$j$ выбирается <<универсальный>> порог $T_j\hm=
\sigma_j\sqrt{2\ln 2^J}$, который поз\-во\-ля\-ет достичь хороших 
результатов при подавлении шума и~обеспечивает бли\-зость сред\-не\-квад\-ра\-тич\-но\-го 
рис\-ка к~минимальному как при жесткой, так и~при мягкой пороговой обработке~\cite{DJ94}. В следующем разделе устанавливаются свойства асимптотической нормальности и~сильной состоятельности оценки (\ref{Risk_Estimate}). Эти свойства служат обоснованием использования $\widehat{R}_J(f)$ при построении доверительных интервалов для $R_J(f)$.

\vspace*{-3pt}

\section{Статистические свойства оценки среднеквадратичного риска}

Покажем, что оценка (\ref{Risk_Estimate}) является асимптотически нормальной.

\smallskip

\noindent
\textbf{Теорема~1.}\ \textit{Пусть $\alpha\hm>1/2$, а функция~$f$ 
задана на отрезке $[0,1]$ и~равномерно регулярна по Липшицу с~показателем 
$\gamma\hm > (4\alpha\hm-2)^{-1}$. Тогда при стабилизированной жест\-кой пороговой 
обработке с~<<универсальными>> порогами~$T_j$ имеет место схо\-ди\-мость по 
распределению}
\begin{align}\label{risk_norm}
{\sf P}\left(\fr{\widehat{R}_J(f) - R_J(f)}{ D_J }<x\right) \to \Phi(x) \
\mbox{при } J \rightarrow \infty\,.
\end{align}
\textit{Здесь $\Phi(x)$~--- функция распределения стандартного нормального закона; 
$D_J^2\hm=C_\alpha 2^J$, где  $C_\alpha$~--- константа, за\-ви\-ся\-щая только от~$\alpha$ 
и~выбранного вейв\-лет-ба\-зиса}.

\smallskip

\noindent
\textbf{Замечание.}\ 
На практике, например при по\-стро\-ении асимптотических доверительных интервалов 
для сред\-не\-квад\-ра\-тич\-но\-го рис\-ка, необходимо знать константу~$C_\alpha$. 
В~отличие от случая независимых наблюдений эта константа зависит от выбранного 
вейв\-лет-ба\-зи\-са (напомним, что по предположению базис строится на основе 
вейвлетов Мейера). Константу~$C_\alpha$ мож\-но вы\-чис\-лить достаточно точ\-но,
 пользуясь методикой работы~\cite{E15}.
 
 \smallskip

\noindent
Д\,о\,к\,а\,з\,а\,т\,е\,л\,ь\,с\,т\,в\,о\,. 
По\-сколь\-ку в~силу условий тео\-ре\-мы $(2\gamma+1)^{-1}\hm<1\hm-(2\alpha)^{-1}$, 
мож\-но выбрать такое~$p$, что
$(2\gamma+1)^{-1}\hm<p\hm<1\hm-(2\alpha)^{-1}$. Обозначим сла\-га\-емые 
в~(\ref{Risk_Estimate}) через~$F_{jk}$ и~запишем дробь под ве\-ро\-ят\-ностью 
в~\eqref{risk_norm} в~виде:

\vspace*{-4pt}

\noindent
\begin{multline}
\label{Two_Sums}
\fr{\widehat{R}_J(f)-R_J(f)}{D_J}=\fr{1}{D_J}\sum\limits_{j=0}^{[pJ]}
\sum\limits_{k=0}^{2^j-1}\left[F_{jk}-\Expect F_{jk}\right]+{}\\
{}+
\fr{1}{D_J}\sum\limits_{j=[pJ]+1}^{J-1}\sum\limits_{k=0}^{2^j-1}
\left[F_{jk}-\Expect F_{jk}\right].
\end{multline}
Учитывая вид~$T_j$, мож\-но убедиться, что существует такая константа $C_F\hm>0$, что

\vspace*{3pt}

\noindent
\begin{equation}
\label{Bound}
\abs{F_{jk}-\Expect F_{jk}} \leqslant C_F J2^{(J-j)(1-\alpha)}\enskip \mbox{п.\ в.}
\end{equation}
Таким образом,

\vspace*{-2pt}

\noindent
\begin{multline*}
\abs{\sum\limits_{i=0}^{[pJ]}\sum\limits_{l=0}^{2^i-1} [F_{jk}-\Expect F_{jk}]}
\leqslant{}\\
{}\leqslant C_F\sum\limits_{i=0}^{[pJ]}\sum\limits_{l=0}^{2^i-1}
 С J2^{(J-i)(1-\alpha)} \leqslant C'_F  J 2^{J(1-\alpha+\alpha p)}\ \  
 \mbox{п.\ в.}
\end{multline*}
с некоторой константой $C'_F\hm>0$.
Следовательно, так как $1\hm-\alpha\hm+\alpha p\hm<1/2$, первая сумма 
в~\eqref{Two_Sums} стремится к~нулю п.\ в.\ при $J\hm\to\infty$.

 Повторяя рассуждения работ~\cite{JS97,J99}, мож\-но убедиться, что 
 по\-сле\-до\-ва\-тель\-ность $\bigl\{F_{jk}\bigr\}$ обладает свойством $\rho$-пе\-ре\-ме\-ши\-ва\-ния 
 и,~следовательно, обладает свойством $\alpha$-пе\-ре\-ме\-ши\-ва\-ния~\cite{B05}.

Далее, рассуждая как в~работе~\cite{ESH14} и~используя~\eqref{Coeff_Decay}, 
мож\-но показать, что при выполнении условий тео\-ре\-мы существует такая 
константа $C_\alpha\hm>0$, за\-ви\-ся\-щая только от~$\alpha$ и~вы\-бран\-но\-го 
вейв\-лет-ба\-зи\-са,~что

\vspace*{4pt}

\noindent
\begin{equation*}
%\label{Var_Order}
\lim\limits_{J\to\infty} \fr{1}{D^2_J}\,\Variance\left[
\sum\limits_{j=[pJ]+1}^{J-1}\sum\limits_{k=0}^{2^j-1}   
\left[F_{jk}-\Expect F_{jk}\right]\right] =1\,.
\end{equation*}
Кроме того, легко видеть, что

\vspace*{2pt}

\noindent
\begin{equation*}
\sup_{J>0} \fr{1}{D^2_J}\sum\limits_{j=[pJ]+1}^{J-1}\sum\limits_{k=0}^{2^j-1} 
\Variance  F_{jk} < \infty\,.
\end{equation*}

 Наконец, выполнено условие Линдеберга: для любого $\eps\hm>0$
 
\columnbreak
 
 \noindent
\begin{multline}
\label{Norm_Cond}
\hspace*{-3mm}\fr{1}{D^2_J}\sum\limits_{j=[pJ]+1}^{J-1}\!\sum\limits_{k=0}^{2^j-1} \!\Expect  
\left( F_{jk} - \Expect F_{jk}\right)^2
\mathbf{1}\left( |F_{jk} - \Expect  F_{jk}| >\right.\\
\left.>\eps D_J\right) \rightarrow 0\ \mbox{при }J\rightarrow\infty\,.
\end{multline}
Действительно, так как 
$$
\abs{F_{jk}-\Expect F_{jk}} \leqslant C_F
J2^{(J-j)(1-\alpha)}\ \mbox{п.~в.}\,,
$$ 
а~$D_J^2\hm=C_\alpha 2^J$, то при
$\alpha\hm>1/2$, начиная с~некоторого~$J$, все индикаторы 
в~(\ref{Norm_Cond}) обращаются в~ноль.

 Таким образом, выполнены все условия тео\-ре\-мы~2.1 из работы~\cite{P96} 
 и~справедливо~\eqref{risk_norm}. Тео\-ре\-ма доказана.

\smallskip

Оценка риска~$\widehat{R}_J(f)$ также является силь\-но со\-сто\-ятель\-ной, причем 
при более слабых ограничениях на~$\alpha$ и~ре\-гу\-ляр\-ность~$f$.

\smallskip

\noindent
\textbf{Теорема~2.}\ \textit{Пусть функция $f\hm\in  L^2([0,1])$. Тогда}
\begin{equation*}
\fr{\widehat{R}_J(f)-R_J(f)}{2^{\lambda J}}\rightarrow 0 \ \mbox{п. в.}\ 
\mbox{при } J\rightarrow\infty
\end{equation*}
\textit{при любом $\lambda>1/2$ в~случае $1/2\hm\leqslant\alpha\hm<1$ 
и~любом $\lambda\hm>1\hm-\alpha$ в~случае} $0\hm<\alpha\hm<1/2$.

\smallskip

Принимая во внимание оценку~\eqref{Bound}, доказательство этого утверж\-де\-ния 
практически пол\-ностью повторяет доказательство со\-от\-вет\-ст\-ву\-юще\-го свойства оцен\-ки 
рис\-ка в~работе~\cite{SH16}.

\vspace*{-9pt}

{\small\frenchspacing
 {%\baselineskip=10.8pt
 \addcontentsline{toc}{section}{References}
 \begin{thebibliography}{99}
\bibitem{Mall99}
\Au{Mallat S.} A~wavelet tour of signal processing.~--- New York, NY, USA: 
Academic Press, 1999. 857~p.

\bibitem{DonJ95}
\Au{Donoho D., Johnstone~I.\,M.} Adapting to unknown smoothness via wavelet shrinkage~// 
J.~Am. Stat. Assoc., 1995. Vol.~90. P.~1200--1224.

\bibitem{Mar09}
\Au{Маркин А.\,В.} Предельное распределение оценки рис\-ка при пороговой обработке 
вейв\-лет-ко\-эф\-фи\-ци\-ен\-тов~// Информатика и~её
применения, 2009. Т.~3. Вып.~4. С.~57--63.

\bibitem{MSH10-1}
\Au{Маркин А.\,В., Шестаков~О.\,В.} 
О~со\-сто\-ятель\-ности оценки рис\-ка при пороговой обработке вейв\-лет-ко\-эф\-фи\-ци\-ен\-тов~// 
Вестн. Моск.
 ун-та. Сер.~15: Вычисл. матем. и~киберн., 2010. №\,1. C.~26--34.

\bibitem{SH12-1}
\Au{Шестаков О.\,В.} Асимптотическая нор\-маль\-ность оценки рис\-ка пороговой обработки 
вейв\-лет-ко\-эф\-фи\-ци\-ен\-тов при выборе адап\-тив\-но\-го порога~// 
Докл. РАН, 2012. Т.~445. №\,5. С.~513--515.

\bibitem{ESH14}
\Au{Ерошенко А.\,А., Шестаков~О.\,В.} 
Асимптотические свойства оценки риска при пороговой обработке вейв\-лет-ко\-эф\-фи\-ци\-ен\-тов 
в~модели
с~коррелированным шумом~// Информатика и~её применения, 2014. Т.~8. Вып.~1. С.~36--44.

\bibitem{SH16}
\Au{Шестаков О.\,В.} Сходимость почти всюду оценки рис\-ка пороговой обработки 
вейв\-лет-ко\-эф\-фи\-ци\-ен\-тов в~модели с~коррелированным шумом~// 
Вестн. Моск.
\linebreak\vspace*{-12pt}

\pagebreak

\noindent
 ун-та. Сер.~15: Вычисл. матем. и~киберн., 2016. №\,3. C.~19--22.

\bibitem{HL10}
\Au{Huang H.-C., Lee~T.\,C.\,M.} 
Stabilized thresholding with generalized sure for image denoising~// 
IEEE 17th  Conference (International) on Image Processing Proceedings.~--- 
IEEE, 2010. P.~1881--1884.

\bibitem{JS97}
\Au{Johnstone I.\,M., Silverman~B.\,W.} 
Wavelet threshold estimates for data with correlated noise~// J.~Roy. Stat.
Soc.~B, 1997. Vol.~59. P.~319--351.

%\pagebreak
\bibitem{T75}
\Au{Taqqu M.\,S.} Weak convergence to fractional Brownian motion and to the 
Rosenblatt process~// Z.~Wahrscheinlichkeit., 1975. Vol.~31. P.~287--302.

\bibitem{J99}
\Au{Johnstone I.\,M.} 
Wavelet shrinkage for correlated data and inverse problems adaptivity results~// 
Stat. Sinica, 1999. Vol.~9. P.~51--83.



\bibitem{DJ94}
\Au{Donoho D., Johnstone~I.\,M.} 
Ideal spatial adaptation via wavelet shrinkage~// Biometrika, 1994. Vol.~81. No.\,3.
P.~425--455.

\bibitem{E15}
\Au{Ерошенко А.\,А.} 
Статистические свойства оценок сигналов и~изображений при пороговой обработке 
коэффициентов в~вейв\-лет-раз\-ло\-же\-ни\-ях. Дис.\ \ldots\ канд. физ.-мат. наук.~--- 
М.:~МГУ, 2015. 82~с.

\bibitem{B05}
\Au{Bradley R.\,C.} 
Basic properties of strong mixing conditions. A~survey and some open questions~// 
Probab. Surveys, 2005. Vol.~2. P.~107--144.

\bibitem{P96}
\Au{Peligrad M.} On the asymptotic normality of sequences of weak dependent 
random variables~// J.~Theor. Probab., 1996. Vol.~9. No.\,3. P.~703--715.

 \end{thebibliography}

 }
 }

\end{multicols}

\vspace*{-6pt}

\hfill{\small\textit{Поступила в~редакцию 09.10.17}}

\vspace*{6pt}

%\newpage

%\vspace*{-24pt}

\hrule

\vspace*{2pt}

\hrule

\vspace*{-5pt}


\def\tit{UNBIASED RISK ESTIMATE 
OF~STABILIZED HARD THRESHOLDING IN~THE~MODEL WITH~A~LONG-RANGE DEPENDENCE}

\def\titkol{Unbiased risk estimate 
of~stabilized hard thresholding in~the~model with~a~long-range dependence}

\def\aut{O.\,V.~Shestakov$^{1,2}$}

\def\autkol{O.\,V.~Shestakov}

\titel{\tit}{\aut}{\autkol}{\titkol}

\vspace*{-14pt}


\noindent
$^1$Department of Mathematical Statistics, Faculty of Computational Mathematics 
and Cybernetics,\linebreak
$\hphantom{^1}$M.\,V.~Lomonosov Moscow State University, 1-52~Leninskiye Gory, 
GSP-1, Moscow 119991, Russian\linebreak
$\hphantom{^1}$Federation


\noindent
$^2$Institute of Informatics Problems, Federal Research Center 
``Computer Science and Control'' of the Russian\linebreak
$\hphantom{^1}$Academy of Sciences, 
44-2~Vavilov Str., Moscow 119333, Russian Federation


\def\leftfootline{\small{\textbf{\thepage}
\hfill INFORMATIKA I EE PRIMENENIYA~--- INFORMATICS AND
APPLICATIONS\ \ \ 2018\ \ \ volume~12\ \ \ issue\ 2}
}%
 \def\rightfootline{\small{INFORMATIKA I EE PRIMENENIYA~---
INFORMATICS AND APPLICATIONS\ \ \ 2018\ \ \ volume~12\ \ \ issue\ 2
\hfill \textbf{\thepage}}}

\vspace*{2pt}


\Abste{De-noising methods for processing signals and images, based on 
the thresholding of wavelet decomposition coefficients, have become 
popular due to their simplicity, speed, and the ability to adapt to signal 
functions that have a~different degree of regularity at different locations. 
An analysis of inaccuracies of these methods is an important practical task, 
since it makes it possible to evaluate the quality of both the methods 
themselves and the equipment used for processing. The present author 
considers the recently proposed stabilized hard thresholding method which avoids 
the main disadvantages of the popular soft and hard thresholding techniques. 
The statistical properties of this method are studied. In the model with an 
additive Gaussian noise, the author analyzes the unbiased risk estimate. 
Assuming that the noise coefficients have a~long-range dependence, the author 
formulates the conditions under which strong consistency and asymptotic normality 
of the unbiased risk estimate take place. The results obtained make it possible 
to construct asymptotic confidence intervals 
for the threshold processing errors using only observable data.}

\KWE{wavelets; thresholding; unbiased risk estimate; correlated noise; 
asymptotic normality}




\DOI{10.14357/19922264180202}

\vspace*{-18pt}

 
\Ack
\noindent
The work was partly supported by the Russian Foundation for 
Basic Research (project 16-07-00736).


\vspace*{-1pt}

  \begin{multicols}{2}

\renewcommand{\bibname}{\protect\rmfamily References}
%\renewcommand{\bibname}{\large\protect\rm References}

{\small\frenchspacing
 {%\baselineskip=10.8pt
 \addcontentsline{toc}{section}{References}
 \begin{thebibliography}{99}

\bibitem{1-ss}
\Aue{Mallat, S.} 1999. 
\textit{A~wavelet tour of signal processing}. New York, NY: Academic Press. 857~p.

\bibitem{2-ss}
\Aue{Donoho, D., and I.\,M.~Johnstone.} 1995. Adapting to unknown smoothness via 
wavelet shrinkage. \textit{J.~Am. Stat. Assoc.} 90:1200--1224.

\columnbreak

\bibitem{3-ss}
\Aue{Markin, A.\,V.} 2009. Predel'noe raspredelenie otsenki riska pri porogovoy 
obrabotke veyvlet-koeffitsientov [Limit distribution of risk estimate of wavelet 
coefficient thresholding]. \textit{Informatika i~ee Primeneniya~--- Inform. Appl.}
3(4):57--63.

\vspace*{-2pt}

\bibitem{4-ss}
\Aue{Markin, A.\,V., and O.\,V.~Shestakov}. 2010. Consistency of risk estimation 
with thresholding of wavelet coefficients. \textit{Mosc. Univ. Comput. Math. 
Cybern.} 34(1):22--30.

\pagebreak

\bibitem{5-ss}
\Aue{Shestakov, O.\,V.}
 2012. Asymptotic normality of adaptive wavelet thresholding risk estimation. 
 \textit{Dokl. Math.} 86(1):556--558.

\bibitem{6-ss}
\Aue{Eroshenko, A.\,A., and O.\,V.~Shestakov.} 2014. Asimptoticheskie svoystva 
otsenki riska pri porogovoy obrabotke veyvlet-koeffitsientov v~modeli 
s~korrelirovannym shumom
[Asymptotic properties of wavelet thresholding risk estimate in the model 
of data with correlated noise]. \textit{Informatika i~ee Primeneniya~--- Inform. Appl.}
8(1):36--44.

\bibitem{7-ss}
\Aue{Shestakov, O.\,V.} 2016. Almost everywhere convergence of 
a~wavelet thresholding risk estimate in a model with correlated noise. 
\textit{Mosc. Univ. Comput. Math. Cybern.} 40(3):114--117.

\bibitem{8-ss}
\Aue{Huang, H.-C. and T.\,C.\,M.~Lee.} 2010. 
Stabilized thresholding with generalized sure for image denoising. 
\textit{IEEE 17th Conference (International) on Image Processing Proceedings}. 
IEEE. 1881--1884.

\bibitem{9-ss}
\Aue{Johnstone, I.\,M., and B.\,W.~Silverman.} 1997.  
Wavelet threshold estimates for data with correlated noise. 
\textit{J.~Roy. Stat. Soc.~B} 59:319--351.

\bibitem{11-ss}
\Aue{Taqqu, M.\,S.} 1975.  Weak convergence to fractional Brownian motion and to 
the Rosenblatt process. \textit{Z.~Wahrscheinlichkeit.} 31:287--302.

\bibitem{10-ss}
\Aue{Johnstone, I.\,M.} 1999. Wavelet shrinkage for correlated data and inverse 
problems adaptivity results.  \textit{Stat. Sinica} 9:51--83.



\bibitem{12-ss}
\Aue{Donoho, D., and I.\,M.~Johnstone.} 1994. Ideal spatial adaptation via 
wavelet shrinkage. \textit{Biometrika} 81(3):425--455.

\bibitem{13-ss}
\Aue{Eroshenko, A.\,A.} 2015. Statisticheskie svoystva otsenok signalov 
i~izobrazheniy pri porogovoy obrabotke ko\-ef\-fi\-tsi\-en\-tov v~veyvlet-razlozheniyakh
[Statistical properties of signal and 
image estimates under thresholding of coefficients in wavelet 
decompositions].   Moscow: MSU. PhD Diss. 82~p. 

\bibitem{14-ss}
\Aue{Bradley, R.\,C.} 2005. 
Basic properties of strong mixing conditions. A~survey and some open questions. 
\textit{Probab. Surveys}  2:107--144.

\bibitem{15-ss}
\Aue{Peligrad, M.} 1996. On the asymptotic normality of sequences of 
weak dependent random variables. \textit{J.~Theor. Probab.} 9(3):703--715.
\end{thebibliography}

 }
 }

\end{multicols}

\vspace*{-3pt}

\hfill{\small\textit{Received October 9, 2017}}

%\vspace*{-24pt}



\Contrl

\noindent
\textbf{Shestakov Oleg V.} (b.\ 1976)~--- 
Doctor of Sciences in physics and mathematics, associate professor, 
Department of Mathematical Statistics, Faculty of Computational Mathematics 
and Cybernetics, M.\,V.~Lomonosov Moscow State University, 1-52~Leninskiye Gory, 
GSP-1, Moscow 119991, Russian Federation; senior scientist, Institute of 
Informatics Problems, Federal Research Center ``Computer Science and Control''
of the Russian Academy of Sciences, 44-2~Vavilov Str., Moscow 119333, 
Russian Federation; \mbox{oshestakov@cs.msu.su}
\label{end\stat}


\renewcommand{\bibname}{\protect\rm Литература}   %2
\newcommand{\mujk}{\mu_{j,k}}
\newcommand{\psijk}{\psi_{j,k}}
\newcommand{\sumk}{\sum\limits_{k=0}^{2^j-1}}
\newcommand{\betajk}{\beta_{j,k}}



\def\stat{kudr+shest}

\def\tit{МИНИМИЗАЦИЯ ОШИБОК ВЫЧИСЛЕНИЯ ВЕЙВЛЕТ-КОЭФФИЦИЕНТОВ\\
ПРИ~РЕШЕНИИ ОБРАТНЫХ ЗАДАЧ$^*$}

\def\titkol{Минимизация ошибок вычисления вейвлет-коэффициентов
при~решении обратных задач}

\def\aut{А.\,А.~Кудрявцев$^1$, О.\,В.~Шестаков$^2$}

\def\autkol{А.\,А.~Кудрявцев, О.\,В.~Шестаков}

\titel{\tit}{\aut}{\autkol}{\titkol}

\index{Кудрявцев А.\,А.}
\index{Шестаков О.\,В.}
\index{Kudryavtsev A.\,A.}
\index{Shestakov O.\,V.}




{\renewcommand{\thefootnote}{\fnsymbol{footnote}} \footnotetext[1]
{Работа выполнена при частичной финансовой поддержке РФФИ (проект 18-07-00252).}}


\renewcommand{\thefootnote}{\arabic{footnote}}
\footnotetext[1]{Московский государственный университет им.~М.\,В.~Ломоносова, 
факультет вычислительной математики и~кибернетики, \mbox{nubigena@mail.ru}}
\footnotetext[2]{Московский государственный университет им.~М.\,В.~Ломоносова, факультет 
 вычислительной математики и~кибернетики; Институт проб\-лем информатики 
 Федерального исследовательского центра <<Информатика и~управ\-ле\-ние>> 
 Российской академии наук, \mbox{oshestakov@cs.msu.su}}

%\vspace*{-6pt}


\Abst{Статистические обратные задачи возникают во многих прикладных областях, 
вклю\-чая медицину, астрономию, био\-ло\-гию, физику плазмы, химию и~т.\,п. 
При этом в~наблюдаемых данных всегда присутствуют по\-греш\-ности, 
связанные с~несовершенством оборудования, фоновыми шумами, дискретизацией 
данных и~др. Для уменьшения этих погрешностей необходимо 
применять специальные методы регуляризации, поз\-во\-ля\-ющие строить при\-бли\-жен\-ные 
устойчивые решения обратных задач. Классические методы регуляризации базируются 
на использовании оконного сингулярного разложения. Однако при таком подходе 
учитывается лишь вид оператора, участвующего в~формировании наблюдаемых 
данных, и~никак не учитываются свойства самого объекта наблюдения. Для линейных 
однородных операторов эта проб\-ле\-ма решается с~по\-мощью специальных методов 
вейв\-лет-ана\-ли\-за, поз\-во\-ля\-ющих адаптироваться одновременно к~виду 
оператора и~локальным особенностям функции, описывающей объект. 
В~данной работе рас\-смат\-ри\-ва\-ет\-ся задача обращения линейного однородного оператора 
при наличии шума в~наблюдаемых данных с~по\-мощью пороговой обработки коэффициентов 
вейв\-лет-раз\-ло\-же\-ния наблюдаемой функции. Вычисляются асимптотически 
оптимальные пороги и~порядки функции потерь при минимизации усред\-нен\-ной 
ве\-ро\-ят\-ности ошибки вычисления вейв\-лет-ко\-эф\-фи\-ци\-ен\-тов.}

\KW{вейвлеты; пороговая обработка; линейный однородный оператор; функция потерь}

\DOI{10.14357/19922264180203}
  
\vspace*{6pt}


\vskip 10pt plus 9pt minus 6pt

\thispagestyle{headings}

\begin{multicols}{2}

\label{st\stat}

\section{Введение}

Математические модели, лежащие в~основе многих прикладных задач анализа и~обработки 
сигналов, предполагают решение задачи обращения некоторого линейного оператора. 
Например, в~медицинских исследованиях бывает необходимо с~по\-мощью неинвазивных 
методов получить изоб\-ра\-же\-ние структуры ка\-ко\-го-ли\-бо внут\-рен\-не\-го органа. 
В~этом случае возникает задача реконструкции изоб\-ра\-же\-ния путем обращения 
проекционного оператора. Как правило, эта задача является некорректно 
поставленной, и~для ее решения требуется использовать методы регуляризации, 
поскольку в~реальных наблюдениях всегда присутствует шум. Классические методы, 
основанные на сингулярном разложении в~сочетании с~линейной регуляризацией, 
адаптируются к~виду оператора, но не принимают в~расчет свойства функции сигнала, 
которую необходимо оценить. При рассмотрении моделей с~линейными однородными 
операторами с~этими недостатками позволяют справиться методы вейв\-лет-раз\-ло\-же\-ния, 
предложенные в~работах~\cite{Abr98, Don95}. 

В~данной работе рас\-смат\-ри\-ва\-ет\-ся 
метод нелинейной регуляризации при обращении линейных однородных операторов, 
основанный на пороговой обработке, и~вы\-чис\-ля\-ют\-ся па\-ра\-мет\-ры этого метода, 
асимптотически оптимальные в~смыс\-ле функции потерь, основанной на вероятностях 
ошибок вы\-чис\-ле\-ния коэффициентов вейв\-лет-раз\-ло\-же\-ния.

\section{Метод обращения линейных однородных операторов}

Линейный оператор~$K$ называется однородным с~показателем $\alpha\hm>0$, если
$$
K\left[f\left(a\left(x-x_0\right)\right)\right]=a^{-\alpha}(Kf)\left[a\left(x-x_0\right)\right]
$$
для любого $x_0\hm\in\r$ и~любого $a\hm>0$.

Для построения оценки функции сигнала~$f$ в~данной работе рас\-смат\-ри\-ва\-ет\-ся метод 
вейг\-лет-вейв\-лет-раз\-ло\-же\-ния (vaguelette-wavelet decomposition). Функция~$Kf$ 
пред\-став\-ля\-ет\-ся в~виде ряда
\begin{equation*}
%\label{Kf_decomp}
Kf=\sum\limits_{j,k\in {\mathbb {Z}}}\langle Kf,\psi_{j,k}\rangle\psi_{j,k} 
\end{equation*}
по некоторому вейвлет-ба\-зи\-су~$\{\psi_{j,k}\}$. 
Здесь $\psi_{j,k}(x)\hm=2^{j/2}\psi(2^jx-k)$; $\psi(x)$~--- 
заданная материнская вейв\-лет-функ\-ция. Обозначим $\beta_{j,k}\hm=
\norm{K^{-1}\psijk}$. Тогда
в~силу од\-но\-род\-ности оператора $\beta_{j,k}\hm=\beta_{0,0}2^{\alpha j}$~\cite{Abr98}. 
Функция~$f$ пред\-став\-ля\-ет\-ся в~виде ряда:
\begin{equation}
\label{f_decomp}
f=\sum\limits_{j,k\in {\mathbb {Z}}}\beta_{j,k}\langle Kf,\psi_{j,k}\rangle u_{j,k}\,,
\end{equation}
где $u_{j,k}=K^{-1}\psi_{j,k}/\beta_{j,k}$. Функции~$u_{j,k}$ называются 
<<вейглетами>>. По\-сле\-до\-ва\-тель\-ность~$\{u_{j,k}\}$ не является ортонормированной,
однако если выполнены некоторые условия глад\-кости на~$K^*\psi$ и~$K^{-1}\psi$, 
то по\-сле\-до\-ва\-тель\-ность~$\{u_{j,k}\}$ образует
устойчивый базис в~$L^2(\r)$~\cite{Lee97}. Формула~\eqref{f_decomp} 
и~пред\-став\-ля\-ет собой метод обращения, называемый вейг\-лет-вейв\-лет-раз\-ло\-же\-нием.

\section{Модель данных}

Рассмотрим следующую модель данных:
$$
X_i=(Kf)_i+z_i\,,\enskip i=1,\ldots,2^J\,,
$$
где $X_i$~--- наблюдаемые данные; $K$~--- некоторый линейный однородный оператор; 
$f$~--- истинная  функция сигнала; $z_i$~--- случайные по\-греш\-ности
измерения, которые предполагаются независимыми и~име\-ющи\-ми одинаковое гауссово
распределение с~нулевым средним и~дисперсией~$\sigma^2$.

Дискретное вейвлет-пре\-об\-ра\-зо\-ва\-ние пред\-став\-ля\-ет 
собой умножение вектора значений функции~$Kf$ на ортогональную матрицу~$W$, 
опре\-де\-ля\-емую вейв\-лет-функ\-ци\-ей~$\psi$~\cite{Mal99}. 
При этом дискретные вейв\-лет-ко\-эф\-фи\-ци\-ен\-ты 
приближенно рав\-ны~$2^{J/2}\langle Kf,\psi_{jk}\rangle$. Это приближение 
тем точнее, чем больше~$J$. Таким образом, для дискретных вейв\-лет-ко\-эф\-фи\-ци\-ен\-тов 
разложения~$Kf$ принимается сле\-ду\-ющая модель:
$$
Y_{j,k}=\mujk+\zeta_{j,k}\,, \enskip j=0,\ldots,J-1\,,\ k=0,\ldots,2^j-1\,,
$$
где  $\mujk=2^{J/2}\langle Kf,\psi_{j,k}\rangle$, а~случайные величины~$\zeta_{j,k}$ 
в~силу ор\-то\-го\-наль\-ности~$W$ также независимы и~име\-ют 
нормальное распределение с~нулевым средним и~дисперсией~$\sigma^2$.

\section{Функция потерь для~метода пороговой обработки}

Одним из самых популярных методов подавления шума является пороговая обработка, 
смысл которой заключается в~обнулении коэффициентов, чьи абсолютные значения 
не превышают заданного порога.

Через $\hat{Y}_{j,k}$ обозначим оценку вейв\-лет-ко\-эф\-фи\-ци\-ен\-та, которая 
получается 
с~по\-мощью пороговой функции~$\rho_T(x)$ с~порогом~$T$: 
$\hat{Y}_{j,k}\hm=\rho_T(Y_{j,k})$. В~данной работе рас\-смат\-ри\-ва\-ют\-ся функции 
жесткой пороговой обработки $\rho_T^{(h)}(x)\hm=x \mathbf{1} (|x|\hm>T)$ 
и~мягкой пороговой обработки $\rho_T^{(s)}(x)\hm=\mathrm{sign}\,(x)(|x|\hm-T)_+$.

Рассмотрим функцию потерь, основанную на вероятностях ошибок 
вы\-чис\-ле\-ния вейв\-лет-ко\-эф\-фи\-ци\-ен\-тов. Пусть $(\xi,\eta)$~--- 
двумерная случайная величина, не зависящая от всех~$\zeta_{j,k}$ и~име\-ющая 
дискретное равномерное распределение на множестве индексов 
$j\hm=0,\ldots,J-1$, $k\hm=0,\ldots,2^j-1$. Для заданного критического
 уровня  $\varepsilon\hm>0$ определим функцию потерь:
 
 \noindent
\begin{multline*}
r_J(f)={\sf E}{\sf P}\left(\left|\beta_{\xi,\eta}\hat{Y}_{\xi,\eta}-
\beta_{\xi,\eta}\mu_{\xi,\eta}\right|>\varepsilon\ \big|\ \xi,\eta\right)={}\\[-1pt]
{}=\fr{\sum\nolimits_{j=0}^{J-1}\sum\nolimits_{k=0}^{2^j-1} 
{\sf P}\left(\abs{\betajk\hat{Y}_{j,k}-\betajk\mujk}>\varepsilon\right)}{2^J}\,,
\end{multline*}
т.\,е.~$r_J(f)$ представляет собой усредненную вероятность того, что ошибка 
вычисления вейв\-лет-ко\-эф\-фи\-ци\-ен\-та превысит критический уровень~$\varepsilon$. 
Такое определение функции потерь является обобщением определения, 
предложенного в~работе~\cite{SMS14}. В~той же работе показано, что оценки, 
целью которых является минимизация функции потерь, основанной на вероятностях 
ошибок вычисления вейв\-лет-ко\-эф\-фи\-ци\-ен\-тов, дают срав\-ни\-мые, а~иногда и~лучшие 
результаты, чем оценки, минимизирующие сред\-не\-квад\-ра\-тич\-ный риск.

Целью данной работы является поиск асимптотически оптимального 
порога и~оценивание максимального порядка функции потерь~$r_J$ пороговой 
обработки наблюдаемого сигнала~$Kf$ в~классе регулярных по Липшицу 
функций $\mathrm{Lip}(\gamma,L)$, где $\gamma\hm>0$~--- показатель; 
$L\hm>0$~--- константа Липшица~\cite{Mal99}:
\noindent
\begin{equation}
\label{Risk_Definition}
R_J=\sup\limits_{Kf\in\mathrm{Lip}(\gamma,L)}{r_J(f)}\,,
\end{equation}
т.\,е.\ поиск порога, асимптотически оптимального в~минимаксном смысле. 
Подробное исследование поведения асимптотически оптимального порога 
для среднеквадратичного рис\-ка можно найти в~работах~\cite{DJ98, Jan01}. 
Также в~\cite{DJ95} был предложен метод поиска адаптивного оптимального порога,
 с~по\-мощью ко-\linebreak\vspace*{-12pt}
 
 \pagebreak
 
 \noindent
 торого можно оценить среднеквадратичный риск пороговой
  обработки конкретной функции. Этот метод основан на построении несмещенной 
  оценки риска, статистические свойства которой подробно исследованы 
  в~работах~\cite{MSH10, SH12}. Для задачи оценивания функции сигнала по 
  прямым зашумленным наблюдениям асимптотически оптимальные значения порогов, 
  минимизирующие функцию потерь, основанную на вероятностях ошибок, в~том 
  числе для моделей с~негауссовым распределением шума, найдены 
  в~работах~\cite{KS16-1, KS16-2}.

Заметим, что <<разумный>> порог должен воз\-рас\-тать по~$J$ со ско\-ростью, 
не выше чем $C_T\sqrt{\ln2^J}$, где~$C_T$~--- 
положительная константа, зависящая от оператора~$K$ 
(подробнее см.~\cite{Abr98, Jan01}).

Далее символом $\asymp$ обозначается порядок рассматриваемой величины по~$J$, 
т.\,е.\ $a_J\asymp b_J$, если начиная с~некоторого~$J$ выполнено 
$C_1  b_J\hm\leq a_J\hm\leq  C_2 b_J$ для некоторых положительных констант~$C_1$ 
и~$C_2$.

Везде далее дополнительно будем предполагать, что вейв\-лет-функ\-ция имеет~$M$~нулевых 
моментов ($M\hm\geq\gamma$), $M$~раз непрерывно дифференцируема и~достаточно быст\-ро 
убывает на бес\-ко\-неч\-ности вместе со своими производными~\cite{Mal99}.

\vspace*{-6pt}

\section{Жесткая пороговая обработка}

Пусть $\hat{Y}_{j,k}=\rho_T^{(h)}(Y_{j,k})$. Пусть функция $g(J)\hm>0$ сколь 
угодно медленно убывает по~$J$ к~нулю.

\vspace{2pt}

\noindent
\textbf{Теорема~1.}\ \textit{Пусть $\gamma>\alpha$. При выборе асимптотически 
оптимального порога для жесткой пороговой обработки функция 
потерь}~(\ref{Risk_Definition}) \textit{начиная с~некоторого~$J$ 
удовле\-тво\-ря\-ет неравенствам}:

\vspace*{-2pt}

\noindent
\begin{multline*}
C_1^{(h)}\cdot2^{-({2\gamma-2\alpha})J/({2\gamma-2\alpha+1})}
g(J)\leq R_J\leq{}\\[-1pt]
{}\leq
 C_2^{(h)}\cdot2^{-(({2\gamma-2\alpha})/({2\gamma-2\alpha+1}))J}
 J^{1/(2\gamma-2\alpha+1)}\,,
 \end{multline*}
 
 \vspace*{-4pt}
 
 \noindent
\textit{где $C_1^{(h)}$ и~$C_2^{(h)}$~--- некоторые положительные константы.
Для асимптотически оптимального зна\-чения порога, ми\-ни\-ми\-зи\-ру\-юще\-го 
порядок функции\linebreak потерь}~(\ref{Risk_Definition}) \textit{при жест\-кой пороговой обработке, 
справедливо неравенство}:
$$
T_*^{(h)}-T_2^{(h)}\leq T\leq T_*^{(h)}-T_1^{(h)},
$$ 
\textit{где}
%\begin{equation}\label{T_Main_Hard}

\noindent
\begin{align*}
T_*^{(h)}&=\sigma\sqrt{\fr{4\gamma-4\alpha}{2\gamma-2\alpha+1}\,\ln2^J};
\\[-2pt]
T_1^{(h)}&=\sigma\sqrt{\fr{2\gamma-2\alpha+1}{4\gamma-4\alpha}}
\, \fr{\ln\left((\ln 2^J)^{1/2}g(J)\right)}{\sqrt{\ln 2^J}}\,;
\\[-2pt]
T_2^{(h)}&=\sigma\sqrt{\fr{2\gamma-2\alpha+1}{4\gamma-4\alpha}}\,
\fr{\ln\left((\ln 2^J)^{1/2}J^{1/(2\gamma-2\alpha+1)}\right)}{\sqrt{\ln 2^J}}\,.
\end{align*}

\noindent
{Д\,о\,к\,а\,з\,а\,т\,е\,л\,ь\,с\,т\,в\,о\,.}\ \ 
При выполнении приведенных выше ограничений на вейв\-лет-функ\-цию 
справедливо неравенство~\cite{Mal99}:

\noindent
\begin{multline}
\label{Wavelet_CoeffDecacy}
\abs{\mujk}\leq\fr{A2^{J/2}}{2^{j\left(\gamma+1/2\right)}}\,, \\
j=0,\ldots,J-1\,,\ k=0,\ldots,2^j-1\,,
\end{multline}
где $A$~--- некоторая константа, зависящая от~$\gamma$ и~$L$ и~не зависящая от~$J$.


Поскольку $\gamma\hm>\alpha$, неравенство~(\ref{Wavelet_CoeffDecacy}) 
поз\-во\-ля\-ет разбить все множество индексов $\{0,\ldots,J-1\}$ 
на три класса в~за\-ви\-си\-мости от величины~$|\mujk|$. Пусть индексы~$j_1$ и~$j_2$ 
($j_1\hm< j_2$) таковы, что
\begin{equation*}
\left\vert \mujk\right\vert \leq 
\begin{cases}
2^{-\alpha j}(g(J))^{-(\gamma-\alpha+1/2)}\,,&\enskip
j_1\leq j\leq j_2-1\,;\\
 2^{-\alpha j}J^{-1/2}\,, &\enskip
 j_2\leq j \leq J-1\,.
 \end{cases}
 \end{equation*}
При этом в~силу~(\ref{Wavelet_CoeffDecacy})

\noindent
\begin{equation} %\label{Indices_Soft_j1}
\left.
\begin{array}{rl}
\hspace*{-2mm}j_1&=\fr{J}{2\gamma-2\alpha+1}+\log_2 g(J)+ \fr{\log_2 A}{\gamma-\alpha+1/2}\,;
\\[6pt]
\hspace*{-2mm}j_2&=\fr{J}{2\gamma-2\alpha+1}+\fr{\log_2 J}{2\gamma-2\alpha+1}+{}\\[6pt]
&\hspace*{41mm}{}+ \fr{\log_2 A}{\gamma-\alpha+1/2}\,.
\end{array}
\right\}
\label{Indices_Soft_j2}
\end{equation}

Разобьем сумму в~чис\-ли\-те\-ле~(\ref{Risk_Definition}) на три со\-став\-ля\-ющие:

\noindent
\begin{multline}
\sum\limits_{j=0}^{J-1}\sumk{\sf P}
\left(\left|\betajk\hat{Y}_{j,k}-\betajk\mujk\right|>\varepsilon\right)={}\\
{}=
\sum\limits_{j=0}^{j_1-1}\sumk{\sf P}
\left(\left|\betajk\hat{Y}_{j,k}-\betajk\mujk\right|>\varepsilon\right)+{}\\
{}+\sum\limits_{j=j_1}^{j_2-1}\sumk{\sf P}
\left(\left|\betajk\hat{Y}_{j,k}-\betajk\mujk\right|>\varepsilon\right)+{}\\
{}\sum\limits_{j=j_2}^{J-1}\sumk{\sf P}
\left(\left|\betajk\hat{Y}_{j,k}-\betajk\mujk\right|>\varepsilon\right)\equiv{}\\ 
{}\equiv S_1+S_2+S_3\,.
\label{Risk_Decomposition}
\end{multline}

Рассмотрим $S_3$. Заметим, что для любого $\varepsilon\hm>0$ найдется такое 
$J_0\hm=J_0(\varepsilon)$, что 
$2^{-\alpha j}J^{-1/2}\hm\leq\varepsilon/\betajk$ и~$\varepsilon/\betajk\hm\leq cT$, 
$0\hm<c\hm<1$, для всех $J\hm>J_0$, и~найдется такое $J_1\hm=J_1(\varepsilon,c)
\hm\geq J_0$, что для всех $J\hm>J_1$ имеют место соотношения:

\noindent
\begin{alignat*}{2}
\mujk+\fr{\varepsilon}{\betajk}&\geq0\,;&\quad \mujk-\fr{\varepsilon}{\betajk}&\leq0\,;\\
\mujk+\fr{\varepsilon}{\betajk}&\leq T\,;&\quad \mujk-\fr{\varepsilon}{\betajk}&\geq -T
\end{alignat*}
для $j_2\leq j\leq J-1$. Следовательно, для одного сла\-га\-емо\-го из~$S_3$ имеем при 
$J\hm>J_1$:

\pagebreak

\noindent
\begin{multline}
{\sf P}\left(\left|\betajk\hat{Y}_{j,k}-\betajk\mujk\right|>\varepsilon\right)={}\\
{}=
{\sf P}\left(\left|\rho_{T}^{(h)}(Y_{j,k})-\mujk\right|>
\fr{\varepsilon}{\betajk}\right)={}\\
{}={\sf P}\left(\mathbf{1}\left(|Y_{j,k}|>T\right)Y_{j,k}>\mujk+
\fr{\varepsilon}{\betajk}\right)+{}\\
{}+
{\sf P}\left(\mathbf{1}\left(|Y_{j,k}|>T\right)Y_{j,k}<
\mujk-\fr{\varepsilon}{\betajk}\right)={}\\
{}={\sf P}\left(Y_{j,k}>T,Y_{j,k}>\mujk+\fr{\varepsilon}{\betajk}\right)+{}\\
{}+
{\sf P}\left(Y_{j,k}<-T,Y_{j,k}<\mujk-\fr{\varepsilon}{\betajk}\right)={}\\
{}=
{\sf P}\left(\left|Y_{j,k}\right|>T\right)\asymp{}\\
{}\asymp \fr{\exp\{-{(T+\mujk)^2}/({2\sigma^2})\}}{T}+{}\\
{}+
\fr{\exp\{-{(T-\mujk)^2}/({2\sigma^2})\}}{T}\,,
\label{S3_Term}
\end{multline}
поскольку $Y_{j,k}$ имеют нормальное распределение со средним~$\mujk$ 
и~дисперсией~$\sigma^2$, а~для стандартной нормальной функ\-ции распределения 
при до\-ста\-точ\-но больших~$x$ справедливо соотношение:
$$
1-\Phi(x)\asymp\fr{\Phi'(x)}{x}\,.
$$

Заметим, что при $j_2\leq j\hm\leq J-1$ величины~$|\mujk|$ не влияют на 
порядок правой час\-ти~(\ref{S3_Term}). Учитывая, что чис\-ло слагаемых в~$S_3$ 
имеет порядок~$2^J$, получаем:
\begin{multline}
S_3=\sum\limits_{j=j_2}^{J-1}\sumk{\sf P}\left(\left|
\betajk\hat{Y}_{j,k}-\betajk\mujk\right|>\varepsilon\right)\asymp{}\\
{}\asymp
\fr{2^J\exp\{-T^2/(2\sigma^2)\}}{T}\,.
\label{S3_Order}
\end{multline}

Найдем верхнюю оценку для функции потерь~(\ref{Risk_Definition}) при жест\-кой 
пороговой обработке. Для этого предположим, что все сла\-га\-емые в~суммах~$S_1$ и~$S_2$ 
из~(\ref{Risk_Decomposition}) отделены от нуля некоторой константой. Поэтому 
из~(\ref{Indices_Soft_j2}) получаем, что
\begin{multline}
S_1+S_2=\sum\limits_{j=0}^{j_2-1}\sumk{\sf P}\left(\left|
\betajk\hat{Y}_{j,k}-\betajk\mujk\right|>\varepsilon\right)\asymp{}\\
{}\asymp 2^{j_2}\asymp 2^{{J}/({2\gamma-2\alpha+1})}J^{1/(2\gamma-2\alpha+1)}\,.
\label{S1S2_Order}
\end{multline}

Порог $T_*^{(h)}-T_2^{(h)}$
обеспечивает равенство порядков правых час\-тей~(\ref{S3_Order}) и~(\ref{S1S2_Order}) 
и,~таким образом, является нижней границей (с~точ\-ностью до величины 
порядка~$O(1/\sqrt{\ln 2^J})$) для асимптотически оптимального в~смыс\-ле
 функции потерь~$R_J$ порога.

Теперь найдем нижнюю границу для функции потерь~(\ref{Risk_Definition}). 
Заметим, что найдется такая функция $Kf\hm\in\mathrm{Lip}(\gamma,L)$, 
что в~неравенстве~(\ref{Wavelet_CoeffDecacy}) будет достигаться равенство 
для $0\hm\leq j\hm\leq j_1-1$~\cite{Mal99}. Следовательно, существует такое $J_2\hm>0$, 
что для всех $\varepsilon\hm>0$ и~$J\hm>J_2$ выполняется $|\mujk|
\hm>\varepsilon/\betajk$ при $0\hm\leq j\hm\leq j_1-1$. Тогда
\begin{multline*}
{\sf P}\left(\left|\betajk\hat{Y}_{j,k}-\betajk\mujk\right|>\varepsilon\right)\geq{}\\
{}+\geq
{\sf P}\left(\left|Y_{j,k}-\mujk\right|>\fr{\varepsilon}{\betajk}\right)\geq
2-2\Phi\left(\fr{\varepsilon}{\sigma\beta_{0,0}}\right)\,.
\end{multline*}

В этом случае порядок суммы~$S_1$ в~(\ref{Risk_Decomposition}) равен чис\-лу 
сла\-га\-емых, т.\,е.
\begin{multline}
\hspace*{-3.23184pt}S_1=\sum\limits_{j=0}^{j_1-1}\sumk{\sf P}
\left(\left|\betajk\hat{Y}_{j,k}-\betajk\mujk\right|>\varepsilon\right)\asymp
 2^{j_1}\asymp{}\\
 {}\asymp 2^{{J}/({2\gamma-2\alpha+1})}g(J)\,.
\label{S1_Order}
\end{multline}
Приравняем порядки $S_1$ и~$S_3$ из~(\ref{S1_Order}) и~(\ref{S3_Order}). 
В~этом случае порог равен $T_*^{(h)}\hm-T_1^{(h)}.$

Заметим, что сумма~$S_2$ в~данных рассуждениях не присутствует. Это означает, 
что истинное значение~$R_J$ имеет порядок не ниже данного, т.\,е.\
 рас\-смат\-ри\-ва\-емый порядок является нижней оценкой для истинного порядка 
 функции потерь, а~$T_*^{(h)}\hm-T_1^{(h)}$~--- 
 верхней границей для асимптотически оптимального порога~$T$, 
 поскольку, чтобы не удалить важ\-ные компоненты функции сигнала, следует 
 выбирать наименьший порог, не ухудшающий порядок функции потерь.

Теорема доказана.

\smallskip

\noindent
\textbf{Замечание~1.}\ Пороги $T_*^{(h)}\hm-T_2^{(h)}$ и~$T_*^{(h)}\hm-T_1^{(h)}$ 
имеют одинаковую воз\-рас\-та\-ющую по~$J$ компоненту~$T_*^{(h)}$, причем 
$|T_1^{(h)}\hm-T_2^{(h)}|$ стремится к~нулю. Это означает, что истинное значение 
порога, ми\-ни\-ми\-зи\-ру\-юще\-го порядок функции потерь при жест\-кой пороговой обработке, 
так\-же будет иметь главную часть~$T_*^{(h)}$.


\section{Мягкая пороговая обработка}

Пусть $\hat{Y}_{j,k}=\rho_T^{(s)}(Y_{j,k})$.
Пусть функция $g_1(J)\hm>0$ сколь угодно медленно убывает по~$J$ к~нулю,
 а~$g_2(J)\hm>0$ неограниченно возрастает по~$J$, причем
\begin{equation}
\ln  g_2(J)=o\left(\sqrt{\ln 2^J}\right)\,, \enskip  J\to\infty\,.
\label{g_2_Grow}
\end{equation}

\noindent
\textbf{Теорема~2.}\
\textit{Пусть $\gamma\hm>\alpha\hm-1/2$. При выборе асимптотически оптимального 
порога для мяг\-кой пороговой обработки функция потерь}~(\ref{Risk_Definition}), 
\textit{начиная с~некоторого~$J$, удовле\-тво\-ря\-ет неравенствам}:
\begin{multline*}
C_1^{(s)}\cdot2^{-({2\gamma-2\alpha})J/({2\gamma-2\alpha+1})}g_1(J)\leq 
R_J\leq{}\\
{}\leq C_2^{(s)}\cdot2^{-({2\gamma-2\alpha})J/({2\gamma-2\alpha+1})}g_2(J)\,,
\end{multline*}
\textit{где $C_1^{(s)}$ и~$C_2^{(s)}$~--- некоторые положительные константы.
Для асимптотически оптимального значения порога, ми\-ни\-ми\-зи\-ру\-юще\-го 
порядок функции потерь}~(\ref{Risk_Definition}) \textit{при мягкой пороговой 
обработке, справедливо неравенство}:
$$
T_*^{(s)}-T_2^{(s)}\leq T\leq  T_*^{(s)}-T_1^{(s)}\,, 
$$
\textit{где}
\begin{align*}
T_*^{(s)}&=\sigma\sqrt{\fr{4\gamma-4\alpha}{2\gamma-2\alpha+1}\ln2^J}\,;
\\
T_i^{(s)}&=\sigma\sqrt{\fr{2\gamma-2\alpha+1}{4\gamma-4\alpha}}\,
\fr{\ln\left((\ln 2^J)^{1/2}g_i(J)\right)}{\sqrt{\ln 2^J}}\,, \\
&\hspace*{60mm}  i=1,2\,.
\end{align*}


\noindent
{Д\,о\,к\,а\,з\,а\,т\,е\,л\,ь\,с\,т\,в\,о\,.}\ \
 Разобьем, как в~разд.~5, все множество индексов $\{0,\ldots,J-1\}$ 
 на три класса в~за\-ви\-си\-мости от величины~$|\mujk|$. Пусть индексы~$j_1$ 
 и~$j_2$ ($j_1\hm< j_2$) таковы, что
\begin{equation*}
|\mujk|\leq
\begin{cases}
 2^{-\alpha j}(g_1(J))^{-(\gamma-\alpha+1/2)}\,, &\ j_1\leq j\leq j_2-1\,;\\
2^{-\alpha j}(g_2(J))^{-(\gamma-\alpha+1/2)}\,, &\ j_2\leq j \leq J-1\,.
\end{cases}
\end{equation*}
При этом в~силу~(\ref{Wavelet_CoeffDecacy}) для $i\hm=1,2$

\vspace*{2pt}

\noindent
$$
j_i=\fr{J}{2\gamma-2\alpha+1}+\log_2 g_i(J)+ \fr{\log_2 A}{\gamma-\alpha+1/2}\,.
$$
Разобьем сумму в~чис\-ли\-те\-ле~(\ref{Risk_Definition}) на три суммы~(\ref{Risk_Decomposition}).

Рассмотрим $S_3$. Зафиксируем некоторое положительное чис\-ло~$\varepsilon$.
 Заметим, что для любого $\varepsilon\hm>0$ найдется такое $J_0\hm=J_0(\varepsilon)$, 
 что $ 2^{-\alpha j}(g_2(J))^{-(\gamma-\alpha+1/2)}\hm\leq\varepsilon/\betajk$ 
 для всех $J\hm>J_0$. Следовательно, имеет мес\-то соотношение
$|\mujk|\hm\leq\varepsilon/\betajk$ для $j_2\h\leq j\hm\leq J-1$. 
Для одного сла\-га\-емо\-го из~$S_3$ имеем при $J\hm>J_0$:
\begin{multline*}
{\sf P}\left(\left|\betajk\hat{Y}_{j,k}-\betajk\mujk\right|>\varepsilon\right)={}\\
{}=
{\sf P}\left(\left|\rho_{T}^{(s)}(Y_{j,k})-\mujk\right|>\fr{\varepsilon}{\betajk}\right)={}\\
{}={\sf P}\left(
\vphantom{\fr{\varepsilon}{\betajk}}
|\mathbf{1}\left(Y_{j,k}>T\right)(Y_{j,k}-T)+{}\right.\\
\left.{}+
\mathbf{1}(Y_{j,k}<-T)(Y_{j,k}+T)-\mujk|>\fr{\varepsilon}{\betajk}\right)={}\\
{}={\sf P}\left(Y_{j,k}>T,Y_{j,k}>T+\mujk+\fr{\varepsilon}{\betajk}\right)+{}
\end{multline*}

\noindent
\begin{multline}
{}+
{\sf P}\left(Y_{j,k}<-T,Y_{j,k}<-T+\mujk-\fr{\varepsilon}{\betajk}\right)={}\\
{}={\sf P}\left(|Y_{j,k}-\mujk|>T+\fr{\varepsilon}{\betajk}\right).
\label{S3_Term_Soft}
\end{multline}

Поскольку $\betajk$ воз\-рас\-та\-ет как~$2^{cJ}$ и~$Y_{j,k}$ 
имеют нормальное распределение со сред\-ним~$\mujk$ и~дис\-пер\-си\-ей~$\sigma^2$, 
из~(\ref{S3_Term_Soft}) имеем:
\begin{equation}
S_3\asymp \fr{2^J\exp\{-T^2/(2\sigma^2)\}}{T}\,.
\label{S3_Order_Soft}
\end{equation}

Найдем верх\-нюю оценку для функции потерь~(\ref{Risk_Definition}) при мягкой 
пороговой обработке. Для этого предположим, что все сла\-га\-емые в~суммах~$S_1$ и~$S_2$ 
из~(\ref{Risk_Decomposition}) отделены от нуля некоторой константой. 
В~этом случае получаем:
\begin{equation}
S_1+S_2\asymp2^{{J}/({2\gamma-2\alpha+1})}g_2(J)\,.
\label{S1S2_Order_Soft}
\end{equation}

Порог $T_*^{(s)}-T_2^{(s)}$
обеспечивает равенство порядков правых час\-тей~(\ref{S3_Order_Soft}) 
и~(\ref{S1S2_Order_Soft}) и,~таким образом, является ниж\-ней границей 
(с~точ\-ностью до величины порядка~$O(1/\sqrt{\ln 2^J})$) для асимптотически 
оптимального в~смысле функции потерь~$R_J$ порога.

Теперь найдем нижнюю границу для функции потерь~(\ref{Risk_Definition}). 
Пользуясь рас\-суж\-де\-ни\-ями, приведенными в~разд.~5, вви\-ду убывания к~нулю~$g_1(J)$, 
для любого $\varepsilon\hm>0$ и~до\-ста\-точ\-но больших~$J$ найдется такая 
функция $Kf\hm\in\mathrm{Lip}\,(\gamma,L)$, что в~неравенстве~(\ref{Wavelet_CoeffDecacy}) 
будет достигаться равенство и~выполняться $|\mujk|\hm>\varepsilon/\betajk$ при 
$0\hm\leq j\hm\leq j_1\hm-1$. Тогда поскольку

\vspace*{-6pt}

\noindent
\begin{multline*}
{\sf P}\left(\left|\betajk\hat{Y}_{j,k}-\betajk\mujk\right|>\varepsilon\right)={}\\
{}=
{\sf P}\left(|\mujk|>\fr{\varepsilon}{\betajk},\left\vert Y_{j,k}\right\vert\leq T\right)+{}\\
{}+{\sf P}\left(\left\vert Y_{j,k}-T-\mujk\right\vert >\fr{\varepsilon}{\betajk},
Y_{j,k}>T\right)+{}\\
{}+
{\sf P}\left(\left\vert Y_{j,k}+T-\mujk\right\vert >\fr{\varepsilon}{\betajk},
Y_{j,k}<-T\right)\,,
\end{multline*}
учитывая, что~$Y_{j,k}$ имеют нормальное распределение с~максимумом плот\-ности, 
до\-сти\-га\-емым в~точке~$\mujk$, получаем:
$$
{\sf P}\left(\left|\betajk\hat{Y}_{j,k}-\betajk\mujk\right|>\varepsilon\right)\geq
2-2\Phi\left(\fr{\varepsilon}{\sigma\beta_{0,0}}\right).
$$
Отсюда следует оценка, аналогичная~(\ref{S1_Order}):
\begin{equation}
S_1\asymp2^{{J}/({2\gamma-2\alpha+1})}g_1(J)\,.
\label{S1_Order_Soft}
\end{equation}

Приравняем порядки~(\ref{S3_Order_Soft}) и~(\ref{S1_Order_Soft}). 
В~этом случае порог равен $T_*^{(s)}\hm-T_1^{(s)}$.
Воспользовавшись рас\-суж\-де\-ни\-ями, приведенными в~разд.~5, 
получаем, что порог $T_*^{(s)}\hm-T_1^{(s)}$ является верх\-ней оценкой для 
истинного значения асимптотически оптимального порога~$T$.

Теорема доказана.

\smallskip

\noindent
\textbf{Замечание~2.}\ Пороги $T_*^{(s)}\hm-T_1^{(s)}$ и~$T_*^{(s)}\hm-T_2^{(s)}$ 
имеют одинаковую воз\-рас\-та\-ющую по~$J$ компоненту~$T_*^{(s)}$, причем, поскольку 
выполнено~(\ref{g_2_Grow}), $|T_1^{(s)}\hm-T_2^{(s)}|$ стремится к~нулю. 
Это означает, что истинное значение порога, минимизирующего порядок функции 
потерь при мягкой пороговой обработке, также будет иметь глав\-ную часть~$T_*^{(s)}$.

\smallskip

\noindent
\textbf{Замечание~3.}\ В~случае мягкой пороговой обработки функции~$g_i(J)$, $i\hm=1,2$, 
фигурирующие в~верх\-ней и~ниж\-ней оценках, имеют сколь угодно малую ско\-рость 
сходимости по~$J$ (в~отличие от логарифмической функции в~одной из оценок при
 жест\-кой пороговой обработке). Это позволяет сделать вывод о~том, что верх\-няя 
 оценка функции потерь и~ниж\-няя оценка асимптотически оптимального порога 
 при мягкой пороговой обработке точ\-нее, чем при жест\-кой.

\smallskip

\noindent
\textbf{Замечание~4.}\ Если дополнительно предположить, что вейв\-лет-функ\-ция~$\psi$ 
имеет компактный носитель, то требование равномерной ре\-гу\-ляр\-ности по Липшицу 
можно заменить на требование кусочной ре\-гу\-ляр\-ности (подробнее см.~\cite{Jan01}).


{\small\frenchspacing
 {%\baselineskip=10.8pt
 \addcontentsline{toc}{section}{References}
 \begin{thebibliography}{99}




\bibitem{Don95} %1
\Au{Donoho~D.} Nonlinear solution of linear inverse problems by wavelet-vaguelette 
decomposition~// Appl. Comput. Harmon.~A., 1995. Vol.~2. P.~101--126.

\bibitem{Abr98}  %2
\Au{Abramovich F., Silverman~B.\,W.} Wavelet decomposition approaches 
to statistical inverse problems~// Biometrika, 1998. Vol.~85. No.\,1. P.~115--129.

\bibitem{Lee97} 
\Au{Lee N.} Wavelet-vaguelette decompositions and homogenous equations.~---
 West Lafayette, IN, USA: Purdue University, 1997.   PhD Thesis. 103~p.


\bibitem{Mal99} 
\Au{Mallat S.} A~wavelet tour of signal processing.~--- New York, NY, USA:
Academic Press, 1999. 857~p.


\bibitem{SMS14}  %5
\Au{Sadasivan J., Mukherjee~S., Seelamantula~C.\,S.} 
An optimum shrinkage estimator based on minimum-probability-of-error criterion 
and application to signal denoising~// 
    %Proceedings of 39th International Conference on Acoustics, Speech and Signal Processing (ICASSP). 2014. Italy, Florence.
    39th IEEE  Conference (International) on Acoustics, 
    Speech and Signal Processing Proceedings.~--- Piscataway, NJ, USA: IEEE, 2014. 
    P.~4249--4253.

\bibitem{DJ98} 
\Au{Donoho D., Johnstone~I.\,M.} Minimax estimation via wavelet shrinkage~// 
Ann. Stat., 1998. Vol.~26. No.\,3. P.~879--921.



\bibitem{Jan01} 
\Au{Jansen M.} Noise reduction by wavelet thresholding.~--- 
Lecture notes in statistics ser.~---  New York, NY, USA: Springer Verlag, 2001. Vol.~161. 217~p.

\bibitem{DJ95} 
\Au{Donoho D., Johnstone~I.} Adapting to unknown smoothness via wavelet shrinkage~// 
J.~Am. Stat. Assoc., 1995. Vol.~90. P.~1200--1224.

\bibitem{MSH10} 
\Au{Маркин А.\,В., Шестаков~О.\,В.} О~состоятельности оценки риска при 
пороговой обработке вейв\-лет-ко\-эф\-фи\-ци\-ен\-тов~// Вестн. Моск. ун-та. Сер.~15: 
Вычисл. матем. и~киберн., 2010. №\,1. C.~26--34.

\bibitem{SH12} 
\Au{Шестаков О.\,В.} Асимптотическая нор\-маль\-ность оцен\-ки риска пороговой 
обработки вейв\-лет-ко\-эф\-фи\-ци\-ен\-тов при выборе адаптивного порога~// 
Докл. РАН, 2012. Т.~445. №\,5. С.~513--515.

\bibitem{KS16-1} 
\Au{Кудрявцев А.\,А., Шестаков~О.\,В.} Асимптотическое поведение порога, 
минимизирующего усред\-нен\-ную ве\-ро\-ят\-ность ошиб\-ки вы\-чис\-ле\-ния 
вейв\-лет-ко\-эф\-фи\-ци\-ен\-тов~// 
Докл. РАН, 2016. Т.~468. №\,5. С.~487--491.

\bibitem{KS16-2} 
\Au{Кудрявцев~А.\,А., Шестаков~О.\,В.} Асимптотически оптимальная пороговая 
обработка вейв\-лет-ко\-эф\-фи\-ци\-ен\-тов в~моделях с~негауссовым 
рас\-пре\-де\-ле\-ни\-ем шума~// Докл. РАН, 2016. Т.~471. №\,1. С.~11--15.

 \end{thebibliography}

 }
 }

\end{multicols}

\vspace*{-6pt}

\hfill{\small\textit{Поступила в~редакцию 25.02.18}}

\vspace*{8pt}

%\newpage

%\vspace*{-24pt}

\hrule

\vspace*{2pt}

\hrule

%\vspace*{8pt}


\def\tit{MINIMIZATION OF~ERRORS OF~CALCULATING WAVELET COEFFICIENTS 
WHILE~SOLVING INVERSE PROBLEMS}

\def\titkol{Minimization of~errors of~calculating wavelet coefficients 
while~solving inverse problems}

\def\aut{A.\,A.~Kudryavtsev$^1$ and O.\,V.~Shestakov$^{1,2}$}

\def\autkol{A.\,A.~Kudryavtsev and O.\,V.~Shestakov}

\titel{\tit}{\aut}{\autkol}{\titkol}

\vspace*{-9pt}


\noindent
$^1$Department of Mathematical Statistics, Faculty of Computational 
Mathematics and Cybernetics,\linebreak
$\hphantom{^1}$M.\,V.~Lomonosov Moscow State University, 
1-52~Leninskiye Gory, GSP-1, Moscow 119991, Russian Fed-\linebreak
$\hphantom{^1}$eration

\noindent
$^2$Institute of Informatics Problems, Federal Research Center ``Computer 
Science and Control'' of the Russian\linebreak
$\hphantom{^1}$Academy of Sciences, 44-2~Vavilov Str., 
Moscow 119333, Russian Federation


\def\leftfootline{\small{\textbf{\thepage}
\hfill INFORMATIKA I EE PRIMENENIYA~--- INFORMATICS AND
APPLICATIONS\ \ \ 2018\ \ \ volume~12\ \ \ issue\ 2}
}%
 \def\rightfootline{\small{INFORMATIKA I EE PRIMENENIYA~---
INFORMATICS AND APPLICATIONS\ \ \ 2018\ \ \ volume~12\ \ \ issue\ 2
\hfill \textbf{\thepage}}}

\vspace*{3pt}


\Abste{Statistical inverse problems arise in many applied fields, including 
medicine, astronomy, biology, plasma physics, chemistry, etc. At the same time, 
there are always errors in the observed data due to imperfect equipment, 
background noise, data discretization, and other reasons. To reduce these errors, 
it is necessary to apply special regularization methods that allow constructing 
approximate stable solutions of inverse problems. The classical\linebreak\vspace*{-12pt}}

\Abstend{regularization 
methods are based on the use of windowed singular value decomposition. However, 
this approach takes into account only the type of operator involved in the 
formation of observable data and does not take into account the properties 
of the object of observation. For linear homogeneous operators, this problem 
is solved with the help of special methods of wavelet analysis, which allow 
adapting simultaneously to the form of the operator and local features of 
the function describing the object. In this paper, the authors consider the 
problem of inverting a linear homogeneous operator in the presence of noise 
in the observational data by thresholding the wavelet expansion coefficients 
of the observed function. The asymptotically optimal thresholds and orders 
of the loss function are calculated when minimizing the averaged probability 
of error of wavelet coefficient calculation.}

\KWE{wavelets; thresholding; linear homogeneous operator; loss function}



\DOI{10.14357/19922264180203}

\vspace*{-14pt}

\Ack
\noindent
The work was partly supported by the Russian Foundation for Basic Research 
(project 18-07-00252).



%\vspace*{-3pt}

  \begin{multicols}{2}

\renewcommand{\bibname}{\protect\rmfamily References}
%\renewcommand{\bibname}{\large\protect\rm References}

{\small\frenchspacing
 {%\baselineskip=10.8pt
 \addcontentsline{toc}{section}{References}
 \begin{thebibliography}{99}



\bibitem{2-ks}
\Aue{Donoho, D.} 1995. Nonlinear solution of linear inverse problems by 
wavelet-vaguelette decomposition. \textit{Appl. Comput. Harmon.~A.} 2:101--126.

\bibitem{1-ks}
\Aue{Abramovich, F., and B.\,W.~Silverman.} 1998. 
Wavelet decomposition approaches to statistical inverse problems. 
\textit{Biometrika} 85(1):115--129.

\bibitem{3-ks}
\Aue{Lee, N.} 1997. Wavelet-vaguelette decompositions and homogenous equations.
 West Lafayette, IN: Purdue University. PhD Thesis. 103~p.

\bibitem{4-ks}
\Aue{Mallat, S.} 1999. \textit{A~wavelet tour of signal processing}. 
New York, NY: Academic Press. 857~p.

\bibitem{5-ks}
\Aue{Sadasivan, J., S.~Mukherjee, and C.\,S.~Seelamantula}. 2014. 
An optimum shrinkage estimator based on minimum-probability-of-error 
criterion and application to signal denoising. 
\textit{39th IEEE  Conference (International) on Acoustics, 
    Speech and Signal Processing Proceedings}. Piscataway, NJ: IEEE. 4249--4253.

\bibitem{6-ks}
\Aue{Donoho, D., and I.\,M.~Johnstone}. 1998. Minimax estimation via wavelet shrinkage. 
\textit{Ann. Stat.} 26(3):879--921.

\bibitem{7-ks}
\Aue{Jansen, M.} 2001. \textit{Noise reduction by wavelet thresholding}. 
Lecture notes in statistics ser. New York, NY: Springer Verlag. Vol.~161. 217~p.

\bibitem{8-ks}
\Aue{Donoho, D., and I.~Johnstone.} 1995. Adapting to unknown smoothness via wavelet 
shrinkage.  \textit{J.~Am. Stat. Assoc.} 90:1200--1224.

\bibitem{9-ks}
\Aue{Markin, A.\,V.,  and O.\,V.~Shestakov.} 2010. Consistency of risk 
estimation with thresholding of wavelet coefficients. 
\textit{Mosc. Univ. Comput. Math. Cybern.} 34(1):22--30.

\bibitem{10-ks}
\Aue{Shestakov, O.\,V.} 2012. Asymptotic normality of adaptive wavelet
 thresholding risk estimation. \textit{Dokl. Math.} 86(1):556--558.

\bibitem{11-ks}
\Aue{Kudryavtsev, A.\,A., and O.\,V.~Shestakov.} 2016. 
Asymptotic behavior of the threshold minimizing the average 
probability of error in calculation of wavelet coefficients. 
\textit{Dokl. Math.} 93(3):295--299.

\bibitem{12-ks}
\Aue{Kudryavtsev, A.\,A., and O.\,V.~Shestakov.} 2016. 
Asymptotically optimal wavelet thresholding in the models with 
non-Gaussian noise distributions. \textit{Dokl. Math.} 94(3):615--619.
\end{thebibliography}

 }
 }

\end{multicols}

\vspace*{-3pt}

\hfill{\small\textit{Received February 25, 2018}}

%\vspace*{-24pt}



\Contr

\noindent
\textbf{Kudryavtsev Alexey A.} (b.\ 1978)~--- 
Candidate of Science (PhD) in physics and mathematics, associate professor, 
Department of Mathematical Statistics, Faculty of Computational 
Mathematics and Cybernetics, M.\,V.~Lomonosov Moscow State University, 
1-52~Leninskiye Gory, GSP-1, Moscow 119991, Russian Federation; \mbox{nubigena@mail.ru}

\vspace*{3pt}

\noindent
\textbf{Shestakov Oleg V.} (b.\ 1976)~--- 
Doctor of Science in physics and mathematics, associate professor, 
Department of Mathematical Statistics, Faculty of Computational Mathematics 
and Cybernetics, M.\,V.~Lomonosov Moscow State University, 1-52~Leninskiye Gory, 
GSP-1, Moscow 119991, Russian Federation; senior scientist, 
Institute of Informatics Problems, Federal Research Center 
``Computer Science and Control'' of the Russian Academy of Sciences, 
44-2~Vavilov Str., Moscow 119333, Russian Federation; \mbox{oshestakov@cs.msu.su}

\label{end\stat}


\renewcommand{\bibname}{\protect\rm Литература}   %3
\def\stat{kondranin+ushakov}

\def\tit{СИСТЕМА ОБСЛУЖИВАНИЯ С~ОТНОСИТЕЛЬНЫМ ПРИОРИТЕТОМ  И~ПРОФИЛАКТИКАМИ ПРИБОРА$^*$}

\def\titkol{Система обслуживания с~относительным приоритетом  и~профилактиками прибора}

\def\aut{Е.\,С.~Кондранин$^1$,  В.\,Г.~Ушаков$^2$}

\def\autkol{Е.\,С.~Кондранин,  В.\,Г.~Ушаков}

\titel{\tit}{\aut}{\autkol}{\titkol}

\index{Кондранин Е.\,С.}
\index{Ушаков В.\,Г.}
\index{Kondranin E.\,S.}
\index{Ushakov V.\,G.}




{\renewcommand{\thefootnote}{\fnsymbol{footnote}} \footnotetext[1]
{Работа выполнена при финансовой поддержке РФФИ (проект 18-07-00678).}}


\renewcommand{\thefootnote}{\arabic{footnote}}
\footnotetext[1]{Факультет вычислительной математики и~кибернетики Московского государственного 
университета им.\ М.\,В.~Ломоносова, \mbox{ekondranin@yandex.ru}}
\footnotetext[2]{Факультет вычислительной математики и~кибернетики
Московского государственного университета им.\ М.\,В.~Ломоносова;
Институт проб\-лем информатики Федерального исследовательского
центра <<Информатика и~управ\-ле\-ние>> Российской академии наук,
\mbox{vgushakov@mail.ru}}

\vspace*{-10pt}




\Abst{Изучена одноканальная система
массового обслуживания с~двумя типами требований, бесконечным
числом мест для ожидания, гиперэкспоненциальным входящим потоком 
и~профилактиками обслуживающего прибора при освобождении системы.
Тип  требования определяется случайно с~заданными вероятностями 
в~момент его поступления в~систему обслуживания. Требования первого
типа имеют относительный приоритет перед требованиями второго
типа. Найдено нестационарное совместное распределение числа
требований каждого типа в~системе. Профилактики прибора
заключаются в~том, что в~момент освобождения системы от требований
прибор на случайное время с~заданным распределением становится
недоступным для обслуживания. Если за время профилактики поступает
хотя бы одно требование, то начинается нормальное функционирование
системы. Если требования не поступают, то прибор отправляется на
новую профилактику. Такие системы хорошо описывают
функционирование большого числа реальных вычислительных и~информационных систем.}

\KW{гиперэкспоненциальный поток; профилактики
обслуживающего прибора; одноканальная система; относительный
приоритет; длина очереди}

\DOI{10.14357/19922264180405}
  
%\vspace*{4pt}


\vskip 10pt plus 9pt minus 6pt

\thispagestyle{headings}

\begin{multicols}{2}

\label{st\stat}

\section{Введение}

В классической системе массового обслуживания ожидание требований
в очереди связано только с~занятостью обслуживающего прибора. В~то
же время в~реальных системах сам  прибор может пребывать как 
в~активном, так и~в~неактивном состоянии. Такое неактивное
состояние прибора (в~литературе на английском языке используется
термин vacation, а~на русском~--- профилактика или прогулка) может
быть связано со многими причинами. В~част\-ности, сис\-те\-мы
обслуживания с~профилактиками прибора хорошо описывают
функционирование  реальных вычислительных и~информационных систем,
в которых наряду с~основными требованиями имеются второстепенные.
Второстепенные требования всегда присутствуют в~сис\-те\-ме, а~их
обслуживание может проводиться только тогда, когда нет основных,
т.\,е.\ в~фоновом режиме.

С точки зрения самого процесса профилактики прибора существует
несколько ее разновидностей. Во-пер\-вых, могут быть разными
правила, задающие условия начала профилактики: прибор может брать
перерыв только при  полном исчерпании требований в~очереди
(exhaustive service) либо при наличии определенного их числа
(nonexhaustive service). Во-вто\-рых, могут быть разными правила
возвращения прибора в~работу. С~этой точки зрения различают случаи
однократного (single vacation) и~многократного (multiple vacation)
перерыва в~работе. В~первом случае ушедший на профилактику прибор
после ее окончания находится в~рабочем состоянии независимо от
наличия требований в~системе. Во втором случае прибор, не
обнаружив новых требований в~очереди, уходит на новую
профилактику.


В работах~[1--4] можно найти обзор известных результатов, большое
число постановок задач, описание различных приложений и~обширную
библиографию по анализу систем с~профилактиками обслуживающего
прибора.


В настоящей работе исследуется совместное распределение длин
очередей в~нестационарном режиме в~однолинейной системе 
с~ожиданием, гиперэкспоненциальным входящим потоком, двумя типами
требований и~относительным приоритетом. Аналогичная неприоритетная
система обслуживания исследована в~[5].

\vspace*{-6pt}

\section{Описание модели}

Рассматривается однолинейная система массового обслуживания 
с~двумя приоритетными классами требований. Входящий поток~---
гиперэкспоненциальный с~функцией распределения интервалов между
поступлениями требований вида:
\begin{multline*}
A(t)=\sum\limits_{i=1}^kc_i\left(1-e^{-a_it}\right),\enskip t>0,\enskip
a_i>0,\enskip c_i>0,\\
a_i\ne a_j\,,\enskip i\ne j\,,\enskip  \sum\limits_{i=1}^k c_i=1\,.
\end{multline*}

Каждое поступившее требование направляется в~первый класс 
с~вероятностью~$p$ и~во второй класс с~вероятностью $1\hm-p$
независимо от остальных требований. Требования первого класса
обладают относительным приоритетом перед требованиями второго
класса. Длительности обслуживания требований $i$-го приоритетного
класса~--- независимые в~совокупности и~не зависящие от входящего
потока случайные величины с~функцией распределения~$B_i(x)$,
$i\hm=1,2.$
 Если в~некоторый момент времени система освободилась от требований, 
 то обслуживающий прибор
 отправляется на профилактику, которая длится случайное время с~функцией 
 распределения~$C(x).$
 Не ограничивая общности, будем считать, что $B_i(x)\hm<1$
 и~$C(x)\hm<1$  для любого~$x$ 
 и~существуют плотности
 распределения~$b_i(x)$ и~$c(x).$
  Обозначим:
$$
 \beta_i(s)=\int\limits_0^{\infty}e^{-sx}b_i(x)\,dx\,;\enskip 
  \gamma(s)=\int\limits_0^{\infty}e^{-sx}c(x)\,dx\,.
$$
Пока прибор находится на профилактике, он не доступен для
обслуживания. Если за время профилактики поступают требования,
после ее завершения начинается их обслуживание. Если ни одно
требование не поступает, то прибор отправляется на новую
профилактику. Длительности различных профилактик являются
независимыми случайными величинами 
и~не зависят от входящего потока и~времен обслуживания.

\section{Вспомогательные результаты}

  Рассмотрим многочлен по $\mu$ степени $k$ вида:
\begin{multline}
\label{1}
\prod\limits_{i=1}^k\left(\mu+a_i\right)-{}\\
{}-
\left(pz_1+(1-p)z_2\right)\sum\limits_{j=1}^kc_ja_j\prod\limits_{i\ne
j}\left(\mu+a_i\right)\,.
\end{multline}
Занумеруем его корни $\mu_1(z_1,z_2),\ldots,\mu_k(z_1,z_2)$ таким образом,
чтобы они были непрерывными функциями и~$\mu_1(1,1)\hm=0.$ Тогда
$\mathrm{Re}\, \mu_j\left(z_1,z_2\right)\hm<0$, $|z_1|\hm<1$, 
$|z_2|\hm<1,$ $\mu_i(z_1,z_2)\hm\ne \mu_j(z_1,z_2),$ $ i\hm\ne j$,
$j\hm=1,\ldots,k.$ Обозначим:
$$
\alpha_m(z_1,z_2)=\prod\limits_{j\ne m}\left(\mu_m\left(z_1,z_2\right)-
\mu_j\left(z_1,z_2\right)\right)\,.
$$
Справедливы следующие леммы.

\smallskip

\noindent
\textbf{Лемма~1.}\
\textit{Для любого $l=1,\ldots,\:k$ система уравнений}
$$
z_j=\beta_j(s-\mu_l(z_1,z_2)),\ \ j=1,2,
$$
\textit{имеет единственное решение $z_i=z_{il}(s)$ такое, 
что $|z_{il}(s)|\hm<1$ при $l\hm=2,\ldots, k,$ $\mathrm{Re}\, s\hm\geqslant 0,$ 
а~$z_{i1}(0)\hm=1$, $|z_{i1}(s)|\hm<1$ при} $\mathrm{Re}\, s\hm> 0$, $i\hm=1,2.$

\smallskip

\noindent
\textbf{Лемма~2.}\
\textit{При каждом $l\hm=1,\ldots,k$ уравнение}
$$
z_1=\beta_1\left(s-\mu_l(z_1,z_2)\right)
$$
\textit{имеет единственное решение $z_1\hm=z_{1l}(z_2,s),$ 
аналитическое в~области $\mathrm{Re}\, s\hm>0$, $|z_2|\hm<1.$
}

\smallskip

Положим
$$
\lambda_l(s)=\mu_l\left(z_{1l}(s),z_{2l}(s)\right)\,.
$$




\section{Распределение длины очереди}

  Гиперэкспоненциальный поток можно рас\-смат\-ри\-вать как
пуассоновский поток со случайной интен\-сив\-ностью~$a,$ которая
принимает $k$ различных значений $a_1,\ldots,a_k$  с~вероятностями
$c_1,\ldots,c_k.$ Текущее значение~$a$ разыгрывается в~момент
поступления требования и~не меняется между двумя соседними
поступлениями. Введем случайный процесс~$j(t)$ такой, что если
$a\hm=a_j$ в~момент времени $t,$ то $j(t)\hm=j.$

Целью работы является нахождение распределения случайного процесса
$\left(L_1(t),L_2(t)\right),$ где $L_i(t)$~--- число требований из
$i$-го приоритетного класса, находящихся в~системе в~момент
времени~$t.$

При сделанных предположениях относительно параметров изучаемой
системы обслуживания\linebreak процесс $\left(L_1(t),L_2(t)\right)$ не
является, вообще говоря, марковским. Пусть $i(t)=i$, $i\hm=1,2,$ если
в~момент времени~$t$ обслуживается требование из $i$-го
приоритетного класса, и~$i(t)\hm=0,$ если в~момент времени~$t$ прибор
находится на профилактике. Случайный процесс~$x(t)$ определим
следующим образом. Если $i(t)\hm\ne 0,$ то $x(t)$ есть
время, прошедшее с~начала обслуживания требования, находящегося на
приборе, до момента~$t.$ Если $i(t)\hm=0,$ то $x(t)$ есть время,
прошедшее с~начала профилактики прибора до момента~$t.$ Случайный
процесс $\left(L_1(t),L_2(t),i(t),j(t),x(t)\right)$ является
однородным марковским процессом. Положим
\begin{multline*}
P_{ij}(n_1,n_2,x,t)=\fr{\partial}{\partial x}
\mathbf{P}\left(L_1(t)=n_1,L_2(t)=n_2,\right.\\
\left. i(t)=i,j(t)=j,x(t)<x
\vphantom{L_1}\right)\,,\enskip 
 x\geqslant 0,\\ 
 j=1,\ldots,k,\enskip i=0,1,2;
\end{multline*}
\begin{gather*}
\eta_i(x)=\fr{b_i(x)}{1-B_i(x)},\ i=1,2;\enskip 
\eta_0(x)=\fr{c(x)}{1-C(x)}\,;\\
\delta_{i,j}=\begin{cases}
1,&\ i=j;\\ 
0,&\ i\ne j\,.
\end{cases}
\end{gather*}
Функции $P_{ij}(n_1,n_2,x,t)$  удовлетворяют при $x\hm>0$
системам дифференциальных уравнений:
\begin{multline}
\label{3}
\fr{\partial P_{ij}(n_1,n_2,x,t)}{\partial t}+\fr{\partial
P_{ij}(n_1,n_2,x,t)}{\partial
x}={}\\
{}=-(a_j+\eta_i(x))P_{ij}(n_1,n_2,x,t)+ {}\\
{}+
c_j\sum\limits_{l=1}^ka_l\left(p\:P_{il}(n_1-1,n_2,x,t)+{}\right.\\
\left.{}+
(1-p)P_{il}(n_1,n_2-1,x,t)\right)
\end{multline}
и краевым условиям при $x\hm=0$:
\begin{multline}
\label{5}
P_{0j}(n_1,n_2,0,t)=0,\ n_1+n_2>0;\\
P_{0j}(0,0,0,t)=\int\limits_0^{\infty}P_{0j}(0,0,x,t)\eta_0(x)\,dx+{}\\
 {}+\int\limits_0^{\infty}P_{1j}(1,0,x,t)\eta_1(x)dx+{}\\
 {}+
\int\limits_0^{\infty}P_{2j}(0,1,x,t)\eta_2(x)\,dx\,;
\end{multline}

\vspace*{-12pt}

\noindent
\begin{multline}
\label{6}
P_{1j}(n_1,n_2,0,t)+P_{2j}(n_1,n_2,0,t)={}\\
{}=\int\limits_0^{\infty}P_{1j}(n_1+1,n_2,x,t)\eta_1(x)\,dx+{}\\
{}+
\int\limits_0^{\infty}P_{2j}(n_1,n_2+1,x,t)\eta_2(x)\,dx+{}\\
{}+\int\limits_0^{\infty}P_{0j}(n_1,n_2,0,t)\eta_0(x)\,dx\,.
\end{multline}

Будем предполагать, что в~начальный момент времени $t\hm=0$ система
свободна от требований, а~с~начала профилактики прибора прошло
случайное время с~заданным распределением с~плотностью $d(x).$
Таким образом,
\begin{align*}
P_{ij}\left(n_1,n_2,x,0\right)&=0,\ i=1,2;
\\
P_{0j}\left(n_1,n_2,x,0\right)&=c_jd(x)\delta_{n_1+n_2,0},\ \
j=1,\ldots,k\,.
\end{align*}
Положим
\begin{multline*}
p_{ij}\left(z_1,z_2,x,s\right)={}\\
{}=\sum\limits_{n_1=0}^{\infty}
\sum\limits_{n_2=0}^{\infty}z_1^{n_1}z_2^{n_2}\!
\int\limits_0^{\infty}e^{-st}P_{ij}(n_1,n_2,x,t)\,dt\,;
\end{multline*}
$$
  \psi(s)=\int\limits_0^{\infty}e^{-sx}\,dx
  \int\limits_0^{\infty}\fr{c(u+x)d(u)}{1-C(u)}\,du\,.
$$
Тогда, учитывая начальные условия,  из \eqref{3}
получаем:
\begin{multline}
\label{7} 
\fr{\partial p_{ij}(z_1,z_2,x,s)}{\partial x}={}\\
{}=-\left(s+a_j+\eta_i(x)\right)p_{ij}
\left(z_1,z_2,x,s\right)+{}\\
{}+c_j\left(pz_1+(1-p)z_2\right)
\sum\limits_{l=1}^ka_lp_{il}\left(z_1,z_2,x,s\right),\\ 
i=1,2;
\end{multline}

\vspace*{-12pt}

\noindent
\begin{multline}
\label{8} 
\fr{\partial p_{0j}(z_1,z_2,x,s)}{\partial x}={}\\
{}=-\left(s+a_j+\eta_0(x)\right)p_{0j}\left(z_1,z_2,x,s\right)+{}\\
{}+c_j\left(pz_1+(1-p)z_2\right)\sum\limits_{l=1}^ka_lp_{0l}\left(z_1,z_2,x,s\right)+{}\\
{}+ c_jd(x).
\end{multline}
Решения \eqref{7} и~\eqref{8} имеют вид:
\begin{multline}
\label{9}
p_{ij}\left(z_1,z_2,x,s\right)=\left(1-B_i(x)\right)c_j\times{}\\
{}\times \sum\limits_{m=1}^k\fr{\gamma_i^{(m)}(z_1,z_2,s)}{\mu_m(z_1,z_2)+a_j}\,
e^{-(s-\mu_m(z_1,z_2))x}\,,\\
 i=1,2\,,
\end{multline}
\vspace*{-12pt}

\noindent
\begin{multline}
\label{10}
p_{0j}\left(z_1,z_2,x,s\right)={}\\
{}=\left(1-C(x)\right)
c_j\!\!\sum\limits_{m=1}^k\!\! e^{-(s-\mu_m(z_1,z_2))x}\!
\!\left(\!
\vphantom{\int\limits_{l=1}^k}
\delta^{(m)}\left(z_1,z_2,s\right)+{}\right.\\
%\left.
{}+\alpha_m^{-1}\left(z_1,z_2\right)
\prod\limits_{l=1}^k
\left(\mu_m\left(z_1,z_2\right)+a_l\right)\times{}\\
\left.{}\times \int\limits_0^x\!
e^{(s-\mu_m(z_1,z_2))u}
\fr{d(u)}{1-C(u)}\,du
\right)
\!\Bigg/ \!\left(\mu_m\left(z_1,z_2\right)+{}\right.\\
\left.{}+a_j\right)\,,
\end{multline}
где функции $\gamma_i^{(m)}(z_1,z_2,s)$  и~$\delta^{(m)}(z_1,z_2,s)$ являются
произвольными функциями указанных переменных и~определяются из
краевых условий. Из~\eqref{5} и~\eqref{6} получаем:
\begin{multline}
\label{11}
p_{1j}\left(z_1,z_2,0,s\right)+p_{2j}\left(z_1,z_2,0,s\right)={}\\
{}=z_1^{-1}\int\limits_0^{\infty}p_{1j}\left(z_1,z_2,x,s\right)\eta_1(x)\,dx+{}
\\
+z_2^{-1}\int\limits_0^{\infty}p_{2j}\left(z_1,z_2,x,s\right)\eta_2(x)\,dx+{}\\
{}+
\int\limits_0^{\infty}p_{0j}\left(z_1,z_2,x,s\right)\eta_0(x)\,dx
-p_{0j}\left(z_1,z_2,0,s\right)\,.
\end{multline}
Заметим, что $p_{0j}(z_1,z_2,0,s)$ не зависит от $z_1$ и~$z_2,$ т.\,е.\
$p_{0j}(z_1,z_2,0,s)\hm=q_j(s).$ 
Подставляя~\eqref{9} и~\eqref{10} в~\eqref{11}, получаем:
\begin{multline}
\label{12}
\gamma_1^{(m)}\left(z_1,z_2,s\right)\left(1-z_1^{-1}\beta_1(s-\mu_m(z_1,z_2))\right)+{}\\
{}+
\gamma_2^{(m)}(z_1,z_2,s)\left(1-z_2^{-1}\beta_2(s-\mu_m(z_1,z_2))\right)={}\\
{} =
\delta^{(m)}\left(z_1,z_2,s\right)\left(\gamma\left(s-\mu_m\left(z_1,z_2\right)\right)-1\right)+{}\\
{}+
\alpha_m^{-1}\left(z_1,z_2\right)\prod\limits_{l=1}^k
\left(\mu_m\left(z_1,z_2\right)+a_l\right)\psi\left(s-\mu_m(z_1,z_2)\right),\\
j=1,\ldots,k.
\end{multline}
В силу леммы~1 левая часть~\eqref{12} обращается в~0 при
$z_1\hm=z_{1m}(s)$ и~$z_2\hm=z_{2m}(s)$, $m\hm=1,\ldots,k.$ Следовательно,
\begin{multline}
\label{13}
\delta^{(m)}\left(z_{1m}(s),z_{2m}(s),s\right)={}\\
{}=\fr{\psi(s-\lambda_m(s))}{\alpha_m(z_{1m}(s),z_{2m}(s))
(1-\gamma(s-\lambda_m(s)))}\times{}\\
{}\times \prod\limits_{l=1}^k\left(\lambda_m(s)+a_l\right).
\end{multline}
Из \eqref{10} следует, что
$$
q_j(s)=c_j\sum\limits_{m=1}^k\fr{\delta^{(m)}(z_1,z_2,s)}{\mu_m(z_1,z_2)+a_j},\
j=1,\ldots,k .
$$
Решая эту систему уравнений относительно
$\delta^{(m)}(z_1,z_2,s),$ получаем:
\begin{multline}
\label{n1}
\delta^{(m)}(z_1,z_2,s)=\left(pz_1+(1-p)z_2\right)\times{}\\
{}\times
\fr{\prod\nolimits_{j=1}^k(\mu_m(z_1,z_2)+a_j)}
{\alpha_m(z_1,z_2)}\sum\limits_{l=1}^k\frac{a_lq_l(s)}{\mu_m(z_1,z_2)+a_l}.
\end{multline}
Подставляя в~\eqref{n1} $z_1\hm=z_{1m}(s)$ и~$z_2\hm=z_{2m}(s),$ имеем:
\begin{multline}
\label{14}
\delta^{(m)}\left(z_{1m}(s),z_{1m}(s),s\right)={}\\
{}=
\left(pz_{1m}(s)+(1-p)z_{2m}(s)\right)\times{}\\
{}\times
\fr{\prod\nolimits_{j=1}^k
(\lambda_m(s)+a_j)}{\alpha_m(z_{1m}(s),z_{1m}(s))}
\sum\limits_{l=1}^k\fr{a_lq_l(s)}{\lambda_m(s)+a_l}\,.
\end{multline}
Сравнивая два представления~\eqref{13} в~\eqref{14} для
$\delta^{(m)}(z_m(s),s),$ получаем систему уравнений для~$q_l(s)$:
\begin{multline*}
\sum\limits_{l=1}^k\fr{a_lq_l(s)}{\lambda_m(s)+a_l}={}\\
{}=\fr{\psi(s-\lambda_m(s))}{(pz_{1m}(s)+(1-p)z_{2m}(s))
(1-\gamma(s-\lambda_m(s)))},\\
m=1,\ldots,k\,,
\end{multline*}
из которой находим
\begin{multline}
\hspace*{-3pt}q_l(s)=c_l\prod\limits_{j=1}^k
\left(\lambda_l(s)+a_j\right) 
\sum\limits_{m=1}^k
%\fr
\psi(s-\lambda_m(s))\!\Bigg/ \!
\Bigg(\left(1-{}\right.\\
\left.
{}-\gamma\left(s-\lambda_m(s)\right)\right)(\lambda_m(s)+a_l)\times{}\\
{}\times \prod\limits_{n\ne m}(\lambda_m(s)-\lambda_n(s))\!\Bigg).
\label{15}
\end{multline}
Подставляя \eqref{15} в~\eqref{n1} и~учитывая~\eqref{1}, получаем:
\begin{multline*}
\delta^{(m)}(z_1,z_2,s)=\fr{(pz_1+(1-p)z_2)}{\alpha_m(z_1,z_2)}\times
\\
\times\sum\limits_{j=1}^k
\fr{\psi(s-\lambda_j(s))\prod\nolimits_{l=1}^k(\lambda_j(s)+a_l)}
{(pz_{1j}(s)+(1-p)z_{2j}(s))(1-\gamma(s-\lambda_j(s)))}\times{}\\
{}\times\prod\limits_{\nu\ne j}
\fr{\mu_m(z_1,z_2)-\lambda_{\nu}(s)}{\lambda_j(s)-\lambda_{\nu}(s)}\,.
\end{multline*}
Положим
$$
\lambda_m(z_2,s)=\mu_m\left(z_{1m}(z_2,s),z_2\right),\enskip m=1,\ldots,k\,.
$$
Подставляя в~\eqref{12} $z_1\hm=z_{1m}(z_2,s)$, имеем:
\begin{multline}
\label{1q}
\gamma_2^{(m)}\left(z_{1m}(z_2,s),z_2,s\right)={}\\
{}=\fr{\delta^{(m)}(z_{1m}(z_2,s),z_2,s)(\gamma_m(s-\lambda_m(z_2,s))-1)}
{1-z_2^{-1}\beta_2(s-\lambda_m(z_2,s))}+{}
\\
{}+\alpha_m^{-1}(z_{1m}(z_2,s),z_2)\psi(s-\lambda_m(z_2,s))
\prod\limits_{l=1}^k\left(\lambda_m(z_2,s)+{}\right.\\
\left.{}+a_l\right)\!\Bigg/\!
\left(
1-z_2^{-1}\beta_2(s-\lambda_m(z_2,s))\right).
\end{multline}
Далее, из~\eqref{9} следует:
$$
p_{2j}(z_1,z_2,0,s)=c_j\sum\limits_{m=1}^k
\fr{\gamma_2^{(m)}(z_1,z_2,s)}{\mu_m(z_1,z_2)+a_j}\,.
$$
Отсюда
\begin{multline}
\label{2q}
\gamma_2^{(m)}(z_1,z_2,s)=\fr{pz_1+(1-p)z_2}{\alpha_m(z_1,z_2)}\times{}\\
{}\times
\prod\limits_{j=1}^k(\mu_m(z_1,z_2)+a_j)
\sum\limits_{l=1}^k\fr{a_lp_{2l}(z_1,z_2,0,s)}{\mu_m(z_1,z_2)+a_l}\,.
\end{multline}
Так как $p_{2j}(z_1,z_2,0,s)$ не зависит от $z_1$, то
\begin{multline}
\label{3q}
p_{2j}\left(z_1,z_2,0,s\right)={}\\
{}=c_j
\sum\limits_{m=1}^k\fr{\gamma_2^{(m)}\left(z_{1m}(z_2,s),z_2,s\right)}{\lambda_m(z_2,s)+a_j}\,.
\end{multline}
Таким образом, соотношения~\eqref{1q}--\eqref{3q} полностью
определяют $\gamma_2^{(m)}(z_1,z_2,s)$ при любых $z_1$ и~$z_2$.
Теперь из~\eqref{12} можно найти $\gamma_2^{(m)}(z_1,z_2,s)$.

Все функции, необходимые для вычисления $p_{ij}(z_1,z_2,x,s)$,
$i\hm=0,1,2$, $j\hm=1,\ldots,k,$ найде-\linebreak\vspace*{-12pt}

\columnbreak

\noindent
ны. Искомая производящая функция
процесса $(L_1(t),L_2(t))$ равна:

\noindent
\begin{multline*}
\int\limits_0^{\infty}e^{-st}\mathbf{E}
z_1^{L_1(t)} z_2^{L_2(t)}\,dt={}\\
{}=
\sum\limits_{i=0}^2\sum\limits_{j=1}^k\int\limits_0^{\infty}p_{ij}
\left(z_1,z_2,x,s\right)\,dx\,.
\end{multline*}

\vspace*{-18pt}

{\small\frenchspacing
 {%\baselineskip=10.8pt
 \addcontentsline{toc}{section}{References}
 \begin{thebibliography}{9}
\bibitem{1-u}
\Au{Doshi B.\,T.} Queueing systems with vacations~--- a~survey~// 
Queueing Syst., 1986. Vol.~1.  P.~29--66.
\bibitem{2-u}
\Au{Takagi H.} Time-dependent analysis of $M\vert G\vert 1$ vacation models 
with exhaustive service~// Queueing Syst.,
1990. Vol.~6.  P.~369--390.
\bibitem{3-u}
\Au{Li J., Tian N., Zhang~Z.\,G. , Luh~H.\,P.} 
Analysis of the $M\vert G\vert 1$ queue with exponentially working vacations~--- 
a~matrix analytic approach~// Queueing Syst., 2009. Vol.~61.
P.~139--166.
\bibitem{4-u}
\Au{Bouman N., Borst S.\,C., Boxma~O.\,J., Leeuwaarden~J.\,S.\,H.} 
Queues with random back-offs~// Queueing Syst.,
2014. Vol.~77. P.~33--74.
\bibitem{5-u}
\Au{Ушаков~В.\,Г.} Система обслуживания с~гиперэкспоненциальным входящим потоком 
и~профилактиками прибора~// Информатика и~её применения, 2016. Т.~10. 
Вып.~2. С.~93--98.
 \end{thebibliography}

 }
 }

\end{multicols}

\vspace*{-9pt}

\hfill{\small\textit{Поступила в~редакцию 11.05.18}}

\vspace*{6pt}

%\pagebreak

%\newpage

%\vspace*{-28pt}

\hrule

\vspace*{2pt}

\hrule

%\vspace*{-2pt}

\def\tit{A~HEAD OF~THE~LINE PRIORITY QUEUE\\ WITH~WORKING VACATIONS}

\def\titkol{A head of the line priority queue with working vacations}

\def\aut{E.\,S.~Kondranin$^1$ and~V.\,G.~Ushakov$^{1,2}$}

\def\autkol{E.\,S.~Kondranin and~V.\,G.~Ushakov}

\titel{\tit}{\aut}{\autkol}{\titkol}

\vspace*{-11pt}


\noindent
$^1$Department of 
Mathematical Statistics, Faculty of Computational Mathematics and Cybernetics, 
M.\,V.~Lo\-mo-\linebreak
$\hphantom{^1}$no\-sov Moscow State University, 1-52~Leninskiye Gory, 
Moscow 119991, GSP-1, Russian Federation

\noindent
$^2$Institute of Informatics Problems, Federal Research Center 
``Computer Science and Control'' of the Russian\linebreak
$\hphantom{^1}$Academy of Sciences,  44-2~Vavilov Str., Moscow 119333, Russian Federation

\def\leftfootline{\small{\textbf{\thepage}
\hfill INFORMATIKA I EE PRIMENENIYA~--- INFORMATICS AND
APPLICATIONS\ \ \ 2018\ \ \ volume~12\ \ \ issue\ 4}
}%
 \def\rightfootline{\small{INFORMATIKA I EE PRIMENENIYA~---
INFORMATICS AND APPLICATIONS\ \ \ 2018\ \ \ volume~12\ \ \ issue\ 4
\hfill \textbf{\thepage}}}

\vspace*{3pt}



\Abste{The authors analyze the single-server queueing system with 
two types of customers, head of the line priority, hyperexponential 
input stream, and working vacations. The authors obtain the Laplace 
transform (with respect to an arbitrary point in time) of the joint 
distribution of server state, queue size, and elapsed time in that state. 
The authors restrict themselves to a~system with exhaustive service (the 
queue must be empty when the server starts a vacation) and multiple vacations. 
The queueing systems with vacations have been well studied because of their 
applications in modeling computer networks, communication, and manufacturing 
systems. For example, in many digital systems, the processor is multiplexed 
among a~number of jobs and, hence, is not available all the time to handle one job type. 
Besides such an application, theoretical interest in vacation models 
has been aroused with respect to their relationship with polling models.}

\KWE{hyperexponential input stream; working vacations; single server; 
head of the line priority; queue length}



\DOI{10.14357/19922264180405}

\vspace*{-14pt}

\Ack
\noindent
This work was supported by the Russian Foundation for Basic Research 
(project 18-07-00678).


%\vspace*{6pt}

  \begin{multicols}{2}

\renewcommand{\bibname}{\protect\rmfamily References}
%\renewcommand{\bibname}{\large\protect\rm References}

{\small\frenchspacing
 {%\baselineskip=10.8pt
 \addcontentsline{toc}{section}{References}
 \begin{thebibliography}{9}
\bibitem{1-u-1}
\Aue{Doshi, B.\,T.} 1986. Queueing systems with vacations~--- a~survey. 
\textit{Queueing Syst.} 1:29--66.
\bibitem{2-u-1}
\Aue{Takagi, H.} 1990. Time-dependent analysis of $M\vert G\vert M\vert 1$ 
vacation models with exhaustive service. \textit{Queueing Syst.} 6:369--390.
\bibitem{3-u-1}
\Aue{Li, J., N. Tian, Z.\,G.~Zhang,  and H.\,P.~Luh.} 2009. Analysis of the 
$M\vert G\vert 1$ queue with exponentially working vacations~--- 
a~matrix analytic approach. \textit{Queueing Syst.} 61:139--166.
{\looseness=1

}
\bibitem{4-u-1}
\Aue{Bouman, N., S.\,C.~Borst, O.\,J.~Boxma, and J.\,S.\,H.~Leeuwaarden.} 
2014. Queues with random back-offs. \textit{Queueing Syst.} 77:33--74.
\bibitem{5-u-1}
\Aue{Ushakov, V.\,G.} 2016. Sistema obsluzhivaniya s~gipereksponentsialnym 
vkhodyashchim potokom i~profilaktikami\linebreak pribora [Queueing system with working 
vacations and hyperexponential input stream]. 
\textit{Informatika i~ee Primeneniya~--- Inform. Appl.} 10(2):93--98.
\end{thebibliography}

 }
 }

\end{multicols}

\vspace*{-6pt}

\hfill{\small\textit{Received May 11, 2018}}

%\pagebreak

%\vspace*{-18pt}

\Contr

\noindent
\textbf{Kondranin Egor S.} (b.\ 1995)~---  MSc student, Department of 
Mathematical Statistics, Faculty of Computational Mathematics and Cybernetics, 
M.\,V.~Lomonosov Moscow State University, 1-52~Leninskiye Gory, 
Moscow 119991, GSP-1, Russian Federation; \mbox{ekondranin@yandex.ru}

\vspace*{6pt}

\noindent
\textbf{Ushakov Vladimir G.} (b.\ 1952)~--- 
Doctor of Science in physics and mathematics, professor, Department of Mathematical 
Statistics, Faculty of Computational Mathematics and Cybernetics, 
M.\,V.~Lomonosov Moscow State University, 1-52~Leninskiye Gory, Moscow 119991, 
GSP-1, Russian Federation; 
senior scientist, Institute of Informatics Problems, Federal Research Center 
``Computer Science and Control'' of the Russian Academy of Sciences, 
44-2~Vavilov Str., Moscow 119333, Russian Federation; \mbox{vgushakov@mail.ru}
\label{end\stat}

\renewcommand{\bibname}{\protect\rm Литература}        %4
\def\stat{nazarov}

\def\tit{ВЕРОЯТНОСТНАЯ МОДЕЛЬ ВЛИЯНИЯ КНИГИ ЗАКАЗОВ
НА~ПРОЦЕСС ЦЕНЫ}

\def\titkol{Вероятностная модель влияния книги заказов
на~процесс цены}

\def\aut{Е.\,В.~Быковец$^1$, В.\,В.~Лаврентьев$^2$,  Л.\,В.~Назаров$^3$}

\def\autkol{Е.\,В.~Быковец, В.\,В.~Лаврентьев,  Л.\,В.~Назаров}

\titel{\tit}{\aut}{\autkol}{\titkol}

\index{Быковец Е.\,В.}
\index{Лаврентьев В.\,В.}
\index{Назаров Л.\,В.}
\index{Nazarov L.\,V.}
\index{Lavrentyev V.\,V.}
\index{Bykovets E.\,V.}




%{\renewcommand{\thefootnote}{\fnsymbol{footnote}} \footnotetext[1]
%{Работа поддержана РНФ (проект 16-11-10227).}}


\renewcommand{\thefootnote}{\arabic{footnote}}
\footnotetext[1]{Московский государственный университет им.\ М.\,В.~Ломоносова, 
факультет вычислительной математики и~кибернетики, 
\mbox{eugene.bykovets@stud.cs.msu.su}}
\footnotetext[2]{Московский государственный университет им.\ М.\,В. Ломоносова, 
факультет вычислительной математики и~кибернетики, \mbox{lavrent@cs.msu.ru}}
\footnotetext[3]{Московский государственный университет им.\ М.\,В. Ломоносова, 
факультет вычислительной математики и~кибернетики, 
\mbox{nazarov@cs.msu.ru}}

\vspace*{-9pt}


   

\Abst{Рассматривается модель книги заказов, в~которой заказы на покупку 
и~продажу образуют два независимых процесса Кокса. Предложен механизм
        влияния поступающих заказов на цену актива на основе физической модели 
        абсолютно упругого соударения. В~этой модели цена представляет собой 
        материальную точку с~некоторой массой, движущуюся по прямой без трения. 
        Приходящие заказы на покупку и~уходящие заказы на продажу упруго 
        сталкиваются с~ней и~придают дополнительный импульс в~одном 
        направлении, а~приходящие заказы на продажу и~уходящие заказы на покупку~--- 
        в~противоположном. Получена функциональная предельная теорема для процесса 
        цены при высокой интенсивности входящего потока заказов, позволяющая 
        аппроксимировать его некоторым процессом Леви.}

\KW{лимитные заявки; абсолютно упругий удар; модель книги заказов; процесс цены; процесс Кокса; 
функциональная предельная тео\-рема}

\DOI{10.14357/19922264180205}
  
\vspace*{-3pt}


\vskip 10pt plus 9pt minus 6pt

\thispagestyle{headings}

\begin{multicols}{2}

\label{st\stat}

\section{Введение}

Рассмотрим некоторый торгуемый на бирже актив, в~отношении которого 
могут приходить два вида запросов: на покупку и~на продажу. 
Список таких запросов формирует книгу заказов для данного актива. 
Информация, содержащаяся в~книге, позволяет делать прогнозы относительно 
возможного движения цены рассматриваемого актива. 
Особенный интерес эта информация начала представлять с~развитием высокочастотной
 торговли.

В работе рассматривается модель, которая описывает влияние книги 
заказов на цену актива. Базовой моделью для исследования была выбрана модель 
книги заказов, близкая к~описанной в~\cite{first}. Основное отличие состоит
 в~следующем: на бирже торгуемый актив имеет цену, которая размещается в~узлах 
 сетки~$nh$, где $n$~--- некоторое целое число; $h$~--- тик, т.\,е.\ 
 минимальное изменение цены. Однако высокая частота узлов сетки позволяет считать, 
 что рассматриваемый актив может иметь произвольную цену, равно как и~заявки на 
 покупку и~продажу торгуемого актива. С~учетом этого рассматриваем следующую модель 
 влияния заявок на цену актива, используя физическую аналогию.  
 %
 Рассмотрим 
 материальную точку массой~$M$, которая может двигаться по прямой (числовой оси) 
 в~любом на\-прав\-ле\-нии без трения. При этом, связывая модель 
 физическую и~математическую, будем считать, что текущее положение на оси~--- 
 это текущая цена~$X(t)$.  Будем далее для краткости называть ценой и~саму 
 указанную материальную точку массой~$M$, т.\,е.\ 
 будем говорить о~ско\-рости цены, импульсе цены и~т.\,п. 
 Каждый заказ на продажу (поступающий по цене~$A_{i}$
не ниже, чем~$X(t)$) придает цене дополнительный импульс в~направлении от~$A_{i}$ 
к~$X(t)$. Заказы живут экспоненциальное время, после чего уходят из книги 
за счет исполнения или отмены. Уход заказа из книги  придает цене 
дополнительный импульс той же абсолютной величины, что и~при ее 
поступлении, но противоположного направления. С~заказами на продажу все аналогично, 
только приходят они с~ценой, не превосходящей~$X(t)$.

Цель данной работы состоит в~том, чтобы выяснить, какой процесс движения цены 
порождает такая сис\-те\-ма при интенсивном потоке приходящих заказов.
Близкая задача решалась авторами в~работе~\cite{second}, но там рас\-смат\-ри\-вал\-ся 
другой механизм влияния по\-сту\-па\-ющих заказов на цену. Здесь следует упомянуть 
и~работу~\cite{Korolev1}, в~которой также изуча\-ет\-ся связь механизма 
функционирования книги заказов на микроуровне с~процессом цены.

\vspace*{-12pt}

\section{Описание модели}

\vspace*{-4pt}

Будем рассматривать работу книги заказов на временн$\acute{\mbox{о}}$м интервале $t\hm\in[0,T]$. 
В~начальный момент времени $t\hm=0$ заказов в~книге нет. Считаем, что поток 
приходящих заказов является процессом Кокса следующего вида:
\begin{equation*}
\left\{N(\Lambda(t))=N_1(\Lambda(t)), t\geqslant0\right\}\,,
\end{equation*}
где
$N_1$~--- пуассоновский процесс, интенсивность которого равна~1; 
$\Lambda(t)$~--- стартующий из нуля случайный процесс, у~которого траектории 
являются неубывающими и~непрерывными справа функциями, а~также справедливо 
$\mathbb{P}(\Lambda(t)\hm<\infty)$.

Каждый заказ, приходящий в~книгу, находится в~ней некоторое случайное время. 
Более точно, время нахождения конкретного заказа в~книге является случайной 
величиной, распределенной по экспоненциальному закону.

Для каждого приходящего заказа определен набор параметров $(h_i, \gamma_i, \eta_i)$, 
где 
$h_i$~--- абсолютное значение разности между ценой заказа и~текущей ценой; 
$\gamma_i$~--- разность между ско\-ростью приходящего заказа и~текущей ско\-ростью цены; 
$\eta_i$~--- время пребывания заказа в~книге. 

Случайные величины $h_i$, $\gamma_i$ и~$\eta_i$, $i\hm=1, 2, \ldots,$ независимы 
в~совокупности и~не зависят от потока заявок, a~$\eta_i$ распределены 
экспоненциально с~параметром~$\mu$. Параметр~$\gamma_i$ фактически определяет 
тип заказа. Для заказов на покупку~$\gamma_i$ положительны, для заказов на 
продажу~--- отрицательны.

В качестве физической модели влияния заказа на цену возьмем модель абсолютно 
упругого удара.  Считаем, что $i$-й заказ, поступающий в~момент~$t$,~--- 
это материальная точка массой $m_0(h_i)\hm>0$, которая движется по той же прямой, 
что и~цена, и~имеет в~момент времени~$t$ скорость, равную $u_i \hm= v_{i-1}\hm+\gamma_i$, 
где~$v_{i-1}$~--- скорость цены до столкновения с~$i$-м заказом. В~момент~$t$ 
происходит их упругое соударение. На распределения~$h_i$ и~$\gamma_i$ 
наложим следующие ограничения:
\begin{equation}
\mathbb{E}\gamma_i = 0;\enskip
\mathbb{E}\gamma_i^2 = \overline{\gamma}<\infty;\enskip
\mathbb{E}m_0(h_i)^2 = \overline{m} < \infty\,.
\label{e1-naz}
\end{equation}
Смысл первого условия заключается в~том, что разности между скоростями 
приходящих заказов и~текущей скоростью актива для заказов на покупку и~продажу 
в~среднем равны. Остальные ограничения имеют технический характер и~лишь постулируют 
конечность соответствующих моментов.

Функция $m_0$ является убывающей на интервале~$(0, \infty)$, поскольку 
воздействие заказа на цену тем больше, чем ближе его цена к~текущей цене актива.
Это соответствует реальному положению дел на рынке, где заказы на уровнях, 
близких к~текущей цене, выставляются более ответственно, так как могут быть 
тут же удовлетворены. В~то же время заказы на более удаленных уровнях чаще 
ставятся для дезориентации других участников  рынка и~снимаются\linebreak\vspace*{-12pt}

\columnbreak

\noindent
 при приближении к~ним 
цены. Иными словами, они не отражают реальный спрос. 

\vspace*{-7pt}

\section{Процесс цены}

Рассмотрим точку на числовой прямой, которая представляет собой текущую цену 
актива. Положим 
$M$~--- масса точки; 
$v_{i}$~--- текущая скорость цены, полученная после удара $i$-й частицы, 
полагаем $v_ {0}=0$; 
$u_{i}$~--- ско\-рость $i$-го заказа до соударения; 
$m_0(h_i)$~--- масса $i$-го заказа. 
Как было сказано в~предыду\-щем разделе, данная точ\-ка (исследуемая цена) 
в~определенные моменты времени абсолютно упруго соударяется с~другими 
частицами (заказами). В~этом случае есть возможность выразить ско\-рость точки после 
удара $i$-й час\-ти\-цы через массу точ\-ки и~массу час\-ти\-цы, а~также их ско\-рости 
до столк\-но\-ве\-ния (это следует из закона сохранения импульса и~закона сохранения 
энергии, см.~\cite[гл.~4, \S\,28]{third}):
\begin{equation*}
v_i = -v_{i-1} +2\fr{Mv_{i-1}+ m_0(h_i)u_i}{M+ m_0(h_i)}\,.
\end{equation*}
Обозначим $\Delta v_i \hm= v_i - v_{i-1}$, тогда

\noindent
\begin{multline*}
\Delta v_i = -2v_{i-1} +2\fr{Mv_{i-1}+ m_0\left(h_i\right)u_i}{M+ m_0\left(h_i\right)} = {}\\
{}=
\fr{2 m_0\left(h_i\right)}{M+ m_0\left(h_i\right)}\left(u_{i}-v_{i-1}\right).
\end{multline*}
Фактически $\Delta v_i$ показывает изменение ско\-рости цены после соударения 
с~$i$-м заказом.

Пусть в~начальный момент книга заказов пус\-та, а~начальная скорость $v_0 \hm= 0$. 
Далее полагаем, что заказ с~номером~$i$ приходит в~момент времени~$\tau_{i0}$ и~уходит 
в~момент времени~$\tau_{i1}$. В~итоге получаем, что скорость цены является случайным 
процессом $\{V(t), t\hm\geqslant0\}$ с~ку\-соч\-но-по\-сто\-ян\-ны\-ми траекто\-риями:

\noindent
\begin{equation*}
V(t) = \sum\limits_{i=1}^{N_1(\Lambda(t))}\Delta v_i\mathbb{I}_{\{\tau_{i0}\le t\le 
\tau_{i1}\}}(t)\,.
\end{equation*}
Тогда изменение цены за время~$T$ будет иметь вид:
\begin{equation}
X(T) = \int\limits_{0}^{{T}} V(t)\, dt = \sum\limits_{i=1}^{N_1(\Lambda(T))}X_i(T)\,,
\label{e2-naz}
\end{equation}
где $X_i(T)$~--- изменение цены на интервале [0,T] за счет удара $i$-го заказа:

\vspace*{-2pt}

\noindent
\begin{multline*}
X_i(T) = \int\limits_{0}^{{T}} \Delta v_i\mathbb{I}_{\{\tau_{i0}\leqslant 
t\leqslant \tau_{i1}\}}(t)\,dt ={}\\
{}= \fr{2 m_0(h_i)}{M+ m_0(h_i)}
\int\limits_{0}^{{T}}(u_{i}-v_{i-1})\mathbb{I}_{\{\tau_{i0}\leqslant t
\leqslant \tau_{i1}\}}(t)\,dt\,.
\end{multline*}
Поскольку по определению $u_{i}\hm-v_{i-1} \hm= \gamma_i$ , то последнее 
выражение можем переписать в~виде ($a \wedge b\hm = \min(a,b)$):
\begin{multline*}
X_i(T)=\fr{2 m_0(h_i)\gamma_i}{M+ m_0(h_i)}\int\limits_{0}^{{T}}
\mathbb{I}_{\{\tau_{i0}\leqslant t\leqslant \tau_{i1}\}}(t)\,dt = {}\\
{}=
\fr{2 m_0(h_i)\gamma_i}{M+ m_0(h_i)}\left(T\wedge\tau_{i1}-
T\wedge\tau_{i0}\right) ={}\\
{}=\fr{2 m_0(h_i)\gamma_i}{M+ m_0(h_i)}
\left(T\wedge\left(\tau_{i0}+\eta_i\right)-T\wedge\tau_{i0}\right).
\end{multline*}
Строго говоря, случайные величины $\{X_i(T)$, $i\hm=1,2,\dots\}$ не являются 
независимыми, но суммы в~(\ref{e2-naz}) можно представить в~виде сумм 
независимых случайных величин. 

Рассмотрим распределение вектора 
моментов прихода заявок $\tau_0\hm=\{\tau_{10},\ldots ,\tau_{n0}\}$. По свойству 
пуассоновского потока при $N_1(\Lambda(T))\hm=n$ распределение~$\tau_0$ 
есть распределение вариационного ряда выборки из~$n$~независимых случайных 
величин,\linebreak равномерно распределенных на $[0, \Lambda(T)]$.
Поскольку значение конечной суммы при перестановке\linebreak слагаемых не меняется, 
далее будем считать, что в~каждой из сумм~(\ref{e2-naz})~$\tau_{i0}$ 
независимы и~равномерно распределены на $[0, \Lambda(T)]$, а~следовательно, 
случайные величины~$\{X_i(T)$, $i\hm=1,2,\dots\}$ также независимы.

Изучим асимптотические свойства моментов~$X_i(T)$.

\smallskip

\noindent
\textbf{Лемма~1.}\ \textit{ Пусть случайная величина~$\xi$ 
равномерно распределена на $[0,T]$, $\eta_0$ не зависит от~$\xi$ и~имеет 
экспоненциальное распределение с~параметром~$\mu$ и}
\begin{equation*}
s = T\wedge\left(\xi+\eta_0\right)-\xi\,.
\end{equation*}

\vspace*{-8pt}

\noindent
\textit{Тогда}
\begin{enumerate}[(1)]
\item $s\stackrel{d}=\xi\wedge\eta_0$;
\item \textit{моменты случайной величины s обладают следующими асимптотическими 
свойствами}: 
\begin{equation*}
\lim\limits_{\mu\rightarrow\infty}\mu\mathbb{E}s = 1;\
 \lim\limits_{\mu\rightarrow\infty}\mu^2\mathbb{E}s^2 = 2;\
  \lim\limits_{\mu\rightarrow\infty}\mu^2\mathbb{D}s = 1.
\end{equation*}
\end{enumerate}

\noindent
{Д\,о\,к\,а\,з\,а\,т\,е\,л\,ь\,с\,т\,в\,о\,.}\ \ 
Вычислим математическое ожидание случайной величины~$s$ с~учетом независимости~$\xi$ 
и~$\eta_0$. По определению
\begin{equation*}
s = T\wedge\left(\xi+\eta_0\right)-\xi = (T-\xi)\wedge\eta_0\,.
\end{equation*}
Справедливость первого утверждения леммы следует из независимости~$\xi$ и~$\eta_0$ 
и~одинаковой распределенности~$\xi$ и~$T\hm-\xi$. Таким образом, математическое 
ожидание~$s$ есть
\begin{equation*}
\mathbb{E}s  = \mathbb{E}\left(\xi\wedge\eta_0\right).
\end{equation*}
При вычислении моментов неоднократно будет требоваться значение интеграла
\begin{equation*}
\int\limits_{0}^{{T}}y^ne^{-\mu y}\,dy = \fr{n!}{\mu^{n+1}}\,F_{n+1}(T)\,,
\end{equation*}
где $F_{n+1}$~--- 
функция распределения Эрланга $(n+1)$-го порядка:
$$
F_{n+1}(x) = 1 - e^{-\mu x}\sum\limits_{i=1}^{n}\fr{\mu^iT^i}{i!}\,.
$$ 
Вычислим 
$\mathbb{E}(\xi\wedge\eta_0)$:
\begin{multline*}
\mathbb{E}\left(\xi\wedge\eta_0\right) =
\fr{\mu}{T}\int\limits_{0}^{\infty}\!e^{-\mu y}\,dy\int\limits_{0}^{T}(x\wedge y)\,dx 
 ={}\\
 {}=\fr{\mu}{T}\int\limits_{0}^{\infty}\!e^{-\mu y}\,dy\left\{
 \mathbb{I}_{\{y<T\}}\left[\int\limits_{0}^{y}x\,dx+
 \int\limits_{y}^{T}y\,dx\right] + {}\right.\\
\left. {}+
 \mathbb{I}_{\{y\geqslant T\}}\int\limits_{0}^{T}x\,dx\right\} = \fr{\mu}{T}\left[T\int\limits_{0}^{T}\!ye^{-\mu y}\,dy -{}\right.\\
\left.{}-
\fr{1}{2}\int\limits_{0}^{T}\!y^2e^{-\mu y}\,dy + 
\fr{T^2}{2}\int\limits_{T}^{\infty}\!e^{-\mu y}\,dy \right]={} \\
{}=\left[\fr{1}{\mu}-\fr{1}{\mu^2 T}\right]+\left[\fr{T}{2}+\fr{1}{\mu^2 T}\right]
e^{-\mu T}.
\end{multline*}
И,~соответственно,
\begin{multline*}
\lim\limits_{\mu \rightarrow \infty}\mu\mathbb{E}s ={}\\
{}= 
\lim\limits_{\mu \rightarrow \infty}\left\{\left[1-\fr{1}{\mu T}\right]+
\left[\fr{T}{2}+\fr{1}{\mu^2 T}\right]\mu e^{-\mu T}\right\}=1. 
\end{multline*}
Вычислим $\mathbb{E}s^2\hm=\mathbb{E}(\xi\wedge\eta_0)^2$:
\begin{multline*}
\mathbb{E}\left(\xi\wedge\eta_0\right)^2=
\fr{\mu}{T}\int\limits_{0}^{\infty}\!e^{-\mu y}\,dy
\int\limits_{0}^{T}\left(x\wedge y\right)^2\,dx = {}\\
{}=\fr{\mu}{T}\int\limits_{0}^{\infty}\!e^{-\mu y}\,dy\left\{
\mathbb{I}_{\{y<T\}}\left[{\int\limits_{0}^{y}\!x^2\,dx+
\int\limits_{y}^{T}\!y^2\,dx}\right] + {}\right.\\
\left.{}+
\mathbb{I}_{\{y\geqslant T\}}\int\limits_{0}^{T}\!x^2\,dx\right\}= 
\fr{\mu}{T}\left[T\int\limits_{0}^{T}\!y^2e^{-\mu y}\,dy -{}\right.\\
\left.{}-
\fr{2}{3}\int\limits_{0}^{T}\!y^3e^{-\mu y}dy + 
\fr{T^3}{3}\int\limits_{T}^{\infty}\!e^{-\mu y}dy \right]= {}\\
{}=\left[\fr{2}{\mu^2}-\fr{4}{\mu^3 T}\right]+\left[
\fr{2}{\mu^2}+\fr{4}{\mu^3 T}\right]e^{-\mu T}.
\end{multline*}
Отсюда получаем асимптотику второго момента
\begin{multline*}
\lim\limits_{\mu \rightarrow \infty}\mu^2\mathbb{E}s^2 = {}\\
{}=
\lim\limits_{\mu \rightarrow \infty}\left\{\left[2-\fr{4}{T\mu}\right]+
\left[\fr{2}{\mu^2}+\fr{4}{\mu^3 T}\right]\mu^2e^{-\mu T}\right\} = 2
\end{multline*}
и дисперсии
\begin{multline*}
\lim\limits_{\mu \rightarrow \infty}\mu^2\mathbb{D}s =  
\lim\limits_{\mu \rightarrow \infty}\mu^2\left[\mathbb{E}s^2-(\mathbb{E}s)^2\right] ={}\\
{}=
 \lim\limits_{\mu \rightarrow \infty}\mu^2\mathbb{E}s^2 - 
  \lim\limits_{\mu \rightarrow \infty}(\mu\mathbb{E}s)^2 =1\,.
\end{multline*}
Утверждение леммы доказано.

\smallskip

Рассмотрим следующую последовательность:
\begin{equation}
\left\{X_{n}(t) = \sum\limits_{i=1}^{N_1(\Lambda_n(t))}X_{ni}(t),\ 
 t \in [0,T]\right\}.
 \label{e3-naz}
\end{equation}
При этом каждому члену~$\{X_n\}$ соответствует процесс $\{\Lambda_n(t)$, 
$t\hm\in [0,T]\}$, параметр~$\mu_n$ и~функция массы ударяющей частицы (приходящего 
заказа)~$m_{n0}$. При увеличении~$n$ будем увеличивать интенсивность входящего 
потока заявок~$\Lambda_n(t)$ и~уменьшать время пребывания заказов в~книге 
посредством увеличения~$\mu_n$ ($\Lambda_n(t)\hm\Rightarrow\infty$, 
$\mu_n\hm\rightarrow\infty$ при $n \hm\rightarrow \infty$). Будем также 
уменьшать влияние отдельного заказа на цену:
\begin{equation*}
m_{n0} = \alpha_n m_0,\enskip
 \alpha_n >0\,,\enskip
  \alpha_n \rightarrow 0\,,\enskip
   n\rightarrow\infty\,.
\end{equation*}
Получим асимптотические свойства моментов случайных величин $X_{n1}(T)$ 
при установленных параметрических зависимостях. Аргумент~$T$ у~них одинаков и~для 
краткости будем его опускать.

\smallskip

\noindent
\textbf{Лемма~2.}\  \textit{Пусть $\mu_n\hm \rightarrow \infty$, 
$\alpha_n \hm\rightarrow 0$, $k_n \hm= {\mu_n^2}/{\alpha_n^2}$.  Тогда}
\begin{enumerate}[(1)]
\item $k_n\mathbb{E}X_{n1}\hm\rightarrow 0$, 
$k_n\mathbb{D}X_{n1}\hm\rightarrow {8\overline{m}\overline{\gamma}}/{M^2}$,
$n\hm\rightarrow\infty$;
\item \textit{Выполняется условие Линдеберга, т.\,е.\ для любого} $\varepsilon \hm>0$
\begin{equation*}
\lim\limits_{n\rightarrow\infty}k_n
\mathbb{E}\left[X_{n1}^2\mathbb{I}(|X_{n1}|>\varepsilon)\right]=0\,.
\end{equation*}
\end{enumerate}

\noindent
Д\,о\,к\,а\,з\,а\,т\,е\,л\,ь\,с\,т\,в\,о\,.\ \
Как было показано выше, 
$$
X_{n1} = \fr{2 m_{n0}(h_1)\gamma_{1}}{M+ m_{n0}(h_1)}s_n\,,
$$
 где $s_n\stackrel{d} =\xi\wedge\eta_{n0}$ и~$\eta_{n0}$ распределена 
 экспоненциально с~параметром~$\mu_n$, а~величины $h_1$, $\xi$, $\eta_{n0}$
и~$\gamma_{1}$ независимы. Так как в~соответствии с~(\ref{e1-naz}) 
 $\mathbb{E}\gamma_1 \hm= 0$, то $k_n\mathbb{E}X_{n1}\hm=0$ для любого~$n$. 
 Проверим соотношение для дисперсии:
 
 \noindent
\begin{multline*}
k_n\mathbb{D}X_{n1} = \fr{\mu_n^2}{\alpha_n^2}\, \mathbb{E} 
\left[\fr{2 m_{n0}(h_1)\gamma_{1}}{M+ m_{n0}(h_1)}\,s_n\right]^2={}\\
{}=4\mathbb{E}\gamma_{1}^2\mathbb{E}\left[
\fr{m_{n0}(h_1)}{\alpha_n}\,\fr{1}{M+m_{n0}(h_1)}\right]^2\mu_n^2\mathbb{E}s_n^2 = {}\\
{}=
4\overline{\gamma}\mathbb{E}\left[\fr{m_0(h_1)}{M+m_{n0}(h_1)}\right]^2\mu_n^2
\mathbb{E}s_n^2.
\end{multline*}

Последовательность случайных величин 
$\{[{m_0(h_1)}/({M+m_{n0}(h_1)})]^2\}$ мажорируется интегрируемой случайной 
величиной $[{m_0(h_1)}/{M}]^2$ и~поточечно сходится к~ней, так как $\alpha_n 
\hm\rightarrow 0$, $n\hm\rightarrow\infty$. Поэтому

\vspace*{-6pt}

\noindent
\begin{multline*}
\lim\limits_{n\rightarrow\infty}k_n\mathbb{D}X_{n1} =  
\lim\limits_{n\rightarrow\infty}4\overline{\gamma}\mathbb{E}
\left[\fr{m_0(h_1)}{M}\right]^2\mu_n^2\mathbb{E}s_n^2 ={}\\
{}=
\fr{4\overline{m}\overline{\gamma}}{M^2}
\lim\limits_{n\rightarrow\infty}\mu_n^2\mathbb{E}s_n^2 = 
\fr{8\overline{m}\overline{\gamma}}{M^2}\,.
%\label{e4-naz}
\end{multline*}

Докажем справедливость условия Линдеберга. Рассмотрим

\vspace*{-6pt}

\noindent
\begin{multline*}
\fr{\mu_n}{\alpha_n}\left\vert X_{n1}\right\vert = 
 \fr{\mu_n}{\alpha_n}\left\vert \fr{2 m_{n0}(h_1)\gamma_{1}}{M+ m_{n0}
 \left(h_1\right)}\,s_n\right\vert = {}\\
 {}=
 2\fr{m_{n0}(h_1)}{\alpha_n}\,\fr{1}{M+ m_{n0}(h_1)}\left\vert \gamma_1\right\vert
 \mu_ns_n \leqslant {}\\
{}\leqslant 2\fr{m_0(h_1)}{M}\left\vert \gamma_1\right\vert \mu_ns_n
\leqslant {}\\
{}\leqslant \fr{2m_0(h_1)|\gamma_1|\mu_n\eta_{n0}}{M} \stackrel{d}= 
\fr{2m_0(h_1)|\gamma_1|\hat{\eta}}{M}\,,
\end{multline*}
где $\hat{\eta}$ распределена экспоненциально с~па\-ра\-мет\-ром~1. 
Распределение случайной величины 
$$
Y_n =   \fr{{2m_0(h_1)|\gamma_1|\mu_n\eta_{n0}}}{M}
$$ 
не зависит от~$n$ и~согласно~(1) имеет конечный второй момент, поэтому

\noindent
\begin{multline*}
k_n\mathbb{E}\left[X_{n1}^2\mathbb{I}(|X_{n1}|>\varepsilon)\right] ={}\\
{}= 
k_n\mathbb{E}\left[X_{n1}^2\mathbb{I}(\sqrt{k_n}|X_{n1}|>
\sqrt{k_n}\varepsilon)\right] \leqslant {} \\
{}\leqslant \mathbb{E}\left[Y_{n}^2\mathbb{I}(|Y_{n}|>
\sqrt{k_n}\varepsilon)\right]
= \mathbb{E}\left[Y_{1}^2\mathbb{I}(|Y_{1}|>
\sqrt{k_n}\varepsilon)\right].\hspace*{-3.79228pt}
\end{multline*}
Последнее математическое ожидание стремится к~нулю при $n\hm\rightarrow\infty$ по 
тео\-ре\-ме Лебега о~ма\-жо\-ри\-ру\-емой сходимости. Утверждение леммы доказано.

\vspace*{2pt}

Сформулируем доказанную в~работе~\cite{fourth} функ\-циональную центральную 
предельную тео\-ре\-му, устанавливающую условия, при которых процессы вида~(\ref{e3-naz}) 
сходятся к~некоторому предельному процессу~$X$ в~про\-стран\-ст\-ве Скорохода 
$\mathcal{D}\hm = \mathit{(D[0,1], d_0)}$ (см.~\cite[гл.~3]{five}). 
Позднее были получены более сильные результаты, касающиеся схо\-ди\-мости обобщенных 
процессов Кокса~(\ref{e3-naz}) (см., на\-при\-мер,~\cite{Korolev_FLT}), 
но достаточно будет приводимого ниже утверж\-де\-ния.
{ %\looseness=1

}

\smallskip

\noindent
\textbf{Теорема}~\cite{fourth}.\ 
\textit{Пусть для некоторой неограниченно возрастающей последовательности 
чисел~$\{k_n\}_{n\geqslant1}$ выполнены условия}:
\begin{enumerate}[(1)]
\item \textit{существуют числа $a \hm\in \mathbb{R}$ и~$\sigma \hm> 0$ такие, что}
\begin{equation*}
k_n\mathbb{E}X_{n1} \rightarrow a;\enskip
 k_n\mathbb{D}X_{n1} \rightarrow \sigma^2 (n\rightarrow \infty);
\end{equation*}
\item \textit{условие Линдеберга, т.\,е.\ для любого} $\varepsilon\hm>0$
\begin{equation*}
\lim\limits_{n\rightarrow\infty}k_n\mathbb{E}\left[(X_{n1}-a_n)^2\mathbb{I}
\left(\left\vert X_{n1}-a_n\right\vert >\varepsilon\right)\right]=0\,,
\end{equation*}
\textit{где $\mathbb{I}(A)$~--- индикатор события}~$A$; $a_n \hm= \mathbb{E}X_{n1}$;
\item \textit{существует безгранично делимая случайная величина~$U$ такая, 
что $\mathbb{P}(U=0) \hm< 1$, $\mathbb{P}(U\geqslant0) \hm= 1$, 
$\mathbb{E}U^2 \hm< \infty$ и}
\begin{equation*}
k_n^{-1}\Lambda_n(1)\Rightarrow U,n\rightarrow\infty;
\end{equation*}
\item
$\displaystyle \sup\limits_nk_n^{-2}\mathbb{E}\Lambda_n(1)^2<\infty.
$
\end{enumerate}
\textit{Тогда обобщенные процессы Кокса $\{X_n\}$ 
слабо сходятся в~пространстве Скорохода~$\mathcal{D}$ к~процессу Леви~$X$ такому, что}
\begin{equation*}
X(1) \stackrel{d}=\sigma\sqrt{U}N(0,1)+aU\,,
\end{equation*}
\textit{где $N(0,1)$~--- случайная величина, имеющая стандартное нормальное 
распределение, при этом не зависящая от}~$U$.

Для последовательности $\{k_n = {\mu_n^2}/{\alpha_n^2}\}$ при $a\hm=0$ первые 
два условия теоремы выполняются по лемме~2. Таким образом, достаточно наложить 
определенные условия на ин\-тен\-сив\-ность входящего потока заявок, чтобы была 
справедлива сле\-ду\-ющая теорема.

\smallskip

\noindent
\textbf{Теорема 1.}  \textit{Пусть $\mu_n \hm\rightarrow \infty$, 
$\alpha_n \hm\rightarrow 0$, $k_n = {\mu_n^2}/{\alpha_n^2}$, 
$\sup_nk_n^{-2}\mathbb{E}\Lambda_n(1)^2\hm<\infty$ и~существует 
безгранично делимая случайная величина~$U$ такая, что}
\begin{equation*}
\mathbb{P}(U=0) < 1\,;\enskip
\mathbb{P}(U\geqslant0) = 1\,;\enskip
 \mathbb{E}U^2 < \infty
\end{equation*}
и

\noindent
\begin{equation*}
k_n^{-1}\Lambda_n(1)\Rightarrow U\,,\enskip n\rightarrow\infty\,.
\end{equation*}
\textit{Тогда обобщенные процессы Кокса~$\{X_n\}$ слабо сходятся в~пространстве 
Скорохода $\mathcal{D}$ к~процессу Леви~$X$ такому, что}
\begin{equation*}
X(1) \stackrel{d}=\sigma\sqrt{U}N(0,1)\,,
\end{equation*}

%\columnbreak

\noindent
\textit{где $\sigma = {8\overline{m}\overline{\gamma}}/{M^2}$, а $N(0,1)$~--- 
случайная величина со стандартным нормальным распределением, независимая от}~$U$. 

%\vspace*{-24pt}

\section{Заключение}

В настоящей работе была предложена модель механизма влияния по\-сту\-па\-ющих 
заказов на цену актива на основе физической модели абсолютно упругого 
соударения час\-тиц. 

Была установлена справедливость функциональной предельной 
тео\-ре\-мы, на основании результатов которой можно аппроксимировать процесс 
цены при интенсивном потоке приходящих заявок процессом Леви, приращения 
которого являются смесью нормальных законов и~поддаются более точному анализу. 
Такая аппроксимация дает также воз\-мож\-ность оценки риска динамических
 стратегий~\cite{six}.


%\vspace*{-48pt}

    {\small\frenchspacing
 {%\baselineskip=10.8pt
 \addcontentsline{toc}{section}{References}
 \begin{thebibliography}{9}
\bibitem{first} 
\Au{Kukanov A.} Stochastic models of limit order markets.~--- 
Columbia University, 2013. Ph.D. Thesis. 131~p.

\bibitem{second} 
\Au{Лаврентьев В.\,В., Назаров~Л.\,В.} 
Процесс движения цены, порожденный непрерывной моделью книги заказов~// 
Вестн. Тверского государственного ун-та. Сер. Прикладная математика, 2015. 
№~4. С.~55--63.

\bibitem{Korolev1}
\Au{Korolev V.\,Yu., Chertok~A.\,V., Korchagin~A.\,Yu, Zeifman~A.\,I.} 
Modeling high-frequency order flow imbalance by functional limit theorems for 
two-sided risk processes~// Appl. Math. Comput., 2015. Vol.~253. P.~224--241.

\bibitem{third} 
\Au{Сивухин Д.\,В.} Общий курс физики.~--- 
В~5 т.~--- Т.~1. Механика.~--- 4-е изд.~--- М.: МФТИ, 2005. 560~с.

\bibitem{fourth} 
\Au{Кащеев Д.\,Е.} Моделирование динамики финансовых временных рядов и~оценивание 
производных ценных бумаг: Дис.\ \ldots\ канд. физ.-мат. наук.~--- 
Тверь: ТвГУ, 2001. 191~c.

\bibitem{five} 
\Au{Биллингсли П.} Сходимость вероятностных мер~/
Пер. с~англ.~--- М.: Наука, 1977. 353~с.
(\Au{Billingsley~P.}  
{Convergence of probability measures}.~--- New York, NY, USA: John Wiley \& Sons, Inc., 
1977. 277~p.)

\bibitem{Korolev_FLT} 
\Au{Korolev V.\,Yu., Chertok~A.\,V., Korchagin~A.\,Yu, Kossova~E.\,V., Zeifman~A.\,I.} 
A~note on functional limit theorems for compound Cox processes~// 
J.~Math. Sci., 2016. Vol.~218. No.\,2. P.~182--194.

\bibitem{six} 
\Au{Balasanov~Y., Doynikov~A., Lavrent'ev~V., Nazarov~L.} 
Estimating risk of dynamic trading strategies from high frequency data flow~// 
Advances in data mining: Applications and theoretical aspects~/
 Ed.\ P.~Perner.~---
Lecture notes in computer science ser.~--- Springer, 2015.  
 Vol.~9165. P.~153--165.
 \end{thebibliography}

 }
 }

\end{multicols}

\vspace*{-3pt}

\hfill{\small\textit{Поступила в~редакцию 07.12.17}}

%\vspace*{6pt}

\newpage

\vspace*{-28pt}

%\hrule

%\vspace*{2pt}

%\hrule

%\vspace*{8pt}


\def\tit{A~PROBABILITY MODEL OF~THE~INFLUENCE\\ OF~THE~ORDER BOOK ON~THE~PRICE PROCESS}

\def\titkol{A probability model of the influence of the order book on the price process}

\def\aut{L.\,V.~Nazarov, V.\,V.~Lavrentyev, and~E.\,V.~Bykovets}

\def\autkol{L.\,V.~Nazarov, V.\,V.~Lavrentyev, and~E.\,V.~Bykovets}

\titel{\tit}{\aut}{\autkol}{\titkol}

\vspace*{-9pt}


\noindent
Faculty of Computational Mathematics and Cybernetics, 
M.\,V.~Lomonosov Moscow State University, 1-52~Leninskiye Gory, GSP-1, Moscow 119991, 
Russian Federation 


\def\leftfootline{\small{\textbf{\thepage}
\hfill INFORMATIKA I EE PRIMENENIYA~--- INFORMATICS AND
APPLICATIONS\ \ \ 2018\ \ \ volume~12\ \ \ issue\ 2}
}%
 \def\rightfootline{\small{INFORMATIKA I EE PRIMENENIYA~---
INFORMATICS AND APPLICATIONS\ \ \ 2018\ \ \ volume~12\ \ \ issue\ 2
\hfill \textbf{\thepage}}}

\vspace*{3pt} 
 


\Abste{The Limit Order Book model is considered, with buy and sell orders arriving 
as two independent Cox processes. It includes the price impact model built on the 
basis of a physical model of perfectly elastic collision. Price is treated as 
a~particle of some mass, moving along a~straight line without friction. The 
incoming buy orders and outgoing sell orders hit the price giving it additional 
momentum in one direction, while incoming sell orders and outgoing buy orders do 
the same in the opposite direction. A~functional limit theorem for the price 
process is obtained at a~high intensity 
of incoming order flow, which allows approximation by some L$\acute{\mbox{e}}$vy process}

\KWE{limit orders; perfectly elastic collision; limit order book model; 
price process; Cox process; functional limit theorem}

 
\DOI{10.14357/19922264180205} %

%\vspace*{-14pt}

  %\Ack
  % \noindent
  


%\vspace*{-3pt}

  \begin{multicols}{2}

\renewcommand{\bibname}{\protect\rmfamily References}
%\renewcommand{\bibname}{\large\protect\rm References}

{\small\frenchspacing
 {%\baselineskip=10.8pt
 \addcontentsline{toc}{section}{References}
 \begin{thebibliography}{9}

\bibitem{1-naz}
\Aue{Kukanov, A.} 2013. Stochastic models of limit order markets. 
Columbia University. Ph.D. Thesis.  131~p.

\bibitem{2-naz}
\Aue{Lavrent'ev, V.\,V., and L.\,V.~Nazarov.} 2015. Protsess dvizheniya tseny, 
porozhdennyy nepreryvnoy model'yu knigi zakazov 
[Price process, generated by the continuous model of the order book]. 
\textit{Vestnik Tverskogo gosudarstvennogo un-ta. Ser. 
Prikladnaya matematika} [Bull. of the Tverskoy State University. Ser. 
Appl. Math.] 4:55--63.

\bibitem{3-naz}
\Aue{Korolev, V.\,Yu., A.\,V.~Chertok, A.\,Yu.~Korchagin, and A.\,I.~Zeifman.} 
2015. Modeling high-frequency order flow imbalance by functional limit theorems
 for two-sided risk processes. \textit{Appl. Math. Comput.} 253:224--241.

\bibitem{4-naz}
\Aue{Sivukhin, D.\,V.} 2005. 
\textit{Obshchiy kurs fiziki. Mekhanika}
[General course of physics. Mechanics].
4~ed. Moscow: MIPT Publs. Vol.~1.  560~p. 

\bibitem{5-naz}
\Aue{Kashcheev, D.\,E.} 2001. Modelirovanie dinamiki finansovykh vremennykh ryadov
 i~otsenivanie proizvodnykh tsennykh bumag [Modeling of dynamics of financial time series and 
 estimation of derivative securities].  
 Tver'. PhD Thesis. 191~p.

\bibitem{6-naz}
\Aue{Billingsley, P.} 1977. 
\textit{Convergence of probability measures}. New York, NY: John Wiley \& Sons, Inc. 
277~p.

\bibitem{7-naz}
\Aue{Korolev, V.\,Yu., A.\,V.~Chertok, A.\,Yu.~Korchagin, E.\,V.~Kossova, 
and A.\,I.~Zeifman.} 2016. 
A~note on functional limit theorems for compound Cox processes. 
\textit{J.~Math. Sci.} 218(2):182--194. 

\bibitem{8-naz}
\Aue{Balasanov, Y., A.~Doynikov, V.~Lavrent'ev, and L.~Nazarov}. 
2015. Estimating risk of dynamic trading strategies from high frequency data flow.
\textit{Advances in data mining: Applications and theoretical aspects.} 
Ed.\ P.~Perner.  Lecture notes in computer science ser.  
Springer. 9165:153--165.
\end{thebibliography}

 }
 }

\end{multicols}

\vspace*{-3pt}

\hfill{\small\textit{Received December 7, 2017}}

%\vspace*{-24pt}




\Contr

\noindent
\textbf{Bykovets Eugene V.} (b.\ 1994)~--- MSc student,  
Faculty of Computational Mathematics and Cybernetics, M.\,V.~Lomonosov Moscow 
State University, 1-52~Leninskiye Gory, GSP-1, Moscow 119991, Russian Federation; 
\mbox{eugene.bykovets@stud.cs.msu.su}

\vspace*{3pt}

\noindent
\textbf{Lavrentyev Victor V.} (b.\ 1955)~---  
Candidate of Science (PhD) in physics and mathematics, scientist, 
Faculty of Computational Mathematics and Cybernetics, M.\,V.~Lomonosov Moscow 
State University, 1-52~Leninskiye Gory, GSP-1, Moscow 119991, Russian Federation; 
\mbox{lavrent@cs.msu.ru}

\vspace*{3pt}

\noindent
\textbf{Nazarov Leonid V.} (b.\ 1957)~--- 
Candidate of Science (PhD) in physics and mathematics, senior scientist, 
Faculty of Computational Mathematics and Cybernetics, M.\,V.~Lomonosov Moscow 
State University, 1-52~Leninskiye Gory, GSP-1, Moscow 119991, Russian Federation; 
\mbox{nazarov@cs.msu.ru}
\label{end\stat}


\renewcommand{\bibname}{\protect\rm Литература}  %5
\def\stat{lebedev}

\def\tit{МАКСИМУМЫ АКТИВНОСТИ В~БЕЗМАСШТАБНЫХ СЛУЧАЙНЫХ СЕТЯХ С~ТЯЖЕЛЫМИ ХВОСТАМИ$^*$}

\def\titkol{Максимумы активности в~безмасштабных случайных сетях с~тяжелыми хвостами}

\def\autkol{А.\,В.~Лебедев}
\def\aut{А.\,В.~Лебедев$^1$}

\titel{\tit}{\aut}{\autkol}{\titkol}

{\renewcommand{\thefootnote}{\fnsymbol{footnote}}\footnotetext[1]
{Работа выполнена при поддержке РФФИ, грант 11-01-00050.}}


\renewcommand{\thefootnote}{\arabic{footnote}}
\footnotetext[1]{Механико-математический факультет
    Московского государственного университета им.\ М.~В.~Ломоносова,
    avlebed@yandex.ru}
    
  %  \vspace*{-6pt}

\Abst{Рассматриваются ориентированные степенные случайные графы как
    модели информационных сетей, где каждый узел обладает случайной информационной
    активностью, распределение которой имеет тяжелый (правильно меняющийся)
    хвост. Используется модель случайного графа, в которой входящие
    степени вершин независимы и имеют распределение со степенным хвостом.
    Выведены достаточные условия, при которых максимум
    суммарных активностей (по узлу и его входящим соседям) растет асимптотически
    так же, как и максимум индивидуальных активностей, и в силу этого для них
    имеет место предельный закон Фреше.}

 %   \vspace*{-2pt}
    
    \KW{максимумы; случайные суммы; безмасштабные сети;
    степенной закон; случайный граф; тяжелый хвост; правильное изменение;
    распределение Фреше}
    
%                    \vspace*{-9pt}
    
     \vskip 14pt plus 9pt minus 6pt

      \thispagestyle{headings}

      \begin{multicols}{2}
      
            \label{st\stat}
            


\section{Введение}

    Степенными (\textit{power-law}) или безмасштабными (\textit{free-scale}) называют
    случайные графы, у которых степени вершин подчиняются асимптотически
    степенному закону (с вероятностями $p_k\hm\sim ck^{-\beta}$, $k\hm\to\infty$,
    $\beta\hm>1$).
    Активные исследования данного класса графов и их приложений
    в последнее десятилетие были инициированы работой~\cite{Bar},
    где приведен ряд интересных примеров (Интернет, электрическая сеть,
    социальная сеть киноактеров).
    С~тех пор были предложены и изучены различные модели степенных графов.
    В~одних степенной закон возникает благодаря некоторому случайному
    процессу~\cite{Bar, Komp}, в других он постулируется изначально~\cite{Pow, Reit}.
    Следует отметить, что некоторые асимптотические свойства графов при
    одинаковом распределении степеней вершин могут оказаться общими,
    а другие зависят от выбора модели.

    Степенной граф может служить моделью некоторой информационной сети.
    Например, исследования кириллического сегмента <<Живого журнала>>
    ({\sf livejournal.com})~\cite{Komp}
    показывают, что он хорошо описывается степенным графом с $\beta\hm\approx 3$.
    Пусть каждый узел этой сети обладает случайной информационной активностью
    (интенсивностью производства информации). Имеется в виду среднее количество
    информации, производимой узлом в единицу времени. Активность будем
    полагать индивидуальной характеристикой, присущей узлу. Например, речь может идти
    о пользователях, которые пишут сообщения в Интернет.
    Предположим, что активности узлов
    независимы и одинаково распределены, причем их распределение~$F$ имеет тяжелый
    (правильно меняющийся) хвост, т.\,е.\ ${\bar F}(x)\hm\sim x^{-a}L(x)$,
    $x\hm\to\infty$, $a>0$, где $L(x)$~--- медленно меняющаяся функция~\cite[\S\ 8.8]{Fel}. 
    Такое предположение находится в русле современных
    представлений о распространенности степенных законов в природе, технике и
    человеческой деятельности. Активности и степени вершин (узлов) для
    простоты будем полагать независимыми.

    Рассмотрим суммарную активность в узле (т.\,е.\ сумму его собственной и
    ближайших соседей). Например, в <<Живом журнале>>
    каждый пользователь может оставлять свои записи и читать записи
    своих друзей, объединяемые для удобства в общую <<ленту друзей>> (френдленту).
    Далее будем интересоваться вопросом: когда максимум суммарных активностей
    растет асимптотически так же, как и максимум индивидуальных
    активностей узлов? В~этом случае для максимумов легко выводится
    предельный закон Фреше
    $\Phi_a(x)\hm=\exp\{-x^{-a}\}$, $x\hm>0$~[5, \S\ 8.8; 6, \S~3.3.1].

    Для модели степенного графа, введенной в~\cite{Pow},
    этот вопрос был изучен автором в~\cite{Leb2}.
    Там число вершин степени~$k$ полагалось
    детерминированным и равным $\lfloor e^\alpha/k^\beta\rfloor$,
    $\alpha$, $\beta>0$, $1\hm\le k\hm\le e^{\alpha/\beta}$,
    а распределение на множестве графов,
    удовле\-тво\-ря\-ющих этому условию, равномерным.
    Были получены достаточные условия того,
    что максимум сумм с ростом числа узлов (при $\alpha\hm\to\infty$)
    растет асимптотически так же, как и макcимум
    индивидуальных активностей: $a\hm<\beta\hm-3/2$, если $3/2\hm<\beta\hm<3$, и
    $a\hm<\beta/2$, если $\beta\hm\ge 3$. При этом применялись
    ранее полученные автором результаты для
    общей схемы максимумов сумм независимых случайных величин~\cite{Leb1}.

    Рассмотрим теперь модель ориентированного
    случайного графа, где направления ребер соответствуют направлениям
    передачи информации. Пусть имеется $n$ вершин и заданы независимые
    неотрицательные целочисленные случайные величины $K_1,\dots, K_n$,
    имеющие одинаковое
    распределение, заданное вероятностями $p_k\hm\sim ck^{-\beta}$, $k\hm\to\infty$,
    $\beta>1$. Положим $D_i\hm=\min\{K_i,n-1\}$. Для $i$-й вершины выберем
    случайным образом (равновероятно и независимо от выбора для других
    вершин) $D_i$ различных вершин из числа остальных (кроме $i$-й) и
    выпустим из них ребра, направленные в \mbox{$i$-ю} вершину. Полученный в
    результате граф можно отнести к степенным в том смысле, что входящие
    степени вершин распределены асимптотически по степенному закону.
    Суммарной активностью в узле в данном случае будем считать сумму
    собственной активности узла и всех узлов, из которых в него
    поступает информация (его входящих соседей).

    К сожалению, метод, использованный в~\cite{Leb2}, здесь
    не работает при $\beta\hm<3$, так как второй момент
    входящей степени вершины тогда растет слишком быстро при $n\hm\to\infty$.
    Эта проблема решается с по\-мощью урезания.
    При этом получаются более сильные ограничения на
    параметры, что связано с более быст\-рым ростом максимальной (входящей)
    степени вершины в графе. Однако поскольку используются лишь
    достаточные условия, не исключено, что эти ограничения в
    будущем могут быть ослаблены.

    %Результаты \cite{Leb2} и настоящей работы были кратко изложены в \cite{Leb3}.

    Отметим, что асимптотическая эквивалентность хвостов распределений
    суммы и максимума конечного числа независимых одинаково распределенных
    случайных величин в случае тяжелых хвостов
    представляет собой давно известный факт~\cite[\S\ 8.8]{Fel},
    обусловленный тем, что основной вклад в сумму дает самое большое
    слагаемое (максимум), а сумма остальных слагаемых по сравнению с
    ним оказывается мала. Теперь обобщим это утверждение
    на модель, где имеется
    некоторый набор случайных сумм со случайными числами слагаемых
    и от сумм берется максимум. По-преж\-не\-му оказывается, что основной
    вклад (в одну или несколько сумм, а значит, и в их максимум) дает только
    одно, максимальное слагаемое. Однако для этого хвост распределения
    слагаемых должен быть достаточно тяжелым.

    Проверка наличия подобного эффекта в реальных сетях, разумеется,
    требует экспериментального исследования, выходящего
    за рамки данной работы, которая имеет теоретический характер.
    
        \vspace*{-9pt}

    
    \section{Основной результат}
    
    \vspace*{-2pt}

    Будем рассматривать сети из $n$ узлов, затем устремляя $n$ к бесконечности.
    Обозначим через $M(n)$ максимум суммарных активностей (самого узла и его
    входящих соседей), а через $M_0(n)$~--- максимум индивидуальных активностей
    узлов. Требуется определить условия, при которых
    \begin{equation}
    \label{MP}
\fr{M(n)}{M_0(n)}\stackrel{P}{\to} 1\,,\quad n\to\infty\,.
    \end{equation}

    Введем неотрицательную функцию
    $u(s)$ такую, что $s{\bar F}(u(s))\hm\to 1$, $s\hm\to\infty$.
    Заметим, что $u(s)$ заведомо существует и правильно
    меняется с показателем $1/a$, т.\,е.\ $u(s)\hm\sim s^{1/a}L_2(s)$,
    $s\to\infty$, где $L_2(s)$~--- медленно меняющаяся функция~\cite[\S\ 1.5]{Sen}.

    Тогда имеет место предельный закон для максимумов независимых случайных
    величин в случае правильно меняющихся хвостов~[5, \S~8.8; 6, \S~3.3.1]:
    $$
    \lim\limits_{n\to\infty}{\bf P}\left(\fr{M_0(n)}{u(n)}\le x\right)=\Phi_a(x)\,,\quad x>0\,,
    $$
    что в сочетании с~(\ref{MP}) дает
    \begin{equation*}
%    \label{PZ}
    \lim\limits_{n\to\infty}{\bf P}\left(\fr{M(n)}{u(n)}\le x\right)=\Phi_a(x)\,,\quad x>0\,.
    \end{equation*}
    %В этом и заключается польза соотношения (\ref{MP}).

\smallskip

\noindent
\textbf{Теорема 1.} \textit{Соотношение}~(\ref{MP}) \textit{выполняется при
    $a\hm<\beta-2$, если $2\hm<\beta\hm<3$, и при $a\hm<(\beta-1)/2$,
    если $\beta\hm\ge 3$.}

    
    \smallskip
    
    \noindent
    Д\,о\,к\,а\,з\,а\,т\,е\,л\,ь\,с\,т\,в\,о\ теоремы будет приведено в разд.~4.

    
    \section{Общая схема максимумов сумм}

    Напомним введенную в~\cite{Leb1} схему (немного изменив обозначения).
    Пусть заданы случайный процесс $\Upsilon(t)$, $t\hm\in T$,
    значениями которого являются конечные
    классы конечных подмножеств ${\bf N}$,
    и семейство $\Xi\hm=\{\xi_{i,t},i\in {\bf N},t\in T\}$
    неотрицательных случайных величин, независимых и одинаково распределенных
    при любом фиксированном значении параметра $t\hm\in T$.
    Полагаем, что $\Upsilon$ и~$\Xi$ независимы.

    Для любых $A\subset {\bf N}$, $t\hm\in T$
    обозначим максимум набора случайных величин $\{\xi_{i,t}$, $i\hm\in A\}$
    через $M_t(A)$, $r$-й максимум (т.\,е.\ чис\-ло, стоящее $r$-м с конца в
    вариационном ряду)~--- через $M_t^{(r)}$, сумму~--- через $S_t(A)$.
    Пусть 
    $$
    U(t)\hm=\bigcup\limits_{A\in\Upsilon(t)}A\,.
    $$

    Введем случайные процессы, порожденные $\Upsilon$ и~$\Xi$:
    \begin{gather*}
    Z(t)=\!\!\sup\limits_{A\in\Upsilon(t)}\!\!S_t(A)\,;\enskip
    \kappa(t)=\!\!\sup\limits_{A\in\Upsilon(t)}\!\!|A|\,;
    \enskip\nu(t)=\left|U(t)\right|\,;\\
    \mu_1(t)=M_t(U(t))\,;\quad \mu_r(t)=M_t^{(r)}(U(t))\,,
 \end{gather*}
    где через $|A|$ обозначен размер (число элементов) множества~$A$.
    Через $|\Upsilon(t)|$ обозначим число различных множеств $A\hm\in\Upsilon(t)$.

    Предполагается, что $\nu(t)<\infty$ почти наверное (п.\,н.)\
    при всех $t\in T$, откуда следует конечность п.\,н.\
    всех процессов, введенных выше.

    Рассмотрим предельное поведение $Z(t)$ при $t\hm\to\infty$.

    Пусть существует случайный процесс $\rho(t)$ со
    значениями в ${\bf Z}_+$, измеримый относительно~$\Upsilon$ и такой, что
    $\rho(t)\hm\ge 1$ при $\nu(t)\hm\ge 1$, $\rho(t)\hm\le\nu(t)$ п.\,н. при всех $t\hm\in T$.

    Обозначим через $\pi(t)$ вероятность того, что для
    множества~$B$, равновероятно выбранного среди всех подмножеств $U(t)$,
    состоящих из $\rho(t)$ элементов,
    имеет место $\sup\limits_{A\in\Upsilon(t)}|A\cap B|\hm>1$.

\smallskip

\noindent
\textbf{Теорема I.} \textit{Пусть выполнены условия
    \begin{align}
    \label{us1}
    (\kappa(t)-1)\fr{\mu_{\rho(t)}(t)}{\mu_1(t)}\stackrel{P}{\to} 0,\quad t\to\infty\,;
\\
\label{us2}
    \pi(t)\to 0\,,\quad t\to\infty\,,
    \end{align}
    тогда
    \begin{equation}
    \label{res1}
    \fr{Z(t)}{\mu_1(t)}\stackrel{P}{\to} 1\,,\quad t\to\infty\,.
    \end{equation}
    }

    Доказано также следующее свойство порядковых статистик
    в случае правильно меняющихся хвостов.
    Пусть $X_n$, $n\ge 1$, независимы и имеют распределение~$F$
    с правильно меняющимся хвос\-том ${\bar F}(x)\hm\sim x^{-a}L(x)$,
    $x\hm\to\infty$, $a\hm>0$.
    Обозначим максимум
    $X_1,\dots, X_n$ через ${\tilde X}_n$ и $r$-й максимум через
    ${\tilde X}^{(r)}_n$.

\smallskip

\noindent
\textbf{Следствие II.}  \textit{Пусть $r_n\hm\sim n^\gamma$, $n\hm\to\infty$, $0\hm<\gamma\hm<1$ и
    $0\hm<\delta\hm<\gamma/a$, тогда
    $$
    \fr{n^\delta {\tilde X}^{(r_n)}_n}{{\tilde X}_n}\stackrel{P}{\to} 0\,,\quad
    n\to\infty\,.
    $$}
    
\vspace*{-12pt}

    
\section{Приложение к случайным графам}

    Адаптируем общую схему к изучению случайных графов.
    Рассмотрим процесс~$\Upsilon$ с дискретным временем, соответствующим
    числу узлов~$n$.
    Случайные величины $\xi_{i,n}$, $1\hm\le i\hm\le n$, описывают
    информационные активности узлов. Обозначим через
    $A_i$ множество из индекса~$i$ и индексов входящих соседей
    $i$-го узла, тогда набор множеств~$A$ получается из набора
    $A_1,\dots, A_n$ удалением повторов (если они есть). Имеем
    $|A_i|\hm=D_i\hm+1$ и $\kappa(n)\hm=\max\limits_{1\le i\le n}D_i\hm+1$.
    Очевидно, $|\Upsilon(n)|\hm\le n$ и $\nu(n)\hm=n$.
    Последовательность $\rho(n)$ далее будем полагать детерминированной.
    Кроме того, в используемых обозначениях $M(n)\hm=Z(n)$, $M_0(n)\hm=\mu_1(n)$ 
    и~(\ref{res1}) эквивалентно~(\ref{MP}).

    Обозначим
    $$
    Q(n,m)=n{\bf M}\left((D+1)D{\bf I}\{D\le m-1\}\right)\,,
    $$
    где $D\stackrel{d}{=}D_1$.

\medskip

\noindent
\textbf{Лемма 1.} \textit{При $n>2$ и $\rho(n)<n$ верно неравенство}
    $$
    \pi(n)\le\fr{\rho(n)(\rho(n)-1)}{2(n-1)(n-2)}\,Q(n,m)+
    {\bf P}(\kappa(n)>m)\,.
    $$

    \smallskip
    
    \noindent
    Д\,о\,к\,а\,з\,а\,т\,е\,л\,ь\,с\,т\,в\,о\,.\
    Событие $\{\sup\limits_{A\in\Upsilon(t)}|A\cap B|\hm>1\}$ представляет собой
    объединение событий $\{|A_i\cap B|\hm>1\}$, $1\hm\le i\hm\le n$.
    Пусть для простоты $B\hm=\{1,2,\dots, \rho(n)\}$ (в противном
    случае можно перенумеровать~$A_i$). Зафиксируем входящие степени
    вершин $d_1,\dots, d_n$. Тогда для $1\hm\le i\hm\le \rho(n)$ один элемент
    множества~$B$ заведомо принадлежит~$A_i$ (а именно, индекс~$i$),
    а любой другой принадлежит с вероятностью $d_i/(n-1)$. Для
    $\rho(n)+1\hm\le i\hm\le n$ любая пара индексов из~$B$ принадлежит~$A_i$
    с ве\-ро\-ят\-ностью $d_i(d_i-1)/((n-1)(n-2))$, а всего таких пар
    $\rho(n)(\rho(n)-1)/2$. Суммируя вероятности, получаем оценку сверху:
    $$
    \fr{\rho(n)-1}{n-1}\sum\limits_{i=1}^{\rho(n)}d_i+
    \fr{\rho(n)(\rho(n)-1)}{2(n-1)(n-2)}\sum\limits_{i=\rho(n)+1}^nd_i(d_i-1)\,.
    $$
    Обозначим 
    \begin{align*}
    q_1&={\bf M}(D{\bf I}\{D\le m-1\})\,;\\
q_2&={\bf M}(D(D-1){\bf I}\{D\le m-1\})\,.
\end{align*}

Усредняя по наборам входящих
    степеней вершин в области $\kappa(n)\hm\le m$, получаем оценку сверху:
\begin{multline*}
    \fr{\rho(n)-1}{n-1}\,\rho(n)q_1+
    \fr{\rho(n)(\rho(n)-1)}{2(n-1)(n-2)}\left(n-\rho(n)\right)q_2\le{}\\
{}\le\fr{\rho(n)(\rho(n)-1)}{2(n-1)(n-2)}\,n\left(2q_1+q_2\right)={}\\
{}=
    \fr{\rho(n)(\rho(n)-1)}{2(n-1)(n-2)}\,Q(n,m)\,.
\end{multline*}
    Учитывая также вероятность события $\{\kappa(n)>m\}$, получаем
    утверждение леммы.

\medskip

\noindent
\textbf{Лемма 2.} \textit{Пусть выполнены следующие условия:}
\begin{enumerate}[(1)]
\item \textit{все $\xi_{i,n}$ имеют одинаковое распределение $F$ на~${\bf R}_+$
    с хвостом ${\bar F}(x)\sim x^{-a}L(x)$, $x\to\infty$, $a>0$};
\item
    $m\sim n^\delta$, $n\to\infty$, $\delta>0$;
\item
    $Q(n,m)=O(n^b)$, $n\to\infty$,   $0<b<2$;
\item
    $\kappa(n)=o_p(m)$, $n\to\infty$;
\item
    $a<(2-b)/(2\delta)$.
    \end{enumerate}
    \textit{Тогда верно}~(\ref{MP}).

    
    \medskip
    
    \noindent
    Д\,о\,к\,а\,з\,а\,т\,е\,л\,ь\,с\,т\,в\,о\,.\
    Можно выбрать $\gamma\hm\in (0,1)$
    так, чтобы выполнялось неравенство $a\delta\hm<\gamma\hm<(2-b)/2$.
    Положим $\rho(n)\hm=[n^\gamma]$, тогда
    по следствию~II получаем~(\ref{us1}), а по лемме~1~---~(\ref{us2}),
    так что условия теоремы~I выполняются и верно соотношение~(\ref{res1}), 
    эквивалентное~(\ref{MP}).

\smallskip

\noindent
Д\,о\,к\,а\,з\,а\,т\,е\,л\,ь\,с\,т\,в\,о\ теоремы~1.\
    Поскольку $p_k\hm\sim ck^{-\beta}$,
    $k\hm\to\infty$, то хвост распределения имеет
    асимптотику ${\bar F}_K(k)\hm\sim c_1k^{-(\beta-1)}$.
    Отсюда получаем
    $\kappa(n)\hm=O_p(n^{1/(\beta-1)})$, $n\hm\to\infty$,
    и, следовательно, $\kappa(n)\hm\sim o_p(m)$ при любом
    $\delta\hm=(1+\varepsilon)/(\beta-1)$, $\varepsilon\hm>0$. Имеем
    \begin{multline*}
    Q(n,m)=n\sum\limits_{k=1}^{m-1}(k+1)kp_k={}\\
    {}=
\begin{cases}
    O(n^{1+\delta(3-\beta)})\,,& 1<\beta<3\,;\\
    O(n\ln n)\,,& \beta=3\,;\\
    O(n)\,, & \beta>3\,.
    \end{cases}
    \end{multline*}
    При $2<\beta<3$ применяем лемму 2 с $b\hm=2\delta\hm>1\hm+\delta(3\hm-\beta)$ и,
    устремляя~$\varepsilon$ к нулю,
    получаем достаточное условие $a\hm<\beta\hm-2$ для выполнения~(\ref{MP}).
    При $\beta\hm\ge 3$ применяем лемму 2 с $b\hm=1\hm+\varepsilon$ и
    аналогично получаем достаточное условие $a\hm<(\beta-1)/2$.
    
    {\small\frenchspacing
{%\baselineskip=10.8pt
\addcontentsline{toc}{section}{Литература}
\begin{thebibliography}{9}


    \bibitem{Bar}
    \Au{Barab\'asi A., Albert R.} Emergence of scaling in random networks~//
    Science, 1999. Vol.~286. P.~509--512.

    \bibitem{Komp}
    \Au{Захаров П.} Народ-бло\-го\-но\-сец~// Компьютерра,
    2007. №\,27-28. C.~36--39. {\sf http://offline.computerra.ru/2007/ 695/327726}.

    \bibitem{Pow}
    \Au{Aiello W., Chung F., Lu~L.} A random graph model for power law
    graphs~// Experimental Math., 2001. Vol.~10. No.~1. P.~53--66.

    \bibitem{Reit}
    \Au{Reittu H., Norros~I.} On the power-law random graph model
    of massive data network~// Performance Evaluation, 2004. Vol.~55. P.~3--23.

    \bibitem{Fel}
    \Au{Феллер В.} Введение в теорию вероятностей и ее приложения. Т.~2.~---
    М.: Мир, 1984.

    \bibitem{EKM}
    \Au{Embrechts P., Kl$\ddot{\mbox{u}}$ppelberg C., Mikosh~T.} Modelling
    extremal events for insurance and finance.~--- Springer-Verlag, 2003.

    \bibitem{Leb2}
    \Au{Лебедев А.\,В.} Максимумы активности в случайных сетях в
    случае тяжелых хвостов~// Проблемы передачи информации, 2008. Т.~44.
    №\,2. С.~96--100.

    \bibitem{Leb1}
\Au{Лебедев А.\,В.} Общая схема максимумов сумм независимых
    случайных величин и ее приложения~// Математические заметки, 2005.
    Т.~77. №\,4. С.~544--550.

\label{end\stat}

    \bibitem{Sen}
\Au{Сенета Е.} Правильно меняющиеся функции.~--- М.: Наука, 1985.
 \end{thebibliography}
}
}


\end{multicols}         %6
\def\stat{grusho}

\def\tit{АРХИТЕКТУРНЫЕ РЕШЕНИЯ В~ЗАДАЧЕ ВЫЯВЛЕНИЯ МОШЕННИЧЕСТВА ПРИ~АНАЛИЗЕ 
ИНФОРМАЦИОННЫХ ПОТОКОВ В~ЦИФРОВОЙ ЭКОНОМИКЕ$^*$}

\def\titkol{Архитектурные решения в~задаче выявления мошенничества при~анализе 
информационных потоков в
%~цифровой 
экономике}

\def\aut{А.\,А.~Грушо$^1$, М.\,И.~Забежайло$^2$, Н.\,А.~Грушо$^3$, 
Е.\,Е.~Тимонина$^4$}

\def\autkol{А.\,А.~Грушо, М.\,И.~Забежайло, Н.\,А.~Грушо, 
Е.\,Е.~Тимонина}

\titel{\tit}{\aut}{\autkol}{\titkol}

\index{Грушо А.\,А.}
\index{Забежайло М.\,И.}
\index{Грушо Н.\,А.}
\index{Тимонина Е.\,Е.}
\index{Grusho A.\,A.}
\index{Zabezhailo M.\,I.}
\index{Grusho N.\,A.}
\index{Timonina E.\,E.}


{\renewcommand{\thefootnote}{\fnsymbol{footnote}} \footnotetext[1]
{Работа частично поддержана РФФИ (проекты 18-29-03081 и~18-07-00274).}}


\renewcommand{\thefootnote}{\arabic{footnote}}
\footnotetext[1]{Институт проблем информатики Федерального исследовательского центра <<Информатика и~управление>> 
Российской академии наук, grusho@yandex.ru}
\footnotetext[2]{Институт проблем информатики Федерального исследовательского центра <<Информатика и~управление>> 
Российской академии наук, m.zabezhailo@yandex.ru}
\footnotetext[3]{Институт проблем информатики Федерального исследовательского центра <<Информатика и~управление>> 
Российской академии наук, info@itake.ru}
\footnotetext[4]{Институт проблем информатики Федерального исследовательского центра <<Информатика и~управление>> 
Российской академии наук, eltimon@yandex.ru}

\vspace*{-12pt}
   

 
  
  \Abst{Cформулирован подход к~исследованию некоторых видов мошенничества в~цифровой 
экономике с~использованием причинно-следственных связей. Во всех видах рассматриваемых 
мошенничеств должно наблюдаться несоответствие между целями финансовых транзакций 
и~реальной стоимостью достижения этих целей. Данные о транзакциях можно собирать, 
наблюдая информационные потоки, в~которых отражаются эти транзакции. Архитектура сбора 
данных и~их анализа может быть организована с~помощью распределенных реестров 
с~централизованным консенсусом, что позволяет создать аналог электронной бухгалтерской 
книги, фиксирующей финансово-экономическую деятельность субъектов цифровой экономики в~регионе. 
  Рассматриваемые методы выявления мошенничества основаны на противоречиях 
между действиями, описанными в~транзакциях, и~информацией, содержащейся в~планах, 
стандартах, прецедентах и~др. Рассмотрен метод, основанный на некоторой упрощенной схеме 
реализации абстрактного проекта. Для выявления противоречий необходимо проводить анализ 
от следствия к~причине, т.\,е.\ искать аномалии в~информации, описывающей порождение 
наблюдаемых следствий. 
  Показано, как в~реализации проекта можно выделять простые <<необходимые условия>> 
нарушения при\-чин\-но-след\-ст\-вен\-ных связей, т.\,е.\ множество <<необходимых условий>>, 
нарушение которых свидетельствует о наличии мошенничества. Это множество <<необходимых 
условий>> можно назвать метаданными для контроля проекта на выявление мошенничества.} 
 
 
  \KW{цифровая экономика; информационные потоки; при\-чин\-но-след\-ст\-вен\-ные связи; 
выявление мошеннических схем} 

\DOI{10.14357/19922264190204}
  
\vspace*{-4pt}


\vskip 10pt plus 9pt minus 6pt

\thispagestyle{headings}

\begin{multicols}{2}

\label{st\stat}

\section{Введение}

\vspace*{3pt}

  В работе сформулирован подход к~исследованию некоторых видов 
мошенничества в~цифровой экономике с~использованием  
при\-чин\-но-след\-ст\-вен\-ных связей. Рассматриваются три вида мошенничества, 
а именно:
  \begin{enumerate}[(1)]
\item отмыв денег; 
\item обман при выполнении договорных обязательств при реализации 
технических проектов (строительные проекты и~др.); 
\item незаконный вывод денег. 
\end{enumerate}

  Названные виды мошенничества могут быть сведены к~решению одного типа 
задач. Для отмывания денег источник должен заключать фиктивные контракты, 
в~соответствии с~которыми будут переводиться средства за заведомо ненужную 
работу и~материалы. 
  
  Мошенничество, связанное с~невыполнением договорных обязательств, связано 
со снижением качества услуг, качества и~количества закупаемых 
материалов, выполнением работ с~ненадлежащим качеством. 
  
  Вывод денег связан с~переводом средств фир\-мам-од\-но\-днев\-кам, которые 
заведомо не могут выполнить обязательства по контрактам, за которые им 
переводятся средства. 
  
  Таким образом, во всех трех видах рассматриваемых мошенничеств должно 
наблюдаться несоответствие между целями финансовых транзакций и~реальной 
стоимостью достижения этих целей. Данные о транзакциях можно собирать, 
наблюдая информационные потоки, в~которых отражаются эти транзакции. 
  
  Однако для наблюдения таких информационных потоков необходимо создавать 
архитектуру\linebreak телекоммуникационной системы, позволяющей перехватывать 
и~собирать данные о всех транзакциях. Например, такая архитектура может быть 
организована с~помощью распределенных реестров с~централизованным 
консенсусом, т.\,е.\ все информационные потоки, сформированные в~цифровой 
экономике и~несущие информацию о транзакциях, проходят через некоторый 
центральный узел, запоминающий их в~форме распределенного реестра. Такие 
реестры могут дублироваться в~аналогичных центрах различных регионов, что 
позволяет создать аналог электронной бухгалтерской книги, фиксирующей 
фи\-нан\-со\-во-эко\-но\-ми\-че\-скую деятельность субъектов цифровой экономики. Такой 
подход предложено реализовать на базе системы ситуационных центров, что 
отражено в~работах~[1, 2].
  
  Собранная из информационных потоков информация о~транзакциях, т.\,е.\ 
о~контрактах, договорах, платежах, отчетах, закупленных материалах, 
характеристиках исполнителей работ и~др., собирается в~базе данных в~указанном 
центре. Согласно теории интеллектуальных сис\-тем~[3], эту базу данных можно 
называть базой фактов (БФ). Базу фактов можно представить как бинарную мат\-ри\-цу, 
строки которой описывают характеристики, входящие в~транзакции, а столбцы 
нумеруются характеристиками. Строки матрицы будем называть 
\textit{объектами}~[4, 5]. 
  
  Рассматриваемые в~работе методы выявления мошенничества будут основаны 
на противоречиях между действиями, описанными в~транзакциях, и~информацией, 
содержащейся в~планах, стандартах, прецедентах и~др. Для нахождения 
противоречий в~архитектуре центра предусмотрена другая база данных~--- база 
знаний (БЗ)~\cite{3-gr, 6-gr}, которая устроена так же, как БФ. 
  
  Информация в~БЗ собирается на основе положительного опыта или расчетов. 
Используя БЗ, можно выводить факты нарушения при\-чин\-но-след\-ст\-вен\-ных 
связей. Нарушения при\-чин\-но-след\-ст\-вен\-ных связей будем называть 
\textit{аномалиями}. 
  
  Для упрощения дальнейшее изложение будет вестись в~рамках поиска 
противоречий при выполнении некоторого абстрактного проекта. Выявление 
аномалий будет происходить на основе фактов из БФ с~помощью знаний из БЗ 
методами искусственного интеллекта и~интеллектуального анализа 
данных~\cite{6-gr}. 

\vspace*{-10pt}
  
  \section{Модели}
  
  \vspace*{-3pt}
  
  Наиболее сложная из рассмотренных выше задач~--- выявление противоречий, 
т.\,е.\ использование БЗ для получения новых знаний и~выявление аномалий из 
полученных фактов. 
  
  Все способы выявления противоречий основаны на определении 
  причинно-следственных связей. При этом противоречия в~параметрах транзакций по 
отношению к~требуемым в~БЗ составляют сущность аномалий. 
  
   Далее будет рассмотрен метод, основанный на некоторой упрощенной схеме 
реализации абстрактного проекта. 
  
  Каждый проект имеет цель: например, цель представляет собой построение 
некоторой системы. Воспользуемся структурным подходом, который позволяет 
строить проект на основе разбиения системы на подсистемы и~определения 
взаимодействий подсистем~\cite{7-gr}. При этом каждая подсистема также 
представима структурной моделью. 
  
  Как сама система, так и~каждая ее подсистема имеют свой функционал 
и~спецификацию, па\-ра\-мет\-ры настройки и~домены параметров настройки. Кроме 
этих характеристик существует множество характеристик, связанных 
с~<<жизненным циклом>> создания системы. Сюда входят работы, ресурсы, 
сроки выполнения работ по созданию подсистем и~самой системы, стоимости 
компонентов и~материалов, стоимости работ, схемы поставок, договорные 
обязательства и~др. Все характеристики связаны между собой, поэтому можно 
говорить о стоимости и~времени изготовления структурных компонентов системы. 
  
  Одной из важнейших характеристик является смета (система смет для 
подсистем). Смета сопоставляет каждому компоненту системы стоимость его 
изготовления и~настройки. 
  
  Схема построения системы может быть пред\-став\-ле\-на диаграммой, 
изображенной на рис.~1. 

{ \begin{center}  %fig1
 \vspace*{9pt}
   \mbox{%
 \epsfxsize=79mm 
 \epsfbox{gru-1.eps}
 }


\vspace*{9pt}


\noindent
{{\figurename~1}\ \ \small{Диаграмма достижения цели}}
\end{center}
}

\vspace*{9pt}

\addtocounter{figure}{1}
  
  


  Представленная на рис.~1 диаграмма позволяет описать основные классы 
возможных противоречий при достижении цели. Противоречия возникают, когда 
данные БФ не соответствуют требуемым характеристикам. 
  
  
  \section{Потенциальные классы аномалий при~достижении цели}
  
  Выделим четыре потенциальных класса противоречий, которые показывают, 
каким образом нужно искать эти противоречия.
  
 
  Противоречие цели и~проекта (рис.~2) возникает при отсутствии обоснования 
или в~случае логического противоречия между возможностями проектируемого 
функционала и~целью системы. Отметим, что в~проект входят сроки, перечень 
работ, материалы, настройки, которые описываются соответствующими 
параметрами и~допустимыми значениями этих параметров. Проект формируется 
на основе БЗ и~расчетов, исходя из информации, полученной по аналогии 
с~другими проектами и~решениями, которые считаются апробированными. 
  
  Отметим, что цель порождает проект и~в этом смысле является причиной 
проекта. Однако для анализа противоречий необходимо двигаться по штриховой 
стрелке диаграммы (см.\ рис.~2) от проекта к~цели. В~самом деле, любой компонент 
проекта направлен на теоретическое достижение цели. Цель~--- сложный объект, 
поэтому в~проекте могут возникнуть характеристики, противоречащие хотя бы 
некоторым характеристикам цели. Это делает проект противоречивым, но вывод 
об этом может быть сделан только на уровне описания цели. 
  

  Противоречия между проектом и~его реализацией, исключая настройки 
(рис.~3), могут возникать, например, при закупке исполнителем материалов более 
низкого качества по более низким ценам, при попытках достижения требуемых 
сроков работы за счет снижения качества выполнения работ, за счет нахождения 
<<объективных>> причин для увеличения сроков работы и,~следовательно, 
увеличения цены реализации проекта. 


  Для выявления указанных противоречий необходимо двигаться по диаграмме 
(см.\ рис.~3) в~обратную сторону в~соответствии со~штриховыми стрелками. 
Действительно, выявить противоречия между характеристиками закупленных 
материалов и~требуемыми по проекту можно только при обращении к~проекту 
и~его спецификациям. Манипуляции со сроками работы также можно выявить 
только при обращении к~соответствующим расчетам в~проекте. Задержки в~сроках 
работы, связанные с~поставками материалов, можно определить только на 
предыдущем этапе диаграммы (см.\ рис.~3) в~описании проекта. 


  


  Противоречия между реализацией проекта и~его настройкой (рис.~4) возникает, 
когда не удается добиться требуемых значений параметров функционала, не 
удается обеспечить необходимый уровень\linebreak\vspace*{-12pt}

{ \begin{center}  %fig2
 \vspace*{-6pt}
   \mbox{%
 \epsfxsize=16mm 
 \epsfbox{gru-2.eps}
 }


\vspace*{6pt}


\noindent
{{\figurename~2}\ \ \small{Противоречия цели и~проекта}}
\end{center}
}

%\vspace*{9pt}

\addtocounter{figure}{1}

{ \begin{center}  %fig3
 \vspace*{6pt}
    \mbox{%
 \epsfxsize=79mm 
 \epsfbox{gru-3.eps}
 }


\end{center}

\vspace*{-2pt}


\noindent
{{\figurename~3}\ \ \small{Противоречия проекта и~его реализации (без настройки)}}
}

\vspace*{6pt}

\addtocounter{figure}{1}

{ \begin{center}  %fig4
 \vspace*{1pt}
   \mbox{%
 \epsfxsize=54.5mm 
 \epsfbox{gru-4.eps}
 }


\end{center}


\noindent
{{\figurename~4}\ \ \small{Противоречия реализации проекта и~его на\-стройки}}
}

%\vspace*{9pt}

\addtocounter{figure}{1}

{ \begin{center}  %fig5
 \vspace*{5pt}
    \mbox{%
 \epsfxsize=79mm 
 \epsfbox{gru-5.eps}
 }


\end{center}



\noindent
{{\figurename~5}\ \ \small{Противоречия цели и~достигнутой реализации проекта}}
}

\vspace*{6pt}

\addtocounter{figure}{1}

\noindent
 качества реализации проекта. Для 
определения противоречия в~настройках надо опять же двигаться по диаграмме 
(см.\ рис.~4) в~обратную сторону по штриховым стрелкам, так как для выявления 
характеристик результатов работы, которые не дают возможности реализации 
определенного функционала, необходимо иметь информацию о результатах этой 
работы. 


  



  Противоречие между целью и~достигнутой реализацией проекта (рис.~5) 
возникает, когда реализованная система не позволяет достичь цели. В~этом случае 
опять противоречие нужно искать, двигаясь от цели к~реальному достигнутому 
функционалу по штриховой стрелке (см.\ рис.~5).
  
  Суммируя положения, изложенные в~данном разделе, приходим к~выводу, что 
для выявления противоречий необходимо проводить анализ от следствия 
к~причине, т.\,е.\ искать аномалии в~информации, описывающей порождение 
наблюдаемых следствий. 
  
  
  \section{Связь противоречий и~причин}
  
  Прежде чем построить связь между причинами и~противоречиями, кратко 
опишем простейшую модель связи этих понятий. Причины и~противоречия будут 
сформулированы для представления компонентов системы как объектов, 
обладающих наборами известных характеристик~\cite{4-gr, 5-gr}. 
  
  Пусть $U\hm=\{\alpha, \beta, \ldots\}$~--- совокупность характеристик 
(пространство характеристик). Согласно~\cite{4-gr} \textit{объектом}~$O$ 
называется любое подмножество характеристик $O\hm\subseteq U$. Рассмотрим 
последовательность объектов, возможно в~различных пространствах 
характеристик. 
  
  \smallskip
  
  \noindent
  \textbf{Определение~1.}\ Объект~$P$ с~числом характеристик, большим или 
равным~2, является \textit{причиной} объекта (\textit{свойства})~$B$ в~цепочке 
наблюдаемых объектов тогда и~только тогда, когда выполнены следующие 
условия:
  \begin{enumerate}[(1)]
\item для каждого объекта~$C$, если $P\hm\subseteq C$, то $C\mapsto B$, где 
$C\mapsto B$ означает, что объект~$B$ присутствует в~объекте, следующем за 
объектом~$C$;
\item объект~$P$ является минимальным объектом, удовлетворяющим 
условию~1, а~именно: $\forall \alpha\hm\in P$ объект~$P\backslash \{\alpha\}$ 
не является причиной, т.\,е.\ $\exists C:\ \alpha\not\in C$, $P\backslash 
\{\alpha\}\hm\subseteq C$ и~$C\not\mapsto B$, где $C\not\mapsto B$ означает, 
что~$B$ не может содержаться в~объекте, следующем за объектом~$C$. 
\end{enumerate}

  Приведенное определение причины является упрощением причин, 
возникающих в~реальном мире. Например, реальные причины могут возникать\linebreak 
как совокупность характеристик из разных пространств. Одно следствие может 
порождаться разными причинами или возникать из внешних\linebreak и~ненаблюдаемых 
характеристик. Однако пред\-став\-лен\-ная далее формализация позволяет доступно 
изложить при\-чин\-но-след\-ст\-вен\-ные истоки противоречий, которые 
инициируют в~дальнейшем глубокое исследование рассматриваемых процессов.
  
  Будем считать, что для любого интересующего нас свойства~$B$ существует 
причина. Тогда справедлива следующая теорема.
  
  \smallskip
  
  \noindent
  \textbf{Теорема~1.}\ \textit{Для любого свойства~$B$ существует 
единственная причина}. 
  
  \smallskip
  
  \noindent
  Д\,о\,к\,а\,з\,а\,т\,е\,л\,ь\,с\,т\,в\,о\,.\ \ Доказательство будем вести от противного, 
т.\,е.\ предположим, что существуют две причины свойства~$B$: $P$ 
и~$P^\prime$, $P\hm\not= P^\prime$. Тогда существует $\alpha\hm\in U$, которое 
удовлетворяет одному из двух условий:
  \begin{itemize}
\item[(а)] $\alpha\in P$, $\alpha\notin P^\prime$;
\item[(б)] $\alpha\notin P$, $\alpha \in P^\prime$.
\end{itemize}

  Пусть выполняется условие~(б). Тогда $P^\prime\backslash \{\alpha\}$ не 
является причиной по условию~2 определения~1, т.\,е.\ $\exists C$ такое, что 
$\alpha\notin C$, $P^\prime\backslash \{\alpha\}\hm\subseteq C$ и~$C\not\mapsto B$. 
Но если~$B$ произошло и~$P$ его причина, то $C\mapsto B$, что противоречит 
предположению. Теорема~1 доказана.
  
  \smallskip
  
  \noindent
  \textbf{Лемма.} \textit{Если $P$~--- причина появления свойства~$B$, то 
объект~$B$ определяет существование свойства~$P$ в~объекте, 
предшествующем~$B$. }
  
  \smallskip
  
  \noindent
  Д\,о\,к\,а\,з\,а\,т\,е\,л\,ь\,с\,т\,в\,о\,.\ \ Из предположения, что у~каж\-до\-го 
свойства~$B$ есть причина, и~условия, что~$P$ является причиной~$B$, следует, 
что при появлении в~данных свойства~$B$ объект~$C$, предшествующий 
появлению~$B$, содержит как часть объект~$P$. Это следует из теоремы~1 
и~определения причины. 
  
  Докажем принцип <<необходимого условия>>, который, несмотря на простоту 
доказательства, будет играть в~дальнейшем существенную роль.
  
  \smallskip
  
  \noindent
  \textbf{Теорема~2.} \textit{Если~$P$~--- причина появления свойства~$B$ 
и~$A\hm\subseteq P$, то объект~$B$ определяет наличие свойства~$A$ 
в~объекте, предшествующем~$B$}. 
  
  \smallskip
  
  \noindent
  Д\,о\,к\,а\,з\,а\,т\,е\,л\,ь\,с\,т\,в\,о\,.\ \ Пусть в~данных имеется объект~$B$ 
и~$P\mapsto B$, тогда в~силу существования и~единственности причины~$B$ 
в~данных должен существовать объект~$C$, предшествующий~$B$ 
и~содержащий причину~$P$. Поскольку $A\hm\subseteq P$ и~$B$ содержит 
причину~$P$, то $B\mapsto A$. С~учетом леммы теорема~2 доказана.
  
  \smallskip
  
  Пусть даны пространства $U_1, U_2,\ldots$ и~имеется последовательность 
данных (процесс выполнения этапов проекта в~соответствии с~рис.~1) $A, B, 
\ldots$, где каждый объект является подмножеством некоторого 
пространства~$U_i$, $i\hm=1,\ldots$ Тогда в~объекте~$A$ присутствует 
причина~$P$ появления интересующего нас свойства~$C$ в~объекте~$B$. Пусть 
$P\hm\subseteq A$, тогда по теореме~2 $\forall \alpha\hm\in P$:  
$C\mapsto \{\alpha\}$, т.\,е.\ из появления~$C$ следует появление 
характеристики~$\alpha$ в~предшествующем объекте. Это необходимое условие 
того, что~$C$ удовлетворяет причинно-следственным связям развития процесса 
выполнения проекта. Если для~$C$ нет характеристики~$\alpha$, которую можно 
отнести к~причине~$C$, то можно считать, что нарушена  
при\-чин\-но-след\-ст\-вен\-ная связь и~$C$~--- аномальный объект. 
  
  \smallskip
  
  \noindent
  \textbf{Пример.} Если объект~$C$ состоит в~получении суммы~$a$ 
фирмой~$K$, то согласно теореме~2 в~пред\-шест\-ву\-ющем объекте должна 
существовать причина перевода суммы~$a$ на фирму~$K$. Если эта причина 
в~проекте отсутствует, то это можно считать признаком мошеннической схемы. 
Все проекты по предположению собираются из <<кубиков>>, содержащихся в~БЗ. 
Тогда можно сравнить цену объекта~$C$, породившего получение суммы~$a$, 
и~сумму, присутствующую в~смете проекта. Если разница велика, то это либо 
ошибка проекта, либо признак мошеннической схемы.
  
  \section{Поиск противоречий на~основе~принципа <<необходимых~условий>>}
   
  Как было показано в~разд.~3, нахождение противоречий соответствуют 
движению от следствия к~причине. Для каждого объекта в~наблюдаемых данных 
выявление причин его появления является трудоемкой задачей. Кроме того, при 
реализации контроля соблюдения при\-чин\-но-след\-ст\-вен\-ных связей на 
большом множестве участников экономической деятельности задача анализа 
причин становится трудоемкой. Поэтому процедуру контроля необходимо разбить 
на два этапа, где первый этап состоит в~анализе простых <<необходимых 
условий>> проявления мошенничества, когда используется хотя бы одна 
известная характеристика причины. Второй этап (в~режиме офлайн) состоит 
в~выявлении причин, позволяющих провести анализ источников мошеннических 
схем. 
  
  Один из подходов к~выбору <<необходимых условий>> состоит в~построении 
множества подцелей исходной цели проекта (структурный метод построения 
проекта~\cite{7-gr}). Каждая подцель описывается диаграммой на рис.~1, 
и~реализации подцелей должны образовывать полный функционал цели. Это 
является необходимым, но не достаточным условием достижения цели, так как 
при таком подходе отсутствует компонент согласования всех подцелей в~единую 
систему. Однако такой подход значительно упрощает анализ выполнения проекта 
на предмет поиска мошенничества. Если признаки мошенничества будут 
обнаружены в~реализации хотя бы одной из подцелей, то это значит, что 
мошенничество присутствует в~реализации всего проекта. 
  
  Аналогично в~реализации каждого этапа в~любой из подцелей можно выделять 
простые <<необходимые условия>> нарушения при\-чин\-но-след\-ст\-венн\-ых 
связей. 
  
  Таким образом, получается множество <<необходимых условий>>, нарушение 
которых свидетельствует о наличии мошенничества. Это множество 
<<необходимых условий>> можно назвать метаданными~[8, 9] для контроля 
проекта на выявление мошенничества. 
  
  
  \section{Заключение }
  
  В поиске противоречий необходимо от транзакций, соответствующих 
следствиям при\-чин\-но-след\-ст\-вен\-ных связей, переходить к~анализу причин 
наблюдаемых следствий. Это сложная задача, которая связана с~описанием причин 
определенных свойств. 
  
  В работе представлена модель, позволяющая строить множество необходимых 
условий соответствия наблюдаемого следствия вызвавшей его причине. Этот 
подход делает поиск противоречий вполне вычислимой задачей, но не гарантирует 
успех. 
  
  {\small\frenchspacing
 {%\baselineskip=10.8pt
 \addcontentsline{toc}{section}{References}
 \begin{thebibliography}{9}
\bibitem{1-gr}
\Au{Грушо А.\,А., Зацаринный~А.\,А., Тимонина~Е.\,Е.} Блокчейны цифровой экономики на базе 
системы ситуационных центров и~централизованного консенсуса~// Радиолокация, навигация, 
связь: Мат-лы XXV Междунар. научн.-технич. конф.~---
Воронеж: Издательский дом ВГУ, 2019. Т.~6. С.~183--191. 
\bibitem{2-gr}
\Au{Grusho A., Zatsarinny~A., Timonina~E.} A~system approach to information security in 
distributed ledgers on the situational centers platform.~---
Lecture notes in computer science ser.~--- Springer, 2019 
(in press).
\bibitem{3-gr}
\Au{Финн В.\,К.} Искусственный интеллект: Методология, применения, философия.~--- М.: 
Красанд, 2011. 448~с.

\bibitem{5-gr} %4
\Au{Аншаков~О.\,М., Фабрикантова~Е.\,Ф.} ДСМ-ме\-тод автоматического порождения 
гипотез: Логические и~эпистемологические основания.~--- М.: Либроком, 2009. 432~с.

\bibitem{4-gr} %5
\Au{Poelmans J., Elzinga~P., Viaene~S., Dedene~G.} Formal concept analysis in knowledge 
discovery: A~survey~// Conceptual structures: From information to intelligence~/ Eds.\ M.~Croitoru, 
S.~Ferr$\acute{\mbox{e}}$, and D.~Lukose.~--- Lecture notes in computer science 
ser.~--- Berlin--Heidelberg: Springer, 2010. Vol.~6208.  P.~139--153.

\bibitem{6-gr}
\Au{Панкратова~Е.\,С., Финн~В.\,К.} Автоматическое по\-рож\-де\-ние гипотез в~интеллектуальных 
системах.~--- М.: Либроком, 2009. 528~с. 
\bibitem{7-gr}
\Au{Денисов А.\,А., Колесников~Д.\,Н.} Теория больших систем управления.~--- Л.: Энергоиздат, 1982. 488~с.

\bibitem{9-gr}
\Au{Грушо А.\,А., Грушо Н.\,А., Забежайло~М.\,И., Смирнов~Д.\,В., Тимонина~Е.\,Е.} 
Параметризация в~прикладных задачах поиска эмпирических причин~// Информатика и~её 
применения, 2018. Т.~12. Вып.~3. С.~62--66.

\bibitem{8-gr}
\Au{Грушо А.\,А., Грушо Н.\,А., Левыкин~М.\,В., Тимонина~Е.\,Е.} Методы идентификации 
захвата хоста в~распределенной ин\-фор\-ма\-ци\-он\-но-вы\-чис\-ли\-тель\-ной сис\-те\-ме, 
защищенной с~помощью метаданных~// Информатика и~её применения, 2018. Т.~12. Вып.~4. 
С.~41--45.

 \end{thebibliography}

 }
 }

\end{multicols}

\vspace*{-3pt}

\hfill{\small\textit{Поступила в~редакцию 03.04.19}}

%\vspace*{8pt}

%\pagebreak

\newpage

\vspace*{-28pt}

%\hrule

%\vspace*{2pt}

%\hrule

%\vspace*{-2pt}

\def\tit{ARCHITECTURAL DECISIONS IN~THE~PROBLEM 
OF~IDENTIFICATION OF~FRAUD IN~THE~ANALYSIS 
OF~INFORMATION FLOWS IN~DIGITAL ECONOMY\\[-5pt]}


\def\titkol{Architectural decisions in~the~problem 
of~identification of~fraud in~the~analysis 
of~information flows in~digital economy}

\def\aut{A.\,A.~Grusho, M.\,I.~Zabezhailo, N.\,A.~Grusho, and~E.\,E.~Timonina}

\def\autkol{A.\,A.~Grusho, M.\,I.~Zabezhailo, N.\,A.~Grusho, and~E.\,E.~Timonina}

\titel{\tit}{\aut}{\autkol}{\titkol}

\vspace*{-13pt}


 \noindent
   Institute of Informatics Problems, Federal Research Center ``Computer Sciences and 
Control'' of the Russian Academy of Sciences; 44-2~Vavilov Str., Moscow 119133, 
Russian Federation

\def\leftfootline{\small{\textbf{\thepage}
\hfill INFORMATIKA I EE PRIMENENIYA~--- INFORMATICS AND
APPLICATIONS\ \ \ 2019\ \ \ volume~13\ \ \ issue\ 2}
}%
 \def\rightfootline{\small{INFORMATIKA I EE PRIMENENIYA~---
INFORMATICS AND APPLICATIONS\ \ \ 2019\ \ \ volume~13\ \ \ issue\ 2
\hfill \textbf{\thepage}}}

\vspace*{3pt}


   
     
   \Abste{An approach to a~research of some types of fraud in digital economy with the usage of relationships of 
cause and effect is formulated. In all types of the considered frauds, the discrepancy between the 
purposes of financial transactions and actual cost of achievement of these purposes
has to be observed. Data on 
transactions can be collected by observing information flows in which these transactions are reflected. 
The architecture of data collection and their analysis can be organized by means of the distributed 
ledgers with the centralized consensus that allows creating an analog of the electronic account book 
fixing financial and economic activity of subjects of digital economy in the region. 
   The methods of fraud identification considered are based on the contradictions 
between actions described in transactions and information, which is contained in plans, standards, 
precedents, etc. 
   The method based on a~simplified scheme of implementation of the abstract project is considered. 
For identification of contradictions, it is necessary to carry out the analysis from the effect to the cause, 
i.\,e., to look for anomalies in information describing the generation of the observed effects. 
   It is shown how in implementation of the project it is possible to allocate simple ``necessary 
conditions'' of violation of cause and effect relationships, i.\,e., a~set of ``necessary conditions'' 
violation of which demonstrates fraud existence. It is possible to call this set of "necessary conditions" 
by metadata for control of the project for fraud identification.} 
   
   \KWE{digital economy; information flows; relationships of reason and effect; detection of 
fraudulent schemes}
   
  

 \DOI{10.14357/19922264190204}

\vspace*{-20pt}

 \Ack
   \noindent
   The work was partially supported by the Russian Foundation for Basic Research (projects  
18-29-03081 and 18-07-00274).



%\vspace*{6pt}

  \begin{multicols}{2}

\renewcommand{\bibname}{\protect\rmfamily References}
%\renewcommand{\bibname}{\large\protect\rm References}

{\small\frenchspacing
 {\baselineskip=10.5pt
 \addcontentsline{toc}{section}{References}
 \begin{thebibliography}{9}
\bibitem{1-gr-1}
\Aue{Grusho, A.\,A., A.\,A.~Zatsarinny, and E.\,E.~Timonina.} 2019. Blokcheyny tsifrovoy ekonomiki 
na baze sistemy situatsionnykh tsentrov i~tsentralizovannogo konsensusa [Blockchains of digital 
economy on the basis of the system of the situational centres and the centralized consensus]. 
\textit{25th Scientific and Technical Conference (International) ``Radar-Location, Navigation, 
Communication'' Proceedings}. Voronezh: VSU Publs. 6:183--191.
\bibitem{2-gr-1}
\Aue{Grusho, A., A.~Zatsarinny, and E.~Timonina.} 2019 (in press). 
A~system approach to information security 
in distributed ledgers on the situational centers platform. 
Lecture notes in computer science ser. Springer.
\bibitem{3-gr-1}
\Aue{Finn, V.\,K.} 2011. \textit{Iskusstvennyy intellekt: Metodologiya, primeneniya, filosofiya} 
[Artificial intelligence: Methodology, applications, philosophy]. Moscow: KRASAND. 448~p.

\bibitem{5-gr-1}
\Aue{Anshakov, O.\,M., and E.\,F.~Fabrikantova}. 2009. \textit{DSM-metod avtomaticheskogo porozhdeniya gipotez: Logicheskie 
i~epistemologicheskie osnovaniya} [JSM-method of automatic hypothesis generation: Logical and 
epistemological]. Moscow: KD LIBROKOM. 432~p.
\bibitem{4-gr-1} %5
\Aue{Poelmans, J., P.~Elzinga, S.~Viaene, and G.~Dedene.} 2010. Formal concept analysis in 
knowledge discovery: A~survey. \textit{Conceptual structures: From information to intelligence}. 
Eds.\ M.~Croitoru, S.~Ferr$\acute{\mbox{e}}$, and D.~Lukose. Lecture notes in 
computer science ser. Berlin--Heidelberg: Springer. 6208:139--153.

\bibitem{6-gr-1}
\Aue{Pankratov, E.\,S., and V.\,K.~Finn}. 
2009. \textit{Avtomaticheskoe porozhdenie gipotez v~intellektual'nykh 
sistemakh} [Automatic hypotheses generation in intelligent systems]. Moscow: KD 
\mbox{LIBROKOM}.  528~p. 
\bibitem{7-gr-1}
\Aue{Denisov, A.\,A., and D.\,N.~Kolesnikov.} 1982. \textit{Teoriya bol'shikh 
sistem upravleniya} [Theory of big control systems]. Leningrad: Energoizdat. 488~p.

\bibitem{9-gr-1}
\Aue{Grusho, A.\,A., N.\,A.~Grusho, M.\,I.~Zabezhailo, D.\,V.~Smirnov, and 
E.\,E.~Timonina.} 2018. 
Parametrizatsiya v~prikladnykh zadachakh poiska empiricheskikh prichin 
[Parametrization in applied 
problems of search of the empirical reasons]. 
\textit{Informatika i~ee Primeneniya~--- 
Inform. Appl.} 12(3):62--66.

\bibitem{8-gr-1}
\Aue{Grusho, A.\,A., N.\,A.~Grusho, M.\,V.~Levykin, and E.\,E.~Timonina.} 2018. Metody 
identifikatsii zakhvata khosta v~raspredelennoy informatsionno-vychislitel'noy sisteme, 
zashchishchennoy s~pomoshch'yu metadannykh [Methods of identification of host capture 
in the  distributed information system which is protected on the base of meta data].
\textit{Informatika i~ee 
Primeneniya~--- Inform. Appl.} 12(4):41--45.
{ %\looseness=1

}

\end{thebibliography}

 }
 }

\end{multicols}

\vspace*{-12pt}

\hfill{\small\textit{Received April 3, 2019}}

%\pagebreak

%\vspace*{-18pt}

\Contr

\noindent
\textbf{Grusho Alexander A.} (b.\ 1946)~--- Doctor of Science in physics and 
mathematics, professor, principal scientist, Institute of Informatics Problems, 
Federal Research Center ``Computer Sciences and Control'' of the Russian 
Academy of Sciences; 44-2~Vavilov Str., Moscow 119133, Russian Federation; 
\mbox{grusho@yandex.ru} 

\vspace*{3pt}

\noindent
\textbf{Zabezhailo Michael I.} (b.\ 1956)~--- Doctor of Science in physics and 
mathematics, principal scientist, Institute of Informatics Problems, Federal Research 
Center ``Computer Sciences and Control'' of the Russian Academy of Sciences;  
44-2~Vavilov Str., Moscow 119133, Russian Federation; 
\mbox{m.zabezhailo@yandex.ru} 

\vspace*{3pt}


\noindent
\textbf{Grusho Nikolai A.} (b.\ 1982)~--- Candidate of Science (PhD) in physics 
and mathematics, senior scientist, Institute of Informatics Problems, Federal 
Research Center ``Computer Sciences and Control'' of the Russian Academy of 
Sciences; 44-2~Vavilov Str., Moscow 119133, Russian Federation; 
\mbox{info@itake.ru} 

\vspace*{3pt}


\noindent
\textbf{Timonina Elena E.} (b.\ 1952)~--- Doctor of Science in technology, 
professor, leading scientist, Institute of Informatics Problems, Federal Research 
Center ``Computer Sciences and Control'' of the Russian Academy of Sciences;  
44-2~Vavilov Str., Moscow 119133, Russian Federation; 
\mbox{eltimon@yandex.ru} 

\label{end\stat}

\renewcommand{\bibname}{\protect\rm Литература}   %7
\def\stat{tirsin}

\def\tit{МОДЕЛИ УПРАВЛЕНИЯ РИСКОМ В~ГАУССОВСКИХ СТОХАСТИЧЕСКИХ 
СИСТЕМАХ$^*$}

\def\titkol{Модели управления риском в~гауссовских стохастических 
системах}

\def\aut{А.\,Н.~Тырсин$^1$, А.\,А.~Сурина$^2$}

\def\autkol{А.\,Н.~Тырсин, А.\,А.~Сурина}

\titel{\tit}{\aut}{\autkol}{\titkol}

\index{Тырсин А.\,Н.}
\index{Сурина А.\,А.}
\index{Tyrsin A.\,N.}
\index{Surina A.\,A.}




{\renewcommand{\thefootnote}{\fnsymbol{footnote}} \footnotetext[1]
{ Работа выполнена при финансовой поддержке РФФИ (проект 17-01-00315а).}}


\renewcommand{\thefootnote}{\arabic{footnote}}
\footnotetext[1]{Уральский федеральный университет имени первого Президента России Б.\,Н. Ельцина; Институт 
экономики Уральского отделения Российской академии наук, \mbox{at2001@yandex.ru}}
\footnotetext[2]{Южно-Уральский государственный университет (национальный 
исследовательский университет),  \mbox{dallila87@mail.ru}}

\vspace*{-2pt}

 
  
  \Abst{Описан новый подход к~исследованию риска многомерных стохастических 
сис\-тем. Он основан на гипотезе о~том, что рис\-ком можно управ\-лять за счет изменения 
вероятностных свойств компонент многомерной стохастической сис\-те\-мы, в~качестве 
которых используют факторы рис\-ка. Исследован случай гауссовских стохастических сис\-тем, 
опи\-сы\-ва\-емых случайными векторами, име\-ющи\-ми многомерное нормальное распределение. 
Как показало моделирование, не учтенные в~яв\-ном виде многомерность сис\-те\-мы и~взаимная 
коррелированность ее компонент могут привести к~существенному занижению фактического 
риска. Приведены результаты расчета ве\-ро\-ят\-ности опасного исхода в~зависимости от 
чис\-ло\-вых характеристик многомерной гауссовской случайной величины~--- ковариационной 
мат\-ри\-цы и~вектора математических ожиданий. Выполнена апро\-ба\-ция предложенной модели 
на примере анализа популяционного рис\-ка сер\-деч\-но-со\-су\-ди\-стых заболеваний. Описаны 
модели управ\-ле\-ния рис\-ком в~виде задач его минимизации или достижения заданного уров\-ня. 
Управляющими переменными являются чис\-ло\-вые характеристики случайного вектора~--- 
ковариационная мат\-ри\-ца и~век\-тор математических ожиданий. Проведена апро\-ба\-ция метода 
управ\-ле\-ния рис\-ком с~по\-мощью статистического моделирования методом Мон\-те Карло.}
  
  \KW{риск; модель; стохастическая сис\-те\-ма; случайный вектор; управ\-ле\-ние; нормальное 
рас\-пре\-де\-ление}

\DOI{10.14357/19922264180208}
  
%\vspace*{-6pt}


\vskip 10pt plus 9pt minus 6pt

\thispagestyle{headings}

\begin{multicols}{2}

\label{st\stat}
  
\section{Введение}

\vspace*{-4pt}

  Уже не вызывает сомнений наличие общемировой тенденции быст\-ро\-го рос\-та 
ущерба от природных катаклизмов, техногенных катастроф, террористических 
актов и~экономических потрясений. Многие авторы отмечают, что тем\-пы рос\-та 
ущерба значительно превосходят темпы рос\-та экономики~[1--3]. Это можно 
объяснить по\-сто\-ян\-ным возрастанием рис\-ка в~условиях на\-уч\-но-тех\-ни\-че\-ской 
революции и~форсированного развития техносферы~[4]. Очевидно, что для 
снижения ущер\-ба от природных катаклизмов, техногенных катастроф, 
террористических актов и~экономических потрясений необходимо повысить 
без\-опас\-ность функционирования со\-от\-вет\-ст\-ву\-ющих сис\-тем, а~значит, снизить 
риск. Для этого необходимы адекватные модели и~эффективные методы 
управ\-ле\-ния риском сис\-тем.
  
  Реальные системы, как правило, являются многомерными, их 
функционирование во многом носит стохастический характер, у~них час\-то 
мож\-но выделить десятки различных факторов риска~\cite{1-t}. При решении 
задачи управления рис\-ком необходимо опираться на модель рис\-ка. 

Обычно 
моделирование рис\-ка сводится к~выделению опасных исходов, 
количественному заданию по\-след\-ст\-вий от их наступления и~оцениванию 
вероятностей этих исходов~\cite[с.~37--43]{5-t}. При этом вклад компонент 
многомерной сис\-те\-мы объединяют и~рас\-смат\-ри\-ва\-ют уже одномерную сис\-те\-му 
как случайную величину~[5, с.~148--156; 6, с.~82--87]. 

Но вопрос взаимного 
влияния опас\-ных ситуаций, вызванных разными элементами многомерной 
сис\-те\-мы, мало исследован, чаще всего им пренебрегают или существенно 
упрощают, считая разные опас\-ные исходы взаимно независимыми, 
и~пренебрегают ве\-ро\-ят\-ностью их одновременного наступления. 

Для 
относительно прос\-тых объектов, когда можно априори указать все опасные 
исходы, при наличии статистической информации или экспертных оценок 
о~шан\-сах их по\-яв\-ле\-ния в~целом данный подход дает приемлемые на практике 
результаты. 
%
Обычно здесь удается накопить достаточную статистику для 
оценивания вероятностей на\-ступ\-ле\-ния опас\-ных исходов, а~форма взаимосвязи 
между элементами сис\-те\-мы является до\-ста\-точ\-но прос\-той и~может быть 
описана, например, с~по\-мощью ло-\linebreak\vspace*{-12pt}

\pagebreak

\noindent
ги\-ко-ве\-ро\-ят\-ност\-ных моделей 
риска~\cite{7-t} в~рам\-ках тео\-рии струк\-тур\-но-слож\-ных сис\-тем~\cite{8-t}.
  
  Однако у сложных систем структуру взаимодействия между элементами 
обычно не удается описать с~по\-мощью ло\-ги\-ко-ве\-ро\-ят\-ност\-ных моделей~--- 
стохастические связи между элементами не позволяют их адекватно 
моделировать с~по\-мощью алгебры логики (AND, OR, NOT), а~изменения 
со\-сто\-яния\linebreak элементов и~самой сис\-те\-мы носят непрерывный харак\-тер. Понятия 
опас\-ных исходов также могут размываться, делая невозможным их конкретное 
выделение. К~таким сис\-те\-мам мож\-но отнес\-ти,\linebreak например, 
со\-ци\-аль\-но-эко\-но\-ми\-че\-ские сис\-те\-мы, 
вклю\-чая территориальные и~региональные сис\-те\-мы, 
живые сис\-те\-мы, например человека с~точ\-ки зрения со\-сто\-яния здоровья.
  
  Таким образом, несмотря на большое число исследований, взаимному 
влиянию элементов и~различных фак\-то\-ров рис\-ка на без\-опас\-ность слож\-ных 
многомерных сис\-тем уделяется недостаточно внимания. Во многих случаях, 
когда нет воз\-мож\-ности явно связать разные факторы рис\-ка в~виде  
ло\-ги\-ко-ве\-ро\-ят\-ност\-ной модели, их корреляция при расчете рис\-ка не 
учитывается, поэтому проб\-ле\-ма\-ти\-ка исследований в~об\-ласти анализа рис\-ка, 
особенно в~час\-ти со\-зда\-ния эффективных моделей описания и~управ\-ле\-ния 
рис\-ком слож\-ных многомерных сис\-тем, в~на\-сто\-ящее время становится одной из 
актуальных.
  
  В~\cite{9-t, 10-t} предложен подход к~моделированию риска, со\-глас\-но 
которому стохастическую сис\-те\-му представляют в~виде случайного вектора со 
взаимно коррелированными компонентами, а~в~качестве управ\-ля\-ющих 
переменных используют его чис\-ло\-вые характеристики. Целью \mbox{статьи} является 
описание моделей управ\-ле\-ния рис\-ком на основе данного подхода.

\section{Модель риска в~гауссовских стохастических системах}

  Пусть $S$~--- некоторая многомерная стохастическая сис\-те\-ма. Выделим 
в~этой сис\-те\-ме фак\-то\-ры рис\-ка $X_1, X_2, \ldots, X_m$. В~результате получим 
пред\-став\-ле\-ние сис\-те\-мы в~виде случайного вектора $\mathbf{X}\hm= (X_1, X_2, 
\ldots, X_m)$ с~некоторой плот\-ностью 
ве\-ро\-ят\-ности~$p_{\mathbf{X}}(\mathbf{x})$.
  
  Вместо общепринятого выделения конкретных опасных ситуаций будем 
задавать гео\-мет\-ри\-че\-ские области неблагоприятных исходов. Они могут 
выглядеть произвольным образом в~за\-ви\-си\-мости от конкретной задачи 
и~определяются на основе име\-ющей\-ся априорной информации. Для 
опре\-де\-лен\-ности опишем пред\-ла\-га\-емый под\-ход на примере распространенной 
концепции нежелательных событий как больших и~маловероятных отклонений 
случайной величины относительно ее математического ожидания. Тогда 
опасными ситуациями будем считать большие и~маловероятные отклонения 
вы\-бо\-роч\-ных значений~$x_{ij}$ любой из компонент~$X_j$ относительно 
математических ожиданий $\mu_j\hm=Х{\sf M}[X_j]$, $j\hm=1, 2,\ldots ,m$. 
Вероятность неблагоприятного исхода для каж\-дой из компонент~$X_j$ 
зададим как
 \begin{multline*}
  {\sf P}\left(D_j\right)={\sf P}\left(X_j\in D_j\right)=
  {\sf P}\left( X_j\notin \overline{D}_j\right)\,,\\
  \overline{D}_j=\left\{ x:\ \mu_j-A_{1j}\sigma_j<x<\mu_j+A_{2j}\sigma_j\right\}\,,
\end{multline*}
где $\sigma_j$~--- среднее квад\-ра\-ти\-че\-ское отклонение случайной 
величины~$X_j$; $A_{1j}$ и~$A_{2j}$~--- заданные ниж\-ний и~верх\-ний 
пороговые уров\-ни (в~единицах~$\sigma_j$), т.\,е.\ об\-ласть благоприятных 
исходов ограничена диапазоном $(\mu_j\hm-A_{1j}\sigma_j; 
\mu_j+A_{2j}\sigma_j)$.

  Теперь необходимо задать многомерную область опас\-ных ситуаций~$D$, 
учтя взаимное вли\-яние компонент на по\-яв\-ле\-ние неблагоприятных исходов. Она 
равна $D\hm= \mathbf{R}^m\backslash \overline{D}$, где $\overline{D}$~--- 
об\-ласть допустимых значений фак\-то\-ров рис\-ка. Опишем 
об\-ласть~$\overline{D}$. Это можно сделать различными способами. Наиболее 
оправданным с~гео\-мет\-ри\-че\-ской точки зрения пред\-став\-ля\-ет\-ся задать ее в~виде 
внут\-рен\-ней об\-ласти $m$-ос\-но\-го эллипсоида
  $$
  \overline{D}= \left\{ \mathbf{x}=\left( x_1, x_2, \ldots , x_m\right): 
\sum\limits_{j=1}^m \fr{(x_j-\mu_j^\prime)^2}{A^2_j \sigma_j^2}<1\right\}
  $$
с~центром в~точке $\boldsymbol{\mu}^\prime \hm= (\mu_1^\prime, \mu_2^\prime, 
\ldots , \mu_m^\prime)$, $\mu_j^\prime\hm= \mu_j\hm+A_j\sigma_j$, 
$A_j\hm=(A_{1j}\hm+ A_{2j})/2$, $j\hm=1, 2,\ldots, m$. Тогда для случайного 
вектора~$\mathbf{X}$ ве\-ро\-ят\-ность неблагоприятного исхода будет равна
\begin{multline}
{\sf P}(D) ={\sf P}(\mathbf{X}\in D)\,,\quad
D={}\\
\!\!{}=\left\{ \mathbf{x}=\left( x_1, x_2, \ldots ,x_m\right): \sum\limits^m_{j=1} 
\fr{(x_j-\mu_j)^2}{A_j^2\sigma_j^2}\geq 1\right\}.\!\!
\label{e1-t}
\end{multline}

\begin{figure*}[b] %fig1
     \vspace*{1pt}
 \begin{center}
 \mbox{%
 \epsfxsize=162.957mm 
 \epsfbox{tyr-1.eps}
 }
 \end{center}
\vspace*{-9pt}
\Caption{Реализации стандартного нормального случайного вектора: 
(\textit{а})~$\rho\hm= 0$; (\textit{б})~$\rho\hm = 0{,}9$}
\end{figure*}
  
  Заметим, что в~(\ref{e1-t}) об\-ласть~$D$ неблагоприятных исходов 
пред\-став\-ля\-ет собой внешнюю об\-ласть \mbox{$m$-ос}\-но\-го эл\-лип\-со\-ида, у~которого 
полуоси по каж\-дой из координат равны~$A_j\sigma_j$ соответственно, т.\,е.\ по 
каж\-дой $j$-й оси эта об\-ласть соответствует одномерному случаю~$D_j$. 
Очевидно, когда исход не лежит на одной из осей, событие~$D$ может 
реализоваться и~при отсутствии рис\-ко\-вых отклонений по всем компонентам 
(воз\-мож\-ны ситуации $\mathbf{X}\hm\in D$ и~$\forall j\ X_j\notin D_j$).
  
  Задав функцию по\-след\-ст\-вий от опасных си\-ту\-аций в~виде $g(\mathbf{x})$, 
получим модель для количественной оцен\-ки риска:
  \begin{equation*}
  r(\mathbf{X})=\idotsint\limits_{\mathbf{R}^m} g(\mathbf{x}) 
p_{\mathbf{X}}(\mathbf{x})\,d\mathbf{x}\,.
 % \label{e2-t}
  \end{equation*}
  Если принять
  \begin{equation}
  g(\mathbf{x})=\begin{cases}
  1\,, &\ \mathbf{x}\in D\,;\\
  0\,, &\ \mathbf{x}\notin D\,,
  \end{cases}
  \label{e3-t}
  \end{equation}
то $r(\mathbf{X})={\sf P}(\mathbf{X}\in D)$, т.\,е.\ риск оцениваем как ве\-ро\-ят\-ность 
неблагоприятного исхода. Если на ранней стадии исследования сис\-те\-мы 
слож\-но до\-ста\-точ\-но точ\-но описать функцию $g(\mathbf{x})$, то 
формула~(\ref{e3-t}) становится оценкой~${\sf P}(D)$ и~является удоб\-ным 
начальным приб\-ли\-же\-ни\-ем модели риска.

  Рассмотрим далее наиболее рас\-про\-стра\-нен\-ный част\-ный случай, 
когда~$\mathbf{X}$ имеет совместное нормальное рас\-пре\-де\-ле\-ние с~плот\-ностью 
ве\-ро\-ят\-ности
\begin{multline*}
  p_{\mathbf{X}}(\mathbf{x})={}\\
  {}=\fr{1}{\sqrt{(2\pi)^m\vert\boldsymbol{\Sigma}\vert}}\,\exp 
\left\{ -\fr{1}{2}\left( \mathbf{x}-\mathbf{a}\right)^{\mathrm{T}} 
\boldsymbol{\Sigma}^{-1}(\mathbf{x}-\mathbf{a})\right\}\,,
  \end{multline*}
где $\mathbf{a}=(a_1, a_2, \ldots, a_m)^{\mathrm{T}}$~--- век\-тор 
математических ожиданий; $\boldsymbol{\Sigma}\hm= \{ \sigma_{ij}\}_{m\times 
m}$~--- ковариационная мат\-рица.
  
  Использование гауссовского случайного век\-то\-ра опирается на цент\-раль\-ную 
предельную тео\-ре\-му~\cite{11-t}. Как показала апробация на ряде примеров, 
такая идеализация не столь критична, и~если есть ка\-кие-ли\-бо основания 
считать, что плот\-но\-сти вероятностей компонент вектора~$\mathbf{X}$ име\-ют 
более вытянутые хвос\-ты, то это можно скорректировать за счет 
со\-от\-вет\-ст\-ву\-юще\-го задания функции~$g(\mathbf{x})$.
  
  Исследуем влияние многомерности и~коррелированности факторов риска на 
ве\-ро\-ят\-ность по\-яв\-ле\-ния неблагоприятных исходов.
  
  \smallskip
  
  \noindent
  \textbf{Пример~1.}\ Для наглядности рас\-смот\-рим двумерный гауссовский 
случайный вектор ($X_1, X_2$) с~плот\-ностью ве\-ро\-ят\-ности
  \begin{equation}
   p_{X_1, X_2}(x_1, x_2) =\fr{e^{-Q(x_1-a_1, x_2-a_2)/2}}{2\pi 
\sigma_1\sigma_2 \sqrt{1-\rho^2}}\,.
 \label{e4-t}
\end{equation}
Здесь
$$
  Q\left(y_1,y_2\right)=\fr{1}{1-\rho^2}\left( \fr{y_1^2}{\sigma_1^2} -\fr{2\rho 
y_1y_2}{\sigma_1\sigma_2}+\fr{y_2^2}{\sigma_2^2}\right)\,,
$$
где $y_i=x_i-a_i$, $i=1, 2$; $\rho\hm= \sigma_{12}/(\sigma_1\sigma_2)$~--- 
коэффициент корреляции между~$X_1$ и~$X_2$.
  
  На рис.~1 показаны примеры реализаций стандартного нормального 
случайного вектора ($X_1, X_2$) для некоррелированных ($\rho\hm=0$) 
и~коррелированных ($\rho\hm = 0{,}9$) компонент. Видим, что увеличение 
тес\-но\-ты корреляционной связи между компонентами приводит к~вытягиванию 
диа\-грам\-мы рас\-се\-яния и~увеличению ве\-ро\-ят\-ности по\-яв\-ле\-ния больших 
укло\-не\-ний случайного век\-тора.
  
  
 % \smallskip
 
  { \begin{center}  %fig2
 \vspace*{-2pt}
  \mbox{%
 \epsfxsize=78.141mm 
 \epsfbox{tyr-2.eps}
 }


\end{center}


\noindent
{{\figurename~2}\ \ \small{Зависимости $\lg P(D)$ от порогового уров\-ня~$A$: 
(\textit{а})~$D_e(\mathbf{X})\hm = 0$; (\textit{б})~$D_e(\mathbf{X}) \hm= 0{,}5$; 
(\textit{в})~$D_e(\mathbf{X}) \hm=1$; \textit{1}~--- $m \hm= 1$; 
\textit{2}~---2; 
\textit{3}~--- 3; \textit{4}~--- 4; \textit{5}~--- $m\hm= 5$}}
}

\vspace*{18pt}

\setcounter{figure}{2}

  
  \noindent
  \textbf{Пример~2.}\ Зададим для определенности раз\-мер\-ность 
вектора~$\mathbf{X}$ от~1 до~5. В~[12] введен коэффициент тес\-но\-ты 
со\-вмест\-ной линейной корреляционной связи компонент случайного 
век\-то\-ра~$\mathbf{X}$, равный $D_e(\mathbf{X})\hm=1\hm- \vert 
\mathbf{R}_{\mathbf{X}}\vert^{1/m}$, где $\mathbf{R}_{\mathbf{X}}$~--- 
корреляционная мат\-ри\-ца случайного век\-то\-ра~$\mathbf{X}$. Очевидно, что 
$0\hm\leq D_e(\mathbf{X})\hm\leq 1$. Случай $D_e(\mathbf{X})\hm=0$ 
соответствует не\-за\-ви\-си\-мости компонент~$X_1, X_2, \ldots , X_m$, а~при 
$D_e(\mathbf{X})\hm=1$ имеем строгую линейную за\-ви\-си\-мость компонент.
  
  Рассмотрим три случая: $D_e(\mathbf{X}) \hm= 0$, $D_e(\mathbf{X}) \hm= 
0{,}5$ и~$D_e(\mathbf{X}) \hm= 1$. Результаты рас\-че\-та ве\-ро\-ят\-ности 
неблагоприятного исхода~(\ref{e1-t}) приведены на рис.~2. Для большей на\-гляд\-ности 
примем  $A_1\hm=A_2= \cdots =A_m\hm=A$.
  
 
  
  Анализ графиков на рис.~2 говорит о~сле\-ду\-ющем. Увеличение 
раз\-мер\-ности~$m$ и~тес\-но\-ты кор\-ре\-ля\-ционной связи меж\-ду компонентами 
случайного вектора~$\mathbf{X}$ приводит к~резкому рос\-ту ве\-ро\-ят\-ности 
не\-бла\-го\-при\-ят\-но\-го исхода.
  
  Особенно важным оказалось то, что даже относительно малая тес\-но\-та 
корреляционной связи ($D_e(\mathbf{X}) \hm= 0{,}5$), которая почти всегда 
наблюдается на прак\-ти\-ке, уже приводит к~значительному рос\-ту 
ве\-ро\-ят\-ности~${\sf P}(D)$. Эффект усиливается с~увеличением значений~$A_j$, что 
соответствует менее вероятным, но более опас\-ным неблагоприятным исходам. 
Например, при $A\hm = 6$ ве\-ро\-ят\-ность неблагоприятного исхода более чем 
в~7000~раз выше у~коррелированной сис\-те\-мы ($D_e(\mathbf{X}) \hm= 1$) по 
срав\-не\-нию с~некоррелированной ($D_e(\mathbf{X}) \hm= 0$). Поэтому при 
моделировании рис\-ка необходимо учитывать как фактор мно\-го\-мер\-ности, так 
и~тес\-но\-ту корреляционных связей.

  

\section{Апробация модели риска на~примере анализа 
популяционного риска сердечно-сосудистых заболеваний}

  Одной из малоизученных проблем в~медицине является комплексная оценка 
популяционного здоровья одновременно по нескольким факторам риска в~их 
взаимосвязи. Это объясняется тем, что не\-яс\-но, как учитывать вклад каждого 
фак\-то\-ра рис\-ка в~общую оценку со\-сто\-яния здоровья. Обычно в~таких случаях 
используются экспертные оценки, которые нельзя считать в~полной мере 
объективными~[13, 14].
  
  Исследуем динамику изменения с~возрастом популяционного риска 
сер\-деч\-но-со\-су\-ди\-стых за\-бо\-леваний по основным биологическим факторам риска, 
к~которым относят артериальную ги\-пер\-тен\-зию, дис\-ли\-пи\-де\-мию, повышенный 
уро\-вень глюкозы в~крови и~избыточную массу тела~[15]. 

В~качестве 
биологических па\-ра\-мет\-ров, ха\-рак\-те\-ри\-зу\-ющих эти факторы рис\-ка, используют 
уровень общего холестерина (ОХС), сис\-то\-ли\-че\-ское артериальное дав\-ле\-ние 
(САД), индекс массы тела (ИМТ), уровень глюкозы (УГ). 

Статистический 
материал получен в~результате комплексного сплош\-но\-го углуб\-лен\-но\-го  
кли\-ни\-ко-эпи\-де\-мио\-ло\-ги\-че\-ско\-го обследования муж\-ской сельской 
популяции с~гнез\-до\-вой выборкой. Всего было обследовано~1402~мужчины 
одного из сел Челябинской об\-ласти, что со\-ста\-ви\-ло~93\% от списочного со\-ста\-ва 
села. Для всех пациентов был проведен необходимый комплекс клинических, 
лабораторных и~инструментальных методов обследования для 
квалифицированного заключения о~со\-сто\-янии здоровья. Работу проводила 
бригада специалистов, со\-сто\-ящая из со\-труд\-ни\-ков ка\-фед\-ры госпитальной 
терапии и~семейной медицины Челябинской государственной медицинской 
академии и~врачей Челябинской об\-ласт\-ной клинической больницы №\,1~[16].
  
  Пороговые значения биологических па\-ра\-мет\-ров, характеризующих основ\-ные 
биологические фак\-то\-ры рис\-ка, при превышении которых риск 
сер\-деч\-но-со\-су\-ди\-стых 
ослож\-не\-ний резко воз\-рас\-та\-ет (ниже этого значения~--- норма), в~соответствии 
с~[15] рав\-ны: САД~--- 140~мм\ рт.\ ст.; ИМТ~--- 25~кг/м$^2$; ОХС~--- 
5~ммоль/л; УГ~--- 5,5~ммоль/л. 
Исследование проводилось сле\-ду\-ющим 
образом. 

Было сформировано четыре группы по воз-\linebreak рас\-там: 18--24~года,  
25--34~года, 35--44~года и~45--54~года. Проверка по критерию со\-гла\-сия 
\mbox{$\chi^2$-Пир}\-со\-на статистической гипотезы о~соответствии\linebreak каж\-дой группы 
наблюдений для всех фак\-то\-ров рис\-ка нормальному распределению на уровне 
зна\-чи\-мости~0,05 не была отклонена. Поэтому считаем, что имеем гауссовскую 
стохастическую сис\-те\-му раз\-мер\-ности $m\hm = 4$.
  
  Затем для каждой группы были определены средние значения 
и~ковариационные мат\-ри\-цы. Вы\-чис\-ле\-ние вероятности ${\sf P}(D)$ можно 
выполнять двумя способами~--- с~по\-мощью чис\-лен\-но\-го интегрирования для 
малых размерностей ($m\hm\leq 4$) или методом статистических испытаний  
Мон\-те Кар\-ло~[17] при раз\-мер\-ности $m\hm>4$. Результаты расчета 
приведены в~таб\-лице.



  
  Видим, что наблюдается тенденция рос\-та риска возникновения 
сердечно-сосудистых ослож\-не\-ний.\linebreak\vspace*{-12pt}

\vspace*{6pt}

%\begin{table*}
{\small
  \begin{center}
  \begin{tabular}{|c|c|}
\multicolumn{2}{p{43mm}}{Значения вероятностей риска возникновения 
сер\-деч\-но-со\-су\-ди\-стых ослож\-не\-ний}\\[-6pt]
\multicolumn{2}{c}{\ }\\
\hline
Возраст, лет&Вероятность\\
\hline
18--24&0,70\\
25--34&0,78\\
35--44&0,95\\
45--54&0,98\\
\hline
\end{tabular}
\vspace*{2pt}
\end{center}
}
%\end{table*}

\columnbreak
  
  
  \noindent
   Полученные в~целом высокие значения 
вероятностей ${\sf P}(D)$ соответствуют фактическому со\-сто\-янию
   здоровья. 
   
   Как 
показали результаты комплексного сплош\-но\-го углуб\-лен\-но\-го  
кли\-ни\-ко-эпи\-де\-мио\-ло\-ги\-че\-ско\-го обследования, в~обследованной 
популяции здоровых лиц в~воз\-рас\-те старше~34~лет практически не оказалось.

 
\section{Модели управления риском}

  Введенная модель риска позволяет на практике осуществлять управ\-ле\-ние 
сто\-ха\-сти\-че\-ской сис\-те\-мой с~целью его снижения.
  
  \bigskip
  
  \noindent
  \textbf{Пример~3.}\ Проиллюстрируем данный подход на прос\-тей\-шем 
примере гауссовской стохастической сис\-те\-мы с~раз\-мер\-ностью $m\hm=2$. На 
рис.~3 показана воз\-мож\-ность уменьшения ве\-ро\-ят\-ности неблагоприятного 
исхода~${\sf P}(D)$, а~значит, и~рис\-ка за счет варьирования па\-ра\-мет\-ров плот\-ности 
$p_{\mathbf{X}}(\mathbf{x})$. Об\-ласть неблагоприятного исхода~$D$ 
рас\-по\-ло\-же\-на выше линии в~правом верх\-нем углу.
  
  Видим, что возможные варианты изменения па\-ра\-мет\-ров случайного вектора 
($X_1, X_2$): уменьшение ковариации (или коэффициента корреляции), 
изменение математических ожиданий случайных величин, уменьшение 
дис\-пер\-сий~$\sigma_1^2$ или~$\sigma_1^2$~--- могут привести к~снижению 
ве\-ро\-ят\-ности~${\sf P}(D)$.
  

  
  Суть управ\-ле\-ния риском гауссовской стохастической сис\-те\-мы со\-сто\-ит 
в~сле\-ду\-ющем. Задав функцию по\-след\-ст\-вий от опасных ситуаций 
$g(\mathbf{x})$ и~введя ограничения на допустимые значения элементов 
ковариационной мат\-ри\-цы $G(\boldsymbol{\Sigma})$ и~сред\-них значений 
компонент сис\-те\-мы $H(\mathbf{a})$, сформулируем задачу минимизации рис\-ка 
с~переменными~$\boldsymbol{\Sigma}$ и~$\mathbf{a}$:
  \begin{multline}
    r(\boldsymbol{\Sigma}, \mathbf{a})=\displaystyle \idotsint\limits_{\mathbf{R}^m} 
g(\mathbf{x}) p_{\mathbf{X}}(\mathbf{x})\,d\mathbf{x} \to 
\min\limits_{\boldsymbol{\Sigma}, \mathbf{a}}\,,\\
\boldsymbol{\Sigma} \in G(\boldsymbol{\Sigma})\,,\enskip \mathbf{a}\in 
H(\mathbf{a})\,.
    \label{e5-t}
  \end{multline}
  
  Задача~(\ref{e5-t}) является задачей нелинейного про\-грам\-ми\-ро\-ва\-ния. Ее 
мож\-но решить разными методами. Одним из них является метод барьерных 
функ\-ций (внут\-рен\-них штраф\-ных функ\-ций)~[18]. Его основная идея со\-сто\-ит 
в~приведении задачи поиска услов\-но\-го экстремума к~по\-сле\-до\-ва\-тель\-ности задач 
на\-хож\-де\-ния без\-услов\-но\-го экстремума вспомогательной функции:
  $$
  F(\mathbf{X}, b_k) =r(\boldsymbol{\Sigma}, \mathbf{a}) 
+{\sf P}\left(\boldsymbol{\Sigma}, \mathbf{a}, b_k\right)\,,
  $$
где ${\sf P}(\boldsymbol{\Sigma}, \mathbf{a}, b_k)$~--- штраф\-ная функ\-ция; $b_k$~--- 
па\-ра\-метр штрафа.

\pagebreak
  
  
 % \smallskip
 \end{multicols}
 
  \begin{figure*} %fig3
  \vspace*{1pt}
 \begin{center}
 \mbox{%
 \epsfxsize=164.954mm 
 \epsfbox{tyr-3.eps}
 }
 \end{center}
\vspace*{-4pt}
\Caption{Снижение риска: (\textit{а})~исходное со\-сто\-яние; (\textit{б})~за счет уменьшения 
корреляции; (\textit{в}) и~(\textit{г})~за счет изменения 
математических ожиданий случайных величин~$X_1$ или~$X_2$
соответственно; 
(\textit{д}) и~(\textit{е})~за счет уменьшения дис\-пер\-сии~$\sigma_1^2$ 
или~$\sigma_2^2$ соответственно}
\vspace*{12pt}
\end{figure*}
  
  
  \begin{multicols}{2}
  
  \noindent
  \textbf{Пример~4.}\ Рас\-смот\-рим двумерный гауссовский случайный век\-тор 
с~плот\-ностью ве\-ро\-ят\-ности~(\ref{e4-t}). Задача минимизации будет выглядеть 
как
\begin{multline*}
  r(\boldsymbol{\Sigma}, \mathbf{a})={}\\
  {}=\iint\limits_{\mathbf{R}^2} 
   \fr{g(x_1, x_2)}{2\pi \sigma_1\sigma_2 \sqrt{1-\rho^2}} %\times{}\\
%{}\times 
e^{-Q(x_1-a_1, x_2-
a_2)/2} \,dx_1 dx_2 \to {}\\
{}\to \min\limits_{\boldsymbol{\Sigma}, \mathbf{a}}
\end{multline*}
с ограничениями
\begin{equation*}
\begin{array}{l}
\sigma_1^2\sigma_2^2> \sigma_{12}^2\,;\\[6pt]
a_i^- <a_i< a_i^+\,,\enskip i=1,2\,;\\[6pt]
\sigma_{ij}^-<\sigma_{ij}<\sigma_{ij}^+\,,\enskip j=1,2\,.
\end{array}
\end{equation*}
  
  Зададим конкретные значения: $a_1^-\hm=a_2^- \hm=-3$; $a_1^+\hm= a_2^+ 
\hm=3$; $\sigma_{12}^- \hm= 0{,}1$; $\sigma_{12}^+\hm=3$. Для 
опре\-де\-лен\-ности считаем, что $\forall\ \mathbf{x}\hm\notin D$ 
$g(\mathbf{x})\hm=0$. Получаем задачу

\noindent
  \begin{multline*}
  r(\boldsymbol{\Sigma}, \mathbf{a})= \iint\limits_{\mathbf{R}^2} 
\fr{  g(x_1, x_2)}{1{,}2\pi}\times{}\\
\!{}\times e^{-\left((x_1-2)^2-1{,}6x_1 x_2+(x_2-2)^2\right)/\left(2\cdot 0{,}6^2\right)}\, 
dx_1 dx_2\to \min\limits_{\boldsymbol{\Sigma}, \mathbf{a}}\hspace*{-5.74925pt}
\end{multline*}
с ограничениями
\begin{equation*}
\begin{array}{l}
\sigma_1^2\sigma_2^2>\sigma_{12}^2\,;\\[6pt]
-3<a_1, a_2<3\,;\\[6pt]
0{,}1<\sigma_1, \sigma_2<3\,;\\[6pt]
0{,}1< \sigma_2<3\,.
\end{array}
\end{equation*}
  
    
    Выберем в~качестве штрафной функ\-ции обратную: 
    $$
    {\sf P}\left(\boldsymbol{\Sigma},\mathbf{a}, b^k\right) =-b^k \sum\limits^m_{j=1} 
\fr{1}{t_j(\Sigma, a)}\,.
$$
    Тогда с~учетом всех ограничений вспомогательная функ\-ция примет вид:
    
    \noindent
    \begin{multline*}
    F\left( \mathbf{X}, b_k\right) ={}\\
    {}=\iint\limits_{\mathbf{R}^2} \!\fr{1}{1{,}2\pi} 
\,e^{- \left((x_1-2)^2 -1{,}6x_2x_2+(x_2-2)^2\right)/\left(2\cdot 0{,}6^2\right)} \,d\mathbf{x} -{}\\
{}-
b^k\left( \fr{1}{\sigma_1} +\fr{1}{\sigma_2}+\fr{1}{\sigma_1^2 \sigma_2^2-
\sigma_{12}^2} +\fr{1}{a_1+3}+ {}\right.\\
{}+\fr{1}{3-a_1}+ \fr{1}{a_2+3} +\fr{1}{3-a_1} 
+\fr{1}{\sigma_1-0{,}1} +{}\\
\left.{}+\fr{1}{3-\sigma_1} +\fr{1}{\sigma_2 -0{,}1} +\fr{1}{3-
\sigma_2}\right) \to \min\limits_{\boldsymbol{\Sigma}, \mathbf{a}}\,.
    \end{multline*}
  
  Поиск минимума вспомогательной функ\-ции находим с~по\-мощью 
покоординатного спус\-ка. Начальные значения па\-ра\-мет\-ров:
  $$
  \boldsymbol{\Sigma}^0=\begin{pmatrix}
  1 & 0{,}8\\
  0{,}8 & 1
  \end{pmatrix}\,;\enskip \mathrm{a}^0=\begin{pmatrix}
  2\\ 2\end{pmatrix}\,;\enskip b^0=10\,.
  $$
  
  Задача минимизации была решена при значениях па\-ра\-мет\-ров: 
$a_1\hm=2{,}6$; $a_2\hm=0{,}6$; $\sigma_1\hm=2{,}9$; $\sigma_2\hm=2{,}9$; 
$\sigma_{12}\hm= 1{,}39\cdot 10^{-16}$. При этом минимум целевой функ\-ции 
с~точ\-ностью до~0,001:  $r(\boldsymbol{\Sigma}^*, \mathbf{a}^*)\hm= 
0{,}041$.
  
  Задача~(\ref{e5-t}) полезна лишь в~качестве первого приб\-ли\-же\-ния модели 
управ\-ле\-ния рис\-ком, так как не учитывает ограничений, связанных с~за\-тра\-та\-ми 
на изменения варь\-и\-ру\-емых па\-ра\-мет\-ров относительно своих начальных 
значений.
  
  Если ввести ограничения на за\-тра\-ты, связанные с~изменением 
переменных~$\boldsymbol{\Sigma}$ и~$\mathbf{a}$, то получим задачу:
  \begin{equation}
  \left.
  \begin{array}{c}
 \displaystyle r(\boldsymbol{\Sigma}, \mathbf{a}) =\idotsint\limits_{\mathbf{R}^m} 
g(\mathbf{x}) p_{\mathbf{X}}(\mathbf{x})\,d\mathbf{x}\to 
\min\limits_{\boldsymbol{\Sigma}, \mathbf{a}}\,,\\[6pt]
  %\hspace*{27mm}
  \boldsymbol{\Sigma}\in G(\boldsymbol{\Sigma})\,,\enskip \mathbf{a}\in 
H(\mathbf{a})\,,\\[6pt]
  a_i=a_i^0+\delta_i\,,\enskip v_i(\delta_i)\leq V_i\,,\enskip i=1,\ldots ,m\,,\\[6pt]
  \sigma_{ij} =\sigma_{ij}^0 +\Delta_{ij}\,,\enskip w_{ij}(\Delta_{ij})\leq 
W_{ij}\,,\\[6pt]
\hspace*{40mm} i, j=1,\ldots ,m\,,
  \end{array}
  \right\}
  \label{e6-t}
  \end{equation}
где $v_i(\delta_i)$~--- функ\-ция за\-трат на изменение среднего значения $i$-й 
компоненты, име\-ющей начальное значение~$a_i^0$; $V_i$~--- предельная 
величина за\-трат на это изменение; $w_{ij}(\Delta_{ij})$~--- функция за\-трат на 
изменение ковариации между $i$-й и~$j$-й компонентами; $\sigma_{ij}^0$~--- 
начальное значение ковариации; $W_{ij}$~--- предельная величина за\-трат на 
это изменение.
  
  Минимизация риска не всегда может быть приемлемым управ\-ле\-ни\-ем. 
Альтернативой является достижение приемлемого рис\-ка~$r^*$ при 
минимальных изменениях чис\-ло\-вых характеристик гауссовской 
сис\-те\-мы~$\mathbf{X}$. Здесь возможны два варианта по\-ста\-нов\-ки задачи.
  
  Во-первых, на основе~(\ref{e6-t}) получаем альтернативный вариант:
  
  \noindent
  \begin{equation*}
  \begin{array}{c}
  \displaystyle \sum\limits_{i=1}^m v_i(\delta_i) +\sum\limits^m_{i-1} 
\sum\limits^m_{j=i} w_{ij}(\Delta_{ij}) \to \min\limits_{\boldsymbol{\Sigma}, 
\mathbf{a}}\,,\\[6pt]
%  \hspace*{27mm}
\boldsymbol{\Sigma} \in G(\boldsymbol{\Sigma})\,,\enskip \mathbf{a}\in 
H(\mathbf{a})\,,\\[6pt]
  \delta_i =a_i-a_i^0\,,\enskip \Delta_{ij}=\sigma_{ij}-\sigma_{ij}^0\,,\enskip 
i,j=1,\ldots ,m\,,\\[6pt]
  r(\boldsymbol{\Sigma}, \mathbf{a}) =r^*\,.
  \end{array}
  \end{equation*}
  
  
  Во-вторых, если сложно задать функции за\-трат~$v_i(\cdot)$ 
и~$w_{ij}(\cdot)$, то мож\-но минимизировать суммарное 
изменение~$\boldsymbol{\Sigma}$ и~$\mathbf{a}$, перейдя к~задаче
  \begin{equation*}
  \begin{array}{c}
  \displaystyle \sum\limits_{i=1}^m \alpha_i \delta_i^2 +\sum\limits^m_{i-1} 
\sum\limits^m_{j=i} \beta_{ij} \Delta_{ij}^2 \to 
\min\limits_{\boldsymbol{\Sigma}, \mathbf{a}}\,,\\[6pt]
 % \hspace*{27mm}
 \boldsymbol{\Sigma} \in G(\boldsymbol{\Sigma})\,,\enskip \mathbf{a}\in 
H(\mathbf{a})\,,\\[6pt]
  \delta_i =a_i-a_i^0\,,\enskip \Delta_{ij}=\sigma_{ij}-\sigma_{ij}^0\,,\enskip 
i,j=1,\ldots ,m\,,\\[6pt]
  r(\boldsymbol{\Sigma}, \mathbf{a}) =r^*\,,
  \end{array}
  \end{equation*}
где $\alpha_i$ и~$\beta_{ij}$~--- весовые коэффициенты.

\section{Заключение}

\noindent
  \begin{enumerate}[1.]
  \item  Предложен новый подход к~исследованию рис\-ка сложных сис\-тем. 
В~его основе лежит моделирование сис\-те\-мы в~виде мно\-го\-мер\-ной случайной 
величины, компоненты которой являются факторами риска.
  \item  Для гауссовских стохастических сис\-тем предложены модели 
управления рис\-ком на основе его минимизации или до\-сти\-же\-ния заданного 
уров\-ня, используя в~качестве управ\-ля\-ющих переменных чис\-ло\-вые 
характеристики случайного век\-то\-ра~--- вектор математических ожиданий 
и~ковариационную мат\-рицу.
  \item  В настоящее время обычно при исследовании риска слож\-ных 
многомерных сис\-тем не выделяют в~явном виде их компоненты и~их 
коррелированность. Как показало моделирование, неучет в~явном виде 
многомерности сис\-те\-мы и~взаимной коррелированности ее компонент может 
привести к~существенному занижению фактического рис\-ка. Усиление тес\-но\-ты 
корреляционной связи между факторами риска приводит к~значительному 
рос\-ту ве\-ро\-ят\-ности одновременного принятия ими опас\-ных значений.
  \item Предложенная гипотеза об управ\-ле\-нии рис\-ком слож\-ной сис\-те\-мы на 
основе изменения чис\-ло\-вых характеристик ее математической модели в~форме 
случайного век\-то\-ра носит предварительный характер. Необходимо выполнить 
апро\-ба\-цию данного подхода на ряде задач.
  \end{enumerate}
  \vspace*{-8pt}
  
{\small\frenchspacing
 {%\baselineskip=10.8pt
 \addcontentsline{toc}{section}{References}
 \begin{thebibliography}{99}
 
 \bibitem{3-t} %1
\Au{Гор А.} Земля на чаше весов. В~по\-ис\-ках новой об\-щей цели~// Новая 
пост\-ин\-ду\-стри\-аль\-ная волна на Западе: Антология~/ Пер. с~англ.~--- М.: 
Academia, 1999. С.~557--571. (\Au{Gore~A.} Earth in the balance. Forging a~new 
common purpose.~--- London: Earthscan Publications Ltd., 1992.)
\bibitem{1-t} %2
\Au{Воробьев Ю.\,Л., Малинецкий~Г.\,Г., Махутов~Н.\,А.} Управ\-ле\-ние рис\-ком 
и~устойчивое развитие: Человеческое измерение~// Известия вузов. 
Прикладная нелинейная динамика, 2000. Т.~8. №\,6. С.~12--26.
\bibitem{2-t} %3
\Au{Порфирьев Б.\,Н.} Снижение природных рис\-ков экономического развития 
России: роль государства~// Актуальные проб\-ле\-мы гражданской защиты:  
Мат-лы 11-й Междунар. науч.-практич. конф. по проб\-ле\-мам защиты 
населения и~территорий от чрезвычайных ситуаций.~--- 
Н.~Новгород: Вектор-ТиС, 2006. С.~44--50. {\sf 
gov.mari.ru/debzn/omgo/46.djvu}.

\bibitem{4-t}
\Au{Порфирьев Б.\,Н.} Управ\-ле\-ние в~чрез\-вы\-чай\-ных ситуациях. Итоги науки 
и~техники. Проб\-ле\-мы без\-опас\-ности: чрез\-вы\-чай\-ные ситуации. Т.~1.~--- М.: 
ВИНИТИ, 1991. 204~с.
\bibitem{5-t}
\Au{Вишняков Я.\,Д., Радаев~Н.\,Н.} Общая теория рисков.~--- 
2-е изд., испр.~--- М.: Академия, 2008. 368~с.
\bibitem{6-t}
\Au{Акимов В.\,А., Лесных~В.\,В., Радаев~Н.\,Н.} Риски в~природе, техносфере, 
обществе и~экономике.~--- М.: Деловой экспресс, 2004. 352~с.
\bibitem{7-t}
\Au{Соложенцев Е.\,Д.} Сценарное ло\-ги\-ко-ве\-ро\-ят\-ност\-ное управ\-ле\-ние 
рис\-ком в~бизнесе и~технике.~--- 2-е изд.~--- СПб.: Биз\-нес-прес\-са, 2006. 560~с.
\bibitem{8-t}
\Au{Рябинин И.\,А.} На\-деж\-ность и~без\-опас\-ность струк\-тур\-но-слож\-ных  
сис\-тем.~--- СПб.: Политехника, 2000. 248~с.
\bibitem{9-t}
\Au{Тырсин А.\,Н.} О~моделировании рис\-ка в~сис\-те\-мах критичных 
инфра\-струк\-тур~// Экономические и~технические аспекты без\-опас\-ности 
строительных критичных инфраструктур: Тезисы Междунар. конф.~--- 
Екатеринбург: УрФУ, 2015. С.~205--208. {\sf 
http://elar.urfu. ru/bitstream/10995/33468/1/safety\_2015.pdf.}
\bibitem{10-t}
\Au{Тырсин А.\,Н., Сурина~А.\,А.} Моделирование рис\-ка в~многомерных 
сто\-ха\-сти\-че\-ских сис\-те\-мах~// Вестн. Томского государственного университета. 
Управ\-ле\-ние, вы\-чис\-ли\-тель\-ная техника и~информатика, 2017. №\,2(39).  
С.~65--72.
\bibitem{11-t}
\Au{Гнеденко Б.\,В.} Курс тео\-рии вероятностей.~--- 8-е изд., испр. и~доп.~--- М.: 
Едиториал УРСС, 2005. 448~с.
\bibitem{12-t}
\Au{Pena D., Rodriguez~J.} Descriptive measures of multivariate scatter and linear 
dependence~// J.~Multivariate Anal., 2003. Vol.~85. P.~361--374.
\bibitem{13-t}
\Au{Кирьянов Б.\,Ф., Токмачёв~М.\,С.} Математические модели 
в~здравоохранении.~--- Великий Новгород: НовГУ им.\ Ярослава Мудрого, 
2009. 279~с.
\bibitem{14-t}
\Au{Цинкер М.\,Ю., Кирьяков~Д.\,А., Камалтдинов~М.\,Р.} Применение 
комплексного ин\-дек\-са нарушения здо\-ровья населения для оцен\-ки 
популяционного здо\-ровья в~Пермском крае~// Из\-вес\-тия Самарского научного 
цент\-ра РАН, 2013. Т.~15. №\,3(6). С.~1988--1992.
\bibitem{15-t}
Профилактика хронических неинфекционных заболеваний. Рекомендации.~--- 
М., 2013. 128~с. {\sf http:// www.webmed.irkutsk.ru/doc/pdf/prevent.pdf}.
\bibitem{16-t}
\Au{Тырсин А.\,Н., Калев~О.\,Ф., Яшин~Д.\,А., Лебедева~О.\,В.} Оцен\-ка 
со\-сто\-яния здо\-ровья популяции на основе энтропийного моделирования~// 
Математическая био\-ло\-гия и~био\-ин\-фор\-ма\-ти\-ка, 2015. Т.~10. Вып.~1.  
С.~206--219. doi: 10.17537/2015.10.206.
\bibitem{17-t}
\Au{Михайлов Г.\,А., Войтишек~А.\,В.} Чис\-лен\-ное ста\-ти\-сти\-че\-ское 
моделирование. Методы Мон\-те-Кар\-ло.~--- М.: Академия, 2006. 368~с.
\bibitem{18-t}
\Au{Пантелеев А.\,В., Летова~Т.\,А.} Методы оптимизации в~примерах 
и~задачах.~--- 3-е изд., стер.~---М.: Выс\-шая школа, 2008. 544~с.

 \end{thebibliography}

 }
 }

\end{multicols}

\vspace*{-6pt}

\hfill{\small\textit{Поступила в~редакцию 21.08.17}}

\vspace*{6pt}

%\newpage

%\vspace*{-24pt}

\hrule

\vspace*{2pt}

\hrule

%\vspace*{8pt}


\def\tit{A~MODEL OF RISK MANAGEMENT IN~GAUSSIAN\\ STOCHASTIC SYSTEMS}

\def\titkol{A~model of risk management in Gaussian stochastic systems}

\def\aut{A.\,N.~Tyrsin$^{1,2}$ and~A.\,A.~Surina$^3$}

\def\autkol{A.\,N.~Tyrsin and~A.\,A.~Surina}

\titel{\tit}{\aut}{\autkol}{\titkol}

\vspace*{-9pt}


\noindent
$^1$Ural Federal University named after first President of Russia B.\,N.~Yeltsin, 
19~Mira Str., Ekaterinburg 620002,\linebreak
$\hphantom{^1}$Russian Federation 

\noindent
$^2$Institute of Economics, Ural Branch of the Russian Academy of Sciences, 
29~Moskovskaya Str., Yekaterinburg\linebreak
$\hphantom{^1}$620014, Russian Federation

\noindent
$^3$Institute of Natural Sciences, South Ural State University, 87~Lenin Ave., 
Chelyabinsk 454080, Russian Federation


\def\leftfootline{\small{\textbf{\thepage}
\hfill INFORMATIKA I EE PRIMENENIYA~--- INFORMATICS AND
APPLICATIONS\ \ \ 2018\ \ \ volume~12\ \ \ issue\ 2}
}%
 \def\rightfootline{\small{INFORMATIKA I EE PRIMENENIYA~---
INFORMATICS AND APPLICATIONS\ \ \ 2018\ \ \ volume~12\ \ \ issue\ 2
\hfill \textbf{\thepage}}}

\vspace*{3pt}




\Abste{A new approach to research of risk of multidimensional 
stochastic systems is described. It is based on a~hypothesis that 
the risk can be managed by changing probabilistic properties of a~component of 
a~multidimensional stochastic system. The case of Gaussian stochastic systems 
described by random vectors having
the multidimensional normal distribution 
is investigated. Modeling has shown that multidimensionality of a~system
and relative\linebreak\vspace*{-12pt}}

\Abstend{
 correlation of components unaccounted in an explicit form, 
can lead to essential understating of risk factors. Results of calculation 
of the probability of a~dangerous outcome depending on numerical characteristics of 
a~multidimensional Gaussian random variable (a~covariance matrix and 
a~vector of mathematical expectations) are given. Approbation of the suggested model 
is executed by the example of the analysis of the risk of cardiovascular 
diseases in population. Models of risk management in the form of 
a~minimization problem or achievement of the given level are described. 
Control variables are the numerical characteristics of a~random vector covariance 
matrix and a~vector of mathematical expectations. Approbation of the method of 
risk management was carried 
out by means of statistical model operation by the Monte-Carlo method.}

\KWE{risk; model; stochastic system;  random vector; control; normal distribution}


\DOI{10.14357/19922264180208} %

%\vspace*{-14pt}

  \Ack
   \noindent
   The work was supported by the Russian Foundation
   for Basic Research (project 17-01-00315а).
   



%\vspace*{-3pt}

  \begin{multicols}{2}

\renewcommand{\bibname}{\protect\rmfamily References}
%\renewcommand{\bibname}{\large\protect\rm References}

{\small\frenchspacing
 {%\baselineskip=10.8pt
 \addcontentsline{toc}{section}{References}
 \begin{thebibliography}{99}
 
 \bibitem{3-t-1} %1
\Aue{Gore, A.} 1992. \textit{Earth in the balance. Forging a~new common purpose.} 
London: Earthscan Publications Ltd. 440~p.
\bibitem{1-t-1} %2
\Aue{Vorob'ev, Yu.\,L., G.\,G.~Malinetskiy, and N.\,A.~Makhutov}. 2000. Uprav\-le\-nie 
ris\-kom i~ustoy\-chi\-voe raz\-vi\-tie: che\-lo\-ve\-che\-skoe iz\-m\-ere\-nie [Management of risk and 
sustainable development: Human measurement]. \textit{Izvestiya vuzov. Pri\-klad\-naya 
nelineynaya dinamika}  [Proceedings of the Universities. Applied Nonlinear 
Dynamics] 8(6):12--26.
\bibitem{2-t-1} %3
\Aue{Porfir'ev, B.\,N.} 2006. Sni\-zhe\-nie pri\-rod\-nykh ris\-kov eko\-no\-mi\-che\-sko\-go 
raz\-vi\-tiya Ros\-sii: rol' go\-su\-dar\-st\-va
 [The reduction of natural risks of economic 
development of Russia: The role of the state]. \textit{Aktu\-al'\-nye prob\-le\-my 
grazh\-dan\-skoy za\-shchi\-ty: Mat-ly 11-y Mezhdunar.  
nauch.-praktich. konf. po problemam zashchity naseleniya i~territoriy ot\linebreak 
chrezvychaynykh situatsiy} [Actual Problems of Civil\linebreak Protection:  11th Scientific and 
Practical Conference (International) on Problems of Protection of the Population and 
Territories from Emergency Situations Proceedings]. 
N.~Novgorod: Vector-TiS. 44--50. Available at: {\sf 
http://gov.mari.ru/debzn/omgo/46.djvu} (accessed  August~7, 2017).

\bibitem{4-t-1}
\Aue{Porfir'ev,  B.\,N.} 1991. \textit{Up\-rav\-le\-nie v~chrez\-vy\-chay\-nykh si\-tu\-a\-tsi\-yakh. 
T.~1. Ito\-gi nau\-ki i~tekh\-ni\-ki. Prob\-le\-my bez\-opas\-nosti: 
chrezvychaynye situatsii} 
[Management in emergency situations. Vol.~1. The results of science and 
technology. Security concerns: Emergency situations]. Moscow: VINITI. 204~p.
\bibitem{5-t-1}
\Aue{Vishnyakov, Ya.\,D., and N.\,N.~Radaev.} 2008. \textit{Ob\-shhaya teo\-riya 
ris\-kov} [Common theory of risks]. 2nd ed. Moscow: Academy. 368~p.
\bibitem{6-t-1}
\Aue{Akimov, V.\,A.,  V.\,V.~Lesnykh, and N.\,N.~Radaev.} 2004. \textit{Riski 
v~pri\-ro\-de, tekh\-no\-sfe\-re, ob\-shchest\-ve i~eko\-no\-mi\-ke} 
[Risks in the nature, technosphere, 
society, and the economy]. Moscow: Business Express. 352~p.
\bibitem{7-t-1}
\Aue{Solozhentsev, E.\,D.} 2006. \textit{Stse\-nar\-noe 
lo\-gi\-ko-ve\-ro\-yat\-nost\-noe 
uprav\-le\-nie ris\-kom v~biz\-ne\-se i~tekh\-ni\-ke} [Scenario logic and probabilistic 
management of risk in business and engineering]. 2nd ed. St.\ Petersburg: Biznes 
pressa. 560~p.
\bibitem{8-t-1}
\Aue{Ryabinin, I.\,A.} 2000. \textit{Na\-dezh\-nost' i~bezopas\-nost'  
struk\-tur\-no-slozh\-nykh sis\-tem} [Reliability and safety of the structural and composite 
systems]. St.\ Petersburg: Polytechnique. 248~p.
\bibitem{9-t-1}
\Aue{Tyrsin, A.\,N.} 2015. O~mo\-de\-li\-ro\-va\-nii ris\-ka v~sis\-te\-makh kri\-tich\-nykh 
infra\-struk\-tur [About model operation of risk in the systems of critical infrastructures]. 
\textit{Economic and Technical Aspects of
Safety of Civil Engineering Critical Infrastructures
Conference (International) Abstracts}.
Ekaterinburg: Ural Federal University. 205--208.
 Available at: {\sf 
http://elar.urfu.ru/bitstream/10995/33468/1/safety\_\linebreak 2015.pdf}  (accessed August~7, 
2017).
\bibitem{10-t-1}
\Aue{Tyrsin, A.\,N., and A.\,A.~Surina.} 2017. Modelirovanie ris\-ka 
v~mno\-go\-mer\-nykh sto\-kha\-sti\-che\-skikh sis\-te\-makh [Modeling of risk in 
multidimensional stochastic systems]. \textit{Vestn. Tomskogo gosudarstvennogo 
universiteta. Upravlenie, vy\-chis\-li\-tel'\-naya tekh\-ni\-ka 
i~in\-for\-ma\-ti\-ka} [Bull. Tomsk State 
University. Management, Computer Facilities, and Informatics] 2(39):65--72.
\bibitem{11-t-1}
\Aue{Gnedenko, B.\,V.} 2005. \textit{Kurs teo\-rii ve\-ro\-yat\-no\-stey} [Course of 
probability theory]. 8th ed. Moscow: Editorial URSS. 448~p.
\bibitem{12-t-1}
\Aue{Pena, D., and J.~Rodriguez.} 2003. Descriptive measures of multivariate 
scatter and linear dependence.  \textit{J.~Multivariate Anal.} 85:361--374.
\bibitem{13-t-1}
\Aue{Kir'yanov, B.\,F., and M.\,S.~Tokmachev.} 2009. \textit{Ma\-te\-ma\-ti\-che\-skie 
modeli v~zdra\-vo\-okh\-ra\-ne\-nii} [Mathematical models in health care]. Veliky 
Novgorod: NovSU. 279~p.
\bibitem{14-t-1}
\Aue{Tsinker, M.\,Yu., D.\,A.~Kir'yakov, and M.\,R.~Kamaltdinov.} 2013. Pri\-me\-ne\-nie 
komp\-leks\-no\-go indek\-sa na\-ru\-she\-niya zdo\-rov'ya na\-se\-le\-niya dlya otsen\-ki 
po\-pu\-lya\-tsi\-on\-no\-go zdo\-rov'ya v~Permskom krae [The integrated index of health 
situation of the population to assess population health in the Perm region.
\textit{Izvestiya Samarskogo nauchnogo tsent\-ra RAN} [Proceedings of the Samara 
Scientific Center of RAS] 15(3(6)):1988--1992.
\bibitem{15-t-1}
Profilaktika khro\-ni\-che\-skikh ne\-in\-fek\-tsi\-on\-nykh za\-bo\-le\-va\-niy. 
Re\-ko\-men\-da\-tsii 
[Prevention of pre-existing conditions. Recommendations]. Moscow. 128~p.
Available at: {\sf 
http://www.webmed.irkutsk.ru/doc/pdf/prevent.pdf} (accessed August~7, 2017).
\bibitem{16-t-1}
\Aue{Tyrsin, A.\,N., O.\,F.~Kalev, D.\,A.~Yashin, and O.\,V.~Lebedeva.} 2015. 
Otsen\-ka so\-sto\-yaniya zdo\-rov'ya po\-pu\-lya\-tsii 
na osnove entropiynogo mo\-de\-li\-ro\-va\-niya 
[Assessment of health status of a~population on the basis of entropy modeling]. 
\textit{Math. Biol. Bioinf.} 10(1):206--219. doi: 10.17537/2015.10.206.
\bibitem{17-t-1}
\Aue{Mihaylov, G.\,A., and A.\,V.~Voytishek}. 2006. \textit{Chis\-len\-noe 
sta\-ti\-sti\-che\-skoe mo\-de\-li\-ro\-va\-nie. Metody Monte-Karlo} [Numerical statistical model 
operation. Monte-Carlo methods]. Moscow: Akademy. 368~p.
\bibitem{18-t-1}
\Aue{Panteleev, A.\,V., and T.\,A.~Letova.} 2008. \textit{Metody op\-ti\-mi\-za\-tsii 
v~pri\-me\-rakh i~za\-da\-chakh} [Optimization methods in examples and tasks]. 3rd ed. 
Moscow: Higher School. 544~p.
\end{thebibliography}

 }
 }

\end{multicols}

\vspace*{-3pt}

\hfill{\small\textit{Received August 21, 2017}}

%\vspace*{-24pt}

\Contr

\noindent
\textbf{Tyrsin Alexander N.} (b.\ 1961)~-- Doctor of Science in technology, Head of 
Department of Applied Mathematics, Ural Federal University named after first 
President of Russia B.\,N.~Yeltsin, 19~Mira Str., Ekaterinburg 620002, Russian 
Federation; senior scientist, Institute of Economics, Ural Branch of the Russian 
Academy of Sciences, 29~Moskovskaya Str., Yekaterinburg 620014, Russian 
Federation; \mbox{at2001@yandex.ru} 

\vspace*{3pt}

\noindent
\textbf{Surina Alfiya A.} (b.\ 1987)~--- PhD student, Institute of Natural Sciences, 
South Ural State University, 87~Lenin Ave., Chelyabinsk 454080, Russian 
Federation; \mbox{dallila87@mail.ru} 



\label{end\stat}


\renewcommand{\bibname}{\protect\rm Литература}  %8
\def\stat{vaskan}

\def\tit{АЛГОРИТМ ВИЗУАЛИЗАЦИИ ПЛОСКОГО ЯДРА ВЕРОЯТНОСТНОЙ МЕРЫ$^*$}

\def\titkol{Алгоритм визуализации плоского ядра вероятностной меры}

\def\aut{С.\,Н.~Васильева$^1$, Ю.\,С.~Кан$^2$}

\def\autkol{С.\,Н.~Васильева, Ю.\,С.~Кан}

\titel{\tit}{\aut}{\autkol}{\titkol}

\index{Васильева С.\,Н.}
\index{Кан Ю.\,С.}
\index{Vasil'eva S.\,N.}
\index{Kan Yu.\,S.}




{\renewcommand{\thefootnote}{\fnsymbol{footnote}} \footnotetext[1]
{Результаты работы получены в~рамках выполнения государственного задания 
 Минобрнауки №\,2.2461.2017/ПЧ, а также при финансовой поддержке РФФИ 
 (проект 15-08-02833а).}}


\renewcommand{\thefootnote}{\arabic{footnote}}
\footnotetext[1]{Московский авиационный институт (национальный исследовательский университет), 
\mbox{sofia\_mai@mail.ru}}
\footnotetext[2]{Московский авиационный институт (национальный исследовательский университет), 
\mbox{yu\_kan@mail.ru}}

%\vspace*{-6pt}



\Abst{Предлагается алгоритм по\-стро\-ения многогранной аппроксимации ядра 
вероятностной меры для двумерного случайного вектора с~независимыми компонентами. 
Ядро является одним из важ\-ных понятий, используемых в~алгоритмах решения задач 
стохастического программирования с~вероятностными критериями.  Наиболее эффективно 
ядро применяется в~случаях, когда постановки указанных задач имеют свойство 
линейности по отношению к~случайным параметрам.  
В~силу линейности  максимум по случайным параметрам определяется путем перебора 
всех вершин аппроксимирующего многогранника. Предложенный в~статье алгоритм основан 
на построении пересечения конечного чис\-ла доверительных полупространств, 
параметры которых оцениваются методом Мон\-те Кар\-ло. Результатом работы 
предложенного алгоритма является определение множества вершин аппроксимирующего 
многогранника. Аппроксимация ядра является их выпуклой оболочкой. 
Приводятся результаты расчетов для ряда типовых непрерывных законов распределения.}



\KW{задача квантильной оптимизации;  метод линеаризации; ядро вероятностной меры}

\DOI{10.14357/19922264180209}
  
\vspace*{-6pt}


\vskip 10pt plus 9pt minus 6pt

\thispagestyle{headings}

\begin{multicols}{2}

\label{st\stat}


\section{Введение} 

Ядро вероятностной меры заданного уровня~$\alpha$ для случайного вектора
 определяется как пересечение всех вы\-пук\-лых, замк\-ну\-тых, $\alpha$-до\-ве\-ри\-тель\-ных 
 множеств в~про\-стран\-ст\-ве реализаций этого случайного вектора. Это понятие 
 играет клю\-че\-вую роль в~тео\-ре\-ти\-че\-ских аспектах задач сто\-ха\-сти\-че\-ско\-го 
 программирования с~квантильным критерием качества~\cite{kankibzun}, называемых 
 ниже задачами квантильной оптимизации. Последние с~прикладной точки зрения 
 моделируют принятие решений в~условиях не\-опре\-де\-лен\-ности с~учетом риска или 
 требований   на\-деж\-ности.

Впервые $\alpha$-яд\-ро было введено в~рассмотрение в~\cite{malkib} для 
гауссовского случая с~целью доказательства асимптотической точ\-ности 
доверительного метода решения задач минимизации функции квантили при 
использовании доверительного множества в~виде эл\-лип\-со\-ида, яв\-ля\-юще\-го\-ся 
множеством уровня плот\-ности вероятности. Исследованию свойств ядра 
посвящены работы~\cite{kanrus, kansur}. Следует также отметить, что $\alpha$-яд\-ро 
может использоваться в~задачах опре\-де\-ле\-ния стар\-то\-вой точ\-ки для чис\-лен\-ных методов 
оптимизации функ\-ции квантили~\cite{kibkurb, kibkurben}.

Квантильный критерий пред\-став\-ля\-ет собой \mbox{$\alpha$-кван}\-тиль 
распределения некоторой функции потерь, зависящей от вектора оптимизируемой 
стратегии и~случайных па\-ра\-мет\-ров. Среди задач квантильной оптимизации можно
 выделить важный класс задач, в~которых функция потерь линейна по случайным 
 па\-ра\-мет\-рам.
В~этот класс включаются задачи оптимизации портфеля ценных бумаг с~учетом 
риска по квантильному критерию, впервые рас\-смот\-рен\-ные в~\cite{moeseke}.
В~этом случае задача минимизации квантильного критерия при выполнении 
некоторых условий регулярности ядра эквивалентна минимаксной 
задаче для функции потерь~\cite{kankibzun}, где внут\-рен\-ний максимум берется по реализациям 
случайных па\-ра\-мет\-ров, а~внеш\-ний минимум~--- по оптимизируемой стратегии. 

Таким образом, сто\-ха\-сти\-че\-ская задача квантильной оптимизации равносильна 
минимаксной со специально за\-да\-ва\-емым множеством неопределенности, по 
которому производится максимизация, и~это множество совпадает с~$\alpha$-яд\-ром. 

Эта минимаксная задача исследовалась в~\cite{vaskan} для случая, когда функция 
потерь является линейной и~по стратегии. В~этой же статье предложены общие 
алгоритмические схемы аппроксимации $\alpha$-яд\-ра\linebreak выпуклы\-ми многогранниками, 
с~использованием которых указанная минимаксная задача сводится к~задаче 
линейного программирования с~большим чис\-лом ограничений. 
%
На основе одной 
из этих схем и~разработан пред\-ла\-га\-емый ниже алгоритм, реализованный в~программном 
пакете MATLAB.

Во многих математических моделях прикладных задач встречается ситуация, 
когда случайные па\-ра\-мет\-ры являются в~некотором смысле малыми. В~этих случаях 
возможно использование метода линеаризации, точ\-ное описание которого изложено 
в~\cite{vaskan2}. В~соответствии с~этим методом исходную нелинейную функцию 
потерь можно линеаризовать по случайным па\-ра\-мет\-рам и~получить линейную по 
случайным па\-ра\-мет\-рам модель. Тем самым задача квантильной оптимизации с~нелинейной 
функцией потерь может быть приближенно аппроксимирована вышеупомянутой ми\-ни\-макс\-ной 
задачей, в~которой роль множества не\-опре\-де\-лен\-ности играет $\alpha$-ядро.

К настоящему времени мас\-штаб\-ные исследования гео\-мет\-рии $\alpha$-яд\-ра 
для типовых законов распределения не проводились. Предлагаемый ниже 
программно реализованный алгоритм предназначен для того, чтобы час\-тич\-но 
ликвидировать этот пробел.


В~разд.~2 вводятся основные определения и~утверж\-де\-ния, описывается 
рас\-смат\-ри\-ва\-емая задача кван\-тиль\-ной оптимизации, приведены некоторые свойства ядра, 
продемонстрирована роль ядра вероятностной меры в~задачах квантильной оптимизации. 
В~разд.~3 пред\-став\-ле\-но описание алгоритма по\-стро\-ения аппроксимации ядра. В~разд.~4 
представлены результаты рас\-че\-тов $\alpha$-ядер для двумерных логнормального 
и~экспоненциального рас\-пре\-де\-ле\-ний, по\-стро\-ен\-ные с~использованием пакета MATLAB.


\section{Постановка задачи и~предварительные результаты}

\noindent
\textbf{Определение~1.}\
{Множество $S_\alpha \subset \mathbb R^t$  называется  
\mbox{$\alpha$-до}\-ве\-ри\-тель\-ным 
множеством для случайного вектора~$\mathbf X$ с~реализациями $\mathbf x \hm\in 
\mathbb R^t$, если}
${\sf P}(\mathbf X \in S_\alpha)\geqslant \alpha\,.$

\smallskip

\noindent
\textbf{Определение~2.}
{$\alpha$-яд\-ро может быть определено как}
\begin{equation*}
K_\alpha=\bigcap\limits_{S\in \mathrm{E_\alpha} }S,
\end{equation*}
где $\mathrm{E_\alpha}$~--- семейство всех вы\-пук\-лых замк\-ну\-тых 
\mbox{$\alpha$-до}\-ве\-ри\-тель\-ных множеств.

\smallskip

Рассматривается задача о~построении границы $\alpha$-яд\-ра в~случае, когда век\-тор~$X$ 
имеет размерность $m\hm=2$  и~его компоненты независимы с~абсолютно 
непрерывным заданным законом рас\-пре\-де\-ле\-ния.

Поскольку для большинства распределений не удается аналитически 
по\-стро\-ить границу ядра, то вместо на\-хож\-де\-ния точ\-ной границы будем 
искать ее аппроксимацию.


Функцию $\Phi(\mathbf X,\mathbf u)$, зависящую от век\-то\-ра случайных 
па\-ра\-мет\-ров~$\mathbf X$ и~вектора стратегии $\mathbf u\hm\in U\subset \mathbb R^n$, 
будем называть функ\-ци\-ей потерь.


Введем необходимые определения и~утверждения из~\cite{kankibzun}.

\smallskip

\noindent
\textbf{Определение~3.}\
{Функция вероятности для функции потерь $\Phi(\mathbf X,\mathbf u)$ имеет вид}:
\begin{equation*}
 {\sf P}_\varphi (\mathbf u)\;{\stackrel{\Delta}{=}}\;
 {\sf P}\{\Phi(\mathbf X,\mathbf u)\leqslant \varphi\}\,,
 %\label{e2-vas}
\end{equation*}
{где ${\sf P}$~--- вероятность; $\varphi$~--- допустимый уровень потерь.}

\smallskip

\noindent
\textbf{Определение~4.}\
{Функция квантили для функции потерь $\Phi(\mathbf X,\mathbf u)$ 
определяется следующим образом:}
\begin{equation*}
\varphi_\alpha (\mathbf u)\;{\stackrel{\Delta}{=}}\;\min 
\left\{\varphi: {\sf P}_\varphi (\mathbf u)\geqslant \alpha \right\} 
\,{\stackrel{\Delta}{=}}\,[\Phi(\mathbf X,\mathbf u)]_\alpha\,.
%\label{e3-vas}
\end{equation*}
Задача квантильной оптимизации имеет вид:
\begin{equation}
[\Phi(\mathbf X,\mathbf u)]_\alpha\to \min\limits_{u\in U}\,.
\label{e4-vas}
\end{equation}
$\alpha$-ядро допускает следующее представление:
\begin{equation*}
K_\alpha \; = \bigcap\limits_{\| c\|=1}
\left \{z\in \mathbb R^t:c^\mathrm{T} z\leqslant 
\left[c^\mathrm{T} \mathbf X\right]_\alpha\right\}\,,
%\label{e5-vas}
\end{equation*}
 где $c$~--- вектор внешней нормали к~границе $\alpha$-до\-ве\-ри\-тель\-но\-го 
 полупространства; $\parallel \!\!\cdot\!\! \parallel$~--- евклидова норма.
Таким образом, $\alpha$-яд\-ро является пересечением всех замк\-ну\-тых 
$\alpha$-до\-ве\-ри\-тель\-ных полупространств.

Определим аппроксимацию $\alpha$-яд\-ра как пересечение конечного числа 
замкнутых $\alpha$-до\-ве\-ри\-тель\-ных полуплоскостей:
\begin{equation}
V_{\alpha N}=\bigcap\limits_{j=1}^{N} \left\{ 
\mathbf x: c_j^\mathrm{T} \mathbf  x \leqslant  
\left [ c_j^\mathrm{T} \mathbf X \right ]_\alpha\right \}.
\label{e6-vas}
\end{equation}

В случае использования точ\-но\-го значения квантили $\left [ \mathrm c_j^\mathrm{T} 
\mathbf X \right ]_\alpha$ полученный многогранник является внеш\-ней 
аппроксимацией $\alpha$-яд\-ра. В~общем случае явно определить эту величину 
не удается, поэтому для ее вы\-чис\-ле\-ния ниже, в~разд.~5, будет использована 
выборочная оценка квантили.

Рассмотрим функцию потерь вида:
\begin{equation}
\Phi(\mathbf X, \mathbf u)=a(\mathbf u)+b^{\mathrm{T}}(\mathbf u) \mathbf X\,.
\label{e7-vas}
\end{equation}

Введем следующее

\smallskip

\noindent
\textbf{Определение~5}~\cite{kankibzun}. $\alpha$-яд\-ро~$K_\alpha$ 
называется регулярным, если всякое замк\-ну\-тое полупространство, содержащее 
это ядро, автоматически является  \mbox{$\alpha$-до}\-ве\-ри\-тель\-ным.

\smallskip

\noindent
\textbf{Теорема 1}~\cite{kankibzun}.
 \textit{Если случайный вектор $\mathbf X$ имеет регулярное $\alpha$-ядро $K_\alpha$, 
 то для любого детерминированного вектора $\mathbf u \hm\in R^n$ справедливо}
\begin{equation*}
 \left [ \mathbf u^\mathrm{T} \mathbf X \right ]_\alpha=\max_{\mathbf x\in K_\alpha} \mathbf u^\mathrm{T} \mathbf x.
% \label{e8-vas}
\end{equation*}


В соответствии с~теоремой~1 задача квантильной оптимизации~(\ref{e4-vas}) 
для функ\-ции~(\ref{e7-vas}) 
может быть сведена к~задаче:
\begin{equation}
a (\mathbf{u}) + \max_{\mathbf{x}\in K_\alpha}\left(b^{\mathrm{T}}(\mathbf{u}) 
\mathbf{x}\right)\to \min\limits_{\mathbf {u} \in U}.
\label{e9-vas}
\end{equation}


После замены многогранника на его аппроксимацию получаем:
 \begin{equation*}
\varphi_\alpha^N (u) = a (\mathbf {u}) + \max\limits_{j=\overline{1,N}}
\left(b^{\mathrm{T}}(\mathbf {u}) \mathbf  {v}_j\right) 
\to \min\limits_{\mathbf {u}\in U}\,,
%\label{e10-vas}
\end{equation*}
где $v_j$~--- вершины $ V_{\alpha N}$,  $j\hm=\overline{1,N}.$ 
Функция $\varphi_\alpha^N (u)$ пред\-став\-ля\-ет собой оценку исходного 
квантильного критерия, при на\-хож\-де\-нии которой исходный максимум по 
случайному век\-то\-ру из множества~$K_\alpha$ заменен на максимум по 
его аппроксимации~$V_{\alpha N}.$ 

Максимум по вершинам выпуклого многогранного 
множества может быть найден как максимум из значений функции в~вершинах 
этого множества~$\max\nolimits_{j=\overline{1,N}}.$



Сходимость
\begin{equation*}
\min\limits_{u\in U} \varphi_\alpha^N (u)-\min\limits_{u\in U} \varphi_\alpha 
(u)\xrightarrow[{N\to \infty}]{} 0
%\label{e11-vas}
\end{equation*}
доказана в~\cite{vaskan} для случая, когда $a (\mathbf {u})$ и~$b(\mathbf {u})$ 
линейны по~$\mathbf {u}$ и~множество~$U$ компактно.


Задача~(\ref{e4-vas}) для функции потерь, нелинейно зависящей от век\-то\-ра  
случайных па\-ра\-мет\-ров, является слож\-но разрешимой. В~\cite{vaskan2} доказана 
спра\-вед\-ли\-вость\linebreak
 использования линеаризованной по случайным\linebreak
  па\-ра\-мет\-рам 
функции потерь в~случае малых случайных па\-ра\-мет\-ров. 
В~качестве вектора малых случайных па\-ра\-мет\-ров используется вектор~$\mathbf{X}^\mu$,  
со\-став\-лен\-ный из покоординатных произведений вектора малых па\-ра\-мет\-ров~$\mu$ 
и~случайного вектора~$\mathbf{X}$, т.\,е.\ $\mathbf{X}^\mu\hm=M\mathbf{X}$, 
где $M\hm=\mathrm{diag}\,(\mu_1, \mu_2, \ldots , \mu_t).$ 
Функция потерь имеет вид $\Phi(\mathbf {X}^\mu, \mathbf {u}).$ 
Разложение этой функции потерь в~ряд Тейлора по вектору $\mathbf {X}^\mu$
 в~окрест\-ности нуля справедливо, если функция является дваж\-ды непрерывно 
 дифференцируемой в~окрест\-ности нуля, а~так\-же непрерывна по $\mathbf {u}\hm\in U$ 
 вместе со своими част\-ны\-ми производными по~$\mathbf{X}^\mu$ до второго 
 порядка включительно. Разложение может быть пред\-став\-ле\-но следующим выражением:
\begin{equation*}
\Phi\left(\mathbf {X}^\mu, \mathbf {u}\right)= 
a(\mathbf {u})+b^{\mathrm{T}}(\mathbf {u})\mathbf {X}^\mu+r_1
\left(\mathbf {X}^\mu, \mathbf {u}\right)\,,
%\label{e12-vas}
\end{equation*}
где $a(\mathbf{u})\hm=\Phi(\mathbf {0},\mathbf {u})$~--- 
значение функции потерь в~нуле; $b (\mathbf {u})\hm=({\partial 
\Phi(\mathbf x^\mu,\mathbf u)}/{\partial \mathbf x^\mu})|_{\mathbf x^\mu=0}$~--- 
вектор значений покоординатных производных в~нуле;
  $r_1(\mathbf{X}^\mu, \mathbf {u})\hm= 
  (X_1^\mu \partial _1 \Phi(\theta \mathbf {X}^\mu, \mathbf {u})+\cdots+
  X_t^\mu \partial_t \Phi(\theta \mathbf {X}^\mu, \mathbf {u}))/2$~--- 
  остаточный член ряда Тейлора в~форме Ла\-гран\-жа,
  $\theta \hm\in(0,1)$; $\partial_i \Phi(\mathbf {X}^\mu, \mathbf{u})\hm=
  {\partial \Phi(\mathbf{X}^\mu, \mathbf {u})}/{\partial {\mathbf {X}_i}^\mu}$~--- 
  част\-ная производная функции $\Phi(\mathbf{X}^\mu, \mathbf{u})$ по $i$-й 
  координате случайного век\-то\-ра~$\mathbf {X}^\mu.$


Под линеаризованной моделью подразумевается линейная часть 
разложения исходной функ\-ции в~ряд Тейлора по случайным па\-ра\-мет\-рам в~окрест\-ности нуля. 
Линеаризованную модель мож\-но пред\-ста\-вить в~виде~(\ref{e7-vas}):
\begin{equation*}
\Phi_l\left(\mathbf {X}^\mu, \mathbf {u}\right)=
 a(\mathbf {u})+b^{\mathrm{T}}(\mathbf {u})\mathbf {X}^\mu\,.
 %\label{e13-vas}
\end{equation*}
Такая замена приводит к~следующему результату:
\begin{equation}
\!\!\min_{u\in U} \left[\Phi(\mathbf{X}^\mu, \mathbf {u})\right]_\alpha\!=\!
\min\limits_{u\in U} \left[\Phi_l\left(\mathbf {X}^\mu, \mathbf {u}\right)\right]_\alpha 
+O\left(||\mu||^2\right).\!
\label{e14-vas}
\end{equation}


Результат~(\ref{e14-vas}) обоснован в~работе~\cite{vaskan2} для случая, 
когда множество~$U$ является компактом, а~носитель случайного вектора~$X$ 
ограничен, и~для случая, когда множество~$U$ конечно, а~носитель вектора 
случайных па\-ра\-мет\-ров~$X$ неограничен.

Учитывая~(\ref{e9-vas}), заключаем, что с~точ\-ностью до $O(||\mu||^2)$ задача
\begin{equation*}
[\Phi(\mathbf {X},\mathbf {u})]_\alpha\to \min\limits_{u\in U}
%\label{e15-vas}
\end{equation*}
может быть аппроксимирована минимаксной задачей
\begin{equation*}
a(\mathbf {u})+\max\limits_{\mathbf {x}\in K_\alpha} b^{\mathrm{T}}
(\mathbf {u}) M \mathbf {x} \to \min\limits_{u\in U},
%\label{e16-vas}
\end{equation*}
если, конечно, ядро~$K_\alpha$ регулярно и~справедливо соотношение~(\ref{e14-vas}).

В данной статье рассматривается задача построения сколь угодно 
точ\-ной аппроксимации ядра многогранником.

\section{Алгоритм построения аппроксимации $\alpha$-ядра}

Для построения аппроксимации ядра сгенерируем выборку $X_1, X_2, \ldots, X_k$ 
случайного вектора~$\mathbf {X}.$
Рас\-смот\-рим вспомогательную линейную функцию потерь вида
$$
Z(c,\mathbf {X})=c^{\mathrm {T}} \mathbf{X}\,,
$$
где $c$~--- единичный вектор нормали, т.\,е.\ \mbox{$\parallel c\parallel \hm= 1.$}
С~учетом этого обозначения аппроксимация ядра~(\ref{e6-vas}) может быть записана в~виде:
\begin{equation*}
V_{\alpha N}=\bigcap\limits_{j=1}^{N} \left\{ \mathbf {x}: c_j^{\mathrm{T}} 
\mathbf{x} \leqslant  \left [ Z\left(c_j, \mathbf {X} \right) \right]_\alpha\right \}\,.
%\label{e17-vas}
\end{equation*}
Функцию квантили $[Z(c,\mathbf {X})]_\alpha$ будем оценивать методом Мон\-те Карло.

Используя сгенерированную ранее выборку случайного вектора~$\mathbf {X},$ 
по\-стро\-им выборку значений функ\-ции потерь $Z(c_j,\mathbf{X})$ по формуле:
$$
Z_i^j=c_j^{\mathrm{T}} \mathbf {X}_i\,,\enskip i=\overline{1,k}\,.
$$
Для каждого номера~$j$ построим вариационный ряд этой выборки:
$$
Z_{(1)}^j \leq Z_{(2)}^j\leq \dots \leq Z_{(k)}^j.
$$
Выборочная оценка $\widehat Z^j_\alpha$ функции квантили~$\left[Z(c_j,X)\right]_\alpha$  
вы\-чис\-ля\-ет\-ся по формуле  из~\cite{kankibzun}:
\begin{equation*}
\widehat Z^j_\alpha=
\begin{cases}
Z ^j_{(\alpha k)}\,, &\ \alpha k\in \mathbb{ N}\,;\\
Z ^j_{([\alpha k]+1)}\,, &\ \alpha k\notin \mathbb{ N}\,,
\end{cases}
\end{equation*}
где $[\cdot]$~--- целая часть от числа.

Точность выборочной оценки обоснована сле\-ду\-ющей тео\-ре\-мой~\cite{bakh}.

\smallskip

\noindent
\textbf{Теорема~2.}\ 
\textit{Если $\alpha\hm\in (0,1)$ и~случайная величина~$\Phi$  
имеет плот\-ность ве\-ро\-ят\-ности~$p(\varphi)$, непрерывную в~некоторой окрест\-ности 
точ\-ки~$\varphi_\alpha$, причем $ p(\varphi_\alpha)\hm>0,$ то}
\begin{equation*}
\widehat \Phi_\alpha-\varphi_\alpha=
\fr{\widehat {\sf P}(\varphi_\alpha)-\alpha}{p(\varphi_\alpha)}+o_p\left(k^{-1/2}\right)\,,
\end{equation*}
\textit{где $\widehat {\sf P}(\varphi_\alpha)$~--- 
значение выборочной оценки ве\-ро\-ят\-ности в~точ\-ке $\varphi_\alpha$;
 $\varphi_\alpha$~--- 
точ\-ное значение квантили; $k^{1/2}o_p(k^{-1/2})\hm\to 0$ по ве\-ро\-ят\-ности при
 $k\to \infty$.}
 
 \smallskip

Как уже отмечалось выше, аппроксимация \mbox{$\alpha$-яд}\-ра строится путем 
пересечения конечного чис\-ла по\-лу\-плос\-костей,
определяемых своими векторами нормалей. В~работе~\cite{vaskan} пред\-ло\-жен 
алгоритм, который
 позволяет построить гус\-тую сеть векторов~$c_j$, $j\hm=\overline{1,N}.$ 
 При этом векторы из этого множества в~двумерном случае
 образуют неравномерную сет\-ку на единичной окруж\-ности 
 с~центром в~начале координат.

Этот недостаток алгоритма из ~\cite{vaskan} можно устранить в~случае 
двумерного распределения случайного вектора. Новый алгоритм включает в~себя 
сле\-ду\-ющую по\-сле\-до\-ва\-тель\-ность шагов.
\begin{enumerate}[1.]
\item  Нанесем равномерную сетку из заданного чис\-ла~$N$ точек~$c_j$, $j\hm=\overline{1,N}$, 
на по\-верх\-ность единичной окруж\-ности.


\item  Сгенерируем выборку заданного объема~$n$ 
для рас\-смат\-ри\-ва\-емо\-го двумерного распределения случайного вектора~$\mathbf{X}.$

\item  Для каждого вектора нормали~$c_j$,  $j\hm=\overline{1,N}$, сформируем 
вариационный ряд $Z_{(1)}^j, Z_{(2)}^j, \ldots, Z_{(k)}^j$
и~построим выборочную оценку квантили~$[Z(c_j,\mathbf X)]_\alpha$
в~соответствии с~процедурой, описанной в~начале данного раздела.

\item  Найдем точки пересечения границ всех пар рас\-смат\-ри\-ва\-емых доверительных 
полуплоскостей $\{ \mathbf {x}: c_j^{\mathrm{T}} \mathbf {x} \hm\leqslant  
\widehat  {Z}^j_\alpha\}.$

\end{enumerate}

После этого производится проверка их принадлежности остальным по\-лу\-плос\-костям.
 Точки, удовле\-тво\-ря\-ющие всем ограничениям, яв\-ля\-ют\-ся
 вершинами аппроксимирующего многогранника.\linebreak
За счет использования выборочной оценки кван\-тили, не совпадающей в~общем 
случае с~точ\-ным значением квантили, чис\-ло
вершин многогранника может оказаться меньше~$N$.
Вершины аппроксимирующего многогранника отображаются на итоговом рисунке, 
со\-сед\-ние вершины соединяются прямолинейным отрезком.

\vspace*{-4pt}

\section{Программная реализация алгоритма визуализации {\boldmath{$\alpha$}}-ядра}

\vspace*{-2pt}

Для реализации алгоритма была использована программная среда MATLAB. 
Для удобства использования программы с~по\-мощью встро\-ен\-но\-го инструмента 
GUIDE был создан пользовательский интерфейс программы, вид которого 
пред\-став\-лен на рис.~1.

\begin{figure*} %fig1
 \vspace*{1pt}
 \begin{center}
 \mbox{%
 \epsfxsize=163.202mm  
 \epsfbox{vas-1.eps}
 }
 \end{center}
\vspace*{-11pt}
\Caption{Пример работы программы: (\textit{а})~логнормальное распределение, %\textit{1}~--- 
$\alpha\hm=0{,}9$; (\textit{б})~экспоненциальное распределение %:
при $\alpha\hm= 0{,}8$, 0,9 и~0,99}
\vspace*{5pt}
\end{figure*}




В окне программы пользователь имеет воз\-мож\-ность выбрать или ввести 
сле\-ду\-ющие параметры: вид рас\-пре\-де\-ле\-ния (среди предложенных); 
па\-ра\-мет\-ры 
(па\-ра\-метр) распределения; объем выборки; чис\-ло доверительных полуплоскостей; 
чис\-ло аппроксимаций; па\-ра\-метр~$\alpha$ для каждой из аппроксимаций. 
В~меню выбора доступны для выбора сле\-ду\-ющие распределения: нормальное; равномерное; 
логнормальное; экспоненциальное; распределение Коши. На одном рисунке может 
быть по\-стро\-ено не более трех аппроксимаций. Виды линий границы указаны в~легенде.
Для удоб\-ст\-ва все необходимые данные изначально заполнены.
Для начала работы алгоритма программы необходимо нажать кнопку 
<<Build approximation>>,
после чего справа от меню появится рисунок с~выбранным пользователем 
чис\-лом аппроксимаций.
После этого па\-ра\-мет\-ры в~об\-ласти меню могут быть изменены.
После нажатия кноп\-ки 
<<Build approximation>> будет показан рисунок, по\-стро\-ен\-ный для новых па\-ра\-мет\-ров.

На рис.~1,\,\textit{б} показано окно программы. На рисунке построены три аппроксимации 
ядер для различных заданных~$\alpha$. В~качестве распределения выбрано 
экспоненциальное рас\-пре\-де\-ле\-ние с~па\-ра\-мет\-ром $\lambda\hm=1.$


Для  генерации выборки распределения используются стандартные функции  из пакета 
MATLAB Statistics Toolbox.

Для удобства была создана функция Kernel, которая по заданным па\-ра\-мет\-рам находит  
аппроксимацию и~возвращает мас\-сив значений. На вход необходимо подать сле\-ду\-ющие 
данные: массив реа-\linebreak\vspace*{-12pt}

\pagebreak

\noindent
ли\-за\-ций случайной величины, уровень~$\alpha$, объем выборки,
 чис\-ло полуплоскостей.

После генерации точек всех ядер, используется стандартная функция plot, которая 
по точ\-кам строит границу ядра.

\vspace*{-3pt}

\section{Результаты расчетов}

\vspace*{-2pt}

В данном разделе представлены результаты для экспоненциального и~логнормального 
распределений компонент случайного вектора~$\mathbf{X}.$ На рис.~2--5 точками
 обозначены вершины многогранников, аппроксимирующих $\alpha$-яд\-ро. 
 Как было отмечено выше, чис\-ло вершин
не превосходит~$N.$


С помощью описанной выше программы построим $\alpha$-яд\-ра для объема выборки~$10^6$ 
для $\alpha\hm=0{,}9.$
Результаты работы алгоритма построения аппроксимации $\alpha$-яд\-ра при 
$N\hm=8$, 16 и~70 для случая, когда
 компоненты случайного вектора~$\mathbf{X}$ независимы и~$X_i\sim \log
 N(0,1)$, $i\hm=1,2,$
показаны на рис.~2.






Эмпирически установлено, что для выборок порядка $k=10^6$ и~выше
выборочная оценка достаточно точ\-на, поэтому уменьшения чис\-ла
вер\-шин аппроксимирующего многогранника не наблюдается.


Аналогичные результаты по\-стро\-ения аппроксимации $\alpha$-яд\-ра при $N=8$, 16 и~70 
для случая, когда
компоненты случайного вектора~$\mathbf X$ независимы и~$X_i\hm\sim E(1)$, $i\hm=1,2$,
показаны на рис.~3.

\begin{figure*} %fig2
\vspace*{1pt}
\begin{minipage}[t]{80mm}
 \begin{center}
 \mbox{%
 \epsfxsize=65.102mm 
 \epsfbox{vas-3.eps}
 }
 \end{center}
\vspace*{-9pt}
\Caption{Логнормальное распределение,  $k\hm=10^6$, $\alpha\hm = 0{,}9$:
(\textit{а})~$N\hm=8$; (\textit{б})~16;
(\textit{в})~$N\hm=70$}
\end{minipage}
%\end{figure*}
\hfill
%\begin{figure*} %fig3
\vspace*{1pt}
\begin{minipage}[t]{80mm}
 \begin{center}
 \mbox{%
 \epsfxsize=64.929mm 
 \epsfbox{vas-6.eps}
 }
 \end{center}
\vspace*{-9pt}
\Caption{Экспоненциальное распределение,  $k\hm=10^6$, $\alpha \hm= 0{,}9$:
(\textit{а})~$N\hm=8$; (\textit{б})~16; (\textit{в})~$N\hm=70$}
\end{minipage}
\end{figure*}


Для экспоненциального распределения при объеме выборки порядка $k\hm=10^6$ и~выше
 выборочная оценка кван\-ти\-ли также достаточно\linebreak точна.

Разработанный программный модуль позволяет строить на одном рисунке одновременно 
две или три аппроксимации для различных значений~$\alpha.$ На рис.~4 
продемонстрированы границы ядер для $\alpha\hm\in\{0,7; 0,9; 0,99\}.$




На рис.~5 показано, что при уменьшении объема выборки в~случае 
логнормального распределения при $\alpha\hm=0{,}9$
чис\-ло вершин аппроксимирующего многогранника уменьшается.


В частности, при $k\hm=10^4$ чис\-ло вершин аппроксимирующего многогранника 
уменьшилось на~20.


При использовании выборки объема $k\hm=10^3$ снижение 
точ\-ности выборочной оценки квантили приводит к~тому,
что остается всего лишь~38~вершин многогранника из~70~возможных.

За счет того, что процедура построения выборочной оценки квантили 
производится для всех единичных векторов нормали и~пересечение ограничений, 
задающих доверительные по\-лу\-плос\-кости, производится попарно, приходится 
производить значительную часть расчетов и~операций, не влия\-ющих на конечный 
результат. В~случае, когда объем выборки увеличить невозможно или пользователю 
необходимо увеличить ско\-рость работы алгоритма, рекомендуется уменьшить чис\-ло 
пересекаемых полупространств. Чис\-ло вершин многогранника при этом уменьшится, 
но незначительно. Если пользователю необходимо увеличить точ\-ность аппроксимации, 
то необходимо увеличить па\-ра\-метр~$n$ для повышения точ\-ности выборочной оценки 
кван\-тили.



Соотношения параметров~$\alpha,$ $N$ и~$k$ не могут быть заданы априори, 
поскольку они зависят как от вида распределения, так и~от параметров распределения. 
Число вершин многогранника является случайной величиной. Для рас\-смат\-ри\-ва\-емо\-го 
распределения при фиксированном~$\alpha$ могут быть по-\linebreak\vspace*{-12pt}

\pagebreak

\end{multicols}

\begin{figure*} %fig4
\vspace*{1pt}
 \begin{center}
 \mbox{%
 \epsfxsize=161.751mm 
 \epsfbox{vas-9.eps}
 }
 \end{center}
\vspace*{-9pt}
\Caption{Экспоненциальное~(\textit{а}) 
и~логнормальное~(\textit{б}) распределения, 70~точек:
\textit{1}~--- $\alpha\hm=0{,}7$; \textit{2}~--- 0,9; \textit{3}~--- 
$\alpha\hm= 0{,}99$}
%\end{figure*}
%\begin{figure*} %fig5
\vspace*{6pt}
 \begin{center}
 \mbox{%
 \epsfxsize=162.254mm 
 \epsfbox{vas-11.eps}
 }
 \end{center}
\vspace*{-9pt}
\Caption{Логнормальное распределение, 70~точек,  $\alpha\hm= 0{,}9$:
(\textit{а})~выборка~$10^4$; (\textit{б})~выборка~$10^3$}
\end{figure*}

\begin{multicols}{2}

\noindent
стро\-ены выборочные оценки 
при фиксированных
 па\-ра\-мет\-рах~$N$ и~$k$   для числа вершин (например, выборочного 
среднего, выборочной дис\-пер\-сии и~т.\,д.) в~за\-ви\-си\-мости от па\-ра\-мет\-ров~$N$ и~$k.$

\vspace*{-9pt}

\section{Заключение }

Предложенный алгоритм позволяет более точно (по сравнению с~методом, предложенным 
в~\cite{vaskan}) аппроксимировать
 $\alpha$-яд\-ро. Этот эффект достигается за счет того, что используется 
 равномерная сеть точек на окруж\-ности, за\-да\-ющих
 векторы нор\-малей.
 {\looseness=1
 
 }

На рисунках показано, как уменьшается число точек многогранника при 
уменьшении объема выборки распределения случайного вектора.

При решении практических задач при заданном~$\alpha$ па\-ра\-мет\-ры~$N$ и~$k$  
необходимо выбирать так, чтобы чис\-ло вершин многогранника было как можно ближе к~$N.$

Помимо графических иллюстраций разработанный программный модуль позволяет 
построить\linebreak сис\-те\-му линейных ограничений, за\-да\-ющих аппроксимацию $\alpha$-яд\-ра, 
которую можно использовать в~алгоритмах решения задач с~квантильным критерием 
качества. Кроме того, определяются координаты вер\-шин многогранника, 
удовле\-тво\-ря\-ющие данной сис\-те\-ме ограничений.



{\small\frenchspacing
 {%\baselineskip=10.8pt
 \addcontentsline{toc}{section}{References}
 \begin{thebibliography}{99}

\bibitem{kankibzun}
\Au{Кибзун А.\,И., Кан~Ю.\,С.}
Задачи стохастического программирования с~вероятностными критериями.~--- 
М.: Физматлит, 2009. 372~с.

\bibitem{malkib}
\Au{Малышев В.\,В., Кибзун~А.\,И.}
Анализ и~синтез высокоточного управления летательными аппаратами.~--- 
М.: Машиностроение, 1987. 303~с.

\bibitem{kanrus}
\Au{Кан Ю.\,С., Русяев~А.\,В.}
Задача квантильной минимизации с~билинейной функцией потерь~// Автоматика
и~телемеханика, 1998. 
№\,7. С.~67--75.

\bibitem{kansur} %4
\Au{Кан Ю.\,С., Суринов~Р.\,Т.}
О~неравенстве треугольника  для критерия VaR~//  
Моделирование и~анализ без\-опас\-ности и~рис\-ка в~слож\-ных сис\-те\-мах: 
Тр. Междунар. научной школы МАБР-2004.~--- СПб.: СПбГУАП,  2004. С.~243--248.



\bibitem{kibkurben} %5
\Au{Kibzun A.\,I., Kurbakovskiy V.\,Yu.}
Guaranteeing approach to solving quantile optimization problem~// 
Ann. Oper. Res., 1991. Vol.~30. P.~81--93.

\bibitem{kibkurb} %6
\Au{Кибзун А.\,И., Курбаковский~В.\,Ю.}
Численные алгоритмы квантильной оптимизации и~их применение к~решению 
задач с~вероятностными ограничениями~// Изв. РАН, Техническая кибернетика, 
1992. №\,1. С.~75--81.

\bibitem{moeseke}
{\it Van Moes$\Grave{\mbox{e}}$ke P.}
Stochastic linear programming~// Yale Econ. Essays, 1965. Vol.~5. P.~197--253.

\bibitem{vaskan}
\Au{Васильева С.\,Н.,  Кан~Ю.\,С.}
Метод решения задачи квантильной оптимизации с~билинейной функцией потерь~// Автоматика
и~телемеханика, 
2015. №\,9. С.~83--101.

\bibitem{vaskan2}
\Au{Васильева С.\,Н.,  Кан~Ю.\,С.}
Метод линеаризации для решения задачи квантильной оптимизации с~функцией потерь, 
зависящей от вектора малых случайных параметров~// 
Автоматика
и~телемеханика, 2017. №\,7. С.~95--109.

\bibitem{bakh}
\Au{Bahadur R.\,R.}
A~note on quantiles in large samples~// Ann. Math. Stat., 
1996. Vol.~37. P.~577--580.

 \end{thebibliography}

 }
 }

\end{multicols}

\vspace*{-6pt}

\hfill{\small\textit{Поступила в~редакцию 26.04.17}}

\vspace*{6pt}

%\newpage

%\vspace*{-24pt}

\hrule

\vspace*{2pt}

\hrule

%\vspace*{8pt}


\def\tit{A~VISUALIZATION ALGORITHM FOR~THE~PLANE PROBABILITY MEASURE KERNEL}

\def\titkol{A~visualization algorithm for~the~plane probability measure kernel}

\def\aut{S.\,N.~Vasil'eva and Yu.\,S.~Kan}

\def\autkol{S.\,N.~Vasil'eva and Yu.\,S.~Kan}

\titel{\tit}{\aut}{\autkol}{\titkol}

\vspace*{-9pt}


 \noindent
Moscow Aviation Institute (National Research University), 4~Volokolamskoe Shosse, 
Moscow 125993, Russian Federation 


\def\leftfootline{\small{\textbf{\thepage}
\hfill INFORMATIKA I EE PRIMENENIYA~--- INFORMATICS AND
APPLICATIONS\ \ \ 2018\ \ \ volume~12\ \ \ issue\ 2}
}%
 \def\rightfootline{\small{INFORMATIKA I EE PRIMENENIYA~---
INFORMATICS AND APPLICATIONS\ \ \ 2018\ \ \ volume~12\ \ \ issue\ 2
\hfill \textbf{\thepage}}}

\vspace*{3pt}



 \Abste{The authors propose an algorithm for constructing a~probability 
 measure kernel polyhedral approximation for a two-dimensional random vector 
 with independent components. The kernel is one of the important concepts 
 used in algorithms for solving stochastic programming problems with probabilistic 
 criteria. The kernel is most effectively used in cases when the statements of 
 the indicated problems have the property of linearity with respect to random 
 parameters. Because of linearity, the maximum in random parameters is determined 
 by searching all vertices of the approximating polyhedron. The authors propose 
 an algorithm for constructing a polyhedral approximation of the kernel of 
 a~probability measure for a two-dimensional random vector with independent 
 components. The algorithm is based on  construction of the intersection of 
 a~finite number of confidence half-spaces, the parameters of which are estimated 
 by the Monte-Carlo method. The result of the proposed algorithm 
 is the definition of the set of vertices of the approximating polyhedron. 
 Approximation of the nucleus is their convex hull. The results of calculations for 
a~number of typical continuous distribution laws are presented.}

\KWE{quantile optimization problem; linearization method; probability measure kernel}



\DOI{10.14357/19922264180209} %

%\vspace*{-14pt}

\Ack
\noindent
The work was supported by the Russian Ministry of Education and Science (Project 
No.\,2.2461.2017/PCh) and by the Russian Foundation for Basic Research (grant 
No.\,15-08-02833a).

\pagebreak



%\vspace*{-3pt}

  \begin{multicols}{2}

\renewcommand{\bibname}{\protect\rmfamily References}
%\renewcommand{\bibname}{\large\protect\rm References}

{\small\frenchspacing
 {%\baselineskip=10.8pt
 \addcontentsline{toc}{section}{References}
 \begin{thebibliography}{99}

\bibitem{1-kan}
\Aue{Kibzun, A.\,I., and Yu.\,S.~Kan.} 2009. 
\textit{Zadachi sto\-kha\-sti\-che\-skogo programmirovaniya 
s~veroyatnostnymi kriteriyami}  
[Stochastic programming problems with probabilistic criteria]. Moscow: Fizmatlit.  372~p.
\bibitem{2-kan} 
\Aue{Malyshev, V.\,V., and A.\,I.~Kibzun.} 1987. 
\textit{Analiz i~sintez vysokotochnogo upravleniya 
letatel'nymi apparatami}  [Analysis and synthesis of high-precision aircraft control].
 Moscow: Mashinostroenie.  303~p.
\bibitem{3-kan} 
\Aue{Kan, Yu.\,S., and A.\,V.~Rusyaev.} 1998. 
Quantile minimization with bilinear loss function.
\textit{Automat.  Rem. Contr.} 59(7(1)):960--966.
\bibitem{4-kan}
\Aue{Kan, Yu.\,S., and R.\,T.~Surinov.} 2004.  
O~neravenstve treugol'nika  dlya kriteriya 
VaR [On the triangle inequality for the VAR criterion].
\textit{Modeling and Analysis of Safety and Risk in Complex Systems: Scientific 
 School (International) MASR-2004 Proceedings}. St.\ Petersburg:
SPbGUAP. 243--248.

\bibitem{6-kan}  %5
\Aue{Kibzun, A.\,I., and V.\,Yu.~Kurbakovskiy.} 1991. 
Guaranteeing approach to solving quantile optimization problem.
\textit{Ann. Oper. Res.} 30:81--93.

\bibitem{5-kan}  %6 
\Aue{Kibzun, A.\,I., and V.\,Yu.~Kurbakovskij.} 1992.
 Chislennye algoritmy kvantil'noy optimizatsii i~ikh primenenie k~re\-she\-niyu zadach 
 s~veroyatnostnymi ogranicheniyami [Numerical algorithms for quantitative 
 optimization and their application to solving problems with probability constraints].
 \textit{J.~Comput. Sys. Sc. Int.}  1:75--81.
 
\bibitem{7-kan}
\Aue{Van Moes$\Grave{\mbox{e}}$ke, P.} 1965. Stochastic linear programming.
\textit{Yale Econ. Essays} 5:197--253.
\bibitem{8-kan} 
\Aue{Vasil'eva, S.\,N., and Yu.\,S.~Kan.} 2015.
 A~method for solving quantile optimization problems with a~bilinear loss function.
 \textit{Automat. Rem. Contr.} 76(9):1582--1597.
\bibitem{9-kan} 
\Aue{Vasil'eva, S.\,N., and Yu.\,S.~Kan}. 2017. 
Linearization method for solving quantile optimization problems 
with loss function depending on a~vector of small random parameters.
\textit{Automat. Rem. Contr.} 78(7):1248--1260.
\bibitem{10-kan} 
\Aue{Bahadur, R.\,R.} 1996. A~note on quantiles in large samples.
\textit{Ann. Math. Stat.} 37:577--580.
\end{thebibliography}

 }
 }

\end{multicols}

\vspace*{-3pt}

\hfill{\small\textit{Received April 26, 2017}}

%\vspace*{-24pt}

\Contr


\noindent
\textbf{Vasil'eva Sofia N.} (b.\ 1993)~--- 
PhD student, Moscow Aviation Institute (National Research University), 
4~Volokolamskoe  Shosse, Moscow 125993, Russian Federation; 
\mbox{sofia\_mai@mail.ru}

\vspace*{3pt}

\noindent
\textbf{Kan Yuri S.} (b.\ 1960)~--- 
Doctor of Science in physics and mathematics, 
Professor, Moscow Aviation Institute (National Research University), 
4~Volokolamskoe  Shosse, Moscow 125993, Russian Federation; \mbox{yu\_kan@mail.ru}

\label{end\stat}


\renewcommand{\bibname}{\protect\rm Литература}  %9
\def\stat{maniakov}

\def\tit{ТЕКСТУРИРОВАНИЕ ВОКСЕЛЬНЫХ МОДЕЛЕЙ НА ОСНОВЕ
ЦВЕТОВОЙ ИНФОРМАЦИИ ОБ ОПОРНЫХ ТОЧКАХ}

\def\titkol{Текстурирование воксельных моделей на основе
цветовой информации об опорных точках}

\def\aut{О.\,П.~Архипов$^1$, Ю.\,А.~Маньяков$^2$}

\def\autkol{О.\,П.~Архипов, Ю.\,А.~Маньяков}

\titel{\tit}{\aut}{\autkol}{\titkol}

\renewcommand{\thefootnote}{\arabic{footnote}}
\footnotetext[1]{Орловский филиал Института проблем информатики Российской академии наук, arkhipov12@yandex.ru}
\footnotetext[2]{Орловский филиал Института проблем информатики Российской академии наук,
maniakov\_yuri@mail.ru}

\vspace*{3pt}


\Abst{Описана технология текстурирования воксельных моделей на основе
имеющейся цветовой информации об опорных точках (ОТ). Опорные точки представляют
собой особые точки в трехмерном пространстве, описывающие определенную часть
поверхности объекта и содержащие обобщенную информацию о цвете соответствующей
им области.
  Цвет воксела в области, ограниченной ОТ, определяется цветом каждой
из ОТ, ограничивающих эту область. Вес цветовой со\-став\-ля\-ющей каждой
ОТ при расчете цвета воксела обратно пропорционален евклидову расстоянию
от воксела до нее. Пред\-став\-лен механизм расчета весовых коэффициентов
цветовых компонент ОТ для вычисления цветов вокселов, зависящих от
рас\-сто\-яния до каждой ОТ, ограничивающей область. Приведены описание и
результаты экспериментов, в рамках которых осуществлялось текстурирование
синтетических моделей и моделей натурных объектов.}

\KW{трехмерная реконструкция; текстурирование; воксельная модель}

\DOI{10.14357/19922264140311}

\vspace*{16pt}

\vskip 14pt plus 9pt minus 6pt

\thispagestyle{headings}

\begin{multicols}{2}

\label{st\stat}



    В современных условиях роста популярности медиатехнологий и
востребованности компьютерной графики решение проблемы быстрого
создания трехмерных моделей натурных объектов и их анимации является
одной из наиболее актуальных задач. Трехмерная реконструкция является
одним из перспективных направлений в области компьютерной графики с
точки зрения повышения скорости и уменьшения сложности процесса
трехмерного моделирования.

   Результатом применения трехмерной реконструкции являются
трехмерные модели объектов, которые могут быть представлены в
различных форматах.
%
Одним из наиболее перспективных способов
представления трехмерных моделей является воксельное представление.
Воксельные модели, в частности, позволяют легко отображать деформации и
разрушения объектов, а также их внутреннюю структуру, что может быть
весьма полезно при создании анимации различных взаимодействий объектов.

  Одним из этапов трехмерной реконструкции является текстурирование.
В~случае с воксельными моделями этот термин подразумевает присвоение
цветов вокселам трехмерной модели. На сегодняшний день большинство
технологий текстурирования подразумевают достаточно большое количество
творческой, сложно автоматизируемой работы.

  Перспективной является технология, основанная на понятии
<<мегатекстура>>, которая позволяет работать с воксельными объектами и
используется в игровом графическом движке id~Tech~4 и id~Tech~5~[1].
Данная технология относится к методике распределения текстур, когда вся
поверхность трехмерной сцены покрывается одной большой текстурой с
заданным уровнем детализации вместо множества мелких. Однако такой
подход используется в основном для текстурирования достаточно прос\-тых
поверхностей, например ландшафтов, и не подходит для объектов со
сложной геометрией.

  Целью настоящей работы является разработка технологии
текстурирования воксельной модели, позволяющей на основе цветовой
информации, содержащейся в ОТ, производить перенос цветовой
информации с изображений натурного объекта на его воксельную модель.

  Опорную точку можно представить в виде кортежа
  $  \langle x, y, z, r, g, b\rangle,$
где ($x, y, z$)~--- координаты ОТ в трехмерном пространстве, $(r, g, b)$~---
цветовые координаты ОТ в пространстве RGB (red, green, blue).
{\looseness=1

}

  Опорные точки генерируются на основе смещенных изображений,
получаемых в результате круговой съемки объекта, что позволяет создать
замкнутое описание поверхности модели объекта в трехмерном
пространстве~[2]. При этом цветом ОТ является цвет соответствующего ей
сегмента, полученного в результате аппроксимации на основе
сгенерированной палитры различимых цветов.

  Технология построения воксельной модели~[3] подразумевает, что на
основе известных координат ОТ в трехмерном пространстве производится
интерполяция координат вокселов, заключенных между ОТ. Таким образом,
единичной областью интерполяции как декартовых координат вокселов
модели, так и их цветовых координат в пространстве RGB является область,
ограниченная пространственным треугольником, образованным тремя
взаимно ближайшими (ограничива\-ющи\-ми)~ОТ.

  Цвет воксела в области, ограниченной ОТ, определяется цветом каждой из
ОТ, ограничивающих эту область. Вес цветовой составляющей каждой ОТ
при расчете цвета воксела обратно пропорционален евклидову расстоянию от
воксела до нее. Таким образом, необходимо вычислить коэффициенты
пропорциональности цветовых значений каждого воксела расстоянию до
  $i$-й ОТ, ограничивающей область:
  $$ k_i=\fr{1}{dd_i}\,,\quad d=\sum\limits_{i=1}^3 \fr{1}{d_i}\,,
  $$
где $d_i$~--- евклидово расстояние от рассматриваемого воксела до $i$-й ОТ.

  Сумма произведений соответствующих коэффициентов и цветовых
составляющих каждой ограничивающей ОТ образует цвет каждого воксела.

  Таким образом, зависимость, описывающая способ вычисления цвета
воксела в области, ограниченной ОТ, может быть представлена в виде:
  $$
  f(c)=\sum\limits_{i=1}^3 c_i k_i\,,
  $$
где $c_i$~--- значение цвета $i$-й ограничивающей ОТ:
$$
c=\begin{bmatrix}
r\\[-2pt]
g\\
b
\end{bmatrix}\,.
$$

  В результате получим текстурированную воксельную модель, которую
можно формально представить в виде совокупности вокселов:
  $$
  V_T= \mathop{\bigcup}\limits_{i=1}^m v_i\,,
  $$
где $m$~--- количество вокселов модели;
$v$~--- воксел модели; представляющий собой структуру вида
$v = \langle x,y,z,r,g,b\rangle,$
в которой ($x, y, z$)~--- координаты воксела в трехмерном пространстве,
$(r, g, b)$~--- координаты цвета воксела в пространстве RGB.

{\small \begin{center}
\begin{tabular}{|c|r|r|c|r|r|r|}
\multicolumn{7}{c}{Координаты ОТ контрольной модели}\\[6pt]
\hline
№ ОТ&$X$&$Y$&$Z$&$R$&$G$&$B$\\
\hline
1&10&$-39$&0&255&0&0\\
2&$-10$&17&0&0&255&0\\
3&14&17&0&0&0&255\\
\hline
\end{tabular}
\end{center}
}
%\end{table*}

\vspace*{9pt}

  На основе представленной модели был создан алгоритм и реализующий
его программный\linebreak модуль, а также были проведены экспериментальные
исследования. Экспериментальные исследования про\-водились на
совокупности тестовых воксель\-ных моделей. В~ходе исследования
осуществлялось текстурирование данных моделей с использованием
информации об ОТ, на основе которых они были построены, в
соответствии с технологией пофрагментного анализа и представления
натурного объекта и изменения его пространственного положения с
последующей интеграцией полученных данных в трехмерные
изображения~[4].

  На начальном этапе с целью повышения точности проверки корректности
работы алгоритма использовалась контрольная воксельная модель,
представляющая собой пространственный треугольник с расположенными в
вершинах ОТ разных цветов, координаты которых представлены в таблице.


%\begin{table*}\small


  Результат текстурирования модели, пред\-став\-лен\-ный на рис.~1, показывает,
что цвета составля-\linebreak

\vspace*{3pt}

\noindent
\begin{center}  %fig1
\mbox{%
\epsfxsize=60mm %78mm
\epsfbox{man-1.eps}
}
  \end{center}

%  \vspace*{3pt}

\noindent
{{\figurename~1}\ \ \small{Визуализация текстурированной контрольной модели}}


%\vspace*{8pt}

\addtocounter{figure}{1}

\end{multicols}

\begin{figure} %fig2
\vspace*{1pt}
\begin{center}
\mbox{%
\epsfxsize=160mm
\epsfbox{man-2.eps}
}
\end{center}
\vspace*{-6pt}
\Caption{Тестовые воксельные модели}
%\end{figure*}
%\begin{figure*} %fig3
\vspace*{24pt}
\begin{center}
\mbox{%
\epsfxsize=160mm
\epsfbox{man-3.eps}
}
\end{center}
\vspace*{-9pt}
\Caption{Визуализация текстурированных тестовых воксельных моделей:
(\textit{а})~текстурирование синтетической воксельной модели;
(\textit{б})~текстурирование модели натурного объекта}
\end{figure}

\begin{multicols}{2}


\noindent
ющих ее вокселов образуют плавный градиентный
переход между цветами ОТ, расположенными в вершинах
треугольника.

  В связи с тем, что алгоритм текстурирования работает с совокупностью
пространственных треугольников, аппроксимирующих трехмерную \mbox{модель},
учитывая корректность текстурирования контрольной модели, можно
сделать вывод о корректности работы алгоритма и модуля текстурирования
воксельных моделей. Это подтверждают исследования с использованием
других тестовых моделей, представленных на рис.~2.



  В результате текстурирования рассмотренных моделей были получены
результаты, представленные на рис.~3.
  При этом стоит уточнить, что источником исходной информации для
генерации ОТ для тестовой модели натурного объекта (рис.~3,\,\textit{б}) в
соответствии с технологией представления натурного объекта и изменения
его пространственного положения~[4] являлись фотоснимки натурного
объекта, один из которых представлен на рис.~4.

	Таким образом, можно заключить, что рассмотренная технология
текстурирования позволяет с  необходимой точностью получить отображение\linebreak

\noindent
\begin{center}  %fig1
\mbox{%
\epsfxsize=78mm
\epsfbox{man-4.eps}
}
  \end{center}

%  \vspace*{3pt}

\noindent
{{\figurename~4}\ \ \small{Изображение исходного натурного объекта для создания тестовой модели
(см.\ рис.~3,\,\textit{б})}}


\vspace*{8pt}

%\addtocounter{figure}{1}


\noindent
цветов поверхности натурного объекта на воксельной модели при условии
ограниченного объема информации о цветах ОТ. Необходимая точность
текстурирования достигается путем варьирования уровня детализации
модели, который, в свою очередь, определяется количеством ОТ.

{\small\frenchspacing
 {%\baselineskip=10.8pt
 \addcontentsline{toc}{section}{References}
 \begin{thebibliography}{9}
\bibitem{1-man}
Использование мегатекстур (megatexture, clipmaps)~// Gamedev.ru. 9~ноября
2007. {\sf http://www.gamedev.ru/ code/articles/Megatexture}.
\bibitem{2-man}
\Au{Архипов О.\,П., Маньяков Ю.\,А., Сиротинин~Д.\,О.} Метод генерации
виртуальной сетки опорных точек на цветных изображениях~//
Информационные технологии в науке, образовании и производстве
(ИТНОП-2012): Мат-лы V Междунар. науч.-технич. конф.~--- Орел:
Госуниверситет-УНПК, 2012. CD-ROM.
\bibitem{3-man}
\Au{Маньяков Ю.\,А.} Технология регистрации поведения объектов в
трехмерном пространстве~// Информационные технологии в науке,
образовании и производстве (ИТНОП-2010): Мат-лы IV Междунар.
науч.-технич. конф.~--- Орел: ОрелГТУ, 2010. Т.~3. С.~182--186.
\bibitem{4-man}
\Au{Архипов О.\,П., Маньяков Ю.\,А., Сиротинин~Д.\,О.} Информационная
модель технологии представления натурного объекта и изменения его
пространственного положения~// Информатика и её применения, 2014. Т.~8.
Вып.~1. С.~71--76.


 \end{thebibliography}

 }
 }

\end{multicols}

\vspace*{-9pt}

\hfill{\small\textit{Поступила в редакцию 03.06.14}}

%\newpage

\vspace*{12pt}

\hrule

\vspace*{2pt}

\hrule

%\vspace*{12pt}

\def\tit{VOXEL MODELS TEXTURING BASED ON~REFERENCE POINTS'
COLOR INFORMATION}

\def\titkol{Voxel models texturing based on~reference points'
color information}

\def\aut{O.\,P.~Arkhipov and Y.\,A.~Maniakov}

\def\autkol{O.\,P.~Arkhipov and Y.\,A.~Maniakov}

\titel{\tit}{\aut}{\autkol}{\titkol}

\vspace*{-9pt}

\noindent
Orel Branch of Institute of Informatics Problems, Russian Academy of Sciences,
137 Moskovskoe Highway, Orel 302025, Russian Federation


\def\leftfootline{\small{\textbf{\thepage}
\hfill INFORMATIKA I EE PRIMENENIYA~--- INFORMATICS AND
APPLICATIONS\ \ \ 2014\ \ \ volume~8\ \ \ issue\ 3}
}%
 \def\rightfootline{\small{INFORMATIKA I EE PRIMENENIYA~---
INFORMATICS AND APPLICATIONS\ \ \ 2014\ \ \ volume~8\ \ \ issue\ 3
\hfill \textbf{\thepage}}}

\vspace*{6pt}

\Abste{The article describes a technology for texturing voxel models based on color information
contained in reference points. Reference points are
special points in three-dimensional space which describe a definite part of an object
surface and contain generalized color information about a corresponding area.
  The color of each voxel located in the area bounded by reference points is
defined by each reference point bounding this area. The color components weights
of each reference point are proportional to Euclidean distance from each voxel to
these reference points. The article presents a method for calculating
reference points color components weights
based on dependence on the distance from reference points to voxels.
This method is used for calculating colors of voxels. The
description and the results of experiments are also provided. The
experiments consisted in texturing generated models and real objects models.}

\KWE{three-dimensional reconstruction; texture mapping; voxel model}

\DOI{10.14357/19922264140311}

\vspace*{3pt}

  \begin{multicols}{2}

\renewcommand{\bibname}{\protect\rmfamily References}
%\renewcommand{\bibname}{\large\protect\rm References}

{\small\frenchspacing
 {%\baselineskip=10.8pt
 \addcontentsline{toc}{section}{References}
 \begin{thebibliography}{9}


\bibitem{1-man-1}
Ispol'zovanie megatekstur (megatexture, clipmaps) [Using megatextures
(megatexture, clipmaps)]. Available at:\linebreak\vspace*{-12pt}

\columnbreak


\noindent {\sf
http://www.gamedev.ru/code/articles/Megatexture} (accessed March~25,
2014).
\bibitem{2-man-1}
\Aue{Arhipov, O.\,P., Yu.\,A.~Maniakov, and D.\,O.~Sirotinin}. 2012. Metod
generatsii virtual'noy setki opornykh tochek \linebreak\vspace*{-12pt}

\pagebreak

\noindent
na tsvetnykh izobrazheniyakh
[Method of virtual markers grid generation on color images]. \textit{Mat-ly VI
Mezhdunar. Nauch.-Tekhnich. Konf. Informatsionnye Tekhnologii v Nauke,
Obrazovanii i Proizvodstve ITNOP-2012} [Information Technologies in
Science, Education, and Production Science and Technology Conference
(International) Proceedings]. Orel. CD-ROM.
\bibitem{3-man-1}
\Aue{Maniakov, Y.\,A.} 2010. Tekhnologiya registratsii povedeniya ob''ektov v
trekhmernom prostranstve [Three-dimensional space object behavior tracking technology].
\textit{Informatsionnye Tekhnologii v Nauke, Obrazovanii i Proizvodstve
(ITNOP). Mater. Mezhdunar. Nauch.-Tehnich. Konf.} [Information
Technologies in Science, Education, and Production Science and Technology
Conference (International) Proceedings]. Orel. 3:182--186.
\bibitem{4-man-1}
\Aue{Arhipov, O.\,P., Yu.\,A.~Maniakov, and D.\,O.~Sirotinin}. 2014.
Informatsionnaya model tekhnologii predstavleniya naturnogo ob''ekta i
izmeneniya ego prostranstvennogo polozheniya [Information model of the
full-scale object and its attitude changes representation technology].
\textit{Informatika i ee Primeneniya}~--- \textit{Inform. Appl.} 1:71--76.

\end{thebibliography}

 }
 }

\end{multicols}

\vspace*{-6pt}

\hfill{\small\textit{Received June 03, 2014}}

\vspace*{-18pt}

\Contr

  \noindent
  \textbf{Arkhipov Oleg P.} (b.\ 1948)~--- Candidate of Science (PhD) in
technology, Director, Orel Branch of Institute of Informatics Problems, Russian
Academy of Sciences, 137 Moskovskoe Highway, Orel 302025, Russian
Federation; arkhipov12@yandex.ru

  \vspace*{3pt}

  \noindent
  \textbf{Maniakov Yury A.} (b.\ 1984)~---
  Candidate of Science (PhD) in technology, scientist, Orel Branch of Institute of
Informatics Problems, Russian Academy of Sciences, 137 Moskovskoe Highway,
Orel 302025, Russian Federation; maniakov\_yuri@mail.ru

\label{end\stat}

\renewcommand{\bibname}{\protect\rm Литература}


  %10
\def\stat{ogaltsov}

\def\tit{АВТОМАТИЧЕСКОЕ ИЗВЛЕЧЕНИЕ МЕТАДАННЫХ\\ ИЗ НАУЧНЫХ PDF-ДОКУМЕНТОВ$^*$}

\def\titkol{Автоматическое извлечение метаданных из научных PDF-документов}

\def\aut{А.\,В.~Огальцов$^{1}$, О.\,Ю. Бахтеев$^{2}$}

\def\autkol{А.\,В.~Огальцов, О.\,Ю. Бахтеев}

\titel{\tit}{\aut}{\autkol}{\titkol}

\index{Огальцов А.\,В.}
\index{Бахтеев О.\,Ю.}
\index{Ogaltsov A.\,V.}
\index{Bakhteev O.\,Y.}




{\renewcommand{\thefootnote}{\fnsymbol{footnote}} \footnotetext[1]
{Работа выполнена при поддержке РФФИ (проект 18-07-01441).}}


\renewcommand{\thefootnote}{\arabic{footnote}}
\footnotetext[1]{Высшая школа экономики; ЗАО <<Антиплагиат>>, 
\mbox{ogaltsov@ap-team.ru}}
\footnotetext[2]{Московский физико-технический институт; 
ЗАО <<Антиплагиат>>, \mbox{bahteev@ap-team.ru}}

%\vspace*{-6pt}



\Abst{Исследуется извлечение метаданных документа. 
Рас\-смат\-ри\-ва\-ют\-ся научные PDF-до\-ку\-мен\-ты на русском языке. 
Особенностью формата PDF 
является раз\-но\-об\-ра\-зие рас\-по\-ло\-же\-ния текста на 
страницах документа. Это создает 
труд\-ности для автоматического извлечения метаданных. Предложенный метод 
извлечения метаданных основан на рас\-смот\-ре\-нии текстовых блоков, полученных при 
помощи PDF-пар\-се\-ра, как объектов в~задаче машинного обучения. Признаковое 
пространство содержит не только текс\-то\-вые признаки, но и~признаки, связанные 
с~форматированием и~расположением блока, которые получены из PDF-пар\-се\-ра. В~работе 
измерено качество классификации предложенного алгоритма и~проведено срав\-не\-ние 
с~базовым алгоритмом.}


\KW{извлечение метаданных; обработка естественного языка; признаки форматирования; 
извлечение информации; метаописания}

\DOI{10.14357/19922264180211}
  
\vspace*{2pt}


\vskip 10pt plus 9pt minus 6pt

\thispagestyle{headings}

\begin{multicols}{2}

\label{st\stat}


\section{Введение}

\vspace*{-2pt}

Метаданные~--- это информация об объекте, рас\-по\-ло\-жен\-ном в~ка\-ком-ли\-бо репозитории. 
Репозитории, такие как электронные биб\-лио\-те\-ки, Semantic Web или Open Archives, 
могут содержать миллионы документов. Авторы достаточно ред\-ко предоставляют 
метаданные, а~электронные документы, со\-бран\-ные в~сети Интернет автоматически, 
име\-ют метаданные еще реже; извлечение метаданных из документов вруч\-ную 
пред\-став\-ля\-ет собой труд\-ную задачу. По этой причине возникает не\-об\-хо\-ди\-мость 
в~сис\-те\-мах автоматического извлечения метаданных из самих документов. В~данной 
статье рас\-смат\-ри\-ва\-ет\-ся извлечение \textit{Заголовка, Автора, Аннотации, 
Содержания, Библиографии.}

Задача извлечения метаданных может быть решена с~помощью правил. Преимущество 
этого подхода в~том, что его достаточно легко реализовать без размеченного 
корпуса. Однако ни правила, ни регулярные выражения не могут обеспечить 
до\-ста\-точ\-но\-го покрытия все\-воз\-мож\-ных стилей форматирования, 
так как сам процесс 
со\-зда\-ния правил содержит в~себе предположение о~схо\-жести форматирования 
ис\-сле\-ду\-емых документов~[1--3]. Недостатки правил за\-клю\-ча\-ют\-ся также 
в~не\-об\-хо\-ди\-мости экспертных знаний и~до\-ста\-точ\-но большого времени, необходимого для 
их создания.

Помимо правил существуют еще несколько подходов к~решению задачи извлечения 
метаданных. Основной из них~--- рас\-смат\-ри\-вать текстовые блоки, полученные из 
pdf-пар\-се\-ра, как объекты. Ставится задача классификации рас\-смат\-ри\-ва\-емо\-го блока 
с~признаковым про\-стран\-ст\-вом, содержащим информацию о~форматировании и~расположении 
блока, а~так\-же текс\-то\-вую и~кон\-текст\-ную информацию.
{\looseness=1

}

Наиболее известные алгоритмы обуче\-ния с~учителем, при\-ме\-ня\-емые в~данной задаче: 
Support Vector Machine (SVM)~\cite{Han, Kovacevic}, Hidden Markov Models (HMMs)~\cite{Seymore} 
и~Conditional Random Fields (CRFs)~\cite{Councill}.

Существует семейство подходов, основанных на нейронных 
сетях~\cite{Rangoni, Rangoni_1}. Здесь объектом является изображение PDF-до\-ку\-мен\-та. 
Для обработки 
изображения применяются OCR (Optical Character Recognition) сис\-те\-мы, но они показывают низкую скорость 
и~точ\-ность работы, поэтому в~данной \mbox{статье} этот подход не используется.

Качество работы алгоритмов извлечения метаданных час\-то измеряется на небольшой 
коллекции достаточно однотипных документов, например из одной об\-ласти знания 
или одного журнала~\cite{Kovacevic, Tao}.

\begin{table*}[b]\small %tabl1
%\vspace*{-12pt}
    \begin{center}
\Caption{Описание признаков}
\label{tab:table1}
\vspace*{2ex}


    \begin{tabular}{|c|l|l|} %\multicolumn{1}{|c|}{\raisebox{-6pt}[0pt][0pt]{
        \hline
\multicolumn{2}{|c|}{Признаки} & \multicolumn{1}{c|}{Описание} \\
\hline

\multicolumn{1}{|c|}{\raisebox{-60pt}[0pt][0pt]
{\tabcolsep=0pt\begin{tabular}{c}Текстовые\\ признаки\end{tabular}}}& isUpper & Написан ли 
весь текст в~верхнем регистре \\
%\cline{2-3}
        & blockLen & Число токенов в~блоке \\
        %\cline{2-3}
        & hasDigit & Есть ли в~блоке  цифра \\
        %\cline{2-3}
        &NumOfPoints & Число точек\\
        %\cline{2-3}
        &NumOfCommas & Число  запятых \\
        %\cline{2-3}
        & NumOfUpCase & Число  символов в~верхнем регистре \\
        %\cline{2-3}
        & UpcaseFraction & Доля  символов в~верхнем регистре \\
        %\cline{2-3}
        & NumOfDigits & Число цифр\\
        %\cline{2-3}
        & NumOfBibSymbols & Число  символов из словаря частых слов библиографии \\
        %\cline{2-3}
        & hasYear & Есть ли год\\
        %\cline{2-3}
        & hasAnno & Есть ли слово аннотация \\
        %\cline{2-3}
        &isEmpty & Является ли пустым  (без текста) \\
        %\cline{2-3}
        & hasKeyW & Есть ли слова  из словаря ключевых слов \\
        %\cline{2-3}
            \hline
   \multicolumn{1}{|c|}{\raisebox{-16pt}[0pt][0pt]
   { \tabcolsep=0pt\begin{tabular}{c}Признаки\\ форматирования\end{tabular} }}& 
Max/Mean/MedianAdv & 
\tabcolsep=0pt\begin{tabular}{l}Максимальное, среднее и~медианное значение толщины\\ 
символов блока\end{tabular} \\
\cline{2-3}
        & Max/Mean/MedianCharH & 
\tabcolsep=0pt\begin{tabular}{l}Максимальное, среднее и~медианное значение высоты\\ 
символов блока\end{tabular} \\
\cline{2-3}
        & fontIndex & Индекс семейства шрифтов \\
        %\cline{2-3}
            \hline
      \multicolumn{1}{|c|}{\raisebox{-22pt}[0pt][0pt]
      { \tabcolsep=0pt\begin{tabular}{c}Признаки\\ расположения\end{tabular} }}& normBlockH/W/A & 
Нормированные высота, ширина и~площадь блока\\
%\cline{2-3}
 & verticalPos & Позиция сверху  страницы \\
% \cline{2-3}
        & leftIndent & Левый отступ\\
   %     \cline{2-3}
        & rightIndent & Правый отступ\\
     %   \cline{2-3}
        &relativeBlockCenter & Позиция  центра блока \\
        %\cline{2-3}
            \hline
                  \multicolumn{1}{|c|}{\raisebox{-38pt}[0pt][0pt]
{    \tabcolsep=0pt\begin{tabular}{c}Контекстные\\ признаки\end{tabular} }}&  
    sameFontAbove/Below &  Блок сверху/снизу имеет такой же индекс шрифта\\
    \cline{2-3}
        & sameCenterAbove/Below & 
Блок сверху/снизу имеет такой же центр \\
\cline{2-3}
        &{sameLeftIndentAbove/Below} & 
Блок сверху/снизу имеет такой же левый отступ\\
\cline{2-3}
        &{hasYearInNeigh} & В фиксированном окне вокруг блока есть год \\
        \cline{2-3}
        &{isEmptyAbove/Below} & Блок  сверху/снизу пуст \\
        \cline{2-3}
        &{hasAnnoInNeigh} & \tabcolsep=0pt\begin{tabular}{l}В~фиксированном окне вокруг блока 
        есть слова из словаря\\ аннотации\end{tabular} \\
        \cline{2-3}
        &{isCharHSmallerThanInNeigh} & 
\tabcolsep=0pt\begin{tabular}{l}Средний размер шрифта блока меньше, чем средний размер\\ 
шрифта в~фиксированном окне\end{tabular} \\
%\cline{2-3}
         \hline
 \end{tabular}
 \end{center}
 \end{table*}   

Научная новизна данной работы заключается в~том, что авторы не ограничиваются 
типом \mbox{\textit{Статья}}, а~работают с~документами различных типов (\textit{Статья, 
Автореферат, Диссертация, Монография, Учебное пособие}) на рус\-ском языке. Оценка\linebreak 
качества алгоритмов производится при помощи\linebreak
 таких мет\-рик, как точ\-ность, полнота 
и~F-ме\-ра, на коллекции PDF-до\-ку\-мен\-тов различных жанров, полученной авторами из 
открытых источников. Проведено сравнение результатов с~базовым SVM-ал\-го\-рит\-мом~\cite{Kovacevic}. 
По сравнению с~базовым алгоритмом использовалось больше 
небинарных при\-зна\-ков и~расширенная информация о~расположении блока.


\section{Формальная постановка задачи}

Задана выборка 
$$
\mathfrak{D} = \left\{({\bf{x}}_i, y_i)\right\}\,, \enskip i = 1,\dots,m\,,
$$ 
состоящая из множества пар <<объ\-ект--класс>>, $\mathbf{x}_i \hm\in  \mathbb{R}^n$. 
Каждый объект~$\mathbf{x}_i$ принадлежит одному из~$Z$~классов с~меткой $y_i \hm\in 
\mathbb{Y} \hm= \{1,\dots,Z\}.$

Пусть также задано множество моделей~$\mathfrak{F}$, среди которых производится 
поиск подходящей модели классификации объектов выборки~$\mathfrak{D}$.
Требуется найти отоб\-ра\-же\-ние $\hat{f} \hm\in \mathfrak{F}: \mathbb{R}^d \hm\to 
\mathbb{Y}$, которое бы минимизировало эмпирический риск на вы\-бор\-ке~$\mathfrak{D}$:
{\looseness=1

}

\noindent
$$
\hat{f} = \underset{f \in \mathfrak{F}}{\argmin}\sum\limits_{\mathbf{x}_i, y_i \in 
\mathfrak{D}}\left[f({\bf{x}}_i) \ne y_i\right].
$$


\section{Описание метода и~признаков}

Для получения блоков документ был обработан парсером. Был выбран парсер Apache 
PDFBox~\cite{PDFBox}, так как он показал наилучшую ско\-рость работы и~воз\-мож\-ность 
на\-строй\-ки. Парсер возвращает текс\-то\-вый блок, который является прямоугольником 
с~текс\-том внут\-ри. Все признаки блока можно раз\-де\-лить на четыре группы (описание 
всех признаков содержится в~табл.~\ref{tab:table1}):
\begin{enumerate}[(1)]
\item \textbf{текстовые признаки.} Данная группа извлекалась из текс\-та 
напрямую. Признаки этой группы проверяют наличие в~строке клю\-че\-вых слов 
и~собирают различные текс\-то\-вые статистики;
\item \textbf{признаки форматирования}. В~эту группу входят такие 
при\-зна\-ки, как размер шриф\-та, семейство шриф\-тов и~др.;

\item \textbf{признаки расположения}. В~этой группе содержится информация 
о~положении блока на странице, такая как правый и~левый отступ, координата 
цент\-ра блока и~др.;
\item \textbf{контекстные признаки}. Эти признаки описывают контекст 
рас\-смат\-ри\-ва\-емо\-го блока, как пред\-ло\-же\-но в~\cite{Tao}.
\end{enumerate}

\vspace*{-12pt}

\section{Эксперимент}

\subsection{Описание коллекции}

Коллекция содержит 550 размеченных PDF-до\-ку\-мен\-тов из различных областей знаний 
и~разных типов (табл.~2).


  
Одна из основных характеристик задачи~--- сильная несбалансированность классов 
(табл.~3). Видно, что общее чис\-ло блоков в~табл.~3 
варьируется для категорий метаданных. Это связано с~эвристиками, примененными на 
этапе обработки документа. Например, для заголовка, автора и~аннотации 
обрабатывалась только первая и~вторая страницы документа, так как обработка 
занимает достаточно много времени и~нет не\-об\-хо\-ди\-мости обрабатывать всю 
диссертацию, чтобы извлечь данные категории. Для библиографии извлекались 
последние $k = \max(0{,}1p, 1)$ стра\-ниц документа, для содержания~--- первые $n\hm = 
\max(0{,}1p, 4)$ стра\-ниц документа, где $p$~--- чис\-ло стра\-ниц в~документе. 
Количество блоков с~автором меньше, чем чис\-ло работ. Это вызвано ошибками работы 
пар\-се\-ра, когда блок с~именем автора не извлекался.

%\begin{table*}
{\small %tabl2
    \begin{center}
    \vspace*{6pt}
    
    \noindent
{{\tablename~2}\ \ \small{Распределение жанров}}

%    \label{tab:table2}  
\vspace*{2ex}

    \begin{tabular}{|l|c|}
        \hline
       \multicolumn{1}{|c|}{Жанр} & Число документов \\ 
        \hline
        Статья & 148 \\ 
       % \hline
        Автореферат & 161 \\ 
       % \hline
        Диссертация & 196 \\ 
        %\hline
        Монография, пособие & \hphantom{9}45 \\
        \hline
    \end{tabular}
    \end{center}}
%\end{table*}
%\begin{table*}
{\small %tabl3
\vspace*{3pt}
\begin{center}
  \noindent
{{\tablename~3}\ \ \small{Распределение блоков по категориям}}

    \vspace*{2ex}
    
    \begin{tabular}{|l|c|c|}
        \hline
        \multicolumn{1}{|c|}{Категория} & 
        \tabcolsep=0pt\begin{tabular}{c}Число блоков\\ данной\\ категории\end{tabular} & 
        \tabcolsep=0pt\begin{tabular}{c}Полное число\\ извлеченных\\ 
блоков\end{tabular} \\ 
\hline
        Заголовок & 957 & 76\,986 \\ %\hline
        Автор & 432 & 76\,986 \\ %\hline
        Аннотация & 1386\hphantom{9} & 76\,986 \\ %\hline
        Библиография & 45\,323\hphantom{\,99} & 316\,015\hphantom{9} \\ %\hline
        Содержание &  5866\hphantom{9} & 309\,681\hphantom{9} \\
        \hline
    \end{tabular}
    \end{center}
    }
%    \end{table*}

\setcounter{table}{3}


\subsection{Описание эксперимента}

Для обоснованного выбора алгоритма была снижена раз\-мер\-ность данных 
с~по\-мощью t-SNE~\cite{Maaten}. Визуализация пред\-став\-ле\-на на рис.~1. 
Видно, что в~про\-стран\-ст\-ве сниженной раз\-мер\-ности данные не 
являются линейно разделимыми, но есть однородные клас\-те\-ры. Этот факт по\-слу\-жил 
одним из обос\-но\-ва\-ний применения ан\-самб\-ля ре\-ша\-ющих деревьев. Также этот алгоритм 
показал наилучшее качество во время экспериментов.


Для проведения экспериментов выборка была разделена на обуча\-ющую и~тес\-то\-вую 
в~пропорции~80/20 стратифицированно в~соответствии с~табл.~3. 
Качество измерялось стандартными мет\-ри\-ка\-ми бинарной классификации: точ\-ность, 
пол\-но\-та и~F-ме\-ра. Результаты приведены в~табл.~\ref{tab:table4}.\linebreak Качество работы 
алгоритма в~данной работе срав\-ни\-ва\-лось с~базовым алгоритмом, взятым из 
\mbox{статьи}~\cite{Kovacevic}. Отличия признакового про\-стран\-ст\-ва базового алгоритма от 
предложенного:
\begin{itemize}
\item
в базовом алгоритме есть признаки, со\-би\-ра\-ющие раз\-лич\-ные html-те\-ги, так как 
использовался другой pdf-пар\-сер. В~проведенном эксперименте такие признаки не 
собирались;

\item
в отличие от базового, в~предложенном алгоритме использовались небинарные 
текстовые статистики, больше признаков расположения и~кон\-текст\-ные признаки.
\end{itemize}


В базовом алгоритме использовался SVM и~извлекалась только первая страница 
статей узкой на\-прав\-лен\-ности. Базовый алгоритм показал низ-\linebreak\vspace*{-12pt}

 { \begin{center}  %fig1
 \vspace*{9pt}
  \mbox{%
 \epsfxsize=74.724mm 
 \epsfbox{oga-1.eps}
 }


\end{center}


\noindent
{{\figurename~1}\ \ \small{Визуализация данных при помощи t-SNE: 
    \textit{1}~--- заголовок;
        \textit{2}~--- автор;  \textit{3}~--- биб\-лио\-гра\-фия;
            \textit{4}~--- содержание;
        \textit{5}~--- другое}}
}

%\vspace*{9pt}

\setcounter{figure}{1}

\begin{table*}\small %tabl4
\begin{center}
\Caption{Результаты эксперимента
    \label{tab:table4}}
\vspace*{2ex}

    \begin{tabular}{|l|l|c|c|c|}
        \hline
\multicolumn{1}{|c|}{Категория}&\multicolumn{1}{c|}{Метод} &
{P} &
{R} &
{F} \\
\hline
      & 
                  {Предложенный} & \textbf{0,74}&
\textbf{0,79} &
        \textbf{0,76} \\
        %\cline{2-5}
     {Заголовок}   & Предложенный только на текстовых 
признаках & 0,67 &
        0,66 &
        0,66 \\
       % \cline{2-5}
        & Базовый & 0,20 &
0,77 &
    0,32 \\
        \hline
 & {Предложенный} & \textbf{0,78}&
0,71 &
    \textbf{0,74} \\
    %\cline{2-5}
       {Автор}  & Предложенный только на текстовых 
признаках & 0,45 &
        0,74 &
        0,56 \\
       % \cline{2-5}
        & Базовый & 0,33 &
        \textbf{0,75} &
        0,46 \\
                \hline
 &  
        {Предложенный} & \textbf{0,76} 
&\textbf{0,85} &
\textbf{0,80} \\
%\cline{2-5}
              Библиография  & Предложенный только на текстовых 
признаках & 0,59 &
        \textbf{0,85} &
0,69 \\
%\cline{2-5}
        & {Базовый} & ---& ---&    --- \\
           \hline
& Предложенный & \textbf{0,72} 
&
\textbf{0,74} &
        \textbf{0,73} \\
       % \cline{2-5}
         {Содержание}    & Предложенный только на текс\-то\-вых 
признаках & 0,71 & 0,69 & 0,70 \\
%\cline{2-5}
        & Базовый & --- & --- &--- \\
        \hline
 &  {Предложенный} & \textbf{0,73}  &
\textbf{0,71} & \textbf{0,72} \\
%\cline{2-5}
     Аннотация   & Предложенный только на текс\-то\-вых 
признаках & 0,22 & 0,54 & 0,31 \\
%\cline{2-5}
        & Базовый & 0,09 & 0,63 & 0,16 \\
                \hline
            \end{tabular}
    \end{center}
    \vspace*{-6pt}
\end{table*}


{ \begin{center}  %fig2
 %\vspace*{0.5pt}
  \mbox{%
 \epsfxsize=78.254mm 
 \epsfbox{oga-2.eps}
 }


\end{center}


\noindent
{{\figurename~2}\ \ \small{Кривые полнота--точ\-ность:
           \textit{1}~--- заголовок;
               \textit{2}~--- автор; 
           \textit{3}~--- биб\-лио\-гра\-фия; \textit{4}~--- содержание;
    \textit{5}~--- аннотация}}
}

\vspace*{9pt}

\setcounter{figure}{2} 



\noindent
кое качество, так как 
в~текущей по\-ста\-нов\-ке 
эксперимента стиль форматирования документов очень сильно 
варьируются.


В табл.~\ref{tab:table4} приведены результаты эксперимента, в~котором 
признаковое пространство со\-сто\-яло только из текстовых статистик. Вид\-но, что 
расширение признакового про\-стран\-ст\-ва информацией о~расположении и~форматировании 
блока заметно повышает качество классификации.

Из-за несбалансированности классов была проведена редукция блоков, относящихся 
к~классу \textit{Прочее}. Если~150~элементов до и~после блока \textit{Прочее} не 
содержали других классов, то такой блок удалялся. Проводились эксперименты 
с~порогом ве\-ро\-ят\-ности классификации и~весами классов. Эксперименты показали, что 
порог классификации~0,5 и~вес редкого класса, равный двум, демонстрирует 
наилучший результат для  \textit{Содержания, Библиографии, Аннотации}. Для 
\textit{Заголовка} и~\textit{Автора} перенос порога классификации в~0,6 и~0,65 
соответственно в~ком\-би\-на\-ции с~весом редкого класса, равным~20, дали 
повышение точ\-ности ($+0{,}05$ и~$+0{,}18$) с~падением пол\-но\-ты ($-0{,}02$ и~$-0{,}16$), 
но с~небольшим повышением F-ме\-ры. 
Кривая пол\-но\-та--точ\-ность изоб\-ра\-же\-на на 
рис.~2.

 







Был проведен анализ значимости признаков, которая была получена стандартным для 
алгоритма Random Forest способом~\cite{Breiman}. Три первых по значимости 
при\-зна\-ка для каж\-до\-го класса приведены в~табл.~5. 
Видно, что почти в~каждом классе\linebreak\vspace*{-12pt}

%\begin{table*}
{\small %tabl5
\vspace*{1pt}
    \begin{center}
    \noindent
{{\tablename~5}\ \ \small{Результаты анализа значимости признаков}}
    \vspace*{2ex}
    
    \begin{tabular}{|l|l|c|}
        \hline
 \multicolumn{1}{|c|}{Категория}& \multicolumn{1}{c|}{Признак} &
        \multicolumn{1}{c|}{Значимость} \\
        \hline
        & {NumOfUpCase} & 0,65 \\
%\cline{2-3}
{Заголовок}        & {verticalPos} & 0,06 
\\
%\cline{2-3}
        & {MeanAdv} & 0,06 \\
          \hline
&  {UpcaseFraction} & 0,34 \\
%\cline{2-3}
    Автор     & {NumOfUpCase} & 0,23\\
%\cline{2-3}
        & {verticalPos} & 0,11\\
        %\cline{2-3}
                \hline
 &  NumOfPoints & 0,41\\
        %\cline{2-3}
        Содержание        & blockLen & 0,18 
\\
%\cline{2-3}
        & {NumOfDigits} & 0,08\\
        %\cline{2-3}
                \hline
&  {blockLen} & 0,14\\
        %\cline{2-3}
        Библиография  & {relativeBlockCenter} & 
0,12 \\
%\cline{2-3}
        & {normBlockW} & 0,10\\
        %\cline{2-3}
                \hline
 & {verticalPos} & 0,10\\
        %\cline{2-3}
               Аннотация & normBlockA & 0,10\\
        %\cline{2-3}
        & MeanAdv & 0,09\\
        %\cline{2-3}
        \hline  
     \end{tabular}
               \vspace*{1pt}   
   \end{center}
   }
%\end{table*}

\setcounter{table}{5}

\end{multicols}

\begin{figure*} %fig3
\vspace*{1pt}
 \begin{center}
 \mbox{%
 \epsfxsize=165.062mm 
 \epsfbox{oga-3.eps}
 }
 \end{center}
\vspace*{-9pt}
    \Caption{Визуализация данных для заголовка
    (\textit{1}~--- заголовок; \textit{2}~--- прочее)
     в~проекции на про\-стран\-ст\-во 
значимых при\-зна\-ков }
    \label{fig:figure3}
\end{figure*}

\begin{multicols}{2}

\noindent
 признаки расположения и~форматирования вошли в~число значимых. 
Данные были визуализированы в~проекции на про\-стран\-ст\-во значимых при\-зна\-ков. На 
рис.~\ref{fig:figure3} и~\ref{fig:figure4} показаны примеры таких визуализаций 
для категорий \textit{Заголовок} и~\textit{Автор}. На рисунках видны сегменты, 
где преимущественно расположен целевой класс.





    
\vspace*{-6pt}


\subsection{Анализ ошибок}

С целью выявления основных ошибок предложенного алгоритма был произведен анализ 
результатов.

\vspace*{-12pt}

\paragraph*{Заголовок.}
В~табл.~5 видно, что для категории \textit{Заголовок} самым 
значимым признаком является число символов в~верхнем регистре. Поэтому 
лож\-но-по\-ло\-жи\-тель\-ные примеры наблюдаются вбли\-зи заголовка по расположению 
и~с~большим 
количеством символов в~верхнем ре\-гистре.

\textbf{Примеры:}

\begin{itemize}
\item[$\bullet$]
\textit{ГОСУДАРСТВЕННЫЙ УНИВЕРСИТЕТ УПРАВ\-ЛЕ\-НИЯ}

\item[$\bullet$]
\textit{05.13.11 МАТЕМАТИЧЕСКОЕ И ПРО\-ГРАМ\-МНОЕ ОБЕСПЕЧЕНИЕ}
\end{itemize}

Основной причиной лож\-но-от\-ри\-ца\-тель\-ных примеров оказались длинные заголовки 
с~форматированием, близким к~обыкновенному тексту. 
Примеры:

\begin{itemize}
\item[$\bullet$]
\textit{Французское органное искусство Барокко: музыка, органостроение, 
исполнительство}\\[-14.5pt]

\item[$\bullet$]
\textit{Нейтральный внешнеполитический курс Нидерландов: от Мюнстерского мира 
1648~г.\ до конца Первой мировой войны}\\[-14.5pt]

\item[$\bullet$]
\textit{Музыкальная жизнь Москвы XIX столетия и~ее отражение в~фортепианной 
практике}
\end{itemize}

\vspace*{-13pt}

\paragraph*{Автор.}

Если извлеченный блок располагался вверху страницы и~содержал имя и~фамилию, то 
наблюдался лож\-но-по\-ло\-жи\-тель\-ный пример. Примеры такого рода наблюдались, когда 
блок содержал комбинацию точек и~символов в~верх\-нем регистре, схожую с~именем 
и~фамилией.

\textbf{Примеры:}

\begin{itemize}
\item[$\bullet$]
\textit{им.\ М.\,В.~Келдыша РАН}\\[-15pt]

\item[$\bullet$]
\textit{(ПЖ). Одним}

Ложно-отрицательные примеры возникали, когда блок с~автором находился в~очень 
нетипичном месте.\\[-15pt]
\end{itemize}

\end{multicols}

\begin{figure*} %fig4
\vspace*{1pt}
 \begin{center}
 \mbox{%
 \epsfxsize=165.459mm 
 \epsfbox{oga-4.eps}
 }
 \end{center}
\vspace*{-9pt}
       \Caption{Визуализация данных для автора 
       (\textit{1}~--- автор; \textit{2}~--- прочее)
       в~проекции на про\-стран\-ст\-во 
значимых при\-зна\-ков
}
    \label{fig:figure4}
\end{figure*}

\begin{multicols}{2}

\begin{itemize}
\item[$\bullet$]
\textit{Огальцов А.\,В.} (если находится где-то в~очень нетипичном месте 
страницы)
\end{itemize}

\vspace*{-10pt}

\paragraph*{Аннотация.}

Самые частые ошибки обоих типов в~этой категории возникали в~связи с~тем, что 
блоки с~аннотацией легко спутать с~обычным текс\-том.

\vspace*{-10pt}

\paragraph*{Содержание.}

Ложно-положительные примеры~--- это в~основном блоки с~большим чис\-лом точек, 
скорее всего вызванных ошибками обработки.

\textbf{Примеры:}

\begin{itemize}

\item[$\bullet$]
\textit{В ТОМСКОЙ ОБЛАСТИ ............}

\item[$\bullet$]
\textit{Случай d(k)=0 ...........................}

Ложно-отрицательные примеры вызваны форматированием, близким к~обычному тексту.

\item[$\bullet$]
\textit{Функции ввода, вывода и~работы с~символами}

\item[$\bullet$]
\textit{Циклы, блоки и~присваивания}
\end{itemize}

\vspace*{-10pt}

\paragraph*{Библиография.}

Ложно-положительные примеры этой категории похожи на биб\-лио\-гра\-фию 
расположением и~форматированием.

\textbf{Примеры:}

\begin{itemize}
\item[$\bullet$]
\textit{Тел.: (3472) 1234567, 1234567}

\item[$\bullet$]
\textit{Печать офсетная. Гарнитура NewtonC. Бумага офсетная}

\item[$\bullet$]
\textit{как отмечают А.\,Я.~Гаев и~др. 2005, нереально.}


Ложно-отрицательные примеры были вызваны сильной схо\-жестью с~обычным текс\-том.
\end{itemize}


Большое число блоков с~обычным текст\-ом делает возможным ситуацию сходства этих 
блоков с~целевыми по ряду признаков, поэтому одним из на\-прав\-ле\-ний улуч\-ше\-ния 
является снижение несбалансированности выборки различными методами.
    
    
\section{Заключение}

В работе была исследована возможность применения методов машинного обучения 
к~задаче извлечения метаданных. Использовано расширенное признаковое про\-стран\-ст\-во, 
вклю\-ча\-ющее не только текс\-то\-вые признаки, но также признаки форматирования 
и~расположения, которые оказались важ\-ны\-ми в~решении проб\-ле\-мы большой вариативности 
стилей форматирования документа. Измерено качество предлагаемого алгоритма на 
коллекции научных PDF-до\-ку\-мен\-тов на рус\-ском языке, взятых из открытых 
источников. Проведен анализ ошибок. Показано, что предложенный алгоритм 
значительно превосходит базовый на документах различных типов.

Будущие исследования будут направлены на расширение признакового пространства 
и~на обзор возможностей новых PDF-пар\-се\-ров. Также будет исследована воз\-мож\-ность 
применения графических моделей, таких как услов\-ные случайные поля.

\bigskip

Авторы выражают свою благодарность к.ф.-м.н.\ Чеховичу~Ю.\,В.\ за ценные советы 
при планировании исследования и~рекомендации по оформлению статьи.


{\small\frenchspacing
 {%\baselineskip=10.8pt
 \addcontentsline{toc}{section}{References}
 \begin{thebibliography}{99}
\bibitem{Bergmark}
  \Au{Bergmark D.} Automatic extraction of reference linking information 
from online documents.~--- Ithaca, NY, USA: Cornell University, 2000. 20~p.

\bibitem{Klink}
\Au{Klink S., Dengel~A., Kieninger~T. } Document structure analysis 
based on layout and textual features~//  Workshop (International) on 
Document Analysis Systems Proceedings.~---
        Boston, MA, USA, 2000. P.~99--111.

\bibitem{Mao}
    \Au{Mao S., Kim J.\,W., Thoma~G.\,R.} A~dynamic feature generation 
system for automated metadata extraction in preservation of digital materials~// 
1st  Workshop (International) on Document Image Analysis for 
Libraries Proceedings.~--- Palo Alto, CA, USA: IEEE Computer Society, 2004. P.~225--232.
    
\bibitem{Han}
\Au{Han H., Giles C.\,L., Manavoglu~E., Zha~H., Zhang~Z., Fox~E.\,A.} 
Automatic document metadata extraction using support vector machines~// 
3rd ACM/IEEE-CS Joint Conference on Digital Libraries Proceedings, 2003. 
P.~37--48.

\bibitem{Kovacevic}
\Au{Kovacevi$\acute{\mbox{c}}$~A., Ivanovi$\acute{\mbox{c}}$~D., Milosavljevi$\acute{\mbox{c}}$~B., 
Konjovi$\acute{\mbox{c}}$~Z., Surla~D.} 
Automatic extraction of metadata from scientific publications for CRIS systems~// 
Program, 2011. Vol.~45. Iss.~4. P.~376--396.

\bibitem{Seymore}
\Au{Seymore K., McCallum~A., Rosenfeld~R.} Learning hidden Markov model 
structure for information extraction~// AAAI~99 Workshop on 
Machine Learning for Information Extraction Proceedings, 1999. P.~37--42.

\bibitem{Councill}
\Au{Councill I., Giles~C.\,L., Kan~M.-Y.} ParsCit: An open-source CRF 
reference string parsing package~// 6th Conference 
(International) on Language Resources and Evaluation Proceedings.~---
Marrakech, Morocco, 2008. P.~661--667.


\bibitem{Rangoni}
\Au{Rangoni Y., \mbox{Bela$\ddot{\mbox{\!\hspace*{-0.2pt}{\ptb{\!\i}}}}$d~A.}} Document logical structure analysis based on 
perceptive cycles~// IAPR Workshop on Document Analysis Systems, 2006. P.~117--128.

\bibitem{Rangoni_1}
\Au{Rangoni Y., Bela$\ddot{\mbox{\!\hspace*{-0.2pt}{\ptb{\!\i}}}}$d~A., Vajda~S.} Labelling logical structures of 
document images using a~dynamic perceptive neural network~// Int. 
J.~Doc. Anal. Recog., 2012. Vol.~15. Iss.~1. P.~45--55.

\bibitem{Tao}
\Au{Tao X., Tang~Z., Xu~C.} Document page structure learning for fixed-layout 
e-books using conditional random fields~// 
 Document Recognition and Retrieval XXI:
 Proc. SPIE, 2014. 
Vol.~9021. P.~1--9.

\bibitem{PDFBox}
Apache PDFBox. {\sf http://pdfbox.apache.org}.

\bibitem{Maaten}
\Au{Van der Maaten L.\,J.\,P., Hinton~G.\,E.} Visualizing 
high-dimensional data using t-SNE~// J.~Mach. Learn. Res., 2008. 
Vol.~9. P.~2579--2605.

\bibitem{Breiman}
\Au{Breiman L.} Random forests~// Mach. Learn., 2001. Vol.~45. P.~5--32.

 \end{thebibliography}

 }
 }

\end{multicols}

\vspace*{-6pt}

\hfill{\small\textit{Поступила в~редакцию 20.12.17}}

\vspace*{6pt}

%\newpage

%\vspace*{-24pt}

\hrule

\vspace*{2pt}

\hrule

%\vspace*{8pt}


\def\tit{AUTOMATIC METADATA EXTRACTION FROM SCIENTIFIC\\ PDF DOCUMENTS}

\def\titkol{Automatic metadata extraction from scientific PDF documents}

\def\aut{A.\,V.~Ogaltsov$^{1,2}$ and~O.\,Y.~Bakhteev$^{2,3}$}

\def\autkol{A.\,V.~Ogaltsov and~O.\,Y.~Bakhteev}

\titel{\tit}{\aut}{\autkol}{\titkol}

\vspace*{-9pt}


\noindent
$^1$National Research University Higher School of Economics, 
20~Myasnitskaya Str., Moscow 101000, Russian\linebreak
$\hphantom{^1}$Federation

\noindent
$^2$Antiplagiat JSC, 33~Varshavskoe Shosse, Moscow 117105, Russian Federation

\noindent
$^3$Moscow Institute of Physics and Technology, 9~Institutskiy Per., Dolgoprudny, 
Moscow Region 141700, Russian\linebreak
$\hphantom{^1}$Federation

\def\leftfootline{\small{\textbf{\thepage}
\hfill INFORMATIKA I EE PRIMENENIYA~--- INFORMATICS AND
APPLICATIONS\ \ \ 2018\ \ \ volume~12\ \ \ issue\ 2}
}%
 \def\rightfootline{\small{INFORMATIKA I EE PRIMENENIYA~---
INFORMATICS AND APPLICATIONS\ \ \ 2018\ \ \ volume~12\ \ \ issue\ 2
\hfill \textbf{\thepage}}}

\vspace*{3pt}




\Abste{The authors investigate the task of metadata extraction. The authors consider 
scientific PDF documents in Russian. One of the features of PDF is a~rich layout. 
It is difficult to extract metadata due to this fact. The authors propose 
a~method based on considering blocks from pdf-parser as objects in 
a~machine learning task. The feature space is constructed not only of text 
statistics but also of formatting and positioning features of the block. 
The authors performed computational experiments and compared their approach 
with the baseline.}

\KWE{metadata extraction; natural language processing; layout features; 
information retrieval; metadescriptions} 

\DOI{10.14357/19922264180211} %

%\vspace*{-14pt}

  \Ack
   \noindent
   The paper was supported by the Russian Foundation 
   for Basic Research (project 18-07-01441).



%\vspace*{-3pt}

  \begin{multicols}{2}

\renewcommand{\bibname}{\protect\rmfamily References}
%\renewcommand{\bibname}{\large\protect\rm References}

{\small\frenchspacing
 {%\baselineskip=10.8pt
 \addcontentsline{toc}{section}{References}
 \begin{thebibliography}{99}
        \bibitem{Bergmark-1}
        \Aue{Bergmark, D.} 2000. Automatic extraction of reference 
        linking information from online documents.  Ithaca, NY: Cornell University.  20~p.
        
        \bibitem{Klink-1}
        \Aue{Klink, S., A.~Dengel, and T.~Kieninger. } 2010. 
        Document structure analysis based on layout and textual features. 
        \textit{Workshop (International) on Document Analysis Systems Proceedings}. 
        Boston, MA. 99--111.
        
        \bibitem{Mao-1}
        \Aue{Mao, S., J.\,W.~Kim, and G.\,R.~Thoma.} 2004. 
        A~dynamic feature generation system for automated metadata extraction in 
        preservation of digital materials. 
        \textit{1st  Workshop (International) on Document Image Analysis for Libraries
        Proceedings}. Palo Alto, CA: IEEE Computer Society. 225-232.
        
        \bibitem{Han-1}
        \Aue{Han,~H., C.\,L.~Giles, E.~Manavoglu, H.~Zha, Z.~Zhang, and E.\,A.~Fox.} 
        2003. Automatic document metadata extraction using support vector machines. 
        \textit{3rd ACM/IEEE-CS Joint Conference on Digital Libraries Proceedings}. 
        37--48.
        
        \bibitem{Kovacevic-1}
        \Aue{Kovacevi$\acute{\mbox{c}}$, A., D.~Ivanovi$\acute{\mbox{c}}$, 
        B.~Milosavljevi$\acute{\mbox{c}}$, Z.~Konjovi$\acute{\mbox{c}}$, and D.~Surla.} 
        2011. Automatic extraction of metadata from scientific publications 
        for CRIS systems. \textit{Program}  45(4):376--396.
        
        \bibitem{Seymore-1}
        \Aue{Seymore, K., A.~McCallum, and R.~Rosenfeld.} 1999. 
        Learning hidden Markov model structure for information extraction. 
        \textit{AAAI~99 Workshop on Machine Learning for Information Extraction
        Proceedings}. 37--42.
        
        \bibitem{Councill-1}
        \Aue{Councill, I., C.\,L.~Giles, and M.-Y.~Kan.} 2008. ParsCit: An 
        open-source CRF reference string parsing package. 
        \textit{6th  Conference (International) on Language Resources and Evaluation 
        Proceedings}. Marrakech, Morocco. 661--667.
        
        
        \bibitem{Rangoni-1}
        \Aue{Rangoni, Y., and A.~\mbox{Bela$\ddot{\mbox{{\ptb{\!\i}}}}$d.}} 2006. 
        Document logical structure analysis based on perceptive cycles.
         \textit{IAPR Workshop on Document Analysis Systems}. 117--128.
        
        \bibitem{Rangoni_1-1}
        \Aue{Rangoni, Y., A.~\mbox{Bela$\ddot{\mbox{\ptb{\!\i}}}$d}, and S.~Vajda.} 
        2012. Labelling logical structures of document images using a~dynamic 
        perceptive neural network \textit{Int. J.~Doc. Anal. Recog.} 
        15(1):45--55.
        
        \bibitem{Tao-1}
        \Aue{Tao, X., Z.~Tang, and C.~Xu.} 2013. Document page structure learning 
        for fixed-layout e-books using conditional random fields. 
        \textit{Document Recognition and Retrieval XXI: Proc. SPIE} 9021:1--9.
        
        \bibitem{PDFBox-1}
        {Apache PDFBox}. Available at: {\sf http://pdfbox.apache.\linebreak org/}
        (accessed June~19, 2018).
        
        \bibitem{Maaten-1}
        \Aue{Van der Maaten, L.\,J.\,P., and G.\,E.~Hinton.} 
        2008. Visualizing high-dimensional data using t-SNE. \textit{J.~Mach. 
        Learn. Res.} 9:2579--2605.
        
        \bibitem{Breiman-1}
        \Aue{Breiman, L.} 2001. Random forests. \textit{Mach. Learn.} 45:5--32.
        
        \end{thebibliography}

 }
 }

\end{multicols}

\vspace*{-3pt}

\hfill{\small\textit{Received December 20, 2017}}

%\vspace*{-24pt}
        
\Contr

\noindent
\textbf{Ogaltsov Alexander V.} (b.\ 1993)~--- 
student, National Research University Higher School of Economics, 
20~Myasnitskaya Str., Moscow 101000, Russian Federation; researcher, 
Antiplagiat JSC, 33~Varshavskoe Shosse, Moscow 117105, Russian Federation; 
\mbox{avogaltsov@edu.hse.ru}

\vspace*{3pt}

\noindent
\textbf{Bakhteev  Oleg Y.} (b.\ 1993)~--- PhD student, 
Moscow Institute of Physics and Technology, 9~Institutskiy Per., Dolgoprudny, 
Moscow Region 141700, Russian Federation; 
researcher, Antiplagiat JSC, 33~Varshavskoe Shosse, Moscow 117105, Russian Federation; 
\mbox{bakhteev@phystech.edu}


\label{end\stat}


\renewcommand{\bibname}{\protect\rm Литература}   %11

\def\stat{shnurkov}

\def\tit{АНАЛИТИЧЕСКОЕ РЕШЕНИЕ ЗАДАЧИ ОПТИМАЛЬНОГО УПРАВЛЕНИЯ ПОЛУМАРКОВСКИМ ПРОЦЕССОМ\\ 
С~КОНЕЧНЫМ МНОЖЕСТВОМ СОСТОЯНИЙ$^*$}

\def\titkol{Аналитическое решение задачи оптимального управления полумарковским 
процессом} %с~конечным множеством состояний}

\def\aut{П.\,В.~Шнурков$^1$, А.\,К.~Горшенин$^2$, В.\,В.~Белоусов$^3$}

\def\autkol{П.\,В.~Шнурков, А.\,К.~Горшенин, В.\,В.~Белоусов}

\titel{\tit}{\aut}{\autkol}{\titkol}

\index{Шнурков П.\,В.}
\index{Горшенин А.\,К.}
\index{Белоусов В.\,В.}
\index{Shnurkov P.\,V.}
\index{Gorshenin A.\,K.}
\index{Belousov V.\,V.}


{\renewcommand{\thefootnote}{\fnsymbol{footnote}} \footnotetext[1]
{Работа выполнена при частичной поддержке РФФИ (проект 15-07-05316).}}


\renewcommand{\thefootnote}{\arabic{footnote}}
\footnotetext[1]{Национальный исследовательский университет <<Высшая школа экономики>>, 
\mbox{pshnurkov@hse.ru}}
\footnotetext[2]{Институт проблем информатики Федерального исследовательского центра <<Информатика 
и~управ\-ле\-ние>> Российской академии наук, \mbox{agorshenin@frccsc.ru}}
\footnotetext[3]{Институт проблем информатики Федерального исследовательского центра <<Информатика 
и~управление>> Российской академии наук, \mbox{vbelousov@ipiran.ru}}

%\vspace*{-6pt}

\Abst{Настоящее исследование посвящено теоретическому обоснованию нового метода 
нахождения оптимальной стратегии управления полумарковским процессом с~конечным 
множеством состояний. Рассматриваются марковские рандомизированные стратегии 
управления, определяемые конечным набором вероятностных мер, соответствующих 
каждому состоянию. Характеристикой качества управления служит стационарный 
стоимостной показатель. Данный показатель представляет собой дроб\-но-ли\-ней\-ный 
интегральный функционал от набора вероятностных мер, задающих стратегию управления. 
Для этого функционала известны явные аналитические представления подынтегральных 
функций числителя и~знаменателя. Дальнейшие результаты основываются на новой 
усиленной и~обобщенной форме теоремы об экстремуме дроб\-но-ли\-ней\-но\-го интегрального 
функционала. Доказывается, что проблемы существования оптимальной стратегии управления 
полумарковским процессом и~ее нахождения сводятся к~задаче численного исследования 
на глобальный экстремум заданной функции от конечного числа вещественных переменных.}

\KW{оптимальное управление полумарковским процессом; стационарный стоимостной 
показатель качества управления; дроб\-но-ли\-ней\-ный интегральный функционал}

\DOI{10.14357/19922264160408} 

\vspace*{9pt}


\vskip 10pt plus 9pt minus 6pt

\thispagestyle{headings}

\begin{multicols}{2}

\label{st\stat}

\section{Введение}

Теория оптимального управления марковскими и~полумарковскими случайными 
процессами интенсивно развивается с~начала 1960-х~гг. Еще в~первых 
основополагающих исследованиях рассматривались не только проблемы существования 
оптимальных стратегий управления, но и~способы нахождения этих стратегий. 

Для решения таких проблем, имеющих алгоритмическое содержание, использовались 
открытые незадолго до этого мощные методы прикладной математики: линейное 
программирование и~динамическое программирование. Отметим, прежде всего, 
классическую работу Р.~Ховарда~\cite{1}, в~которой метод динамического 
программирования был применен для решения проблемы оптимального управления 
марковским процессом с~непрерывным временем. В~дальнейшем В.\,В.~Рыков~\cite{2} 
доказал, что для аналогичной модели управления марковским процессом с~учетом 
переоценки оптимальной стратегией также является стационарная.

Важную роль в~развитии теории управления случайными процессами сыграла работа 
В.~Джевелла~\cite{3}, в~которой были впервые рассмотрены полумарковские модели 
управления для вариантов с~переоценкой и~без переоценки. Данная работа была 
переведена на русский язык и~послужила основой для многих последующих работ 
отечественных и~зарубежных специалистов. В~частности, Б.~Фокс показал~\cite{4}, 
что оптимальной стратегией управления полумарковским процессом в~варианте без 
переоценки является стационарная; аналогичные результаты были получены Э.~Денардо 
и~для варианта с~переоценкой~\cite{5}.

Среди последующих исследований алгоритмической направленности отметим работы 
Р.~Ховарда~\cite{6}, Б.~Фокса~\cite{4}, а также С.~Осаки и~Х.~Майна~\cite{7}. 
В~этих работах для нахождения оптимальных стратегий управления полумарковскими 
процессами использовался метод линейного программирования.

В 1970~г.\ была опубликована фундаментальная монография Х.~Майна и~С.~Осаки~\cite{8}, 
переведенная на русский язык в~1977~г., в~которой были систе\-ма\-ти\-зи\-ро\-ва\-ны и~изложены 
основные результаты по теории оптимального управления марковскими и~полумарковскими 
случайными процессами. Фактически данная книга стала итогом исследований по проблемам 
стохастического управления\linebreak
 за~10~лет. Отметим, что в~этой монографии рас\-смат\-ри\-ва\-лись 
марковские и~полумарковские модели управления с~конечными множествами состояний 
и~допустимых решений, принимаемых \mbox{в~каждом} состоянии. Были получены принципиальные 
тео\-ре\-ти\-че\-ские результаты, заключающиеся в~том, что оптимальные стратегии управ\-ле\-ния 
для основных видов рас\-смат\-ри\-ва\-емых моделей с~переоценкой и~без переоценки являются 
детерминированными и~стационарными. Были разработаны и~обоснованы процедуры нахождения 
оптимальных стратегий управления. В~частности, для модели управления полумарковским 
процессом без переоценки, когда множество со\-сто\-яний образует один эргодический класс, 
а~показатель качества управления пред\-став\-ля\-ет собой стационарный средний удельный 
доход (см.~[8, гл.~5, п.~5.5]), процедура поиска оптимальной рандомизированной 
стратегии осуществлялась методом линейного программирования. Обратим особое внимание 
на данный результат, поскольку аналогичная модель управления полумарковским 
процессом будет рассмотрена в~настоящей работе.

Принципиальную роль в~развитии теории стохастического управления сыграла 
монография И.\,И.~Гихмана и~А.\,В.~Скорохода~\cite{9}. В~этой книге были впервые 
систематически изложены основы теории оптимального управления случайными процессами 
с~дискретным и~непрерывным временем, включая теорию управления процессами, которые 
описываются стохастическими дифференциальными уравнениями. Отдельно были рас\-смот\-ре\-ны 
проблемы управления марковскими процессами с~дискретным временем и~скачкообразными 
марковскими процессами с~непрерывным временем. Роли множеств состояний и~допустимых 
управ\-ле\-ний играли пространства весьма общей структуры. Для широких классов функционалов 
качества управ\-ле\-ния (так называемых эволюционных функционалов в~марковских моделях 
с~дискретным временем и~интегральных функционалов накопления в~марковских моделях 
с~непрерывным временем) были доказаны теоремы о~существовании и~формах пред\-став\-ле\-ния 
оптимальных стратегий управ\-ле\-ния. Было установлено, что для однородных марковских 
моделей оптимальные стратегии управ\-ле\-ния существуют, являются стационарными 
и~детерминированными. Иначе говоря, такие стратегии задаются детерминированными 
функциями, аргументом которых является со\-сто\-яние сис\-те\-мы в~момент принятия решения, 
и~не зависящими от самого момента принятия решения. Что же касается важного вопроса 
о~формах представления этих функций, то их можно охарактеризовать следующим образом. 
Были найдены функциональные уравнения, осложненные условием экстремума, которым 
удовле\-тво\-ря\-ют упомянутые функции. По существу эти соотношения пред\-став\-ля\-ют собой 
уравнения Беллмана для соответствующих динамических стохастических моделей.

Особо отметим, что в~монографии~\cite{9} не рас\-смат\-ри\-ва\-лись проблемы управления 
полумарковскими процессами. Однако дальнейшее развитие общей теории управления 
такими процессами шло по пути, идейно намеченному в~указанной книге.

В последующие годы развитие теории управ\-ле\-ния полумарковскими процессами 
осуществля-\linebreak лось по направлению усложнения моделей и~обобщения исходных предположений. 
Например,\linebreak в~работах~\cite{10, 11} рассмотрены управляемые по\-лумарковские процессы при 
весьма общих предположениях относительно характера пространств состояний и~управлений. 
Проблемы управления исследовались по отношению к~различным видам целевых показателей, 
обобщающих упомянутый выше стационарный показатель средней удельной прибыли. В~этих 
работах доказывается, что оптимальная стратегия управления по отношению к~каж\-до\-му из 
показателей существует и~является одной и~той же стационарной детерминированной 
стратегией, определяемой некоторой функцией, заданной на множестве со\-сто\-яний процесса. 
Об этой функции известно лишь то, что она удовлетворяет некоторому интегральному 
уравнению, которое по содержанию пред\-став\-ля\-ет собой уравнение Бел\-лма\-на для 
соответствующей задачи управ\-ления.

Среди исследований, предшествовавших настоящему, отметим работу 
В.\,А.~Каштанова~[12, гл. 13]. В этом разделе коллективной монографии~\cite{12} 
автором была рассмотрена проблема оптимального управления полумарковским 
процессом с~конечным множеством состояний и~множеством возможных решений, 
которое представляет собой произвольный интервал множества вещественных чисел. 
Модель относится к~виду моделей без переоценки, показателем качества управления 
служит стационарное значение среднего удельного дохода, определяемое аналогично 
классическим работам~\cite{3, 8}. Рандомизированное управление в~каждом состоянии 
определяется в~соответствии с~вероятностным распределением, совокупность которых 
задает\linebreak
 стратегию управления. В.\,А.~Каш\-та\-но\-вым было\linebreak сформулировано утверждение о том, 
что стацио\-нарное значение среднего удельного дохода представляет собой 
дроб\-но-ли\-ней\-ный интегральный функционал от набора вероятностных распределений, 
образующих стратегию управления. При этом\linebreak ранее~[12, гл.~10; 13] было уста\-нов\-ле\-но, 
что дроб\-но-ли\-ней\-ный функционал достигает экстремума на вырожденных распределениях. 
Отсюда естест-\linebreak венно следует, что оптимальная стратегия управ\-ле-ния является 
детерминированной и~должна\linebreak определяться точкой экстремума функции, представляющей 
собой отношение подынтегральных функций чис\-ли\-те\-ля и~знаменателя данного 
дроб\-но-ли\-ней\-но\-го функционала. Однако в~\cite{12} не были получены явные 
представления для указан-\linebreak ных функций. Кроме того, приведенный в~гл.~10 
монографии~\cite{12} вариант теоремы об экстремуме дроб\-но-ли\-ней\-но\-го 
интегрального функционала требовал проверки выполнения условия существования 
этого экстремума. Такие условия указаны не были. В~связи с~этими обстоятельствами 
использовать полученные в~\cite{12} результаты для доказательства существования 
оптимальной детерминированной стратегии управ\-ле\-ния полумарковским процессом и~для 
строгого обоснования способа нахождения такой стратегии оказалось невозможным.

Настоящее исследование посвящено теоретическому обоснованию нового метода 
нахождения\linebreak оптимальной стратегии управления полумарковским процессом с~конечным 
множеством со\-сто\-яний. Рассматриваются марковские рандомизи\-рованные стратегии 
управления, определяемые конеч\-ным набором вероятностных мер, соответствующих 
каждому состоянию. Показателем качества управления служит уже упоминавшийся 
классический  показатель~--- стационарное значение средней удельной прибыли. 
Доказано, что этот показатель представляет собой дроб\-но-ли\-ней\-ный интегральный 
функционал от набора вероятностных мер, задающих стратегию управления. При этом, 
в~отличие от~\cite{12}, получены явные аналитические представления для подынтегральных 
функций числителя и~знаменателя этого дроб\-но-ли\-ней\-но\-го\linebreak
 функционала. Дальнейшие 
результаты основываются на новой усиленной и~обобщенной форме\linebreak
 теоремы об экстремуме 
дроб\-но-ли\-ней\-но\-го интегрального функционала, впервые опубликованной 
в~работе П.\,В.~Шнуркова~\cite{14}. Согласно\linebreak
 утверж\-де\-нию этой теоремы, если 
существует глобальный экстремум так называемой основной функции дроб\-но-ли\-ней\-но\-го 
функционала, которая пред\-став\-ля\-ет собой отношение подынтегральных функций чис\-ли\-те\-ля 
и~знаменателя, то существует безусловный экстремум самого дроб\-но-ли\-ней\-но\-го 
функционала, который достигается на наборе вырожденных вероятностных распределений, 
сосредоточенных в~точке глобального экстремума. В~этом случае оптимальная стратегия 
управ\-ле\-ния существует, является стационарной и~детерминированной и~определяется точкой, 
в~которой основная\linebreak функция достигает глобального экстремума. Таким\linebreak образом, проблемы 
существования оптимальной стратегии управ\-ле\-ния полумарковским процессом и~ее 
нахождения сводятся к~задаче чис\-лен\-но\-го исследования на глобальный экстремум 
заданной функции от конечного чис\-ла вещественных переменных.

\section{Общее описание модели управления полумарковским случайным процессом}

Построим модель управления полумарковским случайным процессом, следуя общему 
подходу, принятому в~классических работах~\cite{3, 8}. Пусть $\xi(t)$~--- 
случайный полумарковский процесс с~конечным множеством состояний
$X\hm=\{1,2,\ldots, N\}$, $N\hm< \infty$. Обозначим через~$t_n$, $n=0,1,2,\ldots$, 
$t_0\hm=0$, случайные моменты изменения состояний данного процесса, 
$\theta_n\hm=t_{n+1}-t_n$, $n\hm=0,1,2,\ldots$, $\xi_n\hm=\xi(t_n)\hm=\xi(t_n+0)$, 
$n\hm=0,1,2,\ldots$ (предполагается, что траектории процесса~$\xi(t)$ 
непрерывны справа). Случайная последовательность~$\{\xi_n\}$
образует цепь Маркова, вложенную в~полумарковский процесс~$\xi(t)$.
Зададим набор измеримых пространств\linebreak $(U_1, \mathscr{B}_1), 
(U_2, \mathscr{B}_2), \ldots, (U_N, \mathscr{B}_N)$, где $U_i$~--- 
множество возможных допустимых управ\-ле\-ний, $\mathscr{B}_i$~--- $\sigma$-ал\-геб\-ра 
подмножеств множества~$U_i$, вклю\-ча\-ющая в~себя все одноточечные подмножества\linebreak  
множества~$U_i$, т.\,е.\ если $u_i \hm\in U_i$, то $\{u_i\} \hm\in \mathscr{B}_i$, 
$i\hm=1,2,\ldots, N$.
Пусть $\Gamma_i$~--- некоторое множество всевозможных вероятностных мер $\Psi_i 
\hm \in \Gamma_i$, заданных на $\sigma$-ал\-геб\-ре~$\mathscr{B}_i$, $i\hm=1,2,\ldots,N$.

Поскольку идейное содержание и~свойства вероятностных мер существенно используются 
в~данной работе, укажем на некоторые фундаментальные издания, в~которых 
изложена соответствующая тео\-рия. Понятие и~основные свойства вероятностной 
меры определены и~подробно проанализированы в~книге А.\,Н.~Ширяева~\cite[гл.~II]{15}. 
Глубокое изложение основ теории вероятностных мер имеется также в~книге 
А.\,А.~Боровкова~\cite{16}. Заметим попутно, что в~книге~\cite{16} имеются разделы, 
посвященные изложению основ теории полумарковских и~регенерирующих случайных процессов. 
Из зарубежных изданий отметим фундаментальную работу П.~Хеннекена и~А.~Тортра~\cite{17}, 
основная часть которой посвящена изложению математических основ теории вероятностей.

Введем специальное понятие вырожденной вероятностной меры, которое будет часто 
использоваться в~дальнейшем. Пусть $(U_0, \mathscr{B}_0)$~--- некоторое измеримое 
пространство, $\mathscr{B}_0$~--- $\sigma$-ал\-геб\-ра подмножеств множества~$U_0$, 
включающая в~себя все одноточечные подмножества этого множества.

\medskip

\noindent
\textbf{Определение 1.}\ Вероятностная мера~$\Psi^*$, заданная 
на $\sigma$-ал\-геб\-рe~$\mathscr{B}_0$, называется вырожденной, если существует 
такой элемент $u^* \hm\in U_0$, для которого выполняются условия $\Psi^*(\{u^*\})\hm=
1$, $\Psi^*(U_0 \setminus \{u^*\})\hm=0$, где $\{u^*\}=u^*$~--- 
множество, состоящее из единственной точки $u^* \hm\in U_0$. Соответствующая 
точка $u^* \hm\in U_0$ будет называться точкой сосредоточения вырожденной 
вероятностной меры~$\Psi^*$.
Таким образом, всякая вырожденная вероятностная мера~$\Psi^*$ определяется 
своей точкой сосредоточения~$u^*$. В~дальнейшем будем использовать 
обозначение~$\Psi_{u^*}^{*}$, имея в~виду, что вырожденная вероятностная мера~$\Psi^*$ 
сосредоточена в~точке~$u^*$.
Отметим также, что вырожденная вероятностная мера~$\Psi_{u^*}^{*}$ соответствует 
детерминированной величине, которая принимает фиксированное значение $u\hm=u^*$ 
с~вероятностью, равной единице.

\medskip

Обозначим через $\Gamma_0$ множество всех  вероятностных мер, заданных 
на измеримом пространстве ($U_0, \mathscr{B}_0$), а через~$\Gamma_0^*$~--- 
множество всех вырожденных вероятностных мер, заданных на этом пространстве, 
$\Gamma_0^*\hm\in \Gamma_0$. Аналогичные обозначения будут использоваться 
и~в~дальнейшем. Заметим, что множество~$\Gamma_0^*$ находится во взаимно
 однозначном соответствии с~множеством точек сосредоточения вырожденных 
 вероятностных мер, т.\,е.\ с~множеством~$U_0$.

Пусть $\Gamma_i^{*}$~--- множество всех вырожденных мер, заданных на 
$\sigma$-ал\-геб\-ре~$\mathscr{B}_i$, $\Gamma_i^{*}\hm\subset \Gamma_i$.
Произвольная вероятностная мера~$\Psi_i$ описывает случайную величину, 
принимающую значения в~$U_i$, а вырожденная мера~$\Psi_i^*$, сосредоточенная 
в~точке~$u_i^*$, соответствует детерминированной величине $u_i^*\hm\in U_i$.
Предполагается, что соответствующие конструкции определены на всех измеримых 
пространствах управлений $(U_1, \mathscr{B}_1), (U_2, \mathscr{B}_2), \ldots, 
(U_N,\mathscr{B}_N)$.

Предположим, что управления случайным полумарковским процессом~$\xi(t)$ 
осуществляются в~моменты времени~$t_n,$ $n\hm=0,1,2,\ldots,$
непосредственно после изменения состояния процесса. Если\linebreak 
$\xi_n\hm=\xi(t_n)\hm=i \hm\in X$, то значение управления представляет 
собой случайную величину~$u_n$, принимающую значения в~множестве допустимых 
управ\-ле\-ний~$U_i$ и~описываемую вероятностной мерой (распределе\-ни\-ем 
вероятностей) $\Psi_i \hm\in \Gamma_i$.
Будем предполагать, что при фиксированном условии $\xi_n\hm=\xi(t_n)=i$ 
управ\-ле\-ние определяется независимо от прошлого поведения процесса~$\xi(t)$ 
и~вероятностная мера~$\Psi_i$,
описывающая стохастическое управление~$u_n$, зависит только от состояния $i\hm\in X$.
Тогда выбор управ\-ле\-ний в~моменты изменения состояний $\{t_n, n\hm=0,1,2,\ldots \}$ 
описывается набором вероятностных мер (распределений вероятностей) 
$(\Psi_1, \Psi_2,\ldots, \Psi_N)$, 
$\Psi_i \hm\in \Gamma_i$, $i\hm=1,2,\ldots,N$.
Назовем любой такой набор стратегией управ\-ле\-ния полумарковским процессом~$\xi(t)$. 
По своим свойствам такая стратегия является марковской, однородной 
и~рандомизированной.

Следуя классической монографии П.~Халмоша~\cite[гл.~VII]{18}, 
рассмотрим декартово произведение пространств $U\hm=U_1\times U_2\times \cdots\times U_N$ 
и~соответствующих $\sigma$-ал\-гебр $\mathscr{B}\hm=\mathscr{B}_1\times \mathscr{B}_2
\times \cdots \times\mathscr{B}_N$. Обозначим через $\Psi\hm=\Psi_1\times \Psi_2\times \cdots
\times \Psi_N$ вероятностную меру на~$(U,\mathscr{B})$, определяемую как 
произведение мер $\Psi_1,\Psi_2, \ldots , \Psi_N$, где $\Psi_i \hm\in \Gamma_i$, 
$i\hm=1,2,\ldots,N$. Обозначим также через~$\Gamma$ множество вероятностных мер~$\Psi$, 
заданных на~$(U,\mathscr{B})$, которые пред\-став\-ля\-ют собой произведение 
мер $\Psi_1,\Psi_2, \ldots , \Psi_N$, где $\Psi_i \hm\in \Gamma_i$, $i\hm=1,2,\ldots,N$.
Множество~$\Gamma$ можно отож\-де\-ст\-вить с~множеством всех стратегий управ\-ле\-ния 
полумарковским процессом~$\xi(t)$.

Проблема оптимального управления полумар\-ковским процессом~$\xi(t)$ будет в~дальнейшем 
сформулирована в~виде задачи безусловного экстремума некоторого функционала 
$I(\Psi)\hm=I(\Psi_1,\Psi_2, \ldots , \Psi_N)$, заданного на множестве 
допустимых стратегий управления. Содержание показателя качества управления~$I(\Psi)$, 
аналитическое представление для него, а~также описание множества допустимых 
стратегий управления будут приведены в~последующих разделах данной работы.

Для получения дальнейших результатов потребуются различные вероятностные 
характеристики управляемого полумарковского процесса~$\xi(t)$. Как известно из
 общей теории полумарковских процессов~\cite{19, 20}, 
 основной вероятностной характеристикой такого процесса является так называемая 
 полумарковская функция. Определим эту функцию для процесса с~управлением 
 (см.~\cite[гл.~5]{8}):
\begin{multline}
Q_{ij}(t,u)=
{\sf P}\left(\xi_{n+1}=j,\theta_n<t \mid \xi_n=i, u_n=u\right)\,,\\
t\in [0,\infty)\,,\ u\in U_i\,;\ i,j\in X=\{1,2,\ldots,N\}\,. \label{e1}
\end{multline}
Используя полумарковские функции, можно получить вероятности перехода 
управляемой цепи Маркова~$\{\xi_n\}$:
\begin{multline}
p_{ij}(u)={\sf P}\left(\xi_{n+1}=j \mid \xi_n=i, u_n=u\right)= {}\\
{}=
\lim\limits_{t\rightarrow\infty}Q_{ij}(t,u)\,,\enskip
u\in U_i\,;\enskip i,j\in X\,, 
\label{e2}
\end{multline}
а также функции распределения длительностей пребывания полумарковского 
процесса~$\xi(t)$ в~соответствующих состояниях:

\noindent
\begin{multline}
H_{i}(t,u)={\sf P}\left(\theta_n<t \mid \xi_n=i, u_n=u\right)={}\\
{}=
\sum\limits_{j\in X}Q_{ij}(t,u)\,,\enskip
t\in [0,\infty)\,,\  u\in U_i\,; \  i\in X\,. 
\label{e3}
\end{multline}

Обозначим через
\begin{multline}
T_{i}(u)=\mathbf{E}\left[\theta_n \mid \xi_n=i, u_n=u\right]={}\\
{}=
\int\limits_0^{\infty}\left[1-H_i(t,u)\right]\,dt\,,\enskip
u\in U_i\,,\ i\in X\,, 
\label{e4}
\end{multline}
математические ожидания длительностей пребывания полумарковского процесса~$\xi(t)$ 
в~каждом из состояний.

Введенные выше характеристики~(1)--(4) определены для случая, когда 
в~момент изменения состояния~$t_n$ процесс оказывается в~состоянии~$i$ 
и~принимается решение $u\hm\in U_i$. При заданной стратегии управления 
$\Psi\hm=\left(\Psi_1,\Psi_2, \ldots , \Psi_N\right)$ можно записать 
соответствующие вероятностные характеристики без условия на управление, а~именно:
\begin{multline*}
Q_{ij}(t)={\sf P}\left(\xi_{n+1}=j,\theta_n<t \mid \xi_n=i\right)={}\\
{}=
\int\limits_{U_i}Q_{ij}(t,u) \,d\Psi_i(u)\,,\enskip 
t\in [0,\infty)\,,\ i,j\in X\,; %\label{e5}
\end{multline*}

\vspace*{-12pt}

\noindent
\begin{multline}
p_{ij}={\sf P}\left(\xi_{n+1}=j \mid \xi_n=i\right)=
\int\limits_{U_i}p_{ij}(u)\, d\Psi_i(u)\,,\\  
i,j\in X\,; 
\label{e6}
\end{multline}

\vspace*{-9pt}

\noindent
\begin{equation}
T_{i}=\mathbf{E}\left[\theta_n \mid \xi_n=i\right]=
\int\limits_{U_i}T_{i}(u)\,d\Psi_i(u)\,,\enskip i\in X\,. 
\label{e7}
\end{equation}
В дальнейшем будем предполагать, что для рас\-смат\-ри\-ва\-емой 
полумарковской модели заданы вероятностные характеристики 
$p_{ij}(u)$, $u\hm\in U_i$, $i,j\hm\in X$, и~$T_i(u)$, $u\hm\in U_i$, $i\hm\in X$, 
определяемые соотношениями~(\ref{e2}) и~(\ref{e4}). 
Для фиксированной стратегии управления $\Psi\hm=(\Psi_1, \Psi_2,\ldots, \Psi_N)$ 
соответствующие вероятностные характеристики~$p_{ij}$ и~ $T_i$, $i,j\hm\in X,$ 
определены равенствами~(\ref{e6}) и~(\ref{e7}) без условий на управление.

\section{Стационарный стоимостной показатель качества управления}

Определим некоторый стоимостной аддитивный функционал, связанный 
с~рассматриваемым полумарковским процессом~$\xi(t)$. По содержанию этот функционал 
представляет собой случайный\linebreak доход или прибыль, накопленную за период времени $[0,t]$. 
Определения такого функционала приведены в~основополагающих работах~[3; 8, гл.~5].\linebreak 
Обозначим через $\widetilde{v}(t)$, $t\hm\geq 0$, значение этого аддитивного 
функционала в~момент времени~$t$; $\widetilde{v}_n\hm=\widetilde{v}(t_n\hm+0)$~--- 
соответствующее значение непосредственно после очередного момента изменения\linebreak 
состояния~$t_n$, $n\hm=0,1,2,\ldots$; $\widetilde{v}_0\hm=v_0$~--- 
заданное начальное значение в~момент $t\hm=0$. Рассмотрим величину
\begin{multline}
d_i(u)=\mathbf{E}\left[\widetilde{v}_{n+1}-\widetilde{v}_n \mid \xi_n=i\,, 
u_n=u\right]\,,\\
u\in U_i\,, \enskip i\in X\,, \label{e8}
\end{multline}
представляющую собой математическое ожидание приращения стоимостного 
аддитивного функционала за период времени между последовательными 
изменениями состояния полумарковского процесса~$\xi(t)$. Тогда соответствующее 
математическое ожидание, вычисляемое без условия на решение, 
принимаемое в~момент времени~$t_n$, представляется в~виде:
\begin{equation*}
d_i=\mathbf{E}\left[\widetilde{v}_{n+1}-\widetilde{v}_n \mid \xi_n=i\right]=
\!\int\limits_{U_i}\!d_i(u)\,d\Psi_i(u)\,,\ i\in X \,. %\label{e9}
\end{equation*}

Предположим, что для заданной стратегии управ\-ле\-ния 
$\Psi\hm=(\Psi_1,\Psi_2,\ldots,\Psi_N)$ вложенная цепь Маркова~$\{\xi_n\}$ 
имеет ровно один класс возвратных положительных состояний (по терминологии, 
принятой в~\cite{8}, такое множество состояний называется эргодическим классом). 
Как известно~\cite[гл.~VIII]{15}, данное условие является необходимым 
и~достаточным для существования единственного\linebreak стационарного распределения. 
Обозначим это стационарное распределение цепи Маркова~$\{\xi_n\}$ через 
$\pi\hm=(\pi_1, \pi_2,\ldots, \pi_N)$. Заметим, что данное\linebreak распределение зависит  
от стратегии управления $\Psi\hm=(\Psi_1,\Psi_2,\ldots,\Psi_N)$. При указанном 
условии имеет место следующий результат, называемый эргодической теоремой 
для аддитивного стоимостного функционала:
\begin{equation}
I=\lim\limits_{t\rightarrow\infty}\fr{\mathbf{E}\widetilde{v}(t)}{t}=
\fr{\sum\nolimits_{i=1}^N d_i\pi_i}{\sum\nolimits_{i=1}^N T_i\pi_i}\,. 
\label{e10}
\end{equation}

Соотношение~(\ref{e10}) доказано в~работе~\cite[гл.~5]{8}. Заметим, что аналогичные 
результаты имеют мес\-то для гораздо более общих полумарковских моделей~\cite{10, 11}.

По своему прикладному содержанию величина, определяемая соотношением~(\ref{e10}), 
представляет собой
среднюю удельную прибыль, связанную с~эволюцией системы в~стационарном
режиме. Кроме того, величина~$I$ представляет собой функционал от
набора вероятностных распределений~$\Psi_{i}$, $i\hm\in\lbrace 1,\ldots
,N\rbrace $, определяющих стратегию управле-\linebreak\vspace*{-12pt}

\pagebreak

\noindent
ния системой. 
В~дальнейшем будем рассматривать стационарный стоимостной функционал 
$I\hm=I(\Psi_{1},\Psi_{2},\ldots , \Psi_{N})$ как
показатель качества управ\-ле\-ния системой и~построенным полумарковским
процессом~$\xi (t)$.

\section{Представление стационарного показателя в~форме
дробно-линейного интегрального функционала}

В данном разделе будет приведено утверждение об аналитическом
представлении стационарного стоимостного функционала~(\ref{e10}), 
служащего критерием качества управления в~рассматриваемой задаче управления 
полумарковским процессом.

\smallskip

\noindent
\textbf{Теорема 1.} \textit{Стационарный стоимостной показатель, 
определяемый равенством}~(\ref{e10}), \textit{представляет собой дроб\-но-ли\-ней\-ный
функционал от вероятностных распределений~$\Psi_{i}(u_{i})$,
$i\hm\in\{1,\dots,N\}$. Данный функционал задается
аналитически следующей формулой:}
\begin{multline}
I=I(\Psi_{1},\ldots, \Psi_{N})={}\\
\hspace*{-2mm}{}=\!
\fr{\int\nolimits_{U_1}\!{\cdots\! 
\int\nolimits_{U_N}\!{A(u_{1},\ldots ,u_{N})d\Psi_{1}(u_{1})\cdots
\,d\Psi_{N}(u_{N})}}}{\int\nolimits_{U_1}{\!\cdots\! \int\nolimits_{U_N}\!{B(u_{1},\ldots ,u_{N})
\,d\Psi_{1}(u_{1})\ldots
d\Psi_{N}(u_{N})}}},\!\!\! \label{e11}
\end{multline}
\textit{где подынтегральные функции числителя и~знаменателя выражаются
соотношениями}:
\begin{align}
A(u_{1},\ldots
,u_{N})&={}\notag\\
&\hspace*{-20mm}{}=\sum\limits_{i=1}^{N}{d_{i}(u_{i})}{\widehat{D}}^{(i)}(u_{1}, \ldots
,u_{i-1},u_{i+1}, \ldots , u_{N})\,;  \label{e12}\\
 B(u_{1},\ldots
,u_{N})&={}\notag\\
&\hspace*{-20mm}{}=\sum\limits_{i=1}^{N}{T_{i}(u_{i})}{\widehat{D}}^{(i)}(u_{1}, \ldots
,u_{i-1},u_{i+1}, \ldots , u_{N})\,.  \label{e13}
\end{align}
\textit{В свою очередь, функции} ${\widehat{D}}^{(i)}(u_{1}, \ldots
,u_{i-1},u_{i+1}, \ldots$\linebreak $\ldots , u_{N})$, $i\hm\in\{1,\dots,N\}$, 
\textit{входящие в~правые части формул}~(\ref{e12}) и~(\ref{e13}), 
\textit{определяются следующим образом:}

\noindent
\begin{multline}
{\widehat{D}}^{(i)}(u_{1}, \ldots ,u_{i-1},u_{i+1}, \ldots , u_{N})={}
\\
{}=(-1)^{N+i+2}\sum\limits_{\alpha ^{(N),i}}{(-1)}^{\delta (\alpha
^{(N),i}) }{\widehat{D}}_{0}^{(i)}\left(\alpha ^{(N),i},u_{1}, \ldots\right.\\
\left.\ldots , u_{i-1},u_{i+1}, \ldots , u_{N}\right)\,. \label{e14}
\end{multline}
\textit{Здесь} $\alpha ^{(N),i}=(\alpha _{1}, \ldots , \alpha _{i-1},\alpha_{i+1}, \ldots , 
\alpha _{N})$~\textit{--- произвольная
перестановка чисел }$(1, \ldots , i-1, i+1, \ldots , N)$;
$\delta
(\alpha ^{(N),i})$~\textit{--- число инверсий в~перестановке} 
$\alpha ^{(N),i}$;

\noindent
\begin{multline}
{\widehat{D}}_{0}^{(i)}(\alpha ^{(N),i},u_{1}, \ldots ,u_{i-1},u_{i+1},
\ldots , u_{N})={}\\
{} ={\widetilde{p}}_{1,\alpha _{1}}\left(u_{1}\right)\cdots {\widetilde{p}}_{i-1,\alpha
_{i-1}}\left(u_{i-1}\right){\widetilde{p}}_{i+1,\alpha _{i+1}}\left(u_{i+1}\right)\cdots\\
\cdots
{\widetilde{p}}_{N,\alpha _{N}}\left(u_{N}\right)\,, 
\label{e15}
\end{multline}
где
\begin{multline}
 {\widetilde{p}}_{k,\alpha _{k}}(u_{k})=
\begin{cases}
p_{kk}(u_{k})-1,\  & \alpha _{k}=k\,; \\
p_{k,\alpha _{k}}(u_{k}),\  & \alpha _{k}\ne k, \\
\end{cases}\\
 k=1, \ldots , i-1, i+1, \ldots ,N\,. \label{e16}
 \end{multline}
\textit{Функции $p_{ij}(u_i)$, $T_{i}(u_{i})$ и~$d_{i}(u_{i})$,
$u_i\hm\in U_i$, $i,j\hm\in \{1,2,\ldots,N\}$, 
входящие в~соотношения}~(\ref{e12})--(\ref{e16}), 
\textit{определяются равенствами}~(\ref{e2}), (\ref{e4}) \textit{и}~(\ref{e8}) \textit{соответственно.}

\smallskip

\noindent
Д\,о\,к\,а\,з\,а\,т\,е\,л\,ь\,с\,т\,в\,о\ теоремы~1 
в~весьма сжатой форме приведено в~работе~\cite{21}. Читателю, интересующемуся 
более подробным обоснованием данного результата, порекомендуем обратиться к~тексту 
кандидатской диссертации А.\,В.~Иванова~\cite[гл.~3]{22}.

\smallskip

Итак, теорема~1 позволяет получить явное аналитическое представление 
для стационарного стоимостного показателя вида~(\ref{e10}) в~форме 
дроб\-но-ли\-ней\-но\-го интегрального функционала от набора\linebreak вероятностных мер 
$\Psi\hm=(\Psi_{1},\Psi_{2},\ldots , \Psi_{N})$, за\-да\-ющих стратегию управления 
полумарковским процессом~$\xi(t)$. При этом подынтегральные функции числителя 
и~знаменателя задаются формулами~(\ref{e12}), (\ref{e13}) 
и~вспомогательными равенствами~(\ref{e14})--(\ref{e16}). Таким образом, функция
\begin{equation}
C\left(u_1, u_2,\ldots, u_N\right)=\fr{A(u_1, u_2,\ldots, u_N)}{B(u_1, u_2,\ldots, u_N)}\,,
\label{e17}
\end{equation}
которая в~дальнейшем будет называться основной функцией дроб\-но-ли\-ней\-но\-го 
интегрального функционала~(\ref{e11}) и~которая будет играть важную роль 
в~дальнейшем исследовании, также явно определяется формулами~(\ref{e17}), 
(\ref{e12}), (\ref{e13}).

\section{Формальная постановка оптимизационной задачи 
и~условия существования оптимальной стратегии управления полумарковским процессом}

Будем рассматривать проблему управления полумарковским процессом~$\xi(t)$ в~форме 
экстремальной задачи
\begin{multline}
I(\Psi)=I\left(\Psi_1, \Psi_2,\ldots,\Psi_N\right)\rightarrow \mathrm{extr}\,,
\\
\Psi=\left(\Psi_1, \Psi_2,\ldots,\Psi_N\right)\in\Gamma\,. \label{e18}
\end{multline}
При этом показатель качества управления~$I(\Psi)$ представляет собой 
дроб\-но-ли\-ней\-ный интегральный функционал вида~(\ref{e11}).

Для решения экстремальной задачи~(\ref{e18}) воспользуемся некоторым утверждением 
об экстремуме дроб\-но-ли\-ней\-но\-го интегрального функционала. Прежде 
чем сформулировать данное утверждение, отметим, что в~теории оптимизации 
хорошо известны задачи, в~которых целевая функция представляет собой 
отношение двух линейных отображений, а имеющиеся ограничения также линейны. 
Такой раздел называется дроб\-но-ли\-ней\-ным программированием. Основные
 теоретические результаты данного направления изложены в~работе~\cite{23},
  там же приведена подробная библиография. В~дальнейшем потребуется некоторый 
  специальный результат о безусловном экстремуме дроб\-но-ли\-ней\-но\-го 
  интегрального функционала вида~(\ref{e11}), который был впервые сформулирован 
  в~работе~\cite{14}. Заметим, что для использования этого результата необходимо, 
  чтобы выполнялись некоторые предварительные условия, которые в~данном случае 
  можно сформулировать следующим образом:
\begin{enumerate}[1.]
\item Интегральные выражения
\begin{align*}
I_1(\Psi)&=I_1\left(\Psi_1,\Psi_2,\ldots,\Psi_N\right)={}&\\
&\hspace*{-13mm}{}=\int\limits_{U_1}\!\cdots\!
\int\limits_{U_N}\!\!A\left(u_1,\ldots ,u_N\right)\,
d\Psi_1\left(u_1\right) %d\Psi_2\left(u_2\right)
\cdots
 d\Psi_N\left(u_N\right)\,;
\\
I_2(\Psi)&=I_2\left(\Psi_1,\Psi_2,\ldots,\Psi_N\right)={}&\\
&\hspace*{-13mm}{}=\int\limits_{U_1}\!\cdots\!\int\limits_{U_N}\!\!
B\left(u_1,\ldots,u_N\right)\,
d\Psi_1\left(u_1\right)% d\Psi_2\left(u_2\right)\cdots\\
\cdots d\Psi_N\left(u_N\right)
\end{align*}
определены для всех стратегий управления $\Psi\hm=(\Psi_1, \ldots,\Psi_N)
\hm\in \Gamma$.

\item Функционал $I_2(\Psi)=I_2(\Psi_1, \ldots,\Psi_N)\hm\neq 0$ 
для всех $\Psi\hm=(\Psi_1, \ldots,\Psi_N)\hm\in \Gamma$.

\item Множество $\Gamma$ включает в~себя множество всех вырожденных 
вероятностных мер: $\Gamma^* \hm\subset \Gamma$.
\end{enumerate}

Сделаем несколько важных замечаний по поводу введенных предварительных условий.

\smallskip

\noindent
\textbf{Замечание~1.}\ Из условия~2 следует, что функция $B(u_1, u_2,\ldots, u_N)$ 
не может принимать значения разных знаков. С~учетом условия~3 
получаем, что указанная функция должна обладать \mbox{свойством} строгой 
знакопостоянности на всем множестве~$U$. С~другой стороны, если выполняется 
условие строгой знакопостоянности функции $B(u_1, u_2,\ldots, u_N), 
(u_1, u_2,\ldots, u_N)\hm\in U$, то условие~2 выполняется автоматически.

\smallskip

\noindent
\textbf{Замечание~2.}\ Если рассматривать в~качестве целевого функционала 
$I(\Psi_1, \Psi_2,\ldots,\Psi_N)$ экстремальной задачи~(\ref{e18}) 
стационарный стоимостной пока\-затель~(\ref{e10}), то функция $B(u_1,u_2,\ldots,u_N)$ 
имеет\linebreak следующее теоретическое содержание. Данная функция представляет собой условное 
математическое ожидание длительности периода времени между соседними моментами 
изменения со\-сто\-яния полумарковского процесса~$\xi(t)$ при условии, что стратегия 
его управ\-ле\-ния является детерминированной и~задается набором значений аргументов 
$(u_1,u_2,\ldots,u_N)$. Тогда условие строгой положительности функции 
$B(u_1,u_2,\ldots,u_N)$ при всех $(u_1,u_2,\ldots,u_N)\hm\in U$ является естественным 
и~фактически означает, что при любой заданной детерминированной стратегии 
управ\-ле\-ния процесс~$\xi(t)$ не имеет мгновенных со\-сто\-яний, длительность пребывания 
в~которых равна нулю.

\smallskip

\noindent
\textbf{Замечание~3.}\ Сделаем некоторые замечания, связан\-ные с~подынтегральной 
функцией числителя дроб\-но-ли\-ней\-но\-го интегрального функционала~(\ref{e11}). 
Как и~ранее, будем рассматривать в~качестве целевого функционала $I(\Psi_1, \Psi_2,\ldots,\Psi_N)$\linebreak 
экстремальной задачи~(\ref{e18}) стационарный стоимостной показатель~(\ref{e10}). 
Тогда для любого фиксированного набора значений аргументов $(u_1,u_2,\ldots,u_N)\hm\in U$ 
значение функции $A(u_1,u_2,\ldots\linebreak \ldots,u_N)$ представляет собой условное математическое
 ожидание приращения рассматриваемого стоимостного функционала, 
 происшедшее за время пребывания полумарковского процесса~$\xi(t)$ в~некотором 
 фиксированном  состоянии при условии, что стратегия управления является 
 детерминированной и~задается указанным набором $(u_1,u_2,\ldots,u_N)\hm\in U$. 
 Отметим, что в~теореме об экстремуме дроб\-но-ли\-ней\-но\-го интегрального 
 функционала, доказанной в~работе~\cite[гл.~10]{12}, 
 на подынтегральную функцию числителя накладываются условия ограниченности на 
 всем множестве значений аргумента. Для многих математических моделей и~связанных 
 с~ними задач оптимального управления такое условие является излишне ограничительным. 
 В~качестве примера можно привести модели оптимального управления запасом непрерывного 
 продукта, рассмотренные в~работах~\cite{27, 28}. 
 В~настоящем исследовании на функцию $A(u_1,u_2,\ldots,u_N)$ накладывается только 
 условие интегрируемости по любому заданному набору вероятностных мер 
 $\Psi\hm=(\Psi_1, \Psi_2,\ldots,\Psi_N)$, образующему стратегию управления 
 полумарковским процессом~$\xi(t)$ (условие~1 системы предварительных условий).

\smallskip

\noindent
\textbf{Замечание~4.} Условия~1--3 являются необходимыми для корректной 
постановки задачи безусловного экстремума дроб\-но-ли\-ней\-но\-го интегрального 
функционала. Если этот функционал служит показателем качества в~задаче оптимального 
управления случайным процессом, то необходимо добавить к~этим условиям дополнительное, 
связанное с~некоторой регулярностью самого управляемого процесса, а~именно: некоторый 
содержательный показатель, связанный с~поведением этого процесса, должен существовать 
и~быть представимым в~виде дроб\-но-ли\-ней\-но\-го интегрального функционала. 
Если потребовать, чтобы выполнялось эргодическое соотношение~(\ref{e10}), 
то можно использовать\linebreak теорему~1 и~сформулировать задачу оптимального управ\-ле\-ния 
в~виде~(\ref{e18}) для дроб\-но-ли\-ней\-но\-го\linebreak интегрального функционала~(\ref{e11}). 
Таким образом, необходимо ввести условие, обеспечивающее существование единственного 
стационарного распределения вложенной цепи Маркова и~выполнение\linebreak соотношения~(\ref{e10}). 
По аналогии с~[8, гл.~5] сформулируем это дополнительное условие в~следующем виде:
\begin{enumerate}
\setcounter{enumi}{3}
\item Для любой рассматриваемой стратегии управ\-ле\-ния $\Psi\hm=
(\Psi_1, \Psi_2,\ldots,\Psi_N)\hm\in \Gamma$ вложенная цепь Маркова 
полумарковского процесса $\xi(t)$ имеет ровно один класс возвратных 
положительных состояний.
\end{enumerate}

Теперь определим понятие допустимой стратегии управления полумарковским процессом 
с~конечным множеством состояний.

\smallskip

\noindent
\textbf{Определение~2.}\ Назовем стратегию управления 
$\Psi\hm=(\Psi_1, \Psi_2,\ldots,\Psi_N)$ 
допустимой в~данной задаче, если она удовлетворяет условиям~1--4.


\smallskip

\noindent
\textbf{Замечание~5.}\ Как следует из замечания~1, если потребовать, 
чтобы функция $B(u_1, u_2,\ldots,u_N)$ являлась строго знакопостоянной при 
всех $(u_1, u_2,\ldots,u_N)\hm\in U$, то можно считать допустимыми стратегии 
$(\Psi_1, \Psi_2,\ldots,\Psi_N)$, удовлетворяющие условиям~1, 3, 4. С~учетом замечания~2 
о~естественном характере условия строгой знакопостоянности функции $B(u_1,u_2,\ldots,u_N)$ 
при всех значениях аргументов $(u_1, u_2,\ldots,u_N)\hm\in U$ будем требовать 
выполнения этого условия в~формулировке приводимой в~дальнейшем основной 
теоремы об оптимальной стратегии управления полумарковским процессом.

\smallskip

\noindent
\textbf{Замечание~6.}\ Ниже будет сформулирована и~доказана основная 
теорема об оптимальной стра\-тегии управления полумарковским процессом с~конеч\-ным 
множеством состояний. Будем формулировать эту теорему по отношению к~экстремальной 
задаче~(\ref{e18}), в~которой целевой функционал $I(\Psi_1, \Psi_2,\ldots,\Psi_N)$ 
имеет вид дроб\-но-ли\-ней\-но\-го интегрального функционала. 
Это обстоятельство связано с~тем, что целевой функционал в~задаче 
оптимального управления необязательно должен иметь характер стационарного 
стоимостного показателя вида~(\ref{e10}). В~частности, еще в~1983~г.\ П.\,В.~Шнурковым 
было установлено~\cite{24}, что ряд показателей, связанных 
с~временем пребывания управляемого полумарковского процесса в~заданном конечном 
подмножестве состояний, имеет структуру дроб\-но-ли\-ней\-но\-го интегрального 
функционала от набора вероятностных мер, определяющих стратегию управления. 
Таким образом, рассматриваемая задача управления имеет более общий характер, 
чем задача, в~которой целевой функционал представляет собой стационарный 
стоимостной показатель вида~(\ref{e10}).






\smallskip

\noindent
\textbf{Замечание~7.}\ Если рассматривать задачу оптимального управления 
полумарковским процессом, в~кото\-рой целевой функционал не совпадает 
со стационарным стоимостным показателем~(\ref{e10}), то возможно, что могут 
потребоваться другие дополнительные условия, обеспечивающие существование этого 
показателя и~его представление в~форме~(\ref{e11}). В~связи с~этим в~формулировке 
основной теоремы будем использовать термин допустимые стратегии в~широком смысле, 
имея в~виду выполнение всех необходимых условий для каждого рассмат\-ри\-ва\-емо\-го 
показателя качества управления.

\smallskip


\noindent
\textbf{Замечание 8.} Множество допустимых стратегий может 
не совпадать с~множеством всех возможных стратегий управления. 
В~частности, допустимые стратегии могут состоять только из дискретных вероятностных 
мер $\Psi_1, \Psi_2,\ldots,\Psi_N$, т.\,е.\ таких, которые сосредоточены на дискретных 
множествах точек пространств $U_1, U_2,\ldots,U_N$.

\section{Теоретическое решение задачи оптимального управления}

Перейдем к~формулировке и~доказательству тео\-ре\-мы об 
оптимальной стратегии управ\-ле\-ния полумарковским процессом с~конечным 
множеством состояний.

\smallskip

\noindent
\textbf{Теорема~2.} \textit{Рассмотрим проблему оптимального управ\-ле\-ния 
полумарковским процессом~$\xi(t)$ в~виде экстремальной задачи}~(\ref{e18}), 
\textit{определенной на множестве допустимых стратегий $\Gamma$, 
для дроб\-но-ли\-ней\-но\-го 
функционала}~(\ref{e11}). \textit{Пусть функция $B(u_1,u_2,\ldots,u_N)$, 
входящая в~определение функционала}~(\ref{e11}),
\textit{является строго знакопостоянной (строго положительной или строго отрицательной) 
при всех значениях аргументов $(u_1,u_2,\ldots,u_N)\hm\in U$.
Тогда справедливы сле\-ду\-ющие утверждения}:
\begin{enumerate}[1.]
\item \textit{Если функция} $C(u_1,u_2,\ldots,u_N)\hm=A(u_1,u_2,\ldots$\linebreak
$\ldots,u_N)/{B(u_1,u_2,\ldots,u_N)}$ 
\textit{ограничена сверху или снизу и~достигает глобального экст\-ре\-му\-ма на множестве
$U\hm=U_1\times U_2\times \cdots \times U_N$ (максимума или минимума), 
то оптимальная стратегия управления полумарковским процессом~$\xi(t)$ существует, 
является детерминированной и~определяется
вырожденной вероятностной мерой $\Psi^*\hm\in \Gamma^*$, сосредоточенной в~точке, 
в~которой достига\-ет соответствующего экстремума функция $C(u_1,u_2,\ldots,u_N)$,
и~при этом выполняются соотношения}:
\begin{multline}  %{\substack{{i=\overline{1,n}}\\ {j=\overline{1,l}}}}
\max\limits_{\Psi \in \Gamma} I(\Psi)=
\max\limits_{\substack{{\Psi_i \in \Gamma_i\,,}\\ 
{i=\overline{1,N}}}}
I\left(\Psi_1,\Psi_2,\ldots,\Psi_N\right)={}\\
{}=
\max\limits_{\substack{{\Psi_i^* \in \Gamma_i^*,}\\ 
{i=\overline{1,N}}}}
 I\left(\Psi_1^*,\Psi_2^*,\ldots,\Psi_N^*\right)={}\\
{}=\max\limits_{(u_1,u_2,\ldots,u_N)\in U}\fr{A(u_1,u_2,\ldots,u_N)}
{B(u_1,u_2,\ldots,u_N)}\,; \label{e19}
\end{multline}

\vspace*{-12pt}

\noindent
\begin{multline*}
\min\limits_{\Psi \in \Gamma} I(\Psi)=
\min\limits_{\substack{{\Psi_i \in \Gamma_i\,,}\\ 
{i=\overline{1,N}}}} I\left(\Psi_1,\Psi_2,\ldots,\Psi_N\right)={}\\
{}=
\min\limits_{\substack{{\Psi_i^* \in \Gamma_i^*,}\\ 
{i=\overline{1,N}}}}
I\left(\Psi_1^*,\Psi_2^*,\ldots,\Psi_N^*\right)={}\\
{}=\min\limits_{(u_1,u_2,\ldots,u_N)\in U}\fr{A(u_1,u_2,\ldots,u_N)}
{B(u_1,u_2,\ldots,u_N)}\,. %\label{e20}
\end{multline*}
\item \textit{Если функция $C(u_1,u_2,\ldots,u_N)\hm=
{A(u_1,u_2,\ldots,u_N)}/{B(u_1,u_2,\ldots,u_N)}$ ограничена сверху или снизу, 
но не достигает глобального экстремума на множестве $U\hm=U_1\times U_2\times\cdots
\times U_N$,
то для любого $\varepsilon\hm > 0$ можно выбрать $\varepsilon$-оп\-ти\-маль\-ную 
детерминированную стратегию управления полумарковским процессом~$\xi(t)$, 
которая определяется вырожденной
вероятностной мерой $\Psi^{*(+)}(\varepsilon)\hm\in \Gamma^*$ или вырожденной
вероятностной мерой $\Psi^{*(-)}(\varepsilon)\hm\in \Gamma^*$, в~зависимости от 
вида экстремума (максимума или минимума) в~задаче}~(\ref{e18}). 
\textit{При этом вероятностная мера $\Psi^{*(+)}(\varepsilon)\hm\in \Gamma^*$ может быть 
сосредоточена в~любой точке $\left(u_1^{(+)}(\varepsilon),u_2^{(+)}(\varepsilon),\ldots,
u_N^{(+)}(\varepsilon)\right)$, удовлетворяющей соотношению}:
\begin{multline}
\sup\limits_{(u_1,u_2,\ldots,u_N) \in U}
\fr{A(u_1,u_2,\ldots,u_N)}{B(u_1,u_2,\ldots,u_N)}-\varepsilon <{}\\
{}<
\fr{A\left(u_1^{(+)}(\varepsilon),u_2^{(+)}(\varepsilon),\ldots,u_N^{(+)}
(\varepsilon)\right)}
{B\left(u_1^{(+)}(\varepsilon),u_2^{(+)}(\varepsilon),\ldots,u_N^{(+)}
(\varepsilon)\right)}<{}\\
{}<\sup\limits_{(u_1,u_2,\ldots,u_N) \in U}
\fr{A(u_1,u_2,\ldots,u_N)}{B(u_1,u_2,\ldots,u_N)}<\infty\,, 
\label{e21}
\end{multline}
\textit{если функция $C(u_1,u_2,\ldots,u_N)$ ограничена сверху 
и~экстремальная задача}~(\ref{e18}) 
\textit{представляет собой задачу на максимум. Аналогично вероятностная мера 
$\Psi^{*(-)}(\varepsilon)\hm\in \Gamma^*$ может быть сосредоточена в~любой точке 
$\left(u_1^{(-)}(\varepsilon),u_2^{(-)}(\varepsilon),\ldots,u_N^{(-)}(\varepsilon)
\right)$, удовлетворяющей соотношению}:

\noindent
\begin{multline*}
-\infty<\inf\limits_{(u_1,u_2,\ldots,u_N) \in U}\fr{A(u_1,u_2,\ldots,u_N)}
{B(u_1,u_2,\ldots,u_N)} <{}\\
{}<
\fr{A\left(u_1^{(-)}(\varepsilon),u_2^{(-)}
(\varepsilon),\ldots,u_N^{(-)}(\varepsilon)\right)}
{B\left(u_1^{(-)}(\varepsilon),u_2^{(-)}(\varepsilon),\ldots,
u_N^{(-)}(\varepsilon)\right)}<{}\\
{}<\inf\limits_{(u_1,u_2,\ldots,u_N) \in U}
\fr{A(u_1,u_2,\ldots,u_N)}{B(u_1,u_2,\ldots,u_N)}+\varepsilon\,, 
%\label{e22}
\end{multline*}
\textit{если функция $C(u_1,u_2,\ldots,u_N)$ ограничена снизу и~экстремальная 
задача}~(\ref{e18})  \textit{представляет собой задачу на минимум}.
\item \textit{Если функция $C(u_1,u_2,\ldots,u_N)\hm=
{A(u_1,u_2,\ldots,u_N)}/{B(u_1,u_2,\ldots,u_N)}$ не ограничена сверху 
или снизу, то оптимальной стратегии управления в~смысле
соответствующей экстремальной задачи не существует. 
При этом найдется такая последовательность вырожденных вероятностных
мер~$\Psi^{*(+)}(n)$, сосредоточенных в~точках 
$\left(u_1^{(+)}(n),u_2^{(+)}(n),\ldots,u_N^{(+)}(n)\right)$, $n\hm=1,2,\dots $, 
для которых выполняется соотношение}:
\begin{multline*}
I\left(\Psi^*(n)\right)={}\\
{}=
I\left(\Psi_1^{*(+)}(n),\Psi_2^{*(+)}(n),\ldots,\Psi_N^{*(+)}(n)\right)={}\\
{}=\fr{A\left(u_1^{(+)}(n),u_2^{(+)}(n),\ldots,u_N^{(+)}(n)\right)}
{B\left(u_1^{(+)}(n),u_2^{(+)}(n),\ldots,u_N^{(+)}(n)\right)}\to 
\infty\\
\mbox{при}\ n\to\infty\,, 
%\label{e23}
\end{multline*}
\textit{если функция $C(u_1,u_2,\ldots,u_N)$ не ограничена сверху. 
Аналогично найдется такая последовательность вырожденных вероятностных
мер~$\Psi^{*(-)}(n)$, сосредоточенных в~точках 
$\left(u_1^{(-)}(n),u_2^{(-)}(n),\ldots,u_N^{(-)}(n)\right)$, 
$n\hm=1,2,\dots $, для которых выполняется соотношение}:
\begin{multline*}
I\left(\Psi^{*(-)}(n)\right)={}\\
{}= I
\left(\Psi_1^{*(-)}(n),\Psi_2^{*(-)}(n),\ldots,\Psi_N^{*(-)}(n)\right)={}\\
{}=\fr{A\left(u_1^{(-)}(n),u_2^{(-)}(n),\ldots,u_N^{(-)}(n)\right)}
{B\left(u_1^{(-)}(n),u_2^{(-)}(n),\ldots,u_N^{(-)}(n)\right)}\to 
-\infty\\
\mbox{при}~~n\to\infty\,,  
%\label{e24}
\end{multline*}
\textit{если функция $C(u_1,u_2,\ldots,u_N)$ не ограничена \mbox{снизу}}.
\end{enumerate}
\textit{При этом сформулированные утверждения каждого пункта теоремы~$2$ 
могут выполняться как по отдельности, для одного из двух
видов экстремума, так и~совместно, для обоих видов экстремума.}

\smallskip

Прежде чем непосредственно доказывать теорему~2, докажем некоторые 
вспомогательные утверждения.

\smallskip

\noindent
\textbf{Лемма~1.}\ 
\textit{Рассмотрим дроб\-но-ли\-ней\-ный интегральный функционал 
$I(\Psi_1, \Psi_2,\ldots, \Psi_N)$ вида}~(\ref{e11}), 
\textit{заданный на некотором множестве наборов вероятностных мер 
$\Psi\hm=(\Psi_1, \Psi_2,\ldots, \Psi_N)\hm \in \Gamma$. Предположим, что на 
множестве~$\Gamma$ выполняется условие~$1$ из набора предварительных условий 
и~функция $B(u_1, u_2,\ldots, u_N)$  обладает свойством строгой знакопостоянности 
при всех $(u_1, u_2,\ldots, u_N) \hm\in U$. Тогда справедливы следующие утверждения}:
\begin{enumerate}[1.]
\item \textit{Если основная функция 
$C(u_1, u_2,\ldots, u_N)\hm={A(u_1, u_2,\ldots, u_N)}/{B(u_1, u_2,\ldots, u_N)}$ 
ограничена сверху, т.\,е.\ выполняется условие}
\begin{multline}
C\left(u_1, u_2,\ldots, u_N\right)=
\fr{A(u_1, u_2,\ldots, u_N)}{B(u_1, u_2,\ldots, u_N)}\leq {}\\
{}\leq
c_0^{(+)}<\infty \,, \enskip \left(u_1, u_2,\ldots, u_N\right) \in U\,, \label{e25}
\end{multline}
\textit{то имеет место неравенство}:
\begin{equation}
I\left(\Psi_1, \Psi_2,\ldots, \Psi_N\right)\leq c_0^{(+)} 
\label{e26}
\end{equation}
\textit{для всех} $(\Psi_1, \Psi_2,\ldots, \Psi_N) \in \Gamma$.
\item \textit{Если основная функция 
$C(u_1, u_2,\ldots, u_N)\hm={A(u_1, u_2,\ldots, u_N)}/{B(u_1, u_2,\ldots, u_N)}$ 
ограничена снизу, т.\,е.\ выполняется условие}
\begin{multline*}
C\left(u_1, u_2,\ldots, u_N\right)=\fr{A(u_1, u_2,\ldots, u_N)}{B(u_1, u_2,\ldots, 
u_N)}\geq{}\\
{}\geq c_0^{(-)}>-\infty \,, 
\left(u_1, u_2,\ldots, u_N\right) \in U\,, 
%\label{e27}
\end{multline*}
\textit{то имеет место неравенство}:
\begin{equation*}
I\left(\Psi_1, \Psi_2,\ldots, \Psi_N\right)\geq c_0^{(-)} 
%\label{e28}
\end{equation*}
\textit{для всех} $(\Psi_1, \Psi_2,\ldots, \Psi_N) \hm\in \Gamma$.
\end{enumerate}

\noindent
Д\,о\,к\,а\,з\,а\,т\,е\,л\,ь\,с\,т\,в\,о\ \ леммы~1.\ 
Докажем первое утверждение леммы. Предположим сначала, 
что функция $B(u_1, u_2,\ldots,  u_N)$ строго положительна:
\begin{equation}
B\left(u_1, u_2,\ldots, u_N\right)>0\,,\enskip
\left(u_1, u_2,\ldots, u_N\right)\in U\,. \label{e29}
\end{equation}
Заметим, что в~таком случае по свойству интеграла~\cite[гл.~V]{18}
\begin{multline}
\hspace*{-2mm}\int\limits_{U_1}\!\!\cdots\! \!\int\limits_{U_N}\!\!B(u_1, \ldots,u_N) \,
d\Psi_1(u_1)%d\Psi_2(u_2)\cdots\\
\cdots d\Psi_N(u_N)>0 \!\!\!\!\label{e30}
\end{multline}
для любого фиксированного набора $\Psi\hm=(\Psi_1, \ldots, \Psi_N)\hm\in \Gamma$.
Из неравенства~(\ref{e25}) с~уче\-том~(\ref{e29}) получаем:
\begin{multline}
\hspace*{-4mm}A\left(u_1,\ldots, u_N\right)\leq{}\\
\hspace*{-4mm}{}\leq c_0^{(+)} B\left(u_1, \ldots, u_N\right)\,, 
\left(u_1, \ldots, u_N\right)\in U\,. \label{e31}
\end{multline}
В свою очередь, из неравенства~(\ref{e31}) и~свойств интеграла следует:
\begin{multline}
\int\limits_{U_1}\!\!\cdots\! \!\int\limits_{U_N}\!\!A(u_1,\ldots, u_N) \,
d\Psi_1\left(u_1\right)%d\Psi_2\left(u_2\right)\cdots\\
\cdots d\Psi_N\left(u_N\right)\leq\\
\hspace*{-24pt}\leq 
c_0^{(+)}\!\!\int\limits_{U_1}\!\!\cdots\!\! \int\limits_{U_N}\!\!\!B\!\left(u_1,\ldots, u_N\right)
 d\Psi_1\!\left(u_1\right)\!%d\Psi_2\left(u_2\right)\cdots\\
 \cdots d\Psi_N\!\left(u_N\right)\!\! 
 \label{e32}
\end{multline}
для любого фиксированного набора $\Psi\hm=(\Psi_1, \ldots, \Psi_N)\hm\in \Gamma$. 
Но тогда из~(\ref{e32}) с~учетом~(\ref{e30}) получаем:
\begin{multline}
I(\Psi_1, \ldots, \Psi_N)={}\\
{}=
\fr{\int\nolimits_{U_1}\!\cdots\! \int\nolimits_{U_N}\!\!A\left(u_1, \ldots, u_N\right)\,
 d\Psi_1(u_1)\cdots d\Psi_N(u_N)}{
\int\nolimits_{U_1}\!\cdots\! \int\nolimits_{U_N}\!\!B\left(u_1, \ldots, u_N\right)\,
 d\Psi_1(u_1)
 \cdots d\Psi_N(u_N)}\leq{}\\
 {}\leq c_0^{(+)} 
 \label{e33}
\end{multline}
для любого фиксированного набора $(\Psi_1, \ldots\linebreak\ldots, \Psi_N)\hm\in \Gamma$.

Предположим теперь, что функция $B(u_1,\ldots, u_N)$ строго отрицательна:
\begin{equation}
B(u_1,\ldots, u_N)<0 \quad \left(u_1, \ldots, u_N\right)\in U\,. 
\label{e34}
\end{equation}
Тогда
\begin{multline}
\hspace*{-6pt}\int\limits_{U_1}\!\!\cdots\!\! \int\limits_{U_N}\!\!B\!\left(u_1,\ldots, u_N\right)\!
 d\Psi_1(u_1) \cdots d\Psi_N(u_N)<0 \!\!\!
 \label{e35}
\end{multline}
для любого фиксированного набора $(\Psi_1, \ldots\linebreak \ldots, \Psi_N)\hm\in \Gamma$.

Как и~ранее, будем исходить из неравенства~(\ref{e25}). 
При выполнении условий~(\ref{e34}) и~(\ref{e35}) характер неравенств~(\ref{e31}) 
и~(\ref{e32}) меняется на противоположный, но характер неравенства~(\ref{e33}) 
остается неизменным. Таким образом, для любой функции 
$B(u_1, u_2,\ldots, u_N)$, обладающей свойством строгой знакопостоянности, 
из условия~(\ref{e25}) следует выполнение неравенства~(\ref{e33}), 
которое совпадает с~(\ref{e26}). Первое утверждение леммы~1 доказано. 
Второе утверждение доказывается аналогично. Лемма~1 доказана.

\smallskip

\noindent
\textbf{Лемма 2.} \textit{Рассмотрим дроб\-но-ли\-ней\-ный интегральный функционал 
$I(\Psi_1, \Psi_2,\ldots, \Psi_N)$ вида}~(\ref{e11}), 
\textit{заданный на некотором множестве наборов вероятностных мер 
$\Psi\hm=(\Psi_1, \Psi_2,\ldots, \Psi_N)\hm\in \Gamma$. Предпо\-ложим, что на 
множестве~$\Gamma$ выполняется условие~$1$ из набора предварительных условий 
и~функция $B(u_1, u_2,\ldots, u_N)$ обладает свойством строгой знакопостоянности 
при всех $(u_1, u_2,\ldots, u_N)\hm\in U$. Тогда справедливы следующие утверждения}:
\begin{enumerate}[1.]
\item \textit{Если основная функция $C(u_1, u_2,\ldots, u_N)\hm=
{A(u_1, u_2,\ldots, u_N)}/{B(u_1, u_2,\ldots, u_N)}$ ограничена сверху, 
но не достигает своего максимального 
значения, то имеет место неравенство}:
\begin{multline}
I\left(\Psi_1, \Psi_2,\ldots, \Psi_N\right)<{}\\
{}< \sup\limits_{(u_1, u_2,\ldots, u_N)\in U}
 C\left(u_1, u_2,\ldots, u_N\right)<\infty \label{e36}
\end{multline}
\textit{для всех} $(\Psi_1, \Psi_2,\ldots, \Psi_N)\in \Gamma$.
\item \textit{Если основная функция $C(u_1, u_2,\ldots, u_N)\hm=
{A(u_1, u_2,\ldots, u_N)}/{B(u_1, u_2,\ldots, u_N)}$ ограничена снизу, 
но не достигает своего минимального значения, то имеет место неравенство}:
\begin{multline*}
I\left(\Psi_1, \Psi_2,\ldots, \Psi_N\right)>{}\\
{}> \inf\limits_{(u_1, u_2,\ldots, u_N)\in U} 
C\left(u_1, u_2,\ldots, u_N\right)>-\infty 
%\label{e37}
\end{multline*}
\textit{для всех} $(\Psi_1, \Psi_2,\ldots, \Psi_N)\hm\in \Gamma$.
\end{enumerate}

\noindent
Д\,о\,к\,а\,з\,а\,т\,е\,л\,ь\,с\,т\,в\,о\ \ леммы~2. 
Докажем первое утверждение леммы. Поскольку множество значений 
основной функции $C(u_1, u_2,\ldots, u_N)$ ограничено сверху, оно имеет конечную 
верхнюю грань:
$$
\exists \sup\limits_{(u_1, u_2,\ldots, u_N)\in U} 
C\left(u_1, u_2,\ldots, u_N\right)<\infty
$$
(см.~\cite[гл.~1, \S3, п.~3.4, теорема~1]{25}).

По условию функция $C(u_1, u_2,\ldots, u_N)$ не достигает своего максимального 
значения. Следовательно, выполняется неравенство:
\begin{multline}
C(u_1, u_2,\ldots, u_N)=\fr{A(u_1, u_2,\ldots, u_N)}{B(u_1, u_2,\ldots, u_N)}<{}\\
{}< 
\sup\limits_{(u_1, u_2,\ldots, u_N)\in U} C(u_1, u_2,\ldots, u_N)<\infty\,, 
\\
\left(u_1, u_2,\ldots, u_N\right)\in U\,.
\label{e38}
\end{multline}
Взяв за основу строгое неравенство~(\ref{e38}), проведем рассуждения, аналогичные тем, 
которые были проведены в~лемме~1 по отношению к~неравенству~(\ref{e25}). 
В~результате получим строгое неравенство~(\ref{e36}).

Второе утверждение леммы~2 доказывается аналогично. Лемма~2 доказана.

\noindent
Д\,о\,к\,а\,з\,а\,т\,е\,л\,ь\,с\,т\,в\,о\ 
\ теоремы~2.
Начнем с~доказательства утверждения~1. Предположим сначала, что основная 
функция $C(u_1, u_2,\ldots, u_N)={A(u_1, u_2,\ldots, u_N)}/{B(u_1, u_2,\ldots, u_N)}$ 
ограничена сверху и~достигает глобального максимума на множестве~$U$ 
в~некоторой точке $u^{(+)}\hm=\left(u^{(+)}_1,u^{(+)}_2,\ldots,u^{(+)}_N\right)\hm\in U$,
а~именно:
\begin{multline*}
\max\limits_{(u_1, u_2,\ldots, u_N)\in U} C\left(u_1, u_2,\ldots, u_N\right) = {}\\
{}=
C\left(u^{(+)}_1,u^{(+)}_2,\ldots,u^{(+)}_N\right)<\infty\,.
\end{multline*}
Тогда выполняется соотношение:
\begin{multline}
C(u_1, u_2,\ldots, u_N)=\fr{A(u_1, u_2,\ldots, u_N)}{B(u_1, u_2,\ldots, u_N)}
\leq{}\\
{}\leq C\left(u^{(+)}_1,u^{(+)}_2,\ldots,u^{(+)}_N\right)<\infty\,, 
\\
\left(u_1, u_2,\ldots, u_N\right)\in U\,.
\label{e39}
\end{multline}
Условия леммы~1 выполнены, и~можно воспользоваться ее утверждениями. 
Согласно первому из них, если выполняется неравенство~(\ref{e39}), 
то имеет место соотношение:
\begin{equation*}
I(\Psi_1, \Psi_2,\ldots, \Psi_N)\leq 
C\left(u^{(+)}_1,u^{(+)}_2,\ldots,u^{(+)}_N\right)<\infty 
%\label{e40}
\end{equation*}
для всех стратегий управления $\Psi\hm=(\Psi_1, \Psi_2,\ldots\linebreak
\ldots, \Psi_N)\hm\in \Gamma$.

Таким образом, множество значений дроб\-но-ли\-ней\-но\-го интегрального 
функционала $I(\Psi_1, \Psi_2,\ldots, \Psi_N)$ ограничено сверху при всех 
$\Psi\hm=(\Psi_1, \Psi_2,\ldots, \Psi_N)\hm\in \Gamma$. Тогда существует верхняя 
грань этого множества и~выполняется неравенство:
\begin{multline}
\sup\limits_{(\Psi_1, \Psi_2,\ldots, \Psi_N)\in \Gamma} 
I\left(\Psi_1, \Psi_2,\ldots, \Psi_N\right)\leq {}\\
{}\leq
C\left(u^{(+)}_1,u^{(+)}_2,\ldots,u^{(+)}_N\right). \label{e41}
\end{multline}
Рассмотрим детерминированную стратегию управ\-ле\-ния 
$\Psi^{*(+)}\hm=\left(\Psi_1^{*(+)}, \Psi_2^{*(+)},\ldots, \Psi_N^{*(+)}\right)$, 
в~которой каждая вероятностная мера~$\Psi_i^{*(+)}$ является вы\-рож\-ден\-ной 
и~сосредоточена в~точке $u_i^{(+)}$, $i\hm=\overline{1, N}$.
По свойству интеграла
\begin{multline}
I\left(\Psi_1^{*(+)}, \Psi_2^{*(+)},\ldots ,\Psi_N^{*(+)}\right)={}\\
{}=
C\left(u^{(+)}_1,u^{(+)}_2,\ldots,u^{(+)}_N\right). \label{e42}
\end{multline}
Из соотношений~(\ref{e41}) и~(\ref{e42}) получаем:
\begin{multline}
\sup\limits_{(\Psi_1, \Psi_2,\ldots, \Psi_N)\in \Gamma} 
I\left(\Psi_1, \Psi_2,\ldots, \Psi_N\right)\leq{}\\
{}\leq
 I\left(\Psi_1^{*(+)}, 
\Psi_2^{*(+)},\ldots, \Psi_N^{*(+)}\right). \label{e43}
\end{multline}
Заметим дополнительно, что выполняются отношения принадлежности:
\begin{equation}
\Psi^{*(+)}=\left(\Psi_1^{*(+)}, \Psi_2^{*(+)},\ldots, \Psi_N^{*(+)}\right) 
\in \Gamma^* \subset \Gamma\,. \label{e44}
\end{equation}
Из~(\ref{e44}) и~свойства верхней грани следует:
\begin{multline}
\sup\limits_{\left(\Psi_1^{*}, \Psi_2^{*},\ldots, \Psi_N^{*}\right) \in \Gamma^*} 
I\left(\Psi_1^{*}, \Psi_2^{*},\ldots, \Psi_N^{*}\right)\leq {}\\
{}\leq
\sup\limits_{\left(\Psi_1, \Psi_2,\ldots, \Psi_N\right) 
\in \Gamma} I\left(\Psi_1, \Psi_2,\ldots, \Psi_N\right)\,. 
\label{e45}
\end{multline}
Объединяя~(\ref{e42}), (\ref{e43}) и~(\ref{e45}), получаем соотношение:
\begin{multline}
\sup\limits_{\left(\Psi_1^{*}, \Psi_2^{*},\ldots, \Psi_N^{*}\right) 
\in \Gamma^*} I\left(\Psi_1^{*}, \Psi_2^{*},\ldots, 
\Psi_N^{*}\right)\leq{}\\
{}\leq \sup\limits_{\left(\Psi_1, \Psi_2,\ldots, \Psi_N\right) 
\in \Gamma} I\left(\Psi_1, \Psi_2,\ldots, \Psi_N\right)\leq{}\\
{}\leq I\left(\Psi_1^{*(+)}, \Psi_2^{*(+)},\ldots, \Psi_N^{*(+)}\right)={}\\
{}=
\fr{A\left(u^{(+)}_1,u^{(+)}_2,\ldots,u^{(+)}_N\right)}{B\left(u^{(+)}_1,u^{(+)}_2,
\ldots,u^{(+)}_N\right)}\,.
 \label{e46}
\end{multline}
Из соотношения~(\ref{e46}) с~учетом~(\ref{e44}) получаем, что максимум 
функционала $I(\Psi_1, \Psi_2,\ldots, \Psi_N)$ на множестве допустимых стратегий 
$\Psi\hm=(\Psi_1, \Psi_2,\ldots, \Psi_N)\hm\in \Gamma$ существует и~достигается 
на детерминированной стратегии $\left(\Psi_1^{*(+)}, \Psi_2^{*(+)},\ldots, 
\Psi_N^{*(+)}\right)$.

Кроме того, выполняются соотношения~(\ref{e19}). Таким образом, утверждение~1 
в~случае, когда основная функция $C(u_1, u_2,\ldots, u_N)$ достигает глобального 
максимума, доказано. Соответствующее утверждение в~случае, когда основная функция 
$C(u_1, u_2,\ldots, u_N)$ достигает глобального минимума, доказывается аналогично. 
При этом используется второе утверждение леммы~1.

\smallskip

Перейдем к~доказательству второго утверждения теоремы~2. Предположим, что основная 
функция $C(u_1, u_2,\ldots, u_N)\hm=A(u_1, u_2,\ldots$\linebreak
$\ldots, u_N)/{B(u_1, u_2,\ldots, u_N)}$ 
ограничена сверху, но не достигает глобального максимума на множестве 
$U \hm= U_1 \times U_2 \times \cdots \times U_N$. Тогда множество значений 
основной функции имеет конечную верхнюю грань:

\noindent
\begin{multline*}
C\left(u_1, u_2,\ldots, u_N\right)=\fr{A(u_1, u_2,\ldots, u_N)}
{B(u_1, u_2,\ldots, u_N)}<{}\\
{}<
\sup\limits_{(u_1, u_2,\ldots, u_N)\in U} \fr{A(u_1, u_2,\ldots, u_N)}
{B(u_1, u_2,\ldots, u_N)}<\infty\,, 
\\
\left(u_1, u_2,\ldots, u_N\right)\in U\,.
%\label{e47}
\end{multline*}
По определению верхней грани для любого фиксированного $\varepsilon \hm>0$ 
существует точка $(u_1^{(+)}(\varepsilon), u_2^{(+)}(\varepsilon),\ldots, 
u_N^{(+)}(\varepsilon))$ такая, что выполняется двойное неравенство~(\ref{e21}) 
(см.~\cite[гл.~1, \S\,3, п.~3.4]{25}). Иначе говоря, значение основной функции 
в~указанной точке лежит в~левой \mbox{$\varepsilon$-окрест}\-ности верхней грани. 
Рассмотрим детерминированную стратегию управления 
$\Psi^{*(+)}(\varepsilon)\hm=\!\left(\Psi_1^{*(+)}(\varepsilon), 
\Psi_2^{*(+)}(\varepsilon),\ldots, \Psi_N^{*(+)}(\varepsilon)\!\right)$, компонентами\linebreak 
которой являются вырожденные вероятностные меры $\Psi_1^{*(+)}(\varepsilon), 
\Psi_2^{*(+)}(\varepsilon),\ldots, \Psi_N^{*(+)}(\varepsilon)$, причем вырожденная 
мера~$\Psi_i^{*(+)}(\varepsilon)$ сосредоточена в~точке~$u_i^{(+)}(\varepsilon)$,
$i\hm=1,2,\ldots,N$.

По свойству интеграла
\begin{multline}
I\left(\Psi_1^{*(+)}(\varepsilon), \Psi_2^{*(+)}(\varepsilon),\ldots,
 \Psi_N^{*(+)}(\varepsilon)\right)={}\\
 {}=
 C\left(u_1^{(+)}(\varepsilon), u_2^{(+)}(\varepsilon),\ldots, 
 u_N^{(+)}(\varepsilon)\right)\,. 
 \label{e48}
\end{multline}
Из соотношения~(\ref{e48}) с~учетом указанного свойства основной функции получаем:
\begin{multline}
\sup\limits_{(u_1, u_2,\ldots, u_N)\in U} \fr{A(u_1, u_2,\ldots, u_N)}
{B(u_1, u_2,\ldots, u_N)}-\varepsilon<{}\\
{}< I\left(\Psi_1^{*(+)}(\varepsilon), 
\Psi_2^{*(+)}(\varepsilon),\ldots, \Psi_N^{*(+)}(\varepsilon)\right)<{}
\\
{}< \sup\limits_{(u_1, u_2,\ldots, u_N)\in U} \fr{A(u_1, u_2,\ldots, u_N)}
{B(u_1, u_2,\ldots, u_N)}<\infty\,. 
\label{e49}
\end{multline}
Заметим также, что в~рассматриваемом случае выполнены условия леммы~2. 
Воспользуемся первым утверждением этой леммы, а~именно соотношением~(\ref{e36}):
\begin{multline}
I(\Psi_1, \Psi_2,\ldots, \Psi_N)< {}\\
{}<\sup\limits_{(u_1, u_2,\ldots, u_N)
\in U} \fr{A(u_1, u_2,\ldots, u_N)}{B(u_1, u_2,\ldots, u_N)}<\infty 
\label{e50}
\end{multline}
для всех $(\Psi_1, \Psi_2,\ldots, \Psi_N)\in\Gamma$.

Из соотношений~(\ref{e49}) и~(\ref{e50}) следует, что детерминированная стратегия 
$\Psi^{*(+)}(\varepsilon)\hm=\left(\Psi_1^{*(+)}(\varepsilon), \Psi_2^{*(+)}(\varepsilon),
\ldots, \Psi_N^{*(+)}(\varepsilon)\right)$, опре\-де\-ля\-емая набором вырожденных 
вероятностных мер, сосредоточенных в~соответствующих точках 
$\left(u_1^{(+)}(\varepsilon), u_2^{(+)}(\varepsilon),\ldots, 
u_N^{(+)}(\varepsilon)\right)$, является $\varepsilon$-оп\-ти\-маль\-ной. 
Вторая часть утверждения~2 теоремы~2, связанная со свойствами нижней грани, 
доказывается аналогично.

Докажем третье утверждение теоремы~2. Предположим, что множество значений 
основной функции $C(u_1, u_2,\ldots, u_N)\hm=
A(u_1, u_2,\ldots$\linebreak $\ldots, u_N)/{B(u_1, u_2,\ldots, u_N)}$
не является ограниченным сверху на множестве $U\hm=U_1\times U_2 \times \cdots $\linebreak
$\cdots \times U_N$.
Тогда существует последовательность\linebreak точек $\left(u_1^{(+)}(n), u_2^{(+)}(n),
\ldots,u_N^{(+)}(n)\right)\hm\in U$, $n\hm=1,2,\ldots$, для которой
\begin{multline}
C\left(u_1^{(+)}(n), u_2^{(+)}(n),\ldots,u_N^{(+)}(n)\right)={}\\
{}=
\fr{A\left(u_1^{(+)}(n), u_2^{(+)}(n),\ldots,u_N^{(+)}(n)\right)}
{B\left(u_1^{(+)}(n), u_2^{(+)}(n),\ldots,u_N^{(+)}(n)\right)}
\longrightarrow \infty \,,\\
n\rightarrow \infty\,.
\label{e51}
\end{multline}
Зафиксируем некоторую последовательность точек $\left(u_1^{(+)}(n), u_2^{(+)}(n),
\ldots,u_N^{(+)}(n)\right)\hm\in U$, $n\hm=1,2,\ldots$, обладающих указанным свойством, 
и~рассмотрим последовательность детерминированных  стратегий управления 
$\Psi^{*(+)}(n)\hm=\left(\Psi_1^{*(+)}(n), \Psi_2^{*(+)}(n),\ldots, 
\Psi_N^{*(+)}(n)\right)$, $n\hm=1,2,\ldots$, определяемых набором вырожденных 
вероятностных мер, сосредоточенных в~соответствующих точках 
$\left(u_1^{(+)}(n), u_2^{(+)}(n),\ldots,u_N^{(+)}(n)\right)$, $n\hm=1,2,\ldots$ 
По свойству интеграла для любого фиксированного значения $n=1,2,\ldots$ 
выполняется равенство:
\begin{multline}
I \left(\Psi^{*(+)}(n)\right)={}\\
{}=I\left(\Psi_1^{*(+)}(n), \Psi_2^{*(+)}(n),\ldots,
 \Psi_N^{*(+)}(n)\right)={}\\
{}=\fr{A\left(u_1^{(+)}(n), u_2^{(+)}(n),\ldots,u_N^{(+)}(n)\right)}
{B\left(u_1^{(+)}(n), u_2^{(+)}(n),\ldots,u_N^{(+)}(n)\right)}\,. 
\label{e52}
\end{multline}
Из соотношений~(\ref{e51}) и~(\ref{e52}) следует, что
\begin{multline}
I\left(\Psi^{*(+)}(n)\right)={}\\
{}=I\left(\Psi_1^{*(+)}(n), \Psi_2^{*(+)}(n),\ldots, 
\Psi_N^{*(+)}(n)\right)\longrightarrow\infty\,,\\ 
n \rightarrow\infty\,.
 \label{e53}
\end{multline}
Соотношение~(\ref{e53}) означает, что множество значе\-ний дроб\-но-ли\-ней\-но\-го 
интегрального функциона\-ла $I(\Psi_1, \Psi_2,\ldots, \Psi_N)$ вида~(\ref{e11}) 
не ограничено сверху\linebreak на множестве наборов вырожденных вероятностных мер 
$\left(\Psi_1^{*(+)}(n), \Psi_2^{*(+)}(n),\ldots, \Psi_N^{*(+)}(n)\right)\hm\in\Gamma^*$, 
а~следовательно, и~на более широком\linebreak множестве наборов вероятностных 
мер $(\Psi_1, \Psi_2,\ldots$\linebreak $\ldots, \Psi_N)\hm\in\Gamma$. В~таком случае решения экстремальной 
задачи~(\ref{e18}) в~форме задачи на максимум не существует. Соответствующее утвержде\-ние 
для варианта, когда множество значений основной функции $C(u_1, u_2,\ldots,u_N)
\hm=A(u_1, u_2,\ldots$\linebreak $\ldots,u_N)/{B(u_1, u_2,\ldots,u_N)}$ 
не является ограниченным снизу, доказывается аналогично. Третье утверж\-де\-ние теоремы~2 
доказано. Тем самым тео\-ре\-ма~2 доказана полностью.

\smallskip

Применим теорему~2 для решения поставленной задачи оптимального управления. 
Из утверждения этой теоремы следует, что для доказательства су-\linebreak ществования 
оптимального управ\-ле\-ния и~его нахождения необходимо исследовать на 
глобальный экстремум основную функцию дроб\-но-ли\-ней\-но\-го интегрального 
функционала $C(u_1,u_2,\ldots,u_N)$, определяемую формулой~(\ref{e17}) с~учетом 
равенств~(\ref{e12})--(\ref{e16}). В~некоторых случаях, например когда основной 
процесс~$\xi(t)$ является регенерирующим, а~стоимостные характеристики 
модели задаются линейными функциями, такое исследование можно провести 
аналитически. Однако для подавляющего большинства полумарковских моделей 
для этого необходимо использовать численные методы.

\section{Заключение}

В заключительной части работы приведем \mbox{краткое} описание теоретической 
основы метода решения задачи оптимального управления полумарковским 
процессом с~конечным множеством состояний.

\begin{enumerate}[1.]
\item Исходная проблема оптимального управления формулируется в~виде 
экстремальной задачи~(\ref{e18}). Целевым показателем качества управ\-ле\-ния в~данной задаче 
служит величина~(\ref{e10}), которая имеет характер средней удельной прибыли.
\item Доказывается, что стационарный показатель~(\ref{e10}) представим в~виде 
дроб\-но-ли\-ней\-но\-го интегрального функционала~(\ref{e11}), для которого явно 
определяются подынтегральные функции числителя и~знаменателя, а~следовательно, 
и~основная функция данного функционала.
\item Используется теорема об экстремуме дроб\-но-ли\-ней\-но\-го интегрального 
функционала. На основании утверждений этой теоремы уста\-нав\-ли\-ва\-ет\-ся, что 
исходная задача оптимального управления сводится к~исследованию на глобальный 
экстремум основной функции этого функционала, для которой получено явное 
аналитическое представление.
\end{enumerate}

Заметим, что такое исследование задач оптимального управления 
стохастическими системами фактически уже было проведено в~ряде работ П.\,В.~Шнуркова 
и~его соавторов. В~частности, в~работе~\cite{26} была рассмотрена модель 
управления для обрывающегося процесса восстановления, описывающего функционирование 
некоторой технической системы. Задача управления решалась для различных показателей 
эффективности и~надежности этой системы, имеющих структуру дроб\-но-ли\-ней\-но\-го 
интегрального функционала.

В работах~\cite{27, 28} рассматривались модели регенерирующих процессов 
для исследования сис\-тем управления запасами. Различные показатели качества 
управления были представлены в~форме дроб\-но-ли\-ней\-ных интегральных функционалов. 
Основные функции этих функционалов были найде\-ны в~явной форме и~исследовались 
на глобальный экстремум. В~работах~\cite{21,29} рассматривалась достаточно 
сложная полумарковская модель с~конечным множеством состояний, описывающая 
сис\-те\-му управления запасом непрерывного продукта. Показатели качества управления в~этой 
модели также имели структуру дроб\-но-ли\-ней\-ных интегральных функционалов, 
для основных функций которых были найдены явные аналитические представления. 
Упомянем также работы~\cite{30, 31}, в~которых была исследована полумарковская 
модель с~дис\-крет\-но-не\-пре\-рыв\-ным фазовым пространством. Показатели 
качества управления в~этой  модели были найдены в~явной форме как функции от 
двух непрерывных параметров управления.

Фактически во всех упомянутых работах уже был использован метод решения задачи 
оптимального управления регенерирующим или полумарковским случайным процессом, 
основанный на исследовании экстремальных свойств основной функции соответствующего 
дроб\-но-ли\-ней\-но\-го интегрального функционала. Из соображений, изложенных 
во\linebreak введении, следует, что в~период написания и~пуб\-ли\-кации этих работ данный метод 
не имел стро\-гого обоснования. Однако после публикации\linebreak работы~\cite{14} и~настоящего 
исследования можно утверж\-дать, что полученные в~них результаты полностью теоретически 
обоснованы.

Таким образом, изложенный выше метод решения проблемы оптимального управления 
полумарковскими процессами с~конечными множествами состояний может быть успешно 
реализован для многих задач, рассматриваемых в~различных областях прикладной 
теории вероятностей.

Практическая реализация численной процедуры поиска оптимального решения на примере\linebreak 
полумарковской модели управления запасом непрерывного продукта (подробнее 
см.~\cite{21, 29}), ба\-зи\-ру\-юща\-яся на изложенных выше результатах (в~частности, 
теореме~1), была осуществлена А.\,К.~Горшениным и~соавторами 
в~статье~\cite{Gorshenin2015}. Коротко опишем наиболее важные аспекты этой работы.

Для решения поставленной задачи опти\-мального управления была создана 
специальная программа \verb"Inventory" на встроенном языке программирования 
пакета \verb"MATLAB", ее возможности\linebreak кратко представле\-ны в~упомянутой ранее 
\mbox{статье}~\cite{Gorshenin2015}. В~программе \verb"Inventory" реализованы функции 
для оценивания через заданные исходные параметры вероятностных и~стоимостных 
характеристик модели, которые в~дальнейшем используются для поиска значений 
основной функции дроб\-но-ли\-ней\-но\-го функционала~(\ref{e17}). Точка глобального 
экстремума этой функции и~определяет оптимальное управление.

В качестве начальных данных необходимо задание следующих параметров:
\begin{itemize}
\item спрос и~вместимость склада;
\item разбиение множества значений объема запаса;
\item вероятностные характеристики, описывающие модель пополнения запаса;
\item условные математические ожидания длительностей задержек пополнения запаса;
\item функции для характеризации затрат и~доходов.
\end{itemize}

По итогам работы программы \verb"Inventory" ряд вспомогательных функций 
представляется в~аналитической форме (в частности, с~использованием аппарата 
символьных вычислений  \verb"Symbolic Toolbox"\linebreak пакета \verb"MATLAB"), выводится 
точка глобального экстремума функции нескольких вещественных переменных~(\ref{e17}), 
найденная с~помощью применения численных и~при\-бли\-жен\-но-ана\-ли\-ти\-че\-ских\linebreak 
аппроксимаций. 
Также формируются графики оценок значений ве\-ро\-ят\-ност\-но-сто\-и\-мост\-ных 
характеристик 
и~основной функции дроб\-но-ли\-ней\-но\-го функционала~(\ref{e17}), либо трехмерных 
сечений в~случае наличия более трех параметров управления (переменных).

Функциональность пакета \verb"Inventory" может быть расширена для практической 
реализации метода решения задачи поиска оптимального управ\-ле\-ния полумарковскими 
процессами с~конечными множествами состояний, рассмотренного в~данной статье.


 {\small\frenchspacing
 {%\baselineskip=10.8pt
 \addcontentsline{toc}{section}{References}
 \begin{thebibliography}{99}
 \bibitem{1}
\Au{Ховард Р.} Динамическое программирование и~марковские процессы~/ 
Пер. с~англ.~--- М.: Сов. радио, 1964. 189~с.
(\Au{Howard~R.\,A.} Dynamic programming and Markov processes.~--- 
Cambridge, MA, USA: MIT Press, 1960. 136~p.)
\bibitem{2} 
\Au{Рыков В.\,В.} Управляемые марковские процессы с~конечными пространствами 
состояний и~управлений~// Теория вероятностей и~ее применения, 1966. Т.~11. 
Вып.~2. С.~343--351.
\bibitem{3} 
\Au{Джевелл В.} Управляемые полумарковские процессы~// Кибернетич. сборник.~--- 
М.: Мир, 1967. Вып.~4. С.~97--134.
%{\em Jewell W.\,S.} Markov-renewal programming~// Operations Research, 1963. Vol.~11. P.~938--971.
\bibitem{4} 
\Au{Fox B.} Markov renewal programming by linear fractional programming~// 
SIAM J.~Appl. Math., 1966. Vol.~14. P.~1418--1432.
\bibitem{5} 
\Au{Denardo E.\,V.} Contraction mappings in the theory underlying dinamic programming~// 
SIAM Rev., 1967. Vol.~9. P.~165--177.

\bibitem{6} 
\Au{Howard R.\,A.} Research in semi-Markovian decision structures~// 
J.~Oper. Res. Soc. Japan, 1963. Vol.~6. P.~163--199.
\bibitem{7} 
\Au{Osaki S., Mine H.} Linear programming algorithms for Markovian decision processes~//
 J.~Math. Anal. Appl., 1968. Vol.~22. P.~356--381.
\bibitem{8} 
\Au{Майн Х., Осаки С.} Марковские процессы принятия решений~/ Пер. с~англ.~--- 
М.: Наука, 1977. 176~с.
(\Au{Mine~H., Osaki~S.} 
Markovian decision processes.~--- New York, NY, USA: 
American Elsevier Publishing Co., 1970. 142~p.)
\bibitem{9} 
\Au{Гихман И.\,И., Скороход А.\,В.} Управляемые случайные процессы.~--- 
Киев: Наукова думка, 1977. 251~с.
\bibitem{10} 
\Au{Luque-Vasquez F., Herndndez-Lerma~О.} Semi-Markov control models with average costs~// 
Appl. Math., 1999. Vol.~26. No.\,3. P.~315--331.
\bibitem{11} 
\Au{Vega-Amaya O., Luque-Vasquez~F.} Sample-path average cost optimality for 
semi-Markov control processes on Borel spaces: Unbounded costs and mean holding times~// 
Appl. Math., 2000. Vol.~27. No.\,3. P.~343--367.
\bibitem{12} 
Вопросы математической теории надежности~/ Под ред. Б.\,В. Гнеденко.~--- 
М.: Радио и~связь, 1983. 376~с.
\bibitem{13} 
\Au{Барзилович Е.\,Ю., Каштанов~В.\,А.} Некоторые математические вопросы теории 
обслуживания сложных систем.~---  М.: Сов. радио, 1971. 272~с.
\bibitem{14} 
\Au{Шнурков П.\,В.} О~решении проблемы безусловного экстремума для 
дроб\-но-ли\-ней\-но\-го интегрального функционала на множестве вероятностных мер~// 
Докл. РАН. Сер. Математика, 2016. Т.~470. №\,4. C.~387--392.
\bibitem{15} 
\Au{Ширяев А.\,Н.}  Вероятность.~--- М.:~МЦНМО, 2011. Кн.~1. 552~с. Кн.~2. 968~с.
\bibitem{16} 
\Au{Боровков А.\,А.} Теория вероятностей.~--- М.: Либроком, 2009. 656~c.
\bibitem{17} 
\Au{Хеннекен П.\,Л., Тортра А.} Теория вероятностей 
и~некоторые ее приложения.~--- М.: Наука, 1974. 472~c.
\bibitem{18} 
\Au{Халмош П.} Теория меры~/ Пер. с~англ.~--- М.: ИЛ, 1953. 282~c.
(\Au{Halmos~P.} Measure theory.~--- Litton Educational Publishing, Inc. 1950. 304~p.)
\bibitem{19} 
\Au{Королюк В.\,С., Турбин~А.\,Ф.} Полумарковские процессы и~их приложения.~--- 
Киев:~Наукова думка, 1976. 184~с.
\bibitem{20} 
\Au{Janssen J., Manca R.} Applied semi-Markov processes.~--- New York,
NY, USA: Springer, 2006. 309~p.
\bibitem{21} 
\Au{Шнурков П.\,В., Иванов~А.\,В.} Анализ дискретной полумарковской модели
 управления запасом непрерывного продукта при периодическом прекращении потребления~// 
 Дискретная математика, 2014. Т.~26. Вып.~1. С.~143--154.
\bibitem{22} 
\Au{Иванов~А.\,В.} Анализ дискретной полумарковской модели
 управления запасом непрерывного продукта при периодическом прекращении 
 потребления.~--- М.: НИУ ВШЭ, 2014.  Дисс.\ \ldots\ канд. физ.-мат. наук. 120~с.
\bibitem{23}  %23
\Au{Bajalinov~E.\,B.} Linear-fractional programming. 
Theory, methods, applications and software.~--- 
Boston/\linebreak Dordrecht/London: Kluwer Academic Publs., 2003. 423~p.

\bibitem{27} %27
\Au{Шнурков П.\,В., Мельников~Р.\,В.} Оптимальное управление запасом 
непрерывного продукта в~модели регенерации~// Обозрение прикладной 
и~промышленной математики, 2006. Т.~13. Вып.~3. С.~434--452.
\bibitem{28} 
\Au{Шнурков П.\,В., Мельников~Р.\,В.} 
Исследование проб\-ле\-мы управления запасом непрерывного продукта при детерминированной 
задержке поставки~// Автоматика и~телемеханика, 2008. Т.~10. С.~93--113.


\bibitem{24}  %26
\Au{Шнурков П.\,В.} Методы исследования задач оптимального обслуживания 
в~математической теории надежности.~--- 
М.: МИЭМ, 1983.  Дисс.\ \ldots\ канд. физ.-мат. наук.

 \bibitem{25}  %25
\Au{Кудрявцев Л.\,Д.} Курс математического анализа. Т.~1.~--- 
М.: Дрофа, 2006. 704~с.

\bibitem{26} %24
\Au{Шнурков П.\,В.} Оптимальное обслуживание на периоде 
до первого отказа системы~// Применение аналитических методов в~вероятностных
 задачах.~--- Киев: Институт математики АН УССР, 1986. С.~121--129.

\bibitem{29} 
\Au{Шнурков П.\,В., Иванов~А.\,В.} Исследование задачи оптимизации в~дискретной 
полумарковской модели управления непрерывным запасом~// Вестник МГТУ им.\ 
Н.\,Э. Баумана. Сер.\ Естественные науки, 2013. Т.~3. Вып.~50. С.~62--87.
\bibitem{30} 
\Au{Shnourkoff P.\,V.} The two-element system with one 
restoring device optimum maintenance~// Stoch. Anal. Appl., 1997. 
Vol.~15. No.\,5. P.~823--837.
\bibitem{31} 
\Au{Shnourkoff P.\,V.} The two-element system optimum maintenance tills the first fail~// 
Stoch. Anal. Appl., 2001. Vol.~19. No.\,6. P.~1005--1024.
\bibitem{Gorshenin2015} 
\Au{Gorshenin~A.\,K., Belousov~V.\,V., Shnourkoff~P.\,V.,
Ivanov~A.\,V.} Numerical research of the optimal control problem in the semi-Markov 
inventory model~// AIP Conference Proceedings, 2015. Vol.~1648. {250007}. 4~p.
%\bibitem{33} {\em Горшенин А.\,К., Белоусов В.\,В., Шнурков П.\,В.} 2016. Система управления запасами на основе стохастических полумарковских моделей. Свидетельство о государственной регистрации программы для ЭВМ \textnumero 2016619021.
 \end{thebibliography}

 }
 }

\end{multicols}

\vspace*{-6pt}

\hfill{\small\textit{Поступила в~редакцию 15.07.16}}

%\vspace*{8pt}

\newpage

\vspace*{-24pt}

%\hrule

%\vspace*{2pt}

%\hrule

%\vspace*{8pt}


\def\tit{ANALYTICAL SOLUTION OF~THE~OPTIMAL CONTROL TASK OF~A~SEMI-MARKOV 
PROCESS WITH~FINITE SET OF~STATES}

\def\titkol{Analytical solution of~the~optimal control task of~a~semi-Markov 
process with~finite set of~states}

\def\aut{P.\,V.~Shnurkov$^{1}$, A.\,K.~Gorshenin$^{2}$, and~V.\,V.~Belousov$^{2}$}

\def\autkol{P.\,V.~Shnurkov, A.\,K.~Gorshenin, and~V.\,V.~Belousov}

\titel{\tit}{\aut}{\autkol}{\titkol}

\vspace*{-9pt}


    
\noindent
$^1$National Research University Higher School of Economics, 34~Tallinskaya Str., 
Moscow, 123458, Russian\linebreak
$\hphantom{^9}$Federation

\noindent
$^2$Institute of Informatics Problems, Federal Research Center 
``Computer Science and Control'' of the Russian\linebreak
$\hphantom{^9}$Academy of Sciences, 44-2~Vavilova Str., 
Moscow 119333, Russian Federation



\def\leftfootline{\small{\textbf{\thepage}
\hfill INFORMATIKA I EE PRIMENENIYA~--- INFORMATICS AND
APPLICATIONS\ \ \ 2016\ \ \ volume~10\ \ \ issue\ 4}
}%
 \def\rightfootline{\small{INFORMATIKA I EE PRIMENENIYA~---
INFORMATICS AND APPLICATIONS\ \ \ 2016\ \ \ volume~10\ \ \ issue\ 4
\hfill \textbf{\thepage}}}

\vspace*{3pt}


\Abste{The theoretical verification of the new method of finding 
the optimal strategy of control of a~semi-Markov process with finite set of states is 
presented. The paper considers Markov randomized strategies of control, determined by 
a~finite collection of probability measures, corresponding to each state. The quality 
characteristic is the stationary cost index. This index is a~linear-fractional integral 
functional, depending on collection of probability measures, giving the strategy of control. 
Explicit analytical forms of integrands of numerator and denominator of this 
linear-fractional integral functional are known. The basis of consequent results is 
the new generalized and strengthened form of the theorem about an extremum of 
a~linear-fractional integral functional. It is proved that problems of existence 
of an optimal control strategy of a~semi-Markov process and finding this strategy 
can be reduced to the task of numerical analysis of global extremum for 
the given function, depending on finite number of real arguments.}

\KWE{optimal control of a~semi-Markov process; stationary cost index of quality control; 
linear-fractional integral functional}




\DOI{10.14357/19922264160408} 

\vspace*{-16pt}

\Ack
\noindent
The research was partially supported by the Russian Foundation 
for Basic Research (project 15-07-05316).



%\vspace*{3pt}

  \begin{multicols}{2}

\renewcommand{\bibname}{\protect\rmfamily References}
%\renewcommand{\bibname}{\large\protect\rm References}

{\small\frenchspacing
 {%\baselineskip=10.8pt
 \addcontentsline{toc}{section}{References}
 \begin{thebibliography}{99}
\bibitem{1-1}
\Aue{Howard,~R.\,A.} 1960. \textit{Dynamic programming and Markov processes}. 
Cambridge, MA: MIT Press. 136~p.
\bibitem{2-1}
\Aue{Rykov,~V.\,V.} 1966. Upravlyaemye markovskie protsessy 
s~konechnymi prostranstvami sostoyaniy i~upravleniy 
[Controlled Markov processes with finite spaces of states and controls ]. 
\textit{Teoriya veroyatnostey i~ee primeneniya} 
[Theory of Probability and Its Applications] 11(2):343--351.
\bibitem{3-1}
\Aue{Jewell,~W.\,S.} 1963. Markov-renewal programming. 
\textit{Oper. Res.} 11:938--971.
\bibitem{4-1}
\Aue{Fox,~B.} 1966. Markov renewal programming by linear fractional programming. 
\textit{SIAM J.~Appl. Math.} 14:1418--1432.
\bibitem{5-1}
\Aue{Denardo, E.\,V.} 1967. Contraction mappings in the theory underlying dinamic 
programming. \textit{SIAM Rev.} 9:165--177.
\bibitem{6-1}
\Aue{Howard,~R.\,A.} 1963. Research in semi-Markovian decision structures. 
\textit{J.~Oper. Res. Soc. Japan} 6:163--199.
\bibitem{7-1}
\Aue{Osaki,~S., and H.~Mine.} 1968. Linear programming algorithms 
for Markovian decision processes. \textit{J.~Math. Anal. Appl.} 22:356--381.
\bibitem{8-1}
\Aue{Mine,~H., and S.~Osaki.} 1970. 
\textit{Markovian decision processes}. New York, NY: Elsevier. 142~p.
\bibitem{9-1}
\Aue{Gikhman,~I.\,I., and A.\,V.~Skorokhod.} 1977. 
\textit{Upravlyaemye sluchaynye protsessy} 
[Controlled random processes]. Kiev: Naukova Dumka. 251~p.
\bibitem{10-1}
\Aue{Luque-Vasquez,~F., and О.~Herndndez-Lerma.} 1999. 
Semi-Markov control models with average costs. \textit{Appl. Math.} 26(3):315--331.
\bibitem{11-1}
\Aue{Vega-Amaya,~O., and  F.~Luque-Vasquez.} 2000.  
Sample-path average cost optimality for semi-Markov control processes on Borel spaces: 
Unbounded costs and mean holding times. \textit{Appl. Math.} 27(3):343--367.
\bibitem{12-1}
Gnedenko,~B.~V., ed. 1983. 
\textit{Voprosy matematicheskoy teorii nadezhnosti} 
[Problems of the mathematical theory of reliability].  Moscow: Radio i~svyaz'. 376~p.
\bibitem{13-1}
\Aue{Barzilovich,~E.\,Yu., and V.\,A.~Kashtanov.} 1971. 
\textit{Nekotorye matematicheskie voprosy teorii obsluzhivaniya slozhnykh sistem}  
[Some mathematical questions in theory of complex systems maintenance]. 
Moscow: Sovetskoe radio. 272~p.
\bibitem{14-1}
\Aue{Shnurkov,~P.\,V.} 2016. Solution of the unconditional extremum problem for 
a~linear-fractional 
integral functional on a~set of probability measures. 
\textit{Dokl. Math.} 94(2):550--554.
\bibitem{15-1} %14
\Aue{Shiryaev,~A.\,N.} 2016. 
\textit{Probability-1}. Graduate texts in mathematics ser.
New York, NY: Springer. Vol.~95. 503~p.;
2017. \textit{Probability-2.} Vol.~900. 500~p.
\bibitem{16-1}
\Aue{Borovkov,~А.\,А.} 2009. 
\textit{Teoriya veroyatnostey} [Probability theory]. Moscow: Librokom. 656~p.
\bibitem{17-1}
\Aue{Khenneken,~P.\,L., and A.~Tortra.} 1974. 
\textit{Teoriya veroyatnostey i~nekotorye ee prilozheniya} 
[Probability theory and some of its applications]. Moscow: Nauka. 472~p.
\bibitem{18-1}
\Aue{Halmos,~P.} 1950. \textit{Measure theory}. Litton Educational Publishing. 304~p.
\bibitem{19-1}
\Aue{Korolyuk, V.\,S., and A.\,F.~Turbin.} 1976. 
\textit{Polumarkovskie protsessy i~ikh prilozheniya} 
[Semi-Markov processes and their applications]. Kiev: Naukova Dumka. 184~p.
\bibitem{20-1}
\Aue{Janssen,~J., and R.~Manca.} 2006. 
\textit{Applied semi-Markov processes}. New York, NY: Springer. 309~p.
\bibitem{21-1}
\Aue{Shnurkov,~P.\,V, and A.\,V~Ivanov.} 2015. Analysis of a~discrete semi-Markov model of continuous inventory 
control with periodic interruptions of consumption. 
\textit{Discrete Math. \mbox{Appl}.} 25(1):59--67.
\bibitem{22-1} %21
\Aue{Ivanov,~A.\,V.} 2014. Analiz diskretnoy polumarkovskoy modeli upravleniya 
zapasom nepreryvnogo produkta pri periodicheskom prekrashchenii potrebleniya 
[Analysis of a~discrete semi-Markov control model of continuous product inventory 
in a~periodic cessation of consumption].  
Moscow: Natsional'nyy Issledovatel'skiy Universitet ``Vysshaya Shkola Ekonomiki.'' 
PhD Thesis. 120~p.
\bibitem{23-1} %22
\Aue{Bajalinov,~E.\,B.} 2003. 
\textit{Linear-fractional programming. Theory, methods, applications and software}. 
Boston/\linebreak Dordrecht/London: Kluwer Academic Publs. 423~p.
\bibitem{26-1} %24
\Aue{Shnurkov,~P.\,V., and R.\,V.~Mel'nikov.} 2006. Optimal'noe upravlenie 
zapasom nepreryvnogo produkta v modeli regeneratsii [Optimal control of 
a~continuous product inventory in the regeneration model]. 
\textit{Obozrenie prikladnoy i~promyshlennoy matematiki} [Rev. Appl. Ind. Math.]
13(3):434--452.

\bibitem{25-1} %25
\Aue{Shnurkov,~P.\,V., and R.\,V.~Mel'nikov.} 2008. 
Analysis of the problem of continuous-product inventory control under deterministic 
lead time. \textit{Automat. Rem. Contr.} 69(10):1734--1751.

\columnbreak

\bibitem{24-1} %26
\Aue{Shnurkov,~P.\,V.} 1983. Metody issledovaniya zadach optimal'nogo obsluzhivaniya 
v~matematicheskoy teorii nadezhnosti [Research methods of optimal service problems 
in the mathematical theory of reliability].  
Moscow: Moskovskiy Institut Elektronnogo Mashinostroeniya.  PhD Thesis. 


\bibitem{27-1} %27
\Aue{Kudryavtsev,~L.\,D.} 2006. 
\textit{Kurs matematicheskogo analiza} 
[A~course of mathematical analysis]. Vol.~1. Moscow: Drofa. 704~p.

\bibitem{28-1}
\Aue{Shnurkov,~P.\,V.} 1986. Optimal'noe obsluzhivanie na periode do 
pervogo otkaza sistemy [The optimum service period until the first system failure]. 
\textit{Primenenie analiticheskikh metodov v~veroyatnostnykh zadachakh} 
[The application of analytical methods in probabilistic tasks]. Kiev:
Institute of Mathematics of the Academy of Sciences of the USSR. 121--129.

\bibitem{29-1}
\Aue{Shnurkov,~P.\,V., and A.\,V.~Ivanov.} 2013. Issledovanie zadachi optimizatsii 
v~diskretnoy polumarkovskoy modeli upravleniya nepreryvnym zapasom 
[Study of the optimization problem in discrete semi-Markov model of continuous 
inventory control]. \textit{Vestnik MGTU im.\ N.\,E.~Baumana. Ser. 
Estestvennye nauki} [Vestnik of MSTU named after N.\,E.~Bauman. Ser. Natural sciences] 
3(50):62--87.
\bibitem{30-1}
\Aue{Shnourkoff,~P.\,V.} 1997. The two-element system with one restoring device 
optimum maintenance.  \textit{Stoch. Anal. Appl.} 15(5):823--837.
\bibitem{31-1}
\Aue{Shnourkoff,~P.\,V.} 2001. The two-element system optimum maintenance tills 
the first fail. \textit{Stoch. Anal. Appl.} 19(6):1005--1024.
\bibitem{32-1}
\Aue{Gorshenin,~A.\,K., V.\,V.~Belousov, P.\,V.~Shnourkoff, and A.\,V.~Ivanov.}
2015. Numerical research of the optimal control problem in the semi-Markov 
inventory model. \textit{AIP Conference Proceedings} 1648:250007.
\end{thebibliography}

 }
 }

\end{multicols}

\vspace*{-3pt}

\hfill{\small\textit{Received July 15, 2016}}

\Contr

\noindent
\textbf{Shnurkov Peter V.} (b.\ 1953)~---
 Candidate of Science (PhD) in physics and mathematics, 
 associate professor, National Research University Higher School of Economics, 
 34~Tallinskaya Str., Moscow 123458, Russian Federation; \mbox{pshnurkov@hse.ru} 
 
 \vspace*{3pt}
 
 \noindent
\textbf{Gorshenin Andrey K.}  (b.\ 1986)~---
Candidate of Science (PhD) in physics and mathematics, leading scientist, 
Institute of Informatics Problems, Federal Research Center ``Computer Science 
and Control'' of the Russian Academy of Sciences, 44-2~Vavilov Str., Moscow 119333, 
Russian Federation; associate professor, Federal State Budget Educational 
Institution of Higher Education ``Moscow Technological University,'' 
78~Vernadskogo Ave., Moscow 119454, Russian Federation;
\mbox{agorshenin@frccsc.ru}

\vspace*{3pt}

\noindent
\textbf{Belousov Vasiliy V.} (b.\ 1977)~---
Candidate of Science (PhD) in technology, senior scientist, Institute of 
Informatics Problems, Federal Research Center ``Computer Science and Control'' 
of the Russian Academy of Sciences, 44-2~Vavilov Str., Moscow 119333, Russian 
Federation; \mbox{VBelousov@ipiran.ru}
\label{end\stat}


\renewcommand{\bibname}{\protect\rm Литература}       %12
\def\stat{zhukov}

\def\tit{ВЛИЯНИЕ ПЛОТНОСТИ СВЯЗЕЙ НА~КЛАСТЕРИЗАЦИЮ И~ПОРОГ ПЕРКОЛЯЦИИ 
ПРИ~РАСПРОСТРАНЕНИИ ИНФОРМАЦИИ В~СОЦИАЛЬНЫХ СЕТЯХ$^*$}

\def\titkol{Влияние плотности связей на кластеризацию и~порог перколяции 
при~распространении информации в %~социальных 
сетях}

\def\aut{Д.\,О.~Жуков$^1$, Т.\,Ю.~Хватова$^2$, С.\,А.~Лесько$^3$, 
А.\,Д.~Зальцман$^4$}

\def\autkol{Д.\,О.~Жуков, Т.\,Ю.~Хватова, С.\,А.~Лесько, 
А.\,Д.~Зальцман}

\titel{\tit}{\aut}{\autkol}{\titkol}

\index{Жуков Д.\,О.}
\index{Хватова Т.\,Ю.}
\index{Лесько С.\,А.} 
\index{Зальцман А.\,Д.}
\index{Zhukov D.\,O.}
\index{Khvatova T.\,Yu.}
\index{Lesko S.\,A.} 
\index{Zaltsman A.\,D.}




{\renewcommand{\thefootnote}{\fnsymbol{footnote}} \footnotetext[1]
{Работа выполнена при финансовой поддержке РФФИ (проект 16-29-09458~офи\_м).}}


\renewcommand{\thefootnote}{\arabic{footnote}}
\footnotetext[1]{Московский технологический университет (МИРЭА), zhukov\_do@mirea.ru}
\footnotetext[2]{Санкт-Петербургский политехнический университет Петра Великого, 
\mbox{khvatova.ty@spbstu.ru}}
\footnotetext[3]{Московский технологический университет (МИРЭА), sergey@testor.ru}
\footnotetext[4]{Московский технологический университет (МИРЭА), 
\mbox{ad.zaltcman@gmail.com}}





\Abst{Рассматриваются вопросы применения новых тео\-ре\-ти\-че\-ских подходов 
к~описанию процессов передачи и~обработки информации в~социотехнических сис\-те\-мах 
и~сетях социальных связей на основе теории перколяции. Величина порога перколяции 
случайной сети зависит от ее плот\-ности. В~сетях, име\-ющих случайную структуру, пороги 
перколяции как в~задаче узлов, так и~в задаче связей при большой плот\-ности сети достигают 
величины насыщения, причем величина насыщения порога перколяции в~задаче связей 
больше, чем в~задаче узлов. С~точ\-ки зрения информационного влияния сети, имеющей 
случайную структуру, увеличение плот\-ности связей оказывается более предпочтительным, 
чем наличие небольшого чис\-ла отдельных <<центральных>> узлов, имеющих множество 
связей. 
В~практическом плане полученные результаты могут быть применены 
в~междисциплинарных исследованиях, вклю\-чая информатику, математическое 
моделирование и~экономику, с~при\-вле\-че\-ни\-ем социологических данных для прогнозирования 
поведения и~управ\-ле\-ния группами людей в~сетевых сообществах.
Полученные результаты дополняют и~расширяют применение методов и~подходов, принятых 
в~классической информатике, на описание социальных и~социотехнических сис\-тем, что 
может быть полезно для широкого круга исследователей, занимающихся изуче\-ни\-ем 
социальных сетевых структур.}

\KW{теория перколяции; структура социальной сети; плот\-ность связей; кластеризация сети; 
порог перколяции} 

\DOI{10.14357/19922264180213}
  
%\vspace*{-6pt}
\vspace*{6pt}


\vskip 10pt plus 9pt minus 6pt

\thispagestyle{headings}

\begin{multicols}{2}

\label{st\stat}

\section{Введение}

Изначально, в~классическом пред\-став\-ле\-нии, информатика (\textit{фр.}\ informatique, 
\textit{англ.}\ computer science)~--- это наука о~методах и~процессах сбора, хранения, 
обработки, передачи, анализа и~оценки информации с~применением 
компьютерных технологий, обеспечивающих воз\-мож\-ность ее использования 
для принятия решений. Однако стремительное развитие сетевых технологий, 
и~в~первую очередь Интернета, привело к~появлению обширного класса 
социотехнических сис\-тем, ярким пред\-ста\-ви\-те\-лем которых выступают 
социальные сети. Наличие человеческого фактора приводит к~тому, что для 
описания про\-те\-ка\-ющих в~них процессов уже недостаточно классических 
методов и~моделей, принятых в~информатике, для принятия решений требуется 
использование междисциплинарных подходов и~расширение усто\-яв\-ших\-ся 
классических пред\-став\-ле\-ний, особенно при исследовании процессов 
распространения и~обработки информации в~социальных сетях. В~этом плане 
необходимо говорить о том, что применение методов и~моделей информатики 
должно быть расширено на социотехнические и~социальные сис\-те\-мы точ\-но так 
же, как в~свое время Норберт Винер определил кибернетику как науку об 
управ\-ле\-нии в~живом и~неживом~[1]. 

  Исследование процессов распространения информации и~кластеризации 
узлов (клас\-тер~--- группа связанных между собой узлов, выделенных по 
определенным свойствам или типам) в~сетях социальных связей, име\-ющих 
случайную топологию, является очень важной и~актуальной задачей для 
экономики, рекламы, маркетинга, социологии, политологии и~т.\,д., что 
подтверждается достаточно большим чис\-лом работ по данной тематике~[2--4].
  
  Сеть социальных связей можно определить как со\-во\-куп\-ность 
информационных каналов каж\-до\-го человека, свя\-зы\-ва\-ющих его с~другими 
членами сообщества, а~так\-же средств массовой информации (т.\,е.\ не только 
взаимодействие между членами определенной группы в~социальной сети). 
Иными словами, сеть социальных связей является социальной сетью, 
помещенной в~информационную среду (средства массовой информации, книги, 
газеты, журналы и~т.\,д.).
  
  Средства массовой информации (радио, телевидение, ин\-тер\-нет-ре\-сур\-сы, 
социальные сети и~др.), а~так\-же книги, газеты, журналы оказывают 
существенное влияние на со\-сто\-яние отдельных узлов социальной сети (выбор 
предпочтений и~поведенческие реакции людей), которое может меняться 
с~течением времени  за счет как информационной среды, так 
и~взаимодействия пользователей между собой. 
  
  В связи с~этим возникает ряд вопросов. Во-пер\-вых, как может происходить 
изменение в~сети доли узлов, находящихся в~том или ином со\-сто\-янии?  
Во-вто\-рых, как эти узлы свя\-зы\-ва\-ют\-ся между собой в~подгруппы 
(клас\-те\-ри\-за\-ция сети)? В-треть\-их, как информационные процессы зависят от 
всей сети в~целом для различных со\-сто\-яний узлов?
  
  Узлами социальной сети являются отдельные люди, а~реб\-ра\-ми~--- 
коммуникативные связи, чис\-ло которых может иметь произвольное значение. 
  
  В такой сети распространение информации может одновременно 
происходить множеством путей\linebreak
 через разные узлы сети. Отдельный узел сети 
может получать от другого узла некоторую инфор\-мацию (рекламные 
предложения, идеи, политические взгляды, профессиональные сведения, 
\mbox{мнения} и~др.)\ и~передавать ее другим узлам, если имеет со\-глас\-ную 
с~по\-сту\-пив\-шей информацией позицию (является активным проводником), или 
блокирует ее в~противном случае. 
  
  Для моделирования и~анализа информационных процессов, протекающих 
в~социальных сетях со случайной структурой, воз\-мож\-но применение методов 
теории перколяции~[5--7], которая может поз\-во\-лить, например, ответить на 
сле\-ду\-ющие вопросы:
  \begin{enumerate}[1.]
\item Как происходит клас\-те\-ри\-за\-ция сети на группы связанных между собой 
определенными взглядами людей в~за\-ви\-си\-мости, например, от среднего чис\-ла 
связей на узел?
\item При какой доле людей (узлов сети) с~определенными взглядами может 
создаваться условие для беспрепятственного распространения этих взглядов 
между двумя любыми произвольно вы\-бран\-ны\-ми узлами (протекание или 
пер\-ко\-ля\-ция)? 
  \end{enumerate}
  
  Наиболее распространенными задачами тео\-рии перколяции являются 
\textit{решеточные задачи}: задача связей и~задача узлов. В~задаче связей ищут 
ответ на вопрос: какую долю связей нужно удалить (перерезать), чтобы сетка 
рас\-па\-лась на две части? В~задаче узлов блокируют узлы (удаляют узел, 
перерезая все входящие в~узел связи) и~ищут, при какой доле блокированных 
узлов сетка рас\-па\-дется. 
  
  \textbf{Кластеризация социальной сети.} Если переход любого узла 
социальной сети (индивидуума) из одно\-го со\-сто\-яния в~другое рас\-смат\-ри\-вать 
как случайный процесс (с~некоторой вероятностью перехода, определяемой 
множеством случайных факторов, в~том чис\-ле зависящих от вли\-яния средств 
массовой информации), то ве\-ро\-ят\-ность перехода будет влиять на средний 
размер клас\-те\-ра пользователей сети (группы на\-пря\-мую связанных меж\-ду собой 
узлов). 
  
  \textbf{Перколяция в~социальной сети.} В~тео\-рии пер\-ко\-ля\-ции доля 
проводящих (неблокированных) узлов, при которой возникает про\-во\-ди\-мость 
между двумя различными произвольно вы\-бран\-ны\-ми узлами сети, называется 
порогом пер\-ко\-ля\-ции (протекания). 
  
  Доля узлов социальной сети, находящихся в~том или ином со\-сто\-янии, может 
быть выявлена путем социологических опро\-сов, что пред\-остав\-ля\-ет 
воз\-мож\-ность оценить, насколько социальная сеть близка к~порогу пер\-ко\-ля\-ции, 
а~так\-же управ\-лять ее со\-сто\-яни\-ями.

\vspace*{-6pt}
  
\section{Обзор существующих моделей описания характеристик 
и~анализа структуры социальных сетей}
 
  Наиболее часто для проведения исследований операций и~процессов 
в~сетевых структурах сегодня используются готовые средства анализа, 
например инструменты анализа социальных сетей (SNA~--- social network 
analysis), позволяющие получить количественные характеристики па\-ра\-мет\-ров 
графа сети, таких как <<цент\-раль\-ность>>, <<про\-ме\-жу\-точ\-ность>>,  
<<плот\-ность>> (сред\-нее чис\-ло связей, приходящихся на один узел 
сети)~\cite{7-zh}. Цент\-раль\-ность характеризует степень влияния данного узла 
на всю сеть. Про\-ме\-жу\-точ\-ность характеризует степень включенности объекта 
в~марш\-ру\-ты связей между другими участниками сети. Про\-ме\-жу\-точ\-ность 
показывает, насколько час\-то данный узел встречается на кратчайших путях 
меж\-ду другими узлами. 
  
  Использование готовых инструментов имеет как свои преимущества, так 
и~ряд недостатков. К~преимуществам можно отнести то, что готовые 
инструменты социального сетевого анализа поз\-во\-ля\-ют сравнивать между собой 
 однотипные сети
по количественным характеристикам. Основные недостатки 
заключаются в~том, что они не позволяют создавать новые, более 
информативные модели. 
  
  В работе~\cite{8-zh} были изучены статистические свойства реальных 
социальных сетей работников домохозяйств и~их мера фрагментации после 
удаления некоторых долей узлов или ссылок из сети. 
  
  В работе~\cite{9-zh} рас\-смат\-ри\-ва\-ет\-ся взаимосвязь перколяционного перехода 
и~выживаемости узлов в~сложной сети с~3~млн связей, по\-стро\-ен\-ная 
вокруг примерно~300~тыс.\ фирм (узлы). Характер\-ная особенность этих 
реальных сетей за\-клю\-ча\-ет-\linebreak ся в~том, что они являются мас\-шта\-би\-ру\-емы\-ми\linebreak 
и~степень их мас\-шта\-би\-ру\-емости асимптотически следует степенному закону. 
Эта функ\-ция под\-разуме\-ва\-ет, что каждая из таких сетей со\-сто\-ит из нескольких 
крупных цент\-раль\-ных узлов с~тысячами связей, мно\-же\-ст\-вом промежуточных 
узлов и~еще б$\acute{\mbox{о}}$льшим числом очень мелких узлов 
с~несколькими связями~\cite{10-zh}. Масштабируемые сети кластеризуются на 
отдельные блоки при достаточно высокой плот\-ности связей, если узлы 
удаляются в~порядке убывания степени цент\-раль\-ности~\cite{11-zh}. При 
случайном удалении узлов и~связей мас\-шта\-би\-ру\-емые сети плохо 
клас\-те\-ри\-зу\-ют\-ся, даже при очень низкой плот\-ности узлов, и~перколяционные 
переходы не отмечаются. В~данном исследовании с~по\-мощью 
точного чис\-лен\-но\-го расчета~\cite{9-zh} доказано существование перколяционного 
перехода в~сложных сетях при случайном удалении узлов, когда плот\-ность сети 
очень низкая, но не нулевая. 

\vspace*{-6pt}
     
\section{Постановка задачи и~методика проведения исследований}

  Несмотря на существенный прогресс в~исследовании информационных 
процессов, про\-те\-ка\-ющих в~социальных сетях, и~использование для этого 
тео\-рии перколяции, еще очень многие задачи остаются нерешенными. 
Большинство исследователей уделяют большое внимание изучению степени 
про\-ме\-жу\-точ\-ности, плот\-ности сети, средней длине пути (бли\-зости), 
цент\-раль\-ности и~т.\,д. Вместе с~тем никто из исследователей не обращал еще 
це\-ле\-на\-прав\-лен\-но внимание на изуче\-ние таких вопросов, как влияние плот\-ности 
(среднего чис\-ла связей в~расчете на один узел) сети на ее кластеризацию 
и~величину порога перколяции (как в~задаче узлов, так и~в задаче связей) как 
в~масштабируемых, так и~в~случайных сетях. На взгляд авторов, необходимо 
учитывать в~целом всю со\-во\-куп\-ность свойств сети, которые определяют порог 
перколяции и~кластеризацию; значение имеют как узлы, у~которых много 
связей и~они явля\-ют\-ся значимыми или цент\-раль\-ны\-ми, так и~с~малым чис\-лом 
связей. 
  
  Следует отметить, что существующие про\-грам\-мные инструменты 
информационного анализа социальных сетей (SNA) в~данном случае не могут 
быть использованы, поскольку они не позволяют со\-зда\-вать случайные сети 
с~произвольной плот\-ностью связей и~изучать их клас\-те\-ри\-за\-цию и~перколяцию 
при блокировании узлов.
  
  Для изучения случайных сетей с~множеством связей аналитических моделей 
описания перколяционных процессов не существует, и~их исследование 
воз\-мож\-но только методами чис\-лен\-но\-го моделирования~\cite{12-zh, 13-zh} 
с~использованием специально разработанного программного обеспечения. Для 
этого необходимо сначала по\-стро\-ить структурную модель социальной сети, 
со\-сто\-ящую из большого чис\-ла (например, в~данном исследовании~--- 1~млн) узлов. Затем вы\-брать пару произвольных узлов и~с~по\-мощью 
методов чис\-лен\-но\-го моделирования определить, при какой доле 
неблокированных узлов (для задачи узлов) в~рас\-смат\-ри\-ва\-емой сети появляется 
свободный путь между рас\-смат\-ри\-ва\-емы\-ми узлами (или, наоборот, исчезает при 
блокировании). Затем аналогичным образом эта процедура проводится для 
других произвольных пар узлов. После этого со статистическим усреднением 
результатов по отдельным экспериментам необходимо про\-вес\-ти определение 
сред\-не\-го значения порога перколяции по всем рас\-смат\-ри\-ва\-емым парам 
узлов~\cite{13-zh, 14-zh}. В~задаче связей используется практически такой же 
алгоритм исследования, однако блокируются не узлы, а~связи. 

\begin{figure*} %fig1
\vspace*{1pt}
 \begin{center}
 \mbox{%
 \epsfxsize=106.462mm 
 \epsfbox{zhu-1.eps}
 }
 \end{center}
\vspace*{-9pt}
\Caption{Зависимость вероятности не\-воз\-мож\-ности передачи информации между 
двумя произвольно вы\-бран\-ны\-ми узлами случайной сети от ве\-ро\-ят\-ности блокирования 
(разрыва) одной связи}
\end{figure*}

\vspace*{-6pt}

\section{Исследование кластеризации узлов социальных сетей 
и~достижения порога перколяции}

\subsection{Перколяция (протекание) информации в~сетях со~случайной 
структурой}

  Задача связей (разрываются связи между узлами) при определении порогов 
перколяции в~сети, име\-ющей случайную структуру, была решена авторами 
ранее в~работе~\cite{14-zh}. Результаты проведенного чис\-лен\-но\-го 
моделирования~\cite{13-zh, 14-zh} за\-ви\-си\-мости средней ве\-ро\-ят\-ности 
не\-воз\-мож\-ности передачи информации между двумя произвольно выбранными 
узлами случайной сети с~множеством путей между узлами и~различным 
сред\-ним чис\-лом связей на один узел от ве\-ро\-ят\-ности блокирования (разрыва) 
связи пред\-став\-ле\-ны на рис.~1. 
  
  


Кривая~\textit{1} построена для сети, у~которой среднее чис\-ло связей 
в~расчете на один узел со\-став\-ля\-ет~3,99, кривая~\textit{2}~--- 5,99, 
кривая~\textit{3}~--- 7,97, кривая~\textit{4}~--- 9,93, кривая~\textit{5}~--- 13,86 
и~кривая~\textit{6}~--- 15,79~связей. Основной задачей исследования является 
определение порогов перколяции для сетей с~различным средним чис\-лом 
связей в~расчете на один узел, но в~задаче узлов при этом удаляются узлы, 
а~в~задаче связей~--- связи. Можно определить линейные участки в~цент\-ре 
кривых~\textit{1}--\textit{6} и~экстраполировать их до 
пересечения с~осью абсцисс (см.\ рис.~1), значения величин которых можно 
условно принять за величину нижний границы порога перколяции данной сети. 
  
  На рис.~2 пред\-став\-ле\-на зависимость величины нижней границы порога 
перколяции от среднего чис\-ла связей на один узел данной сети, найденная по 
описанной выше методике.

 { \begin{center}  %fig2
 \vspace*{9pt}
  \mbox{%
 \epsfxsize=79mm 
 \epsfbox{zhu-2.eps}
 }


\end{center}

\vspace*{-6pt}


\noindent
{{\figurename~2}\ \ \small{Зависимость порога перколяции в~случайной сети от среднего чис\-ла 
связей (плот\-ности сети) на один ее узел в~задаче связей}}
}

%\vspace*{9pt}


  

  Представленная на рис.~2 зависимость хорошо линеаризуется 
в~координатах $\ln P(x)$ от $z\hm=1/x$ (натуральный логарифм 
порога перколяции~--- величина, обратная среднему чис\-лу связей~$x$, 
приходящихся на один узел) и~поз\-во\-ля\-ет получить линейную за\-ви\-си\-мость 
$y\hm= -6{,}581z\hm-0{,}203$ со значением коэффициента корреляции, 
рав\-ным~0,992. 
  
  Здесь говорится о~пороге перколяции как невозможности передачи 
информации при некоторой ве\-ро\-ят\-ности разрыва связи. Например, если 
в~данном случае для сети с~плот\-ностью связей~3,99 порог перколяции 
равен~0,16, то это значит, что если разорвать меньше~16\% связей, то сеть 
информацию передает, а~если~16\% или больше, то передача информации 
в~сети в~целом прекращается. Но это будет ниж\-няя оценка. Заметим, что 
другие методы будут давать большее значение величины порога (верхнюю 
оценку). Например, если за осно\-ву взять ве\-ро\-ят\-ность невозможности передачи 
информации, равную~0,5, которая показана на рис.~1 горизонтальной штриховой
линией, 
то получим сле\-ду\-ющие значения порогов перколяции: для среднего чис\-ла 
связей~3,99~--- 0,46; для~5,99~--- 0,60; для~7,97~--- 0,66; для~9,93~--- 0,69; 
для~13,86~--- 0,73; для~15,79~--- 0,75.



  
  Задача нахождения порога перколяции не для разорванных связей, а~для 
блокированных узлов в~случайной сети была решена в~работе~\cite{12-zh}. 
Результаты чис\-лен\-но\-го моделирования нахождения порога перколяции для 
случайных сетей с~множеством путей между узлами и~различным средним 
чис\-лом связей на один узел также хорошо линеаризуются в~координатах $\ln 
P(x)$ от $z\hm=1/x$ (натуральный логарифм порога 
пер\-ко\-ля\-ции~--- величина, обратная сред\-не\-му чис\-лу связей~$x$ (плот\-ность 
сети), приходящихся на один узел) и~позволяют получить линейную 
за\-ви\-си\-мость $y\hm=4{,}39z\hm-2{,}41$ со значением коэффициента 
корреляции, рав\-ным~0,95.
  
  Здесь речь идет о~пороге перколяции уже как о~воз\-мож\-ности передачи 
информации при некоторой ве\-ро\-ят\-ности активации узлов (узел становится 
проводящим). Например, если в~дан\-ном случае для сети с~плот\-ностью 
связей~4,70 порог перколяции равен~0,27, то это значит, что если будет 
активировано меньше~27\%~узлов, то сеть информацию не передаст, 
а~если~27\% или больше, то передача информации в~сети в~целом возникает. 
  
  Проанализируем полученные данные. Выберем для задачи узлов и~задачи 
связей в~качестве примера четыре значения плот\-ности сети (среднего чис\-ла 
связей, приходящихся на один узел): 5, 10, 50 и~100. Для задачи узлов 
используем урав\-не\-ние $y\hm=4{,}39z\hm-2{,}41$ и~при $z \hm=1/5\hm=0{,}2$ 
получим $y\hm= -1{,}532$, и~величина порога пер\-ко\-ля\-ции будет рав\-на~0,22 
(в~данном случае доля проводящих узлов, при которой появляется 
про\-во\-ди\-мость). Рас\-чет по урав\-не\-нию $y\hm= -6{,}581z\hm- 0{,}203$ для задачи 
связей при $z\hm=0{,}2$ дает величину доли разорванных связей, при которой 
исчезает про\-во\-ди\-мость всей сети в~целом, рав\-ную~0,22 (т.\,е., если 
разорвем~22\% связей и~более, про\-во\-ди\-мость исчезнет).
  
  Полученные результаты позволяют сделать  ряд важ\-ных выводов
  для перколяционных процессов 
в~случайных сетях. Они приводятся в~разд.~5.
  
\subsection{Кластеризация социальной сети}

  Поскольку проводимость узлов в~большей степени определяет решение 
информационных задач всей сетью в~целом, то будем рас\-смат\-ри\-вать задачу 
узлов. Переход любого узла социальной сети из одного со\-сто\-яния в~другое 
можно рас\-смат\-ри\-вать как случайный процесс (с~некоторой вероятностью 
перехода, опре\-де\-ля\-емой множеством случайных факторов), и~эта ве\-ро\-ят\-ность 
должна влиять на средний раз\-мер клас\-те\-ра (группа напрямую связанных между 
собой узлов). В~проведенных авторами чис\-лен\-ных 
 экспериментах~\cite{12-zh, 14-zh} было исследовано влияние вероятности 
перехода узла из одного со\-сто\-яния в~другое (например, желания голосовать 
<<за>> или <<против>>) на средний раз\-мер клас\-те\-ра пользователей, 
находящихся в~данном со\-сто\-янии (в~долях от общего чис\-ла всех 
индивидуумов) в~случайных сетях с~различным средним чис\-лом связей на один 
узел и~общим чис\-лом узлов, равном~1~млн. Можно рас\-смот\-реть два 
предельных случая: первый~--- небольшое среднее чис\-ло связей на 
один узел социальной сети; второй~--- большое среднее 
чис\-ло связей. Исследования показали, что с~рос\-том среднего чис\-ла связей при 
фиксированной ве\-ро\-ят\-ности воздействия раз\-мер клас\-те\-ра увеличивается. 
Аналогичная ситуация наблю\-да\-ет\-ся и~при большом среднем чис\-ле связей на 
один узел случайной сети.
  
  С ростом среднего числа связей при фиксированной вероятности воздействия 
размер кластера увеличивается, а~ско\-рость рос\-та клас\-те\-ри\-за\-ции узлов, 
находящихся в~данном со\-сто\-янии, наиболее сильно увеличивается в~об\-ласти 
значений ве\-ро\-ят\-ности перехода единичных узлов от~0,4 до~0,6, а~при малых 
и~высоких значениях воз\-рас\-та\-ет не так \mbox{сильно}.

\vspace*{-6pt}
  
  \section{Выводы}
  
  \noindent
  \begin{enumerate}[1.]
\item Как в~задаче связей, так и~в задаче узлов величина порога перколяции 
случайной сети зависит от ее плот\-ности (среднего чис\-ла узлов, приходящихся 
на один узел).\\[-13pt]
\item В сетях, имеющих случайную структуру, пороги перколяции как в~задаче 
узлов, так и~в задаче связей при большой плот\-ности сети (среднее число связей 
на один узел) практически достигают величины насыщения (0,24~для задачи 
связей и~0,10 для задачи узлов) и~затем слабо зависят от нее. Величина 
насыщения порога перколяции в~задаче связей почти в~2,5~раза больше, чем 
в~задаче узлов.\\[-13pt] 
\item С точки зрения создания про\-во\-ди\-мости случайной сети в~целом 
образование про\-во\-дя\-щих связей в~задаче связей менее эффективно, чем 
образование проводящих узлов в~задаче узлов (например, при плотности сети, 
рав\-ной~5, для возникновения про\-во\-ди\-мости необходимо иметь долю 
проводящих узлов, равную~0,22, в~то время как доля проводящих связей 
должна быть равной~0,78).\\[-13pt] 
\item С ростом среднего чис\-ла связей (плот\-ности сети) при фиксированной 
ве\-ро\-ят\-ности воздействия раз\-мер клас\-те\-ра увеличивается, а~ско\-рость роста 
клас\-те\-ри\-за\-ции узлов, находящихся в~данном состоянии, наиболее сильно 
увеличивается в~области значений ве\-ро\-ят\-ности перехода единичных узлов 
от~0,4 до~0,6, а~при малых и~высоких значениях возрастает не так сильно.\\[-13pt] 
\item С точки зрения информационного влияния сети, име\-ющей случайную 
структуру, увеличение плот\-ности связей оказывает большее влияние, чем 
наличие отдельных <<центральных>> узлов, име\-ющих множество связей.\\[-13pt] 
\item В практическом плане полученные результаты могут быть применены 
в~междисциплинар-\linebreak\vspace*{-12pt}

\pagebreak

\noindent
 ных исследованиях, вклю\-чая информатику, ма\-те\-матическое 
моделирование и~экономику, с~при\-вле\-че\-ни\-ем социологических данных для 
прогнозирования поведения и~управ\-ле\-ния группами людей в~сетевых 
сообществах. Кроме того, полученные результаты дополняют и~расширяют 
применение методов и~подходов, принятых в~классической информатике, на 
описание социальных и~социотехнических сис\-тем, что может быть полезно для 
широкого круга исследователей, занимающихся изуче\-ни\-ем социальных 
сетевых струк\-тур.
\end{enumerate}

\vspace*{-6pt}

{\small\frenchspacing
 {%\baselineskip=10.8pt
 \addcontentsline{toc}{section}{References}
 \begin{thebibliography}{99}
 \bibitem{0-zh}
 \Au{Винер Н.} Кибернетика, или Управление и~связь в~животном и~машине~//
 Пер. с~англ.~--- 2-е изд.~--- М.: Наука, 1983. 344~с.
 (\Au{Wiener~N.} Cybernetics: Or control and communication in 
 the animal and the machine.~---
 2nd ed.~--- MIT Press, 1961. 212~p.)
\bibitem{1-zh}
\Au{Баканова С.\,А., Силкина~Г.\,Ю.} Процессы распространения знаний 
в~параметризованной сети информационных обменов~// На\-уч\-но-тех\-ни\-че\-ские 
ведомости Санкт-Пе\-тер\-бург\-ско\-го государственного политехнического 
университета. Экономические науки, 2015. №\,2(216). С.~133--146.
\bibitem{2-zh}
\Au{Сулимов П.\,А.} Методы машинного обучения для предсказания распространения 
инфекции в~сети~// Вестн. НГУЭУ, 2016. №\,1. С.~285--306.
\bibitem{3-zh}
\Au{Торопов Б.\,А.} Модель независимых каскадов распространения репоста 
в~онлайновой социальной сети~// Кибернетика и~программирование, 2016. №\,5.  
С.~61--67.
\bibitem{4-zh}
\Au{Тарасевич Ю.\,Ю.} Перколяции: теория, приложения, алгоритмы.~--- М.: Эдиториал 
УРСС, 2002. 112~с.
\bibitem{5-zh}
\Au{Лесько С.\,А., Жуков~Д.\,О., Самойло~И.\,В.} Математическое моделирование 
перколяционных процессов передачи данных и~потери работоспособности  
в~ин\-фор\-ма\-ци\-он\-но-вы\-чис\-ли\-тель\-ных сетях с~2D и~3D регулярной и~случайной 
структурой~// Качество. Инновации. Образование, 2013. №\,6(97). С.~42--50.
\bibitem{6-zh}
\Au{Лесько С.\,А., Жуков~Д.\,О., Самойло~И.\,В., Брукс~Д.\,У.} Алгоритмы построения 
сетей и~моделирования потери их работоспособности в~результате кластеризации 
блокированных узлов~// Качество. Инновации. Образование, 2013. №\,12(103). С.~82--87. 
\bibitem{7-zh}
\Au{Павлековская И.\,В.} Применение метода анализа социальных сетей в~моделировании 
процессов распространения информации и~знаний в~организации~//\linebreak  
На\-уч\-но-тех\-ни\-че\-ская информация. Сер.~2: Информационные процессы 
и~системы, 2007. №\,3. С.~30--36.
\bibitem{8-zh}
\Au{Chen Y., Paul~G., Cohen~R., Yavlin~S., Borgatti~S.\,P., Liljeros~F., Stanley~H.\,E.} 
Percolation theory and fragmentation measures in social networks~// Physica~A, 2006. Vol.~378. No.\,1. P.~11--19.
\bibitem{9-zh}
\Au{Kawamoto H., Takayasu~H., Jensen~H.\,J., Takayasu~M.} Precise calculation of a~bond 
percolation transition and survival rates of nodes in a~complex network~// 
PLoS One, 2015. {\sf https://doi.org/10.1371/journal.pone.0119979}.
\bibitem{10-zh}
\Au{Barab$\acute{\mbox{a}}$si A.\,L., Albert~R.} Emergence of scaling in random networks~// 
Science, 1999. Vol.~286. P.~509--512.
\bibitem{11-zh}
\Au{Albert R., Jeong~H., Barab$\acute{\mbox{a}}$si~A.\,L.} Error and attack tolerance of 
complex networks~// Nature, 2000. Vol.~406. P.~378--382.
\bibitem{12-zh}
\Au{Zhukov D., Lesko~S.} Percolation models of information dissemination in social 
networks~// IEEE Conference (International) on Smart City/SocialCom/SustainCom together 
with DataCom Proceedings.~--- IEEE, 2015. P.~213--216.
\bibitem{13-zh}
\Au{Block M., Khvatova~T., Zhukov~D., Lesko~S.} Studying the structural topology of the 
knowledge sharing network~// 11th European Conference on Management, Leadership and 
Governance Proceedings.~--- Lisbon, Portugal: Academic Conferences and 
Publishing International Ltd., 2015. P.~20--27.
\bibitem{14-zh}
\Au{Khvatova T., Block~M., Zhukov~D., Lesko~S.} How to measure trust: 
The percolation model 
applied to intra-organisational knowledge sharing networks~// J.~Knowl. Manag., 
2016. Vol.~20. No.\,5. P.~918--935.
%\bibitem{15-zh}
%\Au{Сигов А.\,С., Акимов~Д.\,А., Жуков~Д.\,О., Андрианова~Е.\,Г., Сачков~В.\,Е., 
%Раев~В.\,К.} Психолингвистический анализ русскоязычных текстовых сообщений на 
%основе их фоносемантических статистических характеристик~// Информатика и~её 
%применения, 2017. Т.~11. Вып.~3. С.~77--86.
 \end{thebibliography}

 }
 }

\end{multicols}

\vspace*{-7pt}

\hfill{\small\textit{Поступила в~редакцию 04.07.17}}

\vspace*{6pt}

%\newpage

%\vspace*{-28pt}

\hrule

\vspace*{2pt}

\hrule

\vspace*{-6pt}


\def\tit{THE INFLUENCE OF~THE~CONNECTIONS' DENSITY ON~CLUSTERIZATION AND~PERCOLATION 
THRESHOLD DURING~INFORMATION DISTRIBUTION IN~SOCIAL NETWORKS}

\def\titkol{The influence of~the~connections' density on~clusterization and~percolation 
threshold during~information distribution in %~social 
networks}


\def\aut{D.\,O.~Zhukov$^1$, T.\,Yu.~Khvatova$^2$, S.\,A.~Lesko$^1$, 
and~A.\,D.~Zaltsman$^1$}

\def\autkol{D.\,O.~Zhukov, T.\,Yu.~Khvatova, S.\,A.~Lesko, 
and~A.\,D.~Zaltsman}

\titel{\tit}{\aut}{\autkol}{\titkol}

\vspace*{-11pt}


\noindent
$^1$Moscow Technological University (MIREA), 78~Vernadskogo Ave., 
Moscow 119454, Russian Federation 

\noindent
$^2$Peter the Great St. Petersburg Polytechnic University, 
29~Polytechnicheskaya Str., St.\ Petersburg 195251, Russian\linebreak
$\hphantom{^1}$Federation


\def\leftfootline{\small{\textbf{\thepage}
\hfill INFORMATIKA I EE PRIMENENIYA~--- INFORMATICS AND
APPLICATIONS\ \ \ 2018\ \ \ volume~12\ \ \ issue\ 2}
}%
 \def\rightfootline{\small{INFORMATIKA I EE PRIMENENIYA~---
INFORMATICS AND APPLICATIONS\ \ \ 2018\ \ \ volume~12\ \ \ issue\ 2
\hfill \textbf{\thepage}}}

\vspace*{3pt}

 
  \Abste{The paper is focused on applying new theoretical approaches to describing 
  the processes of information transmission and transformation in sociotechnical 
  systems and in social networks based on the percolation theory.
  Percolation 
  threshold of a~random network depends on its density. In networks with random 
  structure, in both the\linebreak\vspace*{-12pt}}
  
  \Abstend{task of bonds and the task of nodes, percolation 
  thresholds reach saturation when the network's density is high. The saturation 
  value of a~percolation threshold is higher in the task of bonds. From the point 
  of information influence of a~random network, increasing the average connection's 
  density within the network turns out to be more preferable than fostering 
  a~small number of separate `central nodes' with numerous connections. The results 
  obtained in this study can be applied in interdisciplinary research in such 
  areas as informatics, mathematic modeling, and economics involving certain 
  sociological survey data for forecasting behavior and managing groups of individuals 
  in network communities. This research enhances and enlarges the scope of methods 
  and approaches applied in classic informatics for describing social 
  and sociotechnical systems, which can be useful for 
  a~wide range of researchers engaged into studying social network structures.}

\KWE{percolation theory; social network structure; connections' density; 
network clusterisation; percolation threshold}



\DOI{10.14357/19922264180213} %

\vspace*{-19pt}

 \Ack
 
 \vspace*{-3pt}
 
\noindent
This research was performed with financial support of the Russian Foundation for 
Basic Research (project No.\,16-29-09458~ofi\_m) ``Developing percolation 
topological models for describing virtual social systems, their participants' 
clusterization into groups according to their views, stochastic dynamics of influence 
distribution, and for managing transitions.''



%\vspace*{2pt}

  \begin{multicols}{2}

\renewcommand{\bibname}{\protect\rmfamily References}
%\renewcommand{\bibname}{\large\protect\rm References}

{\small\frenchspacing
 {%\baselineskip=10.8pt
 \addcontentsline{toc}{section}{References}
 \begin{thebibliography}{99}
 
\bibitem{0-zh-1}
\Aue{Wiener,~N.} 1961. \textit{Cybernetics: Or control and communication in 
the animal and the machine}.
 2nd ed. MIT Press. 212~p.
\bibitem{1-zh-1}
\Aue{Bakanova, S.\,A., and G.\,Yu.~Silkina.} 2015. 
Knowledge dissemination process in parametrized networks of enterprises. 
\textit{St.\ Petersburg State Polytechnical University~J. Economics}
2(216):133--146.
\bibitem{2-zh-1}
\Aue{Sulimov, P.\,A.} 2016. Metody ma\-shin\-no\-go obucheniya dlya pred\-ska\-za\-niya 
ras\-pro\-stra\-ne\-niya in\-fek\-tsii v~seti [Methods of machine learning for forecasting an 
infection distribution within a~network]. \textit{Vestnik NGUEU} [NGUEU~J.] 
1:285--306.
\bibitem{3-zh-1}
\Aue{Toropov, B.\,A.} 2016. Model' ne\-za\-vi\-si\-mykh kas\-ka\-dov ras\-pro\-stra\-ne\-niya 
re\-pos\-ta v~on\-lay\-no\-voy so\-tsi\-al'\-noy seti [The model of independent cascades of 
a~repost distribution in an online social network]. \textit{Kibernetika 
i~programmirovanie} [Cybernetics and Programming] 5:61--67.

\bibitem{4-zh-1}
\Aue{Tarasevich, Yu.} 2002. \textit{Perkolyatsii: teoriya, prilozheniya, algoritmy} 
[Percolations: Theory, applications, algorithms]. Moscow: Editorial URSS. 112~p.
\bibitem{5-zh-1}
\Aue{Les'ko, S., D.~Zhukov, and I.~Samoylo.} 2013. Ma\-te\-ma\-ti\-che\-skoe 
mo\-de\-li\-ro\-va\-nie per\-ko\-lya\-tsi\-on\-nykh pro\-tses\-sov 
peredachi dan\-nykh i~poteri 
ra\-bo\-to\-spo\-sob\-nosti v~in\-for\-ma\-tsi\-on\-no-vy\-chis\-li\-tel'\-nykh setyakh s~2D 
i~3D re\-gu\-lyar\-noy i~slu\-chay\-noy struk\-tu\-roy [Mathematic modeling of percolation 
processes of data transmission and operability loss in informational-computational 
networks with 2D and 3D regular and random structure]. \textit{Kachestvo. 
Innovatsii. Obrazovanie} [Quality. Innovations. Education] 11(97):42--50.

\bibitem{6-zh-1}
\Aue{Les'ko, S., D.~Zhukov, I.~Samoylo, and D.~Bruks.} 2013b. Algoritmy 
po\-stro\-eniya se\-tey i~mo\-de\-li\-ro\-va\-niya poteri ikh 
ra\-bo\-to\-spo\-sob\-nosti v~re\-zul'\-ta\-te 
klas\-te\-ri\-za\-tsii blo\-ki\-ro\-van\-nykh uz\-lov 
[Algorithms of network construction and 
modeling of their operability loss as a~result of blocked nodes clusterization]. 
\textit{Kachestvo. Innovatsii. Obrazovanie} [Quality. Innovations. Education] 
12(103):25--33. 
\bibitem{7-zh-1}
\Aue{Pavlekovskaya, I.\,V.} 2007. The use
of social network analysis in modeling the organizational
processes of 
 information and knowledge circulation.  
\textit{Automatic Documentation Math. Linguistics} 41(2):65--72.

\bibitem{8-zh-1}
\Aue{Chen, Y., G.~Paul, R.~Cohen, S.~Yavlin, S.\,P.~Borgatti, F.~Liljeros, and 
H.\,E.~Stanley.} 2006. Percolation theory and fragmentation measures in social networks. 
\textit{Physica~A} 378(1):11--19.

\bibitem{9-zh-1}
\Aue{Kawamoto, H., H.~Takayasu, H.\,J.~Jensen, and M.~Ta\-ka\-yasu.} 2015. Precise 
calculation of a~bond percolation transition and survival rates of nodes in a~complex 
network. \textit{PLoS One}. Available at:  {\sf 
http://journals. plos.org/plosone/article?id=10.1371/journal.pone.\linebreak 0119979} (accessed 
December~16, 2017).


\bibitem{10-zh-1}
\Aue{Barab$\acute{\mbox{a}}$si, A.\,L., and R.~Albert.} 1999. Emergence of 
scaling in random networks. \textit{Science} 286:509--512.
\bibitem{11-zh-1}
\Aue{Albert, R., H.~Jeong, and A.\,L.~Barab$\acute{\mbox{a}}$si.} 2000. Error 
and attack tolerance of complex networks. \textit{Nature} 406:378--382.
\bibitem{12-zh-1}
\Aue{Zhukov, D., and S.~Lesko.} 2015. Percolation models of information 
dissemination in social networks. \textit{IEEE Conference (International) on Smart 
City/SocialCom/SustainCom together with DataCom Proceedings}. IEEE.
213--216.
\bibitem{13-zh-1}
\Aue{Block, M., T.~Khvatova, D.~Zhukov, and S.~Lesko.} 2015. Studying the 
structural topology of the knowledge sharing network. \textit{11th European 
Conference on Management, Leadership and Governance Proceedings}. 
Lisbon, Portugal: Academic Conferences and 
Publishing International Ltd. 20--27. 
\bibitem{14-zh-1}
\Aue{Khvatova, T., M.~Block, D.~Zhukov, and S.~Lesko.} 2016. How to measure 
trust: The percolation model applied to intra-organisational knowledge sharing 
networks. \textit{J.~Knowl. Manag.} 20(5):918--935.
%\bibitem{15-zh-1}
%\Aue{Sigov, A.\,S., D.\,A.~Akimov,  D.\,O.~Zhukov, E.\,G.~An\-drianova, 
%V.\,E.~Sachkov, and V.\,K.~Raev.} 2017. Psi\-kholing\-vi\-sti\-che\-skiy analiz 
%russkoyazychnykh tekstovykh soobshche\-niy na osnove ikh fonosemanticheskikh 
%statisticheskikh kharakteristik [Psycho-linguistic analysis of Russian text messages 
%based on their phonosemantic statistical characteristics]. \textit{Informatika i~ee 
%Primeneniya~--- Informatics Appl.} 11(3):77--86.
\end{thebibliography}

 }
 }

\end{multicols}

\vspace*{-9pt}

\hfill{\small\textit{Received July 4, 2017}}

%\pagebreak

%\vspace*{-24pt}


\Contr

\noindent
\textbf{Zhukov Dmitry O.} (b.\ 1965)~--- Doctor of Science in technology, 
professor, Head of Department, Moscow Technological University (MIREA), 
78~Vernadskogo Ave., Moscow 119454, Russian Federation; 
\mbox{zhukov\_do@mirea.ru}

\vspace*{6pt}

\noindent
\textbf{Khvatova Tatiana Yu.} (b.\ 1971)~---  Doctor of Science in economics, 
professor, Peter the Great St.\ Petersburg Polytechnic University, 
29~Polytechnicheskaya Str., St.\ Petersburg 195251, Russian Federation; 
\mbox{khvatova.ty@spbstu.ru}

\vspace*{6pt}

\noindent
\textbf{Lesko Sergey A.} (b.\ 1983)~--- Candidate of Science (PhD) in technology, 
associate professor, Moscow Technological University (MIREA), 78~Vernadskogo 
Ave., Moscow 119454, Russian Federation; \mbox{sergey@testor.ru}

\vspace*{6pt}

\noindent
\textbf{Zaltsman Anastasia D.} (b.\ 1989)~--- lecturer, 
Moscow Technological University (MIREA), 78~Vernadskogo Ave., Moscow 119454, Russian 
Federation; \mbox{ad.zaltcman@gmail.com} 

\label{end\stat}


\renewcommand{\bibname}{\protect\rm Литература}    %13
\def\stat{mirzabekov}

\def\tit{ДИСКРЕТНЫЙ АНАЛИЗ В~СИНТАКСИЧЕСКОМ АНАЛИЗЕ}

\def\titkol{Дискретный анализ в~синтаксическом анализе}

\def\aut{Я.\,М.~Мирзабеков$^1$, Ш.\,Б.~Шихиев$^2$}

\def\autkol{Я.\,М.~Мирзабеков, Ш.\,Б.~Шихиев}

\titel{\tit}{\aut}{\autkol}{\titkol}

\index{Мирзабеков Я.\,М.}
\index{Шихиев Ш.\,Б.}
\index{Mirzabekov Ya.\,M.}
\index{Shihiev Sh.\,B.}


%{\renewcommand{\thefootnote}{\fnsymbol{footnote}} \footnotetext[1]
%{Работа поддержана РНФ (проект 16-11-10227).}}


\renewcommand{\thefootnote}{\arabic{footnote}}
\footnotetext[1]{Дагестанский государственный университет, yash831@mail.ru}
\footnotetext[2]{Дагестанский государственный университет, \mbox{sh\_sh\_b51@mail.ru}}

%\vspace*{-8pt}

    
        
      \Abst{Дано неформальное определение синтаксиса в~терминах 
дискретной математики и~тео\-рии графов. Приведен алгоритм выделения семантически 
замкнутых (самодостаточных по значению) фрагментов предложения. Отсутствие формального 
определения понятия семантики слова и~сочетания словоформ затрудняет компьютерную 
обработку текста, в~частности процесс его сегментации. Показано, каким образом можно 
классифицировать выражения по семантическому признаку, используя их синтаксические 
особенности. Классификация выражений по тому, каким вопросам они отвечают, является 
самым доступным и~упрощенным способом группировки выражений по семантическому 
признаку. Описан алгоритм, распознающий указатели места, т.\,е.\ выражения, 
отвечающие на вопрос <<где?>>. На конкретных примерах рас\-смот\-ре\-ны задача анализа 
выражений и~обратная ей задача синтеза~--- прикладная задача распознавания выражений.}
      
      \KW{естественный язык; дискретная математика; теория графов; синтаксис; 
словоформа; морфологический параметр; согласованные определения; несогласованные 
определения; лексика; семантика}

\DOI{10.14357/19922264180214}
  
\vspace*{3pt}


\vskip 10pt plus 9pt minus 6pt

\thispagestyle{headings}

\begin{multicols}{2}

\label{st\stat}
     
\section{Введение}

\vspace*{-3pt}

     Рассматриваемая задача имеет целью предложить новые подходы 
к~решению задачи \textit{синтаксического анализа}, и,~в~част\-ности, 
предлагается алгоритм решения известных в~литературе задач Text Mining по 
\textit{извлечению понятий из текста}~[1]. Предлагаемый в~статье алгоритм 
отличается от известных авторам работ тем, что он опирается на 
\textit{конструктивную тео\-рию синтаксиса}, которая построена как раздел 
дискретного анализа и~в~которой \textit{синтаксис} и~его элементы 
(\textit{словосочетание, выражение, предложение} и~т.\,д.)\ имеют формальные 
определения~[2].
     
     Синтаксический анализ предложения на компьютере давно является 
предметом исследования различных инициативных групп, лабораторий 
и~академических институтов. Исторически так сложилось, что под 
\textit{синтаксическим анализом} предложения подразумевается его 
и~\textit{морфологический}, и~\textit{синтаксический}, и~\textit{семантический} 
анализ (и~словоформ, и~словосочетаний, и~их значений). \textit{Морфология} 
с~\textit{синтаксисом}, с~одной стороны, и~\textit{семантика}, с~другой стороны, 
изучают предметы несовместимой природы~--- мир материальных вещей (слова 
и~их сочетания, план выражения) и~мир пред\-став\-ле\-ний и~мыслей (план 
содержания). Язык есть только \textit{отображение} (\textit{форма}, по 
выражению Ф.~де Соссюра) этих двух \textit{субстанций}. Попытка сходу объять 
их в~рамках одной тео\-рии, а~тем более в~рам\-ках одного алгоритма изначально 
была рискованным предприятием. 
     
     Хорошо известны и~те, кто стоял у истоков <<\textit{синтаксического 
и~глубинного анализов}>> (Н.~Хомский, Д.~Слобин и~др.), и~труды российских 
ученых по созданию \textit{синтаксических анализаторов} предложений русского 
языка. 

В~трудах В.\,А.~Тузова (СПбГУ), Р.\,Г.~Пиотровского (СПбГПУ им.\ 
А.\,И.~Герцена), сотрудников МГУ 
(А.\,С.~Старостин, М.\,Г.~Мальковский и~др.)\ и~РАН (И.\,А.~Мельчук, 
Ю.\,Д.~Апресян и~др.)\ и~особенно в~работах ежегодного 
сборника <<Диалог-21>> отражены достижения и~тенденции развития 
компьютерных наук. А~в~работах А.\,О.~Казенникова, И.\,П.~Кузнецова 
и~Н.\,В.~Сомина, а~так\-же А.\,С.~Старостина, М.\,Г.~Мальковского 
и~Н.\,В.~Арефь\-ева действенно используются методы дискретного анализа при 
построении анализаторов. И~задача <<извлечения из текс\-та понятий>> 
определенного типа~\cite{3-mir} так\-же известна дав\-но и~решается различными 
эвристическими методами.
     
     На особенностях и~име\-ющих\-ся серьезных достижениях в~построении 
синтаксических анализаторов не будем останавливаться по той прос\-той причине, 
что в~данной статье под \textit{синтаксическим анализом} подразумевается нечто 
отличное от традиционного определения этого понятия. 
     
     На примере урезанного (упрощенного) син\-таксиса русского языка 
приводится описание \textit{формаль\-но\-го синтаксиса} (без участия семантики 
в~конструкциях \textit{синтаксиса}). В~рамках \textit{формального синтаксиса} 
формулируется и~решается немало\linebreak\vspace*{-12pt}

\pagebreak

\noindent
 интересных задач линг\-ви\-сти\-ки. На примере 
построения универсального алгоритма решения задачи \textit{извлечения понятий 
из текста} показана одна из возможностей \textit{формального синтаксиса}.
     
     \textit{Формальный синтаксис}~--- понятие диск\-рет\-ной математики. Как 
структура, определяемая лексикой и~набором словосочетаний (упорядоченных 
пар словоформ), она открыта в~том смысле, что может быть дополнена не только 
новыми словами и~словосочетаниями, но и~тем, что \textit{элементы синтаксиса} 
могут быть инкапсулированы семантикой. Последнее обстоятельство дает 
основание и~воз\-мож\-ность для автономного изучения \textit{синтаксиса} без 
услуг семантики и~ощущение лег\-кости формулировок задач и~алгоритмов их 
решения.
     
     Как известно, <<из текстов извлекаются отдельные факты с~по\-мощью 
лексического анализа>>, а~далее, на втором шаге, <<за этим следует 
синтаксический разбор>>~[1]. Под \textit{фактом}, видимо, следует 
подразумевать категорию семантики (например, \textit{понятие} или 
\textit{фрагмент знания}), выраженную одним словом или выражением из 
нескольких словоформ. Когда носителем языка, т.\,е.\ \textit{слова} и~его 
\textit{значения}, является человек, тогда грамматика и~языкознание опираются на 
\textit{соответствие} между \textit{словом} и~его \textit{значением}, имеющимся 
в~сознании человека. В~задачах компьютерной обработки текс\-та разработчики 
со\-от\-вет\-ст\-ву\-ющих алгоритмов, как правило, пытаются в~самом алгоритме 
реализовать свои интуитивные пред\-став\-ле\-ния о~\textit{значении слова}. 
     
     Основной целью статьи является при\-вле\-че\-ние внимания читателя 
к~сле\-ду\-ющим двум методологическим проб\-ле\-мам, име\-ющим место 
в~компьютерной лингвистике, и~путям их решения. 
     \begin{enumerate}[1.]
     \item Возможность замены эвристических алгоритмов на точ\-ные алгоритмы 
решения задач выявления типа элементов синтаксиса (например, является ли 
предложением данная строка символов?). 
     \item  Приемы построения \textit{отображения} между двумя множествами: 
синтаксически правильных выражений и~их значений (поэтому в~статье 
в~качестве первого множества рас\-смат\-ри\-ва\-ет\-ся множество выражений с~явной 
семантикой; если быть точ\-нее, то выражений, отвечающих на вопрос \textit{где?}).
     \end{enumerate}
     
     Не вдаваясь в~тонкости проб\-ле\-мы семантики языка, отметим: 
значение слова~--- это то, что просыпается в~памяти человека при восприятии 
этого слова. А~значение слова, описанное в~толковом (семантическом) словаре, 
есть со\-по\-став\-ле\-ние одному слову некоторой другой группы слов и~их сочетаний.
     
     Даже тот, кто воспринимает слово \textit{халва}, затрудняется сказать, чего 
больше для него в~значении этого слова: ощущения сла\-дости или пяти звуков, 
об\-ра\-зу\-ющих это слово. Непосредственная связь между словом и~его значением 
давно используется в~решении задач лингвистики.

\vspace*{-2pt}
     
\section{Формулировка задачи}

     В \textit{синтаксическом анализе} предложения самым трудоемким является 
процесс определения \textit{морфологических параметров} словоформы. 
Следующей по тру\-до\-ем\-кости процедурой является \textit{сегментация членов} 
предложения. Понятие \textit{сегментации} нуж\-да\-ет\-ся в~уточ\-нении.
     
     Предполагается, что предложение имеет древовидную структуру. 
Например, \textit{пред\-ло\-же\-нию-по\-сле\-до\-ва\-тель\-ности} (словоформ) 

\vspace*{-3pt}

\noindent
     \begin{multline}
     \mbox{\textbf{маленькая дочка соседа пошла на родник}}\\
     \mbox{\textbf{за холодной водой}}
     \label{e1-mir}
     \end{multline}
     

          \noindent
соответствует \textit{пред\-ло\-же\-ние-де\-ре\-во}, пред\-став\-лен\-ное на рис.~1.

  { \begin{center}  %fig1
 \vspace*{9pt}
   \mbox{%
 \epsfxsize=77.957mm 
 \epsfbox{mir-1.eps}
 }


\vspace*{6pt}


\noindent
{{\figurename~1}\ \ \small{Предложение-дерево $T_1$}}
\end{center}
}

\vspace*{9pt}      
          
     Каждая ветвь \textit{корневого дерева}~\cite{2-mir, 4-mir} на рис.~1 
(обозначим его через~$T_1$) также является корневым деревом. Например, ветвь 
с~кор\-нем в~вершине <<\textbf{пошла}>>~--- ветвь ска\-зу\-емо\-го из 
предложения~(\ref{e1-mir})~--- показана на рис.~2. У~дерева несколько вет\-вей, 
вершины вет\-ви пред\-ло\-же\-ния-де\-ре\-ва образуют \textit{сегмент} в~этом 
дереве.
     



       { \begin{center}  %fig2
       \vspace*{9pt}
       
  \mbox{%
 \epsfxsize=47.652mm 
 \epsfbox{mir-2.eps}
 }


\vspace*{6pt}


\noindent
{{\figurename~2}\ \ \small{Ветвь <<\textbf{пошла}>> дерева~$T_1$}}
\end{center}
}


\vspace*{9pt}

\setcounter{figure}{2}
    
     
     Запишем предложение-де\-ре\-во~$T_1$ в~скобочном виде:
     
     \vspace*{-3pt}

\noindent
     \begin{multline}
     \mbox{\textbf{дочка~(маленькая,\ соседа,}}\\
     \mbox{\textbf{пошла~(на\ родник,\ за\ 
водой~(холодной)))}}\,.
     \label{e2-mir}
     \end{multline}
      
     Трансформация двумерной структуры~(\ref{e2-mir}) в~линейную 
цепь~(\ref{e1-mir}) интуитивно выполняется нашей языковой спо\-соб\-ностью. 
В~конструктивной теории\linebreak\vspace*{-12pt}

\pagebreak

\noindent
 языка, в~част\-ности в~компьютерной лингвистике, этот 
процесс должен быть описан таким образом, чтобы иметь воз\-мож\-ность найти 
соответствие между сегментами пред\-ло\-же\-ния-де\-ре\-ва~(\ref{e2-mir}) 
и~\textit{фрагментами}  
пред\-ло\-же\-ния-по\-сле\-до\-ва\-тель\-ности~(\ref{e1-mir}). 
     
     \textit{Фрагментом} в~последовательности~(\ref{e1-mir}) называется любой 
ее диа\-па\-зон~--- непрерывная часть цепочки словоформ~(\ref{e1-mir}). 
     
     \textit{Фрагментация сегментов}, т.\,е.\ трансформация  
\textit{пред\-ло\-же\-ния-де\-ре\-ва}  
в~\textit{пред\-ло\-же\-ние-по\-сле\-до\-ва\-тель\-ность} таким образом, чтобы 
каж\-дый сегмент первого был фрагментом во втором,~--- принципиальной 
важ\-ности задача в~анализе предложения. Решается эта задача несложно~--- 
простым односторонним обходом дерева~\cite{4-mir}. Однако правила 
трансформации, которыми пользуется наша языковая спо\-соб\-ность, час\-тень\-ко 
отступают от пути обхода дерева. Например, левосторонний обход дерева~$T_1$ 
таков:

\vspace*{-6pt}

\noindent
     \begin{multline*}
     \mbox{\textbf{дочка\ маленькая\ соседа\ пошла}}\\
     \mbox{\textbf{за\ водой\ холодной\ на\ 
родник}}
     %\label{e3-mir}
     \end{multline*}
     
\vspace*{-6pt}
     
     Рассматриваемая задача имеет непосредственное отношение к~правилам 
трансформации вершин пред\-ло\-же\-ния-де\-ре\-ва во фрагменты 
пред\-ло\-же\-ния-по\-сле\-до\-ва\-тель\-ности.
     
     Пусть $(v, w)$~--- некоторая дуга в~пред\-ло\-же\-нии-де\-ре\-ве. Известно, что 
слово~$w$ доопределяет (уточняет) значение слова~$v$. Следовательно, и~вся 
ветвь с~корнем в~$w$ присутствует в~предложении исключительно для уточнения 
определенного семантического признака слова~$v$. Признаком слова~$v$ может 
быть и~\textit{вес}, и~\textit{цвет}, и~\textit{местонахождение} и~т.\,д.\ предмета 
с~именем~$v$. Такие сегменты назовем \textit{описателями признаков} вещи 
с~именем~$v$, или \textit{описателями семантических} признаков слова~$v$. 
     
     Также известно, что \textit{описатель семантического} признака всегда 
образует фрагмент в~предложении. Например, в~предложении <<\textbf{Фермер 
отправил} (\textbf{на пять километров двести метров}) (\textbf{в~двух 
контейнерах и~в~семи ящиках}) (\textbf{три тонны пятьсот килограмм}) 
\textbf{зерна}>> в~круглых скобках выделены три выражения, каж\-дое из которых 
описывает один из трех признаков (глагола \textit{отправил} и~имени 
\textit{зерна}). При любом нарушении фрагментарности этих сегментов их 
значения изменятся. 
     
     Используя это свойство предложения, можно разработать алгоритмы для 
выделения в~предложении отдельных его фрагментов~--- описателей семантических 
признаков имен существительных (ИС) и~глаголов, что существенно упростит 
анализ предложения. (Напомним, что не\-уве\-рен\-ность во \textit{фрагментарности 
сегментов} существенно услож\-ня\-ет работу синтаксических анализаторов.) 
     
     Собирать дерево из нескольких его ветвей намного проще, нежели из 
мно\-го\-чис\-лен\-ных его вер-\linebreak\vspace*{-12pt}

\columnbreak

\noindent
шин. (Напомним также, анализ предложения означает: по 
заданному пред\-ло\-же\-нию-по\-сле\-до\-ва\-тель\-ности построить 
со\-от\-вет\-ст\-ву\-ющее ему пред\-ло\-же\-ние-де\-ре\-во~\cite{4-mir}.) 
     
     Для решения поставленной задачи необходимы некоторые определения 
и~обозначения из синтаксиса естественного языка (ЕЯ). В~данном случае в~качестве 
синтаксиса используется синтаксис русского языка.

\vspace*{-6pt}
     
\section{Синтаксис}

     Задачи компьютерной лингвистики предполагают компьютерный 
морфологический анализ словоформ и~синтаксический анализ предложения на 
\textit{естественном языке}. Морфология и~синтаксис письменного языка 
давно рас\-смат\-ри\-ва\-ют\-ся разработчиками синтаксических анализаторов 
и~про\-грамм-пе\-ре\-вод\-чи\-ков как дискретные сис\-те\-мы с~заданными множествами 
элементов и~операций над ними. 
      
     Компьютерный синтаксический словарь (вернее было бы называть его 
морфологическим словарем) является обязательной \mbox{частью} базы данных\linebreak 
в~синтаксических анализаторах. У~этих словарей такое же предназначение, как 
у~обычных словообразовательных словарей: анализ (распознавание) и~синтез 
(построение) словоформ. Построение компьютерного словаря, разработка 
алгоритмов анализа и~синтеза словоформ по данному словарю является 
при\-клад\-ной задачей дискретного анализа~\cite{2-mir}. 
     
     Под анализом словоформы, например <<\textit{ящику}>>, под\-ра\-зу\-ме\-ва\-ет\-ся 
поиск исходной формы этого слова и~\textit{морфологических параметров} его: 
<<\textit{ящик 01~11~21~33}>>, где 01~--- код ИС; 
11~--- код единственного чис\-ла; 33~--- код дательного падежа.
     
     Далее будем пользоваться следующей кодировкой час\-тей речи 
и~морфологических категорий: 01~--- ИС; 02~--- ИП (полная форма имени 
прилагательного); 10, 20 и~30~--- \textit{категории рода}, \textit{числа} 
и~\textit{падежа} соответственно. Значения категорий так\-же пронумерованы:

\vspace*{-3pt}

\noindent
     \begin{multline*}
\mbox{род} = (\mbox{мужской,\ средний,\ женский}) ={}\!\\
    {}= 10 = 
(11, 12, 13)\,; 
\end{multline*}

\vspace*{-13pt}

\noindent
\begin{multline*}
     \mbox{число} = (\mbox{единственное,\ множественное}) = {}\\
     {}=20=  (21, 22)\,; 
    \end{multline*}
    
    \vspace*{-13pt}

\noindent
    \begin{multline*}
     \mbox{падеж} = (\mbox{именительный},
    \mbox{родительный},\\ 
    \mbox{дательный,\ 
винительный,\ творительный},\\
 \mbox{предложный}) = 30 = (31, 32, 33, 34, 35, 36)\,.
 \end{multline*}
     

     
     Нетрудно заметить, что значения категорий суть унарные операции, 
определенные на ИС и~ИП. Действительно,

\pagebreak

\noindent
     \begin{align*}
22(\mbox{\textbf{большой}}) &= \mbox{\textbf{большие}};\\ 
35(\mbox{\textbf{большой}})& = \mbox{\textbf{большим}};\\ 
33(\mbox{\textbf{ящик}}) &= \mbox{\textbf{ящику}}.
\end{align*}


     
     \noindent
     Композиции этих операций образуют новые операции. Например:
     \begin{align*}
     22\cdot35(\mbox{\textbf{большой}}) &= \mbox{\textbf{большими}};\\ 
22\cdot33(\mbox{\textbf{ящик}}) &= \mbox{\textbf{ящикам}}.
     \end{align*}
     

     
     Таким образом, на ИС и~ИП определены по $3\cdot2\cdot6 \hm= 36$ операций, 
в~виде композиции трех операций типа~10, 20 и~30. (Операции~11, 12 и~13 не 
меняют ИС.) Каж\-дая из них однозначно определяет новое значение своего 
операнда. Например:
     \begin{align*}
     13\cdot21\cdot34(\mbox{\textbf{большие}}) &= \mbox{\textbf{большую}};\\ 
13\cdot22\cdot35(\mbox{\textbf{ящик}})& = \mbox{\textbf{ящиками}}.
     \end{align*}
     
     Каждая из этих операций \textit{образует} (строит, синтезирует) одну 
определенную словоформу.
     
     Каждая из этих операций имеет обратную себе операцию: по заданной 
словоформе~$w$ определяется ее исходная форма~$w_1$ и~операция~$f$ такая, что 
$f(w_1) \hm= w$. Например, для $w \hm= \mbox{\textbf{ящиками}}$ имеем $w_1 
\hm= \mbox{\textbf{ящик}}$, $f \hm= 13\cdot22\cdot35$.
     
     Действие, обратное \textit{синтезу} (построению) словоформы, называется 
\textit{анализом} (распознаванием) словоформы.
     
     Если $f(w_1) = w$, то три операции, обра\-зу\-ющие~$f$, называются 
\textit{морфологическими параметрами} словоформы~$w$. Например, для 
словоформы \textit{ящиками} морфологическими па\-ра\-мет\-ра\-ми являются~13, 22 
и~35. Иногда (для убедительности) за словоформой пишутся ее 
\textit{па\-ра\-мет\-ры}. Например, \textbf{ящиками}~13~22~35.
     
     Область значений операции~$f$ называется \textit{лексической 
группой}~$f$. Например, 01~13~22~35~--- под\-мно\-же\-ст\-во ИС в~форме~13~22~35; 
ИС делится на~12~таких лексических групп. 
     
     Над лексическими группами можно выполнить множественные операции, 
например операцию объединения: 
01~13~20~35\;=\;01~13~21~35\;$\cup$\;01~13~22~35, пред\-став\-ля\-ющую собой 
множество ИС женского рода в~творительном падеже в~единственном 
и~множественном чис\-ле, так как $20 =\hm (21, 22)$. 
     
     Аналогично ИП делятся на~36~лексических групп. 
     
     Введенные обозначения 
можно использовать для формирования более сложных обозначений (формул), 
порождающих определенные группы элементов синтаксиса. Например, 
минимальная неделимая единица синтаксиса~--- \textit{словосоче\-тание}. 
Примером словосочетания является \textit{согла\-со\-ван\-ное определение} 
\textbf{ящиками нижними} или пара\linebreak (\textit{ящиками, нижними}), которая, 
в~свою очередь, является элементом прямого произведения 
$(01~13~22~35)\cdot(02~13~22~35)$. Таких сочетаний ИС с~ИП в~русском языке будут 
десятки тысяч. Резонно обозначить множество этих словосочетаний через 
(01~13~22~35)$\cdot$(02~13~22~35) и~называть его кодом \textit{синтаксического 
отношения} (СО) между группами 01132235 и~02132235. Код лексической группы 
начинается с~нуля, пробелы меж\-ду кодами категорий и~знак~<<$\cdot$>> 
необязательны: 0113223502132235. 
     
     Обозначив значения категорий буквами, получим все~36~СО: 
     \begin{equation}
     01~1x~2y~3z~02~1x~2y~3z\,,
          \label{e4-mir}
          \end{equation}
где  
$x= 1\ldots 3$; $y = 1\ldots 2$;  $z = 1\ldots 6$,
порождающих \textit{согласованные определения}, име\-ющие место 
в~русском языке. 
     
     Элементы множеств~(\ref{e4-mir}) обозначим через СО1. 
Все~36~записей~(\ref{e4-mir}) упорядочим лексикографически по возрастанию 
и~обозначим через СО1001, СО1002, \ldots, СО1036.
     
     Аналогично $3\cdot2\cdot6\cdot3\cdot2 = 216$~пар 
     \begin{equation}
     01~1x~2y~3z~01~1r~2t~32\,,
          \label{e5-mir}
          \end{equation}
где $x, r = 1\ldots 3$; $y, t = 1\ldots 2$;  $z = 1\ldots 6$,
     порождают словосочетания (\textit{несогласованные определения} или СО2) 
вида: \textbf{ящиками фермеров}, \textbf{книгу биб\-лио\-те\-ки}, \textbf{заботы 
врачей} и~т.\,д.
     
     В~(\ref{e5-mir}) присутствуют пять па\-ра\-мет\-ров. Согласно грамматике 
русского языка все первые пять категорий, кроме последней категории~<<32>>, 
могут принимать всевозможные значения независимо друг от друга. Объединение 
всех~216~пар~(\ref{e5-mir}) удобно писать в~виде: 
     \begin{equation}
     01~10~20~30~01~10~20~32\,.
     \label{e6-mir}
     \end{equation}
     
     Заменив в~(\ref{e6-mir}) нули их допустимыми значениями и~упорядочив 
полученные записи в~лексикографическом порядке возрастания, 
получим~216~записей с~кодами СО2001, СО2002, \ldots, СО2216.
     
     В русском языке более~50~тыс.\ ИС и~ИП. Словосочетаний из 
\textit{согласованных определений} будет около миллиона. А~словосочетаний из 
\textit{согласованных и~несогласованных определений} окажется десятки 
миллионов. На ЕЯ самым распространенным 
и~мно\-го\-чис\-лен\-ным видом выражений являются сочетания из ИС и~ИП, 
образованных по правилам~СО1 и~СО2. 
     
     Правила СО1 позволяют привязывать к~ИС несколько ИП, например:
         \begin{equation}
     \mbox{\textbf{хороший\ высокий\ умный\ студент.}}
     \label{e7-mir}
     \end{equation}
     
                 
          \noindent
     (Знаков препинания в~рас\-смат\-ри\-ва\-емом здесь синтаксисе нет.) 
Выражению~(\ref{e7-mir}) соответствует дерево в~форме звезды (рис.~3).

  { \begin{center}  %fig3
 \vspace*{1pt}
  \mbox{%
 \epsfxsize=77.957mm 
 \epsfbox{mir-3.eps}
 }


\vspace*{6pt}


\noindent
{{\figurename~3}\ \ \small{Предложение-дерево. Звезда}}
\end{center}
}

\vspace*{9pt}      




 Или в~ско\-боч\-ной записи:
\begin{equation}
\mbox{\textbf{студент\ (хороший,\ высокий,\ умный)}}\,.
\label{e8-mir}
\end{equation}  
     
     \noindent
\textit{Шириной} выражения с~одним ИС называется чис\-ло ИП в~этом выражении. 
Выражение~(\ref{e8-mir}) имеет ширину три.
     
     
     

     По правилам СО2 образуется линейная цепь связанных ИС, например:
          \begin{equation}
     \mbox{\textbf{отряд\ студентов\ школы\ мужества}}\,.           
     \label{e9-mir}
     \end{equation}
     

     
     Выражению~(\ref{e9-mir}) соответствует цепь (которая на восточных 
языках называется \textit{идафной цепью}; \textit{ид$\acute{\mbox{а}}$фа} 
(\textit{араб.}\hspace*{-1pt}{\raisebox{-4pt}{
\epsfxsize=9mm %8.981mm 
\epsfbox{mir-t.eps}
}}
 \textit{дополнение})~--- способ оформления 
несогласованного определения в~арабском и~ряде других восточных языков), 
пред\-став\-лен\-ная на рис.~4.

  { \begin{center}  %fig4
 \vspace*{9pt}
  \mbox{%
 \epsfxsize=78.036mm 
 \epsfbox{mir-4.eps}
 }


\vspace*{6pt}


\noindent
{{\figurename~4}\ \ \small{Предложение-дерево. Цепь}}
\end{center}
}

\vspace*{9pt} 


     
Или в~скобочной записи: 
\begin{equation*}
     \mbox{\textbf{отряд\ (студентов\ (школы\ (мужества)))}}\,.
     %\label{e10-mir}
     \end{equation*}
     
     
     \textit{Длиной} идафной цепи называется число ИС в~этой цепи.
     
     Далее для определенности и~обозримости выражений вводятся ограничения 
на ширину \textit{согласованного определения}~--- она не более одного, и~на 
длину \textit{несогласованного определения}~--- она не более трех.
     
     В грамматике русского языка кроме~12~\textit{беспредложных форм} ИС 
используются еще \textit{предложные формы} ИС. Что\-бы разнообразить 
рас\-смат\-ри\-ва\-емый синтаксис, в~лексику из ИС и~ИП добавляются сле\-ду\-ющие 
предлоги: \textit{над}, \textit{под}, \textit{на}, \textit{в}~--- и~соответствующие им 
\textit{пред\-лож\-ные формы} ИС: первые два предлога управ\-ля\-ют творительным 
падежом, а~последние два~--- предложным.
     
     Если предположить, что эти предлоги имеют коды~21, 22, 23 и~34, то коды 
\textit{предложных форм}, со\-от\-вет\-ст\-ву\-ющих этим предлогам, будут такими: 
01~10~20~35~21, 01~10~20~35~22, 01~10~20~36~23 и~01~10~20~36~24. 
Например:
  \begin{align*}
     0110~20~35~21(\mbox{\textbf{отряд}}) &= \mbox{\textbf{над\ отрядом}};\\ 
01~10~20~36~24(\mbox{\textbf{отряд}}) &= \mbox{\textbf{в~отряде}}\,.
\end{align*}

     
     Обозначим через~$S$ множество ИС и~ИП в~лексике русского языка и~их 
формы, о которых было сказано выше. Обозначим через~$Q$ множество 
\textit{согласованных} и~\textit{несогласованных определений}, определенных 
на~$S$.
     
     Пара ($S, Q$) представляет собой ориентированный граф, обозначим его 
через~$\mathrm{Sint}$. Имеются сле\-ду\-ющие основания называть этот граф 
\textit{синтаксисом}. 
     
     Представим себе \textit{корневое дерево} в~этом графе: 
     \begin{itemize}
\item[(а)] у которого корнем является ИС; 
\item[(б)] каждое ИС в~нем имеет ширину не более одного; 
\item[(в)] идафная цепь в~нем имеет длину не более трех.
\end{itemize}

     Деревья, удовлетворяющие требованиям~(а)--(в), назовем 
\textit{выражениями} в~синтаксисе $\mathrm{Sint}$. (Требова\-ния~(б) и~(в) здесь 
присутствуют для на\-гляд\-ности деревьев и~\textit{выражений}.) Примерами 
\textit{выражений} являются:
\textbf{дом}; \textbf{под\ домом}; \textbf{дом\ белый}; \textbf{под\ домом\ белым};
\textbf{в\ доме\ соседа}; \textbf{в\ доме\ высоком\ соседа\ хорошего\ врача\ зубного}.
        
              
     Последнее выражение из шести слов является самым длинным в~синтаксисе 
$\mathrm{Sint}$. (Переставляя ИС и~следующее за ним ИП, получим привычные для 
русского языка выражения.)
     
     Данное определение \textit{формального синтаксиса} никак не зависит от 
объема лексики и~числа СО, на которых построен \textit{синтаксис}. 
     
     Этот синтаксис не обременен семантикой. Вопрос инкапсуляции синтаксиса 
семантикой здесь не рассматривается. Хотя известна сила синтаксиса вдохнуть 
смысл даже в~то, как <<Глокая куздра штеко будланула бокра и~курдячит 
бокренка>>. Поэтому выражения (деревья) в~синтаксисе $\mathrm{Sint}$, начинающиеся 
с~\textit{предложной формы} имен, обозначающих предметы, будут отвечать на 
вопрос <<\textit{где?>>}, и~назовем их \textit{указателями места}. 
     
      Элементами синтаксиса $\mathrm{Sint}$ являются обычные (синтаксически 
правильно по\-стро\-ен\-ные) выражения на русском языке, и~задача заключается 
в~поиске \textit{указателей места} в~этом синтаксисе.
     
     Алгоритм извлечения \textit{указателей места} из заданного текста очень 
прост. Имеет смысл рассмотреть работу алгоритма по шагам на прос\-том примере:
     \begin{equation}
         \hspace*{-2.5mm}\mbox{\textbf{в\ высоком\ доме\ хорошего\ соседа\ зубного\ 
врача}}.\!\!
         \label{e11-mir}
         \end{equation}
         
         
         \noindent
         \begin{description}
\item[Шаг~1.] Выделяются словоформы в~(\ref{e11-mir}) и~определяются 
коды лексических групп, к~которым они принадлежат:
\begin{alignat*}{2}
     \mbox{\textbf{в\ доме}}\mbox{~---}&\ 01213624; 
&\enskip \mbox{\textbf{высоком}}\mbox{~---}&\  022136;\\
     \mbox{\textbf{соседа}}\mbox{~---}&\ 012132;  &\enskip 
\mbox{\textbf{хорошего}}\mbox{~---}&\ 022132;\\
     \mbox{\textbf{врача}}\mbox{~---}&\  012132; &\enskip 
\mbox{\textbf{зубного}}\mbox{~---}&\ 022132.
     \end{alignat*}

     

     
    \item[Шаг~2.] Коды лексических групп позволяют группировать попарно 
словоформы из~(\ref{e11-mir}) таким образом, чтобы они были отношениями 
\textit{согласованного} и~\textit{несогласованного определений}; полученные таким 
образом пары словоформ образуют дерево, представленное на рис.~5.
\end{description}

{ \begin{center}  %fig5
 \vspace*{9pt}
  \mbox{%
 \epsfxsize=68.512mm 
 \epsfbox{mir-5.eps}
 }


\end{center}


\noindent
{{\figurename~5}\ \ \small{Дерево, порожденное синтаксисом $\mathrm{Sint}$ на словоформах~(\ref{e11-mir}}}
}

\vspace*{9pt}      

     
     
     
 \begin{description}    
     \item[Шаг~3.] После замены словоформ в~дереве с~рис.~5 на коды лексических 
групп, к~которым они принадлежат, получится дерево, показанное на рис.~6.
     

     
     Дерево на рис.~6 называется \textit{синтаксической формой} (СФ) дерева 
с~рис.~5, а~следовательно, и~выражения~(\ref{e11-mir}). Она является СФ для 
большого чис\-ла выражений; действительно, заменяя в~этой СФ код каждой 
лексической группы некоторой словоформой из этой же группы, можно получить 
новые выражения. Так что СФ~--- \textit{правило}, порождающее выражения 
определенной формы. Из сказанного следует шаг~4. (Сформулирован он в~форме 
утверж\-де\-ния.) 
\end{description}

{ \begin{center}  %fig6
 \vspace*{9pt}
  \mbox{%
 \epsfxsize=68.512mm 
 \epsfbox{mir-6.eps}
 }


\end{center}


\noindent
{{\figurename~6}\ \ \small{Синтаксическая форма дерева, представленного на рис.~5}}
}
    
    \vspace*{6pt}
     
     
\begin{description}
     \item[Шаг~4.] При\-над\-леж\-ность выражения~(\ref{e11-mir}) к~СФ с~рис.~6 
указывает на то, что~(\ref{e11-mir}) есть \textit{указатель места}. 
\end{description}
     
     Рассмотрим еще несколько примеров \textit{указателей места} из шес\-ти 
слов. Все они имеют одинаковую структуру:
\textbf{в~большом ауле нижнего района нашей 
области}; \textbf{под нижним ящиком письменного стола
большого  начальника};
\textbf{на длинном хвосте серого слона бедного индуса}.

      

     
     Анализируя эти выражения по тому же алгоритму из четырех шагов, можно 
убедиться в~том, что и~они~--- \textit{указатели места}. 
     
     Обобщая СФ всевозможных \textit{указателей места}, нетрудно заметить, 
что СФ, по\-рож\-да\-ющие \textit{указатели места}, имеют форму дерева, 
пред\-став\-лен\-но\-го на рис.~7.

{ \begin{center}  %fig7
 \vspace*{9pt}
  \mbox{%
 \epsfxsize=68.512mm 
 \epsfbox{mir-7.eps}
 }


\end{center}


\noindent
{{\figurename~7}\ \ \small{Формы указателей места, объединенные в~дереве $T_{\mathrm{Space}}$}}
}
    
    \vspace*{12pt}

    
     
     Корню этого дерева, обозначенного $T_{\mathrm{Space}}$, соответствует код 
предложной формы ИС $011x2y3z\,\mathrm{MN}$, где $\mathrm{MN}$~--- код предлога, управ\-ля\-юще\-го 
падежной формой~$3z$.
     
     Также нетрудно заметить, что любое поддерево дерева $T_{\mathrm{Space}}$ с~корнем 
в~$011x2y3z\,\mathrm{MN}$ есть СФ, порождающая \textit{указатели места} в~синтаксисе 
$\mathrm{Sint}$. 
     
     Более того, эти СФ порождают все\-воз\-мож\-ные \textit{указатели места} 
в~синтаксисе $\mathrm{Sint}$. 
     
\section{Заключение}

Авторы оперировали синтаксисом $\mathrm{Sint}$~--- вырезкой из синтаксиса 
русского языка~--- для наглядной доступности задачи и~алгоритма ее решения. 
Задача и~алгоритм ее решения не претерпят существенных изменений, если их 
сформулировать для всего синтаксиса русского языка. Изменятся размеры 
словаря и~синтаксиса (графа $\mathrm{Sint}$). Для быстрого поиска слов в~словаре 
(в~упорядоченном массиве строк) можно использовать известные ал\-го\-ритмы.
     
     Приведенное выше описание алгоритма выявления \textit{указателей 
места} в~тексте, написанном в~синтаксисе $\mathrm{Sint}$, указывает на прос\-то\-ту 
и~корректность ал\-го\-рит\-ма. Листинг программы, реализующей этот алгоритм, 
занимает не более одной страницы. Линейная слож\-ность алгоритма обеспечивает 
ему мгновенное исполнение на компь\-ютере.
{\looseness=1

}
     
     Программа, анализирующая предложения в~упрощенном синтаксисе 
и~с~лексикой из~16~слов (синтаксический анализатор), под\-го\-тов\-ле\-на авторами 
и~реализована в~IDE Delphi~7.
     
    {\small\frenchspacing
 {%\baselineskip=10.8pt
 \addcontentsline{toc}{section}{References}
 \begin{thebibliography}{9}
\bibitem{1-mir}
\Au{Барсегян А.\,А., Куприянов~М.\,С., Степаненко~В.\,В., Холод~И.\,И.} Технологии анализа 
данных: Data Mining, Visual Mining, Text Mining, OLAP.~--- СПб.: БХВ-Петербург, 2008. 384~с.
\bibitem{2-mir}
\Au{Шихиев Ф.\,Ш.} Формализация и~сетевая формулировка задачи синтаксического анализа. 
Дис.\ \ldots\ канд. физ.-мат. наук.~--- СПб.: СпбГУ, 2006. 171~с.
\bibitem{3-mir}
\Au{Рубашкин В.\,Ш., Чуприн~Б.\,Ю.} Распознавание количественной информации  
в~ЕЯ-текс\-тах~// Компьютерная лингвистика и~интеллектуальные технологии: Тр. Междунар. 
конф. <<Диалог-2006>>.~--- М.: РГГУ, 2006. С.~456--467.
\bibitem{4-mir}
\Au{Харари~Ф.} Теория графов~/ Пер. с~англ.~--- М.: Едиториал УРСС, 2003. 296~с.
(\Au{Harary~F.} {Graph theory}.~--- Boulder, CO, USA: Westview Press, 1994. 284~p.)
 \end{thebibliography}

 }
 }

\end{multicols}

\vspace*{-6pt}

\hfill{\small\textit{Поступила в~редакцию 13.04.17}}

\vspace*{6pt}

%\newpage

%\vspace*{-24pt}

\hrule

\vspace*{2pt}

\hrule

%\vspace*{8pt}


\def\tit{DISCRETE ANALYSIS IN~PARSING}

\def\titkol{Discrete analysis in~parsing}

\def\aut{Ya.\,M.~Mirzabekov and Sh.\,B.~Shihiev}

\def\autkol{Ya.\,M.~Mirzabekov and Sh.\,B.~Shihiev}

\titel{\tit}{\aut}{\autkol}{\titkol}

\vspace*{-9pt}


 \noindent
Dagestan State University, 43-a~Gadzhiyev Str., Makhachkala 367000, Republic 
of Dagestan, Russian Federation


\def\leftfootline{\small{\textbf{\thepage}
\hfill INFORMATIKA I EE PRIMENENIYA~--- INFORMATICS AND
APPLICATIONS\ \ \ 2018\ \ \ volume~12\ \ \ issue\ 2}
}%
 \def\rightfootline{\small{INFORMATIKA I EE PRIMENENIYA~---
INFORMATICS AND APPLICATIONS\ \ \ 2018\ \ \ volume~12\ \ \ issue\ 2
\hfill \textbf{\thepage}}}

\vspace*{3pt}
      
     
     
     
     \Abste{An informal definition 
     of syntax is given in terms of discrete mathematics and graph theory. 
     The main difficulty for numerous attempts to formalize a~natural language 
     is the semantics of the language. It is shown how it is possible to classify 
     expressions on a~semantic basis, using their syntactic features. 
     Classification of expressions by the kind of questions they answer is the 
     simplest way of grouping expressions on the semantic basis. 
The present authors describe an algorithm that recognizes 
     place pointers, that is, expressions  which answer the question ``where?''. 
     On specific examples, the problem of analysis of expressions and the inverse 
     problem of synthesis, more precisely, 
     the applied problem of recognition of expressions are considered.}
     
     \KWE{natural language; discrete mathematics; graph theory; syntax; 
     word forms; morphological parameters; consistent definitions; 
     inconsistent definitions; vocabulary; semantics}

\DOI{10.14357/19922264180214} %

%\vspace*{-14pt}

 % \Ack
  % \noindent
 



\vspace*{3pt}

  \begin{multicols}{2}

\renewcommand{\bibname}{\protect\rmfamily References}
%\renewcommand{\bibname}{\large\protect\rm References}

{\small\frenchspacing
 {%\baselineskip=10.8pt
 \addcontentsline{toc}{section}{References}
 \begin{thebibliography}{9}
     
    
\bibitem{1-mir-1}
\Aue{Barsegyan, A.\,A., M.\,S.~Kupriyanov, V.\,V.~Stepanenko, and I.\,I.~Cholod}. 
2007. \textit{Tekhnologiya analiza dannykh: Data Mining, Visual Mining, Text 
Mining, OLAP} [Technology of data analysis: Data Mining, Visual Mining, Text 
Mining, OLAP]. St.\ Petersburg: BHV. 384~p.
\bibitem{2-mir-1}
\Aue{Shikhiev, F.\,Sh.} 2006. Formalizatsiya i~setevaya formulirovka zadachi 
sintaksicheskogo analiza [Formalization and network formulation of the task of 
parsing].  
St.\ Petersburg: SpbGU. PhD Diss. 171~p.
\bibitem{3-mir-1}
\Aue{Rubashkin, V.\,Sh., and B.\,Y.~Chuprin.} 2006. Ras\-po\-zna\-va\-nie kolichestvennoy 
informatsii v~EYa-tekstakh [Quantitative data recognition at NLP]. Computational 
Linguistics and Intellectual Technologies: Conference (International) ``Dialogue-2006'' 
Proceedings. Moscow: RGGU. 456--467.
\bibitem{4-mir-1}
\Aue{Harary, F.} 1994. \textit{Graph theory}. Boulder, CO: Westview Press. 284~p.
\end{thebibliography}

 }
 }

\end{multicols}

\vspace*{-3pt}

\hfill{\small\textit{Received April 13, 2017}}

%\vspace*{-24pt}


      \Contr
      
      \noindent
     \textbf{Mirzabekov Yahya M.} (b.\ 1983)~--- senior lecturer, Dagestan State 
University, 43-a Gadzhiyev Str., Makhachkala 367000, Republic of Dagestan, Russian 
Federation; \mbox{yash831@mail.ru}
      
      \vspace*{3pt}
      
      \noindent
      \textbf{Shihiev Shukur B.} (b.\ 1951)~--- Candidate of Sciences (PhD) in physics 
and mathematics, associate professor, Dagestan State University, 43-a~Gadzhiyev Str., 
Makhachkala 367000, Republic of Dagestan, Russian Federation; 
\mbox{sh\_sh\_b51@mail.ru}
\label{end\stat}


\renewcommand{\bibname}{\protect\rm Литература}  %14
\def\stat{nuriev}

\def\tit{МЕТОДОЛОГИЯ КОРПУСНО-ОРИЕНТИРОВАННОГО ИССЛЕДОВАНИЯ 
В~ОБЛАСТИ КОНТРАСТИВНОЙ ПУНКТУАЦИИ$^*$\\[-5pt]}

\def\titkol{Методология корпусно-ориентированного исследования 
в~области контрастивной пунктуации}

\def\aut{В.\,А.~Нуриев$^1$, В.\,И.~Карпов$^2$}

\def\autkol{В.\,А.~Нуриев, В.\,И.~Карпов}

\titel{\tit}{\aut}{\autkol}{\titkol}

\index{Нуриев В.\,А.}
\index{Карпов В.\,И.}
\index{Nuriev V.\,A.}
\index{Karpov V.\,I.}


{\renewcommand{\thefootnote}{\fnsymbol{footnote}} \footnotetext[1]
{Работа выполнена за счет гранта Российского научного фонда (проект 23-28-00548) с~использованием инфраструктуры 
Центра коллективного пользования <<Высокопроизводительные вычисления и~большие данные>> (ЦКП 
<<Информатика>>) ФИЦ ИУ РАН (г.~Москва).}}


\renewcommand{\thefootnote}{\arabic{footnote}}
\footnotetext[1]{Федеральный исследовательский центр <<Информатика и~управление>> Российской академии наук, 
\mbox{nurieff.v@gmail.com}}
\footnotetext[2]{Институт языкознания Российской академии наук; Федеральный исследовательский центр <<Информатика 
и~управ\-ле\-ние>> Российской академии наук, \mbox{wi.karpow@gmail.com}}

\vspace*{-3pt}

  
  
    
  \Abst{Уточняется  подход к~современным исследованиям 
в~об\-ласти контрастивной пунктуации с~точки зрения методологии. С~учетом новейших достижений информатики, 
компьютерной лингвистики и~теории перевода такие исследования очевидным образом 
должны иметь кор\-пус\-но-ори\-ен\-ти\-ро\-ван\-ный характер. В~данной статье представлена 
методологическая схема подобного исследования, направленного на выявление 
межъязыковой пунктуационной асим\-мет\-рии посредством сравнения функционального 
диапазона одного и~того же знака препинания в~разных языках. Показываются основные 
методологические тенденции, характерные для этой научной об\-ласти. Внимание 
уделяется особенностям корпусной методологии при контрастивном изучении 
пунктуации. В~качестве одного из современных методологических инструментов 
предлагаются надкорпусные базы данных (НБД), раз\-ра\-ба\-ты\-ва\-емые в~ФИЦ ИУ РАН.}

%\vspace*{-6pt}
  
  \KW{контрастивная пунктуация; перевод; корпусное переводоведение; кор\-пус\-но-ори\-ен\-ти\-ро\-ван\-ное 
  исследование; параллельный корпус; надкорпусная база данных; 
межъязыковая асим\-мет\-рия; методология}

%\vspace*{-6pt}

\DOI{10.14357/19922264230213}{VBOZAO} 
  
%\vspace*{-3pt}


\vskip 10pt plus 9pt minus 6pt

\thispagestyle{headings}

\begin{multicols}{2}

\label{st\stat}
    
    \section{Введение}
    
    \vspace*{-3pt}
    
  Важность и~необходимость исследований в~области контрастивной 
пунктуации в~научной литературе отмечалась неоднократно (см., 
например,~[1--7]). Обычно эта необходимость выводится из нужд 
переводческой практики, которая предполагает при обработке письменного 
текста обязательную речемыслительную программу, связанную с~исходным 
пунктуационным компонентом и~его переносом в~сис\-те\-му переводящего 
языка. Так, Ньюмарк в~своем <<Учебнике перевода>> пишет, что 
<<пунктуация может быть мощнейшим инструментом, но ее настолько легко 
упус\-тить из виду, что я~советую переводчикам: специально сравнивайте, где 
у~вас рас\-став\-ле\-ны знаки препинания, а~где они стоят 
в~оригинале>>~\cite[с.~58]{4-nu}. В~работе <<Переводчик в~текс\-те: 
о~чтении русской литературы  
по-анг\-лий\-ски>> значение пунктуации отмечает Мей, критикуя 
англоязычных переводчиков за недостаточное внимание к~межъязыковой 
пунктуационной асим\-мет\-рии~--- за <<игнорирование отличительных 
особенностей, присущих знакам препинания>>~\cite[с.~121]{2-nu}. 
О~пунктуации в~переводе говорит Юдейл, выделяя три аспекта:
%\begin{enumerate}[(1)]
%\item 
(1)~<<знаки препинания~--- важ\-ная часть перевода, но, концентрируясь на 
общем смыс\-ле переводимого, ее час\-то не замечают>>; 
%\item 
(2)~<<изменения 
в~пунктуации при переводе могут значительно по\-вли\-ять на вы\-ра\-зи\-тель\-ность 
текс\-та, его свя\-зан\-ность и~ритм>>; 
%\item 
(3)~<<час\-то возникает впечатление, что 
литературные переводчики наделили себя правом менять границы исходного 
предложения и~пунктуационные знаки, как им 
заблагорассудится>>~\cite[с.~121]{5-nu}.
%\end{enumerate}
 Гораздо реже 
в~специализированной литературе подчеркивается роль, которую 
исследования в~об\-ласти контрастивной пунктуации играют при обучении 
иностранным языкам, в~част\-ности при обуче\-нии иноязычной письменной 
речи~\cite{7-nu}.
  
  Признавая безусловную зна\-чи\-мость данного научного на\-прав\-ле\-ния и~его 
дальнейшего развития, необходимо предметно разрабатывать методологию 
исследования в~об\-ласти контрастивной пунктуации, которая учитывала бы 
новейшие достижения информатики, компьютерной лингвистики 
и~корпусного переводоведения. Пред\-став\-ля\-ет\-ся, что такая методология 
долж\-на основываться на использовании современных информационных 
корпусных инструментов, поз\-во\-ля\-ющих автоматизированным образом 
обрабатывать пред\-ста\-ви\-тель\-ные массивы текс\-то\-вых данных, 
и,~следовательно, носить кор\-пус\-но-ори\-ен\-ти\-ро\-ван\-ный характер 
(о~корпусных данных при контрастивном изуче\-нии пунктуации  
см.~\cite{6-nu}).

%\vspace*{-6pt}
    
    \section{Методологические модели  
корпусно-ориентированного исследования контрастивной 
пунктуации}

\vspace*{-3pt}
  
  В мае 2019~г.\ в~Регенсбурге (Германия) про\-шла научная конференция 
под названием <<Punctuation Seen Internationally. System--Norm--Practice>> 
(<<Пунктуация в~мировом мас\-шта\-бе: 
 сис\-те\-ма--нор\-ма--прак\-ти\-ка>>)~--- первая конференция, пол\-ностью\linebreak 
по\-свя\-щен\-ная проб\-ле\-мам контрастивной пунктуации. Оргкомитет, собирая 
заявки на участие, справедливо отмечал, что до на\-сто\-яще\-го времени 
пунктуации едва ли уделялось внимание в~рамках \mbox{типологии}, контрастивной 
лингвистики, прагмалингвистики, а~так\-же в~исследованиях индивидуальной 
языковой манеры на фоне языкового стандарта. Сейчас появляются 
отдельные работы, где проводится сопоставительное изуче\-ние пунктуации, 
однако по-преж\-не\-му ощущается острая не\-об\-хо\-ди\-мость в~исследованиях по 
контрастивной пунктуации, которые бы учитывали типологические 
(сис\-тем\-ные), социолингвистические (нормативные) и~прагматические 
(речевые) ас\-пекты.
  
  Итогом конференции стала коллективная монография~\cite{8-nu}, 
со\-сто\-ящая из шестнадцати статей, которые пред\-став\-ля\-ют собой пио\-нер\-ские 
исследования, на\-прав\-лен\-ные на формирование целостной па\-ра\-диг\-мы 
контрастивного изучения пунктуации и~борьбу с~маргинализацией важ\-ной 
научной от\-расли. Все статьи услов\-но мож\-но разделить на~4~категории, 
первые две из которых имеют в~большей степени тео\-ре\-ти\-че\-ский характер и~связаны с~сис\-те\-мой и~нормой, а~вторые~--- более практической 
на\-прав\-лен\-ности~--- с~узусом и~освоением пунктуационных навыков. 
В~пред\-став\-лен\-ных работах доминируют два подхода к~исследованию 
конт\-растив\-ной пунк\-ту\-ации:
  \begin{enumerate}[(1)]
\item интралингвистический (контрастивный анализ знаков препинания 
и~кон\-ку\-ри\-ру\-ющих с~ними маркеров синтаксических отношений в~рамках 
одного языка)~\cite[с.~110]{9-nu};
  \item  интерлингвистический (контрастивный анализ знаков препинания 
  и~конкурирующих с~ними средств в~разных языках, конт\-растив\-ная пунктуация 
рас\-смат\-ри\-ва\-ет\-ся в~том чис\-ле как часть методики обуче\-ния неродному языку, 
например при интеграции трудовых мигрантов в~иноязычную 
среду)~\cite[с.~57--73]{10-nu}.
  \end{enumerate}
  
  Интралингвистический подход час\-то носит смешанный характер: если 
речь идет об эволюции пунктуационной сис\-те\-мы отдельно взятого языка на 
фоне развития аналогичных сис\-тем других языков, контрастивный анализ 
со\-про\-вож\-да\-ет\-ся 
 ис\-то\-ри\-ко-эти\-мо\-ло\-ги\-че\-ским~\cite[с.~187--206]{11-nu}. В~рамках 
этого подхода в~указанной монографии имеются психолингвистические 
исследования с~нетривиальным корпусным материалом. Так, 
в~статье~\cite[с.~163--186]{12-nu} корпусные данные привлекаются для 
контрастивного анализа пунктуационных предпочтений двух групп 
ис\-пы\-ту\-емых. Автор использует корпус \mbox{CoPaDocs} (Corpus of Patient 
Documents), основу которого со\-ста\-ви\-ли письма и~другие личные документы 
бывших пациентов психиатрических учреж\-де\-ний Германии на рубеже  
XIX--XX~вв. Корпус поз\-во\-ля\-ет установить, зависит ли языковое оформление 
пись\-ма от лич\-ности адресата~--- происходит ли переключение ре\-гист\-ров 
сознательно. Данный корпус создан с~целью разработки интегративной 
методики анализа языковой ва\-риа\-тивн\-ости, в~том чис\-ле и~в~об\-ласти 
пунктуации. \mbox{Изучив} специфику расстановки~12~знаков препинания, 
Эбер-Хам\-мерль приходит к~выводу, что пациенты, чей род де\-я\-тель\-ности 
прежде не был связан с~письменной сферой, использовали больше 
пунктуационных маркеров (но с~меньшей ва\-риа\-тив\-ностью), чем 
представители второй опытной группы~--- канцелярские служащие. 
В~лич\-ной переписке участники обеих групп к~знакам препинания прибегали 
гораздо реже, чем в~документах, адресованных официальным лицам.
  
 В статье~\cite[с.~57--73]{10-nu} представлено контрастивное исследование, выполненное в~интерлигвистическом 
ключе. Со\-по\-став\-ле\-ние 
пунктуации в~италь\-ян\-ском и~немецком языках здесь проводится на основе 
комплексной методологии, вклю\-ча\-ющей приемы дескриптивного, 
просодического, синтаксического и~ком\-му\-ни\-ка\-тив\-но-текс\-то\-во\-го 
анализа. Примеры приводятся из различных источников, причем 
к~корпусным данным в~статье отсылают не напрямую, а~опосредованно~--- 
через более раннюю работу~\cite{13-nu}. По мнению авторов, 
пунктуирование в~этих языках организовано по-раз\-но\-му, что объясняется 
резкими различиями в~пунктуационном узусе: если в~итальянском знаки 
препинания коммуникативно на\-гру\-же\-ны, то в~немецком они подчинены 
фор\-маль\-но-син\-так\-си\-че\-ско\-му принципу. Иначе говоря, итальянская 
пунктуация выполняет не формальную функцию, а~сигнализиру-\linebreak ет о~тон\-ких 
смыс\-ло\-вых нюансах, которых нельзя\linebreak достичь другими языковыми 
средствами (аргументативный конфликт, полифонические эффекты, 
метатекстовые комментарии). В~этом же духе\linebreak выполнена и~другая 
интерлингвистическая работа~\cite{14-nu}, по\-свя\-щен\-ная контрастивному 
исследованию многоточия и~тире в~италь\-ян\-ском и~анг\-лий\-ском языках 
и~продуктивно ис\-поль\-зу\-ющая \mbox{корпусный} метод сбора и~обработки 
эмпирических данных.
     
     Объединенные в~коллективную монографию рабо\-ты позволяют 
вывести обобщенную ме\-то\-до\-ло\-гическую схему контрастивного изуче\-ния 
пунктуации. Она имеет трехфазную структуру. Первая\linebreak фаза включает 
тео\-ре\-ти\-че\-ское описание пунктуации в~изуча\-емом языке с~привлечением 
исторических и~современных нормативных грамматик и~справочников. 
Вторая фаза на\-прав\-ле\-на на описание трансформаций в~других языках, 
оказавших существенное влияние на статус и~мес\-то пунктуации в~сис\-те\-ме 
конкретного языка. Обе фазы нацелены на создание такого 
исследовательского поля, которое поз\-во\-лит выявить значение пунктуации 
для языковой культуры. Это, в~свою очередь, долж\-но стать задачей треть\-ей 
фазы. Вторая и~\mbox{третья} фазы предполагают межъязыковое сравнение как 
функционального диапазона отдельно взятых знаков препинания, так 
и~пунктуационного репертуара в~целом. На этих стадиях применяется 
корпусный метод. Контрастивный анализ в~за\-ви\-си\-мости от по\-став\-лен\-ных 
целей и~задач наряду со знаками препинания может охватывать 
и~кон\-ку\-ри\-ру\-ющие с~ними языковые средства. На\-прав\-ле\-ние контрастивного 
исследования пунктуации может быть и~синхронным, и~диахроническим.

\vspace*{-6pt}
    
    \section{Методологические особенности  
корпусно-ориентированного исследования в~области 
контрастивной пунктуации}

\vspace*{-3pt}
  
  Особенности методологии при корпусном контрастивном изуче\-нии 
пунктуации, как, впрочем, и~при любом 
 кор\-пус\-но-ори\-ен\-ти\-ро\-ван\-ном исследовании, связаны прежде всего 
со стремлением получить непротиворечивые, валидные и~на\-деж\-ные данные. 
Электронный корпус, будучи методологически новаторским инструментом 
для получения научного знания, поз\-во\-ля\-ет, с~одной стороны, автоматическим 
образом обрабатывать большие массивы данных и~тем самым серьезно 
сокращает временные издержки на поиск эмпирического материала. 
С~другой стороны, электронные корпусные ресурсы имеют свои 
особенности, и~без их над\-ле\-жа\-ще\-го учета пользователь рискует получить 
искаженные результаты.
  
  Например, в~указанной выше работе~\cite[с.~291]{14-nu} авторы, описывая 
методологию своего исследования, отмечают, что итальянские примеры 
заимствованы из корпуса, хранящегося в~Базельском университете 
и~со\-сто\-яще\-го из двух частей~--- 33~современных  
ро\-ма\-на-бест\-сел\-ле\-ра (1~млн словоупотреблений) 
и~нехудожественных текс\-та разной на\-прав\-лен\-ности (1~млн 40~тыс.\ 
словоупотреблений), в~то время как англоязычные примеры извлечены из 
подкорпуса <<Книги и~периодические издания>> Британского 
национального корпуса (80~млн словоупотреблений). Итальянский материал, 
по словам авторов, был проанализирован весь, а~для английского из-за 
гораздо большего объема ограничились анализом случайной выборки, объем 
которой со\-по\-ста\-вим с~выборкой из итальянского корпуса. Очевидным 
образом ва\-лид\-ность выводов по результатам анализа англоязычного 
материала здесь может оказаться под вопросом в~силу методологически 
неоднородных установок применительно к~процедуре обработки данных, 
полученных по двум языкам. Примечательно к~тому же, что базельский 
корпус, в~отличие от британского, за\-крыт для общественного пользования.
  
  О подобных ограничениях рассуждает На\-двор\-ни\-ко\-ва в~своей работе, 
по\-свя\-щен\-ной корпусной методологии контрастивного изучения 
пунктуации~\cite{15-nu}, где анализируется час\-тот\-ность упо\-треб\-ле\-ния шести 
знаков препинания (запятой, точки, двоеточия, точ\-ки с~запятой, 
вопросительного и~восклицательного знака) в~английском, французском 
и~чешском языках. Для сбора данных используются со\-по\-ста\-ви\-мые 
веб-кор\-пу\-сы, моноязычные общие (референтные) и~параллельные корпусы. Цель 
автора~--- определить, какой из трех типов корпусных ресурсов наиболее 
подходит для исследований в~об\-ласти контрастивной пунктуации.
  
  Полученные данные показывают, что при изучении пунктуации показатели 
час\-тот\-ности проявляют высокую чув\-ст\-ви\-тель\-ность к~типу текс\-та; 
следовательно, веб-кор\-пу\-сы, которые, как правило, отличают стихийное 
наполнение, не\-упо\-ря\-до\-чен\-ность и~низ\-кая степень струк\-ту\-ри\-ро\-ван\-ности, не 
могут служить источником до\-сто\-вер\-ной информации об упо\-треб\-ле\-нии 
знаков препинания в~том или ином языке. Моноязычный общий корпус, 
наоборот, содержит специальную раз\-мет\-ку (морфологическую, 
синтаксическую и~т.\,д.)\ и~поз\-во\-ля\-ет гиб\-ко настраивать поиск (в~том чис\-ле 
выбирать соответствующий тип текс\-та) в~за\-ви\-си\-мости от конкретных 
исследовательских задач. Такие корпусы располагают большими массивами 
данных, поскольку призваны пред\-ста\-вить язык во всей его пол\-но\-те 
и~многообразии, что, казалось бы, обеспечивает на\-деж\-ность и~ва\-лид\-ность 
полученных результатов. Меж\-ду тем этот тип корпусов имеет существенный 
недостаток~--- ограниченную межъязыковую со\-по\-ста\-ви\-мость. Как правило, 
моноязычные общие корпусы разных языков разительно отличаются по 
объему данных и~их со\-ста\-ву и~поэтому не подходят в~качестве основного 
инструмента контрастивного исследования, а~могут служить лишь 
референтным (проверочным) источником для дополнительной верификации 
ре\-зуль\-ти\-ру\-ющих данных. Кроме того, со\-по\-ста\-ви\-тель\-ный анализ 
относительной час\-тот\-ности упо\-треб\-ле\-ния знаков препинания в~разных 
языках на основе данных, извлеченных из корпусов этого типа, так\-же имеет 
свои ограничения. Он не применим для изучения пунк\-ту\-а\-ции в~языках 
разного строя, которым для кодирования информации требуется 
количественно больше (аналитические языки типа французского) или 
меньше слов (синтетические языки типа русского). Таким образом, лучше 
всего для контрастивного изуче\-ния пунк\-ту\-а\-ции подходят параллельные 
корпусы, которые, не\-смот\-ря на свой сравнительно небольшой объем, 
пред\-став\-ля\-ют существенно больше воз\-мож\-но\-стей для качественного анализа 
упо\-треб\-ле\-ния знаков препинания и~непосредственного со\-по\-став\-ле\-ния их 
абсолютной час\-тот\-ности в~параллельных текс\-тах~--- оригинале и~переводе. 
Однако и~этот тип информационного ресурса не может служить 
универсальным исследовательским инструментом. При его использовании 
необходимо учитывать, что пунктуационные рас\-хож\-де\-ния в~исходном 
и~переводном текс\-те могут быть не результатом сис\-тем\-ных дифференциаций, 
а~возникнуть под влиянием переводческих предпочтений. Следовательно, 
чтобы избежать искажения ре\-зуль\-ти\-ру\-ющих данных, надо следовать 
некоторым методологическим принципам: %\\[-13pt] 
\begin{enumerate}[(1)]
\item данные собираются в~обоих 
переводных на\-прав\-ле\-ни\-ях; %\\[-13pt] 
\item выявленные тенденции проходят 
обязательную проверку с~по\-мощью референтного моноязычного корпуса; %\\[-13pt] 
\item контрастивное изуче\-ние пунктуации с~применением параллельных 
корпусов требует сис\-тем\-но\-го подхода в~том смыс\-ле, что в~функциональном 
диапазоне разных знаков препинания могут быть общие зоны, ука\-зы\-ва\-ющие 
на их потенциальную внут\-ри\-язы\-ко\-вую и~межъ\-язы\-ко\-вую конкуренцию. %\\[-13pt]
\end{enumerate}
    
 \vspace*{-12pt}
 
    \section{Заключение}
    
    \vspace*{-3pt}
    
  В статье представлена обобщенная методологическая схема 
  кор\-пус\-но-ори\-ен\-ти\-ро\-ван\-но\-го 
  исследования в~об\-ласти контрастивной пунктуации~--- 
от\-расли научного знания, интенсивно \mbox{раз\-ви\-ва\-ющей\-ся} и~при\-вле\-ка\-ющей 
внимание специалистов самого широкого профиля. Несмотря на то что 
появляются работы, где описываются сопоставительные исследования 
пунктуации на примере одного произведения или литературного наследия 
отдельно взятого писателя (см., например,~\cite{16-nu,17-nu}), очевидно, что 
для ка\-ких-ли\-бо существенных, круп\-но\-мас\-штаб\-ных обобщений относительно 
межъязыковой пунктуационной асимметрии и~специфики функционирования 
знаков препинания в~разных языках требуется привлечение корпусного 
материала.
  
  Дальнейшее изучение контрастивной пунктуации видится в~нескольких 
направлениях. Необходимо качественное углубление со\-по\-ста\-ви\-тель\-но\-го 
анализа, чтобы его тонкая нюансировка \mbox{поз\-во\-ли\-ла} установить, в~какой мере 
совпадает и~разнится функциональный диапазон того или иного знака 
препинания в~кон\-так\-ти\-ру\-ющих языках в~за\-ви\-си\-мости от жанровой 
при\-над\-леж\-ности текс\-та. Этот анализ целесообразно проводить комплексно, 
охватывая всю со\-во\-куп\-ность синтаксических изменений, которые влекут за 
собой отказ от исходного пунктуирования при переводе с~одного языка на 
другой. Такая ком\-плекс\-ность поможет выявить и~с~большей пол\-но\-той 
описать су\-щест\-ву\-ющие межъ\-язы\-ко\-вые структурные различия, что 
необходимо и~для переводческой практики, и~для обуче\-ния иностранным 
языкам. Требует дальнейшего уточ\-не\-ния вопрос, как на пунктуационные 
преференции переводчика влияет род\-ная языковая культура, 
пунктуационные уста\-нов\-ки которой могут меняться со временем. По мере 
наращивания опыта и~мастерства могут меняться пунктуационные 
предпочтения и~самого переводчика, и~это так\-же пред\-став\-ля\-ет определенный 
научный интерес.
  
  В заключение следует отметить, что одним из современных 
информационных инструментов корпусного исследования в~об\-ласти 
контрастивной пунктуации могут быть НБД, 
раз\-ра\-ба\-ты\-ва\-емые в~отделе~54 Федерального исследовательского цент\-ра 
<<Информатика и~управ\-ле\-ние>> Российской академии наук (о~возможностях 
НБД см.~\cite{6-nu}). В~данный момент этот методологический инструмент 
проходит апро\-ба\-цию в~контрастивном исследовании двоеточия и~многоточия в~трех языках~--- русском, французском и~немецком.

\vspace*{-9pt}
  
{\small\frenchspacing
 {\baselineskip=11.5pt
 %\addcontentsline{toc}{section}{References}
 \begin{thebibliography}{99}
 
 \vspace*{-3pt}
 
 \bibitem{4-nu} %1
\Au{Newmark P.} A~textbook of translation.~--- New York, London, Toronto, Sydney, Tokyo: Prentice 
Hall, 1988. 402~p.
 

\bibitem{2-nu} %2
\Au{May R.} The translator in the text: On reading Russian literature in English.~--- Evanston, IL, USA: 
Northwestern University Press, 1994. 209 p.
\bibitem{3-nu}
\Au{Munday J.} Systems in translation: A~systemic model for descriptive translation studies~// 
Crosscultural transgressions: Research models in translation studies II~--- historical and 
ideological issues~/ Ed. T.~Hermans.~---  Manchester, U.K.: St.\ Jerome, 2002. P.~76--92.
\bibitem{1-nu} %4
\Au{Baker M.} In other words.~--- 2nd ed.~--- London, New York: Routledge, 2011. 352~p.

\bibitem{7-nu} %5
\Au{Сигал К.\,Я.} Контрастивная пунктуация в~начале XXI века~// Язык. Текст. Дискурс: 
Научный альманах Ставропольского отделения РАЛК.~--- Ставрополь: СКФУ, 
2019.  Вып.~17. С.~69--78.
\bibitem{5-nu} %6
\Au{Youdale R.} Using computers in the translation of literary style: Challenges and 
opportunities.~--- London, New York: Routledge, 2020. 242~p.
\bibitem{6-nu} %7
\Au{Нуриев В.\,А., Кружков~М.\,Г.} Корпусные данные при контрастивном изуче\-нии 
пунктуации~// Сис\-те\-мы и~средства информатики, 2023. Т.~33. №\,1. С.~14--23. doi: 10.14357/08696527230102.

\bibitem{8-nu}
Vergleichende Interpunktion~--- comparative punctuation~/ Eds. P.~R$\ddot{\mbox{o}}$ssler, P.~Besl, A.~Saller.~--- 
Berlin, Boston: De Gruyter, 2021. 454~p.
\bibitem{9-nu}
\Au{Rinas K.} Vom genormten Satzbau zur genormten Interpunktion. Zur Funktion der 
Zeichensetzung in $\ddot{\mbox{a}}$lterer und neuerer Zeit~// Vergleichende Interpunktion~--- comparative 
punctuation~/ Eds. P.~R$\ddot{\mbox{o}}$ssler, P.~Besl, A.~Saller.~---
 Berlin, Boston: De Gruyter, 2021. P.~109--136. doi: 10.1515/9783110756319-006.
\bibitem{10-nu}
\Au{Ferrari~A., Stojmenova Weber R.} Das Komma in kontrastiver Perspektive Italienisch-Deutsch~// Vergleichende Interpunktion~--- 
comparative punctuation / Eds. P.~R$\ddot{\mbox{o}}$ssler, P.~Besl, 
A.~Saller.~--- Berlin, Boston: De Gruyter, 2021. P.~57--73. doi: 10.1515/9783110756319-003.

\columnbreak

\bibitem{11-nu}
\Au{Besch W.} Zur Entwicklung der deutschen Interpunktion seit dem sp$\ddot{\mbox{a}}$ten Mittelalter~// 
Interpretation und Edition deutscher Texte des Mittelalters. Festschrift f$\ddot{\mbox{u}}$r John Asher zum 60. 
Geburtstag~/ Eds. K.~Smits, W.~Besch, V.~Lange.~--- Berlin: Erich Schmidt, 1981. P.~187--206.
\bibitem{12-nu}
\Au{Eber-Hammerl F.} Interpunktion in historischen Patientenbriefen // Vergleichende
Interpunktion~--- comparative punctuation~/ Eds. P.~R$\ddot{\mbox{o}}$ssler, 
P.~Besl, A.~Saller.~--- Berlin, Boston: De Gruyter, 2021. P.~163--186.
\bibitem{13-nu}
\Au{Ferrari A.} Leggere la virgola. Una prima ricognizione~// Chimera Romance Corpora 
Linguistic Studies, 2017. Vol.~4. Iss.~2. P.~145--162. doi: 
10.15366/chimera2017. 4.2.001.
\bibitem{14-nu}
\Au{Pecorari F., Longo~F.} The ellipsis and the dash in Italian and English: A~contrastive 
perspective~// Vergleichende Interpunktion~--- comparative punctuation~/ Eds.
 P.~R$\ddot{\mbox{o}}$ssler, P.~Besl, A.~Saller.~--- Berlin, Boston: De Gruyter, 2021. P.~289--314.
 doi: 10.1515/9783110756319-013.
\bibitem{15-nu}
\Au{N$\acute{\mbox{a}}$dvorn$\acute{{\iota}}$kov$\acute{\mbox{a}}$~O.}
The use of English, Czech and French punctuation marks in reference, 
parallel and comparable web corpora: A~question of methodology~// 
Linguist. Prag.,  2020. Vol.~30. Iss.~2. P.~30--50. doi: 
10.14712/ 18059635.2020.1.2.
\bibitem{16-nu}
\Au{Сигал К.\,Я.} Пунктуация как средство создания эмоционального под\-текс\-та (на 
материале рассказа М.\,А.~Шолохова <<Судьба человека>> и~его переводов на английский 
язык)~// Известия РАН. Серия литературы и~языка, 2014. Т.~73. №\,6. С.~38--50.
\bibitem{17-nu}
\Au{Богданов К.\,А.} Пунктуация как мотив: многоточие и~тире~// НЛО, 2022. №\,2(174). С.~241--253.
doi: 0.53953/ 08696365\_2022\_174\_2\_241.

\end{thebibliography}

 }
 }

\end{multicols}

\vspace*{-8pt}

\hfill{\small\textit{Поступила в~редакцию 15.04.23}}

\vspace*{6pt}

%\pagebreak

%\newpage

%\vspace*{-28pt}

\hrule

\vspace*{2pt}

\hrule

\vspace*{-2pt}

\def\tit{METHODOLOGY OF~THE~CORPUS-BASED STUDIES\\ 
IN~THE~FIELD OF~CONTRASTIVE PUNCTUATION}


\def\titkol{Methodology of~the~corpus-based studies 
in~the~field of~contrastive punctuation}


\def\aut{V.\,A.~Nuriev$^1$ and~V.\,I.~Karpov$^{1,2}$}

\def\autkol{V.\,A.~Nuriev and~V.\,I.~Karpov}

\titel{\tit}{\aut}{\autkol}{\titkol}

\vspace*{-14pt}


\noindent
      $^1$Federal Research Center ``Computer Science and Control'' of the Russian 
Academy of Sciences, 44-2~Vavilov\linebreak
$\hphantom{^1}$Str., Moscow 119333, Russian Federation
      
      \noindent
      $^2$Institute of Linguistics of the Russian Academy of Sciences, 1~bld.~1 
Bolshoy Kislovsky Lane, Moscow 125009,\linebreak
$\hphantom{^1}$Russian Federation

\def\leftfootline{\small{\textbf{\thepage}
\hfill INFORMATIKA I EE PRIMENENIYA~--- INFORMATICS AND
APPLICATIONS\ \ \ 2023\ \ \ volume~17\ \ \ issue\ 2}
}%
 \def\rightfootline{\small{INFORMATIKA I EE PRIMENENIYA~---
INFORMATICS AND APPLICATIONS\ \ \ 2023\ \ \ volume~17\ \ \ issue\ 2
\hfill \textbf{\thepage}}}

\vspace*{3pt}
      
      
    
    \Abste{The paper refines the methodological approach to the contrastive 
studies of punctuation. Given the recent achievements of information science, 
computer linguistics, and translation theory, such studies are most likely to be 
corpus-based. The paper presents a~methodological model of research into 
interlingual punctuation asymmetry, the aim of which is to shed light on the 
functional scope of the same punctuation marks in different languages. It shows 
what methodological trends are characteristic of this research area. The focus is 
also on the specificities of corpus methodology in the contrastive study of 
punctuation. It is argued that one of the methodological tools, tailored specifically 
to the needs of contrastive punctuation research, may be the supracorpora 
databases developed at the Federal Research Center ``Computer Science and 
Control'' of the Russian Academy of Sciences.}
    
    \KWE{contrastive punctuation; translation; corpus-based translation studies; 
corpus-based studies; parallel corpus; supracorpora database; asymmetry between 
languages; methodology}
    
    
    
\DOI{10.14357/19922264230213}{VBOZAO}

%\vspace*{-18pt}

\Ack
    \noindent
    The research was carried out using the infrastructure of the Shared Research 
Facilities ``High Performance Computing and Big Data'' (CKP ``Informatics'') of 
FRC CSC RAS (Moscow). The research was supported by the Russian Science Foundation (project  
No.\,23-28-00548).
 
%\vspace*{4pt}

  \begin{multicols}{2}

\renewcommand{\bibname}{\protect\rmfamily References}
%\renewcommand{\bibname}{\large\protect\rm References}

{\small\frenchspacing
 {%\baselineskip=10.8pt
 \addcontentsline{toc}{section}{References}
 \begin{thebibliography}{99}
 
 \bibitem{4-nu-1} %1
\Aue{Newmark, P.} 1988. \textit{A~textbook of translation}. New York, London, Toronto, Sydney, Tokyo: 
Prentice Hall. 402~p.   

\bibitem{2-nu-1}
\Aue{May, R.} 1994. \textit{The translator in the text: On reading Russian 
literature in English}. Evanston, IL: Northwestern University Press. 209~p.
\bibitem{3-nu-1}
\Aue{Munday, J.} 2002. Systems in translation: A~systemic model for 
descriptive translation studies. \textit{Crosscultural transgressions: Research models in 
translation studies II~--- historical and ideological issues}. Ed. T.~Hermans. 
Manchester, U.K.: St.\ Jerome. 76--92.

\bibitem{1-nu-1} %4
\Aue{Baker, M.} 2011. \textit{In other words}. 2nd ed. London, New York: 
Routledge. 352~p.

\bibitem{7-nu-1} %5
\Aue{Seagal, K.\,Ya.} 2019. Kont\-ras\-tiv\-naya punk\-tu\-a\-tsiya v~na\-cha\-le XXI~veka 
[Contrastive punctuation at the beginning of the XXI century]. \textit{Yazyk. Tekst. 
Diskurs: Nauchnyy al'manakh Stavropol'skogo otdeleniya RALK} [Language. Text. 
Discourse: Scientific almanac of Stavropol Branch of the Russian Cognitive 
Linguists Association].  Stavropol': SKFU. 17:69--78.

\bibitem{5-nu-1} %6
\Aue{Youdale, R.} 2020. \textit{Using computers in the translation of literary style: 
Challenges and opportunities}. London, New York: Routledge. 242~p.
\bibitem{6-nu-1} %7
\Aue{Nuriev, V.\,A., and M.\,G.~Kruzhkov.} 2023. Kor\-pus\-nye dan\-nye pri 
kont\-ras\-tiv\-nom izu\-che\-nii punk\-tu\-a\-tsii [The parallel corpora perspective on studying 
contrastive punctuation]. \textit{Sistemy i~Sredstva Informatiki~--- Systems and Means of 
Informatics} 33(1):14--23. doi: 10.14357/08696527230102.

  \bibitem{8-nu-1}
R$\ddot{\mbox{o}}$ssler, P., P.~Besl, and A.~Saller, eds. 2021. \textit{Vergleichende 
Interpunktion~--- comparative punctuation}. Berlin, Boston: De Gruyter. 454~p.
\bibitem{9-nu-1}
\Aue{Rinas, K.} 2021. Vom genormten satzbau zur genormten interpunktion. 
Zur funktion der zeichensetzung in $\ddot{\mbox{a}}$lterer und neuerer zeit. \textit{Vergleichende 
Interpunktion~--- comparative punctuation}. Eds.\ P.~R$\ddot{\mbox{o}}$ssler, 
P.~Besl, and A.~Saller. 
Berlin, Boston: De Gruyter. 109--136. doi: 10.1515/ 9783110756319-006.
\bibitem{10-nu-1}
\Aue{Ferrari, A., and R.~Stojmenova.} 2021. Weber das komma in kontrastiver 
perspektive Italienisch-Deutsch. \textit{Vergleichende Interpunktion~--- comparative 
punctuation}. Eds. P.~R$\ddot{\mbox{o}}$ssler, P.~Besl, and A.~Saller. Berlin, Boston: De Gruyter.  
57--73. doi: 10.1515/9783110756319-003.
 \bibitem{11-nu-1}
\Aue{Besch, W.} 1981. Zur entwicklung der deutschen interpunktion seit 
dem sp$\ddot{\mbox{a}}$ten mittelalter. \textit{Interpretation und Edition deutscher Texte des Mittelalters. 
Festschrift f$\ddot{\mbox{u}}$r John Asher zum 60. Geburtstag}. Eds. K.~Smits, W.~Besch, and 
V.~Lange. Berlin: Erich Schmidt. 187--206.
 \bibitem{12-nu-1}
\Aue{Eber-Hammerl, F.} 2021. Interpunktion in historischen 
Patientenbriefen. \textit{Vergleichende Interpunktion~--- comparative punctuation}. Eds. 
P.~R$\ddot{\mbox{o}}$ssler, P.~Besl, and A.~Saller. Berlin, Boston: De Gruyter. 163--186.
\bibitem{13-nu-1}
\Aue{Ferrari, A.} 2017. Leggere la virgola. Una prima ricognizione. 
\textit{Chimera Romance Corpora Linguistic Studies} 4(2):145--162. doi: 
10.15366/chimera2017.4.2.001.
\bibitem{14-nu-1}
\Aue{Pecorari, F., and F.~Longo.} 2021. The ellipsis and the dash in Italian 
and English: A~contrastive perspective. \textit{Vergleichende Interpunktion~--- 
comparative punctuation}. Eds. P.~R$\ddot{\mbox{o}}$ssler, P.~Besl, and A.~Saller. Berlin, Boston: 
De Gruyter. 289--314. doi: 10.1515/9783110756319-013.
\bibitem{15-nu-1}
\Aue{N$\acute{\mbox{a}}$dvorn$\!\acute{\mbox{\ptb{\i}}}$kov$\acute{\mbox{a}}$,~O.} 2020. The use of English, Czech and French 
punctuation marks in reference, parallel and comparable web corpora: A~question 
of methodology. \textit{Linguist. Prag.} 30(2):30--50. doi: 
10.14712/18059635.2020.1.2.
\bibitem{16-nu-1}
\Aue{Seagal, K.\,Ya.} 2014. Punk\-tu\-a\-tsiya kak sred\-st\-vo so\-zda\-niya 
emo\-tsi\-o\-nal'\-no\-go pod\-teks\-ta (na ma\-te\-ri\-ale ras\-ska\-za M.\,A.~Sho\-lo\-kho\-va ``Sud'\-ba 
che\-lo\-ve\-ka'' i~ego pe\-re\-vo\-dov na ang\-liy\-skiy yazyk) [Punctuation as a means of 
revealing the emotional subtext (the case of Mikhail Sholokhov's short story ``The 
Fate of a~Man'' and its translations into English)]. \textit{Izvestiya RAN. Seriya literatury i~yazyka}
 [The Bulletin of the Russian Academy of Sciences: Studies in Literature 
and Language]. 73(6):38--50.
\bibitem{17-nu-1}
\Aue{Bogdanov, K.\,A.} 2022. Punk\-tu\-a\-tsiya kak mo\-tiv: mno\-go\-to\-chie i~ti\-re 
[Punctuation as a~motive: The ellipsis and the dash]. \textit{NLO} [New Literary Observer] 
2(174):241--253. doi: 0.53953/08696365\_2022\_174\_2\_241.
\end{thebibliography}

 }
 }

\end{multicols}

\vspace*{-6pt}

\hfill{\small\textit{Received April 15, 2023}} 

\vspace*{-18pt}
    
    
    \Contr
    
    
    \vspace*{-3pt}
    
    \noindent
    \textbf{Nuriev Vitaly A.} (b.\ 1980)~--- Doctor of Science in philology, leading 
scientist, Institute of Informatics Problems, Federal Research Center ``Computer 
Science and Control'' of the Russian Academy of Sciences, 44-2~Vavilov Str., 
Moscow 119333, Russian Federation; \mbox{nurieff.v@gmail.com}
    
    \vspace*{3pt}
    
    \noindent
    \textbf{Karpov Vladimir I.} (b.\ 1971)~--- Candidate of Science (PhD) in 
philology, leading scientist, Institute of Linguistics of the Russian Academy of 
Sciences, 1~bld.~1 Bolshoy Kislovsky lane, Moscow 125009, Russian Federation; 
scientist, Institute of Informatics Problems, Federal Research Center ``Computer 
Science and Control'' of the Russian Academy of Sciences, 44-2~Vavilov Str., 
Moscow 119333, Russian Federation; \mbox{wi.karpow@gmail.com}
     
      
\label{end\stat}

\renewcommand{\bibname}{\protect\rm Литература}     %15
 
 


%%%%%%%%%%%%%%%%%%%%%%%%%%%%%%%%%%%%%%%%%%%%%%%

%\def\stat{rez}
{%\hrule\par
%\vskip 7pt % 7pt
\raggedleft\Large \bf%\baselineskip=3.2ex
Р\,Е\,Ц\,Е\,Н\,З\,И\,И \vskip 17pt
    \hrule
    \par
\vskip 6pt plus 6pt minus 3pt }

%\thispagestyle{headings} %с верхним колонтитулом
%\thispagestyle{myheadings} %с нижним колонтитулом, но в верхнем РЕЦЕНЗИИ

\def\tit{НОВАЯ КНИГА И.\,Н.~СИНИЦЫНА, А.\,С.~ШАЛАМОВА <<ЛЕКЦИИ ПО ТЕОРИИ 
ИНТЕГРИРОВАННОЙ ЛОГИСТИЧЕСКОЙ ПОДДЕРЖКИ>> (М.: ТОРУС ПРЕСС, 2012. 624~с.)}

%1
\def\aut{Д.ф.-м.н., профессор С.\,Я.~Шоргин}

\def\auf{\ }

\def\leftkol{\ % РЕЦЕНЗИИ
}

\def\rightkol{ \ } 

%\def\leftkol{\ } % ENGLISH ABSTRACTS}

%\def\rightkol{\ } %ENGLISH ABSTRACTS}

%\def\leftkol{РЕЦЕНЗИИ}

%\def\rightkol{РЕЦЕНЗИИ}

\titele{\tit}{\aut}{\auf}{\leftkol}{\rightkol}
\vspace*{-18pt}


     \label{st\stat}

     \begin{multicols}{2}
     {\small
     {\baselineskip=10.1pt
     

      В книге представлено системное изложение теоретических основ одного из новейших 
направлений в \mbox{об\-ласти} экономики послепродажного обслуживания изделий наукоемкой 
продукции (ИНП) длительного пользования~--- интегрированной логистической поддержки
(ИЛП). 
{\looseness=1

}

Приведены также результаты новых работ, выполненных в Институте проблем информатики 
Российской академии наук в рамках научного направления <<Информационные технологии и 
анализ сложных сис\-тем>>.
 {%\looseness=1

}
     
      Излагаемые в книге научные подходы позво\-ляют карди\-наль\-но реформировать 
существующие системы производства и эксплуатации ИНП путем создания и внед\-ре\-ния 
методов рационального и оптимального управ\-ле\-ния процессами расходования 
вре\-мен\-н$\acute{\mbox{ы}}$х, 
мате\-ри\-аль\-ных, трудовых и других ресурсов на всех стадиях жизненного цикла изделий (ЖЦИ) по 
критериям экономической целесообразности и эф\-фек\-тив\-ности.
  {\looseness=1

}
    
      В книге приведен краткий обзор причин возник\-новения и
      развития CALS-методологии как основы 
современных международных стандартов по созданию и функционированию глобальных 
ин\-фор\-ма\-ци\-он\-но-ком\-му\-ни\-ка\-ци\-он\-ных систем, ее ключевых возможностей и эффективности 
результатов ее использования. 
Авторы %\linebreak 
предлагают ряд научных обоснований для разработки 
единой теории проектирования и управления систем ИЛП для полноценного использования 
преимуществ %\linebreak
 суще\-ст\-ву\-ющей методологии, определяют \mbox{общую} структурную схему 
комплексной системы <<ИНП-СППО>> и необходимость разработки для ее описания 
гибридных стохастических моделей.
{%\looseness=1

}

%\columnbreak
      
      Книга состоит из пяти частей, где последовательно излагается материал по каждой из 
следующих тем: <<Интегрированная логистическая поддержка>>, <<Теория гибридных 
стохастических систем и компьютерная поддержка исследований и разработок>>, <<Основы 
математического моделирования, анализа и синтеза систем послепродажного обслуживания>>, 
<<Определение и анализ показателей экспортного потенциала ИНП при проектировании>>, 
<<Задачи управления поддержкой послепродажного обслуживания>>, а также 
<<Моделирование инвестиционных процессов ИЛП в условиях неравновесных финансовых 
рынков>>. 
   
      В конце каждой главы приведены выводы и даны вопросы и задания для 
самоконтроля. В~приложениях содержатся основные определения по программам работ по 
анализу ИЛП, логистическим базам данных и компьютерным решениям, эквивалентной статистической 
линеаризации нелинейных преобразований ИЛП, справочный материал, а также развернутые 
уравнения для вероятностных характеристик.


      \def\leftkol{РЕЦЕНЗИИ}

\def\rightkol{РЕЦЕНЗИИ} 

      
      Книга заинтересует широкий круг специалистов и может быть использована научными 
проектными организациями в сфере промышленного производства ИНП. Большое количество 
иллюстраций, примеров и вопросов, обращенных к читателю, позволяет использовать книгу 
также в качестве учебного пособия для студентов и аспирантов машиностроительных, 
транспортных и~других специальностей, а также для самостоятельного изучения. 
{%\looseness=-1

}

Книга 
представляет несомненный интерес для специалистов и студентов в области прикладной 
математики и информатики.
    

}

}
\end{multicols}

%\newpage

\def\stat{authorsrus}
{%\hrule\par
%\vskip 7pt % 7pt
\raggedleft\Large \bf%\baselineskip=3.2ex
О\,Б\ \ А\,В\,Т\,О\,Р\,А\,Х \vskip 17pt
    \hrule
    \par
\vskip 21pt plus 8pt minus 4pt }


\def\tit{\ }

\def\aut{\ }

\def\auf{\ }

\def\leftkol{\ } % ENGLISH ABSTRACTS}

\def\rightkol{ОБ АВТОРАХ} %ENGLISH ABSTRACTS}

\titele{\tit}{\aut}{\auf}{\leftkol}{\rightkol}
      
            \label{st\stat}



\vspace*{24pt}

\begin{multicols}{2}




\noindent
\textbf{Архипов Олег Петрович} (р.\ 1948)~---
кандидат технических наук, директор Орловского филиала Института проб\-лем информатики
Российской академии наук
%302025, г.Орел, Московское шоссе, д.137

\vspace*{3pt}

\noindent
\textbf{Бирюкова Татьяна Константиновна} (р.\ 1968)~---
кандидат фи\-зи\-ко-ма\-те\-ма\-ти\-че\-ских наук, старший научный сотрудник Института проб\-лем информатики
Российской академии наук

\vspace*{3pt}

\noindent 
\textbf{Бобков  Сергей Геннадьевич} (р.\ 1955)~---
доктор технических наук,  заведующий отделением На\-уч\-но-ис\-сле\-до\-ва\-тель\-ско\-го 
института системных исследований Российской академии наук
%117218, Москва, Нахимовский просп., 36, к.1 

\vspace*{3pt}

\noindent \textbf{Васильев Николай Семенович} (р.\ 1952)~--- доктор 
фи\-зи\-ко-ма\-те\-ма\-ти\-че\-ских наук, профессор, 
МГТУ им.\ Н.\,Э.~Баумана 
%, Москва 105005, 2-я Бауманская ул., д.~5,

\vspace*{3pt}

\noindent
\textbf{Гершкович Максим Михайлович} (р.\ 1968)~---
старший научный сотрудник Института проб\-лем информатики
Российской академии наук

\vspace*{3pt}

\noindent 
\textbf{Дьяченко Юрий Георгиевич} (р.\ 1958)~--- кандидат технических наук, 
старший научный сотрудник Института проб\-лем информатики
Российской академии наук

\vspace*{3pt}

\noindent 
\textbf{Ерошенко Александр Андреевич} (р.\ 1989)~--- аспирант кафедры 
математической статистики факультета вычисли\-тельной математики и кибернетики 
Московского государственного университета им.\ М.\,В.~Ломоносова
%119991, Москва ГСП-1, Ленинские горы, д.\ 1, стр. 52

\vspace*{3pt}
 
\noindent 
\textbf{Захаров Виктор Николаевич} (р.\ 1948)~--- 
доктор технических наук, доцент, ученый секретарь Института проб\-лем информатики
Российской академии наук

\vspace*{3pt}

\noindent
\textbf{Зейфман Александр Израилевич} (р.\ 1954)~---
доктор фи\-зи\-ко-ма\-те\-ма\-ти\-че\-ских наук, профессор, 
заведующий кафедрой Вологодского государственного университета; 
старший научный сотрудник Института проб\-лем информатики
Российской академии наук; главный научный сотрудник ИСЭРТ Российской академии наук

\vspace*{3pt}

\noindent
\textbf{Зыкин Сергей Владимирович} (р.\ 1959)~--- 
доктор технических наук, профессор, заведующий лабораторией Института математики 
им.\ С.\,Л.~Соболева Сибирского отделения Российской академии наук, Новосибирск 
%630090, пр.\ ак.\ Коптюга, 4 

\vspace*{4pt}

\noindent
\textbf{Киреев Владимир Иванович} (р.\ 1938)~---
доктор фи\-зи\-ко-ма\-те\-ма\-ти\-че\-ских наук, профессор Московского 
государственного горного университета
%Адрес: Россия, 119991, г. Москва, Ленинский проспект, д. 6

%\columnbreak

\vspace*{4pt}

\noindent
\textbf{Козеренко Елена Борисовна} (р.\ 1959)~---
кандидат филологических наук, заведующая лабораторией Института проб\-лем информатики
Российской академии наук

\vspace*{4pt}

\noindent
\textbf{Королев Виктор Юрьевич} (р.\ 1954)~--- доктор
фи\-зи\-ко-ма\-те\-ма\-ти\-че\-ских наук, профессор кафедры математической 
статистики факультета вычисли\-тельной математики и кибернетики 
Московского государственного университета; 
ведущий научный сотрудник Института проб\-лем информатики
Российской академии наук

\vspace*{4pt}

\noindent
\textbf{Коротышева Анна Владимировна} (р.\ 1988)~---
старший преподаватель Вологодского государственного университета

\vspace*{4pt}

\noindent 
\textbf{Кун Де Турк} (р.\ 1981)~--- научный сотрудник 
исследовательской группы SMACS факультета телекоммуникаций и обработки информации
Университета Гента, Бельгия
%В-9000 Гент, Бельгия

\vspace*{4pt}

\noindent
\textbf{Лупенцов Олег Сергеевич} (р.\ 1986)~---
аспирант Омского государственного института сервиса
%Омск 644043, ул.\ Певцова 13

\vspace*{4pt}

\noindent
\textbf{Лучко Олег Николаевич} (р.\ 1961)~---
кандидат педагогических наук, профессор, заведующий кафедрой 
Омского государственного института сервиса
%Омск 644043, ул.\ Певцова 13

\vspace*{4pt}

\noindent
\textbf{Малашенко Юрий Евгеньевич} (р.\ 1946)~---
доктор фи\-зи\-ко-ма\-те\-ма\-ти\-че\-ских наук, заведующий сектором 
Вычислительного центра им.\ А.\,А.~Дородницына Российской академии наук
%Адрес: 119333, Москва, ул. Вавилова, 40,

\vspace*{4pt}

\noindent
\textbf{Маньяков Юрий Анатольевич} (р.\ 1984)~---
кандидат технических наук, научный сотрудник Орловского филиала Института проб\-лем информатики
Российской академии наук
%302025, г.Орел, Московское шоссе, д.137

\vspace*{4pt}

\noindent
\textbf{Маренко Валентина Афанасьевна} (р.\ 1951)~---
кандидат технических наук, доцент, старший научный сотрудник 
Института математики им.\ С.\,Л.~Соболева Сибирского отделения Российской академии наук
%Новосибирск 630090, пр. ак. Коптюга, 4 

\vspace*{3pt}

\noindent 
\textbf{Морозов Евсей Викторович} (р.\ 1947)~--- доктор 
фи\-зи\-ко-ма\-те\-ма\-ти\-че\-ских, профессор, ведущий научный сотрудник 
Института прикладных математических исследований Карельского научного центра Российской
академии наук; 
%%185910 Россия, Республика Карелия, г.\ Петрозаводск, ул.\ Пушкинская, 11
профессор Петрозаводского государственного университета, Петрозаводск
%185910 Россия, Республика Карелия, г.\ Петрозаводск, пр.\ Ленина, 33

%\pagebreak

\vspace*{3pt}

\noindent
\textbf{Назарова Ирина Александровна} (р.\ 1966)~---
кандидат фи\-зи\-ко-ма\-те\-ма\-ти\-че\-ских наук, 
научный сотрудник Вычислительного центра им.\ А.\,А.~Дородницына Российской академии наук 
%Адрес: 119333, Москва, ул. Вавилова, 40

\vspace*{3pt}

\noindent
\textbf{Павлов Игорь Валерианович} (р.\ 1945)~--- 
доктор фи\-зи\-ко-ма\-те\-ма\-ти\-че\-ских наук, профессор МГТУ им.\ Н.\,Э.~Баумана 
%Москва 105005, 2-я Бауманская ул., д.~5 

%\pagebreak

\vspace*{3pt}

\noindent 
\textbf{Потахина Любовь Викторовна} (р.\ 1989)~--- аспирантка
Института прикладных математических исследований Карельского научного центра
Российской академии наук; 
%%185910 Россия, Республика Карелия, г.\ Петрозаводск, ул.\ Пушкинская, 11
инженер Петрозаводского государственного университета, Петрозаводск
%185910 Россия, Республика Карелия, г.\ Петрозаводск, пр.\ Ленина, 33

\vspace*{3pt}

\noindent 
\textbf{Рождественский Юрий Владимирович} (р.\ 1952)~--- 
кандидат технических наук, заведующий сектором Института проб\-лем информатики
Российской академии наук

\vspace*{3pt}

\noindent 
\textbf{Синицын Игорь Николаевич} (р.\ 1940)~--- доктор технических наук,
профессор, заслуженный деятель\linebreak\vspace*{-12pt}

\columnbreak

\noindent
 науки РФ, заведующий отделом Института проб\-лем информатики
Российской академии наук

\vspace*{7pt}


\noindent
\textbf{Сиротинин Денис Олегович} (р.\ 1984)~---
кандидат технических наук, научный сотрудник Орловского филиала Института проб\-лем информатики
Российской академии наук
%302025, г.Орел, Московское шоссе, д.137

\vspace*{7pt}

%\columnbreak

\noindent 
\textbf{Соколов  Игорь Анатольевич} (р.\ 1954)~--- академик (действительный член) Российской 
академии наук, доктор технических наук, директор Института проб\-лем информатики
Российской академии наук

\vspace*{7pt}

\noindent
\textbf{Степченков Юрий Афанасьевич} (р.\ 1951)~---
кандидат технических наук, заведующий отделом Института проб\-лем информатики
Российской академии наук

\vspace*{7pt}

\noindent
\textbf{Сурков Алексей Викторович} (р.\ 1978)~--- 
старший научный сотрудник На\-уч\-но-ис\-сле\-до\-ва\-тель\-ско\-го 
института системных исследований Российской академии наук
%117218, Москва, Нахимовский просп., 36, к.1 

\vspace*{7pt}

\noindent 
\textbf{Шестаков Олег Владимирович} (р.\ 1976)~--- доктор 
фи\-зи\-ко-ма\-те\-ма\-ти\-че\-ских, доцент кафедры математической статистики 
факультета вычисли\-тельной математики и кибернетики Московского 
государственного университета им.\ М.\,В.~Ломоносова; 
%119991, Москва ГСП-1, Ленинские горы, д.\ 1, стр. 52
старший научный сотрудник Института проб\-лем информатики
Российской академии наук
%, Москва 119333, ул. Вавилова, д.~44, корп.~2

\vspace*{7pt}

\noindent 
\textbf{Шоргин Сергей Яковлевич} (р.\ 1952.)~--- доктор
фи\-зи\-ко-ма\-те\-ма\-ти\-че\-ских наук, профессор, заместитель директора Института 
проб\-лем информатики Российской академии наук





%%%%%%%%%%%%%%%%%%%%%%%%%%%%%%%%%%%%%%%%%%%%%%%%%%%%%%%%%%%%%%%%%%%%%%%%%%%%%%%




%\def\rightkol{ОБ АВТОРАХ}
%\def\leftkol{ОБ АВТОРАХ}

 \label{end\stat}





%\def\leftfootline{\small{\textbf{\thepage}
%\hfill ИНФОРМАТИКА И ЕЁ ПРИМЕНЕНИЯ\ \ \ том~7\ \ \ выпуск~1\ \ \ 2013}
%}%
% \def\rightfootline{\small{ИНФОРМАТИКА И ЕЁ ПРИМЕНЕНИЯ\ \ \ том~7\ \ \ выпуск~1\ \ \ 2013
%\hfill \textbf{\thepage}}}


%\thispagestyle{myheadings}



\end{multicols}

\newpage

%\end{document}

%
\def\stat{rekl}
%\label{preobr}

%\def\tit{АКАДЕМИК ПУГАЧЁВ  ВЛАДИМИР СЕМЁНОВИЧ\\
%25.03.1911--25.03.1998}


%   \vspace*{-48pt}
%   \begin{center}\LARGE
%Академик Пугачёв  Владимир Семёнович\\ (25.03.1911--25.03.1998)
%   \end{center}

   %\vspace*{2.5mm}

   \begin{center}

{\prgsh\LARGE
ЮБИЛЕИ}

\end{center}
%\hrule

\vspace*{6pt}


   \vspace*{8mm}

   \thispagestyle{empty}


%\def\stat{emel}


\section*{К 70-летию заместителя директора ИПИ РАН,\\ члена редколлегии журнала
<<Информатика и её применения>>\\ доктора технических наук В.\,И.~Будзко}

\vspace*{18pt}




          \begin{multicols}{2}

%            \label{st\stat}

\begin{center}
\vspace*{1pt}
\mbox{%
\epsfxsize=78mm
\epsfbox{bud-1.eps}
}
\end{center}

\vspace*{12pt}

      14 августа 2014~г.\ исполнилось 70~лет за\-мес\-ти\-те\-лю директора ИПИ РАН по
научной работе доктору технических наук Владимиру Игоревичу Будзко.

      Владимир Игоревич Будзко родился в г.~Москве. Высшее образование получил на факультете
элект\-рон\-но-вы\-чис\-ли\-тель\-ных устройств в Московском
ин\-же\-нер\-но-фи\-зи\-че\-ском институте
(МИФИ), который он окончил в 1968~г., после чего был на\-прав\-лен для прохождения
службы в одну из войс\-ко\-вых частей, где прошел путь от инженера до первого заместителя
командира войсковой части.

      С приходом В.\,И.~Будзко в ИПИ РАН (2001~г.)\ в институте
сформировалось новое научное на\-прав\-ле\-ние теоретических исследований~--- <<Постро\-ение
ин\-фор\-ма\-ци\-он\-но-те\-ле\-ком\-му\-ни\-ка\-ци\-он\-ных\linebreak сис\-тем
высокой до\-ступ\-ности>>. В~рамках этого
направления выполнен широкий круг фундаментальных исследований по поиску подходов и
определению принципов построения средств обеспечения доступности, конфиденциальности
и целостности современных крупномасштабных
ин\-фор\-ма\-ци\-он\-но-те\-ле\-ком\-му\-ни\-ка\-ци\-он\-ных
сис\-тем (ИТС). Разработаны основные сис\-тем\-но-тех\-ни\-че\-ские принципы и базовые
архитектурные решения построения перспективных для условий России ИТС с
централизованной обработкой и хранением информации, сочетающих в себе свойства
высокой доступности, отказо- и катастрофоустойчивости, информационной защищенности.
Определены принципы, методы и математические основы рационального построения и
оптимизации средств восстановления функционирования центров обработки данных (ЦОД)
после возникновения отказов и катастроф, передачи и хранения данных, обеспечения
информационной безопасности при достижении минимальной совокупной стоимости
владения такими системами. Результаты нашли практическое воплощение при реализации
проектов в интересах ряда отечественных государственных и негосударственных
организаций, таких как Банк России (БР), Внешторгбанк, ОАО <<ГМК <<Норильский Никель>>,
<<Газпром>>, Минэкономразвития России, Правительство Москвы, а также ряд силовых
ведомств.

      Под руководством В.\,И.~Будзко начиная с 2001~г.\ выполнен комплекс
      на\-уч\-но-ис\-сле\-до\-ва\-тель\-ских и
      опыт\-но-кон\-ст\-рук\-тор\-ских работ (свыше 100~проектов),
направленных на развитие электронной информационной технологии БР.
Разработаны концепции развития ИТС БР сначала до 2008~г., а затем до 2013~г., которые
были приняты в качестве основы проведения технической политики. За реализацию проекта
<<Катастрофоустойчивая тер\-ри\-то\-ри\-аль\-но-рас\-пре\-де\-лен\-ная
      ин\-фор\-ма\-ци\-он\-но-те\-ле\-ком\-му\-ни\-ка\-ци\-он\-ная сис\-те\-ма централизованной
обработки банковской информации>> В.\,И.~Будзко удостоен Премии Правительства РФ в
области науки и техники за 2010~г.

      В.\,И.~Будзко возглавлял и возглавляет работы по ряду других прикладных проектов,
связанных с созданием, совершенствованием и развитием крупномасштабных ИТС.

      В.\,И.~Будзко~--- генерал-майор, доктор технических наук, член-кор\-рес\-пон\-дент
Академии криптографии РФ, известный ученый в области информатики и применения
информационных технологий при построении территориально распределенных ИТС
различного назначения. Является автором свыше 250~научных работ, опубликованных в
на\-уч\-но-тех\-ни\-че\-ских и специальных изданиях.

    \thispagestyle{empty}

      В.\,И.~Будзко уделяет большое внимание подготовке научных кадров. Под его
руководством защищено 6~диссертаций на соискание ученой степени кандидата
технических наук. Свыше 30~лет он читает лекции в ИКСИ Академии ФСБ, профессор
кафедры НИЯУ МИФИ. Является членом двух диссертационных советов, главным
редактором журнала <<Системы высокой доступности>> и членом редколлегии журнала
<<Информатика и её применения>>.

      \bigskip

      Редакционный совет и Редакционная коллегия журнала <<Информатика и её
применения>> сердечно поздравляют Владимира Игоревича Будзко с 70-ле\-ти\-ем и желают
крепкого здоровья и новых научных достижений.

\end{multicols}

%\def\stat{cont}
{%\hrule\par
%\vskip 7pt % 7pt
\raggedleft\Large \bf%\baselineskip=3.2ex
А\,В\,Т\,О\,Р\,С\,К\,И\,Й\ \ У\,К\,А\,З\,А\,Т\,Е\,Л\,Ь\ \ З\,А\ \ 2\,0\,1\,0 г. \vskip 17pt
    \hrule
    \par
\vskip 21pt plus 6pt minus 3pt }

\label{st\stat}

\def\tit{\ }

\def\aut{\ }
\def\auf{\ }

\def\leftkol{\ } % ENGLISH ABSTRACTS}

\def\rightkol{\ } %АВТОРСКИЙ УКАЗАТЕЛЬ ЗА 2010 г.} %ENGLISH ABSTRACTS}

\titele{\tit}{\aut}{\auf}{\leftkol}{\rightkol}

\vspace*{-12pt}

{\tabcolsep=3pt
\begin{tabular}{p{388pt}rr}
&\textbf{Выпуск} & \textbf{Стр.}\\[6pt]
\hangindent=23pt\noindent\textbf{Арутюнян~А.\,Р.} Моделирование влияния деформаций отпечатков пальцев на 
точность\linebreak
\vspace*{-12pt}\\
\hspace*{23pt}дактилоскопической идентификации$\dotfill$&1&51\\
\hangindent=23pt\noindent\textbf{Архипов~О.\,П., Зыкова~З.\,П.} Интеграция гетерогенной информации о цветных 
пикселях\linebreak
\vspace*{-12pt}\\
\hspace*{23pt}и их цветовосприятии$\dotfill$&4&15\\
\hangindent=23pt\noindent\textbf{Баранов~С.\,И., Френкель~С.\,Л., Захаров~В.\,Н.} Полуформальная верификация 
цифрового устройства с конвейером, основанная на использовании алгоритмических машин\linebreak
\vspace*{-12pt}\\
\hspace*{23pt}состояния$\dotfill$&4&49\\
\textbf{Бекетова~И.\,В.} см.~Каратеев~С.\,Л.&&\\
\textbf{Белоусов~В.\,В.} см.~Синицын~И.\,Н.&&\\
\hangindent=23pt\noindent\textbf{Бенинг~В.\,Е., Королев~Р.\,А.} О предельном поведении мощностей критериев в 
случае\linebreak
\vspace*{-12pt}\\
\hspace*{23pt}распределения Лапласа$\dotfill$&2&63\\
\hangindent=23pt\noindent\textbf{Бенинг~В.\,Е., Сипина~А.\,В.} Асимптотическое разложение для мощности 
критерия,\linebreak
\vspace*{-12pt}\\
\hspace*{23pt}основанного на выборочной медиане, в случае распределения Лапласа$\dotfill$&1&18\\
\textbf{Бондаренко~А.\,В.} см.~Каратеев~С.\,Л.&&\\
\hangindent=23pt\noindent\textbf{Бородина~А.\,В., Морозов~Е.\,В.} Об оценивании асимптотики вероятности 
большого\linebreak
\vspace*{-12pt}\\
\hspace*{23pt}уклонения стационарной регенеративной очереди с одним прибором$\dotfill$&3&29\\
\hangindent=23pt\noindent\textbf{Бунтман~Н.\,В., Минель~Ж.-Л., Ле~Пезан~Д., Зацман~И.\,М.} Типология и 
компьютерное\linebreak
\vspace*{-12pt}\\
\hspace*{23pt}моделирование трудностей перевода$\dotfill$&3&77\\
\textbf{Визильтер~Ю.\,В.} см.~Каратеев~С.\,Л.&&\\
\hangindent=23pt\noindent\textbf{Гавриленко~С.\,В.} Оценки скорости сходимости распределений случайных сумм с 
безгранично делимыми индексами к нормальному закону$\dotfill$&4&81\\
\hangindent=23pt\noindent\textbf{Григорьева~М.\,Е., Шевцова~И.\,Г.} Уточнение неравенства 
Каца--Берри--Эссеена$\dotfill$&2&75\\
\hangindent=23pt\noindent\textbf{Грушо~А.\,А., Грушо~Н.\,А., Тимонина~Е.\,Е.} Поиск конфликтов в политиках 
безопасности: модель случайных графов$\dotfill$&3&38\\
\textbf{Грушо~Н.\,А.} см.~Грушо~А.\,А.&&\\
\hangindent=23pt\noindent\textbf{Гудков~В.\,Ю.} Математические модели изображения отпечатка пальца на основе 
описания линий$\dotfill$&1&58\\
\textbf{Гуртов~А.\,В.} см.~Лукьяненко~А.\,С.&&\\
\textbf{Желтов~С.\,Ю.} см.~Каратеев~С.\,Л.&&\\
\hangindent=23pt\noindent\textbf{Захаров~А.\,А., Серебряков~В.\,А.} Система управления электронной библиотекой 
LibMeta$\dotfill$&4&2\\
\textbf{Захаров~В.\,Н.} см.~Баранов~С.\,И.&&\\
\textbf{Захарова~Т.\,В.} см.~Матвеева~С.\,С.&&\\
\hangindent=23pt\noindent\textbf{Зацаринный~А.\,А., Чупраков~К.\,Г.} Некоторые аспекты выбора технологии для 
постро-\linebreak
\vspace*{-12pt}\\
\hspace*{23pt}ения систем отображения информации ситуационного центра$\dotfill$&3&59\\
\textbf{Зацман~И.\,М.} см.~Бунтман~Н.\,В.&&\\
\hangindent=23pt\noindent\textbf{Зейфман~А.\,И., Коротышева~А.\,В., Сатин~Я.\,А., Шоргин~С.\,Я.} Об 
устойчивости нестаци-\linebreak
\vspace*{-12pt}\\
\hspace*{23pt}онарных систем обслуживания с катастрофами$\dotfill$&3&9\\
\textbf{Зыкова~З.\,П.} см.~Архипов~О.\,П.&&\\
\hangindent=23pt\noindent\textbf{Илюшин~Г.\,Я., Соколов~И.\,А.} Организация управляемого доступа пользователей 
к\linebreak
\vspace*{-12pt}\\
\hspace*{23pt}разнородным ведомственным информационным ресурсам$\dotfill$&1&24\\
\hangindent=23pt\noindent\textbf{Кавагучи~Ю., Ульянов~В.\,В., Фуджикоши~Я.} Приближения для статистик, 
описывающих\linebreak
\vspace*{-12pt}\\
\hspace*{23pt}геометрические свойства данных большой размерности, с оценками 
ошибок$\dotfill$&1&12\\
\hangindent=23pt\noindent\textbf{Каратеев~С.\,Л., Бекетова~И.\,В., Ососков~М.\,В., Князь~В.\,А., 
Визильтер~Ю.\,В., Бондаренко~А.\,В., Желтов~С.\,Ю.} Автоматизированный контроль 
качества цифровых\linebreak
\vspace*{-12pt}\\
\hspace*{23pt}изображений для персональных документов$\dotfill$&1&65\\
\end{tabular}
}

\pagebreak

\def\leftkol{АВТОРСКИЙ УКАЗАТЕЛЬ ЗА 2010 г.} % ENGLISH ABSTRACTS}

\def\rightkol{АВТОРСКИЙ УКАЗАТЕЛЬ ЗА 2010 г.} %ENGLISH ABSTRACTS}

{\tabcolsep=3pt
\begin{tabular}{p{388pt}rr}
&\textbf{Выпуск} & \textbf{Стр.}\\[3pt]
\hangindent=23pt\noindent\textbf{Козеренко~Е.\,Б.} Лингвистические фильтры в статистических моделях машинного\linebreak
\vspace*{-12pt}\\
\hspace*{23pt}перевода$\dotfill$&2&83\\
\hangindent=23pt\noindent\textbf{Козеренко~Е.\,Б., Кузнецов~И.\,П.} Когнитивно-лингвистические представления в 
систе-\linebreak
\vspace*{-12pt}\\
\hspace*{23pt}мах обработки текстов$\dotfill$&3&69\\
\textbf{Князь~В.\,А.} см.~Каратеев~С.\,Л.&&\\
\hangindent=23pt\noindent\textbf{Колесников~А.\,В., Солдатов~С.\,А.} Алгоритм координации для гибридной 
интеллектуальной системы решения сложной задачи оперативно-производственного\linebreak
\vspace*{-12pt}\\
\hspace*{23pt}планирования$\dotfill$&4&61\\
\hangindent=23pt\noindent\textbf{Коновалов~М.\,Г.} О планировании потоков в системах вычислительных 
ресурсов$\dotfill$&2&3\\
\textbf{Конушин~А.\,С.} см.~Конушин~В.\,С.&&\\
\hangindent=23pt\noindent\textbf{Конушин~В.\,С., Кривовязь~Г.\,Р., Конушин~А.\,С.} Алгоритм распознавания людей 
в видео-\linebreak
\vspace*{-12pt}\\
\hspace*{23pt}последовательности по одежде$\dotfill$&1&74\\
\textbf{Корепанов~Э.\, Р.} см.~Синицын~И.\,Н.&&\\
\textbf{Королев~В.\,Ю.} см.~Соколов~И.\,А.&&\\
\textbf{Королев~Р.\,А.} см.~Бенинг~В.\,Е.&&\\
\textbf{Коротышева~А.\,В.} см.~Зейфман~А.\,И.&&\\
\hangindent=23pt\noindent\textbf{Кривенко~М.\,П.} Непараметрическое оценивание элементов байесовского 
клас\-си-\linebreak
\vspace*{-12pt}\\
\hspace*{23pt}фикатора$\dotfill$&2&13\\
\textbf{Кривовязь~Г.\,Р.} см.~Конушин~В.\,С.&&\\
\textbf{Крылов~А.\,С.} см.~Павельева~Е.\,А.&&\\
\hangindent=23pt\noindent\textbf{Крылов~В.\,А.} Моделирование и классификация многоканальных дистанционных\linebreak
\vspace*{-12pt}\\
\hspace*{23pt}изображений с использованием копул$\dotfill$&4&34\\
\hangindent=23pt\noindent\textbf{Крючин~О.\,В.} Разработка параллельных эвристических алгоритмов подбора 
весовых\linebreak
\vspace*{-12pt}\\
\hspace*{23pt}коэффициентов искусственной нейтронной сети$\dotfill$&2&53\\
\hangindent=23pt\noindent\textbf{Кудрявцев~А.\,А., Шоргин~С.\,Я.} Байесовские модели массового обслуживания и 
надеж-\linebreak
\vspace*{-12pt}\\
\hspace*{23pt}ности: характеристики среднего числа заявок в системе $M\vert M \vert 1\vert 
\infty$$\dotfill$&3&16\\
\hangindent=23pt\noindent\textbf{Кузнецов~А.\,А.} Связь между временными и структурно-топологическими 
характери-\linebreak
\vspace*{-12pt}\\
\hspace*{23pt}стиками диаграмм ритма сердца здоровых людей$\dotfill$&4&39\\
\textbf{Кузнецов~И.\,П.} см.~Козеренко~Е.\,Б.&&\\
\textbf{Ле~Пезан~Д.} см.~Бунтман~Н.\,В.&&\\
\hangindent=23pt\noindent\textbf{Лукьяненко~А.\,С., Морозов~Е.\,В., Гуртов~А.\,В.} Анализ сетевого протокола с общей 
функ-\linebreak
\vspace*{-12pt}\\
\hspace*{23pt}цией расширения окна передачи сообщения при конфликтах$\dotfill$&2&46\\
\hangindent=23pt\noindent\textbf{Лямин~О.\,О.} О предельном поведении мощностей критериев в случае обобщенного\linebreak
\vspace*{-12pt}\\
\hspace*{23pt}распределения Лапласа$\dotfill$&3&47\\
\hangindent=23pt\noindent\textbf{Маркин~А.\,В., Шестаков~О.\,В.} Асимптотики оценки риска при пороговой 
обработке\linebreak
\vspace*{-12pt}\\
\hspace*{23pt}вейвлет-вейглет коэффициентов в задаче томографии$\dotfill$&2&36\\
\hangindent=23pt\noindent\textbf{Матвеева~С.\,С., Захарова~Т.\,В.} Сети массового обслуживания с наименьшей 
длиной\linebreak
\vspace*{-12pt}\\
\hspace*{23pt}очереди$\dotfill$&3&22\\
\hangindent=23pt\noindent\textbf{Матюшенко~С.\,И.} Стационарные характеристики двухканальной системы 
обслужива-\linebreak
\vspace*{-12pt}\\
\hspace*{23pt}ния с переупорядочиванием заявок и распределениями фазового типа$\dotfill$&4&68\\
\textbf{Минель~Ж.-Л.} см.~Бунтман~Н.\,В.&&\\
\textbf{Морозов~Е.\,В.} см.~Бородина~А.\,В.&&\\
\textbf{Морозов~Е.\,В.} см.~Лукьяненко~А.\,С.&&\\
\textbf{Ососков~М.\,В.} см.~Каратеев~С.\,Л.&&\\
\hangindent=23pt\noindent\textbf{Павельева~Е.\,А., Крылов~А.\,С.} Поиск и анализ ключевых точек радужной 
оболочки\linebreak
\vspace*{-12pt}\\
\hspace*{23pt}глаза методом преобразования Эрмита$\dotfill$&1&79\\
\textbf{Печинкин~А.\,В.} см.~Френкель~С.\,Л.,&&\\
\hangindent=23pt\noindent\textbf{Протасов~В.\,И.} Составление субъективного портрета с использованием 
эволюционно-\linebreak
\vspace*{-12pt}\\
\hspace*{23pt}го морфинга и квалиметрия метода$\dotfill$&1&83\\
\hangindent=23pt\noindent\textbf{Рудаков~К.\,В., Торшин~И.\,Ю.} Вопросы разрешимости задачи распознавания 
вторичной\linebreak
\vspace*{-12pt}\\
\hspace*{23pt}структуры белка$\dotfill$&2&25\\
\textbf{Сатин~Я.\,А.} см.~Зейфман~А.\,И.&&\\
\hangindent=23pt\noindent\textbf{Сейфуль-Мулюков~Р.\,Б.} Нефть как носитель информации о своем 
происхождении,\linebreak
\vspace*{-12pt}\\
\hspace*{23pt}структуре и эволюции$\dotfill$&1&41\\
\end{tabular}
}

{\tabcolsep=3pt
\begin{tabular}{p{388pt}rr}
&\textbf{Выпуск} & \textbf{Стр.}\\[6pt]
\textbf{Семендяев~Н.\,Н.} см.~Синицын~И.\,Н.&&\\
\textbf{Серебряков~В.\,А.} см.~Захаров~А.\,А.&&\\
\textbf{Синицын~В.\,И.} см.~Синицын~И.\,Н.&&\\
\hangindent=23pt\noindent\textbf{Синицын~И.\,Н., Синицын~В.\,И., Корепанов~Э.\, Р., Белоусов~В.\,В., 
Семендяев~Н.\,Н.} Оперативное построение информационных моделей движения полюса 
Земли\linebreak
\vspace*{-12pt}\\
\hspace*{23pt}методами линейных и линеаризованных фильтров$\dotfill$&1&2\\
\textbf{Сипина~А.\,В.} см.~Бенинг~В.\,Е.&&\\
\hangindent=23pt\noindent\textbf{Соколов~И.\,А.} О работах заслуженного деятеля науки Российской Федерации 
И.\,Н.~Синицына в области информационных технологий и автоматизации (к 70-летию\linebreak
\vspace*{-12pt}\\
\hspace*{23pt}со дня рождения)$\dotfill$&3&84\\
\textbf{Соколов~И.\,А.} см.~Илюшин~Г.\,Я.&&\\
\hangindent=23pt\noindent\textbf{Соколов~И.\,А., Королев~В.\,Ю.} Предисловие$\dotfill$&2&2\\
\textbf{Солдатов~С.\,А.} см.~Колесников~А.\,В.&&\\
\hangindent=23pt\noindent\textbf{Степанов~С.\,Ю.} Использование координатного метода фрагментации 
коммутаторной\linebreak
\vspace*{-12pt}\\
\hspace*{23pt}нейронной сети для сокращения трафика$\dotfill$&2&57\\
\textbf{Тимонина~Е.\,Е.} см.~Грушо~А.\,А.&&\\
\textbf{Торшин~И.\,Ю.} см.~Рудаков~К.\,В.&&\\
\textbf{Ульянов~В.\,В.} см.~Кавагучи~Ю.&&\\
\textbf{Фазекаш~И.} см.~Чупрунов~А.\,Н.&&\\
\textbf{Френкель~С.\,Л.} см.~Баранов~С.\,И.&&\\
\hangindent=23pt\noindent\textbf{Френкель~С.\,Л., Печинкин~А.\,В.} Оценка времени самовосстановления в 
цифровых\linebreak
\vspace*{-12pt}\\
\hspace*{23pt}системах после сбоев, вызываемых переходными помехами$\dotfill$&3&2\\
\textbf{Фуджикоши~Я.} см.~Кавагучи~Ю.&&\\
\hangindent=23pt\noindent\textbf{Цискаридзе~А.\,К.} Математическая модель и метод восстановления позы человека 
по\linebreak
\vspace*{-12pt}\\
\hspace*{23pt}стереопаре силуэтных изображений$\dotfill$&4&27\\
\hangindent=23pt\noindent\textbf{Чупраков~К.\,Г.} К вопросу о размещении коллективных средств отображения в 
ситуа-\linebreak
\vspace*{-12pt}\\
\hspace*{23pt}ционном зале с заданными параметрами$\dotfill$&4&89\\
\textbf{Чупраков~К.\,Г.} см.~Зацаринный~А.\,А.&&\\
\hangindent=23pt\noindent\textbf{Чупрунов~А.\,Н., Фазекаш~И.} Законы повторного логарифма для числа 
безошибочных\linebreak
\vspace*{-12pt}\\
\hspace*{23pt}блоков при помехоустойчивом кодировании$\dotfill$&3&42\\
\textbf{Шевцова~И.\,Г.} см.~Григорьева~М.\,Е.&&\\
\hangindent=23pt\noindent\textbf{Шестаков~О.\,В.} Аппроксимация распределения оценки риска пороговой 
обработки вейвлет-коэффициентов нормальным распределением при использовании 
выбо-\linebreak
\vspace*{-12pt}\\
\hspace*{23pt}рочной дисперсии$\dotfill$&4&73\\
\textbf{Шестаков~О.\,В.} см.~Маркин~А.\,В.&&\\
\textbf{Шоргин~С.\,Я.} см.~Зейфман~А.\,И.&&\\
\textbf{Шоргин~С.\,Я.} см.~Кудрявцев~А.\,А.&&\\
\end{tabular}
}

%\thispagestyle{myheadings}
\def\leftfootline{\small{\textbf{\thepage}
\hfill ИНФОРМАТИКА И ЕЁ ПРИМЕНЕНИЯ\ \ \ том~4\ \ \ выпуск~4\ \ \ 2010}
}%
 \def\rightfootline{\small{ИНФОРМАТИКА И ЕЁ ПРИМЕНЕНИЯ\ \ \ том~4\ \ \ выпуск~4\ \ \ 2010
 \hfill \textbf{\thepage}}}
 \label{end\stat}




%
%Том 10 Выпуск 1-4 Год 2016

\def\stat{cont-e}
{%\hrule\par
%\vskip 7pt % 7pt
\raggedleft\Large \bf%\baselineskip=3.2ex
2\,0\,1\,6\ \ A\,U\,T\,H\,O\,R\ \ I\,N\,D\,E\,X \vskip 17pt
 \hrule
 \par
\vskip 21pt plus 6pt minus 3pt }

\label{st\stat}

\def\tit{\ }

\def\aut{\ }
\def\auf{\ }

\def\leftkol{\ } %2016 AUTHOR INDEX} % ENGLISH ABSTRACTS}

\def\rightkol{\ } %2016 AUTHOR INDEX} %ENGLISH ABSTRACTS}

\titele{\tit}{\aut}{\auf}{\leftkol}{\rightkol}

\def\leftfootline{\small{\textbf{\thepage}
\hfill INFORMATIKA I EE PRIMENENIYA~--- INFORMATICS AND APPLICATIONS\ \ \ 2016\
\ \ volume~10\ \ \ issue\ 4}
}%
 \def\rightfootline{\small{INFORMATIKA I EE PRIMENENIYA~--- INFORMATICS AND APPLICATIONS\ \ \ 2016\ \ \ volume~10\ \ \ issue\ 4
\hfill \textbf{\thepage}}}

\vspace*{-12pt}
\vspace*{-18pt}

{\tabcolsep=2.8pt
\begin{tabular}{p{382pt}cc}
&\textbf{Issue} & \textbf{Page}\\[6pt]
\Avtors{Agalarov~M.\,Ya.} see~Agalarov~Ya.\,M.&&\\
\Avtors{Agalarov~Ya.\,M., Agalarov~M.\,Ya., and
Shorgin~V.\,S.} About the optimal threshold of queue\linebreak
\\[-12pt]
\hspace*{23pt}length in a~particular problem of profit maximization
in the $M/G/1$ queuing system&2&70--79\\
\Avtors{Alexeyevsky~D.\,A.} BioNLP ontology extraction from 
a~restricted language corpus with\linebreak
\\[-12pt]
\hspace*{23pt}context-free grammars&1&119--128\\
\Avtors{Andreev~S.\,D.} see~Gaidamaka~Yu.\,V.&&\\
\Avtors{Andreev~S.\,D.} see~Ometov~A.\,Ya.&&\\
\Avtors{Arkhipov~O.\,P., Arkhipov~P.\,O., and Sidorkin~I.\,I.} The
option to create a~local coordinate\linebreak
\\[-12pt]
\hspace*{23pt}system for synchronization of selected images&3&91--97\\
\Avtors{Arkhipov~P.\,O.} see~Arkhipov~O.\,P.&&\\
\Avtors{Belousov~V.\,V.} see~Shnurkov~P.\,V.&&\\
\Avtors{Belousov~V.\,V.} see~Shnurkov~P.\,V.&&\\
\Avtors{Bening~V.\,E.} Calculation of~the~asymptotic deficiency
of~some statistical procedures based\linebreak
\\[-12pt]
\hspace*{23pt}on~samples with~random sizes&4&34--45\\
\Avtors{Borisov~A.\,V., Bosov~A.\,V., and Miller~G.\,B.} Modeling and
monitoring of VoIP connection&2&\hphantom{1}2--13\\
\Avtors{Bosov~A.\,V.} see~Borisov~A.\,V.&&\\
\Avtors{Briukhov~D.\,O.} see~Stupnikov~S.\,A.&&\\
\Avtors{Callaos~N.\,K.\ and Seyful-Mulyukov~R.\,B.} Complexity and
its information content&1&129--139\\
\Avtors{Chertok~A.\,V., Kadaner~A.\,I., Khazeeva~G.\,T., and
Sokolov~I.\,A.} Regime switching detection\linebreak
\\[-12pt]
\hspace*{23pt}for~the~Levy driven
Ornstein--Uhlenbeck process using CUSUM methods&4&46--56\\
\Avtors{Chichagov~V.\,V.} Asymptotic expansions of mean absolute
error of uniformly minimum variance unbiased and maximum likelihood
estimators on the one-parameter exponential\linebreak
\\[-12pt]
\hspace*{23pt}family model of lattice distributions&3&66--76\\
\Avtors{Danishevsky~V.\,I.} see~Kolesnikov A.\,V.&&\\
\Avtors{Fazliev~A.\,Z.} see~Kalinichenko~L.\,A.&&\\
\Avtors{Fedoseev~A.\,A.} What is behind the concept of ``knowledge in
small packages''&3&105--110\\
\Avtors{Gaidamaka~Yu.\,V., Andreev~S.\,D., Sopin~E.\,S.,
Samouylov~K.\,E., and Shorgin~S.\,Ya.} Interference analysis
of~the~device-to-device communications model with~regard to~a~signal\linebreak
\\[-12pt]
\hspace*{23pt}propagation environment&4&\hphantom{1}2--10\\
\Avtors{Gasilov~A.\,V.} see~Yakovlev~O.\,A.&&\\
\Avtors{Goncharov~A.\,V.\ and Strijov~V.\,V.} Metric time series
classification using weighted dynamic\linebreak
\\[-12pt]
\hspace*{23pt}warping relative to centroids of classes&2&36--47\\
\Avtors{Gordov~E.\,P.} see~Kalinichenko~L.\,A.&&\\
\Avtors{Gorshenin~A.\,K.} Concept of online service for stochastic
modeling of real processes&1&72--81\\
\Avtors{Gorshenin~A.\,K.} see~Shnurkov~P.\,V.&&\\
\Avtors{Gorshenin~A.\,K.} see~Shnurkov~P.\,V.&&\\
\Avtors{Grusho~A.\,A., Grusho~N.\,A., Zabezhailo~M.\,I., and
Timonina~E.\,E.} Integration of statistical and\linebreak
\\[-12pt]
\hspace*{23pt}deterministic methods for
analysis of information security&3&2--8\\
\Avtors{Grusho~A.\,A., Zabezhailo~M.\,I., and Zatsarinny~A.\,A.} On
the advanced procedure to reduce\linebreak
\\[-12pt]
\hspace*{23pt}calculation of Galois closures&4&\hphantom{1}96--104\\
\Avtors{Grusho~N.\,A.} see~Grusho~A.\,A.&&\\
\Avtors{Havanskov~V.\,A.} see~Minin~V.\,A.&&\\
\Avtors{Inkova~O.\,Yu.} see~Zatsman~I.\,M.&&\\
\Avtors{Isachenko~R.\,V.\ and Strijov~V.\,V.} Metric learning in
multiclass time series classification\linebreak
\\[-12pt]
\hspace*{23pt}problem&2&48--57\\
\end{tabular}
}
\pagebreak

\def\leftfootline{\small{\textbf{\thepage}
\hfill INFORMATIKA I EE PRIMENENIYA~--- INFORMATICS AND APPLICATIONS\ \ \ 2016\
\ \ volume~10\ \ \ issue\ 4}
}%
 \def\rightfootline{\small{INFORMATIKA I EE PRIMENENIYA~---
INFORMATICS AND APPLICATIONS\ \ \ 2016\ \ \ volume~10\ \ \ issue\ 4
\hfill \textbf{\thepage}}}

\def\leftkol{2016 AUTHOR INDEX} % ENGLISH ABSTRACTS}

\def\rightkol{2016 AUTHOR INDEX} %ENGLISH ABSTRACTS}


{\tabcolsep=2.83pt
\begin{tabular}{p{382pt}cc}
&\textbf{Issue} & \textbf{Page}\\[6pt]
\Avtors{Kadaner~A.\,I.} see~Chertok~A.\,V.&&\\[.255pt]
\Avtors{Kalinichenko~L.\,A., Volnova~A.\,A., Gordov~E.\,P.,
Kiselyova~N.\,N., Kovaleva~D.\,A., Malkov~O.\,Yu., Okladnikov~I.\,G.,
Podkolodnyy~N.\,L., Pozanenko~A.\,S., Ponomareva~N.\,V.,
Stupnikov~S.\,A.,} \textbf{and Fazliev~A.\,Z.} Data access challenges for data
intensive\linebreak
\\[-12pt]
\hspace*{23pt}research in Russia&1& 2--22\\[.255pt]
\Avtors{Karasikov~M.\,E.\ and Strijov~V.\,V.} Feature-based
time-series classification&4&121--131\\[.255pt]
\Avtors{Khazeeva~G.\,T.} see~Chertok~A.\,V.&&\\[.255pt]
\Avtors{Khokhlov~Yu.\,S.} Multivariate fractional Levy motion and its
applications&2&\hphantom{1}98--106\\[.255pt]
\Avtors{Kirikov~I.\,A., Kolesnikov~A.\,V., Listopad~S.\,V., and
Rumovskaya~S.\,B.} Fine-grained hybrid\linebreak
\\[-12pt]
\hspace*{23pt}intelligent systems. Part 2:
Bidirectional hybridization&1&\hphantom{1}96--105\\[.255pt]
\Avtors{Kirikov~I.\,A., Kolesnikov~A.\,V., Listopad~S.\,V., and
Rumovskaya~S.\,B.} ``Virtual council''~---\linebreak
\\[-12pt]
\hspace*{23pt}source environment
supporting complex diagnostic decision making&3&81--90\\[.255pt]
\Avtors{Kiselyova~N.\,N.} see~Kalinichenko~L.\,A.&&\\[.255pt]
\Avtors{Kolesnikov A.\,V., Listopad~S.\,V., Rumovskaya~S.\,B., and
Danishevsky~V.\,I.} Informal axiomatic\linebreak
\\[-12pt]
\hspace*{23pt}theory of~the~role visual models&4&114--120\\[.255pt]
\Avtors{Kolesnikov~A.\,V.} see~Kirikov~I.\,A.&&\\[.255pt]
\Avtors{Kolesnikov~A.\,V.} see~Kirikov~I.\,A.&&\\[.255pt]
\Avtors{Kolin~K.\,K.} Humanitarian aspects of information
security&3&111--121\\[.255pt]
\Avtors{Konovalov~M.\,G.\ and Razumchik~R.\,V.} Dispatching
to~two parallel nonobservable queues using\linebreak
\\[-12pt]
\hspace*{23pt}only static
information&4&57--67\\[.255pt]
\Avtors{Korchagin~A.\,Yu.} see~Korolev~V.\,Yu.&&\\[.255pt]
\Avtors{Korchagin~A.\,Yu.} see~Korolev~V.\,Yu.&&\\[.255pt]
\Avtors{Korepanov~E.\,R.} see~Sinitsyn~I.\,N.&&\\[.255pt]
\Avtors{Korepanov~E.\,R.} see~Sinitsyn~I.\,N.&&\\[.255pt]
\Avtors{Korolev~V.\,Yu., Korchagin~A.\,Yu., and Zeifman~A.\,I.} The
Poisson theorem for Bernoulli trials\linebreak
\\[-12pt]
\hspace*{23pt}with~a~random probability
of~success and~a~discrete analog of~the~Weibull distribution&4&11--20\\[.255pt]
\Avtors{Korolev~V.\,Yu., Zeifman~A.\,I., and Korchagin~A.\,Yu.}
Asymmetric Linnik distributions as~limit\linebreak
\\[-12pt]
\hspace*{23pt}laws for~random sums
of~independent random variables with~finite variances&4&21--33\\[.255pt]
\Avtors{Koucheryavy~E.\,A.} see~Ometov~A.\,Ya.&&\\[.255pt]
\Avtors{Kovaleva~D.\,A.} see~Kalinichenko~L.\,A.&&\\[.255pt]
\Avtors{Kovalyov~S.\,P.} Metaprogramming to increase
manufacturability of large-scale software-\linebreak
\\[-12pt]
\hspace*{23pt}intensive systems&1&56--66\\[.255pt]
\Avtors{Krivenko~M.\,P.} Significance tests of feature selection for
classification&3&32--40\\[.255pt]
\Avtors{Kruzhkov~M.\,G.} see~Zalizniak~Anna~A.&&\\[.255pt]
\Avtors{Kruzhkov~M.\,G.} see~Zatsman~I.\,M.&&\\[.255pt]
\Avtors{Kudryavtsev~A.\,A.} Bayesian queueing and reliability models:
\textit{A~priori} distributions with\linebreak
\\[-12pt]
\hspace*{23pt}compact support&1&67--71\\[.255pt]
\Avtors{Kudryavtsev~A.\,A.} Characteristics dependent on the balance
coefficient in Bayesian models\linebreak
\\[-12pt]
\hspace*{23pt}with compact support of \textit{a priori}
distributions&3&77--80\\[.255pt]
\Avtors{Kudryavtsev~A.\,A.\ and Palionnaia~S.\,I.} Bayesian recurrent
model of reliability growth:\linebreak
\\[-12pt]
\hspace*{23pt}Parabolic distribution of parameters&2&80--83\\[.255pt]
\Avtors{Kudryavtsev~A.\,A.\ and Titova~A.\,I.} Bayesian queuing
and~reliability models: Degenerate-\linebreak
\\[-12pt]
\hspace*{23pt}Weibull case&4&68--71\\[.255pt]
\Avtors{Leontyev~N.\,D.\ and Ushakov~V.\,G.} Analysis of a queueing
system with autoregressive arrivals\linebreak
\\[-12pt]
\hspace*{23pt}and nonpreemptive priority&3&15--22\\[.255pt]
\Avtors{Listopad~S.\,V.} see~Kirikov~I.\,A.&&\\[.255pt]
\Avtors{Listopad~S.\,V.} see~Kirikov~I.\,A.&&\\[.255pt]
\Avtors{Listopad~S.\,V.} see~Kolesnikov A.\,V.&&\\[.255pt]
\Avtors{Malkov~O.\,Yu.} see~Kalinichenko~L.\,A.&&\\[.255pt]
\Avtors{Markov~A.\,S., Monakhov~M.\,M., and
Ulyanov~V.\,V.} Generalized Cornish--Fisher expansions\linebreak
\\[-12pt]
\hspace*{23pt}for distributions of statistics based on samples
of random size&2&84--91\\[.255pt]
\Avtors{Melnikov~A.\,K.\ and Ronzhin~A.\,F.} Generalized statistical
method of~text analysis based\linebreak
\\[-12pt]
\hspace*{23pt}on~calculation of~probability distributions
of~statistical values&4&89--95\\
\end{tabular}
}
\pagebreak

\def\leftfootline{\small{\textbf{\thepage}
\hfill INFORMATIKA I EE PRIMENENIYA~--- INFORMATICS AND APPLICATIONS\ \ \ 2016\
\ \ volume~10\ \ \ issue\ 4}
}%
 \def\rightfootline{\small{INFORMATIKA I EE PRIMENENIYA~---
INFORMATICS AND APPLICATIONS\ \ \ 2016\ \ \ volume~10\ \ \ issue\ 4
\hfill \textbf{\thepage}}}

\def\leftkol{2016 AUTHOR INDEX} % ENGLISH ABSTRACTS}

\def\rightkol{2016 AUTHOR INDEX} %ENGLISH ABSTRACTS}


{\tabcolsep=3pt
\begin{tabular}{p{381pt}cc}
&\textbf{Issue} & \textbf{Page}\\[6pt]
\Avtors{Meykhanadzhyan~L.\,A.} Stationary characteristics of the finite
capacity queueing system with\linebreak
\\[-12pt]
\hspace*{23pt}inverse service order and generalized
probabilistic priority&2&123--131\\[.23pt]
\Avtors{Miller~G.\,B.} see~Borisov~A.\,V.&&\\[.23pt]
\Avtors{Minin~V.\,A., Zatsman~I.\,M., Havanskov~V.\,A., and
Shubnikov~S.\,K.} Intensity of citation of scientific publications in
inventions on information and computer technologies patented\linebreak
\\[-12pt]
\hspace*{23pt}in Russia by domestic and foreign applicants&2&107--122\\[.23pt]
\Avtors{Monakhov~M.\,M.} see~Markov~A.\,S.&&\\[.23pt]
\Avtors{Naumov~V.\,A.\ and Samouylov~K.\,E.} On relationship
between queuing systems with resources\linebreak
\\[-12pt]
\hspace*{23pt}and Erlang networks&3&\hphantom{1}9--14\\[.23pt]
\Avtors{Okladnikov~I.\,G.} see~Kalinichenko~L.\,A.&&\\[.23pt]
\Avtors{Ometov~A.\,Ya., Andreev~S.\,D., Turlikov~A.\,M., and
Koucheryavy~E.\,A.} Performance analysis of\linebreak
\\[-12pt]
\hspace*{23pt}a wireless data
aggregation system with contention for contemporary sensor
networks&3&23--31\\[.23pt]
\Avtors{Palionnaia~S.\,I.} see~Kudryavtsev~A.\,A.&&\\[.23pt]
\Avtors{Podkolodnyy~N.\,L.} see~Kalinichenko~L.\,A.&&\\[.23pt]
\Avtors{Ponomareva~N.\,V.} see~Kalinichenko~L.\,A.&&\\[.23pt]
\Avtors{Popkova~N.\,A.} see~Zatsman~I.\,M.&&\\[.23pt]
\Avtors{Pozanenko~A.\,S.} see~Kalinichenko~L.\,A.&&\\[.23pt]
\Avtors{Razumchik~R.\,V.} see~Konovalov~M.\,G.&&\\[.23pt]
\Avtors{Ronzhin~A.\,F.} see~Melnikov~A.\,K.&&\\[.23pt]
\Avtors{Rumovskaya~S.\,B.} see~Kirikov~I.\,A.&&\\[.23pt]
\Avtors{Rumovskaya~S.\,B.} see~Kirikov~I.\,A.&&\\[.23pt]
\Avtors{Rumovskaya~S.\,B.} see~Kolesnikov A.\,V.&&\\[.23pt]
\Avtors{Samouylov~K.\,E.} see~Gaidamaka~Yu.\,V.&&\\[.23pt]
\Avtors{Samouylov~K.\,E.} see~Naumov~V.\,A.&&\\[.23pt]
\Avtors{Serebryanskii~S.\,M.} see~Tyrsin~A.\,N.&&\\[.23pt]
\Avtors{Seyful-Mulyukov~R.\,B.} see~Callaos~N.\,K.&&\\[.23pt]
\Avtors{Shestakov~O.\,V.} Statistical properties of the denoising method
based on the stabilized hard\linebreak
\\[-12pt]
\hspace*{23pt}thresholding&2&65--69\\[.23pt]
\Avtors{Shestakov~O.\,V.} The strong law of large numbers for the risk
estimate in the problem of\linebreak
\\[-12pt]
\hspace*{23pt}tomographic image reconstruction from
projections with a correlated noise&3&41--45\\[.23pt]
\Avtors{Shestakov~O.\,V.} see~Zakharova~T.\,V.&&\\[.23pt]
\Avtors{Shnurkov~P.\,V., Gorshenin~A.\,K., and Belousov~V.\,V.}
Analytical solution of~the~optimal control\linebreak
\\[-12pt]
\hspace*{23pt}task of~a~semi-Markov
process with~finite set of~states&4&72--88\\[.23pt]
\Avtors{Shnurkov~P.\,V., Zasypko~V.\,V., Belousov~V.\,V., and
Gorshenin~A.\,K.} Development of the algorithm of numerical solution
of the optimal investment control problem\linebreak
\\[-12pt]
\hspace*{23pt}in the closed dynamical model of three-sector economy&1&82--95\\[.23pt]
\Avtors{Shorgin~S.\,Ya.} see~Gaidamaka~Yu.\,V.&&\\[.23pt]
\Avtors{Shorgin~V.\,S.} see~Agalarov~Ya.\,M.&&\\[.23pt]
\Avtors{Shubnikov~S.\,K.} see~Minin~V.\,A.&&\\[.23pt]
\Avtors{Sidorkin~I.\,I.} see~Arkhipov~O.\,P.&&\\[.23pt]
\Avtors{Sinitsyn~I.\,N.} Analytical modeling of processes in stochastic
systems with complex fractional\linebreak
\\[-12pt]
\hspace*{23pt}order Bessel nonlinearities&3&55--65\\[.23pt]
\Avtors{Sinitsyn~I.\,N.} Orthogonal supoptimal filters for nonlinear
stochastic systems on manifolds&1&34--44\\[.23pt]
\Avtors{Sinitsyn~I.\,N.\ and Korepanov~E.\,R.} Normal Pugachev
conditionally-optimal filters and extra-\linebreak
\\[-12pt]
\hspace*{23pt}polators for state linear stochastic systems&2&14--23\\[.23pt]
\Avtors{Sinitsyn~I.\,N.\ and Sinitsyn~V.\,I.} Analytical modeling of
distributions in stochastic systems on\linebreak
\\[-12pt]
\hspace*{23pt}manifolds based on ellipsoidal approximation&1&45--55\\[.23pt]
\Avtors{Sinitsyn~I.\,N., Sinitsyn~V.\,I., and
Korepanov~E.\,R.} Ellipsoidal suboptimal filters for nonlinear\linebreak
\\[-12pt]
\hspace*{23pt}stochastic systems on manifolds&2&24--35\\[.23pt]
\Avtors{Sinitsyn~V.\,I.} see~Sinitsyn~I.\,N.&&\\[.23pt]
\Avtors{Sinitsyn~V.\,I.} see~Sinitsyn~I.\,N.&&\\[.23pt]
\Avtors{Skvortsov~N.\,A.} see~Stupnikov~S.\,A.&&\\[.23pt]
\Avtors{Sokolov~I.\,A.} see~Chertok~A.\,V.&&\\
\end{tabular}
}
\pagebreak

\def\leftfootline{\small{\textbf{\thepage}
\hfill INFORMATIKA I EE PRIMENENIYA~--- INFORMATICS AND APPLICATIONS\ \ \ 2016\
\ \ volume~10\ \ \ issue\ 4}
}%
 \def\rightfootline{\small{INFORMATIKA I EE PRIMENENIYA~---
INFORMATICS AND APPLICATIONS\ \ \ 2016\ \ \ volume~10\ \ \ issue\ 4
\hfill \textbf{\thepage}}}

\def\leftkol{2016 AUTHOR INDEX} % ENGLISH ABSTRACTS}

\def\rightkol{2016 AUTHOR INDEX} %ENGLISH ABSTRACTS}


{\tabcolsep=3pt
\begin{tabular}{p{382pt}cc}
&\textbf{Issue} & \textbf{Page}\\[6pt]
\Avtors{Sopin~E.\,S.} see~Gaidamaka~Yu.\,V.&&\\
\Avtors{Strijov~V.\,V.} see~Goncharov~A.\,V.&&\\
\Avtors{Strijov~V.\,V.} see~Isachenko~R.\,V.&&\\
\Avtors{Strijov~V.\,V.} see~Karasikov~M.\,E.&&\\
\Avtors{Stupnikov~S.\,A., Briukhov~D.\,O., and Skvortsov~N.\,A.}
Co-lending systemic risk analysis over\linebreak
\\[-12pt]
\hspace*{23pt}heterogeneous data collections&1&23--33\\
\Avtors{Stupnikov~S.\,A.} see~Kalinichenko~L.\,A.&&\\
\Avtors{Suchkov~A.\,P.} see~Zatsarinny~A.\,A.&&\\
\Avtors{Timonina~E.\,E.} see~Grusho~A.\,A.&&\\
\Avtors{Titova~A.\,I.} see~Kudryavtsev~A.\,A.&&\\
\Avtors{Turlikov~A.\,M.} see~Ometov~A.\,Ya.&&\\
\Avtors{Tyrsin~A.\,N.\ and Serebryanskii~S.\,M.} Recognition of
dependences on the basis of inverse\linebreak
\\[-12pt]
\hspace*{23pt}mapping&2&58--64\\
\Avtors{Ulyanov~V.\,V.} see~Markov~A.\,S.&&\\
\Avtors{Ushakov~V.\,G.} Queueing system with working vacations and
hyperexponential input stream&2&92--97\\
\Avtors{Ushakov~V.\,G.} see~Leontyev~N.\,D.&&\\
\Avtors{Volnova~A.\,A.} see~Kalinichenko~L.\,A.&&\\
\Avtors{Yakovlev~O.\,A.\ and Gasilov~A.\,V.} Speeded-up stereo
matching using geodesic support weights&3&\hphantom{1}98--104\\
\Avtors{Zabezhailo~M.\,I.} see~Grusho~A.\,A.&&\\
\Avtors{Zabezhailo~M.\,I.} see~Grusho~A.\,A.&&\\
\Avtors{Zakharova~T.\,V.\ and Shestakov~O.\,V.} Precision analysis of
wavelet processing of aerodynamic\linebreak
\\[-12pt]
\hspace*{23pt}flow patterns&3&46--54\\
\Avtors{Zalizniak~Anna~A.\ and Kruzhkov~M.\,G.} Database
of~Russian impersonal verbal constructions&4&132--141\\
\Avtors{Zasypko~V.\,V.} see~Shnurkov~P.\,V.&&\\
\Avtors{Zatsarinny~A.\,A.\ and Suchkov~A.\,P.} Systems engineering
approaches to~the~establishment of\linebreak
\\[-12pt]
\hspace*{23pt}a~system for~decision support based
on~situational analysis&4&105--113\\
\Avtors{Zatsarinny~A.\,A.} see~Grusho~A.\,A.&&\\
\Avtors{Zatsman~I.\,M., Inkova~O.\,Yu., Kruzhkov~M.\,G., and
Popkova~N.\,A.} Representation of cross-\linebreak
\\[-12pt]
\hspace*{23pt}lingual knowledge about
connectors in supracorpora databases&1&106--118\\
\Avtors{Zatsman~I.\,M.} see~Minin~V.\,A.&&\\
\Avtors{Zeifman~A.\,I.} see~Korolev~V.\,Yu.&&\\
\Avtors{Zeifman~A.\,I.} see~Korolev~V.\,Yu.&&\\
\end{tabular}
}

%\thispagestyle{myheadings}
\def\leftfootline{\small{\textbf{\thepage}
\hfill INFORMATIKA I EE PRIMENENIYA~--- INFORMATICS AND APPLICATIONS\ \ \ 2016\
\ \ volume~10\ \ \ issue\ 4}
}%
 \def\rightfootline{\small{INFORMATIKA I EE PRIMENENIYA~---
INFORMATICS AND APPLICATIONS\ \ \ 2016\ \ \ volume~10\ \ \ issue\ 4
\hfill \textbf{\thepage}}}

 \label{end\stat}

\newpage

%\def\stat{rekl}
%\label{preobr}

%\def\tit{АКАДЕМИК ПУГАЧЁВ  ВЛАДИМИР СЕМЁНОВИЧ\\
%25.03.1911--25.03.1998}


%   \vspace*{-48pt}
%   \begin{center}\LARGE
%Академик Пугачёв  Владимир Семёнович\\ (25.03.1911--25.03.1998)
%   \end{center}
   
   %\vspace*{2.5mm}
   
   \begin{center}

{\prgsh\LARGE
ОБЪЯВЛЕНИЯ О КОНФЕРЕНЦИЯХ}

\end{center}
%\hrule

\vspace*{6pt}

   
   \vspace*{10mm}
   
   \thispagestyle{empty}

\noindent
\begin{tabular}{cc}
%\begin{center}
\multicolumn{1}{c}{\raisebox{-40pt}[0pt][0pt]{\mbox{%
\epsfxsize=33mm
\epsfbox{vspu.eps}
}}}
%\end{center}
&
\tabcolsep=0pt\begin{tabular}{c}
{\prg{\Large\textbf{XII Всероссийское совещание}}}\\[6pt]
{\prg{\Large\textbf{по проблемам управления}}}\\[12pt]
{\prg{\large 16--19 июня 2014~г.}}\\[6pt] 
{\prg{\large Институт проблем управления имени В.\,А.~Трапезникова РАН}}\\[6pt]
{\prg{\large Москва, Россия}}
\end{tabular}
\end{tabular}

\vspace*{60pt}

     
 { %\large    
 XII Всероссийское совещание по проблемам управления (ВСПУ XII), посвященное 75-летию 
Института проблем управления (ИПУ) имени В.\,А.~Трапезникова РАН, проводится 16--19~июня 
2014~г.\ 
в ИПУ РАН (г.~Москва, Россия). ВСПУ XII организуется ИПУ РАН при поддержке РФФИ, Отделения 
энергетики, машиностроения, механики и процессов управления Российской академии наук, 
Российского 
национального комитета по автоматическому управлению, Академии навигации и управ\-ле\-ния 
движением, 
Научного совета РАН по комплексным проблемам управления и автоматизации, Совета по 
мехатронике и робототехнике РАН. Официальный язык Совещания~--- русский.

\vspace*{24pt}
     
     \textbf{Направления работы}
     \begin{enumerate}[1.]
\item Теория систем управления
\item Управление подвижными объектами и навигация
\item Интеллектуальные системы управления
\item Управление в промышленности, транспортом и логистикой
\item Управление системами междисциплинарной природы
\item Средства измерения, вычислений и контроля в управлении
\item Системный анализ и принятие решений в задачах управления
\item Информационные технологии в управлении
\item Проблемы образования в области управления: современное содержание и технологии обучения
\end{enumerate}

\vspace*{24pt}

     Подробная информация о Совещании находится на сайте {\sf http://vspu2014.ipu.ru}. Срок 
окончательной подачи докладов через систему подачи докладов на сайте~--- \textbf{30~ноября} 
2013~г.
}

%\include{rekl-1}

%\end{document}

%   \vspace*{-48pt}

\begin{center}
\vspace*{6pt}
\mbox{%
\epsfxsize=53.502mm
\epsfbox{foto-1.eps}
}
\end{center}

\vspace*{6pt} %Академик


   \begin{center}
\fbox{\Large\textbf{Профессор Игорь Алексеевич Ушаков}}\\[12pt]
\textbf{\large 22.01.1935--27.02.2015}
   \end{center}


   %\vspace*{2.5mm}

   \vspace*{5mm}

   \thispagestyle{empty}

%\

%\vspace*{-12pt}


Редакционный совет и редакционная коллегия журнала <<Информатика и~её применения>> с~глубоким прискорбием извещают, что 27~февраля 2015~г.\ после тяжелой
и~продолжительной болезни скончался Игорь Алексеевич Ушаков~--- доктор технических наук, профессор, член редколлегии журнала <<Информатика и ее применения>>.

Игорь Алексеевич Ушаков окончил Московский авиационный институт, в~1963~г.\ защитил кандидатскую, а~в~1968~г.~--- докторскую диссертацию. С~1958 по 1989~гг.\ работал в~ряде научно-исследовательских организаций СССР, в~том числе руководил отделами в~НИИ АА и~ВЦ АН СССР; с 1969 по 1989 гг. преподавал в~МФТИ (был профессором, а~затем заведующим кафедрой) и~в~МЭИ. С~1989~г.~---- в~США: являлся профессором университета Дж.\ Вашингтона, университета Дж.\ Мэйсона и~Калифорнийского университета, сотрудником компаний MCI, Qualcomm и Hughes.

И.\,А.~Ушаков с момента основания журнала <<Надежность и~контроль качества>> был заместителем ответственного редактора, а~затем на протяжении многих лет членом редколлегии. В~2006~г.\ основал электронный международный журнал ``Reliability: Theory \& Application'', главным редактором которого оставался до конца жизни.

Учебниками и справочниками по теории надежности, написанными И.\,А.~Ушаковым, пользовались и~пользуются несколько поколений ученых и~специалистов в~разных странах мира.

Игорь Алексеевич всегда уделял огромное внимание работе с~молодежью; более~50 его учеников защитили докторские и~кандидатские диссертации.

И.\,А.~Ушаков вел активную научно-про\-све\-ти\-тель\-скую деятельность. В~частности, он был одним из организаторов и~руководителей Московского кабинета качества и~надежности при Политехническом музее (целью этого Кабинета было оказание консультаций работникам промышленных предприятий и~чтение курсов лекций для инженеров, занимающихся проблемой надежности). Находясь в~США, И.\,А.~Ушаков создал международный ин\-тер\-нет-фо\-рум им.\ Б.\,В.~Гнеденко, объединивший около~400~видных специалистов по приложениям теории вероятностей и~математической статистики, преимущественно в~об\-ласти теории надежности и~анализа риска, из десятков стран мира; коллективным членов этого Форума является и~наш журнал. Цели Форума~--- содействие контактам между специалистами из разных стран, организация обмена профессиональными 
новостями и~информацией (новые публикации, предстоящие события и~др.). Также необходимо отметить большое число на\-уч\-но-по\-пу\-ляр\-ных работ, опубликованных И.\,А.~Ушаковым.

И.\,А.~Ушаков обладал большим личным обаянием, имел широкий круг интересов. Все знавшие И.\,А.~Ушакова всегда будут помнить его как замечательного ученого и~прекрасного человека.

\bigskip

Редакционный совет и редакционная коллегия журнала <<Информатика и~её применения>> 
выражают глубокие соболезнования родным и близким покойного, всем, кто его знал и~работал с~ним.



%\end{document}

%\include{IPPM-25}

\def\stat{cont-rus}
{%\hrule\par
%\vskip 7pt % 7pt
\vspace*{-24pt}
\raggedleft\Large \bf%\baselineskip=3.2ex
Правила подготовки рукописей  для публикации в журнале
<<Информатика~и~её~применения>> \vskip 8pt
    \hrule
    \par
\vskip 14pt plus 6pt minus 3pt }

\label{st\stat}

\def\tit{\ }

\def\aut{\ }
\def\auf{\ }

\def\leftkol{\ }
% Правила подготовки рукописей  для публикации в журнале
%<<Информатика и её применения>>

\def\rightkol{\ }
%Правила подготовки рукописей  для публикации в журнале
%<<Информатика и её применения>>}


\titele{\tit}{\aut}{\auf}{\leftkol}{\rightkol}


\vspace*{-60pt}
{ %\small

Журнал <<Информатика и её применения>>
публикует теоретические, обзорные и дискуссионные статьи,
посвященные научным исследованиям и разработкам в области
информатики и ее приложений.

Журнал издается на русском языке. По специальному решению
редколлегии отдельные статьи могут печататься на английском языке.

Тематика журнала охватывает следующие направления:
\begin{itemize}
\item теоретические основы информатики;\\[-15pt]
      \item
математические методы исследования сложных систем и процессов;\\[-15pt]
           \item
информационные системы и сети;\\[-15pt]
                \item
информационные технологии;\\[-15pt]
                     \item
архитектура и программное обеспечение вычислительных комплексов и сетей.\\[-15pt]
\end{itemize}


\noindent
\begin{enumerate}[1.]
\item В журнале печатаются статьи, содержащие результаты, ранее не опубликованные и
не предназначенные к одновременной публикации в других изданиях.

%Публикация не должна нарушать закон об авторских правах.
Публикация предоставленной автором(ами) рукописи не должна нарушать 
положений глав~69, 70 раздела~VII части~IV Гражданского кодекса, 
которые определяют права на результаты интеллектуальной деятельности 
и~средства индивидуализации, в~том числе авторские права, в~РФ.

Ответственность за нарушение авторских прав, в~случае предъявления претензий к~редакции журнала,  
несут авторы статей.



Направляя рукопись в редакцию, авторы сохраняют свои права на данную
рукопись и при этом передают учредителям и редколлегии журнала неисключительные права на
издание статьи на русском языке 
(или на языке статьи, если он отличен от рус\-ско\-го) и~на перевод ее на английский
язык, а~также на
ее распространение в России и за рубежом. 
Каждый автор должен представить в~редакцию подписанный 
с~его стороны <<Лицензионный договор о~передаче неисключительных прав 
на использование произведения>>, текст которого размещен по адресу 
{\sf http://www.ipiran.ru/publications/licence.doc}. 
Этот договор может быть пред\-став\-лен в~бумажном (в~2-х экз.)\ 
или в~электронном виде (отсканированная копия заполненного и~подписанного документа).




Редколлегия вправе запросить у авторов экспертное заключение о возможности
пуб\-ли\-ка\-ции пред\-став\-лен\-ной статьи в открытой печати.\\[-13.5pt]

\item К статье прилагаются данные автора (авторов) (см.\ п.~8). При наличии нескольких
авторов указывается фамилия автора, ответственного за переписку с редакцией.\\[-13.5pt]

\item Редакция журнала осуществляет экспертизу присланных статей в соответствии с
принятой в журнале процедурой рецензирования.

Возвращение рукописи на доработку не означает ее принятия к печати.

Доработанный вариант с ответом на замечания рецензента необходимо прислать в
редакцию.\\[-13.5pt]

\item Решение редколлегии о публикации статьи или ее отклонении сообщается авторам.

Редколлегия может также направить авторам текст рецензии на их статью. Дискуссия по
поводу отклоненных статей не ведется.\\[-13.5pt]

%\pagebreak

\item Редактура статей высылается авторам для просмотра. Замечания к редактуре должны
быть присланы авторами в кратчайшие сроки.\\[-13.5pt]

\item Рукопись предоставляется в электронном виде в форматах MS WORD (.doc или
.docx) или \LaTeX\  (.tex), дополнительно~--- в формате .pdf, на дискете, лазерном диске
или электронной почтой. Предоставление бумажной рукописи необязательно.\\[-13.5pt]

\item При подготовке рукописи в MS Word рекомендуется использовать следующие
настройки.

Параметры страницы:
формат~--- А4; ориентация~--- книжная; поля (см): внутри~--- 2,5, снаружи~--- 1,5,
сверху~--- 2, снизу~--- 2, от края до нижнего колонтитула~--- 1,3.

Основной текст: стиль~--- <<Обычный>>, шрифт~--- Times New Roman, размер~---
14~пунк\-тов, абзацный отступ~--- 0,5~см, 1,5~интервала, выравнивание~--- по ширине.

\pagebreak

\def\leftkol{Правила подготовки рукописей  для публикации в журнале
<<Информатика и её применения>>}

\def\rightkol{Правила подготовки рукописей  для публикации в журнале
<<Информатика и её применения>>}



Рекомендуемый объем рукописи~--- не свыше 10~страниц указанного формата.
При превышении указанного объема редколлегия вправе потребовать от 
автора сокращения объема рукописи.


Сокращения слов, помимо стандартных, не допускаются. Допускается минимальное
количество аббревиатур.


Все страницы рукописи нумеруются.

Шаблоны оформления представлены в интернете:

\noindent
 {\sf
http://www.ipiran.ru/journal/template\_iiep\_ssi\_2024.zip}\\[-14pt]

\item Статья должна содержать следующую информацию на {\bfseries\textit{русском и
английском языках}}:\\[-16pt]

\begin{itemize}
\item название статьи;\\[-15pt]
\item Ф.И.О.\ авторов, на английском можно только имя и фамилию;\\[-15pt]
\item место работы, с указанием почтового адреса организации и электронного адреса каждого
автора;\\[-15pt]
\item сведения об авторах, в соответствии с форматом, образцы которого
представлены на страницах:



\def\leftfootline{\small{\textbf{\thepage}
\hfill ИНФОРМАТИКА И ЕЁ ПРИМЕНЕНИЯ\ \ \ том\ 18\ \ \ выпуск\ 3\ \ \ 2024}
}%
 \def\rightfootline{\small{ИНФОРМАТИКА И ЕЁ ПРИМЕНЕНИЯ\ \ \ том\ 18\ \ \ выпуск\ 3\ \ \ 2024
\hfill \textbf{\thepage}}}



{\sf http://www.ipiran.ru/journal/issues/2013\_07\_01/authors.asp} и

{\sf http://www.ipiran.ru/journal/issues/2013\_07\_01\_eng/authors.asp};
\item аннотация (не менее 100~слов на каждом из языков). Аннотация~--- это краткое
резюме работы, которое может публиковаться отдельно. Она является основным
источником информации в~ин\-фор\-ма\-ци\-он\-ных системах и базах данных. Английская
аннотация должна быть оригинальной, может не быть дословным переводом русского
текста и должна быть написана хорошим английским языком. В~аннотации не должно
быть ссылок на литературу и, по возможности, формул;\\[-15pt]
\item ключевые слова~--- желательно из принятых в мировой
на\-уч\-но-тех\-ни\-че\-ской литературе тематических тезаурусов. Предложения не
могут быть ключевыми словами;\\[-15pt]
\item источники финансирования работы (ссылки на гранты, проекты,
поддерживающие организации и~т.\,п.).
\end{itemize}



%\pagebreak

\item  Требования к спискам литературы.\\[-14pt]

Ссылки на литературу в тексте статьи нумеруются (в квадратных скобках) и
располагаются в каждом из списков литературы в порядке  первых упоминаний. Если источник имеет DOI и/или EDN,
то их необходимо указывать.

Списки литературы представляются в двух вариантах:\\[-14pt]


\noindent
\begin{enumerate}[(1)]
\item \textbf{Список литературы к русскоязычной части}. Русские и английские
работы~---  на языке и в алфавите оригинала;\\[-14.5pt]
\item  \textbf{References}. Русские работы и работы на других языках~--- в латинской
транслитерации с переводом на английский язык; английские работы и работы на других
языках~--- на языке оригинала.
\end{enumerate}

Необходимо для составления списка ``References'' пользоваться размещенной на сайте
{\sf http://www. translit.net/ru/bgn/} бесплатной программой транслитерации русского
 текста в~латиницу. %, при этом в~за\-клад\-ке <<варианты\ldots>> следует выбратьопцию BGN.

Список литературы ``References'' приводится полностью отдельным блоком, повторяя все
позиции из списка литературы к русскоязычной части, независимо от того, имеются или
нет в нем иностранные источники. Если в списке литературы к русскоязычной части есть
ссылки на иностранные публикации, набранные латиницей, они полностью повторяются в
списке ``References''.

Ниже приведены примеры ссылок на различные виды публикаций в списке ``References''.

\def\leftfootline{\small{\textbf{\thepage}
\hfill ИНФОРМАТИКА И ЕЁ ПРИМЕНЕНИЯ\ \ \ том\ 18\ \ \ выпуск\ 3\ \ \ 2024}
}%
 \def\rightfootline{\small{ИНФОРМАТИКА И ЕЁ ПРИМЕНЕНИЯ\ \ \ том\ 18\ \ \ выпуск\ 3\ \ \ 2024
\hfill \textbf{\thepage}}}

{\small

\noindent
\textbf{Описание статьи из журнала:}

\Aue{Zagurenko, A.\,G., V.\,A.~Korotovskikh, A.\,A.~Kolesnikov, A.\,V.~Timonov, and D.\,V.~Kardymon}. 2008.
Tekhniko-ekonomicheskaya optimizatsiya dizayna gidrorazryva plasta [Technical and
economic optimization of the design
of hydraulic fracturing]. \textit{Neftyanoe hozyaystvo} [\textit{Oil Industry}] 11:54--57.

\Aue{Zhang, Z., and D.~Zhu}. 2008. Experimental research on the localized
electrochemical micromachining. \textit{Russ. J.~Electrochem.}  44(8):926--930.
{\sf doi:10.1134/S1023193508080077}.

\noindent
\textbf{Описание статьи из электронного журнала:}

\Aue{Swaminathan, V., E.~Lepkoswka-White, and B.\,P.~Rao}. 1999. Browsers or buyers in cyberspace? An
investigation of electronic factors influencing electronic exchange. \textit{JCMC}
5(2). Available at: {\sf http://www.ascusc.org/jcmc/vol5/issue2/} (accessed April~28, 2011).

\def\leftkol{Правила подготовки рукописей  для публикации в журнале
<<Информатика и её применения>>}

\def\rightkol{Правила подготовки рукописей  для публикации в журнале
<<Информатика и её применения>>}


\noindent
\textbf{Описание статьи из продолжающегося издания (сборника трудов):}

\Aue{Astakhov, M.\,V., and T.\,V.~Tagantsev}. 2006. Eksperimental'noe
issledovanie prochnosti soedineniy ``stal'--kompozit'' [Experimental study of
the strength of joints ``steel--composite'']. \textit{Trudy MGTU
``Matematicheskoe modelirovanie slozhnykh tekh\-ni\-che\-skikh sistem''}
[\textit{Bauman MSTU ``Mathematical Modeling of Complex Technical
Systems'' Proceedings}]. 593:125--130.


\pagebreak



\noindent
\textbf{Описание материалов конференций:}

\Aue{Usmanov, T.\,S., A.\,A.~Gusmanov, I.\,Z.~Mullagalin, R.\,Ju.~Muhametshina, A.\,N.~Chervyakova, and
A.\,V.~Sveshnikov}. 2007. Osobennosti proektirovaniya razrabotki mestorozhdeniy
s primeneniem gidrorazryva
plasta [Features of the design of field development with the use of hydraulic fracturing].
\textit{Trudy 6-go
Mezhdu\-na\-rod\-no\-go Simpoziuma ``Novye resursosberegayushchie tekhnologii nedropol'zovaniya i povysheniya
neftegazootdachi''} [\textit{6th  Symposium (International) ``New Energy Saving Subsoil Technologies and
the Increasing of the Oil and Gas Impact'' Proceedings}]. Moscow. 267--272.



\def\leftfootline{\small{\textbf{\thepage}
\hfill ИНФОРМАТИКА И ЕЁ ПРИМЕНЕНИЯ\ \ \ том\ 18\ \ \ выпуск\ 3\ \ \ 2024}
}%
 \def\rightfootline{\small{ИНФОРМАТИКА И ЕЁ ПРИМЕНЕНИЯ\ \ \ том\ 18\ \ \ выпуск\ 3\ \ \ 2024
\hfill \textbf{\thepage}}}



\noindent
\textbf{Описание книги (монографии, сборники):}



Lindorf, L.\,S., and L.\,G.~Mamikoniants, eds. 1972.
\textit{Ekspluatatsiya turbogeneratorov s neposredstvennym
okhlazhdeniem} [\textit{Operation of turbine generators with direct cooling}].
Moscow: Energy Publs. 352~p.


\Aue{Latyshev, V.\,N.} 2009. \textit{Tribologiya rezaniya. Kn.~1: Friktsionnye protsessy
pri rezanii metallov}
[\textit{Tribology of cutting. Vol.~1: Frictional processes in metal cutting}]. Ivanovo: Ivanovskii
State Univ. 108~p.

\def\leftkol{Правила подготовки рукописей  для публикации в журнале
<<Информатика и её применения>>}

\def\rightkol{Правила подготовки рукописей  для публикации в журнале
<<Информатика и её применения>>}

\noindent
\textbf{Описание переводной книги}
(в списке литературы к русскоязычной части необходимо указать:~/ Пер.\ с англ.~---
после названия книги, а в конце ссылки указать оригинал книги в круглых скобках):
\begin{enumerate}[1.]
\item  В русскоязычной части:

\def\leftfootline{\small{\textbf{\thepage}
\hfill ИНФОРМАТИКА И ЕЁ ПРИМЕНЕНИЯ\ \ \ том\ 18\ \ \ выпуск\ 3\ \ \ 2024}
}%
 \def\rightfootline{\small{ИНФОРМАТИКА И ЕЁ ПРИМЕНЕНИЯ\ \ \ том\ 18\ \ \ выпуск\ 3\ \ \ 2024
\hfill \textbf{\thepage}}}

\Au{Тимошенко С.\,П., Янг Д.\,Х., Уивер~У.}
Колебания в инженерном деле~/ Пер.\ с англ.~--- М.: Машиностроение, 1985. 472~с.
(\Au{Timoshenko~S.\,P., Young~D.\,H., Weaver~W.}
Vibration problems in engineering.~--- 4th ed.~--- New York, NY, USA: Wiley, 1974. 521~p.)\\[-13.5pt]
\item  В англоязычной части:

\Aue{Timoshenko, S.\,P., D.\,H.~Young, and W.~Weaver}.
1974. \textit{Vibration problems in engineering}. 4th ed. New York: 
Wiley. 521~p.
\end{enumerate}

\vspace*{-3pt}


\noindent
\textbf{Описание неопубликованного документа:}


\Aue{Latypov, A.\,R., M.\,M.~Khasanov, and V.\,A.~Baikov}.
2004 (unpubl.). Geologiya i~dobycha (NGT GiD) [Geology and production (NGT GiD)]. Certificate on official registration of the computer program
No.\,2004611198. 

\noindent
\textbf{Описание интернет-ресурса:}


Pravila tsitirovaniya istochnikov [Rules for the citing of sources]. Available at: {\sf
http://www.scribd.com/doc/1034528/} (accessed February~7, 2011).

%\pagebreak

\noindent
\textbf{Описание диссертации или автореферата диссертации:}

\Aue{Semenov, V.\,I.}
2003. Matematicheskoe modelirovanie plazmy v sisteme kompaktnyy tor [Mathematical
modeling of the plasma in the compact torus].  Moscow.  D.Sc.\ Diss. 272~p.

\Aue{Kozhunova, O.\,S.} 2009. Tekhnologiya razrabotki semanticheskogo
slovarya informatsionnogo monitoringa [Technology of development of
semantic dictionary of information monitoring system].  Moscow: IPI RAN. PhD Thesis. 23~p.


\noindent
\textbf{Описание ГОСТа:}

GOST 8.586.5-2005. 2007. Metodika vypolneniya izmereniy. Izmerenie raskhoda i~kolichestva zhidkostey i~gazov
s~pomoshch'yu standartnykh suzhayushchikh ustroystv [Method of measurement.
Measurement of flow rate and volume of liquids and gases by means of orifice devices]. Moscow:
Standardinform  Publs. 10~p.

\noindent
\textbf{Описание патента:}

\Aue{Bolshakov, M.\,V., A.\,V.~Kulakov, A.\,N.~Lavrenov, and M.\,V.~Palkin}.
2006. Sposob orientirovaniya po krenu letatel'nogo
apparata s opti\-che\-skoy golovkoy
samonavedeniya [The way to orient on the roll of aircraft with optical homing head].
Patent RF No.\,2280590.
}

\item Присланные в редакцию материалы авторам не возвращаются.\\[-13.5pt]

\item При отправке файлов по электронной почте просим придерживаться следующих
правил:
\begin{itemize}
\item указывать в поле subject (тема) название журнала и фамилию автора;\\[-13.5pt]
\item указывать в тексте письма название статьи, авторов и~журнал, в~который направляется статья;\\[-13.5pt]
\item использовать attach (присоединение);\\[-13.5pt]
\item в состав электронной версии статьи должны входить: файл, содержащий текст
статьи, и файл(ы), содержащий(е) иллюстрации.\\[-13.5pt]
\end{itemize}

\item Журнал <<Информатика и её применения>> является некоммерческим изданием.
Плата за публикацию не взимается, гонорар авторам не выплачивается.
\end{enumerate}



\def\leftfootline{\small{\textbf{\thepage}
\hfill ИНФОРМАТИКА И ЕЁ ПРИМЕНЕНИЯ\ \ \ том\ 18\ \ \ выпуск\ 3\ \ \ 2024}
}%
 \def\rightfootline{\small{ИНФОРМАТИКА И ЕЁ ПРИМЕНЕНИЯ\ \ \ том\ 18\ \ \ выпуск\ 3\ \ \ 2024
\hfill \textbf{\thepage}}}


\vspace*{-1mm}

\begin{center}

\textbf{Адрес редакции журнала <<Информатика и её применения>>:} \\




Москва 119333, ул.~Вавилова, д.~44, корп.~2, ФИЦ ИУ РАН\\[-10pt]

\

Тел.: +7\,(499)\,135-86-92\ \ Факс:  +7\,(495)\,930-45-05\\[-10pt]

 \

e-mail:   {\sf iiep@frccsc.ru} (Стригина Светлана Николаевна)\\[-10pt]

\

{\sf http://www.ipiran.ru/journal/issues/}
\end{center}
}


\def\leftkol{Правила подготовки рукописей  для публикации в журнале
<<Информатика и её применения>>}

\def\rightkol{Правила подготовки рукописей  для публикации в журнале
<<Информатика и её применения>>}


\def\leftfootline{\small{\textbf{\thepage}
\hfill ИНФОРМАТИКА И ЕЁ ПРИМЕНЕНИЯ\ \ \ том\ 18\ \ \ выпуск\ 3\ \ \ 2024}
}%
 \def\rightfootline{\small{ИНФОРМАТИКА И ЕЁ ПРИМЕНЕНИЯ\ \ \ том\ 18\ \ \ выпуск\ 3\ \ \ 2024
\hfill \textbf{\thepage}}} 
\def\stat{podg-e}
{%\hrule\par
%\vskip 7pt % 7pt
\vspace*{-24pt}
\raggedleft\Large \bf%\baselineskip=3.2ex
Requirements for manuscripts submitted to Journal
``Informatics~and~Applications'' \vskip 8pt
    \hrule
    \par
\vskip 21pt plus 6pt minus 3pt }

\label{st\stat}

\def\tit{\ }

\def\aut{\ }
\def\auf{\ }

\def\leftkol{\ }

\def\rightkol{\ }
%Requirements for manuscripts submitted to Journal
%``Informatics~and~Applications''}

\titele{\tit}{\aut}{\auf}{\leftkol}{\rightkol}

\def\leftfootline{\small{\textbf{\thepage}
\hfill INFORMATIKA I EE PRIMENENIYA~--- INFORMATICS AND APPLICATIONS\ \ \ 2019\
\ \ volume~13\ \ \ issue\ 4}
}%
 \def\rightfootline{\small{INFORMATIKA I EE PRIMENENIYA~--- INFORMATICS AND APPLICATIONS\ \ \ 2019\ \ \ volume~13\ \ \ issue\ 4
\hfill \textbf{\thepage}}}

\vspace*{-60pt}

{\small

\noindent
Journal ``Informatics and Applications'' (Inform.\ Appl.)
publishes theoretical, review, and discussion
articles on the research and development in the
field of informatics and its applications.

The journal is published in Russian.
By a special decision of the editorial
board, some articles can be published in English.


The topics covered include the following areas:
\begin{itemize}
               \item
     theoretical fundamentals of informatics; \\[-14pt]
\item
mathematical methods for studying complex systems and processes; \\[-14pt]
\item
information systems and networks;\\[-14pt]
\item
information technologies; and \\[-14pt]
\item
architecture and software of computational complexes and networks. \\[-14pt]
\end{itemize}

\noindent
\begin{enumerate}[1.]
\item The Journal publishes original articles which have not been published before and are not
intended for simultaneous publication in other editions. An article submitted to the Journal must not violate the
Copyright law. Sending the manuscript to the Editorial Board, the authors retain all rights of the
owners of the manuscript and transfer the nonexclusive rights to publish the article in Russian
(or the language of the article, if not Russian) and its distribution in Russia and abroad to the
Founders and the Editorial Board. Authors should submit a letter to the Editorial Board in the
following form:

{\bfseries\textit{Agreement on the transfer of rights to publish:}}

``\textit{We, the undersigned authors of the manuscript ``\ldots'', pass to the
Founder and the Editorial Board of the Journal ``Informatics and Applications''
the nonexclusive right to publish the manuscript of the article in Russian (or
in English) in both print and electronic versions of the Journal. We affirm
that this publication does not violate the Copyright of other persons or
organizations.}

\textit{Author(s) signature(s): (name(s), address(es), date).}

This agreement should be submitted in paper form or in the form of a scanned copy (signed by
the authors).


%The Editorial Board has the right to request from the authors an official expert conclusion that
%the submitted article has no secret data prohibited for publication. \\[-13.5pt]
\item
A submitted article should be attached with \textbf{the data on the author(s)} (see item~8). If
there are several authors, the contact person should be indicated who is responsible for
correspondence with the Editorial Board and other authors about revisions and final approval
of the proofs.\\[-13.5pt]

\item The Editorial Board of the Journal examines the article according to the established
reviewing procedure. If the authors receive their article for correction after reviewing, it does not
mean that the article is approved for publication. The corrected article should be sent to the
Editorial Board for the subsequent review and approval.\\[-13.5pt]

\item The decision on the article publication or its rejection is communicated to the authors. The
Editorial Board may also send the reviews on the submitted articles to the authors. Any
discussion upon the rejected articles is not possible.\\[-13.5pt]

\item The edited articles will be sent to the authors for proofread. The comments of the authors
to the edited text of the article should be sent to the Editorial Board as soon as possible.\\[-13.5pt]

\item The manuscript of the article should be presented electronically in the MS WORD (.doc or
.docx) or \LaTeX\ (.tex) formats, and additionally in the .pdf format. All documents
 may be sent
by e-mail or provided on a CD or diskette. A~hard copy submission is not necessary.\\[-13.5pt]

\item The recommended typesetting instructions for manuscript.

Pages parameters: format A4, portrait orientation, document margins (cm): left~--- 2.5, right~---
1.5, above~--- 2.0, below~--- 2.0, footer 1.3.

Text: font~---Times New Roman, font size~--- 14, paragraph indent~--- 0.5, line spacing~--- 1.5,
justified alignment.

The recommended manuscript size: not more than 15~pages of the specified format.
If the specified size exceeded, the editorial board is entitled to require the author
to reduce the manuscript.

Use only standard abbreviations. Avoid  abbreviations in the title and
abstract. The full term for which an abbreviation stands should precede
its first use in the text unless it is a standard unit of measurement.

All pages of the manuscript should be numbered.

The templates for the manuscript typesetting are presented on site: {\sf
http://www.ipiran.ru/journal/template.doc}.\\[-13.5pt]


%\def\leftkol{Requirements for manuscripts submitted to Journal
%``Informatics~and~Applications''}

\item The articles should enclose data both in \textbf{Russian and English}:
\begin{itemize}
\item title;\\[-13.5pt]
\item author's name and surname;\\[-13.5pt]
\item affiliation~--- organization, its address with ZIP code, city, country, and
official e-mail address;\\[-13.5pt]
\item data on authors according to the format: (see site)

{\sf http://www.ipiran.ru/journal/issues/2013\_07\_01/authors.asp}  and

{\sf  http://www.ipiran.ru/journal/issues/2013\_07\_01\_eng/authors.asp};\\[-13.5pt]

\pagebreak

\def\leftfootline{\small{\textbf{\thepage}
\hfill INFORMATIKA I EE PRIMENENIYA~--- INFORMATICS AND APPLICATIONS\ \ \ 2019\
\ \ volume~13\ \ \ issue\ 4}
}%
 \def\rightfootline{\small{INFORMATIKA I EE PRIMENENIYA~--- INFORMATICS AND APPLICATIONS\ \ \ 2019\ \ \ volume~13\ \ \ issue\ 4
\hfill \textbf{\thepage}}}


%\def\leftkol{Requirements for manuscripts submitted to Journal
%``Informatics~and~Applications''}

%\def\rightkol{Requirements for manuscripts submitted to Journal
%``Informatics~and~Applications''}



\item abstract (not less than 100 words) both in Russian and in English. Abstract is a short
summary of the article that can be published separately. The abstract is the
main source of information on the article and it could be included in leading information
systems and data bases. The abstract in English has to be an original text and should
not be an exact translation of the Russian one. Good English is required.
In abstracts, avoid references and formulae;\\[-13.5pt]
\item indexing is performed on the basis of keywords. The use of keywords from the
internationally accepted thematic Thesauri is recommended.

%\def\leftkol{Requirements for manuscripts submitted to Journal
%``Informatics~and~Applications''}

%\def\rightkol{Requirements for manuscripts submitted to Journal
%``Informatics~and~Applications''}

Important! Keywords must not be sentences;
\item Acknowledgments.
\end{itemize}

\item References. Russian references have to be presented both in English translation and Latin
transliteration (refer {\sf http://www.translit.net/ru/bgn/}).

Please take into account the following examples of Russian references appearance:

\noindent
\textbf{Article in journal:}

\Aue{Zhang, Z., and D.~Zhu}. 2008. Experimental research on the localized electrochemical
micromachining.
\textit{Rus. J.~Electrochem.}  44(8):926--930. {\sf doi:10.1134/S1023193508080077}.


\noindent
\textbf{Journal article in electronic format:}

\Aue{Swaminathan, V., E.~Lepkoswka-White, and B.\,P.~Rao}. 1999. Browsers or buyers in
cyberspace? An
investigation of electronic factors influencing electronic exchange. \textit{JCMC}
5(2). Available at: {\sf http://www.ascusc.org/jcmc/vol5/issue2/} (accessed April~28, 2011).




\noindent
\textbf{Article from the continuing publication (collection of works, proceedings):}

\Aue{Astakhov, M.\,V., and T.\,V.~Tagantsev}. 2006. Eksperimental'noe
issledovanie prochnosti soedineniy ``stal'--kompozit'' [Experimental study of
the strength of joints ``steel--composite'']. \textit{Trudy MGTU
``Matematicheskoe modelirovanie slozhnykh tekh\-ni\-che\-skikh sistem''}
[\textit{Bauman MSTU ``Mathematical Modeling of Complex Technical
Systems'' Proceedings}]. 593:125--130.

\def\leftfootline{\small{\textbf{\thepage}
\hfill INFORMATIKA I EE PRIMENENIYA~--- INFORMATICS AND APPLICATIONS\ \ \ 2019\
\ \ volume~13\ \ \ issue\ 4}
}%
 \def\rightfootline{\small{INFORMATIKA I EE PRIMENENIYA~--- INFORMATICS AND APPLICATIONS\ \ \ 2019\ \ \ volume~13\ \ \ issue\ 4
\hfill \textbf{\thepage}}}

\def\leftkol{Requirements for manuscripts submitted to Journal
``Informatics~and~Applications''}

\def\rightkol{Requirements for manuscripts submitted to Journal
``Informatics~and~Applications''}

\noindent
\textbf{Conference proceedings:}

\Aue{Usmanov, T.\,S., A.\,A.~Gusmanov, I.\,Z.~Mullagalin, R.\,Ju.~Muhametshina,
A.\,N.~Chervyakova, and
A.\,V.~Sveshnikov}. 2007. Osobennosti proektirovaniya razrabotki mestorozhdeniy
s primeneniem gidrorazryva
plasta [Features of the design of field development with the use of hydraulic fracturing].
\textit{Trudy 6-go
Mezhdu\-na\-rod\-no\-go Simpoziuma ``Novye resursosberegayushchie tekhnologii
nedropol'zovaniya i povysheniya
neftegazootdachi''} [\textit{6th  Symposium (International) ``New Energy Saving Subsoil
Technologies and
the Increasing of the Oil and Gas Impact'' Proceedings}]. Moscow. 267--272.


\noindent
\textbf{Books and other monographs:}




Lindorf, L.\,S., and L.\,G.~Mamikoniants, eds. 1972.
\textit{Ekspluatatsiya turbogeneratorov s neposredstvennym
okhlazhdeniem} [\textit{Operation of turbine generators with direct cooling}].
Moscow: Energy Publs. 352~p.


%\Aue{Latyshev, V.\,N.} 2009. \textit{Tribologiya rezaniya. Kn.~1: Frikcionnye prosessy
%pri rezanii metallov}
%[\textit{Tribology of cutting. Vol.~1: Frictional processes in metal cutting}]. Ivanovo: Ivanovskii
%State Univ. 108~p.


%\noindent
%\textbf{Unpublished material:}

%\Aue{Latypov, A.\,R., M.\,M.~Khasanov, and V.\,A.~Baikov}.
%2004. Geology and production (NGT GiD). Certificate on official registration of the computer
%program
%No.\,2004611198. (In Russian, unpubl.)

%\noindent
%\textbf{Internet-source:}

%APA Style. 2011. Available at: {\sf http://www.apastyle.org/apa-style-help.aspx} (accessed
%February~5, 2011).

%Pravila citirovaniya istochnikov [Rules for the citing of sources]. Available at: {\sf
%http://www.scribd.com/doc/1034528/} (accessed February~7, 2011).


\noindent
\textbf{Dissertation and Thesis:}

%\Aue{Semenov, V.\,I.}
%2003. Matematicheskoe modelirovanie plazmy v sisteme kompaktnyy tor. [Mathematical
%modeling of the plasma in the compact torus]. D.Sc.\ Diss. Moscow. 272~p.

\Aue{Kozhunova, O.\,S.} 2009. Tekhnologiya razrabotki semanticheskogo
slovarya informatsionnogo monitoringa [Technology of development of
semantic dictionary of information monitoring system]. PhD Thesis. Moscow: IPI RAN. 23~p.


\noindent
\textbf{State standards and patents:}

GOST 8.586.5-2005. 2007. Metodika vypolneniya izmereniy. Izmerenie raskhoda i~kolichestva
zhidkostey i gazov 
s~pomoshch'yu standartnykh suzhayushchikh ustroystv [Method of measurement.
Measurement of flow rate and volume of liquids and gases by means of orifice devices]. M.:
Standardinform
Publs. 10~p.

%\noindent
%\textbf{Patent:}

\Aue{Bolshakov, M.\,V., A.\,V.~Kulakov, A.\,N.~Lavrenov, and M.\,V.~Palkin}.
2006. Sposob orientirovaniya po krenu letatel'nogo
apparata s opti\-che\-skoy golovkoy
samonavedeniya [The way to orient on the roll of aircraft with optical homing head].
Patent RF No.\,2280590.

References in Latin transcription are presented in the original language.

References in the text are numbered according to the order of their
first appearance; the number is
placed in square brackets. All items from the reference list should be
cited.\\[-13.5pt]

\item Manuscripts and additional materials are not returned to Authors by the Editorial Board.\\[-13.5pt]

\item Submissions of files by e-mail must include:\\[-13.5pt]
\begin{itemize}
\item   the journal title and author's name in the ``Subject'' field; \\[-13.5pt]
\item   an article and additional materials have to be attached using the ``attach'' function;\\[-13.5pt]
\item   an electronic version of the article should contain the file with the text and a separate file
with figures.\\[-13.5pt]
\end{itemize}

\item ``Informatics and Applications'' journal is not a profit publication. There are no
charges for the authors as well as there are no royalties.\\[-13.5pt]
\end{enumerate}

\def\leftfootline{\small{\textbf{\thepage}
\hfill INFORMATIKA I EE PRIMENENIYA~--- INFORMATICS AND APPLICATIONS\ \ \ 2019\
\ \ volume~13\ \ \ issue\ 4}
}%
 \def\rightfootline{\small{INFORMATIKA I EE PRIMENENIYA~--- INFORMATICS AND APPLICATIONS\ \ \ 2019\ \ \ volume~13\ \ \ issue\ 4
\hfill \textbf{\thepage}}}

\def\leftkol{Requirements for manuscripts submitted to Journal
``Informatics~and~Applications''}

\def\rightkol{Requirements for manuscripts submitted to Journal
``Informatics~and~Applications''}


%\vspace*{5mm}


\begin{center}
\textbf{Editorial Board address:} \\

%ABOUT AUTHORS



FRC CSC RAS, 44, block~2, Vavilov Str., Moscow 119333, Russia\\[-10pt]

\

Ph.: +7\,(499)\,135\,86\,92,\ \ Fax: +7\,(495)\,930\,45\,05\\[-10pt]

\

 e-mail: {\sf rust@ipiran.ru} (to Prof.\ Rustem Seyful-Mulyukov)\\[-10pt]

\

 {\sf http://www.ipiran.ru/english/journal.asp}
\end{center}
 }
%\thispagestyle{myheadings}

\def\leftkol{Requirements for manuscripts submitted to Journal
``Informatics~and~Applications''}

\def\rightkol{Requirements for manuscripts submitted to Journal
``Informatics~and~Applications''}

\def\leftfootline{\small{\textbf{\thepage}
\hfill INFORMATIKA I EE PRIMENENIYA~--- INFORMATICS AND APPLICATIONS\ \ \ 2019\
\ \ volume~13\ \ \ issue\ 4}
}%
 \def\rightfootline{\small{INFORMATIKA I EE PRIMENENIYA~--- INFORMATICS AND APPLICATIONS\ \ \ 2019\ \ \ volume~13\ \ \ issue\ 4
\hfill \textbf{\thepage}}}

 \label{end\stat}

\newpage

%\vspace*{-60pt} {\small
{\baselineskip=9.1pt
\section*{Правила подготовки рукописей статей для публикации в журнале
<<Информатика и её применения>>}

\thispagestyle{empty}

 Журнал <<Информатика и её применения>> публикует
теоретические, обзорные и дискуссионные статьи, посвященные научным
исследованиям и разработкам в области информатики и ее приложений. Журнал
издается на русском языке. По специальному решению редколлегии отдельные статьи,
в виде исключения, могут печататься на английском языке.
Тематика журнала охватывает следующие направления:
\begin{itemize}
\item теоретические основы информатики; %\\[-13.5pt]
\item математические методы исследования сложных систем и процессов; %\\[-13.5pt]
\item информационные системы и сети; %\\[-13.5pt]
\item информационные технологии; %\\[-13.5pt]
\item архитектура и программное
обеспечение вычислительных комплексов и сетей.
\end{itemize}
\begin{enumerate}
\item В журнале печатаются результаты, ранее не
опубликованные и не предназначенные к одновременной публикации в других
изданиях. Публикация не должна нарушать закон об авторских правах. Направляя
свою рукопись в редакцию, авторы автоматически передают учредителям и
редколлегии неисключительные права на издание данной статьи на русском языке и
на ее распространение в России и за рубежом. При этом за авторами сохраняются
все права как собственников данной рукописи. В связи с этим авторами должно
быть представлено в редакцию письмо в следующей форме:
Соглашение о передаче права на публикацию:

\textit{<<Мы, нижеподписавшиеся, авторы рукописи <<$\qquad\qquad$>>, передаем
учредителям и редколлегии журнала <<Информатика и её применения>>
неисключительное право опубликовать данную рукопись статьи на русском языке как
в печатной, так и в электронной версиях журнала. Мы подтверждаем, что данная
публикация не нарушает авторского права других лиц или организаций. Подписи
авторов: (ф.\,и.\,о., дата, адрес)>>.}

Указанное соглашение может быть представлено 
как в бумажном виде, так и в виде отсканированной копии (с подписями авторов).


Редколлегия вправе запросить у авторов экспертное заключение о возможности
опубликования представленной статьи в открытой печати. %\\[-13.5pt]
\item Статья
подписывается всеми авторами. На отдельном листе представляются данные автора
(или всех авторов): фамилия, полные имя и отчество, телефон, факс, e-mail,
почтовый адрес. Если работа выполнена несколькими авторами, указывается фамилия
одного из них, ответственного за переписку с редакцией. %\\[-13.5pt]
\item Редакция журнала
осуществляет самостоятельную экспертизу присланных статей. Возвращение рукописи
на доработку не означает, что статья уже принята к печати. Доработанный вариант
с ответом на замечания рецензента необходимо прислать в редакцию. %\\[-13.5pt]
\item Решение
редакционной коллегии о принятии статьи к печати или ее отклонении сообщается
авторам. Редколлегия не обязуется направлять рецензию авторам отклоненной
статьи. %\\[-13.5pt]
\item Корректура статей высылается авторам для просмотра. Редакция
просит авторов присылать свои замечания в кратчайшие сроки. %\\[-13.5pt]
\item При
подготовке рукописи в MS Word рекомендуется использовать следующие настройки.
Параметры страницы: формат~--- А4; ориентация~--- книжная; поля (см): внутри~---
2,5, снаружи~--- 1,5, сверху~--- 2, снизу~--- 2, от края до нижнего
колонтитула~--- 1,3. Основной текст: стиль~--- <<Обычный>>: шрифт Times New
Roman, размер 14~пунктов, абзацный отступ~--- 0,5~см, 1,5 интервала,
выравнивание~--- по ширине. Рекомендуемый объем рукописи~--- не свыше
25~страниц указанного формата. Ознакомиться с шаблонами, содержащими примеры
оформления, можно по адресу в Интернете:
\textsf{http://www.ipiran.ru/journal/template.doc}.
\item К рукописи, предоставляемой в 2-х
экземплярах, обязательно прилагается электронная версия статьи (как правило, в
форматах MS WORD (.doc) или \LaTeX\ (.tex), а также~--- дополнительно~--- в
формате .pdf) на дискете, лазерном диске или по электронной почте. Сокращения
слов, кроме стандартных, не применяются. Все страницы рукописи должны быть
пронумерованы. %\\[-13.5pt]
\item Статья должна содержать следующую информацию на русском и
английском языках: название, Ф.И.О. авторов, места работы авторов и их
электронные адреса, подробные сведения об авторах, оформленные в соответствии с форматом, 
определяемым файлами {\sf http://www.ipiran.ru/journal/issues/2011\_05\_01/authors.asp} и 
{\sf http://www.ipiran.ru/journal/issues/2011\_01\_eng/authors.asp},
аннотация (не более 100~слов), ключевые слова. Ссылки на
литературу в тексте статьи нумеруются (в квадратных скобках) и располагаются в
порядке их первого упоминания. В~списке литературы не должно быть позиций, на которые нет ссылки в тексте статьи.
Все фамилии авторов, заглавия статей, названия
книг, конференций и~т.\,п.\ даются на языке оригинала, если этот язык
использует кириллический или латинский алфавит. %\\[-13.5pt]
\item Присланные в редакцию материалы авторам не возвращаются.
\item При отправке файлов по электронной
почте просим придерживаться следующих правил:
\begin{itemize}
\item указывать в поле subject (тема) название журнала и фамилию автора; %\\[-13.5pt]
\item использовать attach (присоединение); %\\[-13.5pt]
\item в случае больших объемов информации возможно
использование общеизвестных архиваторов (ZIP, RAR); %\\[-13.5pt]
\item в состав электронной версии статьи должны входить: файл, содержащий текст статьи, и файл(ы),
содержащий(е) иллюстрации. %\\[-13.5pt]
\end{itemize}
\item Журнал <<Информатика и её применения>> является некоммерческим изданием. 
Плата за публикацию с авторов не взимается, гонорар авторам не выплачивается.
\end{enumerate}
\thispagestyle{empty}
\textbf{Адрес редакции:} Москва 119333,
ул.~Вавилова, д.~44, корп.~2, ИПИ РАН\\
\hphantom{\textbf{Адрес редакции:} }Тел.: +7 (499) 135-86-92\ \
Факс:  +7 (495) 930-45-05\ \  E-mail:   rust@ipiran.ru }
}

%\include{ipi-ind}

%\tableofcontents

\end{document}


%\tableofcontents

%\end{document}





%\def\stat{cont}
{%\hrule\par
%\vskip 7pt % 7pt
\raggedleft\Large \bf%\baselineskip=3.2ex
А\,В\,Т\,О\,Р\,С\,К\,И\,Й\ \ У\,К\,А\,З\,А\,Т\,Е\,Л\,Ь\ \ З\,А\ \ 2\,0\,0\,7 г. \vskip 17pt
    \hrule
    \par
\vskip 21pt plus 6pt minus 3pt }

\label{st\stat}

\def\tit{\ }

\def\aut{\ }
\def\auf{\ }

\def\leftkol{\ } % ENGLISH ABSTRACTS}

\def\rightkol{\ } %ENGLISH ABSTRACTS}

\titele{\tit}{\aut}{\auf}{\leftkol}{\rightkol}


\contentsline {chapter}{\ }{Выпуск \quad Стр.} 
\contentsline {section}{\textbf{Батракова Д.\,А., Королев В.\,Ю., Шоргин С.\,Я.}\ \ Новый метод вероятностно-ста\-ти\-сти\-че\-ско\-го анализа информационных потоков в\nobreakspace {}телекоммуникационных сетях}{\qquad 1 \qquad 40} 
\contentsline {section}{\textbf{Борисов А.\,В.}\ \ Байесовское оценивание в системах наблюдения с\nobreakspace {}марковскими скачкообразными процессами: игровой подход}{\qquad 2 \qquad 65}
\contentsline {section}{\textbf{Босов А.\,В., Иванов А.\,В.}\ \ Программная инфраструктура информационного Web-пор\-тала}{\qquad 2 \qquad 50}
\contentsline {section}{\textbf{Захаров В.\,Н., Калиниченко Л.\,А., Соколов И.\,А., Ступников С.\,А.}\ \ Конструирование канонических информационных моделей для интегрированных информационных систем}{\qquad 2 \qquad 15}
\contentsline {section}{\textbf{Захаров В.\,Н., Козмидиади В.\,А.}\ \ Средства обеспечения отказоустойчивости при\-ло\-жений}{\qquad 1 \qquad 14} 
\contentsline {section}{\textbf{Иванов А.\,В.}\ \ см. Босов А.\,В.\hfill\hfill\hfill\hfill\hfill\hfill\hfill\hfill\hfill\hfill\hfill\hfill\hfill\hfill\hfill\hfill\hfill\hfill\hfill\hfill\hfill\hfill\hfill\hfill\hfill\hfill\hfill\hfill\hfill\hfill\hfill\hfill\hfill\hfill\hfill}{\ }
\contentsline {section}{\textbf{Ильин В.\,Д., Соколов И.\,А.}\ \ Символьная модель системы знаний информатики в\nobreakspace {}че\-ло\-ве\-ко-автоматной среде}{\qquad 1 \qquad 66} 
\contentsline {section}{\textbf{Калиниченко Л.\,А.}\ \ см. Захаров В.\,Н.\hfill\hfill\hfill\hfill\hfill\hfill\hfill\hfill\hfill\hfill\hfill\hfill\hfill\hfill\hfill\hfill\hfill\hfill\hfill\hfill\hfill\hfill\hfill\hfill\hfill\hfill\hfill\hfill\hfill\hfill\hfill\hfill\hfill\hfill\hfill}{\ }
\contentsline {section}{\textbf{Козеренко Е.\,Б.}\ \ Лингвистическое моделирование для систем машинного перевода и обработки знаний}{\qquad 1 \qquad 54} 
\contentsline {section}{\textbf{Козмидиади В.\,А.}\ \ см. Захаров В.\,Н.\hfill\hfill\hfill\hfill\hfill\hfill\hfill\hfill\hfill\hfill\hfill\hfill\hfill\hfill\hfill\hfill\hfill\hfill\hfill\hfill\hfill\hfill\hfill\hfill\hfill\hfill\hfill\hfill\hfill\hfill\hfill\hfill\hfill\hfill\hfill }{\ } 
\contentsline {section}{\textbf{Королев В.\,Ю.}\ \ см. Батракова Д.\,А.\hfill\hfill\hfill\hfill\hfill\hfill\hfill\hfill\hfill\hfill\hfill\hfill\hfill\hfill\hfill\hfill\hfill\hfill\hfill\hfill\hfill\hfill\hfill\hfill\hfill\hfill\hfill\hfill\hfill\hfill\hfill\hfill\hfill\hfill\hfill}{\ } 
\contentsline {section}{\textbf{Кудрявцев А.\,А., Шоргин С.\,Я.}\ \ Байесовский подход к\nobreakspace {}анализу систем массового обслуживания и\nobreakspace {}показателей надежности}{\qquad 2 \qquad 76}
\contentsline {section}{\textbf{Печинкин А.\,В., Соколов И.\,А., Чаплыгин В.\,В.}\ \ Многолинейная система массового обслуживания с конечным накопителем и ненадежными приборами}{\qquad 1 \qquad 27} 
\contentsline {section}{\textbf{Печинкин А.\,В., Соколов И.\,А., Чаплыгин В.\,В.}\ \ Стационарные характеристики многолинейной\nobreakspace {}системы массового обслуживания с\nobreakspace {}одновременными отказами приборов}{\qquad 2 \qquad 39}
\contentsline {section}{\textbf{Синицын И.\,Н.}\ \ Корреляционные методы построения аналитических информационных моделей флуктуаций полюса Земли по априорным данным}{\qquad 2 \qquad \hphantom{9}2}
\contentsline {section}{\textbf{Синицын И.\,Н.}\ \ Развитие теории фильтров Пугачева для оперативной обработки информации в стохастических системах}{{\qquad 1 \qquad \hphantom{9}3}} 
\contentsline {section}{\textbf{Соколов И.\,А.}\ \ см. Захаров В.\,Н.\hfill\hfill\hfill\hfill\hfill\hfill\hfill\hfill\hfill\hfill\hfill\hfill\hfill\hfill\hfill\hfill\hfill\hfill\hfill\hfill\hfill\hfill\hfill\hfill\hfill\hfill\hfill\hfill\hfill\hfill\hfill\hfill\hfill\hfill\hfill}{\ }
\contentsline {section}{\textbf{Соколов И.\,А.}\ \ см. Ильин В.\,Д.\hfill\hfill\hfill\hfill\hfill\hfill\hfill\hfill\hfill\hfill\hfill\hfill\hfill\hfill\hfill\hfill\hfill\hfill\hfill\hfill\hfill\hfill\hfill\hfill\hfill\hfill\hfill\hfill\hfill\hfill\hfill\hfill\hfill\hfill\hfill}{\ } 
\contentsline {section}{\textbf{Соколов И.\,А.}\ \ см. Печинкин А.\,В.\hfill\hfill\hfill\hfill\hfill\hfill\hfill\hfill\hfill\hfill\hfill\hfill\hfill\hfill\hfill\hfill\hfill\hfill\hfill\hfill\hfill\hfill\hfill\hfill\hfill\hfill\hfill\hfill\hfill\hfill\hfill\hfill\hfill\hfill\hfill}{\ } 
\contentsline {section}{\textbf{Соколов И.\,А.}\ \ см. Печинкин А.\,В.\hfill\hfill\hfill\hfill\hfill\hfill\hfill\hfill\hfill\hfill\hfill\hfill\hfill\hfill\hfill\hfill\hfill\hfill\hfill\hfill\hfill\hfill\hfill\hfill\hfill\hfill\hfill\hfill\hfill\hfill\hfill\hfill\hfill\hfill\hfill}{\ }
\contentsline {section}{\textbf{Ступников С.\,А.}\ \ см. Захаров В.\,Н.\hfill\hfill\hfill\hfill\hfill\hfill\hfill\hfill\hfill\hfill\hfill\hfill\hfill\hfill\hfill\hfill\hfill\hfill\hfill\hfill\hfill\hfill\hfill\hfill\hfill\hfill\hfill\hfill\hfill\hfill\hfill\hfill\hfill\hfill\hfill}{\ }
\contentsline {section}{\textbf{Чаплыгин В.\,В.}\ \ см. Печинкин А.\,В.\hfill\hfill\hfill\hfill\hfill\hfill\hfill\hfill\hfill\hfill\hfill\hfill\hfill\hfill\hfill\hfill\hfill\hfill\hfill\hfill\hfill\hfill\hfill\hfill\hfill\hfill\hfill\hfill\hfill\hfill\hfill\hfill\hfill\hfill\hfill}{\ } 
\contentsline {section}{\textbf{Чаплыгин В.\,В.}\ \ см. Печинкин А.\,В.\hfill\hfill\hfill\hfill\hfill\hfill\hfill\hfill\hfill\hfill\hfill\hfill\hfill\hfill\hfill\hfill\hfill\hfill\hfill\hfill\hfill\hfill\hfill\hfill\hfill\hfill\hfill\hfill\hfill\hfill\hfill\hfill\hfill\hfill\hfill}{\ }
\contentsline {section}{\textbf{Шоргин С.\,Я.}\ \ см. Батракова Д.\,А.\hfill\hfill\hfill\hfill\hfill\hfill\hfill\hfill\hfill\hfill\hfill\hfill\hfill\hfill\hfill\hfill\hfill\hfill\hfill\hfill\hfill\hfill\hfill\hfill\hfill\hfill\hfill\hfill\hfill\hfill\hfill\hfill\hfill\hfill\hfill}{\ } 
\contentsline {section}{\textbf{Шоргин С.\,Я.}\ \ см. Кудрявцев А.\,А.\hfill\hfill\hfill\hfill\hfill\hfill\hfill\hfill\hfill\hfill\hfill\hfill\hfill\hfill\hfill\hfill\hfill\hfill\hfill\hfill\hfill\hfill\hfill\hfill\hfill\hfill\hfill\hfill\hfill\hfill\hfill\hfill\hfill\hfill\hfill}{\ }
%\thispagestyle{myheadings}
\def\leftfootline{\small{\textbf{\thepage}
\hfill ИНФОРМАТИКА И ЕЁ ПРИМЕНЕНИЯ\ \ \ том~1\ \ \ выпуск~2\ \ \ 2007}
}%
 \def\rightfootline{\small{ИНФОРМАТИКА И ЕЁ ПРИМЕНЕНИЯ\ \ \ том~1\ \ \ выпуск~2\ \ \ 2007
 \hfill \textbf{\thepage}}}
 \label{end\stat}

%\def\stat{cont-e}
{%\hrule\par
%\vskip 7pt % 7pt
\raggedleft\Large \bf%\baselineskip=3.2ex
2\,0\,0\,7\ \ A\,U\,T\,H\,O\,R\ \ I\,N\,D\,E\,X \vskip 17pt
    \hrule
    \par
\vskip 21pt plus 6pt minus 3pt }

\label{st\stat}

\def\tit{\ }

\def\aut{\ }
\def\auf{\ }

\def\leftkol{\ } % ENGLISH ABSTRACTS}

\def\rightkol{\ } %ENGLISH ABSTRACTS}

\titele{\tit}{\aut}{\auf}{\leftkol}{\rightkol}


\contentsline {chapter}{\ }{Issue \quad Page} 
\contentsline {subsection}{\textbf{Batrakova D.\,A., Korolev V.\,Yu., Shorgin S.\,Ya.}\ \ A New Method for the Probabilistic and Statistical Analysis of Information Flows in Telecommunication Networks}{\qquad 1 \qquad 40} 
\contentsline {subsection}{\textbf{Borisov A.\,V.}\ \ Bayesian Estimation in\nobreakspace {}Observation Systems with\nobreakspace {}Markov Jump Processes: Game-Theoretic Approach}{\qquad 2 \qquad 65} 
\contentsline {subsection}{\textbf{Bosov A.\,V., Ivanov A.\,V.}\ \ Linguistic Simulation for Machine Translation and Knowledge Management Systems}{\qquad 2 \qquad 50} 
\contentsline {subsection}{\textbf{Chaplygin V.\,V.} see Pechinkin A.\,V.\hfill\hfill\hfill\hfill\hfill\hfill\hfill\hfill\hfill\hfill\hfill\hfill\hfill\hfill\hfill\hfill\hfill\hfill\hfill\hfill\hfill\hfill\hfill\hfill\hfill\hfill\hfill\hfill\hfill\hfill\hfill\hfill\hfill\hfill\hfill}{\ }
\contentsline {subsection}{\textbf{Chaplygin V.\,V.} see Pechinkin A.\,V.\hfill\hfill\hfill\hfill\hfill\hfill\hfill\hfill\hfill\hfill\hfill\hfill\hfill\hfill\hfill\hfill\hfill\hfill\hfill\hfill\hfill\hfill\hfill\hfill\hfill\hfill\hfill\hfill\hfill\hfill\hfill\hfill\hfill\hfill\hfill}{\ }
\contentsline {subsection}{\textbf{Ilyin V.\,D., Sokolov I.\,A.}\ \ The Symbol Model of Informatics Knowledge System in Human-Automaton Environment}{\qquad 1 \qquad 66} 
\contentsline {subsection}{\textbf{Ivanov A.\,V.} see Bosov A.\,V.\hfill\hfill\hfill\hfill\hfill\hfill\hfill\hfill\hfill\hfill\hfill\hfill\hfill\hfill\hfill\hfill\hfill\hfill\hfill\hfill\hfill\hfill\hfill\hfill\hfill\hfill\hfill\hfill\hfill\hfill\hfill\hfill\hfill\hfill\hfill}{\ }
\contentsline {subsection}{\textbf{Kalinichenko L.\,A.} see Zakharov V.\,N.\hfill\hfill\hfill\hfill\hfill\hfill\hfill\hfill\hfill\hfill\hfill\hfill\hfill\hfill\hfill\hfill\hfill\hfill\hfill\hfill\hfill\hfill\hfill\hfill\hfill\hfill\hfill\hfill\hfill\hfill\hfill\hfill\hfill\hfill\hfill}{\ }
\contentsline {subsection}{\textbf{Korolev V.\,Yu.} see Batrakova D.\,A.\hfill\hfill\hfill\hfill\hfill\hfill\hfill\hfill\hfill\hfill\hfill\hfill\hfill\hfill\hfill\hfill\hfill\hfill\hfill\hfill\hfill\hfill\hfill\hfill\hfill\hfill\hfill\hfill\hfill\hfill\hfill\hfill\hfill\hfill\hfill}{\ }
\contentsline {subsection}{\textbf{Kozerenko E.\,B.}\ \ Linguistic Simulation for Machine Translation and Knowledge Management Systems}{\qquad 1 \qquad 54} 
\contentsline {subsection}{\textbf{Kozmidiady V.\,A.} see Zakharov V.\,N.\hfill\hfill\hfill\hfill\hfill\hfill\hfill\hfill\hfill\hfill\hfill\hfill\hfill\hfill\hfill\hfill\hfill\hfill\hfill\hfill\hfill\hfill\hfill\hfill\hfill\hfill\hfill\hfill\hfill\hfill\hfill\hfill\hfill\hfill\hfill}{\ }
\contentsline {subsection}{\textbf{Kudryavtsev A.\,A., Shorgin S.\,Ya.}\ \ Bayesian Approach to Queueing Systems and Reliability Characteristics}{\qquad 2 \qquad 76} 
\contentsline {subsection}{\textbf{Pechinkin A.\,V., Sokolov I.\,A., Chaplygin V.\,V.}\ \ Multichannel Queuing System with Finite Buffer and Unreliable Servers}{\qquad 1 \qquad 27} 
\contentsline {subsection}{\textbf{Pechinkin A.\,V., Sokolov I.\,A., Chaplygin V.\,V.}\ \ Stationary Characteristics of a Multichannel Queueing System with\nobreakspace {}Simultaneous Refusals of Servers}{\qquad 2 \qquad 39} 
\contentsline {subsection}{\textbf{Shorgin S.\,Ya.} see Batrakova D.\,A.\hfill\hfill\hfill\hfill\hfill\hfill\hfill\hfill\hfill\hfill\hfill\hfill\hfill\hfill\hfill\hfill\hfill\hfill\hfill\hfill\hfill\hfill\hfill\hfill\hfill\hfill\hfill\hfill\hfill\hfill\hfill\hfill\hfill\hfill\hfill}{\ }
\contentsline {subsection}{\textbf{Shorgin S.\,Ya.} see Kudryavtsev A.\,A.\hfill\hfill\hfill\hfill\hfill\hfill\hfill\hfill\hfill\hfill\hfill\hfill\hfill\hfill\hfill\hfill\hfill\hfill\hfill\hfill\hfill\hfill\hfill\hfill\hfill\hfill\hfill\hfill\hfill\hfill\hfill\hfill\hfill\hfill\hfill}{\ }
\contentsline {subsection}{\textbf{Sinitsyn I.\,N.}\ \ Correlational Methods for Analytical Informational Models of the Earth Pole Fluctuations Design Based on a priori Data}{\qquad 2 \qquad \hphantom{9}2}
\contentsline {subsection}{\textbf{Sinitsyn I.\,N.}\ \ Development of Pugachev Filtering for Stochastic Systems}{\qquad 1 \qquad \hphantom{9}3}
\contentsline {subsection}{\textbf{Sokolov I.\,A.} see Ilyin V.\,D.\hfill\hfill\hfill\hfill\hfill\hfill\hfill\hfill\hfill\hfill\hfill\hfill\hfill\hfill\hfill\hfill\hfill\hfill\hfill\hfill\hfill\hfill\hfill\hfill\hfill\hfill\hfill\hfill\hfill\hfill\hfill\hfill\hfill\hfill\hfill}{\ }
\contentsline {subsection}{\textbf{Sokolov I.\,A.} see Pechinkin A.\,V.\hfill\hfill\hfill\hfill\hfill\hfill\hfill\hfill\hfill\hfill\hfill\hfill\hfill\hfill\hfill\hfill\hfill\hfill\hfill\hfill\hfill\hfill\hfill\hfill\hfill\hfill\hfill\hfill\hfill\hfill\hfill\hfill\hfill\hfill\hfill}{\ }
\contentsline {subsection}{\textbf{Sokolov I.\,A.} see Pechinkin A.\,V.\hfill\hfill\hfill\hfill\hfill\hfill\hfill\hfill\hfill\hfill\hfill\hfill\hfill\hfill\hfill\hfill\hfill\hfill\hfill\hfill\hfill\hfill\hfill\hfill\hfill\hfill\hfill\hfill\hfill\hfill\hfill\hfill\hfill\hfill\hfill}{\ }
\contentsline {subsection}{\textbf{Sokolov I.\,A.} see Zakharov V.\,N.\hfill\hfill\hfill\hfill\hfill\hfill\hfill\hfill\hfill\hfill\hfill\hfill\hfill\hfill\hfill\hfill\hfill\hfill\hfill\hfill\hfill\hfill\hfill\hfill\hfill\hfill\hfill\hfill\hfill\hfill\hfill\hfill\hfill\hfill\hfill}{\ }
\contentsline {subsection}{\textbf{Stupnikov S.\,A.} see Zakharov V.\,N.\hfill\hfill\hfill\hfill\hfill\hfill\hfill\hfill\hfill\hfill\hfill\hfill\hfill\hfill\hfill\hfill\hfill\hfill\hfill\hfill\hfill\hfill\hfill\hfill\hfill\hfill\hfill\hfill\hfill\hfill\hfill\hfill\hfill\hfill\hfill}{\ }
\contentsline {subsection}{\textbf{Zakharov V.\,N., Kalinichenko L.\,A., Sokolov I.\,A., Stupnikov S.\,A.}\ \ Development of Canonical Information Models for Integrated Information Systems}{\qquad 2 \qquad 15} 
\contentsline {subsection}{\textbf{Zakharov V.\,N., Kozmidiady V.\,A.}\ \ Means Providing Applications Fault Tolerance}{\qquad 1 \qquad 14} 
\def\leftfootline{\small{\textbf{\thepage}
\hfill ИНФОРМАТИКА И ЕЁ ПРИМЕНЕНИЯ\ \ \ том~1\ \ \ выпуск~2\ \ \ 2007}
}%
 \def\rightfootline{\small{ИНФОРМАТИКА И ЕЁ ПРИМЕНЕНИЯ\ \ \ том~1\ \ \ выпуск~2\ \ \ 2007
 \hfill \textbf{\thepage}}}
 \label{end\stat}


%\tableofcontents


\end{document}

\newcommand{\Ack}{\subsection*{\protect\large\bf Acknowledgments}}