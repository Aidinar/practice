\def\stat{vaskan}

\def\tit{АЛГОРИТМ ВИЗУАЛИЗАЦИИ ПЛОСКОГО ЯДРА ВЕРОЯТНОСТНОЙ МЕРЫ$^*$}

\def\titkol{Алгоритм визуализации плоского ядра вероятностной меры}

\def\aut{С.\,Н.~Васильева$^1$, Ю.\,С.~Кан$^2$}

\def\autkol{С.\,Н.~Васильева, Ю.\,С.~Кан}

\titel{\tit}{\aut}{\autkol}{\titkol}

\index{Васильева С.\,Н.}
\index{Кан Ю.\,С.}
\index{Vasil'eva S.\,N.}
\index{Kan Yu.\,S.}




{\renewcommand{\thefootnote}{\fnsymbol{footnote}} \footnotetext[1]
{Результаты работы получены в~рамках выполнения государственного задания 
 Минобрнауки №\,2.2461.2017/ПЧ, а также при финансовой поддержке РФФИ 
 (проект 15-08-02833а).}}


\renewcommand{\thefootnote}{\arabic{footnote}}
\footnotetext[1]{Московский авиационный институт (национальный исследовательский университет), 
\mbox{sofia\_mai@mail.ru}}
\footnotetext[2]{Московский авиационный институт (национальный исследовательский университет), 
\mbox{yu\_kan@mail.ru}}

%\vspace*{-6pt}



\Abst{Предлагается алгоритм по\-стро\-ения многогранной аппроксимации ядра 
вероятностной меры для двумерного случайного вектора с~независимыми компонентами. 
Ядро является одним из важ\-ных понятий, используемых в~алгоритмах решения задач 
стохастического программирования с~вероятностными критериями.  Наиболее эффективно 
ядро применяется в~случаях, когда постановки указанных задач имеют свойство 
линейности по отношению к~случайным параметрам.  
В~силу линейности  максимум по случайным параметрам определяется путем перебора 
всех вершин аппроксимирующего многогранника. Предложенный в~статье алгоритм основан 
на построении пересечения конечного чис\-ла доверительных полупространств, 
параметры которых оцениваются методом Мон\-те Кар\-ло. Результатом работы 
предложенного алгоритма является определение множества вершин аппроксимирующего 
многогранника. Аппроксимация ядра является их выпуклой оболочкой. 
Приводятся результаты расчетов для ряда типовых непрерывных законов распределения.}



\KW{задача квантильной оптимизации;  метод линеаризации; ядро вероятностной меры}

\DOI{10.14357/19922264180209}
  
\vspace*{-6pt}


\vskip 10pt plus 9pt minus 6pt

\thispagestyle{headings}

\begin{multicols}{2}

\label{st\stat}


\section{Введение} 

Ядро вероятностной меры заданного уровня~$\alpha$ для случайного вектора
 определяется как пересечение всех вы\-пук\-лых, замк\-ну\-тых, $\alpha$-до\-ве\-ри\-тель\-ных 
 множеств в~про\-стран\-ст\-ве реализаций этого случайного вектора. Это понятие 
 играет клю\-че\-вую роль в~тео\-ре\-ти\-че\-ских аспектах задач сто\-ха\-сти\-че\-ско\-го 
 программирования с~квантильным критерием качества~\cite{kankibzun}, называемых 
 ниже задачами квантильной оптимизации. Последние с~прикладной точки зрения 
 моделируют принятие решений в~условиях не\-опре\-де\-лен\-ности с~учетом риска или 
 требований   на\-деж\-ности.

Впервые $\alpha$-яд\-ро было введено в~рассмотрение в~\cite{malkib} для 
гауссовского случая с~целью доказательства асимптотической точ\-ности 
доверительного метода решения задач минимизации функции квантили при 
использовании доверительного множества в~виде эл\-лип\-со\-ида, яв\-ля\-юще\-го\-ся 
множеством уровня плот\-ности вероятности. Исследованию свойств ядра 
посвящены работы~\cite{kanrus, kansur}. Следует также отметить, что $\alpha$-яд\-ро 
может использоваться в~задачах опре\-де\-ле\-ния стар\-то\-вой точ\-ки для чис\-лен\-ных методов 
оптимизации функ\-ции квантили~\cite{kibkurb, kibkurben}.

Квантильный критерий пред\-став\-ля\-ет собой \mbox{$\alpha$-кван}\-тиль 
распределения некоторой функции потерь, зависящей от вектора оптимизируемой 
стратегии и~случайных па\-ра\-мет\-ров. Среди задач квантильной оптимизации можно
 выделить важный класс задач, в~которых функция потерь линейна по случайным 
 па\-ра\-мет\-рам.
В~этот класс включаются задачи оптимизации портфеля ценных бумаг с~учетом 
риска по квантильному критерию, впервые рас\-смот\-рен\-ные в~\cite{moeseke}.
В~этом случае задача минимизации квантильного критерия при выполнении 
некоторых условий регулярности ядра эквивалентна минимаксной 
задаче для функции потерь~\cite{kankibzun}, где внут\-рен\-ний максимум берется по реализациям 
случайных па\-ра\-мет\-ров, а~внеш\-ний минимум~--- по оптимизируемой стратегии. 

Таким образом, сто\-ха\-сти\-че\-ская задача квантильной оптимизации равносильна 
минимаксной со специально за\-да\-ва\-емым множеством неопределенности, по 
которому производится максимизация, и~это множество совпадает с~$\alpha$-яд\-ром. 

Эта минимаксная задача исследовалась в~\cite{vaskan} для случая, когда функция 
потерь является линейной и~по стратегии. В~этой же статье предложены общие 
алгоритмические схемы аппроксимации $\alpha$-яд\-ра\linebreak выпуклы\-ми многогранниками, 
с~использованием которых указанная минимаксная задача сводится к~задаче 
линейного программирования с~большим чис\-лом ограничений. 
%
На основе одной 
из этих схем и~разработан пред\-ла\-га\-емый ниже алгоритм, реализованный в~программном 
пакете MATLAB.

Во многих математических моделях прикладных задач встречается ситуация, 
когда случайные па\-ра\-мет\-ры являются в~некотором смысле малыми. В~этих случаях 
возможно использование метода линеаризации, точ\-ное описание которого изложено 
в~\cite{vaskan2}. В~соответствии с~этим методом исходную нелинейную функцию 
потерь можно линеаризовать по случайным па\-ра\-мет\-рам и~получить линейную по 
случайным па\-ра\-мет\-рам модель. Тем самым задача квантильной оптимизации с~нелинейной 
функцией потерь может быть приближенно аппроксимирована вышеупомянутой ми\-ни\-макс\-ной 
задачей, в~которой роль множества не\-опре\-де\-лен\-ности играет $\alpha$-ядро.

К настоящему времени мас\-штаб\-ные исследования гео\-мет\-рии $\alpha$-яд\-ра 
для типовых законов распределения не проводились. Предлагаемый ниже 
программно реализованный алгоритм предназначен для того, чтобы час\-тич\-но 
ликвидировать этот пробел.


В~разд.~2 вводятся основные определения и~утверж\-де\-ния, описывается 
рас\-смат\-ри\-ва\-емая задача кван\-тиль\-ной оптимизации, приведены некоторые свойства ядра, 
продемонстрирована роль ядра вероятностной меры в~задачах квантильной оптимизации. 
В~разд.~3 пред\-став\-ле\-но описание алгоритма по\-стро\-ения аппроксимации ядра. В~разд.~4 
представлены результаты рас\-че\-тов $\alpha$-ядер для двумерных логнормального 
и~экспоненциального рас\-пре\-де\-ле\-ний, по\-стро\-ен\-ные с~использованием пакета MATLAB.


\section{Постановка задачи и~предварительные результаты}

\noindent
\textbf{Определение~1.}\
{Множество $S_\alpha \subset \mathbb R^t$  называется  
\mbox{$\alpha$-до}\-ве\-ри\-тель\-ным 
множеством для случайного вектора~$\mathbf X$ с~реализациями $\mathbf x \hm\in 
\mathbb R^t$, если}
${\sf P}(\mathbf X \in S_\alpha)\geqslant \alpha\,.$

\smallskip

\noindent
\textbf{Определение~2.}
{$\alpha$-яд\-ро может быть определено как}
\begin{equation*}
K_\alpha=\bigcap\limits_{S\in \mathrm{E_\alpha} }S,
\end{equation*}
где $\mathrm{E_\alpha}$~--- семейство всех вы\-пук\-лых замк\-ну\-тых 
\mbox{$\alpha$-до}\-ве\-ри\-тель\-ных множеств.

\smallskip

Рассматривается задача о~построении границы $\alpha$-яд\-ра в~случае, когда век\-тор~$X$ 
имеет размерность $m\hm=2$  и~его компоненты независимы с~абсолютно 
непрерывным заданным законом рас\-пре\-де\-ле\-ния.

Поскольку для большинства распределений не удается аналитически 
по\-стро\-ить границу ядра, то вместо на\-хож\-де\-ния точ\-ной границы будем 
искать ее аппроксимацию.


Функцию $\Phi(\mathbf X,\mathbf u)$, зависящую от век\-то\-ра случайных 
па\-ра\-мет\-ров~$\mathbf X$ и~вектора стратегии $\mathbf u\hm\in U\subset \mathbb R^n$, 
будем называть функ\-ци\-ей потерь.


Введем необходимые определения и~утверждения из~\cite{kankibzun}.

\smallskip

\noindent
\textbf{Определение~3.}\
{Функция вероятности для функции потерь $\Phi(\mathbf X,\mathbf u)$ имеет вид}:
\begin{equation*}
 {\sf P}_\varphi (\mathbf u)\;{\stackrel{\Delta}{=}}\;
 {\sf P}\{\Phi(\mathbf X,\mathbf u)\leqslant \varphi\}\,,
 %\label{e2-vas}
\end{equation*}
{где ${\sf P}$~--- вероятность; $\varphi$~--- допустимый уровень потерь.}

\smallskip

\noindent
\textbf{Определение~4.}\
{Функция квантили для функции потерь $\Phi(\mathbf X,\mathbf u)$ 
определяется следующим образом:}
\begin{equation*}
\varphi_\alpha (\mathbf u)\;{\stackrel{\Delta}{=}}\;\min 
\left\{\varphi: {\sf P}_\varphi (\mathbf u)\geqslant \alpha \right\} 
\,{\stackrel{\Delta}{=}}\,[\Phi(\mathbf X,\mathbf u)]_\alpha\,.
%\label{e3-vas}
\end{equation*}
Задача квантильной оптимизации имеет вид:
\begin{equation}
[\Phi(\mathbf X,\mathbf u)]_\alpha\to \min\limits_{u\in U}\,.
\label{e4-vas}
\end{equation}
$\alpha$-ядро допускает следующее представление:
\begin{equation*}
K_\alpha \; = \bigcap\limits_{\| c\|=1}
\left \{z\in \mathbb R^t:c^\mathrm{T} z\leqslant 
\left[c^\mathrm{T} \mathbf X\right]_\alpha\right\}\,,
%\label{e5-vas}
\end{equation*}
 где $c$~--- вектор внешней нормали к~границе $\alpha$-до\-ве\-ри\-тель\-но\-го 
 полупространства; $\parallel \!\!\cdot\!\! \parallel$~--- евклидова норма.
Таким образом, $\alpha$-яд\-ро является пересечением всех замк\-ну\-тых 
$\alpha$-до\-ве\-ри\-тель\-ных полупространств.

Определим аппроксимацию $\alpha$-яд\-ра как пересечение конечного числа 
замкнутых $\alpha$-до\-ве\-ри\-тель\-ных полуплоскостей:
\begin{equation}
V_{\alpha N}=\bigcap\limits_{j=1}^{N} \left\{ 
\mathbf x: c_j^\mathrm{T} \mathbf  x \leqslant  
\left [ c_j^\mathrm{T} \mathbf X \right ]_\alpha\right \}.
\label{e6-vas}
\end{equation}

В случае использования точ\-но\-го значения квантили $\left [ \mathrm c_j^\mathrm{T} 
\mathbf X \right ]_\alpha$ полученный многогранник является внеш\-ней 
аппроксимацией $\alpha$-яд\-ра. В~общем случае явно определить эту величину 
не удается, поэтому для ее вы\-чис\-ле\-ния ниже, в~разд.~5, будет использована 
выборочная оценка квантили.

Рассмотрим функцию потерь вида:
\begin{equation}
\Phi(\mathbf X, \mathbf u)=a(\mathbf u)+b^{\mathrm{T}}(\mathbf u) \mathbf X\,.
\label{e7-vas}
\end{equation}

Введем следующее

\smallskip

\noindent
\textbf{Определение~5}~\cite{kankibzun}. $\alpha$-яд\-ро~$K_\alpha$ 
называется регулярным, если всякое замк\-ну\-тое полупространство, содержащее 
это ядро, автоматически является  \mbox{$\alpha$-до}\-ве\-ри\-тель\-ным.

\smallskip

\noindent
\textbf{Теорема 1}~\cite{kankibzun}.
 \textit{Если случайный вектор $\mathbf X$ имеет регулярное $\alpha$-ядро $K_\alpha$, 
 то для любого детерминированного вектора $\mathbf u \hm\in R^n$ справедливо}
\begin{equation*}
 \left [ \mathbf u^\mathrm{T} \mathbf X \right ]_\alpha=\max_{\mathbf x\in K_\alpha} \mathbf u^\mathrm{T} \mathbf x.
% \label{e8-vas}
\end{equation*}


В соответствии с~теоремой~1 задача квантильной оптимизации~(\ref{e4-vas}) 
для функ\-ции~(\ref{e7-vas}) 
может быть сведена к~задаче:
\begin{equation}
a (\mathbf{u}) + \max_{\mathbf{x}\in K_\alpha}\left(b^{\mathrm{T}}(\mathbf{u}) 
\mathbf{x}\right)\to \min\limits_{\mathbf {u} \in U}.
\label{e9-vas}
\end{equation}


После замены многогранника на его аппроксимацию получаем:
 \begin{equation*}
\varphi_\alpha^N (u) = a (\mathbf {u}) + \max\limits_{j=\overline{1,N}}
\left(b^{\mathrm{T}}(\mathbf {u}) \mathbf  {v}_j\right) 
\to \min\limits_{\mathbf {u}\in U}\,,
%\label{e10-vas}
\end{equation*}
где $v_j$~--- вершины $ V_{\alpha N}$,  $j\hm=\overline{1,N}.$ 
Функция $\varphi_\alpha^N (u)$ пред\-став\-ля\-ет собой оценку исходного 
квантильного критерия, при на\-хож\-де\-нии которой исходный максимум по 
случайному век\-то\-ру из множества~$K_\alpha$ заменен на максимум по 
его аппроксимации~$V_{\alpha N}.$ 

Максимум по вершинам выпуклого многогранного 
множества может быть найден как максимум из значений функции в~вершинах 
этого множества~$\max\nolimits_{j=\overline{1,N}}.$



Сходимость
\begin{equation*}
\min\limits_{u\in U} \varphi_\alpha^N (u)-\min\limits_{u\in U} \varphi_\alpha 
(u)\xrightarrow[{N\to \infty}]{} 0
%\label{e11-vas}
\end{equation*}
доказана в~\cite{vaskan} для случая, когда $a (\mathbf {u})$ и~$b(\mathbf {u})$ 
линейны по~$\mathbf {u}$ и~множество~$U$ компактно.


Задача~(\ref{e4-vas}) для функции потерь, нелинейно зависящей от век\-то\-ра  
случайных па\-ра\-мет\-ров, является слож\-но разрешимой. В~\cite{vaskan2} доказана 
спра\-вед\-ли\-вость\linebreak
 использования линеаризованной по случайным\linebreak
  па\-ра\-мет\-рам 
функции потерь в~случае малых случайных па\-ра\-мет\-ров. 
В~качестве вектора малых случайных па\-ра\-мет\-ров используется вектор~$\mathbf{X}^\mu$,  
со\-став\-лен\-ный из покоординатных произведений вектора малых па\-ра\-мет\-ров~$\mu$ 
и~случайного вектора~$\mathbf{X}$, т.\,е.\ $\mathbf{X}^\mu\hm=M\mathbf{X}$, 
где $M\hm=\mathrm{diag}\,(\mu_1, \mu_2, \ldots , \mu_t).$ 
Функция потерь имеет вид $\Phi(\mathbf {X}^\mu, \mathbf {u}).$ 
Разложение этой функции потерь в~ряд Тейлора по вектору $\mathbf {X}^\mu$
 в~окрест\-ности нуля справедливо, если функция является дваж\-ды непрерывно 
 дифференцируемой в~окрест\-ности нуля, а~так\-же непрерывна по $\mathbf {u}\hm\in U$ 
 вместе со своими част\-ны\-ми производными по~$\mathbf{X}^\mu$ до второго 
 порядка включительно. Разложение может быть пред\-став\-ле\-но следующим выражением:
\begin{equation*}
\Phi\left(\mathbf {X}^\mu, \mathbf {u}\right)= 
a(\mathbf {u})+b^{\mathrm{T}}(\mathbf {u})\mathbf {X}^\mu+r_1
\left(\mathbf {X}^\mu, \mathbf {u}\right)\,,
%\label{e12-vas}
\end{equation*}
где $a(\mathbf{u})\hm=\Phi(\mathbf {0},\mathbf {u})$~--- 
значение функции потерь в~нуле; $b (\mathbf {u})\hm=({\partial 
\Phi(\mathbf x^\mu,\mathbf u)}/{\partial \mathbf x^\mu})|_{\mathbf x^\mu=0}$~--- 
вектор значений покоординатных производных в~нуле;
  $r_1(\mathbf{X}^\mu, \mathbf {u})\hm= 
  (X_1^\mu \partial _1 \Phi(\theta \mathbf {X}^\mu, \mathbf {u})+\cdots+
  X_t^\mu \partial_t \Phi(\theta \mathbf {X}^\mu, \mathbf {u}))/2$~--- 
  остаточный член ряда Тейлора в~форме Ла\-гран\-жа,
  $\theta \hm\in(0,1)$; $\partial_i \Phi(\mathbf {X}^\mu, \mathbf{u})\hm=
  {\partial \Phi(\mathbf{X}^\mu, \mathbf {u})}/{\partial {\mathbf {X}_i}^\mu}$~--- 
  част\-ная производная функции $\Phi(\mathbf{X}^\mu, \mathbf{u})$ по $i$-й 
  координате случайного век\-то\-ра~$\mathbf {X}^\mu.$


Под линеаризованной моделью подразумевается линейная часть 
разложения исходной функ\-ции в~ряд Тейлора по случайным па\-ра\-мет\-рам в~окрест\-ности нуля. 
Линеаризованную модель мож\-но пред\-ста\-вить в~виде~(\ref{e7-vas}):
\begin{equation*}
\Phi_l\left(\mathbf {X}^\mu, \mathbf {u}\right)=
 a(\mathbf {u})+b^{\mathrm{T}}(\mathbf {u})\mathbf {X}^\mu\,.
 %\label{e13-vas}
\end{equation*}
Такая замена приводит к~следующему результату:
\begin{equation}
\!\!\min_{u\in U} \left[\Phi(\mathbf{X}^\mu, \mathbf {u})\right]_\alpha\!=\!
\min\limits_{u\in U} \left[\Phi_l\left(\mathbf {X}^\mu, \mathbf {u}\right)\right]_\alpha 
+O\left(||\mu||^2\right).\!
\label{e14-vas}
\end{equation}


Результат~(\ref{e14-vas}) обоснован в~работе~\cite{vaskan2} для случая, 
когда множество~$U$ является компактом, а~носитель случайного вектора~$X$ 
ограничен, и~для случая, когда множество~$U$ конечно, а~носитель вектора 
случайных па\-ра\-мет\-ров~$X$ неограничен.

Учитывая~(\ref{e9-vas}), заключаем, что с~точ\-ностью до $O(||\mu||^2)$ задача
\begin{equation*}
[\Phi(\mathbf {X},\mathbf {u})]_\alpha\to \min\limits_{u\in U}
%\label{e15-vas}
\end{equation*}
может быть аппроксимирована минимаксной задачей
\begin{equation*}
a(\mathbf {u})+\max\limits_{\mathbf {x}\in K_\alpha} b^{\mathrm{T}}
(\mathbf {u}) M \mathbf {x} \to \min\limits_{u\in U},
%\label{e16-vas}
\end{equation*}
если, конечно, ядро~$K_\alpha$ регулярно и~справедливо соотношение~(\ref{e14-vas}).

В данной статье рассматривается задача построения сколь угодно 
точ\-ной аппроксимации ядра многогранником.

\section{Алгоритм построения аппроксимации $\alpha$-ядра}

Для построения аппроксимации ядра сгенерируем выборку $X_1, X_2, \ldots, X_k$ 
случайного вектора~$\mathbf {X}.$
Рас\-смот\-рим вспомогательную линейную функцию потерь вида
$$
Z(c,\mathbf {X})=c^{\mathrm {T}} \mathbf{X}\,,
$$
где $c$~--- единичный вектор нормали, т.\,е.\ \mbox{$\parallel c\parallel \hm= 1.$}
С~учетом этого обозначения аппроксимация ядра~(\ref{e6-vas}) может быть записана в~виде:
\begin{equation*}
V_{\alpha N}=\bigcap\limits_{j=1}^{N} \left\{ \mathbf {x}: c_j^{\mathrm{T}} 
\mathbf{x} \leqslant  \left [ Z\left(c_j, \mathbf {X} \right) \right]_\alpha\right \}\,.
%\label{e17-vas}
\end{equation*}
Функцию квантили $[Z(c,\mathbf {X})]_\alpha$ будем оценивать методом Мон\-те Карло.

Используя сгенерированную ранее выборку случайного вектора~$\mathbf {X},$ 
по\-стро\-им выборку значений функ\-ции потерь $Z(c_j,\mathbf{X})$ по формуле:
$$
Z_i^j=c_j^{\mathrm{T}} \mathbf {X}_i\,,\enskip i=\overline{1,k}\,.
$$
Для каждого номера~$j$ построим вариационный ряд этой выборки:
$$
Z_{(1)}^j \leq Z_{(2)}^j\leq \dots \leq Z_{(k)}^j.
$$
Выборочная оценка $\widehat Z^j_\alpha$ функции квантили~$\left[Z(c_j,X)\right]_\alpha$  
вы\-чис\-ля\-ет\-ся по формуле  из~\cite{kankibzun}:
\begin{equation*}
\widehat Z^j_\alpha=
\begin{cases}
Z ^j_{(\alpha k)}\,, &\ \alpha k\in \mathbb{ N}\,;\\
Z ^j_{([\alpha k]+1)}\,, &\ \alpha k\notin \mathbb{ N}\,,
\end{cases}
\end{equation*}
где $[\cdot]$~--- целая часть от числа.

Точность выборочной оценки обоснована сле\-ду\-ющей тео\-ре\-мой~\cite{bakh}.

\smallskip

\noindent
\textbf{Теорема~2.}\ 
\textit{Если $\alpha\hm\in (0,1)$ и~случайная величина~$\Phi$  
имеет плот\-ность ве\-ро\-ят\-ности~$p(\varphi)$, непрерывную в~некоторой окрест\-ности 
точ\-ки~$\varphi_\alpha$, причем $ p(\varphi_\alpha)\hm>0,$ то}
\begin{equation*}
\widehat \Phi_\alpha-\varphi_\alpha=
\fr{\widehat {\sf P}(\varphi_\alpha)-\alpha}{p(\varphi_\alpha)}+o_p\left(k^{-1/2}\right)\,,
\end{equation*}
\textit{где $\widehat {\sf P}(\varphi_\alpha)$~--- 
значение выборочной оценки ве\-ро\-ят\-ности в~точ\-ке $\varphi_\alpha$;
 $\varphi_\alpha$~--- 
точ\-ное значение квантили; $k^{1/2}o_p(k^{-1/2})\hm\to 0$ по ве\-ро\-ят\-ности при
 $k\to \infty$.}
 
 \smallskip

Как уже отмечалось выше, аппроксимация \mbox{$\alpha$-яд}\-ра строится путем 
пересечения конечного чис\-ла по\-лу\-плос\-костей,
определяемых своими векторами нормалей. В~работе~\cite{vaskan} пред\-ло\-жен 
алгоритм, который
 позволяет построить гус\-тую сеть векторов~$c_j$, $j\hm=\overline{1,N}.$ 
 При этом векторы из этого множества в~двумерном случае
 образуют неравномерную сет\-ку на единичной окруж\-ности 
 с~центром в~начале координат.

Этот недостаток алгоритма из ~\cite{vaskan} можно устранить в~случае 
двумерного распределения случайного вектора. Новый алгоритм включает в~себя 
сле\-ду\-ющую по\-сле\-до\-ва\-тель\-ность шагов.
\begin{enumerate}[1.]
\item  Нанесем равномерную сетку из заданного чис\-ла~$N$ точек~$c_j$, $j\hm=\overline{1,N}$, 
на по\-верх\-ность единичной окруж\-ности.


\item  Сгенерируем выборку заданного объема~$n$ 
для рас\-смат\-ри\-ва\-емо\-го двумерного распределения случайного вектора~$\mathbf{X}.$

\item  Для каждого вектора нормали~$c_j$,  $j\hm=\overline{1,N}$, сформируем 
вариационный ряд $Z_{(1)}^j, Z_{(2)}^j, \ldots, Z_{(k)}^j$
и~построим выборочную оценку квантили~$[Z(c_j,\mathbf X)]_\alpha$
в~соответствии с~процедурой, описанной в~начале данного раздела.

\item  Найдем точки пересечения границ всех пар рас\-смат\-ри\-ва\-емых доверительных 
полуплоскостей $\{ \mathbf {x}: c_j^{\mathrm{T}} \mathbf {x} \hm\leqslant  
\widehat  {Z}^j_\alpha\}.$

\end{enumerate}

После этого производится проверка их принадлежности остальным по\-лу\-плос\-костям.
 Точки, удовле\-тво\-ря\-ющие всем ограничениям, яв\-ля\-ют\-ся
 вершинами аппроксимирующего многогранника.\linebreak
За счет использования выборочной оценки кван\-тили, не совпадающей в~общем 
случае с~точ\-ным значением квантили, чис\-ло
вершин многогранника может оказаться меньше~$N$.
Вершины аппроксимирующего многогранника отображаются на итоговом рисунке, 
со\-сед\-ние вершины соединяются прямолинейным отрезком.

\vspace*{-4pt}

\section{Программная реализация алгоритма визуализации {\boldmath{$\alpha$}}-ядра}

\vspace*{-2pt}

Для реализации алгоритма была использована программная среда MATLAB. 
Для удобства использования программы с~по\-мощью встро\-ен\-но\-го инструмента 
GUIDE был создан пользовательский интерфейс программы, вид которого 
пред\-став\-лен на рис.~1.

\begin{figure*} %fig1
 \vspace*{1pt}
 \begin{center}
 \mbox{%
 \epsfxsize=163.202mm  
 \epsfbox{vas-1.eps}
 }
 \end{center}
\vspace*{-11pt}
\Caption{Пример работы программы: (\textit{а})~логнормальное распределение, %\textit{1}~--- 
$\alpha\hm=0{,}9$; (\textit{б})~экспоненциальное распределение %:
при $\alpha\hm= 0{,}8$, 0,9 и~0,99}
\vspace*{5pt}
\end{figure*}




В окне программы пользователь имеет воз\-мож\-ность выбрать или ввести 
сле\-ду\-ющие параметры: вид рас\-пре\-де\-ле\-ния (среди предложенных); 
па\-ра\-мет\-ры 
(па\-ра\-метр) распределения; объем выборки; чис\-ло доверительных полуплоскостей; 
чис\-ло аппроксимаций; па\-ра\-метр~$\alpha$ для каждой из аппроксимаций. 
В~меню выбора доступны для выбора сле\-ду\-ющие распределения: нормальное; равномерное; 
логнормальное; экспоненциальное; распределение Коши. На одном рисунке может 
быть по\-стро\-ено не более трех аппроксимаций. Виды линий границы указаны в~легенде.
Для удоб\-ст\-ва все необходимые данные изначально заполнены.
Для начала работы алгоритма программы необходимо нажать кнопку 
<<Build approximation>>,
после чего справа от меню появится рисунок с~выбранным пользователем 
чис\-лом аппроксимаций.
После этого па\-ра\-мет\-ры в~об\-ласти меню могут быть изменены.
После нажатия кноп\-ки 
<<Build approximation>> будет показан рисунок, по\-стро\-ен\-ный для новых па\-ра\-мет\-ров.

На рис.~1,\,\textit{б} показано окно программы. На рисунке построены три аппроксимации 
ядер для различных заданных~$\alpha$. В~качестве распределения выбрано 
экспоненциальное рас\-пре\-де\-ле\-ние с~па\-ра\-мет\-ром $\lambda\hm=1.$


Для  генерации выборки распределения используются стандартные функции  из пакета 
MATLAB Statistics Toolbox.

Для удобства была создана функция Kernel, которая по заданным па\-ра\-мет\-рам находит  
аппроксимацию и~возвращает мас\-сив значений. На вход необходимо подать сле\-ду\-ющие 
данные: массив реа-\linebreak\vspace*{-12pt}

\pagebreak

\noindent
ли\-за\-ций случайной величины, уровень~$\alpha$, объем выборки,
 чис\-ло полуплоскостей.

После генерации точек всех ядер, используется стандартная функция plot, которая 
по точ\-кам строит границу ядра.

\vspace*{-3pt}

\section{Результаты расчетов}

\vspace*{-2pt}

В данном разделе представлены результаты для экспоненциального и~логнормального 
распределений компонент случайного вектора~$\mathbf{X}.$ На рис.~2--5 точками
 обозначены вершины многогранников, аппроксимирующих $\alpha$-яд\-ро. 
 Как было отмечено выше, чис\-ло вершин
не превосходит~$N.$


С помощью описанной выше программы построим $\alpha$-яд\-ра для объема выборки~$10^6$ 
для $\alpha\hm=0{,}9.$
Результаты работы алгоритма построения аппроксимации $\alpha$-яд\-ра при 
$N\hm=8$, 16 и~70 для случая, когда
 компоненты случайного вектора~$\mathbf{X}$ независимы и~$X_i\sim \log
 N(0,1)$, $i\hm=1,2,$
показаны на рис.~2.






Эмпирически установлено, что для выборок порядка $k=10^6$ и~выше
выборочная оценка достаточно точ\-на, поэтому уменьшения чис\-ла
вер\-шин аппроксимирующего многогранника не наблюдается.


Аналогичные результаты по\-стро\-ения аппроксимации $\alpha$-яд\-ра при $N=8$, 16 и~70 
для случая, когда
компоненты случайного вектора~$\mathbf X$ независимы и~$X_i\hm\sim E(1)$, $i\hm=1,2$,
показаны на рис.~3.

\begin{figure*} %fig2
\vspace*{1pt}
\begin{minipage}[t]{80mm}
 \begin{center}
 \mbox{%
 \epsfxsize=65.102mm 
 \epsfbox{vas-3.eps}
 }
 \end{center}
\vspace*{-9pt}
\Caption{Логнормальное распределение,  $k\hm=10^6$, $\alpha\hm = 0{,}9$:
(\textit{а})~$N\hm=8$; (\textit{б})~16;
(\textit{в})~$N\hm=70$}
\end{minipage}
%\end{figure*}
\hfill
%\begin{figure*} %fig3
\vspace*{1pt}
\begin{minipage}[t]{80mm}
 \begin{center}
 \mbox{%
 \epsfxsize=64.929mm 
 \epsfbox{vas-6.eps}
 }
 \end{center}
\vspace*{-9pt}
\Caption{Экспоненциальное распределение,  $k\hm=10^6$, $\alpha \hm= 0{,}9$:
(\textit{а})~$N\hm=8$; (\textit{б})~16; (\textit{в})~$N\hm=70$}
\end{minipage}
\end{figure*}


Для экспоненциального распределения при объеме выборки порядка $k\hm=10^6$ и~выше
 выборочная оценка кван\-ти\-ли также достаточно\linebreak точна.

Разработанный программный модуль позволяет строить на одном рисунке одновременно 
две или три аппроксимации для различных значений~$\alpha.$ На рис.~4 
продемонстрированы границы ядер для $\alpha\hm\in\{0,7; 0,9; 0,99\}.$




На рис.~5 показано, что при уменьшении объема выборки в~случае 
логнормального распределения при $\alpha\hm=0{,}9$
чис\-ло вершин аппроксимирующего многогранника уменьшается.


В частности, при $k\hm=10^4$ чис\-ло вершин аппроксимирующего многогранника 
уменьшилось на~20.


При использовании выборки объема $k\hm=10^3$ снижение 
точ\-ности выборочной оценки квантили приводит к~тому,
что остается всего лишь~38~вершин многогранника из~70~возможных.

За счет того, что процедура построения выборочной оценки квантили 
производится для всех единичных векторов нормали и~пересечение ограничений, 
задающих доверительные по\-лу\-плос\-кости, производится попарно, приходится 
производить значительную часть расчетов и~операций, не влия\-ющих на конечный 
результат. В~случае, когда объем выборки увеличить невозможно или пользователю 
необходимо увеличить ско\-рость работы алгоритма, рекомендуется уменьшить чис\-ло 
пересекаемых полупространств. Чис\-ло вершин многогранника при этом уменьшится, 
но незначительно. Если пользователю необходимо увеличить точ\-ность аппроксимации, 
то необходимо увеличить па\-ра\-метр~$n$ для повышения точ\-ности выборочной оценки 
кван\-тили.



Соотношения параметров~$\alpha,$ $N$ и~$k$ не могут быть заданы априори, 
поскольку они зависят как от вида распределения, так и~от параметров распределения. 
Число вершин многогранника является случайной величиной. Для рас\-смат\-ри\-ва\-емо\-го 
распределения при фиксированном~$\alpha$ могут быть по-\linebreak\vspace*{-12pt}

\pagebreak

\end{multicols}

\begin{figure*} %fig4
\vspace*{1pt}
 \begin{center}
 \mbox{%
 \epsfxsize=161.751mm 
 \epsfbox{vas-9.eps}
 }
 \end{center}
\vspace*{-9pt}
\Caption{Экспоненциальное~(\textit{а}) 
и~логнормальное~(\textit{б}) распределения, 70~точек:
\textit{1}~--- $\alpha\hm=0{,}7$; \textit{2}~--- 0,9; \textit{3}~--- 
$\alpha\hm= 0{,}99$}
%\end{figure*}
%\begin{figure*} %fig5
\vspace*{6pt}
 \begin{center}
 \mbox{%
 \epsfxsize=162.254mm 
 \epsfbox{vas-11.eps}
 }
 \end{center}
\vspace*{-9pt}
\Caption{Логнормальное распределение, 70~точек,  $\alpha\hm= 0{,}9$:
(\textit{а})~выборка~$10^4$; (\textit{б})~выборка~$10^3$}
\end{figure*}

\begin{multicols}{2}

\noindent
стро\-ены выборочные оценки 
при фиксированных
 па\-ра\-мет\-рах~$N$ и~$k$   для числа вершин (например, выборочного 
среднего, выборочной дис\-пер\-сии и~т.\,д.) в~за\-ви\-си\-мости от па\-ра\-мет\-ров~$N$ и~$k.$

\vspace*{-9pt}

\section{Заключение }

Предложенный алгоритм позволяет более точно (по сравнению с~методом, предложенным 
в~\cite{vaskan}) аппроксимировать
 $\alpha$-яд\-ро. Этот эффект достигается за счет того, что используется 
 равномерная сеть точек на окруж\-ности, за\-да\-ющих
 векторы нор\-малей.
 {\looseness=1
 
 }

На рисунках показано, как уменьшается число точек многогранника при 
уменьшении объема выборки распределения случайного вектора.

При решении практических задач при заданном~$\alpha$ па\-ра\-мет\-ры~$N$ и~$k$  
необходимо выбирать так, чтобы чис\-ло вершин многогранника было как можно ближе к~$N.$

Помимо графических иллюстраций разработанный программный модуль позволяет 
построить\linebreak сис\-те\-му линейных ограничений, за\-да\-ющих аппроксимацию $\alpha$-яд\-ра, 
которую можно использовать в~алгоритмах решения задач с~квантильным критерием 
качества. Кроме того, определяются координаты вер\-шин многогранника, 
удовле\-тво\-ря\-ющие данной сис\-те\-ме ограничений.



{\small\frenchspacing
 {%\baselineskip=10.8pt
 \addcontentsline{toc}{section}{References}
 \begin{thebibliography}{99}

\bibitem{kankibzun}
\Au{Кибзун А.\,И., Кан~Ю.\,С.}
Задачи стохастического программирования с~вероятностными критериями.~--- 
М.: Физматлит, 2009. 372~с.

\bibitem{malkib}
\Au{Малышев В.\,В., Кибзун~А.\,И.}
Анализ и~синтез высокоточного управления летательными аппаратами.~--- 
М.: Машиностроение, 1987. 303~с.

\bibitem{kanrus}
\Au{Кан Ю.\,С., Русяев~А.\,В.}
Задача квантильной минимизации с~билинейной функцией потерь~// Автоматика
и~телемеханика, 1998. 
№\,7. С.~67--75.

\bibitem{kansur} %4
\Au{Кан Ю.\,С., Суринов~Р.\,Т.}
О~неравенстве треугольника  для критерия VaR~//  
Моделирование и~анализ без\-опас\-ности и~рис\-ка в~слож\-ных сис\-те\-мах: 
Тр. Междунар. научной школы МАБР-2004.~--- СПб.: СПбГУАП,  2004. С.~243--248.



\bibitem{kibkurben} %5
\Au{Kibzun A.\,I., Kurbakovskiy V.\,Yu.}
Guaranteeing approach to solving quantile optimization problem~// 
Ann. Oper. Res., 1991. Vol.~30. P.~81--93.

\bibitem{kibkurb} %6
\Au{Кибзун А.\,И., Курбаковский~В.\,Ю.}
Численные алгоритмы квантильной оптимизации и~их применение к~решению 
задач с~вероятностными ограничениями~// Изв. РАН, Техническая кибернетика, 
1992. №\,1. С.~75--81.

\bibitem{moeseke}
{\it Van Moes$\Grave{\mbox{e}}$ke P.}
Stochastic linear programming~// Yale Econ. Essays, 1965. Vol.~5. P.~197--253.

\bibitem{vaskan}
\Au{Васильева С.\,Н.,  Кан~Ю.\,С.}
Метод решения задачи квантильной оптимизации с~билинейной функцией потерь~// Автоматика
и~телемеханика, 
2015. №\,9. С.~83--101.

\bibitem{vaskan2}
\Au{Васильева С.\,Н.,  Кан~Ю.\,С.}
Метод линеаризации для решения задачи квантильной оптимизации с~функцией потерь, 
зависящей от вектора малых случайных параметров~// 
Автоматика
и~телемеханика, 2017. №\,7. С.~95--109.

\bibitem{bakh}
\Au{Bahadur R.\,R.}
A~note on quantiles in large samples~// Ann. Math. Stat., 
1996. Vol.~37. P.~577--580.

 \end{thebibliography}

 }
 }

\end{multicols}

\vspace*{-6pt}

\hfill{\small\textit{Поступила в~редакцию 26.04.17}}

\vspace*{6pt}

%\newpage

%\vspace*{-24pt}

\hrule

\vspace*{2pt}

\hrule

%\vspace*{8pt}


\def\tit{A~VISUALIZATION ALGORITHM FOR~THE~PLANE PROBABILITY MEASURE KERNEL}

\def\titkol{A~visualization algorithm for~the~plane probability measure kernel}

\def\aut{S.\,N.~Vasil'eva and Yu.\,S.~Kan}

\def\autkol{S.\,N.~Vasil'eva and Yu.\,S.~Kan}

\titel{\tit}{\aut}{\autkol}{\titkol}

\vspace*{-9pt}


 \noindent
Moscow Aviation Institute (National Research University), 4~Volokolamskoe Shosse, 
Moscow 125993, Russian Federation 


\def\leftfootline{\small{\textbf{\thepage}
\hfill INFORMATIKA I EE PRIMENENIYA~--- INFORMATICS AND
APPLICATIONS\ \ \ 2018\ \ \ volume~12\ \ \ issue\ 2}
}%
 \def\rightfootline{\small{INFORMATIKA I EE PRIMENENIYA~---
INFORMATICS AND APPLICATIONS\ \ \ 2018\ \ \ volume~12\ \ \ issue\ 2
\hfill \textbf{\thepage}}}

\vspace*{3pt}



 \Abste{The authors propose an algorithm for constructing a~probability 
 measure kernel polyhedral approximation for a two-dimensional random vector 
 with independent components. The kernel is one of the important concepts 
 used in algorithms for solving stochastic programming problems with probabilistic 
 criteria. The kernel is most effectively used in cases when the statements of 
 the indicated problems have the property of linearity with respect to random 
 parameters. Because of linearity, the maximum in random parameters is determined 
 by searching all vertices of the approximating polyhedron. The authors propose 
 an algorithm for constructing a polyhedral approximation of the kernel of 
 a~probability measure for a two-dimensional random vector with independent 
 components. The algorithm is based on  construction of the intersection of 
 a~finite number of confidence half-spaces, the parameters of which are estimated 
 by the Monte-Carlo method. The result of the proposed algorithm 
 is the definition of the set of vertices of the approximating polyhedron. 
 Approximation of the nucleus is their convex hull. The results of calculations for 
a~number of typical continuous distribution laws are presented.}

\KWE{quantile optimization problem; linearization method; probability measure kernel}



\DOI{10.14357/19922264180209} %

%\vspace*{-14pt}

\Ack
\noindent
The work was supported by the Russian Ministry of Education and Science (Project 
No.\,2.2461.2017/PCh) and by the Russian Foundation for Basic Research (grant 
No.\,15-08-02833a).

\pagebreak



%\vspace*{-3pt}

  \begin{multicols}{2}

\renewcommand{\bibname}{\protect\rmfamily References}
%\renewcommand{\bibname}{\large\protect\rm References}

{\small\frenchspacing
 {%\baselineskip=10.8pt
 \addcontentsline{toc}{section}{References}
 \begin{thebibliography}{99}

\bibitem{1-kan}
\Aue{Kibzun, A.\,I., and Yu.\,S.~Kan.} 2009. 
\textit{Zadachi sto\-kha\-sti\-che\-skogo programmirovaniya 
s~veroyatnostnymi kriteriyami}  
[Stochastic programming problems with probabilistic criteria]. Moscow: Fizmatlit.  372~p.
\bibitem{2-kan} 
\Aue{Malyshev, V.\,V., and A.\,I.~Kibzun.} 1987. 
\textit{Analiz i~sintez vysokotochnogo upravleniya 
letatel'nymi apparatami}  [Analysis and synthesis of high-precision aircraft control].
 Moscow: Mashinostroenie.  303~p.
\bibitem{3-kan} 
\Aue{Kan, Yu.\,S., and A.\,V.~Rusyaev.} 1998. 
Quantile minimization with bilinear loss function.
\textit{Automat.  Rem. Contr.} 59(7(1)):960--966.
\bibitem{4-kan}
\Aue{Kan, Yu.\,S., and R.\,T.~Surinov.} 2004.  
O~neravenstve treugol'nika  dlya kriteriya 
VaR [On the triangle inequality for the VAR criterion].
\textit{Modeling and Analysis of Safety and Risk in Complex Systems: Scientific 
 School (International) MASR-2004 Proceedings}. St.\ Petersburg:
SPbGUAP. 243--248.

\bibitem{6-kan}  %5
\Aue{Kibzun, A.\,I., and V.\,Yu.~Kurbakovskiy.} 1991. 
Guaranteeing approach to solving quantile optimization problem.
\textit{Ann. Oper. Res.} 30:81--93.

\bibitem{5-kan}  %6 
\Aue{Kibzun, A.\,I., and V.\,Yu.~Kurbakovskij.} 1992.
 Chislennye algoritmy kvantil'noy optimizatsii i~ikh primenenie k~re\-she\-niyu zadach 
 s~veroyatnostnymi ogranicheniyami [Numerical algorithms for quantitative 
 optimization and their application to solving problems with probability constraints].
 \textit{J.~Comput. Sys. Sc. Int.}  1:75--81.
 
\bibitem{7-kan}
\Aue{Van Moes$\Grave{\mbox{e}}$ke, P.} 1965. Stochastic linear programming.
\textit{Yale Econ. Essays} 5:197--253.
\bibitem{8-kan} 
\Aue{Vasil'eva, S.\,N., and Yu.\,S.~Kan.} 2015.
 A~method for solving quantile optimization problems with a~bilinear loss function.
 \textit{Automat. Rem. Contr.} 76(9):1582--1597.
\bibitem{9-kan} 
\Aue{Vasil'eva, S.\,N., and Yu.\,S.~Kan}. 2017. 
Linearization method for solving quantile optimization problems 
with loss function depending on a~vector of small random parameters.
\textit{Automat. Rem. Contr.} 78(7):1248--1260.
\bibitem{10-kan} 
\Aue{Bahadur, R.\,R.} 1996. A~note on quantiles in large samples.
\textit{Ann. Math. Stat.} 37:577--580.
\end{thebibliography}

 }
 }

\end{multicols}

\vspace*{-3pt}

\hfill{\small\textit{Received April 26, 2017}}

%\vspace*{-24pt}

\Contr


\noindent
\textbf{Vasil'eva Sofia N.} (b.\ 1993)~--- 
PhD student, Moscow Aviation Institute (National Research University), 
4~Volokolamskoe  Shosse, Moscow 125993, Russian Federation; 
\mbox{sofia\_mai@mail.ru}

\vspace*{3pt}

\noindent
\textbf{Kan Yuri S.} (b.\ 1960)~--- 
Doctor of Science in physics and mathematics, 
Professor, Moscow Aviation Institute (National Research University), 
4~Volokolamskoe  Shosse, Moscow 125993, Russian Federation; \mbox{yu\_kan@mail.ru}

\label{end\stat}


\renewcommand{\bibname}{\protect\rm Литература} 