

\def\stat{lebedev}

\def\tit{МАКСИМАЛЬНЫЕ ВЕТВЯЩИЕСЯ ПРОЦЕССЫ\\ В~СЛУЧАЙНОЙ СРЕДЕ}

\def\titkol{Максимальные ветвящиеся процессы в~случайной среде}

\def\aut{А.\,В.~Лебедев$^1$}

\def\autkol{А.\,В.~Лебедев}

\titel{\tit}{\aut}{\autkol}{\titkol}

\index{Лебедев А.\,В.}
\index{Lebedev A.\,V.}


%{\renewcommand{\thefootnote}{\fnsymbol{footnote}} \footnotetext[1]
%{Работа поддержана РНФ (проект 16-11-10227).}}


\renewcommand{\thefootnote}{\arabic{footnote}}
\footnotetext[1]{Московский государственный университет им. М.\,В.~Ломоносова,
кафедра теории вероятностей ме\-ха\-ни\-ко-ма\-те\-ма\-ти\-че\-ско\-го факультета, 
\mbox{avlebed@уandex.ru}}

%\vspace*{-6pt}

    


\Abst{Работа продолжает многолетние исследования автора по теории максимальных ветвящихся
процессов (МВП), которые получаются из классических ветвящихся процессов путем замены операции
суммирования чисел потомков на максимум. Можно сказать, что в~каж\-дом поколении 
выживают
потомки только одной час\-ти\-цы, у~которой их больше всего. Ранее автором было проведено
обобщение процессов с~це\-ло\-чис\-лен\-ны\-ми значениями до процессов с~произвольными
неотрицательными значениями, исследованы их свойства и~доказаны предельные тео\-ре\-мы.
Далее были введены и~изучались процессы с~несколькими типами час\-тиц.
В~на\-сто\-ящей работе вводится понятие МВП в~случайной среде (МВПСС)
(с~одним типом час\-тиц) и~важ\-но\-го случая <<степенной>> случайной среды
(МВПССС). В~по\-след\-нем случае
изучены свойства МВП и~доказана эргодическая тео\-ре\-ма.
В~качестве приложений рассмотрены вентильные бесконечнолинейные сис\-те\-мы массового
обслуживания.}

\KW{максимальные ветвящиеся процессы; случайная среда;
эргодическая теорема; устойчивые распределения; теория экстремумов}

\DOI{10.14357/19922264180206}
  
%\vspace*{-6pt}


\vskip 10pt plus 9pt minus 6pt

\thispagestyle{headings}

\begin{multicols}{2}

\label{st\stat}

\section{Введение}

Классическими объектами исследования в~тео\-рии случайных процессов
являются ветвящиеся процессы Галь\-то\-на--Ват\-со\-на
(с~одним типом час\-тиц и~дискретным временем)~\cite{Har}.
Их экстремальные аналоги называются максимальными
ветвящимися процессами, а~именно: суммирование чисел потомков час\-тиц
(при определении чис\-лен\-ности очередного поколения) заменяется на максимум.

Напомним историю вопроса.
Максимальные вет\-вя\-щи\-еся процессы были введены и~изучались Дж.~Ламперти~\cite{Lamp1, Lamp2} в~1970--1972~гг., 
однако в~дальнейшем были совершенно всеми за\-бро\-ше\-ны 
(хотя и~упомянуты в~обзоре~\cite{VatZub2}).
Новый этап исследований МВП был начат А.\,В.~Лебедевым с~2001~г. Было проведено
обобщение процессов с~це\-ло\-чис\-лен\-ных значений на произвольные 
не\-от\-ри\-ца\-тель\-ные~\cite{Leb-2005d}.
Изучались МВП сначала с~одним типом час\-тиц (см.\ обзор~\cite{Leb-2009a}), 
а~затем с~несколькими типами
(многотипные)~\cite{Leb-2012a}. Наиболее пол\-но на данный момент 
результаты и~биб\-лио\-гра\-фия
пред\-став\-ле\-ны в~диссертации~\cite[гл.~4 и~5]{LebDiss}. 
До недавнего времени исследования автора 
по этой тематике оставались одинокими.
Лишь в~2012~г.\ они были неожиданно подхвачены в~работе зарубежных ученых
О.~Айдогмуса, А.\,П.~Гхоша, С.~Гхоша
и А.~Ройтерштейна~\cite{CMBP}, где введены раскрашенные 
МВП. Их отличие от многотипных МВП заключается в~том, что типы (цвета) час\-тиц определяются уже после
формирования поколения, случайным образом, причем тип влияет на дальнейшую пло\-до\-ви\-тость. Другим
отличием от подхода автора стало рас\-смот\-ре\-ние только процессов, уходящих в~бес\-ко\-неч\-ность.

Напомним основные понятия и~факты об~МВП с~одним типом час\-тиц.

Рассмотрим случайные процессы со значениями в~$\mathbb{Z}_+$,
заданные стохастически рекуррентными формулами вида:

\noindent
\begin{equation}
\label{f1-41}
Z_{n+1}=\bigvee_{m=1}^{Z_n}\xi_{m,n\,},
\end{equation}
где через $\bigvee$ обозначена операция взятия максимума; $\xi_{m,n}$,
$m\hm\ge 1$, $n\hm\ge 0$,~---
независимые случайные величины с~общим распределением~$F$ на~$\mathbb{Z}_+$.
Полагаем (как и~в~случае суммирования), что результат взятия максимума
<<ноль раз>> (при $Z_n\hm=0$) равен нулю.

Можно сказать (по аналогии с~процессами Галь\-то\-на--Ват\-со\-на), что 
в~МВП в~каж\-дом поколении выживают потомки
только одной час\-ти\-цы, у~которой их больше всего. Понятно так\-же, что
множество воз\-мож\-ных значений МВП (при $n\hm\ge 1$)
совпадает с~множеством воз\-мож\-ных значений чис\-ла потомков.
Из~(\ref{f1-41}) следует, что процесс является однородной цепью Маркова
на этом множестве.

Другую интерпретацию МВП можно предложить в~тео\-рии массового
обслуживания, рас\-смот\-рев вентильные бесконечнолинейные сис\-те\-мы.
Так на-\linebreak\vspace*{-12pt}

\pagebreak

\noindent
зывают сис\-те\-мы 
с~бесконечным чис\-лом приборов, в~которых доступ заявок к~обслуживанию
регулируется вентилем. Предполагается, что вентиль открыт только
в~том случае, когда все приборы свободны. Заявки по\-сту\-па\-ют в~очередь
с~бесконечным числом мест ожидания, а~обслуживание происходит по стадиям.
В~начале стадии, когда вентиль открывается, все заявки из очереди
мгновенно получают доступ к~приборам и~далее обслуживаются параллельно
и~независимо до полного осво\-бож\-де\-ния всех приборов.
В~момент освобождения всех приборов вентиль вновь открывается для новой
пар\-тии заявок (пришедших за это время) и~сле\-ду\-ющей стадии.


Заметим, что
система очень проста в~управлении: нет не\-об\-хо\-ди\-мости постоянного учета
приходящих и~уходящих заявок, свободных и~занятых приборов и~т.\,п.
Распределение заявок по приборам (которые на тот момент все свободны)
производится однократно и~единовременно в~начале каждой стадии.
Другое преимущество вентильной сис\-те\-мы может проявиться в~ситуации,
когда заявки в~очереди и~об\-слу\-жи\-ва\-ющие приборы ка\-ким-то образом
разделены между собой, а~установление связи сопряжено с~затратами.
Например, может оказаться невыгодно (или не\-воз\-мож\-но) держать постоянно
вклю\-чен\-ным канал передачи данных, а~предпочтительней недолгие
под\-клю\-че\-ния время от времени.

Разумеется, любая бесконечнолинейная сис\-те\-ма является лишь
приближением для случая, когда реальное чис\-ло приборов велико.
Вместе с~тем, имеет смысл изучение таких сис\-тем для оценки различных
характеристик качества обслуживания (которые для любой сис\-те\-мы
с~конечным числом приборов могут быть только хуже).

Рассмотрим подобную сис\-те\-му с~диск\-рет\-ным временем, и~пусть в~каж\-дый
момент времени по\-сту\-па\-ет ров\-но по одной заявке. Тогда
в~силу па\-рал\-лель\-ности работы приборов время обслуживания очередной
пар\-тии заявок (а~значит, и~чис\-ло заявок в~сле\-ду\-ющей) равно
максимуму из времен их обслуживания. Таким образом, обозначая
через~$Z_n$ дли\-тель\-ность $n$-й стадии, а~через~$\xi_{m,n}$~---
времена обслуживания заявок на ней, получаем в~точ\-ности~(\ref{f1-41}).

Вентильные бесконечнолинейные системы с~непрерывным временем 
и~пуассоновским входным потоком изучались в~[10--13]
(другими методами) и~автором в~\cite{Leb-2003, Leb-2004}.
В~этом случае необходимо оговорить, что
происходит, если вентиль открывается при пус\-той очереди.
Наиболее естественно предположить, что сис\-те\-ма ждет по\-ступ\-ле\-ния
новой заявки, с~которой и~начинается сле\-ду\-ющая стадия.

Отметим, что модели с~параллельной обработкой данных в~последние годы
стали весьма популярны в~связи с~развитием технологии облач\-ных вы\-чис\-ле\-ний.
При этом возникает необходимость изучать максимумы случайных величин.
В~качестве недавней работы на эту тему отметим~\cite{Obl}.

Максимальные ветвящиеся процессы были введены в~\cite{Lamp1} (в связи с~моделями дальнодействующей
перколяции), и~там же были получены критерии их воз\-врат\-ности, а~именно
(в~предположении $F(0)\hm=0$): при выполнении условия
\begin{equation}
\label{ulam}
\limsup\limits_{x\to+\infty}x(1-F(x))<e^{-\gamma}\,,
\end{equation}
где $\gamma=0{,}577\ldots$~--- константа Эйлера, цепь~$\{Z_n\}$ положительно
возвратна и,~напротив, при
$$
\liminf\limits_{x\to+\infty}x(1-F(x))>e^{-\gamma}$$
имеет место $Z_n\to \hm+\infty$, $n\hm\to\infty$, поч\-ти наверное (п.~н.).

Далее в~\cite{Lamp2} рассмотрен критический случай 
$$
x(1-F(x))\to e^{-\gamma},\enskip x\to+\infty\,,
$$
 с~учетом дальнейших членов раз\-ло\-же\-ния хвос\-та на бес\-ко\-неч\-ности.
Показано, что если 
$$
\left(e^\gamma x(1-F(x))-1\right)\ln x\to d,\enskip
x\to+\infty\,,
$$
 то процесс возвратен при $d\hm<\pi^2/12$ и~уходит 
в~бес\-ко\-неч\-ность п.~н.\  при $d\hm>\pi^2/12$.

Согласно~(\ref{f1-41}) и~предположению о~случае $Z_n\hm=0$ процесс
имеет переходные вероятности
$$
{\bf P}\left(Z_{n+1}\le j|Z_n=i\right)=F^i(j)\,,\quad i,j\in \mathbb{Z}_+$$
(где полагаем $0^0\hm=1$),
что подсказало  рассмотреть в~\cite{Leb-2005d} цепи Маркова на произвольном измеримом
множестве $T\hm\subset \mathbb{R}_+$ с~переходными вероятностями
\begin{equation}
\label{obo}
{\bf P}\left(Z_{n+1}\le y|Z_n=x\right)=F^x(y)\,,\enskip x,y\in T\,,
\end{equation}
где $F$ также сосредоточено на~$T$.

Подобные процессы могут рас\-смат\-ри\-вать\-ся как самостоятельно,
так и~в~качестве предельных (в~ка\-ком-ли\-бо смысле) для
МВП на~${\bf Z}_+$ (нормированных определенным образом).
Например, они могут пригодиться для описания поведения
вентильных бесконечнолинейных сис\-тем, в~том числе предельного
при большой загрузке и~т.\,п.

В частности, для вентильной бесконечнолинейной сис\-те\-мы с~непрерывным временем
и~пуассоновским входным потоком последовательность длительностей стадий
на протяжении периода за\-ня\-тости удовле\-тво\-ря\-ет~(\ref{obo}) с~$F(x)\hm=\exp\{-\lambda
{\bar B}(x)\}$, $x\hm\ge 0$,
где $\lambda$~--- ин\-тен\-сив\-ность входного потока; $B(x)$~---
функция распределения времени обслуживания одной заявки,
${\bar B}(x)\hm=1\hm-B(x)$.
Действительно,
обозначим дли\-тель\-ность $n$-й стадии через~$Z_n$. При условии $Z_n\hm=x$
чис\-ло заявок, по\-сту\-пив\-ших на данной стадии, пуассоновское с~па\-ра\-мет\-ром~$\lambda x$, 
а~$Z_{n+1}$ есть максимум этого (случайного) числа
независимых случайных величин с~распределением~$B$. Таким образом,

\noindent
\begin{multline*}
{\bf P}(Z_{n+1}\le y|Z_n=x)=
\sum\limits_{k=0}^\infty\fr{(\lambda x)^k}{k!}\,e^{-\lambda x}B(y)^k={}\\
{}=
\exp\left\{-\lambda x {\bar B}(y)\right\}=F(y)^x.
\end{multline*}

Для описания сис\-те\-мы на всем протяжении времени требуются уже МВП
с~иммиграцией в~момент обнуления и~т.\,п.~\cite[\S\ 4.3]{LebDiss}

Далее будем говорить об~МВП
на~$T$ и~обозначать их МВП($T$). Заметим, что аналогичным обобщением
для процессов Галь\-тон\-а--Ват\-со\-на являются процессы Иржины~\cite{Jr2}
(с~непрерывным множеством со\-сто\-яний и~дискретным временем), 
однако в~данном случае~$T$ может быть любым измеримым подмножеством~$\mathbb{R}_+$.

Формула~(\ref{f1-41}) для таких МВП в~общем случае уже не имеет мес\-та,
но в~соответствии с~(\ref{obo}) они допускают эквивалентное
пред\-став\-ле\-ние стохастической рекуррентной последовательностью вида

\noindent
\begin{equation*}
%\label{obo2}
Z_{n+1}=\begin{cases}
\displaystyle
F^{-1}(U_{n+1}^{1/Z_n}),&\ Z_n>0\,;\\
\displaystyle
0,&\ Z_n=0\,,\end{cases}\enskip n\ge 0\,,
\end{equation*}
где $F^{-1}(y)=\inf\{x: F(x)\ge y\}$ и~$U_n$, $n\hm\ge 1$,~---
независимые равномерно распределенные на $(0,1)$ случайные величины,
а~$Z_0\hm\ge 0$ и~не зависит от них. При этом распределение~$F$
по-преж\-не\-му называется распределением чис\-ла (непосредственных) потомков.

В~\cite{Leb-2005d} был получен ряд свойств МВП (свойство преобразования подобия,
ассоциированность, монотонность по па\-ра\-мет\-рам, условие вырождения)
и~доказана тео\-ре\-ма эргодичности.

Заметим, что для МВП нуль всегда является
по\-гло\-ща\-ющим со\-сто\-яни\-ем. Так, для МВП($\mathbb{Z}_+$) 
при условии~(\ref{ulam})
и~$F(0)\hm>0$ это приводит к~вы\-рож\-де\-нию п.~н.~\cite{Lamp1}. Для МВП($T$), если 
$F(0)\hm=0$,
нуль можно\linebreak просто исключить из множества со\-сто\-яний, рас\-смат\-ри\-вая процесс 
с~ненулевым начальным условием. Однако, если нуль является предельной\linebreak
 точ\-кой~$T$,
остается воз\-мож\-ность асимптотической схо\-ди\-мости к~нему при $n\hm\to\infty$.
Следующая тео\-ре\-ма дает достаточные условия для того, чтобы исключить уход
процесса как в~нуль, так и~в~бес\-ко\-неч\-ность, и~сделать его эргодическим.
Здесь и~далее имеется в~виду эр\-го\-дич\-ность по Харрису~\cite[гл. 1]{Borov}.

\smallskip

\noindent
\textbf{Теорема~A}~\cite[теорема 1]{Leb-2005d}. 
\textit{Если для $\mathrm{МВП}(T)$, $T\hm\subset (0,+\infty)$,
выполнено}~(\ref{ulam}) \textit{и}

\noindent
$$
\liminf\limits_{x\to 0}x(-\ln F(x))>e^{-\gamma}\,,
$$
\textit{то процесс эргодический.}

\columnbreak

%\smallskip

Определим теперь МВПСС.

В приложениях <<случайная среда>> может описывать различные 
факторы природного, технического
или общественного характера, ме\-ня\-ющи\-еся со временем случайным 
образом и~ока\-зы\-ва\-ющие влияние
на сис\-те\-му, например уско\-ря\-ющие или за\-мед\-ля\-ющие работу 
сис\-те\-мы массового об\-слу\-жи\-ва\-ния.

Пусть заданы последовательность~$F_l$, $l\hm\ge 1$, распределений на~$\mathbb{Z}_+$,
набор независимых случайных величин~$\xi_{m,n,l}$, $m\hm\ge 1$, $n\hm\ge 0$, $l\hm\ge 1$,
с распределениями~$F_l$, $l\hm\ge 1$, и~по\-сле\-до\-ва\-тель\-ность независимых случайных 
величин~$\nu_n$, $n\hm\ge 0$, с~общим распределением~$G$ на~$\mathbb{N}$, не зависящих 
от~$\xi_{m,n,l}$, $m\hm\ge 1$, $n\hm\ge 0$, $l\hm\ge 1$. Тогда МВПСС($\mathbb{Z}_+$) 
можно определить
с~по\-мощью стохастически рекуррентной формулы вида
\begin{equation}
\label{f1-41-mvpss}
Z_{n+1}=\bigvee\limits_{m=1}^{Z_n}\xi_{m,n,\nu_n}\,.
\end{equation}
Случайную среду здесь отражают случайные величины~$\nu_n$, $n\hm\ge 1$, 
от которых зависит
выбор распределения~$F_l$, $l\hm\ge 1$, чис\-ла по\-том\-ков час\-тиц на каж\-дом шаге.

Согласно~(\ref{f1-41-mvpss}) и~предположению о~случае $Z_n\hm=0$ процесс
имеет переходные вероятности
$$
{\bf P}\left(Z_{n+1}\le j|Z_n=i\right)={\bf E}
\left(F^i_\nu(j)\right),\enskip i,j\in \mathbb{Z}_+\,,
$$
где $\nu$ имеет распределение~$G$, что подсказывает 
рас\-смот\-реть процесс на произвольном измеримом
множестве $T\hm\subset \mathbb{R}_+$ с~переходными вероятностями
\begin{equation}
\label{obo-mvpss}
{\bf P}\left(Z_{n+1}\le y|Z_n=x\right)={\bf  E}\left(F^x_\nu(y)\right),\enskip x,y\in T\,,
\end{equation}
где имеется в~виду семейство распределений~$F_s$, $s\hm>0$, на~$T$ 
и~случайная величина~$\nu$ с~распределением~$G$ на~$(0,+\infty)$. 
Таким образом, определяем МВПСС($T$).

Подобные процессы могут рас\-смат\-ри\-вать\-ся как самостоятельно,
так и~в~качестве предельных (в~ка\-ком-ли\-бо смысле) для
МВПСС($\mathbb{Z}_+$), нормированных определенным образом.

Далее будем изучать класс процессов со степенным семейством $F_s(x)\hm=F^s(x)$, $s\hm>0$,
где~$F$~--- распределение на~$T$. В~таком случае будем говорить 
о~<<степенной>> случайной среде,
а~со\-от\-вет\-ст\-ву\-ющие процессы обозначим МВПССС($T$).

Введем преобразование Лап\-ла\-са--Стилть\-еса 
$\varphi(u)\hm={\bf E}e^{-u\nu}$. В~этом случае формула~(\ref{obo-mvpss}) 
преобразуется к~виду:
\begin{equation}
\hspace*{-2mm}
\label{obo-mvpss-2}
{\bf P}\left(Z_{n+1}\le y|Z_n=x\right)=\varphi(-x\ln F(y)),\enskip x,y\in T.\!\!
\end{equation}
Поставим задачу изуче\-ния свойств таких процессов.

\section{Основные свойства}

Прежде всего отметим, что для МВПССС так\-же имеет мес\-то конструктивное
пред\-став\-ле\-ние, сле\-ду\-ющее из~(\ref{obo-mvpss-2}), а~именно:
\begin{multline}
\label{obo2-mvpss}
%\hspace*{-2.23734pt}\!\!
Z_{n+1}=\begin{cases}
\displaystyle
F^{-1}\left(\exp\left\{-\fr{\varphi^{-1}(U_{n+1})}{Z_n}\right\}\right),& Z_n>0\,;\\
\displaystyle
0,& Z_n=0\,,\end{cases}\\ n\ge 0\,,
\end{multline}
где $U_n$, $n\ge 1$,~--- независимые равномерно распределенные на $(0,1)$ 
случайные величины,
$Z_0\hm\ge 0$ и~не зависит от них.

Докажем ряд свойств МВПССС по аналогии с~\cite{Leb-2005d} в~форме утверж\-де\-ний.

\smallskip

\noindent
\textbf{Утверждение~1.}\ \textit{Если~$\{Z_n\}$ является $\mathrm{МВПССС}(T)$ с~$F(x)$, то
$\{\lambda Z_n\}$ при любом $\lambda>0$ является $\mathrm{МВПССС}(\lambda T)$ 
c~$F^{1/\lambda}(x/\lambda)$.}

\smallskip

\noindent
Д\,о\,к\,а\,з\,а\,т\,е\,л\,ь\,с\,т\,в\,о\,.\ \
Используем подстановку в~(\ref{obo-mvpss-2}).
Действительно,
\begin{multline*}
{\bf P}\left(\lambda Z_{n+1}\le y|\lambda Z_n=x\right)={}\\
{}=
{\bf P}\left(Z_{n+1}\le \fr{y}{\lambda}|\lambda Z_n=\fr{x}{\lambda}\right)={}\\
\!{}=\varphi\left(-\left(\fr{x}{\lambda}\right)\ln 
F\left(\fr{y}{\lambda}\right)\!\right)=
\varphi\!\left(-x\ln F^{1/\lambda}\left(\fr{y}{\lambda}\right)\!\right)\!.\hspace*{2mm}\square\hspace*{-8.36485pt}
\end{multline*}

Из данного свойства следует замк\-ну\-тость класса МВПССС($\mathbb{R}_+$) относительно
умножения на $\lambda\hm>0$ и~замк\-ну\-тость МВПССС($\mathbb{Z}_+$) при $\lambda
\hm\in \mathbb{N}$.

\smallskip

\noindent
\textbf{Лемма~1.}\ \textit{Для любых чисел $Z'_0\hm\le Z''_0$ и~$U'_n\hm\le U''_n$,
$n\ge 1$, где $Z'_0, Z''_0\hm\ge 0$ и~$U'_n,U''_n\hm\in (0,1)$, чис\-ло\-вые
последовательности $\{Z'_n\}$ и~$\{Z''_n\}$, по\-стро\-ен\-ные по формуле}~(\ref{obo2-mvpss}), 
\textit{удовле\-тво\-ря\-ют условию $Z'_n\hm\le Z''_n$ для всех $n\hm\ge 0$.}

\smallskip

\noindent
Д\,о\,к\,а\,з\,а\,т\,е\,л\,ь\,с\,т\,в\,о\,.\ \
По условию $Z'_0\hm\le Z''_0$. Пусть верно
$Z'_n\hm\le Z''_n$ для некоторого $n\hm\ge 0$. Тогда, поскольку~$F^{-1}$~---
не\-убы\-ва\-ющая функция, из $U'_{n+1}\hm\le U''_{n+1}$ и~формулы~(\ref{obo2-mvpss})
получаем $Z'_{n+1}\hm\le Z''_{n+1}$. Утверж\-де\-ние лем\-мы вер\-но по принципу
математической индукции.\hfill$\square$


\smallskip


Напомним понятие ассоциированности случайных величин~\cite{EPW, Bul}.

Функцию многих переменных~$f(x)$, где $x\hm=(x_1,\dots , x_n)$, назовем
монотонно неубывающей, если из $x'_i\hm\le x''_i$, $1\hm\le i\hm\le n$, следует
$f(x')\hm\le f(x'')$. Случайные величины набора $\zeta\hm=(\zeta_1,\dots,\zeta_n)$
называются ассоциированными, если $\mathrm{cov}\,(f(\zeta),g(\zeta))\hm\ge 0$ для
всех тех монотонно неубывающих~$f$ и~$g$, для которых эта ковариация
существует. Говорят, что случайный процесс или поле $\{\zeta(t): t\hm\in
{\cal T}\}$ ассоциированы, если ассоциированы их значения $\zeta(t_1),
\dots,\zeta(t_n)$ для любого конечного множества $\{t_1,\dots,t_n\}\hm\subset
{\cal T}$.

Согласно~\cite{EPW} и~\cite[теорема 1.8]{Bul}, независимые случайные величины
ассоциированы;
монотонно неубывающие функции от ассоциированных случайных величин
так\-же обладают этим свойством.

\smallskip

\noindent
\textbf{Утверждение~2.} \textit{Любой МВПCCC ассоциирован.}

\smallskip

\noindent
Д\,о\,к\,а\,з\,а\,т\,е\,л\,ь\,с\,т\,в\,о\,.\ \
До\-ста\-точ\-но заметить, что по лем\-ме~1
любые $Z_{i_1},\dots, Z_{i_m}$ пред\-став\-ля\-ют собой неубывающие функции
от независимых случайных величин~$Z_0$ и~$U_n$, $1\hm\le n\hm\le\max\{i_1,\dots,i_m\}$.\hfill$\square$

\smallskip

Для установления монотонности по па\-ра\-мет\-рам введем отношение (час\-тич\-но\-го) порядка
меж\-ду распределениями: $F_1\hm\prec F_2$, если $F_1(x)\hm\ge F_2(x)$, 
$\forall\ x$. %\label{s119}
Заметим, что из $F_1\hm\prec F_2$ следует $F^{-1}_1(y)\hm\le F^{-1}_2(y)$, 
$\forall\ y\hm\in (0,1)$.

Обозначим через $Z\hm={\cal Z}(F,G,H)$ МВПCCC, где~$Z_0$ имеет распределение~$H$.

\smallskip

\noindent
\textbf{Утверждение 3.} \textit{Если $F'\prec F''$, $G'\prec G''$ и~$H'\prec H''$, то можно построить
процессы $Z'\hm={\cal Z}(F',G',H')$ и~$Z''\hm={\cal Z}(F'',G'',H'')$ на одном вероятностном
пространстве так, что $Z'_n\hm\le Z''_n$ для всех $n\hm\ge 0$ п.~н.}

\smallskip

\noindent
Д\,о\,к\,а\,з\,а\,т\,е\,л\,ь\,с\,т\,в\,о\,.\ \ Пусть $U_0$~--- 
равномерно распределенная
на $(0,1)$ случайная величина, не зависящая от~$U_n$, $n\hm\ge 1$. Полагая
$Z'_0\hm=(H')^{-1}(U_0)$ и~$Z''_0\hm=(H'')^{-1}(U_0)$, получаем $Z'_0\hm\le Z''_0$.
Пусть верно $Z'_n\hm\le Z''_n$ для некоторого $n\hm\ge 0$.
Из $G'\hm\prec G''$ следует $\varphi'(u)\hm\ge \varphi''(u)$, $\forall\ u\hm>0$, 
и~$(\varphi'(v))^{-1}\hm\ge (\varphi''(v))^{-1}$, $\forall\ v\hm\in (0,1)$.
Из $F'\hm\prec F''$ следует $(F')^{-1}(y)\hm\le (F'')^{-1}(y)$, $\forall\ y\hm\in (0,1)$.
По формуле~(\ref{obo2-mvpss}) получаем $Z'_{n+1}\hm\le Z''_{n+1}$.
Утверж\-де\-ние~3 доказано по принципу математической индукции.\hfill$\square$


Обозначим предельное распределение~$Z_n$ при $n\hm\to\infty$ 
через~$\Psi$ (если оно существует).

\smallskip

\noindent
\textbf{Утверждение 4.} \textit{Если для двух $\mathrm{МВПCCC}(T)$ верно $F'\hm\prec F''$,
$G'\hm\prec G''$, то $\Psi'\hm\prec\Psi''$.}

\smallskip

\noindent
Д\,о\,к\,а\,з\,а\,т\,е\,л\,ь\,с\,т\,в\,о\,.\ \
Возьмем произвольные $H'\hm=H''$ на~$T$
и~по\-стро\-им процессы на одном вероятностном про\-стран\-ст\-ве 
со\-глас\-но утверж\-де\-нию~3.
Тогда из $Z'_n\hm\le Z''_n$ п.~н.\  следует
${\bf P}(Z'_n\hm\le x)\hm\ge {\bf P}(Z''_n\hm\le x)$, $n\hm\ge 1$,
откуда при $n\hm\to\infty$ получаем $\Psi'(x)\hm\ge\Psi''(x)$, $x\hm>0$.\hfill$\square$


\section{Эргодическая теорема}

Напомним функцию распределения Гумбеля
$\Lambda(x)\hm=\exp\{-e^{-x}\}$, игра\-ющую важ\-ную роль в~сто\-ха\-сти\-че\-ской 
тео\-рии экстремумов.

Если функция распределения~$F(x)$ непрерывна 
и~строго возрастает, а~$F(0)\hm=0$, то~(\ref{obo2-mvpss}) преобразованием
$\zeta_n\hm=\Lambda^{-1}(F(Z_n))$ приводится
к~форме общей (нелинейной) авторегрессии первого порядка
\begin{equation}
\label{nepr}
\zeta_{n+1}=f(\zeta_n)+\eta_{n+1}\,,\enskip n\ge 0\,,
\end{equation}
где $f(u)=\ln F^{-1}(\Lambda(u))$ и~независимые случайные величины
$\eta_n\hm=-\ln \varphi^{-1}(U_n)$, $n\hm\ge 1$, име\-ют функцию 
распределения~$\varphi(e^{-x})$.

Эргодичность подобных моделей изучалась, например, 
в~\cite[\S~8.4]{Borov} и~\cite{Bhat}.

Обозначим $\delta\hm={\bf E}\eta_1$ и~пусть это сред\-нее существует. Имеем
$$
\varphi(e^{-x})={\bf E}\exp\left\{-\nu e^{-x}\right\}={\bf E}\Lambda(x-\ln\nu)\,,
$$
откуда следует

\noindent
\begin{equation}
\label{oprdelta}
\delta=\gamma+{\bf E}\ln\nu\,,
\end{equation}
поскольку распределение Гумбеля~$\Lambda$ имеет математическое ожидание~$\gamma$.

Заметим, что $\delta$ как математическое ожидание при функции распределения
$\varphi(e^{-x})$ может быть так\-же пред\-став\-ле\-но формулой

\vspace*{-4pt}

\noindent
\begin{multline}
\label{formdelta}
\delta=\int\limits_0^{+\infty}\left(1-\varphi\left(e^{-x}\right)\right)\,dx-
\int\limits_{-\infty}^0\varphi\left(e^{-x}\right)\,dx={}\\[-1pt]
{}=
\int\limits_0^\infty\left(1-\varphi\left(e^{-x}\right)\right)\,dx-
\int\limits_0^\infty\varphi\left(e^x\right)\,dx\,,
\end{multline}
когда все интегралы сходятся.

\vspace*{2pt}

\noindent
\textbf{Пример~1.}\
Пусть $F(x)\hm=\exp\{-(x/c)^{-\beta}\}$, $x,c,\beta\hm>0$ (распределение
Фреше), тогда МВПССС допускает конструктивное пред\-став\-ле\-ние:
\begin{equation}
\label{zw}
Z_{n+1}=W_{n+1}Z_n^{1/\beta}\,,
\end{equation}
где $W_n$, $n\hm\ge 1$, независимы и~име\-ют функцию распределения 
$\varphi((x/c)^{-\beta})$,
а~(\ref{nepr}) переписывается в~форме линейной авторегрессии:
\begin{equation}
\label{zb}
\zeta_{n+1}=\fr{\zeta_n}{\beta}+\ln c+\eta_{n+1}\,,\quad n\ge 0\,.
\end{equation}

При $\beta<1$ процесс~$\{\zeta_n\}$
уходит в~$\pm\infty$ в~за\-ви\-си\-мости от знака начального усло\-вия~$\zeta_0$.
При $\beta\hm>1$ процесс~$\{\zeta_n\}$ эргодический. При $\beta\hm=1$ имеем
прос\-тое случайное блуж\-да\-ние, уходящее в~$+\infty$ при $c\hm>e^{-\delta}$,
в~$-\infty$ при $c\hm<e^{-\delta}$ и~ос\-цил\-ли\-ру\-ющее меж\-ду~$\pm\infty$ при
$c\hm=e^{-\delta}$.

Отсюда легко получить результаты для~$\{Z_n\}$, если
учесть, что $Z_n\hm=F^{-1}(\Lambda(\zeta_n))$, откуда $Z_n\hm\to 0$ при
$\zeta_n\hm\to -\infty$ и~$Z_n\hm\to +\infty$ при $\zeta_n\hm\to +\infty$,
а~эргодичность со\-хра\-ня\-ется.

Данный пример наводит на мысль, что верна следующая теорема.

\smallskip

\noindent
\textbf{Теорема~1.}\ \textit{Если для $\mathrm{МВПCCC}(T)$, $T\hm\subset (0,+\infty)$, 
выполнены условия}

\noindent
\begin{align}
\label{usl1}
\liminf\limits_{x\to+\infty} x(-\ln F(x))&<e^{-\delta}\,;
\\
\label{usl2}
\liminf_{x\to 0} x(-\ln F(x))&>e^{-\delta},
\end{align}
\textit{где $\delta$ из}~(\ref{oprdelta}) \textit{существует, то процесс эргодический.}

\columnbreak

%\smallskip

\noindent
Д\,о\,к\,а\,з\,а\,т\,е\,л\,ь\,с\,т\,в\,о\,.\ \
Без ограничения общ\-ности мож\-но считать
$T\hm=(0,+\infty)$. Прежде всего заметим, что условия~(\ref{usl1}) 
и~(\ref{usl2}) эквивалентны утверж\-де\-нию о~существовании
таких $0\hm<x_1\hm<x_2$ и~$0\hm<c_2\hm<e^{-\delta}\hm<c_1$, что $F(x)\hm\le
e^{-c_1/x}$ при $x\hm\le x_1$ и~$F(x)\hm\ge e^{-c_2/x}$ при $x\hm\ge x_2$.

В качестве функции Ляпунова рас\-смот\-рим 

\noindent
$$
g(x)=\left(\ln\left(\fr{x}{x_2}\right)\right)_+ +\left(\ln\left(\fr{x_1}{x}\right)\right)_+\,,
$$
где $y_+\hm=\max\{0,y\}$. Обозначим 

\noindent
$$
\mu(x)={\bf E}\left(\left.g\left(Z_{n+1}\right)\right\vert Z_n=x\right)-g(x)\,,
$$
тогда

\vspace*{-6pt}

\noindent
\begin{multline*}
\mu(x)={\bf E}\left\{\ln\left(\fr{Z_{n+1}}{x_2}\right)_+|Z_n=x\right\}+{}\\[-1pt]
{}+
{\bf E}\left\{\ln\left(\left\vert \fr{x_1}{Z_{n+1}}\right)_+
\right\vert Z_n=x\right\}-g(x)={}\\[-1pt]
{}=\int\limits_0^\infty{\bf P}
\left(\ln\left(\fr{Z_{n+1}}{x_2}\right)>y|Z_n=x\right)\,dy+{}\\[-1pt]
{}+
\int\limits_0^\infty{\bf P}\left(\ln\left(\fr{x_1}{Z_{n+1}}\right)>y|Z_n=x\right)\,dy-
g(x)={}\\[-1pt]
{}=\int\limits_0^\infty\left(1-\varphi\left(-x\ln F\left(x_2e^y\right)\right)\right)\,dy+{}\\[-1pt]
{}+
\int\limits_0^\infty\!\varphi\left(-x\ln F\left(x_1e^{-y}\right)\right)\,dy-
g(x)\le{}\\ %\displaystyle
{}\le\int\limits_0^\infty
\left(1-\varphi\left(\left(\fr{c_2x}{x_2}\right)e^{-y}\right)\right)\,dy+{}\\[-1pt]
{}+
\int\limits_0^\infty\!\varphi\left(\left(\fr{c_1x}{x_1}\right)e^y\right)\,dy-g(x)={}\\[-1pt]
{}=\int\limits_{-\ln(c_2x/x_2)}^\infty\hspace*{-6mm}
\left(1-\varphi\left(e^{-z}\right)\right)\,dz+\hspace*{-2mm}
\int\limits_{\ln(c_1x/x_1)}^\infty\hspace*{-3mm}\varphi\left(e^z\right)\,dz-g(x)={}\\[-1pt]
{}=\int\limits_0^\infty\left(1-\varphi\left(e^{-z}\right)\right)\,dz+
\int\limits_0^\infty\varphi\left(e^z\right)\,dz+{}\\[-1pt]
{}+  \hspace*{-2mm}
\int\limits_{-\ln(c_2x/x_2)}^0\hspace*{-6mm} \left(1-\varphi\left(e^{-z}\right)\right)\,dz-
\hspace*{-4mm}\int\limits_0^{\ln(c_1x/x_1)}\hspace*{-4mm}\varphi\left(e^z\right)\,dz-{}\\[-1pt]
{}-\left(\left(\ln\left(\fr{x}{x_2}\right)\right)_+
+\left(\ln\left(\fr{x_1}{x}\right)\right)_+\right).
\end{multline*}
Обозначим полученную оценку сверху для $\mu(x)$ через~$\mu^*(x)$ и~пусть
\begin{multline*}
{\tilde\mu}(x)=\int\limits_{-\ln(c_2x/x_2)}^0
\left(1-\varphi\left(e^{-z}\right)\right)\,dz-{}\\
{}-\hspace*{-9.46472pt}\int\limits_0^{\ln(c_1x/x_1)}\hspace*{-4.5mm}\varphi
\left(e^z\right)\,dz
-\left(\left(\ln\left(\fr{x}{x_2}\right)\right)_+
+\left(\ln\left(\fr{x_1}{x}\right)\right)_+\right),
\end{multline*}
тогда
\begin{multline}
\label{formmu}
\mu(x)\le \mu^*(x)={}\\
{}=\int\limits_0^\infty
\left(1-\varphi\left(e^{-z}\right)\right)\,dz+
\int\limits_0^\infty\!\varphi\left(e^{-z}\right)\,dz+{\tilde\mu}(x)\,.
\end{multline}

При $x\ge x_2$ имеем:
\begin{multline*}
{\tilde\mu}(x)=\hspace*{-2mm}\int\limits_{-\ln(c_2x/x_2)}^0\hspace*{-4mm}
\left(1-\varphi\left(e^{-z}\right)\right)\,dz-\hspace*{-4mm}
\int\limits_0^{\ln(c_1x/x_1)}\hspace*{-4mm}\varphi\left(e^z\right)\,dz-{}\\
{}-
\ln\left(\fr{c_2x}{x_2}\right)+\ln c_2={}\\
{}=-\hspace*{-2mm}\int\limits_{-\ln(c_2x/x_2)}^0\hspace*{-4mm}\varphi\left(e^{-z}\right)\,dz-
\int\limits_0^{\ln(c_1x/x_1)}\hspace*{-4mm}\varphi\left(e^z\right)\,dz+
\ln c_2\to{}\\
{}\to -2\int\limits_0^\infty \varphi\left(e^z\right)\,dz+\ln c_2,\enskip
 x\to+\infty\,,
\end{multline*}
откуда в~силу~(\ref{formdelta}) и~(\ref{formmu}) верно
$$
\mu(x)\hm\le\mu^*(x)\to\delta+\ln c_2=\ln \left(\!\fr{c_2}{e^{-\delta}}\!\right)<0, \
x\to +\infty.
$$

При $x\le x_1$ имеем:
\begin{multline*}
{\tilde\mu}(x)=\hspace*{-2mm}\int\limits_{-\ln\left({c_2x}/{x_2}\right)}^0
\hspace*{-6mm}\left(1-\varphi\left(e^{-z}\right)\right)\,dz-
\int\limits_0^{\ln(c_1x/x_1)}\hspace*{-4mm}\varphi\left(e^z\right)\,dz+{}\\
{}+
\ln\left(\fr{c_1x}{x_1}\right)-\ln c_1=
-\hspace*{-2mm}\int\limits_{-\ln(c_2x/x_2)}^0\hspace*{-4mm}\varphi\left(e^{-z}\right)\,dz-{}\\
{}-
\int\limits_0^{-\ln(c_1x/x_1)}\hspace*{-4mm}
\left(1-\varphi\left(e^{-z}\right)\right)\,dz-\ln c_1\to{}\\
{}\to -2\int\limits_0^\infty
\left(1-\varphi\left(e^{-z}\right)\right)\,dz-\ln c_2,\enskip x\to 0\,,
\end{multline*}
откуда в~силу (\ref{formdelta}) и~(\ref{formmu}) верно
$$
\mu(x)\le\mu^*(x)\to -\delta-\ln c_1=-\ln \left(\fr{c_1}{e^{-\delta}}\!\right)<0,\
x\to 0.
$$

Следовательно, существуют такие $\varepsilon\hm>0$
и~$0\hm<v_1\hm\le v_2$, что $\mu(x)\hm\le-\varepsilon$ при $x\hm\notin V\hm=[v_1,v_2]$.
Кроме того, 
$$
\sup\limits_{x\in V}{\bf E}\left(g(Z_{n+1})|Z_n=x\right)<\infty\,.
$$
 Таким
образом, условия Ляпунова~\cite[\S~4.2]{Borov} выполнены.

Проверим теперь условие перемешивания.

Пусть $0<u_1\hm\le u\hm\le u_2$, $0\hm<a\hm<b$. В~силу вы\-пук\-лости и~убывания~$\varphi(s)$ верно
\begin{multline*}
\varphi(-u\ln F(b))-\varphi(u\ln F(a))\ge{}\\
{}\ge\fr{u_1}{u_2}
\left(\varphi(-u_2\ln F(b))-\varphi(-u_2\ln F(a))\right),
\end{multline*}
откуда следует, что для любого измеримого~$B$ выполняется
\begin{multline}
\label{perem}
\!\!{\bf P}\left(Z_{n+1}\in B|Z_n=x\right)\ge\fr{v_1}{v_2}\,{\bf P}
\left(Z_{n+1}\in B|Z_n=v_2\right),
\\
 \forall x\in V\,.
\end{multline}

Кроме того, любой МВПCCC неприводим и~апериодичен (иначе говоря,
из любого со\-сто\-яния $x\hm\in T$ можно по\-пасть в~любое множество
$B\hm\subset T$, причем за один шаг).
Из условий Ляпунова и~(\ref{perem}) по~\cite[\S~2, тео\-ре\-ма~2]{Borov} следует
эр\-го\-дич\-ность \mbox{МВПCCC}.\hfill$\square$


Отметим, что при $\nu=1$ п.~н.\  тео\-ре\-ма~1 сводится к~тео\-ре\-ме~А, 
поскольку тогда $\delta\hm=\gamma$.

В некоторых случаях $\delta$ удобней вы\-чис\-лять по определению, чем 
по формуле~(\ref{oprdelta}).

\smallskip

\noindent
\textbf{Пример~2.}\ Пусть~$\nu$ имеет показательное распределение со средним~$\theta$, тогда
$\varphi(u)\hm=(1\hm+\theta u)^{-1}$ и~$$
\varphi \left(e^{-x}\right)=\fr{1}{1+\theta e^{-x}}=\fr{1}{1+e^{-(x-\ln\theta)}}\,,
$$
т.\,е.\ получаем логистическое распределение с~па\-ра\-мет\-ром сдвига~$\ln\theta$. 
Следовательно,
$\delta\hm=\ln\theta$.

\smallskip

\noindent
\textbf{Пример~3.}\ Пусть~$\nu$ имеет строго устойчивое распределение 
с~$\varphi(u)\hm=e^{-cu^\alpha}$,
$c\hm>0$, $0\hm<\alpha\hm<1$, тогда
$$
\varphi\left(e^{-x}\right)=\exp\{-ce^{-\alpha x}\}=\Lambda(\alpha x-\ln c)\,,
$$
откуда
$$
\delta=\fr{\gamma+\ln c}{\alpha}\,.
$$

Это, кстати, дает удобный способ вы\-чис\-ле\-ния сред\-не\-го логарифма от
устойчивой случайной величины. С~по\-мощью~(\ref{oprdelta}) получаем:
$$
{\bf E}\ln\nu=\fr{\gamma(1-\alpha)+\ln c}{\alpha}\,.
$$

Для эргодического $\mathrm{МВПCCC}(T)$
обозначим случайную величину с~предельным распределением через~${\tilde Z}$.
В~некоторых случаях удается найти чис\-ло\-вые характеристики этого рас\-пре\-де\-ления.

\smallskip

\noindent
\textbf{Пример~4.}\ Пусть~$\nu$ имеет строго устойчивое рас\-пре\-де\-ле\-ние с~$\varphi(u)
\hm=e^{-u^\alpha}$,
$0\hm<\alpha\hm<1$,\linebreak и~$F(x)\hm=\exp\{-x^{-\beta}\}$, $x,\beta\hm>0$,
 тогда получаем пред\-став\-ле\-ние~(\ref{zw}), где $W_n$, $n\hm\ge 1$, 
 име\-ют функцию распределения Фреше $\exp\{-x^{-\alpha\beta}\}$, $x\hm>0$, и~$$
{\bf E}W_1^s=\Gamma\left(1-\fr{s}{\alpha\beta}\right),\enskip 0<s<\alpha\beta\,.
$$
Из представления~(\ref{zw}) и~эргодичности процесса следует, что предельным является
распределение сле\-ду\-юще\-го бесконечного произведения, которое сходится п.~н.:
$$
{\tilde Z}\stackrel{d}{=}\prod\limits_{n=0}^\infty W_n^{1/\beta^n}\,,
$$
откуда
$$
{\bf E}{\tilde Z}^s=\prod\limits_{n=1}^\infty \Gamma\left(1-\fr{s}{\alpha\beta^n}
\right),\enskip 0<s<\alpha\beta\,.
$$

Очевидно, утверж\-де\-ние~4 верно и~в~тех случаях, когда одно или оба предельных
распределения со\-сре\-до\-то\-че\-ны в~нуле. Отсюда, в~частности, можно получить условие вырождения
для процессов с~$F(0)\hm>0$. Здесь под вы\-рож\-де\-ни\-ем понимается обращение процесса в~нуль
начиная с~некоторого (случайного) момента.

\smallskip

\noindent
\textbf{Следствие~1.}\ 
Если для МВПCCC($\mathbb{R}_+$) выполнено~(\ref{usl1}) 
и~$F(0)\hm>0$, то процесс вырождается п.~н.


\noindent
Д\,о\,к\,а\,з\,а\,т\,е\,л\,ь\,с\,т\,в\,о\,.\ \
Заметим, что для любых $C\hm>0$ и~$n\hm\ge 0$ по формуле~(\ref{obo-mvpss-2}) верно
\begin{multline*}
{\bf P}\left(Z_{n+1}=0\right)={\bf E}\varphi\left(-Z_n\ln F(0)\right)\ge{}\\
{}\ge
{\bf P}\left(Z_n=0\right)+\varphi(-C\ln F(0)){\bf P}(0<Z_n\le C)\,.
\end{multline*}
Последовательность ${\bf P}(Z_n\hm=0)$ монотонно неубывает и~ограничена,
а~значит, стремится к~некоторому пределу $p_0\hm\in (0,1]$. Получаем:
$$
\sum\limits_{n=0}^\infty{\bf P}\left(0<Z_n\le C\right)\le 
\fr{p_0}{\varphi(-C\ln F(0))}<\infty\,,
$$
так что по лемме Бо\-ре\-ля--Кан\-тел\-ли~$Z_n$ попадает в~$(0,C]$ ко\-неч\-ное
чис\-ло раз п.~н.\  при любом $C\hm>0$, что может означать либо вы\-рож\-де\-ние,
либо уход в~бес\-ко\-неч\-ность при $n\hm\to\infty$.

Пусть $F^*(x)\hm=F(x)\exp\{-x^{-2}\}{\bf I}(x\hm>0)$,
тогда $F\hm\prec F^*$. Согласно утверж\-де\-нию~3
мож\-но по\-стро\-ить МВПССС($\mathbb{R}_+$) c~$F^*$ такой, что $Z_n\hm\le Z^*_n$ п.~н.
Заметим, что~$F^*$ удовле\-тво\-ря\-ет условиям тео\-ре\-мы~1, так что
${\bf P}(Z^*_n\hm\to\hm+\infty)\hm=0$, а~следовательно, и~${\bf P}(Z_n\hm\to+\infty)\hm=0$.
Таким образом, происходит вы\-рож\-де\-ние п.~н.\hfill$\square$

\smallskip

До сих пор предполагалось, что~$\delta$ конечно. Однако предельное распределение
может существовать в~некоторых случаях и~при $\delta\hm=+\infty$, что соответствует
сверх\-тя\-же\-лым хвос\-там~$G$.

\smallskip

\noindent
\textbf{Пример~5.}\ Пусть $F(x)\hm=\exp\{-x^{-\beta}\}$, $x\hm>0$, $\beta\hm>1$, 
и~$G(x)\hm=1-\hm1/\ln x$, $x\hm\ge e$,
тогда по тауберовой тео\-ре\-ме $1\hm-\varphi(u)\hm\sim -1/\ln u$, $u\hm\to 0$, откуда
\begin{equation}
\label{etaphi}
{\bf P}\left(\eta_1>x\right)=1-\varphi\left(e^{-x}\right)\sim \fr{1}{x},\enskip 
x\to+\infty\,.
\end{equation}
Вместе с~тем, поскольку $\nu\hm\ge e$ п.~н., то $\varphi(u)\hm\le e^{-eu}$, 
$u\hm\ge 0$, откуда
\begin{equation}
\label{etaphi2}
{\bf P}\left(\eta_1<-x\right)=\varphi\left(e^x\right)\le \exp\{-e^{x+1}\},\enskip x>0\,.
\end{equation}

Из~(\ref{zb}) следует, что существование предельного распределения эквивалентно 
схо\-ди\-мости
сле\-ду\-юще\-го случайного ряда п.~н.:
\begin{equation}
\label{rzb}
{\tilde\zeta}\stackrel{d}{=}\sum\limits_{n=0}^\infty\beta^{-n}\eta_n\,,
\end{equation}
где ${\tilde\zeta}\hm=\Lambda^{-1}(F({\tilde Z}))$.
Для любого $1/\beta\hm<\varepsilon\hm<1$ в~силу~(\ref{etaphi}) и~(\ref{etaphi2}) 
получаем:
$$
\sum\limits_{n=0}^\infty {\bf P}\left(\left\vert \eta_n\right\vert 
>(\varepsilon\beta)^n\right)<\infty\,,
$$
откуда по лемме Бо\-ре\-ля--Кан\-тел\-ли следует, что события
$A_n\hm=\{\beta^{-n}|\eta_n|\hm>\varepsilon^n\}$ происходят не 
более чем конечное чис\-ло раз;
следовательно, ряд~(\ref{rzb}) сходится п.~н.

\section{Заключение}

В работе введены МВПСС (с~одним
типом час\-тиц). Разобран случай <<степенной>> случайной среды, для него изучен ряд свойств,
доказана эргодическая тео\-ре\-ма, приведены примеры. Отмечена связь 
МВП с~бесконечнолинейными сис\-те\-ма\-ми массового обслуживания.

Дальнейшие исследования могут быть связаны с~изуче\-ни\-ем более широкого класса
случайных сред и~процессов с~иммиграцией.

{\small\frenchspacing
 {%\baselineskip=10.8pt
 \addcontentsline{toc}{section}{References}
 \begin{thebibliography}{99}
\bibitem{Har}
\Au{Харрис T.} Теория вет\-вя\-щих\-ся случайных процессов~/ Пер. с~англ.~--- 
М.: Мир, 1966. 356~c.
(\Au{Harris~T.} {The theory of branching processes.}~---
Berlin: Springer-Verlag, 1963. 230~p.)
\bibitem{Lamp1}
\Au{Lamperti J.} Maximal branching processes and long-range
per\-co\-la\-ti\-on~// J.~Appl. Probab., 1970. Vol.~7. No.\,1. P.~89--96.
\bibitem{Lamp2}
\Au{Lamperti J.} Remarks on maximal branching processes~// Теория
вероятностей и~ее применения, 1972. T.~17. №\,1. C.~46--54.
\bibitem{VatZub2}
\Au{Vatutin V.\,A., Zubkov A.\,M.} Branching processes. II~// J.~Sov. Math., 1993. 
Vol.~67. No.\,6. P.~3407--3485.
\bibitem{Leb-2005d}
\Au{Лебедев А.\,В.} Максимальные вет\-вя\-щи\-еся процессы 
с~неотрицательными значениями~// Теория вероятностей и~ее применения,
2005. T. 50. №\,3. C.~564--570.
\bibitem{Leb-2009a}
\Au{Лебедев А.\,В.} Максимальные вет\-вя\-щи\-еся процессы~//
Современные проб\-ле\-мы математики и~механики, 2009. T.~4. №\,1. C.~93--106.
\bibitem{Leb-2012a}
\Au{Лебедев А.\,В.}
Максимальные вет\-вя\-щи\-еся процессы с~несколькими типами час\-тиц~// 
Вестн. Моск. ун-та. 
Сер.~1:
Математика. Механика, 2012. №\,3. C.~8--13.
\bibitem{LebDiss}
\Au{Лебедев А.\,В.}  Неклассические задачи сто\-ха\-сти\-че\-ской тео\-рии 
экстремумов:
Дис.\ \ldots\ докт. физ.-мат. наук.~--- М.: МГУ, 2016.
\bibitem{CMBP}
\Au{Aydogmus O., Ghosh~A.\,P., Ghosh~S., Roitershtein~A.} 
Coloured maximal branching process~//
Теория вероятностей и~ее применения, 2014. Т.~59. №\,4. С. 790--800.
\bibitem{Brown1}
\Au{Browne S., Coffman~E.\,G., Gilbert~E.\,N., Wright~P.\,E.} Gated, exhaustive,
parallel service~// Probab. Eng. Inform. Sc., 1992. Vol.~2. No.\,2. P.~217--239.
\bibitem{Brown2}
\Au{Browne S., Coffman~E.\,G., Gilbert~E.\,N., Wright~P.\,E.} The gated
infinite-server queue: Uniform service times~// SIAM J.~Appl. Math.,
1992. Vol.~52. No.\,6. P.~1751--1762.
\bibitem{Knessl}
\Au{Tan X., Knessl Ch.} Heavy traffic asymptotics for a~gated,
infinite-server queue with uniform service times~// SIAM J.~Appl. Math.,
1994. Vol.~54. No.\,6. P.~1768--1779.
\bibitem{Zaz}
\Au{Pinotsi D., Zazanis M.\,A.} Stability conditions for gated
$M|G|\infty$ queues~// Probab. Eng. Inform. Sc., 2004. Vol.~18. No.\,1. P.~103--110.
\bibitem{Leb-2003}
\Au{Лебедев А.\,В.} Вентильная бесконечнолинейная сис\-те\-ма с~неограниченными
временами обслу\-жи\-ва\-ния и~большой за\-груз\-кой~// Проб\-ле\-мы передачи информации,
2003. T.~39. №\,3. C.~87--94.
\bibitem{Leb-2004}
\Au{Лебедев А.\,В.} Вентильная бесконечнолинейная сис\-те\-ма
с~большой за\-груз\-кой и~степенным хвостом~// Проб\-ле\-мы передачи
информации, 2004. T.~40. №\,3. C.~62--68.
\bibitem{Obl}
\Au{Горбунова А.\,В., Зарядов~И.\,С., Матюшенков~С.\,И., Самуйлов~К.\,Е., Шоргин~С.\,Я.}
Аппроксимация времени отклика сис\-те\-мы облач\-ных вы\-чис\-ле\-ний //
Информатика и~её применения, 2015. Т.~9. Вып.~3. С.~32--38.
\bibitem{Jr2}
\Au{\mbox{Ji{\!\!\ptb{\v{r}}}ina M.}} Stochastic branching processes with continuous
state space~// Chech. Math.~J., 1958. Vol.~8. No.\,2. P.~292--313.
\bibitem{Borov}
\Au{Боровков А.\,А.} Эргодичность и~устой\-чи\-вость случайных процессов.~---
М.: УРСС, 1999. 440~c.
%\bibitem{Baj}
%{\it Кудрявцев А.~А., Шоргин С.~Я.} Байесовский подход к~анализу систем
%массового обслуживания и~показателей надежности //
%Информ. и~ее примен., 2007. Т. 1. \No~2. С. 76--82.
\bibitem{EPW}
\Au{Esary J., Prochan~F., Walkup~D.} Association of random variables
with applications~// Ann. Math. Stat., 1967. Vol.~38. No.\,5. P.~1466--1474.
\bibitem{Bul}
\Au{Булинский А.\,В., Шашкин~А.\,П.} Предельные тео\-ре\-мы для ассоциированных
случайных полей и~родственных сис\-тем.~--- М.: Физматлит, 2008. 480~c.
%\bibitem{Shto}
%{\it Штойян Д.} Качественные свойства и~оценки стохастических моделей. ---
%М.: Мир, 1979. 272 c.
\bibitem{Bhat}
\Au{Bhattacharya R.\,N., Lee~C.} Ergodicity of nonlinear first order
autoregressive models~// J.~Theor. Probab., 1995. Vol.~8. No.\,1. P.~207--219.
 \end{thebibliography}

 }
 }

\end{multicols}

\vspace*{-6pt}

\hfill{\small\textit{Поступила в~редакцию 07.08.17}}

\vspace*{6pt}

%\newpage

%\vspace*{-24pt}

\hrule

\vspace*{2pt}

\hrule

%\vspace*{8pt}


\def\tit{MAXIMAL BRANCHING PROCESSES IN~RANDOM ENVIRONMENT}


\def\titkol{Maximal branching processes in~random environment}

\def\aut{A.\,V.~Lebedev}

\def\autkol{A.\,V.~Lebedev}

\titel{\tit}{\aut}{\autkol}{\titkol}

\vspace*{-9pt}


\noindent
 Department of Probability Theory, 
Faculty of Mechanics and Mathematics, 
M.\,V.~Lomonosov Moscow State University, Main Building, 1~Leninskiye Gory, 
Moscow 119991, Russian Federation



\def\leftfootline{\small{\textbf{\thepage}
\hfill INFORMATIKA I EE PRIMENENIYA~--- INFORMATICS AND
APPLICATIONS\ \ \ 2018\ \ \ volume~12\ \ \ issue\ 2}
}%
 \def\rightfootline{\small{INFORMATIKA I EE PRIMENENIYA~---
INFORMATICS AND APPLICATIONS\ \ \ 2018\ \ \ volume~12\ \ \ issue\ 2
\hfill \textbf{\thepage}}}

\vspace*{3pt}



\Abste{The work continues the author's long research in the theory of maximal branching processes
that are obtained from classical branching processes by replacing the sum of 
offsping numbers by the maximum.
One can say that the next generation is formed by the offspring of the most 
productive particle.
Earlier, the author generalized processes with integer values up to processes 
with arbitrary 
nonnegative values,
investigated their properties, and proved the limit theorems. Further, maximal 
branching processes with several
types of particles were introduced and studied.
In this paper, the author introduces the concept of maximal branching processes 
in random environment
(with one type of particles) and an important case of the ``power'' random 
environment.
In the latter case, the basic properties of maximal branching processes are 
studied and the ergodic
theorem is proved. As an application, the author considers gated 
infinite-server queues.}

\KWE{maximal branching processes; random environment;
ergodic theorem; stable distributions; extreme value theory}

\DOI{10.14357/19922264180206} %

%\vspace*{-14pt}

%  \Ack
 %  \noindent




%\vspace*{-3pt}

  \begin{multicols}{2}

\renewcommand{\bibname}{\protect\rmfamily References}
%\renewcommand{\bibname}{\large\protect\rm References}

{\small\frenchspacing
 {%\baselineskip=10.8pt
 \addcontentsline{toc}{section}{References}
 \begin{thebibliography}{99}
\bibitem{Har-1}
\Aue{Harris, T.} 1963. \textit{The theory of branching processes.} 
Berlin: Springer-Verlag. 230~p.
\bibitem{Lamp1-1}
\Aue{Lamperti, J.} 1970. Maximal branching processes and long-range
per\-co\-la\-ti\-on. \textit{J.~Appl. Probab.} 7(1): 89--96.
\bibitem{Lamp2-1}
\Aue{Lamperti, J.} 1972. Remarks on maximal branching processes.
\textit{Theor. Probab. Appl.} 17(1): 44--53.
\bibitem{VatZub2-1}
\Aue{Vatutin, V.\,A., and A.\,M.~Zubkov.} 1993. Branching processes.~II. 
\textit{J.~Sov. Math.} 67(6):3407--3485.
\bibitem{Leb-2005d-1}
Lebedev, A.~V. 2006. Maximal branching processes with nonnegative values.
{\it Theor. Probab. Appl.} 50(3):482--488.
\bibitem{Leb-2009a-1}
\Aue{Lebedev, A.\,V.} 2009. Maksimal'nye vetvyashchiesya protsessy 
[Maximal branching processes].
\textit{Sovremennye problemy matematiki i~mekhaniki} 
[Modern Problems of Mathematics and Mechanics]
4(1):93--106.
\bibitem{Leb-2012a-1}
\Aue{Lebedev, A.\,V.} 2012. Maximal branching processes with 
several types of particles.
\textit{Mosc. Univ. Math. Bull.} 67(3):97--101.
\bibitem{LebDiss-1}
\Aue{Lebedev, A.\,V.} 2016. {Neklassicheskie zadachi sto\-kha\-sti\-che\-skoy 
teorii ekstremumov}
[Nonclassical problems of extreme value theory].  Moscow. DSc Diss. 257~p.
\bibitem{CMBP-1}
\Aue{Aydogmus, O., A.\,P.~Ghosh, S.~Ghosh, and A.~Roitershtein.} 
2015. Colored maximal branching process.
\textit{Theor. Probab. Appl.} 59(4): 663--672.
\bibitem{Brown1-1}
\Aue{Browne, S., E.\,G.~Coffman, E.\,N.~Gilbert, and P.\,E.~Wright.} 1992. Gated, exhaustive,
parallel service. \textit{Probab. Eng. Inform. Sc.} 2(2): 217--239.
\bibitem{Brown2-1}
\Aue{Browne, S., E.\,G.~Coffman, E.\,N.~Gilbert, and P.\,E.~Wright.} 1992. The gated
infinite-server queue: Uniform service times. \textit{SIAM J.~Appl. Math.}
52(6):1751--1762.
\bibitem{Knessl-1}
\Aue{Tan, X., and Ch.~Knessl.} 1994. Heavy traffic asymptotics for a gated,
infinite-server queue with uniform service times. \textit{SIAM J.~Appl. Math.}
54(6):1768--1779.
\bibitem{Zaz-1}
\Aue{Pinotsi, D., and M.\,A.~Zazanis.} 2004. Stability conditions for gated
$M|G|\infty$ queues. \textit{Probab. Eng. Inform. Sc.} 18(1):103--110.
\bibitem{Leb-2003-1}
\Aue{Lebedev, A.\,V.} 2003. The gated infinite-server queue 
with unbounded service times and
heavy traffic. \textit{Probl. Inf. Transm.} 39(3):309--316.
\bibitem{Leb-2004-1}
\Aue{Lebedev, A.\,V.} 2004. Gated infinite-server queue with heavy traffic and power tail.
\textit{Probl. Inf. Transm.} 40(3):237--242.
\bibitem{Obl-1}
\Aue{Gorbunova, A.\,V., I.\,S.~Zaryadov, S.\,I.~Matyushenko, K.\,E.~Samouylov, 
and S.\,Ya.~Shorgin.} 2015.
Approksimatsiya vremeni otklika sistemy oblachnykh vychisleniy
[The approximation of responce time of a cloud computing system]. \textit{Informatika i~ee
Primeneniya~--- Inform. Appl.} 9(4):32--38.
\bibitem{Jr2-1}
\Aue{\mbox{Ji{\!\ptb{\v{r}}}ina}, M.} 1958. Stochastic branching processes with continuous
state space. \textit{Chech. Math.~J.} 8(2):292--313.
\bibitem{Borov-1}
\Aue{Borovkov, A.\,A.} 1998. \textit{Ergodicity and stability of stochastic processes}. Wiley. 618~p.
\bibitem{EPW-1}
\Aue{Esary, J., F.~Prochan, and D.~Walkup.} 1967. Association of random variables
with applications. \textit{Ann. Math. Stat.} 38(5):1466--1474.
\bibitem{Bul-1}
\Aue{Bulinski, A.\,V., and A.\,P.~Shashkin.} 2007. \textit{Limit theorems for associated random fields
and related systems.} World Scientific Publishing. 448~p.
\bibitem{Bhat-1}
\Aue{Bhattacharya, R.\,N., and C.~Lee.} 1995. Ergodicity of nonlinear first order
autoregressive models. \textit{J.~Theor. Probab.} 8(1):207--219.
\end{thebibliography}

 }
 }

\end{multicols}

\vspace*{-3pt}

\hfill{\small\textit{Received August 7, 2017}}

%\vspace*{-24pt}

\Contrl

\noindent
\textbf{Lebedev Alexey V.} (b.\ 1971)~--- Doctor of Science in physics and mathematics,
associate professor, Department of Probability Theory, Faculty of Mechanics 
and Mathematics,
M.\,V.~Lomonosov Moscow State University, Main Building, 1~Leninskiye Gory,
Moscow 119991, Russian Federation; \mbox{avlebed@yandex.ru}
\label{end\stat}


\renewcommand{\bibname}{\protect\rm Литература} 