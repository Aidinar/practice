\def\stat{maniakov}

\def\tit{МАТЕМАТИЧЕСКАЯ МОДЕЛЬ ОПТИМАЛЬНОЙ ТРИАНГУЛЯЦИИ}

\def\titkol{Математическая модель оптимальной триангуляции}

\def\aut{А.\,А.~Батенков$^1$, Ю.\,А.~Маньяков$^2$, А.\,В.~Гасилов$^3$, 
О.\,А.~Яковлев$^4$}

\def\autkol{А.\,А.~Батенков, Ю.\,А.~Маньяков, А.\,В.~Гасилов, 
О.\,А.~Яковлев}

\titel{\tit}{\aut}{\autkol}{\titkol}

\index{Батенков А.\,А.}
\index{Маньяков Ю.\,А.}
\index{Гасилов А.\,В.} 
\index{Яковлев О.\,А.}
\index{Batenkov A.}
\index{Maniakov Yu.}
\index{Gasilov A.}
\index{Yakovlev O.}




%{\renewcommand{\thefootnote}{\fnsymbol{footnote}} \footnotetext[1]
%{Работа поддержана РНФ (проект 16-11-10227).}}


\renewcommand{\thefootnote}{\arabic{footnote}}
\footnotetext[1]{Орловский филиал Федерального исследовательского центра <<Информатика и~управление>> 
Российской  академии наук, \mbox{batenkov1957@mail.ru}}
\footnotetext[2]{Орловский филиал Федерального исследовательского центра <<Информатика и~управление>> 
Российской академии наук, \mbox{maniakov\_yuri@mail.ru}}
\footnotetext[3]{Орловский филиал Федерального исследовательского центра <<Информатика и~управление>> 
Российской академии наук, \mbox{gasilov.av@yandex.ru}}
\footnotetext[4]{Орловский филиал Федерального исследовательского центра 
<<Информатика и~управление>> Российской академии наук, \mbox{maucra@gmail.com}}

\vspace*{-6pt}
  
  \Abst{Формализована постановка задачи синтеза оптимальной плос\-кой выпуклой 
триангуляции. Данная задача относится к~вопросам приложений информатики и~весьма 
актуальна в~таких областях, как компьютерная графика и~геоинформационные сис\-те\-мы. 
Математическая модель пред\-став\-ле\-на в~трех вариантах: экстремальная задача 
с~бесконечным чис\-лом ограничений, минимаксная задача со связанными переменными 
и~экстремальная задача с~дополнительным ограничением на пересечения отрезков 
триангуляции, но конечным чис\-лом всех ограничений. Путем введения идемпотентных 
ограничений на булевы переменные исходная целочисленная задача по\-гру\-же\-на в~общую 
задачу математического программирования на континуальном мно\-же\-ст\-ве решений. 
Произведен сравнительный анализ решений, по\-лу\-ча\-емых жадным алгоритмом на основе 
пред\-став\-лен\-ной математической модели и~ал\-го\-рит\-мом триангуляции Делоне.}
  
  \KW{математическая модель; триангуляция; триангуляция Делоне}
  
  \DOI{10.14357/19922264180210}
  
%\vspace*{-6pt}


\vskip 10pt plus 9pt minus 6pt

\thispagestyle{headings}

\begin{multicols}{2}

\label{st\stat}

\section{Введение}

  Задача построения триангуляции весьма востребована в~таких разделах 
информатики, как компьютерная графика, рас\-по\-зна\-ва\-ние образов 
и~геоинформационные сис\-те\-мы, для моделирования поверхно\-стей и~решения 
про\-стран\-ст\-вен\-ных задач, задач навигации автономных роботизированных 
сис\-тем. Кроме того, триангуляция может быть использована в~алгоритмах 
трехмерной реконструкции~[1].
  
  Отсутствие формализованной постановки задачи синтеза триангуляции 
в~виде математической модели привело к~по\-яв\-ле\-нию множества эвристических 
при\-бли\-жен\-ных алгоритмов~[2, 3] по различным критериям. Этому также 
способствовало подозрение на NP-пол\-но\-ту задачи поиска оптимальной 
триангуляции.
  
  Наиболее применима в~настоящее время триангуляция, предложенная 
советским матема\-ти\-ком Б.\,Н.~Делоне, которая имеет вы\-чис\-ли\-тель\-ную 
\mbox{слож\-ность} порядка $O(N\log N)$~\cite{3-man, 4-man} ($N$~--- число точек 
триангуляции), но слабый критерий оптимизации~[5]. Прежде всего это 
обусловлено ориентированностью триангуляции Делоне на алгоритмический 
аспект ре\-ша\-емой задачи, а~никак не на поиск оптимальной триангуляции, под 
которой понимается триангуляция с~минимальной суммой длин ребер среди 
всех воз\-мож\-ных триангуляций, по\-стро\-ен\-ных на тех же исходных точ\-ках~[6]. 
Также необходимо отметить, что алгоритмы триангуляции Делоне лишь 
улучшают некоторую заданную первичную триангуляцию, поиск которой 
требует дополнительных вы\-чис\-ли\-тель\-ных затрат.

\begin{figure*}[b] %fig1
 \vspace*{1pt}
 \begin{center}
 \mbox{%
 \epsfxsize=162.307mm 
 \epsfbox{man-1.eps}
 }
 \end{center}
\vspace*{-9pt}
      \Caption{Пересекающиеся~(\textit{а}) и~непересекающиеся~(\textit{б})  отрезки}
\end{figure*}

  \vspace*{-6pt}
  
  
\section{Идентификация пересечения отрезков на~множестве 
прямых}

  Так как под триангуляцией понимается пла\-нарный граф, все внут\-рен\-ние 
области которого являются треугольниками, то центральное мес\-то 
в~математиче\-ской модели долж\-на занимать идентификация пересечений 
отрезков прямых, входящих в~триангуляцию. Известные алгоритмы 
обнаружения пересечения отрезков~[7] используют их па\-ра\-мет\-ри\-че\-ское 
задание, что затрудняет использование этих алгоритмов при формулировке 
экстремальных задач, поэтому необходим функционал, желательно с~булевым 
множеством об\-ласти значений~[8].
  
  С этой целью используем оконную функцию в~виде:
  
\noindent
  \begin{multline*}
  \mathrm{Ok} (z, \mathrm{sn}, \mathrm{sv}) =
  \left( \Phi\left(z-\mathrm{sn}-\mathrm{sign}\,(\mathrm{sv}-
  \mathrm{sn}) \mathrm{zap}\right) - {}\right.\hspace*{-4.52142pt}\\
\left.  {}-
\Phi(z-\mathrm{sv}+\mathrm{sign}\,(\mathrm{sv}-
\mathrm{sn}) \mathrm{zap})\right) \mathrm{sign}\,(\mathrm{sv}-\mathrm{sn})\,,
  %\label{e1-man}
  \end{multline*}
где $\mathrm{sn}$ и~$\mathrm{sv}$~--- верх\-няя и~ниж\-няя границы окна; $\Phi(x)$~--- функция 
Хевисайда~[9]; $\mathrm{sign}\,(x)$~--- функция знака~\cite{10-man}; $\mathrm{zap}$~--- 
вспомогательная константа.
  
  Сформируем переопределенную сис\-те\-му линейных урав\-не\-ний с~двумя 
неизвестными:
  \begin{equation}
    a_k x +b_k y=c_k\,,\quad
k=\overline{1,D}\,,
  \label{e2-man}
\end{equation}
где
\begin{gather*}
  a_k=\fr{1}{x_j-x_i}\,;\quad b_k=-\fr{1}{y_j-y_i}\,;\\
   c_k=\fr{x_i}{x_j-x_i}-  \fr{y_i}{y_j-y_i}\,;\quad i,j=\overline{1,N}\,;\\
  A Z =B\,;
    \end{gather*}
$D$~--- чис\-ло прямых, соединяющих каждую пару точек триангуляции:
$D\hm={N(N\hm-1)}/{2}$;
$A$~--- матрица коэффициентов размерности $D\times 2$;
$Z^{\mathrm{T}}\hm=(x,y)$~--- вектор неизвестных раз\-мер\-ности $2\times 1$;
$B$~--- вектор свободных членов раз\-мер\-ности $D\times 1$.

  Введем вектор неизвестных $Y$ размерности $D\times 1$, опре\-де\-ля\-ющий 
активные урав\-не\-ния сис\-те\-мы~(\ref{e2-man}), если $Y\not= 0$, и~свяжем их 
с~(\ref{e2-man}):
  \begin{equation}
  \mathrm{diag}\,(Y) AZ=\mathrm{diag}\,(Y)B\,,
  \label{e3-man}
  \end{equation}
где $\mathrm{diag}\,(Y)$~--- диагональная мат\-ри\-ца с~вектором~$\vec{Y}$ на 
главной диа\-го\-нали.

  Решим систему~(\ref{e3-man}) методом наименьших квад\-ра\-тов:
  \begin{equation*}
  Z=\left( M^{\mathrm{T}} M\right)^{-1} M^{\mathrm{T}} C\,,
  %\label{e4-man}
  \end{equation*}
где $M=\mathrm{diag}\,(Y) A$; $C\hm= \mathrm{diag}\,(Y) B$.


      
  Анализ рис.~1 приводит к~выводу, что точка пересечения отрезков 
существует, когда рав\-ны единице оконные функ\-ции:
  \begin{align*}
 & (x_i, y_i), (x_j, y_j); \ (x_k, y_k), (x_l, y_l)\,;\ \ i,j,k,l\in \{\overline{1,N}\}\,;\\
 & \mathrm{OK}1(x^*, y^*)_R =
 \mathrm{Ok}\left(x^*, x_i, x_j\right)\cdot \mathrm{Ok} \left(y^*, y_i, 
y_j\right)\,;\\
 &\hspace*{60mm}R\in\{ \overline{1,D}\}\,;
 \end{align*}
 
 \noindent
 \begin{multline*}
  \mathrm{OK}2\left(x^*, y^*\right)_S= \mathrm{Ok}\left(x^*, x_k, x_l\right)\cdot 
 \mathrm{Ok} (y^*, y_k, y_l)\,;\\
\ S\in \{\overline{1,D}\}\,.
  \end{multline*}
  
  В результате получаем мат\-ри\-цу идентификации пересечения пар отрезков 
в~за\-ви\-си\-мости от вектора~$\vec{Y}$ при известных координатах точек 
триангуляции на плос\-кости:

\noindent
  \begin{multline}
  \mathrm{Cp}(Y)={}\\
 \!\!\hspace*{-2mm} {}=\left\{ \begin{pmatrix}
  (\mathrm{OK}1((M^{\mathrm{T}}M)^{-1}M^{\mathrm{T}}C)\bullet Y)\\
  (\mathrm{OK}2((M^{\mathrm{T}}M)^{-1} M^{\mathrm{T}} C)\bullet Y)^{\mathrm{T}}
  \end{pmatrix}\bullet \mathrm{EDW}\right\}\,,\!\!
  \label{e5-man}
  \end{multline}
где $\mathrm{EDW}$~---  верхняя треугольная мат\-ри\-ца раз\-мер\-ности $D\times D$: 

\noindent
$$
\mathrm{EDW}=\begin{bmatrix}
0&1&1&\cdots& 1\\
0&0&1&\ddots &1\\
0&0&0&\ddots &1\\
\vdots& \ddots& \ddots& \ddots&\vdots\\
0&0&0&\cdots &0
\end{bmatrix}\,;
$$
($\bullet$)~--- произведение Адамара~\cite{11-man}.

%\vspace*{-3pt}

\section{Постановка задачи оптимальной триангуляции}

  Использование функционала пересечения отрезков на основе~(\ref{e5-man}) 
по\-рож\-да\-ет множество по\-ста\-но\-вок задач оптимальной триангуляции.
  
  Здесь приведем перспективную математическую модель оптимальной 
триангуляции с~дополнительным ограничением на пересечения отрезков 
триангуляции:

\vspace*{2pt}

\noindent
  \begin{equation}
  W^{\mathrm{T}} X\Rightarrow \min\limits_{X,Y}
  \label{e6-man}
  \end{equation}
  
  \pagebreak
  
  \noindent
при ограничениях

\noindent
\begin{gather}
\mathrm{Ed}^{\mathrm{T}}X =3(N-1)-N_c\,;\label{e7-man}\\
X^{\mathrm{T}} P X=0\,;\label{e8-man}\\
X_i=0\ \mbox{или } 1\,, \label{e9-man}
\end{gather}
где $W^{\mathrm{T}}\hm= (w_1, w_2, \cdots, w_D)$~--- вектор длин отрезков 
между парами исходных точек триангуляции; $\mathrm{Ed}_R\hm=1$, $R\hm= 
\overline{1,D}$~--- единичный вектор; $N$~--- чис\-ло точек триангуляции;  
$N_c$~--- чис\-ло точек выпуклой оболочки.
  
  Целевая функция~(\ref{e6-man}) обеспечивает минимальную суммарную 
длину ребер триангуляции. Ограничение~(\ref{e7-man}) сформировано на 
основе леммы о~чис\-ле ребер триангуляции~\cite{6-man}.  
Условие~(\ref{e9-man}) определяет булевый характер искомых 
переменных~$X$. В~такой по\-ста\-нов\-ке введено дополнительное 
ограничение~(\ref{e8-man}) на пересечения ребер триангуляции с~по\-мощью 
предварительно вы\-чис\-лен\-ной мат\-ри\-цы~$[{P}]$, формирующей 
множество пар пересекающихся отрезков для исходного множества точек 
триангуляции. Мат\-ри\-ца~$[{P}]$ является верхней треугольной булевой 
мат\-ри\-цей раз\-мер\-ности $D\times D$ вида:

\vspace*{18pt}

\noindent
  $$
  P=\begin{matrix}
  1\\ \vdots \\ i\\ \vdots\\ j\\ \vdots \\ k \\\vdots \\ D
  \end{matrix}
  \left[\,
  \begin{matrix}
    \\[-24pt]
  1 & \cdots & l&\cdots &  m& \cdots & n&\cdots& D\\
  0&\cdots& 0& \cdots & 0 &\cdots & 0 &\cdots & 0/1\\
  \vdots &\vdots&\vdots&\vdots&\vdots&\vdots&\vdots&\vdots&\vdots\\
  0&\cdots&1 & \cdots & 0&\cdots& 0&\cdots& 0/1\\
  \vdots &\vdots&\vdots&\vdots&\vdots&\vdots&\vdots&\vdots&\vdots\\
  0&\cdots&0 & \cdots & 1&\cdots& 0&\cdots& 0/1\\
  \vdots &\vdots&\vdots&\vdots&\vdots&\vdots&\vdots&\vdots&\vdots\\
  0&\cdots&0 & \cdots & 0&\cdots& 1&\cdots& 0/1\\
  \vdots &\vdots&\vdots&\vdots&\vdots&\vdots&\vdots&\vdots&\vdots\\
  0&\cdots&0 & \cdots & 0&\cdots& 0&\cdots& 1/0
  \end{matrix}
  \,\right]\,.
  $$
  
  
  
  Такая матрица идентифицирует пересечение\linebreak ($j$-го, $m$-го) и~($k$-го,  
$n$-го) отрезков.
  
  Для формирования мат\-ри\-цы $[{P}]$ применим~(\ref{e5-man}). Тогда 
$RS$-й элемент определится как
  $$ 
  P_{RS}=\mathrm{Cp}\left( Y^{RS}\right)\,,
  $$
где $Y^{RS}$~--- вектор с~компонентами $Y_R^{RS}\hm= 1$, $Y_S^{RS}\hm= 
1$, $Y_i^{RS}\hm= 0$, $i\hm=\overline{1,D}$, $i\not= R, S$.
  
  Решение задачи~(\ref{e6-man})--(\ref{e9-man}) было найдено жад\-ным 
алгоритмом.
  
  Условия булевости~(\ref{e9-man}) приводят к~NP-пол\-но\-те ре\-ша\-емой 
задачи. Однако его мож\-но преобразовать к~континуальному множеству 
решений задачи~(\ref{e6-man})--(\ref{e9-man}) с~по\-мощью ограничений на 
идемпотентность переменных~\cite{12-man}:
  \begin{equation}
  X_i^2-X_i=0\,,\enskip i=\overline{1,D}\,.
  \label{e10-man}
  \end{equation}
  
  Так, в~последующих работах планируется погрузить дискретную 
математическую модель~(\ref{e6-man})--(\ref{e9-man}) в~непрерывную путем 
замены ограничений~(\ref{e9-man}) на~(\ref{e10-man}). Это позволит 
применить градиентные методы поиска~\cite{13-man} оптимальной 
триангуляции, что должно существенно снизить вы\-чис\-ли\-тель\-ные затраты.

\vspace*{-6pt}

\section{Апробирование математических моделей}

  Произведем сравнительный анализ решений, по\-лу\-ча\-емых при использовании 
разработанной математической модели и~алгоритма триангуляции Делоне 
(рис.~2).


  
  Апробирование математических моделей и~их сравнение проводилось при 
сле\-ду\-ющих исходных данных. Известны координаты~50~точек двумерных 
проекций на плоскости трехмерного изображения по\-верх\-ности в~пространстве.
  
  Используя представленные координаты, были найдены одиннадцать точек 
выпуклой оболочки.
  
  Для двух триангуляций были найдены по~136~ребер для каж\-дой
  из них. Общая длина для жад\-ной триангуляции на основе разработанной 
модели составила 9\,667,487, а~для триангуляции Делоне~--- 10\,096,338. При 
этом графы триангуляций различаются~13~уникальными ребрами.
  
  Для сравнения пар по\-лу\-ча\-емых решений применялась нормированная 
величина:
  \begin{equation}
  L(x,y) =\left\vert  \fr{x-y}{\max(x,y)}\right\vert \cdot 100\%\,,\enskip x>0\,,\ 
y>0\,,
  \label{e11-man}
  \end{equation}
  
  \vspace*{9pt}
  
   { \begin{center}  %fig2
 \vspace*{1pt}
  \mbox{%
 \epsfxsize=76.837mm 
 \epsfbox{man-3.eps}
 }

\end{center}


\noindent
{{\figurename~2}\ \ \small{Сравнение триангуляций как множеств ребер:
толстые линии~--- общие ребра; тонкая сплошная линия~--- реб\-ра, принадлежащие только триангуляции 
Делоне; тон\-кая пунктирная линия~--- реб\-ра, присутствующие только в~жад\-ной 
триангуляции}}
}

%\vspace*{9pt}

\noindent
где $x, y$~--- суммарные длины ребер для срав\-ни\-ва\-емых триангуляций.

  Выигрыш по суммарной длине ребер со\-ста\-вил~4,436\%.
  
  Далее отметим использование найденных триангуляций для решения задач 
навигации и~задач трехмерной реконструкции. Здесь может быть использована 
задача коммивояжера с~целью формирования оптимального маршрута обхода 
заданных опорных точек триангуляции. Это позволит существенно снизить 
размерность ре\-ша\-емой NP-пол\-ной задачи коммивояжера. Так, для~50~точек 
раз\-мер\-ность без триангуляции со\-ста\-вит~1225~ребер, 
а~с~триангуляцией~--- 136~ребер.
  
  При исследовании математических моделей были найдены марш\-ру\-ты 
мажоритарными алгоритмами на основе сравниваемых триангуляций. Здесь 
выигрыш разработанной модели при использовании па\-ра\-мет\-ра~(\ref{e11-man}) 
со\-ста\-вил~14,79\%.
  

\vspace*{-6pt}
   
  

\section{Выводы}

\noindent
\begin{enumerate}[1.]
\item Центральное место в~работе занимает поиск функционала пересечения 
отрезков с~булевым множеством значений.
\item Математическая модель оптимальной триангуляции пред\-став\-ле\-на в~виде 
по\-ста\-нов\-ки экстремальной задачи с~учетом ограничений на пересечение 
отрезков.
\item Математическая модель может позволить преодолеть пред\-по\-ла\-га\-емую 
NP-пол\-но\-ту задачи оптимальной триангуляции за счет применения 
градиентных методов непрерывной оптимизации.
\item Разработанные варианты математической модели оптимальной 
триангуляции позволяют обосно\-вать применение эвристических приближенных 
алгоритмов триангуляции, а~так\-же использование большего чис\-ла 
расширенных классов алгоритмов оптимизации.
\item Даже в~случае жадного алгоритма разработанная математическая модель 
приводит к~меньшему значению целевого функционала на~4,436\% по 
сравнению с~алгоритмом триангуляции Делоне.
\item Использование решения задачи триангуляции в~качестве исходных 
данных в~задаче коммивояжера позволяет снизить ее размерность с~$N(N-1)/2$ 
до~$3(N\hm-1)\hm- N_c$.
  \item Применение в~задаче навигации жадного алгоритма коммивояжера 
к~разработанной модели триангуляции поз\-во\-ли\-ло получить выигрыш 
в~14,79\% по сравнению с~применением его к~триангуляции Делоне.
  \end{enumerate}
  
  Следующие два вывода предопределяют на\-прав\-ле\-ния дальнейших 
при\-клад\-ных исследований предлагаемой математической модели.
  \begin{enumerate}[1.]
  \setcounter{enumi}{7}
  \item Так как реб\-ра двумерной триангуляции пред\-став\-ля\-ют собой отрезки 
прямых, то и~при восстановлении их в~трехмерном пространстве они также 
являются отрезками прямых. Это свойство может быть использовано для 
трехмерной реконструкции концов выделенных отрезков триангуляции.
  \item При трехмерной реконструкции для выделения отрезков использование 
решения задачи коммивояжера в~условиях разработанной модели триангуляции 
может обеспечить меньшие искажения по срав\-не\-нию с~условиями 
триангуляции Делоне.
  \end{enumerate}
  
{\small\frenchspacing
 {%\baselineskip=10.8pt
 \addcontentsline{toc}{section}{References}
 \begin{thebibliography}{99}
\bibitem{1-man}
\Au{Гасилов А.\,В., Яковлев~О.\,А.} Создание реалистичных наборов данных для алгоритмов 
трехмерной реконструкции с~по\-мощью виртуальной съем\-ки компьютерной модели~// 
Системы и~средства информатики, 2016. Т.~26. №\,2. С.~98--107.

\bibitem{3-man} %2
\Au{Костюк Ю.\,Л., Фукс~А.\,Л.} При\-бли\-жен\-ное вы\-чис\-ле\-ние оптимальной триангуляции~// 
Геоинформатика: Теория и~практика, 1998. Вып.~1. С.~61--66.

\bibitem{2-man} %3
\Au{Лисин А.\,В., Файзуллин~Р.\,Т.} Эвристический алгоритм поиска при\-бли\-жен\-но\-го решения 
задачи Штейнера, основанный на физических аналогиях~// Компьютерная оптика, 2013. 
Т.~37. №\,4. С.~503--510.

\bibitem{4-man}
\Au{Препарата Ф., Шеймос~М.} Вы\-чис\-ли\-тель\-ная гео\-мет\-рия: Введение~/ Пер. с~англ.~--- 
М.: Мир, 1989. 478~c. (\Au{Preparata~F., Shamos~M.} Computational geometry. An 
introduction.~--- New York\,--\,Berlin\,--\,Heidelberg\,--\,Tokyo: Springer-Verlag, 
1985. 398~p.)
\bibitem{5-man}
\Au{Скворцов А.\,В., Мирза~Н.\,С.} Алгоритмы по\-стро\-ения и~анализа триангуляции.~--- 
Томск: ТГУ, 2006. 168~с.
\bibitem{6-man}
\Au{Скворцов А.\,В.} Триангуляция Делоне и~ее применение.~--- Томск: ТГУ, 2002. 128~с.
\bibitem{7-man}
\Au{Ивановский С.\,А., Симончик~С.\,К.} Алгоритмы вы\-чис\-ли\-тель\-ной гео\-мет\-рии. 
Пересечение отрезков: метод заметания плос\-кости~// Компьютерные инструменты 
в~образовании, 2007. №\,4. С.~18--33.
\bibitem{8-man}
\Au{Игошин В.\,И.} Математическая логика и~тео\-рия алгоритмов.~--- М.: Академия, 2008. 
448~с.
\bibitem{9-man}
\Au{Волков И.\,К., Канатников~А.\,Н.} Интегральные преобразования и~операционное 
ис\-чис\-ле\-ние.~--- 2-е изд.~--- М.: МГТУ им.\ Н.\,Э.~Баумана, 2002. 
228~с.
\bibitem{10-man}
\Au{Бронштейн И.\,Н., Семендяев~К.\,А.} Справочник по математике.~--- М.: Наука, 1964. 
608~с.

%\pagebreak

\bibitem{11-man}
\Au{Потехина Е.\,А.} Применение произведения Адамара в~задаче пе\-ре\-чис\-ле\-ния 
упорядоченных разбиений~// Наукоемкие технологии, 2012. Т.~13. №\,8. С.~58--65.
\bibitem{12-man}
\Au{Гантмахер Ф.\,Р.} Тео\-рия мат\-риц.~--- 2-е изд., доп.~--- М.: Наука, 1966. 576~с.
\bibitem{13-man}
\Au{Химмельблау Д.} Прикладное нелинейное программирование~/ Пер. с~англ.~--- М.: Мир, 
1975. 526~с. (\Au{Himmelblau~D.\,M.} Applied nonlinear programming.~--- New York, NY, USA: 
McGraw-HilI Book Co., 1972. 498~p.)
{\looseness=1

}
 \end{thebibliography}

 }
 }

\end{multicols}

\vspace*{-6pt}

\hfill{\small\textit{Поступила в~редакцию 24.08.17}}

\vspace*{6pt}

%\newpage

%\vspace*{-24pt}

\hrule

\vspace*{2pt}

\hrule

%\vspace*{8pt}


\def\tit{MATHEMATICAL MODEL OF~OPTIMAL TRIANGULATION}

\def\titkol{Mathematical model of~optimal triangulation}

\def\aut{A.~Batenkov, Yu.~Maniakov, A.~Gasilov, and~O.~Yakovlev}

\def\autkol{A.~Batenkov, Yu.~Maniakov, A.~Gasilov, and~O.~Yakovlev}

\titel{\tit}{\aut}{\autkol}{\titkol}

\vspace*{-9pt}


\noindent
  Orel Branch of the Institute of Informatics Problems, Federal Research Center 
``Computer Science and Control'' of the Russian Academy of Sciences, 
137~Moskovskoe Shosse, Orel 302025, Russian Federation


\def\leftfootline{\small{\textbf{\thepage}
\hfill INFORMATIKA I EE PRIMENENIYA~--- INFORMATICS AND
APPLICATIONS\ \ \ 2018\ \ \ volume~12\ \ \ issue\ 2}
}%
 \def\rightfootline{\small{INFORMATIKA I EE PRIMENENIYA~---
INFORMATICS AND APPLICATIONS\ \ \ 2018\ \ \ volume~12\ \ \ issue\ 2
\hfill \textbf{\thepage}}}

\vspace*{3pt}

    
  
\Abste{The problem of synthesis of optimal planar convex triangulation is 
formalized. This problem arises in different applications of informatics problems and 
is very actual for its sections such as computer graphics and geographical information 
systems. The mathematical model is represented as an extremum problem with 
infinite number of constraints, as a minimax problem with bound variables, and as an 
extremum problem with additional constraints on line segments intersections of 
triangulation with limited number of constraints. By putting idempotent limitations 
on Boolean variables, the initial integer-valued problem could be solved as a general 
mathematical programming problem on a continuum set of answers. In addition, the 
comparison of results obtained by the greedy algorithm based on the represented 
model and Delaunay triangulation is provided.}

\KWE{mathematical model; triangulation; Delaunay triangulation}

\DOI{10.14357/19922264180210} %

%\vspace*{-14pt}

  %\Ack
  % \noindent
 %  The paper was supported by the Russian Scientific Foundation (project 16-11-10227).



%\vspace*{-3pt}

  \begin{multicols}{2}

\renewcommand{\bibname}{\protect\rmfamily References}
%\renewcommand{\bibname}{\large\protect\rm References}

{\small\frenchspacing
 {%\baselineskip=10.8pt
 \addcontentsline{toc}{section}{References}
 \begin{thebibliography}{99}
\bibitem{1-man-1}
\Aue{Gasilov, A.\,V., and O.\,A.~Yakovlev.} 2016. Sozdanie rea\-li\-stich\-nykh 
naborov dannykh dlya al\-go\-rit\-mov trekh\-mer\-noy rekonstruktsii s~pomoshch'yu 
virtual'noy s"em\-ki komp'yuternoy modeli [Genereating realistic  
structure-from-motion datasets with a~virtual shooting]. \textit{Sistemy i~Sredstva 
Informatiki~---Systems and Means of Informatics} 26(2):98--107.

\bibitem{3-man-1}
\Aue{Kostyuk, Yu.\,L., and A.\,L.~Fuks.} 1998. Priblizhennoe vychislenie 
optimal'noy triangulyatsii [Approximate calculation of optimal triangulation]. 
\textit{Geoinformatika: teoriya i~praktika} [Geoinformatics: Theory and practice] 
 1:61--66.
\bibitem{2-man-1}
\Aue{Lisin, A.\,V., and R.\,T.~Faizullin.} 2013. Evristicheskiy algoritm poiska 
priblizhennogo resheniya zadachi Shteynera, osnovannyy na fizicheskikh 
analogiyakh [Heuristic algorithm for finding approximate solution of Steiner 
problem based on physical analogies]. \textit{Computer 
Optics} 37(4):503--510.

\bibitem{4-man-1}
\Aue{Preparata, F., and M.~Shamos.} 1985. \textit{Computational geometry: An 
introduction}.~--- New York\,--\,Berlin\,--\,Heidelberg\,--\,Tokyo: Springer-Verlag. 398~p.
\bibitem{5-man-1}
\Aue{Skvortsov, A.\,V., and N.\,S.~Mirza.} 2006. \textit{Algoritmy postroeniya 
i~analiza triangulyatsii} [Generation and analysis algorithms of triangulation]. 
Tomsk: Tomsk State University. 168~p.
\bibitem{6-man-1}
\Aue{Skvortsov, A.\,V.} 2002. \textit{Triangulyatsiya Delone i~ee primenenie} 
[Delaunay triangulation and its application]. Tomsk: Tomsk State University. 128~p.
\bibitem{7-man-1}
\Aue{Ivanovskiy, S.\,A., and S.\,K.~Simonchik.} 2007. Algoritmy vychislitel'noy 
geometrii. Peresechenie otrezkov: metod zametaniya ploskosti [Computational 
geometry algorythms: Plane covering method]. \textit{Komp'yuternye instrumenty 
v~obrazovanii} [Computer instruments in education] 4:18--33.
\bibitem{8-man-1}
\Aue{Igoshin, V.\,I.} 2008. \textit{Matematicheskaya logika i~teoriya algoritmov} 
[Mathematical logic and algorithms' theory]. Moscow: Akademiya. 448~p.
\bibitem{9-man-1}
\Aue{Volkov, I.\,K., and A.\,N.~Kanatnikov.} 2002. \textit{Integral'nye 
preobrazovaniya i~operatsionnoe ischislenie} [Integral 
transformation and operational calculus]. 2nd ed. Moscow: MSTU. 228~p.
\bibitem{10-man-1}
\Aue{Bronshtein, I.\,N., and K.\,A.~Semendyaev.} 1964. \textit{Spra\-voch\-nik po 
matematike} [Mathematics handbook]. Moscow: Nauka. 608~p.
\bibitem{11-man-1}
\Aue{Potekhina, E.\,A.} 2012. Primenenie proizvedeniya Adamara v~zadache 
perechisleniya uporyadochennykh razbieniy [Application of Adamar's 
multiplication for enumeration of ordered decompositions problem]. 
\textit{Naukoemkie tekhnologii} [Science Technologies]  13(8):58--65.
\bibitem{12-man-1}
\Aue{Gantmakher, F.\,R.} 1966. \textit{Teoriya matrits} [The matrix theory]. 2nd 
ed.  Moscow: Nauka. 576~p.
\bibitem{13-man-1}
\Aue{Himmelblau, D.\,M.} 1972. Applied nonlinear programming. New York, 
NY: McGraw-Hill Book Co. 498~p.
\end{thebibliography}

 }
 }

\end{multicols}

\vspace*{-3pt}

\hfill{\small\textit{Received August 24, 2017}}

%\vspace*{-24pt}
  
  \Contr
  
  \noindent
  \textbf{Batenkov Alexander A.} (b.\ 1957)~--- research engineer, Orel Branch of 
the Institute of Informatics Problems, Federal Research Center ``Computer Science 
and Control'' of the Russian Academy of Sciences, 137~Moskovskoe Shosse, Orel 
302025, Russian Federation; \mbox{batenkov1957@mail.ru}
  
  \vspace*{3pt}
  
  \noindent
  \textbf{Maniakov Yuri A.} (b.\ 1984)~--- senior scientist, Orel Branch of the 
Institute of Informatics Problems, Federal Research Center ``Computer Science and 
Control'' of the Russian Academy of Sciences, 137~Moskovskoe Shosse, Orel 
302025, Russian Federation; \mbox{maniakov\_yuri@mail.ru}
  
  \vspace*{3pt}
  
  \noindent
  \textbf{Yakovlev Oleg A.} (b.\ 1992)~--- junior scientist, Orel Branch of the 
Institute of Informatics Problems, Federal Research Center ``Computer Science and 
Control'' of the Russian Academy of Sciences, 137~Moskovskoe Shosse, Orel 
302025, Russian Federation; \mbox{maucra@gmail.com}
  
  \vspace*{3pt}
  
  \noindent
  \textbf{Gasilov Artur A.} (b.\ 1992)~--- junior scientist, Orel Branch of the 
Institute of Informatics Problems, Federal Research Center ``Computer Science and 
Control'' of the Russian Academy of Sciences, 137~Moskovskoe Shosse, Orel 
302025, Russian Federation; \mbox{gasilov.av@ya.ru}
\label{end\stat}


\renewcommand{\bibname}{\protect\rm Литература} 
  