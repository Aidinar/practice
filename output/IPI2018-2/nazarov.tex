\def\stat{nazarov}

\def\tit{ВЕРОЯТНОСТНАЯ МОДЕЛЬ ВЛИЯНИЯ КНИГИ ЗАКАЗОВ
НА~ПРОЦЕСС ЦЕНЫ}

\def\titkol{Вероятностная модель влияния книги заказов
на~процесс цены}

\def\aut{Е.\,В.~Быковец$^1$, В.\,В.~Лаврентьев$^2$,  Л.\,В.~Назаров$^3$}

\def\autkol{Е.\,В.~Быковец, В.\,В.~Лаврентьев,  Л.\,В.~Назаров}

\titel{\tit}{\aut}{\autkol}{\titkol}

\index{Быковец Е.\,В.}
\index{Лаврентьев В.\,В.}
\index{Назаров Л.\,В.}
\index{Nazarov L.\,V.}
\index{Lavrentyev V.\,V.}
\index{Bykovets E.\,V.}




%{\renewcommand{\thefootnote}{\fnsymbol{footnote}} \footnotetext[1]
%{Работа поддержана РНФ (проект 16-11-10227).}}


\renewcommand{\thefootnote}{\arabic{footnote}}
\footnotetext[1]{Московский государственный университет им.\ М.\,В.~Ломоносова, 
факультет вычислительной математики и~кибернетики, 
\mbox{eugene.bykovets@stud.cs.msu.su}}
\footnotetext[2]{Московский государственный университет им.\ М.\,В. Ломоносова, 
факультет вычислительной математики и~кибернетики, \mbox{lavrent@cs.msu.ru}}
\footnotetext[3]{Московский государственный университет им.\ М.\,В. Ломоносова, 
факультет вычислительной математики и~кибернетики, 
\mbox{nazarov@cs.msu.ru}}

\vspace*{-9pt}


   

\Abst{Рассматривается модель книги заказов, в~которой заказы на покупку 
и~продажу образуют два независимых процесса Кокса. Предложен механизм
        влияния поступающих заказов на цену актива на основе физической модели 
        абсолютно упругого соударения. В~этой модели цена представляет собой 
        материальную точку с~некоторой массой, движущуюся по прямой без трения. 
        Приходящие заказы на покупку и~уходящие заказы на продажу упруго 
        сталкиваются с~ней и~придают дополнительный импульс в~одном 
        направлении, а~приходящие заказы на продажу и~уходящие заказы на покупку~--- 
        в~противоположном. Получена функциональная предельная теорема для процесса 
        цены при высокой интенсивности входящего потока заказов, позволяющая 
        аппроксимировать его некоторым процессом Леви.}

\KW{лимитные заявки; абсолютно упругий удар; модель книги заказов; процесс цены; процесс Кокса; 
функциональная предельная тео\-рема}

\DOI{10.14357/19922264180205}
  
\vspace*{-3pt}


\vskip 10pt plus 9pt minus 6pt

\thispagestyle{headings}

\begin{multicols}{2}

\label{st\stat}

\section{Введение}

Рассмотрим некоторый торгуемый на бирже актив, в~отношении которого 
могут приходить два вида запросов: на покупку и~на продажу. 
Список таких запросов формирует книгу заказов для данного актива. 
Информация, содержащаяся в~книге, позволяет делать прогнозы относительно 
возможного движения цены рассматриваемого актива. 
Особенный интерес эта информация начала представлять с~развитием высокочастотной
 торговли.

В работе рассматривается модель, которая описывает влияние книги 
заказов на цену актива. Базовой моделью для исследования была выбрана модель 
книги заказов, близкая к~описанной в~\cite{first}. Основное отличие состоит
 в~следующем: на бирже торгуемый актив имеет цену, которая размещается в~узлах 
 сетки~$nh$, где $n$~--- некоторое целое число; $h$~--- тик, т.\,е.\ 
 минимальное изменение цены. Однако высокая частота узлов сетки позволяет считать, 
 что рассматриваемый актив может иметь произвольную цену, равно как и~заявки на 
 покупку и~продажу торгуемого актива. С~учетом этого рассматриваем следующую модель 
 влияния заявок на цену актива, используя физическую аналогию.  
 %
 Рассмотрим 
 материальную точку массой~$M$, которая может двигаться по прямой (числовой оси) 
 в~любом на\-прав\-ле\-нии без трения. При этом, связывая модель 
 физическую и~математическую, будем считать, что текущее положение на оси~--- 
 это текущая цена~$X(t)$.  Будем далее для краткости называть ценой и~саму 
 указанную материальную точку массой~$M$, т.\,е.\ 
 будем говорить о~ско\-рости цены, импульсе цены и~т.\,п. 
 Каждый заказ на продажу (поступающий по цене~$A_{i}$
не ниже, чем~$X(t)$) придает цене дополнительный импульс в~направлении от~$A_{i}$ 
к~$X(t)$. Заказы живут экспоненциальное время, после чего уходят из книги 
за счет исполнения или отмены. Уход заказа из книги  придает цене 
дополнительный импульс той же абсолютной величины, что и~при ее 
поступлении, но противоположного направления. С~заказами на продажу все аналогично, 
только приходят они с~ценой, не превосходящей~$X(t)$.

Цель данной работы состоит в~том, чтобы выяснить, какой процесс движения цены 
порождает такая сис\-те\-ма при интенсивном потоке приходящих заказов.
Близкая задача решалась авторами в~работе~\cite{second}, но там рас\-смат\-ри\-вал\-ся 
другой механизм влияния по\-сту\-па\-ющих заказов на цену. Здесь следует упомянуть 
и~работу~\cite{Korolev1}, в~которой также изуча\-ет\-ся связь механизма 
функционирования книги заказов на микроуровне с~процессом цены.

\vspace*{-12pt}

\section{Описание модели}

\vspace*{-4pt}

Будем рассматривать работу книги заказов на временн$\acute{\mbox{о}}$м интервале $t\hm\in[0,T]$. 
В~начальный момент времени $t\hm=0$ заказов в~книге нет. Считаем, что поток 
приходящих заказов является процессом Кокса следующего вида:
\begin{equation*}
\left\{N(\Lambda(t))=N_1(\Lambda(t)), t\geqslant0\right\}\,,
\end{equation*}
где
$N_1$~--- пуассоновский процесс, интенсивность которого равна~1; 
$\Lambda(t)$~--- стартующий из нуля случайный процесс, у~которого траектории 
являются неубывающими и~непрерывными справа функциями, а~также справедливо 
$\mathbb{P}(\Lambda(t)\hm<\infty)$.

Каждый заказ, приходящий в~книгу, находится в~ней некоторое случайное время. 
Более точно, время нахождения конкретного заказа в~книге является случайной 
величиной, распределенной по экспоненциальному закону.

Для каждого приходящего заказа определен набор параметров $(h_i, \gamma_i, \eta_i)$, 
где 
$h_i$~--- абсолютное значение разности между ценой заказа и~текущей ценой; 
$\gamma_i$~--- разность между ско\-ростью приходящего заказа и~текущей ско\-ростью цены; 
$\eta_i$~--- время пребывания заказа в~книге. 

Случайные величины $h_i$, $\gamma_i$ и~$\eta_i$, $i\hm=1, 2, \ldots,$ независимы 
в~совокупности и~не зависят от потока заявок, a~$\eta_i$ распределены 
экспоненциально с~параметром~$\mu$. Параметр~$\gamma_i$ фактически определяет 
тип заказа. Для заказов на покупку~$\gamma_i$ положительны, для заказов на 
продажу~--- отрицательны.

В качестве физической модели влияния заказа на цену возьмем модель абсолютно 
упругого удара.  Считаем, что $i$-й заказ, поступающий в~момент~$t$,~--- 
это материальная точка массой $m_0(h_i)\hm>0$, которая движется по той же прямой, 
что и~цена, и~имеет в~момент времени~$t$ скорость, равную $u_i \hm= v_{i-1}\hm+\gamma_i$, 
где~$v_{i-1}$~--- скорость цены до столкновения с~$i$-м заказом. В~момент~$t$ 
происходит их упругое соударение. На распределения~$h_i$ и~$\gamma_i$ 
наложим следующие ограничения:
\begin{equation}
\mathbb{E}\gamma_i = 0;\enskip
\mathbb{E}\gamma_i^2 = \overline{\gamma}<\infty;\enskip
\mathbb{E}m_0(h_i)^2 = \overline{m} < \infty\,.
\label{e1-naz}
\end{equation}
Смысл первого условия заключается в~том, что разности между скоростями 
приходящих заказов и~текущей скоростью актива для заказов на покупку и~продажу 
в~среднем равны. Остальные ограничения имеют технический характер и~лишь постулируют 
конечность соответствующих моментов.

Функция $m_0$ является убывающей на интервале~$(0, \infty)$, поскольку 
воздействие заказа на цену тем больше, чем ближе его цена к~текущей цене актива.
Это соответствует реальному положению дел на рынке, где заказы на уровнях, 
близких к~текущей цене, выставляются более ответственно, так как могут быть 
тут же удовлетворены. В~то же время заказы на более удаленных уровнях чаще 
ставятся для дезориентации других участников  рынка и~снимаются\linebreak\vspace*{-12pt}

\columnbreak

\noindent
 при приближении к~ним 
цены. Иными словами, они не отражают реальный спрос. 

\vspace*{-7pt}

\section{Процесс цены}

Рассмотрим точку на числовой прямой, которая представляет собой текущую цену 
актива. Положим 
$M$~--- масса точки; 
$v_{i}$~--- текущая скорость цены, полученная после удара $i$-й частицы, 
полагаем $v_ {0}=0$; 
$u_{i}$~--- ско\-рость $i$-го заказа до соударения; 
$m_0(h_i)$~--- масса $i$-го заказа. 
Как было сказано в~предыду\-щем разделе, данная точ\-ка (исследуемая цена) 
в~определенные моменты времени абсолютно упруго соударяется с~другими 
частицами (заказами). В~этом случае есть возможность выразить ско\-рость точки после 
удара $i$-й час\-ти\-цы через массу точ\-ки и~массу час\-ти\-цы, а~также их ско\-рости 
до столк\-но\-ве\-ния (это следует из закона сохранения импульса и~закона сохранения 
энергии, см.~\cite[гл.~4, \S\,28]{third}):
\begin{equation*}
v_i = -v_{i-1} +2\fr{Mv_{i-1}+ m_0(h_i)u_i}{M+ m_0(h_i)}\,.
\end{equation*}
Обозначим $\Delta v_i \hm= v_i - v_{i-1}$, тогда

\noindent
\begin{multline*}
\Delta v_i = -2v_{i-1} +2\fr{Mv_{i-1}+ m_0\left(h_i\right)u_i}{M+ m_0\left(h_i\right)} = {}\\
{}=
\fr{2 m_0\left(h_i\right)}{M+ m_0\left(h_i\right)}\left(u_{i}-v_{i-1}\right).
\end{multline*}
Фактически $\Delta v_i$ показывает изменение ско\-рости цены после соударения 
с~$i$-м заказом.

Пусть в~начальный момент книга заказов пус\-та, а~начальная скорость $v_0 \hm= 0$. 
Далее полагаем, что заказ с~номером~$i$ приходит в~момент времени~$\tau_{i0}$ и~уходит 
в~момент времени~$\tau_{i1}$. В~итоге получаем, что скорость цены является случайным 
процессом $\{V(t), t\hm\geqslant0\}$ с~ку\-соч\-но-по\-сто\-ян\-ны\-ми траекто\-риями:

\noindent
\begin{equation*}
V(t) = \sum\limits_{i=1}^{N_1(\Lambda(t))}\Delta v_i\mathbb{I}_{\{\tau_{i0}\le t\le 
\tau_{i1}\}}(t)\,.
\end{equation*}
Тогда изменение цены за время~$T$ будет иметь вид:
\begin{equation}
X(T) = \int\limits_{0}^{{T}} V(t)\, dt = \sum\limits_{i=1}^{N_1(\Lambda(T))}X_i(T)\,,
\label{e2-naz}
\end{equation}
где $X_i(T)$~--- изменение цены на интервале [0,T] за счет удара $i$-го заказа:

\vspace*{-2pt}

\noindent
\begin{multline*}
X_i(T) = \int\limits_{0}^{{T}} \Delta v_i\mathbb{I}_{\{\tau_{i0}\leqslant 
t\leqslant \tau_{i1}\}}(t)\,dt ={}\\
{}= \fr{2 m_0(h_i)}{M+ m_0(h_i)}
\int\limits_{0}^{{T}}(u_{i}-v_{i-1})\mathbb{I}_{\{\tau_{i0}\leqslant t
\leqslant \tau_{i1}\}}(t)\,dt\,.
\end{multline*}
Поскольку по определению $u_{i}\hm-v_{i-1} \hm= \gamma_i$ , то последнее 
выражение можем переписать в~виде ($a \wedge b\hm = \min(a,b)$):
\begin{multline*}
X_i(T)=\fr{2 m_0(h_i)\gamma_i}{M+ m_0(h_i)}\int\limits_{0}^{{T}}
\mathbb{I}_{\{\tau_{i0}\leqslant t\leqslant \tau_{i1}\}}(t)\,dt = {}\\
{}=
\fr{2 m_0(h_i)\gamma_i}{M+ m_0(h_i)}\left(T\wedge\tau_{i1}-
T\wedge\tau_{i0}\right) ={}\\
{}=\fr{2 m_0(h_i)\gamma_i}{M+ m_0(h_i)}
\left(T\wedge\left(\tau_{i0}+\eta_i\right)-T\wedge\tau_{i0}\right).
\end{multline*}
Строго говоря, случайные величины $\{X_i(T)$, $i\hm=1,2,\dots\}$ не являются 
независимыми, но суммы в~(\ref{e2-naz}) можно представить в~виде сумм 
независимых случайных величин. 

Рассмотрим распределение вектора 
моментов прихода заявок $\tau_0\hm=\{\tau_{10},\ldots ,\tau_{n0}\}$. По свойству 
пуассоновского потока при $N_1(\Lambda(T))\hm=n$ распределение~$\tau_0$ 
есть распределение вариационного ряда выборки из~$n$~независимых случайных 
величин,\linebreak равномерно распределенных на $[0, \Lambda(T)]$.
Поскольку значение конечной суммы при перестановке\linebreak слагаемых не меняется, 
далее будем считать, что в~каждой из сумм~(\ref{e2-naz})~$\tau_{i0}$ 
независимы и~равномерно распределены на $[0, \Lambda(T)]$, а~следовательно, 
случайные величины~$\{X_i(T)$, $i\hm=1,2,\dots\}$ также независимы.

Изучим асимптотические свойства моментов~$X_i(T)$.

\smallskip

\noindent
\textbf{Лемма~1.}\ \textit{ Пусть случайная величина~$\xi$ 
равномерно распределена на $[0,T]$, $\eta_0$ не зависит от~$\xi$ и~имеет 
экспоненциальное распределение с~параметром~$\mu$ и}
\begin{equation*}
s = T\wedge\left(\xi+\eta_0\right)-\xi\,.
\end{equation*}

\vspace*{-8pt}

\noindent
\textit{Тогда}
\begin{enumerate}[(1)]
\item $s\stackrel{d}=\xi\wedge\eta_0$;
\item \textit{моменты случайной величины s обладают следующими асимптотическими 
свойствами}: 
\begin{equation*}
\lim\limits_{\mu\rightarrow\infty}\mu\mathbb{E}s = 1;\
 \lim\limits_{\mu\rightarrow\infty}\mu^2\mathbb{E}s^2 = 2;\
  \lim\limits_{\mu\rightarrow\infty}\mu^2\mathbb{D}s = 1.
\end{equation*}
\end{enumerate}

\noindent
{Д\,о\,к\,а\,з\,а\,т\,е\,л\,ь\,с\,т\,в\,о\,.}\ \ 
Вычислим математическое ожидание случайной величины~$s$ с~учетом независимости~$\xi$ 
и~$\eta_0$. По определению
\begin{equation*}
s = T\wedge\left(\xi+\eta_0\right)-\xi = (T-\xi)\wedge\eta_0\,.
\end{equation*}
Справедливость первого утверждения леммы следует из независимости~$\xi$ и~$\eta_0$ 
и~одинаковой распределенности~$\xi$ и~$T\hm-\xi$. Таким образом, математическое 
ожидание~$s$ есть
\begin{equation*}
\mathbb{E}s  = \mathbb{E}\left(\xi\wedge\eta_0\right).
\end{equation*}
При вычислении моментов неоднократно будет требоваться значение интеграла
\begin{equation*}
\int\limits_{0}^{{T}}y^ne^{-\mu y}\,dy = \fr{n!}{\mu^{n+1}}\,F_{n+1}(T)\,,
\end{equation*}
где $F_{n+1}$~--- 
функция распределения Эрланга $(n+1)$-го порядка:
$$
F_{n+1}(x) = 1 - e^{-\mu x}\sum\limits_{i=1}^{n}\fr{\mu^iT^i}{i!}\,.
$$ 
Вычислим 
$\mathbb{E}(\xi\wedge\eta_0)$:
\begin{multline*}
\mathbb{E}\left(\xi\wedge\eta_0\right) =
\fr{\mu}{T}\int\limits_{0}^{\infty}\!e^{-\mu y}\,dy\int\limits_{0}^{T}(x\wedge y)\,dx 
 ={}\\
 {}=\fr{\mu}{T}\int\limits_{0}^{\infty}\!e^{-\mu y}\,dy\left\{
 \mathbb{I}_{\{y<T\}}\left[\int\limits_{0}^{y}x\,dx+
 \int\limits_{y}^{T}y\,dx\right] + {}\right.\\
\left. {}+
 \mathbb{I}_{\{y\geqslant T\}}\int\limits_{0}^{T}x\,dx\right\} = \fr{\mu}{T}\left[T\int\limits_{0}^{T}\!ye^{-\mu y}\,dy -{}\right.\\
\left.{}-
\fr{1}{2}\int\limits_{0}^{T}\!y^2e^{-\mu y}\,dy + 
\fr{T^2}{2}\int\limits_{T}^{\infty}\!e^{-\mu y}\,dy \right]={} \\
{}=\left[\fr{1}{\mu}-\fr{1}{\mu^2 T}\right]+\left[\fr{T}{2}+\fr{1}{\mu^2 T}\right]
e^{-\mu T}.
\end{multline*}
И,~соответственно,
\begin{multline*}
\lim\limits_{\mu \rightarrow \infty}\mu\mathbb{E}s ={}\\
{}= 
\lim\limits_{\mu \rightarrow \infty}\left\{\left[1-\fr{1}{\mu T}\right]+
\left[\fr{T}{2}+\fr{1}{\mu^2 T}\right]\mu e^{-\mu T}\right\}=1. 
\end{multline*}
Вычислим $\mathbb{E}s^2\hm=\mathbb{E}(\xi\wedge\eta_0)^2$:
\begin{multline*}
\mathbb{E}\left(\xi\wedge\eta_0\right)^2=
\fr{\mu}{T}\int\limits_{0}^{\infty}\!e^{-\mu y}\,dy
\int\limits_{0}^{T}\left(x\wedge y\right)^2\,dx = {}\\
{}=\fr{\mu}{T}\int\limits_{0}^{\infty}\!e^{-\mu y}\,dy\left\{
\mathbb{I}_{\{y<T\}}\left[{\int\limits_{0}^{y}\!x^2\,dx+
\int\limits_{y}^{T}\!y^2\,dx}\right] + {}\right.\\
\left.{}+
\mathbb{I}_{\{y\geqslant T\}}\int\limits_{0}^{T}\!x^2\,dx\right\}= 
\fr{\mu}{T}\left[T\int\limits_{0}^{T}\!y^2e^{-\mu y}\,dy -{}\right.\\
\left.{}-
\fr{2}{3}\int\limits_{0}^{T}\!y^3e^{-\mu y}dy + 
\fr{T^3}{3}\int\limits_{T}^{\infty}\!e^{-\mu y}dy \right]= {}\\
{}=\left[\fr{2}{\mu^2}-\fr{4}{\mu^3 T}\right]+\left[
\fr{2}{\mu^2}+\fr{4}{\mu^3 T}\right]e^{-\mu T}.
\end{multline*}
Отсюда получаем асимптотику второго момента
\begin{multline*}
\lim\limits_{\mu \rightarrow \infty}\mu^2\mathbb{E}s^2 = {}\\
{}=
\lim\limits_{\mu \rightarrow \infty}\left\{\left[2-\fr{4}{T\mu}\right]+
\left[\fr{2}{\mu^2}+\fr{4}{\mu^3 T}\right]\mu^2e^{-\mu T}\right\} = 2
\end{multline*}
и дисперсии
\begin{multline*}
\lim\limits_{\mu \rightarrow \infty}\mu^2\mathbb{D}s =  
\lim\limits_{\mu \rightarrow \infty}\mu^2\left[\mathbb{E}s^2-(\mathbb{E}s)^2\right] ={}\\
{}=
 \lim\limits_{\mu \rightarrow \infty}\mu^2\mathbb{E}s^2 - 
  \lim\limits_{\mu \rightarrow \infty}(\mu\mathbb{E}s)^2 =1\,.
\end{multline*}
Утверждение леммы доказано.

\smallskip

Рассмотрим следующую последовательность:
\begin{equation}
\left\{X_{n}(t) = \sum\limits_{i=1}^{N_1(\Lambda_n(t))}X_{ni}(t),\ 
 t \in [0,T]\right\}.
 \label{e3-naz}
\end{equation}
При этом каждому члену~$\{X_n\}$ соответствует процесс $\{\Lambda_n(t)$, 
$t\hm\in [0,T]\}$, параметр~$\mu_n$ и~функция массы ударяющей частицы (приходящего 
заказа)~$m_{n0}$. При увеличении~$n$ будем увеличивать интенсивность входящего 
потока заявок~$\Lambda_n(t)$ и~уменьшать время пребывания заказов в~книге 
посредством увеличения~$\mu_n$ ($\Lambda_n(t)\hm\Rightarrow\infty$, 
$\mu_n\hm\rightarrow\infty$ при $n \hm\rightarrow \infty$). Будем также 
уменьшать влияние отдельного заказа на цену:
\begin{equation*}
m_{n0} = \alpha_n m_0,\enskip
 \alpha_n >0\,,\enskip
  \alpha_n \rightarrow 0\,,\enskip
   n\rightarrow\infty\,.
\end{equation*}
Получим асимптотические свойства моментов случайных величин $X_{n1}(T)$ 
при установленных параметрических зависимостях. Аргумент~$T$ у~них одинаков и~для 
краткости будем его опускать.

\smallskip

\noindent
\textbf{Лемма~2.}\  \textit{Пусть $\mu_n\hm \rightarrow \infty$, 
$\alpha_n \hm\rightarrow 0$, $k_n \hm= {\mu_n^2}/{\alpha_n^2}$.  Тогда}
\begin{enumerate}[(1)]
\item $k_n\mathbb{E}X_{n1}\hm\rightarrow 0$, 
$k_n\mathbb{D}X_{n1}\hm\rightarrow {8\overline{m}\overline{\gamma}}/{M^2}$,
$n\hm\rightarrow\infty$;
\item \textit{Выполняется условие Линдеберга, т.\,е.\ для любого} $\varepsilon \hm>0$
\begin{equation*}
\lim\limits_{n\rightarrow\infty}k_n
\mathbb{E}\left[X_{n1}^2\mathbb{I}(|X_{n1}|>\varepsilon)\right]=0\,.
\end{equation*}
\end{enumerate}

\noindent
Д\,о\,к\,а\,з\,а\,т\,е\,л\,ь\,с\,т\,в\,о\,.\ \
Как было показано выше, 
$$
X_{n1} = \fr{2 m_{n0}(h_1)\gamma_{1}}{M+ m_{n0}(h_1)}s_n\,,
$$
 где $s_n\stackrel{d} =\xi\wedge\eta_{n0}$ и~$\eta_{n0}$ распределена 
 экспоненциально с~параметром~$\mu_n$, а~величины $h_1$, $\xi$, $\eta_{n0}$
и~$\gamma_{1}$ независимы. Так как в~соответствии с~(\ref{e1-naz}) 
 $\mathbb{E}\gamma_1 \hm= 0$, то $k_n\mathbb{E}X_{n1}\hm=0$ для любого~$n$. 
 Проверим соотношение для дисперсии:
 
 \noindent
\begin{multline*}
k_n\mathbb{D}X_{n1} = \fr{\mu_n^2}{\alpha_n^2}\, \mathbb{E} 
\left[\fr{2 m_{n0}(h_1)\gamma_{1}}{M+ m_{n0}(h_1)}\,s_n\right]^2={}\\
{}=4\mathbb{E}\gamma_{1}^2\mathbb{E}\left[
\fr{m_{n0}(h_1)}{\alpha_n}\,\fr{1}{M+m_{n0}(h_1)}\right]^2\mu_n^2\mathbb{E}s_n^2 = {}\\
{}=
4\overline{\gamma}\mathbb{E}\left[\fr{m_0(h_1)}{M+m_{n0}(h_1)}\right]^2\mu_n^2
\mathbb{E}s_n^2.
\end{multline*}

Последовательность случайных величин 
$\{[{m_0(h_1)}/({M+m_{n0}(h_1)})]^2\}$ мажорируется интегрируемой случайной 
величиной $[{m_0(h_1)}/{M}]^2$ и~поточечно сходится к~ней, так как $\alpha_n 
\hm\rightarrow 0$, $n\hm\rightarrow\infty$. Поэтому

\vspace*{-6pt}

\noindent
\begin{multline*}
\lim\limits_{n\rightarrow\infty}k_n\mathbb{D}X_{n1} =  
\lim\limits_{n\rightarrow\infty}4\overline{\gamma}\mathbb{E}
\left[\fr{m_0(h_1)}{M}\right]^2\mu_n^2\mathbb{E}s_n^2 ={}\\
{}=
\fr{4\overline{m}\overline{\gamma}}{M^2}
\lim\limits_{n\rightarrow\infty}\mu_n^2\mathbb{E}s_n^2 = 
\fr{8\overline{m}\overline{\gamma}}{M^2}\,.
%\label{e4-naz}
\end{multline*}

Докажем справедливость условия Линдеберга. Рассмотрим

\vspace*{-6pt}

\noindent
\begin{multline*}
\fr{\mu_n}{\alpha_n}\left\vert X_{n1}\right\vert = 
 \fr{\mu_n}{\alpha_n}\left\vert \fr{2 m_{n0}(h_1)\gamma_{1}}{M+ m_{n0}
 \left(h_1\right)}\,s_n\right\vert = {}\\
 {}=
 2\fr{m_{n0}(h_1)}{\alpha_n}\,\fr{1}{M+ m_{n0}(h_1)}\left\vert \gamma_1\right\vert
 \mu_ns_n \leqslant {}\\
{}\leqslant 2\fr{m_0(h_1)}{M}\left\vert \gamma_1\right\vert \mu_ns_n
\leqslant {}\\
{}\leqslant \fr{2m_0(h_1)|\gamma_1|\mu_n\eta_{n0}}{M} \stackrel{d}= 
\fr{2m_0(h_1)|\gamma_1|\hat{\eta}}{M}\,,
\end{multline*}
где $\hat{\eta}$ распределена экспоненциально с~па\-ра\-мет\-ром~1. 
Распределение случайной величины 
$$
Y_n =   \fr{{2m_0(h_1)|\gamma_1|\mu_n\eta_{n0}}}{M}
$$ 
не зависит от~$n$ и~согласно~(1) имеет конечный второй момент, поэтому

\noindent
\begin{multline*}
k_n\mathbb{E}\left[X_{n1}^2\mathbb{I}(|X_{n1}|>\varepsilon)\right] ={}\\
{}= 
k_n\mathbb{E}\left[X_{n1}^2\mathbb{I}(\sqrt{k_n}|X_{n1}|>
\sqrt{k_n}\varepsilon)\right] \leqslant {} \\
{}\leqslant \mathbb{E}\left[Y_{n}^2\mathbb{I}(|Y_{n}|>
\sqrt{k_n}\varepsilon)\right]
= \mathbb{E}\left[Y_{1}^2\mathbb{I}(|Y_{1}|>
\sqrt{k_n}\varepsilon)\right].\hspace*{-3.79228pt}
\end{multline*}
Последнее математическое ожидание стремится к~нулю при $n\hm\rightarrow\infty$ по 
тео\-ре\-ме Лебега о~ма\-жо\-ри\-ру\-емой сходимости. Утверждение леммы доказано.

\vspace*{2pt}

Сформулируем доказанную в~работе~\cite{fourth} функ\-циональную центральную 
предельную тео\-ре\-му, устанавливающую условия, при которых процессы вида~(\ref{e3-naz}) 
сходятся к~некоторому предельному процессу~$X$ в~про\-стран\-ст\-ве Скорохода 
$\mathcal{D}\hm = \mathit{(D[0,1], d_0)}$ (см.~\cite[гл.~3]{five}). 
Позднее были получены более сильные результаты, касающиеся схо\-ди\-мости обобщенных 
процессов Кокса~(\ref{e3-naz}) (см., на\-при\-мер,~\cite{Korolev_FLT}), 
но достаточно будет приводимого ниже утверж\-де\-ния.
{ %\looseness=1

}

\smallskip

\noindent
\textbf{Теорема}~\cite{fourth}.\ 
\textit{Пусть для некоторой неограниченно возрастающей последовательности 
чисел~$\{k_n\}_{n\geqslant1}$ выполнены условия}:
\begin{enumerate}[(1)]
\item \textit{существуют числа $a \hm\in \mathbb{R}$ и~$\sigma \hm> 0$ такие, что}
\begin{equation*}
k_n\mathbb{E}X_{n1} \rightarrow a;\enskip
 k_n\mathbb{D}X_{n1} \rightarrow \sigma^2 (n\rightarrow \infty);
\end{equation*}
\item \textit{условие Линдеберга, т.\,е.\ для любого} $\varepsilon\hm>0$
\begin{equation*}
\lim\limits_{n\rightarrow\infty}k_n\mathbb{E}\left[(X_{n1}-a_n)^2\mathbb{I}
\left(\left\vert X_{n1}-a_n\right\vert >\varepsilon\right)\right]=0\,,
\end{equation*}
\textit{где $\mathbb{I}(A)$~--- индикатор события}~$A$; $a_n \hm= \mathbb{E}X_{n1}$;
\item \textit{существует безгранично делимая случайная величина~$U$ такая, 
что $\mathbb{P}(U=0) \hm< 1$, $\mathbb{P}(U\geqslant0) \hm= 1$, 
$\mathbb{E}U^2 \hm< \infty$ и}
\begin{equation*}
k_n^{-1}\Lambda_n(1)\Rightarrow U,n\rightarrow\infty;
\end{equation*}
\item
$\displaystyle \sup\limits_nk_n^{-2}\mathbb{E}\Lambda_n(1)^2<\infty.
$
\end{enumerate}
\textit{Тогда обобщенные процессы Кокса $\{X_n\}$ 
слабо сходятся в~пространстве Скорохода~$\mathcal{D}$ к~процессу Леви~$X$ такому, что}
\begin{equation*}
X(1) \stackrel{d}=\sigma\sqrt{U}N(0,1)+aU\,,
\end{equation*}
\textit{где $N(0,1)$~--- случайная величина, имеющая стандартное нормальное 
распределение, при этом не зависящая от}~$U$.

Для последовательности $\{k_n = {\mu_n^2}/{\alpha_n^2}\}$ при $a\hm=0$ первые 
два условия теоремы выполняются по лемме~2. Таким образом, достаточно наложить 
определенные условия на ин\-тен\-сив\-ность входящего потока заявок, чтобы была 
справедлива сле\-ду\-ющая теорема.

\smallskip

\noindent
\textbf{Теорема 1.}  \textit{Пусть $\mu_n \hm\rightarrow \infty$, 
$\alpha_n \hm\rightarrow 0$, $k_n = {\mu_n^2}/{\alpha_n^2}$, 
$\sup_nk_n^{-2}\mathbb{E}\Lambda_n(1)^2\hm<\infty$ и~существует 
безгранично делимая случайная величина~$U$ такая, что}
\begin{equation*}
\mathbb{P}(U=0) < 1\,;\enskip
\mathbb{P}(U\geqslant0) = 1\,;\enskip
 \mathbb{E}U^2 < \infty
\end{equation*}
и

\noindent
\begin{equation*}
k_n^{-1}\Lambda_n(1)\Rightarrow U\,,\enskip n\rightarrow\infty\,.
\end{equation*}
\textit{Тогда обобщенные процессы Кокса~$\{X_n\}$ слабо сходятся в~пространстве 
Скорохода $\mathcal{D}$ к~процессу Леви~$X$ такому, что}
\begin{equation*}
X(1) \stackrel{d}=\sigma\sqrt{U}N(0,1)\,,
\end{equation*}

%\columnbreak

\noindent
\textit{где $\sigma = {8\overline{m}\overline{\gamma}}/{M^2}$, а $N(0,1)$~--- 
случайная величина со стандартным нормальным распределением, независимая от}~$U$. 

%\vspace*{-24pt}

\section{Заключение}

В настоящей работе была предложена модель механизма влияния по\-сту\-па\-ющих 
заказов на цену актива на основе физической модели абсолютно упругого 
соударения час\-тиц. 

Была установлена справедливость функциональной предельной 
тео\-ре\-мы, на основании результатов которой можно аппроксимировать процесс 
цены при интенсивном потоке приходящих заявок процессом Леви, приращения 
которого являются смесью нормальных законов и~поддаются более точному анализу. 
Такая аппроксимация дает также воз\-мож\-ность оценки риска динамических
 стратегий~\cite{six}.


%\vspace*{-48pt}

    {\small\frenchspacing
 {%\baselineskip=10.8pt
 \addcontentsline{toc}{section}{References}
 \begin{thebibliography}{9}
\bibitem{first} 
\Au{Kukanov A.} Stochastic models of limit order markets.~--- 
Columbia University, 2013. Ph.D. Thesis. 131~p.

\bibitem{second} 
\Au{Лаврентьев В.\,В., Назаров~Л.\,В.} 
Процесс движения цены, порожденный непрерывной моделью книги заказов~// 
Вестн. Тверского государственного ун-та. Сер. Прикладная математика, 2015. 
№~4. С.~55--63.

\bibitem{Korolev1}
\Au{Korolev V.\,Yu., Chertok~A.\,V., Korchagin~A.\,Yu, Zeifman~A.\,I.} 
Modeling high-frequency order flow imbalance by functional limit theorems for 
two-sided risk processes~// Appl. Math. Comput., 2015. Vol.~253. P.~224--241.

\bibitem{third} 
\Au{Сивухин Д.\,В.} Общий курс физики.~--- 
В~5 т.~--- Т.~1. Механика.~--- 4-е изд.~--- М.: МФТИ, 2005. 560~с.

\bibitem{fourth} 
\Au{Кащеев Д.\,Е.} Моделирование динамики финансовых временных рядов и~оценивание 
производных ценных бумаг: Дис.\ \ldots\ канд. физ.-мат. наук.~--- 
Тверь: ТвГУ, 2001. 191~c.

\bibitem{five} 
\Au{Биллингсли П.} Сходимость вероятностных мер~/
Пер. с~англ.~--- М.: Наука, 1977. 353~с.
(\Au{Billingsley~P.}  
{Convergence of probability measures}.~--- New York, NY, USA: John Wiley \& Sons, Inc., 
1977. 277~p.)

\bibitem{Korolev_FLT} 
\Au{Korolev V.\,Yu., Chertok~A.\,V., Korchagin~A.\,Yu, Kossova~E.\,V., Zeifman~A.\,I.} 
A~note on functional limit theorems for compound Cox processes~// 
J.~Math. Sci., 2016. Vol.~218. No.\,2. P.~182--194.

\bibitem{six} 
\Au{Balasanov~Y., Doynikov~A., Lavrent'ev~V., Nazarov~L.} 
Estimating risk of dynamic trading strategies from high frequency data flow~// 
Advances in data mining: Applications and theoretical aspects~/
 Ed.\ P.~Perner.~---
Lecture notes in computer science ser.~--- Springer, 2015.  
 Vol.~9165. P.~153--165.
 \end{thebibliography}

 }
 }

\end{multicols}

\vspace*{-3pt}

\hfill{\small\textit{Поступила в~редакцию 07.12.17}}

%\vspace*{6pt}

\newpage

\vspace*{-28pt}

%\hrule

%\vspace*{2pt}

%\hrule

%\vspace*{8pt}


\def\tit{A~PROBABILITY MODEL OF~THE~INFLUENCE\\ OF~THE~ORDER BOOK ON~THE~PRICE PROCESS}

\def\titkol{A probability model of the influence of the order book on the price process}

\def\aut{L.\,V.~Nazarov, V.\,V.~Lavrentyev, and~E.\,V.~Bykovets}

\def\autkol{L.\,V.~Nazarov, V.\,V.~Lavrentyev, and~E.\,V.~Bykovets}

\titel{\tit}{\aut}{\autkol}{\titkol}

\vspace*{-9pt}


\noindent
Faculty of Computational Mathematics and Cybernetics, 
M.\,V.~Lomonosov Moscow State University, 1-52~Leninskiye Gory, GSP-1, Moscow 119991, 
Russian Federation 


\def\leftfootline{\small{\textbf{\thepage}
\hfill INFORMATIKA I EE PRIMENENIYA~--- INFORMATICS AND
APPLICATIONS\ \ \ 2018\ \ \ volume~12\ \ \ issue\ 2}
}%
 \def\rightfootline{\small{INFORMATIKA I EE PRIMENENIYA~---
INFORMATICS AND APPLICATIONS\ \ \ 2018\ \ \ volume~12\ \ \ issue\ 2
\hfill \textbf{\thepage}}}

\vspace*{3pt} 
 


\Abste{The Limit Order Book model is considered, with buy and sell orders arriving 
as two independent Cox processes. It includes the price impact model built on the 
basis of a physical model of perfectly elastic collision. Price is treated as 
a~particle of some mass, moving along a~straight line without friction. The 
incoming buy orders and outgoing sell orders hit the price giving it additional 
momentum in one direction, while incoming sell orders and outgoing buy orders do 
the same in the opposite direction. A~functional limit theorem for the price 
process is obtained at a~high intensity 
of incoming order flow, which allows approximation by some L$\acute{\mbox{e}}$vy process}

\KWE{limit orders; perfectly elastic collision; limit order book model; 
price process; Cox process; functional limit theorem}

 
\DOI{10.14357/19922264180205} %

%\vspace*{-14pt}

  %\Ack
  % \noindent
  


%\vspace*{-3pt}

  \begin{multicols}{2}

\renewcommand{\bibname}{\protect\rmfamily References}
%\renewcommand{\bibname}{\large\protect\rm References}

{\small\frenchspacing
 {%\baselineskip=10.8pt
 \addcontentsline{toc}{section}{References}
 \begin{thebibliography}{9}

\bibitem{1-naz}
\Aue{Kukanov, A.} 2013. Stochastic models of limit order markets. 
Columbia University. Ph.D. Thesis.  131~p.

\bibitem{2-naz}
\Aue{Lavrent'ev, V.\,V., and L.\,V.~Nazarov.} 2015. Protsess dvizheniya tseny, 
porozhdennyy nepreryvnoy model'yu knigi zakazov 
[Price process, generated by the continuous model of the order book]. 
\textit{Vestnik Tverskogo gosudarstvennogo un-ta. Ser. 
Prikladnaya matematika} [Bull. of the Tverskoy State University. Ser. 
Appl. Math.] 4:55--63.

\bibitem{3-naz}
\Aue{Korolev, V.\,Yu., A.\,V.~Chertok, A.\,Yu.~Korchagin, and A.\,I.~Zeifman.} 
2015. Modeling high-frequency order flow imbalance by functional limit theorems
 for two-sided risk processes. \textit{Appl. Math. Comput.} 253:224--241.

\bibitem{4-naz}
\Aue{Sivukhin, D.\,V.} 2005. 
\textit{Obshchiy kurs fiziki. Mekhanika}
[General course of physics. Mechanics].
4~ed. Moscow: MIPT Publs. Vol.~1.  560~p. 

\bibitem{5-naz}
\Aue{Kashcheev, D.\,E.} 2001. Modelirovanie dinamiki finansovykh vremennykh ryadov
 i~otsenivanie proizvodnykh tsennykh bumag [Modeling of dynamics of financial time series and 
 estimation of derivative securities].  
 Tver'. PhD Thesis. 191~p.

\bibitem{6-naz}
\Aue{Billingsley, P.} 1977. 
\textit{Convergence of probability measures}. New York, NY: John Wiley \& Sons, Inc. 
277~p.

\bibitem{7-naz}
\Aue{Korolev, V.\,Yu., A.\,V.~Chertok, A.\,Yu.~Korchagin, E.\,V.~Kossova, 
and A.\,I.~Zeifman.} 2016. 
A~note on functional limit theorems for compound Cox processes. 
\textit{J.~Math. Sci.} 218(2):182--194. 

\bibitem{8-naz}
\Aue{Balasanov, Y., A.~Doynikov, V.~Lavrent'ev, and L.~Nazarov}. 
2015. Estimating risk of dynamic trading strategies from high frequency data flow.
\textit{Advances in data mining: Applications and theoretical aspects.} 
Ed.\ P.~Perner.  Lecture notes in computer science ser.  
Springer. 9165:153--165.
\end{thebibliography}

 }
 }

\end{multicols}

\vspace*{-3pt}

\hfill{\small\textit{Received December 7, 2017}}

%\vspace*{-24pt}




\Contr

\noindent
\textbf{Bykovets Eugene V.} (b.\ 1994)~--- MSc student,  
Faculty of Computational Mathematics and Cybernetics, M.\,V.~Lomonosov Moscow 
State University, 1-52~Leninskiye Gory, GSP-1, Moscow 119991, Russian Federation; 
\mbox{eugene.bykovets@stud.cs.msu.su}

\vspace*{3pt}

\noindent
\textbf{Lavrentyev Victor V.} (b.\ 1955)~---  
Candidate of Science (PhD) in physics and mathematics, scientist, 
Faculty of Computational Mathematics and Cybernetics, M.\,V.~Lomonosov Moscow 
State University, 1-52~Leninskiye Gory, GSP-1, Moscow 119991, Russian Federation; 
\mbox{lavrent@cs.msu.ru}

\vspace*{3pt}

\noindent
\textbf{Nazarov Leonid V.} (b.\ 1957)~--- 
Candidate of Science (PhD) in physics and mathematics, senior scientist, 
Faculty of Computational Mathematics and Cybernetics, M.\,V.~Lomonosov Moscow 
State University, 1-52~Leninskiye Gory, GSP-1, Moscow 119991, Russian Federation; 
\mbox{nazarov@cs.msu.ru}
\label{end\stat}


\renewcommand{\bibname}{\protect\rm Литература} 