\def\stat{cont}
{%\hrule\par
%\vskip 7pt % 7pt
\raggedleft\Large \bf%\baselineskip=3.2ex
А\,В\,Т\,О\,Р\,С\,К\,И\,Й\ \ У\,К\,А\,З\,А\,Т\,Е\,Л\,Ь\ \ З\,А\ \ 2\,0\,0\,9 г. \vskip 17pt
    \hrule
    \par
\vskip 21pt plus 6pt minus 3pt }

\label{st\stat}

\def\tit{\ }

\def\aut{\ }
\def\auf{\ }

\def\leftkol{\ } % ENGLISH ABSTRACTS}

\def\rightkol{АВТОРСКИЙ УКАЗАТЕЛЬ ЗА 2009 г.} %ENGLISH ABSTRACTS}

\titele{\tit}{\aut}{\auf}{\leftkol}{\rightkol}

\vspace*{-12pt}

{\tabcolsep=3pt
\begin{tabular}{p{388pt}rr}
&\textbf{Выпуск} & \textbf{Стр.}\\[6pt]
\hangindent=23pt\noindent\textbf{Агаларов~Я.\,М.} Алгоритм вычисления загруженности телекоммуникационной сети\linebreak
\vspace*{-12pt}\\
\hspace*{23pt}с~повторными передачами$\dotfill$&4&22\\
\hangindent=23pt\noindent\textbf{Агаларов~Я.\,М.} Приближенный метод вычисления характеристик узла телекоммуни-\linebreak
\vspace*{-12pt}\\
\hspace*{23pt}кационной сети с повторными передачами$\dotfill$&2&34\\
\hangindent=23pt\noindent\textbf{Артюхов~С.\,В.} Оценки скорости сходимости распределений экстремумов обобщенных\linebreak
\vspace*{-12pt}\\
\hspace*{23pt}процессов Кокса с ненулевым средним к сдвиговым смесям нормальных законов$\dotfill$&1&69\\
\hangindent=23pt\noindent\textbf{Арутюнян~А.\,Р.} см. Ушмаев~О.\,С.&&\\
\hangindent=23pt\noindent\textbf{Бенинг~В.\,Е., Лямин~О.\,О.} О мощности критериев в случае обобщенного распределения\linebreak
\vspace*{-12pt}\\
\hspace*{23pt}Лапласа$\dotfill$&3&79\\
\hangindent=23pt\noindent\textbf{Бородакий~В.\,Ю.} Вероятностная модель обслуживания трафика в системе сетецентри-\linebreak
\vspace*{-12pt}\\
\hspace*{23pt}ческого типа$\dotfill$&3&35\\
\hangindent=23pt\noindent\textbf{Гапонова~М.\,О., Шевцова~И.\,Г.} Асимптотические оценки абсолютной постоянной\linebreak
\vspace*{-12pt}\\
\hspace*{23pt}в~неравенстве Бери--Эссеена для распределений, не имеющих третьего момента$\dotfill$&4&41\\
\hangindent=23pt\noindent\textbf{Горькавый~И.\,Н.} см. Сухомлин~В.\,А.&&\\
\hangindent=23pt\noindent\textbf{Егоров~В.\,Б.} Вопросы реализации объединяющей среды в архитектуре децентрализо-\linebreak
\vspace*{-12pt}\\
\hspace*{23pt}ванной пакетной коммутации$\dotfill$&2&43\\
\hangindent=23pt\noindent\textbf{Егоров~В.\,Б}. Концепция создания отечественных интегрированных коммуникацион\linebreak
\vspace*{-12pt}\\
\hspace*{23pt}ных микроконтроллеров для~пакетной коммутации$\dotfill$&1&34\\
\hangindent=23pt\noindent\textbf{Зацман~И.\,М.} Нестационарная семиотическая модель компьютерного кодирования\linebreak
\vspace*{-12pt}\\
\hspace*{23pt}концептов, информационных объектов и денотатов$\dotfill$&4&87\\
\hangindent=23pt\noindent\textbf{Зацман~И.\,М.} Семиотическая модель взаимосвязей концептов, информационных объ\linebreak
\vspace*{-12pt}\\
\hspace*{23pt}ектов и компьютерных кодов$\dotfill$&2&65\\
\hangindent=23pt\noindent\textbf{Зейфман~А.\,И., Сатин~Я.\,А., Коротышева~А.\,В., Терёшина~Н.\,А.} О предельных характе-\linebreak
\vspace*{-12pt}\\
\hspace*{23pt}ристиках системы обслуживания $M(t)/M(t)/S$ с катастрофами$\dotfill$&3&16\\
\hangindent=23pt\noindent\textbf{Зейфман~А.\,И., Сатин~Я.\,А., Чегодаев~А.\,В.} О нестационарных системах обслуживания\linebreak
\vspace*{-12pt}\\
\hspace*{23pt}с катастрофами$\dotfill$&1&47\\
\hangindent=23pt\noindent\textbf{Кириков~И.\,А., Колесников~А.\,В., Листопад~С.\,В.} Моделирование самоорганизации\linebreak
\vspace*{-12pt}\\
\hspace*{23pt}групп интеллектуальных агентов в зависимости от степени согласованности их\linebreak
\vspace*{-12pt}\\
\hspace*{23pt}взаимодействия$\dotfill$&4&76\\
\hangindent=23pt\noindent\textbf{Козмидиади~В.\,А.} Резервное копирование, использующее снимки$\dotfill$&2&15\\
\hangindent=23pt\noindent\textbf{Колесников~А.\,В.} см. Кириков~И.\,А.&&\\
\hangindent=23pt\noindent\textbf{Королёв~В.\,Ю.} О распределении размеров частиц при дроблении$\dotfill$&3&60\\
\hangindent=23pt\noindent\textbf{Королёв~В.\,Ю.} см. Соколов~И.\,А.&&\\
\hangindent=23pt\noindent\textbf{Коротышева~А.\,В.} см. Зейфман~А.\,И.&&\\
\hangindent=23pt\noindent\textbf{Крылов~А.\,С.} см. Насонов~А.\,В.&&\\
\hangindent=23pt\noindent\textbf{Кудрявцев~А.\,А., Шоргин~В.\,С., Шоргин~С.\,Я.} Байесовские модели массового обслужи-\linebreak
\vspace*{-12pt}\\
\hspace*{23pt}вания и надежности: общий эрланговский случай$\dotfill$&4&30\\
\hangindent=23pt\noindent\textbf{Кудрявцев~А.\,А., Шоргин~С.\,Я.} Байесовские модели массового обслуживания и надеж-\linebreak
\vspace*{-12pt}\\
\hspace*{23pt}ности: экспоненциально-эрланговский случай$\dotfill$&1&55\\
\hangindent=23pt\noindent\textbf{Кучеренко~С.} см. Темнов~Г.&&\\
\hangindent=23pt\noindent\textbf{Листопад~С.\,В.} см. Кириков~И.\,А.&&\\
\hangindent=23pt\noindent\textbf{Лямин~О.\,О.} см. Бенинг~В.\,Е.&&\\
\hangindent=23pt\noindent\textbf{Маркин~А.\,В.} Предельное распределение оценки риска при пороговой обработке\linebreak
\vspace*{-12pt}\\
\hspace*{23pt}вейвлет-коэффициентов$\dotfill$&4&57\\
\hangindent=23pt\noindent\textbf{Морозов~Е.\,В.} Асимптотики вероятностей больших уклонений стационарной очереди$\dotfill$&3&23
\end{tabular}
}

{\tabcolsep=3pt
\begin{tabular}{p{388pt}rr}
&\textbf{Выпуск} & \textbf{Стр.}\\[6pt]
\hangindent=23pt\noindent\textbf{Насонов~А.\,В., Крылов~А.\,С., Ушмаев~О.\,С. } Развитие методов повышения качества\linebreak
\vspace*{-12pt}\\
\hspace*{23pt}изображений лиц в~видеопотоке$\dotfill$&1&19\\
\hangindent=23pt\noindent\textbf{Петрова~О.\,В., Ушаков~В.\,Г.} Асимптотический анализ системы массового обслужива-\linebreak
\vspace*{-12pt}\\
\hspace*{23pt}ния $E_r(t)|G|1$$\dotfill$&4&35\\
\hangindent=23pt\noindent\textbf{Печинкин~А.\,В., Соколов~И.\,А., Чаплыгин~В.\,В.} Многолинейная система массового\linebreak
\vspace*{-12pt}\\
\hspace*{23pt}обслуживания с групповым отказом приборов$\dotfill$&3&4\\
\hangindent=23pt\noindent\textbf{Печинкин~А.\,В., Френкель~С.\,Л.} Вероятностный анализ времени проявления неисправ-\linebreak
\vspace*{-12pt}\\
\hspace*{23pt}ности в сети автоматов$\dotfill$&2&2\\
\hangindent=23pt\noindent\textbf{Сатин~Я.\,А.} см. Зейфман~А.\,И.&&\\
\hangindent=23pt\noindent\textbf{Сатин~Я.\,А.} см. Зейфман~А.\,И.&&\\
\hangindent=23pt\noindent\textbf{Синицын~И.\,Н.} Вероятностные методы построения информационных моделей нерав-\linebreak
\vspace*{-12pt}\\
\hspace*{23pt}номерности вращения Земли$\dotfill$&4&2\\
\hangindent=23pt\noindent\textbf{Синицын~И.\,Н.} Методы построения информационных моделей эредитарных флуктуа-\linebreak
\vspace*{-12pt}\\
\hspace*{23pt}ций неравномерности вращения Земли$\dotfill$&1&2\\
\hangindent=23pt\noindent\textbf{Соколов~И.\,А., Королёв~В.\,Ю.} Предисловие$\dotfill$&3&2\\
\hangindent=23pt\noindent\textbf{Соколов~И.\,А.} см. Печинкин~А.\,В.&&\\
\hangindent=23pt\noindent\textbf{Сухомлин~В.\,А., Горькавый~И.\,Н.} Технологическая система для построения про-\linebreak
\vspace*{-12pt}\\
\hspace*{23pt}граммных комплексов автоматизации обработки трехмерных данных лазерного\linebreak
\vspace*{-12pt}\\
\hspace*{23pt}сканирования$\dotfill$&2&53\\
\hangindent=23pt\noindent\textbf{Темнов~Г., Кучеренко~С.} Подход к актуарному моделированию на основе применения\linebreak
\vspace*{-12pt}\\
\hspace*{23pt}метода квази-Монте-Карло для случайных сумм, зависящих от стохастических\linebreak
\vspace*{-12pt}\\
\hspace*{23pt}факторов$\dotfill$&3&40\\
\hangindent=23pt\noindent\textbf{Терёшина~Н.\,А.} см. Зейфман~А.\,И.&&\\
\hangindent=23pt\noindent\textbf{Торчигин~А.\,В.} Об одном подходе к формированию изображений без использования\linebreak
\vspace*{-12pt}\\
\hspace*{23pt}экрана$\dotfill$&1&60\\
\hangindent=23pt\noindent\textbf{Ушаков~В.\,Г., Шестаков~О.\,В. } Восстановление вероятностных характеристик случай-\linebreak
\vspace*{-12pt}\\
\hspace*{23pt}ных функций в~задачах однофотонной эмиссионной томографии$\dotfill$&1&29\\
\hangindent=23pt\noindent\textbf{Ушаков~В.\,Г.} см. Петрова~О.\,В.&&\\
\hangindent=23pt\noindent\textbf{Ушмаев~О.\,С.} Адаптация биометрической системы к искажающим факторам на при-\linebreak
\vspace*{-12pt}\\
\hspace*{23pt}мере дактилоскопической идентификации$\dotfill$&2&25\\
\hangindent=23pt\noindent\textbf{Ушмаев~О.\,С.} Проблемы распараллеливания биометрических вычислений в~крупно-\linebreak
\vspace*{-12pt}\\
\hspace*{23pt}масштабных идентификационных системах$\dotfill$&1&8\\
\hangindent=23pt\noindent\textbf{Ушмаев~О.\,С., Арутюнян~А.\,Р.} Влияние деформаций на качество биометрической иден-\linebreak
\vspace*{-12pt}\\
\hspace*{23pt}тификации по отпечаткам пальцев$\dotfill$&4&12\\
\hangindent=23pt\noindent\textbf{Ушмаев~О.\,С. } см. Насонов~А.\,В.&&\\
\hangindent=23pt\noindent\textbf{Френкель~С.\,Л.} см. Печинкин~А.\,В.&&\\
\hangindent=23pt\noindent\textbf{Хамдеев~Б.\,И.} см. Чупрунов~А.\,Н.&&\\
\hangindent=23pt\noindent\textbf{Чаплыгин~В.\,В.} см. Печинкин~А.\,В.&&\\
\hangindent=23pt\noindent\textbf{Чегодаев~А.\,В.} см. Зейфман~А.\,И.&&\\
\hangindent=23pt\noindent\textbf{Черников~Б.\,В.} Технология хранения слабоформализуемых документов на основе лек-\linebreak
\vspace*{-12pt}\\
\hspace*{23pt}сикологического синтеза$\dotfill$&4&64\\
\hangindent=23pt\noindent\textbf{Чупрунов~А.\,Н., Хамдеев~Б.\,И.} О вероятности исправления ошибок при помехоустой-\linebreak
\vspace*{-12pt}\\
\hspace*{23pt}чивом кодировании, когда число ошибок принадлежит некоторому конечному\linebreak
\vspace*{-12pt}\\
\hspace*{23pt}множеству$\dotfill$&3&52\\
\hangindent=23pt\noindent\textbf{Шевцова~И.\,Г.} Некоторые оценки для характеристических функций с применением\linebreak
\vspace*{-12pt}\\
\hspace*{23pt}к~уточнению неравенства Мизеса$\dotfill$&3&69\\
\hangindent=23pt\noindent\textbf{Шевцова~И.\,Г.} см. Гапонова~М.\,О.&&\\
\hangindent=23pt\noindent\textbf{Шестаков~О.\,В.} Об устойчивости реконструкции изображений в задачах эмиссионной\linebreak
\vspace*{-12pt}\\
\hspace*{23pt}томографии$\dotfill$&3&47\\
\hangindent=23pt\noindent\textbf{Шестаков~О.\,В. } см. Ушаков~В.\,Г.,&&\\
\hangindent=23pt\noindent\textbf{Шоргин~В.\,С.} см. Кудрявцев~А.\,А.&&\\
\hangindent=23pt\noindent\textbf{Шоргин~С.\,Я.} см. Кудрявцев~А.\,А.&&\\
\hangindent=23pt\noindent\textbf{Шоргин~С.\,Я.} см. Кудрявцев~А.\,А.&&\\
\end{tabular}
}

%\thispagestyle{myheadings}
\def\leftfootline{\small{\textbf{\thepage}
\hfill ИНФОРМАТИКА И ЕЁ ПРИМЕНЕНИЯ\ \ \ том~3\ \ \ выпуск~4\ \ \ 2009}
}%
 \def\rightfootline{\small{ИНФОРМАТИКА И ЕЁ ПРИМЕНЕНИЯ\ \ \ том~3\ \ \ выпуск~4\ \ \ 2009
 \hfill \textbf{\thepage}}}
 \label{end\stat}