\def\stat{ushmaev}


\def\tit{ВЛИЯНИЕ ДЕФОРМАЦИЙ НА КАЧЕСТВО БИОМЕТРИЧЕСКОЙ 
ИДЕНТИФИКАЦИИ ПО ОТПЕЧАТКАМ ПАЛЬЦЕВ$^*$}
\def\titkol{Влияние деформаций на качество биометрической 
идентификации по отпечаткам пальцев} 

\def\autkol{О.\,С.~Ушмаев,  А.\,Р.~Арутюнян}
\def\aut{О.\,С.~Ушмаев$^1$,  А.\,Р.~Арутюнян$^2$}

\titel{\tit}{\aut}{\autkol}{\titkol}

{\renewcommand{\thefootnote}{\fnsymbol{footnote}}\footnotetext[1]
{Работа поддержана грантами РФФИ (проект~07-07-00031) и Программой ОНИТ РАН <<Информационные технологии
и методы анализа сложных систем>>. Работа выполнена в рамках НОЦ ИПИ РАН\,--\,ВМК МГУ <<Биометрическая информатика>>.
}}

\renewcommand{\thefootnote}{\arabic{footnote}}
\footnotetext[1]{Институт проблем информатики Российской академии наук, oushmaev@ipiran.ru}
\footnotetext[2]{Институт безопасного развития атомной энергетики Российской академии наук, artem@ac.ibrae.ru}

\vspace*{-8pt}

\Abst{В статье проведен статистический анализ влияния упругих деформаций на качество 
распознавания по отпечаткам пальцев. Для анализа влияния деформаций используется 
механическая модель деформаций изображений отпечатков пальцев. Приведен численный 
метод решения и качественный анализ вычисленных деформаций отпечатков пальцев. 
Статистический анализ проводился на общедоступных дактилоскопических массивах.}

\KW{дактилоскопическая идентификация; нелинейные деформации отпечатков пальцев}

 \vskip 18pt plus 9pt minus 6pt

      \thispagestyle{headings}

      \begin{multicols}{2}

      \label{st\stat}

     \section*{Введение}
      
      На сегодняшний день основными факторами, негативно влияющими на качество 
автома-\linebreak тической дактилоскопической идентификации,\linebreak являются шумы, малые области 
пересечения отпечатков пальцев и нелинейные деформации отпечатков~[1--4].  Малые 
области пересечения отпечатков являются скорее субъективным фактором, который 
устраняется путем обучения пользователей или\linebreak операторов (для крупномасштабных 
автоматизированных идентификационных систем).
      
      Для работы с зашумленными отпечатками используются различные методы и 
алгоритмы обработки изображений, инвариантные к этому негативному фактору. При 
этом использование схожего подхода для работы с деформированными отпечатками 
затруднено. На сегодняшний день абсолютное большинство технологий распознавания 
отпечатков не являются инвариантными к деформациям.
      
      Присутствие деформаций обусловлено спецификой процесса сканирования отпечатков. 
Изначально трехмерный объект в области контакта со сканером становится плоским, а его 
поверхность, которая определяет конкретное изображение отпечатка пальцев, оказывается 
заведомо деформированной (рис.~1). В~каж\-дом приложении деформации могут 
быть различными из-за разницы в трении, силе и направления прижатия, различной 
начальной точки контакта.

      Целью настоящей статьи является оценка влияния деформаций на качество 
распознавания отпечатков пальцев. 
     
      В литературе известны следующие основные подходы к анализу деформаций. 
В~\cite{5ushm} пред\-став\-лен метод измерения деформации, основанный на так 
называемых <<резиновых масках>>. В~\cite{6ushm} применяются методы thin-plate splines 
(TPS)~\cite{7ushm} к моделированию деформаций отпечатков пальцев. Однако 
результаты были в целом неудовлетворительны, так как по своим механическим 
свойствам пальцы сильно отличаются от металлической пластины. В~\cite{8ushm, 9ushm} 
предложен более мягкий аналог TPS. В~\cite{10ushm} предложен систематический метод 
моделирования деформаций, основывающийся на эмпирическом разложении изображения 
отпечатка на неподвижную и подвижную зоны. 

\begin{center} %fig1
%\vspace*{6pt}
\mbox{%
\epsfxsize=75mm %78.39mm
\epsfbox{ush-1.eps}
}
%\end{center}
\vspace*{6pt}

{{\figurename~1}\ \ \small{Возникновение деформации отпечатков пальцев}}
\end{center}
\vspace*{-6pt}


\bigskip
\addtocounter{figure}{1}

Однако перечисленные подходы не 
позволяют установить влияние деформаций на качество распознавания, выраженное 
соотношением ошибок 1-го и 2-го рода. Это связано с тем, что данные работы направлены 
на априорное моделирование деформаций и в большинстве случаев не позволяют 
апостериорно установить, какая именно деформация привела к тому или иному 
искажению изображения (рис.~\ref{f2ushm}).
      
\begin{figure*} %fig2
\vspace*{1pt}
\begin{center}
\mbox{%
\epsfxsize=165.556mm
\epsfbox{ush-2.eps}
}
\end{center}
\vspace*{-9pt}
\Caption{Пример деформации изображения (стрелкой указано каноническое направление 
центра, линия соединяет центр и дельту)
\label{f2ushm}}
\end{figure*}
      
      Далее статья организована следующим образом. В~разд.~1 приведена 
математическая модель деформаций отпечатков пальцев. Раздел~2 посвящен 
качественному анализу деформаций. В~разд.~3 приведены результаты экспериментов.
В разд.~4 исследовано влияние деформаций на прямое наложение изображений отпечатков пальцев.
      
     \section{Математическая модель упругих деформаций}
      
      Для оценки влияния деформаций используется механическая модель динамики 
деформации отпечатков пальцев, предложенная в~\cite{11ushm}.
      
      Если рассматривать отпечаток пальца как упругий объект, то динамика его 
деформаций достаточно точно описывается следующим уравнением~\cite{12ushm}:
      \begin{equation}
      L\mathbf{u} (x,y,z,t) =-\mathbf{f}(x,y,z)\,,
      \label{e1ushm}
      \end{equation}
где $L$~--- дифференциальный оператор
$$
L=\mu\nabla^2+(\lambda+\mu)\nabla(\nabla\cdot)-\rho\fr{\partial^2}{\partial t^2}\,;
$$
\textbf{u}~--- вектор смещений; \textbf{f}~--- внешняя сила. Коэффициенты 
Ламе~$\lambda$ и~$\mu$ могут быть выражены через модуль Юнга~$E$ и коэффициент 
Пуассона~$\nu$ по следующим формулам:
$$
 E=\fr{\mu (3\lambda +2\mu)}{\lambda+\mu}\,;\enskip \nu =\fr{\lambda}{2(\lambda+\mu)}\,.
$$
      
      Вектор смещений задает отображение из одного состояния отпечатка в другое. 
Формально~\textbf{u} является отображением изображения на изображение.
      
      Современные технологии сканирования отпечатка пальцев обычно фиксируют 
изображения в тот момент, когда палец неподвижен, т.\,е.\ действующие силы уже 
уравновесили напряжение, вызванное смещениями. Условиями окончания деформации 
является равенство нулю первых частных\linebreak производных смещений по времени. В~таком 
случае уравнение~(\ref{e1ushm}) приводится к следующему виду:
      \begin{equation}
      \mu\nabla^2\mathbf{u}+\left(\lambda+\mu\right)\nabla 
\left(\nabla\mathbf{u}\right)+\mathbf{f}=0\,.
      \label{e2ushm}
      \end{equation}
      
      Решение уравнения~(\ref{e2ushm}) определяет деформацию отпечатка. Однако 
проблема решения уравнения~(\ref{e2ushm}) поставлена некорректно. Во-первых, не 
существует недеформированного состояния отпечатка, так как он изначально является 
следствием деформации трехмерного объекта в процессе сканирования 
(см.\ рис.~1). Во-вторых, действующие силы неизвестны. При этом не существует 
легко реализуемого способа их измерения. 
      
      В этой связи целесообразно рассматривать задачу апостериорного вычисления 
относительных деформаций отпечатков пальцев. А~именно, если не существует 
недеформированного отпечатка и абсолютной деформации, предлагается решать задачу 
определения относительной деформации двух отпечатков~$FI_1$ и~$FI_2$. А именно 
задачу вычисления карты смещений~\textbf{u}: $FI_1\rightarrow FI_2$.

\begin{figure*} %fig3
\vspace*{1pt}
\begin{center}
\mbox{%
\epsfxsize=125.086mm
\epsfbox{ush-3.eps}
}
\end{center}
\vspace*{-9pt}
\Caption{Соответствие контрольных точек отпечатков пальцев
\label{f3ushm}}
\vspace*{12pt}
\end{figure*}
      
      Известно, что решение уравнения~(\ref{e2ushm}) минимизирует энергию, которая 
вычисляется как
      \begin{multline}
      E_0=-A+E_d=-\int\limits_S \left ( uf_x+vf_y\right )\,dS+{}\\
      {}+\fr{1}{2}\int\limits_S \left ( 
\varepsilon_1^1\sigma_1^1+\varepsilon_2^2\sigma_2^2+\varepsilon_1^2\sigma_1^2\right )\,dS\,,
      \label{e3ushm}
      \end{multline}
где $f_x$ и $f_y$~--- соответствующие компоненты вектора действующих сил~\textbf{f}; 
$u$ и~$v$~--- компоненты вектора смещений $\mathbf{u} =(u,v)$; $\varepsilon_j^i$ 
и~$\sigma_j^i$~--- компоненты тензоров натяжений и давлений, вычисляемых по 
следующим формулам:
\pagebreak

\noindent
\begin{equation*}
\varepsilon  = 
\begin{pmatrix}
\fr{\partial u}{\partial x} & \fr{1}{2}\left ( \fr{\partial u}{\partial y}+\fr{\partial v}{\partial 
x}\right )\\[12pt]
\fr{1}{2}\left ( \fr{\partial u}{\partial y}+\fr{\partial v}{\partial x}\right ) & \fr{\partial 
v}{\partial y}
\end{pmatrix}\,;
\end{equation*}
\begin{multline*}
\sigma  ={} \\[6pt]
\!{}=\!
\begin{pmatrix}
\!\fr{E(1-v)}{(1+v)(1-2v)}\left ( \varepsilon_1^1+\varepsilon_1^2\right )\!\!\!\! \!\!& \!\!\!\!\!\!
\fr{E}{2(1+v)}\,\varepsilon_1^2\!\\[12pt]
\!\fr{E}{2(1+v)}\,\varepsilon_2^1 \!\! \!\!\!\!\!&\!\!\!\!\!\! \fr{E(1-v)}{(1+v)(1-2v)}\left ( 
\varepsilon_2^2+\varepsilon_2^1\right )\!\!
\end{pmatrix}\hspace*{-9.65pt}
\end{multline*}

\medskip
      
      Часть энергии, связанная с действующими силами, не поддается непосредственной 
оценке. При этом для двух произвольных изображений одного отпечатка пальца можно 
найти соответствующие друг другу контрольные точки изображения (рис.~\ref{f3ushm}), 
что потенциально дает информацию об общем направлении действующих сил. 
      
      
      Предположим, что соответствующие наборы контрольных точек заданы и 
обозначены через~$\{\mathbf{p}_i\}$ и~$\{\mathbf{q}_i\}$ соответственно. Тогда в 
качестве меры близости двух наборов контрольных точек можно рассматривать 
следующий квадратичный функционал невязки:
      \begin{equation}
      D\left( \mathbf{u}\right) = \sum \left ( \mathbf{p}_i+\mathbf{u}\left (\mathbf{p}_i\right 
) -\mathbf{q}_i\right )^2\,,
      \label{e4ushm}
      \end{equation}
      
      Качественно функционал~(\ref{e4ushm}) соотносится с работой действующих сил 
функционала энергии~(\ref{e3ushm}). Действительно, если произошло смещение 
контрольной точки, то логично предположить, что в направлении смещения действуют 
внешние силы. Поэтому рассмотрим задачу минимизации следующего функционала, 
состоящего из собственной энергии деформации и регуляризирующей 
поправки~(\ref{e4ushm}) на невязку:
      \begin{equation}
      W(\mathbf{u})=E_d(\mathbf{u}) +\alpha S(\mathbf{u})\,,
      \label{e5ushm}
      \end{equation}
где $\alpha$~--- весовой коэффициент. 

      Субъективно наилучшим решением является деформация, при которой 
контрольные точки совмещаются, т.\,е.\ функционал~(\ref{e4ushm}) обнуляется и, как 
следствие, находится в точке абсолютного минимума. Однако скорее всего такая точка не 
будет экстремумом~(\ref{e5ushm}) из-за возросшей энергии деформации. Таким образом, 
минимизация регуляризированного функционала дает не совсем оптимальное решение с 
субъективной точки зрения. Весовой коэффициент~$\alpha$ регуляризирующей поправки 
определяет толерантность метода решения к предельной деформации.
      
      Численное решение функционала~(\ref{e5ushm}) может быть найдено методом 
конечных элементов. Для объективной оценки качества деформаций может быть 
использовано прямое наложение изоб\-ра\-же\-ний (с компенсацией деформаций и с 
компенсацией только движения). Примеры приведены на рис.~\ref{f4ushm}. Контрастная 
часть результатов наложения указывает на резонанс папиллярных линий. Серая часть 
соответствует неточному наведению.

\end{multicols}
      
      \begin{figure} %fig4
      \vspace*{1pt}
\begin{center}
\mbox{%
\epsfxsize=163.871mm
\epsfbox{ush-4.eps}
}
\end{center}
\vspace*{-9pt}
      \Caption{Пример вычисления деформаций: (\textit{а}) и~(\textit{б})~два изображения 
отпечатка; (\textit{в})~прямое наложение при подгонке жестким движением; 
(\textit{г})~прямое наложение после подгонки деформации
      \label{f4ushm}}
\vspace*{6pt}
      \end{figure}
      
\begin{multicols}{2}

     \section{Структура деформаций}
      
      Для определения структуры деформаций проведем анализ численных решений 
уравнения~(\ref{e2ushm}). Для анализа использовались три базы данных 
FVC2002~\cite{13ushm}, содержащие изображения с оптических и емкостных сканеров 
отпечатков пальцев. Каждая база содержит 8~образцов для 100~человек, т.\,е.\ для анализа 
деформаций всего доступно три набора по 2800~естественных деформаций отпечатков. 
Деформация является векторным полем на плоскости, т.\,е.\ характеризуется двумя 
числами в узлах решетки дискретизации. 
      
      Основное уравнение~(\ref{e2ushm}) является линейным, поэтому над его 
решениями, т.\,е.\ деформациями, определены операции сложения и умножения на число. 
Также для деформаций можно определить операцию проекции друг на друга. В этой связи 
целесообразно изучать структуру деформаций путем проекции на основные деформации. 
      
      Основные деформации могут вычисляться как из качественных 
соображений~\cite{10ushm} так и путем фор-\linebreak\vspace*{-12pt}
\pagebreak

\end{multicols}

\begin{figure} %fig5
\vspace*{1pt}
\begin{center}
\mbox{%
\epsfxsize=98.317mm
\epsfbox{ush-5.eps}
}
\end{center}
\vspace*{-9pt}
\Caption{Накопленная доля дисперсии главных компонент деформаций: \textit{1}~--- 
DB1; \textit{2}~--- DB2; \textit{3}~--- DB4
\label{f5ushm}}
\end{figure}
%\vspace*{6pt}
\begin{figure} %fig6
\vspace*{1pt}
\begin{center}
\mbox{%
\epsfxsize=162.582mm
\epsfbox{ush-6.eps}
}
\end{center}
\vspace*{-9pt}
\Caption{Карта смещений главных компонент деформаций. Верхний ряд~--- база FVC2002 
DB1, нижний ряд~--- FVC2002 DB2: (\textit{a})~вращение; (\textit{б})~кручение; 
(\textit{в})~поперечное смещение; (\textit{г})~продольное смещение
\label{f6ushm}}
%\vspace*{-6pt}
\end{figure}

\begin{multicols}{2}

\noindent
мального статистического 
анализа~\cite{14ushm}. В~настоящей статье будем исследовать характер деформаций 
методом разложения деформаций в пространстве главных компонент в частотной области. 

\begin{figure*} %fig7
\vspace*{1pt}
\begin{center}
\mbox{%
\epsfxsize=160.357mm
\epsfbox{ush-7.eps}
}
\end{center}
\vspace*{-9pt}
\Caption{Схематичное представления основных деформаций, <<Н>>~--- 
недеформированная область, стрелкой указаны основные смещения: (\textit{a})~вращение; 
(\textit{б})~кручение; (\textit{в})~поперечное смещение; (\textit{г})~продольное смещение
\label{f7ushm}}
\end{figure*}

На рис.~\ref{f5ushm} представлены данные по накопленной доли дис\-пер\-сии главных 
компонент. Из рисунка видно, что первые десять главных деформаций прак\-ти\-чески без 
потерь приближают полученные численные решения для дефор\-маций.
{\looseness=1

}
      
      На рис.~\ref{f6ushm} представлены первые четыре главные деформации. Они 
допускают физическую интерпретацию. А~именно, это вращение, кручение и два 
сдвига~--- продольный и поперечный. Соответствующие движения самого пальца 
схематично представлены на рис.~\ref{f7ushm}. Такие деформации отвечают 
субъективному представлению об основных способах приложения пальца к сканеру.


\begin{figure*} %fig8
\vspace*{1pt}
\begin{center}
\mbox{%
\epsfxsize=163.878mm
\epsfbox{ush-8.eps}
}
\end{center}
\vspace*{-9pt}
\Caption{Пример вычисления деформации в чужом сравнении: 
(\textit{а}) и~(\textit{б})~изображения двух разных отпечатков (темными точками отмечены кластеры 
попарно соответствующих контрольных точек); (\textit{в})~прямое наложение при 
подгонке жестким движением; (\textit{г})~прямое наложение после подгонки деформации
\label{f8ushm}}
\end{figure*}
      \begin{figure*} %fig9 
      \vspace*{1pt}
\begin{center}
\mbox{%
\epsfxsize=162.137mm
\epsfbox{ush-9.eps}
}
\end{center}
\vspace*{-9pt}
      \Caption{Изменение гистограмм распределений меры сходства в <<своих>> (слева) 
и <<чужих>> (справа) сравнениях с компенсацией деформации~(\textit{1}) и без 
компенсации деформации~(\textit{2}): (\textit{а})~FVC2002 DB1; (\textit{б})~FVC2002 
DB3
      \label{f9ushm}}
      \end{figure*}

     \section{Влияние на качество распознавания}
      
      Для анализа влияния на качество деформаций проведем следующие три 
эксперимента. В первом проведем эксперименты по оценке качества распознавания 
алгоритмов BioLink MST~\cite{15ushm} для деформированных и недеформированных 
(после компенсации деформации методами разд.~1) отпечатков. Во втором и третьем 
рассмотрим изменение статистики двух основных объективных мер сходства отпечатков: 
среднего расстояния между контрольными точками и коэффициента прямого наложения.

      
      Деформации можно корректно определить только для <<своих>>\footnote{Когда 
два предъявленных для сравнения изображения являются изображениями отпечатков 
одного и того же пальца.} сравнений, а именно можно сказать, что два изображения 
отпечатка одного и того же пальца могут быть трансформированы друг в друга картой 
смещений какой-то деформации. Но формально вычислительная схема разд.~1 применима 
в любых случаях, когда есть попарное соответствие контрольных точек, что вполне 
определено и для <<чужих>>\footnote{Когда два предъявленных для сравнения 
изображения являются изображениями отпечатков разных пальцев.} сравнений. Пример 
применения компенсации деформаций в <<чужом>> сравнении приведен на 
рис.~\ref{f8ushm}. 

      
      Рассмотрим результаты экспериментов по сравнению качества распознавания 
\mbox{BioLink} MST на база FVC2002. На рис.~\ref{f9ushm} представлены гистограммы 
распределений меры сходства в <<своих>> и <<чужих>> сравнениях для экспериментов с 
компенсаций и без компенсации деформаций. Как видно из рисунков, наличие 
деформации приводит к сдвигу распределения в своих сравнениях в сторону меньших 
значений меры сходства, что подтверждает негативное влияние деформаций на качество 
распознавания.
   

      Для численной оценки негативного эффекта применим метод нормальной 
аппроксимации искажающих факторов~\cite{16ushm, 17ushm}, который предполагает 
разделение дис\-пер\-сии меры сходства на дис\-пер\-сию отдельных искажающих факторов. На 
рис.~\ref{f10ushm} показаны нормальные аппроксимации гистограмм рис.~\ref{f9ushm}. 
Умеренные деформации (DB1) дают примерно 10-процентный вклад в дисперсию меры сходства в 
<<своих>> сравнениях и 5-процентный сдвиг среднего. При этом деформации ожидаемо не влияют 
на распределения в <<чужих>> сравнения. При этом даже такие, на первый взгляд, 
незначительные изменения статистических параметров приводят к значительному 
изменению качества распознавания (рис.~\ref{f11ushm}). На рис.~\ref{f11ushm} также 
приведены прогнозы ошибок при 50-процентном увеличении деформаций (эксперимент проводился 
путем применения к 
изображению отпечатка вычисленного поля на\-прав\-ле\-ния деформации, умноженного на 1,5). 

\end{multicols}

\begin{figure} %fig10
\vspace*{1pt}
\begin{center}
\mbox{%
\epsfxsize=165.452mm
\epsfbox{ush-10.eps}
}
\end{center}
\vspace*{-9pt}
\Caption{Нормальная аппроксимация мер сходства (с компенсацией 
деформации~(\textit{1}); без компенсации деформации~(\textit{2});  чужие сравнения~(\textit{3})): 
(\textit{а})~FVC2002DB1; (\textit{б})~FVC2002DB3
\label{f10ushm}}
\end{figure}
\begin{figure} %fig11
\vspace*{1pt}
\begin{center}
\mbox{%
\epsfxsize=165.498mm
\epsfbox{ush-11.eps}
}
\end{center}
\vspace*{-9pt}
\Caption{Качество распознавания (без деформаций~(\textit{1}); с 
деформациями~(\textit{2});  с дополнительной деформацией~(\textit{3})): 
(\textit{а})~FVC2002 DB1; (\textit{б})~FVC2002 DB3
\label{f11ushm}}
\vspace*{6pt}
\end{figure}

            \begin{table}[b]\small
\vspace*{-3pt}
      \begin{center}
      \Caption{Изменение мер сходства изображений после устранения деформаций
      \label{t1ushm}}
      \vspace*{2ex}
      
      \begin{tabular}{|c|c|c|c|c|c|c|}
      \hline
Тестовая база&\multicolumn{2}{c|}{
\tabcolsep=0pt\begin{tabular}{c}Изменение\\ среднего расстояния\\ между контрольными\\ 
точками (пикселов)\end{tabular}}&
\multicolumn{2}{c|}{\tabcolsep=0pt\begin{tabular}{c}Коэффициент\\ прямого наложения\\ 
(относительное\\ изменение, \%)\end{tabular}}&
\multicolumn{2}{c|}{\tabcolsep=0pt\begin{tabular}{c}Коэффициент\\ прямого наложения \\
(абсолютное\\ изменение)\end{tabular}}\\
\cline{2-7}
&свои&чужие&свои&чужие&свои&чужие\\
&\multicolumn{1}{p{40pt}|}{\hspace*{40pt}}
&\multicolumn{1}{p{45pt}|}{\hspace*{45pt}}
&\multicolumn{1}{p{40pt}|}{\hspace*{40pt}}
&\multicolumn{1}{p{45pt}|}{\hspace*{45pt}}
&\multicolumn{1}{p{40pt}|}{\hspace*{40pt}}
&\multicolumn{1}{p{45pt}|}{\hspace*{45pt}}
\\[-11pt]
\hline
FVC2002DB1&$- 2{,}4$&$-7{,}6$&$+36$&$+16$&$\mathbf{+5{,}8}$&$\mathbf{+0{,}5}$\\
FVC2002DB3&$-2{,}2$&$-3{,}4$&$+24$&$+14$&$\mathbf{+3{,}7}$&$\mathbf{+0{,}6}$\\
\hline
\end{tabular}
\end{center}
\end{table}      
\begin{multicols}{2}



     \section{Влияние на естественные меры сходства изображений}

      Как было отмечено в разд.~3, естественными мерами сходства двух изображений 
отпечатков пальцев является среднее расстояние между контрольными точками и 
изменение коэффициента прямого наложения. В~табл.~\ref{t1ushm} представлены 
результаты измерений на двух тестовых базах FVC2002.
      

\begin{figure*} %fig12
\vspace*{1pt}
\begin{center}
\mbox{%
\epsfxsize=165.122mm
\epsfbox{ush-12.eps}
}
\end{center}
\vspace*{-9pt}
\Caption{Качество распознавания с учетом деформаций~(\textit{1}) и без учета 
деформаций~(\textit{2}): (\textit{а})~FVC2002 DB1; (\textit{б})~FVC2002 DB3
\label{f12ushm}}
\end{figure*}
      Как видно из табл.~\ref{t1ushm}, компенсация деформаций значительно влияет на 
абсолютное значения коэффициента прямого наложения. Расстояние между 
контрольными точками уменьшается и для своих, и для чужих сравнений в силу характера 
регуляризирующей поправки~(\ref{e4ushm}), которая имеет тенденцию к уменьшению 
при вычислении деформаций. При этом эффект больше для чужих сравнений.
      
      Для более детальной оценки влияния деформации на прямое наложение 
рассмотрим качество распознавания методом прямого наложения изоб\-ра\-же\-ний. 
Результаты экспериментов приведены на рис.~\ref{f12ushm}. Как видно из рисунков, даже 
при улучшении наложения чужих отпечатков (за счет лучшей подгонки 
малоинформативной дистальной зоны), деформации значительно увеличивают степень 
корреляции изображений. 
 
     \section*{Заключение}
      
      В статье представлены результаты статистического анализа влияния деформаций 
на качество распознавания по отпечаткам пальцев. Полученные результаты позволяют 
сделать заключение о степени влияния деформаций отпечатков на качество 
дактилоскопической идентификации. На\-прав\-ле\-ни\-ем дальнейших исследований является 
более детальный разбор структуры деформаций и создание технологии синтеза 
искусственных деформаций отпечатков пальцев.

{\small\frenchspacing
{%\baselineskip=10.8pt
\addcontentsline{toc}{section}{Литература}
\begin{thebibliography}{99}    
\bibitem{1ushm}
\Au{Eleccion M.}
Automatic fingerprint identification~// IEEE Spectrum, 1973. Vol.~10. P.~36--45.

\bibitem{4ushm} %2
\Au{Halici U., Jain~L.\,C., Erol~A.}
Introduction to fingerprint recognition, intelligent 
biometric techniques in fingerprint and face recognition.~--- CRC Press, 1999.

\bibitem{2ushm} %3
\Au{Maltoni~D., Maio~D., Jain~A.\,K., Prabhakar~S.} 
Handbook of fingerprint recognition.~--- New-York: Springer-Verlag, 2003.

\bibitem{3ushm} %4
\Au{Wilson C.\,L., Watson C.\,I., Garris~M.\,D., Hicklin~A.} 
Studies of fingerprint matching using the NIST verification test bed (VTB). 
National  Institute of Standards and Technology (NIST). Technical Report NISTIR 7020, 2004.


\bibitem{5ushm}
\Au{Burr D.\,J.}
A dynamic model for image registration~// Computer Graphics Image 
Processing, 1981. Vol.~15. P.~102--112.

\bibitem{6ushm}
\Au{Bazen A.\,M., Gerez~S.\,H.}
Thin-plate spline modelling of elastic deformation in 
fingerprints~// 3rd IEEE Benelux Signal Processing Symposium Proceedings, 2002.

\bibitem{7ushm}
\Au{Bookstein F.\,L.}
 Principal warps: Thin-plate splines and the decomposition of  
deformations~// IEEE Trans. PAMI, 1989. Vol.~11. No.\,6. P.~567--585.

\bibitem{8ushm}
\Au{Fornefett M., Rohr~K., Stiehl~H.\,S.} 
Elastic medical image registration using 
surface landmarks with automatic finding of correspondences~// 
Proc. Workshop Bildverarbeitung fur die Medizinl, Informatik actuell~/ Eds. A.~Horsch and T.~Lehmann.~--- 
Munchen, Germany: Springer-Verlag Berlin Heidelberg, 2000. P.~48--52.

\bibitem{9ushm}
\Au{Fornefett M., Rohr K., Stiehl~H.\,S.}
Radial basis functions with compact support for 
elastic registration of medical images~// Image Vision Computing, 2001. Vol.~19. No.\,1--2.
P.~87--96.

\bibitem{10ushm}
\Au{Cappelli~R., Maio~D., Maltoni~D.} 
Modelling plastic distortion in 
fingerprint images~// ICAPR2001, 2001. P.~369--376.

\bibitem{11ushm}
\Au{Ushmaev O., Novikov~S.}
Registration of elastic deformations of fingerprint images 
with automatic finding of correspondences~// Workshop on Multi Modal User 
Authentication (MMUA03) Proceedings, 2003. P.~196--201.

\bibitem{12ushm}
\Au{Ландау Л.\,Д., Лифшиц~Е.\,М.}
Теоретическая физика. Том~VII. Теория упругости.~--- 
М.: Физматлит, 2003.  264~с.

\bibitem{13ushm}
\Au{Maio~D., Maltoni~D., Cappelli~R., Wayman~J.\,L., Jain~A.\,K.} 
FVC2002: Second 
fingerprint verification competition~// 16th  Conference  (International)
on Pattern Recognition Proceedings, 2002.  Vol.~3. P.~811--814.

\bibitem{14ushm}
\Au{Ушмаев О.\,С.} 
Статистическая модель деформаций отпечатков пальцев~// Докл. 
13-й Всероссийской конференции <<Математические методы распознавания 
образов. ММРО-13>>. Ленинградская обл., Зеленогорск, 30~сентября\,--\,06~октября 
2007.~--- М.: МАКС ПРЕСС, 2007. С.~406--408.

\bibitem{15ushm}
\Au{Wilson~C., Hicklin~R.\,A., Korves~H., Ulery~B., Zoepfl~M., Bone~M., Grother~P.,  
Micheals~R., Otto~S., Watson~C.}
Fingerprint vendor technology evaluation 2003: 
Summary of results and analysis report. National Institute of Standards and Technology 
(NIST). Technical Report, 2004.

\bibitem{16ushm}
\Au{Ушмаев~О., Арутюнян~А.}
Метод оценки качества биометрической идентификации 
в операционных условиях на примере дактилоскопической идентификации~// 
ГрафиКон'2009: 19-я Международная конференция по компьютерной графике и 
зрению. Москва, МГУ им.\ М.\,В.~Ломоносова, 5--9~октября 2009: Труды 
конференции.~--- М.: МАКС ПРЕСС, 2009. С.~232--235.

\label{end\stat}

\bibitem{17ushm}
\Au{Ушмаев О.\,С.} 
Адаптация биометрической системы к искажающим факторам на 
примере дактилоскопической идентификации~// Информатика и её применения, 
2009. Т.~3. Вып.~2. С.~25--33.
 \end{thebibliography}
}
}
\end{multicols}