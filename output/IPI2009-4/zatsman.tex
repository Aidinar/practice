\def\stat{zatsman}

\def\tit{НЕСТАЦИОНАРНАЯ СЕМИОТИЧЕСКАЯ МОДЕЛЬ КОМПЬЮТЕРНОГО 
КОДИРОВАНИЯ КОНЦЕПТОВ, ИНФОРМАЦИОННЫХ ОБЪЕКТОВ 
И~ДЕНОТАТОВ$^*$}
\def\titkol{Нестационарная семиотическая модель компьютерного 
кодирования} 


\def\autkol{И.\,М.~Зацман}
\def\aut{И.\,М.~Зацман$^1$}

\titel{\tit}{\aut}{\autkol}{\titkol}

{\renewcommand{\thefootnote}{\fnsymbol{footnote}}\footnotetext[1]
{Работа выполнена при поддержке РФФИ, грант №\,09-07-00156.}}

\renewcommand{\thefootnote}{\arabic{footnote}}
\footnotetext[1]{Институт проблем информатики Российской академии наук, iz\_ipi@a170.ipi.ac.ru}


\Abst{Рассмотрена нестационарная семиотическая модель, которая была разработана в 
процессе исследования проблем целенаправленной генерации и эволюции целевых систем 
знаний, а также отображения в электронных библиотеках и других видах информационных 
систем процессов их эволюции во времени и множестве знаковых систем. Эти проблемы 
относятся к новому направлению исследований, получившему название <<когнитивная 
информатика>>. Предлагаемая нестационарная семиотическая модель позиционируется 
как теоретическая основа компьютерного кодирования концептов, информационных объектов и 
денотатов с учетом их эволюции во времени.}

\KW{семиотическая модель; целевые системы знаний; денотаты; концепты; информационные 
объекты; компьютерные коды; трехкомпонентная кодировка; описание траектории эволюции 
концептов}

 \vskip 18pt plus 9pt minus 6pt

      \thispagestyle{headings}

      \begin{multicols}{2}

      \label{st\stat}
      
\section{Введение}

   В работах~\cite{1zat, 2zat} описана новая область ис\-следо\-ва\-ний, которые ставят 
своей целью разработку\linebreak теоре\-ти\-че\-ских основ создания ин\-фор\-ма\-ци\-он\-но-ком\-му\-ни\-ка\-ци\-он\-ных 
технологий (ИКТ), обеспечивающих процессы \textit{целенаправленного формирования} 
новых систем знаний. Такие системы знаний предлагается называть \textit{целевыми}. 
Проблематика целевых систем знаний позиционируется как междисциплинарная область 
исследований, но постановка ряда актуальных проблем позволяет отнести их в первую 
очередь к информатике как ин\-фор\-ма\-ци\-он\-но-компью\-тер\-ной науке~[1--3]. В~рамках 
информатики эта новая область исследований относится к когнитивной 
информатике\footnote[2]{Когнитивная информатика~--- направление в информатике как 
   ин\-фор\-ма\-ци\-он\-но-компью\-тер\-ной науке, которое при исследовании вычислительных процессов и разработке 
ин\-фор\-ма\-ци\-он\-но-компью\-тер\-ных систем использует методы когнитивной науки, изучающей ментальные 
процессы (познавательные и креативные) и ментальные объекты (концепты), а при исследовании форм 
представления концептов, их эволюции, познавательных и креативных процессов использует методы 
информационной и компьютерной наук~[2--6].}, которая находится в начальной стадии описания 
ее предметной области и составляющих ее проблем~[4--6].
   
   
   Целью разработки теоретических основ создания ИКТ, которые должны обеспечивать 
процессы формирования целевых систем знаний (далее по тексту~--- ЦСЗ), является решение 
как минимум трех актуальных проблем: (1)~идентификации концептов ЦСЗ; (2)~оценивания 
релевантности ЦСЗ технологическим или другим общественно значимым потребностям, 
ради удовлетворения которых формируется ЦСЗ; (3)~направляемого развития ЦСЗ.
   
   В интересах постановки, исследования и решения этих проблем, включая исследование 
процессов генерации и эволюции ЦСЗ, в работе~\cite{2zat} была предпринята попытка 
описать концепты, соответствующие им информационные объекты и денотаты с помощью 
цифровых (компьютерных) кодов в рамках институциональной информационной сис\-те\-мы 
(ИИС). Характерные черты ИИС описаны в работе ~\cite{7zat}.

\begin{figure*} %fig1
\begin{center}
\vspace*{1pt}
\mbox{%
\epsfxsize=153.251mm
\epsfbox{zat-1.eps}
}
\end{center}
\vspace*{-3pt}
\Caption{Стационарная семиотическая модель компьютерного кодирования концептов, 
информационных объектов и денотатов~\cite{2zat}
   \label{f1zat}}
   \vspace*{6pt}
   \end{figure*}
   
   Отметим следующие существенные для данной статьи связи (отношения) ИИС с 
ментальной сферой, материальной сферой физических объектов и явлений, отражаемых в 
ИИС, социально-ком\-му\-ни\-ка\-ци\-он\-ной и цифровой средами (см.\ рис.~\ref{f1zat}, на котором 
условно показаны перечисленные сферы и среды)~\cite{2zat, 7zat}:
   \begin{itemize}
\item представление в цифровой среде ИИС личностных, коллективных и конвенциональных 
концептов, принадлежащих к \textit{ментальной сфере} пользователей ИИС, а также 
отражение эволюции этих концептов с помощью \textit{семантических идентификаторов} 
дескрипторов тезауруса ИИС;
\item описание в цифровой среде ИИС информационных объектов 
\textit{со\-ци\-аль\-но-ком\-му\-ни\-ка\-ци\-он\-ной среды} ИИС с помощью 
\textit{информационных идентификаторов} дескрипторов тезауруса;
\item описание в цифровой среде ИИС денотатов\linebreak \textit{материальной сферы} 
физических объектов (явлений) или денотатов цифровой среды ИИС с помощью 
\textit{объектных идентификаторов} дескрипторов тезауруса;
\item категоризация кодов \textit{цифровой среды} ИИС, позволяющая различать 
цифровые денотаты и их компьютерные коды в цифровой среде ИИС, а также 
различать три категории кодов для концептов, информационных объектов и 
денотатов (на рис.~\ref{f1zat} они обозначены соответственно как <<Код 
категории~1>>, <<Код категории~2>> и <<Код категории~3>>), представляющие собой 
соответственно семантические, информационные и объектные идентификаторы 
дескрипторов тезауруса ИИС.
\end{itemize}
   
   
   В работе~\cite{2zat}  была предложена \textit{семиотическая модель}, которая 
предназначена для компьютерного кодирования концептов, информационных объектов и 
денотатов при условии, что \textit{они не изменяются во времени}. Поэтому предложенную 
ранее семиотическую модель будем называть стационарной. В данной статье будет 
рассмотрена \textit{нестационарная семиотическая модель}, предназначенная для 
компьютерного кодирования концептов, информационных объектов и денотатов при 
условии, что \textit{они изменяются во времени}. Слово <<нестационарная>> в названии 
этой модели относится к их изменяемости во времени.
   
   Построение нестационарной семиотической модели основано на предложенной ранее 
стационарной модели, которая по определению из работы~\cite{2zat} включает пять 
составляющих:
   \begin{itemize}
\item \textit{ментальную сферу} пользователей ИИС, \textit{материальную сферу} 
физических объектов и явлений, отражаемых в ИИС, 
\textit{социально-ком\-му\-ни\-ка\-ци\-он\-ную и цифровую среды} ИИС;
\item денотаты, соответствующие им концепты как значения знаков и 
информационные объекты как формы знаков, которые образуют 
\textit{семиотические треугольники Фреге} с вершинами <<денотат\,--\,значение 
знака\,--\,форма знака>>~\cite{8zat, 9zat}, а также компьютерные коды 
следующих трех категорий:
\begin{enumerate}[(1)]
\item
коды \textit{концептов} как значений знаков~--- первая категория 
компьютерных кодов, которые будем называть \textit{семантическими} (на 
рис.~\ref{f1zat} показан один концепт, который относится к ментальной 
сфере пользователей ИИС);\\[-15pt]
\item коды \textit{информационных объектов} как форм знаков~--- вторая 
категория компьютерных кодов, которые будем называть 
\textit{информационными} (на рис.~\ref{f1zat} показан один 
информационный объект, который относится к 
со\-ци\-аль\-но-ком\-му\-ни\-ка\-ци\-он\-ной среде ИИС);\\[-15pt]
\item коды \textit{денотатов} материальной или цифровой природы~--- 
третья категория компьютерных кодов, которые будем называть 
\textit{объектными} (на рис.~\ref{f1zat} показан один денотат, который 
относится к сфере материальных объектов и явлений);\\[-15pt]
\end{enumerate}
\item авторские, коллективные или конвенциональные знаки, состоящие из 
концептов как значений знаков и информационных объектов как форм знаков (на 
рис.~\ref{f1zat} знаки, являющиеся элементами знаковой системы ИИС, не 
пока\-заны); %[-15pt]
\item формокоды, состоящие из информационных объектов и компьютерных кодов 
второй категории, являющиеся элементами формокодовой кодировки, 
используемой для генерирования информационных компьютерных кодов в ИИС 
(на рис.~\ref{f1zat} показан один формокод, который расположен на границе 
между со\-ци\-аль\-но-ком\-му\-ни\-ка\-ци\-он\-ной и цифровой средами ИИС);
\item семокоды, состоящие из концептов и компьютерных кодов первой категории, 
являющиеся элементами семокодовой кодировки, используемой для 
генерирования семантических компьютерных кодов в ИИС (на рис.~\ref{f1zat} 
показан один семокод, который расположен на границе между ментальной сферой 
пользователей ИИС и ее цифровой средой). %[-15pt]
\end{itemize}

   На рис.~\ref{f1zat} граница между ментальной сферой и цифровой средой условно 
обозначена двойной сплошной линией, между социально-ком\-му\-ни\-ка\-ци\-он\-ной и цифровой 
средами ИИС~--- штрих-пунктирной линией, а между ментальной сферой и 
   социально-ком\-му\-ни\-ка\-ци\-он\-ной средой~--- штриховой линией.
   
   Как следует из приведенного определения, стаци\-онарная семио\-тическая модель основана 
на семиотическом треугольнике Фреге\footnote{В 
данной статье три вершины семиотического треугольника Г.~Фреге~--- это значение знака (его концепта), 
форма знака (как частный случай информационного объекта) и денотат знака (материальной, цифровой или 
иной природы)~\cite{8zat, 9zat}.\label{prim}}, стороны которого, обозначенные на рис.~\ref{f1zat} полужирными\linebreak 
отрезками, соединяют три вершины треугольника: концепт как значение знака, 
соответствующий ему денотат и информационный объект как форму этого знака. Для 
каждой из трех вершин треугольника Фреге в работе~\cite{2zat} было предложено 
использовать свою систему кодировки: семокодовую для концептов, объектную для 
денотатов и формокодовую для информационных объектов.
   
   Важно отметить, что для описания стационарной семиотической модели использовалась 
система из 12 взаимосвязанных терминов, определенная в работах~\cite{2zat, 7zat, 10zat, 
11zat}. Эта система терминов включает понятия <<знания>>, <<концепт>>, 
<<информация>>, <<информационный объект>>, <<данные>>, <<цифровые данные>> и 
<<коды>>. При этом каж\-дый из 12~терминов системы принадлежит только к одной сфере 
(среде) в зависимости от природы соответствующего понятия: ментальной, материальной, 
социально-ком\-му\-ни\-ка\-ци\-он\-ной или цифровой (см.\ рис.~\ref{f1zat}). Идея подобного 
разделения сфер (сред) используется в достаточно широком спектре проблем информатики, 
описание которых можно найти в работах~[12--15].
{ %\looseness=-2

}
   
   Новые ключевые понятия в этой модели выражены терминами <<формокод>> и 
<<семокод>>, каж\-дый из которых по своей природе является двуединой сущностью. По 
определению из работы~\cite{10zat} формокод состоит из информационного объекта как 
формы знака и компьютерного кода второй категории, т.\,е.\ имеет социально-циф\-ро\-вую 
природу, а семокод состоит из концепта как значения знака и компьютерного кода первой 
категории, т.\,е.\ имеет ментально-цифровую природу.
   
   Сочетание трех компьютерных кодов трех разных категорий~--- семантического, 
информаци\-онного и объектного~--- для трех вершин любого треугольника Фреге 
предлагается называть <<циф\-ро\-вым семиотическим треугольником>> (на рис.~\ref{f1zat} 
три его стороны изображены двойными точечными отрезками, которые соединяют три 
компьютерных кода цифровой среды ИИС). Таким образом, стационарная семиотическая 
модель отображает треугольник Фреге, вершины которого принадлежат разным сферам и 
средам, в цифровой семиотический треугольник, принадлежащий полностью только 
цифровой среде ИИС.
   Включение в модель формокода, семокода и компьютерных кодов трех перечисленных 
категорий позволило предложить в работе~\cite{2zat} принципиально новые методы 
трехкомпонентной кодировки концептов, соответствующих им информационных объектов и 
денотатов\footnote{Под трехкомпонентной кодировкой, которая была рассмотрена в 
работе~\cite{2zat}, понимается присвоение компьютерных кодов трем вершинам семиотического 
треугольника Фреге, т.\,е.\ концепту как значению знака, информационному объекту как форме знака и 
денотату.} (см.\ рис.~\ref{f1zat}). Эти методы в дополнение к литерной (посимвольной) 
кодировке текстов обеспечивают формирование (генерирование) в ИИС следующих 
компьютерных кодов трех категорий для концептов, информационных объектов и денотатов:

\noindent
   \begin{enumerate}[(1)]
\item семантические компьютерные коды для концептов как значений знаков, 
пред\-став\-ля\-ющих собой \textit{семантические идентификаторы} дескрипторов 
тезауруса\footnote{Каждый из таких концептов, принадлежащих к ментальной сфере 
пользователей ИИС, в паре со своим кодом образует один элемент семокодовой кодировки.};
\item информационные компьютерные коды для информационных объектов как 
форм знаков, представляющих собой \textit{информационные идентификаторы} 
дескрипторов тезауруса\footnote{Каждый из таких информационных объектов, 
принадлежащих социально-ком\-му\-ни\-ка\-ци\-он\-ной среде ИИС, в паре со своим кодом образует один 
элемент формокодовой кодировки.};
\item объектные компьютерные коды для денотатов, представляющих собой 
\textit{объектные идентификаторы} дескрипторов тезауруса (объектная 
кодировка денотатов, принадлежащих материальной сфере, цифровой или иной 
среде).
\end{enumerate}

   В работе~\cite{2zat} были рассмотрены примеры построения семокодовой и 
формокодовой кодировок, а также формирование с их помощью семантических и 
информационных компьютерных кодов, используемых для решения задач концептуального 
индексирования и семантического поиска в электронной библиотеке геоизображений. Эти 
примеры содержат алгоритмы формирования следующих кодов:
   \begin{itemize}
\item для концептов как значений вербальных или образных знаков, 
соответствующих геообъектам как денотатам, которые представляют собой 
устьевые области рек;
\item для информационных объектов как форм вербальных или образных знаков, 
со\-от\-вет\-ст\-ву\-ющих концептам как значениям этих же вербальных или образных 
знаков.
\end{itemize}
   
   Рассмотренные в работе~\cite{2zat} примеры иллюстрируют сферы применения 
стационарной семиотической модели, а именно: стабильные концепты, устойчивые формы 
их представления и неизменные денотаты. Для постановки, исследования и решения 
проблем идентификации концептов ЦСЗ и оценивания их релевантности технологическим 
или другим общественно значимым потребностям, а также проблемы направляемого 
развития ЦСЗ в данной статье предлагается построить \textit{нестационарную 
семиотическую модель} компьютерного кодирования концептов, информационных объектов 
и денотатов с учетом необходимости фиксации их эволюции во времени в виде 
последовательности\linebreak \vspace*{-12pt}
\columnbreak

\noindent
 дискретных стадий. Построение этой модели является основной целью 
настоящей статьи.

\section{Пространство Фреге}

   Для достижения этой цели сначала определим множество точек, которое будем называть 
<<пространством Фреге>>. Это пространство будем строить для отображения во времени 
стадий эволюции концептов ЦСЗ, а также соответствующих им информационных объектов и 
денотатов. Основная задача этого раздела заключается в том, чтобы эволюцию концептов 
ЦСЗ представить в виде последовательностей семантических кодов, а эволюцию 
соответствующих им информационных объектов и денотатов~--- в виде информационных и 
объектных компьютерных кодов соответственно.
   
   Эволюцию концептов ЦСЗ будем фиксировать в последовательные моменты времени, 
которые обозначим как $\{t_i, \ i = 1, 2, \ldots\}$, где $t_i$~--- $i$-й момент времени описания в 
тезаурусе ИИС вершин семиотических треугольников Фреге, т.\,е.\ эволюционирующих 
концептов ЦСЗ, соответствующих им информационных объектов и денотатов. При этом за 
$t_1$ принимается тот момент времени, когда пользователи ИИС начали фиксировать 
эволюцию концептов ЦСЗ, информационных объектов и денотатов в виде дескрипторов 
тезауруса. Сам процесс формирования ЦСЗ мог начаться и до момента времени~$t_1$, но 
самый первый момент времени построения дескрипторов для концептов этой ЦСЗ обозначим 
именно как~$t_1$. Вопросам описания концептов в виде дескрипторов тезауруса посвящена 
работа~\cite{7zat}.
   
   Как отмечалось ранее в сноске~1 на с.~\pageref{prim}, три вершины семиотического треугольника Фреге, 
построенного пользователями ИИС в процессе описания эволюции любого концепта ЦСЗ, 
представляют собой:
   \begin{itemize}
\item концепт ЦСЗ как значение знака, принадлежащий ментальной сфере 
пользователей ИИС;
\item информационный объект как форму знака, принадлежащий 
социально-ком\-му\-ни\-ка\-ци\-он\-ной среде ИИС;
\item денотат как объект означивания, в результате анализа и интерпретации 
которого генерируются новые и/или изменяются существующие концепты, 
формируются новые и/или изменяются существующие информационные объ\-екты.
\end{itemize}

   Предполагается, что в моменты времени $\{t_i$, $i =$\linebreak $=\;1, 2, \ldots\}$ в процессе описания 
пользователями эволюционирующих концептов ЦСЗ, со\-от\-вет\-ст\-ву\-ющих им информационных 
объектов и денотатов ИИС генерирует соответствующие им цифровые семиотические 
треугольники. При этом в каж\-дый момент времени~$t_i$ может быть сгенерировано 
одновременно несколько цифровых семиотических треугольников с использованием 
уникальных идентификаторов дескрипторов тезауруса, если пользователи ИИС фиксируют в 
тезаурусе в этот момент времени несколько эволюционирующих концептов. Факторы, 
влияющие на направление эволюции концептов ЦСЗ во времени, будут рассмотрены далее в 
статье.
   
   Пространство Фреге предлагается строить, исходя из следующего требования: любому 
циф\-ровому семиотическому треугольнику (см.\ рис.~\ref{f1zat})\linebreak должна соответствовать 
одна точка~($t, n, m, k$), состоящая из следующих элементов:
   \begin{itemize}
   \item $t$~--- момент времени описания в тезаурусе ИИС некоторого треугольника Фреге и 
генерации соответствующего ему цифрового семиотического треугольника;
\item $n$~--- семантический компьютерный код для концепта как значения знака, 
сгенерированный с помощью семокодовой кодировки ИИС в процессе описания в 
тезаурусе ИИС треугольника Фреге;
\item $m$~--- информационный компьютерный код для информационного объекта 
как формы знака, сгенерированный с помощью формокодовой кодировки ИИС в 
процессе описания в тезаурусе ИИС семиотического треугольника Фреге;
\item $k$~--- объектный компьютерный код для денотата, сгенерированный в 
соответствии с регламентом идентификации денотатов в процессе описания в 
тезаурусе ИИС треугольника \mbox{Фреге}.
\end{itemize}

   Следовательно, пространство Фреге должно включать ось времени, а также оси 
семантических, информационных и объектных компьютерных кодов. Если в тезаурусе ИИС 
используется $N$~видов семантических кодов, $M$~видов информационных кодов и 
$K$~видов объектных кодов, то размерность пространства Фреге будет равна $N + M + K + 
1$ (пример множественности видов информационных кодов приведен в работе~\cite{2zat}). 
В~этой статье рас\-смат\-ри\-ва\-ет\-ся случай, когда $N = M = K = 1$.
   
   Время в этом пространстве может изменяться от нуля до бесконечности. Что касается 
осей семантических, информационных и объектных кодов, то в соответствии с определением 
термина <<код>> в работах~\cite{2zat, 7zat, 10zat, 11zat} любому коду, 
пред\-став\-ля\-юще\-му собой идентификатор некоторого дескриптора, можно поставить в 
соответствие целочисленное двоичное число. Так как со временем число дескрипторов и 
число их идентификаторов увеличивается, то предлагается не устанавливать верхнюю 
границу для числовых значений кодов по каждой из осей в пространстве Фреге, т.\,е.\ 
числовые пред\-став\-ле\-ния семантических, информационных и объектных компьютерных 
кодов также могут изменяться от нуля до бесконечности.
   
   Любая точка~$(t, n, m, k)$ в пространстве Фреге определяется на основе семантического, 
информационного и объектного идентификаторов того дескриптора тезауруса, который 
соответствует некоторому треугольнику Фреге. Поэтому и предлагается построенное 
пространство называть <<пространством Фреге>>.
   
   Важно отметить, что построение этого пространства, размерность которого в общем 
случае равна $N + M +K+ 1$, предполагает наличие правил установления однозначного 
отображения сгенерированных цифровых семиотических треугольников на пространство 
Фреге. Эти правила представляют собой формализованный аппарат, позволяющий 
вычислять точки~$(t, n, m, k)$ пространства Фреге на основе кодов вершин генерируемых 
цифровых семиотических треугольников. Пространство Фреге не содержит описания 
взаимосвязей компьютерных кодов с концептами, информационными объектами и 
денотатами. Однако в любой дискретный момент времени~$t_i$ для любого семиотического 
треугольника Фреге описание взаимосвязей компьютерных кодов с его концептом как 
значением знака, информационным объектом как формой знака и денотатом фиксируется с 
помощью стационарной семиотической модели, определение которой приведено в 
предыдущем разделе (см.\ рис.~\ref{f1zat}).
   
   Сформулируем четыре \textit{правила установления однозначного отображения} 
сгенерированных цифровых семиотических треугольников на пространство Фреге, которые 
должны позволять однозначно определять только одну точку~$(t, n, m, k)$ в этом 
пространстве для любого цифрового семиотического треугольника:
   \begin{itemize}
\item $t$ определим как дискретный момент времени описания в тезаурусе ИИС 
семиотического треугольника Фреге и одновременной генерации 
рассматриваемого цифрового семиотического треугольника;
\item точку~$n$ на семантической оси определим как числовое значение 
семантического компьютерного кода рассматриваемого цифрового треугольника;
\item точку~$m$ на информационной оси определим как числовое значение 
информационного компьютерного кода рассматриваемого цифрового 
треугольника;
\item точку~$k$ на объектной оси определим как числовое значение объектного 
компьютерного кода рассматриваемого цифрового треугольника.
\end{itemize}

   Предположим, что в один и тот же момент времени два разных цифровых семиотических 
треугольника были отображены в одну и ту же точку~$(t, n, m, k)$ пространства Фреге. 
Отличие двух цифровых семиотических треугольников означает, что соответствующие два 
семиотических треугольника Фреге различаются хотя бы одной вершиной. Тогда случай 
отображения двух разных цифровых семиотических треугольников в одну и ту же точку~$(t, 
n, m, k)$ означает, что двум разным концептам и/или двум разным информационным 
объектам и/или двум разным денотатам в тезаурусе ИИС в момент времени~$t$ был 
приписан один и тот же код, что противоречит положению об уникальности 
идентификаторов дескрипторов тезауруса ИИС. Таким образом, любому цифровому 
семиотическому треугольнику соответствует только одна точка~$(t, n, m, k)$ в пространстве 
Фреге и наоборот.
   
   Используя пространство Фреге и предлагаемое однозначное соответствие, определим 
\textit{нестационарную семиотическую модель} компьютерного кодирования концептов, 
информационных объектов и денотатов с учетом их эволюции во времени как сочетание 
следующих трех составляющих:
   \begin{enumerate}[(1)]
\item любое семейство семиотических треугольников Фреге, построенных 
пользователями ИИС в процессе эволюции ЦСЗ и описанных ими в виде 
дескрипторов тезауруса ИИС в дискретные моменты времени~$\{t_i, \ i = 1, 2, 
\ldots\}$;
\item соответствующие цифровые семиотические треугольники (см.\ 
рис.~\ref{f1zat});
\item множество точек~$\{(t_i, n_{i,j}, m_{i,j}, k_{i,j})$, где $j=$\linebreak $=1, \ldots , L_i$, $L_i$~--- 
число цифровых семиотических треугольников, сгенерированных в момент 
времени~$t_i$, $i = 1, 2, \ldots\}$, полученных в результате однозначного 
отображения циф\-ро\-вых семиотических треугольников на пространство Фреге с 
использованием четырех сформулированных правил установления однозначного 
отобра\-жения.
\end{enumerate}

   Завершая построение нестационарной се\-миотической модели отметим, что любое 
мно-\linebreak\vspace*{-12pt}
\columnbreak

\noindent
жество точек~$\{(t_i, n_{i,j}, m_{i,j}, k_{i,j})\}$ пространства Фреге яв\-ляется зависимым от 
тезауруса ИИС, так как информационный, семантический и объектный компьютерные коды 
вершин любого цифрового семиотического треугольника представляют собой 
соответственно информационный, семантический и объектный идентификаторы дескриптора 
тезауруса, соответствующего этому треугольнику. Эти коды генерируются автоматически в 
процессе описания пользователями ИИС <<траекторий>> эволюции концептов ЦСЗ, их 
информационных объектов и денотатов в виде дескрипторов тезауруса в дискретные 
моменты времени~$\{t_i, i = 1, 2, \ldots \}$. При этом в тезаурусе фиксируются только 
дискретные моменты времени эволюции, но сам процесс эволюции может быть и 
непрерывным.

\vspace*{-4pt}

\section{Пример нестационарной семиотической модели}
\vspace*{-2pt}

   Прежде чем рассмотреть факторы, влияющие на формирование ЦСЗ, приведем 
иллюстративный пример описания эволюционирующих концептов ЦСЗ в виде дескрипторов 
тезауруса. Как уже отмечалось, концепты по определению относятся к ментальной сфере 
пользователей ИИС как генераторов ЦСЗ. В~общем случае ЦСЗ может включать личностные 
концепты одного или нескольких пользователей ИИС, коллективные и конвенциональные 
концепты~\cite{7zat, 11zat}. В~процессе построения дескрипторов тезауруса, 
относящихся к \textit{цифровой среде ИИС}, пользователи ИИС формируют их, используя в 
качестве своеобразных посредников между концептами и кодами информационные объекты 
социально-ком\-му\-ни\-ка\-ци\-он\-ной среды ИИС\footnote{В~работе~\cite{2zat} приведено описание 
микрокомпьютера MITS Altair~8800, при работе с которым пользователь мог формировать коды без 
применения информационных объектов, используя только двоичные переключатели и лампочки на передней 
панели как формы представления компьютерных кодов вне цифровой среды.}.
   
   В этом примере рассмотрим случай одного пользователя, целью которого является 
формирование ЦСЗ о группе индикаторов распределения по возрастным группам 
публикационной активности научного коллектива (далее~--- ЦСЗ публикационной 
активности). Эта группа новых индикаторов была описана в~\cite{7zat}. В этой работе 
были рассмотрены традиционные индикаторы возрастного распределения научных 
сотрудников и по аналогии с ними формировалась группа новых индикаторов распределения 
публикационной активности. Построим нестационарную семиотическую модель, 
отра\-жа\-ющую два момента времени~$t_1$ и~$t_2$ процесса построения ЦСЗ публикационной 
активности.
\pagebreak
\end{multicols}

\begin{figure} %fig2
\begin{center}
\vspace*{1pt}
\mbox{%
\epsfxsize=162.621mm
\epsfbox{zat-2.eps}
}
\end{center}
\vspace*{-3pt}
\Caption{Два состояние ЦСЗ публикационной активности в моменты времени~$t_1$ и~$t_2$
\label{f2zat}}
\vspace*{9pt}
\end{figure}

\begin{multicols}{2}
   
   Предположим, что в момент времени~$t_1$ пользователь описал ЦСЗ публикационной 
активности, включающую два его личностных концепта, в виде набора из двух дескрипторов 
тезауруса ИИС (обозначим состояние ЦСЗ в момент времени~$t_1$ как~$S(t_1)$). Эти два 
личностных концепта обозначены на рис.~\ref{f2zat} как Concept1$(t_1)$ и~Concept2$(t_1)$.




   Концепт Concept1$(t_1)$ был сформирован пользователем при следующих исходных 
данных. Учитывались публикации в любых научных журналах, сборниках и~т.\,д., 
зарегистрированные в течение заданного периода времени в базе данных ИИС. Если у 
публикации было несколько авторов, то соответствующим возрастным группам добавлялось 
по одному баллу. На основе этих исходных данных была разработана программа вычисления 
значений первого варианта нового индикатора пуб\-ли\-ка\-ци\-он\-ной активности.
   
   Концепт Concept2$(t_1)$ был сформирован почти при тех же исходных данных, но 
которые отличались порядком учета публикаций с несколькими авторами: если у 
публикации было $X$~авторов, то соответствующим возрастным группам добавлялось по 
$1/X$~балла~\cite{7zat}. На основе этих исходных данных была разработана программа 
вычисления значений второго варианта нового индикатора пуб\-ли\-ка\-ци\-он\-ной активности.
   
   Структурированное описание пользователем в момент времени~$t_1$ исходных данных и 
алго\-рит\-мов программ, соответствующих концептам Concept1$(t_1)$ и~Concept2$(t_1)$, 
представляет собой\linebreak дефиницию для каждого из двух новых дескрипторов тезауруса. 
Отображения этих дефиниций на экране монитора обозначены на рис.~\ref{f2zat} как два 
информационных объекта~Object1$(t_1)$ и\linebreak Object2$(t_1)$.
{ %\looseness=1

}
   
   Затем в момент времени~$t_2$ этот же пользователь продолжил описание ЦСЗ 
публикационной активности и сформировал новое состояние ЦСЗ, которое обозначим 
как~$S(t_2)$, включающее три его личностных концепта. Эти концепты на рис.~\ref{f2zat}\linebreak 
обозначены как Concept1$(t_2)$, Concept2$(t_2)$ и Concept3$(t_2)$.
   
   Концепт Concept1$(t_2)$ был сформирован при следующих исходных данных. 
Учитывались пуб\-ли\-ка\-ции только в журналах из списка ВАК, зарегистрированные в течение 
заданного периода времени в базе данных ИИС. Если у публикации было несколько авторов, 
то соответствующим возрастным группам добавлялось по одному баллу. Концепт 
Concept2$(t_2)$ был сформирован почти при тех же исходных данных, но которые 
отличались порядком учета публикаций с несколькими авторами (при $X$~авторах 
добавлялось по~$1/X$~балла). При этом использовались те же программы, что и для 
концептов~Concept1$(t_1)$ и~Concept2$(t_1)$, но этими программами обрабатывались только 
публикации в журналах из списка ВАК. Концепт~Concept3$(t_2)$ был сформирован при 
существенно иных исходных данных. Проводилась нормализация публикационной 
активности с учетом численности каждой воз\-раст\-ной группы, проставлялись разные 
коэффициенты постоянным научным сотрудникам и совместителям, учитывались 
публикации как в журналах из списка ВАК, так и в любых иностранных жур\-налах.
{ %\looseness=1

}
   
   Первые два концепта в момент времени~$t_2$ пользователь описал в виде дескрипторов 
тезауруса как варианты концептов~Concept1$(t_1)$ и~Concept2$(t_1)$, а третий свой личностный 
концепт~Concept3$(t_2)$ он отметил в тезаурусе как новый. Структурированное описание 
пользователем в момент времени~$t_2$ исходных данных и алгоритмов программ, 
соответствующих концептам~Concept1$(t_2)$, Concept2$(t_2)$ и~Concept3$(t_2)$, 
представляет собой три дефини-\linebreak ции для  трех новых дескрипторов тезауруса. 
Отобра\-жения этих дефиниций на экране монитора обозна\-чены на рис.~\ref{f2zat} как три 
информационных объекта~Object1$(t_2)$, Object2$(t_2)$ и~Object3$(t_2)$. Все 
рассматриваемые информационные объекты относятся к социально-ком\-му\-ни\-ка\-ци\-он\-ной 
среде ИИС.
   
   Важно отметить, что описание дескрипторов тезауруса ИИС в моменты времени~$t_1$ 
и~$t_2$ пользователь строил на основе своих личностных концептов, сформированных в 
процессе анализа и интерпретации цифровых (\textit{а не материальных}!) денотатов. 
Каж\-дый денотат в этом примере представляет собой сочетание программы вычисления 
значений одного из индикаторов публикационной ак\-тив\-ности и используемых этой 
программой цифровых ресурсов базы данных ИИС, содержащей сведения о публикациях 
научных сотрудников.
\columnbreak
   
Таким образом, в отличие от рис.~\ref{f1zat}, где денотат является материальным 
объектом, все рассматриваемые на рис.~\ref{f2zat} денотаты относятся к цифровой среде 
ИИС, а не к материальной. На рис.~\ref{f2zat} концепты~Concept1$(t_1)$ и~Concept2$(t_1)$ 
сформированы в процессе анализа и интерпретации цифровых денотатов~Denotatum1$(t_1)$ 
и~Denotatum2$(t_1)$, а концепты~Concept1$(t_2)$, Concept2$(t_2) $ и~Concept3$(t_2)$~--- в 
процессе анализа и интерпретации денотатов~Denotatum1$(t_2)$, Denotatum2$(t_2)$ 
и~Denotatum3$(t_2)$.
   
   Каждый из пяти концептов~Concept1$(t_1)$, Concept2$(t_1)$, Concept1$(t_2)$, 
Concept2$(t_2)$ и Concept3$(t_2)$ с соответствующим ему денотатом, в процессе анализа и 
интерпретации которого этот концепт и был сформирован, а также информационным 
объектом как результатом представления дефиниции этого концепта в виде формы знака или 
устойчивого знакового выражения, образует один семиотический треугольника 
Фреге~\cite{2zat}. Каждому семиотическому треугольнику Фреге соответствует один 
дескриптор тезауруса ИИС с тремя идентификаторами: семантическим, информационным и объектным.
   
   В процессе описания каждого дескриптора тезауруса ИИС автоматически генерирует 
коды трех разных категорий. Например, на рис.~\ref{f2zat} показано, что ИИС генерирует в 
момент времени~$t_1$ коды для первого концепта~Concept1$(t_1)$, его информационного 
объекта~Object1$(t_1)$ и дено\-тата Denotatum1$(t_1)$, которые обозначены как\linebreak 
   Com\-put\-er\ code1.I$(t_1)$, Com\-put\-er\ code1.II$(t_1)$ и~Com\-put\-er\ code1.III$(t_1)$ 
соответственно, а в момент времени~$t_2$~--- коды для нового варианта этого же концепта 
Concept1$(t_2)$, его информационного \mbox{объекта} Object1$(t_2)$ и денотата~Denotatum1$(t_2)$, 
которые обозначены как Com\-put\-er\ code1.I$(t_2)$, Com\-put\-er\ code1.II$(t_2)$ и Com\-put\-er\ 
code1.III$(t_2)$. Сочетание трех кодов Com\-put\-er\ code1.I$(t_1)$, Com\-put\-er\ code1.II$(t_1)$ 
и Com\-put\-er\ code1.III$(t_1)$ на рис.~\ref{f2zat} помечено как <<Коды дескриптора 
первого концепта в~$t_1$>>, а трех кодов Com\-put\-er\ code1.I$(t_2)$, Com\-put\-er\ code1.II$(t_2)$ 
и Com\-put\-er\ code1.III$(t_2)$~--- как <<Коды дескриптора первого концепта в~$t_2$>>. 
Римскими циф\-ра\-ми в этих кодах обозначены категории кодов (I~обозначает семантические 
коды, II~--- информационные, а III~--- объектные).
   
   Отметим, что на рис.~\ref{f2zat} в явном виде показаны только шесть кодов из~15 (для 
5~концеп\-тов, 5~информационных объектов и 5~денота\-тов) и не показаны связи концептов,
информационных объектов и денотатов. Причина в том, что для этого понадобилось бы существенно усложнить 
рис.~\ref{f2zat}. Таким образом, эти связи и 9~кодов из~15 на этом рисунке не показаны. Для того 
чтобы показать\linebreak
\vspace*{-12pt}
\pagebreak

\end{multicols}

\begin{figure} %fig3
\begin{center}
\vspace*{1pt}
\mbox{%
\epsfxsize=159.984mm
\epsfbox{zat-3.eps}
}
\end{center}
\vspace*{-6pt}
\Caption{Два цифровых семиотических треугольника описания двух концептов Concept1$(t_1)$ 
и~Concept1$(t_2)$ ЦСЗ публикационной активности
\label{f3zat}}
\vspace*{6pt}
\end{figure}

\begin{multicols}{2}

\noindent
 в явном виде связи
 концептов, 
информационных объектов и денотатов, предназначен рис.~\ref{f3zat}.
  
   На этом рисунке показаны шесть кодов для двух концептов 
Concept1$(t_1)$ и~Concept1$(t_2)$, двух информационных объектов Object1$(t_1)$ 
и~Object1$(t_2)$, а также двух денотатов~Denotatum1$(t_1)$ и~Denotatum1$(t_2)$. 
Остальные три концепта, три информационных объекта, три денотата и их коды на 
рис.~\ref{f3zat} не показаны, т.\,е.\ на этом рисунке в явном виде показаны только два 
концепта, два информационных объекта, два денотата, а также шесть их кодов, которые 
образуют два цифровых семиотических треугольника в моменты времени~$t_1$ и~$t_2$. 
Однако в пространстве Фреге на рис.~\ref{f4zat} показаны все пять точек с числовыми 
значениями всех 15~компьютерных кодов (5~чисел~$n_{1,1}$, $n_{1,2}$, $n_{2,1}$, $n_{2,2}$, 
$n_{2,3}$ для семантических кодов концептов, 5~чисел $m_{1,1}$, $m_{1,2}$, $m_{2,1}$, $m_{2,2}$, 
$m_{2,3}$ для информационных кодов и 5~чисел $k_{1,1}$, $k_{1,2}$, $k_{2,1}$, $k_{2,2}$, $k_{2,3}$ 
для объектных кодов денотатов).
\columnbreak
   
   
   На рис.~\ref{f3zat} концепт Concept1$(t_1)$ является вершиной левого треугольника 
Фреге, а концепт Concept1$(t_2)$~--- вершиной правого треугольника. Оба семиотических 
треугольника Фреге обозначены на этом рисунке одинарными точечными линиями. Каждому 
из этих двух треугольников Фреге соответствует свой цифровой семиотический треугольник, 
обозначенный двойными точечными линиями: один в момент времени~$t_1$ и один~--- 
в~$t_2$.

   \begin{figure*} %fig4
   \begin{center}
\vspace*{1pt}
\mbox{%
\epsfxsize=122.316mm
\epsfbox{zat-4.eps}
}
\end{center}
\vspace*{-6pt}
\Caption{Отображение пяти цифровых семиотических треугольников на пространство Фреге
\label{f4zat}}
\end{figure*}

   
   Левый цифровой семиотический треугольник, сгенерированный ИИС в момент 
времени~$t_1$, включает коды трех категорий:
   \begin{enumerate}[(1)]
\item семантический код~Computer\ code1.I$(t_1)$ для концепта Concept1$(t_1)$ как 
значения соответствующего авторского знака пользователя;
\item информационный код Computer\ code1.II$(t_1)$ для информационного 
объекта Object1$(t_1)$ как формы этого же авторского знака;
\pagebreak
\item объектный код Computer\ code1.III$(t_1)$ того денотата Denotatum1$(t_1)$, в 
результате анализа и интерпретации которого был сформирован концепт 
Concept1$(t_1)$.
\end{enumerate}
   
   Правый цифровой семиотический треугольник, сгенерированный ИИС в момент 
времени~$t_2$, также включает коды трех категорий:
   \begin{enumerate}[(1)]
\item семантический код Computer\ code1.I$(t_2)$ для концепта Concept1$(t_2)$ как 
значения соответствующего авторского знака пользователя;
\item информационный код Computer\ code1.II$(t_2)$ для информационного 
объекта~Object1$(t_2)$ как формы этого же авторского знака;
\item объектный код Computer\ code1.III$(t_2)$ того денотата~Denotatum1$(t_2)$, в 
результате анализа и интерпретации которого был сформирован 
концепт~Concept1$(t_2)$.
\end{enumerate}

   Авторские знаки, определение и описание которых можно найти в работах~\cite{7zat, 
11zat}, на рис.~\ref{f2zat} и~\ref{f3zat} не показаны. Рисунок~\ref{f3zat} иллюстрирует 
на примере двух вариантов первого концепта~Concept1$(t_1)$ и~Concept1$(t_2)$ построение 
следующих двух со\-став\-ля\-ющих нестационарной семиотической модели компьютерного 
кодирования концептов, информационных объектов и денотатов:
   \begin{itemize}
\item семейство из двух семиотических треугольников Фреге для Concept1$(t_1)$ 
и~Concept1$(t_2)$, описанных пользователем в моменты времени~$t_1$ и~$t_2$ в 
виде дескрипторов тезауруса ИИС в процессе построения ЦСЗ публикационной 
активности;
\item соответствующие этим двум треугольникам Фреге два цифровых 
семиотических треугольника, автоматически сгенерированные ИИС для 
Concept1$(t_1)$ и~Concept1$(t_2)$ в моменты времени~$t_1$ и~$t_2$.
\end{itemize}


   Для завершения построения примера нестационарной семиотической модели определим в 
пространстве Фреге множество из пяти точек~$\{(t_i, n_{i,j}, m_{i,j}, k_{i,}j)$, где $j = 1, \ldots , 
L_i$, $L_1 = 2$, $L_2 = 3$, $i = 1, 2\}$. В этом примере число циф\-ро\-вых семиотических 
треугольников~$L_1$, сгенерированных ИИС в момент времени~$t_1$, равно~2, а число 
циф\-ро\-вых семиотических треугольников~$L_2$, сгенерированных ИИС в момент 
времени~$t_2$, равно~3. Эти пять точек получены в результате следующего однозначного 
отображения пяти циф\-ро\-вых семиотических треугольников для концептов Concept1$(t_1)$, 
Concept2$(t_1)$, Concept1$(t_2)$, Concept2$(t_2)$ и~Concept3$(t_2)$ на пространство Фреге с 
использованием четырех сформулированных ранее правил установления отображения (см.\ 
рис.~\ref{f4zat}):
\begin{itemize}
\item
$t_1$ и~$t_2$~--- моменты времени описания пользователем в тезаурусе ИИС пяти 
семиотических треугольников Фреге (двух в момент времени~$t_1$ и трех в 
момент времени~$t_2$);
\item $n_{1,1}$, $n_{1,2}$, $n_{2,1}$, $n_{2,2}$, $n_{2,3}$~--- числовые значения 
семантических компьютерных кодов Computer\ code1.I$(t_1)$, Computer\ 
code2.I$(t_1)$, Computer\ code1.I$(t_2)$, Computer\ code2.I$(t_2)$ и Computer\ 
code3.I$(t_2)$ концептуальных вершин пяти циф\-ро\-вых семиотических 
треугольников;
\item $m_{1,1}$, $m_{1,2}$, $m_{2,1}$, $m_{2,2}$, $m_{2,3}$~--- числовые значения 
информационных компьютерных кодов Computer\ code1.II$(t_1)$, Computer\ 
code2.II$(t_1)$, Computer\ code1.II$(t_2)$, Computer\ code2.II$(t_2)$ и Computer\ 
code3.II$(t_2)$ информационных вершин этих треугольников;
\item $k_{1,1}$, $k_{1,2}$, $k_{2,1}$, $k_{2,2}$, $k_{2,3}$~--- чис\-ло\-вые значения объектных 
компьютерных кодов Computer\ code1.III$(t_1)$, Computer\ code2.III$(t_1)$, 
Computer\ code1.III$(t_2)$, Computer\ code2.III$(t_2)$ и Computer\ code3.III$(t_2)$ 
объектных вершин этих треугольников.
\end{itemize}

   Рисунок~\ref{f4zat} иллюстрирует построение однозначного отображения пяти цифровых 
семиотических треугольников на пространство Фреге, размерность которого равна~4. Так 
как отображаемое семейство включает цифровые семиотические треугольники, которые 
сгенерированы ИИС только в два момента времени~$t_1$ и~$t_2$, то ось времени на 
рис.~\ref{f4zat} в явном виде не показана. В этом примере для множества из пяти точек в 
пространстве Фреге показаны только три оси: семантическая ось~$n$, информационная 
ось~$m$ и объектная ось~$k$.
   
   Пять точек $\{(t_i, n_{i,j}, m_{i,j}, k_{i,j})$, где $j = 1,  \ldots , L_i$, $L_1 = 2$, $L_2 = 3$, $i = 1, 
2\}$, полученных в результате отображения пяти цифровых семиотических треугольников на 
пространство Фреге, образуют третью составляющую нестационарной семиотической модели 
компьютерного кодирования 5~концептов, 5~информационных объектов и 5~денотатов для 
рассматриваемого примера построения ЦСЗ публикационной активности. Были рассмотрены 
два момента времени~$t_1$ и~$t_2$, когда пользователем были описаны два состояния его 
первого личностного концепта и два состояния второго концепта, а также одно состояние 
третьего концепта в момент времени~$t_2$.

\section{Факторы эволюции целевых систем знаний}

   В рассмотренном примере описаны пять дескрипторов тезауруса в моменты 
времени~$t_1$ и~$t_2$ на основе личностных концептов, сформированных в процессе анализа 
и интерпретации циф\-ро\-вых денотатов Denotatum1$(t_1)$, Denotatum2$(t_1)$, 
Denotatum1$(t_2)$, Denotatum2$(t_2)$ и Denotatum3$(t_2)$. Каждый денотат в этом примере 
представляет собой сочетание программы вычисления индикатора оценивания 
публикационной активности и используемых этой программой цифровых ресурсов ИИС. 
Денотат Denotatum1$(t_2)$ является с точки зрения пользователя модификацией 
денотата Denotatum1$(t_1)$, Denotatum2$(t_2)$~--- модификацией Denotatum2$(t_1)$, а 
денотат Denotatum3$(t_2)$ определен пользователем как новый. Одной из причин эволюции 
ЦСЗ~$S(t)$ и составляющих ее концептов является модификация программ вычисления 
индикаторов и ис\-поль\-зу\-емых этими программами цифровых ресурсов ИИС, а также 
разработка новых программ и формирование новых цифровых ресурсов.
   
   Разработку новых программ и модификацию существующих будем называть 
алгоритмическим фактором эволюции ЦСЗ и составляющих ее концептов, а формирование 
новых цифровых ресурсов и модификацию существующих будем называть 
информационным фактором эволюции. В любой момент описания дескрипторов тезауруса 
пользователь ИИС формирует их на основе своих личностных концептов, которые 
представляют результаты выполненного им анализа и интерпретации соответствующих 
денотатов. Важно отметить, что, изменяя денотаты, т.\,е.\ программы и ис\-поль\-зу\-емые этими 
программами цифровые ресурсы, пользователь итерационно может влиять и на процесс 
\textit{формирования своих личностных концептов}.
   
   Итерационное изменение пользователями ИИС денотатов с последующим их анализом и 
интерпретацией предлагается называть \textit{концептуализацией денотатов}. Если 
изменения денотатов пользователями являются целенаправленными, т.\,е.\ определяются 
явно сформулированными целями, то такой итерационный процесс предлагается называть 
\textit{целенаправленной концептуализацией денотатов}, который и лежит в основе 
формирования ЦСЗ и ее направляемой эволюции.
   
   Сопоставим два введенных понятия <<кон\-цеп\-туализация денотатов>> и 
<<целенаправленная кон\-цептуализация денотатов>> с термином <<концептуализация>> из 
<<Краткого словаря когнитивных терминов>>~\cite{16zat}. В этом словаре 
концептуализация трактуется как один из важнейших процессов познавательной 
деятельности человека, за\-клю\-ча\-ющий\-ся в осмыслении поступающей к нему инфор\-мации и 
приводящей к образованию концептов, концептуальных структур и всей концептуальной 
системы в мозгу (психике) человека. Согласно этой же трактовке концептуализация может 
также рассматриваться как процесс порождения новых смыс\-лов, и тогда предметом 
рассмотрения становятся вопросы о том, как образуются новые концепты, как создание 
нового концепта ограничивается уже име\-ющи\-ми\-ся концептами в системе знаний, как можно 
объяснить способность человека пополнять эту сис\-те\-му~[16, с.~90--93].
   
   В контексте этого словарного определения предлагаемый в статье термин 
<<концептуализация денотатов>> можно трактовать как процесс формирования концептов 
на основе совокупностей изменяемых компьютерных программ и ис\-поль\-зу\-емых ими 
цифровых ресурсов. С~прикладной точки\linebreak зрения концептуализация денотатов является 
ключевым процессом в задачах построения новых индикаторов и формирования их 
смыслового содержания при проектировании и использовании систем информационного 
мониторинга и оценивания научной деятельности~[7, 11, 17--19]. 

   
  В задачах построения новых индикаторов процесс формирования концептов является, как 
правило, целенаправленным и итерационным. В~этом процессе участвуют 
   эксперты-пользователи систем мониторинга и оценивания, целью которых является 
генерация ЦСЗ в процессе их совместной деятельности. Каждая итерация процесса 
\textit{це\-ле\-на\-прав\-лен\-ной концептуализации денотатов} включает три этапа.
   
   На первом этапе эксперты фиксируют некоторую совокупность компьютерных программ, 
их параметров и цифровых ресурсов, используемых в процессе построения ими новых 
индикаторов и вычисления значений этих индикаторов, смысловое содержание которых и 
должны попытаться согласовать между собой эксперты.
   
   На втором этапе поставленная задача вычисления значений индикаторов решается 
экспертами, т.\,е.\ с помощью программ вычисляются значения каждого нового индикатора, 
а затем, используя исходные данные и результаты вычислений, эксперты пытаются 
согласованно интерпретировать новый индикатор. Для каждого нового индикатора итогом 
второго этапа может быть один из трех следующих результатов:
   \begin{enumerate}[(1)]
\item согласованная между экспертами смысловая интерпретация нового 
индикатора при заданной совокупности компьютерных программ, их параметров и 
цифровых ресурсов;
\item несколько разных смысловых интерпретаций нового индикатора, 
полученных на основе исходных данных и результатов вычислений;
\item отсутствие смысловых интерпретаций, т.\,е.\ фиксация экспертами 
невозможности интерпретировать новый индикатор при заданных исходных 
данных и полученных результатах вычислений.
   \end{enumerate}
   
   В последних двух случаях происходит возврат на первый этап. Третий этап выполняется 
только в том случае, если экспертами получена согласованная смысловая интерпретация 
индикатора. Тогда они позиционируют согласованные итоги концептуализации денотата 
индикатора в рамках формируемой ЦСЗ, создавая и/или уточняя дескрипторы тезауруса, 
фиксируют новую стадию ее развития. Затем эксперты сопоставляют новый вариант ЦСЗ с 
теми технологическими или другими общественно значимыми потребностями, ради которых 
этот вариант ЦСЗ был ими сформирован (как правило, с участием лиц, принимающих 
решения). Если эксперты находят полученный вариант неполным или нерелевантным, то 
происходит возврат на первый этап, на котором изменяются компьютерные программы, их 
параметры и используемые программами цифровые ресурсы. Получив желаемый результат, 
эксперты завершают процесс построения новых индикаторов и формирования ЦСЗ.
   
   Приведенное описание примера построения новых индикаторов и формирования ЦСЗ, 
вклю\-чающего процесс концептуализации денотатов\linebreak индикаторов, говорит о том, что заранее 
не гарантируется формирование именно того варианта ЦСЗ, который отвечает 
технологическим или другим общественно значимым потребностям. Начав процесс 
концептуализации, эксперты могут и не получить желаемого варианта ЦСЗ.
   
   Отметим, что по определению процесс кон\-цептуализации денотатов включает в себя 
ите\-рационно повторяемый этап их анализа и интер\-претации. Возможность 
целенаправленного и\linebreak итеративного изменения совокупностей компьютерных программ, их 
параметров и используемых этими программами цифровых ресурсов является существенной 
степенью свободы экспертов в процессе направляемого ими формирования отдельных 
концептов и ЦСЗ в целом.
   
   Однако существуют факторы, которые могут сущест\-венно ограничивать степень свободы 
экспер\-тов, в первую очередь это методический и нормативный факторы, которые 
рассмотрим в контексте Постановления Правительства РФ от 22~мая 2004~года №\,249 
<<О~мерах по повышению результативности бюджетных расходов>> (далее~--- 
Постановление №\,249)~\cite{20zat}. Это постановление преду\-смат\-ри\-ва\-ет введение в 
стране среднесрочного бюджетирования, ориентированного на результаты (СБОР). 
Основные положения СБОР сформулированы в <<Концепции реформирования бюджетного 
процесса в Российской Федерации в~2004--2006~годах>> (далее~--- Концепция), одобренной 
По\-ста\-нов\-ле\-ни\-ем №\,249. Концепция направлена на повышение результативности управления 
бюджетными ресурсами в соответствии с явно определенными приоритетами 
государственной политики.
   
   Согласно методическим рекомендациям по реализации Концепции, формируемые 
экспертами сис\-те\-мы индикаторов и других категорий показателей результативности должны 
соответствовать сле\-ду\-ющим требованиям~\cite{21zat}:
   \begin{itemize}
\item \textit{точность}: погрешности измерения не должны приводить к искаженному 
представлению о результатах деятельности;
\item \textit{объективность}: не допускается использование показателей, улучшение 
значений которых возможно при ухудшении реального положения дел; используемые 
показатели должны в наименьшей степени создавать стимулы к искажению результатов 
деятельности;
\item \textit{достоверность}: способ сбора и обработки исходных данных должен 
допускать возможность проверки точности данных в процессе независимого мониторинга 
и оценивания;
\item \textit{однозначность}: определение показателя должно обеспечивать его одинаковое 
понимание пользователями;
\item \textit{экономичность}: получение исходных данных должно производиться с 
минимально возможными затратами, применяемые показатели должны в максимальной 
степени основываться на уже существующих программах сбора исходных данных;
\item \textit{своевременность и регулярность}: отчетные данные должны поступать с 
определенной периодичностью и с незначительным временным лагом между моментом 
сбора исходных данных и сроком их использования;
\item \textit{уникальность} (\textit{атомарность}): показатели достижения цели не должны 
представлять собой объединение нескольких показателей, характеризующих решение 
отдельных относящихся к этой цели задач.
\end{itemize}

   Перечисленные требования, в том числе достаточно сложное в реализации требование 
однозначности, представляют собой в совокупности пример методического фактора, 
который согласно этим рекомендациям необходимо принимать во внимание пользователям-экспертам 
в процессе построения ими новых индикаторов и формирования их смыс\-ло\-во\-го 
содержания.
   
   Кроме учета методического фактора эксперты, естественно, обязаны следовать 
положениям соответствующих нормативно-правовых документов (нормативный фактор), 
например части~4 ГК РФ при классификации результатов интеллектуальной деятельности, 
для оценивания которых они проектируют новые индикаторы и формируют ЦСЗ.

\section{Заключение}

   Необходимость построения нестационарной семиотической модели компьютерного 
кодирования концептов, информационных объектов и денотатов с учетом необходимости 
фиксации стадий их эволюции во времени обосновывалась актуальностью постановки и 
решения проблем: идентификации концептов ЦСЗ; оценивания релевантности ЦСЗ 
технологическим или другим общественно значимым потребностям, ради удовлетворения 
которых формируется ЦСЗ; направляемого развития ЦСЗ. Построение этой модели в 
виде трех со\-став\-ля\-ющих (два семейства семиотических треугольников~--- Фреге и 
цифровых, а также множество точек, полученных в результате отображения семейства цифровых семиотических 
треугольников на пространство Фреге) представляют собой теоретическую основу для 
решения трех перечисленных проблем.
   
   Проблема идентификации концептов ЦСЗ, эволюционирующих в процессе анализа и 
интерпретации денотатов, может быть решена полностью в рамках предложенной 
нестационарной семиотической модели, если трактовать словосочетание <<идентификация 
концепта>> как построение дискретной <<траектории>> этого концепта в виде серии 
концептуальных вершин цифровых семиотических треугольников в моменты 
времени~$\{t_i\}$. Иными словами, идентификация эволюционирующего концепта, а также 
его денотата и информационного объекта в момент времени~$t_i$ в предлагаемой трактовке 
эквивалентна построению цифрового семиотического треугольника со следующими 
вер\-ши\-нами:
   \begin{itemize}
   \item концептуальная вершина треугольника представляет собой семантический компьютерный код 
эволюционирующего концепта в момент времени~$t_i$ в виде семантического идентификатора 
соответствующего дескриптора тезауруса ИИС;
\item информационная вершина треугольника представляет собой 
информационный компьютерный код информационного объекта как знаковой 
формы представления этого концепта в момент времени~$t_i$ в виде 
информационного идентификатора дескриптора;
\item объектная вершина треугольника представляет собой объектный 
компьютерный код денотата этого концепта в момент времени~$t_i$ в виде 
объектного идентификатора дескриптора.
\end{itemize}

   В такой трактовке словосочетания <<идентификация концепта>> предложенная 
нестационарная семиотическая модель выступает как достаточная теоре\-тическая основа для 
решения проблемы идентификации концептов ЦСЗ, эволюционирующих в процессе анализа 
и интерпретации денотатов. Нестационарная семиотическая модель является достаточной 
теоретической основой, так как множество точек~$\{(t_i, n_{i,j}, m_{i,j}, k_{i,}j)$, где $j = 1,  
\ldots , L_i$, $L_i$~--- число цифровых семиотических треугольников, сгенерированных в 
моменты времени~$t_i$, $i = 1, 2, \ldots \}$ представляет собой полное описание в цифровой 
среде ИИС дискретной <<траектории>> эволюции любого концепта ЦСЗ и его денотата, а 
также <<траектории>> эволюции формы представления этого концепта в 
   социально-ком\-му\-ни\-ка\-ци\-он\-ной среде ИИС.
   
   Однако для описания отношений между концептами, а также для решения проблем 
оценивания релевантности и направляемого развития ЦСЗ предложенная нестационарная 
семиотическая модель представляет собой только необходимую теоретическую основу, так 
как в пространстве Фреге не определена семантическая метрика. Поэтому для этих двух 
проблем необходимо дополнительно определить семантические метрики и ряд функций, 
например функцию конвенциональности концептов ЦСЗ, функцию релевантности всей ЦСЗ 
технологическим или другим общественно значимым потребностям, используя пространство 
Фреге как область определения этих функций.
   
   Главный результат этой статьи заключается в описании теоретической основы 
компьютерного кодирования концептов, информационных \mbox{объектов} и денотатов с учетом их 
эволюции во времени в виде нестационарной семиотической модели. При этом очевидна 
необходимость дальнейшего развития стационарной и нестационарной семиотических 
моделей. В частности, необходимо построить пространство Фреге, обладающее 
семантической метрикой, которое предлагается назвать \textit{семантико-метрическим 
пространством Фреге}, чтобы использовать его для постановки, исследования и решения 
проблем оценивания релевантности и направляемого развития ЦСЗ.
   
   Другим направлением развития этих моделей является описание связей ЦСЗ с другими 
системами знаний. В рассмотренном примере группа новых индикаторов распределения по 
возрастным группам публикационной активности научных сотрудников формировалась по 
аналогии с традиционными индикаторами возрастного распределения научных сотрудников, 
т.\,е.\ использовалась уже сформировавшаяся система знаний об этой группе индикаторов. 
Эту систему знаний можно рас\-смат\-ри\-вать как некоторое начальное (нулевое) состояние 
формируемой ЦСЗ, которое в явном виде отсутствует в рассмотренной модели. Явное 
отражение начального состояния формируемой ЦСЗ, а также ее связей с другими системами 
знаний является еще одним актуальным направлением развития стационарной и 
нестационарной семиотических моделей компьютерного кодирования концептов, 
информационных объектов и денотатов.

{\small\frenchspacing
{%\baselineskip=10.8pt
\addcontentsline{toc}{section}{Литература}
\begin{thebibliography}{99}    
     \bibitem{1zat}
     FP7 Exploratory Workshop~4 ``Knowledge Anywhere Anytime''. {\sf 
http://cordis.europa.eu/ist/directorate\_f/ f\_ws4.htm}.
     
     \bibitem{2zat}
     \Au{Зацман~И.\,М.}
     Семиотическая модель взаимосвязей концептов, информационных объектов и 
компьютерных кодов~// Информатика и её применения, 2009. Т.~3. Вып.~2. С.~65--81.
     
     \bibitem{3zat}
     \Au{Gorn~S.}
     Informatics (computer and information science): Its ideology, methodology, and sociology~//  
The studies of information: Interdisciplinary messages~/ Eds.F.~Machlup, U.~Mansfield.~--- 
New York: John Wiley and Sons, Inc., 1983. P.~121--140.
     
     \bibitem{4zat}
     \Au{Wang~Y.}
     Cognitive informatics: A new transdisciplinary research field~// Brain Mind, 2003. 
Vol.~4. No.\,2. P.~115--127.
     
     \bibitem{5zat}
     \Au{Wang~Y.}
     On cognitive informatics~// Brain Mind, 2003. Vol.~4. No.~2. P.~151--167.
     
     \bibitem{6zat}
     \Au{Bryant~A.}
     Cognitive informatics, distributed representation and embodiment~// Brain Mind, 2003. 
Vol.~4. No.\,2. P.~215--228.
     
     \bibitem{7zat}
     \Au{Зацман~И.\,М., Косарик~В.\,В., Курчавова~О.\,А.}
     Задачи представления личностных и коллективных концептов в цифровой среде~// 
Информатика и её применения, 2008. Т.~2. Вып. ~3. С.~54--69.
     
     \bibitem{8zat}
     \Au{Успенский~В.\,А.}
     К публикации статьи Г.~Фреге <<Смысл и денотат>>~// Семиотика и информатика.~--- 
М.: Языки русской культуры, 1997. Вып.~35. С.~351--352.
     
     \bibitem{9zat}
     \Au{Фреге~Г.}
     Понятие и вещь~// Семиотика и информатика.~--- М.: Языки русской культуры, 1997. 
Вып.~35. С.~380--396.
     
     \bibitem{10zat}
     \Au{Зацман~И.\,М.}
     Концептуальный поиск и качество информации.~--- М.: Наука, 2003.
     
     \bibitem{11zat}
     \Au{Зацман~И.\,М.}
     Концептуализация данных науко\-мет\-рических исследований в научных электронных 
биб\-ли\-о\-те\-ках~// Труды X Всеросс. конф. <<Электронные библиотеки: 
перспективные методы и технологии, электронные коллекции>>.~--- Дубна: \mbox{ОИЯИ}, 2008. 
С.~45--54.
     
     \bibitem{12zat}
     \Au{Колин~К.\,К.}
     О структуре научных исследований по комплексной проблеме <<Информатика>>~// 
Социальная информатика.~--- М.: ВКШ при ЦК ВЛКСМ, 1990. С.~19--33.
     
     \bibitem{13zat}
     \Au{Колин~К.\,К.}
     Эволюция информатики и проблемы формирования нового комплекса наук об 
информации~// Научно-техническая информация, 1995. Сер.~1. №\,5. С.~1--7.
     
     \bibitem{14zat}
     \Au{Зацман~И.\,М.}
     Концептуальный поиск информационных объектов в электронных библиотеках 
научных документов~// Компьютерная лингвистика и интеллектуальные технологии: Труды 
международной конференции <<Диалог-2003>>.~--- М.: Наука, 2003. С.~710--716.
     
     \bibitem{15zat}
     \Au{Gladney~H.\,M., Bennet~J.\,L.}
     What do we mean by authentic? What's the real McCoy?~// D-Lib Magazine, 2003. Vol.~9. 
No.\,7/8.
     
     \bibitem{16zat}
     \Au{Кубрякова~Е.\,С., Демьянков~В.\,З., Панкрац~Ю.\,Г., Лузина~Л.\,Г.}
     Краткий словарь когнитивных терминов~/ Под общ. ред. Е.\,С.~Кубряковой.~--- М.: 
Филфак МГУ, 1996.
     
     \bibitem{17zat}
     \Au{Клейнер~Г.\,Б., Голиченко~О.\,Г., Зацман~И.\,М.}
     Основные принципы разработки системы мониторинга функционирования 
исследовательских организаций.~--- М.: ЦЭМИ РАН, 2007. 61~с.
     
     \bibitem{18zat}
     \Au{Зацман~И.\,М., Верёвкин~Г.\,Ф., Шубников~С.\,К.}
     Моделирование систем мониторинга.~--- М.: ИПИ РАН, 2008.  115~с.
     
     \bibitem{19zat}
     \Au{Zatsman~I., Kozhunova~O.}
     Evaluating for institutional academic activities: classification scheme for R\&D indicators~// 
10th  Conference (International) on Science and Technology Indicators (STI'2008) Proceedings. 
September 17--20, 2008.~--- Vienna: ARC GmbH, 2008. P.~428--431.
     
     \bibitem{20zat}
     \Au{Зацман~И.\,М.}
     Терминологический анализ нор\-ма\-тив\-но-пра\-во\-во\-го обеспечения создания систем 
мониторинга в сфере науки~// Экономическая наука современной России, 2005. №\,4. 
С.~114--129.

\label{end\stat}
     
     \bibitem{21zat}
     Методические рекомендации по подготовке докладов о результатах и основных 
направлениях дея\-тель\-ности субъектов бюджетного планирования на 2006--2008~годы. 
{\sf http://www1.minfin.ru/budref/ metod\_270705.zip)}.
 
 
\end{thebibliography}
}
}
\end{multicols}