\def\stat{kudr}


\def\tit{БАЙЕСОВСКИЕ МОДЕЛИ МАССОВОГО ОБСЛУЖИВАНИЯ И~НАДЕЖНОСТИ:
ОБЩИЙ ЭРЛАНГОВСКИЙ СЛУЧАЙ$^*$}
\def\titkol{Байесовские модели массового обслуживания и~надежности:
общий эрланговский случай} 

\def\autkol{А.\,А.~Кудрявцев, В.\,С.~Шоргин, С.\,Я.~Шоргин}
\def\aut{А.\,А.~Кудрявцев$^1$, В.\,С.~Шоргин$^2$, С.\,Я.~Шоргин$^3$}

\titel{\tit}{\aut}{\autkol}{\titkol}

{\renewcommand{\thefootnote}{\fnsymbol{footnote}}\footnotetext[1]
{Работа выполнена при поддержке РФФИ, проекты 08-07-00152-а, 08-01-00567-а и 09-07-12032-офи-м.}}

\renewcommand{\thefootnote}{\arabic{footnote}}
\footnotetext[1]{Факультет
вычислительной математики и кибернетики Московского государственного
университета им. М.~В.~Ломоносова, nubigena@hotmail.com}
\footnotetext[2]{Институт проблем информатики Российской академии наук, vshorgin@ipiran.ru}
\footnotetext[3]{Институт проблем информатики Российской академии наук, sshorgin@ipiran.ru}


%\input macros.tex

\vspace*{-6pt}


\Abst{Данная работа является очередной в серии статей, посвященных изучению байесовских моделей массового обслуживания и
надежности. Соответствующий метод предусматривает рандомизацию характеристик системы относительно некоторых априорных
распределений ее параметров. В работе представлены новые результаты для случая, когда в качестве пары априорных
распределений рас\-смат\-ри\-ва\-ют\-ся распределения Эрланга с различными наборами параметров, а также пара <<распределение
Эрланга~--- вырожденное распределение>>.}

\KW{байесовский подход; системы массового обслуживания; надежность; смешанные
распределения; моделирование; эрланговское распределение}

      \vskip 16pt plus 9pt minus 6pt

      \thispagestyle{headings}

      \begin{multicols}{2}

      \label{st\stat}

\section{Введение и основные предположения}

Подробное изложение основ байесовского подхода к моделированию систем массового обслуживания (СМО) и ненадежных
восстанавливаемых сис\-тем содержится в~[1, 2]. Здесь коснемся этого вопроса лишь вкратце.

В реальной практике нередки ситуации, когда исследуемая система задана в определенном смысле <<неточно>>. Скажем, если
даже говорить о простейших системах типа $M\vert G\vert 1$, исследователю может быть априори неизвестен параметр входящего 
потока~$\lambda$ и параметры обслуживания~$\mu$ и~$\sigma^2$. В этом случае естественным оказывается рандомизационный подход,
при котором элементами вероятностного пространства становятся (если рассматривать приведенный выше пример) значения
$\lambda$, $\mu$ и $\sigma^2$. При этом подлежащие вычислению характеристики такой <<рандомизированной>> СМО,
естественно, являются рандомизацией аналогичных характеристик <<обычной>> СМО подобного типа~--- с учетом того
априорного распределения входных па\-ра\-мет\-ров СМО, которое взято исследователем за основу.

Таким образом, в том же примере с системой типа $M\vert G\vert 1$ возникают задачи рандомизации 
<<обычных>> характеристик таких
систем с учетом априорных распределений входных параметров. Скажем, может приниматься предположение о показательном,
равномерном или каком-то другом распределении одной или нескольких величин из~$\lambda$, $\mu$ и~$\sigma^2$ (которые
при таком подходе становятся случайными величинами), об их независимости или зависимости и~т.\,п. Полученные результаты
могут применяться, например, для вычисления средних значений, построения доверительных интервалов для тех или иных
характеристик рассматриваемого класса СМО <<в целом>>. Такой подход к по\-стро\-ению моделей массового обслуживания
естественно назвать байесовским.

Другим направлением применения байесовского подхода является оценка надежности. Как известно (см.~[3]), коэффициент
готовности восстанавливаемого устройства в стационарном режиме может быть вычислен по формуле
$$
k=\fr{\lambda^{-1}}{\lambda^{-1}+\mu^{-1}}=\fr{\mu}{\lambda+\mu}\,,
$$
где $\lambda^{-1}$~--- среднее время безотказной работы, $\mu^{-1}$~--- среднее время восстановления. Если принять
сформулированное выше предположение, в соответствии с которым любое изучаемое устройство\linebreak выбирается случайным образом
из некоторого множества сходных устройств, различающихся средними величинами показателей надежности, то согласно
приведенным выше рассуждениям значения~$\lambda$ и~$\mu$ могут рассматриваться в качестве случайных.
Следовательно, при таких предположениях коэффициент готовности~$k$ также является случайной величиной и его
распределение зависит от распределений величин~$\lambda$ и~$\mu$.

Основным объектом рассмотрения на настоящем этапе является СМО
$M\vert M\vert 1$, в которой интенсивность входящего
потока~$\lambda$ и интенсив-\linebreak\vspace*{-12pt}
\pagebreak

\noindent
ность обслуживания~$\mu$ независимы и
имеют некоторые априори известные распределения. При этом загрузка
рассматриваемой системы имеет вид $\rho=\lambda/\mu$. Как
известно, от значения~$\rho$ зависит наличие стационарного режима
у рассматриваемой сис\-те\-мы; величина~$\rho$ входит во многие
формулы, описывающие характеристики разнообразных СМО. В~связи с
этим рассмотрение величины~$\rho$ выбрано одной из первоочередных
задач, которые следует рассмотреть в рамках байесовской теории
СМО. Кроме того, рассматриваются распределения такой
рандомизированной характеристики, как вероятность потерь $1-\pi$
(здесь $\pi$~--- вероятность того, что входящий в СМО вызов не
будет потерян).

Отметим, что значение коэффициента го\-тов\-ности~$k$ совпадает с величиной
\vspace*{2pt}

\noindent
$$
\pi=\fr{1}{1+\rho}=\fr{\mu}{\lambda+\mu}
$$
для системы $M\vert M\vert 1$, в которой~$\lambda$ --- интенсивность входящего потока, а  
$\mu$~--- интенсивность обслуживания.
Поэтому вычисление вероятностных характеристик величины~$\pi$ означает одновременное вычисление вероятностных
характеристик величины~$k$ при соответствующих распределениях среднего времени безотказной работы и среднего времени
восстановления.

В настоящей работе всюду говорится о на\-хож\-де\-нии распределений величин, относящихся к байесовской модели СМО, 
включая~$\pi$. При этом подразумевается, что распределение величины~$k$ в\linebreak соответствующей <<надежностной>> постановке
специально вычислять не нужно, поскольку оно совпадает с распределением величины~$\pi$.

Данная работа является логическим продолжением статей~\cite{1kud, 2kud, 4kud}, в которых авторы рас\-смат\-ри\-ва\-ли вероятностные
характеристики коэффициента загрузки $\rho$ и вероятности потерь $1-\pi$ в\linebreak
сис\-теме $M\vert M\vert 1\vert 0$ в предположении, что пару\linebreak
априорных распределений (т.\,е.\ пару <<распределение параметра входящего по\-то\-ка\,--\,рас\-пре\-де\-ле\-ние 
па\-ра\-мет\-ра
обслуживания>>) составляют:  рав\-но\-мер\-ное--рав\-но\-мер\-ное, 
экспо\-нен\-ци\-аль\-ное--экс\-по\-нен\-ци\-аль\-ное, вы\-рож\-ден\-ное\,--\,распределение 
Эрланга, экс\-по\-нен\-ци\-аль\-ное\,--\,рас\-пре\-де\-ле\-ние Эрланга и распределение Эрланга\,--\,экспоненциальное 
(естественно, одновременно вычислялись характеристики коэффициента готовности~$k$ в соответствующей
<<надежностной>> постановке).

В настоящей статье рассматриваются пары  <<распределение Эрланга\,--\,распределение Эрланга>> и <<распределение 
Эрланга\,--\,вырожденное>>.

В дальнейшем авторы предполагают продолжить расширение множества пар априорных распределений, по которым производится
ран-\linebreak\vspace*{-12pt}
\columnbreak


\vspace*{-6pt}
\noindent %tabl8
\begin{center}
\parbox{60mm}{{\tablename~1}\ \ \small{Этапы рассмотрения задачи}}
\end{center}
\vspace*{2pt}

{\small
\begin{center}
\tabcolsep=10pt
\begin{tabular}{|c|c|c|c|c|c|}
\hline
\multicolumn{1}{|c|}{\raisebox{-4pt}[0pt][0pt]{$\lambda $}}&\multicolumn{5}{c|}{$\mu$}\\
\cline{2-6}
&D&M&R&E&P\\
\hline
D&*&$+$&$+$&$+$&$-$\\
%\hline
M&$\oplus$&$+$&$-$&$+$&$-$\\
%\hline
R&$+$&$-$&$+$&$-$&$-$\\
%\hline
E&$\oplus$&$+$&$-$&$\oplus$&$-$\\
%\hline
P&$-$&$-$&$-$&$-$&$-$\\
\hline
\end{tabular}
\end{center}
}
%\end{table}
\vspace*{12pt}


\bigskip
\addtocounter{table}{1}


\noindent
домизация параметров~$\lambda$ и~$\mu$. В табл.~1 отображены этапы рассмотрения предложенной задачи. Буквы~D, M,
R, E, P обозначают вырожденное, экспоненциальное, равномерное, Эрланга, Парето распределения соответственно; 
символ~<<$*$>> относится к классической постановке задачи, символ~<<$+$>> соответствует уже рассмотренным ранее
распределениям, символ~<<$\oplus$>>~--- распределениям, о которых пойдет речь в данной работе, символом~<<$-$>>
обозначаются распределения, для которых авторы планируют получить аналогичные результаты в дальнейшем.


\section{Основные результаты}

Приведем несколько утверждений, опи\-сы\-ва\-ющих основные вероятностные характеристики коэффициента загрузки
$\rho=\lambda/\mu$ и вероятности <<непотери>> вызова $\pi=(1+\rho)^{-1}$ в системе~$M\vert M\vert 1\vert 0$.

В дальнейшем изложении будем использовать стандартное обозначение для $\beta$-функции:
$$
B(x, y)=\int\limits_0^1t^{x-1}(1-t)^{y-1}\, dt\,.
$$

Для формулирования следующей теоремы понадобится несколько соотношений, являющихся непосредственными следствиями
формулы~2.111 из~[5]. Для $m\ge2$
\begin{multline}
\int\fr{x^n\, dx}{(a+bx)^{n+m}}={}\\
 {}= -\fr{x^n}{(m-1)b
C_{n+m-1}^n(a+bx)^{n+m-1}}\times{}\\
{}\times
\sum\limits_{l=0}^n C_{n+m-1}^{n-l}\left(\fr{a}{bx} \right)^l+C\,;
\end{multline}
\vspace*{-12pt}

\noindent
\begin{multline}
\!\int\fr{x^n\, dx}{(a+bx)^{n+1}}=-\sum\limits_{k=0}^{n-1}\fr{x^{n-k}}{(n-k)b^{k+1}(a+bx)^{n-k}}\;+{}\\
{}+\fr{\ln\left(a+bx\right)}{b^{n+1}}+C\,;
\end{multline}

\vspace*{-12pt}
\noindent
\begin{multline}
\int\fr{x^n\, dx}{(a+bx)^n}=\fr{x^n}{b(a+bx)^{n-1}}+{}\\
{}+ \fr{an}
{b}\sum\limits_{k=0}^{n-2}\fr{x^{n-k-1}}{(n-k-1)b^{k+1}(a+bx)^{n-k-1}}-{}\\
{}- 
\fr{an}{b^{n+1}}\ln\left(a+bx\right)+C\,,
\end{multline}
где $C$~--- некоторая константа.


Введем обозначение

\noindent
\begin{equation}
A(p,q)=\int\limits_0^1 \fr{x^p\,dx}{(\theta+(\alpha-\theta)x)^q}\,.
\label{e4kud}
\end{equation}

Заметим, что из соотношений~(1)--(3) следует, что при $m=n,\,\ldots,\,n+k-2$
\begin{multline}
\!\!\!A(m,n+k)=-\fr{(k+n-m-1)^{-1}}{C_{n+k-1}^{m}\alpha^{n+k-1}}\sum\limits_{j=0}^{m}\fr{C_{n+k-1}^{m-j}\theta^j}{(\alpha
-\theta)^{j+1}}+{}\\
{}+
\fr{(k+n-m-1)^{-1}\theta^{m-n-k+1}}{C_{n+k-1}^{m}(\alpha-\theta)^{m+1}}\,;
\label{e5kud}
\end{multline}
\vspace*{-12pt}

\noindent
\begin{multline}
\!\!A(n+k-1,n+k)={}\\
{}=-\sum\limits_{i=0}^{n+k-2}\fr{(n+k-i-1)^{-1}}{\alpha^{n+k-i-1}(\alpha-\theta)^{i+1}}+
\fr{\ln \left (\alpha/ \theta\right)} {(\alpha-\theta)^{n+k}}\,;
\label{e6kud}
\end{multline}
\vspace*{-12pt}

\noindent
\begin{multline}
A(n+k,n+k)=\fr{1}{\alpha^{n+k-1}(\alpha-\theta)}+{}\\
{}+\fr{(n+k)\theta}{\alpha-\theta}\sum\limits_{j=0}^{n+k-2}\fr{(n+k-j-1)^{-1}
}{\alpha^{n+k-j-1}(\alpha-\theta)^{j+1}}-{}\\
{}-
\fr{(n+k)\theta\ln\left (\alpha/\theta\right )}{(\alpha -\theta)^{n+k+1}}\,.
\label{e7kud}
\end{multline}

\smallskip

\noindent
\textbf{Теорема 1.} 
{\it Пусть интенсивность входящего потока~$\lambda$ и интенсивность обслуживания~$\mu$ имеют распределение
Эрланга с параметрами $k\ge2$, $\theta>0$ и $n\ge2$, $\alpha>0$ соответственно, причем~$\lambda$ и~$\mu$ независимы.
Тогда функции распределения случайных величин~$\rho$ и~$\pi$ имеют вид}
%\noindent
\begin{equation*}
F_\rho(x)=\left(\fr{\theta x}{\alpha+\theta x}\right)^{n+k-1}\sum\limits_{m=0}^{n-1}\left(\fr{\alpha}{\theta
x}\right)^m\,,\quad x>0\,;
\end{equation*}
\vspace*{-16pt}

\noindent
\begin{multline*}
F_\pi(x)=1-{}\\
{}-\left(\fr{\theta(1-x)}{\alpha x+\theta(1-x)}\right)^{n+k-1}\sum\limits_{m=0}^{n-1}C_{n+k-1}^m\left(\fr{\alpha
x}{\theta(1-x)}\right)^m\!,\\ %quad 
x\in (0,\,1)\,;
\end{multline*}


\noindent
\textit{плотности распределения случайных величин~$\rho$ и~$\pi$ имеют вид}

\vspace*{3pt}

\noindent
\begin{equation*}
f_\rho(x)=\fr{(n+k-1)!}{(k-1)!(n-1)!}\,\fr{\theta^k\alpha^nx^{k-1}}{(\theta x+\alpha)^{n+k}}\,,\quad x>0\,;
\end{equation*}

\columnbreak

\noindent
\begin{multline}
f_\pi(x)=\fr{(n+k-1)!\theta^k\alpha^n}{(k-1)!(n-1)!}\,\fr{(1-x)^{k-1}x^{n-1}}{(\theta+(\alpha-\theta)x)^{n+k}}\,,\\
x\in(0,\,1)\,; 
\label{e8kud}
\end{multline}
\textit{для первых двух моментов соответствующих распределений справедливы соотношения}

\noindent
\begin{gather*}
\e\rho=\fr{k\alpha}{(n-1)\theta}\,;\\
\e\rho^2=\fr{k(k+1)\alpha^2}{(n-1)(n-2)\theta^2}\ \ {\mbox{(\textit{для}
$n\ge3$)}}\,;
\end{gather*}
\vspace*{-6pt}

\noindent
\begin{equation}
\left.
\begin{array}{rcl}
\e\pi\!\!\!&=&\!\!\!nC_{n+k-1}^nB(n+1, k)\,;\\[6pt]
\e\pi^2\!\!\!&=&\!\!\!nC_{n+k-1}^nB(n+2, k) \ \ {\mbox{(\textit{для} $\alpha=\theta$)}}\,;
\end{array}
\right \}
\label{e9kud}
\end{equation}
\vspace*{-12pt}

\noindent
\begin{multline}
\e\pi={}\\[-3pt]
\!\!{}=nC_{n+k-1}^n\theta^k\alpha^n\sum\limits_{l=0}^{k-1}(-1)^lC_{k-1}^lA(n+l,n+k);\!
\label{e10kud}
\end{multline}

\vspace*{-14pt}

\noindent
\begin{multline}
\!\e\pi^2=nC_{n+k-1}^n\theta^k\alpha^n\sum\limits_{l=0}^{k-1}(-1)^lC_{k-1}^lA(n+l+1,n+k)\\
{\mbox{(\textit{для} $\alpha\neq\theta$)}},
\label{e11kud}
\end{multline}
\textit{где интегралы $A(m,n+k)$ вычисляются по формулам}~(5)--(7).

\smallskip

\noindent
Д\,о\,к\,а\,з\,а\,т\,е\,л\,ь\,с\,т\,в\,о\,.\ Поскольку функции распределения случайных величин~$\rho$ и~$\lambda$ связаны соотношением

\noindent
$$
F_\rho(x)=\int\limits_0^\infty F_\lambda(xy)\fr{y^{n-1}\alpha^ne^{-\alpha y}}{(n-1)!}\, dy\,,
$$
для плотности~$\rho$ получаем

\noindent
\begin{multline*}
f_\rho(x)=\int\limits_0^\infty\fr{x^{k-1}y^k\theta^ke^{-\theta xy}}{(k-1)!}\,\fr{y^{n-1}\alpha^ne^{-\alpha
y}}{(n-1)!}\, dy={}\\
{}=\fr{x^{k-1}\theta^k\alpha^n}{(k-1)!(n-1)!(\theta x+\alpha)^{n+k}}\int\limits_0^\infty z^{n+k-1}e^{-z}\,
dz={}\\
{}=
\fr{(n+k-1)!}{(k-1)!(n-1)!}\,\fr{\theta^k\alpha^nx^{k-1}}{(\theta x+\alpha)^{n+k}}\,, \quad x>0\,.
\end{multline*}

Для нахождения соответствующей функции распределения воспользуемся соотношением~(1). %\linebreak\vspace*{-12pt}
 Путем несложных арифметических
преобразований получаем

\vspace*{-4pt}

%\pagebreak

\noindent
\begin{multline*}
F_\rho(x)=\fr{(n+k-1)!\theta^k\alpha^n}{(k-1)!(n-1)!}\int\limits_0^x \fr{t^{k-1}\,
dt}{(\theta t+\alpha)^{k+n}} ={}\\
{}=
\left(\fr{\theta x}{\alpha+\theta
x}\right)^{n+k-1}\sum\limits_{m=0}^{n-1}\left(\fr{\alpha}{\theta x}\right)^m\,,\quad x>0\,.
\end{multline*}
%\pagebreak

Теперь воспользуемся тем же соотношением для вычисления первых моментов~$\rho$. Имеем

\noindent
\begin{multline*}
\e\rho=\fr{(n+k-1)!\theta^k\alpha^n}{(k-1)!(n-1)!}\int\limits_0^\infty \fr{x^k\, dx}{(\alpha+\theta x)^{n+k}}={}\\
{}=-\fr{(n+k-1)!\theta^k\alpha^n}{(k-1)!(n-1)!}\times{}\\
{}\times \left[\fr{x^k}{(n-1)\theta
C_{k+n-1}^k(\alpha+\theta x)^{k+n-1}}\times{}\right.\\
\left.{}\times \sum\limits_{l=0}^kC_{k+n-1}^{k-l}\left(\fr{\alpha}{\theta x}
\right)^l\right]\Bigg|_0^\infty= \fr{k\alpha}{(n-1)\theta}\,.
\end{multline*}
Аналогично при $n\ge3$ получаем
\begin{multline*}
\e\rho^2=\fr{(n+k-1)!\theta^k\alpha^n}{(k-1)!(n-1)!}\int\limits_0^\infty \fr{x^{k+1}\, dx}{(\alpha+\theta
x)^{n+k}}={}\\
{}=\fr{k(k+1)\alpha^2}{(n-1)(n-2)\theta^2}\,.
\end{multline*}

Теперь рассмотрим характеристики вероятности <<непотери>> вызова~$\pi$. Для функции распределения имеем

\vspace*{-2pt}

\noindent
\begin{multline*}
F_\pi(x)=1-\p\left(\rho<\fr{1-x} {x}\right)=
1-{}\\
{}-
\left(\fr{\theta(1-x)}{\alpha
x+\theta(1-x)}\right)^{n+k-1}\sum\limits_{m=0}^{n-1}C_{n+k-1}^m\left(\fr{\alpha x}{\theta(1-x)}\right)^m\!,\\
x\in(0,\,1)\,.
\end{multline*}

Для вычисления плотности случайной величины~$\pi$ достаточно лишь воспользоваться очевидной формулой

\vspace*{-2pt}

\noindent
$$
f_\pi(x)=\fr{1}{x^2}\,f_\rho\left(\fr{1-x}{x}\right)\,,
$$
из которой и получаем соотношение~(8).

Найдем первые моменты случайной величины~$\pi$. В случае $\alpha=\theta$ непосредственно из определения $\beta$-функции
и вида плотности~(8) получаем формулу~(9). Для вычисления моментов $\pi$ при $\alpha\neq\theta$
достаточно воспользоваться соотношениями~(5)--(7). Имеем

\vspace*{-2pt}

\noindent
\begin{multline*}
\e\pi=nC_{n+k-1}^n\theta^k\alpha^n\int\limits_0^1\fr{(1-x)^{k-1}x^{n}}{(\theta+(\alpha-\theta)x)^{n+k}}\, dx={}\\
{}=
nC_{n+k-1}^n\theta^k\alpha^n\int\limits_0^1\fr{\sum_{l=0}^{k-1}(-1)^lC_{k-1}^lx^{n+l}}{(\theta+
(\alpha-\theta)x)^{n+k}}\,dx={}\\
%\end{multline*}
%\begin{equation*}
{}=
nC_{n+k-1}^n\theta^k\alpha^n\sum_{l=0}^{k-1}(-1)^lC_{k-1}^lA(n+l,n+k)\,,
\end{multline*} %equation*}
где интегралы $A(n+l,n+k)$ вычисляются по формулам~(5) и~(6) (нужно положить $m=n+l$).
Аналогично

\vspace*{-6pt}

\noindent
\begin{multline*}
\e\pi^2=nC_{n+k-1}^n\theta^k\alpha^n\int\limits_{0}^{1}\fr{(1-x)^{k-1}x^{n+1}}{(\theta+(\alpha-\theta)x)^{n+k}}\, dx={}\\
{}
=nC_{n+k-1}^n\theta^k\alpha^n\sum_{l=0}^{k-1}(-1)^lC_{k-1}^lA(n+l+1,n+k)\,,
\end{multline*}
где интегралы $A(n+l+1,n+k)$ вычисляются по формулам~(5)--(7) (нужно положить $m=n+l+1$).

Теорема полностью доказана.

\smallskip

\noindent
\textbf{Замечание.} При $n=2$ не существует второго момента коэффициента загрузки~$\rho$. 
В случае $n=1$ или $k=1$ также не
существует математического ожидания~$\rho$~\cite{4kud}.

\smallskip

Рассмотрим описанную выше постановку задачи, в которой интенсивность входящего потока~$\lambda$ имеет
распределение Эрланга, а интенсивность обслуживания~$\mu$ имеет вырожденное распределение.

Пусть $Ei(x)$~--- интегральная показательная функция
\vspace*{-2pt}

\noindent
$$
Ei(x)=-\int\limits_{-x}^{\infty}\fr{e^{-t}}{t}\, dt\,.
$$


\vspace*{-6pt}
\smallskip

\noindent
\textbf{Теорема 2.} {\it Пусть случайная величина~$\lambda$ имеет распределение Эрланга с параметрами $k\ge1$ 
и $\theta>0$, а
случайная величина~$\mu$ имеет вырожденное распределение, причем~$\lambda$ и~$\mu$ независимы.
Тогда коэффициент загрузки~$\rho$ имеет распределение Эрланга с па\-ра\-мет\-ра\-ми~$k$ и $\mu\theta$, а характеристики
вероятности <<непотери>> вызова~$\pi$ определяются соотношениями}
\vspace*{-2pt}

\noindent
\begin{multline*}
f_\pi(x)=\fr{\mu^k\theta^k(1-x)^{k-1}\exp\left\{-\mu\theta (1-x)/x\right\}}{(k-1)!x^{k+1}}\,,\\
 x\in(0,\,1)\,;
 \end{multline*}
 \vspace*{-20pt}
 
 \noindent
 \begin{multline*}
\e\pi=\fr{(-1)^k\mu^k\theta^k}{(k-1)!}\left[e^{\mu\theta}Ei(-\mu\theta)+{}\right.\\
\left.{}+\sum\limits_{n=1}^{k-1}\fr{(-1)^{n+1}(n-1)! }{\mu^n\theta^n}\right]\,;
\end{multline*}

%\pagebreak

\vspace*{-5pt}

\noindent
\begin{equation}
\e\pi^2=\mu\theta-\mu\theta\e\pi\ \ {\mbox{(для $k=1$)}}\,;\label{e12kud}
\end{equation}

\vspace*{-11pt}

\noindent
\begin{multline}
\e\pi^2=\fr{(-1)^k\mu^k\theta^k}{(k-2)!}\left[-e^{\mu\theta}Ei(-\mu\theta)+{}\right.\\
\left.{}+\sum\limits_{n=1}^{k-2}\fr{(-1)^n(n-1)!}
{\mu^n\theta^n}\right ]-\mu\theta\e\pi\enskip
{\mbox{(для $k\ge2$)}}\,.
\label{e13kud}
\end{multline}


\smallskip

\noindent
Д\,о\,к\,а\,з\,а\,т\,е\,л\,ь\,с\,т\,в\,о\,.\ Заметим, что
\vspace*{-4pt}

\noindent
\begin{multline*}
F_\rho(x)=\int\limits_{0}^{\mu x}\fr{t^{k-1}\theta^ke^{-\theta t}}{(k-1)!}\,dt={}\\
{}=
\int\limits_{0}{x}\fr{z^{k-1}\mu^{k}\theta^ke^{-\mu\theta
z}}{(k-1)!}\,dz\,,\quad x>0\,,
\end{multline*}
что соответствует распределению Эрланга с па\-ра\-мет\-ра\-ми~$k$ и~$\mu\theta$. Характеристики такого распределения
хорошо известны.

Для плотности вероятности <<непотери>> вызова~$\pi$ имеем

\vspace*{-6pt}

\noindent
\begin{multline*}
f_\pi(x)=\fr{1}{x^2}\,f_\rho\left(\fr{1-x}{x}\right)={}\\
\!{}=\fr{\mu^k\theta^k(1-x)^{k-1}\exp\left\{-\mu\theta(1-x)/x\right\}}{(k-1)!x^{k+1}}\,,\enskip  x\in(0,\,1)\,.
\hspace*{-7.46pt}
\end{multline*}

Для вычисления моментов случайной величины~$\pi$ воспользуемся введенным выше обозначением~$Ei(x)$.
Тогда по формуле~3.353.5 из~\cite{5kud}

\vspace*{-4pt}

\noindent
\begin{multline*}
\!\e\pi=\int\limits_{0}^{1}\fr{\mu^k\theta^k(1-x)^{k-1}\exp\left\{-\mu\theta(1-x)/x\right\}}{(k-1)!x^k}\,dx={}\\
{}=
\fr{\mu^k\theta^k}{(k-1)!}\int\limits_{0}^{\infty}\fr{t^{k-1}e^{-\mu\theta t}}{t+1}\,dt=
\fr{(-1)^k\mu^k\theta^k}{(k-1)!}\times{}\\
{}\times
\left[
%\vphantom{\sum\limits_{n=1}^{k-1}}
e^{\mu\theta}Ei(-\mu\theta)+\sum\limits_{n=1}^{k-1}\fr{(-1)^{n+1}(n-1)!
}{\mu^n\theta^n}\right]\,;
\end{multline*}

\vspace*{-14pt}

\noindent
\begin{multline*}
\e\pi^2={}\\
{}=\int\limits_{0}^{1}\fr{\mu^k\theta^k(1-x)^{k-1}\exp\left\{-\mu\theta (1-x)/x)\right\}}{(k-1)!x^{k-1}}\,dx={}\\
{}=\fr{\mu^k\theta^k}{(k-1)!}\int\limits_{0}^{\infty}\fr{t^{k-1}e^{-\mu\theta
t}}{(1+t)^2}\,dt\,.
\end{multline*}

\noindent
Последнее равенство при $k=1$ принимает вид

\vspace*{1pt}

\noindent
$$
\e\pi^2=\mu\theta-\mu^2\theta^2\int\limits_{0}^{\infty}\fr{e^{-\mu\theta t}}{t+1}\,dt\,,
$$
а при $k\ge2$~---

\vspace*{-6pt}

\noindent
\begin{multline*}
\e\pi^2=\fr{\mu^k\theta^k}{(k-2)!}\int\limits_{0}^{\infty}\fr{t^{k-2}e^{-\mu\theta
t}}{t+1}\,dt-{}\\
{}-\fr{\mu^{k+1}\theta^{k+1}}{(k-1)!}\int\limits_{0}^{\infty}\fr{t^{k-1}e^{-\mu\theta t}}{t+1}\,dt\,,
\end{multline*}
откуда и следуют соотношения~(\ref{e12kud}) и~(\ref{e13kud}). Теорема доказана.
%\columnbreak

\smallskip

\noindent
\textbf{Следствие.} \textit{Отдельный интерес для исследователя может представлять частный случай распре-}\linebreak\vspace*{-12pt}
\columnbreak

\noindent
 \textit{деления
Эрланга для интенсивности входящего потока $\lambda$,
 а именно~--- экспоненциальное распределение с па\-ра\-мет\-ром $\theta>0$.
Очевидно, что при этом предположении получаются следующие соотношения:}

\vspace*{-8pt}

\noindent
\begin{align*}
f_\rho(x)&=\mu\theta e^{-\mu\theta x}\,,\quad x>0\,;\\
F_\pi(x)&=\exp\left\{-\mu\theta\fr{1-x}{x}\right\}\,,\quad x\in(0,\,1)\,;\\
f_\pi(x)&=\fr{\mu\theta}{x^2}\, \exp\left\{-\mu\theta\fr{1-x}{x}\right\}\,,\quad x\in(0,\,1)\,;\\
\e\pi&=-\mu\theta e^{\mu\theta}Ei\left(-\mu\theta\right)\,;\\
\e\pi^2&=\mu\theta+\mu^2\theta^2e^{\mu\theta}Ei\left(-\mu\theta\right)\,.
\end{align*}

\vspace*{-24pt}
\section{Заключение}

\vspace*{-1pt}

Полученные в работах~\cite{1kud, 2kud, 4kud} и в настоящей статье расчетные формулы для определения параметров рандомизированных
характеристик обслуживания и надежности находят применение в\linebreak исследовании реальных информационных и технических систем.
В настоящее время разработан пакет программ, который осуществляет расчет указанных характеристик для базового набора
априорных распределений. Ведется работа по расширению возможностей этого пакета.

Дальнейшее продвижение в рамках данной проб\-ле\-ма\-ти\-ки требует рассмотрения других
 априорных распределений величин~$\lambda$, 
$\mu$ и других тра\-ди\-ционных входных параметров для СМО и\linebreak вос\-ста\-нав\-ли\-ва\-емых устройств, которые могут
представлять интерес для практики, вычисления со\-от\-вет\-ст\-ву\-ющих распределений показателей функционирования и надежности
различных типов систем (в том числе систем вида $M\vert G\vert 1$, $M\vert M\vert n\vert 0$ и~др.) 
после их рандомизации с учетом наиболее важных для практики априорных распределений параметров.

\vspace*{-12pt}

{\small\frenchspacing
{\baselineskip=10pt
\addcontentsline{toc}{section}{Литература}
\begin{thebibliography}{9}    
\vspace*{-1pt}

\bibitem{1kud}
\Au{D'Apice~C., Manzo R., Shorgin~S.}
Some Bayesian queueing and reliability models~// 
Electronic J.\ ``Reliability: Theory \& Applications'', 2006. Vol.~1. No.\,4.

\bibitem{2kud}
\Au{Кудрявцев А.\,А., Шоргин~С.\,Я.}
Байесовский подход к анализу систем массового обслуживания и показателей надежности~// 
Информатика и её применения, 2007. Т.~1. Вып.~2. С.~76--82.

\bibitem{3kud}
\Au{Kozlov B.\,A., Ushakov~I.\,A.}
Reliability Handbook.~--- New York: Holt, Rinehart \& Winston, 1970.

\bibitem{4kud}
\Au{Кудрявцев А.\,А., Шоргин~С.\,Я.}
Байесовские модели массового обслуживания и надежности: экспо\-нен\-ци\-аль\-но-эр\-лан\-гов\-ский случай~// 
Информатика и её применения, 2009. Т.~3. Вып.~1. С.~44--48.


%\bibitem{Dwhite66}
%\textit{Двайт Г.\/}
%Таблицы интегралов и другие математические формулы / Пер. с англ. --- М.: Наука, 1966. -- 228 с.

\bibitem{5kud}
\Au{Градштейн И.\,С., Рыжик~И.\,М.}
Таблицы интегралов, сумм, рядов и произведений.~--- М.: Наука, 1971. 1108~с.

\label{end\stat}
 \end{thebibliography}
}
}
%\pagebreak
\end{multicols}