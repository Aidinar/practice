\def\stat{rez}
{%\hrule\par
%\vskip 7pt % 7pt
\raggedleft\Large \bf%\baselineskip=3.2ex
Р\,Е\,Ц\,Е\,Н\,З\,И\,И \vskip 17pt
    \hrule
    \par
\vskip 6pt plus 6pt minus 3pt }


\def\tit{МОНОГРАФИЯ И.\,Н.~СИНИЦЫНА <<КАНОНИЧЕСКИЕ 
ПРЕДСТАВЛЕНИЯ СЛУЧАЙНЫХ ФУНКЦИЙ И ИХ ПРИМЕНЕНИЕ В ЗАДАЧАХ 
КОМПЬЮТЕРНОЙ ПОДДЕРЖКИ НАУЧНЫХ ИССЛЕДОВАНИЙ>> 
      (М.: ТОРУС ПРЕСС, 2009. 840~с.)}

%1
\def\aut{Д.ф.-м.н., профессор С.\,Я.~Шоргин}

\def\auf{\ }

\def\leftkol{РЕЦЕНЗИИ}

\def\rightkol{ \ } 

%\def\leftkol{\ } % ENGLISH ABSTRACTS}

%\def\rightkol{\ } %ENGLISH ABSTRACTS}

%\def\leftkol{РЕЦЕНЗИИ}

%\def\rightkol{РЕЦЕНЗИИ}

\titele{\tit}{\aut}{\auf}{\leftkol}{\rightkol}
\vspace*{-18pt}


     \label{st\stat}

     \begin{multicols}{2}
     {\small
     {\baselineskip=10.1pt
     
    
      Канонические представления (канонические разложения и интегральные 
канонические представления)~--- это выражение случайных функций через дискретный и 
непрерывный белый шум. Канонические представления удобны для выполнения 
различных операций анализа над случайными функциями, особенно линейных операций. 
Объясняется это тем, что в каноническом представлении случайной функции ее 
зависимость от аргумента (скалярного или векторного) выражается при помощи вполне 
определенных неслучайных координатных функций, что дает возможность свести 
выполнение различных линейных операций к соответствующим операциям над 
неслучайными координатными функциями. Важное значение канонические 
представления имеют для задач статистической обработки информации (сигналов, 
изображений, сцен и других образов), стохастического системного анализа, идентификации и 
синтеза, аналитического и статистического моделирования, комплексного решения задач, 
связанных с построением компьютерных моделей. Здесь методы канонических 
представлений случайных функций оказываются достаточно универсальным 
инструментом. Создание теории канонических пред\-став\-ле\-ний случайных функций 
связано с именами Лоэва, Колмогорова, Карунена и Пугачёва и относится к 1940--1950~гг. 
Современные проблемы автоматизации научных исследований, тесно связанные 
с информационными технологиями компьютерной поддержки научных исследований, 
ставят новые задачи разработки прикладной теории, а также методического, 
алгоритмического и программного обеспечения в области стохастического системного 
анализа, идентификации и стохастических информационных технологий научных 
исследований на базе канонических представлений случайных функций.
      
      Монография состоит из 12~глав и 20 приложений. Первые пять глав посвящены 
прикладной теории канонических представлений случайных функций и случайных 
элементов, а гл.~6--12~--- ее применению для типовых задач компьютерной поддержки 
научных исследований. Каждый раздел содержит дополнения и задачи. Приводятся новые 
результаты фундаментальных работ, выполненных в Институте проблем информатики в 
рамках научного направления <<Стохастические системы и стохастические 
информационные технологии>> (1998--2008~гг.), а также фундаментальных работ, 
поддержанных грантами РФФИ и Программой работ ОНИТ РАН <<Фундаментальные 
основы информационных технологий и систем>> (2003--2008~гг.). Издание осуществлено 
при поддержке Российского фонда фундаментальных исследований, издательский проект 
09-07-07005.
      
%      \medskip
      Глава~1  содержит элементарное введение в корреляционную теорию 
канонических разложений и интегральных канонических представлений случайных 
функций. В~разд.~1.1  рассматриваются два вида канонических представлений 
случайных функций. Раздел~1.2 посвящен дискретным каноническим разложениям 
скалярных случайных функций. В~разд.~1.3--1.6 описываются точные и 
приближенные методы построения канонических разложений случайных функций. 
Раздел~1.7 посвящен методам построения интегральных канонических представлений 
скалярных случайных функций. 
      
      В гл.~2 дается обобщение теории канонических разложений и  интегральных 
канонических представлений гл.~1 на случай векторных  случайных функций 
(разд.~2.1 и~2.2). Раздел~2.3 посвящен каноническим  представлениям линейных и 
квазилинейных преобразований случайных  функций. Тео\-рия эквивалентной 
регрессионной линеаризации существенно  нелинейных преобразований посредством 
канонических разложений дается  в разд.~2.4. 
      
      Глава~3 посвящена обобщению результатов гл.~1--2 на случай линейных 
функциональных пространств. В~разд.~3.1 изложена линейная корреляционная теория 
канонических разложений (теоремы Пугачёва, способы построения канонических 
разложений, теорема Лоэва--Ка\-ру\-не\-на, теоремы о совместных канонических разложениях 
двух случайных элементов). Раздел~3.2 содержит тео\-рию интегральных канонических 
представлений. Связь между спектральными и интегральными каноническими 
представлениями устанавливается в разд.~3.3. Раздел~3.4 посвящен корреляционной 
теории канонических представлений линейных преобразований. В~разд.~3.5 
рассматриваются вопросы решения линейных операторных уравнений методами 
канонических представлений. 
      
      Глава~4 содержит элементы стохастического анализа на основе канонических 
представлений случайных процессов. В~разд.~4.1 приведены краткие сведения о 
распределениях случайных процессов и их высших моментах, рассматриваются вопросы  
вычисления распределений посредством канонических разложений с независимыми 
компонентами. Раздел~4.2 посвящен параметризации одно- и многомерных 
распределений. Особое внимание уделено вопросам параметризации ортогональных 
разложений плотностей посредством канонических разложений с независимыми 
компонентами. Вопросы задания вероятностных мер каноническими разложениями и 
вычисления с их помощью производных Радона--Ни\-ко\-ди\-ма изучаются в разд.~4.3 
и~4.4. В разд.~4.5 рассмотрены операции анализа над случайными процессами, 
стохастические интегралы от неслучайных функций скалярного и векторного аргумента, а 
также два типа интегральных канонических представлений. Раздел~4.6 посвящен 
стохастическим интегралам и дифференциалам от случайных функций. В~разд.~4.7 
рассматриваются некоторые вопросы теории канонических пред\-став\-ле\-ний в линейных 
функциональных пространствах. 
      
      Глава~5 посвящена методам статистического оценивания и моделирования 
канонических представлений случайных функций. В~разд.~5.1 приведены 
необходимые сведения из общей теории статистического оценивания, а в разд.~5.2 
и~5.3~--- сведения из прикладной теории статистического оценивания и моделирования 
случайных величин и случайных функций. Применение канонических разложений в 
задачах факторного анализа рассматривается в разд.~5.4.

\def\leftkol{РЕЦЕНЗИИ}

\def\rightkol{РЕЦЕНЗИИ} 
      
      Глава~6 посвящена методическим вопросам компьютерной поддержки 
статистических научных исследований. В~разд.~6.1 рассмотрены задачи, 
общесистемные принципы и подходы к компьютерной поддержке научных исследований. 
В~разд.~6.2 обсуждаются особенности компьютерной поддержки статистических 
научных исследований. Раздел~6.3 содержит изложение постановок задач оптимального 
синтеза систем для анализа и обработки информации по различным вероятностным 
критериям. Особое внимание уделено постановкам задач синтеза субоптимальных и 
условно оптимальных фильт\-ров для систем реального времени.
      
      Глава~7 посвящена рассмотрению математических моделей стохастических 
сигналов и систем. В~разд.~7.1 обсуж\-да\-ют\-ся общие вопросы структуры и 
представления стохастических сигналов и систем, в том числе и на основе канонических 
представлений случайных функций. Раздел~7.2  посвящен непрерывным сис\-темам. 
В~разд.~7.3 изучаются модели линейных дифференциальных сис\-тем. Нелинейные 
непрерывные сис\-темы, опи\-сы\-ва\-емые\linebreak стохастическими дифференциальными уравнениями, 
рас\-смат\-ри\-ва\-ют\-ся в разд.~7.4. Вопросы приведения уравнений непрерывной системы к 
стохастическим дифференциальным уравнениями изучаются в разд.~7.5. Модели 
дискретных систем обсуждаются в разд.~7.6 и~7.7. 
      
      Глава~8  посвящена моделям соединений и сложных стохастических систем. 
В~разд.~8.1 рассматриваются правила преобразований структурных схем и графов 
линейных систем. В~разд.~8.2 изучаются весовые функции соединений линейных 
систем. Раздел~8.3 посвящен стационарным линейным системам. Модели сложных 
стохастических систем обсуждаются в разд.~8.4. 

\def\leftkol{РЕЦЕНЗИИ}

\def\rightkol{РЕЦЕНЗИИ} 
      
      Глава~9 посвящена методам вероятностного анализа линейных стохастических 
систем на основе канонических представлений случайных функций. В~разд.~9.1 
излагаются общие методы корреляционного анализа точности непрерывных сис\-тем. 
Спектрально-кор\-ре\-ля\-ци\-он\-ные методы анализа точности изучаются в разд.~9.2. 
Раздел~9.3 посвящен спектрально-кор\-ре\-ля\-ци\-он\-ным методам анализа точности 
дискретных систем. Анализ распределений и их параметров дается в разд.~9.4. 
      
      В гл.~10 рассматриваются методы вероятностного анализа процессов в 
нелинейных и сложных стохастических системах с помощью канонических 
представлений случайных функций. Раздел~10.1 посвящен методам анализа  точности, 
основанным на непосредственной линеаризации нелинейностей с помощью канонических 
пред\-став\-ле\-ний, а также анализу нелинейных систем, приводимых к линейным. 
Спектрально-корреляционные методы анализа существенно нелинейных систем, 
основанные на эквивалентной линеаризации посредством канонических разложений, 
обсуждаются в разд.~10.2. В~разд.~10.3 рассматриваются вопросы анализа 
распределений марковских процессов в нелинейных стохастических системах 
посредством неканонических и канонических разложений. Особое внимание уделяется 
параметризации распределений на основе канонических разложений с независимыми 
компонентами. Раздел~10.4 посвящен основанным на нормализации комбинированным 
методам аналитического и статистического моделирования сложных стохастических 
систем с помощью канонических разложений. 
      
      \def\leftkol{РЕЦЕНЗИИ}

\def\rightkol{РЕЦЕНЗИИ} 

      Глава~11 посвящена вопросам оптимального (в смыс\-ле минимума средней квадратической
      ошибки (с.к.о.)) синтеза 
фильтров для обработки информации на основе методов канонических разложений и 
интегральных канонических представлений. В~разд.~11.1 даются необходимые 
сведения из теории с.к.\ оптимизации. Выводится общее условие минимума с.к.о., 
выписаны уравнения, определяющие с.к.\ оптимальный фильтр. Подробно анализируется 
случай линейной зависимости сигнала от параметров и аддитивной помехи. Особое 
внимание уделяется фильтрационным уравнениям, определяющим с.к.\ оптимальное 
неоднородное преобразование. Получены уравнения, определяющие весовые функции с.к.\
оптимальных одно- и конечномерных непрерывных и дискретных линейных фильтров. 
Методы определения с.к.\ оптимальных одно- и конечномерных фильтров на основе 
интегральных канонических представлений рассматриваются в разд.~11.2. Методы 
определения с.к.\ оптимальных одно- и конечномерных фильтров на основе канонических 
разложений изучаются в разд.~11.3. Дается общее решение, рассматриваются особые 
случаи, единственность решения линейных фильтрационных уравнений и точность 
фильтрации. В~разд.~11.4 рассматриваются вопросы синтеза нелинейных фильтров, 
приводимых к линейным. Раздел~11.5 содержит общее решение задачи синтеза с.к.\ 
оптимального фильтра в классе всех возможных фильтров. Подробно рассматривается 
случай нормального распределения сигнала и наблюдения. В~разд.~11.6 
рассматривается синтез дискретных и непрерывных линейных фильтров Калмана. 
      
\def\leftkol{РЕЦЕНЗИИ}

\def\rightkol{РЕЦЕНЗИИ} 

      Глава~12 является продолжением гл.~11 и посвящена вопросам оптимального (в 
смысле произвольного байесовского критерия) синтеза фильтров. В~разд.~12.1 
обобщаются результаты разд.~11.1--11.5 на случай сложно-ста\-ти\-сти\-че\-ско\-го критерия. 
Подробно рас\-смот\-ре\-ны как линейные системы, так и нелинейные системы, приводимые к 
линейным. Разделы~12.2 и~12.3 посвящены синтезу методом канонических разложений 
оптимального фильтра по байесовому критерию. Рассмотрены задачи фильтрации, 
экстраполяции, воспроизведения и обнаружения сигналов известной структуры со 
случайными параметрами по различным байесовым критериям. Раздел~12.4 посвящен 
вопросам алгоритмического синтеза оптимального фильтра по байесовому критерию при 
нелинейной и стохастической зависимости наблюдений от параметров сигнала. Дан 
общий метод синтеза на основе канонических разложений. Изучены случаи, когда 
функция потерь~--- линейный функционал, а также случай, когда функция потерь 
является функционалом, а сигнал и помеха имеют нормальное распределение. 
Разделы~12.5 и~12.6 посвящены суб- и условно оптимальным фильтрам.  
      
%      \bigskip
      
      \def\leftkol{РЕЦЕНЗИИ}

\def\rightkol{РЕЦЕНЗИИ} 
      
      Книга предназначена для научных работников и специалистов в области 
автоматизации научных исследований и компьютерной поддержки научных 
исследований, прикладной математики и информатики, механики и теории управления, 
физики и астрономии. Может использоваться в учебном процессе при подготовке 
специалистов, аспирантов и магистров. Единая методика, тщательный подбор примеров и 
задач (их свыше~500), а также богатый методический, программно-алгоритмический и 
справочный материал позволяют использовать книгу широкому кругу магистров, 
аспирантов и преподавателей.
      
}

}
\end{multicols}

%\newpage