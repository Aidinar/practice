\def\stat{sinits}


\def\tr{,\,\ldots\,,\,}
\def\rk{\,\right]}
\def\lk{\left[\,}
\def\rf{\right\}}
\def\lf{\left\{}
\def\iii{\int\limits}
\def\sss{\sum\limits}
\def\prt{\partial}
\def\mm{{\mathrm{M}}}


\def\tit{ВЕРОЯТНОСТНЫЕ МЕТОДЫ ПОСТРОЕНИЯ ИНФОРМАЦИОННЫХ МОДЕЛЕЙ\\
НЕРАВНОМЕРНОСТИ ВРАЩЕНИЯ ЗЕМЛИ$^*$}
\def\titkol{Вероятностные методы построения информационных моделей неравномерности 
вращения Земли} 

\def\autkol{И.\,Н.~Синицын}
\def\aut{И.\,Н.~Синицын$^1$}

\titel{\tit}{\aut}{\autkol}{\titkol}

{\renewcommand{\thefootnote}{\fnsymbol{footnote}}\footnotetext[1]
{Работа выполнена при финансовой поддержке РФФИ
(проект №07-07-00031) и программы ОНИТ РАН <<Информационные
технологии и анализ сложных систем>> (проект 1.5).}}

\renewcommand{\thefootnote}{\arabic{footnote}}
\footnotetext[1]{Институт проблем информатики Российской академии наук, sinitsin@dol.ru}

%\def\ss{\textstyle}

\vspace*{8pt}

\Abst{Рассмотрены новые  вероятностные линейные  и нелинейные
методы построения моделей флуктуаций неравномерности вращения Земли по априорным данным.
Методы лежат в основе информационных ресурсов по проблеме <<Статистическая динамика вращения Земли>>.
Приведены тестовые примеры.}

\vspace*{8pt}

\KW{информационная модель; информационные ресурсы;
линейные и нелинейные вероятностные методы;
спектрально-корреляционные характеристики; эредитарные флуктуации;
одно- и многомерные распределения; асимметрия; эксцесс; гауссовы и пуассоновы белые шумы;
параметризация распределений; моментные методы}


%      \vskip 18pt plus 9pt minus 6pt
      \vskip 64pt plus 9pt minus 6pt

      \thispagestyle{headings}

      \begin{multicols}{2}

      \label{st\stat}

%\vspace*{24pt}
      
\section{Введение}

В~[1--4] построены корреляционные  модели внутригодовой приливной неравномерности
вращения Земли, основанные на учете аддитивных и параметрических, гармонических (полигармонических) и случайных  (<<окрашенных>>  и широкополосных)
гравитационно-прилив\-ных флуктуаций  от Солнца и Луны.

В~[5] разработана теория одно- и многомерных распределений для нелинейных пуассоновых флуктуаций полюса
 Земли и неравномерности вращения Земли.

В~[6] применительно к линейным гауссовым и негауссовым <<окрашенным>> и широкополосным
флуктуациям неравномерности вращения Земли найдены одно-
и многомерные характеристические функции. Вычислены первые четыре статистических момента, а также
асимметрия и эксцесс одномерного распределения флуктуаций неравномерности скорости вращения Земли.

Основываясь на~[7, 8], изложим новые вероятностные методы построения стохастических информационных моделей
флуктуаций неравномерности вращения Земли по априорным данным.


\section{Нелинейные стохастические дифференциальные\break уравнения флуктуаций неравномерности вращения Земли}

Обобщая~[2--6], представим дифференциальное уравнение изменения угла собственного осевого\linebreak
вращения~$\delta\varphi$ Земли, вызванного гравитацион\-но-приливными гармоническими
(полигармоническими) и случайными (<<окрашенными>> и широкополосными)  моментами сил от Солнца~$M^S(t)$ и 
Луны~$M^L (t)$ в виде
\begin{equation}
\delta \ddot\varphi = M^S (t) +M^L(t) +\Delta M^{SL}\,.
\label{e1sin}
\end{equation}
Здесь введены следующие обозначения:
\begin{align}
M^S(t) &= M_0^S(t) + Z_1+ Z_2\,;\notag\\
M_0^S (t) &=M_{10}^S \cos \left(2\pi f_\Gamma t +\chi_1^S\right)+{}\notag\\
&\quad\quad\quad\quad\quad{}+M_{20}^S \cos (4\pi  f_\Gamma t +\chi_2^S)\,;\label{e2sin}\\
 M^L (t)&= M_0^L(t) + Z_3+Z_4\,;\notag
 \end{align}
 \begin{gather}
 M_0^L(t)=M_{m0}^L \cos \left(2\pi \nu_m t +\chi_m^L\right) +{}\notag\\
\quad\quad\quad\quad\quad {}+ M_{f0}^L \cos \left(2\pi \nu_f t +\chi_f^L\right)\,;\label{e3sin}\\
\ddot Z_h + 2 \varepsilon_h \omega_h \dot Z_h +\omega_h^2 Z_h = V_h \enskip ( h=\overline{1,4})\,;\label{e4sin}\\
\delta\varphi =\varphi - r_* \omega_*^{-1} t\,, \notag
\end{gather}
где $ \varphi$~--- угол собственного вращения Земли,
значения которого определены на дату~$t$ (он является параметром,
характеризующим вращение земной системы координат по отношению к
небесной); $r_* = 7{,}292115 \cdot 10^{-5}$~рад/с~--- постоянная
средняя составляющая угловой скорости собственного вращения Земли;
$\omega_*$~---  угловая скорость обращения  Земли по орбите; $t$~---
безразмерное время, измеряемое стандартными годами; $M_0^S(t)$ и~$M_0^L(t)$~--- регулярные возмущения,
обус\-лов\-лен\-ные гра\-ви\-та\-ци\-он\-но-при\-лив\-ны\-ми возмущениями от Солнца и Луны; $f_\Gamma $, 
$2 f_\Gamma$, $\nu_m$, $\nu_f $, $M_{10}^S$, $M_{20}^S$,
$M_{m0}^L$, $M_{f0}^L$, $\chi_1^S$, $\chi_2^S$,\linebreak $\chi_m^L$,
$\chi_f^L$~--- частоты, амплитуды и начальные фазы аддитивных
гармонических возмущений,
соответствующих годовому, полугодовому, месячному и двухнедельному
циклам; $\Delta M^{SL} =$\linebreak $=\Delta M^{SL} (\delta \varphi, \delta\dot\varphi, Z_h)$~---  параметрические
и нелинейные возмущения отмеченных переменных;
 $Z_h$ $(h=\overline{1,4})$~---  <<окрашенные>>
 случайные моменты (в общем случае негауссовы случайные процессы), определяемые
линейными дифференциальными уравнениями второго порядка
формирующего фильтра~(\ref{e4sin}) с параметрами~$\varepsilon_h$ $(0<\varepsilon_h <1)$ 
и~$\omega_h$ $(\omega_1 = 2\pi f_{\mathrm{г}}$, $\omega_2 = 4\pi f_{\mathrm{г}}$,
$\omega_3 = 2\pi \nu_m$, $\omega_4 = 2\pi \nu_f)$;
$V_h$ $(h=\overline{1,4})$~--- независимые негауссовы белые шумы с интенсивностями~$G_h^V$ $(h=\overline{1,4})$.
Уравнение~(\ref{e4sin}) для $t\gg t_0$ определяет и асимптотически устойчивый
формирующий фильтр (ФФ) для <<окрашенного>> стационарного случайного процесса~$Z_h$, 
на входе которого действует входной сигнал в виде белого шума~$V_h$.
Для широкополосного случайного процесса $Z_h$ $(h=\overline{1,4})$ уравнения ФФ возьмем в следующем виде~[1--4]:
\begin{equation}
\dot Z_h =-\alpha_h Z_h +\sigma_h \sqrt{2\alpha_h} V_h\quad (h=\overline{1,4})\,,
\label{e5sin}
\end{equation}
где $V_h$~--- белый шум единичной интенсивности, $G_h^V =1$.

Совокупность нелинейных стохастических дифференциальных уравнений~(1)--(4) 
путем введения информационных  $X=\lk X_1 X_2\rk^{\mathrm{T}}$ и инструментальных
 $Y=\lk Y_1\ldots Y_8\rk^{\mathrm{T}}$ переменных на основе соотношений
\begin{align*}
   X_1 =&\delta \varphi\,;&X_2& =\delta \dot \varphi\,;& &&&&&\\
   Y_1 =& Z_1\,;&\dot Z_1& = Y_2\,;& Y_3& = Z_2\,;& \dot Z_2 &= Y_4\,;\\
   Y_5 =& Z_3\,; & \dot Z_3 &= Y_6\,;& Y_7&= Z_4\,;& \dot Z_4; &= Y_{8}
%\label{e6sin}
\end{align*}
вместе с соответствующими начальными условиями приводится к виду

\noindent
    \begin{multline}
    \dot X_1 = X_2\,;\quad \dot X_2 = M_0^{S}+M_0^L +{}\\
    {}+ \sss_{j=1,3,5,7} Y_j + \Delta M^{SL} (X,Y)\,;
    \label{e7sin}
    \end{multline}

    \vspace*{-12pt}
    
    \noindent
    \begin{multline}
    \dot Y_j = Y_{j+1}\,;\quad \dot Y_{j+1}= -2\varepsilon_{j+1} \omega_{j+1}Y_{j+1} -
    \omega_{j+1}^2 Y_j +{}\\
    {}+V_{j+1}\enskip(j=1, 3, 5, 7)
    \label{e8sin}
    \end{multline}
или в векторно-матричной форме
\begin{equation}
\dot X = a_0 + aX +bY + \Delta M^{SL} (X,Y)\,;\enskip X(t_0) = X_0\,;
\label{e9sin}
\end{equation}
\begin{equation}
\dot Y=\alpha_0 +\alpha Y +\beta V\,;\enskip Y(t_0) = Y_0\,,
\label{e10sin}
\end{equation}
где $V =\lk V_1\ldots V_4\rk^{\mathrm{T}}$; элементы векторов~$a_0$ и~$\alpha_0$, матриц $a,\alpha,b,\beta$ находятся 
из~(\ref{e7sin}) и~(\ref{e8sin}).
Для широкополосных  случайных процессов~$Z_h$ имеем уравнения~(1)--(3) и~(5). 
При этом входящие в уравнения~(\ref{e9sin}) и~(\ref{e10sin}) векторы и матрицы
$a_{0}, \alpha_0, a, \alpha, b,\beta$ определяются из~(1)--(3) и~(5).

\smallskip

\noindent
\textbf{Замечание 1.} Нелинейная дифференциальная стохастическая векторно-матричная модель~(\ref{e9sin}), 
(\ref{e10sin}) положена в основу компьютерного статистического моделирования
одно- и многомерных распределений неравномерностей вращения Земли в составе 
информационных ресурсов по проблеме <<Ста\-ти\-сти\-че\-ская динамика вращения Земли>>.

\smallskip

\noindent
\textbf{Замечание 2.} Как было отмечено в~\cite{3sin}, использование укороченных уравнений флуктуаций неравномерности вращения Земли различных порядков позволяет строить
 достаточно простые и эффективные алгоритмы анализа и синтеза стохастических моделей по априорным данным.
 Для построения соответствующих укороченных уравнений необходимо в уравнениях~(\ref{e9sin}) и~(\ref{e10sin}) 
 перейти к со\-от\-вет\-ст\-ву\-ющим нормальным координатам  и воспользоваться формулой Ито~\cite{7sin, 8sin}.

\section{Линейные вероятностные методы определения одно- и~многомерных распределений флуктуаций 
неравномерности вращения Земли и~их~уклонений от гауссовых.
Тестовые примеры}

Рассмотрим линейную дифференциальную стохастическую модель~(\ref{e9sin}) и~(\ref{e10sin}) при $\Delta M^{SL} \equiv 0$.
Обозначим через  
$$
\chi^V (\mu;t) =\fr{\partial}{\partial t}\, \ln h_1^V (\mu;t)
$$
логарифмическую производную от одномерной харак\-теристической функции
векторного белого шума~$V$. Точное выражение для $n$-мерной характеристической функции
информационных переменных~$X$ для моментов времени $t_1\tr t_n$  будет иметь следующий вид~\cite{7sin, 8sin}:
\begin{multline}
g_n^x (\lambda_1\tr \lambda_n; t_1\tr t_n) ={}\\
{}= g_0^x \left( \sss_{k=1}^n u_{11}(t_k, t_0)^{\mathrm{T}}\lambda\right)\times{}\\
{}\times  g_0^y  \left(  
\sss_{k=1}^n u_{12}(t_k, t_0)^{\mathrm{T}}\lambda\right)\exp\left \{
\vphantom{\iii_{t_{k-1}}^{t_k}}
 -i \sss_{k=1}^n \lambda_k^{\mathrm{T}} \times{}\right.\\
{}\times \lk \iii_{t_0}^{t_k} u_{11} (t_k,\tau) a_0 (\tau) d\tau + \iii_{t_0}^{t_k} u_{12} (t_k,\tau) \alpha_0 (\tau) d\tau\rk+{}\\
\left.{}+ \sss_{k=1}^n \iii_{t_{k-1}}^{t_k} \chi^V \left( \beta(\tau)^{\mathrm{T}} \sss_{l=k}^n u_{12} (t_l,\tau)^{\mathrm{T}} \lambda_l;\tau\right)
\right \}\,.
\label{e11sin}
\end{multline}
Здесь $g_0^x (\sigma)$ и $g_0^y(\rho)$~--- характеристические функции начальных значений~$Y_0$ и~$X_0$;
$u_{11}(t,\tau)$, $u_{12}(t,\tau)$, $u_{22}(t,\tau)$~--- фундаментальные решения однородных дифференциальных
уравнений
\begin{equation}
\dot u_{11} = a u_{11}\,,\enskip\dot u_{12} = a u_{12}+b u_{22}\,,\enskip\dot u_{22} = \alpha u_{22}\,,
\label{e12sin}
\end{equation}
где $a$, $b$, $\alpha$~--- матрицы в уравнениях~(\ref{e9sin}) и~(\ref{e10sin}).
Из~(\ref{e11sin}) и~(\ref{e12sin}) по известным формулам теории линейных стохастических систем~\cite{7sin, 8sin} 
находятся одно- и многомерные плотности и функции распределения.

При гауссовых белых шумах~$V$ и гауссовых начальных условиях $X_0, Y_0$
все одно- и многомерные распределения гауссовы. Поэтому достаточно использовать формулы корреляционной теории~[1--5]
соответственно для <<окрашенных>> и широкополосных процессов~$Z_h$.

Для стационарных и независимых негауссовых белых шумов~$V$ при $t_0=-\infty$ и с учетом того, 
что $a_0=a_0(\tau)$, $\alpha_0=0$, $b,\beta={\rm const}$, $\chi^V (\mu;t) =\chi^V(\mu)$,
формула~(\ref{e11sin}) при $n=1$ дает следующее точное выражение
для нестационарной одномерной характеристической функции информационной переменной  $X_2 =\delta \dot\varphi$:
\begin{multline}
g_1^x (\lambda;t) =g_0^x (u_{11} (t,-\infty)^{\mathrm{T}}\lambda)g_0^y (u_{12} (t,-\infty)^{\mathrm{T}}\lambda)\times{}\\
{}\times \exp\left \{ -i\lambda^{\mathrm{T}} \iii_{-\infty}^t u_{11}(t,\tau) a_0(\tau) d\tau+{}\right.\\
\left.{}+
    \iii_{-\infty}^t \chi^V (\beta^{\mathrm{T}} u_{12} (t,\tau)^{\mathrm{T}} \lambda;\tau) d\tau\right \}\,.
    \label{e13sin}
    \end{multline}
Из~(\ref{e13sin}) для производной центрированной информационной переменной $X_2^0 = X_2-m_2^x$ ($m_2^x$~--- математическое 
ожидание~$\delta\dot\varphi$, определяемое~$M_0^S(t)$ и~$M_0^L(t)$)
получаем следующее точное выражение стационарного одномерного распределения:
\begin{equation}
g_1^{\dot x} (\lambda)=\prod\limits_{j=1,3,5,7} g_1^{y_j}(\lambda)\,.
\label{e14sin}
\end{equation}
Здесь $g_1^{y_j}(\lambda)$ $(j=1,3,5,7)$~--- характеристические функции инструментальных переменных
$Z_j =Y_j$ $(j=1,3,5,7)$, определяемые уравнениями ФФ, причем согласно~\cite{7sin, 8sin} точное выражение имеет вид
    \begin{equation}
    g_1^{y_j}(\lambda)=\exp \left \{ \iii_0^\infty \chi^{V_j} (w_j (\rho)\lambda)\,d\rho\right \}\,,
    \label{e15sin}
    \end{equation}
где $\chi^{V_j}(\mu)$~--- логарифмическая производная одномерной характеристической функции
белого шума~$V_j$ $(j=1, 3, 5, 7)$; $w_j(\rho)$~--- весовая ФФ. В част\-ности, для~(\ref{e4sin}) и~(\ref{e5sin}) имеем
\begin{equation}
\left.
\begin{array}{rcl}
w_j (\rho)\!\! &=&\!\!\fr{1}{\bar \omega_j}\, e^{-\varepsilon_j \omega_j \rho} \sin \omega_j\rho\,;\\[6pt]
\bar \omega_j\!\! &=&\!\! \omega \sqrt{1-\varepsilon_j^2}\,,
\end{array}
\right \}
\label{e16sin}
\end{equation}
где $0< \varepsilon_j<1$ ($j=1$, 3, 5, 7),
\begin{equation}
w_h (\rho) =\sigma_h \sqrt{2\alpha_h} e^{-\alpha_h \rho} \enskip (h=\overline{1,4})\,.
\label{e17sin}
\end{equation}

Для расчета уклонения одномерного распределения информационных переменных~$X_1$ и~$X_2$
 от гауссового обычно или составляются уравнения для
начальных~$\alpha_r$ и центральных~$\mu_r$ моментов $r$-го порядка,  или непосредственно используются 
известные формулы~\cite{7sin, 8sin}:
\begin{equation}
\left.
\!\begin{array}{l}
    \alpha_r^{\dot{x}} =\alpha_{r_1\tr r_n}^{\dot{x}} = i^{-r}\lk
    \fr{\partial^r g_1^{\dot{x}}(\lambda)}{ \partial \lambda_1^{r_1}\ldots \partial 
    \lambda_n^{r_n}}\rk_{\lambda=0}\,;\\[9pt]
    \mu_r^{\dot{x}} =\mu_{r_1\tr r_n}^{\dot{x}}= {}\\[6pt]
  {}= i^{-r}\lk
\fr{\partial^r}{ \partial \lambda_1^{r_1}\ldots \partial \lambda_n^{r_n}}\,e^{-i\lambda^{\mathrm{T}} 
m^{\dot{x}}}g_1^{\dot{x}}(\lambda)\rk_{\lambda=0}\,,
\end{array}\!
\right\}\!
\label{e18sin}
\end{equation}
где
$\lambda=\lk \lambda_1\ldots \lambda_n\rk^{\mathrm{T}}$;\  $r_1+\cdots+r_n = r$,\
$r=1, 2,\ldots$


Для производной информационной переменной~$X_2^0$ выберем в качестве меры уклонения распределения 
от гауссового две безразмерные величины: асимметрию~$\gamma_1$ и эксцесс~$\gamma_2$. Тогда будем иметь следующие расчетные формулы:
\begin{equation} 
\gamma_1 =\fr{\mu_3}{\mu_2^{3/2}}\,;\quad \gamma_2 = \fr{\mu_4}{ \mu_2^4}-3\,,
\label{e19sin}
\end{equation}
где в силу~(\ref{e18sin})

\pagebreak

\noindent
    \begin{equation}
    \mu_r = \mu_r^{\dot{x}_2}= (i)^{-r} \lk \fr{d g_1^r(\lambda)}{d \lambda^r}\rk_{\lambda=0} \enskip (r=2, 3, 4)\,;
    \label{e20sin}
    \end{equation}
    
    \vspace*{-12pt}
    
    \noindent
\begin{multline}
g_1(\lambda) =g_1^{\dot{x}_2}(\lambda) ={}\\
{}=\exp\lf \sss_{j=1,3,5,7}\iii_0^\infty \chi^{V_j} \left(w_j (\rho) \lambda\right)\, d\rho\rf\,.
\label{e21sin}
\end{multline}

\smallskip
\noindent
\textbf{Замечание 3.} Выражения~(10)--(20) лежат в основе информационных ресурсов по проблеме
 <<Ста\-ти\-сти\-че\-ская динамика вращения Земли>>.
 

\smallskip

\noindent
\textbf{Тестовый пример 1.} Формулы~(\ref{e14sin})--(\ref{e16sin}), (\ref{e18sin}), 
(\ref{e19sin})  для независимых гауссовых белых шумов~$V_j$ $(j=1,3,5,7)$, когда
    \begin{equation*}
\chi^{V_j} (\mu) =-\fr{1}{2}\, \mu^2 G^V\quad  \left( G^V =\sss_{j=1,3,5,7} G_j^V\right)\,,
%\label{e22sin}
\end{equation*}
принимают вид
\begin{equation}
\left.
\begin{array}{c}
     g_1(\lambda) = \exp\lf -\fr{\lambda^2}{2}\sss_{j=1,3,5,7} G_j^V I_{2j}\rf\,;\\[13pt]
\displaystyle     I_{2j} =\iii_0^\infty w_j^2 (\rho) d\rho\,;\\[13pt]
\displaystyle     \mu_1 =\mu_3=0\,;\enskip \mu_2 =\sss_{j=1,3,5,7} G_j^V I_{2j}\,;\\[13pt]
     \mu_4 = 3 \mu_2^2\,;\enskip  \gamma_1=0\,;\enskip \gamma_2 =0\,.
     \end{array}
     \right\}
     \label{e23sin}
     \end{equation}
Таким образом, смесь гауссовых <<окрашенных>> процессов~$Z_h$ в уравнениях~(1)--(3) приводит к суммарному гауссову процессу. При этом из 
формул~(\ref{e23sin}) для
дисперсии~$\mu_2$ вытекает следующее условие эквивалентной замены
<<окрашенных>> гауссовых процессов с весовыми функциями~$w_j(\rho)$ одним гауссовым <<окрашенным>> процессом 
с весовой функцией $w^{\mathrm{Э}}(\rho)$:
 \begin{equation}
 \left.
 \begin{array}{rcl}
G^{\mathrm{Э}} I_2^{\mathrm{Э}} \!\!&=&\!\!\displaystyle\sss_{j=1,3,5,7} G_j^V I_{2j}\,;\\[13pt]
I_2^{\mathrm{Э}}\!\!&=&\!\!   \displaystyle  \iii_0^\infty \lk w^{\mathrm{Э}}(\rho)\rk^2 d\rho\,.
\end{array}
\right \}
    \label{e24sin}
    \end{equation}

\smallskip

\noindent
\textbf{Тестовый пример 2.}
 Для независимых простых пуассоновых шумов~$V_j$ $(j=1,3,5,7)$, когда
\begin{equation*}
\chi^{V_j} (\mu) =\left( e^{i\mu} -1\right) G_j^V\,,
%\label{e25sin}
\end{equation*}
находим

\noindent
    \begin{equation}
\!\!    \left.
 \begin{array}{c}
 g_1(\lambda) =\displaystyle  
 \exp\left \{ -\fr{\lambda^2}{2} \sss_{j=1,3,5,7} \iii_0^\infty \bigg ( e^{iw_j(\rho)\lambda} -{}\right.\\[12pt]
\quad\quad\quad\quad\quad\quad\quad\left. {}-1 \bigg ) 
    G_j^V\, d\rho\vphantom{\fr{\lambda^2}{2} \sss_{j=1,3,5,7} \iii_0^\infty \bigg ( e^{iw_j(\rho)\lambda}}
    \right \}\,;\\[12pt]
    \mu_1=\mu_3=0\,;\enskip \mu_2 =\sss_{j=1,3,5,7} G_j^V I_{2j}\,;\\[12pt]
    \mu_4 = \sss_{j=1,3,5,7} \lk G_j^V I_{4j} + 3 (G_j^V I_{2j})^2\rk\,;\\[12pt]
    I_{4j} =\displaystyle \iii_0^\infty w_j^4 (\rho) d\rho\,;\\[12pt]
    \gamma_1=0\,;\enskip \gamma_2 = \fr{\sss_{j=1,3,5,7} G_j^V I_{4j}}{\left( \sss_{j=1,3,5,7} G_j^V I_{2j}\right)^2}\,.
    \end{array}\!
     \right \}\!
     \label{e26sin}
     \end{equation}

Таким образом, <<окрашенное>> стационарное негауссово распределение~(\ref{e26sin}) 
будет симметричным  $(\gamma_1=0)$ с положительным
 эксцессом $\gamma_2>0$. При одинаковых шумах $V_j =V$, $G_j^V = G^V$ и одинаковых ФФ $w_j (\rho) = w(\rho)$ получаем
\begin{equation*}
\gamma_2 =\fr{I_4}{4 G^V I_2^2 } < \fr{\varepsilon \omega}{4G^V}\,.
%\label{e27sin}
\end{equation*}
Отсюда видно, что с ростом интенсивности~ $G^V$ <<окрашенный>> пу\-ас\-со\-нов процесс~$Z_h$ стремится к гауссовому.
Как следует из формул~(\ref{e23sin}) и~(\ref{e26sin}), дисперсии совпадают $\mu_2^{\mathrm{Г}}=\mu_2^{\mathrm{П}}$,
а условием эквивалентности будет условие~(\ref{e24sin}).

\smallskip

\noindent
\textbf{Тестовый пример 3.}
Для широкополосных гауссовых шумов единичной интенсивности имеем
    \begin{gather*}
    \chi^{V_h} (\mu)=-\fr{\mu^2}{2}\,;\\
    g_1(\lambda) =\exp\lf -\fr{\lambda^2}{2} \sss_{j=1}^4 \sigma_j^2\rf\,;\\[6pt]
    \mu_1 =\mu_3 =0\,;\quad \mu_2 =\sss_{j=1}^4 \sigma_j^2\,;\quad \mu_4 = 3 \mu_2^2\,;\\[6pt]
    \gamma_1=0\,; \quad\gamma_2 =0\,.
%    \label{e28sin}
    \end{gather*}

\smallskip

\noindent
\textbf{Тестовый пример 4.}
В случае простых пуассоновых шумов для широкополосных флуктуаций находим

\noindent
\begin{equation}
\!    \left.
    \begin{array}{l}
\qquad\qquad\qquad  \chi^{V_h} (\mu) =\left( e^{i\mu}-1\right)\,;\\[12pt]
    g_1(\lambda) ={}\\[12pt]
    {}=\exp\left \{ -\fr{\lambda^2}{2} \sss_{h=1}^4 \displaystyle  
    \iii_0^\infty \bigg( e^{iw_h (\rho) \lambda} -1  \bigg ) d\rho
\right \}\,;\\[12pt]
    \mu_1 =\mu_3 =0\,;\enskip \mu_2= \sss_{h=1}^2 \sigma_h^2\,;\enskip \mu_4 = 2\sss_{h=1}^4 \sigma_h^4\,;\\[12pt]
\qquad   \gamma_1 =0\,;\quad \gamma_2 =\fr{\sss_{h=1}^4 \sigma_h^4}{2\left( \sss_{h=1}^4 \sigma^2_h\right)^2}\,.
   \end{array}\!\!
   \right\}\!\!
   \label{e29sin}
   \end{equation}
Из~(\ref{e29sin}) видно, что при одинаковых процессах $Z_h=Z$, когда $\sigma_h=\sigma$, имеет место $\gamma_2 =1/8$.

 
\section{Линейные спектрально-корреляционные методы
построения информационных моделей неравномерности вращения Земли. Тестовые примеры }

В этом случае уравнения~(\ref{e9sin}) и~(\ref{e10sin}) при $\Delta M_0^{SL} \equiv$\linebreak $\equiv 0$,  
если вместо векторных переменных~$X$ и~$Y$ ввести объединенные векторы $\bar X =\lk X^{\mathrm{T}} Y^{\mathrm{T}}\rk^{\mathrm{T}}$,
$\bar X_0 =\lk X_0^{\mathrm{T}} Y_0^{\mathrm{T}}\rk^{\mathrm{T}}$ и блочные матрицы $\bar a_0, \bar a, \bar b$,
    $$
    \bar a_0 =
    \begin{bmatrix}
     a_0\cr \alpha_0
     \end{bmatrix}\,;
     \quad 
     \bar a =
    \begin{bmatrix}
    a&b\\ 0&\alpha\\
    \end{bmatrix}\,;\quad 
    \bar b=
    \begin{bmatrix}
    0&0\\ 0&\beta
    \end{bmatrix}\,,
$$
принимают вид
\begin{equation}
{\dot{\bar X}} =\bar a_0 +\bar a \bar X + \bar b V\,.
\label{e30sin}
\end{equation}
Тогда в соответствии с теорией линейных стохастических систем (СтС)~\cite{7sin, 8sin} будем иметь следующую
систему линейных обыкновенных дифференциальных уравнений:
\begin{equation}
{\dot{\bar m}} =\bar a_0 + \bar a \bar m\,;\quad \bar m (t_0) =\bar m_0\,;
\label{e31sin}
\end{equation}
\begin{equation}
{\dot{\bar K}} = \bar a \bar K + k\bar a^{\mathrm{T}} +\bar b G^V \bar b^{\mathrm{T}}\,;\quad \bar K_0 (t_0) =\bar K_0\,;
\label{e32sin}
\end{equation}
\begin{equation*}
\fr{\prt \bar K (t_1, t_2)}{\prt t_2} =\bar K (t_1, t_2) \bar a (t_2)^{\mathrm{T}}\,;
\end{equation*}
\begin{equation}
\bar K(t_1, t_1) =
\begin{cases}
\bar K (t_1) &\mbox{при\ \ } t_1< t_2\,;\\[6pt]
\bar K(t_2, t_1)^{\mathrm{T}} &\mbox{при\ \ } t_1> t_2\,.
\end{cases}
\label{e33sin}
\end{equation}
Здесь $\bar m = \bar m(t)$, $ \bar K = \bar K(t)= \lk k_{ij} (t)\rk$ и $\bar K(t_1, t_2)=\lk K_{ij}(t_1,t_2)\rk$~--- 
вектор математических ожиданий $\bar X = \bar X(t)$, ковариационная матрица и матрица
ковариационных функций соответственно; $G^V$~--- матрица интенсивностей белого шума $V= \lk V_1 V_2 V_3 V_4\rk^{\mathrm{T}}$.

\smallskip

\noindent
\textbf{Замечание 4.} Линейная корреляционная дифференциальная модель~(\ref{e31sin})--(\ref{e33sin}) лежит в основе прямого компьютерного аналитического моделирования линейных
флуктуаций неравномерности\linebreak
вращения Земли в составе информационных ресурсов по 
проб\-леме <<Статистическая динамика вращения Земли>>.
 

\smallskip

\noindent
\textbf{Замечание 5.} Далее в качестве основных исходных линейных стохастических уравнений будут использоваться уравнения (30). При
 этом черта над соответствующими величинами для краткости будет опускаться.
 

\smallskip

\noindent
\textbf{Тестовый пример 5.} Уравнение~(\ref{e31sin}) для математических ожиданий информационных~$X_1$, 
$X_2$ и инструментальных 
$X_3 = Z_1$, $X_5 = Z_2$, $X_7 = Z_3$, $X_9 = Z_4$ переменных
 для моментов $t\gg t_0$
приводит к соотношениям
\begin{gather*}
\dot m_1 = m_2\,;\quad \dot m_2 = M_0^S (t) + M_0^L (t)\,; %\label{e34sin}
\\
m_h^z =0\quad (h=\overline{1,4})\,. %\label{e35sin}
\end{gather*}
Эта задача изучена в~\cite{1sin, 3sin}.
Для рассматриваемых интервалов времени, как следует из~(\ref{e31sin})--(\ref{e33sin}), корреляционные характеристики
не зависят от характеристик гармонических возмущений~$M_0^S(t)$  и~$M_0^L(t)$.
Дифференциальные уравнения~(\ref{e32sin}) для инструментальных переменных приводят к конечным уравнениям
для дисперсий и ковариаций вследствие стационарности процессов  $Z_h = Z_h(t)$ $(h=\overline{1,4})$, а уравнения~(\ref{e33sin})~---
к обыкновенным дифференциальным уравнениям для ковариационных функций относительно разности
времен $\tau = t_1-t_2$. В последнем случае вместо постоянных интенсивностей~$G_h$ используют спектральные плотности
$s_h^V (\omega) = G_h /(2\pi)$ и $ s^x (\omega)=\lk s_{ij}^x (\omega)\rk$ шумов~$V_h$ и переменных~$X_i$ $(i=\overline{1,10})$.

\smallskip

\noindent
\textbf{Тестовый пример 6.}
 Построим приближенную аналитическую корреляционную модель гра\-ви\-та\-ци\-он\-но-при\-лив\-ных флуктуаций, вызванную воздействием
стационарного <<окрашенного>> возмущения
$Z_1 = X_3$. Из уравнений~(\ref{e32sin}) для переменных~$X_3$ и~$X_4$ находятся
следующие выражения для стационарных значений дисперсий и ковариации:

\noindent
    \begin{equation}
    \left.
    \begin{array}{c}
    k_{34}^* =0\,;\quad k_{33}^* =\fr{G_1}{4\varepsilon_1 \omega_1^3}\,;\\[3pt]
 k_{44}^* = \fr{G_1}{4\varepsilon_1 \omega_1}\quad
    (G_1^{V} =G_1)\,.
    \end{array}
    \right \}
    \label{e36sin}
    \end{equation}

Уравнения для дисперсий и ковариаций переменных~$X_2$, $X_3$ и~$X_4$ с учетом~(\ref{e36sin}) имеют вид
          \begin{equation}
    \left.
    \begin{array}{c}
    \dot k_{22} = 2 k_{23}^*\,;\quad k_{23}^* = \fr{ 2\varepsilon_1 k_{33}^*}{\omega_1}\,;\\[3pt]
     k_{24}^* =-k_{33}^* =     -\fr{G_1}{4\varepsilon_1 \omega_1^3}\,.
\end{array}
     \right \}
     \label{e37sin}
     \end{equation}
Уравнения~(\ref{e32sin}) для переменных $X_1\tr X_4$ с учетом~(\ref{e36sin}) и~(\ref{e37sin}) приобретают вид
\begin{equation}
\!\!    \left.
    \begin{array}{c}
    \dot k_{11}= 2 k_{12}\,;\enskip \dot k_{12} = k_{22} + k_{13}\,;\\[1pt]
    \dot k_{13} = k_{23}^* + k_{14}\,;\\[1pt]
 \dot k_{14} = k_{24}^* - \omega_1^2 k_{13} - 2\varepsilon_1 \omega_1 k_{14}\,.
 \end{array}\!\!
 \right \}\!\!
 \label{e38sin}
 \end{equation}
Из~(\ref{e37sin}) и~(\ref{e38sin}) получаем искомые аналитические выражения для корреляционных 
характеристик флуктуационного дрейфа по информационным переменным~$X_1$ и~$X_2$:

\noindent
\begin{gather}
k_{22} =\Lambda_{22} t \quad \left( \Lambda_{22} = \fr{G_1}{ 2\omega_1^4}\right)\,;\label{e39sin}\\
k_{11} =\fr{\Lambda_{22}}{3}\, t^3 + \Lambda_{13} t^2\,;\label{e40sin}\\
k_{12} = \fr{1}{2}\, \Lambda_{22} t^2 +\Lambda_{13} t \quad ( \Lambda_{13} = k_{13}^*)\,;\label{e41sin}\\
k_{13}^* =\fr{k_{24}^* + 2 \varepsilon_1 \omega_1 k_{23}^*}{\omega_1^2}\,;\quad k_{14}^* =- k_{23}^*\,,\label{e42sin}
\end{gather}
где $k_{24}^*$ и $k_{23}^*$ определяются формулами~(\ref{e37sin}).

Из формул~(\ref{e39sin})--(\ref{e41sin}) следует, что при $t\gg t_0$ дисперсии $D_{\delta\varphi} =k_{11}$,
$ D_{\delta\dot\varphi} = k_{22}$ и ковариация $k_{\delta\varphi\delta\dot\varphi} = k_{12}$
информационных переменных $X_1=\delta \varphi$ и   $X_2=\delta \dot\varphi$  растут во времени и могут
быть ограничены только введением отрицательных
обратных связей (<<возвращающих>> сил) по переменным~$\delta \varphi$ и~$\delta \dot\varphi$.
Таким образом, из-за отсутствия обратных связей <<окрашенный>> шум в виде флуктуаций~$Z_h$ $(h=\overline{1,4})$
приводит к нестационарному эффекту~--- накапливающимся со временем флуктуационным дрейфам по~$\delta \varphi$ и~$\delta \dot\varphi$.
Уравнения~(\ref{e36sin}) и~(\ref{e37sin}) можно использовать для установления эквивалентности воздействия
гауссового~$(G_1^{\mathrm{г}})$ и негауссового $(G_1^{\mathrm{нг}})$ белого шума для инструментальных переменных,
а~(\ref{e39sin})--(\ref{e42sin})~--- для инструментальных и информационных переменных.

\vspace*{-4pt}
 
\section{Нелинейные методы вероятностного анализа распределений флуктуаций неравномерности вращения Земли. Тестовые примеры}

\vspace*{-1pt}

Обобщением квазилинейных методов построения моделей~[1--5] являются различные
приближенные методы, основанные на параметризации распределений~\cite{5sin, 7sin, 8sin}.
Аппроксимируя одно\-мерную характеристическую функцию~$g_1 (\lambda;t)$
и соответствующую плотность~$f_1 (x,t)$ известными функциями
$g_1^* (\lambda;\theta)$ и $f_1^* (x;\theta)$, зависящими от
конечномерного векторного параметра~$\theta$, сводим задачу
приближенного определения одномерного распределения к выводу из
уравнения для характеристических функций обыкновенных
дифференциальных уравнений, определяющих~$\theta$ как функцию времени.

 Пусть $w(x)$~--- некоторая плотность в $r$-мерном пространстве~$R^r$, для которой
существуют все моменты. Система пар полиномов~$p_\nu (x)$ и~$q_\nu (x)$ 
$(\nu=\;0,1,2,\ldots)$ называется биортонормальной с весом~$w(x)$, если
%\begin{multline}
\begin{equation}
\int\limits_{-\infty}^\infty\!\! w(x) p_\nu (x) q_\mu (x)\, dx
    =\delta_{\nu\mu} = %{}\\
%    {}=
    \begin{cases}
     0 \hbox{ при }\mu\ne\nu\,;\\
    1\hbox{ при }\mu=\nu\,.
    \end{cases}\!\!\!\!\!
    \label{e43sin}
    \end{equation}
%    \end{multline}
Система пар полиномов~$p_\nu (x)$ и $q_\nu (x)$
$(\nu=$\linebreak $=0,1,2,\ldots)$ называется  биортогональной, если
условие~(\ref{e43sin}) выполняется только при $\mu\ne\nu$. Всякая
биортогональная система пар полиномов $\lf p_\nu (x),q_\nu (x)\rf$
может быть сделана биортонормальной путем деления полиномов~$p_\nu (x)$  и
$q_\nu (x)$ соответственно на множители~$\alpha_\nu$ и
$\beta_\nu$, произведение которых равно интегралу~(\ref{e43sin}) при
со\-от\-вет\-ст\-ву\-ющем~$\nu$ и~$\mu=\nu$. Когда $q_\nu (x) \equiv p_\nu (x)$ 
$(\nu=0,1,2,\ldots)$, условие~(\ref{e43sin}) принимает вид
\begin{equation}
\iii_{-\infty}^\infty w(x) p_\nu (x) p_\mu(x)\,dx
    =\delta_{\nu\mu}\,.
    \label{e44sin}
    \end{equation}
В этом случае система полиномов $\lf p_\nu (x)\rf$
ортонормальна, если она удовлетворяет условию~(\ref{e44sin}) при всех~$\nu$ и
$\mu$, и  ортогональна, если она удовлетворяет
условию~(\ref{e44sin}) только при $\mu\ne\nu$. Всякая ортогональная
система полиномов $\lf p_\nu (x)\rf$ может быть нормирована путем
деления~$p_\nu (x)$ на корень квадратный из интеграла~(\ref{e44sin}) при
со\-от\-вет\-ст\-ву\-ющем~$\nu$ и $\mu =\nu$.
Существование всех моментов плот\-ности~$w(x)$ необходимо и
достаточно для существования всех интегралов~(\ref{e43sin}) и~(\ref{e44sin}).

Плотность~$w(x)$ для построения ее ортогонального разложения
 удобно выбирать так, чтобы ее вероятностные моменты первого и второго
порядков совпадали с соответствующими моментами плотности~$f(x)$ случайного вектора~$X$:
\begin{equation}
f(x) =
    w(x) \lk 1+\sss_{k=3}^\infty \sss_{\vert\nu\vert=k} c_\nu p_\nu
    (x)\rk\,,
    \label{e45sin}
    \end{equation}
где
\pagebreak

\noindent
\begin{multline}
c_\nu
    =\iii_{-\infty}^\infty f(x) q_\nu (x)\, dx = \mm q_\nu (X) = q_\nu
    (\alpha) = {}\\
    {}= \lk q_\nu \left(\fr{\partial}{i\partial
\lambda}\right)g(\lambda)\rk_{\lambda=0}\,.
\label{e46sin}
\end{multline}

Формула~(\ref{e45sin}) описывает ортогональное разложение плотности~$f(x)$. Конечным отрезком этого разложения можно практически
пользоваться для приближенного представления~$f(x)$ даже тогда, когда~$f(x)$ не имеет моментов выше некоторого порядка. 
В~этом случае достаточно заменить распределение~$f(x)$ усеченным распределением
    \begin{equation*}
    f_D (x) = \fr{f(x) {\bf 1}_D (x)}{\iii_D f(x)\,dx}\,,
%    \label{e47sin}
    \end{equation*}
аппроксимирующим~$f(x)$ с достаточной точ\-ностью, и затем
аппроксимировать~$f_D(x)$ отрезком ряда~(\ref{e45sin}). Ограничиваясь в~(\ref{e45sin}) 
полиномами не выше $N$-й степени, получим приближенную формулу для плотности~$f(x)$:
     \begin{equation}
    f(x)\approx f^* (x) = w(x)
    \lk 1+\sss_{k=3}^N \sss_{\vert\nu\vert=k} c_\nu p_\nu
    (x)\rk\,.
    \label{e48sin}
    \end{equation}

Функция $f^*(x)$, аппроксимирующая плотность~$f(x)$, полностью
определяется вероятностными моментами случайной величины до \mbox{$N$-го} порядка
включительно. При этом моменты\linebreak
функции~$f^*(x)$ до $N$-го порядка
включительно совпадают с соответствующими моментами величины~$X$:
    \begin{equation*}
    \iii_{-\infty}^\infty f^* (x) q_\mu (x)\, dx =
    c_\mu=q_\mu(\alpha)\quad \mbox{при}\quad \vert \mu\vert \le N\,.
%    \label{e49sin}
    \end{equation*}
 Таким образом, математические ожидания
всех полиномов~$q_\nu (X)$ не выше $N$-й степени, вычисленные с
помощью аппроксимирующей функции~$f^*(x)$, совпадают с
математическими ожиданиями соответствующих полиномов~$q_\nu (X)$,
вычисленными с помощью истинной плотности~$f(x)$. А так как
полиномы~$q_\nu (x)$ любой данной степени~$\nu$ линейно независимы и
число их совпадает с числом моментов $\nu$-го порядка, то из
совпадения математических ожиданий полиномов~$q_\nu (X)$ следует и
совпадение моментов функции~$f^* (x)$ и плотности~$f(x)$ до $N$-го
порядка включительно. Что касается моментов высших порядков
аппроксимирующей  функции~$f^*(x)$, то они выражаются через
моменты до $N$-го порядка включительно из уравнений
\begin{equation*}
q_\mu (\alpha^*) =0\enskip\hbox{при}\enskip \vert \mu \vert
>N\,. %\label{e50sin}
\end{equation*}

Для дальнейшего изложения понадобятся следующие формулы, выражающие
моменты случайной величины~$X$ до $N$-го порядка через
коэффициенты приближенного выражения~(\ref{e48sin}) ее плотности:
\vspace*{-6pt}

\noindent
\begin{multline}
\alpha_{k_1\tr k_r} =\alpha_{k_1\tr k_r}^w +\sss_{k=1}^N
    \sss_{\vert \nu\vert=k} c_\nu p_{\nu, k_1\tr k_r} (\alpha^w)\\
\left(k_1\tr k_r = 0,1\tr N\,;\right.\\
\left.k_1 +\cdots + k_r = \overline{3, N}\right )\,,
    \label{e51sin}
    \end{multline}
где верхним индексом~$w$ отмечены моменты плотности~$w(x)$,
\begin{equation}
p_{\nu, k_1\tr k_r} (x) = x_1^{k_1} \ldots x_r^{k_r} p_\nu  (x)\,,
\label{e52sin}
\end{equation}
а $p_{\nu, k_1\tr k_r} (\alpha^w)$ получается из $p_{\nu, k_1\tr
k_r} (x)$ так же, как~$q_\nu (\alpha)$ из~$q_\nu (x)$ в~(\ref{e46sin}),
 т.\,е.\ заменой всех одночленов $x_1^{h_1} \ldots x_r^{h_r}$
соответствующими моментами $\alpha_{h_1\tr h_r}^w$ плотности~$w(x)$. Формулы~(\ref{e51sin}) 
с~учетом~(\ref{e52sin}) позволяют вычислять моменты величины~$X$ до $N$-го порядка включительно через\linebreak
 моменты
плотности~$w(x)$ и коэффициенты приближенного представления~$f^* (x)$ плотности~$f(x)$ величины~$X$.

\smallskip

\noindent
\textbf{Замечание 6.} Функции $p_\nu (x)$ и~$q_\nu (x)$ не обязательно должны быть
полиномами. Они могут быть любыми функциями, удовлетворяющими
условию биортонормальности  и условию существования всех
интегралов~(\ref{e46sin}).
Все сказанное о разложении~(\ref{e45sin}) справедливо и в этом более
общем случае. Однако если функции~$q_\nu (x)$ не являются
полиномами, то, несмотря на совпадение моментов первого и второго
порядков распределений~$w(x)$ и~$f(x)$, коэффициенты~$c_\nu$ не
будут равны нулю при $\vert \nu\vert =1$ и~2, вследствие чего
суммирование по~$k$ в~(\ref{e45sin}) будет начинаться с $k=1$.
 
\smallskip

\noindent
\textbf{Тестовый пример 7.}  Биортогональной
системой полиномов, построенной с помощью нормального
распределения с нулевым математическим ожиданием, является система
полиномов Эрмита $\{ H_\nu (x)$, $G_\nu (x)\}$ [7, 8]:
    \begin{align*}
        p_\nu (x)&=\fr{1}{\nu_1!\ldots \nu_r!}\, H_\nu (x-m)\,;\\
    q_\nu (x)& = G_\nu (x-m)\,.
%    \label{e53sin}
    \end{align*}
В этом случае формула~(\ref{e48sin}) принимает следующий вид:
%\noindent
    \begin{multline*}
    f(x) \approx f^* (x) ={}\\
    {}= w^{\mathcal N}  (x)\lk 1+
    \sss_{k=3}^N \sss_{\vert\nu\vert=k} \fr{c_\nu H_\nu (x-m)}{\nu_1!\ldots\nu_r!}\rk\,,
%    \label{e54sin}
    \end{multline*}
   где
   \pagebreak
    
    \noindent
    \begin{multline*}
    c_\nu =\iii_{-\infty}^\infty f(x) G_\nu (x-m) dx ={}\\
    {}{\rm M} G_\nu (X-m) = G_\nu (\mu)\,;
%    \label{e55sin}
    \end{multline*}
$G_\nu (\mu)$~--- линейная комбинация центральных моментов~$X$,
полученная в результате замены каждого одночлена вида $(X_1
-m_1)^{h_1} \ldots (X_r - m_r)^{h_r}$ соответствующим моментом
$\mu_{h_1\tr h_r}$. При этом все моменты (как начальные, так и
центральные) функции~$f^*(x)$, аппроксимирующей плотность~$f(x)$,
до порядка~$N$ включительно совпадают с соответствующими моментами
плотности~$f(x)$, а моменты высших порядков функции~$f^* (x)$
выражаются через моменты до порядка~$N$ из соотношений
\begin{equation*}
G_\nu (\mu^*) =0\enskip\hbox{при}\enskip \vert \nu\vert  >N\,,
%\label{e56sin}
\end{equation*}
где звездочкой отмечены моменты функции~$f^*(x)$.

Коэффициенты $c_\nu$ при $H_\nu (x-m) /(\nu_1!\ldots \nu_r!)$ в
разложении плотности~$f(x)$ по полиномам Эрмита называются
квазимоментами случайной величины~$X$. Число $\vert
\nu\vert =\nu_1 +\cdots+\nu_r$ называется  порядком
квазимомента~$c_\nu$. Квазимомент порядка~$\nu$ пред\-став\-ля\-ет собой
линейную комбинацию центральных моментов до порядка~$\nu$ включительно.

\smallskip

\noindent
\textbf{Тестовый пример 8.}
В качестве других тестовых примеров могут быть выбраны распределения,
па\-ра\-мет\-ри\-зу\-емые посредством канонических разложений с независимыми компонентами~\cite{8sin}.


\section{Моментные методы построения
нелинейных информационных моделей неравномерности вращения Земли. Тестовый пример}

Нелинейная негауссова система~(\ref{e9sin}) и~(\ref{e10sin}) в переменных $\bar X=
\lk X^{\mathrm{T}} Y^{\mathrm{T}}\rk^{\mathrm{T}}$ и блочных матрицах
\begin{align*}
\bar \varphi (\bar X,t)& =
\begin{bmatrix}
    \bar a_0 +\bar a \bar X +\tilde\varphi (\bar X,t)\\
    \alpha Y
    \end{bmatrix}\,;\\[3pt]
    \bar \psi (\bar X, t)& =
    \begin{bmatrix}
    0&\tilde\psi (\bar X,t)\\
    0&\beta
    \end{bmatrix}\,;\\[3pt]
\tilde \varphi(\bar X, t) &= \Delta M_0^{Sl} (\bar X,t)\,;\\[3pt]
\tilde \psi (\bar X,t) V&= \Delta M^{SL} (\bar X,t) - M_0^{SL} (\bar X,t)\,,
%    \label{e57sin}
    \end{align*}
может быть записана в виде
\begin{equation}
{\dot{\bar X}} =\bar \varphi (\bar X,t) +\bar\psi (\bar X,t)V\,.
\label{e58sin}
\end{equation}
В соответствии с замечанием~2 опустим далее черту в~(\ref{e59sin}) и~(\ref{e60sin}).

Предположим, что параметр~$\theta$, от которого зависят функции~$g_1^*(\lambda;\theta)$ и
$f_1^*(x;\theta)$, аппроксимирующие
одномерную характеристическую функцию~$g_1 (\lambda;t)$ и
соответствующую плотность~$f_1(x;\theta)$, представляет собой
совокупность моментов вектора~$X_t$ до определенного порядка~$N$
включительно. В качестве аппроксимирующей плотность~$f_1 (x;t)$
функции~$f_1^*(x;\theta)$ удобно взять конечный отрезок ее
ортогонального разложения вида~(\ref{e45sin}):
\begin{multline}
f_1 (x;t) \approx f_1^*(x;\theta) ={}\\
{}= w_1 (x) \lk 1+\sss_{l=3}^N
    \sss_{\vert \nu\vert=l} c_\nu p_\nu (x)\rk\,.
    \label{e59sin}
    \end{multline}
Здесь коэффициенты~$c_\nu$~---  линейные комбинации моментов
случайного вектора~$X_t$ до порядка~$N$ включительно:
\begin{multline}
c_\nu = q_\nu (\alpha) \quad \left(\nu_1\tr \nu_n =\overline{0,1, N}\,;\right.\\
\left. \vert \nu\vert
    =\nu_1 +\cdots+\nu_n =\overline{3, N}\right )\,.
    \label{e60sin}
    \end{multline}

При этом $q_\nu (\alpha)$ представляет собой результат замены всех
одночленов $x_1^{r_1} \ldots x_n^{r_n}$ в выражении полинома
$q_\nu (y)$ соответствующими моментами $\alpha_{r_1\tr r_n}$.
Коэффициенты полиномов~$p_\nu(x)$ и~$q_\nu (x)$ в общем случае
зависят от моментов первого и второго порядка вектора~$X_t$,
поскольку плотность~$w_1 (x)$ в~(\ref{e59sin}) имеет те же моменты
первого и второго порядка, что и~$f_1 (x;t)$.

Для нелинейной  негауссовской СтС~(\ref{e58sin}) основные уравнения для
начальных моментов~$\alpha_r$ имеют вид~\cite{7sin, 8sin}
\begin{multline}
{\dot     \alpha}_r =\iii_{-\infty}^\infty
    \left \{\fr{\partial^{\vert r\vert}}{
    \partial (i\lambda_1)^{r_1}\ldots \partial
    (i\lambda_n)^{r_n}} 
\left [ i\lambda^{\mathrm{T}} \varphi(x,t) +{}\right.\right.\\
\left.\left.{}+\chi(\psi(x,t)^{\mathrm{T}} \lambda;t)\right ] e^{i\lambda^{\mathrm{T}} x}
    \vphantom{\fr{\partial^{\vert r\vert}}{
    \partial (i\lambda_1)^{r_1}\ldots \partial
    (i\lambda_n)^{r_n}} }
    \right\}_{\lambda=0} f_1^*
    (x;\theta)\,dx.
    \label{e61sin}
    \end{multline}

Интегрируя систему уравнений~(\ref{e61sin}) при начальных условиях
\begin{equation*}
\alpha_r (t_0) = \alpha^0_r\enskip (r_1\tr r_n =\overline{0,1,N}; \vert r\vert
    =\overline{1, N})\,,\!
%    \label{e62sin}
    \end{equation*}
найдем все координаты вектора~$\theta$ как функции времени~$t$
($\alpha^0_r$~--- моменты начального значения~$X_0$ вектора~$X_t$
при $t=t_0$).

\smallskip

\noindent
\textbf{Замечание 7.} Уравнения~(\ref{e61sin}) линейны относительно моментов~$\alpha_r$ 
выше второго порядка, $\vert r\vert =3\tr N$, и нелинейны
относительно моментов первого и второго порядка, поскольку
плотность~$w_1 (x)$ и коэффициенты полиномов~$p_\nu(x)$ и~$q_\nu (x)$ 
зависят от моментов первого и второго порядка.

\begin{table*}[b]\small
\vspace*{-12pt}
\begin{center}
\Caption{Зависимость числа уравнений для параметров одномерного распределения}
\vspace*{2ex}

\tabcolsep=8pt
\begin{tabular}{|r|c|c|c|c|c|c|c|c|c|c|}
\hline
&\multicolumn{10}{c|}{$\bar n_x$}\\
\cline{2-11}
\multicolumn{1}{|c|}{\raisebox{4pt}[0pt][0pt]{$Q$}}&1&2&3&4&5&6&7&8&9&10\\
\hline
2&2&\hphantom{9}5&\hphantom{9}9&\hphantom{9}14&\hphantom{9}20&\hphantom{9}27&\hphantom{99}35&\hphantom{99}44&\hphantom{99}54&\hphantom{99}65\\
4&4&14&34&\hphantom{9}69&125&209&\hphantom{9}329&\hphantom{9}494&\hphantom{9}714&1000\\
6&6&27&83&209&461&423&1715&3002&5004&8007\\
8&8&44&164\hphantom{9}&494&1286\hphantom{9}&3002\hphantom{9}&6434& & & \\
10&10\hphantom{9}&65&1000\hphantom{99}&3002\hphantom{9}&8007\hphantom{9}& & & & & \\
\hline
\end{tabular}
\end{center}
\end{table*}


\smallskip

\noindent
\textbf{Замечание 8.} 
При составлении уравнений~(\ref{e61sin}) в конкретных задачах полезно
иметь в виду, что чис\-ло~$Q_n^r$ моментов $r$-го порядка, а также
полное чис\-ло моментов порядка, не превосходящего~$N$, для $n$-мерного
случайного вектора определяется формулами
\begin{equation}
\!\left.
\!\!\begin{array}{rcl}
 Q_{1n}^r \!\!\!\!&=&\!\!\!\! C_{n+r-1}^r =\fr{(n+r-1)!}{r! (n-1)}\,;\\[9pt]
    Q_{2n}^N\!\!\!\! &=&\!\!\!\!\sss_{r=1}^N Q_{1n}^r = C_{N+n}^N -1 =
\fr{(N+n)!}{N!n!}- 1\,.
\end{array}\!\!\!
\right \}\!\!
\label{e63sin}
\end{equation}
 
Для нелинейной негауссовой системы~(\ref{e61sin}) дифференциальные уравнения метода центральных
моментов для $m_t$, $K_t$ и~$\mu_r$ могут быть записаны в
следующем виде~\cite{7sin, 8sin}:
\begin{equation*}
{\dot m}_h=
    \iii_{-\infty}^\infty \varphi (x;t) f_1^*(x,\theta)\,dx\enskip
    (h=\overline{1, p})\,,
%    \label{e64sin}
    \end{equation*}
    
    \vspace*{-12pt}
    
    \noindent
\begin{multline*}
{\dot\mu}_r =\iii_{-\infty}^\infty \left\{ \fr{\partial^{\vert  r\vert-1}}{\partial (i\lambda_1)^{r_1}\ldots
    (i\lambda_n)^{r_n}}
\left [ i\lambda^{\mathrm{T}} \varphi(x,t) +{}\right.\right.\\
\left.\left.{}+\chi(\psi(x,t)^{\mathrm{T}}     \lambda;t)\right ] e^{i\lambda^{\mathrm{T}}(x-m)}
\vphantom{\fr{\partial^{\vert  r\vert-1}}{\partial (i\lambda_1)^{r_1}\ldots
    (i\lambda_n)^{r_n}}}
    \right\}_{\lambda=0}
 f_1^*(x;\theta)\, dx -{}\\
 {}- \sss_{h=1}^n r_h \mu_{r-e_h}
    \iii_{-\infty}^\infty \varphi_h (x,t) f_1^*(x;\theta)\,dx
%    \label{e65sin}
    \end{multline*}
    $$
    (r_1\tr r_n =\overline{0,1, N}\,;\enskip \vert r\vert=\overline{2,N})\,,
    $$
где $e_s$~--- вектор, все компоненты
которого рав\-ны~$0$, кро\-ме~$s$-й, рав\-ной~$1$. Начальными условиями для~(\ref{e61sin}) являются
\begin{multline*}
m(t_0) = m_0\,; \quad \mu_r(t_0) = \mu_r^0\\ 
(r_1\tr r_n =\overline{0,1,N}\,;\quad \vert r\vert=\overline{2,N})\,. 
%\label{e66sin}
\end{multline*}

Основной трудностью практического применения
методов моментов является быстрый рост чис\-ла уравнений с увеличением размерности вектора
$\bar X=\lk X^{\mathrm{T}} Y^{\mathrm{T}}\rk^{\mathrm{T}}$. В соответствии с формулами~(\ref{e63sin}) в табл.~1 
показана зависимость числа уравнений для параметров одномерного распределения от порядка момента~$Q$ и 
размерности~$\bar n_x$.


Квазилинейным методам отвечает строчка, соответствующая $Q=2$.
В~\cite{7sin, 8sin} описаны известные методы сокращения числа уравнений. В~нашем случае вследствие фильтрующих 
свойств ФФ~(\ref{e4sin})
и~(\ref{e5sin}) смешанные моменты нелинейных возмущений~$\Delta M^{SL}$ приближенно вычисляются через дисперсии
и ковариации нормальных векторных случайных процессов~\cite{7sin, 8sin}.

Как показано в~\cite{7sin, 8sin}, для определения двух и других многомерных распределений могут быть использованы моментные параметризационные методы.
Для расчетов числа уравнений можно использовать формулы~(\ref{e63sin}) с заменой~$n$ на $n\bar n_x$.

\smallskip

\noindent
\textbf{Тестовый пример 9.} 
В~качестве тестового примера может быть выбран пример~3 из~\cite{2sin, 3sin} 
в случае нелинейного рэлеевского механизма диссипации.
В~качестве других тестовых примеров выбраны примеры~7.1.1, 7.1.6. 7.1.7 и~7.2.2 из~\cite{7sin}.

\vspace*{-6pt}
\section{Заключение}

\noindent
\begin{enumerate}[1.]
\item В силу известного линейного соотношения между $\delta \dot\varphi(t)$ и длительностью суток
(l.o.d.$(t)$)~\cite{9sin}
    $$
    \mathrm{l.o.d}(t) =- \fr{86400}{r_*}\,\delta \dot\varphi(t)
    $$
полученные в разд.~1--6 результаты обобщаются и на l.o.d.$(t)$.\\[-16pt]
\item
Для эредитарных возмущений~$M^S(t)$, $M^l(t)$, ядра которых допускают аппроксимацию линейными дифференциальными
уравнениями, соответствующие результаты для $t\ll t_0$ получаются путем расширения вектора состояния и
составления линейных уравнений для эредитарных ядер~[6--8].\\[-16pt]
\item
Нелинейные вероятностные методы позволяют строить как стационарные, так и нестационарные негауссовы информационные модели 
флуктуаций. В~отличие от квазилинейных методов~\cite{3sin},
 нелинейные методы позволяют повысить точность построения до 0,1\%--0,01\%. 
 Эта задача очень важна для оценок больших уклонений.\\[-13pt]
 \end{enumerate}

Полученные в разд.~1 уравнения, а также выражения для одно- и многомерных распределений
совместно с уравнениями для функций чувствительности определяющих параметров положены в\linebreak основу
компьютерной модели в среде MATLAB. Она  включена в состав информационных ресурсов по фундаментальной проблеме
<<Статистическая динамика вращения Земли>>. Проведенные вы\-чис\-ли\-тель\-ные эксперименты
методом статистического моделирования подтверждают основные свойства полученных аналитических
информационных моделей неравномерностей вращения \mbox{Земли}.

\medskip
Работа выполнена при финансовой поддержке РФФИ (проект 07-07-00031) и программы ОНИТ РАН <<Информационные технологии и методы
анализа сложных систем>> (проект~1.5).

\vspace*{-5pt}

{\small\frenchspacing
{%\baselineskip=10.8pt
\addcontentsline{toc}{section}{Литература}
\vspace*{-2pt}
\begin{thebibliography}{99}    

\bibitem{1sin}
\Au{Марков Ю.\,Г., Синицын~И.\,Н.} 
Корреляционная модель приливной неравномерности вращения Земли~//
ДАН, 2008. Т.~419. №\,3. С.~338--341.

\bibitem{3sin} %2
\Au{Синицын И.\,Н.} 
Квазилинейные методы построения информационных моделей флуктуаций неравномерности вращения Земли~//
Информатика и её применения, 2008. Т.~2. Вып.~1. С.~35--43.

\bibitem{2sin} %3
\Au{Марков Ю.\,Г., Синицын~И.\,Н.} 
Влияние <<окрашенных>> флуктуаций на неравномерности вращения Земли~//
ДАН, 2009. Т.~327. №\,3. С.~1--4.


\bibitem{4sin}
\Au{Синицын И.\,Н.} 
Построение эредитарных и информационных моделей флуктуаций вращения Земли~// Информатика и её применения, 2009. 
Т.~3. Вып.~1. С.~2--7.

\bibitem{5sin}
\Au{Синицын И.\,Н.} 
Стохастические модели флуктуаций движения Земли в условиях пуассоновых возмущений~//
Системы и средства информатики: Спец.\ вып. <<Геоинформационные технологии>>.~--- М.: ИПИ РАН, 2004.~С. 37--55.

\bibitem{6sin}
\Au{Марков Ю.\,Г., Синицын И.\,Н.} 
Одно- и многомерные распределения флуктуаций неравномерностей вращения Земли~// ДАН, 2009.
Т.~428. Вып.~5. С.~616--619.

\bibitem{7sin}
\Au{Пугачёв В.\,С., Синицын~И.\,Н.} 
Теория стохастических систем. 2-е изд.~--- М.:  Логос, 2004.

\bibitem{8sin}
\Au{Синицын И.\,Н.} 
Канонические представления случайных функций и их применение в задачах компьютерной поддержки 
научных исследований.~--- М.: ТОРУС ПРЕСС, 2009.

\label{end\stat}

\bibitem{9sin}
\Au{Одуан К., Гино~Б.} 
Измерение времени. Основы GPS.~--- М.:  Техносфера, 2002.
 \end{thebibliography}
}
}
\end{multicols}