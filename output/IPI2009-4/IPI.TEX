\documentclass[10pt]{book}
\usepackage[utf8]{inputenc}

\usepackage{latexsym,amssymb,amsfonts,amsmath,indentfirst,shapepar,%fleqn,%
picinpar,shadow,floatflt,enumerate,multicol,ipi}

\usepackage{rotating}

\input{epsf}

%\nofiles

%\includeonly{avtor,avtor-eng}
%\includeonly{avtor-eng}
%\includeonly{pred}

%\includeonly{sinits} %1+pdfАвторы
%\includeonly{ushmaev} %2O+pdfиспрАвторы
%\includeonly{agalar} %+ %3pdfиспр2Авторы
%\includeonly{kudr}  %4+pdfАвторы
%\includeonly{ushakov} %5+pdf
%\includeonly{gaponova} %6+pdf
%\includeonly{markin}  %7+pdf
%\includeonly{chernikov} %8+pdfиспр+Авторы
%\includeonly{kirikov} %9pdf
%\includeonly{zatsman}  %10+pdf


%\includeonly{toc-rus,toc-en}
%\includeonly{toc-en}


%\includeonly{obchak}
%\includeonly{reshal}
%\includeonly{eng-index}
%\includeonly{cover3}

\usepackage{acad}
\usepackage{courier}
\usepackage{decor}
\usepackage{newton}
\usepackage{pragmatica}
\usepackage{zapfchan}
\usepackage{petrotex}
\usepackage{bm}                     % полужирные греческие буквы
\usepackage{upgreek}                % прямые греческие буквы
%\usepackage{verbatim}

\renewcommand{\bottomfraction}{0.99}
\renewcommand{\topfraction}{0.99}
\renewcommand{\textfraction}{0.01}

\setcounter{secnumdepth}{1} %здесь - 3 + chapter = 4

%\usepackage[pdftex]{graphicx}

%\usepackage{oz}

%NEW COMMANDS



\renewcommand{\r}{{\rm I\hspace{-0.7mm}\rm R}}
\newcommand{\I}{{\rm I\hspace{-0.7mm}I}}
\newcommand{\Ik}{\mbox{{\small \tt {1}}\hspace{-1.5mm}{\tt 1}}}
%\newcommand{\Ikl}{{\small \tt{1}}\hspace*{-0.4mm}\mathtt{1}}

%\mathrm{I}\hspace*{-0.7mm}\mathrm{R}

\newcommand{\il}[2]{\int\limits_{#1}^{#2}}%интеграл с пределами #1 и #2

\newcommand{\p}{{\sf P}}  % вероятность
\newcommand{\e}{{\sf E}}  % мат. ожидание
\newcommand{\D}{{\sf D}}  % дисперсия
\newcommand{\eps}{\varepsilon}
\newcommand{\vp}{\mathrm{v.p.}}
\newcommand{\F}{{\mathcal F}}


%\newcommand{\gr}{{\geqslant}}

%\renewcommand{\la}{\lambda}
%\newcommand{\si}{\sigma}
%\renewcommand{\a}{\alpha}

%\newcommand{\pto}{\stackrel{P}{\longrightarrow}} % сходимость по веpоятности

%\newcommand{\eqd}{\stackrel{d}{=}} % равенство по pаспpеделению

%\newcommand{\kp}{\kappa}
%\def\Q{{\cal Q}} \def\H{{\cal H}}
%\newcommand{\bet}{\beta_{2+\delta}}


%\newtheorem{definition}{Определение}
%\renewcommand{\thedefinition}{\arabic{definition}.}
%END NEW COMMANDS

%\renewcommand{\baselinestretch}{1.2}

%\pagestyle{myheadings}

\setlength{\textwidth}{167mm}      % 122mm
\setlength{\textheight}{658pt}
%\setlength{\textheight}{635.6pt}
\setlength{\columnsep}{4.5mm}

\setcounter{secnumdepth}{4}

%\addtolength{\headheight}{2pt}
%\addtolength{\headsep}{-2mm}

%\addtolength{\topmargin}{-20mm}  % for printing


\hoffset=-30mm  % From Yap
%\hoffset=-20mm  % From Acrobat

%\voffset=0mm % From Yap
%\voffset=-15mm   % From Acrobat

\addtolength{\evensidemargin}{-9.5mm} % for printing
\addtolength{\oddsidemargin}{9.5mm}  % for printing

%\renewcommand{\thefootnote}{\fnsymbol{footnote}}
%\renewcommand{\thefootnote}{\arabic{footnote}}
\renewcommand{\figurename}{\protect\bf Рис.}
\renewcommand{\tablename}{\protect\bf Таблица}

\newcommand{\Caption}[1]{\caption{\protect\small %\baselineskip=2.5ex
#1}}

\renewcommand{\thefigure}{\arabic{figure}}
\renewcommand{\thetable}{\arabic{table}}
\renewcommand{\theequation}{\arabic{equation}}
\renewcommand{\thesection}{\arabic{section}}

\renewcommand{\contentsname}{СОДЕРЖАНИЕ}
\newcommand{\fr}[2]{\displaystyle\frac{\displaystyle #1\mathstrut}{\displaystyle #2\mathstrut}}

%\renewcommand{\thefootnote}{\fnsymbol{footnote}}
%\newcommand{\g}{\mbox{\textit{g}}}

%\newcommand{\Caption}[1]{\caption{\protect\small\baselineskip=2ex #1}}
\newcounter{razdel}
\setcounter{razdel}{0}


\newcommand{\titel}[4]{%
\

\vspace*{5pt}

\ifodd\therazdel {\raggedright\noindent\Large\textrm\textbf
 \lineskip .75em
  \baselineskip=3.2ex #1 \par}
\vskip 1em {\noindent\large\textrm\textbf #2 \par}
\addcontentsline{toc}{subsection}{{\textrm\textbf #3}\protect\newline #1}
\def\rightheadline{\underline{\noindent\hbox to \textwidth{\hfill\small\textrm{#4}
%\hfill \large\bf\thepage
}}}
\def\leftheadline{\underline{\noindent\parbox{\textwidth}{
%\raggedleft\large\bf\thepage \hfill
\small\textit{#3}\hfill}}}
\def\leftfootline{\small{\textbf{\thepage}
\hfill ИНФОРМАТИКА И ЕЁ ПРИМЕНЕНИЯ\ \ \ том~3\ \ \ выпуск 4\ \ \ 2009}
}%
 \def\rightfootline{\small{ИНФОРМАТИКА И ЕЁ ПРИМЕНЕНИЯ\ \ \ том~3\ \ \ выпуск~4\ \ \ 2009
\hfill \textbf{\thepage}}} \vskip 2em \setcounter{figure}{0}
\setcounter{table}{0} \setcounter{equation}{0} \setcounter{section}{0}
\setcounter{subsection}{0} \setcounter{subsubsection}{0}
\setcounter{footnote}{0} \setcounter{razdel}{0}
%\end{flushleft}
\else {
 \raggedright\noindent\Large\textrm\textbf
 \lineskip .75em
\baselineskip=3.2ex #1 \par} \vskip 1em
%\begin{flushleft}
{\noindent\large\textrm\textbf #2 \par}
\addcontentsline{toc}{subsection}{{\textrm\textbf #3}\protect\newline #1}
\def\rightheadline{\underline{\noindent\hbox to \textwidth{\hfill\small\textrm{#4}
%\hfill \large\bf\thepage
}}}
\def\leftheadline{\underline{\noindent\parbox{\textwidth}{%\raggedleft\large\bf\thepage \hfill
\small\textit{#3}\hfill}}}
\def\leftfootline{\small{\textbf{\thepage}
\hfill ИНФОРМАТИКА И ЕЁ ПРИМЕНЕНИЯ\ \ \ том~3\ \ \ выпуск~4\ \ \ 2009}
}%
 \def\rightfootline{\small{ИНФОРМАТИКА И ЕЁ ПРИМЕНЕНИЯ\ \ \ том~3\ \ \ выпуск~4\ \ \ 2009
\hfill \textbf{\thepage}}} \vskip 2em \setcounter{figure}{0}
\setcounter{table}{0} \setcounter{equation}{0} \setcounter{section}{0}
\setcounter{subsection}{0} \setcounter{subsubsection}{0}
\setcounter{footnote}{0}
%\end{flushleft}
\fi}

\newcommand{\titelr}[2]{%
\

\vspace*{5pt}

\ifodd\therazdel {\raggedright\noindent\large\textrm\textbf
 \lineskip .75em
  \baselineskip=3.2ex #1 \par}
\vskip 1em {\noindent\normalsize\textrm\textbf #2 \par}
\else {
 \raggedright\noindent\large\textrm\textbf
 \lineskip .75em
\baselineskip=3.2ex #1 \par} \vskip 1em
%\begin{flushleft}
{\noindent\normalsize\textrm\textbf #2 \par}
\fi}

\newcommand{\titele}[5]{%
\

%\vspace*{5pt}

\ifodd\therazdel {\raggedright\noindent%\large
\textrm\textbf
 \lineskip .75em
%  \baselineskip=3.2ex
#1 \par}
\vskip .5em {\noindent\large\textrm\textbf #2 \par}
\vskip .5em
 {\noindent\textrm #3 \par}
\addcontentsline{toc}{subsection}{{\textrm\textbf #1}\protect\newline #2}
\def\rightheadline{\underline{\noindent\hbox to \textwidth{\hfill\small\textrm{#4}
%\hfill \large\bf\thepage
}}}
\def\leftheadline{\underline{\noindent\parbox{\textwidth}{
%\raggedleft\large\bf\thepage \hfill
\small\textrm{#5}\hfill}}}
\def\leftfootline{\small{\textbf{\thepage}
\hfill ИНФОРМАТИКА И ЕЁ ПРИМЕНЕНИЯ\ \ \ том~3\ \ \ выпуск~4\ \ \ 2009}
}%
 \def\rightfootline{\small{ИНФОРМАТИКА И ЕЁ ПРИМЕНЕНИЯ\ \ \ том~3\ \ \ выпуск~4\ \ \ 2009
\hfill \textbf{\thepage}}} \vskip 1em \setcounter{figure}{0}
\setcounter{table}{0} \setcounter{equation}{0} \setcounter{section}{0}
\setcounter{subsection}{0} \setcounter{subsubsection}{0}
\setcounter{footnote}{0} \setcounter{razdel}{0}
%\end{flushleft}
\else {
 \raggedright\noindent%\large
 \textrm\textbf
 \lineskip .75em
%\baselineskip=3.2ex
#1 \par} \vskip .5em
%\begin{flushleft}
{\noindent\large\textrm\textbf #2 \par} \vskip .5em
 {\noindent\textrm #3 \par}
\addcontentsline{toc}{subsection}{{\textrm\textbf #1}\protect\newline #2}
\def\rightheadline{\underline{\noindent\hbox to \textwidth{\hfill\small\textrm{#4}
%\hfill \large\bf\thepage
}}}
\def\leftheadline{\underline{\noindent\parbox{\textwidth}{%\raggedleft\large\bf\thepage \hfill
\small\textrm{#5}\hfill}}}
\def\leftfootline{\small{\textbf{\thepage}
\hfill ИНФОРМАТИКА И ЕЁ ПРИМЕНЕНИЯ\ \ \ том~3\ \ \ выпуск~4\ \ \ 2009}
}%
 \def\rightfootline{\small{ИНФОРМАТИКА И ЕЁ ПРИМЕНЕНИЯ\ \ \ том~3\ \ \ выпуск~4\ \ \ 2009
\hfill \textbf{\thepage}}} \vskip 1em \setcounter{figure}{0}
\setcounter{table}{0} \setcounter{equation}{0} \setcounter{section}{0}
\setcounter{subsection}{0} \setcounter{subsubsection}{0}
\setcounter{footnote}{0}
%\end{flushleft}
\fi}

\def\Abst#1{
\begin{center}\small\nwt
\parbox{150mm}{%\baselineskip=2.5ex
\textbf{Аннотация:}\ \
%\hspace*{\parindent}
#1}
\end{center}}
\def\Abste#1{
\begin{center}\small\nwt
\parbox{150mm}{%\baselineskip=2.5ex
\textbf{Abstract:}\ \
%\hspace*{\parindent}
#1}
\end{center}}

\def\KW#1{
\begin{center}\small\nwt
\parbox{150mm}{%\baselineskip=2.5ex
\textbf{Ключевые слова:}\ \ #1}
\end{center}}

\def\KWE#1{
\begin{center}\small\nwt
\parbox{150mm}{%\baselineskip=2.5ex
\textbf{Keywords:}\ \ #1}
\end{center}}


\def\KWN#1{
%\begin{center}
%\small
%\parbox{150mm}\end{center}
}

\renewcommand{\thesubsection}{\thesection.\arabic{subsection}\hspace*{-5pt}}
\renewcommand{\thesubsubsection}{\thesubsection\hspace*{5pt}.\arabic{subsubsection}\hspace*{-3pt}}

\begin{document}
\Rus

\nwt
%\ptb

%\renewcommand{\contentsname}{\protect\Large\bf Содержание}

\setcounter{tocdepth}{2}

%\tableofcontents

\renewcommand{\bibname}{\protect\rmfamily Литература}
  \def\Au#1{{\it #1}}

%\newcommand{\No}{№}
  \newcommand{\tg}{\rm  tg}
    \newcommand{\ctg}{\mathrm{  ctg}}
  \newcommand{\arctg}{\rm  arctg}

\setcounter{page}{1}

\newpage
\addtocounter{razdel}{1}
%\def\razd{РЕГУЛИРУЕМЫЙ ЭЛЕКТРОПРИВОД ДЛЯ ЭЛЕКТРОЭНЕРГЕТИКИ}
%\newpage
%\def\stat{zakh}
\def\tit{СРЕДСТВА ОБЕСПЕЧЕНИЯ ОТКАЗОУСТОЙЧИВОСТИ ПРИЛОЖЕНИЙ}
\def\titkol{Средства обеспечения отказоустойчивости приложений}

\def\aut{В.\,Н.~Захаров$^1$, В.\,А.~Козмидиади$^2$}
\titel{\razd}{\tit}{\aut}{\titkol}


\Abst{Рассмотрены проблемы построения отказоустойчивых серверов, возникающие в связи с недетерминированностью поведения приложений. Предложена формальная модель, описывающая поведение приложения, основными объектами которой являются ресурсы и события. Предложены алгоритмы протоколирования работы приложения на резервном узле кластера, а также восстановления и продолжения его работы при отказе основного узла. При этом для клиентов сбой остается незаметным, за исключением некоторого увеличения времени обслуживания.}

\KW{сервер приложений $\bullet$ прозрачная отказоустойчивость $\diamond$
 процесс $\diamond$ ресурс $\diamond$ событие $\diamond$ контрольная точка
$\bullet$ детерминированность}

\vskip 12pt plus 6pt minus 3pt

\begin{multicols}{2}

\section*{ВВЕДЕНИЕ}

Средства вычислительной техники стали использоваться в областях,
требующих безотказной работы систем в течение многих лет (24 часа
в сутки, 365 дней в году).

\label{st\stat}

\footnotetext{$^1$ФГУП Центральный институт авиационного моторостроения
им. П.И. Баранова, Москва, Россия}
\footnotetext{$^2$ФГУП Центральный институт авиационного моторостроения
им. П.И. Баранова, Москва, Россия}

К таким областям относятся, например, центры хранения и обработки данных  в сетях (системы резервирования билетов, биллинговые,  банковские и т.д.), массированные распределенные вычисления (GRID-вычисления) и другие.

\thispagestyle{headings}

Обычно в подобных системах применяются частные решения, ориентированные в основном на обеспечение надежного хранения данных (например, файловые серверы, использующие для хранения RAID-контроллеры) и корректного их состояния при отказах (серверы баз данных с транзакционным выполнением запросов). Однако большинство приложений не гарантируют, что не произойдет потери части данных при отказе системы. Обычно предполагается, что клиентские средства должны повторять запросы после восстановления серверов, для того, чтобы данные не были потеряны, или что можно сделать возврат по времени на некоторое время назад и повторить работу с этого места. Однако далеко не все клиентские средства и условия применения приложений допускают это.

Отказоустойчивые системы для критически важных приложений, корректно решающие проблемы восстановления после сбоев,   предлагаемые ведущими производителями, как правило, дороги. Кроме того, они включают специфические серверные и клиентские приложения, не совместимые со стандартными приложениями, не обеспечивающими отказоустойчивость. Примером такого подхода к решению проблемы отказоустойчивости  хранения данных являются системы NetApp FAS компании Network Appliance, работающие на базе специализированной операционной системы Data ONTAP [1].

Построение отказоустойчивых систем, использующих серверы со стандартными приложениями, в свете вышесказанного, является актуальной проблемой, вызывающей значительный интерес. Рассмотрение методов достижения прозрачной отказоустойчивости таких систем и является предметом статьи.
\begin{figure*} %fig1
\vspace*{1pt}
\begin{center}
\mbox{%
\epsfxsize=1.6in
\epsfxsize=100mm
\epsfbox{BbR-1.eps}
}
\end{center}
\vspace*{-9pt}
\Caption{Базовый вариант трубы с разными выходными устройствами
(цилиндрическое, расширяющееся и сужающееся сопло)
\label{f1bab}}
\vspace*{-3pt}
\end{figure*}


\section{ОСНОВНЫЕ ПОНЯТИЯ И ПОДХОДЫ}

Под сервером в данной работе понимается вычислительный центр
(отдельный компьютер или кластер) в сети, предоставляющий клиентам
(пользователям, клиентским компьютерам) определенные услуги, разделяя
между ними свои ресурсы. Подобные серверы названы серверами приложений.
Широко распространенным примером сервера такого типа является файловый сервер, обеспечивающий удаленный коллективный доступ к файловой системе. Часто используются вычислительные серверы, предоставляющие клиентам возможность выполнять на них свои программы (например, в центрах коллективного пользования).


Обычно приложение представляет собой программу или группу программ, работающих в операционной среде, создаваемой операционной системой (в другой терминологии - один или несколько взаимодействующих процессов или потоков (threads)), которые реализуют функциональность сервера. Для построения отказоустойчивых серверов приложений широко используется кластерная технология. Следуя [2], кластером, названа разновидность параллельной или распределенной системы, которая:
\begin{itemize}
\item состоит из нескольких компьютеров (узлов кластера), связанных как минимум необходимыми коммуникационными каналами;
\item используется как единый, унифицированный компьютерный ресурс.
\end{itemize}

Прозрачная отказоустойчивость (Transparent Fault Tolerance, TFT) сервера приложений - это такое его поведение при возникновении аппаратных или программных отказов либо отказов в сети, при котором:
\begin{itemize}
\item отказ не вызывает потери или искажения данных, находящихся в базе данных сервера;
\item сервер продолжает нормально функционировать, несмотря на имевшие место отказы.
\end{itemize}

Клиенты сервера "не замечают" произошедших отказов. Единственным\footnote{допустимым
отклонением сервера от нормального поведения с точки зрения клиента является
некоторое увеличение времени обслуживания} (на несколько секунд или десятков секунд).

Обычно приложения, работающие на серверах приложений, не ориентированы на прозрачную отказоустойчивость. Они "заботятся" лишь о собственной целостности (например, состояния файловой системы или базы данных). Восстановление работоспособности сервера приводит к разрыву соединений с клиентами и потере их запросов. Это замечают клиенты - запросы следует повторять, на что клиентские приложения далеко не всегда рассчитаны. В данной работе предполагается, что приложения (прикладные программные средства), выполняемые на сервере, являются стандартными, то есть не имеют специальных средств, обеспечивающих отказоустойчивость.
\begin{figure*}[b] %fig1
\vspace*{1pt}
\begin{center}
\mbox{%
\epsfxsize=1.6in
\epsfxsize=100mm
\epsfbox{BbR-1.eps}
}
\end{center}
\vspace*{-9pt}
\Caption{Базовый вариант трубы с разными выходными устройствами
(цилиндрическое, расширяющееся и сужающееся сопло)
\label{f1bab}}
\vspace*{-3pt}
\end{figure*}

Серьезные исследования в области обеспечения отказоустойчивости серверов были развернуты после создания вычислительных серверов, предназначенных для решения задач, требующих больших вычислительных ресурсов. Решение этих задач выполняется на суперкомпьютерах, обеспечивающих массово-параллельные вычисления и представляющих собой кластеры из сотен и тысяч узлов (процессоров). Однако даже на этих "монстрах" решение может требовать десятков или сотен часов, и одиночный сбой, если не предприняты специальные меры, может привести к необходимости начинать работу сначала. Обычно решение вычислительной задачи в таких случаях осуществляется в модели относительно редко взаимодействующих между собой процессов, выполняемых на разных узлах кластера. Эти взаимодействия нужны для координации работы процессов, в частности, для обмена данными и промежуточными результатами. Взаимодействия опираются на специальный протокол, называемый MPI (Message-Passing Interface) и представляющий собой стандарт "de facto" [3].

Для преодоления последствий сбоя достаточно давно была разработана и широко применяется технология, опирающаяся на механизм контрольных точек (checkpoints) [4-6]. По этой технологии система должна иметь стабильную память, которая не меняется при отказах. Соответствующие программные средства периодически сохраняют информацию о состоянии процессов приложения в стабильной памяти. Все процессы также имеют доступ к устройству стабильной памяти.  В случае отказа или сбоя, записанная в стабильную память информация используется для повторения вычисления с момента, когда была записана эта информация, то есть выполняется откат назад по времени. Данные, сохранение которых позволяет выполнить откат, называются контрольной точкой. В качестве устройства стабильной памяти может использоваться дисковый том, энергонезависимая оперативная память, память другого узла или узлов кластера. В последнем случае узел, которому требуется сохранить информацию, пересылает ее через быстрый канал связи на другой узел. Стабильная память после отказа одного из узлов должна быть доступной узлу, на котором делается повтор.

Однако решение, опирающееся только на контрольные точки, не является прозрачным, поскольку не скрывает от клиентов факт отказа системы и требует от них выполнения определенных действий. Так как при работе процессы обмениваются сообщениями, возможны два варианта решения проблемы. Первый - все процессы выполняют записи контрольных точек одновременно, что затруднительно. Второй вариант, при несоблюдении синхронности, - возврат в каждом процессе к такому скоординированному набору контрольных точек, при котором невозможна противоречивая ситуация. Такая ситуация возникает, когда один процесс вернулся к контрольной точке, после которой он должен получить сообщение от другого процесса, а этот другой процесс вернулся к точке, которая следует за выдачей этого сообщения. Однако при повторе ожидаемое первым процессом сообщение не поступит. В этом случае  возможен эффект домино, в результате процессы оказываются отброшены как угодно далеко назад.

В этом состоит первая проблема, которую необходимо преодолеть.

Если нужно, чтобы последствия отказа узла не были видны клиенту,  это означает:
\begin{itemize}
\item клиент не должен терять и потом восстанавливать соединения с сервером;
\item клиент не должен повторять свои запросы;
\item клиент не должен повторно получать сообщения, которые он уже получил.
\end{itemize}

Вторая проблема, которую надо решать, связана с недетерминированностью поведения сервера приложений. Приведем пример.  Пусть имеется система продажи билетов на самолеты. Два клиента одновременно обратились к системе с запросом билета на один и тот же рейс. Клиентам безразлично, какие места им зарезервирует система. Система выполняет запросы клиентов параллельно, поэтому в какой-то момент между процессами, обрабатывающими эти запросы, может возникнуть конкуренция за ресурс - в данном случае, скажем, рейс. Один из процессов захватывает ресурс первым, резервирует место и освобождает ресурс. Потом второй процесс проделывает то же самое.

Порядок, в котором в этом примере процессы захватили ресурс, зависит от многих факторов и, в конечном счете, случаен. Однако  это не мешает правильному функционированию системы, поскольку клиентам важно одно - получить билеты, причем на разные места. Однако отсутствие детерминизма в поведении приложения приводит к тому, что при повторном выполнении могут быть получены другие результаты: например, клиенту уже сообщено, что ему зарезервировано место №5, а при повторе может получиться, что зарезервировано место №6. Система должна устранить это несоответствие и сделать его невидимым для клиента.
\begin{figure*} %fig1
\vspace*{1pt}
\begin{center}
\mbox{%
\epsfxsize=1.6in
\epsfxsize=100mm
\epsfbox{BbR-1.eps}
}
\end{center}
\vspace*{-9pt}
\Caption{Базовый вариант трубы с разными выходными устройствами
(цилиндрическое, расширяющееся и сужающееся сопло)
\label{f1bab}}
\vspace*{-3pt}
\end{figure*}

Недетерминированность поведения системы это следствие, по крайней мере, двух обстоятельств. Во-первых, это присущая системам с разделением времени неопределенность в порядке выполнения процессов. Во-вторых, это конкуренция процессов за общие ресурсы. Перечислим некоторые причины недетерминированного поведения приложений:
\begin{itemize}
\item синхронизация процессов с помощью семафоров или атомарных операций над операндами в общей памяти процессов;
\item зависимость от порядка получения клиентских запросов;
\item время, затраченное процессом на обработку полученного запроса;
\item генераторы случайных чисел;
\item системное управление процессами и потоками;
\item локальные таймеры;
\item доступ к реальному времени.
\end{itemize}

По различным  причинам время, которое тратится на выполнение вычислительной задачи с одними и теми же исходными данными, не является константой, то есть повторное выполнение может дать другое время. Процессы используют общие ресурсы, обращение к которым требует организации очередности выполнения (сериализации) - первым пришел, первым захватил. И, наконец,  результат работы процесса может зависеть от состояния ресурса, а это состояние может изменить другой процесс, ранее захвативший ресурс. Все это создает значительные трудности при попытках воспроизведения поведения процессов с сохраненной контрольной точки.

Прозрачная отказоустойчивость серверов приложений обычно осуществляется переносом приложения на другой узел кластера, идентичный первому по конфигурации аппаратных средств и операционной среды. Это делается методом, называемым snapshot/restore. На основном узле (оригинале)  периодически фиксируется состояние приложения на этом узле кластера (так называемый снимок или snapshot). После отказа оригинала на резервном узле (копии) делается восстановление (restore), то есть восстанавливается последнее зафиксированное состояние приложения. Операционная среда при этом приводится в состояние, которое соответствует моменту изготовления снимка. После этого узел-копия продолжает работу с зафиксированного места. Сравнение метода  snapshot/restore с другими подходами приведено в [7].

Ниже рассматриваются информационные  технологии, позволяющие решить ряд принципиальных вопросов, связанных с реализацией прозрачной отказоустойчивости серверов приложений. Ими являются:
\begin{itemize}
\item виртуализация операционной среды, в которой работает серверное приложение;
\item отказоустойчивая реализация протокола TCP;
\item создание контрольных точек состояния приложения и файловой системы, которые делаются внешним по отношению к приложению образом;
\item восстановление серверного приложения на основании контрольной точки.
\end{itemize}
\begin{figure*} %fig1
\vspace*{1pt}
\begin{center}
\mbox{%
\epsfxsize=1.6in
\epsfxsize=100mm
\epsfbox{BbR-1.eps}
}
\end{center}
\vspace*{-9pt}
\Caption{Базовый вариант трубы с разными выходными устройствами
(цилиндрическое, расширяющееся и сужающееся сопло)
\label{f1bab}}
\vspace*{-3pt}
\end{figure*}

\section{МОДЕЛЬ ОПИСАНИЯ ПОВЕДЕНИЯ ПРИЛОЖЕНИЯ}

Предлагаемый подход опирается на построение модели вычислений, связанной с использованием понятия времени в многопроцессных приложениях. Впервые подобные проблемы были изучены в классической работе Л. Лампорта [8].

Многопроцессными приложения называются потому, что в них параллельно работают несколько процессов. Процесс ведет себя детерминированно, пока в предписанном кодом порядке выполняет процессорные инструкции. Конечно, его работа может быть прервана практически в любой момент и процессор передан другому процессу или ядру. Поэтому абсолютное время, которое затрачивает процесс на выполнение определенной работы, не  константа, а случайная  величина. То же  относится к относительному времени, то есть времени, которое процесс занимал процессор,  поскольку одни и те же обращения к операционной среде могут вызвать работы разной длительности, а значит потребовать разное время на свое выполнение.

Кэшированность инструкций и данных, а также длина хэш-списков влияют на действительное время пребывания в операционной среде. Утрачивает смысл понятие одновременность действий, поскольку  нельзя установить, выполнили ли два разных процесса какие-либо действия одновременно или одно из них предшествовало другому. Таким образом, с процессом можно связать только его локальное время, которое линейно упорядочивает события,  происходившие в этом процессе.  Глобальное время, линейно упорядочивающее действия во всех процессах, отсутствует. Расстояние (в этом качестве используется время) между действиями оказывается случайной величиной.

Эти соображения важны, поскольку процессы в интересующих нас приложениях взаимодействуют и используют общие ресурсы. Для взаимодействия они используют средства синхронизации, предоставляемые операционной средой - например, наборы семафоров SVR4 (System V Release 4), POSIX-семафоры, бинарные семафоры и другие примитивы взаимного исключения (POSIX- mutual exclusion locks) и т.д. Подобные средства операционной среды, которые позволяют процессам синхронизировать свою деятельность друг с другом или сериализовать обращения к совместно используемым объектам,  будут ниже  называться ресурсами.

С каждым ресурсом связано свое локальное время, линейно упорядочивающее события в жизни ресурса. Например, в случае двоичных семафоров это создание семафора, а также его захват и освобождение процессом. Заметим, что событие - это не намерение процесса (например, захватить бинарный семафор), а сам факт захвата семафора процессом (т.е. успешное выполнение намерения). От изъявления намерения до его осуществления может многое произойти. Например, семафор, который хочет захватить рассматриваемый процесс, принадлежал другому процессу, потом тот процесс его освободил, но семафор был сначала передан операционной средой третьему процессу, который также на него претендовал, и т.д. Поведение рассматриваемого процесса в это время нас не интересует - он ресурсом еще не овладел, а только его захват определяет его дальнейшее поведение. По причинам,  изложенным выше, расстояние между двумя событиями - случайная величина. Однако, события замечательны тем, что они одновременно присутствуют и в локальном времени процесса, и в локальном времени ресурса. Поэтому все, что произошло в истории процесса или/и ресурса до этого события, предшествует ему. Далее  будет считаться, что истории и ресурсов и процессов состоят только из событий, причем между двумя последовательными событиями в жизни процесса последний ведет себя детерминированно.

Это означает, что на  поведении процесса сказывается только его предыдущая история, то есть состояние ресурсов, с которыми он взаимодействовал. Это свойство процессов ниже будет называться локальной детерминированностью. Этим свойством не обладают ресурсы, поскольку - следующее событие в истории ресурса не определяется однозначно по его предыдущей истории. Утверждение, касающееся детерминированного поведения процессов, неявно опирается на предположение,  что учтены все ресурсы, которые могут привести к  недетерминированности процессов.

Таким образом, описанное нами очень неформально время в многопроцессном комплексе представляет собой отношение частичного порядка, введенное на множестве событий. Зная полное состояние комплекса в некоторый момент времени,  нельзя однозначно определить, какое событие в истории ресурса наступит следующим. Можно говорить только о вероятности наступления того или иного события. Недетерминированность поведения есть следствие двух обстоятельств. Во-первых, это неопределенность времени, которое тратит процесс на переход от одного события к другому. Во-вторых, конкуренция процессов за общие ресурсы.

Выполнение приложения, на множестве событий которого введена частичная упорядоченность, можно описать направленным ациклическим графом выполнения. Вершинами этого графа являются события, с каждым  из которых связаны две входящие в него дуги. Одна дуга начинается в событии, которое непосредственно предшествует данному событию в истории процесса, другая - в истории ресурса.

Построение средств обеспечения прозрачной отказоустойчивости приложений опирается на следующее утверждение: для восстановления работы приложения после отказа достаточно располагать:
\begin{itemize}
\item контрольной точкой, которая отражает на некоторый момент времени состояния процессов и других ресурсов, образующих приложение;
\item графом выполнения приложения, который описывает работу приложения, начинающуюся с контрольной точки и заканчивающуюся отказом. Данные, которые нужны для построения графа выполнения, далее называются протоколом.
\end{itemize}
\begin{figure*} %fig1
\vspace*{1pt}
\begin{center}
\mbox{%
\epsfxsize=1.6in
\epsfxsize=100mm
\epsfbox{BbR-1.eps}
}
\end{center}
\vspace*{-9pt}
\Caption{Базовый вариант трубы с разными выходными устройствами
(цилиндрическое, расширяющееся и сужающееся сопло)
\label{f1bab}}
\vspace*{-3pt}
\end{figure*}

Вся эта информация должна находиться в стабильной памяти, не разрушающейся при отказе.

Ниже неформально описан алгоритм восстановления работы приложения после отказа, который опирается на наличие контрольной точки и графа выполнения. Будем считать, что в распоряжении имеются средства, позволяющие остановить процесс в тот момент, когда он намерен совершить некоторую операцию над ресурсом. Заметим, что событие в графе выполнения соответствует не изъявлению намерения, а его удовлетворению, то есть завершению выполнения операции.

Предварительно сделаем следующее:
\begin{itemize}
\item используя контрольную точку, приведем приложение в состояние, соответствующее этой контрольной точке;
\item в графе выполнения пометим все вершины (события) как "не наступившие". У некоторых вершин графа отсутствуют им непосредственно предшествующие; соответствующие события наступили сразу же после создания контрольной точки. Для каждой такой вершины включим в граф дополнительную вершину, ей предшествующую в истории процесса, и отметим эту дополнительную вершину как "наступившую";
\item разрешим процессам приложения выполняться.
\end{itemize}

Пусть некоторый процесс проявляет намерение выполнить операцию над каким-либо ресурсом. Отыщем для этого процесса в его истории последнее наступившее событие. Следующее в его истории событие - это то, которое соответствует требуемой операции. Посмотрим, наступило ли событие в истории ресурса, которое ему предшествует. Если нет, переведем процесс в состояния ожидания, отметив в предшествующем событии, что данный процесс ожидает его наступления. Если да, разрешим процессу выполняться, то есть выполнить операцию над ресурсом.

Пусть некоторый процесс объявляет, что он выполнил операцию над каким-либо ресурсом (это соответствует моменту протоколирования при оригинальном выполнении). Отыщем для этого процесса в его истории последнее наступившее событие и перейдем к следующему событию в его истории. Это опять то событие, которое мы рассматриваем. Отметим его как "наступившее". Если наступления этого события ожидал какой-нибудь процесс, выведем этот процесс из состояния ожидания. Наконец, разрешим процессу, выполнившему операцию, продолжаться дальше.

Когда выясняется, что наступили все события графа выполнения, повторное выполнение считается законченным.

Естественным следствием из сказанного является следующее утверждение: для того, чтобы размер протокола не рос неограниченно, нужно периодически создавать контрольные точки, очищая при этом протокол.

\section{ФОРМАЛЬНОЕ ОПИСАНИЕ МОДЕЛИ ПОВЕДЕНИЯ МНОГОПРОЦЕССНОГО ПРИЛОЖЕНИЯ}
\begin{figure*} %fig1
\vspace*{1pt}
\begin{center}
\mbox{%
\epsfxsize=1.6in
\epsfxsize=100mm
\epsfbox{BbR-1.eps}
}
\end{center}
\vspace*{-9pt}
\Caption{Базовый вариант трубы с разными выходными устройствами
(цилиндрическое, расширяющееся и сужающееся сопло)
\label{f1bab}}
\vspace*{-3pt}
\end{figure*}

Опишем формально поведение приложения, неформальное описание которого было приведено выше. Рассматриваются два типа объектов:
\begin{itemize}
\item ресурсы (r), например, наборы семафоров (POSIX- или SVR4-семафоры), бинарные семафоры (POSIX-mutex's), таймер реального времени, сокеты (sockets), то есть двусторонние виртуальные соединения с внешним миром;
\item процессы (p), например, процессы или потоки (threads) пользователя.
\end{itemize}

\end{multicols}

\label{end\stat}

%\def\stat{batr}

\def\tit{НОВЫЙ МЕТОД ВЕРОЯТНОСТНО-СТАТИСТИЧЕСКОГО\newline
АНАЛИЗА ИНФОРМАЦИОННЫХ ПОТОКОВ
В~ТЕЛЕКОММУНИКАЦИОННЫХ СЕТЯХ$^*$}
\def\titkol{Новый метод вероятностно-статистического
анализа информационных потоков
в~телекоммуникационных сетях}
\def\autkol{Д.\,А.~Батракова, В.\,Ю.~Королев, С.\,Я.~Шоргин}
\def\aut{Д.\,А.~Батракова$^1$, В.\,Ю.~Королев$^2$, С.\,Я.~Шоргин$^3$}

\titel{\tit}{\aut}{\autkol}{\titkol}

{\renewcommand{\thefootnote}{\fnsymbol{footnote}}\footnotetext[1]{Работа 
выполнена при поддержке РФФИ, проекты №№\,04-01-00671, 05-07-90103.} 
\renewcommand{\thefootnote}{\arabic{footnote}}}
 \footnotetext[1]{ИПИ РАН, 
daria.batrakova@gmail.com} \footnotetext[2]{Факультет вычислительной математики 
и кибернетики МГУ им.~М.\,В.~Ломоносова, ИПИ РАН, vkorolev@comtv.ru} 
\footnotetext[3]{ИПИ РАН, sshorgin@ipiran.ru}



\Abst{В данной работе предлагается метод исследования стохастической структуры
хаотических информационных потоков в сложных телекоммуникационных
сетях. Предлагаемый метод основан на стохастической модели
телекоммуникационной сети, в рамках которой она представляется в виде
суперпозиции некоторых простых последовательно-параллельных структур.
Эта модель естественно порождает смеси гамма-распределений для времени
выполнения (обработки) запроса сетью. Параметры получаемой смеси
гамма-распределений характеризуют стохастическую структуру
информационных потоков в сети. Для решения задачи статистического
оценивания параметров смесей экспоненциальных и гамма-распределений
(задачи разделения смесей) используется ЕМ-алгоритм. Чтобы проследить
изменение стохастической структуры информационных потоков во времени,
ЕМ-алгоритм применяется в режиме скользящего окна. Описывается
программный инструментарий для применения полученного решения к
реальным статистическим данным. Приводится интерпретация результатов.}

\KW{телекоммуникационные сети; информационные потоки;
разделение смесей  распределений;
метод скользящего окна;  программа для разделения смесей}

\vskip 24pt plus 9pt minus 6pt

\thispagestyle{headings}

\begin{multicols}{2}


\label{st\stat}

\section{Введение}

Развитие телекоммуникационных сетей, их усложнение поставило перед
инженерами важную прикладную задачу исследования характеристик
информационных потоков, возникающих в этих сетях. Здесь под
информационным потоком мы будем понимать упорядоченное движение
любого вида информации по сети.

Если на заре эры телекоммуникаций, в эпоху первых телефонных линий и
телеграфа эта проблема не была столь насущной, то со временем, при
постепенном охвате мирового пространства сетями возникла необходимость в
построении и исследовании адекватных моделей сетей и процессов,
происходящих в них.

\thispagestyle{headings}


Современные сети~--- это \textit{конвергентные} сети, т.е.\ совокупность крайне
разнородных как по топологии, так и по физической архитектуре сетей, которые
предлагают конечному пользователю самые разнообразные сервисы. Это~--- огромное
виртуальное и физическое пространство, состоящее из миллионов процессоров,
операционных платформ, линий передачи данных и стыковочных узлов.
%
Существует множество классификаций телекоммуникационных сетей по различным
признакам:
\begin{itemize}
\item масштабу (локальные сети~--- LAN, масштаба города~---
MAN, широкого масштаба~--- WAN);
\item топологии, или логической организации (<<звезда>>,
<<кольцо>>, <<шина>>);
\item физической организации (оптические сети, радио);
\item предлагаемым услугам (сотовые сети, для доступа в
Интернет);
\item назначению (военные, гражданские) и~др.
\end{itemize}


Конвергентная сеть входит во все эти классы, причем, как правило,
обладает всеми этими признаками. Поэтому построение модели для ее анализа
является и очень важной, и очень сложной задачей.

Существуют достаточно многочисленные математические методы, ориентированные на
моделирование и анализ телекоммуникационных сетей. В~большинстве своем они
основываются на теории массового обслуживания, то есть разделе теории
вероятностей, посвященном описанию функционирования сложных систем обслуживания
(в том чис\-ле телекоммуникационных сетей и систем) с помощью стохастических
процессов особого вида и анализу таких процессов. Указанная теория является
весьма развитой и широко применяется на практике. Тем не менее, ее применимость
ограничена~--- во-первых, все возрастающей сложностью структур и дисциплин
обслуживания в рас\-смат\-ри\-ва\-емых реальных сетях. Эта сложность во многих
случаях принципиально не может найти адекватного отображения в моделях
массового обслуживания, даже несмотря на постоянно растущую сложность самих
этих моделей. В результате даже модели, допускающие точный математический
анализ, дают возможность расчета всего лишь приближенных значений характеристик
реальных сетей, ибо предположения, принимаемые при построении моделей, во
многих случаях не соответствуют практике. Во-вторых, для описания
телекоммуникационной сети в виде сети массового обслуживания исследователь
должен располагать детальным описанием структуры сети, что далеко не всегда
имеет мес\-то на практике. В-третьих, разработано крайне мало моделей массового
обслуживания, в которых используется в качестве входной информация о
наблюдаемых (статистических) показателях функционирования сети; в то же время,
такая информация очень часто доступна исследователю, и ее использование при
анализе сети весьма целесообразно.

В данной работе предлагается в определенной степени альтернативный подход к
моделированию сложных телекоммуникационных сетей. Строится и исследуется
вероятностная модель сложной телекоммуникационной сети как суперпозиции
достаточно простых структур. При этом практически никакая априорная информация
о структуре исследуемой сети не используется~--- наоборот, в результате
исследования модели исследователь получает приближенное представление об этой
структуре. Характеристики типовых простых структур, составляющих в совокупности
модель сети, и сети в целом при этом подходе описываются
гам\-ма-рас\-пре\-де\-ле\-ни\-я\-ми. Ставится задача оценки параметров модели
на основе статистических данных о функционировании сети, а также предлагается
математическое решение этой задачи. В статье описан также созданный на основе
разработанных математических методов программный инструментарий и приведены
результаты расчетов для реального трафика. {\looseness=-1

}

\section{Математическая модель и~постановка задачи}

\subsection{Логическая модель сети}
 %1.1

Рассмотрим абстрактную \textit{конвергентную телекоммуникационную
сеть}. Это может быть как крупномасштабная транспортная сеть (WAN), сеть
отдельного оператора масштаба города (MAN) с различными сервисами, так и
локальная сеть (LAN).

Любой из этих случаев можно описать как ($p,\,q$)-\textit{сеть}.

\medskip
\textbf{Определение 1.} В теории графов и сетей под ($p,\,q)$-сетью понимается
набор вида $S =$\linebreak $=(G,\,V^\prime ,\,V^{\prime\prime})$, где $G$~---
граф, а $V^\prime$ и $V^{\prime\prime}$~--- выборки из множества $V(G)$ (вершин
графа) длины~$p$ и $q$ соответственно. При этом выборка $V^\prime$
($V^{\prime\prime}$) считается \textit{входной} (\textit{выходной}) выборкой, а
ее $i$-я вершина называется $i$-\textit{м} \textit{входным} (\textit{выходным})
\textit{полюсом} или, иначе, $i$-\textit{м} \textit{входом} (\textit{выходом})
сети~$S$. Вершины, не участвующие во входной и выходной выборках сети,
считаются ее внутренними вершинами~\cite{1bat}.

Сеть $S$ (рис.~\ref{f1bat}) имеет $p$ точек входа~--- точек соединения
с внешней средой (это могут быть точки стыковки разнородных сетей, сетей
различных операторов, физические подключения к интерфейсам
маршрутизаторов и~т.п.). Под \textit{внешней средой} будем понимать другие
сети, которые передают данные в сеть~$S$. Отдельные <<единицы>> данных
(кадры, сообщения, датаграммы, пакеты) поступают на входы сети~$S$,
обрабатываются и подаются на каждый из $q$ выходов, которые могут быть
соединены как с конечными пользователями, так и с другими сетями.
\begin{figure*} %fig1
\vspace*{1pt}
\begin{center}
\mbox{%
\epsfxsize=139.7mm \epsfbox{bat-1.eps}
%\epsfxsize=139.698mm
%\epsfbox{bek-3.eps}
}
\end{center}
\vspace*{-9pt} \Caption{Абстрактная телекоммуникационная сеть \label{f1bat}}
\end{figure*}

Следует отметить, что структура сложных телекоммуникационных сетей обладает
свойством некоторого самоподобия, т.е.\ на каком бы уровне сетевой архитектуры
мы ни рассматривали поведение информационных потоков, мы можем выделить
некоторые базовые структуры, субпотоки, суперпозицией которых мы можем получить
модель конкретной сети, какой бы уровень <<детализации>> сегментов сети мы ни
взяли. Так, например, физические подключения к интерфейсам оптического
кросс-коннекта в этом смысле подобны <<виртуальным>> подключениям к портам TCP
на сервере приложений.

Итак, независимо от уровня сетевой архитектуры мы можем
рассматривать некоторую величину, характеризующую количество каких-либо
ресурсов сети~$S$, занимаемых в процессе передачи и обработки данных.  Это
могут быть ресурсы, относящиеся как к <<объему>> (памяти сетевого
устройства, количеству занятых линий, размеру пакета), так и ко <<времени>>
(времени обслуживания заявки, времени простоя). Эта величина случайна, т.к.\
мы не можем абсолютно точно сказать для сложной телекоммуникационной
сети, какое сообщение на какой из входов поступит и какого размера оно будет.
Таким образом, случайный характер данной величины определяется
случайностью поведения внешней среды.

Пусть $R$~--- это описанная выше случайная величина, $R>0$. Далее, не
ограничивая общности, будем подразумевать под ней время, необходимое для
какой-либо операции сети (обработки <<единицы>> данных), предполагая, что
время обработки прямо зависит от объема сообщения.

\subsection{Вероятностная модель сети} %1.2.

Даже не зная реальной топологии сети, мы можем описать
функционирование некоторых ее участков как процесс выполнения операций
(задач сети) в последовательном  порядке (например, если доступен только
один канал данных) или как процесс одновременного выполнения субопераций
(когда доступно более одного пути выполнения). Это значит, что мы можем
представить функционирование сложной телекоммуникационной сети как
\textit{суперпозицию} таких <<последовательных>> и <<параллельных>>
блоков.

Для построения вероятностной модели распределения~$R$ используется
комбинация асимптотического подхода, основанного на предельных теоремах
теории вероятностей, и принципа максимальной неопределенности (энтропии).

Рассмотрим следующую модель. Предположим, что мы можем разделить
сеть~$S$ на несколько сегментов $S_i$. Пусть $T$~--- случайная величина,
время выполнения операции в отдельно взятом блоке $S_i$ (сегменте сети).

Если операции выполняются \textit{параллельно}, то время, необходимое
для их выполнения~--- это максимальное время, затрачиваемое на какую-либо
субоперацию:
$$
T = \underset{i}{\max}\, T_i\,,
$$
где $T_i$~--- случайные величины для со\-от\-вет\-ст\-ву\-ющих субопераций.
Модель такого выполнения пред\-став\-ле\-на на рис.~\ref{f2bat}.

\begin{figure*} %fig2
\vspace*{1pt}
\begin{center}
\mbox{%
\epsfxsize=117.271mm
\epsfbox{bat-2.eps}
}
\end{center}
\vspace*{-9pt}
\Caption{Параллельное выполнение
\label{f2bat}}
\end{figure*}

Известно, что предельное распределение экстремальных значений для
выборок ~--- это экспоненциальное распределение с плотностью~\cite{2bat}
$$
f(x) =
\begin{cases}
\lambda e^{-\lambda x}\,, & x>0\,,\\
0\,, & x\leq 0\,,
\end{cases}
$$
где $\lambda >0$~--- параметр масштаба.

 Учитывая также энтропийный подход, естественно будет считать
распределение $T$ экспоненциальным, т.к.\ экспоненциальное распределение
обладает наибольшей энтропией среди всех распределений с $x>0$.

Если же операции сети выполняются \textit{последовательно}, то величина
$T$~--- это сумма времен $T_i$, необходимых для выполнения каждой
субоперации:
$$
T = \sum\limits_i T_i\,,
$$
где $T_i$~--- случайные величины для со\-от\-вет\-ст\-ву\-ющих субопераций.
%
Такая модель представлена на рис.~\ref{f3bat}.

\begin{figure*} %fig3
\vspace*{1pt}
\begin{center}
\mbox{%
\epsfxsize=139.592mm
\epsfbox{bat-3.eps}
}
\end{center}
\vspace*{-9pt}
\Caption{Последовательное  выполнение
\label{f3bat}}
\end{figure*}

Это значит, что распределение общей длительности $T$ выполнения
блока~--- это свертка распределений <<элементарных>> времен $T_i$
(экспоненциально распределенных).

Известно, что результатом свертки экспоненциальных распределений
является гамма-распределение, определяемое плотностью
$$
\g_{\lambda , \alpha} (x) =
\begin{cases}
\fr{\lambda_0^{\alpha_0}}{\Gamma (\alpha_0 )}\,x^{\alpha_0-1}
e^{\lambda_0 x}\,, & x>0\,,\\
0,\, & x\leq 0\,,
\end{cases}
$$
где $\alpha >0$~--- параметр формы,  $\lambda >0$  параметр масштаба, а
$\Gamma (z)$~--- гамма-функция Эйлера:
$$
\Gamma (z) = \int\limits_0^\infty x^{z-1} e^{-x}\,dx\,.
$$

\begin{figure*} %fig4
\vspace*{1pt}
\begin{center}
\mbox{%
\epsfxsize=120.831mm
\epsfbox{bat-4.eps}
}
\end{center}
\vspace*{-9pt}
\Caption{Модель пути  обработки сообщения сетью~$S$
\label{f4bat}}
\end{figure*}

Известно~\cite{2bat}, что класс гамма-распределений замкнут над операцией
свертки, поэтому ре\-зуль\-ти\-ру\-ющее распределение случайной величины~$R$
будет также гамма-распределением
$$
\g_{\lambda , \alpha} (x) =
\begin{cases}
\fr{\lambda^{\alpha}}{\Gamma (\alpha )}\,x^{\alpha -1} e^{-\lambda x}\,, &
x>0\,,\\
0,\, & x\leq 0\,.
\end{cases}
$$

В силу случайного характера ввода данных в сеть~$S$ из внешней среды маршрут
передачи данных становится случайным, что представлено на рис.~\ref{f4bat}. Это
означает, что параметры ре\-зуль\-ти\-ру\-юще\-го распределения~$R$ тоже
случайны. Отсюда имеем следующую модель \textit{смеси
гам\-ма-рас\-пре\-де\-ле\-ний}, определяемой плотностью

\begin{equation} %1
p(x) = \iint \g_{\lambda , \alpha}(x)\,dH (\lambda ,\,\alpha )\,,
\end{equation}
где $H(\lambda , \alpha )$~--- смешивающая функция, функция распределения
входных параметров.

Поясним понятие \textit{смеси распределений}.

\medskip
\textbf{Определение~2.} Пусть имеется двух\-па\-ра\-мет\-ри\-че\-ское
семейство $n$-мерных плотностей  распределения
\begin{equation}
F = \{ f_\omega (x;\, \theta (\omega ))\}\,,
\end{equation}
где одномерный (целочисленный или непрерывный) параметр $\omega$ в
качестве нижнего индекса функции $f$ определяет специфику общего вида
каж\-до\-го компонента~--- распределения смеси, а в качестве аргумента при
многомерном, вообще говоря, параметре $\theta$ определяет зависимость
значений хотя бы части компонентов этого параметра от того, в каком именно
составляющем распределении $f_\omega$ он присутствует. Кроме того, пусть
$P = \{P(\omega )\}$~--- \textit{семейство смешивающих функций}
распределения.

Функция плотности распределения
\begin{equation}
f(x) = \int f_\omega (x;\,\theta(\omega ))\,dP (\omega )
\end{equation}
называется $P$-\textit{смесью} (или просто \textit{смесью})
\textit{распределений} семейства~$F$,  интеграл в~(3) понимается в смысле
Лебега--Стильтьеса~\cite{3bat}.

\medskip
\textbf{Определение 3.} Семейство смесей~(3) называется
\textit{идентифицируемым} (\textit{различимым}), если из равенства
$$
\int f_\omega (x;\,\theta(\omega ))\,dP (\omega ) =\int f_\omega
(x,\,\theta(\omega )) dP^*(\omega )
$$
следует, что $P(\omega ) \equiv P^*(\omega )$ для всех $P \in P(\omega
)$~\cite{3bat}.

\subsection{Постановка задачи} %1.3.

Перед нами встает задача \textit{разделения} такой смеси. Вообще говоря,
задача разделения смесей распределений со смешивающими функциями
общего вида является \textit{некорректно поставленной}, т.к.\ она допускает
существование нескольких решений. Поэтому будем искать решение в классе
\textit{конечных идентифицируемых смесей распределений}, где смешивающая
функция дискретна.

Для этого сузим данное выше определение и будем рассматривать в дальнейшем лишь 
случай конечного числа $k$ возможных значений па\-ра\-мет\-ра~$\omega$, что 
соответствует конечному числу скачков смешивающих функций $P(\omega )$.  
Величины этих скачков как раз и будут играть роль \textit{удельных весов} 
(\textit{априорных вероятностей}) $p_j$ компонентов смеси. Более того, в нашем 
случае мы постулируем также однотипность компонентов распределений $f_j$, т.е.\ 
принадлежность всех $f_j$ к одному общему па\-ра\-мет\-ри\-че\-ско\-му 
семейству $\{ f(X;\,\theta )\}$, где $\theta$~--- многомерный, вообще говоря, 
параметр. Так что~(3) в этом случае может быть записано в виде
\begin{equation} %4
p(x) = \sum\limits^k_{j=1} p_j f_j (x;\,\theta_j )\,.
\end{equation}

Переформулируем понятие идентифицируемости (различимости) смесей
специально применительно к такому виду смесей.

\medskip
\textbf{Определение 4.} \textit{Конечная смесь}~(3) называется
\textit{идентифицируемой} (\textit{различимой}), если из равенства
$$
\sum\limits_{j=1}^k p_j f_j (x;\,\theta_j ) = \sum\limits_{l=1}^{k^*} p_l^* f_l
(x;\,\theta_l^* )
$$
следует, что $k=k^*$ и для любого $j$ ($1\leq j \leq k$) найдется такое $l$ 
($1\leq l \leq k^*$), что $p_j = p_l^*$ и $f_j (x;\,\theta_j ) = f_l 
(x;\,\theta_l^* )$~\cite{3bat}.

Решить эту задачу в выборочном варианте~--- значит по выборке
классифицируемых наблюдений
$X_1,\ldots , X_n, $ извлеченной из генеральной совокупности, яв\-ля\-ющей\-ся смесью~(3)
генеральных совокупностей типа~(2) (при заданном общем виде составляющих
смесь функций $f_j (x;\,\theta_j )$), построить статистические оценки для числа
компонентов смеси~$k$, их удельных весов $p_j$ и, главное, для каждого из
компонентов %f_j (x;\,\theta_j )$ анализируемой смеси. Далее будет считать, что
функции $f_j$ однозначно определяются своими параметрами $\theta_j$: $f_j
=f(x;\,\theta_j)$.

Однако не следует ставить знак тождества между задачей разделения смеси
и задачей статистического оценивания параметров в модели~(4) по выборке $
X_1,\ldots , X_n$, поскольку задача разделения сохраняет смысл и
применительно к генеральным совокупностям, т.е.\ в теоретическом
варианте~\cite{3bat}.

Итак, для статистического анализа на основе реальных данных мы
аппроксимируем нашу общую модель~(1) следующей:
$$
p(x) \approx \hat{p}(x) = \sum\limits_{j=1}^k p_j \g_{\lambda_j , \alpha_j}
(x)\,,
$$
где $p_j$~--- дискретные смешивающие параметры, $\g_{\lambda_j , \alpha_j}
(x)$~--- плотности гамма-распределений.

Такая аппроксимация не только позволяет решить поставленную статистическую
задачу, но и полу\-чить наглядную визуализацию результатов. Существуют
достаточно эффективные методики разделения смесей распределений, среди них~---
семейство так называемых \textit{ЕМ-алгоритмов}
(\textit{Expectation-Maximization Algorithms}).

Полученные результаты могут быть достаточно просто интерпретированы. Число
компонентов смеси символизирует число типичных параллельных или
последовательных структур. Значения параметров составляющих смесь
гам\-ма-рас\-пре\-де\-ле\-ний показывают <<степень параллелизма>>
соответствующей структуры: чем ближе параметр формы к~1, тем выше эта
<<степень>>. И наоборот, чем дальше значение параметра формы от~1, тем больше
последовательных операций выполняется в соответствующем блоке.

Веса компонентов характеризуют примерную долю использования
ресурсов для сообщений, соответствующих каждому распределению входных
данных.

Итак, предложенный подход позволяет получить представление о
стохастической структуре телекоммуникационной сети.

\section{ЕМ-алгоритм разделения смесей распределений}

\subsection{Описание алгоритма} %2.1.

Определяемый ниже итерационный алгоритм решения поставленной в
предыдущем разделе задачи относится к процедурам, базирующимся на
\textit{методе максимального правдоподобия}.

Этот алгоритм позволяет находить максимум логарифмической функции
правдоподобия по параметрам $p_1,\,p_2,\ldots ,\,p_k$, $\theta_1 ,\,\theta_2,\ldots ,\,
\theta_k$ при фиксированном $k$ по выборке $X_1, \ldots , X_n$, т.е.\ решение
оптимизационной задачи вида

\begin{equation} \sum\limits_{i=1}^n \ln \left ( \sum\limits_{j=1}^k p_j f_j
(X_i;\,\theta_j )\right ) \rightarrow \underset{p_j,\,\theta_j}{\max}\,.
\end{equation}

Конкретные алгоритмы, построенные по этой схеме, часто называют
\textit{алгоритмами типа ЕМ}, поскольку в каждом из них можно выделить два
этапа, находящихся по отношению друг к другу в последовательности
итерационного взаимодействия: \textit{оценивание} (\textit{Estimation}) и
\textit{максимизация} (\textit{Maximization})~\cite{4bat}.

Введем в рассмотрение так называемые апостериорные вероятности
$\g_{ij}$ принадлежности наблюдения $X_i$ к $j$-му классу:
\begin{equation} %6
\g_{ij} = \fr{p_j f(X_i;\,\theta_j )}{\sum\limits_{l=1}^k p_l f(X_i;\,\theta_l 
)} \ (i=1,\ldots , n;\ j=1,\ldots ,k)\,.\!\!\end{equation} 
Очевидно, что для 
всех $i=1,\ldots ,n$ и $j=1,\ldots ,k$
$$
\g_{ij} \geq 0,\quad \sum_{j=1}^k \g_{ij} =1\,.
$$


Далее обозначим $\Theta = (p_1,\ldots p_k,\,\theta_1,\ldots ,\theta_k )$ и
представим анализируемую логарифмическую функцию правдоподобия
$$
\ln L(\Theta ) = \sum\limits_{i=1}^n \ln \left (\sum\limits_{j=1}^k p_j f_j
(X_i;\,\theta_j )\right )
$$
в виде
\begin{multline}
\ln L (\Theta ) = \sum\limits_{j=1}^k\sum\limits_{i=1}^n \g_{ij} \ln p_j+{}\\
{}+\sum\limits_{j=1}^k\sum\limits_{i=1}^n \g_{ij} f(X_i;\,\theta_j)-
\sum\limits_{j=1}^k\sum\limits_{i=1}^n \g_{ij} \ln \g_{ij}\,.
\end{multline}

Справедливость этого тождества легко проверяется с учетом
$$
\sum\limits_{j=1}^k \g_{ij} =1\,.
$$

Далее идея построения итерационного алгоритма вычисления оценок
$\hat{\Theta} = (\hat{p}_1,\ldots , \hat{p}_k,\
\hat{\theta}_1,\ldots , \hat{\theta}_k)$
для параметров $\Theta = (p_1,\ldots , p_k,\ \theta_1,\ldots , \theta_k)$ состоит в
следующем:
\begin{enumerate}[1.]
\item Выбирается некоторое \textit{начальное приближение}~$\hat{\Theta}^0$.
\item \textbf{E-step:} вычисляются по формулам~(6) начальные приближения
$\g_{ij}^0$ для апостериорных вероятностей $\g_{ij}$~--- \textit{этап
оценивания}.
\item \textbf{M-step:} затем, возвращаясь к~(7), при вычисленных значениях
$\g^0_{ij}$ следует определить значения $\hat{\Theta}^1$ из условия
максимизации отдельно каждого из первых двух слагаемых правой
части~(7), поскольку первое слагаемое
$$
\sum_{j=1}^k \sum_{i=1}^n \g_{ij} \ln p_j
$$
зависит только от параметров $p_j$, а второе слагаемое
$$
\sum_{j=1}^k \sum_{i=1}^n \g_{ij} f(X_i;\,\theta_j )
$$
зависит только от параметров $\theta_j$~--- \textit{этап максимизации}.
\item Проверяется \textit{условие останова}:
$$
\parallel \Theta^{(t)} - \Theta^{t-1}\parallel <\varepsilon\,,
$$
где $t$~--- номер итерации, а
$\parallel\bullet\parallel$~--- евклидова норма, для некоторого $\varepsilon
>0$.
\end{enumerate}

Очевидно, решение оптимизационной задачи
$$
\sum\limits_{j=1}^k\sum\limits_{i=1}^n \g_{ij}^{(t)}\ln p_j \rightarrow
\underset{p_j}{\max}
$$
дается выражением (с учетом $\sum_{j=1}^k p_j =1$):
$$
p_{ij}^{(t+1)} =\fr{1}{n}\,\sum\limits_{i=1}^n \g_{ij}^{(t)}\,,
$$
где $t$~--- номер итерации, $t = 0$, 1, 2,\,\ldots

Решение оптимизационной задачи
$$
\sum\limits_{j=1}^k \sum\limits_{i=1}^n \g_{ij}^{(t)} f(X_i;\,\theta_j )
\rightarrow \underset{\theta_j}{\max}
$$
получить намного проще решения задачи~(5): выражение для $\theta_j$
записывается с учетом знания конкретного вида функций
$f(X,\,\theta)$~\cite{3bat}.

\subsection{О сходимости алгоритма} %2.2.

В работе М.\,И.~Шлезингера~\cite{5bat}, где эта схема (позднее названная
ЕМ-схемой) впервые предложена, установлены и основные свойства
реа\-ли\-зу\-ющих ее алгоритмов. В частности, было доказано, что при достаточно
широких предположениях \textit{предельные точки} всякой последовательности,
порожденной итерациями ЕМ-алгоритма, являются стационарными точками
оптимизируемой логарифмической функции правдоподобия $\ln L(\Theta )$ и что
найдется неподвижная точка алгоритма, к которой будет сходиться каждая из таких
последовательностей. Если дополнительно потребовать положительной
определенности информационной мат\-ри\-цы Фишера для $\ln L(\Theta )$ при
истинных зна\-че\-ни\-ях па\-ра\-мет\-ра $\Theta$, то можно показать, что
асимптотически по $n\rightarrow\infty$ (т.е.\ при больших выборках) существует
единственное сходящееся (по веро\-ят\-но\-сти) решение $\hat{\Theta}(n)$
уравнений метода максимального правдоподобия и, кроме того, существует в
пространстве параметров $\Theta$ норма, в которой последовательность
$\Theta^{(t)}(n)$, порожденная ЕМ-ал\-го\-рит\-мом, сходится к $\hat{\Theta}
(n)$, если только начальная аппроксимация $\hat{\Theta}^0$ не была слишком
далека от~$\hat{\Theta} (n)$. {%\looseness=1

}

Таким образом, результаты исследования свойств ЕМ-алгоритмов метода
максимального правдоподобия разделения смеси и их практическое
использование показали, что они являются достаточно работоспособными (при
известном чис\-ле компонентов смеси) даже при большом чис\-ле $k$ компонентов и
при высоких размерностях анализируемого признака~$X$~\cite{3bat}.

\subsection{Уравнения для смеси экспоненциальных распределений}
%2.3.

Применим описанный выше алгоритм к разделению смеси
экспоненциальных распределений:
$$
p(x) = \sum\limits_{j=1}^k p_j \lambda_j e^{-\lambda_j x}\,.
$$
Получаем следующие итерационные уравнения:
\begin{align*}
\g_{ij}^{(t+1)} & = \fr{p_j^{(t)}\lambda_j^{(t)}e^{-
\lambda_j^{(t)}X_i}}{\sum\limits_{l=1}^k p_l^{(t)}\lambda_l^{(t)}
e^{-\lambda_l^{(t)}X_i}}\,,\\
p_j^{(t+1)} & = \fr{1}{n}\,\sum\limits_{i=1}^n \g_{ij}^{(t)}\,.
\end{align*}

Чтобы найти  оценки $\lambda_j$, подсчитаем первую производную функции
$$\sum_{j=1}^k\sum_{i=1}^n \g_{ij}^{(t)} \ln (\lambda_j e^{-\lambda_j X_i}):$$
\vspace*{-8pt}
\begin{multline*}
\left ( \sum\limits_{j=1}^k \sum\limits_{i=1}^n
\g_{ij}^{(t)}\ln \left ( \lambda_j
e^{-\lambda_j X_i} \right ) \right )^\prime \lambda_j =\\[-3pt]
{}= \left (
\sum\limits_{j=1}^k\sum\limits_{i=1}^n \g_{ij}^{(t)}\ln (\lambda_j -\lambda_j X_i )
\right )^\prime \lambda_j =\\[-3pt]
{}= \sum\limits_{i=1}^n \g_{ij}^{(t)}\left (
\fr{1}{\lambda_j} - X_i \right )\,.
\end{multline*}

Разрешая уравнение
$$
\sum\limits_{i=1}^n \g_{ij}^{(t)}\left ( \fr{1}{\lambda_j} -X_i\right ) =0
$$
относительно $\lambda_j$, получаем следующее итерационное уравнение:
$$
\lambda_j^{(t+1)} = \fr{\sum\limits_{i=1}^n
\g_{ij}^{(t)}}{\sum\limits_{i=1}^n \g_{ij}^{(t)} X_i}\,.
$$

\subsection{Уравнения для смеси гамма-распределений } %2.4.

Применим теперь ЕМ-алгоритм к смеси гам\-ма-рас\-пре\-де\-ле\-ний вида
$$
p(x) = \sum\limits_{j=1}^k p_j \fr{\alpha_j^{\alpha_j} x^{\alpha_j -
1}}{\lambda_j^{\alpha_j} \Gamma (\alpha_j )}\,e^{-(\alpha_j / \lambda_j)x}\,.
$$

Такая параметризация удобна для нахождения
оценок~$\alpha_j$~\cite{6bat}.

Аналогичным способом выписываются итерационные уравнения:
\begin{align*}
\g_{ij}^{(t+1)} & =   \fr{p_j^{(t)}\fr{(\alpha_j^{\alpha_j} )^{(t)}
x^{\alpha_j - 1}}{(\lambda_j^{\alpha_j} )^{(t)}\Gamma (\alpha_j)}\,
e^{-(\alpha_j /\gamma_j)^{(t)}x}}{\sum\limits_{l=1}^k
p_l^{(t)}\fr{(\alpha_l^{\alpha_l})^{(t)} x^{\alpha_l -
1}}{(\lambda_l^{\alpha_l})^{(t)}\Gamma (\alpha_l )}\,
e^{-(\alpha_l /\lambda_l)^{(t)} x}}\,,\\
p_j^{(t+1)} & = \fr{1}{n}\,\sum\limits_{i=1}^n \g_{ij}^{(t)}\,.
\end{align*}

Далее найдем оценки $\lambda_j$ для данного случая, приравнивая
производную
\begin{equation} %8
\sum\limits_{j=1}^k \sum\limits_{i=1}^n \g_{ij}^{(t)} \ln \left (
\fr{\alpha_j^{\alpha_j} x^{\alpha_j -1}}{\lambda_j^{\alpha_j}\Gamma
(\alpha_j)}\,e^{-(\alpha_j /\lambda_j) x}\right )
\end{equation}
по $\lambda_j$ к нулю и разрешая относительно~$\lambda_j$ уравнение:
$$
\sum\limits_{i=1}^n \g_{ij}^{(t+1)}\left ( \fr{\alpha_j^{(t)}}{\lambda_j^{(t)}}
- \fr{\alpha_j^{(t)}X_i}{\left ( \lambda_j^{(t)}\right )^2}\right ) =0 \,.
$$
Получаем
$$
\lambda_j^{(t+1)} = \fr{\sum\limits_{i=1}^n \g_{ij}^{(t)}
X_i}{\sum\limits_{i=1}^n \g_{ij}^{(t)}}\,.
$$

Для того чтобы получить итерационные уравнения для $\alpha_j$, найдем
первую производную~(8):
\begin{multline*}
\left ( \sum\limits_{j=1}^k\sum\limits_{i=1}^n \g_{ij}^{(t)}\ln \left (
\fr{\alpha_j^{\alpha_j} x^{\alpha_j -1}}{\lambda_j^{\alpha_j}\Gamma (\alpha_j
)}\,e^{-(\alpha_j /\lambda_j ) x} \right ) \right )^\prime \alpha_j ={}\\[-3pt]
{}=\left ( \sum\limits_{j=1}^k\sum\limits_{i=1}^n \g_{ij}^{(t)}\ln \left (
\fr{\alpha_j^{\alpha_j}}{\lambda_j^{\alpha_j}}\right ) - \ln \Gamma (\alpha_j )+{} \right.\\[-3pt]
{}+\left.
(\alpha_j -1 )\ln X_i - \fr{\alpha_j}{\lambda_j}\,X_i \right )^\prime \alpha_j =\\[-3pt]
{}=\sum\limits_{i=1}^n \g_{ij}^{(t)} \left ( \ln \alpha_j +1-\ln \lambda_j -
\fr{\Gamma^\prime (\alpha_j )}{\Gamma (\alpha_j)}\right.+\\[-3pt]
{}+\left. \ln X_i - \fr{X_i}{\lambda_j}\right )\,;
\end{multline*}
\begin{multline*}
\sum\limits_{i=1}^n \g_{ij}^{(t)} \left(  \ln \alpha_j +1 -\ln \lambda_j -{}\right. \\[-3pt]
\left. {}-\fr{\Gamma^\prime (\alpha_j )}{\Gamma (\alpha_j )}+\ln X_i 
-\fr{X_i}{\lambda_j} \right) =0\,;
\end{multline*}
\begin{multline}
\fr{\Gamma^\prime (\alpha_j )}{\Gamma (\alpha_j )} ={}\\[-3pt]
{}= \fr{\sum\limits_{i=1}^n \g_{ij}^{(t)} \left ( \ln \alpha_j +1-\ln\lambda_j 
+\ln X_i -\fr{X_i}{\lambda_j} \right )}{\sum\limits_{i=1}^n \g_{ij}^{(t)}}\,.
\end{multline}
%
Здесь $\Gamma^\prime (\alpha_j ) / \Gamma (\alpha_j )$~--- это
\textit{логарифмическая производная гамма-функции}. Для нее существует так
называемое \textit{разложение Абрамовитца}--\textit{Стигана}~\cite{4bat}:
$$
\fr{\Gamma^\prime (\alpha ) }{ \Gamma (\alpha )} = \mathrm{log}\,\alpha -
\fr{1}{2\alpha }-\fr{1}{12\alpha^2 }+\ldots
$$

Подставим первые три члена разложения в~(9) и разрешим это уравнение
относительно~$\alpha_j$:
$$
\alpha_{ij}^{(t+1)} = \fr{\sum\limits_{i=1}^n
\g_{ij}^{(t+1)}}{2\sum\limits_{i=1}^n \g_{ij}^{(t +1)}\left ( \fr{X_i}{\lambda_j^{(t)}} -
\ln \fr{X_i}{\lambda_j^{(t)}} -1\right )}\,.
$$
В итоге получаем итерационные уравнения для ~$\alpha_j$.

\section{Описание программного обеспечения (программа~ЕМ)}

\subsection{Назначение программы} %3.1.

Разработанная авторами статьи программа ЕМ предназначена для решения задачи
разделения смесей экспоненциальных и гамма-распределений, поставленной в
разд.~2, с использованием ЕМ-ал\-го\-рит\-ма и формул, описанных в разд.~3.

\subsection{Инструменты разработки} %3.2.

Для создания программы была использована среда разработки Microsoft
Visual Studio .NET 2005 и объектно-ориентированный язык C\#. Для
визуализации результатов была использована свободно распространяемая
графическая библиотека ZedGraph~\cite{7bat}.


\subsection{Возможности  программы} %3.3.

\noindent
\begin{itemize}
\item Загрузка выборочных данных из текстового файла
\item Оценивание по выборке параметров смеси экспоненциальных
распределений
\item Оценивание по выборке параметров смеси гамма-распределений
\item Отслеживание изменений параметров смесей распределений во
времени в режиме <<скользящего окна>>
\item Построение гистограммы по выборке
\end{itemize}

\subsection{Входные и выходные данные. Функционирование
программы} %3.4.

В качестве \textit{входных данных} программа ЕМ получает:
\begin{itemize}
\item выборочные данные из текстового файла;
\item число компонентов смеси;
\item размер <<скользящего окна>>;
\item размер класса гистограммы.
\end{itemize}

На \textit{выходе} мы получаем:
\begin{itemize}
\item точечные оценки параметров смеси экспоненциальных
распределений;
\item точечные оценки параметров смеси гамма-распределений;
\item графическое изображение результирующей смеси распределения;
\item графическое изображение компонентов каж\-дой смеси;
\item графическое изображение того, как меняются параметры смесей
распределений с течением времени в режиме <<скользящего окна>>;
\item гистограмма, построенная по выборке;
\item значение статистического теста.
\end{itemize}

Выборочные данные загружаются из текстового файла в память программы и подаются
на вход двум независимо работающим реализациям ЕМ-алгоритма~--- для
идентификации смеси экспоненциальных распределений и для идентификации смеси
гамма-распределений. Результатом их работы являются наборы значений оцениваемых
параметров модели, предложенной в разд.~2. Кроме того, результирующие
распределения визуализируются в виде графиков. В программе можно запустить
режим <<скользящего окна>>, который для всех подвыборок заданного
размера с помощью ЕМ-алгоритма оценивает параметры смесей распределений этих
подвыборок. Все действия программы документируются в окне информации.

\section{Описание тестовых расчетов}

С использованием разработанной программы были проведены тестовые
расчеты на выборочных данных реального сетевого трафика.

На вход программы EM были поданы выборки трафика:
\begin{enumerate}[I]
\item Между лабораторией Lawrence Berkeley (Berkeley, California) и
внешним миром размера примерно 7000~\cite{8bat}~--- \textit{выборка~1}.
\item
Сети радиодоступа ЗАО <<Синтерра>> размера примерно 1000~\cite{9bat}~---
 \textit{выборка~2}.
\end{enumerate}

\subsection{Выборка 1 ``Berkeley''} %5.1.

При числе компонентов смеси~5 и случайном начальном приближении
были получены результаты, представленные в табл.~\ref{t1bat}.


Данные результаты иллюстрирует рис.~\ref{f5bat}.

Гистограмма  на рис.~\ref{f6bat} показывает статистическую значимость
полученных результатов.

Данная выборка обладает той особенностью, что она собиралась в течение
достаточно длительного времени и в ней агрегирован самый разнородный
трафик. Поэтому в ней присутствует не только большое количество
<<коротких>> сообщений (что обычно для выборок из телетрафика), но и
некоторый массив сообщений средней длины, а также определенный
<<выброс>> больших сообщений. Это свидетельствует о \textit{пиковости}
телетрафика на довольно больших промежутках времени.

Как мы видим, ЕМ-алгоритм удачно справился с задачей идентификации
смеси.

\subsection{Выборка~2 ``Synterra''} %5.2.

Результаты применения ЕМ-алгоритма к выборке ``Synterra''
представлены в табл.~\ref{t2bat}.
\begin{table*}\small
\begin{minipage}[t]{76mm}
\begin{center}
\Caption{Результаты применения ЕМ-алго\-рит\-ма к выборке~1 ``Berkeley'' 
\label{t1bat}} \vspace*{2ex}

\tabcolsep=8.7pt
\begin{tabular}{|c|c|c|}
\hline
№&Начальное приближение&Результат\\
\hline
\multicolumn{3}{|c|}{$P$}\\
\hline
0&0,2&0,1896\\
1&0,2&0,1858\\
2&0,2&0,1830\\
3&0,2&0,2259\\
4&0,2&0,2154\\
\hline
\multicolumn{3}{|c|}{$\alpha$}\\
\hline
0&2,7028&10,9783\hphantom{9}\\
1&3,6273&5,8621 \\
2&5,7598&2,7092\\
3&0,2315&1,0235\\
4&0,9110&0,4772\\
\hline
\multicolumn{3}{|c|}{$\lambda$}\\
\hline
0&85,2066&137,1714  \\
1&23,9592&136,7349\\
2&63,8425&132,6482\\
3&\hphantom{9}1,8026&116,7317\\
4&98,3882&102,5278\\
\hline
\end{tabular}
\end{center}
\end{minipage}\hfill
\begin{minipage}[t]{76mm}
%\end{table*}
%\begin{table*}\small
\begin{center}
\Caption{Результаты применения ЕМ-алго\-рит\-ма к выборке~2 ``Synterra'' 
\label{t2bat}} \vspace*{2ex}

\tabcolsep=8.7pt
\begin{tabular}{|c|c|c|}
\hline
№&Начальное приближение&Результат\\
\hline
\multicolumn{3}{|c|}{$P$}\\
\hline
0&0,2&$0{,}3815\hphantom{{}\cdot 10^{-9}}$\\
1&0,2&$0{,}3594\hphantom{{}\cdot 10^{-9}}$\\
2&0,2&$0{,}2589\hphantom{{}\cdot 10^{-9}}$\\
3&0,2&$0{,}4401\cdot 10^{-9}$\\
4&0,2&$0{,}0\hphantom{{}\cdot 10^{-9}999}$\\
\hline
\multicolumn{3}{|c|}{$\alpha$}\\
\hline
0&6,0804&1,5833\\
1&3,1838&0,8554\\
2&1,4886&0,4557\\
3&4,6407&0,2278\\
4&3,7843&0,1139\\
\hline
\multicolumn{3}{|c|}{$\lambda$}\\
\hline
0&17,3387&15,8682\\
1&47,8294&16,9150\\
2&54,1984&19,2866\\
3&\hphantom{1}8,6254&19,2866\\
4&\hphantom{1}5,7252&19,2866\\
\hline
\end{tabular}
\end{center}
\end{minipage}
\end{table*}


Данные результаты иллюстрируют рис.~\ref{f7bat}.


Эти результаты также отражают действительную картину, как показано на
рис.~\ref{f8bat}.


Этот трафик был снят с базовой станции <<Лукойл-Юго-Запад>> сети
широкополосного радиодоступа ЗАО <<Синтерра>>. Сеть радиодоступа
является реализацией так называемой <<последней мили>>, переносящей два
разных вида трафика: данные (Ethernet пакеты) и голос (IP-телефония, VoIP).
Поэтому здесь присутствуют в качестве основной массы короткие, но
интенсивные сообщения (пакеты SIP и голосовые фреймы), а также длинные
сообщения, содержащие данные.

Как мы видим, программная реализация ЕМ-ал\-го\-рит\-ма успешно справилась с
задачей разделения смесей распределений для этих двух выборок, что делает
данную программу удобным инструментом построения стохастической картины
конкретной сети. По полученным данным, используя метод интерпретации,
предложенный в разд.~2, можно получить представление о количестве
последовательных и параллельных структур вероятностной модели сети.

\subsection{Режим <<скользящего окна>>} %5.3.

Результаты для выборки
``Berkeley'' в режиме <<скользящего окна>>  представлены
на рис.~\ref{f9bat}.


Данные графики показывают изменение параметров распределений подвыборок выборки 
``Berkeley''. Видно, что параметры распределений подвыборок не остаются 
неизменными во времени, наоборот, они имеют внешне случайный характер. На 
рис.~\ref{f9bat},\,\textit{в} видна даже своеобразная пульсация первой 
компоненты.
%
На основании расчетов можно сделать вывод о том, что пиковость трафика
обусловливается как формой, так и интенсивностью сообщений.

\section{Заключение}

В данной работе исследована вероятностная модель  информационных потоков,
возникающих в сложных телекоммуникационных конвергентных сетях, построенная с
помощью асимптотического и энтропийного подходов. Эта модель предполагает, что
функционирование сложной телекоммуникационной сети можно представить в виде
суперпозиции довольно простых стохастических структур~--- последовательных и
параллельных, которые по\-рож\-да\-ют смеси гамма-распределений для случайной
величины времени обработки и передачи сообщений в сети. Предложена простая
интерпретация параметров данной модели.
\begin{figure*} %fig5
\vspace*{1pt}
\begin{center}
\mbox{%
\epsfxsize=130mm %145.109mm 
\epsfbox{bat-5.eps} }
\end{center}
\vspace*{-13pt} \Caption{Компоненты смеси начального приближения~(\textit{а}) и 
результата~(\textit{б}) для выборки~1 ``Berkeley'' \label{f5bat}}
%\end{figure*}
%\begin{figure*} %fig6
\vspace*{12pt}
\begin{center}
\mbox{%
\epsfxsize=130mm %148.256mm 
\epsfbox{bat-7.eps} }
\end{center}
\vspace*{-13pt} \Caption{График смеси распределений~(\textit{1}) и гистограмма 
для выборки~1 ``Berkeley''~(\textit{2}) \label{f6bat}}
\end{figure*}



\begin{figure*} %fig7
\vspace*{1pt}
\begin{center}
\mbox{%
\epsfxsize=130mm %144.283mm 
\epsfbox{bat-8.eps} }
\end{center}
\vspace*{-16pt} \Caption{Компоненты смеси начального приближения~(\textit{а}) и 
результата~(\textit{б}) для выборки~2 ``Synterra'' \label{f7bat}}
%\end{figure*}
%\begin{figure*} %fig8
\vspace*{12pt}
\begin{center}
\mbox{%
\epsfxsize=130mm %148.256mm 
\epsfbox{bat-10.eps} }
\end{center}
\vspace*{-11pt} \Caption{График смеси распределений~(\textit{1}) и гистограмма
для выборки~2 ``Synterra''~(\textit{2}) \label{f8bat}}
\end{figure*}

\begin{figure*} %fig9
\vspace*{1pt}
\begin{center}
\mbox{%
\epsfxsize=119.041mm
\epsfbox{bat-11.eps} }
\end{center}
\vspace*{-9pt} \Caption{Изменение  смешивающих параметров~(\textit{а}), 
параметров формы~(\textit{б}) и параметров масштаба~(\textit{в}) во времени для 
выборки~1 ``Berkeley'' \label{f9bat}}
\end{figure*}

Для решения вытекающей из модели задачи предложен итерационный алгоритм,
базирующийся на методе максимального правдоподобия~--- ЕМ-ал\-го\-ритм, для
которого получены формулы для конкретного вида смесей~--- экспоненциальных и
гамма-распределений.
%
Кроме того, разработан программный инструментарий для оценки параметров 
предложенной модели на выборках из реальных трафиковых данных. Проведены 
исследования, которые подтвердили предположения вероятностной модели. 


Получение информации о стохастической структуре
телекоммуникационных сетей и наличие программных инструментов для
выявления более или менее стабильных структур позволит понять причины
возникновения неожиданных больших нагрузок, предотвратить такие нагрузки,
а также поможет в будущем в проектировании надежных, оптимальных по
стоимости и уровню сервиса телекоммуникационных сетей нового поколения.

%\vspace*{-15pt} 
{\small\frenchspacing
{%\baselineskip=10.8pt
\addcontentsline{toc}{section}{Литература}
\begin{thebibliography}{9}
\bibitem{1bat}
Teletraffic Engeneering Handbook. International Telecommunication Union, 
Geneva, 2005 {\sf http://www.itu.int}. \vspace*{5pt} 
\bibitem{2bat}
\Au{Севастьянов~Б.\,А.} Курс теории вероятностей и математической статистики. 
М., 2004. \vspace*{5pt} 
\bibitem{3bat}
\Au{Айвазян~C.\,А., Бухштабер~В.\,М., Енюков~И.\,С, Мешалкин~Л.\,Д.} Прикладная 
статистика. Классификация и снижение размерности~// Финансы и статистика. М., 
1989. \vspace*{5pt} 
\bibitem{4bat}
\Au{Bilmes~J.\,A.} A gentle tutorial of the EM algorithm and its application to 
parameter estimation for Gaussian mixture and hidden Markov models. Berkeley, 
CA, USA: International Computer Science Institute,  1998. \vspace*{5pt} 
\bibitem{5bat}
\Au{Шлезингер~М.\,И.} О самопроизвольном различении образов~// Шлезингер~М.\,И. 
Читающие. автоматы. Киев: Наукова думка, 1965. С.~38--45. \vspace*{5pt} 
\bibitem{6bat}
\Au{Hsiao~I.-T., Rangarajan~A., Gindi~G.}. Joint-MAP 
reconstruction/segmentation for transmission tomography using mixture-models as 
priors. Yale University, 1998. \vspace*{5pt} 
\bibitem{7bat}
{\sf http://zedgraph.org}. \vspace*{4pt} 
\bibitem{8bat}
{\sf http://ita.ee.lbl.gov/html/contrib/LBL-PKT.html}. \vspace*{5pt} 
\bibitem{9bat}
{\sf http://www.synterra.ru}.
\end{thebibliography}

} } \label{end\stat}
\end{multicols}


%\addtocounter{razdel}{1}
%\def\razd{НЕРЕГУЛИРУЕМЫЙ ЭЛЕКТРОПРИВОД ДЛЯ ЭЛЕКТРОЭНЕРГЕТИКИ}

\setcounter{page}{2}

%   { %\Large  
   { %\baselineskip=16.6pt
   
   \vspace*{-48pt}
   \begin{center}\LARGE
   \textit{Предисловие}
   \end{center}
   
   %\vspace*{2.5mm}
   
   \vspace*{25mm}
   
   \thispagestyle{empty}
   
   { %\small 

    
Вниманию читателей журнала <<Информатика и её применения>> предлагается 
очередной тематический выпуск <<Вероятностно-статистические методы и 
задачи информатики и информационных технологий>>. Предыдущие тематические 
выпуски журнала по данному направлению вышли в 2008~г.\ (т.~2, вып.~2), 
в 2009~г.\ (т.~3, вып.~3) и в 2010~г.\ (т.~4, вып.~2). 

Статьи, собранные в данном журнале, посвящены разработке новых вероятностно-статистических 
методов, ориентированных на применение к решению конкретных задач информатики и информационных 
технологий, а также~--- в ряде случаев~--- и других прикладных задач. Проблематика, охватываемая 
публикуемыми работами, развивается в рамках научного сотрудничества между Институтом проблем 
информатики Российской академии наук (ИПИ РАН) и Факультетом вычислительной математики и 
кибернетики Московского государственного университета им.\ М.\,В.~Ломоносова в ходе работ 
над совместными научными проектами (в том числе в рамках функционирования 
Научно-образовательного центра <<Вероятностно-статистические методы анализа рисков>>). 
Многие из авторов статей, включенных в данный номер журнала, являются активными участниками 
традиционного международного семинара по проблемам устойчивости стохастических моделей, 
руководимого В.\,М.~Золотаревым и В.\,Ю.~Королевым; регулярные сессии этого семинара 
проводятся под эгидой МГУ и ИПИ РАН (в 2011~г.\ указанный семинар проводится в октябре 
в Калининградской области РФ). 

Наряду с представителями ИПИ РАН и МГУ в число авторов данного выпуска журнала входят 
ученые из Научно-исследовательского института системных исследований РАН, Института 
проблем технологии микроэлектроники и особочистых материалов РАН, Института 
прикладных математических исследований Карельского НЦ РАН, Московского 
авиационного института, Вологодского государственного педагогического университета, 
НИИММ им.\ Н.\,Г.~Чеботарева, Казанского государственного университета, Дебреценского 
университета (Венгрия).

Несколько статей выпуска посвящено разработке и применению стохастических методов и 
информационных технологий для решения различных прикладных задач. В~работе В.\,Г.~Ушакова 
и О.\,В.~Шестакова рассмотрена задача определения вероятностных характеристик случайных 
функций по распределениям интегральных преобразований, возникающих в задачах эмиссионной 
томографии. В~статье Д.\,О.~Яковенко и М.\,А.~Целищева рассмотрены некоторые вопросы 
математической теории риска и предложен новый подход к диверсификации инвестиционных 
портфелей. Работа И.\,А.~Кудрявцевой и А.\,В.~Пантелеева посвящена построению и 
исследованию математической модели, описывающей динамику сильноионизованной плазмы. 
В~статье П.\,П.~Кольцова изучается качество работы ряда алгоритмов сегментации изображений. 
Статья А.\,Н.~Чупрунова и И.~Фазекаша посвящена вероятностному анализу числа без\-оши\-бочных 
блоков при помехоустойчивом кодировании; получены усиленные законы больших чисел для указанных 
величин.

В данном выпуске традиционно присутствует тематика, весьма активно разрабатываемая в течение 
многих лет специалистами ИПИ РАН и МГУ,~--- методы моделирования и управления для 
информационно-телекоммуникационных и вычислительных систем, в частности методы 
теории массового обслуживания. В~статье А.\,И.~Зейфмана с соавторами рассматриваются 
модели обслуживания, описываемые марковскими цепями с непрерывным временем в случае 
наличия катастроф. В~работе М.\,М.~Лери и И.\,А.~Чеплюковой рассматриваются случайные 
графы Интернет-типа, т.\,е.\ графы, степени вершин которых имеют степенные распределения; 
такие задачи находят применение при исследовании глобальных сетей передачи данных. 
Работа Р.\,В.~Разумчика посвящена исследованию систем массового обслуживания специального 
вида~--- с отрицательными заявками и хранением вытесненных заявок.

Ряд статей посвящен развитию перспективных теоретических 
вероятностно-статистических методов, которые находят широкое применение в различных 
задачах информатики и информационных технологий. В~работе В.\,Е.~Бенинга, А.\,К.~Горшенина 
и В.\,Ю.~Королева рассмотрена задача статистической проверки гипотез о числе компонент 
смеси вероятностных распределений, приводится конструкция асимптотически наиболее мощного 
критерия. Результаты этой работы найдут применение в ряде прикладных задач, использующих 
математическую модель смеси вероятностных распределений (в информатике, моделировании 
финансовых рынков, физике турбулентной плазмы и~т.\,д.). В~статье В.\,Ю.~Королева, 
И.\,Г.~Шевцовой и С.\,Я.~Шоргина строится новая, улучшенная оценка точности нормальной 
аппроксимации для пуассоновских случайных сумм; как известно, указанные случайные суммы 
широко используются в качестве моделей многих реальных объектов, в том числе в информатике, 
физике и других прикладных областях. Работа В.\,Г.~Ушакова и Н.\,Г.~Ушакова посвящена 
исследованию ядерной оценки плотности распределения; эти результаты могут применяться, 
в част\-ности, при анализе трафика в телекоммуникационных системах. Серьезные приложения 
в статистике могут получить результаты работы О.\,В.~Шестакова, в которой доказаны оценки 
скорости сходимости распределения выборочного абсолютного медианного отклонения к нормальному 
закону. 

\smallskip

Редакционная коллегия журнала выражает надежду, что данный тематический  выпуск 
будет интересен специалистам в области теории вероятностей и математической статистики 
и их применения к решению задач информатики и информационных технологий.
     
     %\vfill 
     \vspace*{20mm}
     \noindent
     Заместитель главного редактора журнала <<Информатика и её 
применения>>,\\
     директор ИПИ РАН, академик  \hfill
     \textit{И.\,А.~Соколов}\\
     
     \noindent
     Редактор-составитель тематического выпуска,\\
     профессор кафедры математической статистики факультета\\
      вычислительной математики и кибернетики МГУ им.\ М.\,В.~Ломоносова,\\
     ведущий научный сотрудник ИПИ РАН,\\ 
доктор физико-математических наук \hfill
      \textit{В.\,Ю.~Королев}
     
     } }
     }



%   { %\Large  
   { %\baselineskip=16.6pt
   
   \vspace*{-48pt}
   \begin{center}\LARGE
   \textit{Предисловие}
   \end{center}
   
   %\vspace*{2.5mm}
   
   \vspace*{25mm}
   
   \thispagestyle{empty}
   
   { %\small 

    
Вниманию читателей журнала <<Информатика и её применения>> предлагается 
очередной тематический выпуск <<Вероятностно-статистические методы и 
задачи информатики и информационных технологий>>. Предыдущие тематические 
выпуски журнала по данному направлению вышли в 2008~г.\ (т.~2, вып.~2), 
в 2009~г.\ (т.~3, вып.~3) и в 2010~г.\ (т.~4, вып.~2). 

Статьи, собранные в данном журнале, посвящены разработке новых вероятностно-статистических 
методов, ориентированных на применение к решению конкретных задач информатики и информационных 
технологий, а также~--- в ряде случаев~--- и других прикладных задач. Проблематика, охватываемая 
публикуемыми работами, развивается в рамках научного сотрудничества между Институтом проблем 
информатики Российской академии наук (ИПИ РАН) и Факультетом вычислительной математики и 
кибернетики Московского государственного университета им.\ М.\,В.~Ломоносова в ходе работ 
над совместными научными проектами (в том числе в рамках функционирования 
Научно-образовательного центра <<Вероятностно-статистические методы анализа рисков>>). 
Многие из авторов статей, включенных в данный номер журнала, являются активными участниками 
традиционного международного семинара по проблемам устойчивости стохастических моделей, 
руководимого В.\,М.~Золотаревым и В.\,Ю.~Королевым; регулярные сессии этого семинара 
проводятся под эгидой МГУ и ИПИ РАН (в 2011~г.\ указанный семинар проводится в октябре 
в Калининградской области РФ). 

Наряду с представителями ИПИ РАН и МГУ в число авторов данного выпуска журнала входят 
ученые из Научно-исследовательского института системных исследований РАН, Института 
проблем технологии микроэлектроники и особочистых материалов РАН, Института 
прикладных математических исследований Карельского НЦ РАН, Московского 
авиационного института, Вологодского государственного педагогического университета, 
НИИММ им.\ Н.\,Г.~Чеботарева, Казанского государственного университета, Дебреценского 
университета (Венгрия).

Несколько статей выпуска посвящено разработке и применению стохастических методов и 
информационных технологий для решения различных прикладных задач. В~работе В.\,Г.~Ушакова 
и О.\,В.~Шестакова рассмотрена задача определения вероятностных характеристик случайных 
функций по распределениям интегральных преобразований, возникающих в задачах эмиссионной 
томографии. В~статье Д.\,О.~Яковенко и М.\,А.~Целищева рассмотрены некоторые вопросы 
математической теории риска и предложен новый подход к диверсификации инвестиционных 
портфелей. Работа И.\,А.~Кудрявцевой и А.\,В.~Пантелеева посвящена построению и 
исследованию математической модели, описывающей динамику сильноионизованной плазмы. 
В~статье П.\,П.~Кольцова изучается качество работы ряда алгоритмов сегментации изображений. 
Статья А.\,Н.~Чупрунова и И.~Фазекаша посвящена вероятностному анализу числа без\-оши\-бочных 
блоков при помехоустойчивом кодировании; получены усиленные законы больших чисел для указанных 
величин.

В данном выпуске традиционно присутствует тематика, весьма активно разрабатываемая в течение 
многих лет специалистами ИПИ РАН и МГУ,~--- методы моделирования и управления для 
информационно-телекоммуникационных и вычислительных систем, в частности методы 
теории массового обслуживания. В~статье А.\,И.~Зейфмана с соавторами рассматриваются 
модели обслуживания, описываемые марковскими цепями с непрерывным временем в случае 
наличия катастроф. В~работе М.\,М.~Лери и И.\,А.~Чеплюковой рассматриваются случайные 
графы Интернет-типа, т.\,е.\ графы, степени вершин которых имеют степенные распределения; 
такие задачи находят применение при исследовании глобальных сетей передачи данных. 
Работа Р.\,В.~Разумчика посвящена исследованию систем массового обслуживания специального 
вида~--- с отрицательными заявками и хранением вытесненных заявок.

Ряд статей посвящен развитию перспективных теоретических 
вероятностно-статистических методов, которые находят широкое применение в различных 
задачах информатики и информационных технологий. В~работе В.\,Е.~Бенинга, А.\,К.~Горшенина 
и В.\,Ю.~Королева рассмотрена задача статистической проверки гипотез о числе компонент 
смеси вероятностных распределений, приводится конструкция асимптотически наиболее мощного 
критерия. Результаты этой работы найдут применение в ряде прикладных задач, использующих 
математическую модель смеси вероятностных распределений (в информатике, моделировании 
финансовых рынков, физике турбулентной плазмы и~т.\,д.). В~статье В.\,Ю.~Королева, 
И.\,Г.~Шевцовой и С.\,Я.~Шоргина строится новая, улучшенная оценка точности нормальной 
аппроксимации для пуассоновских случайных сумм; как известно, указанные случайные суммы 
широко используются в качестве моделей многих реальных объектов, в том числе в информатике, 
физике и других прикладных областях. Работа В.\,Г.~Ушакова и Н.\,Г.~Ушакова посвящена 
исследованию ядерной оценки плотности распределения; эти результаты могут применяться, 
в част\-ности, при анализе трафика в телекоммуникационных системах. Серьезные приложения 
в статистике могут получить результаты работы О.\,В.~Шестакова, в которой доказаны оценки 
скорости сходимости распределения выборочного абсолютного медианного отклонения к нормальному 
закону. 

\smallskip

Редакционная коллегия журнала выражает надежду, что данный тематический  выпуск 
будет интересен специалистам в области теории вероятностей и математической статистики 
и их применения к решению задач информатики и информационных технологий.
     
     %\vfill 
     \vspace*{20mm}
     \noindent
     Заместитель главного редактора журнала <<Информатика и её 
применения>>,\\
     директор ИПИ РАН, академик  \hfill
     \textit{И.\,А.~Соколов}\\
     
     \noindent
     Редактор-составитель тематического выпуска,\\
     профессор кафедры математической статистики факультета\\
      вычислительной математики и кибернетики МГУ им.\ М.\,В.~Ломоносова,\\
     ведущий научный сотрудник ИПИ РАН,\\ 
доктор физико-математических наук \hfill
      \textit{В.\,Ю.~Королев}
     
     } }
     }

\def\stat{sinits}

\def\tit{АНАЛИТИЧЕСКОЕ МОДЕЛИРОВАНИЕ
НОРМАЛЬНЫХ ПРОЦЕССОВ В~СТОХАСТИЧЕСКИХ СИСТЕМАХ СО~СЛОЖНЫМИ~НЕЛИНЕЙНОСТЯМИ}

\def\titkol{Аналитическое моделирование
нормальных процессов в~стохастических системах со~сложными нелинейностями}

\def\aut{И.\,Н.~Синицын$^1$, В.\,И.~Синицын$^2$}

\def\autkol{И.\,Н.~Синицын, В.\,И.~Синицын}

\titel{\tit}{\aut}{\autkol}{\titkol}

\renewcommand{\thefootnote}{\arabic{footnote}}
\footnotetext[1]{Институт проблем
информатики Российской академии наук, sinitsin@dol.ru}
\footnotetext[2]{Институт проблем
информатики Российской академии наук, vsinitsin@ipiran.ru}


\Abst{Рассматриваются конечномерные дифференциальные стохастические системы
(ДСтС) и эредитарные (интегродифференциальные) стохастические системы  (ЭСтС)
с винеровскими и пуассоновскими шумами, приводимые к ДСтС со сложными конечными,
дифференциальными и интегральными нелинейностями. Такие модели функционирования
описывают поведение многих современных нано- и кван\-то\-во-оп\-ти\-че\-ских
технических средств информатики. Приводятся уравнения методов нормальной
аппроксимации (МНА) и статистической линеаризации (МСЛ) для аналитического
моделирования нестационарных и стационарных нормальных (гауссовских) процессов
в нелинейных ДСтС и  нелинейных ЭСтС путем аппроксимации эредитарных ядер
линейными операторными уравнениями для дифференцируемых нелинейностей и
сингулярными ядрами для недифференцируемых нелинейностей. Рассматриваются
методы вычисления типовых интегралов МНА (МСЛ) для сложных (многомерных и
векторного аргумента) конечных и дифференциальных нелинейностей. Особое
внимание уделяется иррациональным и дробно-рациональным нелинейностям
скалярного аргумента. Приводятся примеры вычисления интегралов. Подробно
рассматриваются вопросы вычисления типовых интегралов МНА (МСЛ) для сложных
интегральных нелинейностей.}

\KW{аналитическое моделирование;
дифференциальные стохастические системы с винеровскими и пуассоновскими шумами (ДСтС);
метод нормальной аппроксимации (МНА);
метод статистической линеаризации (МСЛ);
сложные иррациональные нелинейности;
сложные конечные, дифференциальные и интегральные нелинейности;
эредитарные стохастические системы (ЭСтС), приводимые к дифференциальным}

\DOI{10.14357/19922264140302}

\vspace*{9pt}

\vskip 16pt plus 9pt minus 6pt

\thispagestyle{headings}

\begin{multicols}{2}

\label{st\stat}


\section{Введение}


Моделями функционирования многих современных технических сис\-тем информатики
служат стохастические системы (СтС), описываемые дифференциальными, интегральными
и интегродифференциальными уравнениями со сложными дроб\-но-ра\-ци\-о\-наль\-ны\-ми,
иррациональными и интегральными нелинейностями. В~[1] дано систематическое
изложение МНА и МСЛ для ДСтС и ЭСтС, приводимых к дифференциальным.

Обобщая~[2--7], рассмотрим развитие МНА и МСЛ для аналитического моделирования
нормальных стохастических процессов (СтП) на случай СтС со сложными конечными,
дифференциальными и интегральными нелинейностями.

Как показано в~\cite{4-sin}, альтернативным подходом к аналитическому моделированию
СтП в ДСтС и ЭСтС служит подход, основанный на дискретизации стохастических
дифференциальных уравнений на основе использования обобщенной формы Ито и
кратных стохастических интегралов от винеровских и пуассоновских СтП с
последующим применением дискретных версий МНА (МСЛ).

Статья состоит из введения, пяти разделов и заключения.

В~разд.~2 и~3
приводятся уравнения МНА и МСЛ для аналитического моделирования одно- и
двумерных распределений стационарных и нестационарных СтП в ДСтС и ЭСтС,
приводимых к ДСтС.

Типовые интегралы МНА и МСЛ рассматриваются в разд.~4.

Особенности аналитического моделирования в ДСтС со сложными конечными и
дифференциальными нелинейностями обсуждаются в разд.~5.

Раздел~6
посвящен аналитическому моделированию СтП в ДСтС со сложными интегральными
нелинейностями.

Приводятся примеры.


\section{Уравнения методов нормальной~аппроксимации и~статистической
линеаризации для~дифференциальных стохастических систем}

Как известно~\cite{2-sin, 3-sin},  уравнения конечномерных непрерывных нелинейных сис\-тем
со стохастическими возмущениями путем расширения вектора состояния ДСтС
могут быть записаны в виде следующего векторного стохастического
дифференциального уравнения Ито:
    \begin{multline}
    dY_t = a(Y_t, t)\, dt + b (Y_t, t) \,dW_0+{}\\
    {}+ \iii_{R_0} c (Y_t, t, v) P^0
    (dt, dv)\,,\enskip Y(t_0) = Y_0\,.\label{e2.1-sin}
    \end{multline}
Здесь $a=a(Y_t, t)$ и $b\hm=b(y_t, t)$~--- известные
$(p\times 1)$-мер\-ная и  $(p\times m)$-мер\-ная функции~$Y_t$ и~$t$;
$W_0\hm= W_0(t)$~--- $r$-мер\-ный винеровский СтП интенсивности
$\nu_0 \hm= \nu_0(t)$; $c(Y_t, t, v)$~--- $(p\times 1)$-мер\-ная функция  $Y_t, t$
и вспомогательного $(q\times 1)$-мер\-но\-го па\-ра\-мет\-ра~$v$;
$\iii_{\Delta} dP^0 (t, A)$~--- центрированная пуассоновская мера,
определяемая
\begin{equation*}
\iii_{\Delta} dP^0 (t, A) = \iii_{\Delta} dP (t,A) =
\iii_{\Delta} \nu_P (t,A)\, dt\,. %\label{e2.2-sin}
\end{equation*}
В~(\ref{e2.1-sin}) принято: $\iii_{\Delta}$~-- число скачков пуассоновского
СтП в интервале времени  $\Delta \hm= (t_1, t_2]$; $\nu_P (t, A)$~---
интенсивность пуассоновского СтП  $P(t,A)$; $A$~--- некоторое борелевское
множество пространства  $R_0^q$ с выколотым началом.
Начальное значение~$Y_0$ представляет собой случайную величину, не зависящую
от приращений СтП  $W_0(t)$ и $P(t,A)$ на интервалах времени, следующих
за~$t_0$, $t_0 \hm\le t_1\hm\le t_2$ для любого множества~$A$.

В случае аддитивных нормальных (гауссовских) и обобщенных
пуассоновских возмущений уравнение~(\ref{e2.1-sin}) имеет вид:
\begin{equation}
\dot Y_t = a(Y_t,t)+ b_0 (t) V\,, \enskip
V = \dot W\,,\enskip Y(t_0) = Y_0\,.\label{e2.3-sin}
\end{equation}
Здесь $W$~--- СтП с независимыми приращениями, представляющий собой
смесь нормального и обобщенного пуассоновского СтП.

Если предположить существование конечных вероятностных
моментов второго порядка для моментов времени~$t_1$ и~$t_2$, то уравнения
МНА примут следующий вид~\cite{2-sin, 3-sin}:
\begin{itemize}
\item  для характеристических функций
    \begin{equation}
    g_1^N (\la;t) =\exp \lk i\la^{\mathrm{T}} m_t - \fr{1}{2}\, \la^{\mathrm{T}} K_t \la\rk\,;\label{e2.4-sin}
    \end{equation}
\begin{equation}
\hspace*{-7.5mm}g_{t_1, t_2}^N (\la_1, \la_2;t_1, t_2 ) =\exp \lk i\bar \la^{\mathrm{T}} \bar m_2 -
\fr{1}{2}\, \bar \la^{\mathrm{T}} \bar K_2 \la\rk\,,\!\!\label{e2.5-sin}
\end{equation}
где
    \begin{gather*}
    \bar \la =\lk \la_1^{\mathrm{T}}\la_2^{\mathrm{T}}\rk^{\mathrm{T}}\,; \quad
        \bar m_2 = \lk m_{t_1}^{\mathrm{T}} m_{t_2}^{\mathrm{T}}\rk^{\mathrm{T}}\,;\\
        \bar K_2= \begin{bmatrix}
    K(t_1, t_1)& K(t_1, t_2)\\
    K(t_2, t_1)& K(t_2, t_2)
    \end{bmatrix}\,;
    \end{gather*}

\item для математических ожиданий  $m_t$, ковариационной матрицы~$K_t$ и
матрицы ковариационных функций $K(t_1, t_2)$:
    \begin{equation}
    \dot m_t = a_1 (m_t, K_t, t)\,,\enskip m_0 = m(t_0)\,;\label{e2.6-sin}
    \end{equation}
\begin{equation}
\dot K_t = a_2 (m_t, K_t, t)\,,\enskip K_0 = K(t_0)\,;\label{e2.7-sin}
\end{equation}

\vspace*{-12pt}

\noindent
\begin{multline}
\fr{\prt K(t_1, t_2)}{\prt t_2 }= K(t_1, t_2) a_{21} (m_{t_2}, K_{t_2}, t_2)^{\mathrm{T}}\,;\\
K(t_1, t_1) = K_{t_1}\,.
\label{e2.8-sin}
\end{multline}
    \end{itemize}
Здесь приняты следующие обозначения:
\begin{equation}
a_1 = a_1 (m_t, K_t, t) = M_N a (Y_t, t)\,;\label{e2.9-sin}
\end{equation}

\vspace*{-12pt}

\noindent
\begin{multline}
a_2 = a_2 (m_t, K_t, t) = a_{21} (m_t, K_t, t)+{}\\
{}+ a_{21} (m_t, K_t, t)^{\mathrm{T}} +
a_{22}(m_t, K_t, t)\,;\label{e2.10-sin}
\end{multline}

\vspace*{-12pt}

\noindent

\begin{equation}
a_{21} = a_{21}(m_t, K_t, t)=  M_N a(Y_t, t) Y_{t}^{0\mathrm{T}}\,;\label{e2.11-sin}
\end{equation}
\begin{equation*}
a_{22} = a_{22}(m_t, K_t, t)= M_N \sigma (Y_t, t)\,;
%\label{e2.12-sin}
\end{equation*}

\vspace*{-12pt}

\noindent
\begin{multline*}
\sigma (Y_t, t) = b(Y_t, t) \nu_0(t) b(Y_t, t)^{\mathrm{T}} +{}\\
{}+
\iii_{R_0^q} c (Y_t, t, v) c(Y_t, t,v)^{\mathrm{T}}
\nu_P (t, dv)\,; %\label{e2.13-sin}
\end{multline*}

\vspace*{-12pt}

\begin{gather*}
m_t = MY_t\,,\quad Y_t^0 = Y_t - m_t\,,\\
K_t = M_N Y_0^0 Y_t^{0\mathrm{T}}\,,\quad K(t_1, t_2) =
M_N Y_{t_1}^0 Y_{t_2}^0\,; %\label{e2.14-sin}
\end{gather*}
$M_N$~--- символ вычисления математического ожидания для нормальных
распределений~(\ref{e2.4-sin}) и~(\ref{e2.5-sin}).

Для стационарных ДСтС нормальные стационарные СтП~--- если они существуют,
то  $m_t \hm=\bar m$, $ K_t \hm=\bar K$, $K(t_1, t_2) \hm= k(\tau)$
$(\tau \hm= t_1\hm-t_2)$,~--- определяются уравнениями~\cite{2-sin, 3-sin}:
   \begin{equation}
   a_1 (\bar m, \bar K) =0\,;\enskip a_2 (\bar m, \bar K)=0\,;\label{e2.15-sin}
   \end{equation}
   \begin{equation}
   \left.
   \hspace*{-2.8mm}\begin{array}{l}
  \dot k_\tau (\tau) = a_{21} (\bar m, \bar K)\bar K^{-1} k(\tau)\,;\\[9pt]
  k(0) =\bar K \enskip (\forall \tau >0)\,, \
  k(\tau) = k(-\tau)^{\mathrm{T}} \enskip
  (\forall\tau <0)\,.
  \end{array}\!\!
  \right\}\!\!
  \label{e2.16-sin}
  \end{equation}
При этом необходимо, чтобы матрица  $a_{21} (\bar m, \bar K)\hm=\bar a_{21}$
была бы асимптотически устойчивой.

Для ДСтС~(\ref{e2.3-sin}) уравнения МНА переходят в уравнения МСЛ
Казакова~\cite{2-sin, 3-sin}, если принять
\begin{equation}
a(Y_t,t) = a_1 (m_t, K_t) + k_1^a (m_t, K_t) Y_t^0\,;\label{e2.17-sin}
\end{equation}
\begin{equation}\left.
\begin{array}{rl}
b(Y_t,t) &= b_0 (t)\,;\\[9pt]
    \si(Y_t, t)&= b_0(t) \nu(t) b_0(t)^{\mathrm{T}} = \si_0(t)\,,
    \end{array}
    \right\}\label{e2.18-sin}
    \end{equation}
    \begin{equation}
k_1^a (m_t, K_t, t) =\lk \left(\fr{\prt}{\prt m_t} \right)
    a_0 (m_t, K_t, t)^{\mathrm{T}}\rk^{\mathrm{T}}\,;\label{e2.19-sin}
    \end{equation}
    \begin{equation}
\dot m_t = a_1 (m_t, K_t, t) \,,\enskip m_0 = m(t_0)\,,\label{e2.20-sin}
\end{equation}

\vspace*{-12pt}

\noindent
\begin{multline}
\dot K_t = k_1^a (m_t, K_t, t) K_t + K_t k_1^a (m_t, K_t, t)^{\mathrm{T}}
    +\si_0(t)\,;\\
    K_0 = K(t_0)\,;
    \label{e2.21-sin}
    \end{multline}

    \vspace*{-12pt}

    \noindent
\begin{multline}
\fr{\prt K(t_1, t_2)}{\prt t_2} =
    K(t_1, t_2) k_{t_2} k_1^a (m_{t_2}, K_{t_2}, t_2)^{\mathrm{T}}\,;\\
    K(t_1, t_2) = K_{t_1}\,.
    \label{e2.22-sin}
\end{multline}

Для стационарных ДСтС~(\ref{e2.3-sin})
при условии асимптотической устойчивости матрицы $k_1^a (\bar m, \bar K)$
в основе МСЛ лежат уравнения~(\ref{e2.15-sin}), записанные в виде:
    \begin{gather}
    a_1 (\bar m, \bar K) =0\,; \label{e2.23-sin}\\
k_1^a (\bar m, \bar K) \bar K + \bar K k_1^a
(\bar m, \bar K)^{\mathrm{T}} +\bar \si_0 =0\,;\label{e2.24-sin}
\end{gather}

\vspace*{-12pt}

\noindent
\begin{multline}
k_\tau (\tau) = k_1^a (\bar m, \bar K)k(\tau)\,,\enskip
k(0) =\bar K \enskip (\forall \tau >0)\,,\\
k(\tau) = k (-\tau)^{\mathrm{T}} \enskip (\forall \tau <0)\,.
\label{e2.25-sin}
\end{multline}

Уравнения~(\ref{e2.4-sin})--(\ref{e2.8-sin})
лежат в основе МНА для ДСтС~(\ref{e2.1-sin}), а уравнения~(\ref{e2.17-sin})--(\ref{e2.22-sin})~---
в основе МСЛ для ДСтС~(\ref{e2.3-sin}). Для определения стационарных СтП
согласно МНА служат соотношения~(\ref{e2.15-sin}) и~(\ref{e2.16-sin}),
а МСЛ~--- (\ref{e2.17-sin})--(\ref{e2.25-sin}).

\section{Уравнения методов нормальной~аппроксимации и~статистической линеаризации
для~эредитарных стохастических систем, приводимых к~дифференциальным}

Рассмотрим ЭСтС, описываемую интегродифференциальным уравнением Ито
следующего вида~\cite{7-sin}:

\noindent
\begin{multline}
dX_t = \lk a(X_t, t) +\iii_{t_0}^t a_1 (X(\tau) ,\tau, t)\,d\tau\rk dt+{}\\
{}+\lk b(X_t, t) +\iii_{t_0}^t b_1 (X(\tau) ,\tau, t)\,d\tau\rk dW_0+{}\\
\hspace*{-1.5mm}{}+\!\!\iii_{R_0^q}\!\!\lk c(X_t, t,v) +\!\iii_{t_0}^t\! c_1 (X(\tau) ,\tau, t,v)\,d\tau\!\rk\! dP^0 (t, dv)
\!\!\!\!\label{e3.1-sin}
\end{multline}
с начальным условием  $X(t_0) = X_0$. В~(\ref{e3.1-sin})
сохранены обозначения разд.~2.

Функции $a=a(X_t, t)$, $a_1\hm = a_1(X (\tau),\tau, t)$,
$b\hm=b(X_t, t)$, $b_1\hm = b_1(X (\tau),\tau, t)$,
$c\hm=c(X_t,t,v)$ и $c_1\hm = c_1(X (\tau),\tau, t,v)$ имеют
соответственно размерности $p\times 1$, $p\times 1$, $p\times r$,
$p\times r$, $p\times 1$ и $p\times 1$ и допускают представления следующего вида:
\begin{equation}
\left.
\begin{array}{rl}
a_1&=A(t,\tau) \vrp (X(\tau) , \tau)\,;\\[9pt]
b_1&=B(t,\tau) \psi (X(\tau) ,  \tau)\,;\\[9pt]
c_1&=C(t,\tau) \chi (X(\tau) ,  \tau, v)\,.
\end{array}
\right\}
\label{e3.2-sin}
\end{equation}
Здесь эредитарные ядра $A\hm=A(t,\tau)\hm=\lk A_{ij}(t,\tau)\rk$
$(i,j\hm=\overline{1,p})$,
$B\hm=B(t,\tau)=\lk B_{i l}(t,\tau)\rk$ $(i\hm=\overline{1,p}$;
$l\hm=\overline{1,r})$ и $C\hm=C(t,\tau)=\lk C_{ij}(t,\tau)\rk$
$(i,j\hm=\overline{1,p})$ имеют соответственно размерности
$p\times p$, $p\times r$ и $p\times p$. Они удовлетворяют следующим условиям
физической реализуемости и асимптотического затухания:
\begin{multline}
A_{ij}(t,\tau)=0;\enskip B_{i l}(t,\tau)=0;\\[1pt]
C_{ij}(t,\tau)=0\enskip \forall \tau >t;\label{e3.3-sin}
\end{multline}

\vspace*{-12pt}

\begin{equation}
\left.
\hspace*{-3mm}\begin{array}{c}
\displaystyle\iin\! \lv A_{ij} (t,\tau) \rv d\tau <\infty\,;\
\displaystyle\iin\! \lv B_{i l} (t,\tau) \rv d\tau <\infty \,;\\[9pt]
\displaystyle\iin \!\lv C_{ij} (t,\tau) \rv d\tau <\infty\,.
\end{array}\!
\right\}\!
\label{e3.4-sin}
\end{equation}

В дальнейшем ограничимся случаем, когда эредитарные ядра удовлетворяют
линейным операторным уравнениям~\cite{6-sin, 5-sin, 7-sin}.

Нелинейные в общем случае функции $\vrp\hm=\vrp(X(\tau),\tau)$,
$\psi \hm=\psi(X(\tau), \tau)$, $\chi \hm=\chi (X(\tau),  \tau, v)$
отражают нелинейные свойства ЭСтС, зависят от  $X(\tau)$ и имеют размерности
$p\times 1$, $p\times p$, $p\times 1$ соответственно.

Важный класс  эредитарных ядер представляют собой
сингулярные (вырожденные) ядра, когда имеют место представления:
\begin{equation}
\left.
\hspace*{-3mm}\begin{array}{rl}
A_{ij} (t,\tau) &= A_{ij}^+(t) A_{ij}^-(\tau)\,;\\[9pt]
B_{i l} (t,\tau)& = B_{il}^+(t) B_{il}^-(\tau)\,;\\[9pt]
C_{ij} (t,\tau) &= C_{ij}^+ ( t) C_{ij}^- (\tau)\
(i,l= \overline{1,p}, j=\overline{1,r}).
\end{array}\!
\right\}\!\!
\label{e3.5-sin}
\end{equation}

В~\cite{6-sin, 5-sin, 7-sin} показано, что для дифференцируемых нелинейных
функций~$\vrp$, $\psi$, $\chi$ путем расширения вектора состояния за счет
инструментальных переменных, аппроксимируемых линейными операторными уравнениями,
определяющими эредитарные ядра в ЭСтС, (\ref{e3.1-sin})--(\ref{e3.4-sin})
приводятся к ДСтС вида~(\ref{e2.1-sin}) или~(\ref{e2.3-sin}).
В~случае недифференцируемых нелинейных функций~$\vrp$, $\psi$, $\chi$
ЭСтС~(\ref{e3.1-sin})--(\ref{e3.4-sin}) приводятся к~(\ref{e2.1-sin}) или~(\ref{e2.3-sin})
на основе аппроксимации вырожденными (сингулярными) ядрами~\cite{6-sin, 5-sin, 7-sin}.

Таким образом, после приведения ЭСтС~(\ref{e3.1-sin}) к ДСтС~(\ref{e2.1-sin})
или~(\ref{e2.3-sin}) можно воспользоваться уравнениями МНА и МСЛ разд.~2.

\section{Типовые интегралы методов нормальной аппроксимации и~статистической
линеаризации}

Как следует из уравнений~(\ref{e2.9-sin})--(\ref{e2.11-sin}),
для МНА необходимо уметь вычислять следующие интегралы:
\begin{multline}
I_0^a = I_0^a (m_t, K_t, t) = a_1 (m_t, K_t, t)={}\\
{}= M_N a(Y_t, t)\,;
\label{e4.1-sin}
\end{multline}

\vspace*{-12pt}

\noindent
\begin{multline}
I_1^a = I_1^a (m_t, K_t, t)= a_{21}(m_t, K_t, t)= {}\\
{}=M_N a(Y_t , t) Y_t^{0\mathrm{T}}\,;\label{e4.2-sin}
\end{multline}

\vspace*{-12pt}

\noindent
\begin{multline}
I_0^{\bar \si} = I_0^{\bar \si} (m_t, K_t, t) = a_{22}(m_t, K_t, t) ={}\\
{}= M_N \bar \si (Y_t, t)\,.\label{e4.3-sin}
\end{multline}
Для МСЛ достаточно вычислить интеграл~(\ref{e4.1-sin}),
причем интеграл~$I_1^a$ вычисляется по формуле~\cite{2-sin, 3-sin, 4-sin}:
\begin{equation*}
k_1^a = k_1^a (m_t, K_t, t)=\lk \left( \fr{\prt}{\prt m_t}\right)
I_0^a (m_t, K_t, t)^{\mathrm{T}}\rk^{\mathrm{T}}. %\label{e4.4-sin}
\end{equation*}

\medskip

\noindent
\textbf{Пример 1.} В~[1] для типовых степенных, тригоно\-мет\-ри\-че\-ских,
показательных и ку\-соч\-но-по\-сто\-ян\-ных нелинейностей $Z_t \hm=\vrp (Y_t, t)$
скалярного и векторного аргумента приведены формулы для интегралов
$I_0^\vrp \hm= I_0^\vrp (m_t^y, K_t^y, t)$, а также
$k_1^\vrp \hm= k_1^\vrp (m_t^y, K_t^y, t)$.

\medskip

\noindent
\textbf{Замечание.}
 Важно иметь в виду, что уравнения МНА (МСЛ) содержат интегралы
 $I_0^a$, $I_1^a$, $I_0^\si$ в виде соответствующих коэффициентов.
 Поэтому процедура вычисления интегралов должна быть согласована с
 методом численного решения обыкновенных дифференциальных уравнений для
 $m_t$, $K_t$ и $K(t_1, t_2)$. Эти коэффициенты допускают дифференцирование
 по~$m_t$ и~$K_t$, так как под интегралом стоит сглаживающая нормальная плотность.

\section{Сложные конечные и~дифференциальные нелинейности}

Важный класс сложных конечных нелинейностей (многомерных и векторного аргумента)
представляют собой сложные функции вида:
    \begin{equation*}
    \xi =\vrp (X_t, Y_t, t)\,,\enskip X_t =\psi (Y_t, t)\,. %\label{e5.1-sin}
    \end{equation*}
В~этом случае вычисление интегралов (см.\ разд.~4) проводится по совокупности
переменных  $\lk X_t^{\mathrm{T}} Y_t^{\mathrm{T}}\rk^{\mathrm{T}}$.
К таким нелинейностям, например, относятся дроб\-но-ра\-ци\-о\-наль\-ные,
иррациональные  нелинейности, выражаемые специальными функциями, многозначные
нелинейности, зависящие от СтП~$X_t$ и его производных~$\dot X_t$,  $\ddot X_t$
и~др.

\medskip

\noindent
\textbf{Пример 2.}
Рассмотрим вычисление интегралов~(\ref{e4.1-sin}) и~(\ref{e4.2-sin})
для сложных одномерных иррациональных нелинейностей скалярного аргумента
\begin{equation}
\vrp (Y_t, t) =\lv Y_t\rrv^{\alpha-1}\, \mathrm{sgn}\, Y_t
\label{e5.2-sin}
\end{equation}
($\alpha$~--- нецелый показатель).

Пользуясь~(\ref{e2.16-sin}) и~(\ref{e2.19-sin}), представим~(\ref{e5.2-sin}) в виде
\begin{equation*}
\vrp(Y_t, t) = \vrp_0 (m_t, D_t, t) + k_1^\vrp(m_t, D_t, t) Y_t^0. %\label{e5.3-sin}
\end{equation*}
Здесь введены следующие обозначения:
\begin{gather*}
\vrp_0(m_t, D_t, t) =\Gamma(\alpha) D_t^{1/2} e^{-\xi^2/4} D_{-\alpha} (\xi)\,;%\label{e5.4-sin}
\\
k_1^a (m_t, D_t, t) =\fr {\prt \vrp_0(m_t, D_t, t)}{\prt m_t}\,,%\label{e5.5-sin}
\end{gather*}
где  $\Gamma(\alpha)$~--- гамма-функция,  $\xi \hm= m_t/\sqrt{D_t}$~---
отношение <<сиг\-нал--шум>>; $D_{-\alpha} (\xi)$~---
функция параболического цилиндра~\cite{9-sin}.
При вычислении были учтены следующие соотношения~\cite{9-sin, 8-sin}:
\begin{multline}
\iii_0^\infty x^{\alpha-1} e^{-\beta x^2 - \gamma x} \,dx ={}\\
{}=
(2\beta)^{-\alpha/2} \Gamma(\alpha) \exp \left(\fr{\gamma^2}{8\beta}\right)
D_{-\alpha} \left(\fr{\gamma}{\sqrt{2\beta}}\right)\,;\label{e5.6-sin}
\end{multline}

\vspace*{-12pt}

\noindent
\begin{multline}
\fr{dD_\rho(\xi)}{d\xi} =
   -\fr{\xi}{2}\, D_\rho (\xi) -\rho D_{\rho-1} (\xi) =
   \fr{\xi}{2}\, D_\rho (\xi) -{}\\
   {}- D_{\rho+1} (\xi) \enskip
   (\mathrm{Re}\, \beta>0\,,\enskip \mathrm{Re}\,\alpha>0\,,\enskip
   \rho=-\alpha)\,.\label{e5.7-sin}
   \end{multline}

Соотношения~(\ref{e5.6-sin}) и~(\ref{e5.7-sin})
могут быть использованы также для вычисления интегралов~(\ref{e4.3-sin}).

\medskip

\noindent
\textbf{Замечание.}
Для вычисления интегралов $I_0^a$, $I_1^a$ и $I_0^{\bar \si}$
применительно к типовым иррациональным нелинейностям вида
    $\lv Y_t\rrv^{\alp-1} e^{\delta Y_t}$, $\lv Y_t\rrv^{\alp-1}  \cos \w Y_t$,
    $\lv Y_t\rrv^{\alp-1}  \sin \w Y_t$
и более общим нелинейностям \mbox{вида}
    \begin{equation*}
    \vrp (Y_t, t) =\Phi^\vrp \left( \lv Y_t\rrv^{\alpha-1}, t\right) %\label{e5.8-sin}
    \end{equation*}
можно рекомендовать известные численные методы вычисления функций на ЭВМ~\cite{8-sin}.

\smallskip

\noindent
\textbf{Пример 3.}
Для нелинейной дроб\-но-ра\-ци\-о\-наль\-ной функции

\noindent
\begin{equation*}
\vrp (Y_t, t) = \fr{a}{(b+Y_t)^2} %\label{e5.9-sin}
\end{equation*}
имеем

\vspace*{-3pt}

\noindent
\begin{gather*}
\vrp_0 (m_t, D_t, t) =a b^{-2} \lk 1+ \chi (m_t, D_t, t)\rk\,; %\label{e5.10-sin}
\\
k_1^\vrp (m_t, D_t, t) =  a b^{-2}\fr{\prt \chi (m_t, D_t, t)}{\prt m_t}\,. %\label{e5.11-sin}
\end{gather*}
Здесь

\vspace*{-3pt}

\noindent
\begin{multline*}
\chi (m_t, D_t, t) ={}\\
{}=\sss_{n=1}^\infty \sss_{l=0}^{E(n/2)}
\fr{(-1)^n (n+1) n!}{(n-2l)! (2l)!}\, b^{-n} m_t^n \left( \fr{D_t}{ 2 m_t^2}
\right)^l, %\label{e5.12-sin}
\end{multline*}
где  $E(n/2)$~--- целая часть~$n/2$; $a\hm=a(t)$; $b\hm= b(t)$.

\vspace*{-6pt}

\section{Сложные интегральные нелинейности}

\vspace*{-2pt}

Пусть сначала векторно-матричная нелинейность имеет эредитарный характер, т.\,е.\
\begin{equation}
\underline{\vrp} (Y_t, t) =\iii_{t_0}^t A(t,\tau) \vrp (Y(\tau), \tau) \,d\tau\,.
\label{e6.1-sin}
\end{equation}
Тогда, как показано в~\cite{6-sin, 5-sin, 7-sin}, следует соответст\-ву\-ющие
интегродифференциальные соотношения путем введения  инструментальных
переменных привести к дифференциальным соотношениям.  Для
дифференцируемых функций~$\vrp$ и асимптотически устойчивых ядер
$A(t,\tau)$ зависимость~(\ref{e3.5-sin}) имеет следующий дифференциальный вид:
\begin{equation*}
F^A (t, D) \underline{\vrp} (Y_t, t) = H^A (t, D) \vrp (Y_t, t)\,. %\label{e6.2-sin}
\end{equation*}
Здесь $F^A (t, D)$ и  $H^A (t, D)$~--- линейные дифференциальные операторы $(D\hm= d/dt)$.

Для недифференцируемых функций~$\vrp$ и асимптотически устойчивых
сингулярных ядер~(\ref{e3.5-sin}) используются соотношения:
\begin{equation*}
\underline{\vrp} (Y_t, t) = A^+ Z\,,\enskip
\dot Z = A^- \vrp\,,\enskip
Z(t_0)=0\,. %\label{e6.3-sin}
\end{equation*}

Многочисленные примеры аналитического моделирования ЭСтС можно найти
в~[1--3, 5, 7, 10, 11].

Как отмечалось в~\cite{3-sin}, часто наряду с интегральными
нелинейностями~(\ref{e6.1-sin}) рассматривают нелинейности вида:

\columnbreak

\noindent
\begin{equation*}
Z_s =\sss_{\rho=1}^R \mathcal{A}_\rho \vrp_\rho (Y_{t_1}\tr Y_{t_r})\,, %\label{e6.2-sin}
\end{equation*}
где $\mathcal{A}_1 \tr \mathcal{A}_R$~--- произвольные линейные операторы,
действующие над функциями~$r$ переменных  $t_1\tr t_r$; $\vrp_\rho
\hm=\vrp_\rho (Y_{t_1} \tr Y_{t_r})$~--- линейные функции отмеченных
переменных. Такие нелинейности называются приводимыми к линейным.
Важным частным случаем~(\ref{e6.1-sin}) являются интегральные нелинейности вида:

\noindent
\begin{gather}
Z_s =\iii_T \vrp (Y_t, t, s)\, dt\,; \notag%\label{e6.3-sin}
\\
Z_s =\!\iii_T \!\cdots\!\iii_T\! \vrp (Y_{t_1}\tr Y_{t_r}; t_1\tr t_r, s)\,dt_1
\ldots dt_r,\notag %\label{e6.4-sin}
\end{gather}
В этом случае используется МСЛ по совокупности переменных  $Y_{t_1} \tr Y_{t_r}$.

\vspace*{-9pt}

\section{Заключение}

\vspace*{-2pt}

Разработаны методы и алгоритмы МНА и МСЛ для ДСтС и ЭСтС,
приводимых к ДСтС со сложными конечными, дроб\-но-ра\-ци\-о\-наль\-ны\-ми,
иррациональными, а также дифференциальными и интегральными нелинейностями.
Приведены примеры.

Результаты допускают обобщение на случай ДСтС и ЭСтС со
стохастическими нелинейностями, заданными каноническими разложениями и
интегральными каноническими  представлениями~\cite{1-sin, 3-sin, 11-sin}.

\vspace*{-9pt}

{\small\frenchspacing
 {%\baselineskip=10.8pt
 \addcontentsline{toc}{section}{References}
 \begin{thebibliography}{99}

 \vspace*{-2pt}

\bibitem{1-sin}
\Au{Синицын И.\,Н.,  Синицын~В.\,И.}
Лекции по нормальной и эллипсоидальной аппроксимации распределений в
стохастических сис\-те\-мах.~--- М.: ТОРУС ПРЕСС, 2013. 488~с.

\bibitem{6-sin} %2
\Au{Синицын И.\,Н. }
Stochastic hereditary control systems~// Проблемы управления и
теории информации, 1986. Т.~15. №\,4. С.~287--298.

\bibitem{2-sin} %3
\Au{Пугачев В.\,С., Синицын~И.\,Н.}
Стохастические дифференциальные сис\-те\-мы. Анализ и фильтрация.~--- М.:
Наука,  1990.  632~с. [Англ. пер.
 Stochastic differential systems.
Analysis and filtering.~--- Chichester, New York: Jonh Wiley, 1987.
549~p.].

\bibitem{5-sin} %4
\Au{Синицын И.\,Н. }
Конечномерные распределения процессов в стохастических интегральных
и интегродифференциальных системах~// Preprints of the 2nd IFAC
Symposium on Stochastic Control.~--- Vilnius: Pergamon Press,
1987.  Vol.~1. P.~144--153.

\bibitem{3-sin} %5
\Au{Пугачев В.\,С., Синицын~И.\,Н.}
Теория стохастических систем.~--- М.: Логос, 2000; 2004. 1000~с.
[Англ. пер.\linebreak\vspace*{-12pt}

\pagebreak

\noindent Stochastic systems. Theory and  applications.~---
Singapore: World Scientific, 2001. 908~p.].

\bibitem{4-sin} %6
\Au{Синицын И.\,Н.}
Параметрическое статистическое и аналитическое моделирование распределений
в нелинейных стохастических сис\-те\-мах на многообразиях~//
Информатика и её применения, 2013. Т.~7. Вып.~2. С.~4--16.

\bibitem{7-sin} %7
\Au{Синицын И.\,Н. }
Анализ и моделирование распределений в эредитарных стохастических
сис\-те\-мах~// Информатика и её применения, 2014. Т.~8. Вып.~1.\linebreak
С.~2--11.



\bibitem{9-sin} %8
\Au{Градштейн И.\,С., Рыжик~И.\,М.}
Таблицы интегралов, сумм, рядов и произведений.~--- М.: ГИФМЛ, 1963. 1100~с.

\bibitem{8-sin} %9
\Au{Попов Б.\,А., Теслер~Г.\,С. }
Вычисление функций на ЭВМ: Справочник.~--- Киев: Наукова Думка, 1984. 599~с.


\bibitem{11-sin} %10
\Au{Синицын И.\,Н.}
Канонические представления случайных функций и их применение в
задачах компьютерной поддержки научных исследований.~--- М.: ТОРУС
ПРЕСС, 2009. 768~с.

\bibitem{10-sin} %11
\Au{Синицын И.\,Н., Синицын~В.\,И., Корепанов~Э.\,Р., Белоусов~В.\,В.,
Сергеев~И.\,В., Басилашвили~Д.\,А.}
Опыт моделирования эредитарных стохастических сис\-тем~//
Кибернетика и высокие технологии XXI века: Сб. докл.  XIII Междунар.
науч.-технич. конф.~--- Воронеж: Саквоее, 2012. Т.~2. C.~346--357.

 \end{thebibliography}

 }
 }

\end{multicols}

\vspace*{-9pt}

\hfill{\small\textit{Поступила в редакцию 05.05.14}}

%\newpage

\vspace*{12pt}

\hrule

\vspace*{2pt}

\hrule

\vspace*{12pt}

\def\tit{ANALYTICAL MODELING OF NORMAL PROCESSES
 IN~STOCHASTIC SYSTEMS WITH~COMPLEX NONLINEARITIES}

\def\titkol{Analytical modeling of normal processes
 in~stochastic systems with~complex nonlinearities}

\def\aut{I.\,N.~Sinitsyn and V.\,I.~Sinitsyn}

\def\autkol{I.\,N.~Sinitsyn and V.\,I.~Sinitsyn}

\titel{\tit}{\aut}{\autkol}{\titkol}

\vspace*{-9pt}

\noindent
Institute of Informatics Problems, Russian Academy of Sciences,
44-2 Vavilov Str., Moscow 119333, Russian Federation


\def\leftfootline{\small{\textbf{\thepage}
\hfill INFORMATIKA I EE PRIMENENIYA~--- INFORMATICS AND
APPLICATIONS\ \ \ 2014\ \ \ volume~8\ \ \ issue\ 3}
}%
 \def\rightfootline{\small{INFORMATIKA I EE PRIMENENIYA~---
INFORMATICS AND APPLICATIONS\ \ \ 2014\ \ \ volume~8\ \ \ issue\ 3
\hfill \textbf{\thepage}}}

\vspace*{6pt}

\Abste{Differential stochastic systems (DStS) with Wiener and Poisson
noises and complex finite, differential, and  integral nonlinearities and
hereditary StS reducible to DStS are considered. Equations and algorithms
of analytical modeling based on the normal approximation method (NAM) and the
statistical linearization method (SLM) are given. The case of complex
continuous and discontinuous nonlinearities of scalar and vector arguments
is considered. Special attention is paid to NAM (SLM) typical integrals
for finite rational and irrational nonlinear and integral scalar and vector
nonlinear functions. The general case of integral nonlinearities reducible to
linear is considered. Test examples are given.}

\KWE{analytical modeling;
complex finite differential and integral nonlinearities;
complex irrational nonlinerarites
differential stochastic system with Wiener and Poisson noises;
method of normal approximation;
method of statistical linearization;
hereditary stochastic systems reducible to differential}

\DOI{10.14357/19922264140302}

  \begin{multicols}{2}

\renewcommand{\bibname}{\protect\rmfamily References}
%\renewcommand{\bibname}{\large\protect\rm References}

{\small\frenchspacing
 {%\baselineskip=10.8pt
 \addcontentsline{toc}{section}{References}
 \begin{thebibliography}{99}



\bibitem{1-sin-1}
\Aue{Sinitsyn, I.\,N., and  V.\,I.~Sinitsyn}.  2013.
Lektsii po normal'noy i ellipsoidal'noy approksimatsii raspredeleniy
v stokhasticheskikh sistemakh [Lectures on normal and ellipsoidal
approximation of distributions in stochastic systems].
Moscow: TORUS PRESS. 488~p.

\bibitem{6-sin-1} %2
\Aue{Sinitsyn, I.\,N.}  1986.
{Stochastic hereditary control systems}.
\textit{Problems Control Inform. Theory} 15(4):287--298.

\bibitem{2-sin-1} %3
\Aue{Pugachev, V.\,S., and  I.\,N.~Sinitsyn}.  1987.
\textit{Stochastic differential systems. Analysis and filtering.}
Chichester, New York: Jonh Wiley. 549~p.

\bibitem{5-sin-1} %4
\Aue{Sinitsyn, I.\,N.}  1987.
Konechnomernye raspredeleniya protsessov v stokhasticheskikh integral'nykh
i in\-teg\-ro\-dif\-fe\-ren\-tsial'nykh sistemakh [Finite dimensional distributions
of processes in stochastic integral and integrodifferential systems].
\textit{2nd  Symposium (International) IFAC on Stochastic Control
Preprints}. Vilnius: Pergamon Press. 1:144--153.

\bibitem{3-sin-1} %5
\Aue{Pugachev, V.\,S., and I.\,N.~Sinitsyn}. 2001.
\textit{Stochastic systems. Theory and  applications}.
Singapore: World Scientific. 908~p.

\bibitem{4-sin-1} %6
\Aue{Sinitsyn, I.\,N.}  2013.
Parametricheskoe statisticheskoe i analiticheskoe modelirovanie
raspredeleniy v nelineynykh stokhasticheskikh sistemakh na mnogoobraziyakh
[Parametric statistical and analytical modeling of distributions in
stochastic systems on manifolds].
\textit{Informatika i ee Primeneniya}~--- \textit{Inform. Appl.} 7(2):4--16.


\bibitem{7-sin-1} %7
\Aue{Sinitsyn, I.\,N.}  2014.
Analiz i modelirovanie raspredeleniy v ereditarnykh stokhasticheskikh sistemakh
[Analysis and modeling of distributions in hereditary stochastic systems].
\textit{Informatika i ee Primeneniya}~--- \textit{Inform. Appl.} 8(1):2--11.

\bibitem{9-sin-1} %8
\Aue{Gradshteyn, I.\,S., and I.\,M.~Ryzhik}.  1963.
\textit{Tablitsy integralov, summ, ryadov i proizvedeniy}
[Tables of integrals, sums, series, and products]. Moscow:  GIFML.   1100~p.

\pagebreak

\bibitem{8-sin-1} %9
\Aue{Popov, B.\,A., and G.\,S.~Tesler}.  1984.
\textit{Vychislenie funktsiy na EVM}. Spravochnik [Computing of functions].
Kiev: Naukova Dumka.  599~p.


\bibitem{11-sin-1} %10
\Au{Sinitsyn, I.\,N.} 2009.
\textit{Kanonicheskie predstavleniya sluchaynykh funktsiy i ikh primenenie v
zadachakh komp'yuternoy podderzhki nauchnykh issledovaniy}
[Canonical expansions of random functions and its application to
scientific computer-aided support]. Moscow: TORUS PRESS. 768~p.

\bibitem{10-sin-1} %11
\Aue{Sinitsyn, I.\,N., V.\,I.~Sinitsyn, E.\,R.~Korepanov,
V.\,V.~Belousov, I.\,V.~Sergeev, and D.\,A.~Basilashvili}.
2012. Opyt modelirovaniya ereditarnykh stokhasticheskikh sistem
[Experience of modeling in hereditary stochastic systems].
\textit{Kibernetika i Vysokie Tekhnologii XXI~Veka:
Sbornik dokladov  XIII Mezhdunar. nauch.-tekhnich. konf.}
[Cybernatics ans High Technologies of the XXI Century: Materials of
XIII  Scientific and Technological Conference (International)].
Voronezh: Sakvoee. 2:346--357.

\end{thebibliography}

 }
 }

\end{multicols}

\vspace*{-6pt}

\hfill{\small\textit{Received May 05, 2014}}

\vspace*{-18pt}

\Contr

\noindent
\textbf{Sinitsyn Igor N.} (b.\ 1940)~---
Doctor of Science in technology, professor, Honored scientist of RF, Head of Department, Institute of
Informatics Problems, Russian Academy of Sciences,
44-2 Vavilov Str., Moscow 119333, Russian
Federation; sinitsin@dol.ru

\vspace*{3pt}

\noindent
\textbf{Sinitsyn Vladimir I.} (b.\ 1968)~--- Doctor of Science in physics
and mathematics, associate professor, Head of Department, Institute of
Information Problems, Russian Academy of Sciences,
44-2 Vavilov Str., Moscow 119333, Russian Federation; VSinitsin@ipiran.ru




\label{end\stat}

\renewcommand{\bibname}{\protect\rm Литература} 


\def\stat{ushmaev}


\def\tit{АДАПТАЦИЯ БИОМЕТРИЧЕСКОЙ СИСТЕМЫ К ИСКАЖАЮЩИМ 
ФАКТОРАМ НА ПРИМЕРЕ ДАКТИЛОСКОПИЧЕСКОЙ 
ИДЕНТИФИКАЦИИ$^*$}
\def\titkol{Адаптация биометрической системы к искажающим 
факторам на примере дактилоскопической 
идентификации} 

\def\autkol{О.\,С.~Ушмаев}
\def\aut{О.\,С.~Ушмаев$^1$}

\titel{\tit}{\aut}{\autkol}{\titkol}

{\renewcommand{\thefootnote}{\fnsymbol{footnote}}\footnotetext[1]
{Работа поддержана грантами РФФИ (проект 07-07-00031) и Программой ОНИТ РАН 
<<Информационные технологии и методы анализа сложных систем>>. Работа выполнена в рамках НОЦ 
ИПИ РАН~--- ВМК МГУ <<Биометрическая информатика>>.}}

\renewcommand{\thefootnote}{\arabic{footnote}}
\footnotetext[1]{Институт проблем
информатики Российской академии наук, oushmaev@ipiran.ru}

%\vspace*{12pt}

\Abst{Рассмотрены проблемы учета искажающих факторов при биометрической 
идентификации. На примере дактилоскопической идентификации проанализировано влияние 
искажений на качество идентификации. Предложены методы учета искажений на основе 
нормальной аппроксимации факторов. Проведенные эксперименты показали эффективность 
предложенного подхода к учету искажений. На основе разработанных методов предложена 
технология адаптации биометрической системы к ис\-ка\-жа\-ющим факторам.}

%\vspace*{4pt}

\KW{биометрическая идентификация; автоматическая дактилоскопическая идентификация; 
искажающие факторы}

 \vskip 36pt plus 9pt minus 6pt

      \thispagestyle{headings}

      \begin{multicols}{2}

      \label{st\stat}
      

\section{Введение}

     На сегодняшний день технологии биометрической идентификации получают 
широкое распространение в различных системах гражданской идентификации. Это и 
паспортно-ви\-зо\-вые документы нового поколения, и идентификация получателей 
социальных услуг, и идентификационные карты работников отдельных категорий 
(государственные служащие, транспортная безопасность) \mbox{и~пр.~[1--5]}. 
{ %\looseness=1

}
     
     Длительное накопление биометрических массивов приводит к ситуации, когда 
одновременно обрабатывается информация, собранная с использованием разных 
технических средств, в разное время и при разных условиях. Постепенное отклонение 
характеристик биометрической информации от эталонных значений приводит к 
отклонению показателей качества биометрической идентификации от проектных 
значений. 
     
     В качестве примеров постепенной деградации биометрической идентификации 
можно привести следующие:
     \begin{itemize}
\item возрастные изменения лицевой биометрии ведут к тому, что качество 
идентификации при увеличении временного лага между регистрацией в системе и 
идентификацией падает;
\item сезонные колебания качества дактилоскопической идентификации: зима и лето 
характеризуются принципиально различными условиями получения отпечатков 
пальцев;
\item изменение в качестве идентификации отпечатков пальцев, полученных из разных 
источников: бумажных дактокарт, <<живым>> сканированием, следов отпечатков 
и~т.\,д. 
\end{itemize}

     В~[6--9] приведены примеры деградации биометрической идентификации для 
широкого круга технологий и систем.
     
     Производители биометрических технологий\linebreak постоянно совершенствуют 
алгоритмы распознавания с целью повышения устойчивости к иска\-жа\-ющим факторам. 
Однако многие проблемы устранить невозможно. 
     
     В этой связи актуальной проблемой проектирования и эксплуатации 
биометрических технологий является учет влияния искажающих факторов. Проблема 
особенно актуальна в случаях, когда искажения невозможно подавить штатными 
средствами биометрической системы. 
    
     Далее статья организована следующим образом. Раздел~2 посвящен оценке 
влияния искажающих\linebreak
факторов при дактилоскопической идентификации. В разд.~3 
изложен подход к адаптации био\-мет\-ри\-ческой системы к искажениям. В разд.~4 
\begin{figure*}  %fig1
\vspace*{1pt}
\begin{center}
\vspace*{6pt}
\mbox{%
\epsfxsize=165.353mm
\epsfbox{ush-1.eps}
}
\end{center}
\vspace*{-9pt}
\Caption{Процесс создания шаблона
\label{f1ush}}
\vspace*{6pt}
\end{figure*}
приведены результаты экспериментов. В разд.~5\linebreak рассмотрены отдельные вопросы 
практического применения изложенного подхода. Основные выводы представлены в 
заключении.
%\pagebreak

    
\section{Влияние искажающих факторов}

     В своей методологической основе большинство биометрических технологий 
являются технологиями распознавания образов. При регистрации человека 
биометрическая информация оцифровывается в количественные данные, пригодные 
для дальнейшей идентификации (рис.~\ref{f1ush})~\cite{7ush}.
  
     На этапе идентификации шаблоны сравниваются. Далее преимущественно 
используются пороговые методы принятия решения. Если результат сравнения 
биометрических шаблонов выше некоторого порогового значения, то принимается 
гипотеза о принадлежности образцов одному человеку. В обратном случае считается, 
что образцы принадлежат разным людям. Изменяя порог, можно управлять 
соотношением ошибок 1-го и 2-го рода, FRR (False Rejection Rate)
и FAR (False Accptance Rate) (рис.~2).

     Соответственно, с точки зрения ошибок принятия решения об идентификации 
биометрическая\linebreak\vspace*{-12pt}
\begin{center} %fig2
\vspace*{12pt}
\mbox{%
\epsfxsize=80mm %.951mm
\epsfbox{ush-2.eps}
}
\end{center}
{{\figurename~2}\ \ \small{Пороговое принятие решения: \textit{1}~--- FRR, \textit{2}~--- FAR}}

\bigskip
\addtocounter{figure}{1}
    
\noindent
система характеризуется двумя распределениями: результаты 
сравнений в <<своих>> сравнениях и результаты сравнения в <<чужих>> сравнениях. 

 
     Под воздействием внешних систематических факторов распределения могут 
претерпевать определенные изменения. На рис.~3 приведен пример 
изменения распределений под влиянием эластичных деформаций отпечатков 
пальцев~\cite{9ush}, достаточно распространенного и значимого искажающего фактора 
в дактилоскопической идентификации. 
   
     Как видно из рис.~3, искажающий фактор вносит смещения и изменения 
дисперсии, при этом общий вид распределений не изменяется. Аналогичная ситуация 
наблюдается для более широкого класса биометрик и искажающих факторов. Среди 
публично доступных данных по проблеме следует выделить протоколы тестирований 
(NIST FRVT, NIST PVT, FpVTE\footnote{NIST~--- National Institute of Standards and Technology;
FRVT~--- Face Recognition Vendor Test;
PVT~--- PHIGS Validation Tests;
PHIGS~--- Programmer's Hierarchical Interactive Graphics System;
FpVTE~--- Fingerprint Vendor Technology Evaluation.})~\cite{10ush, 11ush}.

     
     В то же время незначительное на первый взгляд различие приводит к 
существенным изменениям в качестве распознавания (рис.~4) и в 
зависимостях FAR и FRR от порога (рис.~\ref{f5ush}). Такие изменения необходимо 
учитывать при проектировании и эксплуатации биометрической системы.

\begin{figure*} %fig3-4
\vspace*{1pt}
%\begin{figure*} %fig3
%\vspace*{1pt}
\begin{center}
\vspace*{1pt}
\mbox{%
\epsfxsize=166.178mm %807mm
\epsfbox{ush-3-4.eps}
}
\end{center}
\vspace*{-9pt}
\begin{minipage}[t]{81mm}
\Caption{Изменение распределений: \textit{1}~--- без компенсации деформаций, \textit{2}~--- с 
компенсацией деформаций
\label{f3ush}}
\end{minipage}
\hfill
\begin{minipage}[t]{81mm}
%\begin{figure} %fig4
\Caption{Изменение качества распознавания: \textit{1}~---  с учетом деформаций, \textit{2}~--- 
без учета деформаций
\label{f4ush}}
\end{minipage}
\end{figure*}

\begin{figure*} %fig5
\vspace*{1pt}
\begin{center}
\vspace*{1pt}
\mbox{%
\epsfxsize=165.791mm
\epsfbox{ush-5.eps}
}
\end{center}
\vspace*{-9pt}
\Caption{Изменение ошибок распознавания в зависимости от порога: \textit{1}~--- без 
компенсации деформаций, \textit{2}~--- с компенсацией деформаций (исходная шкала)
\label{f5ush}}
%\end{figure*}
% \begin{figure*} %fig6
     \vspace*{18pt}
\begin{center}
\vspace*{1pt}
\mbox{%
\epsfxsize=151.739mm
\epsfbox{ush-6.eps}
}
\end{center}
\vspace*{-9pt}
     \Caption{Процесс биометрической идентификации
\label{f6ush}}
\end{figure*}

     
     Необходимость учета изменения ошибок связана с типичной логикой принятия 
решения о био\-мет\-ри\-ческой идентификации. Согласно bioAPI (Biometric Application Programming Interface), основному 
био\-мет\-ри\-че\-скому стандарту на API био\-мет\-ри\-ческих библиотек, при идентификации 
предъявляемый образец последовательно сравнивается с хранимыми образцами. 
Полученный результат~--- мера сходства био\-мет\-ри\-ческих образцов~--- приводится к 
единой шкале по таблице ошибок 1-го и~2-го рода (рис.~\ref{f6ush}). Ответ в таком 
случае может быть сформулирован как точные значения ошибок идентификации.
     
     Под влиянием сильных внешних искажений однородность биометрической базы 
нарушается, что приводит к невозможности адекватно оценить ожидаемые ошибки 
\begin{table*}[b]\small %tabl1
\begin{center}
\Caption{Статистики влияния деформаций отпечатков пальцев
\label{t1ush}}
\vspace*{2ex}

\begin{tabular}{|l|c|c|c|c|c|c|c|c|}
\hline
\multicolumn{1}{|c|}{Фактор}&$\mu$&$\sigma$&$\gamma_3$&$\gamma_4$&$\gamma_5$&$\gamma_6$&$\gamma_7$
&$\gamma_8$\\
\hline
C компенсацией деформации (свои)&280&143&0,41&2,73&3,33&14,41&35,71&147,52\\
Без компенсации деформации  
(свои)&265&156&0,84&3,12&4,87&18,36&49,45&208,34\\
C компенсацией деформации 
(чужие)&\hphantom{9}48&\hphantom{9}24&1,07&4,72&15,80\hphantom{9}&77,22&430,55\hphantom{9}&2869,27\hphantom{9}\\
Без компенсации деформации 
(чужие)&\hphantom{9}46&\hphantom{9}24&1,14&4,96&16,35\hphantom{9}&83,01&470,90\hphantom{9}&3027,34\hphantom{9}\\
\hline
\end{tabular}
\end{center}
\end{table*}
распознавания. Как видно из рис.~\ref{f5ush}, различия в ожидаемых уровнях ошибок 
могут быть очень значительными.
     
     
\vspace*{6pt}
\section{Оценка искажающих факторов}
\vspace*{3pt}

     Для количественной оценки влияния искажающего фактора изучим моменты 
распределений результатов сравнения: математическое ожидание, дисперсию, моменты 
больших порядков. Чтобы изолировать влияние математического ожидания и 
дисперсии, заменим моменты больших порядков на производные статистики
     \begin{equation}
     \gamma_k = \mathbf{E}\left [ \left ( \fr{s-\mu}{\sigma}\right )^k\right ]\,,
     \label{e1ush}
     \end{equation}
     где $\mu$~--- математическое ожидание, $\sigma$~--- сред\-не\-квад\-ра\-ти\-ческое 
отклонение.

     Как видно из~(\ref{e1ush}), статистики $\gamma_k$ инварианты относительно 
линейной замены аргументов.
     
     Данные по изменению статистик искажающего фактора приведены в 
табл.~\ref{t1ush}.


   
     Анализ данных табл.~\ref{t1ush} показывает, что при грубой оценке искажающего 
фактора можно ограничиться моментами первых двух порядков. Слабая вариация 
статистик~(\ref{e1ush}) высших порядков подтверждает предположение о том, что 
характер распределения не претерпевает принципиальных изменений. 
     
     Обозначим первые моменты распределений в своих и чужих сравнениях через 
($\mu_g,\sigma_g^2$) и ($\mu_i,\sigma_i^2$). Смещения моментов, сопряженные с 
воздействием искажающего фактора, обозначим через $(\Delta m_g, \Delta s_g^2$) и 
($\Delta m_i, \Delta s_i^2$) соответственно.
     
     Такая оценка искажений на уровне смещений моментов первых двух порядков 
имеет явное практическое преимущество. Пусть имеется эталонная таблица 
соответствий ошибок 1-го рода $\mathrm{FRR}(x)$\linebreak и~2-го рода $\mathrm{FAR} (x)$ в 
зависимости от порога~$x$ принятия решения. Тогда скорректированные с учетом 
искажающего фактора ошибки распознавания определяются следующим образом:

\noindent
     \begin{align*}
     \mathrm{FRR}^c (x) & = \mathrm{FRR}\left ( \alpha_{\mathrm{FRR}} x 
+\beta_{\mathrm{FRR}}\right )\,;\\
     \mathrm{FAR}^c(x) & = \mathrm{FAR}\left ( \alpha_{\mathrm{FAR}} 
x+\beta_{\mathrm{FAR}}\right )\,,
     \end{align*}
где коэффициенты $\alpha$ и $\beta$ являются коэффициентами линейного 
преобразования 
$$
\fr{x-\Delta m -\mu}{\sqrt{\sigma^2+\Delta s^2}} \rightarrow \fr{x-\mu}{\sigma}\,,
$$
которые прямо вычисляются через моменты по следующим формулам:
\begin{align*}
\alpha_{\mathrm{FAR}} & = \fr{\sigma_i}{\sqrt{\sigma_i^2+\Delta s_i^2}}\,;\\
\beta_{\mathrm{FAR}} & = \mu_i -\fr{\sigma_i}{\sqrt{\sigma_i^2+\Delta s_i^2}}\left ( 
\mu_i+\Delta m_i\right )\,;\\
\alpha_{\mathrm{FRR}} & = \fr{\sigma_g}{\sqrt{\sigma_g^2+\Delta s_g^2}}\,;\\
\beta_{\mathrm{FRR}}& = \mu_g -\fr{\sigma_i}{\sqrt{\sigma_g^2+\Delta s_g^2}}\left ( 
\mu_g+\Delta m_g\right )\,.
\end{align*}



     В условиях воздействия нескольких факторов с параметрами ($\Delta m_{jg}, 
\Delta s_{jg}^2$), ($\Delta m_{ji}, \Delta s_{ji}^2$) итоговое воздействие ($\Delta m_g, 
\Delta s_g^2$) и ($\Delta m_i, \Delta s_i^2$) получается суммой отдельных факторов:
     \begin{align*}
     \Delta m_g & = \sum\limits_j \Delta m_{jg}\,;\quad \Delta s_g^2 = \sum\limits_j \Delta 
s_{jg}^2\,;\\
     \Delta m_i & = \sum\limits_j \Delta m_{ji}\,;\quad \Delta s_i^2 = \sum\limits_j \Delta 
s_{ji}^2\,.
     \end{align*}
     
     Сдвиги ошибок $\mathrm{FRR}(x)$ и $\mathrm{FAR}(x)$ вычисляются по всей 
совокупности факторов. Формально можно ожидать, что некоторые факторы уменьшат 
дис\-пер\-сию, в таком случае величина $\Delta s^2$ окажется отрицательной. Примером 
подобного фактора является, например, уменьшение окна сканирования. Уменьшение 
дисперсии в таком случае обусловлено общим снижением информативности 
получаемых биометрических образцов.

\begin{figure*} %fig7
\vspace*{1pt}
\begin{center}
\vspace*{1pt}
\mbox{%
\epsfxsize=161.708mm
\epsfbox{ush-7.eps}
}
\end{center}
\vspace*{-9pt}
\Caption{Гистограммы распределений (слева~--- распределения в <<своих>> 
сравнениях, справа~--- распределения в <<чужих>> сравнениях): оптический~(\textit{1}) и емкостной~(\textit{2})
сканеры; с компенсацией~(\textit{3}) и без компенсации~(\textit{4}) деформаций;
контрольные~(\textit{5}) и откатанные~(\textit{6}) отпечатки
\label{f7ush}}
\end{figure*}

\begin{table*}\small %tabl2
\begin{center}
\Caption{Численные оценки моментов
\label{t2ush}}
\vspace*{2ex}

\begin{tabular}{|c|c|c|c|c|c|c|c|c|}
\hline
&&&&&&&&\\[-8pt]
\multicolumn{1}{|c|}{Эксперименты}&$\mu_g$ &$\sigma_g^2$&$\mu_i$
&$\sigma_i^2$&$\Delta m_g$&$\Delta s_g^2$&$\Delta m_i$&$\Delta s_i^2$\\
\hline
I&265&24\,366&46&580&$-16$&\hphantom{$-$}4\,039&$-2$&\hphantom{$-$}309\\
II&265&24\,366&46&580&\hphantom{$-$}15&$-3\,887$&\hphantom{$-$}2&\hphantom{$-9$}21\\
IIIa&934&168\,434\hphantom{9}&40&3362\hphantom{9}&$-126$\hphantom{9}&$-56\,192$\hphantom{9}&$-14$\hphantom{9}&$-139$\\
IIIb&902&134\,658\hphantom{9}&32&2765\hphantom{9}&$-110$\hphantom{9}&$-47\,744$\hphantom{9}&$-11$\hphantom{9}&$-560$\\
\hline
\end{tabular}
\end{center}
\end{table*}
     
\section{Эксперименты}

     В данном разделе рассмотрим факторы, влияющие на качество распознавания 
отпечатков пальцев. Выделим следующие:
     \begin{itemize}
\item шумы;
\item внешние условия (температура, влажность);
\item временной лаг между регистрацией и идентификацией;
\item деформации отпечатков пальцев;
\item характеристики типичного пользователя системы; 
\item различия в способе получения отпечатков пальцев (например, след отпечатка, 
<<живое>> сканирование и оцифровка бумажных носителей).
\end{itemize}

     Можно предположить, что перечисленные факторы по своему воздействию 
независимы. Поэтому можно проводить их оценку по отдельности.
     
     Базовой технологией распознавания пальцев выберем технологию 
AMIS (Automated Multibiometric Information System)~\cite{12ush}, в качестве основного источника биометрических данных~--- 
оптический сканер. Приведем результаты следующих экспериментов, которые могут 
быть повторены на публично доступных биометрических базах:
     \begin{enumerate}[I.]
     \item Изменение характера шумов: переход от оп\-ти\-че\-ского к емкостному.
     \item Учет деформаций: сравнение качества распознавания AMIS с качеством 
распознавания AMIS, усиленной методами устранения деформаций отпечатков 
пальцев~\cite{12ush, 13ush}.
     \item Различия в способе получения отпечатков: использование для 
идентификации откатанных отпечатков.
     \end{enumerate}
     
     Эксперимент~III проведен раздельно для левого и правого больших пальцев (IIIa 
и~IIIb). 
     
     В качестве тестовых баз использовались: FVC2002 (эксперименты~I 
и~II)~\cite{14ush} и база SD~29 (эксперименты~III)~\cite{15ush}. На рис.~\ref{f7ush} 
приведены гис\-то\-грам\-мы распределений с учетом влияния искажающих факторов.


 {\small \begin{center}
{{\tablename~3}\ \ \small{Поправочные коэффициенты}}

\vspace*{2ex}

\tabcolsep=5.5pt
\begin{tabular}{|c|c|c|c|c|}
\hline
Эксперименты&$\alpha_{\mathrm{FRR}}$ &$\beta_{\mathrm{FRR}}$ &$\alpha_{\mathrm{FAR}}$ &$\beta_{\mathrm{FAR}}$ \\
\hline
I&0,926&$\hphantom{-}34{,}402$&0,807&$10{,}460$\\
II&1,091&$-40{,}454$&0,982&$-1{,}154\hphantom{,}$\\
IIIa&1,225&$-55{,}803$&1,021&$13{,}445$\\
IIIb&1,244&$-83{,}817$&1,030&$10{,}365$\\
\hline
\end{tabular}
\end{center}
}

\bigskip
\addtocounter{table}{1}
    
     Численные результаты оценки моментов искажающих факторов и поправочных 
коэффициентов приведены в табл.~\ref{t2ush} и~3. На 
     рис.~\ref{f8ush}--\ref{f10ush} приведены графики нормальной аппроксимации 
искажающих факторов. 



В частности, из таблиц видно, что влияние ис\-кажающих факторов на 
распределения в чужих сравнениях значительно меньше влияния на распределения в 
своих сравнениях. Поэтому больший интерес представляет оценка ошибки 1-го рода с 
учетом искажающего фактора. На рис.~\ref{f12ush}--\ref{f14ush} приведены примеры 
точности оценки ошибок 1-го рода на основе нормальной аппроксимации ис\-ка\-жа\-юще\-го 
фактора. 

     Как видно из рис.~\ref{f12ush}--\ref{f14ush}, предложенный метод позволяет 
производить достаточно точное приведение шкалы ошибок к эталонному варианту. 
     
     \section{Влияние на качество идентификации}

     Основным практическим применением разработанного подхода к учету 
искажающих факторов является адаптация процедуры идентификации в 
биометрической системе при неоднородных исходных данных. 
     
     В качестве примера рассмотрим адаптацию дактилоскопической системы при 
использовании двух источников биометрических данных: откатанных и контрольных 
отпечатков. Предположим, что на этапе идентификации используются только 
контрольные отпечатки, а база данных собрана с использованием двух типов 
сканирования в равных пропорциях. В качестве данных возьмем эксперименты~IIIa 
и~IIIb. 
\pagebreak

\end{multicols}
     
%\linebreak\vspace*{-12pt}
%\pagebreak


\begin{figure*} %fig8
\vspace*{1pt}
\begin{minipage}[t]{83mm}
\begin{center}
\vspace*{1pt}
\mbox{%
\epsfxsize=82mm %.878mm
\epsfbox{ush-8.eps}
}
\end{center}
\vspace*{-9pt}
\Caption{Нормальная аппроксимация факторов~I (изменение типа сканирования): 
\textit{1}~--- оптический сканер (свои), 
\textit{2}~--- емкостной  сканер (свои), 
\textit{3}~--- оптический сканер (чужие),  
\textit{4}~--- емкостной сканер (чужие)
\label{f8ush}}
%\end{figure*}
\end{minipage}
\hfill
\begin{minipage}[t]{81mm}
%\begin{figure*} %fig9
%\vspace*{1pt}
\begin{center}
\vspace*{1pt}
\mbox{%
\epsfxsize=79.76mm
\epsfbox{ush-9.eps}
}
\end{center}
\vspace*{-9pt}
\Caption{Нормальная аппроксимация факторов~II (учет деформаций): \textit{1}~--- с 
компенсацией деформаций, \textit{2}~--- без компенсации деформаций, \textit{3}~--- чужие 
сравнения
\label{f9ush}}
\end{minipage}
%\vspace*{15pt}
\end{figure*}


\begin{figure*} %fig10
\vspace*{1pt}
\begin{center}
\vspace*{1pt}
\mbox{%
\epsfxsize=166.3mm %792mm
\epsfbox{ush-10.eps}
}
\end{center}
\vspace*{-9pt}
\Caption{Нормальная аппроксимация факторов IIIa (переход от сравнения контрольных 
отпечатков к сравнению контрольных с откатанными, левый большой палец)~(\textit{a})
и IIIb (переход от сравнения контрольных 
отпечатков к сравнению контрольных с откатанными, правый большой палец)~(\textit{б}): \textit{1}~--- 
контрольные отпечатки (свои), \textit{2}~--- 
откатанные отпечатки (свои),  \textit{3}~--- контрольные отпечатки (чужие), 
\textit{4}~--- откатанные отпечатки (чужие) 
\label{f10ush}}
\end{figure*}

\begin{multicols}{2}

     Возможны два основных сценария эксплуатации системы с неоднородностями 
такого сорта.

%\noindent
 Первый заключается в наличии двух функций сравнения: <<контрольные 
с контрольными>> и <<контрольные с откатанными>>. Соответственно, учет\linebreak 
различного характера биометрической информации будет осуществлен силами 
производителя\linebreak алгоритмов распознавания. Но для этого необходима нестандартная 
реализация программного интерфейса биометрических библиотек. К тому же такой 
подход требует постоянной модернизации библиотек при каждом новом искажающем 
факторе. 

Второй сценарий заключается в работе с одной функцией сравнения 
(рис.~\ref{f6ush}), но с учетом предложенного подхода к учету искажений.

     
     Рассмотрим два варианта принятия решения: 
     \begin{enumerate}[(1)]
     \item смешивание двух баз без учета сдвигов в ошибках распознавания;
\item приведение к единой шкале по ошибке 1-го рода или по ошибке 2-го рода.
\end{enumerate}

     Основным критерием качества разработанной методики приведения к единой 
шкале является точность оценки ошибок 1-го и 2-го рода, которая продемонстрирована 
в~разд.~3. 
     
 
%\end{multicols}

\begin{figure*} %fig12  %fig11n
\vspace*{1pt}
\begin{minipage}[t]{81.5mm}
\begin{center}
\vspace*{1pt}
\mbox{%
\epsfxsize=80.497mm
\epsfbox{ush-12.eps}
}
\end{center}
\vspace*{-9pt}
\Caption{Пример приведения шкалы ошибки 1-го рода в эксперименте~I: \textit{1}~--- 
оптический сканер, \textit{2}~--- емкостной сканер (исходная шкала), \textit{3}~--- емкостной 
сканер (приведенная шкала)
\label{f12ush}}
%\end{figure*}
\end{minipage}
\hfill
\begin{minipage}[t]{82mm}
%\begin{figure*} %fig13 %fig12n
%\vspace*{1pt}
\begin{center}
\vspace*{1pt}
\mbox{%
\epsfxsize=81.284mm
\epsfbox{ush-13.eps}
}
\end{center}
\vspace*{-9pt}
\Caption{Пример приведения шкалы ошибки 1-го рода в эксперименте~II: \textit{1}~--- без 
компенсации деформаций,   
\textit{2}~--- с компенсацией деформаций (исходная шкала), 
\textit{3}~--- с компенсацией деформаций (приведенная шкала)
\label{f13ush}}
\end{minipage}
\vspace*{9pt}
%\end{figure*}
\vspace*{2pt}
%\begin{figure*} %fig14 %fig13n
%\vspace*{1pt}
\begin{center}
%\vspace*{1pt}
\mbox{%
\epsfxsize=166.2mm % .912mm
\epsfbox{ush-14.eps}
}
\end{center}
\vspace*{-9pt}
\Caption{Пример приведения шкалы ошибки 1-го рода в экспериментах IIIa~(\textit{а}) и
IIIb~(\textit{б}): \textit{1}~--- 
контрольные, \textit{2}~--- откатанные (исходная шкала), \textit{3}~--- откатанные 
(приведенная шкала) отпечатки
 \label{f14ush}}
%\end{figure*}
\vspace*{2pt}
%\begin{figure*} %fig16 %fig14n
%\vspace*{1pt}
\begin{center}
%\vspace*{1pt}
\mbox{%
\epsfxsize=166.2mm %.638mm
\epsfbox{ush-16.eps}
}
\end{center}
\vspace*{-9pt}
     \Caption{Изменение качества идентификации в экспериментах IIIa~(\textit{а}) и IIIb~(\textit{б}): 
     \textit{1}~--- исходная шкала, 
\textit{2}~--- скорректированная шкала, \textit{3}~--- контрольные отпечатки, \textit{4}~--- откатанные отпечатки
\label{f16ush}}
\end{figure*}

    Дополнительным эффектом от учета искажений является улучшение качества 
идентификации за счет адекватного устранения расслоения базы по неоднородности 
искажающих воздействий.
     
     На рис.~\ref{f16ush} приведены сравнительные графики ошибок 
     1-го и 2-го рода (для простого смешивания результатов сравнения и 
корректировки шкал по ошибке 2-го рода). Оценки идентификации на смешанном 
массиве зависят от доли контрольных и откатанных отпечатков в общем потоке 
обращений к биометрической системе. В данном модельном примере были взяты 
значения 50\% для каждого класса.
     
     Как видно из рисунков, корректировка шкалы позволяет прийти к оптимальному 
решению (примерно среднее арифметическое по ошибкам). В~то же время 
использование ненормированной шкалы ухудшает качество распознавания ниже 
уровня худшего из класса.
     
     
\section{Заключение}
     
     В статье изложены подходы к адаптации биометрических систем к воздействию 
искажающих факторов. Предложен подход к декомпозиции и учету искажающих 
факторов на основе метода нормальной аппроксимации. Преимуществами подхода 
являются:
     \begin{itemize}
     \item учет каждого фактора в отдельности;
\item аддитивность характеристик искажающего фактора.
\end{itemize}

     Разработанные на основе предложенного подхода методы и технологии 
адаптации к ис\-ка\-жа\-ющим факторам позволяют в значительной степени 
компенсировать влияние искажений в задаче дактилоскопической идентификации.
     
     Направлением дальнейших исследований является анализ искажающих факторов 
для других биометрических методов, таких как изображение лица, изображение 
радужной оболочки глаза, рукописный почерк и~т.\,д. 

     
{\small\frenchspacing
{%\baselineskip=10.8pt
\addcontentsline{toc}{section}{Литература}
\begin{thebibliography}{99}    
     
\bibitem{1ush}
\Au{Ушмаев О.\,С.}
Применение биометрии в аэропортах~// Biometrics TTS 2007. 22~ноября 2007~г. {\sf 
http:// www.dancom.ru/rus/AIA/Archive/RUVII\_BioLinkSolu tions\_BiometricsInAirports.pdf}.

\bibitem{2ush}
\Au{Ушмаев О.\,С.}
Реализация концепции многофакторной биометрической идентификации в пра\-во\-ох\-ра\-ни\-тель\-ных 
системах~// Интерполитех-2007. {\sf 
http:// www.dancom.ru/rus/AIA/Archive/RUVI\_BioLinkSolu
tions\_MultimodalBiometricsConcept.pdf}.

\bibitem{3ush}
\Au{Синицын~И.\,Н., Губин~А.\,В., Ушмаев~О.\,С.}
Метрологические и биометрические технологии и системы~// История науки и техники, 2008. №~7. 
С.~41--44.

\bibitem{4ush}
\Au{Ушмаев~О.\,С.}
Сервисно-ориентированный подход к разработке мультибиометрических технологий~// 
Информатика и её применения, 2008. Т.~2. Вып.~3. С.~41--53.

\bibitem{5ush}
\Au{Ушмаев О.\,С.}
Концепция мультибиометрической идентификации в информационно-аналитических системах~// 
Паспортные и правоохранительные сис\-темы-2008. Интерполитех-2008. {\sf 
http://www.dancom. ru/rus/AIA/Archive/RUXIX-IPIRAN-Ushmaev-Multi modalBiometricsFramework.ppt}.

\bibitem{7ush} %6
\Au{Bolle~R.\,M., Connell~J.\,H., Pankanti~S., Ratha~N.\,K.,  Senior~A.\,W.}
Guide to biometrics.~--- New-York: Springer-Verlag, 2003.

\bibitem{9ush} %7
\Au{Novikov~S.\,O., Ushmaev~O.\,S.}
Efficiency of elastic deformation registration for fingerprint identification~// 7th Conference  (International) 
on Pattern Recognition and Image Analysis: New Information Technologies (PRIA-7-2004) Proceedings. St.\ Petersburg, 
October 18--23, 2004. Vol.~III.~--- St.\ Petersburg: \mbox{SPbETU}, 2004. 
P.~833--836.

\bibitem{8ush} %8
Wayman~J., Jain~A., Maltoni~D., Maio~D., eds.
Biometric systems: Technology, design and performance evaluation.~--- London: Springer-Verlag, 2004.

\bibitem{6ush} %9
\Au{Dessimoz~D., Champod~C., Richiadi~J., Drygajlo~A.}
Multimodal biometrics for identity documents. Research Report, PFS 314-08.05. UNIL, June 2006.


\bibitem{10ush}
Face recognition vendor test. {\sf http://www.frvt.org}.  

\bibitem{11ush}
Fingerprint vendor technology evaluation. {\sf http:// fpvte.nist.gov}.

\bibitem{12ush}
\Au{Ушмаев О.\,С., Босов~А.\,В.}
Реализация концепции многофакторной биометрической идентификации в интегрированных 
аналитических системах~// Системы высокой доступности, 2007. Т.~3. Вып.~4. С.~13--23.

\bibitem{13ush}
\Au{Ushmaev O.\,S., Novikov~S.\,O.}
Integral criteria for large-scale multiple fingerprint solutions~/ Biometric Technology for Human 
Identification~// Eds.\ A.\,K.~Jain, and N.\,K.~Ratha. Proceedings of SPIE. Vol.~5404.~--- SPIE, Bellingham, 
WA, 2004. P.~534--543.

\bibitem{14ush}
Second fingerprint verification competition. FVC 2002. {\sf http://bias.csr.unibo.it/fvc2002/ }

\label{end\stat}

\bibitem{15ush}
NIST SD~29. NIST Special Database~29 ``Plain and Rolled Images from Paired Fingerprint Cards.''

 \end{thebibliography}
}
}

\end{multicols} 
 
 
 

\def\stat{agalarov}


\def\tit{АЛГОРИТМ ВЫЧИСЛЕНИЯ ЗАГРУЖЕННОСТИ 
ТЕЛЕКОММУНИКАЦИОННОЙ СЕТИ С~ПОВТОРНЫМИ ПЕРЕДАЧАМИ$^*$}
\def\titkol{Алгоритм вычисления загруженности 
телекоммуникационной сети с~повторными передачами} 

\def\autkol{Я.\,М.~Агаларов}
\def\aut{Я.\,М.~Агаларов$^1$}

\titel{\tit}{\aut}{\autkol}{\titkol}

{\renewcommand{\thefootnote}{\fnsymbol{footnote}}\footnotetext[1]
{Работа выполнена при частичной поддержке РФФИ, проекты 08-07-00152-а и
      09-07-12032-офи-м.}}

\renewcommand{\thefootnote}{\arabic{footnote}}
\footnotetext[1]{Институт проблем
информатики Российской академии наук, agglar@yandex.ru}


\end{document}
      \vskip 18pt plus 9pt minus 6pt

      \thispagestyle{headings}

      \begin{multicols}{2}

      \label{st\stat}

    
      \Abst{Рассмотрены модели сети коммутации пакетов c повторными попытками передачи 
пакетов для двух схем распределения буферной памяти: полнодоступной и полного разделения. 
Предложен итерационный метод расчета интенсивностей потоков в сети и вероятностей блокировок 
узлов, где в качестве модели узла используется СМО типа $
      \begin{matrix}
      M \\ \vec{\lambda}
      \end{matrix}
      \left |
      \begin{matrix}
      M \\ \vec{\lambda}
      \end{matrix}
      \right |
      \vec{m} \vert N$. Получено необходимое условие существования решения системы уравнений 
сохранения баланса потоков в установившемся режиме работы сети и доказана монотонная 
сходимость последовательности значений интенсивностей потоков и вероятностей блокировок, 
получаемых предлагаемым методом, к решению указанной системы.}
      
      \KW{слова: сеть коммутации пакетов; буферная память; повторные передачи; вероятность 
блокировки; итерационный метод}

     
\section{Введение}

     Одной из важных проблем, решаемых на этапе проектирования 
телекоммуникационных сетей, является задача предварительного анализа сети 
на предмет возникновения локальных и глобальных перегрузок.
     
     Причинами перегрузок наряду с другими могут быть:
     \begin{itemize}
     \item ограничение объема буферной памяти коммутационного 
оборудования;
     \item блокировка конечных терминалов;
     \item недостаточная производительность вычислительных ресурсов и 
пропускная способность каналов связи.
     \end{itemize}
     
     Ограничение буферной памяти в реальных сетях вызвано не только 
желанием разработчиков снизить себестоимость коммутаторов, но и 
требованиями к параметрам качества обслуживания (среднее время задержки и 
его разброс, вероятность потери пакетов и~др.). В то же время ограничение 
объема буферной памяти узлов является одной из причин роста числа 
повторных передач в сетях с коммутацией пакетов и, как следствие, резкого 
роста нагрузки на отдельных участках сети или сети в целом. Поэтому одной из 
задач предварительного анализа сетей является оценка загруженности узлов и 
каналов связи с учетом ограниченного объема буферной памяти.
     
     Используемые точные методы анализа сетей с коммутацией пакетов (см., 
например,~[1, 2]) разработаны в рамках экспоненциальных СеМО (сетей 
массового обслуживания) со стохастическими маршрутными матрицами и 
неограниченной буферной памятью. Однако предположение о неограниченной 
буферной памяти исключает возможность учета блокировок узлов из-за 
нехватки буферной памяти, которые и являются одной из основных причин 
возникновения повторных передач пакетов в сети.
     
     Большое число работ в последнее время посвящается системам массового 
обслуживания с повторными заявками, одним каналом (обслуживающим 
устройством), ограниченным накопителем и более общими предположениями 
относительно входящих потоков заявок и длительностей обслуживания~[2--7], 
чем при исследовании СеМО. Однако использование этих моделей при расчете 
сетей вызывает очень большие вычислительные трудности.
     
     Из множества приближенных методов расчета сетей с ограниченной 
буферной памятью в узлах следует выделить методы, используемые в теории 
второго порядка для СеМО~[2, 8], и методы, рассматривающие узлы как 
изолированные СМО с пуассоновскими входящими потоками~[9, 10]. Первые 
из них предполагают: 
     \begin{enumerate}[1)]
     \item внешний поток заявок~--- рекуррентный с известными первым и 
вторым моментами интервалов между поступлениями; 
     \item  узел~--- СМО с одним прибором и накопителем;
     \item  время обслуживания в узле~--- независимое с произвольным 
распределением с известными первым и вторым моментами;
     \item движение заявок по сети происходит согласно неразложимой 
стохастической матрице с возможностью случайного ухода из сети в каждом 
узле. Сущность этих методов состоит в том, что они при расчетах используют 
первые и вторые моменты соответствующих распределений интервалов 
поступления и обслуживания заявок. Второй подход отличается тем, что 
потоки, образованные в узлах суперпозицией внешнего потока, повторениями 
по сети не доставленных пакетов и потоками от других узлов,~--- 
пуассоновские потоки, а времена обслуживания~--- экспоненциальные. Общее 
у этих методов то, что они являются итерационными, причем каждая итерация 
выполняется в два этапа: на первом этапе вычисляются характеристики потоков 
в узлах, на втором~--- уточняются другие характеристики сети (вероятности 
блокировок узлов, моменты времени задержки пакетов в узле и~др.). В этих 
методах в качестве моделей линий связи использованы одноканальные СМО. 
     \end{enumerate}
     
     Ниже будут рассмотрены модели сетей коммутации пакетов с повторами 
из источника и из предыдущего узла. Предлагается итерационный метод 
расчета сетей, который реализует второй из упомянутых выше подходов и в 
качестве модели узла использует СМО типа $
      \begin{matrix}
      M \\ \vec{\lambda}
      \end{matrix}
      \left |
      \begin{matrix}
      M \\ \vec{\lambda}
      \end{matrix}
      \right |
      \vec{m} \vert N$. Получено необходимое условие существования решения 
системы уравнений сохранения баланса потоков в установившемся режиме 
работы сети и доказана монотонная сходимость последовательности значений 
интенсивностей потоков и вероятностей блокировок, получаемых 
предлагаемым методом, к решению системы.
     
\section{Общее описание моделей сети}
     
     Рассматривается модель сети с коммутацией пакетов в виде графа, 
состоящего из $U$~вершин и $V$~дуг. Вершины графа отождествляются с 
узлами связи, дуги~--- с линиями связи. Имеется множество источников и 
получателей пакетов, каждый из которых соединен с одним из узлов связи, 
который называется узлом-входом, если соединен с источником, и узлом-
выходом, если соединен с получателем. Передача пакета в сети происходит по 
заданному пути~$l$, соединяющему узел-вход с узлом-выходом. Будем считать 
(без потери общности), что каждый узел сети входит хотя бы в один путь сети и 
множество путей не разбивается на непересекающиеся подмножества. Известна 
интенсивность потока (первичного потока) пакетов, поступающих извне на 
каждый путь~$l$. Узлы сети имеют ограниченную буферную память с заданной 
схемой распределения, линии связи имеют заданное число однородных 
каналов. Поступивший в промежуточный узел пакет принимается в буферную 
память (занимает одно место буферной памяти), если согласно заданной схеме 
распределения ему можно выделить место в буферной памяти (узел не 
блокирован для данного пакета) и он передан без ошибок, иначе он передается 
повторно согласно процедуре повторов (из источника или из предыдущего 
узла), пока не будет успешно передан адресату. Под блокировкой узла (линии) 
понимается такое состояние узла (линии), когда согласно принятой схеме 
распределения памяти поступивший пакет не может быть принять в буфер 
данного узла (линии). Под успешной передачей (попыткой передачи) пакета 
понимается передача (попытка), когда переданный пакет принимается 
последующим узлом в буферную память. При неуспешной попытке передачи 
по линии пакета занятый им буфер освобождается сразу в случае сети с 
повторными попытками из источника и сохраняется за ним в случае повторов 
из предыдущего узла. После успешной передачи пакета занятое им место в 
буферной памяти через заданное время освобождается. Предполагается, что 
пакеты в сети не теряются.
     
     Введем обозначения:
     
     \noindent
     $v$ (или $v_i$, $i = 1$, 2,\ldots)~--- линия связи;
     
     
     \noindent
     $v^+$~--- узел-сток линии~$v$;
     
     \noindent
     $u$ (или $u_i$, $i = 1$, 2,\ldots)~--- узел связи;
     
     \noindent
     $\Omega_u^+$~--- множество исходящих из узла $u$ линий;
     
     \noindent
     $c_v$~--- канальная емкость линии~$v$;
     
     \noindent
     $N_v$~--- емкость буферной памяти, выделенной для линии~$v$;
     
     \noindent
     $N_u$~--- емкость общей буферной памяти узла~$u$;
     
     \noindent
     $L$~--- заданное множество нециклических путей;
     
     \noindent
     $L_v$~--- множество путей, содержащих линию~$v$, ($L_v\subseteq L$);
     
     \noindent
     $l=\{v_1,\ldots ,v_{S_l}\}$~--- путь, содержащий линии $v_1,\ldots 
,v_{S_l}$, где $S_l$~--- число линий в пути~$l$, индексы 1,\ldots , $S_l$ 
показывают порядок следования линий в пути, $v_l$~--- линия, исходящая из 
     узла-входа, $v_{S_l}$~--- линия, входящая в узел-выход;
     
     \noindent
     $u_{S_l+1}$~--- абонентский узел, соединенный с узлом-выходом 
пути~$l$;
     
     \noindent
     $U_u^+$~--- множество различных узлов, следующих после узла~$u$ по 
направлению к адресату в путях, проходящих через узел~$u$;
     
     \noindent
     $V_v^+$~--- множество различных линий, следующих после линии~$v$ 
по направлению к адресату в путях, проходящих по линии~$v$;
     
     \noindent
     $\lambda(l)$~--- интенсивность потока ($l$-потока) пакетов, поступающих 
из источника на узел-вход и требующих передачи на узел-выход, $\lambda(l) 
>0$, $l\in L$;
     
     \noindent
     $\mu_v$~--- интенсивность обслуживания пакета каналом линии~$v$;
     
     \noindent
     $\delta_v$~--- вероятность безошибочной передачи пакета по линии~$v$;
     
     \noindent
     $\Lambda_v$~--- интенсивность потока пакетов, успешно передаваемых 
по линии~$v$;
     
     \noindent
     $\Lambda_v^*$~--- интенсивность суммарного потока пакетов, 
требующих передачи по линии~$v$;
     
     \noindent
     $\Lambda_v^*(l)$~--- интенсивность $l$-потока, поступающего на 
линию~$v$;
     
     \noindent
     $\pi_u$~--- вероятность блокировки узла~$u$;
     $\pi_v$~--- вероятность блокировки узла для пакетов, требующих 
передачи по исходящей из узла линии~$v$.
     
     Во всех рассматриваемых ниже моделях узла коммутации 
предполагается, что потоки $\Lambda_v^*$~--- пуассоновские, а времена 
обслуживания пакетов каналами связи~--- экспоненциальные с параметрами 
$\mu_v$, $v\in V$. Предполагается также, что внешние нагрузки~--- 
реализуемые, т.\,е.\ в стационарном режиме работы сети интенсивности 
первичных входных потоков равны интенсивностям соответствующих 
выходных (покидающих сеть) потоков. Всюду ниже сеть рассматривается в 
стационарном режиме.
     
\section{Сеть с повторами из источника}
     
     В качестве модели коммутационного узла используется СМО с 
ограниченным накопителем (буферной памятью) и несколькими линиями из 
однотипных каналов, в которой сделаны также следующие предположения:
     \begin{enumerate}[1.]
\item Места в буферной памяти распределяются согласно одной из двух 
схем:
\begin{itemize}
\item полнодоступная схема (CS)~--- каждое свободное место хранения 
доступно любой заявке (пакету);
\item схема полного разделения памяти (CP)~--- заявке, требующей передачи 
по линии~$v$ ($v$-заявке), доступны всего~$N_v$ мест, где 
$\sum\limits_{v\in\Omega_u^+} N_v = N$.
\end{itemize}
\item Суммарные потоки первичных и повторных $v$-заявок являются 
независимыми в совокупности пуассоновскими потоками. Для 
обслуживания $v$-заявки требуется одновременно одно место хранения и 
один канал типа~$v$, $v\in\Omega_u^+$.
\item Поступившей в СМО заявке предоставляется место в накопителе, если 
она передана без ошибок и в момент ее поступления в накопителе есть 
доступное свободное место, иначе заявка получает отказ.
\item Принятые в СМО $v$-заявки обслуживаются линией~$v$ в порядке 
поступления.
\item Время занятия канала $v$-заявкой~--- экспоненциально 
распределенная случайная величина с параметром $0<\mu_v<\infty$, $v\in 
\Omega_u^+$, независимая от других случайных событий в системе.
\item Обслуженная заявка освобождает сразу место в накопителе СМО.
\item Заявка, получившая отказ, повторяется через заданное время из 
источника.
\end{enumerate}

     Пусть во всех узлах сети распределение буферной памяти происходит по 
схеме CS. При полнодоступной схеме и повторах из источника в 
установившемся режиме работы сети справедливы следующие соотношения 
для потоков в узлах:
     \begin{align}
     \Lambda_v(l) =\Lambda_v^*(l)\left (1-\pi_v\right )\,,\quad
     \Lambda^*_{v_i}(l) =\Lambda_{v_{i-1}}(l)\,,\quad
     \Lambda_{v_{S_l}} =\lambda(l)\,,\quad l\in L\,,\notag\\
     \Lambda_v=\sum\limits_{l\in L}\Lambda_v(l)\,,\quad 
\Lambda_v^*=\sum\limits_{l\in L}\Lambda_v^*(l)\,,\quad v\in V\,.
     \end{align}
     
     Из~(1) для вычисления параметра $\Lambda_{v_i}(l), $i=1,\ldots ,S_l$, 
$l\inL$, получаем формулу
     \begin{equation}
     \Lambda_{v_i}(l) =\fr{\Lambda_{v_{i+1}}(l)}{(1-
\pi_{u_{i+1}})\delta_{v_{i+1}}} =\fr{\lambda(l)}{(1-
\pi_{u_{S_l+1}})\prod\limits_{}^{S_l} (1-\pi_{u_j})\delta_{v_j}}\,,\quad i=1,\ldots 
,S_l-1\,.
     \end{equation}
     
     Здесь и далее по тексту статьи считается, что при заданных~$\lambda(l)$, 
$l\inL$, величины $\pi_{u_{S_l+1}}(l)$, $l\in L$, заранее вычислены.
     
     Пусть $\overline{k}_u =\{k_v,\ v\in\Omega_u^+\}$~--- состояние буферной 
памяти узла $u\in U$, $k_v$~--- число пакетов в буферной памяти узла, 
передаваемых по линии~$v$, $A_m = 
\{\overline{k}_u:\sum\limits_{v\in\Omega_u^+} k_v=m\}$~--- множество 
различных состояний, при которых в памяти узла заняты ровно $m$~буферов. 
Тогда с учетом введенных выше обозначений и предположений для 
стационарной вероятности блокировки узла можем написать формулу~[2, 11]
\begin{equation}
\pi_u = \fr{1}{G_{N_u}}\sum\limits_{\overline{k}\in A_N} p\left ( 
\overline{k}_u,\overline{\rho}_u^*\right )\,,
\end{equation}
где 
\begin{align}
p(\overline{k}_u,\overline{\rho}_u^*) & = 
\prod\limits_{v\in \Omega_u^+} z_v(\rho_v^*,k_v)\,,\notag\\
Z_v(\rho_v^*,k_v) & = 
\begin{cases}
\fr{\rho_v^{*k_v}}{k_v!} & \mbox{при}\ k_v<c_v\,,\\
\fr{\rho_v^{*k_v}}{c_v ! c_v^{k_v-c_v}} & \mbox{при}\ k_v\geq c_v\,,
\end{cases}\\
G_{N_u} & = \sum\limits_{m=0}^N \sum\limits_{\overline{k}\in A_m} 
p(\overline{k}_u, \overline{\rho}_u^* )\,,\quad \overline{\rho}_u^*=\{\rho_v^*,\ v\in 
\Omega_u^*\}\quad \rho_v^* =\fr{\Lambda_v^*}{\mu_v}\,,\quad v\in\Omega_u^+\,.
     \end{align}
     
     Таким образом, из соотношений~(1)--(5) относительно неизвестных 
величин~$\pi_u$, $u\in U$, получаем систему нелинейных уравнений вида
     \begin{equation*}
     \pi_u = f_u\left ( 
\overline{\lambda},\overline{\mu},\overline{N},\overline{\pi}\right )\,,\quad u\in 
U\,,
     \end{quation*}
     где $\overline{\lambda} =\{\lambda (l),\ l\in\L_u\}$, $\overline{mu} 
=\{\mu_{u^\prime},\ u^\prime\in u\cup U_u^+\}$, $\overline{N}=\{N_{u^\prime},\  
u^\prime \in u\cup U_u^+\}$, $\overline{\pi} = \{\pi_{u^\prime},\ u^\prime \in u\cup 
U_u^+\}$.
     
     Переобозначив $1-\pi+u$ через~$v_u$, выражение в правой части 
равенства для~$p(\overline{k}_u, \overline{\rho}_u^*)$ из (4)~--- через 
$p_{\overline{k}}(\overline{\rho}_u, y_u)$, выражение в правой части равенства 
для~$\pi_u$ из (3)~--- через $1-q_{N_u}(\overline{\rho}_u, y_u)$, где 
$\overline{\rho}_u = (\rho_v, \ v\in\Omega_u^+)$, $\rho_v = \rho_v^* y_u = 
\Lambda_v/\mu_v$, $v\in\Omega_u^+$, получим систему нелинейных уравнений 
относительно неизвестных переменных~$y_u$
     \begin{equation}
     y_u = q_{N_u}\left ( \overline{\rho}_u,y_u\right ),\quad u\in U\,.
     \end{equation}
     
     Отметим, что $\overline{\rho}_u = \{\rho_v, v\in\Omega_u^+\}$ где 
$\rho_v$~--- функция переменных $\overline{y}_u =\{y_{u^\prime},\ u^\prime \in 
U_u^+\}$.
     
     Обозначим набор $\{y_u, u\in U\}$ через~$\overline{y}$. Будем говорить, 
что решение~$\overline{y}$ положительное, если $y_u\in (0,\,1]$ для всех $u\in 
U$.
     
     \medskip
     
     \noindent
     \textbf{Утверждение~1.} \textit{Если} 
     $\overline{y}^\prime = \{u_u^\prime \in (0,\,],\ u\in U\}~--- \textit{решение 
системы уравнений}~(6), \textit{то необходимо выполнение для всех} $u\in U$ 
\textit{условия}
     \begin{equation}
     \fr{\sum\limits_{\overline{k}\in A_{N_u-1}} 
p_{\overline{k}}(\overline{p}_u^\prime , 1)}
     {\sum\limits_{\overline{k}\in A_{N_u}} 
p_{\overline{k}}(\overline{\rho}_u^\prime, 1)} >1\,,
     \label{e7ag}
     \end{equation}
     \textit{где} $\overline{\rho}_u^\prime$~--- \textit{значение 
переменной}~$\overline{\rho}_u$ \textit{при} $\overline{y} 
=\overline{y}^\prime$.
     
     \medskip
     
     \noindent
     Д\,о\,к\,а\,з\,а\,т\,е\,л\,ь\,с\,т\,в\,о\,.\ Пусть $\overline{y}^\prime = 
\{y_u^\prime \in (0,\,1]$, u\in U\}$~--- решение системы~(5). Фиксируем 
произвольный узел~$u$ и положим $u_{u^\prime} = y_u^\prime$ для всех 
$u^\prime\not= u$, $u^\prime\in U$. Отметим, что значение 
переменной~$\overline{\rho}_u$ при заданных значениях~$\lambda(l)$, $l\in L$, 
однозначно определяется переменными~$y_{u^\prime}$, $u^\prime\not= u$, 
$u^\prime\in U$ (см.~(2)). Рассмотрим уравнение
\begin{equation}
Y_u = q_{N_u} (\overline{\rho}_u^\prime, y_u)\,.
\label{e8ag}
\end{equation}
     
     Из работы~[12] (см.\ утверждение~4) следует, что уравнение~(\ref{e8ag}) 
имеет положительное решение тогда и только тогда, когда в узле~$\u$ 
выполняется условие~(\ref{e7ag}), при этом оно будет единственным 
положительным решением. Так как узел $u$~--- произвольный, то получаем, 
что неравенство~(\ref{e7ag}) должно выполняться для всех $u\in U$.
     
     \medskip
     
     \noindent
     \textbf{Следствие.} \textit{Выполнение неравенств} $\mu_v c_v / 
\Lambda_v >1$, $v\in V$, \textit{является необходимым условием 
существования положительного решения системы уравнений}~(\ref{e6ag}).
     
     \medskip
     
     \noindent
     Д\,о\,к\,а\,з\,а\,т\,е\,л\,ь\,с\,т\,в\,о\ непосредственно вытекает из следствия 
утверждения~4 в~[12].
     
     \smallskip
     Пусть задана последовательность $\overline{y}[n] =\{ y_u [n],\ u\in U\}$, 
$n\geq 0, где $y_n[n+1]=q_{N_u}[n+1]=q_{N_u}(\overline{\rho}_u[n],y_u[n])$, 
y_u[0]=1$, $u\in U$, а $\overline{\rho}_u[n]$~--- это~$\overline{\rho}_u$, 
вычисленное при $y_u =1-\pi_u =y_u[n]$. В дальнейшем будем писать 
$\overline{y}[n+1]<\overline{y}[n]$, если для заданного $n\geq 0$ выполняется 
$y_u[n+1]<y_u[n]$ для всех $u\in U$.
     
     \medskip
     
     \noindent
     \textbf{Утверждение 2.} \textit{Для всех} $n\geq 0$ \textit{верно} 
$\overline{y}[n+1] <\overline{y}[n]$.
     \medskip
     
     \noindent
     Д\,о\,к\,а\,з\,а\,т\,е\,л\,ь\,с\,т\,в\,о\,.\ Докажем, что для любых~$u$, 
$u^\prime \in U$, принадлежащих одновременно хотя бы одному пути, 
справедливо неравенство
     \begin{equation}
     \fr{\partial q_{N_u}(\overline{\rho}_u,y_u)}{\partial y_{u^\prime}}>0\,.
     \end{equation}
     
     Взяв производную от $y_u =q_{N_u}(\overline{\rho}_u,y_u)$ как от 
сложной функции, получаем
     \begin{equation}
     \fr{\partial q_{N_u}(\overline{\rho}_u,y_u)}{\partial y_{u^\prime}} = 
\sum\limits_{v\in\Omega_u^+}\fr{\partial q_{N_u}(\overline{\rho}_u,y_u)}{\partial 
\Lambda}\,\fr{\partial \Lambda_v}{\partial y_{u^\prime}}.
     \end{equation}
     
     Введем обозначения:
        .
     Из (3)--(5), взяв производную по~$\Lambda_v$, имеем
     \begin{equation}
     \fr{\partial q_{N_u}(\overline{\rho}_u,y_u)}{\partial\Lambda_v} = 
\fr{1}{\Lambda_v}\,q_{N_u}(\overline{\rho}_u,y_u)\left [ d_{N_u-
1}(\overline{\rho}_u,y_u)-d_{N_u}(\overline{\rho}_u,y_u)\right ]\,.
     \end{equation}
     
     Из (1) и~(2), взяв производную по~$y_{u^\prime}$, получаем
     \begin{equation}
     
      . (12)
     
     Подставив~(11) и~(12) в~(10), имеем
     \begin{equation*}
     \fr{\partial q_{N_u}(\overline{\rho}_u,y_u)}{\partial y_{u^\prime}} = 
\fr{1}{}\,q_{N_u}(\overline{\rho}_u, y_u)\left [d_{N_u}(\overline{\rho}_u,y_u)-
d_{N_u -1} (\overline{\rho}_u,y_u)\right 
]\sum\limits_{v\in\Omega_u^+}\fr{1}{\Lambda_v} \sum\limits_{l:l\in L_v,\, 
\mu^\prime\in l} \Lambda_v(l)\,.
     \end{equation*}
     
     Так как справедливо неравенство $d_{N_u}(\overline{\rho}_u,y_u)-d_{N_u 
-1} (\overline{\rho}_u,y_u) >0$ (см.\ утверждение~1 из~[12]), то из последнего 
равенства следует доказательство неравенства~(9). Из определения 
последовательности $\overline{y}[n]$, $n\geq 0$, и из~(9) следует 
доказательство утверждения~2.
     
     \medskip
     
     \noindent
     \textbf{Утверждение 3.} \textit{Последовательность}~$\overline{y}$, 
$n\geq 0$, \textit{сходится к положительному решению системы}~(6) 
\textit{тогда и только тогда, когда существует положительное решение 
системы}~(6).
     
     \medskip
     
     \noindent
     Д\,о\,к\,а\,з\,а\,т\,е\,л\,ь\,с\,т\,в\,о\,.\ Пусть $\overline{y}^* = \{y_u^*\in 
(0,\,1],\ u\in U\}$~--- решение системы уравнений~(6), $\overline{p}_u^*$~--- 
значение переменной~$\overline{\rho}_u$ при~$\overline{y}^*$. Очевидно, 
$u_u^*<1$, $u\in U$, так как $q_{N_u}(\overline{\rho}_u,y_u)<1$ при любых 
$y_u\in (0,\,1]$, $u\in U$. Пусть для некоторого $n\geq 0$ 
$\overline{y}[n]>\overline{y}^*$ (существование такого~$n$ вытекает из того, 
что $u_n[0] =1$ и $y_u^*<1$, $u\in U$). Тогда, как следует из~(9), для каждого 
узла $u\in U$ $y_u[n+1]=q_{N_u} (\overline{\rho}_u[n],y_u[n]) > q_{N_u}
     (\overline{\rho}_u^*,y_u^*)=y_u^*$, т.\,е.\ последовательность~$u_u[n]$, 
$n\geq 0$, ограничена снизу величиной~$\overline{y}_u^*$. Значит, 
существуют пределы $\lim\limits_{n\rightarrow\infty} y_u[n]=y_u^0\geq 
y_u^*>0$ для всех $u\in U$. Так как $q_{N_u}(\overline{\rho}_u,y_u), \rho_v$, 
$v\in\Omega_u^+$,~--- непрерывные по $y_{u^\prime}$, $u^\prime\in u\cup 
U_u^+$ функции, то можно написать $\lim\limits_{n\rightarrow\infty} q_{N_u} 
(\overline{\rho}_u[n],y_u[n])=q_{N_u}(\overline{\rho}_u^0,y_u^0)=y_u^0$, где 
$\overline{\rho}_u^0$~--- значение переменной~$\overline{\rho}_u$ при 
$y^0_{u^\prime}$, $u^\prime\in U_u^+$, т.\,е.\ $\overline{y}^0 =\{y_u^0\in (0,\,1),\ 
u\in U\}$~--- положительное решение уравнения~(6).
     
     Пусть теперь $\lim\limits_{n\rightarrow\infty} y_u[n] =y_u^*>0$ для всех 
$u\in U$. Тогда, как показано в первой части доказательства утверждения, 
$\overline{y}^0 = \{y_u^0\in (0,\,1),\ u\inU\}$~--- положительное решение 
уравнения~(6). Утверждение~3 доказано.
     
     \medskip
     
     \noindent
     \textbf{Следствие 2.} \textit{Система}~(6) \textit{не имеет 
положительного решения тогда и только тогда, когда} 
$\lim\limits_{n\rightarrow\infty} y_u[n]=y_u^*=0$ \textit{для всех} $u\in U$.
     
     Пусть во всех узлах сети распределение буферной памяти происходит по 
схеме CP. Тогда формула вероятности блокировки узла $u\inU$ для $v$-заявки 
($v\in \Omega_u^*$) записывается в том же виде, что и~(3)--(5), с заменой 
всюду индекса~$u$ на~$v$, за исключением обозначения~$\Omega_u^*$. 
Нетрудно заметить, что в случае этой схемы система уравнений~(6) примет вид
     \begin{equation}
     y_v = q_{N_u}(\rho_v,y_v)\,,\quad v\in V\,,
     \end{equation}
     где $\rho_v$ является функцией~$y_{v^\prime}$, $v^\prime \in V_v^+$, 
которая обладает всеми свойствами системы~(6), использованными при 
доказательстве утверждений~1, 2 и~3 и следствий. Неравенства вида~(6) в 
данном случае преобразуются в $\mu_v c_v/\Lambda_v >1$, $v\in V$.
     
     Заметим также, что все рассуждения, приведенные выше для сети с одной 
только из указанных выше схем распределения буферной памяти, справедливы 
и в смешанном случае, когда в узлах используется любая из этих схем.
     
\section{Сеть с повторами из предыдущего узла}
     
     Рассмотрим сеть с полнодоступной буферной памятью и повторами из 
предыдущего узла. В качестве модели узла используется СМО, отличающаяся 
от СМО, определенной в предыдущем разделе, только пунктами~6 и~7. Вместо 
действий, указанных в этих пунктах, реализуется следующее: выполненная 
     $v$-заявка с заданной вероятностью~$B_{v^+}$ (вероятность блокировки 
последующего узла или ошибки при передаче пакета по линии~$v$) 
повторяется в данном узле через заданное время~$\tau_v$ (тайм-аут) и с 
вероятностью~$1-B_{v^+}$ покидает систему через время~$t_v$ навсегда, 
сразу освободив занятый канал и место в буферной памяти. Для такой модели 
существует более общая формула для вычисления вероятности блокировки 
системы для $v$-заявок (см.~[11, 12]), которая задает зависимость вероятности 
блокировки узла в виде функции от вероятностей блокировок последующих 
узлов~$B_{v^+}$, $v\in \Omega_u^+$, при заданных значениях остальных 
параметров, в частности~$\Lambda_v$, $v\in\Omega_u^+$.
     
     В сети с повторами из предыдущего узла при установившемся режиме 
работы справедливы следующие уравнения баланса потоков:
     \begin{align*}
     \lambda (l) & = \Lambda_v^*(l) (1-\pi_v)\delta_v\,,\quad l\in L_v\,,\\
     \Lambda_v & = \sum\limits_{l\in L_v} \lambda_v(l)\,,\quad 
\Lambda_v^*=\sum\limits_{l\in L_v} \Lambda_v^*(l)\,,\quad v\in V\,.
     \end{align*}
     
     Тогда с учетом введенных выше обозначений и формул~(1)--(7) и~(18) 
из~[12], заменив обозначение~$B_{v^+}$ на~$y_{v^+}$, $v\in\Omega_u^+$, 
можем написать систему нелинейных уравнений
     \begin{align}
     y_u &= q_{N_u}(\overline{\rho}_u,y_u)\ \mbox{при схеме распределения 
CP
     }\,,\\
     y_v &= q_{N_u}(\rho_v,y_v)\ \mbox{ при схеме распределения CS
     },\, \ v\in\Omega_u^+\,\ u\inU\,,
     \end{align}
где компоненты~$\rho_v$ набора $\overline{\rho}_u$~--- функции 
переменных~$y_{v^+}$, $v\in \Omega_u^+$.

     Нетрудно видеть, что системы~(14) и~(15) обладают всеми свойствами 
системы~(6), использованными при доказательстве утверждений~1, 2, 3 и 
следствий, т.\,е.\ для систем~(13) и~(14) также справедливы утверждения~1, 2, 
3 и следствия.
     
\section{Алгоритм расчета}
     
     Для вычисления характеристик потоков в узлах и вероятностей 
блокировок пакетов предлагается следующий алгоритм, описывающий 
изложенную выше итерационную процедуру. Для описания значений, 
вычисляемых на $k$-м шаге алгоритма, к обозначениям соответствующих 
параметров приписывается знак~$[k]$. Введем новые обозначения:
     
     \noindent
     $y_u^v$~--- вероятность отсутствия блокировки узла $u\in U$ для 
пакетов, направляемых на линию $v\in \Omega_u^+$;
     \begin{align*}
     y_u^v & = \begin{cases}
     y_u & \mbox{для}\ v\in\Omega_u^+\ \mbox{при схеме}\ CS\,,\\
     y_v & \mbox{при схеме распределения CP}\,;
     \end{cases}\\
     \overline{\rho}_u^v & = 
     \begin{cases}
     \overline{\rho}_u & \mbox{для}\ v\in\Omega_u^+\ \mbox{при схеме}\ 
CS\,,\\
     \rho_v & \mbox{при схеме распределения CP}\,;
     \end{cases}\\
     q_{N_u}^v(\overline{\rho}_u^v, y_u^v) & = 
     \begin{cases}
     q_{N_u}(\overline{\rho}_u,y_u) & \mbox{для}\ v\in\Omega_u^+\ 
\mbox{при схеме}\ CS\,,\\
     q_{N_v}(\rho_v,y_v)& \mbox{при схеме распределения CP}\,;
     \end{cases}
     \end{align*}
     
     Тогда система уравнений для смешанного варианта сети, аналогичная 
системам~(6), (13)--(15), записывается в виде
     $$
     y_u^v = q_{N_u}^v(\overline{\rho}_u^v,\overline{y}_u^v)\,,\quad u\in 
U\,,\quad v\in\Omega_u^+\,.
     $$

\textbf{Шаг 1.} \textit{Инициализация}. Вычисление начальных значений 
параметров~$\rho_v^*$, $v\in V$: $\Lambda_v[0]=\sum\limits_{l\in L_v} 
\lambda(l)/((1-\pi_{u_{S_l+1}}(l)\prod\limits_{v^\prime\in 
V^+}\delta_{v^\prime}$, $\rho_v^*[0]=\Lambda_v[0]/\mu_v$, $v\in V$, 
$y_u^v[0]=1$, $u\in U$, $v\in\Omega_u^+$.
     
     \textbf{Шаг} $k$ ($k>1$).
     \begin{enumerate}[1.]
\item \textit{Проверка необходимых условий существования решения}. 
Если для некоторой линии $v\in V$ выполняется условие 
$c_v\mu_v/(\Lambda_v[k-1])\leq 1$, то алгоритм заканчивает работу с 
результатом <<система не имеет решения>>. Если в некотором узле~$u$, 
в котором используется полнодоступная схема, условие 
$c_v\mu_v/(\Lambda_v[k-1])> 1$ выполняется для всех $v\in\Omega_u^+$, 
то проверяется условие~(7) заданных $\Lambda_v[k-1]$, $v\in V$, и при 
невыполнении этого условия алгоритм заканчивает работу с результатом 
<<система не имеет решения>>.
     \item \textit{Вычисление вероятностей блокировок}. Используя 
значения параметров $\overline{\rho}_u^v[k-1]$, $y_u^v[k-1]$, $u\in U$, 
$v\in\Omega_u^+$, с помощью соответствующих формул~(3)--(5) или 
формул~(1)--(7) и~(18) из~[12] (в зависимости от типа схемы 
распределения буферной памяти и процедуры повторов передач) 
вычисляется $y_u^v[k]=1-\pi_u[k]$, $u\in U$, $v\in\Omega_u^+$. При этом 
рекомендуется использовать метод свертки Базена (см.~[13]), 
позволяющий производить рекуррентные вычисления (подробно этот 
метод описан также в~[2, 9]).
     \item \textit{Вычисление значений параметров} $\Lambda_v[k]$, $v\in V$:
     \begin{enumerate}[$i$)]
     \item в случае повторов от источника
     \begin{gather*}
     \Lambda_{v_{S_l}}[k]=\lambda(l)\,,\ \Lambda_{v_i}^*(l)[k]=
     \fr{\Lambda_{v_i}(l)[k]}{y^{v_i}_{u_i}[k-1]\delta_{v_i}}\,,
     \Lambda_{v_i-1}(l)[k]=\Lambda_{v_i}^*(l)[k]\,,\ i=1,\ldots ,S_l\,,\ l\in 
L\,,\\
     \Lambda_v^=[k] = \sum\limits_{l\in L_v} \Lambda_v^=(l)[k]\,,\quad 
v\in V\,;
     \end{gather*}
     \item в случае повторов из предыдущего узла
     \begin{equation*}
     \Lambda_v^*[k]=\fr{\Lambda_v[0]}{y_u^v[k-1]\delta_v}\,,\quad 
v\in\Omega_u^+\,,\quad u\in U\,.
     \end{equation*}
     \end{enumerate}
     \item \textit{Проверка условий останова алгоритма}. Если хотя бы для 
одной $v\in V$ для заданного значения точности $\varepsilon >0$ выполняется 
условие
     $$
     \fr{\vert \Lambda_v^*[k]-\Lambda_v^*[k-1]\vert}{\Lambda_v^*[k]} 
>\varepsilon\,,
     $$
     то вычисляются параметры $\overline{\rho}_u^v[k]$, $u\in U$, 
$v\in\Omega_u^+$, и осуществляется переход к шагу~$k$, положив $k$ 
равным~$k+1$, иначе алгоритм завершает работу.
     \end{enumerate}
     
     По завершении алгоритма либо выявляется, что система уравнений не 
имеет положительного решения, либо вычисляются интенсивности потоков, 
поступающих в узлы и на линии сети, и вероятности блокировок узлов для 
пакетов. Далее эти характеристики могут быть использованы для вычисления 
других характеристик сети (средних задержек, среднего числа повторов в узлах, 
узких участков сети и~др.).
     
\section{Примеры расчета}

     В качестве примера рассматривается сеть с тремя узлами, топология 
которой задается графом, показанным на рис.~1. В рассматриваемой сети 
предполагается полнодоступная схема распределения буферной памяти и 
процедура повторных передач из источника. Для вычисления вероятностей 
блокировок узлов и интенсивностей потоков, поступающих на линии связи, 
был использован алгоритм, представленный в разд.~5, и имитационная модель 
сети. В табл.~1 и на рис.~2 приведены результаты вычислений при следующих 
значениях входных параметров: емкости накопителей $N_u = 15$ для всех 
узлов, множество путей
     $$
     L = \{l_1, l_2, l_3, l_4, l_5,l_6\}\,,\  l_1=\{v_1\}\,,\ l_2=\{v_2\}\,,\ 
l_3=\{v_3\}\,,\ l_4=\{v_2,v_3\}\,,\ l_5=\{v_1,v_2\}\,,\ l_6=\{v_3,v_1\}\,,
     $$
     интенсивности первичных потоков $\lambda (l) =2$, 2,5, 2,7, 2,8, 2,9, 3, 
3,1, 3,2, $l \in L$.  Строки~1, 3 в табл.~1 и графики~\textsl{1}, \textsl{3} на 
рис.~2 соответствуют вариантам расчетов с помощью предложенного 
алгоритма, а \textsl{2}, \textsl{4}, \textsl{5}~--- с помощью имитационной 
модели. В вариантах~1 и~2 канальные емкости $c_v = 10$ для всех линий, 
параметр экспоненциального времени обслуживания   для всех линий, 
интервал повтора для всех заявок равен~10, в вариантах~3 и~4 емкости $c_v = 
1$ для всех линий, параметр экспоненциального времени обслуживания $\mu_v 
= 10$ для всех линий, интервал повтора равен~10, в варианте~5 емкости $c_v = 
10$ для всех линий, время обслуживания пакета каналом связи равно~10, 
интервал повтора равен~10. Во всех вариантах первичные потоки~--- 
пуассоновские, $\pi_{u_{S_l+1}}(l) =0$, $\delta_v =1$ $v\in V$, $l\in 
\begin{figure*} %fig1
     \Caption{Граф сети
     \label{f1ag}}
     \end{figure}
     
\begin{table}\small
\begin{center}
\Caption{Зависимость вероятности отсутствия блокировки узла от интенсивности первичных 
потоков
\label{t1ag}}
\vspace*{1ex}

\begin{tabular}{ccccccccc}
\hline
& \multicolumn{8}{$\lambda (l)$\\
\cline{2-9}
    
&2&2.5&2.7&2.8&2.9&3&3.1&3.2\\
\hline
1&0,9967&0,9758&0,9504&0,9272&0,8825&0,0000&0,0000&0\\
2&0,9905&0,9882&0,9257&0,9211&0,6928&0,0000&0,0000&0\\
3&0,9998&0,9964&0,9904&0,9844&0,9746&0,9568&0,9018&0\\
4&1,0000&0,9954&0,9934&0,9873&0,972&0,949&0,8787&0\\
5&0,9986&0,9917&0,9718&0,9677&0,9569&0,8018&0,0000&0\\
     \hline
     \end{tabular}
     \end{center}
     \end{table}
     
     Как показывают результаты, отраженные в табл.~1 и на рис.~2, а также 
другие вычислительные эксперименты, оценки вероятностей блокировок узлов, 
полученные с помощью представленного алгоритма, дают вполне приемлемые 
по точности значения для предварительного анализа сети на реализуемость 
первичных потоков пакетов.
     
     \begin{figure} %fig2
     \Caption{Зависимость вероятности отсутствия блокировки узла от 
интенсивности первичных потоков
     \label{f2ag}}
     \end{figure}
     
     Кроме того, точность результатов, полученных с помощью предлагаемого 
итерационного метода, увеличивается с ростом разветвленности сети и 
увеличением интервала повторов передач пакета. Эксперименты также 
показывают, что, как правило, погрешность, вносимая заменой многоканальной 
линии одноканальной с равной пропускной способностью, больше, чем 
вносимая предположением о пуассоновских входных потоках и 
экспоненциальных временах обслуживания.
     
\section{Заключение}
     
     Проведенные исследования показали, что алгоритм расчета сетей, 
предложенный в данной статье, обладает следующими достоинствами:
     \begin{enumerate}[1.]
     \item Использует в качестве модели сети СеМО, представляющие собой 
совокупность общепринятых СМО типа $
      \begin{matrix}
      M \\ \vec{\lambda}
      \end{matrix}
      \left |
      \begin{matrix}
      M \\ \vec{\lambda}
      \end{matrix}
      \right |
\vec{m} \vert N$ со схемами распределения CS или CP, связанных 
уравнениями баланса потоков в узлах.
\item При реализуемых первичных потоках сходится к положительному 
решению системы уравнений баланса потоков в узлах.
\item При реализуемых первичных потоках вычисляет вероятности 
блокировок узлов и загруженности узлов и каналов связи с приемлемой 
для предварительного анализа сети точностью (относительная 
погрешность вероятности блокировки $\sim 0.1$) .
\item Позволяет определить реализуемость первичных входных потоков.
\item При использовании алгоритма Базена требует для выполнения 
одного шага порядка $\sum\limits_{u\in U} (N_uK_u+N_u^2/2)$ 
арифметических операций, где $K_u$~--- степень узла~$u$.
    \end{enumerate}


{\small\frenchspacing
{%\baselineskip=10.8pt
\addcontentsline{toc}{section}{Литература}
\begin{thebibliography}{99}    
\bibitem{1ag}
\Au{Клейнрок~Л.}
Теория массового обслуживания.~--- М.: Машиностроение, 1979.

\bibitem{2ag}
\Au{Башарин~Г.\,П., Бочаров~П.\,П., Коган~Я.\,А.}
Анализ очередей в вычислительных сетях.~--- М.: Наука, 1989.

\bibitem{3ag}
\Au{Бочаров~П.\,П., Д'Апиче~Ч., Мандзо~Р., Фонг~Н.\,Х.}
Об обслуживании многомерного пуассоновского потока на одном 
приборе с конечным накопителем и повторными заявками~// Проблемы 
передачи информации, 2001. Т.~37. Вып.~4. С.~130--140.

\bibitem{4ag}
\Au{Tsitsiashvili~G.\,Sh., Osipova~M.\,A.}
Construction of queueing networks with stationary product distributions~// 
Proceedings of 5th Workshop (International ) on Retrial Queues.~--- Seoul: 
Korea University, 2004. Р.~111--115.

\bibitem{5ag}
\Au{Моисеева~С.\,П., Морозова~А.\,С., Назаров~А.\,А.}
Исследования СМО с повторным обращением и неограниченным числом 
обслуживающих приборов методом предельной декомпозиции~// 
Вычислительные технологии, 2008. Т.~13. Спец. вып.~5. С.~88--92.

\bibitem{6ag}
\Au{Wuechner~P., Meer~H., Bolc~G., Roszik~J., Sztrik~J.}
Modeling finite-source retrial queueing systems with unreliable heterogeneous 
servers and different service policies using MOSEL~// Proceedings of the 14th 
International conference on analytical and stochastic modeling techniques and 
applications, 2007, Prague, Czech Republic.~--- Sbr.-Dudweiler: Digitaldruck 
Pirrot GmbH, 2007. P.~75--80.

\bibitem{7ag}
\Au{Artalejo~J., G\'{o}mez-Corral~A.}
Retrial queueing systems. A computational approach.~--- Berlin: Springer 
Berlin Heidelberg, 2008.

\bibitem{8ag}
\Au{Бочаров~П.\,П.}
Приближенный метод расчета разомкнутых неэкспоненциальных сетей 
массового обслуживания конечной емкости с потерями или 
блокировками~// Автоматика и телемеханика, 1987. №\,1. C.~55--65.

\bibitem{9ag}
\Au{Вишневский~В.\,М.}
Теоретические основы проектирования компьютерных сетей.~--- М.: 
Техносфера, 2003.

\bibitem{10ag}
\Au{Таранцев~А.\,А.}
Инженерные методы теории массового обслуживания.~--- М.: Наука, 
2007.

\bibitem{11ag}
\Au{Kamoun~F., Kleinrock~L.}
Analysis of shared finite storage in a computer networks node environment 
under general traffic conditions~// IEEE Trans. on Commun., 1980. Vol.~28. 
No.\,7. P.~992--1003.

\bibitem{12ag}
\Au{Агаларов~Я.\,М.}
Приближенный метод вычисления характеристик узла 
телекоммуникационной сети с повторными передачами~// Информатика 
и её применения, 2009. Т.~3. Вып.~2. С.~2--10.

\label{end\stat}


\bibitem{13ag}
\Au{Buzen~J.\,P.}
Computational algorithm for closed queuing networks with exponential 
servers~// Communications ACM, 1973. Vol.~16. No.\,9. P.~527--531.

 \end{thebibliography}
}
}
\end{multicols} 

\def\stat{kudr}

\def\tit{ПРИБЛИЖЕННЫЕ МЕТОДЫ РЕШЕНИЯ ЗАДАЧИ ДИАГНОСТИКИ ПЛОСКИМ 
ЗОНДОМ СИЛЬНОИОНИЗОВАННОЙ ПЛАЗМЫ С~УЧЕТОМ КУЛОНОВСКИХ 
СТОЛКНОВЕНИЙ}

\def\titkol{Приближенные методы решения задачи диагностики плоским 
зондом сильноионизованной плазмы} %с~учетом Кулоновских  столкновений}

\def\autkol{И.\,А.~Кудрявцева, А.\,В.~Пантелеев}
\def\aut{И.\,А.~Кудрявцева$^1$, А.\,В.~Пантелеев$^2$}

\titel{\tit}{\aut}{\autkol}{\titkol}

%{\renewcommand{\thefootnote}{\fnsymbol{footnote}}\footnotetext[1]
%{Работа поддержана Российским фондом фундаментальных исследований
%(проекты 11-01-00515а и 11-07-00112а), а также Министерством
%образования и науки РФ в рамках ФЦП <<Научные и
%научно-педагогические кадры инновационной России на 2009--2013~годы>>.}}


\renewcommand{\thefootnote}{\arabic{footnote}}
\footnotetext[1]{Московский авиационный институт, irina.home.mail@mail.ru}
\footnotetext[2]{Московский авиационный институт, avpanteleev@inbox.ru}

\vspace*{-2pt}

\Abst{Сформирована математическая модель, описывающая динамику сильноионизованной 
плазмы с учетом столкновений заряженных частиц вблизи плоского зонда. Модель включает уравнение 
Фоккера--Планка и уравнение Пуассона. Предложено два подхода к решению задачи: на основе метода 
статистических испытаний Мон\-те-Кар\-ло и на основе композиции метода крупных частиц и метода 
расщепления.} 

\vspace*{-2pt}

\KW{телекоммуникационные системы; метод Монте-Карло; метод крупных частиц; метод 
расщепления; зонд; уравнение Фоккера--Планка; уравнение Пуассона} 

\vspace*{-4pt}

 \vskip 8pt plus 9pt minus 6pt

      \thispagestyle{headings}

      \begin{multicols}{2}
      
            \label{st\stat}

\section{Введение}

В настоящее время в области телекоммуникаций все более востребованными становятся 
информационные технологии, основанные на использовании математических моделей и численных 
методов физики плазмы. Поэтому особенно актуальным является решение разнообразных задач анализа 
поведения плазмы, включающих в себя формирование новых моделей и методов их исследования. 
Помимо этого, в разработке телекоммуникационного оборудования эффективно используются 
собственно физические свойства плазмы. В~частности, изготовлена антенна, работа которой основана 
на газовом разряде низкотемпературной плазмы~[1], интенсивно ведутся разработки по созданию и 
усовершенствованию источников бесперебойного питания на основе плазменных элементов~[2, 3]. 
      
      Одним из наиболее перспективных направлений для построения систем оптической 
беспроводной связи является использование лазеров~\cite{4-k, 5-k}. В~этой связи большое внимание 
уделяется использованию плазмы при разработке импульсных сильноточных коммутаторов~\cite{6-k}, 
так как практическое применение подобных разработок требует повышения уровня надежности и 
быстродействия лазерных систем.
      
      Исследования низкотемпературной плазмы также связаны с разработками в области дальней 
космической связи, так как моделирование процессов взаимодействия заряженного тела с верхними 
слоями атмосферы позволяет предлагать способы улучшения существующих систем радиосвязи с 
космическими летательными аппаратами~\cite{7-k}. 
      
      Наряду с этим актуальными также являются задачи диагностики плазмы, поскольку перспективы 
ее использования в области телекоммуникаций после более полного изучения физических свойств 
могут значительно расшириться. 

Для диагностики плазмы применяют зондовые методы исследования~[8--11]. Эти методы относятся к 
классу контактных методов; как следствие, возникает сложность в исследовании пристеночной области 
вблизи зонда, которая характеризуется достаточно сложным распределением потенциала и функциями 
распределения, отличными от максвелловских. 

Данная работа посвящена исследованию переходного режима обтекания заряженного тела плазмой. Для 
переходного режима выполняется следующее условие: длина свободного пробега иона до столкновения 
с нейтральным атомом или другим ионом невелика по сравнению с характерными размерами тела. 
В~этом случае возникает необходимость учета столкновений заряженных частиц с нейтральными 
атомами и кулоновских столкновений. В~работах~\cite{10-k, 11-k} подробно рассмотрена модель с 
учетом столкновений заряженных частиц с нейтральными атомами. В~настоящей статье представлена 
теоретическая модель, описывающая влияния ион-ионных и ион-элек\-т\-рон\-ных столкновений на 
измеряемые характеристики плазмы, что ранее детально не исследовалось.
      
      В~рамках данной работы предлагается модель, описывающая динамику сильноионизованной 
плазмы с учетом кулоновских столкновений. Эта модель учитывает такие процессы взаимодействия, 
как перенос частиц и столкновения между заряженными частицами типа <<ион--ион>> и 
      <<ион--электрон>> под влиянием макроскопического электрического поля. Перечисленные 
процессы описываются самосогласованной системой уравнений, включающей уравнение 
      Фок\-ке\-ра--План\-ка и уравнение Пуассона~[12].
      
      Вычислительная модель задачи строится на основе двух методов: метода статистических 
испытаний Мон\-те-Кар\-ло и композиции метода крупных частиц и метода расщепления. Приведены 
результаты численного моделирования, полученные с использованием вышеперечисленных методов.

\vspace*{-4pt}

\section{Постановка задачи}

\vspace*{-2pt}

Рассматривается следующая физическая постановка зондовой задачи~[11]. В~невозмущенную 
бесконечно протяженную плазму, состоящую из электронов и однозарядных ионов, внесена большая\linebreak 
заряженная до потенциала $\varphi_p$ плоскость. Плоскость, расположенная поперек потока плазмы, 
является идеально поглощающей для электронов. Ионы при ударе о плоскость нейтрализуются. 
Предполагается, что частицы в плазме движутся под действием внешнего электрического поля, 
магнитное поле отсутствует. Концентрации ионов $n_{i\infty}$ и электронов $n_{e\infty}$, а также 
температуры данных час\-тиц~$T_{i\infty}$ 
и~$T_{e\infty}$ в невозмущенной плазме заданы. За начальные 
функции распределения обоих типов час\-тиц принимаются функции распределения Максвелла. 
      
      Требуется с учетом столкновений между заряженными частицами найти напряженность 
самосогласованного электрического поля $\vec{E}(\vec{r},t)$, функции распределения однозарядных 
ионов $f_i(\vec{r}, \vec{v}, t)$ и электронов $f_e(\vec{r}, \vec{v}, t)$, 
а также их моменты (плотности 
токов ионов и электронов  $j_i(\vec{r},t)\hm
=q\int f_i(\vec{r}, \vec{v}, t)\vec{v}\,d\vec{v}$, $j_e(\vec{r},t) 
\hm={\sf e}\int f_e(\vec{r},\vec{v},t)\vec{v}\,d\vec{v}$, где $q=Z_i{\sf e}$, $Z_i=1$~--- заряд иона, ${\sf 
e}$~--- заряд электрона; концентрации ионов и электронов $n_i(\vec{r},t)\hm=\int 
f_i(\vec{r},\vec{v},t)\,d\vec{v}$, $n_e(\vec{r},t)\hm=\int f_e(\vec{r},\vec{v}, t)\,d\vec{v}$). 
Поведение частиц во 
времени~$t$ характеризуется ра\-ди\-ус-век\-то\-ром~$\vec{r}$ и вектором скорости~$\vec{v}$.
      
      Математическая модель, соответствующая данной физической постановке задачи, имеет 
вид~\cite{11-k, 13-k}:

\noindent
      \begin{equation}
      \left.
      \begin{array}{c}
      \fr{\partial f_\alpha (\vec{r},\vec{v},t)}{\partial t}+
      \vec{v}\fr{\partial f_\alpha (\vec{r},\vec{v},t)}{ 
\partial \vec{r}}+
\fr{\vec{F}_\alpha(\vec{r},t)}{m_\alpha}\times{}\\[4pt]
{}\times\fr{\partial f_\alpha(\vec{r},\vec{v},t)}{ \partial 
\vec{v}}=
\left(\fr{\partial f_\alpha(\vec{r},\vec{v},t)}{ \partial t}\right)_{\mathrm{с}}+S_\alpha 
(\vec{r},\vec{v},t)\,;\\[6pt]
      \Delta\varphi(\vec{r},t)=-\fr{{\sf e}}{\varepsilon_0}\left( n_i(\vec{r},t)-n_e(\vec{r},t)\right)\,;\\[6pt]
      \vec{E}(\vec{r},t)=-\nabla \varphi(\vec{r},t)\,.
      \end{array}\!\!
      \right\}\!\!
      \label{e1-k}
      \end{equation}
Здесь первое уравнение~--- уравнение Фок\-ке\-ра--План\-ка для частиц сорта~$\alpha$ ($\alpha=i,e$), 
второе~--- уравнение Пуассона для самосогласованного электрического поля; 
$f_\alpha(\vec{r},\vec{v},t)$~--- функция\linebreak
распределения час\-тиц сорта~$\alpha$; $(\partial 
f_\alpha(\vec{r},\vec{v},t)/\partial t)_{\mathrm{с}}$~--- 
оператор столкновений Фок\-ке\-ра--План\-ка; 
функция~$S_\alpha(\vec{r},\vec{v},t)$ описывает источники или стоки\linebreak
 час\-тиц; 
$\vec{F}_\alpha(\vec{r},t)=q_\alpha\vec{E}(\vec{r},t)$, где $\vec{E}(\vec{r},t)$~--- напряженность 
самосогласованного электрического поля, 
$$
q_\alpha =
\begin{cases}
-{\sf e}\,, & \alpha=e\,,\\
{\sf e}\,, & \alpha=i\,;
\end{cases}
$$
$\varphi(\vec{r},t)$~--- потенциал самосогласованного электрического поля; $n_\alpha(\vec{r},t)$ ($\alpha 
\hm=i,e$)~--- концентрация частиц сорта~$\alpha$; $m_\alpha$~--- масса частицы сорта~$\alpha$; 
$\varepsilon_0$~--- электрическая постоянная. 

Оператор столкновений Фок\-ке\-ра--План\-ка имеет вид~\cite{13-k, 14-k}
\begin{multline*}
\fr{1}{\Gamma_\alpha}\left( \fr{\partial f_\alpha}{\partial t}\right)_{\mathrm{с}} 
=\fr{1}{2}\,\nabla_v\nabla_v:\left(f_\alpha\nabla_v\nabla_vg_\alpha(\vec{r},\vec{v},t)\right)-{}\\
{}-
\nabla_v\cdot\left(f_\alpha\nabla_v h_\alpha\right)\,,
\end{multline*}
где $\nabla_v\nabla_v g_\alpha(\vec{r},\vec{v},t)$~--- ковариантная тензорная производная второго ранга, 
знак двоеточия ($:$) обозначает операцию двойного суммирования:
\begin{gather*}
\Gamma_\alpha=\fr{Z_\alpha^4 {\sf e}^4}{4\pi \varepsilon_0^2 m^2_\alpha}\,\ln D_\alpha\,;
\\
D_\alpha =\fr{12\pi\varepsilon_0 kT_{\alpha\infty}}{Z_\alpha^2 {\sf e}^2}\left( \fr{\varepsilon_0 k 
T_{e\infty}}{n_{e\infty} {\sf e}^2}\right)^{1/2}\,;\\
g_\alpha (\vec{r},\vec{v},t)=\sum\limits_{b=i,e}\left( \fr{Z_b}{Z_\alpha}\right) \int f_b 
(\vec{r},{\vec{v}}^{\,\prime},t)\left\vert \vec{v}-{\vec{v}}^{\,\prime}\right\vert\,d\vec{v}^{\,\prime}\,;\\
h_\alpha (\vec{r},\vec{v},t)=\sum\limits_{b=i,e} \fr{m_\alpha+m_b}{m_b} 
\left(\fr{Z_b}{Z_\alpha}\right)
\int
\fr{f_b(\vec{r},{\vec{v}}^{\,\prime}, t)}{\vert \vec{v}-{\vec{v}}^{\,\prime}\vert}
\,d{\vec{v}}^{\,\prime}\,;\\
Z_\alpha =1\,, \quad \alpha=i,e\,.
\end{gather*}
 
К системе уравнений~(\ref{e1-k}) необходимо добавить начальные и краевые условия:
\begin{equation}
\!\left.
\begin{array}{rrl}
t=0:\ & f_\alpha(\vec{r},\vec{v},0)&=f_\alpha^{\mathrm{maksv}}\,,\enskip \alpha=i,e;\\[9pt]
\vec{r}\in \Omega_p:\ & f_\alpha(\vec{r},\vec{v},t)\big\vert_{\vec{r}\in\Omega_p}&=0\,,\enskip \alpha=i,e\,;\\[9pt]
&\varphi(\vec{r},t)\big\vert_{\vec{r}\in\Omega_p}&=\varphi_p\,;\\[9pt]
\vec{r}\in\Omega_\infty:\ & 
f_\alpha(\vec{r},\vec{v},t)\big\vert_{\vec{r}\in\Omega_\infty}&= %{}\\[9pt]
f_\alpha^{\mathrm{maksv}}\,,\enskip \alpha=i,e\,;\\[9pt]
&\varphi(\vec{r},t)\big\vert_{\vec{r}\in\Omega_\infty}&=0\,,
\end{array}\!\!
\right\}\!\!\!\!
\label{e2-k}
\end{equation}
    где 
    
    \noindent
    \begin{multline*}
    f_\alpha^{\mathrm{maksv}}=n_{\alpha\infty}\left(\fr{m_\alpha}{2k\pi T_{\alpha\infty}}\right)^{3/2}\times{}\\
    {}\times
    \exp\left( -
\fr{m_\alpha}{2kT_{\alpha\infty}}\left\vert\vec{v}-\vec{v}_\infty\right\vert^2\right)\,,
\enskip \alpha=i, e\,;
\end{multline*} 
$\Omega_p$ и $\Omega_\infty$~--- множество радиус-векторов час\-тиц, концы которых принадлежат плоскости зонда и 
границе возмущенной зоны соответственно.

Для решения поставленной задачи введем декартову систему координат таким образом, чтобы 
заряженная плоскость совпала с плоскостью~$0xz$. Тогда положение частицы в пространстве будет 
определяться координатами $x,y,z$, а скорость~--- координатами $v_x, v_y, v_z$. В~силу того что 
плоскость является бесконечно большой в сравнении с характерным размером задачи, функции 
распределения частиц будут зависеть только от переменных $y, v_y, t$.

Поставленную задачу предлагается решать независимо двумя методами. Первый метод основывается на 
методе статистических испытаний Мон\-те-Кар\-ло, второй метод является композицией метода 
расщепления и метода крупных частиц.

\section{Применение метода Монте-Карло}

Запишем самосогласованную систему уравнений~(\ref{e1-k}) и~(\ref{e2-k}) в декартовой системе 
координат с учетом сделанных предположений:
\begin{equation}
\left.
\begin{array}{l}
\fr{\partial f_\alpha}{\partial t}+
v_y\fr{\partial f_\alpha}{\partial y}+\fr{F_y^\alpha}{m_\alpha}\,\fr{\partial 
f_\alpha}{\partial v_y}=\fr{1}{2}\,\fr{\partial^2 }{\partial [v_y]^2}\times{}\\
{}\times \left( 
f_\alpha\fr{\partial^2 g_\alpha  }{\partial [v_y]^2}\right) -
\fr{\partial}{\partial v_y}\left( f_\alpha\fr{\partial h_\alpha}{\partial v_y}\right)\,,
\enskip \alpha=i,e\,;\\[6pt]
    \fr{\partial^2\varphi}{\partial y^2} =-\fr{{\sf e}}{\varepsilon_0}\left(n_i-n_e\right)\,;
    \enskip E_y=-
\fr{\partial\varphi}{\partial y}\,;\\[6pt]
\hspace*{3.1mm}    t=0:\  \hspace*{2.6mm}f_\alpha(y,v_y,0)=f_\alpha^{\mathrm{maksv}}\,,\ \alpha=i,e\,;\\[9pt]
\hspace*{2.9mm} y=0:\ \hspace*{2.8mm}f_\alpha(0,v_y,t)=0\,,\ \alpha=i,e\,;\\[9pt]
\hspace*{24.3mm}\varphi(0,t)=\varphi_p\,;\\[9pt]
y=y_\infty:\ f_\alpha(y_\infty, v_y, t)=f_\alpha^{\mathrm{maksv}}\,,\ \alpha=i,e\,;\\[9pt]
\hspace*{21.5mm}\varphi(y_\infty, t)=0\,.
\end{array}
\right \}
\label{e3-k}
\end{equation}

В полученной системе уравнений~(\ref{e3-k}) перейдем к безразмерным величинам, применив 
соотношение $X=M_X \hat{X}$, где $M_X$~--- масштаб размерной величины~$X$, $\hat{X}$~--- 
безразмерная величина~$X$. В~качестве используемых масштабов были взяты следующие: радиус 
Дебая, скорость теплового движения частиц, концентрация частиц в невозмущенной плазме, потенциал, 
возникающий при разделении зарядов в дебаевской сфере, и производные от них величины.

Система безразмерных уравнений имеет следующий вид:
%\noindent
\begin{equation}
\left.
\begin{array}{l}
\fr{\partial 
\hat{f}_\alpha}{\partial\hat{t}}+A_\alpha\fr{\partial\hat{f}_\alpha}{\partial\hat{y}}+
B_\alpha\hat{E}_y\fr{\partial\hat{f}_\alpha}{\partial \hat{v}_y}={}\\
\!{}=
\fr{\partial^2}{\partial[\hat{v}_y]^2}\left(D_\alpha 
\hat{f}_\alpha\right)-\fr{\partial}{\partial\hat{v}_y}\left(K_\alpha \hat{f}_\alpha\right),\enskip 
\alpha=i,e;\\[9pt]
\fr{\partial^2\hat{\varphi}}{\partial\hat{y}^2}=-\left(\hat{n}_i-\hat{n}_e\right)\,;\enskip \hat{e}_y=-
\fr{\partial\hat\varphi}{\partial\hat{y}}\,;\\[9pt]
\hspace*{3.1mm}\hat{t}=0:\ \hspace*{2.6mm}\hat{f}_\alpha(\hat{y},\hat{v}_y,0)=\hat{f}_\alpha^{\mathrm{maksv}}\,,\enskip \alpha-i,e\,;\\[9pt]
\hspace*{2.9mm}\hat{y}=0:\ \hspace*{2.8mm}\hat{f}_\alpha(0,\hat{v}_y,\hat{t})=0\,,\enskip \alpha=i,e\,;\\[9pt]
\hspace*{24.3mm}\hat\varphi(0,\hat{t})=\hat{\varphi}_p\,;\\[9pt]
\hat{y}=\hat{y}_\infty:\ \hat{f}_\alpha(\hat{y}_\infty, \hat{v}_y, \hat{t})=\hat{f}^{\mathrm{maksv}}_\alpha\,,\enskip 
\alpha=i,e\,;\\[9pt]
\hspace*{21.5mm}\hat\varphi(\hat{y}_\infty,\hat{t})=0\,.
\end{array}
\right\}
\label{e4-k}
\end{equation}
Здесь 

\vspace*{-2pt}

\noindent
\begin{gather*}
A_\alpha=\sqrt{\delta_\alpha }\,\hat{v}_y\,;\enskip 
B_\alpha=\sqrt{\delta_\alpha}\,\fr{z_\alpha}{2\varepsilon_\alpha}\,;\\
\delta_\alpha=\fr{\varepsilon_\alpha}{\mu_\alpha}\,;\enskip 
\varepsilon_\alpha=\fr{T_{\alpha\infty}}{T_{i\infty}}\,;\\
\mu_\alpha=\fr{m_\alpha}{m_i}\,;\enskip 
D_\alpha=A_g^\alpha\fr{\partial^2\hat{g}_\alpha}{\partial  [\hat{v}_y]^2}\,;\\
K_\alpha=A_h^\alpha \fr{\partial \hat{h}_\alpha}{\partial \hat{v}_y}\,,\enskip \alpha=i,e\,,
\end{gather*}
где $A_g^\alpha$ и $A_h^\alpha$~--- коэффициенты, определяемые характерными параметрами 
задачи~\cite{15-k}.

Поиск решения самосогласованной системы уравнений~(\ref{e4-k}) осуществляется по следующей 
схе-\linebreak ме. Вначале находятся значения напряженности\linebreak
 электрического поля по значениям потенциала, 
полученным из граничной задачи для уравнения Пуассона. Далее, используя найденные значения 
напряженности, решается уравнение Фок\-ке\-ра--План\-ка путем перехода к стохастическому 
дифференциальному уравнению (СДУ) Ито:

\noindent
\begin{multline*}
d\Theta_\alpha(\hat{t}) = a_\alpha \left(\hat{t},\Theta_\alpha(\hat{t})\right)+{}\\
{}+\sigma\left(
\hat{t},\Theta_\alpha(\hat{t})\right)\,dW(\hat{t})\,,\quad \alpha=i,e\,,
%\label{e5-k}
\end{multline*}
где 

\noindent
\begin{align*}
\Theta_\alpha(\hat{t})&=\begin{bmatrix}
\hat{y}(\hat{t})\\ \hat{v}_y(\hat{t})
\end{bmatrix}\,;\\
a_\alpha\left(\hat{t},\Theta_\alpha(\hat{t})\right)&=\begin{bmatrix}
-A_\alpha\\ -K_\alpha -B_\alpha \hat{E}_y
\end{bmatrix}\,;\\
\sigma_\alpha\left(\hat{t},\Theta_\alpha(\hat{t})\right)\sigma_\alpha^{\mathrm{T}}\left( 
\hat{t},\Theta_\alpha(\hat{t})\right)&=D_\alpha\,,\enskip \alpha=i,e\,;
\end{align*} 
$W(\hat{t})$~--- стандартный винеровский случайный процесс.
\pagebreak

Для нахождения значений вектора состояния~$\Theta_\alpha(\hat{t})$ применим явную разностную 
схему стохастического метода Эйлера~\cite{16-k}:
\begin{multline*}
\Theta_\alpha^{n+1}=\Theta_\alpha^n +h_\tau a_\alpha \left( \hat{t}_n, \Theta_\alpha^n\right)+\sigma_\alpha 
\left( \hat{t}_n, \Theta_\alpha^n\right)\Delta W_n\,,\\ 
n=0,\ldots , N\,,\ \alpha=i,e\,,
%\label{e6-k}
\end{multline*}
где $\Theta_\alpha^n$, $n=0,\ldots , N$,~--- приближенное значение вектора 
состояния~$\Theta_\alpha(\hat{t})$, $\alpha=i,e$, в момент времени $\hat{t}\hm=\hat{t}_n$, 
$\hat{t}_n\hm=n h_\tau$, $n=0,\ldots , N$; $h_\tau$~--- достаточно малый шаг интегрирования; $\Delta 
W_n$, $n=0,\ldots ,N$,~--- величина приращения винеровского процесса~$W(\hat{t})$ на отрезке $\left[ 
\hat{t}_n,\,\hat{t}_{n+1}\right]$, по определению независимая от~$\Theta_\alpha^0$, 
$\Delta W_0,\ldots , 
\Delta W_{n-1}$: $\Delta W_n\hm=W(\hat{t}_{n-1})\hm-W(\hat{t}_n)$; $\Delta W_n\hm\sim N(0,\,h_\tau)$, 
т.\,е.\ $\Delta W_n$ представляют собой гауссовские случайные величины с нулевыми математическими 
ожиданиями и дисперсиями, равными шагу интегрирования; $\Theta_\alpha^0$~--- значение вектора 
состояния $\Theta_\alpha(\hat{t})$, $\alpha\hm=i,e$, в момент времени $\hat{t}=0$, 
$\Theta_\alpha^0\hm\sim \hat{f}_\alpha^{\mathrm{maksv}}$. 

Частные производные $\partial^2\hat{g}_\alpha/\partial[\hat{v}_y]^2$ и $\partial \hat{h}_\alpha/\partial 
\hat{v}_y$, являющиеся составляющими матрицы $\sigma_\alpha (\hat{t}_n, 
\Theta_\alpha^n)\sigma_\alpha^{\mathrm{T}}(\hat{t}_n,\Theta_\alpha^n)$ и вектора $a_\alpha(\hat{t}_n, 
\Theta_\alpha^n)$ соответственно, аппроксимируются со вторым порядком точности на трехточечном 
шаблоне на основе значений~$\hat{g}_\alpha$ и~$\hat{h}_\alpha$~\cite{17-k}.
      
      В выражения для функций~$\hat{g}_\alpha$ и~$\hat{h}_\alpha$ входят интегралы, которые 
вычисляются методом Мон\-те-Кар\-ло с использованием набора значений скоростной компоненты 
вектора состояния~$\hat{v}_y$, полученных из решения СДУ Ито:
      \begin{equation*}
      \int \hat{f}_\alpha \left\vert \hat{v}_y-
\hat{v}_y^\prime\right\vert\,dv_y^\prime=M\left(\zeta\left(\hat{V}_y\right)\right)\,,
\end{equation*}
где
$$
      \zeta\left(\hat{V}_y\right)=\left\vert \hat{v}_y-\hat{V}_y\right\vert\,,\enskip \hat{V}_y\sim 
\hat{f}_\alpha\,.
  $$
      
      Для вычисления напряженности самосогласованного электрического поля $\hat{E}_y=-
\partial\hat{\varphi}/\partial\hat{y}$, входящей в вектор $a_\alpha(\hat{t}_n, \Theta_\alpha^n)$, необходимо 
аналогично аппроксимировать со вторым порядком точности производную 
$\partial\hat{\varphi}/\partial\hat{y}$ на трехточечном шаблоне с использованием значений 
потенциала~$\hat{\varphi}$~\cite{17-k}. Значения потенциала~$\hat\varphi$ находятся из решения 
уравнения Пуассона. 
      
      Граничную задачу для уравнения Пуассона 
      \begin{align*}
      \fr{\partial^2 \hat\varphi}{\partial \hat{y}^2} & = -\left(\hat{n}_i-\hat{n}_e\right)\,;\\
      \hat{\varphi}\big|_{\hat{y}=0} &=\hat{\varphi}_p\,;\\
      \hat{\varphi}\big|_{\hat{y}_\infty=0} &=0
      \end{align*}
    предлагается решать путем перехода к конечно-разностной системе с последующим ее решением 
методом прогонки~\cite{17-k}:

\noindent
\begin{gather*}
\hat{\varphi}^n_{l-1}+2\hat{\varphi}_l^n+\hat{\varphi}^n_{l+1}=
h_y\hat{\delta}_l^n\,,\enskip l=1,\ldots , 
N_y\,;\\
\hat{\delta}_l^n=-\left( \hat{n}^n_{i,l}-\hat{n}^n_{e,l}\right)\,;\enskip 
\hat{\varphi}_0=\hat{\varphi}_p\,;\enskip \hat{\varphi}_{N_y}=0\,,
\end{gather*}
где $N_y$~--- число шагов по переменной~$\hat{y}$, $h_y$~--- величина шагов разбиения по~$\hat{y}$. 
      
      Концентрации $\hat{n}_\alpha$, $\alpha=i,e$, и плотности токов частиц на зонд~$\hat{f}_\alpha$, 
$\alpha=i,e$, вычисляются согласно описанному выше методу Мон\-те-Карло.

\section{Применение метода расщепления и~метода крупных~частиц}

Решение задачи в данном случае предлагается начать с записи правой части уравнения 
Фок\-ке\-ра--План\-ка в декартовой системе координат в виде:
$$
\mathbf{Q} f_\alpha = \fr{1}{2}\,\fr{\partial^2 f_\alpha}{\partial [v_y]^2}\,\fr{\partial^2 g_\alpha}{\partial 
[v_y]^2}+\fr{\partial f_\alpha}{\partial v_y}\,\fr{\partial C_\alpha}{\partial v_y}+H_\alpha\,,\enskip 
\alpha=i,e\,,
$$  
где 
\begin{align*}
C_\alpha(\vec{r},\vec{v},t)&=
\begin{cases}
\fr{1-\gamma}{Z_i^2}\int\fr{f_e(\vec{r},{\vec{v}}^{\,\prime},t)}{|\vec{v}-{\vec{v}}^{\,\prime} |}\,d{\vec{v}}^{\,\prime}\,, 
&\alpha=i\,;\\[9pt]
\fr{Z_i^2(\gamma-1)}{\gamma}\int \fr{f_i(\vec{r},{\vec{v}}^{\,\prime}, t)}
{|\vec{v}-{\vec{v}}^{\,\prime} 
|}\,d{\vec{v}}^{\,\prime}\,, &\alpha=e\,;
\end{cases} 
\\
H_\alpha&=
\begin{cases}
4\pi \left( \fr{\gamma f_e}{Z_i^2}+f_i\right)f_i\,, & \alpha=i\,;\\[9pt]
4\pi\left(\fr{Z_i^2 f_i}{\gamma}+f_e\right)f_e\,, &\alpha=e\,.
\end{cases}
\end{align*}
Тогда при переходе к безразмерным величинам (см.\ разд.~3) система~(\ref{e1-k}) запишется 
следующим образом:
      \begin{equation}
      \left.
\!\!\begin{array}{l}
      \fr{\partial 
\hat{f}_\alpha}{\partial\hat{t}}+A_\alpha\fr{\partial\hat{f}_\alpha}{\partial\hat{y}}+
B_\alpha  \hat{E}_y
\fr{\partial\hat{f}_\alpha}{\partial\hat{v}_\alpha}=\tilde{\mathbf{Q}}\hat{f}_\alpha\,,\enskip 
\alpha=i,e;\\[9pt]
      \fr{\partial^2\hat{\varphi}}{\partial\hat{y}^2}=-\left( \hat{n}_i-\hat{n}_e\right)\,,\enskip \hat{E}_y=-
\fr{\partial\hat\varphi}{\partial\hat{y}}\,,\\[9pt]
\hspace*{3.1mm}\hat{t}=0:\ \hspace*{2.6mm}\hat{f}_\alpha(\hat{y},\hat{v}_y, 0)=\hat{f}_\alpha^{\mathrm{maksv}}\,,\enskip \alpha=i,e\,,\\[9pt]
\hspace*{2.9mm} \hat{y}=0:\ \hspace*{2.8mm}\hat{f}_\alpha(0,\hat{v}_y,\hat{t})=0\,,\enskip \alpha=i,e\,;\\[9pt]
\hspace*{24.3mm}\hat\varphi(0,\hat{t})=\hat{\varphi}_p\,;\\[9pt]
      \hat{y}=\hat{y}_\infty:\ \hat{f}_\alpha(\hat{y}_\infty, 
\hat{v}_y,\hat{t})=\hat{f}_\alpha^{\mathrm{maksv}}\,,\enskip \alpha=i,e\,;\\[9pt]
\hspace*{21.5mm}\hat{\varphi}(\hat{y}_\infty,\hat{t})=0\,,\\[9pt]
    \end{array}
\right\}\!\!
\label{e7-k}
\end{equation}
где 
\begin{gather*}
\tilde{\mathbf{Q}} \hat{f}_\alpha=D_\alpha\fr{\partial^2\hat{f}_\alpha}{\partial 
[\hat{v}_y]^2}+K_\alpha\fr{\partial\hat{f}_\alpha}{\partial\hat{v}_y}+H_\alpha\,;\\
D_\alpha=A_g^\alpha\fr{\partial^2\hat{g}_\alpha}{\partial [\hat{v}_y]^2}\,;\enskip 
K_\alpha=A_h^\alpha \fr{\partial \hat{h}_\alpha}{\partial\hat{v}_y}\,,\ \alpha=i,e\,.
\end{gather*}

Для решения системы уравнений~(\ref{e7-k}) применяется модификация метода 
расщепления~\cite{17-k}, согласно которой исходная задача разбивается на две вспомогательные. Такое 
разбиение можно осуществить, переписав уравнение Фок\-ке\-ра--План\-ка в следующем виде:
$$
\fr{\partial\hat{f}_\alpha}{\partial\hat{t}} =
\tilde{\mathbf{Q}}_1\hat{f}_\alpha+\tilde{\mathbf{Q}}_2\hat{f}_\alpha\,,
$$
где 
\begin{align*}
\tilde{\mathbf{Q}}_1\hat{f}_\alpha &=-
\left(A_\alpha\fr{\partial\hat{f}_\alpha}{\partial\hat{y}}+
B_\alpha\fr{\partial\hat{f}_\alpha}{\partial\hat{y}}
\right)\,;\\
\tilde{\mathbf{Q}}_2\hat{f}_\alpha 
&=\left(D_\alpha\fr{\partial^2\hat{f}_\alpha}{\partial[\hat{v}_y]^2}+K_\alpha\fr{\partial 
\hat{f}_\alpha}{\partial\hat{v}_y}+H_\alpha\right)\,.
\end{align*}

      Правая часть уравнения Фок\-ке\-ра--План\-ка представляет собой сумму двух операторов, 
первый из которых отвечает за перенос частиц, второй~--- за столкновения заряженных частиц. 
В~результате образуются следующие задачи, которые решаются последовательно:
      \begin{itemize}
\item первая задача:
\begin{align*}
&\fr{\partial w_\alpha(\hat{y},\hat{v}_y,\hat{t})}{\partial\hat{t}} =\mathbf{Q}_1 
w_\alpha(\hat{y},\hat{v}_y,\hat{t})\,,\enskip \alpha=i,e\,;\\[9pt]
&\fr{\partial^2\hat\varphi}{\partial\hat{y}^2}=-\left(\hat{n}_i-\hat{n}_e\right)\,;\enskip
\hat{E}_y=-
\fr{\partial\hat\varphi}{\partial\hat{y}}\,;\\[9pt]
&w_\alpha(\hat{y},\hat{v}_y,\hat{t}^n)=\hat{f}_\alpha(\hat{y},\hat{v}_y,\hat{t}^n)\,,\enskip n=0,\ldots ,N-
1\,;\\[9pt]
&\hspace{2.9mm}\hat{y}=0:\ \hspace*{2.9mm}w_\alpha(0,\hat{v}_y,\hat{t})=0\,,\enskip \alpha=i,e\,;\\[9pt]
&\hspace*{25.1mm}\hat\varphi(0,\hat{t})=\hat{\varphi}_p\,;\\[9pt]
&\hat{y}=\hat{y}_\infty:\ w_\alpha(\hat{y}_\infty, \hat{v}_y, \hat{t})=
\hat{f}_\alpha^{\mathrm{maksv}}\,,\enskip 
\alpha=i,e\,;\\[9pt]
&\hspace*{22.5mm}\hat\varphi(\hat{y}_\infty,\hat{t})=0\,;
\end{align*}
\item вторая задача:
\begin{align*}
\!\!\!\!\!\!\!\fr{\partial s_\alpha(\hat{y},\hat{v}_y,\hat{t})}{\partial \hat{t}} &=\mathbf{Q}_2 
s_\alpha(\hat{y},\hat{v}_y,\hat{t})\,, & \alpha&=i,e\,;\\
\!\!\!\!\!\!\!s_\alpha (\hat{y},\hat{v}_y,\hat{t}^n) &=w_\alpha (\hat{y},\hat{v}_y, \hat{t}^{n+1}),& n&=0,\ldots ,N-
1.
\end{align*}
\end{itemize}

Первая задача представляет собой систему безразмерных уравнений Вла\-со\-ва--Пуас\-со\-на. Для ее 
решения применяется метод крупных частиц~\cite{18-k}. Согласно этому методу решение задачи 
осуществляется путем расщепления на два этапа: на первом этапе не учитываются конвективные члены 
и решение получается обычным интегрированием на неподвижной эйлеровой сетке, а на втором этапе 
рассматривается система, которая описывает перенос частиц в лагранжевой системе координат. Кроме 
того, на первом этапе необходимо решить уравнение Пуассона для получения значений потенциала 
самосогласованного электрического поля. Для этого применяется метод, описанный в разд.~3. 

Вторая задача решается путем перехода к ко\-неч\-но-раз\-ност\-ной сис\-те\-ме. При этом частные 
производные $\partial^2\hat{g}_\alpha/\partial[\hat{v}_y]^2$ и $\partial\hat{h}_\alpha/\partial\hat{v}_y$ 
аппроксимируются со вторым порядком точности с использованием трехточечного шаблона, а 
производная $\partial s_\alpha/\partial\hat{t}$ аппроксимируется на двухточечном шаблоне с первым 
порядком точности~\cite{16-k}. К~полученной системе разностных уравнений предлагается применить 
один из классических методов решения систем линейных уравнений, например метод 
Гаусса~\cite{19-k}.
      
      Решением первой задачи является функция $w_\alpha(\hat{y}, \hat{v}_y, \hat{t}^n)$, 
$n\hm=0,\ldots ,N$, , которая дает начальное условие для второй задачи. Решая вторую задачу, находим 
функцию $s_\alpha(\hat{y},\hat{v}_y,\hat{t}^n)\hm=\hat{f}_\alpha(\hat{y},\hat{v}_y,\hat{t}^n)$, 
$n=1,\ldots ,N$, $\alpha=i,e$, которая определяет решение $\hat{f}_\alpha(\hat{y},\hat{v}_y,\hat{t}^n)$, 
$\alpha=i,e$, исходной системы~(\ref{e7-k}) для рассматриваемых моментов времени $n=1,\ldots ,N$.

Моменты функций распределения $\hat{f}_\alpha$, $\alpha=i,e$, находятся с помощью методов 
численного интегрирования, например метода трапеций~\cite{19-k}.

\section{Результаты численного моделирования}

Для двух описанных выше методов реализованы две отдельные программы в среде {Matlab~7.0}. 
Эти программы позволяют по заданным значениям концентраций и температур частиц $n_{i\infty}$, 
$n_{e\infty}$, $T_{i\infty}$ и~$T_{e\infty}$ в невозмущенной плазме, а также потенциала~$\varphi_p$, 
подаваемого на зонд, изучить эволюцию во времени плотностей тока частиц~$j_i$ и~$j_e$, концентраций 
частиц~$n_i$  и~$n_e$ в произвольной точке пространства в возмущенной зоне, а также динамику 
изменения напряженности~$E_y$ самосогласованного электрического поля во времени и пространстве.

С использованием разработанных программ проведены серии расчетных экспериментов, в которых 
значение концентраций варьировалось в пределах $n_{i\infty} \hm = n_{e\infty}\hm =10^{18}\div 
10^{22}$~м$^{-3}$. Значение температур было выбрано неизменным и равным $T_{i\infty}\hm = 
T_{e\infty}\hm=3000$~K, а значения потенциала, подаваемого на зонд, изменялись в пределах 
$\varphi_p\hm=0\div 2{,}6$~В.

На рис.~1  и~2 приведены графики изменения напряженности самосогласованного электрического
 поля (см.\ рис.~1) и плотности токов ионов (см.\linebreak\vspace*{-12pt}

\pagebreak

\end{multicols}

\begin{figure} %fig1
\vspace*{1pt}
\begin{center}
\mbox{%
\epsfxsize=162.594mm
\epsfbox{kud-1.eps}
}
\end{center}
\vspace*{-9pt}
\Caption{Динамика изменения плотности тока ионов во времени в фиксированной точке возмущенной 
зоны для значений потенциала: \textit{1}~--- $\varphi_p=-6$; 
\textit{2}~--- $\varphi_p=-16$; \textit{3}~--- $\varphi_p=- 30$ 
в случае применения методов Монте-Карло~(\textit{а}) 
и крупных частиц~(\textit{б})}
\end{figure}

\begin{figure} %fig2
\vspace*{1pt}
\begin{center}
\mbox{%
\epsfxsize=162.713mm
\epsfbox{kud-2.eps}
}
\end{center}
\vspace*{-9pt}
\Caption{Динамика изменения напряженности электрического поля во времени в фиксированной точке 
возмущенной зоны для значений потенциала: 
\textit{1}~--- $\varphi_p=-6$; \textit{2}~--- $\varphi_p=-16$; 
\textit{3}~--- $\varphi_p=-30$ в случае применения методов Монте-Карло~(\textit{а}) и
крупных частиц~(\textit{б})
}
\end{figure}

\begin{multicols}{2}

\noindent
 рис.~2) во времени в фиксированной точке пространства 
возмущенной зоны в случае применения обоих разработанных алгоритмов.


На основании полученных результатов можно отметить похожее поведение зависимостей 
напряженности электрического поля и плотности тока от времени в двух рассматриваемых случаях. 
Графики кривых сначала убывают, затем начинают возрастать, выходя в некоторый момент 
времени~$t^\prime$ (момент установления) на стационарные значения. 

Одинаковое поведение 
напряженности и плот\-ности тока можно объяснить из следующих соображений: плотность тока ионов в 
данной области пространства равна произведению концентрации ионов на их направленную скорость и 
на заряд иона. Скорость ионов, в свою очередь, зависит от заряда, массы и напряженности 
электрического поля. 
%\columnbreak

При внесении в плазму отрицательно заряженного зонда возникает электрическое поле, которое 
нарушает квазинейтральность плазмы. Для того чтобы компенсировать действие внешнего 
электрического поля, ионы устремляются к зонду, а электроны~--- от зонда. Это приводит к дисбалансу 
концентраций вблизи зонда и, как следствие, к увеличению разности потенциалов; график 
напряженности электрического поля убывает. Вскоре разделение зарядов компенсирует внешнее 
электрическое поле; график выходит на стационарное значение. 

Также можно отметить, что значения 
напряженности электрического поля и плотности тока частиц на зонд в момент установления для двух 
методов совпадают. 

Момент установления~$t^\prime$ зависит от при\-ме\-ня\-емо\-го метода решения. В~случае метода 
Мон\-те-Кар\-ло $t^\prime=3{,}5\div 4$~ед., а для метода крупных частиц совместно с методом 
расщепления $t^\prime\hm=5\div 5{,}5$~ед. Используя ко\-неч\-но-раз\-ност\-ный метод, можно 
получить динамику изменения функций распределения частиц~$f_\alpha$, $\alpha=i,e$, во времени и 
пространстве. Функции распределения позволяют наглядно представить влияние на картину 
распределения частиц вблизи зонда самой поверхности зонда и электрического поля.

\section{Заключение}
      
      В работе найдено решение задачи диагностики плоским зондом сильноионизованной плазмы с 
учетом столкновений заряженных частиц. Разработана математическая модель исследуемого явления, 
описываемая уравнениями Фок\-ке\-ра--План\-ка и Пуассона. Решение получено двумя методами:\linebreak 
статистическим и ко\-неч\-но-раз\-ност\-ным на основе\linebreak сформированных алгоритмов. Приведены 
резуль-\linebreak таты численного моделирования при различных\linebreak характерных параметрах задачи.
 Из  проведенных 
вычислительных экспериментов вытекает, что искомые величины: напряженность 
электрического поля, плотности токов частиц на зонд, концентрации частиц вблизи зонда~--- как по 
характеру зависимости, так и по числовым значениям совпадают. При применении метода 
      Мон\-те-Кар\-ло момент установления наступает быстрее по сравнению с конечно-разностным 
методом, однако конечно-разностный метод позволяет получить более наглядные результаты.

{\small\frenchspacing
{%\baselineskip=10.8pt
\addcontentsline{toc}{section}{Литература}
\begin{thebibliography}{99}

\bibitem{1-k}
\Au{Alexeff I., Anderson T.}
Experimental and theoretical results with plasma antenna~// IEEE Trans. Plasma Sci., 2006. Vol.~34. 
No.\,2. P.~166--172.

\bibitem{2-k}
\Au{Сысун В.\,И.}
Сильноионизованная низкотемпературная плазма в приборах электронной техники: Методы 
исследования, свойства, применение. Дисс. \ldots д-ра физ.-мат. наук в форме науч. докл.: 
01.04.08.~--- Пет\-ро\-за\-водск, 1996.

\bibitem{3-k}
\Au{Тухас В.\,А.}
Методология создания средств измерений и испытаний на устойчивость к кондуктивным помехам~// 
Мат-лы VI Междунар. симп. по электромагнитной совместимости и 
электромагнитной экологии.~--- СПб., 2005. С.~231--234.

\bibitem{4-k}
\Au{Гудзенко Л.\,И., Яковленко С.\,И.}
Плазменные лазеры.~--- М.: Атомиздат, 1978.  256~с.

\bibitem{5-k}
\Au{Звелто О.}
Принципы лазеров.~--- М.: Мир, 1990.  560~с.

\bibitem{6-k}
\Au{Сысун В.\,И., Хромой Ю.\,Д.}
Расширение канала мощного импульсного разряда в парах ртути~// Электронная техника, 1974. 
Сер.~4. Вып.~10. С.~80--85. 

\bibitem{7-k}
\Au{Винклер Дж.\,Р.}
Искусственные пучки частиц в космической плазме.~--- М.: Мир, 1985.  451~с.

\bibitem{8-k}
\Au{Bernstein I.\,B., Rabinowitz I.\,N.}
Theory of electrostatic probes in low-density plasma~// Phys. Fluids, 1959. Vol.~2. No.\,2. P.~112--121. 

\bibitem{9-k}
\Au{Альперт Я.\,Л., Гуревич А.\,В., Питаевский~Л.\,П.}
Искусственные спутники в разреженной плазме.~--- М.: Наука, 1964.  282~с.

\bibitem{10-k}
\Au{Чан П., Тэлбот Л., Турян~К.}
Электрические зонды в неподвижной и движущейся плазме.~--- М.: Мир, 1978.  202~с.

\bibitem{11-k}
\Au{Алексеев Б.\,В., Котельников В.\,А.}
Зондовый метод диагностики плазмы.~--- М.: Энергоатомиздат, 1989.  240~с.

\bibitem{12-k}
\Au{Пантелеев А.\,В., Кудрявцева И.\,А.}
Формирование математической модели двухкомпонентной плазмы с учетом столкновений 
заряженных частиц в случае плоского зонда~// Теоретические вопросы вычислительной техники и 
программного обеспечения: Межвузовский сб. научн. тр.~--- М.: МИРЭА, 2006. С.~11--21.

\bibitem{13-k}
\Au{Олдер Б.}
Вычислительные методы в физике плазмы.~--- М.: Мир, 1974.  111~с.

\bibitem{14-k}
\Au{Montgomery D.\,C., Tidman D.\,A.}
Plasma kinetic theory.~--- New York, 1964. 

\bibitem{15-k}
\Au{Кудрявцева И.\,А., Пантелеев А.\,В.}
Применение метода Мон\-те-Кар\-ло для анализа поведения двухкомпонентной плазмы с учетом 
столкновений между заряженными частицами~// Теоретические вопросы\linebreak
вычислительной техники и 
программного обеспечения: Межвузовский сб. научн. тр.~--- М.: МИРЭА, 2008. С.~122--128. 

\bibitem{16-k}
\Au{Семенов В.\,В., Пантелеев А.\,В., Руденко~Е.\,А., Бор\-та\-ков\-ский~А.\,С.}
Методы описания, анализа и синтеза нелинейных систем управления.~--- М.: МАИ, 1993.  312~с.

\bibitem{17-k}
\Au{Киреев В.\,И., Пантелеев А.\,В.}
Численные методы в примерах и задачах.~--- М.: Высшая школа, 2006.  480~с.

\bibitem{18-k}
\Au{Белоцерковский О.\,М., Давыдов~Ю.\,М.}
Метод крупных частиц в газовой динамике. Вычислительный эксперимент.~--- М.: Наука, 
Физматгиз, 1982.

\label{end\stat}

\bibitem{19-k}
\Au{Вержбицкий В.\,М.}
Основы численных методов.~--- М.: Высшая школа, 2002.  840~с.
 \end{thebibliography}
}
}


\end{multicols}        

\def\stat{kondranin+ushakov}

\def\tit{СИСТЕМА ОБСЛУЖИВАНИЯ С~ОТНОСИТЕЛЬНЫМ ПРИОРИТЕТОМ  И~ПРОФИЛАКТИКАМИ ПРИБОРА$^*$}

\def\titkol{Система обслуживания с~относительным приоритетом  и~профилактиками прибора}

\def\aut{Е.\,С.~Кондранин$^1$,  В.\,Г.~Ушаков$^2$}

\def\autkol{Е.\,С.~Кондранин,  В.\,Г.~Ушаков}

\titel{\tit}{\aut}{\autkol}{\titkol}

\index{Кондранин Е.\,С.}
\index{Ушаков В.\,Г.}
\index{Kondranin E.\,S.}
\index{Ushakov V.\,G.}




{\renewcommand{\thefootnote}{\fnsymbol{footnote}} \footnotetext[1]
{Работа выполнена при финансовой поддержке РФФИ (проект 18-07-00678).}}


\renewcommand{\thefootnote}{\arabic{footnote}}
\footnotetext[1]{Факультет вычислительной математики и~кибернетики Московского государственного 
университета им.\ М.\,В.~Ломоносова, \mbox{ekondranin@yandex.ru}}
\footnotetext[2]{Факультет вычислительной математики и~кибернетики
Московского государственного университета им.\ М.\,В.~Ломоносова;
Институт проб\-лем информатики Федерального исследовательского
центра <<Информатика и~управ\-ле\-ние>> Российской академии наук,
\mbox{vgushakov@mail.ru}}

\vspace*{-10pt}




\Abst{Изучена одноканальная система
массового обслуживания с~двумя типами требований, бесконечным
числом мест для ожидания, гиперэкспоненциальным входящим потоком 
и~профилактиками обслуживающего прибора при освобождении системы.
Тип  требования определяется случайно с~заданными вероятностями 
в~момент его поступления в~систему обслуживания. Требования первого
типа имеют относительный приоритет перед требованиями второго
типа. Найдено нестационарное совместное распределение числа
требований каждого типа в~системе. Профилактики прибора
заключаются в~том, что в~момент освобождения системы от требований
прибор на случайное время с~заданным распределением становится
недоступным для обслуживания. Если за время профилактики поступает
хотя бы одно требование, то начинается нормальное функционирование
системы. Если требования не поступают, то прибор отправляется на
новую профилактику. Такие системы хорошо описывают
функционирование большого числа реальных вычислительных и~информационных систем.}

\KW{гиперэкспоненциальный поток; профилактики
обслуживающего прибора; одноканальная система; относительный
приоритет; длина очереди}

\DOI{10.14357/19922264180405}
  
%\vspace*{4pt}


\vskip 10pt plus 9pt minus 6pt

\thispagestyle{headings}

\begin{multicols}{2}

\label{st\stat}

\section{Введение}

В классической системе массового обслуживания ожидание требований
в очереди связано только с~занятостью обслуживающего прибора. В~то
же время в~реальных системах сам  прибор может пребывать как 
в~активном, так и~в~неактивном состоянии. Такое неактивное
состояние прибора (в~литературе на английском языке используется
термин vacation, а~на русском~--- профилактика или прогулка) может
быть связано со многими причинами. В~част\-ности, сис\-те\-мы
обслуживания с~профилактиками прибора хорошо описывают
функционирование  реальных вычислительных и~информационных систем,
в которых наряду с~основными требованиями имеются второстепенные.
Второстепенные требования всегда присутствуют в~сис\-те\-ме, а~их
обслуживание может проводиться только тогда, когда нет основных,
т.\,е.\ в~фоновом режиме.

С точки зрения самого процесса профилактики прибора существует
несколько ее разновидностей. Во-пер\-вых, могут быть разными
правила, задающие условия начала профилактики: прибор может брать
перерыв только при  полном исчерпании требований в~очереди
(exhaustive service) либо при наличии определенного их числа
(nonexhaustive service). Во-вто\-рых, могут быть разными правила
возвращения прибора в~работу. С~этой точки зрения различают случаи
однократного (single vacation) и~многократного (multiple vacation)
перерыва в~работе. В~первом случае ушедший на профилактику прибор
после ее окончания находится в~рабочем состоянии независимо от
наличия требований в~системе. Во втором случае прибор, не
обнаружив новых требований в~очереди, уходит на новую
профилактику.


В работах~[1--4] можно найти обзор известных результатов, большое
число постановок задач, описание различных приложений и~обширную
библиографию по анализу систем с~профилактиками обслуживающего
прибора.


В настоящей работе исследуется совместное распределение длин
очередей в~нестационарном режиме в~однолинейной системе 
с~ожиданием, гиперэкспоненциальным входящим потоком, двумя типами
требований и~относительным приоритетом. Аналогичная неприоритетная
система обслуживания исследована в~[5].

\vspace*{-6pt}

\section{Описание модели}

Рассматривается однолинейная система массового обслуживания 
с~двумя приоритетными классами требований. Входящий поток~---
гиперэкспоненциальный с~функцией распределения интервалов между
поступлениями требований вида:
\begin{multline*}
A(t)=\sum\limits_{i=1}^kc_i\left(1-e^{-a_it}\right),\enskip t>0,\enskip
a_i>0,\enskip c_i>0,\\
a_i\ne a_j\,,\enskip i\ne j\,,\enskip  \sum\limits_{i=1}^k c_i=1\,.
\end{multline*}

Каждое поступившее требование направляется в~первый класс 
с~вероятностью~$p$ и~во второй класс с~вероятностью $1\hm-p$
независимо от остальных требований. Требования первого класса
обладают относительным приоритетом перед требованиями второго
класса. Длительности обслуживания требований $i$-го приоритетного
класса~--- независимые в~совокупности и~не зависящие от входящего
потока случайные величины с~функцией распределения~$B_i(x)$,
$i\hm=1,2.$
 Если в~некоторый момент времени система освободилась от требований, 
 то обслуживающий прибор
 отправляется на профилактику, которая длится случайное время с~функцией 
 распределения~$C(x).$
 Не ограничивая общности, будем считать, что $B_i(x)\hm<1$
 и~$C(x)\hm<1$  для любого~$x$ 
 и~существуют плотности
 распределения~$b_i(x)$ и~$c(x).$
  Обозначим:
$$
 \beta_i(s)=\int\limits_0^{\infty}e^{-sx}b_i(x)\,dx\,;\enskip 
  \gamma(s)=\int\limits_0^{\infty}e^{-sx}c(x)\,dx\,.
$$
Пока прибор находится на профилактике, он не доступен для
обслуживания. Если за время профилактики поступают требования,
после ее завершения начинается их обслуживание. Если ни одно
требование не поступает, то прибор отправляется на новую
профилактику. Длительности различных профилактик являются
независимыми случайными величинами 
и~не зависят от входящего потока и~времен обслуживания.

\section{Вспомогательные результаты}

  Рассмотрим многочлен по $\mu$ степени $k$ вида:
\begin{multline}
\label{1}
\prod\limits_{i=1}^k\left(\mu+a_i\right)-{}\\
{}-
\left(pz_1+(1-p)z_2\right)\sum\limits_{j=1}^kc_ja_j\prod\limits_{i\ne
j}\left(\mu+a_i\right)\,.
\end{multline}
Занумеруем его корни $\mu_1(z_1,z_2),\ldots,\mu_k(z_1,z_2)$ таким образом,
чтобы они были непрерывными функциями и~$\mu_1(1,1)\hm=0.$ Тогда
$\mathrm{Re}\, \mu_j\left(z_1,z_2\right)\hm<0$, $|z_1|\hm<1$, 
$|z_2|\hm<1,$ $\mu_i(z_1,z_2)\hm\ne \mu_j(z_1,z_2),$ $ i\hm\ne j$,
$j\hm=1,\ldots,k.$ Обозначим:
$$
\alpha_m(z_1,z_2)=\prod\limits_{j\ne m}\left(\mu_m\left(z_1,z_2\right)-
\mu_j\left(z_1,z_2\right)\right)\,.
$$
Справедливы следующие леммы.

\smallskip

\noindent
\textbf{Лемма~1.}\
\textit{Для любого $l=1,\ldots,\:k$ система уравнений}
$$
z_j=\beta_j(s-\mu_l(z_1,z_2)),\ \ j=1,2,
$$
\textit{имеет единственное решение $z_i=z_{il}(s)$ такое, 
что $|z_{il}(s)|\hm<1$ при $l\hm=2,\ldots, k,$ $\mathrm{Re}\, s\hm\geqslant 0,$ 
а~$z_{i1}(0)\hm=1$, $|z_{i1}(s)|\hm<1$ при} $\mathrm{Re}\, s\hm> 0$, $i\hm=1,2.$

\smallskip

\noindent
\textbf{Лемма~2.}\
\textit{При каждом $l\hm=1,\ldots,k$ уравнение}
$$
z_1=\beta_1\left(s-\mu_l(z_1,z_2)\right)
$$
\textit{имеет единственное решение $z_1\hm=z_{1l}(z_2,s),$ 
аналитическое в~области $\mathrm{Re}\, s\hm>0$, $|z_2|\hm<1.$
}

\smallskip

Положим
$$
\lambda_l(s)=\mu_l\left(z_{1l}(s),z_{2l}(s)\right)\,.
$$




\section{Распределение длины очереди}

  Гиперэкспоненциальный поток можно рас\-смат\-ри\-вать как
пуассоновский поток со случайной интен\-сив\-ностью~$a,$ которая
принимает $k$ различных значений $a_1,\ldots,a_k$  с~вероятностями
$c_1,\ldots,c_k.$ Текущее значение~$a$ разыгрывается в~момент
поступления требования и~не меняется между двумя соседними
поступлениями. Введем случайный процесс~$j(t)$ такой, что если
$a\hm=a_j$ в~момент времени $t,$ то $j(t)\hm=j.$

Целью работы является нахождение распределения случайного процесса
$\left(L_1(t),L_2(t)\right),$ где $L_i(t)$~--- число требований из
$i$-го приоритетного класса, находящихся в~системе в~момент
времени~$t.$

При сделанных предположениях относительно параметров изучаемой
системы обслуживания\linebreak процесс $\left(L_1(t),L_2(t)\right)$ не
является, вообще говоря, марковским. Пусть $i(t)=i$, $i\hm=1,2,$ если
в~момент времени~$t$ обслуживается требование из $i$-го
приоритетного класса, и~$i(t)\hm=0,$ если в~момент времени~$t$ прибор
находится на профилактике. Случайный процесс~$x(t)$ определим
следующим образом. Если $i(t)\hm\ne 0,$ то $x(t)$ есть
время, прошедшее с~начала обслуживания требования, находящегося на
приборе, до момента~$t.$ Если $i(t)\hm=0,$ то $x(t)$ есть время,
прошедшее с~начала профилактики прибора до момента~$t.$ Случайный
процесс $\left(L_1(t),L_2(t),i(t),j(t),x(t)\right)$ является
однородным марковским процессом. Положим
\begin{multline*}
P_{ij}(n_1,n_2,x,t)=\fr{\partial}{\partial x}
\mathbf{P}\left(L_1(t)=n_1,L_2(t)=n_2,\right.\\
\left. i(t)=i,j(t)=j,x(t)<x
\vphantom{L_1}\right)\,,\enskip 
 x\geqslant 0,\\ 
 j=1,\ldots,k,\enskip i=0,1,2;
\end{multline*}
\begin{gather*}
\eta_i(x)=\fr{b_i(x)}{1-B_i(x)},\ i=1,2;\enskip 
\eta_0(x)=\fr{c(x)}{1-C(x)}\,;\\
\delta_{i,j}=\begin{cases}
1,&\ i=j;\\ 
0,&\ i\ne j\,.
\end{cases}
\end{gather*}
Функции $P_{ij}(n_1,n_2,x,t)$  удовлетворяют при $x\hm>0$
системам дифференциальных уравнений:
\begin{multline}
\label{3}
\fr{\partial P_{ij}(n_1,n_2,x,t)}{\partial t}+\fr{\partial
P_{ij}(n_1,n_2,x,t)}{\partial
x}={}\\
{}=-(a_j+\eta_i(x))P_{ij}(n_1,n_2,x,t)+ {}\\
{}+
c_j\sum\limits_{l=1}^ka_l\left(p\:P_{il}(n_1-1,n_2,x,t)+{}\right.\\
\left.{}+
(1-p)P_{il}(n_1,n_2-1,x,t)\right)
\end{multline}
и краевым условиям при $x\hm=0$:
\begin{multline}
\label{5}
P_{0j}(n_1,n_2,0,t)=0,\ n_1+n_2>0;\\
P_{0j}(0,0,0,t)=\int\limits_0^{\infty}P_{0j}(0,0,x,t)\eta_0(x)\,dx+{}\\
 {}+\int\limits_0^{\infty}P_{1j}(1,0,x,t)\eta_1(x)dx+{}\\
 {}+
\int\limits_0^{\infty}P_{2j}(0,1,x,t)\eta_2(x)\,dx\,;
\end{multline}

\vspace*{-12pt}

\noindent
\begin{multline}
\label{6}
P_{1j}(n_1,n_2,0,t)+P_{2j}(n_1,n_2,0,t)={}\\
{}=\int\limits_0^{\infty}P_{1j}(n_1+1,n_2,x,t)\eta_1(x)\,dx+{}\\
{}+
\int\limits_0^{\infty}P_{2j}(n_1,n_2+1,x,t)\eta_2(x)\,dx+{}\\
{}+\int\limits_0^{\infty}P_{0j}(n_1,n_2,0,t)\eta_0(x)\,dx\,.
\end{multline}

Будем предполагать, что в~начальный момент времени $t\hm=0$ система
свободна от требований, а~с~начала профилактики прибора прошло
случайное время с~заданным распределением с~плотностью $d(x).$
Таким образом,
\begin{align*}
P_{ij}\left(n_1,n_2,x,0\right)&=0,\ i=1,2;
\\
P_{0j}\left(n_1,n_2,x,0\right)&=c_jd(x)\delta_{n_1+n_2,0},\ \
j=1,\ldots,k\,.
\end{align*}
Положим
\begin{multline*}
p_{ij}\left(z_1,z_2,x,s\right)={}\\
{}=\sum\limits_{n_1=0}^{\infty}
\sum\limits_{n_2=0}^{\infty}z_1^{n_1}z_2^{n_2}\!
\int\limits_0^{\infty}e^{-st}P_{ij}(n_1,n_2,x,t)\,dt\,;
\end{multline*}
$$
  \psi(s)=\int\limits_0^{\infty}e^{-sx}\,dx
  \int\limits_0^{\infty}\fr{c(u+x)d(u)}{1-C(u)}\,du\,.
$$
Тогда, учитывая начальные условия,  из \eqref{3}
получаем:
\begin{multline}
\label{7} 
\fr{\partial p_{ij}(z_1,z_2,x,s)}{\partial x}={}\\
{}=-\left(s+a_j+\eta_i(x)\right)p_{ij}
\left(z_1,z_2,x,s\right)+{}\\
{}+c_j\left(pz_1+(1-p)z_2\right)
\sum\limits_{l=1}^ka_lp_{il}\left(z_1,z_2,x,s\right),\\ 
i=1,2;
\end{multline}

\vspace*{-12pt}

\noindent
\begin{multline}
\label{8} 
\fr{\partial p_{0j}(z_1,z_2,x,s)}{\partial x}={}\\
{}=-\left(s+a_j+\eta_0(x)\right)p_{0j}\left(z_1,z_2,x,s\right)+{}\\
{}+c_j\left(pz_1+(1-p)z_2\right)\sum\limits_{l=1}^ka_lp_{0l}\left(z_1,z_2,x,s\right)+{}\\
{}+ c_jd(x).
\end{multline}
Решения \eqref{7} и~\eqref{8} имеют вид:
\begin{multline}
\label{9}
p_{ij}\left(z_1,z_2,x,s\right)=\left(1-B_i(x)\right)c_j\times{}\\
{}\times \sum\limits_{m=1}^k\fr{\gamma_i^{(m)}(z_1,z_2,s)}{\mu_m(z_1,z_2)+a_j}\,
e^{-(s-\mu_m(z_1,z_2))x}\,,\\
 i=1,2\,,
\end{multline}
\vspace*{-12pt}

\noindent
\begin{multline}
\label{10}
p_{0j}\left(z_1,z_2,x,s\right)={}\\
{}=\left(1-C(x)\right)
c_j\!\!\sum\limits_{m=1}^k\!\! e^{-(s-\mu_m(z_1,z_2))x}\!
\!\left(\!
\vphantom{\int\limits_{l=1}^k}
\delta^{(m)}\left(z_1,z_2,s\right)+{}\right.\\
%\left.
{}+\alpha_m^{-1}\left(z_1,z_2\right)
\prod\limits_{l=1}^k
\left(\mu_m\left(z_1,z_2\right)+a_l\right)\times{}\\
\left.{}\times \int\limits_0^x\!
e^{(s-\mu_m(z_1,z_2))u}
\fr{d(u)}{1-C(u)}\,du
\right)
\!\Bigg/ \!\left(\mu_m\left(z_1,z_2\right)+{}\right.\\
\left.{}+a_j\right)\,,
\end{multline}
где функции $\gamma_i^{(m)}(z_1,z_2,s)$  и~$\delta^{(m)}(z_1,z_2,s)$ являются
произвольными функциями указанных переменных и~определяются из
краевых условий. Из~\eqref{5} и~\eqref{6} получаем:
\begin{multline}
\label{11}
p_{1j}\left(z_1,z_2,0,s\right)+p_{2j}\left(z_1,z_2,0,s\right)={}\\
{}=z_1^{-1}\int\limits_0^{\infty}p_{1j}\left(z_1,z_2,x,s\right)\eta_1(x)\,dx+{}
\\
+z_2^{-1}\int\limits_0^{\infty}p_{2j}\left(z_1,z_2,x,s\right)\eta_2(x)\,dx+{}\\
{}+
\int\limits_0^{\infty}p_{0j}\left(z_1,z_2,x,s\right)\eta_0(x)\,dx
-p_{0j}\left(z_1,z_2,0,s\right)\,.
\end{multline}
Заметим, что $p_{0j}(z_1,z_2,0,s)$ не зависит от $z_1$ и~$z_2,$ т.\,е.\
$p_{0j}(z_1,z_2,0,s)\hm=q_j(s).$ 
Подставляя~\eqref{9} и~\eqref{10} в~\eqref{11}, получаем:
\begin{multline}
\label{12}
\gamma_1^{(m)}\left(z_1,z_2,s\right)\left(1-z_1^{-1}\beta_1(s-\mu_m(z_1,z_2))\right)+{}\\
{}+
\gamma_2^{(m)}(z_1,z_2,s)\left(1-z_2^{-1}\beta_2(s-\mu_m(z_1,z_2))\right)={}\\
{} =
\delta^{(m)}\left(z_1,z_2,s\right)\left(\gamma\left(s-\mu_m\left(z_1,z_2\right)\right)-1\right)+{}\\
{}+
\alpha_m^{-1}\left(z_1,z_2\right)\prod\limits_{l=1}^k
\left(\mu_m\left(z_1,z_2\right)+a_l\right)\psi\left(s-\mu_m(z_1,z_2)\right),\\
j=1,\ldots,k.
\end{multline}
В силу леммы~1 левая часть~\eqref{12} обращается в~0 при
$z_1\hm=z_{1m}(s)$ и~$z_2\hm=z_{2m}(s)$, $m\hm=1,\ldots,k.$ Следовательно,
\begin{multline}
\label{13}
\delta^{(m)}\left(z_{1m}(s),z_{2m}(s),s\right)={}\\
{}=\fr{\psi(s-\lambda_m(s))}{\alpha_m(z_{1m}(s),z_{2m}(s))
(1-\gamma(s-\lambda_m(s)))}\times{}\\
{}\times \prod\limits_{l=1}^k\left(\lambda_m(s)+a_l\right).
\end{multline}
Из \eqref{10} следует, что
$$
q_j(s)=c_j\sum\limits_{m=1}^k\fr{\delta^{(m)}(z_1,z_2,s)}{\mu_m(z_1,z_2)+a_j},\
j=1,\ldots,k .
$$
Решая эту систему уравнений относительно
$\delta^{(m)}(z_1,z_2,s),$ получаем:
\begin{multline}
\label{n1}
\delta^{(m)}(z_1,z_2,s)=\left(pz_1+(1-p)z_2\right)\times{}\\
{}\times
\fr{\prod\nolimits_{j=1}^k(\mu_m(z_1,z_2)+a_j)}
{\alpha_m(z_1,z_2)}\sum\limits_{l=1}^k\frac{a_lq_l(s)}{\mu_m(z_1,z_2)+a_l}.
\end{multline}
Подставляя в~\eqref{n1} $z_1\hm=z_{1m}(s)$ и~$z_2\hm=z_{2m}(s),$ имеем:
\begin{multline}
\label{14}
\delta^{(m)}\left(z_{1m}(s),z_{1m}(s),s\right)={}\\
{}=
\left(pz_{1m}(s)+(1-p)z_{2m}(s)\right)\times{}\\
{}\times
\fr{\prod\nolimits_{j=1}^k
(\lambda_m(s)+a_j)}{\alpha_m(z_{1m}(s),z_{1m}(s))}
\sum\limits_{l=1}^k\fr{a_lq_l(s)}{\lambda_m(s)+a_l}\,.
\end{multline}
Сравнивая два представления~\eqref{13} в~\eqref{14} для
$\delta^{(m)}(z_m(s),s),$ получаем систему уравнений для~$q_l(s)$:
\begin{multline*}
\sum\limits_{l=1}^k\fr{a_lq_l(s)}{\lambda_m(s)+a_l}={}\\
{}=\fr{\psi(s-\lambda_m(s))}{(pz_{1m}(s)+(1-p)z_{2m}(s))
(1-\gamma(s-\lambda_m(s)))},\\
m=1,\ldots,k\,,
\end{multline*}
из которой находим
\begin{multline}
\hspace*{-3pt}q_l(s)=c_l\prod\limits_{j=1}^k
\left(\lambda_l(s)+a_j\right) 
\sum\limits_{m=1}^k
%\fr
\psi(s-\lambda_m(s))\!\Bigg/ \!
\Bigg(\left(1-{}\right.\\
\left.
{}-\gamma\left(s-\lambda_m(s)\right)\right)(\lambda_m(s)+a_l)\times{}\\
{}\times \prod\limits_{n\ne m}(\lambda_m(s)-\lambda_n(s))\!\Bigg).
\label{15}
\end{multline}
Подставляя \eqref{15} в~\eqref{n1} и~учитывая~\eqref{1}, получаем:
\begin{multline*}
\delta^{(m)}(z_1,z_2,s)=\fr{(pz_1+(1-p)z_2)}{\alpha_m(z_1,z_2)}\times
\\
\times\sum\limits_{j=1}^k
\fr{\psi(s-\lambda_j(s))\prod\nolimits_{l=1}^k(\lambda_j(s)+a_l)}
{(pz_{1j}(s)+(1-p)z_{2j}(s))(1-\gamma(s-\lambda_j(s)))}\times{}\\
{}\times\prod\limits_{\nu\ne j}
\fr{\mu_m(z_1,z_2)-\lambda_{\nu}(s)}{\lambda_j(s)-\lambda_{\nu}(s)}\,.
\end{multline*}
Положим
$$
\lambda_m(z_2,s)=\mu_m\left(z_{1m}(z_2,s),z_2\right),\enskip m=1,\ldots,k\,.
$$
Подставляя в~\eqref{12} $z_1\hm=z_{1m}(z_2,s)$, имеем:
\begin{multline}
\label{1q}
\gamma_2^{(m)}\left(z_{1m}(z_2,s),z_2,s\right)={}\\
{}=\fr{\delta^{(m)}(z_{1m}(z_2,s),z_2,s)(\gamma_m(s-\lambda_m(z_2,s))-1)}
{1-z_2^{-1}\beta_2(s-\lambda_m(z_2,s))}+{}
\\
{}+\alpha_m^{-1}(z_{1m}(z_2,s),z_2)\psi(s-\lambda_m(z_2,s))
\prod\limits_{l=1}^k\left(\lambda_m(z_2,s)+{}\right.\\
\left.{}+a_l\right)\!\Bigg/\!
\left(
1-z_2^{-1}\beta_2(s-\lambda_m(z_2,s))\right).
\end{multline}
Далее, из~\eqref{9} следует:
$$
p_{2j}(z_1,z_2,0,s)=c_j\sum\limits_{m=1}^k
\fr{\gamma_2^{(m)}(z_1,z_2,s)}{\mu_m(z_1,z_2)+a_j}\,.
$$
Отсюда
\begin{multline}
\label{2q}
\gamma_2^{(m)}(z_1,z_2,s)=\fr{pz_1+(1-p)z_2}{\alpha_m(z_1,z_2)}\times{}\\
{}\times
\prod\limits_{j=1}^k(\mu_m(z_1,z_2)+a_j)
\sum\limits_{l=1}^k\fr{a_lp_{2l}(z_1,z_2,0,s)}{\mu_m(z_1,z_2)+a_l}\,.
\end{multline}
Так как $p_{2j}(z_1,z_2,0,s)$ не зависит от $z_1$, то
\begin{multline}
\label{3q}
p_{2j}\left(z_1,z_2,0,s\right)={}\\
{}=c_j
\sum\limits_{m=1}^k\fr{\gamma_2^{(m)}\left(z_{1m}(z_2,s),z_2,s\right)}{\lambda_m(z_2,s)+a_j}\,.
\end{multline}
Таким образом, соотношения~\eqref{1q}--\eqref{3q} полностью
определяют $\gamma_2^{(m)}(z_1,z_2,s)$ при любых $z_1$ и~$z_2$.
Теперь из~\eqref{12} можно найти $\gamma_2^{(m)}(z_1,z_2,s)$.

Все функции, необходимые для вычисления $p_{ij}(z_1,z_2,x,s)$,
$i\hm=0,1,2$, $j\hm=1,\ldots,k,$ найде-\linebreak\vspace*{-12pt}

\columnbreak

\noindent
ны. Искомая производящая функция
процесса $(L_1(t),L_2(t))$ равна:

\noindent
\begin{multline*}
\int\limits_0^{\infty}e^{-st}\mathbf{E}
z_1^{L_1(t)} z_2^{L_2(t)}\,dt={}\\
{}=
\sum\limits_{i=0}^2\sum\limits_{j=1}^k\int\limits_0^{\infty}p_{ij}
\left(z_1,z_2,x,s\right)\,dx\,.
\end{multline*}

\vspace*{-18pt}

{\small\frenchspacing
 {%\baselineskip=10.8pt
 \addcontentsline{toc}{section}{References}
 \begin{thebibliography}{9}
\bibitem{1-u}
\Au{Doshi B.\,T.} Queueing systems with vacations~--- a~survey~// 
Queueing Syst., 1986. Vol.~1.  P.~29--66.
\bibitem{2-u}
\Au{Takagi H.} Time-dependent analysis of $M\vert G\vert 1$ vacation models 
with exhaustive service~// Queueing Syst.,
1990. Vol.~6.  P.~369--390.
\bibitem{3-u}
\Au{Li J., Tian N., Zhang~Z.\,G. , Luh~H.\,P.} 
Analysis of the $M\vert G\vert 1$ queue with exponentially working vacations~--- 
a~matrix analytic approach~// Queueing Syst., 2009. Vol.~61.
P.~139--166.
\bibitem{4-u}
\Au{Bouman N., Borst S.\,C., Boxma~O.\,J., Leeuwaarden~J.\,S.\,H.} 
Queues with random back-offs~// Queueing Syst.,
2014. Vol.~77. P.~33--74.
\bibitem{5-u}
\Au{Ушаков~В.\,Г.} Система обслуживания с~гиперэкспоненциальным входящим потоком 
и~профилактиками прибора~// Информатика и~её применения, 2016. Т.~10. 
Вып.~2. С.~93--98.
 \end{thebibliography}

 }
 }

\end{multicols}

\vspace*{-9pt}

\hfill{\small\textit{Поступила в~редакцию 11.05.18}}

\vspace*{6pt}

%\pagebreak

%\newpage

%\vspace*{-28pt}

\hrule

\vspace*{2pt}

\hrule

%\vspace*{-2pt}

\def\tit{A~HEAD OF~THE~LINE PRIORITY QUEUE\\ WITH~WORKING VACATIONS}

\def\titkol{A head of the line priority queue with working vacations}

\def\aut{E.\,S.~Kondranin$^1$ and~V.\,G.~Ushakov$^{1,2}$}

\def\autkol{E.\,S.~Kondranin and~V.\,G.~Ushakov}

\titel{\tit}{\aut}{\autkol}{\titkol}

\vspace*{-11pt}


\noindent
$^1$Department of 
Mathematical Statistics, Faculty of Computational Mathematics and Cybernetics, 
M.\,V.~Lo\-mo-\linebreak
$\hphantom{^1}$no\-sov Moscow State University, 1-52~Leninskiye Gory, 
Moscow 119991, GSP-1, Russian Federation

\noindent
$^2$Institute of Informatics Problems, Federal Research Center 
``Computer Science and Control'' of the Russian\linebreak
$\hphantom{^1}$Academy of Sciences,  44-2~Vavilov Str., Moscow 119333, Russian Federation

\def\leftfootline{\small{\textbf{\thepage}
\hfill INFORMATIKA I EE PRIMENENIYA~--- INFORMATICS AND
APPLICATIONS\ \ \ 2018\ \ \ volume~12\ \ \ issue\ 4}
}%
 \def\rightfootline{\small{INFORMATIKA I EE PRIMENENIYA~---
INFORMATICS AND APPLICATIONS\ \ \ 2018\ \ \ volume~12\ \ \ issue\ 4
\hfill \textbf{\thepage}}}

\vspace*{3pt}



\Abste{The authors analyze the single-server queueing system with 
two types of customers, head of the line priority, hyperexponential 
input stream, and working vacations. The authors obtain the Laplace 
transform (with respect to an arbitrary point in time) of the joint 
distribution of server state, queue size, and elapsed time in that state. 
The authors restrict themselves to a~system with exhaustive service (the 
queue must be empty when the server starts a vacation) and multiple vacations. 
The queueing systems with vacations have been well studied because of their 
applications in modeling computer networks, communication, and manufacturing 
systems. For example, in many digital systems, the processor is multiplexed 
among a~number of jobs and, hence, is not available all the time to handle one job type. 
Besides such an application, theoretical interest in vacation models 
has been aroused with respect to their relationship with polling models.}

\KWE{hyperexponential input stream; working vacations; single server; 
head of the line priority; queue length}



\DOI{10.14357/19922264180405}

\vspace*{-14pt}

\Ack
\noindent
This work was supported by the Russian Foundation for Basic Research 
(project 18-07-00678).


%\vspace*{6pt}

  \begin{multicols}{2}

\renewcommand{\bibname}{\protect\rmfamily References}
%\renewcommand{\bibname}{\large\protect\rm References}

{\small\frenchspacing
 {%\baselineskip=10.8pt
 \addcontentsline{toc}{section}{References}
 \begin{thebibliography}{9}
\bibitem{1-u-1}
\Aue{Doshi, B.\,T.} 1986. Queueing systems with vacations~--- a~survey. 
\textit{Queueing Syst.} 1:29--66.
\bibitem{2-u-1}
\Aue{Takagi, H.} 1990. Time-dependent analysis of $M\vert G\vert M\vert 1$ 
vacation models with exhaustive service. \textit{Queueing Syst.} 6:369--390.
\bibitem{3-u-1}
\Aue{Li, J., N. Tian, Z.\,G.~Zhang,  and H.\,P.~Luh.} 2009. Analysis of the 
$M\vert G\vert 1$ queue with exponentially working vacations~--- 
a~matrix analytic approach. \textit{Queueing Syst.} 61:139--166.
{\looseness=1

}
\bibitem{4-u-1}
\Aue{Bouman, N., S.\,C.~Borst, O.\,J.~Boxma, and J.\,S.\,H.~Leeuwaarden.} 
2014. Queues with random back-offs. \textit{Queueing Syst.} 77:33--74.
\bibitem{5-u-1}
\Aue{Ushakov, V.\,G.} 2016. Sistema obsluzhivaniya s~gipereksponentsialnym 
vkhodyashchim potokom i~profilaktikami\linebreak pribora [Queueing system with working 
vacations and hyperexponential input stream]. 
\textit{Informatika i~ee Primeneniya~--- Inform. Appl.} 10(2):93--98.
\end{thebibliography}

 }
 }

\end{multicols}

\vspace*{-6pt}

\hfill{\small\textit{Received May 11, 2018}}

%\pagebreak

%\vspace*{-18pt}

\Contr

\noindent
\textbf{Kondranin Egor S.} (b.\ 1995)~---  MSc student, Department of 
Mathematical Statistics, Faculty of Computational Mathematics and Cybernetics, 
M.\,V.~Lomonosov Moscow State University, 1-52~Leninskiye Gory, 
Moscow 119991, GSP-1, Russian Federation; \mbox{ekondranin@yandex.ru}

\vspace*{6pt}

\noindent
\textbf{Ushakov Vladimir G.} (b.\ 1952)~--- 
Doctor of Science in physics and mathematics, professor, Department of Mathematical 
Statistics, Faculty of Computational Mathematics and Cybernetics, 
M.\,V.~Lomonosov Moscow State University, 1-52~Leninskiye Gory, Moscow 119991, 
GSP-1, Russian Federation; 
senior scientist, Institute of Informatics Problems, Federal Research Center 
``Computer Science and Control'' of the Russian Academy of Sciences, 
44-2~Vavilov Str., Moscow 119333, Russian Federation; \mbox{vgushakov@mail.ru}
\label{end\stat}

\renewcommand{\bibname}{\protect\rm Литература}        

\include{gaponova}

\newcommand{\cov}{\textrm{cov}}
%\newcommand{\indic}{\mathbb{1}}
\newcommand{\Obig}{\textsf{O}}
\newcommand{\osml}{\textsf{o}}
\newcommand{\hsig}{\hat\sigma^2}
\newcommand{\Yljk}{Y_{\lambda;j,\mathbf{k}}}

\newcommand{\Yljks}{Y_{\lambda';j',\mathbf{k'}}}
\newcommand{\muljk}{\mu_{\lambda;j,\mathbf{k}}}
\newcommand{\solj}{\sigma_{\lambda;j}}
\newcommand{\soljs}{\sigma_{\lambda';j'}}
\newcommand{\solz}{\sigma_{\lambda;0}}
\newcommand{\silz}{\sigma_{1;0}}
\newcommand{\siilz}{\sigma_{2;0}}
\newcommand{\siiilz}{\sigma_{3;0}}
\newcommand{\slj}{\solj^2}
\newcommand{\sljs}{\soljs^2}
\newcommand{\hslj}{\hat\sigma^2_{\lambda;j}}
\newcommand{\hsljs}{\hat\sigma^2_{\lambda';j'}}
\newcommand{\hslz}{\hat\sigma_{\lambda;0}^2}
\newcommand{\Tlj}{T_{\lambda;j}}
\newcommand{\hTlj}{\hat T_{\lambda;j}}

\newcommand{\indYjklTj}{\Ik_{\left|\Yljk\right|\leqslant \Tlj}}
\newcommand{\indYjkgTj}{\Ik_{\left|\Yljk\right|>\Tlj}}
\newcommand{\prbYjkgTj}{\p\left( \left|\Yljk\right|>\Tlj \right)}
\newcommand{\indYjklhTj}{\Ik_{\left|\Yljk\right|\leqslant \hTlj}}
\newcommand{\indYjkghTj}{\Ik_{\left|\Yljk\right|>\hTlj}}
\newcommand{\sumljk}{\sum\limits_{\lambda,j,\mathbf{k}}}

\def\stat{markin}

\def\tit{АСИМПТОТИКИ ОЦЕНКИ РИСКА ПРИ ПОРОГОВОЙ ОБРАБОТКЕ ВЕЙВЛЕТ-ВЕЙГЛЕТ КОЭФФИЦИЕНТОВ В ЗАДАЧЕ ТОМОГРАФИИ}

\def\titkol{Асимптотики оценки риска при пороговой обработке вейвлет-вейглет коэффициентов в задаче томографии}

\def\autkol{А.\,В.~Маркин, О.\,В.~Шестаков}
\def\aut{А.\,В.~Маркин$^1$, О.\,В.~Шестаков$^2$}

\titel{\tit}{\aut}{\autkol}{\titkol}

%{\renewcommand{\thefootnote}{\fnsymbol{footnote}}\footnotetext[1]
%{Исследования выполнены при частичной поддержке РФФИ, гранты 08-01-00567, 08-01-91205, 09-01-12180.}}

\renewcommand{\thefootnote}{\arabic{footnote}}
\footnotetext[1]{Московский государственный университет им.\ М.\,В.~Ломоносова, 
факультет вычислительной математики и кибернетики, кафедра математической статистики, artem.v.markin@mail.ru}
\footnotetext[2]{Московский государственный университет им.\ М.\,В.~Ломоносова, 
факультет вычислительной математики и кибернетики, кафедра математической статистики,
oshestakov@cs.msu.su}


\Abst{Рассмотрена задача реконструкции изображения по радоновскому образу с помощью вейв\-лет-вейг\-лет разложения. 
Исследованы свойства оценки риска пороговой обработки вейг\-лет-коэф\-фи\-ци\-ен\-тов, такие как состоятельность и 
асимптотическая нормальность.}

\KW{вейвлеты; томография; пороговая обработка; оценка риска; предельное распределение}

     \vskip 18pt plus 9pt minus 6pt

      \thispagestyle{headings}

      \begin{multicols}{2}

      \label{st\stat}


\section{Введение}

Вейвлет-преобразование является весьма популярным и удобным методом обработки нестационарных сигналов и 
изображений. Одна из основных задач, для которых используются вейв\-ле\-ты,~---
удаление шума и сжатие. Эти операции производятся путем пороговой обработки 
вейв\-лет-коэффициентов. Кроме того, вейвлеты могут быть использованы для обращения 
линейных операторов, таких, например, как преобразование Радона. В этом случае 
пороговая обработка выполняет задачу регуляризации соответствующей формулы обращения.

Пусть на плоскости $(x,\,y)$ задана функция~$f$. Определим образ 
(или проекции) Радона~$\mathcal{R}f$ как набор интегралов от~$f$ по всевозможным прямым плоскости
\begin{equation}
\label{eq_radonTransform}
\mathcal{R}f(s,\theta)=\int\limits_{L_{s,\theta}}f\left(x,\,y\right)\,dl\,,
\end{equation}
где
\begin{equation*}
L_{s,\theta}=\left\{ (x,\,y): x\cos\theta+y\sin\theta-s=0 \right \}\,.
\end{equation*}
Формула обращения преобразования~(\ref{eq_radonTransform}) впервые была получена 
Радоном, ее можно записать в следующем виде~\cite{Natterer}:
\begin{equation}
\label{eq_radonInverse}
f = \fr{1}{2}\mathcal{R}^{\#}\mathcal{I}^{-1}\mathcal{R}f\,,
\end{equation}
где $\mathcal{R^{\#}}$~--- оператор обратного проецирования:
\begin{equation*}
\left(\mathcal{R^{\#}}g\right)(x,y)=\int\limits_0^{2\pi}g(x\cos\theta+y\sin\theta,\theta)\,d\theta\,;
\end{equation*}
$\mathcal{I}$~--- потенциал Рисса:
\begin{equation}
\label{eq_RieszPoten}
\left(\mathcal{F}_1\mathcal{I}^\alpha g\right) (\omega) = |\omega|^{-\alpha}\left(\mathcal{F}_1 g\right)(\omega)\,,
\end{equation}
а $\mathcal{F}_k$~--- $k$-мерное преобразование Фурье.

Для точного восстановления~$f$ требуется точное знание всевозможных проекций~$\mathcal{R}f(s,\,\theta)$. 
На практике же имеют дело с конечным числом проекций, причем в проекциях присутствует шум.
При этом задача томографии является некорректной, т.\,е.\ малые изменения в проекциях могут 
при\-вес\-ти к восстановлению изображения, существенно отличающегося от исходного. Математически
это выражается в наличии множителя~$|\omega|$ в формуле~(\ref{eq_RieszPoten}) (и, следовательно, 
в~(\ref{eq_radonInverse})), который <<подчеркивает>> высокие частоты.

Выход видится в регуляризации~(\ref{eq_radonInverse}) путем умножения~$|\omega|$ на некоторый множитель, 
называемый частотным фильтром (или стабилизирующим множителем)~\cite{TikhonovArsenin}. 
Общая идея регуляризации такова:\linebreak
немного <<испортить>> проекционные данные, подавив влияние 
высоких частот, но при этом обеспечить реконструкцию, близкую к оригиналу. Подроб\-нее о 
регуляризации формулы обращения можно прочитать в монографии~\cite{Herman}.

\section{Вейвлет-вейглет разложение}

Задачу томографии можно решить и с помощью вейвлетов. Пусть~$\phi(t)$ и~$\psi(t)$~--- 
одномерные отцовский и материнский вейвлеты. Определим
\begin{align*}
\phi_{j,k_1,k_2}(x,y) &= 2^{j} \phi\left(2^jx-k_1\right) \phi\left(2^jy-k_2\right)\,;\\
\psi^{[1]}_{j,k_1,k_2}(x,y) &= 2^{j} \phi\left(2^jx-k_1\right) \psi\left(2^jy-k_2\right)\,;
\end{align*}

\noindent
\begin{align*}
\psi^{[2]}_{j,k_1,k_2}(x,y) &= 2^{j} \psi\left(2^jx-k_1\right) \phi\left(2^jy-k_2\right)\,;\\
\psi^{[3]}_{j,k_1,k_2}(x,y) &= 2^{j} \psi\left(2^jx-k_1\right) \psi\left(2^jy-k_2\right)\,.
\end{align*}
Заметим, что параметр масштаба~$j$ контролирует сразу обе функции в произведении. 
Это так называемое тензорное произведение двух одномерных кратномасштабных анализов~\cite{Daub}. 
Тогда набор функций $\left\{ \phi_{j_0,k_1,k_2}, \, \psi^{[\lambda]}_{j,k_1,k_2}, \right\}$, 
где $j$, $k_1$, $k_2\in\mathbb{Z}$, $j\geq j_0$, $\lambda=\overline{1,3}$, 
будет ортонормированным базисом~$\mathbf{L}^2(\mathbb{R}^2)$.

Донохо~\cite{DonohoWVD} решил задачу обращения ряда линейных операторов 
с помощью вейвлетов и родственных им функций специального вида, названных вейглетами (\textit{vaguelettes}). 
Вейглеты для обращения оператора Радона выглядят так:
\begin{multline*}
\xi^{[\lambda]}_{j,k_1,k_2}(s,\,\theta)=
\int\limits_{-\infty}^\infty|\omega|\left(\mathcal{F}_2\psi^{[\lambda]}_{j,k_1,k_2}\right)\times{}\\
{}\times \left( \omega\cos\theta,\,\omega\sin\theta \right)\exp(i2\pi s\omega)\,d\omega\,.
%\label{eq_vagueletteDef}
\end{multline*}
Идея метода реконструкции заключается в том, что вейглет-коэффициенты проекций~$\mathcal{R}f(s,\theta)$ 
равны вейвлет-коэффициентам исходной функции~$f(x,y)$:
\begin{equation*}
\left[\mathcal{R}f,\,\xi^{[\lambda]}_{j,k_1,k_2}\right] = \left\langle f,\,\psi^{[\lambda]}_{j,k_1,k_2}\right\rangle\,,
\end{equation*}
и поэтому
\begin{multline}
f = \sum\limits_{k_1,k_2}\left[\mathcal{R}f,\,\tau_{j_0,k_1,k_2}\right] \phi_{j_0,k_1,k_2} +{}\\
{}+ \sum\limits_{j\geqslant j_0,k_1,k_2,\lambda} \left[\mathcal{R}f,\,\xi^{[\lambda]}_{j,k_1,k_2}\right] \psi^{[\lambda]}_{j,k_1,k_2}\,,
\label{eq_radonInverseWVD}
\end{multline}
где
\begin{multline*}
\tau_{j_0,k_1,k_2}(s,\,\theta)=\int\limits_{-\infty}^\infty|\omega|\left(\mathcal{F}_2\phi_{j_0,k_1,k_2}\right)\times{}\\
{}\times \left( \omega\cos\theta,\,\omega\sin\theta \right)
\exp\left(i2\pi s\omega\right)\,d\omega\,.
\end{multline*}
Регуляризация вейвлет-вейглет формулы~(\ref{eq_radonInverseWVD}) производится с 
помощью мягкой пороговой обработки вейглет-коэффициентов (см.\ разд.~\ref{sect_ThreshholdingTomo}).

\section{Дискретизация и модель шума}

Пусть функция $f(x,y)$ задана на квадрате $[0,\,1]\;\times$\linebreak $\times\;[0,\,1]$. Разбив стороны квадрата на~$N=2^J$ 
равных частей и вычислив значения~$f$ в точках отсчета, получим дискретизованную версию~$f$. Одна\-ко на практике 
нередко бывает удобно нормировать длину отрезка разбиения и рас\-сматривать вместо~$f$ ее <<растянутую>>
версию~--- функцию~$\bar f(Nx,Ny)\;=$\linebreak $={f}(x,y)$. Тогда для вейвлет-коэффициентов функции~$f$ справедливо равенство:
\begin{multline}
\left\langle f,\, \psi^{[\lambda]}_{j,u_1,u_2}\right\rangle ={}\\
{}= \iint f(x,y)\,2^j\overline{\psi^{[\lambda]}\left(2^jx-k_1,\,2^jy-k_2\right)}\,dx\,dy ={}\\
{}=\left(\mathcal{W}^{[\lambda]}f\right)\left(2^{-j},k_1,k_2\right)={}\\
{}=\fr{1}{N}\left(\mathcal{W}^{[\lambda]}\bar f\right)\left(N\,2^{-j},k_1,k_2\right)\,.
\label{eq_contToDiscrCoeff}
\end{multline}
Заметим, что при работе с растянутой функцией растягиваются и вейвлет-функции.
Коэффициенты аппроксимации, получаемые через скалярное произведение~$f$ и~$\phi$, не рассматриваются, 
так как пороговая обработка (см.\ разд.~4) применяется к коэффициентам деталей, которые дают функции~$\psi^{[\lambda]}$. 
Далее везде, кроме разд.~\ref{sect_RegularityTomo}, предполагается, что используются именно коэффициенты 
растянутой версии функции~$f$.

Задача томографии ставится следующим образом. Имеются наблюдения~$X$, состоящие из 
проекций~$\mathcal{R}f$ функции~$f$ и шума~$\epsilon$:
\begin{equation*}%\label{eq_tomoTask}
X=\mathcal{R}f+\epsilon\,, 
\end{equation*}
$\epsilon$~--- независимые нормальные случайные величины с нулевым средним и дисперсией~$\sigma^2$. 
Необходимо восстановить~$f$ по~$X$. При этом при достаточно большом~$N$~\cite{KolaczykArticle}
\begin{equation}
\left.
\begin{array}{rl}
\e \left[X,\,\xi^{[\lambda]}_{j,k_1,k_2}\right] &= \left[\mathcal{R}f,\,\xi^{[\lambda]}_{j,k_1,k_2}\right]\,;\\[9pt]
\D \left[X,\,\xi^{[\lambda]}_{j,k_1,k_2}\right] &= \sigma^2 \left\|\xi^{[\lambda]}_{j,k_1,k_2}\right\|_2^2=\sigma^2_{\lambda;j}\,;\\[9pt]
\left\|\xi^{[\lambda]}_{j,k_1,k_2}\right\|_2^2 &= 2^j \left\|\xi^{[\lambda]}_{0,0,0}\right\|_2^2\,.
\end{array}
\right \}
\label{eq_expctVageuletteCoef}
\end{equation}
Как видим, дисперсия коэффициентов растет вмес\-те с уровнем разложения. Это является следствием 
некорректности задачи томографии. При этом вейглеты не ортогональны, а почти ортогональны. И, 
стало быть, вейг\-лет-коэф\-фи\-ци\-ен\-ты не независимы, а почти независимы. Однако нередко 
этим фактом пренебрегают, так как исследование этой зависимости сопряжено с рядом трудностей. 
И потому порог выбирается исходя из предположения независимости коэффициентов. Как будет видно 
далее, уже только тот факт, что дисперсия растет на каждом уровне, заметно влияет на оценку 
риска пороговой обработки.

\section{Пороговая обработка}\label{sect_ThreshholdingTomo}

Мягкая пороговая функция определяется следующим образом:
\begin{equation*}
\rho(x, T)=
\begin{cases}
x-T & \text{при } x>T\,;\\
x+T & \text{при } x<-T\,;\\
0 & \text{при } |x|\leq T\,.
\end{cases} 
\end{equation*}
Эта функция применяется к вейглет-ко\-эф\-фи\-ци\-ен\-там проекций.

Допустим, что размер изображения равен $N^2\;=$\linebreak $=2^{2J}=L$, разложение идет до уровня~$J-1$. 
В~качестве порога взят порог Колашика~\cite{KolaczykArticle, KolaczykThesis}:
\begin{equation*}
\Tlj = \sqrt{2\ln 2^{2j}} \, 2^{j/2}\sigma  \left\|\xi^{[\lambda]}_{0,0,0}\right\|_2\,.
\end{equation*}
В случае использования оценки дисперсии шума~$\hsig$ порог принимает вид
\begin{equation*}
\hTlj = \sqrt{2\ln 2^{2j}}\,2^{j/2}\hat\sigma \left\|\xi^{[\lambda]}_{0,0,0}\right\|_2\,.
\end{equation*}
Идея выбора такого порога схожа с идеей выбора порога~\textit{VisuShrink} 
$T=\sigma\sqrt{2\ln N}$ (одномерный случай, $N$~--- размер сигнала): при таком пороге 
убирается почти весь шум~\cite{DJideal, DJunkn}. Это следует из того факта, что если $Z_1,\ldots,Z_N$~--- 
независимые стандартные нормальные случайные величины, то
\begin{equation*}
\p\left( \underset{1\leqslant i\leqslant N}{\max}|Z_i| > \sqrt{2\ln N} \right) \rightarrow 0\
\mbox{при}\ N\rightarrow\infty\,.
\end{equation*}

Пороговая обработка идет с уровня~$j_M$, т.\,е.\ в формуле~(\ref{eq_radonInverseWVD}) 
$j_0=j_M$ ($j_M$ определим ниже). Риск~$r(f)$ такой пороговой обработки определяется следующим образом:
\begin{multline}
r(f)=\sum\limits_{j=j_M}^{J-1}\sum_{\lambda=1}^3\sum_{k_1=0}^{2^j-1}
\sum_{k_2=0}^{2^j-1}\e
\left\{ \left\langle f,\,\psi^{[\lambda]}_{j,k_1,k_2}\right\rangle - {}\right.\\
\left.{}-\rho\left(\left[X,\,\xi^{[\lambda]}_{j,k_1,k_2}\right],\,\Tlj\right) \right\}^2\,.
\label{eq_riskEstimDefTomo}
\end{multline}
Так как на практике коэффициенты $\left\langle f,\,\psi^{[\lambda]}_{j,k_1,k_2}\right\rangle$ 
неизвестны, то строят оценку риска. Например,
на основе функции~$\Phi(x,T)$~\cite{Mallat}:
\begin{equation*}
\Phi(x,\Tlj)=
\begin{cases}
x-\slj & \text{при } x\leqslant \Tlj^2\,;\\
\slj+\Tlj^2 & \text{при } x> \Tlj^2\,.
\end{cases} 
\end{equation*}
Оценка риска принимает вид:
\begin{equation*}
\tilde r(f)=\sum\limits_{j=j_M}^{J-1}\sum\limits_{\lambda,k_1,k_2}\Phi\left( 
\left| \left[X,\,\xi^{[\lambda]}_{j,k_1,k_2}\right] \right|^2 ,\,\Tlj\right)\,.
\end{equation*}
Если вместо~$\sigma^2$ используется оценка~$\hsig$, то
\begin{equation*}
\hat r(f)=\sum\limits_{j=j_M}^{J-1}
\sum\limits_{\lambda,k_1,k_2}\hat\Phi\left( 
\left| \left[X,\,\xi^{[\lambda]}_{j,k_1,k_2}\right] \right|^2 ,\,\hTlj\right)\,,
\end{equation*}
где
\begin{equation*}
\hat\Phi(x,\hTlj)=
\begin{cases}
x-\hslj & \text{при } x\leqslant \hTlj^2\,;\\
\hslj+\hTlj^2 & \text{при } x> \hTlj^2\,.
\end{cases} 
\end{equation*}


В работах~\cite{MarkinShestakovConsist, MarkinLimitDistr} рассмотрены асимптотические свойства оценки 
риска пороговой обработки вейв\-лет-коэффициентов в одномерном случае при прямом наблюдении~$f$. 
Показано, что $(\hat r -r)/N^a$ сходится по вероятности к нулю и по распределению к нормальному 
закону при соответствующих~$a$.
Величина~$a$ существенно зависит от свойств оценки~$\hsig$. Однако даже при весьма общих ограничениях 
на моменты~$\hsig$ порядок $a=1$ обеспечивал
сходимость по вероятности к нулю. Ниже будет показано, что в задаче томографии для сходимости 
по вероятности к нулю недостаточно делить на число коэффициентов ($N^2=L$),
т.\,е.\ некоторый аналог закона больших чисел уже не выполнен. Важнейшим фактором 
является то, что~$f$ наблюдается через оператор Радона~$\mathcal{R}$, обратный к 
которому не является непрерывным (т.\,е.\ ограниченным).

\section{Регулярность функции и~вейвлет-коэффициенты}\label{sect_RegularityTomo}

Известно (см., например,~\cite{Mallat}), что если функция~$f(x,y)$ является регулярной по Липшицу 
с параметром $0\leq\alpha\leq 1$, т.\,е.\
\begin{multline*}
\left|f(x_1,y_1)-f(x_2,y_2)\right|\leq{}\\
{}\leq C \left( |x_1-x_2|^2 + |y_1-y_2|^2 \right)^{\alpha/2}
\end{multline*}
для некоторой константы~$C$, не зависящей от $(x_1,y_1)$ и $(x_2,y_2)$, то существует не зависящая от~$J$, 
$j$, $k_1$ и $k_2$ константа~$A$ такая, что
\begin{equation*}
\left(\mathcal{W}^{[\lambda]}f\right)\left(2^{-j},k_1,k_2\right)\leq \fr{A}{2^{j(\alpha+1)}}\,.
\end{equation*}
В отечественной литературе вместо регулярности по Липшицу обычно используется термин <<непрерывность 
по Гёльдеру>>.
С учетом~(\ref{eq_contToDiscrCoeff}) получаем
\begin{equation*}
\left(\mathcal{W}^{[\lambda]}\bar f\right)\left(N\cdot 2^{-j},k_1,k_2\right) \leqslant \frac{A\cdot 2^J}{2^{j(\alpha+1)}}\,.
\end{equation*}

\textit{Предположение о регулярности~$f$: будем полагать, что функция~$f$ является регулярной по Липшицу с 
показателем~$\alpha>0$}. Будем считать, что пороговая обработка ведется с уровня 
$j_M\geq J/(\alpha+1)$. Заметим, что $J-j_M\rightarrow\infty$ при $J\rightarrow\infty$. 
Тогда при определенном выборе вейвлет-базиса~\cite{Mallat} найдется константа~$C_1$ такая, что для 
всех $j\geq j_M$ выполнено
\begin{equation}
\label{eq_WaveletCoeffUpperBoundTomo}
\left(\mathcal{W}^{[\lambda]}\bar f\right)\left(N\cdot 2^{-j},k_1,k_2\right) \leqslant C_1\,,
\end{equation}
причем $C_1$ не зависит от~$N$. Значит, математические ожидания в~(\ref{eq_expctVageuletteCoef}) ограничены.

В работе используется буква~$C$ (с индексом или без индекса) для обозначения констант, причем в 
разных местах~--- вообще говоря, разных.

\section{Асимптотика оценки риска при~известной дисперсии шума}\label{sect_ConsitKnownSTomo}

В работе~\cite{MarkinLimitDistr} показано, что в одномерном случае при известной дисперсии шума 
разность риска и оценки риска при делении на $\sqrt{N}$ сходится по распределению к нормальному 
закону. В задаче томографии уже надо делить не на~$\sqrt{L}$, а на~$L$.

Для краткости введем обозначения:
\begin{align*}
\Yljk &= \left[X,\,\xi^{[\lambda]}_{j,k_1,k_2}\right]\,;\\
\muljk &= \left\langle f,\,\psi^{[\lambda]}_{j,k_1,k_2}\right\rangle\,,
\end{align*}
где $\mathbf{k}=\left(k_1,\,k_2\right)$. Еще раз напомним, что~$\muljk$ рассматриваются 
как коэффициенты растянутой версии дискретизованной функции~$f$. С учетом предположения об 
ортогональности вейглетов получаем
\begin{equation}
\label{eq_YljkNormalDistributed}
\Yljk \sim \mathcal{N}\left(\muljk,\,\slj\right)\,,
\end{equation}
причем $\Yljk$~--- независимые случайные величины.

\medskip

\noindent
\textbf{Теорема 1.}
\textit{Пусть справедливы предположения о регулярности~$f$ из разд.~\ref{sect_RegularityTomo}. 
При известной дисперсии шума в задаче томографии}
\begin{equation*}
\fr{\tilde r(f)-r(f)}{L \sqrt{ b_2 \left( \silz^4 + \siilz^4 + \siiilz^4 \right) }} \Rightarrow \mathcal{N}(0,\,1)
\end{equation*}
\textit{при} $L\rightarrow\infty$, \textit{где} $b_2=2/(2^4-1)=2/15$.

\medskip

\noindent
Д\,о\,к\,а\,з\,а\,т\,е\,л\,ь\,с\,т\,в\,о.\ 
Представим разность оценки риска и самого риска в виде
\begin{multline*}
\tilde r-r=\sumljk\left(\Yljk^2-\slj\right)\indYjklTj +{}\\
\!\!\!\!{}+ \sumljk\left(\slj+\Tlj^2\right)\indYjkgTj - {}
\end{multline*}

\noindent
\begin{multline}
\ \ {}- \sumljk\e\left(\Yljk^2-\slj\right)\indYjklTj -{}\\
{}- \sumljk\e\left(\slj+\Tlj^2\right)\indYjkgTj ={}\\
{}= \sumljk\left(\Yljk^2-\e\Yljk^2\right) -{}\\
{}- \sumljk\left(\Yljk^2-\slj\right)\indYjkgTj + {}\\
{}+ \sumljk\e\left(\Yljk^2-\slj\right)\indYjkgTj +{}\\
{}+ \sumljk\left(\slj+\Tlj^2\right)\indYjkgTj -{}\\
{}- \sumljk\left(\slj+\Tlj^2\right)\prbYjkgTj\,.
\label{eq_diffRiskEstimKnownS}
\end{multline}
Покажем, что при делении на~$L$ первая сумма в~(\ref{eq_diffRiskEstimKnownS}) сходится по распределению
к нормальному закону, а остальные суммы~--- к нулю по вероятности.

Итак, рассмотрим первую сумму в~(\ref{eq_diffRiskEstimKnownS}). Имеем
\begin{multline}
D_L^2 = \D\sumljk\Yljk^2 = \sumljk \D \Yljk^2={}\\
{}=\sum\limits_\lambda \sum_{j=j_M}^{J-1} \sum_{\mathbf{k}}
\left( 2\solj^4 + 4\muljk^2\slj \right)={}\\
{}= \sum\limits_\lambda \sum_{j=j_M}^{J-1}\left\{ 2\cdot 2^{2j}
\solz^4 \cdot 2^{2j} + \sum_{\mathbf{k}}4\muljk^2 2^j \solz^2 \right\} \simeq{}\\
{}\simeq \sum\limits_\lambda \sum_{j=j_M}^{J-1} 2\cdot 2^{4j}
\solz^4 = \sum\limits_\lambda 2\solz^4 \frac{2^{4J} - 2^{4j_M}}{2^4 - 1} \simeq{}\\
{}\simeq \fr{2}{15} 2^{4J} \left( \silz^4 + \siilz^4 + \siiilz^4 \right)\,.
\label{eq_DLknownS}
\end{multline}
Знак~$\simeq$ означает, что при $J\rightarrow\infty$ предел отношения левой и правой частей~(\ref{eq_DLknownS}) 
равен единице. Если выполнено условие Линдеберга, т.\,е.\ для любого~$\delta>0$
\begin{multline}
\fr{1}{D_L^2}\sumljk\e\left\{ \left( \Yljk^2 - \muljk^2 - \slj \right)^2\times{}\right.\\
\left.{}\times \Ik_{\left|\Yljk^2 - \muljk^2 - \slj\right|>\delta D_L} \right\} \rightarrow 0\,,
\label{eq_LindCondTomo}
\end{multline}
то будет иметь место сходимость к нормальному распределению. Так как~$D_L$ имеет порядок~$L$ и чис\-ло слагаемых 
в~(\ref{eq_LindCondTomo}) имеет порядок~$L$, то достаточно показать, что при $L\rightarrow\infty$
\begin{multline*}
\e\left\{ \fr{\left( \Yljk^2 - \muljk^2 - \slj \right)^2}{D_L} \times{}\right.\\
\left.{}\times\Ik_{\left(\Yljk^2 - \muljk^2 - \slj\right)^2/D_L>\delta^2 D_L} \right\} \rightarrow 0\,.
\end{multline*}
А последнее выполнено потому, что у случайных величин вида $\left( \Yljk^2 - \muljk^2 - \slj \right)^2\!/D_L$ 
конечные математические ожидания и $D_L\rightarrow\infty$.

Теперь рассмотрим вторую сумму в~(\ref{eq_diffRiskEstimKnownS}). В силу~(\ref{eq_YljkNormalDistributed}) имеем
\begin{multline*}
\p\left( |\Yljk| > \Tlj \right) < {}\\
{}< \frac{\exp\left( -(\Tlj-\muljk)^2/(2\slj) \right)}{\Tlj} +{}\\
{}+ \frac{\exp\left( -(\Tlj+\muljk)^2/(2\slj) \right)}{\Tlj} \leqslant \fr{C}{2^{5j/2} \sqrt{j} }
\end{multline*}
при $J \rightarrow \infty$ (и, следовательно, $j \rightarrow \infty$). Это можно получить из 
следующей цепочки равенств:
\begin{multline*}
\exp\left( -\fr{(\Tlj-\muljk)^2}{2\slj} \right) = {}\\
{}=\exp\left( -\fr{\Tlj^2}{2\slj} + \fr{2\Tlj\muljk}{2\slj} - \fr{\muljk^2}{2\slj} \right) ={}\\
{}= \exp\left( -\ln 2^{2j} + \fr{\sqrt{2\ln(2^{2j})}\muljk}{2^{j/2}\solz} - 
\fr{\muljk^2}{2\slj} \right) \simeq{}\\
{}\simeq 2^{-2j}\mbox{ при }j \rightarrow \infty\,,
\end{multline*}
так как
\begin{equation*}
\fr{\sqrt{2\ln(2^{2j})}\muljk}{2^{j/2}\solz} \rightarrow 0\quad\text{и}\quad\fr{\muljk^2}{2\slj} \rightarrow 0\,.
\end{equation*}
С помощью неравенств Чебышёва и Коши--Бу\-ня\-ков\-ского получаем для любого $\delta>0$ при $J\rightarrow\infty$
\begin{multline*}
\p\left( \fr{ \left|\sumljk\left(\Yljk^2-\slj\right)\indYjkgTj \right|}{D_L} > \delta \right) \leq{}\\
{}\leq \fr{ \e\left| \sumljk\left(\Yljk^2-\slj\right)\indYjkgTj \right| }{\delta D_L} \leq {}\\
{}\leq \fr{ \sumljk \e\left| \Yljk^2-\slj\right|\indYjkgTj }{\delta D_L} \leq{}\\
{}\leq \fr{ \sumljk \sqrt{ \e\left( \Yljk^2-\slj\right)^2 \p\left( |\Yljk| > \Tlj \right) } }{\delta D_L} 
\leq{}
\end{multline*}

\noindent
\begin{multline*}
{}\leq \fr{1}{\delta D_L}\sumljk 
\left  ( \vphantom{4\cdot 2^j\solz^2\muljk^2  C\cdot 2^{-5j/2} j^{-1/2}}
\left(
\muljk^4 + 2\cdot 2^{2j}\solz^4 + {}\right.\right.\\
\left.\left.{}+4\cdot 2^j\solz^2\muljk^2 
\right) C\cdot 2^{-5j/2} j^{-1/2} 
\right )^{1/2} \rightarrow 0\,.
\end{multline*}
Аналогично проводятся рассуждения для оставшихся сумм в~(\ref{eq_diffRiskEstimKnownS}).~$\square$

\section{Свойства оценки риска при~использовании оценки дисперсии шума}

В работе~\cite{MarkinShestakovConsist} показано, что при достаточно слабых ограничениях 
на моменты оценки дисперсии шума для сходимости разности риска и его оценки к нулю по 
вероятности ее надо нормировать числом вейвлет-коэффициентов, т.\,е.\ порядок знаменателя 
вырастает почти на~1/2. Покажем, что в задаче томографии порядок тоже повышается почти на~1/2, 
но знаменатель уже будет много больше числа коэффициентов.

Введем обозначение
\begin{equation*}
\hslj = 2^j \hsig \left\|\xi^{[\lambda]}_{0,0,0}\right\|_2^2\,.
\end{equation*}

\medskip
\noindent
\textbf{Теорема 2.} \textit{Пусть справедливы предположения о регулярности~$f$. 
Пусть $\hsig$~--- оценка дисперсии, $\e\hsig-\sigma^2=\nu_L$ и 
$\D\hsig=\theta_L=\Obig(L^{-\beta})$, $\nu_L=\osml(1)$, $\beta>0$. 
Тогда при $L\rightarrow\infty$ выполнено}
\begin{equation}
\label{eq_ConsistTomo32}
\fr{\hat r(f)-r(f)}{L^{3/2}} \xrightarrow{\textsf{P}} 0\,.
\end{equation}

\medskip

\noindent
Д\,о\,к\,а\,з\,а\,т\,е\,л\,ь\,с\,т\,в\,о.\
Подобно доказательству теоремы~3 в~\cite{MarkinShestakovConsist} запишем
\begin{equation*}
\hat r-r = S_1 + S_2\,,
\end{equation*}
где
\begin{multline}
S_1 = \sumljk\left(\Yljk^2-\hslj\right) -{}\\
{}- \sumljk\e\left(\Yljk^2-\slj\right)\,; \label{eq_riskSplitSoTomo}
\end{multline}

\vspace*{-6pt}

\noindent
\begin{multline*}
S_2 = - \sumljk\left(\Yljk^2-\hslj\right)\indYjkghTj +{}\\
\!\!{}+ \sumljk\left(\hslj+\hTlj^2\right)\indYjkghTj +{} 
\end{multline*}

\noindent
\begin{multline}
{}+ \sumljk\e\left(\Yljk^2-\slj\right)\indYjkgTj -{}\\
{}- \sumljk\e\left(\slj+\Tlj^2\right)\indYjkgTj\,.
\label{eq_riskSplitStTomo}
\end{multline}
Далее будет показано, что при делении на $L^{3/2}$ и~$S_1$, и~$S_2$ сходятся к нулю по вероятности.

Сначала рассмотрим~$S_1$: по неравенству Чебышёва при любом $\delta>0$
\begin{multline}
\p\left( \fr{|S_1|}{L^{3/2}} > \delta \right) \leq{}\\
{}\leq
\fr{ \e\left( \sumljk \left( \Yljk^2 - \hslj - \e\Yljk^2 + \slj \right) \right)^2 }{\delta^2 L^3} ={}\\
{}= \fr{ \sumljk \e\left( \Yljk^2 - \hslj - \e\Yljk^2 + \slj \right)^2 }{ \delta^2 L^3 } + {}\\
{}+ \fr{1}{\delta^2 L^3}
 \sum \e\left( \Yljk^2 - \hslj - \e\Yljk^2 + \slj \right)\times{}\\
 {}\times \left( \Yljks^2 - \hsljs - \e\Yljks^2 + \sljs \right)\,.
\label{eq_riskSplitUnknSTomo}
\end{multline}
Во второй сумме~(\ref{eq_riskSplitUnknSTomo}) суммирование идет по индексам 
$(\lambda,j,\mathbf{k})\ne(\lambda',j',\mathbf{k}')$. Понятно, что первое слагаемое в~(\ref{eq_riskSplitUnknSTomo}) 
стремится к нулю~--- в сумме всего порядка~$L$ слагаемых, они имеют порядок не выше~$L$ и 
сумма делится на~$L^3$ (напомним, что $L=2^{2J}$).

Рассмотрим одно из слагаемых второй суммы~(\ref{eq_riskSplitUnknSTomo}):
\begin{multline*}
\e\left( \Yljk^2 - \hslj - \e\Yljk^2 + \slj \right) \times{}\\
{}\times\left( \Yljks^2 - \hsljs - \e\Yljks^2 + \sljs \right) = {}\\
{}= \e\Yljk^2\Yljks^2 - \e\Yljk^2\hsljs - \e\Yljk^2 \e\Yljks^2 +{}\\
{}+ \sljs \e\Yljk^2 - 
 \e\hslj\Yljks^2 + \e\hslj\hsljs +{}\\
 {}+ \e\hslj\e\Yljks^2 - \sljs\e\hslj 
- \e\Yljk^2 \e\Yljks^2 +{}\\
{}+ \e\hsljs\e\Yljk^2 + \e\Yljk^2 \e\Yljks^2 - \sljs\e\Yljk^2 
+ {}\\
{}+\slj\e\Yljks^2 - \slj\e\hsljs - \slj\e\Yljks^2 +{}\\
{}+ \slj\sljs = 
 - \cov\left( \hsljs,\,\Yljk^2 \right) -{}\\
 {}- \cov\left( \hslj,\,\Yljks^2 \right) +
 \fr{\slj\sljs}{\sigma^4}\left( \nu_L^2 + \theta_L \right)\,.
\end{multline*}
С учетом того, что $\D\Yljk^2$ имеет порядок~$2^{2j}$, а ковариацию можно оценить по неравенству Коши--Бу\-ня\-ков\-ско\-го, 
получаем, что каждое слагаемое второй суммы~(\ref{eq_riskSplitUnknSTomo}) можно оценить как 
$2^{j+j'}\cdot\osml(1)$. Всего таких слагаемых порядка~$L^2$, а максимальное значение $2^{j+j'}$ 
равно $2^{J-1+J-1}=L/4$. Следовательно, после суммирования получаем, что второе сла\-га\-емое 
в~(\ref{eq_riskSplitUnknSTomo}) оценивается как~$\osml(1)$. Значит, $S_1/L^{3/2}$ сходится к нулю по вероятности.

Для оценки~$S_2$ используем другую модификацию неравенства Чебышёва:
\begin{equation*}
\p\left( \fr{|S_2|}{L^{3/2}} > \delta \right) \leq \fr{\e|S_2|}{\delta L^{3/2}} = 
\fr{\e\left[|S_2|/L^{1/2}\right]}{\delta L}\,.
\end{equation*}
Величину $\e|S_2|$ можно оценить сверху суммой математических ожиданий модулей сумм, входящих в~$S_2$, 
а эти суммы, в свою очередь,~--- суммой математических ожиданий входящих в них слагаемых.

По формуле полной вероятности для некоторого $0<\gamma<1$ получаем
\begin{multline*}
\p\left( |\Yljk| > \hTlj \right)={}\\
{}=\p\left( |\Yljk| > \hTlj \,|\, \hTlj \leqslant (1-\gamma)\solj\sqrt{2\ln 2^{2j}} \right)\times{}\\
{}\times \p\left( \hTlj \leq (1-\gamma)\solj\sqrt{2\ln 2^{2j}} \right) + {}\\
{}+ \p\left( |\Yljk| > \hTlj \,,\, \hTlj > (1-\gamma)\solj\sqrt{2\ln 2^{2j}} \right)\,.
\end{multline*}
В силу свойств~$\hsig$
\begin{multline*}
\p\left( \hTlj \leq (1-\gamma)\solj\sqrt{2\ln 2^{2j}} \right)={}\\
{}=\p\left(\hsig\leqslant (1-\gamma)^2\sigma^2\right)\leq{}\\
{}\leq\p\left(|\hsig-\sigma^2-\nu_L|\geqslant(2\gamma-\gamma^2)\sigma^2+\nu_L\right)\leq{}\\
{}\leq \fr{\D\hsig}{\left((2\gamma-\gamma^2)\sigma^2+\nu_L\right)^2}=\Obig\left(L^{-\beta}\right)
\end{multline*}
для достаточно большого~$L$. Далее
\begin{multline}
\p\left( |\Yljk| > \hTlj \,,\, \hTlj > (1-\gamma)\solj\sqrt{2\ln 2^{2j}} \right) \leq{}\\
{}\leq \p\left( |\Yljk| > (1-\gamma)\solj\sqrt{2\ln 2^{2j}} \right) = {}\\
{}=
\fr{ C }{ 2^{2j(1-\gamma)^2}\cdot 2^{j/2}\sqrt{j} }\,.
\label{eq_prbSplitGammaTomo}
\end{multline}
Теперь оцениваем математические ожидания компонентов сумм из~$S_2$ при делении на~$L^{1/2}$:
\begin{multline*}
\e\left[\fr{\left| \Yljk^2 - \hslj \right|}{L^{1/2}}\indYjkghTj\right] \leq {}\\
{}\leq\sqrt{ \e\left[\fr{ \left(\Yljk^2 - \hslj\right)^2 }{2^{2J}}\right]  \p\left( |\Yljk| > \hTlj \right) } 
\rightarrow 0\,;
\end{multline*}

\vspace*{-6pt}
\noindent
\begin{multline*}
\e\left[\fr{\left| \hslj + \hTlj^2 \right|}{L^{1/2}}\indYjkghTj\right] \leq{}\\
{}\leq 
\left( 2\ln 2^{2j} + 1 \right) 2^{j-J} \times{}\\
{}\times\sqrt{ \e\left(\hslz\right)^2 \p\left( |\Yljk| > \hTlj \right) } \rightarrow 0
\end{multline*}
при $j\geq j_M$ и $J\rightarrow\infty$.
Остальные слагаемые оцениваются аналогично. Итак, $S_2/L^{3/2}$ тоже сходится к нулю по вероятности.~$\square$

\smallskip

Как и в одномерном случае (см.~\cite{MarkinLimitDistr}), порядок знаменателя в~(\ref{eq_ConsistTomo32}) 
можно понизить, введя дополнительные ограничения на~$\nu_L$.

\medskip

\noindent
\textbf{Теорема 3.}
\textit{Пусть справедливы предположения о регулярности~$f$. Пусть $\hsig$~--- 
оценка дисперсии, $\e\hsig-\sigma^2=$\linebreak $=\nu_L=\Obig(L^{-\upsilon})$ и 
$\D\hsig=\theta_L=\Obig(L^{-\beta})$, $\upsilon$, $\beta>0$. Тогда
при любом $a>1/2-c$, $c=\min\left\{1/2, \upsilon, \beta/2\right\}$ и $L\rightarrow\infty$ выполнено
\begin{equation*}
\fr{\hat r(f)-r(f)}{L^{a+1}} \xrightarrow{\textsf{P}} 0\,.
\end{equation*}}
\medskip

\noindent
Д\,о\,к\,а\,з\,а\,т\,е\,л\,ь\,с\,т\,в\,о.
Заметим, что $0<c\leq 1/2$ и, стало быть, $a>0$. Так же, как и в доказательстве теоремы~2, 
разобьем $\hat r - r$ на те же суммы~$S_1$ и~$S_2$ (см.\ формулы~(\ref{eq_riskSplitSoTomo})
и~(\ref{eq_riskSplitStTomo})), только~$S_1$ запишем в виде
\begin{multline*}
S_1 = \sumljk\left(\Yljk^2-\e\Yljk^2\right) - \sumljk \left(\hslj-\slj\right) = {}\\
{}= \sumljk\left(\Yljk^2-\e\Yljk^2\right) - {}\\
{}-\sum\limits_\lambda \sum_j 2^{2j}\,2^j \left(\hslz-\solz^2\right)\,.
%\label{eq_riskSplitSoLimTomo}
\end{multline*}
Первая сумма при делении на~$L$ сходится по распределению к нормальному закону 
(см.\ разд.~\ref{sect_ConsitKnownSTomo}) и, следовательно, сходится по вероятности к нулю при делении 
на~$L^{a+1}$, где $a>0$. Вторая сумма пред\-став\-ля\-ет собой произведение 
$\left(\hsig-\sigma^2\right)$ и множителя, имеющего порядок $2^{3J}=L^{3/2}$. Легко видеть, что
\begin{equation*}
\fr{L^{3/2}\left(\hsig-\sigma^2\right)}{L^{a+1}} \xrightarrow{\textsf{P}} 0
\end{equation*}
при указанных в формулировке теоремы ограничениях на~$a$.

Покажем теперь, что $S_2/L^{a+1}$ сходится к нулю по вероятности. 
Обозначим $\varkappa = a-1/2+c>$\linebreak $>\;0$. В теореме~2 есть оценки для вероятности 
$\p\left( |\Yljk| > \hTlj \right)$:
\begin{multline}
\p\left( |\Yljk| > \hTlj \right) = {}\\
{}=\max\left\{ \fr{C_1}{2^{2J\beta}},\,\fr{C_2}{2^{2j(1-\gamma)^2}\cdot 2^{j/2}\sqrt{j}} \right\} 
\label{eq_ProbYghTOrdersTomo}
\end{multline}
для некоторого $0<\gamma<1$. При $J\rightarrow\infty$ имеем
\begin{multline}
\label{eq_restEstimConsistTomo1}
\fr{\e \left(\Yljk^2\right)^2 C_1/2^{2J\beta}}{L^{2a}} \leq \fr{C_3\cdot 2^{2j}
\cdot 2^{-2j\beta}}{2^{2J(1-2c+2\varkappa)}} ={}\\
{}= \fr{C_3\cdot 2^{2j}\cdot2^{2J\min\left\{1, 2\upsilon, \beta\right\}}}{2^{2J}\cdot 2^{2J\beta}\cdot 2^{4J\varkappa}}  \rightarrow 0,
\end{multline}

\columnbreak 
%\vspace*{-6pt}

\noindent
\begin{multline}
\fr{\e \left(\Yljk^2\right)^2 C_2\cdot 2^{-2j(1-\gamma)^2}\cdot 2^{-j/2}/\sqrt{j}}{L^{2a}} \leq {}\\
{}\leq
\fr{C_4\cdot 2^{2j}\cdot 2^{2J\min\left\{1, 2\upsilon, \beta\right\}}}{2^{2j(1-\gamma)^2+j/2}\cdot 2^{2J}\cdot 2^{4J\varkappa}\sqrt{j}} \rightarrow 0
\label{eq_restEstimConsistTomo2}
\end{multline}
для достаточно малого~$\gamma$. Отсюда имеем для произвольного $\delta>0$
\begin{multline*}
\p\left(\fr{\sumljk \Yljk^2 \indYjkghTj }{L^{a+1}}>\delta\right) \leq{}\\
{}\leq
\fr{\sumljk \e \left[\Yljk^2/L^a\right] \indYjkghTj  }{\delta L} \rightarrow 0
\end{multline*}
при $J\rightarrow\infty$ в силу неравенств Чебышёва и Коши--Бу\-ня\-ков\-ско\-го. Оценки для суммы 
с членами вида $\hslj \indYjkghTj$ получаются аналогично. А для сумм, в которые входят $\indYjkgTj$, 
оценки получены в теореме~1.~$\square$

Можно сформулировать и доказать теорему сходимости по распределению к нетривиальному пределу.

\medskip

\noindent
\textbf{Теорема 4.} 
\textit{Пусть справедливы предположения о регулярности~$f$. 
Пусть $\hsig$~--- оценка дисперсии, 
$\e\hsig-\sigma^2=\nu_L=\Obig(L^{-\upsilon})$ и 
$\D\hsig=\theta_L=\Obig(L^{-\beta})$, $\upsilon>0$, $\beta>1/2$. 
Пусть $\hsig$ не зависит от $\Yljk$ и $\sqrt{L}\left( \hsig - \sigma^2 \right) 
\Rightarrow \mathcal{N}\left(0,\,\Sigma^2\right)$ при $L\rightarrow\infty$, тогда
\begin{multline*}
\fr{\hat r(f)-r(f)}{ L \sqrt{ b_2 \left( \silz^4 + \siilz^4 + \siiilz^4 \right) } } \Rightarrow{}\\
{}\Rightarrow \mathcal{N}\left( 0, 1+\frac{ \left( \silz^2 + \siilz^2 + \siiilz^2 \right)^2 \Sigma^2 }{ d_2 \,\sigma^4 \left( \silz^4 + \siilz^4 + \siiilz^4 \right) } \right)\,,
\end{multline*}
где $b_2=2/(2^4-1)=2/15$, $d_2 = (2(2^3-1)^2)/(2^4-1)=$\linebreak $=98/15$.}

\medskip

\noindent
Д\,о\,к\,а\,з\,а\,т\,е\,л\,ь\,с\,т\,в\,о.
В теореме~3 было существенным наличие~$\varkappa>0$, которое давало сходимость к нулю 
в~(\ref{eq_restEstimConsistTomo1}) (в~(\ref{eq_restEstimConsistTomo2}) это несущественно). 
Сейчас же $\varkappa=0$, поэтому доказательство необходимо изменить.

Оценим $S_2$ более тонко. Имеем
\vspace*{-9pt}

\noindent
\begin{multline*}
\Yljk^2 \indYjkghTj - \e \Yljk^2 \indYjkgTj = {}\\[3pt]
{}= \Yljk^2 \indYjkghTj - \Yljk^2 \indYjkgTj +{}\\[3pt]
{}+ \Yljk^2 \indYjkgTj - \e \Yljk^2 \indYjkgTj\,.
\vspace*{-3pt}
\end{multline*}
\vspace*{-18pt}

\pagebreak

Вопрос о двух последних слагаемых решен в теореме~1. Рассмотрим два первых:
\begin{multline*}
\e \left| \Yljk^2 \indYjkghTj - \Yljk^2 \indYjkgTj \right| = {}\\
{}=\e \Yljk^2 \Ik_{\Tlj<|\Yljk|\leqslant\hTlj} +{}\\
{}+ \e \Yljk^2 \Ik_{\hTlj<|\Yljk|\leqslant\Tlj}\,.
\end{multline*}
При этом
\begin{multline*}
\e \Yljk^2 \Ik_{\Tlj<|\Yljk|\leqslant\hTlj} \leq \e \hTlj^2 \indYjkgTj \leq{}\\
{}\leq \sqrt{\fr{C\cdot j^2\cdot 2^{2j}}{2^{2j+j/2}\sqrt{j}}}\rightarrow 0,\quad J\rightarrow\infty\,,
\end{multline*}

%\vspace*{-3pt}

\noindent
и

%\vspace*{-3pt}
\noindent
\begin{multline}
\e \Yljk^2 \Ik_{\hTlj<|\Yljk|\leq\Tlj} \leq {}\\
{}\leq \Tlj^2 \e \Ik_{\hTlj<|\Yljk|\leq 
 \Tlj} \leq{}\\
{}\leq C j\cdot 2^{j}
\e \indYjkghTj\,.
\label{eq_FineRestEstimTomo}
\end{multline}
С учетом~(\ref{eq_ProbYghTOrdersTomo}) получаем, что
\begin{equation*}
\e \Yljk^2 \Ik_{\hTlj<|\Yljk|\leqslant\Tlj} \rightarrow 0
\end{equation*}
при $J\rightarrow\infty$ и $\beta>1/2$. Отметим, что, в отличие от работы~\cite{MarkinLimitDistr}, 
требование на~$\beta$ повысилось (там требовалось только $\beta>0$). Это является следствием роста дисперсии с 
ростом~$j$, которое выражается в наличии множителя~$2^j$ в~(\ref{eq_FineRestEstimTomo}). 
Аналогично получаем соотношения для~$\hTlj$:
\begin{multline*}
\e \hTlj^2 \Ik_{\Tlj<|\Yljk|\leq\hTlj} \leq \e \hTlj^2 \indYjkgTj \leq{}\\
{}\leq \sqrt{ \fr{ C j^2 \cdot 2^{2j} }{ 2^{2j+j/2}\sqrt{j} } } \rightarrow 0\,;
\end{multline*}

\vspace*{-12pt}

\noindent
\begin{multline*}
\e \hTlj^2 \Ik_{\hTlj<|\Yljk|\leq\Tlj} \leq{}\\
{}\leq \Tlj^2 \e\Ik_{\hTlj<|\Yljk|\leq\Tlj} \rightarrow 0\,.
\end{multline*}
Для $\hslj$ заметим, что $\hslj\leqslant\hTlj^2$. После применения неравенства Чебышёва получим, что 
$S_2/L$ сходится к нулю по вероятности.

В~$S_1$ оба слагаемых сходятся по распределению к нормальному закону и при этом независимы. 
Поэтому их сумма тоже сходится по распределению к нормальному закону. Осталось убедиться в 
правильности параметров. Имеем
\begin{multline*}
\sumljk \left(\hslj-\slj\right) = \sum\limits_\lambda \sum_j 2^{2j}\cdot 2^j \left(\hslz-\solz^2\right) = {}\\
{}= \left( \left\|\xi^{[1]}_{0,0,0}\right\|_2^2 + \left\|\xi^{[2]}_{0,0,0}\right\|_2^2 + 
\left\|\xi^{[3]}_{0,0,0}\right\|_2^2 \right)\times{}\\
{}\times \fr{2^{3J}-2^{3j_M}}{2^3-1} \left(\hsig-\sigma^2\right) = {}
\end{multline*}

\noindent
$$%\begin{multline*}
{}= \fr{  \silz^2 + \siilz^2 + \siiilz^2 }{ \sigma^2 }\, \fr{2^{3J}-2^{3j_M}}{7} \left(\hsig-\sigma^2\right)\,.\hfil\square
$$%\end{multline*}

%\columnbreak
\medskip

\noindent
\textbf{Замечание}. 
Если функция $f$ регулярная с параметром $\alpha\geq 1/4$, а $j_M\geq 4J/5$, то можно ослабить требования 
на~$\hsig$. Достаточно потребовать только состоятельность, асимптотическую нормальность и независимость от~$\Yljk$.

\smallskip

В теоремах~2 и~3 при оценке $\p\left( |\Yljk| > \hTlj \right)$ использовалось число $0<\gamma<1$. 
Можно заменить~$\gamma$ бесконечно малой последовательностью~$\gamma_L$, которая не испортит порядок знаменателя 
в~(\ref{eq_prbSplitGammaTomo}).

По формуле полной вероятности для любого $\delta>0$
\begin{multline}
\label{eq_sumOfIndicTomo}
\p\left( \sumljk\indYjkghTj > \delta \right) ={}\\
{}= \p\left( \hTlj \leqslant \left( 1-\gamma_L \right)\solj\sqrt{2\ln 2^{2j}} \right) \times{} \\
{}\times \p\left( \sumljk \indYjkghTj>\delta\,|\,\hTlj \leqslant{}\right.\\
\left.{}\vphantom{\sumljk\indYjkghTj}\leq \left(1-\gamma_L\right) \solj\sqrt{2\ln 2^{2j}} \right) +{} \\
{}+ \p\left(\sumljk\indYjkghTj>\delta\,,\right. \\
\left.\vphantom{\sumljk\indYjkghTj}\hTlj > \left( 1-\gamma_L \right) \solj\sqrt{2\ln 2^{2j}}\right)\,,
\end{multline}
где
$\gamma_L = 1/J$.
При таком $\gamma_L$ получаем
\begin{multline*}
\p\left( |\Yljk| > (1-\gamma_L)\solj\sqrt{2\ln 2^{2j}} \right) = {}\\
{}=\fr{C}{ 2^{2j(1-\gamma_L)^2}\cdot 2^{j/2}\sqrt{j} } \leq \fr{C_1}{ 2^{2J} \sqrt{j} }
\end{multline*}
в силу выбора $j_M$ и того, что
\begin{equation*}
2^{2j\left(1-\gamma_L\right)^2} = 2^{2j\left( 1-2/J + 1/J^2 \right)} > 2^{2j-4}\,.
\end{equation*}
По неравенству Чебышёва
\begin{multline*}
\p\left(\sumljk\indYjkghTj>\delta\,, \right.\\
\left. \vphantom{\sumljk\indYjkghTj}\hTlj > \left( 1-\gamma_L \right) \solj\sqrt{2\ln 2^{2j}}\right) \leq{}\\
{}\leq \p\left( \sumljk \Ik_{ |\Yljk| > (1-\gamma_L)\solj\sqrt{2\ln 2^{2j}} } > \delta \right) \leq{}
\end{multline*}

\noindent
\begin{multline*}
{}\leq \fr{ \sumljk \p\left( |\Yljk| > (1-\gamma_L)\solj\sqrt{2\ln 2^{2j}} \right) }{\delta} = {}\\
{}=\Obig\left( \fr{1}{\sqrt{j}} \right)\,.
\end{multline*}

Используя свойство асимптотической нормальности~$\hsig$, можно для любого $\delta'>0$ оценить
\begin{equation}
\label{eq_hatTdevProbAsympTomo}
\p\left( \hTlj \leq (1-\gamma_L)\solj\sqrt{2\ln 2^{2j}} \right)< \delta'\,,
\end{equation}
причем отметим, что~$\delta$ здесь фиксировано, а~$\delta'$ можно делать произвольно малым. Имеем
\begin{multline*}
%\label{eq_hatTdevProbAsymp}
\p\left(\hTlj\leqslant(1-\gamma_L)\solj\sqrt{2\ln 2^{2j}}\right) ={}\\
{}= \p\left(\hsig\leqslant(1-\gamma_L)^2\sigma^2\right)={}\\
{}= \p\left( \left(\hsig-\sigma^2\right) \leqslant \sigma^2\left(-2\gamma_L+\gamma_L^2\right) \right) ={}\\
{}= \p\left( \sqrt{L}\left(\hsig-\sigma^2\right) \leqslant -\fr{\sqrt{L}\sigma^2(2J-1)}{J^2} \right)\,.
\end{multline*}
Для произвольного~$\delta'>0$ найдется $J_0$ ($L_0=2^{2J_0}$) такое, что
\begin{equation*}
F_\Sigma\left( -\fr{\sqrt{L_0}\sigma^2(2J_0-1 )}{J_0^2} \right) < \fr{\delta'}{2}\,,
\end{equation*}
где $F_\Sigma$~--- функция распределения нормального закона с нулевым средним и дисперсией~$\Sigma^2$. 
При этом для любого $J\geq J_0$
\begin{multline*}
\p\left( \sqrt{L}\left(\hsig-\sigma^2\right) \leq -\fr{\sqrt{L}\sigma^2(J -1 )}{J^2} \right) \leq{} \\
{}\leq \p\left( \sqrt{L}\left(\hsig-\sigma^2\right) \leq -\fr{\sqrt{L_0}\sigma^2(2J_0 -1 )}{J_0^2}  \right)\,.
\end{multline*}
В силу асимптотической нормальности~$\hsig$ и непрерывности~$F_\Sigma$ для этого же~$\delta'$ 
найдется~$J_1$ $\left(L_1=2^{2J_1}\right)$ такое, что для любого $J\geq J_1$
\begin{equation*}
\left| \p\left( \sqrt{L_1}\left(\hsig-\sigma^2\right) \leq x \right) - F_\Sigma(x)\right| < \fr{\delta'}{2}\,,
\end{equation*}
причем $J_1$ не зависит от~$x$. Возьмем $x_0 =$\linebreak $= -\sqrt{L_0}\sigma^2(2J_0-1)/J_0^2$ и 
$J_2 = \max\{J_0,J_1\}$. Для любого $J\geq J_2$ имеем
\begin{equation*}
\p\left( \sqrt{L}\left(\hsig-\sigma^2\right) \leq x_0 \right) < \delta'\,,
\end{equation*}
а значит, справедливо~(\ref{eq_hatTdevProbAsympTomo}).

Получаем, что сумма индикаторов в~(\ref{eq_sumOfIndicTomo}) сходится к нулю по вероятности:
\begin{equation*}
%\label{eq_sumIndConsisthTTomo}
\p\left( \sumljk\indYjkghTj > \delta \right) \rightarrow 0 \mbox{ при }J\rightarrow\infty\,.
\end{equation*}
Для суммы индикаторов с неслучайным порогом аналогично получаем
\begin{equation*}%\label{eq_sumIndConsistTTomo}
\p\left( \sumljk\indYjkgTj > \delta \right) \rightarrow 0\,.
\end{equation*}
Далее воспользуемся дискретной версией неравенства Коши--Буняковского:
\begin{multline*}
\fr{ \sumljk \Yljk^2\indYjkghTj }{ L } \leq{}\\
{}\leq \sqrt{ \fr{\sumljk \Yljk^4/L}{L} \, \sumljk \indYjkghTj } \,\xrightarrow{\textsf{P}} 0\,,
\end{multline*}
так как $\e\left[ \Yljk^4/L \right]$ ограничено,
\begin{equation*}
\fr{ \sumljk \hTlj^2 \indYjkghTj }{L} \leq \fr{ \sumljk \Yljk^2 \indYjkghTj }{L} \xrightarrow{\mathsf{P}} 0
\end{equation*}
и
\begin{equation*}
\hslj\indYjkghTj \leqslant \hTlj^2\indYjkghTj\,.
\end{equation*}
Оценки для слагаемых с $\indYjkgTj$ получены в теореме~1.

\medskip

\noindent
\textbf{Замечание}. 
Всюду выше в этом разделе предполагалось, что пороговая обработка и суммирование в выражении для 
риска~(\ref{eq_riskEstimDefTomo}) ведутся с уровня~$j_M$, причем $j_M\rightarrow\infty$ при 
$J\rightarrow\infty$. Однако если ввести дополнительные ограничения на регулярность~$f$, 
то можно вести пороговую обработку и суммирование с уровня $j_0\nrightarrow\infty$. Если 
$j_M=J/(\alpha+1)$, то для коэффициентов, соответствующих $j<j_M$, неравенство~(\ref{eq_WaveletCoeffUpperBoundTomo}), 
вообще говоря, не выполнено. Оценим вклад больших коэффициентов в оценку риска:
\begin{multline*}
L^{-1} \sum\limits_{j=j_0}^{j_M-1}\sum\limits_{\lambda,\mathbf{k}} \left\{\left|\Yljk^2-\hslj\right|\indYjklhTj +{}\right.\\
\left.{}+ \left(\hslj+\hTlj^2\right)\indYjkghTj \right\} \leq{}\\
{}\leq L^{-1} \sum\limits_{j=j_0}^{j_M-1}\sum\limits_{\lambda,\mathbf{k}} \left\{ \left(\hslj+\hTlj^2\right) +
 \left(\hslj+\hTlj^2\right) \right\} \xrightarrow{\mathsf{P}}{}\\
\xrightarrow{\mathsf{P}} {} 0
\end{multline*}
в силу состоятельности~$\hsig$ и того, что

\noindent
\begin{multline*}
L^{-1}\left\{\sum\limits_{j=j_0}^{j_M-1}j2^j\cdot2^{2j}\right\} \leq 2^{-2J}
\left\{j_M\sum\limits_{j=j_0}^{j_M-1}2^{3j}\right\} \simeq{}\\
{}\simeq 2^{2J}\cdot j_M\cdot2^{3j_M}\rightarrow 0
\end{multline*}
при $J\rightarrow\infty$, если $3j_M<2J$, т.\,е.\ 
$\alpha>1/2$. Слагаемые риска оцениваются аналогично. Итак, 
при $\alpha>1/2$ суммирование в~(\ref{eq_riskEstimDefTomo}) можно начинать с произвольного~$j_0$.


{\small\frenchspacing
{%\baselineskip=10.8pt
\addcontentsline{toc}{section}{Литература}
\begin{thebibliography}{99}

\bibitem{Natterer} %1
\Au{Наттерер Ф.} 
Математические аспекты компьютерной томографии.~--- М.: Мир, 1990.

\bibitem{TikhonovArsenin}  %2
\Au{Тихонов А.\,Н., Арсенин В.\,Я.} 
Методы решения некорректных задач.~--- М.: Наука, 1979.

\bibitem{Herman}  %3
\Au{Хермен Г.} 
Восстановление изображений по проекциям: основы реконструктивной томографии.~--- М.: Наука, 1983.

\bibitem{Daub}  %4
\Au{Добеши И.} 
Десять лекций по вейвлетам.~--- Ижевск: НИЦ <<Регулярная и хаотическая динамика>>, 2001.

\bibitem{DonohoWVD} 
\Au{Donoho D.\,L.} 
Nonlinear solution of linear inverse problems by wavelet-vaguelette decomposition~// 
Appl. Comput. Harmonic Anal., 1995. Vol.~2. P.~101--126.

\bibitem{KolaczykArticle}  %7
\Au{Kolaczyk E.\,D.} 
A wavelet shrinkage approach to tomographic image reconstruction~// J. Amer. Statistical Association, 1996. 
Vol.~91. No.\,435. P.~1079--1090.

\bibitem{KolaczykThesis} %6
\textit{Kolaczyk E.\,D.} 
Wavelet methods for the inversion of certain homogeneous linear operators in the presence of noisy data.  Ph.D.\ 
Thesis, 1994.

\bibitem{DJideal} 
\textit{Donoho D.\,L., Johnstone I.\,M.} 
Ideal spatial adaptation via wavelet shrinkage~// Biometrika, 1994. Vol.~81. No.\,3. P.~425--455.

\bibitem{DJunkn}  %9
\textit{Donoho D.\,L., Johnstone I.\,M.} 
Adapting to unknown smoothness via wavelet shrinkage~// J. Amer.\ Statistical Association, 1995. Vol.~90. P.~1200--1224.

\bibitem{Mallat} %10
\Au{Mallat S.} 
A wavelet tour of signal processing.~--- Academic Press, 1999.


\bibitem{MarkinLimitDistr}  %11
\Au{Маркин А.\,В.} 
Предельное распределение оценки риска при пороговой обработке вейвлет-ко\-эф\-фи\-ци\-ен\-тов~// 
Информатика и её применения, 2009. Т.~3. Вып.~4. С.~57--63.

\label{end\stat}

\bibitem{MarkinShestakovConsist}  %12
\Au{Маркин А.\,В., Шестаков О.\,В.} 
О состоятельности оценки риска при пороговой обработке вейвлет-ко\-эф\-фи\-ци\-ен\-тов~// Вестник Московского университета. 
Сер.~15. Вычислительная математика и кибернетика, 2010. №\,1. С.~26--33.


 \end{thebibliography}
}
}

\end{multicols}

\include{chernikov} 

\def\stat{kirikov}

\def\tit{<<ВИРТУАЛЬНЫЙ КОНСИЛИУМ>>~--- ИНСТРУМЕНТАЛЬНАЯ 
СРЕДА ПОДДЕРЖКИ ПРИНЯТИЯ 
  СЛОЖНЫХ ДИАГНОСТИЧЕСКИХ РЕШЕНИЙ$^*$}

\def\titkol{<<Виртуальный консилиум>>~--- инструментальная 
среда поддержки принятия сложных диагностических решений}

\def\aut{И.\,А.~Кириков$^1$, А.\,В.~Колесников$^2$, С.\,В.~Листопад$^3$, 
С.\,Б.~Румовская$^4$}

\def\autkol{И.\,А.~Кириков, А.\,В.~Колесников, С.\,В.~Листопад, 
С.\,Б.~Румовская}

\titel{\tit}{\aut}{\autkol}{\titkol}

\index{Кириков И.\,А.}
\index{Колесников А.\,В.}
\index{Листопад С.\,В.} 
\index{Румовская С.\,Б.}
\index{Kirikov I.\,А.}
\index{Kolesnikov А.\,V.}
\index{Listopad S.\,V.}
\index{Rumovskaya S.\,B.}


{\renewcommand{\thefootnote}{\fnsymbol{footnote}} \footnotetext[1]
{Работа выполнена при частичной поддержке РФФИ (проект 16-07-00272 А).}}


\renewcommand{\thefootnote}{\arabic{footnote}}
\footnotetext[1]{Калининградский филиал Федерального исследовательского центра <<Информатика и~управление>> 
Российской академии наук, \mbox{baltbipiran@mail.ru}}
\footnotetext[2]{Балтийский Федеральный университет
имени  И.~Канта, Калининградский филиал Федерального 
исследовательского центра <<Информатика и~управление>> Российской академии наук, 
\mbox{avkolesnikov@yandex.ru}}
\footnotetext[3]{Калининградский филиал Федерального исследовательского центра <<Информатика и~управление>> 
Российской академии наук, \mbox{ser-list-post@yandex.ru}}
\footnotetext[4]{Калининградский филиал Федерального исследовательского центра <<Информатика 
и~управление>> Российской академии наук, \mbox{sophiyabr@gmail.com}}
 
 \vspace*{-3pt}
 
  \Abst{Рассматривается проблема принятия индивидуального решения при диагностике 
пациентов в~ам\-бу\-ла\-тор\-но-по\-ли\-кли\-ни\-че\-ских учреждениях на примере 
диагностики артериальной гипертензии (АГ). Предлагается повысить качество принятия 
индивидуального решения за счет консультаций с~системой поддержки принятия  
решения~--- <<Виртуальным консилиумом>>, моделирующим коллективный интеллект 
врачей стационара многопрофильного больничного учреждения. Приведены результаты 
проектирования и~экспериментального исследования лабораторного прототипа 
<<Виртуального консилиума>>.}

  \KW{система поддержки принятия решения; виртуальный консилиум; функциональная 
гибридная интеллектуальная система}

\DOI{10.14357/19922264160311} 


\vskip 10pt plus 9pt minus 6pt

\thispagestyle{headings}

\begin{multicols}{2}

\label{st\stat}
  

\section{Введение}

  Степень исследования, понимания и~качества диагностики проблемных сред и~их 
окружения отражена в~научной картине мира, онтологи\-зи\-ру\-ющей его представления 
и~делающей рассуждения и~целенаправленную деятельность <<зависимыми>> от них. 
В~искусственном интеллекте понятию <<картина мира>> соответствует понятие <<модель 
внешнего мира>> М.\,Г.~Га\-азе-Рап\-по\-пор\-та и~Д.\,А.~Поспелова~[1]. 
  
  Новая картина мира складывается из многочисленных теорий и~взглядов: <<ноосфера>>, 
<<разумный мир>> (В.\,И.~Вернадский, Н.\,Н.~Моисеев, А.\,В.~Поздняков); <<мир 
диалектики>>~--- мир диалога разных логик (Е.\,Л.~Доценко); социальная парадигма 
искусственного интеллекта (<<The society of mind>>) М.~Минского;  
сис\-тем\-но-ор\-га\-ни\-за\-ци\-он\-ный подход в~искусственном интеллекте 
В.\,Б.~Тарасова; теория иерархических многоуровневых систем М.~Месаровича, Д.~Мако 
и~И.~Такахары и~др.~--- и~укладывается в~семь постулатов~[2]: (1)~признание 
гетерогенности мира и~любого объекта, разнообразия жизни; (2)~неопределенность границ 
объектов и~связь <<всего со всем>>; (3)~относительность любой иерархии и~горизонтальные 
связи; (4)~дополнительность и~сотрудничество; (5)~полицентризм; (6)~относительность 
знания; (7)~соответствие управления сложности объекта. 
  
  Сложная задача диагностики АГ (СЗДАГ)~---
  за\-да\-ча-сис\-те\-ма, вклю\-ча\-ющая диагностические и~технологические подзадачи, 
повышающие эффективность обработки симптоматической информации о пациенте. 
Разнообразие подзадач СЗДАГ с~различными характеристическими свойствами требует 
разнообразия соответствующих методов принятия решений, системного анализа, 
искусственного интеллекта и~инженерии знаний. 
  
  Анализ результатов влияния новой картины мира на ментальную составляющую 
врачебной практики и~медицинской информатики~[3] показал, что, несмотря на стремление 
биомедицины к~гетерогенности восприятия организма человека и~процесса его диагностики 
в~рамках семипостулатной картины мира, человек по-преж\-не\-му остается 
<<расчлененным>> объектом познания, что сформировало <<узких>> специалистов, 
поглощенных решением частных задач. Новый тип ученого <<праг\-ма\-ти\-ка-фак\-то\-ло\-га>> 
утратил системное мышление, перестал задумываться над тем, что делается <<вокруг>> 
и~какое значение могут иметь добытые им факты для понимания работы организма в~целом. 
В~этой связи\linebreak\vspace*{-12pt}

\pagebreak

\noindent
 очевидна необходимость перехода от методов <<конкурентной>> диагностики 
к системному мышлению и~методам гетерогенной диагностики.
  
  В~[3--5] представлены результаты системного анализа СЗДАГ, следуя 
  проблемно-структурной (ПС) методологии, этапы~1--5~[6]: идентификация, редукция сложной задачи, 
спецификация диагностических подзадач, выбор методов их решения, а~также проверка 
неоднородности сложной задачи диагностики. Работы~[3--5] подтвердили релевантность 
применения междисциплинарных инструментариев для решения 
СЗДАГ, мо\-де\-ли\-ру\-ющих разнообразие информации, 
сотрудничество, дополнительность и~относительность знаний, сочетающих методы 
и~методики системного анализа диагностической проблемы с~динамическим синтезом 
метода ее решения и~имитацией работы искусственного гетерогенного коллектива~--- 
<<виртуального консилиума>>.
  
  Разнообразие~--- признак, проявление гетерогенности. Следствие закона необходимого 
разнообразия У.\,Р.~Эшби констатирует, что управ\-ле\-ние обеспечивается, если разнообразие 
средств управ\-ля\-юще\-го не меньше разнообразия управ\-ля\-емой им ситуации. Для отображения 
в информатике ситуативного разнообразия в~естественных гетерогенных системах в~[6] 
введены модели <<гетерогенная, неоднородная задача>> и~<<гомогенная, однородная 
задача>>, а~сам закон трактуется так: только разнообразная, скоординированная клиническая 
деятельность, элементы которой в~комбинации решают одну задачу, сделает результат 
диагностики качественно лучше в~обществе с~новой научной картиной мира. Специфике 
такой работы соответствует коллективный труд экспертов в~малых группах за круглым 
столом~--- консилиумы, совещания, естественные гетерогенные системы для решения 
сложных задач~\cite{3-kir}, где на первый план выходят знания и~опыт лица, принимающего 
решения (ЛПР), и~экспертов.
  
  \begin{figure*} %fig1
\vspace*{1pt}
 \begin{center}  
\mbox{%
 \epsfxsize=147.497mm
 \epsfbox{kir-1.eps}
 }
\end{center} 
%\vspace*{-9pt}
%\Caption{Концептуальная модель процесса диагностики артериальной гипертензии: в~многопрофильном 
%стационарном больничном учреждении~(\textit{а}); в~амбулаторно-поликлиническом~(\textit{б})}
  \end{figure*}

  \addtocounter{figure}{1}
  
  Настоящая работа~--- продолжение работ~[3--5,\linebreak 7] и~имеет целью представить: (1)~результаты 
исследования процесса диагностики АГ  
в~ле\-чеб\-но-про\-фи\-лак\-ти\-че\-ских больничных учреждениях (ЛПУ) широкого 
профиля~--- предлагается повысить эффективность и~качество индивидуальных 
диагностических решений в~ЛПУ широкого профиля ам\-бу\-ла\-тор\-но-по\-ли\-кли\-ни\-че\-ско\-го 
характера (рис.~1,\,\textit{а}) за счет внедрения информационной технологии 
<<Виртуальный консилиум>>, моделирующей коллективное обсуждение; 
(2)~архитектуру <<Виртуального консилиума>> и~результаты лабораторных экспериментов с~
его интегрированными моделями (первые результаты лабораторных экспериментов 
приведены в~[7]).

\section{Диагностика артериальной гипертензии в~многопрофильном 
стационарном больничном учреждении и~в~амбулаторно-поликлиническом 
учреждении}

\vspace*{-9pt}


  В~[8, 9] представлены результаты исследования процесса диагностики 
АГ в~Калининградской клинической областной больнице (КОКБ) 
(см.\ рис.~1,\,\textit{б}) и~ее Диагностическом центре (см.\ рис.~1,\,\textit{а}). 

Для формирования 
полного дифференциального диагноза АГ коллективом врачей во главе с~лечащим врачом, 
ЛПР-кар\-дио\-ло\-гом, в~стационаре привлекаются до тринадцати вра\-чей-экс\-пер\-тов~--- носителей 
знаний из различных разделов медицины: невролог, нефролог, сосудистый хирург, уролог, 
психолог, педиатр, аку\-шер-ги\-не\-ко\-лог, онколог, окулист, врачи функциональной 
диагностики, эндокринолог, терапевт, кардиолог. 

Для исследований выбраны шесть 
специалистов (см.\ рис.~1,\,\textit{б}), решающих двенадцать функциональных подзадач 
(рис.~\ref{f2-kir}), возникающих в~90\%~случаев диагностики АГ, 
каждый из которых формирует промежуточные заключения о~состоянии объекта 
диагностики в~своей области медицинских зна\-ний. 
{\looseness=1

}

Полученные исходные данные об объекте 
диагностики разнородны (содержатся в~медицинской карте): количественные,  
ви\-зу\-аль\-но-графиче\-ские параметры (детерминированные переменные),\linebreak 
лингвистические четкие и~нечеткие переменные. Лицо, при\-ни\-ма\-ющее решение, изучает в~медицинской карте 
симптомы и~частные диагностические мнения вра\-чей-экс\-пер\-тов, множество которых 
подбирает сам, и~ставит заключительный диагноз. Вра\-чам-экс\-пер\-там доступны симптомы 
и~мнения других врачей-экспертов из медицинской карты.
\mbox{Лицо}, при\-ни\-ма\-ющее решение, и~вра\-чи-экс\-пер\-ты 
обследуют пациента и~формируют диагностические заключения согласно нормативным 
документам, например~[10]. В~ЛПУ широкого профиля (см.\ рис.~1,\,\textit{а}) ЛПР~--- это врач 
общей практики или терапевт (иногда кардиолог, но зачастую без опыта работы, к~которому 
направляет терапевт сразу же при выявлении повышенного артериального давления), это 
врач <<праг\-ма\-тик-фак\-то\-лог>>~\cite{9-kir}, объединяющий в~себе роли вра\-ча-ЛПР  
и~вра\-чей-экс\-пер\-тов узкой специализации.

\end{multicols}

\begin{figure} %fig2
\vspace*{1pt}
 \begin{center}  
\mbox{%
 \epsfxsize=163.044mm
 \epsfbox{kir-2.eps}
 }
\end{center} 
\vspace*{-9pt}
\Caption{Архитектура ВКДАГ }
\label{f2-kir}
\vspace*{3pt}
\end{figure}

\begin{multicols}{2}
  

  Исследования диагностического процесса на материалах Диагностического центра КОКБ 
по модели на рис.~1,\,\textit{а} показали, что~70\%~пациентов с~АГ 
амбулаторно-поликлинического учреждения не знают о своем заболевании, в~то время как в~стационарных 
медицинских учреждениях (см.\ рис.~1,\,\textit{б}) практически в~100\%~случаев имеет место 
как адекватное проведение, так и~отображение в~медицинских картах симптоматических 
данных обследования с~подтверждением диагноза  
ла\-бо\-ра\-тор\-но-ин\-ст\-ру\-мен\-таль\-ны\-ми методами исследования. 
  
  В этой связи предлагается повысить эффективность и~качество индивидуальных 
диагностических решений в~ЛПУ широкого профиля амбула\-тор\-но-по\-ли\-кли\-ни\-че\-ско\-го 
характера (см.\ рис.~1,\,\textit{а}) за счет внед\-ре\-ния информационной технологии 
<<Виртуальный консилиум>> (см.\ рис.~\ref{f2-kir}), моделирующей коллективное обсуждение, 
обладающего синергией, опытом и~знаниями в~решении подзадач диагностики 
АГ в~стационаре (см.\ рис.~1,\,\textit{б}). 


  

  
\section{Инструментальная среда <<Виртуальный консилиум для~диагностики 
артериальной гипертензии>>}

\vspace*{-18pt}

  Инструментальная среда <<Виртуальный консилиум>>, архитектура которой 
представлена на рис.~\ref{f2-kir}, а~структура в~\cite{7-kir}, ограничена пациентами 
стар\-ше~18~лет, без особых состояний, нет распознавания снимков, не предусматривается 
назначение лечения и~не диагностируется ряд симптоматических артериальных гипертензий. 

Архитектура <<Виртуального консилиума для диагностики артериальной гипертензии>> 
(ВКДАГ) включает межмодульные интерфейсы~$\zeta^u$ для модулей, реализованных 
посредством различных методологий гибридных интеллектуальных сис\-тем (\mbox{ГиИС}) 
(генетические алгоритмы ($g$), нечеткие 
сис-\linebreak\vspace*{-12pt}

\pagebreak

\end{multicols}

\begin{table*}\small
%\vspace*{-12pt}
\begin{center}
\Caption{Описание блоков архитектуры ВКДАГ}
\vspace*{2ex}

\begin{tabular}{|p{30mm}|p{40mm}|p{39mm}|p{39mm}|}
\hline
\multicolumn{1}{|c|}{Наименование блока}&\multicolumn{1}{c|}{Функции}&\multicolumn{1}{c|}{Вход}&\multicolumn{1}{c|} 
{Выход}\\
\hline
Технологический модуль $i$-й&
Организация эффективной обработки данных и~знаний, выбирается для 
включения в~функциональную \mbox{ГиИС}~--- построение информативного набора 
признаков для диагностики&Популяция 
индивидуумов, накладывающихся как маска на $i$-й функциональный модуль&
Наилучшая особь с~оптимальным набором признаков~--- накладывается как 
маска на $i$-й функциональный модуль\\
\hline
Функциональный модуль $i$-й&Классификация состояния здоровья пациента в~рамках 
\mbox{$i$-й} диагностической 
подзадачи, выбирается для включения в~функциональную \mbox{ГиИС} &
Подмножество $i$-е симптомов с~интерфейса 
пользователя&Частное $i$-е заключение о~со\-сто\-янии здоровья пациента\\
\hline
Функциональный модуль {HCCCC}, моделирующий ЛПР&
Формирование заключительного диагноза 
АГ (всегда в~составе <<Виртуального консилиума>>)&Подмножество симптомов 
с~интерфейса пользователя, множество выходов функциональных модулей&
Заключительный диагноз АГ \\
\hline
Функциональный модуль {ИНСРЭКГ}&Классификация патологического состояния пациента по его 
электрокардиограмме&\multicolumn{2}{p{60mm}|}{Рассмотрены подробно в~\cite{4-kir}}\\
\cline{1-2}
Функциональный модуль {ИНССМАД}&Прогноз нормальных зна\-чений суточного мониторирования 
артериального давле\-ния и~вычисление отклонения &\multicolumn{2}{c|}{\ }\\
\hline
Интерфейс модификации структуры {ВКДАГ}&Исключение из диагностики модулей, решающих не 
интересующие пользователя подзадачи &
Выбранные пользователем подзадачи диагностики &
Функциональная ГиИС, 
синтезированная посредством алгоритма из~\cite{4-kir}\\
\hline
Интерфейс пользователя <<Диагноз>>&Визуализация результатов диагностики и~корректировка их 
пользователем &Заключительный диагноз от функционального модуля НСССС&Отчет, содержащий 
множество симптомов и~диагноз\\
\hline
Интерфейс пользователя &Ввод информации о~со\-сто\-янии здоровья пациента &
Множество значений 
показателей состояния здоровья пациента&
Показатели состояния здоровья пациента, распределенные по 
функциональным модулям \\
\hline
Модификация интерфейса пользователя&Деактивация элементов на интерфейсе пользователя для ввода 
значений показателей состояния здоровья&Множество выходов технологических модулей&Частично 
деактивированный интерфейс пользователя \\
\hline
\end{tabular}
\end{center}
\end{table*}

\begin{multicols}{2}

\noindent 
те\-мы ($f$), искусственные нейронные сети ($n$)).
 В~библиотеке модулей диагностики 
и~препро\-цессии хранятся заранее инициализированные\linebreak в~программной среде 
функциональные и~технологические модели. 
По умолчанию все модули включены 
в~структуру <<Виртуального консилиума>>, их описание пред\-став\-ле\-но в~табл.~1. %\\[-15pt]
%
      <<Виртуальный консилиум>> (см.\ рис.~\ref{f2-kir}) запускает интерфейс пользователя, 
ЛПР-вра\-ча~--- <<{Интерфейс модификации структуры ВКДАГ}>>, посредством 
которого включаются функциональные 
 и~технологические модули в~работу сис\-те\-мы: модуль 
<<Анализ СМАД>>, модуль <<Распознавание ЭКГ>>, модули технологических подзадач из 
группы <<Построение информативного набора признаков\linebreak (симптомов) при диагностике 
заболеваний>> и~модули подзадач из группы <<Диагностика критериев оценки 
сер\-деч\-но-со\-су\-ди\-сто\-го риска и~вторичной АГ у~пациента>> ({ДАГ}$_1$, \ldots , {ДАГ}$_9$): 
диагностики\linebreak поражений ор\-га\-нов-ми\-ше\-ней, факторов риска, цереброваскулярных 
болезней, метаболического синд\-ро\-ма и~сахарного диабета, заболеваний периферических 
артерий, ишемической болезни сердца,\linebreak эндокринной АГ, паренхиматозной нефропатии 
и~реноваскулярной АГ соответственно. Все выбранные $i$-е технологические модули 
запускаются, решают соответствующую подзадачу и~передают информацию на блок 
<<{Модификация интерфейса пользователя}>>. Он деактивирует показатели 
со\-сто\-яния здоровья на <<{Интерфейсе пользователя для\linebreak ввода значений показателей 
состояния здоровья пациента}>> и~корректирует работу $i$-го функционального модуля 
подзадач {ДАГ}$_1$, \ldots\linebreak \ldots , {ДАГ}$_9$. Далее активируется откорректированный 
интерфейс, вводятся симптомы, которые передаются функциональным нечетким модулям, 
решающим подзадачи {ДАГ}$_1$, \ldots , {ДАГ}$_9$\linebreak (моделируют принятие 
решения экспертами, врачами смежных специальностей~--- кардиологом как экспертом, 
неврологом, нефрологом, терапевтом, эндокринологом, урологом). Последние в~свою 
очередь передают информацию о~патологиях, выявленных ими у~пациента, 
функциональному модулю {НСССС} (моделирует принятие решения ЛПР~---  
вра\-чом-кар\-дио\-ло\-гом), решающему подзадачу <<Оценка степени и~стадии 
артериальной гипертензии, степени риска сер\-дечно-сосу\-ди\-стых заболеваний>>. 

В~библиотеке ВКДАГ есть еще два функциональных модуля (см.\ табл.~1), вклю\-ча\-ющих\-ся 
в~работу консилиума посредством <<{Интерфейса модификации структуры 
ВКДАГ}>>: 
      \begin{enumerate}[(1)]
      \item {ИНСРЭКГ}, передающий информацию на модули диагностики поражений 
ор\-га\-нов-ми\-ше\-ней (на рис.~\ref{f2-kir}~--- это {НСДАГ}$_1$), цереброваскулярных 
болезней ({НСДАГ}$_3$) и~ишемической болезни сердца ({НСДАГ}$_6$); 
      \item {ИНССМАД}, формирующий информацию о~нормальных значениях 
суточного артериального давления на функциональный модуль {НСССС}.
      \end{enumerate}
      
\section{Экспериментальное лабораторное исследование программной 
реализации прототипа инструментальной среды <<Виртуальный консилиум>>}
  
  Экспериментальное лабораторное исследование программной реализации 
исследовательского прототипа функциональной гибридной интеллектуальной системы 
ВКДАГ для поддержки принятия сложных диагностических решений необходимо для 
подтверждения его релевантности~[3--5, 7] реальной ситуации диагностики АГ. В~[4] 
пред\-став\-ле\-на информация по особенностям функциональных и~технологических моделей 
гетерогенного модельного поля ВКДАГ, а~в~[7]~--- информация по их инициализации 
в~среде MATLAB-Simulink, результаты исследований качества работы каждой модели 
гетерогенного модельного поля <<Виртуального консилиума>> автономно, а~также 
подтверждена их релевантность работе экспертов~--- врачей узкой специализации, что 
предотвращает распространение ошибок работы автономных моделей на работу 
интегрированной модели. 

В~настоящей работе приведены результаты исследования качества 
интегрированных моделей, синтезированных <<Виртуальным консилиумом>>\linebreak 
и~моделирующих дополнительность и~сотрудничество, которые имитируют коллективные 
рас\-суж\-де\-ния специалистов при постановке диагноза. 

В~табл.~2 представлены критерии 
и~результаты тес\-ти\-ро\-ва\-ния интегрированных моделей <<Виртуального консилиума>> 
с~различными комбинациями знаний врачей, классифицирующих патологическое состояние 
пациента. Порядок работы моделей гетерогенного модельного поля \mbox{ВКДАГ}: запускаются 
модели первой очереди~--- модели технологических элементов {ГАППС}$_{1\mbox{--}9}$, 
корректирующие множества входных переменных моделей {НСДАГ}$_{1\mbox{--}9}$ 
и~{НСССС}; обработка информации передается функциональным элементам: модели 
второй очереди <<отправляют>> информацию на модели третьей, пятой, шес\-той и~седьмой 
очередей~--- \mbox{ИНСРЭКГ} (модель, решающая задачу распознавания электрокардиограммы (ЭКГ)), 
{ИНССМАД} (формирует оптимальные множества показателей суточного давления), 
{НСДАГ}$_9$, {НСДАГ}$_2$ и~{НСДАГ}$_6$; третья\linebreak очередь содержит 
модели НСДАГ$_4$ и~НСДАГ$_5$, передающие выходную информацию на вход моделей четвертой 
и~седьмой очередей; четвертая очередь содержит модель {НСДАГ}$_8$, пе\-ре\-да\-ющую 
информацию  модели пятой очереди {НСДАГ}$_1$, которая в~свою очередь передает 
информацию\linebreak {НСДАГ}$_3$ (шес\-тая очередь); от {НСДАГ}$_3$ передается 
информация {НСДАГ}$_7$ (седьмая очередь); последней запускается модель 
{НСССС}, формирующая заключительный диагноз, на вход которой передается 
выходная информация функциональных моделей вто\-рой--седь\-мой очередей.
  
  Таким образом: (1)~без знаний кардиолога, или нефролога, или эндокринолога 
сред\-не\-квад\-ратическая ошибка наибольшая~--- 0,697; 0,448 и~0,211 соответственно, 
и~объясняется это тем, что кардиолог играет ключевую роль в~обработке ин\-формации, 
поступающей от других врачей\linebreak\vspace*{-12pt}


\pagebreak

\end{multicols}

\begin{table}\small
\begin{center}
\Caption{Параметры и~результаты тестирования интегрированных моделей }
\vspace*{2ex}

\begin{tabular}{|p{66mm}|p{88mm}|}
\hline
\multicolumn{1}{|c|}{\tabcolsep=0pt\begin{tabular}{c}Наименование параметров\\ 
и результатов тестирования\end{tabular}}&
\multicolumn{1}{c|}{Значения параметров и~результатов 
тестирования}\\
\hline
Объем тестовой выборки ВКДАГ, интегрирующего знания всех шести врачей&800 наблюдений~--- 500 с~
диагнозами эссенциальной АГ и~300 с~диагнозами вторичной АГ\\
\hline
Объем тестовой выборки ВКДАГ, интегрирующего знания менее шести врачей&400 наблюдений~--- 200 с~
диагнозами эссенциальной АГ и~200 с~диагнозами вторичной АГ\\
\hline
Источник формирования тестовой вы\-борки&Архив медицинских карт пациентов 1-го кардиологического 
отделения КОКБ\\
\hline
Элемент тестирующей последова\-тель\-ности&
Содержит множество нечетких лингвистических переменных и~вектор образа электрокардиограммы (может отсутствовать)\\
\hline
Эталонный диагноз&Результаты деятельности лечащего вра\-ча-кар\-дио\-ло\-га, подводящего общий итог~--- 
дифференциальный диагноз АГ\\
\hline
Критерии тестирования&Среднеквадратическая ошибка $f$ классификации состояния здоровья пациента~[7]\\
\hline
$f$(шесть врачей)&0,0837\\
\hline
$f$(без кардиолога)&0,697\\
\hline
$f$(без нефролога)&0,448 (в остальных 55,2\% случаях диагноз не вызовет доверия)\\
\hline
$f$(без терапевта)&0,151\\
\hline
$f$(без невролога)&0,149\\
\hline
$f$(без эндокринолога)&0,211 (в остальных 78,9\% случаях диагноз не вызовет доверия)\\
\hline
$f$(без сосудистого хирурга)&0,0798\\
\hline
$f$(без знаний терапевта, невролога, неф\-ро\-ло\-га, эндокринолога, сосудистого хирурга)&0,711\\
\hline
$f$(без знаний терапевта, невролога, эндокринолога, сосудистого хирурга)&0,485\\
\hline
$f$(без знаний невролога, эндокринолога, сосудистого хирурга)&0,334\\
\hline
$f$(без знаний невролога, сосудистого хи\-рурга)&0,167\\
\hline
\end{tabular}
\end{center}
\end{table}

\begin{multicols}{2}


\noindent
 и~от ла\-бораторных исследований, и~в~постановке
заключительного диагноза, а~нефролог и~эндокринолог~--- в~исключении вторичной 
АГ; (2)~знания врача~--- сосудистого хирурга не влияют на 
результаты работы <<Виртуального консилиума>>, и~объясняется это тем, что знания 
сосудистого хирурга, касающиеся диагностики АГ, составляют только~20\% базы знаний 
нечеткой системы, распознающей заболевания периферических артерий (ассоциативные 
клинические состояния), встречающихся не более чем у~10\% населения~\cite{11-kir}, 
и~в~тес\-то\-вую выборку не попала ни одна карта с~данными заболеваниями; (3)~чем больше 
численный состав <<Виртуального консилиума>>, тем с~меньшей среднеквадратической 
ошибкой он классифицирует состояние здоровья пациента; (4)~<<Виртуальный консилиум>> 
в~со\-ста\-ве шести врачей диагностирует АГ со среднеквадратической 
ошибкой постановки диагноза $f = 0{,}0837$, т.\,е.\ дает диагноз, верный в~84\% слу\-чаях. 
{\looseness=1

}
  
  Поскольку <<Виртуальный консилиум>> разра\-ботан на основе всероссийских~\cite{9-kir} 
и~между\-народных рекомендаций по диагностике АГ и~со\-пут\-ст\-ву\-ющих заболеваний, 
которых должен придерживать\-ся каж\-дый врач в~своей практике, при переносе \mbox{ВКДАГ} 
в~другое больничное учреж\-де\-ние необходимо пред\-оста\-вить врачам данного учреждения 
протоколы подтверждения диагностических правил всех баз знаний экспериментальными 
данными из архива КОКБ для ознакомления 
и~внесения при необходимости коррективов в~связи с~возможными особенностями их 
контингента пациентов, а~также возможных требований по устранению ограничений 
системы со стороны персонала нового больничного учреждения. Значительной 
корректировки баз знаний не потребуется.
  
  Таким образом, лабораторные эксперименты с~прототипом <<Виртуального 
консилиума>> дали обнадеживающие результаты. 

Верное решение получено в~84\% 
случаев. В~ам\-бу\-ла\-тор\-но-кли\-ни\-че\-ских учреждениях диагноз не 
выявляется у~70\% пациентов в~основном по причине инертности врачей, недостатка опыта 
врачей узкой специализации и~нехватки кадров в~ЛПУ
широкого профиля, что по результатам экспериментов может быть устранено с~по\-мощью 
применения \mbox{ВКДАГ} во время приема пациентов с~подозрением на АГ.

\section{Заключение}

  Лабораторно подтверждена эффективность предлагаемого подхода для проектирования 
диагностических систем как гетерогенных искусственных диагностических систем со 
свойствами дополнительности, сотрудничества и~относительности\linebreak
 знаний, синтезирующих 
интегрированные методы и~модели, разнообразие которых устраняет разнообразие 
диагностической информации об организме человека~--- <<Виртуальных консилиумов>>,\linebreak 
моделиру\-ющих работу коллектива врачей в~многопрофильном стационарном больничном 
учреждении (на примере КОКБ) и~внедрение 
которых повыша\-ет эффективность и~качество индивидуальных диагностических решений 
в~ам\-бу\-ла\-тор\-но-по\-ли\-кли\-ни\-че\-ском учреждении широкого профиля (на примере 
Диагностического центра КОКБ), где заключение о состоянии больного из-за проблемы 
с~кадрами узкой специализации принимает чаще всего один специалист~--- терапевт или 
врач общей практики, иногда кардиолог, но без опыта работы.

{\small\frenchspacing
 {%\baselineskip=10.8pt
 \addcontentsline{toc}{section}{References}
 \begin{thebibliography}{99}
\bibitem{1-kir}
\Au{Гаазе-Раппопорт М.\,Г., Поспелов~Д.\,А.} От амебы до робота: модели поведения.~--- 
М.: Наука, 1987. 288~с.
\bibitem{2-kir}
\Au{Колесников А.\,В., Кириков~И.\,А., Листопад~С.\,В. %Румовская~С.\,Б. 
и~др.} Решение 
сложных задач коммивояжера методами функциональных гибридных интеллектуальных 
сис\-тем.~--- М.: ИПИ РАН, 2011. 295~с.
\bibitem{3-kir}
\Au{Кириков И.\,А., Колесников~А.\,В., Румовская~С.\,Б.} Исследование сложной задачи 
диагностики артериальной гипертензии в~методологии искусственных гетерогенных  
сис\-тем~// Системы и~средства информатики, 2013. Т.~23. №\,2. С.~81--99. doi: 
10.14357/08696527130208.
\bibitem{4-kir}
\Au{Кириков И.\,А., Колесников~А.\,В., Румовская~С.\,Б.} Функциональная гибридная 
интеллектуальная система для поддержки принятия решений при диагностике артериальной 
гипертензии~// Системы и~средства информатики, 2014. Т.~24. №\,1. С.~153--179. doi: 
10.14357/08696527140110.
\bibitem{5-kir}
\Au{Колесников А.\,В., Румовская~С.\,Б., Листопад~С.\,В., Кириков~И.\,А.} Системный 
анализ в~решении сложных диагностических задач~// Системный анализ и~информационные 
технологии (САИТ-2015): Тр. VI~Междунар. конф.~--- М.: 
ИСА РАН, 2015. Т.~1. С.~157--167.
\bibitem{6-kir}
\Au{Колесников А.\,В., Кириков~И.\,А.} Методология и~технология решения сложных задач 
методами функциональных гибридных интеллектуальных систем.~--- М.: ИПИ РАН, 2007. 
387~с.
\bibitem{7-kir}
\Au{Кириков И.\,А., Колесников~А.\,В., Румовская~С.\,Б.} Исследование лабораторного 
прототипа искусственной гетерогенной системы для диагностики артериальной 
гипертензии~// Системы и~средства информатики, 2014. Т.~24. №\,3. С.~131--143. doi: 
10.14357/08696527140309.
\bibitem{8-kir}
\Au{Румовская С.\,Б.} Методы и~средства информатики для диагностики 
артериальной гипертензии в~ле\-чеб\-но-про\-фи\-лак\-ти\-че\-ских учреждениях 
широкого профиля~// Задачи современной информатики (ЗСИ-2015): Тр. 2-й 
молодежной научной конф.~--- М.: ФИЦ ИУ РАН, 2015. 
С.~168--174.
\bibitem{9-kir}
\Au{Кириков~И.\,А., Румовская~С.\,Б.} Гетерогенная диагностика артериальной 
гипертензии~// Информатика, управление и~системный анализ (ИУСА-2016): Тр. 
4-й Всеросс. научной конф. молодых ученых с~международным участием.~--- 
Тверь: ТвГТУ, 2016. Т.~1. С.~180--188.
\bibitem{10-kir}
Комитет экспертов ВНОК. Диагностика и~лечение артериальной гипертензии. 
Российские рекомендации~// Системные гипертензии, 2010. Вып.~3. С.~5--26.
\bibitem{11-kir}
\Au{Галимзянов Ф.\,В.} Заболевания периферических артерий (клиника, 
диагностика, лечение)~// Международный журнал экспериментального образования, 
2014. Вып.~8. С.~113--114. 

\end{thebibliography}

 }
 }

\end{multicols}

\vspace*{-6pt}

\hfill{\small\textit{Поступила в~редакцию 18.06.16}}

\vspace*{8pt}

%\newpage

%\vspace*{-24pt}

\hrule

\vspace*{2pt}

\hrule

%\vspace*{8pt}



\def\tit{``VIRTUAL COUNCIL''~--- SOURCE ENVIRONMENT SUPPORTING 
COMPLEX DIAGNOSTIC DECISION MAKING}

\def\titkol{``Virtual council''~--- source environment supporting 
complex diagnostic decision making}

\def\aut{I.\,А.~Kirikov$^1$, А.\,V.~Kolesnikov$^{1,2}$, S.\,V.~Listopad$^1$, and 
S.\,B.~Rumovskaya$^1$}

\def\autkol{I.\,А.~Kirikov, А.\,V.~Kolesnikov, S.\,V.~Listopad, and 
S.\,B.~Rumovskaya}

\titel{\tit}{\aut}{\autkol}{\titkol}

\vspace*{-9pt}

\noindent
$^1$Kaliningrad Branch of the Federal Research Center ``Computer Science and 
Control'' of the Russian Academy\linebreak
$\hphantom{^1}$of Sciences, 5~Gostinaya Str., Kaliningrad 236000, 
Russian Federation
   
   \noindent
   $^2$Immanuel Kant Baltic Federal University, 14~Nevskogo Str., Kaliningrad 236041, 
Russian Federation


\def\leftfootline{\small{\textbf{\thepage}
\hfill INFORMATIKA I EE PRIMENENIYA~--- INFORMATICS AND
APPLICATIONS\ \ \ 2016\ \ \ volume~10\ \ \ issue\ 3}
}%
 \def\rightfootline{\small{INFORMATIKA I EE PRIMENENIYA~---
INFORMATICS AND APPLICATIONS\ \ \ 2016\ \ \ volume~10\ \ \ issue\ 3
\hfill \textbf{\thepage}}}

\vspace*{3pt}
  
    
  
\Abste{The paper considers the problem of individual decision making during 
diagnostics of 
patients in outpatient clinics by the example of arterial 
hypertension diagnostics. It is proposed to 
raise the quality of individual decision\linebreak\vspace*{-12pt}}

\Abstend{making by means of consultations with the ``Virtual council'' 
decision support system, which models the work of physician councils in inpatient multifield 
clinics. The results of development and experimental research of the 
laboratory prototype of ``Virtual council'' are presented.}

\KWE{decision support system; virtual council; functional hybrid intellectual system}

\DOI{10.14357/19922264160311} 

\vspace*{-9pt}

\Ack
\noindent
The work was performed with partial support of the Russian
Foundation for Basic Research (grant No.\,16-07-00272~А).


%\vspace*{3pt}

  \begin{multicols}{2}

\renewcommand{\bibname}{\protect\rmfamily References}
%\renewcommand{\bibname}{\large\protect\rm References}

{\small\frenchspacing
 {%\baselineskip=10.8pt
 \addcontentsline{toc}{section}{References}
 \begin{thebibliography}{99}
\bibitem{1-kir-1}
\Aue{Gaaze-Rappoport, M.\,G., and D.\,A.~Pospelov}. 1987. \textit{Ot ameby do robota: Modeli 
povedeniya} [From ameba to robotic mashine: Behavior model] Moscow: Nauka. 288~p.
\bibitem{2-kir-1}
\Aue{Kolesnikov,~A.\,V., I.\,A.~Kirikov, S.\,V.~Listopad, \textit{et al.}}. 2011. \textit{Reshenie 
slozhnykh zadach kommivoyazhera metodami funktsional'nykh gibridnykh intellektual'nykh 
sistem} [Solving of the complex traveling salesman problem by means of functional hybrid 
intellectual systems]. Moscow: IPI RAN. 295~p.
\bibitem{3-kir-1}
\Aue{Kirikov, I.\,A., A.\,V.~Kolesnikov, and S.\,B.~Rumovskaya}.\linebreak
 2013. Issledovanie slozhnoy 
zadachi diagnostiki arterial'noy gipertenzii v~metodologii iskusstvennykh geterogennykh sistem 
[Research of the complex problem at\linebreak diagnosing of the arterial hypertension within the 
methodology of artificial heterogeneous systems]. \textit{Sistemy i~Sredstva Informatiki~--- 
Systems and Means of Informatics} 23(2):81--99. doi: 10.14357/08696527130208.
\bibitem{4-kir-1}
\Aue{Kirikov, I.\,A., A.\,V.~Kolesnikov, and S.\,B.~Rumovskaya}.\linebreak
 2014. Funktsional'naya 
gibridnaya intellektual'naya sistema dlya podderzhki prinyatiya resheniya pri diagnostike 
arterial'noy gipertenzii [Functional hybrid intelligent decision support system for diagnosing of the 
\mbox{arterial} hypertension]. \textit{Sistemy i~Sredstva Informatiki~--- Systems and Means of Informatics} 
24(1):153--179. doi: 10.14357/08696527140110. 
\bibitem{5-kir-1}
\Aue{Kolesnikov, A.\,V., I.\,A.~Kirikov, S.\,V.~Listopad, and S.\,B.~Rumovskaya}. 2015. 
Sistemnyy analiz v~reshenii slozhnykh diagnosticheskikh zadach [Systems analysis for solving 
complex diagnostic tasks]. \textit{Tr. 6-y Mezhdunar. konf. ``Sistemnyy analiz i~informatsionnye 
tekhnologii''} [6th Conference (International) ``Systems Analysis and Information Technology'' 
Proceedings]. Moscow.  1:157--167.
\bibitem{6-kir-1}
\Au{Kolesnikov, A.\,V., and I.\,A.~Kirikov}. 2007. \textit{Metodologiya i~tekhnologiya resheniya 
slozhnykh zadach metodami funk\-tsi\-o\-nal'\-nykh gibridnykh intellektual'nykh sistem} [Methodology 
and technology for solving of complex problems using the methodology of functional hybrid 
artificial systems]. Moscow: IPI RAN. 387~p.
\bibitem{7-kir-1}
\Aue{Kirikov, I.\,A., A.\,V.~Kolesnikov, and S.\,B.~Rumovskaya}. 2014. Issledovanie 
laboratornogo prototipa iskusstvennoy geterogennoy sistemy dlya diagnostiki arterial'noy 
gipertenzii [Research of the laboratory prototype of the artificial heterogeneous system for 
diagnosing of the arterial hypertension]. \textit{Sistemy i~Sredstva informatiki~--- Systems and 
Means of Informatics} 24(3):131--143. doi: 10.14357/08696527140309.
\bibitem{8-kir-1}
\Au{Rumovskaya, S.\,B.} 2015. Metody i~sredstva informatiki dlya diagnostiki 
arterial'noy gipertenzii v~lechebno-profilakticheskikh uchrezhdeniyakh shirokogo profilya 
[Methods and tools of informatics for diagnostics of arterial hypertension in multiskilled 
medical preventive institution]. \textit{Tr. 2-y molodezhnoy nauchnoy konf. ``Zadachi 
sovremennoy informatiki''} [2nd Youth Conference ``Tasks of Modern Informatics'' 
Proceedings]. Moscow: FRC ``Computer Science and Control'' RAS. 168--174.
\bibitem{9-kir-1}
\Aue{Kirikov, I.\,A., and S.\,B.~Rumovskaya}. 2016. Geterogennaya diagnostika arterial'noy 
gipertenzii [Heterogeneous diagnostics of arterial hypertension]. \textit{Tr. 4-y Vseross. 
nauchnoy konf. molodykh uchenykh s~mezhdunarodnym uchastiem ``Informatika, 
upravlenie i~sistemnyy analiz''} [4th Youth Conference (International) ``Informatics, Control 
and Systems Analysis'' Proceedings]. Tver: Tver State Technical University. 1:180--188.
\bibitem{10-kir-1}
Komitet ekspertov VNOK [Committee of experts of All-Russia Scientific Society of Сardiologists]. 
2010. Diagnostika i~lechenie arterial'noy gipertenzii. Rossiyskie 
rekomendatsii [Diagnosing and treatment of arterial 
hypertension. Russian recommenation]. 
\textit{Sistemnye gipertenzii} [Systemic Hypertension] 3:5--26. 
\bibitem{11-kir-1}
\Aue{Galimzyanov, F.\,V.} 2014. Zabolevaniya perifericheskikh arteriy (Klinika, 
diagnostika, lechenie) [Peripheral vascular disease (Clinic, diagnostics, treatment]. 
\textit{Mezhdunarodnyy zhurnal eksperimental'nogo obrazovaniya} [Int. J.~Research 
Education] 8:113--114. 
   \end{thebibliography}

 }
 }

\end{multicols}

\vspace*{-9pt}

\hfill{\small\textit{Received June 18, 2016}}

\vspace*{-3pt}
    
  
  \Contr
  
  \noindent
  \textbf{Kirikov Igor A.}\ (b.\ 1955)~---
  Candidate of  Sciences (PhD) in technology; director, Kaliningrad Branch of the 
  Federal Research Center ``Computer Science and Control'' of the Russian Academy 
  of Sciences, 5~Gostinaya Str., Kaliningrad 236000,  Russian Federation; 
baltbipiran@mail.ru
  
  \pagebreak
%  \vspace*{3pt}
  
  \noindent
  \textbf{Kolesnikov Alexander V.}\ (b.\ 1948)~---
  Doctor of Sciences in technology; professor, 
Department of Telecommunications, 
 Immanuel Kant Baltic Federal University, 14~Nevskogo Str., Kaliningrad 236041, Russian Federation; senior scientist, Kaliningrad Branch of 
  the Federal Research Center ``Computer Science and Control'' of the Russian 
  Academy of Sciences, 5~Gostinaya Str., Kaliningrad 236000,  Russian Federation; 
  avkolesnikov@yandex.ru
  
  \vspace*{4pt}
  
  \noindent
  \textbf{Listopad Sergey V.}\ (b.\ 1984)~---
  Candidate of  Sciences (PhD) in technology; scientist, Kaliningrad Branch of the 
  Federal Research Center ``Computer Science and Control'' of the Russian Academy 
  of Sciences, 5~Gostinaya Str., Kaliningrad 236000,  Russian Federation;   
ser-list-post@yandex.ru
  
  \vspace*{4pt}
  
  \noindent
  \textbf{Rumovskaya Sophiya B.}\ (b.\ 1985)~--- programmer~I, Kaliningrad Branch 
  of the Federal Research Center ``Computer Science and Control'' of the Russian 
  Academy of Sciences, 5~Gostinaya Str., Kaliningrad 236000,  Russian Federation; 
  sophiyabr@gmail.com
  \label{end\stat}
  
  
  \renewcommand{\bibname}{\protect\rm Литература} 

\def\stat{zatsman}

\def\tit{ТРАНСФОРМАЦИИ ОБЪЕКТОВ ПЕРВОГО И~ВТОРОГО ПОРЯДКА 
В~ЛЕКСИКОГРАФИЧЕСКОЙ ИНФОРМАЦИОННОЙ СИСТЕМЕ$^*$}

\def\titkol{Трансформации объектов первого и~второго порядка 
в~лексикографической информационной системе}

\def\aut{И.\,М.~Зацман$^1$}

\def\autkol{И.\,М.~Зацман}

\titel{\tit}{\aut}{\autkol}{\titkol}

\index{Зацман И.\,М.}
\index{Zatsman I.\,M.}


{\renewcommand{\thefootnote}{\fnsymbol{footnote}} \footnotetext[1]
{Исследование выполнено в~ФИЦ ИУ РАН за счет гранта Российского научного фонда №\,24-18-00155, {\sf 
https://rscf.ru/project/24-18-00155}. Работа выполнялась с~использованием инфраструктуры Центра 
коллективного пользования <<Высокопроизводительные вычисления и~большие данные>> (ЦКП 
<<Информатика>>) ФИЦ ИУ РАН (г.\ Москва).}}


\renewcommand{\thefootnote}{\arabic{footnote}}
\footnotetext[1]{ Федеральный исследовательский центр <<Информатика и~управление>> Российской академии наук, 
\mbox{izatsman@yandex.ru}}

\vspace*{-12pt}


  
  \Abst{Рассматриваются теоретические основания проектирования информационных 
технологий (ИТ) интеграции двуязычных словарей и~параллельных корпусов. Дано описание 
первых результатов создания третьего уровня классификации трансформаций объектов 
предметной области информатики, которую предполагается использовать при создании 
концепции лексикографической информационной системы, обеспечивающей интеграцию. 
Все сущности информатики в~статье разделены на два глобальных класса: объекты и~их 
трансформации. Для каждого такого класса конструируется своя классификация. Ранее были 
описаны два верхних уровня классификации трансформаций объектов предметной области. 
В~данной статье рассматривается третий уровень этой классификации. Основанием для 
построения самого верхнего ее уровня служило деление предметной области информатики 
на среды (ментальная, сенсорно воспринимаемая, цифровая и~ряд других сред), каждая из 
которых по определению включает объекты одной природы. Основанием для построения 
второго уровня классификации трансформаций объектов служила типология знаковых  
сис\-тем А.~Соломоника. Цель статьи состоит в~систематизации трансформаций первого 
и~второго порядка объектов предметной области на третьем уровне этой классификации. 
Основанием для систематизации служит средовая версия иерархии Акоффа.}
  
  \KW{объекты предметной области; трансформации объектов; классификация; данные; 
информация; знание; лексикографическая информационная сис\-тема}

\DOI{10.14357/19922264240211}{VZTGVV}
  
\vspace*{3pt}


\vskip 10pt plus 9pt minus 6pt

\thispagestyle{headings}

\begin{multicols}{2}

\label{st\stat}
  
\section{Введение}

\vspace*{-9pt}

  Возникновение параллельных корпусов, в~которых предложениям 
оригинального текста со\-по\-став\-ле\-ны предложения его перевода, обеспечило 
возможность контрастивного лингвистического\linebreak \mbox{анализа} на принципиально 
новом уровне полноты и~точности, недостижимом в~докорпусную эпоху. 
Пионерскими в~этой области стали работы \mbox{1990-х~гг}. Стига Йоханссона  
с~анг\-ло-нор\-веж\-ским корпусом~[1]. В России параллельные корпусы стали 
формироваться в~начале XXI~века в~рамках Национального корпуса русского 
языка~[2].
  
  Создатели двуязычных словарей используют параллельные корпусы для 
сбора материала и~эмпирической проверки своих гипотез, касающихся 
межъязы\-ко\-вой эквивалентности. Ценность параллельных корпусов 
определяется тем, что в~лингвистике этап сбора исходного материала считается 
наиболее трудоемким и~наименее творческим, а~параллельные корпусы 
позволяют значительно сэкономить время и~силы для творческого этапа 
создания словарей~[3].
 % 
  При этом двуязычные словари, создаваемые на основе исходного материала, 
извлеченного из параллельных корпусов, сейчас формируются без связей с~их 
текстами. Другими словами, онлайновые связи созданных словарей 
с~параллельными корпусами, которые служили источниками исходного 
материала, отсутствуют. 

Параллельные корпусы постоянно пополняются 
новыми текстами, в~предложениях которых можно обнаружить новые значения 
слов и~устойчивых словосочетаний. Однако при этом отсутствуют методы 
и~средства оперативного обновления словарей по корпусным данным. 
В~настоящее время проблема установления связей между двуязычными 
словарями и~параллельными корпусами (далее~--- проблема интеграции) 
находится на стадии поиска концептуальных подходов к~их интеграции на 
уровне значений.
  
  Подход к~решению проблемы интеграции, предлагаемый в~статье, учитывает 
  и~появление новых значений слов и~устойчивых словосочетаний, и~динамику 
смысловых значений, которая обусловлена развитием и~пополнением знания 
лингвистов, фиксирующих эти значения в~результате семантического анализа 
пополняемых корпусных данных. Проведенные эксперименты показали, что 
обнаружение нового лингвистического знания обусловливает и~формирование 
дефиниций новых значений, и~пересмотр уже существующих дефиниций~[4, 5].
  
  Например, в~проведенных экспериментах с~использованием ЦКП 
<<Информатика>> ФИЦ ИУ РАН фиксировалась эволюция значений немецких 
модальных глаголов, исходное состояние значений которых было описано 
в~не\-мец\-ко-рус\-ском словаре. В~экспериментальном массиве текстов как 
потенциальных источниках нового знания 16\,268 предложений содержали 
немецкие модальные глаголы и~в~2041 из них встречался глагол sollen. 
В~начале эксперимента в~словаре были описаны~12~значений этого модального 
глагола. По окончании эксперимента лингвисты обнаружили два новых его 
значения, согласовали их дефиниции и~описали эволюцию дефиниций~[6, 7].
  
  Таким образом, для решения проблемы интеграции требуется фиксировать 
новое знание, обнаруженное лингвистами в~текстовых данных параллельных 
корпусов, отслеживать эволюцию знания, представленного в~виде дефиниций 
значений слов и~устойчивых словосочетаний, и,~соответственно, 
актуализировать электронные двуязычные словари. Предлагаемый 
концептуальный подход к~интеграции, который планируется реализовать 
в~процессе проектирования лексикографической информационной сис\-те\-мы, 
фиксирующей эволюцию лингвистического знания, основан на решении 
следующих задач:\\[-14pt]
  \begin{itemize}
  \item категоризация трех базовых понятий информатики, включенных 
  в~иерархию Акоффа~[8] (данные, информация, знание), на объекты 
проектируемой сис\-те\-мы, которая необходима, чтобы фиксировать 
<<кванты>> нового знания и~отслеживать его эволюцию в~этой сис\-теме;\\[-15pt]
  \item  систематизация трансформаций объектов этой сис\-темы.\\[-14pt]
  \end{itemize}
  
  Цель статьи и~состоит в~решении двух задач: категоризации трех базовых 
понятий информатики на объекты лексикографической информационной  
сис\-те\-мы и~сис\-те\-ма\-ти\-за\-ции трансформаций первого и~второго порядка 
ее объектов.
  
  Трансформациями первого порядка, о которых сказано в~формулировке цели 
статьи, называются взаимные преобразования между двумя объектами  
сис\-те\-мы одной природы. Например, перевод в~сис\-те\-ме текста с~русского 
языка на английский относится к~ним. Трансформациями второго порядка 
и~выше называются взаимные преобразования между двумя и~более объектами 
разной природы. Например, кодирование символов текс\-та компьютерными 
кодами и~их декодирование относятся по определению к~трансформациям 
второго порядка.

%\vspace*{-9pt}
  
\section{Процессы трансформаций в~информатике}

%\vspace*{-3pt}

Процессы трансформаций, рассматриваемые в~статье, относятся к~теоретическому ядру информатики, а~не 
только к~проектированию лексикографической информационной сис\-те\-мы. Например, из трех основных 
подходов к~описанию предметной об\-ласти информатики\footnote{В статье предметная область информатики 
трактуется согласно концепции полиадического компьютинга Пола Розенблума~\cite{9-zac}.} (объектный, 
трансформационный и~синтетический) сис\-те\-ма\-ти\-за\-ция трансформаций ближе всего ко второму 
подходу. Примерами первого подхода, в~рамках которого основное внимание уделяется объектам предметной 
области информатики и~в~меньшей степени отношениям\linebreak между ними, могут служить  
работы~\cite{8-zac, 10-zac, 11-zac}; \mbox{примерами} второго подхода, в~рамках которого основное внимание 
уделяется трансформациям и~в~меньшей степени трансформируемым объектам,~---  
работы~\cite{12-zac, 13-zac}; примерами третьего, синтетического подхода, в~котором уделяется внимание 
и~объектам предметной об\-ласти информатики, и~отношениям между ними, могут служить работы~\cite{14-zac, 
15-zac, 16-zac, 17-zac, 18-zac}.

  Таким образом, для описания трансформаций объектов лексикографической 
информационной\linebreak системы предпочтительнее всего трансформационный 
подход, который упоминается и~в определениях информатики. Например, 
в~2009~г.\ П.~Деннинг и~П.~Розенблум сформулировали суть \mbox{информатики} как 
компьютинга следующим образом: <<$\ldots$информатика~--- это не просто 
алгоритмы и~структуры данных; это преобразования [трансформации] 
представлений>>~\cite{12-zac}. Чуть позже, в~контексте краткого описания 
парадигмы информатики как компьютинга, П.~Деннинг и~П.~Фриман изменили 
эту формулировку на такую: <<Центральный объект внимания в~информатике 
можно определить как информационные процессы~--- \textit{естественные или 
искусственные процессы, преобразующие информацию} (курсив мой~--- 
И.\,З.)>>~\cite{13-zac}. Согласно парадигме, предлагаемой авторами этой 
статьи, на начальном этапе проектирования автоматизированных систем 
базовыми элементами моделей их функционирования служат 
\textit{информационные про\-цессы}.
  
  Однако если 15~лет назад в~формулировке из работы~\cite{13-zac} шла речь 
о~процессах, преобразующих информацию, то в~последние~10~лет в~спектр 
процессов трансформаций все чаще стали включать процессы, преобразующие 
не только информацию, но также и~другие объекты автоматизированных 
систем, в~первую очередь данные и~знания~[19--21]. Например, Виктория 
Стодден, позиционируя науку о~данных как одну из дисциплин информатики, 
говорит, что центральный объект исследований в~науке о~данных~--- это 
<<изучение обобщаемого извлечения знания из данных>>~\cite{21-zac}. 
Увеличение и~чис\-ла объектов, и~спект\-ра процессов их трансформаций 
в~автоматизированных сис\-те\-мах обуслов\-ли\-ва\-ет не\-об\-хо\-ди\-мость 
систематизации и~объектов, и~процессов их трансформаций на начальном этапе 
проектирования сис\-тем.
  
  Для создания концепции лексикографической информационной сис\-те\-мы 
и~проектирования ИТ, обеспечивающих интеграцию 
двуязычных словарей и~параллельных корпусов, сначала выполним 
категоризацию на объекты этой сис\-те\-мы трех базовых понятий информатики 
(данные, информация, знание) в~контексте построения классификаций 
сущностей ее предметной об\-ласти.
  
  Необходимость использования классификаций информатики в~процессе 
создания концепции проиллюстрируем, используя иерархию  
Акоффа~\cite{8-zac}. Он использовал принцип их вертикального размещения 
в~иерархии снизу вверх: данные, информация и~знание. Еще в~ней есть термин 
<<мудрость>>, который в~статье не рассматривается. Такое размещение Акофф 
прокомментировал так: <<Каждое из пе\-ре\-чис\-лен\-ных понятий [кроме данных] 
содержит в~себе нижестоящие$\ldots$>>~\cite{8-zac}.
  
  Этому принципу размещения и~комментарию Акоффа свойственны 
недостатки, проанализированные, в~частности, в~работе~\cite{10-zac}. Главный 
вывод, к~которому пришла Роули после изучения иерархии Акоффа, 
заключается в~следующем: <<$\ldots$информация определяется в~терминах 
данных, знание~--- в~терминах информации$\ldots$ но существует меньше 
консенсуса в~описании трансформаций, которые преобразуют сущности, 
расположенные ниже в~иерархии, в~те, которые находятся над ними, что 
приводит к~их терминологической неопределенности>>~\cite{10-zac}. Причина 
этой неопределенности, скорее всего, в~том, что базовые понятия информатики 
включены в~иерархию Акоффа изолированно от общего контекста 
классификаций сущностей ее предметной об\-ласти.

%\vspace*{-9pt}
  
\section{Классификации сущностей информатики}


%\vspace*{-2pt}

  Все сущности предметной области информатики в~работах~[22, 23] 
разделены на два глобальных класса: ее объекты и~их трансформации. Для 
каждого такого класса была предложена своя классификация. 
В~работе~\cite{22-zac} дано описание классификации объектов предметной 
области информатики, первый уровень которой содержит базовые понятия ее 
предметной области (данные, информация, знания и~др.).  
В~работе~\cite{23-zac} дано описание двух верхних уровней классификации 
трансформаций объектов предметной об\-ласти (см.\ рисунок 
в~работе~\cite{23-zac}). Основанием для построения самого верхнего ее уровня послужило деление 
предметной области информатики на среды\footnote{В~работе~\cite{24-zac} дано описание пяти сред 
предметной области информатики (ментальная; сенсорно воспринимаемая, или информационная; 
цифровая; нейро- и~ДНК-среда), каждая из которых по определению включает объекты одной и~той же 
природы.} и~степень разнообразия природы объектов, вовлеченных в~трансформации:
\begin{itemize}
\item  первый класс верхнего уровня классификации включает 
трансформации объектов в~пределах среды только одной природы 
(трансформации первого порядка);
\item  второй класс включает трансформации объектов, относящихся 
к~двум средам разной природы (трансформации второго порядка);
\item третий и~последующие классы включают трансформации объектов, 
относящихся к~трем и~более средам разной природы (трансформации 
третьего и~более высоких порядков).
\end{itemize}

  В работе~\cite{23-zac} были приведены примеры для трех первых классов 
трансформаций, включая пример трансформаций объектов, относящихся 
к~двум средам разной природы (компьютерное кодирование символов текстов 
с~по\-мощью таб\-лиц Unicode).
  
Основанием для построения второго уровня классификации трансформаций объектов послужила типология 
знаковых сис\-тем А.~Соломоника~\cite[c.~131]{25-zac}: естественные знаковые сис\-те\-мы, образные,  
ес\-тест\-вен\-но-язы\-ко\-в$\acute{\mbox{ы}}$е,  
вер\-баль\-но-не\-сло\-вес\-ные сис\-те\-мы записи\footnote{Под системой записи понимается знаковая 
система, сочетающая вербальные знаки с~несловесными (языки нотной записи, карт, таблиц и~др.).} 
и~формализованные знаковые сис\-те\-мы, включая математические. Введем понятие обобщенного текста~--- 
это текст, который может быть создан в~любой из перечисленных знаковых систем. Тогда обобщенные тексты 
могут быть естественными, образными, ес\-тест\-вен\-но-язы\-ко\-в$\acute{\mbox{ы}}$\-ми,  
вер\-баль\-но-не\-сло\-вес\-ны\-ми и~формализованными. Второй уровень классификации трансформаций 
охватывает не все виды объектов предметной  
об\-ласти информатики, а~только перечисленные~5~видов текс\-тов и~их представления, вовлеченные 
в~процессы трансформаций в~одной или более средах вместе с~данными, знанием и~его концептами.

\begin{figure*}[b] %fig1
\vspace*{6pt}
      \begin{center}
     \mbox{%
\epsfxsize=121.191mm 
\epsfbox{zac-1.eps}
}
\end{center}
\vspace*{-6pt}
\Caption{Средовая версия иерархии Акоффа}
\end{figure*}

\section{Классификация трансформаций: построение~третьего 
уровня}

  Основанием для систематизации трансформаций первого и~второго порядка 
на третьем уровне этой классификации служит иерархия Акоффа~\cite{8-zac}, 
на основе которой и~была создана ее средов$\acute{\mbox{а}}$я версия~[26, 
27]. Для создания средов$\acute{\mbox{о}}$й версии была выполнена 
категоризация трех базовых понятий информатики (данные, информация, 
знания) на объекты лексикографической информационной сис\-те\-мы 
в~процессе создания ее концепции\linebreak (рис.~1).
  


  В отличие от классической иерархии Акоффа, в~ее 
средов$\acute{\mbox{о}}$й версии различаются три вида данных: сенсорно 
воспринимаемые, цифровые и~те данные, которые генерируются 
искусственными нейронными сетями (ИНС) в~системах искусственного интеллекта 
(далее~--- ИИ-дан\-ные). Последний вид данных необходим, например, для 
различения входа и~выхода процесса применения обученной 
ИНС в~цифровой модели генерации знания, описанию которой 
посвящена работа~\cite{27-zac}.
  
  Также предлагается различать два вида информации: сенсорно 
воспринимаемая и~цифровая. Кроме знания в~средов$\acute{\mbox{у}}$ю 
версию добавлены концепты и~ментальные образы сенсорно воспринимаемых 
данных. Последние служат промежуточной сущностью между сенсорно 
воспринимаемыми данными и~генерируемым знанием при описании процессов 
извлечения знания из текстовых данных лексикографической информационной 
системы. Описание объектов средов$\acute{\mbox{о}}$й версии иерархии 
Акоффа (см.\ рис.~1) и~отношений между ними дано в~работах~\cite{26-zac, 28-zac}.
  
  В средов$\acute{\mbox{о}}$й версии число объектов равно восьми. Если 
учитывать направления трансформаций, то между восемью объектами на 
рис.~1 она включает~16 их видов (трансформации на границе между сенсорно 
воспринимаемыми данными и~информацией, обозначенные символом~<<?>>, 
в~статье не рас\-смат\-ри\-ва\-ют\-ся). В~будущем число объектов 
в~средов$\acute{\mbox{о}}$й версии, которая выбрана как основание для 
сис\-те\-ма\-ти\-за\-ции трансформаций первого и~второго порядка, может быть 
увеличено. Для построения классификации трансформаций 
важ\-но не возможное увеличение числа объектов 
и~трансформаций между ними, а то, что их виды в~средов$\acute{\mbox{о}}$й 
версии распределены между трансформациями первого и~второго порядка. Из 
16~видов на рис.~1 шесть относятся к~трансформациям первого порядка, это\linebreak 
виды с~номерами~7, 8, 13--16 (далее~--- типология трансформаций первого 
порядка), а~десять~--- к~трансформациям второго порядка, это виды 
с~\mbox{номерами}~1--6 и~9--12 (далее~--- типология трансформаций второго 
порядка). Разместим обе типологии на третьем уровне классификации (см.\ ее 
схему на рис.~2). Перечислим виды трансформаций первой типологии, вводя 
в~скобках их краткие названия, используемые ниже на рис.~3:
  \begin{description}
  \item[\,] 7~--- членение знания на концепты с~помощью одной или нескольких 
знаковых систем (далее~--- членение знания);
  \item[\,] 8~--- формирование знания на основе концептов (формирование 
знания);
  \item[\,] 13~--- обучение ИНС;
  \end{description}
  
  \vspace*{-6pt}
  
  \pagebreak
  
  \end{multicols}
  
  \begin{figure*} %fig2
\vspace*{1pt}
      \begin{center}
     \mbox{%
\epsfxsize=127.513mm 
\epsfbox{zac-2.eps}
}
\end{center}
\vspace*{-9pt}
\Caption{Схема трех верхних уровней классификации трансформаций объектов (объединены 
по три слоя и~для второго, и~для третьего уровней этой классификации)}
\end{figure*}
  
  \begin{multicols}{2}
  
  \noindent
  \begin{description}
  \item[\,] 14~--- восстановление обучающей информации на основе 
содержания обученной ИНС (обращение ИНС);
  \item[\,] 15~--- использование обученной ИНС (использование ИНС);



  \item[\,] 16~--- восстановление исходных данных, соответствующих 
полученным результатам работы обучен\-ной ИНС (восстановление исходных данных 
по результатам ИНС).
  \end{description}
  
  
  Не все виды трансформаций 13--16 поддерживаются в~конкретных системах 
искусственного интеллекта, но с~теоретической точки зрения все их 
предлагается включить в~первую типологию для полноты спектра видов 
трансформаций.
  
  Перечислим виды трансформаций второй типологии:
  \begin{description}
  \item[\,] 1~--- декодирование цифровых данных в~компьютерных системах 
(декодирование данных);
  \item[\,]  2~--- кодирование сенсорно воспринимаемых данных (кодирование 
данных);
  \item[\,] 3~--- ментальное копирование сенсорно воспринимаемых данных 
(ментальное копирование);
  \item[\,] 4~--- восстановление сенсорно воспринимаемых данных по 
ментальным образам (восстановление по образам);
  \item[\,] 5~--- смысловая интерпретация без деления на концепты ментальных 
образов сенсорно воспринимаемых данных (смысловая интерпретация);
  \item[\,] 6~--- восстановление ментальных образов (восстановление образов);
  \item[\,] 9~--- представление концептов в~виде сенсорно воспринимаемой 
информации, например текс\-та\-ми, формулами, таблицами, рисунками и~т.\,д.\ 
(представление концептов);
  \item[\,] 10~--- понимание смысла сенсорно воспринимаемой информации 
(понимание смысла);
  \item[\,] 11~--- кодирование сенсорно воспринимаемой информации 
(кодирование информации);
\end{description}

\vspace*{-6pt}

\pagebreak

\end{multicols}

\begin{figure*} %fig3
\vspace*{1pt}
      \begin{center}
     \mbox{%
\epsfxsize=163mm 
\epsfbox{zac-3.eps}
}
\end{center}
\vspace*{-9pt}
\Caption{Схема частного случая классификации трансформаций объектов (трансформации 
пронумерованы согласно рис.~1)}
\end{figure*}

\begin{multicols}{2}

\noindent
\begin{description}

  \item[\,] 12~--- декодирование цифровой информации (декодирование 
информации).
  \end{description}
  
  Отметим, что в~существующих ИТ
  и~компьютерных системах наиболее часто используются виды 
трансформаций~13 и~15 типологии первого порядка и~1, 2, 11 и~12 типологии 
второго порядка. На рис.~2 в~первом слое третьего уровня классификации 
показаны типологии первого порядка без указания числа трансформаций в~них 
и~без детализации трансформируемых объектов.
  
  Во втором слое третьего уровня классификации условно (без названий) 
показаны типологии второго порядка. Также на рис.~2 в~третьем слое третьего 
уровня классификации условно (также без названий) показаны типологии 
третьего порядка, которые планируется рассмотреть в~отдельной статье. По 
определению они должны включать трансформации между тремя объектами 
разной природы, но средов$\acute{\mbox{а}}$я версия иерархии Акоффа 
включает трансформации только между двумя объектами разной природы. 
Поэтому потребуется другое основание для их систематизации (ранее были 
рассмотрены отдельные примеры трансформаций третьего 
порядка\footnote{Далеко не всегда трансформации третьего и~более высоких порядков можно 
рассматривать как последовательность трансформаций второго порядка. Примером этого могут 
служить трансформации в~процессе обучения пациента пользованию роботизированной рукой, 
охватывающие личностные концепты пациента, релевантные его намерениям, сигналы активности 
мозга как объекты нейросреды и~компьютерные коды~\cite{29-zac}.}~\cite{29-zac}).

\section{Классификация трансформаций: частный~случай}

  Выше было отмечено, что в~будущем число объектов 
в~средов$\acute{\mbox{о}}$й версии иерархии Акоффа может быть увеличено. 
Это означает, что увеличатся и~чис\-ло объектов, и~чис\-ло трансформаций между 
ними в~классификации трансформаций, так как эта средов$\acute{\mbox{а}}$я 
версия служит по определению основанием для систематизации 
трансформаций первого и~второго порядка. Поэтому на третьем уровне рис.~2 
указаны типологии без детализации объектов и~без указания числа 
трансформаций в~каждой из них. С~одной стороны, при таком подходе 
получаем достаточно общий вид этой классификации, так как она не зависит от 
числа объектов в~том или ином варианте средов$\acute{\mbox{о}}$й версии 
(и~это существенно упрощает рис.~2). С~другой стороны, на третьем уровне 
такой общей классификации подразумевается, но не эксплицируется природа 
трансформируемых объектов и~их возможные сочетания в~трансформациях. 

При проектировании лексикографической информационной системы важно 
эксплицировать природу трансформируемых объектов и~их возможные 
сочетания.
  %
  Поэтому в~парадигму информатики~\cite{30-zac} кроме общей 
классификации трансформаций предлагается включать и~ее частные случаи, 
эксплицирующие природу трансформируемых объектов. 

В~этом разделе 
рассмотрим один частный случай, когда используются только естественные 
знаковые сис\-те\-мы из типологии А.~Соломоника~\cite{25-zac} вместе 
с~данными, знанием и~его концептами. Чис\-ло естественных языков при этом не 
ограничено. И~этот частный случай классификации включает только три 
класса природных трансформаций (первого, второго и~третьего порядка, см.\ 
схему классификации на рис.~3).
  
  Первый и~второй уровни схемы общей классификации (см.\ рис.~2) можно 
объединить в~один уровень в~этом частном случае. Ниже этого уровня 
приведено содержание типологий первого и~второго порядка без содержания 
типологий третьего по\-рядка.




  Наполнение типологий первого и~второго порядка соответствует 
средов$\acute{\mbox{о}}$й версии иерархии Акоффа на рис.~1, содержащей 
6~видов трансформаций типологии первого порядка и~10~видов 
трансформаций типологии второго порядка (на рис.~3 стрелки указывают 
направления трансформаций согласно средов$\acute{\mbox{о}}$й версии на рис.~1).
  
  Таким образом, частный случай классификации содержит для этих двух 
типологий 16~теоретически возможных трансформаций, 6 из которых 
в~настоящее время в~существующих ИТ применяются наиболее часто: виды 
трансформаций~1, 2, 11 и~12 типологии второго порядка реализуются 
с~помощью тех или иных методов ко\-ди\-ро\-ва\-ния/де\-ко\-ди\-ро\-ва\-ния 
(например, с~использованием таблиц Unicode), а~виды трансформаций~13 и~15
 в~типологии первого порядка реализуются полностью с~по\-мощью процессов 
цифровой обработки компьютерами.
  
  Остальные виды трансформаций или применяются намного реже (это 
виды~3, 5, 7, 9 и~10), или находятся в~стадии поиска и~разработки (14 и~16) или 
в~настоящее время носят только теоретический характер, обеспечивая полноту 
первой и~второй типологий (4, 6 и~8). Знаком~<<?>> обозначены те виды 
трансформаций, которые по определению не существуют в~используемой 
парадигме информатики~\cite{30-zac}. Однако возможно, что в~других 
будущих подходах к~построению ее парадигмы эти виды трансформаций будут 
существовать.
  
\section{Заключение}

  На сегодняшний день процесс построения классификаций объектов 
предметной области информатики~\cite{22-zac} и~их  
трансформаций~\cite{23-zac} еще не завершен. Однако первые результаты их 
построения уже используются для создания концепции лексикографической 
информационной сис\-те\-мы, обеспечивающей интеграцию двуязычных 
словарей и~параллельных корпусов.
  
  \bigskip
  
  
  Автор признателен рецензентам за помощь в~улучшении статьи.
  
{\small\frenchspacing
 { %\baselineskip=10.6pt
 %\addcontentsline{toc}{section}{References}
 \begin{thebibliography}{99}
\bibitem{1-zac}
\Au{Aijmer K., Altenberg~B.} Advances in corpus-based contrastive linguistics. Studies in honour 
of Stig Johansson.~--- Amsterdam: John Benjamins, 2013. 295~p.  doi: 10.1075/scl.54.
\bibitem{2-zac}
\Au{Добровольский Д.\,О., Кретов~А.\, А., Шаров~С.\,А.} Корпус параллельных текстов~// 
Научная и~техническая информация. Сер.~2: Информационные процессы и~сис\-те\-мы, 2005. 
№\,6. С.~16--27.
\bibitem{3-zac}
\Au{Добровольский Д.\,О.} Корпус параллельных текстов и~сопоставительная 
лексикология~// Труды Института русского языка им.\ В.\,В.~Виноградова, 2015. №\,6. 
С.~413--449. EDN: VJQBHP.
\bibitem{4-zac}
\Au{Гончаров А.\,А., Зацман~И.\,М., Кружков~М.\,Г.} Эволюция классификаций 
в~надкорпусных базах данных~// Информатика и~её применения, 2020. Т.~14. Вып.~4. 
С.~108--116. doi: 10.14357/19922264200415.  
EDN: \mbox{GKWBZT}.
\bibitem{5-zac}
\Au{Гончаров А.\, А., Зацман И. \,М., Кружков~М.\, Г}. Представление новых 
лексикографических знаний в~динамических классификационных сис\-те\-мах~// 
Информатика и~её применения, 2021. Т.~15. Вып.~1. С.~86--93.  doi: 10.14357/19922264210112. EDN: OPEFXW.
\bibitem{6-zac}
\Au{Zatsman I.} Finding and filling lacunas in linguistic typologies~// 15th Forum (International) 
on Knowledge Asset Dynamics Proceedings.~--- Matera, Italy: Institute of Knowledge Asset 
Management, 2020. P.~780--793.
\bibitem{7-zac}
\Au{Zatsman I.} Three-dimensional encoding of emerging meanings in AI-systems~// 21st 
European Conference on Knowledge Management Proceedings.~--- Reading, U.K.: Academic 
Publishing International Ltd., 2020. P.~878--887.
\bibitem{8-zac}
\Au{Ackoff R.} From data to wisdom~// J.~Applied Systems Analysis, 1989. Vol.~16. No.\,1. P.~3--9.
\bibitem{9-zac}
\Au{Rosenbloom P.\,S.} On computing: The fourth great scientific domain.~--- Cambridge, MA, 
USA: MIT Press, 2013. 307~p.
\bibitem{10-zac}
\Au{Rowley J.} The wisdom hierarchy: Representations of the DIKW hierarchy~// J.~Inf. 
Sci., 2007. Vol.~33. Iss.~2. P.~163--180. doi: 10.1177/0165551506070706.
\bibitem{11-zac} 
\Au{Frick$\acute{\mbox{e}}$~M.\,H.} Data--Information--Knowledge--Wisdom (DIKW) pyramid, 
framework, continuum~// Encyclopedia of big data~/ Eds. L.~Schintler, C.~McNeely.~--- Cham: 
Springer, 2018. 4~p. doi: 10.1007/978-3-319-32001-4\_331-1.
\bibitem{12-zac}
\Au{Denning P., Rosenbloom~P.} Computing: The fourth great domain of science~// Commun. 
ACM, 2009. Vol.~52. Iss.~9. P.~27--29.
\bibitem{13-zac}
\Au{Denning P., Freeman~P.} Computing's paradigm~// Commun.  ACM, 2009. Vol.~52. 
Iss.~12. P.~28--30. doi: 10.1145/ 1610252.1610265.
\bibitem{17-zac} %14
\Au{Farradane J.} Knowledge, information, and information science~// J.~Inf. Sci., 
1980. Vol.~2. Iss.~2. P.~75--80. doi: 10.1177/01655515800020020.

\bibitem{15-zac}
\Au{Шрейдер Ю.\,А.} Информация и~знание~// Сис\-тем\-ная концепция информационных 
процессов.~--- М.: ВНИИСИ, 1988. С.~47--52.
\bibitem{16-zac}
\Au{Ingwersen P.} Information and information science~// Enclyclopaedie of library and 
information science~/ Eds. J.\,D.~McDonald, 
M.~Levine-Clark.~--- New York, NY, USA: Marcel Dekker Inc., 1992. Vol.~56. Sup.~19. 
P.~137--174.

\bibitem{14-zac} %17
Информатика как наука об информации: Информационный, документальный, 
технологический, экономический, социальный и~организационный аспекты~/ Под ред. 
Р.\,С.~Гиляревского.~--- М.: Фаир-Пресс, 2006. 592~с.

\bibitem{18-zac}
\Au{Hjorland B.} Library and information science: practice, theory, and philosophical basis~// 
Inform. Process. Manag., 2000. Vol.~36. Iss.~3. P.~501--531. doi:  
10.1016/S0306-\mbox{4573(99)00038-2}.
\bibitem{19-zac}
Deep shift~--- technology tipping points and societal impact.~--- Geneva: WE Forum, 2015. 44~p. 
{\sf http://www3.weforum.org/docs/WEF\_GAC15\_ Technological\_Tipping\_Points\_report\_2015.pdf}.
\bibitem{20-zac}
\Au{Berman F., Rutenbar~R., Hailpern~B., Christensen~H., Davidson~S., Estrin~D., 
Franklin~M., Martonosi~M., Raghavan~P., Stodden~V., Szalay~A.\,S.} Realizing the potential of 
data science~// Commun.  ACM, 2018. Vol.~61. Iss.~4. P.~67--72. doi: 10.1145/3188721.

\bibitem{21-zac}
\Au{Stodden V.} The data science life cycle: A~disciplined approach to advancing data science as 
a~science~// Commun.  ACM, 2020. Vol.~63. Iss.~7. P.~58--66. doi: 10.1145/ 3360646.


\bibitem{23-zac} %22
\Au{Зацман И.\,М.} Научная парадигма информатики: классификация трансформаций 
объектов предметной об\-ласти~// Системы и~средства информатики, 2023. Т.~33. №\,4. 
С.~126--138. doi: 10.14357/08696527230412. EDN: ZIKUWO.

\bibitem{22-zac} %23
\Au{Зацман И.\,М.} Научная парадигма информатики: классификация объектов предметной  
об\-ласти~// Информатика и~её применения, 2023. Т.~17. Вып.~4. С.~96--103. doi: 
10.14357/19922264230413. EDN: FIUQAT.

\bibitem{24-zac}
\Au{Зацман И.\,М.} О~научной парадигме информатики: верхний уровень классификации 
объектов ее предметной об\-ласти~// Информатика и~её применения, 2022. Т.~16. Вып.~4. 
С.~73--79. doi: 10.14357/ 19922264220411. EDN: XZNKVI.

\bibitem{25-zac}
\Au{Соломоник А.\,Б.} Философия знаковых систем и~язык.~--- М.: ЛКИ, 2011. 408~с.
\bibitem{26-zac}
\Au{Зацман И.\,М.} Трансформация иерархии Акоффа в~научной парадигме информатики~// 
Информатика и~её применения, 2023. Т.~17. Вып.~3. С.~107--113. doi: 
10.14357/19922264230315. EDN: UMVRRV.

\bibitem{27-zac}
\Au{Zatsman I.} Building digital spiral models of knowledge generation~// 19th Forum 
(International) on Knowledge Asset Dynamics Proceedings.~--- Matera, Italy: Arts for Business 
Institute, 2024. P.~2185--2196.
\bibitem{28-zac}
\Au{Zatsman I.} Digital spiral model of knowledge creation and encoding its dynamics~// 18th 
Forum (International) on Knowledge Asset Dynamics Proceedings.~--- Matera, Italy: Arts for 
Business Institute, 2023. P.~581--596.
\bibitem{29-zac}
\Au{Зацман И.\,М.} Интерфейсы третьего порядка в~информатике~// Информатика и~её 
применения, 2019. Т.~13. Вып.~3. С.~82--89. doi: 10.14357/19922264190312. EDN: 
EHRQLF.

\bibitem{30-zac}
\Au{Зацман И.\,М.} Научная парадигма информатики как третьей культуры~//  
На\-уч\-но-тех\-ни\-че\-ская информация. Сер.~1: Организация и~методика информационной 
работы, 2023. №\,11. С.~1--14.

\end{thebibliography}

 }
 }

\end{multicols}

\vspace*{-9pt}

\hfill{\small\textit{Поступила в~редакцию 14.04.24}}

\vspace*{4pt}

%\pagebreak

%\newpage

%\vspace*{-28pt}

\hrule

\vspace*{2pt}

\hrule



\def\tit{OBJECT TRANSFORMATIONS OF~THE~FIRST AND~SECOND ORDER
IN~A~LEXICOGRAPHIC INFORMATION SYSTEM\\[-5pt]}


\def\titkol{Object transformations of~the~first and~second order
in~a~lexicographic information system}


\def\aut{I.\,M.~Zatsman}

\def\autkol{I.\,M.~Zatsman}

\titel{\tit}{\aut}{\autkol}{\titkol}

\vspace*{-13pt}


\noindent
Federal Research Center ``Computer Science and Control'' of the Russian Academy of Sciences, 
44-2~Vavilov Str., Moscow 119133, Russian Federation


\def\leftfootline{\small{\textbf{\thepage}
\hfill INFORMATIKA I EE PRIMENENIYA~--- INFORMATICS AND
APPLICATIONS\ \ \ 2024\ \ \ volume~18\ \ \ issue\ 2}
}%
 \def\rightfootline{\small{INFORMATIKA I EE PRIMENENIYA~---
INFORMATICS AND APPLICATIONS\ \ \ 2024\ \ \ volume~18\ \ \ issue\ 2
\hfill \textbf{\thepage}}}

\vspace*{2pt}



\Abste{The theoretical foundations of the design of information technologies used for 
the integration of bilingual dictionaries and parallel corpora are considered. The 
description of the first outcomes of the creation of the third\linebreak\vspace*{-12pt}}

\Abstend{ level of object 
transformations classification in the subject domain of informatics, which is supposed 
to be used
in creating the lexicographic information system providing integration, is 
given. All the entities of informatics are divided into two global classes: objects and 
their transformations. For each such class, its own classification is constructed. 
Previously, the two upper levels of the object transformation classification in the subject 
domain have been described. The present paper discusses the third level of this classification. The 
basis for the construction of its highest level was the division of the subject domain of 
informatics into media (mental, sensory, digital, and a~number of other media), each 
of which by definition includes objects of the same nature. The Solomonick's 
typology of sign systems served as the basis for constructing the second level of the 
object transformation classification. The aim of the paper is to systematize object 
transformations of the first and second orders at the third level of this classification. 
The basis for systematization is the medium version of the Ackoff's hierarchy.}

\KWE{subject domain objects; object transformations; classification; data; 
information; knowledge; lexicographic information system}


\DOI{10.14357/19922264240211}{VZTGVV}

\vspace*{-12pt}

\Ack

\vspace*{-3pt}


\noindent
The reported study was funded by the Russian Science Foundation, project  
No.\,24-18-00155, {\sf 
https://rscf.ru/project/24-18-00155}. The research was carried out using the infrastructure of the Shared 
Research Facilities ``High Performance Computing and Big Data'' (CKP 
``Informatics'') of FRC CSC RAS (Moscow) .
   


  \begin{multicols}{2}

\renewcommand{\bibname}{\protect\rmfamily References}
%\renewcommand{\bibname}{\large\protect\rm References}

{\small\frenchspacing
 {%\baselineskip=10.8pt
 \addcontentsline{toc}{section}{References}
 \begin{thebibliography}{99} 
\bibitem{1-zac-1}
\Aue{Aijmer, K., and B.~Altenberg.} 2013. \textit{Advances in corpus-based 
contrastive linguistics. Studies in honour of Stig Johansson}. Amsterdam: John 
Benjamins. 295~p. doi: 10.1075/scl.54.
\bibitem{2-zac-1}
\Aue{Dobrovolskiy, D.\,O., A.\,A.~Kretov, and S.\,A.~Sharov.} 2005. Korpus 
parallel'nykh tekstov [Corpus of parallel texts]. \textit{Nauchnaya i~tekhnicheskaya 
informatsiya. Ser. 2. Informatsionnye protsessy i~sistemy} [Scientific and Technical 
Information. Ser.~2: Information Processes and Systems] 6:16--27.
\bibitem{3-zac-1}
\Aue{Dobrovolskiy, D.\,O.} 2015. Korpus parallel'nykh tekstov i~sopostavitel'naya 
leksikologiya [The corpus of parallel texts and contrastive lexicology]. \textit{Trudy 
Instituta russkogo yazyka im. V.\,V.~Vinogradova} [Proceedings of the 
V.\,V.~Vinogradov Russian Language Institute] 6:413--449. EDN: VJQBHP.
\bibitem{4-zac-1}
\Aue{Goncharov, A.\,A., I.\,M.~Zatsman, and M.\,G.~Kruzhkov.} 2020. Evolyutsiya 
klassifikatsiy v~nadkorpusnykh ba\-zakh dannykh [Evolution of classifications in 
supracorpora databases]. \textit{Informatika i~ee Primeneniya~--- Inform. \mbox{Appl.}}  
14(4):108--116. doi: 10.14357/19922264200415.  
EDN: GKWBZT.
\bibitem{5-zac-1}
\Aue{Goncharov, A.\,A., I.\,M.~Zatsman, and M.\,G.~Kruzhkov.} 2021. 
Predstavlenie novykh leksikograficheskikh znaniy v~dinamicheskikh 
klassifikatsionnykh sistemakh [Representation of new lexicographical knowledge in 
dynamic classification systems]. \textit{Informatika i~ee Primeneniya~--- Inform. 
Appl.} 15(1):86--93. doi: 10.14357/19922264210112. EDN: OPEFXW.
\bibitem{6-zac-1}
\Aue{Zatsman, I.} 2020. Finding and filling lacunas in linguistic typologies. 
\textit{15th Forum (International) on Knowledge Asset Dynamics Proceedings}. 
Matera, Italy: Institute of Knowledge Asset Management. 780--793.
\bibitem{7-zac-1}
\Aue{Zatsman, I.} 2020. Three-dimensional encoding of emerging meanings in  
AI-systems. \textit{21st European Conference on Knowledge Management 
Proceedings}. Reading, U.K.: Academic Publishing International Ltd. 878--887.
\bibitem{8-zac-1}
\Aue{Ackoff, R.} 1989. From data to wisdom. \textit{J.~Applied Systems Analysis} 
16(1):3--9.
\bibitem{9-zac-1}
\Aue{Rosenbloom, P.\,S.} 2013. \textit{On computing: The fourth great scientific 
domain}. Cambridge, MA: MIT Press. 307~p.
\bibitem{10-zac-1}
\Aue{Rowley, J.} 2007. The wisdom hierarchy: Representations of the DIKW 
hierarchy. \textit{J.~Inf. Sci.} 33(2):163--180. doi: 10.1177/0165551506070706.
\bibitem{11-zac-1}
\Aue{Frick$\acute{\mbox{e}}$, M.\,H.} 2018.  
Data-Information-Knowledge-Wisdom (DIKW) pyramid, framework, continuum. 
\textit{Encyclopedia of big data}. Eds. L.~Schintler and C.~McNeely. Cham: 
Springer. 4~p. doi: 10.1007/978-3-319-32001- 4\_331-1.
\bibitem{12-zac-1}
\Aue{Denning, P., and P.~Rosenbloom.} 2009. Computing: The fourth great domain 
of science. \textit{Commun. ACM} 52(9):27--29.
\bibitem{13-zac-1}
\Aue{Denning, P., and P.~Freeman.} 2009. Computing's paradigm. \textit{Commun. 
ACM} 52(12):28--30. doi: 10.1145/ 1610252.1610265.

\bibitem{17-zac-1} %14
\Aue{Farradane, J.} 1980. Knowledge, information, and information science. 
\textit{J.~Inf. Sci.} 2(2):75--80. doi: 10.1177/ 01655515800020020.

\bibitem{15-zac-1}
\Aue{Shreyder, Yu.\,A.} 1988. Informatsiya i~znanie [Information and knowledge]. 
\textit{Sistemnaya kontseptsiya in\-for\-ma\-tsi\-on\-nykh protsessov} [System concept of 
information processes]. Moscow: VNIISI. 47--52.
\bibitem{16-zac-1}
\Aue{Ingwersen, P.} 1995. Information and information science. 
\textit{Encyclopedia of library and information science}. Eds. J.\,D.~McDonald and 
M.~Levine-Clark. New York, NY: Marcel Dekker Inc. 56(19):137--174.

\bibitem{14-zac-1} %17
Gilyarevskiy, R.\,S., ed. 2006. \textit{Informatika kak nauka ob informatsii: 
informatsionnyy, dokumental'nyy, tekh\-no\-lo\-gi\-che\-skiy, ekonomicheskiy, sotsial'nyy 
i~organizatsionnyy aspekty} [Informatics as information science: Informational, 
documentary, technological, economic, social, and organizational dimensions]. 
Moscow: FAIR-PRESS. 592~p.

\bibitem{18-zac-1}
\Aue{Hjorland, B.} 2000. Library and information science: Practice, theory, and 
philosophical basis. \textit{Inform. Process. Manag.} 36(3):501--531. doi:  
10.1016/S0306-\mbox{4573(99)00038-2}.
\bibitem{19-zac-1}
Deep shift~--- technology tipping points and societal impact. 2015. \textit{World Economic 
Forum}. Geneva. 44~p. Available at: {\sf 
http://www3.weforum.org/docs/WEF\_ GAC15\_Technological\_Tipping\_Points\_report\_2015.pdf} (accessed May~20, 
2024).
\bibitem{20-zac-1}
\Aue{Berman, F., R.~Rutenbar, B.~Hailpern, H.~Christensen, S.~Davidson, 
D.~Estrin, M.~Franklin, M.~Martonosi, P.~Raghavan, V.~Stodden, and 
A.\,S.~Szalay.} 2018. Realizing the potential of data science. \textit{Commun. ACM} 
61(4):67--72. doi: 10.1145/3188721.
\bibitem{21-zac-1}
\Aue{Stodden, V.} 2020. The data science life cycle: A~disciplined approach to 
advancing data science as a~science. \textit{Commun. ACM} 
 63(7):58--66. doi: 10.1145/3360646.

\bibitem{23-zac-1} %22
\Aue{Zatsman, I.\,M.} 2023. Nauchnaya paradigma informatiki: klassifikatsiya 
transformatsiy ob''ektov predmetnoy oblasti [Scientific paradigm of informatics: 
Transformation classification of domain objects]. \textit{Sistemy i~Sredstva 
Informatiki~--- Systems and Means of Informatics} 33(4):126--138. doi: 
10.14357/08696527230412. EDN: ZIKUWO.

\bibitem{22-zac-1} %23
\Aue{Zatsman, I.\,M.} 2023. Nauchnaya paradigma informatiki: klassifikatsiya 
ob''ektov predmetnoy oblasti [Scientific paradigm of informatics: Classification of 
domain objects]. \textit{Informatika i~ee Primeneniya~--- Inform. Appl.} 
 17(4):96--103. doi: 10.14357/19922264230413. EDN: FIUQAT.
 
\bibitem{24-zac-1}
\Aue{   Zatsman, I.\,M.} 2022. O nauchnoy paradigme informatiki: verkhniy uroven' 
klassifikatsii ob''ektov ee predmetnoy oblasti [On the scientific paradigm of 
informatics: The classification high level of its objects]. \textit{Informatika i~ee 
Primeneniya~--- Inform. Appl.} 16(4):73--79. doi: 10.14357/19922264220411. EDN: 
XZNKVI.
\bibitem{25-zac-1}
\Aue{Solomonick, A.\,B.} 2011. \textit{Filosofiya znakovykh system i~yazyk} 
[Philosophy of sign systems and language]. Moscow: LKI. 408~p.
\bibitem{26-zac-1}
\Aue{Zatsman, I.\,M.} 2023. Transformatsiya ierarkhii Akoffa v~nauchnoy 
paradigme informatiki [Transformation of the Ackoff's hierarchy in the scientific 
paradigm of informatics]. \textit{Informatika i~ee Primeneniya~--- Inform. \mbox{Appl.}} 
17(3):107--113. doi: 10.14357/19922264230315. EDN: UMVRRV.
\bibitem{27-zac-1}
\Aue{Zatsman, I.} 2024. Building digital spiral models of knowledge 
generation. \textit{19th Forum (International) on Knowledge Asset Dynamics 
Proceedings}. Matera, Italy: Arts for Business Institute. 2185--2196.
\bibitem{28-zac-1}
\Aue{Zatsman, I.} 2023. Digital spiral model of knowledge creation and encoding its 
dynamics. \textit{18th Forum (International) on Knowledge Asset Dynamics 
Proceedings}. Matera, Italy: Arts for Business Institute. 581--596.
\bibitem{29-zac-1}
\Aue{Zatsman, I.\,M.} 2019. Interfeysy tret'ego poryadka v~informatike 
 [Third-order interfaces in informatics]. \textit{Informatika i~ee Primeneniya~--- 
Inform. Appl.} 13(3):82--89. doi: 10.14357/19922264190312. EDN: EHRQLF.
\bibitem{30-zac-1}
\Aue{Zatsman, I.} 2023. Scientific paradigm of informatics as a~third culture. 
\textit{Scientific Technical Information Processing} 50(4):246--258. doi: 
10.3103/S0147688223040111. EDN: CKHMYS.

\end{thebibliography}

 }
 }

\end{multicols}

\vspace*{-6pt}

\hfill{\small\textit{Received April 14, 2024}} 


\vspace*{-12pt}


\Contrl

\vspace*{-3pt}

\noindent
\textbf{Zatsman Igor M.} (b.\ 1952)~--- Doctor of Science in technology, head of 
department, Federal Research Center ``Computer Science and Control'' of the 
Russian Academy of Sciences, 44-2~Vavilov Str., Moscow 119333, Russian 
Federation; \mbox{izatsman@yandex.ru}





\label{end\stat}

\renewcommand{\bibname}{\protect\rm Литература} 


%\end{document}




\def\stat{rez}
{%\hrule\par
%\vskip 7pt % 7pt
\raggedleft\Large \bf%\baselineskip=3.2ex
Р\,Е\,Ц\,Е\,Н\,З\,И\,И \vskip 17pt
    \hrule
    \par
\vskip 6pt plus 6pt minus 3pt }

%\thispagestyle{headings} %с верхним колонтитулом
%\thispagestyle{myheadings} %с нижним колонтитулом, но в верхнем РЕЦЕНЗИИ

\def\tit{НОВАЯ КНИГА И.\,Н.~СИНИЦЫНА, А.\,С.~ШАЛАМОВА <<ЛЕКЦИИ ПО ТЕОРИИ 
ИНТЕГРИРОВАННОЙ ЛОГИСТИЧЕСКОЙ ПОДДЕРЖКИ>> (М.: ТОРУС ПРЕСС, 2012. 624~с.)}

%1
\def\aut{Д.ф.-м.н., профессор С.\,Я.~Шоргин}

\def\auf{\ }

\def\leftkol{\ % РЕЦЕНЗИИ
}

\def\rightkol{ \ } 

%\def\leftkol{\ } % ENGLISH ABSTRACTS}

%\def\rightkol{\ } %ENGLISH ABSTRACTS}

%\def\leftkol{РЕЦЕНЗИИ}

%\def\rightkol{РЕЦЕНЗИИ}

\titele{\tit}{\aut}{\auf}{\leftkol}{\rightkol}
\vspace*{-18pt}


     \label{st\stat}

     \begin{multicols}{2}
     {\small
     {\baselineskip=10.1pt
     

      В книге представлено системное изложение теоретических основ одного из новейших 
направлений в \mbox{об\-ласти} экономики послепродажного обслуживания изделий наукоемкой 
продукции (ИНП) длительного пользования~--- интегрированной логистической поддержки
(ИЛП). 
{\looseness=1

}

Приведены также результаты новых работ, выполненных в Институте проблем информатики 
Российской академии наук в рамках научного направления <<Информационные технологии и 
анализ сложных сис\-тем>>.
 {%\looseness=1

}
     
      Излагаемые в книге научные подходы позво\-ляют карди\-наль\-но реформировать 
существующие системы производства и эксплуатации ИНП путем создания и внед\-ре\-ния 
методов рационального и оптимального управ\-ле\-ния процессами расходования 
вре\-мен\-н$\acute{\mbox{ы}}$х, 
мате\-ри\-аль\-ных, трудовых и других ресурсов на всех стадиях жизненного цикла изделий (ЖЦИ) по 
критериям экономической целесообразности и эф\-фек\-тив\-ности.
  {\looseness=1

}
    
      В книге приведен краткий обзор причин возник\-новения и
      развития CALS-методологии как основы 
современных международных стандартов по созданию и функционированию глобальных 
ин\-фор\-ма\-ци\-он\-но-ком\-му\-ни\-ка\-ци\-он\-ных систем, ее ключевых возможностей и эффективности 
результатов ее использования. 
Авторы %\linebreak 
предлагают ряд научных обоснований для разработки 
единой теории проектирования и управления систем ИЛП для полноценного использования 
преимуществ %\linebreak
 суще\-ст\-ву\-ющей методологии, определяют \mbox{общую} структурную схему 
комплексной системы <<ИНП-СППО>> и необходимость разработки для ее описания 
гибридных стохастических моделей.
{%\looseness=1

}

%\columnbreak
      
      Книга состоит из пяти частей, где последовательно излагается материал по каждой из 
следующих тем: <<Интегрированная логистическая поддержка>>, <<Теория гибридных 
стохастических систем и компьютерная поддержка исследований и разработок>>, <<Основы 
математического моделирования, анализа и синтеза систем послепродажного обслуживания>>, 
<<Определение и анализ показателей экспортного потенциала ИНП при проектировании>>, 
<<Задачи управления поддержкой послепродажного обслуживания>>, а также 
<<Моделирование инвестиционных процессов ИЛП в условиях неравновесных финансовых 
рынков>>. 
   
      В конце каждой главы приведены выводы и даны вопросы и задания для 
самоконтроля. В~приложениях содержатся основные определения по программам работ по 
анализу ИЛП, логистическим базам данных и компьютерным решениям, эквивалентной статистической 
линеаризации нелинейных преобразований ИЛП, справочный материал, а также развернутые 
уравнения для вероятностных характеристик.


      \def\leftkol{РЕЦЕНЗИИ}

\def\rightkol{РЕЦЕНЗИИ} 

      
      Книга заинтересует широкий круг специалистов и может быть использована научными 
проектными организациями в сфере промышленного производства ИНП. Большое количество 
иллюстраций, примеров и вопросов, обращенных к читателю, позволяет использовать книгу 
также в качестве учебного пособия для студентов и аспирантов машиностроительных, 
транспортных и~других специальностей, а также для самостоятельного изучения. 
{%\looseness=-1

}

Книга 
представляет несомненный интерес для специалистов и студентов в области прикладной 
математики и информатики.
    

}

}
\end{multicols}

%\newpage

\include{obchak}




\def\stat{authorsrus}
{%\hrule\par
%\vskip 7pt % 7pt
\raggedleft\Large \bf%\baselineskip=3.2ex
О\,Б\ \ А\,В\,Т\,О\,Р\,А\,Х \vskip 17pt
    \hrule
    \par
\vskip 21pt plus 8pt minus 4pt }


\def\tit{\ }

\def\aut{\ }

\def\auf{\ }

\def\leftkol{\ } % ENGLISH ABSTRACTS}

\def\rightkol{ОБ АВТОРАХ} %ENGLISH ABSTRACTS}

\titele{\tit}{\aut}{\auf}{\leftkol}{\rightkol}
      
            \label{st\stat}



\vspace*{24pt}

\begin{multicols}{2}




\noindent
\textbf{Архипов Олег Петрович} (р.\ 1948)~---
кандидат технических наук, директор Орловского филиала Института проб\-лем информатики
Российской академии наук
%302025, г.Орел, Московское шоссе, д.137

\vspace*{3pt}

\noindent
\textbf{Бирюкова Татьяна Константиновна} (р.\ 1968)~---
кандидат фи\-зи\-ко-ма\-те\-ма\-ти\-че\-ских наук, старший научный сотрудник Института проб\-лем информатики
Российской академии наук

\vspace*{3pt}

\noindent 
\textbf{Бобков  Сергей Геннадьевич} (р.\ 1955)~---
доктор технических наук,  заведующий отделением На\-уч\-но-ис\-сле\-до\-ва\-тель\-ско\-го 
института системных исследований Российской академии наук
%117218, Москва, Нахимовский просп., 36, к.1 

\vspace*{3pt}

\noindent \textbf{Васильев Николай Семенович} (р.\ 1952)~--- доктор 
фи\-зи\-ко-ма\-те\-ма\-ти\-че\-ских наук, профессор, 
МГТУ им.\ Н.\,Э.~Баумана 
%, Москва 105005, 2-я Бауманская ул., д.~5,

\vspace*{3pt}

\noindent
\textbf{Гершкович Максим Михайлович} (р.\ 1968)~---
старший научный сотрудник Института проб\-лем информатики
Российской академии наук

\vspace*{3pt}

\noindent 
\textbf{Дьяченко Юрий Георгиевич} (р.\ 1958)~--- кандидат технических наук, 
старший научный сотрудник Института проб\-лем информатики
Российской академии наук

\vspace*{3pt}

\noindent 
\textbf{Ерошенко Александр Андреевич} (р.\ 1989)~--- аспирант кафедры 
математической статистики факультета вычисли\-тельной математики и кибернетики 
Московского государственного университета им.\ М.\,В.~Ломоносова
%119991, Москва ГСП-1, Ленинские горы, д.\ 1, стр. 52

\vspace*{3pt}
 
\noindent 
\textbf{Захаров Виктор Николаевич} (р.\ 1948)~--- 
доктор технических наук, доцент, ученый секретарь Института проб\-лем информатики
Российской академии наук

\vspace*{3pt}

\noindent
\textbf{Зейфман Александр Израилевич} (р.\ 1954)~---
доктор фи\-зи\-ко-ма\-те\-ма\-ти\-че\-ских наук, профессор, 
заведующий кафедрой Вологодского государственного университета; 
старший научный сотрудник Института проб\-лем информатики
Российской академии наук; главный научный сотрудник ИСЭРТ Российской академии наук

\vspace*{3pt}

\noindent
\textbf{Зыкин Сергей Владимирович} (р.\ 1959)~--- 
доктор технических наук, профессор, заведующий лабораторией Института математики 
им.\ С.\,Л.~Соболева Сибирского отделения Российской академии наук, Новосибирск 
%630090, пр.\ ак.\ Коптюга, 4 

\vspace*{4pt}

\noindent
\textbf{Киреев Владимир Иванович} (р.\ 1938)~---
доктор фи\-зи\-ко-ма\-те\-ма\-ти\-че\-ских наук, профессор Московского 
государственного горного университета
%Адрес: Россия, 119991, г. Москва, Ленинский проспект, д. 6

%\columnbreak

\vspace*{4pt}

\noindent
\textbf{Козеренко Елена Борисовна} (р.\ 1959)~---
кандидат филологических наук, заведующая лабораторией Института проб\-лем информатики
Российской академии наук

\vspace*{4pt}

\noindent
\textbf{Королев Виктор Юрьевич} (р.\ 1954)~--- доктор
фи\-зи\-ко-ма\-те\-ма\-ти\-че\-ских наук, профессор кафедры математической 
статистики факультета вычисли\-тельной математики и кибернетики 
Московского государственного университета; 
ведущий научный сотрудник Института проб\-лем информатики
Российской академии наук

\vspace*{4pt}

\noindent
\textbf{Коротышева Анна Владимировна} (р.\ 1988)~---
старший преподаватель Вологодского государственного университета

\vspace*{4pt}

\noindent 
\textbf{Кун Де Турк} (р.\ 1981)~--- научный сотрудник 
исследовательской группы SMACS факультета телекоммуникаций и обработки информации
Университета Гента, Бельгия
%В-9000 Гент, Бельгия

\vspace*{4pt}

\noindent
\textbf{Лупенцов Олег Сергеевич} (р.\ 1986)~---
аспирант Омского государственного института сервиса
%Омск 644043, ул.\ Певцова 13

\vspace*{4pt}

\noindent
\textbf{Лучко Олег Николаевич} (р.\ 1961)~---
кандидат педагогических наук, профессор, заведующий кафедрой 
Омского государственного института сервиса
%Омск 644043, ул.\ Певцова 13

\vspace*{4pt}

\noindent
\textbf{Малашенко Юрий Евгеньевич} (р.\ 1946)~---
доктор фи\-зи\-ко-ма\-те\-ма\-ти\-че\-ских наук, заведующий сектором 
Вычислительного центра им.\ А.\,А.~Дородницына Российской академии наук
%Адрес: 119333, Москва, ул. Вавилова, 40,

\vspace*{4pt}

\noindent
\textbf{Маньяков Юрий Анатольевич} (р.\ 1984)~---
кандидат технических наук, научный сотрудник Орловского филиала Института проб\-лем информатики
Российской академии наук
%302025, г.Орел, Московское шоссе, д.137

\vspace*{4pt}

\noindent
\textbf{Маренко Валентина Афанасьевна} (р.\ 1951)~---
кандидат технических наук, доцент, старший научный сотрудник 
Института математики им.\ С.\,Л.~Соболева Сибирского отделения Российской академии наук
%Новосибирск 630090, пр. ак. Коптюга, 4 

\vspace*{3pt}

\noindent 
\textbf{Морозов Евсей Викторович} (р.\ 1947)~--- доктор 
фи\-зи\-ко-ма\-те\-ма\-ти\-че\-ских, профессор, ведущий научный сотрудник 
Института прикладных математических исследований Карельского научного центра Российской
академии наук; 
%%185910 Россия, Республика Карелия, г.\ Петрозаводск, ул.\ Пушкинская, 11
профессор Петрозаводского государственного университета, Петрозаводск
%185910 Россия, Республика Карелия, г.\ Петрозаводск, пр.\ Ленина, 33

%\pagebreak

\vspace*{3pt}

\noindent
\textbf{Назарова Ирина Александровна} (р.\ 1966)~---
кандидат фи\-зи\-ко-ма\-те\-ма\-ти\-че\-ских наук, 
научный сотрудник Вычислительного центра им.\ А.\,А.~Дородницына Российской академии наук 
%Адрес: 119333, Москва, ул. Вавилова, 40

\vspace*{3pt}

\noindent
\textbf{Павлов Игорь Валерианович} (р.\ 1945)~--- 
доктор фи\-зи\-ко-ма\-те\-ма\-ти\-че\-ских наук, профессор МГТУ им.\ Н.\,Э.~Баумана 
%Москва 105005, 2-я Бауманская ул., д.~5 

%\pagebreak

\vspace*{3pt}

\noindent 
\textbf{Потахина Любовь Викторовна} (р.\ 1989)~--- аспирантка
Института прикладных математических исследований Карельского научного центра
Российской академии наук; 
%%185910 Россия, Республика Карелия, г.\ Петрозаводск, ул.\ Пушкинская, 11
инженер Петрозаводского государственного университета, Петрозаводск
%185910 Россия, Республика Карелия, г.\ Петрозаводск, пр.\ Ленина, 33

\vspace*{3pt}

\noindent 
\textbf{Рождественский Юрий Владимирович} (р.\ 1952)~--- 
кандидат технических наук, заведующий сектором Института проб\-лем информатики
Российской академии наук

\vspace*{3pt}

\noindent 
\textbf{Синицын Игорь Николаевич} (р.\ 1940)~--- доктор технических наук,
профессор, заслуженный деятель\linebreak\vspace*{-12pt}

\columnbreak

\noindent
 науки РФ, заведующий отделом Института проб\-лем информатики
Российской академии наук

\vspace*{7pt}


\noindent
\textbf{Сиротинин Денис Олегович} (р.\ 1984)~---
кандидат технических наук, научный сотрудник Орловского филиала Института проб\-лем информатики
Российской академии наук
%302025, г.Орел, Московское шоссе, д.137

\vspace*{7pt}

%\columnbreak

\noindent 
\textbf{Соколов  Игорь Анатольевич} (р.\ 1954)~--- академик (действительный член) Российской 
академии наук, доктор технических наук, директор Института проб\-лем информатики
Российской академии наук

\vspace*{7pt}

\noindent
\textbf{Степченков Юрий Афанасьевич} (р.\ 1951)~---
кандидат технических наук, заведующий отделом Института проб\-лем информатики
Российской академии наук

\vspace*{7pt}

\noindent
\textbf{Сурков Алексей Викторович} (р.\ 1978)~--- 
старший научный сотрудник На\-уч\-но-ис\-сле\-до\-ва\-тель\-ско\-го 
института системных исследований Российской академии наук
%117218, Москва, Нахимовский просп., 36, к.1 

\vspace*{7pt}

\noindent 
\textbf{Шестаков Олег Владимирович} (р.\ 1976)~--- доктор 
фи\-зи\-ко-ма\-те\-ма\-ти\-че\-ских, доцент кафедры математической статистики 
факультета вычисли\-тельной математики и кибернетики Московского 
государственного университета им.\ М.\,В.~Ломоносова; 
%119991, Москва ГСП-1, Ленинские горы, д.\ 1, стр. 52
старший научный сотрудник Института проб\-лем информатики
Российской академии наук
%, Москва 119333, ул. Вавилова, д.~44, корп.~2

\vspace*{7pt}

\noindent 
\textbf{Шоргин Сергей Яковлевич} (р.\ 1952.)~--- доктор
фи\-зи\-ко-ма\-те\-ма\-ти\-че\-ских наук, профессор, заместитель директора Института 
проб\-лем информатики Российской академии наук





%%%%%%%%%%%%%%%%%%%%%%%%%%%%%%%%%%%%%%%%%%%%%%%%%%%%%%%%%%%%%%%%%%%%%%%%%%%%%%%




%\def\rightkol{ОБ АВТОРАХ}
%\def\leftkol{ОБ АВТОРАХ}

 \label{end\stat}





%\def\leftfootline{\small{\textbf{\thepage}
%\hfill ИНФОРМАТИКА И ЕЁ ПРИМЕНЕНИЯ\ \ \ том~7\ \ \ выпуск~1\ \ \ 2013}
%}%
% \def\rightfootline{\small{ИНФОРМАТИКА И ЕЁ ПРИМЕНЕНИЯ\ \ \ том~7\ \ \ выпуск~1\ \ \ 2013
%\hfill \textbf{\thepage}}}


%\thispagestyle{myheadings}



\end{multicols}

\newpage



%\vspace*{-48pt}
\begin{center}\LARGE
\textit{About Authors}
\end{center}

\thispagestyle{empty}
\def\tit{\ }

\def\aut{\ }

\def\auf{\ }


\def\leftkol{ABOUT AUTHORS}

\def\rightkol{ABOUT AUTHORS}

\vspace*{-18pt}

\titele{\tit}{\aut}{\auf}{\leftkol}{\rightkol}

%\vspace*{36pt}

\def\rightmark{{\noindent\hbox to \textwidth{\hfill\small ABOUT AUTHORS
%\hfill \large\bf\thepage
}}}
\def\leftmark{{\noindent\parbox{\textwidth}{
%\raggedleft\large\bf\thepage \hfill
\small\textrm{ABOUT AUTHORS}\hfill}}}


\def\leftfootline{\small{\textbf{\thepage}
\hfill ИНФОРМАТИКА И ЕЁ ПРИМЕНЕНИЯ\ \ \ том~6\ \ \ выпуск~2\ \ \ 2012}
}%
 \def\rightfootline{\small{ИНФОРМАТИКА И ЕЁ ПРИМЕНЕНИЯ\ \ \ том~6\ \ \ выпуск~2\ \ \ 2012
\hfill \textbf{\thepage}}}


\begin{multicols}{2}

\noindent
\textbf{Agalarov Yaver M.} (b.\ 1952)~--- Candidate of Science (PhD)
in technology, 
leading scientist, Institute of Informatics Problems, Russian Academy of Sciences

\vspace*{5pt}


  \noindent
\textbf{Bosov Alexey V.} (b.\ 1969)~--- Doctor of Science in technology, Head of
Laboratory, Institute of Informatics Problems, Russian Academy of Sciences

\vspace*{5pt}


\noindent
\textbf{Dulin Sergey K.} (b.\ 1950)~--- Doctor of Science in technology, 
professor, senior scientist, Institute of Informatics Problems, Russian Academy of Sciences

\vspace*{5pt}

\noindent
\textbf{Gorshenin Andrey K.}~--- (b.\ 1986)~--- Candidate of Science (PhD)
in physics and mathematics,
senior scientist, Institute of Informatics Problems, Russian Academy of Sciences

\vspace*{5pt}

\noindent
\textbf{Kalenov Nikolay E.}  (b.\ 1945)~--- Doctor of Science in technology,
professor, Director, Library for Natural Sciences,  Russian Academy of Sciences 

\vspace*{5pt}

\noindent
\textbf{Kalinichenko Leonid A.} (b.\ 1937)~--- Doctor of Science in physics and mathematics, 
professor, Honored scientist of RF, 
Head of Laboratory, Institute of Informatics Problems, Russian Academy of Sciences 

\vspace*{5pt}

\noindent
\textbf{Karpov Alexey A.} (b.\ 1978)~--- Candidate of Science (PhD) in technology, 
senior scientist, St.\ Petersburg Institute for
Informatics and Automation,  Russian Academy of Sciences

\vspace*{5pt}

\noindent
\textbf{Kuznetsov Igor P.} (b.\ 1938)~--- Doctor of Science in technology, 
professor, principal scientist, Institute of Informatics Problems, Russian Academy of Sciences

\vspace*{5pt}


\noindent
\textbf{Markova Natalia A.} (b.\ 1950)~--- Candidate of Science (PhD) in
physics and mathematics, leading scientist,  
Institute of Informatics Problems, Russian Academy of Sciences

\vspace*{5pt}

\noindent
\textbf{Nikolaev Andrey V.} (b.\ 1985)~--- Candidate of Science (PhD) in technology, 
senior lecturer, Tchaikovsky Technological Institute, Branch of the Izhevsk State Technical 
University

\vspace*{6pt}

\noindent
\textbf{Pavlov Igor V.} (b.\ 1945)~---  Doctor of Science in physics and mathematics,
professor, Bauman Moscow State Technical University

\vspace*{6pt}

%\columnbreak

\noindent
\textbf{Rozenberg Igor N.} (b.\ 1965)~--- Doctor of Science in technology, 
First Deputy Director General, Research \& Design Institute for Information 
Technology, Signalling and Telecommunications on Railway Transport (JSC NIIAS)

\vspace*{6pt}


\noindent
\textbf{Semenov Konstantin K.} (b.\ 1986)~--- MPhil, 
associate professor, St.\ Petersburg State Polytechnical University

\vspace*{6pt}

\noindent
\textbf{Sharnin Mikhail M.} (b.\ 1959)~--- Candidate of Science (PhD) 
in technology, senior scientist, Institute of Informatics Problems, Russian Academy of Sciences

\vspace*{6pt}

\noindent 
\textbf{Shestakov Oleg V.} (b.\ 1976)~--- Candidate of Science (PhD) in physics and mathematics,
associate professor, Department of Mathematical Statistics, Faculty of Computational Mathematics and Cybernetics,
M.\,V.~Lomonosov Moscow State University; senior scientist, Institute of Informatics Problems, 
Russian Academy of Sciences

\vspace*{6pt}

\noindent
\textbf{Stupnikov Sergey A.} (b.\ 1978)~--- Candidate of Science (PhD) in technology, 
senior scientist, Institute of Informatics Problems, Russian Academy of Sciences 

\vspace*{6pt}

\noindent
\textbf{Umansky Vladimir I.} (b.\ 1954)~--- Candidate of Science (PhD) in technology, 
Director General, ``IntechGeoTrans'' Closed Joint Stock Company

\vspace*{6pt}

\noindent
\textbf{Zhevnerchuk Dmitry V.} (b.\ 1978)~--- Candidate of Science (PhD) in technology, 
associate professor, Tchaikovsky Technological Institute, Branch of the Izhevsk State 
Technical University

%\vspace*{6pt}

\def\leftfootline{\small{\textbf{\thepage}
\hfill ИНФОРМАТИКА И ЕЁ ПРИМЕНЕНИЯ\ \ \ том~6\ \ \ выпуск~2\ \ \ 2012}
}%
 \def\rightfootline{\small{ИНФОРМАТИКА И ЕЁ ПРИМЕНЕНИЯ\ \ \ том~6\ \ \ выпуск~2\ \ \ 2012
\hfill \textbf{\thepage}}}



%\thispagestyle{myheadings}

\end{multicols}
\newpage



%\tableofcontents


%\end{document}


\def\stat{cont}
{%\hrule\par
%\vskip 7pt % 7pt
\raggedleft\Large \bf%\baselineskip=3.2ex
А\,В\,Т\,О\,Р\,С\,К\,И\,Й\ \ У\,К\,А\,З\,А\,Т\,Е\,Л\,Ь\ \ З\,А\ \ 2\,0\,1\,0 г. \vskip 17pt
    \hrule
    \par
\vskip 21pt plus 6pt minus 3pt }

\label{st\stat}

\def\tit{\ }

\def\aut{\ }
\def\auf{\ }

\def\leftkol{\ } % ENGLISH ABSTRACTS}

\def\rightkol{\ } %АВТОРСКИЙ УКАЗАТЕЛЬ ЗА 2010 г.} %ENGLISH ABSTRACTS}

\titele{\tit}{\aut}{\auf}{\leftkol}{\rightkol}

\vspace*{-12pt}

{\tabcolsep=3pt
\begin{tabular}{p{388pt}rr}
&\textbf{Выпуск} & \textbf{Стр.}\\[6pt]
\hangindent=23pt\noindent\textbf{Арутюнян~А.\,Р.} Моделирование влияния деформаций отпечатков пальцев на 
точность\linebreak
\vspace*{-12pt}\\
\hspace*{23pt}дактилоскопической идентификации$\dotfill$&1&51\\
\hangindent=23pt\noindent\textbf{Архипов~О.\,П., Зыкова~З.\,П.} Интеграция гетерогенной информации о цветных 
пикселях\linebreak
\vspace*{-12pt}\\
\hspace*{23pt}и их цветовосприятии$\dotfill$&4&15\\
\hangindent=23pt\noindent\textbf{Баранов~С.\,И., Френкель~С.\,Л., Захаров~В.\,Н.} Полуформальная верификация 
цифрового устройства с конвейером, основанная на использовании алгоритмических машин\linebreak
\vspace*{-12pt}\\
\hspace*{23pt}состояния$\dotfill$&4&49\\
\textbf{Бекетова~И.\,В.} см.~Каратеев~С.\,Л.&&\\
\textbf{Белоусов~В.\,В.} см.~Синицын~И.\,Н.&&\\
\hangindent=23pt\noindent\textbf{Бенинг~В.\,Е., Королев~Р.\,А.} О предельном поведении мощностей критериев в 
случае\linebreak
\vspace*{-12pt}\\
\hspace*{23pt}распределения Лапласа$\dotfill$&2&63\\
\hangindent=23pt\noindent\textbf{Бенинг~В.\,Е., Сипина~А.\,В.} Асимптотическое разложение для мощности 
критерия,\linebreak
\vspace*{-12pt}\\
\hspace*{23pt}основанного на выборочной медиане, в случае распределения Лапласа$\dotfill$&1&18\\
\textbf{Бондаренко~А.\,В.} см.~Каратеев~С.\,Л.&&\\
\hangindent=23pt\noindent\textbf{Бородина~А.\,В., Морозов~Е.\,В.} Об оценивании асимптотики вероятности 
большого\linebreak
\vspace*{-12pt}\\
\hspace*{23pt}уклонения стационарной регенеративной очереди с одним прибором$\dotfill$&3&29\\
\hangindent=23pt\noindent\textbf{Бунтман~Н.\,В., Минель~Ж.-Л., Ле~Пезан~Д., Зацман~И.\,М.} Типология и 
компьютерное\linebreak
\vspace*{-12pt}\\
\hspace*{23pt}моделирование трудностей перевода$\dotfill$&3&77\\
\textbf{Визильтер~Ю.\,В.} см.~Каратеев~С.\,Л.&&\\
\hangindent=23pt\noindent\textbf{Гавриленко~С.\,В.} Оценки скорости сходимости распределений случайных сумм с 
безгранично делимыми индексами к нормальному закону$\dotfill$&4&81\\
\hangindent=23pt\noindent\textbf{Григорьева~М.\,Е., Шевцова~И.\,Г.} Уточнение неравенства 
Каца--Берри--Эссеена$\dotfill$&2&75\\
\hangindent=23pt\noindent\textbf{Грушо~А.\,А., Грушо~Н.\,А., Тимонина~Е.\,Е.} Поиск конфликтов в политиках 
безопасности: модель случайных графов$\dotfill$&3&38\\
\textbf{Грушо~Н.\,А.} см.~Грушо~А.\,А.&&\\
\hangindent=23pt\noindent\textbf{Гудков~В.\,Ю.} Математические модели изображения отпечатка пальца на основе 
описания линий$\dotfill$&1&58\\
\textbf{Гуртов~А.\,В.} см.~Лукьяненко~А.\,С.&&\\
\textbf{Желтов~С.\,Ю.} см.~Каратеев~С.\,Л.&&\\
\hangindent=23pt\noindent\textbf{Захаров~А.\,А., Серебряков~В.\,А.} Система управления электронной библиотекой 
LibMeta$\dotfill$&4&2\\
\textbf{Захаров~В.\,Н.} см.~Баранов~С.\,И.&&\\
\textbf{Захарова~Т.\,В.} см.~Матвеева~С.\,С.&&\\
\hangindent=23pt\noindent\textbf{Зацаринный~А.\,А., Чупраков~К.\,Г.} Некоторые аспекты выбора технологии для 
постро-\linebreak
\vspace*{-12pt}\\
\hspace*{23pt}ения систем отображения информации ситуационного центра$\dotfill$&3&59\\
\textbf{Зацман~И.\,М.} см.~Бунтман~Н.\,В.&&\\
\hangindent=23pt\noindent\textbf{Зейфман~А.\,И., Коротышева~А.\,В., Сатин~Я.\,А., Шоргин~С.\,Я.} Об 
устойчивости нестаци-\linebreak
\vspace*{-12pt}\\
\hspace*{23pt}онарных систем обслуживания с катастрофами$\dotfill$&3&9\\
\textbf{Зыкова~З.\,П.} см.~Архипов~О.\,П.&&\\
\hangindent=23pt\noindent\textbf{Илюшин~Г.\,Я., Соколов~И.\,А.} Организация управляемого доступа пользователей 
к\linebreak
\vspace*{-12pt}\\
\hspace*{23pt}разнородным ведомственным информационным ресурсам$\dotfill$&1&24\\
\hangindent=23pt\noindent\textbf{Кавагучи~Ю., Ульянов~В.\,В., Фуджикоши~Я.} Приближения для статистик, 
описывающих\linebreak
\vspace*{-12pt}\\
\hspace*{23pt}геометрические свойства данных большой размерности, с оценками 
ошибок$\dotfill$&1&12\\
\hangindent=23pt\noindent\textbf{Каратеев~С.\,Л., Бекетова~И.\,В., Ососков~М.\,В., Князь~В.\,А., 
Визильтер~Ю.\,В., Бондаренко~А.\,В., Желтов~С.\,Ю.} Автоматизированный контроль 
качества цифровых\linebreak
\vspace*{-12pt}\\
\hspace*{23pt}изображений для персональных документов$\dotfill$&1&65\\
\end{tabular}
}

\pagebreak

\def\leftkol{АВТОРСКИЙ УКАЗАТЕЛЬ ЗА 2010 г.} % ENGLISH ABSTRACTS}

\def\rightkol{АВТОРСКИЙ УКАЗАТЕЛЬ ЗА 2010 г.} %ENGLISH ABSTRACTS}

{\tabcolsep=3pt
\begin{tabular}{p{388pt}rr}
&\textbf{Выпуск} & \textbf{Стр.}\\[3pt]
\hangindent=23pt\noindent\textbf{Козеренко~Е.\,Б.} Лингвистические фильтры в статистических моделях машинного\linebreak
\vspace*{-12pt}\\
\hspace*{23pt}перевода$\dotfill$&2&83\\
\hangindent=23pt\noindent\textbf{Козеренко~Е.\,Б., Кузнецов~И.\,П.} Когнитивно-лингвистические представления в 
систе-\linebreak
\vspace*{-12pt}\\
\hspace*{23pt}мах обработки текстов$\dotfill$&3&69\\
\textbf{Князь~В.\,А.} см.~Каратеев~С.\,Л.&&\\
\hangindent=23pt\noindent\textbf{Колесников~А.\,В., Солдатов~С.\,А.} Алгоритм координации для гибридной 
интеллектуальной системы решения сложной задачи оперативно-производственного\linebreak
\vspace*{-12pt}\\
\hspace*{23pt}планирования$\dotfill$&4&61\\
\hangindent=23pt\noindent\textbf{Коновалов~М.\,Г.} О планировании потоков в системах вычислительных 
ресурсов$\dotfill$&2&3\\
\textbf{Конушин~А.\,С.} см.~Конушин~В.\,С.&&\\
\hangindent=23pt\noindent\textbf{Конушин~В.\,С., Кривовязь~Г.\,Р., Конушин~А.\,С.} Алгоритм распознавания людей 
в видео-\linebreak
\vspace*{-12pt}\\
\hspace*{23pt}последовательности по одежде$\dotfill$&1&74\\
\textbf{Корепанов~Э.\, Р.} см.~Синицын~И.\,Н.&&\\
\textbf{Королев~В.\,Ю.} см.~Соколов~И.\,А.&&\\
\textbf{Королев~Р.\,А.} см.~Бенинг~В.\,Е.&&\\
\textbf{Коротышева~А.\,В.} см.~Зейфман~А.\,И.&&\\
\hangindent=23pt\noindent\textbf{Кривенко~М.\,П.} Непараметрическое оценивание элементов байесовского 
клас\-си-\linebreak
\vspace*{-12pt}\\
\hspace*{23pt}фикатора$\dotfill$&2&13\\
\textbf{Кривовязь~Г.\,Р.} см.~Конушин~В.\,С.&&\\
\textbf{Крылов~А.\,С.} см.~Павельева~Е.\,А.&&\\
\hangindent=23pt\noindent\textbf{Крылов~В.\,А.} Моделирование и классификация многоканальных дистанционных\linebreak
\vspace*{-12pt}\\
\hspace*{23pt}изображений с использованием копул$\dotfill$&4&34\\
\hangindent=23pt\noindent\textbf{Крючин~О.\,В.} Разработка параллельных эвристических алгоритмов подбора 
весовых\linebreak
\vspace*{-12pt}\\
\hspace*{23pt}коэффициентов искусственной нейтронной сети$\dotfill$&2&53\\
\hangindent=23pt\noindent\textbf{Кудрявцев~А.\,А., Шоргин~С.\,Я.} Байесовские модели массового обслуживания и 
надеж-\linebreak
\vspace*{-12pt}\\
\hspace*{23pt}ности: характеристики среднего числа заявок в системе $M\vert M \vert 1\vert 
\infty$$\dotfill$&3&16\\
\hangindent=23pt\noindent\textbf{Кузнецов~А.\,А.} Связь между временными и структурно-топологическими 
характери-\linebreak
\vspace*{-12pt}\\
\hspace*{23pt}стиками диаграмм ритма сердца здоровых людей$\dotfill$&4&39\\
\textbf{Кузнецов~И.\,П.} см.~Козеренко~Е.\,Б.&&\\
\textbf{Ле~Пезан~Д.} см.~Бунтман~Н.\,В.&&\\
\hangindent=23pt\noindent\textbf{Лукьяненко~А.\,С., Морозов~Е.\,В., Гуртов~А.\,В.} Анализ сетевого протокола с общей 
функ-\linebreak
\vspace*{-12pt}\\
\hspace*{23pt}цией расширения окна передачи сообщения при конфликтах$\dotfill$&2&46\\
\hangindent=23pt\noindent\textbf{Лямин~О.\,О.} О предельном поведении мощностей критериев в случае обобщенного\linebreak
\vspace*{-12pt}\\
\hspace*{23pt}распределения Лапласа$\dotfill$&3&47\\
\hangindent=23pt\noindent\textbf{Маркин~А.\,В., Шестаков~О.\,В.} Асимптотики оценки риска при пороговой 
обработке\linebreak
\vspace*{-12pt}\\
\hspace*{23pt}вейвлет-вейглет коэффициентов в задаче томографии$\dotfill$&2&36\\
\hangindent=23pt\noindent\textbf{Матвеева~С.\,С., Захарова~Т.\,В.} Сети массового обслуживания с наименьшей 
длиной\linebreak
\vspace*{-12pt}\\
\hspace*{23pt}очереди$\dotfill$&3&22\\
\hangindent=23pt\noindent\textbf{Матюшенко~С.\,И.} Стационарные характеристики двухканальной системы 
обслужива-\linebreak
\vspace*{-12pt}\\
\hspace*{23pt}ния с переупорядочиванием заявок и распределениями фазового типа$\dotfill$&4&68\\
\textbf{Минель~Ж.-Л.} см.~Бунтман~Н.\,В.&&\\
\textbf{Морозов~Е.\,В.} см.~Бородина~А.\,В.&&\\
\textbf{Морозов~Е.\,В.} см.~Лукьяненко~А.\,С.&&\\
\textbf{Ососков~М.\,В.} см.~Каратеев~С.\,Л.&&\\
\hangindent=23pt\noindent\textbf{Павельева~Е.\,А., Крылов~А.\,С.} Поиск и анализ ключевых точек радужной 
оболочки\linebreak
\vspace*{-12pt}\\
\hspace*{23pt}глаза методом преобразования Эрмита$\dotfill$&1&79\\
\textbf{Печинкин~А.\,В.} см.~Френкель~С.\,Л.,&&\\
\hangindent=23pt\noindent\textbf{Протасов~В.\,И.} Составление субъективного портрета с использованием 
эволюционно-\linebreak
\vspace*{-12pt}\\
\hspace*{23pt}го морфинга и квалиметрия метода$\dotfill$&1&83\\
\hangindent=23pt\noindent\textbf{Рудаков~К.\,В., Торшин~И.\,Ю.} Вопросы разрешимости задачи распознавания 
вторичной\linebreak
\vspace*{-12pt}\\
\hspace*{23pt}структуры белка$\dotfill$&2&25\\
\textbf{Сатин~Я.\,А.} см.~Зейфман~А.\,И.&&\\
\hangindent=23pt\noindent\textbf{Сейфуль-Мулюков~Р.\,Б.} Нефть как носитель информации о своем 
происхождении,\linebreak
\vspace*{-12pt}\\
\hspace*{23pt}структуре и эволюции$\dotfill$&1&41\\
\end{tabular}
}

{\tabcolsep=3pt
\begin{tabular}{p{388pt}rr}
&\textbf{Выпуск} & \textbf{Стр.}\\[6pt]
\textbf{Семендяев~Н.\,Н.} см.~Синицын~И.\,Н.&&\\
\textbf{Серебряков~В.\,А.} см.~Захаров~А.\,А.&&\\
\textbf{Синицын~В.\,И.} см.~Синицын~И.\,Н.&&\\
\hangindent=23pt\noindent\textbf{Синицын~И.\,Н., Синицын~В.\,И., Корепанов~Э.\, Р., Белоусов~В.\,В., 
Семендяев~Н.\,Н.} Оперативное построение информационных моделей движения полюса 
Земли\linebreak
\vspace*{-12pt}\\
\hspace*{23pt}методами линейных и линеаризованных фильтров$\dotfill$&1&2\\
\textbf{Сипина~А.\,В.} см.~Бенинг~В.\,Е.&&\\
\hangindent=23pt\noindent\textbf{Соколов~И.\,А.} О работах заслуженного деятеля науки Российской Федерации 
И.\,Н.~Синицына в области информационных технологий и автоматизации (к 70-летию\linebreak
\vspace*{-12pt}\\
\hspace*{23pt}со дня рождения)$\dotfill$&3&84\\
\textbf{Соколов~И.\,А.} см.~Илюшин~Г.\,Я.&&\\
\hangindent=23pt\noindent\textbf{Соколов~И.\,А., Королев~В.\,Ю.} Предисловие$\dotfill$&2&2\\
\textbf{Солдатов~С.\,А.} см.~Колесников~А.\,В.&&\\
\hangindent=23pt\noindent\textbf{Степанов~С.\,Ю.} Использование координатного метода фрагментации 
коммутаторной\linebreak
\vspace*{-12pt}\\
\hspace*{23pt}нейронной сети для сокращения трафика$\dotfill$&2&57\\
\textbf{Тимонина~Е.\,Е.} см.~Грушо~А.\,А.&&\\
\textbf{Торшин~И.\,Ю.} см.~Рудаков~К.\,В.&&\\
\textbf{Ульянов~В.\,В.} см.~Кавагучи~Ю.&&\\
\textbf{Фазекаш~И.} см.~Чупрунов~А.\,Н.&&\\
\textbf{Френкель~С.\,Л.} см.~Баранов~С.\,И.&&\\
\hangindent=23pt\noindent\textbf{Френкель~С.\,Л., Печинкин~А.\,В.} Оценка времени самовосстановления в 
цифровых\linebreak
\vspace*{-12pt}\\
\hspace*{23pt}системах после сбоев, вызываемых переходными помехами$\dotfill$&3&2\\
\textbf{Фуджикоши~Я.} см.~Кавагучи~Ю.&&\\
\hangindent=23pt\noindent\textbf{Цискаридзе~А.\,К.} Математическая модель и метод восстановления позы человека 
по\linebreak
\vspace*{-12pt}\\
\hspace*{23pt}стереопаре силуэтных изображений$\dotfill$&4&27\\
\hangindent=23pt\noindent\textbf{Чупраков~К.\,Г.} К вопросу о размещении коллективных средств отображения в 
ситуа-\linebreak
\vspace*{-12pt}\\
\hspace*{23pt}ционном зале с заданными параметрами$\dotfill$&4&89\\
\textbf{Чупраков~К.\,Г.} см.~Зацаринный~А.\,А.&&\\
\hangindent=23pt\noindent\textbf{Чупрунов~А.\,Н., Фазекаш~И.} Законы повторного логарифма для числа 
безошибочных\linebreak
\vspace*{-12pt}\\
\hspace*{23pt}блоков при помехоустойчивом кодировании$\dotfill$&3&42\\
\textbf{Шевцова~И.\,Г.} см.~Григорьева~М.\,Е.&&\\
\hangindent=23pt\noindent\textbf{Шестаков~О.\,В.} Аппроксимация распределения оценки риска пороговой 
обработки вейвлет-коэффициентов нормальным распределением при использовании 
выбо-\linebreak
\vspace*{-12pt}\\
\hspace*{23pt}рочной дисперсии$\dotfill$&4&73\\
\textbf{Шестаков~О.\,В.} см.~Маркин~А.\,В.&&\\
\textbf{Шоргин~С.\,Я.} см.~Зейфман~А.\,И.&&\\
\textbf{Шоргин~С.\,Я.} см.~Кудрявцев~А.\,А.&&\\
\end{tabular}
}

%\thispagestyle{myheadings}
\def\leftfootline{\small{\textbf{\thepage}
\hfill ИНФОРМАТИКА И ЕЁ ПРИМЕНЕНИЯ\ \ \ том~4\ \ \ выпуск~4\ \ \ 2010}
}%
 \def\rightfootline{\small{ИНФОРМАТИКА И ЕЁ ПРИМЕНЕНИЯ\ \ \ том~4\ \ \ выпуск~4\ \ \ 2010
 \hfill \textbf{\thepage}}}
 \label{end\stat}


%Том 10 Выпуск 1-4 Год 2016

\def\stat{cont-e}
{%\hrule\par
%\vskip 7pt % 7pt
\raggedleft\Large \bf%\baselineskip=3.2ex
2\,0\,1\,6\ \ A\,U\,T\,H\,O\,R\ \ I\,N\,D\,E\,X \vskip 17pt
 \hrule
 \par
\vskip 21pt plus 6pt minus 3pt }

\label{st\stat}

\def\tit{\ }

\def\aut{\ }
\def\auf{\ }

\def\leftkol{\ } %2016 AUTHOR INDEX} % ENGLISH ABSTRACTS}

\def\rightkol{\ } %2016 AUTHOR INDEX} %ENGLISH ABSTRACTS}

\titele{\tit}{\aut}{\auf}{\leftkol}{\rightkol}

\def\leftfootline{\small{\textbf{\thepage}
\hfill INFORMATIKA I EE PRIMENENIYA~--- INFORMATICS AND APPLICATIONS\ \ \ 2016\
\ \ volume~10\ \ \ issue\ 4}
}%
 \def\rightfootline{\small{INFORMATIKA I EE PRIMENENIYA~--- INFORMATICS AND APPLICATIONS\ \ \ 2016\ \ \ volume~10\ \ \ issue\ 4
\hfill \textbf{\thepage}}}

\vspace*{-12pt}
\vspace*{-18pt}

{\tabcolsep=2.8pt
\begin{tabular}{p{382pt}cc}
&\textbf{Issue} & \textbf{Page}\\[6pt]
\Avtors{Agalarov~M.\,Ya.} see~Agalarov~Ya.\,M.&&\\
\Avtors{Agalarov~Ya.\,M., Agalarov~M.\,Ya., and
Shorgin~V.\,S.} About the optimal threshold of queue\linebreak
\\[-12pt]
\hspace*{23pt}length in a~particular problem of profit maximization
in the $M/G/1$ queuing system&2&70--79\\
\Avtors{Alexeyevsky~D.\,A.} BioNLP ontology extraction from 
a~restricted language corpus with\linebreak
\\[-12pt]
\hspace*{23pt}context-free grammars&1&119--128\\
\Avtors{Andreev~S.\,D.} see~Gaidamaka~Yu.\,V.&&\\
\Avtors{Andreev~S.\,D.} see~Ometov~A.\,Ya.&&\\
\Avtors{Arkhipov~O.\,P., Arkhipov~P.\,O., and Sidorkin~I.\,I.} The
option to create a~local coordinate\linebreak
\\[-12pt]
\hspace*{23pt}system for synchronization of selected images&3&91--97\\
\Avtors{Arkhipov~P.\,O.} see~Arkhipov~O.\,P.&&\\
\Avtors{Belousov~V.\,V.} see~Shnurkov~P.\,V.&&\\
\Avtors{Belousov~V.\,V.} see~Shnurkov~P.\,V.&&\\
\Avtors{Bening~V.\,E.} Calculation of~the~asymptotic deficiency
of~some statistical procedures based\linebreak
\\[-12pt]
\hspace*{23pt}on~samples with~random sizes&4&34--45\\
\Avtors{Borisov~A.\,V., Bosov~A.\,V., and Miller~G.\,B.} Modeling and
monitoring of VoIP connection&2&\hphantom{1}2--13\\
\Avtors{Bosov~A.\,V.} see~Borisov~A.\,V.&&\\
\Avtors{Briukhov~D.\,O.} see~Stupnikov~S.\,A.&&\\
\Avtors{Callaos~N.\,K.\ and Seyful-Mulyukov~R.\,B.} Complexity and
its information content&1&129--139\\
\Avtors{Chertok~A.\,V., Kadaner~A.\,I., Khazeeva~G.\,T., and
Sokolov~I.\,A.} Regime switching detection\linebreak
\\[-12pt]
\hspace*{23pt}for~the~Levy driven
Ornstein--Uhlenbeck process using CUSUM methods&4&46--56\\
\Avtors{Chichagov~V.\,V.} Asymptotic expansions of mean absolute
error of uniformly minimum variance unbiased and maximum likelihood
estimators on the one-parameter exponential\linebreak
\\[-12pt]
\hspace*{23pt}family model of lattice distributions&3&66--76\\
\Avtors{Danishevsky~V.\,I.} see~Kolesnikov A.\,V.&&\\
\Avtors{Fazliev~A.\,Z.} see~Kalinichenko~L.\,A.&&\\
\Avtors{Fedoseev~A.\,A.} What is behind the concept of ``knowledge in
small packages''&3&105--110\\
\Avtors{Gaidamaka~Yu.\,V., Andreev~S.\,D., Sopin~E.\,S.,
Samouylov~K.\,E., and Shorgin~S.\,Ya.} Interference analysis
of~the~device-to-device communications model with~regard to~a~signal\linebreak
\\[-12pt]
\hspace*{23pt}propagation environment&4&\hphantom{1}2--10\\
\Avtors{Gasilov~A.\,V.} see~Yakovlev~O.\,A.&&\\
\Avtors{Goncharov~A.\,V.\ and Strijov~V.\,V.} Metric time series
classification using weighted dynamic\linebreak
\\[-12pt]
\hspace*{23pt}warping relative to centroids of classes&2&36--47\\
\Avtors{Gordov~E.\,P.} see~Kalinichenko~L.\,A.&&\\
\Avtors{Gorshenin~A.\,K.} Concept of online service for stochastic
modeling of real processes&1&72--81\\
\Avtors{Gorshenin~A.\,K.} see~Shnurkov~P.\,V.&&\\
\Avtors{Gorshenin~A.\,K.} see~Shnurkov~P.\,V.&&\\
\Avtors{Grusho~A.\,A., Grusho~N.\,A., Zabezhailo~M.\,I., and
Timonina~E.\,E.} Integration of statistical and\linebreak
\\[-12pt]
\hspace*{23pt}deterministic methods for
analysis of information security&3&2--8\\
\Avtors{Grusho~A.\,A., Zabezhailo~M.\,I., and Zatsarinny~A.\,A.} On
the advanced procedure to reduce\linebreak
\\[-12pt]
\hspace*{23pt}calculation of Galois closures&4&\hphantom{1}96--104\\
\Avtors{Grusho~N.\,A.} see~Grusho~A.\,A.&&\\
\Avtors{Havanskov~V.\,A.} see~Minin~V.\,A.&&\\
\Avtors{Inkova~O.\,Yu.} see~Zatsman~I.\,M.&&\\
\Avtors{Isachenko~R.\,V.\ and Strijov~V.\,V.} Metric learning in
multiclass time series classification\linebreak
\\[-12pt]
\hspace*{23pt}problem&2&48--57\\
\end{tabular}
}
\pagebreak

\def\leftfootline{\small{\textbf{\thepage}
\hfill INFORMATIKA I EE PRIMENENIYA~--- INFORMATICS AND APPLICATIONS\ \ \ 2016\
\ \ volume~10\ \ \ issue\ 4}
}%
 \def\rightfootline{\small{INFORMATIKA I EE PRIMENENIYA~---
INFORMATICS AND APPLICATIONS\ \ \ 2016\ \ \ volume~10\ \ \ issue\ 4
\hfill \textbf{\thepage}}}

\def\leftkol{2016 AUTHOR INDEX} % ENGLISH ABSTRACTS}

\def\rightkol{2016 AUTHOR INDEX} %ENGLISH ABSTRACTS}


{\tabcolsep=2.83pt
\begin{tabular}{p{382pt}cc}
&\textbf{Issue} & \textbf{Page}\\[6pt]
\Avtors{Kadaner~A.\,I.} see~Chertok~A.\,V.&&\\[.255pt]
\Avtors{Kalinichenko~L.\,A., Volnova~A.\,A., Gordov~E.\,P.,
Kiselyova~N.\,N., Kovaleva~D.\,A., Malkov~O.\,Yu., Okladnikov~I.\,G.,
Podkolodnyy~N.\,L., Pozanenko~A.\,S., Ponomareva~N.\,V.,
Stupnikov~S.\,A.,} \textbf{and Fazliev~A.\,Z.} Data access challenges for data
intensive\linebreak
\\[-12pt]
\hspace*{23pt}research in Russia&1& 2--22\\[.255pt]
\Avtors{Karasikov~M.\,E.\ and Strijov~V.\,V.} Feature-based
time-series classification&4&121--131\\[.255pt]
\Avtors{Khazeeva~G.\,T.} see~Chertok~A.\,V.&&\\[.255pt]
\Avtors{Khokhlov~Yu.\,S.} Multivariate fractional Levy motion and its
applications&2&\hphantom{1}98--106\\[.255pt]
\Avtors{Kirikov~I.\,A., Kolesnikov~A.\,V., Listopad~S.\,V., and
Rumovskaya~S.\,B.} Fine-grained hybrid\linebreak
\\[-12pt]
\hspace*{23pt}intelligent systems. Part 2:
Bidirectional hybridization&1&\hphantom{1}96--105\\[.255pt]
\Avtors{Kirikov~I.\,A., Kolesnikov~A.\,V., Listopad~S.\,V., and
Rumovskaya~S.\,B.} ``Virtual council''~---\linebreak
\\[-12pt]
\hspace*{23pt}source environment
supporting complex diagnostic decision making&3&81--90\\[.255pt]
\Avtors{Kiselyova~N.\,N.} see~Kalinichenko~L.\,A.&&\\[.255pt]
\Avtors{Kolesnikov A.\,V., Listopad~S.\,V., Rumovskaya~S.\,B., and
Danishevsky~V.\,I.} Informal axiomatic\linebreak
\\[-12pt]
\hspace*{23pt}theory of~the~role visual models&4&114--120\\[.255pt]
\Avtors{Kolesnikov~A.\,V.} see~Kirikov~I.\,A.&&\\[.255pt]
\Avtors{Kolesnikov~A.\,V.} see~Kirikov~I.\,A.&&\\[.255pt]
\Avtors{Kolin~K.\,K.} Humanitarian aspects of information
security&3&111--121\\[.255pt]
\Avtors{Konovalov~M.\,G.\ and Razumchik~R.\,V.} Dispatching
to~two parallel nonobservable queues using\linebreak
\\[-12pt]
\hspace*{23pt}only static
information&4&57--67\\[.255pt]
\Avtors{Korchagin~A.\,Yu.} see~Korolev~V.\,Yu.&&\\[.255pt]
\Avtors{Korchagin~A.\,Yu.} see~Korolev~V.\,Yu.&&\\[.255pt]
\Avtors{Korepanov~E.\,R.} see~Sinitsyn~I.\,N.&&\\[.255pt]
\Avtors{Korepanov~E.\,R.} see~Sinitsyn~I.\,N.&&\\[.255pt]
\Avtors{Korolev~V.\,Yu., Korchagin~A.\,Yu., and Zeifman~A.\,I.} The
Poisson theorem for Bernoulli trials\linebreak
\\[-12pt]
\hspace*{23pt}with~a~random probability
of~success and~a~discrete analog of~the~Weibull distribution&4&11--20\\[.255pt]
\Avtors{Korolev~V.\,Yu., Zeifman~A.\,I., and Korchagin~A.\,Yu.}
Asymmetric Linnik distributions as~limit\linebreak
\\[-12pt]
\hspace*{23pt}laws for~random sums
of~independent random variables with~finite variances&4&21--33\\[.255pt]
\Avtors{Koucheryavy~E.\,A.} see~Ometov~A.\,Ya.&&\\[.255pt]
\Avtors{Kovaleva~D.\,A.} see~Kalinichenko~L.\,A.&&\\[.255pt]
\Avtors{Kovalyov~S.\,P.} Metaprogramming to increase
manufacturability of large-scale software-\linebreak
\\[-12pt]
\hspace*{23pt}intensive systems&1&56--66\\[.255pt]
\Avtors{Krivenko~M.\,P.} Significance tests of feature selection for
classification&3&32--40\\[.255pt]
\Avtors{Kruzhkov~M.\,G.} see~Zalizniak~Anna~A.&&\\[.255pt]
\Avtors{Kruzhkov~M.\,G.} see~Zatsman~I.\,M.&&\\[.255pt]
\Avtors{Kudryavtsev~A.\,A.} Bayesian queueing and reliability models:
\textit{A~priori} distributions with\linebreak
\\[-12pt]
\hspace*{23pt}compact support&1&67--71\\[.255pt]
\Avtors{Kudryavtsev~A.\,A.} Characteristics dependent on the balance
coefficient in Bayesian models\linebreak
\\[-12pt]
\hspace*{23pt}with compact support of \textit{a priori}
distributions&3&77--80\\[.255pt]
\Avtors{Kudryavtsev~A.\,A.\ and Palionnaia~S.\,I.} Bayesian recurrent
model of reliability growth:\linebreak
\\[-12pt]
\hspace*{23pt}Parabolic distribution of parameters&2&80--83\\[.255pt]
\Avtors{Kudryavtsev~A.\,A.\ and Titova~A.\,I.} Bayesian queuing
and~reliability models: Degenerate-\linebreak
\\[-12pt]
\hspace*{23pt}Weibull case&4&68--71\\[.255pt]
\Avtors{Leontyev~N.\,D.\ and Ushakov~V.\,G.} Analysis of a queueing
system with autoregressive arrivals\linebreak
\\[-12pt]
\hspace*{23pt}and nonpreemptive priority&3&15--22\\[.255pt]
\Avtors{Listopad~S.\,V.} see~Kirikov~I.\,A.&&\\[.255pt]
\Avtors{Listopad~S.\,V.} see~Kirikov~I.\,A.&&\\[.255pt]
\Avtors{Listopad~S.\,V.} see~Kolesnikov A.\,V.&&\\[.255pt]
\Avtors{Malkov~O.\,Yu.} see~Kalinichenko~L.\,A.&&\\[.255pt]
\Avtors{Markov~A.\,S., Monakhov~M.\,M., and
Ulyanov~V.\,V.} Generalized Cornish--Fisher expansions\linebreak
\\[-12pt]
\hspace*{23pt}for distributions of statistics based on samples
of random size&2&84--91\\[.255pt]
\Avtors{Melnikov~A.\,K.\ and Ronzhin~A.\,F.} Generalized statistical
method of~text analysis based\linebreak
\\[-12pt]
\hspace*{23pt}on~calculation of~probability distributions
of~statistical values&4&89--95\\
\end{tabular}
}
\pagebreak

\def\leftfootline{\small{\textbf{\thepage}
\hfill INFORMATIKA I EE PRIMENENIYA~--- INFORMATICS AND APPLICATIONS\ \ \ 2016\
\ \ volume~10\ \ \ issue\ 4}
}%
 \def\rightfootline{\small{INFORMATIKA I EE PRIMENENIYA~---
INFORMATICS AND APPLICATIONS\ \ \ 2016\ \ \ volume~10\ \ \ issue\ 4
\hfill \textbf{\thepage}}}

\def\leftkol{2016 AUTHOR INDEX} % ENGLISH ABSTRACTS}

\def\rightkol{2016 AUTHOR INDEX} %ENGLISH ABSTRACTS}


{\tabcolsep=3pt
\begin{tabular}{p{381pt}cc}
&\textbf{Issue} & \textbf{Page}\\[6pt]
\Avtors{Meykhanadzhyan~L.\,A.} Stationary characteristics of the finite
capacity queueing system with\linebreak
\\[-12pt]
\hspace*{23pt}inverse service order and generalized
probabilistic priority&2&123--131\\[.23pt]
\Avtors{Miller~G.\,B.} see~Borisov~A.\,V.&&\\[.23pt]
\Avtors{Minin~V.\,A., Zatsman~I.\,M., Havanskov~V.\,A., and
Shubnikov~S.\,K.} Intensity of citation of scientific publications in
inventions on information and computer technologies patented\linebreak
\\[-12pt]
\hspace*{23pt}in Russia by domestic and foreign applicants&2&107--122\\[.23pt]
\Avtors{Monakhov~M.\,M.} see~Markov~A.\,S.&&\\[.23pt]
\Avtors{Naumov~V.\,A.\ and Samouylov~K.\,E.} On relationship
between queuing systems with resources\linebreak
\\[-12pt]
\hspace*{23pt}and Erlang networks&3&\hphantom{1}9--14\\[.23pt]
\Avtors{Okladnikov~I.\,G.} see~Kalinichenko~L.\,A.&&\\[.23pt]
\Avtors{Ometov~A.\,Ya., Andreev~S.\,D., Turlikov~A.\,M., and
Koucheryavy~E.\,A.} Performance analysis of\linebreak
\\[-12pt]
\hspace*{23pt}a wireless data
aggregation system with contention for contemporary sensor
networks&3&23--31\\[.23pt]
\Avtors{Palionnaia~S.\,I.} see~Kudryavtsev~A.\,A.&&\\[.23pt]
\Avtors{Podkolodnyy~N.\,L.} see~Kalinichenko~L.\,A.&&\\[.23pt]
\Avtors{Ponomareva~N.\,V.} see~Kalinichenko~L.\,A.&&\\[.23pt]
\Avtors{Popkova~N.\,A.} see~Zatsman~I.\,M.&&\\[.23pt]
\Avtors{Pozanenko~A.\,S.} see~Kalinichenko~L.\,A.&&\\[.23pt]
\Avtors{Razumchik~R.\,V.} see~Konovalov~M.\,G.&&\\[.23pt]
\Avtors{Ronzhin~A.\,F.} see~Melnikov~A.\,K.&&\\[.23pt]
\Avtors{Rumovskaya~S.\,B.} see~Kirikov~I.\,A.&&\\[.23pt]
\Avtors{Rumovskaya~S.\,B.} see~Kirikov~I.\,A.&&\\[.23pt]
\Avtors{Rumovskaya~S.\,B.} see~Kolesnikov A.\,V.&&\\[.23pt]
\Avtors{Samouylov~K.\,E.} see~Gaidamaka~Yu.\,V.&&\\[.23pt]
\Avtors{Samouylov~K.\,E.} see~Naumov~V.\,A.&&\\[.23pt]
\Avtors{Serebryanskii~S.\,M.} see~Tyrsin~A.\,N.&&\\[.23pt]
\Avtors{Seyful-Mulyukov~R.\,B.} see~Callaos~N.\,K.&&\\[.23pt]
\Avtors{Shestakov~O.\,V.} Statistical properties of the denoising method
based on the stabilized hard\linebreak
\\[-12pt]
\hspace*{23pt}thresholding&2&65--69\\[.23pt]
\Avtors{Shestakov~O.\,V.} The strong law of large numbers for the risk
estimate in the problem of\linebreak
\\[-12pt]
\hspace*{23pt}tomographic image reconstruction from
projections with a correlated noise&3&41--45\\[.23pt]
\Avtors{Shestakov~O.\,V.} see~Zakharova~T.\,V.&&\\[.23pt]
\Avtors{Shnurkov~P.\,V., Gorshenin~A.\,K., and Belousov~V.\,V.}
Analytical solution of~the~optimal control\linebreak
\\[-12pt]
\hspace*{23pt}task of~a~semi-Markov
process with~finite set of~states&4&72--88\\[.23pt]
\Avtors{Shnurkov~P.\,V., Zasypko~V.\,V., Belousov~V.\,V., and
Gorshenin~A.\,K.} Development of the algorithm of numerical solution
of the optimal investment control problem\linebreak
\\[-12pt]
\hspace*{23pt}in the closed dynamical model of three-sector economy&1&82--95\\[.23pt]
\Avtors{Shorgin~S.\,Ya.} see~Gaidamaka~Yu.\,V.&&\\[.23pt]
\Avtors{Shorgin~V.\,S.} see~Agalarov~Ya.\,M.&&\\[.23pt]
\Avtors{Shubnikov~S.\,K.} see~Minin~V.\,A.&&\\[.23pt]
\Avtors{Sidorkin~I.\,I.} see~Arkhipov~O.\,P.&&\\[.23pt]
\Avtors{Sinitsyn~I.\,N.} Analytical modeling of processes in stochastic
systems with complex fractional\linebreak
\\[-12pt]
\hspace*{23pt}order Bessel nonlinearities&3&55--65\\[.23pt]
\Avtors{Sinitsyn~I.\,N.} Orthogonal supoptimal filters for nonlinear
stochastic systems on manifolds&1&34--44\\[.23pt]
\Avtors{Sinitsyn~I.\,N.\ and Korepanov~E.\,R.} Normal Pugachev
conditionally-optimal filters and extra-\linebreak
\\[-12pt]
\hspace*{23pt}polators for state linear stochastic systems&2&14--23\\[.23pt]
\Avtors{Sinitsyn~I.\,N.\ and Sinitsyn~V.\,I.} Analytical modeling of
distributions in stochastic systems on\linebreak
\\[-12pt]
\hspace*{23pt}manifolds based on ellipsoidal approximation&1&45--55\\[.23pt]
\Avtors{Sinitsyn~I.\,N., Sinitsyn~V.\,I., and
Korepanov~E.\,R.} Ellipsoidal suboptimal filters for nonlinear\linebreak
\\[-12pt]
\hspace*{23pt}stochastic systems on manifolds&2&24--35\\[.23pt]
\Avtors{Sinitsyn~V.\,I.} see~Sinitsyn~I.\,N.&&\\[.23pt]
\Avtors{Sinitsyn~V.\,I.} see~Sinitsyn~I.\,N.&&\\[.23pt]
\Avtors{Skvortsov~N.\,A.} see~Stupnikov~S.\,A.&&\\[.23pt]
\Avtors{Sokolov~I.\,A.} see~Chertok~A.\,V.&&\\
\end{tabular}
}
\pagebreak

\def\leftfootline{\small{\textbf{\thepage}
\hfill INFORMATIKA I EE PRIMENENIYA~--- INFORMATICS AND APPLICATIONS\ \ \ 2016\
\ \ volume~10\ \ \ issue\ 4}
}%
 \def\rightfootline{\small{INFORMATIKA I EE PRIMENENIYA~---
INFORMATICS AND APPLICATIONS\ \ \ 2016\ \ \ volume~10\ \ \ issue\ 4
\hfill \textbf{\thepage}}}

\def\leftkol{2016 AUTHOR INDEX} % ENGLISH ABSTRACTS}

\def\rightkol{2016 AUTHOR INDEX} %ENGLISH ABSTRACTS}


{\tabcolsep=3pt
\begin{tabular}{p{382pt}cc}
&\textbf{Issue} & \textbf{Page}\\[6pt]
\Avtors{Sopin~E.\,S.} see~Gaidamaka~Yu.\,V.&&\\
\Avtors{Strijov~V.\,V.} see~Goncharov~A.\,V.&&\\
\Avtors{Strijov~V.\,V.} see~Isachenko~R.\,V.&&\\
\Avtors{Strijov~V.\,V.} see~Karasikov~M.\,E.&&\\
\Avtors{Stupnikov~S.\,A., Briukhov~D.\,O., and Skvortsov~N.\,A.}
Co-lending systemic risk analysis over\linebreak
\\[-12pt]
\hspace*{23pt}heterogeneous data collections&1&23--33\\
\Avtors{Stupnikov~S.\,A.} see~Kalinichenko~L.\,A.&&\\
\Avtors{Suchkov~A.\,P.} see~Zatsarinny~A.\,A.&&\\
\Avtors{Timonina~E.\,E.} see~Grusho~A.\,A.&&\\
\Avtors{Titova~A.\,I.} see~Kudryavtsev~A.\,A.&&\\
\Avtors{Turlikov~A.\,M.} see~Ometov~A.\,Ya.&&\\
\Avtors{Tyrsin~A.\,N.\ and Serebryanskii~S.\,M.} Recognition of
dependences on the basis of inverse\linebreak
\\[-12pt]
\hspace*{23pt}mapping&2&58--64\\
\Avtors{Ulyanov~V.\,V.} see~Markov~A.\,S.&&\\
\Avtors{Ushakov~V.\,G.} Queueing system with working vacations and
hyperexponential input stream&2&92--97\\
\Avtors{Ushakov~V.\,G.} see~Leontyev~N.\,D.&&\\
\Avtors{Volnova~A.\,A.} see~Kalinichenko~L.\,A.&&\\
\Avtors{Yakovlev~O.\,A.\ and Gasilov~A.\,V.} Speeded-up stereo
matching using geodesic support weights&3&\hphantom{1}98--104\\
\Avtors{Zabezhailo~M.\,I.} see~Grusho~A.\,A.&&\\
\Avtors{Zabezhailo~M.\,I.} see~Grusho~A.\,A.&&\\
\Avtors{Zakharova~T.\,V.\ and Shestakov~O.\,V.} Precision analysis of
wavelet processing of aerodynamic\linebreak
\\[-12pt]
\hspace*{23pt}flow patterns&3&46--54\\
\Avtors{Zalizniak~Anna~A.\ and Kruzhkov~M.\,G.} Database
of~Russian impersonal verbal constructions&4&132--141\\
\Avtors{Zasypko~V.\,V.} see~Shnurkov~P.\,V.&&\\
\Avtors{Zatsarinny~A.\,A.\ and Suchkov~A.\,P.} Systems engineering
approaches to~the~establishment of\linebreak
\\[-12pt]
\hspace*{23pt}a~system for~decision support based
on~situational analysis&4&105--113\\
\Avtors{Zatsarinny~A.\,A.} see~Grusho~A.\,A.&&\\
\Avtors{Zatsman~I.\,M., Inkova~O.\,Yu., Kruzhkov~M.\,G., and
Popkova~N.\,A.} Representation of cross-\linebreak
\\[-12pt]
\hspace*{23pt}lingual knowledge about
connectors in supracorpora databases&1&106--118\\
\Avtors{Zatsman~I.\,M.} see~Minin~V.\,A.&&\\
\Avtors{Zeifman~A.\,I.} see~Korolev~V.\,Yu.&&\\
\Avtors{Zeifman~A.\,I.} see~Korolev~V.\,Yu.&&\\
\end{tabular}
}

%\thispagestyle{myheadings}
\def\leftfootline{\small{\textbf{\thepage}
\hfill INFORMATIKA I EE PRIMENENIYA~--- INFORMATICS AND APPLICATIONS\ \ \ 2016\
\ \ volume~10\ \ \ issue\ 4}
}%
 \def\rightfootline{\small{INFORMATIKA I EE PRIMENENIYA~---
INFORMATICS AND APPLICATIONS\ \ \ 2016\ \ \ volume~10\ \ \ issue\ 4
\hfill \textbf{\thepage}}}

 \label{end\stat}

\newpage

%\include{cover3}

%\def\stat{cont}
{%\hrule\par
%\vskip 7pt % 7pt
\raggedleft\Large \bf%\baselineskip=3.2ex
А\,В\,Т\,О\,Р\,С\,К\,И\,Й\ \ У\,К\,А\,З\,А\,Т\,Е\,Л\,Ь\ \ З\,А\ \ 2\,0\,0\,7 г. \vskip 17pt
    \hrule
    \par
\vskip 21pt plus 6pt minus 3pt }

\label{st\stat}

\def\tit{\ }

\def\aut{\ }
\def\auf{\ }

\def\leftkol{\ } % ENGLISH ABSTRACTS}

\def\rightkol{\ } %ENGLISH ABSTRACTS}

\titele{\tit}{\aut}{\auf}{\leftkol}{\rightkol}


\contentsline {chapter}{\ }{Выпуск \quad Стр.} 
\contentsline {section}{\textbf{Батракова Д.\,А., Королев В.\,Ю., Шоргин С.\,Я.}\ \ Новый метод вероятностно-ста\-ти\-сти\-че\-ско\-го анализа информационных потоков в\nobreakspace {}телекоммуникационных сетях}{\qquad 1 \qquad 40} 
\contentsline {section}{\textbf{Борисов А.\,В.}\ \ Байесовское оценивание в системах наблюдения с\nobreakspace {}марковскими скачкообразными процессами: игровой подход}{\qquad 2 \qquad 65}
\contentsline {section}{\textbf{Босов А.\,В., Иванов А.\,В.}\ \ Программная инфраструктура информационного Web-пор\-тала}{\qquad 2 \qquad 50}
\contentsline {section}{\textbf{Захаров В.\,Н., Калиниченко Л.\,А., Соколов И.\,А., Ступников С.\,А.}\ \ Конструирование канонических информационных моделей для интегрированных информационных систем}{\qquad 2 \qquad 15}
\contentsline {section}{\textbf{Захаров В.\,Н., Козмидиади В.\,А.}\ \ Средства обеспечения отказоустойчивости при\-ло\-жений}{\qquad 1 \qquad 14} 
\contentsline {section}{\textbf{Иванов А.\,В.}\ \ см. Босов А.\,В.\hfill\hfill\hfill\hfill\hfill\hfill\hfill\hfill\hfill\hfill\hfill\hfill\hfill\hfill\hfill\hfill\hfill\hfill\hfill\hfill\hfill\hfill\hfill\hfill\hfill\hfill\hfill\hfill\hfill\hfill\hfill\hfill\hfill\hfill\hfill}{\ }
\contentsline {section}{\textbf{Ильин В.\,Д., Соколов И.\,А.}\ \ Символьная модель системы знаний информатики в\nobreakspace {}че\-ло\-ве\-ко-автоматной среде}{\qquad 1 \qquad 66} 
\contentsline {section}{\textbf{Калиниченко Л.\,А.}\ \ см. Захаров В.\,Н.\hfill\hfill\hfill\hfill\hfill\hfill\hfill\hfill\hfill\hfill\hfill\hfill\hfill\hfill\hfill\hfill\hfill\hfill\hfill\hfill\hfill\hfill\hfill\hfill\hfill\hfill\hfill\hfill\hfill\hfill\hfill\hfill\hfill\hfill\hfill}{\ }
\contentsline {section}{\textbf{Козеренко Е.\,Б.}\ \ Лингвистическое моделирование для систем машинного перевода и обработки знаний}{\qquad 1 \qquad 54} 
\contentsline {section}{\textbf{Козмидиади В.\,А.}\ \ см. Захаров В.\,Н.\hfill\hfill\hfill\hfill\hfill\hfill\hfill\hfill\hfill\hfill\hfill\hfill\hfill\hfill\hfill\hfill\hfill\hfill\hfill\hfill\hfill\hfill\hfill\hfill\hfill\hfill\hfill\hfill\hfill\hfill\hfill\hfill\hfill\hfill\hfill }{\ } 
\contentsline {section}{\textbf{Королев В.\,Ю.}\ \ см. Батракова Д.\,А.\hfill\hfill\hfill\hfill\hfill\hfill\hfill\hfill\hfill\hfill\hfill\hfill\hfill\hfill\hfill\hfill\hfill\hfill\hfill\hfill\hfill\hfill\hfill\hfill\hfill\hfill\hfill\hfill\hfill\hfill\hfill\hfill\hfill\hfill\hfill}{\ } 
\contentsline {section}{\textbf{Кудрявцев А.\,А., Шоргин С.\,Я.}\ \ Байесовский подход к\nobreakspace {}анализу систем массового обслуживания и\nobreakspace {}показателей надежности}{\qquad 2 \qquad 76}
\contentsline {section}{\textbf{Печинкин А.\,В., Соколов И.\,А., Чаплыгин В.\,В.}\ \ Многолинейная система массового обслуживания с конечным накопителем и ненадежными приборами}{\qquad 1 \qquad 27} 
\contentsline {section}{\textbf{Печинкин А.\,В., Соколов И.\,А., Чаплыгин В.\,В.}\ \ Стационарные характеристики многолинейной\nobreakspace {}системы массового обслуживания с\nobreakspace {}одновременными отказами приборов}{\qquad 2 \qquad 39}
\contentsline {section}{\textbf{Синицын И.\,Н.}\ \ Корреляционные методы построения аналитических информационных моделей флуктуаций полюса Земли по априорным данным}{\qquad 2 \qquad \hphantom{9}2}
\contentsline {section}{\textbf{Синицын И.\,Н.}\ \ Развитие теории фильтров Пугачева для оперативной обработки информации в стохастических системах}{{\qquad 1 \qquad \hphantom{9}3}} 
\contentsline {section}{\textbf{Соколов И.\,А.}\ \ см. Захаров В.\,Н.\hfill\hfill\hfill\hfill\hfill\hfill\hfill\hfill\hfill\hfill\hfill\hfill\hfill\hfill\hfill\hfill\hfill\hfill\hfill\hfill\hfill\hfill\hfill\hfill\hfill\hfill\hfill\hfill\hfill\hfill\hfill\hfill\hfill\hfill\hfill}{\ }
\contentsline {section}{\textbf{Соколов И.\,А.}\ \ см. Ильин В.\,Д.\hfill\hfill\hfill\hfill\hfill\hfill\hfill\hfill\hfill\hfill\hfill\hfill\hfill\hfill\hfill\hfill\hfill\hfill\hfill\hfill\hfill\hfill\hfill\hfill\hfill\hfill\hfill\hfill\hfill\hfill\hfill\hfill\hfill\hfill\hfill}{\ } 
\contentsline {section}{\textbf{Соколов И.\,А.}\ \ см. Печинкин А.\,В.\hfill\hfill\hfill\hfill\hfill\hfill\hfill\hfill\hfill\hfill\hfill\hfill\hfill\hfill\hfill\hfill\hfill\hfill\hfill\hfill\hfill\hfill\hfill\hfill\hfill\hfill\hfill\hfill\hfill\hfill\hfill\hfill\hfill\hfill\hfill}{\ } 
\contentsline {section}{\textbf{Соколов И.\,А.}\ \ см. Печинкин А.\,В.\hfill\hfill\hfill\hfill\hfill\hfill\hfill\hfill\hfill\hfill\hfill\hfill\hfill\hfill\hfill\hfill\hfill\hfill\hfill\hfill\hfill\hfill\hfill\hfill\hfill\hfill\hfill\hfill\hfill\hfill\hfill\hfill\hfill\hfill\hfill}{\ }
\contentsline {section}{\textbf{Ступников С.\,А.}\ \ см. Захаров В.\,Н.\hfill\hfill\hfill\hfill\hfill\hfill\hfill\hfill\hfill\hfill\hfill\hfill\hfill\hfill\hfill\hfill\hfill\hfill\hfill\hfill\hfill\hfill\hfill\hfill\hfill\hfill\hfill\hfill\hfill\hfill\hfill\hfill\hfill\hfill\hfill}{\ }
\contentsline {section}{\textbf{Чаплыгин В.\,В.}\ \ см. Печинкин А.\,В.\hfill\hfill\hfill\hfill\hfill\hfill\hfill\hfill\hfill\hfill\hfill\hfill\hfill\hfill\hfill\hfill\hfill\hfill\hfill\hfill\hfill\hfill\hfill\hfill\hfill\hfill\hfill\hfill\hfill\hfill\hfill\hfill\hfill\hfill\hfill}{\ } 
\contentsline {section}{\textbf{Чаплыгин В.\,В.}\ \ см. Печинкин А.\,В.\hfill\hfill\hfill\hfill\hfill\hfill\hfill\hfill\hfill\hfill\hfill\hfill\hfill\hfill\hfill\hfill\hfill\hfill\hfill\hfill\hfill\hfill\hfill\hfill\hfill\hfill\hfill\hfill\hfill\hfill\hfill\hfill\hfill\hfill\hfill}{\ }
\contentsline {section}{\textbf{Шоргин С.\,Я.}\ \ см. Батракова Д.\,А.\hfill\hfill\hfill\hfill\hfill\hfill\hfill\hfill\hfill\hfill\hfill\hfill\hfill\hfill\hfill\hfill\hfill\hfill\hfill\hfill\hfill\hfill\hfill\hfill\hfill\hfill\hfill\hfill\hfill\hfill\hfill\hfill\hfill\hfill\hfill}{\ } 
\contentsline {section}{\textbf{Шоргин С.\,Я.}\ \ см. Кудрявцев А.\,А.\hfill\hfill\hfill\hfill\hfill\hfill\hfill\hfill\hfill\hfill\hfill\hfill\hfill\hfill\hfill\hfill\hfill\hfill\hfill\hfill\hfill\hfill\hfill\hfill\hfill\hfill\hfill\hfill\hfill\hfill\hfill\hfill\hfill\hfill\hfill}{\ }
%\thispagestyle{myheadings}
\def\leftfootline{\small{\textbf{\thepage}
\hfill ИНФОРМАТИКА И ЕЁ ПРИМЕНЕНИЯ\ \ \ том~1\ \ \ выпуск~2\ \ \ 2007}
}%
 \def\rightfootline{\small{ИНФОРМАТИКА И ЕЁ ПРИМЕНЕНИЯ\ \ \ том~1\ \ \ выпуск~2\ \ \ 2007
 \hfill \textbf{\thepage}}}
 \label{end\stat}

%\def\stat{cont-e}
{%\hrule\par
%\vskip 7pt % 7pt
\raggedleft\Large \bf%\baselineskip=3.2ex
2\,0\,0\,7\ \ A\,U\,T\,H\,O\,R\ \ I\,N\,D\,E\,X \vskip 17pt
    \hrule
    \par
\vskip 21pt plus 6pt minus 3pt }

\label{st\stat}

\def\tit{\ }

\def\aut{\ }
\def\auf{\ }

\def\leftkol{\ } % ENGLISH ABSTRACTS}

\def\rightkol{\ } %ENGLISH ABSTRACTS}

\titele{\tit}{\aut}{\auf}{\leftkol}{\rightkol}


\contentsline {chapter}{\ }{Issue \quad Page} 
\contentsline {subsection}{\textbf{Batrakova D.\,A., Korolev V.\,Yu., Shorgin S.\,Ya.}\ \ A New Method for the Probabilistic and Statistical Analysis of Information Flows in Telecommunication Networks}{\qquad 1 \qquad 40} 
\contentsline {subsection}{\textbf{Borisov A.\,V.}\ \ Bayesian Estimation in\nobreakspace {}Observation Systems with\nobreakspace {}Markov Jump Processes: Game-Theoretic Approach}{\qquad 2 \qquad 65} 
\contentsline {subsection}{\textbf{Bosov A.\,V., Ivanov A.\,V.}\ \ Linguistic Simulation for Machine Translation and Knowledge Management Systems}{\qquad 2 \qquad 50} 
\contentsline {subsection}{\textbf{Chaplygin V.\,V.} see Pechinkin A.\,V.\hfill\hfill\hfill\hfill\hfill\hfill\hfill\hfill\hfill\hfill\hfill\hfill\hfill\hfill\hfill\hfill\hfill\hfill\hfill\hfill\hfill\hfill\hfill\hfill\hfill\hfill\hfill\hfill\hfill\hfill\hfill\hfill\hfill\hfill\hfill}{\ }
\contentsline {subsection}{\textbf{Chaplygin V.\,V.} see Pechinkin A.\,V.\hfill\hfill\hfill\hfill\hfill\hfill\hfill\hfill\hfill\hfill\hfill\hfill\hfill\hfill\hfill\hfill\hfill\hfill\hfill\hfill\hfill\hfill\hfill\hfill\hfill\hfill\hfill\hfill\hfill\hfill\hfill\hfill\hfill\hfill\hfill}{\ }
\contentsline {subsection}{\textbf{Ilyin V.\,D., Sokolov I.\,A.}\ \ The Symbol Model of Informatics Knowledge System in Human-Automaton Environment}{\qquad 1 \qquad 66} 
\contentsline {subsection}{\textbf{Ivanov A.\,V.} see Bosov A.\,V.\hfill\hfill\hfill\hfill\hfill\hfill\hfill\hfill\hfill\hfill\hfill\hfill\hfill\hfill\hfill\hfill\hfill\hfill\hfill\hfill\hfill\hfill\hfill\hfill\hfill\hfill\hfill\hfill\hfill\hfill\hfill\hfill\hfill\hfill\hfill}{\ }
\contentsline {subsection}{\textbf{Kalinichenko L.\,A.} see Zakharov V.\,N.\hfill\hfill\hfill\hfill\hfill\hfill\hfill\hfill\hfill\hfill\hfill\hfill\hfill\hfill\hfill\hfill\hfill\hfill\hfill\hfill\hfill\hfill\hfill\hfill\hfill\hfill\hfill\hfill\hfill\hfill\hfill\hfill\hfill\hfill\hfill}{\ }
\contentsline {subsection}{\textbf{Korolev V.\,Yu.} see Batrakova D.\,A.\hfill\hfill\hfill\hfill\hfill\hfill\hfill\hfill\hfill\hfill\hfill\hfill\hfill\hfill\hfill\hfill\hfill\hfill\hfill\hfill\hfill\hfill\hfill\hfill\hfill\hfill\hfill\hfill\hfill\hfill\hfill\hfill\hfill\hfill\hfill}{\ }
\contentsline {subsection}{\textbf{Kozerenko E.\,B.}\ \ Linguistic Simulation for Machine Translation and Knowledge Management Systems}{\qquad 1 \qquad 54} 
\contentsline {subsection}{\textbf{Kozmidiady V.\,A.} see Zakharov V.\,N.\hfill\hfill\hfill\hfill\hfill\hfill\hfill\hfill\hfill\hfill\hfill\hfill\hfill\hfill\hfill\hfill\hfill\hfill\hfill\hfill\hfill\hfill\hfill\hfill\hfill\hfill\hfill\hfill\hfill\hfill\hfill\hfill\hfill\hfill\hfill}{\ }
\contentsline {subsection}{\textbf{Kudryavtsev A.\,A., Shorgin S.\,Ya.}\ \ Bayesian Approach to Queueing Systems and Reliability Characteristics}{\qquad 2 \qquad 76} 
\contentsline {subsection}{\textbf{Pechinkin A.\,V., Sokolov I.\,A., Chaplygin V.\,V.}\ \ Multichannel Queuing System with Finite Buffer and Unreliable Servers}{\qquad 1 \qquad 27} 
\contentsline {subsection}{\textbf{Pechinkin A.\,V., Sokolov I.\,A., Chaplygin V.\,V.}\ \ Stationary Characteristics of a Multichannel Queueing System with\nobreakspace {}Simultaneous Refusals of Servers}{\qquad 2 \qquad 39} 
\contentsline {subsection}{\textbf{Shorgin S.\,Ya.} see Batrakova D.\,A.\hfill\hfill\hfill\hfill\hfill\hfill\hfill\hfill\hfill\hfill\hfill\hfill\hfill\hfill\hfill\hfill\hfill\hfill\hfill\hfill\hfill\hfill\hfill\hfill\hfill\hfill\hfill\hfill\hfill\hfill\hfill\hfill\hfill\hfill\hfill}{\ }
\contentsline {subsection}{\textbf{Shorgin S.\,Ya.} see Kudryavtsev A.\,A.\hfill\hfill\hfill\hfill\hfill\hfill\hfill\hfill\hfill\hfill\hfill\hfill\hfill\hfill\hfill\hfill\hfill\hfill\hfill\hfill\hfill\hfill\hfill\hfill\hfill\hfill\hfill\hfill\hfill\hfill\hfill\hfill\hfill\hfill\hfill}{\ }
\contentsline {subsection}{\textbf{Sinitsyn I.\,N.}\ \ Correlational Methods for Analytical Informational Models of the Earth Pole Fluctuations Design Based on a priori Data}{\qquad 2 \qquad \hphantom{9}2}
\contentsline {subsection}{\textbf{Sinitsyn I.\,N.}\ \ Development of Pugachev Filtering for Stochastic Systems}{\qquad 1 \qquad \hphantom{9}3}
\contentsline {subsection}{\textbf{Sokolov I.\,A.} see Ilyin V.\,D.\hfill\hfill\hfill\hfill\hfill\hfill\hfill\hfill\hfill\hfill\hfill\hfill\hfill\hfill\hfill\hfill\hfill\hfill\hfill\hfill\hfill\hfill\hfill\hfill\hfill\hfill\hfill\hfill\hfill\hfill\hfill\hfill\hfill\hfill\hfill}{\ }
\contentsline {subsection}{\textbf{Sokolov I.\,A.} see Pechinkin A.\,V.\hfill\hfill\hfill\hfill\hfill\hfill\hfill\hfill\hfill\hfill\hfill\hfill\hfill\hfill\hfill\hfill\hfill\hfill\hfill\hfill\hfill\hfill\hfill\hfill\hfill\hfill\hfill\hfill\hfill\hfill\hfill\hfill\hfill\hfill\hfill}{\ }
\contentsline {subsection}{\textbf{Sokolov I.\,A.} see Pechinkin A.\,V.\hfill\hfill\hfill\hfill\hfill\hfill\hfill\hfill\hfill\hfill\hfill\hfill\hfill\hfill\hfill\hfill\hfill\hfill\hfill\hfill\hfill\hfill\hfill\hfill\hfill\hfill\hfill\hfill\hfill\hfill\hfill\hfill\hfill\hfill\hfill}{\ }
\contentsline {subsection}{\textbf{Sokolov I.\,A.} see Zakharov V.\,N.\hfill\hfill\hfill\hfill\hfill\hfill\hfill\hfill\hfill\hfill\hfill\hfill\hfill\hfill\hfill\hfill\hfill\hfill\hfill\hfill\hfill\hfill\hfill\hfill\hfill\hfill\hfill\hfill\hfill\hfill\hfill\hfill\hfill\hfill\hfill}{\ }
\contentsline {subsection}{\textbf{Stupnikov S.\,A.} see Zakharov V.\,N.\hfill\hfill\hfill\hfill\hfill\hfill\hfill\hfill\hfill\hfill\hfill\hfill\hfill\hfill\hfill\hfill\hfill\hfill\hfill\hfill\hfill\hfill\hfill\hfill\hfill\hfill\hfill\hfill\hfill\hfill\hfill\hfill\hfill\hfill\hfill}{\ }
\contentsline {subsection}{\textbf{Zakharov V.\,N., Kalinichenko L.\,A., Sokolov I.\,A., Stupnikov S.\,A.}\ \ Development of Canonical Information Models for Integrated Information Systems}{\qquad 2 \qquad 15} 
\contentsline {subsection}{\textbf{Zakharov V.\,N., Kozmidiady V.\,A.}\ \ Means Providing Applications Fault Tolerance}{\qquad 1 \qquad 14} 
\def\leftfootline{\small{\textbf{\thepage}
\hfill ИНФОРМАТИКА И ЕЁ ПРИМЕНЕНИЯ\ \ \ том~1\ \ \ выпуск~2\ \ \ 2007}
}%
 \def\rightfootline{\small{ИНФОРМАТИКА И ЕЁ ПРИМЕНЕНИЯ\ \ \ том~1\ \ \ выпуск~2\ \ \ 2007
 \hfill \textbf{\thepage}}}
 \label{end\stat}


%\tableofcontents


\end{document}