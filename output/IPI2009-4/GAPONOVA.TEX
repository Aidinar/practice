
%\renewcommand{\le}{\leqslant}
%\renewcommand{\ge}{\geqslant}


\newcommand{\R}{\mathbb R}
\newcommand{\C}{\mathbb C}


%\newcommand{\I}{{\bf1}} %индикатор
\newcommand{\To}{\longrightarrow}
\newcommand{\bet}{\beta_{2+\delta}}
%\newcommand{\eps}{\varepsilon}

%\newcommand{\eqd}{\stackrel{d}{=}} % совпадение по распределению

%\renewcommand{\kappa}{\varkappa}

%\DeclareMathOperator{\Si}{Si}


%\renewcommand{\Re}{\mathop{\mathrm{Re}}\nolimits}
\renewcommand{\Im}{\mathop{\mathrm{Im}}\nolimits}

\def\stat{gaponova}


\def\tit{АСИМПТОТИЧЕСКИЕ ОЦЕНКИ АБСОЛЮТНОЙ ПОСТОЯННОЙ В~НЕРАВЕНСТВЕ
БЕРРИ--ЭССЕЕНА ДЛЯ~РАСПРЕДЕЛЕНИЙ, НЕ~ИМЕЮЩИХ ТРЕТЬЕГО
МОМЕНТА$^*$}
\def\titkol{Асимптотические оценки абсолютной постоянной в неравенстве
Берри--Эссеена} % для распределений, не имеющих третьего момента}

\def\autkol{М.\,О.~Гапонова,  И.\,Г.~Шевцова}
\def\aut{М.\,О.~Гапонова$^1$,  И.\,Г.~Шевцова$^2$}

\titel{\tit}{\aut}{\autkol}{\titkol}

{\renewcommand{\thefootnote}{\fnsymbol{footnote}}\footnotetext[1]
{Работа поддержана Российским фондом
фундаментальных исследований, проекты 08-01-00563, 08-01-00567 и
08-07-00152, грантом Президента Российской
Федерации МК-654.2008.1, а также Федеральным агентством по образованию, государственные контракты П-958
и П-1181.}}

\renewcommand{\thefootnote}{\arabic{footnote}}
\footnotetext[1]{Факультет вычислительной математики и
кибернетики Московского государственного университета им.\
М.\,В.~Ломоносова, margarita.gaponova@gmail.com}
\footnotetext[2]{Факультет вычислительной математики и
кибернетики Московского государственного университета им.\
М.\,В.~Ломоносова, ishevtsova@cs.msu.su}

\Abst{Уточнены асимптотические оценки Правитца абсолютной
постоянной в неравенстве Берри--Эссеена для случая одинаково
распределенных слагаемых с конечными третьими моментами. Впервые
построены аналогичные оценки для случая, когда слагаемые не имеют
моментов третьего порядка. Найдены верхние оценки асимптотически
правильных постоянных в центральной предельной теореме.}

\KW{центральная предельная теорема; нормальная
аппроксимация; оценка скорости сходимости; сумма независимых
случайных величин; неравенство Берри--Эссеена; дробь Ляпунова;
асимптотически правильная постоянная}

      \vskip 18pt plus 9pt minus 6pt

      \thispagestyle{headings}

      \begin{multicols}{2}

      \label{st\stat}

\section{Введение}

Одним из основных подходов при построении математических моделей
вероятностно-ста\-тис\-ти\-ческих закономерностей поведения тех или иных
характеристик стохастических ситуаций является тот, при котором в
качестве аналитической модели распределения вероятностей берется его
асимптотическая аппроксимация, воз\-ни\-ка\-ющая в соответствующей
предельной теореме. Примером применения такого подхода является
повсеместное использование в прикладных задачах нормальной
аппроксимации, основанное на центральной предельной теореме. Однако
при этом столь же часто обходится стороной вопрос о точности такой
аппроксимации, тогда как этот вопрос имеет крае\-уголь\-ное значение с
точки зрения аде\-кват\-ности используемых нормальных моделей, например
при оценивании тех или иных рисков. Наиболее общая оценка точности
нормальной аппроксимации дается хорошо известным неравенством
Берри--Эс\-се\-ена и представляет собой произведение некоторой
постоянной и дроби Ляпунова, которая в используемых моделях, как
правило, мала. Данная работа посвящена верхним асимптотическим
оценкам этой постоянной, справедливым для достаточно малых значений
ляпуновской дроби~--- очень важному случаю, который имеет место при
решении многих прикладных задач.

Пусть $X_1,X_2,\ldots$~--- независимые одинаково распределенные
случайные величины с общей функцией распределения
${F(x)=\p(X_1<x)}$, ${x\in\R}$, удов\-ле\-тво\-ря\-ющие условиям
\begin{equation}
\label{EDX1}
\e X_1 =  0\,;\quad\D X_1\ =\ 1\,.
\end{equation}
В этом случае выполняется центральная предельная теорема, согласно
которой
$$
\rho(F_n,\Phi)\equiv\sup_x|F_n(x)-\Phi(x)|\To 0,\quad n\to\infty\,,
$$
где
\begin{align*}
F_n(x)&=\p\left(\fr{X_1+\ldots+X_n} {\sqrt n}<x\right)=F^{*n}(x\sqrt n)\,;\\
\Phi(x)&=\fr{1}{\sqrt{2\pi}}\int\limits_{-\infty}^x e^{-t^2/2}\,dt\,.
\end{align*}
Для конструирования в центральной предельной теореме оценок скорости
сходимости, стремящихся к нулю с ростом числа слагаемых~$n$,
условий~\eqref{EDX1} оказывается недостаточно: в 1983~г.\
Мацкявичюс~\cite{Matskjavichus1983} показал, что как бы
медленно ни сходилась к нулю последовательность неотрицательных
чисел~$\delta_n$, найдется последовательность независимых одинаково
распределенных случайных величин, удовлетворяющих
условиям~\eqref{EDX1}, такая что для функции распределения~$F_n$
соответствующей нормированной суммы при достаточно больших~$n$
выполняется неравенство
$$
\rho(F_n,\Phi)\ge\delta_n\,.
$$
Поэтому в данной работе предполагается ограниченность момента
порядка $2+\delta$ с некоторым $0<\delta\le1$:
\begin{equation}
\label{bet}
\bet\equiv \e |X_1|^{2+\delta}<\infty\,.
\end{equation}
Обозначим через $\F_{2+\delta}$ класс всех функций распределения~$F$
случайной величины~$X_1$, удов\-ле\-тво\-ря\-ющих  условиям~\eqref{EDX1}
и~\eqref{bet}. Тогда для любого $F\in\F_{2+\delta}$ справедливо
неравенство (см., например,~\cite{Petrov1972})
\begin{equation}
\label{Bikelis}
\rho(F_n,\Phi)\le C(\delta) L_n^{2+\delta}\,,
\end{equation}
где $L_n^{2+\delta}$~--- дробь Ляпунова порядка ${2+\delta}$:
$$
L_n^{2+\delta}=\fr{\bet}{n^{\delta/2}}\,,
$$
а $C(\delta)>0$ зависит только от~$\delta$.

Случай $\delta=1$ представляет особый интерес, поскольку, как известно,
выполнение условия~\eqref{bet} при $\delta>1$ не приводит, вообще
говоря, к повышению скорости сходимости в центральной предельной
тео\-реме.

Впервые неравенство~\eqref{Bikelis} было доказано при $\delta=$\linebreak $=1$ в
1941--1942~гг.\ Берри~\cite{Berry1941} и
Эс\-се\-е\-ном~\cite{Esseen1942}. История отыскания
наименьшего возможного значения постоянной~$C(1)$ богата
результатами: этой задаче посвящены работы
Берри~\cite{Berry1941}, Эссеена~\cite{Esseen1942,
Esseen1956}, Сюя~\cite{Hsu1945},
Бергстрёма~\cite{Bergstrom1949}, Такано~\cite{Takano1951}, Колмогорова~\cite{Kolmogorov1953},
Уоллеса~\cite{Wallace1958}, 
Рогозина~\cite{Rogozin1960},
Золотарёва~[12--14], ван~Бика~\cite{VanBeek1971, VanBeek1972},
Правитца~\cite{Prawitz1975}, Шиганова~\cite{Shiganov1982}, Феллера~\cite{Feller1967},
Бенткуса~\cite{Bentkus1991, Bentkus1994},
Чистякова~[22--24], Нагаева и
Чеботарёва~\cite{NagaevChebotarev2006},
Шевцовой~\cite{Shevtsova2006, Shevtsova2008} (более подробный
обзор можно найти, например, в работе~\cite{Shevtsova2008}).
Наилучшая на сегодняшний день оценка абсолютной постоянной
$C(1)\le0.7005$ получена в~\cite{Shevtsova2008}.

Случай $0<\delta<1$ изучен гораздо меньше. Верхние оценки величины~$C(\delta)$
найдены  в 1983~г.\ Тысиаком~\cite{Tysiak1983} и
уточнены в работе авторов~\cite{GaponovaKorchaginShevtsova2009} (табл.~1).
%В частности, для некоторых $0<\delta<1$ получены следующие оценки:



В то же время во многих прикладных задачах неравенство
Берри--Эссеена используется при малых значениях ляпуновской дроби,
поэтому возникает естественный интерес к конструированию {\it
асимптотических} верхних оценок константы~$C(\delta)$, обеспечивающих
выполнение неравенства~\eqref{Bikelis} при малых~$L_n^{2+\delta}$. Один
подход к решению этой задачи предложен еще в 1966~г.\
Золотарёвым~\cite{Zolotarev1966} для $\delta=1$ и заключается в
уточнении структуры неравенства Берри--Эссеена за счет внесения
дополнительных  слагаемых, имеющих более высокий порядок ма-\linebreak\vspace*{-12pt}
\columnbreak


\noindent %tabl1
\begin{center}
\parbox{38mm}{{\tablename~1}\ \ \small{Верхние оценки величины $C(\delta)$ при некоторых~$\delta$ из
работы~\cite{GaponovaKorchaginShevtsova2009}}}
\end{center}
\vspace*{6pt}

{\small
\begin{center}
\tabcolsep=15pt
\begin{tabular}{|c|c|}
 \hline
$\delta$& $C$\\
\hline
0,1& $\le 1{,}0739$\\
0,2& $\le 1{,}0001$\\
0,3& $\le 0{,}9407$\\
0,4& $\le 0{,}8876$\\
0,5& $\le 0{,}8454$\\
0,6& $\le 0{,}8126$\\
0,7& $\le 0{,}7876$\\
0,8& $\le 0{,}7720$\\
0,9& $\le 0{,}7671$\\
\hline
\end{tabular}
\end{center}
}
%\end{table}
\vspace*{6pt}


\bigskip
\addtocounter{table}{1}


\noindent
лости, чем~$L_n^3$,
при $L_n^3\to0$. Развивая эту идею, автор получил
асимптотическое разложение
\begin{multline*}
\rho(F_n,\Phi)\le 1{,}300356
L_n^3+0{,}1387335\left(L_n^3\right)^{4/3}+{}\\
{}+
o\left(\left(L_n^3\right)^{4/3}\right)\,,\quad L_n^3\to 0\,,
\end{multline*}
справедливое и в общем случае разнораспределенных случайных
слагаемых. Два года спустя он уточнил~\cite{Zolotarev1967a,
Zolotarev1967b} этот результат:
\begin{multline}
\label{ZolotarevExpDifferentDistr}
\rho(F_n,\Phi)\le{}\\
{}\le 0{,}81968 L_n^3+0{,}05895\left(L_n^3\right)^{4/3}+
C_{\mathrm{З}}\left(L_n^3\right)^{5/3}
\end{multline}
в общем случае и
\begin{multline}
\label{ZolotarevExpIdenticalDistr}
\rho(F_n,\Phi)\le{}\\
{}\le 0{,}81968
L_n^3-0{,}9995\left(L_n^3\right)^2+\widetilde C_{\mathrm{З}}\left(L_n^3\right)^3
\end{multline}
в случае одинаково распределенных случайных слагаемых, где~$C_{\mathrm{З}}$ и
$\widetilde C_{\mathrm{З}}$~--- некоторые положительные постоянные.

В 1975~г.\ был опубликован замечательный результат Правитца~\cite{Prawitz1975}
\begin{multline*}
\rho(F_n,\Phi)\le 0{,}2731\eps_n+0{,}1997\eps_n'+ 0{,}0912
\eps_n''\,,\\
\eps_n+\eps_n'\le 0{,}2\,;
\end{multline*}
\begin{multline*}
\rho(F_n,\Phi)\le 0{,}2763\eps_n+0{,}2000\eps_n'+ 0{,}0972
\eps_n''\,,\\
\eps_n+\eps_n'\le \fr{2}{7}\,;
\end{multline*}
где
$$
\eps_n=L_n^3\left(\fr{n}{n-1}\right)^{3/2}\,;\quad
\eps_n'=\fr{1}{\sqrt n}\left(\fr{n}{n-1}\right)^{3/2}\,;
$$
$$
\eps_n''=\fr{1}{n}\left(\fr{n}{n-1}\right)^2\,,
$$
а также серия следствий
\pagebreak

\noindent
$$
C(1)\le
\begin{cases}
    0.5151\,, & \eps_n\le0{,}1\,; \\
    0.5270\,, & \eps_n\le \fr{1}{7}\,.
\end{cases}
$$
Результат Правитца можно переписать в традиционных обозначениях. Для
этого достаточно заметить, что из условий вида ${L_n^3\le\ell}$,
$0<\ell<1$, вытекает
$$
n\ge\left\lceil\ell^{-2}\right\rceil\,,\quad \eps_n\le
\fr{L_n^3}{(1-\ell^2)^{3/2}}\le
\fr{\ell}{(1-\ell^2)^{3/2}}\,;
$$
$$
\eps_n+\eps_n'\le\fr{2\ell}{(1-\ell^2)^{3/2}}\,,
$$
в частности условие ${L_n^3\le 0{,}0985}$ влечет
$$
n\ge 104\,,\quad\eps_n+\eps_n'<2\cdot 0{,}09996<0{,}2\,,
$$
а из условия ${L_n^3\le 0{,}1387}$ вытекает, что
$$
n\ge52\,,\quad\eps_n+\eps_n'<2\cdot0{,}1428<\fr{2}{7}\,.
$$
Таким образом, результатам Правитца можно придать следующий вид:
\begin{multline}
\rho(F_n,\Phi)\le 0{,}2772 L_n^3+\fr{0{,}2027}{\sqrt n}+
\fr{0{,}0930}{n}\,,\\
 L_n^3\le 0{,}0985\,; \label{Prawitz_exp1}
\end{multline}
\begin{multline}
\rho(F_n,\Phi)\le 0{,}2845 L_n^3+\fr{0{,}2060}{\sqrt n}+
\fr{0{,}1011}{n}\,,\\
 L_n^3\le0{,}1387\,; \label{Prawitz_exp2}
\end{multline}
\begin{equation}
\label{Prawitz_abs_asympt_estim}
C(1)\le
\begin{cases}
0{,}5151,& L_n^3\le0{,}0985\,;\\
0{,}5270,& L_n^3\le0{,}1387\,.\end{cases}
\end{equation}
Данные оценки справедливы и в общем случае разнораспределенных
случайных слагаемых.

В этой же работе Правитц анонсировал асимптотическое разложение,
существенно уточняющее результаты Золотарёва:
\begin{multline}
\label{Prawitz_estim_expansion}
\rho(F_n,\Phi)\le{}\\
{}\le
 \fr{2}{3\sqrt{2\pi}}\, L_{n-1}^3+
\fr{1}{2\sqrt{2\pi(n-1)}}+ C_{P}\left(L_n^3\right)^2\,,
\end{multline}
где $C_P$~--- некоторая положительная постоянная, а также высказал
предположение, что коэффициент
$$
\fr{2}{3\sqrt{2\pi}}=0{,}2659\ldots
%\ge\limsup_{\ell\to0}\limsup_{n\to\infty} \sup_{F\in\F_3\colon
%L_n^3=\ell}\rho(F_n,\Phi)/L_n^3
$$
при~$L_{n-1}^3$ не может быть уменьшен. По-видимому, этот мало
цитируемый результат не был замечен общественностью или, по крайней
мере, вызвал некоторое недоумение со стороны исследователей,
поскольку к тому времени был хорошо известен результат
Эссеена~\cite{Esseen1956}:
$$
\sup_{F\in\F_3}\limsup_{n\to\infty}\fr{\rho(F_n,\Phi)}{L_n^3}=
\fr{\sqrt{10}+3}{6\sqrt{2\pi}}=0{,}4097\ldots\,,
$$
на первый взгляд плохо согласующийся с анонсом Правитца. К тому же
доказательство неравенства~\eqref{Prawitz_estim_expansion},
по-ви\-ди\-мо\-му, автор предполагал опуб\-ли\-ко\-вать во второй части
работы~\cite{Prawitz1975} (название\linebreak указанной работы содержит в
конце римскую циф\-ру~I, указывающую на наличие продолжения), которая
по ка\-ким-то причинам опубликована не \mbox{была}. 
{\looseness=1

}

Тем не менее
анонсированная оценка является верной: по крайней мере, из
результатов опубликованной работы~\cite{Prawitz1975} вытекают
асимптотические оценки (подробнее см.\ разд.~4)
\begin{multline*}
\limsup_{\ell\to0}\limsup_{n\to\infty} \sup_{F\in\F_3\colon
L_n^3=\ell} \fr{\rho(F_n,\Phi)}{L_n^3}\le{}\\
{}\le \fr{2}{3\sqrt{2\pi}}<0{,}2660\,;
\end{multline*}

\vspace*{-12pt}

\noindent
\begin{multline*}
\limsup_{\ell\to0}\sup_{n,\,F\in\F_3\colon L_n^3=\ell}
\fr{\rho(F_n,\Phi)}{L_n^3} \le{}\\
{}\le \fr{7}{6\sqrt{2\pi}}<0{,}4655\,.
\end{multline*}

Последняя оценка обычно ассоциируется с именем Бенткуса,
воспринявшего идеи Правитца. Эти идеи были развиты Бенткусом в
работах~\cite{Bentkus1991, Bentkus1994} для неравномерных оценок
скорости сходимости в центральной предельной теореме, а для
равномерной он доказал асимптотическое разложение
\begin{multline}
\label{Bentkusexp}
\rho(F_n,\Phi)\le{}\\
{}\le \fr{2}{3\sqrt{2\pi}}\, L_n^3 +
\fr{1}{2\sqrt{2\pi}}\,\fr{1}{\sqrt{n}} + C_B\left(L_n^3\right)^{5/3}\,,
\end{multline}
где $C_B$~--- некоторая положительная постоянная. Оно несколько
лучше по сравнению с~\eqref{Prawitz_estim_expansion} в отношении
вида основных членов оценки и хуже в отношении остаточного. Из этого
разложения в силу неравенства $\beta_3\ge1$ вытекает оценка более
простой структуры
\begin{equation}
\label{Bentkusexp2}
\rho(F_n,\Phi)\le \fr{7}{6\sqrt{2\pi}}  L_n^3 +
C_B\left(L_n^3\right)^{5/3}\,,
\end{equation}
также являющаяся следствием
неравенства~\eqref{Prawitz_estim_expansion}, %шестнадцатью годами ранее 
анонсированного Правитцем.

В 2001~г.\ Г.\,П.~Чистяков~[22--24] доказал неравенство
\begin{multline}
\label{Chistexp}
\rho(F_n,\Phi)\le{}\\
{}\le \fr{\sqrt{10}+3}{6\sqrt{2\pi}}\, L_n^3+
C_{\mathrm{Ч}}\left(L_n^3\right)^{40/39}\left|\log L_n^3\right|^{7/6}\,,
\end{multline}
где $C_{\mathrm{Ч}}$~--- некоторая положительная абсолютная постоянная, в
котором коэффициент при~$L_n^3$ совпадает с асимптотически наилучшей
константой Эссеена.

К сожалению, из-за отсутствия конкретных чис\-ло\-вых оценок величин~$C_З$,
$\widetilde C_{\mathrm{З}}$, $C_{\mathrm{B}}$ и~$C_{\mathrm{Ч}}$ даже при малых значениях
ляпуновской дроби~$L_n^3$ асимптотическими неравенствами
Золотарёва~\eqref{ZolotarevExpDifferentDistr},
\eqref{ZolotarevExpIdenticalDistr}, Бенткуса~\eqref{Bentkusexp},
\eqref{Bentkusexp2} и Чистякова~\eqref{Chistexp} нельзя пользоваться
на практике. Наиболее выгодно с этой точки зрения выглядит результат
Правитца~\eqref{Prawitz_abs_asympt_estim}.

Отметим еще одну интересную работу. В~2006~г.\ Нагаев и
Чеботарёв~\cite{NagaevChebotarev2006} получили оценку
\begin{multline}
\rho(F_n,\Phi)\le 0{,}515489 L_n^3+ 0{,}427675\fr{1}{\sqrt n}+{}\\
{}+
0{,}153597\fr{1}{n}\,,\quad n\ge1\,, \label{NagaevChebotarev}
\end{multline}
выгодно отличающуюся от асимптотических разложений Золотарёва,
Бенткуса и Чистякова наличием конкретных числовых значений всех
входящих в нее констант и от результатов Правитца~--- отсутствием
каких бы то ни было условий на~$L_n^3$. Из~\eqref{NagaevChebotarev}
вытекают некоторые <<условные>> верхние оценки постоянной~$C(1)$ в
неравенстве Берри--Эс\-се\-ена (табл.~2).

\vspace*{12pt}
\noindent %tabl2
\begin{center}
\parbox{62mm}{{\tablename~2}\ \ \small{Некоторые <<условные>> верхние
оценки постоянной~$C(1)$ в неравенстве Берри--Эссеена}}
\end{center}
\vspace*{-6pt}

{\small
\begin{center}
\tabcolsep=7pt
\begin{tabular}{|l|c|c|}
\hline
\multicolumn{1}{|c|}{$\beta_3$}  & $n\ge3$ & $n\ge10$ \\
  \hline
$\ge3 $&$ C_0\le0{,}687607$ &$ C_0\le0{,}674238$ \\
%\hline
$\ge4 $& $C_0\le0{,}644578 $& $C_0\le0{,}634551 $\\
%\hline
 $\ge10$ & $C_0\le0{,}567124$ & $C_0\le0{,}563114$ \\
  \hline
\end{tabular}
\end{center}
%\end{table*}
}
%\end{table}
\vspace*{6pt}


\bigskip
\addtocounter{table}{1}

Сравнивая~\eqref{NagaevChebotarev} с оценками
Правитца~\eqref{Prawitz_exp1} и~\eqref{Prawitz_exp2}, можно
заметить, что значения всех входящих в нее констант почти в два раза
больше со\-от\-вет\-ст\-ву\-ющих коэффициентов в~\eqref{Prawitz_exp1}. Более
того, в качестве асимптотической верхней оценки абсолютной
постоянной в неравенстве Берри--Эссеена из~\eqref{NagaevChebotarev}
не может быть получено значение, меньшее~0,515489, из работы же
Правитца можно получить значение, сколько угодно близкое к
$7/(6\sqrt{2\pi})=0{,}4654\ldots,$ что примерно на пять сотых меньше.
%В этом смысле результат Правитца лучше.

%\vspace*{12pt}
\noindent %tabl3
\begin{center}
\parbox{37mm}{{\tablename~3}\ \ \small{Верхние оценки величины~$C(\delta)$ в~\eqref{Bikelis}
для $L_n^{2+\delta}\le0{,}4$ при некоторых~$\delta$}}
\end{center}
\vspace*{-6pt}

{\small
\begin{center}
\tabcolsep=15.5pt
\begin{tabular}{|c|c|}
\hline
$\delta$& $C$\\
\hline
0,1&0,6995\\
0,2&0,9230\\
0,3&0,8527\\
0,4&0,7722\\
0,5&0,7049\\
0,6&0,6674\\
0,7&0,6518\\
0,8&0,6508\\
0,9&0,6611\\
1,0&0,6861\\
 \hline
\end{tabular}
\end{center}
%\end{table*}
}
%\end{table}
\vspace*{6pt}


\bigskip
\addtocounter{table}{1}

Для случая ${0<\delta<1}$ аналогичные асимптотические результаты
отсутствуют вовсе. Цель данной работы~--- восполнить этот пробел. В
частности, будет показано, что наилучшие на сегодняшний день верхние
абсолютные оценки величины~$C(\delta)$, полученные в
работе~\cite{GaponovaKorchaginShevtsova2009}, могут быть существенно
уточнены уже при $L_n^{2+\delta}\le0{,}4$ (табл.~3).


Кроме того, будет уточнено следствие результата Правитца для одинаково
распределенных слагаемых:
$$
C(1)\le
\begin{cases}
0{,}4889\,,&L_n^3\le0{,}0985\,;\\
0{,}5010,&L_n^3\le0{,}1387.
\end{cases}
$$

\section{Вспомогательные результаты}

В данном разделе приведено неравенство сглаживания Правитца, а также
некоторые оценки для характеристических функций, являющиеся
следствиями результатов работы одного из авторов~\cite{Shevtsova2009}.

Обозначим через~$\mathcal G$ класс комплекснозначных функций~$K(t)$,
$t\in[-1,\,1]$, удовлетворяющих следующим условиям:
\begin{enumerate}[($i$)]
\item $\mathrm{Re}\, K(t)$ четна, $\Im K(t)$ нечетна;
\item $\mathrm{Re}\, K(t)$ абсолютно интегрируема на $[-1,1]$;
\item для некоторого $\zeta\in\C$, $\zeta=const\neq0$, функция
$\Im K(t)-\zeta/t$ абсолютно {ин\-те\-гри\-ру\-е\-ма} на $[-1,1]$;
\item для всех $x\in\R$
$$
\vp\int\limits_{-1}^1{e^{itx}K(t)}\,dt\ge \fr{1}{2}-E(x)\,,
$$
где $E(x)$~--- функция вырожденного в нуле распределения:
$$ E(x)=
\begin{cases}
0\,,&x\le0\,;\\
1\,,&x> 0\,.
\end{cases}
$$
\end{enumerate}

\smallskip

\noindent
\textbf{Лемма 1} (см.~\cite{Prawitz1972}). %\label{LemObrawenija}
\textit{Пусть $K(t)\in \mathcal G$ с $\zeta=1/(2\pi)$, $F(x)$ и~$G(x)$~---
произвольные функции распределения с характеристическими функциями
$f(t)$ и $g(t)$ соответственно. Если $G(x)$ непрерывна, то для любых
$T\ge T_1>0$ справедливо неравенство}
\begin{multline*}
\sup_{x\in\R}\left|F(x)-G(x)\right|\le
\fr{1}{T}\int\limits_{-T_1}^{T_1}\left|K\left(\fr
{t}{T}\right)\right| \left|f(t)-{}\right.\\[3pt]
\left.{}-g(t)\right|\,dt+\fr{1}{T}\int\limits_{T_1<|t|\le
T}{\left|K\left(\fr{t}{T}\right)\right|\left|f(t)\right|\,dt}+{}\\
{}+\int\limits_{-T_1}^{T_1}\!\left|\fr{1}{T} K\left(\fr{t}{T}\right)-\fr {i}{2\pi t}
\right|\left|g(t)\right|\,dt+
\fr{1}{2\pi}\int\limits_{|t|>T_1}\!\!\!\left|g(t)\right|\fr{dt}{|t|}\,.\hspace*{-0.153pt}
\end{multline*}

\bigskip

\noindent
\textbf{Лемма 2} (см.~\cite{Prawitz1972}). %\label{Ocenka|K|and|K-i/2pit|}
\textit{Функция
\begin{multline*}
K(t)=\fr{1}{2}\left(1-|t|\right)+\fr{i}{2}\left[(1-|t|)\cot \pi t+ \fr{\mathrm{sign}\,
t}\pi\right]\,,\\[3pt]
-1\le t\le1\,,
\end{multline*}
принадлежит классу~$\mathcal G$ с $\zeta=1/(2\pi)$. При этом для
всех $t\in [-1,\,1]$ справедливы следующие оценки:}
$$
\left| K(t)\right| \leqslant \fr{1{,}0253}{2\pi|t|}\,,
$$
$$
\left| K(t)-\fr i{2\pi t}\right| \leqslant \fr{1}{2}\left(1-|t|+
\fr{\pi^2t^2}{18}\right)\,.
$$

\smallskip

Следующий довольно очевидный факт впервые был замечен
Тысиаком~\cite{Tysiak1983} при $n=2$.

\bigskip

\noindent
\textbf{Лемма 3.} %\label{LemShevtsova2}
\textit{Для любого $0<\delta\le1$  и целого $n\ge0$}
$$
\gamma (n,\delta)= \sup\limits_{x>0} \fr{1}{x^{n+\delta}}\left\vert e^{ix}-
\sum\limits_{k=0}^n {\fr{(ix)^k}{k!}}\right\vert <\infty.
$$
\smallskip

\noindent
Д\,о\,к\,а\,з\,а\,т\,е\,л\,ь\,с\,т\,в\,о\,.\ Функция
$$
g(x)=\fr{1}{|x|^{n+\delta}}\left|\,e^{ix}-\sum\limits_{k=0}^n
{\fr{(ix)^k}{k!}}\right|\,,\quad x\in\R\,,
$$
($g(0)$ доопределяем по непрерывности) 
положительна для всех $x\neq0$, причем
\begin{align*}
\lim_{x \rightarrow 0}{g(x)}&\le \lim_{x \rightarrow
0}{\fr{|x|^{1-\delta}}{(n+1)!}}\le \fr{1}{(n+1)!}<\infty\,,\\[3pt]
\lim_{x \to \infty}{g(x)}&\le \lim_{x \to \infty}\fr{1}{|x|^{n+\delta}}
\bigg(1+\sum\limits_{k=0}^n{\fr{|x|^k}{k!}}\bigg)=0\,,
\end{align*}
следовательно, $g(x)$ ограничена на~$\R$. Кроме того,
$g(x)$ четна, поэтому
$$
\gamma (n,\delta)=\sup\limits_{x>0}{g(x)}=\sup\limits_{x\in
R}{g(x)}<\infty\,,
$$
что и требовалось доказать. $\square$

\smallskip

Из доказанной леммы очевидным образом вытекает следующее
утверждение.

\smallskip

\noindent
\textbf{Следствие 1.} (см.\ также~\cite{Tysiak1983}) %\label{SlGamma}
\textit{Для любого $0<\delta\le1$ справедлива оценка}
$$
\left|e^{ix}-1-ix+\fr{x^2}2\right|\le \gamma(2,\delta)|x|^{2+\delta},
\quad x\in \mathbb R\,.
$$
%где $\gamma(\d)\equiv\gamma(2,\d)$.
%\begin{equation}\label{gamma_delta}
%\gamma(\d)\equiv\gamma(2,\d)= \sup_{x>0}\sqrt{\left(\frac{\cos
%x-1+x^2/2}{x^{2+\d}}\right)^2 + \left(\frac{\sin
%x-x}{x^{2+\d}}\right)^2 }.
%\end{equation}

Перейдем теперь к конструированию оценок разности между допредельной
и предельной нормальной характеристическими функциями. Обозначим

\noindent
$$
f(t)=\e e^{itX_1}=\int\limits_{-\infty}^\infty e^{itx}\,dF(x)\,,
$$

\vspace*{-12pt}

\noindent
\begin{multline*}
f_n(t)\equiv\e\exp\left\{it\fr{X_1+\ldots+X_n}{\sqrt{n}}\right\}={}\\
{}=
\left(f\left(\fr{t}{\sqrt n}\right)\right)^n,\quad t\in\R\,.
\end{multline*}

\smallskip

\noindent
\textbf{Лемма 4.} \textit{В условиях~\eqref{EDX1} и~\eqref{bet} с некоторым} $0<\delta\le$\linebreak $\le 1$
\textit{справедлива оценка}
$$
\left|f(t)-e^{-t^2/2}\right|\le \gamma(\delta)\bet|t|^{2+\delta}+\fr{t^4}{8}\,, \quad
t\in\R\,,
$$
\textit{где}
\vspace*{-2pt}

\noindent
\begin{multline*}
\gamma(\delta)\equiv\gamma(2,\delta)={}\\
{}= \sup_{x>0}\sqrt{\left(\fr{\cos
x-1+x^2/2}{x^{2+\delta}}\right)^2 + \left(\fr{\sin
x-x}{x^{2+\delta}}\right)^2 }\,,\\
 0<\delta\le1\,.
\end{multline*}

\smallskip

Значения величины~$\gamma(\delta)$ для некоторых~$\delta$ приведены в
табл.~4.


\smallskip

\noindent
Д\,о\,к\,а\,з\,а\,т\,е\,л\,ь\,с\,т\,в\,о\,. \
В силу условий леммы имеем
\begin{multline*}
\!\!f(t)-e^{-t^2/2}= \!\int\limits_{-\infty}^\infty\!\left(e^{itx}-1-itx+
\fr{(tx)^2}{2}\right)\,dF(x)+{}\\
{}+1-\fr{t^2}{2}-e^{-t^2/2}\,,\quad
t\in\R\,.
\end{multline*}
Оценивая подынтегральное выражение при помощи
следствия~1 и применяя элементарное неравенство
\pagebreak

\noindent %tabl4
\begin{center}
\parbox{74mm}{{\tablename~4}\ \ \small{Значения величин~$\gamma(\delta)$, $\varkappa(\delta)$
и $\theta_0(\delta)$
для некоторых~$\delta$}}
\end{center}
%\vspace*{2pt}

{\small
\begin{center}
\tabcolsep=15pt
\begin{tabular}{|c|c|c|c|}
\hline
$\delta$ & $\gamma(\delta)$ & $\varkappa(\delta)$& $\theta_0(\delta)$  \\
\hline
 0+   & 0,5316 & 0,5000 & 6,2831  \\
%\hline
 0,05 & 0,4886 & 0,4564 & 6,1331  \\
%\hline
 0,10 & 0,4499 & 0,4171 & 5,9941  \\
%\hline
 0,15 & 0,4150 & 0,3816 & 5,8631  \\
%\hline
 0,20 & 0,3834 & 0,3495 & 5,7384  \\
%\hline
 0,25 & 0,3549 & 0,3204 & 5,6183  \\
%\hline
 0,30 & 0,3291 & 0,2941 & 5,5021  \\
%\hline
 0,35 & 0,3059 & 0,2702 & 5,3887 \\
%\hline
 0,40 & 0,2848 & 0,2485 & 5,2778 \\
%\hline
 0,45 & 0,2658 & 0,2288 & 5,1686 \\
%\hline
 0,50 & 0,2487 & 0,2109 & 5,0609 \\
%\hline
 0,55 & 0,2332 & 0,1946 & 4,9542 \\
%\hline
 0,60 & 0,2194 & 0,1797 & 4,8483 \\
%\hline
 0,65 & 0,2071 & 0,1662 & 4,7427 \\
%\hline
 0,70 & 0,1961 & 0,1538 & 4,6374 \\
%\hline
 0,75 & 0,1866 & 0,1425 & 4,5320 \\
%\hline
 0,80 & 0,1784 & 0,1322 & 4,4263 \\
%\hline
 0,85 & 0,1716 & 0,1228 & 4,3200 \\
%\hline
 0,90 & 0,1666 & 0,1143 & 4,2131 \\
%\hline
 0,95 & 0,1638 & 0,1064 & 4,1051 \\
%\hline
 1,00 & 0,1667 & 0,0992 & 3,9959 \\
\hline
\end{tabular}
\end{center}
%\end{table}
\vspace*{4pt}

}


\bigskip
\addtocounter{table}{1}


$$
|e^{-x}-1+x|\le \fr{x^2}{2}\,,\quad x\ge0\,,
$$
к оставшимся слагаемым, получаем
\begin{multline*}
|f(t)-e^{-t^2/2}|\le
\gamma(\delta)\int\limits_{-\infty}^\infty\left|tx\right|^{2+\delta}\,dF(x)+
\fr{t^4}{8}={}\\
{}= \gamma(\delta)\bet|t|^{2+\delta}+\fr{t^4}{8}\,,\quad t\in\R\,,
\end{multline*}
что и требовалось доказать. $\square$

\smallskip

\noindent
\textbf{Лемма 5} (см.~\cite{Shevtsova2009}). %\label{SlStep}
\textit{В условиях~\eqref{EDX1} и~\eqref{bet} с некоторым ${0<\delta\le1}$
справедлива оценка}
\begin{multline*}
\!\!|f(t)|\le  \exp\left\{-\fr{\sigma^2t^2}{2}+
\varkappa(\delta)\left(\beta_{2+\delta}+
\sigma^{2+\delta}\right)|t|^{2+\delta}\right\}\,,\\ t\in\R\,,
\end{multline*}
\textit{где}
$$
\varkappa(\delta)=\sup_{x>0}\fr{\left|\cos
x-1+x^2/2\right|}{x^{2+\delta}}\,,\quad  0<\delta\le1\,.
$$

\smallskip

Значения величины~$\varkappa(\delta)$ для некоторых~$\delta$ приведены в
табл.~4.

\smallskip
\noindent
\textbf{Лемма 6} (см.~\cite{Shevtsova2009}). %\label{SlCos}
\textit{В условиях~\eqref{EDX1} и~\eqref{bet} с некоторым ${0<\delta\le1}$
справедлива оценка}

\noindent
%\begin{equation}\label{OcenkaLgf(t)Cos}
\begin{multline*}
|f(t)|\le \exp\left\{-\fr{1-\cos\left((\bet+1)^{1/\delta} t\right)}
{(\beta_{2+\delta}+1)^{2/\delta}}\right\}\,,\\
 \theta_0(\delta)\le (\bet+1)^{1/\delta}|t|  \le2\pi\,,
\end{multline*}
\textit{где $\theta_0(\delta)$~--- единственный корень уравнения}

\noindent
$$
\fr{\delta\theta^2}2+ \theta\sin \theta + (2+\delta)(\cos \theta - 1)=0,
$$
\textit{лежащий в интервале $(\pi,\,2\pi)$}.


\smallskip

\noindent
\textbf{Замечание 1.}
%\label{optim_ocenka|f(t)|}
В работе~\cite{Shevtsova2009} также было показано, что минимум из
трех оценок для~$|f(t)|$, устанавли\-ва\-емых
леммами~5,~6 и неравенством $|f(t)|\le1$, имеет
сле\-ду\-ющий вид:
$$
|f(t)|\le \exp\{-\psi_\delta(t)\}\,,
$$
где
$$
\psi_\delta(t)=
\begin{cases}
\fr{t^2}2-\varkappa\left(\delta\right)\left(\beta_{2+\delta}+
 1\right)\left|t\right|^{2+\delta}\,, \\
\quad\quad\quad\quad\quad\quad\quad  |t|\le\fr{\theta_0(\delta)}{(\bet+1)^{1/\delta}}\,, \\[6pt]
\fr{1-\cos\left(\left(\bet+1\right)^{1/\delta}t\right)}
 {\left(\beta_{2+\delta}+1\right)^{2/\delta}}\,,\\
\quad\quad\quad\quad \theta_0(\delta)\le \left(\bet+1\right)^{1/\delta}\left|t\right|  \le2\pi\,, \\
    0\,, \quad\quad\quad\quad\qquad |t|\ge\fr{2\pi}{(\bet+1)^{1/\delta}}\,.
\end{cases}
$$


Значения величины $\theta_0(\delta)$ при некоторых $\delta$ приведены в
табл.~4.

Как несложно убедиться,
$$
\lim_{\delta\to0+}\theta_0(\delta)=2\pi=6{,}2831\ldots
$$

\noindent
\textbf{Лемма 7.}
%\label{LemOcenka|fn(t)-exp|}
\textit{В условиях~\eqref{EDX1} и~\eqref{bet} с некоторым ${0<\delta\le1}$ для
всех $t\in\R$ справедлива оценка
\begin{multline*}
\left|f_n(t)- e^{-t^2/2}\right|\le \left(\gamma(\delta)
\fr{\bet|t|^{2+\delta}}{n^{\delta/2}}+ \fr{t^4}{8n}\right)\times{}\\
{}\times
\exp\left\{-\fr{t^2}{2}\,\fr{n-1}n \left(1-
2\varkappa(\delta)(\bet+1)\fr{|t|^{\delta}}{n^{\delta/2}} \right)\right\},
\end{multline*}
%\begin{multline*}%\label{OcenkaF(u)-expOconch}
%\left|f_n(t)- e^{-t^2/2}\right|\le \left(\gamma(\d)
%\frac{\bet|t|^{2+\d}}{n^{\d/2}}+ \frac{t^4}{8n}\right)\times \\
%\times \exp\left\{-\frac{t^2}2\cdot\frac{n-1}n \left(1-
%2\kappa(\d)(\bet+1)\frac{|t|^{\d}}{n^{\d/2}}
%\right)\right\},
%\end{multline*}
где $\gamma(\delta)$ и $\varkappa(\delta)$ определены в
леммах~4 и~5 соответственно.
}

\smallskip

\noindent
Д\,о\,к\,а\,з\,а\,т\,е\,л\,ь\,с\,т\,в\,о\,.\ Из леммы~5 вытекает, что
\begin{multline*}
\max\left\{f\left(\fr t{\sqrt n}\right),e^{-t^2/(2n)}\right\}\le{}\\
{}\le
\exp\left\{-\fr{t^2}{2n}+
\varkappa(\delta)(\bet+1)\fr{|t|^{2+\delta}}{n^{1+\delta/2}} \right\}\,,\quad
t\in\R\,.
\end{multline*}
Применяя элементарное неравенство
\begin{multline*}
\left|u^n-v^n\right|= |u-v|\left \vert
\sum\limits_{k=0}^{n-1}{u^kv^{n-k-1}}\right \vert
\le{}\\
{}\le |u-v|\left(\sum\limits_{k=0}^{n-1}|u|^k|v|^{n-k-1}\right)\le
n|u-v|w^{n-1}\,,
\end{multline*}
справедливое для любых комплексных чисел~$u$, $v$ и
$w\ge\max\{|u|,|v|\}$ с
$$
u=\exp\left\{-\fr{t^2}{2n}\right\}\,;\quad v=f\left(\fr t{\sqrt
n}\right)\,,
$$
$$
w=\exp\left\{-\fr{t^2}{2n}+
\varkappa(\delta)(\bet+1)\fr{|t|^{2+\delta}}{n^{1+\delta/2}} \right\}\,,
$$
и оценивая разность $|u-v|$ при помощи
леммы~4:
\begin{multline*}
|u-v|=\left|f\left(\fr t{\sqrt n}\right)- e^{-t^2/(2n)}\right|\le{}\\
{}\le
\gamma(\delta)\bet \fr{|t|^{2+\delta}}{n^{1+\delta/2}}+ \fr{t^4}{8n^2}\,,\quad
t\in\R\,,
\end{multline*}
получаем требуемое.  $\square$

\section{Основной результат}

Всюду далее будем использовать следующие обозначения:
\begin{equation}
\left.
\begin{array}{c}
R_n\equiv\rho(F_n,\Phi)=\displaystyle\sup_x|F_n(x)-\Phi(x)|\,;
\\[6pt]
\eps=L_n^{2+\delta}=\fr{\beta_{2+\delta}}{n^{\delta/2}}\,;\quad
\quad \lambda=\sqrt{\fr{n-1}n}\,;
\\[6pt]
\gamma(\delta)={}\ \ \ \ \ \ \ \ \ \ \ \ \ \ \ \ \ \ \hspace*{50mm} \\
\!\!\!\!\!\!\!\!\!{}=\displaystyle\sup_{x>0}\sqrt{\left(\fr{\cos
x-1+x^2/2}{x^{2+\delta}}\right)^2 + \left(\fr{\sin
x-x}{x^{2+\delta}}\right)^2 }\,;\!\!\!
\\[6pt]
\varkappa(\delta)= \displaystyle\sup_{x>0}\fr{\left|\cos x-1+x^2/2\right|}{x^{2+\delta}}\,.
\end{array}\!
\right \}\!\!\!
\label{eps'_eps''_lambda}
\end{equation}

\smallskip

\noindent
\textbf{Теорема 1.}
%\label{ThRavnOcenkaTt0}
\textit{Пусть $\theta_0(\delta)$~--- единственный корень уравнения
$$
\fr{\delta\theta^2}2+\theta\sin \theta - (2+\delta)(1-\cos \theta)=0\,,\quad
0<\theta\le2\pi\,,
$$
лежащий в интервале $(\pi,\,2\pi)$. Обозначим
$$
T=\fr{2\pi}{(\eps+\eps')^{1/\delta}}\,,\quad t_1(\delta)=
\fr{\theta_0(\delta)}{2\pi}\,.
$$
Тогда для всех $n\ge2$ и $0<t_0\le t_1(\delta)$ справедлива оценка}

\noindent
\begin{multline*}
R_n\le  I_1(\lambda T,t_0,\delta)\fr{\eps}{\lambda^{2+\delta}}+{}\\
{}+
\fr{(\eps+\eps')^{1/\delta}}{2\sqrt{2\pi}}
+I_2(\lambda T,t_0,\delta)\fr{(\eps')^{2/\delta}}{\lambda^4} +{} \\
{}+I_3(T,t_0,\delta) (\eps+\eps')^{2/\delta}+
\fr{(\eps+\eps')^{3/\delta}}{144\sqrt{2\pi}}\,,
\end{multline*}
\textit{где}

\vspace*{-12pt}

\noindent
\begin{multline*}
I_1(v,t_0,\delta)
=2\gamma(\delta)v^{2+\delta}\int\limits_0^{t_0}\left|K(t)\right|\times{}\\
{}\times
\exp\left\{-\fr{v^2t^2}2 \left(1-2(2\pi)^\delta\varkappa(\delta)t^{\delta}
\right)\right\} t^{2+\delta}\,dt,
\end{multline*}

\vspace*{-12pt}

\noindent
\begin{multline*}
I_2(v,t_0,\delta)= \fr{v^4}4\int\limits_0^{t_0}\left|K(t)\right|\times{}\\
{}\times
\exp\left\{-\fr{v^2t^2}2 \left(1-2(2\pi)^\delta\varkappa(\delta)t^{\delta}
\right)\right\} t^4\, dt\,,\enskip v>0\,;
\end{multline*}

\vspace*{-12pt}

\noindent
\begin{multline*}
I_3(T,t_0,\delta)={}\\
{}= \fr{I_{31}(T,t_0)+
I_{32}(T,\delta)+I_{33}(T,t_0,\delta)-1}{4\pi^2}\,;
\end{multline*}

\vspace*{-12pt}

\noindent
\begin{multline*}
I_{31}(T,t_0)={}\\
{}=
T^2\int\limits_{t_0}^{\infty}{\left(\fr{1}{\pi t}
-1+t-\fr{\pi^2t^2}{18}\right)e^{-T^2t^2/2}\,dt}\,;
\end{multline*}

\vspace*{-12pt}

\noindent
\begin{multline*}
I_{32}(T,\delta)={}\\
{}=
2T^2\int\limits_{t_1(\delta)}^1\left|K(t)\right|
\exp\left\{-T^2\fr{1-\cos 2\pi t}{4\pi^2}\right\} \,dt;
\end{multline*}

\vspace*{-12pt}

\noindent
\begin{multline*}
I_{33}(T,t_0,\delta)=\fr{1{,}0253}{\pi}\, T^2\times{}\\
{}\times
\int\limits_{t_0}^{t_1(\delta)}
\exp\left\{-\fr{T^2t^2}{2}\left(1-
2\left(2\pi\right)^\delta \varkappa(\delta)t^\delta \right)\right\}
\fr{dt}t\,;
\end{multline*}

\vspace*{-12pt}

\noindent
\begin{multline*}
K(t)=\fr{1}{2}\left(1-|t|\right)+\fr{i}{2}\left[(1-|t|)\cot \pi t+ \fr{\mathrm{sign}\,
t}\pi\right]\,,\\ |t|\le1\,.
\end{multline*}
При этом

\noindent
$$
\sup_{v>0}I_k(v,t_0,\delta)<\infty\,,\quad k=1, 2, 3,
$$
\textit{для всех  $0<\delta\le1$ и $0<t_0\le t_1(\delta)$.
}
%\pagebreak


Значения величин $\gamma(\delta)$ и~$\varkappa(\delta)$ для некоторых $\delta$
приведены в леммах~4 и~5 соответственно, а значения $\theta_0(\delta)$~--- в
замечании~1.

\smallskip

\noindent
Д\,о\,к\,а\,з\,а\,т\,е\,л\,ь\,с\,т\,в\,о\,.\
Напомним, что~$f(t)$ обозначает характеристическую функцию случайной
величины~$X_1$, а~$f_n(t)$~--- характеристическую функцию
нормированной суммы ${(X_1+\ldots+X_n)/\sqrt n}$. Из
лемм~1 и~2 вытекает,
что для величины~$R_n$ при всех ${0<t_0\le1}$ и $T>0$ справедлива
оценка (полагаем $T_1=t_0T$)
\begin{multline*}
R_n\le \int\limits_{-t_0}^{t_0}{\left|K(t)\right|\,\left|f_n(Tt)-e^{-T^2t^2/2}\right|\,dt}+{}\\
{}+\int\limits_{t_0<|t|\le1}{\left|K(t)\right|\,\left|f_n(Tt)\right|\,dt}+{}\\
{}+\int\limits_{-t_0}^{t_0}{\left|K(t)-\fr {i}{2\pi t}\right|e^{-T^2t^2/2}\,dt}+{}\\
{}+ \fr{1}{2\pi}
\int\limits_{|t|>t_0}{e^{-T^2t^2/2}\,\fr{dt}{|t|}}\equiv
\widetilde I_1+ \widetilde I_2+\widetilde I_3+\widetilde I_4\,.
\end{multline*}
Учитывая результат леммы~2
$$
\left| K(t)-\fr{i}{2\pi t}\right| \leqslant \fr{1}{2}\left(1-|t|+
\fr{\pi^2t^2}{18}\right)\,,\quad |t|\le1\,,
$$
для $\widetilde I_3$ и $\widetilde I_4$ получаем
\begin{multline*}
\widetilde I_3+\widetilde I_4\equiv \int\limits_{-t_0}^{t_0}{\left|K(t)-\fr{i}{2\pi
t}\right|e^{-T^2t^2/2}\,dt}+{}\\
{}+
 \fr{1}{2\pi} \int\limits_{|t|>t_0}{e^{-T^2t^2/2}\,\fr{dt}{|t|}}\le{}\\
{}\le\fr{1}{2}\int\limits_{-t_0}^{t_0} {\left(1-|t|+\fr{\pi^2t^2}{18}\right)
e^{-T^2t^2/2}\,dt}+{}\\
{}+
\fr{1}{\pi}\int\limits_{t_0}^{\infty}e^{-T^2t^2/2}\fr{dt}{t}={}\\
{}=\int\limits_0^\infty {\left(1-t+\fr{\pi^2t^2}{18}\right) e^{-T^2t^2/2}\,dt}+ {}\\
{}+
\int\limits_{t_0}^{\infty}\left(\fr{1}{\pi t}-1+
t-\fr{\pi^2t^2}{18}\right)e^{-T^2t^2/2}\,dt\equiv {}\\
{}\equiv \sqrt{\fr{\pi}{2}}\,\fr{1}{T}-\fr{1}{T^2}+
\fr{\pi^{5/2}}{18\sqrt2}\,\fr{1}{T^3}+\fr{I_{31}(T,t_0)}{T^2}\,.
\end{multline*}

Оценим $\widetilde I_2$. Пусть $t_1$~--- произвольное число на
отрезке $[t_0,\,1]$. Разобьем область интегрирования в $\widetilde
I_2$ на две части

\noindent
\begin{multline*}
\widetilde I_2\equiv
\int\limits_{t_0\le|t|\le1}|K(t)|\,|f_n(Tt)|\,dt={}\\[12pt]
{}=
\int\limits_{t_0\le |t|\le t_1}+
\int\limits_{t_1\le|t|\le1}|K(t)|\,|f_n(Tt)|\,dt\equiv \widetilde
I_{21}+\widetilde I_{22}\,.
\end{multline*}
Для оценки первого интеграла применим результаты леммы~5:
\begin{equation*}%\label{OcenkaF(u)forAll}
|f_n(t)|\le\exp\left\{-\fr{t^2}2\left(1-
2\varkappa(\delta)(\eps+\eps')|t|^{\delta}\right)\right\}\,,\enskip t\in\R\,,
\end{equation*}
и леммы~2:
$$
\left| K(t)\right| \leqslant \fr{1{,}0253}{2\pi|t|}\,,\quad |t|\le1\,,
$$
а второй интеграл оценим при помощи более тонкого неравенства из
леммы~6:
\begin{align*}
\left|f_n(t)\right|&\le \exp\left\{
-\fr{1-\cos\left((\eps+\eps')^{1/\delta}t\right)}
{(\eps+\eps')^{2/\delta}} \right\}\,;\\[6pt]
\theta_0(\delta) &\le
(\eps+\eps')^{1/\delta}|t|\le 2\pi\,.
\end{align*}
Последнее неравенство можно использовать для оценки второго
интеграла в $\widetilde I_2$, если величины~$t_1$ и~$T$
удовлетворяют соотношениям
$$
\theta_0(\delta)\le t_1 T(\eps+\eps')^{1/\delta}\,,\quad
T(\eps+\eps')^{1/\delta}\le 2\pi\,.
$$
С другой стороны, принимая во внимание сла\-га\-емое вида~$1/T$ в оценке
для $\widetilde I_3+\widetilde I_4$, величину~$T$ нужно выбирать как
можно б$\acute{\mbox{о}}$льшей, поэтому окончательно положим
\begin{equation}
\left.
\begin{array}{rcl}
T\!\!\!&=&\!\!\!\fr{2\pi}{(\eps+\eps')^{1/\delta}}\,;\\[12pt]
t_1\!\!\!&=&\!\!\!t_1(\delta)=\fr{\theta_0(\delta)}{T(\eps+\eps')^{1/\delta}}=
\fr{\theta_0(\delta)}{2\pi}\,.
\end{array}
\right \}
\label{choice T_t1}
\end{equation}
Тогда для любого $t_0\le t_1(\delta)$ получим
\begin{multline*} %\label{OconI_2}
\widetilde I_{21} \le
\fr{1{,}0253}{\pi}\times{}\\
{}\times
\int\limits_{t_0}^{t_1(\delta)}
\exp\left\{-\fr{T^2t^2}{2}\left(1-2\left(2\pi\right)^\delta \varkappa(\delta)t^\delta\right)\right\}
\,\fr{dt}t\equiv{}\\
{}\equiv
 \fr{I_{33}(T,t_0,\delta)}{T^2}\,,
\end{multline*}
\begin{multline*}%\label{OconI_3}
\widetilde I_{22} \le2\int\limits_{t_1(\delta)}^1|K(t)|\,
\exp\left\{-T^2\fr{1-\cos 2\pi t}{4\pi^2}\right\} \,dt\equiv{}\\
{}\equiv
\fr{I_{32}(T,\delta)}{T^2}\,.
\end{multline*}

Величину $\widetilde I_1$ оценим при помощи
леммы~7, согласно которой в
обозначениях~\eqref{eps'_eps''_lambda} справедлива оценка
\begin{multline*}
\left|f_n(t)- e^{-t^2/2}\right|
\le \left(\gamma(\delta)\varepsilon |t|^{2+\delta}+
\fr{t^4}{8n}\right)\times{}\\
{}\times
 \exp\left\{-\fr{\lambda^2t^2}2 \left(1-
2\varkappa(\delta)(\varepsilon+\varepsilon')|t|^{\delta} \right)\right\}\,,\enskip
t\in\R\,,
\end{multline*}
откуда с учетом~\eqref{choice T_t1} и соотношения
$1/n=(\varepsilon')^{2/\delta}$ получаем
\begin{multline*}
\widetilde I_1\le \int\limits_0^{t_0}\left(2\gamma(\delta)\varepsilon(Tt)^{2+\delta}+
\fr{1}{4}\left(\eps'\right)^{2/\delta}(Tt)^4\right)\times{}\\
{}\times  \exp\left\{-\fr{(\lambda
Tt)^2}2 \left(1-2(2\pi)^\delta\varkappa(\delta)t^{\delta} \right)\right\}\,dt
\equiv{}\\
{}\equiv I_1(\lambda T,t_0,\delta)\fr{\varepsilon}{\lambda^{2+\delta}}+{}\\
{}+I_{2}(\lambda T,t_0,\delta)\fr{(\eps')^{2/\delta}}{\lambda^4}\,.
\end{multline*}
Можно убедиться, что введенные функции $I_k(v,t_0,\delta)$, $k=1,2,3$,
непрерывны и имеют конечные пределы при $v\to\infty$ и $v\to 0$ для
всех рассматриваемых значений~$t_0$ и~$\delta$, а потому являются
ограниченными при всех $v>0$.~$\square$

\section{Асимптотические оценки} \label{asympt_estims}

\noindent
\textbf{Теорема 2.}
\textit{Для асимптотически правильных постоянных}
\begin{multline*}
C_*(\delta)\equiv \limsup_{\ell\to0+}\limsup_{n\to\infty}
\sup_{F\in\F_{2+\delta}\colon L_n^{2+\delta}=\ell}
\fr{\rho(F_n,\Phi)}{L_n^{2+\delta}}\,,\\
 0<\delta \le 1\,;
\end{multline*}

\vspace*{-12pt}

\noindent
\begin{multline*}
C^*(\delta)\equiv \limsup_{\ell\to0+}\sup_{n,\,F\in\F_{2+\delta}\colon
L_n^{2+\delta}=\ell}\fr{ \rho(F_n,\Phi)}{L_n^{2+\delta}}\,,\\
 0<\delta\le1\,,
\end{multline*}
\textit{справедливы оценки}

\noindent
\begin{multline*}
C_*(\delta)\,\le C^*(\delta) \le
2^{\delta/2}\gamma(\delta)\fr{\Gamma\left((2+\delta)/2\right)}{\pi}\equiv{}\\
{}\equiv 
\overline{C}(\delta)\,,\\
 0<\delta < 1\,;
\end{multline*}

\vspace*{-12pt}

\noindent
\begin{multline*}
C_*(1) \le \lim_{\delta\to1-}\overline{C}(\delta)+\fr{1}{2\sqrt{2\pi}}=
\fr{1}{\sqrt{6\pi}}+{}\\
{}+
\fr{1}{2\sqrt{2\pi}}
=\fr{2}{3\sqrt{2\pi}}<0{,}2660\,;
\end{multline*}

\vspace*{-12pt}

\noindent
\begin{multline*}
C^*(1) \le \lim_{\delta\to 1-}\overline{C}(\delta)+{}\\
{}+
\fr{1}{\sqrt{2\pi}}=
\fr{1}{\sqrt{6\pi}}+\fr{1}{\sqrt{2\pi}} =
\fr{7}{6\sqrt{2\pi}}<0{,}4655.
\end{multline*}
\textit{Здесь $\Gamma(\cdot)$ обозначает гамма-функцию Эйлера.}

\smallskip

Значения величины $\overline{C}(\delta)$ при некоторых~$\delta$ приведены в
табл.~5.

\vspace*{12pt}
\noindent %tabl1
\begin{center}
\parbox{38mm}{{\tablename~5}\ \ \small{Значения величины
$\overline{C}(\delta)$ для некоторых~$\delta$}}
\end{center}
\vspace*{2pt}

{\small
\begin{center}
\tabcolsep=16pt
\begin{tabular}{|c|c|}
\hline &\\[-9pt]
 $\delta$ & $\overline{C}(\delta)$ \\[2pt]
\hline
&\\[-8pt]
 0+    &  0,1692  \\
 0,05  &  0,1561  \\
 0,10  &  0,1444  \\
 0,15  &  0,1339  \\
 0,20  &  0,1245  \\
 0,25  &  0,1161  \\
 0,30  &  0,1085  \\
 0,35  &  0,1017  \\
 0,40  &  0,0956  \\
 0,45  &  0,0902  \\
 0,50  &  0,0854  \\
 0,55  &  0,0810  \\
 0,60  &  0,0772  \\
 0,65  &  0,0738  \\
 0,70  &  0,0709  \\
 0,75  &  0,0685  \\
 0,80  &  0,0665  \\
 0,85  &  0,0650  \\
 0,90  &  0,0642  \\
 0,95  &  0,0642  \\
 1$-$  &  0,0665  \\
\hline
\end{tabular}
\end{center}
}
%\end{table}
\vspace*{6pt}


\bigskip
\addtocounter{table}{1}

\noindent
Д\,о\,к\,а\,з\,а\,т\,е\,л\,ь\,с\,т\,в\,о\,.\
Из теоремы~1 вытекает, что для величины
\begin{multline*}
R_\delta(n,\ell)=\sup_{F\in\F_{2+\delta}\colon L_n^{2+\delta}=\ell}
\fr{\rho(F_n,\Phi)}{L_n^{2+\delta}}\,,\\
 0<\delta\le1\,,\ n\ge\ell^{-2/\delta}\,,\
\ell >0\,,
\end{multline*}
справедливы оценки
\begin{multline*}
\!\!\!\!R_\delta(n,\ell)\le
I_1\left(\fr{2\pi}{(\ell+n^{-\delta/2})^{1/\delta}},t_0,\delta\right)\!
\left(\fr{n}{n-1}\right)^{1+\delta/2}\! +{}\\
{}+C_1\ell^{1/\delta-1}\,,\enskip 0<\delta<1\,,
\end{multline*}
\begin{multline*}
R_1(n,\ell)\le I_1\left(\fr{2\pi}{\ell+n^{-1/2}},t_0,1\right)
\left(\fr{n}{n-1}\right)^{3/2} +{}\\
{}+
\fr{1}{2\sqrt{2\pi}}\left(1+\fr{1}{\ell\sqrt{n}}\right)+ C_2\ell\,,
\end{multline*}
где $I_1$ определено в формулировке указанной тео\-ре\-мы, $t_0$~---
произвольное число на интервале $(0,\theta_0(\delta)/(2\pi))$, $C_1$
и~$C_2$ не зависят от~$\ell$ и~$n$. Обозначим

\vspace*{-2pt}

\noindent
$$
v_\ell=\fr{2\pi}{(2\ell)^{1/\delta}}\,.
$$
Тогда для всех $n\ge\ell^{-2/\delta}$ имеет место неравенство
\begin{multline*}
I_1\left(\fr{2\pi}{(\ell+n^{-\delta/2})^{1/\delta}},t_0,\delta\right)\le{}\\
{}\le
\sup_{v\ge v_\ell}I_1(v,t_0,\delta)\equiv J_\ell(t_0,\delta)\,,
\end{multline*}
с учетом которого для достаточно малых~$\ell$ получаем оценки

\noindent
\begin{align*}
\limsup_{n\to\infty}R_\delta(n,\ell)&\le J_\ell(t_0,\delta)
+C_1\ell^{1/\delta-1}\,,\quad 0<\delta<1\,;\\
\limsup_{n\to\infty}R_1(n,\ell)&\le J_\ell(t_0,1)
+\fr{1}{2\sqrt{2\pi}}+ C_2\ell\,,
\end{align*}
а также
\begin{multline*}
\sup_{n\ge\ell^{-2/\delta}}R_\delta(n,\ell)\le
J_\ell(t_0,\delta)\left(1-\ell^{2/\delta}\right)^{-(1+\delta/2)}+{}\\
{}+
C_1\ell^{1/\delta-1}\,,\quad 0<\delta<1\,,
\end{multline*}
$$
\sup_{n\ge\ell^{-2}}R_1(n,\ell)\le
J_\ell(t_0,1)\left(1-\ell^2\right)^{-3/2} +\fr{1}{\sqrt{2\pi}}+
C_2\ell\,.
$$
Теперь для доказательства теоремы достаточно показать, что для всех
$t_0>0$
$$
\limsup_{\ell\to0}J_\ell(t_0,\delta)=\overline C(\delta)\,.
$$
Действительно,
\begin{multline*}
\limsup_{\ell\to 0}J_\ell(t_0,\delta)= \limsup_{v\to\infty}
I_1(v,t_0,\delta)={}
\\
{}
=2\gamma(\delta)\limsup_{v\to\infty}v^{2+\delta}\int\limits_0^{t_0}|K(t)|\times{}\\
{}\times
\exp\left\{-\fr{v^2t^2}2 \left(1-2(2\pi)^\delta\varkappa(\delta)t^{\delta}
\right)\right\} t^{2+\delta}\,dt={}\\
{}
=2\gamma(\delta)\limsup_{v\to\infty}\int\limits_0^{vt_0}
\fr{t}{v}\left|K\left(\fr{t}{v}\right)\right| \times{}\\
{}\times
\exp\left\{-\fr{t^2}2
\left(1-2(2\pi)^\delta\varkappa(\delta)\left(\fr{t}{v}\right)^{\delta}
\right)\right\} t^{1+\delta}\,dt\,,
\end{multline*}
откуда в силу соотношения
\columnbreak

\noindent
$$
\limsup_{v\to\infty}
\fr{t}{v}\left|K\left(\fr{t}{v}\right)\right|=\fr{1}{2\pi}\,,
$$
справедливого для всех $t>0$, получаем
\begin{multline*}
\limsup_{\ell\to0}J_\ell(t_0,\delta)= \fr{\gamma(\delta)}\pi
\int\limits_0^{\infty}t^{1+\delta}e^{-{t^2}/2}\,dt={}\\
{}=
\fr{\gamma(\delta)}\pi 2^{\delta/2}\Gamma\left(\fr{2+\delta}{2}\right)\equiv
\overline{C}(\delta)\,,
\end{multline*}
что и требовалось доказать.~$\square$

%\smallskip

%\vspace*{-24pt}

\section{Вычислительная часть}

В данном разделе для введенных в теореме~1
функций построены некоторые мажоранты, позволяющие существенно
упростить процесс вычисления коэффициентов, фигурирующих в оценке
величины~$R_n$.

Обозначим через~$\mathcal D$ класс вещественных непрерывных
неотрицательных функций~$J(z)$, определенных для $z\ge0,$ которые
имеют единственный максимум и для $z>0$ не имеют минимума.

Краткое доказательство следующего утверждения содержится в работе
Правитца~\cite{Prawitz1975}, ниже приводится его полная версия.

\bigskip

\noindent
\textbf{Лемма 8} (см.\ также~\cite{Prawitz1975}). %\label{LemKlassD}
\textit{Пусть $a<b$ и $k>0$~--- произвольные постоянные, $g(s)$ и~$G(s)$~---
положительные монотонно возрастающие дифференцируемые функции на
$a\le s\le b$. Если
$$
\fr{G(s)-G(a)}{(g(s))^k}\,,\quad a\le s\le b\,,
$$
монотонно возрастает, то функция
$$
J(z)=z^k\int\limits_a^b{e^{-zg(s)}\,dG(s)}\,,\quad z\ge0\,,
$$
принадлежит классу $\mathcal D$.}

\bigskip

\noindent
Д\,о\,к\,а\,з\,а\,т\,е\,л\,ь\,с\,т\,в\,о\,.
Вычислим производную~$J(z)$:
\begin{multline*}
J'(z)=kz^{k-1}\int\limits_a^b{e^{-zg(s)}\,dG(s)}-{}\\
{}-
z^k\int\limits_a^b{e^{-zg(s)}g(s)\,dG(s)}={}
\end{multline*}
\begin{multline*}
{}=z^{k-1}e^{-zg(b)}\left[k\int\limits_a^be^{z(g(b)-g(s))}\,dG(s)-{}\right.\\
\left.{}-
z\int\limits_a^b e^{z(g(b)-g(s))}g(s)\,dG(s)\right]\,.
\end{multline*}
%\pagebreak
Домножая на $z^{1-k}e^{zg(b)},$ получаем
\begin{multline}
J'(z)z^{1-k}e^{zg(b)}=
k\int\limits_a^b{e^{z(g(b)-g(s))}\,dG(s)}-{}\\
{}-z\int\limits_a^b e^{z(g(b)-g(s))}g(s)\,dG(s)\,.
\label{LevChast}
\end{multline}

Рассмотрим функцию
\begin{equation*}
I(z)= z\!\int\limits_a^b \!{e^{z(g(b)-g(s))}g(s)^{k+1}\,
d\fr{G(s)-G(a)}{g(s)^k}}\,,\enskip z\ge0\,.
\end{equation*}
Преобразовав дифференциал, интегрированием по частям получаем
\begin{multline*}
I(z)=z\int\limits_a^b{e^{z(g(b)-g(s))}g(s)\,dG(s)}-{}\\
{}-
kz\int\limits_a^b e^{z(g(b)-g(s))}(G(s)-G(a))\,dg(s)={}\\
{}=z\int\limits_a^b{e^{z(g(b)-g(s))}g(s)\,dG(s)}+{}\\
{}+
k\int\limits_a^b{(G(s)-G(a))\,de^{z(g(b)-g(s))}}={}\\
{}=z\int\limits_a^b{e^{z(g(b)-g(s))}g(s)\,dG(s)}+{}\\
{}+
k(G(b)-G(a))-k\int\limits_a^b{e^{z(g(b)-g(s))}dG(s)}\,,
\end{multline*}
откуда в силу~\eqref{LevChast} вытекает представление
$$
J'(z)z^{1-k}e^{zg(b)}= k(G(b)-G(a))-I(z)\,,
$$
или
\begin{multline}
J'(z)z^{1-k}e^{zg(b)} =k(G(b)-G(a))-{}\\
{}-
z\int\limits_a^b{e^{z(g(b)-g(s))}g(s)^{k+1}\,
d\fr{G(s)-G(a)}{g(s)^k}}\,.
\label{Sootnowenie}
\end{multline}
По условию леммы
\begin{gather*}
g(b)-g(s)>0\,;\\
\fr{d}{ds}\left(\fr{G(s)-G(a)}{g(s)^k}\right)>0\,;\\ 
%\quad\quad\quad\quad 
a< s< b\,,
\end{gather*}
поэтому правая часть~\eqref{Sootnowenie} положительна при малых~$z$,
монотонно убывает по~$z$ и не ограничена снизу при $z\to\infty$.
Поскольку знаки~$J'(z)$ и правой части~\eqref{Sootnowenie}
совпадают, из сказанного выше заключаем, что~$J'(z)$ положительна
при малых~$z$ и имеет единственный нуль на $(0,\,\infty)$, а это, в
свою очередь, означает, что $J(z)\in \mathcal D.$

\smallskip

Следующее следствие леммы~8 сформулировано в работе
Правитца~\cite{Prawitz1975}, ниже приводится его полное
доказательство.


\smallskip

\noindent
\textbf{Следствие 2.} %Ъ %\label{SlKlassD}
\textit{Утверждение леммы~8 остается справедливым, если
условие возрастания функции
$$
\varphi(s)=\fr{G(s)-G(a)}{g(s)^k}\,,\quad a\le s\le b\,,
$$
заменить на условие возрастания функции
$$
\psi(s)=
%\label{Mest2}
\fr{G'(s)}{\left(g(s)^k\right)'}\,,\quad a\le
s\le b\,,
$$
а также потребовать, чтобы $G(a)=g(a)=0$.}

\smallskip

\noindent
Д\,о\,к\,а\,з\,а\,т\,е\,л\,ь\,с\,т\,в\,о\,.\
Покажем, что если~$\psi(s)$ возрастает на $[a,\,b]$, то~$\varphi(s)$
тоже возрастает на $[a,\,b]$. Вычислим производную~$\varphi(s)$:
$$
\varphi'(s)=\left(\fr{G(s)}{g(s)^k}\right)'=
\fr{(g(s)^k)'}{g(s)^k}\left(\fr{G'(s)}{(g(s)^k)'}-
\fr{G(s)}{g(s)^k}\right)\,.
$$
Применяя формулу Коши к функциям~$G(s)$ и~$g(s)^k$, заключаем, что
для любого $s>a$ найдется такая точка $s_0\in(a,\,s)$, что
$$
\fr{G(s)}{g(s)^k}=
\fr{G(s)-G(a)}{g(s)^k-g(a)^k}=\fr{G'(s_0)}{(g(s_0)^k)'}<
\fr{G'(s)}{(g(s)^k)'}\,,
$$
поскольку ${G'(s)/(g(s)^k)'}$ по условию возрастает. Отсюда
вытекает, что $\varphi'(s)>0$ при $a<s<b$, а следовательно,
$\varphi(s)$ возрастает на $[a,\,b]$.

\smallskip

Рассмотрим теперь функции~$I_k$, $k=1, 2, 3$, и построим для них
мажоранты специального вида.

Обозначим
$
H=H(\delta)=2(2\pi)^\delta\widetilde{\gamma}(\delta)$, $0<\delta\le2\pi$.
Из леммы~2 для интегралов~$I_1$ и~$I_2$ вытекают оценки

\noindent
\begin{multline*}
I_2(v,t_0,\delta)\le{}\\
{}\le
 \fr{1{,}0253}{8\pi}v^4\int\limits_0^{t_0}
\exp\left\{-\fr{v^2t^2}2 \left(1-Ht^{\delta} \right)\right\} t^{3}\,dt={}\\
{}=
 \fr{1{,}0253}{8\pi} J(v,t_0,\delta,2)\,,
\end{multline*}
$$
I_1(v,t_0,\delta)\le \fr{1{,}0253}{\pi}\,\gamma(\delta)
J\left(v,t_0,\delta,1+\fr{\delta}{2}\right)\,,
$$
где

\noindent
\begin{multline*}
J(v,t_0,\delta,k)={}\\
{}= \fr{v^{2k}}{2k}\int\limits_0^{t_0}
\exp\left\{-\fr{v^2t^2}{2}\left(1-H(\delta)t^{\delta}\right) \right\}
\,dt^{2k}\,,\\
 k>0\,,\quad v>0\,.
\end{multline*}

Согласно лемме~8 $J\in\mathcal D$, если
$t^2\left(1-H(\delta)t^{\delta}\right)$ является возрастающей функцией на $0\le t\le
t_0$. Последнее условие выполнено, когда для всех $0\le t\le t_0$
$$
\left(t^2(1-H(\delta)t^{\delta})\right)'_t= 2t-(2+\delta)H(\delta)t^{1+\delta}\ge0\,,
$$
т.\,е.\

\noindent
\begin{multline*}
t_0\le \left(\fr{2}{(2+\delta)H(\delta)}\right)^{1/\delta}
={}\\
{}=
\fr{\left((2+\delta)\widetilde \gamma(\delta)
\right)^{-1/\delta}}{2\pi}\equiv t_{\max}(\delta)\,.
\end{multline*}

Рассмотрим $I_{31}(T,t_0).$ Функция
$$
g(t)=\fr{1}{t}\left(\fr{1}{\pi t}-1+t-\fr{\pi^2t^2}{18}\right)\,,\quad
t>0\,,
$$
монотонно убывает при $t>0$, так как
\begin{multline*}
\sup_{t>0}g'(t)=\sup_{t>0}\left( -\fr{2}{\pi t^3} + \fr{1}{t^2}-
\fr{\pi^2}{18} \right)={}\\
{}= \left( -\fr{2}{\pi t^3} + \fr{1}{t^2}-
\fr{\pi^2}{18} \right)\bigg|_{t=3/\pi}
-\fr{\pi^2}{54}<0\,,
\end{multline*}
поэтому
\begin{multline*}
I_{31}(T,t_0)\le{}\\
{}\le
 \!\fr{1}{t_0}\left(\fr{1}{\pi
t_0}-1+t_0-\fr{\pi^2t_0^2}{18}\right) T^2\int\limits_{t_0}^\infty t e^{-T^2t^2/2}\,dt={}\\
\!{}= \fr{1}{t_0}\!\left(\fr{1}{\pi t_0}-1+ t_0-\fr{\pi^2t_0^2}{18}\right)e^{-T^2t^2/2}\equiv
J_{31}(T,t_0)\,.
\end{multline*}
Функция $J_{31}$, очевидно, принадлежит классу~$\mathcal D$.
\columnbreak

Рассмотрим $I_{32}(T,\delta).$ Сделаем замену переменных:
\begin{multline*}
I_{32}(T,\delta)={}\\
{}=
2T^2\!\int\limits_0^{1-t_1(\delta)}{|K(1-t)|\,\exp\left(-T^2\fr{1-\cos 2\pi
t}{4\pi^2}\right)}\,dt\,.
\end{multline*}
Положим
\begin{multline*}
g(t)=\fr{1-\cos 2\pi t}{4\pi^2}\,;\quad dG(t)=\left|K(1-t)\right|\,,\\
 0\le t\le1-t_1\,.
\end{multline*}
На отрезке $0\le t\le1-t_1$ функции~$g(t)$ и
\begin{multline*}
\fr{G'(t)}{\left(g(t)\right)'}={}\\
{}= \fr{4\pi^2|K(1-t)|}{(1-\cos 2\pi
t)'}=\fr {\pi t}{\sin 2\pi s}\sqrt{1+\left(\cot \pi t-\fr{1}{\pi
t}\right)^2}
\end{multline*}
являются возрастающими, $g(0)=0,$ поэтому согласно
следствию~2 $I_{32}\in \mathcal D.$

Рассмотрим $I_{33}(T,t_0,\delta)$. Можно убедиться, что функция
${t^2(1-H(\delta)t^{\delta})}$ положительна при ${t\in(0,t_1(\delta)]}$ и имеет
единственный максимум в точке ${t=t_{\max}(\delta)\in(0,t_1(\delta))}$.
Отсюда вытекает, что существует единственный корень
$$
t_2=t_2(\delta)\in(0,t_{\max}(\delta))
$$
уравнения
$$
t^2(1-H(\delta)t^{\delta})=t_1^2(1-H(\delta)t_1^{\delta})\,,\quad 0<t< t_1(\delta)\,.
$$
Таким образом, на интервале $[t_2,t_1]$ имеем
$$
t^2\left(1-H\left(\delta\right)t^{\delta}\right)\ge t_1^2\left(1-H\left(\delta\right)t_1^{\delta}\right)\,.
$$

Будем считать, что $t_0<t_2$, и разобьем область интегрирования в
$I_{33}(T,t_0,\delta)$ на две части: $(t_0,t_2)$ и~$(t_2,t_1)$. Тогда
получим
$$
I_{33}(T,t_0,\delta)\le J_{33}(T,t_0,\delta)+J_{34}(T,t_0)\,,
$$
где
\begin{multline*}
J_{33}(T,t_0,\delta)={}\\
{}=
 \fr{1{,}0253}{\pi}\,T^2\int\limits_{t_0}^{t_2(\delta)}
{\exp\left\{-\fr{T^2t^2}2\left(1-H(\delta)t^{\delta}\right)\right\}}
\fr{dt}{t}\,;
\end{multline*}
\begin{multline*}
J_{34}(T,\delta)= \fr{1{,}0253}{\pi}\,T^2\times{}\\
{}\times
\exp\left\{-\fr{T^2t_1^2(\delta)}2\left(1-H(\delta)t_1^{\delta}(\delta)\right)
\right\} \ln\fr{t_1(\delta)}{t_2(\delta)}\,.
\end{multline*}
Функция $J_{34}$, очевидно, принадлежит классу~$\mathcal D$.
\pagebreak

Рассмотрим $J_{33}(T)=J_{33}(T,t_0,\delta)$. На отрезке $[t_0,t_2]$
функция $t^2(1-Ht^{\delta})$ возрастает, поэтому согласно
лемме~8 для того, чтобы $J_{33}\in \mathcal D$,
достаточно потребовать, чтобы функция
$\left(\ln t-\ln t_0\right)\!/\!\left(t^2(1-Ht^{\delta})\right)$
возрастала на $[t_0,t_2],$ а это эквивалентно следующему
неравенству:
\begin{multline*}
\left(\fr {\ln t-\ln t_0}{t^2(1-Ht^{\delta})}\right)'={}\\
{}=
\fr{t(1-Ht^{\delta})-(\ln t-\ln
t_0)(2t-(2+\delta)Ht^{1+\delta})}{t^4(1-Ht^{\delta})^2}\ge{}\\
{}\ge 0\,,\quad t_0\le t\le
t_2\,.
\end{multline*}
Последнее условие выполняется, если~$t_0$ удовлетворяет соотношению
$$
\ln t_0\ge \max\limits_{t\in[t_0,\,t_2]}M(t)\,,
$$
где
$$
M(t)=\ln t-\fr{1-Ht^{\delta}}{2-(2+\delta)Ht^{\delta}}\,.
$$
Максимум функции~$M(t)$ достигается в точке
$$
t_M=\left(\fr{9+4\delta -\sqrt{17+8\delta}}{2(2+\delta)^2}\right)^{1/\delta}\,;
$$
следовательно, для $t_0\ge \exp\left\{M(t_M)\right\}$ имеем
$J_{33}\in$\linebreak $\in \mathcal D$.

Таким образом построена мажоранта для~$R_n$, допускающая
представление, в котором фигурируют функции из~$\mathcal D$. Оформим
это утверждение в виде теоремы.

\smallskip

\noindent
\textbf{Теорема 3}.
\textit{Пусть
\begin{gather*}
T=\fr{2\pi}{(\eps+\eps')^{1/\delta}}\,;\\
t_1(\delta)=
\fr{\theta_0(\delta)}{2\pi}\,;\quad
H(\delta)=2(2\pi)^{\delta}\widetilde{\gamma}(\delta)\,,
\end{gather*}
$t_2(\delta)$~--- положительный, отличный от $t_1(\delta)$ корень уравнения
\begin{align*}
t^2-H(\delta)t^{2+\delta}&=t_1^2-H(\delta)t_1^{2+\delta}\,;
\\
t_*(\delta)&=
t_M\exp\left\{-\fr{1-Ht_M^{\delta}}{2-(2+\delta)Ht_M^{\delta}}\right\}\,;\\
t_M&=\left(\fr{9+4\delta -\sqrt{17+8\delta}}{2(2+\delta)^2}\right)^{1/\delta}\,.
\end{align*}
Тогда для всех $n\ge2$ и ${t_*(\delta)\le t_0\le t_2(\delta)}$
справедлива оценка}

\noindent
\begin{multline*}
R_n\le \fr{1{,}0253\gamma(\delta)}{\pi} J_1(\lambda T,t_0,\delta)
\fr{\eps}{\lambda^{2+\delta}} +
\fr{(\eps+\eps')^{1/\delta}}{2\sqrt{2\pi}}+{}\\
+ \fr{1{,}0253}{8\pi}J_2(\lambda T,t_0,\delta)
\fr{(\eps')^{2/\delta}}{\lambda^4}+ J_3(T,t_0,\delta) (\eps+\eps')^{2/\delta}+{}\\
{}+
\fr{(\eps+\eps')^{3/\delta}}{144\sqrt{2\pi}}\,,
\end{multline*}
\textit{где}
\begin{gather*}
J_1(v,t_0,\delta)=J\left(v,t_0,\delta,1+\fr{\delta}{2}\right)\,,\\
 J_2(v,t_0,\delta)= J(v,t_0,\delta,2)\,,
\end{gather*}

\vspace*{-12pt}

\noindent
\begin{multline*}
J(v,t_0,\delta,k)={}\\
{}= v^{2k}\int\limits_0^{t_0} t^{2k-1}
\exp\left\{-\fr{v^2t^2}2\left(1-H(\delta)t^{\delta}\right) \right\} \,dt\,;
\end{multline*}

\vspace*{-12pt}

\noindent
\begin{multline*}
J_3(T,t_0,\delta)=
\left (J_{31}(T,t_0,\delta)+
J_{32}(T,\delta)+{}\right.\\
\left.{}+J_{33}(T,t_0)+J_{34}(T,t_0)-1\right )\Big/
(4\pi^2)\,;
\end{multline*}

\vspace*{-12pt}

\noindent
$$
J_{31}(T,t_0)=\fr{1}{t_0}\left(\fr{1}{\pi
t_0}-1+t_0-\fr{\pi^2t_0^2}{18}\right)e^{-T^2t^2/2}\,;
$$

\vspace*{-12pt}

\noindent
\begin{multline*}
J_{32}(T,\delta)=  I_{32}(T,\delta)= {}\\
{}=
 T^2\int\limits_0^{1-t_1(\delta)}
\sqrt{1+\left(\cot \pi t-\fr{1}{\pi t}\right)^2}\times{}\\
{}\times
\exp\left(-T^2\fr{1-\cos 2\pi t}{4\pi^2}\right)t\,dt\,;
\end{multline*}

\vspace*{-12pt}

\noindent
\begin{multline*}
J_{33}(T,t_0,\delta)={}\\
{}= \fr{1{,}0253}{\pi}\,T^2\int\limits_{t_0}^{t_2(\delta)}
{\exp\left\{-\fr{T^2t^2}2\left(1-H(\delta)t^{\delta}\right)\right\}}
\fr{dt}{t}\,;
\end{multline*}

\vspace*{-12pt}

\noindent
\begin{multline*}
J_{34}(T,\delta)= \fr{1{,}0253}{\pi}\,T^2\times{}\\
{}\times
\exp\left\{-\fr{T^2t_1^2(\delta)}2\left(1-H(\delta)t_1^{\delta}(\delta)\right)
\right\} \ln\fr{t_1(\delta)}{t_2(\delta)}\,,
\end{multline*}
\textit{причем функции $J_1(v,t_0,\delta)$,
$J_2(v,t_0,\delta)$, $J_{31}(v,t_0)$,
$J_{32}(v,\delta)$, $J_{33}(v,t_0,\delta)$ и
$J_{34}(v,\delta)$, рассматриваемые
как функции аргумента~$v$, принадлежат классу~$\mathcal D$ для
указанных $t_0$ и всех ${0<\delta\le 1}$.}

\setcounter{table}{6}
\begin{table*}[b]\small %tabl7
\begin{center}
\Caption{Значения величин $C_1(\ell,\delta,t_0)$ и $C_2(\ell,\delta,t_0)$ для~$t_0$,
доставляющих минимум~$C_1$, и некоторых~$\ell$ и~$\delta$, округленные
в б$\acute{\mbox{о}}$льшую сторону с точностью до четвертого знака}
\vspace*{2ex}

\tabcolsep=14.4pt
\begin{tabular}{|c|c|c|c|c|c|c|c|c|}
\hline
 & \multicolumn {8}{c|}{$\ell$}\\
 \cline{2-9}
&\multicolumn{2}{c|}{0,4}& \multicolumn{2}{c|}{0,3}& \multicolumn{2}{c|}{0,2}& \multicolumn{2}{c|}{0,1}\\
\cline{2-9}
\multicolumn{1}{|c|}{\raisebox{10pt}[0pt][0pt]{$\delta$}}& $C_1$&$C_2$&$ C_1$&$C_2$&$ C_1$&$C_2$&$ C_1$&$C_2$\\
\hline
0,1 &   0,6213  &   0,0886  &   0,1500  &   0,0021  &   0,1481  &    0,0001 &   0,1480  &   0,0001  \\
0,2 &   0,6425  &   0,2805  &   0,2915  &   0,0327  &   0,1327  &    0,0052 &   0,1279  &   0,0004  \\
0,3 &   0,5577  &   0,2950  &   0,2776  &   0,0714  &   0,1916  &    0,0236 &   0,1159  &   0,0047  \\
0,4 &   0,4862  &   0,2861  &   0,2765  &   0,1044  &   0,1954  &    0,0508 &   0,1159  &   0,0179  \\
0,5 &   0,4333  &   0,2717  &   0,2800  &   0,1382  &   0,2068  &    0,0812 &   0,1274  &   0,0400  \\
0,6 &   0,3987  &   0,2705  &   0,2852  &   0,1655  &   0,2209  &    0,1124 &   0,1608  &   0,0688  \\
0,7 &   0,3766  &   0,2772  &   0,2912  &   0,1921  &   0,2361  &    0,1436 &   0,1848  &   0,1016  \\
0,8 &   0,3642  &   0,2869  &   0,2987  &   0,2166  &   0,2536  &    0,1735 &   0,2128  &   0,1368  \\
0,9 &   0,3601  &   0,3010  &   0,3085  &   0,2399  &   0,2725  &    0,2013 &   0,2435  &   0,1731  \\
1,0 &   0,3688  &   0,3174  &   0,3254  &   0,2624  &   0,2963  &    0,2291 &   0,2794  &   0,2100  \\
 \hline
\end{tabular}
\end{center}
\end{table*}


\smallskip

Значения функций~$t_1(d)$, $t_2(\delta)$ и~$t_*(\delta)$ для некоторых~$\delta$ приведены в табл.~6.


Из теоремы~3 с использованием полезных свойств
фигурирующих в ней функций можно получать асимптотические оценки
коэффициентов при~$\eps$ и~$\eps'$. Будем рассматривать условия вида 
\begin{equation}
\eps+\eps'\le2\ell\,,
\label{eps+eps'<=2ell}
\end{equation}
%\pagebreak
%\linebreak\vspace*{-12pt}

%tabl6
\noindent %tabl6
\begin{center}
\parbox{74mm}{{\tablename~6}\ \ \small{Значения функций $t_1(d)$,
$t_2(\delta)$ и~$t_*(\delta)$ для некоторых~$\delta$}}
\end{center}
%\vspace*{2pt}

{\small
\begin{center}
\tabcolsep=18pt
\begin{tabular}{|c|c|c|c|}
\hline
$\delta$ & $t_*$ & $t_2$ & $t_1$ \\
\hline
 0+   & 0,00 & 0,00 & 1,00 \\
 0,01 & 0,01 & 0,03 & 1,00 \\
 0,05 & 0,01 & 0,07 & 0,98 \\
 0,10 & 0,01 & 0,11 & 0,96 \\
 0,15 & 0,02 & 0,14 & 0,94 \\
 0,20 & 0,04 & 0,16 & 0,92 \\
 0,25 & 0,06 & 0,18 & 0,90 \\
 0,30 & 0,07 & 0,20 & 0,88 \\
 0,35 & 0,09 & 0,22 & 0,86 \\
 0,40 & 0,10 & 0,24 & 0,84 \\
 0,45 & 0,11 & 0,26 & 0,83 \\
 0,50 & 0,12 & 0,28 & 0,81 \\
 0,55 & 0,13 & 0,29 & 0,79 \\
 0,60 & 0,14 & 0,31 & 0,78 \\
 0,65 & 0,14 & 0,33 & 0,76 \\
 0,70 & 0,15 & 0,34 & 0,74 \\
 0,75 & 0,15 & 0,36 & 0,73 \\
 0,80 & 0,16 & 0,37 & 0,71 \\
 0,85 & 0,16 & 0,38 & 0,69 \\
 0,90 & 0,16 & 0,40 & 0,68 \\
 0,95 & 0,16 & 0,41 & 0,66 \\
 1,00 & 0,16 & 0,42 & 0,64 \\
\hline
\end{tabular}
\end{center}
}
%\end{table}
\vspace*{2pt}


\bigskip
\addtocounter{table}{1}


\noindent
где $\ell$~--- некоторое число из интервала $(0,\,1)$. В силу соотношения
$\eps'\le\eps$ из~\eqref{eps+eps'<=2ell} вытекает, что
$\eps'\le\ell$. При ограничении~\eqref{eps+eps'<=2ell} имеем

\noindent
\begin{gather*}
T\ge \fr{2\pi}{(2\ell)^{1/\delta}}\,;\quad n=(\eps')^{-2/\delta}\ge
\ell^{-2/\delta}\,;\\
\lambda=\sqrt{1-\fr{1}{n}}\ge\sqrt{1-\ell^{2/\delta}}\,.
\end{gather*}

\smallskip

\noindent
\textbf{Теорема 4.}
\textit{Пусть $\ell$~--- произвольное число из интервала $(0,\,1)$. Обозначим}

\noindent
$$
T_\ell=\fr{2\pi}{(2\ell)^{1/\delta}}\,;\quad
v_\ell=T_\ell\sqrt{1-\ell^{2/\delta}}\,;
$$

\vspace*{-12pt}

\noindent
\begin{multline*}
A_1(\ell,\delta,t_0)={}\\
{}=
\fr{1{,}0253}{\pi}\,\gamma(\delta)
\left(1-\ell^{2/\delta}\right)^{-(1+\delta/2)} \sup_{v\ge
v_\ell}J_1(v,t_0,\delta)\,;
\end{multline*}
$$
A_2(\ell,\delta,t_0)=\fr{1{,}0253}{8\pi}
\left(1-\ell^{2/\delta}\right)^{-2}\sup_{v\ge v_\ell}J_2(v,t_0,\delta)\,;
$$
$$
A_3(\ell,t_0,\delta)=\sup_{v\ge T_\ell}J_3(v,t_0,\delta)\,.
$$
\textit{Тогда для всех ${t_*(\delta)\le t_0\le t_2(\delta)}$, ${n\ge2}$ и~$\bet$,
удовлетворяющих условию ${\eps+\eps'\le 2\ell}$, справедлива
оценка}
\begin{multline*}
\!R_n\le A_1(\ell,\delta,t_0)\eps+ 0{,}19948(\eps+\eps')^{1/\delta}+{}\\
{}+A_2(\ell,\delta,t_0)(\eps')^{2/\delta}+
A_3(\ell,t_0,\delta)(\eps+\eps')^{2/\delta}+{}\\
{}+ 0{,}00278(\eps+\eps')^{3/\delta}\,.
\end{multline*}

\smallskip

Вычисление супремумов~$J_k$, $k=1, 2, 3$, не представляет больших
затруднений, поскольку указанные функции имеют не более одного
максимума. Из теоремы~4 можно легко получить
линейную по~$\eps$ и~$\eps'$ оценку, используя вытекающие из
условия~\eqref{eps+eps'<=2ell} неравенства
$$
\eps'\le\ell\,,\quad
(\eps+\eps')^{1/\delta}\le(2\ell)^{1/\delta-1}(\eps+\eps')
$$
и подобные~им.

\setcounter{table}{8}
\begin{table*}[b]\small
\vspace*{-12pt}
\begin{center}
\parbox{150mm}{\Caption{Значения величины $C(\ell,\delta)$ для некоторых~$\ell$ и~$\delta$, округленные в
б$\acute{\mbox{о}}$льшую сторону с точностью до четвертого знака}

}

\vspace*{2ex}


\tabcolsep=19pt
\begin{tabular}{|c|c|c|c|c|c|c|}
\hline
 & \multicolumn {6}{c|}{$\ell$}\\
 \cline{2-7}
\multicolumn{1}{|c|}{\raisebox{4pt}[0pt][0pt]{$\delta$}}&0,4&0,3&0,2&0,1&0,05&0,01\\ 
\hline
0,05&0,7268&0,1601&0,1601&0,1601&0,1601&0,1601\\
0,10&0,6996&0,1520&0,1481&0,1480&0,1480&0,1480\\
0,20&0,9230&0,3241&0,1378&0,1283&0,1277&0,1276\\
0,30&0,8527&0,3453&0,2151&0,1206&0,1131&0,1113\\
0,40&0,7722&0,3795&0,2462&0,1338&0,1107&0,0992\\
0,50&0,7050&0,4157&0,2880&0,1674&0,1274&0,0955\\
0,60&0,6675&0,4498&0,3332&0,2295&0,1652&0,1085\\
0,70&0,6519&0,4823&0,3797&0,2864&0,2218&0,1474\\
0,80&0,6508&0,5144&0,4265&0,3495&0,2985&0,2183\\
0,90&0,6611&0,5477&0,4737&0,4166&0,3818&0,3244\\
1,00&0,6861&0,5875&0,5249&0,4894&0,4767&0,4680\\
\hline
\end{tabular}
\end{center}
\end{table*}

%\smallskip

%tabl8


\noindent
\textbf{Теорема 5.}
\textit{Пусть $\ell$~--- произвольное число из интервала $(0,\,1)$. Обозначим
\begin{multline*}
C_1(\ell,\delta,t_0)=A_1(\ell,\delta,t_0)+
\fr{(2\ell)^{1/\delta-1}}{2\sqrt{2\pi}}+{}\\
{}+
(2\ell)^{2/\delta-1}A_3(\ell,\delta,t_0) +
\fr{(2\ell)^{3/\delta-1}}{144\sqrt{2\pi}}\,;
\end{multline*}
\begin{multline*}
C_2(\ell,\delta,t_0)=\fr{(2\ell)^{1/\delta-1}}{2\sqrt{2\pi}}+
\ell^{2/\delta-1}A_2(\ell,\delta,t_0)+{}\\
{}+ (2\ell)^{2/\delta-1}A_3(\ell,\delta,t_0)
+ \fr{(2\ell)^{3/\delta-1}}{144\sqrt{2\pi}}\,.
\end{multline*}
Тогда для всех ${t_*(\delta)\le t_0\le t_2(\delta)}$, ${n\ge2}$
и~$\bet$, удовлетворяющих условию ${\eps+\eps'\le 2\ell}$, справедлива
оценка
$$
R_n\le C_1(\ell,\delta,t_0)\eps+ C_2(\ell,\delta,t_0)\eps'\,.
$$
}

\smallskip

Значения величин $C_1(\ell, \delta, t_0)$ и $C_2(\ell,\delta,t_0)$ для~$t_0$, доставляющих минимум~$C_1$,
и некоторых~$\ell$ и~$\delta$ приведены в табл.~7.


Ввиду особой важности случая $\delta=1$ мы приводим более подробную таблицу $C_1(\ell,1,t_0)$
и $C_2(\ell,1,t_0)$ для~$t_0$, доставляющих минимум~$C_1$, и некоторых~$\ell$ (табл.~8).


И, наконец, учитывая соотношение $\eps'\le\eps$, можно получить
оценку в классическом неравенстве Берри--Эссеена.

\noindent %tabl8
\begin{center}
\parbox{60mm}{{\tablename~8}\ \ \small{Значения величин $C_1(\ell,1,t_0)$ и $C_2(\ell,1,t_0)$ для~$t_0$, доставляющих минимум~$C_1$, и
некоторых~$\ell$, округленные в б$\acute{\mbox{о}}$льшую сторону с точностью до четвертого знака}}
\end{center}
%\vspace*{2pt}

{\small
\begin{center}
\tabcolsep=17pt
\begin{tabular}{|c|c|c|c|c|c|c|c|c|}
\hline
$\ell$&$C_1$&$C_2$\\
\hline
0,31  & 0,3292  & 0,2674  \\
0,32  & 0,3331  & 0,2718  \\
0,33  & 0,3373  & 0,2772  \\
0,34  & 0,3415  & 0,2821  \\
0,35  & 0,3459  & 0,2878  \\
0,36  & 0,3502  & 0,2931  \\
0,37  & 0,3548  & 0,2992  \\
0,38  & 0,3594  & 0,3049  \\
0,39  & 0,3640  & 0,3112  \\
 \hline
\end{tabular}
\end{center}
}
%\end{table}
%\vspace*{6pt}


%\bigskip
\addtocounter{table}{1}


%\smallskip
\noindent
\textbf{Теорема 6.}
\textit{Пусть $\ell$~--- произвольное число из интервала $(0,\,1)$. Обозначим
$$
C(\ell,\delta)=\inf_{t_*(\delta)\le t_0\le
t_2(\delta)}\left\{C_1(\ell,\delta,t_0)+ C_2(\ell,\delta,t_0)\right\}\,.
$$
Тогда для всех ${n\ge2}$ и~$\bet$, удовлетворяющих условию
${\eps\le \ell}$, справедлива оценка
$$
R_n\le C(\ell,\delta)\eps\,.
$$
}



%\smallskip

Значения величины $C(\ell,\delta)$ для некоторых~$\ell$ и~$\delta$
приведены в табл.~9.


Также для $\delta=1$ при $\ell=0{,}0985$ и $\ell=0{,}1387$ можно вычислить, что
$$
C(\ell,1)= \begin{cases}
0{,}4889,&\ell=0{,}0985\,;\\
0{,}5010,&\ell=0{,}1387\,.
\end{cases}
$$
%тем самым уточнив результат Правитца примерно на две сотых.
%Чтобы понять, за счет чего достигнуто это уточнение, необходимо заметить, что
%условия $\varepsilon<0{,}0985$ и $\varepsilon<0{,}1387$, как можно убедиться, являются более сильными при
%конечных~$n$, чем соответствующие условия Правитца: $\varepsilon_n\le0{,}1$ и
%$\varepsilon_n\le1/7$ (напомним, что $\eps_n=(1+1/n)^{3/2}\varepsilon$, $\varepsilon'_n=(1+1/n)^{3/2}\varepsilon'$
%при $\delta=1$).

\bigskip

Пользуясь случаем, авторы выражают искреннюю признательность
В.\,Ю.~Королёву за постоянное внимание к работе.

{\small\frenchspacing
{%\baselineskip=10.8pt
\addcontentsline{toc}{section}{Литература}
\begin{thebibliography}{99}

\bibitem{Matskjavichus1983} %1
\Au{Мацкявичюс В.\,К.} О нижней оценке скорости сходимости в
центральной предельной теореме~// Теория вероятн. и ее примен.,
1983. Т.~28. Вып.~3. С.~565--569.

\bibitem{Petrov1972} %2
\Au{Петров В.\,В.} Суммы независимых случайных величин.~--- М.:
Наука, 1972.

\bibitem{Berry1941} %3
\Au{Berry A.\,C.} The accuracy of the Gaussian approximation to the
sum of independent variates~// Trans. Amer. Math. Soc., 1941.
Vol.~49. P.~122--139.

\bibitem{Esseen1942} %4
\Au{Еsseen C.-G.} On the Liapunoff limit of error in the theory of
probability~// Ark. Mat. Astron. Fys., 1942. Vol.~A28. No.\,9.
P.~1--19.

\bibitem{Esseen1956} %5
\Au{Еsseen C.-G.} A moment inequality with an application to the
central limit theorem~// Skand. Aktuarietidskr., 1956. Vol.~39. P.~160--170.

\bibitem{Hsu1945} %6
\Au{Hsu P.\,L.} The approximate distributions of the mean and
variance of a sample of independent variables~// Ann. Math.
Statist., 1945. Vol.~16. No.\,1. P.~1--29.

\bibitem{Bergstrom1949} %7
\Au{Bergstr$\,\ddot{\!\mbox{o}}$m H.} On the central limit theorem in the case of
not equally distributed random variables~// Skand. Aktuarietidskr.,
1949. Vol.~33. P.~37--62.

\bibitem{Takano1951} %8
\Au{Takano K.} A remark to a result of A.\,C.~Berry~// Res. Mem.
Inst. Math., 1951. Vol.~9. No.\,6. P.~4.08--4.15.

\bibitem{Kolmogorov1953} %9
\Au{Колмогоров А.\,Н.} Некоторые работы последних лет в области
предельных теорем теории вероятностей~// Вестник Моск. ун-та, 1953.
№\,10. С.~29--38.

\bibitem{Wallace1958} %10
\Au{Wallace D.\,L.} Asymptotic approximations to distributions~//
Ann. Math. Statist., 1958. Vol.~29. P.~635--654.

\bibitem{Rogozin1960} %11
\Au{Рогозин Б.\,А.} Одно замечание к работе Эссеена <<Моментное
неравенство с применением к центральной предельной теореме>>~//
Теория вероятн. и ее примен., 1960. Т.~5. Вып.~1. С.~125--128.

\bibitem{Zolotarev1966} %12
\Au{Золотарёв В.\,М.}
Абсолютная оценка остаточного члена в центральной предельной теореме~// Теория вероятн. и ее примен.,
1966. Т.~11. Вып.~1. С.~108--119.

\bibitem{Zolotarev1967a} %13
\Au{Золотарёв В.\,М.} Некоторые неравенства теории вероятностей и их
применение к уточнению теоремы А.\,М.~Ляпунова // ДАН, 1967.
Т.~177. №\,3. С.~501--504.

\bibitem{Zolotarev1967b} %14
\Au{Zolotarev V.\,M.} A sharpening of the inequality of
Berry--Esseen~// Wahrsch. verw. Geb., 1967. Bd.~8. P.~332--342.

\bibitem{VanBeek1971} %15
\Au{Van~Beek P.}
Fourier-analytische Methoden zur Verscharfung der
Berry--Esseen Schranke. Doctoral dissertation. Friedrich Wilhelms
Universitat, Bonn, 1971.

\bibitem{VanBeek1972} %16
\Au{Van~Beek P.} An application of Fourier methods to the problem
of sharpening the Berry--Esseen inequality~// Z.~Wahrsch. verw.
Geb., 1972. Bd.~23. P.~187--196.

\bibitem{Prawitz1975} %17
\Au{Prawitz H.} On the remainder in the central limit theorem. Part~I.
One-dimensional independent variables with finite absolute moments of third order~//
Scand. Actuarial J., 1975. No.\,3. P.~145--156.

\bibitem{Shiganov1982} %18
\Au{Шиганов И.\,С.} Об уточнении верхней константы в остаточном
члене центральной предельной теоремы~// Проблемы устойчивости
стохастических моделей. Труды ВНИИСИ, 1982. С.~109--115.

\bibitem{Feller1967} %19
\Au{Феллер В.} Введение в теорию вероятностей и ее приложения.
Т.~2.~--- М.: Мир, 1984.

\bibitem{Bentkus1991} %20
\Au{Bentkus~V.} On the asymptotical behaviour of the constant in
the Berry--Esseen inequality. Preprint 91--078, Universit\"at
Bielefeld, 1991.

\bibitem{Bentkus1994} %21
\Au{Bentkus V.} On the asymptotical behaviour of the constant in
the Berry--Esseen inequality~// J.~Theoret. Probab., 1994. Vol.~7.
No.\,2. P.~211--224.

\bibitem{Chistyakov2001a} %22
\Au{Чистяков Г.\,П.} Новое асимптотическое разложение и
асимптотически наилучшие постоянные в теореме Ляпунова. I~// Теория
вероятн. и ее примен., 2001. Т.~46. Вып.~2. С.~326--344.

\bibitem{Chistyakov2001b} %23
\Au{Чистяков Г.\,П.} Новое асимптотическое разложение и
асимптотически наилучшие постоянные в теореме Ляпунова. II~// Теория
вероятн. и ее примен., 2001. Т.~46. Вып.~3. С.~573--579.

\bibitem{Chistyakov2002} %24
\Au{Чистяков Г.\,П.} Новое асимптотическое разложение и
асимптотически наилучшие постоянные в теореме Ляпунова. III~//
Теория вероятн. и ее примен., 2002. Т.~47. Вып.~3. С.~475--497.

\bibitem{NagaevChebotarev2006} %25
\Au{Нагаев С.\,В., Чеботарёв~В.\,И.}
Новый подход к оценке абсолютной константы в неравенстве Берри--Эссеена~// Тез. докл. XXXI
Дальневосточной школы-семинара им.\ акад.\ Е.\,В.~Золотова.~---
Владивосток, 2006. C.~19.

\bibitem{Shevtsova2006} %26
\Au{Шевцова И.\,Г.} Уточнение верхней оценки абсолютной постоянной в
неравенстве Берри--Эссеена~// Теория вероятн. и ее примен., 2006.
Т.~51. Вып.~3. С.~622--626.

\bibitem{Shevtsova2008} %27
\Au{Шевцова И.\,Г.} Уточнение абсолютной константы в классическом
неравенстве Берри--Эссеена~// Статистические методы оценивания и
проверки гипотез.~--- Пермь: Изд-во Пермского гос.
ун-та, 2008. Т.~21. С.~159--168.

\bibitem{Tysiak1983} %28
\Au{Tysiak W.} Gleichm{\"a}$\beta$ige und
nicht-gleichm{\"a}$\beta$ige Berry--Esseen--Absch{\"a}tzungen.
Dissertation. Wuppertal, 1983.

\bibitem{GaponovaKorchaginShevtsova2009} %29
\Au{Гапонова М.\,О., Корчагин А.\,Ю., Шевцова~И.\,Г.}
Об абсолютных
константах в равномерной оценке точности нормальной аппроксимации
для распределений, не имеющих третьего момента~// Сб. статей
молодых ученых факультета ВМК МГУ, М.: Макс Пресс, 2009. Вып.~6.
С.~81--89.

\bibitem{Shevtsova2009} %30
\Au{Шевцова И.\,Г.} Некоторые оценки для характеристических функций
с применением к уточнению неравенства Мизеса // Информатика и её применения, 2009.
Т.~3. Вып.~3. С.~69--78.

\label{end\stat}

\bibitem{Prawitz1972} %31
\Au{Prawitz H.} Limits for a distribution, if the characteristic
function is given in a finite domain~// Skand. Aktuarietidskr.,
1972. P.~138--154.

%\bibitem{Prawitz1973}
%\Au{Prawitz H.} Ungleichungen f\"{u}r den absoluten Betrag einer
%charakteristischen funktion~// Skand. Aktuarietidskr., 1973. No.\,1.
%P.~11--16.


%\bibitem{Ushakov1999}
%\Au{Ushakov N.\,G.} Selected Topics in Characteristic Functions.~---
%Utrecht: VSP, 1999.


 \end{thebibliography}
}
}
\end{multicols}