\renewcommand{\Re}{\mathrm{Re}\,}
\renewcommand{\Im}{\mathrm{Im}\,}

\def\stat{petr}


\def\tit{АСИМПТОТИЧЕСКИЙ АНАЛИЗ СИСТЕМЫ МАССОВОГО ОБСЛУЖИВАНИЯ
{\boldmath$E_r(t)\vert G \vert 1$}$^*$}
\def\titkol{Асимптотический анализ системы массового обслуживания
$E_r(t)\vert G \vert 1$ }

\def\autkol{О.\,В.~Петрова, В.\,Г.~Ушаков}
\def\aut{О.\,В.~Петрова$^1$, В.\,Г.~Ушаков$^2$}

\titel{\tit}{\aut}{\autkol}{\titkol}

{\renewcommand{\thefootnote}{\fnsymbol{footnote}}\footnotetext[1]
{Работа выполнена при финансовой поддержке РФФИ, грант~08-01-00567-а.}}

\renewcommand{\thefootnote}{\arabic{footnote}}
\footnotetext[1]{Факультет
вычислительной математики и кибернетики Московского государственного
университета им. М.~В.~Ломоносова, o.petrova@inbox.ru}
\footnotetext[2]{Факультет
вычислительной математики и кибернетики Московского государственного
университета им. М.~В.~Ломоносова; Институт проблем информатики Российской академии наук, 
ushakov@akado.ru}

\Abst{Рассмотрена одноканальная
система массового обслуживания (СМО) с эрланговским входящим потоком,
интенсивность которого зависит от времени. Длительности обслуживания заявок имеют произвольное распределение. Получено асимптотическое разложение производящей функции длины
очереди при малом изменении интенсивности потока в случае, когда
загрузка системы в течение рассматриваемого промежутка времени меньше~1.}

\KW{cистемы массового обслуживания $E_{r}\vert G\vert 1$;
 эрланговский входящий поток c интенсивностью, зависящей от времени;
асимптотический анализ}

      \vskip 18pt plus 9pt minus 6pt

      \thispagestyle{headings}

      \begin{multicols}{2}

      \label{st\stat}

\section {Введение}

В литературе по теории массового обслуживания в основном рассматриваются модели, в
которых интенсивность входного потока и
интенсивность обслуживания  не зависят от времени. Однако во многих реальных системах такая
зависимость есть,  и пренебрежение ею может привести к значительному искажению результатов.

Анализ СМО с параметрами, зависящими от времени, намного сложнее анализа аналогичных систем
с постоянными параметрами. В первую оче\-редь это связано с невозможностью непосредст\-вен\-но\-го применения интегральных
преобразований (Лапласа и Лапласа--Стил\-тье\-са) к искомым характеристикам при их нахождении. Поэтому большое значение
при анализе таких моделей имеет получение различных приближенных и асимптотических результатов.

Одним из направлений асимптотического анализа является 
получение асимптотического разложения характеристик систем, в которых из\-менение
параметров мал$\acute{\mbox{о}}$ в течение рассматриваемого промежутка времени. Одними из первых в этом направлении являются работы~[1--3],
в которых проведен асимптотический анализ различных ха\-рактеристик, связанных с длиной очереди в сис\-те\-ме $M(t)\vert M\vert 1.$ 
В работе~[4] эти
исследования продолжены. В ней вводится параметр~$\varepsilon$: отношение среднего времени между двумя последовательными
поступлениями заявок в систему к длине достаточно большого промежутка времени. Если интенсивность потока изменяется слабо, 
то~$\varepsilon$ является малым параметром. Пусть $\lambda(t)$~--- мгновенная интенсивность потока в момент времени $t$, $m$~--- среднее время
обслуживания, $\lambda(\tau)=\lambda(\varepsilon t)$, $\rho(\tau)=\lambda(\tau) m$~--- загрузка системы.
В зависимости от поведения интенсивности рассматривается пять
периодов: начальный переходный период, период малой загрузки,
переходный период насыщения, период перенасыщения, переходный период
в конце фазы перенасыщения. Для каждого периода получается своя
асимптотика. Так, показано, что в период малой загрузки, когда
$\rho(\tau) < 1$, для длины очереди характерно $N(t) = O(1)$   при
$\varepsilon \rightarrow 0$. В~переходный период насыщения при
$\rho(\tau)\uparrow 1$ сис\-те\-ма становится более загружена, и
показано, что $N(t) = O(\varepsilon^{-1/3})$ при $\rho(\tau)\approx 1$. При $\rho(\tau) > 1$, т.\,е.\ в период
перенасыщения, очередь имеет еще большую длину: $N(t) =
O(\varepsilon^{-1})$. И наконец, в переходный период в конце фазы
перенасыщения при  $\rho(\tau) \downarrow 1$ $N(t) = O(1)$ и $N(t)=
O(\varepsilon^{-1/2})$ с разной ве\-ро\-ят\-ностью.

В работе~[5] эти результаты обобщаются на сис\-те\-му $M\vert G\vert 1$.

 В данной статье рассматривается эрланговский
входящий поток с параметрами~$\lambda,\ r$. У системы с эрланговским
входящим потоком интервал между двумя соседними моментами
поступления заявок можно разбить на~$r$ последовательно проходящих фаз
(этапов). Времена прохождения фаз независимы между собой и
одинаково распределены по экспоненциальному закону с параметром~$\lambda$. 
При такой вероятностной интерпретации эрланговского
потока реальная заявка поступает в систему лишь после прохождения
всех $r$~фаз. Частным случаем  эрланговского потока являются пуассоновский (при $r = 1$), а в
качестве предельного при $ r \rightarrow\infty$~--- детерминированный поток.
Отмеченное свойство потоков  Эрланга является одной из причин, по которой
они широко используются для аппроксимации реальных потоков. Изменяя~$r$ от~$1$ до~$\infty$, 
можно получить широкий спектр различных случаев, располагающихся по своим свойствам между этими двумя
крайними полюсами.

 В данной статье исследуется одноканальная СМО $E_{r}(t)\vert G\vert 1$
 с эрланговским входящим потоком, зависящим от времени, и
произвольным распределением времени обслуживания заявок. Получено
асимптотическое разложение длины очереди в случае, когда загрузка
меньше~1 на всем рассматриваемом промежутке времени.

\section {Описание системы} 

Рассмотрим СМО $E_{k}(t)\vert G\vert 1\vert \infty$,  т.\,е.\ одноканальную
систему с неограниченным числом мест для ожидания, в которую
поступает эрланговский входящий поток с интенсивностью, зависящей от
времени, и с произвольной абсолютно непрерывной функцией распределения времени
обслуживания. Обозначим~$B(y)$ и~$b(y)$ функцию распределения и плотность распределения времени
обслуживания  соответственно,
$m_{k}=\int\limits_{0}^{\infty}y^{k} b(y)\,dy$~--- $k$-й момент времени обслуживания.
Входящий поток Эрланга задается параметрами
$(\widetilde{\lambda}(t),k).$ Здесь $k$~--- порядок потока,
$\widetilde{\lambda}(t) k$~--- среднее время между поступлениями заявки.

Будем изучать марковский процесс
$(N(t),Y(t),J(t))$, где $N(t)$~---  число заявок, находящихся в системе в момент~$t$,
 $Y(t)$~--- время, прошедшее с начала обслуживания заявки, находящейся на
приборе в момент~$t$.  (Для определенности $Y(t)=0$, если  в момент~$t$ система свободна.)
Случайный процесс~$J(t)$ со значением $1,\ldots,k$ описываем
следующим образом. Интервал времени между поступлением заявок в
потоке Эрланга порядка~$k$ можно представить в виде суммы $k$~независимых 
случайных величин, имеющих показательное распределение с
па\-ра\-мет\-ром~$\widetilde{\lambda}(t)$. Таким образом, поступление
заявки в систему можно рассматривать как прохождение ею $k$~независимых фаз поступления, длительность каж\-дой из которых имеет
показательное распределение. Тогда $J(t)$~--- номер фазы поступления,
на которой находится поступающая заявка в момент~$t$.

Положим
\begin{multline*}
P_{j}(n,y,t)\,dy=P(N(t)=n,Y(t)\in(y,y+dy)\,,\\
 J(t)=j)\,,\quad  n\geq1\,,\quad k\geq j\geq1\,;
\end{multline*}
\vspace*{-14pt}

\noindent
\begin{align*}
P_{0j}(t)&=P(N(t)=0,J(t)=j)\,;\\
\mu(y)&=\fr{b(y)}{1-B(y)}=\fr{b(y)}{\int\limits^{\infty}_{y}b(z)\,dz}\,;\\
b(y)&= \mu(y)\exp\left(-\int\limits_{0}^{y}\mu(z)\,dz\right)\,.
\end{align*}

Справедливы  следующие уравнения:
\begin{multline}
\fr{\partial P_{j}(n,y,t)}{\partial t}+\fr{\partial
P_{j}(n,y,t)}{\partial y}={}\\
{}=
-[\widetilde{\lambda}(t)+\mu(y)]P_{j}(n,y,t) +{}\\
{}+(1-\delta_{j,1})\widetilde{\lambda}(t)P_{j-1}(n,y,t)+{}\\
{}+
\delta_{j,1}\widetilde{\lambda}(t)P_{k}(n-1,y,t)(1-\delta_{n,1})
\label{q1}
\end{multline}
с граничными условиями
\begin{multline}
P_{j}(n,0,t)=\int\limits_{0}^{\infty}P_{j}(n+1,z,t)\mu(z)\,dz+{}\\
{}+\delta_{n,1}\delta_{j,1}
\widetilde{\lambda}(t)P_{0,k}(t)\,,\quad n\geq1\,,
\label{q2}
\end{multline}
и
\begin{multline}
\fr{\partial P_{0,j}(t)}{\partial t}=-\widetilde{\lambda}(t)P_{0,j}(t)+(1-\delta_{j,1})
\widetilde{\lambda}(t)P_{0,j-1}(t)+{}\\
{}+\int\limits_{0}^{\infty}P_{1,j}(1,z,t)\mu(z)\,dz
\label{q3}
\end{multline}
и начальными условиями
\begin{align}
P_{0,j}(0) &=
\begin{cases}
1\,, &  j=1\,;\\
0\,, & 2 \leq j\leq k\,;
\end{cases}
\notag\\
P_{j}(n,y,0)&=0\,,\quad 1\leq j \leq k\,,\enskip n \geq1\,,\enskip y>0\,;\label{q4}
\end{align}

\vspace*{-12pt}

\begin{equation}
\sum_{j=1}^{k} P_{0,j}(t)+\sum_{j=1}^{k}\sum_{n=1}^{\infty}\int\limits_{0}^{\infty}P_{0,j}(y,t)\,dt=1\,.
\label{q5}
\end{equation}

\section{Постановка задачи}

Будем решать  задачу асимптотически, полагая, что интенсивность
входящего потока медленно меняется. Обозначим маштаб замедления
$\tau=\varepsilon t$.  Положим
$\widetilde{\lambda}(t)=\lambda(\varepsilon t)=\lambda(\tau)$
и
$
P_{j}(n,y,\varepsilon t)=P_{n,j}(y,\tau)$.

Точное определение~$\varepsilon$ не важно для анализа.
Из (\ref{q1})--(\ref{q5})  получаем

\noindent
\begin{multline}
\varepsilon\partial_{\tau}P_{n,j}(y,\tau)={}\\
{}= -\partial_{y}P_{n,j}(y,\tau)+
\delta_{j,1}\lambda(\tau)P_{n-1,k}(y,\tau)- {} \\
{}-\left [\mu(y)+\lambda(\tau)\right]P_{n,j}(y,\tau)+{}\\
{}+
(1-\delta_{j,1})\lambda(\tau)P_{n,j-1}(y,\tau)\,, \quad n\geq2\,;
\label{q6}
\end{multline}

\vspace*{-12pt}

\noindent
\begin{equation}
P_{n,j}(0,\tau)=\int\limits_{0}^\infty P_{n+1,j}(z,\tau)\mu(z)\,dz\,;
\label{q7}
\end{equation}

\vspace*{-12pt}

\noindent
\begin{multline}
\varepsilon\partial_{\tau}P_{1,j}(y,\tau)={}\\
{}=-\partial_{y}P_{1,j}(y,\tau)-\left[\mu(y)+\lambda(\tau)\right]P_{1,j}(y,\tau)+{} \\
{}+(1-\delta_{j,1})\lambda(\tau)P_{1,j-1}(y,\tau)\,;
\label{q8}
\end{multline}

\vspace*{-12pt}

\noindent
\begin{multline}
\varepsilon\partial_{\tau}P_{0,j}(\tau)=-\lambda(\tau)P_{0,j}(\tau)+\int\limits_{0}^{\infty}P_{1,j}(z,\tau)\mu(z)\,dz+{} \\
{}+(1-\delta_{j,1})\lambda(\tau)P_{0,j-1}(\tau)\,;
\label{q9}
\end{multline}

\vspace*{-12pt}
\noindent
\begin{multline}
P_{1,j}(0,\tau)=\delta_{j,1}\lambda(\tau)P_{0,k}(\tau)+{}\\
{}+\int\limits_{0}^{\infty}P_{2,j}(z,\tau)\mu(z)\,dz\,;
\label{q10}
\end{multline}

\vspace*{-8pt}

\noindent
\begin{equation*}
\sum_{j=1}^{k} P_{0,j}(\tau)+\sum_{j=1}^{k}
\sum_{n=1}^\infty\int\limits_{0}^\infty P_{n,j}(y,\tau)\,dy=1\,.
\end{equation*}

В дальнейшем мы будем предполагать, что функция~$\lambda(\tau)$ имеет необходимое число производных. Обозначим
$$
P_{n,j}(\tau)=\int\limits_{0}^{\infty} P_{n,j}(y,\tau)\,dy\,,\quad n \geq 1\,.
$$

Целью работы является нахождение разложения~$P_{n,j}(y,\tau)$ по степеням~$\varepsilon$.

\section{Основные результаты}


Пусть  $0<\tau<\tau_{0}$ и в этом промежутке $\rho(\tau)<1$.
Положим
$$
G_{j}(y,\tau,z)=\sum_{n=1}^{\infty} z^{n}P_{n,j}(y,\tau)\,,\quad j=1,\ldots ,k\,.
$$
Тогда из (\ref{q6})--(\ref{q10}) получаем:
\begin{multline}
\varepsilon\partial_{\tau}G_{j}=-\partial_{y}G_{j}-\mu(y)G_{j}+\lambda(\tau)[\delta_{j,1}zG_{k}-G_{j}]+{}\\
{}+
(1-\delta_{j,1})\lambda(\tau)G_{j-1}\,,\quad  j=1,\ldots , k\,;
\label{q11}
\end{multline}

%\vspace*{-12pt}

\noindent
\begin{multline}
G_{j}(0,\tau,z)=\fr{1}{z}\int\limits_{0}^{\infty}G_{j}(u,\tau,z)\mu(u)\,du+{}\\
{}+\lambda(\tau)\left[\delta_{j,1}zP_{0,k}(\tau)- P_{0,j}(\tau)+
(1-\delta_{j,1})P_{0,j-1}(\tau)\right]-{}\\
{}- \varepsilon\partial_{\tau}P_{0,j}(\tau)\,,
\quad j=1,\ldots , k\,.
\label{q12}
\end{multline}

 Для малого~$\varepsilon$ справедливы разложения
\begin{gather*}
G_{j}(y,\tau,z)=G_{0,j}(y,\tau,z)+\varepsilon
G_{1,j}(y,\tau,z)+O(\varepsilon^{2})\,;
%\label{q13}
\\
P_{n,j}(y,\tau)=A_{n,j}(y,\tau)+\varepsilon
B_{n,j}(y,\tau)+O(\varepsilon^{2})\,,\enskip n \geq 1\,;\hspace*{-1.9pt}
%\label{q14}
\\[2pt]
P_{0,j}(\tau)=A_{0,j}(\tau)+\varepsilon B_{0,j}(\tau)+O(\varepsilon^{2})\,.
\end{gather*}

Из (\ref{q11}) и~(\ref{q12}) получаем
\begin{multline*}
\partial_{y}G_{0,j}+\mu(y)G_{0,j}-\lambda(\tau)[\delta_{j,1}zG_{0,k}-G_{0,j}]-{}\\
{}-(1-\delta_{j,1})\lambda(\tau)G_{0,j-1}\equiv M_{j}G_{0}=0\,;
\end{multline*}
\vspace*{-16pt}

\noindent
\begin{multline*}
G_{0,j}(0,\tau,z)=\fr{1}{z}\int\limits_{0}^{\infty}G_{0,j}(u,\tau,z)\mu(u)\,du+{}\\
{}+ \lambda(\tau)[\delta_{j,1}zA_{0,k}(\tau)-{}\\
{} -A_{0,j}(\tau)+(1-\delta_{j,1})A_{0,j-1}(\tau)]\,,
\end{multline*}
\vspace*{-12pt}

\noindent
\begin{equation*}
M_{j}G_{1,j}=-\partial_{\tau}G_{0,j}\,,\quad j=1,\ldots , k\,;
\end{equation*}

\vspace*{-12pt}

\noindent
\begin{multline}
G_{1,j}(0,\tau,z)=\fr{1}{z}\int\limits_{0}^{\infty}G_{1,j}(u,\tau,z)\mu(u)\,du+{}\\
{}+
\lambda(\tau)[\delta_{j,1}zB_{0,k}(\tau)- B_{0,j}(\tau)+{}\\
{}+(1-\delta_{j,1})B_{0,j-1}(\tau)]-\partial_{\tau}A_{0,j}(\tau)\,.
\label{q16}
\end{multline}

 Получим решение системы~(\ref{q1})--(\ref{q5}) при $\widetilde{\lambda}(t)=a$ для любого~$t$.

 Введем функции
\begin{align*}
p_{j}(z,y,s)&=\sum_{n=1}^{\infty}z^{n}\int\limits_{0}^{\infty}e^{-st}P_{j}(n,y,t)\,dt\,;\\
p_{0,j}(s)&=\sum_{n=1}^{\infty}z^{n}\int\limits_{0}^{\infty}e^{-st}P_{0,j}(t)\,dt\,.
\end{align*}

Тогда из~(\ref{q1})--(\ref{q5}) получим
\begin{multline}
\fr{\partial p_{j}(z,y,s)}{\partial y}
=-\left[s+a+\mu(y)\right]p_{j}(z,y,s) +{}\\
{}+(1-\delta_{j,1})ap_{j-1}(z,y,s) +
\delta_{j,1}azp_{k}(z,y,s)\,;
\label{q18}
\end{multline}

\vspace*{-12pt}

\noindent
\begin{multline}
p_{j}(z,0,s)=\fr{1}{z}\int\limits_{0}^{\infty}p_{j}(z,u,s)\mu(u)\,du+{}\\
{}+\delta_{j,1}-(s+a)p_{0,j}(s)+
\left(1-\delta_{j,1}\right)ap_{0,j-1}(s)+{}\\
{}+az\delta_{j,1}p_{0,k}(s)\,.
\label{q19}
\end{multline}

Общее решение системы уравнений~$(\ref{q18})$ записывается в виде
%\pagebreak

\noindent
\begin{multline}
p_{j}(z,y,s)=
(1-B(y)) \times{}\\
{}\times  \sum_{m=1}^{k}(z_{k,m})^{k-j}\psi_{m}(z,s)e^{-(s+a-az_{k,m})y}\,,
\label{q20}
\end{multline}
где $z_{k,j}$~---  корни $k$-й степени из~$z$ 
(если $z=\rho e^{i\varphi},$ то $z_{k,j}=\sqrt[k]{\rho}\exp\left(i(\varphi+2 \pi j)/k\right))$.
Обозначим $\gamma_{j}(z)=$\linebreak $=-\;a+az_{k,j},$ тогда $\gamma_{j}(z)$, $j=1,\ldots , k $~--- корни уравнения 
$((\gamma+a)/a)^{k}=z$.
Заметим, что при такой нумерации $\gamma_{1}(1)=0$.

Подставляя (\ref{q20}) в~(\ref{q19}), получим
\begin{equation}
\sum_{m=1}^{k}\left(\fr{a+\gamma_{m}(z)}{a}\right)^{k-j} 
\delta_{m}(z,s)=f_{j}(z,s)\,,
\label{q21}
\end{equation}
где
$$
\delta_{m}(z,s)=\left(1-z^{-1}\beta(s-\gamma_{m}(z))\right) \psi_{m}(z,s)\,;
$$

\vspace*{-12pt}

\noindent
\begin{multline*}
f_{j}(z,s)=\delta_{j,1}-(s+a)p_{0,j}(s)+{}\\
{}+
(1-\delta_{j,1})ap_{0,j-1}(s)+az\delta_{j,1}p_{0,k}(s)\,;
%\label{q22}
\end{multline*}

\vspace*{-12pt}

\noindent
$$
\beta(s)=\int\limits_{0}^{\infty}e^{-sy}b(y)\,dy\,.
$$

Решив~(\ref{q21}) как систему линейных уравнений относительно~$\delta_{m}(z,s),$ получим
\begin{multline*}
\delta_{m}(z,s)=k^{-1}\left(\gamma_{m}(z)+a\right)^{-(k-1)} \times{}\\
{}\times
\sum_{j=1}^{k}f_{j}(z,s)a^{k-j}\left(\gamma_{m}(z)+a\right)^{j-1}\,.
\end{multline*}
Отсюда
\begin{multline}
k(\gamma_{m}(z)+a)^{k-1}\delta_{m}(z,s)={}\\
{}=
 \sum_{j=1}^{k}f_{j}(z,s)a^{k-j}(\gamma_{m}(z)+a)^{j-1}\,.
 \label{q23}
\end{multline}
%\pagebreak

\noindent
Заметим, что $f_{j}(z,s)$ при  $j=2, \ldots , k$ не зависит от~$z$, а
\begin{multline*}
f_{1}(z,s)=1-(s+a)p_{0,1}(s)+azp_{0,k}(s)={}\\
=1-(s+a)p_{0,1}(s)+\left(\fr{\gamma_{m}(z)+a}
{a}\right)^{k}ap_{0,k}(s)
\end{multline*}
при любых~$m$.

\smallskip

\noindent
\textbf{Лемма.} \textit{Уравнение}
$$
z=\beta(s-\gamma_{m}(z))
$$
\textit{имеет единственное решение} $z=z^{(m)}(s)$,  $\Re s>0$, $\vert z_{(m)}(s)\vert <1$, \textit{причем}
$\lim\limits_{s\rightarrow0}z^{(1)}(s)=1$.

\smallskip

Обозначим
$$
\alpha_{m}(s)=\gamma_{m}(z^{(m)}(s))\,.
$$
Используя лемму, из~(\ref{q23}) получаем
%\noindent
\begin{multline}
k(\gamma_{m}(z)+a)^{k-1}\delta_{m}(z,s)={}\\
{}=p_{0,k}(s)\prod_{i=1}^{k}(\gamma_{m}(z)-\alpha_{i}(s))\,.
\label{q24}
\end{multline}

Приравняв~(\ref{q23}) и~(\ref{q24}), получаем систему линейных уравнений для
определения~$p_{0,j}(s)$ для любых~$j$
\begin{multline*}
\sum_{j=l+1}^{k}[ap_{0,j-1}(s)-(s+a)p_{0,j}(s)]a^{k-l-1}C^{l}_{j-1}+{}\\
{}+p_{0,k}(s)C_{k}^{l}a^{k-l}={}\\
{}=(-1)^{k-l}p_{0k}(s)\sum_{1\leq i_{1}<\ldots<i_{k-l}\leq
k}\alpha_{i_{1}}(s)\ldots\alpha_{i_{k-l}}(s)\,,\\
\qquad  0\leq l \leq k-1\,, 
\end{multline*}
где $ap_{0,0}(s)=1$.
Находим
\vspace*{2pt}

\noindent
$$
p_{0,k}(s)=\fr{a^{k-1}}{\prod\limits_{j=1}^{k}[s-\alpha_{j}(s)]}\,;
$$

\vspace*{-14pt}

\noindent
\begin{multline*}
p_{0,j}(s)=\left[\fr{(s+a)^{k-j}}{a^{k-j}}+\sum_{m=j}^{k-1}\left(\fr{s+a}{a}\right)^{m-j}\times{}\right.\\
\left.{}\times
\sum_{n=k-1}^{k-m}(-1)^{n-m}C_{n}^{m}
\left(
(-1)^{k-n}\times{}\right.\right.\\
{}\times
\sum_{1\leq i_{1}<\ldots<i_{k-n}\leq k}\alpha_{i_{1}}(s)\ldots\alpha_{i_{k-n}}(s)-{}\\
\left.\left.{}-C_{k}^{n}a^{k-n}\right)
\fr{1}{a^{k-n}}
\vphantom{\sum_{m=j}^{k-1}}
\right]\vphantom{\sum_{m=j}^{k-1}}
p_{0,k}(s)\,,\quad  1\leq j \leq k-1\,.
\end{multline*}

 Пусть $t\rightarrow+\infty.$ Тогда при условии $a m_1<1$
существуют пределы:
\begin{align*}
\lim\limits_{t\rightarrow\infty}P_{j}(n,y,t)&=P_{j}(n,y)\,;\\
\lim\limits_{t\rightarrow\infty}P_{0,j}(t)&=P_{0,j}\,.
\end{align*}

Так как
\begin{align*}
\sum_{n=1}^{\infty}z^{n}P_{j}(n,y)&=\lim\limits_{s\rightarrow0}sp_{j}(z,y,s)\,;\\
P_{0,j}&=\lim\limits_{s\rightarrow0}sp_{0,j}(s)\,,
\end{align*}
то

\noindent
\begin{multline}
\sum^{\infty}_{n=1}z^{n}P_{j}(n,y)={}\\
{}=
\fr{1}{\prod\limits_{j=2}^{k}[-\alpha_{j}(0)]}\,\fr{k-am_{1}}{k}\left(1-B(y)\right)\times{}\\
{}\times \sum_{m=1}^{k}
\left (
(z_{k,m})^{k-j}\,
e^{-(a-az_{k,m})y}\times{}\right.\\
%\left.
{}\times\prod\limits_{j=1}^{k}(\gamma_{m}(z)-\alpha_{i}(0))\Big / %\right.
\left (k(\gamma_{m}(z)+a)^{k-1}\left(1-{}\right.\right.\\
\left.\left.\left.{}- z^{-1}
\beta(-\gamma_{m}(z))\right)\right )
\vphantom{\left(z_{k,m}\right)^{k-j}}\right )\, a^{k-1}\,.
\label{e33}
\end{multline}


При  $a=\lambda(\tau)$ из~(\ref{e33}) имеем
\begin{multline*}
\!\!\!G_{0,j}(y,\tau,z)=
\fr{1}{\prod\limits_{j=2}^{k}[-\alpha_{j}(0)]}\,\fr{k-\lambda(\tau)m_{1}}{k}\,
(1-B(y))\;\times{}\\ 
{}\times\sum\limits_{m=1}^{k}
\left (\left(z_{k,m}\right)^{k-j}
e^{-(\lambda(\tau)-\lambda(\tau)z_{k,m})y}\times{}\right.\\
{}\times
\prod\limits_{j=1}^{k}(\gamma_{m}(z)-\alpha_{i}(0))\Big /
\left (k(\gamma_{m}(z)+\lambda(\tau))^{k-1}\left(1-{}\right.\right.\\
\left.\left.\left.{}- z^{-1}
\beta(-\gamma_{m}(z))\right )\right)
\vphantom{\left(z_{k,m}\right)^{k-j}}
\right)\,\lambda(\tau)^{k-1}\,,
%\label{q17}
\end{multline*}
где $z_{k,m}$~--- корень $k$-й степени из~$z$,
$\gamma_{m}(z)=$\linebreak $=\;-\lambda(\tau)+\lambda(\tau)z_{k,m}$, а~$\alpha_{m}$
определяется из леммы с заменой~$a$ на~$\lambda(\tau)$.
\begin{multline*}
A_{0,j}(\tau)=A_{0,k}(\tau)+{}\\
{}+
\sum_{m=j}^{k-1}\sum_{n=k-1}^{k-m}(-1)^{n-m}A_{0,k}\left(\tau\right)C_{n}^{m}\times{}
\\
{} \times\left((-1)^{k-n}\sum_{1\leq i_{1}<\ldots<i_{k-n}\leq
k}\alpha_{i_{1}}(0)\ldots\alpha_{i_{k-n}}(0)-{}\right.\\
\left.{}-C_{k}^{n}\,\lambda(\tau)^{k-n}
\vphantom{\sum_{1\leq i_{1}<\ldots<i_{k-n}\leq
k}}
\right)
\fr{1}{\lambda(\tau)^{k-n}}\,,\quad  1\leq j \leq k-1\,;
\end{multline*}

\noindent
\begin{equation*}
A_{0,k}(\tau)=\fr{\lambda^{k-1}(\tau)}{\prod\limits_{j=2}^{k}(-\alpha_{j}(0))}\,\fr{k-\lambda(\tau)m_{1}}{k}\,.
\end{equation*}

Теперь рассмотрим систему
\begin{multline*}
\fr{\partial \widehat{p}_{j}(z,y,s)}{\partial y}
=-\left[s+a+\mu(y)\right]\widehat{p}_{j}(z,y,s)
+{}\\
{}+(1-\delta_{j,1})a\widehat{p}_{j-1}(z,y,s) +
\delta_{j,1}az\widehat{p}_{k}(z,y,s)\,;
%\label{q25}
\end{multline*}
%
%\vspace*{-12pt}
%
%\noindent
\begin{multline*}
\widehat{p}_{j}(z,0,s)=\fr{1}{z}\int\limits_{0}^{\infty}\widehat{p}_{j}(z,u,s)\mu(u)\,du+\delta_{j,1}-{}\\
{}-
(s+a)\widehat{p}_{0,j}(s)+ (1-\delta_{j,1})a\widehat{p}_{0,j-1}(s)+{}\\
{}+
az\delta_{j,1}\widehat{p}_{0,k}(s)-\fr{1}{s}h_{j}(s)\,.
%\label{q26}
\end{multline*}

Ее решение можно найти аналогично  решению системы~(\ref{q18}) и (\ref{q19}):
\begin{multline*}
\widehat{p}_{j}(z,y,s)={}\\
{}=(1-B(y))\sum_{m=1}^{k}(z_{k,m})^{k-j}
e^{-(s+a-az_{k,m})y}\times{}\\ 
{}\times\fr{p_{0,k}(s)\prod\limits_{j=1}^{k}(\gamma_{m}(z)-\alpha_{i}(s))}
{k(\gamma_{m}(t)+a)^{k-1}(1-z^{-1}\beta(s-\gamma_{m}(z)))}\,,
\end{multline*}
где
$$
p_{0,k}(s)=\fr{a^{k-1}[1-\sum\limits_{m=1}^{k}((s+a)/a)^{m-1}h_{m}(s)]}{\prod\limits_{j=1}^{k}[s-\alpha_{j}(s)]
}\,.
$$

Отсюда

\noindent
\begin{multline}
\widehat{p}_{j}(z,y)=\lim\limits_{s\rightarrow 0}s\widehat{p}_{j}(z,y,s)={}\\
{}=
\fr{1}{\prod\limits_{j=2}^{k}[-\alpha_{j}(0)]}\,\fr{k-am_{1}}{k}\left(1-\sum\limits_{m=1}^{k}h_{m}(0)\right)
\times{}\\
{}\times \left (1-B(y)\right )
\sum_{m=1}^{k}\left (\left(z_{k,m}\right)^{k-j}
e^{-(a-az_{k,m})y}\times{}\right.\\
{}\times\prod\limits_{j=1}^{k}(\gamma_{m}(z)-\alpha_{i}(0))\Big /
\left ( k(\gamma_{m}(z)+a)^{k-1}\left (1-{}\right.\right.\\
\left.\left.\left.   {}-z^{-1}
\beta\left(-\gamma_{m}(z)\right)\right )\right )
\vphantom{\left(z_{k,m}\right)^{k-j}}
\right )\, a^{k-1}\,.
\label{e44}
\end{multline}


Для решения системы~(\ref{q16}) рассмотрим сначала следующую систему:
$$
M_{j}G_{1,j}=0\,;
$$


\vspace*{-12pt}

\noindent
\begin{multline*}
G_{1,j}(0,\tau,z)=\fr{1}{z}\int\limits_{0}^{\infty}G_{1,j}(u,\tau,z)\mu(u)\,du+{}\\
{}+\lambda(\tau)[\delta_{j,1}zB_{0,k}(\tau)-B_{0,j}(\tau)+{}\\
{}+(1-\delta_{j,1})B_{0,j-1}(\tau)]-\partial_{\tau}A_{0,j}(\tau)\,.
\end{multline*}
Из~(\ref{e44}) при $a=\lambda(\tau)$, $h_{m}=\partial_{\tau}A_{0,j}(\tau)$ находим
\pagebreak

\noindent
\begin{multline*}
G_{1,j}(y,\tau,z)=
\fr{1}{\prod\limits_{j=2}^{k}\left[-\alpha_{j}(0)\right]}\,\fr{k-\lambda(\tau)m_{1}}{k}\times{}\\
{}\times
\left(1-\sum_{m=1}^{k}\partial_{\tau}A_{0,m}(\tau)\right)\left(1-B(y)\right)\times{}\\
{}\times\sum_{m=1}^{k}
\left (\left(z_{k,m}\right)^{k-j}
e^{-(\lambda(\tau)-\lambda(\tau)z_{k,m})y}\times{}\right.\\
{}\times
\prod\limits_{j=1}^{k}(\gamma_{m}(z)-\alpha_{i}(0))\Big /
 \left (k(\gamma_{m}(z)+\lambda(\tau))^{k-1}\left(1-{}\right.\right.\\
\left.\left.\left. {}-z^{-1}
 \beta(-\gamma_{m}(z))\right)\right )
 \vphantom{\left(z_{k,m}\right)^{k-j}}
 \right )\,\lambda(\tau)^{k-1}\,.
\end{multline*}

Теперь рассмотрим систему
\begin{align*}
M_{j}G_{1,j}&=-\partial_{\tau}G_{0,j}\,,\quad  j=1,\ldots ,k\,;
\\
G_{1,j}(0,\tau,z)&=0\,. 
\end{align*}
Легко видеть, что
\begin{multline*}
\xi_{j}=\sum_{l=1}^{k}(1-B(y))\exp\left[\lambda(\tau)(z_{k,l}-1)y\right]\times{}\\
\!\!{}\times \left[y(\psi_{l}(\tau,
z))^{'}_{\tau}+
\fr{1}{2}\,\left(z_{k,l}-1\right)y^{2}\lambda^{'}(\tau)\psi_{l}(\tau,
z)\right ]z^{k-j}_{k,l}\,,\\
j=1,\ldots , k\,,
\end{multline*}
где
\begin{multline*}
\psi_{l}(\tau, z)=\fr{1}{\prod\limits_{j=2}^{k}[-\alpha_{j}(0)]}\,\fr{k-\lambda(\tau)m_{1}}{k}\,
\left(
\vphantom{\sum_{m=1}^{k}}
 1-{}\right.\\
\left.{}-\sum_{m=1}^{k}\partial_{\tau}A_{0,j}(\tau)\right)\times{}\\
{}\times
\fr{\prod\limits_{j=1}^{k}(\gamma_{l}(z)-\alpha_{i}(0))}{k(\gamma_{l}(z)+\lambda(\tau))^{k-1}(1-z^{-1}
\beta(-\gamma_{l}(z)))}\,\lambda(\tau)^{k-1}
\end{multline*}
является частным решением данной системы.
 Следовательно, решение системы~(\ref{q16}) запишется в следующем виде:
\begin{multline*}
G_{1,j}(y,\tau,z)=\fr{1}{\prod\limits_{j=2}^{k}[-\alpha_{j}(0)]}
\,\fr{k-\lambda(\tau)m_{1}}{k}\,\left(1-{}\right.\\
\left.{}-\sum_{m=1}^{k}\partial_{\tau}A_{0,m}(\tau)\right) (1-B(y))\times{}\\
{} \times
\sum_{m=1}^{k}
\left (\left(z_{k,m}\right)^{k-j} e^{-(\lambda
(\tau)-\lambda(\tau)z_{k,m})y}\times{}\right.\\
{}\times\prod\limits_{j=1}^{k}(\gamma_{m}(z)-\alpha_{i}(0))\Big /
\left (k(\gamma_{m}(z)+\lambda
(\tau))^{k-1}\left(1-{}\right.\right.\\
\left.\left.\left.{}-z^{-1}
\beta(-\gamma_{m}(z))\right)\right )
\vphantom{\left(z_{k,m}\right)^{k-j}}
\right )\,\lambda^{k-1}(\tau)+\xi_{j}\,.
\end{multline*}


{\small\frenchspacing
{%\baselineskip=10.8pt
\addcontentsline{toc}{section}{Литература}
\begin{thebibliography}{9}    

\bibitem{1ush}
\Au{Newell G.\,F.} 
Queues with time-dependent arrival rates I~--- the transition through saturation~// 
J. Appl. Prob., 1968. Vol.~5. P.~436--451.

\bibitem{2ush}
\Au{Newell G.\,F.} 
Queues with time-dependent arrival rates II~--- the maximum queue and the return to equilibrium~// 
\mbox{J.~Appl.} Prob., 1968. Vol.~5. P.~579--590.

\bibitem{3ush}
\Au{Newell G.\,F.} 
Queues with time-dependent arrival rates III~--- a mild rush hour~// J. Appl. Prob., 1968. Vol.~5. P.~591--606.

\bibitem{4ush}
\Au{Keller J.\,B.} 
Time-dependent queues~// SIAM Review, 1982. Vol.~24. No.\,4. P.~401--412.

\label{end\stat}

\bibitem{5ush}
\Au{Yang~Y., Knessl~C.} 
Asymptotic analysis of the $M\vert G\vert 1$ queue with a time-dependent arrival rate~// 
Queueing Systems, 1997. Vol.~26. P.~23--68.
 \end{thebibliography}
}
}
\end{multicols}