\def\stat{abstr}
{%\hrule\par
%\vskip 7pt % 7pt
\raggedleft\Large \bf%\baselineskip=3.2ex
A\,B\,S\,T\,R\,A\,C\,T\,S \vskip 17pt
    \hrule
    \par
\vskip 21pt plus 6pt minus 3pt }


\def\tit{PROBABILISTIC METHODS OF~INFORMATION MODEL BUILDING FOR~THE~EARTH ROTATION IRREGULARITY}

%1
\def\aut{I.\,N.~Sinitsyn}

\def\auf{IPI RAN, sinitsin@dol.ru}

\def\leftkol{\ } % ENGLISH ABSTRACTS}

\def\rightkol{\ } %ENGLISH ABSTRACTS}

\titele{\tit}{\aut}{\auf}{\leftkol}{\rightkol}


\noindent 
New linear and nonlinear probabilistic methods  of model building for fluctuation of the Earth 
irregularity are considered. Methods are the basis of \textit{a priori} data processing for the problem
 ``Statistical Dynamics of the Earth Rotation.'' Test examples are provided.

\label{st\stat}

 \KWN{information model; information resources; linear and nonlinear probabilistic methods; 
 hereditary; fluctuations; one- and multidimensional distributions; asymmetry; excess; 
 Gaussian and Poisson white noises; distribution parameterization; moments methods
}

\vskip 14pt plus 6pt minus 3pt


\def\tit{ELASTIC DEFORMATIONS IMPACT ON FINGERPRINT RECOGNITION PERFORMANCE}

%2
\def\aut{O.\,S.~Ushmaev$^1$ and A.\,R.~Arutyunyan$^2$}
\def\auf{$^1$IPI RAN, oushmaev@ipiran.ru\\[1pt]
$^2$Nuclear Safety Institute RAN, artem@ac.ibrae.ru}

\def\leftkol{\ } % ENGLISH ABSTRACTS}

\def\rightkol{\ } %ENGLISH ABSTRACTS}

\titele{\tit}{\aut}{\auf}{\leftkol}{\rightkol}

\noindent
Elastic deformations are the strong negative factor 
in fingerprint recognition. The mechanical approach to fingerprint 
deformation modeling was employed to determine elastic deformation impact on
fingerprint recognition. The statistic analysis of structure of elastic deformations was
carried out. It revealed that 
an arbitrary deformation is the combination of elementary principal deformations: 
rotation, torsion, and traction. Finally, numerical measure for the impact 
of deformations on fingerprint recognition was found.

%\label{st\stat}

\KWN{fingerprint recognition; nonlinear distortions of fingerprints
}

%\pagebreak

% \thispagestyle{headings}

\vskip 14pt plus 6pt minus 3pt

%\vfil

%3
\def\tit{CALCULATION ALGORITHM OF WORKLOAD OF~TELECOMMUNICATION NETWORK WITH~REPETITIVE  TRANSMISSIONS}

\def\aut{Ya.\,M.~Agalarov}
\def\auf{IPI RAN, agglar@yandex.ru}

\titele{\tit}{\aut}{\auf}{\leftkol}{\rightkol}

\noindent
The models of packet switching network with repetitive transmissions for two 
schemes of buffer memory distributions~--- complete sharing and complete partitioning~--- are considered. 
The iterative method of calculation of stream intensity in network and probabilities 
of node blocking where node model is the queueing system of type %\linebreak\vspace*{-8pt}
%\noindent
  $\begin{matrix}
      M \\ \vec{\lambda}
      \end{matrix}
      \left |
      \begin{matrix}
      M \\ \vec{\lambda}
      \end{matrix}
      \right |
      \vec{m} \vert N$ is proposed. The
necessary condition for existence of solution of stream balance 
conservation equations in steady-state regime was established. The 
monotone convergence of stream intensities sequence and  
probabilities of blocking derived in the proposed method to the
solution of these combined equations was proved.


\KWN{network of packets switching; buffer memory; repetitive transmissions; 
probabilities of blocking; iteration method}
\pagebreak


%\vfil
%\vskip 6pt plus 6pt minus 3pt
% \vskip 24pt plus 9pt minus 6pt

%4
\def\tit{BAYESIAN QUEUEING AND RELIABILITY MODELS: GENERAL ERLANG CASE
}

\def\aut{A.\,A.~Kudriavtsev$^1$, V.\,S.~Shorgin$^2$, and S.\,Ya.~Shorgin$^3$}
\def\auf{$^1$Department of Mathematical Statistics, Faculty of
Computational Mathematics and Cybernetics, \\ 
\hphantom{$^1$}M.\,V.~Lomonosov Moscow State University, nubigena@hotmail.com\\[1pt]
$^2$IPI RAN, vshorgin@ipiran.ru\\[1pt]
$^3$IPI RAN, sshorgin@ipiran.ru}


%\def\leftkol{ENGLISH ABSTRACTS}

%\def\rightkol{ENGLISH ABSTRACTS}

\titele{\tit}{\aut}{\auf}{\leftkol}{\rightkol}

\vspace*{-6pt}

\noindent
The paper is a next step in bayesian queueing and reliability models investigation.
The method provides the randomization of system characteristics with regard to \textit{a priori} 
distributions of input parameters. The new results are presented for two cases when both \textit{a priori} 
distributions are Erlang distributions and when a pair of \textit{a~priori} 
distributions is the pair ``Erlang distribution\,--\,degenerate distribution.''

\vspace*{-2pt}

\KWN{bayesian approach; queueing systems; reliability; mixed distributions; modeling; Erlang distribution
}
%\pagebreak

%\vful

 \vskip 6pt plus 6pt minus 3pt

% \vskip 24pt plus 9pt minus 6pt
%\vskip 6pt plus 3pt minus 3pt
%\vspace*{12pt}

%5
\def\tit{ASYMPTOTIC ANALYSIS OF THE $E_r(t)\vert G \vert 1$ QUEUE}


\def\aut{O.\,V.~Petrova$^1$ and V.\,G.~Ushakov$^2$}

\def\auf{$^1$Department of Mathematical Statistics, Faculty of
Computational Mathematics and Cybernetics, \\ 
\hphantom{$^1$}M.\,V.~Lomonosov Moscow State University, o.petrova@inbox.ru\\[1pt]
$^2$Department of Mathematical Statistics, Faculty of
Computational Mathematics and Cybernetics, \\ 
\hphantom{$^1$}M.\,V.~Lomonosov Moscow State University; IPI RAN, ushakov@akado.ru}


\def\leftkol{ENGLISH ABSTRACTS}

\def\rightkol{ENGLISH ABSTRACTS}

\titele{\tit}{\aut}{\auf}{\leftkol}{\rightkol}

\vspace*{-6pt}

\noindent
The single server queue with time-dependent Erlangian input is considered. The service times have general
distribution. The asymptotic behavior of the queue length with traffic intensity less than~1 and with
arrival rate weakly depending upon time is investigated.

\vspace*{-2pt}

\KWN{$E_r\vert G\vert 1$ queue; time-dependent Erlangian input; asymptotic analysis}

%\vskip 18pt plus 6pt minus 3pt

 \vskip 6pt plus 6pt minus 3pt

% \pagebreak

%6
\def\tit{ASYMPTOTIC ESTIMATES OF THE ABSOLUTE CONSTANT IN~THE~BERRY--ESSEEN
INEQUALITY FOR~DISTRIBUTION WITH~UNBOUNDED THIRD MOMENT
}

\def\aut{M.\,O.~Gaponova$^1$ and I.\,G.~Shevtsova$^2$}
\def\auf{$^1$Faculty of
Computational Mathematics and Cybernetics, 
M.\,V.~Lomonosov Moscow State University,\\
\hphantom{$^1$}margarita.gaponova@gmail.com\\[1pt]
$^2$Faculty of
Computational Mathematics and Cybernetics, M.\,V.~Lomonosov Moscow State University,\\
\hphantom{$^1$}ishevtsova@cs.msu.su}

\titele{\tit}{\aut}{\auf}{\leftkol}{\rightkol}

\vspace*{-6pt}

\noindent
The Prawitz' asymptotic estimates for the absolute constant in
the Berry--Esseen inequality are sharpened for the case of independent identically
distributed random variables with finite third moments. Similar estimates
are constructed for the case of unbounded third absolute moment. Also,
upper estimates of the asymptotically exact constants in the central limit
theorem are presented.

\vspace*{-2pt}

\KWN{central limit theorem; normal approximation; convergence rate estimate;
sum of independent random variables; Berry--Esseen inequality; Lyapounov
fraction; asymptotically exact constant
}
%\pagebreak

 \vskip 6pt plus 6pt minus 3pt

%7
\def\tit{LIMIT DISTRIBUTION OF RISK ESTIMATE OF WAVELET COEFFICIENT THRESHOLDING
}


\def\aut{A.\,V.~Markin}
\def\auf{Department of Mathematical Statistics, Faculty of
Computational Mathematics and Cybernetics,\\ M.\,V.~Lomonosov Moscow State University, artem.v.markin@mail.ru}


\titele{\tit}{\aut}{\auf}{\leftkol}{\rightkol}

\vspace*{-6pt}

\noindent
Asymptotic properties of risk estimate of wavelet coefficient thresholding are studied. 
Under certain conditions, there is a convergence of the difference of risk estimate and risk 
itself to normal distribution. 

\vspace*{-2pt}

\KWN{wavelets;  thresholding; risk estimate; limit distribution}
\pagebreak

 \vskip 12pt plus 6pt minus 3pt

%8
\def\tit{TECHNOLOGY OF POORLY FORMALIZED DOCUMENTS STORAGE
ON~THE~BASIS OF~LEXICOLOGICAL SYNTHESIS
}

\def\aut{B.\,V.~Chernikov}
\def\auf{Limited Liability Company ``ANT-Inform,'' bor-cher@yandex.ru
}

\titele{\tit}{\aut}{\auf}{\leftkol}{\rightkol}

\noindent
The technology of storage of poorly formalized documents that 
are created using lexicological synthesis is considered. The technology provides formation of the kept index 
sequences containing indexes of document forms and their substantial components. 
Additionally, thanks to simultaneous preparation of documents and creation of kept index sequences, 
the economy of time is provided. The experiments have shown the efficiency of the 
approach for the documents created for management of different kinds of activity.


\KWN{poorly formalized document; lexicological synthesis; index; index sequence; compression}
%\pagebreak

 \vskip 12pt plus 6pt minus 3pt

%9

\def\tit{SELF-ORGANIZATION OF INTELLIGENT AGENTS GROUPS SIMULATION DEPENDING~ON~DEGREE OF~INTERACTION
}

\def\aut{I.\,A.~Kirikov$^1$, A.\,V.~Kolesnikov$^2$, and S.\,V.~Listopad$^3$}
\def\auf{$^1$Kaliningrad branch of the IPI RAN, kfipiran@yandex.ru\\[1pt]
$^2$Kaliningrad branch of the IPI RAN, avkolesnikov@yandex.ru\\[1pt]
$^3$Kaliningrad branch of the IPI RAN, ser-list-post@yandex.ru}

\titele{\tit}{\aut}{\auf}{\leftkol}{\rightkol}

\noindent
One of the approaches to the creation of self-organizing intellectual computer system for 
decision-making support based on an analysis of the experts' goals is considered. An algorithm 
for determining the type of multiagent system architecture based on the extent of interaction 
between agents is considered, that is relevant to determining the effectiveness of expert groups 
and to improving the quality of decision-making.

\KWN{decision support computer system; self-organizing multiagent system; 
similarity measure of agents' fuzzy goals; algorithm for determining the type 
of multiagent system architecture based on the extent of interaction between agents}
%\pagebreak



\vskip 12pt plus 6pt minus 3pt

%10
\def\tit{TIME-DEPENDENT SEMIOTIC MODEL FOR COMPUTER CODING OF~CONCEPTS, INFORMATION OBJECTS, AND~DENOTATA
}
\def\aut{I.\,M.~Zatsman}


\def\auf{IPI RAN, iz\_ipi@a170.ipi.ac.ru
}

%\def\leftkol{ENGLISH ABSTRACTS}

%\def\rightkol{ENGLISH ABSTRACTS}

\titele{\tit}{\aut}{\auf}{\leftkol}{\rightkol}

\noindent
The time-dependent semiotic model, which has been developed during research of problems of 
generation and evolution of goal-oriented knowledge systems in digital libraries and other 
kinds of information systems, is considered. These problems concern to the new direction of 
the researches, which have been named ``Cognitive Informatics.'' This model is positioned as 
theoretical foundations for computer coding of concepts, information objects, and 
denotata in view of their evolution in time.


\KWN{semiotic model; goal-oriented knowledge systems; denotata; 
concepts; information objects; computer codes; three-component coding; 
concept evolution trajectory
}


 \label{end\stat}
 %\pagebreak