\def\stat{kirikov}


\def\tit{МОДЕЛИРОВАНИЕ САМООРГАНИЗАЦИИ ГРУПП 
ИНТЕЛЛЕКТУАЛЬНЫХ АГЕНТОВ В~ЗАВИСИМОСТИ\\
ОТ~СТЕПЕНИ СОГЛАСОВАННОСТИ ИХ~ВЗАИМОДЕЙСТВИЯ}
\def\titkol{Моделирование самоорганизации групп 
интеллектуальных агентов} % в~зависимости от~степени согласованности их~взаимодействия} 

\def\autkol{И.\,А.~Кириков, А.\,В.~Колесников, С.\,В.~Листопад}
\def\aut{И.\,А.~Кириков$^1$, А.\,В.~Колесников$^2$, С.\,В.~Листопад$^3$}

\titel{\tit}{\aut}{\autkol}{\titkol}

%{\renewcommand{\thefootnote}{\fnsymbol{footnote}}\footnotetext[1]
%{Работа выполнена при поддержке РФФИ, проекты 08-07-00152-а, 08-01-00567-а и 09-07-12032-офи-м.}}

\renewcommand{\thefootnote}{\arabic{footnote}}
\footnotetext[1]{Калининградский филиал Института проблем информатики РАН, kfipiran@yandex.ru}
\footnotetext[2]{Калининградский филиал Института проблем информатики РАН, avkolesnikov@yandex.ru}
\footnotetext[3]{Калининградский филиал Института проблем информатики РАН, ser-list-post@yandex.ru}

%\vspace*{-4pt}


  \Abst{Рассмотрен один из подходов к созданию интеллектуальной компьютерной 
системы поддержки принятия решений (КСППР) с самоорганизацией на основе анализа целей 
экспертов. Приведен алгоритм определения типа архитектуры многоагентной системы по 
степени согласованности взаимодействия агентов, что актуально для определения 
эффективности работы групп экспертов и улучшения качества принятия решений.}
  
  \KW{компьютерная система поддержки принятия решений; многоагентная система с 
самоорганизацией; мера сходства нечетких целей агентов; алгоритм определения типа 
архитектуры многоагентой системы по степени согласованности взаимодействия агентов}

      \vskip 18pt plus 9pt minus 6pt

      \thispagestyle{headings}

      \begin{multicols}{2}

      \label{st\stat}
  
\section{Введение}
  
  При подготовке коллективных решений в сис\-те\-мах поддержки принятия
  решений (\mbox{СППР}), особенно в сложных, уникальных ситуациях, для поиска 
вариантов решения задачи или ее %\linebreak 
частей\linebreak привлекаются эксперты-консультанты. %\linebreak 
Эффективным средством повышения уровня информационного обеспечения 
руководителей и экспертов при подготовке и принятии коллективных решений 
служат \mbox{КСППР}~[1].
  
  Единого определения \mbox{КСППР} не существует. Обобщая~[1--4], \mbox{КСППР} 
можно рассматривать как человеко-машинный объект для рациональной 
организации и реализации процесса коллективного обсуждения, использования 
данных, знаний, объективных и субъективных моделей анализа и решения 
слабоструктурированных и неструктурированных проблем.
  
  Однако даже в условиях компьютеризации подбор участников, выявление и 
согласование их целей остаются за лицом, принимающим решения (ЛПР). 
Поэтому и результаты работы СППР во многом зависят от опыта, знаний и 
личностных особенностей ЛПР.
  
  В этой связи актуальны интеллектуальные \mbox{КСППР}, способные на основе 
анализа ситуации взаимодействия участников в группе консультировать ЛПР о 
необходимости изменения состава участников, переопределения и 
согласования их целей с предпочтениями ЛПР. Для подобных \mbox{КСППР} важно 
исследование группового взаимодействия как единомышленников, согласных 
друг с другом, так и конкурентов, спорящих о правильности решения. 
Подобные \mbox{КСППР} могли бы имити\-ро\-вать работу ЛПР по анализу текущей 
ситуации, вычислять сходство позиций участников группы и выбирать 
стратегию дальнейшего поведения \mbox{КСППР} при решении сложной задачи.

\vspace*{-6pt}
  
\section{Организация и~самоорганизация в~системах поддержки 
принятия решений}
\vspace*{-2pt}
  
  Принятие решений в СППР, например на военных советах, коллегиях 
министерств, совещаниях, в информационных центрах, отличается от принятия 
индивидуальных решений. В~[5] отмечается, что каждый участник группы 
преследует собственные цели, которые могут совпадать или вступать в 
противоречие с целями других участников. Концептуальная модель СППР~[6] 
представлена на рис.~\ref{f1kir}. Стрелки, связывающие экспертов, показывают 
их многогранное взаимодействие.
  
  Какая-то часть экспертов подчиняется одному или нескольким другим 
экспертам по службе, т.\,е.\ могут существовать отношения под\-чи\-нен\-ности, 
образующие организационную структуру \mbox{СППР}. Взаимодействуя в ходе 
обсуждения, эксперты обмениваются данными, знаниями, объяснениями и 
частичными решениями общей задачи.
  
\begin{figure*} %fig1
\begin{center}
\vspace*{1pt}
\mbox{%
\epsfxsize=102.136mm
\epsfbox{kir-1.eps}
}
\end{center}
\vspace*{-12pt}
\Caption{Модель принятия решений в СППР
\label{f1kir}}
\vspace*{-3pt}
\end{figure*}
  
  Могут существовать группы экспертов, не связанных подчиненностью. 
Среди них могут быть явные или неявные лидеры, что в еще большей степени 
<<обостряет>> неоднородность коллективного принятия решений. При этом 
возникают процессы самоорганизации, направляемые отношениями\linebreak 
кооперации, компромисса, содействия, конкуренции, конформизма (подобия), 
приспособления, солидарности, уклонения и~др. Самоорганизация~--- основа 
интенсивного развития компании, ее\linebreak способность чутко реагировать на 
изменения во внешней среде, обоснованно и своевременно корректируя не 
только свое внешнее поведение, но и основополагающие принципы 
собственного устройства и функционирования.
  
  Каждый член группы (эксперт или ЛПР) выслушивает других участников и 
высказывает свое мнение. Процесс принятия решения в СППР~--- поиск 
компромисса, управляемый ЛПР. Цель такого поиска~--- найти <<резонансное 
состояние>> хода обсуждения в СППР, следствием которого стало бы 
возникновение синергетического эффекта, когда коллективное, 
интегрированное решение оказывается качественно лучше и лишено 
недостатков частных мнений экспертов.
  
  Каковы условия возникновения этого эффекта? Как он зависит от сходства 
или различия целей участников СППР? Для ответа на эти вопросы выделим 
возможные типы отношений между участниками: \textit{конкуренцию}, при 
которой достижение цели одним участником исключает возможность 
достижения цели другим; \textit{сотрудничество}, когда цели участников 
совпадают; \textit{нейтралитет}, если цели не совпадают, но и не 
противоречат. В зависимости от типа отношений между участниками их
коллектив можно определить как (1)~\textit{группу сотрудников}, 
состоящую только из сотрудничающих и нейтральных участников; здесь 
полностью отсутствует конкуренция; (2)~\textit{группу нейтралов}, в которой 
присутствуют %\linebreak
 только нейтральные отношения; (3)~\textit{группу конкурентов}, 
в которой есть хотя бы одна пара участников-конкурентов; здесь могут 
присутствовать нейтральные и сотрудничающие участники; при наличии %\linebreak 
сотрудничающих участников их можно рассматривать как коалицию, 
подгруппу, представляемую единым мнимым участником (ниже такого 
мнимого участника будем называть <<суперагентом>>).
  
  Успех работы СППР зависит от знаний и опыта ЛПР по организации 
процесса поиска решения, т.\,е.\ взаимодействия экспертов. Возможно, на 
определенном этапе обсуждения ЛПР нужно, чтобы эксперты, конкурируя 
между собой, как можно шире исследовали область допустимых решений, а на 
другом~--- чтобы они совместно обосновали выбор одного из них. Таким 
образом, ЛПР должно владеть способами определения (классификации, 
распознавания) вида взаимодействия в СППР и возможностями его изменения. 
Отсюда важно, чтобы \mbox{КСППР} вела мониторинг взаимодействия экспертов и 
консультировала ЛПР о не\-об\-хо\-ди\-мости своевременной замены одного типа 
отношений участников на другой.

  \begin{figure*} %fig2
    \begin{center}
\vspace*{1pt}
\mbox{%
\epsfxsize=116.917mm
\epsfbox{kir-2.eps}
}
\end{center}
\vspace*{-9pt}
\Caption{Фрагмент схемы концептуальных моделей предметной области
\label{f2kir}}
%\vspace*{-6pt}
\end{figure*}
  
\vspace*{-6pt}

\section{Концептуальный базис моделирования коллективных~решений}

\vspace*{-2pt}
  
  Описание отношений участников группы и их взаимодействия выполним на 
базе концептуальной модели~[6].
  
  \smallskip
  \noindent
  \textbf{Определение 1.} \textit{Концептуальная модель~--- модель 
предметной области, состоящая из перечня понятий, используемых для ее 
описания, вместе со свойствами и характеристиками, классификацией этих 
понятий и законов протекания процессов в ней}~[7].
  
  \smallskip
  
  \noindent
  \textbf{Определение 2.} \textit{Схема концептуальных моделей~--- 
совокупность классов понятий и отношений между ними, определяющая 
состав и структуру концептуальных моделей предметной области}.
  
  \smallskip
  
  Назначение схемы концептуальных моделей~--- структурирование и 
извлечение знаний о предметной области. Это некий шаблон для разработчика, 
позволяющий получать от экспертов или извлекать из профессиональных 
текстов информацию о предметной области, записывая ее и храня в удобном 
для компьютерной обработки виде. При построении концептуальной модели 
классы понятий и отношений схемы концептуальных моделей наполняются 
понятиями и отношениями предметной области.
  
  Построение концептуальных моделей выполним на основе схемы 
концептуальных моделей~[6], содержащей 11~категорий концептов. 
В~настоящей статье используются категории: 
  \begin{enumerate}[(1)]
  \item ресурсов~--- понятий, отображающих вещи, имеющиеся у субъекта для 
решения задач (множество ресурсов обозначим  $\mathrm{RES} =$\linebreak $=\{\mathrm{res}_1,\ldots , 
\mathrm{res}_{N_{\mathrm{res}}}\}$); 
  \item свойств~--- всего того, что не является границами ресурса, характеризуя 
ресурс, не образует новых ресурсов ($\mathrm{PR} =\{ \mathrm{pr}_1,\ldots , \mathrm{pr}_{N_{\mathrm{pr}}}\}$); 
  \item действий~--- понятий, обозначающих отношения на ресурсах как 
следствие деятельности ($\mathrm{ACT} =\{\mathrm{act}_1, \ldots , \mathrm{act}_{N_{\mathrm{act}}}\}$); 
  \item значений~--- понятий или чисел, по\-ка\-зы\-ва\-ющих количество единиц 
измерения свойства ($\mathrm{VAL} = \{ \mathrm{val}_1,\ldots , \mathrm{val}_{N_{\mathrm{val}}}\}$).
  \end{enumerate}
  
  Так, рассматривая систему <<СППР\,--\,объект управления>>, множество 
субъектов управления~(su), т.\,е.\ участников СППР, обозначим $\mathrm{RES}^{\mathrm{su}} 
=\{\mathrm{res}_1^{\mathrm{su}}, \ldots , \mathrm{res}^{\mathrm{su}}_{N_{\mathrm{res}}^{\mathrm{su}}}\}$, 
а множество объектов 
управления (ОУ, ou), воспринимающих воздействия СППР,~--- $\mathrm{RES}^{\mathrm{ou}} 
=\{\mathrm{res}_1^{\mathrm{ou}}, \ldots , \mathrm{res}^{\mathrm{ou}}_{N_{\mathrm{res}}^{\mathrm{ou}}}\}$, 
причем $\mathrm{RES}^{\mathrm{su}}, 
\mathrm{RES}^{\mathrm{ou}}\subseteq \mathrm{RES}$; множество целей\linebreak субъектов управления~---
$\mathrm{PR}^{\mathrm{gsu}} = \{\mathrm{pr}_1^{\mathrm{gsu}},\ldots$\linebreak $\ldots , \mathrm{pr}^{\mathrm{gsu}}_{N_{\mathrm{pr}}^{\mathrm{gsu}}}\}$, 
$\mathrm{PR}^{\mathrm{gsu}}\subseteq \mathrm{PR}$; 
множество их субъективных ценностей~--- $\mathrm{PR}^{\mathrm{csu}} =\{\mathrm{pr}_1^{\mathrm{csu}}, \ldots , 
\mathrm{pr}^{\mathrm{csu}}_{N_{\mathrm{pr}}^{\mathrm{csu}}}\}$, $\mathrm{PR}^{\mathrm{csu}}\subseteq \mathrm{PR}$; 
множество состояний 
ОУ~--- $\mathrm{PR}^{\mathrm{pou}}\;=$\linebreak $=\;\{\mathrm{pr}_1^{\mathrm{pou}}, \ldots , \mathrm{pr}^{\mathrm{pou}}_{N_{\mathrm{pr}}^{\mathrm{pou}}}\}$, 
%$\mathrm{PR}^{\mathrm{pou}}\subseteq \mathrm{PR}$, 
$\mathrm{PR}^{\mathrm{pou}}\subseteq \mathrm{PR}$; множество свойств ОУ~--- 
$\mathrm{PR}^{\mathrm{ou}} =\{\mathrm{pr}_1^{\mathrm{ou}}, \ldots , 
\mathrm{pr}^{\mathrm{ou}}_{N_{\mathrm{pr}}^{\mathrm{ou}}}\}$, 
$\mathrm{PR}^{\mathrm{ou}}\subseteq  \mathrm{PR}$, множество интервалов времени действия~--- 
$\mathrm{PR}^t = \{\mathrm{pr}_1^t, \ldots , \mathrm{pr}^t_{N^t_{\mathrm{pr}}}\}$, 
$\mathrm{PR}^t\subseteq \mathrm{PR}$; множество действий субъектов управ\-ле\-ния~--- 
$\mathrm{ACT}^{\mathrm{su}} =\{\mathrm{act}_1^{\mathrm{su}}, \ldots , \mathrm{act}^{\mathrm{su}}_{N^{\mathrm{su}}_{\mathrm{act}}}\}$, 
$\mathrm{ACT}^{\mathrm{su}}\subseteq \mathrm{ACT}$; множество последовательностей их действий~--- 
$\mathrm{ACT}^{\mathrm{dsu}} =\{\mathrm{act}_1^{\mathrm{dsu}}, \ldots , \mathrm{act}^{\mathrm{dsu}}_{N^{\mathrm{dsu}}_{\mathrm{act}}}\}$, 
$\mathrm{ACT}^{\mathrm{dsu}}\subseteq \mathrm{ACT}$; множество значений свойств ОУ~--- $\mathrm{VAL}^{\mathrm{ou}} 
=$\linebreak 
$=\;\{ \mathrm{val}_1^{\mathrm{ou}}, \ldots , \mathrm{val}_{N^{\mathrm{ou}}_{\mathrm{val}}}^{\mathrm{ou}}\}$, 
$\mathrm{VAL}^{\mathrm{ou}}\subseteq \mathrm{VAL}$.
  
  Фрагмент схемы концептуальных моделей предметной области, 
включающий перечисленные категории концептов и отношений, представлен 
на рис.~\ref{f2kir}.
  

  
  Здесь выделены следующие классы отношений: <<\textit{ресурс--ресурс}>> 
$R^{\mathrm{res},\mathrm{res}}$; <<\textit{свойство--свойство}>> $R^{\mathrm{pr},\mathrm{pr}}$; 
<<\textit{действие--действие}>> $R^{\mathrm{act},\mathrm{act}}$ и другие. Для моделирования 
целеполагания в СППР особое значение имеют классы отношений: 
<<\textit{ресурс--свойство}>> $R^{\mathrm{res},\mathrm{pr}}$; 
  <<\textit{свойство--свойство}>>~$R^{\mathrm{pr},\mathrm{pr}}$; 
  <<\textit{свойство--действие}>>~$R^{\mathrm{pr},\mathrm{act}}$; 
  <<\textit{дейст\-вие--свойст\-во}>>~$R^{\mathrm{act},\mathrm{pr}}$; 
  <<\textit{дейст\-вие--дейст\-вие}>>~$R^{\mathrm{act},\mathrm{act}}$; 
  <<\textit{свойство--зна\-че\-ние}>>~$R^{\mathrm{pr},\mathrm{val}}$.
  
  Разработка концептуальной модели по схеме концептуальных моделей~--- 
первый этап системного анализа сложной задачи, необходимый для извлечения 
знаний. Он предшествует исследованию ее составных частей (подзадач) 
ограниченным набором методов: аналитических, статистических, логических, 
лингвистических, нечетких~--- каждый\linebreak из которых <<работает>> со своим 
типом переменных: детерминированными, стохастическими, логическими, 
лингвистическими четкими, нечеткими лингвистическими соответственно. 
Ниже\linebreak концепты <<свойство>> (свойства ОУ и экспертов, принимающих 
участие в работе СППР) будут представлены переменными, а отношения между 
участниками СППР~--- функциями.
  
  В терминах рассмотренного концептуального базиса выполним 
моделирование взаимодействий между экспертами в СППР. Разработка метода 
определения степени сплоченности группы, а в дальнейшем тестирование 
эффективности работы различных коллективов будут идти на примере 
интеллектуальной многоагентной системы (МАС)~--- компьютерной модели 
коллектива экспертов. Это позволит исключить посторонние факторы, 
например неискренность участников, не\-отъем\-ле\-мо присутствующие в 
человеческих коллективах и влияющие на результаты исследований.
  
\section{Знания и цели коллектива интеллектуальных агентов}
  
  Основа многоагентного подхода~--- распределенный, социальный интеллект 
взаимо\-дей\-ст\-ву\-ющих интеллектуальных систем (подсистем, агентов) как 
противопоставление классическому подходу в искусственном интеллекте, 
согласно которому единственная интеллектуальная система должна обладать 
глобальным видением сложной проблемы, иметь все необходимые 
способности, знания и ресурсы для ее решения. В работе~\cite{8kir} МАС~--- 
это совокупность четырех объектов:
    \begin{equation}
\mathrm{MAS} = (A,\,E,\,\mathbf{CL},\,\mathrm{ORG})\,,
\label{e1kir}
\end{equation}
где $A$~--- множество агентов, отображающих участников СППР (экспертов и 
ЛПР) на рис.~\ref{f1kir}; $E =$\linebreak $=\;\{e\}$~--- среда, в которой находится МАС, т.\,е.\
отображение внешней среды СППР; \textbf{CL}~--- межагентные отношения, 
изображенные стрелками на рис.~\ref{f1kir}; ORG~--- множество базовых 
архитектур МАС.
  
  В МАС агент имеет частичное представление о задаче и ограниченные 
ресурсы для ее решения, поэтому в сложных ситуациях требуется 
взаимодействие агентов, неотделимое от организации МАС. Способ 
организации взаимодействия агентов в МАС определяется ее базовой 
архитектурой.
  
  В зависимости от степени согласованности взаимодействия агентов базовую 
архитектуру МАС будем относить к одному из трех типов: 
  \begin{enumerate}[(1)]
  \item
  \textit{МАС с сотрудничающими агентами}, которая состоит только из 
сотрудничающих и нейт\-раль\-ных агентов и где полностью от\-сут\-ст\-ву\-ют 
отношения конкуренции (множество\linebreak таких базовых архитектур МАС 
обозначим $\mathrm{ORG_{coop}} \subseteq \mathrm{ORG}$); 
  \item
  \textit{МАС с нейтральными агентами}, где присутст\-вуют только 
нейтральные отношения; такая\linebreak ситуация возможна, например, если цели 
агентов определены в различных независимых измерениях многомерного 
пространства свойств ОУ (множество таких базовых архитектур МАС 
обозначим $\mathrm{ORG}_{\mathrm{neut}} \subseteq \mathrm{ORG}$); 
  \item
  \textit{МАС с конкурирующими агентами}, в которой есть хотя бы одна пара 
агентов с отношением конкуренции; в таких МАС могут быть также 
нейтральные и сотрудничающие агенты; при наличии сотрудничающих агентов 
они могут рассматриваться как единый <<суперагент>>, и тогда все агенты 
станут только конкурирующими или нейтральными (множество таких базовых 
архитектур МАС обозначим $\mathrm{ORG}_{\mathrm{comp}} \subseteq \mathrm{ORG}$).
  \end{enumerate}
  
  \noindent
  \textbf{Определение 3.} \textit{Агент~--- аппаратно или (что\linebreak встречается 
чаще) программно реализованная вы\-чис\-лительная система, обладающая 
свойствами автономности, коммуникабельности, активности, реактивности, 
а также <<ментальными свойствами>>}~\cite{9kir}.
  
  \smallskip
  
  Под автономностью понимается способность агента функционировать без 
вмешательства человека, осуществлять самоконтроль над своими действиями и 
внутренним состоянием. Коммуникабельность агента~--- умение общаться с 
другими агентами, а возможно и с человеком, посредством некоторого языка. 
Реактивность~--- свойство агента воспринимать окружающую среду и 
реагировать на ее изменения. Под активностью агента понимается его 
способность не только реагировать на внешние события, но и предпринимать 
самостоятельные действия по достижению своих целей. <<Ментальными 
свойствами>> (интенциональными понятиями) называются компоненты 
(знания, цели, желания, намерения, убеждения, обязательства и~т.\,п.) агента, 
которые отвечают за формирование предпочтений агента при выборе стратегий 
своего поведения. Ключевые ментальные свойства, на основе которых 
формируются все остальные,~--- это знания и цели.
  
  Знания~--- часть информации агента о себе, среде и других агентах, не 
меняющаяся в процессе его функционирования~\cite{10kir}.
  
  Цель~--- положение вещей, которое стремится достичь ЛПР и имеющее для 
него определенную субъективную ценность~\cite{5kir}. В~\cite{11kir}
цель~--- идеальное предвосхищение результата деятельности, выступающее ее 
регулятором, а в~\cite{12kir}~--- ситуация или область ситуаций, которая 
должна быть достигнута при функционировании системы за определенное 
время. Обобщая эти определения, выделим основные характеристики цели: она 
представляет состояние ОУ, является регулятором деятельности, имеет 
темпоральный характер (функция времени), субъективно полезна для ЛПР.
  
  \smallskip
  
  \noindent
  \textbf{Определение 4.} \textit{Цель $\mathrm{pr}^{\mathrm{gsu}}$ агента как субъекта 
управления $\mathrm{res}^{\mathrm{su}}$~--- состояние $\mathrm{pr}^{\mathrm{pou}}$ объекта 
управ\-ле\-ния~$\mathrm{res}^{\mathrm{ou}}$, 
имеющее для агента ценность (полезность) $\mathrm{pr}^{\mathrm{csu}}$, определяющее его 
деятельность (последовательность действий) $\mathrm{act}^{\mathrm{dsu}}$ и которое должно 
быть достигнуто за определенный промежуток времени~$\mathrm{pr}^t$.}
  
  \smallskip
  Схема концептуальных моделей цели может быть записана в виде
  \begin{multline*}
\mathrm{pr}^{\mathrm{gsu}} = R^{\mathrm{res},\mathrm{pr}}(\mathrm{res}^{\mathrm{ou}},\mathrm{pr}^{\mathrm{pou}}) \wedge 
R^{\mathrm{pr},\mathrm{pr}}(\mathrm{pr}^{\mathrm{pou}},\mathrm{pr}^{\mathrm{csu}})\wedge\\
\wedge R^{\mathrm{res},\mathrm{pr}}(\mathrm{res}^{\mathrm{su}},\mathrm{pr}^{\mathrm{csu}})\wedge 
R^{\mathrm{pr},\mathrm{act}}(\mathrm{pr}^{\mathrm{pou}}, \mathrm{act}^{\mathrm{dsu}})\,,
\end{multline*}
где $R^{\mathrm{pr},\mathrm{pr}}$~--- отношения <<\textit{свойство--свойство}>>, которые определяют 
субъективную полезность состояния ОУ, а $R^{\mathrm{pr},\mathrm{act}}$~--- отношения 
<<\textit{свойство--действие}>>, ставящие в соответствие целевому\linebreak состоянию 
последовательность действий
  \begin{multline*}
  \mathrm{act}^{\mathrm{dsu}} ={}\\
  {}= R^{\mathrm{act,act}}(\mathrm{ACT}^{\mathrm{su}},\mathrm{ACT}^{\mathrm{su}})\wedge 
R^{\mathrm{act,pr}}(\mathrm{ACT}^{\mathrm{su}},\mathrm{PR}^t)\,,
  \end{multline*}
    где $\mathrm{ACT}^{\mathrm{su}}$~--- множество возможных действий агента; $R^{\mathrm{act,act}}$~--- 
отношения <<\textit{действие--действие}>>, устанавливающие порядок действий 
$\mathrm{act}^{\mathrm{su}}\in \mathrm{ACT}^{\mathrm{su}}$ в 
последовательности~$\mathrm{act}^{\mathrm{dsu}}$, $R^{\mathrm{act,pr}}$~--- отношения 
<<\textit{действие--свойство}>>, определяющие время выполнения действий 
из~$\mathrm{ACT}^{\mathrm{su}}$.
  
  Состояние pr$^{\mathrm{pou}}$  объекта res$^{\mathrm{ou}}$ определяется значениями его 
свойств
  \begin{multline*}
\mathrm{pr}^{\mathrm{pou}} ={}\\
  {}=
   R^{\mathrm{res,pr}} (\mathrm{res}^{\mathrm{ou}},\mathrm{PR}^{\mathrm{ou}})\wedge 
R^{\mathrm{pr,val}}(\mathrm{PR}^{\mathrm{ou}},\mathrm{VAL}^{\mathrm{ou}})\,,
  \end{multline*}
где $R^{\mathrm{res,pr}}$~--- отношения <<\textit{ресурс--свойство}>>, за\-да\-ющие 
множество свойств ОУ, а $R^{\mathrm{pr,val}}$~--- отношения 
<<\textit{свойство--значение}>>, каждому из свойств ОУ ставящие в 
соответствие множество значений. Одним из свойств во множестве~PR$^{\mathrm{ou}}$ 
может быть время, связанное с функционированием ОУ. Тогда цель агента по 
определению~4 также становится динамической (изменяется во времени).
  
  Поскольку, как отмечалось выше, свойства ОУ представляются переменными 
при записи причинно-следственных связей в том или ином методе 
моделирования, то в целеполагании мо\-жет использоваться несколько 
инструментариев.\linebreak Это обусловливает сложности моделирования принятия 
решения, если требуется сопоставлять\linebreak подцели, описанные разными методами. 
Такая ситуация возникает, например, если существуют решения, оптимальные 
по Парето, и требуется выбрать только одно из них. Допустим, есть ОУ с двумя 
свойствами pr$_1^{\mathrm{ou}}$ и~pr$_2^{\mathrm{ou}}$, а также два состояния объекта 
управления pr$_1^{\mathrm{pou}}$ и~pr$_2^{\mathrm{pou}}$, причем pr$_1^{\mathrm{pou}}$ ближе к 
целевому состоянию pr$^{\mathrm{pou}}$, чем pr$_2^{\mathrm{pou}}$, по первому 
критерию~pr$_1^{\mathrm{ou}}$, а pr$_2^{\mathrm{pou}}$~--- по второму критерию~pr$_2^{\mathrm{ou}}$. Если 
свойства представлены разными переменными (например, стохастической и 
нечеткой лингвистической), обрабатываемыми разными методами, выбрать 
одно из решений будет не\-прос\-то, так как оперировать переменными различных 
типов можно только в интегрированных средах. Однако если свойства 
характеризуются переменными одного типа, можно задать метрику в двухмерном 
пространстве векторов, представляющих допустимые состояния ОУ, и 
определить расстояние между~pr$_1^{\mathrm{pou}}$ и~pr$^{\mathrm{pou}}$, а также между 
pr$_2^{\mathrm{pou}}$ и~pr$^{\mathrm{pou}}$, после чего сравнить их между собой. Чтобы 
избежать подобных ситуаций, выберем один метод пред\-став\-ле\-ния всех свойств, 
определяющих состояние ОУ, а значит, и используемых при описании целей 
ЛПР и агентов.
  
  Анализ показал, что наиболее удобен для этого аппарат теории нечетких 
множеств. Он адекватно учитывает все виды неопределенности в СППР и 
сводит воедино всю имеющуюся неоднородную информацию~\cite{13kir}.
  
  \smallskip
  
  \noindent
  \textbf{Определение 5.} \textit{Нечеткая цель агента~$\mathrm{pr}^{\mathrm{gsu}}$~---\linebreak нечеткое 
множество, заданное на множестве\linebreak со\-сто\-яний объекта управления 
$\mathrm{PR}^{\mathrm{pou}}\subseteq \mathrm{PR}$, с функцией принадлежности~$\mu^{\mathrm{pr}^{\mathrm{gsu}}} 
(\mathrm{pr}^{\mathrm{pou}})$, или для крат\-кости}~$\mu^{\mathrm{gsu}}(\mathrm{pr}^{\mathrm{pou}})$.

  \smallskip

  Функция принадлежности $\mu^{\mathrm{gsu}}(\mathrm{pr}^{\mathrm{pou}})$ принимает значения на 
множестве~[0;\,1]. При этом чем выше ее значение, тем ближе состояние 
объекта управления~$\mathrm{pr}^{\mathrm{pou}}$ к цели агента~$\mathrm{pr}^{\mathrm{gsu}}$. В~общем случае 
состояние~$\mathrm{pr}^{\mathrm{pou}}$ описывается набором его свойств $\mathrm{PR}^{\mathrm{ou}} 
=\{\mathrm{pr}_1^{\mathrm{ou}}, \ldots , \mathrm{pr}^{\mathrm{ou}}_{N_{\mathrm{pr}}^{\mathrm{ou}}}\}$, 
представленных переменными 
одного из перечисленных в разд.~3 типов, т.\,е.
  \begin{equation}
  \mu^{\mathrm{gsu}} (\mathrm{pr}^{\mathrm{pou}}) = \mu^{\mathrm{gsu}}(\mathrm{pr}_1^{\mathrm{ou}}, \ldots , 
\mathrm{pr}^{\mathrm{ou}}_{N_{\mathrm{pr}}^{\mathrm{ou}}})\,.
  \label{e2kir}
  \end{equation}
  
  Значение функции принадлежности определяется подстановкой 
в~(\ref{e2kir}) значений из множества~$\mathrm{VAL}^{\mathrm{ou}}$ свойств ОУ для данного 
состояния, т.\,е.\ описывается выражением ~$\mu^{\mathrm{gsu}} (\mathrm{val}_1^{\mathrm{ou}},\ldots , 
\mathrm{val}^{\mathrm{ou}}_{N^{\mathrm{ou}}_{\mathrm{val}}})$.
  
  Нечеткая цель агента может быть представлена одним из рассмотренных 
в~\cite{14kir} методов по\-стро\-ения функций принадлежности нечетких 
множеств. Выбор метода остается за разработчиком \mbox{КСППР}. В~\cite{14kir} они 
разделены на два класса: прямые, когда функция принадлежности~$\mu^{\mathrm{gsu}}$ 
задается разработчиком \mbox{КСППР} непосредственно с помощью таб\-лиц, формул 
или примеров, и косвенные, когда функция принадлежности~$\mu^{\mathrm{gsu}}$ 
определяется после обработки экспертных оценок по определенному 
алгоритму. Прямые методы рекомендуется применять, если свойства ОУ 
измеримы, в противном случае более подходят косвенные методы. Ниже для 
записи причинно-следственных связей между целями и отношениями 
взаимодействия агентов используются прямые методы построения нечеткой 
цели. Рассмотрев понятие нечеткой цели, вернемся к вопросу сопоставления 
целей агентов и введем меру близости нечетких целей.
  
\section{Мера сходства нечетких целей агентов}
  
  Один из вариантов определения степени бли\-зости целей агентов~--- 
рассчитать расстояние Евклида или Хэмминга между нечеткими множествами. 
Пусть нечеткие множества~$A$ и~$B$ представляют цели агентов~$A$ и~$B$ 
соответственно. Евклидово расстояние между дискретными конечными 
нечеткими множествами~$A$ и~$B$ в одномерном пространстве (когда 
состояние~$\mathrm{pr}^{\mathrm{pou}}$ объекта управления описывается единственным 
свойством~$\mathrm{pr}^{\mathrm{ou}}$, при\-ни\-ма\-ющим значения  $\mathrm{val}_i^{\mathrm{ou}} \in
\mathrm{VAL}^{\mathrm{ou}}$, 
определяется формулой~\cite{15kir, 16kir}
  \begin{multline}
  d(A,B) = \sum\limits_{i=1}^n \left| \mu_A^{\mathrm{gsu}}\left(\mathrm{val}_i^{\mathrm{ou}}\right)-
\mu_B^{\mathrm{gsu}}\left(\mathrm{val}_i^{\mathrm{ou}}\right)\right| \\
 \mathrm{val}_i^{\mathrm{ou}} \in \mathrm{VAL}^{\mathrm{ou}}\,,
  \label{e3kir}
  \end{multline}
а между непрерывными бесконечными множествами~$A$ и~$B$~--- 
выражением
\begin{multline}
d(A,B) =  \int\limits_{\mathrm{val}_{\min}^{\mathrm{ou}}}^{\mathrm{val}^{\mathrm{ou}}_{\max}} \left\vert 
\mu_A^{\mathrm{gsu}}\left(\mathrm{pr}^{\mathrm{ou}}\right) 
-\mu_B^{\mathrm{gsu}}\left(\mathrm{pr}^{\mathrm{ou}}\right)\right\vert 
d(\mathrm{pr}^{\mathrm{ou}})\,,\\
\mathrm{pr}^{\mathrm{ou}}\in PR^{\mathrm{ou}}\,,\quad \mathrm{val}_{\min}^{\mathrm{ou}},\,\mathrm{val}_{\max}^{\mathrm{ou}}
 \in \mathrm{VAL}^{\mathrm{ou}}\,,
\label{e4kir}
\end{multline}
    где $\mathrm{val}_{\min}^{\mathrm{ou}}$ и $\mathrm{val}_{\max}^{\mathrm{ou}}$~--- минимальное и максимальное 
значение свойства~$\mathrm{pr}^{\mathrm{ou}}$ соответственно. Расстояние Хэмминга между 
дискретными конечными нечеткими множествами~$A$ и~$B$ определяется в виде
\begin{multline}
e(A,B) =  \sqrt{\sum\limits_{i=1}^n \left (\mu_A^{\mathrm{gsu}}\left(\mathrm{val}_i^{\mathrm{ou}}\right) -
\mu_B^{\mathrm{gsu}}\left(\mathrm{val}_i^{\mathrm{ou}}\right)\right )^2}\,, \\
 \mathrm{val}_i^{\mathrm{ou}}\in \mathrm{VAL}^{\mathrm{ou}}\,,
\label{e5kir}
\end{multline}
а между непрерывными бесконечными множествами~$A$ и~$B$ как
\begin{multline}
e(A,B) = {}\\
{}=
\sqrt{\int\limits_{\mathrm{val}_{\min}^{\mathrm{ou}}}^{\mathrm{val}_{\max}^{\mathrm{ou}}} \left 
(\mu_A^{\mathrm{gsu}}(\mathrm{pr}^{\mathrm{ou}})-\mu_B^{\mathrm{gsu}}(\mathrm{pr}^{\mathrm{ou}})\right )^2 
d(\mathrm{pr}^{\mathrm{ou}})}\,,\\ 
\mathrm{pr}^{\mathrm{ou}}\in \mathrm{PR}^{\mathrm{ou}}\,,\ \ \mathrm{val}_{\min}^{\mathrm{ou}},\,\mathrm{val}_{\max}^{\mathrm{ou}}
\in \mathrm{VAL}^{\mathrm{ou}}\,.
\label{e6kir}
\end{multline}

  Обратим внимание на то, что в~(\ref{e3kir}) и~(\ref{e5kir}) фигурируют 
значения $\mathrm{val}_i^{\mathrm{ou}}\in \mathrm{VAL}^{\mathrm{ou}}$, а в~(\ref{e4kir}) и~(\ref{e6kir})~--- 
свойство~pr$^{\mathrm{ou}}$. Дело в том, что при вычислении суммы в~(\ref{e3kir}) 
и~(\ref{e5kir})  используются значения функции принадлежности для каждого 
значения $\mathrm{val}_i^{\mathrm{ou}}\in \mathrm{VAL}^{\mathrm{ou}}$ свойства~pr$^{\mathrm{ou}}$. При вычислении же 
интегралов в~(\ref{e4kir}) и~(\ref{e6kir})  нужно сначала аналитически 
определить первообразную и лишь потом, подставив в формулу 
  Ньютона--Лейбница значения $\mathrm{val}^{\mathrm{ou}}_{\min}$ и~$\mathrm{val}^{\mathrm{ou}}_{\max}$, 
вычислить значение определенного интеграла.
  
  Однако применение (\ref{e3kir})--(\ref{e6kir})  к определению степени 
сходства целей агентов проблематично. Они могут быть вычислены, если ряд 
в~(\ref{e3kir}) и~(\ref{e5kir})  или интеграл в~(\ref{e4kir}) и~(\ref{e6kir}) 
сходится. Иначе при $\mathrm{val}^{\mathrm{ou}}_{\min} =-\infty$ или $\mathrm{val}^{\mathrm{ou}}_{\max} =\infty$ 
расстояние будет равно бесконечности, даже если одно множество включает 
другое.
  
  В этой связи предлагается следующая мера сходства нечетких целей агентов:
  \begin{multline}
  s(A,B) = 0{,}5\left ( \fr{\int\limits_{\mathrm{val}_{\min}^{\mathrm{ou}}}^{ \mathrm{val}_{\max}^{\mathrm{ou}}} 
\mu^{\mathrm{gsu}}_{A\cap B}(\mathrm{pr}^{\mathrm{ou}})\, d(\mathrm{pr}^{\mathrm{ou}})}
{\int\limits_{\mathrm{val}_{\min}^{\mathrm{ou}}}^{\mathrm{val}_{\max}^{\mathrm{ou}}} 
\mu^{\mathrm{gsu}}_{A}(\mathrm{pr}^{\mathrm{ou}}) \,d(\mathrm{pr}^{\mathrm{ou}})}\;+{}\right.\\[6pt]
\left.{}+\;
\fr{\int\limits_{\mathrm{val}_{\min}^{\mathrm{ou}}}^{ \mathrm{val}_{\max}^{\mathrm{ou}}} 
\mu^{\mathrm{gsu}}_{A\cap B}(\mathrm{pr}^{\mathrm{ou}})\, d(\mathrm{pr}^{\mathrm{ou}})}
{\int\limits_{\mathrm{val}_{\min}^{\mathrm{ou}}}^{\mathrm{val}_{\max}^{\mathrm{ou}}}
\mu^{\mathrm{gsu}}_{B}(\mathrm{pr}^{\mathrm{ou}})\, d(\mathrm{pr}^{\mathrm{ou}})}\right )\,.
  \label{e7kir}
  \end{multline}
  
   \begin{figure*} %fig3
    \begin{center}
\vspace*{1pt}
\mbox{%
\epsfxsize=102.278mm
\epsfbox{kir-3.eps}
}
\end{center}
\vspace*{-6pt}
\Caption{Мера сходства нечетких целей агентов
\label{f3kir}}
\vspace*{6pt}
\end{figure*}

  
  Формула~(\ref{e7kir})~--- полусумма отношений площади серой 
заштрихованной области к площади серой области и площади серой 
заштрихованной об\-ласти к площади заштрихованной области на 
рис.~\ref{f3kir}. Значения меры сходства~--- действительные числа в 
интервале~[0;\,1].
  
 
  Анализ показывает, что, в отличие от выражений~(\ref{e3kir})--(\ref{e6kir}) 
соотношение~(\ref{e7kir}) следует считать мерой сходства нечетких множеств, 
а не расстоянием между ними, так как оно не удовлетворяет некоторым из 
условий, предъявляемых к функции расстояния в математике:
  \begin{align}
  d (X,Y) &\geq 0\,; \notag\\[2pt]
  d(X,Y) & =d(Y,X)\,;\notag\\[2pt]
  d(X,Z) &\leq d(X,Y)+d(Y,Z)\,;\label{e10kir}\\[2pt]
  d(X,X) & =0\,,\label{e11kir}
  \end{align}
а именно условиям~(\ref{e10kir}) и~(\ref{e11kir}).

  Рассмотрим вычисление меры сходства нечетких целей на примере. Пусть 
имеется объект управ\-ле\-ния~$\mathrm{res}^{\mathrm{ou}}$, состояние~$\mathrm{pr}^{\mathrm{pou}}$ которого 
определяется единственным свойством~$\mathrm{pr}^{\mathrm{ou}}$, значение~$\mathrm{val}^{\mathrm{ou}}$ 
которого~--- действительное число в интервале [0;\,100], $\mathrm{val}^{\mathrm{ou}}\in \r$, 
$\mathrm{val}^{\mathrm{ou}}\in [0;\,100]$, и есть два агента~$\mathrm{res}_1^{\mathrm{su}}$ и~$\mathrm{res}_2^{\mathrm{su}}$ с 
функциями принадлежности нечетких целей 
соответственно:
  \begin{align*}
  \mu_1^{\mathrm{gsu}}\left(\mathrm{pr}^{\mathrm{pou}}\right) &=\mu_1^{\mathrm{gsu}}\left(\mathrm{pr}^{\mathrm{ou}}\right) = 
\fr{\mathrm{pr}^{\mathrm{ou}}}{100}\,;\\[2pt]
  \mu_2^{\mathrm{gsu}}\left(\mathrm{pr}^{\mathrm{pou}}\right ) &=\mu_2^{\mathrm{gsu}}\left(\mathrm{pr}^{\mathrm{ou}}\right) = \fr{100- 
\mathrm{pr}^{\mathrm{ou}}}{100}\,, %\label{e13kir}
  \end{align*}
  
  Требуется найти меру сходства~$s (1,2)$ этих нечетких целей. Определим ее 
по выражению

\noindent
  \begin{multline}
  s(1,2) ={}\\
  {}= 0{,}5\left (\fr{\int\limits_0^{100} \min \left ( 
\mu_1^{\mathrm{gsu}}\left(\mathrm{pr}^{\mathrm{ou}}\right);\,\mu_2^{\mathrm{gsu}}\left(\mathrm{pr}^{\mathrm{ou}}\right)\right )\,
d\left(\mathrm{pr}^{\mathrm{ou}}\right)}{\int\limits_0^{100} 
\mu_1^{\mathrm{gsu}}(\mathrm{pr}^{\mathrm{ou}})\,d(\mathrm{pr}^{\mathrm{ou}})}+{}\right.\\
\!\!\left.{}+\fr{\int\limits_0^{100} \min \left ( 
\mu_1^{\mathrm{gsu}}\left(\mathrm{pr}^{\mathrm{ou}}\right);\,\mu_2^{\mathrm{gsu}}\left(\mathrm{pr}^{\mathrm{ou}}\right)\right )\, 
d\left(\mathrm{pr}^{\mathrm{ou}}\right)}{\int\limits_0^{100} 
\mu_2^{\mathrm{gsu}}\left(\mathrm{pr}^{\mathrm{ou}}\right)\,d\left(\mathrm{pr}^{\mathrm{ou}}\right)}\right )\,.\!\!
  \label{e14kir}
  \end{multline}
  
  Для вычисления интеграла в числителях дробей требуется найти точки 
пересечения функций принадлежности~$\mu_1^{\mathrm{gsu}}(\mathrm{pr}^{\mathrm{ou}})$ 
и~$\mu_2^{\mathrm{gsu}}(\mathrm{pr}^{\mathrm{ou}})$:
  \begin{equation*}
  \fr{\mathrm{pr}^{\mathrm{ou}}}{100}  = \fr{100-\mathrm{pr}^{\mathrm{ou}}}{100}\,;\quad
\mathrm{pr}^{\mathrm{ou}} =50\,,
\end{equation*}
и определить, при каких значениях~$\mathrm{val}_i^{\mathrm{ou}}$ $\mu_1^{\mathrm{gsu}}\left(\mathrm{pr}^{\mathrm{ou}}\right)$ 
меньше~$\mu_2^{\mathrm{gsu}}\left(\mathrm{pr}^{\mathrm{ou}}\right)$:
\begin{multline*}
\mu_1^{\mathrm{gsu}} \left (50+\Delta \mathrm{pr}^{\mathrm{ou}}\right ) -\mu_2^{\mathrm{gsu}}(50) 
=
    \fr{50+\Delta \mathrm{pr}^{\mathrm{ou}}}{100} -{}\\
    {}- \fr{100-50}{100} =    \fr{\Delta \mathrm{pr}^{\mathrm{ou}}}{100}
  \begin{cases}
  \geq 0 &\mbox{при}\ \Delta \mathrm{pr}^{\mathrm{ou}}\geq 0\,;\\
  <0 & \mbox{при}\ \Delta \mathrm{pr}^{\mathrm{ou}} <0\,.
  \end{cases}
  \end{multline*}

Таким образом, $\mu_1^{\mathrm{gsu}}(\mathrm{pr}^{\mathrm{ou}}) < \mu_2^{\mathrm{gsu}}(\mathrm{pr}^{\mathrm{ou}})$ при 
$\mathrm{pr}^{\mathrm{ou}}<50$ и $\mu_1^{\mathrm{gsu}}(\mathrm{pr}^{\mathrm{ou}}) \geq  \mu_2^{\mathrm{gsu}}(\mathrm{pr}^{\mathrm{ou}})$ при 
$\mathrm{pr}^{\mathrm{ou}}\geq 50$. Тогда~(\ref{e14kir}) переписывается в виде
\pagebreak

\end{multicols}

%\hrule

\noindent
\begin{multline*}
s(1,2) = 0{,}5\left ( \fr{\int\limits_0^{50} \left(\mathrm{pr}^{\mathrm{ou}}/100\right)\,
d\left(\mathrm{pr}^{\mathrm{ou}}\right)+\int\limits_{50}^{100} 
\left((100-\mathrm{pr}^{\mathrm{ou}}\right)/100)\,d\left(\mathrm{pr}^{\mathrm{ou}}\right)}{\int\limits_{0}^{100}  
\left(\mathrm{pr}^{\mathrm{ou}}/100\right)\,d\left(\mathrm{pr}^{\mathrm{ou}}\right)}+{}\right.\\
\left.{}+\fr{\int\limits_0^{50} 
\left(\mathrm{pr}^{\mathrm{ou}}/100\right)\,d\left(\mathrm{pr}^{\mathrm{ou}}\right)+\int\limits_{50}^{100} \left(\left(100-
\mathrm{pr}^{\mathrm{ou}}\right)/100\right)\,d\left(\mathrm{pr}^{\mathrm{ou}}\right)}{\int\limits_0^{100} 
\left(\left(100-\mathrm{pr}^{\mathrm{ou}}\right)/100\right)\,d\left(\mathrm{pr}^{\mathrm{ou}}\right)}\right )={}\\
{}=
0{,}5\left ( \fr{0{,}5\left(\mathrm{pr}^{\mathrm{ou}}\right)^2\big\vert_0^{50} +\left(100 \mathrm{pr}^{\mathrm{ou}}-
0{,}5\left(\mathrm{pr}^{\mathrm{ou}}\right)^2\right)\big\vert_{50}^{100}}{0{,}5\left(\mathrm{pr}^{\mathrm{ou}}\right)^2\big\vert_0^{100}}+
\fr{0{,}5\left(\mathrm{pr}^{\mathrm{ou}}\right)^2\big\vert_0^{50} +\left(100 \mathrm{pr}^{\mathrm{ou}}-
0{,}5\left(\mathrm{pr}^{\mathrm{ou}}\right)^2\right)\big\vert_{50}^{100}}{\left(100 \mathrm{pr}^{\mathrm{ou}} -
0{,}5\left(\mathrm{pr}^{\mathrm{ou}}\right)^2\right)\big\vert_0^{100}}\right ) = 0{,}5.\hspace*{-7.75pt}
\end{multline*}

\medskip

\hrule

\bigskip

\begin{multicols}{2}

  Найденное значение меры сходства $s(1,2)=0{,}5$ нечетких целей двух 
агентов означает, что нечеткие цели агентов совпадают недостаточно, чтобы 
назвать их сотрудниками, но и конкурентами в чис\-том виде они не являются, 
их следует, скорее, отнести к нейтральным агентам.
  
  В следующем разделе будет дан формальный метод определения типа 
отношений между агентами на основе значения меры сходства их нечетких 
целей. Рассмотренный способ определения меры сходства нечетких целей в 
одномерном пространстве, несложно обобщить и на многомерный случай.
 
%\vspace*{-6pt}
 
\section{Определение типа отношений между агентами по степени 
согласованности их~взаимодействия}

%\vspace*{-2pt}
  
  В разд.~2 выделено три типа отношений между участниками СППР: 
конкуренция, нейтралитет и сотрудничество. Представим их нечеткими 
множествами на универсуме значений меры сходства (на множестве 
действительных чисел в интервале~[0; 1]). Очевидно, что чем выше значение 
меры сходства целей агентов~(\ref{e14kir}), тем теснее взаимодействие между 
ними. Таким образом, функция принадлежности~$\mu_{\mathrm{сотр}}(s)$
нечеткого множества 
<<сотрудничество>> должна принимать свое максимальное значение 
при $s=1$, а функция принадлежности~$\mu_{\mathrm{конк}}(s)$ нечеткого мно\-жества 
<<конкуренция>>~--- при $s = 0$. Максимум же\linebreak 
  \begin{center} %fig1
\vspace*{-6pt}
\mbox{%
\epsfxsize=66.912mm
\epsfbox{kir-4.eps}
}
\end{center}
\vspace*{4pt}
{{\figurename~4}\ \ \small{Функции принадлежности нечетких множеств типов отношений между агентами 
по степени согласованности их взаимодействия: \textit{1}~--- конкуренция; \textit{2}~--- 
нейтралитет; \textit{3}~--- сотрудничество}}
%\end{center}
\vspace*{6pt}


\bigskip
\addtocounter{figure}{1}
  
  \noindent
функции 
принадлежности~$\mu_{\mathrm{нейт}}(s)$ нечеткого  мно\-жества   <<нейтралитет>> должен быть 
равноудален от максимумов этих функции, т.\,е.\ находиться в точке\linebreak $s = 
0{,}5$. Учитывая вышесказанное, зададим нечеткие множества отношений 
конкуренции, нейтралитета и сотрудничества функциями 
принад\-леж\-ности, которые 
изображены на рис.~4:
  \begin{align*}
  \mu_{\mathrm{конк}}(s) & = \fr{1}{1+(3s)^8}\,;\\
  \mu_{\mathrm{нейтр}}(s) & = \fr{1}{1+(6(s-0{,}5))^8}\,;\\
  \mu_{\mathrm{сотр}}(s) & = \fr{1}{1+(3(s-1))^8}\,.
  \end{align*}
  
    Тип отношений между агентами по степени согласованности их 
взаимодействия представим следующей лингвистической переменной:
  

\noindent
  \begin{equation}
\mathrm{cl} = \langle \beta , T, U, G, M\rangle\,,
  \label{e18kir}
  \end{equation}
где $\beta$\;=\;\textit{тип отношений}~--- наименование линг-\linebreak вистической 
переменной; $T = \{\mbox{конкуренция}$;\linebreak нейтралитет;\ 
$\mbox{сотрудничество}\}$~--- множество ее\linebreak
 значений (терм-множество), каждое из 
которых представляет собой наименование нечеткой пе\-ременной; $U = 
[0;\,1]$~--- область определения\linebreak (универсум) нечетких переменных, входящих в 
определение лингвистической переменной; $G=$\linebreak $=\oslash$~--- синтаксическая 
процедура, описывающая процесс образования из элементов множества~$T$ 
новых термов; $M = \{\mu_{\mathrm{конк}}(s), \mu_{\mathrm{нейтр}}(s), 
\mu_{\mathrm{сотр}}(s)\}$~--- семантическая процедура, ставящая в соответствие 
каждому терму множества~$T$, а также каждому новому терму, образуемому 
процедурой~$G$, осмысленное содержание посредством формирования 
соответствующего нечеткого множества.
  
  Значение лингвистической переменной cl (\textit{тип отношений})~--- 
терм с наибольшим значением функции принадлежности. Следовательно, для 
ее вычисления нужно определить значение функции принадлежности к 
каждому нечеткому множеству, представляющему тип отношений, и сравнить 
их между собой. Нечеткое множество с максимальным значением функции 
принадлежности соответствует типу отношений, установленному между парой 
агентов. Лингвистическая переменная~cl моделирует представление ЛПР об 
отношениях между парой экспертов в СППР.
  
  Совокупность значений лингвистической переменной~cl для всех пар 
агентов МАС формирует матрицу~\textbf{CL}. По этой матрице можно 
определить степень сплоченности СППР и принять решение о внесении 
изменений в структуру СППР. Отметим, что <<\textit{тип отношений}>> 
здесь~--- название лингвистической переменной, он не связан с понятием 
<<нечеткого отношения>> в нечеткой логике.
  
  В случае если между парой конкурирующих или нейтральных агентов и 
некоторым третьим агентом установлены отношения сотрудничества, они 
должны рассматриваться как нейтральные, так как цель третьего агента 
безразлична к состоянию~ОУ. Если в МАС существует группа 
взаимодействующих агентов, которые состоят в одних и тех же отношениях со 
всеми остальными агентами МАС, она может быть представлена единым 
<<суперагентом>> (применительно к СППР~--- мнимый участник).
  
  После того как для каждой пары агентов определено значение 
лингвистической переменной~cl (\textit{тип отношений}), набор ее значений 
должен быть проанализирован, с тем чтобы определить тип архитектуры МАС 
по степени согласованности взаимодействия агентов.
  
\section{Алгоритм определения архитектуры МАС по~степени~согласованности 
взаимодействия~агентов}
  
  Пусть имеется интеллектуальная \mbox{КСППР}. Тогда в ее состав может быть 
включен элемент <<\textit{агент, принимающий решения}>>, имитирующий 
работу ЛПР по декомпозиции задачи, распределению подзадач между 
агентами, интеграции решений, полученных от каждого из агентов, анализу 
взаимодействий между участниками \mbox{КСППР}, организации их взаимодействия и~пр. 
Функция \textit{анализ взаимодействия}~$f_{\mathrm{ia}}$ используется как для 
мониторинга взаимодействия агентов-экспертов, так и для идентификации 
архитектуры МАС на основе анализа целей этих агентов. По его результатам 
<<\textit{агент, принимающий решения}>> может установить необходимость 
замены одного типа отношений агентов на другой и корректировки их целей. 
Фактически это означает, что с некоторого момента \mbox{КСППР} изменяет алгоритм 
своего функционирования, проявляя свойство самоорганизации, и переходит на 
базовую архитектуру МАС другого типа.
  
  Обобщая вышесказанное, можно предложить алгоритм выполнения 
<<\textit{агентом, принимающим решения}>> функции \textit{анализ 
взаимодействия} в интеллектуальной МАС. Исходная информация~--- цели 
агентов. Результат~--- тип базовой архитектуры МАС по степени 
согласованности взаимодействия агентов в виде строки текста, т.\,е.\ отнесение 
текущей архитектуры МАС~$\mathrm{org}_i$, $i = 1, \ldots , N$, к одному из 
множеств~$\mathrm{ORG}_{\mathrm{coop}}$, $\mathrm{ORG}_{\mathrm{neut}}$ или~$\mathrm{ORG}_{\mathrm{comp}}$, где 
$\mathrm{ORG}_{\mathrm{coop}}, \mathrm{ORG}_{\mathrm{neut}}, \mathrm{ORG}_{\mathrm{comp}} 
\subseteq \mathrm{ORG}$, а  $N$~--- число 
базовых архитектур МАС (число элементов множества~ORG в~(\ref{e1kir})). 
Алгоритм определяется следующей последовательностью шагов:
  \begin{enumerate}[(1)]
\item представить цели агентов в виде нечетких целей; для этого целевые 
значения каждого свойства, описывающего состояние ОУ, представить 
нечетким множеством;
\item исключить из рассмотрения агентов, безразличных к состоянию ОУ, чья 
нечеткая цель~--- константа;
\item для каждой пары агентов вычислить меру сходства целей в соответствии с 
выражением~(\ref{e7kir});
\item для каждой пары агентов на основании значения меры сходства целей 
определить значение лингвистической переменной \textit{тип отношений}, 
заданной выражением~(\ref{e18kir}), получив таким образом 
матрицу~\textbf{CL} ее значений;
\item для каждой пары конкурирующих (нейтральных) агентов найти агента, 
сотрудничающего с обоими агентами и заменить в матрице, полученной на 
четвертом шаге, тип отношений между этим агентом и каждым из 
конкурирующих (нейт\-раль\-ных) агентов на нейтральный; если ни для одной 
пары конкурирующих (нейт\-раль\-ных) агентов не существует ни одного агента, 
сотрудничающего с обоими агентами, перейти к~шагу~6;
\item на основе модифицированной матрицы~\textbf{CL} значений 
лингвистической переменной~cl (\textit{тип отношений}) идентифицировать 
тип архитектуры МАС в целом; если в~\textbf{CL} присутствуют только 
нейтральные отношения, архитектура МАС относится к типу МАС с 
нейтральными\linebreak агентами, если же в ней присутствуют отношения 
сотрудничества или конкуренции, архитектура МАС относится к типу МАС с\linebreak 
со\-труд\-ни\-ча\-ющи\-ми агентами или МАС с конкурирующими агентами 
соответственно.
  \end{enumerate}
  
  В результате выполнения функции \textit{анализ взаимодействий} в \mbox{КСППР} 
идентифицируется базовая архитектура МАС, что позволит <<\textit{агенту, 
принимающему решения}>> определить необходимость (или отсутствие 
таковой) ее модификации с целью повышения качества решений. Таким 
образом происходит смена организационных структур (архитектур) МАС из 
множества~ORG в выражении~(\ref{e1kir}) и самоорганизация \mbox{КСППР}.
  
  Рассмотренный алгоритм позволяет уточнить понятие 
<<\textit{самоорганизация многоагентной системы}>>.
  
  \smallskip
  
  \noindent
  \textbf{Определение 6.} \textit{Самоорганизация в МАС~--- процесс изменения 
архитектуры МАС агентом, имитиру\-ющим ЛПР и входящим в ее состав, на основе 
анализа взаимодействия других агентов с целью повышения качества принимаемых 
решений.}
  
  \smallskip
  
  В итоге МАС~(\ref{e1kir}) с самоорганизацией может быть представлена 
следующим образом:
$$
\mathrm{MAS}_S = (A^*, E, \mathbf{CL}, \mathrm{ORG}, \mathrm{ACT}, \tau)\,,
  $$
    где кроме введенных в определении~(\ref{e1kir}) объектов будем 
выделять~$A^*$~--- множество агентов МАС, включающее в себя 
<<\textit{агента, принимающего решения}>>~$a^{\mathrm{dm}}$, т.\,е.\ $A^*=\{a_1,\ldots , 
a_n,a^{\mathrm{dm}}\}$, где $n$~--- чис\-ло агентов-экспертов в МАС, ACT~--- 
множество действий агентов системы, $\tau$~--- отображение множества 
агентов~$A^*$ на множество действий агентов МАС ACT: $\mathrm{ACT}\rightarrow 
A^*$, $\mathrm{ACT}_a=\tau(a)$, причем множество действий <<\textit{агента, 
принимающего решения}>>~$a^{\mathrm{dm}}$ включает в себя \textit{анализ 
взаимодействий}~$f_{\mathrm{ia}}$, алгоритм которого был рассмотрен выше, т.\,е.\ 
$f_{\mathrm{ia}} \in \mathrm{ACT}_{a^{\mathrm{dm}}}$.
  
  Очевидно, что для принятия решения о замене одной архитектуры МАС на 
другую одних лишь сведений о текущем взаимодействии агентов МАС 
недостаточно. Для подобного решения <<\textit{агенту, принимающему 
решения}>> необходимы знания о том, какой из типов архитектур МАС 
эффективнее в тех или иных условиях. Для получения таких знаний требуется 
провести серию вычислительных экспериментов с МАС различных типов 
архитектуры. При определении типа архитектуры МАС в этой серии 
экспериментов также должен быть использован предложенный алгоритм. 
В~результате для каждого типа архитектур МАС требуется найти долю 
экспериментов от их общего числа, в которых решение, принятое МАС, 
оказалось лучше, чем любое из решений, найденных каждым агентом 
индивидуально. Таким образом, нужно вычислить, в каких МАС выше 
вероятность возникновения синергетического эффекта, при котором принятое 
\mbox{КСППР} решение будет лучше, чем любое из предложенных каждым отдельным 
агентом. Это~--- предмет дальнейших исследований.

\section{Заключение}
  
  Рассмотрен подход к созданию интеллектуальной \mbox{КСППР} с 
самоорганизацией, один из элементов которой имитирует работу ЛПР по 
организации работы коллектива экспертов над сложной задачей. Такая \mbox{КСППР} 
на основе сведений о целях участников сможет давать рекомендации ЛПР о 
необходимости изменения состава и взаимодействия участников либо 
корректировки поставленных перед ними целей.
  
  На примере МАС предложена оригинальная мера сходства нечетких целей 
интеллектуальных агентов, показаны ее преимущества при сравнении нечетких 
целей перед другими мерами близости\linebreak нечетких множеств. В соответствии с 
этой мерой выделены три типа отношений между агентами по степени 
согласованности их взаимодействия: конкуренция, нейтралитет или 
сотрудничество. Предложен алгоритм определения архитектуры МАС по 
степени согласованности взаимодействия агентов, а также многоагентая 
система с самоорганизацией.
  
  Алгоритм определения архитектуры МАС по степени согласованности 
взаимодействия агентов позволяет рассчитать данные для последующего 
сравнения архитектур МАС различных типов. Результаты сопоставления 
архитектур МАС позволят определить, в каких организациях выше вероятность 
возникновения синергетического эффекта, когда принятое коллективом 
решение будет лучше, чем любое из решений, предлагаемых отдельными 
интеллектуальными агентами. На основе этой информации интеллектуальная 
\mbox{КСППР} после распознавания класса коллектива агентов сможет предложить 
необходимые действия по совершенствованию его архитектуры, что повысит 
качество принимаемых с использованием \mbox{КСППР} решений.
  
{\small\frenchspacing
{%\baselineskip=10.8pt
\addcontentsline{toc}{section}{Литература}
\begin{thebibliography}{99}    

\bibitem{1kir}
\Au{Петровский А.\,Б.}
Компьютерная поддержка принятия решений: современное состояние и 
перспективы развития~// Системные исследования. Методологические 
проблемы. Ежегодник~/ Под ред. Д.\,М.~Гвишиани, В.\,Н.~Садовского.~--- М.: 
Эдиториал УРСС, 1996. №\,24. С.~146--178.

\bibitem{4kir} %2
\Au{Wierzbicki A.}
Types of decision support systems and polish contributions to their development~// 
User-oriented methodology and techniques of decision analysis and support~/ Eds.\ 
J.~Wessels, A.\,P.~Wierzbicki.~--- Berlin: Springer-Verlag, 1993. 
P.~158--175.

\bibitem{3kir}
\Au{Трахтенгерц Э.\,А.}
Методы генерации, оценки и согласования решений в распределенных 
системах поддержки принятия решений~// Автоматика и телемеханика, 1995. 
№\,4. С.~3--52.

\bibitem{2kir} %4
\Au{Ларичев О.\,И., Мошкович~Е.\,М.}
Качественные методы принятия решений.~--- М.: Наука, 1996.

\bibitem{5kir}
\Au{Козелецкий Ю.}
Психологическая теория решений.~--- М.: Прогресс, 1979.  503~с.

\bibitem{6kir}
\Au{Колесников А.\,В.}
Гибридные интеллектуальные сис\-те\-мы. Теория и технология разработки~/ Под 
ред. А.\,М.~Яшина.~--- СПб.: Изд-во СПбГТУ, 2001.

\bibitem{7kir}
Толковый словарь по искусственному интеллекту~/ Авторы-составители 
А.\,Н.~Аверкин, М.\,Г.~Гаазе-Рапопорт, Д.\,А.~Поспелов.~--- М.: Радио и связь, 
1992. 256~с.

\bibitem{8kir}
\Au{Тарасов В.\,Б.}
От многоагентных систем к интеллектуальным организациям: философия, 
психология, информатика.~--- М.: Эдиториал УРСС, 2002.  352~с.

\bibitem{9kir}
\Au{Wooldridge~M., Jennings~N.\,R.}
Intelligent agents: Theory and practice~// Knowledge Eng. Rev., 1995. 
Vol.~10. No.\,2. P.~115--152.

\bibitem{10kir}
\Au{Городецкий В.\,И., Грушинский~М.\,С., Хабалов~А.\,В.}
Многоагентные системы (обзор)~// Новости искусственного интеллекта, 1998. 
№\,2. С.~64--116.

\bibitem{11kir}
\Au{Сурмин Ю.\,П.}
Теория систем и системный анализ: Учебное пособие.~--- Киев: МАУП, 2003.  
368~с.

\bibitem{12kir}
\Au{Анфилатов В.\,С., Емельянов~А.\,А., Кукушкин~А.\,А.}
Сис\-тем\-ный анализ в управлении: Учебное пособие.~--- М.: Финансы и 
статистика, 2002.  368~с.

\bibitem{13kir}
\Au{Алтунин А.\,Е., Семухин М.\,В.}
Модели и алгоритмы принятия решений в нечетких условиях.~--- Тюмень: 
Изд-во Тюменского государственного университета, 2000.  352~с.

\bibitem{14kir}
Нечеткие множества в моделях управления и искусственного интеллекта~/ Под 
ред. Д.\,А.~Поспелова.~--- М.: Наука, 1986.


\label{end\stat}

\bibitem{15kir}
\Au{Кофман А.}
Введение в теорию нечетких множеств~/ Пер. с франц.~--- М.: Радио и связь, 
1982.

\bibitem{16kir}
\Au{Рыжов А.\,П.}
Элементы теории нечетких множеств и ее приложений.~--- М.: Диалог-МГУ, 
1998.  81~c.
 \end{thebibliography}
}
}
\end{multicols}