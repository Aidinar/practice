\def\stat{nuriev}

\def\tit{РЕДУЦИРОВАНИЕ СПЕКТРА МОДЕЛЕЙ ПЕРЕВОДА 
В~НАДКОРПУСНЫХ БАЗАХ ДАННЫХ}

\def\titkol{Редуцирование спектра моделей перевода в~надкорпусных 
базах данных}

\def\aut{В.\,А.~Нуриев$^1$, И.\,М.~Зацман$^2$}

\def\autkol{В.\,А.~Нуриев, И.\,М.~Зацман}

\titel{\tit}{\aut}{\autkol}{\titkol}

\index{Нуриев В.\,А.}
\index{Зацман И.\,М.}
\index{Nuriev V.\,A.}
\index{Zatsman I.\,M.}
 

%{\renewcommand{\thefootnote}{\fnsymbol{footnote}} \footnotetext[1]
%{Работа выполнена при частичной поддержке РФФИ (проект 19-07-00187-A).}}


\renewcommand{\thefootnote}{\arabic{footnote}}
\footnotetext[1]{Институт проблем информатики Федерального исследовательского 
центра <<Информатика 
и~управление>> Российской академии наук, \mbox{nurieff.v@gmail.com}}
\footnotetext[2]{Институт проблем информатики Федерального исследовательского 
центра <<Информатика 
и~управление>> Российской академии наук, \mbox{izatsman@yandex.ru}}

\vspace*{-12pt}
  
    \Abst{Описывается подход к~редуцированию спектра моделей перевода, которые 
регистрируются и~хранятся в~надкорпусных базах данных (НБД). Этот подход ориентирован 
на применение в~профессиональном (<<человеческом>>) и~машинном переводе. В~настоящее 
время информационные ресурсы НБД обеспечивают решение широкого круга задач, 
представляющих интерес для информатики, компьютерной лингвистики и~языкознания. 
В~данной статье акцент ставится на возможностях использования НБД в~интересах переводной 
деятельности~--- для редуцирования спектра моделей перевода. Как правило, в~каждом 
конкретном случае переводчик оказывается в~ситуации многовариантного выбора: ввиду 
синонимического потенциала, которым обладают естественные языки, вместо единственно 
возможного решения при переводе имеется некоторое множество относительно 
взаимозаменяемых по смыслу эквивалентов, т.\,е.\ <<веер альтернатив>>. Выбирая один 
вариант из целого ряда потенциально возможных, переводчик, чтобы максимально сузить 
диапазон поиска, руководствуется некоторыми релевантными характеристиками переводимого 
текста. Цель статьи~--- описать подход к~применению НБД, позволяющий сузить диапазон 
поиска релевантной модели перевода.}
  \KW{параллельные тексты; перевод; модели перевода; надкорпусная база данных; 
многовариантный выбор}
  
\DOI{10.14357/19922264200217} 
 
\vspace*{4pt}


\vskip 10pt plus 9pt minus 6pt

\thispagestyle{headings}

\begin{multicols}{2}

\label{st\stat}

  
  \section{Введение}
  
  \vspace*{-3pt}
  
  В последние несколько десятилетий существенно возрос научный 
и~практический интерес к~таким информационным ресурсам, как электронные 
текстовые корпусы и~лингвистические базы данных. Эти ресурсы обеспечивают 
получение нового фундаментального знания, позволяют верифицировать, 
корректировать и~дополнять научные знания, полученные в~докорпусную 
эпоху~[1--3]. Они открывают широкие возможности для 
ис\-сле\-до\-ва\-те\-лей-лин\-г\-вис\-тов, 
работа которых связана с~большими массивами текстов на 
естественном языке (см., например,~[4]). Также представители некоторых 
профессий~--- переводчики, писатели, преподаватели (русского языка 
и~литературы, иностранных языков, культуры речи и~т.\,д.)~--- получают новые 
инструменты, с~помощью которых рабочий процесс можно значительно 
компьютеризировать и~оптимизировать (см., например,~[5--7]).
  
  Разновидностью таких информационных ресурсов являются НБД (о~НДБ 
см.~[8]), функционирующие на основе текстов параллельных корпусов, входящих 
в~состав Национального корпуса русского языка (НКРЯ). Надкорпусные
 базы данных разрабатываются 
с~2012~г.\ в~Институте проб\-лем информатики Федерального исследовательского 
центра <<Информатика и~управ\-ле\-ние>> Российской академии наук 
и~пред\-став\-ля\-ют собой уникальный инструмент, который имеет ярко выраженную 
на\-уч\-но-ис\-сле\-до\-ва\-тель\-скую на\-прав\-лен\-ность, поскольку используется для 
извлечения нового лингвистического знания из параллельных текстов. Также он 
может широко применяться как в~профессиональной переводческой практике, 
так и~для формирования лингвистических онтологий, ориентированных на 
повышение качества машинного перевода, в~том числе при отсутствии или 
недостаточном объеме доступных параллельных текстов~[9].

\addtocounter{footnote}{2}
  
  В данной статье рассматриваются возможности применения НБД в~интересах 
переводной деятельности, и~основная цель статьи, таким образом,~--- 
представление подхода к~редуцированию спектра моделей перевода, которые 
регистрируются и~хранятся в~НБД. Демонстрация 
этих возможностей проводится на примере НБД коннекторов\footnote{Об НБД 
коннекторов см.\ [10--13].} (см.\ разд.~3). Однако прежде, чем перейти к~описанию 
указанного подхода, следует дать общее представление о~том, как происходит 
процесс перевода и~как НБД регистрирует модели перевода.

%\vspace*{-6pt}
  
  \section{Модели перевода}
  
%\vspace*{-3pt}
  
  Результат переводной деятельности~--- это реализация в~конечном тексте 
конкретной переводной модели (устойчивого межъязыкового соответствия) или 
окказионального варианта. Изучая перевод, исследователь стремится 
реконструировать переводческое действие, т.\,е.\ зафиксировать изменения, 
которые претерпевает исходный текст (ИТ), оформляясь на переводящем языке 
(ПЯ).
  
  В некоторых работах по теории и~практике перевода отмечается: <<Основная 
особенность$\ldots$ перевода состоит в~жестко детерминированных от\-ношениях 
между системами языка-источника\linebreak и~принимающего языка, которые 
в~большинстве случаев предписывают вполне однозначное решение той или иной 
конкретной переводческой задачи>>~\cite[с.~47]{14-nur}. В~отдельных случаях, 
возможно, это так, однако обычно при переводе однозначность переводческого 
решения (в~выборе конкретного варианта или модели) не настолько очевидна. 
Вряд ли было бы корректно <<говорить о~возможности лишь одного единственно 
правильного варианта перевода (как некорректно$\ldots$ говорить об одной 
единственно возможной интерпретации$\ldots$ в~обычных условиях 
общения)>>~\cite[с.~22]{15-nur}. В~силу синонимического потенциала, которым 
обладают естественные языки, вместо единственно возможного решения при 
переводе, как правило, имеется некоторое множество относительно 
взаимозаменяемых по смыслу эквивалентов, т.\,е.\ <<веер альтернатив>>.
{ %\looseness=1

}
  
  Переводчик интерпретирует ИТ и, руководствуясь своей интерпретацией, 
занимается отбором в~ПЯ соответствующих средств выражения. Если ИТ~--- это 
определенным образом упорядоченная информационно нагруженная 
последовательность знаков, то <<интерпретационная сторона знака~--- это вся 
совокупность ситуационных факторов понимания (предметных, 
коммуникативных, социокультурных), которые обусловливают необходимость 
употребления знака в~момент речи>>~\cite[с.~25]{15-nur}. Другими словами, ИТ 
представляет собой кодированное средствами естественного языка смысловое 
содержание, объем которого может изменяться в~процессе перевода.
{%\looseness=-1
%
}
 % 
  %\columnbreak
  %
  С точки зрения информатики процессы профессионального (выполняемого 
че\-ло\-ве\-ком-спе\-циа\-ли\-стом) перевода суть информационные трансформации 
объектов ментальной и~{социоинформационной} сред, а~процессы машинного 
перевода~--- информационные трансформации объектов социоинформационной 
и~цифровой сред\footnote{Понятие информационной трансформации во многом совпадает 
с~трактовкой, предложенной П.~Розенблумом (см.~\cite{16-nur, 17-nur}).}. 
Рисунок 
иллюстрирует средовую принадлежность информационных трансформаций 
в~профессиональном и~машинном переводе на примере перевода книги.




  С точки зрения теории перевода переводящий субъект обрабатывает 
конкретный языковой знак, перебирая возможные его смысловые значения, 
актуализированные в~заданном языковом окружении~--- контексте, а~затем 
выбирает соответствующие средства, доступные на ПЯ.
{\looseness=-1

}
  
  Выбор происходит из объектов, пред\-став\-ля\-ющих собой переводные 
соответствия~--- \textit{модели перевода}, которые уже систематизированы 
и~генерализованы по разным основаниям, а~также зафиксированы 
в~контрастивных грамматиках, двуязычных словарях, учебниках по переводу 
и~т.\,д. Для использования в~машинном переводе этим моделям присваиваются 
формальные признаки, характеризующие сферу применения, что обеспечивает 
НБД, регистрирующая модели перевода. Регистрация/интеграция информации 
о~моделях перевода и~их признаках в~НБД осуществляется на основе анализа 
параллельных текстов (ИТ и~его перевода, выполненного человеком-специалистом). 
Эта информация, в~свою очередь, может быть использована для 
повышения качества машинного перевода путем редуцирования всего спектра 
моделей на этапе спецификации, т.\,е.\ выбора одного конкретного 
межъязыкового соответствия из этого спектра.
  
  В наборе зафиксированных моделей перевода может не оказаться нужного 
варианта, и~тогда перед переводчиком встает задача выявления смысловых 
компонентов (=\;концептов), необходимых для интерпретации обрабатываемой 
языковой единицы, чтобы впоследствии путем обобщения отнести ее к~той или 
иной семантической общности и~обнаружить ее возможные переводные 
соответствия\footnote{Таким образом, в~процессе перевода используются 
две категории моделей 
перевода: (1)~<<внешние>>, которые описаны в~словарях,
 справочниках, грамматиках и~т.\,д.; 
(2)~<<индивидуализированные>>, которые созданы переводящим субъектом и~не являются частью 
существующего множества <<внешних>> моделей перевода.}.

  



  
     Спектр моделей перевода для разных языковых единиц наряду 
с~двуязычными словарями и~контрастивными грамматиками может 
регистрироваться и~аннотироваться (т.\,е.\ описываться в~соответствии 
с~определенным набором рубрик) также в~электронных базах данных, в~том 
числе в~НБД. Одной из таких баз данных служит НБД 
коннекторов\footnote{Коннектор~--- языковая единица, функция которой состоит в~выражении 
логико-семантических отношений между двумя соединенными с~ее помощью текстовыми фрагментами 
(см.\ подробнее~\cite{18-nur, 19-nur, 20-nur}).}, которая упоминалась выше. Она 
предоставляет ряд возможностей в~отношении динамической (on-the-fly) 
генерализации примеров и~моделей перевода
по разным основаниям 
(ло\-ги\-ко-се\-ман\-ти\-че\-ские от-\linebreak\vspace*{-12pt}

\pagebreak

\end{multicols}

\begin{figure*}
\vspace*{1pt}
 \begin{center}
 \mbox{%
 \epsfxsize=146.085mm 
\epsfbox{nur-1.eps}
 }
 \end{center}
\vspace*{3pt}

{\small Информационные трансформации в~профессиональном и~машинном переводе:
префикс~И (И-кон\-цепт, И-сло\-во, И-знак) относится к~системе 
исходного языка; префикс~П (П-кон\-цепт, П-сло\-во, П-знак)~--- к~системе 
ПЯ; фигурные скобки указывают направление процессов 
профессионального и~машинного перевода}
\vspace*{-2pt}
\end{figure*}

\begin{multicols}{2}

\noindent
 ношения, уста\-нав\-ли\-ва\-емые коннектором; 
структура коннектора; 
его позиция, статус и~т.\,д.)~\cite{21-nur}. В~НБД предусмот\-ре\-но формирование 
и~визуализация спект\-ра моделей перевода каждого коннектора (см.\ пример 
в~~[22, табл.~2]).
     
     Для параллельных текстов пользователь НБД имеет возможность 
интерактивно осуществить обратный переход от генерализованных моделей 
перевода к~примерам их использования в~текстах, т.\,е.\ полностью обратить 
процесс генерализации. Чтобы обеспечить обратимость, в~НБД каждый пример 
перевода связан с~соответствующей динамически сформированной моделью 
перевода, которая имеет гипертекстовую ссылку на список всех примеров ее 
реализации. Иными словами, наряду с~динамической генерализацией моделей 
перевода НБД обеспечивает и~их спецификацию с~помощью гипертекстовых 
ссылок\cite{21-nur}.

\vspace*{-10pt}
  
  \section{Спектр моделей перевода}
  
  \vspace*{-3pt}
  
  Подход к~редуцированию спектра переводов, представляемый здесь, носит 
пошаговый характер и~включает в~себя несколько основных этапов. 
Реализуемость подхода демонстрируется на примере аннотаций НБД коннекторов 
и~текс\-тов фран\-цуз\-ско-рус\-ско\-го 
параллельного подкорпуса\footnote{Аннотации 
коннекторов в~НБД формируются по\-сред\-ст\-вом обработки параллельных текстов, включающих ИТ и~их 
переводы. Для эксперимента, описываемого в~данной статье,
 привлекались профессиональные 
переводы.}, входящего в~состав НКРЯ. По со\-сто\-янию на 31.03.2020 объем этих 
текстов в~НБД\linebreak составлял 649\,245 французских словоупотреблений и~509\,033 
русских, а~всего 1\,158\,278 словоупотреблений (=\;токенов). 
{ %\looseness=1

}

На первом этапе 
был отобран один из наиболее частотных однокомпонентных французских 
коннекторов~--- \textit{c'est-$\grave{\mbox{a}}$-dire} (204~аннотации). На основе 
аннотаций НБД для него был сформирован экспериментальный массив моделей 
перевода в~виде кортежей  
$\langle$\textit{c'est-$\grave{\mbox{a}}$-dire}; вариант перевода$\rangle$. 


Например, для наиболее продуктивной модели перевода кортеж принимает 
следующий вид: $\langle$\textit{c'est-$\grave{\mbox{a}}$-dire}; то есть$\rangle$ 
(см.\ варианты перевода в~табл.~1). Для этого массива генерализованная 
модель перевода имеет вид $\langle$\textit{c'est-$\grave{\mbox{a}}$-dire}; 
\{вариант перевода\}$\rangle$, где {вариант перевода} обозначает весь спектр 
таких вариантов во втором столбце табл.~1.
{\looseness=1

}


\pagebreak

{ %\begin{table*}\small %tabl1
%\vspace*{0.5pt}

%\noindent
\begin{center}
\vspace*{0.5pt}
\parbox{70mm}{{{\tablename~1}\ \ \small{Варианты перевода французского коннектора \textit{c'est-$\grave{\mbox{a}}$-dire}
}}
}

\vspace*{6pt}


{\small
\begin{tabular}{|c|l|c|}
\hline
№&\multicolumn{1}{c|}{Вариант перевода}&\tabcolsep=0pt\begin{tabular}{c} Число \\ 
аннотаций\end{tabular}\\ 
\hline 
1&то есть&157\hphantom{9}\\ 
2&zero (нулевой эквивалент)&29\\ 
3&иными словами&\hphantom{9}8\\ 
4&иначе говоря&\hphantom{9}2\\ 
5&--- (тире)&\hphantom{9}5\\ 
6&а&\hphantom{9}1\\ 
7&итого&\hphantom{9}1\\ 
8&: (двоеточие)&\hphantom{9}1\\ 
\hline
\multicolumn{2}{|r|}{Всего аннотаций:}&204\hphantom{9}\\ 
\hline 
\end{tabular} }
\end{center}
%\end{table*}
}

\vspace*{8pt}
\smallskip
  
  
    

  Затем было определено число аннотаций с~различными сочетаниями рубрик, 
использованных для аннотирования текстовых фрагментов, где\linebreak употреблен 
коннектор \textit{c'est-$\grave{\mbox{a}}$-dire} (табл.~2)\footnote{Исследуемые сочетания 
рубрик встретились в~46 из~204 аннотаций с~коннектором  
\textit{c'est-$\grave{\mbox{a}}$-dire.}}. Эта таб\-ли\-ца показывает, какие рубрики 
характеризуют модель перевода в~НБД, а~также насколько продуктивна эта 
модель при том или ином их сочетании.
  
  Экспериментальные данные включают следующие рубрики: порядок 
следования связываемых коннектором текстовых фрагментов (p~CNT~q); 
встраи\-ва\-ющий/встро\-ен\-ный коннектор (SubCNT/SuperCNT); маркирует часть 
предложения без пре\-ди\-ка\-ции\,/\,с~пре\-ди\-ка\-ци\-ей; 
ди\-стант\-ное/кон\-такт\-ное расположение составляющих\linebreak в~неоднословных 
коннекторах  (кон\-такт/дис\-тант); позиция  
(на\-чаль\-ная/не\-на\-чаль\-ная/ко\-неч\-ная); ло\-ги\-ко-се\-ман\-ти\-че\-ские 
отношения, устанавливаемые\linebreak

\columnbreak

\noindent
 коннектором (переформулирование и~т.\,д.); 
маркирует сверхфразовое единство, самостоятельное повествовательное 
предложение, сложное предложение и~т.\,д. Релевантность приведенных рубрик 
определялась в~ходе предыдущих кор\-пус\-но-ори\-ен\-ти\-ро\-ван\-ных 
контрастивных исследований, инструментом в~которых служила НБД 
коннекторов. Результаты некоторых исследований см.\ в~\cite{19-nur, 20-nur, 
21-nur, 22-nur, 23-nur, 24-nur, 25-nur}.
  
  Приведенные в~табл.~2 рубрики, характери\-зу\-ющие коннектор и~его контекст, 
были проанализированы на предмет их релевантности для выбора переводного 
варианта\footnote{Анализ проводился одним из соавторов данной статьи В.\,А.~Нуриевым~--- 
профессиональным переводчиком, имеющим опубликованные переводы с~французского и~английского 
языков.}. Следовало определить, позволяет ли имеющийся в~НБД (исходно 
заданный  
ис\-сле\-до\-ва\-те\-ля\-ми-линг\-ви\-ста\-ми\footnote{Набор рубрик исходно формировался лингвистами, 
исходя из определенных задач, напрямую не связанных с~переводческой деятельностью.}) набор 
рубрик \mbox{сузить} об\-ласть переводческого поиска, т.\,е.\ редуцировать спектр моделей 
перевода.
  
  Планировалось ответить экспериментально на следующий вопрос: 
<<Насколько исходно заданная в~НБД система классификационных рубрик дает 
возможность редуцировать спектр конкретных моделей перевода?>> Для 
получения ответа на этот вопрос был проведен семантический анализ 
распределения аннотаций как по рубрикам, так и~по их сочетаниям.
  
  В табл.~1 показано, что французский коннектор имеет более чем одно 
переводное соответствие на русском языке: 
для \textit{c'est-$\grave{\mbox{a}}$-dire} в~НБД на 31.03.2020 
зарегистрировано 8~вариантов перевода, включая~2 
с~пунктуационными знаками. Чтобы переводчик мог существенно редуцировать 
этот спектр переводных вариантов, указанных в~табл.~2 рубрик недостаточно. 
По-ви\-ди\-мо\-му, в~интересах переводной\linebreak
%\vspace*{-11pt}
\end{multicols}

\setcounter{table}{1}
\begin{table*}[h]\small  %tabl2

\vspace*{-24pt}
\begin{center}
\parbox{150mm}{\Caption{Число аннотаций с~моделями перевода для \textit{c'est-$\grave{\mbox{a}}$-dire} 
(первые два столбца) при сочетании рубрик (третий столбец)}
}
\vspace*{2ex}

\begin{tabular}{|c|l|l|c|}
\hline
Оригинал&\multicolumn{1}{c|}{\tabcolsep=0pt\begin{tabular}{c} Вариант\\ 
перевода\end{tabular}}&\multicolumn{1}{c|}{ Сочетание 
рубрик}&\tabcolsep=0pt\begin{tabular}{c} Число\\ аннотаций\end{tabular}\\
\hline
\textit{c'est-$\grave{\mbox{a}}$-dire}&то 
есть&\tabcolsep=0pt\begin{tabular}{l}p~CNT~q\;+\;SubCNT\;+\;SuperCNT\;+\;без предикации\;+\\
 +\;контакт\;+\;начальная\;+\;переформулирование\end{tabular}&10\\
 \cline{2-4}
&zero&\tabcolsep=0pt\begin{tabular}{l} p~CNT~q\;+\;SubCNT\;+\;SuperCNT\;+\;без предикации\;+\\
 +\;контакт\;+\;начальная\;+\;переформулирование\end{tabular}&\hphantom{9}1\\
\cline{2-4}
&то есть&\tabcolsep=0pt\begin{tabular}{l}p~CNT~q\;+\;SubCNT\;+\;SuperCNT\;+\;контакт\; +\;начальная\;+\\
 +\;переформулирование\;+\;сложное предложение\end{tabular}&\hphantom{9}1\\
\cline{2-4}
&то есть&\tabcolsep=0pt\begin{tabular}{l}p~CNT~q\;+\;SubCNT\;+\;без 
предикации\;+\;контакт\;+\;начальная\;+\\
 +\;переформулирование\end{tabular}&33\\
\cline{2-4}
&zero&\tabcolsep=0pt\begin{tabular}{l}p~CNT~q\;+\;SubCNT\;+\;вставка\;+\;контакт\;+\;начальная\;+\\ 
+\;переформулирование\end{tabular}&\hphantom{9}1\\
\cline{2-4}
&---&Другие сочетания рубрик&158\hphantom{9}\\
\hline
\multicolumn{3}{|r|}{Итого:}&204\hphantom{9}\\
  \hline
  \end{tabular}
  \end{center}
  \end{table*}
  
  \begin{multicols}{2}
  
  \noindent
    деятельности число рубрик 
необходимо увеличить. Этот промежуточный вывод позволили сделать несколько 
следующих наблюдений:
  \begin{itemize}
  \item 83\% случаев употребления \textit{c'est-$\grave{\mbox{a}}$-dire} (170 
аннотаций из 204) в~НБД приходится на сборник научных статей Ж.~Женетта 
<<Фигуры>>;
  \item 86\% случаев перевода \textit{c'est-$\grave{\mbox{a}}$-dire} нулевым 
эквивалентом (25 аннотаций из~29) также приходится на эту работу, что, 
по-ви\-ди\-мо\-му, мотивировано необходимостью синонимического \mbox{варьирования} 
в~подборе эквивалентов; 
  \item этой же необходимостью мотивировано употребление при переводе 
\textit{c'est-$\grave{\mbox{a}}$-dire} в~<<Фигурах>> Ж.~Женетта других 
переводных соответствий типа \textit{иными словами} (5~вхождений); 
\textit{иначе говоря} (2~вхождения), тире (1~вхож\-де\-ние);
  \item нулевой эквивалент часто требует значительной синтаксической 
переработки высказывания и~влечет за собой изменение его общего смысла;
  \item реализация модели перевода $\langle$\textit{c'est-$\grave{\mbox{a}}$-dire}; 
\textit{то есть}$\rangle$ больше характерна для научного стиля~--- 87\% случаев 
употребления \textit{то есть} зафиксировано в~переводе <<Фигур>> 
Ж.~Женетта;
  \item при переводе текстов художественных жанров наблюдается тенденция 
к~большему синонимическому варьированию, здесь реализуются такие модели 
перевода \textit{c'est-$\grave{\mbox{a}}$-dire}, как тире (4~вхождения), 
\textit{иными словами} (3~вхождения), \textit{а} (1~вхождение), \textit{итого} 
(1~вхождение), двоеточие (1~вхождение);
  \item нулевой эквивалент для текстов художественных жанров менее 
характерен (14\%~всех случаев, причем в~разных переводах).
  \end{itemize}
  На основе этих наблюдений можно предположить, что в~НБД необходимо 
ввести дополнительную руб\-ри\-ку 
<<ху\-до\-жест\-вен\-ный/не\-ху\-до\-жест\-вен\-ный текст>>, которая 
позволит редуцировать спектр моделей перевода.
  
  \section{Заключение}
  
  В статье предложен подход к~применению НБД для редуцирования спектра 
моделей перевода, на основе которого планируется вычислить их условные 
частотности с~разными сочетаниями рубрик, проставленными в~процессе 
аннотирования. Предлагаемый подход иллюстрируется на примере анализа 
моделей перевода французского коннектора \textit{c'est-$\grave{\mbox{a}}$-dire}. 
Планируется выполнить проверку реализуемости этого подхода на примере 
нескольких французских коннекторов, включая \textit{comme}, чье 
употребление\footnote{По данным НБД коннекторов на 31.03.2020.} характеризуется 
большей равномерностью~--- случаи его употребления 
относительно равномерно распределяются по целому ряду разножанровых 
текстов, принадлежащих разным авторам (О.~де~Бальзак, Ф.~Бегбедер, Ж.~Верн, 
Р.~Госинни, Ж.~Женетт, Ж.~Кокто, П.~Модиано и~др.). Кроме того, для \textit{comme} 
зафиксировано большее многообразие вариантов перевода 
(31~вариант на~206~аннотаций в~НБД по состоянию на 31.03.2020) и~этот 
коннектор, в~отличие от \textit{c'est-$\grave{\mbox{a}}$-dire}, уста\-нав\-ли\-ва\-ющего 
только один тип ло\-ги\-ко-се\-ман\-ти\-че\-ских отношений (переформулирования),
 может 
сигнализировать о~четырех разных типах отношений (пропозициональной 
причины; аддитивные пропозициональные отношения; отношение аналогии; 
сравнительные отношения).

  
{\small\frenchspacing
 {%\baselineskip=10.8pt
 \addcontentsline{toc}{section}{References}
 \begin{thebibliography}{99}
  
\bibitem{1-nur}
\Au{Zatsman I., Buntman~N., Kruzhkov~M., Nuriev~V., Zalizniak Anna~A.} Conceptual framework 
for development of computer technology supporting cross-linguistic knowledge discovery~// 15th 
European Conference on Knowledge Management Proceedings.~--- Reading: Academic Publishing 
International Ltd., 2014. Vol.3. P.~1063--1071.
\bibitem{2-nur}
\Au{Zatsman I., Buntman N.} Outlining goals for discovering new knowledge and computerised 
tracing of emerging meanings~// 16th European Conference on Knowledge Management 
Proceedings.~--- Reading: Academic Publishing International Ltd., 2015. P.~851--860.
\bibitem{3-nur}
\Au{Zatsman I., Buntman N., Coldefy-Faucard~A., Nuriev~V.} WEB knowledge base for 
asynchronous brainstorming~// 17th European Conference on Knowledge Management 
Proceedings.--- Reading: Academic Publishing International Ltd., 2016. Vol.~1. P.~976--983.
\bibitem{4-nur}
\Au{Ляшевская О.\,Н.} Корпусные инструменты в~грамматических исследованиях русского 
языка.~--- М.: ЯСК: Рукописные памятники Древней Руси, 2016. 520~с.

\bibitem{6-nur} %5
Corpus use and translating: Corpus use for learning to translate and learning corpus use to 
translate~/ Eds. A.~Beeby, P.~\mbox{Rodr{\!\ptb{\'{\i}}}guez} In$\acute{\mbox{e}}$s, 
P.~S$\acute{\mbox{a}}$nchez-Gij$\acute{\mbox{o}}$n.~--- Amsterdam--Philadelphia: John 
Benjamins Publishing, 2009. 149~p.
\bibitem{7-nur} %6
\Au{Flowerdew L.} Corpora and language education.~--- New York\,--\,Basingstoke: Palgrave Macmillan, 
2012. 358~p.
\bibitem{5-nur} %7
Corpora in translator education~/ Eds. F.~Zanettin, S.~Bernardini, D.~Stewart.~--- New York, NY, 
USA: Routledge, 2014. 156~p.
\bibitem{8-nur}
\Au{Егорова А.\,Ю., Зацман~И.\,М., Мамонова~О.\,С.} Надкорпусные базы данных 
в~лингвистических проектах~// Системы и~средства информатики, 2019. 
Т.~29. №\,3. С.~77--91.
\bibitem{9-nur}
\Au{Anthes G.} Dead languages come to life~// Commun. ACM, 2020. Vol.~63. 
No.\,4. P.~13--15.
\bibitem{10-nur}
\Au{Зализняк А.\,А., Зацман~И.\,М., Инькова~О.\,Ю.} Надкорпусная база данных 
коннекторов: построение системы терминов~// Информатика и~её применения, 2017. Т.~11. 
Вып.~1. С.~100--108.

\bibitem{13-nur} %11
\Au{Inkova O., Popkova~N.} Statistical data as information source for linguistic analysis of Russian 
connectors~// Информатика и~её применения, 2017. Т.~11. Вып.~3. С.~123--131.

\bibitem{11-nur} %12
\Au{Зацман И.\,М., Кружков~М.\,Г.} Надкорпусная база данных коннекторов: развитие 
системы терминов проектирования~// Системы и~средства информатики, 2018. Т.~28. №\,4. 
С.~156--167.
\bibitem{12-nur} %13
\Au{Инькова О.\,Ю.} Надкорпусная база данных как инструмент формальной вариативности 
коннекторов~// Компьютерная лингвистика и~интеллектуальные технологии: по мат-лам 
ежегодной Международ. конф. <<Диалог>>.--- М.: РГГУ, 2018. Вып.~17(24). С.~240--253.

\bibitem{14-nur}
\Au{Щетинкин В.\,Е.} Пособие по переводу с~французского языка на русский.~--- М.: 
Просвещение, 1987. 160~с.
\bibitem{15-nur}
\Au{Иванов Н.\,В.} Интерпретация в~знаковой онтологии языка и~в~переводе.~--- М.: 
Международные отношения, 2018. 152~с.
\bibitem{16-nur}
\Au{Rosenbloom P.} On computing: The fourth great scientific domain.~--- 
Cambridge, MA, USA: MIT Press, 
2013. 308~p.
\bibitem{17-nur}
\Au{Зацман И.\,М.} Интерфейсы третьего порядка в~информатике~// Информатика и~её 
применения, 2019. Т.~13. Вып.~3. С.~82--89.
\bibitem{18-nur}
\Au{Инькова-Манзотти~О.\,Ю.} Коннекторы противопоставления во французском 
и~русском языках. Сопоставительное исследование.~--- М.: Информэлектро, 2001. 429~с.
\bibitem{19-nur}
Семантика коннекторов: контрастивное исследование~/ Под ред. О.\,Ю.~Иньковой.~--- М.: 
ТОРУС ПРЕСС, 2018. 396~с.
\bibitem{20-nur}
Структура коннекторов и~методы ее описания~/ Под. ред. О.\,Ю. Иньковой.~--- М.: ТОРУС 
ПРЕСС, 2019. 316~с.
\bibitem{21-nur}
\Au{Зацман И.\,М., Мамонова~О. С., Щурова~А.\,Ю. }Обратимость и~альтернативность 
генерализации моделей перевода коннекторов в~параллельных текстах~// Сис\-темы 
и~средства информатики, 2017. Т.~27. №\,2.\linebreak С.~125--142.
\bibitem{22-nur}
\Au{Зацман И.\,М.} Методология обратимой генерализации в~контексте классификации 
информационных трансформаций~// Сис\-те\-мы и~средства информатики, 2018. Т.~28. №\,2. 
С.~128--144.

\bibitem{24-nur} %23
\Au{Амеличева В.} Формальное и~семантическое варьирование русского коннектора 
\textit{не только$\ldots$\ но~и} и~его французские эквиваленты~// Съпоставително 
езикознание (Сопоставительное языкознание~/ Contrastive Linguistics), 2017. Т.~42. №\,4. 
С.~9--20.

\bibitem{23-nur} %24
\Au{Инькова О., Манзотти Э.} Tra l'altro, между прочим, entre autres: сходства 
и~различия~// Съпоставително езикознание (Сопоставительное языкознание~/ Contrastive 
Linguistics), 2017. Т.~42. №\,4. С.~35--47.
\bibitem{25-nur}
\Au{Нуриев В.} Коннектор \textit{раз$\ldots$ то} и~модели его перевода на французский 
язык: между условием, причиной и~следствием~// Съпоставително езикознание 
(Сопоставительное языкознание~/ Contrastive Linguistics), 2017. Т.~42. №\,4. С.~63--76.
\end{thebibliography}

 }
 }

\end{multicols}

\vspace*{-6pt}

\hfill{\small\textit{Поступила в~редакцию 13.04.20}}

\vspace*{6pt}

%\pagebreak

%\newpage

%\vspace*{-28pt}

\hrule

\vspace*{2pt}

\hrule

%\vspace*{-2pt}

\def\tit{REDUCING THE SPECTRUM OF~TRANSLATION MODELS IN~SUPRACORPORA 
DATABASES}


\def\titkol{Reducing the spectrum of translation models in supracorpora 
databases}

\def\aut{V.\,A.~Nuriev and I.\,M.~Zatsman}

\def\autkol{V.\,A.~Nuriev and I.\,M.~Zatsman}

\titel{\tit}{\aut}{\autkol}{\titkol}

\vspace*{-9pt}


\noindent
Institute of Informatics Problems, Federal Research Center ``Computer Science 
and Control'' of the Russian Academy of Sciences, 44-2~Vavilov Str., Moscow 
119333, Russian Federation

\def\leftfootline{\small{\textbf{\thepage}
\hfill INFORMATIKA I EE PRIMENENIYA~--- INFORMATICS AND
APPLICATIONS\ \ \ 2020\ \ \ volume~14\ \ \ issue\ 2}
}%
 \def\rightfootline{\small{INFORMATIKA I EE PRIMENENIYA~---
INFORMATICS AND APPLICATIONS\ \ \ 2020\ \ \ volume~14\ \ \ issue\ 2
\hfill \textbf{\thepage}}}

\vspace*{3pt} 


\Abste{The paper describes an approach aimed at reducing the spectrum of translation models that are 
registered in supracorpora databases (SDB). This approach may be applied to both professional (made by the 
human) and machine translation. As of today, one can use the information resources of the SDB to 
research the  
problems of interest to computing, computer linguistics and linguistic theory. Here, the focus is on 
examining if and how the SDB can be used in translation practice~--- for reducing the spectrum of translation 
models. Very often while translating, a~translator finds himself/herself in a~multiple-choice situation: due to the 
synonymic potential, characteristic of natural languages, in translation instead of the only possible solution, there 
is a~set of relatively interchangeable equivalents, i.\,e., ``fan of alternatives.'' Choosing from a~(sometimes 
wide) range of variants, a~translator, in order to narrow the choice, relies on some specific characteristics of the 
source text. Hence, the goal of the paper is to describe the approach that would allow one to use the SDB for 
narrowing the choice set of relevant translation models.}

\KWE{parallel texts; translation; translation models; supracorpora database; multiple choice}



\DOI{10.14357/19922264200217} 

%\vspace*{-20pt}

%\Ack
%\noindent


%\vspace*{6pt}

 \begin{multicols}{2}

\renewcommand{\bibname}{\protect\rmfamily References}
%\renewcommand{\bibname}{\large\protect\rm References}

{\small\frenchspacing
 {\baselineskip=10.75pt
 \addcontentsline{toc}{section}{References}
 \begin{thebibliography}{99}

\bibitem{1-nur-1}
\Aue{Zatsman, I., N. Buntman, M.~Kruzhkov, V.~Nuriev, and A.~Zalizniak.} 2014. Conceptual framework for 
development of computer technology supporting cross-linguistic knowledge discovery. \textit{15th European 
Conference on Knowledge Management Proceedings.} Reading:
Academic Publishing International Ltd. 3:1063--1071.
\bibitem{2-nur-1}
\Aue{Zatsman, I., and N. Buntman.} 2015. Outlining goals for discovering new knowledge and computerised 
tracing of emerging meanings. \textit{16th European Conference on Knowledge Management Proceedings}. 
Reading: Academic Publishing International Ltd. 851--860.
\bibitem{3-nur-1}
\Aue{Zatsman I., N. Buntman, A.~Coldefy-Faucard, and V.~Nuriev.} 2016. WEB knowledge base for 
asynchronous brainstorming. \textit{17th European Conference on Knowledge Management Proceedings.}  
Reading: Academic Publishing International Ltd. 1:976--983.
\bibitem{4-nur-1}
\Aue{Lyashevskaya, O.\,N.} 2016. Korpusnye instrumenty v grammaticheskikh issledovaniyakh russkogo 
yazyka [Corpus instruments in grammatical studies of Russian]. Moscow: YASK: Rukopisnye Pamyatniki 
Drevney Rusi. 520~p.

\bibitem{6-nur-1} %5
Beeby, A., P.~\mbox{Rodr{\!\ptb{\'{\i}}}guez} In$\acute{\mbox{e}}$s, and 
P.~S$\acute{\mbox{a}}$nchez-Gij$\acute{\mbox{o}}$n, eds. 2009. 
\textit{Corpus use and translating: Corpus use for learning to translate 
and learning corpus use to translate}. Amsterdam--Philadelphia: John Benjamins Publishing. 149~p.
\bibitem{7-nur-1} %6
\Aue{Flowerdew, L.} 2012. \textit{Corpora and language education.} New York\,--\,Basingstoke: Palgrave Macmillan. 
358~p.
\bibitem{5-nur-1} %7
Zanettin, F., S. Bernardini, and D. Stewart, eds. 2014. \textit{Corpora in translator education}. New York, NY: 
Routledge. 156~p.
\bibitem{8-nur-1}
\Aue{Egorova, A.\,Yu., I.\,M.~Zatsman, and O.\,S.~Mamonova.} 2019. Nadkorpusnye basy dannykh 
v~lingvisticheskikh proektakh [Supracorpora databases in linguistic projects]. \textit{Sistemy i Sredstva 
Informatiki~--- Systems and Means of Informatics} 29(3):77--91.
\bibitem{9-nur-1}
\Aue{Anthes, G.} 2020. Dead languages come to life. \textit{Commun. ACM} 63(4):13--15.
\bibitem{10-nur-1}
\Aue{Zaliznyak, A.\,A., I. M. Zatsman, and O. Yu. In'kova.} 2017. Nadkorpusnaya basa dannykh konnektorov: 
postroenie sistemy terminov [Supracorpora database of connectives: Term system development]. 
\textit{Informatika i ee Primeneniya~--- Inform. Appl.} 11(1):100--108.

\bibitem{13-nur-1} %11
\Aue{Inkova, O., and N.~Popkova.} 2017. Statistical data as information source for linguistic analysis of 
Russian connectors. \textit{Informatika i~ee Primeneniya~--- Inform. Appl.} 11(3):123--131.
\bibitem{11-nur-1} %12
\Aue{Zatsman, I.\,M., and M.\,G.~Kruzhkov.} 2018. Nadkorpusnaya baza dannykh konnektorov: razvitie 
sistemy terminov proektirovaniya [Supracorpora database of connectives: Design-oriented evolution of the term 
system]. Sistemy i~Sredstva Informatiki~--- Systems and Means of Informatics 28(4):156--167.
\bibitem{12-nur-1} %13
\Aue{Inkova, O.\,Yu.} 2018. Nadkorpusnaya baza dannykh kak instrument formal'noy variativnosti 
konnektorov [Supracorpora database as an instrument of the study of the formal variability of connectives]. 
\textit{Komp'yuternaya lingvistika 
i~intellektual'nye tekhnologii: po mat-lam ezhegodnoy Mezhdunar. konf. 
``Dialog''} [Computer Linguistic and Intellectual Technologies: Conference (International) ``Dialog'' 
Proceedings]. Moscow. 17(24):240--253.

\bibitem{14-nur-1}
\Aue{Schetinkin, V.\,E.} 1987. \textit{Posobie po perevodu s~frantsuzskogo yazyka na russkiy} [Manual of 
translation from French into Russian]. Moscow: Prosveschenie. 160~p.
\bibitem{15-nur-1}
\Aue{Ivanov, N.\,V.} 2018. \textit{Interpretatsiya v~znakovoy ontologii yazyka i~v~perevode} [Interpretation 
in the semiotic ontology of the language and in translation]. Moscow: Mezhdunarodnye otnosheniya. 152~p.
\bibitem{16-nur-1}
\Aue{Rosenbloom, P.} 2013. \textit{On computing: 
The fourth great scientific domain.} Cambridge, MA: MIT Press. 
308~p.
\bibitem{17-nur-1}
\Aue{Zatsman, I.\,M.} 2019. Interfeysy tret'ego poryadka v informatike [Third-order interfaces in informatics]. 
\textit{Informatika i ee Primeneniya~--- Inform. Appl.} 13(3):82--89.
\bibitem{18-nur-1}
\Aue{In'kova-Manzotti, O.\,Yu.} 2001. \textit{Konnektory protivopostavleniya vo frantsuzskom i russkom 
yazykakh. Sopostavitel'noe issledovanie} [Connectors of opposition in French and Russian: A~comparative 
study]. Moscow: Informelektro. 429~p.
\bibitem{19-nur-1}
In'kova, O.\,Yu., ed. 2018. \textit{Semantika konnektorov: kontrastivnoe issledovanie} [Semantics of 
connectives: Contrastive study]. Moscow: TORUS PRESS. 396~p.
\bibitem{20-nur-1}
In'kova, O. Yu., ed. 2019. \textit{Struktura konnektorov i~metody ee opisaniya} [Structure of connectives and 
methods of its description]. Moscow: TORUS PRESS. 316~p.
\bibitem{21-nur-1}
\Aue{Zatsman, I.\,M., O.\,S.~Mamonova, and A.\,Yu.~Shchurova.} 2017. Obratimost' i~al'ternativnost' 
generalizatsii modeley perevoda konnektorov v~parallel'nykh tekstakh [Reversibility and alternativeness of 
generalization of connectives translations models in parallel texts]. \textit{Sistemy i~Sredstva Informatiki~--- 
Systems and Means of Informatics} 27(2):125--142.
\bibitem{22-nur-1}
\Aue{Zatsman, I.\,M.} 2018. Metodologiya obratimoy ge\-ne\-ra\-li\-za\-tsii 
v~kontekste klassifikatsii informatsionnykh 
transformatsiy [Methodology of reversible generalization in context of classification of information 
transformations]. \textit{Sistemy i~Sredstva Informatiki~--- Systems and 
Means of Informatics} 28(2):128--144.

\bibitem{24-nur-1} %23
\Aue{Amelicheva, V.} 2017. Formal'noe i~semanticheskoe var'irovanie russkogo konnektora \textit{ne 
tol'ko$\ldots$ no i} i~ego frantsuzskie ekvivalenty [Formal and semantic variations of the Russian connector 
\textit{ne tol'ko$\ldots$ no~i} and its French equivalents]. 
\textit{Sopostavitelnoye yazykoznaniye} 
[Contrastive Linguistics] 42(4):9--20.

\bibitem{23-nur-1} %24
\Aue{Inkova, O., and E.~Manzoti.} 2017. Tra l'altro, mezhdu prochim, entre autres: skhodstva i~razlichiya [Tra 
l'altro, mezhdu prochim, entre autres: Similarities and differences]. \textit{Sopostavitelnoye yazykoznaniye} 
[Contrastive Linguistics] 42(4):35--47.

\bibitem{25-nur-1}
\Aue{Nuriev, V.} 2017. Konnektor \textit{raz$\ldots$ to} i~modeli ego perevoda na frantsuzskiy yazyk: 
mezhdu usloviem, prichinoy i~sledstviem [The connector \textit{raz}$\ldots$ \textit{to} and variants of 
rendering it into French: Between condition, cause, and effect]. \textit{Sopostavitelnoye yazykoznaniye} 
[Contrastive Linguistics] 42(4):63--76.

\end{thebibliography}

 }
 }

\end{multicols}

\vspace*{-14pt}

\hfill{\small\textit{Received April 13, 2020}}

\pagebreak



\Contr

\noindent
\textbf{Nuriev Vitaly A.} (b.\ 1980)~--- Candidate of Science (PhD) in philology, leading scientist, Institute of
Informatics Problems, Federal Research Center ``Computer Science and Control'' of the Russian Academy
of Sciences, 44-2~Vavilov Str., Moscow 119333, Russian Federation; \mbox{nurieff.v@gmail.com}

\vspace*{6pt}

\noindent
\textbf{Zatsman Igor M.} (b.\ 1952)~--- Doctor of Science in technology, Head of Department, Institute of 
Informatics
Problems, Federal Research Center ``Computer Science and Control'' of the Russian Academy of Sciences,
44-2~Vavilov Str., Moscow 119333, Russian Federation; \mbox{izatsman@yandex.ru}
\label{end\stat}

\renewcommand{\bibname}{\protect\rm Литература} 