   \vspace*{-48pt}

\begin{center}
\vspace*{6pt}
\mbox{%
%\epsfxsize=50mm %56.519mm  
%\epsfbox{smu-1.eps} 

\epsfxsize=50mm %46.402 mm
\epsfbox{nec-rb.eps}
}
%\end{center}

\vspace*{9pt} %Академик


%   \begin{center}
\fbox{\large\textbf{Рустем Бадриевич Сейфуль-Мулюков}}\\[6pt]
\textbf{\large 1928--2020}
   \end{center}


   %\vspace*{2.5mm}

   \vspace*{5mm}

   \thispagestyle{empty}

%\

%\vspace*{-12pt}

  
      Редакция журнала <<Информатика и~её применения>> с глубоким 
      прискорбием сообщают, что 17~марта 2020~г.\ на 93-м~году жизни 
      скончался заведующий редакцией журнала, главный научный сотрудник Федерального исследовательского центра <<Информатика и~управление>> Российской академии наук
      Рустем Бадриевич Сейфуль-Мулюков.
           
     Всю свою жизнь Рустем Бадриевич посвятил служению науке. Закончив в~1956~г.\ аспирантуру Московского ордена Трудового Красного знамени Нефтяного института им.\ академика
     И.\,М.~Губкина, он прошел путь от заведующего отделом Института геологии зарубежных стран Министерства геологии СССР до заместителя директора ВИНИТИ
     АН СССР, доктора гео\-ло\-го-ми\-не\-ра\-ло\-ги\-че\-ских наук, профессора.
     
     С марта 2002~г.\ Рустем Бадриевич успешно применял свои знания и~организационный талант в ИПИ
     РАН (в~дальнейшем~--- ФИЦ ИУ РАН), в~котором руководил лабораторией и~отделом, занимающимися вопросами технологий информационной технической деятельности. 
Р.\,Б.~Сейфуль-Мулюков, являясь автором значительного количества научных трудов и~монографий по геологии, информационным технологиям и~теоретической информатике, осуществлял организацию издания монографий ИПИ РАН и~ФИЦ ИУ РАН, библиографий научных сотрудников Центра.
     
     Р.\,Б.~Сейфуль-Мулюков являлся заведующим редакцией журналов <<Информатика и~её применения>> и~<<Системы и~средства информатики>>, членом редколлегии журнала <<Системы и~средства информатики>>. Он вложил огромный вклад в становление и~развитие этих журналов, организацию их регистрации, функционирования, редактуры и~издания. Включение этих журналов в ряд отечественных и~зарубежных информационных баз и~систем цитирования во многом является его личной заслугой.
     
     На всех занимаемых должностях Рустем Бадриевич отличался высоким профессионализмом, преданностью делу и~вниманием к коллегам.
     
     \thispagestyle{empty}
     
     Рустема Бадриевича отличали доброта, отзывчивость, неиссякаемый
      оптимизм, простота и~сердечность.
     
     Коллеги Рустема Бадриевича запомнят его как многогранного в~своих увлечениях человека, живописца,
     эрудита и~энциклопедиста, интересующегося историей, литературой и~искусством.
     
     Выражаем глубокое
     соболезнование семье, родственникам, друзьям и~коллегам по работе в~связи с~тяжелой невосполнимой утратой.
     Светлый образ Рустема Бадриевича навсегда сохранится в~нашей памяти.
     

      