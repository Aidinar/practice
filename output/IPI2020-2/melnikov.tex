\def\stat{melnikov}

\def\tit{СТАТИСТИЧЕСКИЕ СВОЙСТВА ДВОИЧНЫХ 
НЕАВТОНОМНЫХ РЕГИСТРОВ СДВИГА С~ВНУТРЕННИМ 
СУММИРОВАНИЕМ$^*$}

\def\titkol{Статистические свойства двоичных 
неавтономных регистров сдвига с~внутренним 
суммированием}

\def\aut{С.\,Ю.~Мельников$^1$, К.\,Е.~Самуйлов$^2$}

\def\autkol{С.\,Ю.~Мельников, К.\,Е.~Самуйлов}

\titel{\tit}{\aut}{\autkol}{\titkol}

\index{Мельников С.\,Ю.}
\index{Самуйлов К.\,Е.}
\index{Melnikov S.\,Yu.}
\index{Samouylov K.\,E.}
 

{\renewcommand{\thefootnote}{\fnsymbol{footnote}} \footnotetext[1]
{Публикация подготовлена при поддержке Программы РУДН <<5-100>> 
(К.\,Е.~Самуйлов, постановка задачи) и~при финансовой поддержке РФФИ (проекты 
18-00-01555, 18-00-01685 и~19-07-00933).}}


\renewcommand{\thefootnote}{\arabic{footnote}}
\footnotetext[1]{Российский университет дружбы народов, melnikov@linfotech.ru}
\footnotetext[2]{Российский университет дружбы народов, ksam@sci.pfu.edu.ru}

%\vspace*{-6pt}

  \Abst{Проводится сравнение статистических и~алгебраических свойств двоичных 
неавтономных регистров сдвига и~регистров сдвига с~внутренним суммированием, при 
шаге которых вектор состояния суммируется со своим сдвигом на один шаг. Доказан 
изоморфизм графов переходов этих автоматов. Показано, что при бернуллиевском 
случайном входе стационарное распределение на состояниях регистра с~внутренним 
суммированием равномерное. Получен вид вероятностной функции этих регистров. 
Показано, что при определенных ограничениях на функцию выходов регистры 
с~внутренним суммированием не че\-за\-ро\-во-на\-след\-ст\-вен\-ные. Предъявлены входные 
последовательности, которые обладают свойством устойчивости относительных частот 
произвольных мультиграмм, в~то время как выходные последовательности таким свойством 
не обладают.} 
  \KW{автомат со случайным входом; регистр сдвига; граф де Брейна; вероятностная 
функция}
  
\DOI{10.14357/19922264200211}
 
 
%\vspace*{9pt}


\vskip 10pt plus 9pt minus 6pt

\thispagestyle{headings}

\begin{multicols}{2}

\label{st\stat}
  
   
\section{Введение}

  При конструировании генераторов случайных последовательностей широко 
используются как линейные, так и~нелинейные регистры сдвига с~теми или 
иными элементами усложнения или обратной связи~[1]. Это во многом 
обусловлено совокупностью <<хороших>> комбинаторных и~структурных 
свойств графов де Брейна, описывающих преобразования информации в~таких 
регистрах.
  
  Пусть $f(x_1,x_2,\ldots , x_n)$~--- булева функция от~$n$ двоичных 
переменных, $n\hm= 1,2,\ldots$ Будем рас\-смат\-ри\-вать два автомата Мура, 
множество состояний каждого из которых есть $V_n\hm= \{0,1\}^n$, входной 
и~выходной алфавиты~--- множества $\{0,1\}$, выходом служит значение 
функции~$f$ от текущего состояния. 
  
  Автомат (регистр сдвига), который обозначим $A_f$, под действием 
входного символа $a_0\hm\in \{0,1\}$ из состояния $(a_1, a_2, \ldots ,a_n)$ 
переходит в~состояние $(a_0, a_1, \ldots , a_{n-1})$.
   
  Регистром сдвига с~внутренним суммированием назовем автомат, который 
под действием входного символа $a_0\hm\in \{0,1\}$ из состояния 
$(a_1,a_2,\ldots , a_n)$ переходит в~состояние $(a_0\oplus a_1, a_1\oplus 
a_2,\ldots , a_{_n-1}\oplus a_n)$, где $\oplus$~--- суммирование по модулю~2. 
Такой автомат будем обозначать~$A_f^{\oplus}$.
  
  Алгебраические и~статистические свойства обычных регистров хорошо 
изучены. Тео\-ре\-ти\-ко-ав\-то\-мат\-ные свойства регистров, 
аналогичных~$A_f^{\oplus}$, в~автономном случае рассматривались в~[2], 
а~вопросы их аппаратной реализации~--- в~[3]. В~настоящей работе доказывается 
изоморфизм графов переходов и~проводится сравнение статистических свойств 
автоматов~$A_f$ и~$A_f^{\oplus}$.
  
\section{Связь графов переходов автоматов~$A_f$ 
и~$A_f^{\oplus}$}

  Графом переходов автомата~$A_f$, как нетрудно видеть, служит 
ориентированный граф с~множеством вершин~$V_n$, с~дугами, выходящими 
из вершин $(a_1, a_2, \ldots , a_n)$ и~заходящими в~вершины 
$(a_0,a_1,a_2,\ldots , a_{n-1})$, $a_i\hm\in \{0,1\}$. Это хорошо известный граф 
де Брейна, который будем обозначать~$G_n$.
  
  Граф переходов регистра сдвига с~внутренним суммированием 
с~накопителем размера $n\hm=1,2,\ldots$~--- это ориентированный граф 
с~множеством вершин~$V_n$, содержащий дуги, выходящие из вершин $(a_1, 
a_2, \ldots , a_n)$, и~заходящие в~вершины $(a_0\oplus a_1, a_1\oplus a_2, \ldots 
, a_{n-1}\oplus a_n)$, $a_i\hm\in \{0,1\}$. Такой граф будем 
обозначать~$\Gamma_n$. Графы~$G_3$ и~$\Gamma_3$ пред\-став\-ле\-ны на 
рисунке.
  
  \begin{figure*} %fig1
\vspace*{1pt}
 \begin{center}
 \mbox{%
 \epsfxsize=158.296mm 
\epsfbox{mel-1.eps}
 }
\vspace*{6pt}

  {\small Графы $G_3$~(\textit{а}) и~$\Gamma_3$~(\textit{б})}
 \end{center}
\vspace*{-12pt}
  \end{figure*}
  
  Изоморфизм графов~$G_n$ и~$\Gamma_n$ может быть задан с~помощью 
комбинаторных формул обращения~[4].
  
  \smallskip
  
  \noindent
  \textbf{Утверждение~1.}\ \textit{Пусть $(a_1, a_2, \ldots , a_n)$, $a_i\hm\in 
\{0,1\}$,~--- вершина графа~$G_n$, $(b_1, b_2, \ldots , b_n)$, $b_i\hm\in 
\{0,1\}$,--- вершина графа~$\Gamma_n$. Отображение $\varphi_n: V_n\hm\to 
V_n$, определяемое формулой
  $\varphi_n (a_1, a_2, \ldots , a_n)\hm= (b_1, b_2, \ldots , b_n)$, где}
  $$
  b_t=\sum\limits^n_{j=t}  
\begin{pmatrix}  
n-t\\ n-j
  \end{pmatrix} a_j\,\mathrm{mod}\,2\,,\enskip t=1,\ldots ,n\,,
  $$
\textit{является изоморфизмом графов~$G_n$ и~$\Gamma_n$}.

\smallskip

\noindent
  Д\,о\,к\,а\,з\,а\,т\,е\,л\,ь\,с\,т\,в\,о\,.\ \ Отображение~$\varphi_n$~--- это 
линейное преобразование пространства~$V_n$ с~треугольной матрицей, 
составленной из биномиальных коэффициентов:
  $$\
  A=\left(
  \begin{array}{cccc}
  \begin{pmatrix} n-1\\ n-1\end{pmatrix}&
  \begin{pmatrix} n-1\\ n-2\end{pmatrix} &\cdots& \begin{pmatrix} n-1\\ 
0\end{pmatrix}\\[9pt]
  0&\begin{pmatrix} n-2\\ n-2\end{pmatrix} & \cdots & 
  \begin{pmatrix} n-2\\ 0\end{pmatrix}\\[6pt]
  \cdots &\cdots &\cdots &\cdots\\[6pt]
 0& 0 &\cdots &\begin{pmatrix}  0 \\ 0\end{pmatrix}
  \end{array}
  \right) 
  \left( \mathrm{mod}\,2\right)\,.
    $$
  %
  Поскольку $\mathrm{Det}\,(A)\hm=1$, отображение~$\varphi_n$ биективно.
  
  Осталось показать, что данное отображение соседние вершины графа~$G_n$ 
переводит в~соседние вершины графа~$\Gamma_n$. Пусть вершины $(a_1, a_2, 
\ldots , a_n)$ и~$(a_0, a_1,\ldots , a_{n-1})$ соединены дугой в~графе~$G_n$. 
Покажем, что их образы $\varphi_n (a_1, a_2, \ldots , a_n)$ 
и~$\varphi_n(a_0,a_1,\ldots , a_{n-1})$ также соединены дугой 
в~графе~$\Gamma_n$.
  
  Рассмотрим $k$-ю и~($k+1$)-ю координаты первого вектора  
и~($k +1$)-ю координату второго, $k\hm= 1,2,\ldots , n-1$:

\noindent
  \begin{align*}
  \left[ \varphi_n\left(a_1,a_2,\ldots ,a_n\right)\right]_k&= \sum\limits_{i=k}^n 
\begin{pmatrix}
  n-k\\ n-i\end{pmatrix} a_i\,;\\
  \left[ \varphi_n\left(a_1,a_2,\ldots ,a_n\right)\right]_{k+1}&= 
\sum\limits_{i=k+1}^n \begin{pmatrix}
  n-k-1\\ n-i\end{pmatrix} a_i\,;\\
  \left[ \varphi_n\left(a_0,a_1,\ldots , 
a_{n-1}\right)\right]_{k+1}&= \displaystyle\sum\limits_{i=k+1}^n \begin{pmatrix}
  n-k-1\\ n-i\end{pmatrix} a_{i-1}\,,\\
  \end{align*}
  
  Покажем, что для $k\hm= 1,2,\ldots , n-1$ выполняется соотношение:
  
  \noindent
  \begin{multline*}
  \left[ \varphi_n\left( a_1,a_2,\ldots , a_n\right)\right]_k+
  \left[ \varphi_n\left( a_1,a_2,\ldots , a_n\right) \right]_{k+1}={}\\
  {}=\left[ \varphi_n\left( a_0,a_1,\ldots , a_{n-1}\right)\right]_{k+1} 
  (\mathrm{mod}\,2)\,.
  \end{multline*}
  
  В самом деле: 
  
  \noindent
  \begin{multline*}
  \left[ \varphi_n\left( a_1,a_2,\ldots , a_n\right)\right]_k+ 
  \left[ \varphi_n\left( a_1,a_2,\ldots , a_n\right) \right]_{k+1}={}\\
  {}=\sum\limits^n_{i=k} \begin{pmatrix} n-k\\ n-i\end{pmatrix} 
  a_i\left( \mathrm{mod}\,2\right)+{}\\
  {} +\sum\limits^n_{i=k+1} \begin{pmatrix} n-k-1\\ n-i \end{pmatrix} 
  a_i \left(\mathrm{mod}\,2\right)={}
  \end{multline*}
  
\noindent
  \begin{multline*}
    {}=\sum\limits^n_{i=k} \left( \begin{pmatrix} n-k\\ n-i\end{pmatrix}+ 
 \begin{pmatrix} n-k-1\\ n-i\end{pmatrix}\right) 
 a_i \left( \mathrm{mod}\,2\right) ={}\\
  {}= 
  \sum\limits^n_{i=k} \begin{pmatrix} n-k-1\\ n-i-1\end{pmatrix} 
  a_i\left( \mathrm{mod}\,2\right)={}\\
  {}=\sum\limits^n_{j=k+1} \begin{pmatrix} n-k-1\\ n-j\end{pmatrix} 
  a_{j-1} \left(\mathrm{mod}\,2\right)={}\\
  {}=
  \left[ \varphi_n\left( a_0, a_1, \ldots , a_{n-1}\right)\right]_{k+1}\,.
  \end{multline*}
  %
  Здесь использовалось соотношение: 
  $$
  \begin{pmatrix}
  a\\b\end{pmatrix} +\begin{pmatrix}
  a-1\\b \end{pmatrix}= \begin{pmatrix}
  a-1\\ b-1 \end{pmatrix} \left( \mathrm{mod}\,2\right)\,,
  $$
  вытекающее из известного комбинаторного тождества
  $$
  \begin{pmatrix}
  a\\b\end{pmatrix}=\begin{pmatrix}
  a-1\\ b \end{pmatrix}+\begin{pmatrix}
  a-1\\ b-1\end{pmatrix}\,.
  $$
  
  
  \noindent
  \textbf{Пример.}\ $n=4$. Изоморфизм $G_4\hm\cong \Gamma_4$ 
описывается следующим образом:
  $$
  A=\!\begin{pmatrix} 
  1&3&3&1\\
  0&1&2&1\\
  0&0&1&1\\
  0&0&0&1\end{pmatrix}\!\left(\mathrm{mod}\,2\right)\!;\enskip
  \varphi_4=\!\begin{pmatrix}
  a_1\oplus a_2\oplus a_3\oplus a_4\\
  a_2\oplus a_4\\
  a_3\oplus a_4\\
  a_4 \end{pmatrix}\!.
  $$
  
  \noindent
  \textbf{Утверждение~2.}\ \textit{Справедливо равенство} $\varphi_n^{-
1}\hm=\varphi_n$, $n\hm=1,2,\ldots$
  
  \smallskip
  
  \noindent
  Д\,о\,к\,а\,з\,а\,т\,е\,л\,ь\,с\,т\,в\,о\,.\ \  
В~\cite[п.~2.2]{5-mel} показано, что мат\-ри\-ца, обратная к
  $$
  A=\displaystyle \left( \begin{pmatrix} n-1-i\\  
n-1-j\end{pmatrix}\right)_{i,j=0}^{n-1},
$$
 имеет вид: 
 $$
 A^{-1}= 
\displaystyle\left( (-1)^{i+j}\begin{pmatrix} n-1-i\\ n-1-
j\end{pmatrix}\right)_{i,j=0}^{n-1}.
$$
 По модулю два эти матрицы равны.
  
  В~[6] доказано, что у~графа~$G_n$ существуют ровно два автоморфизма, 
тождественный и~соответствующий инверсии координат вершин.
  
  \smallskip
  
  \noindent
  \textbf{Утверждение~3.} \textit{Пусть~$\phi$ и~$\eta$~--- преобразования 
линейного пространства~$V_n$ вида}:
  \begin{align*}
  \phi\left(x_1,x_2,\ldots , x_n\right)&= 
  \left( \overline{x_1},\overline{x_2},\ldots , \overline{x_n}\right)\,;\\
  \eta\left(x_1,x_2,\ldots , x_n\right) &= \left( x_1, x_2,\ldots ,  
x_{n-1}, \overline{x_n}\right)\,.
  \end{align*}
\textit{Тогда} 
\begin{itemize}
\item[\,] $\phi: G_n\to G_n$~--- \textit{автоморфизм графа}~$G_n$;
\item[\,] $\eta: \Gamma_n\to\Gamma_n$--- \textit{автоморфизм 
графа}~$\Gamma_n$,
\end{itemize}
\textit{диаграмма} 
$$
\begin{array}{ccc}
G_n&\overset{\varphi_n}{\cong}&\Gamma_n\\[3pt]
\downarrow\ \ \phi & & \downarrow\ \ \eta\\[3pt]
G_n & \overset{\varphi_n}{\cong}&\Gamma_n
\end{array}
$$
\textit{коммутативна}.
  
  \smallskip
  
  \noindent
  Д\,о\,к\,а\,з\,а\,т\,е\,л\,ь\,с\,т\,в\,о\,.\ \ Очевидно, что инверсия координат 
вершин задает автоморфизм графа де Брейна. Для доказательства 
коммутативности диаграммы и~того, что отображение $\eta: \Gamma_n\to 
\Gamma_n$ есть\linebreak автоморфизм графа~$\Gamma_n$, равенство 
$\phi\varphi_n\hm=\varphi_n\eta$ проверяется непостредственно. 
  
\section{Вероятностная функция двоичных регистров сдвига 
с~внутренним суммированием}

  
  Вероятностная функция~[7] конечного сильносвязного автомата определяется 
как действительная функция, заданная на множестве стохастических векторов, 
соответствующих возможным \mbox{полиномиальным} распределениям на входном 
алфавите. Значение функции есть предел относительной частоты встречаемости 
знака в~выходной последовательности автомата в~предположении, что на его 
вход поступает последовательность независимых одинаково распределенных по 
заданной полиномиальной схеме случайных величин. В~двоичном случае 
входная последовательность бернуллиевская с~параметром~$p$, $0\hm< p 
\hm<1$. В~качестве знака выходной последовательности рассматривается 
единица. Для вывода формулы вероятностной функции 
автомата~$A_f^{\oplus}$ получим стационарное распределение на состояниях 
автомата. Определим случайное блуж\-да\-ние на графе~$\Gamma_n$ следующим 
образом. Начальная вершина выбирается в~соответствии с~некоторой 
полиномиальной схемой на множестве~$V_n$. Шаг блуждания проходит по 
одной из исходящих из нее дуг. Предположим, что переходы из вершины 
в~вершину независимы и~вероятность шага блуждания из вершины $(a_1, a_2, 
\ldots , a_n)$ в~вершину 
$(1\oplus a_1, a_1\oplus a_2,\ldots , a_{n-1}\oplus a_n)$ 
равна~$p$, а~в~вершину $(a_1,a_1\oplus a_2,\ldots ,  
a_{n-1}\oplus a_{n})$ равна $1-p$, $0\hm< p\hm<1$. 
  
  \smallskip
  
  \noindent
  \textbf{Утверждение~4.} \textit{Стационарное распределение описанного 
случайного блуждания имеет вид}:
  $$
  P\left( a_1, a_2,\ldots , a_n\right)=\fr{1}{2^n}\,.
  $$
  
  
  \noindent
  Д\,о\,к\,а\,з\,а\,т\,е\,л\,ь\,\,с\,т\,в\,о\,.\ \ Покажем, что матрица переходных 
вероятностей марковской цепи, опи\-сы\-ва\-ющей рассматриваемое случайное 
блуждание, дваж\-ды стохастическая. Ориентированный граф~$\Gamma_n$~--- 
регулярный степени~2, т.\,е.\ из каждой вершины выходят две дуги и~в~каждую 
вершину заходят две дуги, что вытекает из доказанного выше 
изоморфизма~$\Gamma_n$ и~$G_n$. Рассмотрим вершину $(a_1, a_2, \ldots , 
a_n)$. В~нее можно попасть из двух разных вершин, которые обозначим 
$(b_1,b_2,\ldots , b_n)$ (по дуге, которая помечена символом~$b_0$) 
и~$(c_1,c_2,\ldots ,c_n)$ (по дуге, которая помечена символом~$c_0$). Тогда 
справедливо соотношение:
\begin{multline*}
  (b_1,b_2,\ldots , b_n)\oplus (b_0,b_1,\ldots , b_{n-1})= {}\\
  {}=(c_1,c_2,\ldots , c_n)\oplus (c_0,c_1,\ldots , c_{n-1})\,.
\end{multline*}
  
  Поскольку $(b_1,b_2,\ldots ,b_n)\not= (c_1,c_2,\ldots , c_n)$, то из этого 
соотношения нетрудно вывести, что $b_0\not= c_0$. Это означает, что одна из 
дуг, заходящих в~вершину $(a_1, a_2, \ldots , a_n)$, помечена символом~1, 
а~другая~--- символом~0. Следовательно, в~матрице переходных 
вероятностей в~столбце, соответствующем вершине $(a_1, a_2, \ldots , a_n)$, 
расположено ровно два ненулевых элемента: $p$ и~$1\hm-p$. Поэтому данная 
матрица~--- дваж\-ды стохастическая, что означает равномерность стационарного 
распределения. 
  
  \smallskip
  
  \noindent
  \textbf{Утверждение~5.}\ \textit{Вероятностная функция регистра сдвига 
с~внутренним суммированием с~выходной функцией~$f$ имеет вид}:
  $$
  P_{A_f^{\oplus}}(p)=\fr{\| f\|}{2^n}\,,
  $$ 
  где $\|f\|$~--- вес функции~$f$:
  $$
  \|f\|= \sum\limits_{(x_1,x_2,\ldots ,x_n)\in V_n} f\left(x_1,x_2,\ldots , x_n\right)\,.
  $$

\section{Чезарово-наследственность регистра сдвига 
с~внутренним суммированием}

  В~[8] дано следующее определение: слово называется чезаровским для 
бесконечной последовательности, если определен предел относительной 
частоты его встречаемости в~растущих начальных отрезках этой 
последовательности. Последовательность над некоторым алфавитом 
называется чезаровской, если произвольное слово над этим алфавитом является 
для нее чезаровским. Конечный автомат называют че\-за\-ро\-во-на\-след\-ст\-вен\-ным, 
если из любого начального состояния чезаровские 
последовательности во входном алфавите он перерабатывает в~чезаровские 
последовательности в~выходном алфавите. В~[8] показано, что при любой 
функции~$f$ автомат~$A_f$ че\-за\-ро\-во-на\-след\-ст\-вен\-ный.
  
  \smallskip
  
  \noindent
  \textbf{Утверждение~6.}\ \textit{Если для булевой функции $f(x_1,x_2,\ldots , 
x_n)$ выполнено условие 
$$
f(0,0,\ldots ,0)\not= f(0,0,\ldots , 0,1),
$$ 
то~$A_f^{\oplus}$ не  
че\-за\-ро\-во-на\-след\-ст\-вен\-ный автомат}.
  
  \smallskip
  
  \noindent 
  Д\,о\,к\,а\,з\,а\,т\,е\,\,л\,ь\,с\,т\,в\,о\,.\ \ В~графе автомата~$A_f$, как нетрудно 
видеть, существует ровно один цикл, движение по которому происходит при 
подаче на вход последовательности, состоящей только из нулей, это петля 
в~нулевой вершине. Для автомата~$A_f^{\oplus}$ таких циклов существует 
несколько. В~самом деле, поскольку при нулевом символе на входе автомата 
состояние $(a_1, a_2, \ldots , a_n)$ переходит в~состояние $(a_1,a_1\oplus 
a_2,\ldots , a_{n-1}\oplus a_n)$, условие наличия цикла длины~  (петли) 
выглядит так: $a_2\hm= a_1\oplus a_2, a_3\hm=a_2\oplus a_3,\ldots , 
a_n\hm=a_{n-1}\oplus a_n$. Отсюда получаем равенства: $a_i\hm=0$, 
$i\hm=1,2,\ldots , n-1$, $a_n\hm=0,1$. Поэтому в~графе 
автомата~$A_f^{\oplus}$ имеются две петли, движение по которым происходит 
при подаче~  на вход: петля в~вершине $(0,0,\ldots ,0)$ и~петля в~вершине 
$(0,0,\ldots , 0,1)$ (см.\ рисунок). 
  
  Пусть теперь выполнено условие утверждения. Из него следует, что при 
движении по одной из указанных выше петель выходная последовательность 
автомата~$A_f^{\oplus}$ состоит из единиц, а~при движении по другой эта 
последовательность состоит из нулей. Поскольку диаметр графа~$\Gamma_n$ 
равен~$n$, из состояния $(0,0,\ldots ,0)$ в~состояние $(0,0,\ldots , 0,1)$ 
и~наоборот можно перейти за не более чем $n$ шагов. Через~$\xi_{01}$ 
и~$\xi_{10}$   обозначим входные последовательности, которые 
обеспечивают эти переходы. Рассмотрим теперь бесконечную входную 
последовательность
  $$
  \chi=\xi_0\, 0^{k_1}\, \xi_{01}\, 0^{k_2}\, \xi_{10}\,
   0^{k_3}\, \xi_{01}\, 0^{k_4} \,
\xi_{10}\ldots ,
  $$
где $\xi_0$~--- входная последовательность, переводящая 
автомат~$A_f^{\oplus}$ из заданного состояния в~состояние $(0,,0,\ldots ,0)$; 
$0^{k_i}$~--- последовательность, состоящая из~$k_i$ нулей, $k_i$~--- целые 
числа, $i\hm=1,2,\ldots$ Выходной последовательностью, очевидно, будет
$$
\gamma= \zeta_0 \,0^{k_1} \,\zeta_{01}\, 1^{k_2}\, \zeta_{10}\, 0^{k_3}\, \zeta_{01} \,
1^{k_4}\, \zeta_{10}\ldots\,,
$$
где $\zeta_0$, $\zeta_{01}$ и~$\zeta_{10}$~--- некоторые двоичные 
последовательности, длина каждой из которых не превосходит~$n$; 
$1^{k_j}$~--- последовательность, состоящая из $k_j$ единиц.
  
  Нетрудно убедиться в~том, что при $k_i\hm= 2^{2^i}$ 
последовательность~$\xi$ чезаровская, а~для последовательности~$\gamma$ не 
существует предела отностительной частоты встречаемости единицы 
в~растущих начальных отрезках. Поэтому она не чезаровская 
и~автомат~$A_f^{\oplus}$ не че\-за\-ро\-во-на\-след\-ст\-вен\-ный.
  
\section{Заключение}

  Регистры сдвига различных типов широко применяются в~качестве узлов 
генераторов случайных последовательностей. В~работе исследованы 
алгебраические и~статистические свойства семейства двоичных регистров 
сдвига с~внутренним суммированием. Доказан изоморфизм графов переходов 
регистров сдвига с~внутренним суммированием и~обычных регистров. 
Доказано, что, в~отличие от\linebreak обычного регистра сдвига, при случайном 
бернуллиевском входе вероятностное распределение на состояниях регистра 
с~внутренним суммированием равномерное. Показано, что рассматриваемые 
регистры не обладают свойством че\-за\-ро\-во-на\-след\-ст\-вен\-ности, которым 
обладают обычные регистры сдвига. 
  
{\small\frenchspacing
 {%\baselineskip=10.8pt
 \addcontentsline{toc}{section}{References}
 \begin{thebibliography}{9}
  
\bibitem{1-mel}
\Au{Грушо А.\,А., Применко Э.\,А., Тимонина~Е.\,Е.} Теоретические основы компьютерной 
безопасности.~--- М.: 
Академия, 2009. 267~с.
\bibitem{2-mel}
\Au{Golomb S.\,W.} Shift register sequences.~--- Laguna Hills, CA, USA: Aegean Park Press, 
1981. 247~p.
\bibitem{3-mel}
The VLSI handbook~/ Ed. W.-K.~Chen.~--- 2nd ed.~--- Chicago, IL, USA: CRC Press, 2006. 2320~p. 
\bibitem{4-mel}
\Au{Сачков В.\,Н.} Курс комбинаторного анализа.~--- Ижевск: НИЦ РХД, 2013. 336~с. 
\bibitem{5-mel}
\Au{Риордан Дж.} Комбинаторные тождества~/ Пер. с~англ.~--- М.: Наука, 1982. 256~с. 
(\Au{Riordan~J.} 1968. Combinatorial identities.~--- New York, NY, USA: Wiley. 256~p.)
\bibitem{6-mel}
\Au{Liu M.} Homomorphisms and automorphisms of 2-D de Bruijn--Good graphs~// Discrete 
Math., 1990. Vol.~85. Iss.~1. P.~105--109. 
\bibitem{7-mel}
\Au{Melnikov S.\,Yu., Samouylov~K.\,E.} The recognition of the output function of a~finite 
automaton with random input~// Distributed Computer and Communication Networks: 
21st Conference (International) Revised Selected Papers~/ Eds. 
V.\,M.~Vishnevskiy, D.\,V.~Kozyrev.~--- Communications in computer and information 
science ser.~--- Springer, 2018. Vol.~919. P.~525--531. doi: 10.1007/978-3-319-99447-5\_45.
\bibitem{8-mel}
\Au{Мельников С.\,Ю.} О переработке конечными автоматами чезаровских 
последовательностей~// Лесной вестник, 2004. №\,1(32). 
С.~169--174.
\end{thebibliography}

 }
 }

\end{multicols}

\vspace*{-9pt}

\hfill{\small\textit{Поступила в~редакцию 14.04.20}}

\vspace*{6pt}

%\pagebreak

%\newpage

%\vspace*{-28pt}

\hrule

\vspace*{2pt}

\hrule

%\vspace*{-2pt}

\def\tit{STATISTICAL PROPERTIES OF~BINARY NONAUTONOMOUS SHIFT REGISTERS 
WITH~INTERNAL XOR}


\def\titkol{Statistical properties of binary nonautonomous shift registers with 
internal XOR}

\def\aut{S.\,Yu.~Melnikov and K.\,E.~Samouylov}

\def\autkol{S.\,Yu.~Melnikov and K.\,E.~Samouylov}

\titel{\tit}{\aut}{\autkol}{\titkol}

\vspace*{-15pt}


\noindent
Peoples' Friendship University of Russia (RUDN University), 6~Miklukho-Maklaya Str., Moscow 117198, Russian 
Federation

\def\leftfootline{\small{\textbf{\thepage}
\hfill INFORMATIKA I EE PRIMENENIYA~--- INFORMATICS AND
APPLICATIONS\ \ \ 2020\ \ \ volume~14\ \ \ issue\ 2}
}%
 \def\rightfootline{\small{INFORMATIKA I EE PRIMENENIYA~---
INFORMATICS AND APPLICATIONS\ \ \ 2020\ \ \ volume~14\ \ \ issue\ 2
\hfill \textbf{\thepage}}}

\vspace*{3pt} 


\Abste{ The statistical and algebraic properties of binary nonautonomous shift registers and shift registers with internal 
XOR are compared, during which the state vector is summed with its one-step shift. The isomorphism of transition 
graphs of these automata is proved. It is shown that, with a~Bernoulli random input, the stationary distribution of the 
register states with internal XOR is uniform. The form of the probability function of these registers is obtained. It is 
shown that, under certain conditions on the output function, registers with internal XOR are not Cesaro-hereditary. The 
authors show input sequences that possess the property of stability of the relative frequencies of arbitrary multigrams, 
while output sequences do not have this property.}

\KWE{random input automata; shift register; de Bruijn graph}


\DOI{10.14357/19922264200211} 

\vspace*{-20pt}

\Ack
\noindent
 The publication has been prepared with the support of the ``RUDN University 
Program 5-100'' (recipient K.\,E.~Samouylov, problem statement). The reported 
study was partially funded by the Russian Foundation for Basic Research, projects  
18-00-01555, 18-00-01685, and 19-07-00933.


\vspace*{5pt}

 \begin{multicols}{2}

\renewcommand{\bibname}{\protect\rmfamily References}
%\renewcommand{\bibname}{\large\protect\rm References}

{\small\frenchspacing
 {%\baselineskip=10.8pt
 \addcontentsline{toc}{section}{References}
 \begin{thebibliography}{9}
 
\vspace*{-8pt}

\bibitem{1-mel-1}
\Aue{Grusho, A.\,A., E.\,A. Primenko, and E.\,E.~Timonina.} 2009. 
\textit{Teoreticheskie osnovy komp'yuternoy bezopasnosti} 
[Theoretical foundations of computer 
security]. Moscow: Akademiya. 267~p.
\bibitem{2-mel-1}
\Aue{Golomb, S.\,W.} 1981. \textit{Shift register sequences}. Laguna Hills, CA: 
Aegean Park Press. 247~p.
\bibitem{3-mel-1}
Chen, W.-K., ed. 2006. \textit{The VLSI handbook}. 2nd ed. Chicago, IL: CRC Press. 
2320~p.

\pagebreak 
 
\bibitem{4-mel-1}
\Aue{Sachkov, V.\,N.} 2013. \textit{Kurs kombinatornogo analiza} [Combinatorial 
analysis course]. Izhevsk: NITS RKHD. 336~p.
\bibitem{5-mel-1}
\Aue{Riordan, J.} 1968. \textit{Combinatorial identities}. New York, NY: Wiley. 
256~p.
\bibitem{6-mel-1}
\Aue{Liu, M.} 1990. Homomorphisms and automorphisms of \mbox{2-D} de Bruijn-good 
graphs. \textit{Discrete Math.} 85(1):105--109.
\bibitem{7-mel-1}
\Aue{Melnikov, S.\,Yu., and K.\,E.~Samouylov.} 2018. The recognition of the output 
function of a~finite automaton with random input. \textit{Distributed Computer and 
Communication Networks: 21st Conference (International) Revised Selected 
Papers}. Eds. V.\,M.~Vishnevskiy and D.\,V.~Kozyrev. Communications in 
computer and information science ser. Springer. 919:525--531.
\bibitem{8-mel-1}
\Aue{Melnikov, S.\,Yu.} 2004. O~pererabotke konechnymi avtomatami 
chezarovskikh posledovatel'nostey [On the
 finite automa with Cesaro sequences input]. 
\textit{Lesnoy Vestnik} [Forestry Bull.] 1(32):169--174.

\end{thebibliography}

 }
 }

\end{multicols}

\vspace*{-9pt}

\hfill{\small\textit{Received April 14, 2020}}

%\pagebreak



\Contr

\noindent
\textbf{Melnikov Sergey Yu.} (b.\ 1963)~--- Candidate of Science (PhD) in physics 
and mathematics, doctoral student, Peoples' Friendship University of Russia (RUDN 
University), 6~Miklukho-Maklaya Str., Moscow 117198, Russian Federation; 
\mbox{melnikov@linfotech.ru}

\vspace*{3pt}

\noindent
\textbf{Samouylov Konstantin E.} (b.\ 1955)~--- Doctor of Science in technology, 
professor, Head of Department, Peoples' Friendship University of Russia (RUDN 
University), 6~Miklukho-Maklaya Str., Moscow 117198, Russian Federation; 
\mbox{ksam@sci.pfu.edu.ru}


\label{end\stat}

\renewcommand{\bibname}{\protect\rm Литература} 