
%\def\l{\lambda}
%\def\ld{\ldots}

\def\stat{razumchik}

\def\tit{СТАЦИОНАРНЫЕ ХАРАКТЕРИСТИКИ СИСТЕМЫ $M/G/2/\infty$ С~ОДНИМ 
ЧАСТНЫМ СЛУЧАЕМ 
ДИСЦИПЛИНЫ ИНВЕРСИОННОГО ПОРЯДКА ОБСЛУЖИВАНИЯ С~ОБОБЩЕННЫМ ВЕРОЯТНОСТНЫМ 
ПРИОРИТЕТОМ$^*$}

\def\titkol{Стационарные характеристики системы $M/G/2/\infty$ с~одним 
частным случаем дисциплины
инверсионного порядка} % обслуживания с~обобщенным вероятностным приоритетом}

\def\aut{Л.\,А.\ Мейханаджян$^1$, Р.\,В.~Разумчик$^2$}

\def\autkol{Л.\,А.\ Мейханаджян, Р.\,В.~Разумчик}

\titel{\tit}{\aut}{\autkol}{\titkol}

\index{Мейханаджян Л.\,А.}
\index{Разумчик Р.\,В.}
\index{Meykhanadzhyan L.\,A.}
\index{Razumchik R.\,V.}
 

{\renewcommand{\thefootnote}{\fnsymbol{footnote}} \footnotetext[1]
{Работа выполнена при поддержке РФФИ (проект 18-37-00283).}}


\renewcommand{\thefootnote}{\arabic{footnote}}
\footnotetext[1]{Финансовый университет при Правительстве РФ,
lamejkhanadzhyan@fa.ru}
\footnotetext[2]{Институт проблем информатики Федерального 
исследовательского центра <<Информатика 
и~управление>> Российской академии наук, \mbox{rrazumchik@ipiran.ru}}
%\vspace*{-6pt}


\Abst{Рассматривается система массового обслуживания (СМО) $M/G/2/\infty$ 
с~идентичными приборами.
Предполагается, что в~любой момент времени известна остаточная длина 
каждой заявки в~системе.
В~системе реализован частный случай дисциплины инверсионного порядка 
обслуживания 
с обобщенным вероятностным приоритетом, заключающийся в~следующем. 
В~момент поступления новой заявки в~систему все заявки на приборах 
приобретают новую случайную длину. 
При этом, если есть хотя бы один свободный прибор, новая заявка занимает 
его; иначе она становится на первое место в~очереди. Длины заявок 
представляют собой независимые одинаково распределенные 
случайные величины с~произвольным абсолютно непрерывным распределением. 
В~предположении, что стационарный режим существует, найдены основные 
стационарные характеристики функционирования, включая
совместное стационарное распределение общего числа заявок в~системе 
и~остаточных длин
заявок на приборах.}


\KW{многолинейная система; инверсионный порядок обслуживания; 
вероятностный приоритет}

\DOI{10.14357/19922264200209}
 
 
%\vspace*{9pt}


\vskip 10pt plus 9pt minus 6pt

\thispagestyle{headings}

\begin{multicols}{2}

\label{st\stat}


\section{Введение}

В работе~\cite{xx2} были изучены стационарные характеристики СМО
 $M/G/1/\infty$ с~одним из частных случаев дисциплины инверсионного порядка обслуживания 
с обобщенным вероятностным приоритетом~\cite{xx3}, 
названной там дисциплиной ресамплинга. Она заключается в~следующем. 
Предполагается, что в~момент поступления новой заявки становится известно 
ее
остаточное время обслуживания.
В~дальнейшем удобно называть случайную величину остаточного времени 
обслуживания остаточной длиной, подразумевая, что длина измеряется 
в~единицах времени.
Каждая поступающая в~непустую систему заявка назначает новую остаточную 
длину заявке на приборе
и становится на первое место в~очереди. Когда остаточная длина заявки на 
приборе
становится равной нулю, заявка навсегда покидает систему и~на 
обслуживание
выбирается первая заявка из очереди. 
Также в~\cite{xx2,xx3,xx4,xx5,xx6} было показано, что описанная выше СМО 
оказывается полезной
в задачах оценки среднего времени пребывания в~однолинейных системах 
с неточной априорной информацией о~временах обслуживания. 

Представляет несомненный интерес изучение обобщения полученных 
в~\cite{xx2}
результатов на случай многолинейных систем\footnote[3]{Отметим, что вопросы
анализа многолинейных СМО $M/G/n/\infty$ остаются по большей части 
открытыми
и связаны с~серьезными трудностями. 
Одно из немногих исключений~--- СМО $M/G/2/\infty$, для которой известен 
ряд
аналитических результатов
(см., например, \cite{xx7,xx8,xx9}).}. В этой работе исследуется одно
такое обобщение: находятся основные стационарные характеристики в~системе
$M/G/2/\infty$ с~дисциплиной ресамплинга и~идентичными приборами. 



\section{Описание системы}

Рассмотрим систему с~двумя идентичными приборами
и очередью неограниченной емкости, на вход которой поступает
пуассоновский поток заявок интенсивности~$\lambda$.


Определим дисциплину обслуживания сле\-ду\-ющим образом.
В~момент прихода очередной заявки в~систему 
становится известно ее остаточное время обслуживания (далее~--- 
остаточная длина, 
или прос\-то длина) и~прерывается обслуживание заявок на всех приборах.
Каждой заявке, обслуживание которой было прервано, 
независимо от всей предыс\-то\-рии функционирования системы
назначается новая остаточная длина. 
Затем их обслуживание возобновляется. Новая заявка становится на 
свободный прибор,
если такой имеется; в~противном случае она занимает первое место 
в~очереди. 
Когда остаточная длина заявки на приборе становится равной нулю, она 
покидает систему
и~на обслуживание выбирается заявка с~первого мес\-та в~очереди. 
Остаточные длины заявок пред\-став\-ля\-ют собой независимые одинаково 
распределенные случайные величины 
с~произвольной функцией распределения~$B(x)$. Для упрощения выкладок будем 
предполагать
существование у~распределения~$B(x)$ непрерывной плотности 
$b(x)\hm=B'(x)$.
Также всюду в~дальнейшем будем предполагать, что 
выполняются условия существования стационарного распределения.

\vspace*{-2pt}

\section{Вспомогательные функции}

\vspace*{-2pt}

Пусть в~некоторый момент в~систему поступила новая заявка и~
сразу после ее поступления в~системе оказалось $n\hm\ge 3$ заявок
с длинами\footnote{Предполагается, что 
на первом приборе заявка длины $y_1$, на втором~--- $y_2$,
в очереди на первом месте заявка длины~$y_3$, 
на втором~--- $y_4$ и~т.\,д.} $y_1,\dots,y_n$. 
Обозначим через $f_n(s;x_1,x_2,y_4,\dots,y_{n}|y_1,\dots,y_n)$
преобразование Лап\-ла\-са--Стилть\-е\-са (ПЛС) времени до момента, когда 
в~системе впервые останется $(n-1)$ заявка и~
плотность вероятности того, что в~тот же момент длины оставшихся 
в~системе заявок будут
равны $x_1,x_2,y_4,\dots,y_{n}$.
Из описания системы и~свойств дисциплины обслуживания
следует, что функции $f_n$ 
симметричны на паре переменных $(x_1,x_2)$,
не зависят от $y_4, y_5, \dots, y_n$ и~совпадают при $n\hm\ge 3$.
Воспользовавшись формулой полной вероятности, получаем уравнение для 
расчета
условной плотности $f=f_n$, $n\ge 3$:
\begin{multline}
\label{fx1x2}
f\left(s;x_1,x_2|y_1,y_2,y_3\right) ={}\\
{}={\mathbf{1}_{(y_1 \le y_2)}}
e^{-(\lambda+s) y_1}
\delta \left ( y_2-(y_1+x_2)\right)
\delta(y_3-x_1)
+{}\\
{}+
{\mathbf{1}_{(y_1>y_2)}}
e^{-(\lambda+s) y_2}
\delta \left ( y_1-(y_2+x_1) \right )
\delta(y_3-x_2)
+{}\\
{}+
\fr{\lambda}{\lambda+ s} \left (1 - e^{-(\lambda+s) \min(y_1,y_2)} 
\right)\times{}\\
\!\!{}\times \int\limits_0^\infty\! \int\limits_0^\infty  \!\!
f\left(s;u_1,u_2\right)
f\left(s;x_1,x_2|u_1,u_2,y_3\right)du_1 du_2,\!\!
\end{multline}
где 

\noindent
\begin{multline*}
f(s;x_1,x_2)=
\int\limits_0^\infty \int\limits_0^\infty 
\int\limits_0^\infty f(s;x_1,x_2|u_1,u_2,u_3)\times{}\\
{}\times b(u_1)b(u_2)b(u_3)\,du_1 du_2 du_3; 
\end{multline*}
$\delta$~--- дельта-функция Дирака; 
${\mathbf{1}_{(A)}}$~--- индикатор множества~$A$.
Решая~\eqref{fx1x2}, получаем, что для $f(s;x_1,x_2)$
справедлива формула 
$$
f\left(s;x_1,x_2\right)=b\left(x_1\right)g\left(s;x_2\right)+b\left(x_2\right)
g\left(s;x_1\right),
$$
в которой неизвестная функция~$g$ есть решение интегрального уравнения: 

\noindent
\begin{equation}
\label{g}
g(s;x)=y(s;x)+ \gamma(s) \int\limits_0^\infty K(s;u,x)g(u)\,du\,, 
\end{equation}
где 
\begin{align*}
y(s;x)&=\int\limits_{0}^\infty
e^{-(\lambda+s) u} b(u)b(u+x)\, du\,;\\
K(s;u,x)&=\theta(u-x) e^{-(\lambda+s) (u-x)} b(u-x)+ {}\\
&\hspace*{25mm}{}+e^{-(\lambda+s) u}b(u+x);
\end{align*}
$\theta$~--- функция Хевисайда\footnote{Это означает, что $\theta(x)=1$ при $x\ge 
0$ и~$\theta(x)\hm=0$ иначе.};
\begin{multline*}
\gamma(s) ={}\\
{}= %\fr{
\!\left[\fr{2\lambda}{\lambda+ s} \!\int\limits_{0}^\infty \!
\int\limits_{0}^u b(u)b(v) 
\left (1 -e^{- (\lambda+s) v}\right )dv du\right]\times{}\\ 
{}\times 
\left[1- \fr{2\lambda}{\lambda+ s}
\int\limits_{0}^\infty
\int\limits_{0}^{u}
\left (1 - e^{-(\lambda+s) v} \right)\right.
\left( b(u)g(s;v)\!+{}\right.\\
\left.\left.{}+ b(v)g(s;u) \right)
\vphantom{\int\limits_0^T }
dv du\right]^{-1}\,.
\end{multline*}

\noindent
Отметим, что значение~$\gamma(s)$ зависит от неизвестной функции~$g$,
и уравнение~\eqref{g}, по-видимому, не обладает хорошими особенностями, 
кроме одной: 
свободный член и~ядро являются неотрицательными функциями.
В некоторых частных случаях\footnote{Например, при $s\hm=0$
в случае $B(x)\hm=1-e^{-\mu x}$ решение~\eqref{g} есть 
$g(0;x)\hm=(1/2)\mu e^{-\mu x}$.}
решение~\eqref{g} может быть выписано в~явном виде. В~общем же случае его 
приходится искать численно.
Хорошие результаты дает итерационный метод, причем в~качестве начальной 
итерации
необходимо брать нулевое приближение. Тогда итерации будут возрастающими, 
что
позволит контролировать сходимость к~точному решению. При $s\hm=0$ для 
контроля точности
можно пользоваться условием нормировки, из которого следует, что 
$$
\int\limits_0^\infty g(0;x)\,dx=\fr{1}{2}\,.
$$

Пусть теперь в~некоторый момент в~систему поступила новая заявка и~
сразу после ее поступления в~системе оказалось две заявки
с остаточными длинами~$y_1$ и~$y_2$. Обозначим через $f_2(s;x|y_1,y_2)$ 
ПЛС времени до момента, когда в~системе впервые останется одна заявка, и~
плот\-ность ве\-ро\-ят\-ности того, что в~тот же момент ее остаточная длина будет 
равна~$x$.
Выписывая для $f_2(s;x|y_1,y_2)$ уравнение, аналогичное~\eqref{fx1x2},
и решая его, получаем, что $f_2(s;x)\hm=2g(s;x)$.

\section{Стационарные вероятности состояний}


Рассмотрим процесс $(\xi_{1}(t),\ldots,\xi_{\nu(t)}(t))$.
Здесь $\nu(t)\hm=n$, когда в~момент~$t$ в~системе находится~$n$~заявок. 
Координаты $\xi_{1}(t)$ и~$\xi_{2}(t)$~--- это
остаточные времена обслуживания заявок на приборах,
$\xi_{3}(t)$~--- длина первой заявки в~очереди, а
$\xi_{\nu(t)}(t)$~--- последней.
В~том случае, когда в~системе отсутствуют заявки, 
координаты $\xi_i(t)$ не определяются.
Наконец, при $\nu(t)\hm=1$ координата~$\xi_{1}(t)$
хранит остаточное время обслуживания находящейся на приборе 
единственной заявки.
Процесс $(\xi_{1}(t),\ldots,\xi_{\nu(t)}(t))$
является марковским и~описывает характеристики очереди в~момент~$t$.

Обозначим через
\begin{align*}
%\label{(2.1)}
P_{0}&=
\lim\limits_{t\to\infty}
P_{0}(t),
\\
%\label{(2.1)}
P_{n}\left(x_1,\ldots,x_{n}\right)
&=
\lim\limits_{t\to\infty}
P_{n}\left(t;x_1,\ldots,x_{n}\right),\enskip \ n\ge 1\,,
\end{align*}
\noindent стационарное распределение процесса 
$(\xi_{1}(t),\hm\ldots,\xi_{\nu(t)}(t))$,
где $P_{0}(t)\hm=\mathsf{P} \left (\nu(t)\hm=0 \right )$,
$P_{n}(t;x_1,\hm\ldots,x_{n})\hm=\mathsf{P}\left (\nu(t)\hm=n,\, 
\xi_{1}(t)<x_{1},\hm\ldots,\xi_{n}(t)\hm<x_{n} \right )$.
Для плотностей введенных вероятностей (в предположении, что они 
существуют)
можно выписать интегродифференциальные уравнения, на основе которых 
теоретически можно производить вы\-чис\-ле\-ния. Однако на практике 
они не реализуемы уже при совсем небольших значениях~$n$,
поскольку размерность уравнений растет пропорционально~$n$.
%
Поэтому далее речь пойдет только о~маргинальных стационарных вероятностях
$P_{1}(x)$, $P_{n}(x_1,x_2)\hm=P_{n}(x_1,x_2,\infty,\ldots,\infty)$, $n 
\hm\ge 2$,
для которых предполагается существование плотностей: 
$$
p_1(x)=P'_{1}(x)\,;\ 
p_n(x_1,x_2)=\fr{\partial^2 P_{n}(x_1,x_2)}{\partial x_1\, \partial 
x_2},\ n \ge 2\,.
$$

Принятая дисциплина обслуживания обладает известным свойством, которое 
в данном случае позволяет рекуррентно вычислять стационарные плотности
(подробнее см.~\cite{xx1}) и~которое заключается в~следующем.
Пусть $n\hm>1$~--- произвольное целое число. Выделим для процесса
$\nu(t)$ те интервалы времени, когда число заявок в~системе будет 
больше~$n$, т.\,е.\ $\nu(t)\hm>n$. Тогда, в~силу дисциплины обслуживания, 
с того момента, как в~системе впервые появится 
$(n+1)$-я заявка, и~до того момента, как в~системе снова будет~$n$
заявок, последние $(n-1)$ компонент процесса 
$(\xi_{1}(t),\ldots,\xi_{\nu(t)}(t))$
не меняются. Следовательно, если для процесса 
выкинуть все те интервалы времени, когда $\nu(t)\hm>n$,
и~оставшиеся куски склеить, то вероятностные характеристики 
получившегося после склейки процесса будут одинаковыми для всех~$n$
и плотности 
$p_1(x)$, $p_k(x_1,x_2)$, $2 \hm\le k \hm\le n$, 
будут совпадать с~точностью до постоянного множителя, не зависящего 
от~$k$.

Положим 
$$
P_1=\int\limits_0^\infty p_1(u)\,du\,;\ 
P_n=\int\limits_0^\infty \int\limits_0^\infty
 p_n(u,v)\,dudv,\ n \ge 2.
 $$ 
С учетом описанного выше свойства дисциплины обслуживания,
уравнения Кол\-мо\-го\-р\-ова--Чеп\-ме\-на для стационарных плотностей
будут иметь вид:
\begin{gather}
\label{p1x}
-p'_{1}(x)=-\lambda p_{1}(x)+\lambda b(x) P_{0}+\lambda f_2(0;x) P_1 ;\\
!\!fr{\partial p_{n}(x_1,x_2) }{\partial x_1}-
\fr{\partial p_{n}(x_1,x_2)}{\partial x_2}=- \lambda p_{n}(x_1,x_2)+{}\notag\\
\!\!\!{}+ \lambda b(x_1)b(x_2) P_{n-1} +
\lambda f_{n+1}(0;x_1,x_2) P_{n}, \ n \ge 2,\!
\label{p2xy}
\end{gather}
с начальными условиями 
\begin{align*}
\lim\limits_{x\rightarrow \infty }p_1(x)&=0\,;\\
\lim\limits_{x_1\rightarrow \infty } p_n(x_1,x_2)&=0\,;\\
\lim\limits_{x_2\rightarrow \infty } p_n(x_1,x_2)&=0,\ n \ge 2\,.
\end{align*}

Систему~\eqref{p1x}--\eqref{p2xy} можно решить следующим образом.
Заметим, что~\eqref{p2xy} есть уравнение в~частных производных первого 
порядка.
Воспользовавшись методом характеристик, находим выражение для 
плотности $p_{n}(x_1,x_2)$:
\begin{multline}
\label{pnx1x2}
p_{n}(x_1,x_2)
={}\\
{}=
e^{\lambda x_1}
\int\limits_{x_1}^{\infty}
e^{-\lambda u}
\left (
\lambda b(u)b\left(x_2-x_1+u\right) P_{n-1} +{}\right.\\
\left.{}+ \lambda f\left(0;u,x_2-
x_1+u\right) P_{n}
\right )\,du\,,\enskip n\geq 2\,,
\end{multline}
\noindent 
где 
$$
f\left(0;x_1,x_2\right)=b\left(x_1\right)g\left(0;x_2\right)+b\left(x_2\right)
g\left(0;x_1\right);
$$
функция~$g$ есть 
решение уравнения~\eqref{g} при $s\hm=0$.
Для нахождения неизвестных в~\eqref{pnx1x2} вероятностей~$P_{n-1}$ 
и~$P_n$
проинтегрируем~\eqref{pnx1x2} по всем значениям~$x_1$ и~$x_2$.
С~по\-мощью обычных преобразований, вводя для сокращения записи 
обозначения 
\begin{align*}
{\hat \beta}(\lambda)
&=
\lambda \int\limits_{0}^{\infty}
e^{-\lambda u}(1-B(u))^2
\,du\,;
\\
{\bar \beta}(\lambda)
&=
2 \lambda \int\limits_{0}^{\infty}
\int\limits_{0}^{\infty}
e^{-\lambda z} (1-B(z))
g(0;z+x) \,dx dz\,,
\end{align*}

\noindent 
получим соотношение 
$$
P_{n}={\hat \beta}(\lambda)
P_{n-1} + {\bar \beta}(\lambda) P_{n}\,, 
$$
из которого следует, что 
\begin{equation*}
\label{pn}
P_{n}
=
\left (
\fr{
{\hat \beta}(\lambda)}{1- {\bar \beta}(\lambda)}
\right )^{n-1} P_1, \ n\geq 1.
\end{equation*}
\noindent Поступая аналогичным образом с~решением уравнения~\eqref{p1x},
которое имеет вид: 

\noindent
\begin{eqnarray*}
\label{p1xsol}
p_{1}(x) = 
e^{\lambda x}
\int\limits_{x}^{\infty}
e^{-\lambda u}
(\lambda b(u) P_0 + 2 \lambda g(0;u) P_1)\,du,
\end{eqnarray*}

\noindent
находим 
$$
P_1=P_0\fr{1 - \beta(\lambda)}{1 - {\tilde \beta}(\lambda)},
$$
где $\beta(\lambda)$ --- ПЛС $B(x)$;
$$
{\tilde \beta}(\lambda)=
2 \lambda \int\limits_{0}^{\infty}
\int\limits_{0}^{\infty}
e^{-\lambda z}
g(0;z+x) \,dx dz\,.
$$

Оставшаяся неизвестной вероятность~$P_0$, как обычно, находится из 
условия нормировки 
$\sum\nolimits_{i=0}^\infty P_i\hm=1$, т.\,е.
\begin{eqnarray}
\label{p0}
P_0=
\left (1+
\fr{1 - \beta(\lambda) }{1 - {\tilde \beta}(\lambda)}\,
\fr{1- {\bar \beta}(\lambda)
}{1- {\bar \beta}(\lambda)-{\hat \beta}(\lambda)}
\right )^{-1}.
\end{eqnarray}
\noindent
Таким образом, стационарное распределение общего числа заявок
в системе образует начиная с~$P_1$ геометрическую прогрессию,
что позволяет легко находить моменты числа заявок в~системе. 
В~частности, среднее число~$N$ заявок в~системе в~стационарном режиме
равно\footnote{Примечательно, что такая же формула для $N$ 
через вероятности состояний системы имеет место и~в~случае одноканальной 
системы~\cite[Remark~3]{xx2}.} $N\hm=(1-P_0)^2/P_1$. 

В заключение раздела отметим, что из~\eqref{p0} и~того, что $P_0\hm>0$, 
следует необходимое условие существования стационарного распределения.


\section{Стационарное распределение времени пребывания}

Время пребывания в~системе заявки пред\-став\-ля\-ет собой сумму двух 
независимых частей: 
времени ожидания начала обслуживания и~собственно времени нахождения
заявки на приборе. 
Обозначим соответствующие ПЛС через~$\chi(s)$, $\psi(s)$ и~$\omega(s)$.
Поскольку с~вероятностью $P_0\hm+P_1$ поступающая заявка 
попадает сразу на прибор, а~с дополнительной вероятностью~--- в~очередь, 
то ПЛС~$\chi(s)$ времени пребывания заявки в~системе имеет вид:
\begin{equation}
\label{chi}
\chi(s)=
\left(P_0+P_1\right)\psi(s)+
\left(1-P_0-P_1\right)
\psi(s) \omega(s).
\end{equation}

\noindent Из результатов разд.~3 немедленно следует, что
$$
\omega(s)=\int\limits_0^\infty f_2(s;u)\,du\,. 
$$
Нахождение же ПЛС $\psi(s)$ времени пребывания заявки на приборе 
(с учетом всех возможных прерываний, т.\,е.\ ее истинного времени 
обслуживания) 
ничем не отличается от на\-хож\-де\-ния этого ПЛС в~случае однолинейной 
системы. Поэтому, согласно 
\cite[формула (13)]{xx2}, имеем:
$$
\psi(s)
=
\fr{\beta(\lambda+s) (\lambda + s) 
}{s + \lambda \beta(\lambda+s)}\,.
$$

Остановимся на нахождении стационарного распределения периода занятости 
системы (ПЗ).
Будем считать, что он начинается в~момент поступления заявки в~пустую 
систему 
и заканчивается в~тот момент, когда система впервые оказалась свободной 
от заявок. 
Обозначим ПЛС ПЗ, открываемого заявкой длины~$x$,
через~$\vartheta(s;x)$. Воспользовавшись формулой полной вероятности, 
получим интегральное уравнение для~$\vartheta(s;x)$, решение 
которого имеет вид:
\begin{multline*}
\vartheta(s;x)= e^{-(s+\lambda)x}+\left(1-e^{-(\lambda+s) x}\right) %\fr{
\times{}\\
{}\times\left[\fr{2\lambda}{\lambda+s}
\int\limits_{0}^\infty
e^{-(s+\lambda)u}
g(s;u)\,dv\right]\times{}\\
{}\times
\left[1- 
\fr{2 \lambda}{\lambda+s} \int\limits_{0}^\infty
\left(1-e^{-(\lambda+s) u}\right)
g(s;u)\,du\right]^{-1}\,.
\end{multline*}
\noindent
Безусловное ПЛС~$\vartheta(s)$ ПЗ получается усреднением~$\vartheta(s;x)$ 
по длине заявки, т.\,е.\
$$
\vartheta(s)=\int\limits_0^\infty \vartheta(s;u)b(u)\,du\,.
$$

К~сожалению, найденные выражения для стационарных распределений
времени пребывания и~ПЗ, по-видимому, не могут 
привести к~явным выражениям для моментов и~
позволяют их находить лишь численно, 
причем при расчетах основная сложность связана с~решением 
уравнения~\eqref{g}. 


\section{Заключение}


В~\cite{xx2} доказано, что для аналогичной, но одноканальной 
системы справедлива формула Литтла. Предложенный в~этой работе 
метод анализа, по-видимому, не позволяет установить 
ее справедливость для двухканальной системы.
Однако чис\-лен\-ные эксперименты показывают, что формула Литтла имеет место 
и~в~этом случае. 
Тогда для расчета стационарного среднего времени~$v$ 
пребывания заявки в~системе нет необходимости 
дифференцировать~\eqref{chi} и~можно пользоваться формулой 
$$
v=\fr{N}{\lambda} =\fr{(1\hm-P_0)^2}{\lambda P_1}\,.
$$

Всюду выше предполагалось существование стационарного распределения,
но критерий не был найден; получено лишь необходимое условие.
Результаты, полученные для одноканальной системы в~\cite{xx2}, 
подсказывают, что для двухканальной системы 
необходимое и~достаточное условие существования стационарного режима,
по-видимому, не должно зависеть от моментов 
длины заявки какого-либо порядка, т.\,е.\ для любого
распределения длины заявки при достаточно малой интенсивности~$\lambda$ 
существует стационарное распределение. 
Как показывают численные эксперименты, средняя длина ПЗ 
рассматриваемой системы равна\footnote[1]{В отличие от случая 
одноканальной системы, в~которой стационарное среднее время пребывания 
заявки в~системе совпадает со средней длиной ПЗ~\cite[Corollary~3]{xx2}.} 
${3}/{2}v$.
Отсюда получаем (при условии справедливости формулы Литтла),
что $P_0\hm>0$ является не только необходимым, но и~достаточным условием 
существования стационарного распределения.
 
Отметим наконец, что полученные формулы не позволяют производить расчеты 
в~случае, 
когда длины заявок принимают только конечное число значений. 
Этот случай требует специального исследования. 

{\small\frenchspacing
 {%\baselineskip=10.8pt
 \addcontentsline{toc}{section}{References}
 \begin{thebibliography}{9}

\bibitem{xx2} 
\Au{Horv$\acute{\mbox{a}}$th I., Razumchik R., Telek~M.} 
The resampling $M/G/1$ non-preemptive LIFO queue and its application to 
systems with uncertain service time~// Perform. Evaluation, 2019. 
Vol.~134. Art. ID: 102000.

\bibitem{xx3}  %2
\Au{Мейханаджян Л.\,А., Милованова~Т.\,А., Печинкин~А.\,В., 
Разумчик~Р.\,В.} 
Стационарные вероятности состояний в~системе обслуживания с~инверсионным 
порядком
обслуживания и~обобщенным вероятностным приоритетом~// Информатика и~её 
применения, 2014. Т.~8. Вып.~3. С.~28--38.



\bibitem{xx5}  %3
\Au{Meykhanadzhyan L., Razumchik~R.} New scheduling policy
for estimation of stationary performance characteristics in
single server queues with inaccurate job size information~//
30th European Conference on Modelling and Simulation
Proceedings.~--- Dudweiler, Germany: Digitaldruck Pirrot
GmbH, 2016. P.~710--716.

\bibitem{xx6}  %4
\Au{Milovanova T.\,A., Meykhanadzhyan~L.\,A., Razumchik~R.\,V.}
Bounding moments of Sojourn time in $M/G/1$ FCFS queue with inaccurate job 
size information and additive error: Some observations from numerical 
experiments~// CEUR Workshop Proceedings, 2018. Vol.~2236. P.~24--30.

\bibitem{xx4}  %5
\Au{Мейханаджян Л.\,А., Разумчик~Р.\,В.} Система массового обслуживания 
Geo$/G/1/\infty$ с~инверсионным порядком обслуживания и~ресамплингом 
в~дискретном времени~// Информатика и~её применения, 2019. Т.~13. Вып.~4. 
С.~60--67.

\bibitem{xx8} %6
\Au{Hokstad P.} On the steady-state solution of the $M/G/2$ 
queue~// Adv. Appl. Probab., 1979. Vol.~11. Iss.~1. P.~240--255.

\bibitem{xx9} %7
\Au{Wiens D.\,P.} 
On the busy period distribution of the $M/G/2$ queueing system~//
J.~Appl. Probab., 1989. Vol.~26. Iss.~4. P.~858--865.

\bibitem{xx7} %8
\Au{Knessl C., Matkowsky B.\,J., Schuss~Z., Tier~C.} An integral equation 
approach
to the $M/G/2$ queue~// Oper. Res., 1990. Vol.~38. Iss.~3. P.~506--518.

\bibitem{xx1} 
\Au{Печинкин А.\,В.} Об одной инвариантной системе массового 
обслуживания~// Math.
Operationsforsch. Statist. Ser. Optimization, 1983. Vol.~14. Iss.~3. 
P.~433--444.

\end{thebibliography}

 }
 }

\end{multicols}

\vspace*{-3pt}

\hfill{\small\textit{Поступила в~редакцию 14.04.20}}

%\vspace*{8pt}

%\pagebreak

\newpage

\vspace*{-28pt}

%\hrule

%\vspace*{2pt}

%\hrule

%\vspace*{-2pt}

\def\tit{STATIONARY CHARACTERISTICS OF $M/G/2/\infty$ QUEUE 
WITH~IDENTICAL SERVERS, LIFO SERVICE, AND~RESAMPLING~POLICY\\[-7pt]}


\def\titkol{Stationary characteristics of $M/G/2/\infty$ queue with 
identical servers, LIFO service, and 
resampling policy}

\def\aut{L.\,A.\ Meykhanadzhyan$^1$ and 
R.\,V.~Razumchik$^2$\\[-7pt]}

\def\autkol{L.\,A.\ Meykhanadzhyan and 
R.\,V.~Razumchik}

\titel{\tit}{\aut}{\autkol}{\titkol}

\vspace*{-18pt}


\noindent
$^1$Financial University under the Government of the Russian Federation,
49~Leningradsky Prospekt, Moscow\linebreak
$\hphantom{^1}$125993, Russian Federation

\noindent
$^2$Institute of Informatics Problems, Federal Research Center ``Computer 
Science and Control'' of the Russian\linebreak
$\hphantom{^1}$Academy of Sciences, 44-2~Vavilov Str., Moscow 119333, Russian Federation

\def\leftfootline{\small{\textbf{\thepage}
\hfill INFORMATIKA I EE PRIMENENIYA~--- INFORMATICS AND
APPLICATIONS\ \ \ 2020\ \ \ volume~14\ \ \ issue\ 2}
}%
 \def\rightfootline{\small{INFORMATIKA I EE PRIMENENIYA~---
INFORMATICS AND APPLICATIONS\ \ \ 2020\ \ \ volume~14\ \ \ issue\ 2
\hfill \textbf{\thepage}}}

\vspace*{2pt} 



\Abste{Consideration is given to the $M/G/2/\infty$ queue with identical 
servers,
LIFO (last in, first out) service discipline and one special case of the generalized 
probabilistic priority 
policy called resampling.
The latter implies that a~customer arriving to the nonidle system
assigns independently new remaining service time to each customer 
currently in service.
The new customer itself either enters a~free server, if there is any,
or occupies a~place in the queue. 
Remaining service times are assumed to be independent 
identically distributed random variables with 
the known general absolute continuous distribution. 
Under the assumption that the stationary regime exists,
the main performance characteristics of the system,
including the joint stationary distribution of the 
total number of customers in the system 
and the remaining service times of customers in service,
are derived.} 

\KWE{multiserver system; inverse service order; probabilistic priority; 
resampling}


\DOI{10.14357/19922264200209}
 

\vspace*{-24pt}

\Ack

\vspace*{-3pt}

\noindent
The reported study was funded by the Russian Foundation for
Basic Research according to the research project 18-37-00283.


%\vspace*{6pt}

 \begin{multicols}{2}

\renewcommand{\bibname}{\protect\rmfamily References}
%\renewcommand{\bibname}{\large\protect\rm References}

{\small\frenchspacing
 {%\baselineskip=10.8pt
 \addcontentsline{toc}{section}{References}
 \begin{thebibliography}{9}

\bibitem{xx-12} 
\Aue{Horv$\acute{\mbox{a}}$th I., R.~Razumchik, and M.~Telek.} 2019. The
resampling $M/G/1$ non-preemptive LIFO queue and its
application to systems with uncertain service time. \textit{Perform. 
Evaluation} 134:102000. 


\bibitem{xx3-1}  %2
\Aue{Meykhanadzhyan, L.\,A., T.\,A.~Milovanova, A.\,V.~Pechinkin, and 
R.\,V.~Razumchik}. 2014. Statsionarnye veroyatnosti so\-sto\-yaniy v~sisteme 
obsluzhivaniya s~inversionnym
poryadkom obsluzhivaniya i~obobshchennym veroyatnostnym prioritetom 
[Stationary distribution in a~queueing
system with inverse service order and generalized probabilistic 
priority]. \textit{Informatika i~ee Primeneniya~--- Inform. Appl.} 
8(3):28--38.



\bibitem{xx5-1} %3
\Aue{Meykhanadzhyan, L., and R.~Razumchik.} 2016. New
scheduling policy for estimation of stationary performance
characteristics in single server queues with inaccurate job
size information. \textit{30th European Conference on Modelling
and Simulation Proceedings}. Dudweiler, Germany: Digitaldruck Pirrot 
GmbH. 710--716.

\bibitem{xx6-1}  %4
\Aue{Milovanova, T.\,A., L.\,A.~Meykhanadzhyan, and R.\,V.~Razumchik.} 
2018.
Bounding moments of sojourn time in $M/G/1$ FCFS queue with inaccurate job 
size information and additive error: Some observations from numerical 
experiments. \textit{CEUR Workshop Proceedings} 2236:24--30.

\bibitem{xx4-1}  %5
\Aue{Meykhanadzhyan, L.\,A., and R.\,V.~Razumchik.} 2019. Sistema 
massovogo obsluzhivaniya Geo$/G/1/\infty$ s~inversionnym poryadkom 
obsluzhivaniya i~resamplingom v~diskretnom vremeni [Discrete-time 
Geo$/G/1/\infty$ LIFO queue with resampling policy].
\textit{Informatika i~ee Primeneniya~--- Inform. Appl.} 13(4):60--67.

\bibitem{xx8-1} %6
\Aue{Hokstad, P.} 1979. On the steady-state solution of the $M/G/2$ queue. 
\textit{Adv. Appl. Probab.} 11(1):240--255.

\bibitem{xx9-1} %7
\Aue{Wiens, D.\,P.} 1989. On the busy period distribution of the $M/G/2$ 
queueing system.
\textit{J.~Appl. Probab.} 26(4):858--865.

\bibitem{xx7-1} %8
\Aue{Knessl, C., B.\,J. Matkowsky, Z.~Schuss, and C.~Tier.} 1990. An 
integral equation approach
to the $M/G/2$ queue. \textit{Oper. Res.} 38(3):506--518.

\bibitem{xx1-1} %9
\Aue{Pechinkin, A.\,V.} 1983. Ob odnoy invariantnoy 
sisteme
massovogo obsluzhivaniya [On an invariant queuing system]. \textit{Math. 
Operationsforsch. Statist. Ser. Optimization}
14(3):433--444. doi: 10.1080/02331938308842876.

\end{thebibliography}

 }
 }

\end{multicols}

\vspace*{-12pt}

\hfill{\small\textit{Received April 14, 2020}}

%\pagebreak

\vspace*{-24pt}

\Contr

\vspace*{-4pt}

\noindent
\textbf{Meykhanadzhyan Lusine A.} (b.\ 1990)~---
Candidate of Science (PhD) in physics and
mathematics, associate professor,
Department of Data Analysis, Decision-Making and Financial Technology,
Financial University under the Government of the Russian Federation,
49 Leningradsky Prospekt, Moscow 125993, Russian Federation;
\mbox{lamejkhanadzhyan@fa.ru}

%\vspace*{3pt}

\noindent
\textbf{Razumchik Rostislav V.} (b.\ 1984)~---
Candidate of Science (PhD) in physics and mathematics, leading scientist,
Institute of Informatics Problems, Federal Research Center ``Computer 
Science and Control'' of the Russian Academy of Sciences, 44-2~Vavilov 
Str., Moscow 119333, Russian Federation; rrazumchik@ipiran.ru


\label{end\stat}

\renewcommand{\bibname}{\protect\rm Литература} 