\def\stat{agalarov}

\def\tit{ОПТИМИЗАЦИЯ ЕМКОСТИ ОСНОВНОГО НАКОПИТЕЛЯ 
В~СИСТЕМЕ МАССОВОГО ОБСЛУЖИВАНИЯ ТИПА $G/M/1/K$ С~ДОПОЛНИТЕЛЬНЫМ 
НАКОПИТЕЛЕМ$^*$}

\def\titkol{Оптимизация емкости основного накопителя в~СМО типа 
$G/M/1/K$ с~дополнительным накопителем}

\def\aut{Я.\,М.~Агаларов$^1$}

\def\autkol{Я.\,М.~Агаларов}

\titel{\tit}{\aut}{\autkol}{\titkol}

\index{Агаларов Я.\,М.}
\index{Agalarov Ya.\,M.}
 

{\renewcommand{\thefootnote}{\fnsymbol{footnote}} \footnotetext[1]
{Работа выполнена при частичной финансовой поддержке РФФИ (проекты 18-07-00692, 19-07-00739 
и~20-07-00804).}}


\renewcommand{\thefootnote}{\arabic{footnote}}
\footnotetext[1]{Институт проблем информатики Федерального исследовательского центра 
<<Информатика и~управление>> Российской академии наук, \mbox{agglar@yandex.ru}}
  
   \vspace*{-12pt}

\Abst{Для системы массового обслуживания (СМО) типа $G/M/1/K$ c дополнительным 
накопителем сформулирована задача оптимизации емкости основного накопителя при 
стоимостной целевой функции, учитывающей затраты системы, связанные с~потерей 
заявок, хранением заявок в~накопителях, техническим обслуживанием накопителей 
и~прибора, простоем прибора. Заявка, поступившая в~систему, принимается в~основной 
накопитель, если есть свободное место, иначе согласно заданному вероятностному 
распределению либо покидает систему (теряется), либо поступает в~дополнительный 
накопитель, если там есть свободное место. Если оба накопителя переполнены, заявка 
покидает систему (теряется). Если в~основном накопителе освобождается место, то одна из 
заявок из очереди в~дополнительном накопителе (если он не пуст) сразу поступает 
в~очередь в~основной накопитель. Доказана унимодальность целевой функции 
и~предложена процедура гарантированного поиска оптимальной емкости основного 
накопителя.}
\KW{система массового обслуживания; оптимизация; накопитель; емкость накопителя}

\DOI{10.14357/19922264200210} 
 
%\vspace*{9pt}


\vskip 10pt plus 9pt minus 6pt

\thispagestyle{headings}

\begin{multicols}{2}

\label{st\stat}

  
  \section{Введение}
  
  В качестве математических моделей многих производственных, 
информационных и~транспортных систем, в~которых генерируются случайные 
потоки заявок на обслуживание, как правило, используют СМО 
с~ограниченными накопителями~[1]. Одним из параметров, от значения 
которого существенным образом может зависеть экономическая эффективность 
такой системы, служит объем накопителя. При этом кроме стоимости самого 
накопителя объем накопителя влияет на эффективность системы\linebreak через 
зависящие от него другие характеристики, такие как потери заявок, задержки 
заявок в~на\-ко\-пи\-теле, время простоя обслуживающего устройства, 
загруженность обслуживающего устройства, интенсивность повторных 
поступлений заявок и~т.\,д.

  \begin{figure*}[b] %fig1
\vspace*{3pt}
 \begin{center}
 \mbox{%
 \epsfxsize=117.503mm 
 \epsfbox{aga-1.eps}
 }
 \end{center}
   \vspace*{-9pt}
  \Caption{Схема системы: $I$~--- источник потока первичных заявок; 
  $\pi_1$~--- вероятность 
переполнения основного накопителя; $\pi_2$~--- вероятность переполнения 
дополнительного накопителя; $\alpha$~--- вероятность повтора обслуживания;
 $\beta$~--- 
вероятность поступления в~дополнительный накопитель заявки, не допущенной 
в~основной}
  \end{figure*}
  
   
  Вопросам оптимизации СМО с~ограниченными накопителями посвящено 
большое число работ, в~которых можно найти множество различных 
постановок оптимизационных задач. Приведем краткий обзор некоторых 
работ~[2--9], опубликованных в~последние годы и~наиболее близких 
к~рассматриваемой в~данной статье тематике оптимизации емкости 
накопителя при стоимостной целевой функции.

 В~\cite{2-aga} рассматривается 
$M/M/1$ с~управ\-ля\-емой нагрузкой, в~которой можно отклонять заявки по мере 
прибытия и~корректировать ско\-рость обслуживания в~каждый момент 
принятия решения. Цель состоит в~минимизации предельных (долгосрочных) 
средних затрат, которые включают затраты на эксплуатацию мест хранения, 
затраты за единицу времени обслуживания заявки при заданной ско\-рости 
обслуживания, штраф за отклонение задания. Разработан алгоритм вычисления 
оптимальной политики, который позволяет на каждом шаге вычислять 
достигнутую точность. 

В~\cite{3-aga} решается задача выбора оптимальной 
емкости буферной памяти узла сети массового обслуживания в~условиях 
интенсивного трафика, где в~качества модели узла используется СМО типа 
$M/M/1$ и~в~качестве целевой функции~--- функционал затрат, учитывающий 
три типа затрат: <<стоимость перегрузки>>, зависящую от числа клиентов 
в~системе, <<стоимость контроля>>, связанную с~динамически 
контролируемой скоростью обслуживания, и~<<стоимость штрафа>> за 
отклонение клиентов. Предложено приближенное решение, полученное для 
аппроксимирующей диффузионной модели как решение одномерной 
броуновской задачи управления (Brownian control problem, BCP). 


В~работе~\cite{4-aga} ставится задача оптимального управ\-ле\-ния очередью 
системы типа $GI/GI/1$ с~нетерпеливыми заявками (заявка после поступления 
в~очередь через экспоненциально распределенный интервал времени, если она 
за это время не обслужилась, покидает сис\-те\-му) в~условиях интенсивного 
трафика и~в~качестве оптимальной политики ограничения длины очереди 
в~смыс\-ле \mbox{минимизации} затрат предлагается использовать решение 
диффузионной аппроксимирующей задачи. Сис\-те\-ма при поступлении заявки 
в~очередь получает платеж, при отклонении заявки и~уходе необслуженной 
заявки из очереди платит соответствующие штрафы. 

В~работе~\cite{5-aga} 
рассмотрена проб\-ле\-ма управ\-ле\-ния доступом заявок в~систему 
$M/M/N/M$, когда существуют затраты, связанные с~отказом от клиентов, 
простоем прибора и~отказывающимися заявками. Сформулирован 
соответствующий марковский процесс принятия решений
 (MDP, Markov decision process), и~показано, 
что оптимальная политика~--- пороговая. Предлагается итерационный 
алгоритм, который при определенных условиях обеспечивает точное 
оптимальное решение, когда он останавливается; в~противном случае алгоритм 
останавливается, когда достигается предусмотренная заранее точность 
приближения. Решена также соответствующая ап\-прок\-си\-ми\-ру\-ющая задача 
управления диффузией (DCP, diffusion control problem) и~проведен численный анализ минимальных 
затрат системы для MDP и~DCP. 

В~работе~\cite{6-aga} рассматривается 
последовательность СМО типа $M/M/1$ с~воз\-мож\-ностью ограничения длины 
очереди и~корректировки ско\-рости обслуживания с~целью минимизации 
предельных средних затрат при интенсивном трафике. Функция затрат 
включает штраф за каждую отвергнутую заявку, затраты, связанные 
с~корректировкой скорости обслуживания, и~штраф за каждую 
отказывающуюся заявку. Для построения оптимального управления для СМО 
в~условиях интенсивного трафика (выбора оптимальной емкости буфера 
и~оптимальной скорости обслуживания) предлагается использовать 
оптимальную стратегию для предельной задачи управления диффузией (задача 
броуновского управ\-ле\-ния, или BCP~--- Brownian control problems). 

В~работе~\cite{7-aga} рассмотрена 
задача выбора оптимальной емкости накопителя системы $G/M/1/K$ с~учетом 
штрафов из-за задержки заявок в~очереди, потерь заявок и~простоя прибора, 
доказана унимодальность целевой функции. 

Задача выбора оптимальной 
емкости накопителя системы $M/M/s/K$ при целевой функции, учитывающей 
штрафы за пребывание заявки в~очереди дольше заданного времени (дедлайна) 
и~потери заявки исследована в~работе~\cite{8-aga}. 
Доказано, что функция 
дохода~--- унимодальна. 

В~\cite{9-aga} рас\-смот\-ре\-на СМО типа $G/G/1/K$, 
в~которой при\-бы\-ва\-ющие заявки, когда число заявок в~буфере превышает 
значение~$K$, отклоняются и~с~фиксированной вероятностью~$p$ покидают 
сис\-те\-му. В~противном случае они поступают на орбиту и~повторяют попытку 
позже через экспоненциально распределенный интервал времени с~па\-ра\-мет\-ром 
$\mu\hm> 0$ для всех заявок. Повторные заявки рассматриваются сис\-те\-мой как 
первичные. В~работе ставится задача оптимизации емкости накопителя 
независимо от ем\-кости орбиты для стоимостной целевой функции, 
учи\-ты\-ва\-ющей потери и~за\-держ\-ки заявок. Получен асимптотически 
оптимальный размер буфера для диффузионной модели СМО для случаев 
$\mu\hm\to 0$ и~$\mu\hm\to\infty$.
  
  Задача, которая рассматривается ниже, близка к~задаче из работы~\cite{9-aga} 
с~точки зрения модели и~структуры затрат для случая $\mu\hm\to\infty$.
  
  \section{Описание задачи}
  
  Рассматривается одноканальная СМО с~двумя накопителями (основной 
и~дополнительный) ограниченной емкости и~заданной функцией 
распределения интервалов времени между поступлениями заявок $A(t)$ 
и~экспоненциальной функцией распределения времени обслуживания заявок 
с~заданным параметром $\gamma\hm>0$ (рис.~1). Поступившая извне 
заявка принимается в~основ\-ной накопитель сис\-те\-мы (занимает любое 
свободное место в~этом накопителе), если в~момент ее поступления чис\-ло 
занятых в~нем мест меньше его емкости $K\hm>0$. Если заявка допущена 
в~накопитель, она занимает любое свободное место в~нем и,~если прибор 
свободен, немедленно начинает обслуживаться прибором, в~противном случае 
становится к~нему в~очередь. На время обслуживания заявки место 
в~накопителе сохраняется за ней. Заявка после завершения обслуживания 
с~фиксированной вероятностью $(1\hm-\alpha)\hm>0$ покидает систему 
(успешно обслуживается), освободив одновременно прибор и~накопитель, а~с 
вероятностью $\alpha\hm\geq0$ освобождает прибор, но продолжает занимать 
место в~накопителе и~повторно становится в~очередь к~прибору. 
Освободившийся прибор сразу приступает к~обслуживанию очередной заявки, 
если очередь не пус\-та, иначе простаивает до момента поступления 
в~основной накопитель новой заявки. Заявка, заставшая основной накопитель 
сис\-те\-мы пол\-ностью занятым, с~вероятностью $(1\hm-\beta)\hm\geq0$ теряется, 
а~с~вероятностью $\beta\hm\geq0$ поступает на дополнительный накопитель 
ограниченной емкости $M\hm\geq0$, если там есть свободное место, иначе 
теряется. В~момент освобождения места в~основном накопителе одна из 
заявок перемещается в~него из дополнительного накопителя, освободив там 
место. Время перемещения в~рассматриваемой модели считается равным нулю. 
  
  
  Система получает доход, включающий в~себя следующие составляющие:
  \begin{description}
  \item[\,]
  $d_0$~--- плата, получаемая системой, если поступившая заявка успешно 
обслужена системой ($0\hm\leq d_0\hm<\infty$); 
  \item[\,]
  $d_1$~--- величина штрафа, который платит система, если поступившая 
заявка потеряна ($0\hm\leq d_1\hm<\infty$);
  \item[\,]
  $d_{21}$~--- затраты системы в~единицу времени на хранение заявки 
в~основном накопителе ($0\hm\leq d_{21}\hm<\infty$);
  \item[\,]
  $d_{22}$~--- затраты системы в~единицу времени на хранение 
в~дополнительном накопителе ($0\hm\leq d_{22}\hm<\infty$);
  \item[\,]
  $d_3$~--- величина штрафа за единицу времени простоя прибора ($0\hm\leq 
d_3\hm<\infty$);
  \item[\,]
  $d_4$~--- затраты системы в~единицу времени на обслуживание заявки 
прибором ($0\hm\leq d_4\hm<\infty$);
  \item[\,]
  $d_5$ и~$d_6$~--- затраты системы в~единицу времени на техническое 
обслуживание одного мес\-та в~основном и~дополнительном накопителе 
соответственно ($0\hm\leq d_5\hm<\infty$, $0\hm\leq d_6\hm<\infty$).
  \end{description}
  
  Обозначим через $D(K)$ предельный (стационарный) доход, получаемый 
системой в~единицу времени, и~рассмотрим его как функцию переменной~$K$ 
при фиксированных значениях остальных па\-ра\-мет\-ров системы. 
  
  Ставится задача оптимизации объема основного накопителя при 
фиксированных остальных параметрах системы в~смысле максимизации 
функции дохода $D(K)$ по $K\hm>0$. В~следующих разделах приводятся 
результаты исследования данной задачи.
  
  \section{Решение задачи}
  
  Получим явное аналитическое выражение для функции~$D(K)$. Введем 
обозначения:
  \begin{description}
  \item[\,]
  $\overline{v}=\int\nolimits_0^\infty v\,dA(v)\hm>0$~--- среднее время между 
моментами поступления заявок извне;
  \item[\,]
  $W_1(K)$~--- среднее время пребывания заявки в~основном накопителе;
  \item[\,]
  $W_2(K)$~--- среднее время пребывания заявки в~дополнительном 
накопителе;
  \item[\,]  
$\overline{T}(K)$~--- среднее время простоя прибора.
  \end{description}
  
  Как следует из описания модели, среднее чис\-ло попыток обслуживания для 
каждой заявки равно  
$(1\hm-\alpha)^{-1}$ и~функция распределения суммарного времени 
обслуживания одной заявки имеет вид:
$$
B(t)= 1-\exp(-\mu t),\ 
\mu=\gamma(1-\alpha),\ t\geq0\,.
$$
 Как известно, процесс 
обслуживания заявок в~данной системе описывается цепью Mаркова~[1], где 
переходы цепи определяются моментами поступления заявок, а~состояние 
цепи~--- суммарное число заявок, находящихся в~двух накопителях в~момент 
поступления. Для вероятностей переходов~$p_{ij}$ вложенной цепи Маркова 
справедливы формулы:
  \begin{multline}
  p_{ij}(K)={}\\
  {}=
  \begin{cases}
  r_{i+1-j}\,, & 0\leq i\leq K-1\,, \ 1\leq j\leq i+1\,;\\
  \beta r_{i+1-j}+(1-\beta)r_{i-j}\,, \hspace*{-35mm}& \\
  &\hspace*{-10mm}K\leq i\leq K+M-1\,,\ 1\leq j\leq i+1\,;\\
  r_{K+M-j}, & i=K+M\,,\ 1\leq j\leq K+M\,;\\
  \displaystyle 1-\sum\limits^i_{m=0}\, r_m\,, & i\leq K-1\,,\ j=0\,;\\
  \displaystyle 1-\beta\sum\limits^i_{m=0} \, 
  r_m-(1-\beta)\sum\limits_{m=0}^{i-1}\, r_m\,, \hspace*{-35mm}& \\
&K\leq i \leq K+M-1\,,\ j=0\,;\\
  \displaystyle 1-\sum\limits_{m=0}^{K+M-1} \,r_m\,, \hspace*{-5mm}&\hspace*{5mm} i=K+M\,,\ j=0\,;\\
  0 & \mbox{в\ остальных случаях},
  \end{cases}\hspace*{-5mm}
  \label{e1-aga}
  \end{multline}
  где
  $$
  r_m=\int\limits_0^\infty \fr{(\mu t)^m}{m!}\,e^{-\mu t}\, dB(t)\ \mbox{при } 
m\geq 0\,.
  $$
  
  Для данной цепи система уравнений равновесия в~матричном виде задается 
равенством:
  \begin{equation}
  \overline{\pi}(K)=\overline{\pi}^{\mathrm{T}}(K) P(K)\,,
  \label{e2-aga}
  \end{equation}
где $\overline{\pi}(K)=(\pi_0(K), \ldots , \pi_{K+M}(K))$~--- столбец 
стационарных вероятностей состояний сис\-те\-мы; $\overline{\pi}^{\mathrm{T}}$~--- строка 
стационарных вероятностей состояний сис\-те\-мы; $P(K)\hm=  
(p_{i,j}(K))$~--- треугольная мат\-ри\-ца переходных вероятностей 
с~элементами~$p_{i,j}(K)$ вида~(1). Из~(2) следует, что стационарные 
вероятности состояний вычисляются по формуле:
$$
\pi_j(K)=\fr{R_j^H}{\sum\nolimits_{i=0}^{h_2} R_i^H}\,,\enskip 0\leq j\leq 
K+M\,,
$$
  где
  \begin{multline*}
  R_j(K)={}\\
  {}=\!\begin{cases}
  \displaystyle \fr{R_{j+1}(K)(1-r_1) - R_{K+M}(K) r_{K+M-1-j}}{r_0} -{}\hspace*{-60mm}\\
%\hspace*{40pt}
{}-\fr{\sum\nolimits_{i=j+2}^{K-1}  
R_i(K) r_{i-j}}{r_0}-{}\hspace*{-60mm}\\
  \displaystyle -\fr{\sum\nolimits_{i=K}^{K+M-1} R_i(K) [\beta r_{i-j} +(1-\beta) 
r_{i-j-1}]}{r_0}\,,\hspace*{-60mm}\\ 
&0\leq j\leq K-2\,;\\[9pt]
  \displaystyle \fr{R_K(K)[1\!-\!\beta r_1\!-\!(1-\beta)r_0] \!-\!R_{K+M}(K) r_M}{r_0}-
{}\hspace*{-60mm}\\
\!\!\!  \displaystyle
 {}-\fr{\sum\nolimits_{i=K+1}^{K+M-1} \!R_i(K) [\beta r_{i+1-K}\! + \!
(1-\beta) r_{i-K}]}{r_0}\,,\hspace*{-60mm}\\
&j=K-1\,;\\
  1\,, & j=K+M\,,\ \beta>0\,;\\
  \displaystyle \fr{1-r_0}{\beta r_0}\,, & j=K+M-1\,,\ \beta>0\,;\\[9pt]
  \fr{R_{j+1}(K) [1-\beta r_1 -(1-\beta)r_0] }{\beta r_0}-{}\hspace*{-60mm}&\\[9pt]
  {}-\fr{ R_{K+M}(K) r_{K+M-1-j}}{\beta r_0}-{}\hspace*{-60mm}&\\[9pt]
{} \displaystyle - \fr{\sum\nolimits_{i=j+2}^{K+M-1} R_i(K) [\beta r_{i-j} + 
  (1-\beta) r_{i-j-1}]}{\beta r_0}\,, \hspace*{-136.2pt}\\
  &  K\leq j\leq K+M-2\,,\ \beta>0\,;\\
  0\,, & K+1\leq j\leq K+M\,,\ \beta=0\,;\\
  1\,,& j=K\,,\ \beta=0\,.
  \end{cases}
  \end{multline*}
  
  Заметим, что 
  \begin{multline}
  \pi_{j+1}(K+1)=\left[ 1-\pi_0(K+1)\right] \pi_j(K)\,,\\
  \enskip 0\leq j\leq K+M\,.
  \label{e4-aga}
  \end{multline}
  
  Обозначим через $q_j(K)$ средний доход, получаемый системой 
в~состоянии~$j$. Тогда
  \begin{equation}
  D(K)=\fr{1}{\overline{v}} \sum\limits_{j=0}^{K+M} \pi_j(K) q_j(K)\,.
  \label{e5-aga}
  \end{equation}
  
  Справедливо следующее утверждение.
  
  \smallskip
  
  
  \noindent
  \textbf{Утверждение~1.}\ \textit{$D(K)$~--- унимодальная функция по 
переменной $K\hm>0$. При этом}
\begin{enumerate}[(1)]
\item \textit{если} $\mathop{\mathrm{inf}}\nolimits_{K>0} 
G(K)\hm< \mathop{\mathrm{sup}}\nolimits_{K>0} D(K)$ и~$d_{21}\hm>0$, \textit{то cуществует 
$K^*\hm<\infty$, иначе, если $D(1)\hm< G(1)$, то $K^*\hm=
  \infty$}; 
  \item \textit{если $D(1)\hm\geq G(1)$, то $K^*\hm=1$.}
  \end{enumerate}
  
  \smallskip
  
  \noindent
  Д\,о\,к\,а\,з\,а\,т\,е\,л\,ь\,с\,т\,в\,о\,.\ \ Верны равенства:
  \begin{multline}
q_j(K)=\sum\limits^\infty_{m=0} \int\limits_0^\infty r_m(v) q_j(K, m, v)\,dv\,,\\[-6pt]
  j=0,\ldots , K+M\,,
  \label{e6-aga}
  \end{multline}
где $q_j(K, m, v)$~--- средний доход, получаемый системой в~состоянии~$j$ 
при условии, что время пребывания в~этом состоянии равно~$v$ и~за время   
завершится полное обслуживание ровно~$m$ заявок (в~дальнейшем 
условие~$B_{m,v}$). 
  
  Ниже, чтобы получить явные выражения для $q_j(K,m,v)$, найдем 
соответствующие выражения для следующих величин: 
\begin{itemize}
\item $W_{i,1}(m,v)$~--- 
среднее суммарного времени занятия мест в~основном накопителе; 
\item $W_{i,2}(m,v)$~--- среднее суммарного времени занятия мест 
в~дополнительном накопителе; 
\item $W_{i,\mathrm{пр}}(m,v)$~--- среднее время 
простоя прибора при условии~$B_{m,v}$ и~при условии, что число заявок 
в~сис\-те\-ме в~момент поступления с~учетом поступившей равно~$i$.
\end{itemize}

Как известно~\cite{7-aga}, в~СМО $G/M/1$ среднее время ожидания $l$-й по 
очереди заявки при условии~$B_{m,v}$ равно $lv/(m+1)$. После несложных 
преобразований для заданного~~$i$ получим равенства:
  \begin{equation}
\!\,  W_{i,1}(m,v)=\!\begin{cases}
  Kv\,, & 0\leq m \leq i-K\,,\ i\geq K\,;\\[9pt]
  \displaystyle iv-\fr{mv}{2}- \fr{(i-K)(i-K+1)v}{2(m+1)}\,,\hspace*{-120pt} &\\
  & i\geq m>i-K\,,\ i\geq K\,;\\[9pt]
  \displaystyle \fr{i(i+1)v}{2(m+1)}-\fr{(i-K)(i-K+1)v}{2(m+1)}\,,\hspace*{-120pt} &\\
  &\hspace*{15pt} i\geq K\,,\ m>i\,;\\
  \displaystyle iv-\fr{mv}{2}\,,\hspace*{-15pt} &\hspace*{15pt} i<K\,,\ 0\leq m<i\,;\\[9pt]
  \displaystyle \fr{i(i+1)v}{2(m+1)}\,,\hspace*{-15pt} & \hspace*{15pt}i<K\,,\ m\geq i\,,
  \end{cases}
\hspace*{-15pt} \label{e7a-aga}
  \end{equation}
  \begin{equation}
  W_{i,2}(m,v)=\begin{cases}
  \displaystyle (i-K)v-\fr{mv}{2}\,, \\
  &\hspace*{-30mm} 0\leq m\leq i-K\,,\ i\geq K\,;\\
  \displaystyle \fr{(i-K)(i-K+1)v}{2(m+1)}\,,\\
   &\hspace*{-20mm}m>i-K\,,\ i\geq K\,;\\
  0\,, & i\leq K\,,\end{cases}
  \label{e7-aga}
  \end{equation}
  \begin{equation}
  W_{i,\mathrm{пр}}(m,v)=\begin{cases}
  \displaystyle \fr{(m-i+1)v}{m+1}\,, &m\geq i\,;\\
  0\,, & 0\leq m <i\,.
  \end{cases}
  \label{e7b-aga}
  \end{equation}
Отсюда находим
\begin{multline}
q_j(K,m,v)={}\\
{}=\begin{cases}
d_0-\tilde{q}_{j+1}(K,m,v)\,, &j\leq K-1\,;\\
\beta[d_0-\tilde{q}_{j+1}(K,m,v)]-{}&\\
{}-(1-\beta) [d_1+\tilde{q}_j(K,m,v)]\,,\hspace*{-18mm} &\\
& \hspace*{-15mm}K\leq j\leq K+M-1\,;\\
-\tilde{q}_{K+M-1}(K,m,v)-d_1\,, & j=K+M\,,
\end{cases}
\label{e8-aga}
\end{multline}
где 
\begin{multline*}
\tilde{q}_i(K,m,v)=d_{21}W_{i,1}(m,v)+d_{22}W_{i,2}(m,v)+{}\\
{}+d_3W_{i,\mathrm{пр
}}(m,v)+d_4 v +d_5 K v +d_6 M v\,,\\  1\leq i\leq K+M\,.
\end{multline*}
  
  По формуле~(\ref{e6-aga}), воспользовавшись~(\ref{e1-aga}), 
(\ref{e7a-aga})--(\ref{e8-aga}), получим:
  \begin{equation}
 \!\, q_j(K)=\!\begin{cases}
  d_0+\tilde{q}_{j+1}(K)\,,& 0\leq j\leq K-1\,;\\
  \beta[d_0\!+\!\tilde{q}_{j+1}(K)]\!-\!(1\!-\!\beta)[d_1\!-\!\tilde{q}_j(K)]\,,\hspace*{-120pt}\\
   & \hspace*{-10pt}K\leq j\leq K+M-1\,;\\
  \tilde{q}_{K+M}(K)\,,& j=K+M\,,
  \end{cases}\!\!
  \hspace*{-20pt}\label{e9-aga}
  \end{equation}
  где 
  \begin{multline}
  \tilde{q}_i(K)=\fr{d_{21}}{\mu} \left[ \fr{1}{2}\sum\limits_{m=2}^{i+1} (m-1) 
mr_m -{}\right.\\
\left.{}- i \sum\limits_{m=1}^{i+1} m r_m-
\fr{1}{2}\,i(i+1)\sum\limits^\infty_{m=i+2} r_m\right]+{}\\
  {}+\theta_{(i-K)}\fr{d_{22}-d_{21}}{\mu} \left[ 
\fr{1}{2}\sum\limits_{m=2}^{i-K+1} (m-1) m r_m-{}\right.\\
{}-(i-K)\sum\limits_{m=1}^{i- K+1} m r_m-{}\\
  \left.{}- \fr{(i-K)(i-K+1)}{2}\sum\limits^\infty_{m=i-K+2} r_m\right] -{}\\
{}-
\fr{d_3}{\mu} \sum\limits^\infty_{m=i+1} (m-i) r_m-d_4\overline{v} -d_5 K 
\overline{v} -d_6 M \overline{v}\,;
  \label{e10-aga}
  \end{multline}
$\theta_{(i-K)}$~--- функция Хевисайда переменной $(i-K)$, $1\hm\leq i\hm\leq 
K\hm+M\hm-1$; $\tilde{q}_{K+M}(K)\hm= \tilde{q}_{K+M-1}(K) \hm- d_1$.
  
  Из~(\ref{e9-aga}) и~(\ref{e10-aga}) для величин $\Delta_j(K)\hm= 
q_j(K+1)\hm- q_{j-1}(K)$, $1\hm\leq j\hm\leq K\hm+M\hm-1$, 
и~$\tilde{\Delta}_i(K)\hm= \tilde{q}_i (K+1) \hm- \tilde{q}_{i-1}(K)$, $2\hm\leq 
i\hm\leq K\hm+M$, следуют равенства:
  $$
  \Delta_j(K)=\begin{cases}
  \tilde{\Delta}_{j+1}(K)\,, & 1\leq j\leq K\,;\\[6pt]
  \beta\tilde{\Delta}_{j+1}(K)+(1-\beta) \tilde{\Delta}_j(K-1)\,,\hspace*{-70pt}\\
& K\leq j\leq K+M-1\,;\\
  \tilde{\Delta}_{K+M}\,, &j=K+M\,.
  \end{cases}
  $$
  
  \noindent
  \begin{multline}
  \tilde{\Delta}_i(K) =\fr{d_3}{\mu} \sum\limits^\infty_{m=i+2} r_m -
\fr{d_{21}}{\mu} \sum\limits_{m=1}^{i+1} m r_m-{}\\
{}-\fr{d_{21}(i+1)}{\mu} 
\sum\limits^\infty_{m=i+2} r_m- d_5 \overline{v}\,,\\
 K>0\,,\ i=1,\ldots , 
K+M\,.
  \label{e11-aga} 
\end{multline}
  
  Из~(\ref{e2-aga}) и~(\ref{e4-aga}) следует:
  \begin{multline}
  r_0\pi_0(K+1)={}\\
  {}=\left[ 1-\pi_0 (K+1)\right] \Bigg \{ 
  \sum\limits_{j=0}^{K-1} \pi_j(K)\! \sum\limits^\infty_{m=j+2} \! r_m+{}%\right.
  \\
  {}+\sum\limits_{j=K}^{K+M-1} \pi_j(K) \left[ \beta \sum\limits^\infty_{m=j+2} 
\! r_m+(1-\beta)\! \sum\limits^\infty_{m=j+1}\! r_m\right]+{}\\
%\left.
  {}+\pi_{K+M}(K)\sum\limits^\infty_{m=K+M+1}\! r_m\Bigg\}\,.
  \label{e12-aga}
  \end{multline}
  
\begin{figure*}[b] %fig2
\vspace*{1pt}
\begin{minipage}[c]{80mm}
 \begin{center}
 \mbox{%
 \epsfxsize=75.253mm 
 \epsfbox{aga-2.eps}
 }
 \end{center}
   \vspace*{-9pt}
   \Caption{Зависимость дохода системы от емкости основного накопителя}
\end{minipage}
%  \end{figure*}
\hfill
%
%\begin{figure*} %fig3
\begin{minipage}[c]{80mm}
\vspace*{1pt}
 \begin{center}
 \mbox{%
 \epsfxsize=75.253mm 
 \epsfbox{aga-3.eps}
 }
 \end{center}
   \vspace*{-9pt}
\Caption{Зависимость максимального дохода от емкости дополнительного накопителя}
\end{minipage}
\end{figure*}
  
  Введем обозначения $\Psi(K)$ и~$H(K)$:
  \begin{multline*}
  \Psi(K) =\overline{v}-\fr{1}{\mu}\! \Bigg\{\! \sum\limits_{j=0}^{K-1} \pi_j(K) 
\!\sum\limits^\infty_{m=j+2}\!\! (m-j-1)r_m+{}\!%\right. 
\\
  {}+\sum\limits_{j=K}^{K+M-1} \pi_j(K) \left[ \beta \sum\limits^\infty_{m=j+2} 
(m-j-1)r_m +{}\right.\\
\left.{}+(1-\beta)\sum\limits^\infty_{m=j+1} (m-j)r_m\right]+{}\\
%\left.
{}+ \pi_{K+M}(K) \sum\limits^\infty_{m=K+M+1} (m-K-M) r_m\Bigg\}\,.
  \end{multline*}
  
  \vspace*{-12pt}
  
  \noindent
  \begin{multline*}
  H(K) =\sum\limits^{K-1}_{j=0} \! \pi_j(K) \sum\limits^\infty_{m=j+2} \! r_m + {}\\
  {}+
\sum\limits_{j=K}^{K+M-1} \! \pi_j (K) \left[ \beta \sum\limits^\infty_{m=j+2} \! 
r_m +(1-\beta)\sum\limits^\infty_{m=j+1}\! r_m\right]+{}\\
  {}+\pi_{K+M}(K)\sum\limits^\infty_{m=K+M+1}\! r_m,.
  \end{multline*} 



Рассмотрим функцию:
\begin{equation}
G(K)=\fr{d_3r_0}{\mu} - \fr{d_5r_0\overline{v}}{H(K)}-d_{21} 
r_0\fr{\Psi(K)}{H(K)}+q_0(K)\,.
\label{e13-aga}
\end{equation}
  
  Использовав~(\ref{e4-aga}), (\ref{e9-aga})--(\ref{e12-aga}), из~(\ref{e5-aga}) 
после несложных преобразований получим равенство:
  \begin{equation}
  D(K)-D(K+1)=\pi_0(K+1) [D(K)-G(K)]\,.
  \label{e14-aga}
  \end{equation} 

  Отметим, что $G(K)$~--- убывающая функция от переменной $K\hm>0$. 
Действительно, так как $\Psi(K)$~--- среднее время занятости прибора 
в~произвольно взятом состоянии системы (следует из определения~$\Psi(K)$ 
и~формулы для $W_{i,\mathrm{пр}}(m,v)$) (\ref{e7-aga}) и~$\pi_0(K)$\linebreak 
убывает по $K\hm>0$ (доказательство аналогично доказательству этого 
свойства~$\pi_0(K)$ в~работе~\cite{7-aga}), то~$\Psi(K)$ возрастает, 
а~$H(K)\hm= r_0\pi_0(K+1)/(1\hm-\pi_0(K+1))$ убывает по $K\hm>0$ 
и,~как следует из~(\ref{e13-aga}), $G(K)$ убывает по $K\hm>0$.

  Далее, так как функция~$D(K)$ удовлетворяет условиям теоремы 
из~\cite{10-aga} (т.\,е.\ для $D(K)$ имеет место равенство~(\ref{e14-aga}), 
$0\hm< \pi_0(K)\hm< 1$, $G(K)$~--- убывающая функция по $K\hm>0$), то 
непосредственно получаем доказательство унимодальности функции~$D(K)$. 
Доказательство условий~1 и~2 утверждения~1 следует из свойств 
функций~$D(K)$ и~$G(K)$, а~именно: если $D(K)\hm< G(K)$, то $D(K)\hm< 
D(K+1)$ возрастает по $K\hm>0$, иначе $D(K)\hm> D(K+1)$ (см.\ 
доказательство теоремы в~\cite{10-aga}).
  
  Как следствие утверждения~1 получим простое правило вычисления 
оптимальной емкости основного накопителя: если $D(1)\hm\geq D(2)$, то 
оптимальная емкость основного накопителя $K^*\hm=1$, иначе минимальное 
значение~$K^*$, удовлетворяющее условию $D(K^*)\hm\geq D(K^*+1)$, 
является оптимальным. 
  
  \section{Пример}
  
  Здесь приводятся результаты вычислительных экспериментов, проведенных 
для СМО $M/H_n/1$ с~функцией распределения времени обслуживания 
$A(t)\hm= \sum\nolimits^n_{i=1} f_i (1\hm- e^{1-\delta_i t})$ при 
$n\hm=2$; 
$f_1\hm=0{,}3$; $f_2\hm= 0{,}7$; $\delta_1\hm= 2$; $\delta_2\hm= 1$; 
$\gamma\hm= 1$; $d_0\hm= 20$; $d_1\hm= 10$; $d_{21}\hm= d_{22}\hm= 
0{,}5$; $d_3\hm= d_4\hm= 0{,}01$; $d_5\hm=0{,}1$; $d_6\hm=0$.
  
Кривые \textit{1}, \textit{2}, \textit{3} и~\textit{4}
на рис.~2 иллюстрируют соответствующие зависимости предельного дохода 
системы для четырех значений~$\beta$: 1,0; 0,7; 0,5 и~0,1. 
Соответствующие оптимальные значения для~$K$ и~$D(K)$ равны: 1, 4, 6, 7 
и~11,95, 12,82, 13,16, 13,2. 

  На рис.~3 показаны графики, иллюстриру\-ющие зависимость предельного 
максимального дохода системы от значения параметра~$M$, для трех значений 
параметра~$\beta$ (соответствующие кривые~\textit{1}, \textit{2} 
и~\textit{3}): 1,0, 0,5 и~0,1.
  
  \section{Заключение}
  
  В рассмотренной модели емкость дополнительного накопителя считается 
ограниченной, и~согласно утверждению~1 функция~$D(K)$ унимодальна для 
любого конечного значения $M\hm\geq 0$. Так как сумма~(\ref{e5-aga}) 
сходится при $M\hm\to \infty$ (ряд 
$(1/\overline{v})\sum\nolimits^K_{j=0}\pi_j(K) q_j(K)\hm+ 
(1/\overline{v})\sum\nolimits^\infty_{j=K+1} \pi_j(K) q_j(K)$ сходится) при 
условиях, что $d_6\hm=0$ и~что средняя длина очереди в~дополнительном 
накопителе ограничена, то есть основание считать, что функция~$D(K)$ 
унимодальна при выполнении этих условий и~в~случае $M\hm=\infty$. 
Вычислительные эксперименты показывают, что при $d_6\hm=0$ оптимальная 
емкость основного накопителя начиная с~некоторого~$M$ остается 
постоянной (соответствующее максимальное значение дохода по~$K$ слабо 
зависит от~$M$, когда $M\hm\gg 0$) (см.\ рис.~3) и~правило вы\-чис\-ле\-ния 
оптимального значения~$K$, предложенное в~конце разд.~3, может найти 
применение и~в~случае $M\hm=\infty$.
  
  Результаты данной работы могут также найти применение при исследовании 
более сложных моделей систем с~повторными заявками (например, в~качестве 
предельной модели для систем с~ненулевым временем~$\tau$ перехода из 
дополнительного накопителя в~основной при $\tau\hm\to 0$). 
  
{\small\frenchspacing
 {%\baselineskip=10.8pt
 \addcontentsline{toc}{section}{References}
 \begin{thebibliography}{99}
  
\bibitem{1-aga}
\Au{Бочаров П.\,П., Печинкин~А.\,В.} Теория массового обслуживания.~--- М.: РУДН, 1995. 
529~с.
\bibitem{2-aga}
\Au{Ata B., Shneorson S.} Dynamic control of an $M/M/1$ service system with adjustable arrival and 
service rates~// Manage. Sci., 2006. Vol.~52. Iss.~11. P.~1778--1791.
\bibitem{3-aga}
\Au{Ghosh A.\,P., Weerasinghe~A.\,P.} Optimal buffer size for a~stochastic processing network 
in heavy traffic~// Queueing Syst., 2007. Vol.~55. Iss.~3. P.~147--159.
\bibitem{4-aga}
\Au{Ward A., Kumar S.} Asymptotically optimal admission control of a~queue with impatient 
customers~// Math. Oper. Res., 2008. Vol.~33. Iss.~1. P.~167--202.
\bibitem{5-aga}
\Au{\mbox{Ko{\!\!\ptb{\c{c}}}a{\!\!\ptb{\v{g}}}a} Y.\,L., Ward A.\,R.} Admission control for a~multi-server 
queue with abandonment~// Queueing Syst., 2010. Vol.~65. Iss.~3. P.~275--323.
\bibitem{6-aga}
\Au{Ghosh A.\,P., Weerasinghe~A.\,P.} Optimal buffer size and dynamic rate control for 
a~queueing system with impatient customers in heavy traffic~// Stoch. Proc. 
Appl., 2010. Vol.~120. Iss.~11. P.~2103--2141.
\bibitem{7-aga}
\Au{Агаларов Я.\,М., Агаларов~М.\,Я., Шоргин~В.\,С.} Об оптимальном пороговом значении 
длины очереди в~одной задаче максимизации дохода системы массового обслуживания 
типа $M/G/1$~// Информатика и~её применения, 2016. Т.~10. Вып.~2. С.~70--79.

\bibitem{9-aga}
\Au{Atar R., Lev-Ari A.} Optimizing buffer size for the retrial queue: Two state space collapse 
results in heavy traffic~// Queueing Syst., 2018. Vol.~90. Iss.~3-4. P.~225--255. 

\bibitem{8-aga}
\Au{Агаларов Я.\,М., Коновалов~М.\,Г.} Доказательство унимодальности целевой функции 
в~задаче порогового управления нагрузкой на сервер~// Информатика и~её применения, 
2019. Т.~13. Вып.~2. С.~2--6.

\bibitem{10-aga}
\Au{Агаларов Я.\,М.} Признак унимодальности целочисленной функции одной 
переменной~// Обозрение прикладной и~промышленной математики, 2019. Т.~26. Вып.~1. 
С.~65--66.
\end{thebibliography}

 }
 }

\end{multicols}

%\vspace*{-12pt}

\hfill{\small\textit{Поступила в~редакцию 15.04.20}}

\vspace*{8pt}

%\pagebreak

%\newpage

%\vspace*{-28pt}

\hrule

\vspace*{2pt}

\hrule

%\vspace*{-2pt}

\def\tit{OPTIMIZATION OF THE CAPACITY OF~THE~MAIN STORAGE IN~$G/M/1/K$ QUEUEING 
SYSTEM\\ WITH~AN~ADDITIONAL STORAGE DEVICE}


\def\titkol{Optimization of the capacity of the main storage in $G/M/1/K$ queueing 
system with an additional storage device}

\def\aut{Ya.\,M.~Agalarov}

\def\autkol{Ya.\,M.~Agalarov}

\titel{\tit}{\aut}{\autkol}{\titkol}

\vspace*{-9pt}


    \noindent
Institute of Informatics Problems, Federal Research Center ``Computer Science and Control'' of the Russian 
Academy of Sciences, 44-2~Vavilov Str., Moscow 119333, Russian Federation 

\def\leftfootline{\small{\textbf{\thepage}
\hfill INFORMATIKA I EE PRIMENENIYA~--- INFORMATICS AND
APPLICATIONS\ \ \ 2020\ \ \ volume~14\ \ \ issue\ 2}
}%
 \def\rightfootline{\small{INFORMATIKA I EE PRIMENENIYA~---
INFORMATICS AND APPLICATIONS\ \ \ 2020\ \ \ volume~14\ \ \ issue\ 2
\hfill \textbf{\thepage}}}

\vspace*{3pt} 
   
   
\Abste{The problem of optimizing the capacity of the main storage 
device of a~queuing system of the type $G/M/1/K$ with an additional 
storage device with the cost objective function is formulated taking 
into account the costs of the system associated with the loss of 
requests, storage of requests, maintenance of storage devices, and 
device downtime. A~request arriving to the system is accepted into 
the main storage device if there is a~free space; otherwise, according 
to the given probability distribution, it goes to the additional device 
if there is a~free space. A~request leaves the system (is lost) if both 
storage devices are full. If space is freed up in the main storage 
device, then one of the requests from the queue in the additional 
device immediately enters the queue in the main device. The unimodality 
of the objective function is proved and the procedure for finding the 
optimal capacity of the main storage device is proposed.}

   \KWE{queueing system; optimization; storage device; storage capacity}
   
\DOI{10.14357/19922264200210} 

%\vspace*{-20pt}

   \Ack
   \noindent
   The reported study was partly funded by the Russian Foundation 
   for Basic Research according to the 
research projects Nos.\,18-07-00692,  
19-07-00739, and 20-07-00804.

%\vspace*{6pt}

 \begin{multicols}{2}

\renewcommand{\bibname}{\protect\rmfamily References}
%\renewcommand{\bibname}{\large\protect\rm References}

{\small\frenchspacing
 {%\baselineskip=10.8pt
 \addcontentsline{toc}{section}{References}
 \begin{thebibliography}{99}
   
   \bibitem{1-aga-1}
   \Aue{Bocharov, P.\,P., and A.\,V.~Pechinkin.} 1995. \textit{Teoriya massovogo obsluzhivaniya} 
[Queueing theory]. Moscow: RUDN. 529~p.
   \bibitem{2-aga-1}
   \Aue{Ata, B., and S. Shneorson.} 2006. Dynamic control of an $M/M/1$ service system with adjustable 
arrival and service rates. \textit{Manage. Sci.} 52(11):1778--1791.
   \bibitem{3-aga-1}
   \Aue{Ghosh, A.\,P., and A.\,P.~Weerasinghe.} 2007. Optimal buffer size for a~stochastic processing 
network in heavy traffic. \textit{Queueing Syst.} 55(3):1572--9443.
   \bibitem{4-aga-1}
   \Aue{Ward, A., and S. Kumar.} 2008. Asymptotically optimal admission control of a~queue with 
impatient customers. \textit{Math. Oper. Res.} 33(1):167--202.
   \bibitem{5-aga-1}
   \Aue{\mbox{Ko{\!\ptb{\c{c}}}a{\!\ptb{\v{g}}}a}, Y.\,L., and A.\,R.~Ward.} 2010. 
Admission control for 
a~multi-server queue with abandonment. \textit{Queueing Syst.} 65(3):275--323.
   \bibitem{6-aga-1}
   \Aue{Ghosh, A.\,P., and A.\,P.~Weerasinghe.} 2010. Optimal buffer 
size and dynamic rate control for a~queueing system with impatient customers 
in heavy traffic. \textit{Stoch. Proc. Appl.} 120(11):2103--2141. 
   \bibitem{7-aga-1}
   \Aue{Agalarov, Ya.\,M., M.\,Ya.~Agalarov, and V.\,S.~Shorgin.} 2016. 
Ob optimal'nom porogovom znachenii dliny ocheredi v~odnoy zadache maksimizatsii dokhoda 
sistemy massovogo obsluzhivaniya tipa $M/G/1$ [About the optimal 
threshold of queue length in particular problem of profit maximization in $M/G/1$ 
queueing system]. 
\textit{Informatika i~ee Primeneniya~--- Inform. Appl.} 10(2):70--79.
   
   \bibitem{9-aga-1} %8
\Aue{Atar, R., and A. Lev-Ari.} 2018. Optimizing buffer size for the 
retrial queue: Two state space collapse results in heavy traffic. 
\textit{Queueing Syst.} 90(3-4):225--255. 

\bibitem{8-aga-1} %9
   \Aue{Agalarov, Ya.\,M., and M.\,G.~Konovalov.} 2019. Dokazatel'stvo unimodal'nosti tselevoy funktsii 
v~zadache porogovogo upravleniya nagruzkoy na server [Proof of the unimodality of the objective function 
in $M/M/N$ queue with threshold-based congestion control]. \textit{Informatika i~ee primeneniya~--- Inform. 
Appl.} 13(2):2--6. 
   \bibitem{10-aga-1}
   \Aue{Agalarov, Ya.\,M.} 2019. Priznak unimodal'nosti tselochislennoy funktsii odnoy peremennoy 
[A~sign of unimodality of an integer function of one variable]. \textit{Obozrenie prikladnoy 
i~promyshlennoy matematiki } [Surveys Applied and Industrial Mathematics] 26(1):65--66.
\end{thebibliography}

 }
 }

\end{multicols}

\vspace*{-6pt}

\hfill{\small\textit{Received April 15, 2020}}

%\pagebreak

%\vspace*{-24pt}
   


   \Contrl
   
   \noindent
   \textbf{Agalarov Yaver M.} (b.\ 1952)~--- Candidate of Science (PhD) in technology, associate 
professor, leading scientist, Institute of Informatics Problems, Federal Research Center ``Computer Science 
and Control'' of the Russian Academy of Sciences, 44-2~Vavilov Str., Moscow 119333, Russian Federation; 
\mbox{agglar@yandex.ru}
   
\label{end\stat}

\renewcommand{\bibname}{\protect\rm Литература} 