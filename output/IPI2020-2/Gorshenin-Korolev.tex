
%\usepackage{algorithm}
%\floatname{algorithm}{Алгоритм}


\def\stat{gorshenin-kor}

\def\tit{АППРОКСИМАЦИЯ РАСПРЕДЕЛЕНИЙ РАЗМЕРОВ ЧАСТИЦ ЛУННОГО РЕГОЛИТА 
НА~ОСНОВЕ МЕТОДА СТАТИСТИЧЕСКОЙ СИМУЛЯЦИИ ВЫБОРОК$^*$}

\def\titkol{Аппроксимация распределений размеров частиц лунного реголита 
на~основе метода статистической симуляции} % выборок}

\def\aut{А.\,К.~Горшенин$^1$, В.\,Ю.~Королев$^2$}

\def\autkol{А.\,К.~Горшенин, В.\,Ю.~Королев}

\titel{\tit}{\aut}{\autkol}{\titkol}

\index{Горшенин А.\,К.}
\index{Королев В.\,Ю.}
\index{Gorshenin A.\,K.}
\index{Korolev V.\,Yu.}
 

{\renewcommand{\thefootnote}{\fnsymbol{footnote}} \footnotetext[1]
{Работа выполнена при поддержке Российского научного фонда 
(проект 18-11-00155). Для обработки данных использовался гибридный 
высокопроизводительный вычислительный комплекс ЦКП <<Информатика>> ФИЦ ИУ 
РАН: {\sf http://ckp.frccsc.ru}.}}


\renewcommand{\thefootnote}{\arabic{footnote}}
\footnotetext[1]{Федеральный исследовательский центр <<Информатика 
и~управление>> Российской академии наук; факультет вычислительной 
математики и~кибернетики Московского государственного университета имени 
М.\,В.~Ломоносова,
\mbox{agorshenin@frccsc.ru}}
\footnotetext[2]{Факультет вычислительной математики и~кибернетики
Московского государственного университета имени М.\,В.~Ломоносова;
Федеральный исследовательский центр <<Информатика и~управление>> 
Российской академии наук,
\mbox{vkorolev@cs.msu.ru}}

\vspace*{2pt}


\Abst{Рассмотрена задача моделирования распределения размеров пылевых 
частиц лунного реголита. В~качестве теоретической модели использованы 
конечные смеси логнормальных законов. Они позволяют учесть стохастический 
характер интенсивностей процессов  
дроб\-ле\-ния/спе\-ка\-ния при формировании ансамблей пылевых частиц 
в~результате различных воздействий (бомбардировка метеоритами, 
излучение). Разработан метод статистической аппроксимации неизвестных 
распределений на основе симуляции выборок, который демонстрирует высокое 
согласие с~реальными данными,
содержащимися в~каталоге {\sf NASA} (все 317~проб, 
доставленные миссиями <<Луна-24>> и~<<Аполлон-11, 12, 14--17>>).}

\KW{конечные логнормальные смеси; бутстреп; EM-алгоритм; статистические 
методы}

\DOI{10.14357/19922264200207} 
 
\vspace*{9pt}


\vskip 10pt plus 9pt minus 6pt

\thispagestyle{headings}

\begin{multicols}{2}

\label{st\stat}


\section{Введение}

\vspace*{-3pt}

В статье рассмотрена задача моделирования распределений размеров пылевых 
частиц лунного ре\-го\-лита, возникающих в~результате различных воздействий, 
например при бомбардировках \mbox{поверхности} Луны метеоритами. При таких 
воз\-действиях развиваются как взрывные процессы разлета час\-тиц с~их 
дроблением, так и~спекание\linebreak час\-тиц в~экзотермических плазмохимических 
реакциях синтеза~\cite{Popel2018}.

Изучение закономерностей, определяющих размеры частиц, образующих лунные 
и~другие планетарные реголиты, имеет очень большое значение при 
планировании автоматических и~пи\-ло\-ти\-ру\-емых миссий для изучения 
космических тел (Луны, астероидов, планет и~их спутников). В~условиях 
малой гравитации и~отсутствия плотной атмосферы пылевые структуры 
приобретают качества, не типичные для земных условий, представляя собой 
облака заряженных частиц, обладающих высокими абразивными свойствами. Они 
осаждаются на элементах аппаратуры (например, солнечных батареях) 
и~скафандрах, что быстро выводит их из строя. 

Таким образом, решение 
рассматриваемой задачи может оказаться весьма полезным при подготовке 
новых космических миссий для повышения уровня безопасности и~общей 
успешности подобных проектов.

Изначально в~исследованиях на эту тему рассматривались процессы чистого 
дробления частиц. В~работе~\cite{Razumovskii1940} указано много случаев, 
в~которых логарифмы размеров частиц (золотин в~золотоносных россыпях, 
частиц горных пород при их дроблении и~т.\,п.)\ имеют примерно нормальное 
распределение. На эту работу обратил внимание А.\,Н.~Колмогоров,\linebreak который 
предложил математическую модель процесса дробления частиц, аналитически 
объяс\-ня\-ющую возникновение логнормального распределения размеров частиц 
при дроблении, а~также \mbox{содержания} минералов в~отдельных 
пробах~\cite{Kolmogorov1941}. Данная модель базируется на изучении 
изменения во времени числа частиц, размер которых не превосходит заданный 
порог.

 Результат Колмогорова справедлив при довольно сильных 
предположениях, в~част\-ности для логнормальности распределения час\-тиц при 
дроблении необходимо, чтобы ско\-рость дроб\-ле\-ния была постоянной, т.\,е.\ 
не зависела от размеров самих\linebreak час\-тиц. Однако необходимо учитывать тот 
факт, что с~уменьшением размеров час\-ти\-цы интенсивность ее соударений 
с~другими частицами может изменяться, например уменьшаться в~силу того, 
что вероятность столкновения в~определенном смыс\-ле пропорциональна 
размерам час\-тицы. 
{\looseness=1

}

Известны эмпирические свидетельства того, что 
в~некоторых случаях и~логнормальная модель для распределения размера 
частиц при дроблении не адекватна. Так, в~книге~\cite{Bagnold1954} 
отмечено, что логарифм функции плот\-ности распределения для логарифма 
размера частицы в~естественных запасах песка больше похож на гиперболу, 
чем на параболу. Это означает, что распределение размера не логнормально, 
а~скорее имеет экспоненциально уменьшающиеся хвосты.

Последнее обстоятельство побудило некоторых исследователей обратить 
внимание на модели типа  
лог-не\-сим\-мет\-рич\-но\-го распределения Лапласа, так называемого 
двойного (двустороннего)  
Па\-ре\-то-лог\-нор\-маль\-но\-го распределения, а~также 
лог-гаус\-сов\-ско\-го/\!/об\-рат\-но\-го гауссовского распределения. В~част\-ности, 
в~работах~\cite{ReedJorgensen2003,Sorensen2006} рассмотрены модели 
формирования распределения размера дробящихся частиц, в~основе которых 
лежит предположение о~том, что час\-ти\-ца при перемещении из одного места 
в~другое может разделиться на несколько меньших частиц вследствие 
соударения или другого воздействия, что обусловливает случайность масс 
частиц после разделения. Не все частицы одновременно добираются до места 
назначения. Некоторые могут поменять направление и~застрять надолго, 
тогда как другие могут, не задев других, пройти весь путь намного 
быстрее. Очевидно, состояние частицы зависит от того, на каком расстоянии 
от конечного пункта она находится. Это учитывается в~упомянутых моделях 
Ри\-да--Йор\-ген\-се\-на и~Соренсена за счет случайности времени, 
в~течение которого наблюдается частица. По сути, 
в~работах~\cite{ReedJorgensen2003,Sorensen2006} в~модель Колмогорова 
введена рандомизация.

Данные модели демонстрируют хорошее согласие с~экспериментальными 
данными, отражающими размеры частиц в~природных залежах. Однако они не 
учитывают отмеченное Колмогоровым\linebreak обстоятельство: интенсивность процесса 
дробления остается постоянной. В~работе~\cite{Korolev2009} рас\-смот\-рена 
более общая модель, в~которой данная интенсивность может быть переменной 
и~даже\linebreak случайной. Показано, что случайность интенсивности процесса 
дробления может привести к~тому, что распределение размеров частиц будет 
иметь вид произвольной смеси логнормальных рас\-пре\-де\-ле\-ний.
{ %\looseness=1

}

В докладе~\cite{SK2019} была отмечена возможность распространения выводов 
работы~\cite{Korolev2009} на процессы  
дроб\-ле\-ния/спе\-ка\-ния. Эти результаты используются в~данной статье 
в~качестве теоретической базы для создания метода статистической 
обработки всех образцов лунного реголита, доставленных миссиями 
<<Аполлон-11, 12, 14--17>> и~<<Луна 24>>. Данные 317~проб взяты из 
каталога \verb"NASA"~\cite{Graf1993}, в~котором размеры частиц 
представлены в~так называемой $\phi$-шка\-ле~\cite{Donoghue2016}, а~для 
характеризации эмпирического распределения размеров частиц использованы 
величины долей частиц того или иного размера в~образце, полученные при 
просеивании. Для всех\linebreak представленных данных будет продемонстрировано 
высокое соответствие модели типа смеси конечных логнормальных законов 
с~представленными в~каталоге \verb"NASA" распределениями размеров 
ансамб\-лей пылевых частиц, возникающих в~лунном реголите.


\section{Конечные логнормальные смеси как~аппроксимации для~эмпирических 
распределений размеров частиц лунного реголита}

В соответствии с~результатами работ~\cite{Korolev2009, SK2019} в~качестве 
модели распределения логарифма размера частиц следует рассматривать 
асимптотическую аппроксимацию вида, вообще говоря, произвольной смеси 
логнормальных законов. 

На практике разумно использовать дискрет\-ную 
аппроксимацию для смешивающего закона, т.\,е.\ изначально предполагать, 
что распределение логарифмов размеров частиц является конечной смесью 
нормальных законов, что позволяет искать оценки параметров смеси (веса 
нормальных компонент, их математические ожидания и~дисперсии) с~помощью 
хорошо известных статистических процедур, традиционно используемых для 
этих целей, например с~использованием ЕМ (expectation-maximization)\linebreak алгоритма или его модификаций. 
Подобные модели допускают удобную интерпретацию, связанную с~описанием 
типичных видов частиц (породы или конкреций) и/или типов плазмохимических\linebreak 
реакций, влияющих на формирование ре\-го\-лита.
{ %\looseness=1

}

\renewcommand{\tablename}{\protect\bf Алгоритм}
\begin{table*}[b]\small %[!h]
\vspace*{9pt}
\begin{center}
%\hline
%\small
%\%label{AlgGeneratingSamples}
%\begin{algorithmic}
\begin{tabular}{l}
\hline
\\[-8pt]
\multicolumn{1}{c}{{\tablename\ 1}\ \  Имитационное моделирование выборок для аппроксимации 
и~статистического теста}\\[3pt]
\hline
\\[-8pt]
\textbf{function} [{\normalsize{S}}AMPLE, 
{\normalsize{T}}EST{\normalsize{S}}AMPLE]$\gets$
{\normalsize{G}}ENERATING{\normalsize{S}}AMPLES ({\normalsize{ECDF}})\\
\hspace*{5mm}// \textit{SampleSize~--- объем выборки для оценивания 
параметров}\\
\hspace*{5mm}// \textit{TestSampleSize~--- объем выборки для критерия 
Колмогорова}\\
\hspace*{5mm}// \textit{Случайные векторы для метода обратных функций}\\
\hspace*{5mm}r$\gets${\normalsize{R}}AND(SampleSize);\\
\hspace*{5mm}rTest$\gets${\normalsize{R}}AND(TestSampleSize);\\
\textit{//~Имитационное 
моделирование}\\
\hspace*{5mm}\textbf{for} i\;=\;1:SampleSize \textbf{do}\\
\hspace*{10mm}Sample(i)$\gets$ {FSOLVE} ({\normalsize{ECDF}, r(i)}); // 
\textit{Метод обратных функций}\\
\hspace*{10mm}\textbf{if}\ i$\leqslant$TestSampleSize\ \textbf{then}\\
\hspace*{15mm}TestSample(i) $\gets$ {{FSOLVE}} ({\normalsize{ECDF}}, rTest(i));\\
\hline
\end{tabular}
%%\EndFor
%%\EndFunction
%%\end{algorithmic}
%%\end{algorithm}
\end{center}
\end{table*}


\renewcommand{\tablename}{\protect\bf Таблица}

Итак, в~качестве модели распределения размеров частиц в~$\phi$-шка\-ле 
(т.\,е., фактически, логарифмов исходных размеров) будем использовать 
конечную смесь нормальных законов вида ($k\hm\in\mathbb{N}$, 
$a_i\hm\in\mathbb{R}$, $\sigma_i\hm>0$, $p_i\hm\geqslant 0$,
$i\hm=1,\ldots,k$, $p_1+\cdots+p_k\hm=1$):
\begin{equation}
\label{LogPhi}
{\mathbb P}(Z<x)=\sum\limits_{i=1}^k p_i\Phi\left(\fr{\log
x-a_i}{\sigma_i}\right).
\end{equation}


\section{Метод на~основе статистической симуляции~выборок}

В каталогах \verb"NASA"~\cite{Graf1993} данные представлены в~таб\-лич\-ной 
форме в~виде пар <<размер  
час\-ти\-цы\,--\,до\-ля (в~процентах) частиц такого размера 
в~про\-сеи\-ва\-емых образцах>>. Доступны сведения только о~нескольких (как 
правило, не более десяти) точках роста (выбранных, вообще говоря, 
бессистемно) эмпирической функции распределения, но не о~ее поведении 
между ними. В~работе~\cite{Graf1993} описано применение 
интерполяции Стайнемана~\cite{Stineman1980} для их соединения, благодаря которому
с~использованием макросов пакета \verb"Excel" возможно построение
 соответствующих  гистограмм.

В данной статье в~аналогичных целях применяются кусочные кубические 
полиномы Эрмита (см., например,~\cite{Fritsch1980})~--- они позволили 
получить наиболее близкие кривые по сравнению с~пред\-став\-лен\-ны\-ми 
в~каталоге \verb"NASA". Для симуляции выборок была апробирована 
и~кусочно-линейная интерполяция. Итоговые результаты в~этом случае 
получаются похожими, но менее наглядными, поэтому для сравнения в~данной 
статье не при\-во\-дятся.
{ %\looseness=1

}

Предложенная аппроксимация полиномами Эрмита позволила получить 
непрерывную эмпирическую функцию распределения (ECDF,
empirical cumulative distribution function), а~значит, стало 
возможным использовать метод обратных функций для генерации тестовых 
выборок. Объем выборки для оценивания параметров составлял $10\,000$ 
наблюдений; кроме того, для проверки соответствия приближающей смеси 
и~исходной эмпирической функции распределения генерировалась еще одна 
независимая выборка объемом 2500~наблюдения (для проверки гипотез 
с~помощью критерия согласия Колмогорова). 
%
Данная процедура в~целом близка 
к~такому статистическому подходу, как бутстреп. Ее описание в~виде 
псевдокода приведено в~алгоритме~1.




Полученные тестовые выборки используются для получения оценок 
максимального правдоподобия параметров аппроксимирующего смешанного 
распределения~\eqref{LogPhi} c помощью EM-алгоритма для нормальных 
законов~\cite{Korolev2011}. Примеры применения данной 
бут\-стреп-про\-це\-ду\-ры к~реальным пробам лунного реголита представлены на 
рис.~\ref{FigSoil12042} и~2.




На графиках слева представлены исходные данные из таблиц каталога 
\verb"NASA" (\textit{1}), полученные в~процессе 
просеивания, их интерполяция с~по\-мощью полиномов Эрмита~(\textit{2}) 
и~аппроксимиру\-ющая смесь~(\textit{3}). 
Дополнительная вставка на рис.~\ref{FigSoil12042} демонстрирует 
увеличенную область $[5,10]$. Видно, что обе кривые практически всюду 
совпадают. Такая ситуация повторяется для абсолютного большинства 
анализируемых проб. Отметим, что для числа компонент~$k$ 
в~формуле~\eqref{LogPhi} проверялись разные значения. Эмпирически было 
установлено, что необходимый баланс между качеством аппроксимации 
и~вычислительной сложностью достигается при значении $k\hm=4$, которое 
и~использовано в~данной статье при обработке всех 317~выборок.




На графиках справа на рис.~\ref{FigSoil12042} и~2 
приведены гистограммы для имитационных выборок и~просеянных данных, 
построенных по ECDF, для конкретной пробы, а~также их приближение 
плотностью смешанного распределения. Необходимо отметить, что точность 
аппроксимации определялась именно по сравнению с~эмпирической функцией 
распределения, а~данные графики служат лишь для более подробной 
иллюстрации работы метода.

Очевидно, что как функции распределения, так и~гистограммы приближаются 
визуально очень хорошо, даже с~учетом различных особенностей в~них. Ясно, 
что форма распределений в~каждом из случаев далека от стандартного 
гауссовского вида.

На рис.~3 приведены результаты проверки\linebreak с~помощью критерия 
однородности Колмогоро-\linebreak\vspace*{-10pt}

\pagebreak

\end{multicols}

\begin{figure*} %fig1
\vspace*{1pt}
 \begin{center}
 \mbox{%
 \epsfxsize=160.683mm 
 \epsfbox{gor-1.eps}
 }
 \end{center}
\vspace*{-9pt}
\Caption{Пробы лунного грунта 12042{,}24 (миссия <<Апполон-12>>)~(\textit{а})
и~14156{,}22 (миссия <<Апполон-14>>)~(\textit{б}):
\textit{1}~--- исходные данные из таблиц каталога NASA;
\textit{2}~--- интерполяция с~помощью полиномов Эрмита;
\textit{3}~---  аппроксимирующая смесь; 
\textit{4}~--- гистограммы для имитационных выборок;
\textit{5}~--- гистограммы для просеянных данных по ECDF;
\textit{6}~--- приближение плотностью смешанного распределения}
\label{FigSoil12042}
\vspace{12pt}
\end{figure*}



\begin{multicols}{2}

\noindent
ва и~дополнительно симулированных выборок. Для 
большей наглядности на графиках обозначены стандартные критические уровни 
0{,}01 и~0{,}05. Первый из них превышен $P$-зна\-че\-ни\-ями для 84{,}5\% 
выборок (268 из 317), а~ второй~--- для 70{,}7\% наборов (224 из 317), т.\,е.\
 для абсолютного большинства проб с~помощью бут\-стреп-ме\-то\-да 
получены достаточно хорошие результаты аппроксимации.



Отметим, что описанная процедура обработки данных была реализована на 
языке про\-грам\-ми\-ро\-ва\-ния \verb"MATLAB". Для расчетов использовались\linebreak ресурсы 
гибридного высокопроизводительного вычислительного кластера архитектуры 
\verb"Intel x86_64": сервер \verb"Huawei XH 622 V3" (два процессора 
\verb"Intel Xeon E5-2683V4" с~тактовой частотой 2{,}1~ГГц (16~ядер), 
512~ГБ оперативной памяти и~2~видеокарты \verb"NVIDIA Tesla K80". Это 
позволило повысить скорость вычислений в~среднем в~3{,}8~раза по 
сравнению со стандартными настольными решениями.


\section{Кластеризация параметров аппроксимирующих смесей}

\vspace*{-12pt}

В данном разделе приводятся результаты анализа параметров сразу всех 
аппроксимирующих конечных нормальных смесей. Кроме того, используется 
тривиальное обратное преобразование для перехода от $\phi$-шкалы для 
размеров к~метрической (в~микрометрах). На рис.~4 иллюстрируется взаимозависимость 
математического ожидания и~среднего квадратического отклонения для 
$\phi$-шкалы и~стандартных единиц измерения. Верхние графики на рис.~4,\,\textit{а} 
и~4,\,\textit{б} демонстрируют вид этой за\-ви\-си\-мости, при этом размер 
и~интенсивность цвета точек соответствуют их весам (см.\ 
формулу~\eqref{LogPhi} и~цветовую
шкалу вверху). Эти рисунки существенно уточняют предложенную 
в~статье~\cite{Slyuta2014} линейную аппроксимацию для данной за\-ви\-си\-мости, 
которая, как видно из рис.~4, оказывается чрезмерным загрублением.

\pagebreak

\end{multicols}

\begin{figure*} %fig2
\vspace*{1pt}
 \begin{center}
 \mbox{%
 \epsfxsize=160.683mm 
 \epsfbox{gor-3.eps}
 }
 \end{center}
\vspace*{-15pt}
\Caption{Пробы лунного грунта 72141{,}15~(\textit{а}) и~74221,82~(\textit{б})
 (миссия <<Апполон-17>>): \textit{1}~--- исходные данные из таблиц каталога NASA;
\textit{2}~--- интерполяция с~помощью полиномов Эрмита;
\textit{3}~---  аппроксимирующая смесь; 
\textit{4}~---  гистограммы для имитационных выборок;
\textit{5}~--- гистограммы для просеянных данных по ECDF;
\textit{6}~--- приближение плотностью смешанного распределения}
\label{FigSoil72141}
\end{figure*}

\begin{figure*} %fig3
\vspace*{1pt}
 \begin{center}
 \mbox{%
 \epsfxsize=160.077mm 
 \epsfbox{gor-5.eps}
 }
 \end{center}
\vspace*{-15pt}
\Caption{Ошибки аппроксимации (критерий Колмогорова): \textit{1}~--- 
$P$-значения; \textit{2} и~\textit{3}~--- критические уровни
 $\alpha\hm= 0{,}05$ и~$0{,}01$ соответственно}
\label{FigKSTests}
\vspace*{-6pt}
\end{figure*}

\begin{multicols}{2}



\begin{figure*} %fig4
\vspace*{1pt}
 \begin{center}
 \mbox{%
 \epsfxsize=154.153mm 
 \epsfbox{gor-6.eps}
 }
 \end{center}
\vspace*{-9pt}
\Caption{Кластеризация параметров аппроксимирующих смесей:
(\textit{а})~$\phi$-шкала;
(\textit{б})~метрическая шкала (мкм);
\textit{1}--\textit{5}~--- клас\-те\-ры;
\textit{6}~--- медоиды кластеров; 
\textit{7}~--- центры клас\-те\-ров по методу $c$-сред\-них}
\label{FigClustersHermite}
\end{figure*}

Два нижних графика на каждом из рис.~4,\,\textit{а}
и~4,\,\textit{б} демонстрируют разбиение параметрического 
пространства на $5$ групп методами $k$-медоид 
(см.\ рис.~4,\,\textit{а}) и~нечеткой кластеризации 
$c$-сред\-них (см. рис.~4,\,\textit{б}) 
(в~этом случае в~качестве окончательного решения выбирается 
кластер, для которого достигается максимальное значение величины 
вероятности членства 
для данного элемента среди всех возможных. Очевидно, что решения обоих 
методов в~каждом из случаев оказываются достаточно близкими. 
Представленные на рис.~4 данные позволяют выделить некоторые 
типичные кластеры, которые могут быть использованы, например, для 
соотнесения с~химическим составом проб или иными характеристиками 
реголита. На рис.~4 нумерация клас\-те\-ров произведена в~произвольном порядке и~не
связана с~ка\-ки\-ми-ли\-бо их характеристиками, например весом.

\vspace*{-6pt}


\section{Заключение}

\vspace*{-2pt}

В работе рассмотрена методология аппрок\-си\-мации распределений размеров 
частиц лунного ре\-голита с~помощью логнормальных смешанных мо\-делей. 
Продемонстрировано высокое согласие получа\-емых таким образом 
вероятностных распределений и~данных просеивания образцов лунного 
реголита. Дальнейшие исследования будут ориентированы на отказ от 
необходимости интерполяции точек~--- исходных данных 
в~каталоге~\verb"NASA". При этом их приближение также возможно функциями, 
имеющими вид конечной смеси логнормальных законов, однако подобная 
процедура не требует статистической симуляции выборок. Это может повысить 
точность аппроксимации, а~также уменьшить затрачиваемое на расчеты время.


\vspace*{-6pt}

{\small\frenchspacing
 {%\baselineskip=10.8pt
 \addcontentsline{toc}{section}{References}
 \begin{thebibliography}{99}


\vspace*{-2pt}

\bibitem{Popel2018} 
\Au{Попель~С.\,И., Голубь~А.\,П., Захаров~А.\,В. и~др.}
Формирование плазменно-пылевых облаков при ударе метеороида о~поверхность 
Луны~// Письма в~Журнал экспериментальной и~теоретической физики, 2018. 
Т.~108. Вып.~6. С.~379--387.

\bibitem{Razumovskii1940} 
\Au{Разумовский~Н.\,К.} Характер распределения содержания металлов 
в~рудных месторождениях~// Докл. Акад. наук СССР, 1940. Т.~28. Вып.~9. 
С.~815--817.

\bibitem{Kolmogorov1941} 
\Au{Колмогоров~А.\,Н.} О~логарифмически нормальном законе распределения 
размеров частиц при дроблении~// Докл. Акад. наук СССР, 1941. Т.~31. 
Вып.~2. С.~99--101.

%\bibitem{Florenskii1975} 
%\Au{Флоренский~К.\,П., Базилевский~А.\,Т., Николаева~О.\,В.} Лунный 
%грунт: свойства и~аналоги.~--- М.: Наука, 1975.

\bibitem{Bagnold1954} 
\Au{Bagnold~R.\,A.} The physics of blown sand and desert dunes.~--- 
London: Methuen, 1954. 265~p.

\bibitem{ReedJorgensen2003}  %5
\Au{Reed~W.\,J., Jorgensen~M.} The double Pareto-Lognormal distribution~--- 
a~new parametric model for size distribution~//
Commun. Stat. Theor.~M., 2004. Vol.~33. 
Iss.~8. P.~1733--1753.

\bibitem{Sorensen2006}  %6
\Au{\mbox{S{\!\!\ptb{{\o}}}\,rensen M.}} On the size distribution of sand.~--- 
Copenhagen: Department of Applied Mathematics and Statistics, 
University of Copenhagen, 2006.  Working 
paper. 11~p.

\bibitem{Korolev2009}
\Au{Королев~В.\,Ю.} О~распределении размеров частиц при дроблении~// 
Информатика и~её применения, 2009. Т.~3. Вып.~3. С.~60--68.

\bibitem{SK2019} 
\Au{Korolev~V.\,Yu., Skvortsova~N.\,N.} Size distribution of dust grains 
created in plasma-chemical reactions initiated by the microwave radiation 
of a~gyrotron in regolith~// Complex Systems of Charged Particles and 
Their Interactions with Electromagnetic Radiation: 17th Workshop 
(International) Proceedings.~--- Мoscow: Prokhorov General Physics Institute 
of the Russian Academy of Sciences, 2019. P.~25.

\bibitem{Graf1993} 
\Au{Graf~J.\,C.} Lunar soils grain size catalog.~--- NASA, 1993. 
NASA Reference 
Publication 1265. 484~p.
{\sf  
https://ntrs.nasa.gov/search.jsp?R=19930012474/}.

\bibitem{Donoghue2016} 
\Au{Donoghue~J.\,F.} Phi scale~// Encyclopedia of estuaries~/ 
Ed. M.\,J. Kennish.~--- Encyclopedia of Earth sciences ser.~---
Dordrecht: Springer, 2016. 789~p. doi: 10.1007/978-94-017-8801-4.

\bibitem{Stineman1980} 
\Au{Stineman~R.\,W.} A~consistently well behaved method of 
interpolation~// Creative Comput., 1980. Vol.~6. Iss.~7. P.~54--57.

\bibitem{Fritsch1980} 
\Au{Fritsch~F.\,N., Carlson~R.\,E.} Monotone piecewise cubic 
interpolation~// SIAM J.~Numer. Anal., 1980. Vol.~17. P.~238--246.

\bibitem{Korolev2011} 
\Au{Королев~В.\,Ю.} Ве\-ро\-ят\-ност\-но-ста\-ти\-сти\-че\-ские методы 
декомпозиции волатильности хаотических процессов.~--- М.:  
Изд-во Московского ун-та, 2011. 512~c.

\bibitem{Slyuta2014} 
\Au{Слюта~Е.\,Н.} Фи\-зи\-ко-ме\-ха\-ни\-че\-ские свойства лунного грунта 
(обзор)~// Астрономический вестник, 2014. Т.~48. Вып.~5. С.~358--382.

\end{thebibliography}

 }
 }

\end{multicols}

\vspace*{-6pt}

\hfill{\small\textit{Поступила в~редакцию 15.04.20}}

\vspace*{8pt}

%\pagebreak

%\newpage

%\vspace*{-28pt}

\hrule

\vspace*{2pt}

\hrule

%\vspace*{-2pt}

\def\tit{APPROXIMATION OF PARTICLE SIZE DISTRIBUTIONS\\ OF~LUNAR REGOLITH 
BASED ON~THE~RESAMPLING}


\def\titkol{Approximation of particle size distributions of lunar 
regolith based on the
resampling}

\def\aut{A.\,K.~Gorshenin$^{1,2}$ and~V.\,Yu.~Korolev$^{1,2}$}

\def\autkol{A.\,K.~Gorshenin and~V.\,Yu.~Korolev}

\titel{\tit}{\aut}{\autkol}{\titkol}

\vspace*{-9pt}


\noindent
$^1$Federal Research Center ``Computer Science and Control'' of the 
Russian Academy of Sciences, 44-2 Vavilov\linebreak
$\hphantom{^1}$Str., Moscow 119333, Russian 
Federation

\noindent
$^2$Faculty of Computational Mathematics and Cybernetics, 
M.\,V.~Lomonosov Moscow
State University, GSP-1,\linebreak
$\hphantom{^1}$Leninskie Gory, Moscow 119991, Russian 
Federation


\def\leftfootline{\small{\textbf{\thepage}
\hfill INFORMATIKA I EE PRIMENENIYA~--- INFORMATICS AND
APPLICATIONS\ \ \ 2020\ \ \ volume~14\ \ \ issue\ 2}
}%
 \def\rightfootline{\small{INFORMATIKA I EE PRIMENENIYA~---
INFORMATICS AND APPLICATIONS\ \ \ 2020\ \ \ volume~14\ \ \ issue\ 2
\hfill \textbf{\thepage}}}

\vspace*{3pt}


\Abste{The paper considers the problem of modeling the size distribution 
of dust particles of lunar regolith based on approximating with finite 
lognormal mixtures. These models make it possible to take into account 
the stochastic nature of the intensities of splitting/baking processes 
during the formation of ensembles of dust particles as a~result\linebreak\vspace*{-12pt}}

\Abstend{of 
various influences (bombardment by meteorites, radiation). A~method for 
statistical approximation of unknown distributions based on simulation of 
samples was developed. It is demonstrated that the model distributions 
fit very well to the real observations of lunar regolith gathered by 
missions ``Apollo 11, 12, 14--17'' and ``Luna-24'' that had been presented 
in the {\sf NASA}'s grain size catalog (317~samples).}



\KWE{finite lognormal mixtures; bootstrap; EM algorithm; statistical 
methods}


\DOI{10.14357/19922264200207} 

%\vspace*{-20pt}

\Ack
\noindent
The research is supported by the Russian Science Foundation 
(project~18-11-00155). The calculations were performed using Hybrid high-performance 
computing cluster of FRC CSC RAS ({\sf http://ckp.frccsc.ru/}).

%\vspace*{6pt}

 \begin{multicols}{2}

\renewcommand{\bibname}{\protect\rmfamily References}
%\renewcommand{\bibname}{\large\protect\rm References}

{\small\frenchspacing
 {%\baselineskip=10.8pt
 \addcontentsline{toc}{section}{References}
 \begin{thebibliography}{99}


\bibitem{1-gor}
\Aue{Popel,~S.\,I., A.\,P.~Golub, A.\,V.~Zakharov, \textit{et al.}} 2018.
Formation of dusty plasma clouds at meteoroid impact on the surface of 
the Moon. \textit{JETP Lett.} 108(6):356--
363.

\bibitem{2-gor}
\Aue{Razumovskii,~N.\,K.} 1940. Kharakter raspredeleniya soderzhaniya 
metallov v~rudnykh mestorozhdeniyakh [The nature of the distribution of 
metal content in ore deposits]. \textit{Dokl. Akad. Nauk SSSR}  28(9):815--817. 

\bibitem{3-gor}
\Aue{Kolmogorov,~A.\,N.} 1941. O logarifmicheski normal'nom zakone 
raspredeleniya razmerov chastits pri droblenii [On the log-normal law of 
particle size distribution during fracturing]. \textit{Dokl. Akad. Nauk 
SSSR} 31(2):99--101.

%\bibitem{4-gor}
%\Aue{Florenskii,~K.\,P., A.\,T.~Bazilevskii, and O.\,V.~Nikolaeva.} 1975. 
%\textit{Lunnyy grunt: svoystva i~analogi} [Lunar soil: Properties and 
%analogues]. Moscow: Nauka.

\bibitem{5-gor}
\Aue{Bagnold,~R.\,A.} 1954. \textit{The physics of blown sand and desert 
dunes}. London: Methuen. 265~p.

\bibitem{6-gor}
\Aue{Reed,~W.\,J., and M.~Jorgensen.} 2004. The double Pareto-Lognormal 
distribution~--- a~new parametric model for size distribution. 
\textit{Commun. Stat. Theor.~M.} 33(8):1733--1753.

\bibitem{7-gor}
\Aue{\mbox{S\!{\ptb{\o}}\,rensen, M.}} 2006. {On the size distribution of 
sand}. Copenhagen: Department of Applied Mathematics and Statistics,
University of Copenhagen. Working paper. 11~p.
\bibitem{8-gor}
\Aue{Korolev,~V.\,Yu.} 2009. O raspredelenii razmerov chastits pri 
droblenii [On the distribution of particle size under fracturing]. 
\textit{Informatika i ee Primeneniya~--- Inform. Appl.} 3(3):60--68. 

\bibitem{9-gor}
\Aue{Korolev,~V.\,Yu., and N.\,N.~Skvortsova.} 2019. Size distribution of 
dust grains created in plasma-chemical reactions initiated by the 
microwave radiation of a~gyrotron in regolith. \textit{17th Workshop 
(International) ``Complex Systems of Charged Particles and Their 
Interactions with Electromagnetic Radiation'' Proceedings}. Moscow:
Prokhorov General Physics Institute 
of the Russian Academy of Sciences. 25.

\bibitem{10-gor}
\Aue{Graf,~J.\,C.} 1993. Lunar soils grain size catalog. 
NASA Reference 
Publication 1265. Available at: 
{\sf  
https://ntrs.nasa.gov/search.jsp?R=19930012474/} (accessed May~25, 2020).

\bibitem{11-gor}
\Aue{Donoghue,~J.\,F.} 2016. Phi scale. \textit{Encyclopedia of 
estuaries}. Ed. M.\,J.~Kennish. 
Encyclopedia of Earth sciences ser.
Dordrecht: Springer. 789~p. doi: 10.1007/978-94-017-8801-4.

\bibitem{12-gor}
\Aue{Stineman,~R.\,W.} 1980. A~consistently well behaved method of 
interpolation. \textit{Creative Comput.}  
6(7):54--57.

\bibitem{13-gor}
\Aue{Fritsch,~F.\,N., and R.\,E.~Carlson.} 1980. Monotone piecewise cubic 
interpolation. \textit{SIAM J.~Numer. Anal.} 17:238--246.

\bibitem{14-gor}
\Aue{Korolev,~V.\,Yu.} 2011. \textit{Veroyatnostno-statisticheskie metody 
dekompozitsii 
volatil'nosti khaoticheskikh protsessov} [Probabilistic and statistical 
methods of decomposition of volatility of chaotic processes]. Moscow: 
Izd-vo Moskovskogo un-ta. 512~p.

\bibitem{15-gor}
\Aue{Slyuta,~E.\,N.} 2014. Physical and mechanical properties of the 
lunar soil (a~review). \textit{Solar Syst. Res.} 48(5):330--353.

\end{thebibliography}

 }
 }

\end{multicols}

\vspace*{-9pt}

\hfill{\small\textit{Received April 15, 2020}}

%\pagebreak

%\vspace*{-24pt}



\Contr

\noindent
\textbf{Gorshenin Andrey K.} (b.\ 1986)~--- Candidate of Science (PhD) in 
physics and
mathematics, associate professor, leading scientist, Federal Research 
Center ``Computer Science and Control'' of the Russian Academy of 
Sciences, 44-2~Vavilov Str., Moscow 119333, Russian Federation; leading 
scientist, Faculty of Computational Mathematics and Cybernetics, 
M.\,V.~Lomonosov Moscow State University, GSP-1, Leninskie Gory, Moscow 
119991, Russian Federation; \mbox{agorshenin@frccsc.ru}

\vspace*{3pt}

\noindent
\textbf{Korolev Victor Yu.} (b.\ 1954)~--- Doctor of Science in 
physics and
mathematics, professor, head of department, Faculty of Computational 
Mathematics and Cybernetics, M.\,V.~Lomonosov Moscow State University, 
GSP-1, Leninskie Gory, Moscow 119991, Russian Federation; leading 
scientist, Federal Research Center ``Computer Science and Control'' of 
the Russian Academy of Sciences, 44-2~Vavilov Str., Moscow 119333, 
Russian Federation; \mbox{vkorolev@cs.msu.ru}


\label{end\stat}

\renewcommand{\bibname}{\protect\rm Литература} 