\def\stat{shnurkov}

\def\tit{РЕШЕНИЕ ЗАДАЧИ БЕЗУСЛОВНОГО ЭКСТРЕМУМА 
ДЛЯ~ДРОБНО-ЛИНЕЙНОГО ИНТЕГРАЛЬНОГО 
ФУНКЦИОНАЛА, ЗАВИСЯЩЕГО ОТ~ПАРАМЕТРА}

\def\titkol{Решение задачи безусловного экстремума 
для~дробно-линейного интегрального функционала} %, зависящего  от~параметра}

\def\aut{П.\,В.~Шнурков$^1$, К.\,А.~Адамова$^2$}

\def\autkol{П.\,В.~Шнурков, К.\,А.~Адамова}

\titel{\tit}{\aut}{\autkol}{\titkol}

\index{Шнурков П.\,В.}
\index{Адамова К.\,А.}
\index{Shnurkov P.\,V.}
\index{Adamova K.\,A.}
 

%{\renewcommand{\thefootnote}{\fnsymbol{footnote}} \footnotetext[1]
%{Работа выполнена при финансовой поддержке РФФИ (проект 18-07-00252).}}


\renewcommand{\thefootnote}{\arabic{footnote}}
\footnotetext[1]{Национальный исследовательский университет <<Высшая школа экономики>>, 
pshnurkov@hse.ru}
\footnotetext[2]{Научно-производственный центр автоматики и~приборостроения им. академика 
Н.\,А.~Пилюгина, \mbox{ksenya\_an@mail.ru}}

%\vspace*{-6pt}

\Abst{Работа посвящена исследованию задачи безусловного экстремума 
дробно-линейного интегрального функционала, заданного на множестве вероятностных 
распределений. В~отличие от результатов, полученных ранее, в~рассматриваемой задаче 
подынтегральные функции интегральных выражений, находящихся в~числителе 
и~знаменателе, зависят от некоторого векторного вещественного параметра оптимизации. 
Таким образом, задача оптимизации исследуется на декартовом произведении множества 
вероятностных распределений и~множества допустимых значений векторного 
вещественного параметра. Доказаны три утверж\-де\-ния об экстремуме  
дроб\-но-ли\-ней\-но\-го интегрального функционала. Установлено, что во всех вариантах 
решение исходной задачи полностью определяется экстремальными свойствами основной 
функции дроб\-но-ли\-ней\-но\-го интегрального функционала, которая представляет собой 
отношение подынтегральных функций числителя и~знаменателя. Описаны возможные 
применения полученных результатов в~задачах оптимального управления стохастическими 
системами.}
      
      \KW{дробно-линейный интегральный функционал; задача безусловного экстремума 
дробно-ли\-ней\-но\-го интегрального функционала; основная функция; задачи оптимального 
управления марковскими и~полумарковскими случайными процессами}
      
\DOI{10.14357/19922264200214} 
 
%\vspace*{9pt}


\vskip 10pt plus 9pt minus 6pt

\thispagestyle{headings}

\begin{multicols}{2}

\label{st\stat}
      
      
\section{Введение}

     Настоящая работа продолжает цикл исследований проб\-ле\-мы 
безусловного экстремума для 
     дроб\-но-ли\-ней\-но\-го интегрального функционала, начатый 
в~работах~[1, 2].
     
     Данная проблема имеет не только общетеорети\-ческое значение, но 
и~служит основой для реше\-ния задач оптимального управ\-ле\-ния различными\linebreak 
классами случайных процессов (регенерирующими, марковскими, 
полумарковскими). В~свою очередь, задачи оптимального управ\-ле\-ния 
в~указанных классах процессов возникают при анализе многочисленных 
при\-клад\-ных моделей в~математической теории эффективности и~на\-деж\-ности, 
тео\-рии управ\-ле\-ния запасами и~других об\-ластях прикладной тео\-рии 
вероятностей.
     
     Сделаем некоторые замечания биб\-лио\-гра\-фи\-че\-ско\-го характера, 
относящиеся к~общей проб\-ле\-ме безусловного экстремума для 
дроб\-но-ли\-ней\-но\-го интегрального функционала.
     
     Обычно дробно-ли\-ней\-ным программированием называется раздел 
тео\-рии оптимизации, в~котором целевой функционал рас\-смат\-ри\-ва\-емой 
экстремальной задачи пред\-став\-ля\-ет собой отношение двух линейных 
функционалов, а~име\-ющи\-еся ограничения носят линейный характер. В~данной 
об\-ласти оптимизации имеется обширная научная литература, основная часть 
которой посвящена исследованию соответствующих задач в~конечномерных 
пространствах.
     
     Современная теория указанного научного на\-прав\-ле\-ния изложена 
в~фундаментальной монографии~[3]. В~этой книге не только приводятся 
теоретические результаты решения со\-от\-вет\-ст\-ву\-ющих экстремальных задач, но 
и~описываются чис\-лен\-ные методы нахождения таких решений. Кроме того, 
в~ней приведена подробная библиография научных исследований в~области 
дроб\-но-ли\-ней\-но\-го\linebreak программирования. Отметим также некоторые\linebreak 
значительные работы по\-след\-них лет, в~которых исследовались теоретические 
и~вы\-чис\-ли\-тель\-ные проб\-ле\-мы, связанные с~данным научным на\-прав\-ле\-ни\-ем~[4--6].
     
     Специальный раздел дроб\-но-ли\-ней\-но\-го программирования 
со\-став\-ля\-ют экстремальные задачи, в~которых целевой функционал 
пред\-став\-ля\-ет\linebreak собой отношение двух интегралов. Подынтегральные функции 
в~этих интегралах предполагаются известными, а~интегрирование проводится 
по вероятностной мере, принадлежащей некоторому множеству вероятностных 
мер, определенных на заданном измеримом пространстве. Решением задачи 
служит вероятностная мера, до\-став\-ля\-ющая глобальный экстремум такому 
функ\-ци\-о\-на\-лу. Функционалы указанного вида можно назвать интегральными 
дроб\-но-ли\-ней\-ны\-ми функционалами. Экстремальные задачи для 
интегральных дроб\-но-ли\-ней\-ных целевых функционалов, заданных на 
множестве вероятностных распределений в~конечномерном про\-стран\-ст\-ве, 
рассматривались в~работах~[7, 8]. Результаты  
исследований по тео\-рии безуслов\-но\-го экстремума для таких функционалов 
в~наиболее завершенной форме изложены в~гл.~10 коллективной 
монографии~\cite{8-shn}. Основное содержание этих результатов заключается 
в~том, что безусловный экстремум функционалов такого вида достигается на 
вырожденных вероятностных распределениях, име\-ющих одну точ\-ку рос\-та. 
Однако этот результат был получен в~работе~\cite{8-shn} при сильных 
ограничительных условиях, главное из которых~--- предположение 
о~существовании экстремума целевого функционала, т.\,е.\ существовании 
решения исходной задачи. 

В~работах~[1, 2] было предложено 
новое решение задачи без\-услов\-но\-го экстремума для дроб\-но-ли\-ней\-но\-го 
интегрального функционала, которое существенно обобщает и~усиливает 
результаты работы~\cite{8-shn}. Принципиальное отличие результатов 
указанных работ от пред\-ше\-ст\-ву\-ющих за\-клю\-ча\-ет\-ся в~том, что в~основном 
утверж\-де\-нии указываются условия, при выполнении которых экстремум 
дроб\-но-ли\-ней\-но\-го интегрального функционала существует и~достигается на 
вы\-рож\-ден\-ном распределении, сосредоточенном в~одной точке. При этом точка, 
в~которой сосредоточена вся вероятностная мера,~--- это точка глобального 
экстремума функ\-ции, для которой известно явное аналитическое 
представление. 
     
     Таким образом, исследование экстремальных свойств данной функции 
позволяет одновременно доказать существование решения исходной 
экстремальной задачи и~найти само это решение, опре\-де\-ля\-емое точ\-кой 
глобального экстремума. 
     
     В работах~[1, 2] предполагалось, что подынтегральные функции 
числителя и~знаменателя 
     дроб\-но-ли\-ней\-но\-го интегрального функционала не зависят от 
вероятностной меры, характеризующей управ\-ле\-ние. Данное обстоятельство 
оправдано тем, что во многих задачах прикладного содержания\linebreak целевой 
показатель пред\-став\-ля\-ет собой стационарный стоимостный функционал, 
структура которого установлена и~обладает указанными особенностями. 
Однако в~ряде задач целевой показатель, 
представляемый в~форме дроб\-но-ли\-ней\-но\-го интегрального функционала, 
устроен таким образом, что подынтегральные функции его чис\-ли\-те\-ля 
и~знаменателя зависят от некоторого набора детерминированных па\-ра\-мет\-ров 
управления. В~этом случае экстремальная задача меняет свой характер 
и~нуж\-да\-ет\-ся в~специальном исследовании. Такому исследованию 
и~посвящена данная работа.

\vspace*{-6pt}

\section{Постановка основной экстремальной задачи}

\vspace*{-2pt}

     Обозначим через $(U,\mathcal{B})$ некоторое измеримое пространство, 
где  
$U$~--- множество произвольной природы; $\mathcal{B}$~--- $\sigma$-ал\-геб\-ра 
подмножеств множества~$U$, вклю\-ча\-ющая в~себя все одноточечные 
множества.
     
     В дальнейшем пространство $(U,\mathcal{B})$ будет служить 
множеством допустимых значений стохастических параметров оптимизации или 
управлений в~рас\-смат\-ри\-ва\-емой экстремальной задаче.
     
     Пусть~$\Gamma$~--- некоторое множество вероятностных мер, заданных 
на  
$\sigma$-ал\-геб\-ре~$\mathcal{B}$, элементы которого будем обозначать 
$\Psi\hm\in\Gamma$. Некоторые требования к~множеству~$\Gamma$ и~его 
элементам, связанные с~постановкой основной экстремальной задачи, будут 
сформулированы в~дальнейшем.
      
     Классическая теория измеримых пространств и~мер, за\-да\-ва\-емых на этих 
пространствах, изложена в~известной работе П.~Халмоша~[9].
     
     \smallskip
     
     \noindent
     \textbf{Определение~1.}\ Назовем вероятностную меру~$\Psi^*$, 
заданную на~$\mathcal{B}$, вырожденной, если существует точка $u^*\hm\in U$ 
такая, что $\Psi^*(\{u^*\})\hm=1$, $\Psi^*(B^*)=0$, где $u^*\hm= \{ 
u^*\}$~--- множество, со\-сто\-ящее из одной точ\-ки, $B^*$~--- произвольное 
множество из системы~$\mathcal{B}$, не содержащее точку~$u^*$. 
Точку~$u^*$ будем называть точ\-кой сосредоточения меры~$\Psi^*$ 
и~обозначать меру со своей точ\-кой сосредоточения символами $\Psi^*\hm= 
\Psi^*_{u^*}$.
     
     Обозначим через~$\Gamma^*$ множество всех возможных вырожденных 
вероятностных мер, заданных на измеримом пространстве $(U,\mathcal{B})$. 
Множество~$\Gamma^*$ находится во взаимно однозначном соответствии 
с~множеством всех точек сосредоточения вы\-рож\-ден\-ных вероятностных мер, 
т.\,е.\ с~самим множеством~$U$.
     
     Теперь введем $S\subseteq R^r$~--- множество значений векторного 
па\-ра\-мет\-ра $\alpha\hm= (\alpha_1,\alpha_2,\hm\ldots , \alpha_r)\hm\in S$. В~дальнейшем 
параметр~$\alpha$ будет служить дополнительным детерминированным 
параметром оптимизации (управ\-ле\-ния) в~рассматриваемой экстремальной 
задаче.
     
     Зададим некоторые измеримые чис\-ло\-вые функции:
$$
A(\alpha, u): S\times U \to R\,;\enskip B(\alpha, u): S\times U\to R\,,
$$
где $u\in U$, $\alpha\hm\in S$.
     
Введем интегральные преобразования, за\-да\-ва\-емые функциями $A(\alpha,u)$
и~$B(\alpha, u)$:

\noindent
\begin{equation}
\left. 
\begin{array}{rl}
I_{1,\alpha}(\Psi)&=\displaystyle\int\limits_U A(\alpha, u)\,d\Psi(u)\,;\\
I_{2,\alpha}(\Psi) &=\displaystyle\int\limits_U B(\alpha, u)\,d\Psi(u)\,.
\end{array}
\right\}
\label{e1-shn}
\end{equation}

В своей наиболее общей форме интегральные выражения 
в~соотношениях~(1) пред\-став\-ля\-ют собой интегралы Лебега по вероятностной 
мере $\Psi\hm\in \Gamma$. Согласно общей вероятностной модели, по\-дроб\-но 
изложенной в~работе~\cite[гл.~II]{10-shn}, эти интегралы имеют смысл 
математических ожиданий некоторых случайных величин $A(\alpha, u)$ 
и~$B(\alpha, u)$, зависящих от элементарного исхода $u\hm\in U$ случайного 
эксперимента на вероятностном пространстве~$(U,\mathcal{B},\Psi)$. Если 
множество~$U$~--- конечномерное вещественное про\-стран\-ст\-во, то 
вероятностная мера~$\Psi$ может быть задана функцией распределения 
одномерной или многомерной случайной величины. Тогда интегралы 
в~соотношениях~(1) можно понимать как интегралы Ле\-бе\-га--Стилть\-еса по 
вероятностному распределению~$\Psi$. Заметим, что теория интегралов 
данного вида в~краткой форме изложена в~упомянутых работах~\cite{9-shn, 
10-shn}, а~в~более развернутом виде представлена в~книге~\cite{11-shn}.
     
     Интегральные преобразования~(1) также определяют функционалы, 
заданные на множестве вероятностных мер~$\Gamma$: 
     $$
     I_{1,\alpha}(\Psi):\Gamma\to R\,,\enskip I_{2,\alpha}(\Psi): \Gamma\to R\,,
     $$
которые зависят от параметра $\alpha\hm\in S$.
     
     Теперь введем понятие дроб\-но-ли\-ней\-но\-го интегрального 
функционала, зависящего от па\-ра\-метра.

\smallskip

\noindent
\textbf{Определение~2.}\ Отображение~$I_\alpha: \Gamma\to R$, 
опре\-де\-ля\-емое соотношением:
\begin{equation}
I_\alpha(\Psi) = \fr{I_{1,\alpha}(\Psi)}{ 
I_{2,\alpha}(\Psi)}=\fr{\int\nolimits_U A(\alpha,u)\,d\Psi(u)} 
{\int\nolimits_U B(\alpha,u)\,d\Psi(u)}\,,
\label{e2-shn}
\end{equation}
будем называть дроб\-но-ли\-ней\-ным интегральным функционалом, 
зависящим от па\-ра\-мет\-ра~$\alpha$.
     
     Рассмотрим следующую экстремальную задачу:
     \begin{equation}
     I_\alpha(\Psi)\to \mathrm{extr}\,,\enskip \Psi\in \Gamma,\ \alpha\in S\,,
     \label{e3-shn}
     \end{equation}
для дробно-ли\-ней\-но\-го интегрального функционала, зависящего от чис\-ло\-во\-го 
па\-ра\-мет\-ра~$\alpha$.
     
     Уточним, что данная экстремальная задача рассматривается на множестве 
пар $(\alpha, \Psi)$, которые и~пред\-став\-ля\-ют собой двумерный параметр 
оптимизации.
     
     \smallskip
     
     \noindent
     \textbf{Определение~3.}\ Будем называть функцию $C(\alpha,u)\hm= 
A(\alpha,u)/B(\alpha,u)$, $u\hm\in U$, $\alpha\hm\in S$, основной функцией  
дроб\-но-ли\-ней\-но\-го интегрального функционала~(2).
     
     Задача безусловного экстремума для дроб\-но-ли\-ней\-но\-го 
интегрального функционала вида~(2), не зависящего от па\-ра\-мет\-ра, была 
исследована в~работах~[1, 2]. Отличие рассматриваемой задачи от 
предшествующей заключается в~том, что в~целевом функционале~(2) 
присутствует зависимость подынтегральных функций чис\-ли\-те\-ля и~знаменателя 
от некоторого дополнительного неслучайного параметра оптимизации 
(управ\-ле\-ния) $\alpha\hm\in S$. 

\section{Решение поставленной экстремальной задачи}

     Введем некоторые предварительные условия для основных объектов, 
входящих в~описание экстремальной задачи~(3). Эти условия призваны 
обеспечить корректность по\-став\-лен\-ной задачи.
     \begin{enumerate}[1.]
\item Интегральные выражения~(1) 
существуют для всех $\Psi\hm\in \Gamma$, $\alpha\hm\in S$.
\item $\int\nolimits_U B(\alpha,u)\,d\Psi(u)\not= 0$, $\Psi\hm\in 
\Gamma$, $\alpha\hm\in S$. 
\item $\Gamma^*\subset \Gamma$.
\end{enumerate}

\noindent
     \textbf{Замечание~1.}\ Если функция~$B(\alpha, u)$ обладает свойством 
строгого знакопостоянства, т.\,е.\ ~$B(\alpha, u)\hm>0$, $u\hm\in U$, 
$\alpha\hm\in S$, или ~$B(\alpha, u) \hm<0$, $u\hm\in U$, $\alpha\hm\in S$, то 
условие~2 из указанной сис\-те\-мы предварительных условий выполняется 
автоматически. В~то же время условие строгой по\-ло\-жи\-тель\-ности 
функции~$B(\alpha, u)$ является естественным для многих задач оптимального 
управления регенерирующими и~полумарковскими случайными процессами 
(см.\ соответствующие пояснения в~работе~[2]). В~связи с~этим 
в~формулировках всех по\-сле\-ду\-ющих основных утверж\-де\-ний об экстремуме 
дроб\-но-ли\-ней\-но\-го интегрального функционала будет предполагаться, что 
выполнены условия~1, 3 и~условие строгого знакопостоянства 
функции~$B(\alpha, u)$.
     
     \smallskip
     
     \noindent
\textbf{Теорема~1.}\ \textit{Предположим, что в~экстремальной 
задаче}~(3) \textit{основные объекты удовле\-тво\-ря\-ют предварительным
условиям~$1$ и~$3$, а~функция~$B(\alpha, u)$ строго знакопостоянна (строго положительна 
или строго отрицательна) для всех $u\hm\in U$, $\alpha\hm\in S$. Предположим 
также, что основная функ\-ция~$C(\alpha, u)$ достигает глобального экстремума 
на всем множестве $(\alpha, u)\hm\in S\times U$ в~точке~$(\alpha^*, u^*)$.}
     
     \textit{Тогда решение экстремальной задачи}~(3) \textit{существует 
и~достигается на паре $(\alpha^*, \Psi^*_{u^*})$, где~$\Psi^*_{u^*}$~--- 
вы\-рож\-ден\-ная вероятностная мера, сосредоточенная в~точ\-ке~$u^*$, и~при этом 
выполняются соотношения}:
     \begin{multline*}
     \max\limits_{(\alpha,\Psi)\in S\times \Gamma} I_\alpha(\Psi) =  
\max\limits_{\alpha\in S} \max\limits_{\Psi^*\in \Gamma^*} I_\alpha(\Psi^*)={}\\
     {}= \max\limits_{(\alpha,u)\in S\times U} \fr{A(\alpha,u)}{B(\alpha,u)}=
     \fr{A(\alpha^*,u^*)}{B(\alpha^*,u^*)}\,,
     \end{multline*}
\textit{если $(\alpha^*,u^*)$~--- точка глобального максимума функции} 
$C(\alpha, u)$;
     \begin{multline*}
     \min\limits_{(\alpha,\Psi)\in S\times \Gamma} I_\alpha(\Psi) =  
\min\limits_{\alpha\in S} \min\limits_{\Psi^*\in \Gamma^*} I_\alpha(\Psi^*)={}\\
     {}= \min\limits_{(\alpha,u)\in S\times U} \fr{A(\alpha,u)}{B(\alpha,u)}=
     \fr{A(\alpha^*,u^*)}{B(\alpha^*,u^*)}\,,
\end{multline*}
\textit{если $(\alpha^*,u^*)$~--- точка глобального минимума 
функции}~$C(\alpha, u)$.

\smallskip

     Полное доказательство тео\-ре\-мы~1 приведено в~приложении 
к~на\-сто\-ящей работе~[12], а~также в~пуб\-ли\-ка\-ции~[13].
     
     Теперь исследуем решение экстремальной задачи~(3) для вариантов, 
в~которых основная функция  
дроб\-но-ли\-ней\-но\-го интегрального функционала~(2) не достигает 
глобального экстремума.
     
     \smallskip
     
     \noindent
     \textbf{Теорема~2.}\ \textit{Предположим, что в~экстремальной 
задаче}~(3) \textit{основные объекты удовлетворяют условиям~$1$ и~$3$, 
а~функ\-ция~$B(\alpha, u)$ строго знакопостоянна (строго положительна или 
строго отрицательна). Предположим также, что основная функция~$C(\alpha, 
u)$ ограничена (сверху или снизу), но при этом не достигает глобального 
экстремума (максимума или минимума) на множестве}~$S\times U$.
     
     \textit{Тогда справедливы следующие утверж\-де\-ния}.
     \begin{enumerate}[1.]
\item \textit{Если основная функция $C(\alpha, u)$ ограничена сверху и~не 
достигает глобального максимума, то для любого заданного 
$\varepsilon\hm>0$ существует пара $\alpha^{(+)} (\varepsilon)\hm\in S$ 
и~$u^{(+)}(\varepsilon)\hm\in U$ такая, что выполняется двойное неравенство}:
\begin{multline}
\mathop{\mathrm{sup}}\limits_{(\alpha, \Psi)\in S\times \Gamma} I_\alpha(\Psi)-
\varepsilon < I_{\alpha^{(+)}(\varepsilon)}\left( 
\Psi^*_{u^{(+)}(\varepsilon)}\right)={}\\
{}= \fr{A(\alpha^{(+)}(\varepsilon), u^{(+)}(\varepsilon))}{ 
B(\alpha^{(+)}(\varepsilon), u^{(+)}(\varepsilon))}<\sup\limits_{(\alpha,\Psi)\in 
S\times \Gamma} I_\alpha (\Psi)\,,
\label{e4-shn}
\end{multline}
\textit{где $\Psi^*_{u^{(+)}(\varepsilon)}\in \Gamma^*$~--- вы\-рож\-ден\-ная 
вероятностная мера, сосредоточенная в~точке} $u^{(+)}(\varepsilon)$.
\item \textit{Если основная функция $C(\alpha, u)$ ограничена снизу и~не 
достигает глобального минимума, то для любого заданного $\varepsilon \hm >0$ 
существует пара $\alpha^{(-)} (\varepsilon)\hm\in S$ и~$u^{(-)}
(\varepsilon)\hm\in U$ такая, что выполняется двойное неравенство}:
\begin{multline}
\mathop{\mathrm{inf}}\limits_{(\alpha,\Psi)\in S\times \Gamma} I_\alpha(\Psi)< 
I_{\alpha^{(-)}(\varepsilon)} \left( \Psi^*_{u^{(-)}(\varepsilon)}\right)={}\\
\hspace*{-5mm}{}= \fr{A(\alpha^{(-)}(\varepsilon), u^{(-)}(\varepsilon))}{ B(\alpha^{(-)}
(\varepsilon), u^{(-)}(\varepsilon))} <\mathop{\mathrm{inf}}\limits_{(\alpha,\Psi)\in 
S\times\Gamma} I_\alpha(\Psi)+\varepsilon,\!
\label{e5-shn}
\end{multline}
\textit{где $\Psi^*_{u^{(-)}(\varepsilon)}\hm\in \Gamma^*$~--- 
вы\-рож\-ден\-ная вероятностная мера, со\-сре\-до\-то\-чен\-ная в~точке}~$u^{(-)}
(\varepsilon)$.
     \end{enumerate}
     
     Заметим, что утверж\-де\-ния~1 и~2, т.\,е.\ соотношения~(4) и~(5), могут 
выполняться как по отдельности для верхней или нижней грани множества 
значений функционала, так и~со\-вместно.
     
     \smallskip
     
     Доказательство теоремы~2 приведено в~приложении к~на\-сто\-ящей 
работе~[12], а~так\-же в~пуб\-ли\-ка\-ции~[13].
     
     \smallskip
     
     Сущность утверж\-де\-ния тео\-ре\-мы~2 заключается в~том, что если основная 
функция  
дроб\-но-ли\-ней\-но\-го интегрального функционала ограничена, но не 
достигает экстремума, то для любого заданного $\varepsilon\hm>0$ существует 
так называемая $\varepsilon$-оп\-ти\-маль\-ная детерминированная стратегия 
управления, за\-да\-ва\-емая значениями $\alpha^{(+)} (\varepsilon)\hm\in S$ 
и~$u^{(+)}(\varepsilon)\hm\in U$ для задачи на максимум или значениями 
$\alpha^{(-)} (\varepsilon)\hm\in S$  
и~$u^{(-)}(\varepsilon)\hm\in U$ для задачи на минимум.
     
     Перейдем к~формулировке еще одного утверж\-де\-ния, связанного 
с~экстремальной проблемой для  
дроб\-но-ли\-ней\-но\-го интегрального функционала, зависящего от па\-ра\-метра.
     
     \smallskip
     
     \noindent
     \textbf{Теорема~3.}\ \textit{Предположим, что в~экстремальной 
задаче}~(3) \textit{основные объекты удовлетворяют условиям~$1$ и~$3$, 
а~функция~$B(\alpha,u)$ знакопостоянна (строго положительна или строго 
отрицательна). Предположим также, что основная функция~$C(\alpha, u)$ не 
ограничена (сверху или снизу). Тогда соответствующий дроб\-но-ли\-ней\-ный 
интегральный функционал также не ограничен сверху или снизу 
и~справедливы сле\-ду\-ющие утверж\-де\-ния}:
     \begin{enumerate}[1.]
     \item
\textit{Существует последовательность $(\alpha_n^{(+)}, 
\Psi_n^{*(+)})$, $\alpha_n^{(+)}\hm\in S$, $\Psi_n^{*(+)}\hm\in 
\Gamma^*$, $n\hm=1,2,\ldots$, такая что}
 \begin{equation}
I\left( \alpha_n^{(+)}, \Psi_n^{*(+)} \right)\to \infty\,,\enskip n\to \infty\,,
\label{e6-shn}
\end{equation}
\textit{если основная функция не ограничена сверху}.
\item 
\textit{Существует последовательность $(\alpha_n^{(-)}, \Psi_n^{*(-)})$, 
$\alpha_n^{(-)}\hm\in S$, $\Psi_n^{*(-)}\hm\in \Gamma^*$, 
$n\hm=1,2,\ldots$, такая что} 
\begin{equation}
I\left( \alpha_n^{(-)}, \Psi_n^{*(-)} \right)\to -\infty\,,\enskip n\to \infty\,,
\label{e7-shn}
\end{equation}
\textit{если основная функция не ограничена снизу}.
\end{enumerate}
     Утверждения~1 и~2, т.\,е.\ соотношения~(6) и~(7), могут выполняться 
как по от\-дель\-ности, так и~со\-вместно.
     
     \smallskip
     
     Доказательство тео\-ре\-мы~3 приведено в~приложении к~на\-сто\-ящей 
работе~[12], а~так\-же  
в~пуб\-ли\-ка\-ции~ [13].
     
     \smallskip
     
     Из теоремы~3 непосредственно следует, что если основная функция 
дроб\-но-ли\-ней\-но\-го интегрального функционала не ограничена сверху или 
снизу, то и~сам этот функционал не ограничен и~решения соответствующей 
экстремальной задачи на максимум или минимум не существует.

\section{Заключение}

     В работе получено исчерпывающее решение задачи безуслов\-но\-го 
экстремума для  
дроб\-но-ли\-ней\-но\-го интегрального функционала, зависящего от 
дополнительного детерминированного па\-ра\-мет\-ра\linebreak
 оптимизации. Установлено, 
что решение по\-став\-лен\-ной экстремальной задачи пол\-ностью опреде\-ляется 
свойствами основной функции данного\linebreak функционала, которая предполагается 
известной. Полученные результаты обобщают со\-от\-вет\-ст\-ву\-ющие утверж\-де\-ния 
о~решении экстремальной задачи для дроб\-но-ли\-ней\-но\-го интегрального 
функционала, у~которого основная функция не зависит от дополнительного 
па\-ра\-мет\-ра оптимизации~[1, 2]. Данные результаты можно применять для 
решения различных прикладных задач, в~которых параметр управ\-ле\-ния имеет 
смешанный характер и~состоит из детерминированной и~стохастической 
компонент.
     
{\small\frenchspacing
 {%\baselineskip=10.8pt
 \addcontentsline{toc}{section}{References}
 \begin{thebibliography}{99}
    
  \bibitem{1-shn}
  \Au{Шнурков П.\,В.} О решении задачи безусловного экстремума для дробно-линейного 
интегрального функционала на множестве вероятностных мер~// Докл. Акад. наук, 2016. 
Т.~470. №\,4. С.~387--392.
  \bibitem{2-shn}
  \Au{Шнурков П.\,В., Горшенин А.\,К., Белоусов~В.\,В.} Аналитическое решение задачи 
оптимального управления полумарковским процессом с~конечным множеством 
состояний~// Информатика и~её применения, 2016. Т.~10. Вып.~4. С.~72--88.
  \bibitem{3-shn}
  \Au{Bajalinov Е.\,В.} Linear-fractional programming. Theory, methods, applications and 
software.~--- Boston--Dordrecht--London: Kluwer Academic Publs., 2003. 423~p.
  
  \bibitem{5-shn} %4
  \Au{Joshi V.\,D., Singh E., Gupta~N.} Primal-dual approach to solve linear fractional 
programming problem~// J.~Appl. Mathematics Statistics Informatics, 2008. Vol.~4. 
Iss.~1. P.~61--69.
  \bibitem{6-shn} %5
  \Au{Hasan M.\,B., Acharjee~S.} Solving LFP by converting it into a~single LP~// Int. 
J.~Operations Research, 2011. Vol.~8. Iss.~3. P.~1--14.

\bibitem{4-shn} %6
  \Au{Borza M., Rambely~A.\,S., Saraj~M.} Solving linear fractional programming problems with 
interval coefficients in the objective function. A~new approach~// Appl. Math. Sciences, 2012. 
Vol.~6. No.\,69-72.  
P.~3443--3452.

  \bibitem{7-shn}
  \Au{Барзилович E.\,Ю., Каштанов~В.\,А.} Некоторые математические вопросы теории 
обслуживания сложных систем.--- М.: Советское радио, 1971. 272~с.
  \bibitem{8-shn}
  Вопросы математической теории надежности~/ Под ред. Б.\,В.~Гнеденко.~--- М.: Радио 
и~связь, 1983. 376~с.
  \bibitem{9-shn}
  \Au{Халмош П.} Теория меры~/ Пер. с~англ. Д.\,А.~Василькова.~--- 
  М.: ИИЛ, 1953. 291~с.
  (\Au{Halmos~P.} {Measure theory}.~--- Graduate texts in mathematics ser.~---
Springer-Verlag New York, 1950. 316~p.)
  \bibitem{10-shn}
  \Au{Ширяев А.\,Н.} Вероятность-1.~--- М.: МЦНМО, 2011. 552~с.
  \bibitem{11-shn}
  \Au{Колмогоров А.\,Н., Фомин~С.\,В.} Элементы теории функций и~функционального 
анализа.~--- М.: Физматлит, 2004. 572~с.
  \bibitem{12-shn}
  \Au{Шнурков П.\,В., Адамова К.\,А.} Приложение к~\mbox{статье}: 
  Решение задачи безусловного 
экстремума для дроб\-но-ли\-ней\-но\-го интегрального функционала, зависящего от параметра, 
2020. 17~с. {\sf http://www.ipiran. ru/publications/publications/supplement.pdf}.
  \bibitem{13-shn}
\Au{Shnurkov P., Adamova K.} Solution of the unconditional extremal problem for 
a~linear-fractional integral functional depending on the parameter~// 
arXiv.org, 2019. arXiv:1906.05824 
[math.OC]. 14~p. {\sf 
https://arxiv.org/abs/1906.05824}.
\end{thebibliography}

 }
 }

\end{multicols}

\vspace*{-3pt}

\hfill{\small\textit{Поступила в~редакцию 15.04.20}}

\vspace*{8pt}

%\pagebreak

%\newpage

%\vspace*{-28pt}

\hrule

\vspace*{2pt}

\hrule

%\vspace*{-2pt}

\def\tit{SOLUTION OF THE UNCONDITIONAL EXTREMAL PROBLEM 
FOR~A~LINEAR-FRACTIONAL INTEGRAL FUNCTIONAL DEPENDENT ON~THE~PARAMETER}


\def\titkol{Solution of the unconditional extremal problem for 
a~linear-fractional integral functional dependent on the parameter}

\def\aut{P.\,V.~Shnurkov$^1$ and K.\,A.~Adamova$^2$}

\def\autkol{P.\,V.~Shnurkov and K.\,A.~Adamova}

\titel{\tit}{\aut}{\autkol}{\titkol}

\vspace*{-9pt}


\noindent
$^1$National Research University Higher School of Economics, 34~Tallinskaya Str., Moscow 123458, 
Russian\linebreak
$\hphantom{^1}$Federation

\noindent
$^2$Academician Pilyugin Center, 1~Vvedenskogo Str., Moscow, 117342, Russian Federation

\def\leftfootline{\small{\textbf{\thepage}
\hfill INFORMATIKA I EE PRIMENENIYA~--- INFORMATICS AND
APPLICATIONS\ \ \ 2020\ \ \ volume~14\ \ \ issue\ 2}
}%
 \def\rightfootline{\small{INFORMATIKA I EE PRIMENENIYA~---
INFORMATICS AND APPLICATIONS\ \ \ 2020\ \ \ volume~14\ \ \ issue\ 2
\hfill \textbf{\thepage}}}

\vspace*{3pt} 

\Abste{The paper is devoted to the study of the unconditional extremal problem for a~fractional linear 
integral functional defined on a~set of probability distributions. In contrast to results proved earlier, the 
integrands of the integral expressions in the numerator and the denominator in the problem under 
consideration depend on a~real optimization parameter vector. Thus, the optimization problem is studied on 
the Cartesian product of a~set of probability distributions and a~set of admissible values of a~real parameter 
vector. Three statements on the extremum
of a~fractional linear integral functional are proved. It is 
established that, in all the variants, the solution of the\linebreak\vspace*{-12pt}}

\Abstend{original problem is completely determined by the 
extremal properties of the test function of the linear-fractional integral functional; this function is the ratio of 
the integrands of the numerator and the denominator. Possible applications of the results obtained to 
problems of optimal control of stochastic systems are described.}

\KWE{linear-fractional integral functional; unconditional extremal problem for a~fractional linear integral 
functional; test function; optimal control problems for Markov and semi-Markov random processes}

\DOI{10.14357/19922264200214}
 

%\vspace*{-20pt}

%\Ack
%\noindent

%\vspace*{6pt}

 \begin{multicols}{2}

\renewcommand{\bibname}{\protect\rmfamily References}
%\renewcommand{\bibname}{\large\protect\rm References}

{\small\frenchspacing
 {%\baselineskip=10.8pt
 \addcontentsline{toc}{section}{References}
 \begin{thebibliography}{99}

\bibitem{1-shn-1}
\Aue{Shnurkov, P.\,V.} 2016. Solution of the unconditional extremum problem for a~linear fractional 
integral functional on a~set of probability measures. \textit{Dokl. Math.} 94(2):550--554.
\bibitem{2-shn-1}
\Aue{Shnurkov, P.\,V., A.\,K.~Gorshenin, and V.\,V.~Belousov.} 2016. Analiticheskoe reshenie zadachi 
optimal'nogo upravleniya polumarkovskim protsessom s~konechnym mnozhestvom sostoyaniy [An analytic 
solution of the optimal control problem for a~semi-Markov process with a~finite set of states]. 
\textit{Informatika i ee Primeneniya~--- Inform. Appl.} 10(4):72--88.
\bibitem{3-shn-1}
\Aue{Bajalinov, Е.\,В.} 2003. \textit{Linear-fractional programming. Theory, methods, applications and 
software}. Boston--Dordrecht--London: Kluwer Academic Publs. 423~p.

\bibitem{5-shn-1} %4
\Aue{Joshi, V.\,D., E. Singh, and N.~Gupta}. 2008. Primal-dual approach to solve linear fractional 
programming problem. \textit{J.~Appl. Mathematics Statistics Informatics} 4(1): 61-69.
\bibitem{6-shn-1} %5
\Aue{Hasan, M.B., and S. Acharjee}. 2011. Solving LFP by converting
 it into a~single LP. \textit{Int. J.~Operations Research} 8(1): 1-14.
 
 \bibitem{4-shn-1} %6
\Aue{Borza, M., A.\,S. Rambely, and M.~Saraj}. 2012. Solving linear fractional programming problems 
with interval coefficients in the objective function. A~new approach. 
\textit{Appl. Math. Sciences}  
6(69-72):3443--3452.

\bibitem{7-shn-1}
\Aue{Barzilovich, E.Yu., and V.A. Kashtanov}. 1971. 
\textit{Nekotorye matematicheskie voprosy teorii 
obsluzhivaniya slozhnykh sistem} [Some mathematical questions in theory of complex systems maintenance]. 
Moscow: Sovetskoe radio. 272~p.
\bibitem{8-shn-1}
Gnedenko, B.\,V., ed. 1983. \textit{Voprosy matematicheskoy teorii nadezhnosti} [Questions of mathematics 
reliability theory]. Moscow: Radio i svyaz'. 376~p.
\bibitem{9-shn-1}
\Aue{Halmos, P.} 1950. \textit{Measure theory}. Graduate texts in mathematics ser.
Springer-Verlag New York. 316~p.
\bibitem{10-shn-1}
\Aue{Shiryaev, A.\,N.} 2011. \textit{Veroyatnost'-1} [Probability-1]. Moscow: MTSNMO. 552~p.
\bibitem{11-shn-1}
\Aue{Kolmogorov, A.\,N., and S.\,V.~Fomin.} 2004. \textit{Elementy teorii funktsiy i funktsional'nogo 
analiza} [Elements of function theory and functional analysis]. Moscow: Fizmatlit. 572~p.
\bibitem{12-shn-1}
\Aue{Shnurkov, P.\,V., and K.\,A.~Adamova.} 2020. Prilozhenie k~stat'e: reshenie zadachi bezuslovnogo 
ekstremuma dlya drobno-lineynogo integral'nogo funktsionala, zavisyashchego ot parametra [Appendix to 
article: Solution of the unconditional extremal problem for a~linear-fractional integral 
functional depending on the parameter]. Available at: {\sf 
http://www.ipiran.ru/publications/\linebreak  publications/supplement.pdf} (accessed April~15, 2020).
\bibitem{13-shn-1}
\Aue{Shnurkov, P., and K. Adamova.} 2019. Solution of the unconditional extremal problem for 
a~linear-fractional integral functional depending on the parameter. 
\textit{arXiv.org}. 14~p. Available at: {\sf 
https://arxiv.org/abs/1906.05824} (accessed April ~5, 2020).

\end{thebibliography}

 }
 }

\end{multicols}

\vspace*{-6pt}

\hfill{\small\textit{Received April 15, 2020}}

%\pagebreak

%\vspace*{-24pt}

\Contr

\noindent
\textbf{Shnurkov Peter V.} (b.\ 1953)~--- Candidate of Science (PhD) in physics and mathematics, associate 
professor, National Research University Higher School of Economics, 34~Tallinskaya Str., Moscow 123458, 
Russian Federation; \mbox{pshnurkov@hse.ru}

\vspace*{3pt}

\noindent
\textbf{Adamova Kseniia A.} (b.\ 1994)~--- engineer, Academician Pilyugin Center, 
1~Vvedenskogo Str., Moscow, 117342, Russian Federation; \mbox{ksenya\_an@mail.ru}
\label{end\stat}

\renewcommand{\bibname}{\protect\rm Литература} 