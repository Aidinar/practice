\newcommand{\vzeta}{\boldsymbol{\zeta}}
\newcommand{\pmbpi}{\boldsymbol{\pi}}

%@@@
\newcommand{\vhz}{{\hat{\mathbf{z}}}}


\def\stat{zeifman}

\def\tit{О ПОДХОДАХ К ПОСТРОЕНИЮ ПРЕДЕЛЬНЫХ РЕЖИМОВ ДЛЯ~НЕКОТОРЫХ МОДЕЛЕЙ 
МАССОВОГО ОБСЛУЖИВАНИЯ$^*$}

\def\titkol{О подходах к~построению предельных режимов для некоторых 
моделей массового обслуживания}

\def\aut{Я.\,А.~Сатин$^1$, А.\,И.~Зейфман$^2$, Г.\,Н.~Шилова$^3$}

\def\autkol{ Я.\,А.~Сатин, А.\,И.~Зейфман, Г.\,Н.~Шилова}

\titel{\tit}{\aut}{\autkol}{\titkol}

\index{Сатин Я.\,А.}
\index{Зейфман А.\,И.}
\index{Шилова Г.\,Н.}
\index{Satin Ya.\,A.}
\index{Zeifman A.\,I.}
\index{Shilova G.\,N.}
 

{\renewcommand{\thefootnote}{\fnsymbol{footnote}} \footnotetext[1]
{Исследования в~разд.~3 и~4 выполнены Я.~Сатиным и~А.~Зейфманом при 
поддержке Российского
научного фонда (проект 19-11-00020.)}}


\renewcommand{\thefootnote}{\arabic{footnote}}
\footnotetext[1]{Вологодский государственный университет, yacovi@mail.ru}
\footnotetext[2]{ Вологодский государственный университет;
Институт проб\-лем информатики Федерального исследовательского центра
<<Информатика и~управление>> Российской академии наук; Вологодский
научный центр Российской академии наук, \mbox{a\_zeifman@mail.ru}}
\footnotetext[3]{Вологодский государственный университет,
shgn@mail.ru}

%\vspace*{-6pt}


\Abst{Рассматриваются нестационарные модели
массового обслуживания, число требований в~которых описывается
конечными марковскими цепями с~периодическими интенсивностями. Для
многих классов таких моделей в~предыдущих статьях разработаны
методы получения верхних оценок скорости сходимости к~предельному
режиму, используя которые можно находить сами предельные
характеристики системы, исследовать их устойчивость по отношению 
к~малым возмущениям интенсивностей поступления и~обслуживания
требований, а~также получать информацию о~том, насколько в~каждый
момент времени текущие характеристики системы отличаются от
предельных. В~настоящей работе изучается другая ситуация, а~именно:
предполагается, что явные оценки скорости сходимости к~предельному
режиму получить не удается. Рассмотрены способы построения
предельных режимов таких систем и~методики получения информации о
 скорости сходимости к~ним. В качестве примера рассмотрена простая
 модель нестационарной системы с~достаточно медленной скоростью 
сходимости к
 предельному режиму.}


\KW{система массового обслуживания; конечные
марковские цепи; периодические интенсивности; предельные
характеристики; скорость сходимости}

\DOI{10.14357/19922264200201} 
 
%\vspace*{9pt}


\vskip 10pt plus 9pt minus 6pt

\thispagestyle{headings}

\begin{multicols}{2}

\label{st\stat}


\section{Введение}

Рассматривается класс нестационарных моделей массового обслуживания,
описываемых конечными неоднородными марковскими цепями 
с~1-пе\-рио\-ди\-че\-ски\-ми интенсивностями. Одной из важнейших задач при 
этом
становится по\-стро\-ение основных предельных характеристик системы.
Проще всего проводить такое построение, если удается получить оценки
скорости схо\-ди\-мости к~предельному режиму. Общие подходы для
получения таких оценок, классы соответствующих моделей, а~также
возможные приложения для изуче\-ния устойчивости и~построения усечений
таких моделей рассмотрены в~работах~[1--7]. Однако иногда такие
оценки получить в~явном виде не удается (см., 
например,~\cite{Markova20}). 
В~этой ситуации обычно или строится решение при
конкретных значениях~$t$ с~заданными начальными условиями (см.,
например,~\cite{aa2018, DiCrescenzo2016})~--- но в~таком случае не
будет информации о~предельном режиме системы; или, наоборот,
изучаются предельные характеристики без обсуждения скорости
сходимости к~ним (см., например,~\cite{Chakravarthy2017})~--- в~этом
случае неизвестно, насколько допустима замена текущих характеристик
системы предельными. В~настоящей статье обсуж\-да\-ют\-ся возможные
подходы к~нахождению предельных характеристик системы в~такой
ситуации.


Число требований $X(t)$ в~таких системах описывается конечной
марковской цепью с~пространством со\-сто\-яний $\{0, 1, \ldots, N\}$
 и~непрерывным временем $ t\hm \ge 0$.

Обозначим через 
\begin{multline*}
p_{ij}(s,t)=\mathrm{Pr}\left\{ X(t)=j\left| X(s)\hm=i\right.
\right\},\\
 i,j \ge 0\,,\enskip 0\leq s\leq t\,,
\end{multline*}
 переходные вероятности
процесса $X\hm=X(t)$, а~через $p_i(t)\hm=\mathrm{Pr}\left\{ X(t) \hm=i \right\}$~---
его вероятности со\-сто\-яний.

Будем обозначать через $\|\bullet\|$ $l_1$-нор\-му вектора 
и~матрицы, т.\,е.\ $\|{\bf x}\|\hm=\sum|x_i|$, $\|B\| \hm= \max\nolimits_j 
\sum\nolimits_i
|b_{ij}|$ при $B \hm= (b_{ij})_{i,j=0}^{N}$, а~через $\Omega$~---
множество всех стохастических векторов, т.\,е.\ множество векторов 
с~неотрицательными координатами и~единичной $l_1$-нор\-мой.


Через $E(t,k) = E\left\{X(t)\left|X(0)=k\right.\right\}$ будем далее
обозначать математическое ожидание процесса (среднее число
требований) в~момент~$t$ при условии, что в~нулевой момент времени
он находится в~со\-сто\-янии~$k$.

Напомним, что марковская цепь~$X(t)$ называется слабо эргодичной,
если $\| {\bf p^*}(t) \hm- {\bf p^{**}}(t) \| \hm\to 0 $ при $t \hm\to
\infty$ для любых начальных условий ${\bf p^*}(s), {\bf p^{**}}(s)$
и любом $ s \hm\ge 0$. Марковская цепь~$X(t)$ имеет предельное среднее
$\phi (t)$, если $E(t,k)\hm - \phi (t) \hm\to 0$ при $t \hm\to \infty$ и
любом~$k$.

\vspace*{-6pt}

\section{Слабая эргодичность конечномерных процессов с~периодической 
матрицей интенсивностей}

\vspace*{-2pt}

\noindent
\textbf{Определение 1.}
Состояние~$i$ для однородной марковской цепи $X(t)$ называется
достижимым, если для любого~$j$ существует~$t$, при котором
выполняется: $p_{ji}(t)\hm>0$.



\smallskip

Пусть $A(t) \hm= \left(a_{ij}(t)\right)_{i,j=1}^N$~--- транспонированная
матрица интенсивностей рассматриваемой цепи, далее будем предполагать,
что все $a_{ij}(t)$ 1-пе\-рио\-дич\-ны и~интегрируемы на $[0,1]$. Введем
<<усредненную>> матрицу
\begin{equation*}
A_0 = \int\limits_{0}^{1} A(t)\,dt\,.
\end{equation*}


\noindent
\textbf{Теорема 1.}
%\label{ch311n} 
\textit{Пусть однородная цепь с~матрицей интенсивностей~$A_0$
имеет достижимое состояние~$i_*$. Тогда марковская цепь~$X(t)$ с
1-пе\-рио\-ди\-че\-ской мат\-ри\-цей интенсивностей~$A(t)$ слабо эргодична,
имеет 1-пе\-рио\-ди\-че\-ские предельный режим~$\pmbpi (t)$ и~предельное
среднее $\phi (t)$, а~также двойное среднее ${\sf E}$, причем
найдется $\alpha \hm>0$ такое, что при всех натуральных $n$ справедливы
следующие неравенства}:
\begin{enumerate}[(1)]
\item \textit{при всех} $t \hm\ge n$
\begin{equation}
\|{\bf p}(t) - \mathbf{p}_{i_*}(t)\| \le \left(1-\alpha\right)^n\|{\bf
p}(0) - \mathbf{p}_{i_*}(0)\| 
\label{ch311}
\end{equation}
\textit{для любого} ${\bf p}(0)$;

\item \textit{ при всех} $t \hm\ge n$
\begin{equation}
|E\left\{X(t)\left|X(0)=k\right.\right\} - \phi (t)| \le 2N
\left(1-\alpha\right)^n 
\label{ch312}
\end{equation}
\textit{для любого}~$k$.
\end{enumerate}


\noindent
Д\,о\,к\,а\,з\,а\,те\,л\,ь\,с\,т\,в\,о\ \ этого утверждения проводится с
помощью интегрирования дифференциальных неравенств, как это показано
в~\cite{zbs} (см.\ рассуждения леммы~1, теоремы~19 для стационарного
случая и~теорем~35 и~36 в~случае периодических интенсивностей.

\smallskip

\noindent
\textbf{Замечание 1.}
Условие существования достижимого состояния у~однородной цепи с
матрицей интенсивностей $A_0$ является и~необходимым для слабой
эргодичности $X(t)$.


\smallskip

\noindent
\textbf{Замечание 2.}
В случае непрерывности интенсивностей и~наличия достижимого при всех
$t$ состояния у~рассматриваемой неоднородной цепи можно
гарантировать справедливость оценок теоремы уже при $n\hm=1$ (см.\
разд.~4).


\smallskip

Следует отметить, что оценки~(\ref{ch311}) и~(\ref{ch312}) 
обычно очень грубые, как показано далее в~примере. Они
только гарантируют существование предельных характеристик, но не
дают реальной возможности их построения, поэтому и~приходится
использовать другие методы.

\vspace*{-6pt}

\section{Построение периодического решения}
%\label{ddddpSectionOne}

\vspace*{-2pt}

Опишем способы получения предельного периодического решения
произвольной марковский цепи $X(t)$, у~которой есть хотя бы одно
достижимое состояние и~интенсивности 1-периодические.


Итак, рассмотрим прямую систему Колмогорова, описывающую вектор
вероятностей состояний процесса~$X(t)$:

\noindent
\begin{equation}
\fr{d\mathbf{p}}{dt}=A(t)\mathbf{p}(t)\,. 
\label{dddpdiffA}
\end{equation}

Пусть для определенности достижимым для усредненной цепи является
нулевое состояние, т.\,е.\ $i^*\hm = 0$. Тогда при некотором $n$ по
теореме~1 найдет\-ся число $\omega \hm\in (0,1)$ такое, что
верно неравенство $\|{\bf p}(n) \hm- \mathbf{p}_{i_*}(n)\|\hm\le \omega$. 
Для
удобства и~простоты вычислений будем считать, что $n\hm=1$, это условие
на практике не будет обременительным.

После подстановки 
$$
p_0(t)=1-\sum\limits_{i\ge 1} p_i(t)
$$ 
переходим к
системе:
\begin{equation}
\fr{d\vz}{dt}=B(t)\vz(t)+\vf(t)\,. 
\label{dddpdiffB}
\end{equation}
%
Данная система имеет решение:
\begin{equation}
\vz(t)=U(t,0)\vz(0)+\int\limits_0^t U(t,s) \vf(s) \, ds\,,
\label{dddpsolvB}
\end{equation}
где $U(t,s)$~--- матрица Коши для соответствующей
однородной системы
\begin{equation*}
\fr{d\mathbf{x}}{dt}=B(t)\mathbf{x}(t)\,. 
%\label{dddpdiffBB}
\end{equation*}

Обозначим через 
$\mathbf{p}^i(t)\hm=(p_{i0},p_{i1},p_{i2},\ldots,p_{in})^{\mathrm{T}}$ 
и~$\vz^i(t\hm)=(z_{i1},z_{i2},\ldots,z_{in})^{\mathrm{T}}\hm=(p_{i1},p_{i2},\ldots,p_{
in})^{\mathrm{T}}$
решения систем (\ref{dddpdiffA}) и~(\ref{dddpdiffB})
соответственно, с~началь-\linebreak\vspace*{-12pt}

\pagebreak

\noindent
ным условием $\mathbf{p}(0)\hm=e_i$ (все 
координаты
нулевые, кроме $i$-й, которая равна единице).

Возьмем начальное условие $e_0$. Для него
$\mathbf{p}(0)\hm=(1,0,0,\ldots,0)^{\mathrm{T}}$ и~$\vz(0)\hm=(0,0,\ldots,0)^{\mathrm{T}}$. 
После
подстановки в~(\ref{dddpsolvB}) получаем:
\begin{equation*}
\vz^0(t)=\int\limits_0^t U(t,s)\vf(s) \, ds.
\end{equation*}

Для начального условия $e_1$ имеем $\mathbf{p}(0)\hm=(0,1,0,\ldots)^{\mathrm{T}}$ и
$\vz^1(0)\hm=(1,0,\ldots)^{\mathrm{T}}$, получаем:
\begin{multline*}
\vz^1(t)=U(t,0)\vz^1(0)+\int\limits_0^t U(t,s)\vf(s) \,
ds={}\\
{}=U(t,0)\vz^1(0)+\vz^0(t).
\end{multline*}

Аналогично для всех $i \hm\le N$ имеем
\begin{equation*}
U(t,0)\vz^i(0)=\vz^i(t)-\vz^0(t)\,,
\end{equation*}
откуда получаем столбцы матрицы $U(t,0)$ и~ее саму в~следующей
форме:
\begin{multline*}
U(t,0)=\left(
\begin{array}{cccc}
z_{11}-z_{01} & z_{21}-z_{01} & z_{31}-z_{01} & \cdots \\
z_{12}-z_{02} & z_{22}-z_{02} & z_{32}-z_{02} & \cdots \\
\vdots & \vdots & \vdots &\vdots\\
z_{1N}-z_{0N} & z_{2N}-z_{0N} & z_{3N}-z_{0N} & \cdots \\
\end{array}\right.\\
\left.\begin{array}{cc}
\cdots& z_{N1}-z_{01}\\  
\cdots& z_{N2}-z_{02}\\  
\vdots& \vdots\\                   
\cdots& z_{NN}-z_{0N}
\end{array}
\right)
\end{multline*}
или
\begin{multline*}
U(t,0)=\left(
\begin{array}{cccc}
p_{11}-p_{01} & p_{21}-p_{01} & p_{31}-p_{01} & \cdots \\ 
p_{12}-p_{02} & p_{22}-p_{02} & p_{32}-p_{02} & \cdots\\  
\vdots & \vdots & \vdots &\vdots\\
p_{1N}-p_{0N} & p_{2N}-p_{0N} & p_{3N}-p_{0N} & \cdots\\  
\end{array}      \right. \\
\left.\begin{array}{cc}
\cdots &p_{N1}-p_{01}\\
\cdots &p_{N2}-p_{02}      \\   
\vdots&\vdots\\
\cdots &       p_{NN}-p_{0N}  
\end{array}
\right).
\end{multline*}
Из сделанного предположения $\|\mathbf{p}^i(1)-\mathbf{p}^0(1)\|\hm\le 
\omega\hm<1$
вытекает, что $\|U(1,0)\|\hm\le \omega<1$. А тогда матрица $I-U(1,0)$
обратима и~из~(\ref{dddpsolvB}) можно {\it в~принципе точно}
выразить начальное состояние, соответствующее периодическому
решению~$\vzeta(t)$:
\begin{equation}
\vzeta(0)=\left(I-U(1,0)\right)^{-1}\int\limits_0^1 U(1,s) \vf(s) \,
ds\,. 
\label{dddpper}
\end{equation}
  %
Однако реальное применение этой формулы затруднительно. Рассмотрим
более простой способ приближенного вычисления~$\vzeta(0)$.


Для разности двух произвольных решений получаем при любых $t \hm\ge s$
\begin{equation*}
\vzeta(t)-\vz(t)=U(t,s)(\vzeta(s)-\vz(s)).
\end{equation*}

Обозначив $\vhz(t)=\vzeta(t)\hm-\vz(t)$, получаем
\begin{equation*}
\vhz(t)=U(t,s)\vhz(s)\,.
\end{equation*}


С учетом 1-периодичности получаем для любого натурального $n$
равенство: 
$$
U(n+1,n)=U(1,0)\,.
$$
 Тогда $\|\vhz(n)\| \le \omega^n
\|\vhz(0)\|$, а~значит, 
$$
\lim\limits_{n\to \infty} \|\vhz(n)\|=0\,.
$$

Следовательно,
\begin{multline*}
\vhz(n+2)-\vhz(n+1)=U(n+2,n+1)\vhz(n+1)- {}\\
{}-
U(n+1,n)\vhz(n)=U(n+1,n)\left(\vhz(n+1)-\vhz(n)\right).
\end{multline*}
 Отсюда
\begin{equation*}
\|\vhz(n+2)-\vhz(n+1)\|\le \omega\|\vhz(n+1)-\vhz(n)\|
\end{equation*}
и для любого натурального~$m$:
\begin{multline*}
\|\vhz(n)-\vhz(m)\|\le \|\vhz(n)-\vhz(n+1)\|+{}\\
{}+ \|\vhz(n+1)-
\vhz(n+2)\|+\cdots 
+\|\vhz(m-1)-\vhz(m)\|
\le {}\\
{}\le
(1+\omega+\omega^2+\cdots)\|\vhz(n)-\vhz(n+1)\|\le{}\\
{}\le \fr
{1}{1-\omega}\|\vhz(n)-\vhz(n+1)\|.
\end{multline*}

\begin{figure*}[b]
\vspace*{1pt}
 \begin{center}
 \mbox{%
 \epsfxsize=110.178mm 
 \epsfbox{Zey-0.eps}
 }
 \end{center}
 \vspace*{-6pt}
 \Caption{Граф процесса}
\end{figure*}
%\end{verbatim}

Устремив $m$ к~бесконечности, получаем
\begin{equation*}
\|\vhz(n)\|\le \fr{1}{1-\omega}\|\vhz(n)-\vhz(n+1)\|.
\end{equation*}
С учетом
\begin{multline*}
\hspace*{-9pt}\|\vhz(n+1)-\vhz(n)\|=\|\vz(n+1)-
\vzeta(n+1)+\vzeta(n)+\vz(n)\|={}\hspace*{-0.46747pt}\\
{}=\|\vz(n+1)-\vz(n)\|
\end{multline*}
имеем:
\begin{multline*}
\|\vz(n)-\vzeta(0)\|\le \fr{1}{1-\omega}\|\vz(n)-\vz(n+1)\| \le{}\\
{}\le
\fr {\omega}{1-\omega}\|\vz(n-1)-\vz(n)\|.
\end{multline*}

Таким образом, зная $\vz(n-1)$ и~$\vz(n)$, можно оценить точность
вычисления~$\vzeta(0)$.

\pagebreak

С учетом неравенства
\begin{multline*}
\|\mathbf{p}(t)-{\boldsymbol{\pi}}(s)\|=|p_0(t)-\pi_0(s)|+\|\vz(t)-
\vzeta(s)\|\le{}\\
{}\le
2\|\vz(t)-\vzeta(s)\|
\end{multline*}
получаем следующее утверждение.

\smallskip

\noindent
\textbf{Теорема 2.}
%\label{thper}
 \textit{Любая марковская цепь $X(t)$ с~1-пе\-рио\-ди\-че\-ской матрицей 
интенсивностей $A(t)$ и~хотя бы одним достижимым состоянием слабо 
эргодична,
 имеет 1-периодический предельный режим $\pmbpi(t)$ и~для нее 
гарантировано выполнение неравенства}:
\begin{equation*}
\|\mathbf{p}(n)-\pmbpi(0)\|\le \fr {2\omega}{1-\omega}\|\mathbf{p}(n-1)-
\mathbf{p}(n)\|,
%\label{dddpth2}
\end{equation*}
\textit{где $\pmbpi(t)$~--- периодическое решение и~$\mathbf{p}(t)$~--- 
произвольное решение
системы}~(\ref{dddpdiffA}).


Таким образом, можно указать несколько способов нахождения
предельного периодического решения.
\begin{enumerate}
\item Зная матрицу Коши $V(1,0)$ системы~(\ref{dddpdiffA}), можно 
получить
начальное условие периодического решения из системы
\begin{align*}
%\left\{
%\begin{array}{c}
V(1,0)\pmbpi(0)&=\pmbpi(0)\,;\\
\|\pmbpi(0)\|&=1\,.
%\end{array}
%\right. 
\end{align*}
\item
Зная матрицу Коши $U(1,0)$ системы~(\ref{dddpdiffB}), можно получить
начальное условие с~помощью формулы~(\ref{dddpper}).
\item
Решать систему~(\ref{dddpdiffA}) на отрезке $[0,n]$ до тех пор, пока
$({2\omega}/({1-\omega}))\|\mathbf{p}(n\hm-1)\hm-\mathbf{p}(n)\|$ не 
будет получено
достаточно точно. В~качестве вектора, приближенно задающего
периодическое решение, можно тогда взять $\mathbf{p}(n)$. Здесь 
$\mathbf{p}(t)$~--- произвольное решение.
\item Из неравенства (\ref{ch311}) можно аналогично предыдущему пункту 
получить следующее правило.
Решать систему~(\ref{dddpdiffA}) на отрезке $[0,n]$ до тех пор, пока
$(({1-\alpha})/{\alpha})\|\mathbf{p}(n\hm-1)\hm-\mathbf{p}(n)\|$ не 
станет меньше значения
требуемой точности. В~качестве вектора, приближенно задающего
периодическое решения, берем $\mathbf{p}(n)$. Здесь $\mathbf{p}(t)$~--- 
решение, которое в~начальный момент находится в~достижимом состоянии.
\item С~помощью оценки скорости сходимости \mbox{найти} целое~$t_*$, после 
которого векторы решений станут практически неразличимы.
Решить сис\-те\-му~(\ref{dddpdiffA}) с~произвольным начальным условием
на отрезке $[0,t_*]$. В~качестве вектора, приближенно задающего
периодическое решение, взять~$\mathbf{p}(t_*)$.
\end{enumerate}

\vspace*{-6pt}

\section{Пример}

\vspace*{-2pt}

Пусть требования поступают на сервер только
при наличии подключения к~интернету с~интенсивностью~$\lambda(t)$.
Если есть подключение к~интернету, требование обслуживается с
интенсивностью~$\mu(t)$. В~случае обрыва соединения с~интернетом
требование не обслуживается. При возобновлении соединения
обслуживание требования начинается заново. Интенсивность обрыва
$\eta(t)$. Интенсивность восстановления соединения~$\nu(t)$. Сервер
обслуживает только одно требование. Требования обслуживаются по
очереди. Очередь считаем конечной.

Граф процесса представлен на рис.~1.



Состояния $2k$ соответствуют ситуации нахождения в~системе $k$
требований и~наличия соединения с~интернетом. Состояния $2k+1$
соответствуют ситуации нахождения в~системе~$k$ требований 
и~отсутствия соединения с~интернетом.

Выпишем матрицу $A(t)$ для данного процесса:
{%\small
\begin{equation*}
\begin{pmatrix}
-(\lambda+\eta) & \nu & \mu & 0 & 0 & \cdots \\
\eta & -\nu & 0 & 0 & 0 & \cdots \\
\lambda & 0 & -(\lambda+\mu+\eta) & \nu & \mu & \cdots \\
0 & 0 & \eta & -\nu & 0 & \cdots \\
0 & 0 & \lambda & 0 & -(\lambda+\mu+\eta) & \cdots \\
0 & 0 & 0 & 0 & \eta & \cdots \\
0 & 0 & 0 & 0 & \lambda & \cdots \\
\vdots & \vdots & \vdots & \vdots & \vdots & \vdots\\
\end{pmatrix}.
\end{equation*}
}


\begin{figure*} %fig2
\vspace*{1pt}
 \begin{center}
 \mbox{%
 \epsfxsize=154.459mm 
 \epsfbox{zey-1.eps}
 }
 \end{center}
\vspace*{-9pt}
\begin{minipage}[t]{80mm}
\Caption{ Математическое ожидание $ E(t, k)$ при $t \hm\in
[0, 40]$ с~начальными условиями $X(0) \hm= 0$~(\textit{1}) и~$29$~(\textit{2})}
\end{minipage}
%\end{figure*}
\hfill
%\begin{figure*} %fig3
\vspace*{-9pt}
\begin{minipage}[t]{80mm}
\Caption{Аппроксимация предельного математического
ожидания $ E(t, 0)$ при $t \hm\in [40, 41]$}
\end{minipage}
\vspace*{12pt}
\end{figure*}

\bigskip

Рассмотрим вначале для наглядности простейшую ситуацию, считая $N=5$.

Пусть интенсивности $\lambda(t)\hm=2\hm+\cos(2\pi t)$; 
$\mu(t)\hm=3\hm+\sin(2 \pi t)$;
$\eta(t)\hm=1\hm+\cos(2 \pi t)$; $\nu(t)\hm=2\hm+\sin(2 \pi t)$.

\pagebreak

Вычислим матрицы Коши за период для сис\-тем~(\ref{dddpdiffA}) 
и~(\ref{dddpdiffB}).
\begin{multline*} 
V(1,0)=
\left(
\begin{array}{cccc}
 0{,}290760 & 0{,}297274 & 0{,}251463 & 0{,}214199\\ 
 0{,}221605 & 0{,}317077 & 0{,}155889 & 0{,}115851 \\
 0{,}202991 & 0{,}178127 & 0{,}210925 & 0{,}212677 \\
 0{,}105084 & 0{,}077259 & 0{,}141829 & 0{,}251418 \\
 0{,}130241 & 0{,}098805 & 0{,}161605 & 0{,}147329 \\
 0{,}049318 & 0{,}031459 & 0{,}078289 & 0{,}058525 
\end{array}\right.\\
\left.
 \begin{array}{cc}   
       0{,}212764 & 0{,}151714\\   
 0{,}116136 & 0{,}072951\\  
 0{,}206426 & 0{,}181198\\  
 0{,}116579 & 0{,}089175\\  
 0{,}204803 & 0{,}241294\\  
 0{,}143292 & 0{,}263669
\end{array}
\right),
\end{multline*}
откуда $ \alpha \ge 0{,}1781$.

Далее,
\begin{multline*} 
U(1,0)=
\left(
\begin{array}{ccc}
0{,}09547 & -0{,}02486 & -0{,}02783\\    
-0{,}06572 & 0{,}00793 & 0{,}03675 \\      
-0{,}10575 & 0{,}00969 & 0{,}14633  \\     
-0{,}10547 & 0{,}00343 & 0{,}01149 \\      
-0{,}14865 & -0{,}02179 & -0{,}01591   
\end{array}   \right.\\
\left.
\begin{array}{cc}
 -0{,}03144 & -0{,}01786\\ 
 0{,}03136 & 0{,}02897\\       
 0{,}01709 & 0{,}00921\\       
 0{,}07456 & 0{,}09397\\       
 0{,}11105 & 0{,}21435
\end{array}
\right),
\end{multline*}
откуда $\|U(1,0)\| =\omega \hm\le 0{,}51176$.

Отметим, что при этом получается 
$$
\fr{1-\alpha}{\alpha}> \fr{2\omega}{1-\omega}\,,
$$
так что в~этой ситуации третий способ приводит к~цели быстрее, чем 
четвертый.

С другой стороны, первый и~второй способы позволяют только найти начальное 
условие, соответствующее
периодическому решению, но не дают возможности оценить скорость 
сходимости к~нему.

Отметим, что для получения оценки 
$$
\|\mathbf{p}^*(t)-\mathbf{p}^{**}(t)\|\le
\varepsilon=10^{-3}
$$ 
достаточно, в~соответствии с~пятым способом, взять $t^* \hm\ge 12$.


Отметим еще, что вектор, дающий начальное условие для предельного 
периодического режима, имеет вид:

\noindent
\begin{multline*}
\vzeta(0) = \left(0{,}250499, 0{,}183664, 0{,}199890, 0{,}126187, \right.\\
\left.0{,}153903, 0{,}085858\right)^{\mathrm{T}}.
\end{multline*}


\medskip

Рассмотрим теперь ситуацию с~большей размерностью. Пусть $N\hm=49$, 
а~интенсивности
$\lambda(t)\hm=2\hm+\cos(2 \pi t)$; $\mu(t)\hm=4\hm+\sin(2 \pi t)$; 
$\eta(t)\hm=1\hm+\cos(2 \pi t)$; $\nu(t)\hm=12\hm+6\sin(2 \pi t)$.

Тогда, находя матрицы Коши систем~(\ref{dddpdiffA}) и~(\ref{dddpdiffB}) 
за первый единичный период, получаем оценки: 
$\alpha \hm< 10^{-5}$, а~$ \omega\hm\le 1,5570$, так что в~этой ситуации
выбрать $n\hm=1$, как это делалось в~начале п.~3, не удается.

При нахождении же матриц Коши $V(10,0)$ и~$U(10,0)$ удается получить
приемлемые оценки: $ \alpha \hm\ge 0{,}1439 $, и~$ \omega\hm\le 0{,}7578 
$,
что позволяет, пользуясь третьим способом, при $t \hm\ge 40$ получить
погрешность для среднего меньше~$0{,}02$.

Графики соответствующих средних (сходимость с~двумя разными начальными 
условиями и~предельное среднее) приведены на рис.~2 и~3.





{\small\frenchspacing
 {%\baselineskip=10.8pt
 \addcontentsline{toc}{section}{References}
 \begin{thebibliography}{99}



\bibitem{Zeifman2017tpa}   %1
\Au{А.\,А.~Зейфман, А.\,В.~Коротышева, В.\,Ю.~Королев, Я.\,А.~Сатин}.
Оценки погрешности аппроксимаций неоднородных марковских цепей 
с~непрерывным временем~// Теория вероятностей и~ее применения, 2017. Т.~61. №\,3. С.~563--569. doi:
10.4213/tvp5073.

\bibitem{Zeifman2018c}     %2
\Au{Zeifman A., Razumchik R., Satin~Y., Kiseleva~K., Korotysheva~A., 
Korolev~V.}
Bounds on the rate of convergence for one class of inhomogeneous 
{M}arkovian queueing
models with possible batch arrivals and services~// Int.
J.~Appl. Math. Comp., 2018. Vol.~28.
No.\,1. P.~66--72.

\bibitem{Zeifman2019amc}     %3
\Au{Zeifman A., Satin Y., Kiseleva~K., Korolev~V., Panfilova~T.}
On limiting characteristics for a~non-stationary two-processor
heterogeneous system~// Appl. Math. Comput., 2019.
Vol.~351. P.~48--65.

\bibitem{Ammar2020}  %4
\Au{Ammar S.\,I., Zeifman A., Satin~Y., Kiseleva~K., Korolev~V.}
On limiting characteristics for a~non-stationary two-processor
heterogeneous system with catastrophes, server failures and repairs~// 
J.~Ind. Manag. Optim., 2020. 
doi: 10.3934/jimo.2020011.

\bibitem{Zeifman2020amcs}  %5
\Au{Zeifman A., Satin Y., Kryukova~A., Razumchik~R., Kiseleva~K.,
Shilova~G.} On the three methods for bounding the rate of
convergence for some continuous-time Markov chains~// Int.
J.~Appl. Math. Comp., 2020. Vol.~30.
P.~1--43. 
{\sf https://arxiv.org/abs/1911.04086}.


\bibitem{Zeifman2020mdpi} 
\Au{Zeifman A., Korolev V., Satin~Y.} Two approaches to the
construction of perturbation bounds for continuous-time Markov
chains~// Mathematics, 2020. Vol.~8. Iss.~2. Art. No.~253. 25~p.
%{\sf https://www.mdpi.com/2227-7390/8/2/253}.


\bibitem{Zeifman2020spl}  %7
\Au{Zeifman A.\,I., Satin Y.\,A., Kiseleva~K.\,M.} On obtaining
sharp bounds of the rate of convergence for a~class of
continuous-time Markov chains~// Stat. Probabil. Lett.,
2020. Vol.~161. Art. ID: 108730.
%https://www.sciencedirect.com/science/article/pii/S016771522030033X



\bibitem{Markova20} %8
\Au{Markova E., Sinitcina A., Satin~Y., Gudkova~I., Samouylov~K., 
Zeifman~A.}
Admission control scheme model described as queuing system with
unreliable servers under licensed shared access~// Mathematics,
 2020 (in press).


\bibitem{DiCrescenzo2016}  %9
\Au{Di Crescenzo A., Giorno V., Nobile~A.\,G.}
Constructing transient birth-death processes by means of suitable
transformations~// Appl. Math. Comput., 2016. Vol.~281. P.~152--171.

\bibitem{aa2018}  %10
\Au{Ammar S.\,I., Alharbi Y.\,F.} Time-dependent
analysis for a~two-processor heterogeneous system with time-varying
arrival and service rates~// Appl. Math. Model., 2018. 
Vol.~54. P.~743--751.

\bibitem{Chakravarthy2017}  %11
\Au{Chakravarthy S.\,R.} A~catastrophic queueing model
with delayed action~// Appl. Math. Model., 2017. Vol.~46. 
P.~631--649.



%\bibitem{Crawford2014} Crawford, F. W., Minin, V. N., Suchard, M. A. 
%2014. Estimation for general birth-death  
%processes // Journal of the American Statistical Association, 109(506), 
%730--747.

%\bibitem{Cruz2017} Cruz, F. R. B., Quinino, R. C., Ho, L. L. 2017. 
%Bayesian estimation of traffic intensity based on  
%queue length in a~multi-server M/M/s queue // Communications in 
%Statistics-Simulation and Computation, 1--13.

%\bibitem{Dong2015} Dong J., Whitt W. 2015. Stochastic grey-box modeling 
%of queueing systems: fitting birth-and-death processes to data // 
%Queueing Systems. 79, 391--426.

%\bibitem{E17} Erlang, A. K. 1917. L{\o}sning af nogle Problemer fra 
%Sandsynlighedsregningen af Betydning for de  
%automatiske Telefoncentraler // Elektroteknikeren. V.13. P. 5--13.

%\bibitem{Ho2017} Ho, L. S. T., Xu, J., Crawford, F. W., Minin, V. N., 
%Suchard, M. A. 2017. Birth/birth-death  
%processes and their computable transition probabilities with biologica%l 
%applications // Journal of Mathematical  
%Biology, 1--34.

%\bibitem{Zhu2016} Zhu, D. M., Ching, W. K., Guu, S. M. 2016. Sufficient 
%conditions for the ergodicity of fuzzy  
%Markov chains // Fuzzy Sets and Systems, 304, 82-93, Guo, Y., Wang, Z. 
%2013. Stability of Markovian jump systems  
%with generally uncertain transition rates. Journal of the Franklin 
%Institute, 350(9), 2826--2836

%\bibitem{dz} Van Doorn, E. A., Zeifman, A. I. 2009. On the speed of 
%convergence to stationarity of the Erlang loss  
%system. Queueing Syst. 63, 241--252.

%\bibitem{Doorn2010} Van Doorn,~E.\,A., Zeifman,~A.\,I., 
%Panfilova,~T.\,L. 2010. Bounds and asymptotics for the rate  
%of convergence of birth-death processes. Th. Probab. Appl. 54, 97--113.

%\bibitem{FRT} Fricker, C., Robert, P., Tibi, D. 1999. On the rate of 
%convergence of Erlang's model. J. Appl.  
%Probab. 36, 1167--1184.

%\bibitem{ki90} Kijima, M. 1990. On the largest negative eigenvalue of 
%the infinitesimal generator associated with  
%$M/M/n/n$ queues. Oper. Res. Let. 9, 59--64.


%\bibitem{voit} Voit, M. 2000. A note of the rate of convergence to 
%equilibrium for Erlang's model in the  
%subcritical case. J. Appl. Probab. 37, 918--923.

%\bibitem{z89} Зейфман А. И. 1989. Некоторые свойства системы с~потерями 
%в~случае переменных интенсивностей //  
%Автоматика и~телемеханика, No~1, c.~107--113.

%\bibitem{z95} Zeifman, A. I. 1995. Upper and lower bounds on the rate of 
%convergence for nonhomogeneous birth and 
%death processes. Stoch. Proc. Appl. 59, 157--173.

%\bibitem{z06} Zeifman, A., Leorato, S., Orsingher, E., Satin, Ya., 
%Shilova, G. 2006. Some universal limits for  
%nonhomogeneous birth and death processes. Queueing systems. 52, 139--
%151.

\bibitem{zbs} 
\Au{Зейфман А.\,И., Бенинг В.\,Е., Соколов~И.\,А.} Марковские цепи 
и~модели с~непрерывным временем.~--- М.: ЭЛЕКС-КМ, 2008. 168~с.


\end{thebibliography}

 }
 }

\end{multicols}

%\vspace*{-12pt}

\hfill{\small\textit{Поступила в~редакцию 16.03.20}}

\vspace*{8pt}

%\pagebreak

%\newpage

%\vspace*{-28pt}

\hrule

\vspace*{2pt}

\hrule

%\vspace*{-2pt}

\def\tit{ON APPROACHES TO CONSTRUCTING LIMITING REGIMES FOR~SOME 
QUEUING MODELS}


\def\titkol{On approaches to constructing limiting regimes for some 
queuing models}

\def\aut{Ya.\,A.~Satin$^1$, A.\,I.~Zeifman$^{1,2,3}$, and G.\,N.~Shilova$^1$}

\def\autkol{Ya.\,A.~Satin, A.\,I.~Zeifman, and G.\,N.~Shilova}

\titel{\tit}{\aut}{\autkol}{\titkol}

\vspace*{-9pt}


\noindent
$^1$Vologda State University, 15~Lenin Str., Vologda 160000, Russian Federation

\noindent
$^2$Institute of Informatics Problems, Federal Research Center ``Computer 
Sciences and Control'' of the Russian\linebreak
$\hphantom{^1}$Academy of Sciences, 
 44-2~Vavilov Str., Moscow 119133, Russian Federation

\noindent
$^3$Vologda Research Center of the Russian Academy of Sciences, 56A~Gorky 
Str., Vologda 160014, Russian\linebreak
$\hphantom{^1}$Federation

\def\leftfootline{\small{\textbf{\thepage}
\hfill INFORMATIKA I EE PRIMENENIYA~--- INFORMATICS AND
APPLICATIONS\ \ \ 2020\ \ \ volume~14\ \ \ issue\ 2}
}%
 \def\rightfootline{\small{INFORMATIKA I EE PRIMENENIYA~---
INFORMATICS AND APPLICATIONS\ \ \ 2020\ \ \ volume~14\ \ \ issue\ 2
\hfill \textbf{\thepage}}}

\vspace*{3pt} 



\Abste{ The authors consider nonstationary queuing models, the number of customers in 
which is described by finite Markov chains with periodic intensities. For many classes of such 
models, the methods of obtaining upper bounds on the rate of convergence to the limiting regime 
were developed in previous papers of the authors. Using these methods, one can find the main 
limiting characteristics of the system, study their stability with respect to small perturbations of 
the arrival and service intensities, and receive information on how current characteristics of the 
system differ from the limiting characteristics at each moment of time. In 
the present paper, the authors 
study a~different situation, namely, it is assumed that explicit estimates of the rate of 
convergence to the limiting regime cannot be obtained. 
The methods for constructing the limiting 
regimes of such systems and for obtaining information on the rate of convergence to 
them are considered. As an example, the authors consider a~simple model of 
 a~nonstationary system with a~rather slow rate of convergence to the limiting regime.}

\KWE{queuing system; finite Markov chains; periodic intensities; limiting characteristics; rate 
of convergence}

\DOI{10.14357/19922264200201} 

%\vspace*{-20pt}

\Ack
\noindent
The results of Sections 3 and 4 were obtained by Ya.\,A.~Satin and 
A.\,I.~Zeifman supported 
by the Russian Science Foundation under grant  
19-11-00020.

%\vspace*{6pt}

 \begin{multicols}{2}

\renewcommand{\bibname}{\protect\rmfamily References}
%\renewcommand{\bibname}{\large\protect\rm References}

{\small\frenchspacing
 {%\baselineskip=10.8pt
 \addcontentsline{toc}{section}{References}
 \begin{thebibliography}{99}


\bibitem{2-zei-1} %1
\Aue{Zeifman, A.\,I., A.\,V. Korotysheva, V.\,Y.~Korolev, and Ya.\,A.~Satin.} 2017. 
Truncation bounds for approximations of inhomogeneous continuous-time Markov chains. 
\textit{Theor. Probab. Appl.} 61(3):513--520.
\bibitem{3-zei-1} %2
\Aue{Zeifman, A., R. Razumchik, Y.~Satin, K.~Kiseleva, A.~Korotysheva, and V.~Korolev.} 
2018. Bounds on the rate of convergence for one class of inhomogeneous Markovian queueing 
models with possible batch arrivals and services. \textit{Int. J.~Appl. 
Math. Comp.} 
28(1):66--72.
\bibitem{4-zei-1} %3
\Aue{Zeifman, A., Ya. Satin, K.~Kiseleva, V.~Korolev, and T.~Panfilova.} 2019. On limiting 
characteristics for a~non-stationary two-processor heterogeneous system. 
\textit{Appl. Math. Comput.} 351:48--65.

\bibitem{1-zei-1} %4
\Aue{Ammar, S.\,I., A. Zeifman, Ya.~Satin, K.~Kiseleva, and V.~Korolev.} 2020. On limiting 
characteristics for a~non-stationary two-processor heterogeneous system with catastrophes, 
server failures and repairs. \textit{J.~Ind. Manag. Optim.} 
doi: 10.3934/jimo.2020011.

\bibitem{5-zei-1}
\Aue{Zeifman, A., Y.~Satin, A.~Kryukova, R.~Razumchik, K.~Kiseleva, and G.~Shilova.} 
2020. On the three methods for bounding the rate of convergence for some continuous-time 
Markov chains. \textit{Int.
J.~Appl. Math. Comp.} 30:1--43.
 Available at: https://arxiv.org/abs/1911.04086 (accessed 
April 28, 2020).
\bibitem{6-zei-1}
\Aue{Zeifman, A., V. Korolev, and Y.~Satin.} 2020. Two approaches to the construction of 
perturbation bounds for continuous-time Markov chains. 
\textit{Mathematics} 8(2):253. 25~p.
\bibitem{7-zei-1}
\Aue{Zeifman, A.\,I., Y.\,A. Satin, and K.\,M.~Kiseleva.} 2020. On obtaining sharp bounds of 
the rate of convergence for a~class of  
continuous-time Markov chains. \textit{Stat. Probabil. Lett.} 161:108730.
\bibitem{8-zei-1}
\Aue{Markova, E., A. Sinitcina, Ya.~Satin, I.~Gudkova, K.~Samouylov, and A.~Zeifman.} 
2020 (in press). Admission control scheme model described as queuing system with unreliable servers 
under licensed shared access. \textit{Mathematics}.

\bibitem{10-zei-1} %9
\Aue{Di Crescenzo, A., V. Giorno, and A.\,G.~Nobile.} 2016. Constructing transient birth-
death processes by means of suitable transformations. \textit{Appl. 
Math. Comput.} 281:152--171.

\bibitem{9-zei-1} %10
\Aue{Ammar, S.\,I., and Y.\,F. Alharbi.} 2018. Time-dependent analysis for a~two-processor 
heterogeneous system with time-varying arrival and service rates. \textit{Appl.
Math. Model.} 54:743--751.

\bibitem{11-zei-1}
\Aue{Chakravarthy, S.\,R.} 2017. A~catastrophic queueing model with delayed action. 
\textit{Appl. Math. Model.} 46:631--649.
\bibitem{12-zei-1}
\Aue{Zeifman, A.\,I., V.\,E. Bening, and I.\,A.~Sokolov.} 2008. \textit{Markovskie tsepi 
i~modeli s~nepreryvnym vremenem} [Markov chains and models with continuous time]. 
Moscow: ELEKS-KM. 168~p.
\end{thebibliography}

 }
 }

\end{multicols}

\vspace*{-6pt}

\hfill{\small\textit{Received March 16, 2020}}

%\pagebreak

%\vspace*{-24pt}


\Contr

\noindent
\textbf{Satin Yacov A.} (b.\ 1978)~--- Candidate of Science (PhD) in physics and 
mathematics, associate professor, Vologda State University, 15~Lenin Str., Vologda 160000, 
Russian Federation; \mbox{yacovi@mail.ru}

\vspace*{3pt}

\noindent
\textbf{Zeifman Alexander I.} (b.\ 1954)~--- Doctor of Science in physics and mathematics, 
professor, Head of Department, Vologda State University, 15~Lenin Str., Vologda 160000, 
Russian Federation; senior scientist, Institute of Informatics Problems, Federal Research Center 
``Computer Sciences and Control'' of the Russian Academy of Sciences, 44-2~Vavilov Str., 
Moscow 119133, Russian Federation; principal scientist, Vologda Research Center of the 
Russian Academy of Sciences, 56A~Gorky Str., Vologda 160014, Russian Federation; 
\mbox{a\_zeifman@mail.ru}

\vspace*{3pt}

\noindent
\textbf{Shilova Galina N.} (b.\ 1961)~--- Candidate of Science (PhD) in physics and 
mathematics, associate professor, Vologda State University, 15~Lenin Str., Vologda 160000, 
Russian Federation; \mbox{shgn@mail.ru}
\label{end\stat}

\renewcommand{\bibname}{\protect\rm Литература} 