\def\stat{krivenko}

\def\tit{ПОСЛЕДОВАТЕЛЬНЫЙ АНАЛИЗ СЕРИЙ ДАННЫХ\\ НА~ОСНОВЕ 
МНОГОМЕРНЫХ РЕФЕРЕНСНЫХ РЕГИОНОВ}

\def\titkol{Последовательный анализ серий данных на~основе 
многомерных референсных регионов}

\def\aut{М.\,П. Кривенко$^1$}

\def\autkol{М.\,П. Кривенко}

\titel{\tit}{\aut}{\autkol}{\titkol}

\index{Кривенко М.\,П.}
\index{Krivenko M.\,P.}
 

%{\renewcommand{\thefootnote}{\fnsymbol{footnote}} \footnotetext[1]
%{Работа выполнена при частичной поддержке РФФИ (проект 19-07-00187-A).}}


\renewcommand{\thefootnote}{\arabic{footnote}}
\footnotetext[1]{Институт проблем информатики Федерального 
исследовательского центра <<Информатика 
и~управление>> Российской академии наук, \mbox{mkrivenko@ipiran.ru}}

%\vspace*{-6pt}

\Abst{Рассматриваются последовательные процедуры анализа серий данных. Развивается 
подход, при котором набор многомерных характеристик некоторого объекта, изменяющийся 
во времени, представляется как единый вектор наблюденных значений. За счет увеличения 
размерности данных удается получить единую картину описания объектов, учесть 
объективно существующую зависимость отдельных наблюдений, смоделировать изменения 
во времени. В~основу решения задач классификации положено использование многомерных 
референсных областей. Предлагаются три варианта процедур обработки данных, 
исследуются их свойства, вырабатываются рекомендации по их применению на практике. 
Для демонстрации возможностей этих процедур рассматривается задача ранней диагностики 
рака с~использованием биомаркера PSA (prostate-specific antigen~---
про\-стат-спе\-ци\-фи\-че\-ский антиген). Указаны особенности применения 
последовательных методов анализа серий данных, сформированы рекомендации по их 
эффективному использованию, выявлены преимущества консолидирующего подхода 
в~анализе данных.}
 
\KW{серии данных; консолидирующий подход; последовательные процедуры; анализ PSA}

\DOI{10.14357/19922264200212}
 
 
\vspace*{6pt}


\vskip 10pt plus 9pt minus 6pt

\thispagestyle{headings}

\begin{multicols}{2}

\label{st\stat}

 
\section{Введение}

\vspace*{-3pt}

     В~[1] рассматривался метод представления результатов наблюдений 
отдельных серий данных в~виде единой совокупности данных. Подобная 
консолидация фрагментированных по времени и~по характеристикам 
наблюдений подключает преимущества обработки многомерных данных, 
а~также позволяет избежать потери общей эффективности статистического 
вывода из-за декомпозиции общей задачи анализа данных при лонгитюдном 
подходе. Проведенные эксперименты продемонстрировали продуктивность 
предлагаемого подхода, полученные при этом результаты создали предпосылки 
для постановки и~решения задачи упреждения результатов обследований. 
     
     Основная цель исследования серий данных~--- описать изменения во 
времени и~охарактеризовать влияющие на них факторы. Желание 
прогнозировать при этом процессы, порождающие серии наблюдений, 
естественно приводит к~постановкам задач последовательного анализа данных. 
Соответствующая экономия статистического материала приобретает особую 
жизненную остроту в~области медицинской диагностики.
     
     Тема последовательного анализа была ини\-ци\-иро\-ва\-на А.~Вальдом~[2] 
в~ответ на возникшую потребность создания более эффективных процедур 
выборочного контроля. Был введен в~рассмотрение и~проанализирован 
последовательный критерий отношения вероятностей (sequential probability 
ratio test~--- SPRT) для двух простых гипотез $H_0:\ f\hm=f_0$ против $H_1:\ 
f\hm= f_1$, использующий последовательность независимых наблюдений $X_1, 
X_2,\ldots ,$ имеющих общую плот\-ность распределения~$f$. Критерий 
прекращает обработку данных на шаге, когда появится первое наблюденное 
значение~$x_n$, при котором $r_n\hm\leq B$ или $r_n\hm\geq A$, где $0\hm< 
B\hm< 1\hm< A$, $r_n\hm= \prod\nolimits^n_{i=1} \left[ f_1(X_i)/f_0(X_i)\right]$. 
По определению $n\hm=\infty$, если $B\hm< r_n\hm<A$ для всех~$n$. 
Критерий SPRT 
принимает гипотезу~$H_0$, если $r_n\hm\leq B$, и~$H_1$, если $r_n\hm\geq 
A$. Была обоснована оптимальность этого критерия в~смысле ожидаемого 
размера используемой выборки, даны приближенные формулы для нахождения 
границ~$A$ и~$B$ при заданных ошибках 1-го и~2-го рода. 
     
     Практически сразу же после появления фундаментальных работ по 
поводу SPRT было признано, что последовательное тестирование гипотез\linebreak 
оказывается полезным инструментом в~биомедицинских исследованиях. 
Впоследствии был предложен производный от SPRT и~его вариантов 
подход~--- <<повторный критерий значимости>> (\mbox{repeated} significance test~--- RST). 
Основная мотивация при создании RST состояла в~том, чтобы в~ходе 
испытаний многократно применять процедуру, эффективную для отдельного 
эксперимента. Однако общий\linebreak уровень значимости~$\alpha^*$, достижимый на 
некоторой стадии обработки данных, мог оказаться существенно больше, 
чем~$\alpha$ для отдельного эксперимента. Поэтому изменения SPRT в~виде 
RST \mbox{включали} в~себя поиск подходящих скорректированных об\-ла\-стей 
принятия решений таким образом, чтобы фактический уровень значимости не 
превышал~$\alpha$.\linebreak В~виде конкретных результатов речь шла об отдельных 
распределениях с~фиксированными значениями параметров. Историческая 
справка о~зарож\-дении последовательных процедур и~их развитии дана, 
например, в~отдельных стать\-ях энциклопедии~[3], обширная биб\-лио\-гра\-фия 
и~представление о~современном состоянии в~об\-ласти последовательного 
анализа данных могут быть почерпнуты из~[4,~5].
     
     Объединить потребности обработки серий многомерных данных 
и~сформировавшиеся на данный момент возможности последовательного 
анализа преимущественно одномерных данных предлагается в~данной статье 
за счет использования многомерных референсных регионов
 (MRR~--- 
multivariate reference region)~[6].
      
     Напомним, если плотность распределения референсных значений есть 
$f(\boldsymbol{y})$, то MMR
 есть область $A_t\hm= \left\{ \boldsymbol{y}\hm\in 
\mathcal{R}^d\vert f(\boldsymbol{y})\hm\geq t\right\}$ для некоторого 
порогового значения~$t$. Для нормального распределения это обычный 
эллипсоид равной вероятности. Если задается вероятность $(1\hm-\alpha_0)$ 
попадания в~$A_t$, то пороговое значение~$t$ есть решение уравнения 
$$
\int\limits_{A_t} f(\boldsymbol{u})\,d\boldsymbol{u}=1-
\alpha_0,
$$
 получить которое аналитически в~случае произвольной плот\-ности 
распределения вряд ли возможно. Здесь присутствуют две проб\-ле\-мы: 
вычисление многомерного интеграла и~зависимость об\-ласти интегрирования 
от неизвестного значения. Для решения их предлагается привлечь метод 
моделирования.
     
     Сгенерируем выборку из~$f(\boldsymbol{y})$, которую обозначим как 
$\mathbf{Y}^f\hm= \left\{ \boldsymbol{y}_1^f, \ldots , 
\boldsymbol{y}_m^f\right\}$. Для оценки~$\int\nolimits_{A_t} 
f(\boldsymbol{u})\,d\boldsymbol{u}$
      используем отношение:
\begin{multline*}
\fr{\vert \{ y_i^f\vert y_i^f\in A_t\}\vert}{m}= \fr{\vert \{ y_i^f\vert 
f(y_i^f)\geq t\}\vert }{m}={}\\
{}=1-\fr{\vert \{ y_i^f\vert f(y_i^f)< t\}\vert }{m}=1-F_m(t)\,,
\end{multline*}
где $F_m(t)$~--- эмпирическая функция распределения случайной 
величины~$f(\boldsymbol{y})$, т.\,е.\ случайной величины, представляющей 
собой результат преобразования с~помощью функции~$f(\cdot)$ случайной 
величины, имеющей плотность распределения~$f(\boldsymbol{u})$.
     
     Таким образом, искомая оценка~$t^*$ должна удовле\-тво\-рять уравнению 
$F_m(t^*)\hm= \alpha_0$ и~может быть получена как непараметрическая 
оценка квантиля порядка~$\alpha_0$ из распределения~$F_m(\cdot)$. Заметим, 
что для такой оценки можно указать доверительный интервал.
     
     Для построения MRR необходимо знать распределение данных. При 
описании области точек высокой плотности можно обратиться 
к~параметрической модели смеси нормальных распределений, имеющей 
плотность распределения
     $$
     f(\boldsymbol{u})=\sum\limits^k_{j=1} 
p_j\varphi\left(\boldsymbol{u},\boldsymbol{\mu}_j, \boldsymbol{\Sigma}_j\right).
     $$ 
Если~$\hat{f}(\boldsymbol{u})$~--- оценка смеси, то~$t^*$ строится 
следующим образом: генерация выборки $\{ \boldsymbol{y}_1^f, \ldots , 
\boldsymbol{y}_m^f\}$ из~$\hat{f}$, подсчет значений 
$\hat{f}(\boldsymbol{y}_i^f)$, непараметрическая оценка квантиля 
порядка~$\alpha_0$.
     
     Использование MRR высокой плотности для диагностирования сводится 
к~реализации так называемого слабого критерия значимости. Для 
наблюденного элемента серии~$\boldsymbol{x}$ нулевая гипотеза~$H_0$ 
заключается в~том, что $\boldsymbol{x}\hm\in A_t$; статистика критерия есть 
$\hat{f}(\boldsymbol{x})$ и~решение о~принадлежности критической 
области~$A_t$ принимается при больших значениях~$\hat{f}(\boldsymbol{x})$. 
Но из-за наличия зависимостей в~сериях получающийся для ее элементов 
набор значений не является, вообще говоря, последовательностью испытаний 
Бернулли. Поэтому применение SPRT должно осуществляться 
с~осторожностью. 

\section{Процедуры последовательного анализа}

     Рассмотрим задачу выбора на основании поступающих данных одного из 
двух решений (диагнозов). Одно из них соответствует нормальному состоянию 
объекта исследования (нулевая гипотеза~$H_0$ и~диагноз <<здоров>>), 
другое~--- отклонению от нормы (конкурирующая гипотеза~$H_1$). Основой 
для построения статистического вывода служит слабый критерий значимости. 
Для задач медицинской диагностики в~этом случае речь идет об использовании 
референсных значений в~виде MRR.
     
     В основе построения MRR лежит модель смеси нормальных 
многомерных распределений, которая может эффективно использоваться как 
средство аппроксимации реальных данных и~при этом быть доступной с~точки 
зрения теоретического анализа. Выбор структурного параметра модели (число 
элементов смеси) предлагается искать, комбинируя применение 
информационных критериев или простого перебора для формирования 
начальных приближений с~последующим уточнением получающихся 
решений~[7]. Принимая во внимание, что статистический вывод строится на 
основе слабого критерия значимости, качество полученных результатов можно 
охарактеризовать через оценку по обучающей выборке значения~$\alpha_0$.
      
     Рассмотрим возможные процедуры последовательного анализа 
отдельных серий.
     
     \textbf{Обычная обработка (процедура R).} В~медицинской 
диагностике распространено простое правило\linebreak
 использования референсных 
значений: ожидать такое первое наблюдение, которое с~этими значениями не 
согласуется. Таким образом, последовательное тестирование завершается 
отклонением~$H_0$,\linebreak если в~пределах серии встретилось событие 
$x_n\hm\notin A_t$, или ее принятием в~противном случае. Недостаток данной 
процедуры заключается в~том, что она приводит к~б$\acute{\mbox{о}}$льшим 
ошибкам 1-го рода, чем в~случае одного наблюдения. Если это существенно, то 
можно занизить уровень значимости при по\-стро\-ении~$A_t$. Кроме того, при 
реализации процедуры~R целесообразно учесть рекомендации~[1]: 
обрабатывать данные в~совокупности, т.\,е.\ перейти от последовательности 
отдельных наблюдений $x_1, x_2, \ldots , x_s$ к~фрагментированным данным 
$\hat{x}_1, \hat{x}_2, \ldots , \hat{x}_{\hat{s}}$, где 
$\hat{x}_i\hm= ( x_{(i-
1)g+1}, x_{(i-1)g+2}, \ldots , x_{(i-1)g+g})$ для $i\hm= 1, \ldots , \hat{s}-1$ 
и~$\hat{x}_{\hat{s}}$ может оказаться с~неполным набором компонент.
     
     \textbf{Обработка по Вальду (процедура W).} Проверки 
принадлежности элементов серии области MRR приводят 
к~последовательности успехов или неуспехов. Если принять предположение, 
что при этом формируется последовательность испытаний Бернулли $b_1,\ldots 
, b_l$ с~вероятностью появления успеха~$p$ то можно обратиться 
к~последовательному анализу Вальда и~построить процедуру, 
обеспечивающую гарантированные ошибки при обработке \mbox{серий}. 
{\looseness=1

}
     
     В этом случае необходимо проверить нулевую гипотезу~$H_0:\ 
p\hm=p_0$ против конкурирующей гипотезы $H_\delta:\ p\hm=p_0\hm+\delta$, 
где $p_0, p_0\hm+\delta \hm\in (0,1)$, с~вероятностями~$\alpha_W$ 
и~$\beta_W$ ошибок 1-го и~2-го рода соответственно, $\alpha_W, \beta_W\hm< 
0{,}5$. Если через~$r_m$ обозначить число успехов (успеху соответствует 
значение~1) в~$m$ испытаниях, т.\,е.\ $r_m\hm= \sum\nolimits^m_{i=1} b_i$, 
$m\hm= 1,\ldots , m_{\max}$,\linebreak то в~соответствии с~общими принципами~[2] 
и~поправкой на ограниченность длины серии (значение~$m_{\max}$ для 
каждой серии свое) процедура SPRT основывается на отношении правдоподобия и~в~случае 
$\delta\hm<0$ включает сле\-ду\-ющие шаги.

\columnbreak

\noindent
     \begin{enumerate}[1.]
\item Положить $m=1$.
\item Если 
$$
r_m \geq \fr{1}{c_\delta}\ln \fr{\beta_W}{1-\alpha_W}-\fr{1}{c_\delta}
 \ln \fr{1-(p_0\hm+\delta)}{1-p_0}\,m\,,
 $$ 
 то принять гипотезу~$H_0$; если 
$$r_m\leq 
\fr{1}{c_\delta}\ln \fr{1-\beta_W}{\alpha_W}-\fr{1}{c_\delta} \ln 
\fr{1-(p_0\hm+\delta)}{1\hm-p_0}\,m\,,
$$ 
то принять гипотезу~$H_\delta$; 
если принята одна из гипотез, то завершить обработку.
\item Если $m=m_{\max}$, то принять гипотезу~$H_0$ и~завершить 
обработку; в~противном случае увеличить~$m$ на единицу и~перейти 
к~шагу~2, т.\,е.\ продолжить обработку серии.
\end{enumerate}
Для $c_\delta $ верно представление 
$$
c_\delta = \fr{\ln((p_0+\delta)(1-p_0))}{p_0(1-(p_0+\delta))}\,.
$$
Изменение постановки задачи привело к~необходимости задания значений 
дополнительных параметров $p_0$, $\delta$, $\alpha_W$ и~$\beta_W$.

\medskip

\textbf{Обработка с~консолидацией (процедура~C).} Две описанные 
выше процедуры основываются на единственном для всех элементов серии 
MRR. Но в~рассматриваемой ситуации, когда регион строится с~помощью 
обучающей выборки (оценивание параметров смеси нормальных 
распределений и~уровня высокой плотности), в~принципе, ничто не мешает на 
каждом шаге последовательной обработки элементов серий опираться на 
различные MRR. Таким образом, обработка 
включает шаги: допустить, что для серии длины~$s$ выполняется~$H_0$, 
далее для $i\hm=1,\ldots ,s$ осуществить проверку и, если$(x_1,\ldots, 
x_i)\hm\notin A_t(\alpha_0,i)$, то отклонить~$H_0$ и~завершить обработку. 
Здесь область $A_t(\alpha_0,i)$ по\-стро\-ена для заданного~$\alpha_0$ 
и~объединенной совокупности данных из~$i$ элементов серии.
     
\section{Эксперименты}

     Исследовалась возможность сокращения объема данных при обработке 
серий измерений уровня PSA при 
раннем диагностировании рака предстательной железы. Для экспериментов 
использовался набор данных~[8]. Результаты его предварительного анализа~[1] 
показали, что целесообразно рас\-смат\-ри\-вать только один маркер~--- общий PSA 
(tPSA~--- total PSA). Для каждого измерения уровня PSA фиксировался промежуток 
времени до установки окончательного диагноза. Всего имелось 
683~наблюдения, каждое из которых включало идентификатор пациента, 
окончательный диагноз (отсутствие или наличие рака предстательной железы, 
далее~--- $D\hm=0/1$), промежуток времени до установки окончательного 
диагноза, уровни tPSA, возраст пациента. Данные распадались на серии 
наблюдений для отдельных пациентов: общее число серий~--- 70 для $D\hm=0$ 
и~71 для $D\hm=1$; длина серий колебалась от~2 до~9 для $D\hm=0$ и~от~1 
до~7 для $D\hm=1$. 
     
     Формирование предварительного диагноза было сведено к~задаче 
обучаемой классификации данных. Для описания данных из класса, 
соответствующего $D\hm=0$, принималась вероятностная \mbox{модель} смеси 
анализаторов главных компонент~[7], включая выбор значений параметров 
чис\-ла элементов смеси. Задача выбора эффективной раз\-мер\-ности для 
отдельных элементов смеси не решалась: на начальном этапе исследования 
возможностей консолидирующих методов анализа серий из-за небольших 
размерностей вектора признаков в~этом не было особой необходимости. 
     
     Применение информационных критериев оказалось малоэффективным, 
так как уже при $g\hm>3$ оценивание итоговой ошибки 1-го рода с~по\-мощью 
обучающей выборки приводило к~большим ошибкам. Поэтому выбор числа 
элементов смеси~$k$ осуществлялся перебором от б$\acute{\mbox{о}}$льших 
значений к~меньшим до удовлетворительных величин разброса оценок ошибки 
1-го рода. Проделывалось это для ряда необходимых значений $\alpha_0$ и~$g$. 
Примером получившихся результатов при $\alpha_0\hm=5\%$ служит набор 
значений: для $g\hm=1$ число элементов смеси $k^*\hm=8$, для $g\hm=2$~--- 
$k^*\hm=5$, для $g\hm=3$~--- $k^*\hm=6$ и~т.\,д.
     
     При реализации процедуры~W необходимо определиться 
с~дополнительными параметрами. Можно предложить следующее: принятые 
в~медицинской диагностике рекомендации приводят к~$p_0\hm=95\%$; 
эксперименты с~построением различных классификаторов серий измерений 
PSA показали, что для диагноза~<<1>> достижимы вероятности принятия 
правильного решения (выход за границы MRR) порядка 60\%, т.\,е.\ $1\hm-
(p_0\hm+\delta)\hm\approx 0{,}6$  
и~$\delta\hm\approx -55\%$, и,~наконец, не противоречат сложившимся 
традициям статистического анализа значения 
$\alpha_W\hm=\beta_W\hm=0{,}1$.
     


Применение описанных процедур практически сразу же выявило 
несоответствие уровней значимости, задаваемых при построении MRR, 
и~итоговых характеристик решающих последовательных правил: с~ростом 
длины серии ошибка 1-го рода увеличивается, что связано с~самой природой 
процедур, для чего надо обратить внимание на точные значения в~столбцах 
табл.~1, озаглавленных как~$\alpha_R$. При этом реальные значения этой 
характеристики (столбцы~$\alpha_R^*$), в~принципе, значимо не 
отличаясь  от 
теоретических, стабильно ниже, чем теоретические (в столбцах $\alpha_E^*$ 
приведены критические уровни\linebreak\vspace*{-12pt}

\columnbreak

{ %\begin{table*}\small %tabl1
%\vspace*{16pt}

\noindent
{{\tablename~1}\ \ \small{Теоретические и~эмпирические вероятности 
отклонения нулевых гипотез для процедуры~R (вероятности $\alpha_R$, 
$\alpha_R^*$ и~$\alpha_E^*$ заданы в~\%)}}

%\vspace*{2ex}

\begin{center}
{\small
\tabcolsep=10.5pt
\begin{tabular}{|c|c|c|c|c|c|c|}
\hline
\multicolumn{1}{|c|}{\raisebox{-6pt}[0pt][0pt]{$l$}}&\multicolumn{3}{c|}{$\alpha_0 =5\%$}&\multicolumn{3}{c|}{$\alpha_0 =10\%$}\\
\cline{2-7}
&$\alpha_R$&$\alpha_R^*$&$\alpha_E^*$&$\alpha_R$&$\alpha_R^*$&$\alpha_E^*$\\
\hline
1&\hphantom{9}5&\hphantom{9}5&99&10&10&73\hphantom{9}\\
2&10&\hphantom{9}6&\hphantom{9}6&19&14&4\\
3&14&10&15&27&18&3\\
4&18&10&\hphantom{9}4&34&21&1\\
5&23&11&\hphantom{9}6&41&21&0\\
\hline
\end{tabular}}
\end{center}
%\end{table*}
}


\vspace*{12pt}

\noindent
 значимости при сравнении теоретической 
и~эмпирической частот). Причина здесь, скорее всего, в~предсказанной 
зависимости элементов серий~[9, разд.~2.5], как теперь оказывается, 
и~в~случае $D\hm=0$. Заметим, что при переходе от $\alpha_0\hm=5\%$ 
к~$\alpha_0\hm=10\%$ значимость отличий возрастает.

В результате пришлось расширить диапазон значений~$\alpha_0$, для 
которых оценивались ошибки классификации, так, чтобы имелась возможность 
сравнивать эффективность различных процедур. Далее\linebreak при изложении 
результатов оставлены значения $\alpha_0\hm=2\%$, 4\%, 6\%,  
8\%, 10\%.
      


 Результаты сравнительного анализа процедур (см.\ рисунок) позволяют 
сделать следующие выводы:
     \begin{itemize}
\item группирование результатов измерений дает ощутимые 
преимущества;
\item управление рассмотренными последовательными процедурами 
может осуществляться через установление априорных требований 
к~используемым MRR;
\item с~точки зрения соотношения ошибок классификации более 
перспективной оказывается процедура~C.
\end{itemize}

{ \begin{center}  %fig1
 \vspace*{6pt}
    \mbox{%
 \epsfxsize=79mm 
\epsfbox{kri-1.eps}
 }

\end{center}

\vspace*{-6pt}

\noindent
 {\small Соотношение ошибок 1-го и~2-го рода для последовательных процедур 
 (\textit{1}~--- R; \textit{2}~--- W; \textit{3}~--- C)
 при обработке отдельных ($g\hm=1$, пунктирные кривые) 
 и~группированных наблюдений ($g\hm=3$, штриховые кривые)}
 }

%\vspace*{6pt}

\pagebreak

{ %\begin{table*}\small %tabl2
\vspace*{0.5pt}

\noindent
\begin{center}
\parbox{46mm}{{{\tablename~2}\ \ \small{Доля ожидаемого объема выборки для последовательных процедур
}}
}

\vspace*{6pt}


{\small
\tabcolsep=11pt
\begin{tabular}{|c|c|c|}
\hline
Процедура &$g$ & $e$\\
\hline
R & \tabcolsep=0pt\begin{tabular}{c} 1\\ 3\end{tabular} & 
\tabcolsep=0pt\begin{tabular}{c} 56\\ 88\end{tabular}\\
\hline
W & \tabcolsep=0pt\begin{tabular}{c} 1\\3\end{tabular} & 
\tabcolsep=0pt\begin{tabular}{c} 52\\ 87\end{tabular}\\
\hline
C & \tabcolsep=0pt\begin{tabular}{c} 1\\ 3\end{tabular} & 
\tabcolsep=0pt\begin{tabular}{c} 56\\ 62\end{tabular}\\
\hline
\end{tabular}}
\end{center}
%\end{table*}
}

\vspace*{10pt}
     
     Есть еще одно важное для практики свойство процедур обработки 
данных~--- возможность сократить объем данных, необходимых для принятия 
решений. Для имеющихся данных, когда каждая серия по результатам 
измерений tPSA и~дополнительных исследований завершалась установлением 
окончательного диагноза, было интересно измерить долю ожидаемого объема 
выборки по отношению к~длине серии~--- величину~$e$. Она характеризует 
прогностические свойства применяемой процедуры статистического вывода: 
чем~$e$ меньше, тем выше упреждение при установлении диагноза $D\hm=1$. 
Значения этой характеристики в~диапазоне значений $\alpha^*\hm\in (1\%, 
30\%)$ оказались практически одинаковыми; для исследуемых тестов 
и~вариантов их применения они приведены в~табл.~2.
     

     
     Полученные результаты говорят о~том, что:
     \begin{itemize}
\item с~точки зрения минимизации ожидаемого объема выборки 
предпочтение надо отдать вариантам процедур с~наименьшим значением 
длины обрабатываемой группы;
\item в~любом случае использования последовательных процедур удается 
заблаговременно предупредить о~возможном значимом отклонении от 
нормы.
\end{itemize}

\section{Заключение}

     Последовательные методы анализа данных решают применительно 
к~задачам медицинской диагностики важнейшую задачу скорейшего 
получения предварительных выводов с~гарантированными свойствами. 
Эффект от подобных действий можно пояснить, переписав фрагмент табл.~2 
в~других единицах измерения: например, для процедуры~R при $g\hm=1$ 
значение $e\hm = 56$ эквивалентно периоду времени в~4,1~года, а~при $g\hm = 
3$ значение $e\hm = 88$ соответствует 1,8~года. 
     
     Построение соответствующих процедур осуществляется на основе 
реальных данных, объем которых, к~сожалению, обычно невелик. Как 
следствие, правила формирования выводов \mbox{относительно} состояния пациента 
принимают форму рекомендаций (вариантов, сценариев) типа: <<если врача 
удовле\-тво\-ря\-ют значения ошибок диагностирования и~время упреждения при 
формировании предположения об окончательном диагнозе, то решение на 
основе данных о~пациенте таково>> (для подтверждения распространенности 
подобного подхода см., например,~[10]).
     
     Дальнейшее развитие методов анализа серий данных в~рамках 
консолидирующего подхода может пойти по пути построения 
соответствующего SPRT, 
ближе всего к~которому лежит предложенная в~данной работе процедура~C.
     
{\small\frenchspacing
 {%\baselineskip=10.8pt
 \addcontentsline{toc}{section}{References}
 \begin{thebibliography}{99}

\bibitem{1-kri}
\Au{Кривенко М.\,П.} Байесовская классификация серий многомерных данных~// 
Системы и~средства информатики, 2020. Т.~30. №\,1. С.~34-45.
\bibitem{2-kri}
\Au{Wald A.} Sequential tests of statistical hypotheses~// Ann. Math. Stat., 
1945. Vol.~16. No.\,2. P.~117--186.
\bibitem{3-kri}
\Au{Armitage P., Colton T.} Encyclopedia of biostatistics: In 8~vols.~--- 2nd ed.~--- 
Hoboken, NJ, USA: Wiley, 2005. 6257~p.
\bibitem{4-kri}
\Au{Bartroff J., Lai T.\,L., Shih M.-C.} Sequential experimentation in clinical trials. Design 
and analysis.~--- New York, NY, USA: Springer, 2013. 237~p.
\bibitem{5-kri}
\Au{Wassmer G., Brannath W.} Group sequential and confirmatory adaptive designs in 
clinical trials.~--- Cham, Switzerland: Springer, 2016. 301~p.
\bibitem{6-kri}
\Au{Кривенко М.\,П.} Многомерный референсный регион высокой плотности~// 
Информатика и~её применения, 2017. Т.~11. Вып.~2. С.~59--64.
\bibitem{7-kri}
\Au{Кривенко М.\,П.} Выбор модели данных в~задачах медицинской диагностики~// 
Информатика и~её применения, 2019. Т.~13. Вып.~4. С.~27--29.
\bibitem{8-kri}
\Au{Thornquist M.\,D., Omenn~G.\,S., Goodman~G.\,E., \textit{et al.}} 
Statistical design and monitoring of the carotene and 
retinol efficacy trial (CARET)~// Control. Clin. Trials, 1993. Vol.~14. Iss.~4. P.~308--
324.
\bibitem{9-kri}
\Au{Fitzmaurice G.\,M., Laird N.\,M., Ware~J.\,H.} Applied longitudinal analysis.~--- 2nd 
ed.~--- Hoboken, NJ, USA: Wiley, 2011. 701~p.
\bibitem{10-kri}
\Au{Etzioni R., Pepe M., Longton G., Hu~C., Goodman~G.} Incorporating the time 
dimension in receiver operating characteristic curves: A~case study of prostate cancer~// 
Med. Decis. Making, 1999. Vol.~19. Iss.~3. P.~242--251.
\end{thebibliography}

 }
 }

\end{multicols}

%\vspace*{-12pt}

\hfill{\small\textit{Поступила в~редакцию 05.03.20}}

%\vspace*{8pt}

%\pagebreak

\newpage

\vspace*{-28pt}

%\hrule

%\vspace*{2pt}

% \hrule

%\vspace*{-2pt}

\def\tit{SEQUENTIAL ANALYSIS OF SERIAL MEASUREMENTS\\ BASED ON~MULTIVARIATE 
REFERENCE REGIONS}


\def\titkol{Sequential analysis of serial measurements based on multivariate 
reference regions}


\def\aut{M.\,P.~Krivenko}

\def\autkol{M.\,P.~Krivenko}

\titel{\tit}{\aut}{\autkol}{\titkol}

\vspace*{-9pt}


\noindent
Institute of Informatics Problems, Federal Research Center ``Computer 
Science 
and Control'' of the Russian Academy of Sciences, 44-2~Vavilov Str., Moscow 
119333, Russian Federation

\def\leftfootline{\small{\textbf{\thepage}
\hfill INFORMATIKA I EE PRIMENENIYA~--- INFORMATICS AND
APPLICATIONS\ \ \ 2020\ \ \ volume~14\ \ \ issue\ 2}
}%
 \def\rightfootline{\small{INFORMATIKA I EE PRIMENENIYA~---
INFORMATICS AND APPLICATIONS\ \ \ 2020\ \ \ volume~14\ \ \ issue\ 2
\hfill \textbf{\thepage}}}

\vspace*{3pt} 

\Abste{ Sequential data series analysis procedures are considered. An approach is 
developed when a~set of multivariate features of a~certain object, which varies in 
time, is presented as a~single vector of observed values. By increasing the 
dimensionality of the data, it is possible to obtain a~single picture of the description 
of objects, to take into account the objectively existing dependence of individual 
observations, and to simulate changes over time. The basis for solving classification 
problems is the use of multivariate reference regions. Three options for data 
processing procedures are proposed, their properties are investigated, and
recommendations for practical application are developed. To demonstrate the 
capabilities of these procedures, the task of early diagnosis of cancer using the 
PSA (prostate-specific antigen) 
biomarker is considered. Features of the application of sequential methods for 
analyzing data series are indicated, recommendations for their effective use are 
formed, and the advantages of the consolidating approach in data analysis are identified.}

\KWE{serial measurements; consolidation approach; sequential procedures; analysis 
of prostate-specific antigen (PSA)}

\DOI{10.14357/19922264200212}
 

%\vspace*{-20pt}

%\Ack
%\noindent


%\vspace*{6pt}

 \begin{multicols}{2}

\renewcommand{\bibname}{\protect\rmfamily References}
%\renewcommand{\bibname}{\large\protect\rm References}

{\small\frenchspacing
 {%\baselineskip=10.8pt
 \addcontentsline{toc}{section}{References}
 \begin{thebibliography}{99}

\bibitem{1-kri-1}
\Aue{Krivenko, M.\,P.} 2020. Bayesovskaya klassifikatsiya seriy mnogomernykh 
dannykh [Bayesian classification of serial multivariate data]. \textit{Sistemy 
i~Sredstva Informatiki~--- Systems and Means of Informatics} 30(1): 34--45.
\bibitem{2-kri-1}
\Aue{Wald, A.} 1945. Sequential tests of statistical hypotheses. \textit{Ann. 
Math. Stat.} 16(2):117--186.
\bibitem{3-kri-1}
\Aue{Armitage, P., and T. Colton.} 2005. 
\textit{Encyclopedia of biostatistics}: In 8~vols. 
2nd ed. Hoboken, NJ: Wiley. 6257~p.
\bibitem{4-kri-1}
\Aue{Bartroff, J., T.\,L. Lai, and M.-C. Shih.} 2013. \textit{Sequential 
experimentation in clinical trials. Design and analysis}. New York, NY: Springer. 
237~p.
\bibitem{5-kri-1}
\Aue{Wassmer, G., and W. Brannath.} 2016. \textit{Group sequential and 
confirmatory adaptive designs in clinical trials.} Cham, Switzerland: Springer. 
301~p.
\bibitem{6-kri-1}
\Aue{Krivenko, M.\,P.} 2017. Mnogomernyy referensnyy region vysokoy plotnosti 
[High-density multivariate reference region]. \textit{Informatika i~ee 
Primeneniya~--- Inform. Appl.} 11(2):59--64.
\bibitem{7-kri-1}
\Aue{Krivenko, M.\,P.} 2019. Vybor modeli dannykh v~zadachakh meditsinskoy 
diagnostiki [Data model selection in medical diagnostic tasks]. \textit{Informatika 
i~ee Primeneniya~--- Inform. Appl.}  
13(4):27--29.
\bibitem{8-kri-1}
\Aue{Thornquist, M.\,D., G.\,S.~Omenn, G.\,E.~Goodman, \textit{et al.}} 1993. Statistical design and monitoring of 
the carotene and retinol efficacy trial (CARET). \textit{Control. Clin. Trials} 
14(4):308--324.
\bibitem{9-kri-1}
\Aue{Fitzmaurice, G.\,M., N.\,M. Laird, and J.\,H.~Ware.} 2011. \textit{Applied 
longitudinal analysis}. 2nd ed. Hoboken, NJ: Wiley. 701~p.
\bibitem{10-kri-1}
\Aue{Etzioni, R, M. Pepe, G.~Longton, C.~Hu, and G.~Goodman.} 1999. 
Incorporating the time dimension in receiver operating characteristic curves: A~case 
study of prostate cancer. \textit{Med. Decis. Making} 19(3):242--251.


\end{thebibliography}

 }
 }

\end{multicols}

\vspace*{-9pt}

\hfill{\small\textit{Received March 5, 2020}}

%\pagebreak

%\vspace*{-24pt}


\Contrl

\noindent
\textbf{Krivenko Michail P.} (b.\ 1946)~--- Doctor of Science in technology, 
professor, leading scientist, Institute of Informatics Problems, Federal Research 
Center ``Computer Science and Control'' of the Russian Academy of Sciences, 
\mbox{44-2}~Vavilov Str., Moscow 119333, Russian Federation; \mbox{mkrivenko@ipiran.ru}

\label{end\stat}

\renewcommand{\bibname}{\protect\rm Литература} 