\def\stat{grusho}

\def\tit{МЕТОДЫ НАХОЖДЕНИЯ ПРИЧИН СБОЕВ В~ИНФОРМАЦИОННЫХ 
ТЕХНОЛОГИЯХ\\ С~ПОМОЩЬЮ МЕТАДАННЫХ$^*$}

\def\titkol{Методы нахождения причин сбоев в~информационных 
технологиях с~помощью метаданных}

\def\aut{Н.\,А.~Грушо$^1$, А.\,А.~Грушо$^2$, М.\,И.~Забежайло$^3$, 
Е.\,Е.~Тимонина$^4$}

\def\autkol{Н.\,А.~Грушо, А.\,А.~Грушо, М.\,И.~Забежайло, 
Е.\,Е.~Тимонина}

\titel{\tit}{\aut}{\autkol}{\titkol}

\index{Грушо Н.\,А.}
\index{Грушо А.\,А.}
\index{Забежайло М.\,И.}
\index{Тимонина Е.\,Е.}
\index{Grusho N.\,A.}
\index{Grusho A.\,A.}
\index{Zabezhailo M.\,I.}
\index{Timonina E.\,E.}
 

{\renewcommand{\thefootnote}{\fnsymbol{footnote}} \footnotetext[1]
{Работа частично поддержана РФФИ (проекты 18-29-03081 и~18-07-00274).}}


\renewcommand{\thefootnote}{\arabic{footnote}}
\footnotetext[1]{Институт проблем информатики Федерального исследовательского 
центра <<Информатика 
и~управление>> Российской академии наук, \mbox{info@itake.ru}}
\footnotetext[2]{Институт проблем информатики Федерального исследовательского 
центра <<Информатика 
и~управление>> Российской академии наук, \mbox{grusho@yandex.ru}}

\footnotetext[3]{Вычислительный центр им.\ А.\,А.~Дородницына Федерального 
исследовательского центра <<Информатика и~управление>> Российской академии наук, 
\mbox{m.zabezhailo@yandex.ru}}
\footnotetext[4]{Институт проблем информатики Федерального исследовательского 
центра <<Информатика 
и~управление>> Российской академии наук, \mbox{eltimon@yandex.ru}}


%\vspace*{-6pt}
   
  \Abst{Статья посвящена вопросам удаленного выявления и~локализации сбоя 
в~информационных системах. Определены информационные ресурсы для решения 
перечисленных выше задач и~исследованы модели использования этих информационных 
ресурсов. В данной статье метаданные описываются естественными для моделей 
биз\-нес-про\-цес\-сов и~информационных технологий (ИТ) ориентированными ациклическими графами
 (DAG~--- directed acyclic graph). 
  Рассматриваемая задача состоит в~следующем. При сбое или ошибке необходимо быстро 
выйти на содержащий причину сбоя блок, который уже не настолько дорог, чтобы его беречь, 
и~он был бы заменяем. Если блок содержит очень дорогие компоненты, например сервер, то его 
замена может быть экономически недоступна организации. Если говорить о~программных 
приложениях, то они могут быть легко реинсталлированы, и~цена такой замены невелика. 
Следовательно, причину надо искать в~направлении сверху вниз по уровням иерархической 
декомпозиции (ИД) DAG информационной технологии.
  Проведено исследование по анализу и~выявлению сбойных данных в~ИТ
   при условии правильности работы всех блоков информационной системы.}
  
  \KW{модели информационных технологий; метаданные; ориентированные ациклические 
графы; причины сбоев и~ошибок в~информационных технологиях} 
  
\DOI{10.14357/19922264200205}
 
 
%\vspace*{9pt}


\vskip 10pt plus 9pt minus 6pt

\thispagestyle{headings}

\begin{multicols}{2}

\label{st\stat}
  

\section{Введение} 
  
  В задачи, решаемые системными администраторами, входит выявление ошибок 
и~сбоев ИТ в~распределенных ин\-фор\-ма\-ци\-он\-но-вы\-чис\-ли\-те\-ль\-ных 
сис\-те\-мах (\mbox{РИВС}). Слож\-ность этих задач привела к~тому, что 
наблюдается дефицит высококвалифицированных системных администраторов.\linebreak 
В~связи с~расширением числа информационных сис\-тем поддержки цифровой 
экономики высококвалифицированные системные администраторы все чаще 
переходят на удаленную работу по обслуживанию информационных систем 
малого и~среднего бизнеса. При этом необходимо удаленно решать следующие 
задачи:
  \begin{itemize}
\item как узнать о~сбое;
\item какую информацию необходимо получить о~деталях сбоя;
\item как определить причину сбоя;
\item как дистанционно осуществить восстановление~ИТ;
\item как обеспечить безопасность дистанционной работы системного 
администратора.
  \end{itemize}
  
  В работе не рассматриваются задачи безопас\-ной связи и~сбор информации 
мониторинга в~контролируемых информационных системах~[1]. В~\mbox{статье} 
рассмотрены вопросы удаленного выявления и~локализации сбоя 
в~информационных системах. Определены информационные ресурсы для 
решения перечисленных выше задач и~исследованы модели использования этих 
информационных ресурсов. 
  
  В работах~[2--4] определены метаданные для управления соединениями 
в~РИВС. Метаданные представляют информацию о~необходимых и~до\-пус\-ти\-мых 
взаимодействиях хостов, на которых\linebreak реализуется ИТ. Метаданные отражают 
логику биз\-нес-про\-цес\-сов и~позволяют дистанционно контролировать ИТ, 
связанные с~их выполнением. В~[5] показано, что метаданные можно также 
использовать для поиска сбоев и~ошибок. Поскольку в~работе не 
рассматриваются сетевые вопросы, то поиск причин сбоев и~ошибок в~SDN 
(software-defined networking)
можно найти в~работе~[6]. 
  
  В отличие от предыдущих работ в~данной статье метаданные описываются 
естественным для моделей 
  бизнес-процессов и~ИТ способом~--- в~виде DAG. 
  
  Причина~$A$ свойства~$P$ в~данных определяется как детерминированное 
появление свойства~$P$ в~данных при появлении в~них причины~$A$~[7]. 
В~этом определении свойство~$P$ называется следствием причины~$A$. 
Исследования формализованных описаний причин, следствий и~связанных 
с~этими понятиями исследований можно найти в~работах В.\,К.~Финна (см., 
например,~[8]). 
  
  Следует отметить, что причина свойства~$P$ не всегда является единственной. 
Например, предположим, что некоторая система выполняет свои функции 
последовательно на устройствах A, B и~C. Сбой работы системы выявляется на 
устройстве~C (свойство~$P$) и~может быть связан с~выходом из строя 
устройства~A или устройства~B. Каждое из этих событий порождает в~системе 
свойство~$P$. Тогда сбои~A и~B являются причинами свойства~$P$. Но 
отсюда не следует, что в~случае появления свойства~$P$ на устройстве~C 
причина однозначно определена. Вместе с~тем уточнение того, какая из причин 
повлияла на сбой в~системе, в~этом примере имеет практическое значение. Если 
можно заменить любое из устройств~A или~B, то экономически целесообразно 
сначала узнать, какое из них вышло из строя. 

\vspace*{-6pt}
  
  \section{Модели графов для~описания информационных технологий} 
  
  \vspace*{-3pt}
  
  В работе рассмотрены ИТ~[9], представимые в~виде DAG. Прописными 
латинскими буквами $A, B,\ldots$ будем обозначать \textit{данные} (объекты), 
служащие входными или выходными данными преобразований информации 
в~ИТ, на рисунках данные будем представлять окружностями. Преобразования 
будем называть \textit{блоками}, обозначать строчными латинскими буквами и~на 
рисунках обозначать прямоугольниками. 
  
  Каждый блок соответствует преобразованию информации и~решает одну или 
несколько задач, необходимых для реализации ИТ. Дуги DAG соответствуют 
передаче данных блокам от предыдущих блоков, т.\,е.\ дуга выходит из выходных 
данных выполненного преобразования и~направляется во входные данные для 
следующего преобразования. 
  
  В ИТ могут быть ориентированные циклы преобразований. Очевидно, что 
входные и~выходные данные каждого цикла отличаются. Следовательно, в~графе 
ИТ ориентированные циклы отсутствуют. Если выходные данные преобразования 
полностью определяют входные данные одного или несколько следующих 
преобразований, то из выходных данных такого преобразования выходят 
несколько дуг, при этом часть выходных данных в~следующем блоке может не 
использоваться. 
  
  При рассмотрении ИД ИТ может возникнуть 
необходимость рас\-смат\-ри\-вать преобразования на уровне операционной
сис\-те\-мы (ОС) и~аппаратной 
платформы. Тогда преобразование верхнего блока можно представить в~виде 
суперпозиции час\-ти преобразования этого блока, блока, связанного 
с~преобразованием информации на уровне ОС и~аппаратной платформы, 
и~оставшейся части преобразования верхнего блока. Повтор преобразований 
допустим, если они имеют разные входные данные. 
  
  Простейший DAG, описывающий произвольную ИТ, имеет вид, 
представленный на рис.~1, где $A$~--- входные данные ИТ; $B$~--- выходные 
данные ИТ; $f$~--- преобразование, реализующее ИТ. 

{ \begin{center}  %fig1
 \vspace*{3pt}
    \mbox{%
 \epsfxsize=45mm 
 \epsfbox{gru-1.eps}
 }
\vspace*{6pt}

\noindent
{{\figurename~1}\ \ \small{Простейший DAG}
}
\end{center}
}

%\vspace*{6pt}



  Если выходные данные~$A$ какого-то преобразования или выходные 
данные~$B$ другого какого-то преобразования используются для 
преобразования~$f$, то входные данные для преобразования~$f$ представляют 
собой вектор $(A, B)$ (рис.~2). 

{ \begin{center}  %fig2
 \vspace*{3pt}
   \mbox{%
 \epsfxsize=79mm 
\epsfbox{gru-2.eps}
 }
\vspace*{6pt}

\noindent
{{\figurename~2}\ \ \small{Векторные входные данные 
}}

\end{center}
}

%\vspace*{6pt}


  Далее потребуется ИД DAG. Иерархическая декомпозиция определяется 
итерационно с~по\-мощью двух операций. 




  \begin{enumerate}[1.]
  \item \textit{Операция деления блока.} Пусть преобразование данных~$f$ 
в~текущем DAG имеет вид, показанный на рис.~1. Предположим, что~$f$ может 
быть представлено в~виде суперпозиции двух преобразований: 
$$
B\hm= f(A)\hm= 
f_2(f_1(A));\enskip  f_1(A)\hm=C\,.
$$ 

Тогда фрагмент DAG на рис.~1 может быть 
графически представлен в~виде, отраженном на рис.~3.
%
  Такое преобразование сохраняет ацикличность исходного графа. 
  
  \end{enumerate}
  
  { \begin{center}  %fig3
 \vspace*{6pt}
    \mbox{%
 \epsfxsize=79mm 
 \epsfbox{gru-3.eps}
 }
\vspace*{6pt}

\noindent
{{\figurename~3}\ \ \small{Операция деления блока
}}

\end{center}
}

\vspace*{3pt}

\begin{enumerate}[1.]
\setcounter{enumi}{1}
  \item \textit{Операция детализации блоков.} Пусть преобразование~$f$ зависит 
от исходных данных $(A, B)$ и~может быть представлено в~виде $C\hm= f(A,B)$. 
Предположим, что существуют функции~$f_1$ и~$f_2$ такие, что $f(A,B)\hm= 
f_2(f_1(A),B)$. Тогда результат этой детализации имеет вид, показанный на 
рис.~4.
  \end{enumerate}
  

\vspace*{3pt}
  
{ \begin{center}  %fig4
 %\vspace*{3pt}
    \mbox{%
 \epsfxsize=79mm 
 \epsfbox{gru-4.eps}
 }
\vspace*{6pt}

\noindent
{{\figurename~4}\ \ \small{Операция детализации блока
}}

\end{center}
}

\vspace*{6pt}

   Очевидно, что операция детализации блоков не порождает ориентированных 
циклов в~по\-рож\-да\-емом графе. 

   Применяя операции деления и~детализации, можно из общего представления 
ИТ (см.\ рис.~1) получить DAG, в~котором блоки соответствуют используемым в~ИТ 
приложениям. Выделим этот этап ИД в~связи с~тем, что создатели приложений, 
как правило, готовят для них хорошие средства тестирования и~описания входных 
и~выходных данных. 
   
  \section{Использование иерархической декомпозиции ориентированных 
  ациклических графов 
для~анализа сбоев и~ошибок}
   
   Рассмотрим этот вопрос на уровне DAG блоков приложений, реализующихся 
в~ИТ. Задача состоит в~следующем. При сбое или ошибке необходимо быстро 
выйти на содержащий причину сбоя блок, который уже не настолько дорог, чтобы 
его беречь, и~он был бы заменяем. Если блок содержит очень дорогие 
компоненты, например сервер, то его замена может быть экономически 
недоступна организации. Если говорить о~программных приложениях, то они 
могут быть легко реинсталлированы, и~цена такой замены невелика. 
Следовательно, причину надо искать в~направлении сверху вниз по уровням 
иерархической декомпозиции DAG ИТ. При этом если выявлены блоки, ставшие 
причиной выхода из строя ИТ, и~их удается заменить или <<отремонтировать>>, 
то апостериорная вероятность выхода из строя таких же блоков в~дальнейшем 
увеличивается. Тогда необходимо готовиться к~тому, что эти блоки, возможно, 
будут выходить из строя и~в~других системах. 
   
   Если все приложения (включая ОС и~про\-граммно-аппаратные платформы) 
работают исправно, то причину сбоя следует искать в~данных. Этот вопрос 
рассмотрен в~следующем разделе. 
  
  \section{Поиск причин сбоя в~данных} 
  
  Сбои могут возникать даже при правильной работе всех приложений 
и~технических систем. Эти сбои связаны с~неправильными входными данными. 
В~ИТ, представленной на рис.~1, сбой данных~$B$ при правильно работающем 
преобразовании~$f$ связан с~неправильными исходными данными~$A$. 
  
  Более детальный анализ причины сбоя в~этой ситуации потребует введения 
новых обозначений и~доказательства ряда утверждений.
  
  Пусть $A$, $B$ и~$f$~--- параметры блока на рис.~1. Обозначим через $D(f)$ 
множество возможных исходных данных, которые могут быть поданы на вход 
преобразования~$f$. Если на данных $A\hm\in D(f)$ преобразование~$f$ не 
может быть вычислено, то $f(A)\hm=0$. Если сбой произошел на предыдущих 
блоках ИТ, то формально положим $0\hm\in D(f)$ и~$f(0)\hm=0$. Таким образом, 
значение~0~--- это очевидно идентифицируемый сбой. 
  
  Пусть $J(f)$~--- это множество выходных данных преобразования~$f$ при 
входных данных, принадлежащих $D(f)$, $0\hm\in J(f)$. Предположим, что 
определена функция~$\varphi(B)$, принимающая значение~~1, когда~$B$ можно 
признать допустимым результатом преобразования~$f$, и~значение~0 
в~противном случае. Тогда $J(f)$ разбивается на два непересекающихся 
подмножества $J^+(f)$ и~$J^-(f)$, $J^+(f)\cap J^-(f)\hm= \emptyset$, 
соответственно допустимых и~недопустимых значений. Недопустимое значение 
означают сбой или ошибку в~ИТ по результатам выполнения преобразования~$f$. 
  
  Разбиение $J(f)\hm= J^+(f)\cup J^-(f)$ индуцирует разбиение множества 
$D(f)\hm= D^+(f)\cup D^-(f)$, где~$D^-(f)$~--- множество тех исходных 
данных~$A$,\linebreak которые преобразуются в~$J^-(f)$. Пусть при выполнении ИТ 
встретился фрагмент DAG, изображенный на рис.~3, когда последовательно 
выполняются преобразования~$f_1$ и~$f_2$, а~выходные\linebreak данные 
преобразования~$f_1$ служат выходными данными преобразования~$f_2$. Тогда 
из недопустимости выходных данных~$B$ следует, что входные данные 
преобразования~$f_2$ принадлежат $D^-(f_2)$. Но тогда выходные данные 
преобразования~$f_1$ должны удовлетворять условию $J^-(f_1)\hm\subseteq D^-
(f_2)$. Иначе данные~$B$~--- до\-пус\-ти\-мы. Это означает, что недопустимость 
данных~$B$ (результата блока~$f_2$) при правильности работы блоков~$f_1$ 
и~$f_2$ требует, чтобы данные $A\hm\in D^-(f_1)$.
  
  Предположим, что в~ИТ выделен фрагмент, изображенный на рис.~2. При 
правильности работы~$f$ и~предыдущих блоков (обозначим их~$f_1$\linebreak и~$f_2$), 
если данные $C\hm\in J^-(f)$, то вектор данных $(A,B)\hm\in D^-(f)$. При этом 
важно, что данные~$A$ и~$B$ могут быть допустимыми для 
преобразований~$f_1$ и~$f_2$, т.\,е.\ помимо недопустимости\linebreak данных~$A$ 
или~$B$ ошибку может инициировать сочетание $(A, B)$. Прообраз данных $(A, 
B)$ находится в~произведении $D(f_1)\times D(f_2)\hm= D((f_1,f_2))$. Отсюда 
$(A,B)\hm\in D^- ((f_1,f_2))$. Справедливо следующее утверждение. 
  
  \smallskip
  
  \noindent
  \textbf{Утверждение 1.}\ \textit{Если все блоки работают правильно, но 
результат данного блока является ошибочным, то такой результат возможен, 
когда исходные данные этого блока представимы в~виде вектора, 
не\-со\-вмес\-ти\-мо\-го с~допустимыми данными~ИТ.} 
  
  Продолжая последовательность приведенных выше рассуждений (возможно, 
с~увеличением размерности данных и~преобразований), получим следующее 
утверждение.
  
  \smallskip
  
  \noindent
  \textbf{Утверждение 2.}\ \textit{В~случае недопустимости выходных данных 
ИТ их причинами могут быть либо сбои и~ошибки в~блоках, либо 
недопустимость исходных данных~ИТ.} 
  
  Смысл этого утверждения состоит в~том, что недопустимость выходных 
данных при правильной работе блоков не может возникнуть из-за 
не\-до\-пус\-ти\-мости данных промежуточных блоков, а~является следствием 
недопустимости входных дан\-ных~ИТ. 
  
  \smallskip
  
  \noindent
  \textbf{Следствие.} Для поиска причин ошибок и~сбоев необходимо иметь 
хорошие тесты для блоков и~две бинарные функции: 
  \begin{enumerate}[(1)]
\item $\varphi$ для определения допустимости выходных данных ИТ;
\item $\psi$ для проверки допустимости входных данных.
  \end{enumerate}
  
  Проведенный анализ позволяет определить источник ошибок в~данных.
  
  \section{Примеры быстрого поиска сбоев блоков и~выявление 
ошибок в~данных}
  
  Поиск причин скрытых ошибок и~сбоев~--- важная функция системного 
администратора информационной системы. Для сложных ИТ или для большого 
количества ИТ метаданные в~виде DAG позволяют выделить сбойные фрагменты 
информационной системы или отдельные блоки, которые необходимо проверить 
на наличие сбоев. Однако определение множеств $D^+(f)$, $D^-(f)$,\linebreak $J^+(f)$ 
и~$J^-(f)$~--- труднореализуемая задача. По\-мимо определения этих множеств 
необходимо строить эффективно вычислимые функции при\-над\-леж\-ности к~этим 
множествам (например, $\varphi$ и~$\psi$). 
  %
  \mbox{Поэтому} можно воспользоваться методом контейнерной 
виртуализации~\cite{10-gr}. 

Пусть выделен хотя бы один сервер, на котором 
можно быстро устанавливать виртуальные контейнеры, эмулирующие 
проверяемые блоки (приложения и~ОС). 
  
  Предположим, что входные и~выходные данные приложений запоминаются 
(например, в~соседних блоках). Тогда можно использовать известный метод 
контроля программ для проверки работы блока. Для этого на эмуляторе 
запускается преобразование контролируемого блока и~сравниваются результаты 
этого преобразования с~полученными ранее результатами. Если они совпадают, то 
можно считать, что блок работает правильно. Если результаты не совпадают, то 
причина этого может быть или в~сбое преобразования, или в~сбое ОС, или 
в~сбое аппаратной платформы. Для каждого из этих компонентов должны быть 
построены свои тесты. При этом целесообразно построить отдельный DAG для 
преобразований в~блоке, позволяющий рассматривать отдельные процессы ИТ. 
  
  Хранение контейнеров возможно в~базе данных ресурсов системного 
администратора, что позволит ему быстро на основании метаданных выявлять 
сбои и~заменять сбойный блок временно работающим виртуальным эмулятором 
(с замыканием на него соответствующих сетевых соединений). 
  
  Метаданные можно также использовать для контроля сетевых 
соединений~\cite{5-gr}. Аналогичный подход был разработан в~\cite{11-gr} для 
контроля критических систем. 
  
  \section{Заключение}
   
   В работе рассмотрена задача поиска и~локализации причин сбоев и~ошибок 
в~ИТ, представимых ИД DAG. Введенные модели 
используют нетрадиционное определение DAG в~том смысле, что дуги графа 
характеризуют связи между входными и~выходными данными различных 
преобразований в~ИТ. 
   
   Построена ИД таких DAG, позволяющая решать 
задачу поиска наименьшей глубины абстракции для экономически осуществимой 
замены программного обеспечения или элементов аппаратной платформы 
информационной системы. 
   
   Проведено исследование по анализу и~выявлению сбойных данных в~ИТ при 
условии пра\-виль\-ности работы всех блоков информационной системы. 
   
{\small\frenchspacing
 {%\baselineskip=10.8pt
 \addcontentsline{toc}{section}{References}
 \begin{thebibliography}{99}
\bibitem{1-gr}
\Au{Грушо Н.\,А.} Метод интеграции многоагентного поиска информации со средствами анализа 
за\-щи\-щен\-ности и~информационными сервисами в~цифровых инфраструктурах~// Проблемы 
информационной безопас\-ности. Компьютерные системы, 2019. №\,2. С.~45--55. 

\bibitem{4-gr} %2
\Au{Грушо А.\,А., Забежайло М.\,И., Грушо~Н.\,А., Тимонина~Е.\,Е.}
 Информационная безопасность на основе\linebreak метаданных в~компонентно-интеграционных 
архитектурах информационных систем~// Системы и~средства информатики, 2018. Т.~28. 
№\,2. С.~34--41.
 
\bibitem{3-gr} %3
\Au{Грушо А.\,А., Тимонина Е.\,Е., Шоргин~С.\,Я.} Иерархический метод порождения 
метаданных для управления сетевыми соединениями~// Информатика и~её применения, 
2018. Т.~12. Вып.~2. С.~44--49. 

\bibitem{2-gr} %4
\Au{Грушо А.\,А., Забежайло М.\,И., Грушо~Н.\,А., Тимонина~Е.\,Е.} Поиск эмпирических 
причин сбоев и~ошибок в~компьютерных системах и~сетях с~использованием 
метаданных~// Системы и~средства информатики, 2019. Т.~29. №\,4. С.~28--38.


\bibitem{5-gr}
\Au{Grusho A., Grusho N., Timonina~E.} Information flow control on the basis of meta data~// 
Distributed Computer and Communication Networks, 22-nd International Conference, DCCN 2019, 
Revised Selected Papers~/ Eds. V.\,M.~Vishnevskiy, K.\,E.~Samouylov, 
D.\,V.~Kozyrev.--- Lecture 
notes in computer science ser.~--- Springer, 2019. Vol.~11965. P.~548--562.
\bibitem{6-gr}
\Au{Grusho A., Grusho N., Zabezhailo~M., Zatsarinny~A., Timonina~E.} Information security of SDN 
on the basis of metadata~// Computer network security~/ Eds. J.~Rak, J.~Bay, I.\,V.~Kotenko, 
\textit{et al.}~--- Lecture notes in computer science ser.~--- Springer, 2017. Vol.~10446.  
P.~339--347.
\bibitem{7-gr}
\Au{Милль Дж. С.} Система логики силлогической и~индуктивной: Изложение принципов 
доказательства в~связи с~методами научного исследования~/ 
Пер. с~англ.~--- 5-е изд., испр. 
и~доп.~--- Сер.\ <<Из наследия мировой философской мысли: Логика>>.~--- 
М.: Ленанд, 2011. 832~с. 
(\Au{Mill~J.\,S.} A~system of logic, ratiocinative and inductive: Being a~connected view of 
the principles of evidence, and the methods of scientific investigation.~--- 
Cambridge library collection~--- philosophy ser.~--- 
Cambridge University Press. Vol.~1.   602~p.)
\bibitem{8-gr}
\Au{Финн В.\,К.} Искусственный интеллект: Методология, применения, философия.~--- М.: 
Красанд, 2011. 448~с.
\bibitem{9-gr}
\Au{Самуйлов К.\,Е., Чукарин А.\,В., Яркина~Н.\,В.} Бизнес-про\-цес\-сы и~информационные 
технологии в~управлении телекоммуникационными компаниями.~--- М.: Альпина 
Паблишерс, 2009. 442~с.
\bibitem{10-gr}
\Au{Кондрашин М.} Контейнерная виртуализация: преимущества и~проблемы 
безопасности~// ИнформКурьер-Связь, 2019. №\,4.  
С.~92--95.
\bibitem{11-gr}
\Au{Шубинский И.\,Б., Николаев В.\,И., Колганов~С.\,К., Заяц~А.\,М.} Активная защита от 
отказов управляющих модульных вычислительных систем.~--- СПб.: Наука, 1993. 284~с.
\end{thebibliography}

 }
 }

\end{multicols}

%\vspace*{-12pt}

\hfill{\small\textit{Поступила в~редакцию 08.04.20}}

\vspace*{8pt}

%\pagebreak

%\newpage

%\vspace*{-28pt}

\hrule

\vspace*{2pt}

\hrule

%\vspace*{-2pt}

\def\tit{METHODS OF~FINDING THE~CAUSES OF~INFORMATION TECHNOLOGY FAILURES BY~MEANS 
OF~METADATA}


\def\titkol{Methods of finding the causes of information technology failures 
by means of metadata}

\def\aut{N.\,A.~Grusho$^1$, A.\,A.~Grusho$^1$, M.\,I.~Zabezhailo$^2$, and 
E.\,E.~Timonina$^1$}

\def\autkol{N.\,A.~Grusho, A.\,A.~Grusho, M.\,I.~Zabezhailo, and E.\,E.~Timonina}

\titel{\tit}{\aut}{\autkol}{\titkol}

\vspace*{-9pt}


\noindent
$^1$Institute of Informatics Problems, Federal Research Center ``Computer 
Science 
and Control'' of the Russian\linebreak
$\hphantom{^1}$Academy of Sciences, 44-2~Vavilov Str., Moscow 
119333, Russian Federation

\noindent
$^2$Dorodnicyn Computing Center, Federal Research Center ``Computer Science 
and 
Control'' of the Russian\linebreak
$\hphantom{^1}$Academy of Sciences, 40~Vavilov Str., Moscow 119333, Russian 
Federation


\def\leftfootline{\small{\textbf{\thepage}
\hfill INFORMATIKA I EE PRIMENENIYA~--- INFORMATICS AND
APPLICATIONS\ \ \ 2020\ \ \ volume~14\ \ \ issue\ 2}
}%
 \def\rightfootline{\small{INFORMATIKA I EE PRIMENENIYA~---
INFORMATICS AND APPLICATIONS\ \ \ 2020\ \ \ volume~14\ \ \ issue\ 2
\hfill \textbf{\thepage}}}

\vspace*{3pt} 


\Abste{The work is devoted to remote detection and localization of failures in information systems. 
Information resources for these tasks have been identified and models for the use of these information 
resources have been investigated. This article describes metadata by firected acyclic graphs (DAG) 
which are used for business processes and information technology descriptions. The task in question is 
as follows. In case of a~failure or an error, it is necessary to find quickly the block containing the 
cause of failure, which is not so expensive that it could\linebreak
\vspace*{-12pt}}

\Abstend{not be replaced, and it would be replaced. If 
a~unit contains very expensive components, such as a~server, replacing it may not be cost-effective 
for an organization. In terms of software applications, they can be easily reinstalled and the cost of 
such a~replacement is low. Therefore, the cause should be sought in the top-down direction along the 
levels of hierarchical decomposition of DAG of information technology. A~study has been carried out 
on analysis and detection of failed data in information technology, provided that all blocks of an 
information system operate correctly.}

\KWE{models of information technologies; metadata; directed acyclic graphs; causes of information 
technology failures and errors}


\DOI{10.14357/19922264200205} 

%\vspace*{-20pt}

\Ack
\noindent
The paper was partially supported by the Russian Foundation for Basic Research 
(projects 18-29-03081 and 18-07-00274).

%\vspace*{6pt}

 \begin{multicols}{2}

\renewcommand{\bibname}{\protect\rmfamily References}
%\renewcommand{\bibname}{\large\protect\rm References}

{\small\frenchspacing
 {%\baselineskip=10.8pt
 \addcontentsline{toc}{section}{References}
 \begin{thebibliography}{99}


\bibitem{1-gr-1}
\Aue{Grusho, N.} 2019. Metod integratsii mnogoagentnogo poiska informatsii so sredstvami analiza 
zashchishchennosti i~informatsionnymi servisami v~tsifrovykh infrastrukturakh [Method of 
integration of multi-agent information search with security analysis tools and information services for 
digital infrastructures]. \textit{Problemy Informatsionnoy Bezopasnosti. Komp'yuternye Sistemy} 
[Information Security Problems. Computer Systems] 2:45--55.

\bibitem{4-gr-1} %2
\Aue{Grusho, A.\,A., M.\,I.~Zabezhailo, N.\,A. Grusho, and E.\,E.~Timonina.}
 2018. Informatsionnaya bezopasnost' na osnove metadannykh v~komponentno-integratsionnykh 
arkhitekturakh informatsionnykh sistem [Information security on the basis of meta data in enterprise 
application integration architecture of information systems]. \textit{Sistemy i Sredstva Informatiki~--- 
Systems and Means of Informatics} 28(2):34--41.

\bibitem{3-gr-1} %3
\Aue{Grusho, A.\,A., E.\,E. Timonina, and S.\,Ya.~Shorgin.} 2018. Ierarkhicheskiy metod 
porozhdeniya metadannykh dlya upravleniya setevymi soedineniyami [Hierarchical method of meta 
data generation for control of network connections]. \textit{Informatika i ee Primeneniya~--- Inform. 
Appl.} 12(2):44--49.

\bibitem{2-gr-1} %4
\Aue{Grusho, A.\,A., M.\,I.~Zabezhailo, N.\,A. Grusho, and E.\,E.~Timonina.} 
2019. Poisk empiricheskikh prichin sboev i~oshibok v~komp'yuternykh sistemakh i~setyakh 
s~ispol'zovaniem metadannykh [Search of empirical causes of failures and errors in computer systems 
and networks using metadata]. \textit{Sistemy i Sredstva Informatiki~--- Systems and Means of 
Informatics} 29(4):28--38.


\bibitem{5-gr-1}
\Aue{Grusho, A.\,A., N.\,A. Grusho, and E.\,E.~Timonina.} 2019. Information flow control on the 
basis of meta data. \textit{Distributed computer and communication networks}.
 Eds. V.\,M.~Vishnevskiy, K.\,E.~Samouylov, and 
D.\,V.~Kozyrev. Lecture notes in computer science ser. Springer. 11965:548--562.
\bibitem{6-gr-1}
\Aue{Grusho, A., N. Grusho, M.~Zabezhailo, A.~Zatsarinny, and E.~Timonina.} 2017. Information 
security of SDN on the basis of metadata. \textit{Computer network security}. Eds. J.~Rak, J.~Bay, 
I.\,V.~Kotenko, \textit{et al.} Lecture notes in computer science ser. Springer. 10446:339--347.
\bibitem{7-gr-1}
\Aue{Mill, J.\,S.} 2011. \textit{A system of logic, ratiocinative and inductive: Being a~connected view of 
the principles of evidence, and the methods of scientific investigation}. 
Cambridge library collection~--- philosophy ser. 
Cambridge University Press. Vol.~1.   602~p.
\bibitem{8-gr-1}
\Aue{Finn, V.\,K.} 2011. \textit{Iskusstvennyy intellekt: Metodologiya, primeneniya, filosofiya} 
[Artificial intelligence: Methodology, applications, philosophy]. Moscow: Krasand. 448~p.
\bibitem{9-gr-1}
\Aue{Samuylov, K.\,E., A.\,V. Chukarin, and N.\,V.~Yarkina.} 2009. 
\textit{Biznes-protsessy 
i~informatsionnye tekhnologii v~upravlenii telekommunikatsionnymi kompaniyami} [Business 
processes and information technologies in management of the telecommunication companies]. 
Moscow: Alpina Publs. 442~p.
\bibitem{10-gr-1}
\Aue{Kondrashin, M.} 2019. Konteynernaya virtualizatsiya: preimushchestva i problemy bezopasnosti 
[Container virtualization: advantages and security problems]. \textit{InformKur'er-Svyaz'} 
[InformCourier-Communications] 4:92--95.
\bibitem{11-gr-1}
\Aue{Shubinsky, I.\,B., V.\,I.~Nikolaev, S.\,K.~Kolganov, and A.\,M.~Zayats.} 1993. \textit{Aktivnaya 
zashchita ot otkazov upravlyayushchikh modul'nykh vychislitel'nykh system} [Active protection against 
failures of control modular computing systems]. St.\ Petersburg: Nauka. 284~p.

\end{thebibliography}

 }
 }

\end{multicols}

\vspace*{-9pt}

\hfill{\small\textit{Received April 8, 2020}}

%\pagebreak

%\vspace*{-24pt}



\Contr

\noindent
\textbf{Grusho Nikolai A.} (b.\ 1982)~--- Candidate of Science (PhD) in physics and mathematics, 
senior scientist, Institute of Informatics Problems, Federal Research Center ``Computer Sciences and 
Control'' of the Russian Academy of Sciences; 44-2~Vavilov Str., Moscow 119133, Russian 
Federation; \mbox{info@itake.ru}

%\vspace*{3pt}

\noindent
\textbf{Grusho Alexander A.} (b.\ 1946)~--- Doctor of Science in physics and mathematics, professor, 
principal scientist, Institute of Informatics Problems, Federal Research Center ``Computer Sciences 
and Control'' of the Russian Academy of Sciences; 44-2~Vavilov Str., Moscow 119133, Russian 
Federation; \mbox{grusho@yandex.ru}

\vspace*{6pt}

\noindent
\textbf{Timonina Elena E.} (b.\ 1952)~--- Doctor of Science in technology, professor, leading scientist, 
Institute of Informatics Problems, Federal Research Center ``Computer Sciences and Control'' of the 
Russian Academy of Sciences; 44-2~Vavilov Str., Moscow 119133, Russian Federation; 
\mbox{eltimon@yandex.ru}


\vspace*{6pt}

\noindent
\textbf{Zabezhailo Michael I.} (b. 1956)~--- Doctor of Science in physics and mathematics, principal scientist, 
Dorodnicyn Computing Centre, Federal Research Center ``Computer Science 
and Control'' of the Russian 
Academy of Sciences, 40 Vavilov Str., Moscow 119333, Russian Federation; m.zabezhailo@yandex.ru

\label{end\stat}

\renewcommand{\bibname}{\protect\rm Литература} 