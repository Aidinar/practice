\newcommand{\me}[2]{\mathbf{E}_{ #1 }\left\{ \mathop{#2} \right\} }




%\newcommand {\s}{^{(s)}}
%\newcommand {\g}{\gamma}

\def\stat{borisov}

\def\tit{ЧИСЛЕННЫЕ СХЕМЫ ФИЛЬТРАЦИИ МАРКОВСКИХ СКАЧКООБРАЗНЫХ ПРОЦЕССОВ 
ПО~ДИСКРЕТИЗОВАННЫМ НАБЛЮДЕНИЯМ~III:~СЛУЧАЙ МУЛЬТИПЛИКАТИВНЫХ ШУМОВ$^*$}

\def\titkol{Численные схемы фильтрации марковских скачкообразных 
процессов по %дискретизованным наблюдениям~III:~случай мультипликативных 
шумов}

\def\aut{А.\,В.~Борисов$^1$}

\def\autkol{А.\,В.~Борисов}

\titel{\tit}{\aut}{\autkol}{\titkol}

\index{Борисов А.\,В.}
\index{Borisov A.\,V.}
 

{\renewcommand{\thefootnote}{\fnsymbol{footnote}} \footnotetext[1]
{Работа выполнена при частичной поддержке РФФИ (проект 19-07-00187~А).}}


\renewcommand{\thefootnote}{\arabic{footnote}}
\footnotetext[1]{Институт проблем информатики Федерального 
исследовательского центра <<Информатика 
и~управление>> Российской академии наук, \mbox{aborisov@frccsc.ru}}

\vspace*{-12pt}

\Abst{Статья завершает цикл исследований, начатых в~работах <<Численные 
схемы фильтрации марковских скачкообразных процессов по дискретизованным 
наблюдениям~I:~ характеристики точ\-ности>> и~<<Численные схемы фильтрации 
марковских скачкообразных процессов по дискретизованным 
наблюдениям~II:~случай аддитивных шумов>>.
На основании представленных ранее теоретических результатов разработан 
алгоритм численной реализации задачи фильтрации состояний однородных 
марковских скачкообразных процессов (МСП) по косвенным непрерывным зашумленным 
наблюдениям, дискретизованным по времени.
Класс систем наблюдения ограничен системами с~\textit{мультипликативными} 
винеровскими шумами: аддитивная полезная составляющая в~наблюдениях 
отсутствует, а~интенсивность шумов является функцией оцениваемого 
состояния.
Для вычисления интегралов, присутствующих в~оценках, использовался 
составной вариант численной схемы <<средних>> прямоугольников порядка 
точ\-ности~$3$ для вычисления одномерных интегралов, а~также формула 
среднего порядка~$4$ для интегрирования по треугольнику. В~итоге были 
получены численные схемы порядка точности~$1$ и~$2$.}

\KW{марковский скачкообразный процесс; оптимальная фильтрация; 
мультипликативные шумы в~наблюдениях; стохастическое дифференциальное 
уравнение; аналитическая и~численная аппроксимация}

\DOI{10.14357/19922264200202} 
 
%\vspace*{9pt}


\vskip 10pt plus 9pt minus 6pt

\thispagestyle{headings}

\begin{multicols}{2}

\label{st\stat}


 \section{Введение}

Работа завершает цикл статей~\cite{B_20_1_IA, B_20_2_IA}. Она посвящена 
созданию и~сравнительному анализу алгоритмов численного решения задач 
оптимальной фильтрации состояний однородных МСП
 по дискретизованным косвенным наблюдениям 
с~мультипликативными винеровскими шумами. Наличие шумов такого рода 
означает, что их интенсивность является случайной и~зависит от 
оцениваемого со\-сто\-яния. Тео\-ре\-ти\-че\-ские основы решения такой задачи 
заложены в~\cite{B_18, B_18_IA}. Так как оценка оптимальной фильтрации 
имеет почти наверное неотрицательные компоненты и~удовлетворяет условию 
нормировки, то пред\-став\-лен\-ные в~данном цикле алгоритмы чис\-лен\-но\-го решения 
задачи фильтрации сохраняют для своих реализаций указанные свойства 
и~названы за это \textit{устойчивыми}.
Точность оценок зависит от шага дискретизации наблюдений 
и~\textit{порядка~$s$ аналитической аппроксимации}~--- числа возможных скачков 
оцениваемого состояния на интервале дискретизации, учитываемого 
в~алгоритме.


Статья организована следующим образом. 
%
Раздел~2 содержит постановку 
задачи оптимальной фильтрации по дискретизованным наблюдениям, а~также 
сведения из~\cite{B_20_1_IA, B_20_2_IA}, необходимые для ее чис\-лен\-но\-го 
решения. 

В~разд.~3 для систем с~мультипликативными шумами разработаны 
численные схемы фильтрации, соответствующие аналитическим аппроксимациям 
порядка $s\hm=1$ и~$2$.
Для численного интегрирования при вычислении аппроксимации порядка 
$s\hm=1$ выбрана составная схема <<средних>> прямоугольников, а~для 
$s\hm=2$~--- та же схема прямоугольников для вычисления одномерных 
интегралов, и~составной вариант метода средних~--- для вычисления 
двумерных интегралов по треугольникам. 
Полученные в~результате чис\-лен\-ные 
аппроксимации имеют порядки~$1$ и~$2$. 

Заключительные замечания 
представлены в~разд.~4.

%\vspace*{-16pt}
 
\section{Оценка фильтрации и~ее~аппроксимации}

%\vspace*{-6pt}

 На триплете с~фильтрацией $(\Omega^X \times \Omega^W,\mathcal{F}^X 
\hm\times \mathcal{F}^W,
 \mathcal{P}^X \hm\times \mathcal{P}^W, \{\mathcal{F}^X_t \times 
\mathcal{F}^W_t\}_{t \geqslant 0})$ рассматривается система наблюдения

%\pagebreak

\noindent
\begin{equation}
 \left.
 \begin{array}{ll}
\displaystyle
X_t =X_0 + \int\limits_0^t \Lambda^{\top}X_{s}ds + \mu_t\,; & \\[6pt]
 \displaystyle Y_r = \int\limits_{(r-1)h}^{rh}\!\! fX_sds+ 
\int\limits_{(r-1)h}^{rh} \sum\limits_{n=1}^NX_s^ng_n^{1/2} dW_s, &\\[6pt]
 &\hspace*{-10mm}r \in \mathbb{N},
 \end{array}
 \right\}
 \label{eq:obsys_1}
 \end{equation}
 где
 % \item[]
 $X_t \triangleq \mathrm{col}\left(X_t^1,\ldots,X_t^N\right) \in 
\mathbb{S}^N$~--- не\-на\-блю\-да\-емое состояние системы~--- однородный МСП 
с~конечным множеством состояний $ \mathbb{S}^N \triangleq 
\{e_1,\ldots,e_N\}$ ($\mathbb{S}^N$~--- множество единичных векторов 
евклидова пространства $\mathbb{R}^N$), матрицей интенсивностей\linebreak переходов 
$\Lambda$ и~начальным распределением $\pi$;
 %\item[]
 $\mu_t \hm\triangleq \mathrm{col}\left(
 \mu_t^1,\ldots,\mu_t^N\right)\hm\in \mathbb{R}^N$~--- 
$\mathcal{F}_t^X$-со\-гла\-со\-ван\-ный мартингал;
% \item[]
 $\{Y_r\}_{r \in \mathbb{N}}:\; Y_r \hm\triangleq 
\mathrm{col}\left(Y_r^1,\ldots,Y_r^M\right) \hm\in \mathbb{R}^M$~--- 
последовательность дискретизованных наблюдений, доступных в~известные 
равноотстоящие\linebreak моменты времени $\{rh\}_{r \in \mathbb{N}}$;
$W_t \triangleq \mathrm{col}(W_t^1,\ldots\linebreak \ldots ,W_t^M) \hm\in \mathbb{R}^M$ 
является
$\mathcal{F}_t^W$-со\-гла\-со\-ван\-ным стандартным винеровским процессом;
 $f$~--- $(M \times N)$-мер\-ная мат\-ри\-ца;
 $\{g_n\}_{n=\overline{1,N}}$~--- симметричные положительно определенные 
матрицы; процессы~$X$ и~$W$ независимы.
 %\end{itemize}

 \textit{Задача оптимальной фильтрации состояния $X$ по дискретизованным 
наблюдениям~$Y$} заключается в~нахождении условного математического 
ожидания
 \begin{equation*}
 \widehat{X}_r \triangleq \me{}{X_{t_r}|\mathcal{O}_{r} },
 %\label{eq:fest_1}
 \end{equation*}
 где $\mathcal{O}_r \triangleq \sigma\{ Y_{\ell}: \; 1 \leqslant \ell 
\leqslant r\}$~--- $\sigma$-ал\-геб\-ра, по\-рож\-ден\-ная наблюдениями, 
полученными до момента времени $rh$ включительно; $\mathcal{O}_0 
\triangleq \{\varnothing,\; \Omega\}$.

 Пусть $N_r^X(\omega)$~--- число скачков процесса~$X$,\linebreak произошедших на 
отрезке $[(r-1)h,rh]$, $\tau_r\hm \triangleq \int\nolimits_0^t X_s\,ds$~--- 
случайный вектор времени пребывания~$X$ в~различных состояниях на  
$[(r-1)h,rh]$, а
 $\rho^{n,j,m}(\cdot)$~--- распределение вектора
 $\tau_{r}X_{t_{r}}^{j}\mathbf{I}_{\{m\}}(N_{r}^X)$ при условии $X_{t_{r-
1}}\hm=e_k$. Это означает, что
 для любого $\mathcal{G} \hm\in \mathcal{B}(\mathbb{R}^M)$ верно 
равенство
 $$
\me{}{\mathbf{I}_{\mathcal{G}}\left(\tau_r\right)X_{t_r}^j
\mathbf{I}_{\{m\}}(N_r^X)|X_{t_{r-1}}=e_n}
=
 \hspace*{-3pt}\int\limits_{\mathcal{G}}\hspace*{-3pt} \rho^{n,j,m}(du).
 $$

\textit{Аналитическая аппроксимация $\overline{X}_r$ порядка~$s$} 
опре\-де\-ля\-ет\-ся рекурсивной схемой
 \begin{equation}
\overline{X}_r = (\mathbf{1}\xi_{r}^{\top}\overline{X}_{r-1})^{-1} 
\xi_{r}^{\top}\overline{X}_{r-1}, \enskip r \geqslant 1, \enskip 
\overline{X}_0=\pi,\!\!
 \label{eq:filt_3}
 \end{equation}
 где $\mathbf{1} = \mathrm{row}\, (1,\ldots,1)$~--- вектор-строка 
подходящей размерности;
$\xi_q \triangleq \|\xi^{ij}(Y_q)\|_{i,j=\overline{1,N}},\; q \hm\in 
\mathbb{N}$~--- $(N \times N)$-мер\-ные случайные матрицы~--- функции 
наблюдений~$Y_q$:
 \begin{equation} %\textstyle
 \xi^{ij}(y)\triangleq
\sum\limits_{m=0}^s \int\limits_{\mathcal{D}}
 \mathcal{N}\left(y,f u,\sum_{p=1}^N u^p g_p\right)
 \rho^{i,j,m}(du).
 \label{eq:xi_def}
 \end{equation}
 В~(4) 
 $\mathcal{N}(y,m,K) \triangleq (2\pi)^{-M/2} \mathrm{det}^{-1/2} K \times\linebreak
 \times
\exp\left\{ -({1}/{2})\|y-m)\|^2_{K^{-1}}\right\}
 $~--- это $M$-мер\-ная\linebreak плот\-ность гауссовского распределения с~математическим 
ожиданием~$m$ и~невырожденной ковариационной матрицей $K$; $\mathcal{D} 
\triangleq \{u \hm\in \mathbb{R}^M_+:\; \mathbf{1}u=h\}$.

 Из построения оценки $\overline{X}_r$~(\ref{eq:filt_3}) следует, что она 
обладает свойством устойчивости.

 Интегралы $\xi^{ij}(y)$ приближенно вычисляются в~виде сумм:
 \begin{align*}
%\left.
% \begin{array}{rl}
 \!\!\xi^{ij}(y) &\displaystyle
\approx \psi^{ij}(y) \triangleq
 \sum\limits_{\ell=1}^{L} \mathcal{N}\!\left( \! y,f 
w_{\ell},\sum\limits_{p=1}^N w^p_{\ell} g_p \!\right)\varrho_{\ell}^{ij}; \\
\! \!\psi(y) &\triangleq \|\psi^{ij}(y)\|_{i,j=\overline{1,N}}, 
% \quad \psi_q % 
%\triangleq \psi(Y_q),%\\
 %\Psi_{q,p}(Y_q,\ldots,Y_p) \triangleq % 
%\psi_q(Y_q)\psi_{q+1}(Y_{q+1})\ldots\psi_p(Y_p),
% \end{array}\!\!
%\right\}\!
% \label{eq:int_sum}
 \end{align*}
 определяемых набором пар 
$\{(w_{\ell},\varrho_{\ell}^{ij})\}_{\ell=\overline{1,L}}$. Здесь 
$\varrho_{\ell}^{ij} \hm\geqslant 0$, $\ell\hm=\overline{1,L},$~--- веса; 
$\sum\nolimits_{\ell=1}^L\varrho_{\ell}^{ij} \hm\leqslant 1$; 
$w_{\ell}\hm\triangleq 
\mathrm{col}\left(w^1_{\ell},\ldots,w^N_{\ell}\right) \hm\in 
\mathcal{D}$~--- точки; $\psi_q \hm\triangleq \|\psi^{ij}(Y_q)\|_{ij=\overline{1,N}}$.

 По построению $\psi^{ij}_q$ являются положительными случайными 
величинами, поэтому \textit{численная аппроксимация} $\widetilde{X}_r$ 
оценки~$\overline{X}_r$
 \begin{equation*}
 \widetilde{X}_r \triangleq (\mathbf{1}\psi_r^{\top}  
\widetilde{X}_{r-1})^{-1}\psi_r^{\top} \widetilde{X}_{r-1}, \enskip 
r\geqslant 1, \enskip \widetilde{X}_{0} = \pi,
% \label{eq:pp_est}
 \end{equation*}
 также обладает свойством устойчивости.

Если для схемы численного интегрирования выполнено условие
$$ %\textstyle
\max\limits_{k=\overline{1,N}} \sum\limits_{j=1}^N
\int\limits_{\mathbb{R}^M}|\psi^{kj}(y) - \xi^{kj}(y)|dy < \delta\,,
$$
то глобальный показатель точности приближения оптимальной оценки
$\widehat{X}_r$ ее численной аппроксимацией $\widetilde{X}_r$
ограничен сверху
 \begin{multline} %\textstyle
\sup\limits_{\pi \in \Pi}\me{}{\|\widehat{X}_r - \widetilde{X}_r\|_1} 
\leqslant{}\\
{}\leqslant  4
\left[ 1-\left(1-\fr{(\overline{\lambda}h)^{s+1}}{(s+1)!}\right)^r
\right] + 2r\delta,
 \label{eq:tot_glob}
 \end{multline}
 где $\Pi \triangleq \{\pi \in \mathbb{R}^N_+:\; \mathbf{1}\pi=1\}$~--- 
вероятностный симплекс; $\overline{\lambda} \triangleq 
\max_k|\lambda_{kk}|$.

 Для фиксированного момента времени $T$ с~уменьшением шага дискретизации 
$h \hm\to 0$ неравенство~(\ref{eq:tot_glob}) принимает асимптотический 
вид:
 \begin{multline}
\sup\limits_{\pi \in \Pi}\me{}{\|\widetilde{X}_{T/h} - 
\widehat{X}_{T/h}\|_1}
\leqslant{}\\
{}\leqslant
2T\left(2\overline{\lambda} \fr{(\overline{\lambda}h)^s}{(s+1)!}+
 \fr{\delta}{h}\right).
 \label{eq:asympt_glob}
\end{multline}
Поэтому для эффективного выбора схемы численного интегрирования,
чтобы не понизить порядок точности, обеспечиваемый аналитической 
аппроксимацией $s$-го порядка,
необходимо подбирать~$\delta$ так, чтобы 
${\overline{\lambda}\delta}/{(\overline{\lambda}h)^{s+1}}\hm \to C 
\hm\geqslant 0$ при $h \hm\to 0$.

 \section{Приближенное решение задачи фильтрации по~наблюдениям 
с~мультипликативными шумами}

\vspace*{-2pt}

 Для построения численных схем фильтрации состояний однородных МСП по 
наблюдениям с~мультипликативными шумами и~исследования их точности без 
ограничения общности рассмотрим\linebreak случай <<чисто мультипликативных шумов>>. 
Это означает, что в~(\ref{eq:obsys_1}) $f\hm=0$, т.\,е.\ аддитивный 
полезный сигнал полностью отсутствует, а~информация об оцениваемом 
состоянии скрыта в~интенсивности шумов в~наблюдениях. Будем также 
считать, что все условные интенсивности шумов мультипликативных 
наблюдений $\{g_n\}_{n=\overline{1,N}}$ различны.

 Ниже исследуются аппроксимации порядка $s\hm=1$ и~$2$. Для них 
с~помощью обобщенной формулы полной вероятности легко получить вид 
интегралов~(\ref{eq:xi_def}), используемых в~дальнейшем изложении:

\noindent
 \begin{multline} %\textstyle
 \int\limits_{\mathcal{D}} \mathcal{N}\left(y,f u,\sum\limits_{p=1}^N u^p 
g_p\right) \rho^{k,j,0}(du) ={}\\
{}=
 \delta_{kj}e^{\lambda_{kk}h} \mathcal{N}\left(y,0 ,h g_k\right);
 \label{eq:h0}
 \end{multline}
 
 \vspace*{-12pt}
 
 \noindent
 \begin{multline*}
 \displaystyle \int\limits_{\mathcal{D}} \mathcal{N}\left(y,f 
u,\sum_{p=1}^N u^p g_p\right) \rho^{k,j,1}(du) ={}\\
{}=(1- \delta_{kj})\lambda_{kj}e^{\lambda_{jj}h}
 \int\limits_0^{h}
 Q^{kj}(y,u)du\,;
 \end{multline*}
 
 \vspace*{-12pt}
 
 \noindent
 \begin{multline*}
  Q^{kj}(y,u) \triangleq{}\\
 {}\triangleq e^{(\lambda_{kk}-\lambda_{jj})u}
 \mathcal{N}\left(y,0, u g_k+(h-u)g_j\right)\,;
 % \label{eq:h1}
 \end{multline*}
 
 \noindent
\begin{equation}
\left.
\begin{array}{l}
\displaystyle
\int\limits_{\mathcal{D}} \mathcal{N}\left(y,f u,\sum\limits_{p=1}^N u^p 
g_p\right) \rho^{k,j,2}(du) ={}\\[6pt]
{}=\displaystyle
 \sum\limits_{\substack{{i:i \neq k,}\\ {i \neq j}}} \lambda_{k 
i}\lambda_{i j} e^{\lambda_{jj}h}
\int\limits_0^{h} \int\limits_0^{h-u} R^{kij}(y,u,v)dvdu,\\[6pt]
R^{kij}(y,u,v) \triangleq
e^{(\lambda_{kk}-\lambda_{i i})u+(\lambda_{i i}-\lambda_{jj})v}\times{}\\[6pt]
\hspace*{5mm}{}\times
\mathcal{N}\left(y,0,
 u g_k+v g_{i} + (h - u - v )g_j
 \right).
 \end{array}
\right\}
 \label{eq:h2}
 \end{equation}
 
За исключением~(\ref{eq:h0}), остальные интегралы не имеют явного 
аналитического представления, и~для них будут рассмотрены различные 
варианты чис\-лен\-ной реализации.

\vspace*{-10pt}

 \subsection{Порядок $s=1$, схема <<средних>> прямоугольников}
 
 \vspace*{-2pt}
 
 Рассмотрим аналитическую аппроксимацию $\overline{X}_r$ порядка 
$s\hm=1$.
 В~этом случае
 
 \noindent
 \begin{multline*} %\textstyle
 \xi^{kj}(y)=\delta_{kj}e^{\lambda_{jj}h}\mathcal{N}(y,0,hg_j)+{}\\
 {}+
 \left(1-\delta_{kj}\right)\lambda_{kj}e^{\lambda_{jj}h}
 \int\limits_0^h Q^{kj}(y,u)\,du\,,
% \label{eq:xi_1}
 \end{multline*}
 а~для ее приближения попробуем использовать схему <<средних>> 
прямоугольников без дополнительного дробления отрезка $[0,h]$:

\noindent
 \begin{multline*} %\textstyle
 \psi^{kj}(y)=\delta_{kj}e^{\lambda_{jj}h}\mathcal{N}(y,0,hg_j)
+{}\\
{}+
 \left(1-\delta_{kj}\right)\lambda_{kj}hQ^{kj}\left(y,\fr{h}{2}\right).
 %\label{eq:psi_mid_tr}
 \end{multline*}
 При этом ошибка численного интегрирования~\cite{IK_94} определяется 
формулой:

\noindent
 \begin{multline*} %\textstyle
 \gamma^{kj}(y)=\left(1-
\delta_{kj}\right)\fr{\lambda_{kj}h^3e^{\lambda_{jj}h}}{24}\,
\fr{\partial^2}{\partial u^2}Q^{kj}(y,u)\Bigl|_{u=z} ={}\\
\hspace*{-2pt}{}=
 \left(1-\delta_{kj}\right)\fr{\lambda_{kj}h^3e^{\lambda_{jj}h}}{24}
 \,Q^{kj}(y,z) \left[\zeta_0^2(y,z)-\zeta_1(y,z)\right]\!,\hspace*{-7.95522pt}
 \end{multline*}
 где $z = z (y) \in [0,h]$~--- некоторый параметр, зависящий от $y$;
 
\noindent
 \begin{multline*} %\textstyle
 \zeta_0 (y,z) \triangleq
 \lambda_{kk}-\lambda_{jj}-{}\\
 {}-\fr{d}{dz}\left\vert z g_k+(h-z)g_j\right\vert \Big/
 \left(2\left\vert z g_k+(h-
z)g_j \right\vert\right)
 +{}\\
{} +\fr{1}{2}\,y^{\top}[z g_k+(h-z)g_j]^{-1}\left( g_k-g_j\right)\times{}\\
{}\times\left[z 
g_k+(h-z)g_j\right]^{-1}y\,;
 %\label{eq:zeta_0_def}
 \end{multline*}

\vspace{-16pt}

\noindent
 \begin{multline*} %\textstyle
 \zeta_1(y,z) \triangleq{}\\
 {}\triangleq
 \left( \vphantom{\left(\fr{d}{dz}\left\vert z 
g_k+(h-z)g_j\right\vert \right)^2}
 \left\vert z g_k+(h-z)g_j\right\vert  \fr{d^2}{dz^2}
 \left\vert z g_k
 +(h-z)g_j\right\vert -{}\right.\\
\left.{}-\left(\fr{d}{dz}\left\vert z 
g_k+(h-z)g_j\right\vert \!\right)^{\!2}
\right)\!\!\Bigg/ \!\!
\left( \vphantom{(h-z)g_j|^2}
2|z g_k+{}\right.\\
\left.{}+(h-z)g_j|^2
\right)
 +y^{\top}\left[z g_k+(h-z)g_j\right]^{-1}\times{}\\
 {}\times \left( g_k-g_j\right)\left[z g_k+(h-z)g_j\right]^{-1}\times{}\\
 {}\times
 \left( g_k-g_j\right)\left[z g_k+(h-z)g_j\right]^{-1}y\,.
% \label{eq:zeta_1_def}
 \end{multline*}
 Непосредственно интегрировать абсолютную величину~$\gamma^{kj}$ 
проблематично, так как
 $\int\nolimits_{\mathbb{R}^M}|\gamma^{kj}(y)|\,dy\hm= 
\int\nolimits_{\mathbb{R}^M}|\gamma^{kj}(y, z^{kj}(y))|\,dy$, а~зависимость 
$z^{kj} (y)$ в~общем случае неизвестна. Поэтому предварительно оценим 
$|\gamma^{kj}|$ сверху.

 Свойства системы~(\ref{eq:obsys_1}) гарантируют, что существуют такие 
симметричные матрицы~$g$ и~$G$, что $0\hm < g \hm\leqslant g_n 
\hm\leqslant G$ для всех $n=\overline{1,N}$. Поэтому выполняется 
неравенство
 \begin{equation*}
 Q^{kj}(y,u) \leqslant K_1 \mathcal{N}(y,0,hg),
 %\label{eq:Q_ineq}
 \end{equation*}
 
 %\vspace*{-3pt}
 
 \noindent
  где
  
\vspace*{-3pt}

  \noindent
 $$
 K_1 = \exp\left(
 \max\limits_{\substack{{k,j=\overline{1,N}:k \neq j}\\ {u \in[0,h]}}} 
(\lambda_{kk}-\lambda_{jj})u
 \right)\fr{|G|}{|g|}.
% \frac{|hg|}{\max_{k,j=\overline{1,N}:k \neq j \atop u \in[0,h]}|u % 
%g_k+(h-u)g_j|}.
 $$
 
 \noindent
 Из свойства определителей~\cite{MN_2019} следует, что
 \begin{multline}
 |z g_k+(h-z)g_j| =\left\vert z (g_k-g_j)+hg_j\right\vert ={}\\
 {}=\sum\limits_{n=0}^N z^nh^{N-
n}G_{kjn},
 \label{eq:det}
 \end{multline}
 где $G_{kjn}$~--- сумма всех определителей матриц, полученных из матрицы 
$u (g_k-g_j)$ путем замены $n$ столбцов соответствующими столбцами 
мат\-ри\-цы~$hg_j$. Отсюда следует, что
 \begin{equation}
 \fr{d}{dz} |z g_k+(h-z)g_j| = \sum\limits_{n=1}^N nz^{n-1}h^{N-
n}G_{kjn},
 \label{eq:det_der}
 \end{equation}
 поэтому верно неравенство:
\begin{multline*}
\left|
 \fr{d}{dz}\left\vert z g_k+(h-z)g_j\right\vert \Big/ 
 \left(2\left\vert z g_k+(h-z)g_j\right\vert \right) \right|
 ={}\\
 {}=h^{-1}
 \left|
 \fr{\sum\nolimits_{n=1}^N n\left({u}/{h}\right)^{n-
1}G_{kjn}}{\sum\nolimits_{m=0}^N \left({u}/{h}\right)^mG_{kjm}}
 \right|
 \leqslant \fr{K_2}{h},
 \end{multline*}
 где 
$$
K_2 = \max\limits_{\substack{{k,j=\overline{1,N}:}\\ {k\neq 
j}}}\fr{\sum\nolimits_{n=1}^N n\left|G_{kjn}\right|}{2\min_{w \in[0,1]} 
\left| \sum\nolimits_{m=0}^N G_{kjm} w^m\right|}.
$$
 Таким образом, для функции $|\zeta_0(y,u)|$ верна следующая оценка 
сверху:
$$
 |\zeta_0(y,u)| \leqslant K_3 + \fr{K_2}{h} + \fr{K_4}{h^2}\|y\|^2_I,
 $$
 где $K_3 = \max\nolimits_{\substack{{k,j=\overline{1,N}:}\\ { k \neq 
j}}}|\lambda_{kk}-\lambda_{jj}|$, $K_4 \hm= \|g^{-1}Gg^{-1}\|^2_2$~--- 
квадрат спектральной нормы мат\-ри\-цы. Из этого следует оценка сверху для 
квадрата $\zeta_0(y,u)$:
 \begin{multline*} %\textstyle
 \zeta_0^2(y,u)= K_3^2 + \fr{K_2^2}{h^2}+\fr{K_4^2}{h^4}\|y\|^4_I+{}\\
 {}+
 2\fr{K_2K_3}{h}+ 2\fr{K_3K_4}{h^2}\|y\|^2_I + 
2\fr{K_2K_4}{h^3}\|y\|^2_I.
 %\label{eq:zeta0_2_upbound}
 \end{multline*}

 Используя формулы~(\ref{eq:det}) и~(\ref{eq:det_der}), можно получить 
оценку сверху для абсолютного значения первого слагаемого 
в~$\zeta_1(y,u)$:

\noindent
 \begin{multline*}
 \left\vert 
 \left(
 \vphantom{\left(\fr{d}{dz}  \left\vert z g_k+\left(h-z\right)g_j\right\vert \right)^{\!2}}
\left\vert z g_k+(h-z)g_j\right\vert \fr{d^2}{dz^2}
 \left\vert z g_k+(h-z)g_j\right\vert -{}\right.\right.\\
\left. {}-\left(\fr{d}{dz}
 \left\vert z g_k+\left(h-
z\right)g_j\right\vert \right)^{\!2}
\right)\left(
\vphantom{(h-z)g_j|^2}
2|z g_k+{}\right.\\
\left.\left.{}+(h-z)g_j|^2\right)^{-1} 
\vphantom{\left(\left(\fr{d}{dz}\right)  \left\vert z g_k+\left(h-z\right)g_j\right\vert \right)^{\!2}}
\right\vert= 
\left|
\left(
\vphantom{\left(\left(\sum\limits_{\ell=1}^N\ell z^{\ell-1}h^{N-
\ell}G_{kj\ell}\right)^{\!2}\right)^{-1}}
\sum\limits_{n=0}^Nz^sh^{N-n}G_{kjn}\times{}\right.\right.\\
{}\times\sum\limits_{m=2}^Nm(m-1)z^{m-2}h^{N-m}G_{kjm} -{}\\
\left.{}- \left(\sum\limits_{\ell=1}^N\ell z^{\ell-1}h^{N-\ell}
G_{kj\ell}\right)^{\!2}\right)\times{}\\
{}\times
\left.\left(2\left(\sum\limits_{s=0}^Nz^sh^{N-s}G_{kjs}\right)^2\right)^{-1}
\right| = \fr{1}{h^2}\times{}\\
{}\times
\left|
\left( 
\vphantom{\left(\left(\sum\limits_{\ell=1}^N\ell z^{\ell-1}h^{N-
\ell}G_{kj\ell}\right)^{\!2}\right)^{-1}}
\sum\limits_{n=0}^N\left(\fr{z}{h}\right)^s \hspace*{-1.5pt}G_{kjn}
\sum\limits_{m=2}^Nm(m-1)\left(\fr{z}{h}\right)^{m-2}\hspace*{-1.5pt}G_{kjm} - {}\right.\right.\\
\left.{}-\left(\sum\limits_{\ell=1}^N \ell \left(\fr{z}{h}\right)^{\ell-1}\!\!
G_{kj\ell}\right)^2
\right)\times{}\\
{}\times
\left.\left(2\left(\sum\limits_{s=0}^N\left({z}/{h}\right)^sG_{kjs}\right)^2\right)^{-1}
 \right| \leqslant \fr{K_5}{h^2},
 \end{multline*}
 где
 \begin{multline*}
 \hspace*{-7.7pt}K_5 = \hspace*{-13pt}\max\limits_{%\substack
 {k,j=\overline{1,N}:} %\\ 
 {k \neq j}}
\left(
\vphantom{\left(\sum\limits_{\ell=1}^N\ell |G_{kj\ell}|\right)^2}
\!\left( 
\sum\limits_{n=0}^N|G_{kjn}|\sum\limits_{m=2}^Nm(m-1)|G_{kjm}| + {}\right.\right.\vspace*{-3pt}\\
\left.\left.{}+
\left(\sum\limits_{\ell=1}^N\ell |G_{kj\ell}|\right)^2
\right)\!\Bigg/ \!\left(2\min\limits_{w \in [0,1]}
(\sum\limits_{s=0}^N w^sG_{kjs})^2\right)
\vphantom{\left(\sum\limits_{\ell=1}^N\ell |G_{kj\ell}|\right)^{\!2}}
\!\right).\hspace*{-5pt}
 \end{multline*}
 Абсолютное значение второго слагаемого в~$\zeta_1(y,u)$ также 
оценивается сверху:
 \begin{multline*}
 y^{\top}\left[z g_k+(h-z)g_j\right]^{-1}\left( g_k-g_j\right)\times{}\\
 {}\times \left[z g_k+(h-z)g_j\right]^{-1}
 ( g_k-g_j)\left[z g_k+{}\right.\\
\left. {}+(h-z)g_j\right]^{-1}y \leqslant
 \fr{K_6}{h^3}\,\|y\|^2_I\,,
 \end{multline*}
 где $K_6 = 4 \|g^{-1}Gg^{-1}Gg^{-1}\|^2_2$.


\begin{figure*} %fig1
\vspace*{1pt}
 \begin{center}
 \mbox{%
 \epsfxsize=162.006mm 
 \epsfbox{bor-1.eps}
 }
 \end{center}
   \vspace*{-9pt}
   \begin{minipage}[t]{80mm}
   \Caption{Общий вид функции $\mathfrak{Q}(y,u)$ в~случае аддитивных шумов 
в~наблюдениях
} 
\end{minipage}
\label{pic:pic1}
%\end{figure*}
\hfill
% \begin{figure*} %fig2
    \vspace*{-9pt}
       \begin{minipage}[t]{80mm}
\Caption{Общий вид функции $Q(y,u)$ в~случае мультипликативных шумов в~наблюдениях
} \label{pic:pic3}
\end{minipage}
\vspace*{12pt}
\end{figure*}


 Обозначим 
 
 \noindent
 $$
 K_0 \triangleq 
\max\limits_{\substack{{k,j=\overline{1,N}:}\\ {k \neq 
j}}}\fr{\lambda_{kj}e^{\lambda_{jj}h}}{24}
$$ 
и~оценим сверху
 $\int_{\mathbb{R}^M}|\gamma^{kj}(y)|\,dy$, используя связь моментов \mbox{2-го} 
и~4-го порядка гауссовского распределения:

\noindent
 \begin{multline*} %\textstyle
 \int\limits_{\mathbb{R}^M}|\gamma^{kj}(y)|dy \leqslant{}\\
 {}\leqslant h^3 K_0K_1
 \left[
 K_3^2 + \fr{2K_2K_3}{h} + \fr{K_2^2+K_5}{h^2}
 \right] + h^3 K_0K_1\times{}\\
 {}\times
 \int\limits_{\mathbb{R}^M}
 \left[
 \fr{2K_3K_4}{h^2} + \fr{2K_2K_4+K_6}{h^3}
 \right]
 \|y\|^2_I
 \mathcal{N}(y,0,hG)dy
 + {}\\ 
{}+ 
 h^3 K_0K_1\fr{K_4^2}{h^4}
 \int\limits_{\mathbb{R}^M}
 \|y\|^4_I
 \mathcal{N}(y,0,hG)dy ={}\\
 {}= K_7h + K_8h^2 +K_9h^3
 \end{multline*}
 для некоторых положительных констант $K_7$, $K_8$ и~$K_9$. Это значит, 
что 
$$
\int\limits_{\mathbb{R}^M}|\gamma^{kj}(y)|dy \hm= O(h)
$$ 
и~согласно~(\ref{eq:asympt_glob}) 
$$
\sup\limits_{\pi \in 
\Pi}\me{}{\|\widetilde{X}_{T/h} - \widehat{X}_{T/h}\|_1}\hm = O(h^0)
$$ 
и~точности \textit{несоставного} метода <<средних>> прямоугольников 
недостаточно для построения численного алгоритма фильтрации любой 
степени. Какая-либо замена этой схемы на другую несоставную (например, на 
схему Симпсона, квадратуры Гаусса и~пр.)\ к~улучшению не приведет. 
Причиной этому является связь между порядком производной и~степенью~$h$ 
в~оценке ошибки интеграла по остатку ряда Тейлора.
 Дело в~том, что функция $Q_{kj}(y,u)$ при малых значениях параметра~$h$ 
и~некоторых фиксированных значениях~$y$ быстро меняется по аргументу~$u$. 
Примечательно, что в~случае наблюдений с~аддитивными 
шумами~\cite{B_20_2_IA} аналогичная функция
 $$
\mathfrak{Q}_{kj}(y,u) = e^{(\lambda_{kk}-\lambda_{jj})u}
 \mathcal{N}\left(y,u f^k+(h-u)f^j, h g\right)
$$
 такими отрицательными свойствами не обладает. В последней формуле $f^k$ 
и~$f^j$ означают $k$-й и~$j$-й столбцы матрицы $f$.

 Для иллюстрации данного факта рассмотрим два примера наблюдений 
одинакового масштаба:
 \begin{enumerate}
 \item
 \textit{Наблюдения c аддитивными шумами:}
 \begin{gather*}
N = 2;\enskip M = 1; \enskip f = [1\;\; 10]; \enskip 
\lambda_{11}=\lambda_{22}=-1\,. \\ 
g_1=g_2=1\,.
\end{gather*}
 \item
 \textit{Наблюдения c мультипликативными шумами:}
 \begin{gather*}
N = 2;\enskip M = 1; \enskip f = [0\;\; 0 ]; \enskip 
\lambda_{11}=\lambda_{22}=-1; \\
 g_1=1;\enskip g_2=100.
\end{gather*}
 \end{enumerate}
 Общие трехмерные графики функций $\mathfrak{Q}(y,u)$ и~$Q(y,u)$ 
представлены на рис.~1 и~3, а~отдельные их 
сечения~--- на рис.~3 и~4 соответственно. На 
рис.~4 видно, что в~окрестности $y=0$ функция $Q(y,u)$ 
меняется быстро и~любая простая (несоставная) схема не обеспечит 
необходимой точности интегрирования.
 

 \begin{figure*} %fig3
 \vspace*{1pt}
 \begin{minipage}[t]{80mm}
 \begin{center}
 \mbox{%
 \epsfxsize=79mm 
 \epsfbox{bor-3.eps}
 }
 \end{center}
   \vspace*{-9pt}
\Caption{
Сечения $\mathfrak{Q}(y,u)$ для некоторых фиксированных~$y$ в~случае 
аддитивных шумов в~наблюдениях:
\textit{1}~--- $y\hm=-0{,}083$; \textit{2}~--- $-0{,}0485$;
\textit{3}~--- $-0{,}014$; \textit{4}~--- $-0{,}0205$; \textit{6}~--- $y\hm=0{,}055$
} \label{pic:pic2}
\end{minipage}
%\end{figure*}
\hfill
% \begin{figure*} %fig4
\vspace*{1pt}
\begin{minipage}[t]{80mm}
 \begin{center}
 \mbox{%
 \epsfxsize=79mm 
 \epsfbox{bor-4.eps}
 }
 \end{center}
   \vspace*{-9pt}
\Caption{
Сечения $Q(y,u)$ для некоторых фиксированных~$y$ в~случае 
мультипликативных шумов в~наблюдениях: \textit{1}~--- $y\hm=-0{,}9$; \textit{2}~--- $-0{,}6$;
\textit{3}~--- $-0{,}3$; \textit{4}~--- $-0{,}15$; \textit{5}~--- $y\hm=0$
} \label{pic:pic4}
\end{minipage}
\end{figure*}

Используем для приближенного вычисления $\xi^{kj}$ составную схему 
<<средних>> прямоугольников, разбив отрезок интегрирования $[0,h]$ 
с~шагом $h^2$. В~этом случае
 \begin{equation*} %\textstyle
 \gamma^{kj}(y)=(1-\delta_{kj})\fr{\lambda_{kj}h^5e^{\lambda_{jj}h}}{24}
\,\fr{\partial^2}{\partial u^2}Q_{kj}(y,u)\Bigl|_{u=z}.
 \end{equation*}
 Повторяя все выводы этого подраздела для составной схемы <<средних>> 
прямоугольников, можно проверить, что она обеспечивает порядок 
точ\-ности~$3$:
 $$
 \int\limits_{\mathbb{R}^M}|\gamma^{kj}(y)|\,dy = O(h^3)
 $$ 
и~согласно~(\ref{eq:asympt_glob}) глобальный показатель точности имеет 
порядок~$1$:
 $$
\sup\limits_{\pi \in \Pi}\me{}{\|\widetilde{X}_{T/h} - 
\widehat{X}_{T/h}\|_1} \leqslant CTh.
$$
 Окончательно схема вычисления $\psi^{kj}$ при $s=1$ примет вид:
 \begin{multline*} %\textstyle
 \psi^{kj}(y)=\delta_{kj}e^{\lambda_{jj}h}\mathcal{N}(y,0,hg_j)
+{}\\
{}+
 (1-\delta_{kj})\lambda_{kj}h^2 
\sum\limits_{i=1}^{\left[{1}/{h}\right]-1} 
Q^{kj}\left(y,h^2\left(i-\fr{1}{2}\right)\right).
 %\label{eq:psi_mid_tr_comp}
 \end{multline*}
 
\subsection{Порядок $s=2$, схема средних}

 В случае $s=2$ элементы $\xi^{jk}$ вычисляются по формуле:
 \begin{multline} %\textstyle
 \xi^{kj}(y)= \delta_{kj}e^{\lambda_{jj}h}\mathcal{N}(y,0,hg_j)+{} \\
{}+ 
 (1-\delta_{kj})\lambda_{kj}e^{\lambda_{jj}h}
 \int\limits_0^h Q^{kj}(y,u)\,du +{}\\
 {}+
 \sum\limits_{\substack{{i:i \neq k,}\\ {i \neq j}}} \lambda_{k 
i}\lambda_{i j} e^{\lambda_{jj}h}
\int\limits_0^{h} \int\limits_0^{h-u}
R^{kij} (y,u,v)\,dvdu.
 \label{eq:xi_2}
 \end{multline}
Согласно~(\ref{eq:asympt_glob}) для сохранения второго порядка точности 
аналитической аппроксимацией необходимо, чтобы локальная ошибка 
численного интегрирования на каждом шаге была не более~$O(h^3)$.\linebreak 
Составная схема <<средних>> прямоугольников, предложенная в~предыдущем 
подразделе, обеспечивает эту точность для вычисления одномерного 
интеграла~--- второго слагаемого в~(\ref{eq:xi_2}).

Выберем подходящую схему вычисления двойных интегралов по треугольнику, 
входящих в~третье слагаемое~(\ref{eq:xi_2}).
 Прежде всего, определим величину ошибки приближения интеграла 
в~(\ref{eq:h2}) простым методом средних:
 \begin{multline*} %\textstyle
 \lambda_{k i}\lambda_{i j} e^{\lambda_{jj}h}
\int\limits_0^{h} \int_0^{h-u} R^{kij}(y,u,v)\,dvdu ={}\\
{}= \fr{h^2}{2}\,
\lambda_{k i}\lambda_{i j} e^{\lambda_{jj}h}
R^{kij}\left(y,\fr{h}{3},\fr{h}{3}\right)+{} \\ %\textstyle
{}+\lambda_{k i}\lambda_{i j} e^{\lambda_{jj}h} \int\limits_0^{h} 
\int\limits_0^{h-u}\chi_2^{kij}(y,u,v)\,dvdu,
 \end{multline*}
где функция $\chi_2^{kij}(y,u,v)$ имеет вид:
\begin{multline*} %\textstyle
\chi_2^{kij}(y,u,v)\triangleq
\fr{1}{2}
\left(\left(z-\fr{h}{3}\right)\fr{\partial}{\partial z}+{}\right.\\
\left.{}+\left(w-
\fr{h}{3}\right)\fr{\partial}{\partial w}
\right)^2R^{kij}(y,z,w)\Bigl|_{(z(y,u),w(y,v))}.
\end{multline*}
Согласно~\cite{IK_94}, для некоторой положительной константы~$C_1$ верно 
неравенство:
 \begin{multline*} %\textstyle
\lambda_{k i}\lambda_{i j} e^{\lambda_{jj}h} \int\limits_0^{h} 
\int\limits_0^{h-u}\chi_2^{kij}(y,u,v)\,dvdu \leqslant{}\\
{}\leqslant h^4C_1
\max\limits_{\substack{{\ell=0,1,2;}\\ { (z,w) \in \mathcal{D}}}}
\left|\fr{\partial^2}{\partial z^{\ell}\partial w^{2-
\ell}}\,\chi_2^{kij}(y,z,w)\right|.
 \end{multline*}
 В предыдущем разделе также оценивалась вторая производная, однако от 
другой функции~--- $Q^{kj}$. Она содержала $h^2$ в~знаменателе. Сравнивая 
$Q^{kj}$ и~$R^{kij}$, можно заключить, что вторая производная от 
$R^{kij}$ также будет содержать $h^2$ в~знаменателе, т.\,е.
 $$
 \lambda_{k i}\lambda_{i j} e^{\lambda_{jj}h} \int\limits_0^{h} 
 \int\limits_0^{h-u}\chi_2^{kij}(y,u,v)\,dvdu \leqslant h^2C_2
 $$
 для некоторой положительной константы $C_2$. Так как требуемый порядок 
точности~--- третий, последнее равенство позволяет сделать вывод о~том, 
что простой метод средних в~данном случае нужной точности не 
обеспечивает.

 Используем для вычисления двойного интег\-рала составной метод средних, 
разбив область\linebreak интегрирования, прямоугольный треугольник с~катетами длины 
$h$, на подобные треугольники с~катетами~$h^2$. В этом случае
 $$
\lambda_{k i}\lambda_{i j} e^{\lambda_{jj}h} \int\limits_0^{h} 
\int\limits_0^{h-u}\chi_2^{kij}(y,u,v)\,dvdu \leqslant h^4 C_3
 $$
 для некоторой положительной константы $C_3$. Тогда 
согласно~(\ref{eq:asympt_glob})
 $$
\sup\limits_{\pi \in \Pi}\me{}{\|\widetilde{X}_{T/h} - 
\widehat{X}_{T/h}\|_1} \leqslant CTh^2,
$$
 т.\,е.\ для реализации аналитической аппроксимации порядка $s\hm=2$ 
правомерно использование составных схем средних при вычислении одномерных 
и~двойных интегралов.

 Окончательно схема вычисления $\psi^{kj}$ при $s\hm=2$ примет вид:
 \begin{multline*} %\textstyle
 \psi^{kj}(y)=\delta_{kj}e^{\lambda_{jj}h}\mathcal{N}(y,0,hg_j)
+{}\\
{}+
 (1-\delta_{kj})\lambda_{kj}h^2\sum_{i=1}^{\left[{1}/{h}\right]-1}
Q^{kj}\left(y,h^2\left(i-\fr{1}{2}\right)\right)+ {}\\
{} +
 \fr{h^4}{2}\sum\limits_{\substack{{i:i \neq k,}\\ {i \neq j}}} 
\lambda_{k i}\lambda_{i j} e^{\lambda_{jj}h}\times{}\\
{}\times
\sum\limits_{n=1}^{\left[{1}/{h}\right]-1}
\sum\limits_{m=1}^{\left[{1}/{h}\right]-1-n} \hspace*{-10pt}R^{kij}\!\left(y,h^2
\left(n-\fr{2}{3}\right),h^2\left(m-\fr{2}{3}\right)\!\right).\hspace*{-5.93419pt}
 %\label{eq:psi_mid_tr_comp-t}
 \end{multline*}

 \section{Заключение}

 Данный цикл работ посвящен разработке алгоритмов численного решения 
задач фильтрации МСП по косвенным наблюдениям в~присутствии винеровских 
шумов. Перед обработкой непрерывные наблюдения дискретизуются по времени. 
Таким образом, полученные результаты могут применяться для практических 
задач оценивания как по исходным непрерывным наблюдениям, так и~по 
заранее дискретизованным данным. Формула~(\ref{eq:asympt_glob}) при этом 
представляется ключевой. Она описывает влияние <<входных>> характеристик 
на итоговую точность численного алгоритма фильтрации. К~<<входным>> 
относятся: характеристика темпа изменения состояний МСП (величина 
$\overline{\lambda}$), шаг дискретизации по времени $h$, порядок 
реализуемой аналитической аппроксимации $s$ и~точность используемых схем 
численного интегрирования $\delta$. Основываясь на этих данных, 
(\ref{eq:asympt_glob}) позволяет оценить величину $\sup\nolimits_{\pi \in 
\Pi}\me{}{\|\widetilde{X}_{T/h} \hm- \widehat{X}_{T/h}\|_1}$~--- потерю 
точности при переходе от оптимальной оценки к~некоторой ее численной 
аппроксимации. Данная формула позволяет решать ряд практических задач 
системного анализа. Во-пер\-вых, по фиксированным $\overline{\lambda}$ 
и~$h$ можно выбирать схему численной реализации оценки с~учетом требуемой 
точности и~ресурсоемкости различных методов численного интегрирования. 
Во-вторых, при проектировании реальных сис\-тем наблюдения можно подбирать 
максимальный шаг дискретизации $h$, обеспечивающий для выбранной схемы 
численной реализации тре\-бу\-емую точность.

 Важно отметить еще одну особенность полученных результатов. В~отличие от 
классической\linebreak $\mathcal{L}_2$-мет\-ри\-ки, используемой для характеризации 
ошибок численного решения стохастических дифференциальных 
уравненией~\cite{KP_92}, в~предложенном цик\-ле для этого используется 
$\mathcal{L}_1$-метрика. Дело в~том, что решением задачи оптимальной 
фильт\-ра\-ции является условное распределение со\-сто\-яния МСП по имеющимся 
наблюдениям. Приближенные решения, предлагаемые в~данной работе, обладают 
свойством устойчивости, т.\,е.\ также могут рас\-смат\-ри\-вать\-ся как некоторые 
распределения. Очевидно, что $\mathcal{L}_1$-метрика оказывается более 
естественной характеристикой, описывающей расстояние между 
распределениями.

В данных статьях оценено расстояние между оптимальной оценкой фильтрации 
по дискретизованным наблюдениям и~ее численной аппроксимацией, выбранной 
из предложенного класса. Однако итоговая точность предложенных численных 
аппроксимаций, т.\,е.\ расстояние между ними и~истинным текущим 
состоянием оцениваемого МСП, пока не охарактеризовано. Решение данной 
задачи представляется одним из направлений дальнейших исследований.


{\small\frenchspacing
 {%\baselineskip=10.8pt
 \addcontentsline{toc}{section}{References}
 \begin{thebibliography}{9}

 \bibitem{B_20_1_IA}
 \Au{Борисов А.} Численные схемы фильтрации марковских скачкообразных 
процессов по дискретизованным наблюдениям~I:~характеристики точности~// 
Информатика и~её применения,~2019. Т.~13. Вып.~4. C.~68--75. doi: 
10.14357/19922264190411.

\bibitem{B_20_2_IA}
 \Au{Борисов А.} Численные схемы фильтрации марковских скачкообразных 
процессов по дискретизованным наблюдениям II: случай аддитивных шумов~// 
Информатика и~её применения,~2020. Т.~14. Вып.~1. C.~17--23. 
doi:  10.14357/19922264200103.

\bibitem{B_18}
\Au{Борисов А.} Фильтрация Вонэма по наблюдениям с~мультипликативными 
шумами~// Автоматика и~телемеханика, 2018.
№\,1. C.~52--65. % doi: 10.1134/S0005117918010046.

 \bibitem{B_18_IA}
 \Au{Борисов А.} Фильтрация состояний марковских скачкообразных процессов 
по дискретизованным наблюдениям~// Информатика и~её применения,~2018. 
Т.~12. Вып.~3. C.~115--121. doi: 10.14357/19922264180316.

%\bibitem{KP_92}
%{\it Kloeden P., Platen E.} Numerical solution of stochastic 
%differential equations.~--- % 
%Berlin: Springer, 1992. 636 p. %doi: 10.1007/978-3-662-12616-5.

%\bibitem{YZL_04}
%{\it Yin G., Zhang Q., Liu Y.}
%Discrete-time approximation of Wonham filters~//
%Journal of Control Theory and Applications. 2004. №.\,2. P.~1--10.

%\bibitem{PR_10}
%{\it Platen E., Rendek R.}
%Quasi-Exact Approximation of Hidden Markov Chain Filters~// 
%Communicat.~Stoch.~Analys. 2010.  
%Vol.~4. №.~1. P.~129--142.

%\bibitem{BGH_16}
%{\it B\"auerle, N., Gilitschenski I., Hanebeck U.} Exact and approximate 
%hidden Markov chain filters based on discrete observations // Statistics 
%\& Risk Modeling. 2016. V.~32.~ №.3-4.~
%P.~159--176.

\bibitem{IK_94}
\Au{Isaacson E., Keller H.} Analysis of numerical methods.~--- New York, 
NY, USA: Dover Publications, 1994. 541~p.

\bibitem{MN_2019}
 \Au{Magnus J., Neudecker H.}
 Matrix differential calculus with applications in statistics and 
econometrics.~---
 New York, NY, USA: Wiley, 2019. 504~p.

 \bibitem{KP_92}
\Au{Kloeden P., Platen E.} Numerical solution of stochastic differential 
equations.~--- Berlin: Springer, 1992. 636~p. %doi: 10.1007/978-3-662-12616-5.

\end{thebibliography}

 }
 }

\end{multicols}

\vspace*{-6pt}

\hfill{\small\textit{Поступила в~редакцию 11.10.19}}

\vspace*{8pt}

%\pagebreak

%\newpage

%\vspace*{-28pt}

\hrule

\vspace*{2pt}

\hrule

%\vspace*{-2pt}

\def\tit{NUMERICAL SCHEMES OF MARKOV JUMP PROCESS FILTERING GIVEN 
DISCRETIZED OBSERVATIONS III: MULTIPLICATIVE NOISES CASE}


\def\titkol{Numerical schemes of Markov jump process filtering given 
discretized observations III: Multiplicative noises case}

\def\aut{A.\,V.~Borisov}

\def\autkol{A.\,V.~Borisov}

\titel{\tit}{\aut}{\autkol}{\titkol}

\vspace*{-9pt}


\noindent
Institute of Informatics Problems, Federal Research Center ``Computer 
Science 
and Control'' of the Russian Academy of Sciences, 44-2~Vavilov Str., 
Moscow 
119333, Russian Federation

\def\leftfootline{\small{\textbf{\thepage}
\hfill INFORMATIKA I EE PRIMENENIYA~--- INFORMATICS AND
APPLICATIONS\ \ \ 2020\ \ \ volume~14\ \ \ issue\ 2}
}%
 \def\rightfootline{\small{INFORMATIKA I EE PRIMENENIYA~---
INFORMATICS AND APPLICATIONS\ \ \ 2020\ \ \ volume~14\ \ \ issue\ 2
\hfill \textbf{\thepage}}}

\vspace*{3pt} 



\Abste{The paper presents the final part of investigations initialized in 
the papers
\Aue{Borisov,~A.} 2019. Numerical schemes of Markov jump process 
filtering given discretized observations I: Accuracy characteristics. 
\textit{Inform.~Appl.} 13(4):68--75
and 
\Aue{Borisov,~A.} 2020. Numerical schemes of Markov jump process 
filtering given discretized observations II: Multiplicative noises case. 
\textit{Inform.~Appl.} 14(1):17--23. 
Relying on the theoretical results, this paper presents a~numerical 
algorithm of the state filtering of homogeneous Markov jump processes 
(MJP) given indirect noisy continuous time observations discretized by 
time. The class of observation systems under consideration is restricted 
by ones with multiplicative noises: any additive payload component is 
absent in the observable signal, but the observation noise intensity is 
a~function of the MJP state under estimation. To calculate the integrals 
in the estimate, the author uses the composite midpoint rule of the 
precision order $3$, along with the composite midpoint rule for triangles 
of the precision order~$4$. The constructed numerical algorithms of 
filtering have the final precision of the orders~$1$ and~$2$.}

\KWE{Markov jump process; optimal filtering; additive and multiplicative 
observation noises; stochastic differential equation; analytical and 
numerical approximation}



\DOI{10.14357/19922264200202}

\vspace*{-6pt}

\Ack
\noindent
The work was supported in part by the Russian Foundation
for Basic Research (project No.\,19-07-00187~A).


\vspace*{6pt}

 \begin{multicols}{2}

\renewcommand{\bibname}{\protect\rmfamily References}
%\renewcommand{\bibname}{\large\protect\rm References}

{\small\frenchspacing
 {%\baselineskip=10.8pt
 \addcontentsline{toc}{section}{References}
 \begin{thebibliography}{9}


 

 \bibitem{B_20_1_IA-1}
\Aue{Borisov, A.} 2019. Chislennye skhemy fil'tratsii markovskikh 
skachkoobraznykh protsessov po diskretizovannym nablyudeniyam I: 
kharakteristiki tochnosti [Numerical schemes of Markov jump process 
filtering given discretized observations I: Accuracy characteristics]. 
\textit{Informatika i~ee Primeneniya~--- Inform. Appl.} 13(4):68--75. 
doi: 10.14357/19922264190411.

 \bibitem{B_20_2_IA-1}
\Aue{Borisov, A.} 2020. Chislennye skhemy fil'tratsii markovskikh 
skachkoobraznykh protsessov po diskretizovannym nablyudeniyam II: sluchay 
additivnykh shuchmov [Numerical schemes of Markov jump process filtering 
given discretized observations II: Additive noises case]. 
\textit{Informatika i~ee primeneniya~--- Inform. Appl.} 14(1):17--23. 
doi: 10.14357/19922264200103.

\bibitem{B_18-1}
\Aue{Borisov, A.} 2018. 
Wonham filtering by observations with multiplicative noises. 
\textit{Automat. Rem. Contr.} 79(1):39--50. 
doi: 10.1134/S0005117918010046.

 \bibitem{B_18_IA-1}
\Aue{Borisov, A.} 2018. Fil'tratsiya sostoyaniy markovskikh 
skachkoobraznykh protsessov po diskretizovannym nablyu-\linebreak\vspace*{-12pt}

\pagebreak

\noindent
deniyam [Filtering 
of Markov jump processes by discretized observations]. 
\textit{Informatika i~ee Primeneniya~--- Inform.~Appl.} 12(3):115--121. 
doi: 10.14357/19922264180316.

\bibitem{IK_94-1}
\Aue{Isaacson, E., and H. Keller.} 1994.
\textit{Analysis of numerical methods.} New York, NY: Dover Publications. 
541~p.



\bibitem{MN_2019-1}
\Aue{Magnus, J., and H.~Neudecker}. 2019.
\textit{Matrix differential calculus with applications in statistics and 
econometrics.} 
New York, NY: Wiley. 504~p.

\bibitem{KP_92-1}
\Aue{Kloeden,~P., and E.~Platen.} 1992. \textit{Numerical solution of 
stochastic
differential equations.} Berlin: Springer. 636~p.


\end{thebibliography}

 }
 }

\end{multicols}

\vspace*{-3pt}

\hfill{\small\textit{Received October 11, 2019}}

%\pagebreak

%\vspace*{-24pt}


\Contrl

\noindent
\textbf{Borisov Andrey V.} (b.\ 1965)~--- Doctor of Science in physics and 
mathematics, principal scientist, Institute of Informatics Problems, 
Federal Research Center ``Computer Science and Control'' of the Russian 
Academy of Sciences, 44-2 Vavilov Str., Moscow 119333, Russian 
Federation; \mbox{aborisov@frccsc.ru}

\label{end\stat}

\renewcommand{\bibname}{\protect\rm Литература} 