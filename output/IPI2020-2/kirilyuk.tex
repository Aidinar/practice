\def\stat{kirilyuk}

\def\tit{ВЫБОР МОДЕЛЕЙ ОПТИМАЛЬНОЙ СЛОЖНОСТИ МЕТОДАМИ МОНТЕ-КАРЛО (НА 
ПРИМЕРЕ МОДЕЛЕЙ ПРОИЗВОДСТВЕННЫХ ФУНКЦИЙ РЕГИОНОВ РОССИЙСКОЙ 
ФЕДЕРАЦИИ)$^*$}

\def\titkol{Выбор моделей оптимальной сложности методами Монте-Карло 
(на примере моделей производственных функций} % регионов Российской Федерации)}

  \def\aut{И.\,Л.~Кирилюк$^1$, О.\,В.~Сенько$^2$}

  \def\autkol{И.\,Л.~Кирилюк, О.\,В.~Сенько}

\titel{\tit}{\aut}{\autkol}{\titkol}

  \index{Кирилюк И.\,Л.}
  \index{Сенько О.\,В.}
\index{Kirilyuk I.\,L.}
\index{Sen'ko O.\,V.}
 

{\renewcommand{\thefootnote}{\fnsymbol{footnote}} \footnotetext[1]
{Исследование выполнено в~рамках государственного задания по теме <<Феномен мезоуровня 
в~экономическом анализе: новые теории и~их практическое применение>>.}}


\renewcommand{\thefootnote}{\arabic{footnote}}
\footnotetext[1]{Институт экономики Российской академии наук, igokir@rambler.ru}
\footnotetext[2]{Федеральный исследовательский центр <<Информатика 
и~управление>> Российской академии наук, \mbox{senkoov@mail.ru}}


%\vspace*{-6pt}
  
  
  \Abst{Описан подход к~сравнению альтернативных вариантов линейных регрессионных моделей на 
временных рядах и~к~определению целесообразности их усложнения (посредством добавления новых 
переменных) с~использованием нескольких вариантов методов 
  Монте-Карло. Предложенные методы исследования с~помощью генерации псевдовыборок 
позволяют учесть как эффекты, связанные с~возможными отличиями распределений в~эмпирических 
данных от распределения Гаусса, так и~эффекты, связанные с~возможной нестационарностью 
исследуемых временных рядов. Для этого применяется генерирование псевдовыборок~--- временных 
рядов, являющихся гауссовым белым шумом или случайным блужданием, а~также перестановочный 
тест и~метод бутстрепа. Достоверность получаемых результатов оценивается при помощи процедур 
ресэмплинга. Применимость рассматриваемых методов демонстрируется на примере моделей 
инвестиционных производственных функций регионов Российской Федерации, рассчитываемых на 
основе данных Федеральной службы государственной статистики.}
  
  \KW{методы Монте-Карло; перестановочные тесты; ложная регрессия; производственные функции; 
селекция моделей; мезоуровень экономики}
  
\DOI{10.14357/19922264200216} 
 
%\vspace*{9pt}


\vskip 10pt plus 9pt minus 6pt

\thispagestyle{headings}

\begin{multicols}{2}

\label{st\stat}
  
  
\section{Введение}

  Во многих областях исследований, например в~общественных науках, предметная 
область не описывается надежно верифицированными и~строго обоснованными 
моделями. В этой ситуации приобретает актуальность сравнение альтернативных 
вариантов регрессионных моделей, проверка оправданности усложнения таких моделей, 
например посредством включения в~них новых переменных. До сих пор в~разных 
областях науки применяются простейшие методы сравнения моделей посредством 
сравнения коэффициентов детерминации, получаемых при применении этих моделей 
к~единому набору данных. Однако такой подход имеет тот недостаток, что коэффициент 
детерминации может только расти при добавлении в~модель новых переменных. 

Дальнейшее развитие математической статистики привело к~появлению новых способов 
сравнения с~использованием скорректированных ко\-эффициентов детерминации, 
критерия Акаике, байесовского критерия Шварца и~др. В~литературе подобные подходы 
в~ряде публикаций объединены под названием выбор моделей (\textit{англ.}\ Model 
selection)~[1]. Однако перечисленные критерии имеют свои ограничения, включая 
невозможность прямых оценок статистической до\-сто\-вер\-ности превосходства 
усложненных моделей.
  
  Одно из направлений Model selection~--- применение методов непараметрической 
статистики. В~отличие от вышеперечисленных критериев они не предполагают 
конкретную форму распределения ошибок модели, но при этом часто требуют долгих 
вычислений. Здесь применяются перестановочные тесты~[2--4] и~метод бутстрепа~[5]. 
Эти методы основаны на генерации множества псевдовыборок, полученных из 
распределений эмпирических данных. Методом генерации служит взятие 
последовательности эмпирических данных в~произвольном порядке. Для 
перестановочных тестов в~результате получается перемешивание, для бутстрепа же 
данные берутся <<с~возвращением>> (т.\,е.\ ка\-кие-то значения могут дублироваться). 
Используемым альтернативным вариантом метода Монте-Карло служит генерация 
псевдовыборок с~нормальным распределением.
  
  Одной из трудностей, имеющих место при поиске моделей, корректно описывающих 
данные, является проблема ложных регрессий~[6]. Этот эффект имеет место, когда 
расчет показывает достоверность регрессионной зависимости (например, высокие 
значения коэффициентов детерминации), хотя на самом деле такой закономерной 
зависимости нет. Явление ложной регрессии характерно для стохастически 
нестационарных процессов, например для случайных блужданий. В~таких\linebreak
 случаях для 
верификации зависимостей использовать стандартную статистику (например, 
t-ста\-ти\-сти\-ку) неправомерно, в~част\-ности из-за несоблюдения требования независимости 
отдельных \mbox{наблюдений}. Поэтому важ\-ную роль в~проверке достоверности регрессионной 
зависимости играет проверка временн$\acute{\mbox{ы}}$х рядов на стационарность. Известным способом 
такой проверки служит тест\linebreak Ди\-ки--Фул\-ле\-ра. Представляемое 
исследование на-\linebreak правлено 
на разработку методов верификации, применимых для регрессионных моделей, 
вклю\-чающих нестационарные переменные, т.\,е.\ переменные, значения которых 
являются элементами нестационарного временн$\acute{\mbox{о}}$го ряда. При моделировании 
учитывается также возможность отличия распределения ошибок от нормального.
   
    Интересным объектом для исследования методами Model selection пред\-став\-ля\-ют\-ся 
широко применимые в~экономике производственные функции, связывающие объем 
выпускаемой продукции с~факторами производства (классическими факторами 
выступают труд и~капитал). Простейшая\linebreak спецификация~--- функция Коб\-ба--Дуг\-ла\-са~--- 
посредством логарифмирования сводится к~модели линейной регрессии, для которой 
получено особенно много результатов, в~том чис\-ле с~применением непараметрических 
методов. За время исследований накопилось множество обобщений этой функции 
и~альтернативных спецификаций. Производственные функции применяются как на 
микроуровне экономики, так и~на мезо- и~макроуровне. В~данной работе они 
исследуются на мезоуровне~--- на уровне российских регионов. Их анализ позволяет 
лучше понять особенности конкретных регионов, прогнозировать их развитие 
и~управлять ими.
{ %\looseness=1

}

\section{Описание исследуемых моделей и~данных}

  Ранее в~[7] исследовались различные варианты производственных функциий, 
построенных по временн$\acute{\mbox{ы}}$м рядам, характеризующим развитие Российской Федерации 
в~целом. Выводы о~качестве моделей делались на основе вычисления обычных 
и~скорректированных коэффициентов детерминации. В~данной работе они исследуются 
для отдельных регионов Российской Федерации (по данным за 1996--2014~гг.). 
Поскольку удобно исследовать временн$\acute{\mbox{ы}}$е ряды фиксированной длины, в~расчетах не 
использовались данные для Чеченской республики, Крыма, Севастополя (они доступны 
за меньшее число лет). Регионы, входящие в~состав других субъектов Российской 
Федерации, а~не подчиняющиеся ей напрямую, также в~расчетах не участвовали.
  
  В предыдущей работе~[8] для производственных функций регионов проводилась 
оценка применимости классической модели функции Коб\-ба--Дуг\-ла\-са и~определялась 
роль трендов по времени в~обеспечении достоверности зависимости с~использованием 
псвевдовыборок в~виде гауссова белого шума и~случайных блужданий. В~данной 
работе методы Монте-Карло используются для сравнения альтернативных вариантов 
моделей, анализа достоверности отдельных переменных с~привлечением 
непараметрических методов (перестановочных тес\-тов и~бутстрепов).
  
  В литературе ведется дискуссия о~целесообразности использования 
в~производственных функциях в~качестве фактора производства \textit{инвестиций} 
вместо \textit{капитала}~[9, 10]. Это связано, в~част\-ности, с~тем, что для инвестиций 
проще оценивать объективные количественные значения. Кроме того, расчеты показали, 
что использование ин\-вес\-ти\-ций при аппроксимации исследуемых данных функцией 
Коб\-ба--Дуг\-ла\-са приводит к~более высоким значениям коэффициента детерминации, чем 
при использовании капитала. Поэтому (в отличие
 от~\cite{7-ki, 8-ki}) в~данной работе в~качестве исходной модели используется 
инвестиционная производственная функция. 
  
  Также, в~отличие от~\cite{7-ki, 8-ki}, в~переменную \textit{труд} производственной функции 
введена в~качестве сомножителя к~числу занятых в~экономике заработная плата, что 
существенно увеличило значимость труда в~исследуемых вариантах производственных 
функций. В~литературе инвестиционные производственные функции с~трудом, 
выражаемым через зарплату, были представлены ранее, например в~[10].
  
  Основная используемая в~статье модель (<<инвестиционный>> аналог функции 
Коб\-ба--Дуг\-ла\-са) выражается формулой:
  \begin{equation}
  Y=A I^\alpha L^\beta\,,
  \end{equation}
  где $Y$~--- валовой региональный продукт;
  $I$~--- инвестиции в~основной капитал;
  $L$~--- среднегодовая численность занятых в~экономике, помноженная на 
среднемесячную номинальную начисленную заработную плату работающих 
в~экономике; 
  $A$, $\alpha$ и~$\beta$~--- вычисляемые параметры.
  
  Иногда на (1) накладывается условие $\alpha\hm+\beta\hm=1$, но здесь это условие не 
используется.
  
  Данные взяты из~[11] и~аналогичных сборников\linebreak
   за 2003, 2005 и~2012~гг. Некоторые 
данные за одинаковые годы в~разных сборниках слегка различаются, что, видимо, 
обусловлено внесением поправок статистическими органами, в~таких случаях\linebreak 
использовались данные из более поздних сборников. Было использовано 
приведение~$Y$, $L$ и~$I$ к~постоянным ценам с~по\-мощью индексов потребительских 
цен (в~\cite{7-ki} использовался ин\-декс-дефля\-тор валового внут\-рен\-не\-го
продукта, а~в~\cite{8-ki}~--- индекс 
физического объема валового регионального продукта).
  
  Посредством логарифмирования переменных из~(1) получаем линейное соотношение
  \begin{equation}
  \ln Y=\ln A+\alpha \ln I+\beta \ln L\,,
  \end{equation}
которое теперь можно исследовать методами линейного регрессионного анализа.
  
  Однако имеет смысл проверять альтернативные гипотезы, например о~том, что 
производственная функция достоверно зависит только от инвестиций или только от 
труда: 
  \begin{align}
  \ln Y&= \ln A+\alpha \ln I\,;\label{e-ki}\\
  \ln Y &= \ln A+\beta \ln L\,. \label{e4-ki}
  \end{align} 

\section{Расчеты по~альтернативным моделям производственных 
функций}

  Исследуемый в~статье набор данных в~терминах эконометрики является частным 
случаем панельных данных (т.\,е.\ данных, описывающих совокупность однотипных 
объектов, каждый из которых характеризуется некоторым числом временн$\acute{\mbox{ы}}$х рядов 
показателей). Для них разработаны особые методы исследования, в~частности методы 
выяснения того, есть ли у~объектов закономерные индивидуальные признаки (модель 
с~фиксированными эффектами) или они характеризуются единым набором признаков со 
случайными отклонениями (модель со случайными эффектами). Этот подход ранее 
использовался для оценки производственных функций регионов Российской 
Федерации~[12] (на основе иного набора данных) и~показал, что они\linebreak
 описываются 
моделью с~фиксированными эф\-фектами. Проведенные авторами расчеты также 
подтвердили справедливость этой модели. Это делает более обоснованными 
исследования производственных функций регионов независимо друг от друга. 
  
  Классические методы оценки достоверности, применяемые в~регрессионном анализе 
(вычисление $R^2$,  
$p$-зна\-че\-ния для модели в~целом с~использованием F-ста\-ти\-сти\-ки и~для 
коэффициентов при\linebreak
 отдельных переменных с~использованием t-ста\-тисти\-ки 
Стьюдента), показывают высокую достоверность получаемых зависимостей, однако из 
литературы известно, что применение стандартных методов верификации 
к~регрессионным моделям приводит к~возникновению ложных регрессий~--- формально 
весьма достоверных, но фактически бессмысленных.
  
  Важная причина возникновения ложных регрессий кроется в~несправедливости для 
временн$\acute{\mbox{ы}}$х рядов предположения о~взаимной не\-за\-ви\-си\-мости\linebreak
 отдельных наблюдений. 
Избежать подобного предположения позволяют методы верификации, основанные на 
искусственной генерации псевдовыборок в~соответствии с~предполагаемой нулевой 
\mbox{гипотезой}. В~этом случае генерация данных, в~част\-ности, будет производиться 
с~использованием предположения о~соответствии исследуемого временн$\acute{\mbox{о}}$го ряда 
процессу случайного блуж-\linebreak дания.
  
  Для оценки достоверности (1)--(4) и~сравнения моделей были рассмотрены 
следующие модели генерации данных методами Монте-Карло.
  \begin{enumerate}[1.]
  \item Для совокупности используемых показателей генерировалось по 
5000~псевдовыборок по формуле
  \begin{equation}
  X_t=e_t\,,
  \label{e5-ki}
  \end{equation}
  где $e_t$~--- гауссов белый шум. Известно, что он характеризуется средним значением 
и~дисперсией, но значение коэффициентов детерминации для его реализаций конечной 
длины от них не зависит и~определяется числом наблюдений во временн$\acute{\mbox{ы}}$х рядах 
(поэтому среднее и~дисперсия могут устанавливаться произвольно, в~данном случае они 
выбирались в~соответствии с~эмпирическими данными для регионов).
  
  \item Генерировалось по 5000~псевдовыборок по формуле
  \begin{equation}X_t=X_{t-1}+e_t\,,
  \label{e6-ki}
  \end{equation}
  где $e_t$~--- гауссов белый шум с~нулевым средним и~дисперсией, соответствующей 
дисперсии эмпирических рядов данных. Процесс~(\ref{e6-ki}) называется случайным 
блужданием и~относится к~нестационарным процессам.
  \item Генерировалось по 5000~псевдовыборок, для получения которых внутри 
используемых эмпирических временн$\acute{\mbox{ы}}$х рядов данных происходило перемешивание 
случайным образом порядка\linebreak\vspace*{-12pt}

\pagebreak

\item[ ] их значений во времени (этот подход называется 
перестановочными тестами).\\[-8pt]
  \item Генерировалось по 5000~псевдовыборок, где внутри используемых 
эмпирических временн$\acute{\mbox{ы}}$х рядов данных происходило перемешивание случайным 
образом порядка
 их значений %\linebreak
 во времени, но при этом (в отличие от предыду\-ще\-го 
случая) некоторые значения могли повторяться за счет других данных (выборка 
с~возвращением, может интерпретироваться как \mbox{бутстреп}).\\[-8pt]
  \item Генерировалось по 5000~псевдовыборок, где внут\-ри используемых 
эмпирических вре\-мен\-н$\acute{\mbox{ы}}$х рядов данных происходило перемешивание случайным 
образом порядка приращений %\linebreak
 во времени, при этом некоторые значения могли 
повторяться за счет других данных. Дополнительно проводилось исключение влияния 
прироста значений величин реальных \mbox{данных} за весь период наблюдений.
  \end{enumerate}
  
  Перестановочного теста для приращений не проводилось, поскольку его 
использование остав\-ля\-ет инвариантным суммарное приращение величин за период 
наблюдений, что вносит искажения в~результаты.
  
  Использовалась следующая методология верификации: величины параметра~$R^2$, 
характеризующего точность регрессионной модели на искусственно сгенерированных 
выборках, сравнивались с~величинами на истинной выборке. Для полученных каждым 
из пяти способов псевдовыборок вычислялись коэффициенты детерминации, 
получающиеся при аппроксимации данных ис\-сле\-ду\-емы\-ми моделями~(2)--(4). Затем 
5000~полученных коэффициентов детерминации ранжировались по величине 
и~реальное значение~$R^2$ сравнивалось с~250-м по величине значением от 
максимального среди полученных для псевдовыборок (которое\linebreak далее будем называть 
95\%-ным квантилем эмпирического распределения, или просто 95\%-ным квантилем). 
Это способ оценки достоверности полученного результата на уровне 95\% (в~данном 
случае того, что значение~$R^2$ получено не случайно, а~выражает некоторую 
закономерность, если реальное~$R^2$ больше 250-го по величине псевдовыборочного).
{ %\looseness=1

}

  Можно считать, что для конкретных регионов, для которых значение~$R^2$ 
превышает соответствующее значение 95\%-ного квантиля, выявлена закономерность, 
достоверная на уровне ниже~0,05. Однако авторам доступна информация 
о~79~регионах, и~это дает дополнительную возможность проверить, не является ли 
выявленная закономерность на самом деле случайной (такая проб-\linebreak\vspace*{-12pt}


{ \begin{center}  %fig1
 \vspace*{-1pt}
    \mbox{%
 \epsfxsize=79mm %81.035mm 
 \epsfbox{kir-1.eps}
 }

\end{center}

\noindent
{{\figurename~1}\ \ \small{Ранговое распределение $R^2$, рассчитанных по формуле~(2), 
в~сравнении с~95\%-ными квантилями~$R^2$ для симуляций, расчитанных по 
формулам~(5) и~(6): \textit{1}~--- реальные значения~$R^2$; 
\textit{2} и~\textit{3}~--- 95\%-ные квантили эмпирического распределения $R^2$ для 
псевдовыборок, генерируемых процессами~(\ref{e5-ki}) и~(\ref{e6-ki})
соответственно
}}}

\vspace*{12pt}


\noindent
ле\-ма становится особо 
актуальной при множественном тестировании). Если, например, лишь для одного 
региона из~79~для какой-либо модели превзойден 95\%-ный квантиль, то достоверность 
выявленной закономерности значительно меньше, чем когда он превзойден для всех 
регионов. 
{\looseness=1

}


 Классические оценки по F-критерию соответствия модели~(2) реальным данным 
показывают высокие значения коэффициента детерминации (рис.~1). Они варьируют 
в~диапазоне от~0,994 для Дагестана до~0,783 для Мурманской области. 
Соответствующие $p$-значения варьируют в~диапазоне ($1{,}55\cdot 10^{-18}$; 
$4{,}96\cdot 10^{-6}$). При этом  
95\%-ный квантиль для сгенерированных согласно формуле~(\ref{e5-ki}) псевдовыборок равен 
примерно~0,313, а~для псевдовыборок, сгенерированных 
согласно формуле~(\ref{e6-ki}),~---
0,813. Только для двух регионов из~79 имеет место соотношение $R^2\hm< 
0{,}813$.

 Описываемая моделью зависимость в~данных, таким образом, не может быть 
полностью объяснена не только чистой случайностью, но и~ложной регрессией, 
обусловленной не\-ста\-ци\-о\-нар\-ностью вида~(\ref{e6-ki}) рас\-смат\-ри\-ва\-емых процессов, из 
чего можно сделать вывод о~до\-сто\-вер\-ности за\-ко\-но\-мер\-ности, вы\-ра\-жа\-емой 
соотношениями~(1)--(2) на уровне меньше~0,05. Результат, полученный 
с~использованием~(\ref{e5-ki}) соответствует обыч\-но\-му $p$-зна\-че\-нию для~$R^2$, 
вы\-чис\-ля\-емо\-му в~стандартных пакетах в~рамках параметрической статистики, 
а~использование~(\ref{e6-ki}) позволяет получить более надежные результаты в~оценке 
наличия закономерностей с~учетом возможной нестационарности исследуемых 
процессов. В~качестве альтернативы~(\ref{e5-ki}) в~вы\-чис\-ле\-ни\-ях использовались бутстрепы, но 
они дали близкие к~полученным для псевдовыборок, сгенерированных по моделям~(\ref{e5-ki}) 
и~(\ref{e6-ki}), результаты: соответственно $R^2\hm\approx 0{,}313$ (как и~перестановочный 
тест) и~$R^2\hm\approx 0{,}87$~--- и~не привели к~качественно новым выводам 
о~до\-сто\-вер\-ности ис\-сле\-ду\-емых моделей.
{\looseness=1

}
  
  Тест Дики--Фуллера в~целом формально показывает нестационарность временн$\acute{\mbox{ы}}$х 
рядов и~остатков модели~(2) по~19~значениям во времени, но для получения 
достоверных результатов не хватает данных (ряды должны были бы быть хотя бы 
в~несколько раз длиннее), и~потому используется сравнение реальных результатов со 
сгенерированными по формуле~(6) в~качестве альтернативы этому тесту, пригодной для 
малых выборок. 
  
  Возникает вопрос, не описываются ли данные с~настолько приемлемой точ\-ностью 
более прос\-ты\-ми соотношениями, например формулами~(3) или~(4), что усложнение 
модели до~(2) может оказаться нецелесообразным. Дальнейший анализ выявляет 
подходы к~ответу на этот вопрос.
   
  Достоверность отличия от нуля отдельных коэффициентов~~--- $\ln A$, $\alpha$, 
$\beta$~--- в~формуле~(2) по классическому t-кри\-те\-рию Стьюдента следующая: 
условие $p\hm < 0{,}05$ выполняется для $\ln A$ у~41~региона, для~$\alpha$~--- у~37 
регионов и~для $\beta$~--- у~60~регионов.
{\looseness=1

}
  
  Зависимость, определяемая формулой~ (3), характеризуется для используемых 
в~статье данных коэффициентами детерминации в~диапазоне $(0{,}346; 0{,}986)$. При 
этом 95\%-ные квантили эмпирического распределения~$R^2$ для псевдовыборок, 
генерируемых по формулам~(5) и~(6), равны соответственно~0,209 и~0,707. Для четырех 
регионов из~79 имеет место соотношение $R^2\hm< 0{,}707$.
{\looseness=1

}
  
  Зависимость~(4) характеризуется $R^2$ в~диапазоне (0,62; 0,99), 
  95\%-ные квантили $R^2$ для 
(\ref{e5-ki}) и~(\ref{e6-ki}) те же, что для~(3). 
Лишь для одного региона имеет мес\-то соотношение $R^2\hm< 
0{,}707$.
  Итак, модели~(3) и~(4) сами по себе с~весьма высокой достоверностью описывают 
данные, хотя достоверность на уровне меньше~0,05 отличия от нуля коэффициентов 
формулы~(2) по t-кри\-те\-рию для значительного числа регионов не 
имеет мес\-та. Это 
связано с~наличием мультиколлинеарности, корреляции между~$L$ и~$I$. Существенный 
вклад в~мультиколлинеарность вносит наличие общего тренда по времени, значение 
которого (для несколько другого набора показателей и~с~применением другого подхода) 
обсуждалось в~[5]. В~следующем разделе описаны способы оценки целесообразности 
для описания исследуемого набора данных усложнения моделей~(3) и~(4) 
до модели~(2).
{ %\looseness=1

}

\section{Использование метода генерации псевдовыборок для~оценки 
целесообразности усложнения модели}

\vspace*{-13pt}

  Объединенная модель~(2) статистически достоверна, как и~более простые модели~(3) 
и~(4). Но существует ли при этом комбинированный, синергетический эффект, можно ли 
доказать, что объединенная модель действительно лучше? Подходы к~ответу на этот 
вопрос будут приведены в~данном разделе.
  
  Опишем алгоритм, применявшийся для оценки целесообразности использования для 
описания динамики~$Y$ обоих факторов производства~--- $I$ и~$L$. Поскольку 
обычный~$R^2$ может только увеличиваться при введении в~модель новой переменной, 
даже при использовании в~качестве всех данных случайно сгенерированных 
псевдовыборок расчет по модели с~дополнительной имитируемой случайным процессом 
переменной даст некоторый прирост в~значении~$R^2$. Прирост будет и~при 
добавлении такой переменной к~реальным данным. Рассчитав эти приросты для 
достаточно большого числа псевдовыборок и~определяя квантили распределения 
приростов, можно оценить, насколько достоверно усложненный вариант модели 
в~действительности лучше описывает ситуацию по сравнению с~исходной моделью, 
насколько вероятно, что увеличение значения~$R^2$ (будем обозначать его~$\Delta 
R^2$) не является артефактом.
  
  Было проведено сравнение <<длинной>> модели~(2) с~<<короткими>> моделями~(3) 
и~(4). Набор из 5000~имитируемых длинных моделей рассчитывался на основе 
 реальных данных короткой модели и~случайных 
псевдовыборок,
описываемых формулами~(\ref{e5-ki}) и~(\ref{e6-ki}) и~имитирующих реальные данные той из переменных длинной 
модели~(2), которая отсутствует в~сравниваемой короткой модели. Для полученного 
набора вычислялись значения коэффициентов детерминации. Затем значения~$R^2$ 
<<короткой модели>>~(3) или~(4), посчитанные для реальных данных, вычитались из 
соответствующих значений~$R^2$ <<длинной модели>>~(2) и~определялись 95\%-ные 
квантили полученных наборов~$\Delta R^2$. Они сравнивались с~реальными 
значениями~$\Delta R^2$ (т.\,е.\ с~такими разностями, где и~в~короткой, и~в~длинной моделях все 
данные реальны).
  


  Расчеты по сравнению моделей~(2) и~(3) показали, что для 61~региона реальные 
значения $\Delta R^2$ больше 95\%-ных квантилей симуляций~(\ref{e5-ki}) и~для 44~регионов 
больше 95\%-ных квантилей симуляций~(\ref{e6-ki}).

\pagebreak

\end{multicols}

\setcounter{figure}{1}
\begin{figure*} %fig2
\vspace*{1pt}
 \begin{center}
 \mbox{%
 \epsfxsize=151.133mm 
 \epsfbox{kir-2.eps}
 }
 \end{center}
   \vspace*{-9pt}
\Caption{Ранговое распределение разностей $\delta\hm = (\Delta R^2_{\mathrm{реал}} \hm- \Delta 
R^2_{250})$ реальных значений~$\Delta R^2$, рассчитанных по формулам~(2) и~(3)~(\textit{а}) и~(2) 
и~(4)~(\textit{б}), и~разностей 95\%-ных квантилей эмпирических распределений $\Delta R^2$ для 
симуляций, рассчитанных по формулам~(\ref{e5-ki}) (левый столбец) 
и~(\ref{e6-ki}) (правый столбец) для случаев зависимостей от одной 
и~двух переменных}
\vspace*{6pt}
\end{figure*}

\begin{multicols}{2}
   
  Сравнение моделей~(2) и~(4) показало, что значения $\Delta R^2$ для 38~регионов 
больше 95\%-ных квантилей симуляций~(\ref{e5-ki}) и~для 13~регионов больше 95\%-ных 
квантилей симуляций~(\ref{e6-ki}).





 Как видно из результатов расчетов, пред\-став\-лен\-ных на рис.~2, сравнение 
с~симуляциями, основанными на модели~(\ref{e6-ki}), не позволяет сделать вывод со 
значимостью на уровне $p\hm< 0{,}05$ об информативности переменной~$I$ 
(инвестиции) для~34 из 78~регионов. Симуляции по модели~(6) не позволяют сделать 
вывод со значимостью на уровне $p\hm< 0{,}05$ об информативности переменной~$L$ 
(труд) уже для~65 из 78~регионов.
  
  В описанных выше расчетах информация о~каж\-дом из регионов использовалась 
независимо от информации о~других регионах. Вместе с~тем для каждого 
отдельного региона вывод основывается на анализе весьма короткого 
временн$\acute{\mbox{о}}$го ряда. Совместное использование информации обо всех регионах 
может дать более достоверные результаты об информативности переменных~$I$ 
и~$L$ в~целом безотносительно к~каждому конкретному региону. Поэтому были 
проведены дополнительные расчеты. Были вычислены суммы~$\Delta R^2$ по всем 
регионам для пар моделей~(2),~(3) и~(2),~(4). 
Такие же суммы были вычислены в~каждом случае для 5000~наборов из 
79~псевдовыборок, где переменная, не вошедшая в~короткую модель, 
генерировалась по формулам~(5) и~(6). Полученные суммы для реальных данных 
сравнивались с~95\%-ными квантилями эмпирического распределения разностей 
коэффициентов 
детерминации псевдовыборок. Для обеих пар моделей была показана высокая 
достоверность исследуемых разностей~$\Delta R^2$. Для случая с~короткой 
моделью~(3) получены интервалы сумм~$\Delta R^2$ псевдовыборок по формуле~(5)~--- 
$ (0{,}28; 1{,}24)$ и~по формуле~(6) ~--- $(0{,}99; 3{,}23)$ при реальном 
значении~6,02. Для случая с~короткой моделью~(4) получены интервалы сумм~$\Delta 
R^2$ псевдовыборок по формуле~(5)~--- $ (0{,}15; 0{,}70)$ и~по формуле~~(6)--- 
$(0{,}42; 1{,}37)$, при реальном значении~1,57. Таким образом,
 в~обоих случаях имеет место 
высокая достоверность, оцениваемая по крайней мере на уровне $p\hm < 0{,}0002$.
  
  Все использованные в~статье расчеты проводились с~помощью программ, 
написанных на язы-\linebreak ке~$R$.

%\vspace*{-24pt}

\section{Заключение}

  Использованные в~статье методы оценки целесообразности усложнения моделей 
имеют преимущества перед классической методологией оценки достоверности регрессии 
по F- и~t-ста\-ти\-сти\-кам\linebreak и~могут быть использованы для широкого класса моделей. 
Преимущество заключается, в~част\-ности, в~том, что легко учитывается эффект 
нестационарности, а~также существуют \mbox{непараметрические} варианты методов. 
В~исследованном случае перестановочные тесты, бутстрепы и~симуляции 
с~\mbox{использованием} нормального распределения дали близкие результаты, но такая 
ситуация не является общей, непараметрические методы имеют более широкую сферу 
применимости. Представляет интерес интеграция методов исследования нестационарных 
процессов непараметрическими методами с~существующими подходами 
к~исследованию панельных данных, что предполагается возможным предметом 
дальнейших исследований.
  
{\small\frenchspacing
 {%\baselineskip=10.8pt
 \addcontentsline{toc}{section}{References}
 \begin{thebibliography}{99}
  
  \bibitem{1-ki}
  \Au{Стрижов В.\,В., Крымова Е.\,А.} Методы выбора регрессионных моделей.~--- М.: ВЦ РАН, 2010. 
60~с.
  \bibitem{2-ki}
  \Au{Kennedy P.\,E., Cade B.\,S.} Randomization tests for multiple 
  regression~// Commun. 
Stat. Simul.~C., 1996. Vol.~25. Iss.~4. P.~923--936.
  \bibitem{3-ki}
  \Au{Anderson M.\,J., Robinson~J.} Permutation tests for linear models~// 
  Aust. NZ J.~Stat., 2001. Vol.~43. Iss.~1.  
P.~75--88.
  \bibitem{4-ki}
  \Au{Senko O.\,V., Dzyba D.\,S., Pigarova~E.\,A., Rozhinskaya~L.\,Ya., Kuznetsova~A.\,V.} A~method for 
evaluating validity of piecewise-linear models~// Conference (International) on Knowledge Discovery and 
Information Retrieval Short Papers.--- Scitepress, 2014. P.~437--443.
doi: 10.5220/0005156904370443.
  \bibitem{5-ki}
  \Au{Skrobotov А.} On bootstrap implementation of likelihood 
  ratio test for a~unit root~// Econ. Lett., 
2018. Vol.~171(C). P.~154--158.
  \bibitem{6-ki}
  \Au{Granger C.\,J., Newbold P.} Spurious regressions in econometrics~// J.~Econometrics, 1974. Vol.~2. 
Iss.~2. P.~111--120.
  \bibitem{7-ki}
  \Au{Кирилюк И.\,Л.} Модели производственных функций для российской экономики~// 
Компьютерные исследования и~моделирование, 2013. Т.~5. №\,2. С.~293--312.
  \bibitem{8-ki}
  \Au{Кирилюк И.\,Л., Сенько~О.\,В.} Исследования соотношений между нестационарными 
временными рядами на примере производственных функций~// Машинное обучение и~анализ данных, 
2018. Т.~4. №\,3. С.~142--151.
   \bibitem{9-ki}
  \Au{Поспелов И.\,Г., Поспелова~И.\,И., Хохлов~М.\,А., Шипулина~Г.\,Е.} Новые принципы и~методы 
разработки макромоделей экономики и~модель современной экономики России.~--- М.: ВЦ РАН, 2006. 
242~с.
  \bibitem{10-ki}
  \Au{Гребнев М.\,И.} Построение производственных функций регионов России~// ВУЗ. XXI~век, 2015. 
№\,2. С.~50--56.
  \bibitem{11-ki}
  Регионы России. Социально-экономические показатели. 
  2017.~--- М.: Росстат, 2017. 1402~с.
  {\sf https://gks.ru/bgd/regl/B17\_14p/Main.htm}.
  \bibitem{12-ki}
  \Au{Бахитова Р.\,Х., Ахметшина~Г.\,А., Лакман~И.\,А.} Панельное моделирование объема выпуска 
продукции для регионов России~// Управление большими системами, 2014. Вып.~50. С.~99--109.
\end{thebibliography}

 }
 }

\end{multicols}

\vspace*{-6pt}

\hfill{\small\textit{Поступила в~редакцию 22.05.19}}

\vspace*{8pt}

%\pagebreak

%\newpage

%\vspace*{-28pt}

\hrule

\vspace*{2pt}

\hrule

%\vspace*{-2pt}

\def\tit{SELECTION OF OPTIMAL COMPLEXITY MODELS BY~METHODS OF~NONPARAMETRIC STATISTICS 
(ON~THE~EXAMPLE OF~PRODUCTION FUNCTION MODELS OF~THE~REGIONS OF~THE~RUSSIAN FEDERATION)}


\def\titkol{Selection of optimal complexity models by~methods 
of~nonparametric statistics 
(on~the~example of~production function models)} % of~the
%regions of~the~Russian Federation)}

\def\aut{I.\,L.~Kirilyuk$^1$ and O.\,V.~Sen'ko$^2$}

\def\autkol{I.\,L.~Kirilyuk and O.\,V.~Sen'ko}

\titel{\tit}{\aut}{\autkol}{\titkol}

\vspace*{-9pt}


\noindent
$^1$Institute of Economics of the Russian Academy of Sciences, 32~Nakhimovskiy Pr., Moscow 117218, 
Russian\linebreak
$\hphantom{^1}$Federation

\noindent
$^2$Federal Research Center ``Computer Science and Control'' of the Russian Academy of Sciences, 
44-2~Vavilov\linebreak
$\hphantom{^1}$Str., Moscow 119333, Russian Federation


\def\leftfootline{\small{\textbf{\thepage}
\hfill INFORMATIKA I EE PRIMENENIYA~--- INFORMATICS AND
APPLICATIONS\ \ \ 2020\ \ \ volume~14\ \ \ issue\ 2}
}%
 \def\rightfootline{\small{INFORMATIKA I EE PRIMENENIYA~---
INFORMATICS AND APPLICATIONS\ \ \ 2020\ \ \ volume~14\ \ \ issue\ 2
\hfill \textbf{\thepage}}}

\vspace*{3pt} 
  
\Abste{The article describes an approach to comparing alternative variants 
of linear regression models on time series
and determining the appropriateness of 
complicating them (by adding new variables) using several variants of 
Monte-Carlo methods. The proposed research methods using pseudosampling 
generation allow taking into account both the effects associated with 
possible differences of distributions in empirical data from the Gauss 
distribution and the effects associated with possible nonstationarity 
of the time series under study. For this purpose,
pseudosampling 
generation is used~--- time series, which are Gaussian white noise, 
random walk generation, as well as the permutation test and the bootstrap 
method. Reliability of the obtained results is estimated using resampling.\linebreak\vspace*{-12pt}}

\Abstend{Applicability of the considered methods is demonstrated by the example 
of models of investment production functions of regions of the Russian 
Federation, calculated on the basis of data from the Federal State 
Statistics Service.}

\KWE{Monte-Carlo methods; permutation tests; spurious regression; 
production functions; model selection; meso level of the economy}

\DOI{10.14357/19922264200216} 

%\vspace*{-20pt}

\Ack
\noindent
The research was carried out within the framework of the state task 
``The phenomenon of mesolevel in economic analysis: New theories and 
their practical application.''


%\vspace*{6pt}

 \begin{multicols}{2}

\renewcommand{\bibname}{\protect\rmfamily References}
%\renewcommand{\bibname}{\large\protect\rm References}

{\small\frenchspacing
 {%\baselineskip=10.8pt
 \addcontentsline{toc}{section}{References}
 \begin{thebibliography}{99}

\bibitem{1-ki-1}
\Aue{Strizhov, V.\,V., and E.\,A.~Krymova.} 2010. \textit{Metody 
vybora regressionnykh modeley} [Methods 
for choosing regression models]. Moscow: CC RAS. 60~p.
\bibitem{2-ki-1}
\Aue{Kennedy, P.\,E., and B.\,S. Cade.} 1996. Randomization tests for 
multiple regression. 
\textit{Commun. Stat. Simul.~C.} 25(4):923--936.
\bibitem{3-ki-1}
\Aue{Anderson, M.\,J., and J.~Robinson.} 2001. Permutation tests for 
linear models. \textit{Aust. NZ J.~Stat.} 43(1):75--88.
\bibitem{4-ki-1}
\Aue{Senko, O.\,V., D.\,S. Dzyba, E.\,A.~Pigarova, L.\,Ya.~Rozhinskaya, 
and A.\,V.~Kuznetsova.} 2014. 
A~method for evaluating validity of piecewise-linear models. 
\textit{Conference (International) on Knowledge Discovery and 
Information Retrieval  Short Papers}. 
Scitepress. 437--443. doi: 10.5220/0005156904370443.
\bibitem{5-ki-1}
\Aue{Skrobotov, А.} 2018. On bootstrap implementation of likelihood 
ratio test for a~unit root. 
\textit{Econ. Lett.} 171(C):154--158.
\bibitem{6-ki-1}
\Aue{Granger, C.\,J., and P.~Newbold.} 1974. Spurious regressions 
in econometrics. \textit{J.~Econometrics} 2(2):111--120.
\bibitem{7-ki-1}
\Aue{Kirilyuk, I.\,L.} 2013. Modeli proizvodstvennykh funk\-tsiy 
dlya rossiyskoy ekonomiki [Models of 
production functions for the Russian economy]. \textit{Komp'yuternye 
issledovaniya i~modelirovanie} 
[Computer Research Modeling] 5(2):293--312.
\bibitem{8-ki-1}
\Aue{Kirilyuk, I.\,L., and O.\,V.~Senko.} 2018. Issledovaniya 
sootnosheniy mezhdu nestatsionarnymi vremennymi ryadami na primere 
proizvodstvennykh funktsiy [Studies of the relationship between 
nonstationary time series on the example of production functions]. 
\textit{Mashinnoe Obuchenie i~Analiz 
Dannykh} [Machine Learning Data Analysis] 4(3):142--151.
 \bibitem{9-ki-1}
\Aue{Pospelov, I.\,G., I.\,I. Pospelova, M.\,A.~Khokhlov, and G.\,E.~Shipulina.}
 2006. \textit{Novye printsipy i metody razrabotki makromodeley 
 ekonomiki i~model' sovremennoy ekonomiki Rossii} [New principles and 
methods for developing macromodels of the economy and a~model 
of the modern economy of Russia]. Moscow: CC RAS. 242~p.
\bibitem{10-ki-1}
\Aue{Grebnev, M.\,I.} 2015. Postroenie proizvodstvennykh funk\-tsiy 
regionov Rossii [Construction of production functions of Russian regions]. 
\textit{VUZ. XXI~vek} [High School. XXI~Century] 2:50--56.
\bibitem{11-ki-1}
Regiony Rossii. Sotsial'no-ekonomicheskie pokazateli. 2017 [Regions of Russia. 
Socio-economic indicators. 2017]. Moscow: Rosstat. 1402~p. Available at: 
{\sf https://gks.ru/bgd/regl/B17\_14p/Main.htm} (accessed May~1, 2020).
\bibitem{12-ki-1}
\Aue{Bakhitova, R.\,Kh., G.\,A.~Akhmetshina, and I.\,A.~Lakman.} 2014. 
Panel'noe modelirovanie ob''ema vypuska produktsii dlya regionov Rossii 
[Panel modeling of production output in Russian regions]. 
\textit{Upravlenie bol'shimi sistemami} [Large-Scale Systems Control] 50:99--109.

\end{thebibliography}

 }
 }

\end{multicols}

\vspace*{-9pt}

\hfill{\small\textit{Received May 22, 2019}}

%\pagebreak

%\vspace*{-24pt}

\Contr

\noindent
\textbf{Kirilyuk Igor L.} (b.\ 1974)~--- scientist, Institute of Economics of the Russian Academy of 
Sciences, 32~Nakhimovskiy Pr., Moscow 117218, Russian Federation; \mbox{igokir@rambler.ru}

\vspace*{3pt}

\noindent
\textbf{Sen'ko Oleg V.} (b.\ 1957)~--- Doctor of Science in physics and mathematics, academician of RAS, 
leading scientist, Federal Research Center ``Computer Science and Control'' of the Russian Academy of 
Sciences, 44-2~Vavilov Str., Moscow 119333, Russian Federation; \mbox{senkoov@mail.ru}
  
\label{end\stat}

\renewcommand{\bibname}{\protect\rm Литература} 
  