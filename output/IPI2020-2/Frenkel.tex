\def\stat{frenkel}

\def\tit{СОВМЕСТНАЯ ОЦЕНКА ПРЕДСКАЗУЕМОСТИ ДАННЫХ 
И~КАЧЕСТВА ПРЕДИКТОРОВ$^*$}

\def\titkol{Совместная оценка предсказуемости данных 
и~качества предикторов}

\def\aut{С.\,Л.~Френкель$^1$, В.\,Н.~Захаров$^2$}

\def\autkol{С.\,Л.~Френкель, В.\,Н.~Захаров}

\titel{\tit}{\aut}{\autkol}{\titkol}

\index{Френкель С.\,Л.}
\index{Захаров В.\,Н.}
\index{Frenkel S.\,L.}
\index{Zakharov V.\,N.}
 

{\renewcommand{\thefootnote}{\fnsymbol{footnote}} \footnotetext[1]
{Работа выполнена при частичной финансовой поддержке РФФИ (проекты 18-07-00669,
18-29-03100 и~18-07-01434).}}


\renewcommand{\thefootnote}{\arabic{footnote}}
\footnotetext[1]{Федеральный исследовательский центр <<Информатика 
и~управление>> Российской академии наук, \mbox{fsergei51@gmail.com}}
\footnotetext[2]{Федеральный исследовательский центр <<Информатика 
и~управление>> Российской академии наук, \mbox{vzakharov@ipiran.ru}}

%\vspace*{-6pt}


  \Abst{Предлагается и~анализируется новый подход к~выбору предикторов, 
необходимых для предсказания будущих значений в~последовательностях данных 
в~конкретный временной период.  
Цель~--- недорогие реализуемые техники, обеспечивающие выбор приемлемого 
предиктора для текущего сеанса предсказания или же принятия решения 
о~невозможности выполнить надежный прогноз, если обнаружится, что данный участок 
последовательности не обладает свойством предсказуемости. Для этого предсказуемость 
последовательности определяется как максимальная по множеству доступных 
предикторов условная вероятность правильного предсказания при данном наборе 
наблюденных значений. Выбор (отсеивание) предикторов выполняется как по величине 
оценки данной вероятности, так и~по степени отличия конкретного предиктора от 
предиктора, оптимального для предсказаний следующего исхода последовательности 
испытаний Бернулли.}
   
  \KW{предсказание случайных последовательностей; предикторы; анализ данных}
  

\DOI{10.14357/19922264200206} 
 
%\vspace*{9pt}


\vskip 10pt plus 9pt minus 6pt

\thispagestyle{headings}

\begin{multicols}{2}

\label{st\stat}
  
  
  
\section{Введение}
 
  В настоящее время значительное число практических задач в~области 
экономического и~финансового управления, медицины, управления 
вычислительными и~коммуникационными сетями решается 
с~использованием различных математических методов прогнозирования. 
  
  Примером служит задача онлайн-прог\-но\-зи\-ро\-ва\-ния, выполняемого на 
основе больших наборов ранее накопленных данных, используемых 
в~алгоритмах \textit{машинного обучения} (МО, \textit{англ}.\ ML~--- Machine 
Learning), с~помощью готовых программных инструментов предсказания 
(далее~--- \textit{предикторов}), например для предсказания объемов 
рассматриваемого трафика в~определенные моменты в~будущем или 
возможных вредоносных атак~[1]. Целью может быть прогноз для 
администратора сети, чтобы тот успел начать действовать еще до того, как 
возникнет проблема. 
  
  Часто онлайн-работа предикторов требует настройки их параметров, что 
влияет на суммарные вычислительные затраты всей схемы управления сетью. 
Разработчик подсистемы прогнозирования системы управления трафиком 
и~безопасности должен предусмотреть возможность оценки ожидаемого 
качества и~возможных затрат на прогнозирование в~каждый требуемый 
момент, когда работает подсистема прогнозирования, чтобы избежать 
лишних затрат. Для этого необходим инструмент оценки эффективности 
прогнозирования на данном временном интервале. Качество прогноза 
в~некоторый момент времени зависит от того, насколько правильно 
используемый предиктором алгоритм на основе данных в~прошлом отражает 
ситуацию в~будущем. 
  %
  Поэтому важными становятся задачи разработки моделей предсказуемости 
данных, в~которых <<предсказуемость>> (predictability) выступает как одна 
из характеристик набора данных, и~задачи оценки прогностической 
способности предиктора для набора данных с~теми или иными 
характеристиками, поддающимися измерению (оценке) в~тре\-бу\-емый период 
времени. В~настоящее время можно выделить два основных подхода 
к~решению этих задач: ве\-ро\-ят\-ност\-но-ста\-ти\-сти\-че\-ский 
и~ло\-ги\-ко-се\-ман\-ти\-че\-ский.
  
  Предметом исследований предсказания поведения систем с~помощью 
семантических моделей выступают в~значительной степени языки описания 
предметной области, для которой будут построены модели предсказания, 
и~теории, на основе которых выполняются предсказания~[2]. 
  
  Будем рассматривать ве\-ро\-ят\-ност\-но-ста\-ти\-сти\-че\-ские подходы 
как наиболее распространенные в~современных инструментальных 
средствах прогнозирования. В~рамках этого подхода наборы данных 
(например, записи трафика в~тех или иных терминах) моделируются как 
случайные последовательности или процессы. 
  
  Итак, рассматривается решение следующей проблемы: 
  \begin{itemize}
  \item имеется множество последовательно рас\-смат\-ри\-ва\-емых данных; 
  \item имеется доступный набор предикторов, реализующих различные 
алгоритмы предсказания для этой последовательности.
  \end{itemize}
   
  Как осуществить выбор предиктора для предсказания следующего 
значения элемента последовательности, используя только результаты 
предсказаний на предыдущих элементах, и~какие свойства индивидуальных 
последовательностей могут позволить решить эту задачу?
  
  Прежде всего укажем, что данная проблема связана с~известными 
результатами относительно разрешимости задач предсказания. В~самом 
общем виде они формулируются в~так называемой теореме об отсутствии 
<<бесплатного обеда>> (The No Free Lunch Theorem~[3]) в~задачах МО, 
которая гласит, что не существует единственного алгоритма, который мог бы 
быть лучшим для решения любых проблем МО, и~необходимо пробовать 
разные алгоритмы, принимая во внимание размер и~структуру набора 
данных, чтобы определить, какой из них окажется лучшим для решения 
конкретной проблемы. Известно также о~невозможности построения 
единственного предиктора с~нулевой средней ошибкой предсказания для 
семейств дискретных последовательностей с~неизвестным вероятностным 
распределением генерирующего их источника~[4].
   
  Подчеркнем, что рассматривается ситуация, когда разработчик системы, 
в~которую должна входить подсистема прогноза, не является экспертом 
в~МО настолько, чтобы разрабатывать новые алгоритмы, а~намерен 
использовать лишь доступные готовые предикторы и~выбор возможен лишь 
по наборам данных, формирующимся в~результате работы этих предикторов. 
  
   В свое время в~работах~[5, 6] были сформулированы и~предложены 
подходы к~решению задачи оценки двоичных последовательностей 
относительно некоторого множества предикторов, однако получены 
в~основном результаты асимптотического характера и~их использование 
в~практических случаях затруднительно. В~последние годы появилось много 
работ по статистическим оценкам работы предикторов из известных 
облачных ресурсов (MS AWS Azure Machine Learning, Google Cloud Machine 
Learning и~т.\,д.), однако оценки получены относительно так называемой 
средней доли ошибок предсказаний, что, как будет показано в~разд.~3, 
требует существенного уточнения. 
  
  В данной работе предлагается использовать эвристические методы 
априорного выбора пре\-дик\-то\-ров и/или подмножества двоичных наборов 
данных для данного предиктора, позволяющих улучшить качество прогноза. 
Данные рассматриваются как двоичные последовательности, поскольку 
многие практические задачи прогнозирования (предсказания финансовых 
индексов, вредоносных атак на сетевые ресурсы и~т.\,д.)\ требуют 
предсказания не самих абсолютных значений, а~знаков (на\-прав\-ле\-ния) их 
изменения, когда задача может быть сведена к~предсказанию двоичных 
последовательностей~[7].
{\looseness=1

}
   
  В качестве эвристической меры эффективности использования данного 
предиктора для предсказания двоичной последовательности в~следующий 
после последнего наблюдения момент времени предлагается некоторая мера 
<<близости>> характеристик\linebreak качества предсказания рассматриваемой 
последовательности данным предиктором, к~характеристикам качества 
предсказания последовательности независимых испытаний (испытаний 
Бернулли)\linebreak некоторым тестовым предиктором, который обеспечивает 
минимальные потери предсказания, при функции потерь (loss function~[8]) 
Хэмминга. В~дальнейшем будем называть такой тестовый предиктор 
<<предиктором Бернулли>> (ПБ). Данный критерий качества обобщенно 
учитывает как структуру множества ошибок предсказаний оцениваемым 
предиктором, так и~статистическую оценку условной вероятности 
правильного предсказания элемента последовательности в~момент~$t$ по 
ранее наблюденным ее значениям, и~чем ближе эти характеристики 
к~характеристикам ПБ для данной последовательности, тем менее 
пригодным считается предиктор. Заметим, что в~разговоре о~бли\-зости не 
предполагалось, чтобы она формально отвечала требованиям метрики 
расстояния, неравенству треугольника прежде всего. При этом выполняется 
указанное выше требование обеспечить реализацию предлагаемого 
эвристического подхода без существенных дополнительных затрат на его 
программную имплементацию.

\vspace*{-9pt}

\section{Задача оценки предсказуемости данных}

\vspace*{-2pt}

  Определим задачу прогнозирования последовательности. Наблюдатель 
последовательно наблюдает значения $x_1, x_2, \ldots , x_t$ известных типов, 
например вещественные значения (тогда речь идет о~~временн$\acute{\mbox{ы}}$х рядах) или 
символьные в~некотором алфавите~$A$. В~момент времени $t-1$, получив 
значения $x_1, \ldots , x_{t-1}$, наблюдатель \textit{прогнозирует} сле\-ду\-ющее 
значение~$x_t$, т.\,е.\ предсказывает, что~$x_t$ примет значение~$b_t$. 
Проблема состоит в~том, какое прави-\linebreak\vspace*{-12pt}

\pagebreak

\noindent
ло должен использовать наблюдатель 
для принятия решения о~значении~$x_t$.
    
  В рамках широко используемой теории принятия решений~[8] для 
решения этой задачи предполагается наличие некоторой числовой функции 
потерь~$l:\ \{b_t, x_t\}\hm\to {R}$, где ${R}$~--- некоторое 
измеримое множество, на котором задается мера качества прогноза. Эта 
функция может быть, например, расстоянием Хэмминга (для двоичных 
последовательностей) или квадратом ошибки (для числовых). Функция 
потерь $l(p, x)$ при значениях $p\hm = b_t$, $x\hm = x_t$~--- это величина 
потери, которая связана с~результатом прогноза (принятием решения 
о~значении~$b_t$), если истинным оказалось значение~$x_t$, т.\,е.\ потеря, 
убыток, штраф от неправильного прогноза. 
  
  Соответственно, в~самом общем виде задача предсказания состоит 
в~построении алгоритма назначения (выбора) значения~$b_t$, 
обеспечивающего минимум потерь (максимум качества прогноза) 
относительно принятой функции потерь.
  
  \textit{Предсказуемость последовательности данных} служит 
характеристикой эффективности решения задачи предсказания 
(прогнозирования) значения данных в~будущем, или появления событий 
в~будущем, по ранее наблюденным данным или со\-бы\-тиям. 
  
  Концепция предсказуемости (называемая, в~том числе и~в~русскоязычной 
литературе, predictability) широко изучалась в~многочисленных работах по 
теории случайных процессов~[9]. Однако предлагаемый подход к~оценке 
предсказуемости основывается на работах~\cite{4-fr, 5-fr, 6-fr, 11-fr}, 
относящихся к~задачам\linebreak построения так называемых <<универсальных>> 
алгоритмов предсказания <<индивидуальных>> последо\-ва\-тель\-ностей, т.\,е.\ 
таких методов и~таких случайных последовательностей, для которых 
невозможно сделать ка\-кие-ли\-бо предположения о~том, к~каким классам 
вероятностных распределений они принадлежат~\cite{10-fr}.
   
  В рамках данного подхода сформулированное\linebreak в~работах~\cite{10-fr, 11-fr} 
понимание предсказуемости\linebreak двоичной последовательности предиктором, 
реализующим алгоритм~$A$, позволяет оценивать предсказуемость 
степенью отличия вероятности предсказания этим предиктором следующего 
значения от~$1/2$, т.\,е.\ от вероятности правильного предсказания, которое 
реализует предиктор, эквивалентный подбрасыванию честной монеты.
  
  Решению описанной во введении общей задачи в~большей степени 
отвечает подход, сформулированный в~\cite{6-fr}. Дадим его краткое 
изложение (в~более подходящих для рассматриваемой задачи терминах).
  
  Пусть $x_1, x_2, \ldots , x_n$, $x_t\hm\in \{0,1\}$,~--- двоичная 
последовательность независимых испытаний длины~$n$ 
(последовательность Бернулли) с~параметром (вероятностью единицы) 
$\theta\hm = \mathrm{Pr}\left\{x_t=1\right\}$. Предиктор $f\hm = (f_1, f_2, 
\ldots)$~--- это последовательность функций, принадлежащих некоторому 
семейству. В~момент~$t$ используется функция предсказания $f_t: \{0,1\}^t 
\hm\to \{0,1\}$, т.\,е.\ она вычисляет по наблюденной двоичной 
последовательности $x_1, x_2, \ldots , x_t$ длины~$t$ следующее значение: 
$x^\prime_{t+1}\hm= f_t(x_1, x_2, \ldots , x_t)$.\linebreak Отметим, что~$x^\prime_{t+1}$ 
означает именно оценку истинного значения~$x_{t+1}$ и~может не совпадать 
с~ним. Обозначим число неправильных предсказаний\linebreak после применения 
предиктора~$f$ для всей указанной последовательности длины~$n$ как 
$n_e(f)\hm = \sum\nolimits_{t=1,n} 1\{ x^\prime_t \not= x_t\}$, где 
$1\{\cdot\}$~---  
функ\-ция-ин\-ди\-ка\-тор, равная единице при выполнении условия 
в~скобках и~нулю в~противном случае.
  
  Задача состоит в~минимизации 
  $$
  \pi_\theta (f)= \lim\limits_{n\to\infty} 
\mathrm{sup}\,\fr{\mathrm{E}_\theta n_e(f)}{n}\,,
$$
 где $\mathrm{E}_\theta$~--- 
математическое ожидание случайной величины~$n_e(f)$ для 
последовательности Бернулли $x_1, x_2, \ldots , x_n$ с~параметром~$\theta$. 
Предсказуемость последовательности определяется как значение
  \begin{equation}
  \pi_\theta = \mathop{\mathrm{inf}}\nolimits_f \left( \pi_\theta (f)\right) = 
\mathop{\mathrm{inf}}\nolimits_f \left( \lim\limits_{n\to\infty} 
\mathrm{sup}\,\fr{\mathrm{E}_\theta n_e(f)}{n}\right)\,.
  \label{e1-fr}
  \end{equation}
  %
  Точная нижняя грань $\mathrm{inf}_f$ берется по всем возможным 
предикторам~$f$.
   
  Рассмотрим возможность практического вы\-чис\-ле\-ния~(1) для 
последовательности Бернулли. Известно~\cite{6-fr}, что $\mathrm{inf}_f$ для 
последовательности Бернулли с~параметром~$\theta$ достигается на 
предикторе, реализующем следующую функцию предсказания: 
  $$
  x^\prime_{t+1}\hm=\begin{cases}
   1, &\mbox{если } \theta>\fr{1}{2}\,;\\
  0\,, &\mbox{если } \theta\leq \fr{1}{2}.
  \end{cases}
  $$ 
  Однако, если $\theta$ точно не известно, то неясно, какой предиктор 
следует использовать. Интуитивно предиктор 
\begin{multline*}
x^*_{t+1}= f^*_t\left(x_1, x_2, 
\ldots , x_t\right) = {}\\
{}=
\begin{cases}
0, & \mbox{если }  \theta^\prime(t) < \fr{1}{2};\\
 1,& \mbox{если } \theta^\prime(t)\hm> \fr{1}{2};\\
 0 \mbox{ или } 1& \mbox{в~зависимости\ от\ результата}\\ 
 &\mbox{подбрасывания\
  <<честной>>\ монеты,}\hspace*{-4.04109pt}\\
&\mbox{если } \theta=\fr{1}{2},
  \end{cases}
  \end{multline*}
   где 
$\theta^\prime(t)\hm = t^{-1}\sum\nolimits_{r=1,t}(x_r)$~--- текущая статистическая 
оценка~$\theta$, представляется в~некотором смысле лучшим, если 
значение~$\theta$ неизвестно. В~дальнейшем будем называть такой 
предиктор <<предиктором Бернулли>>.
  
  В~\cite{6-fr} было доказано, что 
  \begin{itemize}
  \item[(а)] для каждого предиктора~$f$ и~для любого $\theta\not=$\linebreak
  $\not= 1/2$ либо 
$$\mathrm{E}_\theta n_e(f)\geq n\pi_\theta+ c_0(\theta) - o(1),
$$ 
либо 
$$
\mathrm{E}_{\underline{\theta}} n_e(f)\geq 
n\pi_{\underline{\theta}} + c_0(\underline{\theta}) - o(1),
$$
 где 
$\underline{\theta}\hm= 1\hm- \theta$, $c_0(\theta) \hm= [2(1\hm- 2\pi_\theta)]^{-1}$; 
  \item[(б)] предиктор~$f^*$ удовлетворяет обоим условиям: 
\begin{align*}
\mathrm{E}_{\theta} n_e(f^*)&\leq n\pi_\theta  + c_0(\theta)\,;\\ 
\mathrm{E}_{\underline{\theta}} n_e(f^*)&\leq n\pi_{\underline{\theta}} 
 + c_0(\underline{\theta}).
 \end{align*}
  Тем самым было доказано, что предиктор~$f^*$ 
оказывается в~определенном смысле лучшим, если значение~$\theta$ 
неизвестно. 
  \end{itemize}
  
  Однако, учитывая, что в~используемых на практике предикторах 
основным показателем качества служит доля правильных предсказаний 
(success rate~--- SR), оцениваемая по достаточно большой выборке 
(аналогично рассмотренной выше  
$n$-по\-сле\-до\-ва\-тель\-ности), перейдем к~неравенствам именно для этого 
показателя, связанного с~долей неправильных предсказаний, как $1\hm- 
n_e(f)/n$ и~перепишем указанные в~(б) неравенства как
  \begin{equation}
  \left.
  \begin{array}{rl}
  \mathrm{E}_\theta \mathrm{SR}\,(f^*) &\geq 1-\pi_\theta -
\fr{c_0(\theta)}{n}\,;\\[6pt]
  \mathrm{E}_{\underline{\theta}} \mathrm{SR}\,(f^*) &\geq 1-
\pi_{\underline{\theta}} -\fr{c_0(\underline{\theta})}{n}\,.
  \end{array}
  \right\}
  \label{e2-fr}
  \end{equation}
  
  Итак, для последовательности независимых испытаний Бернулли известен 
оптимальный пре\-диктор и~можно оценить ошибки предсказания с~учетом 
только статистически оценочного знания параметра Бернулли. Нетрудно 
показать, что описанный ПБ обеспечивает оптимальное предсказание также 
по критерию минимизации потерь при использовании расстояния Хэмминга 
в~качестве функции потерь, а~именно: присваивает функции значения:
$$
l(x_t,  x^\prime_t) =\begin{cases}
 0 & \mbox{при } x_t = x^\prime_t\,;\\
 1 & \mbox{в~противном\ случае},
 \end{cases}
 $$
  т.\,е.\ <<потерял все>>, если предсказал неправильно, 
и~ничего не потерял, если правильно (совершенно разумный принцип при 
оценивании, например, прогнозирования DDoS (distributed denial of service)
атак в~сети). Этот предиктор, 
разумеется, довольно примитивный, но при этом все-та\-ки обеспечивает 
б$\acute{\mbox{о}}$льшую вероятность правильного прогноза, чем простое 
угадывание (подбрасывание <<честной>> монеты).
  
  Однако, как указывалось, предсказуемость в~понимании~(1) зависит 
только от свойств последовательности, а~не от предикторов. Поэтому 
ставится задача: на основании~(1) сформулировать количественное 
определение прогнозируемости случайной последовательности~$x$ 
в~момент~$t$ относительно заданного множества предикторов~$f$, т.\,е.\ как 
наименьшее значение доли ошибочных предсказаний, которые могут 
обеспечить предикторы из данного набора доступных предикторов при их 
независимом использовании на достаточно длинных отрезках 
последовательности. 
   
  Очевидно, что любой предиктор, обеспечивающий лучшую долю 
предсказаний, чем БП, не должен повторять структуру множества 
ошибочных решений БП. Учитывая это обстоятельство, будем оценивать 
предсказуемость таких индивидуальных последовательностей для 
конкретных предикторов, сравнивая их с~ПБ.

\section{Вычисление количественной меры предсказуемости 
случайной последовательности}

  Из сопоставления формул~(1) и~(2) нетрудно видеть, что доля правильных 
предсказаний $\mathrm{SR}\,(\cdot)$ представляет собой статистическую 
оценку условной вероятности: 
  $$
  \pi (x, t) = \mathrm{Prob}\left ( b_t = x_t \vert x_{t-k}, \ldots , x_{t-1}\right),
  $$ 
где $x_t$--- истинное значение члена по\-сле\-до\-ва\-тель\-ности; $k$~--- длина 
наблюденной по\-сле\-до\-ва\-тель\-ности; $b_t \hm= f_p(x_{t-m}, \ldots , x_{t-1})$~--- 
предсказанное предиктором~$f_p$ значение случайной последовательности 
в~момент~$t$,  
$m\hm \leq k$~--- чис\-ло членов последовательности, предшествующих~$t$, 
которое используется в~данном предикторе~$p$ для получения прогноза. 

 \begin{figure*}[b] %fig1
  \vspace*{1pt}
 \begin{center}
 \mbox{%
 \epsfxsize=144.644mm 
 \epsfbox{fre-1.eps}
 }
 \end{center}
   \vspace*{-9pt}
  \Caption{Интервал прогнозирования, $(t-M+m)$~--- момент первого предсказания 
предиктором с~$m$~входными (наблюденными) значениями последовательности}
  \end{figure*}
  

  Соответственно, в~определении пред\-ска\-зу\-емости согласно~(1) можно от 
минимизации математического ожидания доли неправильных предсказаний 
по данному распределению вероятностей нулей и~единиц 
последовательности длиной~$n$ перейти к~максимизации вероятности 
(доли, в~практическом аспекте) правильных предсказаний, понимая под 
количественной мерой предсказуемости статистическую оценку 
величины~$\pi(x,t)$.
   
  Заметим, что отличием длины наблюдаемой последовательности~$k$ 
и~числа элементов последовательности~$m$, используемой предиктором, 
учитывается то, что выбор некоторой обучающей выборки, по которой 
в~конечном итоге вычисляется прогнозируемое значение, путем ли 
автоматической установки параметров используемой модели предиктора 
(например, нейронной сетью некоторого типа) или вручную пользователем 
программы, которая реализует алгоритм, является неотъемлемым 
компонентом алгоритма предсказания. 

%\vspace*{-6pt}
  
\section{Вычисление $\pi (x, t)$}

%\vspace*{-2pt}

  Приведем порядок вычисления $\pi (x, t)$ как эмпирической функции 
распределения. Пусть $x_{t-M}, \ldots , x_t$ (рис.~1)~--- участок двоичной 
последовательности $x(t)$, рассматриваемый в~пределах временного 
окна~$W_M$ размером~$M$ в~предположении, что используемый 
предиктор~$f_p$ выполняет предсказание по~$m$ прошлым наблюденным 
значениям последовательности.
  
 
  Первое предсказание предиктором~$f_p$ в~данном окне выполняется для 
члена последовательности в~момент $(t\hm- M \hm+ m)$, зная значения 
членов последовательности в~моменты $(t- M), (t - M + 1), \ldots , (t\hm- M \hm+ m \hm-
1)$. Следующее предсказание делается для момента $(t \hm-
 M \hm+ m \hm+ 1)$ по данным, начиная со следующего за $(t - M)$ момента 
$(t\hm- M \hm+ 1)$ до момента $(t\hm- M \hm+ m)$, вычисляя $f_p(x_{t-M+1}, 
\ldots , x_{t-M+m})$, и~так далее до последнего в~данном окне момента~$t$, 
в~котором вычисляют $f_p(x_{t-m}, x_{t-m+1},\ldots , x_{t-1})$.
   
  Тогда, рассматривая поток предсказаний как последовательность Бернулли 
с~единичными исходами (общим числом~$N_1$) для правильных 
предсказаний и~нулевыми для неправильных, в~качестве оценки 
$\mathrm{SR}\,(W_M, p, t)$ данной условной ве\-ро\-ят\-ности можно взять долю 
правильных предсказаний на временн$\acute{\mbox{о}}$м окне~$W_M$: 
%\noindent
  \begin{multline*}
\mathrm{SR}\,(W_M, p, t) = \fr{N_1}{M} ={}\\
{}= M^{-1}\hspace*{-8pt}\sum\limits_{i=1,M-m+1} \hspace*{-11.8348pt}1 
\left( f_p\left ( x_{t-M+i-1}, \ldots , x_{t-M+i+m-2} \right) ={}\right.\\
\left.{}= x_{t-M+i+m-1}\right),
%\label{e3-fr}
\end{multline*}
где $x_{t-M+i-1}, \ldots , x_{t-M+i+m-2}$~--- значения членов последовательности 
в~$m$~моментов времени, предшествующих моменту $(t\hm-
 M \hm+ i \hm+ m\hm-1)$, $i\hm = 1, \ldots , M-m+1$, которые предиктор 
использует в~качестве входных (в~соответствии с~его алгоритмом).
  
  Очевидно, что величина $\mathrm{SR}\,(\cdot)$, часто на\-зы\-ва\-емая 
в~документации по использованию про\-граммных реализаций алгоритмов 
предсказаний <<коэффициентом успешных предсказаний>>, 
служит естественной мерой успешных предсказаний к~моменту~$t$.
   
   Вероятность единицы $p_1$ (параметр Бернулли, аналогичный~$\theta$ 
в~предыдущем разделе) точно не\linebreak известна, а~известна ее статистическая 
оценка,\linebreak позволяющая судить о~ее величине в~пределах\linebreak доверительного 
интервала. Однако~(2) и~\cite{5-fr} показывают, что ширина этих интервалов 
имеет асимп\-то\-ти\-че\-скую оценку, в~зависимости от предположений 
о~моделях предикторов, между $O\,(1/\sqrt{M})$ 
и~$O\,(1/M)$, т.\,е.\ не шире, чем доверительные интервалы 
стандартных статистических оценок, что допускает практическое 
использование оценок $\mathrm{SR}\,(\cdot)$.
{\looseness=1

}

\section{Использование оценок прогнозирования для~выбора 
предиктора или~отказа от~выполнения прогноза в~данный 
момент}

Предположим, что предсказываемая последовательность~--- это 
последовательность независимых испытаний с~неизвестным параметром 
распределения Бернулли. Пусть к~некоторому моменту~$t$ 
выполнено~$M$~предсказаний с~предиктором~$f_p$ и~вы\-чис\-ле\-но 
$\mathrm{SR}\,(\cdot)$. Определим, разумно ли использовать этот предиктор 
для дальнейших предсказаний или надо выбрать другой предиктор из 
доступного списка предикторов. Для оценки качества предиктора используем 
оценку $\mathrm{SR}\,(\cdot)$ и~ее сравнение с~оптимальным 
ПБ. Поясним смысл предлагаемого сравнения на примере нескольких 
хорошо известных предикторов XGD
(eXtreme 
Gradient Boosting), SGD
 (stochastic gradient descent) и~SVR 
 (Support Vector Regression) из библиотек Python (из-за 
ограниченности размера статьи их описание не приводится. Читатель без 
труда найдет многочисленные описания в~Сети, например, в~\cite{14-fr}). 
  
  Пусть использован один из таких предикторов, 
  например XGD, и~вы\-чис\-ле\-на оценка $\mathrm{SR}\,(\cdot)$. Оче\-видно, 
что логика использования $\mathrm{SR}\,(\cdot)$ как ха\-рактеристики 
предсказуемости со\-сто\-ит в~том, что\linebreak некоторые явные или скрытые 
характеристики по\-сле\-до\-ва\-тель\-ности, значимые для алгоритма работы 
данного предиктора, не изменятся в~сле\-ду\-ющий момент и~можно считать, 
что ожидаемая доля успешных предсказаний будет примерно такой же, как 
ранее вы\-чис\-лен\-ная величина $\mathrm{SR}\,(\cdot)$. 
  
  Опишем кратко, что это за характеристики последовательности, для 
XGBoost. 
  
  В основе XGBoost лежит алгоритм градиентного\linebreak
   бустинга деревьев 
решений, который строит модель предсказания в~форме ансамбля 
предсказывающих моделей, обычно деревьев решений. Обуче\-ние ансамбля 
проводится последовательно на\linebreak входных данных предсказываемой 
последовательности, в~которой выделяются паттерны определенной длины 
(например, <<$0110$>>), за которыми должны следовать 
предсказываемые~<<$0$>> или~<<$1$>>.
  
  Пусть рассматривается упоминавшаяся последовательность в~окне $W_M$ 
(т.\,е.\ $(t-M, t)$) просто как последовательность Бернулли и~используется 
для предсказания также оптимальный предсказатель Бернулли. Поэтому 
можно говорить об отличии некоторого предиктора двоичной 
последовательности~$f_p$ от такого простого предиктора (ПБ) как по 
оцененной величине $\mathrm{SR}\,(\cdot)$, так и~по структуре ошибок, 
т.\,е.\ по соотношению ошибок <<$0$ вмес\-то~$1$>> (<<$0\hm\to 1$>>) 
и~<<$1$ вместо~$0$>> (<<$1\hm\to 0$>>). Очевидно, что если сколь\-ко-ни\-будь 
существенного отличия нет, то можно говорить о~неэффективности 
предиктора~$f_p$, поскольку аналогичное решение дает существенно более 
прос\-той~ПБ.
  
  Критерий перехода к~другому предиктору для прогнозирования в~момент 
времени $(t\hm- M \hm+ 1)$ формулируется следующим образом (считая, что 
величина~$M$ достаточно велика, чтобы обеспечить необходимый узкий 
доверительный интервал оценок вероятностей нулей, единиц 
и~$\mathrm{SR}\,(\cdot)$). 
  
  Предположим, что на интервале длиной~$M$, на котором предиктор~$f_p$ 
делает~$M$ предсказаний значений $x_{t-M+i}$ из предыдущих значений, 
частота единиц равна~$p_1$, частота правильных прогнозов нулевых 
значений относительно всех нулевых значений~--- $p_{V=1}\left(W_M, 
f_p = 0\right)$, где $(V = 1)$~--- знак правильного предсказания 
предиктором~$f_p$ внутри окна. 
  
  Тогда прогноз этого предиктора~$f_p$ принимается и~для следующего 
интервала, если при $p_1\hm > 1/2$


\noindent
  \begin{equation}
  \left.
  \begin{array}{rl}
  p_1 &< \mathrm{SR}\left( W_M, p, t\right);\\[6pt]
  p_{V=1}\left( W_M, f_p = 
0\right)& > \fr{1-p_1}{\tau}\,,
\end{array}
\right\}
  \label{e4-fr}
  \end{equation}
а при $p_1<1/2$
\begin{equation}
\left.
\begin{array}{rl}
  (1-p_1) &< \mathrm{SR}\left( W_M, p, t\right);\\[6pt]
  p_{V=1}\left( W_M, f_p = 
0\right) &> \fr{p_1}{\tau}\,,
\end{array}
\right\}
  \label{e5-fr}
  \end{equation}
где $\tau=2\mbox{--}5$ выбирается в~зависимости от получаемых $p_{V=1}$ 
и,~очевидно, отражает уменьшение дисбаланса между ошибками 
<<$1\hm\to0$>> и~<<$0\hm\to1$>>, присущими ПБ.
  
  Если эти условия не выполнены, то выбирается следующий предиктор из 
списка доступных. Действительно, $p_1$~--- это доля единиц в~данном окне, 
а~$(1\hm- p_1)$~--- доля нулей. Оптимальным ПБ 
в~данном окне все биты последовательности предсказываются как единицы, 
если единицы более час\-тые ($p_1\hm >1/2$), или как нули, если нули более 
частые в~данном окне. Поэтому первое условие означает, что 
предиктор~$f_p$ обеспечивает более высокую вероятность правильного 
предсказания до последнего наблюденного момента, чем ПБ, а~вторая~--- 
что определенная часть предсказанных нулей (или единиц) является 
результатом работы алгоритма предсказания предиктора~$f_p$, а~не 
механического обращения нулей в~единицы (или единиц в~нули), как это 
делает~ПБ.
   
  Был выполнен ряд экспериментов для двоичных последовательностей, 
соответствующих знакам изменения значений временн$\acute{\mbox{ы}}$х рядов (<<$1$>>~--- 
воз\-рас\-та\-ние, <<$-1$>>~--- убывание), полученных как из различных 
литературных источников, посвященных исследованию изменения объема 
трафика в~сетях, так и~из измерений с~помощью инструмента IBM QRadar 
NETflow числа так называемых <<потоков событий>> для трафика в~сети, 
реализующей географически распределенный гибридный 
высокопроизводительный вычислительный кластер РАН в~разное время 
суток~\cite{16-fr} (рис.~2).
   
  \begin{figure*} %fig2
\vspace*{1pt}
 \begin{center}
 \mbox{%
 \epsfxsize=102.531mm 
 \epsfbox{fre-2.eps}
 }
 \end{center}
   \vspace*{-6pt}
   \Caption{Примеры записей Flow (от IBM QRadar NETflow). 
   Ось абсцисс~--- время суток; 
ось ординат~--- число зафиксированных потоков событий}
\vspace*{9pt}
\end{figure*}

  Для проверки предсказаний предиктором, реализующим технику XGB, 
а~также доступными предикторами SGD и~SVR 
(например класс sklearn.svm.SVR в~Python) были 
получены массивы данных в~окне размером $M\hm = 100$~мин.
   
  Исследовалась следующая полученная знаковая последовательность 
длиной~100, соответствующая знакам приращения трафика (алфавит $Z\hm = 
\{-1, 1\}$ используется вместо двоичного $B\hm = \{0, 1\}$ ввиду формата 
используемых программ):
  
  \noindent 
1 1 1 1 $-$1 1 1 $-$1 1 $-$1 1 1 1 $-$1 1 1 1 $-$1 $-$1 $-$1 1 $-$1 1 1 1 $-$1 1 1 1 1 1 1 1 1 $-
$1 1 1 1 1 1 $-$1 $-$1 $-$1 1 $-$1 $-$1 $-$1 1$-$1 $-$1 $-$1 1 $-$1 $-$1 1 1 $-$1 1 1 1 1 $-$1 
$-$1 1 $-$1 1 1 $-$1 1 $-$1 $-$1 1 $-$1 1 1 1 1 1 $-$1 1 1 $-$1 1 $-$1 $-$1 1 1 1 $-$1 1 $-$1 $-
$1 1 $-$1 $-$1 1 1 $-$1 1 1
  
  
\noindent
  Эта наблюдаемая последовательность содержит 61~единицу и~39~нулей, 
т.\,е.\ $p_1\hm = 0{,}61$. Для указанных данных в~результате работы 
предиктора XGB был получен следующий массив предсказанных значений:


  
  \noindent
1 1 1 1 1 1 1 1 1 1 1 1 1 1 1 1 1 1 1 1 1 1 1 1 1 1 1 1 1 $-$1 
1 1 1 1 1 1 1 1 1 1 1 1 1 1 1 1 1 1 1 1 1 1 1 1 $-$1 1 1 1 1 1 1 1 1 1 1
$-$1 1 1 1 1 1 1 1 1 1 1 1 1 1 1 1 1 1 1 1 1 1 1 1 1 1 1 1 1 1 1 1 1 1 1

\vspace*{2pt}
 
 \noindent 
  В этом массиве 97~единиц и~3~нуля. Правильно предсказанных единиц~--- 
  59, правильно предсказанных нулей~--- 1, неправильно предсказанных 
единиц~--- 2, неправильно предсказанных нулей~--- 38. Всего правильных 
предсказаний~60, неправильных~--- 40; $\mathrm{SR}\,(W_M, p, t) \hm= 
0{,}6$; $p_{V=1}(W_M, f_p\hm= 0) \hm= 1/39 \hm= 0{,}02$: $0{,}6 \hm< p_1$ 
и~$0{,}02 \hm\ll 0{,}39/\tau$ для $\tau \hm = 2\mbox{--}5$, т.\,е.\ оба
 условия~(\ref{e4-fr}) не выполняются. Предиктор отклоняется, так как он 
дает результаты хуже, чем~ПБ.
  
  Далее была проведена проверка предиктора SGD, он дал следующий 
массив предсказанных значений:

%  $$
%  \begin{matrix}
\noindent
 1   1   1  1  1   1   1   1   1   $-1$   1  1  1   1  1  1  1  1  1  1  1  1  1  1  1  $-1$  1   1 1 1\\
  1 1 1 1  1  1  1  1  1 1  $-1$ 1  1   1 1  1  $-1$ 
      1  1  1  1  1  1  1  $-1$  1  1  1  1  1\\
       1  1 $-1$  1  1  1  1  1  1  1  1  1  1  1  1  1  1  1  1  1  1  $-1$  $-1$ 1
 1  1 1  1  1\\
   1  1  1  1  1  1  1  1  1  1 1 
%\end{matrix} 

\vspace*{2pt}
%$$
  
 \noindent
  В этом массиве 92 единицы и~8 нулей. Правильно предсказанных 
единиц~--- 59, правильно предсказанных нулей~--- 6, неправильно 
предсказанных единиц~--- 2, неправильно предсказанных нулей~--- 33. Всего 
правильных предсказаний~65, неправильных~--- 35. На данном отрезке 
последовательности значение $\mathrm{SR}\,(\cdot) \hm= 0{,}65 \hm> p_1$ 
и~$p_{V=1}(W_M, f_p= 0) \hm= 6/39 \hm= 0{,}15 \hm> 0{,}39/3$, т.\,е.\ оба 
условия~(\ref{e4-fr}) выполняются, предиктор получил результаты, 
отличающиеся от ПБ. 
  %
   Поэтому в~данном случае можно говорить о~высокой предсказуемости 
последовательности для следующего момента времени для предиктора SGD, 
этот предиктор предпочтительнее предиктора XGB для предсказания 
следующего значения.
  
  Был доступен также предиктор SVR, он дал следующий массив 
предсказанных значений для той же наблюдаемой последовательности: 

\noindent
 1  1 1  $-1$  $-1$  $-1$ 1  $z$  $-1$  1  1  1  1  1  1  1  $-1$  $-1$   $-1$  $-$1  $-$1
 1  $-1$  1  1  1  $-1$  1  1 $-1$  1  1  1  1  1 $-1$  1  1 1 1  
1 1 
 1  $z$  1  1  1  1  1  1  1 1  1  1  1  1  1  1  1 1 $-1$  1  1
 1  1  1  1  1  1  1  1  1 1  1  $-1$  1 $-1$  1  1 1 1 1 1 1 
 1  1  1  1  1  1  1  1  1 1  1  1  $-1$  1  1  1

\vspace*{2pt}

\noindent
  В этом массиве 81 единица и~17 нулей, $z \hm= 2$ (знак~~$z$ здесь 
соответствует ситуации, когда программа, реализующая предиктор, 
прерывает вы\-чис\-ле\-ние\linebreak из-за того, что на имеющихся значениях не удается 
построить разделяющую плоскость классификатора). Правильно 
предсказанных единиц~--- 48, правильно предсказанных нулей~--- 4, 
неправильно предсказанных единиц~--- 13 (в~том числе в~двух случаях 
имеем прерывание~$z$), неправильно предсказанных нулей~--- 35 (в~том 
числе два~$z$). Всего правильных предсказаний~52, неправильных~--- 48 
(в~том числе два~$z$). На данном отрезке последовательности значение 
$\mathrm{SR}\,(\cdot) \hm= 0{,}52 \hm< p_1= 0{,}61$ и~$p_{V=1}(W_M, f_p= 0) 
\hm= 4/39 \hm= 0{,}1 \hm> 0{,}39/4$.
  
  Предиктор обеспечивает весьма малое значение $\mathrm{SR}(\cdot)$, 
и~условие~(\ref{e4-fr}) заведомо не выполняется, хотя структура ошибок 
благоприятна с~точки зрения удаленности от ПБ~--- в~семнадцати случаях 
предиктор предсказывал нулевые значения (13~раз ошибочно и~4~раза 
правильно).
  
  Таким образом, и~этот предиктор отклоняется, а~лучшим предиктором, 
обеспечивающим условия~(\ref{e4-fr}), признается XGB.
  
  Исследовалась еще одна знаковая последовательность длиной~100: 

\noindent
1 1 $-1$ 1 1 1 1 $-1$ $-1$ $-1$ 1 $-1$ $-1$ $-1$ 1 $-1$ 1 $-1$ 1 $-1$ 1 $-1$ 
1 $-1$ $-1$ 1 1 1 1 
1 1 $-1$ 1 1 $-1$ 1 $-1$ 1 $-1$ 1 $-1$ $-1$ $-1$ 1 $-1$ 1 1 $-1$ 1 $-1$ 1 $-1$ 
$-1$ $-1$ 1 1 1 
$-1$ 1 $-1$ 1 $-1$ $-1$ $-1$ $-1$ 1 $-1$ -1 $-1$ $-1$ 
$-1$ 1 $-1$ $-1$ 1 $-1$ 1 1 $-1$ $-1$ $-1$ $-1$ 1 $-1$ $-1$ 
$-1$ $-1$ 1 1 $-1$ $-1$ 1 $-1$ $-1$ 1 1 1 1 $-1$ 1

\vspace*{2pt}

\noindent
  Эта наблюдаемая последовательность содержит 47~единиц и~53~нуля, т.\,е.\
   $p_1 \hm= 0{,}47$, $1\hm- p_1 \hm= 0{,}53 \hm> 1/2$, т.\,е.\ в~данном 
случае оптимальный ПБ предсказывает все значения как нулевые. 
В~эксперименте опять были доступны предикторы XGB, SGD и~SVR. Для 
указанных данных в~результате работы предиктора XGB был получен 
следующий массив предсказанных значений:
  
  \noindent
1 1 1 1 1 1 1 $-1$ 1 1 1 1 1 1 1 1 1 1 1 1 1 1 1 1 1 1 1 1 1 1 1 1 1 $-1$ 1 1 1 1 1 1 1 1 1 1 1 1 1 1 
1 1 1 1 1 1 1 1 1 1 1 1 1 1 1 1 1 1 1 1 1 1 1 1 $-1$ 1 1 1 1 1 1 1 1 1 1 1 1 1 1 1 $-1$ 1 1 1 1 1 1 1 
1 1 1 1

\vspace*{2pt}
  
  \noindent
  В этом массиве 96~единиц и~4~нуля. Правильно предсказанных единиц~--- 
  45, правильно предсказанных нулей~--- 2, неправильно предсказанных 
единиц~--- 2, неправильно предсказанных нулей~--- 53. Всего правильных 
предсказаний 47, неправильных~--- 53; $\mathrm{SR}\,(W_M, p, t)\hm = 
0{,}47$; $p_{V=1}(W_M, f_p= 0)\hm = 2/53 \hm= 0{,}04$: $0{,}47 \hm< 1\hm-
 p_1 \hm= 0{,}53$ и~$0{,}04 \hm\ll 0{,}39/\tau$ для $\tau\hm = 2\mbox{--}5$. 
Число правильных предсказаний $\mathrm{SR}\,(\cdot) \hm= 0{,}47$
оказывается ниже частоты нулевых значений, и,~следовательно, этот 
предиктор может быть отвергнут вне зависимости от структуры множества 
ошибок~(\ref{e5-fr}). Предиктор не годится, он дает результаты хуже, чем 
ПБ.
  
  Далее была произведена проверка предиктора SGD, он дал следующий 
массив предсказанных значений: 
 
 \noindent
1  1 1  1 1  1 1 1  1   1 1  1  1  1  1  1  1  1  $-1$ 1  1 
 1  1 1  1  1  1 1  $-1$  1 1 1  1  1  1  1  $-1$ 1 1 1  1  1 
 1  1 1  1  1  1 1  1  1  1  1  1  1  1  1 1  1  1 1  1 1
 1  1 1 1  1 1 1 1  1  1  1  1  1 1  1  1  1 1 1  1 1
 $-1$ 1 1 1 $-1$ 1  1 1 1 1 1  1 1 1 1 1
 
 \vspace*{2pt}
  
  
  \noindent
  В этом массиве 95~единиц и~5~нулей. Правильно предсказанных 
единиц~--- 44, правильно предсказанных нулей~--- 2, неправильно 
предсказанных единиц~--- 3, неправильно предсказанных нулей~--- 51. Всего 
правильных предсказаний~46, неправильных~--- 54; $\mathrm{SR}\,(W_M, p, 
t) \hm= 0{,}46$; $p_{V=1}(W_M, f_p= 0) \hm= 2/53 \hm= 0{,}04$: $0{,}46\hm< 
1\hm- p_1 \hm= 0{,}53$ и~$0{,}04 \hm\ll 0{,}39/\tau$ для $\tau\hm = 2\mbox{--}5$. 
Этот предиктор может быть отвергнут вне зависимости от структуры 
множества ошибок~(\ref{e5-fr}). Предиктор не годится, он дает результаты 
хуже, чем~ПБ.
  
  Была также проведена проверка для предиктора SVR, для которого 
ситуация усугубилась большим числом прерываний ($z$):

\noindent
 1 1 1 1 1 1 1 1 1 1 $z$ 1 1 1 1 1 1 $-1$ 1 $-1$ $-1$
 1 1 $-1$ 1 1 $-1$ 1 1 1 1 1 1 1 1 1 1 1 1 1 1 1
 1 1 $z$ $z$ 1 1 1 1 1 1 1 1 1 1 1 1 1 1 1 $z$ 1
 1 1 1 1 $z$ 1 $z$ $z$ $z$ 1 $-1$ $-1$ 1 $z$ $z$ $z$ $-1$ $-1$ $-1$ $-1$ 1
 $-1$ 1 $-1$ $-1$ $-1$ $-1$ $-1$ $z$ 1 $-1$ $-1$ $-1$ $z$ 1 $-1$ $-1$
 
 \vspace*{2pt}
  
 \noindent
  В этом массиве 65 единиц и~22 нуля, а~также $z \hm= 13$. Правильно 
предсказанных единиц~--- 31, правильно предсказанных нулей~--- 13, 
неправильно предсказанных единиц~--- 16 (в том числе из 13~значений~$z$), 
неправильно предсказанных нулей~--- 40 (в~том числе из 13~значений~$z$). Всего 
правильных предсказаний~44, неправильных~--- 43 (и~еще 13~значений~$z$). 
На данном отрезке последовательности значение 
$\mathrm{SR}\,(\cdot) \hm= 0{,}44\hm< p_1 \hm= 0{,}53$, поэтому этот 
предиктор может быть отвергнут вне зависимости от структуры множества 
ошибок~(\ref{e5-fr}).
   
  Таким образом, с~точки зрения предлагаемого понимания 
предсказуемости данная последовательность является непредсказуемой 
относительно набора предикторов $\{$XGB, SGD, SVR$\}$. Заметим, что 
в~данном случае и~оценка параметра Бернулли, и~вычисленные величины 
$\mathrm{SR}\,(\cdot)$ близки к~1/2, что, согласно~\cite{6-fr}, соответствует 
области не\-опре\-де\-лен\-ности предсказания. В~этом случае соответствующая 
программа оценки предсказуемости информирует, например, 
администратора сети, что предсказания в~данный момент не будет и~ему 
следует использовать ка\-кие-ли\-бо другие средства для принятия решения. 

\vspace*{-6pt}
  
\section{Обсуждение и~заключение}

\vspace*{-2pt}

  В статье предложен и~рассмотрен новый подход к~выбору предикторов 
в~конкретный временной период, необходимых для предсказания будущих 
значений в~последовательностях данных, связанных с~функционированием 
различных ИТ-сис\-тем. Согласно сформулированным во введении 
требованиям к~реализации метода, рассмотрены недорогие реализуемые 
техники выбора предиктора и~принятия решения о~невозможности 
выполнить надежный прогноз, если обнаруживается, что данный участок 
последовательности не обладает свойством предсказуемости. Для этого 
предсказуемость данной последовательности определяется как максимальная 
по множеству доступных предикторов условная вероятность правильного 
предсказания на данном множестве предикторов~\cite{17-fr} при данном 
наборе наблюденных значений. Выбор (отсеивание) предикторов 
выполняется как по величине данной вероятности, так и~по степени отличия 
конкретного предиктора от простейшего~ПБ. 
  
  Существенно, что пользователь, не будучи экспертом в~области создания 
предикторов, не разрабатывает новых предикторов, в~част\-ности не 
вы\-чис\-ля\-ет тех или иных статистических характеристик наблюденных 
данных, более сложных, чем прос\-тая оценка частот событий, а~использует 
только доступные ему предикторы и~только ту информацию, 
которую он может извлечь из результатов работы доступного предиктора, 
а~именно: результаты предсказаний на накопленных к~моменту очередного 
предсказания данных и~истинные значения предсказываемых данных (т.\,е.\ 
обучающую выборку).
{\looseness=-1

}

\vspace*{-6pt}
   
{\small\frenchspacing
 {%\baselineskip=10.8pt
 \addcontentsline{toc}{section}{References}
 \begin{thebibliography}{99}
 
 \vspace*{-2pt}
  
  \bibitem{1-fr}
  \Au{Buczak L., Guven~E.} A~survey of data mining and machine learning methods for cyber 
security intrusion detection~// IEEE Commun. Surv. Tut., 2016. Vol.~18. Iss.~2. 
P.~1153--1176.
  \bibitem{2-fr}
  \Au{Rooba R., Vallimayil~V.} Semantic aware future page prediction based on domain~// Int. 
J.~Pure Appl. Math., 2018. Vol.~118. Iss.~9. P.~911--919.
  \bibitem{3-fr}
  \Au{Wolpert D., Macready~W.} No free lunch theorems for optimization~// IEEE~T. 
Evolut. Comput., 1997. Vol.~1. Iss.~1.  
P.~67--83.
  \bibitem{4-fr}
  \Au{Ryabko D., Ryabko~B.} Predicting the outcomes of every process for which an 
asymptotically accurate stationary predictor exists is impossible~// IEEE Symposium 
(International) on Information Theory.~--- IEEE, 2015. P.~1204--1206.
  
  \bibitem{6-fr} %5
\Au{Merhav N., Feder~M., Gutman~M.} Some properties of sequential predictors for binary 
Markov sources~// IEEE~T. Inform. Theory, 1993. Vol.~39. Iss.~3. P.~887--893.

\bibitem{5-fr} %6
  \Au{Feder M., Merhav N.} Universal prediction~// IEEE~T. Inform. Theory, 1998. 
Vol.~44. Iss.~6. P.~2124--2147.

  \bibitem{7-fr} %7
  \Au{Hodge V.\,J., Krishnan~R., Austin~J., Polak~J., Jackson ~T.} Short-term prediction of 
traffic flow using a~binary neural network~// Neural Comput. Appl., 2014. Vol.~25. 
P.~1639--1655. doi: 10.1007/s00521-014-1646-5.
  \bibitem{8-fr}
  \Au{Even-Dar E., Kearns~M., Mansour~Y., Wortman~J.} Regret to the best vs.\ regret to the 
average~// Mach. Learn., 2008. Vol.~72.  
P.~21--37.
  \bibitem{9-fr}
  \Au{Bass D.} Stochastic processes.~--- Cambridge, U.K.: Cambridge University Press, 2011. 
392~p.
  
  \bibitem{11-fr} %10
  \Au{Lavasani A., Eghlidos~T.} Practical next bit test for evaluating pseudorandom 
sequences~// J.~Sci. Technol., 
2009. Vol.~16. Iss.~1. P.~19--33.

\bibitem{10-fr} %11
  \Au{Nobel A.} Some stochastic properties of memoryless individual 
  sequences~// IEEE~T. 
Inform. Theory, 2004. Vol.~50. P.~1497--1505. 

  %\bibitem{12-fr}
  %\Au{Gu S., Kelly~B.} Empirical asset pricing via machine learning~// The Rev. 
%Financ. Stud., 2020. Vol.~33. P.~2223--2273.
 % \bibitem{13-fr}
 % \Au{Freud Y.} Predicting a~binary sequence almost as well as the optimal biased coin~// 
%Inform. Comput., 2003. Vol.~82. P.~73--94. 
  \bibitem{14-fr}
  \Au{Chen T., Guestrin~ C.} XGBoost: A~scalable tree boosting system~// 
arXiv:1603.02754v3 [cs.LG], 10 Jun 2016.
  %\bibitem{15-fr}
 % \Au{Qiang X., Che H., Ye~X., Su~R., Wei~L.} M6AMRFS: Robust prediction of 
 % N6-methyladenosine sites with sequence-based features in multiple species~// Frontiers Genetics, 
%2018. Vol.~9. Art. No.\,495.
  \bibitem{16-fr}
  \Au{Volovich~K., Denisov~S., Shabanov~A., Malkovsky~S.} Aspects of the assessment of the 
quality of loading hybrid high-performance computing cluster~// CEUR Workshop Proceedings, 
2019. Vol.~2426. P.~7--11.
  \bibitem{17-fr}
  \Au{Frenkel S.} On a~priory estimation of random sequences predictability~// 
  6th Workshop on Computational Data Analysis and Numerical Methods
  Book of Abstracts.~--- 
Covilh$\tilde{\mbox{a}}$, Portugal, 2019. P.~109--111. {\sf  
 http://www.wcdanm-ubi19.uevora.pt/wp-content/uploads/2019/09/Book-of-Abstracts.pdf}.
  
\end{thebibliography}

 }
 }

\end{multicols}

\vspace*{-6pt}

\hfill{\small\textit{Поступила в~редакцию 15.04.20}}

\vspace*{8pt}

%\pagebreak

%\newpage

%\vspace*{-28pt}

\hrule

\vspace*{2pt}

\hrule

%\vspace*{-2pt}

\def\tit{JOINT ASSESSMENT OF~DATA PREDICTABILITY AND~QUALITY~PREDICTORS}


\def\titkol{Joint assessment of data predictability and quality 
predictors}

\def\aut{S.\,L.~Frenkel and V.\,N.~Zakharov}

\def\autkol{S.\,L.~Frenkel and V.\,N.~Zakharov}

\titel{\tit}{\aut}{\autkol}{\titkol}

\vspace*{-9pt}


\noindent
Federal Research Center ``Computer Science and Control'' of the Russian Academy 
of Sciences, 44-2~Vavilov Str., Moscow 119333, Russian Federation


\def\leftfootline{\small{\textbf{\thepage}
\hfill INFORMATIKA I EE PRIMENENIYA~--- INFORMATICS AND
APPLICATIONS\ \ \ 2020\ \ \ volume~14\ \ \ issue\ 2}
}%
 \def\rightfootline{\small{INFORMATIKA I EE PRIMENENIYA~---
INFORMATICS AND APPLICATIONS\ \ \ 2020\ \ \ volume~14\ \ \ issue\ 2
\hfill \textbf{\thepage}}}

\vspace*{3pt} 
  


\Abste{The paper proposes and analyzes a~new approach to the selection of predictors necessary 
for predicting future values in data sequences in a~specific time period. Our goal is low-cost 
implemented techniques that ensure the selection of an acceptable predictor for the current 
prediction session, or the decision about the impossibility of making a~reliable forecast 
if one 
finds that this section of the sequence does not have the predictability property. For this, the 
predictability of this sequence is defined as the maximum conditional probability of the correct 
prediction in the set of available predictors for a~given set of observed values. The selection of 
predictors is performed by both the magnitude of the conditional probability 
estimation and 
the degree of difference between a~specific predictor and a~predictor that is optimal for 
predicting the next outcome of the Bernoulli trials sequence.}

\KWE{random sequences prediction; predictors; data analysis}


\DOI{10.14357/19922264200206} 

\vspace*{-19pt}

\Ack

\vspace*{-5pt}

\noindent
This research was partially supported by the Russian Foundation for Basic Research (projects 
18-07-00669,  
18-29-03100, and 18-07-01434).

\vspace*{-3pt}

 \begin{multicols}{2}

\renewcommand{\bibname}{\protect\rmfamily References}
%\renewcommand{\bibname}{\large\protect\rm References}

{\small\frenchspacing
 {%\baselineskip=10.8pt
 \addcontentsline{toc}{section}{References}
 \begin{thebibliography}{99}

\bibitem{1-fr-1}
\Aue{Buczak, L., and E.~Guven.} 2016. A~survey of data mining and machine learning 
methods for cyber security intrusion detection. \textit{IEEE Commun. Surv. 
Tut.} 18(2):1153--1176.
{\looseness=1

}


\bibitem{2-fr-1}
\Aue{Rooba, R., and V.~Vallimayil.}| 2018. Semantic aware future page prediction based on 
domain. \textit{Int. J.~Pure Appl. Math.} 118(9):911--919.

\vspace*{-2pt}

\bibitem{3-fr-1}
\Aue{Wolpert, D., and W.~Macready.} 1997. No free lunch theorems for optimization. 
\textit{IEEE T. Evolut. Comput.} 1(1):67--83.

\bibitem{4-fr-1}
\Aue{Ryabko, D., and B.~Ryabko.} 2015. Predicting the outcomes of every process for which 
an asymptotically accurate stationary predictor exists is impossible. \textit{IEEE 
Symposium (International) on Information Theory}. IEEE.\linebreak 1204--1206.

\bibitem{6-fr-1} %5
\Aue{Merhav, N., M.~Feder, and M.~Gutman.} 1993. Some properties of sequential predictors 
for binary Markov sources. \textit{IEEE~T. Inform. Theory} 39(3):887--893.

\bibitem{5-fr-1} %6
\Aue{Feder, M., and N.~Merhav.} 1998. Universal prediction. \textit{IEEE T. Inform. 
Theory} 44(6):2124--2147.

\bibitem{7-fr-1}
\Aue{Hodge, V., R.~Krishnan, J.~Austin, J.~Polak, and T.~Jackson.} 2014. Short-term 
prediction of traffic flow using a~binary neural network. \textit{Neural Comput.
 Appl.}  25:1639--1655.
\bibitem{8-fr-1}
\Aue{Even-Dar, E., M.~Kearns, Y.~Mansour, and J.~Wortman.} 2008. Regret to the best vs.\ 
regret to the average. \textit{Mach. Learn.}  72:21--37.
\bibitem{9-fr-1}
\Aue{Bass, D.} 2011. \textit{Stochastic processes}. Cambridge, U.K., Cambridge University 
Press. 392~p.

\bibitem{11-fr-1} %10
\Aue{Lavasani, A., and T.~Eghlidos.} 2009. Bit test for evaluating pseudorandom sequences. 
\textit{J.~Sci. Technol.} 16(1):19--33.

\bibitem{10-fr-1} %11
\Aue{Nobel, A.} 2004. Some stochastic properties of memoryless individual sequences. 
\textit{IEEE~T. Inform. Theory} 50:1497--1505.

%\bibitem{12-fr-1}
%\Aue{Gu., S., and B.~Kelly.} 2004. Empirical asset pricing via machine learning. 
%\textit{Rev. Financ. Stud.} 33:2223--2273.
%\bibitem{13-fr-1}
%\Aue{Freud, Y.} 2003. Predicting a~binary sequence almost as well as the optimal biased coin. 
%\textit{Inform. Comput.} 82:73--94. 
\bibitem{14-fr-1}
\Aue{Chen, T., and C.~Guestrin.} 2016. XGboost: A~scalable tree boosting system. 
\textit{arXiv.org}. Available at: {\sf https://arxiv.org/pdf/1603.02754.pdf } (accessed on 
April~15, 2020).
%\bibitem{15-fr-1}
%\Aue{Qiang, X., H. Che, X.~Ye, R.~Su, and L.~Wei.} 2018. M6AMRFS: Robust prediction of 
%N6-methyladenosine sites with sequence-based features in multiple species. \textit{Frontiers 
%Genetics} 9:495.
\bibitem{16-fr-1}
\Aue{Volovich, K., S.~Denisov, A.~Shabanov, and S.~Malkovsky.} 2019. Aspects of the 
assessment of the quality of loading hybrid high-performance computing cluster. \textit{CEUR 
Workshop Proceedings} 2426:7--11.
\bibitem{17-fr-1}
\Aue{Frenkel, S.} 2019. On \textit{a~priory} estimation of random sequences 
predictability. \textit{6th 
Workshop on Computational Data Analysis and Numerical Methods Book of Abstracts}.
Covilh$\tilde{\mbox{a}}$, Portugal. 
 109--111.
 Available at: {\sf  
 http://www.wcdanm-ubi19.uevora.pt/wp-content/uploads/2019/09/Book-of-Abstracts.pdf}
 (accessed June~22, 2020).

\end{thebibliography}

 }
 }

\end{multicols}

\vspace*{-6pt}

\hfill{\small\textit{Received April 15, 2020}}

%\pagebreak

%\vspace*{-24pt}


\Contr

\noindent
\textbf{Frenkel Sergey L.} (b.\ 1951)~--- Candidate of Science (PhD) in technology, associate 
professor, senior scientist, Institute of Informatics Problems, Federal Research Center 
``Computer Sciences and Control'' of the Russian Academy of Sciences, 44-2~Vavilov Str., 
Moscow 119333, Russian Federation; \mbox{fsergei51@gmail.com}

\vspace*{3pt}

\noindent
\textbf{Zakharov Victor N.} (b.\ 1948)~--- Doctor of Science in technology, associate professor; 
Scientific Secretary, Federal Research Center ``Computer Science and Control'' of the Russian 
Academy of Sciences,  
44-2~Vavilov Str., Moscow 119333, Russian Federation; \mbox{vzakharov@ipiran.ru}

\label{end\stat}

\renewcommand{\bibname}{\protect\rm Литература} 