\def\stat{bosov}

\def\tit{УПРАВЛЕНИЕ ВЫХОДОМ СТОХАСТИЧЕСКОЙ 
ДИФФЕРЕНЦИАЛЬНОЙ СИСТЕМЫ\\ ПО~КВАДРАТИЧНОМУ 
КРИТЕРИЮ.\\ V.~СЛУЧАЙ НЕПОЛНОЙ ИНФОРМАЦИИ 
О~СОСТОЯНИИ$^*$}

\def\titkol{Управление выходом стохастической 
дифференциальной системы по квадратичному критерию. V}
%.~Случай неполной информации о~состоянии}

\def\aut{А.\,В. Босов$^1$}

\def\autkol{А.\,В. Босов}

\titel{\tit}{\aut}{\autkol}{\titkol}

\index{Босов А.\,В.}
\index{Bosov A.\,V.}
 

{\renewcommand{\thefootnote}{\fnsymbol{footnote}} \footnotetext[1]
{Работа выполнена при частичной поддержке РФФИ (проект 19-07-00187-A).}}


\renewcommand{\thefootnote}{\arabic{footnote}}
\footnotetext[1]{Институт проблем информатики Федерального 
исследовательского центра <<Информатика 
и~управление>> Российской академии наук, \mbox{ABosov@frccsc.ru}}
    
     \vspace*{-12pt}
    
    \Abst{Рассматривается обобщение задачи оптимального управления 
для диффузионного процесса Ито и~линейного управляемого выхода 
с~квадратичным критерием качества на случай косвенных наблюдений за 
состоянием. Имеющееся решение задачи с~полной информацией 
используется для синтеза управления по косвенным наблюдениям на 
основании принципа разделения. Возможность разделения задач 
управления и~фильтрации состояния обосновывается свойствами 
используемого квадратичного критерия. Вместо решения возникающей 
вспомогательной задачи оптимальной фильтрации, описываемого общими 
уравнениями нелинейной фильтрации на основе обновляющих процессов, 
предлагается использовать оценку условно-оптимального фильтра (УОФ)
В.\,С.~Пугачева. Таким образом, субоптимальное решение 
рассматриваемой задачи управления получается как результат 
традиционного подхода к~синтезу управления в~задаче с~неполной 
информацией, состоящего в~формальной замене в~решении 
соответствующей задачи с~полной информацией переменной состояния на 
ее оценку. Окончательно предлагается вариант численной реализации 
предложенного субоптимального управления на основе метода 
имитационного компьютерного моделирования, использующий общий 
пучок моделируемых траекторий как для расчета параметров 
УОФ, так и~для вычисления параметров 
в~исходной задаче управления.}
    \KW{стохастическое дифференциальное уравнение; стохастическая 
дифференциальная система; оптимальное управление; стохастическая 
фильтрация; условно-оптимальная фильтрация; имитационное 
компьютерное моделирование}
    
\DOI{10.14357/19922264200203} 
 
%\vspace*{9pt}


\vskip 10pt plus 9pt minus 6pt

\thispagestyle{headings}

\begin{multicols}{2}

\label{st\stat}



\section{Введение}

     Задача управления линейным выходом стохастической 
дифференциальной системы по квадратичному критерию качества, 
постановка и~оптимальное решение, анализ свойств оптимального решения 
и~основанные на нем варианты численной реализации, а~также 
выполненные модельные расчеты представлены в~работах~[1--4]. 
В~постановке (здесь и~далее для простоты рассматриваются скалярные 
величины) используется состояние~$y_t$ стохастической дифференциальной 
системы, описываемое нелинейным стохастическим дифференциальным 
уравнением Ито:
     \begin{equation}
     dy_t=A_t(y_t)\,dt+\Sigma_t(y_t)\,dv_t\,,\enskip y_0=Y\,,
     \label{e1-bos}
     \end{equation}
где $v_t$~--- стандартный винеровский процесс; $Y$~--- случайная величина 
с~конечным вторым моментом; функции~$A_t$ и~$\Sigma_t$ удовлетворяют 
условиям Ито, обеспечивающим существование единственного 
решения~\cite{5-bos}.

     С состоянием~$y_t$ линейно связан выход~$z_t$:
     \begin{equation}
     dz_t= a_ty_t\,dt+b_tz_t\,dt+c_tu_t\,dt+\sigma_t\,dw_t\,,\  z_0=Z\,,
     \label{e2-bos}
     \end{equation}
где $w_t$~--- не зависящий от~$v_t$, $Y$ и~$Z$ стандартный винеровский 
процесс; $Z$~--- случайная величина с~конечным вторым моментом; 
$u_t$~--- допустимое управление. Функции~$a_t$, $b_t$, $c_t$ и~$\sigma_t$ 
предполагаются ограниченными, процесс управления~--- допустимым 
неупреждающим~\cite{5-bos}, что обеспечивает существование решения 
уравнения~(\ref{e2-bos}) для любого допустимого управления.

     Используется квадратичный целевой функционал следующего вида:
     
     \noindent
     \begin{multline}
    \!\! J\left( U_0^T\right)={\sf E}\left\{ \int\limits_0^T 
\left( S_t(s_ty_t-g_tz_t-h_tu_t)^2+G_tz_t^2+{}\!\right.\right.\\
\left.\left.{}+H_tu_t^2\right)dt+S_T(s_Ty_T-g_Tz_T)^2+G_Tz_T^2
\vphantom{\int\limits_0^T }
     \right\}\,,\\
     \enskip
     U_0^T=\left\{ u_t,0\leq t\leq T\right\}\,,
     \label{e3-bos}
     \end{multline}
где $S_t$, $G_t$ и~$H_t$~--- неотрицательные ограниченные функции.
     
     Задача была сформулирована в~предположении наличия полной 
информации о~состоянии~$y_t$ и~выходе~$z_t$ (соответствующая 
$\sigma$-ал\-геб\-ра, порожденная величинами~$y_\tau$, $z_\tau$, $0\hm\leq 
\tau\hm\leq t$, обозначается~$\mathcal{F}_t^{y,z}$), поэтому допустимое 
управление искалось в~классе  
$\mathcal{F}_t^{y,z}$-из\-ме\-ри\-мых неупреждающих функций и~оказалось 
управлением с~обратной связью:
\begin{multline}
u_t^*=u_t^*(y_t,z_t)= -
\fr{1}{2}\left( S_th_t^2+H_t\right)^{-1}\times {}\\
{}\times \left( c_t(2\alpha_tz_t+\beta_t(y_t))+2S_t(s_ty_t-g_tz_t)h_t\right).
\label{e4-bos}
\end{multline}
     %
     Это решение обеспечено представлением функции Беллмана в~виде 
$$
V_t(y,z)= \alpha_tz^2+\beta_t(y)z+\gamma_t(y)$$ 
и~решением 
уравнений динамического программирования, приведшим к~линейными 
уравнениями в~частных производных второго порядка параболического типа 
относительно коэффициентов $\beta_t(y)$ и~$\gamma_t(y)$  и~уравнению 
Риккати для коэффициента~$\alpha_t$. Подходы к~численной реализации 
по-\linebreak лученного оптимального решения обеспечили, во-пер\-вых, традиционные 
сеточные методы решения уравнений параболического типа, во-вто\-рых, 
альтернативное описание этих коэффициентов, базирующееся на уравнении 
А.\,Н.~Колмогорова~\cite{6-bos} или эквивалентной ему интегральной 
формуле Фейнмана--Каца~\cite{7-bos}, обосновавшие применение 
в~расчетах имитационного компьютерного моделирования~\cite{4-bos}.
     
     Цель данной работы, подводящей окончательный теоретический итог 
исследованию задачи, рассмотреть случай неполной информации 
о~состоянии, т.\,е.\ предположить, что полная информация 
о~состоянии~$y_t$ из~(\ref{e1-bos}) недоступна, и~интерпретировать 
уравнение~(\ref{e2-bos}) как уравнение наблюдений. Формально такая 
постановка требует определения допустимых управлений в~классе 
$\mathcal{F}_t^z$-из\-ме\-ри\-мых неупреждающих функций 
($\mathcal{F}_t^z$~--- $\sigma$-ал\-геб\-ра, по\-рож\-ден\-ная~$z_\tau$, 
$0\hm\leq \tau\hm\leq t$; также 
далее используется обозначение~${\sf E}\{\cdot\vert \mathcal{F}_t^z\}$~--- 
оператор условного математического ожидания 
относительно~$\mathcal{F}_t^z$). Но, даже несмотря на отсутствие 
управления в~уравнении состояния~(1), линейность наблюдений~(2) 
и~квад\-ра\-тич\-ный критерий~(3), получить оптимальное решение не удастся. 
Обсуждение этого положения и~вариант субоптимального управ\-ле\-ния, 
опи\-ра\-юще\-го\-ся на принцип разделения, обосно\-вы\-ва\-емый в~данной задаче 
свойствами квадратичного функционала качества, приводятся во втором 
разделе статьи. 
     
     Соображения по принципу разделения и~формулировка отдельно 
задачи фильтрации для системы наблюдения~(1), (2) с~точки зрения 
практической реализуемости приводят к~выбору в~качестве метода 
оценивания УОФ 
В.\,С.~Пугачева~\cite{8-bos}. При этом структура оценки УОФ 
такова, что позволяет еще предложить 
вариант численной реализации на основе метода имитационного 
компьютерного моделирования, вклю\-ча\-ющий в~том чис\-ле расчет 
параметров оптимального управ\-ле\-ния методом, представленным 
и~апробированным в~\cite{4-bos}. Этот алгоритм приведен в~разд.~4. Итоги 
исследования подведены в~заключении.

\section{Управление на~основе принципа разделения}
     
     Понятие <<принципа разделения>> интерпрети\-руется в~смысле 
классического результата о~разделении задач управления и~фильтрации 
в~линей-\linebreak но-гауссовской стохастической системе~\cite{9-bos}. Этот\linebreak результат 
базируется на двух положениях. Во-пер\-вых, это дифференциальная форма 
оценки фильт\-ра Калмана, условного математического ожидания\linebreak состояния 
в~ли\-ней\-но-гаус\-сов\-ской системе наблюдения, имеющая тот же вид, что 
и~уравнение со\-стояния, т.\,е.\ то же самое линейное уравнение, 
отли\-чающееся лишь ковариацией шума, роль которого выполняет 
обновляющий процесс, явля\-ющий\-ся винеровским. Во-вто\-рых, это 
воз\-мож\-ность пред\-став\-ле\-ния квад\-ра\-тич\-но\-го критерия\linebreak двумя сла\-га\-емы\-ми, 
отвечающими за качество управ\-ле\-ния и~качество оценивания состояния.
     
     Оба эти положения имеют свою интерпретацию в~рассматриваемой 
задаче, но вначале следует обсудить важный дополнительный аспект 
разделения, а~именно: отсутствие в~задаче управления дуального 
эффекта~\cite{10-bos}, т.\,е.\ влияния закона управ\-ле\-ния на качество 
оценивания состояния. С~этой целью видоизменим исходную постановку 
сле\-ду\-ющим образом. Выполним замену переменных в~(2), избавляясь от 
слагаемых $b_tz_t\,dt$ и~$c_tu_t\,dt$, приводя уравнение выхода к~некоторой 
канонической форме (традиционной форме записи наблюдений за 
состоянием). Для этого введем новую переменную 
$$
\tilde{z}_t= B_tz_t- \int\limits_0^t B_s c_s u_s\,ds\,,
$$
 где обозначено 
$$
B_t= \exp \left\{-\int\limits_0^t b_s\,ds\right\}.
$$

\pagebreak

 С~учетом того, что 
$$
dB_t= - b_t B_t\,dt\,,
$$ 
дифференциал введенной переменной принимает вид:
$$
d\tilde{z}_t= 
B_t\left(dz_t- b_t z_t\,dt- c_t u_t\,dt\right)=
 \tilde{a}y_t\,dt+ \tilde{\sigma}_t 
\,dw_t\,,
$$
 где обозначено 
 $$
 \tilde{a}_t=a_tB_t\,;\enskip \tilde{\sigma}_t=  \sigma_tB_t\,.
 $$
     
     Заметим, что проведенное преобразование обеспечивает совпадение 
$\sigma$-алгебр $\mathcal{F}_t^z$ и~$\mathcal{F}_t^{\tilde{z}}$ для любого 
допустимого управления~$u_t$. Действительно, $\mathcal{F}_t^z\hm= 
\mathcal{F}_t^{\tilde{z}}$, поскольку выполненная замена~--- линейное, 
невырожденное ($B_t\hm>0$) преобразование~$z_t$, а~$u_t$ по условию 
$\mathcal{F}_t^z$-измеримо. При этом полученное уравнение 
для~$\tilde{z}_t$ показывает, что $\tilde{z}_t$ не зависит от~$u_t$, т.\,е., 
используя~$\tilde{z}_t$ в~качестве наблюдения, можно строить оценку 
состояния, не зависящую от конкретного закона управления. Таким образом, 
вместо задачи оценивания состояния по наблюдениям~$z_t$, зависящим от 
реализуемого закона управления, можно рассматривать эквивалентную 
задачу оценивания~$y_t$ по наблюдениям~$\tilde{z}_t$, не зависящим 
от~$u_t$.
     
     Проведенная замена не означает разделение задач управления 
и~фильтрации в~рассматриваемой постановке, но обосновывает отсутствие 
в~ней дуального эффекта, так что, не ограничивая общ\-ности, задачу 
фильтрации можно решать, используя вместо уравнения выхода~(2) 
уравнение
     \begin{equation}
     d\tilde{z}_t=\tilde{a}y_t\,dt+\tilde{\sigma}_t\,dw_t\,,\enskip \tilde{z}_0=Z\,.
     \label{e5-bos}
     \end{equation}
     
     Оптимальную оценку состояния системы~(1) по наблюдениям~(\ref{e5-bos}), 
т.\,е.\ условное математическое ожидание ${\sf E}\{y_t\vert 
\mathcal{F}_t^{\tilde{z}}\}$, обозначим через~$\hat{y}_t$. Наложенные выше 
на систему~(1), (2) ограничения представляют, по-видимому, наиболее 
жесткий вариант условий, обеспечивающих существование этой оценки. 
Соотношения для нее основаны на принципиальных результатах, 
полученных для общей задачи нелинейной фильтрации на основе 
обновляющих процессов~\cite{11-bos}. Применительно к~рассматриваемой 
задаче оптимальную оценку дает следующее соотношение:
     \begin{equation}
     d\hat{y}_t= 
\widehat{A_t}\,dt+\fr{\widehat{\Sigma_t}}{\tilde{\sigma}_t}\,d\hat{v}_t\,,\enskip 
\hat{y}_0={\sf E}\{Y\}\,,
     \label{e6-bos}
     \end{equation}
где 
\begin{align*}
\widehat{A_t}&={\sf E}\left\{ A_t(y_t)\vert \mathcal{F}_t^{\tilde{z}}\right\};\\ 
\widehat{\Sigma_t}&= \tilde{a}_t {\sf E}\left\{ y_t(y_t\hm- \hat{y}_t)\vert 
\mathcal{F}_t^{\tilde{z}}\right\}\,;\\
d\hat{v}_t&=\fr{1}{\tilde{\sigma}_t}\left(d\tilde{z}_t- \tilde{a}_t\hat{y}_t\,dt\right)\,.
\end{align*}
Кроме того, 
обновляющий процесс~$\hat{v}_t$ является стандартным винеровским 
относительно~$\mathcal{F}_t^{\tilde{z}}$.

     Если исходить из того, что~(\ref{e6-bos}) используется в~качестве 
уравнения состояния, то наблюдения~(\ref{e5-bos}) можно задать 
следующим уравнением:
     \begin{equation}
     d\tilde{z}_t=\tilde{a}\hat{y}_t\,dt+\tilde{\sigma}_t\,d\hat{v}_t\,,\enskip 
\tilde{z}_0=Z\,.
     \label{e7-bos}
     \end{equation}
     %
     Это уравнение действительно описывает линейные наблюдения для 
состояния~$\hat{y}_t$, при том что эти наблюдения совпадают 
с~(\ref{e5-bos}). Таким образом, полученную пару уравнений~(\ref{e6-bos}), 
(\ref{e7-bos}) можно использовать в~качестве эквивалентного описания системы 
наблюдения.
     
     На этом возможности разделения, обеспечи\-ва\-емые средствами 
оптимальной фильтрации, в~рассматриваемой постановке исчерпываются: 
хотя и~удалось сохранить структуру наблюдений, уравнение оценки 
оказалось принципиально иным. Уравнение~(\ref{e6-bos}) никоим образом 
не повторяет форму~(1), так что готовое решение задачи управления по 
полной информации применить не получится. Кроме того, решение 
уравнения для оптимального фильт\-ра само по себе является существенной 
задачей. Это, строго говоря, не уравнение Ито, так как его параметры 
$\widehat{A_t}$ и~$\widehat{\Sigma_t}$ зависят от всей 
траектории~$\tilde{z}_\tau$, $0\hm\leq \tau\hm\leq t$, и~вычисление их~--- 
отдельная задача, требующая привлечения сложных дополнительных 
уравнений. Таким образом, все, на что позволяет опираться это уравнение, 
выражается целесообразностью использовать в~рассматриваемой задаче 
управления оценку фильтрации и~предполагать, что результат будет тем 
успешнее, чем ближе эта оценка к~эталонной, определяемой~(\ref{e6-bos}).
     
     Поддерживает этот вывод и~рассмотрение второго базового положения 
принципа разделения~--- разделение квадратичного критерия. Учитывая 
обозначения 
$$
\hat{y}_t= {\sf E}\left\{ y_t\vert \mathcal{F}_t^{\tilde{z}}\right\} =
{\sf E}\left\{ 
y_t\vert \mathcal{F}_t^{z}\right\}
$$ 
и~формулу полного математического ожидания, 
используемый функционал~(3) можно представить в~виде:
\begin{multline*}
 \!J\left( U_0^T\right)={\sf E}\left( \int\limits_0^T 
\left(S_t \left(s_t\hat{y}_t-g_tz_t-h_tu_t\right)^2 +G_tz_t^2+{}\right.\right.\hspace*{-3.52704pt}\\
\left.\left.{}
+H_tu_t^2
\vphantom{S_t\left(\right)^2}
\right)\,dt+S_T(s_T\hat{y}_T-g_Tz_T)^2+G_T 
z_T^2\vphantom{\int\limits_0^T}\right)+{}\\
    {}+{\sf E}\left\{ \int\limits_0^T S_ts_t^2 (y_t-\hat{y}_t)^2dt+
S_Ts_T^2(y_t-\hat{y}_T)^2\right\}.
     \end{multline*} 
     %
     В этом представлении квадратичного критерия выделено второе 
слагаемое, составляющее плату за ошибку оценивания. С~учетом сказанного 
выше в~отношении независимости оценки фильтрации~$\hat{y}_t$ от закона 
управления это слагаемое можно убрать из целевого функционала и,~таким 
образом, обосно\-вать использование в~оптимизационной постановке целевой 
функции вида:
     \begin{multline*}
 \!\!    J\left( U_0^T\right)={\sf E}\left\{ \int\limits_0^T 
\left(S_t(s_t\hat{y}_t-g_tz_t-h_tu_t)^2 +G_tz_t^2+{}\right.\!\right.\\
     \left.\left.{}
    +H_tu_t^2\right)dt+ 
\vphantom{\int\limits_0^T }
S_T(s_T\hat{y}_T-g_Tz_T)^2+G_Tz_T^2\right\}.
     %\label{e8-bos}
     \end{multline*} 
     
     Сказанное дает основание в~качестве результата неформального 
применения принципа разделения в~рассматриваемой задаче предложить 
субоптимальный вариант управления~$u_t^S$ вида
     \begin{equation}
     u_t^S=u_t^*\left( \hat{y}_t, z_t\right)\,,
     \label{e9-bos}
     \end{equation}
где~$u_t^*$ определено соотношением~(\ref{e4-bos})~--- решением исходной 
задачи оптимизации~(1)--(3); $\hat{y}_t$~--- оценка состояния~$y_t$ по 
преобразованным наблюдениям~(\ref{e5-bos}), не зависящая от закона 
управления~$u_t$. \mbox{Верхний} индекс~$S$ можно интерпретировать как 
подчеркивание субоптимальности управления~(\ref{e9-bos}) или как первую 
букву от английского separation (разделение). Нетрудно видеть, что это 
управление было бы оптимальным для задачи с~полной информацией 
с~состоянием~$y_t$ вида 
$$
dy_t= A_t(y_t)\,dt+ 
\fr{\Sigma_t(y_t)}{\sigma_t}\,dv_t,
$$
 выходом~(2) и~целевым функционалом~(3). 
Такое уравнение состояния можно интерпретировать как нулевое 
приближение решения задачи оптимальной фильтрации.
     
     Итак, получен вариант~(\ref{e9-bos}) субоптимального управления 
в~исходной задаче, выполняющий формально принцип разделения 
и~обладающий потенциалом практической реализуемости при условии 
приемлемого качества собственно управления. В~отношении этого варианта 
управления придется сделать еще одно упрощение. Поскольку шансов 
реализовать расчет условного математического ожидания ${\sf E}\{ y_t\vert 
\mathcal{F}_t^{\tilde{z}}\}$ практически нет, то вместо оптимального 
фильтра~(\ref{e6-bos}) в~расчетах придется использовать другую оценку 
фильтрации. Качество управления при этом, видимо, упадет, но можно 
предполагать, что потери будут тем меньше, чем лучше, т.\,е.\ ближе 
к~оптимальной оценке~(\ref{e6-bos}), будет используемый фильтр.

\section{Управление на~основе условно-оптимального фильтра}
     
     Итак, имеется необходимость отказаться от использования 
оптимальной оценки~--- условного математического ожидания ${\sf E}\{ y_t\vert 
\mathcal{F}_t^{\tilde{z}}\}$~--- в~пользу другого решения задачи 
фильтрации, не являющегося оптимальным, но обладающего хорошим 
качеством в~решении задачи оценивания состояния~$y_t$ по 
наблюдениям~$\tilde{z}_\tau$, $0\hm\leq \tau\hm\leq t$. Такую оценку будем 
обозначать через~$\tilde{y}_t$ и~определять ее методом 
услов\-но-оп\-ти\-маль\-ной фильтрации В.\,С.~Пугачева~\cite{8-bos}.
     
     Условно-оптимальная оценка~$\tilde{y}_t$ состояния сис\-те\-мы~(1) по 
наблюдениям~(2) ищется в~виде:
     \begin{multline}
     d\tilde{y}_t=\tilde{\alpha}_t\xi_t\left( 
\tilde{y}_t,\tilde{z}_t\right)dt+\tilde{\beta}_t\zeta_t\left( 
\tilde{y}_t,\tilde{z}_t\right)d\tilde{z}_t+\tilde{\gamma}_t\,dt\,,\\[1pt]
\enskip 
\tilde{y}_0={\sf E}\{Y\}\,,
     \label{e10-bos}
     \end{multline}
где $\tilde{\alpha}_t$, $\tilde{\beta}_t$ и~$\tilde{\gamma}_t$~--- ограниченные 
функции времени, выполняющие роль параметров; функции~$\xi_t(y,z)$ 
и~$\zeta_t(y,z)$~--- заданные структурные функции, выбираемые из 
эмпирических соображений.

     Для обеспечения условий существования решения дифференциального 
уравнения~(\ref{e10-bos}) будем накладывать на структурные функции те же 
ограничения, что и~на функции сноса и~диффузии в~(1) с~учетом 
ограничений, наложенных на параметры уравнения~(2), т.\,е.
\begin{multline*}
\left\vert \xi_t(y,z)\right\vert + \left\vert \zeta_t(y,z)\right\vert 
\leq C(1+\vert y\vert +\vert z\vert )\\[1pt] \mbox{для\ всех } 0\leq t\leq T\,;
\end{multline*}

\vspace*{-12pt}

\noindent
\begin{multline*}
\left\vert \xi_t (y_1,z_1)-\xi_t(y_1,z_1)\right\vert +\left\vert 
\zeta_t(y_1,z_1)-\zeta_t(y_2,z_2)\right\vert \leq{}\\[1pt]
{}\leq C\left( \left\vert y_1-y_2\right\vert 
+\left\vert z_1-z_2\right\vert \right)\\[1pt]
 \mbox{для\ всех }0\leq t\leq T\ 
\mbox{и}\ y_1, y_2, z_1, z_2\in\mathbb{R}^1\,.
\end{multline*}
     
     В рассматриваемой задаче в~качестве эмпирических соображений, 
обосновывающих выбор структуры УОФ, предлагается использовать 
записанные выше уравнения оптимальной оценки~(\ref{e6-bos}), точнее их 
структуру, а~именно: представление правой час\-ти~(\ref{e6-bos}) в~виде 
$(\widehat{A_t}\hm- (\widehat{\Sigma_t}/\tilde{\sigma}_t^2)
\tilde{a}_t\hat{y}_t)\,dt\hm+ 
(\widehat{\Sigma_t}/\tilde{\sigma}_t^2)\,d\tilde{z}_t$ обосно\-вы\-ва\-ет 
предложение использовать в~качестве структурных функций 
\begin{align*}
\xi_t(y,z)&=\xi_t(y)= A_t(y)-
\fr{\Sigma_t(y)}{\tilde{\sigma}_t^2}\,\tilde{a}_t y\,;\\ 
\zeta_t(y,z)&= \zeta_t(y)= \fr{\Sigma_t(y)}{\tilde{\sigma}_t^2}\,,
\end{align*}
 т.\,е.\ 
оценку фильтрации в~виде:

\noindent
     \begin{multline}
d\tilde{y}_t= \tilde{\alpha}_t\left( A_t(\tilde{y}_t) - 
\fr{\Sigma_t(\tilde{y}_t)}{\tilde{\sigma}_t^2}\,\tilde{a}_t\tilde{y}_t\right) dt 
+\tilde{\beta}_t 
\fr{\Sigma_t(\tilde{y}_t)}{\tilde{\sigma}_t^2}\,d\tilde{z}_t+{}\\
{}+\tilde{\gamma}_t\,dt\,,\enskip \tilde{y}_0={\sf E}\{Y\}\,.
     \label{e11-bos}
     \end{multline} 
     %
     Здесь, как и~для варианта управления~(\ref{e9-bos}), величины 
$A_t(\tilde{y}_t)$ и~$\Sigma_t(\tilde{y}_t)$ можно интерпретировать как нулевые 
приближения для $\widehat{A_t}$ и~$\widehat{\Sigma_t}$, которые уточняются 
согласно методу услов\-но-оп\-ти\-маль\-ной фильт\-ра\-ции коэффициентами 
$\tilde{\alpha}_t$, $\tilde{\beta}_t$ и~$\tilde{\gamma}_t$, обеспечивающими оценке 
фильтра~(\ref{e11-bos}) несмещенность и~гарантированное качество, лучшее 
на некотором классе допустимых фильтров~\cite{8-bos}.
     
     Соотношения для коэффициентов УОФ приведены в~\cite{8-bos} 
и~представляют собой комбинации моментных характеристик 
состояния~$y_t$, наблюдений~$\tilde{z}_t$ и~структурных функций~$\xi_t$ 
и~$\zeta_t$, т.\,е.\ не зависят от реализуемого закона управления~$u_t$. 
Приемлемый вариант для расчета этих коэффициентов на основе 
имитационного компьютерного моделирования приведен в~следующем 
разделе.

\section{Численная реализация управления~$u_t^S$}

     Подведем итог рассуждениям предыдущего раздела, перечислив шаги, 
выполняемые для реализации варианта управления~$u_t^S$ для решения 
исходной задачи оптимизации, включающей уравнение~(1) ненаблюдаемого 
состояния, уравнение~(2) выхода, формирующего косвенные наблюдения, 
и~критерия~(3), минимизируемого на классе $\mathcal{F}_t^z$-из\-ме\-ри\-мых 
неупреждающих управлений.
     
    \textbf{Шаг~1.} Рассматривается вспомогательная задача условно-оптимальной 
фильтрации для системы наблюдения и~фильтра вида
     \begin{equation}
     \left.
     \begin{array}{rl}
     dy_t&= A_t(y_t)\,dt+\Sigma_t(y_t)\,dv_t\,,\enskip y_0=Y\,;\\[6pt]
     d\tilde{z}_t&= \tilde{a}_t y_t\,dt+\tilde{\sigma}_t \,dw_t\,,\enskip 
z_0=Z\,;\\[6pt]
     d\tilde{y}_t&= \tilde{\alpha}_t\xi_t(\tilde{y}_t)\,dt 
+\tilde{\beta}_t\zeta_t(\tilde{y}_t)\,d\tilde{z}_t+\tilde{\gamma}_t\,dt\,,
\\[6pt]
&\hspace*{35mm}\tilde{y}_0={\sf E}\{Y\}\,,
\end{array}
\right\}
     \label{e12-bos}
\end{equation}
где
$$
     \xi(\tilde{y}_t)=A_t(\tilde{y}_t)-
\fr{\Sigma_t(\tilde{y}_t)}{\tilde{\sigma}_t^2}\,\tilde{a}_y\tilde{y}_t\,;\enskip 
\zeta_t(\tilde{y}_t)= \fr{\Sigma_t(\tilde{y}_t)}{\tilde{\sigma}_t^2}\,.
  $$
     
     Решение~(\ref{e12-bos})~--- коэффициенты $\tilde{\alpha}_t$, 
$\tilde{\beta}_t$ и~$\tilde{\gamma}_t$ УОФ~--- определяются следующей системой 
уравнений (это частный случай решения задачи УОФ для системы 
наблюдения общего вида~\cite{8-bos}):

\noindent
     \begin{equation}
     \left.
     \begin{array}{l}
     \hspace*{-2.5mm}\tilde{\gamma}_t={\sf E}\left\{A_t\right\} - \tilde{\alpha}_t {\sf E}\{\xi_t\} -
\tilde{\beta}_t\tilde{a}_t {\sf E}\{y_t\}\,;\\[9pt]
  \hspace*{-2.5mm} \tilde{\beta}_t= \fr{\tilde{a}_t {\sf E}\left\{(y_t-\tilde{y}_t) y_t 
\zeta_t\right\}}{\tilde{\sigma}_t^2 {\sf E}\{\zeta_t^2\}}\,;\\[9pt]
    \hspace*{-2.5mm} \tilde{\alpha}_t= {\sf E}\left\{\!\left(\!
y_t-\tilde{y}_t\!\right)\! \Bigg(\!\fr{\partial \xi_t}{\partial t} +
\fr{1}{2}\,\fr{\partial^2\xi_t}{\partial 
(\tilde{y}_t)^2}\left(\tilde{\beta}_t\zeta_t \tilde{\sigma}_t)^2\right)+ {}\right.\\[6pt]
\hspace*{-2.5mm}{}+\left(y_t-\tilde{y}_t\right) 
\fr{\partial \xi_t}{\partial \tilde{y}_t}\,{\sf E}\{A_t\} +
\tilde{\beta}_t 
\tilde{a}_t\overline{\zeta_t y_t}\Bigg)-{}\\[6pt]
{}-\fr{\partial \xi_t}{\partial 
\tilde{y}_t}\,\left(\tilde{\beta}_t\zeta_t\tilde{\sigma}_t\right)^2+
\xi_t\left(\overline{A}_t-
\tilde{\beta}_t\tilde{a}_t \overline{\zeta_t y_t}
\right)\Bigg\}\times{}\\[6pt]
\hspace*{14mm}{}\times
\left(
{\sf E}\{ \xi_t \overline{\xi}_t-(y_t- \tilde{y}_t)
\fr{\partial \xi_t}{\partial \tilde{y}_t}\,\overline{\xi}_t\}\right)^{-1}\,,
     \end{array}\!\!\!
     \right\}\!\!
     \label{e13-bos}
     \end{equation}
где обозначено: 
\begin{alignat*}{2}
A_t&= A_t\left(y_t\right);&\enskip \overline{A}_t&=A_t- {\sf E}\{A_t\};\\ 
\xi_t&= \xi_t\left(\tilde{y}_t\right);&\enskip \overline{\xi}_t&=\xi_t- 
{\sf E}\left\{\xi_t\right\}\,;\\
\zeta_t&= \zeta_t(\tilde{y}_t);&\enskip 
\overline{\zeta_t y_t}&= \zeta_t y_t- {\sf E}\left\{\zeta_t y_t\right\}\,.
\end{alignat*}
     
     В качестве численной реализации оценки~$\tilde{y}_t$ предлагается 
использовать метод имитационного компьютерного  моделирования, 
а~именно: разбивая ось времени на интервалы равной малой длины~$\Delta 
t$, для вычисления коэффициентов $\tilde{\alpha}_t$, $\tilde{\beta}_t$ 
и~$\tilde{\gamma}_t$ в~правой части~(\ref{e13-bos}) использовать вычисленные 
на предыдущем шаге $\tilde{\alpha}_{t-\Delta t}$ и~$\tilde{\beta}_{t-\Delta t}$, 
а~моменты~${\sf E}\{\cdot\}$ заменять на их оценки статистическим средним, для 
чего моделировать пучок траекторий процессов $y_t$, $\tilde{z}_t$
и~$\tilde{y}_t$ 
Заметим, что аналогичный метод, основанный на формуле  
Фейн\-ма\-на--Ка\-ца, использован в~\cite{4-bos} для расчета коэффициентов 
$\alpha_t$, $\beta_t(y)$ и~$\gamma_t(y)$ в~задаче оптимального управления по 
полной информации и~хорошо себя показал. Кроме того, для обоих расчетов 
(параметров управления и~коэффициентов УОФ) можно воспользоваться 
одним и~тем же ресурсом~--- смоделированным пучком траекторий.
     
     \textbf{Шаг~2.} Рассматривается вспомогательная задача управления с~тем же 
целевым функционалом~(3), линейным выходом~(2) и~состоянием~(1), но 
в~предположении наблюдаемости состояния. Решение~--- коэффициенты 
$\alpha_t$ и~$\beta_t(y)$, формирующие закон управления $u_t^*\hm= u_t^*(y_t, 
z_t)$ согласно~(4),~--- находится любым из двух численных методов, 
предложенных в~\cite{2-bos, 4-bos}.
     
     \textbf{Шаг~3.} Формируется расширенная (эквивалентная) постановка для 
исходной задачи, вклю\-ча\-ющая уравнение состояния~(1), линейный 
выход~(2) и~псевдонаблюдения 
$$
\tilde{z}_t= B_tz_t- \int\limits_0^t 
B_s c_s u_s^S\,ds\,.
$$
 Предложенный вариант управления~$u_t^S$ вычисляется 
по формуле~(\ref{e9-bos}), где используются параметры управления 
$\alpha_t$ и~$\beta_t(y)$, вычисленные на шаге~2, и~оценка УОФ~$\tilde{y}_t$ 
по наблюдениям~$\tilde{z}_t$ с~коэффициентами~$\tilde{\alpha}_t$, 
$\tilde{\beta}_t$ и~$\tilde{\gamma}_t$ вычисленными на шаге~1.

\section{Заключение}

     В статье подведен итог исследованию задачи оптимизации линейного 
выхода нелинейной дифференциальной системы по квадратичному 
критерию, выполненному в~[1--4]. Оптимальные алгоритмы, полученные для 
рассматриваемой задачи оптимизации в~случае полной информации, т.\,е.\ 
при наличии возможности наблюдать и~выход сис\-те\-мы, и~ее состояние, 
проанализированы в~связи с~постановкой задачи при неполной информации, а~именно:
в~случае, когда выход выполняет функцию косвенных наблюдений за 
неизвестным состоянием системы. Возможности получить и~в~такой 
постановке оптимальные решения не выявлены, поэтому предложен вариант 
субоптимального решения, опирающийся на концепцию разделения задач 
управления и~фильтрации и~на метод услов\-но-оп\-ти\-маль\-ной 
фильт\-ра\-ции состояний стохастических дифференциальных систем наблюдения 
В.\,С.~Пугачева.
     
{\small\frenchspacing
 {%\baselineskip=10.8pt
 \addcontentsline{toc}{section}{References}
 \begin{thebibliography}{99}

\bibitem{1-bos}
\Au{Босов А.\,В., Стефанович А.\,И.} Управление выходом 
стохастической дифференциальной системы по квад\-ра\-тич\-но\-му 
критерию. I.~Оптимальное решение методом динамического 
программирования~// Информатика и~её применения, 2018. Т.~12. 
Вып.~3. С.~99--106.
\bibitem{2-bos}
\Au{Босов А.\,В., Стефанович А.\,И.} Управление выходом 
стохастической дифференциальной системы по квад\-ра\-тич\-но\-му 
критерию. II.~Численное решение уравнений динамического 
программирования~// Информатика и~её применения, 2019. Т.~13. 
Вып.~1. С.~9--15.
\bibitem{3-bos}
\Au{Босов А.\,В., Стефанович А.\,И.} Управление выходом 
стохастической дифференциальной системы по квад\-ра\-тич\-но\-му 
критерию. III.~Анализ свойств оптимального управления~// 
Информатика и~её применения, 2019. Т.~13. Вып.~3. С.~41--49.
\bibitem{4-bos}
\Au{Босов А.\,В., Стефанович А.\,И.} Управление выходом стохастической 
дифференциальной системы по квад\-ра\-тич\-но\-му критерию. 
IV.~Альтернативное чис\-лен\-ное решение~// Информатика и~её 
применения, 2020. Т.~14. Вып.~1. С.~24--30.
\bibitem{5-bos}
\Au{Флеминг У., Ришел Р.} Оптимальное управление 
детерминированными и~стохастическими системами~/ Пер. с~англ.~--- 
М.: Мир, 1978. 316~с. (\Au{Fleming~W.\,H., Rishel~R.\,W.} Deterministic 
and stochastic optimal control.~--- New York, NY, USA: Springer-Verlag, 
1975. 222~p.)
\bibitem{6-bos}
\Au{Гихман И.\,И., Скороход А.\,В.} Теория случайных процессов.~--- М.: Наука, 1975. 
Т.~III. 496~с.
\bibitem{7-bos}
\Au{{\ptb{\O}}\,\,ksendal~B.} Stochastic differential equations. An introduction 
with applications.~--- New York, NY, USA: Springer-Verlag, 2003. 379~p.
\bibitem{8-bos}
\Au{Пугачев В.\,С., Синицын И.\,Н.} Стохастические 
дифференциальные системы. Анализ и~фильтрация.~--- М.: Наука, 
1990. 632~с.
\bibitem{9-bos}
\Au{Davis M.\,H.\,A.} Linear estimation and stochastic control.~--- London: 
Chapman and Hall, 1977. 224~p.
\bibitem{10-bos}
\Au{Фельдбаум А.\,А.} Основы теории оптимальных автоматических 
систем.~--- 2-е изд., испр. и~доп.~--- М.: Наука, 1966. 624~с.
\bibitem{11-bos}
\Au{Липцер Р.\,Ш., Ширяев А.\,Н.} Статистика случайных процессов 
(нелинейная фильтрация и~смежные вопросы).~--- М.: Наука, 1974. 
696~с.

\end{thebibliography}

 }
 }

\end{multicols}

%\vspace*{-12pt}

\hfill{\small\textit{Поступила в~редакцию 30.01.20}}

\vspace*{8pt}

%\pagebreak

%\newpage

%\vspace*{-28pt}

\hrule

\vspace*{2pt}

\hrule

%\vspace*{-2pt}

\def\tit{STOCHASTIC DIFFERENTIAL SYSTEM OUTPUT CONTROL BY~THE~QUADRATIC 
CRITERION.\\ V.~CASE OF~INCOMPLETE STATE INFORMATION}



\def\titkol{Stochastic differential 
system output control by the quadratic 
criterion. V. Case of incomplete state information}

\def\aut{A.\,V.~Bosov}

\def\autkol{A.\,V.~Bosov}

\titel{\tit}{\aut}{\autkol}{\titkol}

\vspace*{-9pt}


\noindent
Institute of Informatics Problems, Federal Research Center ``Computer 
Science 
and Control'' of the Russian Academy of Sciences, 44-2~Vavilov Str., 
Moscow 
119333, Russian Federation

\def\leftfootline{\small{\textbf{\thepage}
\hfill INFORMATIKA I EE PRIMENENIYA~--- INFORMATICS AND
APPLICATIONS\ \ \ 2020\ \ \ volume~14\ \ \ issue\ 2}
}%
 \def\rightfootline{\small{INFORMATIKA I EE PRIMENENIYA~---
INFORMATICS AND APPLICATIONS\ \ \ 2020\ \ \ volume~14\ \ \ issue\ 2
\hfill \textbf{\thepage}}}

\vspace*{3pt} 

\Abste{ A generalization of the optimal control problem for the Ito diffusion process and a~linear 
controlled output with a~quadratic quality criterion for the case of indirect observation of the state is 
considered. The available solution of the problem with full information is used to synthesize control 
from indirect observations based on the principle of separation. The ability to separate control and 
state filtering tasks is justified by the properties\linebreak\vspace*{-12pt}}

\Abstend{ of the quadratic criterion used. Instead of solving the 
arising auxiliary problem of optimal filtering described by
the general equations of nonlinear filtering 
based on innovation processes, it is proposed to use the estimate of the conditionally optimal filter of 
V.\,S.~Pugachev. Thus, the suboptimal solution of the control problem under consideration is 
obtained as a~result of the traditional approach to control synthesis in the problem with incomplete 
information, consisting in a~formal replacement in solving the corresponding problem with complete 
information of the state variable by its estimate. Finally, a~case of numerical implementation of the 
obtained suboptimal control is proposed. It is based on the method of computer simulation, using 
a~common beam of simulated paths both for calculating the parameters of a~conditionally optimal 
filter and for calculating the parameters in the original control problem.}
\KWE{stochastic differential equation; stochastic differential system; optimal control; stochastic 
filtering; conditionally-optimal filtering; computer simulations}

\DOI{10.14357/19922264200203}

%\vspace*{-20pt}

\Ack
\noindent
This work was partially supported by the Russian Foundation for Basic Research (grant 19-07-00187-A).

%\vspace*{6pt}

 \begin{multicols}{2}

\renewcommand{\bibname}{\protect\rmfamily References}
%\renewcommand{\bibname}{\large\protect\rm References}

{\small\frenchspacing
 {%\baselineskip=10.8pt
 \addcontentsline{toc}{section}{References}
 \begin{thebibliography}{99}


\bibitem{1-bos-1}
\Aue{Bosov, A.\,V., and A.\,I.~Stefanovich.} 2018. Upravlenie vykhodom 
stokhasticheskoy differentsial'noy sistemy po kvadratichnomu kriteriyu. 
I.~Optimal'noe reshenie metodom dinamicheskogo programmirovaniya 
[Stochastic differential system output control by the quadratic criterion. 
I.~Dynamic programming optimal solution]. \textit{Informatika i ee 
Primeneniya~--- Inform. Appl.} 12(3):99--106.
\bibitem{2-bos-1}
\Aue{Bosov, A.\,V., and A.\,I.~Stefanovich.} 2019. Upravlenie vykhodom 
stokhasticheskoy differentsial'noy sistemy po kvadratichnomu kriteriyu. 
II.~Chislennoe reshenie uravneniy dinamicheskogo programmirovaniya 
[Stochastic differential system output control by the quadratic criterion. 
II.~Dynamic programming equations numerical solution]. 
\textit{Informatika i ee Primeneniya~--- Inform. Appl.} 13(1):9--15.
\bibitem{3-bos-1}
\Aue{Bosov, A.\,V., and A.\,I.~Stefanovich.} 2019. Upravlenie vykhodom 
stokhasticheskoy differentsial'noy sistemy po kvadratichnomu kriteriyu. 
III.~Analiz svoystv optimal'nogo upravleniya [Stochastic differential system 
output control by the quadratic criterion. III.~Optimal control properties 
analysis]. \textit{Informatika i~ee Primeneniya~--- Inform. Appl.} 13(3):41--49.
\bibitem{4-bos-1}
\Aue{Bosov, A.\,V., and A.\,I.~Stefanovich.} 2020. Upravlenie vykhodom 
stokhasticheskoy differentsial'noy sistemy po kvadratichnomu kriteriyu. 
IV.~Al'ternativnoe chislennoe reshenie [Stochastic differential system 
output control by the quadratic criterion. IV.~Alternative numerical 
decision]. \textit{Informatika i~ee Primeneniya~--- Inform. Appl.} 14(1):24--30.
\bibitem{5-bos-1}
\Aue{Fleming, W.\,H., and R.\,W.~Rishel.} 1975. \textit{Deterministic and 
stochastic optimal control.} New York, NY: Springer-Verlag. 222~p.
\bibitem{6-bos-1}
\Aue{Gihman, I.\,I., and A.\,V.~Skorohod.} 2012. \textit{The theory of 
stochastic processes}. New York, NY: Springer-Verlag. Vol.~3. 388~p.
\bibitem{7-bos-1}
\Aue{{\ptb{\O}}ksendal, B.} 2003. \textit{Stochastic differential equations. 
An introduction with applications}. New York, NY: Springer-Verlag. 379~p.
\bibitem{8-bos-1}
\Aue{Pugachev, V.\,S., and I.\,N.~Sinitsyn.} 1987. \textit{Stochastic 
differential systems. Analysis and filtering}. Chichester, NY: J.~Wiley \& 
Sons. 549~p.
\bibitem{9-bos-1}
\Aue{Davis, M.\,H.\,A.} 1977. \textit{Linear estimation and stochastic 
control}. London: Chapman and Hall. 224~p.
\bibitem{10-bos-1}
\Aue{Feldbaum, A.\,A.} 1966. \textit{Osnovy teorii optimal'nykh 
avtomaticheskikh system} [Foundations of theory of optimal automatic 
systems]. Moscow: Nauka. 624~p.
\bibitem{11-bos-1}
\Aue{Liptser, R.\,S., and A.\,N.~Shiryaev}. 2001. \textit{Statistics of 
random processes~II.~Applications}. Berlin: Springer-Verlag. 402~p.
\end{thebibliography}

 }
 }

\end{multicols}

\vspace*{-9pt}

\hfill{\small\textit{Received January 30, 2020}}

%\pagebreak

%\vspace*{-24pt}



\Contrl

\noindent
\textbf{Bosov Alexey V.} (b.\ 1969)~--- Doctor of Science in technology, principal scientist, 
Institute of Informatics Problems, Federal Research Center ``Computer Science and Control'' of 
the Russian Academy of Sciences, 44-2~Vavilov Str., Moscow 119333, Russian Federation; 
\mbox{AVBosov@ipiran.ru}

\label{end\stat}

\renewcommand{\bibname}{\protect\rm Литература} 