\def\stat{abgaryan}

\def\tit{ИНТЕГРАЦИОННАЯ ПЛАТФОРМА ДЛЯ 
МНОГОМАСШТАБНОГО МОДЕЛИРОВАНИЯ НЕЙРОМОРФНЫХ 
СИСТЕМ$^*$}

\def\titkol{Интеграционная платформа для многомасштабного 
моделирования нейроморфных систем}

\def\aut{К.\,К.~Абгарян$^1$, Е.\,С.~Гаврилов$^2$}

\def\autkol{К.\,К.~Абгарян, Е.\,С.~Гаврилов}

\titel{\tit}{\aut}{\autkol}{\titkol}

\index{Абгарян К.\,К.}
\index{Гаврилов Е.\,С.}
\index{Abgaryan K.\,K.}
\index{Gavrilov E.\,S.}
 

{\renewcommand{\thefootnote}{\fnsymbol{footnote}} \footnotetext[1]
{Работа выполнена при финансовой поддержке РФФИ (проект 19-29-03051~мк).}}


\renewcommand{\thefootnote}{\arabic{footnote}}
\footnotetext[1]{Вычислительный центр им.\ А.\,А.~Дородницына Федерального 
исследовательского центра <<Информатика и~управление>> Российской академии наук; 
Московский авиационный институт (национальный исследовательский университет), 
\mbox{kristal83@mail.ru}}
\footnotetext[2]{Вычислительный центр им.\ А.\,А.~Дородницына Федерального 
исследовательского центра <<Информатика и~управление>> Российской академии наук; 
Московский авиационный институт (национальный исследовательский университет), 
\mbox{eugavrilov@gmail.com}}

%\vspace*{-6pt}

     \Abst{Актуальные сегодня многоуровневые элементы резистивной памяти позволяют 
увеличить плотность интеграции энергонезависимой памяти, а~также спроектировать 
и~создать системы с~механизмом параллельных вычислений. В~основе таких устройств лежат 
мемристорные элементы, необходимые для разработки основ аналоговых нейроморфных сетей, 
которые используются для решения задач интеллектуального анализа данных. Однако 
использование мемристоров в~составе нейроморфных устройств сталкивается с~рядом 
проблем, таких как разброс значений параметров переключения (напряжение, окно памяти) от 
ячейки к~ячейке, асимметричность и~нелинейные эффекты и~др. Такие проблемы диктуют 
необходимость создания оригинальных имитационных моделей и~новых программных 
инструментов, которые позволят оценить влияние возмущающих факторов на 
предсказательную точность и~процесс обучения сети. В~данной работе для решения задачи 
многомасштабного моделирования нейроморфных систем применяется оригинальная 
информационная технология построения многомасштабных моделей. Для ее практической 
реализации построена интеграционная платформа, которая позволяет оценить влияние 
возмущающих факторов на предсказательную точность и~процесс обучения нейроморфной 
сети, а~в~дальнейшем сможет обеспечить формирование информации для обоснованного 
выбора материалов, конфигурации и~топологии ячеек памяти компьютеров нового поколения.}
     
     \KW{многомасштабное моделирование; многоуровневые элементы памяти; 
нейроморфные сети; предсказательное моделирование; мемристор; интеграционная платформа; 
программный комплекс}
     
\DOI{10.14357/19922264200215} 
 
\vspace*{-6pt}


\vskip 10pt plus 9pt minus 6pt

\thispagestyle{headings}

\begin{multicols}{2}

\label{st\stat}
     
\section{Введение}

    В настоящее время искусственные нейронные сети стали мощным 
инструментом интеллектуального анализа данных, построения 
аппрок\-си\-ма\-ционных моделей сложных систем и~процессов, распознавания 
образов, классификации и~кластеризации. Актуальные сегодня многоуровневые 
элементы резистивной памяти позволяют увеличить плотность интеграции 
энергонезависимой памяти, а~также спроектировать и~создать системы 
с~механизмом параллельных вычислений, которые необходимы для разработки 
основ аналоговых нейроморфных сетей. Ожидается, что на основе таких 
яче\-ек/мат\-риц резистивной памяти будет создана элементная база компьютеров 
следующего поколения, работающих на новых физических прин\-ципах.
{\looseness=1

}
    
     \begin{table*}\small %tabl1
     \begin{center}

    %\tabcolsep=0pt
    \begin{tabular}{|c|l|l|}
     \multicolumn{3}{c}{Распределение БК по масштабным 
уровням}\\
     \multicolumn{3}{c}{\ }\\[-6pt]
     \hline
\tabcolsep=0pt\begin{tabular}{c}№\\ масштаб-\\ ного \\ уровня\end{tabular}&
\multicolumn{1}{c|}{\tabcolsep=0pt\begin{tabular}{c} 
Обозначение и~название БК\end{tabular}}&
\multicolumn{1}{c|}{\tabcolsep=0pt\begin{tabular}{c} Название\\ масштабного\\
 уровня\end{tabular}}\\
\hline
0&$\mathbf{MC}_0^1$ <<АТОМ $A_0^i$>>&\tabcolsep=0pt\begin{tabular}{l} 
Уровень\\ химических\\ элементов\end{tabular}\\
\hline
1&\tabcolsep=0pt\begin{tabular}{l}$\mathbf{MC}_1^1$<<КРИСТАЛЛОХИМИЧЕСКАЯ ФОРМУЛА>>\\
$\mathbf{MC}_1^2$  
<<КВАНТОВО-МЕХАНИЧЕСКАЯ 
ЯЧЕЙКА>>\end{tabular}&\tabcolsep=0pt\begin{tabular}{l} Квантово-\\ 
механический\\ уровень\end{tabular}\\
\hline
2&\tabcolsep=0pt\begin{tabular}{l} 
$\mathbf{MC}_2^1$ <<АТОМНЫЙ КЛАСТЕР\,--\,СТАТИКА>>\\
$\mathbf{MC}_2^2$ <<АТОМНЫЙ КЛАСТЕР\,--\,ДИНАМИКА>> 
\end{tabular}&\tabcolsep=0pt\begin{tabular}{l} Нано-\\ уровень\end{tabular}\\
\hline
3&\tabcolsep=0pt\begin{tabular}{l} 
$\mathbf{MC}_3^1$ <<Модель генерации/рекомбинации ионов кислорода\\ 
\hspace*{5mm}и~кислородных вакансий (GR-model)>>\\
$\mathbf{MC}_3^2$ <<Модель дрейфа-диффузии ионов кислорода (DD-model)>>\\
$\mathbf{MC}_3^3$ <<Модель силового поля (E-model)>>\\
$\mathbf{MC}_3^4$ <<Модель переноса электронов (электрического тока) 
(J-model)>>\\
$\mathbf{MC}_3^5$ <<Модель теплопереноса (HT-model)>>\end{tabular}&
\tabcolsep=0pt\begin{tabular}{l} Уровень\\ элемента\\ 
резистивной\\ памяти\\ (мемристора) \end{tabular}\\
\hline
4&\tabcolsep=0pt\begin{tabular}{l} 
$\mathbf{MC}_4^1$ <<Фитинг моделей. ВАХ>>\\
$\mathbf{MC}_4^2$ <<Cхемотехническое представление>>\end{tabular}&
\tabcolsep=0pt\begin{tabular}{l} Уровень\\ 
формирования\\ нейроморфной\\ сети\end{tabular}\\
\hline
5&\tabcolsep=0pt\begin{tabular}{l}$\mathbf{MC}_5^1$ 
<<Нейросетевое моделирование>>. Обучение по прецедентам\end{tabular}&
\tabcolsep=0pt\begin{tabular}{l} Логический\\ уровень\end{tabular}\\
\hline
\end{tabular}
\end{center}
\end{table*}

    Вместе с~тем для решения задач интеллектуального анализа данных 
использование мемристорных элементов в~составе нейроморфных устройств 
сталкивается с~рядом проблем. Среди них~--- разброс значений параметров 
переключения (напряжение, окно памяти) от ячейки к~ячейке, асимметричность 
и~нелинейные эффекты, сложность применения градиентных алгоритмов 
обучения в~связи с~дискретным характером синаптических весов. 

Такие 
проблемы диктуют необходимость создания оригинальных имитационных 
моделей и~новых программных инструментов, которые позволят оценить влияние 
возмущающих факторов на предсказательную точность и~процесс обучения сети. 
Кроме того, такие модели обеспечат формирование информации для 
обоснованного выбора материалов, конфигурации и~топологии ячеек памяти, 
селекторов и~электродов, выработки оптимальных схемных и~алгоритмических 
решений. 

Сложность создания моделей данного класса связана, в~первую 
очередь, с~необходимостью учета широкого спектра 
про\-стран\-ст\-вен\-но-вре\-мен\-н$\acute{\mbox{ы}}$х масштабов. В~случаях, когда необходимо в~рамках одной модели 
провести исследование многомасштабного физического процесса или явления, 
возникает проб\-ле\-ма взаимосогласованности име\-ющих\-ся моделей, что требует 
разработки теоретических основ их объединения. 

В~настоящей работе для 
решения задачи многомасштабного моделирования нейроморфных сис\-тем 
применяется оригинальная информационная технология построения 
многомасштабных моделей, основанная на теоретико-множественном 
пред\-став\-ле\-нии фи\-зи\-ко-ма\-те\-ма\-ти\-че\-ских моделей и~использовании 
информационных структур, объ\-еди\-ня\-ющих данные и~методы их обработки~[1, 
2]. Для ее практической реализации была разработана интеграционная платформа 
для многомасштабного моделирования, объединяющая информационные потоки 
на разных масштабных уровнях: на уровне элементов резистивной памяти, на 
уровне\linebreak нейроморфной сети, на уровне имитации обучения нейроморфной сети по 
прецедентам (логический уровень). Для более детального и~последовательного 
построения схемы многомасштабного\linebreak моделирования можно выделить шесть 
характерных масштабных уровней: уровень элементов таб\-ли\-цы Менделеева; 
кван\-то\-во-ме\-ха\-ни\-че\-ский уровень и~наноуровень (их объединение ранее 
условно называли уровнем атомарной структуры); уровень элемента памяти 
(мемристора); уровень нейроморфной сети; логический уровень (см.\ таблицу).
 
    В данной работе представлена оригинальная интеграционная платформа для 
многомасштабного моделирования нейроморфных систем, размещенная на 
гибридном кластере ЦКП ФИЦ ИУ РАН, которая позволяет проводить 
эффективное параллельное решение задач разных масштабов с~учетом 
постоянного обмена данными. Далее приведено описание используемых 
теоретических моделей и~сценариев работы комплекса, на\-прав\-лен\-ных на 
решение существенных проблем, которые возникают при проектировании 
элементов мем\-ристора на основе магнитных туннельных пе\-реходов (МТП) при 
последовательной миниатюризации этих устройств. Это придает задаче 
\mbox{междисциплинарный} характер и~обусловливает актуальность применения 
информационной технологии многомасштабного моделирования в~контексте 
поставленной задачи

\section{Принципы формирования информационной поддержки 
многомасштабного моделирования физических явлений 
и~процессов}

    В данной работе при построении многомасштабной модели для имитации 
работы нейроморфных систем используются подходы, разработанные ранее 
и~представленные в~[1, 2]. В~их основе лежит идея о~том, что спецификация 
информационных объектов предметной области может быть сформулирована 
в~терминах 
 ко\-неч\-но-мно\-жест\-вен\-ных представлений~[3] которые, в~част\-ности, могут 
быть реализованы посредством доменных моделей~[4, 5]~--- конечных наборов 
(множеств) данных с~поддержкой операций по их обработке. 
В~\mbox{статье}~\cite{2-ab} даны понятия и~определения, применяемые при описании информационной 
технологии многомасштабного моделирования. Вводится понятие <<базовая 
мод\-ель-ком\-по\-зи\-ция>> (БК), для ее описания применяется 
тео\-ре\-ти\-ко-мно\-жест\-вен\-ный аппарат. Используется обозначение базовой  
мо\-де\-ли-ком\-по\-зи\-ции: $\mathbf{MC}_i^j$, где~$i$~--- номер 
масштабного уровня; $j$~--- номер данной БК на этом масштабном уровне~[1, 
2]. 
    
    Как было показано в~работах~[1, 2], БК служат информационными 
аналогами базовых  
фи\-зи\-ко-ма\-те\-ма\-ти\-че\-ских моделей, которые применяются 
в~вычислительном процессе для решения \mbox{конкретных} задач на своем масштабном 
уровне. Их можно представить в~виде таблиц данных разного структурного типа 
(вход\-ных/вы\-ход\-ных данных,\linebreak моделей и~алгоритмов и~др.). Экземпляры БК 
(таб\-ли\-цы с~конкретными данными) представляются объектами классов~--- 
наследников БК  
в~объект\-но-ори\-ен\-ти\-ро\-ван\-ном языке программирования. Они хранятся 
в~виде документов 
    в~до\-ку\-мент\-но-ори\-ен\-ти\-ро\-ван\-ной базе данных~\cite{6-ab}. 
Многомасштабная композиция представляет собой коллекцию, состоящую из 
сгруппированных документов с~иерархической структурой, отражающей 
последовательность передачи данных в~общем вычислительном процессе.
    
    Покажем, как при решении задачи о~применении методов 
многомасштабного моделирования для имитации работы нейроморфной сети 
с~по\-мощью данной технологии из конкретных БК\linebreak 
 со\-став\-ля\-ют\-ся многомасштабные композиции (МК)~--- информационные 
аналоги многомасштабных моделей, передающие содержание многомасштабных 
вычислительных процессов. 

\section{Интеграционная платформа для~моделирования 
работы нейроморфной сети}
 
    Интеграционная платформа для мно\-го\-мас\-штабного моделирования 
нейроморфной сети объединяет информационные потоки на разных мас\-штаб\-ных 
уровнях~--- на кван\-то\-во-ме\-ха\-ни\-че\-ском и~наноуровнях, на уровне 
элементов резистивной памяти, на уровне нейроморфной сети, на уровне 
имитации обучения нейроморфной сети по прецедентам. Для более детального 
описания моделируемого вы\-чис\-ли\-тель\-но\-го уровня введем дополнительные 
уровни.

    Общее представление о~распределении БК 
по масштабным уровням, задействованным в~вы\-чис\-ли\-тель\-ном процессе, 
показано в~таблице. 

   Покажем, как организуется вычислительным процесс, имитирующий работу 
нейроморфной сети, в~основе которой лежит мемристивный элемент на базе 
оксида гафния.
   
    На нулевом масштабном уровне с~БК $\mathbf{MC}_0^1$ <<АТОМ $A_0^i$>> ($i$~--- номер элемента 
в~таблице Менделеева) задаются основные данные по конкретным химическим 
элементам, входящим в~состав соединений, участвующих в~вычислительном 
процессе (атомный номер химического элемента, масса атома, заряд ядра, радиус 
атома, электронная конфигурация, структура решетки и~др.). В~данном случае 
формируются экземпляры БК 
$\mathbf{MC}_0^{72}$ <<АТОМ $\mathrm{Hf}_0^{72}$>> и~$\mathbf{MC}_0^{16}$ 
<<АТОМ $\mathrm{O}_0^{16}$>>.

    К первому масштабному уровню отнесены БК
$\mathbf{MC}_1^1$ <<КРИСТАЛЛОХИМИЧЕСКАЯ ФОРМУЛА>> 
и~$\mathbf{MC}_1^2$ <<КВАН\-ТО\-ВО-МЕ\-ХА\-НИ\-ЧЕ\-СКАЯ ЯЧЕЙКА>>. C~их по\-мощью 
данные, полученные с~нулевого масштабного уровня из 
$\mathbf{MC}_0^{72}$ <<АТОМ $\mathrm{Hf}_0^{72}$>> и~$\mathbf{MC}_0^{16}$ 
<<АТОМ $\mathrm{O}_0^{16}$>>, передаются на кван\-то\-во-ме\-ха\-ни\-че\-ский уровень 
первоначально в~$\mathbf{MC}_1^1$ <<КРИСТАЛЛОХИМИЧЕСКАЯ ФОРМУЛА>>, 
где, используя знания о~химическом составе и~кристаллографической структуре, 
с~помощью простых фи\-зи\-ко-ма\-те\-ма\-ти\-че\-ских моделей 
(ион\-но-атом\-ных радиусов и~др.)\ определяется кристаллохимическая структура соединения 
(метрические параметры крис\-тал\-ли\-че\-ской решетки, координаты базисных атомов 
и~др.). Данная БК программно реализована в~двух расчетных модулях (модель 
<<Плотная упаковка>> и~программный комплекс Materials Studio ({\sf 
https://www.3dsbiovia.com/products/collaborative-science/biovia-materials-studio})).
     
     Далее полученные в~ходе вычислительного процесса данные передаются 
в~БК $\mathbf{MC}_1^2$ <<КВАН\-ТО\-ВО-МЕ\-ХА\-НИ\-ЧЕ\-СКАЯ 
ЯЧЕЙКА>>. Данная БК программно реализована в~двух 
расчетных модулях (про\-грам\-мный комплекс VASP ({\sf https://www.\linebreak vasp.at}) 
и~пакет программ с~открытым кодом Quantum Espresso 
({\sf https://www.quantum-espresso.org})). Здесь на базе теории функционала электронной плотности решается 
задача по определению итоговых значений параметров кристаллической решетки, 
рассчитывается электронная плотность~[7], полная энергия для заданной 
конфигурации базисных атомов. Далее проводятся расчеты констант упругости, 
поляризация, фононное рассеяние, рассчитываются энергетические барьеры, 
энергия активации и~др.
    
На втором масштабном уровне, используя данные, полученные на предыдущем 
уровне, проводится выбор потенциалов межатомного взаимодействия с~учетом 
типа связи моделируемого \mbox{соединения} (в~данном случае~--- оксида гафния). 
Например, используется модель погруженного атома (embedded atom model, 
EAM)~[8] либо потенциал межатомного взаимодействия MEAM~[9]. При 
помощи БК~$\mathbf{MC}_2^1$ <<АТОМНЫЙ КЛАСТЕР\,--\,СТА\-ТИ\-КА>> 
проводится идентификация параметров потенциала межатомного 
взаимодействия. Далее потенциал с~идентифицированными параметрами 
используется в~ходе мо\-ле\-ку\-ляр\-но-ди\-на\-ми\-че\-ско\-го моделирования 
в~рамках БК~$\mathbf{MC}_2^2$ <<АТОМНЫЙ КЛАС\-ТЕР\,--\,ДИ\-НА\-МИ\-КА>>. 

Ранее полученные данные о~координатах позиций атомов 
кислорода, рассчитанные данные об энергии активации, энергии рекомбинации, 
энергетических барьерах миграции и~другие поступают на третий масштабный 
уровень. Туда же поступают дополнительные данные, такие как значения 
потенциалов на электродах, коэффициенты, учитывающие вклад поляризации 
света в~энергетические барьеры, и~др.\ (см.\ рисунок) 
    
На третьем масштабном уровне сформирована математическая модель 
обра\-зо\-ва\-ния/раз\-ру\-ше\-ния\linebreak
 проводящих каналов (филаментов) в~мемристорных 
элементах на основе оксидных пленок. Ей\linebreak соответствует <<Композитный модуль 
расчета мемристора>>. Необходимо отметить, что в~мире весьма активно ведутся 
работы по исследованию возможностей проектирования и~оптимизации ячеек 
памяти на основе оксидов металлов. Так, в~\mbox{статье}~\cite{10-ab} пред\-став\-лен 
электротермический симулятор, позволяющий исследовать физику и~потенциал 
ячеек резистивной памяти на основе SiO$_x$ с~произвольным доступом (RRAM,
resistive random-access memory). 
В~данной же работе в~контексте многомасштабного подхода используется 
модельное пред\-став\-ле\-ние, вклю\-ча\-ющее в~себя~5~взаимосвязанных 
подмоделей, каждой из которых по\-став\-ле\-ны в~соответствие БК: 
$\mathbf{MC}_3^1$ <<Модель  
ге\-не\-ра\-ции/ре\-ком\-би\-на\-ции ионов кислорода и~кис\-ло\-род\-ных вакансий  
(GR-model)>>; $\mathbf{MC}_3^2$ <<Модель дрей\-фа-диф\-фу\-зии ионов 
кислорода  
(DD-model)>>; $\mathbf{MC}_3^3$ <<Модель силового поля (E-model)>>; 
$\mathbf{MC}_3^4$ <<Модель переноса электронов (электрического тока) (J-model)>>;
$\mathbf{MC}_3^5$ <<Модель теплопереноса  
(HT-model)>>. Из~5~БК формируется композиция для расчета свойств 
мемристора $K_3^{3,1; 3,2; 3,3; 3,4; 3,5}$. Здесь\linebreak нижний индекс $i\hm=3$, так как 
все БК, об\-ра\-зующие композицию,~--- с~треть\-его уровня;\linebreak верхние индексы 
(3,1;3,2;3,3;3,4;3,5) через точку с~запятой обозначают, какие именно БК 
ис\-поль\-зу\-ются.
 
     
     На четвертом масштабном уровне многомасштабной модели для имитации работы 
нейроморфной сети представлена БК
$\mathbf{MC}_4^1$ <<Фитинг моделей. ВАХ>> 
и~БК~$\mathbf{MC}_4^2$ <<Схемотехническая>>. На 
вход~$\mathbf{MC}_4^1$ подаются сведения (графики) по вольтамперным 
характеристикам (ВАХ) моделируемого соединения с~нижнего масштабного 
уровня либо сведения о~модели и~начальном наборе ее па\-ра\-мет\-ров. На выходе 
получаем уточненные значения па\-ра\-мет\-ров модели. Далее полученные данные 
передаются в~БК~$\mathbf{MC}_4^2$ <<Схемотехническая>>, туда же 
по\-сту\-па\-ют данные по па\-ра\-мет\-рам электронной схемы.
     
     На пятом масштабном уровне проводится моделирование работы аналоговой 
нейроморфной сети~$\mathbf{MC}_5^1$ <<Нейросетевое моделирование>>. 
Данные, рассчитанные на нижних масштабных уровнях (электронная схема, 
уточненные ВАХ, уточненные названия параметров модели и~др.), а~также 
параметры нейроморфной сети и~наборы паттернов распознавания подаются на 
вход~$\mathbf{MC}_5^1$ <<Нейросетевое моделирование>>. С~помощью 
данной модели проводится имитация обучения нейроморфной сети. Более 
подробно работа модели этого масштабного уровня описана в~работе~[11].
    
    На рисунке представлена архитектурная схема расчетных модулей 
и~основных потоков данных. В~общем случае модуль представлен программным 
компонентом, состоящим:
    \begin{itemize}
\item из программы, непосредственно выполняющей расчет в~пакетном 
режиме; 
\item микросервиса~[12], обеспечивающего связь с~расчетной программой 
и~предоставляющего программный интерфейс (API, application programming interface) для интеграции 
в~общий сценарий расчета~[13];
\item интерфейса пользователя для ввода параметров и~мониторинга расчета;
\item локальной базы данных расчетного модуля для хранения внутренних 
данных и~промежуточных результатов.
\end{itemize}

\begin{figure*} %fig1
\vspace*{1pt}
 \begin{center}
 \mbox{%
 \epsfxsize=156.22mm 
\epsfbox{abg-1.eps}
 }
\vspace*{12pt}

{\small Интеграционная платформа для многомасштабного моделирования нейроморфных 
систем}
 \end{center}
\end{figure*}

      На следующем этапе с~помощью созданной интеграционной платформы 
    планируется проведение серии вычислительных экспериментов. Будут 
    рассмотрены различные модели многоуровневых устройств памяти с~
    мемристивными системами на основе оксидов металлов и~широкозонных 
    полупроводников, таких как MgO, VO$_2$, TiO$_2$, SiO$_2$, ZrO$_2$, HfO$_2$, 
TaO$_x$ и~др. 
      
    В связи с~тем, что разработанная интеграционная платформа позволяет 
оценить влияние возмущающих факторов на предсказательную точность 
и~процесс обучения нейроморфной сети, накопление большого числа данных 
вычислительных экспериментов в~дальнейшем может обеспечить формирование 
информации для обоснованного выбора материалов, конфигурации и~топологии 
ячеек памяти компьютеров нового поколения.
  
{\small\frenchspacing
 {%\baselineskip=10.8pt
 \addcontentsline{toc}{section}{References}
 \begin{thebibliography}{99}
  
 \bibitem{1-ab}
 \Au{Абгарян К.\,К.} Многомасштабное моделирование в~задачах структурного 
материаловедения.~--- М.: МАКС Пресс, 2017. 284~с.
 \bibitem{2-ab}
 \Au{Абгарян К.\,К.} Информационная технология по\-стро\-ения многомасштабных моделей 
в~задачах вычислительного материаловедения~// Системы высокой доступности, 
2018. Т.~14.  №\,2. С.~9--15.
 \bibitem{3-ab}
 \Au{Бродский Ю.\,И.} Модельный синтез и~модельно-ори\-ен\-ти\-ро\-ван\-ное 
 программирование.~--- М.: ВЦ РАН, 2013. 142~с.
 \bibitem{4-ab}
 \Au{Evans E.} Domain-driven design: Tackling complexity in the heart of software.~--- 
 Addison-Wesley Professional, 2003. 560~p.
 \bibitem{5-ab}
 \Au{Абгарян~К.\,К., Гаврилов~Е.\,С., Марасанов~А.\,М.} Информационная поддержка задач 
многомасштабного моделирования композиционных материалов~// Int.~J. Open Information 
Technologies, 2017. №\,12. C.~24--29.
 \bibitem{6-ab}
 \Au{Fowler M., Sadalage~P.\,J.} NoSQL distilled: 
 A~brief guide to the emerging world of polyglot 
persistence.~--- Addison-Wesley Professional, 2012. 190~p.
 \bibitem{7-ab}
 \Au{Kohn W., Sham L.\,J.} Self-consistent equations including exchange and correlation effects~// 
Phys. Rev.~A, 1965. Vol.~140. P.~1133--1138.     
 \bibitem{8-ab}
 \Au{Gao~D., Deng~H., Heinisch~H., Kurtz~R.} A~new Fe--He interatomic potential based on 
\textit{ab initio} calculations in $\alpha$-Fe~// J. Nucl. Mater., 2011. Vol.~418. Iss.~1. P.~115--120. 
 \bibitem{9-ab}
 \Au{Lee B.-J., Baskes M.\,I., Kim~H., Cho~Ya.\,K.} Second nearest-neighbor modified embedded 
atom method potentials for bcc transition metals~// Phys. Rev.~B, 2001. Vol.~64. Art. ID: 184102. 
doi: 10.1103/PhysRevB.64.184102.
  \bibitem{10-ab}
 \Au{Sadi T., Mehonic~A., Montesi~L., Buckwell~M., Kenyon~A., Asenov~A.} Investigation of 
resistance switching in SiO$_x$ RRAM cells using a~3D multi-scale kinetic Monte Carlo simulator~// 
J.~Phys. Condens. Matter, 2018. Vol.~30. Iss.~8. Art. ID: 084005. doi: 
10.1088/1361-648X/aaa7C1.
 \bibitem{11-ab}
 \Au{Морозов А.\,Ю., Ревизников~Д.\,Л., Абгарян~К.\,К.} Вопросы реализации нейросетевых 
алгоритмов на мемристорных кроссбарах~// Известия высших учебных заведений. Материалы 
электронной техники, 2019. Т.~22. №\,4. %doi: 10.17073/1609-3577-2019-4-. @@@
 \bibitem{12-ab}
 \Au{Newman S.} Building microservices.~--- Sebastopol, CA, USA: O'Reilly Media, 2015. 282~p.
 \bibitem{13-ab}
 \Au{Абгарян К.\,К., Гаврилов~Е.\,С.} Информационная поддержка интеграционной платформы 
многомасштабного моделирования~// Системы и~средства информатики, 2019. Т.~29. №\,1. 
С.~53--62.
\end{thebibliography}

 }
 }

\end{multicols}

\vspace*{-6pt}

\hfill{\small\textit{Поступила в~редакцию 15.04.20}}

\vspace*{8pt}

%\pagebreak

%\newpage

%\vspace*{-28pt}

\hrule

\vspace*{2pt}

\hrule

%\vspace*{-2pt}

\def\tit{INTEGRATION PLATFORM FOR MULTISCALE MODELING OF~NEUROMORPHIC SYSTEMS}


\def\titkol{Integration platform for multiscale modeling of neuromorphic systems}

\def\aut{K.\,K.~Abgaryan$^{1,2}$ and E.\,S.~Gavrilov$^{1,2}$}

\def\autkol{K.\,K.~Abgaryan and E.\,S.~Gavrilov}

\titel{\tit}{\aut}{\autkol}{\titkol}

\vspace*{-9pt}


\noindent
$^1$Dorodnicyn Computing Center, Federal Research Center 
``Computer Science and Control'' of the Russian\linebreak
$\hphantom{^1}$Academy of Sciences, 40~Vavilov Str., Moscow 119333, Russian Federation

\noindent
$^2$Moscow Aviation Institute (National Research University),
 4~Volokolamskoe Shosse, Moscow 125080, Russian\linebreak 
$\hphantom{^1}$Federation

\def\leftfootline{\small{\textbf{\thepage}
\hfill INFORMATIKA I EE PRIMENENIYA~--- INFORMATICS AND
APPLICATIONS\ \ \ 2020\ \ \ volume~14\ \ \ issue\ 2}
}%
 \def\rightfootline{\small{INFORMATIKA I EE PRIMENENIYA~---
INFORMATICS AND APPLICATIONS\ \ \ 2020\ \ \ volume~14\ \ \ issue\ 2
\hfill \textbf{\thepage}}}

\vspace*{3pt} 

\Abste{ The current multilevel resistive memory elements allow increasing the integration density of 
nonvolatile memory as well as designing and creating systems with a~parallel computing mechanism. Such devices 
are based on memristor elements necessary for developing the foundations of analog neuromorphic networks 
that are used to solve data mining problems. However, the use of memristors as a~part of neuromorphic devices 
encounters a~number of problems such as the scatter of the switching parameters 
(voltage and memory window) 
from cell to cell, asymmetry and nonlinear effects, and others. Such problems dictate the need to create original 
simulation models and new software tools that will allow one to evaluate the influence of disturbing factors on 
the predictive accuracy and network learning process. In this paper, to solve the problem of multiscale 
modeling of neuromorphic systems, the authors use the original information technology for constructing 
multiscale models. For its practical implementation, an integration platform has been built that allows one to evaluate 
the influence of disturbing factors on the predictive accuracy and learning process of a~neuromorphic 
network 
and in the future, it will be able to provide information for 
a~reasonable choice of materials, configuration, and 
topology of memory cells of new-generation computers.}

\KWE{multi-scale modeling; multilevel memory elements; neuromorphic networks; predictive modeling; 
memristor; integration platform; software package}

\DOI{10.14357/19922264200215} 

%\vspace*{-20pt}

\Ack
\noindent
The work was supported by the Russian Foundation for Basic Research (project 19-29-03051~mk).

%\vspace*{6pt}

 \begin{multicols}{2}

\renewcommand{\bibname}{\protect\rmfamily References}
%\renewcommand{\bibname}{\large\protect\rm References}

{\small\frenchspacing
 {%\baselineskip=10.8pt
 \addcontentsline{toc}{section}{References}
 \begin{thebibliography}{99}

 \bibitem{1-ab-1}
\Aue{Abgaryan, K.\,K.} 2017. \textit{Mnogomasshtabnoe modelirovanie v~zadachakh strukturnogo 
materialovedeniya} [Multiscale modeling in material science problems]. Moscow: MAKS Press. 284~p.
 \bibitem{2-ab-1}
\Aue{Abgaryan, K.\,K.} 2018. Informatsionnaya tekhnologiya postroyeniya mnogomasshtabnykh modeley 
v~zadachakh vychislitel'nogo materialovedeniya 
[Information technology is the construction of multi-scale 
models in problems of computational materials science]. \textit{Sistemy Vysokoy Dostupnosti} [High 
Availability Systems] 14(2):9--15.
 \bibitem{3-ab-1}
\Aue{Brodskiy, Yu.\,I.} 2013. \textit{Model'nyy sintez i~model'no-oriyentirovannoe programmirovanie} 
[Model synthesis and model oriented programming]. Moscow: CC RAS. 142~p.
 \bibitem{4-ab-1}
\Aue{Evans, E.} 2003. \textit{Domain-driven design: Tackling complexity in the heart of software}. 
Addison-Wesley Professional. 560~p.
 \bibitem{5-ab-1}
\Aue{Abgaryan, K.\,K., E.\,S. Gavrilov, and A.\,M.~Marasanov.} 2017. Informatsionnaya podderzhka zadach 
mno\-go\-mas\-shtab\-no\-go modelirovaniya kompozitsionnykh materialov [Multiscale modeling for composite 
materials computer simulation support]. \textit{Int.~J. Open Information Technologies} 12:24--29.
 \bibitem{6-ab-1}
\Aue{Fowler, M., and P.\,J. Sadalage.} 2012. \textit{NoSQL distilled: A~brief guide to the emerging world of 
polyglot persistence}. Addison-Wesley Professional. 190~p.
 \bibitem{7-ab-1}
\Aue{Kohn, W., and L.\,J. Sham.} 1965. Self-consistent equations including exchange and correlation effects. 
\textit{Phys. Rev.~A} 140:1133--1138.
 \bibitem{8-ab-1}
\Aue{Gao, D., H. Deng, H.~Heinisch, and R.~Kurtz.} 2011. A new Fe--He interatomic potential based on 
\textit{ab initio} calculations in $\alpha$-Fe. \textit{J.~Nucl. Mater.} 418(1):115--120.
 \bibitem{9-ab-1}
\Aue{Lee, B.-J., M.\,I. Baskes, H.~Kim, and Ya.\,K.~Cho.} 2001. Second nearest-neighbor modified embedded 
atom method potentials for bcc transition metals. \textit{Phys. Rev.~B} 64:184102. doi: 
10.1103/PhysRevB.64.184102.
 \bibitem{10-ab-1}
\Aue{Sadi, T., A. Mehonic, L.~Montesi, M.~Buckwell, A.~Kenyon, and A.~Asenov.} 2018. Investigation of 
resistance switching in SiOx RRAM cells using a~3D multi-scale kinetic Monte Carlo simulator. 
\textit{J.~Phys. Condens. Matter} 30(8):084005. doi: 10.1088/1361-648X/aaa7C1.
 \bibitem{11-ab-1}
\Aue{Morozov, A.\,Yu., D.\,L. Reviznikov, and K.\,K.~Abgaryan.} 2019. Voprosy realizatsii neyrosetevykh 
algoritmov na memristornykh krossbarakh [Implementation of neural network algorithms using memristor 
crossbars]. \textit{Izvestiya Vysshikh Uchebnykh Zavedeniy. Materialy Elektronnoy Tekhniki } [Proceedings of 
Higher Educational Institutions. High Materials of Electronics Engineering] 
22(4). %Available at: {\sf 
%https://met.misis.ru/jour/article/view/346} (accessed April~15, 2020). 
 \bibitem{12-ab-1}
\Aue{Newman, S.} 2015. \textit{Building microservices}. Sebastopol, CA: O'Reilly Media. 282~p.
 \bibitem{13-ab-1}
\Aue{Abgaryan, K.\,K., and E.\,S.~Gavrilov.} 2019. In\-for\-ma\-tsi\-on\-naya podderzhka integratsionnoy platformy 
mnogomasshtabnogo modelirovaniya [Informational support of the multiscale modeling integration platform]. 
\textit{Sistemy i~Sredstva Informatiki~--- Systems and Means of Informatics} 29(1):53--62.
\end{thebibliography}

 }
 }

\end{multicols}

\vspace*{-6pt}

\hfill{\small\textit{Received April 15, 2020}}

%\pagebreak

%\vspace*{-24pt}



\Contr

\noindent
\textbf{Abgaryan Karine K.} (b.\ 1963)~--- Doctor of Science in physics and mathematics, principal scientist, 
Dorodnicyn Computing Center, Federal Research Center ``Computer Science and Control'' of the Russian 
Academy of Sciences, 40~Vavilov Str., Moscow 119333, Russian Federation; Head of Department, Moscow 
Aviation Institute (National Research University), 4~Volokolamskoe Shosse, Moscow 125080, Russian 
Federation; \mbox{kristal83@mail.ru}

\vspace*{3pt}

\noindent
\textbf{Gavrilov Evgeny S.} (b.\ 1982)~--- scientist, Dorodnicyn Computing Center, Federal Research Center 
``Computer Science and Control'' of the Russian Academy of Sciences, 40~Vavilov Str., Moscow 119333, 
Russian Federation; senior lecturer, Moscow Aviation Institute (National Research University), 
4~Volokolamskoe Shosse, Moscow 125080, Russian Federation; \mbox{eugavrilov@gmail.com}
\label{end\stat}

\renewcommand{\bibname}{\protect\rm Литература} 