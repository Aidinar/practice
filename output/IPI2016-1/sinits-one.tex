\def\ss2{\mathop {\sum\limits^{n^\Theta}\sum\limits^{n^\Theta}}}


\def\stat{sin-one}

\def\tit{ОРТОГОНАЛЬНЫЕ СУБОПТИМАЛЬНЫЕ ФИЛЬТРЫ
ДЛЯ~НЕЛИНЕЙНЫХ СТОХАСТИЧЕСКИХ
СИСТЕМ НА~МНОГООБРАЗИЯХ$^*$}

\def\titkol{Ортогональные субоптимальные фильтры
для нелинейных стохастических
систем на многообразиях}

\def\aut{И.\,Н.~Синицын$^1$}

\def\autkol{И.\,Н.~Синицын}

\titel{\tit}{\aut}{\autkol}{\titkol}

{\renewcommand{\thefootnote}{\fnsymbol{footnote}} \footnotetext[1]
{Работа выполнена при поддержке РФФИ (проект 15-07-02244).}}


\renewcommand{\thefootnote}{\arabic{footnote}}
\footnotetext[1]{Институт проблем информатики Федерального исследовательского
центра <<Информатика и~управление>> Российской академии наук,
sinitsin@dol.ru}

\vspace*{6pt}

\Abst{Для нелинейных дифференциальных стохастических систем 
на гладких многообразиях с~винеровскими и~пуассоновскими шумами в~уравнениях состояния 
и~винеровскими шумами в~наблюдениях разработана теория синтеза ортогональных субоптимальных 
фильтров (ОСОФ)
по среднеквадратическому критерию. Получены точные фильтрационные уравнения для  
стохастических систем на многообразиях (МСтС). Обсуждаются вопросы упрощения точных 
фильтрационных уравнений. Приводятся уравнения субоптимальных фильтров (СОФ) на основе 
методов нормальной аппроксимации (МНА) и~статистической линеаризации (МСЛ). Для решения задач 
в~реальном времени использование нормальных СОФ (НСОФ) не обеспечивает 
необходимой точности, поэтому в~основу синтеза положены методы ортогональных разложений (МОР)
и~квазимоментов (МКМ) для апостериорной одномерной плотности. Получены уравнения точности 
и~чувствительности алгоритмов. В~качестве тестового примера рассмотрена одномерная 
нелинейная стохастическая система с~аддитивным и~мультипликативным белым шумом. 
Рассмотрены некоторые обобщения разработанных алгоритмов.}

\KW{апостериорное одномерное распределение; винеровский шум;
квазимомент (КМ); коэффициент ортогонального разложения (КОР);
метод квазимоментов (МКМ); метод ортогональных разложений (МОР);
нормальный фильтр; ортогональный СОФ (ОСОФ); первая функция чувствительности;
пуассоновский шум; стохастическая система на многообразиях (МСтС);
субоптимальный фильтр (СОФ)}

\DOI{10.14357/19922264160103} %


\vspace*{12pt}

\vskip 14pt plus 9pt minus 6pt

\thispagestyle{headings}

\begin{multicols}{2}

\label{st\stat}

\section{Введение}

В~[1, 2] МОР и~МКМ развиты для аналитического моделирования одно- и~многомерных
распределений в~стохастических системах  на гладких многообразиях
и~дано их применение для задач надежности и~безопасности
технических систем.

Рассмотрим развитие теории СОФ на базе МОР 
и~МКМ для МСтС. 

Раздел~2 посвящен точным  фильтрационным уравнениям для МСтС, 
а~также СОФ на основе МНА и~МСЛ. 

В~разд.~3 рассматриваются ОСОФ на основе МОР 
и~МКМ аппроксимаций нормированной  апостериорной одномерной плотности. 

В~разд.~4 обсуждаются вопросы точности и~чувствительности ОСОФ. 

В~разд.~5 
приводится тестовый пример. 

Заключение содержит выводы и~некоторые обобщения.

\section{Точные фильтрационные уравнения. Нормальный субоптимальный фильтр}

\subsection{Уравнения процессов. Вспомогательные формулы}

Пусть векторный стохастический процесс (СтП) 
$\lk X_t^{\mathrm{T}} Y_t^{\mathrm{T}} \rk^{\mathrm{T}}$
определяется системой векторных стохастических дифференциальных
уравнений Ито:
    \begin{align}
   \hspace*{-1mm}dX_t &=\varphi (X_t,Y_t,\Theta, t)\, dt + \psi' (X_t,Y_t,\Theta, t)\, dW_0 +{}\notag\\
    &\hspace*{-9.5mm}{}
+\iii_{R_0^q} \psi'' (X_t,Y_t,\Theta, t,v) P^0 \,(dt, dv)\,,\enskip 
X(t_0) = X_0\,;\label{e2.1-s1}\\
\hspace*{-1mm}dY_t &=\varphi_1 (X_t,Y_t,\Theta, t) \,dt +
    \psi_1' (X_t,Y_t,\Theta, t)\, dW_0 + {}\notag\\
&\hspace*{-9.5mm}{}+\int\limits_{R_0^q} \psi_1'' (X_t,Y_t,\Theta, t,v) P^0
    (dt,dv)\,,\enskip Y(t_0) = Y_0\,.\label{e2.2-s1}
    \end{align}
Здесь $Y_t=Y(t)$~--- $n_y$-мер\-ный наблюдаемый
СтП, $Y_t \hm\in \Delta^y$ ($\Delta^y$~--- гладкое многообразие наблюдений); 
$X_t \hm=X(t)$~--- $n_x$-мер\-ный ненаблюдаемый
СтП (вектор состояния), $X_t \hm\in \Delta^x$ ($\Delta^x$~--- 
гладкое многообразие состояний); $W_0 \hm=W_0(t)$~--- $n_w$-мер\-ный
винеровский СтП $(n_w\hm\ge n_y)$; $P^0(\Delta,A)\hm=P(\Delta,A)\hm-\mu_P (\Delta,A)$, 
$P(\Delta,A)$~--- представляет \mbox{собой} для любого множества~$A$ 
простой пуассоновский СтП, а~$\mu_P (\Delta,A)$~--- его математическое ожидание, причем

    \vspace*{2pt}

\noindent
    $$
    \mu_P (\Delta,A)=\mm P (\Delta,A)=\iii_\Delta \nu_P(\tau, A)\, d\tau\,;
    $$
    
        \vspace*{-2pt}
        
        \noindent
$\nu_P (\Delta, A)$~--- интенсивность соответствующего пуассоновского потока событий, 
$\Delta \hm=(t_1,t_2]$; интегрирование по~$v$ распространяется на все пространство~$R^q$ 
с~выколотым началом координат; $\Theta$~--- вектор случайных параметров 
размерности~$n_\Theta$;\linebreak
$\varphi\hm=\varphi(X_t,Y_t,\Theta, t)$, $\varphi_1\hm=\varphi_1(X_t,Y_t,\Theta, t)$,
 $\psi'\hm=\psi'(X_t,Y_t,\Theta, t)$ и~$\psi_1'\hm=\psi_1'(X_t,Y_t,\Theta, t)$~--- 
 известные функции, отображающие
$R^{n_x}\times R^{n_y}\times  R$ соответ\-ственно в~$R^{n_x}$,
$R^{n_y}$, $R^{n_xn_w}$ и~$R^{n_yn_w}$;
$\psi''\hm=\psi''(X_t,Y_t,\Theta, t,v)$ и~$\psi_1''\hm=\psi_1''(X_t,Y_t,\Theta,
t,v)$~--- известные функции, отображающие $R^{n_x}\times
R^{n_y}\times R^q$ в~$R^{n_x}$ и~$R^{n_y}$. Требуется найти оценку~$\hat X_t$ 
СтП~$X_t$ в~каж\-дый момент времени~$t$ по результатам
наблюдения СтП $Y(\tau)$ до момента~$t$, $Y_{t_0}^t \hm=
 \{ Y(\tau) \,: t_0 \hm\le \tau\hm< t\}$.

Предположим, что
\begin{itemize}
\item уравнение состояния имеет вид~(\ref{e2.1-s1});
\item уравнение наблюдения~(\ref{e2.2-s1}), во-пер\-вых, не содержит
пуассоновского шума $(\psi_1'' \hm\equiv 0)$,
во-вто\-рых, коэффициент при винеровском шуме~$\psi_1'$  
в~уравнениях наблюдения не зависит от состояния $(\psi_1' (X_t, Y_t,\Theta, t)\hm=\psi_1'
(Y_t,\Theta, t))$.
\end{itemize}

В этом случае уравнения задачи нелинейной фильтрации имеют следующий вид:

\noindent
\begin{align}
    dX_t &=\varphi (X_t, Y_t,\Theta, t)\,dt+\psi'(X_t, Y_t,\Theta, t)\,dW_0+{}\notag\\
&\hspace*{10mm}{}+\int\limits_{R_0^q} \psi''(X_t, Y_t,\Theta, t,v) P^0 (dt, dv)\,;
        \label{e2.3-s1}\\
    dY_t &=\varphi_1 (X_t, Y_t,\Theta, t)dt +\psi_1 (Y_t,\Theta, t)\, dW_0\,.
    \label{e2.4-s1}
    \end{align}


Будем считать, что выполнены условия существования и~единственности 
СтП  $\lk X_t^{\mathrm{T}}\ Y_t^{\mathrm{T}}\rk^{\mathrm{T}}$, 
определяемого~(\ref{e2.3-s1}) и~(\ref{e2.4-s1}) при соответствующих начальных 
условиях~[3--5].

В дальнейшем потребуется для стохастического уравнения

    \vspace*{2pt}

\noindent
    \begin{equation}
    dZ=a \,dt + b\, dW_0 + \iii_{R_0^q} c P^0 (dt, dv)
    \label{e2.5-s1}
    \end{equation}
    
    \vspace*{-2pt}
    
    \noindent
следующая обобщенная формула Ито~\cite{4-s1, 7-s1} для дифференциала  $U\hm=U(Z,t)$:

\columnbreak

\noindent
\begin{multline}
    dU =\lf U_t + U_z^{\mathrm{T}} a + \fr{1}{2}\,\mathrm{tr}\, 
    \lk U_{zz} b\nu b^{\mathrm{T}}\rk\rf dt +{}\\
{}+ \iii_{R_0^q} \lk U(Z+c,t)^{\mathrm{T}} - U(Z,t)^{\mathrm{T}} - U_z^{\mathrm{T}} c\rk 
\mu_P (dt, dv)+{}\\
\hspace*{-3mm}{}+U_Z^{\mathrm{T}} b\, dW_0 + \!\iii_{R_0^q}\! \lk U(Z+c,t) - U(Z,t) \rk P^0 (dt, dv).\!\!\!
  \label{e2.6-s1}
\end{multline}
Здесь $a$, $b$ и~$c$~--- известные функции~$Z$ и~$t$.

\subsection{Фильтрационные уравнения}

Как известно~\cite{5-s1, 7-s1, 6-s1}, для любых СтП $X_t$ и~$Y_t$ оптимальная оценка  
$\hat X^t$, минимизирующая средний квадрат ошибки в~каждый момент времени~$t$ 
представляет собой апостериорное математическое ожидание СтП  $X_t$: $\hat X_t \hm= 
\mm \lk X_t \mid Y_{t_0}^t\rk$. Чтобы найти это условное математическое ожидание, 
необходимо знать $p_t \hm= p_t (x)$~--- апостериорное одномерное распределение 
СтП~$X_t$.

В основе уравнений оптимальной (в~смыс\-ле минимума средней квадратической ошибки) 
фильт\-рации для уравнений~(\ref{e2.3-s1}) и~(\ref{e2.4-s1}) в~силу~(\ref{e2.6-s1}) 
лежит следу\-ющая формула для стохастического дифференциала апостериорного математического 
ожидания скалярной функции  $f\hm=f(X,t)$ вектора состояния:
\begin{multline}
d \hat f = \mm_{\Delta^x}^{p_t} \left[ 
f_t (X,t) + f_x (X,t)^{\mathrm{T}} \vrp (X,Y,t) + {}\right.\\
{}+\fr{1}{2}\,\mathrm{tr}\, 
\lf f_{xx} (X,t) \left(\psi' \nu {\psi'}^{\mathrm{T}}\right) (X,Y,t)\rf+{}\\
{}+ \iii_{R_0^q}  \left\{  \vphantom{f_x (X,t)^{\mathrm{T}}}
f \left(X+ \psi'' , t\right) - f(X,t) -{}\right.\\
\left.\left.{}- f_x (X,t)^{\mathrm{T}} 
\psi''(X,Y,t)\right\} \nu_P (t, dv)\mid Y_{t_0}^t \right]\, dt+{}\\
{}+\mm_{\Delta^x}^{p_t} \left\{ f(X,t) \left[ \vrp_1 (X,Y,t)^{\mathrm{T}} -
\hat \vrp_1^{\mathrm{T}}\right] +{}\right.\\
\hspace*{-2mm}\left.{}+ f_x (X,t)^{\mathrm{T}} \left(\psi\nu\psi_1^{\mathrm{T}}\right) 
(X,Y,t) \mid Y_{t_0}^t\right\} \times{}\\
    {}\times \left(\psi_1 \nu \psi_1^{\mathrm{T}}\right)^{-1} (Y,t) 
    \left(dY-\hat\vrp_1\, dt\right).
    \label{e2.7-s1}
    \end{multline}
Здесь для краткости аргумент~$\Theta$ опущен; $X\hm=X_t$, $ Y\hm=Y_t$; $\nu\hm=\nu_0$ 
и~$\nu_P $~--- интенсивности~$W_0$ и~$P^0$;
$\hat\vrp_1$~--- апостериорное математическое ожидание~$\vrp_1$ при заданной условной 
плотности  $p_t\hm=p_t (x, \Theta)$:
\begin{equation*}
\hat\vrp_1 = \mm_{\Delta^x}^{p_t} \lk\vrp_1(X,Y,t)\rk\,. %\label{e2.8-s1}
\end{equation*}

Полагая в~(\ref{e2.5-s1}) 
$$
f(X,t) \equiv g_t (\la,\Theta) =\mm_{\Delta^x}^{p_t}\exp (i\la^{\mathrm{T}} X)\,,
$$
 получим
точное нелинейное фильт\-ра\-ци\-он\-ное урав\-не\-ние для  характеристической функции 
$g_t (\la,\Theta)$:

\noindent
    \begin{multline}
    dg_t (\la,\Theta) = {\mm}_{\Delta^x}^{p_t} \left[ 
    \left\{ \vphantom{\fr{1}{2}}
     i\la^{\mathrm{T}} \varphi (X_t,Y_t,\Theta, t) -{}\right.\right.\\
\left. \!\!\!\!\!  {}- \fr{1}{2}\, 
    \la^{\mathrm{T}}\! \left(\psi\nu\psi^{\mathrm{T}}\right) (X_t,Y_t,\Theta, t) \la+
    \gamma (\la, X_t, Y_t, \Theta, t) \right\}\times{}\hspace*{-6.32312pt}
   \\
\hspace*{-7pt}\left.{}\times e^{i\la^{\mathrm{T}} X_t} \mid Y_{t_0}^t 
\vphantom{\fr{1}{2}}\right] dt+ 
{\mm}_{\Delta^x}^{p_t} %\left[ 
\left\{ 
\vphantom{e^{i\la^{\mathrm{T}} X_t} \mid Y_{t_0}^t}
\varphi_1 (X_t,Y_t,\Theta, t)^{\mathrm{T}} 
-\hat\varphi_1^{\mathrm{T}} +{}\right. %\hspace*{-1.6932pt}
\\
{}+
    i\la^{\mathrm{T}} (\psi\nu\psi_1^{\mathrm{T}}) (X_t,Y_t, \Theta,t) \times{}\\
\hspace*{-2.5mm}{}\left.{}\times e^{i\la^{\mathrm{T}} X_t} \mid Y_{t_0}^t \right\} \!
\left(\psi_1\nu\psi_1^{\mathrm{T}}\right)^{-1} \!(Y_t,\Theta,t) (dY_t-\hat\varphi_1
    \,dt) %\right]
    \!\!\! \!\label{e2.9-s1}
    \end{multline}
где

\noindent
\begin{multline}
    \hspace*{-4pt}\gamma=\gamma (\la, X_t, Y_t, \Theta, t)=
    \int\limits_{R_0^q}\big[ e^{i\la^{\mathrm{T}} \psi''(X_t,Y_t,\Theta, t,v)} -
    {}\\
 {}- 1- i\la^{\mathrm{T}} \psi''(X_t,Y_t,\Theta, t,v)\big] \nu_P (\Theta, t, dv)\,.
\label{e2.10-s1}
    \end{multline}

Функции $g_t (\la,\Theta)$ и~$p_t(x,\Theta)$ связаны между собой преобразованием Фурье.

Отсюда для гауссовской МСтС~(\ref{e2.3-s1}) и~(\ref{e2.4-s1}) $(\psi''\hm\equiv 0)$ уравнение~(\ref{e2.7-s1}) 
при  $\gamma\hm=0$ упрощается и~приобретает вид:

\noindent
\begin{multline}
dg_t (\la,\Theta) = {\mm}_{\Delta^x}^{p_t} \left[\left\{
\vphantom{\fr{1}{2}}
 i\la^{\mathrm{T}} 
\varphi (X_t,Y_t,\Theta, t) -{}\right.\right.\\
\left.\left.{}- \fr{1}{2}\, \la^{\mathrm{T}} 
\left(\psi\nu\psi^{\mathrm{T}}\right) (X_t,Y_t,\Theta, t) \la\right\} 
e^{i\la^{\mathrm{T}} X_t} \mid Y_{t_0}^t\right] dt+{}\\
{}+ {\mm}_{\Delta^x}^{p_t} %\left[ 
\left\{ 
\vphantom{\left(\psi_1\nu\psi_1^{\mathrm{T}}\right)^{-1}}
\varphi_1 (X_t,Y_t,\Theta, t)^{\mathrm{T}} -\hat\varphi_1^{\mathrm{T}} +{}%\right.
\right.\\
{}+
    i\la^{\mathrm{T}} \left(\psi\nu\psi_1^{\mathrm{T}}\right) (X_t,Y_t, \Theta,t) 
\left. e^{i\la^{\mathrm{T}} X_t} \mid Y_{t_0}^t \right\}\times{}\\
%\left.
{}\times 
\left(\psi_1\nu\psi_1^{\mathrm{T}}\right)^{-1} \!(Y_t,\Theta,t) 
\left(dY_t-\hat\varphi_1
    \,dt\right) %\right]
    \,,
    \label{e2.11-s1}
    \end{multline}

Таким образом, в~основу синтеза СОФ для МСтС~(\ref{e2.3-s1}) и~(\ref{e2.4-s1})
могут быть положены  следующие утверждения.

\vspace*{3pt}

\noindent
\textbf{Теорема~2.1.}\
\textit{Пусть для МСтС}~(\ref{e2.3-s1}) и~(\ref{e2.4-s1})
\textit{выполнены условия существования и~единственности решения, а матрица 
$\si_1 \hm=\psi_1\nu\psi_1^{\mathrm{T}}$ не вырождена. 
Тогда при условии ограниченности соответствующих условных математических 
ожиданий в}~(\ref{e2.10-s1}) \textit{точное фильтрационное уравнение для условной 
одномерной характеристической функции имеет вид}~(\ref{e2.9-s1}).

\smallskip

\noindent
\textbf{Теорема 2.2.}\ 
\textit{Пусть для гауссовской МСтС}~(\ref{e2.3-s1})  и~(\ref{e2.4-s1})
$(\psi''\hm\equiv 0)$ \textit{выполнены условия существования и~единственности решения, 
а~матрица $\si_1 \hm=\psi_1\nu\psi_1^{\mathrm{T}}$ не\linebreak вырождена. 
Тогда точное
%при условии ограниченности соответствующих  математических ожиданий 
фильт\-ра\-ционное уравнение для условной одномерной характеристической 
функции имеет вид}~(\ref{e2.11-s1}).

\vspace*{3pt}

\noindent
\textbf{Замечание~2.1.}\
Точное решение фильтрационных уравнений теорем~2.1 и~2.2  возможно
только в~случаях, когда уравнения гауссовской дифференциальной МСтС
линейны или линейны лишь относительно вектора состояния~$X_t$ при
независимой от состояния функции~$\psi$. Эти уравнения
 дают точ\-ное решение задачи оптимальной нелинейной фильтрации.  Однако это решение
  не может быть\linebreak реали\-зовано практически. Для
 нахождения оптимальной оценки вектора состояния необходимо\linebreak решить
 фильтрационное уравнение  для апостериорной характеристической функции
 (или  фильтрационное уравнение  для апостериорной плотности   вектора
 состояния~$X_t$) после получения результатов наблюдений, затем вычислить 
 оптимальную оценку вектора~$X_t$. Но методов точного решения этих
 уравнений  в~общем случае пока еще не существует.
 
 \vspace{3pt}
 
 \noindent
\textbf{Замечание~2.2.}\
 Численное решение фильтрационных уравнений в~задачах реального 
 времени (или он\-лайн-оце\-ни\-ва\-ния) тоже
 невозможно, так как для этого требуется много времени, а~решать их
 необходимо каждый раз после получения результатов наблю\-де\-ний.
 Кроме того, практическое применение точной теории оптимальной нелинейной фильтрации
 имеет смысл только в~тех случаях, когда оценки можно вычислять 
 в~реальном масштабе времени по мере получения результатов
 наблюдений. Точная теория дает оптимальные
 оценки в~каждый момент~$t$ по результатам наблюдений, полученным
 к~этому моменту, без использования последующих результатов
 наблюдений. Если эти оценки не могут быть вычислены в~тот же
 момент~$t$ или хотя бы с~фиксированным приемлемым запаздыванием
 и~их вычисление приходится откладывать на будущее, то нет
 никакого смысла отказываться от использования наблюдений,
 получаемых после момента~$t$, для оценивания состояния системы в~момент~$t$. 
 Поэтому для статистической обработки результатов
 после окончания наблюдений, т.\,е.\ для оф\-лайн-оце\-ни\-ва\-ния,
 целесообразно применять известные из математической статистики методы
 постобработки информации~\cite{5-s1}.
 
 \vspace*{-6pt}

\subsection{О~приближенных методах нелинейной фильтрации}

 Необходимость обработки результатов наблюдений в~реальном
 масштабе времени непосредственно в~процессе эксперимента
 привела  к~появлению ряда приближенных методов оптимальной нелинейной  фильтрации,
 называемых обычно методами \textit{субоптимальной фильтрации}~\cite{7-s1}. Одни
 приближенные методы основаны на  приближенном решении фильтрационных
 уравнений, а~другие~---  на превращении формул  для стохастических дифференциалов оптимальной
 оценки~$\hat X_t$ и~апостериорной ковариационной  матрицы ошибки~$R_t$ 
 в~стохастические дифференциальные уравнения  для~$\hat X_t$ и~$R_t$ путем разложения 
 функций~$\varphi$, $\varphi_1$ и~$\psi_1$ или $\varphi$,\linebreak\vspace*{-12pt}
 
 \pagebreak
 
 \noindent
  $\varphi_1$, 
 $\psi'\psi''$ и~$\psi, \psi_1$ в~степенные ряды и~отбрасывания остаточных членов.

 Для приближенного решения уравнения  для апостериорной
одномерной характеристической функции  $g_1(\la, \Theta)$ вектора~$X_t$ можно
использовать методы, основанные на параметризации одномерных
 распределений СтП, определяемого стохастическим
 дифференциальным уравнением~\cite{7-s1, 6-s1}.  Эти\linebreak
  методы  позволяют изучить
 стохастические диф\-ференциальные уравнения для параметров
 апостериорного распределения. Простейшим таким методом является
 МНА апостериорного распределения. Исключительно важное практическое значение имеют квазилинейные
фильтры, по\-лу\-ча\-емые с~по\-мощью методов эквивалентной линеаризации~[5--7].

%\vspace*{-3pt}

\subsection{Субоптимальный фильтр на~основе метода нормальной
аппроксимации}

Так как нормальное (гауссовское) распределение, 
аппроксимирующее апостериорное одномерное распределение~$X_t$, полностью 
определяется математическим ожиданием~$\hat X_t$ и~ковариационной матрицей~$R_t$ 
вектора~$X_t$, то при аппроксимации
апостериорного одномерного распределения вектора~$X_t$ нормальным
распределением все математические ожидания в~правых частях формул
для~$d\hat X_t$ и~$dR_t$ будут определенными
функциями~$\hat X_t$, $R_t$ и~$t$. Для гауссовских МСтС ($\psi'' \hm=0$
и~$\psi_1''\hm=0$), 
пользуясь формулой~(\ref{e2.7-s1}), можно показать, что фильтрационные уравнения будут
представлять собой стохастические дифференциальные уравнения,
определя\-ющие~$\hat X_t$ и~$R_t$:
\begin{equation}
\left.
\begin{array}{rl}
\hspace*{-3mm}d\hat X_t &= f \left(\hat X_t, Y_t,R_t,\Theta, t\right)dt +{}\\
&{}+  h\left(\hat X_t,Y_t, R_t,\Theta, t\right)\times{}\\[6pt]
& {}\times\left[ dY_t - f^{(1)} \left(\hat X_t,Y_t,
    R_t,\Theta, t\right)dt\right]\,; %\label{e2.12-s1}
\\[6pt]
\hspace*{-3mm}dR_t&=\left\{
\vphantom{\left({\hat X}_t, Y_t,R_t,\Theta, t\right)^{\mathrm{T}}}
 f^{(2)}\left(\hat X_t, Y_t,R_t,\Theta, t\right)-{}\right.\\[6pt]
&\hspace*{-10mm}{}-h\left(\hat
    X_t, Y_t,R_t,\Theta, t\right)\left(\psi_1\nu\psi_1^{\mathrm{T}}\right) 
    \left(Y_t,\Theta, t\right) \times{}\\[6pt]
&\left.{}\times h \left({\hat X}_t, Y_t,R_t,\Theta, t\right)^{\mathrm{\!T}}\right\} dt+{}\\[6pt]
&\displaystyle{}+\sss_{r=1}^{n_y}\! \rho_r \!\left({\hat X}_t,Y_t, R_t,\Theta, t\right)\times{}\\[6pt]
&\hspace*{-1mm}{}\times\left[
    dY_r -f_r^{(1)}\left({\hat X}_t,Y_t, R_t,\Theta, t\right) dt\right]\,.
    \end{array}
    \right\}
    \label{e2.13-s1}
    \end{equation}
Здесь введены следу\-ющие обозначения:

\noindent
    \begin{multline}
    f=f\left(\hat X_t, Y_t,R_t,t\right)={}\\[6pt]
{}=  \mm_{\Delta^x}^N \lk \varphi\left(Y_t,X_t,\Theta, t\right)\rk =\hat\vrp\,;
\label{e2.14a-s1}
\end{multline}

\columnbreak

\noindent
\begin{multline}
f^{(1)}=f^{(1)}\left(\hat X_t, Y_t,R_t,\Theta, t\right)={}\\
{}=
\lf f_r^{(1)} \left( \hat X_t, Y_t, R_t,\Theta,  t\right)\rf={}\\
{}=
\mm_{\Delta^x}^N \left[  \varphi_1\left(Y_t,X,\Theta, t\right) \right]=
\hat\vrp_1^{\mathrm{T}}\,;
\end{multline}

    \vspace*{-14pt}

\noindent
\begin{multline}
h=h\left(\hat X_t, Y_t,R_t,t\right)={}\\
{}=
    \left\{
    \vphantom{\left(\hat X_t, Y_t,R_t,\Theta, t\right)^{\mathrm{T}}}
    \mm_{\Delta^x}^N \left[
     \hat X_t\varphi_1\left(Y_t,X_t,\Theta, t\right)^{\mathrm{T}} +{}
     \right.\right.\\
     \left. {}+ \psi\nu\psi_1^{\mathrm{T}} 
     \left(Y_t,X_t,\Theta, t\right)
     \vphantom{     \hat X_t\left(Y_t, Theta\right)^T}\right]-{}\\
\left.  {} -\hat X_t f^{(1)}\left(\hat X_t, Y_t,R_t,\Theta, t\right)^{\mathrm{T}}
\right\}\times{}\\
{}\times \left(\psi_1\nu\psi_1^{\mathrm{T}}\right)^{-1} \left(Y_t,\Theta, t\right)\,;
\end{multline}

\vspace*{-14pt}

\noindent
\begin{multline}
f^{(2)}=f^{(2)}\left(\hat X_t, Y_t,R_t,\Theta, t\right)={}\\
{}=\mm_{\Delta^x}^N
    \left\{  \left(X_t-\hat X_t\right)\varphi\left(Y_t,X_t,\Theta, t\right)^{\mathrm{T}} + {}\right.\\
{}+ \varphi \left(Y_t,X_t,\Theta, t\right) 
\left(X_t^{\mathrm{T}}-\hat X_t^{\mathrm{T}}\right) +{}\\
\left.{}+\psi\nu\psi^{\mathrm{T}} \left(Y_t,X_t,\Theta, t\right)
     \vphantom{     \hat X_t\left(Y_t, Theta\right)^T}
\right\}\,;
\end{multline}

\vspace*{-14pt}

\noindent
\begin{multline}
    \rho_r=\rho_r\left(\hat X_t,Y_t, R_t,\Theta, t\right)={}\\
{}=\mm_{\Delta^x}^N
    \left\{  \left(X_t-\hat X_t\right)\left(X_t^{\mathrm{T}}-\hat X_t^{\mathrm{T}}\right) 
    \times{}\right.\\
   {}\times a_r \left(Y_t,X_t,\Theta, t\right)+
    \left(X_t-\hat X_t\right)\times{}\\
   {}\times b_r\left(Y_t,X_t,\Theta, t\right)^{\mathrm{T}} 
\left(X_t^{\mathrm{T}}-\hat X_t^{\mathrm{T}}\right)+ {}\\
\left.{}+
b_r \left(Y_t,X_t,\Theta, t\right) \left(X_t^{\mathrm{T}}-\hat
    X_t^{\mathrm{T}}\right)\right\} \\
(r=1\tr n_y)\,,
    \label{e2.14-s1}
    \end{multline}
где $a_r$~--- $r$-й элемент мат\-ри\-цы-стро\-ки $(\vrp_1^{\mathrm{T}} \hm-
\hat\vrp_1^{\mathrm{T}}) (\psi_1\nu\psi_1^{\mathrm{T}})$; $b_{kr}$~--- 
элемент $k$-й строки и~$r$-го столбца $(\psi\nu\psi_1^{\mathrm{T}})
(\psi_1\nu\psi_1^{\mathrm{T}})^{-1}$; $b_r$~--- 
$r$-й столбец матрицы $(\psi\nu\psi_1^{\mathrm{T}})(\psi_1\nu\psi_1^{\mathrm{T}})$, 
$b_r \hm= \lk b_{1r}\cdots b_{n_x r}\rk^{\mathrm{T}}$.

Число уравнений для апостериорного одномерного распределения
определяется по формуле:
    $$
    Q_{\mathrm{МНА}} = n_x + \fr{n_x (n_x+1)}{2} = \fr{n_x(n_x+3)}{2}\,.
    $$

За начальные значения $\hat X_t$ и~$R_t$  при интегрировании 
уравнений~(\ref{e2.13-s1}), естественно, следует принять
условные математическое ожидание и~ковариационную матрицу величины~$X_0$ относительно~$Y_0$:
\begin{equation}
\left.
\begin{array}{rl}
\hat X_0 &= \mm_{\Delta_x}^N\lk X_0 \mid Y_0\rk\,;\\[6pt] 
R_0 &= \mm_{\Delta_x}^N \lk \left(X_0 -\hat X_0\right) \left(X_0^{\mathrm{T}} -
\hat X_0^{\mathrm{T}}\right)\mid Y_0\rk\,.
\end{array}
\right\}
\label{e2.15-s1}
\end{equation}
 Если нет
информации об условном распределении~$X_0$ относительно~$Y_0$, то
начальные условия
 можно взять в~виде:  
 $ \hat X_0 \hm= \mathrm{M}\,X_0$ и~$R_0\hm= \mathrm{M}(X_0 
 \hm-\mathrm{M}\,X_0) (X_0^{\mathrm{T}} \hm- 
\mathrm{M}\,X_0^{\mathrm{T}})$. 
Если
же и~об этих величи-\linebreak\vspace*{-12pt}

\pagebreak

\noindent
 нах нет никакой информации, то начальные
значения~$\hat X_t$ и~$R_t$ приходится задавать произвольно.

Таким образом, имеем утверждение.

\smallskip

\noindent
\textbf{Теорема~2.3.}\
\textit{Пусть МСтС}~(\ref{e2.3-s1}) и~(\ref{e2.4-s1})~--- 
\textit{гауссовская  $(\psi''\hm=0)$, выполнены условия существования 
и~единственности решения, а~матрица $\si_1\hm=\psi_1 \nu \psi_1^{\mathrm{T}}$ 
не вырождена. Тогда алгоритм СОФ на основе МНА определяется  
уравнениями}~(\ref{e2.13-s1}) \textit{и}~(\ref{e2.15-s1}) 
\textit{при условиях ограниченности функций}~(\ref{e2.14a-s1})--(\ref{e2.14-s1}).

\smallskip

В основе соответствующей теоремы для МСтС~(\ref{e2.3-s1}) и~(\ref{e2.4-s1}) 
с~пуассоновскими шумами в~(\ref{e2.3-s1}) и~невырожденной матрицей 
$\si\hm=\psi_1 \nu \psi_1^{\mathrm{T}}$ лежат уравнения теоремы~2.3. 
При этом, если учесть формулу~(\ref{e2.7-s1}), потребуется ограниченность 
функций~$f$, $f^{(1)}$, $h$ и~$\rho_r$, определяемых~(\ref{e2.14a-s1})--(\ref{e2.14-s1}), и~функции
    \begin{equation}
    \bar f^{(2)}=f^{(2)}+ \mm_{\Delta^x}^N \lk \iii_{R_0^q} 
    \psi''{\psi''}^{\mathrm{T}} \nu_P (\Theta, t, dv)\rk.\label{e2.16-s1}
    \end{equation}

Таким образом, приходим к~утверждению.

\smallskip

\noindent
\textbf{Теорема~2.4.}\
\textit{Пусть МСтС}~(\ref{e2.3-s1}), (\ref{e2.4-s1}) 
\textit{удовлетворяет условиям существования и~единственности решения, а~матрица 
$\si\hm=\psi_1 \nu \psi_1^{\mathrm{T}}$ не вырождена. Тогда  СОФ на основе МНА 
определяется  уравнениями}~(\ref{e2.13-s1}) и~(\ref{e2.15-s1}) 
\textit{при условиях ограниченности функций~$f$, $f^{(1)}$, $\bar f^{(2)}$, $h$
и~$\rho_r$.}

\smallskip

\noindent
\textbf{Замечание~2.3.}\
Для гладких функций $\vrp$, $\vrp_1$, $\psi'$ и~$\psi_1'$ и~гауссовских 
МСтС~(\ref{e2.3-s1}) и~(\ref{e2.4-s1}) СОФ на основе МНА называется гауссовским 
фильтром~[5--7].

%\smallskip

\subsection{Квазилинейный субоптимальный фильтр на~основе метода 
статистической линеаризации} %2.5

Для МСтС~(\ref{e2.1-s1}), (\ref{e2.2-s1}) при $\psi'\hm=\psi'(\Theta,t)$, 
$\psi''\hm=\psi''(\Theta,t,v)$, $\psi_1'\hm=\psi_1'(\Theta,t)$
и~$\psi_1''\hm=\psi_1''(\Theta,t,v)$ (т.\,е.\ с~аддитивными винеровскими и~пуассоновскими 
шумами) уравнения НСОФ получаются проще, если нелинейные функции~$\vrp$ 
и~$\vrp_1$ на основе гауссовского  (нормального) распределения заменить на статистически 
линеаризованные~\cite{5-s1, 4-s1}:
    \begin{equation}
    \left.
\hspace*{-2mm}\begin{array}{rl}
    \vrp &=\vrp \left( X_t, Y_t, \Theta, t\right) \approx{}\\[6pt]
    & {}\approx\vrp_0 + k_x^\vrp 
    \left(X_t - m_t^x\right) + k_y^\vrp \left(Y_t - m_t^y\right)\,;
\\[6pt]
    \vrp_1 &= \vrp_1\left( X_t, Y_t, \Theta, t\right) \approx {}\\[6pt]
    &{}\approx\vrp_{10} + k_x^{\vrp_1} 
    \left(X_t - m_t^x\right) + k_y^{\vrp_1} \left(Y_t - m_t^y\right)\,,
    \end{array}
    \right\}
    \label{e2.17-s1}
    \end{equation}
а затем использовать уравнения линейной фильтрации~\cite{5-s1}. Входящие 
в~(\ref{e2.17-s1}) коэффициенты статистической линеаризации зависят от 
математических ожиданий, дисперсий и~ковариаций:
    $$
    Z_t =\begin{bmatrix} X_t\\ Y_t\end{bmatrix}\,; \enskip 
    m_t^z =\begin{bmatrix} m_t^x\\ m_t^y\end{bmatrix}\,;\enskip 
    K_t^z=\begin{bmatrix} K_t^x&K_t^{xy}\\ K_t^{xy}&K_t^y\end{bmatrix}\,.
    $$
Они определяются из уравнений
$$
\dot Z_t = A^z Z_t + A_0^z + B_0^z V\,,\enskip V= \dot W\,;
%\label{e2.18-s1}
$$
\begin{equation}
\left.
\begin{array}{c}
\dot m_t^z = A^z m_t^z + A_0^z \,,\enskip m_{t_0}^Z = m_0^z\,;\\[6pt] %\label{e2.19-s1}
\hspace*{-5mm}\dot K_t^z = B^z K_t^z + K_t^z \left(B^z\right)^{\mathrm{T}} + B_0^z 
\nu^m (B_0^z)^{\mathrm{T}}\,;\\[6pt]
\hspace*{43mm}K_{t_0}^z = K_0^z\,.
\end{array}
\right\}
\label{e2.20-s1}
\end{equation}
Здесь введены следующие обозначения:
    $$
    A_0^z = \begin{bmatrix} a_0\\ b_0\end{bmatrix}\,;\enskip A^z =
    \begin{bmatrix} a_1&a\\ b_1&b\end{bmatrix}\,;\enskip 
    B_0^z =\begin{bmatrix} \bar \psi\\ \bar\psi_1\end{bmatrix}\,,
    $$
    где
   \begin{alignat*}{3}
    a_0 &=\vrp_0 - k_x^\vrp m_t^x - k_y^\vrp m_t^y\,; &\enskip
 a_1 &= k_x^\vrp\,;&\enskip          a&= k_y^\vrp\,;\\
    b_0&=\vrp_0 -k_x^{\vrp_1} m_t^x -k_y^{\vrp_1}m_t^y\,;&\enskip
b_1&=k_x^{\vrp_1}\,;&\enskip        b&= k_y^{\vrp_1}\,;  %\label{e2.21-s1}
    \end{alignat*}
\begin{equation}
\left.
\begin{array}{c}
   \displaystyle \psi\, dW_0 + \iii_{R_0^q} \psi'' P^0 (dt, dv) =
    \bar \psi \, dW_0\,;\\[6pt] 
   \displaystyle    \psi_1'\, dW_0 + \iii_{R_0^q} \psi_1'' P^0 (dt, dv)= \bar \psi_1 \,dW\,.
    \end{array}
    \right\}
    \label{e2.22-s1}
    \end{equation}
 
 Тогда уравнения квазилинейного 
НСОФ будут иметь вид:
\begin{align}
\dot{\hat X}_t &= a Y_t + a_1\hat X_t + a_0 +{}\notag\\
&\hspace*{12mm}{}+ \beta_t 
\left[ Z_t - \left(bY_t + b_1 \hat X_t + b_0\right)\right]\,;\label{e2.23-s1}\\
\beta_t &= \left(R_t b_1^{\mathrm{T}} + \bar\psi \nu^W \bar\psi_1^{\mathrm{T}}\right) 
\left(\bar\psi_1\nu^W\bar\psi_1^{\mathrm{T}}\right)^{-1}\,;\label{e2.24-s1}\\
\dot R_t &= a_1 R_t + R_t a_1^{\mathrm{T}} + \bar\psi \nu^W \bar\psi^{\mathrm{T}} -
\left(R_t b_1^{\mathrm{T}} +\bar\psi \nu^W\bar\psi_1^{\mathrm{T}}\right)\times{}\notag\\
&\hspace*{8mm}\times\left(\bar\psi_1 \nu^W\bar\psi_1^{\mathrm{T}}\right)^{-1} 
\left(b_1 R_t + \bar\psi_1 \nu^W\bar\psi^{\mathrm{T}}\right)\,,
\label{e2.25-s1}
\end{align}
где $\nu^W$~--- интенсивность СтП с~независимыми приращениями, состоящего из 
винеровской и~пуассоновской частей~(\ref{e2.22-s1}).

%\smallskip

\noindent
\textbf{Теорема~2.5.}\ \textit{Пусть МСтС}~(\ref{e2.1-s1}), (\ref{e2.2-s1}) 
\textit{содержит только
аддитивные винеровские и~пуассоновские шумы и~допускает замену
статистически линеаризованной системой, а матрица $\si_1
\hm=\bar\psi_1 \nu^W \bar\psi_1^{\mathrm{T}}$ не вырождена. Тогда в~основе
алгоритма квазилинейного НСОФ лежат уравнения}~(\ref{e2.23-s1})--(\ref{e2.25-s1}) \textit{при
начальных условиях}~(\ref{e2.15-s1}).

\smallskip 

\noindent
\textbf{Замечание~2.4.}\
Уравнения теоремы~2.3 сохраняют свой вид, если коэффициенты статистической 
линеаризации в~(\ref{e2.17-s1}) вычислять для известного эквивалентного 
(негауссовского) распределения. При этом уравнения~(\ref{e2.20-s1}), 
(\ref{e2.23-s1})--(\ref{e2.25-s1}), как известно~\cite{7-s1}, 
имеют место для любого негауссовского~СтП.


\smallskip

\noindent
\textbf{Замечание~2.5.}\
Из теоремы~2.3 немедленно следуют уравнения НСОФ для фильтрации стационарных 
процессов в~установившемся режиме для стационарных МСтС, если приравнять нулю правые 
части уравнений~(\ref{e2.20-s1}), (\ref{e2.23-s1}) и~(\ref{e2.25-s1}).

\section{Ортогональные субоптимальные фильтры}

\subsection{Гауссовские шумы}

При аппроксимации апостериорной одномерной плотности отрезком ее ортогонального 
разложения~\cite{1-s1, 2-s1}:
\begin{multline}
p_t (x, \Theta)= p^* (x; \Theta, \vartheta) ={}\\
{}= w (x; \Theta) 
\lk 1+ \sss_{k=3}^N \sss_{|\nu | =k} c_\nu p_\nu (x)\rk
\label{e3.1-s1}
\end{multline}
естественно принять за параметры,
 образующие вектор~$\vartheta$, апостериорные математическое
 ожидание~$\hat X_t$, ковариационную матрицу~$R_t$ вектора~$X_t$, а~так\-же
 коэффициенты ортогонального разложения (КОР)~$c_\nu$ $(\lv \nu\rv \hm= 3\tr N)$.
 Здесь КОР определяется формулой:
\begin{equation}
c_\kappa = \lk q_\kappa \left(\fr{\partial}{i\partial \la}\right) 
g_t (\la,\Theta)\rk_{\la=0}\,.\label{e3.2-s1}
\end{equation}
Заметим, что полином~$q_\kappa$ зависит от~$\hat X_t$ и~$R_t$.

На основе~(\ref{e2.7-s1}) и~(\ref{e2.9-s1}) для  гауссовской 
МСтС~(\ref{e2.3-s1}) и~(\ref{e2.4-s1}) при  $\psi''\hm=0$ получим, что 
ОСОФ определяется следующими уравнениями:
\begin{equation}
\left.
\begin{array}{rl}
d\hat X_t&=f\,dt + h     \left(dY_t- f^{(1)} \,dt\right)\,; %\label{e3.3-s1}
\\[6pt]
dR_t&= \left(f^{(2)} -h\psi_1\nu\psi_1^{\mathrm{T}} h^{\mathrm{T}}\right) dt +{}\\[6pt]
&\hspace*{15mm}\displaystyle{}+\sss_{r=1}^{n_y} \rho_r \left(dY_r -f_r^{(1)}\, dt\right)\,.
\end{array}
\right\}
\label{e3.4-s1}
\end{equation}
Здесь введены обозначения:
    \begin{equation}
    \left.
    \begin{array}{rl}
    f&= f\left(Y_t,\vartheta,\Theta, t\right)=\mm_{\Delta^x}^{p^*} 
    \left[\varphi(Y_t,X,\Theta,t)\right]\,;
\\[6pt]
f^{(1)}&= \lf f_r^{(1)}\rf= f^{(1)}\left(Y_t,\vartheta,\Theta, t\right)={}\\[6pt]
&\hspace*{21mm}{}=
    \mm_{\Delta^x}^{p^*}  \lk\varphi_1\left(Y_t,X,\Theta,t\right)\rk\,;
    \end{array}
    \right\}
    \label{e3.5a-s1}
\end{equation}

\vspace*{-12pt}

\noindent
   \begin{multline}
f^{(2)}= f^{(2)}\left(Y_t,\vartheta,\Theta,t\right)={}\\
{}=
    \mm_{\Delta^x}^{p^*} \left[ \left(X-\hat X_t\right)\varphi
    \left(Y_t,X,\Theta,t\right)^{\mathrm{T}}+{}\right.\\
{}+\varphi\left(Y_t,X,\Theta,t\right) 
\left(X^{\mathrm{T}}-\hat X_t^{\mathrm{T}}\right)
 + {}\\
\left. {}+\left(\psi\nu\psi^{\mathrm{T}}\right) \left(Y_t,X,\Theta,t\right)
\vphantom{\hat X \left(Y-t\Theta\right)^T}
\right] \,;
\label{e3.5b-s1}
 \end{multline}
 
 \vspace*{-12pt}
 
 \noindent
  \begin{multline}
h= h\left(Y_t,\vartheta,\Theta,t\right)=\left\{
    \mm_{\Delta^x}^{p^*} \left[ X\varphi_1\left(Y_t,X,\Theta,t\right)^{\mathrm{T}}+{}\right.\right.\\
\left.{}+ \left(\psi\nu\psi_1^{\mathrm{T}}\right) 
\left(Y_t,X,\Theta,t\right)
\vphantom{\hat X \left(Y-t\Theta\right)^T}
\right]-{}\\
\left.{}-
    \hat X_t f^{(1)T}\right\} (\psi_1\nu\psi_1^{\mathrm{T}})^{-1} \left(Y_t,\Theta,t\right)\,;
    \label{e3.5c-s1}
    \end{multline}
    

\noindent
\begin{multline}
\rho_r= \rho_r\left(Y_t,\vartheta\Theta,,t\right)={}\\
{}=
   \mm_{\Delta^x}^{p^*}  \left[ \left(X-\hat X_t\right) 
   \left(X^{\mathrm{T}}-\hat X_t^{\mathrm{T}}\right) a_r
   \left(Y_t,X,\Theta,t\right)+{}\right.\\
{}+ \left(X-\hat X_t\right) b_r\left(Y_t,X,\Theta,t\right)^{\mathrm{T}}+ {}\\
\left.{}+
b_r\left(Y_t,X,\Theta,t\right)\left(X^{\mathrm{T}}-\hat X_t^{\mathrm{T}}\right)
\right]\\ (r=1\tr n_y)\,.
\label{e3.5d-s1}
\end{multline}

Далее перепишем~(\ref{e3.4-s1}) покоординатно:
\begin{equation}
\left.
\begin{array}{rl}
d\hat X_s &= f_s dt + h_s \left(dY_t- f^{(1)}\, dt\right) ={}\\[6pt]
&\hspace*{-5mm}{}= A^{\hat X_s} \,dt + 
B^{\hat X_s} \,dY_t\enskip (s=1\tr  n_x)\,; %\label{e3.6-s1}
\\[6pt]
dR_{sq} &=\left(f_{sq}^{(2)} - h_s\psi_1\nu\psi_1^{\mathrm{T}}
h_q^{\mathrm{T}} \right) dt +{}\\[6pt]
&\hspace*{-10mm}{}+\eta_{sq} \left(dY_t - f^{(1)}\, dt\right)=
A^{R_{sq}} \,dt + B^{R_{sq}} \,dY_t\,,
\end{array}
\right\}
\label{e3.7-s1}
\end{equation}
где $\hat X_s (t_0) \hm= X_{s0}$; $R_{sq} (t_0) \hm= R_{sq0}$;  $s,q\hm=1\tr n_x$; 
$\eta_{sq}$~--- мат\-ри\-ца-стро\-ка, элементами которой служат
соответствующие элементы матрицы  $\rho_1\tr \rho_{n_1}$:
\begin{equation*}
\eta_{sq} =\eta_{e_s+e_q} = \lk \rho_{1sq}\cdots \rho_{msq}\rk
    \enskip (s,q,=1\tr n_x)\,.
%    \label{e3.8-s1}
    \end{equation*}
Здесь и~далее для краткости индекс~$t$ сохраним только у~$Y_t$. 
По формуле дифференцирования Ито для винеровского СтП~\cite{3-s1, 4-s1}, 
учитывая~(\ref{e3.7-s1}), находим в~силу~(\ref{e3.2-s1}) 
стохастический дифференциал:
\begin{multline*}
dc_\kappa =\lk d\lf q_\kappa \left(\fr{\partial}{i\partial \la}\right) g_t
    (\la,\Theta)\rf\rk_{\la=0}={}\\
{}=\sss_{s=1}^{n_x} \lk\partial q_\kappa
    \left(\fr{\partial}{i\partial \la}\right) \partial \hat X_s\cdot g_t
    (\la,\Theta)\rk_{\la=0}d \hat X_s+{}\\
{}+\sss_{s,u=1}^{n_x} \lk\partial
    q_\kappa \left(\fr{\partial }{i\partial \la}\right) \partial R_{su}\cdot g_t
    (\la,\Theta)\rk_{\la=0}dR_{su} +{}\\
    {}+ \lk q_\kappa \left(\fr{\partial}{i\partial  \la}\right) d g_t (\la,\Theta)\rk_{\la=0}+{}\\
{}+ \Bigg\{ \fr{1}{2}\hspace*{-2pt} \sss_{s,u=1}^{n_x}\! \lk\!
\fr{\partial^2 q_\kappa
    \left(\partial /(i\partial \la)\right)\cdot g_t (\la,\Theta) }
    {\partial \hat X_s
    \partial \hat X_u }\rk_{\la=0}\hspace*{-5mm} h_s \psi_1\nu\psi_1^{\mathrm{T}} h_u^{\mathrm{T}} +{}\\
{}+ \fr{1}{2} \hspace*{-4pt}\sss_{s,u,k,l=1}^{n_x} \!\lk\!
\fr{\partial^2 q_\kappa \left(\partial /(i\partial \la)\right)\cdot
    g_t (\la,\Theta)}{\partial R_{su} \partial R_{kl}}
    \rk_{\la=0}\hspace*{-5.5mm} \eta_{su} \psi_1\nu\psi_1^{\mathrm{T}} \eta_{kl}^{\mathrm{T}}+{}\\
{}+\!\!\sss_{s,k,l=1}^{n_x} \!\lk\!
\fr{\partial^2 q_\kappa \left(\partial /(i\partial \la)\right)\cdot g_t (\la,\Theta)}
    {\partial \hat X_s \partial R_{kl}}\rk_{\la=0}\hspace*{-4mm}h_s \psi_1\nu\psi_1^{\mathrm{T}} 
    \eta_{kl}^{\mathrm{T}}
    \!\Bigg\}\, dt. \hspace*{-2.2177pt}
    %\label{e3.9-s1}
    \end{multline*}
Подставив сюда выражения~(\ref{e3.7-s1}) и~(\ref{e2.7-s1})
дифференциалов $d\hat X_s$, $dR_{sq}$  и~$dg_t (\la,\Theta)$ и~вспомнив,
что для любого полинома  $P(x)$ $P\lk (\partial /(i\partial
\la)) g_t(\la)\rk_{\la=0}\hm=P(\alp)$, получаем стохастические
дифференциальные уравнения:

\noindent
\begin{multline}
dc_\kappa =\left\{ 
\vphantom{\sss_{s,k,l=1}^{n_x}}
F_\kappa +\sss_{s=1}^{n_x} \fr{\partial q_\kappa
    (\alp)}{\partial \hat X_s}\,f_s +{}\right.\\
    {}+\sss_{s,u=1}^{n_x} \fr{\partial
    q_\kappa (\alp)}{\partial R_{su}} \left( 
    f_{su}^{(2)} - h_s \psi_1\nu\psi_1^{\mathrm{T}} h_u^{\mathrm{T}}\right)+{}\\
{}+\fr{1}{2}\sss_{s,u=1}^{n_x} \fr{\partial^2 q_\kappa (\alp)}
{\partial \hat X_s \partial \hat X_u}\, 
    h_s \psi_1\nu\psi_1^{\mathrm{T}} h_u^{\mathrm{T}} +{}\\
    {}+ \fr{1}{2}
    \sss_{s,u,k,l=1}^{n_x} \fr{\partial^2 q_\kappa (\alp)}
    {\partial R_{su}\partial R_{kl}}\,
    \eta_{su} \psi_1\nu\psi_1^{\mathrm{T}} \eta_{kl}^{\mathrm{T}}+{}\\
\left.{}+\sss_{s,k,l=1}^{n_x} \fr{\partial^2 q_\kappa (\alp)}
{\partial \hat X_s \partial R_{kl}}\,h_s \psi_1\nu\psi_1^{\mathrm{T}} 
\eta_{kl}^{\mathrm{T}}\right\} dt +{}\\
{}+\left\{ H_\kappa +
    \sss_{s=1}^{n_x} \fr{\partial q_\kappa (\alp)}{\partial \hat X_s}\,h_s 
+\sss_{s,u=1}^{n_x} \fr{\partial
    q_\kappa (\alp)}{\partial R_{su}}\,\eta_{su}\right\}\times{}
    \\
    {}\times
     \left(dY_t -
    f^{(1)} dt\right)= A^{c_\kappa} dt + B^{c_\kappa} dY_t\,,\enskip 
    c_\kappa (t_0) = c_{\kappa0} \\
     (\lv\kappa\rv= 3\tr N)\,.
    \label{e3.10-s1}
    \end{multline}
Здесь в~дополнение к~прежним обозначениям принято:

\vspace*{4pt}

\noindent
\begin{equation}
\left.
\begin{array}{rl}
 \hspace*{-2mm}F_\kappa &= F_\kappa \left(Y_t,\Theta,\vartheta,t\right) ={}\\[3pt]
 &{}=\displaystyle\sss_{s=1}^{n_x}
   \mm_{\Delta^x}^{p^*} \left[ \varphi_s \left(Y_t,X,\Theta,t\right)
   \fr{\partial q_\kappa (X)}{\partial X_s}\right]+{}\\[5pt]
&\hspace*{-7mm}{}+\fr{1}{2} \displaystyle\sss_{s,u=1}^{n_x}
     \mm_{\Delta^x}^{p^*}\lk \si_{su} \left(Y_t,X,\Theta,t\right)\fr{\partial^2 q_\kappa
    (X)}{\partial X_s\partial X_u}\rk;
   \\[5pt]
   \hspace*{-2mm}H_\kappa &= H_\kappa \left(Y_t,\vartheta,\Theta,t\right) ={}\\[3pt]
  & \hspace*{4mm}{}=
   \Bigg\{  \mm_{\Delta^x}^{p^*}
   \lk \varphi_1 \left(Y_t,X,\Theta,t\right)^{\mathrm{T}} q_\kappa (X)\rk+{}\\[5pt]
&\left.\hspace*{-7.5mm}{}+ \displaystyle\!\sss_{s=1}^{n_x}\!
     \mm_{\Delta^x}^{p^*}\!\lk \left(\psi\nu\psi_1^{\mathrm{T}}\right)_s 
     \left(Y_t,X,\Theta,t\right) \fr{\partial
    q_\kappa (X)}{\partial X_s} \rk-{}\right.\\[5pt]
&    \hspace*{6mm}{}- c_\kappa
    f^{(1)T} \Bigg\} \left(\psi_1\nu\psi_1^{\mathrm{T}}\right)^{-1} \left(Y_t,\Theta,t\right)\,,
    \end{array}
    \right\}
    \label{e3.11-s1}
    \end{equation}
    
    \vspace*{-2pt}

\noindent
где через $(\psi\nu\psi_1^{\mathrm{T}})_s$ обозначена  $s$-я строка матрицы
$\psi\nu\psi_1^{\mathrm{T}}$; $\si\hm=\psi\nu\psi_1^{\mathrm{T}} \hm=\lf \si_{su}\rf$.

Функции  $f_s$, $f^{(1)}$, $f^{(2)}_{su}$, $h_s$, $\eta_{su}$, $F_\kappa$ 
и~$H_\kappa$ в~уравнениях~(\ref{e3.7-s1}) и~(\ref{e3.10-s1}) представляют
собой линейные комбинации величин  $c_\nu$ $(\lv\nu\rv \hm= 3\tr N)$
с~коэффициентами, зависящими от~$\hat X_t$ и~$R_t$. Величины
$\partial q_\kappa (\alp)/\partial \hat X_s$, $\partial q_\kappa
(\alp)/\partial R_{su}$, $\partial^2 q_\kappa (\alp)/(\partial \hat
X_s \partial \hat X_u)$, $\partial^2 q_\kappa (\alp)/(\partial
R_{su} \partial R_{kl})$ и~$\partial^2 q_\kappa (\alp)/(\partial \hat
X_s \partial R_{kl})$ после замены моментов их выражениями
через~$c_\nu$ тоже будут линейными комбинациями величин~$c_\nu$ 
с~коэффициентами, зависящими от~$\hat X_t$ и~$R_t$.

В частном случае разложений~(\ref{e3.1-s1}) по
полиномам Эрмита КОР~$c_\nu$ представляют собой
квазимоменты (КМ). В~этом случае, как показано в~\cite{7-s1, 6-s1}, 
для производных полиномов Эрмита~$G_\nu$, формулы~(\ref{e3.11-s1}) приводятся к~виду:

\columnbreak

\noindent
  \begin{equation}
  \left.
  \begin{array}{rl}
    \hspace*{-3mm}F_\kappa &={}\\
&\hspace*{-7.5mm}    {}=\displaystyle\sss_{s=1}^{n_x}\! \kappa_s
    \mm_{\Delta^x}^{p^*}\lk \varphi_s \left(Y_t,X,\Theta,t\right)G_{\kappa-e_s} (X-m)\rk+{}\!\!\\[6pt]
&   \hspace*{-5mm} {}+\fr{1}{2} \displaystyle\sss_{s=1}^{n_x}  \kappa_s \left(\kappa_s-1\right)
     \mm_{\Delta^x}^{p^*}\left[ 
     \si_{ss} \left(Y_t,X,\Theta,t\right)\times {}\right.\\[6pt]
&\left.    {}\times G_{\kappa-2e_s}(X-m)
     \right]+ \displaystyle\sss_{u=2}^{n_x} \sss_{s=1}^{u-1}  \kappa_s
    \kappa_u  \times{}\\[6pt]
 &   \hspace*{-6mm}{}\times \mm_{\Delta^x}^{p^*}\lk \si_{su} \left(Y_t,X,\Theta,t\right)
    G_{\kappa-e_s-e_u}(X-m)\rk\,;
\\[6pt]
      \hspace*{-3mm} H_\kappa &= \left\{
     \mm_{\Delta^x}^{p^*}\left[
    \varphi_1 \left(Y_t,X,\Theta,t\right)^{\mathrm{T}} G_{\kappa} (X-m) \right]+{}\right.\\[6pt]
&\displaystyle{}+\sss_{s=1}^{n_x} \kappa_s
     \mm_{\Delta^x}^{p^*}\left[ \left(\psi\nu\psi_1^{\mathrm{T}}\right) 
     \left(Y_t,X,t\right)\times{}\right.\\
     &\left.\hspace*{20mm}{}\times G_{\kappa-e_s} (X-m) 
     \vphantom{\psi^T}\right]-{}\\[6pt]
&\left.\hspace*{10mm}{}-f^{(1)T} c_\kappa 
\vphantom{Y_t\left(\Theta\right)^T}\right\} \left(\psi_1\nu\psi_1^{\mathrm{T}}
\right)^{-1} 
\left(Y_t,t\right)\,,
\end{array}\!
\right\}\!\!\!
\label{e3.12-s1}
\end{equation}
где
    $$\fr{\partial q_\kappa (\alp)}{\partial \hat X_s} = -\kappa_s
    c_{\kappa-e_s}\,;
    $$
    $$
    \fr{\partial q_\kappa (\alp)}{\partial R_{ss}}= -\fr{1}{2}\,
    \fr{\partial q_\kappa (\alp) }{\partial \hat X_s^2} = -\fr{1}{2}\,
    \kappa_s \left(\kappa_s-1\right) c_{\kappa- 2 e_s}\,;
    $$
    $$
    \fr{\partial q_\kappa (\alp)}{\partial   R_{su}}= -
    \fr{\partial^2 q_\kappa (\alp)}{\partial \hat X_s \partial
    \hat X_u} = -\kappa_s \kappa_u c_{\kappa- e_s-e_u}\,;
    $$
    $$
    \fr{\partial^2 q_\kappa  (\alp)}{\partial R_{ss}^2}= \fr{1}{4}\,
    \kappa_s\left(\kappa_s -1\right)
    \left(\kappa_s -2\right) \left(\kappa_s-3\right) c_{\kappa- 4e_s}\,;
    $$
    $$
    \fr{ \partial^2 q_\kappa     (\alp)}{\partial R_{ss} \partial R_{kk}}= 
    \fr{1}{4}\,\kappa_s\left(\kappa_s -1\right)\kappa_s \left(\kappa_s -1\right) 
     c_{\kappa- 2e_s- 2  e_k}\,;
     $$
    $$
    \fr{\partial^2 q_\kappa     (\alp)}{\partial R_{ss} \partial R_{sl}}= \fr{1}{2}\,
    \kappa_s\left(\kappa_s -1\right) \left(\kappa_s -2\right) \kappa_l c_{\kappa- 3e_s- e_l}\,;
    $$
    $$
    \fr{\partial^2 q_\kappa     (\alp)}{\partial R_{ss} \partial R_{kl}}= \fr{1}{2}\,
    \kappa_s\left(\kappa_s -1\right) \kappa_k \kappa_l c_{\kappa- 2e_s-e_k- e_l}\,;
    $$
    $$
    \fr{\partial^2 q_\kappa     (\alp)}{\partial R_{su} \partial R_{sl}}= 
    \kappa_s\left(\kappa_s -1\right)
    \kappa_u \kappa_l c_{\kappa- 2e_s-e_u-  e_l}\,;
    $$
    $$
    \fr{\partial^2 q_\kappa (\alp)}{\partial R_{su} \partial R_{kl}}=
    \kappa_s\kappa_u \kappa_k\kappa_l c_{\kappa- e_s-e_u-e_k-e_l}\,;
    $$
    $$
    \fr{\partial^2 q_\kappa     (\alp)}{\partial \hat X_s \partial R_{ss}}= \fr{1}{2}\,
    \kappa_s\left(\kappa_s -1\right) \left(\kappa_s -2\right) c_{\kappa- 3e_s}\,;
    $$
    $$
    \fr{\partial^2 q_\kappa     (\alp)}{\partial \hat X_s \partial R_{sl}}= 
    \kappa_s\left(\kappa_s -1\right)  \kappa c_{\kappa- 2e_s- e_l}\,;
    $$
    $$
    \fr{\partial^2 q_\kappa     (\alp)}{\partial \hat X_s \partial R_{kk}}= \fr{1}{2}\,
    \kappa_s\kappa_k \left(\kappa_k -1\right) c_{\kappa- e_s-2     e_k}\,;
    $$
    $$
    \fr{\partial^2 q_\kappa     (\alp)}{\partial \hat X_s \partial R_{kl}}= \kappa_s
    \kappa_k     \kappa_l c_{\kappa- e_s-e_k-     e_l}\,.
%    \label{e3.13-s1}
 $$

Таким образом, имеем следующие утверждения.

\smallskip

\noindent
\textbf{Теорема~3.1.}\ \textit{Пусть МСтС}~(\ref{e2.3-s1}), (\ref{e2.4-s1})~--- 
\textit{гауссовская
$(\psi'' \hm=0)$, выполнены условия существования и~единственности
решения, а матрица $\si_1 \hm= \psi_1 \nu \psi_1^{\mathrm{T}}$ не вырождена. Тогда
в основе алгоритма ОСОФ по МОР лежат уравнения}~(\ref{e3.1-s1}), (\ref{e3.7-s1})
и~(\ref{e3.10-s1}) \textit{при условии ограниченности функций}~(\ref{e3.11-s1}).

\vspace*{3pt}

\noindent
\textbf{Теорема~3.2.}\
\textit{В~условиях теоремы~$3.1$ алгоритм ОСОФ по МКМ определяется 
уравнениями}~(\ref{e3.1-s1}),
(\ref{e3.7-s1}) \textit{и}~(\ref{e3.10-s1}) 
\textit{при условии ограниченности функций}~(\ref{e3.12-s1}).

\vspace*{-3pt}

\subsection{Негауссовские шумы}

Пользуясь формулами~(\ref{e2.7-s1}) и~(\ref{e2.9-s1}), устанавливаем, что наличие
 пуассоновского шума влияет только на функцию~$f^{(2)}$. В~результате заменим~$f^{(2)}$ 
 в~(\ref{e3.5b-s1}) на~$\bar f^{(2)}$ согласно~(\ref{e2.16-s1}) 
 и~придем к~следующим утверждениям.

\vspace*{3pt}

\noindent
\textbf{Теорема~3.3.}\
\textit{Пусть для МСтС}~(\ref{e2.3-s1}), (\ref{e2.4-s1}) 
\textit{выполнены условия существования и~единственности решения, а~мат\-ри\-ца $\si_1\hm=
\psi_1\nu\psi_1^{\mathrm{T}}$ не вырождена. Тогда алгоритм ОСОФ, согласно МОР, задается
 уравнениями}~(\ref{e3.1-s1}), (\ref{e3.7-s1}) и~(\ref{e3.10-s1}) 
 \textit{при условии ограниченности\linebreak функций~$f$, $f^{(1)}$, $\bar f^{(2)}$, 
 $h$, $\rho_r$, $F_\kappa$ и~$H_\kappa$, определя\-емых}~(\ref{e3.5a-s1})--(\ref{e3.5d-s1}), 
 (\ref{e2.16-s1})
 \textit{и}~(\ref{e3.11-s1}).

\vspace*{3pt}

\noindent
\textbf{Теорема~3.4.}\
\textit{Пусть для МСтС}~(\ref{e2.3-s1}), (\ref{e2.4-s1}) 
\textit{выполнены условия существования и~единственности решения, а~мат\-ри\-ца 
$\si_1\hm=\psi_1\nu\psi_1^{\mathrm{T}}$ не вырождена. Тогда алгоритм ОСОФ, 
согласно МКМ, задается уравнениями}~(\ref{e3.1-s1}), (\ref{e3.7-s1}) 
и~(\ref{e3.10-s1}) \textit{при условии ограниченности функций}~$f$, 
$f^{(1)}$, $\bar f^{(2)}$, $h$, $\rho_r$, $F_\kappa$ и~$H_\kappa$, 
\textit{определя\-емых}~(\ref{e3.5a-s1})--(\ref{e3.5d-s1}), (\ref{e2.16-s1}) \textit{и}~(\ref{e3.12-s1}).

\vspace*{-3pt}

\section{Точность и~чувствительность ортогонального субоптимального фильтра}

Точность СОФ на базе МНА (МСЛ) оценивается на основе ОСОФ по МОР или МКМ путем 
удержания конечного числа членов в~разложении~(\ref{e3.1-s1}).

Применяя методы теории чувствительности~\cite{9-s1, 10-s1} 
для приближенного анализа фильтрационных уравнений в~теоремах~3.1--3.4 и~учитывая 
случайность параметров~$\Theta$, придем к~следующим уравнениям для функций 
чувствительности первого порядка:

\vspace*{-4pt}

\noindent
\begin{multline*}
d\nabla^\Theta \hat X_s = \nabla^\Theta A^{\hat X_s} \,dt + 
\nabla^\Theta B^{\hat X_s}\,dY_t,\\
\nabla^\Theta B^{\hat X_s}(t_0) =0\,; % \label{e4.1-s1}
\end{multline*}

\vspace*{-14pt}

\noindent
\begin{multline*}
d\nabla^\Theta R_{sq} = \nabla^\Theta A^{R_{sq}}\, dt + \nabla^\Theta B^{R_{sq}}\,dY_t, \\
\nabla^\Theta R_{sq}(t_0) =0\,; %\label{e4.2-s1}\\
\end{multline*}

\vspace*{-3pt}

\noindent
\begin{equation*}
d\nabla^\Theta c_{\kappa} = \nabla^\Theta A^{c_\kappa}\, dt + \nabla^\Theta 
B^{c_\kappa}\,dY_t\,,\enskip \nabla^\Theta c_\kappa(t_0) =0\,.
%\label{e4.3-s1}
\end{equation*}

\columnbreak

\noindent
Здесь вычисление взятия производных ведется по всем входящим переменным, 
а~коэффициенты чувствительности вычисляются при  $\Theta\hm=m^\Theta$. При этом 
предполагается малость дисперсий по сравнению с~их математическими ожиданиями. 
Очевидно, что при дифференцировании по~$\Theta$ $(\nabla^\Theta \hm= \prt /\prt\Theta)$
порядок уравнений возрастает пропорционально числу производных. Аналогично составляются 
уравнения для элементов матриц вторых функций чувствительности.

Следуя~\cite{1-s1, 2-s1}, для оценки качества ОСОФ, определяемых теоремами~3.1--3.4 
при гауссовских~$\Theta$ с~математическим ожиданием~$m^\Theta$ и~ковариационной 
матрицей~$K^\Theta$, введем условную функцию потерь, допускающую квадратичную 
аппроксимацию:
\begin{multline*}
    \rho^{\hat X_s}=\rho^{\hat X_s}(\Theta) =\rho (m^\Theta) +\sss_{ii=1}^{n^\Theta} 
    \rho_i' \left(m^\Theta\right)\Theta_s^0+ {}\\
    {}+\ss2\limits_{i,j=1} 
    \rho_{ij}'' \left(m^\Theta\right)\Theta_i^0 \Theta_j^0\,,
    %\label{e4.4-s1}
    \end{multline*}
а также показатель~$\varepsilon$, определяемый формулой:
  \begin{equation*}
  \varepsilon =\varepsilon_2^{1/4}\,.
  %\label{e4.5-s1}
  \end{equation*}
Здесь
    $$
    \varepsilon_2 = \mm^N \lk \rho (\Theta)^2\rk -\rho (m^\Theta)^2\,, %\label{e4.6-s1}
    $$
где
\begin{multline*}
\mm^N \left[ \rho(\Theta)^2\right] = \rho \left(m^\Theta\right)^2 +
\rho' \left(m^\Theta\right)^{\mathrm{T}} K^\Theta \rho'\left(m^\Theta\right)+ {}\\
{}+
2\rho \left(m^\Theta\right) \mathrm{tr}\, \left[ 
\rho''\left(m^\Theta\right)K^\Theta\right]+{}\\
{}+\left\{ \mathrm{tr}\, \lk \rho'' \left(m^\Theta\right) K^\Theta\rk \right\}^2+2 \mathrm{tr}\, 
\left[ \rho''\left(m^\Theta\right) K^\Theta\right]^2\,,
%\label{e4.7-s1}
\end{multline*}
а функции $\rho'$ и~$\rho''$ по известным формулам~\cite{9-s1, 10-s1} 
определяются на основе первых и~вторых функций чувствительности.

Изложенные выше методы синтеза ОСОФ дают
принципиальную возможность получить фильтр, близкий к~оптимальному, по
оценке с~любой степенью точности.
Чем выше максимальный порядок~$N$~учитываемых моментов, КОР и~КМ, тем выше будет точность
приближения к~оптимальной оценке. Однако число уравнений,
определяющих параметры апостериорного одномерного распределения, быстро растет
с увеличением числа учитываемых параметров.
Соответствующие оценки можно найти в~[5--7].

\section{Тестовый пример}

Рассмотрим нелинейную с~мультипликативным шумом гауссовскую  стохастическую систему~\cite{6-s1}:

\noindent
\begin{gather*}
\dot X_t =- X_t^3 + X_t V_1(\Theta)\,;\enskip Z_t (\Theta) = 
\dot Y_t= X_t + V_2(\Theta)\,; %\label{e5.1-s1}
\\
\hat X_{t_0} = \mm X(t_0)\,;\enskip R_{t_0} = {\sf D}X\left(t_0\right)\,.
%\label{e5.2-s1}
\end{gather*}
Здесь предполагается, что  интенсивности гауссовских белых шумов~$\nu_1$ и~$\nu_2$ 
зависят от одного скалярного параметра~$\Theta$, т.\,е.\ $\nu_{1,2} \hm= 
\nu_{1,2} (\Theta)$.

Положим $\nabla =\nabla^\Theta\hm= (\prt / \prt \Theta)$. Тогда уравнения точности 
и~чувствительности СОФ на основе МНА имеют вид:
  \begin{equation}
  \left.
  \hspace*{-3mm}\begin{array}{rl}
  \dot{\hat X}_t &= - \hat X_t \left( \hat X_t^2 + 3 R_t\right) + 
  \nu_2^{-1} R_t \left(Z_t - \hat X_t\right)\,; %\label{e5.3-s1}
  \\[6pt]
  \dot R_t &= \left(\nu_1 - 6 R_t\right) \left( \hat X_t^2 + R_t\right) - 
  \nu_2^{-1} R_t^2\,;
  \end{array}
  \right\}
  \label{e5.4-s1}
  \end{equation}
    \begin{equation}
    \left.
    \begin{array}{l}
\hspace*{-1.5mm}\nabla \dot{\hat X}_t = - \lk 3\hat X_t +\left(3 + \nu_2^{-1} \right)\rk 
\nabla\hat X_t+ {}\\
\hspace*{3mm}{}+\nu_2^{-1} R_t \nabla Z_t - \nu_2^{-2} R_t \left(Z_t -\hat X_t\right)
    \nabla \nu_2\,,\\    
    \hspace*{45mm} \nabla \hat X_{t_0}=0\,;
    %\label{e5.5-s1}
    \\[6pt]
\hspace*{-1.5mm}\nabla \dot R_t = 2\left(\nu_1 - 6 R_t\right) \hat X_t \nabla \hat X_t + \big[ 
\left(\nu_1 - 6 R_t\right) - {}\\
\hspace*{12mm}{}-6 \left(\hat X_t^2 + R_t\right) - 
2 \nu_2^{-1} R_t\big] \nabla R_t +{}\\[6pt]
{}+ \left(\hat X_t^2 + R_t\right) \nabla \nu_1 + \nu_2^{-2}R_t \nabla \nu_2,\ 
\nabla R_{t0}=0.
\end{array}
\right\}
\label{e5.6-s1}
\end{equation}

Уравнения~(\ref{e5.4-s1}) нелинейны относительно~$\hat X_t$ и~$R_t$, 
причем фильтр существует только при наличии шума в~наблюдениях  $\nu_2 \hm\ne 0$ 
и~произвольном шуме интенсивности~$\nu_1$ в~уравнении состояния. 
Уравнения~(\ref{e5.6-s1}) 
для первых функций чувствительности являются линейными неоднородными уравнениями 
вследствие зависимости $\nu_{1,2} \hm= \nu_{1,2} (\Theta)$ и~$Z_t\hm = Z_t (\Theta)$.

 В соответствии с~теоремой~3.1 (с~точ\-ностью до вероятностных моментов третьего порядка) 
 имеем следующие уравнения ОСОФ:
 \begin{equation}
 \left.
 \begin{array}{rl}
 \dot{\hat X}_t &=-\hat X_t \left(\hat X_t^2 + 3 R\right) + {}\\[6pt]
 &\hspace*{17mm}{}+\nu_2^{-1} R_t 
 \left(Z_t - \hat X_t\right) - c_3\,; %\label{e5.7-s1}
 \\[6pt]
\dot R_t &= \lk\nu_1 - 6 \left(\hat X_t^2 + R_t\right)\rk R_t - 
  \nu_2^{-1} R_t^2 - {}\\[6pt]
  &\hspace*{12mm}{}-
 6 \hat X_t c_3 + \nu_2^{-1} c_3 \left(Z_t - \hat X_t\right)\,; %\label{e5.8-s1}
 \\[6pt]
  \dot c_3 &= -18 \hat X_t R_t^2 - 9 \left(\hat X_t^2 + 3 R_t\right) 
  c_3+ {}\\[6pt]
& \hspace*{10mm} {}+
 3 \nu_1 \left( 2 \hat X_t R_t + c_3\right) - \fr{3}{2}\, \nu_2^{-1} c_3\,.
 \end{array}
 \right\}
 \label{e5.9-s1}
 \end{equation}
Уравнениям~(\ref{e5.9-s1}) отвечают следующие уравнения 
для первых функций чувствительности:
\begin{multline*}
\nabla \dot{\hat X}_t=- \left(3 \hat X_t^2 + 3 R_t + \nu_2^{-1} R_t\right) 
\nabla \hat X_t +{}\\
{}+\lk \nu_2^{-1} \left(Z_t -\hat X_t\right) - 3 \hat X_t)\rk \nabla R_t +
 \nu_2^{-1} R_t\nabla Z_t - {}\\
 {}-\nu_2^{-2} R_t 
\left(Z_t -\hat X_t\right) \nabla \nu_2 - \nabla c_3\,,\enskip 
\nabla \hat X(t_0)=0\,; %\label{e5.10-s1}
\end{multline*}

  
  \noindent
  \begin{multline*}
\nabla \dot R_t =-\left(12\hat X_t R_t + 6 c_3 + \nu_2^{-1} c_3\right) 
\nabla \hat X_t -{}\\
{}- \left(\nu_1+\nu_2^{-1} R_t + 6 \hat X_t^2 + 12 R_t\right) 
\nabla R_t+\nu_2^{-1} c_3 \nabla Z_t +{}\\
{}+  \lk \nu_2^{-1} \left(Z_t -\hat X_t\right) - 6 \hat X_t\rk \nabla c_3 + 
R_t\nabla \nu_1+ {}\\
{}+\nu_2^{-2} \lk R_t^2 - c_3 \left(Z_t-X_t\right)\rk
\nabla \nu_2\,,\enskip \nabla R(t_0)=0\,;
%\label{e5.11-s1}
\end{multline*}

 \vspace*{-12pt}
  
  \noindent
  \begin{multline*}
\nabla \dot c_3= - 9 \left(c_3 + 2 \hat X_t c_3 + 2 R_t^2\right) 
\nabla \hat X_t + {}\\
{}+3 \left(2 \nu_1 \hat X_t - 9 c_3 - 2 \hat X_t R_t\right) \nabla R_t+{}\\
{}+3 \left(\nu_1 + \fr{9}{2}\,\nu_2^{-1} - 3 \hat X_t - 9 R_t - 3 \hat X_t^2\right) 
\nabla c_3 + {}\\
{}+3 \left(2 \hat X_t R_t + c_3\right) \nabla \nu_1 + \fr{3}{2}\,
\nu_2^{-2} \nabla \nu_2\,,
\enskip \nabla c_3 \left(t_0\right) =0\,. %\label{e5.13-s1}
\end{multline*}
Уравнения точности этого ОСОФ нелинейны относительно~$\hat X_t$, $R_t$ и~$c_3$ 
и~справедливы только при  $\nu_2 \hm\ne 0$ и~произвольном~$\nu_1$.
Уравнения для первых функций чувствительности являются 
линейными неоднородными уравнениями вследствие зависимости $Z_t \hm= Z_t(\Theta)$, 
$\nu_{1,2}\hm = \nu_{1,2} (\Theta)$.

\vspace*{-3pt}

\section{Заключение}

Для нелинейных дифференциальных систем с~винеровскими и~пуассоновскими шумами 
в~уравнениях состояния и~винеровскими шумами в~уравнениях наблюдения, понимаемых 
в~смысле Ито (в~том числе и~на многообразиях), разработаны методы синтеза ОСОФ 
на основе аппроксимации апостериорного одномерного 
распределения МНА и~МСЛ, 
а~также МОР и~МКМ. Получены уравнения точности 
и~чувствительности  СОФ. Приведен тестовый пример для одномерной 
нелинейной сис\-те\-мы с~параметрическим шумом.
Алгоритмиче-\linebreak ское обеспечение положено в~основу инструментального программного 
обеспечения в~библиотеке <<StS--Filter>>.

В качестве дальнейших обобщений можно рассмотреть дискретные 
и~не\-пре\-рыв\-но-дис\-крет\-ные МСтС, 
а~также модифицированные СОФ на основе ненормированной 
апостериорной плотности. Развития требуют также методы экстраполяции и~интерполяции 
в~таких системах.

\vspace*{-3pt}

{\small\frenchspacing
 {%\baselineskip=10.8pt
 \addcontentsline{toc}{section}{References}
 \begin{thebibliography}{99}

\bibitem{1-s1}
\Au{Синицын И.\,Н. }
Аналитическое моделирование распределений на основе ортогональных разложений 
в~нелинейных стохастических системах на многообразиях~// 
Информатика и~её применения, 2015. Т.~9. Вып.~3. C.~17--24.

\bibitem{2-s1}
\Au{Синицын И.\,Н.}
Применение ортогональных разложений для аналитического моделирования многомерных 
распределений в~нелинейных стохастических системах на многообразиях~// 
Системы и~средства информатики, 2015. Т.~25. №\,3. С.~3--22.

\bibitem{3-s1}
\Au{Ватанабэ С., Икэда Н.} Стохастические дифференциальные уравнения 
и~диффузионные процессы~/ Пер. с~англ.~--- М.: Наука, 1986. 448~с.
(\Au{Watanabe~S, Ikeda~N.} 
Stochastic differential equations and diffusion processes.~--- 
Amsterdam\,--\,Oxford\,--\,New York: North-Holland Publishing Co.; 
Tokyo: Kodansha Ltd., 1981. 476~p.)

\bibitem{5-s1} %4
\Au{Королюк В.\,С.,
Портенко Н.\,И., Скороход~А.\,В., Турбин~А.\,Ф.}
Справочник по теории вероятностей и~математической статистике.~---
М.: Наука, 1985. 640~с.


\bibitem{4-s1} %5
\Au{Пугачёв В.\,С., Синицын~И.\,Н.}
Теория стохастических систем.~--- М.: Логос, 2000; 2004. 1000~с.
%(\Au{Pugachev~V.\,S., Sinitsyn~I.\,N.} Stochastic systems. Theory and  applications.~---
%Singapore: World Scientific, 2001. 908~p.)


\bibitem{7-s1} %6
\Au{Синицын И.\,Н.} 
Фильтры Калмана и~Пугачёва.~--- 2-е изд.~--- М.: Логос, 2007.
776~с.


\bibitem{6-s1} %7
 \Au{Пугачёв В.\,С., Синицын~И.\,Н.}
Стохастические дифференциальные системы. Анализ и~фильтрация.~--- М.:
Наука,  1990.  632~с. (\Au{Pugachev~V.\,S., Sinitsyn~I.\,N.}
Stochastic differential systems.
Analysis and filtering.~--- Chichester\,--\,New York, NY, USA: Jonh Wiley, 1987.
549~p.)



%\bibitem{8-s1} %8
%\Au{Wonham W.\,M.}
%Some applications of stochastic differential equations to optimal nonlinear 
%filtering~// J.~Soc. Ind. Appl. Math. Ser. A Control, 1964. Vol.~2. Iss.~3. P.~347--369.

\bibitem{9-s1}
\Au{Евланов А.\.Г., Константинов В.\,М. }
Системы со случайными параметрами.~--- М.: Наука, 1987. 568~с.

\bibitem{10-s1}
Справочник по теории автоматического управления~/ Под ред. А.\,А.~Красовского.~--- 
М.: Наука, 1987. 712~с.

\end{thebibliography}

 }
 }

\end{multicols}

\vspace*{-3pt}

\hfill{\small\textit{Поступила в~редакцию 29.10.15}}

\vspace*{10pt}

%\newpage

%\vspace*{-24pt}

\hrule

\vspace*{2pt}

\hrule

\vspace*{10pt}

\def\tit{ORTHOGONAL SUPOPTIMAL FILTERS FOR~NONLINEAR STOCHASTIC SYSTEMS~ON~MANIFOLDS}

\def\titkol{Orthogonal supoptimal filters for~nonlinear stochastic systems~on manifolds}

\def\aut{I.\,N.~Sinitsyn}

\def\autkol{I.\,N.~Sinitsyn}

\titel{\tit}{\aut}{\autkol}{\titkol}

\vspace*{-9pt}

\noindent
Institute of Informatics Problems, Federal Research Center 
``Computer Science and Control'' of the Russian Academy of Sciences,
44-2 Vavilov Str., Moscow 119333, Russian Federation

\def\leftfootline{\small{\textbf{\thepage}
\hfill INFORMATIKA I EE PRIMENENIYA~--- INFORMATICS AND
APPLICATIONS\ \ \ 2016\ \ \ volume~10\ \ \ issue\ 1}
}%
 \def\rightfootline{\small{INFORMATIKA I EE PRIMENENIYA~---
INFORMATICS AND APPLICATIONS\ \ \ 2016\ \ \ volume~10\ \ \ issue\ 1
\hfill \textbf{\thepage}}}

\vspace*{3pt}

\Abste{The authors developed the
  synthesis theory of suboptimal filers (SOF) based on normal approximation 
method (NAM), statistical linearization method (SLM), orthogonal expansions method (OEM), 
and quasi-moment method (QMM) for nonlinear differential stochastic systems on manifolds
(MStS) with Wiener and Poisson noises. Exact optimal (for mean square error criteria) 
equations 
for MStS with Gaussian noises in observation equations for the one-dimensional \textit{a~posteriori} 
characteristic function are derived. Problems of approximate solving of exact 
equations are 
discussed. Accuracy and sensitivity equations are presented. A~test example for 
the nonlinear scalar 
differential equation with additive and multiplicative noises is given. Some generalizations 
are mentioned.}

\KWE{\textit{a posteriori} one-dimensional distribution;
coefficient of orthogonal expansion;
first sensitivity function;
normal approximation method; normal suboptimal filter;
orthogonal expansion method; orthogonal suboptimal filter;
quasi-moment method; quasi-moment; statistical linearization method;
stochastic system on manifolds; suboptimal filter;
Wiener white noise}



\DOI{10.14357/19922264160103}

\Ack
\noindent
The research was supported by the Russian Foundation for Basic Research 
(project 15-07-002244).



%\vspace*{6pt}

  \begin{multicols}{2}

\renewcommand{\bibname}{\protect\rmfamily References}
%\renewcommand{\bibname}{\large\protect\rm References}



{\small\frenchspacing
 {%\baselineskip=10.8pt
 \addcontentsline{toc}{section}{References}
 \begin{thebibliography}{99}


\bibitem{1-s1-1}
\Aue{Sinitsyn, I.\,N.} 2015.
Analiticheskoe modelirovanie raspredeleniy na osnove ortogonal'nykh 
razlozheniy v~neli\-ney\-nykh stokhasticheskikh sistemakh na mnogo\-ob\-ra\-zi\-yakh 
[Analytical modeling in stochastic systems on manifolds based on orthogonal expansions].
\textit{Informatika i~ee Primeneniya}~--- \textit{Inform. Appl.}  9(2):17--24.

\bibitem{2-s1-1}
\Aue{Sinitsyn, I.\,N.} 2015.
Primenenie ortogonal'nykh raz\-lo\-zhe\-niy dlya analiticheskogo modelirovaniya mnogomernykh 
raspredeleniy v~nelineynykh stokhasticheskikh sistemakh na mnogoobraziyakh [Applications 
of orthogonal expansions for analytical modeling of multidimensional distributions in 
stochastic systems on manifolds]. \textit{Sistemy i~Sredstva Informatiki}~---
\textit{Systems and Means of Informatics} 25(3):3--22.

\bibitem{3-s1-1}
\Aue{Watanabe,~S., and N. Ikeda}. 1981. 
\textit{Stochastic differential equations and diffusion processes}. 
Amsterdam\,--\,Oxford\,--\,New York: North-Holland Publishing Co.; 
Tokyo: Kodansha Ltd. 476~p.

\bibitem{5-s1-1} %4
Korolyuk, V.\,S., N.\,I.~Portenko, A.\,V.~Skorokhod, and A.\,F.~Turbin, eds. 
1985.
\textit{Spravochnik po teorii veroyatnostey}
\textit{i~matematicheskoy statistike}
[Probability theory and\linebreak mathematical statistics: Handbook].
Moscow: Nauka. 640~p.


\bibitem{4-s1-1} %5
 \Aue{Pugachev, V.\,S., and I.\,N.~Sinitsyn.} 
 2001.  \textit{Stochastic systems. Theory and  applications}.
Singapore: World Scientific. 908~p.

\bibitem{7-s1-1} %6
\Aue{Sinitsyn, I.\,N.} 2007. \textit{Fil'try Kalmana i~Pugacheva} [Kalman and Pugachev
filters]. 2nd ed. Moscow: Logos.  776~p.


\bibitem{6-s1-1} %7
 \Aue{Pugachev, V.\,S., and I.\,N.~Sinitsyn.} 
1987. \textit{Stochastic differential systems.
Analysis and filtering}. Chichester\,--\,New York, NY: Jonh Wiley.
549~p.

%\bibitem{8-s1-1}
%\Aue{Wonham, W.\,M.} 1964.
%Some application of stochastic differential equations to optimal nonlinear filtering.
%\textit{J.~Soc. Ind. Appl. Math. Ser. A Control} 2(3):347--369.

\bibitem{9-s1-1}
\Aue{Evlanov, A.\,G., and V.\,M.~Konstantinov}. 1976.
\textit{Sistemy so slozhnymi parametrami} [Systems with random parameters]. 
Moscow: Nauka. 568~p.

\bibitem{10-s1-1}
Krasovskii, A.\,A., ed. 1987.
\textit{Spravochnik po teorii avtomaticheskogo upravleniya} 
[Handbook for automatic control].   Moscow: Nauka. 712~p.
\end{thebibliography}

 }
 }

\end{multicols}

\vspace*{-3pt}

\hfill{\small\textit{Received October 29, 2015}}

\Contrl

\noindent
\textbf{Sinitsyn Igor N.} (b.\ 1940)~---
Doctor of Science in technology, professor,
Honored scientist of RF, Head of Department, Institute of Informatics Problems, Federal Research Center ``Computer Science and
Control'' of the Russian Academy of Sciences, 44-2 Vavilov Str.,
Moscow 119333, Russian Federation; sinitsin@dol.ru


\label{end\stat}


\renewcommand{\bibname}{\protect\rm Литература}