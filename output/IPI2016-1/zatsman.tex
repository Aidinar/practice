\def\stat{zatsman}

\def\tit{ПРЕДСТАВЛЕНИЕ КРОССЯЗЫКОВЫХ ЗНАНИЙ 
О~КОННЕКТОРАХ В~НАДКОРПУСНЫХ БАЗАХ ДАННЫХ$^*$}

\def\titkol{Представление кроссязыковых знаний 
о~коннекторах в~надкорпусных базах данных}

\def\aut{И.\,М.~Зацман$^1$, О.\,Ю.~Инькова$^2$, М.\,Г.~Кружков$^3$, 
Н.\,А.~Попкова$^4$}

\def\autkol{И.\,М.~Зацман, О.\,Ю.~Инькова, М.\,Г.~Кружков, 
Н.\,А.~Попкова}

\titel{\tit}{\aut}{\autkol}{\titkol}

{\renewcommand{\thefootnote}{\fnsymbol{footnote}} \footnotetext[1]
{Работа выполнена в~Институте проб\-лем информатики Федерального исследовательского центра 
<<Информатика и~управ\-ле\-ние>> Российской академии наук при поддержке РФФИ
(проект 14-07-00785), РГНФ (проект 16-24-41002) и~ШННФ (проект IZLRZ1\_164059).}}


\renewcommand{\thefootnote}{\arabic{footnote}}
\footnotetext[1]{Институт проб\-лем информатики Федерального исследовательского центра 
<<Информатика и~управ\-ле\-ние>> Российской академии наук, izatsman@yandex.ru}
\footnotetext[2]{Институт проб\-лем информатики Федерального исследовательского центра 
<<Информатика и~управ\-ле\-ние>> Российской академии наук; 
Женевский университет, Olga.Inkova@unige.ch}
\footnotetext[3]{Институт проб\-лем информатики Федерального исследовательского центра 
<<Информатика и~управ\-ле\-ние>> Российской академии наук, magnit75@yandex.ru}
\footnotetext[4]{Институт проб\-лем информатики Федерального исследовательского центра 
<<Информатика и~управ\-ле\-ние>> Российской академии наук, natasha\_\_popkova@mail.ru}


   \Abst{Рассматриваются используемые в~контрастивных лингвистических 
исследованиях базы данных (БД), получившие название <<надкорпусных>>. Они 
формируются в~результате обработки текстов, хранящихся в~двуязычных 
параллельных подкорпусах Национального корпуса русского языка. В~них 
каждому тексту на русском языке соответствует один или несколько его 
переводов на другой язык, а~каждому тексту на иностранном языке~--- один 
его перевод на русский язык. Каждый текст на языке оригинала и~его 
переводы выровнены по предложениям. Надкорпусные БД (НБД) 
представляют собой новый вид лингвистических ресурсов, которые 
предназначены для целенаправленного извлечения новых знаний о~широком 
спектре языковых единиц (ЯЕ). Эти знания необходимы для повышения качества 
машинного перевода, актуализации моно- и~двуязычных грамматик, а~также 
для обновления многообразных образовательных курсов по лингвистике, 
теории и~практике перевода. В~статье дается описание концепции 
формирования НБД и~примера реализации такой базы 
для представления знаний о~коннекторах русского языка и~об их переводах на 
французский язык.}

   \KW{кроссязыковые знания; коннекторы русского языка; представление 
знаний о коннекторах; надкорпусные базы данных}

\DOI{10.14357/19922264160110} %

\vspace*{4pt}

\vskip 14pt plus 9pt minus 6pt

\thispagestyle{headings}

\begin{multicols}{2}

\label{st\stat}

\section{Введение}

\vspace*{-2pt}

  Создание электронных корпусов текстов позволяет значительно ускорить 
сбор исходных данных для проведения лингвистических исследований, 
включая выполнение запросов на поиск исследуемых языковых единиц~[1]. 
При решении задач контрастивной лингвистики используются параллельные 
корпуса, в~которых хранятся тексты на одном языке и~их переводы на другой 
язык, выровненные, как правило, по предложениям. Параллельные корпуса 
могут объединяться в~рамках более крупных хранилищ текстов. 
  
  Примером такого хранилища является Национальный корпус русского 
языка ({\sf http://ruscorpora.\linebreak ru}). На данный момент он содержит параллельные 
подкорпуса для английского, армянского, белорусского, болгарского, 
бурятского, испанского, итальянского, латышского, немецкого, польского, 
украинского, французского и~эстонского языков.
  
  Когда исследователь находит в~корпусе интересующие его объекты, 
у~него нередко возникает потребность в~создании для этих объектов 
формализованных описаний, которые могли бы значительно расширить 
данные, содержащиеся в~корпу\-се. При этом создание таких описаний иногда 
подразумевает формирование и~использование авторских сис\-тем 
классификации, которые отсутствуют в~корпусе. В~некоторых корпусах 
существует возможность экспорта данных, найденных в~результате  
лек\-си\-ко-грам\-ма\-ти\-че\-ско\-го поиска исследуемых языковых единиц, 
но их последующее хранение, авторское описание и~классификация не 
входят в~перечень функций корпусов. Более того, сама идея создания 
электронных корпусов текстов не предполагает реализации этих функций.
  
  В работах~[2--4] была описана технология формирования БД, 
которая существенно дополняет функциональность параллельных кор\-пу\-сов 
при решении задач контрастивной лингвистики. В~период 2013--2015~гг.\ 
эта БД использовалась для исследования глагольных форм русского 
языка в~зеркале их переводов на французский язык~[5]. В~ней реализован 
двуязычный лек\-си\-ко-грам\-ма\-ти\-ческий поиск, который позволяет 
находить интересующие пользователя глагольные формы русского\linebreak языка, 
задавая для них сочетания лексических и~грамматических признаков 
и~одновременно соответствующие им функционально эквивалентные 
фрагменты в~текстах на французском языке, а~так\-же получать информацию 
о~час\-тот\-ности моделей перевода этих глагольных форм ({\sf 
http://a179.ipi. ac.ru/corpora\_dynasty/main.aspx}).
  
  В стадии формирования находятся еще две БД~[6], 
предназначенные для контрастивного исследования лингвоспецифичных 
единиц (ЛСЕ) и~коннекторов русского языка; описание второй из них будет 
дано в~разд.~3. Опыт их применения показал, что, с~одной стороны, 
расширение функциональности параллельных корпусов необходимо для 
решения современных задач контрастивной лингвистики, а~с~другой стороны, 
три БД (глагольных форм, ЛСЕ и~коннекторов) пред\-став\-ля\-ют собой 
новый вид лингвистических информационных ресурсов, которые получили 
название <<надкорпусные БД>>~[7, 8].
  
  Основным логическим элементом НБД является установленное 
лингвистом переводное соответствие, т.\,е.\ двухместный кортеж, 
вклю\-ча\-ющий\linebreak языковую единицу в~оригинальном тексте и~ее функционально 
эквивалентный фрагмент\footnote{Термин <<функционально эквивалентный фрагмент>> 
был введен Д.\,О.~Добровольским в~его работах с~соавторами~[9, 10].} в~тексте перевода 
в~привязке к~контексту их употребления (такие кортежи далее будут 
называться двуязычными кортежами). Двуязычные кортежи формируются 
профессиональными лингвистами, которые анализируют и~сопоставляют 
выровненные параллельные тексты, загруженные в~НБД (см.\ 
подробнее~\cite{3-zat}). В~процессе их формирования нередко выявляются 
лакуны в~существующих сопоставительных грамматических описаниях. 
Например, лингвист может обнаружить некоторые переводные соответствия, 
которые отсутствуют в~этих описаниях, но которые дают важную 
дополнительную информацию о~семантике и~функционировании изучаемых 
языковых явлений. Возможность специфицирования таких новых для 
контрастивной лингвистики переводных соответствий с~по\-мощью 
двуязычных кортежей является характерной чертой НБД, отличающей ее от 
других видов лингвистических информационных ресурсов (параллельных 
корпусов, типологических БД и~т.\,д.~[7, 8]).
  
  Основная цель этой статьи состоит в~описании концепции формирования 
НБД как нового вида лингвистических информационных ресурсов, а~также 
примера ее реализации для представления в~НБД кроссязыковых знаний 
о~коннекторах русского языка и~об их переводах на французский язык.

\vspace*{-6pt}
  
\section{Основные термины и~концептуальные положения}
  
  В этом разделе приведены определения терминов из  
работ~\cite{3-zat, 4-zat}, предложенные Анной Зализняк и~адаптированные 
к~описанию концепции формирования НБД. В~этих работах для 
типологизации глагольных форм применяются лингвистические понятия 
<<базовый вид>> и~<<дополнительный признак>> формы. В~терминах 
информатики как ин\-фор\-ма\-ци\-он\-но-компью\-тер\-ной науки~[11, 12] 
применение этих понятий для типологизации представляет собой 
одновременное и~независимое использование двух классификаций по разным 
основаниям.
  
  Таким образом, по сути, применяется фасетная классификация глагольных 
форм, в~которой выделены одно главное и~одно дополнительное основания. 
Рубрика, присвоенная языковой единице по главному основанию, в~НБД 
глагольных конструкций названа базовым видом, а~в~НБД коннекторов~--- 
речевой реализацией. По дополнительному основанию ЯЕ
могут быть присвоены ноль, одна или несколько руб\-рик, значения которых 
были названы ее дополнительными признаками во всех НБД. Отметим, что 
названия руб\-рик фасетной классификации (ФК) выбираются лингвистами. От 
этого выбора не зависит функционирование НБД. Использование ФК с~двумя 
основаниями иллюстрируется в~статье на примере формирования НБД 
коннекторов.
  
  Отметим три важных свойства ФК, которая применяется для 
категоризации языковых единиц в~процессе проведения контрастивных 
лингвистических исследований. Во-пер\-вых, по главному основанию может 
быть присвоена только одна рубри\-ка. Во-вто\-рых, по дополнительному 
основанию может быть присвоено любое число руб\-рик. В-третьих, эта 
классификация является темпоральной, т.\,е.\ чис\-ло уровней классификации 
по каждому основанию, руб\-рик каждого уровня, их названия и~дефиниции 
могут изменяться во времени. Это, с~одной стороны, позволяет 
удовлетворить требование адап\-ти\-ру\-емости НБД к~видам исследуемых 
ЕЯ и~специфике решаемых лингвистических задач, с~другой 
стороны, решать задачи представления в~НБД сис\-те\-мы кроссязыковых 
знаний, которые формируются и~пополняются в~процессе проведения 
контрастивных исследований.



  
  Принимая во внимание требование адапти-\linebreak руемости НБД в~качестве 
исходного, сформулируем первое положение концепции формирова-\linebreak ния 
НБД.
  
  \smallskip
  
  \noindent
  \textbf{Положение~1.}\ \textit{Должна быть обеспечена возможность 
формировать и~редактировать ФК, используемую 
в~НБД для описания ис\-сле\-ду\-емых ЯЕ, которая включает одно 
главное основание и~одно дополнительное.}
  
\vspace*{12pt}

\noindent
{{\tablename~1}\ \ \small{Названия девяти рубрик первого уровня по главному основанию и~число рубрик 
на втором уровне}}
 
% \vspace*{19pt}

{\small
\begin{center}
\tabcolsep=7.4pt
\begin{tabular}{|c|l|c|}
\hline
№&\multicolumn{1}{c|}{
\tabcolsep=0pt\begin{tabular}{c}Рубрика\\ первого уровня\\ по главному основанию\end{tabular}}&
\tabcolsep=0pt\begin{tabular}{c}Число рубрик\\ второго  уровня\end{tabular}\\
\hline
1&при&5\\
2&иначе&2\\
3&в\_связи&3\\
4&прочие&21\hphantom{9}\\
5&сочетания&6\\
6&только&3\\
7&то CNT part$^*$&2\\
8&кстати&4\\
9&не то чтобы&6\\
\hline
\multicolumn{3}{p{76mm}}{\footnotesize \hspace*{3mm}$^*$Рубрика <<то CNT part>> используется в~тех 
случаях, когда слово \textit{то} является элементом многокомпонентного коннектора, например 
<<\textit{Не то что к~Соне, 
 а~к~Дуне пойдешь!}>> (\Au{Ф.\,М.~Достоевский}. <<Преступление и~наказание>>).}\\
\end{tabular}
\end{center}
}

\addtocounter{table}{1}

\vspace*{6pt}
  
  Для иллюстрации этого положения приведем краткое описание ФК 
с~одним главным и~одним дополнительным основаниями, используемой 
в~НБД коннекторов русского языка. Для ее главного основания определены 
два уровня классификации. На первом уровне используется~9~рубрик по 
состоянию на 1~января 2016~г.\ (табл.~1).

 В~табл.~2 приведены 
названия шести руб\-рик второго уровня для 9-й руб\-ри\-ки первого уровня <<не 
то чтобы>> и~число двуязычных кортежей, сформированных по каждой из 
этих руб\-рик на конец каждого квартала 2015~г.




  Рубрика <<прочие>> (см.\ строку №\,4 в~табл.~1) включает те коннекторы, 
для которых на 1~января 2016~г.\ не была проведена классификация по 
восьми другим рубрикам первого уровня главного
 основания. В~результате 
анализа содержания этой рубрики в~дальнейшем могут добавляться новые 
рубрики первого уровня или редактироваться существующие. Начиная 
с~2016~г.\ помимо изменения чис\-ла руб\-рик могут редактироваться их названия 
и~дефиниции.
  

  
  Приведем пример предложения с~коннектором, который относится 
к~рубрике второго уровня <<не то что $\langle$расстояние$\rangle$~а>> и~его перевод 
на французский язык, кроме текста в~квад\-рат\-ных скобках.
  
  \smallskip
  
  \noindent
  \textbf{Пример 1} [\Au{И.\,А.~Гончаров}. <<Обломов>>.]:
  
  <<[--- Как: ты бы и~мне не поверил?]
  
  --- Ни за что; \textbf{не то что} тебе, \textbf{а} все может случиться: [ну, как лопнет, 
вот я и~без гроша]>>;
  
  <<Pour rien au monde \textbf{non que} je ne te fasse confiance \textbf{mais} tout peut 
arriver>>.
  
  \smallskip
  
  В этом примере в~процессе построения двуязычного кортежа для 
коннектора \textit{не то что}\ldots, \textit{а}\linebreak
 лингвист сначала определяет 
в~переводе его функционально эквивалентный фрагмент 
(ФЭФ)\footnote{В~этом примере ФЭФ~--- это слова \textit{non que$\ldots$ mais} в~тексте 
перевода, которые, по мнению лингвиста, являются переводным соответствием для коннектора 
\textit{не то что$\ldots$,~а}.} и~контекст ФЭФ. Затем путем объединения 
коннектора \textit{не то что}\ldots, \textit{а} и~его ФЭФ \textit{non que$\ldots$ 
mais} в~НБД строится двуязычный кортеж типа $\langle$\textit{не то что$\ldots$, 
а}; \textit{non que$\ldots$\ mais}$\rangle$.
  
  Кроме главного основания ФК для ЯЕ (ФК ЯЕ) в~НБД 
коннекторов есть еще дополнительное. По состоянию на~1~января 2016~г.\ 
для него также определены два уровня классификации, включая~6~рубрик 
первого уровня (табл.~3).


\setcounter{table}{1}
\begin{table*}\small %tabl2
%\vspace*{-12pt}
\begin{center}
\parbox{410pt}
{\Caption{Названия шести рубрик второго уровня для рубрики первого уровня <<не то 
чтобы>> и~чис\-ло двуязычных кортежей на конец каждого квартала 2015~г.}

}

\vspace*{2ex}

\begin{tabular}{|c|l|c|c|c|c|}
\hline
№&\multicolumn{1}{c|}{Рубрика} &\multicolumn{4}{c|}{Число кортежей в~НБД}\\
\cline{3-6}
 & \multicolumn{1}{c|}{второго уровня} &на 31.03.2015&
 на 30.06.2015&
на 30.09.2015&
на  31.12.2015\\
\hline
1&не то что&0&0&4&24\hphantom{9}\\
2&не то что $\langle$расстояние$\rangle$~а&0&0&1&11\hphantom{9}\\
3&не то что $\langle$расстояние$\rangle$ но&0&0&0&5\\
4&не то чтобы&0&0&7&8\\
5&не то чтобы $\langle$расстояние$\rangle$ а&0&0&5&5\\
6&не то чтобы $\langle$расстояние$\rangle$ но&0&1&1&1\\
\hline
\end{tabular}
\end{center}
\vspace*{-12pt}
\end{table*}

\setcounter{table}{3}
  \begin{table*}[b]\small %4
  \vspace*{-12pt}
  \begin{center}
  \Caption{Названия десяти рубрик второго уровня для рубрики <<Отношения>> 
и~число двуязычных кортежей}
  \vspace*{2ex}
  
  \begin{tabular}{|c|l|c|c|c|c|}
  \hline
\multicolumn{1}{|c|}{\raisebox{-6pt}[0pt][0pt]{№}}&\multicolumn{1}{c|}{Рубрика второго уровня} & \multicolumn{4}{c|}{Число кортежей в~НБД}\\
\cline{3-6}
&\multicolumn{1}{c|}{по дополнительному основанию}&
на 31.03.2015&
на 30.06.2015&
 на  30.09.2015&
на 31.12.2015\\
\hline
1&Временн$\acute{\mbox{ы}}$е отношения&0&0&0&\hphantom{9}0\\
2&Отношение подлежит определению&38\hphantom{9}&78\hphantom{9}&365\hphantom{99}&381\hphantom{9}\\
3&Отношения замещения&0&3&15\hphantom{9}&35\\
4&Присоединительные отношения&0&0&0&\hphantom{9}0\\
5&Противительные отношения&4&8&8&\hphantom{9}8\\
6&Соединительные отношения&38\hphantom{9}&38\hphantom{9}&38\hphantom{9}&38\\
7&Сопоставительные отношения&4&4&4&\hphantom{9}4\\
8&Сравнительные отношения&0&0&0&\hphantom{9}0\\
9&Условные отношения&0&0&0&\hphantom{9}0\\
10\hphantom{9}&Уступительные отношения&8&8&8&\hphantom{9}8\\
\hline
\end{tabular}
\end{center}
\end{table*}


  

  В табл.~4 приведены названия~10~рубрик вто\-рого уровня для рубрики 
первого уровня <<Отно\-шения>>, включая название особой рубрики 
<<отношение подлежит определению>>, а~так\-же число двуязыч\-ных 
кортежей, сформированных по каждой из этих рубрик на конец каждого 
квартала 2015~г.
  

  Для построения двуязычных кортежей на основе параллельных текстов 
лингвисты должны про\-водить анализ исследуемой ЯЕ
оригиналь\-ного\linebreak текс\-та, идентифицировать и~классифицировать 
соответствующий ей ФЭФ с~использованием текста
 перевода. При этом для 
классификации ФЭФ создается еще одна темпоральная ФК для ФЭФ (ФК 
ФЭФ). Анализ приведенного примера позволяет сформулировать следующее 
положение концепции формирования НБД.


  
  \smallskip
  
  \noindent
  \textbf{Положение~2.}\ \textit{Должна быть обеспечена возможность 
загружать и~использовать параллельные тексты, содержащие исследуемые 
ЯЕ и~их ФЭФ в~тексте перевода, которые 
идентифицируются и~классифицируются с~использованием ФК ФЭФ.}
  
  \smallskip
  
  Для иллюстрации этого положения приведем описание ФК ФЭФ, которая 
по состоянию на 1~января 2016~г.\ включала два уровня по главному 
основанию. Первый уровень включает три рубрики:
  \begin{enumerate}[(1)]
\item <<CNT>> для ФЭФ, которые являются коннекторами французского 
языка;
\item <<TBD (прочие)>> для ФЭФ, рубрика которых временно не 
определена;
\item <<Zero>> для отсутствующих ФЭФ (к этой рубрике относятся те 
ситуации, когда отсутствует перевод исследуемой ЯЕ).
\end{enumerate}

%\vspace*{12pt}

\noindent
{{\tablename~3}\ \ \small{Названия шести руб\-рик первого уровня по дополнительному 
основанию и~чис\-ло  руб\-рик на втором уровне}}
 
% \vspace*{19pt}

{\small
\begin{center}
\tabcolsep=9pt
\begin{tabular}{|c|l|c|}
\hline
№&\multicolumn{1}{c|}{\tabcolsep=0pt\begin{tabular}{c}Рубрика первого уровня\\
 по дополнительному\\ основанию\end{tabular}}&
 \tabcolsep=0pt\begin{tabular}{c}Число рубрик \\ второго уровня\end{tabular} \\
\hline
1&Отношения&10\hphantom{9}\\
2&Структура&7\\
3&Позиция&3\\
4&Порядок&3\\
5&Статус&6\\
6&Расположение&2\\
\hline
\end{tabular}
\end{center}
}

\addtocounter{table}{1}

\vspace*{6pt}

  
 
  В НБД коннекторов ФК ЯЕ и~ФК ФЭФ сейчас полностью совпадают по 
дополнительному основанию. Однако в~общем случае такое совпадение 
может отсутствовать~--- лингвисты при необходи\-мости могут формировать 
разные ФК ЯЕ и~ФК ФЭФ по дополнительному основанию. Например, в~НБД 
глагольных форм классификации по дополнительному основанию в~ФК ЯЕ 
и~ФК ФЭФ не совпадают (см.\ табл.~3 и~4 в~работе~\cite{3-zat}).
  
  Возможность формировать и~редактировать ФК ЯЕ зафиксирована 
в~первом концептуальном положении. Приведем аналогичное положение 
для ФК ФЭФ.
  
  \smallskip
  
  \noindent
  \textbf{Положение~3.}\ \textit{Должна быть обеспечена возможность 
формировать и~редактировать ФК ФЭФ, используемую в~НБД для описания 
и классификации ФЭФ, которая включает одно главное основание и~одно 
дополнительное. В общем случае ФК ЯЕ и~ФК ФЭФ по дополнительному 
основанию не совпадают и~могут создаваться независимо друг от друга.}
  
  \smallskip
  
  Число двуязычных кортежей, построенных в~НБД глагольных форм, 
сейчас превышает десять тысяч. Формирование ФК ЯЕ и~ФК ФЭФ дает 
возможность реализовать не только лексический, но и~грамматический поиск 
кортежей, используя их руб\-ри\-ки, причем для решения задач контрастивной 
лингвистики часто необходим двуязычный лек\-си\-ко-грам\-ма\-ти\-че\-ский 
поиск. Он дает возможность одновременно задать в~запросе поисковые 
признаки для исследуемой ЯЕ на одном языке и~для ее ФЭФ на другом. 
Такая поисковая возможность редко присутствует в~параллельных корпусах. 
Опишем эту возможность в~виде отдельного концептуального положения.
  
  \smallskip
  
  \noindent
  \textbf{Положение~4.}\ \textit{В НБД должна быть предусмотрена 
возможность формировать двуязычные кортежи и~задавать  
лек\-си\-ко-грам\-ма\-ти\-че\-ские критерии их поиска, используя 
одновременно слова, а~также рубрики ФК для исследуемой ЯЕ и~ФК ФЭФ 
для ее перевода, включая случаи его отсутствия.}
  
  \smallskip
  
  В НБД каждому тексту на русском языке может соответствовать несколько 
его переводов на другой язык, а каждому тексту на иностранном языке~--- 
несколько его переводов на русский язык. Вернемся к~примеру~1 
с~предложением, содержащим коннектор \textit{не то что$\ldots$,~а}. Для 
этого предложения в~НБД есть еще один перевод, так как в~НБД коннекторов 
есть доступ к~текстам двух переводов этого произведения.

\smallskip
  
\noindent
  \textbf{Пример~2} со вторым переводом текста на русском языке из примера~1 на 
французский язык [\Au{И.\,А.~Гончаров}. <<Обломов>>]:
  
  <<$\ldots$Aucunement. Et \textbf{ce n'est pas}, je t'assure, \textbf{parce que} c'est toi, 
\textbf{mais} je pense {\ptb{\`{a}}} tout ce qui peut se passer$\ldots$>> (ФЭФ \textit{ce n'est 
pas$\ldots$\ parce que$\ldots$, mais}).
  
  Для второго перевода был построен еще один двуязычный кортеж типа 
$\langle$\textit{не то что$\ldots$, а}; \textit{ce n'est pas$\ldots$\ parce que$\ldots$, 
mais}$\rangle$. В~этом случае недостаточно иметь возможность  
лек\-си\-ко-грам\-ма\-ти\-че\-ско\-го поиска для одного варианта перевода. 
Нужно также учесть случаи, когда один текст имеет несколько переводных 
вариантов.
  
  \smallskip
  
  \noindent
  \textbf{Положение~5.}\ \textit{При использовании НБД должна быть 
обеспечена возможность формировать описания и~классифицировать 
несколько вариантов перевода одной и~той же ЯЕ; должен 
быть предусмотрен лек\-си\-ко-грам\-ма\-ти\-че\-ский поиск 
с~использованием одновременно лексики, руб\-рик ФК исследуемой ЯЕ и~ФК 
ФЭФ для нескольких вариантов ее перевода.}
  
  \smallskip
  
  Приведенные в~этом разделе концептуальные положения формирования 
НБД являются основой для разработки ее концептуальной и~логической схем 
данных, а~также информационной технологии создания НБД и~их 
применения для решения задач контрастивной лингвистики по исследованию 
широкого спект\-ра ЯЕ, в~частности коннекторов русского 
языка.

\vspace*{-9pt}
  
\section{Надкорпусные базы данных коннекторов и~категории двуязычных кортежей}

  Выровненные параллельные тексты являются потенциальным источником 
новых, но труднодоступных кроссязыковых знаний. Тексты переводов, 
выровненные относительно оригинальных текстов, являются уникальным 
и~постоянно пополняемым источником таких знаний. Однако для их 
извлечения необходимо существенно дополнить функциональность 
параллельных корпусов, что и~предлагается делать с~помощью НБД, 
используя методы и~модели теории генерации  
знаний~[5, 13--16]. Согласно концепции 
формирования НБД, основные положения которой были рассмотрены 
в~предыдущем разделе, для описания новых знаний предназначены 
следующие четыре компонента НБД: ФК исследуемых ЯЕ, ФК их ФЭФ, 
двуязычные кортежи и~система связей между первыми тремя компонентами. 
В~данном разделе дано описание этой сис\-те\-мы и~предлагается подход 
к~категоризации двуязычных кортежей на примере НБД коннекторов, 
структура которых и~некоторые аспекты функционирования на данный 
момент малоизучены.
  
Коннектор~--- это <<языковая единица, функция которой состоит 
в~выражении определенного типа 
отношений~---  
ло\-ги\-ко-се\-ман\-ти\-че\-ских, иллокутивных\footnote{То есть устанавливаемых 
на уровне речевых 
актов. Так, в~высказывании \textit{Он не пришел, потому что заболел} коннектор \textit{потому 
что} устанавливает ло\-ги\-ко-се\-ман\-ти\-че\-ское отношение причины 
на уровне двух ситуаций, 
описанных в~главном и~в придаточном предложениях. 
А~в высказывании \textit{Позвони ему, 
пожалуйста, потому что он может обидеться} тот же коннектор устанавливает то же 
отношение, но на уровне речевого акта, совершенного в~главном предложении (я тебя 
\textbf{прошу} позвонить), и~вводит причину этой просьбы (иначе он обидится).}, 
структурных,~--- 
существующих между двумя соединенными с~ее помощью компонентами, имеющими предикативный 
характер, выраженными имплицитно или эксплицитно>>~\cite{17-zat}.  
Лек\-си\-ко-грам\-ма\-ти\-че\-ская природа коннекторов и~их структура разнообразны. 
Коннектор может 
состоять из одного (\textit{хотя, притом, кстати}) и~большего числа слов 
(\textit{не только$\ldots$, но~и; 
скорее$\ldots$, чем}$\ldots$); первые являются по своему составу 
однокомпонентными (пример~3),  вторые~--- многокомпонентными (примеры~1 и~4). 

\smallskip

  \noindent
  \textbf{Пример~3.} [\Au{И.\,А.~Гончаров}. <<Обломов>>]:
  
  Нет, у него чернильница полна чернил, на столе лежат письма, бумага, даже гербовая, 
\textbf{притом} исписанная его рукой.
  
  \smallskip
  
  \noindent
  \textbf{Пример~4.}\ [\Au{И.\,А.~Гончаров}. <<Обломов>>]: 
  
  Со времени смерти стариков хозяйственные дела в~деревне \textbf{не только} не 
улучшились, \textbf{но}, как видно из письма старосты, становились хуже.
  
  С лингвистической точки зрения интерес представляют как 
многокомпонентные коннекторы в~целом, так и~их составные части; этот 
последний аспект особенно важен для изучения структуры 
коннекторов~\cite{18-zat}. Поэтому основным принципом формирования 
НБД коннекторов является сле\-ду\-ющий: многокомпонентный коннектор как 
ЯЕ описывается с~по\-мощью двуязычных кортежей сначала 
в~целом, а~затем описываются по от\-дель\-ности составляющие его 
компоненты. На всех этапах описания у~исследователя есть возможность 
приписать левому и~правому элементам сформированного двуязычного 
кортежа:
  \begin{itemize}
\item по одной рубрике ФК ЯЕ и~ФК ФЭФ по главному основанию;
\item ноль, одну или несколько рубрик ФК ЯЕ и~ФК ФЭФ по 
дополнительному основанию этих классификаций.
\end{itemize}

  Таким образом, традиционные и~новые знания о~структуре коннекторов 
и~их функциях в~тексте представлены в~НБД в~виде двуязычных кортежей, 
т.\,е.\ упорядоченных пар, состоящих из исследуемой ЯЕ
и~ее ФЭФ, рубрик эволюционирующих ФК ЯЕ и~ФК ФЭФ, а~так\-же сис\-те\-мы 
связей между элементами кортежей и~руб\-ри\-ка\-ми. Для иллюстрации 
приведем еще один пример двуязычного кортежа, построенного в~НБД для 
коннектора \textit{не только$\ldots$, но} из примера~4 и~сле\-ду\-юще\-го его 
перевода на французский язык:
  
  \smallskip
  
  \noindent
  \textbf{Пример~5}\ с~переводом текста на русском языке из примера~4 на 
французский язык [\Au{И.\,А.~Гончаров}. <<Обломов>>]:
  
  Depuis la mort des parents, les affaires du domaine \textbf{non seulement} ne 
s'am$\acute{\mbox{e}}$lioraient pas, \textbf{mais}, {\ptb{\`{a}}} en croire la lettre du 
r$\acute{\mbox{e}}$gisseur, empiraient.
  
  Для этого коннектора и~его ФЭФ в~предложении из примера~5 построен 
двуязычный кортеж типа $\langle$\textit{не только$\ldots$, но}; \textit{non 
seulement$\ldots$ mais}$\rangle$. В~НБД коннекторов этот кортеж сопровождается 
рубриками ФК ЯЕ и~ФК ФЭФ, которые будут описаны в~следующем разделе.
  
  В завершение этого раздела опишем кратко~9~категорий двуязычных 
кортежей, формиру\-емых в~НБД коннекторов. В~процессе сопостави\-тельного 
анализа параллельных русских текстов и~их французских переводов, 
полученных из Наци\-о\-нального корпуса русского языка (НКРЯ), в~настоящее 
время лингвистами формируются рус\-ско-фран\-цуз\-ские соответствия в~виде 
трех из девяти\linebreak тео\-ре\-ти\-че\-ски определенных категорий. Следу\-ющие три 
категории используются в~тех случаях, когда некоторому текс\-ту 
соответствует только один перевод:
  \begin{enumerate}[(1)]
\item $\langle$c, Fc$\rangle$, где c~--- коннектор русского языка; Fc~--- его 
ФЭФ на французском языке;
\item $\langle$b, Fb$\rangle$, где b~--- блок коннектора (его неэлементарная 
составная часть); Fb~--- его ФЭФ;
\item $\langle$e, Fe$\rangle$, где e~--- элемент коннектора (его элементарная 
составная часть); Fe~-- его ФЭФ.
\end{enumerate}

  В процессе описания каждого многокомпонентного коннектора могут 
использоваться двуязычные кортежи всех трех перечисленных категорий, но 
необходимость их построения и~количество кортежей определяются 
лингвистом. Обязательным является только построение кортежа первой 
категории, т.\,е.\ для всего многокомпонентного коннектора.
  
  Приведем пример коннектора и~его ФЭФ, для описания структуры 
которого может понадобиться использование двуязычных кортежей 
нескольких категорий.
  
  \smallskip
  
  \noindent
   \textbf{Пример 6} [\Au{И.\,А.~Гончаров}. <<Обломов>.]:
   
  $\ldots$мягкость, которая была господствующим и~основ\-ным выражением, \textbf{не} 
лица \textbf{только}, \textbf{а}~всей \mbox{души};
  
  $\ldots$l'expression de douceur, qui dominait \textbf{non seulement} le visage \textbf{mais 
aussi} l'$\hat{\mbox{a}}$me.
  
  Для многокомпонентного коннектора из этого примера \textit{не\ $\ldots$ 
\  только,~а} могут быть построены кортежи всех трех категорий, поскольку 
он состоит из разложимого блока \textit{не только}, для которого может быть 
построен кортеж второй категории, и~из элемента коннектора~\textit{а}, для 
которого может быть построен кортеж третьей категории (также при 
необходимости могут быть построены кортежи для элементов, составляющих 
блок \textit{не только}).
  
  Для текстов, имеющих несколько вариантов перевода, после построения 
лингвистом двуязычных кортежей первых трех категорий автоматически, 
т.\,е.\ программным способом, без участия лингвистов, могут быть 
сформированы рус\-ско-фран\-цуз\-ские кортежи еще трех категорий, 
в~которых второй компонент является множеством ФЭФ из разных 
переводов одного и~того же текста:
  \begin{enumerate}[(1)]
  \setcounter{enumi}{3}
\item $\langle$c, \{Fc$_1$, \ldots, Fc$_n$\}$\rangle$, где c~--- коннектор русского 
языка; Fc$_1$~--- его ФЭФ в~первом переводе; $\ldots$; Fc$_n$~--- его ФЭФ 
в $n$-м переводе;
\item $\langle$b, \{Fb$_1$, \ldots, Fb$_n$\}$\rangle$, где b~--- блок коннектора; 
Fb$_1$~--- его ФЭФ в~первом переводе; $\ldots$; Fb$_n$~--- его ФЭФ 
в~$n$-м переводе;
\item $\langle$e, \{Fe$_1$, $\ldots$, Fe$_n$\}$\rangle$, где e~--- неделимый элемент 
коннектора; Fe$_1$~--- его ФЭФ в~первом переводе; $\ldots$; Fe$_n$~--- его 
ФЭФ в~$n$-м переводе.
  \end{enumerate}
  
  \setcounter{table}{4}
  
    \begin{table*}[b]\small %tabl5
    \vspace*{-9pt}
\begin{center}
\Caption{Первые четыре пары выровненных предложений, найденных в~НБД 
коннекторов по сочетанию слов <<не только>>}
\vspace*{2ex}

  \begin{tabular}{|p{79mm}|p{79mm}|}
  \hline
\multicolumn{1}{|c|}{Оригинальный текст}&\multicolumn{1}{c|}{Перевод}\\
\hline
Захар не старался изменить \textbf{не только} данного ему Богом образа, но и~своего костюма, в~
котором ходил в~деревне.&Zakhar n'avait rien fait pour changer l'apparence que Dieu lui avait 
donn$\acute{\mbox{e}}$e ni le costume qu'il avait port$\acute{\mbox{e}}$ {\ptb{\`{a}}} la campagne.\\
\hline
Он его представлял себе чем-то вроде второго отца, который только и~дышит тем, как бы за дело 
и не за дело, сплошь да рядом, награждать своих подчиненных и~заботиться \textbf{не только} о 
их нуждах, но и~об удовольствиях.&Il se l'imaginait comme une sorte de second p{\ptb{\`{e}}}re qui ne 
pensait qu'{\ptb{\`{a}}} distribuer des primes {\ptb{\`{a}}} ses employ$\acute{\mbox{e}}$s, qu'ils le 
m$\acute{\mbox{e}}$ritent ou non, {\ptb{\`{a}}} tort et {\ptb{\`{a}}} travers, et qu'{\ptb{\`{a}}} veiller 
non seulement {\ptb{\`{a}}} leurs besoins mais aussi {\ptb{\`{a}}} leur bien-$\hat{\mbox{e}}$tre.\\
\hline
Это происходило, как заметил Обломов впоследствии, оттого, что есть такие начальники, которые 
в испуганном до одурения лице подчиненного, выскочившего к~ним навстречу, видят \textbf{не 
только} почтение к~себе, но даже ревность, а иногда и~способности к~службе.&Comme Oblomov le 
remarqua plus tard, la cause en $\acute{\mbox{e}}$tait que certains sup$\acute{\mbox{e}}$rieurs 
voyaient dans la mine effray$\acute{\mbox{e}}$e d'un employ$\acute{\mbox{e}}$ qui s'empressait 
{\ptb{\`{a}}} leur rencontre, non seulement une preuve de respect pour eux, mais aussi un signe de 
z$\acute{\mbox{e}}$le et m$\hat{\mbox{e}}$me d'aptitude au service.\\
\hline
Со времени смерти стариков хозяйственные дела в~деревне \textbf{не только} не улучшились, но, 
как видно из письма старосты, становились хуже.&Depuis la mort des parents, les affaires du 
domaine non seulement ne s'am$\acute{\mbox{e}}$lioraient pas, mais, {\ptb{\`{a}}} en croire la lettre 
du r$\acute{\mbox{e}}$gisseur, empiraient.\\
\hline
\end{tabular}
\end{center}
%  \end{table*}
%  \begin{table*}\small %tabl6
\vspace*{-2pt}
\begin{center}
\Caption{Результаты рубрицирования левого и~правого элементов двуязычного кортежа 
типа $\langle$\textit{не только$\ldots$, но; non seulement$\ldots$ mais}$\rangle$}
\vspace*{2ex}

\begin{tabular}{|p{44mm}|p{30mm}|p{44mm}|p{30mm}|}
\hline
\multicolumn{2}{|c|}{Оригинальный текст} & \multicolumn{2}{c|}{Перевод}\\
\hline
Со времени смерти стариков хозяйственные дела в~деревне \textbf{не только} не улучшились, 
\textbf{но}, как видно из письма старосты, становились хуже. 
&\textbf{не}$\|$\textbf{только}$\|$\textbf{но} \newline  
$\langle$~TBD~$\rangle$\newline 
$\langle$~CNT p CNT q~$\rangle$\newline
$\langle$~CNT~$\rangle$\newline
$\langle$~Дистант~$\rangle$
&Depuis la mort des parents, les affaires du domaine \textbf{non seulement} ne 
s'am$\acute{\mbox{e}}$lioraient pas, \textbf{mais}, {\ptb{\`{a}}} en croire la lettre du 
r$\acute{\mbox{e}}$gisseur, empiraient.&\textbf{non seulement}$\|$\textbf{mais}\newline
$\langle$~TBD~$\rangle$\newline 
$\langle$~CNT p CNT q~$\rangle$\newline
$\langle$~CNT~$\rangle$\newline
$\langle$~Дистант~$\rangle$\\
\hline
\end{tabular}
\end{center}
\end{table*}
  
  В перспективе также планируется провести сопоставительный анализ 
параллельных текстов, состоящих из выровненных по предложениям 
оригинальных произведений на французском языке и~их переводов на 
русский, полученных на основе соответствующего подкорпуса НКРЯ. Это 
даст возможность сформировать в~НБД коннекторов  
фран\-цуз\-ско-рус\-ские кортежи еще трех категорий:
  \begin{enumerate}[(1)]
    \setcounter{enumi}{6}
\item $\langle$Sc, c$\rangle$, где c~--- многокомпонентный коннектор русского 
языка; Sc~--- <<стимул>> французского языка, обусло\-вив\-ший 
появление этого коннектора в~переводе на русский язык;
\item $\langle$Sb, b$\rangle$, где b~--- блок коннектора русского языка; Sb~--- 
<<стимул>> французского языка, обусло\-вив\-ший появление этого 
блока коннектора в~переводе на русский язык;
\item $\langle$Se, e$\rangle$, где e~--- неделимый элемент коннектора русского 
языка; Se~--- <<стимул>> французского языка, обусло\-вив\-ший 
появление этого элемента коннектора в~переводе на русский язык.\\[-13pt]
\end{enumerate}

  Одновременное формализованное описание в~НБД не только всего 
многокомпонентного коннектора как единой ЯЕ, но и~его 
со\-став\-ля\-ющих позволяет, с~одной стороны, фиксировать все возможные 
блоки и~элементы, составляющие коннекторы, характеризующиеся высокой 
сте\-пенью вариативности, а~с~другой стороны, даст возможность 
отслеживать, в~какие многокомпонентные коннекторы может входить тот 
или иной блок или его элементы.

\vspace*{-8pt}

\section{Рубрицирование коннекторов и~их функционально эквивалентных фрагментов}

\vspace*{-2pt}

  Приведенные ранее примеры построения двуязычных кортежей 
предполагают, что предварительно был выполнен поиск в~НБД по шаблонам, 
которые формируются для каждого исследуемого коннектора. В~результате 
поиска по сформированному шаблону из заданного массива параллельных 
текстов отбираются все пары предложений, удовлетворяющие критерию 
поиска (см.\ табл.~5 с~первыми четырьмя парами выровненных 
предложений, найденных в~НБД коннекторов по сочетанию <<не только>>; 
для четвертой пары с~использованием текстов примеров~4 и~5 был построен 
двуязычный кортеж типа $\langle$\textit{не только$\ldots$, но; non 
seulement$\ldots$ mais}$\rangle$).
  

  
  Лингвист анализирует каждую найденную пару предложений 
в~отдельности, отмечает все со\-став\-ля\-ющие коннектора и~его контекст, 
а~также ФЭФ и~его контекст. Коннектор русского языка и~его ФЭФ во 
французском тексте~--- это левый и~правый элементы формируемого 
двуязычного кортежа соответственно.
  
  Как отмечалось ранее, для рубрицирования коннекторов (левый элемент 
двуязычных кортежей) и~их ФЭФ (правый элемент) используются две 
в~общем случае независимые системы классификации: ФК ЯЕ и~ФК ФЭФ. 
Таблица~6 отражает результаты рубрицирования в~НБД коннекторов левого 
и~правого элементов (а~также их контекстов) двуязычного кортежа типа $\langle$\textit{не только$\ldots$, 
но; non seulement$\ldots$ mais}$\rangle$ в~виде рубрик по двум основаниям ФК ЯЕ 
и~ФК ФЭФ.
  
\begin{table*}\small %tabl7
\begin{center}
\Caption{Дополнительные признаки двуязычных кортежей}
\vspace*{2ex}

\begin{tabular}{|c|l|l|}
\hline
\multicolumn{1}{|c|}{\raisebox{-6pt}[0pt][0pt]{№}}&
\tabcolsep=0pt\begin{tabular}{c}Код\\ дополнительного\\ признака\\ кортежа\end{tabular}&
\multicolumn{1}{c|}{Описание дополнительного 
признака кортежа}\\
\hline
1&Exp&Требуется экспертиза кортежа экспертом\\
\hline
2&Ext\_Up&Для построения кортежа необходимо учесть контекст предыдущей пары 
предложений\\
\hline
3&Ext\_Down&Для построения кортежа необходимо учесть контекст следующей пары 
предложений\\
\hline
4&Up\_Down&\tabcolsep=0pt\begin{tabular}{l}Для построения кортежа необходимо 
учесть контекст предыдущей и~следующей \\ пар предложений\end{tabular}\\
\hline
5&NB&Интересный лингвистический пример (для использования в~публикациях)\\
\hline
6&Cngrn&Коннектор русского языка переведен коннектором французского языка\\
\hline
7&Dvrg&\tabcolsep=0pt\begin{tabular}{l}Коннектор русского языка переведен ЯЕ 
или конструкцией другой 
категории\\ (не коннектором)\end{tabular}\\
\hline
\multicolumn{3}{p{160mm}}{\footnotesize \hspace*{3mm}\textbf{Замечание:}\ 
Определения дополнительных признаков Cngrn 
(конгруэнтность) и~Dvrg (дивергентность) заимствованы из работы~[1] и~адаптированы для 
случая контрастивного исследования коннекторов. В~основу использования признака Cngrn 
положено функциональное соответствие исследуемой языковой единицы и~ее ФЭФ, т.\,е.\ 
в~данном случае принадлежность ФЭФ к~функциональному классу коннекторов.}
\end{tabular}
\end{center}
\vspace*{-9pt}
\end{table*}
  
  Первый ее столбец содержит коннектор \textit{не только$\ldots$, но} 
в~контексте всего предложения на русском языке. Второй столбец содержит 
рубрики ФК ЯЕ для этого коннектора. По главному основанию рубрика 
второго уровня <<не$\|$только$\|$но>> для этого коннектора принадлежит 
рубрике первого уровня <<только>> (см.\ строку~6 в~табл.~1). По 
дополнительному основанию этому коннектору присвоено~4~руб\-рики:
  \begin{enumerate}[(1)]
\item $\langle$TBD$\rangle$ относится к~рубрике первого уровня <<Отношения>> 
(см.\ строку~1 в~табл.~3) и~говорит о~том, что отношение, выражаемое 
этим коннектором, будет определено позже;
\item $\langle$CNT p CNT q$\rangle$ относится к~рубрике первого уровня 
<<Порядок>> (см.\ строку~4 в~табл.~3) и~говорит о том, что элементы 
многокомпонентного коннектора находятся в~каждом из соединяемых 
фрагментов текста p и~q;
\item $\langle$CNT$\rangle$ относится к~рубрике первого уровня <<Статус>> (см.\ 
строку~5 в~табл.~3) и~говорит о~том, что кортеж построен для всего 
коннектора, а~не для отдельных его блоков или элементов;
\item $\langle$Дистант$\rangle$ относится к~рубрике первого уровня 
<<Расположение>> (см.\ строку~6 в~табл.~3) и~говорит о~том, что части 
коннектора разделены текстом.
\end{enumerate}

  Третий столбец таблицы содержит ФЭФ \textit{non seulement$\ldots$ mais} 
в~контексте перевода всего предложения на французский язык. Четвертый 
столбец содержит рубрики ФК ФЭФ. По главному основанию рубрика 
второго уровня <<non seulement$\ldots$\ mais>> (коннектор и~имя его 
рубрики посимвольно могут совпадать, поэтому имя ставится в~кавычки) 
говорит о том, что ФЭФ \textit{non seulement$\ldots$ mais} относится 
к~категории коннекторов французского языка, т.\,е.\ принадлежит рубрике 
первого уровня <<CNT>>. По дополнительному основанию этому 
коннектору присвоены те же~4~рубрики, что и~коннектору \textit{не 
только$\ldots$ но} в~тексте оригинала. Однако эти рубрики могут, 
в~принципе, не совпадать.
  
  После завершения рубрицирования левого и~правого элементов каждого 
кортежа лингвист может присвоить несколько дополнительных признаков 
всему кортежу в~целом, используя табл.~7.
  
\vspace*{-6pt}

\section{Двуязычный поиск в~надкорпусной базе данных коннекторов}

%\vspace*{-2pt}

Построенные двуязычные кортежи могут быть найдены по лексике обеих своих составляющих, рубрикам 
ФК ЯЕ, ФК ФЭФ и~дополнительным признакам кортежей. Приведем три примера, иллюстрирующих 
отдельно лексические и~грамматические двуязычные поисковые возможности, а также  
лек\-си\-ко-грам\-ма\-ти\-че\-ский двуязычный запрос на поиск. Для первого примера в~шаблоне на поиск 
был задан тип кортежа $\langle$\textit{притом; au surplus}$\rangle$. 
В~результате лексического поиска были 
найдены\footnote{В~этом и~следующих двух примерах результаты поиска 
даны по состоянию НБД 
коннекторов на 23.01.2016.} три двуязычных кортежа, соответствующих заданному типу, 
с~разными 
наборами рубрик ФК ЯЕ и~ФК ФЭФ, так как контексты коннектора 
\textit{притом} и~его ФЭФ \textit{au 
surplus} в~этих трех случаях различаются (см.\ табл.~8 с~контекстами и~наборами рубрик).

\begin{table*}\small %tabl8
\begin{center}
\Caption{Результат двуязычного лексического поиска~--- три двуязычных кортежа, 
соответствующих 
заданному типу $\langle$\textit{притом; au~surplus}$\rangle$, 
с~контекстами и~наборами рубрик ФК ЯЕ и~ФК ФЭФ}
\vspace*{2ex}

\begin{tabular}{|p{37mm}|p{37mm}|p{37mm}|p{37mm}|}
\hline
\multicolumn{2}{|c|}{Оригинальный текст} & \multicolumn{2}{c|}{Перевод}\\
\hline
\textbf{Притом} этот человек не любил неизвестности, &\textbf{притом}\newline 
$\langle$~TBD~$\rangle$\newline $\langle$~повествовательное~$\rangle$\newline 
$\langle$~начальная~$\rangle$\newline
$\langle$~p CNT q~$\rangle$\newline 
$\langle$~CNT~$\rangle$ 
&\textbf{Au surplus}, cet homme n'aimait pas l'incertitude &\textbf{au surplus} \newline
$\langle$~TBD~$\rangle$\newline
$\langle$~повествовательно~$\rangle$\newline 
$\langle$~начальная~$\rangle$\newline 
$\langle$~p CNT q~$\rangle$\newline
$\langle$~CNT~$\rangle$\\
\hline
\textbf{Притом} она ничего бы и~не поняла. &\textbf{притом}\newline 
$\langle$~TBD~$\rangle$\newline
$\langle$~повествовательное~$\rangle$\newline
$\langle$~начальная~$\rangle$\newline
                            $\langle$~p CNT q~$\rangle$\newline
$\langle$~CNT~$\rangle$ 
&\textbf{Au surplus}, elle n'aurait rien compris! &\textbf{au surplus}\newline 
$\langle$~TBD~$\rangle$\newline
$\langle$~восклицательное~$\rangle$\newline
$\langle$~начальная~$\rangle$\newline
$\langle$~p CNT q~$\rangle$\newline
$\langle$~CNT~$\rangle$\\
\hline
Не противоречу вам и~\textbf{притом} не мастер я философствовать. &\textbf{притом} 
$\langle$~Part~$\rangle$ 
&Je ne vous contredis pas et \textbf{au surplus} je ne suis pas fort pour faire de la philosophie. 
&\textbf{au surplus} \newline
$\langle$~Part~$\rangle$\\
\hline
\end{tabular}
\end{center}
\vspace*{-3pt}
\end{table*}

\begin{table*}[b]\small %tabl9
\vspace*{-9pt}
\begin{center}
\Caption{Результат двуязычного грамматического поиска~--- два двуязычных кортежа, 
контексты и~рубрики}
\vspace*{2ex}

\begin{tabular}{|p{37mm}|p{37mm}|p{37mm}|p{37mm}|}
\hline
\multicolumn{2}{|c|}{Оригинальный текст} & \multicolumn{2}{c|}{Перевод}\\
\hline
\textbf{Притом} она ничего бы и~не поняла. &\textbf{притом}\newline 
$\langle$~TBD~$\rangle$\newline
$\langle$~повествовательное~$\rangle$\newline 
$\langle$~начальная~$\rangle$\newline 
$\langle$~p CNT q~$\rangle$\newline 
$\langle$~CNT~$\rangle$ 
&\textbf{Au surplus}, elle n'aurait rien compris! &\textbf{au surplus}\newline 
$\langle$~TBD~$\rangle$\newline
$\langle$~восклицательное~$\rangle$\newline 
$\langle$~начальная~$\rangle$\newline 
$\langle$~p CNT q~$\rangle$\newline 
$\langle$~CNT~$\rangle$\\
\hline
\textbf{Притом же} в~деревне одному очень скучно. &\textbf{притом}$\|$\textbf{же}\newline
$\langle$~TBD~$\rangle$\newline
$\langle$~повествовательное~$\rangle$\newline 
$\langle$~начальная~$\rangle$\newline
$\langle$~p CNT q~$\rangle$\newline 
$\langle$~CNT~$\rangle$\newline 
$\langle$Контакт$\rangle$
&\textbf{Et enfin}, vivre {\ptb{\`{a}}} la campagne tout seul, 
c'est tellement ennuyeux! 
&\textbf{et}$\|$\textbf{enfin }\newline
$\langle$~TBD~$\rangle$\newline
$\langle$~восклицательное~$\rangle$\newline 
$\langle$~начальная~$\rangle$\newline 
$\langle$~p CNT q~$\rangle$\newline 
$\langle$~CNT~$\rangle$\newline
$\langle$Контакт$\rangle$\\
\hline
\end{tabular}
\end{center}
\end{table*}

  В табл.~8 рубрики $\langle$TBD$\rangle$, $\langle$p CNT q$\rangle$
  и~$\langle$CNT$\rangle$ были 
описаны выше (с небольшой разницей: в~отличие от $\langle$CNT p CNT q$\rangle$ 
рубрика $\langle$p CNT q$\rangle$ означает, что коннектор находится между 
соединяемыми фрагментами текста или во втором из них). Руб\-ри\-ка 
$\langle$начальная$\rangle$ относится к~рубрике <<Позиция>> (см.\ строку~3 
в~табл.~3) и~говорит о~том, что коннектор находится в~начале маркируемого 
им фрагмента текста. Рубрика $\langle$Part$\rangle$ относится к~рубрике первого 
уровня <<Статус>> и~означает, что в~данном случае \textit{притом} входит 
в~состав многокомпонентного коннектора (\textit{и~притом}), являясь его 
частью. Здесь для него построен кортеж третьей категории (см.\ разд.~3 
с~описанием категорий). Наконец, рубрики $\langle$повествовательное$\rangle$ 
и~$\langle$восклицательное$\rangle$ по состоянию на 1~января 2016~г.\ относятся 
к~руб\-ри\-ке первого уровня <<Структура>> (см.\ строку~2 в~табл.~3), но 
в~будущем они, возможно, будут выделены в~отдельную рубрику первого 
уровня в~процессе развития ФК ЯЕ и~ФК ФЭФ. Эти две рубрики 
используются ниже для иллюстрации двуязычных грамматических 
поисковых возможностей.
  
  Во втором примере надо было найти любые двуязычные кортежи, 
коннекторы которых маркируют повествовательные предложения (в запросе 
была задана рубрика $\langle$повествовательное$\rangle$ для контекс\-та коннектора 
в~русском языке), а~их ФЭФ~--- восклицательные (т.\,е.\ одновременно задан 
поиск и~с~использовани\-ем рубрики $\langle$восклицательное$\rangle$ для контекста 
ФЭФ). В~этом запросе на поиск присутствуют только эти две рубрики 
и~отсутствуют лексемы. В~результате поиска были найдены два двуязычных 
кортежа: один типа $\langle$\textit{притом; au surplus}$\rangle$ и~второй типа 
$\langle$\textit{притом же; et enfin}$\rangle$ (см.\ табл.~9 с~двумя кортежами, 
контекстами и~наборами руб\-рик).

  \begin{table*}\small %tabl10
  \begin{center}
  \Caption{Результат двуязычного лексико-грамматического поиска~--- три кортежа, 
соответствующих типу $\langle$\textit{притом; zero}$\rangle$ 
в~комбинации с~рубрикой $\langle$начальная$\rangle$}
   \vspace*{2ex}
   
\begin{tabular}{|p{50mm}|p{37mm}|p{50mm}|p{10mm}|}
\hline
\multicolumn{2}{|c|}{Оригинальный текст} & \multicolumn{2}{c|}{Перевод}\\
\hline
\textbf{Притом} их связывало детство и~школа~--- &\textbf{притом}\newline 
$\langle$~TBD~$\rangle$\newline
$\langle$~повествовательное~$\rangle$\newline 
$\langle$~начальная~$\rangle$\newline 
$\langle$~p CNT q~$\rangle$\newline 
$\langle$~CNT~$\rangle$ 
&Les deux hommes $\acute{\mbox{e}}$taient unis 
par les souvenirs de l'$\acute{\mbox{e}}$cole et de l'enfance, &\textbf{zero}\\
\hline 
\textbf{Притом} карет неслось такое множество взад и~вперед и~с~такою быстротою, что трудно 
было даже приметить; &\textbf{притом}\newline 
$\langle$~TBD~$\rangle$\newline
$\langle$~повествовательное~$\rangle$\newline
$\langle$~начальная~$\rangle$\newline
$\langle$~p CNT q~$\rangle$\newline 
$\langle$~CNT~$\rangle$ 
&Les $\acute{\mbox{e}}$quipages se croisaient si nombreux et roulaient {\ptb{\`{a}}} si belle allure 
qu'il $\acute{\mbox{e}}$tait difficile d'en distinguer un parmi les autres; &\textbf{zero }\\
\hline
\textbf{Притом} история о танцующих стульях в~Конюшенной улице была еще свежа, 
&\textbf{притом}\newline 
$\langle$~TBD~$\rangle$\newline
$\langle$~повествовательное~$\rangle$\newline
$\langle$~начальная~$\rangle$\newline
$\langle$~p CNT q~$\rangle$\newline
$\langle$~CNT~$\rangle$ 
&L'histoire des chaises tournantes de la rue des  
Grandes-$\acute{\mbox{E}}$curies $\acute{\mbox{e}}$tait encore pr$\acute{\mbox{e}}$sente 
{\ptb{\`{a}}} toutes les m$\acute{\mbox{e}}$moires. &\textbf{zero}\\
\hline
\end{tabular}
\end{center}
\end{table*}


  
  В третьем примере был задан одновременно поиск кортежа типа 
$\langle$\textit{притом; zero}$\rangle$, чтобы найти случаи, когда коннектор 
\textit{притом} не переведен на французский язык, и~задана рубрика второго 
уровня $\langle$начальная$\rangle$, т.\,е.\ этот коннектор должен быть в~начале 
предложения на русском языке. В~результате  
лек\-си\-ко-грам\-ма\-ти\-че\-ско\-го поиска были найдены три кортежа, 
соответствующих заданному типу, в~комбинации с~рубрикой 
$\langle$начальная$\rangle$.

  
  Таблицы~8--10 содержат примеры реализации в~НБД тех функций 
двуязычного поиска, которые отсутствуют в~корпусах параллельных текстов, 
но необходимы для сопоставительных исследований и~получения новых 
кроссязыковых знаний.

\vspace*{-6pt}
  
\section{Заключение}

  Необходимость разработки и~применения НБД для описания разных видов 
ЯЕ проявляется наиболее наглядно в~случае машинного 
перевода, качество которого на сегодняшний день остается мало 
удовлетворительным. Для повышения качества\linebreak машинного перевода 
требуется существенное развитие двуязычных грамматических описаний,\linebreak 
благодаря полученным на основе сопоставления параллельных текстов 
новым кроссязыковым знаниям. Сопоставление выровненных предложений 
двух (а~в~перспективе и~большего чис\-ла) языков с~помощью НБД позволяет 
выявлять и~заполнять лакуны в~системе лингвистических знаний о разных 
видах языковых единиц и~явлений~\cite{5-zat, 16-zat}. В~статье были 
рассмотрены примеры из НБД коннекторов, грамматическое описание 
которых (как в~одном языке, так и~в сопоставительном аспекте~\cite{17-zat}) 
должно быть приведено в~соответствие с~современным уровнем развития 
лингвистической \mbox{науки}.
{\looseness=1

}
  
  Рассмотренные примеры построения двуязычных кортежей позволяют 
сделать вывод о том, что в~общем случае рубрицирование коннекторов и~их 
ФЭФ дает возможность исследователям, применяющим НБД, выполнять 
следующие задачи:
  \begin{enumerate}[(1)]
\item разрабатывать методики формирования новых лингвистических 
знаний, ориентированные на применение современных  
ин\-фор\-ма\-ци\-он\-но-компью\-тер\-ных технологий локального и/или 
сетевого <<мозгового штурма>> для сопоставительного анализа 
параллельных текстов;
\item используя новые методики, выявлять лакуны в~системах моно- 
и~кроссязыковых знаний;
\item специфицировать лакуны, маркируя их особыми рубриками 
фасетных классификаций НБД (см., например, рубрику $\langle$отношение 
подлежит определению$\rangle$ в~табл.~4);
\item формулировать цели лингвистических исследований в~интересах 
генерации новых лингвистических знаний, направленной на заполнение 
выявленных лакун;
\item пополнять моно- и~кроссязыковые знания, заполняя лакуны 
в~процессе развития фасетных классификаций НБД, включая создание их 
новых оснований, уровней, названий и~дефиниций рубрик, используемых 
для описания исследуемых ЯЕ, явлений и~их контекстов.
\end{enumerate}

  Первые две и~четвертая функции являются общими для параллельных 
корпусов и~НБД. Однако третья и~пятая функции реализуются в~настоящее 
время только в~НБД для разных видов ЯЕ.
  
  В рамках российско-швей\-цар\-ско\-го проекта <<Контрастивное 
корпусное исследование коннекторов русского языка>> в~НБД сейчас 
проектируются еще две функции, которые позволят:
  \begin{enumerate}[(1)]
\setcounter{enumi}{5}
\item выявлять лакуны в~системах моно- и~кросс\-язы\-ко\-вых знаний, 
используя одновременно тексты, созданные профессиональными 
переводчиками, и~машинные переводы;
\item специфицировать отдельно лакуны в~системе лингвистических 
знаний и~отдельно ошибки систем машинного перевода.
\end{enumerate}

  Иначе говоря, эти новые функции НБД ориентированы на прикладное 
применение и~будут непосредственно способствовать повышению качества 
машинного перевода.
  
{\small\frenchspacing
 {%\baselineskip=10.8pt
 \addcontentsline{toc}{section}{References}
 \begin{thebibliography}{99}
\bibitem{1-zat}
\Au{Johansson S.} Seeing through Multilingual Corpora: 
On the use of corpora in contrastive 
studies.~--- Amsterdam: John Benjamins, 2007. 377~p.
\bibitem{2-zat}
\Au{Loiseau S., Sitchinava~D.\,V., Zalizniak~Anna~A., Zatsman~I.\,M.} Information 
technologies for creating the database of equivalent verbal forms in the 
Russian--French 
multivariant parallel corpus~// Информатика и~её применения, 2013. Т.~7. Вып.~2.  
С.~100--109.
\bibitem{3-zat}
\Au{Kruzhkov M.\,G., Buntman N.\,V., Loshchilova~E.\,Ju., Sitchinava~D.\,V., 
Zalizniak~Anna~A., Zatsman~I.\,M.} A~database of Russian verbal forms and their French 
translation equivalents~// Компьютерная лингвистика и~интеллектуальные технологии: По 
мат-лам ежегодной Междунар. конф. <<Диалог>>.~--- М.: РГГУ, 2014. Вып.~13(20). 
C.~284--296.
\bibitem{4-zat}
\Au{Бунтман Н.\,В., Зализняк~Анна\,А., Зацман~И.\,М., Кружков~М.\,Г., 
Лощилова~Е.\,Ю., Сичинава~Д.\,В.} Информа\-ционные технологии корпусных 
исследований: принципы построения кросс-лингвистических баз данных~// Информатика 
и~её применения, 2014. Т.~8. Вып.~2. С.~98--110.
\bibitem{5-zat}
\Au{Zatsman I., Buntman N.} Outlining goals for discovering new knowledge 
and computerised 
tracing of emerging meanings discovery~// 16th European Conference on Knowledge 
Management Proceedings.~--- Reading: Academic Publishing International Ltd., 2015. 
P.~851--860.
\bibitem{6-zat}
\Au{Зализняк Анна А., Зацман~И.\,М., Инькова~О.\,Ю., Кружков~М.\,Г.} Надкорпусные 
базы данных как лингвистический ресурс~// Корпусная лингвистика-2015: Тр. 7-й 
Междунар. конф.~--- СПб.: СПбГУ, 2015. С.~211--218.
\bibitem{7-zat}
\Au{Кружков М.\,Г.} Информационные ресурсы контрастивных лингвистических 
исследований: типологические базы данных~// Системы и~средства информатики, 2015. 
Т.~25. №\,1. С.~198--212.
\bibitem{8-zat}
\Au{Кружков М.\,Г.} Информационные ресурсы контрастивных лингвистических 
исследований: электронные корпуса текстов~// Системы и~средства информатики, 2015. 
Т.~25. №\,2. С.~140--159.
\bibitem{9-zat}
\Au{Добровольский Д.\,О., Кретов А.\,А., Шаров~С.\,А.} Корпус параллельных текстов: 
архитектура и~возможности использования~// Национальный корпус русского языка: 
2003--2005.~--- М.: Индрик, 2005. С.~263--296.
\bibitem{10-zat}
\Au{Добровольский Д.\,О., Кретов А.\,А., Шаров~С.\,А.} Корпус параллельных текстов~// 
Научная и~техническая информация. Сер.~2: Информационные процессы и~сис\-те\-мы, 2005. 
№\,6. С.~16--27.
\bibitem{11-zat}
\Au{Zatsman I.} Computer and information science: Background of formation~// 
Scientific   Technical Information Processing, 2013. Vol.~40. No.\,3. P.~119--130.
\bibitem{12-zat}
\Au{Zatsman I.} Table of interfaces of informatics as computer and information science~// 
Scientific Technical Information Processing, 2014. Vol.~41. No.\,4. P.~233--246.
\bibitem{13-zat}
\Au{Зацман~И.\,М., Косарик В.\,В., Курчавова~О.\,А.} Задачи представления личностных 
и~коллективных концептов в~цифровой среде~// Информатика и~её применения, 2008. 
Т.~2. Вып.~3. С.~54--69.
\bibitem{14-zat}
\Au{Зацман И.\,М.} Семиотическая модель взаимосвязей концептов, информационных 
объектов и~компьютерных кодов~// Информатика и~её применения, 2009. Т.~3. Вып.~2. 
С.~65--81.
\bibitem{15-zat}
\Au{Зацман И.\,М.} Нестационарная семиотическая модель компьютерного кодирования 
концептов, информационных объектов и~денотатов~// Информатика и~её применения, 
2009. Т.~3. Вып.~4. С.~87--101.
\bibitem{16-zat}
\Au{Zatsman I., Buntman N., Kruzhkov~M., Nuriev~V., Zalizniak~Anna~A.} Conceptual 
framework for development of computer technology supporting cross-linguistic knowledge 
discovery~// 15th European Conference on Knowledge Management Proceedings.~--- Reading: 
Academic Publishing International Ltd., 2014. Vol.~3. P.~1063--1071.
\bibitem{17-zat}
\Au{Инькова-Манзотти О.\,Ю.} Коннекторы противопоставления во французском 
и~русском языках: сопоставительное исследование.~---  М.: Информэлектро, 2001. 434~с.
\bibitem{18-zat}
\Au{Инькова О.\,Ю.} К~проблеме описания многокомпонентных коннекторов русского 
языка: \textit{не только$\ldots$ но~и}~// Вопросы языкознания, 2016. №\,2. С.~37--60.
\end{thebibliography}

 }
 }

\end{multicols}

\vspace*{-3pt}

\hfill{\small\textit{Поступила в~редакцию 28.01.16}}

%\vspace*{8pt}

\newpage

\vspace*{-24pt}

%\hrule

%\vspace*{2pt}

%\hrule

%\vspace*{8pt}



\def\tit{REPRESENTATION OF~CROSS-LINGUAL KNOWLEDGE ABOUT~CONNECTORS
IN~SUPRACORPORA DATABASES}

\def\titkol{Representation of cross-lingual knowledge about connectors
in~supracorpora databases}

\def\aut{I.\,M.~Zatsman$^1$, O.\,Yu.~Inkova$^{1,2}$, M.\,G.~Kruzhkov$^1$, 
and~N.\,A.~Popkova$^1$}

\def\autkol{I.\,M.~Zatsman, O.\,Yu.~Inkova, M.\,G.~Kruzhkov, 
and~N.\,A.~Popkova}

\titel{\tit}{\aut}{\autkol}{\titkol}

\vspace*{-9pt}


\noindent
$^1$Institute of Informatics Problems, Federal Research Center 
``Computer Science and Control'' of the Russian\linebreak
 $\hphantom{^1}$Academy of Sciences,
44-2 Vavilov Str., Moscow 119333, Russian Federation


\noindent
$^2$University of Geneva, 22~Bd des Philosophes, CH-1205 Geneva 4, Switzerland

\def\leftfootline{\small{\textbf{\thepage}
\hfill INFORMATIKA I EE PRIMENENIYA~--- INFORMATICS AND
APPLICATIONS\ \ \ 2016\ \ \ volume~10\ \ \ issue\ 1}
}%
 \def\rightfootline{\small{INFORMATIKA I EE PRIMENENIYA~---
INFORMATICS AND APPLICATIONS\ \ \ 2016\ \ \ volume~10\ \ \ issue\ 1
\hfill \textbf{\thepage}}}

\vspace*{3pt}

   
   
   \Abst{The article considers ``supracorpora databases,'' which are used in 
contrastive linguistic studies. Such databases result from processing of parallel 
texts from bilingual parallel subcorpora within the Russian National Corpus. Each 
of these parallel texts contains either one original Russian text with one or more 
translations into a~foreign language, or one original text in a foreign language with 
one translation into Russian. Every source text is aligned with its translation(s) at 
the level of sentences. Supracorpora databases are a new type of linguistic 
resources designed for goal-oriented discovery of new knowledge about various 
linguistic units. This knowledge is needed to improve the quality of machine 
translation, to update monolingual and bilingual grammars, and to modernize 
a~wide range of academic courses in such fields as linguistics and translation studies. 
The article describes the underlying conceptual foundations of the database and 
gives an example of how it can be implemented to represent knowledge about 
Russian connectors and their French translation correspondences.}
   
   \KWE{cross-lingual studies; Russian connectors; representation of 
knowledge about connectors; supracorpora databases}
  
\DOI{10.14357/19922264160110} 

\Ack
\noindent
This research was performed at the Institute of Informatics Problems, 
Federal Research Center 
``Computer Science and Control'' of the Russian Academy of Sciences, and 
supported  by the Russian Foundation for Basic Research
(grant No.\,14-07-00785), by the Russian Foundation for Humanities 
(grant No.\,16-24-41002), by Swiss Nastional Science Foundation 
(grant No.\,IZLRZ1\_164059).





%\vspace*{3pt}

  \begin{multicols}{2}

\renewcommand{\bibname}{\protect\rmfamily References}
%\renewcommand{\bibname}{\large\protect\rm References}

{\small\frenchspacing
 {%\baselineskip=10.8pt
 \addcontentsline{toc}{section}{References}
 \begin{thebibliography}{99}
\bibitem{1-zat-1}
\Aue{Johansson, S.} 2007. \textit{Seeing through Multilingual Corpora: On the use 
of corpora in contrastive studies}. Amsterdam: John Benjamins. 377~p.
\bibitem{2-zat-1}
\Aue{Loiseau, S., D.\,V. Sitchinava, Anna~A.~Zalizniak, and I.\,M.~Zatsman}. 
2013. Information technologies for creating the database of equivalent verbal forms 
in the Russian--French multivariant parallel corpus. \textit{Informatika i ee 
Primeneniya}~--- \textit{Inform. Appl.} 7(2):100--109.
\bibitem{3-zat-1}
\Aue{Kruzhkov, M.\,G., N.\,V.~Buntman, E.\,Ju.~Loshchilova, 
D.\,V.~Sitchinava, Anna A.~Zalizniak, and I.\,M.~Zatsman}. 2014. A~database 
of Russian verbal forms and their French translation equivalents. 
\textit{Komp'yuternaya Lingvistika i~Intellektual'nye Tekhnologii. Po mat-lam 
Ezhegodnoy Mezhdunar. Konf. ``Dialog-2014''} [Computational Linguistics and 
Intellectual Technologies: Conference (International) ``Dialog-2014'' 
Proceedings]. Moscow. 13(20):284--297.
\bibitem{4-zat-1}
\Aue{Buntman, N.\,V., Anna A.~Zaliznyak, I.\,M.~Zatsman, M.\,G.~Kruzhkov, 
E.\,Yu.~Loshchilova, and D.\,V.~Sichinava}. 2014. Informatsionnye tekhnologii 
korpusnykh issledovaniy: printsipy postroeniya kross-lingvisticheskikh baz dannykh 
[Information technologies for corpus studies: Underpinnings for cross-linguistic 
database creation]. \textit{Informatika i ee Primeniya}~--- \textit{Inform. Appl.} 
8(2):98--110.
\bibitem{5-zat-1}
\Aue{Zatsman, I., and N.~Buntman}. 2015. Outlining goals for discovering new 
knowledge and computerised tracing of emerging meanings discovery. \textit{16th 
European Conference on Knowledge Management Proceedings}. Reading: 
Academic Publishing International Ltd. 851--860.
\bibitem{6-zat-1}
\Aue{Zaliznyak, Anna A., I.\,M.~Zatsman, O.\,Yu.~In'kova, and 
M.\,G.~Kruzhkov}. 2015. Nadkorpusnye bazy dannykh kak lingvisticheskiy resurs 
[Subcorpora databases as linguistic resource]. \textit{Korpusnaya Lingvistika: Tr.  
7-y Mezhdunar. Konf.} [7th Conference (International) on Corpus Linguistics 
Proceedings]. St.\ Petersburg: St.\ Petersburg State University. 211--218.
\bibitem{7-zat-1}
\Aue{Kruzhkov, M.\,G.} 2015. Informatsionnye resursy kontrastivnykh 
lingvisticheskikh issledovaniy: Tipologicheskie bazy dannykh [Information 
resources for contrastive studies: Typological databases]. \textit{Sistemy i~Sredstva 
Informatiki}~--- \textit{Systems and Means of Informatics} 25(1):198--212.
\bibitem{8-zat-1}
\Aue{Kruzhkov, M.\,G.} 2015. Informatsionnye resursy kontrastivnykh 
lingvisticheskikh issledovaniy: Elektronnye korpusa tekstov [Information resources 
for contrastive studies: Electronic text corpora]. \textit{Sistemy i~Sredstva 
Informatiki}~--- \textit{Systems and Means of Informatics} 25(2):140--159.

\bibitem{9-zat-1}
\Aue{Dobrovol'skiy, D.\,O., A.\,A.~Kretov, and S.\,A.~Sharov}. 2005. Korpus parallel'nykh tekstov: 
Arkhitektura i~vozmozh\-no\-sti ispol'zovaniya [Corpus of parallel texts: Architecture and applications]. 
\textit{Natsional'nyy korpus russkogo yazyka: 2003--2005} 
[Russian National Corpus: 2003--2005]. Moscow: Indrik. 263--296.
\bibitem{10-zat-1}
\Aue{Dobrovol'skiy, D.\,O, A.\,A.~Kretov, and S.\,A.~Sharov}. 2005. Korpus 
parallel'nykh tekstov [Corpus of parallel texts]. \textit{Nauchnaya i~Tekhnicheskaya 
Informatsiya} [Scientific and Technical Information]. Ser.~2: Informatsionnye 
protsessy i~sistemy [Informational processes and systems]. 6:16--27.
\bibitem{11-zat-1}
\Aue{Zatsman, I.} 2013. Computer and information science: Background of 
formation. \textit{Scientific Technical Information Processing} 40(3):119--130.
\bibitem{12-zat-1}
\Aue{Zatsman, I.} 2014. Table of interfaces of informatics as computer and 
information science. \textit{Scientific Technical Information Processing} 
41(4):233--246.

\bibitem{13-zat-1}
\Aue{Zatsman, I.\,M., V.\,V.~Kosarik, and O.\,A.~Kurchavova}. 2008. Zadachi 
predstavleniya lichnostnykh i~kollektivnykh kontseptov v~tsifrovoy srede 
[Representation of individual and collective concepts in digital medium].  
\textit{Informatika i~ee Primeneniya}~--- \textit{Inform. Appl.} 2(3):54--69.
\bibitem{14-zat-1}
\Aue{Zatsman, I.} 2009. Semioticheskaya model' vzaimosvyazey kontseptov, 
informatsionnykh ob"ektov i~komp'yuternykh kodov [Semiotic model of 
relationships of concepts, information objects, and computer codes]. 
\textit{Informatika i~ee Primeneniya}~--- \textit{Inform. Appl.} 3(2):65--81.
\bibitem{15-zat-1}
\Aue{Zatsman, I.} 2009. Nestatsionarnaya semioticheskaya model' 
komp'yuternogo kodirovaniya kontseptov, informatsionnykh ob"ektov 
i~denotatov [Nonstationary semiotic model of computer coding of concepts, 
information objects, and denotata]. \textit{Informatika i~ee Primeneniya}~--- 
\textit{Inform. Appl.} 3(4):87--101.
\bibitem{16-zat-1}
\Aue{Zatsman, I., N.~Buntman, M.~Kruzhkov, V.~Nuriev, and Anna 
A.~Zalizniak}. 2014. Conceptual framework for development of computer 
technology supporting cross-linguistic knowledge discovery. \textit{15th European 
Conference on Knowledge Management Proceedings}. Reading: Academic 
Publishing International Ltd. 3:1063--1071.
\bibitem{17-zat-1}
\Aue{Inkova-Manzotti, O.~Yu.} 2001. \textit{Konnektory protivopostavleniya 
vo frantsuzskom i~russkom yazykakh: Sopostavitel'noe issledovanie} [Connectors 
of opposition in French and Russian: A~comparative study]. Moscow: 
Informelektro. 434~p.
\bibitem{18-zat-1}
\Aue{Inkova, O.\,Yu.} 2016. K~probleme opisaniya mnogokomponentnykh 
konnektorov russkogo yazyka: Ne tol'ko$\ldots$ no~i [Towards the 
description of multiword connectives in Russian: Ne tol'ko$\ldots$ no~i 
(non only$\ldots$ but also)]. \textit{Voprosy Jazykoznanija} [Topics in the Study 
of Language] 2:37--60.
\end{thebibliography}

 }
 }

\end{multicols}

\vspace*{-3pt}

\hfill{\small\textit{Received January 28, 2016}}


\Contr

\noindent
\textbf{Zatsman Igor M.} (b.\ 1952)~--- Doctor of Science in technology, Head of Department, Institute 
of Informatics Problems, Federal Research Center ``Computer Science and Control'' of the Russian 
Academy of Sciences, 44-2~Vavilov Str., Moscow 119333, Russian Federation; izatsman@yandex.ru 

\vspace*{3pt}

\noindent
\textbf{Inkova Olga Yu.} (b.\ 1965)~---  Doctor of Science in philology, Faculty member, University of 
Geneva, 22~Bd des Philosophes, CH-1205 Geneva 4, Switzerland;  senior scientist, Institute of Informatics 
Problems, Federal Research Center ``Computer Science and Control'' of the Russian Academy of 
Sciences, 44-2~Vavilov Str., Moscow 119333, Russian Federation; Olga.Inkova@unige.ch 

\vspace*{3pt}

\noindent
\textbf{Kruzhkov Mikhail G.} (b.\ 1975)~---
leading programmer, Institute of Informatics Problems, Federal Research Center ``Computer Science 
and Control'' of the Russian Academy of Sciences, 44-2 Vavilov Str., Moscow 119333, Russian 
Federation; magnit75@yandex.ru 

\vspace*{3pt}

\noindent
\textbf{Popkova Natalia A.} (b.\ 1992)~--- junior scientist, Institute of Informatics Problems, Federal 
Research Center ``Computer Science and Control'' of the Russian Academy of Sciences, 44-2 Vavilov Str., 
Moscow 119333, Russian Federation; natasha\_\_popkova@mail.ru


   
\label{end\stat}


\renewcommand{\bibname}{\protect\rm Литература}