\def\stat{gorshenin+Sh}

\def\tit{РАЗРАБОТКА АЛГОРИТМА ЧИСЛЕННОГО РЕШЕНИЯ ЗАДАЧИ ОПТИМАЛЬНОГО УПРАВЛЕНИЯ ИНВЕСТИЦИЯМИ 
В~ЗАКРЫТОЙ ДИНАМИЧЕСКОЙ МОДЕЛИ ТРЕХСЕКТОРНОЙ ЭКОНОМИКИ}

\def\titkol{Разработка алгоритма численного решения задачи оптимального управления 
инвестициями} 
%в~закрытой динамической модели трехсекторной экономики}

\def\aut{П.\,В.~Шнурков$^1$, В.\,В.~Засыпко$^2$, В.\,В.~Белоусов$^3$, А.\,К.~Горшенин$^4$}

\def\autkol{П.\,В.~Шнурков, В.\,В.~Засыпко, В.\,В.~Белоусов, А.\,К.~Горшенин}

\titel{\tit}{\aut}{\autkol}{\titkol}

%{\renewcommand{\thefootnote}{\fnsymbol{footnote}} \footnotetext[1]
%{Работа выполнена при поддержке РФФИ (проект 15-07-02244).}}


\renewcommand{\thefootnote}{\arabic{footnote}}
\footnotetext[1]{НИУ Высшая школа экономики, pshnurkov@hse.ru}
\footnotetext[2]{НИУ Высшая школа экономики, vzasypko@gmail.com}
\footnotetext[3]{Институт проб\-лем информатики Федерального исследовательского центра 
<<Информатика и~управ\-ле\-ние>> Российской академии наук, vbelousov@ipiran.ru}
\footnotetext[4]{Институт проб\-лем информатики Федерального исследовательского центра 
<<Информатика и~управ\-ле\-ние>> Российской академии наук; 
%Федеральное государственное бюджетное образовательное учреждение высшего образования 
Московский технологический университет (МИРЭА),
 agorshenin@frccsс.ru}

\Abst{Настоящее исследование посвящено разработке численного метода решения 
задачи оптимального управления инвестициями в~закрытой динамической модели 
трехсекторной экономики. В~предшествующих работах было проведено аналитическое 
исследование поставленной задачи оптимального управления на основе принципа максимума. 
В~данной работе полученные аналитические представления для функций состояний 
и~сопряженных переменных используются как основа для численного алгоритма. Предлагаемый 
алгоритм позволяет проанализировать класс допустимых функций управления, имеющих не 
более заданного конечного числа точек переключения, и~найти среди них те, которые 
удовлетворяют необходимым условиям экстремума и~ограничениям исходной задачи. Общая 
схема предложенного алгоритма может быть использована и~при решении других задач 
оптимального управления, связанных с~различными предметными областями. В~ходе 
проведенного исследования разработанный алгоритм реализован в~комплексе прикладных 
программ.}

\KW{модель трехсекторной экономики; принцип максимума Понтрягина; численное решение 
задачи оптимального управления}

\DOI{10.14357/19922264160108} %

\vspace*{3pt}


\vskip 14pt plus 9pt minus 6pt

\thispagestyle{headings}

\begin{multicols}{2}

\label{st\stat}


\section{Введение}

В работах П.\,В.~Шнуркова и~В.\,В.~Засыпко~[1, 2] была поставлена и~исследована задача 
оптимального управ\-ле\-ния инвестициями, сформулированная на основе закрытой динамической 
модели трехсекторной экономики. По форме поставленная задача представляет собой 
классическую задачу оптимального управ\-ле\-ния на заданном конечном интервале времени 
с~закрепленным левым концом траектории. Теоретическое исследование задачи производилось 
на основе принципа максимума Понтря\-гина.
{ %\looseness=-1

}

Исследованию задач оптимального управ\-ле\-ния в~экономических системах посвящена 
обширная литература. Отметим, например, фундаментальные издания~[3--5]. Работы, 
в~которых исследуются многосекторные экономические модели, встречаются достаточно 
редко~[6]. Данное обстоятельство определяет актуальность проведенного исследо\-вания.

Следует отметить, что метод исследования задач оптимального управ\-ле\-ния, основанный на 
использовании принципа максимума, позволяет получить аналитические решения только для 
сравнительно небольшого числа задач, и~такие решения хорошо известны в~научной 
и~учебной литературе~[7, 8]. Для остальных задач принцип максимума позволяет только 
определить структуру, т.\,е.\ аналитическое устройство функций оптимального управ\-ле\-ния. 

Для продолжения исследования требуется решение весьма сложной системы соотношений, 
состоящей из необходимых условий экстремума и~ограничений исходной задачи. Именно 
такая ситуация возникает и~при решении рассматриваемой задачи.

 Дальнейшее аналитическое исследование проводилось для функций управ\-ле\-ния, 
 удовлетворяющих условию максимума и~при этом имеющих произвольное конечное число 
 переключений. Для таких функций выписаны аналитические решения системы сопряженных 
 уравнений и~системы уравнений дифференциальной связи. В~результате получены явные 
 представления для сопряженных переменных, которые по содержанию являются множителями 
 Лагранжа в~рассматриваемой экстремальной задаче, и~функций удельного капитала 
 в~различных секторах, которые играют роль состояний в~этой задаче~\cite{1-gor}. 
 Однако на этом аналитические средства исследования исчерпывают свои возможности. 
 Дальнейшее исследование по\-став\-лен\-ной задачи оптимального управ\-ле\-ния возможно только 
 при помощи численных методов.

В работе~\cite{1-gor} была кратко описана общая схема алгоритма, позволяющего завершить 
исследование поставленной задачи численно. Для заданного набора исходных данных 
(параметров модели) этот алгоритм позволяет определить конкретную функцию управ\-ле\-ния 
и~соответствующий набор функций состояний, которые удовлетворяют системе соотношений, 
состоящей из необходимых условий и~ограничений исходной задачи, т.\,е.\ 
является допустимой экстремалью.


В настоящей работе результаты проведенного исследования представлены 
в~целом. При этом основное внимание уделяется описанию и~реализации указанного 
численного алгоритма. Приводится математическая постановка задачи оптимального 
управ\-ле\-ния. Формулируется утверждение о~необхо\-димых условиях экстремума в~форме 
принципа максимума. Условие максимума позволяет определить структуру оптимального 
управ\-ле\-ния. Приводятся формулировки отдельных теорем об аналитических представлениях 
функций состояний и~сопряженных переменных. Вычисление значений этих функций по 
найденным аналитическим формулам используется в~ходе реализации основного алго\-ритма.

Разработанный алгоритм реализован в~виде комплекса прикладных программ. 
В~данной работе приведено описание структуры этого комплекса и~его отдельных 
составляющих. Дано также описание исходных данных, для которых произведены численные 
исследования. Результатом такого исследования является конкретный управ\-ля\-емый процесс, 
удовлетворяющий необходимым условиям и~ограничениям поставленной задачи.

\section{Математическая постановка задачи оптимального управления}

Рассматриваемая задача оптимального управ\-ле\-ния формулируется на основе 
трехсекторной модели экономики. Трехсекторная модель предложена 
В.\,А.~Колемаевым~[9, 10]. Дополнительные предположения, внесенные в~данную модель, 
изложены в~работе~\cite{2-gor}. Измененную версию основной модели можно условно 
называть трехсекторной инвестиционной динамической моделью макроэкономической 
системы (национальной экономики). В~исходной задаче оптимального управ\-ле\-ния~\cite{2-gor} 
состояние системы описывается трехмерным вектором, компонентами которого являются 
функции фондовооруженности (удельного капитала) в~каж\-дом из секторов. Параметром 
управ\-ле\-ния служит скалярная функция, представляющая собой долю инвестиций в~первый 
(фондосоздающий) сектор по отношению к~общему объему инвестиций в~сис\-теме.

Математическая постановка задачи управ\-ле\-ния имеет следующий вид.
\begin{enumerate}[1.]
\item  Целевой функционал смешанного типа с~интегральной и~терминальной
частями:
\begin{multline}
\int\limits^T_0 e^{-\delta t}A_2\theta_2k^{\alpha_2}_2(t)\,dt +{}\\
{}+
e^{-\delta T}\psi(k_0(T),k_1(T),k_2(T))\longrightarrow \max\,.
\label{e1-gor}
\end{multline}



\item  Дифференциальная связь:
\begin{equation}
\left.
\begin{array}{rl}
\dot k_0&=-\lambda_0k_0+l^{(1)}_0\rho A_1k^{\alpha_1}_1(1-u_1)\,;
\\[6pt]
\dot k_1&=-\lambda_1k_1+A_1k^{\alpha_1}_1 u_1\,;
\\[6pt]
\dot k_2&=-\lambda_2k_2+l^{(1)}_2(1-\rho)A_1k^{\alpha_1}_1(1-u_1)\,.
\end{array}
\right\}
\label{e2-gor}
\end{equation}


\item Начальные условия на основные параметры (состояния системы):
\begin{equation}
 k_0(0)=k_{0,0}\,,\enskip k_1(0)=k_{1,0}\,,\enskip
k_2(0)=k_{2,0}\,.
\label{e3-gor}
\end{equation}


\item Ограничения на допустимое управление:
\begin{equation}
 0 \leq u_1(t)\leq 1\ , 0 \leq t \leq T\,.
 \label{e4-gor}
 \end{equation}
\end{enumerate}

Приведем описание исходных параметров и~основных характеристик модели, 
входящих в~указанную постановку задачи. Отметим, что во введенных обозначениях 
нижний индекс~$j$ соответствует номеру сектора рассматриваемой экономической системы, 
$j\hm=0$, 1, 2.
\begin{description}
\item[\,] $K_j$ --- объем основных производственных фондов (капитал) в~$j$-м секторе.

\item[\,] $L_j$ --- число занятых (объем трудовых ресурсов) в~$j$-м секторе.

\item[\,] $I_j$ --- объем инвестиций в~$j$-й сектор.

\item[\,] $k_j={K_j}/{L_j}$~--- фондовооруженность $j$-го сектора
экономики (удельный капитал).

\item[\,] $i_j={I_j}/{L_j}$~--- удельные инвестиции в~$j$-й сектор
экономики.
\end{description}

Указанные выше характеристики являются функциями от времени.
\begin{description}
\item[\,] $\nu$ --- доля прироста единицы объема трудовых ресурсов за
единицу времени во всей экономической системе.

\item[\,] $\mu_j$ --- доля выбывших за единицу времени основных
производственных фондов в~$j$-м секторе экономики.
\end{description}

Заметим, что в~рассматриваемой модели выполняются следующие соотношения:
\begin{description}
\item[\,]  $L_j=L_{j,0} e^{\nu t}$, $L_{j,0}$~--- объем трудовых ресурсов 
 в~$j$-м секторе в~начальный момент времени.
\item[\,] 
$f_j(k_j)=A_j k^{\alpha_j}_j$~--- объем произведенного продукта на 
одну единицу трудовых ресурсов (производительность труда) в~$j$-м секторе.
\item[\,] 
$A_j$~---  нормирующий коэффициент $A_j \hm> 0$.
\item[\,] 
$\alpha_j$~---  коэффициент эластичности $0\hm< \alpha_j \hm< 1$.
\end{description}

Параметры $\nu$, $\mu_j$, $L_{j,0}$, $A_j$ и~$\alpha_j$, $j\hm=0$, 1, 2, 
предполагаются заданными.

Введем дополнительные обозначения:
\begin{gather*} 
\lambda_j = \mu_j + \nu\,, \quad j=0,1,2\,;
\\
\theta_j = \fr{L_j}{L}=\fr{L_j}{L_0 +L_1 +L_2}=\fr{L_{j,0}}{L_{0,0} +L_{1,0} +L_{2,0}}, 
\\
\hspace*{54mm}j=0,1,2;
\\
l^{(1)}_0 = \fr{L_{1,0}}{L_{0,0}}\,; \enskip  
l^{(1)}_2 = \fr{L_{1,0}}{L_{2,0}}\,; \enskip\ B_2=A_2 \theta_2\,.
\end{gather*}
Функция
\begin{equation*} 
u_1(t) = \fr{I_1(t)}{I_0(t)+I_1(t)+I_2(t)} = \fr{i_1(t)}{A_1 k^{\alpha_1}_1(t)}
\end{equation*}
играет в~данной модели роль параметра управ\-ления.

Постоянная величина $0 \hm\leq \rho \hm\leq 1$ определяет распределение инвестиций 
между нулевым (материальным) и~вторым (потребительским) секторами после задания 
параметра управ\-ле\-ния, который, как уже отмечалось, определяет долю инвестиций, 
направляемых в~первый (фондосоздающий) сектор. Эта величина предполагается заданной.

Постоянная величина $\delta \hm> 0$ называется коэффициентом дисконтирования и~также 
предполагается известной. Данная величина характеризует темп инфляции в~рассматриваемой 
экономической системе.

С формальной точки зрения задача оптимизации~(\ref{e1-gor})--(\ref{e4-gor}) представляет собой классическую задачу
оптимального управ\-ле\-ния с~фиксированным интервалом времени,
закрепленным левым и~свободным правым концом траектории~\cite{2-gor, 1-gor}. 
Эко\-номическое содержание этой задачи описано\linebreak в~работе~\cite{2-gor}. 
Напомним его вкратце. Целевой функционал~(\ref{e1-gor}) 
состоит из интегральной и~терминальной составляющих. Интегральная часть представляет 
собой накопленный объем удельного производства (производительность труда) 
потребительского сектора, т.\,е.\ накопленный удельный объем потребительских благ. 
Терминальная часть характеризует уровень производства во всей сис\-те\-ме в~конечный 
момент времени~$T$ и~зависит от значений удельного капитала в~различных секторах. 
Система соотношений~(\ref{e2-gor}) или дифференциальная связь описывает изменение 
состояний системы~$k_0(t)$, $k_1(t)$, $k_2(t)$ при заданной функции управ\-ле\-ния $u_1(t)$. 
Данная сис\-тема непосредственно выводится из исходных динамических соотношений, 
характеризующих трехсекторную модель экономиче\-ской системы. Соотноше\-ния~(\ref{e3-gor}), 
или начальные условия, означают, что заданы фиксированные значения функций состояний 
в~начальный момент времени $t\hm=0$. Соотношение~(\ref{e4-gor}) 
пред\-став\-ля\-ет собой ограничение на управ\-ле\-ние, связанное с~экономическим содержанием 
функции~$u_1(t)$.

\section{Необходимые условия экстремума и~их~теоретический~анализ}

Приведем в~краткой форме результаты анализа задачи~(1)--(4) методом, основанным на 
принципе максимума Понтрягина~\cite{8-gor, 7-gor, 11-gor}. 
Основным тео\-ре\-ти\-ческим утверждением является утверждение о~необходимых условиях 
экстремума в~по\-став\-лен\-ной задаче, доказанное в~работе~\cite{2-gor}. 
В~данной работе это утверждение приводится  в~новой форме.

\smallskip

\noindent
\textbf{Теорема~1.} 
\textit{Пусть $(k_{0*}(t), k_{1*}(t),k_{2*}(t); u_{1*}(t))$~--- 
оптимальный управ\-ля\-емый процесс, т.\,е.\ решение задачи оптимального управ\-ле\-ния}~(\ref{e1-gor})--(\ref{e4-gor}). 
\textit{Тогда \mbox{найдутся} не равные нулю одновременно множители Лагранжа $\lambda^{*}_0 \hm
\in R$, $(p_0(t),p_1(t),p_2(t))$, $t \hm\in [0,T]$, такие что выполняются следующие 
соотношения}:
\begin{enumerate}[1.]
\item \textit{Сопряженные уравнения $($система дифференциальных уравнений относительно 
функций $(p_0(t),p_1(t),p_2(t))$$)$, которые в~теории называются сопряженными 
переменными}~\cite{8-gor, 7-gor, 11-gor}:
\begin{equation}
\hspace*{-6mm}
\left.
\begin{array}{rl}
\dot p_0(t)&=\lambda_0 p_0(t)\,;
\\[6pt]
\dot p_1(t)&=\lambda_1p_1(t)-A_1\alpha_1k^{\alpha_1-1}_1(t) 
\left[l^{(1)}_0\rho p_0(t)+{}\right.\\[6pt]
&\hspace*{10mm}\left.{}+l^{(1)}_2(1-\rho)p_2(t)\right]
\left(1-u_1(t)\right)-{}\\[6pt]
&\hspace*{25mm}{}-A_1\alpha_1k^{\alpha_1-1}_1(t)u_1(t)\,;
\\[6pt]
\dot p_2(t)&=\lambda_2 p_2(t)-B_2 e^{-\delta t} \alpha_2 k^{\alpha_2-1}_2(t)
\end{array}\!\right\}\!\!\!\!
\label{e5-gor}
\end{equation}
\textit{при $k_{0}(t)\hm=k_{0*}(t),$ $k_{1}(t)\hm=k_{1*}(t),$ $k_{2}(t)\hm=k_{2*}(t)$ 
и~$u_{1}(t)\hm=u_{1*}(t)$}.

\item \textit{Условия трансверсальности $($система граничных условий 
к~сопряженным уравнениям в~точке $t\hm=T$$)$}:
\begin{equation}
\hspace*{-3mm}\left.
\begin{array}{rl}
p_0(T)&=e^{- \delta T} \psi_{k^{(1)}_0}\left(k_0(T),k_1(T),k_2(T)\right)={}\\[6pt]
&\hspace*{38mm}{}=\psi^{(0)}_0(T);
\\[6pt]
p_1(T)&=e^{- \delta T} \psi_{k^{(1)}_1}\left(k_0(T),k_1(T),k_2(T)\right)={}\\[6pt]
&\hspace*{38mm}{}=\psi^{(0)}_1(T);
\\[6pt] 
p_2(T)&=e^{- \delta T} \psi_{k^{(1)}_2}\left(k_0(T),k_1(T),k_2(T)\right)={}\\[6pt]
&\hspace*{38mm}{}=\psi^{(0)}_2(T)
\end{array}
\right\}\!\!\!
\label{e6-gor}
\end{equation}
\textit{при $k_{0}(t)\hm=k_{0*}(t),$ $k_{1}(t)\hm=k_{1*}(t)$ и~$k_{2}(t)\hm=k_{2*}(t)$}.

\item \textit{Условие максимума функции Понтрягина}:
\begin{multline} 
\max_{u_1 \in U}H(t,k_0,k_1,k_2;u_1,p_0,p_1,p_2)={}\\
{}=\max_{u_1 \in U} \left[
\vphantom{\left[-l^{(1)}_0\rho p_0+p_1-l^{(2)}_0(1-\rho)p_2
\right]}
-\left[\lambda_0k_0p_0+\lambda_1k_1p_1+\lambda_2k_2p_2
\right]+{}\right.\\
{}+B_2 e^{-\delta t}k^{\alpha_2}_2+ A_1k^{\alpha_1}_1 [l^{(1)}_0 \rho p_0+l^{(2)}_0(1-\rho)p_2]+{}\\
\left.{}+
A_1k^{\alpha_1}_1\left[-l^{(1)}_0\rho p_0+p_1-l^{(2)}_0(1-\rho)p_2
\right]u_1 \right]=
\\
=H\left(t,k_0,k_1,k_2;u_{1^*},p_0,p_1,p_2\right) 
\label{e7-gor}
\end{multline}
\textit{при $k_{0}(t)=k_{0*}(t),$ $k_{1}(t)\hm=k_{1*}(t),$ $k_{2}(t)\hm=k_{2*}(t)$
и~$u_{1}(t)\hm=u_{1*}(t)$, где $U\hm=\{u: 0 \leq u \hm\leq 1 \}$~--- 
множество допустимых значений параметра управ\-ле\-ния $u_1=u_1(t)$}.
\end{enumerate}

Теорема~1 и~аналогичные утверждения в~теории оптимального управ\-ле\-ния 
называются теоремами о~необходимых условиях экстремума в~форме принципа максимума 
или просто принципом максимума Понтрягина. В~классических задачах управ\-ле\-ния 
эти условия состоят из сопряженных уравнений~(\ref{e5-gor}), граничных условий 
к~сопряженным уравнениям вида~(\ref{e6-gor}), которые обычно называются условиями 
трансверсальности, а~также из условия максимума некоторой специальной функции, 
называемой функцией Понтрягина или гамильтонианом (соотношение~(\ref{e7-gor})).

Дальнейшее исследование заключается в~анализе общей системы соотношений, состоящей 
из необходимых условий~(\ref{e5-gor})--(\ref{e7-gor}) и~ограничений исходной 
задачи~(\ref{e2-gor})--(\ref{e4-gor}). Важную роль в~этом анализе играет условие 
максимума функции Понтрягина.

Из условия максимума непосредственно следует, что структура функции 
оптимального управ\-ле\-ния $u_{1^*}(t)$ зависит от значения некоторой вспомогательной 
функции $Q(p_0(t),p_1(t),p_2(t))$ в~каждой точке $t \hm\in [0,T]$ и~может быть выражена 
следующей формулой:

\noindent
\begin{multline}
u_{1*}(t)={}\\
{}=\begin{cases}
1, &\mbox{если } Q(p_0(t),p_1(t),p_2(t))>0\,;
\\
u^{(0)}_1(t),  &\mbox{если } Q(p_0(t),p_1(t),p_2(t))=0\,;
\\
0, &\mbox{если } Q(p_0(t),p_1(t),p_2(t))<0,
\end{cases}
\label{e8-gor}
\end{multline}
где 
\begin{multline*}
Q(t)=Q\left(p_0(t),p_1(t),p_2(t)\right)={}\\
{}=-l^{(1)}_0 \rho p_0(t) \hm+ p_1(t) 
\hm- l^{(2)}_0 ( 1 \hm-\rho) p_2(t)\,.
\end{multline*}
Данная функция называется функцией переключений в~рассматриваемой задаче.
В соотношении~(\ref{e8-gor}) через $u^{(0)}_1 (t)$ обозначено особое управ\-ле\-ние, 
возникающее при условии, когда $Q(p_0(t),p_1(t),p_2(t))\hm=0$ и~функция Понтрягина 
не зависит явно от па\-ра\-мет\-ра управ\-ле\-ния~$u_1$. Особое управ\-ле\-ние не определяется 
из условий максимума функции Понтрягина, и~для его нахождения необходимо отдельное 
исследование. Аналитические представления для особых режимов управ\-ле\-ния удается 
определить в~очень редких случаях. В~связи с~этим  ограничимся только стандартным 
вариантом структуры управ\-ле\-ния без особого режима.

Функция $Q(p_0(t),p_1(t),p_2(t))$ является непрерывно дифференцируемой 
функцией от аргументов $p_0$, $p_1$ и~$p_2$. В~свою очередь, сопряженные переменные 
$p_0(t)$, $p_1(t)$ и~$p_2(t)$ непрерывны  и~непрерывно дифференцируемы при всех 
значениях $t \hm\in [0,T]$, кроме точек разрыва функции $u_1(t)$, определяющей значение 
управления. Однако, не\-смот\-ря на эти хорошие аналитические свойства, поведение 
функции $Q(p_0(t),p_1(t),p_2(t))$ на конечном интервале  $[0,T]$ может быть 
достаточно сложным. В~част\-ности, теоретически возможен вариант, когда эта функция 
бесконечное чис\-ло раз меняет знак, что приведет к~необходимости бесконечного чис\-ла 
переключений управления на данном конечном интервале времени. С~точки зрения 
экономического содержания такое управ\-ле\-ние неоправданно и~практически не реализуемо. 
В~связи с~этим в~данной работе будет рас\-смат\-ри\-вать\-ся только вариант, когда функция  
$Q(p_0(t),p_1(t),p_2(t))$ лишь конечное чис\-ло раз меняет знак на интервале времени 
$[0,T]$.

Будем предполагать, что точки, в~которых\linebreak функция $Q(p_0(t),p_1(t),p_2(t))$ 
изменяет знак, изолированы. Из соотношения~(\ref{e8-gor}) следует, что в~указанных 
точках характер управ\-ле\-ния должен \mbox{изменяться}. В~связи с~этим такие точки на 
рас\-смат\-ри\-ва\-емом интервале времени $[0,T]$ будем называть точками переключения 
управления. Обозначим точки переключения через $\tau_1,\tau_2,\ldots,\tau_n$, 
$0\hm<\tau_1\hm<\tau_2<\cdots<\tau_{n-1}\hm<\tau_n\hm<T$, $\tau_0\hm=0$, 
$\tau_{n+1}\hm=T$. Аналитические представления решений систем сопряженных уравнений 
будут зависеть от того, является ли число переключений~$n$~четным или нечетным, 
а~также от знака функции $Q(p_0(t),p_1(t),p_2(t))$ на начальном интервале времени 
$[0,\tau_1)$. В~соответствии с~этим рассмотрено четыре возможных варианта структуры 
оптимального управления. Для каждого из них получены аналитические решения системы 
сопряженных уравнений и~уравнений дифференциальной связи. 
В~данной работе приводится только часть полученных результатов.

\section{Аналитические решения для~функций состояний и~сопряженных переменных}

В ходе дальнейшего изложения будем использовать для функции переключений
$Q(p_0(t),p_1(t),p_2(t))$ наряду с~указанным также и~краткое обозначение 
$Q(t)\hm=Q(p_0(t)$, $p_1(t),p_2(t))$.

 Для краткости изложения выпишем решения уравнений системы сопряженных 
 уравнений только для одного из вариантов, остальные имеют аналогичный вид. 
 Пусть число переключений нечетно $n\hm=2m\hm-1$, $m \hm\geq 1$~--- 
 заданное целое число:
\begin{equation} 
\left.
\begin{array}{rl}
Q(t)&>0 \ \mbox{при} \ 0 \leq t < \tau_1\,; \\[6pt]
Q(t)&<0 \ \mbox{при} \ \tau_1 < t < \tau_2\,;\\[6pt]
\ldots&\ldots\ldots\ldots\ldots\ldots\ldots\ldots\\[6pt]
Q(t)&<0 \ \mbox{при} \ \tau_{2m-1} < t < \tau_{2m}=T\,.
\end{array}
\right\}
\label{e9-gor}
\end{equation}
Для данного варианта функция управления имеет вид:
\begin{equation}
u_{1*}(t)=
\begin{cases}
1\,, &\  \tau_{2j-2} \leq t < \tau_{2j-1}\,,\\
&\hspace*{20mm} j=1,2,\ldots,m\,;
\\
0\,, &\ \tau_{2j-1}\leq t< \tau_{2j}\,, \\
&\hspace*{20mm} j=1,2,\ldots,m\,.
\end{cases}
\label{e10-gor}
\end{equation}

\noindent
\textbf{Теорема~2.}\  
\textit{Предположим, что в~исходной задаче оптимального управления функция 
переключений  $Q(t)$ удовлетворяет условиям}~(\ref{e9-gor}), 
\textit{а~соответствующая функция управления $u_{1*}(t)$ задается формулой}~(\ref{e10-gor}). 
\textit{Тогда решение системы сопряженных уравнений определяется формулами}:
\begin{equation*} 
p^{(2j-1)}_0(t)=p_{0,\tau_{2j}} e^{\lambda_0(t-\tau_{2j})}\,;
%\label{e11-gor}
\end{equation*}

\vspace*{-12pt}

\noindent
\begin{multline*}
p^{(2j-1)}_1(t)=  e^{\lambda_1 t} \left [ 
\vphantom{\int\limits^{\tau_{2j}}_{z_3}}
p_{1,\tau_{2j}}e^{-\lambda_1 \tau_{2j}} +{}\right.\\
{}+
A_1\alpha_1 \int\limits^{\tau_{2j}}_t e^{-\lambda_1 z_3}
k^{\alpha_1-1}_1(z_3) \!\left(\!
\vphantom{\int\limits^{\tau_{2j}}_{z_3}}
l^{(1)}_0 \rho p_{0,\tau_{2j}} e^{\lambda_0
\left(z_3 -\tau_{2j}\right)} +{}\right.\hspace*{-0.9pt}
\end{multline*}

\noindent
\begin{multline*}
{}+l^{(2)}_0 (1- \rho) \left(
\vphantom{\int\limits^{\tau_{2j}}_t e^{(-\delta -\lambda_2)z_1}}
e^{\lambda_2(z_3-\tau_{2j})}
p_{2,\tau_{2j}}+{}\right.\\
\left.\left.\left.{}+e^{\lambda_2 z_3}  B_2
\int\limits^{\tau_{2j}}_{z_3} e^{(-\delta -\lambda_2)z_4}k^{\alpha_2-1}_2(z_4) \,dz_4\right) 
\right) \, dz_3 \right]\,;
%\label{e11a-gor}
\end{multline*}

\vspace*{-12pt}

\noindent
\begin{multline*}
p^{(2j-1)}_2(t)=e^{\lambda_2t}\left[
\vphantom{\int\limits^{\tau_{2j}}_t e^{(-\delta -\lambda_2)z_1}}
e^{-\lambda_2 \tau_{2j}}p_{2,\tau_{2j}}+ {}\right.\\
\left.{}+ B_2
\int\limits^{\tau_{2j}}_t e^{(-\delta -\lambda_2)z_1}k^{\alpha_2-1}_2(z_1) \,dz_1 \right]\,, 
\\
\tau_{2j-1} \leq t \leq \tau_{2j}\,,
%\label{e11b-gor}
\end{multline*}
\textit{где значения $p_{i,\tau_{2j}}$, $i=0,1,2$, определяются на интервале 
$[\tau_{2j},\tau_{2j+1}]$ при $t\hm=\tau_{2j}$, $j=1,2,\ldots,m$, 
причем $p_{i,\tau_{2m}}\hm=\psi^{(0)}_{i}(T)$, $i\hm=0,1,2$};
\begin{equation*}
p^{(2j-2)}_0(t)=p_{0,\tau_{2j-1}} e^{\lambda_0(t-\tau_{2j-1})}\,;
%\label{e12-gor}
\end{equation*}

\vspace*{-12pt}

\noindent
\begin{multline*}
p^{(2j-2)}_1(t)= p_{1,\tau_{2j-1}}\times{}\\
{}\times e^{ A_1\alpha_1 \hspace*{-2mm}
\int\limits^{ \tau_{2j-1}}_t\hspace*{-2.5mm} k^{\alpha_1-1}_1(z_2) \,dz_2 + 
\lambda_1 (t-\tau_{2j-1})}\,;
%\label{e12a-gor}
\end{multline*}

\vspace*{-12pt}

\noindent
\begin{multline*}
p^{(2j-2)}_2(t)=e^{\lambda_2t} \left[ 
\vphantom{\int\limits^{\tau_{2j-1}}_t}
p_{2,\tau_{2j-1}} 
e^{-\lambda_2 \tau_{2j-1}}+{}\right.\\
\left.{}+B_2
 \int\limits^{\tau_{2j-1}}_t 
e^{(-\delta -\lambda_2)z_1}k^{\alpha_2-1}_2(z_1) \,dz_1 \right]\,,\\
\tau_{2j-2} \leq t \leq \tau_{2j-1}\,,
%\label{e12b-gor}
\end{multline*}
\textit{где значения $p_{i,\tau_{2j-1}}$ определяются равенствами}
$$
p_{i,\tau_{2j-1}}=p_i^{(2j-1)}(\tau_{2j-1}),\ i=0,1,2,\ j=1,2,\ldots,m.
$$

Аналогичные утверждения доказаны для системы уравнений дифференциальной связи. 
Приведем формулировку одного из них, соответствующего рассматриваемому варианту 
структуры функции управления.

\smallskip

\noindent
\textbf{Теорема~3.}\ 
\textit{Предположим, что в~исходной задаче оптимального управления функция 
переключений $Q(t)$ удовлетворяет условиям}~(\ref{e9-gor}), 
\textit{а~соответствующая функция управления $u_{1*}(t)$ задается формулой}~(\ref{e10-gor}). 
\textit{Тогда решение системы уравнений дифференциальной связи определяется формулами}:
\begin{equation}
\hspace*{-1.8mm}\left.
\begin{array}{rl}
k^{(2j-2)}_0(t)&=k_{0,\tau_{2j-2}}e^{-\lambda_0 (t- \tau_{2j-2})}\,;
\\[6pt]
k^{(2j-2)}_1(t)&=\left(
\vphantom{\fr{A_1}{\lambda_1}}
e^{-\lambda_1(1-\alpha_1)(t- \tau_{2j-2})}
\times{}\right.\\[6pt]
&\hspace*{-5mm}\left.{}\times \left( k^{1-\alpha_1}_{1,\tau_{2j-2}}
-
\fr{A_1}{\lambda_1}\right)
+\fr{A_1}{\lambda_1}\right)^{1/({1-\alpha_1})}\,;
\\[6pt]
k^{(2j-2)}_2(t)&=k_{2,\tau_{2j-2}}e^{-\lambda_2 \left(t- \tau_{2j-2}\right)}\,, 
\\[6pt]
&\hspace*{23mm} \tau_{2j-2}\leq t\leq \tau_{2j-1}\,,
\end{array}
\right\}\!\!
\label{e13-gor}
\end{equation}
где значения $k_{i,\tau_{2j-2}}$, $i\hm=0,1,2$, определяются на интервале 
$[\tau_{2j-3}, \tau_{2j-2}]$ при $t\hm=\tau_{2j-2}$, $j\hm=2,3,\ldots ,m$, 
причем $k_{i,\tau_{0}}\hm=k_{i,0}$, $i\hm=0$, 1, 2, при $j\hm=1$;
\begin{equation}
\hspace*{-2mm}\left.
\begin{array}{rl}
k^{(2j-1)}_0(t)&=e^{-\lambda_0t}\left(
\vphantom{\fr{l^{(1)}_0\rho A_1 k^{\alpha_1}_{1,\tau_{2j-1}} 
{e^{ \lambda_1 \alpha_1 \tau_{2j-1}}} }{\lambda_0-\lambda_1\alpha_1} }
k_{0,\tau_{2j-1}} e^{\lambda_0 \tau_{2j-1}}+{}\right.\\[6pt]
&\hspace*{-17mm}{}+
\fr{l^{(1)}_0\rho A_1 k^{\alpha_1}_{1,\tau_{2j-1}} 
{e^{ \lambda_1 \alpha_1 \tau_{2j-1}}} }{\lambda_0-\lambda_1\alpha_1} 
\big(
e^{(\lambda_0-\lambda_1\alpha_1) t} -{}\\[6pt]
&\left.{}- e^{(\lambda_0-\lambda_1\alpha_1) 
\tau_{{2j-1}}}
 \big) 
\vphantom{\fr{l^{(1)}_0\rho A_1 k^{\alpha_1}_{1,\tau_{2j-1}} 
{e^{ \lambda_1 \alpha_1 \tau_{2j-1}}} }{\lambda_0-\lambda_1\alpha_1} }
\right)\,;
\\[6pt]
k^{(2j-1)}_1(t)&=k_{1,\tau_{{2j-1}}}e^{-\lambda_1 (t- \tau_{{2j-1}})}\,;
\\[6pt]
k^{(2j-1)}_2(t)&=e^{-\lambda_2 t}\left(
\vphantom{\fr{l^{(1)}_2(1-\rho) A_1 k^{\alpha_1}_{1,\tau_{2j-1} 
{e^{ \lambda_1 \alpha_1 \tau_{2j-1}}}} }
{\lambda_2-\lambda_1\alpha_1}}
k_{2,\tau_{2j-1}} e^{\lambda_2 \tau_{2j-1}}+{}\right.\\
&\hspace*{-18.5mm}{}+
\fr{l^{(1)}_2(1-\rho) A_1 k^{\alpha_1}_{1,\tau_{2j-1} 
{e^{ \lambda_1 \alpha_1 \tau_{2j-1}}}} }
{\lambda_2-\lambda_1\alpha_1} \big(
e^{(\lambda_2-\lambda_1\alpha_1) t} -{}\\[6pt]
&\left.\hspace*{-10mm}{}- e^{(\lambda_2-\lambda_1\alpha_1) \tau_{2j-1}} 
\big) 
\vphantom{\fr{l^{(1)}_2(1-\rho) A_1 k^{\alpha_1}_{1,\tau_{2j-1} 
{e^{ \lambda_1 \alpha_1 \tau_{2j-1}}}} }
{\lambda_2-\lambda_1\alpha_1}}
\right)\,,
\ \tau_{2j-1}\leq t\leq \tau_{2j}, 
\end{array}\!\!\!
\right\}\!\!\!\!
\label{e14-gor}
\end{equation}
где значения $k_{i, \tau_{2j-1}}$ определяются равенствами
\begin{multline*}
k_{i, \tau_{2j-1}} = k_i^{(2j-2)}(\tau_{2j-1})\,,\\
 i=0,1,2\,,\quad j=1,2,\ldots,m\,.
\end{multline*}

Таким образом, формулы~(\ref{e13-gor}) и~(\ref{e14-gor}) 
пол\-ностью определяют аналитические представления для функций состояний 
в~рассматриваемом варианте поведения функции переключений.

Явные аналитические пред\-став\-ле\-ния для сопряженных переменных $p_0(t)$, 
$p_1(t)$ и~$p_2(t)$ и~функций\linebreak состояний $k_0(t)$, $k_1(t)$ и~$k_2(t)$ получены также 
и~для\linebreak трех остальных вариантов поведения функции переключений в~случае, когда 
чис\-ло переключений~$n$~является произвольным положительным\linebreak чис\-лом $n \hm\geq 1$. 
Отметим, что соответствующие пред\-став\-ле\-ния для сопряженных переменных и~функций 
состояний в~случае отсутствия переключений $n\hm=0$ и~в~случае одного переключения 
$n\hm=1$ были приведены в~работе~\cite{1-gor}.

Полученные результаты позволяют последовательно вычислить значения функций 
состояний~$k_0(t)$, $k_1(t)$ и~$k_2(t)$, а~затем сопряженных 
переменных~$p_0(t)$, $p_1(t)$ и~$p_2(t)$ в~любой точке $t \hm\in [0,T]$. Но тогда и~функция 
переключений $Q(p_0(t),p_1(t),p_2(t))$, определяемая через сопряженные переменные, 
также может быть вычислена в~любой точке $t \hm\in [0,T]$ для любого варианта функции 
управления $u_{1*}(t)$ без переключений или с~произвольным конечным числом точек 
переклю\-чения.
{\looseness=1

}

\section{Описание численного алгоритма решения задачи оптимального управления}

Воспользовавшись обстоятельством, отмеченным в~конце предыдущего раздела,  
можно предложить численный алгоритм определения функции управления $u_{1*}(t)$, 
удовлетворяющей необходимым условиям экстремума в~форме принципа максимума. Идея 
этого алгоритма заключается в~сле\-ду\-ющем.

Для каждого из рассмотренных вариантов\linebreak структуры управления необходимо
проанализировать поведение функции переключения $Q(t)$ на всем интервале времени
$t\hm\in[0,T]$.
Такой анализ можно осуществлять численными
методами компьютерных программ, вычисляющих значения
функции~$Q(t)$ в~отдельных точках на интервале\linebreak $t\hm\in[0,T]$.
%
 При этом задается некоторое разбиение  интервала $[0,T]$ конечным числом точек,\linebreak 
 в~которых будут вычисляться значения функции\linebreak переключений $Q(p_0(t),p_1(t),p_2(t))$. 
 Программа, вы\-чис\-ля\-ющая значения функции $Q(t)$, должна использовать подпрограммы, 
 вычисляющие значения функций $p_0(t)$, $p_1(t)$ и~$p_2(t)$. В~свою очередь,\linebreak подпрограммы, 
 в~результате выполнения которых вычисляются значения сопряженных переменных, 
 должны реализовать аналитические представления этих функций, зависящих от функций 
 со\-сто\-яний~$k_0(t)$, $k_1(t)$ и~$k_2(t)$, полученные в~ходе проведенного исследования.

Если выявленное поведение функции переключения $Q(t)$ соответствует выбранной структуре
функции управления $u_1(t)$, то рассматриваемый вариант функции управления
и соответствующих функций состояний системы $k_0(t)$, $k_1(t)$ и~$k_2(t)$
можно считать управляемым процессом, который удовле\-тво\-ря\-ет системе соотношений, 
состоящей из необходимых условий экстремума и~ограничений исходной задачи. 
При этом соответствие функции переключений $Q(t)$ и~выбранного варианта функции 
управления $u_1(t)$ понимается в~смысле выполнения соотношения~(\ref{e8-gor}).

Перейдем теперь к~последовательному описанию предлагаемого алгоритма нахождения 
до\-пус\-ти\-мых экстремалей в~исходной задаче оптимального управления.
\begin{enumerate}[1.]
\item Выберем в~качестве начального варианта функции управления $u_{1}(t)$ 
один из вариантов с~отсутствием переключений.\\[-15pt]

\item Для выбранного варианта вычислим значения функций  $k_0(t)$, $k_1(t)$ и~$k_2(t)$ 
по имеющимся ана-\linebreak\vspace*{-12pt}

\pagebreak

\noindent
литическим формулам (см.\ формулы~(19) и~(21) работы~\cite{1-gor}).

\item Используя значения выбранной функции управ\-ле\-ния $u_{1*}(t)$  и~найденные 
значения функций $k_0(t)$, $k_1(t)$ и~$k_2(t)$ вычислим значения\linebreak сопряженных 
функций $p_0(t)$, $p_1(t)$ и~$p_2(t)$ по име\-ющим\-ся аналитическим формулам (см.\ 
формулы~(29) и~(30) работы~\cite{1-gor}).

\item Используя найденные  значения сопряженных переменных $p_0(t)$, $p_1(t)$
и~$p_2(t)$, определим чис\-лен\-ное представление функции переключений $Q(t)\hm=Q(p_0(t),
p_1(t),p_2(t))$.

\item Осуществим проверку соответствия характера вычисленной функции $Q(t)$ 
выбранному исходному варианту функции управления. В~данном случае необходимо проверить 
выполнение одного из условий: либо условия $Q(t)\hm>0$, $t \hm\in [0,T]$, если в~п.~1 
данной процедуры был выбран вариант управления $u_{1}(t)\hm=1$, $t \hm\in [0,T]$, 
либо условия $Q(t)\hm<0$, $t \hm\in [0,T]$, если в~п.~1 данной процедуры был 
выбран вариант управ\-ле\-ния $u_{1}(t)\hm=0$, $t \hm\in [0,T]$. Отметим, что 
указанные условия на функцию $Q(t)$ должны быть проверены для всех значений 
аргумента~$t$, принадлежащих заданному изначально разбиению отрезка $[0,T]$.

\item Если указанное условие выполняется, то можно\linebreak считать, что выбранный в~п.~1 
данной проце\-дуры вариант функции управления удовле\-творяет необходимым условиям 
экстремума  в~форме принципа максимума и~ограничениям исходной задачи. 
Тогда управляемый процесс $(u_{1*}(t), k_0(t)$, $k_1(t)$, $k_2(t))$ 
представляет собой допустимую экстремаль в~рассматриваемой задаче оптимального 
управления. Данный вариант управляемого процесса сохраним в~специально отведенном 
месте памяти. Перейдем к~следующему варианту функции управления.

\item Предположим теперь, что проверяемое условие не выполняется. Для определенности 
будем считать, что при выбранном варианте управления $u_{1}(t)\hm=1$,  $t \hm\in [0,T]$ 
функция $Q(t)$ меняет знак в~некоторой точке~$\tau$: $Q(t)\hm>0$, $t \hm\in [0,\tau)$, 
$Q(t)\hm<0$, $t \hm\in (\tau,T]$.
Такой результат означает, что при заданных исходных параметрах модели управляемый 
процесс, состоящий из функции $u_{1}(t)$ и~соответствующих функций 
состояний~$k_0(t)$, $k_1(t)$ и~$k_2(t)$, не удовлетворяет системе соотношений, 
состоящей из необходимых условий экстремума и~ограничений исходной задачи. 
Следует перейти к~анализу другого варианта функции управления.

\item В качестве следующего варианта функции управ\-ле\-ния можно выбрать тот, 
который наиболее близок к~результату, описанному в~п.~7, а~именно:
\begin{equation} 
u_{1}(t)=\begin{cases}
1 &\mbox{ при } 0\leq t \leq \tau\,;
\\
0  &\mbox{ при } \tau \leq t \leq T\,.
\end{cases}
\label{e15-gor}
\end{equation}
В то же время такой выбор нельзя обосновать строго. Можно осуществить переход 
к~любому из вариантов управления с~одной точкой переключения вида~(\ref{e15-gor}), 
где~$\tau$~--- произвольная внутренняя точка из интервала $[0,T]$, совпадающая 
с~одной из точек заданного разбиения отрезка $[0,T]$.

\item Для нового варианта управления вида~(\ref{e15-gor}) повторяются действия, 
описанные в~пп.~2--4 данной процедуры.

\item  Проверяется условие соответствия поведения вычисленной функции $Q(t)$ 
и~выбранного варианта~(\ref{e15-gor}) функции управления.

Если поведение функции $Q(t)$ соответствует выбранному варианту функции управления,
а~именно: выполняется условие $Q(t)\hm>0$, $0 \hm\leq t \hm\leq \tau$, $Q(t)\hm<0$, 
$\tau \hm< t \hm\leq T$, $Q(\tau)\hm=0$, то можно утверждать что выбранный вариант 
функции управления $u_{1*}(t)$ вида~(\ref{e15-gor}) и~соответствующие ему функции 
состояний~$k_{0*}(t)$, $k_{1*}(t)$ и~$k_{2*}(t)$ удовлетворяют системе, состоящей из 
необходимых условий экстремума и~ограничений исходной задачи. Управ\-ля\-емый 
процесс\linebreak $(u_{1*}(t);k_{0*}(t),k_{1*}(t),k_{1*}(t))$ является допустимой 
экстремалью в~исходной задаче оптимального управления. Данный вариант 
управля\-емо\-го процесса сохраняется в~специально\linebreak отведенном месте памяти, 
после чего происходит переход к~анализу следующего варианта функции управления.

\begin{figure*}[b] %fig1
 \vspace*{6pt}
 \begin{center}
 \mbox{%
 \epsfxsize=127.491mm
 \epsfbox{gor-1.eps}
 }
 \end{center}
 \vspace*{-6pt}
\Caption{Теоретическая схема реализации алгоритма}
\end{figure*}


Если же поведение вычисленной функции  $Q(t)$ не соответствует характеру выбранного 
варианта функции управления вида~(\ref{e15-gor}), то набор, состоящий из 
функций $u_{1}(t)$ и~соответству\-ющих ей функций состояний $k_{0}(t)$,
$k_{1}(t)$ и~$k_{2}(t)$, 
не удовле\-тво\-ря\-ет системе соотношений, состоящей из необходимых условий экстремума 
и~ограничений исходной задачи. Данный набор не является допустимой экстремалью.

\item Производится переход к~следующему варианту функции управления $u_{1}(t)$ из 
рассматриваемого множества функций управления. При этом прежде всего перебираются 
все варианты функции управления с~одной точкой переключения вида~(\ref{e15-gor}), 
в~которых точка~$\tau$ пробегает заданное множество точек, составляющих разбиение 
отрезка $[0,T]$.

\item Для каждого нового варианта функции управления производятся аналогичные 
действия, описанные в~пп.~2--4 данного алгоритма. После этого вновь проверяется 
соответствие поведения вычисленной функции $Q(t)$  выбранному варианту функции 
управления. Выводы, которые могут быть сделаны по результатам проверки такого 
соответствия, аналогичны выводам, сформулированным в~пп.~6, 7 и~10 данного алгоритма.

\item Указанные действия производятся для всех вариантов функции управления, входящих 
в~некоторое множество функций, заданных на отрезке $[0,T]$. Более подробно данное 
множество будет описано ниже. Сейчас отметим лишь, что это множество конечно 
и~данный алгоритм будет реализован за конечное число шагов.
\end{enumerate}

Иллюстрацией к~описанному выше алгоритму нахождения допустимых экстремалей 
в~исходной задаче оптимального управления может служить схема, изображенная на рис.~1.


Сделаем важное замечание, относящееся к~п.~13 приведенного алгоритма. Из общей 
тео\-рии экстремальных задач известно, что может существовать несколько допустимых 
экстремалей, удовле\-тво\-ря\-ющих необходимым условиям экстремума 
\mbox{и~ограничениям} исходной 
задачи. Таким образом, процедура поиска должна продолжаться с~использованием 
целенаправленного перебора возможных вариантов поведения функции управ\-ле\-ния. 
Указанный перебор должен осуществляться при помощи изменения положения точек 
переключения и~чис\-ла этих точек. Основываясь на общих особенностях функций 
оптимального управления, следующих из условия максимума и~соотношения~(\ref{e8-gor}), 
а~так\-же на сделанных в~дальнейшем дополнительных предположениях о~характере возможных 
управ\-ле\-ний, можно предложить следующее описание основных аналитических особенностей 
множества возможных вариантов функции управления $u_1(t)$, на котором реализуется 
описанный алгоритм. Данный объект должен представлять собой мно\-жество функций, 
заданных на отрезке $[0,T]$, при\-ни\-ма\-ющих два возможных значения:~0~или~1~--- 
и~имеющих не более заданного конечного чис\-ла\linebreak точек разрыва первого рода (скачков). 
Для определенности будем предполагать также, что эти функции являются непрерывными 
справа.

Проведем теперь формальное описание множества возможных функций управления, 
на котором действует предлагаемый алгоритм.
Зафиксируем целое положительное число $N \hm\geq 1$. Пусть $\Delta \hm= {T}/({N+1})\hm > 0$. 
Обозначим через $t_i \hm= i \Delta$,\linebreak $i\hm=0,1,2,\ldots,N,N+1$,  точки, 
принадлежащие\linebreak отрезку $[0,T]$, $t_0\hm=0$, $t_{N+1}\hm=(N+1) \Delta \hm= T$. 
Совокупность точек $\{t_1,t_2,\ldots,t_N\}$ будем называть раз\-би\-ени\-ем отрезка $[0,T]$. 
В~каждой реализации алгоритма параметры~$T$ и~$N$, а~следовательно, и~все\linebreak раз\-би\-ение 
данного отрезка времени, должны быть заданы. Будем предполагать, что заданное 
разбиение $\{t_1,t_2,\ldots,t_N\}$ представляет собой мно\-жест\-во возможных точек 
переключения у~рас\-смат\-ри\-ва\-емых функций управления.  При этом граничные точки данного 
отрезка $t_0\hm=0$ и~$t_{N+1}\hm=T$ не считаются точками переключения.

Зафиксируем целое положительное число~$n$,\linebreak $1 \hm\leq n \hm\leq N$. Выберем~$n$~различных 
точек из разбиения $\{ t_1, t_2,\ldots, t_N  \}$. Обозначим через~$S^{(N)}_n$ множество 
возможных функций управления $u_1(t)$, обла\-да\-ющих следующими свойствами:
\begin{itemize}
\item[(а)] данные функции принимают только два возможных значения:~0 или~1;

\item[(б)] данные функции являются ку\-соч\-но-по\-сто\-янными и~меняют свое значение 
в~точках переключения. В~точках переключения эти функции обладают свойством 
непрерывности\linebreak справа;

\item[(в)] точками переключения для функций, принадлежащих множеству~$S^{(N)}_n$, 
может служить любой набор, состоящий из~$n$~различных точек, принадлежащих разбиению 
$\{ t_1, t_2,\ldots, t_N  \}$.
\end{itemize}

Совокупность функций~$S^{(N)}_n$, обладающих указанными свойствами, будет 
рассматриваться как набор возможных функций управления с~$n$~переключениями, где 
$1 \hm\leq n \hm\leq N$. Для удобства введем также множество $S_0\hm=S^{(N)}_0$, 
состоящее из двух функций: $u_1(t)\hm=0$, $0 \hm\leq t \hm\leq T$, 
и~$u_1(t)\hm=1$, $0 \hm\leq t \hm\leq T$. Функции управления, входящие в~множество~$S^{(N)}_0$, 
не имеют переключений. Очевидно, что при $n_1  \hm\neq  n_2$ множества~$S^{(N)}_{n_1}$
и~$S^{(N)}_{n_2}$ являются непересекающимися.

Теперь рассмотрим объединение множеств
\begin{equation*} 
\hat S^{(N)}_n =  \bigcup\limits_{k=0}^{n} S^{(N)}_{k}\,.
\end{equation*}

По содержанию множество~$\hat S^{(N)}_n$ представляет собой совокупность функций, 
обладающих свойствами~(а) и~(б), указанными выше, для которых точками переключения 
может служить любой набор, состоящий не более чем из~$n$~различных точек, принадлежащих 
множеству $\{ t_1, t_2,\ldots, t_N  \}$. Именно такие наборы функций будут 
рассматриваться в~данной алгоритмической части настоящего исследования как 
множества возможных функций управления. Для каждой реализации изложенного выше 
алгоритма задаются параметры~$N$ и~$n$, т.\,е.\ шаг разбиения~$\Delta$ 
и~максимально возможное число точек переключения $n \hm\leq N$.

Докажем утверждение о числе элементов множества $\hat S^{(N)}_n$.

\noindent
\textbf{Утверждение~1.}\ 
Число элементов множества $\hat S^{(N)}_n$ определяется следующей формулой:
\begin{equation*} 
\hat N_n =  2 \sum \limits_{k=0}^{n} C^{k}_{N}\,.
\end{equation*}

\noindent
Д\,о\,к\,а\,з\,а\,т\,е\,л\,ь\,с\,т\,в\,о\,.\ \ 
Определим число элементов в~каждом из множеств, входящих в~множество~$\hat S^{(N)}_n$. 
Множество~$S^{(N)}_0$ состоит из двух элементов. Рассмотрим множество~$S^{(N)}_k$ 
с~фиксированным номером~$k$. Число возможных вариантов точек переключения для 
функций из данного множества совпадает с~числом возможных вариантов выбора~$k$~различных 
элементов из~$N$ без возвращения и~без учета порядка элементов. Указанное число 
вариантов переключения равно числу сочетаний $C^k_N \hm= {N!}/({k! (N - k)!})$. 
Каждому фиксированному набору точек переключения соответствуют два варианта 
возможного вида функции управления, которые отличаются значением функции на 
интервале от $t\hm=0$ до первой точки переключения. Таким образом, 
множество~$ S^{(N)}_k$ содержит~$2 C^k_N$~элементов. Поскольку множества~$S^{(N)}_k$ 
не пересекаются при различных~$k$, чис\-ло элементов в~объединении множеств равно 
сумме чис\-ла элементов в~каждом множестве. Отсюда следует, что
\begin{equation*} 
\hat N_n = 2 + 2 C^1_N + 2 C^2_N + \cdots + 2 C^n_N = 2 \sum \limits_{k=0}^{n} C^{k}_{N}\,.
\end{equation*}
Утверждение~1 доказано.

Доказанное утверждение позволяет точно определить число вариантов функции 
управления, которые необходимо исследовать в~ходе реализации предложенного алгоритма. 
Если удается ка\-ким-ли\-бо образом, например при помощи численного эксперимента, 
оценить время, необходимое для анализа каждого варианта, то полученная формула 
позволит оценить общее время работы программы, зависящее от параметров~$N$ и~$n$.


\section{Программная реализация алгоритма и~результаты вычислений}

Предложенный алгоритм численного решения задачи оптимального управления реализован 
в~виде программного комплекса. Комплекс состоит из нескольких функциональных частей, 
которые можно назвать модулями. Перечислим эти модули в~той последовательности, 
в~которой происходит реализация всего алгоритма.
\begin{enumerate}[1.]
\item Модуль, непосредственно реализующий аналитические формулы для функций 
состояний~$k_0(t)$, $k_1(t)$ и~$k_2(t)$.
\item Модуль, непосредственно реализующий аналитические формулы для сопряженных 
переменных $p_0(t)$, $p_1(t)$  и~$p_2(t)$. В~этом модуле используются результаты вычислений 
по программным продуктам первого модуля.
\item Модуль, реализующий аналитическое пред\-став\-ле\-ние для функции 
переключения $Q(t)\hm=Q(p_0(t),p_1(t),p_2(t))$. В~этом модуле используются результаты 
вычислений из модуля~1\linebreak и~модуля~2.
\item  Модуль, реализующий перебор некоторого заданного множества возможных 
функций управ-\linebreak\vspace*{-12pt}

\pagebreak

\noindent
ле\-ния и~проверку соответствия каждого варианта функции управления 
характеру функции переключения, вычисляемой для этого варианта.
В~этом модуле используется результаты вычислений модуля~3.
\end{enumerate}

В результате использования данного комплекса определяются все возможные 
варианты управля\-емых процессов, состоящих из некоторой функции управления $u_{1*}(t)$ 
и~соответствующих ей функций состояний $(k_{0*}(t),k_{1*}(t),k_{2*}(t))$, которые 
удовлетворяют системе, состоящей из необходимых условий экстремума~(\ref{e5-gor})--(\ref{e7-gor}) 
и~ограничений исходной задачи~(\ref{e2-gor})--(\ref{e4-gor}). 
Каждый из таких управляемых процессов $\left(u_{1*}(t);k_{0*}(t),k_{1*}(t),k_{2*}(t)\right)$ 
представляет собой допустимую экстремаль для заданного набора исходных параметров 
математической модели.

Отметим еще одну важную особенность программного комплекса, реализующего 
разработанный алгоритм. Данная особенность связана с~организацией выбора 
точек переключения и~всей функции управления~$u_{1*}(t)$. В~программном комплексе 
предусмотрены три возможности такого выбора и~вычисления функций 
состояний $(k_{0}(t),k_{1}(t),k_{2}(t))$, сопряженных 
переменных $(p_{0}(t),p_{1}(t),p_{2}(t))$ и~функции переключений 
$Q(t)\hm=Q(p_0(t),p_1(t),p_2(t))$, соответствующих выбранной функции управления:
\begin{enumerate}[1.]
\item Случайный выбор точек переключения $(t_1,t_2,\ldots,t_n)$. При этом задается 
чис\-ло $n \hm\geq 1$ и~значение функции управления на начальном интервале $[0,t_1]$. 
Такой выбор условно называется стохастическим моделированием. Полученные результаты 
могут быть использованы для качественного анализа исходной задачи управления.

\item Задание конкретного набора точек переключения $(t_1,t_2,\ldots,t_n)$. 
При этом конкретный вид функции управления $u_1(t)$ определяется при помощи задания 
ее значения на начальном интервале времени $[0,t_1]$. Такая функция позволяет 
исследовать возможные варианты управ\-ля\-емых процессов по отдельности и~изучать 
\mbox{влияние} отдельных параметров на вид управляемого процесса.

\item Организация непосредственного перебора возможных вариантов функций 
управления $u_1(t)$, входящих в~определенный класс~$S^{(N)}_k$, 
$k\hm=0,1,\ldots,n$. При этом должно быть задано число точек переключения~$k$ 
и~значение функции управления $u_1(t)$ на начальном интервале времени $[0,t_1]$. 
Все наборы точек переключения функции управления, входящих в~указанный класс, 
перебираются в~ходе реализации программы. Пользователю необходимо лишь ввести 
информацию о числе точек переключения, а~так\-же о~значении функции управления 
на начальном интервале, которое может быть равно~0 или~1. 
В~результате действия этой функции программного комплекса определяются все 
допустимые экстремали $u_{1*}(t)$ , входящие в~рассматриваемый класс~$S^{(N)}_k$, 
и~соответствующие им функции состояний  $(k_{0*}(t),k_{1*}(t),k_{2*}(t))$. 
Данная функция предназначена для непосредственного численного решения поставленной 
задачи оптимального управления, т.\,е.\ нахождения всех допустимых экстремалей.
\end{enumerate}

Теперь приведем краткое описание основного набора числовых значений исходных параметров 
рассматриваемой математической модели. Указанные числовые значения частично взяты 
из работы Колемаева~\cite{12-gor}, в~которой они были определены на основе 
анализа реальных данных. Указанные значения определены на основе анализа данных 
о~советской экономике за период с~1960 по~1991~гг.\ (в~це\-нах~1983~г.). 
Для описания функционирования национальной экономики СССР за этот период использовалась 
трехсекторная модель и~соответствующие значения параметров были определены для каждого 
сектора. В~част\-ности, в~работе~\cite{12-gor} было установлено, что численные значения 
параметров, входящих в~функции $f_j(k_j)\hm=A_j k^{\alpha_j}_j$, $j\hm=1,2$, принимают 
следующие числовые значения:
\begin{equation*} 
 A_1=1{,}35\,; \quad A_2=2{,}71\,; \quad \alpha_1=0{,}68\,; \quad  \alpha_2=0{,}72\,.
\end{equation*}

Коэффициенты $\mu_j$, $j\hm=0,1,2$, характеризующие скорость выбывания основных фондов 
в~различных секторах, принимают следующие значения:
$$
\mu_0=0{,}1\,; \enskip  \mu_1=0{,}3\,; \enskip \mu_2=0{,}2\,.
$$
Коэффициент $\nu$ характеризует скорость прироста объема 
трудовых ресурсов. Поскольку для различных национальных экономических систем такой 
параметр может быть очень различным, выберем условное расчетное значения для $\nu\hm=0{,}01$ 
исходя из предположения о~достаточно медленном темпе прироста объема трудовых ресурсов.

Поскольку $\lambda_j= \mu_j + \nu$, $j\hm=0,1,2$, получаем:
$$
\lambda_0=0{,}11\,; \quad \lambda_1=0{,}31\,; \quad \lambda_2=0{,}21\,.
$$

Параметр $\delta >0$, характеризующий скорость уменьшения реального 
содержания в~единице денежных ресурсов (показатель инфляции), выберем равным 
$\delta\hm=0{,}06$, что соответствует среднему темпу инфляции.

\begin{figure*}[b] %fig2
 \vspace*{1pt}
 \begin{center}
 \mbox{%
 \epsfxsize=160.121mm
 \epsfbox{gor-2.eps}
 }
 \end{center}
 \vspace*{-9pt}
\Caption{Графическое представление функций фондовооруженности $k_i(t)$~(\textit{а}),
сопряженных переменных $p_i(t)$~(\textit{б}), функции $Q(t)$~(\textit{в}) 
и~структура управления~(\textit{г})}
\end{figure*}

Величина $T$~--- длительность интервала управления, которая иногда также называется 
горизонтом планирования. Рассмотрим краткосрочный вариант, когда горизонт планирования 
представляет собой величину $T\hm=1$ условных единиц времени (лет).

В рассматриваемой модели величины  $\theta_0$, $\theta_1$ и~$\theta_2$ 
представляют собой доли трудовых ресурсов $j$-го сектора в~общем объеме 
трудовых ресурсов системы. Отметим, что в~упомянутой работе В.\,А.~Колемаева~\cite{12-gor} 
эти параметры также были определены на основе данных об экономике СССР 
за~1960--1991~гг., а~именно:
\begin{equation*} 
\theta_0=0{,}22\,; \quad \theta_1=0{,}16\,; \quad \theta_2=0{,}62\,.
\end{equation*}
Учитывая, что в~рассматриваемой задаче пара\-мет\-ры~$l^{(1)}_{0}$ и~$l^{(1)}_{2}$ 
выражаются через $\theta_0$, $\theta_1$ и~$\theta_2$  по формулам
\begin{equation*}
l^{(1)}_{0}=\fr{L_{1,0}}{L_{0,0}}=\fr{\theta_1}{\theta_0}\,; \quad
l^{(1)}_{2}=\fr{L_{1,0}}{L_{2,0}}=\fr{\theta_1}{\theta_2}\,,
\end{equation*}
получаем
\begin{gather*}
l^{(1)}_{0}=\fr{L_{1,0}}{L_{0,0}}=\fr{\theta_1}{\theta_0}=\fr{16}{22}=\fr{8}{11}\,; \\
l^{(1)}_{2}=\fr{L_{1,0}}{L_{2,0}}=\fr{\theta_1}{\theta_2}=\fr{16}{62}=\fr{8}{31}\,.
\end{gather*}

Зададим теперь начальные значения функций фондовооруженности (удельного капитала):
\begin{equation*} 
k_{0,0}=1000\,; \quad  k_{1,0}=2000\,; \quad k_{2,0}=1500\,.
\end{equation*}

Предположим, что в~рассматриваемой задаче терминальный член целевого функционала линейным образом зависит от значений удельного капитала $k_0(T)$, $k_1(T)$, $k_2(T)$:
\begin{multline*} 
\psi(k_0(T),k_1(T),k_2(T))={}\\
{}=a_0 k_0(T)+ a_1 k_1(T) + a_2 k_2(T)\,,
\end{multline*}
где $a_0$, $a_1$ и~$a_2$~--- заданные коэффициенты, характеризующие веса (вклады) 
величин $k_0(T)$, $k_1(T)$ и~$k_2(T)$ в~терминальную часть целевого функционала. Данные
 величины могут быть определены экс\-пер\-та\-ми-эко\-но\-миста\-ми. 
 В~этом случае производные терминального члена целевого функционала имеют вид:
\begin{align*}
\psi_{k_{0}}(k_0(T),k_1(T),k_2(T))&=\psi^{(0)}_{0}(T)=a_0\,;
\\
\psi_{k_{1}}(k_0(T),k_1(T),k_2(T))&=\psi^{(0)}_{1}(T)=a_1\,;
\\
\psi_{k_{2}}(k_0(T),k_1(T),k_2(T))&=\psi^{(0)}_{2}(T)=a_2\,.
\end{align*}
Величины $a_0$, $a_1$ и~$a_2$ определяют граничные условия для системы 
сопряженных уравнений относительно функций~$p_0(t)$, $p_1(t)$ и~$p_2(t)$ в~точке 
$t_{N+1}\hm=T$, 
т.\,е.\ условия трансверсальности. В~рассматрива\-емом случае
\begin{gather*}
p_0(T)=\psi^{(0)}_{0}(T)=a_0\,; \\ 
p_1(T)=\psi^{(0)}_{1}(T)=a_1\,; \enskip p_2(T)=\psi^{(0)}_{2}(T)=a_2\,.
\end{gather*}
Зададим численные значения коэффициентов~$a_0$, $a_1$ и~$a_2$ следующим образом:
$$
a_0=0{,}2\,; \enskip a_1=0{,}5\,; \enskip a_2=0{,}3\,.
$$

Наконец, зададим числовое значение величины~$\rho$, характеризующей 
распределение инвестиций между материальным и~потребительским секторами. 
Предположим, что $\rho\hm=0{,}2$.
Таким образом, все необходимые значения параметров трехсекторной модели экономики 
заданы.

Разбиение отрезка времени $[0,T]$ задается параметром $N\hm=99$, откуда 
$\Delta\hm={T}/({N+1})\hm=0{,}01$. Рассмотрено множество возможных функций 
управ\-ле\-ния~$\hat S^{(N)}_3$, состоящее из управлений, имеющих не более трех точек 
переключения на заданном интервале времени. В~результате установлено, что в~данном 
множестве имеется единственная функция управления $u_{1*}(t)\hm=1$, $t \hm\in [0,T]$, 
которая вместе с~соответствующими ей функциями состояний 
$(k_{0*}(t),k_{1*}(t),k_{2*}(t))$ является решением системы соотношений, 
содержащей необходимые условия и~ограничения исходной задачи. Графические 
представления для функции $u_{1*}(t)$, $t \hm\in [0,T]$, а~также соответствующих 
ей функций состояний $k_{0*}(t)$, $k_{1*}(t)$  и~$k_{2*}(t)$, сопряженных 
переменных $p_0(t)$, $p_1(t)$ и~$p_2(t)$ и~функции переключений 
$Q(t)\hm=Q(p_0(t),p_1(t),p_2(t))$ приведены на рис.~2.




Итак, для рассматриваемого набора численных значений исходных параметров 
в~поставленной задаче оптимального управления имеется единственная допустимая 
экстремаль, задаваемая функциями $(u_{1*(t)};k_{0*}(t),k_{1*}(t),k_{2*}(t))$, 
изображенными на рис.~2,\,\textit{а} и~2,\,\textit{г}. В~этом управляемом процессе переключения отсутствуют.

Отметим, что аналогичные результаты были получены еще для нескольких вариантов набора 
значений исходных параметров, частично отлича\-ющих\-ся от основного.

\vspace*{-6pt}

\section{Заключительные замечания}

Численный алгоритм, изложению и~анализу которого посвящена данная работа, 
позволяет завершить решение поставленной задачи оптимального управления. 
В~ходе реализации данного алгоритма используются программы, вычисляющие значения 
функции состояний и~сопряженных переменных. Алгоритм осуществляет перебор возможных 
вариантов функции управления и~выбирает те из них, которые являются допустимыми 
экстремалями. Таким образом, решение исходной задачи оптимального управления находится 
при помощи сочетания аналитических и~численных методов.

Предлагаемый алгоритм может быть использован не только в~рассмотренной задаче 
оптимального управления. Его можно использовать при решении различных задач, 
которые имеют следующие общие свойства:
\begin{itemize}
\item исследование задачи производится на основе принципа максимума;
\item структура оптимального управления определяется некоторой функцией переключений, 
зависящей от сопряженных переменных;
\item система сопряженных уравнений может зависеть, вообще говоря, от функций состояний и~управлений.
\end{itemize}

Такие общие свойства имеет весьма широкий класс задач оптимального управления, 
возникающих при анализе экономических и~технических систем.


\vspace*{-6pt}

{\small\frenchspacing
 {%\baselineskip=10.8pt
 \addcontentsline{toc}{section}{References}
 \begin{thebibliography}{99}

\bibitem{2-gor} 
\Au{Шнурков П.\,В., Засыпко В.\,В.}
Оптимальное управление инвестициями в~закрытой динамической модели трехсекторной 
экономики: математическая постановка задачи и~общий анализ на основе принципа максимума~// 
Вестник МГТУ им.\ Н.\,Э.~Баумана. Сер. Естественные науки,  2014. Вып.~2.  С.~101--115.

\bibitem{1-gor} %2 
\Au{Шнурков П.\,В., Засыпко В.\,В.} 
Аналитическое исследование задачи оптимального управления инвестициями в~закрытой 
динамической модели трехсекторной экономики~// Вестник МГТУ им. Н.\,Э.~Баумана. 
Сер. Естественные науки, 2014. Вып.~4. С.~101--120.

\bibitem{3-gor} 
\Au{Интрилигатор М.} Математические методы оптимизации и~экономическая теория~/ 
Пер. с~англ.~---  М.: Айрис-Пресс, 2002.  553~с.
(\Au{Intriligator M.\,D.}  Mathematical optimization and economic theory.~--- 
SIAM, 1971. 508~p.)

\bibitem{4-gor} 
\Au{Arrow A., Intriligator M.\,D., Hildenbrand~W., Sonnenschein~H.} 
Handbook of mathematical economics.~---  Amsterdam: North Holland, 1991. 750~p.

\bibitem{5-gor} 
\Au{Barro R., Sala-i-Martin~X.} Economic growth.~--- Boston: MIT Press, 2003. 672~p.

\bibitem{6-gor} 
\Au{Ngai L.\,R., Pissarides C.\,A.} Structural change in a~multi-sector model of 
growth.~--- London: Centre 
for Economic Policy Research, 2004.  Discussion Paper No.\,4763. 25~p.



\bibitem{8-gor} 
\Au{Алексеев В.\,М., Галеев Э.\,М., Тихомиров~В.\,М.}
Сборник задач по оптимизации. Теория. Примеры. Задачи.~--- М.: Физматлит, 2005. 256~c.

\bibitem{7-gor} 
\Au{Арутюнов А.\,А., Мага\-рил-Илья\-ев Г.\,Г., Тихомиров~В.\,М.} 
Принцип максимума Понтрягина.~--- М.: Факториал, 2006. 144~c.

\bibitem{9-gor} 
\Au{Колемаев В.\,А.}  Трехсекторная модель экономики~// Сб. науч. тр. Междунар. 
академии информатизации.~--- М.: Копия-Принт, 1997. С.~335--345.

\columnbreak


\bibitem{10-gor} 
\Au{Колемаев В.\,А.}  Математическая  экономика.~--- М.: Юни\-ти-Да\-на, 2002. 399~c.

%\vspace*{2pt}

\bibitem{11-gor}  
\Au{Алексеев В.\,М., Тихомиров В.\,М., Фомин~С.\,В.} 
Оптимальное управление.~--- М.: Физматлит, 2007. 408~c.

%\vspace*{2pt}

\bibitem{12-gor} 
\Au{Колемаев В.\,А.}
Оптимальный сбалансированный рост открытой трехсекторной экономики~// 
Прикладная эконометрика, 2008. Вып.~3. C.~14--42.
\end{thebibliography}

 }
 }

\end{multicols}

\vspace*{-3pt}

\hfill{\small\textit{Поступила в~редакцию 08.11.15}}

\vspace*{6pt}

%\newpage

%\vspace*{-24pt}

\hrule

\vspace*{2pt}

\hrule

\vspace*{6pt}



\def\tit{DEVELOPMENT OF THE ALGORITHM OF NUMERICAL SOLUTION OF~THE~OPTIMAL INVESTMENT CONTROL PROBLEM 
IN~THE~CLOSED DYNAMICAL MODEL\\ OF~THREE-SECTOR~ECONOMY}

\def\titkol{Development of the algorithm of numerical solution of the optimal 
investment control problem in the closed dynamical model} % of three-sector economy}

\def\aut{P.\,V.~Shnurkov$^1$, V.\,V.~Zasypko$^1$, V.\,V.~Belousov$^{2}$, and~A.\,K.~Gorshenin$^{2,3}$}

\def\autkol{P.\,V.~Shnurkov, V.\,V.~Zasypko, V.\,V.~Belousov, and A.\,K.~Gorshenin}

\titel{\tit}{\aut}{\autkol}{\titkol}

\vspace*{-9pt}

\noindent
$^1$National Research University Higher School of Economics, 34~Tallinskaya Str., 
Moscow 123458, Russian\linebreak
$\hphantom{^1}$Federation


\noindent
$^2$Institute of Informatics Problems, Federal Research Center 
``Computer Science and Control'' of the Russian\linebreak
 $\hphantom{^1}$Academy of Sciences,
44-2 Vavilov Str., Moscow 119333, Russian Federation

\noindent %Federal State Budget Educational Institution  
%of  Higher Education
$^3$Moscow Technological University (MIREA),
78~Vernadskogo Ave.,  Moscow 119454, Russian Federation

\def\leftfootline{\small{\textbf{\thepage}
\hfill INFORMATIKA I EE PRIMENENIYA~--- INFORMATICS AND
APPLICATIONS\ \ \ 2016\ \ \ volume~10\ \ \ issue\ 1}
}%
 \def\rightfootline{\small{INFORMATIKA I EE PRIMENENIYA~---
INFORMATICS AND APPLICATIONS\ \ \ 2016\ \ \ volume~10\ \ \ issue\ 1
\hfill \textbf{\thepage}}}

\vspace*{3pt}



\Abste{The paper develops the numerical method of solution of the optimal investment 
control problem in the closed dynamical model of three-sector economy. The preceding 
papers described an analytical research of this problem by the method based on the 
Pontryagin maximum principle. In the present paper, the authors obtained analytical 
representations for state functions. Conjugate variables are used as the foundation 
of the numerical algorithm. The developed algorithm makes it possible to analyze the 
class of admissible control functions, having not more than the given finite number of 
points of switch, and to find among them those that satisfy the necessary optimality 
conditions and restrictions of the original task. The general scheme of the proposed 
algorithm can be used to investigate another optimal control tasks, connected with 
different subject areas. The developed algorithm is realized in a system of applied 
programs.}

\KWE{model of three-sector economy; Pontryagin maximum principle; numerical method of solution of the optimal control problem}

\DOI{10.14357/19922264160108}

%\Ack
%\noindent



%\vspace*{6pt}

  \begin{multicols}{2}

\renewcommand{\bibname}{\protect\rmfamily References}
%\renewcommand{\bibname}{\large\protect\rm References}

{\small\frenchspacing
 {%\baselineskip=10.8pt
 \addcontentsline{toc}{section}{References}
 \begin{thebibliography}{99}


\bibitem{2-gor-1}
\Aue{Shnurkov, P.\,V., and V.\,V.~Zasypko}. 2014.  
Optimal'noe upravlenie investitsiyami v~zakrytoy dinamicheskoy modeli trekhsektornoy 
ekonomiki: Matematicheskaya postanovka zadachi i~obshchiy analiz na osnove printsipa 
maksimuma   [Optimal control of investments in the closed-form dynamic model of 
three-sector economy: Mathematical statement of the problem and general analysis 
based on the maximum principle]. \textit{Vestnik MGTU im. N.\,E.~Baumana. Ser. 
Estestvennye nauki} [Herarld of the Bauman Moscow State Technical University. Ser.\
 Natural Sciences] 2:101--115.
 
 \bibitem{1-gor-1}
\Aue{Shnurkov, P.\,V., and V.\,V.~Zasypko}.  2014. 
Analiticheskoe issledovanie zadachi optimal'nogo upravleniya investi\-tsi\-yami 
v~zakrytoy dinamicheskoy modeli trekhsektornoy ekonomiki 
[Analytical study of optimal investments control problem in closed-form dynamic 
model of three-sector economics]. \textit{Vestnik MGTU im. N.\,E.~Baumana. Ser. 
Estestvennye nauki} [Herarld of the Bauman Moscow State Technical University. 
Ser. Natural Sciences] 4:101--120.
 

\bibitem{3-gor-1}
\Aue{Intriligator, M.\,D.} 
 1971. \textit{Mathematical optimization and economic theory}. SIAM. 508~p.

\bibitem{4-gor-1}
\Aue{Arrow,~A.,  M.\,D.~Intriligator, W.~Hildenbrand, and H.~Sonnenschein}. 1991. 
\textit{Handbook of mathematical economics}. Amsterdam: North Holland. 750~p.


\bibitem{5-gor-1}
\Aue{Barro, R., and X.~Sala-i-Martin}. 2003. 
\textit{ Economic growth}. MIT Press. 672~p.

%\pagebreak

\bibitem{6-gor-1}
\Aue{Ngai, L.\,R., and C.\,A.~Pissarides}. 2004. Structural change in a~multi-sector
model of growth. London: Centre for Economic Policy Research.
 Discussion Paper No.\,4763.  25~p.



\bibitem{8-gor-1}
\Aue{Alekseev, V.\,M., E.\,M.~Galeev, and V.\,M.~Tikhomirov}. 
2005. \textit{Sbornik zadach po optimizatsii. Teoriya. Primery. Zadachi} 
[Collection of tasks on optimization. Theory. Examples. Tasks].
 Moscow: Fizmatlit. 256~p.
 
 \bibitem{7-gor-1}
\Aue{Arutyunov, A.\,A., G.\,G.~Magaril-Il'yaev, and V.\,M.~Ti\-kho\-mi\-rov}. 2006.
\textit{Printsip maksimuma Pontryagina} 
[The Pontryagin maximum principle]. Moscow: Fak\-to\-ri\-al.  144~p.



\bibitem{9-gor-1}
\Aue{Kolemaev, V.\,A.} 1997. 
Trekhsektornaya model' ekonomiki [Three-sector model]. 
\textit{Sb. nauch. tr. Mezhdunar. Akademii} %\linebreak\vspace*{-12pt}
%\columnbreak
%\noindent
\textit{Informatizatsii}
[International Academy of Informatization Proceedings]. Moscow: Kopiya-Print.
P.~335--345.

\bibitem{10-gor-1}
\Aue{Kolemaev, V.\,A.} 2002.
\textit{Matematicheskaya  ekonomika} [Mathematical economics]. Moscow: Yuniti-Dana.
399~p.

\bibitem{11-gor-1}
\Aue{Alekseev, V.\,M., V.\,M.~Tikhomirov, and S.\,V.~Fomin}. 
2007. \textit{Optimal'noe upravlenie} [Optimal control]. Moscow: Fizmatlit. 408~p.

\bibitem{12-gor-1}
\Aue{Kolemaev, V.\,A.} 2008.  
Optimal'nyy sbalansirovannyy rost otkrytoy trekhsektornoy ekonomiki 
[The optimal balanced growth of the open three-sector economy]. 
\textit{Prikladnaya Ekonometrika} [Applied Econometrics]\linebreak  3:14--42.


\end{thebibliography}

 }
 }

\end{multicols}

\vspace*{-3pt}

\hfill{\small\textit{Received November 8, 2015}}

\Contr

\noindent
\textbf{Shnurkov Peter V.} (b.\ 1953)~--- 
Candidate of Science (PhD) in physics and mathematics, associate professor, 
National Research University Higher School of Economics, 34~Tallinskaya Str., 
Moscow 123458, Russian Federation; pshnurkov@hse.ru 

\vspace*{3pt}

\noindent
\textbf{Zasypko Veronika V.} (b.\ 1988)~--- 
PhD student, National Research University Higher School of Economics, 
34~Tallinskaya Str., Moscow, 123458, Russian Federation; vzasypko@gmail.com 

\vspace*{3pt}

\noindent
\textbf{Belousov Vasiliy V.} (b.\ 1977)~---
Candidate of Science (PhD) in technology, Head of Laboratory, 
Institute of Informatics Problems, Federal Research Center ``Computer Science and 
Control'' of the Russian Academy of Sciences, 44-2 Vavilov Str., Moscow 119333, 
Russian Federation; VBelousov@ipiran.ru 

\vspace*{3pt}

\noindent
\textbf{Gorshenin Andrey K.}  (b.\ 1986)~---
Candidate of Science (PhD) in physics and mathematics, senior scientist, 
Institute of Informatics Problems, Federal Research Center ``Computer Science and Control'' 
of the Russian Academy of Sciences, 44-2 Vavilov Str., Moscow 119333, 
Russian Federation; associate professor, %Federal State Budget Educational Institution  
%of  Higher Education 
Moscow Technological University (MIREA), 
78~Vernadskogo Ave., Moscow 119454, Russian Federation;
 agorshenin@frcсsc.ru

\label{end\stat}


\renewcommand{\bibname}{\protect\rm Литература}