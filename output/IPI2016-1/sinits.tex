\def\prn{p_{r,\nu}}
\def\crn{c_{r,\nu}}
\def\qpn{q_{p,\nu}}
\def\qrm{q_{p,\mu}}
\def\ral{{\mathrm{ЭЛ}}}
\def\mal{{\mathrm{МЭЛ}}}
\def\srn{S_{r,\nu}}



\def\stat{sinits}

\def\tit{АНАЛИТИЧЕСКОЕ МОДЕЛИРОВАНИЕ
 РАСПРЕДЕЛЕНИЙ В~НЕЛИНЕЙНЫХ 
СТОХАСТИЧЕСКИХ  СИСТЕМАХ НА~МНОГООБРАЗИЯХ МЕТОДОМ  ЭЛЛИПСОИДАЛЬНОЙ АППРОКСИМАЦИИ$^*$}

\def\titkol{Аналитическое моделирование
 распределений в~нелинейных  %стохастических  системах на~многообразиях
МСтС методом  ЭА} %эллипсоидальной аппроксимации}

\def\aut{И.\,Н.~Синицын$^1$, В.\,И.~Синицын$^2$}

\def\autkol{И.\,Н.~Синицын, В.\,И.~Синицын}

\titel{\tit}{\aut}{\autkol}{\titkol}

{\renewcommand{\thefootnote}{\fnsymbol{footnote}} \footnotetext[1]
{Работа выполнена при финансовой поддержке  РФФИ (проект 15-07-02244).}}


\renewcommand{\thefootnote}{\arabic{footnote}}
\footnotetext[1]{Институт проблем информатики Федерального исследовательского
центра <<Информатика и~управление>> Российской академии наук,
sinitsin@dol.ru}
\footnotetext[2]{Институт проблем информатики Федерального исследовательского
центра <<Информатика и~управление>> Российской академии наук,
vsinitsin@ipiran.ru}

\vspace*{-6pt}

\Abst{Рассматриваются вопросы оценки точности и~чувствительности алгоритмов 
структурного аналитического моделирования одно- и~многомерных распределений  
методами эл\-липсо\-и\-даль\-ной аппроксимации (ЭА) и~эл\-липсо\-и\-даль\-ной
 линеаризации (ЭЛ). 
Разработаны общие алгоритмы эл\-липсо\-и\-даль\-но\-го анализа распределений в~нелинейных 
стохастических системах (СтС) на многообразиях (МСтС) с~винеровскими и~пуассоновскими шумами. 
Особое внимание уделено алгоритмам для МСтС с~аддитивными негауссовскими шумами.
Получены уравнения точности и~чувствительности к~параметрам. В~качестве иллюстрации 
рассмотрена нелинейная двумерная угловая система с~аддитивным угловым белым шумом. 
Составлены уравнения точности и~чувствительности, позволяющие оценивать моменты до 
четвертого порядка включительно. Приведены результаты численных расчетов, показывающие 
эффективность метода ЭА (ЭЛ) по сравнению с~общим методом ортогональных разложений 
одномерной плотности. Сформулированы некоторые возможные обобщения.}

\KW{метод аналитического моделирования (МАМ);
метод эллипсоидальной аппроксимации (МЭА);
метод эллипсоидальной линеаризации (МЭЛ);
плотность одно- и~многомерного распределения;
стохастическая система на многообразиях (МСтС);
уравнения точности МЭА и~МЭЛ;
уравнения чувствительности МЭА и~МЭЛ}

\DOI{10.14357/19922264160104} 

\vspace*{-2pt}

\vskip 12pt plus 9pt minus 6pt

\thispagestyle{headings}

\begin{multicols}{2}

\label{st\stat}

\section{Введение}


 В~[1, 2] подробно описаны известные методы аналитического моделирования (МАМ) 
 распределений процессов в~СтС, описываемых дифференциальными 
 стохастическими уравнениями Ито с~винеровскими и~пуассоновскими шумами, основанные на 
 параметризации их распределений.
%
Обобщение результатов~[1, 2] на случай многоканальных круговых и~сферических 
СтС выполнено\linebreak в~[3--7].
Статья~[8] посвящена развитию дискрет-\linebreak ных методов параметрического статистического
 и~аналитического моделирования в~МСтС. В~ней рас\-смот\-ре\-ны 
 уравнения МСтС, приближенные методы статистического моделирования (МСМ) различной 
 точности и~методы аналитического моделирования (МАМ). Подробно развита нелинейная 
 корреляционная теория МСМ и~ МАМ. 
 %
 В~[9, 10] пред\-став\-лены МАМ одно- и~многомерных 
 распределений в~МСтС, основанные на методе ортогональных разложений (МОР) плотностей. 
 Рас\-смот\-рим развитие~[9, 10] на случай, когда плотности допускают 
ЭА~\cite{1-sin, 2-sin, 11-sin}.
%
В~[9--12] %\cite{9-sin, 10-sin, 11-sin, 12-sin} 
развиты методы и~алгоритмы аналитического 
моделирования гауссовских (нормальных) процессов в~МСтС.

Статья включает в~себя: введение, 5~разделов, заключение и~список литературы. 
В~разд.~2 представлены уравнения МСтС. Общие уравнения метода ЭА (МЭА) даны в~разд.~3, а для МСтС с~аддитивными шумами разработан 
метод ЭЛ (МЭЛ) в~разд.~4.  В~разд.~5 получены уравнения 
точности и~чувствительности МЭА и~МЭЛ. Раздел~6 содержит иллюстративный пример. 
В~заключении кратко сформулированы основные результаты и~указаны некоторые возможные 
обобщения.

\vspace*{-15pt}


\section{Уравнения нелинейных стохастических систем на~многообразиях}

\vspace*{-6pt}

Рассмотрим сначала дифференциальную  СтС в~конечномерном пространстве следующего вида:

\noindent
    \begin{multline}
    dY_t= a (Y_t,\Theta,t)\, dt + b(Y_t,\Theta,t)\, d W_0 + {}\\
    {}+
    \int\limits_{R_0^q} c(Y_t,\Theta,t,v) \,dP^0 (\Theta, t,
    dv)\,,\enskip Y(t_0)=Y_0\,.\label{e2.1-sin}
    \end{multline}
Здесь $Y_t$~--- $p$-мер\-ный вектор состояния, $Y_t\hm\in \Delta^y$\linebreak ($\Delta^y$~--- 
многообразие состояний); $\Theta$~--- вектор  случайных параметров размерности~$p^\Theta$;
$a\hm=a (y_t,\Theta,t)$ и~$b\hm= b(y_t,\Theta,t)$~--- известные $(p\times 1)$-мер\-ная 
и~$(p\times r)$-мер\-ная функции вектора~$Y_t$ и~времени~$t$; $W_0\hm= W_0(\Theta, t)$~---
$r$-мер\-ный винеровский стохастический процесс (СтП) интенсивности $\nu_0\hm=
\nu_0(\Theta,t)$; $c(y_t,\Theta, t,v)$~--- $(p\times 1)$-мер\-ная функция~$y_t$, 
$t$ и~вспомогательного
$(q\times 1)$-мер\-но\-го парамет\-ра~$v$; $\int\limits d P^0 (\Theta, t,A)$~--- 
центрированная пуассоновская мера:
    $$
    \int\limits_{\Delta}\, dP^0 (\Theta, t,A)= 
    \int\limits_{\Delta} \,dP (\Theta, t,A)-
    \int\limits_{\Delta} \nu_P (\Theta, t,A)\, dt\,,
    $$
где $\int\limits_{\Delta} d P (\Theta, t,A)$~--- число скачков пуассоновского
СтП в~интервале времени~$\Delta$; $\nu_P (\Theta, t,A)$~--- интенсивность
пуассоновского СтП $P(t,A)$;
$A$~--- некоторое борелевское множество пространства~$R^q_0$ 
с~выколотым началом координат.
Интеграл~(1) в~общем случае распространяется на все пространство~$R_0^q$ 
с~выколотым началом координат.
Начальное значение~$Y_0$ вектора~$Y_t$ представляет
собой случайную величину (с.в.), не зависящую от приращений винеровского
СтП $W_0(\Theta, t)$ и~пуассоновского СтП $P(\Theta, t,A)$ на интервалах
времени $\Delta\hm= (t_1, t_2]$, следующих за~$t_0$, $t_0\hm\le t_1\hm\le t_2$,
для любого множества~$A$.

Теперь выделим важный частный случай~(\ref{e2.1-sin}):
    \begin{equation}
    \dot Y_t = a (Y_t,\Theta, t)+ b_0 (\Theta,t) V(\Theta)
    \label{e2.2-sin}
    \end{equation}
при следующих допущениях
\begin{gather*}
    b(Y_t,\Theta, t)= b_0 (\Theta, t)\,;\enskip c(Y_t,\Theta, t,v) = c_0(v,\Theta)\,;
   % \label{e2.3-sin}
\\
V=\dot W\,;\quad W = W_0 + W_P\,;\\
W_P = \iii\limits_{R_0^q} c_0 (v,\Theta) 
P^0(\Theta, t, dv)\,;
%\label{e2.4-sin}
\end{gather*}

\vspace*{-13pt}

\noindent
\begin{align*}
\chi^{W_0} &=\chi^{W_0} (\mu; \Theta, t) =- \fr{1}{2}\, \mu^{\mathrm{T}} 
\nu_0 (\Theta, t) \mu\,;\\
\chi^{W_P}&=\chi^{W_P} (\mu; \Theta, t) = \iii_{R_0^q} \left[ 
e^{i\la^T c_0(v,\Theta)} -1-{}\right.\\
&\left.\hspace*{15mm}{}-i\mu^T c_0(v,\Theta)\vphantom{e^{i\la^T c_0(v,\Theta)}}
\right] \nu_P 
(\Theta, t,v)\,dv\,.
%\label{e2.5-sin}
\end{align*}
Здесь $\nu_0 \hm=\nu_0 (\Theta, t)$ и~$\nu_P\hm=\nu_P(\Theta, t,v)$~--- 
интенсивности СтП и~потока скачков, равных $c_0(v,\Theta)$; $\chi^{W_0}$ 
и~$\chi^{W_P}$~--- логарифмические производные одномерных характеристических 
функций СтП~$W_0$ и~$W_P$.

Для вычисления вероятностей событий,
связанных с~СтП, в~прикладных задачах достаточно
знания многомерных распределений. Поэтому центральной
задачей теории МСтС является задача вероятност\-ного анализа одномерных
распределений СтП, удовлетворяющих дифференциальному стохастическому
уравнению Ито~(\ref{e2.1-sin}) при соответствующих начальных условиях.
В~теории МСтС различают два принципиально разных подхода
к~вычислению распределений. Первый общий подход основан на
статистическом моделировании, т.\,е.\ на прямом численном решении~(\ref{e2.1-sin}) 
с~последу-\linebreak ющей статистической обработкой результатов.\linebreak Второй общий
подход основан на теории непрерывных марковских СтП 
и~предполагает аналитическое\linebreak моделирование, т.\,е.\ решение
детерминированных уравнений в~функциональных пространствах
(уравнений Фок\-ке\-ра--План\-ка--Кол\-мо\-го\-ро\-ва, Фел\-ле\-ра--Кол\-мо\-го\-ро\-ва,
Пугачёва и~др.)\ для одномерных\linebreak распределений. 
В~практических задачах час\-то используют и~комбинированные методы. При
этом будем предполагать, что существуют  одно- и~многомерные
плотности СтП в~МСтС~(\ref{e2.1-sin}) и~(\ref{e2.2-sin}). Достаточные условиях их
существования можно найти, например, в~\cite{1-sin, 2-sin, 12-sin}.

\vspace*{-9pt}

\section{Эллипсоидальная аппроксимация одно- и~многомерных распределений в~стохастической
системе на~многообразиях~(\ref{e2.1-sin})}

\vspace*{-2pt}

Как известно~\cite{11-sin}, для конечномерных МСтС часто оказывается полезной
структурная аппроксимация распределений посредством эллипсоидальных
распределений. Следуя~\cite{11-sin},  для структурной аппроксимации
плотностей вероятности\linebreak случайных векторов будем использовать
плотности, имеющие эллипсоидальную структуру, т.\,е.\ плотности, 
у~которых поверхностями уровней равной вероятности являются подобные
концентрические\linebreak эллипсоиды (эллипсы для двумерных векторов,
эллипсоиды для трехмерных векторов, ги\-пер\-эл\-лип\-со\-иды для векторов
размерности больше трех).\linebreak В~частности, эллипсоидальную структуру
имеет нормальное распределение в~любом конечномерном пространстве.
Характерная особенность таких распределений состоит в~том, что их
плотности вероятности являются функциями  положительно определенной квадратичной
формы $u\hm=u(y)\hm=(y^T\hm-m^T)C(y\hm-m)$, где $m$~--- математическое ожидание
случайного вектора~$Y$; $C$~--- некоторая положительно определенная матрица.

Для нахождения ЭА плотности вероятности\linebreak
$r$-мер\-но\-го случайного вектора будем пользоваться конечным
отрезком разложения по биортонормальной системе полиномов
$\{p_{r,\nu}(u(y)),q_{r,\nu}(u(y))\}$, которые зависят только от
квадратичной формы $u\hm=u(y)$ и~функцией веса для которых служит
некоторая плотность вероятности эллипсоидальной структуры $w(u(y))$:
   \begin{equation}
{\rm M}_{\Delta^y}^w\left[    p_{r,\nu}\left(u(Y)\right)q_{r,\mu}(u(Y))\right]=
    \delta_{\nu\mu}.\label{e3.1-sin}
\end{equation}



Индексы $\nu$ и~$\mu$ у~полиномов означают их степени относительно
переменной~$u$. Конкретный вид и~свойства полиномов определены
ниже. Однако без потери общности можно принять, что
$q_{r,0}(u)\hm=p_{r,0}(u)\hm=1$. Тогда плотность вероятности вектора~$Y$
может быть приближенно представлена в~виде:
\begin{equation}
f(y)\approx   f^*(u)=w(u)\sum\limits_{\nu=0}^N  \crn \prn(u)\,,
\label{e3.2-sin}
\end{equation}
где коэффициенты~$\crn$ определяются по формуле
    \begin{equation}
    \crn=\mathrm{M}_{\Delta^y}^{\mathrm{ЭА}}\left[q_{r,\nu}(U)\right]\,,\enskip 
    (\nu=1,\ldots,N).
    \label{e3.3-sin}
    \end{equation}
Учитывая, что $p_{r,0}(u)$ и~$q_{r,0}(u)$~--- взаимно обратные
постоянные (полиномы нулевой степени), то всегда $c_{r,0}p_{r,0}\hm=1$.
Поэтому из формулы~(\ref{e3.2-sin}) следует,~что
\begin{equation}
f(y)\approx     f^*(u)=w(u)\left[1+\sum\limits_{\nu=2}^N \crn\prn(u)
    \right]\,.\label{e3.4-sin}
    \end{equation}

Для приложений большое значение имеет случай, когда за
распределение $w(u)$ выбирается нормальное (гауссовское) распределение
\begin{equation*}
w(u)=w(y^TCy)=\fr{1}{\sqrt{(2\pi)^r\vert
    K\vert}}\exp\left(-\fr{y^TK^{-1}y}{2}\right)\,. %\label{e3.5-sin}
    \end{equation*}
Учитывая, что $C\hm=K^{-1}$, приведем условие биортонормальности~(\ref{e3.1-sin}) к~виду:
    \begin{equation*}
    \fr{1}{2^{r/2}\Gamma(r/2)}\int\limits_0^{\infty}
    \prn(u)\qrm(u)u^{r/2-1}e^{-u/2}\,du=\delta_{\nu\mu}\,.
%\label{e3.6-sin}
\end{equation*}

 Задача выбора системы полиномов $\{\prn(u),\qrm(u)\}$,
используемой при ЭА плотностей~(\ref{e3.2-sin})
и~(\ref{e3.3-sin}), сводится к~нахождению биортонормальной системы
полиномов, для которой весом служит $\chi^2$-рас\-пре\-де\-ле\-ние 
с~$r$~степенями свободы. При этом используются следующие формулы:

\noindent
\begin{align*}
p_{r,\nu}(u)&=S_{r,\nu}(u)\,;\\[6pt] 
q_{r,\nu}(u)&=\fr{(r-2)!!}
{(r+2\nu-2)!!(2\nu)!!}\,S_{r,\nu}(u)\,,
    \enskip r\ge 2\,,
    %    \label{e3.7-sin}
    \end{align*}
    где
   \begin{multline*}
S_{r,\nu}(u)=S_{\nu}^{r/2-1}(u)={}\\
{}=\sum\limits_{\mu=0}^{\nu}(-1)^{\nu+\mu}
    C_{\nu}^{\mu}\fr{(r+2\nu-2)!!}{(r+2\mu-2)!!}\,u^{\mu}\,.
    %\label{e3.8-sin}
    \end{multline*}
    
    \vspace*{-8pt}

При разложении по полиномам~$\srn(u)$ плот\-ности вероятности
случайного вектора~$Y$ и~всех его возможных проекций согласованы.

Пользуясь обобщенной формулой Ито~[1, 2] для дифференциала нелинейной функции  
$\vrp \hm= \vrp(Y_t,\Theta,t)$
\begin{multline*}
    d\vrp (Y_t, \Theta, t) = {}\\[1pt]
    {}=
    \left\{ 
    \vphantom{b\left(Y_t, \Theta, t\right)^{\mathrm{T}}}
    \vrp_t \left(Y_t, \Theta, t\right)+ \vrp_Y \left(Y_t, \Theta, t\right)^{\mathrm{T}}
    a \left(Y_t, \Theta, t\right)+{}\right.\\[1pt]
{}+\fr{1}{2} \,\mathrm{tr} \left[ 
\vphantom{b\left(Y_t, \Theta, t\right)^{\mathrm{T}}}
\vrp_{YY} \left(Y_t, \Theta, t\right) 
b\left(Y_t, \Theta, t\right)\nu_0(\Theta, t)\times{}\right.\\[1pt]
\left.\left.{}\times b\left(Y_t, \Theta, t\right)^{\mathrm{T}} 
\right]\right\}  dt+{}\\[1pt]
{}+ \iii_{R_0^q} \left[ 
\vphantom{\left(Y_t, \Theta, t\right)^{\mathrm{T}} }
\vrp  \left( Y_t + c \left(Y_t, \Theta, t,v\right) - \vrp \left(Y_t, \Theta, t\right)-{}\right.\right.\\[1pt]
\left.{} -\vrp_Y \left(Y_t, \Theta, t\right)^{\mathrm{T}} 
c(Y_t, \Theta, t,v)\right] \nu_P (\Theta,  dv)\,dt+ {}\\[1pt]
{}+
\vrp_Y\left(Y_t, \Theta, t\right)^{\mathrm{T}} b \left(Y_t, \Theta, t\right)\, dW_0 
(\Theta, t) + {}\\[1pt]
{}+\iii_{R_0^q} \left[ \vrp\left(Y+c\left(Y_t, \Theta, t,v\right), \Theta, t\right) - \right.{}\\[1pt]
\left.{}-\vrp \left(Y_t, \Theta, t\right)\right] P^0 (\Theta, dt,dv)\,,
%\label{e3.9-sin}
\end{multline*}
получаем, что уравнения
для нахождения одномерной плотности
вероятности $f_1(y;t)$ $p$-мер\-но\-го случайного процесса~$Y_t$,
определяемого стохастическим дифференциальным уравнением Ито~(\ref{e2.1-sin}), 
с~по\-мощью МЭА имеют вид:

\vspace*{-8pt}

\noindent
    \begin{multline}
    f_1(y;\Theta, t)\cong f_1^{\mathrm{ЭА}}(\Theta,u)={}\\
{}=w_1(\Theta,u)\left[
    1+\sum\limits_{\nu=2}^N c_{p,\nu} p_{p,\nu}(\Theta,u)\right]\,,
        \label{e3.10-sin}
    \end{multline}
    где
    
    \vspace*{-13pt}
    
    \noindent
\begin{align*}
    u&= (y-m)^{\mathrm{T}} K^{-1} (y-m)\,;\\
     c_{p,\nu}&=\mm_{\Delta^y}^{\mathrm{ЭА}}\left[\qpn\left(U_t\right)\right]\quad
    (\nu=1,\ldots,N)\,.\notag
 %   \label{e3.11-sin}
    \end{align*}
Здесь $m=m_t$, $K\hm=K_t$ удовлетворяют уравнениям

\vspace*{-2pt}

\noindent
   \begin{multline}
  \dot m= A^m \left(m_t, K_t, \Theta, t\right) = 
    \varphi_{10}\left(m_t,K_t,\Theta, t\right)+{}\\
 {}+\sum\limits_{\nu=2}^N
    c_{p,\nu}\varphi_{1\nu}\left(m_t,K_t,\Theta, t\right)\,;
    \label{e3.12-sin}
\end{multline}

%\vspace*{-14pt}

\noindent
\begin{multline}
\dot K_t=A^K \left(m_t, K_t, \Theta, t\right) =
    \varphi_{20}\left(m_t,K_t,\Theta,t\right)+{}\\
{}+\sum\limits_{\nu=2}^N
    c_{p,\nu}\varphi_{2\nu}\left(m_t,K_t,\Theta,t\right),
      \label{e3.13-sin}
    \end{multline}

%    \pagebreak
    
    \noindent
где введены следующие обозначения:
    \begin{equation}
    \left.
\begin{array}{rl}
        \hspace*{-2mm}\varphi_{10}\left(m_t,K_t,t\right)&=\mm_{\Delta^y}^{w_1} 
    [a(Y,\Theta, t)]\,; %\label{e3.14-sin}
    \\[4pt]
    \hspace*{-2mm}\varphi_{1\nu}\left(m_t,K_t,t\right)&=\mm_{\Delta^y}^{p_{p,\nu}w_1}
[a(Y,\Theta, t)]\,; %\label{e3.15-sin}
\\[4pt]
    \hspace*{-3mm}\varphi_{20}\left(m_t,K_t,t\right)&={}\\[4pt]
&\hspace*{-16mm}{}=\mm_{\Delta^y}^{w_1}
    \left[a(Y,\Theta,t)\left(Y^{\mathrm{T}}-m_t^{\mathrm{T}}\right)+{}\right.\\[4pt]
    &\left.\hspace*{-16mm}{}+
    \left(Y-m_t\right) a(Y,\Theta,t)^{\mathrm{T}}  +{\bar\sigma}(Y,\Theta, t)\right]\,;
    %\label{e3.16-sin}
    \\[3pt]
    \hspace*{-2mm}a_{2\nu}\left(m_t,K_t,\Theta, t\right)&={}\\[4pt]
&\hspace*{-16mm}{}=\mm_{\Delta^y}^{p_{p,\nu}w_1}
    \left[a(Y,\Theta,t)\left(y^{\mathrm{T}}-m_t^{\mathrm{T}}\right)+{}\right.\\[4pt]
&\hspace*{-18mm}\left.{}+    \left(Y-m_t\right) a(Y,\Theta, t)^{\mathrm{T}}+{\bar\sigma}(Y,\Theta,t)\right].
\end{array}\!\!
\right\}
    \label{e3.17-sin}
    \end{equation}
Коэффициенты $c_{p,\kappa}= c_{p,\kappa,t}$ определяются уравнением

\vspace*{-2pt}

\noindent
    \begin{multline}
    \dot c_{p,\kappa}=A^{c_p,\kappa} \left(m_t, K_t, \Theta, t\right)={}\\
    {}=
    -\left(\fr{c_{p,\kappa-1}}{2p} - \fr{\kappa
    c_{p,\kappa}}{p}\right) \mathrm{tr}\left\{
    \vphantom{\sum\limits_{\nu=2}^N}
    K_t^{-1}\varphi_{20}\left(m_t,K_t,\Theta,t\right)+{}\right.\\
\left.    {}+
    K_t^{-1}\sum\limits_{\nu=2}^N c_{p,\nu}\varphi_{2\nu}\left(m_t,K_t,\Theta,t\right)\right\}+{}\\
{}+\psi_{\kappa0}\left(m_t,K_t,\Theta,t\right)+
\sum\limits_{\nu=2}^N c_{p,\nu}\psi_{\kappa\nu}\left(m_t,K_t,\Theta,t\right)\,,
    \\
     \kappa=2,\ldots,N\,,\label{e3.18-sin}
    \end{multline}
где введены следующие обозначения:

\vspace*{-2pt}

\noindent
  \begin{multline}
  \psi_{\kappa 0}\left(m_t,K_t,\Theta,t\right)={}\\[-1pt]
  {}=
  \mm_{\Delta^y}^{w_1}\left[
  q_{p,\kappa}'(U)\bigg(2\left(Y-m_t\right)^{\mathrm{T}}K_t^{-1} a(Y,\Theta,t)+{}\right.\\[-1pt]
{}+    \mathrm{tr}\,K_t^{-1}\sigma(Y,\Theta,t)\bigg)+{}\\[-1pt]
{}+ 2q_{\kappa}''(U)\left(Y-m_t\right)^{\mathrm{T}}K^{-1}\sigma(Y,\Theta,t)
\left(Y-m_t\right)+{}\\[-1pt]
{}+\int\limits_{R_0^q} \bigg\{
q_{p,\kappa}\left[
\left(Y^{\mathrm{T}}+{c}^{\mathrm{T}}-m_t^{\mathrm{T}}\right)
K_t^{-1}\left(Y+c-m_t\right)\right]-{}\\[-1pt]
{}-q_{p,\kappa}(U)-{}\\[-1pt]
\left.{}-2q_{p,\kappa}'(U)\left(Y-m_t\right)^{\mathrm{T}}
    K_t^{-1}c \bigg\}\nu_P(\tau, dv)\right]\,;\label{e3.19-sin}
    \end{multline}
    
    \vspace*{-14pt}
    
    \noindent
    \begin{multline*}
\psi_{\kappa \nu}\left(m_t,K_t,\Theta,t\right)={}\\
   {}=\mm_{\Delta^y}^{p_{p,\nu}w_1} \left[q_{p,\kappa}'(U)
   \left(2\left(Y-m_t\right)^{\mathrm{T}}K_t^{-1}a\left(Y,\Theta,t\right)+{}\right.\right.\\
\left.{}+    \hbox{\rm tr}\,K_t^{-1}\sigma(Y,\Theta,t)
\vphantom{q_{p,\kappa}'(U) \left(Y-m_t\right)^{\mathrm{T}}}
\right)+{}\\
    {}+2q_{\kappa}''(U)\left(Y-m_t\right)^{\mathrm{T}}
    K_t^{-1}\sigma(Y,\Theta,t)K_t^{-1}\left(Y-m_t\right)+{}\\
         {}+\int\limits_{R_0^q}
    \left\{ \vphantom{\left(Y-m_t\right)^{\mathrm{T}}}
    q_{p,\kappa}\left(\left(Y^{\mathrm{T}}+c^{\mathrm{T}}-m_t^{\mathrm{T}}\right)
    K_t^{-1}\left(Y+c-m_t\right)\right)-{}\right.
       \end{multline*}
       
    \noindent
    \begin{multline}
{}-q_{p,\kappa}(U) -{}\\
\left.\left.{}-2q_{p,\kappa}'(U)\left(Y-m_t\right)^{\mathrm{T}}K_t^{-1}c\right\}
    \nu_P(t,dv)\right]. 
    \label{e3.20-sin}
    \end{multline}
    
    \vspace*{-2pt}

Таким образом, уравнения~(\ref{e3.12-sin}), (\ref{e3.13-sin}) и~(\ref{e3.18-sin}) при
начальных условиях

\noindent
  \begin{equation}
  \left.
  \begin{array}{c}
  m\left(\Theta,t_0\right)=m_0\,;\enskip K\left(\Theta,t_0\right)=K_0\,;
    \\[6pt]
    c_{p,\kappa}\left(\Theta,t_0\right)=c_{p,\kappa}^0\enskip
    (\kappa=2,\ldots,N) 
    \end{array}
    \right\}
    \label{e3.21-sin}
    \end{equation}
определяют $m_t$, $K_t$, $c_{p,2},\ldots,c_{p,N}$ как функции
времени. Для нахождения величин~$c_{p,\kappa}^0$ следует
аппроксимировать плотность начального значения~$Y_0$ вектора
состояния системы формулой~(\ref{e3.10-sin}).

Количество уравнений
МЭА лишь на $N/2 \hm-1$ больше числа уравнений метода нормальной аппроксимации (МНА): 
$Q^{\mathrm{МЭА}} 
\hm=Q_{\mathrm{МНА}} \hm+ N/2 \hm-1$.

Теперь представим $n$-мер\-ную плотность вероятности
процесса~$Y_t$ в~форме

    \vspace*{-8pt}

\noindent
    \begin{multline}
    f_n\left(y_1,\ldots,y_n;\Theta,t_1,\ldots,t_n\right)
    \approx f_n^{\mathrm{ЭА}}(\Theta,u)={}\\
{}=w_n(\Theta,u)\left[1+\sum\limits_{\nu=2}^N c_{np,\nu}p_{np,\nu}(\Theta,u)\right]\,,
    \label{e3.22-sin}
    \end{multline}
    где
    $$
    c_{np,\nu}=\mm_{\Delta^y}^{f_n} \left[q_{np,\nu}\left(U_t\right)\right].
$$
Здесь использованы следующие обозначения:

    \vspace*{-8pt}

    \noindent
\begin{gather*}
U_t=\left(\tilde Y_n-\tilde     m_n\right)^{\mathrm{T}}
C_n\left(\tilde Y_n-\tilde m_n\right);\enskip C_n=K_n^{-1}\,;\\[2pt]
\tilde  Y_n=\left[Y_1^{\mathrm{T}}\,\,Y_2^{\mathrm{T}}\,\cdots\,T_n^{\mathrm{T}}\right]^{\mathrm{T}}\,;
\enskip
    Y_k=Y\left(t_k\right)\,;\\[2pt]
\tilde
    m_n=\left[ m_1^{\mathrm{T}}\,\,m_2^{\mathrm{T}}\,\cdots\,m_n^{\mathrm{T}}\right]^{\mathrm{T}};
    \enskip m_k=m\left(t_k\right)\,;
   \\[2pt]
%\label{e3.23-sin}
K_n=\begin{bmatrix} 
K(t_1,t_1) &K(t_1,t_2) &\cdots     &K(t_1,t_n)\\ 
    K(t_2,t_1) &K(t_2,t_2) &\cdots &K(t_2,t_n)\\
    \vdots &\vdots &\vdots &\vdots\\
    K(t_n,t_1) &K(t_n,t_2) &\cdots &K(t_n,t_n)\end{bmatrix}
\,,
%\label{e3.24-sin}
\end{gather*}
где $K(t_i,t_j)$~--- ковариационная функция процесса~$Y_t$;
$C_n$~--- матрица, состоящая из блоков $C_{kl}^{(n)}$ размерности
$p\times p$:
    \begin{equation*}
    C_n=\begin{bmatrix}
     C_{11}^{(n)} &C_{12}^{(n)} &\cdots &C_{1n}^{(n)}\cr C_{21}^{(n)} &C_{22}^{(n)} &\ldots
    &C_{2n}^{(n)}\\
    \vdots &\vdots &\vdots &\vdots\\
    C_{n1}^{(n)} &C_{n2}^{(n)} &\cdots &C_{nn}^{(n)}
\end{bmatrix}\,;
%\label{e3.25-sin}
\end{equation*}
 $\bar C_k \hm=\left[C_{k1}^{(n)}\,\cdots\,C_{kn}^{(n)}\right]$~--- 
ее $k$-я блочная строка.

В качестве $w_n(\Theta,u)$ возьмем гауссовскую плотность вероятности

    \vspace*{-2pt}
    
\noindent
\begin{equation*}
w_n(\Theta,u)=\left[(2\pi)^n\vert
    K_n\vert\right]^{-1/2}e^{-u/2}\,.
    %\label{e3.26-sin}
    \end{equation*}
    
    
    \noindent
 Полиномы $p_{np,\nu}(u)$ и~$q_{np,\nu}(u)$  определим форму\-лами:
\begin{align*}
p_{np,\nu}(u)&=S_{np,\nu}(u)={}\notag\\
&{}=\sum\limits_{\mu=0}^{\nu} (-1)^{\nu-\mu}C_{\nu}^{\mu}
    \fr{(np+2\nu-2)!!}{(np+2\mu-2)!!}\,u^{\mu}\,;
    %\label{e3.27-sin}
    \\
q_{np,\nu}(u)&=\fr{(np-2)!!}{(np+2\nu-2)!!(2\nu)!!}\,p_{np,\nu}(u)\,.
    %\label{e3.28-sin}
    \end{align*}
Символ $n$ в~индексе коэффициентов разложения и~полиномов означает
принадлежность к~$n$-мер\-но\-му распределению.

Уравнение для коэффициентов разложения~$c_{np,\kappa}$ примет вид:
\begin{multline}
\fr{\partial c_{np,\kappa}}{\partial t_n}= 
A^{c_{np,\kappa}} \left(\tilde m_n, C_n, K_n,\Theta,t_n\right)
    ={}\\
    {}=-\left(fr{c_{np,\kappa-1}}{2np}+\fr{\kappa c_{np,\kappa}} {np}\right)
    \mathrm{tr}\left[\fr{\partial K_n}{\partial t_n}\,C_n\,\right]+{}\\
{}+\mm_{\Delta^y}^{\mathrm{ЭА}_n} \left\{
    2q_{np,\kappa}'(U)\left(\tilde Y_n-\tilde m_n\right)^{\mathrm{T}} 
    \bar C_n^{\mathrm{T}} a\left(Y_n,\Theta,t_n\right)+{}\right.\\
  {}+2q_{np,\kappa}''(U)
    \left(\tilde Y_n-\tilde m_n\right)^{\mathrm{T}} \bar C_n^{\mathrm{T}}
    \sigma\left(Y_n,\Theta,t_n\right) \times{}\\
    {}\times\bar C_n\left(\tilde Y_n-\tilde m_n\right)
+q_{np,\kappa}'(U)\,\mathrm{tr}\left[
C_{nn}^{(n)}\sigma\left(Y_n,\Theta,t_n\right)\right]+{}\\
{}+
    \int\limits_{R_0^q} \left[q_{np,\kappa}(U)-\right.
2q_{np,\kappa}'(U)\left(\tilde Y_n-\tilde m_n\right)^{\mathrm{T}}
\times{}\\
\left.\left.{}\times \bar  C_n^{\mathrm{T}}  c\left(Y_n,\Theta,t_n,v\right)
    \nu_P(t,dv)\right]
    \vphantom{\left(\tilde Y_n-\tilde m_n\right)^{\mathrm{T}} }
    \right\}\\
     (\kappa=2,3,\ldots,N).\label{e3.29-sin}
    \end{multline}
Будем требовать согласованности $n$- и~$(n-1)$-мер\-но\-го
распределений при $t_n\hm=t_{n-1}\hm+\tau_n$, где $\tau_n$~---
малая величина, достаточная для обеспечения существования матрицы~$C_n$. 
Тогда начальные условия для уравнений~(\ref{e3.29-sin}),
определяющих коэффициенты разложения, можно записать в~виде:
\begin{multline}
c_{np,\nu}\left(t_1,t_2,\ldots,t_{n-1},t_{n-1}+\tau_n\right)={}\\
{}=c_{(n-1)p,\nu}\left(t_1,
    \ldots,t_{n-1}\right)\,.\label{e3.30-sin}
    \end{multline}

В случае двумерного распределения $n\hm=2$ к~уравнениям для
коэффициентов разложения~(\ref{e3.29-sin}) следует добавить уравнение для
ковариационной функции процесса~$Y(t)$:
\begin{equation}
\fr{\partial K(t_1,t_2)}{\partial t_2}=
   \mm_{\Delta^y}^{\mathrm{ЭА}_2}\lk \left(Y_1-m_1\right)a
   \left(Y_2,\Theta,t_2\right)^{\mathrm{T}} \rk\,,
\label{e3.31-sin}
\end{equation}
где $\mm_{\Delta^y}^{\mathrm{ЭА}_2}$ вычисляются на основе
$$
f_2^{\mathrm{ЭА}}(u)=w_2(\Theta,u)\!\left[1+\sum\limits_{\nu=2}^N
    c_{2p,\nu}\left(\Theta,t_1,t_2\right)
    p_{2p,\nu}(u)\right]; %\label{e3.32-sin}
    $$
    \begin{align*}
w_2(\Theta,u)&=\left[(2\pi)^2\vert K_n\vert\right]^{-1/2}\times{}\\
&\hspace*{3mm}{}\times\exp
\left\{-\left(\tilde     y_n-\tilde     m_n\right)^{\mathrm{T}}
K_n^{-1}\left(\tilde y_n-\tilde m_n\right)\right\}\,; %\label{e3.33-sin}
\\
p_{2p,\nu}(u)&=S_{2p,\nu}(u)={}\notag\\
&\hspace*{10mm}{}=\sum\limits_{\mu=0}^{\nu}(-1)^{\nu-\mu}C_{\nu}^{\mu}
    \fr{(2p+2\nu-2)!!}{(2p+2\mu-2)!!}\,u^{\mu}\,. %\label{e3.34-sin}
    \end{align*}

К этому уравнению следует добавить начальное условие
$K(t_1,t_1)\hm=K(t_1)$.
 Ввиду вырожденности двумерного распределения при $t_2\hm=t_1$ уравнение
для ковариационной функции до момента $t_1\hm+\tau_2$ интегрируется при
вырожденном распределении

\noindent
  \begin{multline*}
  f_2^{\mathrm{ЭА}}(\Theta,u)\approx{}\\
  {}\approx
    w_1(\Theta,u)\!\left[1+\!\sum\limits_{\nu=2}^N
    c_{1p,\nu}\left(\Theta,t_1\right)p_{1p,\nu}(u)\!\right]\delta
    \left(y_1-y_2\right),\hspace*{-7.93599pt} %\label{e3.35-sin}
    \end{multline*}
а~начиная с~момента $t_1\hm+\tau_2$~--- совместно с~уравнениями~(\ref{e3.31-sin}). 
Величина~$\tau_2$ определяется в~процессе
интегрирования, когда матрица~$K_2$ станет невы\-рож\-ден\-ной.

 Для приближенного определения МЭА одномерного распределения
стационарного в~узком смысле процесса в~стационарной нелинейной
негауссовской СтС~(\ref{e2.1-sin}) следует положить в~уравнениях~(\ref{e3.12-sin}),
(\ref{e3.18-sin}) и~(\ref{e3.29-sin})
    \begin{equation}
    \dot m=0\,;\  \dot K=0\,;\  \dot c_{p,\kappa}=0\,,\ \kappa=1\,,
    \ldots,N\,.\label{e3.36-sin}
    \end{equation}
Если полученные таким путем уравнения имеют решение, которое может
служить вектором параметров соответствующей ЭА
одномерного распределения, то можно предположить,
что стационарный в~узком смысле процесс в~системе существует. 
В~данном случае для определения других многомерных распределений
этого стационарного процесса следует заменить в~уравнениях~(\ref{e3.29-sin}) 
производные по~$t_n$ производными по
$\tau_{n-1}\hm=t_n\hm-t_1$, а~начальные условия~(\ref{e3.30-sin}) принять в~виде:

\noindent
\begin{multline}
c_{np,\kappa}\left(\Theta,\tau_1,\ldots,\tau_{n-2},\tau_{n-2}+\Delta\right)={}\\
{}=
    c_{(n-1)p,\kappa}\left(\Theta,\tau_1,\ldots,\tau_{n-2}\right)\,,
    \label{e3.37-sin}
    \end{multline}
где $\Delta$~--- малая величина, обеспечивающая невы\-рож\-денность
матрицы~$C$ в~уравнениях для параметров распределения.

Таким образом, в~основе эллипсоидальных МАМ
одно- и~$n$-мер\-ных распределений лежат следующие утверждения.

\smallskip

\noindent
\textbf{Теорема~3.1.}\ \textit{Пусть существует одномерное распределение СтП~$Y_t$ 
в~МСтС}~(\ref{e2.1-sin}). \textit{Тогда при фиксированном векторе параметров~$\Theta$ 
и~полиномиальных $\{ p_\nu, q_\nu\}$ в~основе алгоритма аналитического моделирования 
по}\linebreak\vspace*{-12pt}

\pagebreak

\noindent
\textit{МЭА лежат уравнения}~(\ref{e3.10-sin})--(\ref{e3.13-sin}) 
\textit{и}~(\ref{e3.18-sin}) \textit{при условии конечности интегралов}~(\ref{e3.17-sin}), 
(\ref{e3.19-sin}) \textit{и}~(\ref{e3.20-sin}) 
\textit{и~начальных условиях}~(\ref{e3.21-sin}). \textit{Количество уравнений МЭА 
на  $N/2\hm -1$ больше числа уравнений МНА
$Q^{\mathrm{МЭА}}\hm=Q^{\mathrm{МНА}}\hm+ N/2\hm-1$, 
$Q^{\mathrm{МНА}}\hm= p(p\hm+3)/2$. В~стационарном случае используются конечные 
уравнения}~(\ref{e3.36-sin}).

\smallskip

\noindent
\textbf{Теорема~3.2.}\ \textit{Пусть существует $n$-мер\-ное распределение СтП~$Y_t$ 
в~МСтС}~(\ref{e2.1-sin}). \textit{Тогда при фиксированном векторе параметров~$\Theta$ 
и~полиномиальных $\{ p_{np,\nu}, q_{np,\nu}\}$ в~основе алгоритма аналитического 
моделирования по МЭА при $n\hm=1$ лежат уравнения теоремы~$3.1$, а для  $n\hm>1$~--- 
уравнения}~(\ref{e3.22-sin}) и~(\ref{e3.29-sin}) \textit{при условиях}~(\ref{e3.30-sin}). 
\textit{В стационарном случае используются конечные уравнения}~(\ref{e3.37-sin}).


%\smallskip
\section{Эллипсоидальная аппроксимация одно- и~многомерных распределений 
в~стохастической системе на~многообразиях~(\ref{e2.2-sin})}

Пусть в~условиях теорем~3.1 и~3.2 дополнительно выполнены следующие условия:

\textbf{1.} Нелинейная векторная функция $a(Y_t,\Theta,t)$ допускает ЭЛ
согласно
\begin{multline*}
    a\left(Y_t,\Theta,t\right)\approx a_0 \left(m_t,K_t,c_1,\Theta,t\right) + {}\\
    {}+
    a_1 \left(m_t,K_t,c_1,\Theta,t\right) \left(Y_t-m_t\right)\,, %\label{e4.1-sin}
    \end{multline*}
где $a_{0} \hm=a_0 (m_t,K_t,c_1,\Theta,t)$ и~$a_{1}\hm= a_1
(m_t,K_t,c_1,\Theta,t)$ называются соответственно вектором смещения нуля 
и~матричным коэффициентом усиления ЭЛ, а~$c_1 \hm=\lf c_{p,\nu}\rf$~---
вектором структурных коэффициентов эллипсоидального распределения.

\textbf{2.} Одномерная плотность $f_1 (y;\Theta,t)$ существует и~имеет
эллипсоидальную структуру вида~(\ref{e3.4-sin}):
   \begin{equation}
   f_1^\ral (y;\Theta,t) = w_1 (\Theta,u)\lk 1+\sss\limits_{\nu=2}^N 
   c_{p,\nu} p_{p,\nu} (u)\rk\,,\label{e4.2-sin}
   \end{equation}
   где
   \begin{align*}
   w_1(\Theta,u) &=\fr{1}{\sqrt{(2\pi)^p \lv K_t\rv}}\,e^{-u/2}\,; %\label{e4.3-sin}
\\
     c_{p,\nu}&= \mm^\ral_{\Delta^y} \left[q_{p,\nu} (U_t)\right]\,;\\
   U_t&= \left(Y_t-m_t\right)^{\mathrm{T}} K_t^{-1} \left(Y_t-m_t\right)\,,
 %   \label{e4.4-sin}
\end{align*}
причем первые и~вторые моменты $m_t$ и~$K_t$ неизвестны, 
а~структурные коэффициенты одномерного эллипсоидального
распределения $c_1 \hm=\lf c_{p,\nu}\rf$ известны или известны
функции от~$m_t$  и~$K_t$.


При этих условиях и~соответствующих начальных условиях нелинейная
 дифференциальная МСтС~(\ref{e2.2-sin}) будет эквивалентна следующей
системе уравнений для~$m_t$ и~$Y_t^0\hm = Y_t\hm-m_t$:
\begin{multline}
    \dot m_t =A^{m} \left(m_t, K_t, c_1,\Theta,t\right)={}\\
    {}=
    a_0 \left(m_t,K_t,c_1,\Theta,t\right)\,,\quad
    m(t_0) = m_0\,;\label{e4.5-sin}
    \end{multline}
    
    \vspace*{-9pt}
    
\noindent
    \begin{equation}
    \left.
    \begin{array}{rl}
    \dot X_t^0 &= a_1 \left(m_t,K_t,c_1,\Theta,t\right) Y_t^0 +V\,;\\[6pt] 
    Y^0\!\left (t_0\right)&=
    Y_0^0\,.
    \end{array}
    \right\}
    \label{e4.6-sin}
    \end{equation}
Здесь $V$~--- составной белый шум, равный
   \begin{equation*}
   V= b_0' (\Theta,t) V_0 +V_P\,. %\label{e4.7-sin}
   \end{equation*}

Для вычисления параметров %$a_0$, $a_1$ и~$\overline{\sigma}_{0t}$
МЭЛ используются фор\-мулы:
    \begin{multline*}
    a_{0}=a_0\left(m_t,K_t,c_1,\Theta,t\right)=a_{01} \left(m_t,K_t,\Theta,t\right)
    +{}\\
    {}+\sss_{\nu=2}^N c_{p,\nu}a_{1\nu} \left(m_t,K_t,\Theta,t\right),
    \end{multline*}
где
$$
    a_{01} \left(m_t,K_t,\Theta,t\right) =\mm_{\Delta^y}^{w_1} 
    \left[ a(Y, \Theta, t)\right];
$$
    \begin{multline}
    a_{1}=a_1 \left(m_t,K_t,c_1,\Theta,t\right) =a_{10}
    \left(m_t,K_t,\Theta,t\right) +{}\\
    {}+ \sss_{\nu=2}^N c_{p,\nu} a_{1,p,\nu}
    \left(m_t,K_t,\Theta,t\right)\,,
    \label{e4.8-sin}
    \end{multline}
    где
    \begin{multline*}
    a_{10} \left(m_t,K_t,\Theta,t\right) ={}\\
    {}=\mm_{\Delta^y}^{w_1} 
    \left[a(Y,\Theta, t) \left(Y-m_t\right)^{\mathrm{T}} K_t^{-1}\right]\,,
    \end{multline*}
    
    \vspace*{-12pt}

    \noindent
    \begin{multline*}
a_{1,p,\nu}(m_t,K_t,\Theta,t)={}\\
{}= \mm_{\Delta^y}^{p_{p,\nu}w_1} 
\left[a(Y,\Theta, t) \left(Y-m_t\right)^{\mathrm{T}} K_t^{-1}\right]\,;
\end{multline*}

 \vspace*{-12pt}

    \noindent
    \begin{multline*}
\bar\si_{0t}={}\\
{}=\si_0 (\Theta,t) +\iii_{R_0^q} c_0(\Theta,v) c_0(\Theta,v)^{\mathrm{T}} \nu_P (\Theta,t)
   \, dv\,,
%   \label{e4.10-sin}
    \end{multline*}
    где
    \begin{equation}
\si_{0}(\Theta,t) = b_0(\Theta,t)\nu_0(\Theta,t) b_0(\Theta,t)^{\mathrm{T}}\,.
\label{e4.11-sin}
\end{equation}

В силу линейности~(\ref{e4.6-sin})  уравнение для ковариационной матрицы~$K_t$ 
при соответствующих начальных условиях имеет вид:
\begin{multline}
\dot K_t=A^K \left(m_t, K_t, \Theta, t\right) =
a_{1,t} K_t + K_t a_{1,t}^{\mathrm{T}} +\bar\si_{0,t} \,,
    \\ K(t_0)= K_0\,.
        \label{e4.12-sin}
    \end{multline}
Отсюда следует  утверждение.

\smallskip

\noindent
\textbf{Теорема~4.1.}\ \textit{В основе МЭЛ нахождения одномерной плотности}~(\ref{e4.2-sin}) 
\textit{с известными структурными коэффициентами $c_1\hm=\lf
c_{p,\nu}\rf$ для МСтС}~(\ref{e2.2-sin}) \textit{при услови-}\linebreak\vspace*{-12pt}

\pagebreak

\noindent
\textit{ях~$1$--$2$, 
лежат уравнения}~(\ref{e4.5-sin}) \textit{и}~(\ref{e4.12-sin})
\textit{при условиях}~(\ref{e4.8-sin}) \textit{и}~(\ref{e4.11-sin}).
\textit{Количество уравнений для параметров одномерного распределения
$Q_1^\mal \hm=Q_1^{\mathrm{МНА}}\hm =p(p\hm+3)/2$.
В~основе МЭЛ нахождения эллипсоидальной стационарной
плотности  вида}~(\ref{e4.2-sin}) \textit{с~известными
структурными коэффициентами $c_1\hm=\lf c_{p,\nu}\rf$ для
стационарного (в узком смысле) процесса в~стационарной МСтС}~(\ref{e2.2-sin}) 
\textit{при условиях~$1$--$2$ лежат
уравнения}~(\ref{e4.5-sin}) \textit{и}~(\ref{e4.12-sin}) \textit{при $\dot
m_t \hm=0$ и~$\dot K_t\hm=0$}:
\begin{equation}
\left.
\begin{array}{rl}
A^m \left(m,K,c_1, \Theta\right)& =a_0 \left(m,K,c_1,\Theta\right) =0\,; %\label{e4.13-sin}
\\[6pt]
A^K \left(m,K,c_1, \Theta\right) &=a_1 \left(m,K,c_1,\Theta\right) K+{}\\[6pt]
&\hspace*{-10mm}{}+ 
K a_1 \left(m,K,c_1,\Theta\right)^{\mathrm{T}} +\bar\si_0=0\,.
\end{array}
\right\}
\label{e4.14-sin}
\end{equation}

%\smallskip

Теперь получим основные уравнения МЭЛ для $n$-мер\-ных распределений.
Дополнительно к~условиям~1--2 предположим:

\textbf{3.} 
 $n$-мерная плот\-ность $f_n \hm= f_n(y_1\tr y_n;$\linebreak $\Theta, t_1\tr t_n)$
существует и~имеет эллипсоидальную структуру вида
  \begin{multline*}
  f_n^\ral ( y_1\tr y_n ;\Theta, t_1\tr t_n) = f_n^\ral (u,c_n)={}\\
  {}= w_n (\Theta,u) \lk 1+\sss_{\nu=2}^N c_{np,\nu} p_{np,\nu}
    (u)\rk, %\label{e4.15-sin}
    \end{multline*}
    где
\begin{align*}
 w_n(\Theta,u)&= \lk (2\pi)^n \lv K_n\rv
    \rk^{-1/2} e^{-u/2}\,; %\label{e4.16-sin}
\\
c_{np,\nu}&= \mm_{\Delta^y}^\ral \left[q_{np,\nu} (U)\right]\,;\\
U&= \left(\tilde Y_n - \tilde m_n\right) K_n^{-1} \left(\tilde Y_n -
    \tilde m_n\right).
    %    \label{e4.17-sin}
    \end{align*}
Здесь
\begin{gather*}
\tilde Y_n =\lk Y_1^{\mathrm{T}} Y_2^{\mathrm{T}}
    \cdots Y_n^{\mathrm{T}}\rk^{\mathrm{T}}\,,\enskip Y_l = Y\left(t_l\right)\,; 
   % \label{e4.18-sin}
   \\
\tilde m_n =\lk m_1^{\mathrm{T}} m_2^{\mathrm{T}} \cdots m_n^{\mathrm{T}}\rk^{\mathrm{T}}\,,\enskip m_l = m(t_l)\,;
%    \label{e4.19-sin}
\\
K_n =\begin{bmatrix} 
K(t_1, t_1)& K(t_1, t_2)& \cdots& K(t_1, t_n)\\
K(t_2, t_1)& K(t_2, t_2)& \cdots& K(t_2,     t_n)\\
\vdots& \vdots& \vdots& \vdots\\
K(t_n, t_1)& K(t_n,     t_2)& \cdots& K(t_n, t_n)
\end{bmatrix}\,. %\label{e4.20-sin}
\end{gather*}
Структурные коэффициенты $c_n\hm =\lf c_{np,\nu}\rf$ предполагаются,
во-пер\-вых, известными, во-вто\-рых, удовле\-тво\-ря\-ющими условиям согласованности.

В условиях~1--3 для получения уравнений МЭЛ применительно 
к~$f_n^\ral$ достаточно воспользоваться известными уравнениями
теории линейных СтС, основываясь на уравнениях~(\ref{e4.5-sin}) и~(\ref{e4.6-sin}).

\smallskip

\noindent
\textbf{Теорема~4.2.}\ \textit{В основе МЭЛ $n$-мер\-ной плотности при условиях~$1$--$3$ 
лежат уравнения квазилинейной теории для}~(\ref{e4.5-sin}) \textit{и}~(\ref{e4.6-sin}).

\smallskip

В случае двумерного эллипсоидального распределения $(n\hm=2)$, полагая
\begin{equation*}
a\left(Y\left(t_2\right),\Theta,t_2\right)=a_{0,\Theta,t_2} +
    a_{1,\Theta,t_2} \left(Y_{t_2} - m_{t_2}\right)\,, %\label{e4.21-sin}
    \end{equation*}
где
    \begin{align*}
    a_{0,\Theta,t_2}&= a_0 \left(m_{t_2}, K_{t_2}, c_{1,t_2}, t_2\right)\,;\\
    a_{1,\Theta,t_2}&=a_1 \left(m_{t_2}, K_{t_2}, c_{1,t_2}, \Theta,t_2\right)\,,
    \end{align*}
для стационарного случая, когда
    \begin{gather*}
    K\left(\Theta,t_1, t_2,a_{1,t_2}\right) = k \left(\tau, \alp_1\right)
    \,,\\
     \tau=     t_1-t_2\,,\enskip a_1 = a_1\left(m_t,K_t,c_1,\Theta\right)\,,
     \end{gather*}
уравнения теоремы~4.2 приобретают вид:
\begin{equation*}
\fr{dk(\Theta,\tau,a_1)}{d\tau} =a_1 k\left(\Theta,\tau,a_1\right)\,,
\end{equation*}

\vspace*{-12pt}

\noindent
\begin{alignat*}{2}
  k\left(0,a_1\right)&=K_t, t=t_1=t_2 &\  \mbox{при}\  \tau&>0\,,\ \ \\[3pt]
k\left(\Theta,\tau,a_1\right) &=k\left(\Theta,-\tau,\varphi_1\right)^{\mathrm{T}}&\ 
\mbox{при}\  \tau&<0\,.
%\label{e4.22-sin}
\end{alignat*}

Таким образом,  квазилинейная эллипсоидальная
спект\-раль\-но-кор\-ре\-ля\-ци\-он\-ная тео\-рия процессов в~стационарной
МСтС~(\ref{e2.2-sin}) основана на  уравнениях~(\ref{e4.14-sin}) и~следующих
соотношениях для эквивалент\-ной передаточной функции, спектральной
плотности и~ковариационной функции:
\begin{gather*}
Y^0 =\Phi \left(\Theta, s,a_1\right) V\,,\enskip 
\Phi \left(\Theta,s,a_1\right) =-\left(a_1-I_p s\right)^{-1}\,; %\label{e4.23-sin}
\\
s_x\left(\Theta,\w,a_1\right)=\Phi \left(\Theta,i\w,a_1\right)
s_{_V}\Phi\left(\Theta,i\w,a_1\right)^*\,;\\
k_x\left(\Theta,\tau,a_1\right) =
\int\limits_{-\infty}^\infty s_x \left(\Theta,\w,a_1\right) e^{i\w
    \tau}\, d\tau
%    \label{e4.24-sin}
    \end{gather*}
при условии асимптотической устойчивости функции
$\Phi(\Theta,s,\vrp_1)$.

\section{Точность и~чувствительность алгоритмов, основанных 
на~методах эллипсоидальной аппроксимации и~эллипсоидальной
линеаризации}

В задачах надежности и~безопасности технических систем~[13--15]
точность алгоритмов, основанных на теоремах~3.1, 3.2, 4.1 и~4.2,
оценивается исходя из выбранного критерия качества. Широ\-кое
распространение получили метод сравнения вероятностей попадания на
множества (в~данном случае эллипсоида) и~метод оценки вероятностных
моментов четвертого порядка.

В инженерных приложениях, как правило, в~основе оценок точности и~чувствительности 
ал-\linebreak\vspace*{-12pt}

\pagebreak

\noindent
горитмов к~параметрам~$\Theta$ лежат методы теории чувствительности~\cite{14-sin, 15-sin}. 
В~условиях разд.~2--4 соответствующие уравнения для первых функций чувствительности 
получаются путем дифференцирования по~$\Theta$ правых и~левых частей уравнений МЭА 
(МЭЛ).

В силу~(\ref{e3.12-sin}), (\ref{e3.13-sin}) и~(\ref{e3.18-sin}) 
придем к~следующим уравнениям для функций чувствительности 
первого порядка~$\nabla m_t$, $\nabla K_t$ и~$\nabla c_{p, \kappa,t}$ 
$(\nabla\hm= \prt/\prt\Theta)$:
    \begin{equation}
    \left.
    \begin{array}{rlrl}
    \nabla \dot m_t &= \nabla A^m\,, &\  \nabla m_{t_0} &=0\,; %\label{e5.1-sin}
    \\[6pt]
\nabla \dot K_t &= \nabla A^K\,, &\  \nabla K_{t_0} &=0\,; %\label{e5.2-sin}
\\[6pt]
\nabla \dot c_{p,\kappa, t}& = \nabla A^{c_{p,\kappa}}\,, &\ 
\nabla c_{p,\kappa,t_0} &=0\,.
\end{array}
\right\}
\label{e5.3-sin}
\end{equation}
Заметим, что при дифференцировании по~$\Theta$ порядок уравнений
возрастает пропорционально числу производных.

Аналогично~\cite{9-sin, 10-sin} выписываются уравнения для функций чувствительности 
второго порядка $\nabla(\nabla)^{\mathrm{T}} m_t$, 
$\nabla(\nabla)^{\mathrm{T}} K_t$ и~$\nabla(\nabla)^{\mathrm{T}} c_{p,\kappa,t}$ 
для одномерных распределений, а~также  $n$-мер\-ных распределений.

Алгоритмы МАМ, основанные на МЭА (МЭЛ), содержат только на $N/2\hm-1$ уравнений 
больше, чем алгоритмы, основанные на МНА (МСЛ). В~этом их существенное преимущество 
по сравнению с~алгоритмами МОР и~метода квазимоментов~\cite{9-sin, 10-sin}.

\section{Пример}

Следуя~\cite{1-sin, 2-sin, 11-sin}, рассмотрим двумерную угловую МСтС вида
\begin{alignat*}{2}
\dot Y_1 &=-Y_1 Y_2 \,;&\enskip \dot Y_2 &= -a Y_2 + h V(\Theta)\,;\\
Y_1 (t_0) &= Y_{10}\,;&\enskip Y_2 (t_0) &= Y_{20}\,,
%\label{e6.1-sin}
\end{alignat*}
где $V= V(\Theta)$~--- угловой гауссовский белый шум, зависящий от скалярного 
параметра~$\Theta$ с~интенсивностью  $\nu\hm=\nu(\Theta)$. В~силу теоремы~3.1, 
ограничиваясь моментами до четвертого порядка включительно, получим следующие 
уравнения для $m_1$, $m_2$, $K_{11}$, $K_{12}$, $K_{22}$ и~$c_{2,2}$ параметров 
ЭА одномерной плот\-ности~(\ref{e3.10-sin}):
\begin{equation}
\left.
\begin{array}{rl}
\dot m_1 &=-m_1m_2 - K_{12}\,,\enskip \dot m_2 = -a m_2\,;  %\label{e6.2-sin}
\\[6pt]
\dot K_{11} &= -2 \left(m_2 K_{11} + m_1 K_{12}\right)\,;\\[6pt]
 \dot K_{12} &= - \left(m_2 + a\right) K_{12} - m_1 K_{22}\,;\\[6pt]
 \dot K_{22} &= \nu h^2 - 2 a K_{22}\,;
 \\[6pt]
 %\label{e6.3-sin}
 \dot c_{2,2} &= 4 h^2 \nu  C_{22} + {}\\[6pt]
 &\hspace*{10mm}{}+c_{2,2} \left[ \left(m_2 +a\right) 
\left(6+8C_{12} K_{12}\right) +{}\right.\\[6pt]
&\hspace*{16mm}\left.{}+ 8 m_1 C_{12} K_{12} - 3 h^2 \nu C_{22}\right]\,.
\end{array}\!\!
\right\}\!\!
\label{e6.4-sin}
\end{equation}
Здесь $C=\lk C_{ij}\rk$, $C\hm=K^{-1}$, $K\hm= \lk K_{ij}\rk$ $(i,j\hm=1,2)$. 
Уравнениям~(\ref{e6.4-sin}) в~силу~(\ref{e5.3-sin}) 
отвечают следующие уравнения для функций чувствительности парамет\-ров ЭА:
\begin{equation}
\left.
\begin{array}{c}
\nabla \dot m_1 =- m_2 \nabla m_1 - m_1 \nabla m_2 - \nabla K_{12}\,; \\[6pt]
\nabla m_{10}=0\,; \enskip \nabla \dot m_2 = - a \nabla m_2\,;\enskip 
\nabla m_{20} =0\,;
\\[6pt]
%\label{e6.5-sin}
\hspace*{-10mm}\nabla \dot K_{11} = -2 K_{12} \nabla m_1 - 2 K_{11} \nabla m_2 - {}\\[6pt]
{}-2 m_2 \nabla K_{11} - 2 m_1 \nabla K_{12}\,;\  \nabla K_{11,0} =0\,;\\[6pt]
\hspace*{-12mm}\nabla \dot K_{12} = - K_{22} \nabla m_1 -  K_{12} \nabla m_2 - {}\\[6pt]
\hspace*{-2mm}{}-\left( m_2+a\right) \nabla K_{12} -  m_1 \nabla K_{22}\,;\
\nabla K_{12,0} =0\,;\\[6pt]
\hspace*{-10mm}\nabla \dot K_{22} = h^2  -2a K_{22}\,;\enskip \nabla K_{22,0} =0\,;
%\label{e6.6-sin}
\\[6pt]
\hspace*{-18mm}\nabla \dot c_{2,2} =8c_{2,2} C_{12} K_{12} \nabla m_1 +{}\\[6pt]
\hspace*{5mm}{}+\left(6+8C_{12} K_{12}\right) c_{2,2} \nabla m_2 +{}\\[6pt]
\hspace*{5mm}{}+ 8 C_{12} c_{2,2} 
\left(m_1 + m_2 +a\right) \nabla K_{12}+{}\\[6pt]
\hspace*{5mm}{}+8K_{12} c_{2,2} \left(m_1 + m_2+a\right) \nabla C_{12}+{}\\[6pt]
\hspace*{6mm}{}+  h^2 
\left(4+ 3\nu c_{2,2}\right)\nabla C_{22}+{}\\[6pt]
 \hspace*{12mm} {}+\left[ \left(m_2 +a\right) \left(6+8C_{12} K_{12}\right) + {}\right.\\[6pt]
\hspace*{12mm}  \left.{}+8 C_{12} K_{12} - 
  3 h^2 \nu C_{22}\right] \nabla c_{2,2}\,,\\[6pt] 
  \hspace*{30mm}\nabla c_{2,20} =0\,.
  \end{array}
  \right\}
  \label{e6.7-sin}
  \end{equation}

Для стационарного случая в~уравнениях~(\ref{e6.4-sin}) и~(\ref{e6.7-sin}) 
надо приравнять нулю их  правые части.

При вычислении вероятностных моментов второго порядка МЭА совпадает с~МНА. 
Если, следуя~\cite{11-sin}, ограничиться вероятностными моментами до четвертого порядка 
включительно, положив $t\hm\in [0,1]$, $\alp\hm=5$, $h\hm=1$, $m_1 (0) \hm= 0{,}5$, 
$m_2(0)\hm= 0{,}5$, $K_{11}(0) = D_1(0)=0{,}1$, $K_{22} (0)\hm=D_2(0)=0{,}1$, 
$K_{12}(0)\hm=0$, то точность МЭА при вычислении начального момента~$\alp_4$ составит 
около 2\% по отношению к~точному решению. Такую же чувствительность к~параметру  
$\nu \hm= \Theta$ проявляют  $D_1$, $D_2$, $K_{12}$ и~$c_{2,2}$.

В~\cite{11-sin} показано, что точность вычисления начального момента~$\alp_6$ 
составляет  8\%, а начального момента $\alp_8$~--- 20\%. Этого обычно достаточно 
для приложений в~задачах надежности и~безопасности технических систем.

\section{Заключение}


Даны обобщения алгоритмов эллипсоидального анализа и~моделирования одно- 
и~многомерных распределений в~нелинейных МСтС 
с~винеровскими и~пуассоновскими шумами. Особое
внимание уделено алгоритмам для систем с~аддитивными негауссовскими шумами.

Получены уравнения точности и~чувствительности к~параметрам. 
В~качестве иллюстрации рассмотрена нелинейная двумерная угловая система 
с~аддитивным угловым белым шумом. Составлены\linebreak уравнения точности и~чувствительности, 
позволяющие оценивать моменты до четвертого порядка включительно. Приведены результаты 
численных расчетов, показывающие эффективность МЭА (МЭЛ) по сравнению с~общим МОР
 одномерной плот\-ности. Сформулированы некоторые возможные 
обобщения.

В качестве обобщений можно рассмотреть задачи аналитического моделирования 
процессов в~дискретных МСтС, а~также задачи 
синтеза субоптимальных фильтров для обработки информации\linebreak в~таких стохастических 
нелинейных системах.


{\small\frenchspacing
 {%\baselineskip=10.8pt
 \addcontentsline{toc}{section}{References}
 \begin{thebibliography}{99}

\bibitem{1-sin}
 \Au{Пугачёв В.\,С., Синицын~И.\,Н.}
Стохастические дифференциальные системы. Анализ и~фильтрация.~--- М.:
Наука,  1990.  632~с. (\Au{Pugachev~V.\,S., Sinitsyn~I.\,N.}
Stochastic differential systems.
Analysis and filtering.~--- Chichester\,--\,New York, NY, USA: Jonh Wiley, 1987.
549~p.)

\bibitem{2-sin}
 \Au{Пугачёв В.\,С., Синицын~И.\,Н.}
Теория стохастических систем.~--- М.: Логос, 2000; 2004. 1000~с.
%(\Au{Pugachev~V.\,S., Sinitsyn~I.\,N.} Stochastic systems. Theory and  applications.~---
%Singapore: World Scientific, 2001. 908~p.)


\bibitem{3-sin}
 \Au{Синицын И.\,Н.} 
Стохастические информационные технологии для исследования нелинейных круговых 
стохастических систем~// Информатика и~её применения, 2011. Т.~5. Вып.~4. С.~78--89.

\bibitem{4-sin}
\Au{Sinitsyn I.\,N., Belousov~V.\,V., Konashenkova~T.\,D.} 
Software tools for circular stochastic systems analysis~// 
29th  Seminar (International) on Stability Problems for Stochastic Models: Abstracts.~--- 
Svetlogorsk, Russia, 2011. Р.~86--87.


\bibitem{5-sin}
\Au{Синицын И.\,Н. } Математическое обеспечение для анализа 
нелинейных многоканальных круговых стохастических систем, основанное на 
параметризации распределений~// Информатика и~её применения, 2012. Т.~6. 
Вып.~1. С.~12--18.

\bibitem{6-sin}
\Au{Синицын И.\,Н., Корепанов Э.\,Р., Белоусов~В.\,В., Конашенкова~Т.\,Д.} 
Развитие математического обеспечения для анализа нелинейных многоканальных круговых 
стохастических сис\-тем~// Системы и~средства информатики, 2012. Вып.~22. №~1. С.~29--40.

\bibitem{7-sin}
\Au{Sinitsyn I.\,N., Belousov V.\,V., Konashenkova~T.\,D.} 
Software tools for spherical stochastic systems analysis and filtering~//  
Прикладные задачи теории веро\-ят\-ности и~математической статистики, связанные 
с~моделированием информационных систем.~--- М.: ИПИ РАН, 2012. С.~91--93.

\bibitem{8-sin}
\Au{Синицын И.\,Н.}
Параметрическое статистическое и~аналитическое моделирование распределений 
в~нелинейных стохастических системах на многообразиях~// Информатика и~её применения, 2013. Т.~7. Вып.~2. С.~4--16.


\bibitem{9-sin}
\Au{Синицын И.\,Н.}
Аналитическое моделирование распределений методом ортогональных разложений 
в~нелинейных стохастических системах на многообразиях~// Информатика и~её 
применения, 2015. Т.~9. Вып.~3. С.~17--24.

\bibitem{10-sin}
\Au{Синицын И.\,Н.}
Применение ортогональных разложений для аналитического моделирования многомерных 
распределений в~нелинейных стохастических системах на многообразиях~// Системы 
и~средства информатики, 2015. Т.~25. №\,3. С.~3--22.

\bibitem{11-sin}
\Au{Синицын И.\,Н., Синицын В.\,И.}
Лекции по теории нормальной и~эллипсоидальной аппроксимации распределений 
в~стохастических системах.~--- М.: ТОРУС ПРЕСС, 2013. 488~с.


\bibitem{12-sin}
\Au{Ватанабэ С., Икэда Н.} Стохастические дифференциальные уравнения 
и~диффузионные процессы~/ Пер. с~англ.~--- М.: Наука, 1986. 448~с.
(\Au{Watanabe~S, Ikeda~N.} 
Stochastic differential equations and diffusion processes.~--- 
Amsterdam\,--\,Oxford\,--\,New York: North-Holland Publishing Co.; 
Tokyo: Kodansha Ltd., 1981. 476~p.)






\bibitem{14-sin}
\Au{Евланов А.\,Г., Константинов~В.\,М. }
Системы со случайными параметрами.~--- М.: Наука, 1976. 568~с.

\bibitem{15-sin}
Справочник по теории автоматического управления~/ Под ред. 
А.\,А.~Красовского.~--- М.: Наука, 1987. 712~с.

\bibitem{13-sin} %15
\Au{Александровская Л.\,Н., Аронов И.\,З., Круглов~В.\,И. и~др.}
 Безопасность и~надежность технических сис\-тем.~--- 
 М.: Университетская книга, Логос, 2008. 376~с.
\end{thebibliography}

 }
 }

\end{multicols}

\vspace*{-3pt}

\hfill{\small\textit{Поступила в~редакцию 02.12.15}}

%\vspace*{8pt}

\newpage

\vspace*{-24pt}

%\hrule

%\vspace*{2pt}

%\hrule

%\vspace*{8pt}

\def\tit{ANALYTICAL MODELING OF~DISTRIBUTIONS\\  IN~STOCHASTIC SYSTEMS ON~MANIFOLDS\\
 BASED ON~ELLIPSOIDAL APPROXIMATION}

\def\titkol{Analytical modeling of~distributions  in~stochastic systems on~manifolds based
on~ellipsoidal approximation}

\def\aut{I.\,N.~Sinitsyn and V.\,I.~Sinitsyn}

\def\autkol{I.\,N.~Sinitsyn and V.\,I.~Sinitsyn}

\titel{\tit}{\aut}{\autkol}{\titkol}

\vspace*{-9pt}

\noindent
Institute of Informatics Problems, Federal Research Center 
``Computer Science and Control'' of the Russian Academy of Sciences,
44-2 Vavilov Str., Moscow 119333, Russian Federation

\def\leftfootline{\small{\textbf{\thepage}
\hfill INFORMATIKA I EE PRIMENENIYA~--- INFORMATICS AND
APPLICATIONS\ \ \ 2016\ \ \ volume~10\ \ \ issue\ 1}
}%
 \def\rightfootline{\small{INFORMATIKA I EE PRIMENENIYA~---
INFORMATICS AND APPLICATIONS\ \ \ 2016\ \ \ volume~10\ \ \ issue\ 1
\hfill \textbf{\thepage}}}

\vspace*{3pt}

\Abste{Accuracy and sensitivity problems for algorithms of structural analytical 
modeling of one- and multidimensional distributions based on the
method of ellipsoidal approximation (MEA) and the
method of ellipsoidal linearization (MEL) are considered. 
General algorithms of ellipsoidal stochastic analysis in nonlinear stochastic 
systems on manifolds (MStS) with Wiener and Poisson noises are developed. 
Special attention is paid to MStS with additive noises. Accuracy and sensitivity 
equations are presented. For two-dimensional nonlinear circular MStS, accuracy and sensitivity 
equations are derived. Equations make it possible to calculate probability moments up to 
the fourth order. Algorithms based on orthogonal expansion of one-dimensional density are compared with algorithms based on MEA (MEL). Some generalizations are given.}

\KWE{accuracy equations;
analytical modeling;
method  of ellipsoidal linearization (MEL);
method of ellipsoidal approximation (MEA);
sensitivity equations;
stochastic system on manifold (MStS)}


\DOI{10.14357/19922264160104} 

\Ack
\noindent
The work was supported by the Russian Foundation for
Basic Research (project 15-07-02244).



\vspace*{9pt}

  \begin{multicols}{2}

\renewcommand{\bibname}{\protect\rmfamily References}
%\renewcommand{\bibname}{\large\protect\rm References}

{\small\frenchspacing
 {%\baselineskip=10.8pt
 \addcontentsline{toc}{section}{References}
 \begin{thebibliography}{99}

\bibitem{1-sin-1}
\Aue{Pugachev, V.\,S., and I.\,N.~Sinitsyn}. 1987.
Stochastic differential systems. Analysis
and filtering. -- Chichester\,--\,New York, NY: Jonh Wiley, 1987. 549~p.

\bibitem{2-sin-1}
 \Aue{Pugachev, V.\,S., and I.\,N.~Sinitsyn.} 
 2001.  \textit{Stochastic systems. Theory and  applications}.
Singapore: World Scientific. 908~p.


\bibitem{3-sin-1}
\Aue{Sinitsyn, I.\,N. } 2011.  Stokhasticheskie informatsionnye 
tekhnologii dlya issledovaniya nelineynykh krugovykh stokhasticheskikh sistem 
[Stochastic informational technologies for circular stochastic systems investigation]. 
\textit{Informatika i~ee Primeneniya}~--- \textit{Inform. Appl.} 5(4):78--89.

\bibitem{4-sin-1}
\Aue{Sinitsyn, I.\,N., V.\,V.~Belousov, and T.\,D.~Konashenkova.} 
Software tools for circular stochastic systems analysis. 
\textit{29th Seminar (International) on Stability Problems 
for Stochastic Models: Abstracts}. Svetlogorsk, Russia. 86--87.


\bibitem{5-sin-1}
\Aue{Sinitsyn, I.\,N. } 2012.
Matematicheskoe obespechenie dlya analiza nelineynykh mnogokanal'nykh krugovykh 
stokhasticheskikh sistem, osnovannoe na parametri\-za\-tsii raspredeleniy 
[Mathematical software for analysis of nonlinear multichannel circular stochastic 
systems based on distributions parametrization]. 
\textit{Informatika i~ee Primeneniya}~--- \textit{Inform. Appl}. 6(1):12--18.

\bibitem{6-sin-1}
\Aue{Sinitsyn, I.\,N., E.\,R.~Korepanov, V.\,V.~Belousov, 
and T.\,D.~Konashenkova.}  2012.
Razvitie matematicheskogo obespecheniya dlya analiza nelineynykh mno\-go\-ka\-nal'\-nykh 
krugovykh stokhasticheskikh sistem [Development of mathematical software for 
analysis of nonlinear mutlichannel circular stochastic systems]. 
\textit{Sistemy i~Sredstva Informatiki}~--- \textit{Systems and Means of Informatics}
 22(1):29--40.

\bibitem{7-sin-1}
\Aue{Sinitsyn, I.\,N., V.\,V.~Belousov, and T.\,D.~Konashenkova.}  2012.
Software tools for spherical stochastic systems analysis and filtering.  
\textit{Prikladnye zadachi teorii ve\-ro\-yat\-nosti i~matematicheskoy statistiki, svyazannye 
s~modelirovaniem informatsionnykh sistem}
[Applied problems of probability theory and mathematical
statistics related to modeling
of information systems]. Moscow:   IPI RAN. 91--93.

\bibitem{8-sin-1}
\Aue{Sinitsyn, I.\,N.} 2013.
Parametricheskoe statisticheskoe i~analiticheskoe modelirovanie raspredeleniy 
v~ne\-li\-ney\-nykh stokhasticheskikh sistemakh na mnogoobraziyakh [Parametric statistical 
and analytical modeling of distributions in nonlinear stochastic systems on manifolds]. 
\textit{Informatika i~ee Primeneniya}~--- \textit{Inform. Appl}. 7(2):4--16.

\bibitem{9-sin-1}
\Aue{Sinitsyn, I.\,N.} 2015.
Analiticheskoe modelirovanie ras\-predeleniy metodom ortogonal'nykh razlozheniy 
v~nelineynykh stokhasticheskikh sistemakh na mno\-go\-ob\-ra\-ziyakh [Analytical 
modeling in stochastic systems on\linebreak manifolds based on orthogonal expansions].
\textit{Informatika i~ee Primeneniya}~--- \textit{Inform. Appl}. 9(3):17--24.

\bibitem{10-sin-1}
\Aue{Sinitsyn, I.\,N.} 2015.
Primenenie ortogonal'nykh raz\-lo\-zheniy dlya analiticheskogo modelirovaniya 
mno\-go\-mer\-nykh 
raspredeleniy v~nelineynykh stokhasticheskikh sistemakh na mno\-go\-ob\-ra\-zi\-yakh 
[Applications of orthogonal\linebreak\vspace*{-12pt}

\pagebreak

\noindent
 expansions for analytical modeling of multi-dimensional 
distributions in nonlinear stochastic systems on manifolds].
\textit{Sistemy i~Sredstva Informatiki}~--- \textit{Systems and Means of Informatics}
 25(3):3--22.

\bibitem{11-sin-1}
\Aue{Sinitsyn, I.\,N., and V.\,I.~Sinitsyn.} 2013.
\textit{Lektsii po teorii normal'noy i~ellipsoidal'noy approkskimatsii raspredeleniy
 v~stokhasticheskikh sistemakh} [Lectures on normal and ellipsoidal approximation 
 of distributions in stochastic systems].  Moscow: TORUS PRESS.  488~p.


\bibitem{12-sin-1}
\Aue{Watanabe,~S., and N. Ikeda}. 1981. 
\textit{Stochastic differential equations and diffusion processes}. 
Amsterdam\,--\,Oxford\,--\,New York: North-Holland Publishing Co.; 
Tokyo: Kodansha Ltd. 476~p.



\bibitem{14-sin-1}
\Aue{Evlanov, A.\,G., and V.\,M.~Konstantinov.} 1976.
\textit{Sistemy so slozhnymi parametrami} [Systems with random parameters]. 
Moscow: Nauka. 568~p.

\bibitem{15-sin-1}
Krasovskskii, A.\,A., ed.  1987.
\textit{Spravochnik po teorii avtomaticheskogo upravleniya} 
[Handbook for automatic control]. Moscow:   Nauka. 712~p.

\bibitem{13-sin-1} %15
\Aue{Aleksandrovskaya,~L.\,N., I.\,Z.~Aronov, V.\,I.~Kruglov, \textit{et al.}} 2008.
\textit{Bezopasnost' i~nadezhnost' tekhnicheskikh sistem} [Security and reliability 
of technical systems]. Moscow:  Universitetskaya Kniga, Logos.  
376~p.
\end{thebibliography}

 }
 }

\end{multicols}

\vspace*{-3pt}

\hfill{\small\textit{Received December 2, 2015}}

\Contr

\noindent
\textbf{Sinitsyn Igor N.} (b.\ 1940)~---
Doctor of Science in technology, professor,
Honored scientist of RF, Head of Department, Institute of Informatics Problems, Federal Research Center ``Computer Science and
Control'' of the Russian Academy of Sciences, 44-2 Vavilov Str.,
Moscow 119333, Russian Federation; sinitsin@dol.ru

\vspace*{3pt}

\noindent
\textbf{Sinitsyn Vladimir I.} (b.\ 1968)~---
 Doctor of Science in physics and mathematics,
associate professor, Head of Department, Institute of Informatics Problems, Federal Research Center ``Computer Science and
Control'' of the Russian Academy of Sciences, 44-2 Vavilov Str.,
Moscow 119333, Russian Federation; vsinitsin@ipiran.ru

\label{end\stat}


\renewcommand{\bibname}{\protect\rm Литература}