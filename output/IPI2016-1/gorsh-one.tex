\def\stat{gorshenin-one}

\def\tit{КОНЦЕПЦИЯ ОНЛАЙН-КОМПЛЕКСА ДЛЯ СТОХАСТИЧЕСКОГО МОДЕЛИРОВАНИЯ РЕАЛЬНЫХ ПРОЦЕССОВ$^*$}

\def\titkol{Концепция онлайн-комплекса для стохастического моделирования реальных процессов}

\def\aut{А.\,К.~Горшенин$^1$}

\def\autkol{А.\,К.~Горшенин}

\titel{\tit}{\aut}{\autkol}{\titkol}

{\renewcommand{\thefootnote}{\fnsymbol{footnote}} \footnotetext[1]
{Работа выполнена при поддержке
РФФИ (проект 15-37-20851 мол\_а\_вед).}}


\renewcommand{\thefootnote}{\arabic{footnote}}
\footnotetext[1]{Институт проб\-лем информатики Федерального исследовательского центра 
<<Информатика и~управ\-ле\-ние>> Российской академии наук; 
%Федеральное государственное бюджетное образовательное учреждение высшего образования 
Московский технологический университет (МИРЭА),
 agorshenin@frccsс.ru}


\Abst{Анализ информационных потоков с~использованием разнообразных вероятностных 
моделей достаточно широко распространен в~различных прикладных областях. 
В~статье описаны основные принципы построения новой он\-лайн-сис\-те\-мы 
стохастического моделирования реальных процессов, не имеющей прямых аналогов 
в~силу универсальности предлагаемого набора методов, а также выбранной концепции 
реализации в~виде ин\-тер\-нет-ре\-сур\-са, который позволит конечному пользователю 
не заботиться о соответствии технических характеристик своего компьютера ка\-ким-ли\-бо 
специальным требованиям, а сразу загружать данные на сервер и~обрабатывать их.}

\KW{смеси вероятностных распределений; метод скользящего разделения смесей; 
интеллектуальный анализ данных; он\-лайн-ком\-плекс; матричные вычисления}

\DOI{10.14357/19922264160107} %



\vskip 14pt plus 9pt minus 6pt

\thispagestyle{headings}

\begin{multicols}{2}

\label{st\stat}


\section{Введение}

При практическом решении задачи моделирования и~анализа нестационарных 
информационных потоков в~современных информационных, телекоммуникационных 
и~вычислительных системах и~сетях ключевым этапом является разделение смесей 
вероятностных распределений, т.\,е.\ статистическое оценивание неизвестных параметров. 
При этом в~качестве базовых аппроксимирующих моделей в~рамках так называемого 
метода скользящего разделения смесей (СРС-ме\-тод)~\cite{Korolev2011} могут 
рассматриваться как конечные сдвиг-мас\-штаб\-ные смеси нормальных 
и~гам\-ма-рас\-пре\-де\-ле\-ний~\cite{Korolev2011}, так и~непрерывные распределения, 
например дис\-пер\-си\-он\-но-сдви\-го\-вые смеси нормальных законов~\cite{Korolev2015}. 
Основная идея СРС-ме\-то\-да заключается в~специальном разбиении исходной выборки 
на подвыборки (окна) и~дальнейшем анализе поведения данных на каж\-дом окне. 
Это позволяет отслеживать появление и~исчезновение формирующих структур 
в~изучаемых процессах во времени.

Можно привести широкий спектр примеров успешного использования
вероятностных моделей для реальных данных различной природы.
Например, статьи~[3--5] предлагают анализ финансовых и~физических
рядов, а работа~\cite{Hamedi2015} ориентирована на анализ
биомедицинских сигналов. В статье~\cite{Abanto2010} проводится
анализ данных фондового индекса \verb"S&P500" на основе симметричных
масштабных смесей нормальных распределений, а работа~\cite{Wang2011}
посвящена изучению курса обмена австралийского доллара 
с~использованием представлений распределений Стьюдента в~виде
масштабной смеси нормальных законов с~применением специальной
программы \verb"WinBUGS" для проведения вычислений.

Однако зачастую методы реализованы в~виде программ для специализированных 
пакетов (в частности, \verb"MATLAB"), что накладывает ряд ограничений 
на удобство распространения и~использования.  Во-пер\-вых, конечный пользователь 
должен иметь доступ к~дорогостоящему программному обеспечению и~опыт работы с~ним. 
Во-вто\-рых, далеко не все алгоритмы могут быть запущены на любой физической 
архитектуре. К~таковым можно отнести, например, решения, предназначенные для 
выполнения вычислений с~по\-мощью  \verb"CUDA" (Compute Unified Device
Architecture).  У~конечного пользователя может 
оказаться графическая карта от другого производителя либо его  \verb"GPU"  (graphics processing
unit) не предназначены для осуществления вычислений на должном уровне. В-третьих, 
могут возникнуть различные тонкости в~использовании разработанных третьими 
лицами программ, например связанные с~лицензиями. 

Все это приводит к~идее 
создания специализированного сервиса с~он\-лайн-до\-сту\-пом, который поможет 
преодолеть указанные трудности.

В настоящей статье рассмотрим требования, которым должна удовлетворять такая 
система.\linebreak В~частности, будут описаны функциональные возможности специализированного 
комплекса, разработанного автором на языке программирования пакета \verb"MATLAB", 
который выступает прообразом функциональных возможностей, предназначенных для 
реализации в~рамках сервиса с~он\-лайн-до\-сту\-пом. Кроме того, значительное внимание 
уделяется получению явных формул для ряда моментных характеристик, в~том числе 
в~матричном представлении, для повышения производительности вычислений.

\section{Матричные представления некоторых моментных характеристик конечных 
смесей нормальных законов}

Пусть случайная величина~$Z_t$ имеет функцию распределения $F_Z(x,t)$, представимую 
в~виде конечной сдвиг-мас\-штаб\-ной смеси нормальных законов, а именно:
\begin{multline}
F_Z(x,t)={}\\
{}=\sum\limits_{i=1}^{k(t)}
\fr{p_i(t)}{\sigma_i(t)\sqrt{2\pi}}\int\limits_{-\infty}^{x}\exp
\left\{-\fr{(y-a_i(t))^2}
{2\sigma_i^2(t)}\right\}\,dy, 
\\
\forall x \in \mathbb{R}\,,\quad a_i(t)\in\mathbb{R}\,, \enskip \sigma_i(t)>0\,,\enskip 
i=1,\ldots,k(t)\,,\\ \sum\limits_{i=1}^{k(t)}p_{i}(t)=1\,, \enskip p_{i}(t)\geqslant 0\,.
\label{Mixture}
\end{multline}

Параметр $t$ здесь обозначает время (номер окна) для СРС-ме\-то\-да 
и~подчеркивает тот факт, что распределение изменяется для 
каждого положения окна (возможно, весьма существенным образом). 
В~частности, параметр $k(t)$, описывающий число компонент в~смеси 
вида~\eqref{Mixture}, при аппроксимации\linebreak реальных данных может принимать 
различные значения с~течением времени (см., например, 
\mbox{статьи}~\cite{Gorshenin2008,Gorshenin2012}, при этом вопрос статистической 
значимости дополнительных компонент обсуждается в~работе~\cite{Gorshenin2011b}). 
Это обстоятельство значительно затрудняет прогнозирование, так как появление 
новых компонент в~процессе зачастую объясняется новыми факторами, которые 
отсутствовали при построении первоначальной модели и,~следовательно, не могли быть 
учтены. Поэтому необходимо перейти к~рассмотрению некоторых <<интегральных>> 
характеристик, которые можно рассчитать для любого распределения вида~\eqref{Mixture}, 
независимо от конкретных значений параметров.

В качестве таких величин можно использовать моменты различных порядков, 
а~также производные от них коэффициенты асимметрии и~эксцесса. Ниже получим 
соответствующие выражения для случайной величины с~функцией распределения~\eqref{Mixture} 
для первых четырех моментов.

Известно, что для начальных моментов случайной величины~$X$ с~нормальным распределением 
с~параметрами~$a$ и~$\sigma^2$ (т.\,е.\ $X\sim N(a,\sigma^2)$) справедливы 
следующие соотношения:
\begin{equation}
\mathbb{E} X^m=\begin{cases}
a^2+\sigma^2, & m=2\,;\\
a^3+3a\sigma^2, & m=3\,;\\
a^4+6a^2\sigma^2+3\sigma^4, & m=4\,.
\end{cases}\label{NonCentral}
\end{equation}

Для начальных моментов случайной величины~$Z_t$ с~функцией распределения, 
задаваемой формулой~\eqref{Mixture}, по определению имеем:
\begin{multline*}
\mathbb{E} Z_t^m=\int\limits_{-\infty}^{+\infty} z^m\,dF_Z(z,t)=
\sum\limits_{i=1}^{k(t)} p_i(t)\fr{1}{\sigma_i(t)\sqrt{2\pi}}\times{}\\
{}\times 
\int\limits_{-\infty}^{+\infty} z^m\exp\left\{-\fr{(z-a_i(t))^2}
{2\sigma_i^2(t)}\right\} \,dz={}\\
{}=\sum\limits_{i=1}^{k(t)} p_i(t)\mathbb{E} X_i^m\,, \quad 
X_i\sim N(a_i(t),\sigma_i^2(t))\,.
\end{multline*}

Тогда для начальных моментов случайной величины~$Z_t$ справедлив следующий 
аналог выражений~\eqref{NonCentral}:
\begin{equation}
\mathbb{E} Z_t^m=\begin{cases}
\displaystyle\sum\limits_{i=1}^{k(t)} p_i(t) a_i(t), &\hspace*{-15mm} m=1;\\
\displaystyle\sum\limits_{i=1}^{k(t)} p_i(t)\left(a_i^2(t)+\sigma_i^2(t)\right), &\hspace*{-15mm} m=2;\\
\displaystyle\sum\limits_{i=1}^{k(t)} p_i(t)\left(a_i^3(t)+3a_i\sigma_i^2(t)\right), &\hspace*{-15mm} m=3;\\
\displaystyle\sum\limits_{i=1}^{k(t)} p_i(t)\left(a_i^4(t)+6a_i^2\sigma_i^2(t)+3\sigma_i^4(t)
\right), &\\ 
& \hspace*{-15mm}m=4.
\end{cases}\!\!\! \!\!\label{NonCentralZ}
\end{equation}

Воспользуемся этими выражениями для получения явных формул для дисперсии, 
а~так\-же коэффициентов асимметрии и~эксцесса, зависящих только от параметров 
распределения, а~именно: величин $p_i(t)$, $a_i(t)$ и~$\sigma_i(t)$.

Сразу отметим, что математическое ожидание случайной величины~$Z_t$, согласно 
первой строке в~формуле~\eqref{NonCentralZ}, имеет вид:
\begin{equation}
\mathbb{E} Z_t=\sum\limits_{i=1}^{k(t)} p_i(t) a_i(t)\,. 
\label{Expectation}
\end{equation}

\subsection{Дисперсия}

Дисперсия случайной величины~$Z_t$ с~функцией распределения $F_Z(x,t)$, 
определяемой выражением~\eqref{Mixture}, имеет вид:
\begin{equation}
\mathbb{D} Z_t=\mathbb{E} Z_t^2-\left(\mathbb{E} Z_t\right)^2\,. 
\label{Var}
\end{equation}

Пользуясь выражениями~\eqref{NonCentralZ} и~проводя необходимые 
преобразования, получим:
\begin{multline}
\mathbb{D} Z_t=\int\limits_{-\infty}^{+\infty} z^2\,dF_Z(z,t)-\left(
\sum\limits_{i=1}^{k(t)} p_i(t) a_i(t)\right)^{\!2}={}
\\
{}=\sum\limits_{i=1}^{k(t)} \fr{p_i(t)}{\sigma_i(t)\sqrt{2\pi}} 
\int\limits_{-\infty}^{+\infty} z^2\exp\left\{-\fr{(z-a_i(t))^{\!2}}
{2\sigma_i^2(t)}\right\} \,dz-{}\\
{}-\left(\sum_{i=1}^{k(t)} p_i(t) a_i(t)\right)^2={}\\
{}=\sum\limits_{i=1}^{k(t)} p_i(t) a_i^2(t)-\left(\sum\limits_{i=1}^{k(t)} 
p_i(t) a_i(t)\right)^{\!\!2}\!+\sum\limits_{i=1}^{k(t)} p_i(t) \sigma_i^2(t) ={}\\
{}=\sum\limits_{i=1}^{k(t)} p_i(t) \left(a_i(t)-\sum\limits_{i=1}^{k(t)} 
p_i(t) a_i(t)\right)^2+{}\\
{}+\sum_{i=1}^{k(t)} p_i(t) \sigma_i^2(t)\,. 
\label{Variance}
\end{multline}

Заключительное представление в~формуле~\eqref{Variance} легко получить 
путем раскрытия квадрата в~первой сумме и~приведения подобных слагаемых 
(например, именно в~таком виде дисперсия приводится в~книге~\cite{Korolev2011}).

\subsection{Коэффициент асимметрии}

Общий вид коэффициента асимметрии случайной величины~$Z_t$ с~функцией 
распределения $F_Z(x,t)$~\eqref{Mixture} задается следующим выражением:
\begin{equation*}
\gamma_{Z,\,t}=\fr{\mathbb{E} \left(Z_t-\mathbb{E} 
Z_t\right)^3}{\left(\mathbb{D} Z_t\right)^{3/2}}\,.
\end{equation*}

Выпишем отдельно выражение для числителя дроби, указанной выше:
\begin{multline*}
\mathbb{E} \left(Z_t-\mathbb{E} Z_t\right)^3={}\\
{}=
\mathbb{E} Z_t^3-3\mathbb{E} Z_t\cdot\mathbb{E} Z_t^2+3\left(
\mathbb{E} Z_t\right)^3-\left(\mathbb{E} Z_t\right)^3={}\\
{}=\mathbb{E} Z_t^3-3\mathbb{E} Z_t\cdot\left(\mathbb{E} Z_t^2-\left(
\mathbb{E} Z_t\right)^2\right)-\left(\mathbb{E} Z_t\right)^3={}\\
{}=\mathbb{E} Z_t^3-3\mathbb{E} Z_t\cdot\mathbb{D} Z_t-\left(\mathbb{E} Z_t\right)^3\,.
\end{multline*}

Тогда коэффициент асимметрии может быть представлен в~виде:
\begin{equation}
\gamma_{Z,\,t}=\fr{\mathbb{E} Z_t^3-3\mathbb{E} 
Z_t\cdot\mathbb{D} Z_t-\left(\mathbb{E} Z_t\right)^3}
{\left(\mathbb{D} Z_t\right)^{3/2}}\,.
\label{Skew}
\end{equation}

Воспользуемся формулами~\eqref{NonCentral} и~\eqref{Variance}. Имеем:
\begin{multline*}
\mathbb{E} \left(Z_t-\mathbb{E} Z_t\right)^3={}\\
{}=
\sum\limits_{i=1}^{k(t)} \fr{p_i(t)}{\sigma_i(t)\sqrt{2\pi}} 
\int\limits_{-\infty}^{+\infty} z^3\exp\left\{-\fr{(z-a_i(t))^2}
{2\sigma_i^2(t)}\right\} \,dz-{}\\
{}-
3\mathbb{E} Z_t\cdot\mathbb{D} Z_t-\left(\mathbb{E} Z_t\right)^3={}\\
{}=\sum\limits_{i=1}^{k(t)} p_i(t)\left(a_i^3(t)+3a_i\sigma_i^2(t)\right)-3\mathbb{E} Z_t\cdot\mathbb{D} Z_t-\left(\mathbb{E} Z_t\right)^3=\\
{}=\sum\limits_{i=1}^{k(t)} p_i(t)\left(a_i^3(t)+3a_i\sigma_i^2(t)\right)-
\left(\sum\limits_{i=1}^{k(t)} p_i(t) a_i(t)\right)\times{}\\
{}\times\left(3\sum\limits_{i=1}^{k(t)} p_i(t) \left(a_i(t)-
\sum\limits_{i=1}^{k(t)} p_i(t) a_i(t)\right)^2+{}\right.\\
\left.{}+
3\sum\limits_{i=1}^{k(t)} p_i(t) \sigma_i^2(t)-\left(
\sum\limits_{i=1}^{k(t)} p_i(t) a_i(t)\right)^2\right)\,.
\end{multline*}

Окончательно получим следующее выражение для коэффициента асимметрии:
\begin{multline}
\gamma_{Z,\,t}={}\\
{}=\left[
\vphantom{\left(a_i(t)-
\sum\limits_{i=1}^{k(t)} p_i(t) a_i(t)\right)^2}
\sum\limits_{i=1}^{k(t)} p_i(t)\left(
a_i^3(t)+3a_i\sigma_i^2(t)\!\right)-\left(\sum\limits_{i=1}^{k(t)} 
p_i(t) a_i(t)\right)\times{}\right.\hspace*{-3.48pt}\\
{}\times\left(3\sum\limits_{i=1}^{k(t)} p_i(t) \left(a_i(t)-
\sum\limits_{i=1}^{k(t)} p_i(t) a_i(t)\right)^2+{}\right.\\
\left.\left.{}+
3\sum\limits_{i=1}^{k(t)} p_i(t) \sigma_i^2(t)-\left(
\sum\limits_{i=1}^{k(t)} p_i(t) a_i(t)\right)^2\right)\right]\times{}\\
{}\times\left[\sum\limits_{i=1}^{k(t)} p_i(t) \left(a_i(t)-
\sum\limits_{i=1}^{k(t)} p_i(t) a_i(t)\right)^2+{}\right.\\
\left.{}+
\sum\limits_{i=1}^{k(t)} p_i(t) \sigma_i^2(t)
\vphantom{\sum\limits_{i=1}^{k(t)} p_i(t) \left(a_i(t)-
\sum\limits_{i=1}^{k(t)} p_i(t) a_i(t)\right)^2}
\right]^{-3/2}\,.
\label{Skewness}
\end{multline}

\subsection{Коэффициент эксцесса}

Общий вид коэффициента эксцесса случайной величины~$Z_t$ с~функцией 
распределения $F_Z(x,t)$, определяемой выражением~\eqref{Mixture}, 
задается следу\-ющим выражением:
\begin{equation*}
\kappa_{Z,\,t}=\fr{\mathbb{E} \left(Z_t-\mathbb{E} Z_t\right)^4}
{\left(\mathbb{D} Z_t\right)^2}-3\,.
\end{equation*}

Выпишем выражения для числителя в~указанной выше дроби через известные величины:
\begin{multline*}
\mathbb{E} \left(Z_t-\mathbb{E} Z_t\right)^4=
\mathbb{E} \left(Z_t^4-4Z_t^3\cdot\mathbb{E} Z_t+{}\right.\\
\left.{}+6Z_t^2\cdot
\left(\mathbb{E} Z_t\right)^2-4Z_t\cdot \left(\mathbb{E} Z_t\right)^3+
\left(\mathbb{E} Z_t\right)^4\right)={}\\
{}=\mathbb{E} Z_t^4-3\left(\mathbb{E} Z_t\right)^4-
4\mathbb{E} Z_t\cdot \mathbb{E} Z_t^3+
6\left(\mathbb{E} Z_t\right)^2\cdot \mathbb{E} Z_t^2\,.
\end{multline*}

Тогда коэффициент эксцесса может быть представлен в~виде:
\begin{multline}
\kappa_{Z,\,t}=\left(\mathbb{E} Z_t^4-4\mathbb{E} Z_t\cdot 
\mathbb{E} Z_t^3+6\left(\mathbb{E} Z_t\right)^2\cdot 
\mathbb{E} Z_t^2-{}\right.\\
\left.{}-3\left(\mathbb{E} Z_t\right)^4\right)\Big/
\left(\mathbb{D} Z_t\right)^2-3\,.
\label{Kurt}
\end{multline}

Воспользуемся формулами~\eqref{NonCentral} и~\eqref{Expectation}. Имеем:
\begin{multline*}
\mathbb{E} \left(Z_t-\mathbb{E} Z_t\right)^4={}\\
{}=
\sum\limits_{i=1}^{k(t)} \fr{p_i(t)}{\sigma_i(t)\sqrt{2\pi}} %\times{}\\
%{}\times
\int\limits_{-\infty}^{+\infty} z^4\exp\left\{-\fr{(z-a_i(t))^2}
{2\sigma_i^2(t)}\right\} \,dz-{}\\
{}-
3\left(\sum\limits_{i=1}^{k(t)} p_i(t) a_i(t)\right)^4-
4\left(\sum\limits_{i=1}^{k(t)} p_i(t) a_i(t)\right) \times{}\\
{}\times
\left(\sum\limits_{i=1}^{k(t)} p_i(t)\left(a_i^3(t)+
3a_i\sigma_i^2(t)\right)\right)+{}\\
{}+6\left(\sum\limits_{i=1}^{k(t)} p_i(t) a_i(t)\right)^2\left(
\sum\limits_{i=1}^{k(t)} p_i(t) \left(a_i^2(t) + \sigma_i^2(t)\right)\right)={}\\
{}=\sum\limits_{i=1}^{k(t)} p_i(t)\left(a_i^4(t)+
6a_i^2\sigma_i^2(t)+3\sigma_i^4(t)\right)-{}\\
{}-4\left(\sum\limits_{i=1}^{k(t)} p_i(t) a_i(t)\right) \left(
\sum\limits_{i=1}^{k(t)} p_i(t)\left(a_i^3(t)+3a_i\sigma_i^2(t)\right)\right)+{}\\
{}+6\left(\sum\limits_{i=1}^{k(t)} p_i(t) a_i(t)\right)^2
\left(\sum\limits_{i=1}^{k(t)} p_i(t) \left(a_i^2(t) + \sigma_i^2(t)\right)\right)-{}\\
{}-
3\left(\sum\limits_{i=1}^{k(t)} p_i(t) a_i(t)\right)^4\,.
\end{multline*}

Окончательно получим следующее выражение для коэффициента эксцесса:
\begin{multline}
\kappa_{Z,\,t}=\left[
\vphantom{\left(a_i(t)-
\sum\limits_{i=1}^{k(t)} p_i(t) a_i(t)\right)^2}
\sum\limits_{i=1}^{k(t)} p_i(t)\left(
a_i^4(t)+6a_i^2\sigma_i^2(t)+3\sigma_i^4(t)\right)-{}\right.\\
{}-3\left(
\sum\limits_{i=1}^{k(t)} p_i(t) a_i(t)\right)^{\!4}-
4\left(\sum\limits_{i=1}^{k(t)} p_i(t) a_i(t)\right) \times{}\\
{}\times
\left(\sum\limits_{i=1}^{k(t)} p_i(t)\left(a_i^3(t)+
3a_i\sigma_i^2(t)\right)\right)+{}\\
\left.{}+6\left(\sum\limits_{i=1}^{k(t)} p_i(t) a_i(t)\right)^{\!\!2}\left(
\sum\limits_{i=1}^{k(t)} p_i(t) \left(a_i^2(t) + 
\sigma_i^2(t)\right)\right)\!\right]\times{}\\
{}\times\left[\sum\limits_{i=1}^{k(t)} p_i(t) \left(a_i(t)-
\sum\limits_{i=1}^{k(t)} p_i(t) a_i(t)\right)^{\!2}+{}\right.\\
\left.{}+
\sum\limits_{i=1}^{k(t)} p_i(t) \sigma_i^2(t)
\vphantom{\left(a_i(t)-
\sum\limits_{i=1}^{k(t)} p_i(t) a_i(t)\right)^2}
\right]^{-2}-3\,.
\label{Kurtosis}
\end{multline}

Стоит отметить, что формулы \eqref{Expectation}, 
\eqref{Variance}, \eqref{Skewness} и~\eqref{Kurtosis} могут быть использованы 
для непосредственного определения моментных характеристик по величинам $p_i(t)$, 
$a_i(t)$ и~$\sigma_i(t)$ и~не требуют знания значения момента предыдущего порядка. 
Для упрощения вычислений, в~том числе и~с~точки зрения программирования, при 
<<последовательном>> поиске этих характеристик проще воспользоваться 
формулами~\eqref{Var}, \eqref{Skew} и~\eqref{Kurt}. Более того, многие 
системы оптимизированы для выполнения матричных вычислений. Поэтому разумно 
воспользоваться представлением выражений~\eqref{NonCentralZ} в~виде
\begin{equation*}
\mathbb{E} Z_t^m=\begin{cases}
P_t\cdot A_t^T, & \hspace*{-15mm}m=1;\\
P_t\left(D_{A,\,t}\cdot A_t^T + D_{\Sigma,\,t} \cdot \Sigma_t^T\right), & \hspace*{-15mm}m=2;\\
P_t\cdot D_{A,\,t}\left(D_{A,\,t}\cdot A_t^T+3\cdot D_{\Sigma,\,t}\cdot\Sigma_t^T\right),\\
 & \hspace*{-15mm}m=3;\\
P_t\left( D_{A,\,t}^3\cdot A_t^T +6\cdot D_{\Sigma,\,t}^2\cdot D_{A,\,t}\cdot A_t^T+{}\right.&\\
\hspace*{15mm}\left.{}+3\cdot D_{\Sigma,\,t}^3\cdot\Sigma_t^T\right), & \hspace*{-15mm}m=4,
\end{cases}
%\label{NonCentralZMatrix}
\end{equation*}
где
\begin{gather*}
P_t=\left(p_1,\ldots,p_{k(t)}\right)\,;\enskip
A_t=\left(a_1,\ldots,a_{k(t)}\right)\,;\\
\Sigma_t=\left(\sigma_1,\ldots,\sigma_{k(t)}\right)\,; \\
D_{A,\,t}=\mathrm{diag}\left\{a_1,\ldots,a_{k(t)}\right\}\,;\\
D_{\Sigma,\,t}=\mathrm{diag}\left\{\sigma_1,\ldots,\sigma_{k(t)}\right\},
\end{gather*}
а обозначение $\mathrm{diag}\{\ldots\}$ использовано для записи диагональных матриц 
с~указанными элементами. Получаемое преимущество в~скорости в~рамках подобного 
подхода количественно оценивается в~работе~\cite{Gorshenin2015a} на 
примере классического EM-ал\-го\-ритма  (expectation-maximization).

\vspace*{-6pt}

\section{Реализация метода скользящего разделения смесей
на~встроенном языке программирования пакета MATLAB}

Возможности СРС-метода были реализованы в~рамках единого программного 
комплекса на встроенном языке программирования пакета \verb"MATLAB". 
Для удобства работы был создан оконный пользовательский интерфейс, 
с~по\-мощью которого задаются параметры вычислительных методов, определяется 
диапазон графического вывода для сохранения и~т.\,д. Данный комплекс 
объединил в~себе различные варианты EM-ал\-го\-рит\-мов, включая и~реализацию 
матричных вычислений на \verb"GPU", для различных типов смесей вероятностных 
распределений.



В начале работы пользователю предлагается ввес\-ти имя (путь) 
для файла с~данными. Может быть выбран любой диапазон внутри ряда, 
при этом по умолчанию предлагаются все значения от первого до последнего 
(с~автоматическим определением длины ряда при указании параметра <<{\sf End}>>). 
В~случае ввода пользователем некорректных значений (например, отрицательных или 
превышающих объем выборки) программа возвращает настройки к~предустановленным. 
Вид начального окна пред\-став\-лен на рис.~1.



Затем пользователь может выбрать отображение исходных данных или разностей, 
вывести гистограммы для этих рядов с~автоматической аппрок-\linebreak\vspace*{-12pt}

 \noindent
 \begin{center}  %fig1
 \vspace*{9pt}
\mbox{%
 \epsfxsize=78mm
 \epsfbox{go1-1.eps}
 }



\vspace*{3pt}

\noindent
{{\figurename~1}\ \ \small{Начальное окно программы}}
\end{center} 
% \vspace*{19pt}

 \noindent
 \begin{center}  %fig2
 \vspace*{1pt}
\mbox{%
 \epsfxsize=78mm
 \epsfbox{go1-2.eps}
 }



\vspace*{3pt}

\noindent
{{\figurename~2}\ \ \small{Настройки параметров СРС-метода}}
\end{center} 

\vspace*{6pt}

 
 \addtocounter{figure}{2}


\noindent
симацией конечной 
смесью вероятностных распределений. После этого предлагается возможность 
запуска СРС-ме\-то\-да как для исходной выборки, 
так и~для разностей. Настройка параметров СРС-ме\-то\-да осуществляется 
с~по\-мощью нового диалогового окна, в~котором задается название вычислительного 
алгоритма, размер подвыборки, максимальное число компонент в~аппроксимирующей 
смеси, точность приближений и~диапазон для сдвига (можно выбрать любую часть 
внут\-ри выборки; если заданные пользователем значения некорректны, расчет 
производится от первого элемента выборки до $N\hm-\mathrm{width}\hm+1$, где $N$~--- длина 
всей выборки для анализа, а~width~--- ширина окна). Для каждого из полей 
предусмотрены значения по умолчанию. На рис.~2 приводится 
англоязычный вариант интерфейса с~аббревиатурой MSM 
(от {moving separation of mixtures}).



После нажатия на кнопку <<\verb"ОК">> и~запуска шагов алгоритма 
отображается окно с~индикатором выполнения и~названием выбранного 
вычислительного метода, позволяющего следить за текущим положением окна 
и~состоянием работы программы. После завершения вычислений появляется график 
с~динамической и~диффузионной компонентами волатильности (подробнее см.\ 
в~\cite{Korolev2011}), а~так\-же диалоговое окно, с~по\-мощью 
которого можно настроить диапазон вывода параметров для сохранения, включая 
и~диапазон для окон. После закрытия данного диалогового окна график автоматически 
сохраняется в~формате PNG 
на диск в~папку с~программой. Пример графического 
вывода пред\-став\-лен на рис.~\ref{FigLimits}.

\begin{figure*} %fig3
\vspace*{1pt}
 \begin{center}
 \mbox{%
 \epsfxsize=160mm
 \epsfbox{go1-3.eps}
 }
 \end{center}
 \vspace*{-9pt}
\Caption{Пример графического вывода c окном изменения диапазонов по каждой из осей
\label{FigLimits}}
\vspace*{6pt}
\end{figure*}
\begin{figure*}[b] %fig4
\vspace*{6pt}
 \begin{center}
 \mbox{%
 \epsfxsize=143.08mm
 \epsfbox{go1-4.eps}
 }
 \end{center}
 \vspace*{-9pt}
\Caption{Пример изображения эволюции квантилей различных уровней (от~0,05 до~0,95)
\label{FigQuantiles}}
\end{figure*}

Данный функционал реализуется с~по\-мощью модуля <<Ядро СРС-ме\-то\-да>>
(свидетельство государственной регистрации программ для ЭВМ
№\,2015618673). Он используется с~целью унификации для
пользователя работы с~любым алгоритмом, при этом с~точки зрения
разработчика возможно простое включение в~состав любых новых методов
без изменения пользовательского опыта. Найденные оценки сохраняются
в~процессе расчетов, а~так\-же по окончании работы программы, что
позволяет избежать потери результатов при возникновении прерывания.
Непосредственно в~ядро встроены различные модификации EM-ал\-го\-рит\-ма
(классический, сглаженный, основанный на повторных вычислениях на
окне и~т.\,п.).

Кроме того, в~состав комплекса включен модуль визуализации моментных
характеристик и~квантилей (свидетельство государственной регистрации
программ для ЭВМ №\,2015618564), формулы для которого были
рассмотрены в~предыдущем разделе. Модуль содержит функции для
отыскания математического ожидания, дисперсии, коэффициента
асимметрии и~эксцесса, а также квантилей различного уровня для
конечных смесей нормальных законов. Указанные величины вычисляются
для каждого положения окна в~СРС-ме\-то\-де, 
а~затем выводятся на графиках: четыре моментные характеристики 
в~разных осях на одном листе; эволюция квантилей различного уровня
изображается на отдельном графике в~трехмерном пространстве 
(рис.~\ref{FigQuantiles}). Для заданного набора окон (параметр одной
из функций) предусмотрен вывод приближения гистограммы
аппроксимирующими кривыми с~заданием расположения набора графиков на
лис\-те. С~привлечением возможностей данного модуля были созданы
графики, например в~работе~\cite{Gorshenin2015b}.

\vspace*{-6pt}

\section{Вычислительный комплекс с~онлайн-доступом}

В предыдущем разделе был коротко описан программный комплекс, 
реализованный средствами программирования пакета \verb"MATLAB", функциональные 
возможности которого должны стать базой для он\-лайн-сер\-виса.

Реализация широкого спектра методов интеллектуального анализа данных, основанного 
на идеологии смесей вероятностных распределений (не только конечных нормальных, 
как было рас\-смот\-ре\-но выше) будет представлена в~он\-лайн-сис\-те\-ме на основе 
инструментов, предлагаемых в~пользовательских профилях (с~реализацией  
механизмов регистрации и~авторизации). Пользователи смогут загружать 
на сервер данные, а полученные после обработки модели будут сохраняться 
в~их профиле, что позволит избежать повторной загрузки одних и~тех же рядов 
(возможно, значительного объема) и~сделает возможным многократное дальнейшее 
использование результатов обработки.

Из-за высокой вычислительной сложности в~ряде ситуаций может 
потребоваться значительное время для корректной аппроксимации данных 
выбранной моделью. В~этой ситуации предполагается оповещение исследователей 
о~статусе процесса с~по\-мощью электронных каналов связи. В~случае завершения 
расчетов пользователь сможет в~любое удобное время обратиться 
к~результатам анализа как в~числовом (с~по\-мощью экспорта оцененных параметров), 
так и~в~графическом виде. При этом будет доступна работа с~несколькими рядами 
одновременно, в~том числе в~разных режимах обработки.






Основным элементом интерфейса он\-лайн-сис\-те\-мы является область для
отображения исходных временн$\acute{\mbox{ы}}$х рядов и~визуального представления
результатов СРС-ме\-то\-да, включая вывод
нескольких графиков одновременно (динамической и~диффузионной
компоненты волатильности, моментных характеристик и~др.).

 \noindent
 \begin{center}  %fig5
 \vspace*{-1pt}
\mbox{%
 \epsfxsize=78mm
 \epsfbox{go1-5.eps}
 }



\vspace*{3pt}

\noindent
{{\figurename~5}\ \ \small{Схема архитектуры комплекса}}
\end{center} 

 \vspace*{6pt}

Описание архитектуры, предлагаемой для решения перечисленных задач, 
включая организацию потоков данных в~системе, приведено в~работе~\cite{Gorshenin2015c}. 
Рассмотрим ключевые элементы, представленные на рис.~5.


Первый уровень представляет персональный компьютер (ПК) пользователя, 
обеспечивающий ему доступ к~интерфейсу он\-лайн-ком\-плек\-са для загрузки 
данных, настройки параметров и~получения результатов (в~том числе в~визуальной 
форме). Frontend-сер\-вер предназначен для реализации интерфейса взаимодействия 
между пользователем\linebreak и~вычислительными узлами системы, при этом 
непосредст\-венная обработка данных на нем не производится. Backend-сер\-вер 
осуществляет взаимодействие между Frontend-сер\-ве\-ром, рабочими серверами 
и~серверами баз данных (БД), в~част\-ности готовит данные для обработки 
и~распределяет задачи по вычислительным компонентам системы.

Наибольшая нагрузка по передаче данных ложится на взаимодействие 
Frontend- и~Backend-сер\-ве\-ров, поэтому необходима разработка механизмов 
ускорения их работы (в частности, за счет кэширования). Взаимодействие Backend- 
и~рабочих серверов должно осуществляться с~по\-мощью специализированных API 
(application programming interface) для 
реализации возможностей, связанных с~параллельной обработкой данных. После проведения 
вычислений для одного положения окна СРС-ме\-то\-да данные сохраняются на сервере БД 
и~пересылаются с~помощью Frontend-сер\-ве\-ра пользователю для возможного контроля 
процесса выполнения анализа с~его стороны.

Подобная архитектура позволит интегрировать в~систему различные методы 
интеллектуального анализа данных, при этом их реализация может быть 
основана на специальных программных  и~аппаратных подходах, но для 
конечного пользователя особенности решений будут скрыты.

\section{Заключение}

Анализ информационных потоков с~использованием разнообразных 
вероятностных моделей широко распространен в~различных прикладных областях, 
что подчеркивает актуальность создания описыва\-емой в~статье он\-лайн-сис\-те\-мы. 
При этом следует отметить отсутствие непосредственных аналогов сервиса из-за 
универсальности предлагаемого набора методов и~выбранной концепции, позволяющих 
конечному пользователю не заботиться о соответствии ресурсов своего компьютера 
ка\-ким-ли\-бо специальным техническим требованиям, а сразу загружать данные на 
сервер, обрабатывать их с~по\-мощью различных методов интеллектуального анализа 
и~получать результат. Данная система сможет предложить широкие возможности 
для различных групп исследователей во всем мире.

\bigskip

{Автор выражает признательность д.\,ф.-м.\,н., профессору 
Виктору Юрьевичу Королеву за полезные обсуждения, а также Виктору Кузьмину 
за плодотворное участие в~разработке архитектуры комплекса.}

{\small\frenchspacing
 {%\baselineskip=10.8pt
 \addcontentsline{toc}{section}{References}
 \begin{thebibliography}{99}
\bibitem{Korolev2011} 
\Au{Королев~В.\,Ю.}
Ве\-ро\-ят\-но\-ст\-но-ста\-ти\-сти\-че\-ские методы декомпозиции волатильности
хаотических процессов.~--- М.: Изд-во МГУ, 2011. 512~с.

\bibitem{Korolev2015} 
\Au{Королев~В.\,Ю., Корчагин~А.\,Ю., Горшенин~А.\,К.} 
Некоторые свойства дис\-пер\-си\-он\-но-сдви\-го\-вых смесей нормальных законов~// 
Статистические методы оценивания и~проверки гипотез, 2015. Вып.~26. С.~134--153.

\bibitem{Gorshenin2008} 
\Au{Горшенин~А.\,К., Королев~В.\,Ю.,
Турсунбаев~А.\,М.} Медианные  модификации EM- и~SEM-ал\-го\-рит\-мов
для разделения смесей вероятностных распределений и~их
применение к~декомпозиции волатильности финансовых временных
рядов~// Информатика и~её применения, 2008. Т.~2. Вып.~4. C.~12--47.

\bibitem{Gorshenin2011a}
\Au{Горшенин~А.\,К., Королев~В.\,Ю.,
Малахов~Д.\,В., Скворцова~Н.\,Н.} Анализ тонкой стохастической
структуры хаотических процессов с~по\-мощью ядерных оценок~//
Математическое моделирование, 2011. Т.~23. №\,4. C.~83--89.

\bibitem{Gorshenin2012}
\Au{Горшенин~А.\,К., Королев~В.\,Ю., Малахов~Д.\,В., Скворцова~Н.\,Н.}
Об исследовании плазменной турбулентности на основе анализа спектров~// Компьютерные
исследования и~моделирование, 2012. Т.~4. №\,4. С.~793--802.

\bibitem{Hamedi2015} 
\Au{Hamedi~M., Salleh~S.-H., Chee-Ming~Ting, Samdin~S.\,B., Mohd~Noor~A.} 
Sensor space time-varying information flow analysis of multiclass motor imagery 
through Kalman smoother and EM algorithm~// 
Conference (International) on BioSignal Analysis, Processing and Systems (ICBAPS)
Proceedings, 2015. 
P.~118--122.

\bibitem{Abanto2010} 
\Au{Abanto-Vallea~C.\,A., Bandyopadhyayb~D., Lachosc~V.\,H., Enriquezd~I.} 
Robust Bayesian analysis of heavy-tailed stochastic volatility models 
using scale mixtures of normal distributions~// Comput. Stat. 
Data An., 2010. Vol.~54. No.\,12. P.~2883--2898.

\bibitem{Wang2011} 
\Au{Wang~J.\,J.\,J., Chan~J.\,S.\,K., Choy~S.\,T.\,B.} 
Stochastic volatility models with leverage and heavytailed distributions: 
A~Bayesian approach using scale mixtures~// Comput. Stat. 
Data An., 2011. Vol.~55. No.\,1. P.~852--862.

\bibitem{Gorshenin2011b}
\Au{Горшенин~А.\,К.}
Проверка статистических гипотез в~модели расщепления компоненты~//
Вестник Московского ун-та. Сер.~15: Вычислительная математика и~кибернетика, 2011. 
№\,4. С.~26--32.

\bibitem{Gorshenin2015a} 
\Au{Gorshenin A.\,K.}
On implementation of EM-type algorithms in the stochastic models 
for a~matrix computing on GPU~// AIP Conference Proceedings, 2015. Vol.~1648. P.~250008.


\bibitem{Gorshenin2015b} 
\Au{Королев~В.\,Ю., Горшенин~А.\,К., Гулев~С.\,К., Беляев~К.\,П.} Статистическое
моделирование турбулентных потоков тепла между океаном 
и~атмосферой с~помощью метода скользящего
разделения конечных нормальных смесей~// Информатика и~её применения, 2015. Т.~9. Вып.~4. C.~3--13.

\bibitem{Gorshenin2015c} 
\Au{Gorshenin~A., Kuzmin~V.} Online system for the construction of
structural models of information flows~// 
7th  Congress (International) on Ultra Modern Telecommunications
and Control Systems and Workshops (ICUMT) Proceedings.~--- Piscataway, NJ, USA:
IEEE, 2015. P.~216--219.

\end{thebibliography}

 }
 }

\end{multicols}

\vspace*{-3pt}

\hfill{\small\textit{Поступила в~редакцию 06.02.16}}

%\vspace*{8pt}

\newpage

%\vspace*{-24pt}

%\hrule

%\vspace*{2pt}

%\hrule

\vspace*{-24pt}



\def\tit{CONCEPT OF ONLINE SERVICE FOR~STOCHASTIC MODELING OF~REAL PROCESSES}

\def\titkol{Concept of online service for stochastic modeling of real processes}

\def\aut{A.\,K.~Gorshenin$^{1,2}$}

\def\autkol{A.\,K.~Gorshenin}

\titel{\tit}{\aut}{\autkol}{\titkol}

\vspace*{-9pt}


\noindent
$^1$Institute of Informatics Problems, Federal Research Center 
``Computer Science and Control'' of the Russian\linebreak
 $\hphantom{^1}$Academy of Sciences,
44-2 Vavilov Str., Moscow 119333, Russian Federation

\noindent %Federal State Budget Educational Institution  of  Higher Education 
$^2$Moscow Technological University (MIREA),
78~Vernadskogo Ave., %\linebreak
Moscow 119454, Russian Federation

\def\leftfootline{\small{\textbf{\thepage}
\hfill INFORMATIKA I EE PRIMENENIYA~--- INFORMATICS AND
APPLICATIONS\ \ \ 2016\ \ \ volume~10\ \ \ issue\ 1}
}%
 \def\rightfootline{\small{INFORMATIKA I EE PRIMENENIYA~---
INFORMATICS AND APPLICATIONS\ \ \ 2016\ \ \ volume~10\ \ \ issue\ 1
\hfill \textbf{\thepage}}}

\vspace*{3pt}



\Abste{Information flows analysis based on various probabilistic 
models is widely used in various applied fields. The article describes 
the basic principles of construction of the new online system for stochastic 
modeling of real processes, which has no analogue due to the universality 
of the set of methods and the concept of an Internet resource; 
so, the end user should not check personal computer's specifications and could upload data 
to the server immediately and then process the samples.}

\KWE{probability mixtures; moving separation of mixtures; data mining; online
software; matrix computing}



\DOI{10.14357/19922264160107}

\Ack
\noindent
The research was supported by the Russian Foundation for Basic Research (project
15-37-20851).


\vspace*{12pt}

  \begin{multicols}{2}

\renewcommand{\bibname}{\protect\rmfamily References}
%\renewcommand{\bibname}{\large\protect\rm References}

{\small\frenchspacing
 {%\baselineskip=10.8pt
 \addcontentsline{toc}{section}{References}
 \begin{thebibliography}{99}
\bibitem{1-gg}
\Aue{Korolev, V.\,Yu.} 2011. 
\textit{Veroyatnostno-statisticheskie metody dekompozitsii 
volatil'nosti khaoticheskikh protsessov} [Probabilistic and statistical methods of 
decomposition of volatility of chaotic processes]. Moscow: Moscow University
Publishing House. 512~p.

\bibitem{2-gg}
\Aue{Korolev, V.\,Yu., A.\,Yu. Korchagin, and A.\,K.~Gorshenin}. 
2015. Nekotorye svoystva dispersionno-sdvigovykh\linebreak smesey normal'nykh zakonov 
[Some properties of variance-mean normal mixtures]. 
\textit{Statisticheskie Metody Otsenivaniya i~Proverki Gipotez}
[Statistical Methods of Estimation and Hypothesis Testing] 26:134--153.

\bibitem{3-gg}
\Aue{Gorshenin, A.\,K., V.\,Yu. Korolev, and A.\,M.~Tursunbaev}. 
2008. Mediannye modifikatsii EM- i~SEM-algoritmov dlya razdeleniya smesey 
veroyatnostnykh raspredeleniy i~ikh primenenie k~dekompozitsii volatil'nosti 
finansovykh vremennykh ryadov [Median modification of EM- and SEM-algorithms for 
separation of mixtures of probability distributions and their application to 
the decomposition of volatility of financial time series]. 
\textit{Informatika i~ee Primeneniya}~--- \textit{Inform. Appl.} 2(4):12--47.

\bibitem{4-gg}
\Aue{Gorshenin, A.\,K., V.\,Yu. Korolev, D.\,V.~Malakhov, and N.\,N.~Skvortsova}. 
2011. Analiz tonkoy stokha\-sti\-che\-skoy struktury khaoticheskikh protsessov 
s~pomoshch'yu yadernykh otsenok [Analysis of fine stochastic structure 
of chaotic processes by kernel estimators]. \textit{Matematicheskoe 
Modelirovanie} [Mathematical Modeling] 23(4):83--89.

\bibitem{5-gg}
\Aue{Gorshenin, A.\,K., V.\,Yu. Korolev, D.\,V.~Malakhov, and 
N.\,N.~Skvortsova}. 2012. Ob issledovanii plazmennoy turbulentnosti na osnove 
analiza spektrov [On the investiga-\linebreak\vspace*{-12pt}

\columnbreak

\noindent
tion of plasma turbulence by the analysis of the spectra]. 
\textit{Komp'yuternye Issledovaniya i~Modelirovanie} [Computer Research and 
Modeling] 4(4):793--802.

\bibitem{6-gg}
\Aue{Hamedi, M., S.-H.~Salleh, Ting Chee-Ming, S.\,B.~Sam\-din, and A.~Mohd Noor}. 
2015. Sensor space time-varying information flow analysis of multiclass motor 
imagery through Kalman smoother and EM algorithm. \textit{Conference (International) on 
BioSignal Analysis, Processing and Systems (ICBAPS) Proceedings}. 118--122.

\bibitem{7-gg}
\Aue{Abanto-Vallea, C.\,A., D.~Bandyopadhyayb, V.\,H.~Lachosc, and I.~Enriquezd}. 
2010. Robust Bayesian analysis of heavy-tailed stochastic volatility models using 
scale mixtures of normal distributions. 
\textit{Comput. Stat. Data An.} 54(12):2883--2898.

\bibitem{8-gg}
\Aue{Wang, J.\,J.\,J., J.\,S.\,K.~Chan, and S.\,T.\,B.~Choy}. 2011. 
Stochastic volatility models with leverage and heavytailed distributions: 
A~Bayesian approach using scale mixtures.
\textit{Comput. Stat.  Data An.} 55(1):852--862.

\bibitem{9-gg}
\Aue{Gorshenin, A.\,K.} 2011. Proverka statisticheskikh gipotez 
v~modeli rasshchepleniya komponenty [Testing of statistical hypotheses in the 
splitting component model]. 
\textit{Vestnik Moskovskogo Un-ta. Vychislitel'naya Matematika i~Kibernetika}
[Moscow University Computational Mathematics and Cybernetics] 4:26--32.

\bibitem{10-gg}
\Aue{Gorshenin, A.\,K.} 2015. 
On implementation of EM-type algorithms in the stochastic models for 
a~matrix computing on GPU. \textit{AIP Conference Proceedings}. 1648:250008


\bibitem{11-gg}
\Aue{Korolev, V.\,Yu., A.\,K.~Gorshenin, S.\,K.~Gulev, and K.\,P.~Belyaev}. 
2015. Statisticheskoe modelirovanie turbulentnykh potokov tepla mezhdu okeanom 
i~at\-mo\-sfe\-roy\linebreak\vspace*{-12pt}

\pagebreak

\noindent
 s~pomoshch'yu metoda skol'zyashchego razdeleniya konech\-nykh 
normal'nykh smesey [Statistical modeling of air--sea turbulent heat fluxes by 
the method of moving separation of finite normal mixtures]. \textit{Informatika 
i~ee Primeneniya}~--- \textit{Inform. Appl.} 9(4):3--13.

\bibitem{12-gg}
\Aue{Gorshenin, A., and V.~Kuzmin}. 
2015. Online system for the construction of structural models of information flows. 
\textit{7th  Congress (International) on Ultra Modern Telecommunications and 
Control Systems and Workshops (ICUMT) Proceedings}. Piscataway, NJ: IEEE, 2015. 
216--219.
\end{thebibliography}

 }
 }

\end{multicols}

\vspace*{-3pt}

\hfill{\small\textit{Received February 6, 2016}}

\Contrl


\noindent
\textbf{Gorshenin Andrey K.}  (b.\ 1986)~---
Candidate of Science (PhD) in physics and mathematics, senior scientist, 
Institute of Informatics Problems, Federal Research Center ``Computer Science and Control'' 
of the Russian Academy of Sciences, 44-2 Vavilov Str., Moscow 119333, 
Russian Federation; associate professor, %Federal State Budget Educational Institution   of  Higher Education 
Moscow Technological University (MIREA), 
78~Vernadskogo Ave., Moscow 119454, Russian Federation;
 agorshenin@frcсsc.ru
 




\label{end\stat}


\renewcommand{\bibname}{\protect\rm Литература}