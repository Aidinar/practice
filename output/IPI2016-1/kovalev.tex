\def\stat{kovalev}

\def\tit{ПРИМЕНЕНИЕ МЕТАПРОГРАММИРОВАНИЯ\\ ДЛЯ~ПОВЫШЕНИЯ 
ТЕХНОЛОГИЧНОСТИ\\ БОЛЬШИХ АВТОМАТИЗИРОВАННЫХ СИСТЕМ}

\def\titkol{Применение метапрограммирования для повышения 
технологичности больших автоматизированных систем}

\def\aut{С.\,П.~Ковалёв$^1$}

\def\autkol{С.\,П.~Ковалёв}

\titel{\tit}{\aut}{\autkol}{\titkol}

%{\renewcommand{\thefootnote}{\fnsymbol{footnote}} \footnotetext[1]
%{Работа выполнена при поддержке РФФИ (проект 15-07-02244).}}


\renewcommand{\thefootnote}{\arabic{footnote}}
\footnotetext[1]{Институт проблем управления им.\
 В.\,А.~Трапезникова Российской академии наук, kovalyov@nm.ru}

\Abst{Предложен подход к~снижению затрат на проектирование больших 
автоматизированных систем за счет привлечения современных технологий 
метапрограммирования. В~качестве наиболее перспективных среди таких технологий 
рассматриваются разработка, управляемая моделями (model driven engineering, MDE),  
и~ас\-пект\-но-ори\-ен\-ти\-ро\-ван\-ный подход (aspect-oriented software development). 
Пред\-став\-ле\-ны методы масштабирования этих технологий, позволяющие эффективно 
применять их в~условиях роста размера создаваемых автоматизированных систем путем 
замыкания относительно системообразующих структурных отношений. В~качестве примера 
практического применения подхода приводится проектирование математического 
обеспечения интеллектуальных электроэнергетических сетей. Излагаются принципы 
математического аппарата для построения, анализа и~оптимизации процедур проектирования 
на базе теории категорий. Описан процесс проектирования генератора расчетных 
программных компонентов большой автоматизированной системы с~применением  
тео\-ре\-ти\-ко-ка\-те\-гор\-ных методов.}

\KW{большие автоматизированные системы; метапрограммирование; мегамодель; теория 
категорий; копредел; разработка, управляемая моделями;  
ас\-пект\-но-ори\-ен\-ти\-ро\-ван\-ный подход; интеллектуальная электроэнергетическая 
сеть}

\DOI{10.14357/19922264160105} %



\vskip 14pt plus 9pt minus 6pt

\thispagestyle{headings}

\begin{multicols}{2}

\label{st\stat}

\section{Введение}

   К числу основных тенденций развития совре-\linebreak менных автоматизированных 
систем относится рост масштаба до уровня больших и~даже сверхбольших 
(ultra-large scale systems, ULS)~[1]. 

Примером служит Smart Grid 
(интеллектуальная сеть)~--- сис\-те\-ма сквозной автоматизации \mbox{крупной} 
электроэнергетической сети~[2]. Такие сис\-те\-мы отличаются %\linebreak 
беспрецедентно 
большими значени-\linebreak ями масштабных факторов, таких как количество данных, 
элементов, взаимосвязей, процессов, норма\-тивов, пользователей и~др. Как 
следствие, сверхбольшим системам присущи следующие качественные 
особенности, принципиально отличающие их от традиционных объектов 
системной инженерии~[1]:
   \begin{itemize}
\item децентрализация;
\item заведомо конфликтующие, непостижимые, разноречивые требования;
\item непрерывные процессы эволюции и~внедрения;
\item разнородные, несовместимые, изменчивые элементы;
\item эрозия границ между человеком и~системой;
\item штатный характер сбоев;
\item новые парадигмы приобретения и~нормирования (policy).
\end{itemize}

   Эти особенности не рассматривались в~классическом программировании, 
поэтому применение традиционных технологий проектирования программного 
обеспечения при создании больших сис\-тем сопряжено со значительными 
затратами. Рост масштаба приводит к~взрывному увеличению трудоемкости 
рутинных процедур, которые в~ходе проектирования традиционных систем 
выполнялись <<вручную>> почти незаметно, вследствие чего сравнительно 
редко изучались в~системной инженерии. К~таким процедурам относятся 
согласование порядка внесения изменений в~компоненты, подстройка 
интерфейсов программного доступа к~модулям под потребности их окружения, 
реализация рассеянных задач (crosscutting concerns), обеспечение интегральной 
производительности и~др. Говоря кратко, по мере роста масштаба программные 
изделия стремительно утрачивают \textit{технологичность} (manufacturability).
   
   Снижение трудозатрат на выполнение рутинных процедур возможно за счет 
автоматизации~--- перехода к~метапрограммированию. Однако традиционных 
CASE-средств для такого перехода недоста\-точно: автоматизация 
вышеуказанных процедур требует специализированных инструментов. 
Широкие возможности по специализации инструментария заложены 
в~технологиях разработки сис\-тем, управляемой моделями (MDE)~[3]. 
Технологии MDE нацелены на создание средств, 
позволяющих описать систему набором формализованных компьютерных 
моделей, в~том числе на специализированных языках (domain-specific languages, 
DSL), и~далее автоматически сгенерировать из моделей тексты про\-граммно\-го 
обеспечения системы. Языки моделирования строятся путем специализации на 
уровне метамоделей из языков общего назначения, таких как язык  
объ\-ект\-но-ори\-ен\-ти\-ро\-ван\-но\-го моделирования UML
(unified modeling language)~[4], языки 
реального времени~[5], языки логического вывода~[6] и~др. Сборка моделей 
систем из моделей компонентов описывается <<мегамоделями>> 
и~выполняется при помощи разнообразных механизмов, в~том числе 
связывания (weaving)~--- основного способа сборки систем  
в~ас\-пект\-но-ори\-ен\-ти\-ро\-ван\-ном подходе~[7]. Предоставляются 
инструменты для создания средств автоматического конструирования, анализа, 
сборки и~трансформации моделей, в~том числе для генерации про\-граммно\-го 
кода. Большой набор таких инструментов и~средств создан и~опубликован 
в~открытом исходном коде в~рамках проекта Eclipse Modeling Project~[8].
   
   Разнообразные технологии MDE интенсивно развиваются в~настоящее 
время, и~для них еще не выработан общий концептуальный фундамент. Не 
хватает апробированных типовых решений, применение которых не влечет 
риска массовой генерации программ, содержащих ошибки либо по\-треб\-ля\-ющих 
недопустимо много вычислительных ресурсов. Как следствие, не решены 
проблемы масштабирования технологий MDE~--- затраты труда на разработку, 
управляемую моделями, слишком быстро растут по мере увеличения размера 
и~структурной сложности программных систем~[9]. В~этих условиях новые 
инструменты метапрограммирования больших систем должны быть 
специфицированы и~верифицированы на максимально высоком уровне 
строгости~--- таком, который может дать только математический аппарат. 
Аппарат должен позволять кратко описывать механизмы масштабирования, 
формулировать и~доказывать их основные свойства, не <<потонув>> в~деталях 
описания частных моделей и~языков. Ввиду этого не удается привлечь 
традиционные формальные методы программирования, базирующиеся на 
разноплановых <<тяжеловесных>> математических средствах, слабо 
совместимых друг с~другом~[10].
   
   Перспективным представляется применение тео\-рии категорий~[11]~--- 
раздела высшей алгебры, который позволяет прозрачно и~компактно 
формализовать принципы системной инженерии~[12]. Вводятся категории, 
объектами которых служат модели системных единиц (компонентов, 
подсистем, систем и~т.\,д.), а~морфизмы формально описыва\-ют преобразования 
моделей по ходу процессов раз-\linebreak работки систем. Строятся  
тео\-ре\-ти\-ко-ка\-те\-гор\-ные конструкции, описывающие способы 
выполнения\linebreak трудоемких процедур (технологические карты) на абстрактном 
концептуальном уровне: моделям, задействованным в~процедурах, 
сопоставляются объекты подходящих категорий, технологическим операциям 
сопоставляются морфизмы, переходам между технологиями сопоставляются 
функторы и~т.\,д. Путем вычислений в~категориях оценива\-ются свойства 
способов выполнения процедур, выбирается такой из альтернативных 
способов, который доставляет экстремальное значение целевому функционалу 
технологичности (стоимости, надежности и~т.\,д.). Конструкция, 
соответствующая самому технологичному способу, интерпретируется 
в~терминах подходящей технологии, и~выбираются либо создаются 
инструменты метапрограммирования для автоматизации работ по его 
реализации~[13].
   
   Статья построена следующим образом. В~разд.~2 рассматриваются 
технологические проблемы проектирования больших автоматизированных 
систем (преимущественно в~части программного обеспечения) и~подходы к~их 
решению. В~разд.~3 приводится пример~--- повышение эффективности 
процесса проектирования больших расчетных задач за счет 
метапрограммирования. Раздел~4 посвящен принципам применения теории 
категорий в~качестве математического аппарата метапрограммирования. 
В~заключении подводятся итоги и~намечаются перспективы представленных 
исследований.

\vspace*{-6pt}

\section{Проблемы проектирования больших автоматизированных 
систем}

\vspace*{-2pt}
   
   Основной причиной роста значений масштабных факторов систем являются 
требования полноты (замкнутости) массивов сущностей, опи\-сы\-ва\-ющих объект 
автоматизации, относительно тех или иных структурных отношений. Именно 
такие требования вызывают необходимость масштабировать технологии 
метапрограммирования типа MDE, поскольку сущности и~отношения образуют 
основной предмет моделирования. К~числу главных сис\-те\-мо\-об\-ра\-зу\-ющих 
отношений относятся топологические, каузальные, мереологические  
(<<часть--це\-лое>>), дескриптивные (<<абст\-ракт\-ное--кон\-крет\-ное>>) 
и~телеологические (<<цель--сред\-ст\-во>>)~[14].
   
   В то же время результаты процессов проектирования программных систем 
могут вступать только в~отношения <<часть--це\-лое>>  
и~<<абст\-ракт\-ное--кон\-крет\-ное>>: действия по созданию сис\-те\-мы сводятся 
к~(де)композиции составных результатов (например, детализация требований, 
сборка приложения\linebreak из объектных модулей) и~трансформации (refinement) 
абстрактных результатов в~конкретные (например, реализация спецификации 
на языке программирования)~[15]. Поэтому разработчики\linebreak
 автоматизированных 
систем вынуждены выражать все возможные связи сущностей через 
мереологические и~дескриптивные связи отвечающих им сис\-тем\-ных единиц. 
Для топологических и~каузальных связей эта проблема решается путем 
сохранения их в~базе данных: они отображаются на мереологические связи 
структур данных. Эффективность хранения и~обработки связей достигается 
путем экстенсивного наращивания количества и~мощности вычислительных 
узлов, параллельно выполняющих навигацию по ним. Здесь привлекаются 
технологии параллельных и~распределенных вычислений типа Grid~[16] или 
вычислительных облаков.
   
   Однако с~телеологическими связями поступить подобным образом не 
удается, поскольку с~ростом масштаба они неизбежно приобретают 
конфликтность, запутанность, изменчивость (в~то время как мереологическим 
связям структур данных присуща низкая гибкость). Появляется большое 
количество рассеянных задач~--- таких, которые не поддаются локализации 
в~рамках функционально замкнутых программных модулей с~фиксированным 
интерфейсом. Обеспечение технологичности автоматизированных систем 
в~условиях рассеяния задач является целью  
ас\-пект\-но-ори\-ен\-ти\-ро\-ван\-но\-го программирования (АОП)~[17]. 
Реализация рассеянной задачи в~АОП оформляется как аспект~--- особая 
программная единица, код которой автоматически вставляется в~код других 
единиц в~мес\-тах, явно задаваемых внешним образом. Процедура вставки 
аспектов называется связыванием (weaving). Она предоставляет аспектам 
полный доступ к~контексту в~обход ограничений модульного интерфейса. 
Примером технологии АОП служит AspectJ~---  
ас\-пект\-но-ори\-ен\-ти\-ро\-ван\-ное расширение языка Java~[18].
   
   Однако на практике АОП применяется значительно реже, чем модульные 
подходы, поскольку отсутствует единое непротиворечивое понимание его 
методологической основы~[19]. На семантическом уровне неясно, как 
рационально выделять и~комплексировать аспекты в~программах и~моделях. 
Существующие технологии АОП предлагают лишь частные решения в~рамках 
частных парадигм программирования. Они ограничены поддержкой  
про\-граммно-тех\-ни\-че\-ских рассеянных задач, таких как журналирование, 
кэширование, защита ин-\linebreak формации и~т.\,п. Не хватает технологичных решений %\linebreak 
и~инструментов для моделирования и~реализации семантически богатых 
функциональных рассеянных задач, таких как ведение информационной %\linebreak  
модели (паспорта) объекта автоматизации, опо\-вещение участников процессов 
о~ходе их выполнения, оперативная оценка эффективности процессов, 
проверка правильности действий пользователей и~компонентов системы, 
перевод информации на разные языки и~в разные форматы.
   
   В связи с~этим в~\cite{13-kov, 20-kov} предложен универсальный  
тео\-ре\-ти\-ко-ка\-те\-гор\-ный подход к~расширению <<модульных>> 
технологий проектирования систем ас\-пект\-но-ори\-ен\-ти\-ро\-ван\-ны\-ми 
приемами, пригодный к~применению в~технологиях MDE. В~основе подхода 
лежит направленность на обеспечение трассирования~--- прослеживания 
воплощения абстрактно поставленных задач в~конкретных фрагментах 
результатов процесса проектирования~\cite{21-kov}. По существу, подход 
обеспечивает погружение телеологических связей задач в~дескриптивные связи 
моделей средств их автоматизации. Такое погружение оправдано тем, что 
трассируемость страдает от рассеяния задач больше, чем другие показатели 
качества систем.

\section{Эффективное проектирование расчетных задач}

   В качестве примера процесса, эффективность которого удается повысить за 
счет метапрограммирования, рассмотрим проектирование базового 
математического обеспечения интеллектуальных электроэнергетических сетей, 
состоящего из задач расчета и~анализа показателей распределенных 
энергетических объектов. Исходные данные для расчета поступают в~систему 
от измерительных приборов, из смежных информационных систем 
и~с~автоматизированных рабочих мест пользователей. В~первую очередь 
оперативно вычисляются режимные показатели~--- результаты 
распространения значений параметров количества и~качества энергоресурсов 
вдоль оси времени (статистический анализ, прогнозирование), вдоль линий 
передачи энергии (замещение отсутствующих значений, расчет отпуска из сети 
по приему в~сеть и~др.), вдоль иерархии управления объектом (например, 
формирование сводного топ\-лив\-но-энер\-ге\-ти\-че\-ско\-го баланса города). 
Решение таких задач сводится к~суперпозиции базовых алгоритмов, 
определяющих один шаг распространения (прогноз изменения потребности 
энергоустановки в~энергоресурсе за один расчетный период, объем потери 
энергоресурса в~элементе энергетической сети и~т.\,п.). Обычно базовые 
алгоритмы задаются алгебраическими формулами,\linebreak разностными схемами либо 
задачами математического программирования. Порядок применения\linebreak 
суперпозиции определяется каузальными, топологическими 
и~мереологическими связями между сущностями, составляющими объект.
   
   Расчет путем навигации по истории, топологии и~иерархии объектов 
требуется также для показателей технического состояния, определяющих 
уровень износа оборудования, надежность, потребность в~ремонте и/или 
модернизации. Массивы значений режимных и~технических показателей 
служат входными данными для расчета необходи\-мых управляющих 
воздействий на энергетический объект. Фактические значения сравниваются 
с~целевыми, определяются причины расхождений, формируются 
и~исполняются задания на производство работ по возврату значений в~целевые 
рамки. %\linebreak 
В~завершение цикла расчета вычисляются агрегатные показатели 
уровня развития объекта управ\-ле\-ния, характеризующие его способность 
удовлетворять потребности, новизну техники, энергетическую 
эффективность~\cite{22-kov}. Эти показатели используются в~процессах 
стратегического управления.
   
   При программной реализации расчета показа\-телей традиционные 
<<ручные>> процедуры со\-став\-ле\-ния, верификации и~актуализации программ 
требуют затрат труда, растущих пропорционально масшта\-бу энергетического 
объекта. Кроме того, для достижения необходимой производительности нужно 
выполнять большие расчетные задачи в~среде распределенных вычислений 
типа Grid, развертывание программных модулей в~которой требует 
значительных затрат труда квалифицированных системных 
программистов~\cite{23-kov}. В~то же время вся информация, необходимая для 
формирования расчетных алгоритмов, присутствует в~информационной модели 
(паспорте) объекта управления: в~нем хранятся и~параметры оборудования, 
и~всевозможные связи, и~структура вычислительной среды~\cite{24-kov}. 
Поэтому для снижения затрат целесообразно воспользо\-ваться подходом типа 
MDE~--- создать автоматический генератор расчетных компонентов по 
наполнению паспорта. Генератор <<обходит>> фрагмент паспорта, отвечающий 
участку энергетической сети, показатели которого требуется рассчитать, 
и~порождает тексты программ модулей, вы\-чис\-ля\-ющих значения показателей. 
Тексты содержат расчетные формулы, обращения к~специализированным 
инструментам расчета и~анализа данных, акты межмодульного взаимодействия, 
системные вызовы для развертывания в~среде распределенных вычислений. 
Исполнение сгенерированных модулей позволяет обеспечить информационную 
поддержку принятия решений в~темпе процесса  
опе\-ра\-тив\-но-дис\-пет\-чер\-ско\-го управления объектом.
   
   При разработке такого генератора не удается воспользоваться готовыми 
инструментами метапрограммирования наподобие созданных в~рамках проекта 
Eclipse Modeling Project, поскольку требуется решить две серьезные проблемы. 

Первая из них состоит в~необходимости обеспечить не только правильность 
формирования расчетных формул, но и~корректность и~производительность их 
выполнения при любых входных данных с~учетом особенностей архитектуры 
используемых вычислительных средств. Выполнение программ, составленных 
путем прямой реализации спецификаций алгоритмов без учета таких 
особенностей, может сопровождаться переполнениями, искажениями данных, 
коммуникационными блокиров\-ками. Например, промежуточные результаты 
вы\-чис\-ле\-ния показателей больших объектов могут\linebreak выходить за рамки 
стандартного диапазона машинного представления чисел. Поэтому необходимо 
отобразить алгоритмы на архитектуру вы\-чис\-ли\-тель\-ной среды~\cite{25-kov}~--- 
подобрать типы данных, структуру управления потоком вычислений, режимы 
взаимодействия с~другими модулями так, чтобы избежать этих проблем, 
затрачивая минимальное количество ресурсов машинного времени, памяти, 
емкости каналов связи.
   
   При <<ручной>> реализации вычислительных алгоритмов программист 
выполняет отображение умозрительно, основываясь на документации, 
сопровождающей аппаратуру и~средства программирования. Генератор же 
должен автоматически обеспечивать соответствие структур порождаемой 
программы характеристикам вычислительной среды, указанным в~паспорте. 
Необходимо встроить в~генератор формальную модель процедуры 
отоб\-ра\-же\-ния, позволяющую адаптировать ресурсные потребности 
генерируемого кода к~возможностям узлов вычислительной среды.

\begin{figure*} %fig1
 \vspace*{1pt}
 \begin{center}
 \mbox{%
 \epsfxsize=120.068mm
 \epsfbox{kov-1.eps}
 }
 
\vspace*{6pt}


\noindent
{\small Процесс проектирования генератора расчетных компонентов 
с~применением тео\-ре\-ти\-ко-ка\-те\-гор\-ных методов}
 \end{center}

\vspace*{9pt}

\end{figure*}
   


   При формализации отображения каждый узел вычислительной среды 
представляется моделью вычислений~--- конечной алгеброй, состоящей из 
совокупности всех чисел, помещающихся в~память, и~вычислительных 
примитивов над ними. Интеграция вычислительного узла в~среду заключается 
в~кодировании чисел, поддерживаемых узлом, т.\,е.\ в~сопоставлении им кодов 
чисел, поддерживаемых средой. Формально действиям по интеграции отвечают 
отображения (в обычном математическом смысле) основных множеств алгебр, 
задающие правила кодирования так, что структура тех вычислительных 
операций, которые узел должен выполнять в~составе системы, не разрушается. 
Все модели вычислений и~все действия по их интеграции образуют 
категорию~\cite{26-kov}, конструкции в~которой, в~том числе мегамодели, 
позволяют строго описывать и~оптимизировать распределение ресурсов 
вычислительной среды под выполнение расчетных алгоритмов. Чтобы найти 
узлы среды, на которых можно разместить алгоритм, строится алгебра 
операций, выполняемых алгоритмом. Погружение этой алгебры 
в~вычислительную среду задается вычислительным расширением~--- 
функтором специального вида, который формализует отображение задачи на 
среду: задача размещается в~узлах, алгебраические модели которых 
поддерживают операции ее вычислительного расширения. 

Для оптимизации 
размещения с~точки зрения производительности применяются мощные 
математические методы~\cite{27-kov}: сигнатурным операциям алгебр, 
отвечающих узлам среды, и~связывающим узлы морфизмам приписываются 
оценки длительности, получаемые из паспортных характеристик среды и/или 
результатов прогона тес\-тов производительности, и~строится размещение, 
обладающее минимальной интегральной длительностью расчета. В~терминах 
MDE процедуру вы\-чис\-ли\-тель\-но\-го расширения можно рассматривать как 
частный случай перехода от плат\-фор\-мен\-но-не\-за\-ви\-си\-мой модели 
расчетной задачи (platform independent model, PIM)  
к~плат\-фор\-мен\-но-за\-ви\-си\-мой модели развертывания задачи 
в~вычислительной среде (platform specific model, PSM). Реализация этой 
процедуры и~встраивается в~генератор расчетных модулей.
   
   Вторая проблема, связанная с~автоматизацией\linebreak порождения расчетных 
модулей, состоит в~обеспечении их актуальности~--- соответствия про\-грам\-мной 
реализации алгоритмов фактическому состоянию объекта управления. 
Нетрудно добиться\linebreak соответствия алгоритмов паспорту: нужно авто\-матически 
вызывать генератор при изменениях паспорта. Значительно труднее обеспечить 
соот\-ветствие паспорта объекту, поскольку объект может %\linebreak
 изме\-ниться в~ходе 
совершенно разнородных процессов: реконструкции и~ремонта, смены 
собственника, энергетического обследования и~т.\,д. Необходимо рассеять 
задачу ведения паспорта по\linebreak средствам автоматизации всех таких процес\-сов 
с~обеспечением трассирования каждой записи паспорта к~задачам, вызвавшим 
ее изменение. Для этого применяется процедура аспектного связывания, 
формально специфицированная и~верифицированная при помощи конструкций 
тео\-ре\-ти\-ко-ка\-те\-гор\-но\-го подхода к~АОП~\cite{20-kov}. Благодаря 
связыванию достигается практически полная актуальность и~достоверность 
паспорта, в~противовес традиционным <<ручным>> регламентным процедурам 
его ведения.
   
   Таким образом, процесс проектирования генератора расчетных компонентов 
с~применением тео\-ре\-ти\-ко-ка\-те\-гор\-ных методов можно представить в~виде
схемы (см.\ рисунок)~\cite{28-kov}. 



   Такой процесс был апробирован на практике при проектировании 
подсистемы расчета и~анализа данных программной платформы учета 
и~управ\-ле\-ния энергообеспечением <<Энергиус>>~\cite{29-kov}. Это позволило 
снизить стоимость и~сроки внедрения больших систем диспетчерского 
управ\-ле\-ния, интеллектуального учета электроэнергии, управления 
энергоэффективностью на базе платформы <<Энергиус>> по сравнению 
с~функционально аналогичными программными продуктами.

\section{Теория категорий как~математический аппарат 
метапрограммирования}
   
   Рассмотрим проблему выбора математического аппарата, пригодного для 
строгого формального описания и~анализа технологических процедур 
проектирования больших автоматизированных сис\-тем, в~том числе 
с~применением метапрограммирования. Известно~\cite{30-kov}, что 
традиционное математическое моделирование посредством дифференциальных 
уравнений практически не применимо в~инженерии программных систем. Дело 
в~том, что для вывода уравнений необходимы вариационные принципы, 
законы сохранения либо статистические закономерности, а~в~процессах 
разработки программ обнаружить их не удалось. Фактически отсутствует даже 
устоявшаяся система координат~--- числовых показателей, зависимость 
которых от времени характеризует все значимые аспекты процесса создания 
программной системы. (Очевидные показатели типа числа строк текста 
программы весьма слабо коррелируют с~трудоемкостью написания текста, 
особенно в~условиях применения метапрограммирования.)
   
   Существует альтернативный прагматический подход к~выбору 
математического аппарата, основанный на том наблюдении, что для 
большинства систем доступна (либо легко восстановима) история сборки из 
составных частей. Если известны математические модели частей и~сборочных 
операций, то можно задавать системы математическими <<мегамоделями>>~--- 
ориентированными графами (диаграммами), узлы которых помечены 
обозначениями частей, а~ребра помечены обозначениями операций. Здесь 
требуется формировать и~обрабатывать большие графы, которые целиком даже 
нельзя изобразить, а~можно только описать структурными ограничениями. 
Мощный аппарат для конструирования и~анализа диаграмм такого рода развит 
в~рамках теории категорий~--- раздела высшей алгебры, который <<начинается 
с~наблюдения, что многие свойства математических систем можно представить 
просто и~единообразно посредством диаграмм>>~\cite{11-kov}.
   
   Напомним, что категория $C$~--- это класс абстрактных объектов Ob\,$C$, 
попарно связанных морфизмами (абстрактными аналогами 
отображений)~\cite{11-kov}: каждый морфизм~$f$ имеет область 
$\mathrm{dom}\,f\hm\in\mathrm{Ob}~C$ и~кообласть $\mathrm{codom}\,f\hm\in 
\mathrm{Ob}~C$. Соотношения вида $\mathrm{dom}\,f\hm=A$ 
и~$\mathrm{codom}\,f\hm=B$ наглядно записываются в~форме стрелки $f:\ 
A\hm\to B$, а~множество всех морфизмов, удовлетворяющих этим 
соотношениям, обозначается через $\mathrm{Mor}\left(A,B\right)$. Для любой 
пары морфизмов~$f,g$, такой что $\mathrm{codom}\,f\hm=\mathrm{dom}\,g$, 
определена композиция~--- морфизм $g\circ f:\ \mathrm{dom}\,f\hm\to 
\mathrm{codom}\,g$. Композиция ассоциативна: для любой тройки морфизмов 
$f, g, h$, если $\mathrm{codom}\,f\hm=\mathrm{dom}\,g$ и~
$\mathrm{codom}\,g\hm=\mathrm{dom}\,h$, то $h\circ (g\circ f)\hm=(h\circ g)\circ 
f$. Наконец, любой объект~$A$ обладает тождественным морфизмом $1_A:\ 
A\hm\to A$, таким что для любого морфизма $f:\ A\hm\to B$ выполняется 
соотношение $f\circ 1_A\hm= 1_B\circ f\hm=f$.
   
   При формализации процессов проектирования программных систем главную 
роль играют категории, объекты которых сопоставляются системным 
единицам, а~морфизмы сопоставляются действиям по интеграции 
(мереологическим связям): каждое действие отображает компонент (область 
морфизма) в~содержащую его систему (кообласть). Подчеркнем, что для 
компонента~$P$ и~системы~$S$, как правило, указывается не только 
возможность (или невозможность) интеграции~$P$ в~$S$, но и~совокупность 
всех способов интеграции (вставка, разрешение ссылок, перекомпоновка 
и~т.\,д.), по\-рож\-дающая множество морфизмов $\mathrm{Mor}\left(P, S\right)$. 
Ком\-позиция морфизмов отвечает многошаговым\linebreak действи\-ям (процессам), 
а~тождественные морфизмы~--- <<ничегонеделанию>>. Например, объектами 
категорий, описывающих формальные методы программирования, служат 
модели программ (алгебраические спецификации, графы, термы  
лямб\-да-ис\-чис\-ле\-ния и~т.\,д.), а~в~качестве морфизмов часто выступают 
гомоморфизмы и~их обобщения.
   
   Как указывалось выше, диаграммы в~таких категориях задают 
математические <<мегамодели>> сис\-тем. Процедуре сборки системы из 
мегамодели отвечает копредел (colimit) диаграммы~--- универсальная 
диаграммная конструкция~\cite{11-kov}. Для иллюстрации этой конструкции 
рассмотрим соединение компонента~$P$ с~системой~$S$~--- прием сборки, 
состоящий в~добавлении промежуточного компонента~$G$, называемого 
<<клеем>> (glue) или связкой (connector)~\cite{31-kov}, который способен 
интегрироваться как с~компонентом~$P$, так и~с~системой~$S$. Например, 
путем соединения строятся системы на базе промежуточного программного 
обеспечения (middleware), которое и~служит связкой. Мегамодель соединения 
имеет вид пары морфизмов $f:\ P\leftarrow G\rightarrow S:\ g$. Ее копредел, 
называемый в~тео\-рии категорий кодекартовым квадратом (pushout), задается 
объ\-ек\-том-вер\-ши\-ной~$V$ и~парой мор\-физ\-мов-ре\-бер $p:\ P\rightarrow 
V\leftarrow S:\ s$, таких что $p\circ f\hm= s\circ g$ (т.\,е.\ соблюдается 
структурная корректность системного решения) и,~кроме того, выполняется 
следующее условие универсальности: для любых объекта~$T$ и~пары 
морфизмов $u:\ P\rightarrow T\leftarrow S:\ v$, если $u\circ f\hm= v\circ g$, то 
существует единственный морфизм $w:\ V\hm\to T$, удовлетворяющий 
соотношениям $w\circ p\hm= u$ и~$w\circ s\hm= v$. Тогда объект~$V$ 
действительно отвечает системе, которая собрана из~$S$ и~$P$ путем 
соединения посредством~$G$ и~не содержит ничего <<лишнего>>. Если такой 
объект~$V$ существует, то он определяется однозначно с~точностью до 
изоморфизма~--- формального представления несущественного различия 
между моделями. Если же копредела не существует, то делается вывод, что 
клей~$G$ не способен соединить компонент~$P$ с~$S$ посредством 
действий~$f$ и~$g$, т.\,е.\ что исходная мегамодель неправильно построена.
   
%\begin{figure*}
 \begin{center}
 \mbox{%
 \epsfxsize=64.305mm
 \epsfbox{kov-2.eps}
 }
 \end{center}
%\end{figure*}
   
   Конструкция кодекартова квадрата часто рассматривается в~литературе по 
тео\-ре\-ти\-ко-ка\-те\-гор\-ным методам сборки программных систем~(см.,\linebreak
например,~\cite{32-kov}). 
При некоторых технических ограничениях через нее можно выразить 
копредел любой конечной диаграммы~\cite{11-kov} (т.\,е.\ формально собрать 
систему из любой мегамодели). Например, любая конечная диаграмма может 
быть реализована в~виде системы в~категории из разд.~3, формально\linebreak 
описывающей проектирование средств распределенных  
вычислений~\cite{26-kov}. Это необходимо для\linebreak реализации динамического 
развертывания вы\-чис\-лительных задач в~среде типа Grid: топология размещения 
расчетных компонентов должна диктоваться ресурсными, а~не структурными 
ограничениями.
   
   Путем обобщения кодекартова квадрата строится типовая формальная 
мегамодель основной технологической процедуры АОП~--- аспектного 
связывания~\cite{20-kov}. В~классическом АОП связывание состоит 
в~подключении программы~$W$, называемой советом (advice), к~базовой 
программе (base) $B$ в~заданных местах, называемых точками соединения (join 
points)~\cite{33-kov}. Каждый раз, когда при исполнении базовой программы 
встречается точка соединения, вызывается совет. Поэтому аспект обычно 
выглядит как блок программного кода, охра\-ня-\linebreak емый (guarded) условием, 
идентифицирующим точку соединения; начало блока служит точкой вызова 
совета (entry point). Таким образом, инструмент связывания (weaver) принимает 
на вход две спецификации:
   \begin{enumerate}[(1)]
\item описание точек соединения в~базовой программе, или срез (pointcut);
\item описание точек вызова совета в~точках соединения.
\end{enumerate}

   При связывании на первом шаге (виртуально) создается достаточное 
количество копий совета, по одной на каждую точку соединения, 
с~маркировкой соответствующих им точек вызова. Далее на втором шаге эти 
точки <<склеиваются>> друг с~другом так, чтобы не разрушить аспектную 
структуру базы\linebreak и~совета. Для формальной записи правил связы\-вания 
привлекается дополнительная модель~$C$,\linebreak на\-зы\-ва\-емая связкой 
(connector~\cite{34-kov}), которая интегрируется с~базой в~точках соединения, 
а~с~советом~--- в~точках вызова. В~технологиях типа AspectJ в~роли связки 
выступает регулярное выражение, выделяющее в~тексте базовой программы 
синтаксические единицы, образующие срез (конструкция  
pointcut)~\cite{18-kov}. Соответствие точек соединения точкам вызова задается 
парой морфизмов $j:\ B\leftarrow C\rightarrow W:\ e$ (здесь наглядно 
проявляется отличие связывания от модульной компоновки, формализуемой 
одношаговым действием вида $l:\ M\rightarrow S$, где $M$~--- модуль, $S$~--- 
система). Первому шагу связывания отвечает построение произведения 
$C\times W$ вместе с~морфизмом $\langle 1_C,e\rangle:\ C\hm\to C\times W$, 
который однозначно определяется условиями $p_C\circ \langle 1_C, 
e\rangle\hm=1_C$ и~$p_W\circ \langle 1_C,e\rangle \hm=e$, где $p_C:\ C\times 
W\hm\to C$ и~$p_W:\ C\times W\hm\to W$~--- проекции произведения на 
компоненты. Второй шаг связывания, как легко видеть, формально 
пред\-став\-ля\-ет собой соединение~--- кодекартов квад\-рат над парой морфизмов 
$j:\ B\leftarrow C\rightarrow C\times W:\ \langle 1_C,e\rangle$. Результатом 
связывания является вершина этого квад\-ра\-та, обозначаемая через $j\bowtie e$ 
(при условии, что и~произведение $C\times W$, и~сам кодекартов квадрат 
существуют).
   
%\begin{figure*}
 \begin{center}
 \mbox{%
 \epsfxsize=41.246mm
 \epsfbox{kov-3.eps}
 }
 \end{center}
%\end{figure*}
   
   Путем анализа этой конструкции строго доказан ряд свойств связывания 
(для случаев, когда оно существует)~\cite{20-kov}: единственность результата 
с~точностью до изоморфизма, наличие неразрушающего вложения базы 
в~результат, независимость результата привязывания нескольких взаимно 
независимых советов к~одной базе от порядка связывания, сохранение 
аспектной структуры базы при связывании с~достаточно мелкими единицами 
аспектной декомпозиции. В~свою очередь, существование связывания доказано 
для ряда практически значимых частных случаев, в~том числе для аспекта 
ведения паспорта объекта управления.
   
   Отметим, что сама конструкция копредела, как видно из ее названия, 
является двойственной по отношению к~конструкции предела, которая была 
введена первоначально~--- для нужд приложений тео\-рии категорий 
к~топологии~\cite{11-kov} (и, вообще говоря, не связана непосредственно 
с~понятием предела из математического анализа). Предел отличается от 
копредела обращением направления ребер, так что в~приложениях к~системной 
инженерии предел представляет <<наибольшую общую часть>> составляющих 
мегамодели, в~то время как копредел представляет <<наименьшее объемлющее 
целое>> (систему). Например, частным случаем предела является 
произведение, которое содержится в~мегамодели аспектного связывания.

\section{Заключение}

   Технологии метапрограммирования типа MDE и~АОП обладают большим 
потенциалом в~качестве средств повышения технологичности больших 
автоматизированных систем. Как показано выше, они позволяют заменять 
десятки про\-грам\-ми\-стов-ко\-ди\-ров\-щи\-ков небольшими группами, 
создающими инструменты для автоматического порождения массивного 
программного кода. Ключевую роль в~проектировании, верификации 
и~масштабировании таких инструментов играет формализация приемов 
метапрограммирования, которую целесообразно проводить на базе 
математического аппарата теории категорий.
   
   Однако привлечение указанных технологий служит лишь первым шагом на 
пути к~радикальному повышению эффективности процессов создания систем за 
счет сквозной автоматизации. На следующем шаге необходимо поднять 
уровень интеллектуальности инструментов программной инженерии~--- 
переложить на них построение, анализ и~оптимизацию процедур 
проектирования. Современные CASE-сред\-ст\-ва практически не умеют этого 
делать~--- они способны только механически выполнять команды инженеров. 
Значительного прогресса можно добиться, привлекая теорию категорий: можно 
использовать ее язык для формального внутримашинного представления 
процедур проектирования и~свести интеллектуальные функции 
к~распознаванию и~расчету категорных конструкций. Отметим, что такой 
подход позволил бы повысить эффективность и~в~автоматизированном 
проектировании сложных изделий материального производства.
   
   Уже давно существуют автоматические решатели  
тео\-ре\-ти\-ко-ка\-те\-гор\-ных задач, в~том числе возникающих при разработке 
программных систем~\cite{35-kov}. Но в~дополнение к~решателям для создания 
полноценных средств интеллектуального управления такой интеллектуальной 
деятельностью, как программная инженерия, требуются новые подходы, по 
мощности качественно превосходящие традиционные нейронные сети 
и~численные методы оптимизации. Здесь открываются перспективные 
направления дальнейших исследований.
   
{\small\frenchspacing
 {%\baselineskip=10.8pt
 \addcontentsline{toc}{section}{References}
 \begin{thebibliography}{99}
\bibitem{1-kov}
Ultra-Large-Scale Systems: The software challenge of the future.~--- Pittsburgh: Carnegie Mellon 
Software Engineering Institute, 2006. 134~p.
\bibitem{2-kov}
\Au{Anvaari M., Cruzes D.\,S., Conradi~R.} Smart Grid software applications as an  
ultra-large-scale system: Challenges for evolution~// Innovative Smart Grid Technologies 
(\mbox{ISGT}).~--- IEEE PES, 2012. P.~1--6.
\bibitem{3-kov}
\Au{Schmidt D.\,C.} Model-driven engineering~// IEEE Computer, 2006. Vol.~39. No.\,2.  
P.~25--32.
\bibitem{4-kov}
\Au{Graaf B., van Deursen A.} Visualisation of domain-specific modelling languages using 
UML~// 14th IEEE  Conference (International) on the Engineering of Computer-Based Systems 
ECBS'2007 Proceedings.~--- Tucson, 2007. P.~586--595.
\bibitem{5-kov}
\Au{Benveniste A., Caspi P., Edwards~S.\,A., Halbwachs~N., Le Guernic~P., de Simone~R.} The 
synchronous languages 12~years later~// Proc. IEEE, 2003. Vol.~91. Iss.~1. P.~64--83.
\bibitem{6-kov}
\Au{Васильев С.\,Н.} Формализация знаний и~управление на основе позитивно-образованных 
языков~// Информационные технологии и~вычислительные системы, 2008. №\,1. С.~3--17.
\bibitem{7-kov}
\Au{Jouault F., Vanhooff~B., Bruneliere~H., Doux~G., Berbers~Y., Bezivin~J.} Inter-DSL 
coordination support by combining megamodeling and model weaving~// 2010 ACM Symposium 
Applied Computing Proceedings.~--- Sierre, 2010. P.~2011--2018.
\bibitem{8-kov}
Eclipse Modeling Project. The Eclipse Foundation, 2014. {\sf http://www.eclipse.org/modeling}.
\bibitem{9-kov}
\Au{Kolovos D.\,S., Rose L.\,M., Matragkas~N., Paige~R.\,F., Guerra~E., Cuadrado~J.\,S., De 
Lara~J., R$\acute{\mbox{a}}$th~I., Varr$\acute{\mbox{o}}$~D., Tisi~M., Cabot~J.} A~research 
roadmap towards achieving scalability in model driven engineering~//  Workshop on Scalability in 
Model Driven Engineering Proceedings.~--- Budapest, Hungary: ACM, 2013. P.~2:1--2:10.
\bibitem{10-kov}
\Au{Diskin Z., Maibaum~T.\,S.\,E.} Category theory and model-driven engineering: From formal 
semantics to design patterns and beyond~//  7th Workshop ACCAT'2012 ``Electronic Proceedings 
in Theoretical Computer Science,'' 2012. Vol.~93. P.~1--21.
\bibitem{11-kov}
\Au{Маклейн С.} Категории для работающего математика~/ Пер. с~англ.~--- М.: Физматлит, 
2004. 352~с. (\Au{Mac Lane~S.} Categories for the working mathematician.~--- Springer, 1978. 
317~p.)
\bibitem{12-kov}
\Au{Goguen J.} Categorical foundations for general systems theory~// Advances in cybernetics and 
systems research.~--- London: Transcripta Books, 1973. P.~121--130.
\bibitem{13-kov}
\Au{Ковалёв С.\,П.} Тео\-ре\-ти\-ко-ка\-те\-гор\-ный подход к~метапрограммированию.~--- М.: 
ИПУ РАН, 2014. 112~с.
\bibitem{14-kov}
\Au{Ковалёв С.\,П.} Системный анализ жизненного цикла больших  
ин\-фор\-ма\-ци\-он\-но-управ\-ля\-ющих сис\-тем~// Автоматика и~телемеханика, 2013. №\,9. 
С.~98--118.
\bibitem{15-kov}
\Au{Sannella D.} A~survey of formal software development methods~// Software engineering: 
A~European prospective.~--- IEEE Computer Society Press, 1993. P.~281--297.
\bibitem{16-kov}
Grid computing: Making the global infrastructure a~reality.~---  New York, NY, USA: 
Wiley\,\&\,Sons, 2003. 1060~p.
\bibitem{17-kov}
\Au{Kiczales G., Lamping~J., Mendhekar~A., Maeda~C., Lopes~C.\,V., Loingtier~J.-M., Irwin~J.} 
Aspect-oriented programming~// ECOOP'97~--- object-oriented programming~/
Eds. M.~Aksit, S.~Matsuoka.~--- Lecture notes in computer science ser.~---
Springer, 1997. Vol.~1241.  
P.~220--242.
\bibitem{18-kov}
\Au{Colyer A., Clement~A., Harley~G., Webster~M.} Eclipse AspectJ: Aspect-oriented 
programming with AspectJ and the Eclipse AspectJ development tools.~--- Reading:  
Addison-Wesley, 2004. 504~p.
\bibitem{19-kov}
\Au{Steimann F.} The paradoxical success of aspect-oriented programming~//  Conference 
(International) \mbox{OOPSLA'2006} Proceedings.~--- Portland, 2006. P.~481--497.
\bibitem{20-kov}
\Au{Ковалёв С.\,П.} Семантика аспектно-ори\-ен\-ти\-ро\-ван\-но\-го моделирования данных 
и~процессов~// Информатика и~её применения, 2013. Т.~7. Вып.~3. С.~70--80.
\bibitem{21-kov}
\Au{Kannenberg A., Saiedian~H.} Why software requirements traceability remains a~challenge~// 
J.~Defense Software Engineering, 2009. July/August. P.~14--19.
\bibitem{22-kov}
\Au{Гребенюк Г.\,Г., Лубков Н.\,В., Никишов С.\,М.} Информационные аспекты управления 
муниципальным хозяйством.~--- М.: ЛЕНАНД, 2011. 320~с.
\bibitem{23-kov}
\Au{Manset D., Verjus~H., McClatchey~R., Oquendo~F.} A~formal architecture-centric 
model-driven approach for the automatic generation of Grid applications~// 8th Conference (International)  
``Enterprise Information Systems: Databases and Information Systems Integration''  
Proceedings.~--- Paphos, Cyprus, 2006. P.~322--330.
\bibitem{24-kov}
\Au{Uslar M., Schmedes~T., Lucks~A., Luhmann~T., Winkels~L., Appelrath~H.-J.} Interaction of 
EMS related systems by using the CIM standard~// 2nd  ICSC Symposium (International) 
on Information Technologies in Environmental Engineering ITEE 2005 Proceedings.~--- Magdeburg, 
2005. P.~596--610.
\bibitem{25-kov}
\Au{Воеводин В.\,В.} Отображение проблем вычислительной математики на архитектуру 
вычислительных сис\-тем~// Вычислительные методы и~программирование, 2000. Т.~1. 
С.~37--44.
\bibitem{26-kov}
\Au{Ковалёв С.\,П.} Алгебраический подход к~проектированию распределенных 
вычислительных сис\-тем~// Сибирский журнал индустриальной математики, 2007. Т.~10. 
№\,2. С.~70--84.
\bibitem{27-kov}
\Au{Топорков В.\,В.} Модели распределенных вычислений.~--- М.: Физматлит, 2004. 320~с.
\bibitem{28-kov}
\Au{Ковалёв С.\,П.} Повышение эффективности процессов проектирования больших  
ин\-фор\-ма\-ци\-он\-но-управ\-ля\-ющих сис\-тем~// Тр. XII Всеросс. совещания по  
проб\-ле\-мам управления ВСПУ-2014.~--- М.: ИПУ РАН, 2014. С.~9291--9300.
\bibitem{29-kov}
\Au{Ковалёв С.\,П., Андрюшкевич~С.\,К., Гуськов~А.\,Е.} Интеграционная платформа учета 
и~управления энергообеспечением <<Энергиус>>. Свидетельство о государственной 
регистрации программы для ЭВМ №\,2009613359 от 26.06.2009.
\bibitem{30-kov}
\Au{Соммервилл И.} Инженерия программного обеспечения~/ Пер. с~англ.~--- 6-е изд.~--- М.: 
Вильямс, 2002. 624~с. (\Au{Sommerville~I.} Software engineering.~--- 6th ed.~--- Addison 
Wesley, 2001. 720~p.)
\bibitem{31-kov}
\Au{Allen R.\,J., Garlan~D.} A~formal basis for architectural connection~// ACM Trans. Software 
Engineering Methodology, 1997. Vol.~6. No.\,3. P.~213--249.
\bibitem{32-kov}
\Au{Smith D.\,R.} Composition by colimit and formal software development~// 
Algebra, meaning, and computation~/ Eds. K.~Futatsugi, J.-P.~Jouannaud, J.~Meseguer.~---
Lecture notes  in computer science ser.~--- Springer, 2006. Vol.~4060. P.~317--332.
\bibitem{33-kov}
Aspect-oriented software development.~--- Reading: Addison Wesley, 2004. 800~p.
\bibitem{34-kov}
\Au{Pinto M., Fuentes~L., Troya~J.\,M.} DAOP-ADL: An architecture description language for 
dynamic component and aspect-based development~// Generative programming and component
engineering~/ Eds. F.~Pfenning, Y.~Smaragkaris.~---
Lecture notes in computer science ser.~--- Springer, 2003. 
Vol.~2830. P.~118--137.
\bibitem{35-kov}
\Au{Srinivas Y.\,V., J$\ddot{\mbox{u}}$llig~R.} SPECWARE: Formal support for composing 
software~// Mathematics of program construstion~/
Ed. B.~M$\ddot{\mbox{o}}$ller.~--- Lecture notes in computer science ser.~---
Springer, 1995. Vol.~947. P.~399--422. 
\end{thebibliography}

 }
 }

\end{multicols}

\vspace*{-3pt}

\hfill{\small\textit{Поступила в~редакцию 19.11.15}}

%\vspace*{pt}

\newpage

%\vspace*{-24pt}

%\hrule

%\vspace*{2pt}

%\hrule

\vspace*{-24pt}



\def\tit{METAPROGRAMMING TO INCREASE MANUFACTURABILITY OF~LARGE-SCALE SOFTWARE-INTENSIVE SYSTEMS}

\def\titkol{Metaprogramming to increase manufacturability of large-scale software-intensive systems}

\def\aut{S.\,P.~Kovalyov}

\def\autkol{S.\,P.~Kovalyov}

\titel{\tit}{\aut}{\autkol}{\titkol}

\vspace*{-9pt}

\noindent
Institute of Control Sciences, Russian Academy of Sciences, 65~Profsoyuznaya Str., 
Moscow 117997, Russian Federation


\def\leftfootline{\small{\textbf{\thepage}
\hfill INFORMATIKA I EE PRIMENENIYA~--- INFORMATICS AND
APPLICATIONS\ \ \ 2016\ \ \ volume~10\ \ \ issue\ 1}
}%
 \def\rightfootline{\small{INFORMATIKA I EE PRIMENENIYA~---
INFORMATICS AND APPLICATIONS\ \ \ 2016\ \ \ volume~10\ \ \ issue\ 1
\hfill \textbf{\thepage}}}

\vspace*{3pt}

\Abste{An approach to reduce costs of large-scale software-intensive 
systems design due to applying modern metaprogramming technologies 
is proposed. Model-driven engineering and aspect-oriented software 
development are considered to be the most advanced among such technologies. 
The methods to scale these technologies are presented in order to apply them efficiently 
under growth of the target system size via closure with regard to basic structural 
relations. Design of mathematical software for smart electric grids is considered 
as a~case study for practical applications of the approach. Principles of mathematical 
device for constructing, analysis, and optimization of design technological procedures 
based on the category theory are described. The process to design the generator of 
computational software components 
of large-scale systems applying category-theoretical methods is drawn.}

\KWE{large-scale software-intensive systems; metaprogramming; megamodel; 
category theory; colimit; 
model driven engineering; aspect-oriented software development; smart grid}

\DOI{10.14357/19922264160105} 

%\Ack
%\noindent



%\vspace*{3pt}

  \begin{multicols}{2}

\renewcommand{\bibname}{\protect\rmfamily References}
%\renewcommand{\bibname}{\large\protect\rm References}

{\small\frenchspacing
 {%\baselineskip=10.8pt
 \addcontentsline{toc}{section}{References}
 \begin{thebibliography}{99}
\bibitem{1-kov-1}
 \textit{Ultra-Large-Scale Systems: The software challenge of the future}. 
 2006. Pittsburgh: 
Carnegie Mellon Software Engineering Institute. 134~p.
\bibitem{2-kov-1}
\Aue{Anvaari, M., D.\,S.~Cruzes, and R.~Conradi}. 2012. Smart Grid software applications as an 
ultra-large-scale system: Challenges for evolution. \textit{Innovative Smart Grid Technologies 
(ISGT)}. IEEE PES.  1--6.
\bibitem{3-kov-1}
\Aue{Schmidt, D.\,C.} 2006.  Model-driven engineering. \textit{IEEE Computer} 39(2):25--32.
\bibitem{4-kov-1}
\Aue{Graaf, B., and A.~van Deursen}. 2007. Visualisation of domain-specific modelling languages 
using UML. \textit{14th IEEE  Conference (International) on the Engineering of Computer-Based 
Systems ECBS'2007 Proceedings}. Tucson. 586--595.
\bibitem{5-kov-1}
\Aue{Benveniste, A., P.~Caspi, S.\,A.~Edwards, N.~Halbwachs, P.~Le Guernic, and R.~de 
Simone}. 2003. The synchronous languages 12~years later. \textit{Proc. IEEE}  91(1):64--83.
\bibitem{6-kov-1}
\Aue{Vasil'ev, S.\,N.} 2008. Formalizatsiya znaniy i upravlenie na osnove pozitivno-obrazovannykh 
yazykov [Formalization of knowledge and control on the basis of positively-formed languages]. 
\textit{Informatsionnye Tekhnologii i~Vychislitel'nye Sistemy} [Information Technologies and 
Computer Systems] 1:3--17.
\bibitem{7-kov-1}
\Aue{Jouault, F., B.~Vanhooff, H.~Bruneliere, G.~Doux, Y.~Berbers, and J.~Bezivin}. 2010.  
Inter-DSL coordination support by combining megamodeling and model weaving. \textit{ACM 
2010 Symposium Applied Computing Proceedings}. Sierre.  2011--2018.
\bibitem{8-kov-1}
The Eclipse Foundation. 2014. Eclipse Modeling Project.  Available at: {\sf 
http://www.eclipse.org/modeling/} (accessed March~4, 2016).
\bibitem{9-kov-1}
\Aue{Kolovos, D.\,S., L.\,M.~Rose, N.~Matragkas, R.\,F.~Paige, E.~Guerra, J.\,S.~Cuadrado, J.~De 
Lara, I.~R$\acute{\mbox{a}}$th, D.~Varr$\acute{\mbox{o}}$, M.~Tisi, and J.~Cabot}. 2013. 
A~research roadmap towards achieving scalability in model driven engineering. \textit{Workshop 
on Scalability in Model Driven Engineering Proceedings}. Budapest, Hungary: ACM.  2:1--2:10.
\bibitem{10-kov-1}
\Aue{Diskin, Z., and T.\,S.\,E.~Maibaum}. 2012. Category theory and model-driven engineering: 
From formal semantics to design patterns and beyond. \textit{7th Workshop ACCAT'2012 
``Electronic Proceedings in Theoretical Computer Science.''} 93:1--21.
\bibitem{11-kov-1}
\Aue{Mac Lane, S.} 1978. \textit{Categories for the working mathematician}. Springer. 317~p.
\bibitem{12-kov-1}
\Aue{Goguen, J.} 1973. Categorical foundations for general systems theory. \textit{Advances in 
cybernetics and systems research}. London: Transcripta Books. 121--130.
\bibitem{13-kov-1}
\Aue{Kovalyov, S.\,P.} 2014.  \textit{Teoretiko-kategornyy podkhod k~metaprogrammirovaniyu} 
[Category-theoretical approach to metaprogramming]. Moscow: Institute of Control Sciences. 
112~p.
\bibitem{14-kov-1}
\Aue{Kovalev, S.\,P.} 2013. Systems analysis of life cycle of large-scale information-control 
systems. \textit{Automation Remote Control} 74(9):1510--1524.
\bibitem{15-kov-1}
\Aue{Sannella, D.} 1993. A~survey of formal software development methods. \textit{Software 
engineering: A~European prospective}. IEEE Computer Society Press.  281--297.
\bibitem{16-kov-1}
\textit{Grid computing: Making the global infrastructure a~reality}. 2003. New York, NY: 
Wiley\,\&\,Sons. 1060~p.
\bibitem{17-kov-1}
\Aue{Kiczales, G., J.~Lamping, A.~Mendhekar, C.~Maeda, C.\,V.~Lopes, J.-M.~Loingtier, and 
J.~Irwin}. 1997. Aspect-oriented programming. 
\textit{ECOOP'97~--- object-oriented programming}.
Eds. M.~Aksit and S.~Matsuoka. Lecture notes in computer science ser.
Springer. 1241:220--242.
\bibitem{18-kov-1}
\Aue{Colyer, A., A.~Clement, G.~Harley, and M.~Webster}. 2004. \textit{Eclipse AspectJ:  
Aspect-oriented programming with \mbox{AspectJ} and the Eclipse AspectJ development tools}. Reading: 
Addison-Wesley. 504~p.
\bibitem{19-kov-1}
\Aue{Steimann, F.} 2006. The paradoxical success of aspect-oriented programming. 
\textit{Conference (International)  \mbox{OOPSLA'2006} Proceedings}. Portland.  481--497.
\bibitem{20-kov-1}
\Aue{Kovalyov, S.\,P.} 2013.  Semantika aspektno-orientirovannogo modelirovaniya dannykh 
i~protsessov [Semantics of aspect-oriented modeling of data and processes]. \textit{Informatika 
i~ee Primeneniya}~--- \textit{Inform. Appl.}   7(3):70--80.
\bibitem{21-kov-1}
\Aue{Kannenberg, A., and H.~Saiedian}. 2009. Why software requirements traceability remains 
a~challenge. \textit{J.~Defense Software Engineering}. July/August:14--19.
\bibitem{22-kov-1}
\Aue{Grebenjuk, G.\,G., N.\,V.~Lubkov, and S.\,M.~Nikishov}. 2011. \textit{Informatsionnye 
aspekty upravleniya munitsipal'nym hozyaystvom} [Informational aspects of municipal 
infrastructure management]. Moscow: LENAND. 320~p.
\bibitem{23-kov-1}
\Aue{Manset, D., H.~Verjus, R.~McClatchey, and F.~Oquendo}. 2006. A~formal architecture-
centric model-driven approach for the automatic generation of Grid applications. \textit{8th  
Conference (International) ``Enterprise Information Systems: Databases and Information Systems 
Integration'' Proceedings}. Paphos, Cyprus. 322--330.
\bibitem{24-kov-1}
\Aue{Uslar, M., T.~Schmedes, A.~Lucks, T.~Luhmann, L.~Winkels, and H.-J.~Appelrath}.  2005. 
Interaction of EMS related systems by using the CIM standard. \textit{2nd ICSC Symposium 
(International) on Information Technologies in Environmental Engineering ITEE 2005 Proceedings}. 
Magdeburg. 596--610.
\bibitem{25-kov-1}
\Aue{Voevodin, V.\,V.} 2000. Otobrazhenie problem vychislitel'noy matematiki na arkhitekturu 
vychislitel'nykh sistem [Mapping of computational mathematics problems to computational systems 
architecture]. \textit{Vychislitel'nye Metody i~Programmirovanie} [Numerical Methods and 
Programming]  1:37--44.
\bibitem{26-kov-1}
\Aue{Kovalyov, S.\,P.} 2007. Algebraicheskiy podkhod k~proektirovaniyu raspredelennykh 
vychislitel'nykh sistem [Algebraic approach to distributed computation systems design]. 
\textit{Sibirskiy Zh. Industrial'noy Matematiki} [Siberian J.~Industrial Mathematics]  
10(2):70--84.

\columnbreak

\bibitem{27-kol-1}
\Aue{Toporkov, V.\,V.} 2004.  \textit{Modeli raspredelennykh vychisleniy} [Models of distributed 
computing]. Moscow: Fizmatlit. 320~p.
\bibitem{28-kov-1}
\Aue{Kovalyov, S.\,P.} 2014. Povyshenie effektivnosti pro\-tses\-sov proektirovaniya bol'shikh 
informatsionno-up\-rav\-lya\-yushchikh sistem [Increasing efficiency of large-scale information-control 
systems design]. \textit{Tr. XII Vseross. Soveshchaniya po Problemam Upravleniya VSPU-2014} 
[XII All-Russian Workshop on Control Problems VSPU-2014 Proceedings]. Moscow: Institute of 
Control Sciences.  9291--9300.
\bibitem{29-kov-1}
\Aue{Kovalyov, S.\,P., S.\,K.~Andrjushkevich, and A.\,E.~Gus'kov}. 
June~26, 2009. Integratsionnaya 
platforma ucheta i~upravleniya energoobespecheniem ``Energius'' [Integration platform for energy 
accounting and management ``Energius'']. Certificate of state registration of computer program 
No.\,2009613359. 
\bibitem{30-kov-1}
\Aue{Sommerville, I.} 2001. \textit{Software engineering}. 6th ed. Addison Wesley. 720~p.
\bibitem{31-kov-1}
\Aue{Allen, R.\,J., and D.~Garlan}. 1997. A~formal basis for architectural connection. \textit{ACM 
Trans. Software Engineering Methodology} 6(3):213--249.
\bibitem{32-kov-1}
\Aue{Smith, D.\,R.} 2006.  Composition by colimit and formal software development. 
\textit{Algebra, meaning, and computation}.
Eds. K.~Futatsugi, J.-P.~Jouannaud, and J.~Meseguer.
Lecture notes  in computer science ser. Springer.  4060:317--332.
\bibitem{33-kov-1}
 \textit{Aspect-oriented software development}. 2004. Reading: Addison Wesley. 800~p.
\bibitem{34-kov-1}
\Aue{Pinto, M., L.~Fuentes, and J.\,M.~Troya}. 2003. DAOP-ADL: An architecture description 
language for dynamic component and aspect-based development. 
\textit{Generative programming and component
engineering}. Eds. F.~Pfenning and Y.~Smaragkaris.~---
Lecture notes in computer science ser.  Springer. 2830:118--137.
\bibitem{35-kov-1}
\Aue{Srinivas, Y.\,V., and R.~J$\ddot{\mbox{u}}$llig}. 1995. SPECWARE: Formal support for 
composing software. 
\textit{Mathematics of program construstion}.
Ed.\ B.~M$\ddot{\mbox{o}}$ller.
Lecture notes in computer science ser.  Springer. 947:399--422.
\end{thebibliography}

 }
 }

\end{multicols}

\vspace*{-3pt}

\hfill{\small\textit{Received November 19, 2015}}

\Contrl

\noindent
\textbf{Kovalyov Sergey P.} (b.\ 1972)~---
Doctor of Science in physics and mathematics, leading scientist, Institute of Control 
Problems, Russian Academy of Sciences, 65~Profsoyuznaya Str., Moscow 117997, 
Russian Federation; kovalyov@nm.ru

\label{end\stat}


\renewcommand{\bibname}{\protect\rm Литература}