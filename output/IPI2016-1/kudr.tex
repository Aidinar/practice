\newcommand{\al}{a_\lambda}
\newcommand{\bll}{b_\lambda}
\newcommand{\am}{a_\mu}
\newcommand{\bmm}{b_\mu}
\newcommand{\fl}{f_\lambda}
\newcommand{\fm}{f_\mu}
%\newcommand{\Ik}{\mathbb{1}}

\def\stat{kudr}

\def\tit{БАЙЕСОВСКИЕ МОДЕЛИ МАССОВОГО  ОБСЛУЖИВАНИЯ И~НАДЕЖНОСТИ:
АПРИОРНЫЕ РАСПРЕДЕЛЕНИЯ С~КОМПАКТНЫМ НОСИТЕЛЕМ$^*$}

\def\titkol{Байесовские модели массового  обслуживания и~надежности:
априорные распределения с~компактным носителем}

\def\aut{А.\,А.~Кудрявцев$^1$}

\def\autkol{А.\,А.~Кудрявцев}

\titel{\tit}{\aut}{\autkol}{\titkol}

{\renewcommand{\thefootnote}{\fnsymbol{footnote}} \footnotetext[1]
{Исследование выполнено при поддержке Российского научного фонда (проект 14-11-00397).}}


\renewcommand{\thefootnote}{\arabic{footnote}}
\footnotetext[1]{Московский государственный университет им.~М.\,В.~Ломоносова, 
факультет вычислительной математики и~кибернетики; 
Институт проб\-лем информатики Федерального исследовательского центра 
<<Информатика и~управ\-ле\-ние>> Российской академии наук, nubigena@mail.ru}

\vspace*{-12pt}

\Abst{Данная работа является очередной в~серии статей, посвященных изучению 
байесовских моделей массового обслуживания и~надежности. В~работе приводятся 
соотношения для функции распределения и~плотности частного~$\rho$ независимых 
случайных величин, имеющих априорные распределения с~компактным носителем, 
которые интерпретируются как параметр, <<препятствующий>> функционированию 
системы, и~параметр, <<способствующий>> функционированию. Описание жизненного цикла 
многих реальных систем осуществляется в~терминах~$\rho$, например в~теории 
массового обслуживания~$\rho$ называется параметром загрузки системы и~входит 
во многие формулы, описывающие разнообразные характеристики. Рассматриваются 
частные случаи априорных распределений с~компактным носителем, для которых плотности 
имеют полиномиальный или ку\-соч\-но-по\-ли\-но\-ми\-аль\-ный вид.}


\KW{байесовский подход; системы массового обслуживания; надежность; смешанные
распределения; распределения с~компактным носителем}

\DOI{10.14357/19922264160106} %

\vspace*{-4pt}

\vskip 12pt plus 9pt minus 6pt

\thispagestyle{headings}

\begin{multicols}{2}

\label{st\stat}

\section{Введение}

\vspace*{-2pt}

Во многих областях исследования математических моделей функционирования 
реальных систем аналитические результаты, характеризующие жизненный цикл 
рассматриваемых объектов, так или иначе зависят от параметров, <<способ\-ст\-ву\-ющих>> 
функционированию системы и~<<препятствующих>> функционированию. Так, в~моделях 
массового об-\linebreak служивания к~параметрам, <<способствующим>> функционированию, можно 
отнести интенсивность обслуживания запросов, а~к~па\-ра\-мет\-рам, <<препятствующим>> 
функционированию,~--- интенсивность входящего потока требований. 

Аналогично 
в~теории надежности па\-ра\-метр <<эффективности>> средства, исправляющего ошибки 
в~системе, <<способствует>> функционированию, а~па\-ра\-метр <<дефективности>>~--- 
<<препятствует>>. Очевидно, что итоговые результаты работы системы зависят не 
столько от значений самих па\-ра\-мет\-ров, влияющих на функционирование, сколько от 
их отношения. В~общем случае такое отношение можно назвать 
{\it <<коэффициентом баланса системы>>}.

Хорошо известно, что одним из основных показателей при изучении моделей 
массового обслуживания $M|M|1$ является коэффициент загрузки системы~$\rho$, 
равный отношению параметра входящего потока~$\lambda$ к~параметру обслуживания~$\mu$. 
От значения~$\rho$ зависит наличие стационарного режима у~рассматриваемой системы; 
величина~$\rho$ входит во многие формулы, описывающие характеристики разнообразных 
систем массового обслуживания. По своей сути коэффициент загрузки является коэффициентом 
баланса, характеризующим систему: функционирование системы тем эффективнее, чем ближе 
к~нулю значения~$\rho$. При достаточно больших значениях~$\rho$ ($\rho\hm\ge1$) 
система работает столь неэффективно, что, например, среднее число заявок 
в~системе $M|M|1|\infty$ считается равным бесконеч\-ности.

В рекуррентных моделях роста надежности удобно рассматривать коэффициент 
баланса $\rho\hm=\lambda/\mu$, где~$\lambda$~--- параметра <<эффективности>> 
средства, исправляющего ошибки в~системе, а $\mu$~--- параметр 
<<дефективности>>. При этом, в~отличие от теории массового обслуживания, 
система тем надежнее, чем больше значение~$\rho$.

Байесовский подход к~задачам массового обслуживания и~надежности предполагает 
рандомизацию параметров~$\lambda$ и~$\mu$. 

Подробное описание предпосылок для 
исследования, особенностей и~биб\-ли\-о\-графии байесовских моделей в~теории массового 
обслуживания и~надежности можно найти в~книге~\cite{KuSh2015}. 
В~основе всех результатов для байесовских моделей из~\cite{KuSh2015} лежит 
вероятностное распределение коэффициента баланса~$\rho$. Принципиальное отличие 
байесовских постановок задач в~указанных теориях заключается в~том, что 
в~задачах теории надежности коэффициенты <<эффективности>> и~<<дефективности>> 
соответственно имеют смысл рандомизированных вероятностей исправления ошибки 
в~системе и~внесения новой ошибки, а следовательно, соответствующие вероятностные 
распределения должны быть подмножествами единичного отрезка. 
В~теории массового обслуживания для систем $M|M|1$ параметры входящего потока 
и~обслуживания должны быть положительными числами, поэтому соответствующие 
распределения имеют лишь нижнее ограничение в~нуле.

Далее приводятся результаты для распределения величины $\rho\hm=\lambda/\mu$ 
в~случае, когда носителями распределений~$\lambda$ и~$\mu$ являются отрезки 
на положительной полупрямой. При применении изложенных ниже результатов 
к~надежностным постановкам необходимо ограничивать правые концы носителей 
распределений единицей.

\vspace*{-9pt}

\section{Основные результаты}

Пусть $\lambda$ и~$\mu$~--- независимые абсолютно непрерывные случайные величины, 
причем  ${\sf P}(\lambda\hm\in[\al,\bll])\hm=1$, $0\hm<\al\hm<\bll$, 
и~не существует множества $S\hm\subset[\al,\bll]$ положительной меры Лебега такого, что 
${\sf P}(\lambda\hm\in S)\hm=0$, а для случайной величины~$\mu$ выполнены аналогичные 
требования с~параметрами~$\am$ и~$\bmm$. Плотности случайных величин~$\lambda$ 
и~$\mu$ обозначим через $\fl(x)$ и~$\fm(x)$ соответственно. Во 
всех последующих выкладках будем предполагать, что $x\hm>0$.

Найдем функцию распределения $F_\rho(x)$ и~плотность $f_\rho(x)$ случайной 
величины~$\rho\hm=\lambda/\mu$. Имеем

\noindent
\begin{multline*}
F_\rho(x)=\il{-\infty}{+\infty}{\sf P}(\lambda<xy)\, d{\sf P}(\mu<y)={}\\[-1pt]
{}=\il{\am}{\bmm} \left[\il{\al}{xy}\fl(u)\, du\cdot
\Ik\left(\al\le xy\le \bll\right)+{}\right.\\[-1pt]
\left.{}+\Ik\left(xy>\bll\right)
\vphantom{\il{\am}{\bmm}}
\right] \fm(y)\, dy\,.
\end{multline*}
Рассмотрим всевозможные комбинации расположения точек~$\am$, $\bmm$, 
$\al/x$ и~$\bll/x$ на прямой. При этом существенную роль играет взаимное 
расположение точек~$\al/\am$ и~$\bll/\bmm$. Так, при $\am\hm<\al/x\hm<\bmm\hm<\bll/x$ 
и~$\al/\am\hm<\bll/\bmm$

\noindent
$$
F_\rho(x)=\il{\al/x}{\bmm}\il{\al}{xy}\fl(u)\fm(y)\, dudy\,.
$$
Аналогично рассмотрев остальные случаи, убеждаемся в~справедливости следующего 
утверждения.

\smallskip

\noindent
\textbf{Теорема 1.} \textit{Пусть независимые абсолютно непрерывные случайные 
величины~$\lambda$ и~$\mu$ имеют соответственно носители 
распределений $[\al,\bll]$ и~$[\am,\bmm]$, $0\hm<\al\hm<\bll$, $0\hm<\am\hm<\bmm$, 
и~плот\-ности $\fl(x)$ и~$\fm(x)$. Тогда случайная величина 
$\rho\hm=\lambda/\mu$ имеет функцию распределения}:
\begin{multline*}
F_\rho(x)=\Ik\left(\fr{\al}{\bmm}<x\le 
\min\left\{\fr{\al}{\am},\fr{\bll}{\bmm}\right\}\right)\times{}\\
{}\times \il{\al/x}{\bmm}
\il{\al}{xy}\fl(u)\fm(y)\, dudy+\Ik\left(\fr{\al}{\am}<x\le 
\fr{\bll}{\bmm}\right)\times{}\\
{}\times \il{\am}{\bmm}\il{\al}{xy}\fl(u)\fm(y)\, dudy+
\Ik\left(\fr{\bll}{\bmm}<x\le \fr{\al}{\am}\right)\times{}\\
{}\times \left[\ \il{\al/x}{\bll/x}\il{\al}{xy}\fl(u)\fm(y)\, dudy +
\il{\bll/x}{\bmm}\fm(x)\, dy\right]+{}\\
{}+\Ik\left(\max\left\{\fr{\al}{\am},\fr{\bll}{\bmm}\right\}<x\le\fr{\bll}{\am}\right)\times{}\\
{}\times
\left[\ \il{\am}{\bll/x}\il{\al}{xy}\fl(u)\fm(y)\, dudy+\il{\bll/x}{\bmm}\fm(y)\, dy\right]+{}\\
{}+\Ik\left(x>\fr{\bll}{\am}\right).
\end{multline*}


%\smallskip

Для нахождения плотности случайной величины~$\rho$ достаточно воспользоваться 
соотношением
$$
f_\rho(x)=\il{-\infty}{+\infty}{y}\fl(xy)\fm(y)\, dy
$$
и рассуждениями, приведенными выше, для функции распределения~$F_\rho(x)$.

\smallskip

\noindent
\textbf{Теорема 2.}\ \textit{Пусть независимые абсолютно непрерывные случайные 
величины~$\lambda$ и~$\mu$ имеют соответственно носители распределений 
$[\al,\bll]$ и~$[\am,\bmm]$, $0\hm<\al\hm<\bll$, $0\hm<\am\hm<\bmm$, 
и~плот\-ности $\fl(x)$ и~$\fm(x)$. Тогда случайная величина 
$\rho\hm=\lambda/\mu$ имеет плотность распределения}

\noindent
\begin{multline*}
f_\rho(x)=\Ik\left(\fr{\al}{\bmm}<x\le\min\left\{\frac{\al}{\am},\fr{\bll}{\bmm}\right\}\right)\times{}\\
{}\times\il{\al/x}{\bmm}
{y}\fl(xy)\fm(y)\, dy+{}
\end{multline*}

\noindent
\begin{multline*}
{}+\Ik\left(\fr{\al}{\am}<x\le\fr{\bll}{\bmm}\right)\il{\am}{\bmm}{y}\fl(xy)\fm(y)\, dy+{}\\
{}+\Ik\left(\fr{\bll}{\bmm}<x\le\fr{\al}{\am}\right)\il{\al/x}{\bll/x}{y}\fl(xy)\fm(y)\, dy +{}\\
{}+\Ik\left(\max\left\{\fr{\al}{\am},\fr{\bll}{\bmm}\right\}<x\le
\fr{\bll}{\am}\right)\!\!\il{\am}{\bll/x}\!\!{y}\fl(xy)\fm(y)\, dy.\hspace*{-5.39891pt}
\end{multline*}


\noindent
\textbf{Замечание~1.}\ Применительно к~байесовским моделям массового 
обслуживания $M|M|1$ теоремы~1 и~2 позволяют не только находить распределение 
коэффициента загрузки~$\rho$, но и~распределения таких характеристик, 
как вероятность <<непотери>> вызова 
$\pi\hm=1/(1\hm+\rho)$ в~модели $M|M|1|0$ или среднее чис\-ло требований 
в~очереди $N\hm=\rho/(1\hm-\rho)$ в~модели $M|M|1|\infty$ и~др. Приведенные 
теоремы также упрощают поиск средней надежности системы, поскольку 
для рекуррентных моделей байесовской тео\-рии надежности априорные 
распределения всегда ограничены и~являются подмножествами единичного отрезка.


\smallskip

\noindent
\textbf{Замечание~2.}\ В~формулировке теоремы~1 вместо плотности случайной 
величины~$\lambda$ можно использовать ее функцию распределения, учитывая
$$
\il{\al}{xy}\fl(u)\,du=F_\lambda(xy)\,,\enskip y\in\left[\fr{\al}{x},\fr{\bll}{x}\right]\,.
$$




\noindent
\textbf{Замечание~3.}\ 
В~формулировках теорем~1 и~2 один из индикаторов 
$\Ik\left({\al}/{\am}<x\le{\bll}/{\bmm}\right)$ 
и~$\Ik\left({\bll}/{\bmm}<x\le{\al}/{\am}\right)$ всегда равен нулю 
в~зависимости от взаимного расположения точек~$\al/\am$ и~$\bll/\bmm$.

\smallskip


Отдельный интерес в~классе распределений с~компактным носителем 
(для некоторой случайной величины~$\xi$) представляют распределения, 
плотности которых могут быть представлены в~виде полинома:
\begin{equation}\label{Density_Polynomial}
f_\xi(x)=\sum\limits_{i=0}^{n_\xi}c_{\xi,i}\, x^i\cdot\Ik(x\in[a_\xi,b_\xi])\,.
\end{equation}
К таким распределениям, в~частности, относятся равномерное (при $n_\xi\hm=0$) 
и~параболическое (при $n_\xi\hm=2$) распределения.

Пусть случайные величины~$\lambda$ и~$\mu$ удовлетворяют условиям теоремы~2, 
а их плотности~--- соотношению~(\ref{Density_Polynomial}) 
с~соответствующими параметрами. Для некоторых~$a$ и~$b$, одновременно 
принадлежащих отрезкам $[\am,\bmm]$ и~$[\al/x,\bll/x]$, определим

\noindent
\begin{multline}
I(a,b,x)=\il{a}{b}{y}\fl(xy)\fm(y)\, dy={}\\
{}=
\il{a}{b}\sum\limits_{i=0}^{n_\lambda}\sum\limits_{j=0}^{n_\mu}
c_{\lambda,i}c_{\mu,j}\, x^{i}y^{i+j+1}\, dy={}\\
{}=\sum\limits_{i=0}^{n_\lambda}\sum\limits_{j=0}^{n_\mu}c_{\lambda,i}c_{\mu,j}
\fr{b^{i+j+2}-a^{i+j+2}}{i+j+2}\, x^i.
\label{Integral_General}
\end{multline}

Теорема~2 дает возможность сформулировать следующее утверждение для 
распределений с~плотностями, имеющими полиномиальный вид.

\smallskip

\noindent
\textbf{Следствие~1.}\ Пусть распределения случайных величин~$\lambda$ и~$\mu$ 
удовлетворяют условиям теоремы~2 и~соотношению~(\ref{Density_Polynomial}) 
с~соответствующими пара\-мет\-рами. Тогда случайная величина $\rho\hm=\lambda/\mu$ 
имеет плотность:
\begin{multline}
f_\rho(x)={}\\
{}=\Ik\left(\fr{\al}{\bmm}<x\le\min\left\{\fr{\al}{\am},\fr{\bll}{\bmm}\right\}
\right)I\left(\fr{\al}{x},\bmm,x\right)+{}\\
{}+\Ik\left(\fr{\al}{\am}<x\le\fr{\bll}{\bmm}\right)I\left(\am,\bmm,x\right)+{}\\
{}+
\Ik\left(\fr{\bll}{\bmm}<x\le\fr{\al}{\am}\right)I\left(
\fr{\al}{x},\fr{\bll}{x},x\right)+{}\\
\!{}+\Ik\left(\max\left\{\fr{\al}{\am},\fr{\bll}{\bmm}\right\}<x\le\fr{\bll}{\am}\right)
I\left(\am,\fr{\bll}{x},x\right),\!\!
\label{Density_rho_Polynomial_case}
\end{multline}
где величины $I(a,b,x)$ определены соотношением~(\ref{Integral_General}).

\smallskip


Следствие~1 дает возможность вычислить моменты
$$
{\sf E}\rho^k=\int x^k f_\rho(x)\,dx\,, \enskip k=1,2,\ldots\,,
$$
для случая, когда $f_\rho(x)$ имеет вид~(\ref{Density_rho_Polynomial_case}). 
Будем полагать для определенности, что $c_{\mu,j}\hm=0$ при $j\hm>n_\mu$. 
Введем обозначение:
\begin{multline}
J(d)=\sum\limits_{i=0}^{n_\lambda}\sum\limits_{j=0}^{n_\mu}
\fr{c_{\lambda,i}c_{\mu,j}d^{i+j+2}}{i+j+2}\il{d/\bmm}{d/\am}x^{k-j-2}\, dx={}\\
{}=
\sum\limits_{i=0}^{n_\lambda}\fr{c_{\lambda,i}c_{\mu,k-1}d^{k+i+1}}{k+i+1}\,
\ln\fr{\bmm}{\am}+{}\\
{}+\sum\limits_{i=0}^{n_\lambda}\sum\limits_{j=0}^{n_\mu}\Ik(j\neq k-1)\times{}\\
{}\times
\fr{c_{\lambda,i}c_{\mu,j}(\bmm^{k-j-1}-\am^{k-j-1})d^{k+i+1}}
{(i+j+2)(k-j-1)\am^{k-j-1}\bmm^{k-j-1}}.
\label{Integral_for_moment}
\end{multline}
Непосредственно из следствия~1 вытекает сле\-ду\-ющее утверждение.

\smallskip

\noindent
\textbf{Следствие~2.}\ Пусть распределения случайных величин~$\lambda$ и~$\mu$ 
удовлетворяют условиям теоремы~2 и~соотношению~(\ref{Density_Polynomial}) 
с~соответствующими пара\-мет\-рами. Тогда моменты $k$-го порядка ($k\hm=1,2,\ldots$) 
случайной величины $\rho\hm=\lambda/\mu$ имеют вид:
\begin{multline*}
{\sf E}\rho^k={}\\
{}=\sum\limits_{i=0}^{n_\lambda}\sum\limits_{j=0}^{n_\mu}
c_{\lambda,i}c_{\mu,j}\left(\bll^{k+i+1}-\al^{k+i+1}\right)
\left(\am^{k-j-1}-{}\right.\\
\left.{}-\bmm^{k-j-1}\right)\!\Bigg/ \!
\left((i+j+2)(k+i+1)\am^{k-j-1}\bmm^{k-j-1}\right)+{}\\
{}+J(\bll)-J(\al)\,,
\end{multline*}
где величины $J(d)$ определены соотношением~(\ref{Integral_for_moment}).


\smallskip

\noindent
\textbf{Замечание~4.}\ Следствия~1 и~2 из теоремы~2 обобщают некоторые 
полученные ранее результаты (см., например,~\cite{KuSh2015}).

\smallskip


\noindent
\textbf{Замечание~5.}\ На величину моментов ${\sf E}\rho^k$ не 
влияет взаимное расположение точек~$\al/\am$ и~$\bll/\bmm$.

\smallskip


Изложенный выше метод дает возможность находить характеристики 
распределения частного случайных величин, плотности которых имеют 
полиномиальный вид не на всем носителе, а на некоторых его подмножествах. 
Рассмотрим следующее разбиение отрезка $[a_\xi,b_\xi]$:
$$
0<a_\xi=a_\xi^{(1)}<b_\xi^{(1)}=a_\xi^{(2)}<\cdots<b_\xi^{(l_\xi)}=b_\xi\,.
$$
Пусть плотность некоторой случайной величины~$\xi$, носителем распределения 
которой является отрезок $[a_\xi,b_\xi]$, имеет ку\-соч\-но-по\-ли\-но\-ми\-аль\-ный вид:
\begin{equation}
f_\xi(x)=\sum\limits_{l=1}^{l_\xi}\sum\limits_{i=0}^{n_\xi}c_{\xi,i,l}\, 
x^i\cdot\Ik\left(x\in\left[a_\xi^{(l)},b_\xi^{(l)}\right]\right).
\label{Density_Piecewise_Polynomial}
\end{equation}
Частными случаями таких распределений являются распределение 
Симпсона (при $l_\xi\hm=2$ и~$n_\xi\hm=1$) и~трапецеидальное распределение 
(при $l_\xi\hm=3$ и~$n_\xi\hm=1$).

\smallskip

\noindent
\textbf{Замечание~6.} Для распределений случайных величин~$\lambda$ и~$\mu$, 
удовлетворяющих условиям теоремы~2 и~соотношению~(\ref{Density_Piecewise_Polynomial}) 
с~соответствующими параметрами, при каждом конкретном наборе параметров~$n_\lambda$, 
$n_\mu$, $l_\lambda$ и~$l_\mu$ несложно сформулировать утверждения, 
аналогичные следствиям из теоремы~2, в~которых величины $I(a,b,x)$ будут 
определяться соотношением:
\begin{multline*}
\!\!\!I(a,b,x)=\il{a}{b}\sum\limits_{l=1}^{l_\lambda}\sum\limits_{i=0}^{n_\lambda}
\sum\limits_{m=1}^{l_\mu}\sum\limits_{j=0}^{n_\mu}c_{\lambda,i,l}
c_{\mu,j,m}\, x^iy^{i+j+1}\times{}\ \\
{}\times\Ik\left(y\in\left[a_\lambda^{(l)}/x,b_\lambda^{(l)}/x\right]\right)
\Ik\left(y\in\left[a_\mu^{(m)},b_\mu^{(m)}
\right]\right)\, dy.
\end{multline*}

{\small\frenchspacing
 {%\baselineskip=10.8pt
 \addcontentsline{toc}{section}{References}
 \begin{thebibliography}{9}

\bibitem{KuSh2015}
\Au{Кудрявцев А.\,А., Шоргин С.\,Я.}
Байесовские модели в~тео\-рии массового обслуживания и~надежности.~--- 
М.: ФИЦ ИУ РАН, 2015. 76~с.

\end{thebibliography}

 }
 }

\end{multicols}

\vspace*{-3pt}

\hfill{\small\textit{Поступила в~редакцию 17.01.16}}

\vspace*{8pt}

%\newpage

%\vspace*{-24pt}

\hrule

\vspace*{2pt}

\hrule

\vspace*{8pt}



\def\tit{BAYESIAN QUEUEING AND~RELIABILITY MODELS:\\
\textit{A~PRIORI} DISTRIBUTIONS WITH~COMPACT SUPPORT}

\def\titkol{Bayesian queueing and reliability models: \textit{A~priori} distributions with compact support}

\def\aut{A.\,A.~Kudryavtsev$^{1,2}$}

\def\autkol{A.\,A.~Kudryavtsev}

\titel{\tit}{\aut}{\autkol}{\titkol}

\vspace*{-9pt}

\noindent
$^1$Department of Mathematical Statistics, Faculty of Computational Mathematics 
and Cybernetics,\linebreak
$\hphantom{^1}$M.\,V.~Lomonosov Moscow State University, 1-52~Leninskiye Gory, GSP-1, 
Moscow 119991, Russian\linebreak
$\hphantom{^1}$Federation

\noindent
$^2$Institute of Informatics Problems, Federal Research Center 
``Computer Science and Control'' of the Russian\linebreak
 $\hphantom{^1}$Academy of Sciences, 
44-2~Vavilov Str., Moscow 119333, Russian Federation

\def\leftfootline{\small{\textbf{\thepage}
\hfill INFORMATIKA I EE PRIMENENIYA~--- INFORMATICS AND
APPLICATIONS\ \ \ 2016\ \ \ volume~10\ \ \ issue\ 1}
}%
 \def\rightfootline{\small{INFORMATIKA I EE PRIMENENIYA~---
INFORMATICS AND APPLICATIONS\ \ \ 2016\ \ \ volume~10\ \ \ issue\ 1
\hfill \textbf{\thepage}}}

\vspace*{3pt}

\Abste{This work is the latest in a series of articles devoted to the 
study of Bayesian queueing and reliability models. The paper presents 
relations for the distribution function and the density of the quotient~$\rho$ of 
independent random variables with \textit{a priori} distributions with compact support, 
which are interpreted as a~parameter  ``obstructing'' the functioning of the 
system and a~parameter ``conducing'' to the functioning of the system. Description 
of the life cycle of many real systems is carried out in terms of~$\rho$; for example, 
in the queueing theory, parameter~$\rho$ is called\linebreak\vspace*{-12pt}}

\Abstend{the ``system load factor'' 
and is a~part of many formulas that describe various characteristics. The paper 
considers particular cases of \textit{a~priori} distributions with compact 
support for which densities have polynomial or piecewise polynomial form.}

\KWE{Bayesian approach; mass service theory; reliability theory; mixed distributions; 
distributions with compact support}

\DOI{10.14357/19922264160106} 

\Ack
\noindent
This work was financially supported by the Russian Science Foundation 
(grant No.\,14-11-00397).



%\vspace*{3pt}

  \begin{multicols}{2}

\renewcommand{\bibname}{\protect\rmfamily References}
%\renewcommand{\bibname}{\large\protect\rm References}

{\small\frenchspacing
 {%\baselineskip=10.8pt
 \addcontentsline{toc}{section}{References}
 \begin{thebibliography}{9}

\bibitem{1-kudr}
\Aue{Kudryavtsev, A.\,A., and S.\,Ya.~Shorgin}. 
2015. \textit{Bayesovskie modeli v~teorii massovogo obsluzhivaniya i~nadezhnosti} 
[Bayesian models in mass service and reliability theories]. 
Moscow: Federal Research Center ``Computer Science and Control'' of the Russian
Academy of Sciences, 2015.\linebreak 76~p.

\end{thebibliography}

 }
 }

\end{multicols}

\vspace*{-3pt}

\hfill{\small\textit{Received January 17, 2016}}

\Contrl

\noindent
\textbf{Kudryavtsev Alexey A.} (b.\ 1978)~---
Candidate of Sciences (PhD) in physics and mathematics, associate professor, 
Department of Mathematical Statistics, Faculty of Computational Mathematics 
and Cybernetics, M.\,V.~Lomonosov Moscow State University, 1-52~Leninskiye Gory, 
GSP-1, Moscow 119991, Russian Federation; Institute of Informatics Problems, 
Federal Research Center ``Computer Science and Control'' 
of the Russian Academy of Sciences, 44-2~Vavilov Str., Moscow 119333, 
Russian Federation; nubigena@mail.ru
\label{end\stat}


\renewcommand{\bibname}{\protect\rm Литература}