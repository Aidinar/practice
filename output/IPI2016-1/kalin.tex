\def\stat{kalinichenko}

\def\tit{ПРОБЛЕМЫ ДОСТУПА К ДАННЫМ В~ИССЛЕДОВАНИЯХ 
С~ИНТЕНСИВНЫМ ИСПОЛЬЗОВАНИЕМ ДАННЫХ В~РОССИИ$^*$}

\def\titkol{Проблемы доступа к~данным в~исследованиях с~интенсивным использованием данных в~России}

\def\aut{Л.\,А.~Калиниченко$^1$, А.\,А.~Вольнова$^2$, Е.\,П.~Гордов$^3$, 
Н.\,Н.~Киселева$^4$, Д.\,А.~Ковалева$^5$, 
О.\,Ю.~Малков$^6$, И.\,Г.~Окладников$^7$, Н.\,Л.~Подколодный$^8$, 
А.\,С.~Позаненко$^9$,\\ 
Н.\,В.~Пономарева$^{10}$, С.\,А.~Ступников$^{11}$, А.\,З.~Фазлиев$^{12}$}

\def\autkol{Л.\,А.~Калиниченко, А.\,А.~Вольнова, Е.\,П.~Гордов и~др.}
%$^3$,  Н.\,Н.~Киселева$^4$, Д.\,А.~Ковалева$^5$, 
%О.\,Ю.~Малков$^6$, И.\,Г.~Окладников$^7$, Н.\,Л.~Подколодный$^8$,  А.\,С.~Позаненко$^9$, 
%Н.\,В.~Пономарева$^{10}$, С.\,А.~Ступников$^{11}$, А.\,З.~Фазлиев $^{12}$}

\titel{\tit}{\aut}{\autkol}{\titkol}

{\renewcommand{\thefootnote}{\fnsymbol{footnote}} \footnotetext[1]
{Подготовка настоящего обзора была частично поддержана различными грантами, 
полученными группами из вовлеченных в~эту работу исследовательских организаций: для 
ИПИ ФИЦ ИУ РАН грантами 13-07-00579, 14-07-00548 и~16-07-01028; для ИМКЭС СО 
РАН грантами РФФИ 13-05-12034 и~14-05-00502; для ИМЕТ РАН грантами РФФИ  
14-07-00819 и~15-07-00980; для ИНАСАН РАН грантом РФФИ 15-02-04053 и~грантом 
Президиума РАН по программе П-41; для ИЦиГ СО РАН грантом РНФ 14-24-00123; для 
ИКИ РАН грантом РФФИ 15-02-10203-К; для НЦН грантами РФФИ 15-04-08744  
и~15-04-05066; для ИОА СО РАН грантом РФФИ 13-07-00411.}}


\renewcommand{\thefootnote}{\arabic{footnote}}
\footnotetext[1]{Институт проб\-лем информатики Федерального исследовательского центра 
<<Информатика и~управ\-ле\-ние>> Российской академии наук;
факультет вычислительной математики и~кибернетики Московского
государственного университета им.\ М.\,В.~Ломоносова, leonidandk@gmail.com}
\footnotetext[2]{Институт космических исследований Российской академии наук, alinusss@gmail.com}
\footnotetext[3]{Международный исследовательский центр климатоэкологических
исследований Института мониторинга климатических и~экологических систем Сибирского
отделения Российской академии наук, gordov@scert.ru}
\footnotetext[4]{Институт металлургии и~материаловедения им.\ А.\,А.~Байкова
Российской академии наук, kis@imet.ac.ru}
\footnotetext[5]{Институт астрономии Российской академии наук, dana@inasan.ru}
\footnotetext[6]{Институт астрономии Российской академии наук, malkov@inasan.ru}
\footnotetext[7]{Международный исследовательский центр климатоэкологических
исследований Института мониторинга климатических и~экологических систем Сибирского
отделения Российской академии наук, igor.okladnikov@gmail.com}
\footnotetext[8]{Центр коллективного пользования <<Биоинформатика>> Федерального
исследовательского центра Институт цитологии и~генетики Сибирского
отделения  Российской академии наук, pnl@bionet.nsc.ru}
\footnotetext[9]{Институт космических исследований Российской академии наук, apozanen@iki.rssi.ru}
\footnotetext[10]{Научный центр неврологии, ponomare@yandex.ru}
\footnotetext[11]{Институт проб\-лем информатики Федерального исследовательского центра 
<<Информатика и~управ\-ле\-ние>> Российской академии наук, sstupnikov@ipiran.ru}
\footnotetext[12]{Центр интегрированных информационных систем Института оптики атмосферы Сибирского
отделения  Российской академии наук, faz@iao.ru}

\vspace*{-12pt}


\Abst{Целью данного обзора является анализ глобальных тенденций создания массивных 
коллекций данных в~мире и~обеспечения возможности совместного использования таких 
коллекций при решении задач исследования и~принятия решений в~различных областях 
с~интенсивным использованием данных (ОИИД) в~России. Конкретный набор ОИИД, 
отобранный для обзора, включает астрономию, материаловедение, науки о~Земле, геномику 
и протеомику, нейронауку. По каждой из рассмотренных ОИИД представлены крупные 
стратегические инициативы США и~ЕС, примеры крупных коллекций данных в~мире до 
2025~г., известные проекты информационных и~телекоммуникационных 
инфраструктур и~центров данных. Включенный в~обзор 
набор массивных коллекций данных, планируемых к~получению в~мире, предлагается 
использовать в~качестве ориентира при планировании и~развитии исследовательских 
инфраструктур для накопления и~анализа данных, совместимых с~зарубежными открытыми 
инфраструктурами в~науке. В~частности, рассматриваемые в~обзоре коллекции данных, 
цели их создания и~научные исследования, планируемые к~осуществлению с~их помощью, 
позволяют перейти к~постановке и~решению задач создания компонентов перспективных 
информационных и~телекоммуникационных инфраструктур, таких как, например, средства концептуализации ОИИД, 
необходимые метамодели, средства обеспечения возможности повторного использования 
коллекций данных, воспроизводимости программ и~потоков работ и~др.}

\KW{4-я парадигма; области с~интенсивным использованием данных; исследовательские 
инфраструктуры; коллекции данных; большие данные}

\DOI{10.14357/19922264160101} %

\vspace*{-2pt}

\vskip 10pt plus 9pt minus 6pt

\thispagestyle{headings}

\begin{multicols}{2}

\label{st\stat}

\section{Введение}

\vspace*{-2pt}

  Исследования и~принятие решений в~различных областях деятельности людей 
реализуются на основе анализа данных, накопленных в~соответствующих областях, объем 
и разнообразие которых в~наши дни растут экспоненциально.
  
  В соответствии с~4-й парадигмой научных исследований~[1], проведение исследований, 
движимых данными, становится неотъемлемой частью различных областей науки, 
экономики, бизнеса (далее~--- областей с~интенсивным использованием данных~--- ОИИД, 
или data intensive domains~--- DID). Без обеспечения все новыми данными, являющимися 
результатом наблюдений, измерений в~природе и~обществе, развитие исследований 
в~различных ОИИД становится немыслимым. 

По мере развития ОИИД извлечение данных 
из природы становится все более сложным и~изощренным из-за необходимости 
проникновения во все более масштабные микро-, мезо- и~макроявления. Организуются 
глобальные проекты и~миссии (в~том чис\-ле космические) по сбору и~накоплению данных 
при помощи специализированных новейших высокотехнологичных инструментов, 
размещаемых не только на Земле, но и~в космосе. Получение данных становится все более 
сложным, дорогостоящим делом, требующим развития специальных технологий 
и~серьезных капиталовложений. В~результате удается получать сырые данные, 
подлежащие дальнейшей обработке и~анализу. Сам процесс сбора данных при изучении 
определенного вида явлений в~конкретной ОИИД может занимать многие годы. 

Наряду со 
сбором данных беспрецедентно быстро развиваются методы и~средства накопления, 
обработки, анализа и~управ\-ле\-ния накапливаемыми данными в~разнообразных ОИИД, 
происходит быстрое расширение спектра задач, требующих решения на основе 
полученных данных, накопление опыта решения подобных задач и~обеспечение 
возможности его междисциплинарного использования. 
  
  Главный побудительный мотив настоящей инициативной работы\footnote{Данная статья 
является расширенным русскоязычным вариантом работы \Au{Kalinichenko~L., Fazliev~A., 
Gordov~E., Kiselyova~N., Kovaleva~D., 
Malkov~O., Okladnikov~I., Podkolodny~N., Ponomareva~N., Pozanenko~A., 
Stupnikov~S., Volnova~A.} New data access challenges for data 
intensive research in Russia~// 17th Conference (International) on Data Analytics 
and Management in Data Intensive Domains Proceedings, 2015. P.~215--237.} заключается в~необходимости положить начало 
систематическому анализу развития массивных коллекций данных в~различных ОИИД 
в~мире, создания и~развития инфраструктур для накопления и~использования больших 
коллекций данных, систематизации опыта решения задач в~ОИИД и~пр. Некоторыми 
прагматическими целями такого анализа являются выявление технических, правовых 
и~финансовых проблем на пути обеспечения доступа ученых России в~различных ОИИД 
к~уже накопленным и~ожидаемым коллекциям данных в~мире\footnote{Следует заметить, что на этом 
пути из-за высокой технологической сложности и~стоимости средств извлечения данных во многих ОИИД, высокой стоимости самого 
процесса получения конкретных коллекций данных в~течение ближайших 10~лет ни о~каком <<импортозамещении>> не может быть 
и~речи.}, определение потребности создания специальных инфраструктур технических 
и~программных средств в~России для поддержки такой возможности, а~так\-же способности 
России заметным образом участвовать во вкладе в~мировую сокровищницу данных, 
в~создание соответствующих инфраструктур, методов и~средств решения задач анализа 
данных.
  
  Предварительный анализ показывает, что западный мир весьма озабочен проблемами, 
по\-рож\-ден\-ны\-ми все возрастающим <<наводнением>> \mbox{ОИИД} большими данными, 
проблемами их анализа (включая анализ публикаций в~виде текстов на естественном языке 
как час\-ти когнитивного процесса), накопления и~совместного (в~том чис\-ле 
междисциплинарного) использования данных и~опыта решения задач, планированием 
специальных инфраструктур, позволяющих справиться с~подобным\linebreak наводнением по мере 
ввода в~действие новых инструментов получения данных. Для этого органи\-зуются 
крупные совместные проекты, рабочие\linebreak группы, обсуждаются возможные решения, 
планируются новые инфраструктуры и~уже тестируются их фрагменты, ориентированные 
на получение конкретных коллекций данных после 2020~г., создаются методы и~средства 
поддержки таких инфраструктур, отрабатываются характерные примеры будущих задач, 
проводятся конференции, специализи\-рованные симпозиумы рабочих групп и~пр.
  
  Вместе с~тем в~ряде ОИИД в~России обстановка такова, что если своевременно не 
позаботиться о~том, чтобы иметь эффективный доступ к~данным (наиболее важные из 
которых, увы, собираются и~накапливаются за пределами России), то исследования по 
ряду направлений во многих ОИИД можно просто прекратить.
  
  Все это служит мотивацией для проведения анализа глобальных тенденций создания 
массивных коллекций данных в~мире и~обеспечения возможности совместного 
использования таких коллекций при решении задач исследования и~принятия решений 
в~различных ОИИД в~России. Анализ проведен на выборке коллекций, отражающей 
современные тенденции научных исследований.
  
  Отбор глобальных коллекций данных и~примеров их использования в~данной работе 
ограничен временн$\acute{\mbox{ы}}$ми рамками (учитываются крупные проекты накопления 
и~использования данных в~различных ОИИД, выполняемые до 2025~г.), а~так\-же 
конкретным набором предварительно определенных ОИИД, включающим астрономию, 
материаловедение, науки о~Земле, геномику и~протеомику, нейронауку. Этот список 
дополнен информатикой, в~силу того что все инфраструктурные проекты в~значительной 
мере посвящены проб\-ле\-мам ин\-фор\-ма\-ци\-он\-но-ком\-му\-ни\-ка\-ци\-он\-ных 
технологий (ИКТ).
  
  По каждой из рассмотренных ОИИД авторы стремились представить в~нижеследующих 
разделах работы следующую информацию:
  \begin{itemize}
\item крупные стратегические инициативы США и~ЕС по направлению;
\item примеры крупных коллекций данных в~мире до 2025~г.\ по направлению;
\item известные проекты ИКТ-инфраструктур и~центров данных;
\item сравнимые проекты в~России, при наличии таковых.
\end{itemize}

\section{Астрономические данные}

\subsection{Большой обзорный телескоп}

  Обзорный широкоугольный (поле зрения~--- около~10~кв.\ град.)\ 8-мет\-ро\-вый 
телескоп  (Large Synoptic Survey Telescope, LSST) строится в~Чили на высоте 2700~м 
и~начнет функционировать в~начале  
2020-х~гг. Он предназначен для регистрации объектов всей доступной полусферы неба 
и~прежде всего для обнаружения темной материи и~темной энергии, поиска околоземных 
астероидов, изучения природы транзиентных явлений и~картографирования Млечного 
Пути.
  
  Поток данных, ожидаемый от проекта LSST начиная с~2020~г., будет поступать 
с~беспрецедентной скоростью. Телескоп будет собирать данные о~более чем 
40~млрд объектов, а~так\-же проводить исследования переменных источников 
и~транзиентных событий. Предполагается, что доступная для наблю\-де\-ний полусфера будет 
полностью покрываться наблюдениями LSST в~6~фотометрических фильтрах (ugrizy) не 
реже, чем раз в~неделю.
  
  Объем наблюдений за ночь достигнет 15~ТБ (терабайт), что за 10~лет приведет к~суммарному 
объему около~60~ПБ  (петабайт) сжатых сырых данных, 15~ПБ  баз данных, 0,5~ЭБ
(эксабайт)  в~коллекции 
изображений. Эти данные соответствуют каталогу, содержащему 20~млрд галактик 
и~17~млрд звезд, 7~трлн детектируемых источников и~около 30~трлн 
измерений.
  
  Две группы данных планируется сделать пуб\-лич\-ны\-ми: автоматическая система будет 
посылать оповещения о~транзиентных событиях, а~также пуб\-лич\-но доступными будут 
данные верхнего уровня (катало\-ги). Для институтов, сотрудничающих в~рамках миссии, 
гарантирован доступ к~вычислительным ресурсам для эффективного поиска, запуска 
программ над базой данных в~15~ПБ и~обработки изображений в~базе данных 
в~100~ПБ~[2]. Институты России не участвуют в~этом проекте. 

\subsection{Массив квадратного километра} 

Массив квадратного километра (Square Kilometer Array, SKA)~--- наиболее амбициозный проект в~радиоастрономии. Радиотелескоп, 
расположенный и~Южной Африке, содержит тысячи отдельных антенн, занимающих 
площадь около квадратного километра. Он работает в~широком диапазоне час\-тот 
(от~50~МГц до~14~ГГц), имеет чувствительность, в~пятьдесят раз превышающую 
возможности современных радиотелескопов, и~способен производить обзор неба в~10~тыс.\
 раз быстрее. 
  
  Количество собранной информации ставит сложную задачу хранения и~потребует 
обработки данных в~реальном времени. По оценкам, SKA может создавать эксабайт  
сырых данных ежедневно, который после обработки в~режиме реального времени можно 
будет cжимать до 10~ПБ~[3]. Требования к~мощности компьютеров для обработки данных 
превышают характеристики имеющихся самых быстрых компьютеров в~2015~г., 
а~передача данных в~Интернете требует нового вида высокоскоростных сетей. Ученые 
России не участвуют в~этом проекте.

\subsection{Космическая обсерватория Гайя}

  Гайя  (Gaia) является космической обсерваторией Европейского космического агентства 
  (European Space Agency, ESA), 
созданной для астрометрии и~выведенной на орбиту в~2013~г. Цель~--- создание 
трехмерного каталога 1~млрд астрономических объектов, главным образом 
звезд, позволяющего понять образование и~эволюцию нашей Галактики. Этот каталог 
станет основой для нового взгляда на Галактику и~ключом для решения фундаментальных 
астрономических проблем. Дополнительно ожидается обнаружение от тысяч до десятков 
тысяч планет, подобных Юпитеру, за пределами Солнечной сис\-те\-мы (экзопланет), около 
полумиллиона квазаров и~десятков тысяч астероидов и~комет в~Солнечной сис\-те\-ме. За пять 
лет миссии общий объем данных составит 20~ТБ. Окончательная версия каталога будет 
доступна в~2020~г. Доступ к~полученным данным ограничен. Российские ученые 
участвуют в~поддержке проекта наземными наблюдениями (\textit{Gaia Follow-Up 
Network}). 

\subsection{Детекторы гравитационных волн (gravitational wave astronomy)}

  Проекты LIGO (Laser Interferometer Gravitational-Wave Observatory) или
  Advanced LIGO и~Virgo ориентированы на экспериментальное 
подтверждение поступления гравитационных волн от их наиболее мощных источников~--- 
взрывов кол\-лапси\-ру\-ющих cверхновых и~слияния нейтронных звезд в~тесных двойных 
системах. Исходные данные проекта LIGO составляют~1~ПБ, и~к ним предо\-став\-лен 
пуб\-лич\-ный доступ. В~рамках проектов в~реальном времени работает система оповещения 
об обнаружении и~возможных областях локализации транзиентных источников 
гравитационных волн. Российские ученые участвуют в~коллаборации LIGO. Система 
оповещения для поддержки проекта наблюдениями областей локализации в~оптическом 
диапазоне доступна всем ученым после подписания соглашения с~коллаборацией LIGO.

\subsection{Публичные коллекции данных}
  
  \textit{Sloan Digital Sky Survey} (SDSS)~--- один из основных продолжающихся проектов 
астрономических наблюдений, более 15~лет его поддержки посвящены созданию карты 
Вселенной. Каждую ночь широкоугольный телескоп производит более 200~ГБ 
многоцветных фотометрических обзоров и~спект\-ро\-скопических данных (для сравнения, 
LSST за ночь будет производить 15~ТБ). В~настоящее время фотометрическими 
наблюдениями в~пяти фильтрах покрыто около 37\% всего неба. На их основе создаются 
каталоги, включающие звезды, звездные скопления, галактики, экзопланеты и~др.
  
  \textit{NASA/IPAC Extragalactic Database} (NED)~--- это база данных внегалактических 
объектов, обеспечивающая систематический анализ интегрируемой информации из сотен 
обзоров неба и~десятков тысяч публикаций. Диапазон наблюдаемых спект\-ров~--- от  
гамма- до радиочастотного излучения. По мере опубликования наблюдения 
кросс-иден\-ти\-фи\-ци\-ру\-ют\-ся с~предшествующими данными и~интегрируются в~базу 
данных для упрощения 
запросов и~извлечения требуемых данных. Приблизительный объем базы данных NED около 20~ТБ. 
  
  \textit{Mikulski Archive for Space Telescope} (MAST). Основой MAST является архив 
научных данных, полученных от чрезвычайно успешно до сих пор функционирующего 
космического телескопа им. Э.~Хаббла. В~него включены также данные таких 
космических проектов, как Kepler, IUE (International Ultraviolet Explorer), 
GALEX (Galaxy Evolution Explorer) и~др. Объем данных составляет чуть 
бо-\linebreak лее~100~ТБ, и~они публично доступны~[4]. 
  
  \textit{The ESO (European South Observatory) Science Archive} представляет коллекцию 
данных Европейской южной обсерватории. Месячный поток данных составляет 
$7\sim8$~ТБ, а~полный объем превышает 100~ТБ за последние несколько лет. Данные 
\mbox{после} научной обработки в~большей части становятся доступными через год. Публичная 
часть архива доступна для зарегистрированных пользователей из международного 
сообщества. России пока не удалось стать членом ESO.

\subsection{Примеры астрономических миссий и~коллекций данных в~России}

  В России наиболее близким аналогом архива ESO является архив общих наблюдений 
Специальной астрофизической обсерватории РАН, содержащий в~2010~г.\ данные 
объемом~250~ГБ~[5].
  
  Данные международной космической обсерватории INTEGRAL составляют несколько 
десятков терабайт и~являются публично доступными по про\-шест\-вии одного года, в~течение 
которого исключительные права на данные принадлежат заявителям ежегодных открытых 
программ. 
  
  В России запланировано несколько космических проектов: наряду с~уже работающим 
с~2011~г.\ орбитальным радиотелескопом <<Радиоастрон>>~[6] это  
Спектр-Рент\-ген-Гам\-ма ({\sf http:// hea.iki.rssi.ru/ru/index.php?page=srg}), WSO-UV~[7], 
а~также <<Миллиметрон>>~[8].

\section{Данные в~исследованиях мозга}

  Нейронаука~--- это совокупность анатомии, физиологии, генетики, биохимии, 
патологии нервной системы, психологии. Она является передним краем изучения мозга 
и~мышления. Изучение мозга важно для понимания того, как мы воспринимаем 
и~взаимодействуем с~внешним миром. 
  
  Количество данных, генерируемых в~типовой лаборатории, проводящей исследования 
в~нейронауке, растет с~поражающей быстротой. Интеграция полученных данных в~единую 
картину является сложной задачей. Для ее решения необходима
 ней-\linebreak роинформатика, 
возникающая при сотрудничестве исследователей в~нейронауке с~информатиками, для 
того чтобы как новые, так и~ранее известные данные стали доступнее сообществу 
исследователей для ускорения нашего понимания работы мозга~[9].

\subsection{Исследование мозга в~рамках стратегической инициативы развития 
инновационных нейротехнологий (BRAIN)}
  
  Инициатива Белого дома BRAIN, объявленная в~апреле 2013~г.,~--- это десятилетняя 
программа, нацеленная на создание динамического понимания функций мозга 
и~демонстрацию того, как отдельные клетки и~сложные нейросети взаимодействуют 
в~здоровом или больном организме. 
  
  Главные цели анализа сетей взаимодейст\-ву\-ющих нейронов:
  \begin{itemize}
\item идентификация и~описание компонентов нейронов, определяющих их (клеток) 
синаптические связи друг с~другом, на основе изучения динамики активности во 
время функционирования нейросетей в~живом организме;
\item понимание алгоритмов управления обработкой информации внутри 
нейросетей и~между взаимодействующими нейросетями в~мозге в~целом.
\end{itemize}

  Ожидается, что в~результате данного исследования появится концептуальная база 
понимания биологической основы ментальных процессов вследствие развития новых 
теоретических инструментов, а~также инструментов обработки данных. Теоретические 
и~статистические исследования, а~также моделирование способствуют пониманию 
комплексных, нелинейных функций мозга. 
  
  Программе BRAIN необходима инфраструктура для обобщения и~обмена релевантными 
наборами данных, а~также методами анализа данных. Значительным препятствием, 
которое затрудняет понимание работы мозга, является раздробленность исследований 
мозга и~получаемых в~результате этих исследований данных. Основной целью является 
согласование международных усилий по интеграции этих данных в~единую картину мозга 
как отдельной многоуровневой системы. Объем данных о~мозге на клеточном уровне 
имеет порядок эксабайтов. Планируется построить комплексную систему 
исследовательских платформ, основанных на ИКТ, которая позволила бы нейробиологам, 
ме\-ди\-кам-ис\-сле\-до\-ва\-те\-лям и~разработчикам новых технологий ускорить темпы их 
исследований. 
  
\subsection{Проект Европейского Союза по~исследованию человеческого мозга}

  Human Brain Project (HBP)~--- это главный десятилетний проект Европейского Союза с~бюджетом 
  в~1~млрд Евро, нацеленный на ускорение процесса понимания работы человеческого 
мозга. Данный проект включает исследования по диагностике и~определению расстройств 
мозга, а~также по разработке новых технологий, основанных на принципах работы 
мозга~[10]. 
  
  Human Brain Project состоит из~13~подпроектов, охватывающих стратегические данные 
нейробиологии, когнитивную архитектуру, теорию, этику, менеджмент, а~так\-же развитие 
новых платформ, основанных на информатике.
  
    Основной целью HBP является создание реалистичной симуляции человеческого мозга. 
Для этого потребуется молекулярная и~клеточная информация, позволяющая моделировать 
и понять биологические процессы в~норме и~патологии. Это позволит использовать 
данную информацию для разработки и~применения новых типов компьютеров 
и~робототехники, т.\,е.\ для применения полученных результатов для разработки новых 
технологий (создания нейроморфных устройств). 

    Планируется построить управляемые данными модели, которые отображают то, что 
удалось узнать о~мозге экспериментальным путем, его глубинную механику, а~также 
познать основные принципы, на которых основан мыслительный процесс. Модели мозга 
будут создаваться при помощи правил обучения, максимально приближенных к~реальным 
закономерностям, которые использует мозг. Ожидается, что подобные модели будут 
обучаться с~помощью тех же механизмов, которые используются человеческим мозгом, и~что 
они будут проявлять подобное интеллектуальное поведение.

  Проект HBP развивает 6~новых платформ, основанных на информатике: 
  \begin{enumerate}[(1)]
\item нейроинформатика (поисковые атласы и~анализ данных мозга);\\[-15pt]
\item симуляция мозга (построение и~симуляция многоуровневых моделей нервных сетей 
и~церебральных функций);\\[-15pt]
\item медицинская информатика (анализ клинических данных для лучшего 
понимания болезней мозга);\\[-15pt]
\item нейроморфные вычисления (применение функций, подобных функциям мозга, 
в~аппаратном обеспечении);\\[-15pt]
\item нейроробототехника (тестирование моделей мозга и~их симуляция 
в~виртуальной среде);\\[-15pt]
\item высокопроизводительные вычисления (обеспечивающие необходимую 
вычислительную способность).
\end{enumerate}

\vspace*{-12pt}

\subsection{Проект коннектома человека}
  
  Структурные (анатомические) связи мозга (его коннектом) могут быть отображены на 
нескольких уровнях: макро- (в~сантиметровом и~миллиметровом масштабе), мезо- 
(в~миллиметровом и~микронном масштабе) и~микромасштабе (в~микронном 
и~нанометровом разрешении). Текущие разработки по человеческому коннектому 
(отображению всех нейронных связей в~нервной системе) проводятся только 
в~макромасштабе~[11]. Данные по Human Connectome Project (HCP) (объемом в~десятки терабайт) уже доступны для 
анализа. 

\subsection{Нейробиологические базы данных}

  \textit{Атласы мозга (Аллена)}~--- это проекты по совмещению геномики и~нейроанатомии 
при помощи создания карт экспрессии генов для мозга мыши и~человека~[12]. Данные 
этих проектов будут способствовать развитию различных областей нейронаук, они 
помогут выяснить роль определенного гена в~том или ином заболевании мозга. Разные 
типы клеток центральной нервной системы возникают в~связи с~изменением экспрессии 
генов. Карта экспрессии генов в~мозге позволяет исследовать отношения между формой 
и~функцией. Атлас мозга дает исследователю вид областей с~отличием экспрессии генов 
в~мозге, которые позволяют исследовать пути формирования нейронных связей. Изучение 
этих путей, также и~с помощью методов нейровизуализации, позволит установить 
отношения между экспрессией генов, типами клеток и~функцией различных путей мозга 
в~организации поведения и~фенотипами. Атлас позволит показать, какие гены и~области 
мозга связаны с~неврологическими и~психическими расстройствами. 

\subsection{Данные в~нейронауке в~России}

В ряде российских исследовательских центров (Научный центр неврологии, Институт 
высшей нервной деятельности и~нейрофизиологии РАН, Институт мозга человека им.\ 
Н.\,П.~Бехтеревой и~др.) накоплены большие коллекции данных по анатомии, гистологии, 
генетике и~биохимии нервной системы, компьютерной томографии, структурной 
и~функциональной магнитно-резонансной томографии мозга, электроэнцефалографии 
и~вызванным потенциалам при нормальном развитии и~старении, а~так\-же при 
неврологических и~психических заболеваниях. В~настоящее время эти данные доступны 
в~центрах, где коллекции были получены, и~в~сотрудничающих с~ними организациях. 
Разработка открытых коллекций данных будет способствовать повышению эффективности 
исследований в~области нейронаук.

\section{Данные в~геномике и~протеомике}

  Для современной молекулярной генетики ха\-рак\-терно появление качественно новых 
возможно\-стей, связанных с~использованием в~исследованиях высокопроизводительных 
экспериментальных технологий, которые привели к~беспрецедентному объему 
накопленных данных и~знаний~[13]. Эти\linebreak данные используются для сравнительного 
ана\-лиза геномов, поиска генетических вариаций\linebreak и~биомаркеров, которые применяются 
в~биотехнологии, сельском хозяйстве, фармакологии, клинических исследованиях, 
персонализированной медицине и~т.\,д.
  
  Прогнозные оценки указывают на то, что общий объем геномных данных по всем 
проектам ежегодно будет увеличиваться в~3~раза и~достигнет к~2018~г.\ объема 3300~ПБ.
  
  В настоящее время существует около~7400~высокопроизводительных геномных 
секвенаторов, которые работают в~1027~центрах по всему миру. В~России находится 
только~14~геномных секвенаторов в~6~научных центрах. Поэтому б$\acute{\mbox{о}}$льшая часть 
геномных данных генерируется за рубежом: в~США, Европе, Китае, Южной Корее и~др. 
  
  Мультимодальность, многоуровневость и~широкомасштабность биологических систем 
порождают огромный объем неоднородных и~распределенных данных, для которых 
характерна изменчивость и~несогласованность, необходимость контроля точности этих 
данных. 
  
  Как и~в других предметных областях, отсутствие технологий поиска, распределения, 
хранения, поддержки целостности, передачи, интеграции и~визуализации больших данных 
существенно затрудняют анализ и~систематизацию больших данных~[14--16].

\subsection{Коллекции геномных данных}
  
  Целью проекта <<\textit{1000~геномов}>> является создание наиболее 
детализированного каталога генетического разнообразия человеческого генома, 
основанного на результатах секвенирования геномов более чем~2600~человек 
из~26~популяций по всему миру. 
  
  Проект <<\textit{1001~геном}>> ориентирован на поиск генетиче\-ских вариаций 
в~геномах различных штаммов растения Arabidopsis thaliana, которое используется как 
модель для детального изучения мо\-ле\-ку\-ляр\-но-генетических механизмов у~растений. Эта 
информация открывает новые возможности в~генетике, определяя аллели, ответственные 
за фенотипическое разнообразие целого генома одного вида как на разных уровнях, 
включая биохимический, метаболический, физиологический, морфологический, так и~на 
уровне целого растения. Результаты исследования проекта <<\textit{1001~геном}>> важны 
для развития таких наук, как селекция растений, биотехнология и~медицина.
  
  Проект <<\textit{Геном 10К}>> содержит коллекцию более чем 
16\,000~последовательностей геномов позвоночных, включая ныне живущих и~недавно 
вымерших млекопитающих, птиц, рептилий, амфибий, рыб и~многих других видов, 
находящихся под угрозой исчезновения или вымирающих~[17].
  
  Целью проекта <<\textit{Человеческий микробиом}>> является описание метагенома 
микробных сообществ, найденных во многих частях человеческого тела, а~также поиск 
соотношений между изменениями в~микробиоме и~здоровьем человека. 
  
  Проект <<\textit{Атлас генома рака}>> содержит исследования геномов пациентов, 
страдающих от более~33~видов рака. В~настоящий момент накоплена информация о~более 
чем 7000 вариантах рака~[18]. Эта информация важна для поиска генетических маркеров 
рака и~использования их для диагно\-стики.

\subsection{Атлас протеомы человека}

  Интерактивный атлас протеом человека, созданный в~Стокгольме, в~Royal Institute of 
Technology, ориентирован на фундаментальные исследования в~области биологии человека 
и применение в~трансляционной медицине. В~настоящее время атлас содержит 13~млн 
аннотированных изображений человеческих тканей.
  
  В рамках этого проекта могут изучаться различ\-ного типа протеомы человека, например 
протео-\linebreak ма домашнего хозяйства, включающая белки,\linebreak экспрессирующиеся во всех типах 
тканей, тканеспецифические протеомы, включающие белки, которые показывают 
повышенную экспрессию только в~одной или нескольких типах тканей, или\linebreak протеомы, 
связанные с~определенными функциями, такими как лекарственные протеомы, 
вклю\-ча\-ющие все бел\-ки-ми\-ше\-ни лекарств, раковая\linebreak протеома, включающая белки, 
участвующие в~патогенезе рака, а~так\-же секретома~--- все белки, которые сек\-ре\-ти\-ру\-ют\-ся, 
и~т.\,д.

\subsection{ELIXIR~--- Европейская медико-биологическая инфраструктура 
биологической информации}

  ELIXIR~--- это проект Европейской молекулярной биологической обсерватории 
(European Molecular Biology Laboratory, EMBL), реализуется как панъевропейская исследовательская инфраструктура. Целью 
ELIXIR является пред\-остав\-ле\-ние средств, необходимых для всех исследователей в~области 
медицины и~биологии, начиная с~полевых биологов и~заканчивая хи\-ми\-ко-ин\-фор\-ма\-ти\-ка\-ми, 
позволяя им получить полную информацию из быстрорастущего хранилища информации 
о живых системах. Эти данные являются основой, на которой базируется наше понимание 
жизни. 
  
  Задачей ELIXIR является управление сбором, контролем качества и~архивированием 
больших объемов биологических данных, полученных вследствие биологических 
и~медицинских экспериментов. Некоторые из этих наборов данных ранее были слишком 
специализированными и~доступными лишь для ученых той страны, в~которой они были 
получены.

\subsection{Интеграция BILS-ProteomeXchange на~основе ресурсов EUDAT}
 
  Этот пилотный проект нацелен на интеграцию хранилищ сырых данных  
масс-спектроскопии, данных протеомики, собираемых в~BILS (Швеция) 
и~ProteomeXchange (через базу данных PRIDE (Proteonics Identifications), EMBL-EBI, U.K.), используя Европейскую 
инфраструктуру EUDAT. Проект служит примером объединения национальных хранилищ 
данных и~международных репозиториев посредством ELIXIR.
  
\subsection{Проект BD2K}

  Проект \textit{От больших данных к~знаниям} (Big Data to Knowledge, BD2K) позволяет 
использовать биомедицинские большие данные для укрепления человеческого здоровья 
посредством создания, индексирования и~распространения методов, инструментов 
и~обучающих материалов. Проект BD2K (начатый в~2012~г.)\ имеет четыре главные цели, 
которые в~совокупности расширят использование биомедицинских больших данных:
  \begin{itemize}
\item упростить широкое использование биомедицинских цифровых ресурсов, сделав их 
более доступными, распространенными и~цити\-ру\-емыми;
\item проводить исследования и~разрабатывать методы, программное обеспечение 
и~инструменты, необходимые для анализа биомедицинских больших данных;
\item усилить обучение развитию и~использованию методов и~инструментов, необходимых 
для нау\-ки биомедицинских больших данных;
\item поддержать экосистему данных, ускоряющую открытия.
\end{itemize}

В проекте участвуют 185~институтов, 11~BD2K центров мастерства (centers of excellence).

\section{Данные в~материаловедении}

  Современные материалы во многом определяют развитие человеческой цивилизации. 
Они широко применяются в~промышленности, включая те отрасли, которые 
непосредственно связаны с~национальной безопасностью, разработкой источников чис\-той 
энергии и~обеспечением высокого уровня жизни людей. Особенностью данных 
в~неорганической химии и~материаловедении является то, что они представляют собой 
результат обработки и~систематизации больших (сотни петабайт) объемов исходных 
экспериментальных данных. В~связи с~этим создание инфраструктуры для хранения 
и~поиска данных~--- одна из важнейших проблем разработки информационных сис\-тем 
для материаловедения. 

\subsection{Инициатива генома материалов}

  Согласно Инициативе генома материалов 
  (Materials Genome Initiative, MGI), объявленной Белым домом в~2011~г., 
ускоренное создание новых материалов, обладающих заданными свойствами, критично 
для достижения высокого уровня конкурентоспособности промышленности США~[19]. 
Цель MGI~--- обеспечение разработки и~внедрения новых материалов за счет координации 
исследований и~предоставления доступа к~расчетным моделям и~инструментарию для 
оценки свойств и~поведения материалов, а~также использования прорывных методов 
моделирования и~анализа данных. Главной целью MGI является создание механизмов, 
способствующих обмену данными и~знаниями о~материалах не только между 
исследователями, но и~между академической наукой и~промышленностью.
  
  Инициатива генома материалов будет способствовать поддержке лидиру\-ющей роли США во многих секторах 
современного материаловедения и~промышленности: от энергетики до электроники, от 
обороны до здравоохранения, а~так\-же поддержке недавних прорывов в~тео\-рии, 
моделировании свойств материалов и~data mining для существенного прогресса 
в~материаловедении, что приведет к~снижению затрат на разработку, исследование 
и~получение новых материалов. Основой MGI является \textit{Инфраструктура инноваций 
в~материаловедении} (Materials Innovation Infrastructure), которая обеспечит интеграцию 
методов и~средств современного моделирования, включающего данные, а~так\-же 
экспериментальный и~тео\-ре\-ти\-че\-ский инструментарий. 

\vspace*{-3pt}
  
\subsection{Средства организации данных о~материалах}

\vspace*{-1pt}
  
  В июне 2014~г.\ консорциум Национальных сервисов данных (National Data Service,
  NDS) объявил о~первом 
показательном проекте разработки средств для организации данных, выбрав для этого 
область материаловедения (Materials Data Facility,
MDF)~[20]. Этот проект является реакцией на инициативу 
Белого дома MGI по ускорению разработки современных материалов. MDF обеспечит 
материаловедов масштабируемым репозиторием для хранения экспериментальных 
и~расчетных данных, в~том числе и~до их публикации, снабженных ссылками на 
соответствующие библиографические источники. MDF станет рычагом для создания 
национальной инфраструктуры коллективного использования информации, включая 
разработанные в~мире базы данных по свойствам материалов и~информационные системы 
для расчета и~моделирования, а~также будет способствовать организации обмена данными 
о материалах, в~том числе и~еще не опубликованными. Доступность данных и~средств 
расчета обеспечивается современной информационной и~телекоммуникационной 
инфраструктурой, которая позволяет предоставить данные исследователям материалов для 
многоцелевого использования, дополнительного анализа и~проверки.

\vspace*{-3pt}

\subsection{Программа VAMAS}

\vspace*{-1pt}

  Versailles Project on Advanced Materials and Standards (VAMAS)~[21]~--- это программа 
международного сотрудничества, призванная продвигать исследования и~разработки, 
которые обеспечивают подготовку новых стандартов для современных материалов. 
Предполагается, что эта программа приведет к~согласованию стандартов по всему миру. 
Предварительные исследования при разработке стандартов особенно необходимы в~случае 
современных материалов, поскольку традиционные тес\-ты не всегда подходят для них. 
VAMAS создан для преодоления барьеров в~обмене новыми технологиями, необходимыми 
для исследований на базе международных стандартов.

\vspace*{-3pt}
  
\subsection{Коллекции данных в~материаловедении}

\vspace*{-1pt}

  Коллекция данных Национального института стандартов и~технологии (National
  Institute of Standards and Technology, NIST) США 
содержит информацию о~широком наборе веществ и~материалов: неорганических 
и~органических веществах, включая пластмассы, углеродные нанотрубки, высокопрочные 
сплавы, искусственные кости и~т.\,д., для которых в~институте развиваются стендовые 
испытания и~определяются эталонные тесты.
  
  Коллекция данных Национального института материаловедения (Япония) содержит 
информацию о~веществах и~материалах разной природы: неорганических веществах, 
композитах, промышленных сплавах и~т.\,д.
  
  Немецкая сеть на\-уч\-но-тех\-ни\-че\-ской инфор\-мации STN 
  (Scientific and Technical Network) предо\-став\-ля\-ет доступ 
к~опубликованным экспери\-ментальным данным о~структуре и~свойствах\linebreak материалов, 
патентам и~иной информации.
  
  Коллекция данных Springer Materials (Германия) обеспечивает доступ к~данным 
о~3000~физических и~химических свойств более 250\,000~материалов и~веществ.

\vspace*{-4pt}
  
\subsection{Проекты информационных систем в~области материаловедения 
в~России}

\vspace*{-2pt}

  Развитие информационных систем по материаловедению в~России является 
инициативой разработчиков. Наиболее известные информационные системы разработаны 
в Объединенном институте высоких температур РАН~[22] и~Институте металлургии 
и~материаловедения РАН~[23]. Базы данных в~этих системах объединены с~подсистемами 
расчета термодинамических свойств веществ~[22] и~системами data mining, 
позволяющими конструировать еще не полученные неорганические соединения~[22].

\vspace*{-4pt}

\section{Коллекции данных в~науках о~Земле}

\vspace*{-2pt}

  Объектами исследования наук о~Земле являются планета Земля и~ее атмосфера. 
Комплексные исследования процессов, происходящих в~литосфере, атмосфере, 
гидросфере, биосфере и~криосфере, направлены на понимание функционирования Земли 
как системы. Особенностью наук о~Земле является сложная иерархия предметных 
областей, включающих в~себя как фундаментальные, так и~прикладные науки. Эта 
иерархия накладывает жесткие ограничения как на данные отдельных предметных 
областей наук о~Земле, так и~на структуры интегрированных данных наук о~Земле, 
предназначенных для использования в~таких приложениях, как метеорология, 
климатология, океанология, экология. 
  
  Основные массивы данных в~науках о~Земле получаются в~результате локальных 
и~дистанционных наблюдений, а~также численного моделирования изучаемых процессов. 
При этом объемы соответствующих архивов, например для данных дистанционного 
зондирования, достигают десятков петабайт, а~данных климатических вычислительных 
экспериментов~--- единиц петабайт (CMIP5~--- Coupled Model Intercomparison Project Phase~5, 
ERA CLIM). В~об\-ласти климатологии основные 
усилия направлены на выяснение причин и~последствий происходящих сейчас 
и~возможных в~будущем глобальных климатических изменений. Прикладной целью этих 
исследований является создание <<службы климата>> как аналога службы погоды. 
Инструментами получения данных здесь являются сети метеостанций, сети плавающих 
в~океанах буев, сети наземных измерительных комплексов, осуществляющие наблюдения 
за локальными климатическими и~экологическими характеристиками, системы спутников, 
осуществляющих наблюдения за атмосферой и~поверхностью Земли, и~климатические 
модели.
  
  Основным источником больших массивов данных являются спутники. Петабайтные 
коллекции данных формируются, поддерживаются и~обслуживаются в~США 
профильными национальными ведомствами (NOAA~--- National Oceanic and Atmospheric
Administration, NASA~--- National Aeronautics and Space Administration, 
DoE~--- Department of Energy), а~в~Европе~--- либо 
наднациональными тематическими структурами (ECMWF~--- European Center for
Medium range Weather Forecasting, ESA), либо консорциумами 
ведущих по теме университетов и~исследовательских центров. Эти структуры активно 
участвуют в~реализации указанных программ.

\subsection{Примеры крупных проектов получения и~накопления данных в~науках 
о~Земле}

  В области наук о~Земле, точнее об окружа\-ющей среде, наибольший прогресс в~деле 
обеспечения всего комплекса работ, связанных с~большими объемами данных, достигнут 
в~области дистанционного зондирования Земли. Примеры соответст\-ву\-ющих программ 
рассматриваются далее.
  
  В ЕС наиболее амбициозную программу \textit{Copernicus} возглавляют Европейская 
комиссия и~ESA. Европейское космическое агентство координирует доставку 
данных с~30~спутников, а~комиссия отвечает за проект, устанавливает требования 
и~управляет сервисами. 
  
Европейское космическое агентство создает семейство спутников (Sentinels) для оперативных нужд этой программы. 
Спутники будут проводить уникальный набор наблюдений, таких как всепогодные 
круглосуточные радарные изоб\-ра\-же\-ния, получать оптические изображения высокого 
разрешения для наземных сервисов, данные для сервисов, относящихся к~океану 
и~приземному слою, данные по мониторингу состава атмосферы с~геостационарных 
и~полярных орбит, а~также данные с~радарного высотомера для измерения высоты 
морской поверхности в~океанографии.
  
  Программа \textit{Copernicus}~[24] обеспечит сервисы для предсказания качества 
воздуха,  пред\-упреж\-дения наводнений, раннего обнаружения засух 
и~опус\-ты\-ни\-ва\-ния, оценки качества морской воды и~анализа урожая зерновых, 
мониторинга лесов, контроля изменений землепользования, предсказания 
катастрофических погодных условий, фиксации разливов нефти, слежения за 
отклонением кораблей от курса и~т.\,д. 
  
  Программа \textit{Copernicus} (получая~5~млрд Евро за период 2014--2020~гг.)\ 
предусматривает доставку высококачественных данных (до~8~ТБ в~день) в~рамках 
политики, основанной на полном и~открытом доступе к~данным. Предоставляя данные 
высокого разрешения о~приземном слое, океане и~атмосфере, Copernicus получит 
возможность управлять развитием исследований и~сотрудничества в~новых приложениях 
наук о~Земле.
  
  Развиваемая в~США \textit{Earth Observing System} (EOS)~--- это координированный 
набор спутников для долговременных глобальных наблюдений приземного слоя Земли, 
биосферы, земной по\-вер\-хности, атмосферы и~океанов, позволяющих улучшить понимание 
Земли как сложной интегрированной системы. Информационная инфраструктура EOS 
содержит~12~национальных цент-\linebreak ров в~США, которые хранят и~обеспечивают 
непрерывный доступ к~широкому разнообразию\linebreak геофизической информации о~Земле 
и~космосе: полярных и~приземных процессах; верхней атмосфере, глобальной биосфере, 
атмосферной динамике и~геофизике; физической океанографии, радиационному бюджету, 
тропосферной химии,\linebreak облакам и~аэрозолям; глобальном распределении снега и~льда; 
криосфере; биохимической динамике; воздействии человека на окружающую среду; 
гид\-ро\-логическом цикле; климате и~погоде; геофизике земной тверди, геологии 
и~геофизике морей, сол\-неч\-но-зем\-ной физике, палеоклиматологии; спутниковом 
дистанционном зондировании. Данные и~средства работы с~ними объединены 
в~\textit{Earth Observing System Data and Information System} (\mbox{EOSDIS})~[25]. 
{ %\looseness=1

}
  
  В настоящее время этот опыт активно используется в~мире для создания национальных 
сегментов такой глобальной информационной системы и~для глобальной инфраструктуры 
международного проекта по наблюдению Земли из космоса \textit{GEOSS} (\textit{Global 
Earth Observation System of Systems})~--- международного проекта, рассчитанного на 
несколько десятилетий. Объем финансирования~--- десятки миллиардов долларов. 
  
  Проект \textit{Data Observation Network for Earth} (DataONE, 
  {\sf https://www.dataone.org}) 
является основой для создания науки об окружающей среде\linebreak в~форме распределенной базы 
и устойчивой киберинфраструктуры для открытого, постоянного, устойчивого 
и~безопасного доступа к~качественным описаниям и~легкодоступным данным наблюдений 
о~Земле. Проект не предназначен для хранения данных. Он является основой для 
соединения многочисленных репозиториев в~федеральных сетях для поиска, 
извлечения и~обеспечения репликаций на репозиториях данных внутри сетей. 
  
  В проекте будет создано легкое и~просто инсталлируемое программное обеспечение 
и~развита совместимость программного обеспечения, уже развернутого в~репозиториях по 
всему миру. Новыми чертами проекта будут:
  \begin{itemize}
\item семантический поиск результатов измерений;
\item отслеживание всех этапов жизненного цикла данных;
\item сервисы обработки данных, дающие исследователям возможность простыми 
способами обращаться к~большим данным.
\end{itemize}

  Проект \textit{Satellite Observations for Climate Modeling} (SOCM) посвящен 
  интеграции 
спутниковой информации и~моделированию процессов. Новое поколение инфраструктуры 
будет поддерживать сравнение спутниковых наблюдений с~климатическими моделями. 
Публикация данных дистанционного зондирования вместе с~результатами моделирования 
климата будет способствовать их сравнению и~пониманию. Кроме того, лица, 
принимающие ключевые решения о~будущем климата, состоянии регионального уровня 
туризма, водных ресурсах и~управлении питанием: штаты, федеральное правительство 
и~иностранные структуры,~--- будут использовать эту более полную информацию. 
  
  Следующим шагом является преобразование климатической аналитики в~сервисы~[26]. 
Например, сервис CAaaS (Continuous Analitics as a~Service)
сочетает вычисления высокой производительности и~аналитику 
данных с~масштабируемым управ\-ле\-ни\-ем данными, виртуализацией облачных вычислений, 
представлением адаптивной аналитики и~API (Application Programming Interface), связанными с~предметными областями для 
улучшения доступа к~большим коллекциям климатических данных.
  
  В рамках международного сотрудничества \textit{Earth System Grid Federation} (ESGF) 
созданы порталы, интегрирующие коллекции научных данных, распределенные по всему 
миру. В рамках этого сотрудничества развивается виртуальная среда \textit{Earth\linebreak System 
Grid} (ESG) для содействия анализу глобальных климатических изменений 
и~обеспечивается доступ к~предсказанным климатическим данным. В~частности, 
исторические климатические данные и~результаты моделирования по климатическим 
сценариям, выполненные при подготовке недавнего доклада Межправительственной группы
экспертов по изменению климата, распространялись 
через ESGF. В~настоящее время около~2~ПБ данных архивированы в~узлах ESGF, 
распределенных по всему свету.
  
  В Европе скоординированный подход к~созданию глобальной инфраструктуры данных 
был выработан в~ходе выполнения проекта 7-й Рамочной программы ЕС <<Global Research 
Data Infrastructures: The Big Data Challenges>>. Реализующие эту программу проекты 
в~Европе можно разделить на три группы: проекты наднациональных структур (ESA, 
ECMWF), внутригосударственные проекты развития ведомств, о~которых имеется очень 
скупая информация, и~межгосударственные научные проекты в~рамках программ EC, 
информация о~которых доступна в~Сети ({\sf http://cordis.europa.eu}). В~частности, список 
инфраструктурных проектов 7-й Рамочной программы включает более чем 350~проектов. 
Не менее десятой части этих проектов связано с~науками о~Земле. 
  
  Следует добавить, что основой многих прикладных направлений наук о~Земле являются 
результаты фундаментальных наук. В~частности, существенную роль играют 
количественные данные, полученные в~таких фундаментальных науках, как 
спектроскопия, химия атмосферы и~др. Учитывая, что число молекул, рассматриваемых 
при решении задач, например, прогноза качества воздуха в~регионе, достигает почти 
тысячи, а~с~учетом их изотопов~--- более двух тысяч, объем спектральных данных и~затраты 
на анализ их качества, с~учетом постоянного потока данных в~новых спектральных 
интервалах, делают такие задачи чрезвычайно трудозатратными. Одним из выполненных 
в~Европе проектов, относящихся к~фундаментальным наукам, стал проект VAMDC~--- 
Virtual Atomic and Molecular Data Center~[27, 28]. Этот проект ориентирован на 
исследовательские группы и~институты, играющие центральную роль в~производстве 
атомных и~молекулярных данных, которые критичны для использования в~широкой 
области применений. 
  
\subsection{Сравнимые проекты в~России}

  В области создания информационных ресурсов для наук о~Земле инфраструктурных 
научных проектов, сравнимых по масштабу с~названными в~подразд.~6.1, в~России не 
было. 

Крупным ведомственным проектом является ЕСИМО (Единая государственная система
информации об обстановке в~Мировом океане). 

Более мелкие проекты 
связаны с~пространственными данными субъектов РФ. Проекты в~области наук о~Земле, 
финансируемые РФФИ и~РНФ, не являются частью каких-либо долгосрочных 
государственных программ.

\section{Инфраструктуры данных и~проекты для доступа к~данным 
и~анализа перспективных источников информации}

\subsection{Проекты исследовательских инфраструктур в~Европейском Союзе}

  \textit{Исследовательские инфраструктуры}, со\-зда\-ва\-емые в~ЕС, представляют собой 
средства, ресурсы или сервисы уникальной природы, которые были идентифицированы 
в~различных областях сообществами исследователей Европы для поддержки 
соответствующей деятельности на высоком уровне. Подобное определение 
\textit{исследовательской инфраструктуры}, включая ассоциированные с~ней людские 
ресурсы, охватывает крупное оборудование или наборы инструментов вместе 
с~содержащими знания ресурсами, такими как коллекции данных, архивы или банки 
данных.
  
  Европейский стратегический форум исследовательских инфраструктур (European
  Strategy Forum on Research Infrastructures, ESFRI) является 
стратегическим механизмом, образованным в~2002~г.\ странами~--- членами ЕС 
и~Еврокомиссией, чтобы способствовать научной интеграции Европы и~усилению ее 
международного влияния. Члены ESFRI назначаются министрами науки стран~--- членов 
или ассоциированных членов ЕС, а~также включают представителей Еврокомиссии. Они 
работают совместно для определения объединенного видения и~общей стратегии, 
включающих в~качестве инструментов планирования и~реализации новых 
панъевропейских исследовательских инфраструктур регулярно обновляемые дорожные 
карты, отче\-ты и~критерии. Подобный стратегический подход\linebreak нацелен на обеспечение 
Европы наиболее совре\-менными исследовательскими инфраструктурами, отвечающими 
нуждам быстро развивающихся областей науки, продвижение основанных на знаниях 
технологий и~расширение их применений.
  
  Ряд примеров исследовательских инфраструктур, деятельность которых приводит 
к~образованию новых коллекций данных и~знаний, к~их совместному использованию, 
рассматривается ниже.
  
  ЦЕРН ({\sf http://home.web.cern.ch})~--- наибольшая в~мире лаборатория ядерной физики 
частиц; именно ЦЕРН стал родоначальником идеи исследовательской инфраструктуры.
  
  GEANT ({\sf http://www.geant.net/pages/home.\linebreak aspx})~--- проект высокоскоростной сети, 
является примером инфраструктуры, способствующей совместному использованию 
данных и~знаний учеными.
  
  EMMA~--- Европейский архив мышиных мутантов (European Mouse Mutant Archive, {\sf 
http:// www.emmanet.org})~--- типичный пример распределенной инфраструктуры с~узлами 
в~шести странах, представленной для пользователей в~виде единственного центра.
  
  SIOS (Svalbard Integrated Arctic Earth Observation System,  
{\sf http://www.sios-svalbard.org/servlet/\linebreak 
Satellite?c=Page\&pagename=sios/Hovedsidemal\&cid\linebreak =1234130481072})~--- 
интегрированная система наблюдений Арктики на Шпицбергене, предназначена для 
изучения геофизических, химических и~биологических процессов, охватывая всю 
арктическую систему, начиная от верхних уровней атмосферы до процессов в~морских 
глубинах и~земной коре.
  
  BBMRI-LPC (Biobanking and Biomolecular Resources Research Infrastructure~--- Large 
Prospective Cohorts, {\sf http://www.bbmri-lpc.org/about})~--- исследовательская 
инфраструктура для получения биобанков биомолекулярных ресурсов~--- одна из 
наибольших сетей поддержки биобанков в~Европе; \mbox{целью} проекта является изучение 
подобных коллекций и~связи накопленных данных со здоровьем людей.
  
  EMbaRC (European Consortium of Microbial\linebreak Resource Centres, {\sf http://www.embarc.eu})~--- 
Европейский консорциум центров микробиомных ресурсов, служит для координации 
обеспечения информацион\-ными микробиомными ресурсами исследователей в~Европе 
и~в~мире.
  
  SYNTHESYS (Synthesis of Systematic Resources, {\sf http://www.synthesys.info})~--- 
  проект  создания интегрированной европейской инфраструктуры для поддержки коллекций 
естественной истории.

\columnbreak

\subsection{Панъевропейская инфраструктура данных EUDAT}

%\vspace*{-18pt}

  Европейская комиссия поддерживает развитие панъевропейской междисциплинарной 
инфраструктуры данных в~рамках программы Horizon 2020, следуя нескольким ведущим 
принципам. 

%\smallskip
  
  \textit{Федерализация.} Предполагается, что основные действия над данными 
реализуются в~федерациях данных. Они являются сетями репозиториев и~центров данных, 
которые предоставляют структуры для обработки данных и~действуют на основе 
соглашений о~легальных или этических правилах, интерфейсах и~спецификациях 
протоколов, а~также стека общих сервисов манипулирования данными. Такие центры 
могут являться членами многих федераций. Координированный подход предполагает, что 
каждый центр создает описание своих возможностей, а~каждая федерация может 
использовать одни и~те же описания для извлечения необходимой информации. Такой 
подход способствует открытому представлению исследовательских данных и~помогает 
изменять существующую культуру исследований для поддержки совместного 
использования данных. 

%\smallskip
  
  \textit{Открытое совместное использование данных}. Поскольку научные дисциплины 
интернациональны по своей природе, то критичным является следование международным 
подходам к~снижению барь\-е\-ров при обмене данными или при их повторном\linebreak 
использовании. На этом пути основными препятствиями являются неоднородность данных 
и~языков запросов, способность к~пониманию и~обнаружению данных, перемещение 
данных сквозь семантиче\-ские границы между многозначными контекстами, а~так\-же 
проблемы рассогласования данных (относительно качества, неполноты, абстракции 
данных).

%\smallskip
  
  \textit{Европейская инфраструктура данных EUDAT} является начальным шагом в~этих 
направлениях. EUDAT ({\sf http://www.eudat.eu}) объединяет~25~европейских партнеров, 
включающих центры данных, провайдеры технологий, сообщества исследователей 
и~фондовые агентства из~15~стран. EUDAT предлагает общие сервисы данных в~рамках 
географически распределенной сети, связывающей цент\-ры данных и~специализированные 
репозитории,\linebreak а~также решения для поиска, совместного использования, хранения, 
репликации, ста\-дий\-ности\linebreak первичных и~вторичных данных исследований и~выполнения их 
анализа. Такая сеть образует {Совместную} инфраструктуру данных (Collaborative Data 
Infrastructure), обозначаемую далее СИД, которая развивается как  
сер\-вис-ори\-ен\-ти\-ро\-ван\-ная, междисциплинарная и~устойчивая инфраструктура.
{\looseness=-1

}

%\smallskip
  
  \textit{EUDAT2020}~--- новый трехлетний большой проект развития СИД, начатый 
в~2015~г., целями которого являются: поддержка политики Европейской\linebreak комиссии 
открытого доступа к~данным исследо\-ваний, достижение интероперабельности 
существующих в~Европе инфраструктур научных исследований (ИНИ) для доступа 
ученых к~сетевым,\linebreak вычислительным ресурсам и~ресурсам данных в~различных ИНИ, 
включая гриды и~облачные инфраструктуры. Так, например, будут достигнуты\linebreak 
возможности подключения данных в~СИД к~высокопроизводительным ресурсам, 
организуемым в~рамках PRACE (Partnership for Advanced Computing in Europe), 
для их анализа или в~качестве входных данных моделей 
и~репликации полученных результатов в~систему хранения EUDAT; подключения данных 
в~СИД к~гридам и~облачным ресурсам, поддерживаемым EGI (European Grid
Infrastructure); а~так\-же федерализации 
данных при их подключении к~ряду европейских инициатив (таких как Nebula, GEANT, 
TERENA, \mbox{OpenAIR} и~др.). 

При организации EUDAT2020 достигнута договоренность 
о~партнерстве с~NDS по образованию совместных пилотных проектов 
(междис\-цип\-линарных и~межконтинентальных). В~СИД будет\linebreak поддерживаться функция 
долгосрочного ар\-хи\-ви\-рования данных, репликации, каталогизации,\linebreak
 цитируемости данных 
наряду с~обеспечением обнаружения, доступа, повторного использования коллекций 
и~отдельных объектов данных. Функции анализа данных будут поддерживаться ресурсами 
EGI и~PRACE, а~так\-же средствами, образуемыми на основе виртуализации 
вычислительного оборудования центров данных и~кластерных платформ. 

Специальная 
программа в~рамках EUDAT2020 ориентирована на создание средств оценки качест\-ва 
данных и~сертификации репозиториев данных в~СИД. EUDAT2020 развивает 
мультидисциплинарный подход, охватывая сообщества исследователей в~гуманитарных 
областях и~в~социальных сетях (CLARIN~--- Common Language Resources and Technology Infractructure, 
DARIAH, CESSDA), в~науках о~Земле 
и~атмосфере (EPOS~--- European Plate Observing System, ICOS, EMSO, VERCE, IAGOS, DRIHM), 
науке о~климате (ENES~--- European Network for Earth System), 
биоразнообразии (LifeWatch, LTER, iMarine), науке о~жизни (VPH, ELIXIR, BBMRI, 
ECRIN, INCF, DiXA) и~физике (EISCAT, EURO-VO, ISIS, \mbox{WLCG}, PaNdata). Значительное 
внимание в~проекте будет уделено динамическим данным и~научным потокам работ, 
созданию сервисов управ\-ле\-ния динамическими данными, оставаясь в~рамках СИД. Эти 
исследования будут опираться на сценарии динамического использования данных из ENES 
и~EPOS и~обобщения их для анализа будущих динамических данных при решении 
реальных научных задач. Одним из планируемых результатов будет создание модели 
и~языка представления жизненного цикла данных, сервисных инфраструктур 
и~происхождения данных. Одновременно будут происходить исследования 
инфраструктурных операций более эффективных, надежных, устойчивых и~близких 
к~потребностям научных сообществ. Примерами планируемых задач являются следующие: 
оценка объектно-ориентированной среды хранения для машин баз данных центров данных 
при создании масштабируемой и~интероперабельной СИД на основе облачных решений 
(одной из целей этого анализа является определение возможности реализации B2SHARE 
без использования iRODS); расширение возможностей уровня долговременного хранения 
путем применения распределенной графовой базы данных для поддержки отношений 
между объектами данных вместо собственной базы метаданных, используемой 
в~настоящее время в~B2SHARE-сер\-ви\-се (по замыслу это должно сблизить подходы в~СИД 
с~применениями семантического веба, поддержкой происхождения данных 
и~семантического аннотирования).
{\looseness=1

}

\subsection{Инфраструктура проекта <<Национальные сервисы  данных>>}

  США и~ряд международных научных сообществ нуждаются в~унификации структур 
и~сервисов для хранения, совместного использования, публикации, размещения 
и~верификации данных. Нужны\linebreak стандартные средства доступа к~данным, програм\-мному 
обеспечению, метаданным, инструментам и~иным компонентам, характерным для многих 
дисциплин. Отсутствие таких стандартных средств создает трудности при проведении 
исследований\linebreak и~репродуцировании опубликованных научных результатов. США 
планируют открытую инфраструктуру для поддержки интегрированного набора\linebreak сервисов 
национального масштаба для эффективного, удобного и~безопасного хранения, 
совмест\-ного использования, пуб\-ли\-ка\-ции, обнаружения, верификации и~атрибуции данных 
на уровне индивидуальных, групповых и~кооперативных потребностей. Именно такую 
инфраструктуру и~сервисы формирует проект NDS~[29] (см.\ рисунок).

\begin{figure*}
\vspace*{1pt}
 \begin{center}
 \mbox{%
 \epsfxsize=132.238mm
 \epsfbox{kal-1.eps}
 }

 
% \vspace*{-9pt}
 
{\small Среда NDS}
 \end{center}
 \vspace*{6pt}
\end{figure*}

  Международные партнеры, в~особенности Research Data Alliance (Альянс 
исследовательских\linebreak данных)~--- RDA, будут способствовать NDS в~обеспечении 
транспарентного, глобального доступа к~данным.
  
\subsection{Альянс исследовательских данных}

Альянс исследовательских данных был образован для поддержки совместного использования данных сквозь барьеры 
в~2013~г. Ядро организаторов включало Европейскую комиссию, National
Science Foundation, NIST, Министерство инноваций Австралии. В~настоящее время 
число членов альянса превышает~2600 из~90~стран. В~рамках альянса образовано 
большое число рабочих групп и~групп по интересам. Дважды в~год организуются 
пленарные совещания в~различных местах мира. Например, в~марте 2015~г.\ на совещании 
в~Сан-Дие\-го рас\-смат\-ри\-ва\-лись крупномасштабные инфраструктурные проекты 
организации и~анализа данных (включая EUDAT, DataOne, CLARIN, Supercomputing and 
Big Data, ELIXIR, NDS и~др.). Пока еще RDA находится в~состоянии обсуждения 
и~уточнения целей альянса.

\subsection{Проекты обеспечения доступа к~ожидаемым данным (на примере 
астрономии)}

  Разнообразные проекты (миссии) в~мире в~различных предметных областях, 
рассматриваемые в~настоящем обзоре, недавно начали получать данные или планируют 
начать получать их до либо после~2020~г. В~разных странах исследователи 
в~соответствующих областях X-ин\-фор\-ма\-ти\-ки уже начали или подготавливают 
исследования инфраструктур, поддерживающих доступ к~данным, их анализ и~управ\-ле\-ние 
данными в~подобных проектах (миссиях). В~настоящем обзоре астрономия выбрана для 
того, чтобы показать примеры подобных исследований, относящихся к~проекту LSST, 
миссии Gaia, а~также к~проекту ASTERICS,

\subsubsection{Подготовка к~доступу к~данным в~проекте LSST}

  В марте 2015~г.\ заключено партнерское соглашение между Institut National de Physique 
Nucl$\acute{\mbox{e}}$aire et de Physique des Particules (IN2P3), Корпорацией LSST, проектом LSST, а~также 
NCSA (Национальным центром суперкомпьютерных приложений Иллинойского 
университета) о~вкладе в~доступ и~обработку версий данных LSST во время формирования 
телескопом обзоров неба~[30]. По этому соглашению IN2P3 должен реализовать операции 
обработки данных LSST посредством коммуникаций, средств обработки данных, а~так\-же 
персонала для образования годичных версий обзора в~узле обработки, сателлитном по 
отношению к~Архивному центру NCSA LSST. Цель проекта CNRS/LSST заключается 
в~том, чтобы предоставить получаемые LSST данные ученым и~более широкой аудитории 
в~мире в~виде двух видов данных: (а)~извещения о~транзиентах, посылаемые 
в~течение~60~с после завершения формирования изображения; (б)~годичные релизы 
данных, которые будут содержать наиболее полно обработанные данные обзора. Каталог\linebreak 
годичного релиза будет состоять из более~100~таб\-лиц, самыми важными из которых 
будут каталог объектов, суммирующий для каждого физического источника всю 
информацию, накопленную за вре-\linebreak мя действия проекта, а~также исходный каталог, 
обеспечивающий доступ к~данным каждого конкретного наблюдения одного объекта за 
одну экспозицию. Данные будут представлены в~виде, при\linebreak котором алгоритмы их анализа 
смогут сосредоточиться на извлечении знаний из каталогов, накопленных в~базе данных 
без необходимости до\-ступа к~первоначальным пикселам. Согласно\linebreak проекту, 
предполагается применить мас\-сив\-но-па\-рал\-лель\-ную реляционную технологию баз 
данных (основанную на принципах архитектуры shared nothing), которая при текущем 
уровне развития оценивается как более эффективная, чем Map-Reduce. Предварительные 
измерения производительности и~масштабируемости были проведены в~проекте LSST на 
различных кластерах: от~20~узловых 100-те\-ра\-байт\-ных кластеров до~300~узловых 30-те\-ра\-байт\-ных
кластеров 
с~таблицами, содержащими порядка~50~млрд строк.

\subsubsection{Подготовка к~доступу к~данным миссии Gaia}

  Космический аппарат Gaia в~среднем за день передает 40~ГБ данных. К~концу миссии 
ожидается накопление данных, превышающих~1~ПБ. Для доступа к~данным 
созданный для обработки и~анализа данных Европейский научный консорциум (DPAC) 
образовал шесть центров обработки данных (DPC), разбросанных по Европе: Мадрид 
(DPCE), Тулуза (DPCC~--- Data Processing and Coordinating Center), 
Кембридж (DPCI), Турин (DPCT), Барселона (DPCB) и~Женева 
(DPCG)~[31].
  
  Различаются два вида DPC: (а)~основанные на <<инфраструктурном пакете>>, 
соединенном с~централизованной файловой системой (DPCE, DPCT, DPCB, DPCG); 
(б)~использующие Hadoop (DPCC и~DPCI). DPCC ориентирован на обработку спектров, 
в~конце миссии планируется установка в~нем~6000~ядер в~кластере с~2-ги\-га\-бит\-ной 
сетью для связи между узлами и~10-ги\-га\-бит\-ной сетью между стойками (racks).
  
\subsubsection{Подготовка к~доступу к~разнообразным данным в~комплексном 
проекте исследовательской инфраструктуры ASTERICS}

  В 2015~г.\ стартовал координируемый ESFRI и~финансируемый Horizon~2020 
комплексный проект ASTERICS ({\sf http://www.asterics2020.eu}). Это первый проект, 
в~котором совместно рас\-смат\-ри\-ва\-ют\-ся проблемы астрономии, астрофизики и~физики 
космических частиц. В~проекте\linebreak будут использованы различные координируемые ESFRI 
инфраструктуры (включая SKA, массив телескопов Черенкова CTA ({\sf  
https://portal.cta-obs\linebreak ervatory.org/Pages/Home.aspx}), 
глубоководный\linebreak
 нейт\-рин\-ный телескоп 
KM3NeT ({\sf http://www.km3 net.org}), гигантский, создаваемый ESO телескоп \mbox{E-ELT} ({\sf 
http://www.eso.org/public/unitedkingdom/ teles-instr/e-elt}) и~другие проекты). Основные цели 
ASTERICS~--- поддержка и~ускорение реализации телескопов, находящихся в~ведении 
ESFRI, и~обеспечение их интероперабельности в~рамках интегрированного, 
многочастотного и~многоцелевого посредника. Основные ожидаемые результаты 
четырехлетнего проекта включают: создание технологий для обеспечения надежного 
и~гибкого манипулирования гигантскими потоками данных, генерируемыми названными 
выше инфраструктурами, охватываемыми ASTERICS, адап\-та\-цию и~оптимизацию систем 
управления огромными базами данных для нужд создаваемой инфраструктуры, адап\-та\-цию 
средств виртуальной обсерватории IVOA для использования в~результирующей 
инфраструктуре. Кроме того должны быть проведены исследования возможности анализа 
данных в~создаваемой инфраструктуре, применяя средства статистического анализа и~data 
mining над коллекциями данных петабайтного масштаба.

\section{Заключение}

  Практически во всех ОИИД данные становятся стратегическим ресурсом, 
затрагивающим все сферы деятельности людей и~определяющим конкурентоспособность, 
уровень развития науки, промышленности, здравоохранения, обороноспособности страны. 
Анализ состояния пяти представительных областей науки в~обзоре показал следующее.
  
  Новизна ситуации заключается в~том, что повсеместно в~мире развивается процесс 
образования петабайтных коллекций данных как результат применения новых 
высокотехнологичных наземных или космических миссий и~инструментов в~крупных 
программах (инициативах) исследований, посвященных изучению разнообразных явлений 
окружающей среды в~различных ОИИД. В~некоторых областях массивные коллекции 
данных образуются как результат интеграции большого числа относительно небольших 
баз данных, создаваемых в~различных исследовательских лабораториях мира. В~США 
получение петабайтных коллекций данных часто является одним из естественных 
результатов стратегических инициатив, объявляемых на уровне Президента США, 
вовлекающих большое число государственных ведомств и~ведущих исследовательских 
центров в~выполнение соответствующих проектов. В~Европейском Союзе подобные 
программы являются межгосударственными. 
  
  В России практически нет межведомственных крупных исследовательских программ, 
которые требовали бы создания новейших инструментов изучения природных явлений, 
а~также крупных международных информационных инфраструктур для накопления 
и~анализа данных (например, со странами БРИКС и~ШОС). Большая часть 
исследовательских проектов организуется инициативно в~рамках межличностных, 
академических и~университетских связей. Потребности в~научных данных\linebreak в~стране не 
формируются системно государственными органами науки, ими не регулируются 
процессы дублирования действий разных ведомств,\linebreak научных институтов и~университетов 
в~области накопления, стандартизации и~контроля качества данных.
  
  В результате вклад России в~мировые коллекции данных незначительный; более того, 
трудно прогнозировать изменение ситуации в~{ближайшие} 10~лет ввиду неразвитости 
соответствующих технологий в~стране и~отсутствия возможности образования адекватных 
программ, тре\-бу\-ющих значительных ассигнований. Таким образом, одной из\linebreak важнейших 
проблем сохранения уровня научных исследований в~России является обеспечение\linebreak 
возможности эффективного доступа исследовательских организаций России к~данным, 
накапливаемым в~мире. Доступ к~центрам данных, размещенным на территории 
иностранных государств, требует решения ряда серьезных технических проблем, а~так\-же 
преодоления политических и~финансовых ограничений (часто требующих заключения 
международных соглашений). Эффективный доступ означает возможность проведения 
анализа данных с~темпом их предоставления для ученых в~мире. При этом следует 
понимать, что совершенно недостаточно создания методов решения типовых классических 
проб\-лем~--- статистических, машинного обучения, data mining и~пр. Опыт показывает, 
что в~конкретных ОИИД каждая конкретная задача анализа данных, особенно больших, 
требует проведения исследований и~экспериментов для создания специального подхода 
к~решению задачи, по возможности опираясь на типовые методы.
  
  Анализ показал, что, в~отличие от России, за рубежом идет активная подготовка 
к~использованию новых источников данных (примеры подготовки даны в~разд.~7), 
включая обсуждение и~планирование проектов новых информационных инфраструктур 
(таких как, например, ASTERICS, NDS, EUDAT2020, RDA, DataONE, MDF, ELIXIR и~др.), 
создание и~отработка элементов таких инфраструктур (например, для анализа данных, 
которые нач\-нут поступать в~ближайшее время (миссия Gaia), или не ранее чем через пять 
лет (телескоп LSST), или по завершении проекта (ASTERICS)). Для этого в~каждом 
крупном проекте образуются большие международные междисциплинарные сообщества 
специалистов, рабочие группы для спецификации новых функций, которые должны 
поддерживаться новыми инфраструктурами.
  
  Приведенные в~обзоре примеры коллекций данных, создаваемых в~мире, инфраструктур 
формирования новых коллекций данных в~процессе исследований предполагается 
использовать в~качестве ориентира при планировании создания и~развитии 
исследовательских инфраструктур для накопления и~анализа данных в~России, 
совместимых с~зарубежными открытыми инфраструктурами данных в~науке. В~частности, 
рассматриваемые в~обзоре коллекции данных, цели их создания и~научные исследования, 
планируемые к~осуществлению с~их помощью, позволяют планировать создание в~России 
компонентов перспективных ИКТ-инфра\-струк\-тур, таких как, например, средства 
концептуализации ОИИД, необходимые метамодели, средства обеспечения возможности 
повторного использования коллекций данных, воспроизводимости программ и~потоков 
работ и~др.
  
  Для достижения эффективного доступа исследовательских организаций России 
к~данным, накапливаемым в~мире, с~целью их использования в~исследовательских 
проектах России представляется целесообразной организация целевой 
междисциплинарной программы для реализации пилотного проекта распределенной 
инфраструктуры для накопления и~анализа данных, совместимой с~зарубежными 
открытыми инфраструктурами в~науке.
  %
  Предполагается, что программа должна включать решение следующих основных задач: 
  \begin{itemize}
\item анализ и~выбор вариантов инфраструктур и~платформ для поддержки решения 
задач анализа больших данных в~различных \mbox{ОИИД}, а~так\-же для обеспечения 
доступа исследователей к~разнообразным видам данных \mbox{в~мире} и~совместного 
междисциплинарного их исполь\-зования (наряду с~техническими проблемами,\linebreak в~том 
числе коммуникационными, предполагается решение на международном уровне 
правовых и~финансовых проблем, вызываемых\linebreak установленными ограничениями 
доступа к~конкретным коллекциям данных);
\item организация рабочих групп и~формирование сообществ в~различных областях 
с~интенсивным использованием данных, принятие мер для установления контактов 
с~международными сообществами аналогичного назначения;
\item создание высокопроизводительного междисциплинарного центра 
интенсивного использования данных (МЦИИД) для исследователей и~практиков из 
разнообразных ОИИД, накопление междисциплинарного опыта создания подходов 
к~решению конкретных задач анализа данных в~конкретных ОИИД, реализация 
проектов с~интенсивным использованием данных в~МЦИИД, выработка 
предложений по тиражированию МЦИИД в~стране, их интероперабельности 
и~размещению в~составе распределенной междисциплинарной инфраструктуры 
совместного использования данных.
\end{itemize}


{\small\frenchspacing
 {%\baselineskip=10.8pt
 \addcontentsline{toc}{section}{References}
 \begin{thebibliography}{99}

\bibitem{1-kal}
The fourth paradigm: Data-intensive scientific discovery~/
Eds. T.~Hey, S.~Tansley, K.~Tolle.~--- Redmond, WA, USA: Microsoft Research, 2009. 284~p.
{\sf http:// goo.gl/edvr6W}.
\bibitem{2-kal}
\Au{Juric M., Tyson T.} LSST data management: Entering the era of petascale optical astronomy~// 
High. Astron., 2015. Vol.~16. P.~675.
\bibitem{3-kal}
\Au{Taylor A.\,R.} Data intensive radio astronomy en route to the SKA: The rise of big radio data~// 
High. Astron., 2015. Vol.~16. P.~677.
\bibitem{4-kal}
\Au{Fleming S.\,W., Abney F., Donaldson~T., \textit{et al.}} Beyond the prime directive: The MAST 
discovery portal and high level science products~// American Astronomical Society Meeting (AAS~225), 
2015. \#336.59. 
\bibitem{5-kal}
\Au{Zhelenkova O., Vitkovsky~V., Plyaskina~T.} Electronic archive of observational data of 
astrophysical observatory~// Russ. J.~Digital Libraries, 2010. Vol.~13. Iss.~4. {\sf 
http:// www.elbib.ru/index.phtml?page=elbib/rus/journal/ 2010/part4/ZVP}.
\bibitem{6-kal}
\Au{Kardashev N.\,S., Khartov~V.\,V., Abramov~V.\,V., \textit{et al.}} 
``RadioAstron''~--- a~telescope with a~size 
of~300\,000~km: Main parameters and first observational results~// Astron. Rep., 2013. Vol.~57. 
Iss.~3. P.~153--194.
\bibitem{7-kal}
\Au{Shustov B.\,M., Gomez de Castro~A.\,I., Sachkov~M., \textit{et al.}} WSO-UV progress and 
expectations~// Astrophys. Space Sci., 2014. Vol.~354. Iss.~1. P.~155--161.
\bibitem{8-kal}
\Au{Кардашёв Н.\,С.,  Новиков~И.\,Д.,  Лукаш~В.\,Н.  и~др.}
Обзор научных задач для обсерватории Миллиметрон~// УФН, 2014. Т.~184. №\,12. С.~1319--1352.

\bibitem{9-kal}
Why neuroinformatics? International Neuroinformatics Coordinating Facility. {\sf 
http://www.incf.org/about/why-neuroinformatics}.
\bibitem{10-kal}
Human Brain Project. {\sf https://www.humanbrainproject. eu}.
\bibitem{11-kal}
Human Connectome Project. WU-Minn HCP 500 Subjects Data Release: Reference manual. 2014. 166~p. 
{\sf http:// www.humanconnectome.org/documentation/S500/\linebreak HCP\_S500\_Release\_Reference\_Manual.pdf}.
\bibitem{12-kal}
\Au{Hawrylycz M.\,J., Lein~E.\,S., Guillozet-Bongaarts~A.\,L., \textit{et al.}} An anatomically 
comprehensive atlas of the adult human brain transcriptome~// Nature, 2012. Vol.~489. P.~391--399.
\bibitem{13-kal}
\Au{Gomez-Cabrero D., Abugessaisa~I., Maier~D., Teschendorff~A., Merkenschlager~M., Gisel~A., 
Ballestar~E., Bongcam-Rudloff~E., Conesa~A., Tegner~J.} Data integration in the era of omics: Current 
and future challenges~// BMC Syst. Biol., 2014. Vol.~8. Suppl.~2. P.~I1.
\bibitem{14-kal}
\Au{Greene C.\,S., Tan J., Ung~M., Moore~J.\,H., Cheng~C.} Big data bioinformatics~// J.~Cell. 
Physiol., 2014. Vol.~229. Iss.~12. P.~1896--1900.

\bibitem{16-kal} %15
\Au{Herland M., Khoshgoftaar~T.\,M., Wald~R.} A~review of data mining using big data in health 
informatics~// J.~Big Data, 2014. Vol.~1. Iss.~2. 35~p.

\bibitem{15-kal} %16
\Au{Kamesh D.\,B.\,K., Neelima~V., Ramya Priya~R.} A~review of data mining using bigdata in health 
informatics~// Int. J.~Sci. Res. Publ., 2015. Vol.~5. Iss.~3. 35~p.

\bibitem{17-kal}
Genome 10K community of scientists. Genome 10K: A~proposal to obtain whole-genome sequence 
for~10\,000 vertebrate species~// J.~Heredity, 2009. Vol.~100. Iss.~6. P.~659--674.
\bibitem{18-kal}
\Au{Davis-Dusenbery B., Onder~Z., Locke~D., Kural~D.} Petabyte-scale cancer genomics in the 
cloud~// TCGA Symposium  Oral Presentations, 2015. P.~34.
\bibitem{19-kal}
Materials Genome Initiative for Global Competitiveness. 2011. {\sf 
http://www.whitehouse.gov/sites/default/ \linebreak 
files/microsites/ostp/materials\_genome\_initiative-final.pdf}.
\bibitem{20-kal}
The Materials Data Facility. {\sf http://www.\linebreak nationaldataservice.org/mdf}.
\bibitem{21-kal}
Versailles Project on Advanced Materials and Standards (VAMAS). http://www.vamas.org.
\bibitem{22-kal}
\Au{Belov G.\,V., Iorish~V.\,S., Yungman~V.\,S.} IVTANTHERMO for Windows~--- database on 
thermodynamic properties and related software~// CALPHAD, 1999. Vol.~23. Iss.~2. P.~173--180.
\bibitem{23-kal}
\Au{Киселева Н.\,Н., Дударев~В.\,А., Земсков В.\,С.}
Компьютерные информационные ресурсы неорганической химии и материаловедения~//
Усп. хим., 2010. Т.~79. Вып.~2. С.~162--188.

\bibitem{24-kal}
Copernicus. Observing the Earth. {\sf 
http://www.esa.int/ Our\_Activities/Observing\_the\_Earth/Copernicus/\linebreak Overview3}.
\bibitem{25-kal}
\Au{Ramapriyan H.\,K., Behnke~J., Sofinowski~E., Lowe~D., Esfandiari~M.\,A.} Evolution of the Earth 
Observing System (EOS) data and Information System (EOSDIS)~// Standard-based data and information 
systems for Earth observation~/ Eds. L.~Di, H.\,K.~Ramapriyan.~--- Lecture notes in geoinformation and 
cartography ser.~--- Berlin--Heidelberg: Springer, 2010. P.~63--92.
\bibitem{26-kal}
\Au{Schnase J.\,L., Duffy D.\,Q., McInerney~M.\,A., \textit{et al.}} Climate analytic as a~service~// 
Conference on Big Data from Space (BiDS'14) Proceedings.~--- Luxembourg: Publications Office of the 
European Union, 2014. P.~90--93.
\bibitem{27-kal}
\Au{Dubernet M.\,L., Boudon~V., Culhane~J.\,L., \textit{et al.}} Virtual atomic and molecular data 
centre~// J.~Quant. Spectrosc. Ra. Transfer, 2010. Vol.~111. Iss.~15. P.~2151--2159.
\bibitem{28-kal}
\Au{Rixon G., Dubernet~M.-L., Piskunov~N., \textit{et al.}} VAMDC~--- the Virtual Atomic and 
Molecular Data Centre~--- a~new way to disseminate atomic and molecular data~--- VAMDC Level~1 
Release~// J.~Phys. Conf. Ser., 2011. Vol.~1344. P.~107--115.
\bibitem{29-kal}
National Data Service (NDS). {\sf http://www.\linebreak nationaldataservice.org}.
\bibitem{30-kal}
\Au{Gangler E.} Big data challenge posed by the Large Synoptic Survey Telescope~//  Conference on Big 
Data from Space (BiDS'14) Proceedings.~--- Luxembourg: Publications Office of the European Union, 
2014. P.~194--197.
\bibitem{31-kal}
\Au{Frezouls B., Brunet~P.-M.} Big data technology in the service of the Gaia data processing~//  
Conference on Big Data from Space (BiDS'14) Proceedings.~--- Luxembourg: Publications Office of the 
European Union, 2014. P.~198--201.
\end{thebibliography}

 }
 }

\end{multicols}

\vspace*{-3pt}

\hfill{\small\textit{Поступила в~редакцию 02.12.15}}

\vspace*{8pt}

%\newpage

%\vspace*{-24pt}

\hrule

\vspace*{2pt}

\hrule

\vspace*{8pt}



\def\tit{DATA ACCESS CHALLENGES FOR~DATA INTENSIVE RESEARCH IN~RUSSIA}

\def\titkol{Data access challenges for data intensive research in Russia}

\def\aut{L.\,A.~Kalinichenko$^{1,2}$, A.\,A.~Volnova$^3$, E.\,P.~Gordov$^4$, 
N.\,N.~Kiselyova$^5$,  D.\,A.~Kovaleva$^6$,\\ O.\,Yu.~Malkov$^6$, I.\,G.~Okladnikov$^4$, 
N.\,L.~Podkolodnyy$^7$,  A.\,S.~Pozanenko$^3$, N.\,V.~Ponomareva$^8$,\\ 
S.\,A.~Stupnikov$^1$, and~A.\,Z.~Fazliev$^9$}

\def\autkol{L.\,A.~Kalinichenko, A.\,A.~Volnova, E.\,P.~Gordov, et al.} 
%N.\,N.~Kiselyova$^5$,  D.\,A.~Kovaleva$^6$,O.\,Yu.~Malkov$^6$, I.\,G.~Okladnikov$^4$, 
%N.\,L.~Podkolodnyy$^7$,  A.\,S.~Pozanenko$^3$, N.\,V.~Ponomareva$^8$,  S.\,A.~Stupnikov$^1$, and~A.\,Z.~Fazliev$^9$}

\titel{\tit}{\aut}{\autkol}{\titkol}

\vspace*{-9pt}

\noindent
$^1$Institute of Informatics Problems, 
Federal Research Center ``Computer Science and Control'' 
of the Russian\linebreak
$\hphantom{^1}$Academy of Sciences, 44-2~Vavilov Str., Moscow 119333, Russian Federation

\noindent
$^2$Faculty of Computational Mathematics and Cybernetics, M.\,V.~Lomonosov Moscow State 
University, 1-52~Lenin-\linebreak
$\hphantom{^1}$skiye Gory, GSP-1, Moscow 119991, Russian Federation

\noindent
$^3$Space Research Institute of the Russian Academy of Sciences, 84/32~Profsoyuznaya Str., 
Moscow 117997,\linebreak
$\hphantom{^1}$Russian Federation 
       
\noindent
$^4$Siberian Center for Environmental Research and Training, Institute of Monitoring of 
Climatic and Ecological\linebreak
$\hphantom{^1}$Systems of the Siberian Branch of the Russian Academy of Sciences, 
10/3~Akademicheski Av., Tomsk  634055,\linebreak
 $\hphantom{^1}$Russian Federation 
       
\noindent
$^5$A.\,A.~Baikov Institute of Metallurgy and Materials Science of the Russian Academy of 
Sciences,  49~Leninsky Av.,\linebreak
$\hphantom{^1}$GSP-1, Moscow 119991, Russian Federation 
       
\noindent
$^6$Institute of Astronomy of the Russian Academy of Sciences, 48~Pyatnitskaya Str., Moscow 
119017, Russian\linebreak
$\hphantom{^1}$Federation 

\noindent
$^7$Center for Bioinformatics, Federal Research Research Center Institute of 
Cytology and Genetics of the\linebreak
$\hphantom{^1}$Siberian Branch of the Russian Academy of Sciences, 
10~Acad. Lavrentyeva Av., Novosibirsk 630090, Russian\linebreak
$\hphantom{^1}$Federation

\noindent
$^8$Research Center of Neurology, 80~Volokolamskoe Shosse, Moscow 125367, Russian 
Federation 
       

\noindent
$^9$Integrated Information Systems Center, Institute of Atmospheric Optics of the 
Siberian Branch of the Russian\linebreak
$\hphantom{^1}$Academy of Sciences, 1~Acad. Zuev Sq., Tomsk 
634055, Russian Federation 

\def\leftfootline{\small{\textbf{\thepage}
\hfill INFORMATIKA I EE PRIMENENIYA~--- INFORMATICS AND
APPLICATIONS\ \ \ 2016\ \ \ volume~10\ \ \ issue\ 1}
}%
 \def\rightfootline{\small{INFORMATIKA I EE PRIMENENIYA~---
INFORMATICS AND APPLICATIONS\ \ \ 2016\ \ \ volume~10\ \ \ issue\ 1
\hfill \textbf{\thepage}}}

\vspace*{3pt}



\Abste{The goal of this survey is to analyze the global trends of development of massive data 
collections and related infrastructures in the world aimed at the evaluation of the opportunities for 
the shared usage of such collections during research, decision making, and problem solving in 
various data intensive domains (DIDs) in
Russia. The representative set of DIDs selected for the 
survey includes astronomy, genomics and proteomics, neuroscience 
(human brain investigation), 
materials science, and Earth sciences. For each of such DIDs, the
strategic initiatives (or large 
projects) in the USA and Europe aimed at creation of big data collections and the\linebreak\vspace*{-12pt}}

\Abstend{respective 
infrastructures planned up to~2025 are briefly overviewed. The information technology 
projects aimed at the 
development of the infrastructures supporting access to and analysis of such data collections are 
also briefly overviewed. The set of large data collections included into the survey and expected to 
be created soon is planned to be used as a reference point for the design and development of the 
research infrastructures for data management and analysis making them compatible with the 
foreign open research infrastructures. In particular, the data collections considered in the survey, 
the goals of their creation and the researches planned to be accomplished based on them make it 
possible to proceed to the design and implementation of the advanced components of the research 
infrastructures, such as, for example, conceptualization facilities of the application domains to be 
investigated in data intensive research, respective metamodels, components intended for data 
reuse and reproducing of programs and workflows, etc.}

\KWE{fourth paradigm; data intensive domains; research infrastructures; data collections; big 
data}




\DOI{10.14357/19922264160101} %10.14357/1992264150409}

\Ack
\noindent
This survey was partially supported by different grants for groups 
from participating research institutes: for IPI FRC CSC RAS by RFBR grants 13-07-00579, 
14-07-00548, and 16-07-01028; for IOA SB RAS by RFBR grant 13-07-00411; for IMCES SB RAS 
by RFBR grants 13-05-12034 and 14-05-00502; for IMET RAS by RFBR grants 14-07-00819 and 
15-07-00980; for INASAN RAS by RFBR grant 15-02-04053 and by the Presidium of 
RAS Program  P-41; for ICG SB RAS  by RSF grant 14-24-00123; for RCN by RFBR 
grants 15-04-08744 and 15-04-05066; and for SRI (IKI) RAS by RFBR grant 15-02-10203-K. 


%\vspace*{3pt}

  \begin{multicols}{2}

\renewcommand{\bibname}{\protect\rmfamily References}
%\renewcommand{\bibname}{\large\protect\rm References}

{\small\frenchspacing
 {%\baselineskip=10.8pt
 \addcontentsline{toc}{section}{References}
 \begin{thebibliography}{99}
\bibitem{1-kal-1}
{Hey, T., S. Tansley, and K.~Tolle}, eds. 2009. The fourth paradigm: Data-intensive scientific discovery. 
Redmond, WA: Microsoft Research. 284~p.
Available at: {\sf http://goo.gl/edvr6W} (accessed February~1, 2016).
\bibitem{2-kal-1}
\Aue{Juric, M., and T. Tyson.} 2015. LSST data management: Entering the era of petascale optical astronomy. 
\textit{High. Astron.} 16:675.
\bibitem{3-kal-1}
\Aue{Taylor, A.\,R.} 2015. Data intensive radio astronomy en route to the SKA: The rise of big radio data. 
\textit{High. Astron.} 16:677.
\bibitem{4-kal-1}
\Aue{Fleming, S.\,W., F. Abney, T.~Donaldson, \textit{et al.}} 2015. Beyond the Prime Directive: The MAST 
discovery portal and high level science products.
\textit{American Astronomical Society (AAS) Meeting} \#225. 
\#336.59.
\bibitem{5-kal-1}
\Aue{Zhelenkova, O., V. Vitkovsky, and T.~Plyaskina}. 2010. Electronic archive of observational data of 
astrophysical observatory. \textit{Russ. J.~Digital Libraries} 13(4). Available at: {\sf 
http://www.elbib.ru/index.phtml?page=elbib/rus/ journal/2010/part4/ZVP} 
(accessed February~1, 2016).
\bibitem{6-kal-1}
\Aue{Kardashev, N.\,S., V.\,V.~Khartov, V.\,V.~Abramov, \textit{et al.}} 
``RadioAstron''~--- a~telescope with 
a~size of 300\,000~km: Main parameters and first observational results. \textit{Astron. Rep.} 57(3):153--194.
\bibitem{7-kal-1}
\Aue{Shustov, B.\,M., A.\,I.~Gomez de Castro, M.~Sachkov, \textit{et al.}} 2014. WSO-UV progress and 
expectations. \textit{Astrophys. Space Sci.} 354(1):155--161.
\bibitem{8-kal-1}
\Aue{Kardashev, N.\,S., I.\,D. Novikov, V.\,N.~Lukash, 
\textit{et al.}} 2014. Review of scientific topics for the Millimetron space observatory. \textit{Physics-Uspekhi} 
57(12):1199--1228.
\bibitem{9-kal-1}
Why neuroinformatics? International Neuroinformatics Coordinating Facility. Available at: {\sf 
http://www.incf. org/about/why-neuroinformatics} (accessed February~1, 2016).
\bibitem{10-kal-1}
Human Brain Project. Available at: {\sf https://www.\linebreak humanbrainproject.eu} (accessed February~1, 2016).
\bibitem{11-kal-1}
Human Connectome Project. WU-Minn HCP 500 Subjects Data Release: 
Reference manual. Available at: {\sf 
http://goo.gl/FsfmUb} (accessed February~1, 2016).
\bibitem{12-kal-1}
\Aue{Hawrylycz, M.\,J., E.\,S. Lein, A.\,L.~Guillozet-Bongaarts, \textit{et al.}} 2012. An anatomically 
comprehensive atlas of the adult human brain transcriptome. \textit{Nature} 489:391--399.
\bibitem{13-kal-1}
\Aue{Gomez-Cabrero,~D., I.~Abugessaisa, D.~Maier, A.~Teschen\-dorff, 
M.~Merkenschlager, A.~Gisel, E.~Balle\-star, 
E.~Bongcam-Rudloff, A.~Conesa, and J.~Tegner}. 2014. Data integration in the era of omics: Current and future 
challenges. \textit{BMC Syst. Biol.} 8(2):I1.
\bibitem{14-kal-1}
\Aue{Greene, C.\,S., J. Tan, M.~Ung, J.\,H.~Moore, and C.~Cheng}. 2014. Big data bioinformatics. 
\textit{J.~Cell. Physiol.} 229(12):1896--1900.

\bibitem{16-kal-1}
\Aue{Herland, M., T.\,M.~Khoshgoftaar, and R.~Wald}. 2014. A~review of data mining using big data in health 
informatics. \textit{J.~Big Data} 1(2).  35~p.
\bibitem{15-kal-1} %16
\Aue{Kamesh, D.\,B.\,K., V. Neelima, and R.\,R.~Priya}. 2015. A~review of data mining using bigdata in health 
informatics. \textit{Int. J.~Sci. Res. Publ.} 5(3).

\bibitem{17-kal-1}
Genome 10K community of scientists. 2009. Genome 10K: A~proposal to obtain whole-genome sequence for 
10\,000~vertebrate species. \textit{J.~Heredity} 100(6):659--674.
\bibitem{18-kal-1}
\Aue{Davis-Dusenbery, B., Z. Onder, D.~Locke, and D.~Kural}. 2015. Petabyte-scale cancer genomics in the 
cloud. \textit{TCGA Symposium Oral Presentations}. 34.
\bibitem{19-kal-1}
Materials Genome Initiative for Global Competitiveness. Available at: {\sf 
http://www.whitehouse.gov/sites/\linebreak default/files/microsites/ostp/materials\_genome\_initiati\linebreak ve-final.pdf} (accessed 
February~1, 2016).
\bibitem{20-kal-1}
The Materials Data Facility. Available at: {\sf http://www.\linebreak nationaldataservice.org/mdf/} (accessed February~1, 
2016).
\bibitem{21-kal-1}
Versailles Project on Advanced Materials and Standards (VAMAS). Available at: {\sf http://www.vamas.org/} 
(accessed February~1, 2016).
\bibitem{22-kal-1}
\Aue{Belov, G.\,V., V.\,S.~Iorish, and V.\,S.~Yungman}. 1999. IVTANTHERMO for Windows~--- database on 
thermodynamic properties and related software. \textit{CALPHAD} 23(2):173--180.
\bibitem{23-kal-1}
\Aue{Kiselyova, N.\,N., V.\,A. Dudarev, and V.\,S.~Zemskov}. 2010. Computer information resources in inorganic 
chemistry and materials science. \textit{Russ. Chem. Rev.} 79(2):145--166.
\bibitem{24-kal-1}
Copernicus. Observing the Earth. Available at: {\sf 
http://\linebreak www.esa.int/Our\_Activities/Observing\_the\_Earth/\linebreak  Copernicus/Overview3} (accessed February~1, 2016).
\bibitem{25-kal-1}
\Aue{Ramapriyan, H.\,K., J. Behnke, E.~Sofinowski, D.~Lowe, and M.\,A.~Esfandiari}. 2010. Evolution of the 
Earth Observing System (EOS) data and Information System (\mbox{EOSDIS}). \textit{Standard-based data and Information 
systems for Earth observation}. Eds. L.~Di and H.\,K.~Ramapriyan. Lecture notes in geoinformation and 
cartography ser. Berlin--Heidelberg: Springer. 63--92.
\bibitem{26-kal-1}
\Aue{Schnase, J.\,L., D.\,Q. Duffy, M.\,A.~McInerney, \textit{et al.}} 2014. Climate analytic as a~service. 
\textit{Conference on Big Data from Space BiDS'14 Proceedings}. 
Luxembourg: Publications Office of the 
European Union. 90--93.
\bibitem{27-kal-1}
\Aue{Dubernet, M.\,L., V. Boudon, J.\,L.~Culhane, \textit{et al.}} 2010. Virtual atomic and molecular data centre. 
\textit{J.~Quant. Spectrosc. Ra. Transfer} 111(15):2151--2159.
\bibitem{28-kal-1}
\Aue{Rixon, G., M.-L.~Dubernet, N.~Piskunov, \textit{et al.}} 2011. VAMDC~--- the Virtual Atomic and 
Molecular Data Centre~--- a~new way to disseminate atomic and molecular data~--- VAMDC Level~1 Release. 
\textit{J.~Phys. Conf. Ser.} 1344:107--115.
\bibitem{29-kal-1}
National Data Service (NDS). Available at: {\sf http://www. nationaldataservice.org/} (accessed February~1, 2016).
\bibitem{30-kal-1}
\Aue{Gangler, E.} 2014. Big data challenge posed by the Large Synoptic Survey Telescope. Big data technology in 
the service of the Gaia data processing. \textit{Conference on Big Data from Space BiDS'14 Proceedings}. Luxembourg: Publications Office of the 
European Union. 194-197.
\bibitem{31-kal-1}
\Aue{Frezouls, B., and P.-M. Brunet}. 2014. Big data technology in the service of the Gaia data processing. 
\textit{Conference on Big Data from Space BiDS'14 Proceedings}. 
Luxembourg: Publications Office of the 
European Union. 198--201.
\end{thebibliography}

 }
 }

\end{multicols}

\vspace*{-3pt}

\hfill{\small\textit{Received December 2, 2015}}


\Contr

\noindent
\textbf{Kalinichenko Leonid A.} (b.\ 1937)~--- Doctor of Science in physics and mathematics, 
professor; Head of Laboratory, Institute of Informatics Problems, 
Federal Research Center ``Computer Science and Control'' 
of the Russian Academy of Sciences, 44-2 Vavilov Str., Moscow 119333, Russian Federation; 
professor, Faculty of Computational Mathematics and Cybernetics, M.\,V.~Lomonosov Moscow 
State University, 1-52~Leninskiye Gory, GSP-1, Moscow 119991, Russian Federation; 
leonidandk@gmail.com 

\vspace*{3pt}

\noindent
\textbf{Volnova Alina A.} (b.\ 1986)~--- scientist, Space Research Institute of the Russian 
Academy of Sciences, 84/32~Profsoyuznaya Str, Moscow 117997, Russian Federation; 
alinusss@gmail.com
       
\vspace*{3pt}

\noindent
       \textbf{Gordov Evgeny P.} (b.\ 1946)~--- Doctor of Science in physics and 
mathematics, Head of Siberian Center for Environmental Research and Training, Institute of 
Monitoring of Climatic and Ecological Systems of the Siberian Branch 
of the Russian Academy 
of Sciences, 10/3~Akademicheski Av., 
Tomsk 634055, Russian Federation; gordov@scert.ru
       
\vspace*{3pt}

\noindent
       \textbf{Kiselyova Nadezhda N.} (b.\ 1949)~--- Doctor of Science in chemistry, Head of 
Laboratory, A.\,A.~Baikov Institute of Metallurgy and Materials Science of the Russian 
Academy of Sciences,  49~Leninsky Av., GSP-1, Moscow 119991, Russian Federation; 
kis@imet.ac.ru
       
\vspace*{3pt}

\noindent
       \textbf{Kovaleva Dana A.} (b.\ 1973)~--- Candidate of Science (PhD) in physics and 
mathematics, scientist, Institute of Astronomy of the Russian Academy of Sciences, 
48~Pyatnitskaya Str., Moscow 119017, Russian Federation; dana@inasan.ru
       
\vspace*{3pt}

\noindent
       \textbf{Malkov Oleg Yu.} (b.\ 1961)~--- Doctor of Science in physics and mathematics, 
Head of Department, Institute of Astronomy of the Russian Academy of Sciences, 
48~Pyatnitskaya Str., Moscow 119017, Russian Federation; malkov@inasan.ru
       
\vspace*{3pt}

\noindent
       \textbf{Okladnikov Igor G.} (b.\ 1978)~--- Candidate of Science (PhD) in technology, 
senior scientist, Siberian Center for Environmental Research and Training, Institute of 
Monitoring of Climatic and Ecological Systems of the Siberian Branch of the Russian Academy 
of Sciences, 10/3~Akademicheski Av., 
Tomsk 634055, Russian Federation; oig@scert.ru
       
\pagebreak

\noindent
       \textbf{Podkolodnyy Nikolay L.} (b.\ 1952)~--- Head of Center for Bioinformatics, 
Federal Research Research Center Institute of Cytology and Genetics 
of the Siberian Branch of 
the Russian Academy of Sciences, 10~Acad.\ Lavrentyeva Av., Novosibirsk 630090, 
Russian Federation; pnl@bionet.nsc.ru
       
\vspace*{3pt}

\noindent
       \textbf{Pozanenko Alexey S.} (b.\ 1962)~--- Candidate of Science (PhD) in physics and 
mathematics, Head of Laboratory, Space Research Institute of the Russian Academy of Sciences, 
84/32~Profsoyuznaya Str, Moscow, Russian Federation; apozanen@iki.rssi.ru
       
\vspace*{3pt}

\noindent
       \textbf{Ponomareva Natalya V.} (b.\ 1956)~--- Doctor of Science in medicine, Head of 
Group and leading scientist, Research Center of Neurology, 80~Volokolamskoe Shosse, 
Moscow 125367, Russian Federation; ponomare@yandex.ru
       
\vspace*{3pt}

\noindent
       \textbf{Stupnikov Sergey A.}~--- Candidate of Science (PhD) in technology, senior 
scientist, Institute of Informatics Problems, Federal Research Center ``Computer Science and 
Control'' of the Russian Academy of Sciences, 44-2~Vavilov Str., Moscow 119333, Russian 
Federation; sstupnikov@ipi.ac.ru
       
\vspace*{3pt}

\noindent
       \textbf{Fazliev Alexander Z.} (b.\ 1953)~--- Candidate of Science (PhD) in physics and 
mathematics, Head of Integrated Information Systems Center, Institute of Atmospheric Optics of 
the Siberian Branch of the Russian Academy of Sciences, 1~Acad.\ Zuev Sq., Tomsk
634055, Russian Federation; faz@iao.ru

  
\label{end\stat}


\renewcommand{\bibname}{\protect\rm Литература}
 

 

 