\def\stat{kirikov}

\def\tit{МЕЛКОЗЕРНИСТЫЕ ГИБРИДНЫЕ ИНТЕЛЛЕКТУАЛЬНЫЕ 
  СИСТЕМЫ. ЧАСТЬ~2: ДВУНАПРАВЛЕННАЯ ГИБРИДИЗАЦИЯ}

\def\titkol{Мелкозернистые гибридные интеллектуальные 
  системы. Часть~2: Двунаправленная гибридизация}

\def\aut{И.\,А.~Кириков$^1$, А.\,В.~Колесников$^2$, С.\,В.~Листопад$^3$, 
С.\,Б.~Румовская$^4$}

\def\autkol{И.\,А.~Кириков, А.\,В.~Колесников, С.\,В.~Листопад, 
С.\,Б.~Румовская}

\titel{\tit}{\aut}{\autkol}{\titkol}

%{\renewcommand{\thefootnote}{\fnsymbol{footnote}} \footnotetext[1]
%{Работа выполнена при поддержке РФФИ (проект 15-07-02244).}}


\renewcommand{\thefootnote}{\arabic{footnote}}
\footnotetext[1]{Калининградский филиал Федерального исследовательского
  центра <<Информатика и~управ\-ление>> Российской академии наук, 
baltbipiran@mail.ru}
  \footnotetext[2]{Балтийский федеральный университет им.\ И.~Канта; 
  Калининградский филиал Федерального исследовательского
  центра <<Информатика и~управ\-ление>> Российской академии наук, 
avkolesnikov@yandex.ru}
  \footnotetext[3]{Калининградский филиал Федерального исследовательского
  центра <<Информатика и~управ\-ление>> Российской академии наук,  
ser-list-post@yandex.ru}
  \footnotetext[4]{Калининградский филиал Федерального исследовательского
  центра <<Информатика и~управ\-ле\-ние>> Российской академии наук, 
sophiyabr@gmail.com}

\vspace*{-8pt}
  
  
  \Abst{Рассматривается проблематика междисциплинарных инструментариев и~делается 
вывод об актуальности исследований свойства <<зернистости>> гибридов в~информатике. 
Исследованы свойства функциональной и~инструментальной неоднородности сложных задач 
и приведены результаты моделирования мелкозернистых гибридов в~теории схем ролевых 
концептуальных моделей (РКМ). Результаты исследований показаны в~рамках 
лингвистического подхода, суть которого состоит в~трансформации вербализованной 
информации об объ\-ек\-тах-ори\-ги\-на\-лах (сложных задачах) 
  и~объ\-ек\-тах-про\-то\-ти\-пах (методах моделирования), имеющейся в~полиязыках 
профессиональной деятельности, в~объ\-ек\-ты-ре\-зуль\-та\-ты (функциональные гибридные 
интеллектуальные системы). Трансформация направляется эвристиками~--- схемами 
ролевых концептуальных моделей в~неформальной аксиоматической теории. 
Категориальное ядро теории~---  
<<ре\-сурс--свой\-ст\-во--дей\-ст\-вие--от\-но\-ше\-ние>>. Введено понятие двунаправленной 
гибридизации. Рассмотрены ее преимущества и~приведены первые результаты.}
  
  \KW{логико-математический интеллект; гибридные интеллектуальные системы; 
лингвистический подход; теория ролевых концептуальных моделей; мелкозернистые 
гибриды; двунаправленная гибридизация}

\DOI{10.14357/19922264160109} %

\vspace*{-4pt}


\vskip 12pt plus 9pt minus 6pt

\thispagestyle{headings}

\begin{multicols}{2}

\label{st\stat}

\section{Введение}

  В первой половине 1990-х~гг.\ L.~Medsker (Вашингтон, США) были анонсированы 
гибридные интеллектуальные сис\-те\-мы (ГиИС), по существу\linebreak совпадающие 
с~интеллектуальными гибридными сис\-те\-ма\-ми~[1], гибридными интегрированными 
сис\-те\-ма\-ми~[2, 3], гибридными информационными\linebreak сис\-те\-ма\-ми~[4] и~гибридными 
интеллектуальными адап\-тив\-ны\-ми сис\-те\-ма\-ми~[5]. В~России сложилось несколько школ 
в~этой области междисциплинарных знаний: Д.\,А.~Пос\-пе\-ло\-ва--В.\,Б.~Та\-ра\-со\-ва; 
В.\,Н.~Ва\-ги\-на--А.\,П.~Ере\-ме\-ева; Г.\,В.~Рыбиной в~Москве~\cite{3-kir};  
В.\,Ф.~По\-но\-ма\-ре\-ва--А.\,В.~Ко\-лес\-ни\-ко\-ва в~Калининграде~[6]; Н.\,Г.~Ярушкиной 
в~Ульяновске~[7].
{\looseness=-1

} 
  
  Вследствие противоречий между свойствами объективной реальности, процессами, 
явлениями и~событиями окружающего мира и~научной карти\-ной мира, а~так\-же из-за 
разнообразия информации,\linebreak перерабатываемой при решении задач, и~разнообразия знаний об 
объекте и~окружающем мире\linebreak перспективна выработка принципов, переводящих создание 
гибридов из уникальной ре\-мес\-лен\-ной мастерской в~про\-ект\-но-кон\-струк\-тор\-скую 
деятельность, в~частности мелкозернистых функциональных ГиИС (ФГиИС). 

В~сочетании с~лингвистическим подходом 
  к~гиб\-ри\-дизации~\cite{6-kir,8-kir}, развитым из~[9, 10], в~част\-ности аксиоматической 
теорией ролевых концептуальных моделей и~построенной в~ней многоуровневой моделью 
внешнего мира как взаимоувязанной совокупностью ло\-ги\-ко-лингви\-сти\-че\-ских, 
знаковых представлений <<мира измерений>>, <<мира ресурсов, свойств, процессов>>, 
<<мира состояний и~поведения>>, <<мира субъ\-ек\-тов-ре\-ша\-те\-лей (мира задач)>>, 
<<мира решения задач>>, <<мира методов моделирования задач>>, <<мира  
субъ\-ек\-тов-раз\-ра\-бот\-чи\-ков (мира моделирования)>> с~использованием тет\-ра\-ды 
концептов <<ре\-сурс--свой\-ст\-во--дей\-ст\-вие--от\-но\-ше\-ние>>, мелкозернистые ФГиИС 
позволят объединить естественный и~визуальный языки в~ло\-ги\-ко-ма\-те\-ма\-ти\-че\-ских  
и~ви\-зу\-аль\-но-про\-стран\-ст\-вен\-ных рассуждениях над проблемами, что релевантно 
феномену человеческого мышления~[11]. 

\begin{table*}[b]\small
\begin{center}
%\vspace*{2ex}

\begin{tabular}{|p{70mm}|p{70mm}|}
\multicolumn{2}{c}{Определения и~свойства проблем и~сложных задач}\\
\multicolumn{2}{c}{\ }\\[-6pt]
\hline
\multicolumn{1}{|c|}{\textbf{Определение проблемы}} &
\multicolumn{1}{c|}{\textbf{Свойства проблемы}}\\
\multicolumn{1}{|c|}{\textbf{по В.\,Ф.~Спиридонову}}  & 
\multicolumn{1}{c|}{ (психология, В.\,Ф.~Спиридонов~\cite{14-kir})}\\
Проблема (англ.\ complex problem, ill-structured problem)~--- это затруднительные условия 
без явно сформулированной цели или четкая цель, не связанная со сложившимися 
неблагоприятными условиями&
Отсутствие и~необходимость поиска исходной формулировки, комплексное строение, 
<<навязчивый>> и~сетевой характер, непрозрачность, собственная динамика, <<человеческое>> 
измерение, межпредметное содержание\\
\hline
\multicolumn{2}{|c|}{Формула идентичности терминов
<<\textbf{проблема}>>\;$\cong$\;<<\textbf{сложная задача}>>}\\
\hline
\multicolumn{1}{|c|}{\textbf{Определение сложной задачи}} &
\multicolumn{1}{c|}{\textbf{Свойства сложной задачи}}\\
\textbf{Сложная задача} (англ.\ complex problem)~--- взаимодействие в~форме гетерогенной системы 
с~неопределенностью &
(искусственный интеллект, А.\,В.~Колесников,\newline  
И.\,А.~Кириков~[8, 15--17])\newline
Субъективность, системность, неоднородность, динамичность, неопределенность ситуации решения, 
полиязыковой характер, внутренняя несогласованность\\
\hline
\end{tabular}
\end{center}
\end{table*}
  
  Настоящая работа является продолжением 
  \mbox{статьи}~[12] и~призвана сформулировать понятие класса мелкозернистых 
ГиИС в~терминах РКМ как цели двунаправленной гибридизации и~рассмотреть 
ее методологические  и~технологические стороны. 

\vspace*{-6pt}

\section{Объекты-оригиналы гибридизации }

  Решение задач системами управления шло в~условиях изменяющегося мирового рынка, их 
эволюции от тейлоровских к посттейлоровским организациям и~смены научной картины 
мира~\cite{8-kir}. В~то время как стандартизация и~практика автоматизированного 
управления определяли термины <<задача>> (тех\-ни\-ко-эко\-но\-ми\-че\-ская сущность, 
выходная информация, входная информация, алгоритм решения) и~<<комплекс задач>>, 
наука (математика, исследование операций, теория управления, теория принятия решений, 
искусственный интеллект, системный анализ, лингвистика), исследуя методы решения 
в~мире искусственно упрощенных, игровых ситуаций, выработала широкий спектр терминов 
и~мнений~--- от <<простой задачи>> до <<сложной задачи>>, <<проблемы>>. 

В~целях 
упорядочения терминологии в~[12] введена качественная шкала и~мера сложности 
моделирования задач. Тем не менее в~[13] высказывалось мнение, что, несмотря на 
одинаковое содержание терминов <<проблема>> и~<<задача>>, между ними есть 
ситуативная разница. Первый применяется в~ситуациях, когда <<у~нас %\linebreak
 есть проб\-ле\-ма>>, 
а~второй применялся опосредованно в~ситуациях, когда <<необходимо решить %\linebreak 
задачу>>. 
Измерения по шкале сложности задач (<<прос\-тая>>--<<слож\-ная>>) относили проб\-ле\-му 
ближе к~правой границе и~переводили в~сферу коллективного интеллекта. Такое же мнение 
высказал и~В.\,Ф.~Спиридонов~[14], исследуя психологию решения задач и~проблем. 
Сравнительный анализ свойств проблем и~сложных задач (см.\ таблицу) показывает их 
содержательное, смыс\-ло\-вое совпадение. 
  

  
  Исследования информации о~сложных задачах~\cite{12-kir, 17-kir} показали ее 
разнообразие: функциональную и~инструментальную неоднородность.
  
  \textit{Функциональная неоднородность} сложных задач проявляется в~наличии 
взаимосвязанных обла-\linebreak стей однородных па\-ра\-мет\-ров-пе\-ре\-мен\-ных: детерминированных, 
стохастических, лингвистических, ге\-не\-ти\-че\-ских,~--- на которых аналитическими, 
статистическими, экспертными, нечеткими, нейросетевыми, генетическими представлениями 
заданы при\-чин\-но-след\-ст\-вен\-ные связи в~рассуждениях экспертов. Таким образом, 
функциональная неоднородность связана с~проблемой <<це\-лое--час\-ти>> или <<сложная 
за\-да\-ча\,--\,под\-за\-да\-чи>> (рис.~1). Это явление слабо изучено в~системном анализе 
и~инженерии знаний. Качество ее исследования определяет глубину понимания задачи. 
  
  На рис.~1 модель задачи имеет двухуровневое представление: на макроуровне~--- задача 
как целое и~ее свойства (prb$^u$); на микроуровне~--- система из подзадач~prb$^h$ (светлые 
кружки) и~координирующей задачи prb$^k$ (темный кружок). 
  
  \begin{figure*} %fig1
  \vspace*{1pt}
 \begin{center}
 \mbox{%
 \epsfxsize=115.978mm
 \epsfbox{kir-1.eps}
 }
 \end{center}
 \vspace*{-9pt}
\Caption{Двухуровневое представление сложной задачи:
\textit{1}~--- отношения включения; \textit{2}~---  отношения координации; \textit{3}~---  отношения 
декомпозиции}
\end{figure*}

  Зададим множество языков профессиональной деятельности (ЯПД) 
  $$
  \mathrm{LANG}^p\hm= \{ \mathrm{LANG}^p_1,\ldots , 
\mathrm{LANG}^p_{N_{\mathrm{LANG}}}\}
$$ 
в~системе~$S$ и~соответствие 
$$
\Psi_1\hm\subseteq \mathrm{LANG}^p \times {S}\vert \Psi_1\not= 
\varnothing,
$$
где 
${S}\hm= \{\mathrm{STR_1, STR_2, STR_3}\}$ для модели mod$_1^S$~\cite{11-kir, 17-kir} 
заочных консультаций и~${S}\hm= \{\mathrm{STR_1, \ldots, STR_4}\}$ для модели 
mod$_2^S$~\cite{11-kir, 17-kir} очных консультаций. Одной страте может соответствовать 
более одного $\mathrm{LANG}_1^p \hm\in \mathrm{LANG}^p \vert q \hm= 1,\ldots, 
N_{\mathrm{LANG}}$. Тогда \textit{гетерогенной предметной областью} назовем
  $$
  E^{\mathrm{LANG}} = \langle {S}, \mathrm{LANG}^p, \Psi_1\rangle\,.
  $$
  
  Пусть в~$E^{\mathrm{LANG}}$ есть гетерогенные задачи $\mathrm{PRB}^u\hm= \{ 
\mathrm{prb}_1^u, \ldots, \mathrm{prb}^u_{\mathrm{PRBU}}\}$ и~$\forall\, \mathrm{prb}_l^u$
$\exists\, 
\mathrm{PRB}_l^h\hm= \{ \mathrm{prb}_1^h, \ldots , \mathrm{prb}^h_{N_{lh}}\}$, где $l\hm= 
1,\ldots, N_{\mathrm{PRBU}}$, $\forall\,l (N_{lh}\hm= \mathrm{var}\,y)$, $\mathrm{prb}_l^u\hm\in 
\mathrm{PRB}^u$. Допустим, что\linebreak prb$_l^u$ могут возникать только на $\mathrm{STR}_j\vert 
j\hm= 2,3$ и~STR$_4$ для mod$_1^S$- и~mod$_2^S$-стра\-ти\-фи\-ка\-ций соответственно. 
Зададим соответствия $\Psi_2\subseteq \mathrm{PRB}^u\times {S}$ 
и~$\Psi_3\hm\subseteq \mathrm{PRB}^h\times {S}$, где $\mathrm{PRB}^h\hm= 
  \prod\limits_l^{N_{\mathrm{PRBU}}} \mathrm{PRB}_l^h$, причем более одной  
prb$^h$ есть на STR$_j\vert j\hm= 1,\ldots, 3$  
и~для  
mod$_2^S$-стра\-ти\-фи\-ка\-ции  
$\forall\,  
(\mathrm{prb}_4^u\,\mathrm{STR}_4)\hm\in \Psi_2 \
\exists\, (\mathrm{PRB}_1^h\,\mathrm{STR}_1),
  \exists\, (\mathrm{PRB}_2^h\,\mathrm{STR}_2),\linebreak
  \exists\, (\mathrm{PRB}_3^h\,\mathrm{STR}_3)\hm\in \Psi_3$, где $\mathrm{PRB}_1^h,\ldots, 
\mathrm{PRB}_3^h\hm\subseteq \mathrm{PRB}^h$, а~для mod$_1^S$-стра\-ти\-фи\-ка\-ции 
$\forall\, (\mathrm{prb}_3^u\,\mathrm{STR}_3)\hm\in \Psi_2$\ %\linebreak 
$\exists\, 
(\mathrm{PRB}_1^h\,\mathrm{STR}_1),\ \exists\, (\mathrm{PRB}_2^h\,\mathrm{STR}_2)\hm\in 
\Psi_3$ %\linebreak 
и~$\forall\, (\mathrm{prb}^u_2\,\mathrm{STR}_2)\hm\in \Psi_2\ \exists\, 
(\mathrm{PRB}_1^h\,\mathrm{STR}_1)\hm\in \Psi_3$. 
  
  Тогда \textit{гетерогенной проблемной средой} назовем
  \begin{equation}
  E^u =\langle E^{\mathrm{LANG}}, \mathrm{PRB}^u, \mathrm{PRB}^h, \Psi_2,\Psi_3\rangle\,.
  \end{equation}
    
  Сформулируем свойства~$E^u$. 
  \begin{enumerate}[1.]
  \item  В~силу свойств $E^{\mathrm{LANG}}$ это редуцированное, многоуровневое 
представление системы~$S$, на стратах которого есть свои ЯПД и~задачи. 
  \item  Задачи на страте~STR$_4$ по определению гетерогенные, а на стратах 
STR$_1$--STR$_3$ по определению гомогенные (однородные).
  \item По определению методы решения~prb$^h$ известны, и~на стратах 
STR$_1$--STR$_3$ могут строиться модели решения подзадач. 
  \item По определению методы решения~prb$^u$ на страте~STR$^4$ неизвестны, и~нет 
априори заданных моделей.
  \item В~силу свойств~${S}$ между стратами существуют двунаправленные потоки 
информации, координирующие решение однородных задач в~составе гетерогенной. 
  \end{enumerate}
  
  \vspace*{-6pt}
  
\section{Объекты-прототипы гибридизации}

  \textit{Инструментальная неоднородность} сложных задач проявляется в~разнообразии 
методов решения ее частей~--- подзадач~--- как следствие редукции сложной задачи. Это 
явление из мира субъекта моделирования, оно также еще до конца не изучено.

  В гетерогенной проблемной среде~(1) проявляются ограниченность, достоинства 
и~недостатки метода. Он модифицируется настолько, что превращается в~новый метод. 
Иногда ме\-тод-пер\-во\-осно\-ва перестает существовать, впитывая <<дополнения>> 
и~<<изменения>>, заимствуемые из других методов. Такой процесс можно рассматривать как 
эволюцию популяции методов и~распространить на него подходы генетики. Для этого 
представим методы-сущ\-ности в~макро- и~микроуровнях (рис.~2).

\begin{figure*} %fig2
\vspace*{1pt}
 \begin{center}
 \mbox{%
 \epsfxsize=115.978mm
 \epsfbox{kir-2.eps}
 }
 \end{center}
 \vspace*{-9pt}
\Caption{Двухуровневое представление метода: \textit{1}~---  отношения определения, 
предназначения; \textit{2}~---  отношения ресурсов, ресурсов и~действий; \textit{3}~--- 
отношения включения;  \textit{4}~--- отношения следования;  \textit{5}~---
<<соответствовать>>}
\end{figure*}

  На макроуровне метод~--- ресурс разработчика для решения задачи с~отличительными 
свойствами и~представляется схемой РКМ~\cite{11-kir, 17-kir}:
\begin{multline*}
\mathrm{met}^a=R^{\mathrm{met\,met}}(\mathrm{met, MET}) \circ R^{\mathrm{met\,pr}} 
(\mathrm{met}, \mathrm{ch}_1) \circ{}\\
{}\circ
R^{\mathrm{met\,pr}}(\mathrm{met}, \mathrm{ch}_2) 
\circ R^{\mathrm{met\,pr}}(\mathrm{met},\mathrm{ch}_3)\circ{}\\
{}\circ 
R^{\mathrm{met\,act}} (\mathrm{met, ACT}_1) 
\circ
R^{\mathrm{met\,act}} (\mathrm{met, ACT}_2) \circ{}\\
{}\circ
R^{\mathrm{met\,prb}} (\mathrm{met, prb}^h) \circ
R^{\mathrm{met\,pr}} (\mathrm{met, SPC}^m)\,,
\end{multline*}
где $\mathrm{ch}_i\in \mathrm{CH}\subseteq \mathrm{PR}$~--- 
классификатор (аналитический, статистический, символьный, 
коннекционистский, эволюционный); $\mathrm{ch}_2\hm\in \mathrm{CH}\hm\subseteq
\mathrm{PR}$~--- модель (черный ящик, сис\-те\-ма 
массового\linebreak \mbox{обслуживания}, система автоматического управ\-ле\-ния, 
усло\-вие-дей\-ст\-вие, серый 
ящик, эволюция,\linebreak си\-ту\-а\-ция-ре\-ше\-ние); $\mathrm{ch}_3\hm\in \mathrm{CH}\hm
\subseteq \mathrm{PR}$~--- язык описания (уравнения, алгоритм, 
продукции, матрицы и~др.); $\mathrm{ACT}_1\subseteq \mathrm{ACT}$~--- процедура 
получения решения (прямой, обратный вывод в~экспертных системах, прямое 
распространение в~нейросетях,\linebreak
 нечеткий вывод, методы решения уравнений, машинные 
эксперименты с~генетическим или моделирующим алгоритмом; методы установления 
соответствия на множествах прецедентов, поиска\linebreak
 аналогов и~сохранения единиц опыта 
в~памяти и~др.);  $\mathrm{ACT}_2\subseteq \mathrm{ACT}$~--- процедура обучения 
(обратное 
распространение, алгоритм Кохонена, не\-па\-ра\-мет\-ри\-че\-ское обучение и~др.);   
SPC$^m$~--- 
спецификатор (погрешность, гибридные возможности~\cite{11-kir, 17-kir}, знания 
о~преимуществах и~недостатках, работа с~шумом, адаптивность и~др.~\cite{11-kir, 17-kir});
$R^{\mathrm{met\,met}}$, $R^{\mathrm{met\,act}}$~--- отношения определения;   
$R^{\mathrm{met\,prb}}$~--- отношения предназначения. 
  
  Микроуровневое представление met$^a$ содержит генетическую информацию и~трактует 
метод как отношения модели, языка, процедуры, специфицируемых схемой РКМ: 
  \begin{multline*}
  \mathrm{met}^a= R^{\mathrm{met\,res}}(\mathrm{met, RES})\circ R^{\mathrm{res\,res}} 
(\mathrm{RES, mod})\circ \\
\circ R^{\mathrm{res\,res}}(\mathrm{RES, lang})\circ  R^{\mathrm{res\,act}}
(\mathrm{RES,proc})\circ \\
\circ 
R^{\mathrm{res\,res}}(\mathrm{lang,mod})\circ 
R^{\mathrm{res\,act}}(\mathrm{mod,proc})\circ\\
\circ 
  R^{\mathrm{res\,act}}(\mathrm{lang,proc})\,,
  \end{multline*}
где met~--- метод; $\mathrm{mod}\hm\in \mathrm{MOD}\subseteq \mathrm{RES}$, 
$\mathrm{lang}\hm\in \mathrm{LANG}\hm\subseteq \mathrm{RES}$, $\mathrm{proc}\hm\in 
\mathrm{ACT}$~--- схемы РКМ модели, языка и~процедуры; $R^{\mathrm{res\,res}}$~--- 
отношения ресурсов (<<ме\-тод--мо\-дель>>, <<ме\-тод--язык>>, <<язык--мо\-дель>>); 
$R^{\mathrm{res\,act}}$~--- отношения ресурсов и~действий (<<ме\-тод--про\-це\-ду\-ра>>, 
<<мо\-дель--про\-це\-ду\-ра>>, <<язык--про\-це\-дура>>).

  Составные части <<модель>>, <<язык>> и~<<процедура>> на микроуровне 
($\mathrm{mod}$, $\mathrm{lang}$ и~$\mathrm{proc}$) определяются схемами РКМ:
  \begin{multline*}
  \mathrm{mod}= R^{\mathrm{res\,res}}(\mathrm{mod,RES}) \circ 
R^{\mathrm{res\,res}}(\mathrm{RES,RES})\circ {}\\
{}\circ R^{\mathrm{pr\,pr}}(\mathrm{PR,PR})\circ 
R^{\mathrm{act\,act}}(\mathrm{ACT, ACT})\circ\\
  \circ R^{\mathrm{res\,pr}}(\mathrm{RES,PR}) \circ R^{\mathrm{res\,act}}(\mathrm{RES, 
ACT})\circ{}\\
{}\circ R^{\mathrm{act\,pr}}(\mathrm{ACT,PR})\,;
  \end{multline*}
  
  \vspace*{-12pt}
  
  \noindent
  \begin{multline*}
\mathrm{lang} = R^{\mathrm{res\,res}}(\mathrm{lang,RES})\circ 
R^{\mathrm{res\,res}}(\mathrm{RES,RES}) \circ{}\\
{}\circ R^{\mathrm{pr\,pr}}(\mathrm{PR,PR}) \circ 
R^{\mathrm{act\,act}}(\mathrm{ACT,ACT})\circ\\
\circ R^{\mathrm{res\,pr}}(\mathrm{RES,PR})\circ R^{\mathrm{res\,act}}(\mathrm{RES, 
ACT})\circ{}\\
{}\circ R^{\mathrm{act\,pr}}(\mathrm{ACT,PR})\,;
\end{multline*}

\vspace*{-12pt}

\noindent
\begin{multline*}
\mathrm{proc}= R^{\mathrm{act\,act}}(\mathrm{proc, ACT}) \circ R^{\mathrm{res\,res}} 
(\mathrm{RES, RES}) \circ{}\\
{}\circ R^{\mathrm{pr\,pr}} (\mathrm{PR,PR})\circ 
R^{\mathrm{act\,act}}(\mathrm{ACT,ACT})\circ\\
\circ R^{\mathrm{res\,pr}}(\mathrm{RES, PR}) \circ R^{\mathrm{act\,res}} (\mathrm{ACT, 
RES}) \circ{}\\
{}\circ R^{\mathrm{act\,pr}}(\mathrm{ACT, PR})\,.
\end{multline*}
  
  Таким образом, <<зернистость>> возникает на мик\-ро\-уровне применительно 
к~структурным особенностям метода.
  
  Пусть $\mathrm{Iv}^m= \{ \mathrm{Iv}_1^m, \ldots , \mathrm{Iv}^m_{N_{\mathrm{Iv}m}}\}$~--- множество экспертов в~коллективе 
разработчиков. Введем множество их ЯПД $\mathrm{LANG}^m\hm= \{ 
\mathrm{LANG}_1^m,\ldots$\linebreak $\ldots, \mathrm{LANG}^m_{N_{\mathrm{LANG}m}}\}$ и~установим невзаимно 
однозначное соответствие $\Psi^{\mathrm{LANG,Iv}}:\ \mathrm{LANG}^m\hm\to 
\mathrm{Iv}^m$: не всюду 
определено (есть языки, которыми модельеры не владеют), сюръективно (некомпетентные 
модельеры к~гибридизации не привлекаются), нефункционально (одним и~тем же языком 
владеют несколько экспертов). 
  
  Введем множества макроуровневых и~микроуровневых представлений автономных 
методов: $\mathrm{MET}^a$, $\mathrm{MET}^a$. Зададим взаимно однозначные 
соответствия $\Psi_i^{\mathrm{met\,met}}$ при $i\hm\in \overline{1,5}$ 
и~$(\mathrm{met}^a_{dj}, \mathrm{met}^a_{ql})$, $j\hm=l$; $d,q\hm\in \{ 
\mathrm{An}, \mathrm{St}, \mathrm{Lg}, \mathrm{Li}, \mathrm{Ep}\}$; 
$d\hm= q$: $\Psi_i^{\mathrm{met\,met}}\hm= \mathrm{MET}_d^a\hm\to \mathrm{MET}_q^a$.
  
  Зададим множества кортежей $((\mathrm{met}^a_{dj}, \mathrm{met}^a_{ql}),$\linebreak
  $\mathrm{LANG}_w^m)$, где $w\hm\in \overline{1,N_{\mathrm{LANG}}}$, как соответствия 
$\Psi_i^{\mathrm{MET\,LANG}}:\ \Psi_i^{\mathrm{met\,met}}\hm\to \mathrm{LANG}^m$: не 
всюду определены (не каждый метод <<знаком>> модельерам), сюръективны (модельер 
владеет минимум одним методом), не функциональны (метод выражается на разных языках) 
и~не взаимно однозначны.
  
  \textit{Неоднородной исследовательской средой} назовем
\begin{multline*}
  E^M= {}\\
  {}=\langle \mathrm{LANG}^m, \mathrm{MET}^a, \mathrm{MET}^a, 
\Psi_i^{\mathrm{met\,met}},  \Psi^{\mathrm{MET\,LANG}}\rangle\,.\hspace*{-2pt}
  \end{multline*}
    Данная модель отображает в~ходе гибридизации основные особенности коллективной 
разработки: двухуровневое представление метода моделирования; многообразие методов 
работы с~различными видами знаний; разнообразие языков профессиональной деятельности 
модельеров.

\vspace*{-6pt}
  
\section{Объекты-результаты гибридизации } %4
  
  Актуальность применения в~системах управления мелкозернистых ФГиИС связана 
с~проблемными ситуациями, для которых нет ни знаний, ни опыта и~которые могли бы быть 
отображены в~гетерогенном модельном поле. Причина таких ситуаций~--- уникальность 
проблемы в~сочетании с~усугубляющим трудность ситуации положением, когда известные 
методы не могут быть применены к решению однородных(ой) задач(и). 

Например, пусть для 
решения некоторой однородной задачи подходит экспертная система. Однако поскольку 
предположительно ее база знаний имеет большой размер, то известные алгоритмы ее 
интерпретации будут работать медленно. Такая\linebreak ситуация не позволяет применить при 
гибридизации крупнозернистый элемент, построенный как экспертная система, и~требует 
вмешательства в~мик\-ро\-уровневое представление. Цель~--- заменить отдельные процедуры 
интерпретатора~--- мелкие зерна~--- на, чаще всего, полнофункциональный, т.\,е.\ 
крупнозернистый элемент, построенный по другому методу. Иное объяснение следует из 
необходимости использовать не только классы трансформационных и~ФГиИС, 
как в~случае крупнозернистой гибридизации, но и~полиморфические ГиИС, которые 
сконструировать из крупного зерна невозможно. 

Наконец, в~рамках мелкозернистых 
гибридов проще решать технологические задачи, сопутствующие решению подзадач, 
например: извлечения знаний для экспертной системы из декларативной информации других 
элементов \mbox{ФГиИС}; корректировки знаний элементов ГиИС после того, как 
в~базу знаний одного из элементов были внесены изменения; пополнения исходной 
информации, поступающей на вход элемента как за счет его собственной базы знаний 
и~данных, так и~баз данных и~баз знаний других элементов. Все они требуют разработки 
механизма вмешательства в~декларативную и(или) процедурную составляющие элемента 
\mbox{ФГиИС} априори, еще до синтеза ГиИС. На рис.~3 дана иерархия схем РКМ 
для представления знаний о~мелкозернистых \mbox{ФГиИС}. Для этого в~направлении сверху вниз 
изображены схемы РКМ мелкозернистой \mbox{ФГиИС} ($\alpha^u$), крупнозернистой \mbox{ФГиИС} 
($\alpha^u(t)$), двух вариантов элементов \mbox{ФГиИС} ($\alpha_j^h(t)$), метода на макро- 
(met$^a$) и~микроуровне (met$^a$), модели (mod), языка (lang) и~процедуры (proc) как 
частей метода.
  
  Движение по иерархии сверху вниз~--- детализация, а снизу вверх~--- агрегирование. 
В~месте, где указано отношение <<соответствовать>>, возникает <<разрыв>> включения 
одних знаний в~другие.
  
  На рис.~3 использованы обозначения: RES$^{n-1}$~--- множество эле\-мен\-тов-зе\-рен 
функциональных $\alpha^h\hm\in A^h$ и~технологических $\alpha^\tau\hm\in A^\tau$ 
элементов; $R^{\mathrm{res\,res}}$ и~$\ddot{R}^{\mathrm{res\,res}}$~--- отношения 
включения и~интеграции зерен из RES$^{n-1}$ в~знак ФГиИС res$^n$; \textbf{pr}$^n_1$~--- 
вектор исходных данных~DAT$^u$ задачи~prb$^u$; \textbf{pr}$^n_2$~--- вектор выходных 
данных элементов, цель~GL$^u$ задачи~prb$^u$; \textbf{pr}$^n_3$~--- вектор состояния 
ФГиИС; RES$^n$~--- множество знаков информационного языка; PR$_1^n$ и~PR$_2^n$~--- 
множества свойств <<вход>> и~<<выход>> элементов из~RES$^n$ соответственно; 
$R_6^{\mathrm{pr\,pr}}$, $R_7^{\mathrm{pr\,pr}}$ и~$R_8^{\mathrm{pr\,pr}}$~--- отношения 
функционирования ФГиИС; $\ddot{R}^{\mathrm{res\,res}}$~--- отношения 
интеграции~\cite{11-kir, 17-kir};  $R_9^{\mathrm{pr\,pr}}$ и~$R_{10}^{\mathrm{pr\,pr}}$~--- 
отношения <<входа>> ФГиИС и~<<входов>> элементов, а~так\-же <<выходов>> элементов 
и~<<выхода>> ФГиИС соответственно; $t$~--- автоматное время.
  
     \begin{figure*} %fig3
     \vspace*{1pt}
 \begin{center}
 \mbox{%
 \epsfxsize=156.281mm
 \epsfbox{kir-3.eps}
 }
 \end{center}
 \vspace*{-9pt}
\Caption{Иерархия схем ролевых концептуальных моделей для моделирования знаний о~мелкозернистых 
ФГиИС: \textit{1}~---  отношение замены (<<что>> и~<<на что>> заменяется);  \textit{2}~---  
<<соответствовать>>}
\end{figure*}

  Возникает задача комбинирования на мик\-ро\-уровне. Пусть имеем множества: 
$\mathrm{In}^k\hm= \{ \mathrm{in}_1^k, \mathrm{in}_2^k,\ldots, 
\mathrm{in}^k_{N_{\mathrm{in}}}\}$~--- интерпретаторов моделей вычислений автономных 
методов~\cite{11-kir, 17-kir}); $A^\tau\hm= \{\alpha^\tau_{j1}, \alpha^\tau_{j2},\ldots, 
\alpha^\tau_{jN_{\alpha\tau}}\}$~--- технологических компонентов; 
MOD$^{ak\,\mathrm{Ca}}\hm= \{ 
\mathrm{mod}_1^{ak\,\mathrm{Ca}}, \mathrm{mod}_2^{ak\,\mathrm{Ca}}, \ldots$\linebreak
$\ldots,  
\mathrm{mod}_{N_{\mathrm{met\,Ca}}}^{ak\,\mathrm{Ca}}\}$~--- декларативных составляющих автономных методов; 
ACT$^k\hm= \{\mathrm{act}_1^k, \mathrm{act}_2^k,\ldots$\linebreak $\ldots , 
\mathrm{act}^k_{N_{\mathrm{act}k}}\}$~--- действий разработчика ФГиИС. Тогда имеют 
смысл следующие соответствия:
  \begin{multline*}
  \Psi^{\mathrm{in}\,\alpha} =  \left\{ \left( \mathrm{in}_v^k, \alpha^\tau_{j\omega} 
  \right) \vert \,\,
\mathrm{in}^k_v \in \mathrm{IN}^k\,,\ \alpha^\tau_{j\omega} \in A^\tau\,,\right.\\
 k\in \left\{ \mathrm{An},\mathrm{St},\mathrm{Lg},\mathrm{Li},\mathrm{Ep}\right\}\,,\enskip
 v,j\in \{1,\ldots, 7\}\,,\\
   \left.\omega= 1,\ldots, N_{\alpha,\tau}
  \vphantom{\left( \mathrm{in}_v^k, \alpha^\tau_{j\omega} \right) \mathrm{IN}^k}
  \right\}\,,
  \end{multline*}
  
  \begin{figure*}[b] %fig4
\vspace*{1pt}
 \begin{center}
 \mbox{%
 \epsfxsize=150.888mm
 \epsfbox{kir-4.eps}
 }
 \end{center}
 \vspace*{-9pt}
\Caption{Одно- и~двунаправленная гибридизация в~информатике}
\end{figure*}

 
\noindent
   \begin{multline*}
  \Psi^{\mathrm{met\,Ca}\,\alpha} = {}\\
  {}=\left\{ \left( \mathrm{mod}_1^{ak\,\mathrm{Ca}},\alpha^\tau_{j\omega}\right) 
\vert \mathrm{mod}_l^{ak\,\mathrm{Ca}} \in \mathrm{MOD}^{ak\,\mathrm{Ca}}\,,\right.\\
\left. l\in \{1,\ldots, 7\}
\vphantom{\left( \mathrm{mod}_1^{ak\,Ca},\alpha^\tau_{j\omega}\right) }
\right\}\,,
\end{multline*}



\noindent
$$
  \Psi^{\mathrm{act}k\,\alpha}= \left\{ \left( \mathrm{act}_v^k, \alpha^\tau_{j\omega}\right) \vert 
\mathrm{act}_v^k\in \mathrm{ACT}_k\right\}\,,
  $$
    где in$_v^k$, mod$_l^{ak\,\mathrm{Ca}}$ и~$\alpha_{j\omega}^\tau$~--- 
    части интерпретатора в~модели 
вычислений автономного метода (назовем их \textit{зернами}), декларативной составляющей 
модели вычислений автономного метода и~полнофункциональный технологический элемент 
(назовем их \textit{модулями}) соответственно. 

\vspace*{-6pt}

\section{Двунаправленная гибридизация } %5

  В зависимости от особенностей объектов-ре\-зуль\-та\-тов будем различать и~виды 
гибридизации в~информатике (рис.~4). Исследования гетерогенной проблемной среды на 
искусственных гетерогенных системах класса <<крупнозернистые  
\mbox{ФГиИС}>>~\cite{11-kir, 17-kir} названы \textit{днонаправленной гибридизацией}~\cite{8-kir}. 
Исследования гетерогенной проблемной среды и~гетерогенной исследовательской среды на 
искусственных гетерогенных системах класса <<мелкозернистые  
\mbox{ФГиИС}>>~\cite{8-kir, 11-kir, 17-kir} названы \textit{дву\-на\-прав\-лен\-ной 
гибридизацией}~\cite{8-kir}. 



  Первый вариант связан с~трансформацией информации так, что гиб\-ри\-ды-по\-том\-ки 
как интегрированный метод решения сложной задачи наследуют родительские признаки~--- 
функциональную неоднородность сложной задачи как объ\-ек\-та-ори\-ги\-на\-ла, 
а~родительские признаки методов опосредованно (как есть), через гетерогенное модельное 
поле. Это направление показано номерами~1--4. Ему соответствует однонаправленная 
гибридизация. Ее смысл состоит в~том, что сложная задача должна быть рассмотрена на 
макро- и~микроуровне. Макроуровень (фенотип задачи)~--- вся задача в~целом как сложная 
сущность, система. Микроуровень (генотип сложной задачи)~--- совокупность подзадач, 
связанная в~декомпозициях классами отношений. Макро- и~микроуровневые отображения 
задачи взаимосвязаны и~должны рассматриваться\linebreak
 в~единстве. Однонаправленная 
гибридизация требует исследований и~извлечения знаний о~макро- и~микроуровневом 
представлениях задачи и~их взаимозависимостях. Эти исследования должны\linebreak проводить 
системные аналитики совместно с~экспертами из коллектива, принимающего решения. 

  Второй вариант связан с~трансформацией информации и~наследованием родительских 
признаков как сложной задачи (ее функциональной неоднородности), так и~непосредственно 
(как надо) объ\-ек\-тов-про\-то\-ти\-пов. Это направление показано стрелками~1, 5--9. 
Этому 
направлению соответствует двунаправленная гибридизация. Ее смысл состоит в~том, что 
каждый метод из ограниченной совокупности должен быть рассмотрен на макро- 
и~микроуровне. Макроуровень (фенотип метода)~--- 
метод в~целом как сложная сущность, 
система. Мик\-ро\-уро\-вень (генотип метода)~--- совокупность\linebreak зерен <<модель>>, <<язык 
описания>>, <<процедура получения решения>> или зерен более детального уров-\linebreak ня как 
составных частей процедуры решения. Макро- и~мик\-ро\-уров\-не\-вые отображения метода 
взаимосвязаны и~должны рассматриваться в~единстве. Двунаправленная гибридизация 
помимо работ направления~1--4 требует: исследования и~извлечения знаний о~возможностях 
методов;\linebreak 
исследования и~извлечения знаний о~мак\-ро- и~мик\-ро\-уровневых представлениях 
методов и~взаимозависимостях. Эти исследования должны проводить системные аналитики 
совместно с~экспертами, владеющими методами моделирования (модельерами).

  Каждая из ФГиИС, полученных двунаправленной гибридизацией, должна иметь 
<<сходство>> со сложной задачей, а~для каждой ее части конструируется <<хороший>>  
ме\-тод-по\-томок.

  Преимущества двунаправленной гибридизации:
\begin{enumerate}[(1)]
\item модельная архитектура мелкозернистых \mbox{ФГиИС} отображает на компьютере 
функциональную структуру задачи, области ее однородных параметров 
(детерминированных, стохастических, логических, лингвистических и~др.), представленных 
множествами подзадач, что позволяет вести несколько линий рассуждений;\\[-15pt]
\item для решения подзадач можно использовать не только известные методы моделирования 
(ме\-то\-ды-ро\-ди\-те\-ли), но и~конструировать из набора инструментальных средств (зерен) 
ме\-то\-ды-по\-том\-ки, лишенные родительских недостатков;\\[-15pt] 
\item мелкозернистые ФГиИС разрабатываются методами целенаправленной гибридизации, 
управ\-ля\-емой знаниями о~функциональной и~инструментальной неоднородности сложной 
задачи, позволяющей выявить плюсы и~минусы и~путем декомпозиции процедурной 
со\-став\-ля\-ющей построить наборы типовых инструментальных средств-зе\-рен;\\[-15pt]
\item компьютерная интерпретация модели <<мелкозернистая ФГиИС>> 
может рассматриваться как синтез метода решения задачи, адаптируемого к изменениям во 
внешней среде, условиям задачи, релевантного неоднородности подобных задач, 
наследующего сильные стороны ме\-то\-дов-ро\-ди\-те\-лей, при\-бли\-жа\-юще\-го\-ся к~возможностям 
естественного коллективного интеллекта.
\end{enumerate}

  Для выполнения однонаправленной гибридизации разработана и~апробирована  
проб\-лем\-но-струк\-тур\-ная методология, соответствующая технология 
и~инструментальные средства поддержки~\cite{6-kir, 8-kir}. Для выполнения 
двунаправленной гибридизации разработана и~частично апробирована  
проб\-лем\-но-ин\-ст\-ру\-мен\-таль\-ная методология~\cite{12-kir}. Технология 
и~инструментарий апробированы для узкого класса дигибридов искусственных нейронных 
сетей и~генетических алгоритмов.

\vspace*{-9pt}

\section{Заключение}

  В рамках лингвистического подхода к гибридизации рассмотрено моделирование 
мелкозернистых ФГиИС в~неформальной 
аксиоматической тео\-рии схем ролевых концептуальных моделей. Такие системы обладают 
значительным потенциалом имитации рассуждений специалистов, решающих сложные 
задачи, и~разрабатываются методами целенаправленной гибридизации, управляемой двумя 
видами знаний. Прежде всего, это знания о~функциональной неоднородности сложной 
практической задачи, полученные в~ходе ее системного\linebreak анализа. Второй вид знаний~--- это 
информация о~микроуровневых представлениях методов, полученная в~ходе их системного 
анализа с~использованием триады <<язык--мо\-дель--про\-це\-ду\-ра>> и~позволяющая не 
только выявить и~систематизировать плюсы (силы) и~минусы (слабости), но и~путем 
декомпозиции процедурной составляющей построить наборы типовых инструментальных 
средств-зе\-рен. В~итоге компьютерная интерпретация модели\linebreak
 <<мелкозернистая 
ФГиИС>> рассматривается как синтез метода решения сложной задачи, легко 
адап\-тируемого к изменениям во внешней среде, условиям решения задачи, релевантного 
неоднородности подобных задач и~наследующего сильные стороны  
ме\-то\-дов-ро\-ди\-те\-лей.

\vspace*{-6pt}
  
{\small\frenchspacing
 {%\baselineskip=10.8pt
 \addcontentsline{toc}{section}{References}
 \begin{thebibliography}{99}
\bibitem{1-kir}
\Au{Goonatilake S., Khebbal S.} Intelligent hybrid systems~// 1st Singapore Conference 
(International) on Intelligent Systems Proceedings, 1992. P.~356--364.
\bibitem{2-kir}
\Au{Поспелов Г.\,С.} Искусственный интеллект~--- основа новой информационной 
технологии.~--- М.: Наука, 1988. 280~c.
\bibitem{3-kir}
\Au{Рыбина Г.\,В.} Интегрированные экспертные системы: современное состояние, 
проблемы и~тенденции~// Известия РАН. Теория и~системы управления, 2002. №\,5.  
С.~111--126.
\bibitem{4-kir}
\Au{Medsker L.\,R.} Hybrid intelligent systems.~--- Kluwer Academic Publ., 1995. 295~p.
\bibitem{5-kir}
\Au{Kasabov N., Kozma R.} Hybrid intelligent adaptive systems: A~framework and a~case study 
on speech recognition~// Int. J.~Intell. Syst., 1998. Vol.~13. P.~455--466.
\bibitem{6-kir}
\Au{Колесников А.\,В.} Гибридные интеллектуальные сис\-те\-мы. Теория и~технология 
разработки~/ Под ред. А.\,М.~Яшина.~--- СПб.: СПбГТУ, 2001. 711~c.
\bibitem{7-kir}
\Au{ Ярушкина Н.\,Г.} Основы теории нечетких и~гибридных систем.~--- М.: Финансы 
и~статистика, 2004. 320~с.
\bibitem{8-kir}
\Au{Колесников А.\,В., Кириков И.\,А.} Методология и~технология решения сложных задач 
методами функциональных гибридных интеллектуальных систем.~--- М.: ИПИ РАН, 2007. 
387~с.

\bibitem{10-kir}
\Au{Уемов А.\,И.} Вещи, свойства, отношения.~--- М.: Институт философии АН СССР, 1963. 
184~с.

\bibitem{9-kir}
\Au{Поспелов Д.\,А.} Ситуационное управление: теория и~практика.~--- М.: Наука, 
1986. 288~с.

\bibitem{11-kir}
\Au{Колесников А.\,В.} Моделирование естественных гетерогенных систем 
коллективного принятия решений~// Системный анализ и~информационные 
технологии (САИТ-2015): Тр. 6-й Междунар. конф.~---  М.: ИСА РАН, 2015. Т.~1. 
С.~7--16.
\bibitem{12-kir}
\Au{Колесников А.\,В., Кириков~И.\,А., Листопад~С.\,В., Румовская~С.\,Б.,
Доманицкий~А.\,А.} Решение 
сложных задач коммивояжера методами функциональных гибридных интеллектуальных 
систем~/ Под ред.\ А.\,В.~Колесникова.~--- М.: ИПИ РАН, 2011. 295~с.
\bibitem{13-kir}
\Au{Самсонова М.\,В., Ефимов В.\,В.} Технология и~методы коллективного решения 
проблем.~--- Ульяновск: \mbox{УлГТУ}, 2003. 152~с.
\bibitem{14-kir}
\Au{Спиридонов В.\,Ф.} Психология мышления: Решение задач и~проблем.~--- М.: Генезис, 2006. 319~с.
\bibitem{15-kir}
\Au{Колесников А.\,В., Кириков~И.\,А.} Концептуальная модель двунаправленной 
гибридизации при разработке компьютерных систем поддержки принятия решений~// 
Системы и~средства информатики.~--- М.: Наука, 2008. Доп. вып. С.~21--53.
\bibitem{16-kir}
\Au{Кириков И.\,А., Колесников~А.\,В., Листопад~С.\,В.} Моделирование самоорганизации 
групп интеллектуальных агентов в~зависимости от степени согла\-со\-ван\-ности их 
взаимодействия~// Информатика и~ее \mbox{применения}, 2009. Т.~3. Вып.~4. С.~76--86.
\bibitem{17-kir}
\Au{Колесников А.\,В.} Принципы и~методология разработки информационных гетерогенных 
систем дву\-на\-прав\-лен\-ной гибридизации~// Интегрированные\linebreak
 модели и~мягкие вычисления 
в~искусственном интеллекте: Сб. науч. тр. VIII Междунар. науч.-практич. конф.~--- М.: 
Физматлит, 2015. Т.~1. С.~36--53.

\end{thebibliography}

 }
 }

\end{multicols}

\vspace*{-3pt}

\hfill{\small\textit{Поступила в~редакцию 20.09.15}}

\vspace*{8pt}

%\newpage

%\vspace*{-24pt}

\hrule

\vspace*{2pt}

\hrule

\vspace*{8pt}



\def\tit{FINE-GRAINED HYBRID INTELLIGENT SYSTEMS.\\ PART~2: 
BIDIRECTIONAL HYBRIDIZATION}

\def\titkol{Fine-grained hybrid intelligent systems. Part~2: 
Bidirectional hybridization}

\def\aut{I.\,А.~Kirikov$^1$, А.\,V.~Kolesnikov$^{1,2}$, S.\,V.~Listopad$^1$, 
  and~S.\,B.~Rumovskaya$^1$}

\def\autkol{I.\,А.~Kirikov, А.\,V.~Kolesnikov, S.\,V.~Listopad, 
  and~S.\,B.~Rumovskaya}

\titel{\tit}{\aut}{\autkol}{\titkol}

\vspace*{-9pt}

\noindent
$^1$Kaliningrad Branch of the Federal Research Center ``Computer Science and
   Control'' of the Russian Academy\linebreak
   $\hphantom{^1}$of Sciences, 5~Gostinaya Str.,     Kaliningrad 
   236000,  Russian Federation
      
\noindent
$^2$Immanuel Kant Baltic Federal University, 14~Nevskogo Str., Kaliningrad 236041,
   Russian Federation
   
\def\leftfootline{\small{\textbf{\thepage}
\hfill INFORMATIKA I EE PRIMENENIYA~--- INFORMATICS AND
APPLICATIONS\ \ \ 2016\ \ \ volume~10\ \ \ issue\ 1}
}%
 \def\rightfootline{\small{INFORMATIKA I EE PRIMENENIYA~---
INFORMATICS AND APPLICATIONS\ \ \ 2016\ \ \ volume~10\ \ \ issue\ 1
\hfill \textbf{\thepage}}}

\vspace*{3pt}
 
  
  
  \Abste{The problematic of interdisciplinary tools is considered and the conclusion about the 
relevance of research of the ``grain'' property of hybrids in informatics is made. Properties of 
functional and instrumental heterogeneity of complex tasks are investigated and the results of  
fine-grained hybrids modeling within the theory of the schemes of role conceptual models are 
presented. The results are presented within the linguistic approach, the core of which is in 
transformation of the verbalized information about objects-originals (complex subjects) and 
objects-prototypes (modeling approaches) to objects-results (functional hybrid intelligent systems). 
It exists in poly-languages of professional activity. The transformation is directed by 
heuristics 
which are the schemes of the conceptual role models in the informal axiomatic theory. The 
categorical core of the theory is ``resource--property--operation--relation.'' The notion of 
bidirectional hybridization, its benefits, and the first results are represented.}
  
  \KWE{logical-mathematical intelligence; hybrid intelligent systems; linguistic approach; theory 
of role conceptual models; fine-grained hybrids; bidirectional hybridization}

\DOI{10.14357/19922264160109} 

%\Ack
%\noindent



%\vspace*{3pt}

  \begin{multicols}{2}

\renewcommand{\bibname}{\protect\rmfamily References}
%\renewcommand{\bibname}{\large\protect\rm References}

{\small\frenchspacing
 {%\baselineskip=10.8pt
 \addcontentsline{toc}{section}{References}
 \begin{thebibliography}{99}
\bibitem{1-kir-1}
\Aue{Goonatilake, S., and S. Khebbal}. 1992. Intelligent hybrid systems. \textit{1st Singapore 
Conference (International) on Intelligent Systems Proceedings}. 356--364.
\bibitem{2-kir-1}
\Aue{Pospelov, G.\,S.} 1988. \textit{Iskusstvennyy intellekt~--- osnova novoy informatsionnoy 
tekhnologii} [Artificial intelligence~--- the base of the new information processing technology]. 
Moscow: Nauka. 280~p.
\bibitem{3-kir-1}
\Aue{Rybina, G.\,V.} 2002. Integrated expert systems: State of the art, problems,
and trends]. 
\textit{J.~Comput. Sys. Sc. Int.} 41(5):780--793.
\bibitem{4-kir-1}
\Aue{Medsker, L.\,R.} 1995. \textit{Hybrid intelligent systems}. Kluwer Academic Publ.  295~p.
\bibitem{5-kir-1}
\Aue{Kasabov N., and R. Kozma.} 1998. Hybrid intelligent adaptive systems: A~framework and 
a~case study on speech recognition. \textit{Int.\ J.~Intell. Syst.} 13:455--466.
\bibitem{6-kir-1}
\Aue{Kolesnikov, A.\,V.} 2001. \textit{Gibridnye intellektual'nye sistemy. Teoriya i~tekhnologiya 
razrabotki} [Hybrid artificial systems. Theory and development technology]. St.\ Petersburg: 
\mbox{SPbGTU}. 711~p.
\bibitem{7-kir-1}
\Aue{Jarushkina, N.\,G.} 2004. \textit{Osnovy teorii nechetkikh i~gibridnykh sistem} [Foundations 
of theory of fuzzy and hybrid systems]. Moscow: Finance and Statistics. 320~p.
\bibitem{8-kir-1}
\Aue{Kolesnikov, A.\,V., and I.\,A.~Kirikov}. 2007. \textit{Metodologiya i~tekhnologiya resheniya 
slozhnykh zadach metodami funk\-tsi\-onal'\-nykh gibridnykh intellektual'nykh sistem} [Methodology 
and technology for solving of complex problems using the methodology of functional hybrid 
artificial systems]. Moscow: IPI RAN. 387~p.

\bibitem{10-kir-1}
\Aue{Uemov, A.\,I.} 1963. \textit{Veshchi, svoystva, otnosheniya} [Items, properties, relations]. 
Moscow: Institute of Philosophy of the Academy of Sciences of the USSR. 184~p.

\bibitem{9-kir-1}
\Aue{Pospelov, D.\,A.} 1986. \textit{Situatsionnoe upravlenie: Teoriya i~praktika} [Situation 
control: Theory and practice]. Moscow: Nauka. 288~p.

\bibitem{11-kir-1}
\Aue{Kolesnikov, A.\,V.} 2015. Modelirovanie estestvennykh geterogennykh sistem kollektivnogo 
prinyatiya resheniy [Modeling of the natural heterogeneous systems of collective decision-making]. 
\textit{6th Conference (International) ``Systems Analysis and Information Technology'' 
Proceedings}. 
Moscow: ISA RAS. 1:7--16.
\bibitem{12-kir-1}
\Aue{Kolesnikov, A.\,V., I.\,A. Kirikov, S.\,V.~Listopad, S.\,B. Rumovskaya, and
A.\,A.~Domanitskiy}. 2010. 
\textit{Reshenie slozhnykh zadach kommivoyazhera metodami funktsional'nykh gibridnykh 
intellektual'nykh sistem} [Solving of the complex traveling salesman problem using the 
methodology of functional hybrid artificial systems]. Ed.~ A.\,V.~Kolesnikov. Moscow: IPI RAN. 295~p.
\bibitem{13-kir-1}
\Aue{Samsonova, M.\,V., and V.\,V.~Efimov}. 2003. \textit{Tehnologiya i~metody kollektivnogo 
resheniya problem} [The technology and methods of collective  
decision-making]. Ulyanovsk: UlSTU. 152~p.
\bibitem{14-kir-1}
\Aue{Spiridonov, V.\,F.} 2006. \textit{Psikhologiya myshleniya: Reshenie zadach i~problem} 
[The psychology of mind: Solving of tasks and problems]. Moscow: 
Genesis. 319~p.
\bibitem{15-kir-1}
\Aue{Kolesnikov, A.\,V., and I.\,A.~Kirikov}. 2008. Kontseptual'naya model' dvunapravlennoy 
gibridizatsii pri razrabotke komp'yuternykh sistem podderzhki prinyatiya resheniy [Substructered 
model of bidirectional hybridization within the development of computer-based systems for 
decision-making support]. \textit{Sistemy i~Sredstva Informatiki}~--- \textit{Systems and Means of 
Informatics}. Add. Issue:21--53.
\bibitem{16-kir-1}
\Aue{Kirikov, I.\,A., A.\,V.~Kolesnikov, and S.\,V.~Listopad}. 2009. Modelirovanie 
samoorganizatsii grupp intellektual'nykh agentov v~zavisimosti ot stepeni soglasovannosti ikh 
vzaimodeystviya [Imitation of self-organization of intelligent agents teams 
in case of agreement 
dimension of their interaction]. \textit{Informatika i~ee Primeneniya~--- Inform. Appl.}  
3(4):76-- 86.
\bibitem{17-kir-1}
\Aue{Kolesnikov, A.\,V.} 2015. Printsipy i~metodologiya razra\-botki informatsionnykh 
geterogennykh sistem dvu\-na\-pravlennoy gibridizatsii [Approach and methodology of the 
development of informational heterogeneous systems of %\linebreak
 bidirectional hybridization]. 
\textit{Integrirovannye Modeli i~Myagkie Vychisleniya v~Iskusstvennom Intellekte: Sb. 
nauch. tr. 
\mbox{VIII} Mezhdunar. Nauch.-Praktich. Konf.} [8th Research and Practice Conference (International) 
``Integrated Models and Soft Computing in Artificial Intelligence'' Proceedings]. 
Moscow: 
Fizmatlit. 1:36--53.


\end{thebibliography}

 }
 }

\end{multicols}

\vspace*{-3pt}

\hfill{\small\textit{Received September 20, 2015}}


\Contr

\noindent
\textbf{Kirikov Igor A.}\ (b.\ 1955)~---
Candidate of  Sciences (PhD) in technology; director, Kaliningrad Branch of the 
Federal Research Center ``Computer Science and Control'' of the Russian Academy 
of Sciences, 5~Gostinaya Str., Kaliningrad 236000,  Russian Federation; baltbipiran@mail.ru

\vspace*{3pt}

\noindent
\textbf{Kolesnikov Alexander V.}\ (b.\ 1948)~---
Doctor of Sciences in technology; professor, Department of Telecommunications, 
Immanuel Kant Baltic Federal University, 14~Nevskogo Str., Kaliningrad 236041,
Russian Federation; senior scientist, Kaliningrad Branch of 
the Federal Research Center ``Computer Science and Control'' of the Russian 
Academy of Sciences, 5~Gostinaya Str., Kaliningrad 236000,  Russian Federation; 
avkolesnikov@yandex.ru

\vspace*{3pt}

\noindent
\textbf{Listopad Sergey V.}\ (b.\ 1984)~---
Candidate of  Sciences (PhD) in technology; scientist, Kaliningrad Branch of the 
Federal Research Center ``Computer Science and Control'' of the Russian Academy 
of Sciences, 5~Gostinaya Str., Kaliningrad 236000,  Russian Federation;  ser-list-post@yandex.ru

\vspace*{3pt}

\noindent
\textbf{Rumovskaya Sophiya B.}\ (b.\ 1985)~--- programmer~I, Kaliningrad Branch 
of the Federal Research Center ``Computer Science and Control'' of the Russian 
Academy of Sciences, 5~Gostinaya Str., Kaliningrad 236000,  Russian Federation; 
sophiyabr@gmail.com
\label{end\stat}


\renewcommand{\bibname}{\protect\rm Литература}