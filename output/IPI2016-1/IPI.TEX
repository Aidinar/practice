\documentclass[10pt]{book}
\usepackage[utf8]{inputenc}

\usepackage{latexsym,amssymb,amsfonts,amsmath,indentfirst,shapepar,%fleqn,%
picinpar,shadow,floatflt,enumerate,multicol,colortbl,moreverb,ipi}

\usepackage{rotating}
\usepackage{mathrsfs}
\usepackage[noend]{algorithmic}
\usepackage{ulem}

\input{epsf}

%\nofiles

%\includeonly{avtor} %+pdf
%\includeonly{obchak,avtor}
%\includeonly{pred}      %
%\includeonly{podgot-rus,podgot-eng}  %+pdf
%\includeonly{ocherk} %+
%\includeonly{nekrol} %+


%\includeonly{kalin}             %1+pdf
%\includeonly{stupnikov}         %2+pdfавт
%\includeonly{sinits-one}        %3+pdfавт
%\includeonly{sinits}            %4+pdfавт
%\includeonly{kovalev}           %5pdfавт
%\includeonly{kudr}              %6+pdf
%\includeonly{gorsh-one}         %7+pdf
%\includeonly{gorshenin}         %8+pdf
%\includeonly{kirikov}           %9+pdf
%\includeonly{zatsman}           %10+pdf
%\includeonly{alex}              %11+pdf
%\includeonly{seif-mul} %+12


%\includeonly{toc-rus, toc-en}
%\includeonly{obchak} %,toc-en}

%\includeonly{rekl}
%\includeonly{rekl-1}
%\includeonly{reshal}  %
%\includeonly{eng-index}
%\includeonly{cover3}

\usepackage{acad}
%\usepackage{courier}
\usepackage{decor}
\usepackage{newton}
\usepackage{pragmatica}
\usepackage{zapfchan}
\usepackage{petrotex}
\usepackage{bm}                     % полужирные греческие буквы
\usepackage{upgreek}                % прямые греческие буквы
\usepackage{eufrak}
\usepackage{verbatim}

\renewcommand{\bottomfraction}{0.99}
\renewcommand{\topfraction}{0.99}
\renewcommand{\textfraction}{0.01}

\setcounter{secnumdepth}{1} %здесь - 3 + chapter = 4

\arraycolsep=1.5pt

%\usepackage[pdftex]{graphicx}

%\usepackage{oz}

%NEW COMMANDS


\renewcommand*{\hm}[1]{#1\nobreak\discretionary{}%
            {\hbox{$\mathsurround=0pt #1$}}{}} %% Дублирует знаки операций
                               %при переносе в формуле (перед знаком, который
                               %надо продублировать ставится команда \hm)

%\newcommand{\endproof}{\hfill$\Box$}
%\renewcommand{\r}{\mathbb{R}}
\newcommand{\I}{{\rm I\hspace{-0.7mm}I}}
%\newcommand{\Ikl}{{\tt{1}}\hspace*{-1.44mm}\mathtt{1}}
\newcommand{\Ik}{\mbox{{\small \tt {1}}\hspace{-1.3mm}{\tt 1}}}
\newcommand{\argmin}{\mathop{\mathrm{arg}\,\mathrm{min}}}
\newcommand{\argmax}{\mathop{\mathrm{arg}\,\mathrm{max}}}
%\newcommand{\capr}{\mathop{\cap\,}}
%\newcommand{\cupr}{\mathop{\cup\,}}
%\def\argmin{\mathop{arg\,min}}

\def\vrp{\varphi}
\def\prt{\partial}
\def\mm{\mathrm{M}}
\def\modnop#1{\mathop{#1}\limits_{n}}
\def\eam{\mathbin{{\mathop{=}\limits^{\mathrm{def}}}}}
\def\dey#1#2{#1 (#2)}
\def\deyc#1#2{#1 \cdot  #2}
\def\ra#1{\;\mathop{\to}\limits^{#1}\;}
\def\raz#1{\;\mathop{\longrightarrow}\limits^{\!\!\!#1}\;}
\def\ral#1{\;\mathop{\longrightarrow}\limits^{#1}\;}

\newcommand{\Nor}{\mathcal{N}}
\newcommand{\T}{\mathbb{T}}
\newcommand{\Z}{\mathbb{Z}}



\newcommand{\il}[2]{\int\limits_{#1}^{#2}}%интеграл с пределами #1 и #2

%\def\ss2{\mathop {\sum\limits^p\sum\limits^p}}
\def\sss{\sum\limits}
\def\tr{,\,\ldots\,,\,}
\def\rk{\right]}
\def\lk{\left[}
\def\rf{\right\}}
\def\lf{\left\{}
\def\lv{\,\left\vert}
\def\rv{\right\vert\,}
\def\iii{\int\limits}
\def\iin{\int\limits_{-\infty}^\infty}
\def\rrv{\right\vert}


\def\ee{{\cal E}}
\def\ww{{\cal W}}
\def\yy{{\cal Y}}
\def\vv{{\cal V}}

\newcommand{\R}{\mathbb R}
\newcommand{\E}{\mathbb E}
\newcommand{\N}{\mathbb N}

\renewcommand{\P}{\mathbb{P}}

\newcommand{\h}{{\bf H}}
\newcommand{\p}{{\sf P}}  % вероятность

\newcommand{\e}{{\sf E}}  % мат. ожидание
\newcommand{\D}{{\sf D}}  % дисперсия
\newcommand{\eps}{\varepsilon}
\newcommand{\vp}{{\mathbf p}}
\newcommand{\vz}{{\mathbf z}}
\newcommand{\vx}{{\mathbf x}}
\newcommand{\vf}{{\mathbf f}}
\newcommand{\F}{{\mathcal F}}
\def\ap{{\mathrm{ЭР}}}
\newcommand{\ud}{\Delta_n} %uniform ditance
\newcommand{\nud}{\Delta_n(x)}
\renewcommand{\Re}{\mathrm{Re}\,}

\newcommand{\abs}[1]{\left\vert#1\right\vert}
\newcommand{\norm}[1]{\left\Vert#1\right\Vert}
\def\da{(\Delta_t,A)}

\newcommand{\corr}{\mathrm{corr}}

\newcommand{\cov}{\mathrm{cov}}
\newcommand{\Expect}{\mathbb{E}}

\def\w{\omega}
\def\W{\Omega}

\def\inh{\int\limits_{nh}^{(n+1)h}}

\def\sumin{\sum_{i=1}^N}


\def\bxt{(Y,t)}
\def\xt{(y,t)}

\def\ovth{{\fr{\tau-nh}{h}}}
\def\ov{\overline}
\def\tm{\tilde m}


\DeclareMathOperator{\sign}{sign}

%\newcommand{\gr}{{\geqslant}}


\newcommand{\g}{\mbox{\textit{g}}}

\renewcommand{\la}{\lambda}
\newcommand{\si}{\sigma}
\newcommand{\alp}{\alpha}

%\newcommand{\pto}{\stackrel{P}{\longrightarrow}} % сходимость по веpоятности

\newcommand{\eqd}{\stackrel{\mathrm{d}}{=}} % равенство по pаспpеделению
\newcommand{\eqdelta}{\stackrel{\Delta}{=}} % равенство по pаспpеделению

\def\be#1{\begin{equation}\label{#1}}
\def\ee{\end{equation}}
\def\re#1{(\ref{#1})}

\def\bn{\begin{enumerate}}
\def\en{\end{enumerate}}
\def\bi{\begin{itemize}}
\def\ei{\end{itemize}}
%\def\i{\item}

%\newcommand{\kp}{\kappa}
%\def\Q{{\cal Q}} \def\H{{\cal H}}
%\newcommand{\bet}{\beta_{2+\delta}}


%\newtheorem{definition}{Определение}
%\renewcommand{\thedefinition}{\arabic{definition}.}
%END NEW COMMANDS

%\renewcommand{\baselinestretch}{1.2}

%\pagestyle{myheadings}

\setlength{\textwidth}{167mm}      % 122mm
\setlength{\textheight}{658pt}
%\setlength{\textheight}{635.6pt}
\setlength{\columnsep}{4.5mm}

\setcounter{secnumdepth}{4}

%\addtolength{\headheight}{2pt}
%\addtolength{\headsep}{-2mm}

%\addtolength{\topmargin}{-20mm}  % for printing


%\hoffset=-30mm  % From Yap
\hoffset=-23mm  % From Acrobat

%\voffset=0mm % From Yap
%\voffset=-15mm   % From Acrobat

\addtolength{\evensidemargin}{-9.5mm} % for printing
\addtolength{\oddsidemargin}{9.5mm}  % for printing

%\renewcommand{\thefootnote}{\fnsymbol{footnote}}
%\renewcommand{\thefootnote}{\arabic{footnote}}
\renewcommand{\figurename}{\protect\bf Рис.}
\renewcommand{\tablename}{\protect\bf Таблица}

\newcommand{\Caption}[1]{\caption{\protect\small %\baselineskip=2.5ex
#1}}

\renewcommand{\thefigure}{\arabic{figure}}
\renewcommand{\thetable}{\arabic{table}}
\renewcommand{\theequation}{\arabic{equation}}
\renewcommand{\thesection}{\arabic{section}}

\renewcommand{\contentsname}{СОДЕРЖАНИЕ}
\newcommand{\fr}[2]{\displaystyle\frac{\displaystyle #1\mathstrut}{\displaystyle #2\mathstrut}}

%\renewcommand{\thefootnote}{\fnsymbol{footnote}}
%\newcommand{\g}{\mbox{\textit{g}}}

%\newcommand{\Caption}[1]{\caption{\protect\small\baselineskip=2ex #1}}
\newcounter{razdel}
\setcounter{razdel}{0}


\newcommand{\titel}[4]{%
\

\vspace*{5pt}

\ifodd\therazdel {\raggedright\noindent\Large\textrm\textbf
 \lineskip .75em
  \baselineskip=3.2ex #1 \par}
\vskip 1em {\noindent\large\textrm\textbf #2 \par}
\addcontentsline{toc}{subsection}{{\textrm\textbf #3}\protect\newline #1}
\def\rightheadline{\underline{\noindent\hbox to \textwidth{\hfill\small\textrm{#4}
%\hfill \large\bf\thepage
}}}
\def\leftheadline{\underline{\noindent\parbox{\textwidth}{
%\raggedleft\large\bf\thepage \hfill
\small\textit{#3}\hfill}}}
\def\leftfootline{\small{\textbf{\thepage}
\hfill ИНФОРМАТИКА И ЕЁ ПРИМЕНЕНИЯ\ \ \ том~10\ \ \ выпуск 1\ \ \ 2016}
}%
 \def\rightfootline{\small{ИНФОРМАТИКА И ЕЁ ПРИМЕНЕНИЯ\ \ \ том~10\ \ \ выпуск~1\ \ \ 2016
\hfill \textbf{\thepage}}}
\vskip 2em \setcounter{figure}{0}
\setcounter{table}{0}
\setcounter{equation}{0}
\setcounter{section}{0}
\setcounter{subsection}{0}
\setcounter{subsubsection}{0}
\setcounter{footnote}{0}
\setcounter{razdel}{0}
%\end{flushleft}
\else {
 \raggedright\noindent\Large\textrm\textbf
 \lineskip .75em
\baselineskip=3.2ex #1 \par} \vskip 1em
%\begin{flushleft}
{\noindent\large\textrm\textbf #2 \par}
\addcontentsline{toc}{subsection}{{\textrm\textbf #3}\protect\newline #1}
\def\rightheadline{\underline{\noindent\hbox to \textwidth{\hfill\small\textrm{#4}
%\hfill \large\bf\thepage
}}}
\def\leftheadline{\underline{\noindent\parbox{\textwidth}{%\raggedleft\large\bf\thepage \hfill
\small\textit{#3}\hfill}}}
\def\leftfootline{\small{\textbf{\thepage}
\hfill ИНФОРМАТИКА И ЕЁ ПРИМЕНЕНИЯ\ \ \ том~10\ \ \ выпуск~1\ \ \ 2016}
}%
 \def\rightfootline{\small{ИНФОРМАТИКА И ЕЁ ПРИМЕНЕНИЯ\ \ \ том~10\ \ \ выпуск~1\ \ \ 2016
\hfill \textbf{\thepage}}} \vskip 2em \setcounter{figure}{0}
\setcounter{table}{0} \setcounter{equation}{0} \setcounter{section}{0}
\setcounter{subsection}{0} \setcounter{subsubsection}{0}
\setcounter{footnote}{0}
%\end{flushleft}
\fi}

\newcommand{\titelr}[2]{%
\

\vspace*{5pt}

\ifodd\therazdel {\raggedright\noindent%\Large\textrm\textbf
 \lineskip .75em
  \baselineskip=3.2ex #1 \par}
\vskip 1em {\noindent\normalsize\textrm\textbf #2 \par}
\else {
 \raggedright\noindent\Large\textrm\textbf
 \lineskip .75em
\baselineskip=3.2ex #1 \par} \vskip 1em
%\begin{flushleft}
{\noindent\large\textrm\textbf #2 \par
%\noindent\normalsize\textrm\textbf #2 \par
} \fi}

\newcommand{\titele}[5]{%
\

%\vspace*{5pt}

\ifodd\therazdel {\raggedright\noindent\large
\textrm\textbf
 \lineskip .75em
%  \baselineskip=3.2ex
#1 \par}
\vskip .5em {\noindent\large\textrm\textbf #2 \par}
\vskip .5em
 {\noindent\textrm #3 \par}
\addcontentsline{toc}{subsection}{{\textrm\textbf #1}\protect\newline #2}
\def\rightheadline{\underline{\noindent\hbox to \textwidth{\hfill\small\textrm{#4}
%\hfill \large\bf\thepage
}}}
\def\leftheadline{\underline{\noindent\parbox{\textwidth}{
%\raggedleft\large\bf\thepage \hfill
\small\textrm{#5}\hfill}}}
\def\leftfootline{\small{\textbf{\thepage}
\hfill ИНФОРМАТИКА И ЕЁ ПРИМЕНЕНИЯ\ \ \ том~10\ \ \ выпуск~1\ \ \ 2016}
}%
 \def\rightfootline{\small{ИНФОРМАТИКА И ЕЁ ПРИМЕНЕНИЯ\ \ \ том~10\ \ \ выпуск~1\ \ \ 2016
\hfill \textbf{\thepage}}} \vskip 1em \setcounter{figure}{0}
\setcounter{table}{0} \setcounter{equation}{0} \setcounter{section}{0}
\setcounter{subsection}{0} \setcounter{subsubsection}{0}
\setcounter{footnote}{0} \setcounter{razdel}{0}
%\end{flushleft}
\else {
 \raggedright\noindent\large
 \textrm\textbf
 \lineskip .75em
%\baselineskip=3.2ex
#1 \par} \vskip .5em
%\begin{flushleft}
{\noindent\large\textrm\textbf #2 \par} \vskip .5em
 {\noindent\textrm #3 \par}
\addcontentsline{toc}{subsection}{{\textrm\textbf #1}\protect\newline #2}
\def\rightheadline{\underline{\noindent\hbox to \textwidth{\hfill\small\textrm{#4}
%\hfill \large\bf\thepage
}}}
\def\leftheadline{\underline{\noindent\parbox{\textwidth}{%\raggedleft\large\bf\thepage \hfill
\small\textrm{#5}\hfill}}}
\def\leftfootline{\small{\textbf{\thepage}
\hfill ИНФОРМАТИКА И ЕЁ ПРИМЕНЕНИЯ\ \ \ том~10\ \ \ выпуск~1\ \ \ 2016}
}%
 \def\rightfootline{\small{ИНФОРМАТИКА И ЕЁ ПРИМЕНЕНИЯ\ \ \ том~10\ \ \ выпуск~1\ \ \ 2016
\hfill \textbf{\thepage}}} \vskip 1em \setcounter{figure}{0}
\setcounter{table}{0} \setcounter{equation}{0} \setcounter{section}{0}
\setcounter{subsection}{0} \setcounter{subsubsection}{0}
\setcounter{footnote}{0}
%\end{flushleft}
\fi}

\def\Abst#1{
\begin{center}\small\nwt
\parbox{150mm}{%\baselineskip=2.5ex
\textbf{Аннотация:}\ \
%\hspace*{\parindent}
#1}
\end{center}}
\def\Abste#1{
\begin{center}\small\nwt
\parbox{150mm}{%\baselineskip=2.5ex
\textbf{Abstract:}\ \
%\hspace*{\parindent}
#1}
\end{center}}

\def\DOI#1{
\begin{center}\small\nwt
\parbox{150mm}{%\baselineskip=2.5ex
\textbf{DOI:}\ \
%\hspace*{\parindent}
#1}
\end{center}}

\def\Abstend#1{
\begin{center}\small\nwt
\parbox{150mm}{%\baselineskip=2.5ex
%\hspace*{\parindent}
#1}
\end{center}}


\def\KW#1{
\begin{center}\small\nwt
\parbox{150mm}{%\baselineskip=2.5ex
\textbf{Ключевые слова:}\ \ #1}
\end{center}}

\def\KWE#1{
\begin{center}\small\nwt
\parbox{150mm}{%\baselineskip=2.5ex
\textbf{Keywords:}\ \ #1}
\end{center}}


\def\KWN#1{
%\begin{center}
%\small
%\parbox{150mm}\end{center}
}

\renewcommand{\thesubsection}{\thesection.\arabic{subsection}\hspace*{-5pt}}
\renewcommand{\thesubsubsection}{\thesubsection\hspace*{5pt}.\arabic{subsubsection}\hspace*{-3pt}}

\newcommand{\Ack}{\section*{\protect\rmfamily Acknowledgments}\noindent}
\newcommand{\Contr}{\section*{\protect\rmfamily Contributors}\noindent}
\newcommand{\Contrl}{\section*{\protect\rmfamily Contributor}\noindent}

\makeindex


\begin{document}
\Rus

\nwt
%\ptb


%\renewcommand{\contentsname}{\protect\Large\bf Содержание}

\setcounter{tocdepth}{2}

%\tableofcontents

\renewcommand{\bibname}{\protect\rmfamily Литература}
  \def\Au#1{{\it #1}}
    \def\Aue#1{{#1}}

%\newcommand{\No}{№}
  \newcommand{\tg}{\,\mathrm{tg}\,}
    \newcommand{\ctg}{\,\mathrm{ctg}\,}
  \newcommand{\arctg}{\,\mathrm{arctg}\,}

\def\forallb{\mathop{\forall}}
\def\cupb{\mathop{\cup}}
\def\existsb{\mathop{\exists}}


\newpage
\addtocounter{razdel}{1}
%\def\razd{РЕГУЛИРУЕМЫЙ ЭЛЕКТРОПРИВОД ДЛЯ ЭЛЕКТРОЭНЕРГЕТИКИ}


\setcounter{page}{2}

%   { %\Large  
   { %\baselineskip=16.6pt
   
   \vspace*{-48pt}
   \begin{center}\LARGE
   \textit{Предисловие}
   \end{center}
   
   %\vspace*{2.5mm}
   
   \vspace*{25mm}
   
   \thispagestyle{empty}
   
   { %\small 

    
Вниманию читателей журнала <<Информатика и её применения>> предлагается 
очередной тематический выпуск <<Вероятностно-статистические методы и 
задачи информатики и информационных технологий>>. Предыдущие тематические 
выпуски журнала по данному направлению вышли в 2008~г.\ (т.~2, вып.~2), 
в 2009~г.\ (т.~3, вып.~3) и в 2010~г.\ (т.~4, вып.~2). 

Статьи, собранные в данном журнале, посвящены разработке новых вероятностно-статистических 
методов, ориентированных на применение к решению конкретных задач информатики и информационных 
технологий, а также~--- в ряде случаев~--- и других прикладных задач. Проблематика, охватываемая 
публикуемыми работами, развивается в рамках научного сотрудничества между Институтом проблем 
информатики Российской академии наук (ИПИ РАН) и Факультетом вычислительной математики и 
кибернетики Московского государственного университета им.\ М.\,В.~Ломоносова в ходе работ 
над совместными научными проектами (в том числе в рамках функционирования 
Научно-образовательного центра <<Вероятностно-статистические методы анализа рисков>>). 
Многие из авторов статей, включенных в данный номер журнала, являются активными участниками 
традиционного международного семинара по проблемам устойчивости стохастических моделей, 
руководимого В.\,М.~Золотаревым и В.\,Ю.~Королевым; регулярные сессии этого семинара 
проводятся под эгидой МГУ и ИПИ РАН (в 2011~г.\ указанный семинар проводится в октябре 
в Калининградской области РФ). 

Наряду с представителями ИПИ РАН и МГУ в число авторов данного выпуска журнала входят 
ученые из Научно-исследовательского института системных исследований РАН, Института 
проблем технологии микроэлектроники и особочистых материалов РАН, Института 
прикладных математических исследований Карельского НЦ РАН, Московского 
авиационного института, Вологодского государственного педагогического университета, 
НИИММ им.\ Н.\,Г.~Чеботарева, Казанского государственного университета, Дебреценского 
университета (Венгрия).

Несколько статей выпуска посвящено разработке и применению стохастических методов и 
информационных технологий для решения различных прикладных задач. В~работе В.\,Г.~Ушакова 
и О.\,В.~Шестакова рассмотрена задача определения вероятностных характеристик случайных 
функций по распределениям интегральных преобразований, возникающих в задачах эмиссионной 
томографии. В~статье Д.\,О.~Яковенко и М.\,А.~Целищева рассмотрены некоторые вопросы 
математической теории риска и предложен новый подход к диверсификации инвестиционных 
портфелей. Работа И.\,А.~Кудрявцевой и А.\,В.~Пантелеева посвящена построению и 
исследованию математической модели, описывающей динамику сильноионизованной плазмы. 
В~статье П.\,П.~Кольцова изучается качество работы ряда алгоритмов сегментации изображений. 
Статья А.\,Н.~Чупрунова и И.~Фазекаша посвящена вероятностному анализу числа без\-оши\-бочных 
блоков при помехоустойчивом кодировании; получены усиленные законы больших чисел для указанных 
величин.

В данном выпуске традиционно присутствует тематика, весьма активно разрабатываемая в течение 
многих лет специалистами ИПИ РАН и МГУ,~--- методы моделирования и управления для 
информационно-телекоммуникационных и вычислительных систем, в частности методы 
теории массового обслуживания. В~статье А.\,И.~Зейфмана с соавторами рассматриваются 
модели обслуживания, описываемые марковскими цепями с непрерывным временем в случае 
наличия катастроф. В~работе М.\,М.~Лери и И.\,А.~Чеплюковой рассматриваются случайные 
графы Интернет-типа, т.\,е.\ графы, степени вершин которых имеют степенные распределения; 
такие задачи находят применение при исследовании глобальных сетей передачи данных. 
Работа Р.\,В.~Разумчика посвящена исследованию систем массового обслуживания специального 
вида~--- с отрицательными заявками и хранением вытесненных заявок.

Ряд статей посвящен развитию перспективных теоретических 
вероятностно-статистических методов, которые находят широкое применение в различных 
задачах информатики и информационных технологий. В~работе В.\,Е.~Бенинга, А.\,К.~Горшенина 
и В.\,Ю.~Королева рассмотрена задача статистической проверки гипотез о числе компонент 
смеси вероятностных распределений, приводится конструкция асимптотически наиболее мощного 
критерия. Результаты этой работы найдут применение в ряде прикладных задач, использующих 
математическую модель смеси вероятностных распределений (в информатике, моделировании 
финансовых рынков, физике турбулентной плазмы и~т.\,д.). В~статье В.\,Ю.~Королева, 
И.\,Г.~Шевцовой и С.\,Я.~Шоргина строится новая, улучшенная оценка точности нормальной 
аппроксимации для пуассоновских случайных сумм; как известно, указанные случайные суммы 
широко используются в качестве моделей многих реальных объектов, в том числе в информатике, 
физике и других прикладных областях. Работа В.\,Г.~Ушакова и Н.\,Г.~Ушакова посвящена 
исследованию ядерной оценки плотности распределения; эти результаты могут применяться, 
в част\-ности, при анализе трафика в телекоммуникационных системах. Серьезные приложения 
в статистике могут получить результаты работы О.\,В.~Шестакова, в которой доказаны оценки 
скорости сходимости распределения выборочного абсолютного медианного отклонения к нормальному 
закону. 

\smallskip

Редакционная коллегия журнала выражает надежду, что данный тематический  выпуск 
будет интересен специалистам в области теории вероятностей и математической статистики 
и их применения к решению задач информатики и информационных технологий.
     
     %\vfill 
     \vspace*{20mm}
     \noindent
     Заместитель главного редактора журнала <<Информатика и её 
применения>>,\\
     директор ИПИ РАН, академик  \hfill
     \textit{И.\,А.~Соколов}\\
     
     \noindent
     Редактор-составитель тематического выпуска,\\
     профессор кафедры математической статистики факультета\\
      вычислительной математики и кибернетики МГУ им.\ М.\,В.~Ломоносова,\\
     ведущий научный сотрудник ИПИ РАН,\\ 
доктор физико-математических наук \hfill
      \textit{В.\,Ю.~Королев}
     
     } }
     }


\renewcommand{\figurename}{\protect\bf Figure}
\renewcommand{\tablename}{\protect\bf Table}

\def\stat{kalinich}


\def\tit{CONCEPTUAL MODELING OF~MULTIDIALECT WORKFLOWS}

\def\titkol{Conceptual modeling of~multidialect workflows}

\def\autkol{L.~Kalinichenko, S.~Stupnikov, A.~Vovchenko,
and~D.~Kovalev}

\def\aut{L.~Kalinichenko$^{1,2}$, S.~Stupnikov$^1$, A.~Vovchenko$^1$,
and~D.~Kovalev$^1$}

\titel{\tit}{\aut}{\autkol}{\titkol}

%{\renewcommand{\thefootnote}{\fnsymbol{footnote}}
%\footnotetext[1] {}}

\renewcommand{\thefootnote}{\arabic{footnote}}
\footnotetext[1]{Institute of Informatics Problems, Russian Academy of Sciences,
44-2 Vavilov Str., Moscow 119333, Russian Federation}
\footnotetext[2]{Faculty of
Computational Mathematics and Cybernetics, M.\,V.~Lomonosov Moscow State University,
1-52~Leninskiye Gory, GSP-1, Moscow 119991, Russian Federation}


%\vspace*{6pt}

\def\leftfootline{\small{\textbf{\thepage}
\hfill INFORMATIKA I EE PRIMENENIYA~--- INFORMATICS AND APPLICATIONS\ \ \ 2014\ \ \ volume~8\ \ \ issue\ 4}
}%
 \def\rightfootline{\small{INFORMATIKA I EE PRIMENENIYA~--- INFORMATICS AND APPLICATIONS\ \ \ 2014\ \ \ volume~8\ \ \ issue\ 4
\hfill \textbf{\thepage}}}

%\vspace*{6pt}


\Abste{This paper contributes to the techniques for conceptual representation of
data analysis algorithms and data integration facilities as well as processes to
specify data and behavior semantics in one paradigm. An investigation of a~novel
approach for applying a~combination of semantically different
platform-independent rule-based languages (dialects) for interoperable conceptual
specifications over various rule-based systems (RSs) relying on the rule-based
program transformation technique recommended by the W3C Rule Interchange
Format (RIF) is extended here. Such approach is combined with the facilities
aimed at the semantic rule-based mediation intended for the heterogeneous data
base integration. This paper extends a~previous research of the authors in the
direction of workflow modeling for definition of compositions of algorithmic
modules in a~process structure. A~capability of the multidialect workflow
support specifying the tasks in semantically different languages mostly suited to
the task orientation is presented. A~practical workflow use case, the
interoperating tasks of which are specified in several rule-based languages
(RIF-CASPD, RIF-BLD, RIF-PRD), is introduced. In addition, OWL~2 is used
for the conceptual schema definition, RIF-PRD is used also for the workflow
orchestration. The use case implementation infrastructure includes a~production
rule-based system (IBM ILOG), a~logic rule-based system (DLV), and
a~mediation system.}

\KWE{conceptual specification; workflow; RIF; production rule languages;
database integration; mediators; PRD; multidialect infrastructure}

\DOI{10.14357/19922264140413}

%\vspace*{6pt}


\vskip 12pt plus 9pt minus 6pt

      \thispagestyle{myheadings}

      \begin{multicols}{2}

                  \label{st\stat}

\section{Introduction}

  \noindent
  This work keeps on the intention of developing the facilities for conceptual
declarative problem specification and solving in data intensive domains (DID). In [1]
it was claimed that conceptual data semantics alone (e.\,g., formalized in ontology
languages based on description logic) are insufficient, so that conceptual
representation of data analysis algorithms as well as processes for problem solving are
required to specify data and behavior semantics in one paradigm.

The results presented in this paper\footnote[3]{This paper is an extended for the journal version of
the results presented in the ``Multidialect Workflows'' report at the ADBIS'2014
Conference.} extend the research~[1] aimed at the definition and implementation of the
facilities for conceptually-driven problems specification and solving in DID aiming at
ensuring eventually the following capabilities for expressing the specifications:
\begin{enumerate}[(1)]
\item an ability to provide complete and precise specification of the abstract
structure and behavior of the domain entities, their consistency, relationship, and
interaction;
\item well-grounded diversity of semantics of the modeling facilities providing for
the best attainable expressiveness, compactness, and precision of the definition of
the problem solving algorithm specifications;
\item arrangements for the extensions of the modeling facilities satisfying the
changing technological and practical needs;
\item specification independence from implementation platforms (languages,
systems);
\item specification independence from concrete information resources (IRs)
(databases,
services, ontologies, etc.)\ combined with facilities for their semantic
integration and interoperability; and
\item built-in methodologies for creation of unifying specification languages
providing for construction of semantics-preserving mappings of conceptual
specifications into their implementations in specific platforms.
\end{enumerate}

  The research reported in~[1] investigated the conceptual modeling facilities for
DID applying rule-based declarative logic languages possessing different,
complementary semantics and capabilities combined with the methods and languages
for heterogeneous data mediation and integration. Two fundamental techniques were
combined: ($i$)~constructing of the unifying extensible language providing for
semantics-preserving mapping into it of various IR
specification languages (e.\,g., such as data definition (DDL) and
data manipulation (DML) languages for databases); and ($ii$)~creation of
the unified extensible family of rule-based languages (dialects) and a~model of
interoperability of the programs expressed in such dialects with different semantics.

  The first technique is based on the experience obtained in course of the
SYNTHESIS language development~[2]. The kernel of the SYNTHESIS language is
based on the object-frame data model used together with the declarative rule-based
facilities in the logic language similar to a~stratified Datalog with functions and
negation. The extensions of the kernel are constructed in such a~way that each
extension together with the kernel is a~result of semantic preserving mapping of some
IR language into the SYNTHESIS~[2]. The canonical information model is
constructed as a~union of the kernel with such extensions defined for various resource
languages. Canonical model is used for development of \textit{mediators} positioned
between the users, conceptually formulating problems in terms of the mediators, and
distributed resources. A~schema of a~subject mediator for a~class of problems
includes the specification of the domain concepts defined by the respective
ontologies.
{ %\looseness=1

}

  Another, multidialect technique for rule-based programs interoperability applied is
based on the RIF standard [3] of W3C. The RIF standard introduces a~unified family of rule-based
languages together with a~methodology for constructing of semantic preserving
mappings of specific languages used in various RSs into RIF
dialects. Examples of RSs include {SILK}, {OntoBroker}, {DLV},
{IBM Websphere ILOG JRules}, {RIF4J}\;+\;{IRIS}, and others
(more examples can be found at {\sf
http:// www.w3.org/2005/rules/wiki/Implementations}). From the RIF point of view, an
IR is a~program developed in a~specific language of some RS.

  In [1], the first results obtained were presented including the description of an
approach and an infrastructure supporting:
  \begin{itemize}
\item application domain conceptual specification and problem solving algorithms
definitions based on the combination of the heterogeneous database mediation
technique and the rule-based multidialect facilities;
\item interoperability of distributed multidialect rule-based programs and mediators
integrating heterogeneous databases; and
\item rule delegation approach for the peer interactions in the multidialect
environment.
\end{itemize}

  The proof-of-concept prototype of the infrastructure based on the SYNTHESIS
environment and RIF standards has been implemented. The approach for multidialect
conceptualization of a~problem domain, rule delegation, rule-based programs, and
mediators interoperability were explained in detail and illustrated on an use-case in
the finance domain~[1]. For the conceptual definition of the use-case problem, the
OWL was used for the domain concepts definition and two RIF logic dialects
RIF-BLD~[4] and RIF-CASPD~[5] were used and mapped for implementation into the
SYNTHESIS formula language and the ASP (answer set programming)
based DLV~[6] language, respectively.

  The results obtained so far are quite encouraging for future work: they show that
the mentioned in the beginning capabilities~(1)--(6) sought for conceptual modeling
become feasible. This paper reports the results of extending the research in the
direction of modeling of the processes for the problem solving following the approach
briefly outlined above. These results include extensions of the infrastructure and
specification languages considered in~[1] to the workflow level keeping the same
approach and paradigm as well as aiming at the capabilities of the
conceptualization~(1)--(6) that were stated in~[1] and mentioned in the beginning of
the introduction.

  For investigation of such extension with respect to the choice of rule-based languages, it was
decided not to go outside the limits of the existing set of the published RIF dialects.
Such decision would allow to retain well-defined semantics of the conceptual
rule-based languages with a~possibility to check preservation of their semantics by various
languages of the implementing systems.

  The production rule dialect RIF PRD~[7] has been chosen as the language for the
workflow modeling in such a~way that the tasks of the workflow can have
multidialect rule-based representation (as defined in~[1]). This paper reporting the results
of such investigation is structured as follows. To make the paper self-contained, the
next section provides a~brief overview of the infrastructure supporting multidialect
programming defined in details in~[1]. Here, it is stressed
 that this infrastructure is
suitable for the workflow tasks specification. Workflow-oriented extension of the
multidialect infrastructureis considered in section~3. Use case implementation in
the proof-of-concept prototype is given in section~4. Related works are reviewed
in section~5. Concluding Remarks summarize contributions of the research.

%\vspace*{-24pt}

\begin{figure*} %fig1
\vspace*{1pt}
 \begin{center}
 \mbox{%
 \epsfxsize=164.734mm
 \epsfbox{kal-1.eps}
 }
 \end{center}
 \vspace*{-9pt}
\Caption{Conceptual schema and peer specifications }
\vspace*{-2pt}
  \end{figure*}

\section{Basic Principles of the Workflow Tasks Representation
in~the~Multidialect Infrastructure}

  \noindent
  Each workflow task (besides those that for pragmatic reasons are defined as
externally specified functions) is assumed to be represented in the novel infrastructure
defined in details in~[1]. Conceptual programming of tasks is performed using the
RIF dialects (now not only logic but also PRDs can be
used).

Conceptual tasks are implemented by their transformation into the rule-based
programs of the respective RSs and mediation systems (MSs). \textit{Conceptual
specification of a~task} is defined in the context of a~subject domain and consists of
a~set of RIF-documents (document is a~specification unit of RIF). The
\textit{conceptual schema} of the domain is defined using OWL 2~\cite{8-kal}
ontologies. Such usage of ontology is analogous to~\cite{22-kal}; however, it is
specifically important in the multidialect environment due to the formally defined
compatibility between RIF and OWL. The ontologies contain entities of the domain
and their relationships (Fig.~1, right-hand part). Conceptual specification of a~task is
defined over conceptual schema. Ontologies are imported into the RIF-documents
specifying an import profile, for instance, {OWL Direct}. Documents
\textit{import} other documents having the same semantics (the \textit{Import}
directive), \textit{link} documents defined using other dialects and having different
semantics (remote module directive \textit{Module}) or \textit{refer} to entities
contained in other documents using \textit{external terms}.
{\looseness=1

}

  Semantics of a~conceptual task definition in such setting becomes a~multidialect
one. The specification modules of a~task are treated as peers. Mediation modules are
assumed to be defined in RIF-BLD for representation of the mediator rules (to be
interpreted in SYNTHESIS) supporting schema mapping and semantic integration of
the IRs. Multidialect task is implemented by means of
transformation of conceptual specifications into modular, component-based
peer-to-peer (P2P)
program represented in the languages of the MSs and RSs
with the respective semantics. Interoperability of logic rule components of such
distributed program is carried out by means of the delegation technique [1,
section~3.3]. Production rule components are considered as external functions,
interoperability is achieved through the mechanism of external terms.

  A schema $S_R$ of a~peer~$R$ is a~set of entities (classes or relations and their
attributes) corresponding to extensional and intensional predicates of the resource
implementing the peer~$R$.

  The RS or the MS of each peer~$R$ should be
a~conformant~$D_R$ consumer where~$D_R$ is the~respective RIF dialect (Fig.~1,
left-hand part). Conformance is formally defined using formula entailment and
language mappings~[3].

  The peer $R$ is relevant to a~RIF-document~$d$ of a~conceptual specification of a~problem
  (Fig.~1, right-hand part) if ($i$)~$D_R$ is a~subdialect of the document~$d$
dialect (subdialect is a~language obtained from some dialect by removing certain
syntactic constructsand imposing respective restrictions on its semantics~[4]; each
program that conforms with the subdialect also conforms with the dialect) and
($ii$)~entities of the peer schema~$S_R$ (if they exist) are \textit{ontologically
relevant} to entities of the conceptual schema the names of which are used
in~$d$~for extensional predicates.

  The schema of a~relevant peer is mapped into the conceptual schema. The mapping
establishes the correspondence of the conceptual entities referred in the
document~$d$ to their expressions in terms of entities of the schema~$S_R$ using
rules of the $D_R$ dialect. These schema mapping rules constitute separate
  RIF-document (Fig.~1, middle part).

  Peers communicate using a~technique for distributed execution of the rule-based
programs. The basic notion of the technique is delegation-transferring facts and
rules from one peer to another. A~peer is installed on a~node of the multidialect
infrastructure. A~node is a~combination of a~wrapper, an RS or an MS, and a~peer
(for the details, refer~[1, Fig.~3]). A~wrapper
transforms programs and facts from the specific RIF dialect into the language of the
RS or MS and \textit{vice versa}. A~wrapper also implements the delegation mechanism.
Transferring facts and rules among peers is performed in the RIF dialects.

  A~special component (\textit{Supervisor}) of the architecture defined in~[1] stores
shared information of the environment, i.\,e., conceptual specifications related to the
domain and to the problem, a~list of the relevant resources, RIF-documents combining
rules for the conceptual specification and a~resource schema mapping.

  Implementation of the conceptual specification includes the following steps:
  \begin{enumerate}[(1)]
\item rewriting of the conceptual documents into the RIF-programs of the peers
performed by the \textit{Supervisor}. The rewriting includes also ($i$)~replacing the
document identifiers (used to mark predicates) by peer identifiers and ($ii$)~adding
schema mapping rules to programs (Fig.~1, middle part); %\\[-14pt]
\item a transfer of the rewritten programs to nodes containing peers relevant to the
respective conceptual documents. The transfer is performed by the \textit{Supervisor}
by calling the method \textit{loadRules} of the respective node wrappers; %\\[-14pt]
\item a transformation of the RIF-programs into the concrete RS or MS languages.
The transformation is performed by the \textit{NodeWrapper} or by the RS or MS
itself (if the RS or MS supports the respective RIF dialect); and %\\[-14pt]
\item an execution of the produced programs in P2P environment.
\end{enumerate}

  During the process of rewriting of the conceptual schema into the resource
programs, the relationships between RIF-documents of the conceptual schema defined
by remote or imported terms are replaced by relationships between peers also defined
by remote or imported terms. To implement remote and imported terms, a~\textit{rule
delegation} mechanism is used to transfer facts and rules from one peer to another.
The details of rule delegation approach including description of the related algorithms
are provided in~[1].

\vspace*{-7pt}

\section{Workflow-Oriented Extension of~the~Multidialect
Infrastructure}

\vspace*{-2pt}

  \noindent
  The aim of the infrastructure proposed is a~conceptual programming of problems in
the RIF-dialects and an implementation of conceptual specifications using rule-based
languages of the RSs and MSs. One of the objectives of this particular paper is to
introduce an extension of the existing multidialect infrastructure~[1] aiming at the
conceptual specification of rule-based workflows.

  Conceptual specification of a~problem (class of problems) is defined in the context
of a~subject domain and consists of a~set of
  RIF-documents. Besides the documents expressed in the logic dialects of RIF, the
documents expressed in the production rule dialect (RIF-PRD) also can be a~part of
conceptual specification of a~problem. In particular, these documents are aimed to
express a~process of solving the problem as the production rule-based workflow.

\vspace*{-4pt}

\subsection{Specification of~workflow orchestration}

  A workflow consists of a~set of tasks orchestrated by specific constructs
(\textit{workflow patterns}~\cite{9-kal}, for instance, \textit{sequence}, \textit{split},
\textit{join}) defining the order of tasks execution. The specification of such
orchestration is called here a~\textit{workflow skeleton}. A~skeleton is defined using
RIF-PRD production rules. Workflows and workflow patterns can be represented
using production rules in various ways, e.\,g., as in~\cite{9-kal, 17-kal}. The
approach applied in this paper to represent workflows requires the extension of
  RIF-PRD dialect by several built-in predicates (they are considered to be a~part of
\textit{wkfl} namespace referenced by
  {\sf http://www.w3.org/2014/rif-workflow-predicate\#} URI similarly to
\textit{func} and \textit{pred} namespaces defined in~\cite{21-kal} for built-in
functions and predicates of RIF):
  \begin{itemize}
\item predicate {\sf wkfl:end-of-task(?arg)} where \textit{\sf ?arg} is an identifier of a~task. The value space of
{\sf ?arg} is the XML-Schema built-in data type
{\sf xsd:Name} representing XML names. The predicate turns into true if a~task
\textit{?arg} has been completed;
\item predicate {\sf wkfl:variable-definition(?arg1\, ?arg2)} where {\sf ?arg1}
is the identifier of a~variable and {\sf ?arg2} is the identifier of a~type of the
variable.
The value space for both arguments is {\sf xsd:Name}. Turning the predicate into
true means that a~variable {\sf ?arg1} of type {\sf ?arg2} is defined in the
context of a~workflow;
\item predicate {\sf wkfl:variable-value(?arg1\,?arg2)} where {\sf ?arg1} is the
identifier of a~variable and {\sf ?arg2} is the value of the variable. The value space for
the first argument is {\sf xsd:Name}, the value space for the second argument is the
union of value spaces of all RIF built-in datatypes. Turning the predicate into true
means that a~variable {\sf ?arg1} has the value {\sf ?arg2};
\item predicate {\sf wkfl:parameter-definition(?arg1\,?arg2\,
?arg3)} where
{\sf ?arg1} is the identifier of a~workflow parameter; {\sf ?arg2} is the identifier
of a~type of the parameter; and {\sf ?arg3} is the direction of the parameter. The value
space for the first and for the second arguments is {\sf xsd:Name}. The value space
for the third argument is {\sf \{IN, OUT, IN\_OUT\}}
(\textit{input}, \textit{output}, or
\textit{input--output} parameter).
Turning the predicate into true means that a~parameter {\sf ?arg1} of
type {\sf ?arg2}, and direction {\sf ?arg3} is defined
for a~workflow; and
\item predicate {\sf wkfl:parameter-value(?arg1\,?arg2)} defines values of
workflow parameters in the same way as {\sf wkfl:variable-value} defines values
of workflow variables.
  \end{itemize}

  Predicates {\sf wkfl:variable-definition} and
  {\sf wkfl:}\linebreak {\sf variable-value} allow
to specify workflow variables and their values and thus to organize the data flow
within a~workflow. Predicates {\sf wkfl:parameter-definition} and
{\sf wkfl:parameter-value} allow to specify workflow parameters and their values
and thus to define the interface of a~workflow in terms of input and output parameters.
Using of workflow parameters and variables is illustrated in the Appendix.

  The predicate {\sf wkfl:end-of-task(?arg)} allows to orchestrate the order of
execution of workflows tasks using conditions and actions of production rules. In this
section, the template rules intended for representation of several basic workflow
patterns (Fig.~2) are provided.

\begin{center}  %fig2
\vspace*{6pt}
\mbox{%
 \epsfxsize=76.913mm
 \epsfbox{kal-2.eps}
 }
  \vspace*{2pt}

{{\figurename~2}\ \ \small{Basic workflow patterns}}
  \end{center}

\vspace*{6pt}


\addtocounter{figure}{1}

  Three well-known workflow patterns are considered below: {\sf Sequence},
{\sf AND-Split}, and {\sf AND-Join}.

  The \textit{AND-Split}\footnote{In this paper, the simplified \textit{presentation
syntax}~\cite{7-kal} is used.} workflow pattern is represented in RIF-PRD by the
following production ruletemplate using {\sf wkfl:end-of-task} predicate:
  \begin{verbatim}
If Not(External(wkfl:end-of-task(A)))
Then Do (Act(A)
 Assert(External(wkfl:end-of-task(A))))
If And(Not(External(wkfl:end-of-task(B)))
 External(wkfl:end-of-task(A)))
Then Do (Act(B)
 Assert(External(wkfl:end-of-task(B))))
If And(Not(External(wkfl:end-of-task(C)))
 External(wkfl:end-of-task(A)))
Then Do (Act(C)
 Assert(External(wkfl:end-of-task(C))))
\end{verbatim}

  The template includes three rules for tasks~$A$, $B$, and~$C$, respectively.
${\sf Act}(A)$, ${\sf Act}(B)$, and ${\sf Act}(C)$ denote \textit{actions} associated with tasks~$A$,
$B$, and $C$. Orchestration (tasks~$B$ and~$C$ are executed concurrently right after
task~$A$ is completed) is specified using {\sf wkfl:end-of-task} predicate in
conditions and {\sf Assert} actions of rules.

  Similarly, the {\sf AND-Split} pattern is represented in RIF-PRD by the
following production rule template:

\vspace*{-1.5pt}

\noindent
  \begin{verbatim}
If Not(External(wkfl:end-of-task(A)))
Then Do (Act(A)
 Assert(External(wkfl:end-of-task(A))))
If And(Not(External(wkfl:end-of-task(B)))
 External(wkfl:end-of-task(A)))
Then Do (Act(B)
 Assert(External(wkfl:end-of-task(B))))
If And(Not(External(wkfl:end-of-task(C)))
 External(wkfl:end-of-task(A)))
Then Do (Act(C)
 Assert(External(wkfl:end-of-task(C))))
\end{verbatim}

\vspace*{-1.5pt}

  The {\sf Sequence} pattern is represented in RIF-PRD by the following
production rule template:

\vspace*{-1.5pt}

\noindent
  \begin{verbatim}
If Not(External(wkfl:end-of-task(A)))
Then Do (Act(A)
 Assert(External(wkfl:end-of-task(A))))
If And(Not(External(wkfl:end-of-task(B)))
 External(wkfl:end-of-task(A)))
Then Do (Act(B)
 Assert(External(wkfl:end-of-task(B))))
\end{verbatim}

\vspace*{-1.5pt}

  More complicated patterns like OR-, XOR- splits and joins, structured loops,
subflows, and others are represented in RIF-PRD similarly.

\vspace*{-6pt}

\subsection{Workflow tasks specification}

  \noindent
  Workflow taskscan be specified as:
  \begin{itemize}
\item separate RIF-documents in various logic RIF-dialects (this is the way how
multidialect infrastructure~[1] is extended with workflow capabilities);\\[-15pt]
\item separate RIF-documents in the RIF-PRD dialect;\\[-15pt]
\item set of production rules embedded into the workflow skeleton; and\\[-15pt]
\item external functions treated as ``black boxes.''
\end{itemize}

  Semantics of tasks specified as multidialect logic programs are defined in
accordance with the RIF-FLD~[3] standard and standards for the respective
  RIF-dialects (BLD, CASPD, etc.). Semantics of tasks specified as production
rule programs are defined in accordance with the RIF-PRD standard. Semantics of
external functions ``are assumed to be specified externally in some document''~[3].

  All kinds of tasks (except those that are embedded into a~workflow skeleton) are
referenced in the workflow skeleton as \textit{external terms}~[3] like
${\sf External}\left({\sf t}\right)$
where term~{\sf t} is defined by an external resource identified by internationalized
resource identifier (IRI)~[3].

\begin{figure*} %fig3
\vspace*{1pt}
 \begin{center}
 \mbox{%
 \epsfxsize=163.675mm
 \epsfbox{kal-3.eps}
 }
 \end{center}
 \vspace*{-9pt}
\Caption{Extended multidialect infrastructure}
\end{figure*}


\vspace*{-6pt}

\subsection{Workflow implementation infrastructure}

  \noindent
  Workflows defined in the conceptual specification are implemented in the
environment shown in Fig.~3. Peer-\linebreak\vspace*{-12pt}

\pagebreak

\noindent
to-peer environment~[1] intended to implement logic
programs is extended with a~production rule-based system (PRS) (for
instance, a~production system compliant with the OMG Production Rule
Representation~\cite{16-kal}) and with external functions, implemented as
  web-services. Implementation of the conceptual specification includes the
following steps:
  \begin{enumerate}[(1)]
\item transfer of the conceptual RIF-documents constituting a~workflow skeleton to
the production rule-based system node (performed by the \textit{Supervisor} component);\\[-14pt]
\item transformation of the conceptual RIF-documents constituting a~workflow
skeleton into the language of the production rule-based system (performed by the PRS
Wrapper component);\\[-14pt]
\item transferring RIF logic programs related to tasks to the relevant nodes of the
environment and transformation of the RIF-programs into the concrete RS or MS
languages~[1]; and\\[-14pt]
\item execution of the workflow.
\end{enumerate}


  The interface of the \textit{Supervisor} includes methods for submitting and
executing a~workflow represented as a~set of RIF-documents, and for getting the
result of the workflow execution.

  To provide a~proof of the multidialect infrastructure concept, a~use case in the
financial domain has been implemented. The problem to be solved in the use case is
called the \textit{investment portfolio diversification problem}. The detailed
description of the use case is included in the Appendix.

\vspace*{-9pt}

\section{Related Work}

  \noindent
  Two types of workflow models, namely, abstract and concrete, were
identified~\cite{15-kal}. In the abstract model, a~workflow is described in an abstract
form, without re-\linebreak\vspace*{-12pt}
\columnbreak

\noindent
ferring to specific resources. In this paper, workflow
representation in abstract and platform-independent  form is suggested.

  A classification model for scientific workflow characteristics~\cite{9-kal}
contributes to better understanding of scientific workflow requirements. The list of
structural patterns discovered during this analysis (including sequential, parallel,
parallel-split, parallel-merge, and mesh) influenced the choice of the required workflow
patterns.

  The OMG standard~\cite{16-kal} reflects an attitude to production rules from the
industrial side providing an OMG MDA (model-driven architecture)
platform-independent model (PIM)  with a~high probability of
support at the PSM (platform-specific model) level from the rule engine vendors.
Similar capabilities though formally defined are used as the basis for the
RIF-PRD~\cite{7-kal}.

  Some vendors of such production rule engines have extended their languages with
the workflow specification capabilities. IBM has extended ILOG to provide the
ruleflow capability. Microsoft supports Windows Workflow Foundation as a~platform
providing the workflow and rules capabilities. The examples of specific formalisms for
PIM rule-based process specifications are also provided in~\cite{17-kal}.

  Comparing to the known variants of the PIM production rule representations,
  selection of the RIF-PRD is considered to be well grounded:
  \begin{enumerate}[(1)]
\item the RIF-PRD is formally defined;
\item RIF ensures support of interoperability of modules written in different
rule-based dialects with different semantics;
\item RIF provides foundations for PIM to PSM semantic preserving
transformation; and
\item RIF also provides ability for specification of the concepts in application
domain terms combining rule-based specifications with the OWL ontologies.
\end{enumerate}

  Importance of providing the interdialect interoperation is advocated
  in~\cite{18-kal} for combining the functionalities of production systems and logic
programs for abductive logic programming (ALP). The ALP framework gives a~model-theoretic semantics to both kinds of rules and provides them with powerful
proof procedures, combining backward and forward reasoning.

  Papers related to RIF-PRD experimentations are focused mainly on the issue of the
PRD programs transformation to an implementation system. In~\cite{19-kal}, a~case
study of bridging the ILOG Rule Language (IRL) to RIF-PRD and vice versa is
considered. In~\cite{20-kal}, implementation of RIF-PRD in three different
paradigms: Answer Set Programming, Production Rules, and Logic Programming
(XSB) is investigated.

  The contribution of this paper with regard to previous works of the authors~[1] consists in
extensions of the infrastructure and specification languages considered in~[1] to the
workflow level.

\section{Concluding Remarks}

  \noindent
  Progress in the investigation of the infrastructure~[1] for the conceptual
multidialect interoperable programming in the abstract, rule-based,
platform-independent notations is reported. An extension of the coherent
combination of the multidialect rule-based programming technique recommended by
the W3C RIF with the approach for unifying modeling of heterogeneous data bases
for their semantic mediation is presented. The extension of the infrastructure and specification
languages considered in~[1] in the direction of the workflow modeling is presented.

  Sticking to the limits of the existing set of the published RIF dialects,
   a~capability of the multidialect workflow support
   is presented with the tasks specified in
semantically different languages mostly suited to the task orientation.
Also, a~realistic problem solving use case containing the interoperating tasks
specified in
several platform-independent rule-based languages: RIF-CASPD, RIF-BLD,
  RIF-PRD, is presented. In addition, OWL~2 is used for the conceptual schema definition,
  RIF-PRD is applied for the workflow orchestration. The platforms selected for
implementation of the tasks include: DLV, SYNTHESIS, IBM ILOG. Such approach
retains well-defined semantics of the platform-independent rule-based languages with
a possibility to check preservation of their semantics by various languages of the
implementing systems. The principle of independence of tasks from the specific IRs is
carried out by the heterogeneous database mediation facilitates contributing to the
  reuse of tasks and workflows. Alongside with the further extension of the
approach, in the future work, the authors plan to apply the conceptual multidialect
programming philosophy for support of the experiments in data intensive sciences. In
particular, they plan to investigate modeling hypotheses in astronomy representing them
as a~set of rules applying the multiplicity of the dialects required.


%\setcounter{equation}{0}

\vspace*{12pt}

{{\hfill \textbf{APPENDIX A}}}

\vspace*{-12pt}

\subsection*{MULTIDIALECT WORKFLOW USE CASE}


{\small


 %\section*{\raggedleft Appendix~A.\\ Multidialect Workflow Use-Case}


%\renewcommand{\thesection}{A\arabic{equation}}

\subsection*{A.1\ Investment portfolio diversification\\
\hspace{20pt}problem extended}

  \noindent
  Motivation of the use case that illustrates the proposed approach comes from the
finance area. The use case extends the \textit{investment portfolio diversification
problem} defined in~[1, Appendix] by adding workflow orchestration applying the
RIF-PRD. The idea of the portfolio diversification problem is
as follows. The portfolio is a~collection of securities of companies, and its size is the
number of securities in the portfolio. The problem is to build a~diversified portfolio of
maximum size. Diversification means that the prices of the securities in portfolio
should be almost independent of each other. If the price of one security falls, it will
not significantly affect the prices of others. Thus, the risk of a~portfolio sharp decrease
is reduced.

  The input data for the problem is a~set of securities and respective time series of
indicators of the security price for each security. Time series for each security is a~set
of pairs $(d, v)$ where $d$ is a~date and~$v$ is an indicator of the security price (for
instance, closing price). The financial services \textit{Google Finance} ({\sf
https://www.google.com/finance}) and \textit{Yahoo! Finance} ({\sf
http://finance.yahoo.com/}) are considered. They include various indicators of the
security price for all trading days of the last decades. For the diversified portfolio, the
securities having noncorrelated time series should be used. Noncorrelation of the
time series means that their correlation is less than some predetermined price
correlation value. The output data for the problem is a~set of subsets of securities of
the maximum size, for which the pair wise correlation will be less than the
predetermined one.

  The maximum satisfying subset of securities is calculated in the following way.
Let~$G$ be a~graph where the vertices are the securities. An edge between two
securities exists if absolute value of their correlation is less than a~specified number.
So, any two securities connected by an edge are considered as noncorrelated. In such
case, the problem of finding the portfolio of the maximum size is exactly the problem
of finding a~maximum clique in an undirected graph. A~maximal clique is a~maximal
portfolio. Note that several different maximal portfolios can be found.

  The conceptual specification of the use case~[1] used two RIF-dialects: RIF-BLD
and RIF-CASPD. The use case was implemented in the environment containing a~mediation system used as a~platform for RIF-BLD~[4] and ASP-based DLV
system~\cite{6-kal}~--- a~platform for RIF-CASPD. The RIF-BLD was used to
specify the problem of data integration, and RIF-CASPD~--- the problem of finding
a~maximum clique in an undirected graph.

In this work, the portfolio use case is extended in the following way. The goal is
not only to build a~set of diversified portfolios, but
also to choose the ``best'' of them
according to some criteria. There are several approaches to choose the most
appropriate portfolio.

  The most recognized one is based on the Markovitz portfolio theory~\cite{10-kal}.
The idea is to choose the portfolio, which has the maximum risk/return ratio. The
most well-known metric to operate with risk/return is Sharpe-ratio~\cite{11-kal}:
  $(r_p -r_f)/\sigma^2$. Here, $r_p$ denotes the expected return of the portfolio;
$r_f$ denotes the~risk free rate; and $\sigma^2$ denotes the~portfolio standard deviation (risk).
The more the Sharpe-ratio is, the better the investment is.

  Another approach is based on an idea that with the advent of social networks, it
became possible to monitor ideas, sentiments, actions of people and lots of available
information has to do with the markets and investments. In~\cite{12-kal}, Bollen
\textit{et al.}\ draw the connection between the mood of investor tweets and the move
of Dow Jones Index, stating that correlation between them is more than 80\%. The
idea of using tweets to assess market movements has been implemented in several
hedge funds.

  Combining these two strategies could provide benefits of both of them, which
leads to the following problem statement: having S\&P500 (a stock market index
maintained by the Standard\,\&\,Poor's, comprising 500~large-cap American
companies) list of companies, compute the diversified portfolio of maximum size
with the best risk/return and sentiment ratios.

%\vspace*{-6pt}

\subsection*{A.2\	Conceptual specification\\
\hspace*{20pt}of~the~application domain\\
\hspace*{20pt}and~the~problem}

%\vspace*{-2pt}

  \noindent
  Conceptual schema (ontology) of the application domain of historical prices of
securities is written in the simplified OWL functional syntax~\cite{8-kal}
({\sf Declaration} keyword is omitted; {\sf property}, {\sf domain}, and {\sf range}
declarations are combined).
  \begin{verbatim}
Ontology(<http://synthesis.ipi.ac.ru/portfolio/
    ontology>
 Class(Portfolio)
  ObjectProperty(securities domain(Portfolio)
   range(Portfolio))
  DataProperty(expected_return domain(Portfolio)
   range(xsd:double))
  DataExactCardinality(1 expected_return
   Portfolio)
  DataProperty(std_dev domain(Portfolio)
   range(xsd:double))
  DataExactCardinality(1 std_dev Portfolio)
  DataProperty(sharpe_ratio domain(Portfolio)
   range(xsd:double))
  DataExactCardinality(1 sharpe_ratio Portfolio)
  DataProperty(twitter_positive_ratio
   domain(Portfolio) range(xsd:double))
  DataExactCardinality(1 twitter_positive_ratio
   Portfolio)
  DataProperty(risk_free_rate domain(Portfolio)
   range(xsd:double))
  DataExactCardinality(1 risk_free_rate
   Portfolio)
  DataProperty(recommended domain(Portfolio)
   range(xsd:boolean))
  DataExactCardinality(1 recommended Portfolio)

 Class(Security)
  DataProperty(ticker  domain(Security)
   range(xsd:string))
  DataExactCardinality(1 ticker Security)
  DataProperty(rates  domain(Security)
   range(StockRate))
  DataProperty(positive_tweets domain(Security)
   range(xsd:double))
  DataExactCardinality(1 positive_tweets
   Security)
  DataProperty(sec_expected_return
   domain(Security) range(xsd:double))
  DataExactCardinality(1 sec_expected_return
   Security)
  DataProperty(sec_std_dev domain(Security)
   range(xsd:double))
  DataExactCardinality(1 sec_std_dev Security)

 Class(StockRate)
  DataProperty(date domain(StockRate)
   range(xsd:date))
  DataExactCardinality(1 date StockRate)
  DataProperty(price domain(StockRate)
   range(xsd:double))
  DataExactCardinality(1 price StockRate)
)
  \end{verbatim}

  \vspace*{-6pt}

  A~portfolio (the {\sf Portfolio} class) is characterized by a~set of securities
({\sf securities} attribute) contained in the portfolio, by several metrics: expected
return ({\sf expected\_return} attribute), standard deviation ({\sf std\_dev}
attribute), Sharpe ratio ({\sf sharpe\_ratio attribute}), risk free rate
({\sf risk\_free\_rate} attribute), and
ratio of positive tweets mentioning securities of
the portfolio ({\sf twitter\_positive\_ratio} attribute).

  A security (the {\sf Security} class) is characterized by identifier ({\sf ticker}
attribute), time series of historical prices (attribute {\sf
rates}), ratio of positive
tweets mentioning the security ({\sf positive\_tweets} attribute), expected return
({\sf sec\_expected\_return} attribute), and standard deviation ({\sf sec\_std\_dev}
attribute).

\begin{figure*} %fig4
\vspace*{1pt}
 \begin{center}
 \mbox{%
 \epsfxsize=126.24mm
 \epsfbox{kal-4.eps}
 }
 \end{center}
 \vspace*{-11pt}
\Caption{Portfolio workflow}
\vspace*{-6pt}
  \end{figure*}

The workflow of the extended portfolio problem is demonstrated in Fig. 4. The workflow
contains six tasks\footnote{To save space, specifications are provided only for
{\sf getPortfolios}, {\sf getPositiveTweetRatio}, and
{\sf computePortfolioTwitterMetrics} tasks.}:
\begin{enumerate}[(1)]
\item {\sf getPortfolios}. A~set of diversified portfolio candidates is computed. The
multidialect task specification consists of two RIF-documents in BLD and CASPD
dialects~[1, Appendix]. Portfolios received as a~result contain only security tickers,
they have to be augmented by financial and sentiments ratios;
\item {\sf getPositiveTweetRatio}. This task is responsible for computing a~sentiment ratio of tweets for every security. Every tweet is assessed
to be positive,
negative, or neutral. The task is specified as a~call of external function;
\item {\sf computePortfolioTwitterMetrics}. The portfolio sentiment ratio is
computed as the average of its securities sentiment ratio. The task is specified using
RIF-PRD;
{\looseness=1

}
\item {\sf getSecurityFinancialMetrics}. For every security in a~portfolio the
financial rates (the {\sf expected return} and the {\sf standard deviation}) are
calculated on the basis of historical rates of securities specified as an OWL~2 class of
the ontology of the application domain. The task is specified using RIF-BLD dialect;
\item {\sf computePortfolioFinancialMetrics}. The computation of the portfolio
expected return, risk, and Sharpe-ratio is done within this task. The task is specified
using RIF-PRD dialect; and
\item {\sf choosePortfolio}. The best portfolio is chosen according to maximizing
the (\textit{Sharpe ratio * sentiment ratio}) coefficient. The task is specified using
RIF-PRD dialect.
  \end{enumerate}

  Workflow skeleton is specified as a~RIF-PRD document importing the ontology of
the application domain:
  \begin{verbatim}
Document( Dialect(RIF-PRD)
 Base(<http://synthesis.ipi.ac.ru/portfolio/
  workflow#>)
 Import(<http://synthesis.ipi.ac.ru/portfolio/
  ontology#>
 <http://www.w3.org/ns/entailment/OWL-Direct>)
Prefix(ont<http://synthesis.ipi.ac.ru/portfolio/
 ontology#>)
Prefix(ofws<http://synthesis.ipi.ac.ru/
 synthesis/projects/RuleInt/OpinionFinderWS#>)
Prefix(mws<http://synthesis.ipi.ac.ru/
 synthesis/projects/RuleInt/MediatorWS#>)

Group 2 (
 Do(
  Assert(External(wkfl:parameter-definition(
   startDatexsd:string IN)))
  Assert(External(wkfl:parameter-definition(
   endDatexsd:string IN)))
  Assert(External(wkfl:parameter-definition(
   bestPortfolioont:Portfolio OUT)))
  Assert(External(wkfl:variable-definition(
   ps  List<ont:Portfolio> IN)))
  Assert(External(wkfl:
   variable-value(ps List())))
 )
)
\end{verbatim}

\noindent
\begin{verbatim}

Group 1 (
 Forall ?sd ?ed such that (
  External(wkfl:parameter-value(startDate ?sd))
  External(wkfl:parameter-value(endDate ?ed))  )
( If Not(External(wkfl:
   end-of-task(getPortfolios)))
  Then
   Do( Modify(External(wkfl:variable-value(ps
    External(mws:getPortfolios(?sd ?ed) )))
   Assert(External(wkfl:
    end-of-task(getPortfolios))) )
 )

 Forall ?ps ?p ?scs ?s ?t such that (
  External(wkfl:variable-value(ps ?ps))
  ?p#?ps  ?p[securities->?scs]
   ?s#?scs ?s[ticker->?t] )
( If And( Not(External(wkfl:
     end-of-task(getTweets)))
   External(wkfl:end-of-task(getPortfolios)))
  Then
  Do( Modify(?s[positive_tweets->
   External(ofws:computeSecPosTweets(?t))] )
   Assert(External(wkfl:
    end-of-task(getTweets))) )
)

Forall ?ps ?p such that (
 External(wkfl:variable-value(ps  ?ps))
  ?p#?ps)
( If And(Not(External(wkfl:
   end-of-task(countTwitterMetrics)))
   External(wkfl:end-of-task(getTweets)) )
  Then Do(
   Modify(?p[twitter_positive_ratio->
    External(func:numeric-divide(
    Sum{?pt | Exists
     ?scs ?s(?p[securities->?scs]
     ?s#?scs  ?s[positive_tweets->?pt])}
    External(func:count(?ps))))])
   Assert(External(wkfl:
    end-of-task(countTwitterMetrics)))
)	)) )
\end{verbatim}

\begin{figure*}[b] %fig5
\vspace*{-4pt}
 \begin{center}
 \mbox{%
 \epsfxsize=162.319mm
 \epsfbox{kal-5.eps}
 }
 \end{center}
 \vspace*{-9pt}
\Caption{Portfolio problem implementation infrastructure}
  \end{figure*}

  Production rules of the document are divided into two groups. The first group with
priority~2 contains rules defining workflow parameters and variable. Input parameters
are \textit{start date} and \textit{end date} of historical rates used for calculation of
\textit{portfolio metrics}. Workflow variable {\sf ps} denotes a~set containing
\textit{portfolio candidates}.

  The second group with priority~1 contains the orchestration rules~--- workflow
skeleton. The only orchestration rule provided in the example above corresponds to
the task {\sf getPortfolios}. The external function {\sf getPortfolios}
encapsulates a~multidialect logic program calculating portfolio candidates~[1,
Appendix]. A~{\sf Modify} action is used to call the function and to put the
returned result into the {\sf ps} variable.

\vspace*{-6pt}

\subsection*{A.3\	Revised portfolio problem infrastructure}

  \noindent
  The implementation structure of the use case is shown in Fig.~5.




  The RIF-PRD workflow skeleton was transformed into a~program (rule set) in the
ILOG~\cite{13-kal} language combining production rules and workflow facilities
(like {\sf fork} and {\sf sequence}). The ILOG program was executed in the
{IBM Operational Decision Manager} tool~\cite{24-kal}. In order to execute
ILOG programs, the underlying execution model (XOM)~\cite{25-kal}
was defined as a~set of
Java classes: {\sf Portfolio}, {\sf Security}, and {\sf StockRate}. The
{\sf Portfolio class} corresponds to a~financial portfolio and contains as attributes a~set of
securities in it, its expected return, standard deviation, Sharpe ratio, and twitter
positive ratio. Code of this class is provided below:
  \begin{verbatim}
public class Portfolio {
 private Collection<Security> securities;
 private double expected_return;
 private double std_dev;
 private double sharpe_ratio;
 private double twitter_positive_ratio;
 // as of 05.04.14 US 5-year treasuries
 private static double risk_free_rate = 0.0169;
 private boolean recommended;
}
\end{verbatim}

  Class {\sf Security} corresponds to real world financial securities. The class
contains as attributes a~ticker, ratio of positive tweet number to the sum of positive
and negative tweets, a~set of stock rates, security's standard deviation, and expected
return. These attributes are set as responses to corresponding web services queries:

\vspace*{-2pt}

\noindent
  \begin{verbatim}
public class Security {
 public String ticker;
 public double positive_tweets;
 public Collection<StockRate> rates;
 public double std_dev;
 public double expected_return;
 public static int number_of_periods = 5;
}	
\end{verbatim}

\vspace*{-2pt}

{\sf StockRate} is a~simple class and contains just two attributes~--- price and date:

\vspace*{-2pt}

\noindent
  \begin{verbatim}
public class StockRate {
 public float price;
 public String date;
}
\end{verbatim}

\vspace*{-2pt}

  It is easy to see that the one-to-one mapping exists between conceptual schema
entities and execution model entities.

  Parameters of RIF-PRD workflow skeleton ({\sf startDate}, {\sf endDate}, and
{\sf bestPortfolio}) are mapped into the respective parameters of ILOG rule set
(Fig.~6).

\begin{figure*} %fig6
\vspace*{1pt}
 \begin{center}
 \mbox{%
 \epsfxsize=115mm
 \epsfbox{kal-6.eps}
 }
 \end{center}
 \vspace*{-9pt}
\Caption{Rule set parameters}
  \end{figure*}

  The variable of RIF-PRD workflow skeleton ({\sf ps}) is mapped into a~local variable
of the rule set. Specification of the variable looks as follows:

\vspace*{-2pt}

\noindent
  \begin{verbatim}
<?xml version="1.0" encoding="UTF-8"?>
<ilog.rules.studio.model.base:VariableSetxmi:
  version="2.0"
xmlns:xmi="http://www.omg.org/XMI"
  xmlns:ilog.rules.studio.model.base =
"http://ilog.rules.studio/model/base.ecore">
 <name>local_vars</name>
 <variables name="ps" type="java.util.ArrayList"
   initialValue=""verbalization="ps"/>
</ilog.rules.studio.model.base:VariableSet>
\end{verbatim}

  Rules of the RIF-PRD workflow skeleton are mapped into ILOG
\textit{ruleflow}~\cite{25-kal}:

\vspace*{-6pt}

\noindent
  \begin{verbatim}
flowtask portfolio$_$flow {
 property mainflowtask = true;
 property ilog.rules.business_name =
  "portfolio_flow";
 body {
  portfolio$_$flow#getPortfolios;
  fork {
   portfolio$_$flow#getRates;
   portfolio$_$flow
   #computePortfolioFinancialMetrics;} &&
  { portfolio$_$flow#getTweets;
   portfolio$_$flow#
    computePortfolioTwitterMetrics;}
  portfolio$_$flow#choosePortfolio;
 }
};

ruletask portfolio$_$flow#getPortfolios {
 property ilog.rules.business_name =
   "portfolio_flow>getPortfolios";
 body { getPortfolios.*}
};

ruletask portfolio$_$flow#
   computePortfolioTwitterMetrics {
 propertyilog.rules.business_name =
  "portfolio_flow>
   computePortfolioTwitterMetrics";
 body { computePortfolioTwitterMetrics.* }
};

ruletask portfolio$_$flow#getTweets {
 property ilog.rules.business_name =
  "portfolio_flow>getTweets";
 property ilog.rules.package_name = "";
 body {getTweets.*}
};
\end{verbatim}

  The {\sf computePortfolioTwitterMetrics},
{\sf computePortfolioFinancialMetrics}, and {\sf choosePortfolio} tasks are
implemented as production rules in ILOG:

\vspace*{-6pt}

\noindent
  \begin{verbatim}
package computePortfolioTwitterMetrics {
 use ps;
 import portfolio.*;

 rule computePortfolioTwitterMetrics {
  property status = "new";
  when {	IlrContext() from ?context;	}
  then {
   foreach (Portfolio p in ps) {
    double ?twitter_metrics = 0;
    int ?length = 0;
     foreach (Security security
       in p.securities) {
      ?twitter_metrics= ?twitter_metrics +
       security.positive_tweets;
      ?length = ?length + 1; }
     p.twitter_positive_ratio=
      ?twitter_metrics / ?length;
}}}}
\end{verbatim}

  The {\sf getPortfolios} and {\sf computeSecurityFinancialMetrics} tasks are
implemented by the following production rules in ILOG:


\noindent
  \begin{verbatim}
package getPortfolios {
 use ps;
 import portfolio.*;

 rule getPortfolios {
  when { IlrContext() from ?context; }
  then {
   ps = Supervisor.getPortfolios(startDate,
    endDate);
} } }
\end{verbatim}

\begin{table*}\small
\begin{center}
\Caption{Metrics for the securities}
  \vspace*{2ex}

  \begin{tabular}{cccc}
  \hline
Security identifier&Expected return&Standard deviation&Positive tweet ratio\\
\hline
COG&0.163&0.201&0.507\\
DO&0.015&0.019&0.651\\
EQR&0.150&0.022&0.846\\
FOSL&0.513&0.030&0.579\\
SCG&0.050&0.010&0.622\\
\hline
\end{tabular}
\end{center}
%\end{table*}
%\begin{table*}\small
\begin{center}
\Caption{Metrics for the portfolio candidates}
  \vspace*{2ex}

  \begin{tabular}{lcccccc}
  \hline
\multicolumn{1}{c}{\tabcolsep=0pt\begin{tabular}{c}Portfolio\\ identifier\end{tabular}}&
\tabcolsep=0pt\begin{tabular}{c}Expected\\ return\end{tabular}&
\tabcolsep=0pt\begin{tabular}{c}Standard\\ deviation\end{tabular}&
\tabcolsep=0pt\begin{tabular}{c}Risk free\\ rate\end{tabular}&
\tabcolsep=0pt\begin{tabular}{c}Sharpe\\ ratio\end{tabular}&
\tabcolsep=0pt\begin{tabular}{c}Positive\\ tweet ratio\end{tabular}&
\tabcolsep=0pt\begin{tabular}{c}Sharpe ratio\\$\times$\;Positive tweet ratio\end{tabular}\\
\hline
1&0.111&0.008&0.0169&11.755&0.660&7.758\\
2&2.400&0.507&0.0169&\hphantom{9}4.701&0.508&2.388\\
3&2.381&0.508&0.0169&\hphantom{9}4.662&0.557&2.597\\
4&2.347&0.505&0.0169&\hphantom{9}4.606&0.708&3.261\\
5 (best)&0.178&0.011&0.0169&14.227&0.641&9.120\\
6&0.147&0.008&0.0169&15.577&0.521&8.166\\
\hline
\end{tabular}
\end{center}
\vspace*{-3pt}
\end{table*}



\noindent
  Here, the {\sf Supervisor} is the~Java class wrapping execution of logic programs
in multidialect infrastructure including two nodes~[1]. The nodes correspond to the
mediation system (which integrates \textit{Google Finance} and the \textit{Yahoo!
Finance} services) and to a~rule-based programming system DLV.

  The {\sf getSecurityFinancialMetrics} task uses the same instance of the
mediation system as the {\sf getPortfolios} task. The reason is that financial
metrics are calculated using the historical rates of the securities. This is exactly the
information that is extracted by the mediation system from {Google Finance}
and {Yahoo! Finance}. The difference between two tasks is that the
{\sf getPortfolios} is implemented as a~submission of a~query to the DLV node, but
the {\sf getSecurityFinancialMetrics} is implemented as a~submission of a~different
query to the Mediation Node.

  The {\sf getPositiveTweetRatio} task is implemented by the following
production rule in ILOG:

%\pagebreak

\noindent
  \begin{verbatim}
package getTweets {
 use ps;
 import portfolio.*;

 rule getTweets {
  when { IlrContext() from ?context; }
  then {
   foreach (Portfolio p in ps) {
    foreach (Security s in p.securities) {
     s.positive_tweets =
      WebServices.computeSecPosTweets(s.ticker);
} } } } }
\end{verbatim}




\noindent
Here, {\sf WebServices} is the~Java-class wrapping invocation of a~web-service.
The WSDL specification of the web-service can be found at {\sf
http://synthesis.ipi.ac.ru/synthesis/ projects/RuleInt/OpinionFinderWS}. The
  web-service, in its turn, encapsulates a~Java-program. The program first collects
tweets using the {\sf Twitter Streaming API}. After that, a~sentiment analysis is
done by the {\sf Polarity Classifier} of the {\sf OpinionFinder}
  tool~\cite{14-kal} which assesses if tweet is positive, negative, or neutral. Finally,
the sentiment ratio for every security in a~portfolio is calculated and returned as the
result.

\vspace*{-6pt}

\subsection*{A.4\	Result of~the~use case workflow execution}

  \noindent
  The results obtained by one of the use case runs are as follows. The task
{\sf getPortfolios} computes portfolio candidates on the basis of historical rates of
daily closing prices of securities from S\&P500 list for the 2011--2013. Six portfolios
of size~5 were calculated. Each portfolio is a~set of identifiers (tickers) of
companies:
  \begin{verbatim}
Candidate 1: { ALXN, BF.B, EW, POM, VNO }
Candidate 2: { BMC, JBL, LUK, MNST, POM }
Candidate 3: { AVP, BMC, JPL, MNST, POM }
Candidate 4: { ALTR, BF.B, BMC, DGX, PEG }
Candidate 5: { COG, DO, EQR, FOSL, SCG }
Candidate 6: { ADSK, GILD, INTC, POM, TJX }
\end{verbatim}

  The task {\sf getSecurityFinancialMetrics} computes the expected return and
the standard deviation for every security mentioned in portfolio candidates.
 The task
{\sf getPositiveTweetRatio} computes positive sentiment ratios for every security
mentioned in portfolio candidates (500~latest tweets for every security were used for
the computation). Financial and twitter metrics for several securities are provided in
Table~1.



  The task {\sf computePortfolioFinancialMetrics} computes financial metrics for
every portfolio candidate on the basis of respective metrics for
securities in a~portfolio. The task {\sf computePortfolioTwitterMetrics} computes sentiment
metrics for every portfolio candidate on the basis of sentiment metrics for securities in
a~portfolio. Financial and twitter metrics for portfolio candidates are provided in
Table~2. The task {\sf choosePortfolio} identifies the best portfolio by maximum
value of the products of Sharpe ratio and positive tweet ratio obtained for every
portfolio (see Table~2).


}

\vspace*{-9pt}

\Ack
\noindent
This research has been done under the support of the \mbox{RFBR} (projects13-07-00579,
14-07-00548) and the Program for Basic Research of the Presidium of RAS.

\renewcommand{\bibname}{\protect\rmfamily References}

\vspace*{-9pt}

{\small\frenchspacing
{%\baselineskip=10.8pt
\begin{thebibliography}{99}

\bibitem{1-kal}
\Aue{Kalinichenko, L.\,A., S.\,A.~Stupnikov, A.\,E.~Vovchenko, and D.\,Y.~Kovalev}.
2013. Conceptual declarative problem specification and solving in data intensive domains.
\textit{Informatics and Applications}~--- \textit{Inform \mbox{Appl.}} 7(4):112--139.
Available at: {\sf http://synthesis.ipi.ac.\linebreak ru/synthesis/publications/13ia-multidialect}
 (accessed December~9, 2014).
\bibitem{2-kal}
\Aue{Kalinichenko, L.\,A., S.\,A.~Stupnikov, and D.\,O.~Martynov}. 2007.
\textit{SYNTHESIS: A~language for canonical information modeling and mediator
definition for problem solving in heterogeneous information resource environments}.
Moscow: IPIRAN. 171~p.
\bibitem{3-kal}
Boley, H., and M.~Kifer, eds. 2013. {RIF framework for logic dialects. W3C
recommendation}. 2nd ed. Available at:
{\sf http://www.w3.org/TR/2013/REC-rif-fld-20130205/}
(accessed December~9, 2014).

\bibitem{4-kal}
Boley, H., and M.~Kifer, eds. 2013. {RIF basic logic dialect. W3C Recommendation}.
2nd ed. Available at:
{\sf http://www.w3.org/TR/2013/REC-rif-bld-20130205/}
(accessed December~9, 2014).


\bibitem{5-kal}
Heymans, S., and M.~Kifer, eds. 2009. {RIF core answer set programming dialect}.
Available at: {\sf http:// ruleml.org/rif/RIF-CASPD.html} (accessed November~5, 2014).
\bibitem{6-kal}
\Aue{Leone, N., G.~Pfeifer, W.~Faber, T.~Eiter,  G.~Gottlob, S.~Perri, and F.~Scarcello}.
2006. The DLV system for knowledge representation and reasoning. \textit{ACM Trans.
Comput. Logic} 7(3):499--562.
\bibitem{7-kal}
DeSante, M.\,C., G.~Hallmark, and A.~Paschke, eds. 2013. {RIF production rule
dialect. W3C Recommendation}. 2nd ed.
Available at: {\sf http://www.w3.org/TR/2013/REC-rif-prd-20130205/}
(accessed December~9, 2014).

\bibitem{8-kal}
Motik, B., P.\,F.~Patel-Schneider, and B.~Parsia, eds.
2012. {OWL~2 Web Ontology Language structural
specification and functional-style syntax. W3C Recommendation}. 2nd ed.
Available at:
{\sf http://www.w3.org/TR/owl2-syntax/} (accessed November~5, 2014).
\bibitem{22-kal} %9
\Aue{Calvanese, D., G.~De Giacomo, D.~Lembo, M.~Lenzerini, A.~Poggi, and
R.~Rosati}.
 2007. Ontology-based database access. \textit{15th
Italian Symposium on Advanced Database Systems Proceedings}. 324--331.

\bibitem{9-kal} %10
\Aue{Ramakrishnan, L., and B.~Plale}. 2010. A~multi-dimensional classification model for
scientific workflow\linebreak characteristics. \textit{1st Workshop (International) on Workflow
Approaches to New Data-Centric Science Proceedings}. New York: ACM.
Article No.\,4. 12~p.
Available at: {\sf http://dl.acm.org/citation.cfm?id=1833402}\linebreak
(accessed December~9, 2014).

\bibitem{17-kal} %11
\Aue{Boukhebouze, M, Y.~Amghar, A.-N.~Benharkat,  and Z.~Maamar}. 2011.
A~rule-based approach to model and verify flexible business processes. \textit{Int.
J.~Business Process Integration Management} 5(4):287--307.

\bibitem{21-kal} %12
Polleres, A., H.~Boley, and M.~Kifer, eds. 2013. {RIF datatypes and Built-Ins~1.0
W3C Recommendation.} 2nd ed.
Available at: {\sf http://www.w3.org/TR/2013/REC-rif-dtb-20130205/}
(accessed December~9, 2014).




\bibitem{16-kal} %13
Production Rule Representation (PRR), Version 1.0. OMG Document Number:
formal/2009-12-01. Available at: {\sf http://www.omg.org/spec/PRR/1.0} (accessed
November~5, 2014).

\bibitem{15-kal} %14
\Aue{Yu,~J., and R.~Buyya}. 2005. A~taxonomy of scientific workflow systems for grid
computing. \textit{ACM SIGMOD Records} 34(3):44--49.

\bibitem{18-kal} %15
\Aue{Kowalski, R., and F.~Sadri}. 2009. Integrating logic programming and production
systems in abductive logic programming agents.
\textit{Web reasoning and rule systems}. Eds. A.~Polleres and T.~Swift.
Lecture notes in computer science ser. Springer.
5837:1--23.

\bibitem{19-kal} %16
\Aue{Cosentino, V., M.\,D.~Del Fabro, and A.~El Ghali}. 2012. A~model driven approach
for bridging ILOG rule language and RIF. \textit{6th Symposium (International) on Rules
RuleML Proceedings}. CEUR-WS.org. 874:96--102.
\bibitem{20-kal} %17
\Aue{Veiga, F.\,D.\,J.} 2011. Implementation of the RIF-PRD.  Universidade
Nova de Lisboa. Master Thesis. Available at: {\sf
http://run.unl.pt/bitstream/10362/6310/1/Veiga\_\linebreak 2011.pdf} (accessed November~5, 2014).

\bibitem{10-kal} %18
\Aue{Markowitz, H.\,M.} 1991. \textit{Portfolio selection: Efficient diversification of
investments}. Wiley. 402~p.
\bibitem{11-kal} %19
\Aue{Sharpe, W.\,F.} 1966. Mutual fund performance. \textit{J.~Business}
39(S1):119--138.
\bibitem{12-kal} %20
\Aue{Bollen,~J., H.~Maoa, and X.~Zeng}. 2011. Twitter mood predicts the stockmarket.
\textit{J.~Comput. Sci.} 2(1):1--8.
\bibitem{13-kal} %21
IBM WebSphere ILOG JRules Version~7.0. Online documentation. Available at: {\sf
http://pic.dhe.ibm.com/\linebreak infocenter/brjrules/v7r0/index.jsp} (accessed November~5, 2014).

\bibitem{24-kal} %22
IBM Operational Decision Manager. Available at:
{\sf http:// www-03.ibm.com/software/products/en/odm} (accessed November~5, 2014).
\bibitem{25-kal} %23
IBM Operational Decision Manager Version~8.5 Information Center. Available at: {\sf
http://pic.dhe.ibm.com/\linebreak infocenter/dmanager/v8r5/index.jsp} (accessed November~5, 2014).


\bibitem{14-kal} %24
\Aue{Wilson, T., J.~Wiebe, and P.~Hoffmann}. 2005.
Recognizing contextual polarity in phrase-level sentiment Analysis. \textit{Conference on
Human Language Technology and Empirical Methods in Natural Language Processing
Proceedings}. Stroudsburg: Association for Computational Linguistics. 347--354.


%\bibitem{23-kal}
%Bock, C., \textit{et. al.}, eds. 2012. \textit{OWL~2 Web Ontology Language Structural
%Specification and Functional-Style Syntax. W3C Recommendation}. 2nd ed.


\end{thebibliography} } }

\end{multicols}

\vspace*{-9pt}

\hfill{\small\textit{Received November 3, 2014}}

\vspace*{-18pt}

\Contr

\noindent
\textbf{Kalinichenko Leonid A.} (b.\ 1937)~---
 Doctor of Science in physics and mathematics, professor;
 Head of Laboratory, Institute of Informatics Problems, 44-2 Vavilov Str.,
 Moscow 119333, Russian Federation; professor,
 Faculty of Computational Mathematics and Cybernetics, M.\,V.~Lomonosov Moscow
 State University, 1-52 Leninskiye Gory, GSP-1, Moscow 119991,
 Russian Federation; leonidandk@gmail.com

 \vspace*{3pt}

 \noindent
 \textbf{Stupnikov Sergey A.} (b.\ 1978)~---
 Candidate of Science (PhD) in technology, senior scientist,
 Institute of Informatics Problems, Russian Academy of Sciences,
 44-2 Vavilov Str.,
 Moscow 119333, Russian Federation; ssa@ipi.ac.ru

 \vspace*{3pt}

 \noindent
 \textbf{Vovchenko Alexey E.} (b.\ 1984)~---
 Candidate of Science (PhD) in technology, senior scientist,
 Institute of Informatics Problems, Russian Academy of Sciences,
 44-2 Vavilov Str.,
 Moscow 119333, Russian Federation; itsnein@gmail.com

 \vspace*{3pt}

 \noindent
 \textbf{Kovalev Dmitry Yu.} (b.\ 1988)~---
 junior scientist, Institute of Informatics Problems, Russian Academy of Sciences,
 44-2 Vavilov Str.,
 Moscow 119333, Russian Federation; dm.kovalev@gmail.com


%\vspace*{24pt}

%\hrule

%\vspace*{2pt}

%\hrule

%\vspace*{-6pt}

\newpage


\def\tit{КОНЦЕПТУАЛЬНОЕ МОДЕЛИРОВАНИЕ МУЛЬТИДИАЛЕКТНЫХ ПОТОКОВ РАБОТ$^*$}

\def\aut{Л.\,А.~Калиниченко$^{1,2}$, С.~Ступников$^1$, А.~Вовченко$^1$, Д.~Ковалев$^1$}


\def\titkol{Концептуальное моделирование мультидиалектных потоков работ}

\def\autkol{Л.\,А.~Калиниченко, С. Ступников, А. Вовченко, Д. Ковалев}

{\renewcommand{\thefootnote}{\fnsymbol{footnote}}
\footnotetext[1]{Работа выполнена при поддержке РФФИ (проекты
13-07-00579, 14-07-00548) и~Программы фундаментальных исследований Президиума РАН.}}


\titel{\tit}{\aut}{\autkol}{\titkol}

\vspace*{-12pt}

\noindent
$^1$Институт проблем информатики Российской академии наук

\noindent
$^2$Московский государственный университет им.\ М.\,В.~Ломоносова, факультет вычислительной
матема-\linebreak
$\hphantom{^1}$тики и~кибернетики

\vspace*{6pt}

\def\leftfootline{\small{\textbf{\thepage}
\hfill ИНФОРМАТИКА И ЕЁ ПРИМЕНЕНИЯ\ \ \ том\ 8\ \ \ выпуск\ 4\ \ \ 2014}
}%
 \def\rightfootline{\small{ИНФОРМАТИКА И ЕЁ ПРИМЕНЕНИЯ\ \ \ том\ 8\ \ \ выпуск\ 4\ \ \ 2014
\hfill \textbf{\thepage}}}


\Abst{Рассматриваются методы концептуального представления
алгоритмов анализа данных, средств интеграции данных, а~также процессов,
направленных на спецификацию семантики данных и~поведения в~единой парадигме.
Расширяется новый подход к~применению комбинации семантически различных
плат\-фор\-мо\-не\-за\-ви\-си\-мых языков на правилах (диалектов) для создания
интероперабельных концептуальных спецификаций над различными системами на правилах.
Подход опирается на методику трансформации программ на правилах, рекомендованную
стандартом W3C Rule Interchange Format (RIF). Подход, предлагаемый в~стандарте RIF,
сочетается со технологией семантической интеграции неоднородных баз данных
в~предметных посредниках. Статья расширяет предыдущие исследования авторов
в~направлении моделирования потоков работ для определения композиций
алгоритмических модулей в~процессной структуре. Рассмотрены возможности
спецификации задач в~мультидиалектных потоках работ с~применением семантически
различных языков, наиболее подходящих для конкретных задач. Приведен практический
пример потока работ, задачи которого специфицированы с~использованием нескольких
 языков на правилах (RIF-CASPD, RIF-BLD, RIF-PRD). Для определения концептуальной
 схемы использован язык OWL~2, для оркестровки потока работ использован язык
 RIF-PRD. Инфраструктура реализации примера включает систему на продукционных
 правилах (IBM ILOG), систему на логических правилах (DLV) и~предметный посредник.}

\KW{концептуальные спецификации; потоки работ; RIF; языки продукционных правил;
интеграция баз данных; посредники; PRD; мультидиалектная инфраструктура}

\DOI{10.14357/19922264140413}

\vspace*{6pt}


 \begin{multicols}{2}

\renewcommand{\bibname}{\protect\rmfamily Литература}
%\renewcommand{\bibname}{\large\protect\rm References}

{\small\frenchspacing
{%\baselineskip=10.8pt
\begin{thebibliography}{99}

\bibitem{1-kal-1}
\Au{Kalinichenko L.\,A., Stupnikov S.\,A.. Vovchenko~A.\,E.,
Kovalev~D.\,Y.}
Conceptual declarative problem specification and solving in data intensive domains~//
Информатика и~её применения, 2013. Т.~7. Вып.~4. С.~112--139.
{\sf http://synthesis.ipi.ac.ru/synthesis/publications/13ia-multidialect}.
\bibitem{2-kal-1}
\Au{Kalinichenko L.\,A., Stupnikov~S.\,A., Martynov~D.\,O.}
 SYNTHESIS: A~language for canonical information modeling and mediator definition
 for problem solving in heterogeneous information resource environments.~---
 Moscow: IPI RAN, 2007. 171~p.
\bibitem{3-kal-1}
RIF framework for logic dialects. W3C Recommendation~/
Eds. H.~Boley, M.~Kifer. 2nd ed.
{\sf http:// www.w3.org/TR/2013/REC-rif-fld-20130205/}.
\bibitem{4-kal-1}
RIF basic logic dialect. W3C Recommendation~/
Eds. H.~Boley, M.~Kifer. 2nd ed.
{\sf http://www.w3.org/ TR/2013/REC-rif-bld-20130205/}.
\bibitem{5-kal-1}
RIF core answer set programming dialect~/
Eds.\ S.~Heymans, M.~Kifer, 2009. {\sf  http://ruleml.org/rif/RIF-CASPD.html}.
\bibitem{6-kal-1}
\Au{Leone N., Pfeifer G., Faber~W., Eiter~T., Gottlob~G., Perri~S., Scarcello~F.}
The DLV system for knowledge representation and reasoning~//
 ACM Trans. Comput. Logic, 2006. Vol.~7. No.\,3. P.~499--562.
\bibitem{7-kal-1}
RIF production rule dialect. W3C Recommendation~/
Eds.\ De Sante Marie~C., Hallmark~G., A.~Paschke.~ 2nd ed.
{\sf http://www.w3.org/TR/2013/REC-rif-prd-20130205/}.
\bibitem{8-kal-1}
OWL~2 Web Ontology Language Structural Specification and Functional-Style Syntax.
W3C Recommendation~/ Eds. B.~Motik, P.\,F.~Patel-Schneider, B.~Parsia. 2nd ed.
{\sf http://www.w3.org/TR/owl2-syntax/}.

\bibitem{22-kal-1} %9
\Au{Calvanese, D., De Giacomo~G., Lembo~D., Lenzerini~M., Poggi~A.,
Rosati~R.}
Ontology-based database access~// 15th Italian Symposium on Advanced
Database Systems Proceedings, 2007. P.~324--331.

\bibitem{9-kal-1} %10
\Au{Ramakrishnan L., Plale~B.}
A~multi-dimensional classification model for scientific workflow characteristics~//
1st  Workshop (International) on Workflow Approaches to New Data-Centric Science
Proceedings.  New York: ACM, 2010. Aricle No.\,4. 12~p.
{\sf http://dl. acm.org/citation.cfm?id=1833402}.
\pagebreak

\bibitem{17-kal-1} %11
\Au{Boukhebouze M., Amghar~Y., Benharkat~A.-N., Maamar~Z.}
A~rule-based approach to model and verify flexible business processes~//
Int. J.~Business Process Integration Management, 2011. Vol.~5. No.\,4. P.~287--307.

\bibitem{21-kal-1} %12
RIF Datatypes and Built-Ins 1.0. W3C Recommendation~/
Eds.\ A.~Polleres, H.~Boley, M.~Kifer. 2nd ed.
{\sf http://www.w3.org/TR/2013/REC-rif-dtb-20130205/}.



%\pagebreak

\bibitem{16-kal-1} %13
Production Rule Representation (PRR), Version~1.0.
OMG Document Number: formal/2009-12-01. {\sf http:// www.omg.org/spec/PRR/1.0}.

\bibitem{15-kal-1} %14
\Au{Yu J., Buyya~R.}
 A~taxonomy of scientific workflow systems for grid computing~//
 ACM SIGMOD Records, 2005. Vol.~34. No.\,3. P.~44--49.

 \bibitem{18-kal-1} %15
\Au{Kowalski R., Sadri~F.}
Integrating logic programming and production systems in abductive logic programming
agents~//
Web reasoning and rule systems~/ Eds. A.~Polleres, T.~Swift.
Lecture notes in computer science ser.~--- Springer, 2009. Vol.~5837. P.~1--23.
\bibitem{19-kal-1} %16
\Au{Cosentino V., Del Fabro~M.\,D., El Ghali~A.}
 A~model driven approach for bridging ILOG rule language and RIF~//
 6th  Symposium (International) on Rules, RuleML 2012 Proceedings.  2012.
 CEUR-WS.org. Vol.~874. P.~96--102.
\bibitem{20-kal-1} %17
\Au{Veiga F.\,D.\,J.}
Implementation of the RIF-PRD. Universidade Nova de Lisboa, 2011. Master Thesis.


\bibitem{10-kal-1} %18
\Au{Markowitz H.\,M.}
Portfolio selection: Efficient diversification of investments. Wiley, 1991.
402~p.
\bibitem{11-kal-1} %19
\Au{Sharpe, W.\,F.}
Mutual fund performance~// J.~Business, 1966. Vol.~39(S1). P.~119--138.

\bibitem{12-kal-1} %20
\Au{Bollen J., Maoa H., Zeng~X.}
Twitter mood predicts the stock market~// J.~Comput. Sci., 2011. Vol.~2. No.\,1.
P.~1--8.

\bibitem{13-kal-1} %21
IBM WebSphere ILOG JRules Version 7.0. Online documentation.
{\sf http://pic.dhe.ibm.com/infocenter/\linebreak brjrules/v7r0/index.jsp}.

 \bibitem{24-kal-1} %22
 IBM Operational Decision Manager.
 {\sf http://www-03.\linebreak ibm.com/software/products/en/odm}.
\bibitem{25-kal-1} %23
IBM Operational Decision Manager Version~8.5 Information Center.
{\sf http://pic.dhe.ibm.com/infocenter/\linebreak dmanager/v8r5/index.jsp}.

\bibitem{14-kal-1} %24
\Au{Wilson T., Wiebe~J., Hoffmann~P.}
Recognizing contextual polarity in phrase-level sentiment analysis.
\textit{Conference on Human Language Technology and Empirical Methods in Natural
Language Processing Proceedinhgs}.
Stroudsburg: Association for Computational Linguistics, 2005. P.~347--354.

\end{thebibliography}
} }

\end{multicols}

 \label{end\stat}

 \vspace*{-3pt}

\hfill{\small\textit{Поступила в редакцию 03.11.2014}}
%\renewcommand{\bibname}{\protect\rm Литература}
\renewcommand{\figurename}{\protect\bf Рис.}  %1рис
\def\stat{stupnikov}

\def\tit{ВЕРИФИЦИРУЕМОЕ ОТОБРАЖЕНИЕ МОДЕЛИ ДАННЫХ, ОСНОВАННОЙ НА~МНОГОМЕРНЫХ МАССИВАХ, 
В~ОБЪЕКТНУЮ~МОДЕЛЬ ДАННЫХ$^*$}

\def\titkol{Верифицируемое отображение модели данных, основанной на~многомерных массивах, 
в~объектную модель данных}

\def\autkol{С.\,А.~Ступников}

\def\aut{С.\,А.~Ступников$^1$}

\titel{\tit}{\aut}{\autkol}{\titkol}

{\renewcommand{\thefootnote}{\fnsymbol{footnote}}\footnotetext[1] {Работа 
выполнена при поддержке РФФИ (проект 11-07-00402-а). Статья рекомендована к 
публикации в журнале Программным комитетом конференции <<Электронные 
библиотеки: перспективные методы и технологии, электронные коллекции>> 
(RCDL-2012).}}

\renewcommand{\thefootnote}{\arabic{footnote}}
\footnotetext[1]{Институт проблем информатики Российской академии наук, 
ssa@ipi.ac.ru}

\vspace*{-6pt}       

\Abst{Рассматривается отображение модели данных, основанной на 
многомерных мас\-си\-вах (ММ-мо\-де\-ли), в объектную модель данных. Изложены 
общие принципы отображения ММ-мо\-де\-лей в объектные модели данных. 
Рассмотрено отображение конкретной модели~--- Array Data Model (ADM), 
использующейся в системе управления базами данных (СУБД) SciDB, в язык СИНТЕЗ, 
использующийся в качестве канонической модели данных в технологии предметных 
посредников. Проиллюстрирован метод верификации отображения~--- доказательства 
сохранения информации и семантики операций при отображении. Верификация 
осуществляется при помощи формального языка спецификаций AMN. Практической 
целью работы ставилось создание базы для виртуальной или материализованной 
интеграции ресурсов, основанных на многомерных массивах.}

\vspace*{-1pt}

\KW{многомерные массивы; объектная модель данных; отображение моделей 
данных; интеграция баз данных}

\vspace*{-6pt}

\vskip 14pt plus 9pt minus 6pt

      \thispagestyle{headings}

      \begin{multicols}{2}

            \label{st\stat}
            

\section{Введение}

        Развитие науки и промышленности, широкое распространение 
информационных технологий ведет к накоплению огромных объемов данных 
как в науке, так и в бизнесе. Данные могут быть как наблюдательными, 
экспериментальными, так и полученными в ходе компьютерного 
моделирования. Данные таких масштабов (часто измеряемых уже в петабайтах) 
называются <<большими данными>> (Big Data)~\cite{1-stu}. Они плохо 
поддаются обработке и анализу в рамках широко известных технологий баз 
данных, опирающихся в основном на реляционную модель данных.
        
        Именно поэтому развиваются различные модели данных, нацеленные на 
параллельную обработку и анализ данных в распределенных средах~--- гридах 
и облаках. Важными видами таких моделей являются модели данных, 
основанные на многомерных массивах (array-based data models, или ADM) 
и называемые далее ММ-мо\-де\-ля\-ми. Родственны данным моделям 
так называемые <<кубы данных>>, используемые в 
OLAP (online analytical processing) тех\-но\-ло\-гии~[2--4]. 
Исследования ММ-мо\-де\-лей начались достаточно 
давно~\cite{4-stu, 5-stu} и продолжают развиваться. В~данной статье 
рассматривается конкретная модель, а именно модель, используемая в СУБД 
SciDB~\cite{6-stu}.
        
        История SciDB начинается с 2007~г., когда на симпозиуме по 
экстремально большим базам данных (XLDB~--- extremely large data bases) 
представителями науки и 
промышленности был сделан вывод о том, что существующие СУБД не в 
состоянии манипулировать объемами данных, которые появятся в ближайшем 
будущем. Одним из примеров поставщиков таких данных служит строящийся 
телескоп LSST (Large Synoptic Survey Telescope)~\cite{7-stu}. Был также сделан 
вывод о необходимости разработки СУБД нового поколения, которая должна 
удовлетворять, в частности, следующим требованиям~\cite{8-stu}:
        \begin{itemize}
\item модель данных основывается на многомерных массивах, а не на 
кортежах;
\item модель хранения базируется на версионности, а не на обновлении 
значений;
\item масштабируемость до сотен петабайт и высокая отказоустойчивость;
\item СУБД является свободно распространяемым программным 
обеспечением.
\end{itemize}

        Некоторое время спустя был запущен международный проект под 
руководством Майкла Стоунбрейкера, целью которого стало создание новой 
СУБД, получившей название SciDB. В~настоящее 
время свободно распространяется очередная версия системы для операционных
сис\-тем (ОС) Ubuntu и  RedHat.
        
        Целью данной статьи является исследование вопроса о верифицируемом 
отображении ММ-мо\-де\-лей, и в частности ADM~\cite{9-stu}, 
использующейся в системе SciDB, в объектные 
модели данных для виртуальной или материализованной интеграции ресурсов 
при создании федеративных баз данных или хранилищ данных. 
        
        При материализованной интеграции предполагается создание 
хранилища данных (warehouse), в которое загружаются ресурсы, подлежащие 
интеграции. В~процессе загрузки происходит преобразование данных из схемы 
ресурса в общую схему хранилища.
        
        Виртуальная же интеграция рассматривается в статье применительно к 
предметным посредникам~\cite{10-stu}. Предметные посредники представляют 
собой специальный вид программного обеспечения (ПО), образующий 
промежуточный слой между пользователем (приложением) и неоднородными 
информационными ресурсами. При этом данные из ресурсов не 
материализуются в посреднике. Федеративная схема посредника, описывающая 
некоторую предметную область, создается независимо от существующих 
ресурсов. Ресурсы, релевантные предметной области, затем регистрируются в 
посреднике~--- их схемы связываются специальными семантическими 
отображениями с федеративной схемой. Исполнительная среда посредников 
предо\-став\-ля\-ет возможность пользователям (приложениям) задавать запросы 
(программы) к посреднику в терминах федеративной схемы. Эти запросы 
переписываются в частичные запросы над информационными ресурсами, а 
затем исполняются на ресурсах. Результаты частичных запросов объединяются 
и выдаются пользователю также в терминах федеративной схемы.
        
        Важным понятием технологии систем интеграции баз данных является 
каноническая модель, служащая общим языком, унифицирующим 
разнообразные модели ресурсов.
        
        Необходимым предусловием интеграции ресурсов, основанных на 
многомерных массивах, является построение отображения соответствующей\linebreak 
ММ-мо-де\-ли в каноническую модель данных, сохраняющего информацию и 
семантику операций языка манипулирования данными (ЯМД)~\cite{11-stu}. 
Это обусловлено тем, что семантические отображения, связывающие 
федеративную схему и схемы ресурсов, нужно проводить в единой 
(канонической) модели~\cite{12-stu}. Отображение должно быть 
верифицируемым~--- доказуемо правильным. 
        
        В качестве канонической модели в данной работе рассматривается язык 
СИНТЕЗ~\cite{13-stu}~--- комбинированная слабоструктурированная и 
объектная модель данных, нацеленная на разработку предметных посредников 
для решения задач в средах неоднородных ресурсов. Разработан прототип 
программных средств для поддержки среды предметных посредников с языком 
СИНТЕЗ в роли канонической модели~\cite{14-stu}.
        
        С точки зрения предметных посредников СУБД, основанные на 
многомерных массивах, пред\-став\-ля\-ют собой новый вид ресурсов, подлежащих 
интеграции в посредниках вместе с привычными ресурсами~--- реляционными 
и объектными СУБД, веб-сер\-ви\-са\-ми и~т.\,д. 
        
        Нужно отметить, что ADM подвергается некоторой критике со стороны 
исследователей, продолжающих развитие моделей, основанных на 
многомерных массивах. Так, авторы языка SciQL~\cite{15-stu} отмечают, что 
язык ADM является смесью SQL и деревьев алгебраических операций. По их 
мнению, язык для СУБД, основанных на многомерных массивах, должен быть 
интегрирован с синтаксисом и семантикой SQL:2003. Несмотря на эти 
замечания, модель ADM представляет несомненный практический интерес для 
интеграции баз данных. SciDB используется как в научных проектах, связанных 
с LSST (предполагается после запуска телескопа) и физикой высоких энергий, 
так и в коммерческих, связанных с генетикой, страхованием, финансами. 
Сравнительное тестирование SciDB с СУБД Postgres и статистическим ПО R 
показало преимущества SciDB по производительности и масштабируемости.
        
        Статья организована следующим образом. В~разд.~2 рассмотрены и 
проиллюстрированы основные принципы отобра\-же\-ния модели данных ADM в 
язык СИНТЕЗ. Принципы обобщены на случай моделей, отличных от ADM и 
СИНТЕЗ. В~разд.~3 рассмотрен метод доказательства сохранения информации 
и семантики операций при отоб\-ра\-же\-нии моделей с использованием 
формального языка спецификаций AMN~\cite{16-stu}. Метод 
проиллюстрирован на структурах данных и операциях ЯМД моделей SciDB и 
СИНТЕЗ. В~разд.~4 рассмотрены некоторые родственные исследования и 
направления дальнейшей работы.

\vspace*{6pt}

\section{Отображение модели ADM в~язык СИНТЕЗ}

\vspace*{2pt}

        SciDB поддерживает два языка для работы с массивами: AQL (Array 
Query Language) и AFL (Array\linebreak Functional Language). Язык AQL является 
        SQL (Structured Query Language)
        по\-доб\-ным декларативным языком, включающим как операции 
языка описания данных (ЯОД), так и операции ЯМД. Язык AFL представляет собой функциональный язык 
манипулирования массивами, операции которого можно объединять в 
композиции. Допускается использование операций AFL в запросах AQL.
        
        Операции языков и отображение будут иллюстрироваться на 
адаптированных примерах из сценария применения SciDB в области 
оптической астрономии~\cite{17-stu}, а также на простых примерах из 
документации SciDB~\cite{9-stu}.

\subsection{Отображение языка определения данных}

        Отображение ЯОД в данном разделе описывается независимо от вида 
интеграции~--- виртуальной или материализованной.
        
        Основной единицей определения данных в модели ADM является 
массив, имеющий конечное количество {измерений} $d_1, d_2, \ldots , 
d_n$~[9]. Длиной измерения называется количество упорядоченных значений в 
этом измерении. По умолчанию типом измерения являются 64-бит\-ные целые 
числа. Поддерживаются также нецелочисленные измерения, например строки 
или числа с плавающей точкой. Каждая комбинация значений измерений 
соответствует ячейке массива, которая может содержать конечное количество 
значений, называемых \textit{атрибутами}. Типом атрибута может быть один 
из встроенных типов ADM~\cite{9-stu}.
        
        Основная операция ЯОД ADM~--- создание массива~--- выглядит 
следующим образом:
        \begin{verbatim}
CREATE ARRAY source
< ampExposureId: int64 NULL, 
   filterId: int8,
   apMag: double >
[ ra(double), de(double), objectId=0:*];
\end{verbatim}

        Создается массив оптических источников {\sf source}, измерениями 
которого являются координаты {\sf ra} и {\sf de} типа {\sf double} и целочисленный 
идентификатор объекта. Для целочисленного измерения указаны его нижняя (0) 
и верхняя (<<*>>, обозна\-ча\-ющая бесконечность) границы. Ячейка массива 
состоит из трех атрибутов: {\sf ampExposureId}, {\sf filterId}, 
{\sf apMag}. Указано, что 
атрибут {\sf ampExposureId} может принимать неопределенное значение {\sf NULL}. 
В~данном примере приведены только некоторые из реально используемых 
атрибутов и измерений.
        
        В языке СИНТЕЗ создание массива представляется определением 
одноименного класса:
        \begin{verbatim}
{ source; in: class;
  instance_type:{
  double ra;
  ra2long: {in: function; 
            params: {-ret/long}; };
  double de;
  de2long: {in: function; 
            params: {-ret/long}; };
  long objectId; metaslot lower: 0;  
  higher: inf; end
  objectIdBounds: {in: invariant;
    {{all s(source(s) -> s.objectId >= 0)}}
  };
  long ampExposureId;
  short filterId;
  double apMag;
  key: { unique; { ra, de, objectId } };
  definiteness: {obligatory;
    { ra, de, objectId, filterId, apMag } };
  };
}
\end{verbatim}

        Как измерения, так и атрибуты, составляющие ячейку, представляются в 
языке СИНТЕЗ атрибутами типа экземпляров ({\sf instance\_type}) класса. Между 
встроенными типами ADM ({\sf int8}, {\sf int64}, {\sf double} и~др.)\ и встроенными 
типами языка \mbox{СИНТЕЗ} ({\sf short}, {\sf long}, {\sf double}) устанавливается взаимно 
однозначное соответствие. Совокупность атрибутов, со\-от\-вет\-ст\-ву\-ющих 
измерениям, объявляется уникальной (инвариант {\sf key}, выражаемый 
встроенным утверждением {\sf unique}). Объявляется также, что атрибуты, 
соответствующие измерениям и не-{\sf NULL} атрибутам ADM, должны быть 
определены у всех экземпляров класса (инвариант {\sf definiteness}, выражаемый 
встроенным утверждением {\sf obligatory}).
        
        Таким образом обеспечивается сохранение отличи\-тель\-ных свойств 
многомерных массивов (<<кубов данных>>), существенным образом 
раз\-ли\-ча\-ющих измерения и атрибуты, со\-став\-ля\-ющие \mbox{ячейку}:
        \begin{itemize}
\item по набору значений измерений однозначно определяется набор 
значений атрибутов ячейки (уникальность измерений);
\item ячейка массива всегда определяется полным набором значений 
измерений (определенность измерений).
\end{itemize}

        Заметим также, что отсутствие в коллекции объекта с некоторым 
набором значений измерений означает \textit{пустую ячейку} в массиве.
        
        Для нецелочисленных измерений {\sf ra} и {\sf de} в языке СИНТЕЗ кроме 
атрибутов определяются функции {\sf ra2long}, {\sf de2long}, преобразующие 
нецелочисленные значения в целочисленные. Необходимость при\-вне\-се\-ния этих 
функций вызвана следующим. При попытке описать операции, характерные для 
ММ-мо\-де\-лей, в объектной модели (в частности, в языке СИНТЕЗ) 
выясняется необходимость применения принципиально различных механизмов 
работы с целочисленными и нецелочисленными измерениями. Это вызвано 
различием типов измерений, возможной неравномерностью шага измерения 
и~т.\,д.\linebreak Для того чтобы обеспечить возможность единообразного описания 
операций над цело\-чис\-лен\-ными и нецелочисленными измерениями и 
необходимы функции, приводящие нецелочисленные\linebreak измерения к 
целочисленным.
        
        Ограничения, связанные с нижними и верхними границами 
целочисленных измерений, пред\-став\-ля\-ют\-ся в языке СИНТЕЗ, во-пер\-вых, 
мета\-слотом соответствующего атрибута (например,\linebreak {\sf objectId}). В~метаслоте 
хранится метаинформация, связанная с атрибутом как с отдельной сущностью 
языка. В~данном случае метаслот включает два слота {\sf lower} и {\sf higher}, 
отвечающих соответственно верхней и нижней границе измерения. 
        Во-вто\-рых, создается инвариант (например, {\sf objectIdBounds}), 
предикативная спецификация которого устанавливает ограничения на значения 
измерения для каждого из объектов класса, отвечающего массиву. 
Спецификация инварианта имеет вид формулы первого порядка, где {\sf all}~--- 
квантор существования, <<\verb -> >> --- импликация.
        
        Необходимо отметить, что массив представляется в объектной модели 
множеством объектов класса (фактически кортежей значений атрибутов). При 
этом наблюдается некоторое противоречие со стремлением создателей 
        ММ-мо\-де\-лей \mbox{отойти} от моделей, основанных на кортежах. Однако в 
контексте интеграции ресурсов ММ-мо\-де\-ли это лишь один класс из 
большого множества разнообразных классов моделей данных. Представление 
специфических ММ-мо\-де\-лей в объектной модели является методологически 
и технически гораздо более простым и естественным, нежели использование 
многомерных массивов в качестве канонической модели.
        
        Изложенные принципы отображения ЯОД могут быть обобщены на 
случай, когда канонической является объектная или 
        объ\-ект\-но-ре\-ля\-ци\-он\-ная модель, отличная от языка СИНТЕЗ. 
Также не принципиален выбор модели данных, основанной на многомерных 
массивах. В~общем виде принципы отображения ЯОД выглядят следующим 
образом:
        \begin{itemize}
\item массив отображается в коллекцию типизированных объектов (класс) 
объектной модели;
\item измерения и атрибуты, составляющие ячейку массива, отображаются в 
атрибуты типа экземпляров класса;
\item между встроенными типами модели, основанной на многомерных 
массивах, и встроенными типами объектной модели устанавливается 
взаимно однозначное соответствие;
\item совокупность атрибутов, соответствующих измерениям, объявляется 
уникальной (при помощи механизма ключей, утверждений или 
инвариантов);
\item атрибуты, соответствующие измерениям и не-{\sf NULL} атрибутам ячейки 
массива, объявляются определенными (при помощи утверждений или 
инвариантов);
\item для нецелочисленных измерений в типе экземпляров дополнительно 
определяются методы, преобразующие нецелочисленные значения в 
целочисленные;
\item ограничения, связанные с нижними и верхними границами 
целочисленных измерений, отображаются при помощи инвариантов или 
встроенных утверждений о кардинальности соответствующих атрибутов. 
В~случае использования инвариантов при отображении границы измерений 
представляются также метаданными атрибута.
\end{itemize}

\subsection{Отображение языка манипулирования данными}

        При интеграции баз данных для установления семантических 
соотношений между схемами ресурсов и федеративной схемой необходимо 
отображение ЯОД исходной модели в каноническую. Язык манипулирования данными канонической 
модели, напротив, необходимо отображать в ЯМД исходной модели. Это 
связано с тем, что запросы к посреднику в канонической модели необходимо 
отображать в запросы к ресурсам.
        
        Отметим отличие виртуальной и материализованной интеграции. При 
виртуальной интеграции отображение ЯМД обеспечивает возможность 
трансляции программ на языке посредника в запросы на языке ресурсов. 
        
        В случае материализованной интеграции данные извлекаются из ресурса 
и представляются в хранилище в канонической модели. При этом программы 
на языке канонической модели исполняются непосредственно на данных. 
Отоб\-ра\-же\-ние\linebreak ЯМД нужно лишь для того, чтобы убедиться, что отображение 
моделей сохраняет информацию и семантику операций. Семантически 
правильное\linebreak отоб\-ра\-же\-ние служит базой для процесса 
        <<из\-вле\-че\-ния--пре\-образо\-ва\-ния--за\-груз\-ки>> (ETL), 
формирующего из данных ресурса данные хранилища:\linebreak ETL-про\-цесс может 
быть выражен только в терминах канонической модели.
        
        \smallskip
        
        Язык запросов (программ) модели СИНТЕЗ представляет собой 
        Datalog-по\-доб\-ный язык в объектной среде. Программа представляет 
собой набор конъюнктивных запросов (правил) вида 

\noindent
\begin{multline*}
        q(x/T): - C_1(x_1/T_1),\ldots , C_n(x_n/T_n), (X_1,Y_1), 
\ldots \\
\ldots F_m(X_m,Y_m), B\,.
        \end{multline*}
        Тело запроса представляет собой конъюнкцию 
        пре\-ди\-ка\-тов-кол\-лек\-ций, функциональных предикатов и 
ограничения. Здесь $C_i$~--- имена коллекций (классов), $F_i$~--- имена 
функций, $x_i$~--- имена переменных, значения которых пробегают по 
классам, $T_i$~--- типы переменных, $X_j$ и $Y_j$~--- входные и выходные 
параметры функций, $B$~--- ограничение, налагаемое на $x_i$, $X_j$, $Y_j$. 
Предикаты, находящиеся в голове правил, могут быть использованы в телах 
других правил в качестве пре\-ди\-ка\-тов-кол\-лек\-ций. 
        
        В дальнейшем будет часто использоваться запись 
        пре\-ди\-ка\-та-кол\-лек\-ции вида {\sf source([ra, de])}. Неформально это 
означает, что представляют интерес не объекты класса {\sf source} целиком, а 
лишь их атрибуты {\sf ra} и {\sf de}. Формально запись означает сокращение от 
{\sf source(\_/source.inst[ra, de])}. Здесь знак <<{\sf \_}>> обозначает анонимную 
переменную, {\sf source.inst}~--- анонимный тип экземпляров (instance) класса 
{\sf source}, а {\sf ra} и {\sf de}~--- необходимые атрибуты типа экземпляров класса.
        
        Будет также использоваться запись {\sf source([i, j, val1/val])}, означающая 
переименование атрибута {\sf val} в {\sf val1}.
        
        \medskip
        
        При отображении ЯМД будут сначала рассмотрены основные 
конструкции языка программ СИНТЕЗ, соответствующие конструкциям языка 
AQL. Затем будут рассмотрены конструкции \mbox{СИНТЕЗ}, соответствующие 
конструкциям языка AFL.
        
        Начнем рассмотрение с конструкций языка СИНТЕЗ, соответствующих 
конструкциям языка AQL, связанных с {извлечением} данных.
        
%        \smallskip
        
\paragraph*{Предикаты-классы, условия, подзапросы.} Рас\-смот\-рим 
программу, извлекающую координаты ({\sf ra}, {\sf de}) и апертурную звездную 
величину ({\sf apMag}) астрономических источников из класса  {\sf source} с 
условием на фильтр ({\sf filterId}) и апертурную звездную величину, причем 
запрос~{\sf q} использует результаты запроса~{\sf r}:
        \begin{verbatim}
q([ra,de,apMag]) :- r([ra,de,apMag]),
   filterId= #filterId.
r([ra,de,apMag]) :- source([ra,de,apMag]),
   apMag >= #apMag.
\end{verbatim}
Здесь {\sf \#filterId} и {\sf \#apMag}~--- некоторые константы 
соответствующих типов.
        
        Такая программа представляется в AQL сле\-ду\-ющим запросом:
        \begin{verbatim}
SELECT apMag FROM 
  ( SELECT apMag FROM source
    WHERE apMag >= #apMag )
WHERE filterId = #filterId;
\end{verbatim}
        
        Простые условия отображаются в AQL без изменений, рекурсивные 
запросы представляются вложенными запросами. Заметим, что координаты 
{\sf ra} и {\sf de} не указываются в секции {\sf SELECT}~--- они являются измерениями и 
извлекаются по умолчанию.
        
\paragraph*{Соединение классов.} Соединение по определенным атрибутам 
(например, {\sf objectId})
        \begin{verbatim}
q2([ra, de, filterId, uMag]) :- 
    source([ra, de, objectId, fliterId]), 
    objectSummary([objectId, uMag]).
\end{verbatim}
представляется в AQL конструкцией {\sf JOIN-ON}:
\begin{verbatim}
SELECT filterId, uMag INTO q2
FROM source
JOIN objectSummary 
ON source.objectId = objectSummary.objectId;
\end{verbatim}
где массив {\sf objectSummary} имеет вид: 
\begin{verbatim}
CREATE ARRAY objectSummary
<uMag: float NULL,  gMag: float NULL>
[ objectId=0:* ];
\end{verbatim}
        
\paragraph*{Агрегация.} Рассмотрим запрос, возвращающий объекты с 
минимальной звездной величиной {\sf uMag}:
        \begin{verbatim}
q([objectId, uMag]) :-  
  objectSummary(obj/[objectId, uMag]), 
    uMag = min(obj.uMag).
\end{verbatim}

        Запрос представляется в AQL с использованием агрегирующей функции 
того же рода:
        \begin{verbatim}
SELECT uMag
FROM source, 
 (SELECT min(uMag) AS min FROM Source)
WHERE uMag = min;
\end{verbatim}
        
\paragraph*{Группирование.} Рассмотрим запрос, возвра\-ща\-ющий среднее 
значение звездной величины {\sf uMag}, вычисленное на группе по 
идентификатору объекта {\sf filterId}:
        \begin{verbatim}
q([objectId, avgMag]) :- 
    group_by({objectId}, 
       q2(obj/[ra,de,filterId, uMag])),
    avgMag = average(obj.uMag).
\end{verbatim}

        Здесь коллекция {\sf q2}, на которой производится группирование по 
атрибуту {\sf objectId}~--- результат соединения классов {\sf source} и 
{\sf objectSummary}, рассмотренных выше.
        
        Очевидно, в AQL запрос представляется при помощи конструкции 
GROUP BY:
        \begin{verbatim}
SELECT avg(uMag) AS avgMag
FROM q2 GROUP BY objectId;
\end{verbatim}
        
        Рассмотрим конструкции языка СИНТЕЗ, соответствующие 
конструкциям языка AQL и связанные с {изменением} данных.

        
\paragraph*{Обновление.} Рассмотрим запрос, изменяющий значения в 
квадратной матрице (см.\ предыдущий пример) на значения с обратным знаком 
в том случае, если модуль значения больше~5:
        \begin{verbatim}
source(x/[i, j, val]) :- 
    source(x/[i, j, val1/val]), 
       abs(val) > 5, val = -val1.
\end{verbatim}
        
        В AQL данный запрос представляется сле\-ду\-ющим образом:
        \begin{verbatim}
UPDATE source
SET val =  -val WHERE abs(val) > 5;
\end{verbatim}


        
\paragraph*{Удаление.} Рассмотрим программу, удаляющую из базы данных 
класс и все его содержимое:
        \begin{verbatim}
-source(x) :- source(x).; 
delete_frame(source).
\end{verbatim}

        В правилах со знаком минус в голове осуществляется удаление объектов 
из коллекции. В~данном случае из коллекции удаляются все объекты. Функция 
{\sf delete\_frame} удаляет коллекцию как объект из базы данных. Операция <<{\sf ;}>> 
обозначает последовательную композицию программ. В~AQL данный запрос 
представляется при помощи операции {\sf DROP}:
\begin{verbatim}
DROP ARRAY source;
\end{verbatim}

        Рассмотрим принципы отображения конструкций языка СИНТЕЗ, 
соответствующих конструкциям AFL, на примере {расширения элементов 
мас\-си\-ва в подмассивы}. Каждый элемент массива расширя\-ется в подмассив 
определенного размера. Значения всех ячеек подмассива дублируют значение 
оригинальной ячейки. Пример программы, расширяющей каждую ячейку 
матрицы $3\times3$ в подматрицу $2\times2$:
        \begin{verbatim}
q([i,j,val]) :- {x/[i,j,val] | exists y (
  source(y/[i1/i, j1/j, val]) & 
  ( i = i1*2 & j = j1*2 | i = i1*2 +1 & 
  j = j1*2 | i= i1*2 & 
  j= j1*2 + 1 | i= i1*2 +1 & j= j1*2 +1))}.
\end{verbatim}
    Здесь выражение $\{x/T \vert F(x)\}$, где $F$~--- формула со свободной 
переменной~$x$, обозначает конструкцию выделения множества; {\sf exists} 
обозначает квантор существования. 

\columnbreak
        
        В ADM запрос представляется с использованием операции {\sf xgrid}:
        \begin{verbatim}
SELECT * FROM xgrid(source, 2, 2);
\end{verbatim}
        
        Можно заметить, что операция AFL {\sf xgrid} имеет достаточно сложно 
устроенный прообраз в канонической модели (это справедливо и для многих 
других операций). Между тем эти операции являются естественными для 
массивов. Поэтому при интеграции ресурсов, основанных на многомерных 
массивах, в канонической модели возможно использование специального 
класса {\sf array}, инкапсулирующего специфические операции, характерные для 
многомерных массивов:
        \begin{verbatim}
{ array; in: class;
  instance_type: {
  xgrid: { in: function; 
    params: {
     +dimensions/{sequence; 
      type_of_element: string;},
     +scales/{sequence; 
      type_of_element: integer;}};
  };  };
}
\end{verbatim}
        В приведенном примере рассмотрена сигнатура единственной операции 
{\sf xgrid}, параметрами которой являются последовательность имен измерений\linebreak 
{\sf dimensions} и последовательность масштабов увеличения по каждому из 
измерений {\sf scales}. Па\-ра\-мет\-ром операции по умолчанию также считается 
класс\linebreak {\sf array} как коллекция объектов. При отображении ЯОД каждый класс~--- 
образ массива (например, класс {\sf source} из подразд.~2.1) становится подклассом 
класса {\sf array}:
        \begin{verbatim}
{ source; in: class; superclass: array;
  instance_type: { ... };
}
\end{verbatim}

        Аналогично {\sf xgrid}, операциями класса {\sf array} могут быть 
представлены такие операции AFL, как {\sf substitute}, {\sf sort}, 
{\sf multiply} и~т.\,д. 
        
        Заметим, что решение о представлении операций, характерных для 
многомерных массивов, в рамках специального класса канонической модели 
допускает обобщение на объектные канонические модели, отличные от языка 
СИНТЕЗ, и модели, основанные на многомерных массивах, отличные от ADM.
        
        \smallskip
        
        Разработанные отображения ЯОД и ЯМД были частично реализованы на 
языке ATL (ATLAS\linebreak Transformation Language)~\cite{18-stu}. ATL-программы 
пред\-став\-ля\-ют собой де\-кла\-ра\-тив\-но-им\-пе\-ра\-тив\-ные трансформации, 
реализующие отображения произвольных исходных моделей уровня М1 
(согласно классификации MOF~\cite{19-stu}), конформных исходной 
метамодели уровня М2, в целевые модели уровня М1, конформные целевой 
метамодели уровня М2. Модели уровня М1 являются схемами, 
представленными в канонической модели данных или модели ADM; модели 
уровня М2 есть описание абстрактного синтаксиса канонической модели или 
модели ADM. В~качестве метамодели уровня М3, которой конформны 
метамодели уровня M2, рассматривается модель Ecore~\cite{20-stu}. Cинтаксис 
ЯОД и ЯМД ядра канонической информационной модели (языка СИНТЕЗ) и 
модели ADM был представлен в метамодели Ecore. 
        
        Было осуществлено построение ATL-транс\-фор\-ма\-ций, реализующих 
отображения подмножества ЯОД модели ADM в ЯОД канонической модели и 
подмножества ЯМД канонической модели в ЯМД модели ADM. Подмножества 
ЯМД определялись конструкциями ЯОД и ЯМД канонической модели, 
поддерживаемыми в настоящее время в исполнительной среде предметных 
посредников. Поддерживаемый язык запросов канонической модели включает 
правила, в голове которых могут быть пре\-ди\-ка\-ты-кол\-лек\-ции, а в теле~--- 
конъюнкция пре\-ди\-ка\-тов-кол\-лек\-ций, условий соединения коллекций и 
других условий на значения атрибутов типов экземпляров коллекций. 
Условием соединения может быть только равенство атрибутов. 
Поддерживаются основные арифметические предикаты и функции, а также 
термы~--- вызовы функций. 

\section{Сохранение информации и~семантики операций языка манипулирования данными 
при~отображении}
        
        Метод доказательства сохранения информации и семантики операций 
при отображении моделей данных~\cite{21-stu} основывается на представлении 
семантики моделей в формальном языке спецификаций AMN~\cite{16-stu}. 
        
        Язык AMN представляет собой тео\-ре\-ти\-ко-мо\-дель\-ную нотацию, 
основанную на теории множеств и типизированном языке первого порядка. 
Спецификации AMN называются абстрактными машинами. Язык AMN позволяет 
интегрированно рас\-смат\-ри\-вать спецификацию пространства состояний и 
поведения машины (определенного операциями на состояниях). В~AMN 
формализуется специальное отношение между спецификациями~--- 
{уточнение}. Неформально спецификация~$B$ уточняет 
спецификацию~$A$, если пользователь может использовать $B$ вместо~$A$, 
не замечая факта замены~$A$ на~$B$. 
{\looseness=1

}
        
        Идея метода заключается в следующем. Рассмотрим исходную 
модель~$S$ и целевую модель~$T$. Построим отображение~$\theta$ 
модели~$S$ в модель~$T$ (подобно изложенному в предыдущем разделе). 
Выразим семантику моделей в виде абстрактных машин AMN, построив при 
этом машины $M_S$ и $M_T$ соответственно. При этом структуры данных 
моделей~--- классы, массивы~--- представляются переменными машин, 
различные свойства структур данных представляются инвариантами машин, 
характерные операции моделей данных представляются операциями машин. 
Рассматриваемые операции исходной и целевой модели должны быть связаны 
отображением ЯМД. Отображение ЯОД представляется в виде специального 
\textit{склеивающего инварианта}~--- замкнутой формулы, связывающей 
состояния машин~$M_S$ и~$M_T$.
        
        Будем считать отображение~$\theta$ сохраняющим инфор\-ма\-цию и 
семантику операций, если машина~$M_S$, соответствующая исходной модели, 
уточняет машину~$M_T$, соответствующую целевой модели. Уточнение 
доказывается интерактивно при помощи специальных программных 
средств~\cite{22-stu}.
        
        \smallskip
        
        В качестве иллюстрации основных принципов выражения семантики 
моделей ADM и СИНТЕЗ в AMN рассмотрим частичные 
        AMN-спе\-ци\-фи\-ка\-ции, соответствующие данным моделям.
        
        Cпецификация, выражающая семантику объектной модели языка 
СИНТЕЗ, представляется в языке AMN конструкцией {\sf REFINEMENT}:
\begin{verbatim}
REFINEMENT ObjectDM
\end{verbatim}

        Переменные, составляющие пространство состояний объектной модели, 
объявлены в разделе {\sf ABSTRACT\_VARIABLES} машины {\sf ObjectDM} и 
типизируются в разделе {\sf INVARIANT}:
\begin{verbatim}
ABSTRACT_VARIABLES
    typeNames, classNames, attributeNames,
    instanceType, typeAttributes, 
      attributeType,
    unique, obligatory,
    intAttributeLowerBound, 
      intAttributeHigherBound,
    objectIDs, objectType, objectsOfClass,
    integerAttributeValue,
    adtAttributeValue
INVARIANT ...
\end{verbatim}

        Раздел {\sf INVARIANT} содержит формулу, которая состоит из предикатов, 
типизирующих переменные состояния и налагающих различные совместные 
ограничения на переменные. Предикаты соединяются операцией конъюнкции.
        
        Так, имена типов и классов представлены переменными {\sf typeNames} и 
{\sf classNames}, тип которых~--- подмножество множества строк 
({\sf STRING\_Type}):
        \begin{verbatim}
typeNames: POW(STRING_Type) &
classNames: POW(STRING_Type)
\end{verbatim}
        
        \noindent
        Здесь {\sf POW}~--- конструктор множества подмножеств.
        
        Атрибуты (переменная {\sf attributeNames}) пред\-став\-ле\-ны частичной 
функцией (знак <<\verb +-> >>), ставящей в соответствие уникальному идентификатору 
атрибута (натуральному числу из множества {\sf NAT}) имя атрибута (строку):
        \begin{verbatim}
attributeNames: NAT +-> STRING_Type
\end{verbatim}

        Типы экземпляров классов (переменная\linebreak {\sf instanceType}) представлены 
тотальной функцией (знак \verb -> ) из множества имен классов в 
множество имен типов:
        \begin{verbatim}
instanceType: classNames --> typeNames
\end{verbatim}

        Принадлежность атрибутов типам (переменная {\sf typeAttributes}) 
выражена тотальной функцией из множества имен типов в множество 
подмножеств атрибутов:
        \begin{verbatim}
typeAttributes: 
  typeNames --> POW(dom(attributeNames))
\end{verbatim}
        Здесь {\sf dom}~--- операция, возвращающая область определения 
функции, а {\sf dom(attributeNames)}~--- множество имен атрибутов.
        
        Типы значений атрибутов (переменная\linebreak {\sf attributeType}) представлены 
функцией из множества атрибутов в множество идентификаторов встроенных 
типов данных {\sf BuiltInTypes}:
        \begin{verbatim}
attributeType: 
  dom(attributeNames) +-> BuiltInTypes
\end{verbatim}

        Множества уникальных атрибутов типов {\sf unique}\linebreak и множества 
определенных атрибутов типов\linebreak {\sf obligatory} представлены тотальными 
функциями из множества имен типов в множество подмножеств атрибутов:
\begin{verbatim}
unique: 
  typeNames --> POW(dom(attributeNames))&
obligatory: 
  typeNames --> POW(dom(attributeNames))
\end{verbatim}

        Нижние границы целочисленных атрибутов (переменная 
{\sf intAttributeLowerBound}) представлены час\-тич\-ной функцией из множества 
атрибутов в множество целых чисел:
\begin{verbatim}
intAttributeLowerBound: 
  dom(attributeNames) +-> INT
\end{verbatim}

        Аналогично представляются верхние границы.
        
        Идентификаторы объектов (переменная\linebreak {\sf objectIDs}) представлены 
подмножеством натуральных чисел:
        \begin{verbatim}
objectIDs: POW(NAT)
\end{verbatim}

        Типы объектов (переменная {\sf objectType}) представлены тотальной 
функцией из множества объектных идентификаторов в множество имен типов:
\begin{verbatim}
objectType: objectIDs --> typeNames
\end{verbatim}

        Состав классов (переменная {\sf objectsOfClass}) представлен тотальной 
функцией из множества имен классов в множество подмножеств 
идентификаторов объектов:
        \begin{verbatim}
objectsOfClass: 
  classNames --> POW(objectIDs)
\end{verbatim}
        
        Значения атрибутов объектов (переменные\linebreak {\sf integerAttributeValue}, 
{\sf adtAttributeValue} и~др.)\ пред\-став\-ле\-ны функциями из множества атрибутов\linebreak 
в функции из множества идентификаторов объектов в множества значений 
атрибутов. Для простоты рассмотрены лишь функции для целочисленных 
атрибутов и атрибутов со значениями АТД\linebreak (абстрактного типа данных) (объектами):
        \begin{verbatim}
integerAttributeValue: 
 dom(attributeNames) +-> (objectIDs+->INT)& 
adtAttributeValue: 
 dom(attributeNames) +-> (objectIDs+->NAT)
\end{verbatim}
        
        Дополнительные необходимые свойства переменных состояния 
представлены конъюнктивными компонентами инварианта. Так, каждый 
атрибут является атрибутом некоторого типа:
        \begin{verbatim}
        
UNION(tt).(tt:typeNames|typeAttributes(tt))=
    dom(attributeNames)
\end{verbatim}
        Здесь {\sf UNION}~--- родовая операция объединения, в данном случае 
объединяются множества атрибутов {\sf typeAttributes(tt)} по всем именам 
типов~{\sf tt} из множества {\sf typeNames}. 
        
        Никакой атрибут не принадлежит двум типам одновременно:
        \begin{verbatim}
!(t1, t2).(t1: typeNames & t2: typeNames =>
  (typeAttributes(t1) /\ typeAttributes(t2) 
    = {}))
\end{verbatim}
   Здесь <<\verb ! >>~--- знак квантора всеобщности, <<\verb => >>~--- логическая 
импликация, <<\verb /\ >>~--- символ пересечения множеств, <<\verb {} >>~--- пустое 
множество.
        
        Уникальные и определенные атрибуты типа выбираются из множества 
атрибутов типа:
        \begin{verbatim}
!(tt).(tt: dom(unique) => unique(tt) <: 
typeAttributes(tt)) &
!(tt).(tt: dom(obligatory) => 
    obligatory(tt) <: typeAttributes(tt))
\end{verbatim}
        Здесь <<\verb <: >>~--- символ отношения мно\-жес\-во--под\-мно\-жество.
        
        Нижние и верхние границы могут быть определены только для 
целочисленных атрибутов:
        \begin{verbatim}
!(attr).(attr: dom(intAttributeLowerBound)=> 
    attributeType(attr) = Integer) 
\end{verbatim}

        Тип объекта~--- экземпляра класса есть тип экземпляров этого класса:
        \begin{verbatim}
!(cc).(cc: classNames => 
    !(oo).(oo: objectsOfClass(cc) => 
       objectType(oo) = instanceType(cc))) 
\end{verbatim}

        Для каждого атрибута определена ровно одна функция значений:
        \begin{verbatim}
dom(adtAttributeValue) /\ 
  dom(integerAttributeValue) = {} &
dom(adtAttributeValue) \/ 
  dom(integerAttributeValue) = 
    dom(attributeNames)
\end{verbatim}
   Здесь <<\verb \/ >>~--- символ объединения множеств.
        
        Для любого объекта и любого определенного атрибута типа этого 
объекта функция значений атрибута определена на объекте:
        \begin{verbatim}
!(oo, aa).(oo: dom(objectType) & 
  aa: typeAttributes(objectType(oo)) & 
  aa: obligatory(objectType(oo)) =>
      (attributeType(aa) = Integer => 
       oo: dom(integerAttributeValue(aa))) &
      (attributeType(aa) = ADT =>
       oo: dom(adtAttributeValue(aa)))) 
\end{verbatim}

        Для любого объекта и любого целочисленного атрибута типа объекта, 
определенного на объекте и для которого определена нижняя (верхняя) 
граница, значение атрибута не меньше (не больше) нижней (верхней) границы:
        \begin{verbatim}
!(oo, aa).(oo: objectIDs & 
    aa: typeAttributes(objectType(oo)) &
    oo: dom(integerAttributeValue(aa) => 
    (aa: dom(intAttributeLowerBound) =>
        (integerAttributeValue(aa)(oo) >= 
         intAttributeLowerBound(aa))) ) 
\end{verbatim}

        Объект однозначно идентифицируется набором своих уникальных 
атрибутов:
        \begin{verbatim}
!(oo1, oo2).(oo1: objectIDs & 
  oo2: objectIDs &
    objectType(oo1) = objectType(oo2) & 
    unique(objectType(oo1)) /= {} &
    !(aa).(aa: unique(objectType(oo1)) => 
      (attributeType(aa) = Integer =>
        integerAttributeValue(aa)(oo1) =
         integerAttributeValue(aa)(oo2)) &
      (attributeType(aa) = ADT =>
         adtAttributeValue(aa)(oo1) =
          adtAttributeValue(aa)(oo2)) ) => 
    oo1 = oo2 )
\end{verbatim}

        Из всего ЯМД в спецификации рассмотрена единственная операция 
{\sf update} обновления значений атрибута в объектах класса:
        \begin{verbatim}
OPERATIONS
update(cls, attr, exp, cond) =
PRE cls: classNames & 
  attr: typeAttributes(instanceType(cls)) &
  attributeType(attr) = Integer &
  exp: INT --> INT & cond: NAT --> BOOL
THEN
 integerAttributeValue := 
 integerAttributeValue <+ 
 { xx | xx: (NAT*(NAT<->INT)) &
  #(oo, val).( oo: objectsOfClass(cls) & 
  val: INT &
    xx = attr |-> ({oo |-> val}) & 
  (cond(integerAttributeValue(attr)(oo)) 
  = TRUE =>
  val=exp(integerAttributeValue(attr)(oo)))&
  (cond(integerAttributeValue(attr)(oo)) 
  = FALSE => 
  val=integerAttributeValue(attr)(oo)))}
END
\end{verbatim}

        Параметрами операции являются имя класса {\sf cls}, имя целочисленного 
атрибута {\sf attr} типа экземпляров класса, функция {\sf exp}, отвечающая за 
преобразование атрибута, и функция {\sf cond}, отвечающая условию на значение 
атрибута. Пусть {\sf o}~--- некоторый объект класса {\sf cls}, для которого определено 
значение атрибута {\sf attr}, и это значение есть~{\sf v}. Тогда операция {\sf update} 
изменяет значение атрибута на {\sf exp(v)} в случае, если выражение {\sf cond(v)} 
принимает значение <<истина>>, и оставляет значение атрибута без изменений в 
противном случае. Очевидно, такая операция {\sf update} есть обобщение примера 
обновления, рассмотренного в подразд.~2.2. Действительно, для рассмотренного 
примера {\sf cls} есть {\sf source}, {\sf attr} есть {\sf val}, 
{\sf exp(v)}\;=\;-\,{\sf v}, {\sf cond(v)}\;=\;{\sf abs(v)}.
        
        Заметим, что в рассмотренной спецификации для простоты не 
рассмотрены некоторые черты объектной модели, например отношения 
        тип--под\-тип и класс--под\-класс.
        
        \smallskip
        
        Спецификация, выражающая семантику модели ADM, представляется в 
языке AMN конструкцией
        \begin{verbatim}
REFINEMENT ArrayDM
\end{verbatim}

        Переменные, составляющие пространство состояний объектной модели, 
объявлены в разделе {\sf ABSTRACT\_VARIABLES} машины {\sf ArrayDM}:
        \begin{verbatim}
ABSTRACT_VARIABLES
    arrayNames, dimensionNames, 
    cellAttributeNames,
    arrayDimensions, arrayCellAttributes,    
    cellAtrributeType, nullable, 
    dimLowerBound, dimHigherBound,
    cells, dimensionValue, 
    integerCellAttributeValue
\end{verbatim}

        Имена массивов представлены переменной\linebreak 
{\sf arrayNames}; имена измерений~--- переменной\linebreak 
{\sf  dimensionNames}; имена атрибутов ячеек массива~--- переменной 
\mbox{{\sf cellAttributeNames}}; принадлежность измерений массивам~--- переменной 
\mbox{{\sf arrayDimensions}}; принадлежность атрибутов ячеек~--- переменной 
\mbox{{\sf arrayCellAttributes}}; 
тип атрибута ячейки~--- переменной \mbox{{\sf cellAtrributeType}}; 
атрибуты ячеек массивов, которые могут принимать неопределенные 
значения,~--- переменной \mbox{{\sf nullable}}; верхние (нижние) границы измерений~--- 
переменной \mbox{{\sf dimLowerBound}} (\mbox{{\sf dimHigherBound}}); множества 
идентификаторов ячеек массивов~--- переменной 
\mbox{{\sf cells}}, значения измерений в 
ячейках~--- переменной \mbox{{\sf dimensionValue}}; значения атрибутов ячеек~--- 
переменной \mbox{{\sf integerCellAttributeValue}}. Переменные типизируются в разделе 
\mbox{{\sf INVARIANT}} при помощи частичных и тотальных функций аналогично 
переменным, использующимся для придания семантики объектной модели:
        \begin{verbatim}
INVARIANT
   arrayNames: POW(STRING_Type) &
   dimensionNames: NAT +-> STRING_Type &
   cellAttributeNames: NAT +-> STRING_Type &
   arrayDimensions: arrayNames --> 
   POW(dom(dimensionNames)) &
   arrayCellAttributes: arrayNames --> 
     POW(dom(cellAttributeNames)) &
   cellAtrributeType: 
     dom(cellAttributeNames) --> 
       BuiltInTypes &
   nullable: 
     dom(cellAttributeNames) --> BOOL &
   dimLowerBound: 
     dom(dimensionNames) --> INT &
   dimHigherBound: 
     dom(dimensionNames) +-> INT &
   cells: arrayNames --> POW(NAT) & 
   dimensionValue: 
     NAT*dom(dimensionNames) +-> INT  &
   integerCellAttributeValue: 
     NAT*dom(cellAttributeNames) +-> INT &
\end{verbatim}
        Здесь <<\verb * >>~--- знак декартова произведения множеств.
        
        Дополнительные необходимые свойства переменных состояния 
представлены конъюнктивными компонентами инварианта. Так, любая ячейка 
любого массива однозначно идентифицируется набором значений измерений:
        \begin{verbatim}
!(arr, cell1, cell2).(arr: arrayNames & 
  cell1: cells(arr) &  cell2: cells(arr) &
  !(dim).(dim: arrayDimensions(arr) =>
    dimensionValue(cell1, dim) = 
    dimensionValue(cell2, dim)) =>
    cell1 = cell2)
        \end{verbatim}
        
                \vspace*{-6pt}
        
        Для любой ячейки любого массива определены значения всех измерений 
и значение по крайней мере одного атрибута:
        \begin{verbatim}
!(arr, cell).(arr: arrayNames & 
 cell: cells(arr) =>
  !(dim).(dim: arrayDimensions(arr) => 
   (cell |-> dim): dom(dimensionValue)) &
   #(attr).(attr: arrayCellAttributes(arr) & 
    cellAtrributeType(attr) = Integer & 
    (cell, attr): 
      dom(integerCellAttributeValue)) )
        \end{verbatim}
        
        \vspace*{-6pt}
        
        Аналогично объектной модели рассмотрена единственная операция 
ЯМД~--- операция об\-нов\-ле\-ния {\sf update}:
        \begin{verbatim}
OPERATIONS
update(cls, attr, exp, cond) =
PRE cls: arrayNames & 
 attr: arrayCellAttributes(cls) &
  cellAtrributeType(attr) = Integer &
  exp: INT --> INT & cond: NAT --> BOOL
THEN
  integerCellAttributeValue := 
  integerCellAttributeValue <+
  { yy | yy: (NAT*NAT)*INT &
    #(cell, val).(cell: cells(cls) & 
     val: INT & 
    yy = ((cell |-> attr)|-> val) &
    (cond(integerCellAttributeValue(cell, 
     attr)) = TRUE =>
      val = 
       exp(integerCellAttributeValue(cell,
        attr))) &
      (cond(integerCellAttributeValue(cell, 
       attr))= FALSE  =>
    val = 
     integerCellAttributeValue(cell,attr)))}
END   
END
        \end{verbatim}
        
                \vspace*{-6pt}
        
        Сигнатура операции совпадает с сигнатурой операции объектной 
модели. Семантика операции также аналогична: значение~{\sf v} атрибута {\sf attr} 
массива {\sf cls} заменяется на {\sf exp(v)}, если значение {\sf cond(v)} есть 
<<истина>>, и не изменяется в противном случае. 
        
        Заметим, что в данной спецификации для прос\-то\-ты не рассмотрены 
некоторые черты ADM, например нецелочисленные измерения.
        
        \smallskip
        
        Для формального доказательства того, что машина {\sf ArrayDM} уточняет 
машину {\sf ObjectDM}, необходимо построить {инвариант уточнения}, 
связы\-вающий переменные машин, и добавить его к\linebreak инварианту уточняющей 
машины. 
        
        Инвариант формализует принципы отображения ЯОД, изложенные в 
подразд.~2.1, и объединяет их в одну конъюнкцию.
        
        Так, множество имен массивов совпадает с множеством имен классов:
        \begin{verbatim}
classNames = arrayNames
\end{verbatim}

%                \vspace*{-6pt}
        
        Множество идентификаторов и имен измерений и атрибутов ячеек 
совпадает с множеством идентификаторов и имен атрибутов типов экземпляров 
классов:
        \begin{verbatim}
attributeNames = 
  dimensionNames \/ cellAttributeNames
\end{verbatim}

%                \vspace*{-6pt}

        Любому измерению любого массива соответствует атрибут типа 
экземпляра класса, соответствующего этому массиву:
        \begin{verbatim}
!(arr, dim).(arr: arrayNames & 
  dim: arrayDimensions(arr) =>
    #(attr).(attr: 
     typeAttributes(instanceType(arr)) &
          attr = dim & 
          attributeType(attr) = Integer) )s
        \end{verbatim}
        
                        \vspace*{-6pt}
        
        Любому атрибуту ячейки любого массива соответствует атрибут типа 
экземпляра класса, соответствующего этому массиву, и типы атрибутов 
совпадают:
        \begin{verbatim}
!(arr, cattr).(arr: arrayNames & 
   cattr: arrayCellAttributes(arr) =>
    #(attr).(attr: 
      typeAttributes(instanceType(arr)) & 
         attr = cattr & 
         attributeType(attr) = 
           attributeType(cattr)))
        \end{verbatim}
        
                        \vspace*{-9pt}
        
        Атрибут ячейки массива, который может принимать неопределенные 
значения, соответствует определенному ({\sf obligatory}) атрибуту типа:
        \begin{verbatim}
!(arr, cattr).(arr: arrayNames & 
  cattr /: dom(nullable) &
    cattr: arrayCellAttributes(arr) => 
    cattr: obligatory(instanceType(arr)) )
        \end{verbatim}
        
\vspace*{-9pt}

           Здесь знак <<\verb /: >> обозначает отношение непринадлежности элемента 
множеству.
        
        Измерения соответствуют уникальным атрибутам типов:
        \begin{verbatim}
!(arr, dim).(arr: arrayNames & 
    dim: arrayDimensions(arr) => 
      dim: unique(instanceType(arr)) )
        \end{verbatim}
        
                        \vspace*{-6pt}
        
        Верхние (нижние) границы измерений равны верхним (нижним) 
границам соответствующих атрибутов типов:
        \begin{verbatim}
!(dim).(dim: dom(dimLowerBound) =>
    dim: dom(intAttributeLowerBound) & 
    dimLowerBound(dim) = 
      intAttributeLowerBound(dim))
        \end{verbatim}
        
                        \vspace*{-6pt}
        
        Непустые ячейки массивов соответствуют объектам классов:
        \begin{verbatim}
cells = objectsOfClass
\end{verbatim}

%                \vspace*{-6pt}

        Для любой ячейки значения ее измерений и определенных атрибутов 
совпадают со значениями соответствующих атрибутов объекта, 
соответствующего ячейке:
        \begin{verbatim}
!(cell, dim).(cell: NAT & dim: NAT & 
  (cell |-> dim): dom(dimensionValue) =>
  cell: dom(integerAttributeValue(dim)) &
  dimensionValue(cell, dim) = 
    integerAttributeValue(dim)(cell)) &
!(cell, cattr).(cell: NAT & cattr: NAT & 
   (cell |-> cattr): 
   dom(integerCellAttributeValue) =>
   cell: dom(integerAttributeValue(cattr)) &
   integerCellAttributeValue(cell, cattr) =
     integerAttributeValue(cattr)(cell) )
        \end{verbatim}
        
                        \vspace*{-6pt}
        
        Для указания того, что машина {\sf ArrayDM} уточняет машину 
{\sf ObjectDM}, в машину {\sf ArrayDM} была добавлена директива
        \begin{verbatim}
REFINES ObjectDM
\end{verbatim}

%                \vspace*{-6pt}

        Спецификации {\sf ObjectDM} и {\sf ArrayDM} вместе с инвариантом 
уточнения были загружены в инструментальное средство 
        Atelier~B~\cite{22-stu}. Автоматически были сгенерированы теоремы, 
выражающие уточнение спецификаций. В~частности, для операции {\sf update} 
были сгенерированы 10~тео\-рем. Три из них были доказаны автоматически, 
для доказательства остальных необходимо применять интерактивные средства 
доказательства.

\vspace*{-9pt}
  
\section{Родственные исследования и~направления дальнейшей 
работы}

\vspace*{-2pt}

        Родственными данной работе следует считать исследования, связанные с 
отображением моделей, основанных на многомерных массивах, в реляционную 
модель данных. Обычно они нацелены на реализацию многомерных массивов 
при помощи реляционных СУБД. Такие работы начались одновременно с 
исследованиями моделей, основанных на многомерных массивах~\cite{5-stu}, и 
продолжаются в настоящее время~\cite{23-stu}.
        
        Основные особенности данной работы состоят в следующем. 
В~качестве исходной модели при отображении используется специфическая 
модель, основанная на многомерных массивах СУБД SciDB, язык которой 
представляет собой комбинацию декларативного SQL-по\-доб\-но\-го языка и 
функционального языка, включающего специфические\linebreak операции над 
многомерными массивами. В~качестве целевой модели используется объектная 
модель с Datalog-по\-доб\-ным языком запросов (программ)~--- язык СИНТЕЗ. 
Для отображения\linebreak обеспечивается формальное доказательство сохранения 
информации и семантики операций ЯМД.
        
        Отметим, что результаты работы могут быть с легкостью обобщены и 
использованы при интеграции в системах, использующих каноническую 
модель, отличную от языка СИНТЕЗ, например другую объектную (ODMG) 
или объект\-но-ре\-ля\-ци\-он\-ную модель (SQL:2003). Результаты также могут 
быть использованы для интеграции ресурсов, представленных в модели, 
основанной на многомерных массивах, но отличной от ADM.
        
        Некоторые вопросы отображения требуют дальнейших исследований. 
Например, следует ли иметь в канонической модели при интеграции 
        масс\-сив-ори\-ен\-ти\-ро\-ван\-ных моделей данных операции, 
связанные с размером порции (chunk size) данных в БД~\cite{9-stu}?
        
        Дальнейшую работу можно разбить на два этапа:
        \begin{enumerate}[(1)]
\item расширение инструментальных средств поддержки предметных 
посредников для виртуальной интеграции SciDB-ресурсов: 
\begin{itemize}
\item[(а)] расширение средств регистрации ресурсов в посреднике~\cite{10-stu} 
трансформацией ЯОД\ ADM в каноническую модель; 
\item[(б)] создание 
SciDB-адап\-те\-ра~--- специального ПО, связывающего исполнительную 
среду посредников с SciDB-ресурсами (составной частью адаптера является 
разработанная трансформация ЯМД);
\end{itemize}
\item применение технологии предметных посредников для решения 
научных задач в некоторой предметной области над множеством\linebreak 
неоднородных ресурсов, включающим SciDB-ре\-сурсы.
\end{enumerate}

\bigskip
        Автор выражает благодарность Л.\,А.~Калиниченко, П.\,Е.~Велихову и 
А.\,Е.~Вовченко за полезные замечания, высказанные в ходе обсуждения 
данной работы на семинарах ИПИ РАН.

\vspace*{-6pt}

{\small\frenchspacing
{%\baselineskip=10.8pt
\addcontentsline{toc}{section}{Литература}
\begin{thebibliography}{99}

\vspace*{-2pt}

\bibitem{1-stu} %1
Challenges and opportunities with big data~// A~community white paper developed 
by leading researchers across the United States, 2012. {\sf http://cra.org/ccc/docs/ init/bigdatawhitepaper.pdf}. 

\bibitem{1-2-stu} %2
\Au{Abrial J.-R.} The B-Book: Assigning programs to 
meanings.~--- Cambridge: Cambridge University Press, 1996. 

\bibitem{2-stu} %3
\Au{Vassiliadis P., Sellis T.\,K.} A~survey of logical models for OLAP databases~// SIGMOD 
Record, 1999. Vol.~28. No.\,4. P.~64--69. 

\bibitem{3-stu}
\Au{Pedersen T.\,B., Jensen C.\,S.} Multidimensional database technology~// IEEE Computer, 
2001. Vol.~34. No.\,12. P.~40--46. 

\bibitem{4-stu} %5
\Au{Libkin L., Machlin R., Wong~L.} A~query language for multidimensional arrays: Design, 
implementation, and optimization techniques.~--- SIGMOD, 1996. P.~228--239. 
\bibitem{5-stu} %6
\Au{Baumann P.} A~database array algebra for spatio-temporal data and beyond~// Next 
generation information technologies and systems. Lectures notes in computer science ser.
Springer Verlag KG, 1999. Vol.~1649. P.~76--93.
\bibitem{6-stu} %7
Overview of SciDB: Large scale array storage, processing and analysis. The SciDB development 
team.~--- SIGMOD, 2010. 
\bibitem{7-stu}
Large synoptic survey telescope. {\sf http://www.lsst.org}. 
\bibitem{8-stu}
\Au{Becla J., Lim K.-T.} Report from the First Workshop on Extremely Large Databases~// Data 
Sci.~J., 2008. Vol.~7.
\bibitem{9-stu}
SciDB User's Guide. Version~12.3, 2012. {\sf http:// www.scidb.org}.
\bibitem{10-stu}
\Au{Kalinichenko L.\,A., Briukhov D.\,O., Martynov~D.\,O., Skvortsov~N.\,A., Stupnikov~S.\,A.} 
Mediation framework for enterprise information system infrastructures~// Volume Databases and 
Information Systems Integration: 9th Conference (International) on Enterprise Information 
Systems (ICEIS 2007) Proceedings ~--- Funchal, 2007. P.~246--251.
\bibitem{11-stu}
\Au{Захаров В.\,Н., Калиниченко Л.\,А., Соколов~И.\,А., Ступников~С.\,А.} Конструирование 
канонических информационных моделей для интегрированных информационных 
сис\-тем~// Информатика и её применения, 2007. Т.~1. Вып.~2. C.~15--38. 
\bibitem{12-stu}
\Au{Kalinichenko L.\,A., Stupnikov S.\,A.} Heterogeneous information model unification as a 
prerequisite to resource schema mapping~// Information Systems: People, Organizations, 
Institutions, and Technologies: 5th Conference of the Italian Chapter of Association for 
Information Systems itAIS Proceedings.~--- Berlin--Heidelberg: Springer Physica Verlag, 2010. 
P.~373--380. 
\bibitem{13-stu}
\Au{Kalinichenko L.\,A., Stupnikov S.\,A., Martynov~D.\,O.} SYNTHESIS: A~language for 
canonical information modeling and mediator definition for problem solving in heterogeneous 
information resource environments.~--- Moscow: IPI RAN, 2007. 171~p. 
\bibitem{14-stu}
\Au{Брюхов Д.\,О., Вовченко А.\,Е., Захаров~В.\,Н., Желенкова~О.\,П., Калиниченко~Л.\,А., 
Мартынов~Д.\,О., Скворцов~Н.\,А., Ступников~С.\,А.} Архитектура промежуточного слоя 
предметных посредников для решения \mbox{задач} над множеством интегрируемых 
неоднородных распределенных информационных ресурсов в гиб\-рид\-ной 
грид-ин\-фра\-струк\-ту\-ре виртуальных обсерваторий~// Информатика и её применения, 
2008. Т.~2. Вып.~1. С.~2--34. 

\bibitem{15-stu} %16
\Au{Kersten M.\,L., Zhang~Y., Ivanova~M., Nes~N.} SciQL, a query language for science 
applications~// EDBT/ICDT~--- Workshop on Array Databases 2011 Proceedings.~--- Uppsala, 
Sweden, 2011. P.~1--12.

\bibitem{16-stu} %17
\Au{Abrial J.-R.} The B-Book: Assigning programs to meanings.~--- Cambridge: Cambridge 
University Press, 1996.
\bibitem{17-stu} %18
Astronomy in ArrayDB. 
{\sf http://trac.scidb.org/\linebreak raw-attachment/wiki/UseCases/Astronomy\%20in\%20\linebreak
ArrayDB.pdf }
\bibitem{18-stu} %19
ATL Project. {\sf http://www.eclipse.org/m2m/atl}.
\bibitem{19-stu} %20
\Au{Budinsky F., Steinberg D., Ellersick~R., Grose~T.}
Eclipse modeling framework. Ch.~5: Ecore modeling concepts.~--- Addison Wesley 
Professional, 2004.
\bibitem{20-stu} %21
Meta Object Facility (MOF) 2.0 Core Specification, 2003. 
{\sf http://www.omg.org/cgi-bin/apps/doc?ptc/\linebreak 03-10-04.pdf}. 
\bibitem{21-stu} %22
\Au{Kalinichenko L.\,A.} Method for data models integration in the common paradigm~//  1st 
East-European Symposium on Advances in Databases and Information Systems \mbox{ADBIS'97} 
Proceedings.~--- St.-Petersburg: Nevsky Dialect, 1997. Vol.~1: Regular papers. P.~275--284.
\bibitem{22-stu}
Atelier~B: The industrial tool to efficiently deploy the B Method. 
{\sf http://www.atelierb.eu/index-en.php}.

\label{end\stat}

\bibitem{23-stu} %24
\Au{Van Ballegooij A.} RAM: Array database management through relational mapping~// SIKS 
Dissertation ser. No.\,2009-25. {\sf http://oai.cwi.nl/oai/asset/14074/ 14074D.pdf}.
         
\end{thebibliography}
} }

\end{multicols} %2рис
\def\ss2{\mathop {\sum\limits^{n^\Theta}\sum\limits^{n^\Theta}}}


\def\stat{sin-one}

\def\tit{ОРТОГОНАЛЬНЫЕ СУБОПТИМАЛЬНЫЕ ФИЛЬТРЫ
ДЛЯ~НЕЛИНЕЙНЫХ СТОХАСТИЧЕСКИХ
СИСТЕМ НА~МНОГООБРАЗИЯХ$^*$}

\def\titkol{Ортогональные субоптимальные фильтры
для нелинейных стохастических
систем на многообразиях}

\def\aut{И.\,Н.~Синицын$^1$}

\def\autkol{И.\,Н.~Синицын}

\titel{\tit}{\aut}{\autkol}{\titkol}

{\renewcommand{\thefootnote}{\fnsymbol{footnote}} \footnotetext[1]
{Работа выполнена при поддержке РФФИ (проект 15-07-02244).}}


\renewcommand{\thefootnote}{\arabic{footnote}}
\footnotetext[1]{Институт проблем информатики Федерального исследовательского
центра <<Информатика и~управление>> Российской академии наук,
sinitsin@dol.ru}

\vspace*{6pt}

\Abst{Для нелинейных дифференциальных стохастических систем 
на гладких многообразиях с~винеровскими и~пуассоновскими шумами в~уравнениях состояния 
и~винеровскими шумами в~наблюдениях разработана теория синтеза ортогональных субоптимальных 
фильтров (ОСОФ)
по среднеквадратическому критерию. Получены точные фильтрационные уравнения для  
стохастических систем на многообразиях (МСтС). Обсуждаются вопросы упрощения точных 
фильтрационных уравнений. Приводятся уравнения субоптимальных фильтров (СОФ) на основе 
методов нормальной аппроксимации (МНА) и~статистической линеаризации (МСЛ). Для решения задач 
в~реальном времени использование нормальных СОФ (НСОФ) не обеспечивает 
необходимой точности, поэтому в~основу синтеза положены методы ортогональных разложений (МОР)
и~квазимоментов (МКМ) для апостериорной одномерной плотности. Получены уравнения точности 
и~чувствительности алгоритмов. В~качестве тестового примера рассмотрена одномерная 
нелинейная стохастическая система с~аддитивным и~мультипликативным белым шумом. 
Рассмотрены некоторые обобщения разработанных алгоритмов.}

\KW{апостериорное одномерное распределение; винеровский шум;
квазимомент (КМ); коэффициент ортогонального разложения (КОР);
метод квазимоментов (МКМ); метод ортогональных разложений (МОР);
нормальный фильтр; ортогональный СОФ (ОСОФ); первая функция чувствительности;
пуассоновский шум; стохастическая система на многообразиях (МСтС);
субоптимальный фильтр (СОФ)}

\DOI{10.14357/19922264160103} %


\vspace*{12pt}

\vskip 14pt plus 9pt minus 6pt

\thispagestyle{headings}

\begin{multicols}{2}

\label{st\stat}

\section{Введение}

В~[1, 2] МОР и~МКМ развиты для аналитического моделирования одно- и~многомерных
распределений в~стохастических системах  на гладких многообразиях
и~дано их применение для задач надежности и~безопасности
технических систем.

Рассмотрим развитие теории СОФ на базе МОР 
и~МКМ для МСтС. 

Раздел~2 посвящен точным  фильтрационным уравнениям для МСтС, 
а~также СОФ на основе МНА и~МСЛ. 

В~разд.~3 рассматриваются ОСОФ на основе МОР 
и~МКМ аппроксимаций нормированной  апостериорной одномерной плотности. 

В~разд.~4 обсуждаются вопросы точности и~чувствительности ОСОФ. 

В~разд.~5 
приводится тестовый пример. 

Заключение содержит выводы и~некоторые обобщения.

\section{Точные фильтрационные уравнения. Нормальный субоптимальный фильтр}

\subsection{Уравнения процессов. Вспомогательные формулы}

Пусть векторный стохастический процесс (СтП) 
$\lk X_t^{\mathrm{T}} Y_t^{\mathrm{T}} \rk^{\mathrm{T}}$
определяется системой векторных стохастических дифференциальных
уравнений Ито:
    \begin{align}
   \hspace*{-1mm}dX_t &=\varphi (X_t,Y_t,\Theta, t)\, dt + \psi' (X_t,Y_t,\Theta, t)\, dW_0 +{}\notag\\
    &\hspace*{-9.5mm}{}
+\iii_{R_0^q} \psi'' (X_t,Y_t,\Theta, t,v) P^0 \,(dt, dv)\,,\enskip 
X(t_0) = X_0\,;\label{e2.1-s1}\\
\hspace*{-1mm}dY_t &=\varphi_1 (X_t,Y_t,\Theta, t) \,dt +
    \psi_1' (X_t,Y_t,\Theta, t)\, dW_0 + {}\notag\\
&\hspace*{-9.5mm}{}+\int\limits_{R_0^q} \psi_1'' (X_t,Y_t,\Theta, t,v) P^0
    (dt,dv)\,,\enskip Y(t_0) = Y_0\,.\label{e2.2-s1}
    \end{align}
Здесь $Y_t=Y(t)$~--- $n_y$-мер\-ный наблюдаемый
СтП, $Y_t \hm\in \Delta^y$ ($\Delta^y$~--- гладкое многообразие наблюдений); 
$X_t \hm=X(t)$~--- $n_x$-мер\-ный ненаблюдаемый
СтП (вектор состояния), $X_t \hm\in \Delta^x$ ($\Delta^x$~--- 
гладкое многообразие состояний); $W_0 \hm=W_0(t)$~--- $n_w$-мер\-ный
винеровский СтП $(n_w\hm\ge n_y)$; $P^0(\Delta,A)\hm=P(\Delta,A)\hm-\mu_P (\Delta,A)$, 
$P(\Delta,A)$~--- представляет \mbox{собой} для любого множества~$A$ 
простой пуассоновский СтП, а~$\mu_P (\Delta,A)$~--- его математическое ожидание, причем

    \vspace*{2pt}

\noindent
    $$
    \mu_P (\Delta,A)=\mm P (\Delta,A)=\iii_\Delta \nu_P(\tau, A)\, d\tau\,;
    $$
    
        \vspace*{-2pt}
        
        \noindent
$\nu_P (\Delta, A)$~--- интенсивность соответствующего пуассоновского потока событий, 
$\Delta \hm=(t_1,t_2]$; интегрирование по~$v$ распространяется на все пространство~$R^q$ 
с~выколотым началом координат; $\Theta$~--- вектор случайных параметров 
размерности~$n_\Theta$;\linebreak
$\varphi\hm=\varphi(X_t,Y_t,\Theta, t)$, $\varphi_1\hm=\varphi_1(X_t,Y_t,\Theta, t)$,
 $\psi'\hm=\psi'(X_t,Y_t,\Theta, t)$ и~$\psi_1'\hm=\psi_1'(X_t,Y_t,\Theta, t)$~--- 
 известные функции, отображающие
$R^{n_x}\times R^{n_y}\times  R$ соответ\-ственно в~$R^{n_x}$,
$R^{n_y}$, $R^{n_xn_w}$ и~$R^{n_yn_w}$;
$\psi''\hm=\psi''(X_t,Y_t,\Theta, t,v)$ и~$\psi_1''\hm=\psi_1''(X_t,Y_t,\Theta,
t,v)$~--- известные функции, отображающие $R^{n_x}\times
R^{n_y}\times R^q$ в~$R^{n_x}$ и~$R^{n_y}$. Требуется найти оценку~$\hat X_t$ 
СтП~$X_t$ в~каж\-дый момент времени~$t$ по результатам
наблюдения СтП $Y(\tau)$ до момента~$t$, $Y_{t_0}^t \hm=
 \{ Y(\tau) \,: t_0 \hm\le \tau\hm< t\}$.

Предположим, что
\begin{itemize}
\item уравнение состояния имеет вид~(\ref{e2.1-s1});
\item уравнение наблюдения~(\ref{e2.2-s1}), во-пер\-вых, не содержит
пуассоновского шума $(\psi_1'' \hm\equiv 0)$,
во-вто\-рых, коэффициент при винеровском шуме~$\psi_1'$  
в~уравнениях наблюдения не зависит от состояния $(\psi_1' (X_t, Y_t,\Theta, t)\hm=\psi_1'
(Y_t,\Theta, t))$.
\end{itemize}

В этом случае уравнения задачи нелинейной фильтрации имеют следующий вид:

\noindent
\begin{align}
    dX_t &=\varphi (X_t, Y_t,\Theta, t)\,dt+\psi'(X_t, Y_t,\Theta, t)\,dW_0+{}\notag\\
&\hspace*{10mm}{}+\int\limits_{R_0^q} \psi''(X_t, Y_t,\Theta, t,v) P^0 (dt, dv)\,;
        \label{e2.3-s1}\\
    dY_t &=\varphi_1 (X_t, Y_t,\Theta, t)dt +\psi_1 (Y_t,\Theta, t)\, dW_0\,.
    \label{e2.4-s1}
    \end{align}


Будем считать, что выполнены условия существования и~единственности 
СтП  $\lk X_t^{\mathrm{T}}\ Y_t^{\mathrm{T}}\rk^{\mathrm{T}}$, 
определяемого~(\ref{e2.3-s1}) и~(\ref{e2.4-s1}) при соответствующих начальных 
условиях~[3--5].

В дальнейшем потребуется для стохастического уравнения

    \vspace*{2pt}

\noindent
    \begin{equation}
    dZ=a \,dt + b\, dW_0 + \iii_{R_0^q} c P^0 (dt, dv)
    \label{e2.5-s1}
    \end{equation}
    
    \vspace*{-2pt}
    
    \noindent
следующая обобщенная формула Ито~\cite{4-s1, 7-s1} для дифференциала  $U\hm=U(Z,t)$:

\columnbreak

\noindent
\begin{multline}
    dU =\lf U_t + U_z^{\mathrm{T}} a + \fr{1}{2}\,\mathrm{tr}\, 
    \lk U_{zz} b\nu b^{\mathrm{T}}\rk\rf dt +{}\\
{}+ \iii_{R_0^q} \lk U(Z+c,t)^{\mathrm{T}} - U(Z,t)^{\mathrm{T}} - U_z^{\mathrm{T}} c\rk 
\mu_P (dt, dv)+{}\\
\hspace*{-3mm}{}+U_Z^{\mathrm{T}} b\, dW_0 + \!\iii_{R_0^q}\! \lk U(Z+c,t) - U(Z,t) \rk P^0 (dt, dv).\!\!\!
  \label{e2.6-s1}
\end{multline}
Здесь $a$, $b$ и~$c$~--- известные функции~$Z$ и~$t$.

\subsection{Фильтрационные уравнения}

Как известно~\cite{5-s1, 7-s1, 6-s1}, для любых СтП $X_t$ и~$Y_t$ оптимальная оценка  
$\hat X^t$, минимизирующая средний квадрат ошибки в~каждый момент времени~$t$ 
представляет собой апостериорное математическое ожидание СтП  $X_t$: $\hat X_t \hm= 
\mm \lk X_t \mid Y_{t_0}^t\rk$. Чтобы найти это условное математическое ожидание, 
необходимо знать $p_t \hm= p_t (x)$~--- апостериорное одномерное распределение 
СтП~$X_t$.

В основе уравнений оптимальной (в~смыс\-ле минимума средней квадратической ошибки) 
фильт\-рации для уравнений~(\ref{e2.3-s1}) и~(\ref{e2.4-s1}) в~силу~(\ref{e2.6-s1}) 
лежит следу\-ющая формула для стохастического дифференциала апостериорного математического 
ожидания скалярной функции  $f\hm=f(X,t)$ вектора состояния:
\begin{multline}
d \hat f = \mm_{\Delta^x}^{p_t} \left[ 
f_t (X,t) + f_x (X,t)^{\mathrm{T}} \vrp (X,Y,t) + {}\right.\\
{}+\fr{1}{2}\,\mathrm{tr}\, 
\lf f_{xx} (X,t) \left(\psi' \nu {\psi'}^{\mathrm{T}}\right) (X,Y,t)\rf+{}\\
{}+ \iii_{R_0^q}  \left\{  \vphantom{f_x (X,t)^{\mathrm{T}}}
f \left(X+ \psi'' , t\right) - f(X,t) -{}\right.\\
\left.\left.{}- f_x (X,t)^{\mathrm{T}} 
\psi''(X,Y,t)\right\} \nu_P (t, dv)\mid Y_{t_0}^t \right]\, dt+{}\\
{}+\mm_{\Delta^x}^{p_t} \left\{ f(X,t) \left[ \vrp_1 (X,Y,t)^{\mathrm{T}} -
\hat \vrp_1^{\mathrm{T}}\right] +{}\right.\\
\hspace*{-2mm}\left.{}+ f_x (X,t)^{\mathrm{T}} \left(\psi\nu\psi_1^{\mathrm{T}}\right) 
(X,Y,t) \mid Y_{t_0}^t\right\} \times{}\\
    {}\times \left(\psi_1 \nu \psi_1^{\mathrm{T}}\right)^{-1} (Y,t) 
    \left(dY-\hat\vrp_1\, dt\right).
    \label{e2.7-s1}
    \end{multline}
Здесь для краткости аргумент~$\Theta$ опущен; $X\hm=X_t$, $ Y\hm=Y_t$; $\nu\hm=\nu_0$ 
и~$\nu_P $~--- интенсивности~$W_0$ и~$P^0$;
$\hat\vrp_1$~--- апостериорное математическое ожидание~$\vrp_1$ при заданной условной 
плотности  $p_t\hm=p_t (x, \Theta)$:
\begin{equation*}
\hat\vrp_1 = \mm_{\Delta^x}^{p_t} \lk\vrp_1(X,Y,t)\rk\,. %\label{e2.8-s1}
\end{equation*}

Полагая в~(\ref{e2.5-s1}) 
$$
f(X,t) \equiv g_t (\la,\Theta) =\mm_{\Delta^x}^{p_t}\exp (i\la^{\mathrm{T}} X)\,,
$$
 получим
точное нелинейное фильт\-ра\-ци\-он\-ное урав\-не\-ние для  характеристической функции 
$g_t (\la,\Theta)$:

\noindent
    \begin{multline}
    dg_t (\la,\Theta) = {\mm}_{\Delta^x}^{p_t} \left[ 
    \left\{ \vphantom{\fr{1}{2}}
     i\la^{\mathrm{T}} \varphi (X_t,Y_t,\Theta, t) -{}\right.\right.\\
\left. \!\!\!\!\!  {}- \fr{1}{2}\, 
    \la^{\mathrm{T}}\! \left(\psi\nu\psi^{\mathrm{T}}\right) (X_t,Y_t,\Theta, t) \la+
    \gamma (\la, X_t, Y_t, \Theta, t) \right\}\times{}\hspace*{-6.32312pt}
   \\
\hspace*{-7pt}\left.{}\times e^{i\la^{\mathrm{T}} X_t} \mid Y_{t_0}^t 
\vphantom{\fr{1}{2}}\right] dt+ 
{\mm}_{\Delta^x}^{p_t} %\left[ 
\left\{ 
\vphantom{e^{i\la^{\mathrm{T}} X_t} \mid Y_{t_0}^t}
\varphi_1 (X_t,Y_t,\Theta, t)^{\mathrm{T}} 
-\hat\varphi_1^{\mathrm{T}} +{}\right. %\hspace*{-1.6932pt}
\\
{}+
    i\la^{\mathrm{T}} (\psi\nu\psi_1^{\mathrm{T}}) (X_t,Y_t, \Theta,t) \times{}\\
\hspace*{-2.5mm}{}\left.{}\times e^{i\la^{\mathrm{T}} X_t} \mid Y_{t_0}^t \right\} \!
\left(\psi_1\nu\psi_1^{\mathrm{T}}\right)^{-1} \!(Y_t,\Theta,t) (dY_t-\hat\varphi_1
    \,dt) %\right]
    \!\!\! \!\label{e2.9-s1}
    \end{multline}
где

\noindent
\begin{multline}
    \hspace*{-4pt}\gamma=\gamma (\la, X_t, Y_t, \Theta, t)=
    \int\limits_{R_0^q}\big[ e^{i\la^{\mathrm{T}} \psi''(X_t,Y_t,\Theta, t,v)} -
    {}\\
 {}- 1- i\la^{\mathrm{T}} \psi''(X_t,Y_t,\Theta, t,v)\big] \nu_P (\Theta, t, dv)\,.
\label{e2.10-s1}
    \end{multline}

Функции $g_t (\la,\Theta)$ и~$p_t(x,\Theta)$ связаны между собой преобразованием Фурье.

Отсюда для гауссовской МСтС~(\ref{e2.3-s1}) и~(\ref{e2.4-s1}) $(\psi''\hm\equiv 0)$ уравнение~(\ref{e2.7-s1}) 
при  $\gamma\hm=0$ упрощается и~приобретает вид:

\noindent
\begin{multline}
dg_t (\la,\Theta) = {\mm}_{\Delta^x}^{p_t} \left[\left\{
\vphantom{\fr{1}{2}}
 i\la^{\mathrm{T}} 
\varphi (X_t,Y_t,\Theta, t) -{}\right.\right.\\
\left.\left.{}- \fr{1}{2}\, \la^{\mathrm{T}} 
\left(\psi\nu\psi^{\mathrm{T}}\right) (X_t,Y_t,\Theta, t) \la\right\} 
e^{i\la^{\mathrm{T}} X_t} \mid Y_{t_0}^t\right] dt+{}\\
{}+ {\mm}_{\Delta^x}^{p_t} %\left[ 
\left\{ 
\vphantom{\left(\psi_1\nu\psi_1^{\mathrm{T}}\right)^{-1}}
\varphi_1 (X_t,Y_t,\Theta, t)^{\mathrm{T}} -\hat\varphi_1^{\mathrm{T}} +{}%\right.
\right.\\
{}+
    i\la^{\mathrm{T}} \left(\psi\nu\psi_1^{\mathrm{T}}\right) (X_t,Y_t, \Theta,t) 
\left. e^{i\la^{\mathrm{T}} X_t} \mid Y_{t_0}^t \right\}\times{}\\
%\left.
{}\times 
\left(\psi_1\nu\psi_1^{\mathrm{T}}\right)^{-1} \!(Y_t,\Theta,t) 
\left(dY_t-\hat\varphi_1
    \,dt\right) %\right]
    \,,
    \label{e2.11-s1}
    \end{multline}

Таким образом, в~основу синтеза СОФ для МСтС~(\ref{e2.3-s1}) и~(\ref{e2.4-s1})
могут быть положены  следующие утверждения.

\vspace*{3pt}

\noindent
\textbf{Теорема~2.1.}\
\textit{Пусть для МСтС}~(\ref{e2.3-s1}) и~(\ref{e2.4-s1})
\textit{выполнены условия существования и~единственности решения, а матрица 
$\si_1 \hm=\psi_1\nu\psi_1^{\mathrm{T}}$ не вырождена. 
Тогда при условии ограниченности соответствующих условных математических 
ожиданий в}~(\ref{e2.10-s1}) \textit{точное фильтрационное уравнение для условной 
одномерной характеристической функции имеет вид}~(\ref{e2.9-s1}).

\smallskip

\noindent
\textbf{Теорема 2.2.}\ 
\textit{Пусть для гауссовской МСтС}~(\ref{e2.3-s1})  и~(\ref{e2.4-s1})
$(\psi''\hm\equiv 0)$ \textit{выполнены условия существования и~единственности решения, 
а~матрица $\si_1 \hm=\psi_1\nu\psi_1^{\mathrm{T}}$ не\linebreak вырождена. 
Тогда точное
%при условии ограниченности соответствующих  математических ожиданий 
фильт\-ра\-ционное уравнение для условной одномерной характеристической 
функции имеет вид}~(\ref{e2.11-s1}).

\vspace*{3pt}

\noindent
\textbf{Замечание~2.1.}\
Точное решение фильтрационных уравнений теорем~2.1 и~2.2  возможно
только в~случаях, когда уравнения гауссовской дифференциальной МСтС
линейны или линейны лишь относительно вектора состояния~$X_t$ при
независимой от состояния функции~$\psi$. Эти уравнения
 дают точ\-ное решение задачи оптимальной нелинейной фильтрации.  Однако это решение
  не может быть\linebreak реали\-зовано практически. Для
 нахождения оптимальной оценки вектора состояния необходимо\linebreak решить
 фильтрационное уравнение  для апостериорной характеристической функции
 (или  фильтрационное уравнение  для апостериорной плотности   вектора
 состояния~$X_t$) после получения результатов наблюдений, затем вычислить 
 оптимальную оценку вектора~$X_t$. Но методов точного решения этих
 уравнений  в~общем случае пока еще не существует.
 
 \vspace{3pt}
 
 \noindent
\textbf{Замечание~2.2.}\
 Численное решение фильтрационных уравнений в~задачах реального 
 времени (или он\-лайн-оце\-ни\-ва\-ния) тоже
 невозможно, так как для этого требуется много времени, а~решать их
 необходимо каждый раз после получения результатов наблю\-де\-ний.
 Кроме того, практическое применение точной теории оптимальной нелинейной фильтрации
 имеет смысл только в~тех случаях, когда оценки можно вычислять 
 в~реальном масштабе времени по мере получения результатов
 наблюдений. Точная теория дает оптимальные
 оценки в~каждый момент~$t$ по результатам наблюдений, полученным
 к~этому моменту, без использования последующих результатов
 наблюдений. Если эти оценки не могут быть вычислены в~тот же
 момент~$t$ или хотя бы с~фиксированным приемлемым запаздыванием
 и~их вычисление приходится откладывать на будущее, то нет
 никакого смысла отказываться от использования наблюдений,
 получаемых после момента~$t$, для оценивания состояния системы в~момент~$t$. 
 Поэтому для статистической обработки результатов
 после окончания наблюдений, т.\,е.\ для оф\-лайн-оце\-ни\-ва\-ния,
 целесообразно применять известные из математической статистики методы
 постобработки информации~\cite{5-s1}.
 
 \vspace*{-6pt}

\subsection{О~приближенных методах нелинейной фильтрации}

 Необходимость обработки результатов наблюдений в~реальном
 масштабе времени непосредственно в~процессе эксперимента
 привела  к~появлению ряда приближенных методов оптимальной нелинейной  фильтрации,
 называемых обычно методами \textit{субоптимальной фильтрации}~\cite{7-s1}. Одни
 приближенные методы основаны на  приближенном решении фильтрационных
 уравнений, а~другие~---  на превращении формул  для стохастических дифференциалов оптимальной
 оценки~$\hat X_t$ и~апостериорной ковариационной  матрицы ошибки~$R_t$ 
 в~стохастические дифференциальные уравнения  для~$\hat X_t$ и~$R_t$ путем разложения 
 функций~$\varphi$, $\varphi_1$ и~$\psi_1$ или $\varphi$,\linebreak\vspace*{-12pt}
 
 \pagebreak
 
 \noindent
  $\varphi_1$, 
 $\psi'\psi''$ и~$\psi, \psi_1$ в~степенные ряды и~отбрасывания остаточных членов.

 Для приближенного решения уравнения  для апостериорной
одномерной характеристической функции  $g_1(\la, \Theta)$ вектора~$X_t$ можно
использовать методы, основанные на параметризации одномерных
 распределений СтП, определяемого стохастическим
 дифференциальным уравнением~\cite{7-s1, 6-s1}.  Эти\linebreak
  методы  позволяют изучить
 стохастические диф\-ференциальные уравнения для параметров
 апостериорного распределения. Простейшим таким методом является
 МНА апостериорного распределения. Исключительно важное практическое значение имеют квазилинейные
фильтры, по\-лу\-ча\-емые с~по\-мощью методов эквивалентной линеаризации~[5--7].

%\vspace*{-3pt}

\subsection{Субоптимальный фильтр на~основе метода нормальной
аппроксимации}

Так как нормальное (гауссовское) распределение, 
аппроксимирующее апостериорное одномерное распределение~$X_t$, полностью 
определяется математическим ожиданием~$\hat X_t$ и~ковариационной матрицей~$R_t$ 
вектора~$X_t$, то при аппроксимации
апостериорного одномерного распределения вектора~$X_t$ нормальным
распределением все математические ожидания в~правых частях формул
для~$d\hat X_t$ и~$dR_t$ будут определенными
функциями~$\hat X_t$, $R_t$ и~$t$. Для гауссовских МСтС ($\psi'' \hm=0$
и~$\psi_1''\hm=0$), 
пользуясь формулой~(\ref{e2.7-s1}), можно показать, что фильтрационные уравнения будут
представлять собой стохастические дифференциальные уравнения,
определя\-ющие~$\hat X_t$ и~$R_t$:
\begin{equation}
\left.
\begin{array}{rl}
\hspace*{-3mm}d\hat X_t &= f \left(\hat X_t, Y_t,R_t,\Theta, t\right)dt +{}\\
&{}+  h\left(\hat X_t,Y_t, R_t,\Theta, t\right)\times{}\\[6pt]
& {}\times\left[ dY_t - f^{(1)} \left(\hat X_t,Y_t,
    R_t,\Theta, t\right)dt\right]\,; %\label{e2.12-s1}
\\[6pt]
\hspace*{-3mm}dR_t&=\left\{
\vphantom{\left({\hat X}_t, Y_t,R_t,\Theta, t\right)^{\mathrm{T}}}
 f^{(2)}\left(\hat X_t, Y_t,R_t,\Theta, t\right)-{}\right.\\[6pt]
&\hspace*{-10mm}{}-h\left(\hat
    X_t, Y_t,R_t,\Theta, t\right)\left(\psi_1\nu\psi_1^{\mathrm{T}}\right) 
    \left(Y_t,\Theta, t\right) \times{}\\[6pt]
&\left.{}\times h \left({\hat X}_t, Y_t,R_t,\Theta, t\right)^{\mathrm{\!T}}\right\} dt+{}\\[6pt]
&\displaystyle{}+\sss_{r=1}^{n_y}\! \rho_r \!\left({\hat X}_t,Y_t, R_t,\Theta, t\right)\times{}\\[6pt]
&\hspace*{-1mm}{}\times\left[
    dY_r -f_r^{(1)}\left({\hat X}_t,Y_t, R_t,\Theta, t\right) dt\right]\,.
    \end{array}
    \right\}
    \label{e2.13-s1}
    \end{equation}
Здесь введены следу\-ющие обозначения:

\noindent
    \begin{multline}
    f=f\left(\hat X_t, Y_t,R_t,t\right)={}\\[6pt]
{}=  \mm_{\Delta^x}^N \lk \varphi\left(Y_t,X_t,\Theta, t\right)\rk =\hat\vrp\,;
\label{e2.14a-s1}
\end{multline}

\columnbreak

\noindent
\begin{multline}
f^{(1)}=f^{(1)}\left(\hat X_t, Y_t,R_t,\Theta, t\right)={}\\
{}=
\lf f_r^{(1)} \left( \hat X_t, Y_t, R_t,\Theta,  t\right)\rf={}\\
{}=
\mm_{\Delta^x}^N \left[  \varphi_1\left(Y_t,X,\Theta, t\right) \right]=
\hat\vrp_1^{\mathrm{T}}\,;
\end{multline}

    \vspace*{-14pt}

\noindent
\begin{multline}
h=h\left(\hat X_t, Y_t,R_t,t\right)={}\\
{}=
    \left\{
    \vphantom{\left(\hat X_t, Y_t,R_t,\Theta, t\right)^{\mathrm{T}}}
    \mm_{\Delta^x}^N \left[
     \hat X_t\varphi_1\left(Y_t,X_t,\Theta, t\right)^{\mathrm{T}} +{}
     \right.\right.\\
     \left. {}+ \psi\nu\psi_1^{\mathrm{T}} 
     \left(Y_t,X_t,\Theta, t\right)
     \vphantom{     \hat X_t\left(Y_t, Theta\right)^T}\right]-{}\\
\left.  {} -\hat X_t f^{(1)}\left(\hat X_t, Y_t,R_t,\Theta, t\right)^{\mathrm{T}}
\right\}\times{}\\
{}\times \left(\psi_1\nu\psi_1^{\mathrm{T}}\right)^{-1} \left(Y_t,\Theta, t\right)\,;
\end{multline}

\vspace*{-14pt}

\noindent
\begin{multline}
f^{(2)}=f^{(2)}\left(\hat X_t, Y_t,R_t,\Theta, t\right)={}\\
{}=\mm_{\Delta^x}^N
    \left\{  \left(X_t-\hat X_t\right)\varphi\left(Y_t,X_t,\Theta, t\right)^{\mathrm{T}} + {}\right.\\
{}+ \varphi \left(Y_t,X_t,\Theta, t\right) 
\left(X_t^{\mathrm{T}}-\hat X_t^{\mathrm{T}}\right) +{}\\
\left.{}+\psi\nu\psi^{\mathrm{T}} \left(Y_t,X_t,\Theta, t\right)
     \vphantom{     \hat X_t\left(Y_t, Theta\right)^T}
\right\}\,;
\end{multline}

\vspace*{-14pt}

\noindent
\begin{multline}
    \rho_r=\rho_r\left(\hat X_t,Y_t, R_t,\Theta, t\right)={}\\
{}=\mm_{\Delta^x}^N
    \left\{  \left(X_t-\hat X_t\right)\left(X_t^{\mathrm{T}}-\hat X_t^{\mathrm{T}}\right) 
    \times{}\right.\\
   {}\times a_r \left(Y_t,X_t,\Theta, t\right)+
    \left(X_t-\hat X_t\right)\times{}\\
   {}\times b_r\left(Y_t,X_t,\Theta, t\right)^{\mathrm{T}} 
\left(X_t^{\mathrm{T}}-\hat X_t^{\mathrm{T}}\right)+ {}\\
\left.{}+
b_r \left(Y_t,X_t,\Theta, t\right) \left(X_t^{\mathrm{T}}-\hat
    X_t^{\mathrm{T}}\right)\right\} \\
(r=1\tr n_y)\,,
    \label{e2.14-s1}
    \end{multline}
где $a_r$~--- $r$-й элемент мат\-ри\-цы-стро\-ки $(\vrp_1^{\mathrm{T}} \hm-
\hat\vrp_1^{\mathrm{T}}) (\psi_1\nu\psi_1^{\mathrm{T}})$; $b_{kr}$~--- 
элемент $k$-й строки и~$r$-го столбца $(\psi\nu\psi_1^{\mathrm{T}})
(\psi_1\nu\psi_1^{\mathrm{T}})^{-1}$; $b_r$~--- 
$r$-й столбец матрицы $(\psi\nu\psi_1^{\mathrm{T}})(\psi_1\nu\psi_1^{\mathrm{T}})$, 
$b_r \hm= \lk b_{1r}\cdots b_{n_x r}\rk^{\mathrm{T}}$.

Число уравнений для апостериорного одномерного распределения
определяется по формуле:
    $$
    Q_{\mathrm{МНА}} = n_x + \fr{n_x (n_x+1)}{2} = \fr{n_x(n_x+3)}{2}\,.
    $$

За начальные значения $\hat X_t$ и~$R_t$  при интегрировании 
уравнений~(\ref{e2.13-s1}), естественно, следует принять
условные математическое ожидание и~ковариационную матрицу величины~$X_0$ относительно~$Y_0$:
\begin{equation}
\left.
\begin{array}{rl}
\hat X_0 &= \mm_{\Delta_x}^N\lk X_0 \mid Y_0\rk\,;\\[6pt] 
R_0 &= \mm_{\Delta_x}^N \lk \left(X_0 -\hat X_0\right) \left(X_0^{\mathrm{T}} -
\hat X_0^{\mathrm{T}}\right)\mid Y_0\rk\,.
\end{array}
\right\}
\label{e2.15-s1}
\end{equation}
 Если нет
информации об условном распределении~$X_0$ относительно~$Y_0$, то
начальные условия
 можно взять в~виде:  
 $ \hat X_0 \hm= \mathrm{M}\,X_0$ и~$R_0\hm= \mathrm{M}(X_0 
 \hm-\mathrm{M}\,X_0) (X_0^{\mathrm{T}} \hm- 
\mathrm{M}\,X_0^{\mathrm{T}})$. 
Если
же и~об этих величи-\linebreak\vspace*{-12pt}

\pagebreak

\noindent
 нах нет никакой информации, то начальные
значения~$\hat X_t$ и~$R_t$ приходится задавать произвольно.

Таким образом, имеем утверждение.

\smallskip

\noindent
\textbf{Теорема~2.3.}\
\textit{Пусть МСтС}~(\ref{e2.3-s1}) и~(\ref{e2.4-s1})~--- 
\textit{гауссовская  $(\psi''\hm=0)$, выполнены условия существования 
и~единственности решения, а~матрица $\si_1\hm=\psi_1 \nu \psi_1^{\mathrm{T}}$ 
не вырождена. Тогда алгоритм СОФ на основе МНА определяется  
уравнениями}~(\ref{e2.13-s1}) \textit{и}~(\ref{e2.15-s1}) 
\textit{при условиях ограниченности функций}~(\ref{e2.14a-s1})--(\ref{e2.14-s1}).

\smallskip

В основе соответствующей теоремы для МСтС~(\ref{e2.3-s1}) и~(\ref{e2.4-s1}) 
с~пуассоновскими шумами в~(\ref{e2.3-s1}) и~невырожденной матрицей 
$\si\hm=\psi_1 \nu \psi_1^{\mathrm{T}}$ лежат уравнения теоремы~2.3. 
При этом, если учесть формулу~(\ref{e2.7-s1}), потребуется ограниченность 
функций~$f$, $f^{(1)}$, $h$ и~$\rho_r$, определяемых~(\ref{e2.14a-s1})--(\ref{e2.14-s1}), и~функции
    \begin{equation}
    \bar f^{(2)}=f^{(2)}+ \mm_{\Delta^x}^N \lk \iii_{R_0^q} 
    \psi''{\psi''}^{\mathrm{T}} \nu_P (\Theta, t, dv)\rk.\label{e2.16-s1}
    \end{equation}

Таким образом, приходим к~утверждению.

\smallskip

\noindent
\textbf{Теорема~2.4.}\
\textit{Пусть МСтС}~(\ref{e2.3-s1}), (\ref{e2.4-s1}) 
\textit{удовлетворяет условиям существования и~единственности решения, а~матрица 
$\si\hm=\psi_1 \nu \psi_1^{\mathrm{T}}$ не вырождена. Тогда  СОФ на основе МНА 
определяется  уравнениями}~(\ref{e2.13-s1}) и~(\ref{e2.15-s1}) 
\textit{при условиях ограниченности функций~$f$, $f^{(1)}$, $\bar f^{(2)}$, $h$
и~$\rho_r$.}

\smallskip

\noindent
\textbf{Замечание~2.3.}\
Для гладких функций $\vrp$, $\vrp_1$, $\psi'$ и~$\psi_1'$ и~гауссовских 
МСтС~(\ref{e2.3-s1}) и~(\ref{e2.4-s1}) СОФ на основе МНА называется гауссовским 
фильтром~[5--7].

%\smallskip

\subsection{Квазилинейный субоптимальный фильтр на~основе метода 
статистической линеаризации} %2.5

Для МСтС~(\ref{e2.1-s1}), (\ref{e2.2-s1}) при $\psi'\hm=\psi'(\Theta,t)$, 
$\psi''\hm=\psi''(\Theta,t,v)$, $\psi_1'\hm=\psi_1'(\Theta,t)$
и~$\psi_1''\hm=\psi_1''(\Theta,t,v)$ (т.\,е.\ с~аддитивными винеровскими и~пуассоновскими 
шумами) уравнения НСОФ получаются проще, если нелинейные функции~$\vrp$ 
и~$\vrp_1$ на основе гауссовского  (нормального) распределения заменить на статистически 
линеаризованные~\cite{5-s1, 4-s1}:
    \begin{equation}
    \left.
\hspace*{-2mm}\begin{array}{rl}
    \vrp &=\vrp \left( X_t, Y_t, \Theta, t\right) \approx{}\\[6pt]
    & {}\approx\vrp_0 + k_x^\vrp 
    \left(X_t - m_t^x\right) + k_y^\vrp \left(Y_t - m_t^y\right)\,;
\\[6pt]
    \vrp_1 &= \vrp_1\left( X_t, Y_t, \Theta, t\right) \approx {}\\[6pt]
    &{}\approx\vrp_{10} + k_x^{\vrp_1} 
    \left(X_t - m_t^x\right) + k_y^{\vrp_1} \left(Y_t - m_t^y\right)\,,
    \end{array}
    \right\}
    \label{e2.17-s1}
    \end{equation}
а затем использовать уравнения линейной фильтрации~\cite{5-s1}. Входящие 
в~(\ref{e2.17-s1}) коэффициенты статистической линеаризации зависят от 
математических ожиданий, дисперсий и~ковариаций:
    $$
    Z_t =\begin{bmatrix} X_t\\ Y_t\end{bmatrix}\,; \enskip 
    m_t^z =\begin{bmatrix} m_t^x\\ m_t^y\end{bmatrix}\,;\enskip 
    K_t^z=\begin{bmatrix} K_t^x&K_t^{xy}\\ K_t^{xy}&K_t^y\end{bmatrix}\,.
    $$
Они определяются из уравнений
$$
\dot Z_t = A^z Z_t + A_0^z + B_0^z V\,,\enskip V= \dot W\,;
%\label{e2.18-s1}
$$
\begin{equation}
\left.
\begin{array}{c}
\dot m_t^z = A^z m_t^z + A_0^z \,,\enskip m_{t_0}^Z = m_0^z\,;\\[6pt] %\label{e2.19-s1}
\hspace*{-5mm}\dot K_t^z = B^z K_t^z + K_t^z \left(B^z\right)^{\mathrm{T}} + B_0^z 
\nu^m (B_0^z)^{\mathrm{T}}\,;\\[6pt]
\hspace*{43mm}K_{t_0}^z = K_0^z\,.
\end{array}
\right\}
\label{e2.20-s1}
\end{equation}
Здесь введены следующие обозначения:
    $$
    A_0^z = \begin{bmatrix} a_0\\ b_0\end{bmatrix}\,;\enskip A^z =
    \begin{bmatrix} a_1&a\\ b_1&b\end{bmatrix}\,;\enskip 
    B_0^z =\begin{bmatrix} \bar \psi\\ \bar\psi_1\end{bmatrix}\,,
    $$
    где
   \begin{alignat*}{3}
    a_0 &=\vrp_0 - k_x^\vrp m_t^x - k_y^\vrp m_t^y\,; &\enskip
 a_1 &= k_x^\vrp\,;&\enskip          a&= k_y^\vrp\,;\\
    b_0&=\vrp_0 -k_x^{\vrp_1} m_t^x -k_y^{\vrp_1}m_t^y\,;&\enskip
b_1&=k_x^{\vrp_1}\,;&\enskip        b&= k_y^{\vrp_1}\,;  %\label{e2.21-s1}
    \end{alignat*}
\begin{equation}
\left.
\begin{array}{c}
   \displaystyle \psi\, dW_0 + \iii_{R_0^q} \psi'' P^0 (dt, dv) =
    \bar \psi \, dW_0\,;\\[6pt] 
   \displaystyle    \psi_1'\, dW_0 + \iii_{R_0^q} \psi_1'' P^0 (dt, dv)= \bar \psi_1 \,dW\,.
    \end{array}
    \right\}
    \label{e2.22-s1}
    \end{equation}
 
 Тогда уравнения квазилинейного 
НСОФ будут иметь вид:
\begin{align}
\dot{\hat X}_t &= a Y_t + a_1\hat X_t + a_0 +{}\notag\\
&\hspace*{12mm}{}+ \beta_t 
\left[ Z_t - \left(bY_t + b_1 \hat X_t + b_0\right)\right]\,;\label{e2.23-s1}\\
\beta_t &= \left(R_t b_1^{\mathrm{T}} + \bar\psi \nu^W \bar\psi_1^{\mathrm{T}}\right) 
\left(\bar\psi_1\nu^W\bar\psi_1^{\mathrm{T}}\right)^{-1}\,;\label{e2.24-s1}\\
\dot R_t &= a_1 R_t + R_t a_1^{\mathrm{T}} + \bar\psi \nu^W \bar\psi^{\mathrm{T}} -
\left(R_t b_1^{\mathrm{T}} +\bar\psi \nu^W\bar\psi_1^{\mathrm{T}}\right)\times{}\notag\\
&\hspace*{8mm}\times\left(\bar\psi_1 \nu^W\bar\psi_1^{\mathrm{T}}\right)^{-1} 
\left(b_1 R_t + \bar\psi_1 \nu^W\bar\psi^{\mathrm{T}}\right)\,,
\label{e2.25-s1}
\end{align}
где $\nu^W$~--- интенсивность СтП с~независимыми приращениями, состоящего из 
винеровской и~пуассоновской частей~(\ref{e2.22-s1}).

%\smallskip

\noindent
\textbf{Теорема~2.5.}\ \textit{Пусть МСтС}~(\ref{e2.1-s1}), (\ref{e2.2-s1}) 
\textit{содержит только
аддитивные винеровские и~пуассоновские шумы и~допускает замену
статистически линеаризованной системой, а матрица $\si_1
\hm=\bar\psi_1 \nu^W \bar\psi_1^{\mathrm{T}}$ не вырождена. Тогда в~основе
алгоритма квазилинейного НСОФ лежат уравнения}~(\ref{e2.23-s1})--(\ref{e2.25-s1}) \textit{при
начальных условиях}~(\ref{e2.15-s1}).

\smallskip 

\noindent
\textbf{Замечание~2.4.}\
Уравнения теоремы~2.3 сохраняют свой вид, если коэффициенты статистической 
линеаризации в~(\ref{e2.17-s1}) вычислять для известного эквивалентного 
(негауссовского) распределения. При этом уравнения~(\ref{e2.20-s1}), 
(\ref{e2.23-s1})--(\ref{e2.25-s1}), как известно~\cite{7-s1}, 
имеют место для любого негауссовского~СтП.


\smallskip

\noindent
\textbf{Замечание~2.5.}\
Из теоремы~2.3 немедленно следуют уравнения НСОФ для фильтрации стационарных 
процессов в~установившемся режиме для стационарных МСтС, если приравнять нулю правые 
части уравнений~(\ref{e2.20-s1}), (\ref{e2.23-s1}) и~(\ref{e2.25-s1}).

\section{Ортогональные субоптимальные фильтры}

\subsection{Гауссовские шумы}

При аппроксимации апостериорной одномерной плотности отрезком ее ортогонального 
разложения~\cite{1-s1, 2-s1}:
\begin{multline}
p_t (x, \Theta)= p^* (x; \Theta, \vartheta) ={}\\
{}= w (x; \Theta) 
\lk 1+ \sss_{k=3}^N \sss_{|\nu | =k} c_\nu p_\nu (x)\rk
\label{e3.1-s1}
\end{multline}
естественно принять за параметры,
 образующие вектор~$\vartheta$, апостериорные математическое
 ожидание~$\hat X_t$, ковариационную матрицу~$R_t$ вектора~$X_t$, а~так\-же
 коэффициенты ортогонального разложения (КОР)~$c_\nu$ $(\lv \nu\rv \hm= 3\tr N)$.
 Здесь КОР определяется формулой:
\begin{equation}
c_\kappa = \lk q_\kappa \left(\fr{\partial}{i\partial \la}\right) 
g_t (\la,\Theta)\rk_{\la=0}\,.\label{e3.2-s1}
\end{equation}
Заметим, что полином~$q_\kappa$ зависит от~$\hat X_t$ и~$R_t$.

На основе~(\ref{e2.7-s1}) и~(\ref{e2.9-s1}) для  гауссовской 
МСтС~(\ref{e2.3-s1}) и~(\ref{e2.4-s1}) при  $\psi''\hm=0$ получим, что 
ОСОФ определяется следующими уравнениями:
\begin{equation}
\left.
\begin{array}{rl}
d\hat X_t&=f\,dt + h     \left(dY_t- f^{(1)} \,dt\right)\,; %\label{e3.3-s1}
\\[6pt]
dR_t&= \left(f^{(2)} -h\psi_1\nu\psi_1^{\mathrm{T}} h^{\mathrm{T}}\right) dt +{}\\[6pt]
&\hspace*{15mm}\displaystyle{}+\sss_{r=1}^{n_y} \rho_r \left(dY_r -f_r^{(1)}\, dt\right)\,.
\end{array}
\right\}
\label{e3.4-s1}
\end{equation}
Здесь введены обозначения:
    \begin{equation}
    \left.
    \begin{array}{rl}
    f&= f\left(Y_t,\vartheta,\Theta, t\right)=\mm_{\Delta^x}^{p^*} 
    \left[\varphi(Y_t,X,\Theta,t)\right]\,;
\\[6pt]
f^{(1)}&= \lf f_r^{(1)}\rf= f^{(1)}\left(Y_t,\vartheta,\Theta, t\right)={}\\[6pt]
&\hspace*{21mm}{}=
    \mm_{\Delta^x}^{p^*}  \lk\varphi_1\left(Y_t,X,\Theta,t\right)\rk\,;
    \end{array}
    \right\}
    \label{e3.5a-s1}
\end{equation}

\vspace*{-12pt}

\noindent
   \begin{multline}
f^{(2)}= f^{(2)}\left(Y_t,\vartheta,\Theta,t\right)={}\\
{}=
    \mm_{\Delta^x}^{p^*} \left[ \left(X-\hat X_t\right)\varphi
    \left(Y_t,X,\Theta,t\right)^{\mathrm{T}}+{}\right.\\
{}+\varphi\left(Y_t,X,\Theta,t\right) 
\left(X^{\mathrm{T}}-\hat X_t^{\mathrm{T}}\right)
 + {}\\
\left. {}+\left(\psi\nu\psi^{\mathrm{T}}\right) \left(Y_t,X,\Theta,t\right)
\vphantom{\hat X \left(Y-t\Theta\right)^T}
\right] \,;
\label{e3.5b-s1}
 \end{multline}
 
 \vspace*{-12pt}
 
 \noindent
  \begin{multline}
h= h\left(Y_t,\vartheta,\Theta,t\right)=\left\{
    \mm_{\Delta^x}^{p^*} \left[ X\varphi_1\left(Y_t,X,\Theta,t\right)^{\mathrm{T}}+{}\right.\right.\\
\left.{}+ \left(\psi\nu\psi_1^{\mathrm{T}}\right) 
\left(Y_t,X,\Theta,t\right)
\vphantom{\hat X \left(Y-t\Theta\right)^T}
\right]-{}\\
\left.{}-
    \hat X_t f^{(1)T}\right\} (\psi_1\nu\psi_1^{\mathrm{T}})^{-1} \left(Y_t,\Theta,t\right)\,;
    \label{e3.5c-s1}
    \end{multline}
    

\noindent
\begin{multline}
\rho_r= \rho_r\left(Y_t,\vartheta\Theta,,t\right)={}\\
{}=
   \mm_{\Delta^x}^{p^*}  \left[ \left(X-\hat X_t\right) 
   \left(X^{\mathrm{T}}-\hat X_t^{\mathrm{T}}\right) a_r
   \left(Y_t,X,\Theta,t\right)+{}\right.\\
{}+ \left(X-\hat X_t\right) b_r\left(Y_t,X,\Theta,t\right)^{\mathrm{T}}+ {}\\
\left.{}+
b_r\left(Y_t,X,\Theta,t\right)\left(X^{\mathrm{T}}-\hat X_t^{\mathrm{T}}\right)
\right]\\ (r=1\tr n_y)\,.
\label{e3.5d-s1}
\end{multline}

Далее перепишем~(\ref{e3.4-s1}) покоординатно:
\begin{equation}
\left.
\begin{array}{rl}
d\hat X_s &= f_s dt + h_s \left(dY_t- f^{(1)}\, dt\right) ={}\\[6pt]
&\hspace*{-5mm}{}= A^{\hat X_s} \,dt + 
B^{\hat X_s} \,dY_t\enskip (s=1\tr  n_x)\,; %\label{e3.6-s1}
\\[6pt]
dR_{sq} &=\left(f_{sq}^{(2)} - h_s\psi_1\nu\psi_1^{\mathrm{T}}
h_q^{\mathrm{T}} \right) dt +{}\\[6pt]
&\hspace*{-10mm}{}+\eta_{sq} \left(dY_t - f^{(1)}\, dt\right)=
A^{R_{sq}} \,dt + B^{R_{sq}} \,dY_t\,,
\end{array}
\right\}
\label{e3.7-s1}
\end{equation}
где $\hat X_s (t_0) \hm= X_{s0}$; $R_{sq} (t_0) \hm= R_{sq0}$;  $s,q\hm=1\tr n_x$; 
$\eta_{sq}$~--- мат\-ри\-ца-стро\-ка, элементами которой служат
соответствующие элементы матрицы  $\rho_1\tr \rho_{n_1}$:
\begin{equation*}
\eta_{sq} =\eta_{e_s+e_q} = \lk \rho_{1sq}\cdots \rho_{msq}\rk
    \enskip (s,q,=1\tr n_x)\,.
%    \label{e3.8-s1}
    \end{equation*}
Здесь и~далее для краткости индекс~$t$ сохраним только у~$Y_t$. 
По формуле дифференцирования Ито для винеровского СтП~\cite{3-s1, 4-s1}, 
учитывая~(\ref{e3.7-s1}), находим в~силу~(\ref{e3.2-s1}) 
стохастический дифференциал:
\begin{multline*}
dc_\kappa =\lk d\lf q_\kappa \left(\fr{\partial}{i\partial \la}\right) g_t
    (\la,\Theta)\rf\rk_{\la=0}={}\\
{}=\sss_{s=1}^{n_x} \lk\partial q_\kappa
    \left(\fr{\partial}{i\partial \la}\right) \partial \hat X_s\cdot g_t
    (\la,\Theta)\rk_{\la=0}d \hat X_s+{}\\
{}+\sss_{s,u=1}^{n_x} \lk\partial
    q_\kappa \left(\fr{\partial }{i\partial \la}\right) \partial R_{su}\cdot g_t
    (\la,\Theta)\rk_{\la=0}dR_{su} +{}\\
    {}+ \lk q_\kappa \left(\fr{\partial}{i\partial  \la}\right) d g_t (\la,\Theta)\rk_{\la=0}+{}\\
{}+ \Bigg\{ \fr{1}{2}\hspace*{-2pt} \sss_{s,u=1}^{n_x}\! \lk\!
\fr{\partial^2 q_\kappa
    \left(\partial /(i\partial \la)\right)\cdot g_t (\la,\Theta) }
    {\partial \hat X_s
    \partial \hat X_u }\rk_{\la=0}\hspace*{-5mm} h_s \psi_1\nu\psi_1^{\mathrm{T}} h_u^{\mathrm{T}} +{}\\
{}+ \fr{1}{2} \hspace*{-4pt}\sss_{s,u,k,l=1}^{n_x} \!\lk\!
\fr{\partial^2 q_\kappa \left(\partial /(i\partial \la)\right)\cdot
    g_t (\la,\Theta)}{\partial R_{su} \partial R_{kl}}
    \rk_{\la=0}\hspace*{-5.5mm} \eta_{su} \psi_1\nu\psi_1^{\mathrm{T}} \eta_{kl}^{\mathrm{T}}+{}\\
{}+\!\!\sss_{s,k,l=1}^{n_x} \!\lk\!
\fr{\partial^2 q_\kappa \left(\partial /(i\partial \la)\right)\cdot g_t (\la,\Theta)}
    {\partial \hat X_s \partial R_{kl}}\rk_{\la=0}\hspace*{-4mm}h_s \psi_1\nu\psi_1^{\mathrm{T}} 
    \eta_{kl}^{\mathrm{T}}
    \!\Bigg\}\, dt. \hspace*{-2.2177pt}
    %\label{e3.9-s1}
    \end{multline*}
Подставив сюда выражения~(\ref{e3.7-s1}) и~(\ref{e2.7-s1})
дифференциалов $d\hat X_s$, $dR_{sq}$  и~$dg_t (\la,\Theta)$ и~вспомнив,
что для любого полинома  $P(x)$ $P\lk (\partial /(i\partial
\la)) g_t(\la)\rk_{\la=0}\hm=P(\alp)$, получаем стохастические
дифференциальные уравнения:

\noindent
\begin{multline}
dc_\kappa =\left\{ 
\vphantom{\sss_{s,k,l=1}^{n_x}}
F_\kappa +\sss_{s=1}^{n_x} \fr{\partial q_\kappa
    (\alp)}{\partial \hat X_s}\,f_s +{}\right.\\
    {}+\sss_{s,u=1}^{n_x} \fr{\partial
    q_\kappa (\alp)}{\partial R_{su}} \left( 
    f_{su}^{(2)} - h_s \psi_1\nu\psi_1^{\mathrm{T}} h_u^{\mathrm{T}}\right)+{}\\
{}+\fr{1}{2}\sss_{s,u=1}^{n_x} \fr{\partial^2 q_\kappa (\alp)}
{\partial \hat X_s \partial \hat X_u}\, 
    h_s \psi_1\nu\psi_1^{\mathrm{T}} h_u^{\mathrm{T}} +{}\\
    {}+ \fr{1}{2}
    \sss_{s,u,k,l=1}^{n_x} \fr{\partial^2 q_\kappa (\alp)}
    {\partial R_{su}\partial R_{kl}}\,
    \eta_{su} \psi_1\nu\psi_1^{\mathrm{T}} \eta_{kl}^{\mathrm{T}}+{}\\
\left.{}+\sss_{s,k,l=1}^{n_x} \fr{\partial^2 q_\kappa (\alp)}
{\partial \hat X_s \partial R_{kl}}\,h_s \psi_1\nu\psi_1^{\mathrm{T}} 
\eta_{kl}^{\mathrm{T}}\right\} dt +{}\\
{}+\left\{ H_\kappa +
    \sss_{s=1}^{n_x} \fr{\partial q_\kappa (\alp)}{\partial \hat X_s}\,h_s 
+\sss_{s,u=1}^{n_x} \fr{\partial
    q_\kappa (\alp)}{\partial R_{su}}\,\eta_{su}\right\}\times{}
    \\
    {}\times
     \left(dY_t -
    f^{(1)} dt\right)= A^{c_\kappa} dt + B^{c_\kappa} dY_t\,,\enskip 
    c_\kappa (t_0) = c_{\kappa0} \\
     (\lv\kappa\rv= 3\tr N)\,.
    \label{e3.10-s1}
    \end{multline}
Здесь в~дополнение к~прежним обозначениям принято:

\vspace*{4pt}

\noindent
\begin{equation}
\left.
\begin{array}{rl}
 \hspace*{-2mm}F_\kappa &= F_\kappa \left(Y_t,\Theta,\vartheta,t\right) ={}\\[3pt]
 &{}=\displaystyle\sss_{s=1}^{n_x}
   \mm_{\Delta^x}^{p^*} \left[ \varphi_s \left(Y_t,X,\Theta,t\right)
   \fr{\partial q_\kappa (X)}{\partial X_s}\right]+{}\\[5pt]
&\hspace*{-7mm}{}+\fr{1}{2} \displaystyle\sss_{s,u=1}^{n_x}
     \mm_{\Delta^x}^{p^*}\lk \si_{su} \left(Y_t,X,\Theta,t\right)\fr{\partial^2 q_\kappa
    (X)}{\partial X_s\partial X_u}\rk;
   \\[5pt]
   \hspace*{-2mm}H_\kappa &= H_\kappa \left(Y_t,\vartheta,\Theta,t\right) ={}\\[3pt]
  & \hspace*{4mm}{}=
   \Bigg\{  \mm_{\Delta^x}^{p^*}
   \lk \varphi_1 \left(Y_t,X,\Theta,t\right)^{\mathrm{T}} q_\kappa (X)\rk+{}\\[5pt]
&\left.\hspace*{-7.5mm}{}+ \displaystyle\!\sss_{s=1}^{n_x}\!
     \mm_{\Delta^x}^{p^*}\!\lk \left(\psi\nu\psi_1^{\mathrm{T}}\right)_s 
     \left(Y_t,X,\Theta,t\right) \fr{\partial
    q_\kappa (X)}{\partial X_s} \rk-{}\right.\\[5pt]
&    \hspace*{6mm}{}- c_\kappa
    f^{(1)T} \Bigg\} \left(\psi_1\nu\psi_1^{\mathrm{T}}\right)^{-1} \left(Y_t,\Theta,t\right)\,,
    \end{array}
    \right\}
    \label{e3.11-s1}
    \end{equation}
    
    \vspace*{-2pt}

\noindent
где через $(\psi\nu\psi_1^{\mathrm{T}})_s$ обозначена  $s$-я строка матрицы
$\psi\nu\psi_1^{\mathrm{T}}$; $\si\hm=\psi\nu\psi_1^{\mathrm{T}} \hm=\lf \si_{su}\rf$.

Функции  $f_s$, $f^{(1)}$, $f^{(2)}_{su}$, $h_s$, $\eta_{su}$, $F_\kappa$ 
и~$H_\kappa$ в~уравнениях~(\ref{e3.7-s1}) и~(\ref{e3.10-s1}) представляют
собой линейные комбинации величин  $c_\nu$ $(\lv\nu\rv \hm= 3\tr N)$
с~коэффициентами, зависящими от~$\hat X_t$ и~$R_t$. Величины
$\partial q_\kappa (\alp)/\partial \hat X_s$, $\partial q_\kappa
(\alp)/\partial R_{su}$, $\partial^2 q_\kappa (\alp)/(\partial \hat
X_s \partial \hat X_u)$, $\partial^2 q_\kappa (\alp)/(\partial
R_{su} \partial R_{kl})$ и~$\partial^2 q_\kappa (\alp)/(\partial \hat
X_s \partial R_{kl})$ после замены моментов их выражениями
через~$c_\nu$ тоже будут линейными комбинациями величин~$c_\nu$ 
с~коэффициентами, зависящими от~$\hat X_t$ и~$R_t$.

В частном случае разложений~(\ref{e3.1-s1}) по
полиномам Эрмита КОР~$c_\nu$ представляют собой
квазимоменты (КМ). В~этом случае, как показано в~\cite{7-s1, 6-s1}, 
для производных полиномов Эрмита~$G_\nu$, формулы~(\ref{e3.11-s1}) приводятся к~виду:

\columnbreak

\noindent
  \begin{equation}
  \left.
  \begin{array}{rl}
    \hspace*{-3mm}F_\kappa &={}\\
&\hspace*{-7.5mm}    {}=\displaystyle\sss_{s=1}^{n_x}\! \kappa_s
    \mm_{\Delta^x}^{p^*}\lk \varphi_s \left(Y_t,X,\Theta,t\right)G_{\kappa-e_s} (X-m)\rk+{}\!\!\\[6pt]
&   \hspace*{-5mm} {}+\fr{1}{2} \displaystyle\sss_{s=1}^{n_x}  \kappa_s \left(\kappa_s-1\right)
     \mm_{\Delta^x}^{p^*}\left[ 
     \si_{ss} \left(Y_t,X,\Theta,t\right)\times {}\right.\\[6pt]
&\left.    {}\times G_{\kappa-2e_s}(X-m)
     \right]+ \displaystyle\sss_{u=2}^{n_x} \sss_{s=1}^{u-1}  \kappa_s
    \kappa_u  \times{}\\[6pt]
 &   \hspace*{-6mm}{}\times \mm_{\Delta^x}^{p^*}\lk \si_{su} \left(Y_t,X,\Theta,t\right)
    G_{\kappa-e_s-e_u}(X-m)\rk\,;
\\[6pt]
      \hspace*{-3mm} H_\kappa &= \left\{
     \mm_{\Delta^x}^{p^*}\left[
    \varphi_1 \left(Y_t,X,\Theta,t\right)^{\mathrm{T}} G_{\kappa} (X-m) \right]+{}\right.\\[6pt]
&\displaystyle{}+\sss_{s=1}^{n_x} \kappa_s
     \mm_{\Delta^x}^{p^*}\left[ \left(\psi\nu\psi_1^{\mathrm{T}}\right) 
     \left(Y_t,X,t\right)\times{}\right.\\
     &\left.\hspace*{20mm}{}\times G_{\kappa-e_s} (X-m) 
     \vphantom{\psi^T}\right]-{}\\[6pt]
&\left.\hspace*{10mm}{}-f^{(1)T} c_\kappa 
\vphantom{Y_t\left(\Theta\right)^T}\right\} \left(\psi_1\nu\psi_1^{\mathrm{T}}
\right)^{-1} 
\left(Y_t,t\right)\,,
\end{array}\!
\right\}\!\!\!
\label{e3.12-s1}
\end{equation}
где
    $$\fr{\partial q_\kappa (\alp)}{\partial \hat X_s} = -\kappa_s
    c_{\kappa-e_s}\,;
    $$
    $$
    \fr{\partial q_\kappa (\alp)}{\partial R_{ss}}= -\fr{1}{2}\,
    \fr{\partial q_\kappa (\alp) }{\partial \hat X_s^2} = -\fr{1}{2}\,
    \kappa_s \left(\kappa_s-1\right) c_{\kappa- 2 e_s}\,;
    $$
    $$
    \fr{\partial q_\kappa (\alp)}{\partial   R_{su}}= -
    \fr{\partial^2 q_\kappa (\alp)}{\partial \hat X_s \partial
    \hat X_u} = -\kappa_s \kappa_u c_{\kappa- e_s-e_u}\,;
    $$
    $$
    \fr{\partial^2 q_\kappa  (\alp)}{\partial R_{ss}^2}= \fr{1}{4}\,
    \kappa_s\left(\kappa_s -1\right)
    \left(\kappa_s -2\right) \left(\kappa_s-3\right) c_{\kappa- 4e_s}\,;
    $$
    $$
    \fr{ \partial^2 q_\kappa     (\alp)}{\partial R_{ss} \partial R_{kk}}= 
    \fr{1}{4}\,\kappa_s\left(\kappa_s -1\right)\kappa_s \left(\kappa_s -1\right) 
     c_{\kappa- 2e_s- 2  e_k}\,;
     $$
    $$
    \fr{\partial^2 q_\kappa     (\alp)}{\partial R_{ss} \partial R_{sl}}= \fr{1}{2}\,
    \kappa_s\left(\kappa_s -1\right) \left(\kappa_s -2\right) \kappa_l c_{\kappa- 3e_s- e_l}\,;
    $$
    $$
    \fr{\partial^2 q_\kappa     (\alp)}{\partial R_{ss} \partial R_{kl}}= \fr{1}{2}\,
    \kappa_s\left(\kappa_s -1\right) \kappa_k \kappa_l c_{\kappa- 2e_s-e_k- e_l}\,;
    $$
    $$
    \fr{\partial^2 q_\kappa     (\alp)}{\partial R_{su} \partial R_{sl}}= 
    \kappa_s\left(\kappa_s -1\right)
    \kappa_u \kappa_l c_{\kappa- 2e_s-e_u-  e_l}\,;
    $$
    $$
    \fr{\partial^2 q_\kappa (\alp)}{\partial R_{su} \partial R_{kl}}=
    \kappa_s\kappa_u \kappa_k\kappa_l c_{\kappa- e_s-e_u-e_k-e_l}\,;
    $$
    $$
    \fr{\partial^2 q_\kappa     (\alp)}{\partial \hat X_s \partial R_{ss}}= \fr{1}{2}\,
    \kappa_s\left(\kappa_s -1\right) \left(\kappa_s -2\right) c_{\kappa- 3e_s}\,;
    $$
    $$
    \fr{\partial^2 q_\kappa     (\alp)}{\partial \hat X_s \partial R_{sl}}= 
    \kappa_s\left(\kappa_s -1\right)  \kappa c_{\kappa- 2e_s- e_l}\,;
    $$
    $$
    \fr{\partial^2 q_\kappa     (\alp)}{\partial \hat X_s \partial R_{kk}}= \fr{1}{2}\,
    \kappa_s\kappa_k \left(\kappa_k -1\right) c_{\kappa- e_s-2     e_k}\,;
    $$
    $$
    \fr{\partial^2 q_\kappa     (\alp)}{\partial \hat X_s \partial R_{kl}}= \kappa_s
    \kappa_k     \kappa_l c_{\kappa- e_s-e_k-     e_l}\,.
%    \label{e3.13-s1}
 $$

Таким образом, имеем следующие утверждения.

\smallskip

\noindent
\textbf{Теорема~3.1.}\ \textit{Пусть МСтС}~(\ref{e2.3-s1}), (\ref{e2.4-s1})~--- 
\textit{гауссовская
$(\psi'' \hm=0)$, выполнены условия существования и~единственности
решения, а матрица $\si_1 \hm= \psi_1 \nu \psi_1^{\mathrm{T}}$ не вырождена. Тогда
в основе алгоритма ОСОФ по МОР лежат уравнения}~(\ref{e3.1-s1}), (\ref{e3.7-s1})
и~(\ref{e3.10-s1}) \textit{при условии ограниченности функций}~(\ref{e3.11-s1}).

\vspace*{3pt}

\noindent
\textbf{Теорема~3.2.}\
\textit{В~условиях теоремы~$3.1$ алгоритм ОСОФ по МКМ определяется 
уравнениями}~(\ref{e3.1-s1}),
(\ref{e3.7-s1}) \textit{и}~(\ref{e3.10-s1}) 
\textit{при условии ограниченности функций}~(\ref{e3.12-s1}).

\vspace*{-3pt}

\subsection{Негауссовские шумы}

Пользуясь формулами~(\ref{e2.7-s1}) и~(\ref{e2.9-s1}), устанавливаем, что наличие
 пуассоновского шума влияет только на функцию~$f^{(2)}$. В~результате заменим~$f^{(2)}$ 
 в~(\ref{e3.5b-s1}) на~$\bar f^{(2)}$ согласно~(\ref{e2.16-s1}) 
 и~придем к~следующим утверждениям.

\vspace*{3pt}

\noindent
\textbf{Теорема~3.3.}\
\textit{Пусть для МСтС}~(\ref{e2.3-s1}), (\ref{e2.4-s1}) 
\textit{выполнены условия существования и~единственности решения, а~мат\-ри\-ца $\si_1\hm=
\psi_1\nu\psi_1^{\mathrm{T}}$ не вырождена. Тогда алгоритм ОСОФ, согласно МОР, задается
 уравнениями}~(\ref{e3.1-s1}), (\ref{e3.7-s1}) и~(\ref{e3.10-s1}) 
 \textit{при условии ограниченности\linebreak функций~$f$, $f^{(1)}$, $\bar f^{(2)}$, 
 $h$, $\rho_r$, $F_\kappa$ и~$H_\kappa$, определя\-емых}~(\ref{e3.5a-s1})--(\ref{e3.5d-s1}), 
 (\ref{e2.16-s1})
 \textit{и}~(\ref{e3.11-s1}).

\vspace*{3pt}

\noindent
\textbf{Теорема~3.4.}\
\textit{Пусть для МСтС}~(\ref{e2.3-s1}), (\ref{e2.4-s1}) 
\textit{выполнены условия существования и~единственности решения, а~мат\-ри\-ца 
$\si_1\hm=\psi_1\nu\psi_1^{\mathrm{T}}$ не вырождена. Тогда алгоритм ОСОФ, 
согласно МКМ, задается уравнениями}~(\ref{e3.1-s1}), (\ref{e3.7-s1}) 
и~(\ref{e3.10-s1}) \textit{при условии ограниченности функций}~$f$, 
$f^{(1)}$, $\bar f^{(2)}$, $h$, $\rho_r$, $F_\kappa$ и~$H_\kappa$, 
\textit{определя\-емых}~(\ref{e3.5a-s1})--(\ref{e3.5d-s1}), (\ref{e2.16-s1}) \textit{и}~(\ref{e3.12-s1}).

\vspace*{-3pt}

\section{Точность и~чувствительность ортогонального субоптимального фильтра}

Точность СОФ на базе МНА (МСЛ) оценивается на основе ОСОФ по МОР или МКМ путем 
удержания конечного числа членов в~разложении~(\ref{e3.1-s1}).

Применяя методы теории чувствительности~\cite{9-s1, 10-s1} 
для приближенного анализа фильтрационных уравнений в~теоремах~3.1--3.4 и~учитывая 
случайность параметров~$\Theta$, придем к~следующим уравнениям для функций 
чувствительности первого порядка:

\vspace*{-4pt}

\noindent
\begin{multline*}
d\nabla^\Theta \hat X_s = \nabla^\Theta A^{\hat X_s} \,dt + 
\nabla^\Theta B^{\hat X_s}\,dY_t,\\
\nabla^\Theta B^{\hat X_s}(t_0) =0\,; % \label{e4.1-s1}
\end{multline*}

\vspace*{-14pt}

\noindent
\begin{multline*}
d\nabla^\Theta R_{sq} = \nabla^\Theta A^{R_{sq}}\, dt + \nabla^\Theta B^{R_{sq}}\,dY_t, \\
\nabla^\Theta R_{sq}(t_0) =0\,; %\label{e4.2-s1}\\
\end{multline*}

\vspace*{-3pt}

\noindent
\begin{equation*}
d\nabla^\Theta c_{\kappa} = \nabla^\Theta A^{c_\kappa}\, dt + \nabla^\Theta 
B^{c_\kappa}\,dY_t\,,\enskip \nabla^\Theta c_\kappa(t_0) =0\,.
%\label{e4.3-s1}
\end{equation*}

\columnbreak

\noindent
Здесь вычисление взятия производных ведется по всем входящим переменным, 
а~коэффициенты чувствительности вычисляются при  $\Theta\hm=m^\Theta$. При этом 
предполагается малость дисперсий по сравнению с~их математическими ожиданиями. 
Очевидно, что при дифференцировании по~$\Theta$ $(\nabla^\Theta \hm= \prt /\prt\Theta)$
порядок уравнений возрастает пропорционально числу производных. Аналогично составляются 
уравнения для элементов матриц вторых функций чувствительности.

Следуя~\cite{1-s1, 2-s1}, для оценки качества ОСОФ, определяемых теоремами~3.1--3.4 
при гауссовских~$\Theta$ с~математическим ожиданием~$m^\Theta$ и~ковариационной 
матрицей~$K^\Theta$, введем условную функцию потерь, допускающую квадратичную 
аппроксимацию:
\begin{multline*}
    \rho^{\hat X_s}=\rho^{\hat X_s}(\Theta) =\rho (m^\Theta) +\sss_{ii=1}^{n^\Theta} 
    \rho_i' \left(m^\Theta\right)\Theta_s^0+ {}\\
    {}+\ss2\limits_{i,j=1} 
    \rho_{ij}'' \left(m^\Theta\right)\Theta_i^0 \Theta_j^0\,,
    %\label{e4.4-s1}
    \end{multline*}
а также показатель~$\varepsilon$, определяемый формулой:
  \begin{equation*}
  \varepsilon =\varepsilon_2^{1/4}\,.
  %\label{e4.5-s1}
  \end{equation*}
Здесь
    $$
    \varepsilon_2 = \mm^N \lk \rho (\Theta)^2\rk -\rho (m^\Theta)^2\,, %\label{e4.6-s1}
    $$
где
\begin{multline*}
\mm^N \left[ \rho(\Theta)^2\right] = \rho \left(m^\Theta\right)^2 +
\rho' \left(m^\Theta\right)^{\mathrm{T}} K^\Theta \rho'\left(m^\Theta\right)+ {}\\
{}+
2\rho \left(m^\Theta\right) \mathrm{tr}\, \left[ 
\rho''\left(m^\Theta\right)K^\Theta\right]+{}\\
{}+\left\{ \mathrm{tr}\, \lk \rho'' \left(m^\Theta\right) K^\Theta\rk \right\}^2+2 \mathrm{tr}\, 
\left[ \rho''\left(m^\Theta\right) K^\Theta\right]^2\,,
%\label{e4.7-s1}
\end{multline*}
а функции $\rho'$ и~$\rho''$ по известным формулам~\cite{9-s1, 10-s1} 
определяются на основе первых и~вторых функций чувствительности.

Изложенные выше методы синтеза ОСОФ дают
принципиальную возможность получить фильтр, близкий к~оптимальному, по
оценке с~любой степенью точности.
Чем выше максимальный порядок~$N$~учитываемых моментов, КОР и~КМ, тем выше будет точность
приближения к~оптимальной оценке. Однако число уравнений,
определяющих параметры апостериорного одномерного распределения, быстро растет
с увеличением числа учитываемых параметров.
Соответствующие оценки можно найти в~[5--7].

\section{Тестовый пример}

Рассмотрим нелинейную с~мультипликативным шумом гауссовскую  стохастическую систему~\cite{6-s1}:

\noindent
\begin{gather*}
\dot X_t =- X_t^3 + X_t V_1(\Theta)\,;\enskip Z_t (\Theta) = 
\dot Y_t= X_t + V_2(\Theta)\,; %\label{e5.1-s1}
\\
\hat X_{t_0} = \mm X(t_0)\,;\enskip R_{t_0} = {\sf D}X\left(t_0\right)\,.
%\label{e5.2-s1}
\end{gather*}
Здесь предполагается, что  интенсивности гауссовских белых шумов~$\nu_1$ и~$\nu_2$ 
зависят от одного скалярного параметра~$\Theta$, т.\,е.\ $\nu_{1,2} \hm= 
\nu_{1,2} (\Theta)$.

Положим $\nabla =\nabla^\Theta\hm= (\prt / \prt \Theta)$. Тогда уравнения точности 
и~чувствительности СОФ на основе МНА имеют вид:
  \begin{equation}
  \left.
  \hspace*{-3mm}\begin{array}{rl}
  \dot{\hat X}_t &= - \hat X_t \left( \hat X_t^2 + 3 R_t\right) + 
  \nu_2^{-1} R_t \left(Z_t - \hat X_t\right)\,; %\label{e5.3-s1}
  \\[6pt]
  \dot R_t &= \left(\nu_1 - 6 R_t\right) \left( \hat X_t^2 + R_t\right) - 
  \nu_2^{-1} R_t^2\,;
  \end{array}
  \right\}
  \label{e5.4-s1}
  \end{equation}
    \begin{equation}
    \left.
    \begin{array}{l}
\hspace*{-1.5mm}\nabla \dot{\hat X}_t = - \lk 3\hat X_t +\left(3 + \nu_2^{-1} \right)\rk 
\nabla\hat X_t+ {}\\
\hspace*{3mm}{}+\nu_2^{-1} R_t \nabla Z_t - \nu_2^{-2} R_t \left(Z_t -\hat X_t\right)
    \nabla \nu_2\,,\\    
    \hspace*{45mm} \nabla \hat X_{t_0}=0\,;
    %\label{e5.5-s1}
    \\[6pt]
\hspace*{-1.5mm}\nabla \dot R_t = 2\left(\nu_1 - 6 R_t\right) \hat X_t \nabla \hat X_t + \big[ 
\left(\nu_1 - 6 R_t\right) - {}\\
\hspace*{12mm}{}-6 \left(\hat X_t^2 + R_t\right) - 
2 \nu_2^{-1} R_t\big] \nabla R_t +{}\\[6pt]
{}+ \left(\hat X_t^2 + R_t\right) \nabla \nu_1 + \nu_2^{-2}R_t \nabla \nu_2,\ 
\nabla R_{t0}=0.
\end{array}
\right\}
\label{e5.6-s1}
\end{equation}

Уравнения~(\ref{e5.4-s1}) нелинейны относительно~$\hat X_t$ и~$R_t$, 
причем фильтр существует только при наличии шума в~наблюдениях  $\nu_2 \hm\ne 0$ 
и~произвольном шуме интенсивности~$\nu_1$ в~уравнении состояния. 
Уравнения~(\ref{e5.6-s1}) 
для первых функций чувствительности являются линейными неоднородными уравнениями 
вследствие зависимости $\nu_{1,2} \hm= \nu_{1,2} (\Theta)$ и~$Z_t\hm = Z_t (\Theta)$.

 В соответствии с~теоремой~3.1 (с~точ\-ностью до вероятностных моментов третьего порядка) 
 имеем следующие уравнения ОСОФ:
 \begin{equation}
 \left.
 \begin{array}{rl}
 \dot{\hat X}_t &=-\hat X_t \left(\hat X_t^2 + 3 R\right) + {}\\[6pt]
 &\hspace*{17mm}{}+\nu_2^{-1} R_t 
 \left(Z_t - \hat X_t\right) - c_3\,; %\label{e5.7-s1}
 \\[6pt]
\dot R_t &= \lk\nu_1 - 6 \left(\hat X_t^2 + R_t\right)\rk R_t - 
  \nu_2^{-1} R_t^2 - {}\\[6pt]
  &\hspace*{12mm}{}-
 6 \hat X_t c_3 + \nu_2^{-1} c_3 \left(Z_t - \hat X_t\right)\,; %\label{e5.8-s1}
 \\[6pt]
  \dot c_3 &= -18 \hat X_t R_t^2 - 9 \left(\hat X_t^2 + 3 R_t\right) 
  c_3+ {}\\[6pt]
& \hspace*{10mm} {}+
 3 \nu_1 \left( 2 \hat X_t R_t + c_3\right) - \fr{3}{2}\, \nu_2^{-1} c_3\,.
 \end{array}
 \right\}
 \label{e5.9-s1}
 \end{equation}
Уравнениям~(\ref{e5.9-s1}) отвечают следующие уравнения 
для первых функций чувствительности:
\begin{multline*}
\nabla \dot{\hat X}_t=- \left(3 \hat X_t^2 + 3 R_t + \nu_2^{-1} R_t\right) 
\nabla \hat X_t +{}\\
{}+\lk \nu_2^{-1} \left(Z_t -\hat X_t\right) - 3 \hat X_t)\rk \nabla R_t +
 \nu_2^{-1} R_t\nabla Z_t - {}\\
 {}-\nu_2^{-2} R_t 
\left(Z_t -\hat X_t\right) \nabla \nu_2 - \nabla c_3\,,\enskip 
\nabla \hat X(t_0)=0\,; %\label{e5.10-s1}
\end{multline*}

  
  \noindent
  \begin{multline*}
\nabla \dot R_t =-\left(12\hat X_t R_t + 6 c_3 + \nu_2^{-1} c_3\right) 
\nabla \hat X_t -{}\\
{}- \left(\nu_1+\nu_2^{-1} R_t + 6 \hat X_t^2 + 12 R_t\right) 
\nabla R_t+\nu_2^{-1} c_3 \nabla Z_t +{}\\
{}+  \lk \nu_2^{-1} \left(Z_t -\hat X_t\right) - 6 \hat X_t\rk \nabla c_3 + 
R_t\nabla \nu_1+ {}\\
{}+\nu_2^{-2} \lk R_t^2 - c_3 \left(Z_t-X_t\right)\rk
\nabla \nu_2\,,\enskip \nabla R(t_0)=0\,;
%\label{e5.11-s1}
\end{multline*}

 \vspace*{-12pt}
  
  \noindent
  \begin{multline*}
\nabla \dot c_3= - 9 \left(c_3 + 2 \hat X_t c_3 + 2 R_t^2\right) 
\nabla \hat X_t + {}\\
{}+3 \left(2 \nu_1 \hat X_t - 9 c_3 - 2 \hat X_t R_t\right) \nabla R_t+{}\\
{}+3 \left(\nu_1 + \fr{9}{2}\,\nu_2^{-1} - 3 \hat X_t - 9 R_t - 3 \hat X_t^2\right) 
\nabla c_3 + {}\\
{}+3 \left(2 \hat X_t R_t + c_3\right) \nabla \nu_1 + \fr{3}{2}\,
\nu_2^{-2} \nabla \nu_2\,,
\enskip \nabla c_3 \left(t_0\right) =0\,. %\label{e5.13-s1}
\end{multline*}
Уравнения точности этого ОСОФ нелинейны относительно~$\hat X_t$, $R_t$ и~$c_3$ 
и~справедливы только при  $\nu_2 \hm\ne 0$ и~произвольном~$\nu_1$.
Уравнения для первых функций чувствительности являются 
линейными неоднородными уравнениями вследствие зависимости $Z_t \hm= Z_t(\Theta)$, 
$\nu_{1,2}\hm = \nu_{1,2} (\Theta)$.

\vspace*{-3pt}

\section{Заключение}

Для нелинейных дифференциальных систем с~винеровскими и~пуассоновскими шумами 
в~уравнениях состояния и~винеровскими шумами в~уравнениях наблюдения, понимаемых 
в~смысле Ито (в~том числе и~на многообразиях), разработаны методы синтеза ОСОФ 
на основе аппроксимации апостериорного одномерного 
распределения МНА и~МСЛ, 
а~также МОР и~МКМ. Получены уравнения точности 
и~чувствительности  СОФ. Приведен тестовый пример для одномерной 
нелинейной сис\-те\-мы с~параметрическим шумом.
Алгоритмиче-\linebreak ское обеспечение положено в~основу инструментального программного 
обеспечения в~библиотеке <<StS--Filter>>.

В качестве дальнейших обобщений можно рассмотреть дискретные 
и~не\-пре\-рыв\-но-дис\-крет\-ные МСтС, 
а~также модифицированные СОФ на основе ненормированной 
апостериорной плотности. Развития требуют также методы экстраполяции и~интерполяции 
в~таких системах.

\vspace*{-3pt}

{\small\frenchspacing
 {%\baselineskip=10.8pt
 \addcontentsline{toc}{section}{References}
 \begin{thebibliography}{99}

\bibitem{1-s1}
\Au{Синицын И.\,Н. }
Аналитическое моделирование распределений на основе ортогональных разложений 
в~нелинейных стохастических системах на многообразиях~// 
Информатика и~её применения, 2015. Т.~9. Вып.~3. C.~17--24.

\bibitem{2-s1}
\Au{Синицын И.\,Н.}
Применение ортогональных разложений для аналитического моделирования многомерных 
распределений в~нелинейных стохастических системах на многообразиях~// 
Системы и~средства информатики, 2015. Т.~25. №\,3. С.~3--22.

\bibitem{3-s1}
\Au{Ватанабэ С., Икэда Н.} Стохастические дифференциальные уравнения 
и~диффузионные процессы~/ Пер. с~англ.~--- М.: Наука, 1986. 448~с.
(\Au{Watanabe~S, Ikeda~N.} 
Stochastic differential equations and diffusion processes.~--- 
Amsterdam\,--\,Oxford\,--\,New York: North-Holland Publishing Co.; 
Tokyo: Kodansha Ltd., 1981. 476~p.)

\bibitem{5-s1} %4
\Au{Королюк В.\,С.,
Портенко Н.\,И., Скороход~А.\,В., Турбин~А.\,Ф.}
Справочник по теории вероятностей и~математической статистике.~---
М.: Наука, 1985. 640~с.


\bibitem{4-s1} %5
\Au{Пугачёв В.\,С., Синицын~И.\,Н.}
Теория стохастических систем.~--- М.: Логос, 2000; 2004. 1000~с.
%(\Au{Pugachev~V.\,S., Sinitsyn~I.\,N.} Stochastic systems. Theory and  applications.~---
%Singapore: World Scientific, 2001. 908~p.)


\bibitem{7-s1} %6
\Au{Синицын И.\,Н.} 
Фильтры Калмана и~Пугачёва.~--- 2-е изд.~--- М.: Логос, 2007.
776~с.


\bibitem{6-s1} %7
 \Au{Пугачёв В.\,С., Синицын~И.\,Н.}
Стохастические дифференциальные системы. Анализ и~фильтрация.~--- М.:
Наука,  1990.  632~с. (\Au{Pugachev~V.\,S., Sinitsyn~I.\,N.}
Stochastic differential systems.
Analysis and filtering.~--- Chichester\,--\,New York, NY, USA: Jonh Wiley, 1987.
549~p.)



%\bibitem{8-s1} %8
%\Au{Wonham W.\,M.}
%Some applications of stochastic differential equations to optimal nonlinear 
%filtering~// J.~Soc. Ind. Appl. Math. Ser. A Control, 1964. Vol.~2. Iss.~3. P.~347--369.

\bibitem{9-s1}
\Au{Евланов А.\.Г., Константинов В.\,М. }
Системы со случайными параметрами.~--- М.: Наука, 1987. 568~с.

\bibitem{10-s1}
Справочник по теории автоматического управления~/ Под ред. А.\,А.~Красовского.~--- 
М.: Наука, 1987. 712~с.

\end{thebibliography}

 }
 }

\end{multicols}

\vspace*{-3pt}

\hfill{\small\textit{Поступила в~редакцию 29.10.15}}

\vspace*{10pt}

%\newpage

%\vspace*{-24pt}

\hrule

\vspace*{2pt}

\hrule

\vspace*{10pt}

\def\tit{ORTHOGONAL SUPOPTIMAL FILTERS FOR~NONLINEAR STOCHASTIC SYSTEMS~ON~MANIFOLDS}

\def\titkol{Orthogonal supoptimal filters for~nonlinear stochastic systems~on manifolds}

\def\aut{I.\,N.~Sinitsyn}

\def\autkol{I.\,N.~Sinitsyn}

\titel{\tit}{\aut}{\autkol}{\titkol}

\vspace*{-9pt}

\noindent
Institute of Informatics Problems, Federal Research Center 
``Computer Science and Control'' of the Russian Academy of Sciences,
44-2 Vavilov Str., Moscow 119333, Russian Federation

\def\leftfootline{\small{\textbf{\thepage}
\hfill INFORMATIKA I EE PRIMENENIYA~--- INFORMATICS AND
APPLICATIONS\ \ \ 2016\ \ \ volume~10\ \ \ issue\ 1}
}%
 \def\rightfootline{\small{INFORMATIKA I EE PRIMENENIYA~---
INFORMATICS AND APPLICATIONS\ \ \ 2016\ \ \ volume~10\ \ \ issue\ 1
\hfill \textbf{\thepage}}}

\vspace*{3pt}

\Abste{The authors developed the
  synthesis theory of suboptimal filers (SOF) based on normal approximation 
method (NAM), statistical linearization method (SLM), orthogonal expansions method (OEM), 
and quasi-moment method (QMM) for nonlinear differential stochastic systems on manifolds
(MStS) with Wiener and Poisson noises. Exact optimal (for mean square error criteria) 
equations 
for MStS with Gaussian noises in observation equations for the one-dimensional \textit{a~posteriori} 
characteristic function are derived. Problems of approximate solving of exact 
equations are 
discussed. Accuracy and sensitivity equations are presented. A~test example for 
the nonlinear scalar 
differential equation with additive and multiplicative noises is given. Some generalizations 
are mentioned.}

\KWE{\textit{a posteriori} one-dimensional distribution;
coefficient of orthogonal expansion;
first sensitivity function;
normal approximation method; normal suboptimal filter;
orthogonal expansion method; orthogonal suboptimal filter;
quasi-moment method; quasi-moment; statistical linearization method;
stochastic system on manifolds; suboptimal filter;
Wiener white noise}



\DOI{10.14357/19922264160103}

\Ack
\noindent
The research was supported by the Russian Foundation for Basic Research 
(project 15-07-002244).



%\vspace*{6pt}

  \begin{multicols}{2}

\renewcommand{\bibname}{\protect\rmfamily References}
%\renewcommand{\bibname}{\large\protect\rm References}



{\small\frenchspacing
 {%\baselineskip=10.8pt
 \addcontentsline{toc}{section}{References}
 \begin{thebibliography}{99}


\bibitem{1-s1-1}
\Aue{Sinitsyn, I.\,N.} 2015.
Analiticheskoe modelirovanie raspredeleniy na osnove ortogonal'nykh 
razlozheniy v~neli\-ney\-nykh stokhasticheskikh sistemakh na mnogo\-ob\-ra\-zi\-yakh 
[Analytical modeling in stochastic systems on manifolds based on orthogonal expansions].
\textit{Informatika i~ee Primeneniya}~--- \textit{Inform. Appl.}  9(2):17--24.

\bibitem{2-s1-1}
\Aue{Sinitsyn, I.\,N.} 2015.
Primenenie ortogonal'nykh raz\-lo\-zhe\-niy dlya analiticheskogo modelirovaniya mnogomernykh 
raspredeleniy v~nelineynykh stokhasticheskikh sistemakh na mnogoobraziyakh [Applications 
of orthogonal expansions for analytical modeling of multidimensional distributions in 
stochastic systems on manifolds]. \textit{Sistemy i~Sredstva Informatiki}~---
\textit{Systems and Means of Informatics} 25(3):3--22.

\bibitem{3-s1-1}
\Aue{Watanabe,~S., and N. Ikeda}. 1981. 
\textit{Stochastic differential equations and diffusion processes}. 
Amsterdam\,--\,Oxford\,--\,New York: North-Holland Publishing Co.; 
Tokyo: Kodansha Ltd. 476~p.

\bibitem{5-s1-1} %4
Korolyuk, V.\,S., N.\,I.~Portenko, A.\,V.~Skorokhod, and A.\,F.~Turbin, eds. 
1985.
\textit{Spravochnik po teorii veroyatnostey}
\textit{i~matematicheskoy statistike}
[Probability theory and\linebreak mathematical statistics: Handbook].
Moscow: Nauka. 640~p.


\bibitem{4-s1-1} %5
 \Aue{Pugachev, V.\,S., and I.\,N.~Sinitsyn.} 
 2001.  \textit{Stochastic systems. Theory and  applications}.
Singapore: World Scientific. 908~p.

\bibitem{7-s1-1} %6
\Aue{Sinitsyn, I.\,N.} 2007. \textit{Fil'try Kalmana i~Pugacheva} [Kalman and Pugachev
filters]. 2nd ed. Moscow: Logos.  776~p.


\bibitem{6-s1-1} %7
 \Aue{Pugachev, V.\,S., and I.\,N.~Sinitsyn.} 
1987. \textit{Stochastic differential systems.
Analysis and filtering}. Chichester\,--\,New York, NY: Jonh Wiley.
549~p.

%\bibitem{8-s1-1}
%\Aue{Wonham, W.\,M.} 1964.
%Some application of stochastic differential equations to optimal nonlinear filtering.
%\textit{J.~Soc. Ind. Appl. Math. Ser. A Control} 2(3):347--369.

\bibitem{9-s1-1}
\Aue{Evlanov, A.\,G., and V.\,M.~Konstantinov}. 1976.
\textit{Sistemy so slozhnymi parametrami} [Systems with random parameters]. 
Moscow: Nauka. 568~p.

\bibitem{10-s1-1}
Krasovskii, A.\,A., ed. 1987.
\textit{Spravochnik po teorii avtomaticheskogo upravleniya} 
[Handbook for automatic control].   Moscow: Nauka. 712~p.
\end{thebibliography}

 }
 }

\end{multicols}

\vspace*{-3pt}

\hfill{\small\textit{Received October 29, 2015}}

\Contrl

\noindent
\textbf{Sinitsyn Igor N.} (b.\ 1940)~---
Doctor of Science in technology, professor,
Honored scientist of RF, Head of Department, Institute of Informatics Problems, Federal Research Center ``Computer Science and
Control'' of the Russian Academy of Sciences, 44-2 Vavilov Str.,
Moscow 119333, Russian Federation; sinitsin@dol.ru


\label{end\stat}


\renewcommand{\bibname}{\protect\rm Литература} %3+
\def\stat{sinits}

\def\tit{АНАЛИТИЧЕСКОЕ МОДЕЛИРОВАНИЕ
НОРМАЛЬНЫХ ПРОЦЕССОВ В~СТОХАСТИЧЕСКИХ СИСТЕМАХ СО~СЛОЖНЫМИ~НЕЛИНЕЙНОСТЯМИ}

\def\titkol{Аналитическое моделирование
нормальных процессов в~стохастических системах со~сложными нелинейностями}

\def\aut{И.\,Н.~Синицын$^1$, В.\,И.~Синицын$^2$}

\def\autkol{И.\,Н.~Синицын, В.\,И.~Синицын}

\titel{\tit}{\aut}{\autkol}{\titkol}

\renewcommand{\thefootnote}{\arabic{footnote}}
\footnotetext[1]{Институт проблем
информатики Российской академии наук, sinitsin@dol.ru}
\footnotetext[2]{Институт проблем
информатики Российской академии наук, vsinitsin@ipiran.ru}


\Abst{Рассматриваются конечномерные дифференциальные стохастические системы
(ДСтС) и эредитарные (интегродифференциальные) стохастические системы  (ЭСтС)
с винеровскими и пуассоновскими шумами, приводимые к ДСтС со сложными конечными,
дифференциальными и интегральными нелинейностями. Такие модели функционирования
описывают поведение многих современных нано- и кван\-то\-во-оп\-ти\-че\-ских
технических средств информатики. Приводятся уравнения методов нормальной
аппроксимации (МНА) и статистической линеаризации (МСЛ) для аналитического
моделирования нестационарных и стационарных нормальных (гауссовских) процессов
в нелинейных ДСтС и  нелинейных ЭСтС путем аппроксимации эредитарных ядер
линейными операторными уравнениями для дифференцируемых нелинейностей и
сингулярными ядрами для недифференцируемых нелинейностей. Рассматриваются
методы вычисления типовых интегралов МНА (МСЛ) для сложных (многомерных и
векторного аргумента) конечных и дифференциальных нелинейностей. Особое
внимание уделяется иррациональным и дробно-рациональным нелинейностям
скалярного аргумента. Приводятся примеры вычисления интегралов. Подробно
рассматриваются вопросы вычисления типовых интегралов МНА (МСЛ) для сложных
интегральных нелинейностей.}

\KW{аналитическое моделирование;
дифференциальные стохастические системы с винеровскими и пуассоновскими шумами (ДСтС);
метод нормальной аппроксимации (МНА);
метод статистической линеаризации (МСЛ);
сложные иррациональные нелинейности;
сложные конечные, дифференциальные и интегральные нелинейности;
эредитарные стохастические системы (ЭСтС), приводимые к дифференциальным}

\DOI{10.14357/19922264140302}

\vspace*{9pt}

\vskip 16pt plus 9pt minus 6pt

\thispagestyle{headings}

\begin{multicols}{2}

\label{st\stat}


\section{Введение}


Моделями функционирования многих современных технических сис\-тем информатики
служат стохастические системы (СтС), описываемые дифференциальными, интегральными
и интегродифференциальными уравнениями со сложными дроб\-но-ра\-ци\-о\-наль\-ны\-ми,
иррациональными и интегральными нелинейностями. В~[1] дано систематическое
изложение МНА и МСЛ для ДСтС и ЭСтС, приводимых к дифференциальным.

Обобщая~[2--7], рассмотрим развитие МНА и МСЛ для аналитического моделирования
нормальных стохастических процессов (СтП) на случай СтС со сложными конечными,
дифференциальными и интегральными нелинейностями.

Как показано в~\cite{4-sin}, альтернативным подходом к аналитическому моделированию
СтП в ДСтС и ЭСтС служит подход, основанный на дискретизации стохастических
дифференциальных уравнений на основе использования обобщенной формы Ито и
кратных стохастических интегралов от винеровских и пуассоновских СтП с
последующим применением дискретных версий МНА (МСЛ).

Статья состоит из введения, пяти разделов и заключения.

В~разд.~2 и~3
приводятся уравнения МНА и МСЛ для аналитического моделирования одно- и
двумерных распределений стационарных и нестационарных СтП в ДСтС и ЭСтС,
приводимых к ДСтС.

Типовые интегралы МНА и МСЛ рассматриваются в разд.~4.

Особенности аналитического моделирования в ДСтС со сложными конечными и
дифференциальными нелинейностями обсуждаются в разд.~5.

Раздел~6
посвящен аналитическому моделированию СтП в ДСтС со сложными интегральными
нелинейностями.

Приводятся примеры.


\section{Уравнения методов нормальной~аппроксимации и~статистической
линеаризации для~дифференциальных стохастических систем}

Как известно~\cite{2-sin, 3-sin},  уравнения конечномерных непрерывных нелинейных сис\-тем
со стохастическими возмущениями путем расширения вектора состояния ДСтС
могут быть записаны в виде следующего векторного стохастического
дифференциального уравнения Ито:
    \begin{multline}
    dY_t = a(Y_t, t)\, dt + b (Y_t, t) \,dW_0+{}\\
    {}+ \iii_{R_0} c (Y_t, t, v) P^0
    (dt, dv)\,,\enskip Y(t_0) = Y_0\,.\label{e2.1-sin}
    \end{multline}
Здесь $a=a(Y_t, t)$ и $b\hm=b(y_t, t)$~--- известные
$(p\times 1)$-мер\-ная и  $(p\times m)$-мер\-ная функции~$Y_t$ и~$t$;
$W_0\hm= W_0(t)$~--- $r$-мер\-ный винеровский СтП интенсивности
$\nu_0 \hm= \nu_0(t)$; $c(Y_t, t, v)$~--- $(p\times 1)$-мер\-ная функция  $Y_t, t$
и вспомогательного $(q\times 1)$-мер\-но\-го па\-ра\-мет\-ра~$v$;
$\iii_{\Delta} dP^0 (t, A)$~--- центрированная пуассоновская мера,
определяемая
\begin{equation*}
\iii_{\Delta} dP^0 (t, A) = \iii_{\Delta} dP (t,A) =
\iii_{\Delta} \nu_P (t,A)\, dt\,. %\label{e2.2-sin}
\end{equation*}
В~(\ref{e2.1-sin}) принято: $\iii_{\Delta}$~-- число скачков пуассоновского
СтП в интервале времени  $\Delta \hm= (t_1, t_2]$; $\nu_P (t, A)$~---
интенсивность пуассоновского СтП  $P(t,A)$; $A$~--- некоторое борелевское
множество пространства  $R_0^q$ с выколотым началом.
Начальное значение~$Y_0$ представляет собой случайную величину, не зависящую
от приращений СтП  $W_0(t)$ и $P(t,A)$ на интервалах времени, следующих
за~$t_0$, $t_0 \hm\le t_1\hm\le t_2$ для любого множества~$A$.

В случае аддитивных нормальных (гауссовских) и обобщенных
пуассоновских возмущений уравнение~(\ref{e2.1-sin}) имеет вид:
\begin{equation}
\dot Y_t = a(Y_t,t)+ b_0 (t) V\,, \enskip
V = \dot W\,,\enskip Y(t_0) = Y_0\,.\label{e2.3-sin}
\end{equation}
Здесь $W$~--- СтП с независимыми приращениями, представляющий собой
смесь нормального и обобщенного пуассоновского СтП.

Если предположить существование конечных вероятностных
моментов второго порядка для моментов времени~$t_1$ и~$t_2$, то уравнения
МНА примут следующий вид~\cite{2-sin, 3-sin}:
\begin{itemize}
\item  для характеристических функций
    \begin{equation}
    g_1^N (\la;t) =\exp \lk i\la^{\mathrm{T}} m_t - \fr{1}{2}\, \la^{\mathrm{T}} K_t \la\rk\,;\label{e2.4-sin}
    \end{equation}
\begin{equation}
\hspace*{-7.5mm}g_{t_1, t_2}^N (\la_1, \la_2;t_1, t_2 ) =\exp \lk i\bar \la^{\mathrm{T}} \bar m_2 -
\fr{1}{2}\, \bar \la^{\mathrm{T}} \bar K_2 \la\rk\,,\!\!\label{e2.5-sin}
\end{equation}
где
    \begin{gather*}
    \bar \la =\lk \la_1^{\mathrm{T}}\la_2^{\mathrm{T}}\rk^{\mathrm{T}}\,; \quad
        \bar m_2 = \lk m_{t_1}^{\mathrm{T}} m_{t_2}^{\mathrm{T}}\rk^{\mathrm{T}}\,;\\
        \bar K_2= \begin{bmatrix}
    K(t_1, t_1)& K(t_1, t_2)\\
    K(t_2, t_1)& K(t_2, t_2)
    \end{bmatrix}\,;
    \end{gather*}

\item для математических ожиданий  $m_t$, ковариационной матрицы~$K_t$ и
матрицы ковариационных функций $K(t_1, t_2)$:
    \begin{equation}
    \dot m_t = a_1 (m_t, K_t, t)\,,\enskip m_0 = m(t_0)\,;\label{e2.6-sin}
    \end{equation}
\begin{equation}
\dot K_t = a_2 (m_t, K_t, t)\,,\enskip K_0 = K(t_0)\,;\label{e2.7-sin}
\end{equation}

\vspace*{-12pt}

\noindent
\begin{multline}
\fr{\prt K(t_1, t_2)}{\prt t_2 }= K(t_1, t_2) a_{21} (m_{t_2}, K_{t_2}, t_2)^{\mathrm{T}}\,;\\
K(t_1, t_1) = K_{t_1}\,.
\label{e2.8-sin}
\end{multline}
    \end{itemize}
Здесь приняты следующие обозначения:
\begin{equation}
a_1 = a_1 (m_t, K_t, t) = M_N a (Y_t, t)\,;\label{e2.9-sin}
\end{equation}

\vspace*{-12pt}

\noindent
\begin{multline}
a_2 = a_2 (m_t, K_t, t) = a_{21} (m_t, K_t, t)+{}\\
{}+ a_{21} (m_t, K_t, t)^{\mathrm{T}} +
a_{22}(m_t, K_t, t)\,;\label{e2.10-sin}
\end{multline}

\vspace*{-12pt}

\noindent

\begin{equation}
a_{21} = a_{21}(m_t, K_t, t)=  M_N a(Y_t, t) Y_{t}^{0\mathrm{T}}\,;\label{e2.11-sin}
\end{equation}
\begin{equation*}
a_{22} = a_{22}(m_t, K_t, t)= M_N \sigma (Y_t, t)\,;
%\label{e2.12-sin}
\end{equation*}

\vspace*{-12pt}

\noindent
\begin{multline*}
\sigma (Y_t, t) = b(Y_t, t) \nu_0(t) b(Y_t, t)^{\mathrm{T}} +{}\\
{}+
\iii_{R_0^q} c (Y_t, t, v) c(Y_t, t,v)^{\mathrm{T}}
\nu_P (t, dv)\,; %\label{e2.13-sin}
\end{multline*}

\vspace*{-12pt}

\begin{gather*}
m_t = MY_t\,,\quad Y_t^0 = Y_t - m_t\,,\\
K_t = M_N Y_0^0 Y_t^{0\mathrm{T}}\,,\quad K(t_1, t_2) =
M_N Y_{t_1}^0 Y_{t_2}^0\,; %\label{e2.14-sin}
\end{gather*}
$M_N$~--- символ вычисления математического ожидания для нормальных
распределений~(\ref{e2.4-sin}) и~(\ref{e2.5-sin}).

Для стационарных ДСтС нормальные стационарные СтП~--- если они существуют,
то  $m_t \hm=\bar m$, $ K_t \hm=\bar K$, $K(t_1, t_2) \hm= k(\tau)$
$(\tau \hm= t_1\hm-t_2)$,~--- определяются уравнениями~\cite{2-sin, 3-sin}:
   \begin{equation}
   a_1 (\bar m, \bar K) =0\,;\enskip a_2 (\bar m, \bar K)=0\,;\label{e2.15-sin}
   \end{equation}
   \begin{equation}
   \left.
   \hspace*{-2.8mm}\begin{array}{l}
  \dot k_\tau (\tau) = a_{21} (\bar m, \bar K)\bar K^{-1} k(\tau)\,;\\[9pt]
  k(0) =\bar K \enskip (\forall \tau >0)\,, \
  k(\tau) = k(-\tau)^{\mathrm{T}} \enskip
  (\forall\tau <0)\,.
  \end{array}\!\!
  \right\}\!\!
  \label{e2.16-sin}
  \end{equation}
При этом необходимо, чтобы матрица  $a_{21} (\bar m, \bar K)\hm=\bar a_{21}$
была бы асимптотически устойчивой.

Для ДСтС~(\ref{e2.3-sin}) уравнения МНА переходят в уравнения МСЛ
Казакова~\cite{2-sin, 3-sin}, если принять
\begin{equation}
a(Y_t,t) = a_1 (m_t, K_t) + k_1^a (m_t, K_t) Y_t^0\,;\label{e2.17-sin}
\end{equation}
\begin{equation}\left.
\begin{array}{rl}
b(Y_t,t) &= b_0 (t)\,;\\[9pt]
    \si(Y_t, t)&= b_0(t) \nu(t) b_0(t)^{\mathrm{T}} = \si_0(t)\,,
    \end{array}
    \right\}\label{e2.18-sin}
    \end{equation}
    \begin{equation}
k_1^a (m_t, K_t, t) =\lk \left(\fr{\prt}{\prt m_t} \right)
    a_0 (m_t, K_t, t)^{\mathrm{T}}\rk^{\mathrm{T}}\,;\label{e2.19-sin}
    \end{equation}
    \begin{equation}
\dot m_t = a_1 (m_t, K_t, t) \,,\enskip m_0 = m(t_0)\,,\label{e2.20-sin}
\end{equation}

\vspace*{-12pt}

\noindent
\begin{multline}
\dot K_t = k_1^a (m_t, K_t, t) K_t + K_t k_1^a (m_t, K_t, t)^{\mathrm{T}}
    +\si_0(t)\,;\\
    K_0 = K(t_0)\,;
    \label{e2.21-sin}
    \end{multline}

    \vspace*{-12pt}

    \noindent
\begin{multline}
\fr{\prt K(t_1, t_2)}{\prt t_2} =
    K(t_1, t_2) k_{t_2} k_1^a (m_{t_2}, K_{t_2}, t_2)^{\mathrm{T}}\,;\\
    K(t_1, t_2) = K_{t_1}\,.
    \label{e2.22-sin}
\end{multline}

Для стационарных ДСтС~(\ref{e2.3-sin})
при условии асимптотической устойчивости матрицы $k_1^a (\bar m, \bar K)$
в основе МСЛ лежат уравнения~(\ref{e2.15-sin}), записанные в виде:
    \begin{gather}
    a_1 (\bar m, \bar K) =0\,; \label{e2.23-sin}\\
k_1^a (\bar m, \bar K) \bar K + \bar K k_1^a
(\bar m, \bar K)^{\mathrm{T}} +\bar \si_0 =0\,;\label{e2.24-sin}
\end{gather}

\vspace*{-12pt}

\noindent
\begin{multline}
k_\tau (\tau) = k_1^a (\bar m, \bar K)k(\tau)\,,\enskip
k(0) =\bar K \enskip (\forall \tau >0)\,,\\
k(\tau) = k (-\tau)^{\mathrm{T}} \enskip (\forall \tau <0)\,.
\label{e2.25-sin}
\end{multline}

Уравнения~(\ref{e2.4-sin})--(\ref{e2.8-sin})
лежат в основе МНА для ДСтС~(\ref{e2.1-sin}), а уравнения~(\ref{e2.17-sin})--(\ref{e2.22-sin})~---
в основе МСЛ для ДСтС~(\ref{e2.3-sin}). Для определения стационарных СтП
согласно МНА служат соотношения~(\ref{e2.15-sin}) и~(\ref{e2.16-sin}),
а МСЛ~--- (\ref{e2.17-sin})--(\ref{e2.25-sin}).

\section{Уравнения методов нормальной~аппроксимации и~статистической линеаризации
для~эредитарных стохастических систем, приводимых к~дифференциальным}

Рассмотрим ЭСтС, описываемую интегродифференциальным уравнением Ито
следующего вида~\cite{7-sin}:

\noindent
\begin{multline}
dX_t = \lk a(X_t, t) +\iii_{t_0}^t a_1 (X(\tau) ,\tau, t)\,d\tau\rk dt+{}\\
{}+\lk b(X_t, t) +\iii_{t_0}^t b_1 (X(\tau) ,\tau, t)\,d\tau\rk dW_0+{}\\
\hspace*{-1.5mm}{}+\!\!\iii_{R_0^q}\!\!\lk c(X_t, t,v) +\!\iii_{t_0}^t\! c_1 (X(\tau) ,\tau, t,v)\,d\tau\!\rk\! dP^0 (t, dv)
\!\!\!\!\label{e3.1-sin}
\end{multline}
с начальным условием  $X(t_0) = X_0$. В~(\ref{e3.1-sin})
сохранены обозначения разд.~2.

Функции $a=a(X_t, t)$, $a_1\hm = a_1(X (\tau),\tau, t)$,
$b\hm=b(X_t, t)$, $b_1\hm = b_1(X (\tau),\tau, t)$,
$c\hm=c(X_t,t,v)$ и $c_1\hm = c_1(X (\tau),\tau, t,v)$ имеют
соответственно размерности $p\times 1$, $p\times 1$, $p\times r$,
$p\times r$, $p\times 1$ и $p\times 1$ и допускают представления следующего вида:
\begin{equation}
\left.
\begin{array}{rl}
a_1&=A(t,\tau) \vrp (X(\tau) , \tau)\,;\\[9pt]
b_1&=B(t,\tau) \psi (X(\tau) ,  \tau)\,;\\[9pt]
c_1&=C(t,\tau) \chi (X(\tau) ,  \tau, v)\,.
\end{array}
\right\}
\label{e3.2-sin}
\end{equation}
Здесь эредитарные ядра $A\hm=A(t,\tau)\hm=\lk A_{ij}(t,\tau)\rk$
$(i,j\hm=\overline{1,p})$,
$B\hm=B(t,\tau)=\lk B_{i l}(t,\tau)\rk$ $(i\hm=\overline{1,p}$;
$l\hm=\overline{1,r})$ и $C\hm=C(t,\tau)=\lk C_{ij}(t,\tau)\rk$
$(i,j\hm=\overline{1,p})$ имеют соответственно размерности
$p\times p$, $p\times r$ и $p\times p$. Они удовлетворяют следующим условиям
физической реализуемости и асимптотического затухания:
\begin{multline}
A_{ij}(t,\tau)=0;\enskip B_{i l}(t,\tau)=0;\\[1pt]
C_{ij}(t,\tau)=0\enskip \forall \tau >t;\label{e3.3-sin}
\end{multline}

\vspace*{-12pt}

\begin{equation}
\left.
\hspace*{-3mm}\begin{array}{c}
\displaystyle\iin\! \lv A_{ij} (t,\tau) \rv d\tau <\infty\,;\
\displaystyle\iin\! \lv B_{i l} (t,\tau) \rv d\tau <\infty \,;\\[9pt]
\displaystyle\iin \!\lv C_{ij} (t,\tau) \rv d\tau <\infty\,.
\end{array}\!
\right\}\!
\label{e3.4-sin}
\end{equation}

В дальнейшем ограничимся случаем, когда эредитарные ядра удовлетворяют
линейным операторным уравнениям~\cite{6-sin, 5-sin, 7-sin}.

Нелинейные в общем случае функции $\vrp\hm=\vrp(X(\tau),\tau)$,
$\psi \hm=\psi(X(\tau), \tau)$, $\chi \hm=\chi (X(\tau),  \tau, v)$
отражают нелинейные свойства ЭСтС, зависят от  $X(\tau)$ и имеют размерности
$p\times 1$, $p\times p$, $p\times 1$ соответственно.

Важный класс  эредитарных ядер представляют собой
сингулярные (вырожденные) ядра, когда имеют место представления:
\begin{equation}
\left.
\hspace*{-3mm}\begin{array}{rl}
A_{ij} (t,\tau) &= A_{ij}^+(t) A_{ij}^-(\tau)\,;\\[9pt]
B_{i l} (t,\tau)& = B_{il}^+(t) B_{il}^-(\tau)\,;\\[9pt]
C_{ij} (t,\tau) &= C_{ij}^+ ( t) C_{ij}^- (\tau)\
(i,l= \overline{1,p}, j=\overline{1,r}).
\end{array}\!
\right\}\!\!
\label{e3.5-sin}
\end{equation}

В~\cite{6-sin, 5-sin, 7-sin} показано, что для дифференцируемых нелинейных
функций~$\vrp$, $\psi$, $\chi$ путем расширения вектора состояния за счет
инструментальных переменных, аппроксимируемых линейными операторными уравнениями,
определяющими эредитарные ядра в ЭСтС, (\ref{e3.1-sin})--(\ref{e3.4-sin})
приводятся к ДСтС вида~(\ref{e2.1-sin}) или~(\ref{e2.3-sin}).
В~случае недифференцируемых нелинейных функций~$\vrp$, $\psi$, $\chi$
ЭСтС~(\ref{e3.1-sin})--(\ref{e3.4-sin}) приводятся к~(\ref{e2.1-sin}) или~(\ref{e2.3-sin})
на основе аппроксимации вырожденными (сингулярными) ядрами~\cite{6-sin, 5-sin, 7-sin}.

Таким образом, после приведения ЭСтС~(\ref{e3.1-sin}) к ДСтС~(\ref{e2.1-sin})
или~(\ref{e2.3-sin}) можно воспользоваться уравнениями МНА и МСЛ разд.~2.

\section{Типовые интегралы методов нормальной аппроксимации и~статистической
линеаризации}

Как следует из уравнений~(\ref{e2.9-sin})--(\ref{e2.11-sin}),
для МНА необходимо уметь вычислять следующие интегралы:
\begin{multline}
I_0^a = I_0^a (m_t, K_t, t) = a_1 (m_t, K_t, t)={}\\
{}= M_N a(Y_t, t)\,;
\label{e4.1-sin}
\end{multline}

\vspace*{-12pt}

\noindent
\begin{multline}
I_1^a = I_1^a (m_t, K_t, t)= a_{21}(m_t, K_t, t)= {}\\
{}=M_N a(Y_t , t) Y_t^{0\mathrm{T}}\,;\label{e4.2-sin}
\end{multline}

\vspace*{-12pt}

\noindent
\begin{multline}
I_0^{\bar \si} = I_0^{\bar \si} (m_t, K_t, t) = a_{22}(m_t, K_t, t) ={}\\
{}= M_N \bar \si (Y_t, t)\,.\label{e4.3-sin}
\end{multline}
Для МСЛ достаточно вычислить интеграл~(\ref{e4.1-sin}),
причем интеграл~$I_1^a$ вычисляется по формуле~\cite{2-sin, 3-sin, 4-sin}:
\begin{equation*}
k_1^a = k_1^a (m_t, K_t, t)=\lk \left( \fr{\prt}{\prt m_t}\right)
I_0^a (m_t, K_t, t)^{\mathrm{T}}\rk^{\mathrm{T}}. %\label{e4.4-sin}
\end{equation*}

\medskip

\noindent
\textbf{Пример 1.} В~[1] для типовых степенных, тригоно\-мет\-ри\-че\-ских,
показательных и ку\-соч\-но-по\-сто\-ян\-ных нелинейностей $Z_t \hm=\vrp (Y_t, t)$
скалярного и векторного аргумента приведены формулы для интегралов
$I_0^\vrp \hm= I_0^\vrp (m_t^y, K_t^y, t)$, а также
$k_1^\vrp \hm= k_1^\vrp (m_t^y, K_t^y, t)$.

\medskip

\noindent
\textbf{Замечание.}
 Важно иметь в виду, что уравнения МНА (МСЛ) содержат интегралы
 $I_0^a$, $I_1^a$, $I_0^\si$ в виде соответствующих коэффициентов.
 Поэтому процедура вычисления интегралов должна быть согласована с
 методом численного решения обыкновенных дифференциальных уравнений для
 $m_t$, $K_t$ и $K(t_1, t_2)$. Эти коэффициенты допускают дифференцирование
 по~$m_t$ и~$K_t$, так как под интегралом стоит сглаживающая нормальная плотность.

\section{Сложные конечные и~дифференциальные нелинейности}

Важный класс сложных конечных нелинейностей (многомерных и векторного аргумента)
представляют собой сложные функции вида:
    \begin{equation*}
    \xi =\vrp (X_t, Y_t, t)\,,\enskip X_t =\psi (Y_t, t)\,. %\label{e5.1-sin}
    \end{equation*}
В~этом случае вычисление интегралов (см.\ разд.~4) проводится по совокупности
переменных  $\lk X_t^{\mathrm{T}} Y_t^{\mathrm{T}}\rk^{\mathrm{T}}$.
К таким нелинейностям, например, относятся дроб\-но-ра\-ци\-о\-наль\-ные,
иррациональные  нелинейности, выражаемые специальными функциями, многозначные
нелинейности, зависящие от СтП~$X_t$ и его производных~$\dot X_t$,  $\ddot X_t$
и~др.

\medskip

\noindent
\textbf{Пример 2.}
Рассмотрим вычисление интегралов~(\ref{e4.1-sin}) и~(\ref{e4.2-sin})
для сложных одномерных иррациональных нелинейностей скалярного аргумента
\begin{equation}
\vrp (Y_t, t) =\lv Y_t\rrv^{\alpha-1}\, \mathrm{sgn}\, Y_t
\label{e5.2-sin}
\end{equation}
($\alpha$~--- нецелый показатель).

Пользуясь~(\ref{e2.16-sin}) и~(\ref{e2.19-sin}), представим~(\ref{e5.2-sin}) в виде
\begin{equation*}
\vrp(Y_t, t) = \vrp_0 (m_t, D_t, t) + k_1^\vrp(m_t, D_t, t) Y_t^0. %\label{e5.3-sin}
\end{equation*}
Здесь введены следующие обозначения:
\begin{gather*}
\vrp_0(m_t, D_t, t) =\Gamma(\alpha) D_t^{1/2} e^{-\xi^2/4} D_{-\alpha} (\xi)\,;%\label{e5.4-sin}
\\
k_1^a (m_t, D_t, t) =\fr {\prt \vrp_0(m_t, D_t, t)}{\prt m_t}\,,%\label{e5.5-sin}
\end{gather*}
где  $\Gamma(\alpha)$~--- гамма-функция,  $\xi \hm= m_t/\sqrt{D_t}$~---
отношение <<сиг\-нал--шум>>; $D_{-\alpha} (\xi)$~---
функция параболического цилиндра~\cite{9-sin}.
При вычислении были учтены следующие соотношения~\cite{9-sin, 8-sin}:
\begin{multline}
\iii_0^\infty x^{\alpha-1} e^{-\beta x^2 - \gamma x} \,dx ={}\\
{}=
(2\beta)^{-\alpha/2} \Gamma(\alpha) \exp \left(\fr{\gamma^2}{8\beta}\right)
D_{-\alpha} \left(\fr{\gamma}{\sqrt{2\beta}}\right)\,;\label{e5.6-sin}
\end{multline}

\vspace*{-12pt}

\noindent
\begin{multline}
\fr{dD_\rho(\xi)}{d\xi} =
   -\fr{\xi}{2}\, D_\rho (\xi) -\rho D_{\rho-1} (\xi) =
   \fr{\xi}{2}\, D_\rho (\xi) -{}\\
   {}- D_{\rho+1} (\xi) \enskip
   (\mathrm{Re}\, \beta>0\,,\enskip \mathrm{Re}\,\alpha>0\,,\enskip
   \rho=-\alpha)\,.\label{e5.7-sin}
   \end{multline}

Соотношения~(\ref{e5.6-sin}) и~(\ref{e5.7-sin})
могут быть использованы также для вычисления интегралов~(\ref{e4.3-sin}).

\medskip

\noindent
\textbf{Замечание.}
Для вычисления интегралов $I_0^a$, $I_1^a$ и $I_0^{\bar \si}$
применительно к типовым иррациональным нелинейностям вида
    $\lv Y_t\rrv^{\alp-1} e^{\delta Y_t}$, $\lv Y_t\rrv^{\alp-1}  \cos \w Y_t$,
    $\lv Y_t\rrv^{\alp-1}  \sin \w Y_t$
и более общим нелинейностям \mbox{вида}
    \begin{equation*}
    \vrp (Y_t, t) =\Phi^\vrp \left( \lv Y_t\rrv^{\alpha-1}, t\right) %\label{e5.8-sin}
    \end{equation*}
можно рекомендовать известные численные методы вычисления функций на ЭВМ~\cite{8-sin}.

\smallskip

\noindent
\textbf{Пример 3.}
Для нелинейной дроб\-но-ра\-ци\-о\-наль\-ной функции

\noindent
\begin{equation*}
\vrp (Y_t, t) = \fr{a}{(b+Y_t)^2} %\label{e5.9-sin}
\end{equation*}
имеем

\vspace*{-3pt}

\noindent
\begin{gather*}
\vrp_0 (m_t, D_t, t) =a b^{-2} \lk 1+ \chi (m_t, D_t, t)\rk\,; %\label{e5.10-sin}
\\
k_1^\vrp (m_t, D_t, t) =  a b^{-2}\fr{\prt \chi (m_t, D_t, t)}{\prt m_t}\,. %\label{e5.11-sin}
\end{gather*}
Здесь

\vspace*{-3pt}

\noindent
\begin{multline*}
\chi (m_t, D_t, t) ={}\\
{}=\sss_{n=1}^\infty \sss_{l=0}^{E(n/2)}
\fr{(-1)^n (n+1) n!}{(n-2l)! (2l)!}\, b^{-n} m_t^n \left( \fr{D_t}{ 2 m_t^2}
\right)^l, %\label{e5.12-sin}
\end{multline*}
где  $E(n/2)$~--- целая часть~$n/2$; $a\hm=a(t)$; $b\hm= b(t)$.

\vspace*{-6pt}

\section{Сложные интегральные нелинейности}

\vspace*{-2pt}

Пусть сначала векторно-матричная нелинейность имеет эредитарный характер, т.\,е.\
\begin{equation}
\underline{\vrp} (Y_t, t) =\iii_{t_0}^t A(t,\tau) \vrp (Y(\tau), \tau) \,d\tau\,.
\label{e6.1-sin}
\end{equation}
Тогда, как показано в~\cite{6-sin, 5-sin, 7-sin}, следует соответст\-ву\-ющие
интегродифференциальные соотношения путем введения  инструментальных
переменных привести к дифференциальным соотношениям.  Для
дифференцируемых функций~$\vrp$ и асимптотически устойчивых ядер
$A(t,\tau)$ зависимость~(\ref{e3.5-sin}) имеет следующий дифференциальный вид:
\begin{equation*}
F^A (t, D) \underline{\vrp} (Y_t, t) = H^A (t, D) \vrp (Y_t, t)\,. %\label{e6.2-sin}
\end{equation*}
Здесь $F^A (t, D)$ и  $H^A (t, D)$~--- линейные дифференциальные операторы $(D\hm= d/dt)$.

Для недифференцируемых функций~$\vrp$ и асимптотически устойчивых
сингулярных ядер~(\ref{e3.5-sin}) используются соотношения:
\begin{equation*}
\underline{\vrp} (Y_t, t) = A^+ Z\,,\enskip
\dot Z = A^- \vrp\,,\enskip
Z(t_0)=0\,. %\label{e6.3-sin}
\end{equation*}

Многочисленные примеры аналитического моделирования ЭСтС можно найти
в~[1--3, 5, 7, 10, 11].

Как отмечалось в~\cite{3-sin}, часто наряду с интегральными
нелинейностями~(\ref{e6.1-sin}) рассматривают нелинейности вида:

\columnbreak

\noindent
\begin{equation*}
Z_s =\sss_{\rho=1}^R \mathcal{A}_\rho \vrp_\rho (Y_{t_1}\tr Y_{t_r})\,, %\label{e6.2-sin}
\end{equation*}
где $\mathcal{A}_1 \tr \mathcal{A}_R$~--- произвольные линейные операторы,
действующие над функциями~$r$ переменных  $t_1\tr t_r$; $\vrp_\rho
\hm=\vrp_\rho (Y_{t_1} \tr Y_{t_r})$~--- линейные функции отмеченных
переменных. Такие нелинейности называются приводимыми к линейным.
Важным частным случаем~(\ref{e6.1-sin}) являются интегральные нелинейности вида:

\noindent
\begin{gather}
Z_s =\iii_T \vrp (Y_t, t, s)\, dt\,; \notag%\label{e6.3-sin}
\\
Z_s =\!\iii_T \!\cdots\!\iii_T\! \vrp (Y_{t_1}\tr Y_{t_r}; t_1\tr t_r, s)\,dt_1
\ldots dt_r,\notag %\label{e6.4-sin}
\end{gather}
В этом случае используется МСЛ по совокупности переменных  $Y_{t_1} \tr Y_{t_r}$.

\vspace*{-9pt}

\section{Заключение}

\vspace*{-2pt}

Разработаны методы и алгоритмы МНА и МСЛ для ДСтС и ЭСтС,
приводимых к ДСтС со сложными конечными, дроб\-но-ра\-ци\-о\-наль\-ны\-ми,
иррациональными, а также дифференциальными и интегральными нелинейностями.
Приведены примеры.

Результаты допускают обобщение на случай ДСтС и ЭСтС со
стохастическими нелинейностями, заданными каноническими разложениями и
интегральными каноническими  представлениями~\cite{1-sin, 3-sin, 11-sin}.

\vspace*{-9pt}

{\small\frenchspacing
 {%\baselineskip=10.8pt
 \addcontentsline{toc}{section}{References}
 \begin{thebibliography}{99}

 \vspace*{-2pt}

\bibitem{1-sin}
\Au{Синицын И.\,Н.,  Синицын~В.\,И.}
Лекции по нормальной и эллипсоидальной аппроксимации распределений в
стохастических сис\-те\-мах.~--- М.: ТОРУС ПРЕСС, 2013. 488~с.

\bibitem{6-sin} %2
\Au{Синицын И.\,Н. }
Stochastic hereditary control systems~// Проблемы управления и
теории информации, 1986. Т.~15. №\,4. С.~287--298.

\bibitem{2-sin} %3
\Au{Пугачев В.\,С., Синицын~И.\,Н.}
Стохастические дифференциальные сис\-те\-мы. Анализ и фильтрация.~--- М.:
Наука,  1990.  632~с. [Англ. пер.
 Stochastic differential systems.
Analysis and filtering.~--- Chichester, New York: Jonh Wiley, 1987.
549~p.].

\bibitem{5-sin} %4
\Au{Синицын И.\,Н. }
Конечномерные распределения процессов в стохастических интегральных
и интегродифференциальных системах~// Preprints of the 2nd IFAC
Symposium on Stochastic Control.~--- Vilnius: Pergamon Press,
1987.  Vol.~1. P.~144--153.

\bibitem{3-sin} %5
\Au{Пугачев В.\,С., Синицын~И.\,Н.}
Теория стохастических систем.~--- М.: Логос, 2000; 2004. 1000~с.
[Англ. пер.\linebreak\vspace*{-12pt}

\pagebreak

\noindent Stochastic systems. Theory and  applications.~---
Singapore: World Scientific, 2001. 908~p.].

\bibitem{4-sin} %6
\Au{Синицын И.\,Н.}
Параметрическое статистическое и аналитическое моделирование распределений
в нелинейных стохастических сис\-те\-мах на многообразиях~//
Информатика и её применения, 2013. Т.~7. Вып.~2. С.~4--16.

\bibitem{7-sin} %7
\Au{Синицын И.\,Н. }
Анализ и моделирование распределений в эредитарных стохастических
сис\-те\-мах~// Информатика и её применения, 2014. Т.~8. Вып.~1.\linebreak
С.~2--11.



\bibitem{9-sin} %8
\Au{Градштейн И.\,С., Рыжик~И.\,М.}
Таблицы интегралов, сумм, рядов и произведений.~--- М.: ГИФМЛ, 1963. 1100~с.

\bibitem{8-sin} %9
\Au{Попов Б.\,А., Теслер~Г.\,С. }
Вычисление функций на ЭВМ: Справочник.~--- Киев: Наукова Думка, 1984. 599~с.


\bibitem{11-sin} %10
\Au{Синицын И.\,Н.}
Канонические представления случайных функций и их применение в
задачах компьютерной поддержки научных исследований.~--- М.: ТОРУС
ПРЕСС, 2009. 768~с.

\bibitem{10-sin} %11
\Au{Синицын И.\,Н., Синицын~В.\,И., Корепанов~Э.\,Р., Белоусов~В.\,В.,
Сергеев~И.\,В., Басилашвили~Д.\,А.}
Опыт моделирования эредитарных стохастических сис\-тем~//
Кибернетика и высокие технологии XXI века: Сб. докл.  XIII Междунар.
науч.-технич. конф.~--- Воронеж: Саквоее, 2012. Т.~2. C.~346--357.

 \end{thebibliography}

 }
 }

\end{multicols}

\vspace*{-9pt}

\hfill{\small\textit{Поступила в редакцию 05.05.14}}

%\newpage

\vspace*{12pt}

\hrule

\vspace*{2pt}

\hrule

\vspace*{12pt}

\def\tit{ANALYTICAL MODELING OF NORMAL PROCESSES
 IN~STOCHASTIC SYSTEMS WITH~COMPLEX NONLINEARITIES}

\def\titkol{Analytical modeling of normal processes
 in~stochastic systems with~complex nonlinearities}

\def\aut{I.\,N.~Sinitsyn and V.\,I.~Sinitsyn}

\def\autkol{I.\,N.~Sinitsyn and V.\,I.~Sinitsyn}

\titel{\tit}{\aut}{\autkol}{\titkol}

\vspace*{-9pt}

\noindent
Institute of Informatics Problems, Russian Academy of Sciences,
44-2 Vavilov Str., Moscow 119333, Russian Federation


\def\leftfootline{\small{\textbf{\thepage}
\hfill INFORMATIKA I EE PRIMENENIYA~--- INFORMATICS AND
APPLICATIONS\ \ \ 2014\ \ \ volume~8\ \ \ issue\ 3}
}%
 \def\rightfootline{\small{INFORMATIKA I EE PRIMENENIYA~---
INFORMATICS AND APPLICATIONS\ \ \ 2014\ \ \ volume~8\ \ \ issue\ 3
\hfill \textbf{\thepage}}}

\vspace*{6pt}

\Abste{Differential stochastic systems (DStS) with Wiener and Poisson
noises and complex finite, differential, and  integral nonlinearities and
hereditary StS reducible to DStS are considered. Equations and algorithms
of analytical modeling based on the normal approximation method (NAM) and the
statistical linearization method (SLM) are given. The case of complex
continuous and discontinuous nonlinearities of scalar and vector arguments
is considered. Special attention is paid to NAM (SLM) typical integrals
for finite rational and irrational nonlinear and integral scalar and vector
nonlinear functions. The general case of integral nonlinearities reducible to
linear is considered. Test examples are given.}

\KWE{analytical modeling;
complex finite differential and integral nonlinearities;
complex irrational nonlinerarites
differential stochastic system with Wiener and Poisson noises;
method of normal approximation;
method of statistical linearization;
hereditary stochastic systems reducible to differential}

\DOI{10.14357/19922264140302}

  \begin{multicols}{2}

\renewcommand{\bibname}{\protect\rmfamily References}
%\renewcommand{\bibname}{\large\protect\rm References}

{\small\frenchspacing
 {%\baselineskip=10.8pt
 \addcontentsline{toc}{section}{References}
 \begin{thebibliography}{99}



\bibitem{1-sin-1}
\Aue{Sinitsyn, I.\,N., and  V.\,I.~Sinitsyn}.  2013.
Lektsii po normal'noy i ellipsoidal'noy approksimatsii raspredeleniy
v stokhasticheskikh sistemakh [Lectures on normal and ellipsoidal
approximation of distributions in stochastic systems].
Moscow: TORUS PRESS. 488~p.

\bibitem{6-sin-1} %2
\Aue{Sinitsyn, I.\,N.}  1986.
{Stochastic hereditary control systems}.
\textit{Problems Control Inform. Theory} 15(4):287--298.

\bibitem{2-sin-1} %3
\Aue{Pugachev, V.\,S., and  I.\,N.~Sinitsyn}.  1987.
\textit{Stochastic differential systems. Analysis and filtering.}
Chichester, New York: Jonh Wiley. 549~p.

\bibitem{5-sin-1} %4
\Aue{Sinitsyn, I.\,N.}  1987.
Konechnomernye raspredeleniya protsessov v stokhasticheskikh integral'nykh
i in\-teg\-ro\-dif\-fe\-ren\-tsial'nykh sistemakh [Finite dimensional distributions
of processes in stochastic integral and integrodifferential systems].
\textit{2nd  Symposium (International) IFAC on Stochastic Control
Preprints}. Vilnius: Pergamon Press. 1:144--153.

\bibitem{3-sin-1} %5
\Aue{Pugachev, V.\,S., and I.\,N.~Sinitsyn}. 2001.
\textit{Stochastic systems. Theory and  applications}.
Singapore: World Scientific. 908~p.

\bibitem{4-sin-1} %6
\Aue{Sinitsyn, I.\,N.}  2013.
Parametricheskoe statisticheskoe i analiticheskoe modelirovanie
raspredeleniy v nelineynykh stokhasticheskikh sistemakh na mnogoobraziyakh
[Parametric statistical and analytical modeling of distributions in
stochastic systems on manifolds].
\textit{Informatika i ee Primeneniya}~--- \textit{Inform. Appl.} 7(2):4--16.


\bibitem{7-sin-1} %7
\Aue{Sinitsyn, I.\,N.}  2014.
Analiz i modelirovanie raspredeleniy v ereditarnykh stokhasticheskikh sistemakh
[Analysis and modeling of distributions in hereditary stochastic systems].
\textit{Informatika i ee Primeneniya}~--- \textit{Inform. Appl.} 8(1):2--11.

\bibitem{9-sin-1} %8
\Aue{Gradshteyn, I.\,S., and I.\,M.~Ryzhik}.  1963.
\textit{Tablitsy integralov, summ, ryadov i proizvedeniy}
[Tables of integrals, sums, series, and products]. Moscow:  GIFML.   1100~p.

\pagebreak

\bibitem{8-sin-1} %9
\Aue{Popov, B.\,A., and G.\,S.~Tesler}.  1984.
\textit{Vychislenie funktsiy na EVM}. Spravochnik [Computing of functions].
Kiev: Naukova Dumka.  599~p.


\bibitem{11-sin-1} %10
\Au{Sinitsyn, I.\,N.} 2009.
\textit{Kanonicheskie predstavleniya sluchaynykh funktsiy i ikh primenenie v
zadachakh komp'yuternoy podderzhki nauchnykh issledovaniy}
[Canonical expansions of random functions and its application to
scientific computer-aided support]. Moscow: TORUS PRESS. 768~p.

\bibitem{10-sin-1} %11
\Aue{Sinitsyn, I.\,N., V.\,I.~Sinitsyn, E.\,R.~Korepanov,
V.\,V.~Belousov, I.\,V.~Sergeev, and D.\,A.~Basilashvili}.
2012. Opyt modelirovaniya ereditarnykh stokhasticheskikh sistem
[Experience of modeling in hereditary stochastic systems].
\textit{Kibernetika i Vysokie Tekhnologii XXI~Veka:
Sbornik dokladov  XIII Mezhdunar. nauch.-tekhnich. konf.}
[Cybernatics ans High Technologies of the XXI Century: Materials of
XIII  Scientific and Technological Conference (International)].
Voronezh: Sakvoee. 2:346--357.

\end{thebibliography}

 }
 }

\end{multicols}

\vspace*{-6pt}

\hfill{\small\textit{Received May 05, 2014}}

\vspace*{-18pt}

\Contr

\noindent
\textbf{Sinitsyn Igor N.} (b.\ 1940)~---
Doctor of Science in technology, professor, Honored scientist of RF, Head of Department, Institute of
Informatics Problems, Russian Academy of Sciences,
44-2 Vavilov Str., Moscow 119333, Russian
Federation; sinitsin@dol.ru

\vspace*{3pt}

\noindent
\textbf{Sinitsyn Vladimir I.} (b.\ 1968)~--- Doctor of Science in physics
and mathematics, associate professor, Head of Department, Institute of
Information Problems, Russian Academy of Sciences,
44-2 Vavilov Str., Moscow 119333, Russian Federation; VSinitsin@ipiran.ru




\label{end\stat}

\renewcommand{\bibname}{\protect\rm Литература} %4+
\def\stat{kovalev}

\def\tit{МЕТОДЫ ТЕОРИИ КАТЕГОРИЙ В~МОДЕЛЬНО-ОРИЕНТИРОВАННОЙ СИСТЕМНОЙ 
ИНЖЕНЕРИИ}

\def\titkol{Методы теории категорий в~модельно-ориентированной системной 
инженерии}

\def\aut{С.\,П.~Ковалёв$^1$}

\def\autkol{С.\,П.~Ковалёв}

\titel{\tit}{\aut}{\autkol}{\titkol}

\index{Ковалёв С.\,П.}
\index{Kovalyov S.\,P.}


%{\renewcommand{\thefootnote}{\fnsymbol{footnote}} \footnotetext[1]
%{Исследование выполнено при финансовой поддержке Российского научного фонда (проект 16-11-10227).}}


\renewcommand{\thefootnote}{\arabic{footnote}}
\footnotetext[1]{Институт проблем управления им.\ В.\,А.~Трапезникова 
Российской академии наук,  \mbox{kovalyov@nm.ru}}

%\vspace*{-18pt}

\Abst{Предложен математический аппарат на базе теории категорий, который позволяет 
формально описывать и~строго исследовать процедуры применения моделей в~инженерной 
деятельности, составляющие сущность мо\-дель\-но-ори\-ен\-ти\-ро\-ван\-ной системной 
инженерии (Model-Based Systems Engineering, MBSE). В~основе аппарата лежит 
математическое представление сборочных чертежей (мегамоделей сис\-тем) диаграммами 
в~категориях, объектами которых служат модели, а~морфизмы представляют действия по 
сборке моделей сис\-тем из моделей компонентов. Адекватность аппарата обоснована исходя 
из требований стандартов, регламентирующих описание структуры систем, в~том числе 
IEC~81346. Предложены и~исследованы тео\-ре\-ти\-ко-ка\-те\-гор\-ные методы решения ряда 
практических задач сборки систем. Приведены примеры решения таких задач в~категориях, 
представляющих две ключевые области применения MBSE: гео\-мет\-ри\-че\-ское моделирование 
изделий сложной формы и~дис\-крет\-но-со\-бы\-тий\-ное имитационное моделирование 
поведения технических систем.}

\KW{модельно-ориентированная системная инженерия; мегамодель; теория категорий; 
копредел}



\DOI{10.14357/19922264170305} 


\vspace*{6pt}

\vskip 10pt plus 9pt minus 6pt

\thispagestyle{headings}

\begin{multicols}{2}

\label{st\stat}

\section{Введение}

   Модельно-ориентированная системная инженерия состоит в~формализованном применении моделирования в~
поддержке жизненного цикла сис\-тем, включая сбор требований, 
проектирование, проверку и~приемку, другие стадии~[1]. Модели, 
разрабатываемые в~ходе процедур MBSE, пригодны к~автоматической 
обработке на компьютерах. Это позволяет сначала задавать, верифицировать 
и~оптимизировать проектные решения на моделях <<в циф\-ре~и только потом 
воплощать <<в железе>>, снижая затраты на организацию жизненного цикла 
изделий и~сокращая сроки выполнения работ~[2].
   
   И все же внедрение технологий MBSE в~инженерную деятельность 
происходит медленно. Это связано во многом с~нехваткой единой 
концептуальной базы инженерного моделирования: предлагается много 
частных языков и~технологий, слабо совместимых друг с~другом и~плохо 
приспособленных для совместной разработки моделей большими 
мультидисциплинарными коллективами~[3]. Тем самым затрудняется переход 
от набора электронных чертежей к~полноценному электронно-цифровому 
макету (digital mock-up) промышленного изделия.
   
   Естественный, хотя и~<<трудный>>, подход к~получению результатов 
общего характера, унифи\-ци\-ру\-ющих разнородные технологии, состоит в~том, 
чтобы как можно более строго формализовать процедуры моделирования. 
Формализация позволит совершенствовать процедуры MBSE и~передавать их 
на исполнение компьютеру без пробелов и~искажений. Самый высокий уровень 
строгости достигается при привлечении математического аппарата, поскольку 
математика позволяет надежно доказывать или опровергать утверждения, 
ха\-рак\-те\-ри\-зу\-ющие корректность и~эффективность процедур.
   
   В настоящей работе предложен аппарат, основанный на математическом 
представлении сборочных чертежей (<<мегамоделей>> систем) 
ориенти-\linebreak рованными графами (диаграммами). Узлы такого\linebreak графа помечаются 
обозначениями моделей час\-тей, а~реб\-ра помечаются обозначениями действий\linebreak 
(activities), посредством которых части собираются в~систему. Представление 
структуры систем графами регламентируется, в~частности, стандартом 
IEC~81346~[4]. Естественным источником математических методов 
конструирования и~анализа мегамоделей служит теория категорий (см., 
например,~[5, 6]). Модели рассматриваются как объекты подходящих 
категорий, а~действия формально описываются морфизмами. Строятся 
и~исследу-\linebreak ются тео\-ре\-ти\-ко-ка\-те\-гор\-ные конструкции, опи\-сы\-ва\-ющие процедуры 
MBSE на абстрактном кон-\linebreak цептуальном уровне. Определенный опыт такого\linebreak 
исследования был накоплен в~инженерии программного обеспечения~[7] 
и~теперь может быть обобщен для системной инженерии в~целом. Например, 
сборке системы согласно некоторой мегамодели отвечает построение 
копредела диаграммы~--- универсальной конструкции~\cite{5-kov}.
   
   Статья построена следующим образом. В~разд.~2 приведен обзор 
принципов описания структуры сис\-тем согласно стандарту IEC~81346. 
Раздел~3 посвящен практическим проб\-ле\-мам мегамоделирования и~сборке 
сис\-тем. В~разд.~4 вводятся конструкции тео\-рии категорий, позволяющие 
формально решать задачи мегамоделирования. В~заключении приводятся 
выводы и~намечаются направления дальнейших исследований.

\section{Структура систем и~стандарт~IEC~81346}

   Важной проблемой MBSE, отмеченной во введении, является слабая 
совместимость языков и~инструмен\-тов моделирования от разных поставщиков. 
Основным подходом к~достижению совместимости является стандартизация~--- 
принятие обязывающих документов, устанавливающих требования и~принципы 
взаимозаменяемости инструментов. Многие стандарты определяют конкретные 
форматы машиночитаемой записи моделей, нейтральные относительно 
разработчиков инструментов MBSE. Примером служит формат описания 
твердотельных геометрических моделей STEP, стандартизованный семейством 
ISO~10303. Однако для формализации MBSE в~целом интерес представляют 
в~первую очередь стандарты более общего плана, унифицирующие принципы 
и~методы применения моделей в~жизненном цикле систем независимо от 
способа записи моделей. С~этой точки зрения внимания заслуживает 
международный стандарт IEC 81346-1:2009 <<Промышленные системы, 
установки и~обору\-до\-ва\-ние~--- принципы структурирования и~ссылочные 
обозначения~--- часть~1: основные правила>> (<<Industrial Systems, 
Installations and Equipment and Industrial Products~--- Structuring Principles and 
Reference Designations~--- Part~1: Basic Rules>>)~\cite{4-kov}. Стандарт не 
принят в~России, однако ряду его положений в~области структуры систем 
соответствует российский ГОСТ~2.053-2013 <<ЕСКД. Электронная структура 
изделия. Общие положения>>.
   
   В стандарте IEC~81346 рассматривается ряд вопросов моделирования 
структуры систем и~идентификации отдельных единиц в~составе систем. 
Системная единица названа в~стандарте объектом, причем принципиально не 
проводится различие между объектами реального мира, составляющими 
реально существующие системы, и~объектами мыслительной деятельности~--- 
моделями единиц, составляющими модели систем. Таким образом, стандарт 
выходит за рамки MBSE и~рассматривает ряд вопросов системной инженерии 
вообще. Иерар\-хи\-че\-ская структура системы (холархия~\cite{3-kov}) 
изображается деревом, узлы которого помечены обозначениями объектов. 
Важным достижением стандарта является выявление того факта, что одна и~та 
же система задается не одной, а несколькими в~общем случае различными 
иерархическими структурами, возникающими в~результате декомпозиции 
согласно различным принципам (аспектам). В~их числе:
   \begin{itemize}
\item функциональная (function-oriented) структура, отвечающая разделению 
системных единиц по выполняемым ими функциям в~составе сис\-темы;
\item продуктовая (product-oriented), или модульная, структура, отражающая 
сборочную (технологическую) конфигурацию сис\-темы;
\item структура размещения (location-oriented), в~соответствии с~которой 
единицы располагаются в~физическом пространстве.
\end{itemize}

   Ясно, что один и~тот же объект может входить в~несколько структур и~при 
этом находиться на различных уровнях. В~то же время в~некоторых аспектах 
объект может никак не проявлять себя и~вследствие этого отсутствовать 
в~соответствующих структурах. Полное идентифицирующее ссылочное 
обозначение объекта (reference designation) конструируется путем 
последовательного перечисления всех объектов, находящихся на пути от корня 
дерева рассматриваемой структуры до дан\-ного объекта включительно. 
Наименование каж\-до\-го объекта в~этом перечислении составляется из 
символьного обозначения аспекта, буквенного обозначения класса (типа), 
к~которому относится  объект, и~порядкового номера объекта среди 
экземпляров своего класса. Таким путем обеспечивается\linebreak  уникальность 
наименования любой единицы\linebreak
 в~пределах системы. Например, функциональная 
структура обозначается символом <<=>>, а~функциональный класс 
переключателей потоков ресурсов обозначается буквами QA, так что первая по 
порядку единица, выполняющая функцию переключения, называется =QA1, 
а~ее полное ссылочное обозначение может выглядеть как =WP1=WC1=QA1. 
Если объект присутствует в~нескольких структурах, то он может иметь 
несколько ссылочных обозначений, как показано на рис.~1~\cite{4-kov}.

\begin{figure*} %fig1
    \vspace*{1pt}
\begin{center}
\mbox{%
\epsfxsize=165mm
\epsfbox{kov-1.eps}
}
\end{center}
\vspace*{-9pt}
\Caption{Пример ссылочных обозначений структурных единиц системы}
\vspace*{9pt}
\end{figure*}

   С~точки зрения практики системной инженерии большой интерес 
представляет описание эволюции структурного представления системы по ходу 
жизненного цикла, приведенное в~приложении~B к~стандарту IEC~81346. 
<<Строительный материал>> для структур имеет вид (виртуального) 
справочника или каталога объектов, из которого выбираются объекты для 
включения в~структуру. 

В~начале жизненного цикла системы на основе 
исходных требований к~ней конструктор строит ее функциональную структуру. 
Затем определяется пространственное положение функциональных объектов, 
в~результате чего создается структура размещения. На следующей стадии 
формируются закупочные спецификации, образующие продуктовую структуру. 
В~ходе последующих стадий жизненного цикла эти структуры могут 
трансформироваться. На каждой стадии могут происходить замена, слияние 
и~расщепление объектов. Таким образом, объекты разных структур системы 
связаны отношением вида <<многие ко многим>>, вдоль которого 
прослеживаются (трассируются) исходные требования.
   
   В то же время стандарт не предусматривает указа\-ние способов, какими 
объекты собраны в~сис\-те\-мы. Поэтому структуру сис\-те\-мы можно рас\-смат\-ри\-вать 
как эскизный проект, в~котором отражены лишь факты вхождения системных 
единиц более низкого уровня иерархии в~единицы более высокого уровня. 


Проект такого рода поступает на вход технологу, который определяет 
конкретные операции сборки каждой единицы каждого уровня иерархии. При 
необходимости технолог вносит изменения в~конструкцию объектов (такие как 
нарезка резьбы) и~добавляет связующие интерфейсные объекты (такие как 
клей, трансформатор и~др.). В~результате для каждого составного объекта 
формируется сборочный чертеж, на котором указаны все со\-став\-ля\-ющие 
объекты и~действия по их соединению в~целях получения сис\-те\-мы. 
Технологическая проработка требуется на всех стадиях жизненного цикла, на 
которых формируется либо изменяется ка\-кая-ли\-бо из структур системы.

%\vspace*{-6pt}

\section{Мегамоделирование и~сборка~систем}

   В MBSE объекты, образующие 
структуры\linebreak
 сис\-тем, описываются формализованными ком\-пьютерными моделями 
различных видов: геометрическими фигурами и~телами, численными 
аппроксимациями дифференциальных уравнений, оснащенными графами и~
т.\,д. При этом, как свидетель\-ст\-ву\-ют стандарты типа IEC~81346, для анализа 
структуры систем и~организации сборки необходимо знать не столько 
внутреннюю структуру моделей, сколько ассортимент их возможностей 
соединяться с~другими моделями в~целях формирования моделей составных 
объектов. Иными словами, модели рассматриваются как <<черные ящики>> 
с~известным поведением по отношению к~другим моделям. Каталог объектов, 
упоминавшийся в~предыду\-щем разделе, в~условиях применения \mbox{MBSE} 
составляется из моделей и~описаний действий по их соединению.
   
   Структуры систем и~сборочные чертежи представляют собой частные 
случаи мегамоделей (mega\-mod\-el)~--- моделей, состоящих из моделей и~связей 
между ними~\cite{8-kov}. Мегамодель, в~которой связи описывают соединение 
моделей, образующих некоторую сис\-те\-му, называется конфигурацией этой 
сис\-те\-мы~\cite{5-kov}. Существуют и~другие виды мегамоделей, 
предназначенные для описания других процедур \mbox{MBSE}, таких как 
формирование модели согласно заданной метамодели  
(instantiating)~\cite{9-kov}. Но в~настоящей работе сосредоточимся на 
конфигурациях и~сборке систем.
   
   Например, в~моделировании механических сис\-тем, состоящих из твердых 
тел, моделями деталей и~сборочных единиц служат геометрические тела, 
которые могут быть представлены для компьютерной обработки различными 
способами: конструктивным, воксельным, граничным~\cite{10-kov}. Объекты, 
составляющие механические системы, т.\,е.\ представления экземпляров тел, 
получаются из моделей путем аффинных изометрий и~растяжений. Так, из 
набора цилиндров разных размеров составляется модель штанги (спортивного 
снаряда). В~функциональной структуре штанги по IEC~81346 цилиндры 
представлены разными объектами, поскольку они выполняют разные функции, 
хотя порождаются одной и~той же геометрической моделью. Соответственно, 
в~каталоге моделей содержится тело в~форме цилиндра, допускающее 
несколько разных действий по включению в~состав штанги.
   
   В качестве еще одного примера рассмотрим дис\-крет\-но-со\-бы\-тий\-ное 
имитационное моделирование, поддержка которого относится к~числу 
важнейших достижений MBSE~\cite{1-kov}. Здесь модель имеет вид 
сценария~--- фрагмента предполагаемой истории поведения моделируемой 
системы, пред\-став\-лен\-но\-го потоком дискретных событий различных видов. 
Некоторые события могут вызывать либо запрещать возникновение других 
событий. Описания действий по сборке сценариев поведения систем отражают 
вклад сценариев поведения составляющих. Так, сценарий работы цеха 
составляется из сценариев работы станков, связанных друг с~другом согласно 
маршрутным картам~\cite{11-kov}.
   
   Сформулируем задачу мегамоделирования сборки систем в~общем виде 
следующим образом. По мегамодели, представляющей конфигура\-цию 
некоторой системы, требуется сконструировать модель системы как целого 
и~рассчитать для нее моделируемые параметры, в~том числе эмерджентные~--- 
не присущие никакой из со\-став\-ля\-ющих единиц в~отдельности. Принцип 
конструирования модели системы легко усмотреть из организации 
структур-\linebreak\vspace*{-12pt}

\columnbreak

 { \begin{center}  %fig1
 \vspace*{1pt}
\mbox{%
\epsfxsize=57.246mm
\epsfbox{kov-2.eps}
}


\vspace*{12pt}


\noindent
{{\figurename~2}\ \ \small{Схема склеивания}}
\end{center}
}

\vspace*{18pt}

\addtocounter{figure}{1}

\noindent
ного представления: система должна находиться на иерархическом 
уровне, располагающемся непосредственно над уровнем со\-став\-ля\-ющих ее 
объектов. Иными словами, модель системы должна включать в~себя модели 
всех составляющих с~учетом их конфигурационных связей и~в~то же время 
включаться в~любые модели, включающие в~себя модели всех составляющих 
конфигурации.
   
   Поясним этот принцип на простом примере. Предположим, что нужно 
объединить в~систему два объекта~$P$ и~$S$ и~что технолог решил сделать это 
с~по\-мощью клея~--- третьего объекта~$G$, который может быть соединен 
и~с~$P$, и~с~$S$. Действие клея описывается конфигурацией следующего 
вида: объекты~$G$ и~$P$ порождают в~результате соединения известный 
промежуточный комплексный объект~$P_G$, содержащий их, а~объекты~$G$ 
и~$S$ порождают объект~$S_G$. Система~$R$, полученная путем 
склеивания~$P$ с~$S$ при помощи~$G$, отбирается среди объектов, 
содержащих~$P_G$ и~$S_G$, по следующему структурному критерию: 
объект~$R$ должен содержаться в~любом объекте~$T$, содержащем~$P_G$ 
и~$S_G$. Схематически этот критерий изображен на рис.~2.


   Если объект $R$, удовлетворяющий указанному структурному критерию, 
существует, то он действительно отвечает системе, которая собрана из~$S$ 
и~$P$ путем склеивания посредством~$G$ (и~не содержит ничего 
<<лишнего>>). Более того, легко видеть, что такой объект~$R$ определяется, 
по существу, однозначно в~том смысле, что любые два объекта~$R$ 
и~$R^\prime$, удовлетворяющие структурному критерию, содержатся друг 
в~друге. Если же нужного объекта~$R$ не существует, то делается вывод, что 
технолог ошибся: клей~$G$ не способен соединить объекты~$P$ и~$S$.
   
   В структурное представление, выполненное по стандарту IEC~81346 либо по 
ГОСТу 2.053-2013, входят только объекты~$P$, $S$ и~$R$ и~две композитные 
стрелки: $P\hm\to R$, проходящая через~$P_G$, и~$S\hm\to R$, проходящая 
через~$S_G$ (так что мегамодель склеивания~--- это часть схемы, ограниченная 
треугольником~$PSR$). Кроме того, стрелки на схеме склеивания, в~отличие от 
структуры, представляют не просто факты включения объектов друг в~друга, 
а~конкретные действия по их соединению. При этом соблюдается следующее 
естественное условие структурной корректности: если из одного объекта 
можно прийти в~другой разными путями по схеме, то эти пути задают одно и~то 
же композитное действие. Например, клей~$G$ включается в~состав 
системы~$R$ единственным способом, несмотря на наличие двух путей $G 
\hm\to  P_G \hm\to R$ и~$G \hm\to S_G \hm\to R$: в~действительности не имеет 
значения, через какой промежуточный объект <<прослеживается>> включение 
клея в~систему. Таким образом, мегамодель сборки содержит больше 
информации, чем иерархическая структура системы.
   
   Если модели содержат значения тех или иных параметров, а описание 
действий по их соединению позволяет выявить правила преобразования 
значений, то по мегамодели сборки можно вы\-чис\-лить значения параметров для 
системы. Известны примеры вычислений такого рода в~области разработки 
новых композиционных материалов~\cite{12-kov}. Осредненные 
(эффективные) физические характеристики композитов, такие как модуль Юнга и~коэффициент Пуассона, сложным образом зависят от характеристик 
компонентов и~способов изготовления композита из них. При помощи методов 
теории упру\-гости эти зависимости задаются в~форме линеаризованных 
матричных соотношений, которые приписываются к~стрелкам мегамоделей, 
пред\-став\-ля\-ющим включение компонентов в~композиты. Появляется 
возможность рассчитывать на компьютере свойства композитов по базе данных 
компонентов, без проведения дорогостоящих физических экспериментов.
   
   В заключение раздела отметим, что хотя прямой расчет системы по 
конфигурации имеет большое значение, в~MBSE он играет вспомогательную 
роль. Согласно стандарту IEC~81346 и~практикам системной инженерии, 
система обычно проектируется сверху вниз~--- от корня структурной иерархии 
к~составляющим~\cite{13-kov}. Это означает, что технолог в~основном решает 
не прямую, а~обратную задачу: модель системы, которую нужно собрать, 
известна, а~нужно построить (восстановить) конфигурацию, из которой такая 
система может быть получена путем сборки, с~учетом различных ограничений. 
Формальные математические постановки и~методы решения обратных задач 
мегамоделирования представляют собой крупную перспективную тему 
исследований, выходящую за рамки настоящей статьи.

\section{Теория категорий в~мегамоделировании}

   Как указывалось во введении, естественным источни\-ком математических 
методов кон\-стру\-ирова\-ния и~анализа мегамоделей служит теория категорий. 
Категорией называется коллекция абстрактных объектов, попарно связанных 
морфизмами (стрелками). Точное определение занимает буквально несколько 
строк~\cite{14-kov}: категория~$C$ состоит из совокупности 
объектов~$\mathrm{Ob}\,C$ и~совокупности морфизмов~$\mathrm{Mor}\,C$, 
на которых заданы следующие операции:
\begin{enumerate}[(1)]
\item каждому морфизму~$f$ 
сопоставляется два объекта: область $\mathrm{dom}\,f$ и~кообласть 
$\mathrm{codom}\,f$ (соотношения вида $\mathrm{dom}\,f \hm= A$ и~
$\mathrm{codom}\,f \hm= B$ наглядно записываются в~форме стрелки~$f$: 
$A\hm\to B$, а множество всех морфизмов, удовлетворяющих этим 
соотношениям, обозначается через $\mathrm{Mor}(A, B))$;
\item для 
любой пары морфизмов~$f, g$, удовлетворяющей условию 
$\mathrm{codom}\,f\hm = \mathrm{dom}\,g$, определена композиция~--- 
морфизм $g \circ f : \mathrm{dom}\,f \hm\to  \mathrm{codom}\,g$, причем она 
ассоциативна: для любой тройки морфизмов~$f, g, h$, удовлетворяющей 
условиям $\mathrm{codom}\,f \hm= \mathrm{dom}\,g$ и~$\mathrm{codom}\,g 
\hm= \mathrm{dom}\,h$, выполняется соотношение $h \circ (g \circ f) \hm= (h 
\circ g) \circ f$;
\item любой объект~$A$ обладает тождественным 
морфизмом~$1_A : A \to A$ таким, что для любого морфизма~$f : A\hm\to B$ 
выполняется соотношение $f \circ 1_A \hm= 1_B \circ  f \hm= f$.
\end{enumerate}

Классическим 
примером категории служит $\mathbf{Set}$, состоящая из всех множеств и~всех 
их отображений: закон композиции отображений задается стандартной 
подстановкой, а тождественным морфизмом произвольного множества служит 
его тождественное отображение на себя.
   
   Вместе с~категорией вводится понятие функтора~--- отображения категорий, 
сохраняющего структуру. Функтор $\mathrm{fun}\,: C \hm\to D$, действующий из 
категории~$C$ в~$D$,~--- это пара одноименных отображений $\mathrm{fun}\,: 
\mathrm{Ob}\,C \hm\to \mathrm{Ob}\,D$, $\mathrm{fun}\,: \mathrm{Mor}\,C \hm\to 
\mathrm{Mor}\,D$, удовлетворяющая следующим условиям (для произвольных 
$C$-мор\-физ\-мов~$f, g$ и~$C$-объ\-ек\-та~$A$): 
\begin{enumerate}[(1)]
\item $\mathrm{fun}\,(\mathrm{dom}\,f) 
\hm= \mathrm{dom}\,\mathrm{fun}\,(f), \mathrm{fun}\,(\mathrm{codom}\,f)\hm = 
\mathrm{codom}\,\mathrm{fun}\,(f)$;  
\item $\mathrm{fun}\,(g \circ f) = \mathrm{fun}\,(g) \circ \mathrm{fun}\,(f)$, 
если композиция $g \circ f$ определена; 
\item $\mathrm{fun}\,(1_A) \hm= 1_{\mathrm{fun}\,(A)}$.
\end{enumerate}
 Все категории и~все функторы образуют 
(формальную) категорию~$\mathbf{CAT}$. Чтобы исследовать взаимосвязь 
между функторами, вводится следующее понятие: естественным 
преобразованием~$\varepsilon$ функтора $\mathrm{fun}\, : C\hm\to D$ в~$\mathrm{fun}^\prime\, : C 
\hm\to D$ называется любое семейство $D$-мор\-физ\-мов~$\varepsilon_A : 
\mathrm{fun}\,(A) \hm\to \mathrm{fun}^\prime (A)$, $A \hm\in \mathrm{Ob}\,C$, 
такое что для любого 
\mbox{$C$-мор}\-физ\-ма $f : A\hm\to B$ выполняется соотношение $\varepsilon_B \circ 
\mathrm{fun}\,(f) \hm= \mathrm{fun}^\prime(f) \circ \varepsilon_A$:

%\begin{figure*} %рис
\vspace*{1pt}
\begin{center}
\mbox{%
\epsfxsize=54.473mm
\epsfbox{kov-3.eps}
}
\end{center}
%\vspace*{-9pt}
%\end{figure*}

   Эффективность применения теории категорий в~качестве математического 
аппарата \mbox{MBSE} обуслов\-ле\-на тем, что любой каталог моделей представляет 
собой не что иное, как категорию. Действительно, любая цепочка действий по 
соединению моделей порождает композитное действие (процесс) и, кроме того, 
любая модель допускает пустое действие над самой собою, не 
подразумевающее никаких изменений (процедура <<ничегонеделания>>). 
Например, в~твердотельном моделировании механических систем объектами 
категории\linebreak моделей выступают тела~--- подмножества в~$\mathbb{R}^3$, 
которые являются ограниченными, регулярными\linebreak
 (совпадают с~замыканием 
своей внутренности) и~полуаналитическими (допускают представление 
конечными булевыми комбинациями множеств вида $\{(x, y, z) \vert  F_i(x, y, 
z)\hm\leq 0\}$, где~$F_i : \mathbb{R}^3\hm\to \mathbb{R}$ является 
вещественной аналитической функцией для всех~$i$)~\cite{10-kov}. Чтобы 
было возможно задавать процедуры типа склеивания участков поверхности тел, в~категорию геометрических моделей добавляются ограниченные регулярные 
полуаналитические подмножества в~$\mathbb{R}^n$, $0 \hm\leq n \hm\leq 2$, 
при помощи стандартного вложения~$\mathbb{R}^n$ в~$\mathbb{R}^3$. Далее 
выполняется факторизация: отождествляются друг с~другом все множества, 
переходящие друг в~друга под действием аффинных изометрий. Морфизмы 
таких классов эквивалентности, описывающие действия по сборке составных 
механических сис\-тем, порождаются изометрическими вложениями множеств 
и~растяжениями. Получается подкатегория в~\textbf{Set}, которую будем обозначать 
через $\mathbf{MBS}$ (от Multibody Systems).
   
   Для многих известных технологий MBSE формальное описание каталогов 
поддерживаемых моделей приводит к~категориям множеств со структурой~--- 
алгебраических систем, топологических пространств, графов и~т.\,д. 
Морфизмами в~таких категориях служат отображения множеств, со\-вмес\-ти\-мые 
со структурой. На любой такой категории действует канонический функтор 
в~$\mathbf{Set}$, <<забывающий>> структуру. 

В~качестве примера приведем  
дис\-крет\-но-со\-бы\-тий\-ное моделирование, в~котором математической 
моделью сценария служит множество событий, час-\linebreak тич\-но упорядоченное  
при\-чин\-но-след\-ст\-вен\-ны\-ми зависимостями и~размеченное видами 
событий~\cite{15-kov}. Действия по сборке сложных сценариев задаются 
монотонными отображениями, сохраняющими разметку, поскольку ни 
события, ни зависимости, ни метки не могут быть <<потеряны>> при 
соединении сценариев поведения компонентов в~сценарии поведения систем. 
Получается категория~$\mathbf{Pomset}$, состоящая из всех помеченных 
частично упорядоченных множеств и~всех их монотонных отображений, 
сохраняющих разметку. Имеется функтор $\vert \mbox{--} \vert : 
\mathbf{Pomset}\hm\to \mathbf{Set} : S \mapsto \vert S\vert$, <<забывающий>> 
порядок и~разметку.
   
   Зафиксируем произвольную категорию~$C$, представляющую некоторый 
каталог моделей. Как и~для любой алгебраической системы, определена 
конструкция подкатегории в~$C$~--- это пара, состоящая из подкласса 
в~$\mathrm{Ob}\,C$ и~подкласса в~$\mathrm{Mor}\,C$, замкнутых 
относительно унаследованных из~$C$ операций. Подкатегория в~$C$ 
называется полной, если любой \mbox{$C$-мор}\-физм, область и~кообласть которого 
содержатся в~ней, сам содержится в~ней. Например, подкатегориями 
описываются различные аспекты структурного представления систем согласно 
стандарту IEC~81346. Действительно, композиция двух морфизмов, 
представляющих действия по формированию некоторого аспекта структуры, 
также должна входить в~этот аспект, поскольку стандарт предписывает строить 
цепочки для идентификации объектов в~структуре системы. Кроме того, если 
объект присутствует в~аспекте, то его тождественный морфизм формально 
должен быть включен в~этот аспект. В~то же время подкатегории, 
опи\-сы\-ва\-ющие все аспекты, не обязаны образовывать в~совокупности разбиение 
категории~$C$: как показывает рис.~1, возможны как действия, входящие 
в~несколько аспектов одновременно, так и~композитные действия с~переходом 
между структурами, не входящие ни в~один аспект. Требуется лишь, чтобы 
объединение классов объектов всех этих подкатегорий совпадало 
с~$\mathrm{Ob}\,C$, поскольку не имеет смысла вводить модели, не входящие 
ни в~одну структуру.
   
   Категории можно получать из графов: любой ориентированный мультиграф 
порождает категорию, объектами в~которой служат все узлы, а морфизмами~--- 
все пути. Областью и~кообластью морфизма являются соответственно начало 
и~конец пути, композиция морфизмов действует как конкатенация путей, 
а~тождественным морфизмом узла~$a$ является пустой путь из~$a$ в~$a$, не 
содержащий ни одного ребра. Отсюда получается фундаментальное понятие  
$C$-диа\-грам\-мы~--- это функтор вида~$\Delta : X \hm\to C$, где~$X$~--- 
категория, порожденная некоторым графом и~называемая схемой диаграммы. 
Все $C$-диа\-грам\-мы образуют категорию~$\mathbf{D}C$ (ковариантная 
категория <<сверхзапятой>>~\cite{14-kov}), в~которой морфизмом 
диаграммы~$\Delta : X \hm\to C$ в~$\Xi : Y \hm\to C$ служит любая пара 
вида $\langle\gamma, fd\rangle$, состоящая из функтора~$fd : X\hm\to Y$ 
и~естественного преобразования~$\gamma : \Delta\hm\to \Xi \circ fd$; закон 
композиции морфизмов диаграмм имеет вид:
$$
\langle \gamma, fd\rangle \circ 
\langle \varphi, gd\rangle \hm = \langle \gamma_{gd(-)} \circ \varphi, fd \circ 
gd\rangle\,.
$$ 
В~тео\-рии категорий накоплен богатый арсенал алгебраических 
методов конструирования и~анализа диаграмм.
   
   Любая мегамодель задается $C$-диа\-грам\-мой, так что категорное 
представление каталогов моделей позволяет формально решать задачи 
мегамоделирования. Морфизмы диаграмм описывают структурные 
преобразования мегамоделей, выполняемые при помощи инструментов MBSE. 
Покажем, как решаются средствами теории категорий прямые задачи 
мегамоделирования. Здесь применяется одна из основных  
тео\-ре\-ти\-ко-ка\-те\-гор\-ных конструкций~--- копредел  
диаграммы~\cite{5-kov}, который строится следующим образом. Обозначим 
через~$\mathbf{1}$ категорию,\linebreak состоящую из одного объекта~0 и~одного 
морфизма~$1_0$. Из любой категории~$X$ имеется в~точ\-ности один 
функтор~$!_X : X \hm\to \mathbf{1}$, сопоставляющий объект~0  
любому~$X$-объ\-ек\-ту (иными словами, $\mathbf{1}$ является терминальным 
$\mathbf{CAT}$-объ\-ек\-том). Имеется вложение (инъективный функтор) 
$\ulcorner \mbox{--}\urcorner : C \hookrightarrow \mathbf{D}C$, сопоставляющее 
произвольному $C$-объ\-ек\-ту $Q$~точку~--- диаграмму $\ulcorner Q\urcorner : 
\mathbf{1}\hm\to  C : 0 \mapsto Q$. Коконусом (cocone) называется 
$\mathbf{D}C$-мор\-физм, имеющий точку в~качестве кообласти. Можно 
изобразить коконус $\langle \sigma, !_X\rangle : \Delta\hm\to \ulcorner 
Q\urcorner$ над диаграммой $\Delta : X\hm\to C$ в~виде диаграммы, 
<<пририсовав>> к~$\Delta$ дополнительную вершину, помеченную 
объектом~$Q$, и~набор ребер~--- стрелок, по одной для каждого узла $I\hm\in 
\mathrm{Ob}\,X$, направленной из~$I$ в~вершину и~помеченной морфизмом 
$\sigma_I : \Delta (I) \hm\to Q$. Копределом (colimit) диаграммы~$\Delta$ 
называется коконус $\mathrm{colim}\,\Delta : \Delta\hm\to \ulcorner R\urcorner$, 
универсальный в~том смысле, что для любых \mbox{$C$-объ}\-ек\-та~$T$ 
и~коконуса~$\delta : \Delta\hm\to\ulcorner T\urcorner$ существует единственный 
$C$-мор\-физм~$w : R \hm\to T$ такой, что $\delta\hm= \ulcorner w\urcorner \circ  
\mathrm{colim}\,\Delta$. Легко видеть, что это условие универсальности 
представляет собой в~точности структурный критерий из разд.~3. Таким 
образом, конструирование копредела конфигурации~$\Delta$ описывает на 
строгом математическом языке сборку системы, которой отвечает 
вершина~$R$. В~категориях типа $\mathbf{MBS}$ и~$\mathbf{Pomset}$ 
построение копредела сводится к~факторизации раздельных объединений 
объектов, представляющих компоненты системы, по отношениям 
эквивалентности, индуцированным моделями клея и~других средств сборки.
   
   Копредел любой диаграммы, если он сущест\-вует, определяется однозначно 
   с~точностью до изомор\-физма. Более того, можно описать сборку сис\-тем из 
конфигураций в~виде функтора. Пусть $Cd$~--- некоторый класс  
$C$-диа\-грамм, имеющих копределы. Он порождает полную подкатегорию 
в~$\mathbf{D}C$, из которой в~$C$ действует функтор копредела $\mathrm{colim}$, 
сопоставляя каждой диаграмме из~$Cd$~вершину некоторого ее копредела, а 
каждому \mbox{$\mathbf{D}C$-мор}\-физ\-му~$\theta : \Delta\hm\to \Xi$, 
где~$\Delta, \Xi\hm\in Cd$~--- стрелку копредела $\mathrm{colim}\,(\theta)$ такую, что 
$\mathrm{colim}\,\Xi \circ \theta \hm= \ulcorner \mathrm{colim}\,(\theta)\urcorner \circ 
\mathrm{colim}\,\Delta$.

%\begin{figure*}
\vspace*{1pt}
\begin{center}
\mbox{%
\epsfxsize=56.127mm
\epsfbox{kov-4.eps}
}
\end{center}
%\vspace*{-9pt}
%\end{figure*}

   Например, в~категории \textbf{Set} любая диаграмма имеет 
копредел~\cite[упражнение~5.1.8]{14-kov}, поэтому имеется функтор $\mathrm{colim}\, : 
\mathbf{D}(\mathbf{Set})\hm\to \mathbf{Set}$. Примечательно, что этот функтор 
является рефлектором: он сопряжен слева с~вложением $\ulcorner \mbox{--}\urcorner : 
\mathbf{Set}\hookrightarrow \mathbf{D}(\mathbf{Set})$, причем 
единица рефлексии состоит из $\mathbf{D}(\mathbf{Set})$-мор\-физ\-мов 
$\mathrm{colim}\,\Delta : \Delta\hm\to \ulcorner\mathrm{colim}\,(\Delta)\urcorner$, 
$\Delta\hm\in \mathrm{Ob}\ \mathbf{D}(\mathbf{Set})$. Напомним, что единица 
рефлексии~--- это естественное преобразование тождественного функтора 
в~композицию рефлектора и~вложения (в~данном случае, естественное 
преобразование функтора $1_{\mathbf{D}(\mathbf{Set})}$ в~$\ulcorner \mathrm{colim}\,(  
\mbox{--})\urcorner)$, состоящее из универсальных  
стрелок~\cite[разд.~4.3]{14-kov}. И~для произвольного класса~$Cd$, 
содержащего достаточное количество одноточечных диаграмм, функтор 
$\mathrm{colim}$ сопряжен слева с~ограничением  
вложения~$\ulcorner \mbox{--}\urcorner$ на подходящую полную подкатегорию 
в~$C$. А~поскольку сопряженный функтор задается однозначно с~точностью 
до изоморфизма~\cite[разд.~4.1]{14-kov}, можно сделать вывод, что сборка 
систем в~некотором смысле <<зашифрована>> в~процедуре построения 
одноточечных диаграмм~--- моделей систем как целого без раскрытия 
струк\-туры. 

Так наглядно проявляется двойственность прямых и~обратных задач 
мегамоделирования.

\section{Заключение}

   Аппарат теории категорий обладает большим потенциалом в~области 
повышения полезной отдачи от MBSE, в~том числе путем математически 
строгого решения задач мегамоделирования. Так, базовая процедура системной 
инженерии~--- сборка\linebreak
 системы из заданной конфигурации взаимо\-свя\-занных 
компонентов~--- формально описывается тео\-ретико-ка\-те\-гор\-ной 
конструкцией копредела диа\-граммы. Более сложные конструкции отвечают\linebreak 
сложным процедурам сборки, таким как связывание (weaving) общесистемных 
функций, рассеянных по всем компонентам (crosscutting concerns), например 
мониторинговых или защитных~\cite{16-kov}. Математического представления 
требуют и~другие процедуры MBSE, в~частности коллективная модификация 
мегамоделей и~составляющих моделей, восстановление конфигурации заданной 
системы, оценка взаимозаменяемости компонентов. 

Актуальны вопросы 
внедрения аппарата теории категорий в~практику, в~том числе путем развития 
программных инструментов моделирования и~мегамоделирования. Здесь 
открывается широкий спектр направлений для дальнейших исследований.
   
{\small\frenchspacing
 {%\baselineskip=10.8pt
 \addcontentsline{toc}{section}{References}
 \begin{thebibliography}{99}
\bibitem{1-kov}
Modeling and simulation-based systems engineering handbook~/
Eds.\ D.~Gianni,  A.~D'Ambrogio, A.~Tolk.~--- London: CRC Press, 2014. 513~p.
\bibitem{2-kov}
\Au{Ковалёв С.\,П., Толок~А.\,В.} Применение модельно-ори\-ен\-ти\-ро\-ван\-но\-го подхода 
в~управ\-ле\-нии жизненным циклом технических изделий~// Информационные технологии 
в~проектировании и~производстве, 2015. №\,2. С.~3--9.
\bibitem{3-kov}
\Au{Левенчук А.\,И.} Системноинженерное мышление.~--- М.: TechInvestLab, 2015. 305~с.
\bibitem{4-kov}
IEC 81346-1:2009. Industrial Systems, Installations and Equipment and Industrial Products~--- 
Structuring Principles and Reference Designations~--- Part~1: Basic Rules.~--- Geneva: ISO, 2009. 
168~p.
\bibitem{5-kov}
\Au{Ginali S., Goguen~J.} A~categorical approach to general systems~// 
 Conference (International) on Applied General Systems 
Research Proceedings~/
Ed. G.\,J.~Klir.~--- NATO conference series.~--- New York, NY, USA: Plenum 
Press, 1978. Vol.~5. P.~257--270.
\bibitem{6-kov}
\Au{Mabrok M.\,A., Ryan M.\,J.} Category theory as a~formal mathematical foundation for  
model-based systems engineering~// Appl. Math. Inform. Sci., 2017. Vol.~11. No.\,1. P.~43--51.
\bibitem{7-kov}
\Au{Ковалёв С.\,П.} Тео\-ре\-ти\-ко-ка\-те\-гор\-ный подход к~проектированию программных 
сис\-тем~// Фундаментальная и~прикладная математика, 2014. Т.~19. Вып.~3. С.~111--170.
\bibitem{8-kov}
\Au{B$\acute{\mbox{e}}$zivin J., Jouault~F., Rosenthal~P., Valduriez~P.} Modeling in the large 
and modeling in the small~// Model Driven Architecture: European MDA Workshops on 
Foundations and Applications Proceedings~/
Eds.\ U.~A{\!\ptb{\ss}}mann, M.~Aksit,  A.~Rensink.~--- 
Lecture notes in computer science ser.~--- Springer, 2005. Vol.~3599. 
P.~33--46.
\bibitem{9-kov}
\Au{Diskin Z., Kokaly~S., Maibaum~T.} Mapping-aware mega\-mod\-eling: Design patterns and 
laws~// Software Language Engineering: 6th Conference (International) Proceedings~/
Eds.\ M.~Erwig, R.\,F.~Paige, E.~Van Wyk.~--- 
Lecture notes  in computer science ser.~--- Springer, 2013. Vol.~8225. P.~322--343.
\bibitem{10-kov}
\Au{Requicha A.\,G.} Representations for rigid solids: Theory, methods, and systems~// 
ACM  Comput. Surv., 1980. Vol.~12. Iss.~4. P.~437--464.
\bibitem{11-kov}
\Au{K$\acute{\mbox{a}}$d$\acute{\mbox{a}}$r B., Pfeiffer~A., Monostori~L.} Discrete event 
simulation for supporting production planning and scheduling decisions in digital
 factories~//  37th 
CIRP Seminar (International) on Manufacturing Systems Proceedings.~--- Budapest, 2004.  
P.~444--448.
\bibitem{12-kov}
\Au{Giesa T., Spivak D.\,I., Buehler~M.\,J.} Category theory based solution for the building block 
replacement problem in materials design~// Adv. Eng. Mater., 2012. Vol.~14. 
Iss.~9. P.~810--817.
\bibitem{13-kov}
\Au{Косяков А., Свит У., Сеймур~С., Бимер~С.} Системная инженерия. Принципы 
и~практика~/ Пер. с~англ.~--- М.: ДМК-Пресс, 2014. 636~с. (\Au{Kossiakoff~A., Sweet~W.\,N., 
Seymour~S., Biemer~S.\,M.} Systems engineering principles and practice.~--- 2nd ed.~--- New 
York, NY, USA: John Wiley, 2011. 560~p.)
\bibitem{14-kov}
\Au{Маклейн С.} Категории для работающего математика~/ Пер. с~англ.~--- М.: Физматлит, 
2004. 352~с. (\Au{Mac Lane~S.} Categories for the working mathematician.~--- New York, NY, 
USA: Springer, 1978. 317~p.)
\bibitem{15-kov}
\Au{Pratt V.\,R.} Modeling concurrency with partial orders~// Int. J.~Parallel 
Prog., 1986. Vol.~15. No.\,1. P.~33--71.
\bibitem{16-kov}
\Au{Ковалёв С.\,П.} Семантика ас\-пект\-но-ори\-ен\-ти\-ро\-ван\-но\-го моделирования 
данных и~процессов~// Информатика и~её применения, 2013. Т.~7. Вып.~3. С.~70--80.
 \end{thebibliography}

 }
 }

\end{multicols}

\vspace*{-3pt}

\hfill{\small\textit{Поступила в~редакцию 16.01.17}}

%\vspace*{8pt}

\newpage

\vspace*{-30pt}

%\hrule

%\vspace*{2pt}

%\hrule

%\vspace*{8pt}


\def\tit{METHODS OF CATEGORY THEORY IN~MODEL-BASED SYSTEMS ENGINEERING\\[-7pt]}

\def\titkol{Methods of category theory in~model-based systems engineering}

\def\aut{S.\,P.~Kovalyov\\[-12pt]}

\def\autkol{S.\,P.~Kovalyov}

\titel{\tit}{\aut}{\autkol}{\titkol}

\vspace*{-14pt}


\noindent
Institute of Control Sciences, Russian Academy of Sciences, 65~Profsoyuznaya Str., 
Moscow 117997, Russian Federation



\def\leftfootline{\small{\textbf{\thepage}
\hfill INFORMATIKA I EE PRIMENENIYA~--- INFORMATICS AND
APPLICATIONS\ \ \ 2017\ \ \ volume~11\ \ \ issue\ 3}
}%
 \def\rightfootline{\small{INFORMATIKA I EE PRIMENENIYA~---
INFORMATICS AND APPLICATIONS\ \ \ 2017\ \ \ volume~11\ \ \ issue\ 3
\hfill \textbf{\thepage}}}

\vspace*{1pt}

 

\Abste{A mathematical device based on the category theory is proposed to formally describe and 
rigorously explore procedures of employing models in engineering that constitute the contents of 
model-based systems engineering (MBSE). The essence of the device consists in mathematical 
representation of assembly drawings (megamodels of systems) as diagrams in categories whose 
objects are models, and morphisms represent actions associated with assembling system models 
from component models. The soundness of the device is justified on the basis of standards that 
govern description of the systems' structure such as IEC~81346. Category-theoretical methods for 
solving a number of practical problems of assembling systems are proposed and explored. 
Examples of solving such problems are provided in categories that represent two key application 
areas for MBSE: geometric modeling of complex shapes and discrete-event simulation of the 
behavior of industrial systems.}

\KWE{ model-based systems engineering; megamodel; category theory; colimit}

\DOI{10.14357/19922264170305} 

%\vspace*{-18pt}

%\Ack
%\noindent




\vspace*{-7pt}

  \begin{multicols}{2}

\renewcommand{\bibname}{\protect\rmfamily References}
%\renewcommand{\bibname}{\large\protect\rm References}

{\small\frenchspacing
 {%\baselineskip=10.8pt
 \addcontentsline{toc}{section}{References}
 \begin{thebibliography}{99}
\bibitem{1-kov-1}
Gianni, D., A.~D'Ambrogio, and A.~Tolk, eds. 2014. \textit{Modeling and simulation-based 
systems engineering handbook}. London: CRC Press. 513~p.
\bibitem{2-kov-1}
\Aue{Kovalyov, S.\,P., and A.\,V.~Tolok.} 2015. Primenenie model'no-orientirovannogo podkhoda 
v~upravlenii zhiznennym tsiklom tekhnicheskikh izdeliy [Applying model-based approach 
to product lifecycle management].\linebreak \textit{Informatsionnye tekhnologii v~proektirovanii 
i~proizvod\-st\-ve} [Information Technologies in Design and Industry] 2(158):3--9.
\bibitem{3-kov-1}
\Aue{Levenchuk A.\,I.} 2015. 
\textit{Sistemnoinzhenernoe myshlenie} [Systems engineering thinking]. 
Moscow: TechInvestLab. 305~p.
\bibitem{4-kov-1}
IEC 81346-1:2009. 2009. 
Industrial Systems, Installations and Equipment and Industrial 
Products~--- Structuring Principles and Reference Designations~--- 
Part~1: Basic Rules. Geneva:  ISO. 168~p.
\bibitem{5-kov-1}
\Aue{Ginali, S., and J.~Goguen.} 1978. 
A~categorical approach to general systems. \textit{Conference 
(International) on Applied General Systems Research Proceedings}. Ed.\
 G.\,J.~Klir. \mbox{NATO}  conference ser. Plenum Press. 5:257--270.
\bibitem{6-kov-1}
\Aue{Mabrok, M.\,A., and M.\,J.~Ryan}. 
2017. Category theory as a~formal mathematical foundation for 
model-based systems engineering. \textit{Appl. Math.  Inform. Sci.} 11(1):43--51.
\bibitem{7-kov-1}
\Aue{Kovalyov, S.\,P.} 2016. 
Category-theoretic approach to software systems design. \textit{J.~Math. Sci.} 
214(6):814--853.
\bibitem{8-kov-1}
\Aue{B$\acute{\mbox{e}}$zivin, J., F.~Jouault, P.~Rosenthal, and P.~Valduriez.}
 2005. Modeling in 
the large and modeling in the small. 
\textit{Model Driven Architecture: European MDA Workshops on 
Foundations and Applications Proceedings.} 
Eds.\ U.~\mbox{A{\!\ptb{\ss}}mann}, M.~Aksit, and A.~Rensink. 
Lecture notes in computer science ser. Springer. 3599:33--46.
\bibitem{9-kov-1}
\Aue{Diskin, Z., S.~Kokaly, and T.~Maibaum.} 2013. 
Mapping-aware megamodeling: Design patterns 
and laws. \textit{6th Conference (International) on Software Language Engineering 
Proceedings}. Eds.\ M.~Erwig, R.\,F.~Paige, and E.~Van Wyk. 
Lecture notes in computer science ser. Springer. 
8225:322--343.
\bibitem{10-kov-1}
\Aue{Requicha, A.\,G.} 1980. Representations for rigid solids: 
Theory, methods, and systems. \textit{ACM 
Comput. Surv.} 12(4):437--464.
\bibitem{11-kov-1}
\Aue{K$\acute{\mbox{a}}$d$\acute{\mbox{a}}$r,~B., A.~Pfeiffer, and L.~Monostori.}
2004. Discrete 
event simulation for supporting production planning and scheduling decisions in 
digital factories. \textit{37th CIRP Seminar (International) on Manufacturing 
Systems Proceedings}. Budapest.  444--448.
\bibitem{12-kov-1}
\Aue{Giesa, T., D.\,I.~Spivak, and M.\,J.~Buehler.} 2012. 
Category theory based solution for the building 
block replacement problem in materials design. 
\textit{Adv. Eng. Mater.} 14(9):810--817.
\bibitem{13-kov-1}
\Aue{Kossiakoff, A., W.\,N.~Sweet, S.~Seymour, and S.\,M.~Bie\-mer.}
2011. \textit{Systems engineering 
principles and practice}. 2nd ed. New York, NY: John Wiley. 560~p.
\bibitem{14-kov-1}
\Aue{Mac Lane, S.} 1978. \textit{Categories for the working mathematician}. 
New York, NY: Springer. 317~p.
\bibitem{15-kov-1}
\Aue{Pratt, V.\,R.} 1986. Modeling concurrency with partial orders. 
\textit{Int. J.~Parallel Prog.} 15(1):33--71.
\bibitem{16-kov-1}
\Aue{Kovalyov, S.\,P.} 2013. 
Semantika aspektno-ori\-en\-ti\-ro\-van\-no\-go modelirovaniya dannykh 
i~protsessov [Semantics of aspect-oriented modeling of data and processes]. 
\textit{Informatika i~ee  Primeneniya~--- Inform. Appl.} 7(3):70--80.
\end{thebibliography}

 }
 }

\end{multicols}

\vspace*{-9pt}

\hfill{\small\textit{Received January 16, 2017}}

\vspace*{-18pt}

\Contrl

\noindent
\textbf{Kovalyov Sergey P.} (b.\ 1972)~--- Doctor of Science in physics and 
mathematics, leading scientist, Institute of Control Problems, Russian 
Academy of Sciences, 65~Profsoyuznaya Str., Moscow 117997, Russian 
Federation Federation; \mbox{kovalyov@nm.ru} 

\label{end\stat}


\renewcommand{\bibname}{\protect\rm Литература}  %5рис
\def\stat{kudr}

\def\tit{ПРИБЛИЖЕННЫЕ МЕТОДЫ РЕШЕНИЯ ЗАДАЧИ ДИАГНОСТИКИ ПЛОСКИМ 
ЗОНДОМ СИЛЬНОИОНИЗОВАННОЙ ПЛАЗМЫ С~УЧЕТОМ КУЛОНОВСКИХ 
СТОЛКНОВЕНИЙ}

\def\titkol{Приближенные методы решения задачи диагностики плоским 
зондом сильноионизованной плазмы} %с~учетом Кулоновских  столкновений}

\def\autkol{И.\,А.~Кудрявцева, А.\,В.~Пантелеев}
\def\aut{И.\,А.~Кудрявцева$^1$, А.\,В.~Пантелеев$^2$}

\titel{\tit}{\aut}{\autkol}{\titkol}

%{\renewcommand{\thefootnote}{\fnsymbol{footnote}}\footnotetext[1]
%{Работа поддержана Российским фондом фундаментальных исследований
%(проекты 11-01-00515а и 11-07-00112а), а также Министерством
%образования и науки РФ в рамках ФЦП <<Научные и
%научно-педагогические кадры инновационной России на 2009--2013~годы>>.}}


\renewcommand{\thefootnote}{\arabic{footnote}}
\footnotetext[1]{Московский авиационный институт, irina.home.mail@mail.ru}
\footnotetext[2]{Московский авиационный институт, avpanteleev@inbox.ru}

\vspace*{-2pt}

\Abst{Сформирована математическая модель, описывающая динамику сильноионизованной 
плазмы с учетом столкновений заряженных частиц вблизи плоского зонда. Модель включает уравнение 
Фоккера--Планка и уравнение Пуассона. Предложено два подхода к решению задачи: на основе метода 
статистических испытаний Мон\-те-Кар\-ло и на основе композиции метода крупных частиц и метода 
расщепления.} 

\vspace*{-2pt}

\KW{телекоммуникационные системы; метод Монте-Карло; метод крупных частиц; метод 
расщепления; зонд; уравнение Фоккера--Планка; уравнение Пуассона} 

\vspace*{-4pt}

 \vskip 8pt plus 9pt minus 6pt

      \thispagestyle{headings}

      \begin{multicols}{2}
      
            \label{st\stat}

\section{Введение}

В настоящее время в области телекоммуникаций все более востребованными становятся 
информационные технологии, основанные на использовании математических моделей и численных 
методов физики плазмы. Поэтому особенно актуальным является решение разнообразных задач анализа 
поведения плазмы, включающих в себя формирование новых моделей и методов их исследования. 
Помимо этого, в разработке телекоммуникационного оборудования эффективно используются 
собственно физические свойства плазмы. В~частности, изготовлена антенна, работа которой основана 
на газовом разряде низкотемпературной плазмы~[1], интенсивно ведутся разработки по созданию и 
усовершенствованию источников бесперебойного питания на основе плазменных элементов~[2, 3]. 
      
      Одним из наиболее перспективных направлений для построения систем оптической 
беспроводной связи является использование лазеров~\cite{4-k, 5-k}. В~этой связи большое внимание 
уделяется использованию плазмы при разработке импульсных сильноточных коммутаторов~\cite{6-k}, 
так как практическое применение подобных разработок требует повышения уровня надежности и 
быстродействия лазерных систем.
      
      Исследования низкотемпературной плазмы также связаны с разработками в области дальней 
космической связи, так как моделирование процессов взаимодействия заряженного тела с верхними 
слоями атмосферы позволяет предлагать способы улучшения существующих систем радиосвязи с 
космическими летательными аппаратами~\cite{7-k}. 
      
      Наряду с этим актуальными также являются задачи диагностики плазмы, поскольку перспективы 
ее использования в области телекоммуникаций после более полного изучения физических свойств 
могут значительно расшириться. 

Для диагностики плазмы применяют зондовые методы исследования~[8--11]. Эти методы относятся к 
классу контактных методов; как следствие, возникает сложность в исследовании пристеночной области 
вблизи зонда, которая характеризуется достаточно сложным распределением потенциала и функциями 
распределения, отличными от максвелловских. 

Данная работа посвящена исследованию переходного режима обтекания заряженного тела плазмой. Для 
переходного режима выполняется следующее условие: длина свободного пробега иона до столкновения 
с нейтральным атомом или другим ионом невелика по сравнению с характерными размерами тела. 
В~этом случае возникает необходимость учета столкновений заряженных частиц с нейтральными 
атомами и кулоновских столкновений. В~работах~\cite{10-k, 11-k} подробно рассмотрена модель с 
учетом столкновений заряженных частиц с нейтральными атомами. В~настоящей статье представлена 
теоретическая модель, описывающая влияния ион-ионных и ион-элек\-т\-рон\-ных столкновений на 
измеряемые характеристики плазмы, что ранее детально не исследовалось.
      
      В~рамках данной работы предлагается модель, описывающая динамику сильноионизованной 
плазмы с учетом кулоновских столкновений. Эта модель учитывает такие процессы взаимодействия, 
как перенос частиц и столкновения между заряженными частицами типа <<ион--ион>> и 
      <<ион--электрон>> под влиянием макроскопического электрического поля. Перечисленные 
процессы описываются самосогласованной системой уравнений, включающей уравнение 
      Фок\-ке\-ра--План\-ка и уравнение Пуассона~[12].
      
      Вычислительная модель задачи строится на основе двух методов: метода статистических 
испытаний Мон\-те-Кар\-ло и композиции метода крупных частиц и метода расщепления. Приведены 
результаты численного моделирования, полученные с использованием вышеперечисленных методов.

\vspace*{-4pt}

\section{Постановка задачи}

\vspace*{-2pt}

Рассматривается следующая физическая постановка зондовой задачи~[11]. В~невозмущенную 
бесконечно протяженную плазму, состоящую из электронов и однозарядных ионов, внесена большая\linebreak 
заряженная до потенциала $\varphi_p$ плоскость. Плоскость, расположенная поперек потока плазмы, 
является идеально поглощающей для электронов. Ионы при ударе о плоскость нейтрализуются. 
Предполагается, что частицы в плазме движутся под действием внешнего электрического поля, 
магнитное поле отсутствует. Концентрации ионов $n_{i\infty}$ и электронов $n_{e\infty}$, а также 
температуры данных час\-тиц~$T_{i\infty}$ 
и~$T_{e\infty}$ в невозмущенной плазме заданы. За начальные 
функции распределения обоих типов час\-тиц принимаются функции распределения Максвелла. 
      
      Требуется с учетом столкновений между заряженными частицами найти напряженность 
самосогласованного электрического поля $\vec{E}(\vec{r},t)$, функции распределения однозарядных 
ионов $f_i(\vec{r}, \vec{v}, t)$ и электронов $f_e(\vec{r}, \vec{v}, t)$, 
а также их моменты (плотности 
токов ионов и электронов  $j_i(\vec{r},t)\hm
=q\int f_i(\vec{r}, \vec{v}, t)\vec{v}\,d\vec{v}$, $j_e(\vec{r},t) 
\hm={\sf e}\int f_e(\vec{r},\vec{v},t)\vec{v}\,d\vec{v}$, где $q=Z_i{\sf e}$, $Z_i=1$~--- заряд иона, ${\sf 
e}$~--- заряд электрона; концентрации ионов и электронов $n_i(\vec{r},t)\hm=\int 
f_i(\vec{r},\vec{v},t)\,d\vec{v}$, $n_e(\vec{r},t)\hm=\int f_e(\vec{r},\vec{v}, t)\,d\vec{v}$). 
Поведение частиц во 
времени~$t$ характеризуется ра\-ди\-ус-век\-то\-ром~$\vec{r}$ и вектором скорости~$\vec{v}$.
      
      Математическая модель, соответствующая данной физической постановке задачи, имеет 
вид~\cite{11-k, 13-k}:

\noindent
      \begin{equation}
      \left.
      \begin{array}{c}
      \fr{\partial f_\alpha (\vec{r},\vec{v},t)}{\partial t}+
      \vec{v}\fr{\partial f_\alpha (\vec{r},\vec{v},t)}{ 
\partial \vec{r}}+
\fr{\vec{F}_\alpha(\vec{r},t)}{m_\alpha}\times{}\\[4pt]
{}\times\fr{\partial f_\alpha(\vec{r},\vec{v},t)}{ \partial 
\vec{v}}=
\left(\fr{\partial f_\alpha(\vec{r},\vec{v},t)}{ \partial t}\right)_{\mathrm{с}}+S_\alpha 
(\vec{r},\vec{v},t)\,;\\[6pt]
      \Delta\varphi(\vec{r},t)=-\fr{{\sf e}}{\varepsilon_0}\left( n_i(\vec{r},t)-n_e(\vec{r},t)\right)\,;\\[6pt]
      \vec{E}(\vec{r},t)=-\nabla \varphi(\vec{r},t)\,.
      \end{array}\!\!
      \right\}\!\!
      \label{e1-k}
      \end{equation}
Здесь первое уравнение~--- уравнение Фок\-ке\-ра--План\-ка для частиц сорта~$\alpha$ ($\alpha=i,e$), 
второе~--- уравнение Пуассона для самосогласованного электрического поля; 
$f_\alpha(\vec{r},\vec{v},t)$~--- функция\linebreak
распределения час\-тиц сорта~$\alpha$; $(\partial 
f_\alpha(\vec{r},\vec{v},t)/\partial t)_{\mathrm{с}}$~--- 
оператор столкновений Фок\-ке\-ра--План\-ка; 
функция~$S_\alpha(\vec{r},\vec{v},t)$ описывает источники или стоки\linebreak
 час\-тиц; 
$\vec{F}_\alpha(\vec{r},t)=q_\alpha\vec{E}(\vec{r},t)$, где $\vec{E}(\vec{r},t)$~--- напряженность 
самосогласованного электрического поля, 
$$
q_\alpha =
\begin{cases}
-{\sf e}\,, & \alpha=e\,,\\
{\sf e}\,, & \alpha=i\,;
\end{cases}
$$
$\varphi(\vec{r},t)$~--- потенциал самосогласованного электрического поля; $n_\alpha(\vec{r},t)$ ($\alpha 
\hm=i,e$)~--- концентрация частиц сорта~$\alpha$; $m_\alpha$~--- масса частицы сорта~$\alpha$; 
$\varepsilon_0$~--- электрическая постоянная. 

Оператор столкновений Фок\-ке\-ра--План\-ка имеет вид~\cite{13-k, 14-k}
\begin{multline*}
\fr{1}{\Gamma_\alpha}\left( \fr{\partial f_\alpha}{\partial t}\right)_{\mathrm{с}} 
=\fr{1}{2}\,\nabla_v\nabla_v:\left(f_\alpha\nabla_v\nabla_vg_\alpha(\vec{r},\vec{v},t)\right)-{}\\
{}-
\nabla_v\cdot\left(f_\alpha\nabla_v h_\alpha\right)\,,
\end{multline*}
где $\nabla_v\nabla_v g_\alpha(\vec{r},\vec{v},t)$~--- ковариантная тензорная производная второго ранга, 
знак двоеточия ($:$) обозначает операцию двойного суммирования:
\begin{gather*}
\Gamma_\alpha=\fr{Z_\alpha^4 {\sf e}^4}{4\pi \varepsilon_0^2 m^2_\alpha}\,\ln D_\alpha\,;
\\
D_\alpha =\fr{12\pi\varepsilon_0 kT_{\alpha\infty}}{Z_\alpha^2 {\sf e}^2}\left( \fr{\varepsilon_0 k 
T_{e\infty}}{n_{e\infty} {\sf e}^2}\right)^{1/2}\,;\\
g_\alpha (\vec{r},\vec{v},t)=\sum\limits_{b=i,e}\left( \fr{Z_b}{Z_\alpha}\right) \int f_b 
(\vec{r},{\vec{v}}^{\,\prime},t)\left\vert \vec{v}-{\vec{v}}^{\,\prime}\right\vert\,d\vec{v}^{\,\prime}\,;\\
h_\alpha (\vec{r},\vec{v},t)=\sum\limits_{b=i,e} \fr{m_\alpha+m_b}{m_b} 
\left(\fr{Z_b}{Z_\alpha}\right)
\int
\fr{f_b(\vec{r},{\vec{v}}^{\,\prime}, t)}{\vert \vec{v}-{\vec{v}}^{\,\prime}\vert}
\,d{\vec{v}}^{\,\prime}\,;\\
Z_\alpha =1\,, \quad \alpha=i,e\,.
\end{gather*}
 
К системе уравнений~(\ref{e1-k}) необходимо добавить начальные и краевые условия:
\begin{equation}
\!\left.
\begin{array}{rrl}
t=0:\ & f_\alpha(\vec{r},\vec{v},0)&=f_\alpha^{\mathrm{maksv}}\,,\enskip \alpha=i,e;\\[9pt]
\vec{r}\in \Omega_p:\ & f_\alpha(\vec{r},\vec{v},t)\big\vert_{\vec{r}\in\Omega_p}&=0\,,\enskip \alpha=i,e\,;\\[9pt]
&\varphi(\vec{r},t)\big\vert_{\vec{r}\in\Omega_p}&=\varphi_p\,;\\[9pt]
\vec{r}\in\Omega_\infty:\ & 
f_\alpha(\vec{r},\vec{v},t)\big\vert_{\vec{r}\in\Omega_\infty}&= %{}\\[9pt]
f_\alpha^{\mathrm{maksv}}\,,\enskip \alpha=i,e\,;\\[9pt]
&\varphi(\vec{r},t)\big\vert_{\vec{r}\in\Omega_\infty}&=0\,,
\end{array}\!\!
\right\}\!\!\!\!
\label{e2-k}
\end{equation}
    где 
    
    \noindent
    \begin{multline*}
    f_\alpha^{\mathrm{maksv}}=n_{\alpha\infty}\left(\fr{m_\alpha}{2k\pi T_{\alpha\infty}}\right)^{3/2}\times{}\\
    {}\times
    \exp\left( -
\fr{m_\alpha}{2kT_{\alpha\infty}}\left\vert\vec{v}-\vec{v}_\infty\right\vert^2\right)\,,
\enskip \alpha=i, e\,;
\end{multline*} 
$\Omega_p$ и $\Omega_\infty$~--- множество радиус-векторов час\-тиц, концы которых принадлежат плоскости зонда и 
границе возмущенной зоны соответственно.

Для решения поставленной задачи введем декартову систему координат таким образом, чтобы 
заряженная плоскость совпала с плоскостью~$0xz$. Тогда положение частицы в пространстве будет 
определяться координатами $x,y,z$, а скорость~--- координатами $v_x, v_y, v_z$. В~силу того что 
плоскость является бесконечно большой в сравнении с характерным размером задачи, функции 
распределения частиц будут зависеть только от переменных $y, v_y, t$.

Поставленную задачу предлагается решать независимо двумя методами. Первый метод основывается на 
методе статистических испытаний Мон\-те-Кар\-ло, второй метод является композицией метода 
расщепления и метода крупных частиц.

\section{Применение метода Монте-Карло}

Запишем самосогласованную систему уравнений~(\ref{e1-k}) и~(\ref{e2-k}) в декартовой системе 
координат с учетом сделанных предположений:
\begin{equation}
\left.
\begin{array}{l}
\fr{\partial f_\alpha}{\partial t}+
v_y\fr{\partial f_\alpha}{\partial y}+\fr{F_y^\alpha}{m_\alpha}\,\fr{\partial 
f_\alpha}{\partial v_y}=\fr{1}{2}\,\fr{\partial^2 }{\partial [v_y]^2}\times{}\\
{}\times \left( 
f_\alpha\fr{\partial^2 g_\alpha  }{\partial [v_y]^2}\right) -
\fr{\partial}{\partial v_y}\left( f_\alpha\fr{\partial h_\alpha}{\partial v_y}\right)\,,
\enskip \alpha=i,e\,;\\[6pt]
    \fr{\partial^2\varphi}{\partial y^2} =-\fr{{\sf e}}{\varepsilon_0}\left(n_i-n_e\right)\,;
    \enskip E_y=-
\fr{\partial\varphi}{\partial y}\,;\\[6pt]
\hspace*{3.1mm}    t=0:\  \hspace*{2.6mm}f_\alpha(y,v_y,0)=f_\alpha^{\mathrm{maksv}}\,,\ \alpha=i,e\,;\\[9pt]
\hspace*{2.9mm} y=0:\ \hspace*{2.8mm}f_\alpha(0,v_y,t)=0\,,\ \alpha=i,e\,;\\[9pt]
\hspace*{24.3mm}\varphi(0,t)=\varphi_p\,;\\[9pt]
y=y_\infty:\ f_\alpha(y_\infty, v_y, t)=f_\alpha^{\mathrm{maksv}}\,,\ \alpha=i,e\,;\\[9pt]
\hspace*{21.5mm}\varphi(y_\infty, t)=0\,.
\end{array}
\right \}
\label{e3-k}
\end{equation}

В полученной системе уравнений~(\ref{e3-k}) перейдем к безразмерным величинам, применив 
соотношение $X=M_X \hat{X}$, где $M_X$~--- масштаб размерной величины~$X$, $\hat{X}$~--- 
безразмерная величина~$X$. В~качестве используемых масштабов были взяты следующие: радиус 
Дебая, скорость теплового движения частиц, концентрация частиц в невозмущенной плазме, потенциал, 
возникающий при разделении зарядов в дебаевской сфере, и производные от них величины.

Система безразмерных уравнений имеет следующий вид:
%\noindent
\begin{equation}
\left.
\begin{array}{l}
\fr{\partial 
\hat{f}_\alpha}{\partial\hat{t}}+A_\alpha\fr{\partial\hat{f}_\alpha}{\partial\hat{y}}+
B_\alpha\hat{E}_y\fr{\partial\hat{f}_\alpha}{\partial \hat{v}_y}={}\\
\!{}=
\fr{\partial^2}{\partial[\hat{v}_y]^2}\left(D_\alpha 
\hat{f}_\alpha\right)-\fr{\partial}{\partial\hat{v}_y}\left(K_\alpha \hat{f}_\alpha\right),\enskip 
\alpha=i,e;\\[9pt]
\fr{\partial^2\hat{\varphi}}{\partial\hat{y}^2}=-\left(\hat{n}_i-\hat{n}_e\right)\,;\enskip \hat{e}_y=-
\fr{\partial\hat\varphi}{\partial\hat{y}}\,;\\[9pt]
\hspace*{3.1mm}\hat{t}=0:\ \hspace*{2.6mm}\hat{f}_\alpha(\hat{y},\hat{v}_y,0)=\hat{f}_\alpha^{\mathrm{maksv}}\,,\enskip \alpha-i,e\,;\\[9pt]
\hspace*{2.9mm}\hat{y}=0:\ \hspace*{2.8mm}\hat{f}_\alpha(0,\hat{v}_y,\hat{t})=0\,,\enskip \alpha=i,e\,;\\[9pt]
\hspace*{24.3mm}\hat\varphi(0,\hat{t})=\hat{\varphi}_p\,;\\[9pt]
\hat{y}=\hat{y}_\infty:\ \hat{f}_\alpha(\hat{y}_\infty, \hat{v}_y, \hat{t})=\hat{f}^{\mathrm{maksv}}_\alpha\,,\enskip 
\alpha=i,e\,;\\[9pt]
\hspace*{21.5mm}\hat\varphi(\hat{y}_\infty,\hat{t})=0\,.
\end{array}
\right\}
\label{e4-k}
\end{equation}
Здесь 

\vspace*{-2pt}

\noindent
\begin{gather*}
A_\alpha=\sqrt{\delta_\alpha }\,\hat{v}_y\,;\enskip 
B_\alpha=\sqrt{\delta_\alpha}\,\fr{z_\alpha}{2\varepsilon_\alpha}\,;\\
\delta_\alpha=\fr{\varepsilon_\alpha}{\mu_\alpha}\,;\enskip 
\varepsilon_\alpha=\fr{T_{\alpha\infty}}{T_{i\infty}}\,;\\
\mu_\alpha=\fr{m_\alpha}{m_i}\,;\enskip 
D_\alpha=A_g^\alpha\fr{\partial^2\hat{g}_\alpha}{\partial  [\hat{v}_y]^2}\,;\\
K_\alpha=A_h^\alpha \fr{\partial \hat{h}_\alpha}{\partial \hat{v}_y}\,,\enskip \alpha=i,e\,,
\end{gather*}
где $A_g^\alpha$ и $A_h^\alpha$~--- коэффициенты, определяемые характерными параметрами 
задачи~\cite{15-k}.

Поиск решения самосогласованной системы уравнений~(\ref{e4-k}) осуществляется по следующей 
схе-\linebreak ме. Вначале находятся значения напряженности\linebreak
 электрического поля по значениям потенциала, 
полученным из граничной задачи для уравнения Пуассона. Далее, используя найденные значения 
напряженности, решается уравнение Фок\-ке\-ра--План\-ка путем перехода к стохастическому 
дифференциальному уравнению (СДУ) Ито:

\noindent
\begin{multline*}
d\Theta_\alpha(\hat{t}) = a_\alpha \left(\hat{t},\Theta_\alpha(\hat{t})\right)+{}\\
{}+\sigma\left(
\hat{t},\Theta_\alpha(\hat{t})\right)\,dW(\hat{t})\,,\quad \alpha=i,e\,,
%\label{e5-k}
\end{multline*}
где 

\noindent
\begin{align*}
\Theta_\alpha(\hat{t})&=\begin{bmatrix}
\hat{y}(\hat{t})\\ \hat{v}_y(\hat{t})
\end{bmatrix}\,;\\
a_\alpha\left(\hat{t},\Theta_\alpha(\hat{t})\right)&=\begin{bmatrix}
-A_\alpha\\ -K_\alpha -B_\alpha \hat{E}_y
\end{bmatrix}\,;\\
\sigma_\alpha\left(\hat{t},\Theta_\alpha(\hat{t})\right)\sigma_\alpha^{\mathrm{T}}\left( 
\hat{t},\Theta_\alpha(\hat{t})\right)&=D_\alpha\,,\enskip \alpha=i,e\,;
\end{align*} 
$W(\hat{t})$~--- стандартный винеровский случайный процесс.
\pagebreak

Для нахождения значений вектора состояния~$\Theta_\alpha(\hat{t})$ применим явную разностную 
схему стохастического метода Эйлера~\cite{16-k}:
\begin{multline*}
\Theta_\alpha^{n+1}=\Theta_\alpha^n +h_\tau a_\alpha \left( \hat{t}_n, \Theta_\alpha^n\right)+\sigma_\alpha 
\left( \hat{t}_n, \Theta_\alpha^n\right)\Delta W_n\,,\\ 
n=0,\ldots , N\,,\ \alpha=i,e\,,
%\label{e6-k}
\end{multline*}
где $\Theta_\alpha^n$, $n=0,\ldots , N$,~--- приближенное значение вектора 
состояния~$\Theta_\alpha(\hat{t})$, $\alpha=i,e$, в момент времени $\hat{t}\hm=\hat{t}_n$, 
$\hat{t}_n\hm=n h_\tau$, $n=0,\ldots , N$; $h_\tau$~--- достаточно малый шаг интегрирования; $\Delta 
W_n$, $n=0,\ldots ,N$,~--- величина приращения винеровского процесса~$W(\hat{t})$ на отрезке $\left[ 
\hat{t}_n,\,\hat{t}_{n+1}\right]$, по определению независимая от~$\Theta_\alpha^0$, 
$\Delta W_0,\ldots , 
\Delta W_{n-1}$: $\Delta W_n\hm=W(\hat{t}_{n-1})\hm-W(\hat{t}_n)$; $\Delta W_n\hm\sim N(0,\,h_\tau)$, 
т.\,е.\ $\Delta W_n$ представляют собой гауссовские случайные величины с нулевыми математическими 
ожиданиями и дисперсиями, равными шагу интегрирования; $\Theta_\alpha^0$~--- значение вектора 
состояния $\Theta_\alpha(\hat{t})$, $\alpha\hm=i,e$, в момент времени $\hat{t}=0$, 
$\Theta_\alpha^0\hm\sim \hat{f}_\alpha^{\mathrm{maksv}}$. 

Частные производные $\partial^2\hat{g}_\alpha/\partial[\hat{v}_y]^2$ и $\partial \hat{h}_\alpha/\partial 
\hat{v}_y$, являющиеся составляющими матрицы $\sigma_\alpha (\hat{t}_n, 
\Theta_\alpha^n)\sigma_\alpha^{\mathrm{T}}(\hat{t}_n,\Theta_\alpha^n)$ и вектора $a_\alpha(\hat{t}_n, 
\Theta_\alpha^n)$ соответственно, аппроксимируются со вторым порядком точности на трехточечном 
шаблоне на основе значений~$\hat{g}_\alpha$ и~$\hat{h}_\alpha$~\cite{17-k}.
      
      В выражения для функций~$\hat{g}_\alpha$ и~$\hat{h}_\alpha$ входят интегралы, которые 
вычисляются методом Мон\-те-Кар\-ло с использованием набора значений скоростной компоненты 
вектора состояния~$\hat{v}_y$, полученных из решения СДУ Ито:
      \begin{equation*}
      \int \hat{f}_\alpha \left\vert \hat{v}_y-
\hat{v}_y^\prime\right\vert\,dv_y^\prime=M\left(\zeta\left(\hat{V}_y\right)\right)\,,
\end{equation*}
где
$$
      \zeta\left(\hat{V}_y\right)=\left\vert \hat{v}_y-\hat{V}_y\right\vert\,,\enskip \hat{V}_y\sim 
\hat{f}_\alpha\,.
  $$
      
      Для вычисления напряженности самосогласованного электрического поля $\hat{E}_y=-
\partial\hat{\varphi}/\partial\hat{y}$, входящей в вектор $a_\alpha(\hat{t}_n, \Theta_\alpha^n)$, необходимо 
аналогично аппроксимировать со вторым порядком точности производную 
$\partial\hat{\varphi}/\partial\hat{y}$ на трехточечном шаблоне с использованием значений 
потенциала~$\hat{\varphi}$~\cite{17-k}. Значения потенциала~$\hat\varphi$ находятся из решения 
уравнения Пуассона. 
      
      Граничную задачу для уравнения Пуассона 
      \begin{align*}
      \fr{\partial^2 \hat\varphi}{\partial \hat{y}^2} & = -\left(\hat{n}_i-\hat{n}_e\right)\,;\\
      \hat{\varphi}\big|_{\hat{y}=0} &=\hat{\varphi}_p\,;\\
      \hat{\varphi}\big|_{\hat{y}_\infty=0} &=0
      \end{align*}
    предлагается решать путем перехода к конечно-разностной системе с последующим ее решением 
методом прогонки~\cite{17-k}:

\noindent
\begin{gather*}
\hat{\varphi}^n_{l-1}+2\hat{\varphi}_l^n+\hat{\varphi}^n_{l+1}=
h_y\hat{\delta}_l^n\,,\enskip l=1,\ldots , 
N_y\,;\\
\hat{\delta}_l^n=-\left( \hat{n}^n_{i,l}-\hat{n}^n_{e,l}\right)\,;\enskip 
\hat{\varphi}_0=\hat{\varphi}_p\,;\enskip \hat{\varphi}_{N_y}=0\,,
\end{gather*}
где $N_y$~--- число шагов по переменной~$\hat{y}$, $h_y$~--- величина шагов разбиения по~$\hat{y}$. 
      
      Концентрации $\hat{n}_\alpha$, $\alpha=i,e$, и плотности токов частиц на зонд~$\hat{f}_\alpha$, 
$\alpha=i,e$, вычисляются согласно описанному выше методу Мон\-те-Карло.

\section{Применение метода расщепления и~метода крупных~частиц}

Решение задачи в данном случае предлагается начать с записи правой части уравнения 
Фок\-ке\-ра--План\-ка в декартовой системе координат в виде:
$$
\mathbf{Q} f_\alpha = \fr{1}{2}\,\fr{\partial^2 f_\alpha}{\partial [v_y]^2}\,\fr{\partial^2 g_\alpha}{\partial 
[v_y]^2}+\fr{\partial f_\alpha}{\partial v_y}\,\fr{\partial C_\alpha}{\partial v_y}+H_\alpha\,,\enskip 
\alpha=i,e\,,
$$  
где 
\begin{align*}
C_\alpha(\vec{r},\vec{v},t)&=
\begin{cases}
\fr{1-\gamma}{Z_i^2}\int\fr{f_e(\vec{r},{\vec{v}}^{\,\prime},t)}{|\vec{v}-{\vec{v}}^{\,\prime} |}\,d{\vec{v}}^{\,\prime}\,, 
&\alpha=i\,;\\[9pt]
\fr{Z_i^2(\gamma-1)}{\gamma}\int \fr{f_i(\vec{r},{\vec{v}}^{\,\prime}, t)}
{|\vec{v}-{\vec{v}}^{\,\prime} 
|}\,d{\vec{v}}^{\,\prime}\,, &\alpha=e\,;
\end{cases} 
\\
H_\alpha&=
\begin{cases}
4\pi \left( \fr{\gamma f_e}{Z_i^2}+f_i\right)f_i\,, & \alpha=i\,;\\[9pt]
4\pi\left(\fr{Z_i^2 f_i}{\gamma}+f_e\right)f_e\,, &\alpha=e\,.
\end{cases}
\end{align*}
Тогда при переходе к безразмерным величинам (см.\ разд.~3) система~(\ref{e1-k}) запишется 
следующим образом:
      \begin{equation}
      \left.
\!\!\begin{array}{l}
      \fr{\partial 
\hat{f}_\alpha}{\partial\hat{t}}+A_\alpha\fr{\partial\hat{f}_\alpha}{\partial\hat{y}}+
B_\alpha  \hat{E}_y
\fr{\partial\hat{f}_\alpha}{\partial\hat{v}_\alpha}=\tilde{\mathbf{Q}}\hat{f}_\alpha\,,\enskip 
\alpha=i,e;\\[9pt]
      \fr{\partial^2\hat{\varphi}}{\partial\hat{y}^2}=-\left( \hat{n}_i-\hat{n}_e\right)\,,\enskip \hat{E}_y=-
\fr{\partial\hat\varphi}{\partial\hat{y}}\,,\\[9pt]
\hspace*{3.1mm}\hat{t}=0:\ \hspace*{2.6mm}\hat{f}_\alpha(\hat{y},\hat{v}_y, 0)=\hat{f}_\alpha^{\mathrm{maksv}}\,,\enskip \alpha=i,e\,,\\[9pt]
\hspace*{2.9mm} \hat{y}=0:\ \hspace*{2.8mm}\hat{f}_\alpha(0,\hat{v}_y,\hat{t})=0\,,\enskip \alpha=i,e\,;\\[9pt]
\hspace*{24.3mm}\hat\varphi(0,\hat{t})=\hat{\varphi}_p\,;\\[9pt]
      \hat{y}=\hat{y}_\infty:\ \hat{f}_\alpha(\hat{y}_\infty, 
\hat{v}_y,\hat{t})=\hat{f}_\alpha^{\mathrm{maksv}}\,,\enskip \alpha=i,e\,;\\[9pt]
\hspace*{21.5mm}\hat{\varphi}(\hat{y}_\infty,\hat{t})=0\,,\\[9pt]
    \end{array}
\right\}\!\!
\label{e7-k}
\end{equation}
где 
\begin{gather*}
\tilde{\mathbf{Q}} \hat{f}_\alpha=D_\alpha\fr{\partial^2\hat{f}_\alpha}{\partial 
[\hat{v}_y]^2}+K_\alpha\fr{\partial\hat{f}_\alpha}{\partial\hat{v}_y}+H_\alpha\,;\\
D_\alpha=A_g^\alpha\fr{\partial^2\hat{g}_\alpha}{\partial [\hat{v}_y]^2}\,;\enskip 
K_\alpha=A_h^\alpha \fr{\partial \hat{h}_\alpha}{\partial\hat{v}_y}\,,\ \alpha=i,e\,.
\end{gather*}

Для решения системы уравнений~(\ref{e7-k}) применяется модификация метода 
расщепления~\cite{17-k}, согласно которой исходная задача разбивается на две вспомогательные. Такое 
разбиение можно осуществить, переписав уравнение Фок\-ке\-ра--План\-ка в следующем виде:
$$
\fr{\partial\hat{f}_\alpha}{\partial\hat{t}} =
\tilde{\mathbf{Q}}_1\hat{f}_\alpha+\tilde{\mathbf{Q}}_2\hat{f}_\alpha\,,
$$
где 
\begin{align*}
\tilde{\mathbf{Q}}_1\hat{f}_\alpha &=-
\left(A_\alpha\fr{\partial\hat{f}_\alpha}{\partial\hat{y}}+
B_\alpha\fr{\partial\hat{f}_\alpha}{\partial\hat{y}}
\right)\,;\\
\tilde{\mathbf{Q}}_2\hat{f}_\alpha 
&=\left(D_\alpha\fr{\partial^2\hat{f}_\alpha}{\partial[\hat{v}_y]^2}+K_\alpha\fr{\partial 
\hat{f}_\alpha}{\partial\hat{v}_y}+H_\alpha\right)\,.
\end{align*}

      Правая часть уравнения Фок\-ке\-ра--План\-ка представляет собой сумму двух операторов, 
первый из которых отвечает за перенос частиц, второй~--- за столкновения заряженных частиц. 
В~результате образуются следующие задачи, которые решаются последовательно:
      \begin{itemize}
\item первая задача:
\begin{align*}
&\fr{\partial w_\alpha(\hat{y},\hat{v}_y,\hat{t})}{\partial\hat{t}} =\mathbf{Q}_1 
w_\alpha(\hat{y},\hat{v}_y,\hat{t})\,,\enskip \alpha=i,e\,;\\[9pt]
&\fr{\partial^2\hat\varphi}{\partial\hat{y}^2}=-\left(\hat{n}_i-\hat{n}_e\right)\,;\enskip
\hat{E}_y=-
\fr{\partial\hat\varphi}{\partial\hat{y}}\,;\\[9pt]
&w_\alpha(\hat{y},\hat{v}_y,\hat{t}^n)=\hat{f}_\alpha(\hat{y},\hat{v}_y,\hat{t}^n)\,,\enskip n=0,\ldots ,N-
1\,;\\[9pt]
&\hspace{2.9mm}\hat{y}=0:\ \hspace*{2.9mm}w_\alpha(0,\hat{v}_y,\hat{t})=0\,,\enskip \alpha=i,e\,;\\[9pt]
&\hspace*{25.1mm}\hat\varphi(0,\hat{t})=\hat{\varphi}_p\,;\\[9pt]
&\hat{y}=\hat{y}_\infty:\ w_\alpha(\hat{y}_\infty, \hat{v}_y, \hat{t})=
\hat{f}_\alpha^{\mathrm{maksv}}\,,\enskip 
\alpha=i,e\,;\\[9pt]
&\hspace*{22.5mm}\hat\varphi(\hat{y}_\infty,\hat{t})=0\,;
\end{align*}
\item вторая задача:
\begin{align*}
\!\!\!\!\!\!\!\fr{\partial s_\alpha(\hat{y},\hat{v}_y,\hat{t})}{\partial \hat{t}} &=\mathbf{Q}_2 
s_\alpha(\hat{y},\hat{v}_y,\hat{t})\,, & \alpha&=i,e\,;\\
\!\!\!\!\!\!\!s_\alpha (\hat{y},\hat{v}_y,\hat{t}^n) &=w_\alpha (\hat{y},\hat{v}_y, \hat{t}^{n+1}),& n&=0,\ldots ,N-
1.
\end{align*}
\end{itemize}

Первая задача представляет собой систему безразмерных уравнений Вла\-со\-ва--Пуас\-со\-на. Для ее 
решения применяется метод крупных частиц~\cite{18-k}. Согласно этому методу решение задачи 
осуществляется путем расщепления на два этапа: на первом этапе не учитываются конвективные члены 
и решение получается обычным интегрированием на неподвижной эйлеровой сетке, а на втором этапе 
рассматривается система, которая описывает перенос частиц в лагранжевой системе координат. Кроме 
того, на первом этапе необходимо решить уравнение Пуассона для получения значений потенциала 
самосогласованного электрического поля. Для этого применяется метод, описанный в разд.~3. 

Вторая задача решается путем перехода к ко\-неч\-но-раз\-ност\-ной сис\-те\-ме. При этом частные 
производные $\partial^2\hat{g}_\alpha/\partial[\hat{v}_y]^2$ и $\partial\hat{h}_\alpha/\partial\hat{v}_y$ 
аппроксимируются со вторым порядком точности с использованием трехточечного шаблона, а 
производная $\partial s_\alpha/\partial\hat{t}$ аппроксимируется на двухточечном шаблоне с первым 
порядком точности~\cite{16-k}. К~полученной системе разностных уравнений предлагается применить 
один из классических методов решения систем линейных уравнений, например метод 
Гаусса~\cite{19-k}.
      
      Решением первой задачи является функция $w_\alpha(\hat{y}, \hat{v}_y, \hat{t}^n)$, 
$n\hm=0,\ldots ,N$, , которая дает начальное условие для второй задачи. Решая вторую задачу, находим 
функцию $s_\alpha(\hat{y},\hat{v}_y,\hat{t}^n)\hm=\hat{f}_\alpha(\hat{y},\hat{v}_y,\hat{t}^n)$, 
$n=1,\ldots ,N$, $\alpha=i,e$, которая определяет решение $\hat{f}_\alpha(\hat{y},\hat{v}_y,\hat{t}^n)$, 
$\alpha=i,e$, исходной системы~(\ref{e7-k}) для рассматриваемых моментов времени $n=1,\ldots ,N$.

Моменты функций распределения $\hat{f}_\alpha$, $\alpha=i,e$, находятся с помощью методов 
численного интегрирования, например метода трапеций~\cite{19-k}.

\section{Результаты численного моделирования}

Для двух описанных выше методов реализованы две отдельные программы в среде {Matlab~7.0}. 
Эти программы позволяют по заданным значениям концентраций и температур частиц $n_{i\infty}$, 
$n_{e\infty}$, $T_{i\infty}$ и~$T_{e\infty}$ в невозмущенной плазме, а также потенциала~$\varphi_p$, 
подаваемого на зонд, изучить эволюцию во времени плотностей тока частиц~$j_i$ и~$j_e$, концентраций 
частиц~$n_i$  и~$n_e$ в произвольной точке пространства в возмущенной зоне, а также динамику 
изменения напряженности~$E_y$ самосогласованного электрического поля во времени и пространстве.

С использованием разработанных программ проведены серии расчетных экспериментов, в которых 
значение концентраций варьировалось в пределах $n_{i\infty} \hm = n_{e\infty}\hm =10^{18}\div 
10^{22}$~м$^{-3}$. Значение температур было выбрано неизменным и равным $T_{i\infty}\hm = 
T_{e\infty}\hm=3000$~K, а значения потенциала, подаваемого на зонд, изменялись в пределах 
$\varphi_p\hm=0\div 2{,}6$~В.

На рис.~1  и~2 приведены графики изменения напряженности самосогласованного электрического
 поля (см.\ рис.~1) и плотности токов ионов (см.\linebreak\vspace*{-12pt}

\pagebreak

\end{multicols}

\begin{figure} %fig1
\vspace*{1pt}
\begin{center}
\mbox{%
\epsfxsize=162.594mm
\epsfbox{kud-1.eps}
}
\end{center}
\vspace*{-9pt}
\Caption{Динамика изменения плотности тока ионов во времени в фиксированной точке возмущенной 
зоны для значений потенциала: \textit{1}~--- $\varphi_p=-6$; 
\textit{2}~--- $\varphi_p=-16$; \textit{3}~--- $\varphi_p=- 30$ 
в случае применения методов Монте-Карло~(\textit{а}) 
и крупных частиц~(\textit{б})}
\end{figure}

\begin{figure} %fig2
\vspace*{1pt}
\begin{center}
\mbox{%
\epsfxsize=162.713mm
\epsfbox{kud-2.eps}
}
\end{center}
\vspace*{-9pt}
\Caption{Динамика изменения напряженности электрического поля во времени в фиксированной точке 
возмущенной зоны для значений потенциала: 
\textit{1}~--- $\varphi_p=-6$; \textit{2}~--- $\varphi_p=-16$; 
\textit{3}~--- $\varphi_p=-30$ в случае применения методов Монте-Карло~(\textit{а}) и
крупных частиц~(\textit{б})
}
\end{figure}

\begin{multicols}{2}

\noindent
 рис.~2) во времени в фиксированной точке пространства 
возмущенной зоны в случае применения обоих разработанных алгоритмов.


На основании полученных результатов можно отметить похожее поведение зависимостей 
напряженности электрического поля и плотности тока от времени в двух рассматриваемых случаях. 
Графики кривых сначала убывают, затем начинают возрастать, выходя в некоторый момент 
времени~$t^\prime$ (момент установления) на стационарные значения. 

Одинаковое поведение 
напряженности и плот\-ности тока можно объяснить из следующих соображений: плотность тока ионов в 
данной области пространства равна произведению концентрации ионов на их направленную скорость и 
на заряд иона. Скорость ионов, в свою очередь, зависит от заряда, массы и напряженности 
электрического поля. 
%\columnbreak

При внесении в плазму отрицательно заряженного зонда возникает электрическое поле, которое 
нарушает квазинейтральность плазмы. Для того чтобы компенсировать действие внешнего 
электрического поля, ионы устремляются к зонду, а электроны~--- от зонда. Это приводит к дисбалансу 
концентраций вблизи зонда и, как следствие, к увеличению разности потенциалов; график 
напряженности электрического поля убывает. Вскоре разделение зарядов компенсирует внешнее 
электрическое поле; график выходит на стационарное значение. 

Также можно отметить, что значения 
напряженности электрического поля и плотности тока частиц на зонд в момент установления для двух 
методов совпадают. 

Момент установления~$t^\prime$ зависит от при\-ме\-ня\-емо\-го метода решения. В~случае метода 
Мон\-те-Кар\-ло $t^\prime=3{,}5\div 4$~ед., а для метода крупных частиц совместно с методом 
расщепления $t^\prime\hm=5\div 5{,}5$~ед. Используя ко\-неч\-но-раз\-ност\-ный метод, можно 
получить динамику изменения функций распределения частиц~$f_\alpha$, $\alpha=i,e$, во времени и 
пространстве. Функции распределения позволяют наглядно представить влияние на картину 
распределения частиц вблизи зонда самой поверхности зонда и электрического поля.

\section{Заключение}
      
      В работе найдено решение задачи диагностики плоским зондом сильноионизованной плазмы с 
учетом столкновений заряженных частиц. Разработана математическая модель исследуемого явления, 
описываемая уравнениями Фок\-ке\-ра--План\-ка и Пуассона. Решение получено двумя методами:\linebreak 
статистическим и ко\-неч\-но-раз\-ност\-ным на основе\linebreak сформированных алгоритмов. Приведены 
резуль-\linebreak таты численного моделирования при различных\linebreak характерных параметрах задачи.
 Из  проведенных 
вычислительных экспериментов вытекает, что искомые величины: напряженность 
электрического поля, плотности токов частиц на зонд, концентрации частиц вблизи зонда~--- как по 
характеру зависимости, так и по числовым значениям совпадают. При применении метода 
      Мон\-те-Кар\-ло момент установления наступает быстрее по сравнению с конечно-разностным 
методом, однако конечно-разностный метод позволяет получить более наглядные результаты.

{\small\frenchspacing
{%\baselineskip=10.8pt
\addcontentsline{toc}{section}{Литература}
\begin{thebibliography}{99}

\bibitem{1-k}
\Au{Alexeff I., Anderson T.}
Experimental and theoretical results with plasma antenna~// IEEE Trans. Plasma Sci., 2006. Vol.~34. 
No.\,2. P.~166--172.

\bibitem{2-k}
\Au{Сысун В.\,И.}
Сильноионизованная низкотемпературная плазма в приборах электронной техники: Методы 
исследования, свойства, применение. Дисс. \ldots д-ра физ.-мат. наук в форме науч. докл.: 
01.04.08.~--- Пет\-ро\-за\-водск, 1996.

\bibitem{3-k}
\Au{Тухас В.\,А.}
Методология создания средств измерений и испытаний на устойчивость к кондуктивным помехам~// 
Мат-лы VI Междунар. симп. по электромагнитной совместимости и 
электромагнитной экологии.~--- СПб., 2005. С.~231--234.

\bibitem{4-k}
\Au{Гудзенко Л.\,И., Яковленко С.\,И.}
Плазменные лазеры.~--- М.: Атомиздат, 1978.  256~с.

\bibitem{5-k}
\Au{Звелто О.}
Принципы лазеров.~--- М.: Мир, 1990.  560~с.

\bibitem{6-k}
\Au{Сысун В.\,И., Хромой Ю.\,Д.}
Расширение канала мощного импульсного разряда в парах ртути~// Электронная техника, 1974. 
Сер.~4. Вып.~10. С.~80--85. 

\bibitem{7-k}
\Au{Винклер Дж.\,Р.}
Искусственные пучки частиц в космической плазме.~--- М.: Мир, 1985.  451~с.

\bibitem{8-k}
\Au{Bernstein I.\,B., Rabinowitz I.\,N.}
Theory of electrostatic probes in low-density plasma~// Phys. Fluids, 1959. Vol.~2. No.\,2. P.~112--121. 

\bibitem{9-k}
\Au{Альперт Я.\,Л., Гуревич А.\,В., Питаевский~Л.\,П.}
Искусственные спутники в разреженной плазме.~--- М.: Наука, 1964.  282~с.

\bibitem{10-k}
\Au{Чан П., Тэлбот Л., Турян~К.}
Электрические зонды в неподвижной и движущейся плазме.~--- М.: Мир, 1978.  202~с.

\bibitem{11-k}
\Au{Алексеев Б.\,В., Котельников В.\,А.}
Зондовый метод диагностики плазмы.~--- М.: Энергоатомиздат, 1989.  240~с.

\bibitem{12-k}
\Au{Пантелеев А.\,В., Кудрявцева И.\,А.}
Формирование математической модели двухкомпонентной плазмы с учетом столкновений 
заряженных частиц в случае плоского зонда~// Теоретические вопросы вычислительной техники и 
программного обеспечения: Межвузовский сб. научн. тр.~--- М.: МИРЭА, 2006. С.~11--21.

\bibitem{13-k}
\Au{Олдер Б.}
Вычислительные методы в физике плазмы.~--- М.: Мир, 1974.  111~с.

\bibitem{14-k}
\Au{Montgomery D.\,C., Tidman D.\,A.}
Plasma kinetic theory.~--- New York, 1964. 

\bibitem{15-k}
\Au{Кудрявцева И.\,А., Пантелеев А.\,В.}
Применение метода Мон\-те-Кар\-ло для анализа поведения двухкомпонентной плазмы с учетом 
столкновений между заряженными частицами~// Теоретические вопросы\linebreak
вычислительной техники и 
программного обеспечения: Межвузовский сб. научн. тр.~--- М.: МИРЭА, 2008. С.~122--128. 

\bibitem{16-k}
\Au{Семенов В.\,В., Пантелеев А.\,В., Руденко~Е.\,А., Бор\-та\-ков\-ский~А.\,С.}
Методы описания, анализа и синтеза нелинейных систем управления.~--- М.: МАИ, 1993.  312~с.

\bibitem{17-k}
\Au{Киреев В.\,И., Пантелеев А.\,В.}
Численные методы в примерах и задачах.~--- М.: Высшая школа, 2006.  480~с.

\bibitem{18-k}
\Au{Белоцерковский О.\,М., Давыдов~Ю.\,М.}
Метод крупных частиц в газовой динамике. Вычислительный эксперимент.~--- М.: Наука, 
Физматгиз, 1982.

\label{end\stat}

\bibitem{19-k}
\Au{Вержбицкий В.\,М.}
Основы численных методов.~--- М.: Высшая школа, 2002.  840~с.
 \end{thebibliography}
}
}


\end{multicols}        %6+
\def\stat{gorshenin-one}

\def\tit{КОНЦЕПЦИЯ ОНЛАЙН-КОМПЛЕКСА ДЛЯ СТОХАСТИЧЕСКОГО МОДЕЛИРОВАНИЯ РЕАЛЬНЫХ ПРОЦЕССОВ$^*$}

\def\titkol{Концепция онлайн-комплекса для стохастического моделирования реальных процессов}

\def\aut{А.\,К.~Горшенин$^1$}

\def\autkol{А.\,К.~Горшенин}

\titel{\tit}{\aut}{\autkol}{\titkol}

{\renewcommand{\thefootnote}{\fnsymbol{footnote}} \footnotetext[1]
{Работа выполнена при поддержке
РФФИ (проект 15-37-20851 мол\_а\_вед).}}


\renewcommand{\thefootnote}{\arabic{footnote}}
\footnotetext[1]{Институт проб\-лем информатики Федерального исследовательского центра 
<<Информатика и~управ\-ле\-ние>> Российской академии наук; 
%Федеральное государственное бюджетное образовательное учреждение высшего образования 
Московский технологический университет (МИРЭА),
 agorshenin@frccsс.ru}


\Abst{Анализ информационных потоков с~использованием разнообразных вероятностных 
моделей достаточно широко распространен в~различных прикладных областях. 
В~статье описаны основные принципы построения новой он\-лайн-сис\-те\-мы 
стохастического моделирования реальных процессов, не имеющей прямых аналогов 
в~силу универсальности предлагаемого набора методов, а также выбранной концепции 
реализации в~виде ин\-тер\-нет-ре\-сур\-са, который позволит конечному пользователю 
не заботиться о соответствии технических характеристик своего компьютера ка\-ким-ли\-бо 
специальным требованиям, а сразу загружать данные на сервер и~обрабатывать их.}

\KW{смеси вероятностных распределений; метод скользящего разделения смесей; 
интеллектуальный анализ данных; он\-лайн-ком\-плекс; матричные вычисления}

\DOI{10.14357/19922264160107} %



\vskip 14pt plus 9pt minus 6pt

\thispagestyle{headings}

\begin{multicols}{2}

\label{st\stat}


\section{Введение}

При практическом решении задачи моделирования и~анализа нестационарных 
информационных потоков в~современных информационных, телекоммуникационных 
и~вычислительных системах и~сетях ключевым этапом является разделение смесей 
вероятностных распределений, т.\,е.\ статистическое оценивание неизвестных параметров. 
При этом в~качестве базовых аппроксимирующих моделей в~рамках так называемого 
метода скользящего разделения смесей (СРС-ме\-тод)~\cite{Korolev2011} могут 
рассматриваться как конечные сдвиг-мас\-штаб\-ные смеси нормальных 
и~гам\-ма-рас\-пре\-де\-ле\-ний~\cite{Korolev2011}, так и~непрерывные распределения, 
например дис\-пер\-си\-он\-но-сдви\-го\-вые смеси нормальных законов~\cite{Korolev2015}. 
Основная идея СРС-ме\-то\-да заключается в~специальном разбиении исходной выборки 
на подвыборки (окна) и~дальнейшем анализе поведения данных на каж\-дом окне. 
Это позволяет отслеживать появление и~исчезновение формирующих структур 
в~изучаемых процессах во времени.

Можно привести широкий спектр примеров успешного использования
вероятностных моделей для реальных данных различной природы.
Например, статьи~[3--5] предлагают анализ финансовых и~физических
рядов, а работа~\cite{Hamedi2015} ориентирована на анализ
биомедицинских сигналов. В статье~\cite{Abanto2010} проводится
анализ данных фондового индекса \verb"S&P500" на основе симметричных
масштабных смесей нормальных распределений, а работа~\cite{Wang2011}
посвящена изучению курса обмена австралийского доллара 
с~использованием представлений распределений Стьюдента в~виде
масштабной смеси нормальных законов с~применением специальной
программы \verb"WinBUGS" для проведения вычислений.

Однако зачастую методы реализованы в~виде программ для специализированных 
пакетов (в частности, \verb"MATLAB"), что накладывает ряд ограничений 
на удобство распространения и~использования.  Во-пер\-вых, конечный пользователь 
должен иметь доступ к~дорогостоящему программному обеспечению и~опыт работы с~ним. 
Во-вто\-рых, далеко не все алгоритмы могут быть запущены на любой физической 
архитектуре. К~таковым можно отнести, например, решения, предназначенные для 
выполнения вычислений с~по\-мощью  \verb"CUDA" (Compute Unified Device
Architecture).  У~конечного пользователя может 
оказаться графическая карта от другого производителя либо его  \verb"GPU"  (graphics processing
unit) не предназначены для осуществления вычислений на должном уровне. В-третьих, 
могут возникнуть различные тонкости в~использовании разработанных третьими 
лицами программ, например связанные с~лицензиями. 

Все это приводит к~идее 
создания специализированного сервиса с~он\-лайн-до\-сту\-пом, который поможет 
преодолеть указанные трудности.

В настоящей статье рассмотрим требования, которым должна удовлетворять такая 
система.\linebreak В~частности, будут описаны функциональные возможности специализированного 
комплекса, разработанного автором на языке программирования пакета \verb"MATLAB", 
который выступает прообразом функциональных возможностей, предназначенных для 
реализации в~рамках сервиса с~он\-лайн-до\-сту\-пом. Кроме того, значительное внимание 
уделяется получению явных формул для ряда моментных характеристик, в~том числе 
в~матричном представлении, для повышения производительности вычислений.

\section{Матричные представления некоторых моментных характеристик конечных 
смесей нормальных законов}

Пусть случайная величина~$Z_t$ имеет функцию распределения $F_Z(x,t)$, представимую 
в~виде конечной сдвиг-мас\-штаб\-ной смеси нормальных законов, а именно:
\begin{multline}
F_Z(x,t)={}\\
{}=\sum\limits_{i=1}^{k(t)}
\fr{p_i(t)}{\sigma_i(t)\sqrt{2\pi}}\int\limits_{-\infty}^{x}\exp
\left\{-\fr{(y-a_i(t))^2}
{2\sigma_i^2(t)}\right\}\,dy, 
\\
\forall x \in \mathbb{R}\,,\quad a_i(t)\in\mathbb{R}\,, \enskip \sigma_i(t)>0\,,\enskip 
i=1,\ldots,k(t)\,,\\ \sum\limits_{i=1}^{k(t)}p_{i}(t)=1\,, \enskip p_{i}(t)\geqslant 0\,.
\label{Mixture}
\end{multline}

Параметр $t$ здесь обозначает время (номер окна) для СРС-ме\-то\-да 
и~подчеркивает тот факт, что распределение изменяется для 
каждого положения окна (возможно, весьма существенным образом). 
В~частности, параметр $k(t)$, описывающий число компонент в~смеси 
вида~\eqref{Mixture}, при аппроксимации\linebreak реальных данных может принимать 
различные значения с~течением времени (см., например, 
\mbox{статьи}~\cite{Gorshenin2008,Gorshenin2012}, при этом вопрос статистической 
значимости дополнительных компонент обсуждается в~работе~\cite{Gorshenin2011b}). 
Это обстоятельство значительно затрудняет прогнозирование, так как появление 
новых компонент в~процессе зачастую объясняется новыми факторами, которые 
отсутствовали при построении первоначальной модели и,~следовательно, не могли быть 
учтены. Поэтому необходимо перейти к~рассмотрению некоторых <<интегральных>> 
характеристик, которые можно рассчитать для любого распределения вида~\eqref{Mixture}, 
независимо от конкретных значений параметров.

В качестве таких величин можно использовать моменты различных порядков, 
а~также производные от них коэффициенты асимметрии и~эксцесса. Ниже получим 
соответствующие выражения для случайной величины с~функцией распределения~\eqref{Mixture} 
для первых четырех моментов.

Известно, что для начальных моментов случайной величины~$X$ с~нормальным распределением 
с~параметрами~$a$ и~$\sigma^2$ (т.\,е.\ $X\sim N(a,\sigma^2)$) справедливы 
следующие соотношения:
\begin{equation}
\mathbb{E} X^m=\begin{cases}
a^2+\sigma^2, & m=2\,;\\
a^3+3a\sigma^2, & m=3\,;\\
a^4+6a^2\sigma^2+3\sigma^4, & m=4\,.
\end{cases}\label{NonCentral}
\end{equation}

Для начальных моментов случайной величины~$Z_t$ с~функцией распределения, 
задаваемой формулой~\eqref{Mixture}, по определению имеем:
\begin{multline*}
\mathbb{E} Z_t^m=\int\limits_{-\infty}^{+\infty} z^m\,dF_Z(z,t)=
\sum\limits_{i=1}^{k(t)} p_i(t)\fr{1}{\sigma_i(t)\sqrt{2\pi}}\times{}\\
{}\times 
\int\limits_{-\infty}^{+\infty} z^m\exp\left\{-\fr{(z-a_i(t))^2}
{2\sigma_i^2(t)}\right\} \,dz={}\\
{}=\sum\limits_{i=1}^{k(t)} p_i(t)\mathbb{E} X_i^m\,, \quad 
X_i\sim N(a_i(t),\sigma_i^2(t))\,.
\end{multline*}

Тогда для начальных моментов случайной величины~$Z_t$ справедлив следующий 
аналог выражений~\eqref{NonCentral}:
\begin{equation}
\mathbb{E} Z_t^m=\begin{cases}
\displaystyle\sum\limits_{i=1}^{k(t)} p_i(t) a_i(t), &\hspace*{-15mm} m=1;\\
\displaystyle\sum\limits_{i=1}^{k(t)} p_i(t)\left(a_i^2(t)+\sigma_i^2(t)\right), &\hspace*{-15mm} m=2;\\
\displaystyle\sum\limits_{i=1}^{k(t)} p_i(t)\left(a_i^3(t)+3a_i\sigma_i^2(t)\right), &\hspace*{-15mm} m=3;\\
\displaystyle\sum\limits_{i=1}^{k(t)} p_i(t)\left(a_i^4(t)+6a_i^2\sigma_i^2(t)+3\sigma_i^4(t)
\right), &\\ 
& \hspace*{-15mm}m=4.
\end{cases}\!\!\! \!\!\label{NonCentralZ}
\end{equation}

Воспользуемся этими выражениями для получения явных формул для дисперсии, 
а~так\-же коэффициентов асимметрии и~эксцесса, зависящих только от параметров 
распределения, а~именно: величин $p_i(t)$, $a_i(t)$ и~$\sigma_i(t)$.

Сразу отметим, что математическое ожидание случайной величины~$Z_t$, согласно 
первой строке в~формуле~\eqref{NonCentralZ}, имеет вид:
\begin{equation}
\mathbb{E} Z_t=\sum\limits_{i=1}^{k(t)} p_i(t) a_i(t)\,. 
\label{Expectation}
\end{equation}

\subsection{Дисперсия}

Дисперсия случайной величины~$Z_t$ с~функцией распределения $F_Z(x,t)$, 
определяемой выражением~\eqref{Mixture}, имеет вид:
\begin{equation}
\mathbb{D} Z_t=\mathbb{E} Z_t^2-\left(\mathbb{E} Z_t\right)^2\,. 
\label{Var}
\end{equation}

Пользуясь выражениями~\eqref{NonCentralZ} и~проводя необходимые 
преобразования, получим:
\begin{multline}
\mathbb{D} Z_t=\int\limits_{-\infty}^{+\infty} z^2\,dF_Z(z,t)-\left(
\sum\limits_{i=1}^{k(t)} p_i(t) a_i(t)\right)^{\!2}={}
\\
{}=\sum\limits_{i=1}^{k(t)} \fr{p_i(t)}{\sigma_i(t)\sqrt{2\pi}} 
\int\limits_{-\infty}^{+\infty} z^2\exp\left\{-\fr{(z-a_i(t))^{\!2}}
{2\sigma_i^2(t)}\right\} \,dz-{}\\
{}-\left(\sum_{i=1}^{k(t)} p_i(t) a_i(t)\right)^2={}\\
{}=\sum\limits_{i=1}^{k(t)} p_i(t) a_i^2(t)-\left(\sum\limits_{i=1}^{k(t)} 
p_i(t) a_i(t)\right)^{\!\!2}\!+\sum\limits_{i=1}^{k(t)} p_i(t) \sigma_i^2(t) ={}\\
{}=\sum\limits_{i=1}^{k(t)} p_i(t) \left(a_i(t)-\sum\limits_{i=1}^{k(t)} 
p_i(t) a_i(t)\right)^2+{}\\
{}+\sum_{i=1}^{k(t)} p_i(t) \sigma_i^2(t)\,. 
\label{Variance}
\end{multline}

Заключительное представление в~формуле~\eqref{Variance} легко получить 
путем раскрытия квадрата в~первой сумме и~приведения подобных слагаемых 
(например, именно в~таком виде дисперсия приводится в~книге~\cite{Korolev2011}).

\subsection{Коэффициент асимметрии}

Общий вид коэффициента асимметрии случайной величины~$Z_t$ с~функцией 
распределения $F_Z(x,t)$~\eqref{Mixture} задается следующим выражением:
\begin{equation*}
\gamma_{Z,\,t}=\fr{\mathbb{E} \left(Z_t-\mathbb{E} 
Z_t\right)^3}{\left(\mathbb{D} Z_t\right)^{3/2}}\,.
\end{equation*}

Выпишем отдельно выражение для числителя дроби, указанной выше:
\begin{multline*}
\mathbb{E} \left(Z_t-\mathbb{E} Z_t\right)^3={}\\
{}=
\mathbb{E} Z_t^3-3\mathbb{E} Z_t\cdot\mathbb{E} Z_t^2+3\left(
\mathbb{E} Z_t\right)^3-\left(\mathbb{E} Z_t\right)^3={}\\
{}=\mathbb{E} Z_t^3-3\mathbb{E} Z_t\cdot\left(\mathbb{E} Z_t^2-\left(
\mathbb{E} Z_t\right)^2\right)-\left(\mathbb{E} Z_t\right)^3={}\\
{}=\mathbb{E} Z_t^3-3\mathbb{E} Z_t\cdot\mathbb{D} Z_t-\left(\mathbb{E} Z_t\right)^3\,.
\end{multline*}

Тогда коэффициент асимметрии может быть представлен в~виде:
\begin{equation}
\gamma_{Z,\,t}=\fr{\mathbb{E} Z_t^3-3\mathbb{E} 
Z_t\cdot\mathbb{D} Z_t-\left(\mathbb{E} Z_t\right)^3}
{\left(\mathbb{D} Z_t\right)^{3/2}}\,.
\label{Skew}
\end{equation}

Воспользуемся формулами~\eqref{NonCentral} и~\eqref{Variance}. Имеем:
\begin{multline*}
\mathbb{E} \left(Z_t-\mathbb{E} Z_t\right)^3={}\\
{}=
\sum\limits_{i=1}^{k(t)} \fr{p_i(t)}{\sigma_i(t)\sqrt{2\pi}} 
\int\limits_{-\infty}^{+\infty} z^3\exp\left\{-\fr{(z-a_i(t))^2}
{2\sigma_i^2(t)}\right\} \,dz-{}\\
{}-
3\mathbb{E} Z_t\cdot\mathbb{D} Z_t-\left(\mathbb{E} Z_t\right)^3={}\\
{}=\sum\limits_{i=1}^{k(t)} p_i(t)\left(a_i^3(t)+3a_i\sigma_i^2(t)\right)-3\mathbb{E} Z_t\cdot\mathbb{D} Z_t-\left(\mathbb{E} Z_t\right)^3=\\
{}=\sum\limits_{i=1}^{k(t)} p_i(t)\left(a_i^3(t)+3a_i\sigma_i^2(t)\right)-
\left(\sum\limits_{i=1}^{k(t)} p_i(t) a_i(t)\right)\times{}\\
{}\times\left(3\sum\limits_{i=1}^{k(t)} p_i(t) \left(a_i(t)-
\sum\limits_{i=1}^{k(t)} p_i(t) a_i(t)\right)^2+{}\right.\\
\left.{}+
3\sum\limits_{i=1}^{k(t)} p_i(t) \sigma_i^2(t)-\left(
\sum\limits_{i=1}^{k(t)} p_i(t) a_i(t)\right)^2\right)\,.
\end{multline*}

Окончательно получим следующее выражение для коэффициента асимметрии:
\begin{multline}
\gamma_{Z,\,t}={}\\
{}=\left[
\vphantom{\left(a_i(t)-
\sum\limits_{i=1}^{k(t)} p_i(t) a_i(t)\right)^2}
\sum\limits_{i=1}^{k(t)} p_i(t)\left(
a_i^3(t)+3a_i\sigma_i^2(t)\!\right)-\left(\sum\limits_{i=1}^{k(t)} 
p_i(t) a_i(t)\right)\times{}\right.\hspace*{-3.48pt}\\
{}\times\left(3\sum\limits_{i=1}^{k(t)} p_i(t) \left(a_i(t)-
\sum\limits_{i=1}^{k(t)} p_i(t) a_i(t)\right)^2+{}\right.\\
\left.\left.{}+
3\sum\limits_{i=1}^{k(t)} p_i(t) \sigma_i^2(t)-\left(
\sum\limits_{i=1}^{k(t)} p_i(t) a_i(t)\right)^2\right)\right]\times{}\\
{}\times\left[\sum\limits_{i=1}^{k(t)} p_i(t) \left(a_i(t)-
\sum\limits_{i=1}^{k(t)} p_i(t) a_i(t)\right)^2+{}\right.\\
\left.{}+
\sum\limits_{i=1}^{k(t)} p_i(t) \sigma_i^2(t)
\vphantom{\sum\limits_{i=1}^{k(t)} p_i(t) \left(a_i(t)-
\sum\limits_{i=1}^{k(t)} p_i(t) a_i(t)\right)^2}
\right]^{-3/2}\,.
\label{Skewness}
\end{multline}

\subsection{Коэффициент эксцесса}

Общий вид коэффициента эксцесса случайной величины~$Z_t$ с~функцией 
распределения $F_Z(x,t)$, определяемой выражением~\eqref{Mixture}, 
задается следу\-ющим выражением:
\begin{equation*}
\kappa_{Z,\,t}=\fr{\mathbb{E} \left(Z_t-\mathbb{E} Z_t\right)^4}
{\left(\mathbb{D} Z_t\right)^2}-3\,.
\end{equation*}

Выпишем выражения для числителя в~указанной выше дроби через известные величины:
\begin{multline*}
\mathbb{E} \left(Z_t-\mathbb{E} Z_t\right)^4=
\mathbb{E} \left(Z_t^4-4Z_t^3\cdot\mathbb{E} Z_t+{}\right.\\
\left.{}+6Z_t^2\cdot
\left(\mathbb{E} Z_t\right)^2-4Z_t\cdot \left(\mathbb{E} Z_t\right)^3+
\left(\mathbb{E} Z_t\right)^4\right)={}\\
{}=\mathbb{E} Z_t^4-3\left(\mathbb{E} Z_t\right)^4-
4\mathbb{E} Z_t\cdot \mathbb{E} Z_t^3+
6\left(\mathbb{E} Z_t\right)^2\cdot \mathbb{E} Z_t^2\,.
\end{multline*}

Тогда коэффициент эксцесса может быть представлен в~виде:
\begin{multline}
\kappa_{Z,\,t}=\left(\mathbb{E} Z_t^4-4\mathbb{E} Z_t\cdot 
\mathbb{E} Z_t^3+6\left(\mathbb{E} Z_t\right)^2\cdot 
\mathbb{E} Z_t^2-{}\right.\\
\left.{}-3\left(\mathbb{E} Z_t\right)^4\right)\Big/
\left(\mathbb{D} Z_t\right)^2-3\,.
\label{Kurt}
\end{multline}

Воспользуемся формулами~\eqref{NonCentral} и~\eqref{Expectation}. Имеем:
\begin{multline*}
\mathbb{E} \left(Z_t-\mathbb{E} Z_t\right)^4={}\\
{}=
\sum\limits_{i=1}^{k(t)} \fr{p_i(t)}{\sigma_i(t)\sqrt{2\pi}} %\times{}\\
%{}\times
\int\limits_{-\infty}^{+\infty} z^4\exp\left\{-\fr{(z-a_i(t))^2}
{2\sigma_i^2(t)}\right\} \,dz-{}\\
{}-
3\left(\sum\limits_{i=1}^{k(t)} p_i(t) a_i(t)\right)^4-
4\left(\sum\limits_{i=1}^{k(t)} p_i(t) a_i(t)\right) \times{}\\
{}\times
\left(\sum\limits_{i=1}^{k(t)} p_i(t)\left(a_i^3(t)+
3a_i\sigma_i^2(t)\right)\right)+{}\\
{}+6\left(\sum\limits_{i=1}^{k(t)} p_i(t) a_i(t)\right)^2\left(
\sum\limits_{i=1}^{k(t)} p_i(t) \left(a_i^2(t) + \sigma_i^2(t)\right)\right)={}\\
{}=\sum\limits_{i=1}^{k(t)} p_i(t)\left(a_i^4(t)+
6a_i^2\sigma_i^2(t)+3\sigma_i^4(t)\right)-{}\\
{}-4\left(\sum\limits_{i=1}^{k(t)} p_i(t) a_i(t)\right) \left(
\sum\limits_{i=1}^{k(t)} p_i(t)\left(a_i^3(t)+3a_i\sigma_i^2(t)\right)\right)+{}\\
{}+6\left(\sum\limits_{i=1}^{k(t)} p_i(t) a_i(t)\right)^2
\left(\sum\limits_{i=1}^{k(t)} p_i(t) \left(a_i^2(t) + \sigma_i^2(t)\right)\right)-{}\\
{}-
3\left(\sum\limits_{i=1}^{k(t)} p_i(t) a_i(t)\right)^4\,.
\end{multline*}

Окончательно получим следующее выражение для коэффициента эксцесса:
\begin{multline}
\kappa_{Z,\,t}=\left[
\vphantom{\left(a_i(t)-
\sum\limits_{i=1}^{k(t)} p_i(t) a_i(t)\right)^2}
\sum\limits_{i=1}^{k(t)} p_i(t)\left(
a_i^4(t)+6a_i^2\sigma_i^2(t)+3\sigma_i^4(t)\right)-{}\right.\\
{}-3\left(
\sum\limits_{i=1}^{k(t)} p_i(t) a_i(t)\right)^{\!4}-
4\left(\sum\limits_{i=1}^{k(t)} p_i(t) a_i(t)\right) \times{}\\
{}\times
\left(\sum\limits_{i=1}^{k(t)} p_i(t)\left(a_i^3(t)+
3a_i\sigma_i^2(t)\right)\right)+{}\\
\left.{}+6\left(\sum\limits_{i=1}^{k(t)} p_i(t) a_i(t)\right)^{\!\!2}\left(
\sum\limits_{i=1}^{k(t)} p_i(t) \left(a_i^2(t) + 
\sigma_i^2(t)\right)\right)\!\right]\times{}\\
{}\times\left[\sum\limits_{i=1}^{k(t)} p_i(t) \left(a_i(t)-
\sum\limits_{i=1}^{k(t)} p_i(t) a_i(t)\right)^{\!2}+{}\right.\\
\left.{}+
\sum\limits_{i=1}^{k(t)} p_i(t) \sigma_i^2(t)
\vphantom{\left(a_i(t)-
\sum\limits_{i=1}^{k(t)} p_i(t) a_i(t)\right)^2}
\right]^{-2}-3\,.
\label{Kurtosis}
\end{multline}

Стоит отметить, что формулы \eqref{Expectation}, 
\eqref{Variance}, \eqref{Skewness} и~\eqref{Kurtosis} могут быть использованы 
для непосредственного определения моментных характеристик по величинам $p_i(t)$, 
$a_i(t)$ и~$\sigma_i(t)$ и~не требуют знания значения момента предыдущего порядка. 
Для упрощения вычислений, в~том числе и~с~точки зрения программирования, при 
<<последовательном>> поиске этих характеристик проще воспользоваться 
формулами~\eqref{Var}, \eqref{Skew} и~\eqref{Kurt}. Более того, многие 
системы оптимизированы для выполнения матричных вычислений. Поэтому разумно 
воспользоваться представлением выражений~\eqref{NonCentralZ} в~виде
\begin{equation*}
\mathbb{E} Z_t^m=\begin{cases}
P_t\cdot A_t^T, & \hspace*{-15mm}m=1;\\
P_t\left(D_{A,\,t}\cdot A_t^T + D_{\Sigma,\,t} \cdot \Sigma_t^T\right), & \hspace*{-15mm}m=2;\\
P_t\cdot D_{A,\,t}\left(D_{A,\,t}\cdot A_t^T+3\cdot D_{\Sigma,\,t}\cdot\Sigma_t^T\right),\\
 & \hspace*{-15mm}m=3;\\
P_t\left( D_{A,\,t}^3\cdot A_t^T +6\cdot D_{\Sigma,\,t}^2\cdot D_{A,\,t}\cdot A_t^T+{}\right.&\\
\hspace*{15mm}\left.{}+3\cdot D_{\Sigma,\,t}^3\cdot\Sigma_t^T\right), & \hspace*{-15mm}m=4,
\end{cases}
%\label{NonCentralZMatrix}
\end{equation*}
где
\begin{gather*}
P_t=\left(p_1,\ldots,p_{k(t)}\right)\,;\enskip
A_t=\left(a_1,\ldots,a_{k(t)}\right)\,;\\
\Sigma_t=\left(\sigma_1,\ldots,\sigma_{k(t)}\right)\,; \\
D_{A,\,t}=\mathrm{diag}\left\{a_1,\ldots,a_{k(t)}\right\}\,;\\
D_{\Sigma,\,t}=\mathrm{diag}\left\{\sigma_1,\ldots,\sigma_{k(t)}\right\},
\end{gather*}
а обозначение $\mathrm{diag}\{\ldots\}$ использовано для записи диагональных матриц 
с~указанными элементами. Получаемое преимущество в~скорости в~рамках подобного 
подхода количественно оценивается в~работе~\cite{Gorshenin2015a} на 
примере классического EM-ал\-го\-ритма  (expectation-maximization).

\vspace*{-6pt}

\section{Реализация метода скользящего разделения смесей
на~встроенном языке программирования пакета MATLAB}

Возможности СРС-метода были реализованы в~рамках единого программного 
комплекса на встроенном языке программирования пакета \verb"MATLAB". 
Для удобства работы был создан оконный пользовательский интерфейс, 
с~по\-мощью которого задаются параметры вычислительных методов, определяется 
диапазон графического вывода для сохранения и~т.\,д. Данный комплекс 
объединил в~себе различные варианты EM-ал\-го\-рит\-мов, включая и~реализацию 
матричных вычислений на \verb"GPU", для различных типов смесей вероятностных 
распределений.



В начале работы пользователю предлагается ввес\-ти имя (путь) 
для файла с~данными. Может быть выбран любой диапазон внутри ряда, 
при этом по умолчанию предлагаются все значения от первого до последнего 
(с~автоматическим определением длины ряда при указании параметра <<{\sf End}>>). 
В~случае ввода пользователем некорректных значений (например, отрицательных или 
превышающих объем выборки) программа возвращает настройки к~предустановленным. 
Вид начального окна пред\-став\-лен на рис.~1.



Затем пользователь может выбрать отображение исходных данных или разностей, 
вывести гистограммы для этих рядов с~автоматической аппрок-\linebreak\vspace*{-12pt}

 \noindent
 \begin{center}  %fig1
 \vspace*{9pt}
\mbox{%
 \epsfxsize=78mm
 \epsfbox{go1-1.eps}
 }



\vspace*{3pt}

\noindent
{{\figurename~1}\ \ \small{Начальное окно программы}}
\end{center} 
% \vspace*{19pt}

 \noindent
 \begin{center}  %fig2
 \vspace*{1pt}
\mbox{%
 \epsfxsize=78mm
 \epsfbox{go1-2.eps}
 }



\vspace*{3pt}

\noindent
{{\figurename~2}\ \ \small{Настройки параметров СРС-метода}}
\end{center} 

\vspace*{6pt}

 
 \addtocounter{figure}{2}


\noindent
симацией конечной 
смесью вероятностных распределений. После этого предлагается возможность 
запуска СРС-ме\-то\-да как для исходной выборки, 
так и~для разностей. Настройка параметров СРС-ме\-то\-да осуществляется 
с~по\-мощью нового диалогового окна, в~котором задается название вычислительного 
алгоритма, размер подвыборки, максимальное число компонент в~аппроксимирующей 
смеси, точность приближений и~диапазон для сдвига (можно выбрать любую часть 
внут\-ри выборки; если заданные пользователем значения некорректны, расчет 
производится от первого элемента выборки до $N\hm-\mathrm{width}\hm+1$, где $N$~--- длина 
всей выборки для анализа, а~width~--- ширина окна). Для каждого из полей 
предусмотрены значения по умолчанию. На рис.~2 приводится 
англоязычный вариант интерфейса с~аббревиатурой MSM 
(от {moving separation of mixtures}).



После нажатия на кнопку <<\verb"ОК">> и~запуска шагов алгоритма 
отображается окно с~индикатором выполнения и~названием выбранного 
вычислительного метода, позволяющего следить за текущим положением окна 
и~состоянием работы программы. После завершения вычислений появляется график 
с~динамической и~диффузионной компонентами волатильности (подробнее см.\ 
в~\cite{Korolev2011}), а~так\-же диалоговое окно, с~по\-мощью 
которого можно настроить диапазон вывода параметров для сохранения, включая 
и~диапазон для окон. После закрытия данного диалогового окна график автоматически 
сохраняется в~формате PNG 
на диск в~папку с~программой. Пример графического 
вывода пред\-став\-лен на рис.~\ref{FigLimits}.

\begin{figure*} %fig3
\vspace*{1pt}
 \begin{center}
 \mbox{%
 \epsfxsize=160mm
 \epsfbox{go1-3.eps}
 }
 \end{center}
 \vspace*{-9pt}
\Caption{Пример графического вывода c окном изменения диапазонов по каждой из осей
\label{FigLimits}}
\vspace*{6pt}
\end{figure*}
\begin{figure*}[b] %fig4
\vspace*{6pt}
 \begin{center}
 \mbox{%
 \epsfxsize=143.08mm
 \epsfbox{go1-4.eps}
 }
 \end{center}
 \vspace*{-9pt}
\Caption{Пример изображения эволюции квантилей различных уровней (от~0,05 до~0,95)
\label{FigQuantiles}}
\end{figure*}

Данный функционал реализуется с~по\-мощью модуля <<Ядро СРС-ме\-то\-да>>
(свидетельство государственной регистрации программ для ЭВМ
№\,2015618673). Он используется с~целью унификации для
пользователя работы с~любым алгоритмом, при этом с~точки зрения
разработчика возможно простое включение в~состав любых новых методов
без изменения пользовательского опыта. Найденные оценки сохраняются
в~процессе расчетов, а~так\-же по окончании работы программы, что
позволяет избежать потери результатов при возникновении прерывания.
Непосредственно в~ядро встроены различные модификации EM-ал\-го\-рит\-ма
(классический, сглаженный, основанный на повторных вычислениях на
окне и~т.\,п.).

Кроме того, в~состав комплекса включен модуль визуализации моментных
характеристик и~квантилей (свидетельство государственной регистрации
программ для ЭВМ №\,2015618564), формулы для которого были
рассмотрены в~предыдущем разделе. Модуль содержит функции для
отыскания математического ожидания, дисперсии, коэффициента
асимметрии и~эксцесса, а также квантилей различного уровня для
конечных смесей нормальных законов. Указанные величины вычисляются
для каждого положения окна в~СРС-ме\-то\-де, 
а~затем выводятся на графиках: четыре моментные характеристики 
в~разных осях на одном листе; эволюция квантилей различного уровня
изображается на отдельном графике в~трехмерном пространстве 
(рис.~\ref{FigQuantiles}). Для заданного набора окон (параметр одной
из функций) предусмотрен вывод приближения гистограммы
аппроксимирующими кривыми с~заданием расположения набора графиков на
лис\-те. С~привлечением возможностей данного модуля были созданы
графики, например в~работе~\cite{Gorshenin2015b}.

\vspace*{-6pt}

\section{Вычислительный комплекс с~онлайн-доступом}

В предыдущем разделе был коротко описан программный комплекс, 
реализованный средствами программирования пакета \verb"MATLAB", функциональные 
возможности которого должны стать базой для он\-лайн-сер\-виса.

Реализация широкого спектра методов интеллектуального анализа данных, основанного 
на идеологии смесей вероятностных распределений (не только конечных нормальных, 
как было рас\-смот\-ре\-но выше) будет представлена в~он\-лайн-сис\-те\-ме на основе 
инструментов, предлагаемых в~пользовательских профилях (с~реализацией  
механизмов регистрации и~авторизации). Пользователи смогут загружать 
на сервер данные, а полученные после обработки модели будут сохраняться 
в~их профиле, что позволит избежать повторной загрузки одних и~тех же рядов 
(возможно, значительного объема) и~сделает возможным многократное дальнейшее 
использование результатов обработки.

Из-за высокой вычислительной сложности в~ряде ситуаций может 
потребоваться значительное время для корректной аппроксимации данных 
выбранной моделью. В~этой ситуации предполагается оповещение исследователей 
о~статусе процесса с~по\-мощью электронных каналов связи. В~случае завершения 
расчетов пользователь сможет в~любое удобное время обратиться 
к~результатам анализа как в~числовом (с~по\-мощью экспорта оцененных параметров), 
так и~в~графическом виде. При этом будет доступна работа с~несколькими рядами 
одновременно, в~том числе в~разных режимах обработки.






Основным элементом интерфейса он\-лайн-сис\-те\-мы является область для
отображения исходных временн$\acute{\mbox{ы}}$х рядов и~визуального представления
результатов СРС-ме\-то\-да, включая вывод
нескольких графиков одновременно (динамической и~диффузионной
компоненты волатильности, моментных характеристик и~др.).

 \noindent
 \begin{center}  %fig5
 \vspace*{-1pt}
\mbox{%
 \epsfxsize=78mm
 \epsfbox{go1-5.eps}
 }



\vspace*{3pt}

\noindent
{{\figurename~5}\ \ \small{Схема архитектуры комплекса}}
\end{center} 

 \vspace*{6pt}

Описание архитектуры, предлагаемой для решения перечисленных задач, 
включая организацию потоков данных в~системе, приведено в~работе~\cite{Gorshenin2015c}. 
Рассмотрим ключевые элементы, представленные на рис.~5.


Первый уровень представляет персональный компьютер (ПК) пользователя, 
обеспечивающий ему доступ к~интерфейсу он\-лайн-ком\-плек\-са для загрузки 
данных, настройки параметров и~получения результатов (в~том числе в~визуальной 
форме). Frontend-сер\-вер предназначен для реализации интерфейса взаимодействия 
между пользователем\linebreak и~вычислительными узлами системы, при этом 
непосредст\-венная обработка данных на нем не производится. Backend-сер\-вер 
осуществляет взаимодействие между Frontend-сер\-ве\-ром, рабочими серверами 
и~серверами баз данных (БД), в~част\-ности готовит данные для обработки 
и~распределяет задачи по вычислительным компонентам системы.

Наибольшая нагрузка по передаче данных ложится на взаимодействие 
Frontend- и~Backend-сер\-ве\-ров, поэтому необходима разработка механизмов 
ускорения их работы (в частности, за счет кэширования). Взаимодействие Backend- 
и~рабочих серверов должно осуществляться с~по\-мощью специализированных API 
(application programming interface) для 
реализации возможностей, связанных с~параллельной обработкой данных. После проведения 
вычислений для одного положения окна СРС-ме\-то\-да данные сохраняются на сервере БД 
и~пересылаются с~помощью Frontend-сер\-ве\-ра пользователю для возможного контроля 
процесса выполнения анализа с~его стороны.

Подобная архитектура позволит интегрировать в~систему различные методы 
интеллектуального анализа данных, при этом их реализация может быть 
основана на специальных программных  и~аппаратных подходах, но для 
конечного пользователя особенности решений будут скрыты.

\section{Заключение}

Анализ информационных потоков с~использованием разнообразных 
вероятностных моделей широко распространен в~различных прикладных областях, 
что подчеркивает актуальность создания описыва\-емой в~статье он\-лайн-сис\-те\-мы. 
При этом следует отметить отсутствие непосредственных аналогов сервиса из-за 
универсальности предлагаемого набора методов и~выбранной концепции, позволяющих 
конечному пользователю не заботиться о соответствии ресурсов своего компьютера 
ка\-ким-ли\-бо специальным техническим требованиям, а сразу загружать данные на 
сервер, обрабатывать их с~по\-мощью различных методов интеллектуального анализа 
и~получать результат. Данная система сможет предложить широкие возможности 
для различных групп исследователей во всем мире.

\bigskip

{Автор выражает признательность д.\,ф.-м.\,н., профессору 
Виктору Юрьевичу Королеву за полезные обсуждения, а также Виктору Кузьмину 
за плодотворное участие в~разработке архитектуры комплекса.}

{\small\frenchspacing
 {%\baselineskip=10.8pt
 \addcontentsline{toc}{section}{References}
 \begin{thebibliography}{99}
\bibitem{Korolev2011} 
\Au{Королев~В.\,Ю.}
Ве\-ро\-ят\-но\-ст\-но-ста\-ти\-сти\-че\-ские методы декомпозиции волатильности
хаотических процессов.~--- М.: Изд-во МГУ, 2011. 512~с.

\bibitem{Korolev2015} 
\Au{Королев~В.\,Ю., Корчагин~А.\,Ю., Горшенин~А.\,К.} 
Некоторые свойства дис\-пер\-си\-он\-но-сдви\-го\-вых смесей нормальных законов~// 
Статистические методы оценивания и~проверки гипотез, 2015. Вып.~26. С.~134--153.

\bibitem{Gorshenin2008} 
\Au{Горшенин~А.\,К., Королев~В.\,Ю.,
Турсунбаев~А.\,М.} Медианные  модификации EM- и~SEM-ал\-го\-рит\-мов
для разделения смесей вероятностных распределений и~их
применение к~декомпозиции волатильности финансовых временных
рядов~// Информатика и~её применения, 2008. Т.~2. Вып.~4. C.~12--47.

\bibitem{Gorshenin2011a}
\Au{Горшенин~А.\,К., Королев~В.\,Ю.,
Малахов~Д.\,В., Скворцова~Н.\,Н.} Анализ тонкой стохастической
структуры хаотических процессов с~по\-мощью ядерных оценок~//
Математическое моделирование, 2011. Т.~23. №\,4. C.~83--89.

\bibitem{Gorshenin2012}
\Au{Горшенин~А.\,К., Королев~В.\,Ю., Малахов~Д.\,В., Скворцова~Н.\,Н.}
Об исследовании плазменной турбулентности на основе анализа спектров~// Компьютерные
исследования и~моделирование, 2012. Т.~4. №\,4. С.~793--802.

\bibitem{Hamedi2015} 
\Au{Hamedi~M., Salleh~S.-H., Chee-Ming~Ting, Samdin~S.\,B., Mohd~Noor~A.} 
Sensor space time-varying information flow analysis of multiclass motor imagery 
through Kalman smoother and EM algorithm~// 
Conference (International) on BioSignal Analysis, Processing and Systems (ICBAPS)
Proceedings, 2015. 
P.~118--122.

\bibitem{Abanto2010} 
\Au{Abanto-Vallea~C.\,A., Bandyopadhyayb~D., Lachosc~V.\,H., Enriquezd~I.} 
Robust Bayesian analysis of heavy-tailed stochastic volatility models 
using scale mixtures of normal distributions~// Comput. Stat. 
Data An., 2010. Vol.~54. No.\,12. P.~2883--2898.

\bibitem{Wang2011} 
\Au{Wang~J.\,J.\,J., Chan~J.\,S.\,K., Choy~S.\,T.\,B.} 
Stochastic volatility models with leverage and heavytailed distributions: 
A~Bayesian approach using scale mixtures~// Comput. Stat. 
Data An., 2011. Vol.~55. No.\,1. P.~852--862.

\bibitem{Gorshenin2011b}
\Au{Горшенин~А.\,К.}
Проверка статистических гипотез в~модели расщепления компоненты~//
Вестник Московского ун-та. Сер.~15: Вычислительная математика и~кибернетика, 2011. 
№\,4. С.~26--32.

\bibitem{Gorshenin2015a} 
\Au{Gorshenin A.\,K.}
On implementation of EM-type algorithms in the stochastic models 
for a~matrix computing on GPU~// AIP Conference Proceedings, 2015. Vol.~1648. P.~250008.


\bibitem{Gorshenin2015b} 
\Au{Королев~В.\,Ю., Горшенин~А.\,К., Гулев~С.\,К., Беляев~К.\,П.} Статистическое
моделирование турбулентных потоков тепла между океаном 
и~атмосферой с~помощью метода скользящего
разделения конечных нормальных смесей~// Информатика и~её применения, 2015. Т.~9. Вып.~4. C.~3--13.

\bibitem{Gorshenin2015c} 
\Au{Gorshenin~A., Kuzmin~V.} Online system for the construction of
structural models of information flows~// 
7th  Congress (International) on Ultra Modern Telecommunications
and Control Systems and Workshops (ICUMT) Proceedings.~--- Piscataway, NJ, USA:
IEEE, 2015. P.~216--219.

\end{thebibliography}

 }
 }

\end{multicols}

\vspace*{-3pt}

\hfill{\small\textit{Поступила в~редакцию 06.02.16}}

%\vspace*{8pt}

\newpage

%\vspace*{-24pt}

%\hrule

%\vspace*{2pt}

%\hrule

\vspace*{-24pt}



\def\tit{CONCEPT OF ONLINE SERVICE FOR~STOCHASTIC MODELING OF~REAL PROCESSES}

\def\titkol{Concept of online service for stochastic modeling of real processes}

\def\aut{A.\,K.~Gorshenin$^{1,2}$}

\def\autkol{A.\,K.~Gorshenin}

\titel{\tit}{\aut}{\autkol}{\titkol}

\vspace*{-9pt}


\noindent
$^1$Institute of Informatics Problems, Federal Research Center 
``Computer Science and Control'' of the Russian\linebreak
 $\hphantom{^1}$Academy of Sciences,
44-2 Vavilov Str., Moscow 119333, Russian Federation

\noindent %Federal State Budget Educational Institution  of  Higher Education 
$^2$Moscow Technological University (MIREA),
78~Vernadskogo Ave., %\linebreak
Moscow 119454, Russian Federation

\def\leftfootline{\small{\textbf{\thepage}
\hfill INFORMATIKA I EE PRIMENENIYA~--- INFORMATICS AND
APPLICATIONS\ \ \ 2016\ \ \ volume~10\ \ \ issue\ 1}
}%
 \def\rightfootline{\small{INFORMATIKA I EE PRIMENENIYA~---
INFORMATICS AND APPLICATIONS\ \ \ 2016\ \ \ volume~10\ \ \ issue\ 1
\hfill \textbf{\thepage}}}

\vspace*{3pt}



\Abste{Information flows analysis based on various probabilistic 
models is widely used in various applied fields. The article describes 
the basic principles of construction of the new online system for stochastic 
modeling of real processes, which has no analogue due to the universality 
of the set of methods and the concept of an Internet resource; 
so, the end user should not check personal computer's specifications and could upload data 
to the server immediately and then process the samples.}

\KWE{probability mixtures; moving separation of mixtures; data mining; online
software; matrix computing}



\DOI{10.14357/19922264160107}

\Ack
\noindent
The research was supported by the Russian Foundation for Basic Research (project
15-37-20851).


\vspace*{12pt}

  \begin{multicols}{2}

\renewcommand{\bibname}{\protect\rmfamily References}
%\renewcommand{\bibname}{\large\protect\rm References}

{\small\frenchspacing
 {%\baselineskip=10.8pt
 \addcontentsline{toc}{section}{References}
 \begin{thebibliography}{99}
\bibitem{1-gg}
\Aue{Korolev, V.\,Yu.} 2011. 
\textit{Veroyatnostno-statisticheskie metody dekompozitsii 
volatil'nosti khaoticheskikh protsessov} [Probabilistic and statistical methods of 
decomposition of volatility of chaotic processes]. Moscow: Moscow University
Publishing House. 512~p.

\bibitem{2-gg}
\Aue{Korolev, V.\,Yu., A.\,Yu. Korchagin, and A.\,K.~Gorshenin}. 
2015. Nekotorye svoystva dispersionno-sdvigovykh\linebreak smesey normal'nykh zakonov 
[Some properties of variance-mean normal mixtures]. 
\textit{Statisticheskie Metody Otsenivaniya i~Proverki Gipotez}
[Statistical Methods of Estimation and Hypothesis Testing] 26:134--153.

\bibitem{3-gg}
\Aue{Gorshenin, A.\,K., V.\,Yu. Korolev, and A.\,M.~Tursunbaev}. 
2008. Mediannye modifikatsii EM- i~SEM-algoritmov dlya razdeleniya smesey 
veroyatnostnykh raspredeleniy i~ikh primenenie k~dekompozitsii volatil'nosti 
finansovykh vremennykh ryadov [Median modification of EM- and SEM-algorithms for 
separation of mixtures of probability distributions and their application to 
the decomposition of volatility of financial time series]. 
\textit{Informatika i~ee Primeneniya}~--- \textit{Inform. Appl.} 2(4):12--47.

\bibitem{4-gg}
\Aue{Gorshenin, A.\,K., V.\,Yu. Korolev, D.\,V.~Malakhov, and N.\,N.~Skvortsova}. 
2011. Analiz tonkoy stokha\-sti\-che\-skoy struktury khaoticheskikh protsessov 
s~pomoshch'yu yadernykh otsenok [Analysis of fine stochastic structure 
of chaotic processes by kernel estimators]. \textit{Matematicheskoe 
Modelirovanie} [Mathematical Modeling] 23(4):83--89.

\bibitem{5-gg}
\Aue{Gorshenin, A.\,K., V.\,Yu. Korolev, D.\,V.~Malakhov, and 
N.\,N.~Skvortsova}. 2012. Ob issledovanii plazmennoy turbulentnosti na osnove 
analiza spektrov [On the investiga-\linebreak\vspace*{-12pt}

\columnbreak

\noindent
tion of plasma turbulence by the analysis of the spectra]. 
\textit{Komp'yuternye Issledovaniya i~Modelirovanie} [Computer Research and 
Modeling] 4(4):793--802.

\bibitem{6-gg}
\Aue{Hamedi, M., S.-H.~Salleh, Ting Chee-Ming, S.\,B.~Sam\-din, and A.~Mohd Noor}. 
2015. Sensor space time-varying information flow analysis of multiclass motor 
imagery through Kalman smoother and EM algorithm. \textit{Conference (International) on 
BioSignal Analysis, Processing and Systems (ICBAPS) Proceedings}. 118--122.

\bibitem{7-gg}
\Aue{Abanto-Vallea, C.\,A., D.~Bandyopadhyayb, V.\,H.~Lachosc, and I.~Enriquezd}. 
2010. Robust Bayesian analysis of heavy-tailed stochastic volatility models using 
scale mixtures of normal distributions. 
\textit{Comput. Stat. Data An.} 54(12):2883--2898.

\bibitem{8-gg}
\Aue{Wang, J.\,J.\,J., J.\,S.\,K.~Chan, and S.\,T.\,B.~Choy}. 2011. 
Stochastic volatility models with leverage and heavytailed distributions: 
A~Bayesian approach using scale mixtures.
\textit{Comput. Stat.  Data An.} 55(1):852--862.

\bibitem{9-gg}
\Aue{Gorshenin, A.\,K.} 2011. Proverka statisticheskikh gipotez 
v~modeli rasshchepleniya komponenty [Testing of statistical hypotheses in the 
splitting component model]. 
\textit{Vestnik Moskovskogo Un-ta. Vychislitel'naya Matematika i~Kibernetika}
[Moscow University Computational Mathematics and Cybernetics] 4:26--32.

\bibitem{10-gg}
\Aue{Gorshenin, A.\,K.} 2015. 
On implementation of EM-type algorithms in the stochastic models for 
a~matrix computing on GPU. \textit{AIP Conference Proceedings}. 1648:250008


\bibitem{11-gg}
\Aue{Korolev, V.\,Yu., A.\,K.~Gorshenin, S.\,K.~Gulev, and K.\,P.~Belyaev}. 
2015. Statisticheskoe modelirovanie turbulentnykh potokov tepla mezhdu okeanom 
i~at\-mo\-sfe\-roy\linebreak\vspace*{-12pt}

\pagebreak

\noindent
 s~pomoshch'yu metoda skol'zyashchego razdeleniya konech\-nykh 
normal'nykh smesey [Statistical modeling of air--sea turbulent heat fluxes by 
the method of moving separation of finite normal mixtures]. \textit{Informatika 
i~ee Primeneniya}~--- \textit{Inform. Appl.} 9(4):3--13.

\bibitem{12-gg}
\Aue{Gorshenin, A., and V.~Kuzmin}. 
2015. Online system for the construction of structural models of information flows. 
\textit{7th  Congress (International) on Ultra Modern Telecommunications and 
Control Systems and Workshops (ICUMT) Proceedings}. Piscataway, NJ: IEEE, 2015. 
216--219.
\end{thebibliography}

 }
 }

\end{multicols}

\vspace*{-3pt}

\hfill{\small\textit{Received February 6, 2016}}

\Contrl


\noindent
\textbf{Gorshenin Andrey K.}  (b.\ 1986)~---
Candidate of Science (PhD) in physics and mathematics, senior scientist, 
Institute of Informatics Problems, Federal Research Center ``Computer Science and Control'' 
of the Russian Academy of Sciences, 44-2 Vavilov Str., Moscow 119333, 
Russian Federation; associate professor, %Federal State Budget Educational Institution   of  Higher Education 
Moscow Technological University (MIREA), 
78~Vernadskogo Ave., Moscow 119454, Russian Federation;
 agorshenin@frcсsc.ru
 




\label{end\stat}


\renewcommand{\bibname}{\protect\rm Литература} %7рис
\def\stat{gorshenin}

\def\tit{ЗАШУМЛЕНИЕ ДАННЫХ КОНЕЧНЫМИ СМЕСЯМИ НОРМАЛЬНЫХ 
И~ГАММА-РАСПРЕДЕЛЕНИЙ\\ С~ПРИМЕНЕНИЕМ К~ЗАДАЧЕ ОКРУГЛЕНИЯ НАБЛЮДЕНИЙ$^*$}

\def\titkol{Зашумление данных конечными смесями нормальных 
и~гамма-распределений с~применением к~задаче округления} % наблюдений}

\def\aut{А.\,К.~Горшенин$^1$}

\def\autkol{А.\,К.~Горшенин}

\titel{\tit}{\aut}{\autkol}{\titkol}

\index{Горшенин А.\,К.}
\index{Gorshenin A.\,K.}


{\renewcommand{\thefootnote}{\fnsymbol{footnote}} \footnotetext[1]
{Работа выполнена при поддержке РНФ (проект 18-71-00156).}}


\renewcommand{\thefootnote}{\arabic{footnote}}
\footnotetext[1]{Институт проблем информатики Федерального исследовательского центра 
<<Информатика и~управление>> Российской академии наук, \mbox{agorshenin@frccsc.ru}}

\vspace*{-12pt}




\Abst{Во многих реальных задачах проводится статистический анализ данных, 
содержащих дополнительные ошибки измерения, в~том числе в~виде округления, 
что в~ряде ситуаций может приводить к~достаточно существенным искажениям. 
В~настоящей статье для одной из возможных моделей округления получены оценки 
для неизвестного математического ожидания наблюдений в~предположении, что 
исходные данные дополнительно зашумлены с~по\-мощью случайных величин, 
име\-ющих распределения типа конечных смесей нормальных и~гам\-ма-за\-ко\-нов. 
Построены доверительные интервалы для неизвестного математического ожидания 
с~использованием уточненной оценки для дисперсии целой части случайной величины. 
Обсуждается алгоритм определения значения параметра для искусственного шума, 
добавление которого к~исходным данным способствует повышению качества работы 
метода скользящего разделения смесей.}

\KW{зашумленные данные; округленные наблюдения; конечные смеси нормальных 
распределений; конечные смеси гам\-ма-рас\-пре\-де\-ле\-ний; доверительные интервалы;  
метод скользящего разделения смесей}

\DOI{10.14357/19922264180304}
  
\vspace*{-4pt}


\vskip 10pt plus 9pt minus 6pt

\thispagestyle{headings}

\begin{multicols}{2}

\label{st\stat}


\section{Введение}

Во многих реальных задачах данные, являющиеся непрерывными по своей сути, 
регистрируются с~помощью инструментов, вносящих дополнительные ошибки 
измерения, в~том чис\-ле в~виде округления. Таким образом, статистический 
анализ проводится не для исходных, а для преобразованных некоторым 
случайным образом наблюдений, что в~ряде ситуаций может приводить к~достаточно
 существенным искажениям.

Для преодоления данной проблемы развивались различные подходы, в~том числе 
на основе смешанных моделей (см., например, статью~\cite{Wright2003}, в~которой 
различные компоненты  используются для пред\-став\-ле\-ния уровней округления). 
В~работе~\cite{Bai2009} приводятся результаты для моделей авторегрессии и~скользящего 
среднего для округленных данных, а~в~статье~\cite{Zhang2010} эти результаты 
развиваются и~исследуются их асимптотические свойства. 
В~статье~\cite{Zhao2012} исследован метод оценивания па\-ра\-мет\-ров конечных смесей 
вероятностных распределений (в~том чис\-ле, и~многомерных) 
на основе использования EM (expectation-maximization) 
алгоритма~\cite{Korolev2011-i} с~\mbox{целью} получения состоятельных 
и~асимптотически нормальных оценок.

В настоящей статье развиваются результаты для моделей округления, 
описанных в~работах~\cite{Ushakov2015,Ushakov2017a,Ushakov2017b}. 
В~их рамках будут получены оценки для неизве\-ст\-ного математического ожидания 
наблюдений в~предположении, что исходные данные зашумлены с~по\-мощью случайных 
величин, имеющих распределения типа конечных смесей нормальных и~гам\-ма-за\-ко\-нов. 
Это позволяет учесть большее количество случайных факторов, влия\-ющих на величину 
<<дополнительной>> ошибки. Также будут построены доверительные интервалы для 
неизвестного математического ожидания. Выражения для гам\-ма-рас\-пре\-де\-ле\-ний 
получены впервые. Также обсуждается алгоритм определения значения па\-ра\-мет\-ра для 
искусственного шума, добавление которого к~исходным данным способствует 
повышению качества работы метода скользящего разделения смесей~\cite{Gorshenin2016}.

\vspace*{-12pt}

\section{Предположения и~базовые отношения}

Для сокращения формулировок теорем в~сле\-ду\-ющих разделах сделаем ряд 
предположений, на которые будем ссылаться в~дальнейшем. Итак, пусть:
\begin{itemize}
\item[(A)] $X_1,X_2,\ldots$~--- независимые одинаково распределенные 
случайные величины с~неизвестным математическим ожиданием ${\sf E}_X\hm<+\infty$;
\item[(B)] $\varepsilon_1,\varepsilon_2,\ldots$~--- независимые одинаково 
распределенные случайные величины с~математическим ожиданием 
${\sf E}_\varepsilon\hm<+\infty$; %\label{B}
\item[(C)] $X_1,X_2,\ldots$ и~$\varepsilon_1,\varepsilon_2,\ldots$ 
являются независимыми;
\item[(D)] $Y_j=\left[X_j+\varepsilon_j+1/2\right]$ для всех $j\hm=1,2,\ldots$ 
представляют собой округление значения суммы случайных величин $X_j\hm+\varepsilon_j$ 
до ближайшего целого сверху (при этом запись~$[\cdot]$ соответствует целой 
части выражения).
\end{itemize}

В рамках данных предположений в~статье будут рассмотрены вопросы качества 
приближения неизвестного математического ожидания~${\sf E}_X$ для исходных данных 
в~ситуации, когда наблюдения для анализа получены с~аддитивной ошибкой c известными 
распределениями (см.\ предположение~(B)) и~дополнительно округляются до 
ближайшего целого (см.\ предположение~(D)).

Заметим, что в~силу усиленного закона больших чисел справедливы следующие выражения:
\begin{multline}
\fr{1}{n}\sum\limits_{j=1}^n Y_j\xrightarrow[n\to\infty]{\text{п.н.}}
{\sf E}_Y\equiv\mathbb{E}\left[X_1+\varepsilon_1+\fr{1}{2}\right]={}\\
{}=\mathbb{E}\left(X_j+\varepsilon_j+\fr{1}{2}\right)-\mathbb{E}
\left\{X_j+\varepsilon_j+\fr{1}{2}\right\}={}\\
{}={\sf E}_X+{\sf E}_\varepsilon+\fr{1}{2}-\mathbb{E}\left\{X_j+\varepsilon_j+\fr{1}{2}\right\}. 
\label{Law}
\end{multline}

Запись $\{\cdot\}$ в~формуле~\eqref{Law} соответствует дробной 
части выражения, а~п.н.\ обозначает сходимость в~смысле почти наверное.

Для доказательства результатов в~дальнейшем потребуется следующее 
представления для дробной части  абсолютно непрерывной случайной величины~$Z$ 
с~абсолютно  интегрируемой характеристической функцией~$\varphi_Z(t)$
 (см., например, Лемму~4 в~работе~\cite{Ushakov2017b}):
\begin{equation}
\label{Fract}
\mathbb{E}\{Z\}=\fr{1}{2}-\sum\limits_{n=1}^\infty 
\fr{\mathrm{Im}\left (\varphi_Z(2\pi n)\right)}{\pi n}\,.
\end{equation}

Через $\mathrm{Im}\,(\cdot)$ в~формуле~\eqref{Fract} обозначена мнимая часть 
соответствующей функции.

При построении доверительных интервалов в~дальнейшем будет 
использована следующая оценка, справедливая для любой случайной величины~$Z$:
\begin{equation}
\mathbb{D}[Z]\leqslant \left(\sqrt{\mathbb{D} Z}+\fr{1}{2}\right)^2.
\label{Var}
\end{equation}
Она может быть проверена непосредственно с~учетом представления 
$\mathbb{D} [Z]\hm=\mathbb{D}\left(Z\hm-\{Z\}\right)$, неравенства 
Ко\-ши--Бу\-ня\-ков\-ско\-го для ковариации и~соотношения 
 $\mathbb{D}\{Z\}\hm\leqslant 1/4$, справедливого для любой случайной величины~$Z$ 
 (см., например, статью~\cite{Ushakov2017b}). Отметим, что данная оценка 
 является более точной по сравнению с~использованным для аналогичных 
 целей в~работе~\cite{Ushakov2017b} соотношением 
 $\mathbb{D} [Z]\hm\leqslant 2\mathbb{D} Z\hm+1/2$. Действительно,
\begin{equation*}
2\mathbb{D} Z+\fr{1}{2}-\left(\sqrt{\mathbb{D} Z}+\fr{1}{2}\right)^2=
\left(\sqrt{\mathbb{D} Z}-\fr{1}{2}\right)^2\geqslant0\,,
\end{equation*}
причем для всех $\sqrt{\mathbb{D} Z}\hm\neq {1}/{2}$ 
данное неравенство является строгим.

\section{Конечные смеси нормальных законов}

Для случайной величины~$X$, имеющей распределение типа 
конечной смеси нормальных законов~\cite{Korolev2011-i} с~параметрами 
${\bf a}\hm=(a_1,\ldots, a_k)$, $a_j\hm\in \mathbb{R}$, 
$\boldsymbol{\sigma}\hm=(\sigma_1,\ldots, \sigma_k)$, 
$\sigma_j\hm>0$, ${\bf p}\hm=(p_1,\ldots, p_k)$, $p_j\hm\geqslant 0$, 
$\sum\nolimits_{j=1}^{k}p_j\hm=1$, плот\-ность которого задается выражением
\begin{equation}
f_X(x)=\sum\limits_{j=1}^{k}\fr{p_j}{\sigma_j\sqrt{2\pi}}\,e^{-(x-a_j)^2/(2\sigma_j^2)}\,,
\label{FinNormMixt}
\end{equation}
характеристическая функция имеет вид:
\begin{equation}
\varphi_X(t)=\int\limits_{-\infty}^{+\infty}\!\!e^{itx} f_X(x)\, dx = 
\sum\limits_{j=1}^{k}p_j e^{ita_j-\sigma_j^2 t^2/2}.
\label{ChiFinNormMixt}
\end{equation}

Абсолютная интегрируемость  $\varphi_X(t)$ вытекает из свойств 
характеристической функции нормального распределения. 
Заметим, что в~точке $t\hm=2\pi n$ выражение~\eqref{ChiFinNormMixt} принимает 
сле\-ду\-ющий вид:
\begin{equation}
\label{ChiFinNormMixt2npi}
\varphi_X(2\pi n)= \sum\limits_{j=1}^{k}p_j e^{-2\pi^2 \sigma_j^2 n^2}\,.
\end{equation}

Рассмотрим вопрос точности оценивания неизвестного математического ожидания~${\sf E}_X$ 
при до\-бав\-ле\-нии зашумления.

\smallskip

\noindent
\textbf{Теорема~1.}\ 
\textit{Пусть выполнены предположения}~(A)--(D), 
\textit{причем случайные величины~$\varepsilon_j$, $j\hm=1,2,\ldots$, 
имеют распределение типа конечной $k$-ком\-по\-нент\-ной смеси нормальных законов 
вида}~\eqref{FinNormMixt} \textit{с~па\-ра\-мет\-ра\-ми~${\bf a}$, $\boldsymbol{\sigma}$ 
и~${\bf p}$. Тогда}
\begin{equation}
\label{Th1Eq}
\left\lvert {\sf E}_Y-{\sf E}_X\right\rvert \leqslant 
A+\fr{1}{\pi}\left(1+\fr{1}{4\pi^2\sigma^2}\right)e^{-2\pi^2\sigma^2}\,, 
\end{equation}
\textit{где} $A=\max(|a_1|,\ldots,|a_k|)$, $\sigma\hm=\min(\sigma_1,\ldots,\sigma_k)$.

\smallskip


\noindent
Д\,о\,к\,а\,з\,а\,т\,е\,л\,ь\,с\,т\,в\,о\,.\ \
С~учетом пред\-став\-ле\-ний~\eqref{Law},~\eqref{Fract} и~\eqref{ChiFinNormMixt2npi}, 
ограниченности модуля характеристической функции, а~также не\-за\-ви\-си\-мости 
случайных величин~$X_j$ и~$\varepsilon_j$ имеем:
\begin{multline*}
\left\lvert {\sf E}_Y-{\sf E}_X\right\rvert =
\left\lvert {\sf E}_\varepsilon+\fr{1}{2}-\mathbb{E}\left\{X_j+
\varepsilon_j+\fr{1}{2}\right\}\right\rvert={}\\
{}=\left\lvert {\sf E}_\varepsilon+\sum\limits_{n=1}^\infty
\fr{\mathrm{Im} \left(\varphi_{X_j}(2\pi n)\varphi_{\varepsilon_j}(2\pi n)
\varphi_{1/2}(2\pi n)\right)}{\pi n}\right\rvert={}\\
=\left\lvert 
\vphantom{\fr{(-1)^n\sum\nolimits_{j=1}^{k}p_j e^{-2\pi^2 \sigma_j^2 n^2} 
\mathrm{Im} \left(\varphi_{X_j}(2\pi n)\right)}{\pi n}}
{\sf E}_\varepsilon+{}\right.\\
\left.{}+\sum\limits_{n=1}^\infty
\fr{\mathrm{Im} \left(\varphi_{X_j}(2\pi n) 
\sum\nolimits_{j=1}^{k}p_j e^{-2\pi^2 \sigma_j^2 n^2} 
e^{\pi n}\right)}{\pi n}\right\rvert={}\\
{}=\left\lvert 
\vphantom{\fr{(-1)^n\sum\nolimits_{j=1}^{k}p_j e^{-2\pi^2 \sigma_j^2 n^2} 
\mathrm{Im} \left(\varphi_{X_j}(2\pi n)\right)}{\pi n}}
{\sf E}_\varepsilon+{}\right.\\
\left.{}+\sum\limits_{n=1}^\infty
\fr{(-1)^n\sum\nolimits_{j=1}^{k}p_j e^{-2\pi^2 \sigma_j^2 n^2} 
\mathrm{Im} \left(\varphi_{X_j}(2\pi n)\right)}{\pi n}\right\rvert\leqslant{}\\
{}\leqslant \left\lvert {\sf E}_\varepsilon\right\rvert+\left\lvert\
\sum\limits_{j=1}^{k}p_j\sum\limits_{n=1}^\infty 
\fr{1}{\pi n} e^{-2\pi^2 \sigma_j^2 n^2}\right\rvert\leqslant {}\\
\\
{}\leqslant
\max\left(|a_1|,\ldots,|a_k|\right)+{}\\
{}+\sum\limits_{j=1}^{k} 
\fr{p_j}{\pi} \left(\!1+\fr{1}{4\pi^2\sigma_j^2}\!\right)\!e^{-2\pi^2\sigma_j^2}\leqslant{}\\
{}\leqslant
A+\fr{1}\pi\left(1+\fr{1}{4\pi^2\sigma^2}\right)e^{-2\pi^2\sigma^2}\,.
\end{multline*}

Справедливость использованной оценки 
\begin{equation*}
\sum\limits_{n=1}^\infty
\fr{e^{-2\pi^2 \sigma_j^2 n^2}}{n}\leqslant 
\left(1+\fr{1}{4\pi^2\sigma_j^2}\right)e^{-2\pi^2\sigma_j^2}
\end{equation*}
может быть проверена непосредственно (например, см.\ доказательство Теоремы~6 
в~статье~\cite{Ushakov2017b}).~\hfill$\square$

\smallskip

\noindent
\textbf{Замечание~1.}
В~случае, если зашумление производится нормально распределенными случайными 
величинами c нулевыми средними (т.\,е.\ в~формуле~\eqref{Th1Eq} необходимо считать 
$A\hm=0$, $k\hm=1$), то Тео\-ре\-ма~1 совпадает с~результатом, 
полученным в~работе~\cite{Ushakov2017b}.


\smallskip

Рассмотрим вопросы построения доверительного интервала для неизвестного 
математического ожидания~${\sf E}_X$ в~предположении, что случайные величины~$X_j$ не 
содержат ошибок измерения, а~все погрешности учтены исключительно в~за\-шум\-ля\-ющих 
элементах~$\varepsilon_j$.

\smallskip

\noindent
\textbf{Теорема~2.}\ 
\textit{Пусть выполнены предположения}~(A)--(D), 
\textit{причем случайные величины~$\varepsilon_j$, $j\hm=1,2,\ldots$, имеют 
распределение типа конечной $k$-ком\-по\-нент\-ной смеси нормальных законов 
вида}~\eqref{FinNormMixt} \textit{с~параметрами~${\bf a}$, $\boldsymbol{\sigma}$ 
и~${\bf p}$, а~случайные величины} $X_j\stackrel{\text{п.н.}}{=}{\sf E}_X$, $j\hm=1,2,\ldots$ 
\textit{Тогда доверительный интервал для~${\sf E}_X$ при условии $0\hm<\alpha\hm<1$ имеет вид}:
\begin{equation} 
\label{Th2Eq}
\hat{{\sf E}}_X - f({\bf a},\boldsymbol{\sigma},\alpha,n) 
\leqslant {\sf E}_X \leqslant  \hat{{\sf E}}_X + f({\bf a},\boldsymbol{\sigma},\alpha,n),
\end{equation}
\textit{где}

\vspace*{-2pt}

\noindent
\begin{align}
\hat{{\sf E}}_X&=\fr{1}{n} \sum\limits_{j=1}^{n} Y_j\,; \label{Th2hatE}\\
f({\bf a},\boldsymbol{\sigma},\alpha,n)&=
\fr{z_{1-{\alpha}/2}}{\sqrt{n}} \left(\sqrt{A^2+\Sigma^2}+\fr{1}{2}\right) +{}\notag\\
&{}+A+\fr{1}\pi\left(1+\fr{1}{4\pi^2\sigma^2}\right)e^{-2\pi^2\sigma^2}\,;
  \label{Th2f}
\end{align}
\textit{$z_{1-{\alpha}/2}$~--- $\left(1-{\alpha}/2\right)$-кван\-тиль 
стандартного нормального распределения; $A\hm=\max(|a_1|,\ldots,|a_k|)$; 
$\Sigma\hm=\max(\sigma_1,\ldots,\sigma_k)$; $\sigma\hm=\min(\sigma_1,\ldots,\sigma_k)$}. 


\smallskip

\noindent
\noindent
Д\,о\,к\,а\,з\,а\,т\,е\,л\,ь\,с\,т\,в\,о\,.\ \
Из центральной предельной тео\-ре\-мы с~учетом условия~(A) следует, 
что величина~$\hat{{\sf E}}_X$~\eqref{Th2hatE} асимптотически нормальна с~математическим 
ожиданием 
\begin{equation}
{\sf E}_Y\equiv \mathbb{E}\left[{\sf E}_X+\varepsilon_1+\fr{1}{2}\right] \label{EY}
\end{equation}
и дисперсией
\begin{equation}
\fr{1}{n} {\sf D}_Y\equiv \fr{1}{n}\mathbb{D}\left[{\sf E}_X+\varepsilon_1+
\fr{1}{2}\right]. \label{DY}
\end{equation}

Воспользовавшись оценкой~\eqref{Var}, получим:

\vspace*{-2pt}

\noindent
\begin{multline*}
{\sf D}_Y \leqslant  \left(\sqrt{\mathbb{D} \left({\sf E}_X+\varepsilon_1+\fr{1}{2}\right)}+
\fr{1}{2}\right)^2={}\\
{}=
\left(\sqrt{\mathbb{D}\varepsilon_1}+\fr{1}{2}\right)^2= {}\\
{}= \left(\sqrt{\sum\limits_{j=1}^{k}p_j\left(\left(a_j-\sum\limits_{t=1}^{k}
p_t a_t\right)^2+\sigma_j^2\right)}+\fr{1}{2}\right)^2\leqslant {}\\ 
{}\leqslant \left(\sqrt{A^2+\Sigma^2}+\fr{1}{2}\right)^2\,.
\end{multline*}
Тогда доверительный интервал уровня $1\hm-\alpha$ для математического ожидания~${\sf E}_Y$ 
имеет вид:
\begin{equation*}
\mathbb{P}\left(\left\lvert \hat{{\sf E}}_X-{\sf E}_Y\right\rvert \leqslant 
\fr{z_{1-{\alpha}/2}}{\sqrt{n}} 
\left(\sqrt{A^2+\Sigma^2}+\fr{1}{2}\right)\right)\geqslant 1-\alpha\,.
\end{equation*}

\begin{table*}[b]\small
\begin{center}

\begin{tabular}{|c|c|c|c|c|c|c|c|}
\multicolumn{7}{p{100mm}}{Численные решения уравнений~\eqref{f1} и~\eqref{f2} относительно 
параметра~$\sigma$ для некоторых значений~$n$ и~$\alpha$}\\
\multicolumn{7}{c}{\ }\\[-6pt]
\hline
\multicolumn{1}{|c|}{Размер}  & \multicolumn{2}{c|}{Уровень $\alpha=0{,}1$}& 
\multicolumn{2}{c|}{Уровень $\alpha=0{,}05$}& 
\multicolumn{2}{c|}{Уровень $\alpha=0{,}01$}\\
\cline{2-7}
\multicolumn{1}{|c|}{выборки $n$}&$\sigma_1$&$\sigma_2$&$\sigma_1$&$\sigma_2$&$\sigma_1$&$\sigma_2$\\
\hline
$\hphantom{000}100$&$0{,}4302$&$0{,}435$&$0{,}419$&$0{,}425$&$0{,}4002$&$0{,}408$\\
%\hline
$\hphantom{000}200$&$0{,}452$&$0{,}455$ &$0{,}441$&$0{,}445$&$0{,}424$&$0{,}429$\\
%\hline
$\hphantom{00}1000$&$0{,}499$&$0{,}499$ &$0{,}489$&$0{,}489$&$0{,}473$&$0{,}475$\\
%\hline
$\hphantom{0}10000$&$0{,}558$&$0{,}556$ &$0{,}549$&$0{,}547$&$0{,}536$&$0{,}534$\\
%\hline
$100000$&$0{,}611$&$0{,}607$ &$0{,}603$&$0{,}599$&$0{,}591$&$0{,}588$\\
\hline
\end{tabular}
\end{center}
\end{table*}


\noindent
Откуда следует справедливость соотношения~\eqref{Th2Eq} c~уче\-том 
очевидного неравенства

\pagebreak

\noindent
\begin{equation*}
\left\lvert \hat{{\sf E}}_X-{\sf E}_X\right\rvert \leqslant 
\left\lvert \hat{{\sf E}}_X-{\sf E}_Y\right\rvert +\left\lvert {\sf E}_Y-{\sf E}_X\right\rvert 
\end{equation*}
и оценки~\eqref{Th1Eq} из Теоремы~1.~\hfill$\square$

\smallskip

\noindent
\textbf{Замечание~2.}
В~работе~\cite{Gorshenin2016} было продемонстрировано повышение точ\-ности 
работы метода скользящего разделения конечных нормальных смесей за счет 
введения дополнительной компоненты, имеющей нормальное 
распределение $\mathcal{N}(0,\sigma^2)$ с~математическим ожиданием, равным~$0$, 
и~стандартным отклонением~$\sigma$. При этом была отмечена сложность выбора 
параметра~$\sigma$ для сохранения структуры выборки, близкой к~исходной. 
Результат Теоремы~2 может быть использован с~данной целью, если положить $k\hm=1$, 
$a_j\hm=0$ для всех $j\hm=1,2,\ldots$ и~выбирать величину~$\sigma$ как 
минимизирующую длину доверительного интервала~\eqref{Th2Eq}. Для 
этого необходимо найти производную функции $f(0,\sigma,\alpha,n)$~\eqref{Th2f} 
и~численно решить уравнение
\begin{multline}
f_\sigma'(0,\sigma,\alpha,n)\equiv \fr{z_{1-{\alpha}/2}}{\sqrt{n}} - {}\\
{}-
e^{-2\pi^2\sigma^2}\left(4\pi\sigma+\fr{1}{2\pi^3\sigma^3}+
\fr{1}{\pi\sigma}\right)=0
\label{f1}
\end{multline}
относительно неизвестного параметра~$\sigma$ при выбранных значениях величин~$n$ 
и~$\alpha$. В~качестве альтернативы можно использовать вид доверительного интервала 
из статьи~\cite{Ushakov2017b}, полученный с~помощью неравенства $\mathbb{D} [Z]
\hm\leqslant 2\mathbb{D} Z\hm+{1}/{2}$, и~искать решение уравнения вида:
\begin{multline}
\hspace*{-2.90578pt}\fr{2\sigma z_{1-{\alpha}/2}}{\sqrt{n (2\sigma^2+{1}/{2})}} -
 e^{-2\pi^2\sigma^2}\left(4\pi\sigma+\fr{1}{2\pi^3\sigma^3}+
 \fr{1}{\pi\sigma}\right)={}\\
 {}=0\,.\label{f2}
\end{multline}

Примеры найденных значений~$\sigma$ для типичных размеров выборок в~методе 
скользящего разделения смесей (учитываются как возможная ширина окна, 
так и~общее количество наблюдений в~анализируемом ряде) приведены в~таблице 
(использован метод оптимизации \verb"Trust-Region Dogleg" пакета \verb"MATLAB" 
c~настройками по умолчанию), в~которой через~$\sigma_1$ обозначено приближенное  
решение уравнения~\eqref{f1}, a~через $\sigma_2$~--- уравнения~\eqref{f2}.


Проверка практической эффективности данного подхода в~качестве 
критерия выбора параметров зашумляющего распределения для повышения 
точности работы метода скользящего разделения смесей может быть отмечена 
как задача для дальнейших исследований.


\section{Конечные смеси гамма-распределений}

Для случайной величины~$X$, имеющей распределение типа конечной смеси 
гам\-ма-рас\-пре\-де\-ле\-ний с~параметрами ${\bf r}\hm=(r_1,\ldots, r_k)$,
 $r_j\hm>0$, $\boldsymbol{\lambda}\hm=(\lambda_1,\ldots, \lambda_k)$, $\lambda_j\hm>0$, 
 ${\bf p}\hm=(p_1,\ldots, p_k)$, $p_j\hm\geqslant 0$, $\sum\nolimits_{j=1}^{k}p_j\hm=1$, 
 плот\-ность которого задается выражением
\begin{equation}
f_X(x)=\sum\limits_{j=1}^{k}p_j\fr{\lambda_j^{r_j} e^{-\lambda_j x}}
{\Gamma(r_j)}\,x^{r_j-1}\,,
\label{FinGammaMixt}
\end{equation}
характеристическая функция имеет следующий вид:
%характеристическая функция задается следующим выражением:
\begin{equation}
\varphi_X(t)=\!\int\limits_{-\infty}^{+\infty}\!\!\!e^{itx} f_X(x)\, dx = \!
\sum\limits_{j=1}^{k}p_j \left(\!1-\fr{it}{\lambda_j}\right)^{-r_j}\!.\!
\label{ChiFinGammaMixt}
\end{equation}

Отметим, что подобные модели зашумления разумно использовать в~случае, 
если известно, что данные сосредоточены на положительной полуоси, например 
при анализе различных информационных потоков (см., в~част\-ности, 
 работу~\cite{Gorshenin2013}). 

Проверим абсолютную интегрируемость функции $\varphi_X(t)$~\eqref{ChiFinGammaMixt}. 
Имеем:
\begin{multline*}
\int\limits_{-\infty}^{+\infty}\left\lvert\varphi_X(t)\right\rvert\, dt\leqslant 
\sum\limits_{j=1}^{k}p_j \int\limits_{-\infty}^{+\infty}\left\lvert \left(
1-\fr{it}{\lambda_j}\right)^{-r_j}\right\rvert \, dt={}\\
{}=\sum\limits_{j=1}^{k}p_j \int\limits_{-\infty}^{+\infty} \left\lvert\left(
\fr{\lambda_j(\lambda_j+it)}{\lambda_j^2+t^2}\right)^{r_j}\right\rvert\, dt \leqslant{}\\
{}\leqslant\sum\limits_{j=1}^{k}p_j \lambda_j \int\limits_{-\infty}^{+\infty}\left(
1+y^2\right)^{-{r_j}/{2}}\, dy\,.
\end{multline*}

Подынтегральное выражение при $r_j\hm\geqslant 2$ может быть оценено сверху 
функцией $1/({1+y^2})$, при этом соответствующий интеграл равен~$\pi$, что влечет 
абсолютную интегрируемость характеристической функции для конечной смеси 
гам\-ма-рас\-пре\-де\-ле\-ний. Поэтому в~дальнейшем будем предполагать,
 что $r_j\hm\geqslant 2$ для всех возможных значений $j\hm=1,2,\ldots$

Рассмотрим вопрос точ\-ности оценивания неизвестного математического ожидания ${\sf E}_X\hm>0$ 
при добавлении зашумления.

\smallskip

\noindent
\textbf{Теорема~3.}
\textit{Пусть выполнены предположения}~(A)--(D), 
\textit{причем случайные величины~$\varepsilon_j$, $j\hm=1,2,\ldots$, имеют 
распределение типа конечной $k$-ком\-по\-нент\-ной смеси 
гам\-ма-рас\-пре\-де\-ле\-ний вида}~\eqref{FinGammaMixt} 
\textit{с~па\-ра\-мет\-ра\-ми~${\bf r}$, $\boldsymbol{\lambda}$ и~${\bf p}$. Тогда}
\begin{equation}
\label{Th3Eq}
\left\lvert {\sf E}_Y-{\sf E}_X\right\rvert \leqslant \fr{R}{\lambda}+
\fr{\Lambda^{R}}{2^{r}\pi^{r+1}}\left(1+\frac1{r}\right)\,,
\end{equation}
\textit{где} $r=\min(r_1, \ldots,r_k)$; $R\hm=\max(r_1, \ldots,r_k)$; 
$\lambda\hm=\max(\lambda_1, \ldots,\lambda_k)$; 
$\Lambda\hm=\max(\lambda_1, \ldots,\lambda_k)$.

\smallskip

\noindent
Д\,о\,к\,а\,з\,а\,т\,е\,л\,ь\,с\,т\,в\,о\,.\ \
С~учетом пред\-став\-ле\-ний~\eqref{Law} и~\eqref{Fract}, ограниченности 
модуля характеристической функции, перехода от тригонометрической к~показательной 
записи комплексных чисел, а~также независимости случайных величин~$X_j$ 
и~$\varepsilon_j$ \mbox{имеем}:
\begin{multline*}
\left\lvert {\sf E}_Y-{\sf E}_X\right\rvert
\leqslant \left\lvert {\sf E}_\varepsilon\right\rvert+ {}\\
{}+\left\lvert\sum\limits_{n=1}^\infty
\left(
(-1)^n\mathrm{Im} \left(\sum\limits_{j=1}^{k}p_j \varphi_{X_j}(2\pi n)\left(
\vphantom{\fr{2\pi n}{\lambda_j}}
1-{}\right.\right.\right.\right.\\
\left.\left.\left.\left.{}-i\left(\fr{2\pi n}{\lambda_j}\right)\right)^{-r_j}\right)
\Bigg/ ({\pi n})
\vphantom{\sum\limits_{j=1}^{k}}
\right)\right\rvert={}\\
{}=\left\lvert {\sf E}_\varepsilon\right\rvert+ 
\left\lvert\sum\limits_{n=1}^\infty
\left(\!(-1)^n\mathrm{Im} \!\left(\sum\limits_{j=1}^{k}p_j \left(\!
1+\fr{4\pi^2 n^2}{\lambda_j^2}\right)^{- {r_j}/2}\!\times{}\right.\right.\right.\hspace*{-2.8663pt}\\
\left.\left.\left.{}\times \varphi_{X_j}(2\pi n)\,
e^{-ir_j\mathrm{arctan}\,({{t}/{\lambda_j}})}\right)
\Bigg/
({\pi n})
\vphantom{\left(
1+\fr{4\pi^2 n^2}{\lambda_j^2}\right)^{- {r_j}/2}}
\right)\right\rvert\leqslant{}\\
{}\leqslant \left\lvert {\sf E}_\varepsilon\right\rvert+\sum\limits_{j=1}^{k}
p_j\sum\limits_{n=1}^\infty\fr{1}{\pi n}\left(
1+\fr{4\pi^2 n^2}{\lambda_j^2}\right)^{-{r_j}/2}\leqslant{}\\
{}\leqslant  \fr{R}\lambda + \sum\limits_{j=1}^{k}p_j
\sum\limits_{n=1}^\infty\left(\fr{1}{\pi n}\,
\fr{\lambda_j^{r_j}}{(2\pi)^{r_j} n^{r_j}}\right)\leqslant {}
\\
{}\leqslant  \fr{R}{\lambda} + \sum\limits_{j=1}^{k}p_j 
\fr{\lambda_j^{r_j}}{2^{r_j}\pi^{r_j+1}}\left(1+\int\limits_{1}^{\infty}
\fr{1}{ x^{r_j+1}}\,dx\right)
\leqslant{}\\
{}\leqslant \fr{R}{\lambda}+\fr{\Lambda^{R}}{2^{r}\pi^{r+1}}\left(1+\fr{1}{r}\right).
\end{multline*}

При переходе от суммы к~интегралу используется факт убывания функции как переменной~$n$ 
(или~$x$).~\hfill$\square$


\smallskip

\noindent
\textbf{Замечание~3.}\
Теорема~3 описывает соответ\-ст\-ву\-ющий результат для гам\-ма-рас\-пре\-де\-лен\-ных 
за\-шум\-ля\-ющих случайных величин, если положить $k\hm=1$ в~выражении~\eqref{Th3Eq}. 
При этом, очевидно, $r\hm\equiv R$ и~$\lambda\hm\equiv \Lambda$.


\smallskip

Рассмотрим вопросы построения доверительного интервала для неизвестного 
математического ожидания ${\sf E}_X\hm>0$ в~предположении, что случайные величины~$X_j$ 
не содержат ошибок измерения, а все погрешности учтены исключительно в~за\-шум\-ля\-ющих 
элементах~$\varepsilon_j$.

\smallskip

\noindent
\textbf{Теорема~4.}
\textit{Пусть выполнены предположения}~(A)--(D), 
\textit{причем случайные величины~$\varepsilon_j$, $j\hm=1,2,\ldots$, имеют 
распределение типа конечной $k$-ком\-по\-нент\-ной смеси 
гам\-ма-рас\-пре\-де\-ле\-ний вида}~\eqref{FinGammaMixt} 
\textit{с~па\-ра\-мет\-ра\-ми~${\bf r}$, $\boldsymbol{\lambda}$ и~${\bf p}$, 
а~случайные величины} $X_j\stackrel{\text{п.н.}}{=}{\sf E}_X$, $j=1,2,\ldots$ 
\textit{Тогда доверительный интервал для~${\sf E}_X$ при условии $0\hm<\alpha\hm<1$ имеет вид}:
\begin{equation} 
\label{Th4Eq}
\left\lvert {\sf E}_X - \hat{{\sf E}}_X\right\rvert \leqslant  
f({\bf r},\boldsymbol{\lambda},\alpha,n),
\end{equation}
\textit{где}

\vspace*{-9pt}

\noindent
\begin{align}
\hat{{\sf E}}_X&=\fr{1}{n} \sum\limits_{j=1}^{n} Y_j\,; \label{Th4hatE}\\[-4pt]
f({\bf r}, \boldsymbol{\lambda},\alpha,n)&=\fr{z_{1-{\alpha}/2}}{\sqrt{n}} \left(
\sqrt{\fr{R(R+1)}{\lambda^2}-\fr{r^2}{\Lambda^2}}+\fr{1}{2}\right) +{}\notag\\[-1pt]
&\hspace*{20mm}{}+
\fr{R}{\lambda}+\fr{\Lambda^{R}}{2^{r}\pi^{r+1}}\left(1+\fr{1}{r}\right); \notag
\end{align}
\textit{$z_{1-{\alpha}/2}$~--- $\left(1-{\alpha}/2\right)$-кван\-тиль 
стандартного нормального распределения; $r\hm=\min(r_1, \ldots,r_k)$; 
$R\hm=\max(r_1, \ldots,r_k)$; $\lambda\hm=\max(\lambda_1, \ldots,\lambda_k)$; 
$\Lambda\hm=\max(\lambda_1, \ldots,\lambda_k)$}. 

\smallskip

\noindent
Д\,о\,к\,а\,з\,а\,т\,е\,л\,ь\,с\,т\,в\,о\,.\ \
Из центральной предельной теоремы с~учетом условия~(A) 
следует, что величина~$\hat{{\sf E}}_X$~\eqref{Th4hatE} асимптотически нормальна 
с~математическим ожиданием~${\sf E}_Y$~\eqref{EY} и~дисперсией $(1/n){\sf D}_Y$~\eqref{DY}. 
Пользуясь определением и~свойствами гам\-ма-функ\-ции, а~также оценкой~\eqref{Var} 
получим:

\noindent
\begin{multline*}
{\sf D}_Y \leqslant \left(\sqrt{\sum\limits_{j=1}^k p_j
\fr{\lambda_j^{r_j}}{\Gamma(r_j)} \int\limits_{0}^{+\infty} 
e^{\lambda_j x}x^{r_j+1}\, dx}+\fr{1}{2}\right)^2= {}\\[-0.5pt]
= \left(\sqrt{\sum\limits_{j=1}^{k}p_j
\fr{r_j(r_j+1)}{\lambda_j^2}-\left(\sum\limits_{j=1}^{k}p_j
\fr{r_j}{\lambda_j}\right) ^2}+\fr{1}{2}\right)^2\leqslant {}\\[-1.5pt]
{}\leqslant \left(\sqrt{\fr{R(R+1)}{\lambda^2}-\fr{r^2}{\Lambda^2}}+\fr{1}{2}\right)^2\,.
\end{multline*}

Аналогично доказательству Тео\-ре\-мы~2 с~учетом оценки~\eqref{Th3Eq} 
отсюда следует справедливость соотношения~\eqref{Th4Eq}.~\hfill$\square$

\vspace*{-12pt}

\section{Заключение}

Итак, в~работе получены оценки для математического ожидания наблюдений в~предположении 
зашумления конечными смесями нормальных\linebreak (Тео\-ре\-ма~1) 
и~гам\-ма-рас\-пре\-де\-ле\-ний (Тео\-ре\-ма~3). 
%
Построены доверительные интервалы 
для неизвестного математического ожидания в~этих случаях с~использованием 
уточненной оценки~\eqref{Var} 
(Тео\-ре\-мы~2 и~4 соответственно). Отметим, что соответствующие соотношения 
зависят только от <<экстремальных>> значений параметров смесей, но не от числа 
компонент и~весов в~распределении зашумляющих наблюдений. 
%
Замечание~2 
предлагает подход, который  может быть использован для определения неизвестного 
параметра искусственно добавляемого к~исходным данным шума для улучшения качества 
работы метода скользящего разделения смесей.

\smallskip
Автор выражает признательность доктору фи\-зи\-ко-ма\-те\-ма\-ти\-че\-ских наук, 
профессору Виктору Юрьевичу Королеву за идею использования оценки 
дисперсии вида~\eqref{Var} и~другие полезные обсуждения в~рамках 
работы над данной статьей.

\vspace*{-12pt}

{\small\frenchspacing
 {%\baselineskip=10.8pt
 \addcontentsline{toc}{section}{References}
 \begin{thebibliography}{99}
\bibitem{Wright2003} \Au{Wright~D.\,E., Bray~I.} 
A~mixture model for rounded data~// J.~Roy. Stat. Soc.~D 
Sta., 2003. Vol.~52. P.~3--13.

\columnbreak

\bibitem{Bai2009} \Au{Bai~Z., Zheng~S., Zhang~B., Hu~G.} 
Statistical analysis for rounded data~// J.~Stat. Plan.  Infer., 2009. 
Vol.~139. Iss.~8. P.~2526--2542.

\bibitem{Zhang2010} \Au{Zhang~B., Liu~T., Bai~Z.\,D.} 
Analysis of rounded data from dependent sequences~// 
Ann. I.~Stat. Math., 2010. Vol.~62. Iss.~6. P.~1143--1173.

\bibitem{Zhao2012} \Au{Zhao~N., Bai~Z.} 
Analysis of rounded data in mixture normal model~// Stat. Pap., 2012. 
Vol.~53. P.~895--914.

\bibitem{Korolev2011-i} \Au{Королев~В.\,Ю.} 
Ве\-ро\-ят\-но\-ст\-но-ста\-ти\-сти\-че\-ские методы декомпозиции волатильности 
хаотических процессов.~--- М.: Изд-во Моск. ун-та, 2011. 512~с.

\bibitem{Ushakov2015} \Au{Ушаков В.\,Г., Ушаков Н.\,Г.} 
Об усреднении округленных данных~// Информатика и~её применения, 2015. Т.~9. 
Вып.~4. С.~106--109.

\bibitem{Ushakov2017a} \Au{Ушаков~В.\,Г., Ушаков~Н.\,Г.} 
Границы точ\-ности восстановления информации, 
теряемой при округлении результатов наблюдений~// 
Вестник Московского университета. Серия~15: Вычислительная математика и~кибернетика, 
2017. №\,2. С.~26--30.

\bibitem{Ushakov2017b} \Au{Ushakov~N.\,G., Ushakov~V.\,G.} 
Statistical analysis of rounded data: Recovering of information lost due to rounding~// 
J.~Korean Stat. Soc., 2017.  Vol.~46. No.\,3. P.~426--437.

\bibitem{Gorshenin2016} \Au{Gorshenin~A.\,K., Korolev~V.\,Yu.} 
A~noising method for the identification of the stochastic structure of 
information flows~// Comm. Com. Inf. Sc., 2017. 
Vol.~678. P.~279--289.

\bibitem{Gorshenin2013} 
\Au{Gorshenin~A., Korolev~V.} Modelling of statistical
fluctuations of information flows by mixtures of gamma distributions~// 
27th European Conference on Modelling and Simulation Proceedings.~--- 
Dudweiler, Germany: Digitaldruck Pirrot GmbHP, 2013. P.~569--572.
 \end{thebibliography}

 }
 }

\end{multicols}

\vspace*{-6pt}

\hfill{\small\textit{Поступила в~редакцию 03.08.18}}

\vspace*{6pt}

%\newpage

%\vspace*{-24pt}

\hrule

\vspace*{2pt}

\hrule

\vspace*{-2pt}


\def\tit{DATA NOISING BY FINITE NORMAL AND~GAMMA MIXTURES WITH~APPLICATION 
TO~THE~PROBLEM OF~ROUNDED OBSERVATIONS}


\def\titkol{Data noising by finite normal and~gamma mixtures with~application 
to~the~problem of~rounded observations}



\def\aut{A.\,K.~Gorshenin}

\def\autkol{A.\,K.~Gorshenin}

\titel{\tit}{\aut}{\autkol}{\titkol}

\vspace*{-11pt}


\noindent
Institute of Informatics Problems, Federal Research Center ``Computer Science and
Control'' of the Russian Academy of Sciences, 44-2~Vavilov Str., Moscow 119333,
Russian Federation


\def\leftfootline{\small{\textbf{\thepage}
\hfill INFORMATIKA I EE PRIMENENIYA~--- INFORMATICS AND
APPLICATIONS\ \ \ 2018\ \ \ volume~12\ \ \ issue\ 3}
}%
 \def\rightfootline{\small{INFORMATIKA I EE PRIMENENIYA~---
INFORMATICS AND APPLICATIONS\ \ \ 2018\ \ \ volume~12\ \ \ issue\ 3
\hfill \textbf{\thepage}}}

\vspace*{3pt}



\Abste{In many real problems, statistical analysis of data containing additional 
measurement errors, including rounding, is performed, which in some situations can 
lead to sufficiently significant distortions. In this paper, estimates for an 
unknown expectation of observations are obtained for one of the possible 
rounding models under the assumption that the original data are additionally 
noised with random variables having distributions of the type of finite 
mixtures of normal and gamma laws. Confidence intervals for an 
unknown expectation are constructed using the refined estimate for 
the variance of the integer part of the random variable. An algorithm 
for determining the value of the parameter of artificial noise, which 
can be added to the initial data to improve the quality of the 
method of moving separation of mixtures, is discussed.}


\KWE{noisy data; rounded data; finite normal mixtures; finite gamma mixtures; 
confidence intervals; moving separation of mixtures}



\DOI{10.14357/19922264180304}

%\vspace*{-14pt}

\Ack
\noindent
The research was supported by the Russian Science Foundation (project 18-71-00156).



%\vspace*{6pt}

  \begin{multicols}{2}

\renewcommand{\bibname}{\protect\rmfamily References}
%\renewcommand{\bibname}{\large\protect\rm References}

{\small\frenchspacing
 {%\baselineskip=10.8pt
 \addcontentsline{toc}{section}{References}
 \begin{thebibliography}{99}
\bibitem{1-gor-1}
\Aue{Wright,~D.\,E., and I.~Bray.} 2003.
A~mixture model for rounded data.  \textit{J.~Roy. Stat. Soc.~D Sta.} 52:3--13.

\bibitem{2-gor-1}
\Aue{Bai,~Z., S.~Zheng, B.~Zhang, and G.~Hu.} 2009. 
Statistical analysis for rounded data. \textit{J.~Stat. Plan. 
Infer.} 139(8):2526--2542.

\bibitem{3-gor-1}
\Aue{Zhang,~B., T.~Liu, and Z.\,D.~Bai.} 2010. 
Analysis of rounded data from dependent sequences. 
\textit{Ann. I.~Stat. Math.} 62(6):1143--1173.

\bibitem{4-gor-1}
\Aue{Zhao,~N., and Z.~Bai.} 2012. Analysis of rounded data in mixture normal model. 
\textit{Stat. Pap.} 53:895--914.

\bibitem{5-gor-1}
\Aue{Korolev, V.\,Yu.} 2011. 
\textit{Veroyatnostno-statisticheskie metody dekompozitsii volatil'nosti 
khaoticheskikh protsessov} [Probabilistic and statistical methods of 
decomposition of volatility of chaotic processes]. 
Moscow: Moscow University Publishing House. 512~p.

\bibitem{6-gor-1}
\Aue{Ushakov, V.\,G., and N.\,G.~Ushakov.} 
2015. Ob usrednenii okruglennykh dannykh [On averaging of rounded data].
\textit{Informatika i~ee Primeneniya~--- Inform. Appl.} 9(4):106--109.

\bibitem{7-gor-1}
\Aue{Ushakov,~V.\,G., and N.\,G.~Ushakov.} 2017. 
Boundaries of the precision of restoring information lost after rounding
 the results from observations. 
 \textit{Moscow University Computational Math. Cybernetics} 41(2):76--80.

\bibitem{8-gor-1}
\Aue{Ushakov,~N.\,G., and  V.\,G.~Ushakov.} 2017. 
Statistical analysis of rounded data: Recovering of information lost due to rounding. 
\textit{J.~Korean Stat. Soc.} 46(3):426--437.

\bibitem{9-gor-1}
\Aue{Gorshenin,~A.\,K., and V.\,Yu.~Korolev.} 2016. 
A~noising method for the identification of the stochastic structure of information 
flows. \textit{Comm. Com. Inf. Sc.} 678:279--289.

\bibitem{10-gor-1}
\Aue{Gorshenin,~A., and V.~Korolev.} 2013.  Modelling of statistical fluctuations of
information flows by mixtures of gamma distributions. 
\textit{27th European Conference on Modelling and Simulation Proceedings}. 
Dudweiler, Germany: Digitaldruck Pirrot GmbHP. 569--572.

\end{thebibliography}

 }
 }

\end{multicols}

\vspace*{-6pt}

\hfill{\small\textit{Received August 3, 2018}}

%\pagebreak

%\vspace*{-18pt}

\Contrl

\noindent
\textbf{Gorshenin Andrey K.} (b.\ 1986)~--- Candidate of Science (PhD) in physics and
mathematics, associate professor, leading scientist, Institute of Informatics Problems,
Federal Research Center ``Computer Science and Control'' of the Russian Academy of
Sciences, 44-2 Vavilov Str., Moscow 119333, Russian Federation; 
\mbox{agorshenin@frccsc.ru}
\label{end\stat}

\renewcommand{\bibname}{\protect\rm Литература}       %8рис
\def\stat{kirikov}

\def\tit{<<ВИРТУАЛЬНЫЙ КОНСИЛИУМ>>~--- ИНСТРУМЕНТАЛЬНАЯ 
СРЕДА ПОДДЕРЖКИ ПРИНЯТИЯ 
  СЛОЖНЫХ ДИАГНОСТИЧЕСКИХ РЕШЕНИЙ$^*$}

\def\titkol{<<Виртуальный консилиум>>~--- инструментальная 
среда поддержки принятия сложных диагностических решений}

\def\aut{И.\,А.~Кириков$^1$, А.\,В.~Колесников$^2$, С.\,В.~Листопад$^3$, 
С.\,Б.~Румовская$^4$}

\def\autkol{И.\,А.~Кириков, А.\,В.~Колесников, С.\,В.~Листопад, 
С.\,Б.~Румовская}

\titel{\tit}{\aut}{\autkol}{\titkol}

\index{Кириков И.\,А.}
\index{Колесников А.\,В.}
\index{Листопад С.\,В.} 
\index{Румовская С.\,Б.}
\index{Kirikov I.\,А.}
\index{Kolesnikov А.\,V.}
\index{Listopad S.\,V.}
\index{Rumovskaya S.\,B.}


{\renewcommand{\thefootnote}{\fnsymbol{footnote}} \footnotetext[1]
{Работа выполнена при частичной поддержке РФФИ (проект 16-07-00272 А).}}


\renewcommand{\thefootnote}{\arabic{footnote}}
\footnotetext[1]{Калининградский филиал Федерального исследовательского центра <<Информатика и~управление>> 
Российской академии наук, \mbox{baltbipiran@mail.ru}}
\footnotetext[2]{Балтийский Федеральный университет
имени  И.~Канта, Калининградский филиал Федерального 
исследовательского центра <<Информатика и~управление>> Российской академии наук, 
\mbox{avkolesnikov@yandex.ru}}
\footnotetext[3]{Калининградский филиал Федерального исследовательского центра <<Информатика и~управление>> 
Российской академии наук, \mbox{ser-list-post@yandex.ru}}
\footnotetext[4]{Калининградский филиал Федерального исследовательского центра <<Информатика 
и~управление>> Российской академии наук, \mbox{sophiyabr@gmail.com}}
 
 \vspace*{-3pt}
 
  \Abst{Рассматривается проблема принятия индивидуального решения при диагностике 
пациентов в~ам\-бу\-ла\-тор\-но-по\-ли\-кли\-ни\-че\-ских учреждениях на примере 
диагностики артериальной гипертензии (АГ). Предлагается повысить качество принятия 
индивидуального решения за счет консультаций с~системой поддержки принятия  
решения~--- <<Виртуальным консилиумом>>, моделирующим коллективный интеллект 
врачей стационара многопрофильного больничного учреждения. Приведены результаты 
проектирования и~экспериментального исследования лабораторного прототипа 
<<Виртуального консилиума>>.}

  \KW{система поддержки принятия решения; виртуальный консилиум; функциональная 
гибридная интеллектуальная система}

\DOI{10.14357/19922264160311} 


\vskip 10pt plus 9pt minus 6pt

\thispagestyle{headings}

\begin{multicols}{2}

\label{st\stat}
  

\section{Введение}

  Степень исследования, понимания и~качества диагностики проблемных сред и~их 
окружения отражена в~научной картине мира, онтологи\-зи\-ру\-ющей его представления 
и~делающей рассуждения и~целенаправленную деятельность <<зависимыми>> от них. 
В~искусственном интеллекте понятию <<картина мира>> соответствует понятие <<модель 
внешнего мира>> М.\,Г.~Га\-азе-Рап\-по\-пор\-та и~Д.\,А.~Поспелова~[1]. 
  
  Новая картина мира складывается из многочисленных теорий и~взглядов: <<ноосфера>>, 
<<разумный мир>> (В.\,И.~Вернадский, Н.\,Н.~Моисеев, А.\,В.~Поздняков); <<мир 
диалектики>>~--- мир диалога разных логик (Е.\,Л.~Доценко); социальная парадигма 
искусственного интеллекта (<<The society of mind>>) М.~Минского;  
сис\-тем\-но-ор\-га\-ни\-за\-ци\-он\-ный подход в~искусственном интеллекте 
В.\,Б.~Тарасова; теория иерархических многоуровневых систем М.~Месаровича, Д.~Мако 
и~И.~Такахары и~др.~--- и~укладывается в~семь постулатов~[2]: (1)~признание 
гетерогенности мира и~любого объекта, разнообразия жизни; (2)~неопределенность границ 
объектов и~связь <<всего со всем>>; (3)~относительность любой иерархии и~горизонтальные 
связи; (4)~дополнительность и~сотрудничество; (5)~полицентризм; (6)~относительность 
знания; (7)~соответствие управления сложности объекта. 
  
  Сложная задача диагностики АГ (СЗДАГ)~---
  за\-да\-ча-сис\-те\-ма, вклю\-ча\-ющая диагностические и~технологические подзадачи, 
повышающие эффективность обработки симптоматической информации о пациенте. 
Разнообразие подзадач СЗДАГ с~различными характеристическими свойствами требует 
разнообразия соответствующих методов принятия решений, системного анализа, 
искусственного интеллекта и~инженерии знаний. 
  
  Анализ результатов влияния новой картины мира на ментальную составляющую 
врачебной практики и~медицинской информатики~[3] показал, что, несмотря на стремление 
биомедицины к~гетерогенности восприятия организма человека и~процесса его диагностики 
в~рамках семипостулатной картины мира, человек по-преж\-не\-му остается 
<<расчлененным>> объектом познания, что сформировало <<узких>> специалистов, 
поглощенных решением частных задач. Новый тип ученого <<праг\-ма\-ти\-ка-фак\-то\-ло\-га>> 
утратил системное мышление, перестал задумываться над тем, что делается <<вокруг>> 
и~какое значение могут иметь добытые им факты для понимания работы организма в~целом. 
В~этой связи\linebreak\vspace*{-12pt}

\pagebreak

\noindent
 очевидна необходимость перехода от методов <<конкурентной>> диагностики 
к системному мышлению и~методам гетерогенной диагностики.
  
  В~[3--5] представлены результаты системного анализа СЗДАГ, следуя 
  проблемно-структурной (ПС) методологии, этапы~1--5~[6]: идентификация, редукция сложной задачи, 
спецификация диагностических подзадач, выбор методов их решения, а~также проверка 
неоднородности сложной задачи диагностики. Работы~[3--5] подтвердили релевантность 
применения междисциплинарных инструментариев для решения 
СЗДАГ, мо\-де\-ли\-ру\-ющих разнообразие информации, 
сотрудничество, дополнительность и~относительность знаний, сочетающих методы 
и~методики системного анализа диагностической проблемы с~динамическим синтезом 
метода ее решения и~имитацией работы искусственного гетерогенного коллектива~--- 
<<виртуального консилиума>>.
  
  Разнообразие~--- признак, проявление гетерогенности. Следствие закона необходимого 
разнообразия У.\,Р.~Эшби констатирует, что управ\-ле\-ние обеспечивается, если разнообразие 
средств управ\-ля\-юще\-го не меньше разнообразия управ\-ля\-емой им ситуации. Для отображения 
в информатике ситуативного разнообразия в~естественных гетерогенных системах в~[6] 
введены модели <<гетерогенная, неоднородная задача>> и~<<гомогенная, однородная 
задача>>, а~сам закон трактуется так: только разнообразная, скоординированная клиническая 
деятельность, элементы которой в~комбинации решают одну задачу, сделает результат 
диагностики качественно лучше в~обществе с~новой научной картиной мира. Специфике 
такой работы соответствует коллективный труд экспертов в~малых группах за круглым 
столом~--- консилиумы, совещания, естественные гетерогенные системы для решения 
сложных задач~\cite{3-kir}, где на первый план выходят знания и~опыт лица, принимающего 
решения (ЛПР), и~экспертов.
  
  \begin{figure*} %fig1
\vspace*{1pt}
 \begin{center}  
\mbox{%
 \epsfxsize=147.497mm
 \epsfbox{kir-1.eps}
 }
\end{center} 
%\vspace*{-9pt}
%\Caption{Концептуальная модель процесса диагностики артериальной гипертензии: в~многопрофильном 
%стационарном больничном учреждении~(\textit{а}); в~амбулаторно-поликлиническом~(\textit{б})}
  \end{figure*}

  \addtocounter{figure}{1}
  
  Настоящая работа~--- продолжение работ~[3--5,\linebreak 7] и~имеет целью представить: (1)~результаты 
исследования процесса диагностики АГ  
в~ле\-чеб\-но-про\-фи\-лак\-ти\-че\-ских больничных учреждениях (ЛПУ) широкого 
профиля~--- предлагается повысить эффективность и~качество индивидуальных 
диагностических решений в~ЛПУ широкого профиля ам\-бу\-ла\-тор\-но-по\-ли\-кли\-ни\-че\-ско\-го 
характера (рис.~1,\,\textit{а}) за счет внедрения информационной технологии 
<<Виртуальный консилиум>>, моделирующей коллективное обсуждение; 
(2)~архитектуру <<Виртуального консилиума>> и~результаты лабораторных экспериментов с~
его интегрированными моделями (первые результаты лабораторных экспериментов 
приведены в~[7]).

\section{Диагностика артериальной гипертензии в~многопрофильном 
стационарном больничном учреждении и~в~амбулаторно-поликлиническом 
учреждении}

\vspace*{-9pt}


  В~[8, 9] представлены результаты исследования процесса диагностики 
АГ в~Калининградской клинической областной больнице (КОКБ) 
(см.\ рис.~1,\,\textit{б}) и~ее Диагностическом центре (см.\ рис.~1,\,\textit{а}). 

Для формирования 
полного дифференциального диагноза АГ коллективом врачей во главе с~лечащим врачом, 
ЛПР-кар\-дио\-ло\-гом, в~стационаре привлекаются до тринадцати вра\-чей-экс\-пер\-тов~--- носителей 
знаний из различных разделов медицины: невролог, нефролог, сосудистый хирург, уролог, 
психолог, педиатр, аку\-шер-ги\-не\-ко\-лог, онколог, окулист, врачи функциональной 
диагностики, эндокринолог, терапевт, кардиолог. 

Для исследований выбраны шесть 
специалистов (см.\ рис.~1,\,\textit{б}), решающих двенадцать функциональных подзадач 
(рис.~\ref{f2-kir}), возникающих в~90\%~случаев диагностики АГ, 
каждый из которых формирует промежуточные заключения о~состоянии объекта 
диагностики в~своей области медицинских зна\-ний. 
{\looseness=1

}

Полученные исходные данные об объекте 
диагностики разнородны (содержатся в~медицинской карте): количественные,  
ви\-зу\-аль\-но-графиче\-ские параметры (детерминированные переменные),\linebreak 
лингвистические четкие и~нечеткие переменные. Лицо, при\-ни\-ма\-ющее решение, изучает в~медицинской карте 
симптомы и~частные диагностические мнения вра\-чей-экс\-пер\-тов, множество которых 
подбирает сам, и~ставит заключительный диагноз. Вра\-чам-экс\-пер\-там доступны симптомы 
и~мнения других врачей-экспертов из медицинской карты.
\mbox{Лицо}, при\-ни\-ма\-ющее решение, и~вра\-чи-экс\-пер\-ты 
обследуют пациента и~формируют диагностические заключения согласно нормативным 
документам, например~[10]. В~ЛПУ широкого профиля (см.\ рис.~1,\,\textit{а}) ЛПР~--- это врач 
общей практики или терапевт (иногда кардиолог, но зачастую без опыта работы, к~которому 
направляет терапевт сразу же при выявлении повышенного артериального давления), это 
врач <<праг\-ма\-тик-фак\-то\-лог>>~\cite{9-kir}, объединяющий в~себе роли вра\-ча-ЛПР  
и~вра\-чей-экс\-пер\-тов узкой специализации.

\end{multicols}

\begin{figure} %fig2
\vspace*{1pt}
 \begin{center}  
\mbox{%
 \epsfxsize=163.044mm
 \epsfbox{kir-2.eps}
 }
\end{center} 
\vspace*{-9pt}
\Caption{Архитектура ВКДАГ }
\label{f2-kir}
\vspace*{3pt}
\end{figure}

\begin{multicols}{2}
  

  Исследования диагностического процесса на материалах Диагностического центра КОКБ 
по модели на рис.~1,\,\textit{а} показали, что~70\%~пациентов с~АГ 
амбулаторно-поликлинического учреждения не знают о своем заболевании, в~то время как в~стационарных 
медицинских учреждениях (см.\ рис.~1,\,\textit{б}) практически в~100\%~случаев имеет место 
как адекватное проведение, так и~отображение в~медицинских картах симптоматических 
данных обследования с~подтверждением диагноза  
ла\-бо\-ра\-тор\-но-ин\-ст\-ру\-мен\-таль\-ны\-ми методами исследования. 
  
  В этой связи предлагается повысить эффективность и~качество индивидуальных 
диагностических решений в~ЛПУ широкого профиля амбула\-тор\-но-по\-ли\-кли\-ни\-че\-ско\-го 
характера (см.\ рис.~1,\,\textit{а}) за счет внед\-ре\-ния информационной технологии 
<<Виртуальный консилиум>> (см.\ рис.~\ref{f2-kir}), моделирующей коллективное обсуждение, 
обладающего синергией, опытом и~знаниями в~решении подзадач диагностики 
АГ в~стационаре (см.\ рис.~1,\,\textit{б}). 


  

  
\section{Инструментальная среда <<Виртуальный консилиум для~диагностики 
артериальной гипертензии>>}

\vspace*{-18pt}

  Инструментальная среда <<Виртуальный консилиум>>, архитектура которой 
представлена на рис.~\ref{f2-kir}, а~структура в~\cite{7-kir}, ограничена пациентами 
стар\-ше~18~лет, без особых состояний, нет распознавания снимков, не предусматривается 
назначение лечения и~не диагностируется ряд симптоматических артериальных гипертензий. 

Архитектура <<Виртуального консилиума для диагностики артериальной гипертензии>> 
(ВКДАГ) включает межмодульные интерфейсы~$\zeta^u$ для модулей, реализованных 
посредством различных методологий гибридных интеллектуальных сис\-тем (\mbox{ГиИС}) 
(генетические алгоритмы ($g$), нечеткие 
сис-\linebreak\vspace*{-12pt}

\pagebreak

\end{multicols}

\begin{table*}\small
%\vspace*{-12pt}
\begin{center}
\Caption{Описание блоков архитектуры ВКДАГ}
\vspace*{2ex}

\begin{tabular}{|p{30mm}|p{40mm}|p{39mm}|p{39mm}|}
\hline
\multicolumn{1}{|c|}{Наименование блока}&\multicolumn{1}{c|}{Функции}&\multicolumn{1}{c|}{Вход}&\multicolumn{1}{c|} 
{Выход}\\
\hline
Технологический модуль $i$-й&
Организация эффективной обработки данных и~знаний, выбирается для 
включения в~функциональную \mbox{ГиИС}~--- построение информативного набора 
признаков для диагностики&Популяция 
индивидуумов, накладывающихся как маска на $i$-й функциональный модуль&
Наилучшая особь с~оптимальным набором признаков~--- накладывается как 
маска на $i$-й функциональный модуль\\
\hline
Функциональный модуль $i$-й&Классификация состояния здоровья пациента в~рамках 
\mbox{$i$-й} диагностической 
подзадачи, выбирается для включения в~функциональную \mbox{ГиИС} &
Подмножество $i$-е симптомов с~интерфейса 
пользователя&Частное $i$-е заключение о~со\-сто\-янии здоровья пациента\\
\hline
Функциональный модуль {HCCCC}, моделирующий ЛПР&
Формирование заключительного диагноза 
АГ (всегда в~составе <<Виртуального консилиума>>)&Подмножество симптомов 
с~интерфейса пользователя, множество выходов функциональных модулей&
Заключительный диагноз АГ \\
\hline
Функциональный модуль {ИНСРЭКГ}&Классификация патологического состояния пациента по его 
электрокардиограмме&\multicolumn{2}{p{60mm}|}{Рассмотрены подробно в~\cite{4-kir}}\\
\cline{1-2}
Функциональный модуль {ИНССМАД}&Прогноз нормальных зна\-чений суточного мониторирования 
артериального давле\-ния и~вычисление отклонения &\multicolumn{2}{c|}{\ }\\
\hline
Интерфейс модификации структуры {ВКДАГ}&Исключение из диагностики модулей, решающих не 
интересующие пользователя подзадачи &
Выбранные пользователем подзадачи диагностики &
Функциональная ГиИС, 
синтезированная посредством алгоритма из~\cite{4-kir}\\
\hline
Интерфейс пользователя <<Диагноз>>&Визуализация результатов диагностики и~корректировка их 
пользователем &Заключительный диагноз от функционального модуля НСССС&Отчет, содержащий 
множество симптомов и~диагноз\\
\hline
Интерфейс пользователя &Ввод информации о~со\-сто\-янии здоровья пациента &
Множество значений 
показателей состояния здоровья пациента&
Показатели состояния здоровья пациента, распределенные по 
функциональным модулям \\
\hline
Модификация интерфейса пользователя&Деактивация элементов на интерфейсе пользователя для ввода 
значений показателей состояния здоровья&Множество выходов технологических модулей&Частично 
деактивированный интерфейс пользователя \\
\hline
\end{tabular}
\end{center}
\end{table*}

\begin{multicols}{2}

\noindent 
те\-мы ($f$), искусственные нейронные сети ($n$)).
 В~библиотеке модулей диагностики 
и~препро\-цессии хранятся заранее инициализированные\linebreak в~программной среде 
функциональные и~технологические модели. 
По умолчанию все модули включены 
в~структуру <<Виртуального консилиума>>, их описание пред\-став\-ле\-но в~табл.~1. %\\[-15pt]
%
      <<Виртуальный консилиум>> (см.\ рис.~\ref{f2-kir}) запускает интерфейс пользователя, 
ЛПР-вра\-ча~--- <<{Интерфейс модификации структуры ВКДАГ}>>, посредством 
которого включаются функциональные 
 и~технологические модули в~работу сис\-те\-мы: модуль 
<<Анализ СМАД>>, модуль <<Распознавание ЭКГ>>, модули технологических подзадач из 
группы <<Построение информативного набора признаков\linebreak (симптомов) при диагностике 
заболеваний>> и~модули подзадач из группы <<Диагностика критериев оценки 
сер\-деч\-но-со\-су\-ди\-сто\-го риска и~вторичной АГ у~пациента>> ({ДАГ}$_1$, \ldots , {ДАГ}$_9$): 
диагностики\linebreak поражений ор\-га\-нов-ми\-ше\-ней, факторов риска, цереброваскулярных 
болезней, метаболического синд\-ро\-ма и~сахарного диабета, заболеваний периферических 
артерий, ишемической болезни сердца,\linebreak эндокринной АГ, паренхиматозной нефропатии 
и~реноваскулярной АГ соответственно. Все выбранные $i$-е технологические модули 
запускаются, решают соответствующую подзадачу и~передают информацию на блок 
<<{Модификация интерфейса пользователя}>>. Он деактивирует показатели 
со\-сто\-яния здоровья на <<{Интерфейсе пользователя для\linebreak ввода значений показателей 
состояния здоровья пациента}>> и~корректирует работу $i$-го функционального модуля 
подзадач {ДАГ}$_1$, \ldots\linebreak \ldots , {ДАГ}$_9$. Далее активируется откорректированный 
интерфейс, вводятся симптомы, которые передаются функциональным нечетким модулям, 
решающим подзадачи {ДАГ}$_1$, \ldots , {ДАГ}$_9$\linebreak (моделируют принятие 
решения экспертами, врачами смежных специальностей~--- кардиологом как экспертом, 
неврологом, нефрологом, терапевтом, эндокринологом, урологом). Последние в~свою 
очередь передают информацию о~патологиях, выявленных ими у~пациента, 
функциональному модулю {НСССС} (моделирует принятие решения ЛПР~---  
вра\-чом-кар\-дио\-ло\-гом), решающему подзадачу <<Оценка степени и~стадии 
артериальной гипертензии, степени риска сер\-дечно-сосу\-ди\-стых заболеваний>>. 

В~библиотеке ВКДАГ есть еще два функциональных модуля (см.\ табл.~1), вклю\-ча\-ющих\-ся 
в~работу консилиума посредством <<{Интерфейса модификации структуры 
ВКДАГ}>>: 
      \begin{enumerate}[(1)]
      \item {ИНСРЭКГ}, передающий информацию на модули диагностики поражений 
ор\-га\-нов-ми\-ше\-ней (на рис.~\ref{f2-kir}~--- это {НСДАГ}$_1$), цереброваскулярных 
болезней ({НСДАГ}$_3$) и~ишемической болезни сердца ({НСДАГ}$_6$); 
      \item {ИНССМАД}, формирующий информацию о~нормальных значениях 
суточного артериального давления на функциональный модуль {НСССС}.
      \end{enumerate}
      
\section{Экспериментальное лабораторное исследование программной 
реализации прототипа инструментальной среды <<Виртуальный консилиум>>}
  
  Экспериментальное лабораторное исследование программной реализации 
исследовательского прототипа функциональной гибридной интеллектуальной системы 
ВКДАГ для поддержки принятия сложных диагностических решений необходимо для 
подтверждения его релевантности~[3--5, 7] реальной ситуации диагностики АГ. В~[4] 
пред\-став\-ле\-на информация по особенностям функциональных и~технологических моделей 
гетерогенного модельного поля ВКДАГ, а~в~[7]~--- информация по их инициализации 
в~среде MATLAB-Simulink, результаты исследований качества работы каждой модели 
гетерогенного модельного поля <<Виртуального консилиума>> автономно, а~также 
подтверждена их релевантность работе экспертов~--- врачей узкой специализации, что 
предотвращает распространение ошибок работы автономных моделей на работу 
интегрированной модели. 

В~настоящей работе приведены результаты исследования качества 
интегрированных моделей, синтезированных <<Виртуальным консилиумом>>\linebreak 
и~моделирующих дополнительность и~сотрудничество, которые имитируют коллективные 
рас\-суж\-де\-ния специалистов при постановке диагноза. 

В~табл.~2 представлены критерии 
и~результаты тес\-ти\-ро\-ва\-ния интегрированных моделей <<Виртуального консилиума>> 
с~различными комбинациями знаний врачей, классифицирующих патологическое состояние 
пациента. Порядок работы моделей гетерогенного модельного поля \mbox{ВКДАГ}: запускаются 
модели первой очереди~--- модели технологических элементов {ГАППС}$_{1\mbox{--}9}$, 
корректирующие множества входных переменных моделей {НСДАГ}$_{1\mbox{--}9}$ 
и~{НСССС}; обработка информации передается функциональным элементам: модели 
второй очереди <<отправляют>> информацию на модели третьей, пятой, шес\-той и~седьмой 
очередей~--- \mbox{ИНСРЭКГ} (модель, решающая задачу распознавания электрокардиограммы (ЭКГ)), 
{ИНССМАД} (формирует оптимальные множества показателей суточного давления), 
{НСДАГ}$_9$, {НСДАГ}$_2$ и~{НСДАГ}$_6$; третья\linebreak очередь содержит 
модели НСДАГ$_4$ и~НСДАГ$_5$, передающие выходную информацию на вход моделей четвертой 
и~седьмой очередей; четвертая очередь содержит модель {НСДАГ}$_8$, пе\-ре\-да\-ющую 
информацию  модели пятой очереди {НСДАГ}$_1$, которая в~свою очередь передает 
информацию\linebreak {НСДАГ}$_3$ (шес\-тая очередь); от {НСДАГ}$_3$ передается 
информация {НСДАГ}$_7$ (седьмая очередь); последней запускается модель 
{НСССС}, формирующая заключительный диагноз, на вход которой передается 
выходная информация функциональных моделей вто\-рой--седь\-мой очередей.
  
  Таким образом: (1)~без знаний кардиолога, или нефролога, или эндокринолога 
сред\-не\-квад\-ратическая ошибка наибольшая~--- 0,697; 0,448 и~0,211 соответственно, 
и~объясняется это тем, что кардиолог играет ключевую роль в~обработке ин\-формации, 
поступающей от других врачей\linebreak\vspace*{-12pt}


\pagebreak

\end{multicols}

\begin{table}\small
\begin{center}
\Caption{Параметры и~результаты тестирования интегрированных моделей }
\vspace*{2ex}

\begin{tabular}{|p{66mm}|p{88mm}|}
\hline
\multicolumn{1}{|c|}{\tabcolsep=0pt\begin{tabular}{c}Наименование параметров\\ 
и результатов тестирования\end{tabular}}&
\multicolumn{1}{c|}{Значения параметров и~результатов 
тестирования}\\
\hline
Объем тестовой выборки ВКДАГ, интегрирующего знания всех шести врачей&800 наблюдений~--- 500 с~
диагнозами эссенциальной АГ и~300 с~диагнозами вторичной АГ\\
\hline
Объем тестовой выборки ВКДАГ, интегрирующего знания менее шести врачей&400 наблюдений~--- 200 с~
диагнозами эссенциальной АГ и~200 с~диагнозами вторичной АГ\\
\hline
Источник формирования тестовой вы\-борки&Архив медицинских карт пациентов 1-го кардиологического 
отделения КОКБ\\
\hline
Элемент тестирующей последова\-тель\-ности&
Содержит множество нечетких лингвистических переменных и~вектор образа электрокардиограммы (может отсутствовать)\\
\hline
Эталонный диагноз&Результаты деятельности лечащего вра\-ча-кар\-дио\-ло\-га, подводящего общий итог~--- 
дифференциальный диагноз АГ\\
\hline
Критерии тестирования&Среднеквадратическая ошибка $f$ классификации состояния здоровья пациента~[7]\\
\hline
$f$(шесть врачей)&0,0837\\
\hline
$f$(без кардиолога)&0,697\\
\hline
$f$(без нефролога)&0,448 (в остальных 55,2\% случаях диагноз не вызовет доверия)\\
\hline
$f$(без терапевта)&0,151\\
\hline
$f$(без невролога)&0,149\\
\hline
$f$(без эндокринолога)&0,211 (в остальных 78,9\% случаях диагноз не вызовет доверия)\\
\hline
$f$(без сосудистого хирурга)&0,0798\\
\hline
$f$(без знаний терапевта, невролога, неф\-ро\-ло\-га, эндокринолога, сосудистого хирурга)&0,711\\
\hline
$f$(без знаний терапевта, невролога, эндокринолога, сосудистого хирурга)&0,485\\
\hline
$f$(без знаний невролога, эндокринолога, сосудистого хирурга)&0,334\\
\hline
$f$(без знаний невролога, сосудистого хи\-рурга)&0,167\\
\hline
\end{tabular}
\end{center}
\end{table}

\begin{multicols}{2}


\noindent
 и~от ла\-бораторных исследований, и~в~постановке
заключительного диагноза, а~нефролог и~эндокринолог~--- в~исключении вторичной 
АГ; (2)~знания врача~--- сосудистого хирурга не влияют на 
результаты работы <<Виртуального консилиума>>, и~объясняется это тем, что знания 
сосудистого хирурга, касающиеся диагностики АГ, составляют только~20\% базы знаний 
нечеткой системы, распознающей заболевания периферических артерий (ассоциативные 
клинические состояния), встречающихся не более чем у~10\% населения~\cite{11-kir}, 
и~в~тес\-то\-вую выборку не попала ни одна карта с~данными заболеваниями; (3)~чем больше 
численный состав <<Виртуального консилиума>>, тем с~меньшей среднеквадратической 
ошибкой он классифицирует состояние здоровья пациента; (4)~<<Виртуальный консилиум>> 
в~со\-ста\-ве шести врачей диагностирует АГ со среднеквадратической 
ошибкой постановки диагноза $f = 0{,}0837$, т.\,е.\ дает диагноз, верный в~84\% слу\-чаях. 
{\looseness=1

}
  
  Поскольку <<Виртуальный консилиум>> разра\-ботан на основе всероссийских~\cite{9-kir} 
и~между\-народных рекомендаций по диагностике АГ и~со\-пут\-ст\-ву\-ющих заболеваний, 
которых должен придерживать\-ся каж\-дый врач в~своей практике, при переносе \mbox{ВКДАГ} 
в~другое больничное учреж\-де\-ние необходимо пред\-оста\-вить врачам данного учреждения 
протоколы подтверждения диагностических правил всех баз знаний экспериментальными 
данными из архива КОКБ для ознакомления 
и~внесения при необходимости коррективов в~связи с~возможными особенностями их 
контингента пациентов, а~также возможных требований по устранению ограничений 
системы со стороны персонала нового больничного учреждения. Значительной 
корректировки баз знаний не потребуется.
  
  Таким образом, лабораторные эксперименты с~прототипом <<Виртуального 
консилиума>> дали обнадеживающие результаты. 

Верное решение получено в~84\% 
случаев. В~ам\-бу\-ла\-тор\-но-кли\-ни\-че\-ских учреждениях диагноз не 
выявляется у~70\% пациентов в~основном по причине инертности врачей, недостатка опыта 
врачей узкой специализации и~нехватки кадров в~ЛПУ
широкого профиля, что по результатам экспериментов может быть устранено с~по\-мощью 
применения \mbox{ВКДАГ} во время приема пациентов с~подозрением на АГ.

\section{Заключение}

  Лабораторно подтверждена эффективность предлагаемого подхода для проектирования 
диагностических систем как гетерогенных искусственных диагностических систем со 
свойствами дополнительности, сотрудничества и~относительности\linebreak
 знаний, синтезирующих 
интегрированные методы и~модели, разнообразие которых устраняет разнообразие 
диагностической информации об организме человека~--- <<Виртуальных консилиумов>>,\linebreak 
моделиру\-ющих работу коллектива врачей в~многопрофильном стационарном больничном 
учреждении (на примере КОКБ) и~внедрение 
которых повыша\-ет эффективность и~качество индивидуальных диагностических решений 
в~ам\-бу\-ла\-тор\-но-по\-ли\-кли\-ни\-че\-ском учреждении широкого профиля (на примере 
Диагностического центра КОКБ), где заключение о состоянии больного из-за проблемы 
с~кадрами узкой специализации принимает чаще всего один специалист~--- терапевт или 
врач общей практики, иногда кардиолог, но без опыта работы.

{\small\frenchspacing
 {%\baselineskip=10.8pt
 \addcontentsline{toc}{section}{References}
 \begin{thebibliography}{99}
\bibitem{1-kir}
\Au{Гаазе-Раппопорт М.\,Г., Поспелов~Д.\,А.} От амебы до робота: модели поведения.~--- 
М.: Наука, 1987. 288~с.
\bibitem{2-kir}
\Au{Колесников А.\,В., Кириков~И.\,А., Листопад~С.\,В. %Румовская~С.\,Б. 
и~др.} Решение 
сложных задач коммивояжера методами функциональных гибридных интеллектуальных 
сис\-тем.~--- М.: ИПИ РАН, 2011. 295~с.
\bibitem{3-kir}
\Au{Кириков И.\,А., Колесников~А.\,В., Румовская~С.\,Б.} Исследование сложной задачи 
диагностики артериальной гипертензии в~методологии искусственных гетерогенных  
сис\-тем~// Системы и~средства информатики, 2013. Т.~23. №\,2. С.~81--99. doi: 
10.14357/08696527130208.
\bibitem{4-kir}
\Au{Кириков И.\,А., Колесников~А.\,В., Румовская~С.\,Б.} Функциональная гибридная 
интеллектуальная система для поддержки принятия решений при диагностике артериальной 
гипертензии~// Системы и~средства информатики, 2014. Т.~24. №\,1. С.~153--179. doi: 
10.14357/08696527140110.
\bibitem{5-kir}
\Au{Колесников А.\,В., Румовская~С.\,Б., Листопад~С.\,В., Кириков~И.\,А.} Системный 
анализ в~решении сложных диагностических задач~// Системный анализ и~информационные 
технологии (САИТ-2015): Тр. VI~Междунар. конф.~--- М.: 
ИСА РАН, 2015. Т.~1. С.~157--167.
\bibitem{6-kir}
\Au{Колесников А.\,В., Кириков~И.\,А.} Методология и~технология решения сложных задач 
методами функциональных гибридных интеллектуальных систем.~--- М.: ИПИ РАН, 2007. 
387~с.
\bibitem{7-kir}
\Au{Кириков И.\,А., Колесников~А.\,В., Румовская~С.\,Б.} Исследование лабораторного 
прототипа искусственной гетерогенной системы для диагностики артериальной 
гипертензии~// Системы и~средства информатики, 2014. Т.~24. №\,3. С.~131--143. doi: 
10.14357/08696527140309.
\bibitem{8-kir}
\Au{Румовская С.\,Б.} Методы и~средства информатики для диагностики 
артериальной гипертензии в~ле\-чеб\-но-про\-фи\-лак\-ти\-че\-ских учреждениях 
широкого профиля~// Задачи современной информатики (ЗСИ-2015): Тр. 2-й 
молодежной научной конф.~--- М.: ФИЦ ИУ РАН, 2015. 
С.~168--174.
\bibitem{9-kir}
\Au{Кириков~И.\,А., Румовская~С.\,Б.} Гетерогенная диагностика артериальной 
гипертензии~// Информатика, управление и~системный анализ (ИУСА-2016): Тр. 
4-й Всеросс. научной конф. молодых ученых с~международным участием.~--- 
Тверь: ТвГТУ, 2016. Т.~1. С.~180--188.
\bibitem{10-kir}
Комитет экспертов ВНОК. Диагностика и~лечение артериальной гипертензии. 
Российские рекомендации~// Системные гипертензии, 2010. Вып.~3. С.~5--26.
\bibitem{11-kir}
\Au{Галимзянов Ф.\,В.} Заболевания периферических артерий (клиника, 
диагностика, лечение)~// Международный журнал экспериментального образования, 
2014. Вып.~8. С.~113--114. 

\end{thebibliography}

 }
 }

\end{multicols}

\vspace*{-6pt}

\hfill{\small\textit{Поступила в~редакцию 18.06.16}}

\vspace*{8pt}

%\newpage

%\vspace*{-24pt}

\hrule

\vspace*{2pt}

\hrule

%\vspace*{8pt}



\def\tit{``VIRTUAL COUNCIL''~--- SOURCE ENVIRONMENT SUPPORTING 
COMPLEX DIAGNOSTIC DECISION MAKING}

\def\titkol{``Virtual council''~--- source environment supporting 
complex diagnostic decision making}

\def\aut{I.\,А.~Kirikov$^1$, А.\,V.~Kolesnikov$^{1,2}$, S.\,V.~Listopad$^1$, and 
S.\,B.~Rumovskaya$^1$}

\def\autkol{I.\,А.~Kirikov, А.\,V.~Kolesnikov, S.\,V.~Listopad, and 
S.\,B.~Rumovskaya}

\titel{\tit}{\aut}{\autkol}{\titkol}

\vspace*{-9pt}

\noindent
$^1$Kaliningrad Branch of the Federal Research Center ``Computer Science and 
Control'' of the Russian Academy\linebreak
$\hphantom{^1}$of Sciences, 5~Gostinaya Str., Kaliningrad 236000, 
Russian Federation
   
   \noindent
   $^2$Immanuel Kant Baltic Federal University, 14~Nevskogo Str., Kaliningrad 236041, 
Russian Federation


\def\leftfootline{\small{\textbf{\thepage}
\hfill INFORMATIKA I EE PRIMENENIYA~--- INFORMATICS AND
APPLICATIONS\ \ \ 2016\ \ \ volume~10\ \ \ issue\ 3}
}%
 \def\rightfootline{\small{INFORMATIKA I EE PRIMENENIYA~---
INFORMATICS AND APPLICATIONS\ \ \ 2016\ \ \ volume~10\ \ \ issue\ 3
\hfill \textbf{\thepage}}}

\vspace*{3pt}
  
    
  
\Abste{The paper considers the problem of individual decision making during 
diagnostics of 
patients in outpatient clinics by the example of arterial 
hypertension diagnostics. It is proposed to 
raise the quality of individual decision\linebreak\vspace*{-12pt}}

\Abstend{making by means of consultations with the ``Virtual council'' 
decision support system, which models the work of physician councils in inpatient multifield 
clinics. The results of development and experimental research of the 
laboratory prototype of ``Virtual council'' are presented.}

\KWE{decision support system; virtual council; functional hybrid intellectual system}

\DOI{10.14357/19922264160311} 

\vspace*{-9pt}

\Ack
\noindent
The work was performed with partial support of the Russian
Foundation for Basic Research (grant No.\,16-07-00272~А).


%\vspace*{3pt}

  \begin{multicols}{2}

\renewcommand{\bibname}{\protect\rmfamily References}
%\renewcommand{\bibname}{\large\protect\rm References}

{\small\frenchspacing
 {%\baselineskip=10.8pt
 \addcontentsline{toc}{section}{References}
 \begin{thebibliography}{99}
\bibitem{1-kir-1}
\Aue{Gaaze-Rappoport, M.\,G., and D.\,A.~Pospelov}. 1987. \textit{Ot ameby do robota: Modeli 
povedeniya} [From ameba to robotic mashine: Behavior model] Moscow: Nauka. 288~p.
\bibitem{2-kir-1}
\Aue{Kolesnikov,~A.\,V., I.\,A.~Kirikov, S.\,V.~Listopad, \textit{et al.}}. 2011. \textit{Reshenie 
slozhnykh zadach kommivoyazhera metodami funktsional'nykh gibridnykh intellektual'nykh 
sistem} [Solving of the complex traveling salesman problem by means of functional hybrid 
intellectual systems]. Moscow: IPI RAN. 295~p.
\bibitem{3-kir-1}
\Aue{Kirikov, I.\,A., A.\,V.~Kolesnikov, and S.\,B.~Rumovskaya}.\linebreak
 2013. Issledovanie slozhnoy 
zadachi diagnostiki arterial'noy gipertenzii v~metodologii iskusstvennykh geterogennykh sistem 
[Research of the complex problem at\linebreak diagnosing of the arterial hypertension within the 
methodology of artificial heterogeneous systems]. \textit{Sistemy i~Sredstva Informatiki~--- 
Systems and Means of Informatics} 23(2):81--99. doi: 10.14357/08696527130208.
\bibitem{4-kir-1}
\Aue{Kirikov, I.\,A., A.\,V.~Kolesnikov, and S.\,B.~Rumovskaya}.\linebreak
 2014. Funktsional'naya 
gibridnaya intellektual'naya sistema dlya podderzhki prinyatiya resheniya pri diagnostike 
arterial'noy gipertenzii [Functional hybrid intelligent decision support system for diagnosing of the 
\mbox{arterial} hypertension]. \textit{Sistemy i~Sredstva Informatiki~--- Systems and Means of Informatics} 
24(1):153--179. doi: 10.14357/08696527140110. 
\bibitem{5-kir-1}
\Aue{Kolesnikov, A.\,V., I.\,A.~Kirikov, S.\,V.~Listopad, and S.\,B.~Rumovskaya}. 2015. 
Sistemnyy analiz v~reshenii slozhnykh diagnosticheskikh zadach [Systems analysis for solving 
complex diagnostic tasks]. \textit{Tr. 6-y Mezhdunar. konf. ``Sistemnyy analiz i~informatsionnye 
tekhnologii''} [6th Conference (International) ``Systems Analysis and Information Technology'' 
Proceedings]. Moscow.  1:157--167.
\bibitem{6-kir-1}
\Au{Kolesnikov, A.\,V., and I.\,A.~Kirikov}. 2007. \textit{Metodologiya i~tekhnologiya resheniya 
slozhnykh zadach metodami funk\-tsi\-o\-nal'\-nykh gibridnykh intellektual'nykh sistem} [Methodology 
and technology for solving of complex problems using the methodology of functional hybrid 
artificial systems]. Moscow: IPI RAN. 387~p.
\bibitem{7-kir-1}
\Aue{Kirikov, I.\,A., A.\,V.~Kolesnikov, and S.\,B.~Rumovskaya}. 2014. Issledovanie 
laboratornogo prototipa iskusstvennoy geterogennoy sistemy dlya diagnostiki arterial'noy 
gipertenzii [Research of the laboratory prototype of the artificial heterogeneous system for 
diagnosing of the arterial hypertension]. \textit{Sistemy i~Sredstva informatiki~--- Systems and 
Means of Informatics} 24(3):131--143. doi: 10.14357/08696527140309.
\bibitem{8-kir-1}
\Au{Rumovskaya, S.\,B.} 2015. Metody i~sredstva informatiki dlya diagnostiki 
arterial'noy gipertenzii v~lechebno-profilakticheskikh uchrezhdeniyakh shirokogo profilya 
[Methods and tools of informatics for diagnostics of arterial hypertension in multiskilled 
medical preventive institution]. \textit{Tr. 2-y molodezhnoy nauchnoy konf. ``Zadachi 
sovremennoy informatiki''} [2nd Youth Conference ``Tasks of Modern Informatics'' 
Proceedings]. Moscow: FRC ``Computer Science and Control'' RAS. 168--174.
\bibitem{9-kir-1}
\Aue{Kirikov, I.\,A., and S.\,B.~Rumovskaya}. 2016. Geterogennaya diagnostika arterial'noy 
gipertenzii [Heterogeneous diagnostics of arterial hypertension]. \textit{Tr. 4-y Vseross. 
nauchnoy konf. molodykh uchenykh s~mezhdunarodnym uchastiem ``Informatika, 
upravlenie i~sistemnyy analiz''} [4th Youth Conference (International) ``Informatics, Control 
and Systems Analysis'' Proceedings]. Tver: Tver State Technical University. 1:180--188.
\bibitem{10-kir-1}
Komitet ekspertov VNOK [Committee of experts of All-Russia Scientific Society of Сardiologists]. 
2010. Diagnostika i~lechenie arterial'noy gipertenzii. Rossiyskie 
rekomendatsii [Diagnosing and treatment of arterial 
hypertension. Russian recommenation]. 
\textit{Sistemnye gipertenzii} [Systemic Hypertension] 3:5--26. 
\bibitem{11-kir-1}
\Aue{Galimzyanov, F.\,V.} 2014. Zabolevaniya perifericheskikh arteriy (Klinika, 
diagnostika, lechenie) [Peripheral vascular disease (Clinic, diagnostics, treatment]. 
\textit{Mezhdunarodnyy zhurnal eksperimental'nogo obrazovaniya} [Int. J.~Research 
Education] 8:113--114. 
   \end{thebibliography}

 }
 }

\end{multicols}

\vspace*{-9pt}

\hfill{\small\textit{Received June 18, 2016}}

\vspace*{-3pt}
    
  
  \Contr
  
  \noindent
  \textbf{Kirikov Igor A.}\ (b.\ 1955)~---
  Candidate of  Sciences (PhD) in technology; director, Kaliningrad Branch of the 
  Federal Research Center ``Computer Science and Control'' of the Russian Academy 
  of Sciences, 5~Gostinaya Str., Kaliningrad 236000,  Russian Federation; 
baltbipiran@mail.ru
  
  \pagebreak
%  \vspace*{3pt}
  
  \noindent
  \textbf{Kolesnikov Alexander V.}\ (b.\ 1948)~---
  Doctor of Sciences in technology; professor, 
Department of Telecommunications, 
 Immanuel Kant Baltic Federal University, 14~Nevskogo Str., Kaliningrad 236041, Russian Federation; senior scientist, Kaliningrad Branch of 
  the Federal Research Center ``Computer Science and Control'' of the Russian 
  Academy of Sciences, 5~Gostinaya Str., Kaliningrad 236000,  Russian Federation; 
  avkolesnikov@yandex.ru
  
  \vspace*{4pt}
  
  \noindent
  \textbf{Listopad Sergey V.}\ (b.\ 1984)~---
  Candidate of  Sciences (PhD) in technology; scientist, Kaliningrad Branch of the 
  Federal Research Center ``Computer Science and Control'' of the Russian Academy 
  of Sciences, 5~Gostinaya Str., Kaliningrad 236000,  Russian Federation;   
ser-list-post@yandex.ru
  
  \vspace*{4pt}
  
  \noindent
  \textbf{Rumovskaya Sophiya B.}\ (b.\ 1985)~--- programmer~I, Kaliningrad Branch 
  of the Federal Research Center ``Computer Science and Control'' of the Russian 
  Academy of Sciences, 5~Gostinaya Str., Kaliningrad 236000,  Russian Federation; 
  sophiyabr@gmail.com
  \label{end\stat}
  
  
  \renewcommand{\bibname}{\protect\rm Литература} %9рис
\def\stat{zatsman}

\def\tit{ТРАНСФОРМАЦИИ ОБЪЕКТОВ ПЕРВОГО И~ВТОРОГО ПОРЯДКА 
В~ЛЕКСИКОГРАФИЧЕСКОЙ ИНФОРМАЦИОННОЙ СИСТЕМЕ$^*$}

\def\titkol{Трансформации объектов первого и~второго порядка 
в~лексикографической информационной системе}

\def\aut{И.\,М.~Зацман$^1$}

\def\autkol{И.\,М.~Зацман}

\titel{\tit}{\aut}{\autkol}{\titkol}

\index{Зацман И.\,М.}
\index{Zatsman I.\,M.}


{\renewcommand{\thefootnote}{\fnsymbol{footnote}} \footnotetext[1]
{Исследование выполнено в~ФИЦ ИУ РАН за счет гранта Российского научного фонда №\,24-18-00155, {\sf 
https://rscf.ru/project/24-18-00155}. Работа выполнялась с~использованием инфраструктуры Центра 
коллективного пользования <<Высокопроизводительные вычисления и~большие данные>> (ЦКП 
<<Информатика>>) ФИЦ ИУ РАН (г.\ Москва).}}


\renewcommand{\thefootnote}{\arabic{footnote}}
\footnotetext[1]{ Федеральный исследовательский центр <<Информатика и~управление>> Российской академии наук, 
\mbox{izatsman@yandex.ru}}

\vspace*{-12pt}


  
  \Abst{Рассматриваются теоретические основания проектирования информационных 
технологий (ИТ) интеграции двуязычных словарей и~параллельных корпусов. Дано описание 
первых результатов создания третьего уровня классификации трансформаций объектов 
предметной области информатики, которую предполагается использовать при создании 
концепции лексикографической информационной системы, обеспечивающей интеграцию. 
Все сущности информатики в~статье разделены на два глобальных класса: объекты и~их 
трансформации. Для каждого такого класса конструируется своя классификация. Ранее были 
описаны два верхних уровня классификации трансформаций объектов предметной области. 
В~данной статье рассматривается третий уровень этой классификации. Основанием для 
построения самого верхнего ее уровня служило деление предметной области информатики 
на среды (ментальная, сенсорно воспринимаемая, цифровая и~ряд других сред), каждая из 
которых по определению включает объекты одной природы. Основанием для построения 
второго уровня классификации трансформаций объектов служила типология знаковых  
сис\-тем А.~Соломоника. Цель статьи состоит в~систематизации трансформаций первого 
и~второго порядка объектов предметной области на третьем уровне этой классификации. 
Основанием для систематизации служит средовая версия иерархии Акоффа.}
  
  \KW{объекты предметной области; трансформации объектов; классификация; данные; 
информация; знание; лексикографическая информационная сис\-тема}

\DOI{10.14357/19922264240211}{VZTGVV}
  
\vspace*{3pt}


\vskip 10pt plus 9pt minus 6pt

\thispagestyle{headings}

\begin{multicols}{2}

\label{st\stat}
  
\section{Введение}

\vspace*{-9pt}

  Возникновение параллельных корпусов, в~которых предложениям 
оригинального текста со\-по\-став\-ле\-ны предложения его перевода, обеспечило 
возможность контрастивного лингвистического\linebreak \mbox{анализа} на принципиально 
новом уровне полноты и~точности, недостижимом в~докорпусную эпоху. 
Пионерскими в~этой области стали работы \mbox{1990-х~гг}. Стига Йоханссона  
с~анг\-ло-нор\-веж\-ским корпусом~[1]. В России параллельные корпусы стали 
формироваться в~начале XXI~века в~рамках Национального корпуса русского 
языка~[2].
  
  Создатели двуязычных словарей используют параллельные корпусы для 
сбора материала и~эмпирической проверки своих гипотез, касающихся 
межъязы\-ко\-вой эквивалентности. Ценность параллельных корпусов 
определяется тем, что в~лингвистике этап сбора исходного материала считается 
наиболее трудоемким и~наименее творческим, а~параллельные корпусы 
позволяют значительно сэкономить время и~силы для творческого этапа 
создания словарей~[3].
 % 
  При этом двуязычные словари, создаваемые на основе исходного материала, 
извлеченного из параллельных корпусов, сейчас формируются без связей с~их 
текстами. Другими словами, онлайновые связи созданных словарей 
с~параллельными корпусами, которые служили источниками исходного 
материала, отсутствуют. 

Параллельные корпусы постоянно пополняются 
новыми текстами, в~предложениях которых можно обнаружить новые значения 
слов и~устойчивых словосочетаний. Однако при этом отсутствуют методы 
и~средства оперативного обновления словарей по корпусным данным. 
В~настоящее время проблема установления связей между двуязычными 
словарями и~параллельными корпусами (далее~--- проблема интеграции) 
находится на стадии поиска концептуальных подходов к~их интеграции на 
уровне значений.
  
  Подход к~решению проблемы интеграции, предлагаемый в~статье, учитывает 
  и~появление новых значений слов и~устойчивых словосочетаний, и~динамику 
смысловых значений, которая обусловлена развитием и~пополнением знания 
лингвистов, фиксирующих эти значения в~результате семантического анализа 
пополняемых корпусных данных. Проведенные эксперименты показали, что 
обнаружение нового лингвистического знания обусловливает и~формирование 
дефиниций новых значений, и~пересмотр уже существующих дефиниций~[4, 5].
  
  Например, в~проведенных экспериментах с~использованием ЦКП 
<<Информатика>> ФИЦ ИУ РАН фиксировалась эволюция значений немецких 
модальных глаголов, исходное состояние значений которых было описано 
в~не\-мец\-ко-рус\-ском словаре. В~экспериментальном массиве текстов как 
потенциальных источниках нового знания 16\,268 предложений содержали 
немецкие модальные глаголы и~в~2041 из них встречался глагол sollen. 
В~начале эксперимента в~словаре были описаны~12~значений этого модального 
глагола. По окончании эксперимента лингвисты обнаружили два новых его 
значения, согласовали их дефиниции и~описали эволюцию дефиниций~[6, 7].
  
  Таким образом, для решения проблемы интеграции требуется фиксировать 
новое знание, обнаруженное лингвистами в~текстовых данных параллельных 
корпусов, отслеживать эволюцию знания, представленного в~виде дефиниций 
значений слов и~устойчивых словосочетаний, и,~соответственно, 
актуализировать электронные двуязычные словари. Предлагаемый 
концептуальный подход к~интеграции, который планируется реализовать 
в~процессе проектирования лексикографической информационной сис\-те\-мы, 
фиксирующей эволюцию лингвистического знания, основан на решении 
следующих задач:\\[-14pt]
  \begin{itemize}
  \item категоризация трех базовых понятий информатики, включенных 
  в~иерархию Акоффа~[8] (данные, информация, знание), на объекты 
проектируемой сис\-те\-мы, которая необходима, чтобы фиксировать 
<<кванты>> нового знания и~отслеживать его эволюцию в~этой сис\-теме;\\[-15pt]
  \item  систематизация трансформаций объектов этой сис\-темы.\\[-14pt]
  \end{itemize}
  
  Цель статьи и~состоит в~решении двух задач: категоризации трех базовых 
понятий информатики на объекты лексикографической информационной  
сис\-те\-мы и~сис\-те\-ма\-ти\-за\-ции трансформаций первого и~второго порядка 
ее объектов.
  
  Трансформациями первого порядка, о которых сказано в~формулировке цели 
статьи, называются взаимные преобразования между двумя объектами  
сис\-те\-мы одной природы. Например, перевод в~сис\-те\-ме текста с~русского 
языка на английский относится к~ним. Трансформациями второго порядка 
и~выше называются взаимные преобразования между двумя и~более объектами 
разной природы. Например, кодирование символов текс\-та компьютерными 
кодами и~их декодирование относятся по определению к~трансформациям 
второго порядка.

%\vspace*{-9pt}
  
\section{Процессы трансформаций в~информатике}

%\vspace*{-3pt}

Процессы трансформаций, рассматриваемые в~статье, относятся к~теоретическому ядру информатики, а~не 
только к~проектированию лексикографической информационной сис\-те\-мы. Например, из трех основных 
подходов к~описанию предметной об\-ласти информатики\footnote{В статье предметная область информатики 
трактуется согласно концепции полиадического компьютинга Пола Розенблума~\cite{9-zac}.} (объектный, 
трансформационный и~синтетический) сис\-те\-ма\-ти\-за\-ция трансформаций ближе всего ко второму 
подходу. Примерами первого подхода, в~рамках которого основное внимание уделяется объектам предметной 
области информатики и~в~меньшей степени отношениям\linebreak между ними, могут служить  
работы~\cite{8-zac, 10-zac, 11-zac}; \mbox{примерами} второго подхода, в~рамках которого основное внимание 
уделяется трансформациям и~в~меньшей степени трансформируемым объектам,~---  
работы~\cite{12-zac, 13-zac}; примерами третьего, синтетического подхода, в~котором уделяется внимание 
и~объектам предметной об\-ласти информатики, и~отношениям между ними, могут служить работы~\cite{14-zac, 
15-zac, 16-zac, 17-zac, 18-zac}.

  Таким образом, для описания трансформаций объектов лексикографической 
информационной\linebreak системы предпочтительнее всего трансформационный 
подход, который упоминается и~в определениях информатики. Например, 
в~2009~г.\ П.~Деннинг и~П.~Розенблум сформулировали суть \mbox{информатики} как 
компьютинга следующим образом: <<$\ldots$информатика~--- это не просто 
алгоритмы и~структуры данных; это преобразования [трансформации] 
представлений>>~\cite{12-zac}. Чуть позже, в~контексте краткого описания 
парадигмы информатики как компьютинга, П.~Деннинг и~П.~Фриман изменили 
эту формулировку на такую: <<Центральный объект внимания в~информатике 
можно определить как информационные процессы~--- \textit{естественные или 
искусственные процессы, преобразующие информацию} (курсив мой~--- 
И.\,З.)>>~\cite{13-zac}. Согласно парадигме, предлагаемой авторами этой 
статьи, на начальном этапе проектирования автоматизированных систем 
базовыми элементами моделей их функционирования служат 
\textit{информационные про\-цессы}.
  
  Однако если 15~лет назад в~формулировке из работы~\cite{13-zac} шла речь 
о~процессах, преобразующих информацию, то в~последние~10~лет в~спектр 
процессов трансформаций все чаще стали включать процессы, преобразующие 
не только информацию, но также и~другие объекты автоматизированных 
систем, в~первую очередь данные и~знания~[19--21]. Например, Виктория 
Стодден, позиционируя науку о~данных как одну из дисциплин информатики, 
говорит, что центральный объект исследований в~науке о~данных~--- это 
<<изучение обобщаемого извлечения знания из данных>>~\cite{21-zac}. 
Увеличение и~чис\-ла объектов, и~спект\-ра процессов их трансформаций 
в~автоматизированных сис\-те\-мах обуслов\-ли\-ва\-ет не\-об\-хо\-ди\-мость 
систематизации и~объектов, и~процессов их трансформаций на начальном этапе 
проектирования сис\-тем.
  
  Для создания концепции лексикографической информационной сис\-те\-мы 
и~проектирования ИТ, обеспечивающих интеграцию 
двуязычных словарей и~параллельных корпусов, сначала выполним 
категоризацию на объекты этой сис\-те\-мы трех базовых понятий информатики 
(данные, информация, знание) в~контексте построения классификаций 
сущностей ее предметной об\-ласти.
  
  Необходимость использования классификаций информатики в~процессе 
создания концепции проиллюстрируем, используя иерархию  
Акоффа~\cite{8-zac}. Он использовал принцип их вертикального размещения 
в~иерархии снизу вверх: данные, информация и~знание. Еще в~ней есть термин 
<<мудрость>>, который в~статье не рассматривается. Такое размещение Акофф 
прокомментировал так: <<Каждое из пе\-ре\-чис\-лен\-ных понятий [кроме данных] 
содержит в~себе нижестоящие$\ldots$>>~\cite{8-zac}.
  
  Этому принципу размещения и~комментарию Акоффа свойственны 
недостатки, проанализированные, в~частности, в~работе~\cite{10-zac}. Главный 
вывод, к~которому пришла Роули после изучения иерархии Акоффа, 
заключается в~следующем: <<$\ldots$информация определяется в~терминах 
данных, знание~--- в~терминах информации$\ldots$ но существует меньше 
консенсуса в~описании трансформаций, которые преобразуют сущности, 
расположенные ниже в~иерархии, в~те, которые находятся над ними, что 
приводит к~их терминологической неопределенности>>~\cite{10-zac}. Причина 
этой неопределенности, скорее всего, в~том, что базовые понятия информатики 
включены в~иерархию Акоффа изолированно от общего контекста 
классификаций сущностей ее предметной об\-ласти.

%\vspace*{-9pt}
  
\section{Классификации сущностей информатики}


%\vspace*{-2pt}

  Все сущности предметной области информатики в~работах~[22, 23] 
разделены на два глобальных класса: ее объекты и~их трансформации. Для 
каждого такого класса была предложена своя классификация. 
В~работе~\cite{22-zac} дано описание классификации объектов предметной 
области информатики, первый уровень которой содержит базовые понятия ее 
предметной области (данные, информация, знания и~др.).  
В~работе~\cite{23-zac} дано описание двух верхних уровней классификации 
трансформаций объектов предметной об\-ласти (см.\ рисунок 
в~работе~\cite{23-zac}). Основанием для построения самого верхнего ее уровня послужило деление 
предметной области информатики на среды\footnote{В~работе~\cite{24-zac} дано описание пяти сред 
предметной области информатики (ментальная; сенсорно воспринимаемая, или информационная; 
цифровая; нейро- и~ДНК-среда), каждая из которых по определению включает объекты одной и~той же 
природы.} и~степень разнообразия природы объектов, вовлеченных в~трансформации:
\begin{itemize}
\item  первый класс верхнего уровня классификации включает 
трансформации объектов в~пределах среды только одной природы 
(трансформации первого порядка);
\item  второй класс включает трансформации объектов, относящихся 
к~двум средам разной природы (трансформации второго порядка);
\item третий и~последующие классы включают трансформации объектов, 
относящихся к~трем и~более средам разной природы (трансформации 
третьего и~более высоких порядков).
\end{itemize}

  В работе~\cite{23-zac} были приведены примеры для трех первых классов 
трансформаций, включая пример трансформаций объектов, относящихся 
к~двум средам разной природы (компьютерное кодирование символов текстов 
с~по\-мощью таб\-лиц Unicode).
  
Основанием для построения второго уровня классификации трансформаций объектов послужила типология 
знаковых сис\-тем А.~Соломоника~\cite[c.~131]{25-zac}: естественные знаковые сис\-те\-мы, образные,  
ес\-тест\-вен\-но-язы\-ко\-в$\acute{\mbox{ы}}$е,  
вер\-баль\-но-не\-сло\-вес\-ные сис\-те\-мы записи\footnote{Под системой записи понимается знаковая 
система, сочетающая вербальные знаки с~несловесными (языки нотной записи, карт, таблиц и~др.).} 
и~формализованные знаковые сис\-те\-мы, включая математические. Введем понятие обобщенного текста~--- 
это текст, который может быть создан в~любой из перечисленных знаковых систем. Тогда обобщенные тексты 
могут быть естественными, образными, ес\-тест\-вен\-но-язы\-ко\-в$\acute{\mbox{ы}}$\-ми,  
вер\-баль\-но-не\-сло\-вес\-ны\-ми и~формализованными. Второй уровень классификации трансформаций 
охватывает не все виды объектов предметной  
об\-ласти информатики, а~только перечисленные~5~видов текс\-тов и~их представления, вовлеченные 
в~процессы трансформаций в~одной или более средах вместе с~данными, знанием и~его концептами.

\begin{figure*}[b] %fig1
\vspace*{6pt}
      \begin{center}
     \mbox{%
\epsfxsize=121.191mm 
\epsfbox{zac-1.eps}
}
\end{center}
\vspace*{-6pt}
\Caption{Средовая версия иерархии Акоффа}
\end{figure*}

\section{Классификация трансформаций: построение~третьего 
уровня}

  Основанием для систематизации трансформаций первого и~второго порядка 
на третьем уровне этой классификации служит иерархия Акоффа~\cite{8-zac}, 
на основе которой и~была создана ее средов$\acute{\mbox{а}}$я версия~[26, 
27]. Для создания средов$\acute{\mbox{о}}$й версии была выполнена 
категоризация трех базовых понятий информатики (данные, информация, 
знания) на объекты лексикографической информационной сис\-те\-мы 
в~процессе создания ее концепции\linebreak (рис.~1).
  


  В отличие от классической иерархии Акоффа, в~ее 
средов$\acute{\mbox{о}}$й версии различаются три вида данных: сенсорно 
воспринимаемые, цифровые и~те данные, которые генерируются 
искусственными нейронными сетями (ИНС) в~системах искусственного интеллекта 
(далее~--- ИИ-дан\-ные). Последний вид данных необходим, например, для 
различения входа и~выхода процесса применения обученной 
ИНС в~цифровой модели генерации знания, описанию которой 
посвящена работа~\cite{27-zac}.
  
  Также предлагается различать два вида информации: сенсорно 
воспринимаемая и~цифровая. Кроме знания в~средов$\acute{\mbox{у}}$ю 
версию добавлены концепты и~ментальные образы сенсорно воспринимаемых 
данных. Последние служат промежуточной сущностью между сенсорно 
воспринимаемыми данными и~генерируемым знанием при описании процессов 
извлечения знания из текстовых данных лексикографической информационной 
системы. Описание объектов средов$\acute{\mbox{о}}$й версии иерархии 
Акоффа (см.\ рис.~1) и~отношений между ними дано в~работах~\cite{26-zac, 28-zac}.
  
  В средов$\acute{\mbox{о}}$й версии число объектов равно восьми. Если 
учитывать направления трансформаций, то между восемью объектами на 
рис.~1 она включает~16 их видов (трансформации на границе между сенсорно 
воспринимаемыми данными и~информацией, обозначенные символом~<<?>>, 
в~статье не рас\-смат\-ри\-ва\-ют\-ся). В~будущем число объектов 
в~средов$\acute{\mbox{о}}$й версии, которая выбрана как основание для 
сис\-те\-ма\-ти\-за\-ции трансформаций первого и~второго порядка, может быть 
увеличено. Для построения классификации трансформаций 
важ\-но не возможное увеличение числа объектов 
и~трансформаций между ними, а то, что их виды в~средов$\acute{\mbox{о}}$й 
версии распределены между трансформациями первого и~второго порядка. Из 
16~видов на рис.~1 шесть относятся к~трансформациям первого порядка, это\linebreak 
виды с~номерами~7, 8, 13--16 (далее~--- типология трансформаций первого 
порядка), а~десять~--- к~трансформациям второго порядка, это виды 
с~\mbox{номерами}~1--6 и~9--12 (далее~--- типология трансформаций второго 
порядка). Разместим обе типологии на третьем уровне классификации (см.\ ее 
схему на рис.~2). Перечислим виды трансформаций первой типологии, вводя 
в~скобках их краткие названия, используемые ниже на рис.~3:
  \begin{description}
  \item[\,] 7~--- членение знания на концепты с~помощью одной или нескольких 
знаковых систем (далее~--- членение знания);
  \item[\,] 8~--- формирование знания на основе концептов (формирование 
знания);
  \item[\,] 13~--- обучение ИНС;
  \end{description}
  
  \vspace*{-6pt}
  
  \pagebreak
  
  \end{multicols}
  
  \begin{figure*} %fig2
\vspace*{1pt}
      \begin{center}
     \mbox{%
\epsfxsize=127.513mm 
\epsfbox{zac-2.eps}
}
\end{center}
\vspace*{-9pt}
\Caption{Схема трех верхних уровней классификации трансформаций объектов (объединены 
по три слоя и~для второго, и~для третьего уровней этой классификации)}
\end{figure*}
  
  \begin{multicols}{2}
  
  \noindent
  \begin{description}
  \item[\,] 14~--- восстановление обучающей информации на основе 
содержания обученной ИНС (обращение ИНС);
  \item[\,] 15~--- использование обученной ИНС (использование ИНС);



  \item[\,] 16~--- восстановление исходных данных, соответствующих 
полученным результатам работы обучен\-ной ИНС (восстановление исходных данных 
по результатам ИНС).
  \end{description}
  
  
  Не все виды трансформаций 13--16 поддерживаются в~конкретных системах 
искусственного интеллекта, но с~теоретической точки зрения все их 
предлагается включить в~первую типологию для полноты спектра видов 
трансформаций.
  
  Перечислим виды трансформаций второй типологии:
  \begin{description}
  \item[\,] 1~--- декодирование цифровых данных в~компьютерных системах 
(декодирование данных);
  \item[\,]  2~--- кодирование сенсорно воспринимаемых данных (кодирование 
данных);
  \item[\,] 3~--- ментальное копирование сенсорно воспринимаемых данных 
(ментальное копирование);
  \item[\,] 4~--- восстановление сенсорно воспринимаемых данных по 
ментальным образам (восстановление по образам);
  \item[\,] 5~--- смысловая интерпретация без деления на концепты ментальных 
образов сенсорно воспринимаемых данных (смысловая интерпретация);
  \item[\,] 6~--- восстановление ментальных образов (восстановление образов);
  \item[\,] 9~--- представление концептов в~виде сенсорно воспринимаемой 
информации, например текс\-та\-ми, формулами, таблицами, рисунками и~т.\,д.\ 
(представление концептов);
  \item[\,] 10~--- понимание смысла сенсорно воспринимаемой информации 
(понимание смысла);
  \item[\,] 11~--- кодирование сенсорно воспринимаемой информации 
(кодирование информации);
\end{description}

\vspace*{-6pt}

\pagebreak

\end{multicols}

\begin{figure*} %fig3
\vspace*{1pt}
      \begin{center}
     \mbox{%
\epsfxsize=163mm 
\epsfbox{zac-3.eps}
}
\end{center}
\vspace*{-9pt}
\Caption{Схема частного случая классификации трансформаций объектов (трансформации 
пронумерованы согласно рис.~1)}
\end{figure*}

\begin{multicols}{2}

\noindent
\begin{description}

  \item[\,] 12~--- декодирование цифровой информации (декодирование 
информации).
  \end{description}
  
  Отметим, что в~существующих ИТ
  и~компьютерных системах наиболее часто используются виды 
трансформаций~13 и~15 типологии первого порядка и~1, 2, 11 и~12 типологии 
второго порядка. На рис.~2 в~первом слое третьего уровня классификации 
показаны типологии первого порядка без указания числа трансформаций в~них 
и~без детализации трансформируемых объектов.
  
  Во втором слое третьего уровня классификации условно (без названий) 
показаны типологии второго порядка. Также на рис.~2 в~третьем слое третьего 
уровня классификации условно (также без названий) показаны типологии 
третьего порядка, которые планируется рассмотреть в~отдельной статье. По 
определению они должны включать трансформации между тремя объектами 
разной природы, но средов$\acute{\mbox{а}}$я версия иерархии Акоффа 
включает трансформации только между двумя объектами разной природы. 
Поэтому потребуется другое основание для их систематизации (ранее были 
рассмотрены отдельные примеры трансформаций третьего 
порядка\footnote{Далеко не всегда трансформации третьего и~более высоких порядков можно 
рассматривать как последовательность трансформаций второго порядка. Примером этого могут 
служить трансформации в~процессе обучения пациента пользованию роботизированной рукой, 
охватывающие личностные концепты пациента, релевантные его намерениям, сигналы активности 
мозга как объекты нейросреды и~компьютерные коды~\cite{29-zac}.}~\cite{29-zac}).

\section{Классификация трансформаций: частный~случай}

  Выше было отмечено, что в~будущем число объектов 
в~средов$\acute{\mbox{о}}$й версии иерархии Акоффа может быть увеличено. 
Это означает, что увеличатся и~чис\-ло объектов, и~чис\-ло трансформаций между 
ними в~классификации трансформаций, так как эта средов$\acute{\mbox{а}}$я 
версия служит по определению основанием для систематизации 
трансформаций первого и~второго порядка. Поэтому на третьем уровне рис.~2 
указаны типологии без детализации объектов и~без указания числа 
трансформаций в~каждой из них. С~одной стороны, при таком подходе 
получаем достаточно общий вид этой классификации, так как она не зависит от 
числа объектов в~том или ином варианте средов$\acute{\mbox{о}}$й версии 
(и~это существенно упрощает рис.~2). С~другой стороны, на третьем уровне 
такой общей классификации подразумевается, но не эксплицируется природа 
трансформируемых объектов и~их возможные сочетания в~трансформациях. 

При проектировании лексикографической информационной системы важно 
эксплицировать природу трансформируемых объектов и~их возможные 
сочетания.
  %
  Поэтому в~парадигму информатики~\cite{30-zac} кроме общей 
классификации трансформаций предлагается включать и~ее частные случаи, 
эксплицирующие природу трансформируемых объектов. 

В~этом разделе 
рассмотрим один частный случай, когда используются только естественные 
знаковые сис\-те\-мы из типологии А.~Соломоника~\cite{25-zac} вместе 
с~данными, знанием и~его концептами. Чис\-ло естественных языков при этом не 
ограничено. И~этот частный случай классификации включает только три 
класса природных трансформаций (первого, второго и~третьего порядка, см.\ 
схему классификации на рис.~3).
  
  Первый и~второй уровни схемы общей классификации (см.\ рис.~2) можно 
объединить в~один уровень в~этом частном случае. Ниже этого уровня 
приведено содержание типологий первого и~второго порядка без содержания 
типологий третьего по\-рядка.




  Наполнение типологий первого и~второго порядка соответствует 
средов$\acute{\mbox{о}}$й версии иерархии Акоффа на рис.~1, содержащей 
6~видов трансформаций типологии первого порядка и~10~видов 
трансформаций типологии второго порядка (на рис.~3 стрелки указывают 
направления трансформаций согласно средов$\acute{\mbox{о}}$й версии на рис.~1).
  
  Таким образом, частный случай классификации содержит для этих двух 
типологий 16~теоретически возможных трансформаций, 6 из которых 
в~настоящее время в~существующих ИТ применяются наиболее часто: виды 
трансформаций~1, 2, 11 и~12 типологии второго порядка реализуются 
с~помощью тех или иных методов ко\-ди\-ро\-ва\-ния/де\-ко\-ди\-ро\-ва\-ния 
(например, с~использованием таблиц Unicode), а~виды трансформаций~13 и~15
 в~типологии первого порядка реализуются полностью с~по\-мощью процессов 
цифровой обработки компьютерами.
  
  Остальные виды трансформаций или применяются намного реже (это 
виды~3, 5, 7, 9 и~10), или находятся в~стадии поиска и~разработки (14 и~16) или 
в~настоящее время носят только теоретический характер, обеспечивая полноту 
первой и~второй типологий (4, 6 и~8). Знаком~<<?>> обозначены те виды 
трансформаций, которые по определению не существуют в~используемой 
парадигме информатики~\cite{30-zac}. Однако возможно, что в~других 
будущих подходах к~построению ее парадигмы эти виды трансформаций будут 
существовать.
  
\section{Заключение}

  На сегодняшний день процесс построения классификаций объектов 
предметной области информатики~\cite{22-zac} и~их  
трансформаций~\cite{23-zac} еще не завершен. Однако первые результаты их 
построения уже используются для создания концепции лексикографической 
информационной сис\-те\-мы, обеспечивающей интеграцию двуязычных 
словарей и~параллельных корпусов.
  
  \bigskip
  
  
  Автор признателен рецензентам за помощь в~улучшении статьи.
  
{\small\frenchspacing
 { %\baselineskip=10.6pt
 %\addcontentsline{toc}{section}{References}
 \begin{thebibliography}{99}
\bibitem{1-zac}
\Au{Aijmer K., Altenberg~B.} Advances in corpus-based contrastive linguistics. Studies in honour 
of Stig Johansson.~--- Amsterdam: John Benjamins, 2013. 295~p.  doi: 10.1075/scl.54.
\bibitem{2-zac}
\Au{Добровольский Д.\,О., Кретов~А.\, А., Шаров~С.\,А.} Корпус параллельных текстов~// 
Научная и~техническая информация. Сер.~2: Информационные процессы и~сис\-те\-мы, 2005. 
№\,6. С.~16--27.
\bibitem{3-zac}
\Au{Добровольский Д.\,О.} Корпус параллельных текстов и~сопоставительная 
лексикология~// Труды Института русского языка им.\ В.\,В.~Виноградова, 2015. №\,6. 
С.~413--449. EDN: VJQBHP.
\bibitem{4-zac}
\Au{Гончаров А.\,А., Зацман~И.\,М., Кружков~М.\,Г.} Эволюция классификаций 
в~надкорпусных базах данных~// Информатика и~её применения, 2020. Т.~14. Вып.~4. 
С.~108--116. doi: 10.14357/19922264200415.  
EDN: \mbox{GKWBZT}.
\bibitem{5-zac}
\Au{Гончаров А.\, А., Зацман И. \,М., Кружков~М.\, Г}. Представление новых 
лексикографических знаний в~динамических классификационных сис\-те\-мах~// 
Информатика и~её применения, 2021. Т.~15. Вып.~1. С.~86--93.  doi: 10.14357/19922264210112. EDN: OPEFXW.
\bibitem{6-zac}
\Au{Zatsman I.} Finding and filling lacunas in linguistic typologies~// 15th Forum (International) 
on Knowledge Asset Dynamics Proceedings.~--- Matera, Italy: Institute of Knowledge Asset 
Management, 2020. P.~780--793.
\bibitem{7-zac}
\Au{Zatsman I.} Three-dimensional encoding of emerging meanings in AI-systems~// 21st 
European Conference on Knowledge Management Proceedings.~--- Reading, U.K.: Academic 
Publishing International Ltd., 2020. P.~878--887.
\bibitem{8-zac}
\Au{Ackoff R.} From data to wisdom~// J.~Applied Systems Analysis, 1989. Vol.~16. No.\,1. P.~3--9.
\bibitem{9-zac}
\Au{Rosenbloom P.\,S.} On computing: The fourth great scientific domain.~--- Cambridge, MA, 
USA: MIT Press, 2013. 307~p.
\bibitem{10-zac}
\Au{Rowley J.} The wisdom hierarchy: Representations of the DIKW hierarchy~// J.~Inf. 
Sci., 2007. Vol.~33. Iss.~2. P.~163--180. doi: 10.1177/0165551506070706.
\bibitem{11-zac} 
\Au{Frick$\acute{\mbox{e}}$~M.\,H.} Data--Information--Knowledge--Wisdom (DIKW) pyramid, 
framework, continuum~// Encyclopedia of big data~/ Eds. L.~Schintler, C.~McNeely.~--- Cham: 
Springer, 2018. 4~p. doi: 10.1007/978-3-319-32001-4\_331-1.
\bibitem{12-zac}
\Au{Denning P., Rosenbloom~P.} Computing: The fourth great domain of science~// Commun. 
ACM, 2009. Vol.~52. Iss.~9. P.~27--29.
\bibitem{13-zac}
\Au{Denning P., Freeman~P.} Computing's paradigm~// Commun.  ACM, 2009. Vol.~52. 
Iss.~12. P.~28--30. doi: 10.1145/ 1610252.1610265.
\bibitem{17-zac} %14
\Au{Farradane J.} Knowledge, information, and information science~// J.~Inf. Sci., 
1980. Vol.~2. Iss.~2. P.~75--80. doi: 10.1177/01655515800020020.

\bibitem{15-zac}
\Au{Шрейдер Ю.\,А.} Информация и~знание~// Сис\-тем\-ная концепция информационных 
процессов.~--- М.: ВНИИСИ, 1988. С.~47--52.
\bibitem{16-zac}
\Au{Ingwersen P.} Information and information science~// Enclyclopaedie of library and 
information science~/ Eds. J.\,D.~McDonald, 
M.~Levine-Clark.~--- New York, NY, USA: Marcel Dekker Inc., 1992. Vol.~56. Sup.~19. 
P.~137--174.

\bibitem{14-zac} %17
Информатика как наука об информации: Информационный, документальный, 
технологический, экономический, социальный и~организационный аспекты~/ Под ред. 
Р.\,С.~Гиляревского.~--- М.: Фаир-Пресс, 2006. 592~с.

\bibitem{18-zac}
\Au{Hjorland B.} Library and information science: practice, theory, and philosophical basis~// 
Inform. Process. Manag., 2000. Vol.~36. Iss.~3. P.~501--531. doi:  
10.1016/S0306-\mbox{4573(99)00038-2}.
\bibitem{19-zac}
Deep shift~--- technology tipping points and societal impact.~--- Geneva: WE Forum, 2015. 44~p. 
{\sf http://www3.weforum.org/docs/WEF\_GAC15\_ Technological\_Tipping\_Points\_report\_2015.pdf}.
\bibitem{20-zac}
\Au{Berman F., Rutenbar~R., Hailpern~B., Christensen~H., Davidson~S., Estrin~D., 
Franklin~M., Martonosi~M., Raghavan~P., Stodden~V., Szalay~A.\,S.} Realizing the potential of 
data science~// Commun.  ACM, 2018. Vol.~61. Iss.~4. P.~67--72. doi: 10.1145/3188721.

\bibitem{21-zac}
\Au{Stodden V.} The data science life cycle: A~disciplined approach to advancing data science as 
a~science~// Commun.  ACM, 2020. Vol.~63. Iss.~7. P.~58--66. doi: 10.1145/ 3360646.


\bibitem{23-zac} %22
\Au{Зацман И.\,М.} Научная парадигма информатики: классификация трансформаций 
объектов предметной об\-ласти~// Системы и~средства информатики, 2023. Т.~33. №\,4. 
С.~126--138. doi: 10.14357/08696527230412. EDN: ZIKUWO.

\bibitem{22-zac} %23
\Au{Зацман И.\,М.} Научная парадигма информатики: классификация объектов предметной  
об\-ласти~// Информатика и~её применения, 2023. Т.~17. Вып.~4. С.~96--103. doi: 
10.14357/19922264230413. EDN: FIUQAT.

\bibitem{24-zac}
\Au{Зацман И.\,М.} О~научной парадигме информатики: верхний уровень классификации 
объектов ее предметной об\-ласти~// Информатика и~её применения, 2022. Т.~16. Вып.~4. 
С.~73--79. doi: 10.14357/ 19922264220411. EDN: XZNKVI.

\bibitem{25-zac}
\Au{Соломоник А.\,Б.} Философия знаковых систем и~язык.~--- М.: ЛКИ, 2011. 408~с.
\bibitem{26-zac}
\Au{Зацман И.\,М.} Трансформация иерархии Акоффа в~научной парадигме информатики~// 
Информатика и~её применения, 2023. Т.~17. Вып.~3. С.~107--113. doi: 
10.14357/19922264230315. EDN: UMVRRV.

\bibitem{27-zac}
\Au{Zatsman I.} Building digital spiral models of knowledge generation~// 19th Forum 
(International) on Knowledge Asset Dynamics Proceedings.~--- Matera, Italy: Arts for Business 
Institute, 2024. P.~2185--2196.
\bibitem{28-zac}
\Au{Zatsman I.} Digital spiral model of knowledge creation and encoding its dynamics~// 18th 
Forum (International) on Knowledge Asset Dynamics Proceedings.~--- Matera, Italy: Arts for 
Business Institute, 2023. P.~581--596.
\bibitem{29-zac}
\Au{Зацман И.\,М.} Интерфейсы третьего порядка в~информатике~// Информатика и~её 
применения, 2019. Т.~13. Вып.~3. С.~82--89. doi: 10.14357/19922264190312. EDN: 
EHRQLF.

\bibitem{30-zac}
\Au{Зацман И.\,М.} Научная парадигма информатики как третьей культуры~//  
На\-уч\-но-тех\-ни\-че\-ская информация. Сер.~1: Организация и~методика информационной 
работы, 2023. №\,11. С.~1--14.

\end{thebibliography}

 }
 }

\end{multicols}

\vspace*{-9pt}

\hfill{\small\textit{Поступила в~редакцию 14.04.24}}

\vspace*{4pt}

%\pagebreak

%\newpage

%\vspace*{-28pt}

\hrule

\vspace*{2pt}

\hrule



\def\tit{OBJECT TRANSFORMATIONS OF~THE~FIRST AND~SECOND ORDER
IN~A~LEXICOGRAPHIC INFORMATION SYSTEM\\[-5pt]}


\def\titkol{Object transformations of~the~first and~second order
in~a~lexicographic information system}


\def\aut{I.\,M.~Zatsman}

\def\autkol{I.\,M.~Zatsman}

\titel{\tit}{\aut}{\autkol}{\titkol}

\vspace*{-13pt}


\noindent
Federal Research Center ``Computer Science and Control'' of the Russian Academy of Sciences, 
44-2~Vavilov Str., Moscow 119133, Russian Federation


\def\leftfootline{\small{\textbf{\thepage}
\hfill INFORMATIKA I EE PRIMENENIYA~--- INFORMATICS AND
APPLICATIONS\ \ \ 2024\ \ \ volume~18\ \ \ issue\ 2}
}%
 \def\rightfootline{\small{INFORMATIKA I EE PRIMENENIYA~---
INFORMATICS AND APPLICATIONS\ \ \ 2024\ \ \ volume~18\ \ \ issue\ 2
\hfill \textbf{\thepage}}}

\vspace*{2pt}



\Abste{The theoretical foundations of the design of information technologies used for 
the integration of bilingual dictionaries and parallel corpora are considered. The 
description of the first outcomes of the creation of the third\linebreak\vspace*{-12pt}}

\Abstend{ level of object 
transformations classification in the subject domain of informatics, which is supposed 
to be used
in creating the lexicographic information system providing integration, is 
given. All the entities of informatics are divided into two global classes: objects and 
their transformations. For each such class, its own classification is constructed. 
Previously, the two upper levels of the object transformation classification in the subject 
domain have been described. The present paper discusses the third level of this classification. The 
basis for the construction of its highest level was the division of the subject domain of 
informatics into media (mental, sensory, digital, and a~number of other media), each 
of which by definition includes objects of the same nature. The Solomonick's 
typology of sign systems served as the basis for constructing the second level of the 
object transformation classification. The aim of the paper is to systematize object 
transformations of the first and second orders at the third level of this classification. 
The basis for systematization is the medium version of the Ackoff's hierarchy.}

\KWE{subject domain objects; object transformations; classification; data; 
information; knowledge; lexicographic information system}


\DOI{10.14357/19922264240211}{VZTGVV}

\vspace*{-12pt}

\Ack

\vspace*{-3pt}


\noindent
The reported study was funded by the Russian Science Foundation, project  
No.\,24-18-00155, {\sf 
https://rscf.ru/project/24-18-00155}. The research was carried out using the infrastructure of the Shared 
Research Facilities ``High Performance Computing and Big Data'' (CKP 
``Informatics'') of FRC CSC RAS (Moscow) .
   


  \begin{multicols}{2}

\renewcommand{\bibname}{\protect\rmfamily References}
%\renewcommand{\bibname}{\large\protect\rm References}

{\small\frenchspacing
 {%\baselineskip=10.8pt
 \addcontentsline{toc}{section}{References}
 \begin{thebibliography}{99} 
\bibitem{1-zac-1}
\Aue{Aijmer, K., and B.~Altenberg.} 2013. \textit{Advances in corpus-based 
contrastive linguistics. Studies in honour of Stig Johansson}. Amsterdam: John 
Benjamins. 295~p. doi: 10.1075/scl.54.
\bibitem{2-zac-1}
\Aue{Dobrovolskiy, D.\,O., A.\,A.~Kretov, and S.\,A.~Sharov.} 2005. Korpus 
parallel'nykh tekstov [Corpus of parallel texts]. \textit{Nauchnaya i~tekhnicheskaya 
informatsiya. Ser. 2. Informatsionnye protsessy i~sistemy} [Scientific and Technical 
Information. Ser.~2: Information Processes and Systems] 6:16--27.
\bibitem{3-zac-1}
\Aue{Dobrovolskiy, D.\,O.} 2015. Korpus parallel'nykh tekstov i~sopostavitel'naya 
leksikologiya [The corpus of parallel texts and contrastive lexicology]. \textit{Trudy 
Instituta russkogo yazyka im. V.\,V.~Vinogradova} [Proceedings of the 
V.\,V.~Vinogradov Russian Language Institute] 6:413--449. EDN: VJQBHP.
\bibitem{4-zac-1}
\Aue{Goncharov, A.\,A., I.\,M.~Zatsman, and M.\,G.~Kruzhkov.} 2020. Evolyutsiya 
klassifikatsiy v~nadkorpusnykh ba\-zakh dannykh [Evolution of classifications in 
supracorpora databases]. \textit{Informatika i~ee Primeneniya~--- Inform. \mbox{Appl.}}  
14(4):108--116. doi: 10.14357/19922264200415.  
EDN: GKWBZT.
\bibitem{5-zac-1}
\Aue{Goncharov, A.\,A., I.\,M.~Zatsman, and M.\,G.~Kruzhkov.} 2021. 
Predstavlenie novykh leksikograficheskikh znaniy v~dinamicheskikh 
klassifikatsionnykh sistemakh [Representation of new lexicographical knowledge in 
dynamic classification systems]. \textit{Informatika i~ee Primeneniya~--- Inform. 
Appl.} 15(1):86--93. doi: 10.14357/19922264210112. EDN: OPEFXW.
\bibitem{6-zac-1}
\Aue{Zatsman, I.} 2020. Finding and filling lacunas in linguistic typologies. 
\textit{15th Forum (International) on Knowledge Asset Dynamics Proceedings}. 
Matera, Italy: Institute of Knowledge Asset Management. 780--793.
\bibitem{7-zac-1}
\Aue{Zatsman, I.} 2020. Three-dimensional encoding of emerging meanings in  
AI-systems. \textit{21st European Conference on Knowledge Management 
Proceedings}. Reading, U.K.: Academic Publishing International Ltd. 878--887.
\bibitem{8-zac-1}
\Aue{Ackoff, R.} 1989. From data to wisdom. \textit{J.~Applied Systems Analysis} 
16(1):3--9.
\bibitem{9-zac-1}
\Aue{Rosenbloom, P.\,S.} 2013. \textit{On computing: The fourth great scientific 
domain}. Cambridge, MA: MIT Press. 307~p.
\bibitem{10-zac-1}
\Aue{Rowley, J.} 2007. The wisdom hierarchy: Representations of the DIKW 
hierarchy. \textit{J.~Inf. Sci.} 33(2):163--180. doi: 10.1177/0165551506070706.
\bibitem{11-zac-1}
\Aue{Frick$\acute{\mbox{e}}$, M.\,H.} 2018.  
Data-Information-Knowledge-Wisdom (DIKW) pyramid, framework, continuum. 
\textit{Encyclopedia of big data}. Eds. L.~Schintler and C.~McNeely. Cham: 
Springer. 4~p. doi: 10.1007/978-3-319-32001- 4\_331-1.
\bibitem{12-zac-1}
\Aue{Denning, P., and P.~Rosenbloom.} 2009. Computing: The fourth great domain 
of science. \textit{Commun. ACM} 52(9):27--29.
\bibitem{13-zac-1}
\Aue{Denning, P., and P.~Freeman.} 2009. Computing's paradigm. \textit{Commun. 
ACM} 52(12):28--30. doi: 10.1145/ 1610252.1610265.

\bibitem{17-zac-1} %14
\Aue{Farradane, J.} 1980. Knowledge, information, and information science. 
\textit{J.~Inf. Sci.} 2(2):75--80. doi: 10.1177/ 01655515800020020.

\bibitem{15-zac-1}
\Aue{Shreyder, Yu.\,A.} 1988. Informatsiya i~znanie [Information and knowledge]. 
\textit{Sistemnaya kontseptsiya in\-for\-ma\-tsi\-on\-nykh protsessov} [System concept of 
information processes]. Moscow: VNIISI. 47--52.
\bibitem{16-zac-1}
\Aue{Ingwersen, P.} 1995. Information and information science. 
\textit{Encyclopedia of library and information science}. Eds. J.\,D.~McDonald and 
M.~Levine-Clark. New York, NY: Marcel Dekker Inc. 56(19):137--174.

\bibitem{14-zac-1} %17
Gilyarevskiy, R.\,S., ed. 2006. \textit{Informatika kak nauka ob informatsii: 
informatsionnyy, dokumental'nyy, tekh\-no\-lo\-gi\-che\-skiy, ekonomicheskiy, sotsial'nyy 
i~organizatsionnyy aspekty} [Informatics as information science: Informational, 
documentary, technological, economic, social, and organizational dimensions]. 
Moscow: FAIR-PRESS. 592~p.

\bibitem{18-zac-1}
\Aue{Hjorland, B.} 2000. Library and information science: Practice, theory, and 
philosophical basis. \textit{Inform. Process. Manag.} 36(3):501--531. doi:  
10.1016/S0306-\mbox{4573(99)00038-2}.
\bibitem{19-zac-1}
Deep shift~--- technology tipping points and societal impact. 2015. \textit{World Economic 
Forum}. Geneva. 44~p. Available at: {\sf 
http://www3.weforum.org/docs/WEF\_ GAC15\_Technological\_Tipping\_Points\_report\_2015.pdf} (accessed May~20, 
2024).
\bibitem{20-zac-1}
\Aue{Berman, F., R.~Rutenbar, B.~Hailpern, H.~Christensen, S.~Davidson, 
D.~Estrin, M.~Franklin, M.~Martonosi, P.~Raghavan, V.~Stodden, and 
A.\,S.~Szalay.} 2018. Realizing the potential of data science. \textit{Commun. ACM} 
61(4):67--72. doi: 10.1145/3188721.
\bibitem{21-zac-1}
\Aue{Stodden, V.} 2020. The data science life cycle: A~disciplined approach to 
advancing data science as a~science. \textit{Commun. ACM} 
 63(7):58--66. doi: 10.1145/3360646.

\bibitem{23-zac-1} %22
\Aue{Zatsman, I.\,M.} 2023. Nauchnaya paradigma informatiki: klassifikatsiya 
transformatsiy ob''ektov predmetnoy oblasti [Scientific paradigm of informatics: 
Transformation classification of domain objects]. \textit{Sistemy i~Sredstva 
Informatiki~--- Systems and Means of Informatics} 33(4):126--138. doi: 
10.14357/08696527230412. EDN: ZIKUWO.

\bibitem{22-zac-1} %23
\Aue{Zatsman, I.\,M.} 2023. Nauchnaya paradigma informatiki: klassifikatsiya 
ob''ektov predmetnoy oblasti [Scientific paradigm of informatics: Classification of 
domain objects]. \textit{Informatika i~ee Primeneniya~--- Inform. Appl.} 
 17(4):96--103. doi: 10.14357/19922264230413. EDN: FIUQAT.
 
\bibitem{24-zac-1}
\Aue{   Zatsman, I.\,M.} 2022. O nauchnoy paradigme informatiki: verkhniy uroven' 
klassifikatsii ob''ektov ee predmetnoy oblasti [On the scientific paradigm of 
informatics: The classification high level of its objects]. \textit{Informatika i~ee 
Primeneniya~--- Inform. Appl.} 16(4):73--79. doi: 10.14357/19922264220411. EDN: 
XZNKVI.
\bibitem{25-zac-1}
\Aue{Solomonick, A.\,B.} 2011. \textit{Filosofiya znakovykh system i~yazyk} 
[Philosophy of sign systems and language]. Moscow: LKI. 408~p.
\bibitem{26-zac-1}
\Aue{Zatsman, I.\,M.} 2023. Transformatsiya ierarkhii Akoffa v~nauchnoy 
paradigme informatiki [Transformation of the Ackoff's hierarchy in the scientific 
paradigm of informatics]. \textit{Informatika i~ee Primeneniya~--- Inform. \mbox{Appl.}} 
17(3):107--113. doi: 10.14357/19922264230315. EDN: UMVRRV.
\bibitem{27-zac-1}
\Aue{Zatsman, I.} 2024. Building digital spiral models of knowledge 
generation. \textit{19th Forum (International) on Knowledge Asset Dynamics 
Proceedings}. Matera, Italy: Arts for Business Institute. 2185--2196.
\bibitem{28-zac-1}
\Aue{Zatsman, I.} 2023. Digital spiral model of knowledge creation and encoding its 
dynamics. \textit{18th Forum (International) on Knowledge Asset Dynamics 
Proceedings}. Matera, Italy: Arts for Business Institute. 581--596.
\bibitem{29-zac-1}
\Aue{Zatsman, I.\,M.} 2019. Interfeysy tret'ego poryadka v~informatike 
 [Third-order interfaces in informatics]. \textit{Informatika i~ee Primeneniya~--- 
Inform. Appl.} 13(3):82--89. doi: 10.14357/19922264190312. EDN: EHRQLF.
\bibitem{30-zac-1}
\Aue{Zatsman, I.} 2023. Scientific paradigm of informatics as a~third culture. 
\textit{Scientific Technical Information Processing} 50(4):246--258. doi: 
10.3103/S0147688223040111. EDN: CKHMYS.

\end{thebibliography}

 }
 }

\end{multicols}

\vspace*{-6pt}

\hfill{\small\textit{Received April 14, 2024}} 


\vspace*{-12pt}


\Contrl

\vspace*{-3pt}

\noindent
\textbf{Zatsman Igor M.} (b.\ 1952)~--- Doctor of Science in technology, head of 
department, Federal Research Center ``Computer Science and Control'' of the 
Russian Academy of Sciences, 44-2~Vavilov Str., Moscow 119333, Russian 
Federation; \mbox{izatsman@yandex.ru}





\label{end\stat}

\renewcommand{\bibname}{\protect\rm Литература}  %10
\def\stat{alex}

\def\tit{ПРИМЕНЕНИЕ КОНТЕКСТНО-СВОБОДНЫХ ГРАММАТИК ДЛЯ~ИЗВЛЕЧЕНИЯ ОНТОЛОГИИ ИЗ ТЕКСТОВ 
КОРОТКИХ ОПИСАНИЙ СТАТЕЙ БИОЛОГИЧЕСКОЙ ТЕМАТИКИ$^*$}

\def\titkol{Применение КС-грамматик для извлечения онтологии из текстов 
коротких описаний статей биологической тематики}

\def\aut{Д.\,А.~Алексеевский$^1$}

\def\autkol{Д.\,А.~Алексеевский}

\titel{\tit}{\aut}{\autkol}{\titkol}

{\renewcommand{\thefootnote}{\fnsymbol{footnote}} \footnotetext[1]
{Работа выполнена при частичной поддержке РФФИ (проект 15-07-09306).}}


\renewcommand{\thefootnote}{\arabic{footnote}}
\footnotetext[1]{НИУ Высшая школа экономики, dalexeyevsky@hse.ru}

\vspace*{-6pt}

  
  \Abst{Обработка текстов биологической и~медицинской тематики представляет интерес 
как с~точки зрения биологии, для которой она предоставляет ценные результаты, так 
и~в~качестве источника более сложных задач для обработки текстов. Одной из важных 
задач автоматической обработки текстов является построение онтологий. 
Предложен метод построения онтологий промежуточного уровня по корпусу текстов на 
ограниченном подмножестве английского языка. Онтологии промежуточного уровня служат 
одним из инструментов решения задачи установления соответствия между фактами 
в~априорных онтологиях и~фрагментами текста. Предложен новый подход, основанный на 
расширенном определении кон\-текст\-но-сво\-бод\-ных (КС) грам\-ма\-тик, позволяющий порождать онтологии, 
обладающие указанным свойством. Показаны преимущества использования 
корпусов на ограниченном подмножестве естественного языка для построения таких 
онтологий.}
  
  \KW{КС-грамматики; построение онтологий; биомедицинские тексты}
  
  \DOI{10.14357/19922264160111} %

%\vspace*{-4pt}

\vskip 10pt plus 9pt minus 6pt

\thispagestyle{headings}

\begin{multicols}{2}

\label{st\stat}
  
  \section{Введение}
  
%  \vspace*{-2pt}
  
  За последние десятилетия биология, а следом за ней и~медицина претерпели 
несколько научных переворотов, каждый из которых приводил к~бурному росту 
числа публикаций, а также и~прочих текстов этих тематик. Многие полученные 
данные были собраны в~базы данных, которые играют большую роль в~этих 
науках. В~то же время с~ростом объема опубликованных текстов 
обнаруживаются новые виды данных, доступные в~текстовом виде, требующие 
структурирования и~верификации. Этим объясняется растущая актуальность 
темы извлечения фактов из текстов биологической и~медицинской тематики. 
Следует заметить, что эта тема имеет существенные отличия от автоматической 
обработки текстов в~целом, что обусловливает выделение ее в~отдельную 
область.
  
  Задачам автоматической обработки текстов медицинской и~биологической 
тематики посвящено много работ. Среди современных направлений 
исследова\-ний: извлечение и~нормализация именованных сущностей~[1], 
извлечение событий и~состав\-ных отношений~[2], анализ дискурса 
и~ко\-референции~[3], построение и~пополнение онтологий и~баз данных~[4]. 
Среди наиболее широко\linebreak используемых биологических баз данных встречаются 
ресурсы, совмещающие структурированные данные (ссылки на другие базы 
данных, чис\-ло\-вые характеристики объектов, номенклатурные на\-звания 
объектов и~т.\,п.), неструктурированные тексто\-вые данные (текстовые 
описания, цитаты из статей и~эн\-цик\-ло\-пе\-дий) и~час\-тич\-но формализованные 
текстовые данные (описания на ограниченном подмножестве языка 
с~использованием контролируемых словарей)~[5, 6].
  
  Наряду с~задачей извлечения фактов, соответ\-ст\-вующих заранее заданной 
онтологии, для неко\-торых областей актуальна задача определения 
онтологической структуры и~извлечения самих онтологиче\-ских элементов. 
%
В~настоящей статье {пред\-ло\-жен} метод преобразования частично 
структурированных текстовых описаний в~онтологии, основанный на 
использовании гетерогенных час\-тот\-ных списков и~семантически 
ориентированных КС (СОКС) грамматик.
  
  Для иллюстрации работы метода выбраны краткие аннотации статей, 
используемые в~одной из баз данных (см.\ подразд.~2.4). Приведена 
последовательность действий по преобразованию аннотаций в~онтологическое 
представление, дана оценка применимости метода в~выбранном примере. 
В~насто\-ящее время указанные краткие аннотации заполняются кураторами 
вручную, но затем автоматически посредством простых шаблонов по ним 
определяется уровень доверия к~записи в~базе данных. Приведение таких 
аннотаций к~онтологическому представлению является необходимым первым 
ша-\linebreak\vspace*{-12pt}

\pagebreak

\noindent
гом для последующего автоматического постро\-ения аннотаций по тексту 
статьи.

\vspace*{-6pt}
  
  \section{Контекст работы}
  
  \vspace*{-3pt}
  
  \subsection{Специфика обработки биологических текстов}
  
  \vspace*{-1pt}
  
  Медицинская и~биологическая тематика текстов привносит особенности во 
многие этапы их обработки. В~значительной мере именно это и~обуслов\-ливает 
выделение bionlp как отдельной предметной области.
  
  Один из часто используемых шагов обработки текстов~--- идентификация 
фрагментов текста, соответствующих известным сущностям в~базах данных, 
так называемое извлечение именованных сущностей.
{\looseness=-1

}
  
  В биологических текстах этот шаг обработки усложнен несколькими 
обстоятельствами:
  \begin{itemize}
\item именованная сущность может являться лишь частью слова, например 
в~предложении <<The acid-promoted expression of the PmrD protein was  
phoPQ-dependent, which is in agreement with the fact that PhoP is the only known 
direct transcriptional activator of pmrD (Kox \textit{et al.}, 2000)>> в~слове  
phoPQ-dependent выделяют двухбелковый комплекс <<phoPQ>>, состоящий из 
белка <<phoP>> и~белка~<<Q>>;\\[-15pt]
\item некоторые сущности, такие как белки или химические соединения, имеют множество синонимичных 
названий, при этом в~текстах могут использоваться не полные названия, а~их сокращения, смысл которых 
возможно восстановить лишь из контекста. Например, белок\footnote{UniProt AC P30233, {\sf 
http://www.uniprot.org/uniprot/P30233}.}, имеющий в~базе данных названия <<Sweet protein 
mabinlin-2>>, <<Mabinlin~II>>, <<MAB~II>>, <<Sweet protein mabinlin-2 chain~A>>, 
<<Sweet protein 
\mbox{mabinlin-2} chain~B>>, может встречаться в~стать\-ях как <<heat-stable sweet protein, mabinlin-II>>, 
<<mabinlin>> (в~пределах текста одной статьи это название может в~разных контекстах обозначать как 
название класса белков, так и~конкретный белок), <<Cm-MaIIA>> (обозначение одной цепочки 
модифицированного белка, введенное в~статье)~[7];\\[-15pt]
\item для некоторых сущностей после определения их названия необходимо 
точнее идентифицировать сущность, о которой идет речь: например, одно и~то 
же название белка может иметь несколько аллелей в~одном организме, белок 
может различаться или не различаться в~зависимости от ткани, для которой 
проводился эксперимент, одно имя могут иметь схожие, но различные белки из 
разных организмов. Для каждого белка, имеющего то же имя, имеется 
отдельная запись в~базе данных, и~необходимо определить, о какой именно 
записи идет речь.
\end{itemize}

\vspace*{-10pt}

  \subsection{Задача построения онтологий}
  
  В литературе встречаются разнообразные определения понятия онтологии 
в~зависимости от темы и~специфики выбранной задачи. Встречающиеся 
определения этого понятия в~контексте извлечения фактов описывают способы 
представления знаний, как правило, состоящие из описаний сущностей, их 
свойств, классификации, связей между ними и~логических правил пополнения 
их свойств и~связей~[8].
  
  Выделяют онтологии, построенные априори путем логической 
классификации и~лексические, в~которых отражаются семантические связи 
между языковыми единицами~[9]. Они обладают разными свойствами: 
априорная точнее отражает предметную область и~позволяет применять 
богатые механизмы логического вывода, в~то время как сущности лексических 
онтологий, как правило, проще выделять в~тексте. В~связи с~этим одна из часто 
возникающих задач состоит в~установлении соответствия между сущностями 
лексической и~априорной онтологии~[10].
  
  Подход, предлагаемый в~настоящей статье, позволяет построить онтологию, 
занимающую промежуточное положение. Такая онтология строится частично 
по базам данных как априорная, частично по корпусу текстов как лексическая. 
Это определяет ее главное достоинство: она содержит как ссылки на 
конкретные сущности из базы данных, так и~их текстовое представление.

\vspace*{-10pt}
  
  \subsection{Семантически-ориентированные контекстно-свободные грамматики}
  
  Для настоящей работы в~качестве формализма для описания синтаксической 
структуры предложения были выбраны КС-грам\-ма\-ти\-ки. Контекстно-свободная
грам\-ма\-ти\-ка~--- это способ описания структуры 
предложения в~виде иерархии составляющих частей~[11]. Дадим ей 
формальное определение.
  
  \vspace*{1pt}
  
  \noindent
  \textbf{Определение~1.} Кон\-тек\-ст\-но-сво\-бод\-ной грамматикой 
называется четверка $G\hm=(V,\Sigma, R, S)$, где $V$~--- конечное множество 
нетерминальных символов; $\Sigma$~--- конечное множество терминальных 
символов; $R\subset \{V\times (V\cup \Sigma)^*\}$~--- множество правил вывода 
вида $v\hm\to a_1a_2\cdots$, где $v\hm\in V$, $a_i\hm\in V\cup \Sigma$; $S\hm\in 
V$~--- начальный символ.
  
\pagebreak
  
  Формальным определением для описания онтологии в~настоящей работе 
было выбрано следующее: онтология~--- это ориентированный граф, в~котором 
каждая вершина и~каждое ребро со\-про\-вож\-да\-ют\-ся пометой. С~помощью пометы 
множество вершин делится на вер\-ши\-ны-клас\-сы  
и~вер\-ши\-ны-эк\-земп\-ля\-ры. Пометы на ребрах устанавливают тип 
отношений, в~которых находятся две выбранные вершины. Приведем более 
формализованное определение.
  
  \smallskip
  
  \noindent
  \textbf{Определение 2.} Онтологией называется пара $O\hm= (G_O,L_O)$ из 
ориентированного графа и~меток к~нему. В~свою очередь, граф 
$G_O\hm=(E_O,R_O)$ состоит из множества вершин~$E_O$, называемого 
множеством сущностей, и~множества ребер $R_O\subset E_O\times E_O$, 
называемого множеством отношений; метки $L_O\hm= (T_E, T_R, L_E, L_R)$ 
задаются алфавитом возможных меток для вершин~$T_E$, алфавитом 
возможных меток для ребер~$T_R$, отображением $L_E: E_O\hm\to T_E$ 
вершины на ее метку и~отображением $L_R: R_O\hm\to T_R$ ребра на его 
метку.

  \smallskip
  
  Одно из свойств предлагаемого в~настоящей работе алгоритма состоит 
в~простоте выделения онтологических фактов из деревьев синтаксического 
разбора. Такой алгоритм требует введения нового понятия: семантически 
ориентированной КС-грам\-ма\-ти\-ки. Контекстно-свободная грам\-ма\-ти\-ка является 
семантически ориентированной для данной онтологии, если часть ее правил 
описывает сущности и~отношения в~онтологии. Предлагается следующее 
определение.
  
  \smallskip
  
  \noindent
  \textbf{Определение~3.} Семантически ориентированной  
КС-грамматикой называется тройка $S\hm=(G,O,M)$ 
из КС-грам\-ма\-ти\-ки $G\hm=(V,\Sigma,\ldots)$, 
онтологии $O\hm= ((E-O, R_O), (T_E, T_R, L_E, L_R))$ и~отображения~$M$ 
между ними. Отображение $M\hm= (M_E, M_R)$ состоит из отображения 
$M_E\subset (\Sigma \cup V, E_O)$, где $\forall (v,e), (v^\prime, e^\prime)\hm\in 
M_E: v\hm= v^\prime \Leftrightarrow e\hm=e^\prime$; символов грамматики на 
вершины онтологии и~отображения $M_R\subset (V,L_R)$, где $\forall (v,r), 
(v^\prime, r^\prime)\hm\in M_R:\ v\hm=v^\prime\Leftrightarrow r\hm=r^\prime$.

  \smallskip
  
  Терминальный символ грамматики может быть отображен на  
вер\-ши\-ну-класс или вершину-эк\-земп\-ляр либо не использоваться в~онтологии. 
В~последнем случае терминальный символ будем называть синтаксическим по 
последнему этапу обработки текста, в~котором он используется. 
Нетерминальный символ может быть отображен на вер\-ши\-ну-класс, метку 
ребра (тип отношения), в~том числе одновременно, либо не использоваться 
в~онтологии. Как и~в случае с~терминальными вершинами, в~последнем случае 
такой нетерминал будет называться синтаксическим.

%\pagebreak

%\vspace*{-6pt}
  
  \subsection{База данных UniProt}
  
  Материалом для разработки и~тестирования предлагаемой процедуры 
построения онтологий послужила свободно распространяемая база UniProt~[6].
  
  UniProt является хранилищем аминокислотных последовательностей белков 
наряду с~их краткими описаниями. База содержит ссылки на другие базы 
данных, посвященные исследованиям белков специфическими методами. 
Кроме того, частью описания белка в~базе является список литературы, 
описывающей белок.
  
  Для каждого белка база содержит:
  \begin{itemize}
\item описание его аминокислотной последовательности (поле <<\ \ >>~--- два 
пробела);
\item обозначения белка согласно различным номенклатурам (поля <<DE>> 
и~<<GN>>);
\item идентификаторы в~различных биологических базах данных самого белка 
(поля <<ID>>, <<AC>> и~<<DR>>) и~его носителя (<<OC>> и~<<OX>>);
\item биологический контекст белка (поля <<OS>>, <<OG>> и~<<OH>>);
\item библиографическую информацию (поля <<RN>>, <<RP>>, <<RC>>, 
<<RX>>, <<RG>>, <<RA>>, <<RT>> и~<<RL>>);
\item описания известных свойств белка: текстовые (поле <<CC>>), на 
ограниченном подмножестве английского языка (поля <<RP>> и~<<KW>>), 
формализованные (поле <<FT>>);
\item уровень доверия данной записи (поле <<PE>>);
\item прочую служебную информацию (поля <<DT>> и~<<SQ>>).
\end{itemize}

  Значение <<PE>> уровня достоверности записи базы данных определяется 
тем, какими экспериментальными средствами установлен факт существования 
белка и~его соответствия представленным данным. Описания того, какие 
экспериментальные средства применялись к~белку, хранятся в~базе в~полях 
<<CC>>, <<RP>> и~<<KW>>, для некоторых методов факт их применения 
можно опознать по свойствам в~поле <<FT>>. В~базе заданы формальные 
правила выставления значения уровня доверия (<<PE>>) в~зависимости от 
наличия некоторых шаблонных выражений в~этих полях~\cite{12-al}.



  \begin{figure*}[b] %fig1
  \begin{center}
  
  {\small
  \begin{boxedverbatim}
X-RAY CRYSTALLOGRAPHY (1.80 ANGSTROMS) OF 44-480 OF WILD-TYPE AND MUTANTS TYR-118;
ARG-168 AND ALA-309 IN ACTIVE AND RESTING STATES AND IN COMPLEX WITH PEPTIDE SUBSTRATE, 
FUNCTION, CATALYTIC ACTIVITY, ENZYME REGULATION, SUBSTRATE SPECIFICITY, SUBUNIT, DOMAIN, 
PROTEOLYTIC AUTO-CLEAVAGE, ACTIVE SITES, SITES, DISRUPTION PHENOTYPE, MUTAGENESIS
OF VAL-118; ARG-168; SER-309 AND GLN-338, AND PDZ DOMAIN DELETION MUTANT.
\end{boxedverbatim}

}
\begin{center}
{\small (\textit{а})}
\end{center}

{\small
\begin{boxedverbatim}
> X-RAY CRYSTALLOGRAPHY (1.80 ANGSTROMS) OF 44-480 OF WILD-TYPE AND MUTANTS TYR-118;
ARG-168 AND ALA-309 IN ACTIVE AND RESTING STATES AND IN COMPLEX WITH PEPTIDE SUBSTRATE
> FUNCTION
> CATALYTIC ACTIVITY
> ENZYME REGULATION
> SUBSTRATE SPECIFICITY
> SUBUNIT
> DOMAIN
> PROTEOLYTIC AUTO-CLEAVAGE
> ACTIVE SITES
> SITES
> DISRUPTION PHENOTYPE
  > MUTAGENESIS OF VAL-118; ARG-168; SER-309
\end{boxedverbatim}

}

\begin{center}
{\small (\textit{б})}
\end{center}
  \end{center}
  
  \vspace*{-14pt}
  
  \begin{center}
  \Caption{Примеры описаний в~поле <<RP>>: (\textit{а})~полное описание, (\textit{б}) 
атомарные факты}
  \end{center}
  \vspace*{-12pt}
  \end{figure*}
  
  
  
  База данных UniProt состоит из двух час\-тей: UniProt/TrEMBL, пополняемой 
полностью автоматически, и~UniProt/Swiss-Prot, по\-пол\-ня\-емой кураторами 
вручную на основе материалов UniProt/TrEMBL, существующих публикаций 
и~материалов других баз данных. Поля <<KW>> и~<<FT>> получают 
начальные значения автоматически в~базе данных UniProt/TrEMBL, хотя затем 
могут быть изменены в~процессе курирования. Поля <<CC>> и~<<RP>> 
заполняются только кураторами вручную.
  
  Этими обстоятельствами обусловлено то, что в~качестве материала для 
настоящей работы был собран корпус предложений в~поле <<RP>> из базы 
данных UniProt/Swiss-Prot.

\vspace*{-9pt}
  
  \section{Материалы и~методы}
  
  Материалом исследований послужили данные из базы UniProt/Swiss-Prot 
версии 2015\_01. Собранный корпус уникальных атомарных причин 
цитирования в~поле <<RP>> имеет размер 173\,212~предложений.
  
  Всего база UniProt/Swiss-Prot 2015\_1 содержит:
  \begin{itemize}
\item 547\,357 записей (одна запись описывает один белок);
\item 1\,092\,817 ссылок на литературу и, соответственно, всего предложений 
в~поле <<RP>>, включая повторяющиеся, среди них;
\item 179\,616 уникальных предложений в~поле <<RP>>, состоящих из одного или 
нескольких атомарных описаний (в~свою очередь также вклю\-ча\-ющих 
повторения);
\item 173\,212 уникальных атомарных описаний.
\end{itemize}

  Для дальнейшей работы использовался описанный корпус уникальных 
атомарных описаний, с~тем чтобы наиболее полно покрыть максимально 
возможное количество особых случаев в~языке.

%\vspace*{-9pt}
  
  \subsection{Особенности предложений в~поле <<RP>>}
  
    \vspace*{-2pt}
  
  Предложения в~поле <<RP>> являются полуструктурированными, так как несут 
признаки как структурированных, так и~естественных языковых данных. 
Предложения порождаются кураторами. Для них не существует 
формализованного описания структуры или инструмента для валидации. 
Существует находящаяся на данный момент в~стадии разработки
инициатива по унификации представления названий различных классов 
сущностей в~таких предложениях с~помощью внедрения контро\-ли\-ру\-емых 
словарей~\cite{13-al}. Наряду с~этим для кураторов существует инструкция по заполнению, 
вклю\-ча\-ющая в~себя примеры представления большого числа типов 
фактов~\cite{14-al}.
  
  Каждое атомарное описание является именной группой. Важно заметить, что 
для краткости описания не содержат упоминаний описываемого объекта. 
Объект описания устанавливается из факта принадлежности описания записи 
в~базе данных (рис.~1). 
  
\vspace*{-14pt}
  
  \subsection{Словники}
  
  \vspace*{-6pt}
  
  Для извлечения именованных сущностей и~насыщения списка примитивных 
фактов были использованы словники.
  
  Словник имен белков был построен по значениям в~поле <<DE>>, подразделам 
RecName и~AltName
 базы данных UniProt и~полям Full, Short, Name, Synonyms 
в~них. Суммарный объем словника составил 308\,370 словосочетаний.
  
  Некоторые названия белков совпадают с~общезначимыми словами 
английского языка. Для того чтобы исключить ошибки второго рода в~таких 
случаях, из словника имен белков были удалены все слова, являющиеся 
словами английского языка. Для этой фильтрации был использован словник 
общеупотребительной лексики американского анг\-лий\-ско\-го языка~\cite{15-al} 
объемом 99\,171~словоформа, содержащий все падежные формы слов.

\vspace*{-6pt}
  
  \subsection{Методы}
  
  Для сегментации текста на слова был использован токенизатор, 
сохраняющий все знаки пунктуации, включая дефисы, как отдельные токены. 
Токенизатор был разработан на основе пакета re языка Python~\cite{16-al}.
  
  Для построения СОКС-грамматик был использован парсер Эрли с~проходом 
снизу вверх из пакета nltk~\cite{17-al} для языка Python.
  
  Для построения частотных списков использовались средства shell script 
и~сопутствующие программы текстовой обработки из базового комплекта операционной системы 
GNU: cat, sort, uniq, grep, sed, head, tail, less.

\vspace*{-6pt}
  
  \section{Алгоритм разработки онтологии с~помощью контекстно-свободных грамматик}
  
  Задача алгоритма состоит в~том, чтобы за наименьшее время преобразовать 
наибольшую часть заранее заданного корпуса фактов, представленного в~виде 
полуструктурированных текстовых данных, в~онтологическое представление.
  
  Основная идея алгоритма состоит в~итерационном применении и~пополнении 
КС-грам\-ма\-ти\-ки.\linebreak\vspace*{-12pt}


\noindent После каждого применения грамматики предложения 
корпуса преобразуются в~гетерогенную последовательность из токенов 
и~нетерминальных \mbox{символов} грамматики. Полученный корпус гетерогенных 
последовательностей используется для того, чтобы определить, какое правило 
нужно добавить в~корпус для получения наибольшего прироста количества 
предложений, разбор которых доведен до не\-тер\-ми\-на\-ла-вер\-шины.
  
  При построении КС-грам\-ма\-ти\-ки терминальными символами грамматики 
являются токены из корпуса, множество нетерминальных символов является 
объединением из множества типов сущностей в~онтологии и~множества 
вспомогательных нетерминальных символов.
  
  Входными данными для построения онтологии являются:
  \begin{itemize}
\item корпус разбираемых текстов;\\[-15pt]
\item базы данных и~словники, позволяющие выделять в~тексте релевантные 
именованные сущности.
\end{itemize}

  Алгоритм состоит из пяти шагов:
  \begin{enumerate}[1.]
\item Подготовить начальную грамматику.\\[-15pt]
\item Применить к~корпусу текстов правила грамматики, заменив покрытые 
правилами фрагменты текста соответствующими нетерминалами.\\[-15pt]
\item Оценить покрытие корпуса текстов нетерминалами и~выбрать метод 
пополнения грамматики (см.\ ниже).\\[-15pt]
\item Пополнить грамматику новым правилом (см.\ ниже).\\[-15pt]
\item Перейти на шаг~2.
  \end{enumerate}
  
  Начальная грамматика содержит заранее определенный  
не\-тер\-ми\-нал-вер\-ши\-ну; множество нетерминальных символов, состоящее 
только из не-\linebreak тер\-ми\-на\-ла-вер\-ши\-ны; множество терминальных симво\-лов, 
совпадающее с~множеством токенов корпуса; множество правил, являющееся 
пустым.
  
  Оценка покрытия может производиться одним из двух способов.
  \begin{enumerate}[1.]
\item Выбрать из корпуса случайным образом~100~предложений, среди них 
найти наиболее частую синтаксическую конструкцию или тип именованной 
сущности, который еще не покрыт правилами грамматики.\\[-15pt]
\item Построить частотный список предложений, выбрать из них наиболее 
частое, для которого может быть написано правило СОКС-грам\-ма\-ти\-ки, 
не имеющее ложных срабатываний.
\end{enumerate}

  В результате оценки должно быть порождено правило одного из трех видов:
  \begin{enumerate}[(1)]
\item синтаксическое упрощение;\\[-15pt]
\item создание или пополнение газетира;\\[-15pt]
\item семантическое правило.
\end{enumerate}

  \textit{Синтаксическими упрощениями} называются прави\-ла грамматики, 
которые не отображаются в~результирующей онтологии, но обобщают 
однородные конструкции и~упрощают последующее расширение грамматики.
  
  К этому типу правил относятся, например,
  
\noindent
{\small
  \begin{verbatim}
and -> 'AND' | ',' | ',' 'AND' | ';' | ';' 'AND'
и
det -> 'A' | 'AN' | 'THE'
\end{verbatim}

}

  Необходимость создания или пополнения газетира возникает в~тех случаях, 
когда наиболее частым\linebreak\vspace*{-12pt}

\pagebreak

\noindent
 не покрытым нетерминалами явлением в~корпусе 
оказываются названия именованных сущностей, принадлежащие к~одному 
классу.
  
  Например, в~предложении
  \begin{verbatim}
PALMITOYLATION AT CYS-11, AND MUTAGENESIS
 OF SER-2; ARG-6 AND CYS-11.
\end{verbatim}
\noindent
четыре раза встречаются названия конкретных аминокислотных остатков в~
белке, представленные как название аминокислоты и~номер ее позиции, 
записанные через дефис. В тот момент, когда в~корпусе такие случаи 
становятся самыми частотными из неразобранных, необходимо пополнить 
газетир списком названий аминокислот.

  Третий вариант действий состоит в~том, чтобы пополнить  
СОКС-грам\-ма\-ти\-ку \textit{семантическим правилом}. Для этого 
необходимо выявить самую час\-тот\-ную конструкцию, такую что в~ней нет 
токенов, которые могли бы войти в~именованную сущность; в~ней нет лексики, 
играющей исключительно синтаксическую роль; она не сведена к~нетерминалу, 
являющемуся вершиной онтологии.
  
  Такая конструкция может являться предложением целиком, в~этом случае из 
нее будет образовано новое правило для СОКС-грамматики, в~левой час\-ти 
которого будет находиться вершина онтологии:
  \begin{verbatim}
feature -> 'STRUCTURE' 'BY' method
feature -> modification 'AT' range
\end{verbatim}


  Пример предложений, использующих приведенный фрагмент грамматики:
  \begin{verbatim}
STRUCTURE BY ELECTRON MICROSCOPY
 (9.4 ANGSTROMS).
PHOSPHOPANTETHEINYLATION AT SER-37.
\end{verbatim}

  Такая конструкция может одновременно быть предложением и~сводиться 
к~нетерминалу, который при этом не является вершиной онтологии, например:
  \begin{verbatim}
feature -> method
feature -> interaction
\end{verbatim}

  Пример предложений, использующих приведенный фрагмент грамматики:
  \begin{verbatim}
IDENTIFICATION BY MASS SPECTROMETRY.
CALMODULIN-BINDING.
\end{verbatim}

  Такая конструкция может являться частью предложения, в~этом случае 
нетерминал в~левой части правила не будет являться вершиной онтологии, 
например:
  \begin{verbatim}
interaction -> interaction 'WITH' protein
\end{verbatim}

  Пример предложений, использующих приведенный фрагмент грамматики:
  \begin{verbatim}
INTERACTION WITH MPK6
\end{verbatim}
  
  \subsection{Преобразование деревьев~синтаксического разбора 
в~онтологическое представление данных}
  
  В результате работы СОКС-пар\-се\-ра предложения исходного текста 
преобразуются в~деревья синтаксического разбора. Например, предложение
  \begin{verbatim}
FUNCTION, AND INTERACTION WITH RBM8A;
NXF1 AND THE EXON JUNCTION COMPLEX.
\end{verbatim}
после разбора преобразуется в~следующее дерево:
\begin{verbatim}
(description
 (feature
  (feature FUNCTION)
  (and , AND)
  (feature
    (interaction
     (interaction INTERACTION)
     WITH
     (protein
      (protein (protein RBM8A) (and ;)
       (protein NXF1))
      (and AND)
      (protein (det THE)
       (protein (words EXON JUNCTION)
        COMPLEX))))))
 .)
\end{verbatim}

  Такое дерево содержит набор связей, которые в~точности соответствуют 
онтологическим. Помимо таких связей в~дереве имеются связи и~узлы, 
имеющие синтаксическую роль (сочетание и~детерминанты). Кроме того, связи, 
отвечающие за сочетание, представлены здесь не как однородные связи внутри 
одного объекта, а~как вложенная рекурсивная цепочка связей.

  \begin{figure*}[b] %fig2
  \vspace*{6pt}
\begin{center}

{\small
  \begin{boxedverbatim}
> [X - RAY CRYSTALLOGRAPHY [1 . 80 ANGSTROMS]resolution OF [44 - 480]range 
OF [WILD - TYPE AND MUTANTS [TYR - 118 ; ARG - 168 AND ALA - 309]range]variant
IN [ACTIVE AND RESTING STATES]form AND IN [COMPLEX WITH [PEPTIDE SUBSTRATE]chemical]chemenv]feature
> [FUNCTION]feature
> [CATALYTIC ACTIVITY]feature
> [ENZYME REGULATION]feature
> [SUBSTRATE SPECIFICITY]feature
> [SUBUNIT]feature
> [DOMAIN]feature
> [PROTEOLYTIC AUTO - CLEAVAGE]feature
> [ACTIVE SITES]feature
> [SITES]feature
> [DISRUPTION PHENOTYPE]feature
  > [MUTAGENESIS OF [VAL - 118 ; ARG - 168 ; SER - 309 AND GLN - 338]range]feature AND
  \end{boxedverbatim}
  
  }
  
  \vspace*{5pt}
  
  \Caption{Разбор описания}
  \end{center}
  \end{figure*}
  
  Для преобразования деревьев такого вида в~онтологические факты 
необходимо:
  \begin{itemize}
\item заменить текстовое описание именованных сущностей на 
идентификатор базы данных (например, заменить \verb"RBM8A" на 
\verb"Q9Y5S9; RBM8A" является названием белка, общего для многих 
видов, база данных UniProt содержит 64~белка с~идентичным названием, 
текст данного предложения получен из описания белка, извлеченного из 
h.sapiens; следовательно, нас интересуют и~белки \verb"RBM8A" только из 
h.sapiens, такой белок только один);
\item нормализовать числовые значения (например, заменить 
на~4.2~поддерево 
\begin{verbatim}
(float (digits 4) . (digits 2)));
\end{verbatim}
\item раскрыть случаи сочетания необходимым для данного онтологического 
класса способом;
\item удалить нетерминалы, играющие синтаксическую роль (например, 
поддерево: \verb"(det THE)");
\item в~случаях, когда несколько аргументов обозначаются одним и~тем же 
нетерминалом, дать аргументам различные имена;
\item преобразовать правила грамматики в~объявление онтологических 
классов, отношений класс--под\-класс и~объявлений свойств;
\item преобразовать газетиры в~объявление онтологических индивидов и~отношений класс--ин\-дивид;
\item преобразовать дерево разбора в~объявление набора онтологических 
индивидов, объявление их отношения к~соответствующим онтологическим 
классам и~отношений часть--це\-лое и~атрибут для этих индивидов.
\end{itemize}

  Для приведенного примера фрагмент грамматики (вместе с~вставленными 
  в~него для на\-гляд\-ности фрагментами необходимых газетиров) выглядит 
следующим образом:

        \vspace*{-2pt}
        
        \noindent
  \begin{verbatim}
description -> feature '.'
feature -> feature and feature
feature -> interaction
feature -> 'FUNCTION'
interaction -> interaction 'WITH' protein
interaction -> 'INTERACTION'
protein -> protein and protein
protein -> words 'COMPLEX'
protein -> det protein
protein -> Q9Y5S9 | Q9UBU9
and -> 'AND' | ',' | ',' 'AND' | ';' |
 ';' 'AND'
        \end{verbatim}
        
%        \vspace*{-9pt}
        
  Он однозначным образом преобразуется в~набор определений (здесь авторы 
используют OWL2 functional notation~\cite{18-al}:
  \begin{verbatim}
Declaration(Class(:Description))
Declaration(Class(:Feature))
Declaration(Class(:Function))
Declaration(Class(:Interaction))
Declaration(Class(:Protein))
Declaration(ObjectProperty
 (:InteractionWith))
ObjectPropertyDomain(:InteractionWith 
 :Protein)
        
SubClassOf(:Feature :Description)
SubClassOf(:Interaction :Feature)
SubClassOf(:Function :Feature)
      
Declaration(NamedIndividual(:Q9Y5S9))
ClassAssertion(:Protein :Q9Y5S9)
Declaration(NamedIndividual(:Q9UBU9))
ClassAssertion(:Protein :Q9UBU9)
\end{verbatim}
        
  При этом приведенное описание трансформируется в~набор онтологических 
объектов:
  \begin{verbatim}
Declaration(NamedIndividual(:function1))
ClassAssertion(:Function :function1)

Declaration(NamedIndividual(:interaction1))
ClassAssertion(:Interaction :interaction1)

ObjectPropertyAssertion(:InteractionWith 
:interaction1 :Q9Y5S9)
ObjectPropertyAssertion(:InteractionWith 
:interaction1 :Q9UBU9)

Declaration(NamedIndividual(:protein1))
AnnotationAssertion( rdfs:comment 
:protein1 "EXON JUNCTION COMPLEX" )
ClassAssertion(:Protein :protein1)
\end{verbatim}

\vspace*{-16pt}

\section{Результаты и~обсуждение}
  

  
  В ходе работы была построена СОКС-грам\-ма\-ти\-ка, 
содержащая~179~правил (рис.~2).

\vspace*{-8pt}
  
  \subsection{Оценка покрытия}
  
  Для оценки была выбрана случайным образом тестовая выборка из 
100~предложений, 96~из них уникальные. Тестовая выборка содержит 
205~атомарных причин цитирования, 135~из них уникальные.
  
  Задача построения газетиров находится за пределами настоящей работы, 
поэтому в~тестовой выборке перед тестированием сущности, входящие 
в~газетиры, были вручную заменены на соответствующие им нетерминалы. 
Дополнительно в~грамматику были добавлены правила, позволяющие 
обрабатывать такие предобработанные входные данные.
  
  В тех случаях, где в~тестирующей выборке одна и~та же сущность могла быть 
описана более длинной или более короткой цепочкой, использовалась более 
короткая цепочка. Таким образом вручную были размечены классы: белок, 
вещество, болезнь, лекарство, химическая модификация.
  
  Полученные в~результате тестирования оценки покрытия представлены 
в~таблице.

{\small
  \begin{center}
  \tabcolsep=3pt
  \begin{tabular}{|l|c|}
  \multicolumn{2}{c}{Результаты тестирования покрытия}\\
  \multicolumn{2}{c}{\ }\\[-6pt]
  \hline
  \multicolumn{1}{|c|}{Тестирование} & Доля\\
  \hline
  Все атомарные причины цитирования&73\%\\
  Уникальные атомарные причины цитирования&43\%\\
  Все предложения&54\%\\
  Уникальные предложения&52\%\\
  \hline
  \end{tabular}
  \end{center}
}

\vspace*{6pt}
  


  
  Следует обратить внимание на значительный (в~1,7~раза) прирост покрытия 
при отключении процедуры удаления дубликатов из корпуса ато-\linebreak марных 
причин цитирования. Это является кос\-венным следствием большого числа 
дубликатов, которые, в~свою очередь, являются следствием огра\-ни\-чен\-ности 
выбранного языка (он использует только именные группы) и~его лексической 
огра\-ни\-чен\-ности (кураторы следуют инструкции, рег\-ламентирующей 
используемую лексику). Такие ограничения приводят к~значительному объему 
дуб\-ли\-рования в~корпусе. Это дает возможность при меньшем числе правил 
в~грамматике добиваться более высокого покрытия корпуса, что и~предложено 
в~настоящей статье.

 
  
  Очевидно, что более сложные конструкции обладают б$\acute{\mbox{о}}$льшим 
разнообразием и,~следовательно, меньшей степенью дублирования, что 
и~продемонстрировано на оценке покрытия полных предложений. Таким 
образом, для более сложных или менее ограниченных языков кажется 
осмысленным в~качестве предобработки выделять наиболее узко лишь такие 
конструкции, которые имеют сущности, значимые для составляемой онтологии. 
Для построения онтологий, описывающих объекты и~их свойства, такой 
предобработкой может служить выделение именных групп.

\vspace*{-6pt}
  
  \section{Заключение}
  
  В работе поставлена актуальная задача разработки новых онтологий на 
основе корпусных данных и~предложен подход к~ее решению. Для составления 
онтологий в~работе дано определение и~представлен алгоритм составления 
семантически ориентированных КС-грам\-ма\-тик. Важным аспектом подхода 
является использование в~качестве материала для построения онтологии 
корпуса предложений на ограниченном подмножестве естественного языка.
{\looseness=1

}
  
  Алгоритм опробован для текстов именных групп ограниченного языка, 
используемого в~базе UniProt для описания причин цитирования статьи, 
в~результате чего составлена грамматика и~разработан синтаксический 
анализатор таких причин цитирования.

\vspace*{-6pt}

{\small\frenchspacing
 {%\baselineskip=10.8pt
 \addcontentsline{toc}{section}{References}
 \begin{thebibliography}{99}
 
 \vspace*{-2pt}
\bibitem{1-al}
\Au{\mbox{Do{\!\!\ptb{\u{g}}}an} R.\,I., Leaman R., Lu~Zh.} Ncbi disease corpus: 
A~resource for 
disease name recognition and concept normalization~// J.~Biomed. Inform., 2014. 
Vol.~47. P.~1--10. doi: 10.1016/j.jbi.2013.12.006.
\bibitem{2-al}
\Au{Li Ch., Song R., Liakata~M., Vlachos~A., Seneff~S., Zhang~X.} Using word embedding for 
bio-event extraction~// 2015 Workshop on Biomedical Natural Language Processing 
(BioNLP 2015) Proceedings.~--- Beijing, China: ACL, 2015. P.~121--126. 
\bibitem{3-al}
\Au{Kim  J., Nguyen N., Wang~Yu., Tsujii~J., Takagi~T., Yonezawa~A.} The genia event and 
protein coreference tasks of the BioNLP shared task 2011~// BMC Bioinformatics, 2012. 
Vol.~13. Suppl.~11:S1. doi:10.1186/1471-2105-13-S11-S1.
\bibitem{4-al}
\Au{N$\acute{\mbox{e}}$dellec C., Bossy~R., Kim~Ji., Kim~Ju., Ohta~To., Pyysalo~S., 
Zweigenbaum~P.} Overview of BioNLP shared task 2013~// BioNLP Shared Task 2013 
Workshop (BioNLP-ST 2013) Proceedings.~---  ACL, 2013. P.~1--7.

\bibitem{6-al} %5
\Au{Tanabe M., Kanehisa~M.} Unit~1--12 using the KEGG database resource~// Current protocols in 
bioinformatics.~--- John Wiley\,\&\,Sons, Inc., 2012. P.~1.12.1--1.12.43.  doi: 10.1002/0471250953.bi0112s38.

\bibitem{5-al} %6
The UniProt Consortium. UniProt: A~hub for protein information~// Nucleic Acids Res., 
2015.  Vol.~43. P.~D204--D212. doi: 10.1093/nar/gku989.

\bibitem{7-al}
\Au{Tonkon M.\,J., Miller R.\,R., DeMaria~A.\,N., Vismara~L.\,A., Amsterdam~E.\,A., 
Mason~D.\,T.} Multifactor evaluation of the determinants of ischemic electrocardiographic 
response to maximal treadmill testing in coronary disease~// Am. J.~Med., 1977. Vol.~62. 
Iss.~3. P.~339--346. doi: 10.1016/0002-9343(77)90830-0.
\bibitem{8-al}
\Au{Giaretta P., Guarino~N.} Ontologies and knowledge bases towards a terminological 
clarification~// Towards very large knowledge bases.~--- Amsterdam: IOS Press.  
P.~25--32.
\bibitem{9-al}
\Au{Jones D., Bench-Capon~T., Visser~P.} Methodologies for ontology development~//  
IT\&KNOWS Conference, XV IFIP World Computer Congress Proceedings.~--- Budapest, 
1998.
\bibitem{10-al}
\Au{Reed S.\,L., Lenat D.\,B.} Mapping ontologies into Cyc~// AAAI 2002 Conference Workshop 
on Ontologies For The Semantic Web, 2002. P.~1--6.

\columnbreak


\bibitem{11-al}
\Au{Хомский Н.} Аспекты теории синтаксиса~/ Пер. В.\,А.~Звегинцева.~--- М.: Изд-во Моск. 
ун-та, 1972.  258~с. (\Au{Chomsky~N.} {Aspects of the theory of Syntax}.~---  MIT Press, 
1969. 261~p.)
\bibitem{12-al}
Criteria used to assign the pe level of entries. {\sf http:// www.uniprot.org/docs/pe\_criteria}.
\bibitem{13-al}
Controlled vocabulary. {\sf http://www.uniprot.org/help/\linebreak controlled\_vocabulary}.
\bibitem{14-al}
\textit{UniProt Consortium}. UniProt manual curation sop. {\sf 
http:// www.uniprot.org/docs/sop\_manual\_curation.pdf}.
\bibitem{15-al}
\Au{Beale A.} Spell checker oriented word lists. 1999--2015. 
{\sf http://wordlist.aspell.net/12dicts-readme}.
\bibitem{16-al}
\Au{Van~Rossum  G.} Python programming language~// USENIX Annual Technical 
Conference, 2007.
\bibitem{17-al}
\Au{Bird  S., Klein~E., Loper~E.} Natural language processing with Python.~--- O'Reilly Media, 
2009. 512~p.
\bibitem{18-al}
\Au{Horridge M., Patel-Schneider~P.\,F.} OWL 2 Web Ontology Language Manchester  
Syntax.~--- 2nd ed.~--- W3C Working Group Note, 2009. 
{\sf http://www.w3.org/TR/owl2-manchester-syntax}.
\end{thebibliography}

 }
 }

\end{multicols}

\vspace*{-3pt}

\hfill{\small\textit{Поступила в~редакцию 23.09.15}}

\vspace*{8pt}

%\newpage

%\vspace*{-24pt}

\hrule

\vspace*{2pt}

\hrule

%\vspace*{8pt}



\def\tit{BioNLP ONTOLOGY EXTRACTION FROM A~RESTRICTED LANGUAGE CORPUS 
WITH CONTEXT-FREE GRAMMARS}

\def\titkol{BioNLP ontology extraction from a~restricted language corpus 
with context-free grammars}

\def\aut{D.\,A.~Alexeyevsky}

\def\autkol{D.\,A.~Alexeyevsky}

\titel{\tit}{\aut}{\autkol}{\titkol}

\vspace*{-9pt}

\noindent
National Research University Higher School of Economics; 20~Myasnitskaya 
Str., Moscow 101000, Russian Federation

\def\leftfootline{\small{\textbf{\thepage}
\hfill INFORMATIKA I EE PRIMENENIYA~--- INFORMATICS AND
APPLICATIONS\ \ \ 2016\ \ \ volume~10\ \ \ issue\ 1}
}%
 \def\rightfootline{\small{INFORMATIKA I EE PRIMENENIYA~---
INFORMATICS AND APPLICATIONS\ \ \ 2016\ \ \ volume~10\ \ \ issue\ 1
\hfill \textbf{\thepage}}}

\vspace*{3pt}

  
  
    
  
\Abste{BioNLP is an emerging area of NLP that brings new challenging objects for 
language processing and new valuable resources for bioinformatics and medicine. 
One notable task in BioNLP is creating de-novo ontologies. This is generally 
a~tedious process; however, in some cases, it is possible to automate it to some extent. 
One such case is when a corpus of texts in a restricted subset of natural language is 
available. This paper presents a simple approach to automate ontology creation in 
such cases. The approach is aimed to simplify mapping of entities in natural texts to 
predefined ontologies wherever possible. The paper discusses which properties of the 
corpus enable the approach presented.}

\KWE{BioNLP; ontology creation; context-free grammar}


\DOI{10.14357/19922264160111}

\vspace*{-12pt}

\Ack
\noindent
The work was partly supported by the Russian Foundation for Basic
Research (project 15-07-09306).



%\vspace*{3pt}

  \begin{multicols}{2}

\renewcommand{\bibname}{\protect\rmfamily References}
%\renewcommand{\bibname}{\large\protect\rm References}

{\small\frenchspacing
 {%\baselineskip=10.8pt
 \addcontentsline{toc}{section}{References}
 \begin{thebibliography}{99}
\bibitem{1-al-1}
\Aue{Do{\!\!\ptb{\!\u{g}}}an, R. I., R.~Leaman, and Zh.~Lu.} 2014. Ncbi disease 
corpus: A~resource for disease name and concept normalization. 
\textit{J.~Biomed. Inform.} 47:1--10. doi: 10.1016/j.jbi.2013.12.006.
\bibitem{2-al-1}
\Aue{Li, Ch., R. Song, M.~Liakata, A.~Vlachos, S.~Seneff, and X.~Zhang}. 
2015. Using word embedding for bio-event extraction. \textit{2015 Workshop on 
Biomedical Natural Language Processing (BioNLP 2015) Proceedings}. Beijing, 
China: ACL. 121--126.
\bibitem{3-al-1}
\Aue{Kim,  J., N. Nguyen, Yu.~Wang,  J.~Tsujii, T.~Takagi, and A.~Yonezawa}. 
2012. The genia event and protein coreference tasks of the BioNLP shared task 
2011. \textit{BMC Bioinformatics} 13(Suppl.~11:S1).  
doi: 10.1186/1471-2105-13-S11-S1.
\bibitem{4-al-1}
\Aue{N$\acute{\mbox{e}}$dellec, C., R.~Bossy, Ji.~Kim, Ju.~Kim, To.~Ohta, 
S.~Pyysalo, and P.~Zweigenbaum}. 2013. Overview of BioNLP shared task 2013. 
\textit{BioNLP Shared Task 2013 Workshop (BioNLP-ST 2013) Proceedings}. 
ACL. 1--7.

\bibitem{6-al-1} %5
\Aue{Tanabe, M., and M.~Kanehisa}. 2012. Unit~1--12 using the KEGG database resource. 
\textit{Current protocols in bioinformatics}. 
John Wiley\,\&\,Sons, Inc. 1.12.1--1.12.43.   doi: 
10.1002/0471250953.bi0112s38.

\bibitem{5-al-1} %6
The UniProt Consortium. 2015. Uniprot: A~hub for protein information. 
\textit{Nucleic Acids Res.} 43:D204--D212. doi:10.1093/nar/gku989.
\bibitem{7-al-1}
\Aue{Tonkon, M.\,J., R.\,R.~Miller, A.\,N.~DeMaria, L.\,A.~Vismara, 
E.\,A.~Amsterdam, and D.\,T.~Mason}. 1977. Multifactor evaluation of the 
determinants of ischemic electrocardiographic response to maximal treadmill 
testing in coronary disease. \textit{Am. J.~Med.} 62(3):339--346. doi: 
10.1016/0002-9343(77)90830-0.
\bibitem{8-al-1}
\Aue{Giaretta, P., and N.~Guarino}. 1995. Ontologies and knowledge bases 
towards a terminological clarification. \textit{Towards very large knowledge 
bases}. Amsterdam: IOS Press. 25--32.
\bibitem{9-al-1}
\Aue{Jones, D., T.~Bench-Capon, and P.~Visser}. 1998. Methodologies for 
ontology development. \textit{IT\&KNOWS Conference, XV IFIP World 
Computer Congress Proceedings}. Budapest. 62--75.
\bibitem{10-al-1}
\Aue{Reed, S.\,L., and D.\,B.~Lenat.} 2002. Mapping ontologies into Cyc. 
\textit{AAAI 2002 Conference Workshop on Ontologies For The Semantic Web}. 
1--6.
\bibitem{11-al-1}
\Aue{Chomsky, N.} 1969. \textit{Aspects of the theory of Syntax}.  MIT Press. 
261~p. 
\bibitem{12-al-1}
Criteria used to assign the pe level of entries. Available at: {\sf 
http://www.uniprot. org/docs/pe\_criteria} (accessed January~21, 2016).
\bibitem{13-al-1}
Controlled vocabulary. Available at: {\sf 
http://www.uniprot. org/help/controlled\_vocabulary} (accessed January~21, 
2016).
\bibitem{14-al-1}
UniProt Consortium. Uniprot manual curation sop. Available at: {\sf 
http://www.uniprot.org/docs/sop\_manual\_\linebreak curation.pdf} (accessed January~21, 
2016).
\bibitem{15-al-1}
\Aue{Beale, A.} 1999--2015. \textit{Spell checker oriented word lists}. Available 
at: {\sf http://wordlist.aspell.net/12dicts-readme} (accessed January~21, 2016).
\bibitem{16-al-1}
\Aue{Van Rossum, G.} 2007. Python programming language. \textit{USENIX 
Annual Technical Conference}.
\bibitem{17-al-1}
\Aue{Bird, S., E.~Klein, and E.~Loper}. 2009. \textit{Natural language 
processing with Python}. O'Reilly Media. 512~p.
\bibitem{18-al-1}
\Aue{Horridge, M., and P.\,F.~Patel-Schneider}. 2009. OWL~2 Web Ontology 
Language Manchester Syntax. 2nd ed. W3C Working Group Note. Available at: 
{\sf http://www.w3.org/TR/owl2-manchester-syntax} (accessed January~21, 
2016).

\end{thebibliography}

 }
 }

\end{multicols}

\vspace*{-3pt}

\hfill{\small\textit{Received September 23, 2015}}

\Contrl

\noindent
  \textbf{Alexeyevsky Daniil A.} (b.\ 1983)~--- PhD student, Faculty of 
Humanities, National Research University Higher School of Economics; 
20~Myasnitskaya Str., Moscow 101000, Russian Federation; 
dalexeyevsky@hse.ru

   
\label{end\stat}


\renewcommand{\bibname}{\protect\rm Литература} %11+
\renewcommand{\figurename}{\protect\bf Figure}
\renewcommand{\tablename}{\protect\bf Table}

\def\stat{self-mul}

\def\tit{INFORMATICS AND~ITS~ROLE FOR~THE~STUDY OF~GENESIS 
AND~PROPERTIES OF~COMPLEX NATURAL SYSTEMS$^*$}

\def\titkol{Informatics and its role for the study of genesis 
and properties of complex natural systems}

\def\autkol{R.\,B.~Seyful-Mulyukov}

\def\aut{R.\,B.~Seyful-Mulyukov$^1$}

\titel{\tit}{\aut}{\autkol}{\titkol}


\index{Seyful-Mulyukov R.\,B.}
\index{Сейфуль-Мулюков Р.\,Б.}

{\renewcommand{\thefootnote}{\fnsymbol{footnote}}
\footnotetext[1] { The investigation was carried out according to the Program 
``Informatics 
methods in development of the petroleum origin theory and elaboration of new 
technologies for exploring petroleum and gas accumulations and providing energy 
security of the Russian Federation'' under the general theme ``Society 
informatization and information security.''}}

\renewcommand{\thefootnote}{\arabic{footnote}}
\footnotetext[1]{Institute of Informatics Problems, Federal Research Center 
``Computer Science and Control'' of the Russian Academy of 
Sciences, 44-2~Vavilov Str.,  Moscow 119333, Russian Federation, \mbox{rust@ipiran.ru}}




\def\leftfootline{\small{\textbf{\thepage}
\hfill INFORMATIKA I EE PRIMENENIYA~--- INFORMATICS AND APPLICATIONS\ \ \ 2017\ \ \ volume~11\ \ \ issue\ 1}
}%
 \def\rightfootline{\small{INFORMATIKA I EE PRIMENENIYA~--- INFORMATICS AND APPLICATIONS\ \ \ 2017\ \ \ volume~11\ \ \ issue\ 1
\hfill \textbf{\thepage}}}



\Abste{The paper considers the history of cognition of information as a~phenomenon and 
informatics as its quantitative and qualitative development. The logical connection between such 
notions as information, informatics, complexity, and complex natural self-organizing systems is 
investigated. It is considered that information, besides its usual traditional meaning, is one of the 
main properties of matter. Informatics is considered as an instrument for cognition of 
development and structure of complex natural systems. Petroleum is chosen as an example of 
such system. It is proved that petroleum, as well as each its hydrocarbon molecule, possesses 
corpuscular properties, and petroleum as a~whole has information volume calculated in bits. 
A~new approach is proposed for petroleum accumulations exploration. It is based on the fact 
that petroleum generation is a~discrete process. Consequently, the process of discovering 
petroleum accumulations has two stages. The first stage is characterized by static 
uncertainty
and the second stage is characterized by dynamic uncertainty. Both types of uncertainty need to 
be removed. The paper presents technologies and methods of solving these problems.}

\KWE{informatics; informatization; complex natural system; petroleum origin; petroleum 
exploration; static uncertainty; dynamic uncertainty}

\DOI{10.14357/19922264170111} 

%\vspace*{9pt}


\vskip 10pt plus 9pt minus 6pt

      \thispagestyle{myheadings}

      \begin{multicols}{2}

                  \label{st\stat}

\noindent
   Historically, ``information'' was understood as data about an event, a~state, or 
other characteristics of a~phenomenon that living species transmitted to each other. 
In particular, information is a~light, heat, or sound signal from a~natural 
phenomenon, which plants receive or to which they react. Information is a~signal 
of different types that terrestrial animals, pests, birds, as well as marine creatures 
receive or exchange with each other.
{\looseness=1

}
   
   In economics, main data are expressed with digits. In industry and agriculture, 
``information'' is the name of specifications of goods and services or unit 
measurements, such as weight, mass, volume, size, distance, and others. In 
education, ``information'' is all totality of the basic knowledge about nature, 
history, and laws. This knowledge has vital importance for development of 
mankind. In each sphere of activities, ``information" is understood in a~specific 
way.
   
   The notion of ``informatization'' appeared in the information theory and 
different spheres of information applications. Its appearance was caused by a~rapid 
increase of flow of information, which was used in everyday life, industry, science, 
culture, education, social, and other spheres.
   
   Information has to be transmitted, received, processed, interpreted, stored, and 
undergo many other manipulations. It is necessary for the right positioning of an 
individual or a~community in a~society and has vital importance for economic 
independency and national security. All of these activities are informatization.
   
   Informatization is not a~one-time campaign. Everyday activities introduced the 
public consciousness to the necessity of informatization of all spheres of social 
activities. K.~Kolin proved that informatization has to be perceived in the public 
consciousness as a~powerful instrument for qualitative modification of education, 
science development, new technologies application, improving management, and 
other activities. All of these activities have vital importance for development and 
national security~[1].
   
   This paper presents application of informatization to solving one of 
fundamental problems of natural sciences~--- the problem of genesis of petroleum 
and natural hydrocarbon gas. For this reason, the paper contains a~short introduction 
to the history of cognition of the notion of ``information.'' It does not mean only 
social phenomena or informatics as an instrument, which provides information 
storage, usage, transmission, and processing. ``Information'' also means the 
properties of matter, which are associated with the notion of a~``complex natural 
system.'' These systems can be cognized using the laws of informatics.
   
   Understanding of the notion of ``information'' depends on means of its 
transmission, usage, application, and many other factors and is always subjective. 
There is no exact definition of ``information,'' which would be universally 
recognized, and such definition is not possible in principle. The development of 
civilization and growth of our knowledge about matter, movement, time, and space 
resulted in a~new deeper and more comprehensive understanding of ``information.'' 
The most important achievement was the identification of ``information'' and 
``uncertainty'' that were measured by the C.~Shannon's mathematical theory of 
information~[2]. It was a~qualitative definition of ``information.'' For the first time 
in the history of information science, A.~Ursul presented an integrated 
philosophical definition of ``information'' that shows the relation between its 
quantitative and qualitative content~[3].
   
   The revolution of information cognition was the result of detection of its new 
meaning. Physicists M.~Planck and L.~de~Broglie~[4] investigated matter on 
atomic and subatomic levels and proved that besides its usual meaning as 
something existing in our consciousness, information is also one of the main 
properties of matter that exists beyond men's consciousness or wish. A.~Zeilinger 
proved that each elementary particle of an atom contains one bit of 
information~[5]. 
   
   Further investigation of qualitative and quantitative characteristics of 
information caused the appearance of new disciplines. First among them was 
cybernetics that N.~Wiener defined as a~``scientific study of control and 
communication in animals and machines''~[6]. The practical application of this 
idea was the computer. Later, the term ``informatics'' was introduced by 
K.~Steinbuch~[7]. Since~1966, informatics was positioned as a~science about 
collection, storage, distribution, retrieval, and use of scientific and technical 
information~[8].
    
    Russian and American scientists continued to investigate informatics and 
information science theory and applied problems, which are based on 
achievements of mathematics, physics, cybernetics, and philosophy. The approach 
of American information specialists and their understanding of ``information'' and 
``information science'' is best described in two monographs. The first one considers 
information science as a~metascience~[9]. Its integral parts are mathematics, 
linguistics, psychology, library science, engineering science, and computer 
science~[10]. Physics and cybernetics were predecessors of information science; 
therefore, they are not present in this list. However, these disciplines proved 
physical nature of information as one of its main properties.
    
    Another monograph is an official publication of the American Society for 
Information Science and Technology (ASIS\&T). The monograph follows the 
ideas of Otten and Debone. Information science is considered mainly as 
investigation of mental perception and interpretation of information as 
a~phenomenon existing in our consciousness. A~considerable part of the 
monograph is devoted to information in economics. In the case of market 
economy, ``information'' (in other words, ``uncertainty'' or ``information entropy'') 
is related to the notion of ``value.'' It means that the Shannon's theory, which deals 
with quantity of information, represents the quantitative side of information with 
its value. In this context, information has value since it can be bought or sold in 
order to decrease uncertainty and create a~product with a~larger value.
   
   The development of the information theory relates to I.~Gurevich~[11]. Basing 
on the postulate that information is one of the main fundamental properties of 
matter, he calculated information content of each chemical element of the 
Mendeleev table in bits. Besides, he proved that laws of fundamental sciences 
including informatics make it possible to understand development of complex 
natural and social systems.
{\looseness=1

}
   
   An indicative example, which demonstrates effectiveness of applying 
informatics laws to the study of complex natural systems, is the problem of origin 
of petroleum and natural hydrocarbon gas. According to the currently dominating 
model, petroleum origin is mainly the problem of petroleum geology and 
geochemistry. This model has become insufficient, which results in a~considerable 
decrease of effectiveness of exploration works. Surprisingly, application of 
informatics laws solves this problem. In informatics, the nature and properties of 
petroleum are represented as direct or indirect consequences of the fact that 
petroleum is a~complex self-developing natural system~[12, 13]. Its nature, main 
properties, and genesis become understandable, if one considers them in the 
context of the phenomenon of complexity and informatics laws.
   
The new understanding of deep inorganic nature of petroleum as a~phenomenon 
that is not related to the biosphere in any way leads to the necessity of changing the 
ideology of exploration of its commercial accumulations. The current methodology 
is based on the assumption that remainders of the biosphere's plants and animals 
served as raw material for petroleum and gas generation. However, nobody has 
ever proved the existence of mother rocks generating petroleum and their ability to 
transform organic matter into hydrocarbon molecules for sure. Nobody has ever 
proved that petroleum is able to migrate through geological rocks, so media keeps 
its initial hydrocarbon composition. The deep inorganic nature of petroleum proves 
its place in the hierarchy of organization of matter on the Earth. There are three 
main forms of organization of matter. The higher form is represented by billions of 
\textit{cells} of living species. The intermediary form is represented by thousands 
of petroleum hydrocarbon \textit{molecules}. The lower form is represented by 
several \textit{atoms} of chemical elements composing crystals of rocks. 

The most simple petroleum hydrocarbon molecules transform into the most 
complex ones, which is a~self-organizing discrete process. As the result, a~complex 
natural system appears, which possesses several properties. This paper considers 
the properties, which are the most significant for petroleum generation, namely, 
\textit{existence}, \textit{development}, and \textit{cognoscibility}~[14]. These 
properties are typical for any complex natural system.

System uniqueness is one of existence features. Petroleum is a~unique phenomenon 
that is created by a~unique set of geological, geochemical, fluid-dynamic, and other 
natural conditions. They are always different for each period of the Earth's history. 
Petroleum composition and structure correspond completely to the current period 
of the Earth's development. 
     
     System constant movement is one of \textit{development} features. 
Petroleum is permanently migrating and therefore changing. Petroleum, which is 
conserved in one place by geological media and keeps its original hydrocarbon 
composition for hundreds of millions of years, does not exist anymore. 
Consequently, inside the Earth, there is no Devonian (360~M years old), 
Carboniferous (300~M years old), Permian (251~M years old), and Jurassic 
(145~M years old) petroleum. Its generation does not take millions of years, but 
hundreds or thousands of years. Petroleum generation has the following stages: 
hydrocarbon radicals and methane molecules appear in the Earth's upper mantle, 
migrate through the Earth's crust, interact with rocks of geological media, 
transform into hydrocarbon molecules of different kinds, and finally, accumulate 
within a~reservoir as petroleum.
{\looseness=1

}
     
     \textit{Cognoscibility} of a~complex system pertains to the gnoseological 
aspect of petroleum nature. There is a~principal question: Is our perception of 
petroleum adequate to its real nature and age or nonadequate? There are two 
possible answers to this question. One answer corresponds to the organic paradigm 
of petroleum genesis, which considers that composition, properties, and other 
features of petroleum just pumped from a~reservoir are related to the phenomena, 
which were generated hundreds of millions years ago. It means that animals and 
plants of the previous epoch, which served as a~source for kerogen generation, had 
the same composition as they have now. Another opposite idea is that petroleum 
composition and genesis correspond completely to the modern epoch of the Earth's 
development. 
     
     It is absolutely clear that we cognize composition and features of a~complex 
system as a~natural phenomenon, which is generated by all totality of modern 
geological, geochemical, thermodynamic, and other conditions. Otherwise, one has 
to accept that~300, 200, and~100~M years ago geology, geochemistry, 
thermodynamics, and many other natural conditions of the Earth's interior were the 
same as they are now. It is impossible in principle. So, according to the 
\textit{cognoscibility} feature, petroleum is a~modern complex abiogenic natural 
system, which consists of several thousands of hydrocarbon molecules, which are 
not related to the biosphere.
{\looseness=1

}
     
     The petroleum example allows demonstrating one more important feature. 
The necessity of informatization of scientific research follows from the fact that 
information is one of the main properties of matter. Petroleum characteristics have 
never been considered before in this context. Petroleum possesses physical and 
chemical properties, as well as information content, which corresponds to the 
quantity of carbon and hydrogen atoms, which compose hydrocarbon 
molecules~\cite{13-seif}. In this case, one has to consider that any atom has 
information volume that is calculated in bits. Petroleum consists of hydrocarbon 
molecules on~95\%. They are divided into three main groups: paraffin  
(30\%--35\%), naphthenic (25\%--75\%), and aromatic (10\%--15\%). There are 
different kinds of petroleum with different quantities of hydrocarbon molecules of 
different kinds: light oil (C$_{32}$H$_{66}$SN), low-gravity oil 
(C$_{32}$H$_{66}$OSN), and bitumen (C$_{45}$H$_{51}$O$_2$SN). 
Correspondingly, their information volume is~16\,224, 16\,729, and~17\,789~bits.
     
     Application of the complex natural system ideology and informatics laws 
opens new possibilities for petroleum, such as exploration in different geological 
media. Therefore, the procedure of exploration of commercial petroleum 
accumulations requires reconsideration. Static and dynamic uncertainty are typical 
for a~complex natural self-organizing system. Both types of uncertainty have to be 
removed in order to find the place where petroleum is accumulated. This 
exploration strategy was proposed for the first time in~\cite{12-seif}.
     
     Static uncertainty applies to geological elements, whose location in the 
Earth's interior has not changed during all the time of petroleum generation and 
accumulation. These geological elements are trap, reservoir, impermeable 
covering, and canal for petroleum migration. Petroleum accumulation could not 
have happened without these geological elements. They must be present together, 
and their static uncertainty has to be removed in the Aristotelian logic terms~--- 
``yes'' or ``no'' only. The units used to express exact characteristics of a~given 
geological element are its size, density, permeability, fracturing, degree, as well as 
petrographic and chemical composition of reservoir, trap, and others. All of these 
characteristics can be successfully determined by seismic survey technologies. 
However, problems arise when all geological elements, provided by three-dimensional (3D) 
mathematical models of petroleum bearing basins, are present, but exploration 
wells turn out to be dry. Currently, the average statistical percent of successful 
exploration does not exceed~30\%. 
     
     Therefore, besides static uncertainty, there exists another type of uncertainty 
that does not deal with stable geological elements, but with results of a~dynamic 
discrete process. The result is a~petroleum accumulation, which is located 
sometimes under~10~km of sedimentary rocks. Fixing the stages of petroleum 
generation and the trajectory of its migration in time and space is impossible. It 
means that one cannot present a~discrete process as a~Cantorian set and the result 
of a~discrete process cannot be determined unambiguously by the Boolean classic 
logic, since mathematical calculation of this logic needs alternating values~``1'' 
or~``0'' only. 
     
     Hence, the problem of removing dynamic uncertainty and deciding whether 
a~petroleum accumulation exists or not does not have a~unique unambiguous 
solution. If a~problem does not have a~unique solution in principle, one can pass to 
its multiple-valued solution. In this case, intermediate values are introduced. The 
probability of accumulation presence in deep strata can be expressed with values 
0.1, 0.2, 0.25,\ $\ldots$\,,\ 0.7,\ $\ldots$\ until~0.95. It means that if static uncertainty was 
removed for sure, then the probability of removing its dynamic uncertainty could 
be calculated by applying the fuzzy set theory equations~\cite{15-seif}. Petroleum 
geologists and geophysicists possess all totality of information about the Earth's 
interior, including its 3D mathematical model, as well as experience and intuition. 
However, in many cases, this information is insufficient for planning well drilling 
in a~certain point. One can get additional data on that point calculated 
mathematically to be sure that the probability of discovering a~petroleum 
accumulation is equal to~25\%, 75\%, or even~95\%.
     
     The fuzzy set can consist of multiple direct and indirect geochemical indices 
of hydrocarbon molecules detected in different elements of environment. Light 
gasiform hydrocarbon molecules belong only to petroleum, which migrated from 
the pool to the surface and accumulated in the soil, snow, plants, air located close 
to the surface, and other elements of environment. Hydrocarbon molecules can be 
detected by modern geochemical methods.
     
     The new approach described in the 
paper made it possible to propose a~solution of the fundamental problems of 
petroleum and gas genesis and of exploration of their commercial accumulations in 
a~nontraditional, nontrivial way. 
    
\renewcommand{\bibname}{\protect\rmfamily References}


{\small\frenchspacing
{%\baselineskip=10.8pt
\begin{thebibliography}{99}

\bibitem{1-seif}
\Aue{Kolin, K.\,K.} 2010. \textit{Filosofskie problemy informatiki} 
[Philosophic problems of informatics]. Moscow: BINOM. 259~p.
\bibitem{2-seif}
\Aue{Shannon, C.\,E.} 1948. A~mathematical theory of communication. 
\textit{Bell Syst. Tech.~J.} 27:379--423, 623--656.
\bibitem{3-seif}
\Aue{Ursul, A.\,D.} 2010. \textit{Priroda informatsii: Filosofskiy ocherk} 
[The nature of information: A~philosophic essay]. 2nd ed. Chelyabinsk: 
CHGAKI. 231~p.
\bibitem{4-seif}
\Aue{De Broiglie, L.} 1927. Wave mechanics and the atomic structure of 
matter and radiation. \textit{J.~Phys. Radium} 8(5):225--241.
\bibitem{5-seif}
\Aue{Zeilinger, A.} 1999. A~foundation principal for quantum mechanics. 
\textit{Found. Phys.} 29(4):631--643.
\bibitem{6-seif}
\Aue{Wiener, N.} 1961. \textit{Cybernetics, or control and communication 
in the animal and the machine}. 2nd rev. ed. Cambridge: MIT Press. 
232~p.
\bibitem{7-seif}
\Aue{Steinbuch, K.} 1957. Informatik. \textit{Automatische 
Informationsverarbeitung, SEG-Nachrichten} (Technische Mitteilunger der 
Standard Elektrik Gruppe). No.~4. 171~p.
\bibitem{8-seif}
\Aue{Mikhaylov, A.\,I., A.\,I.~Chernyy, and R.\,S.~Gilyarevskiy}. 1968. 
\textit{Osnovy informatiki} [The foundations of informatics]. Moscow: 
Nauka. 425~p.
\bibitem{9-seif}
\Aue{Otten, K., and A.~Debons}. 1970. Towards a~metascience of 
information: Informatology. \textit{J.~Am. Soc. Inform. 
Sci.} 21:89--94.
\bibitem{10-seif}
\Aue{Norton, M.\,J.} 2010. \textit{Introductory concepts in information 
science}. 2nd ed. ASIST monograph ser. Information Today. 210~p.
\bibitem{11-seif}
\Aue{Gurevich, I.\,M.} 2007. \textit{Zakony informatiki~--- osnova 
stroeniya i~poznaniya slozhnykh system} [The laws of informatics as a~basis 
of cognizing complex systems]. Moscow: TORUS PRESS. 399~p.
\bibitem{12-seif}
\Aue{Heylighen, F.} 2008. Сomplexity and self-organization. 
\textit{Encyclopedia of library and information sciences}. Eds. M.\,J.~Bates 
and M.\,N.~Maack. Taylor \& Frances. 9~p. Available at: {\sf 
http://pespmc1.vub.ac.be/Papers/ELIS-complexity.pdf} (accessed 
February~8, 2017).
\bibitem{13-seif}
\Aue{Seyful-Mulyukov, R.\,B.} 2012. \textit{Neft' i~gaz. Glubinnaya priroda 
i~ee prikladnoe znachenie} [Petroleum and gas. Deep nature and its applied 
meaning].  Moscow: TORUS PRESS. 214~p. 
\bibitem{14-seif}
\Aue{Seyful-Mulyukov, R.\,B.} 2010. \textit{Neft'~--- uglevodorodnye 
posledovatel'nosti: Analiz modeley genezisa i~evolyutsii} [Petroleum~--- 
hydrocarbon sequences: An analysis of models of genesis and evolution]. 
Moscow: 11~format. 173~p.
\bibitem{15-seif}
\Aue{Zadeh, L.\,A.} 1975. The concept of linguistic variable and its 
application to approximate reasoning. \textit{Inform. Sci.} 8:199--249,  
310--357; 9:43--80.
\end{thebibliography} }
 }

\end{multicols}

\vspace*{-6pt}

\hfill{\small\textit{Received October 11, 2016}}

\vspace*{-24pt}

\Contrl


\vspace*{3pt}

\noindent
\textbf{Seyful-Mulyukov Rustem B.} (b.\ 1928)~--- Doctor of Science in 
geology, professor, Head of Laboratory, Institute of Informatics Problems, 
Federal Research Center ``Computer Science and Control'' of the Russian 
Academy of Sciences, 44-2~Vavilov Str,  Moscow 119333, Russian Federation; 
\mbox{rust@ipiran.ru}


%\vspace*{8pt}

%\hrule

%\vspace*{2pt}

%\hrule

\newpage

\vspace*{-24pt}



\def\tit{ИНФОРМАТИКА И~ЕЕ РОЛЬ В~ПОЗНАНИИ 
ОБРАЗОВАНИЯ И~СВОЙСТВ СЛОЖНОЙ ПРИРОДНОЙ 
СИСТЕМЫ}

\def\aut{Р.\,Б.~Сейфуль-Мулюков}


\def\titkol{Информатика и~ее роль в~познании 
образования и~свойств сложной природной 
системы}

\def\autkol{Р.\,Б.~Сейфуль-Мулюков}

%{\renewcommand{\thefootnote}{\fnsymbol{footnote}}
%\footnotetext[1]{Работа проводится при финансовой поддержке Программы
%стратегического развития Петрозаводского государственного университета в~рамках
%на\-уч\-но-ис\-сле\-до\-ва\-тель\-ской деятельности.}}


\titel{\tit}{\aut}{\autkol}{\titkol}

\vspace*{-12pt}

\noindent
Институт проблем информатики Федерального исследовательского центра <<Информатика 
и~управление>>
Российской академии наук

\vspace*{6pt}

\def\leftfootline{\small{\textbf{\thepage}
\hfill ИНФОРМАТИКА И ЕЁ ПРИМЕНЕНИЯ\ \ \ том\ 11\ \ \ выпуск\ 1\ \ \ 2017}
}%
 \def\rightfootline{\small{ИНФОРМАТИКА И ЕЁ ПРИМЕНЕНИЯ\ \ \ том\ 11\ \ \ выпуск\ 1\ \ \ 2017
\hfill \textbf{\thepage}}}



\Abst{Рассматривается история познания феномена <<информации>> 
и~информатики как междисциплинарной науки, изучающей качественные 
и~количественные особенности ее практических приложений. 
Представляется логическая связь таких широко распространенных понятий, 
как информация, информатика, сложность, сложные природные 
самоорганизующиеся системы. Принимается во внимание, что информация 
кроме традиционного, общепринятого значения является одним из свойств 
материи. Информатика наряду с~другими особенностями является 
инструментом познания развития и~строения сложных природных 
самоорганизующихся систем. В~качестве примера такой системы выбрана 
нефть. Доказывается, что нефть обладает корпускулярными свойствами 
и~каждая молекула углеводорода имеет объем информации в~битах. 
Предлагается новый подход к~поиску месторождений нефти, основанный на 
том факте, что ее образование~--- это дискретный процесс. Соответственно, 
обнаружение его результата является раскрытием статической 
и~динамической неопределенности. Рассматриваются методы и~технологии 
их раскрытия.}

\KW{информатика; информатизация; природные сложные системы; 
образование нефти; поиски нефти; статическая неопределенность; 
динамическая неопределенность} 
   
\DOI{10.14357/19922264160111} 

%\vspace*{18pt}


 \begin{multicols}{2}

\renewcommand{\bibname}{\protect\rmfamily Литература}
%\renewcommand{\bibname}{\large\protect\rm References}

{\small\frenchspacing
{%\baselineskip=10.8pt
\begin{thebibliography}{99}
\bibitem{1-seif-1}
\Au{Колин К.\,К.} Философские проблемы информатики.~--- М.: БИНОМ, 
2010. 259~с.
\bibitem{2-seif-1}
\Au{Shannon C.\,E.} A~mathematical theory of communication~// Bell Syst. 
Tech.~J., 1948. Vol.~27. P.~379--423, 623--656.
\bibitem{3-seif-1}
\Au{Урсул А.\,Д.} Природа информации: Философский очерк.~--- 2-е изд.~--- 
Челябинск: ЧГАКИ, 2010. 231~с.
\bibitem{4-seif-1}
\Au{De Broiglie L.} Wave mechanics and the atomic structure of matter and 
radiation~// J.~Phys. Radium, 1927. Vol.~8. No.\,5. P.~225--241.
\bibitem{5-seif-1}
\Au{Zeilinger A.} A~foundation principal for quantum mechanics~// Found. 
Phys., 1999. Vol.~29. No.\,4. P.~631--643.
\bibitem{6-seif-1}
\Au{Wiener N.} Cybernetics, or control and communication in the animal and the 
machine.~--- 2nd rev. ed.~--- Cambridge: MIT Press, 1961. 232~p.
\bibitem{7-seif-1}
\Au{Steinbuch K.} Informatik~// Automatische Informationsverarbeitung,  
SEG-Nachrichten (Technische Mitteilunger der Standard Elektrik Gruppe), 1957. 
No.\,4. 171~p.
\bibitem{8-seif-1}
\Au{Михайлов А.\,И., Черный~А.\,И., Гиляревский~Р.\,С.} Основы 
информатики.~--- М.: Наука, 1968. 425~с.
\bibitem{9-seif-1}
\Au{Otten K., Debons~A.} Towards a~metascience of information: 
Informatology~// J.~Am. Soc. Inform. Sci., 1970. Vol.~21.  
P.~89--94.
\bibitem{10-seif-1}
\Au{Norton M.\,J.} Introductory concepts in information science.~--- 2nd ed.~--- 
ASIST monograph ser.~--- Information Today, 2010. 210~p.
\bibitem{11-seif-1}
\Au{Гуревич И.\,М.} Законы информатики~--- основа строения и познания 
сложных систем.~--- М.: ТОРУС ПРЕСС, 2007. 399~с.
\bibitem{12-seif-1}
\Au{Heylighen F.} Сomplexity and self-organization~// Encyclopedia of library 
and information sciences~/ Eds. M.\,J.~Bates, M.\,N.~Maack.~--- Taylor \& 
Frances, 2008. 20~p. {\sf http://pespmc1.vub.ac.be/Papers/ELIS-complexity.pdf}.
\bibitem{13-seif-1}
\Au{Сейфуль-Мулюков Р.\,Б.} Нефть и газ. Глубинная природа и~ее 
прикладное значение.~--- М: ТОРУС ПРЕСС, 2012. 214~с.
\bibitem{14-seif-1}
\Au{Сейфуль-Мулюков Р.\,Б.} Нефть~--- углеводородные последовательности: 
анализ моделей генезиса и эволюции.~--- М.: 11~формат, 2010. 173~с.
\bibitem{15-seif-1}
\Au{Заде Л.} Понятие лингвистической переменной и~его применение 
к~принятию приближенных решений.~--- М.: Мир, 1976. 176~с.

\end{thebibliography}
} }

\end{multicols}

 \label{end\stat}

 \vspace*{-3pt}

\hfill{\small\textit{Поступила в~редакцию  11.10.2016}}
%\renewcommand{\bibname}{\protect\rm Литература}
\renewcommand{\figurename}{\protect\bf Рис.}
\renewcommand{\tablename}{\protect\bf Таблица}  %+12



%%%%%%%%%%%%%%%%%%%%%%%%%%%%%%%%%%%%%%%%%%%%%%%

%\def\stat{rez}
{%\hrule\par
%\vskip 7pt % 7pt
\raggedleft\Large \bf%\baselineskip=3.2ex
Р\,Е\,Ц\,Е\,Н\,З\,И\,И \vskip 17pt
    \hrule
    \par
\vskip 6pt plus 6pt minus 3pt }

%\thispagestyle{headings} %с верхним колонтитулом
%\thispagestyle{myheadings} %с нижним колонтитулом, но в верхнем РЕЦЕНЗИИ

\def\tit{НОВАЯ КНИГА И.\,Н.~СИНИЦЫНА, А.\,С.~ШАЛАМОВА <<ЛЕКЦИИ ПО ТЕОРИИ 
ИНТЕГРИРОВАННОЙ ЛОГИСТИЧЕСКОЙ ПОДДЕРЖКИ>> (М.: ТОРУС ПРЕСС, 2012. 624~с.)}

%1
\def\aut{Д.ф.-м.н., профессор С.\,Я.~Шоргин}

\def\auf{\ }

\def\leftkol{\ % РЕЦЕНЗИИ
}

\def\rightkol{ \ } 

%\def\leftkol{\ } % ENGLISH ABSTRACTS}

%\def\rightkol{\ } %ENGLISH ABSTRACTS}

%\def\leftkol{РЕЦЕНЗИИ}

%\def\rightkol{РЕЦЕНЗИИ}

\titele{\tit}{\aut}{\auf}{\leftkol}{\rightkol}
\vspace*{-18pt}


     \label{st\stat}

     \begin{multicols}{2}
     {\small
     {\baselineskip=10.1pt
     

      В книге представлено системное изложение теоретических основ одного из новейших 
направлений в \mbox{об\-ласти} экономики послепродажного обслуживания изделий наукоемкой 
продукции (ИНП) длительного пользования~--- интегрированной логистической поддержки
(ИЛП). 
{\looseness=1

}

Приведены также результаты новых работ, выполненных в Институте проблем информатики 
Российской академии наук в рамках научного направления <<Информационные технологии и 
анализ сложных сис\-тем>>.
 {%\looseness=1

}
     
      Излагаемые в книге научные подходы позво\-ляют карди\-наль\-но реформировать 
существующие системы производства и эксплуатации ИНП путем создания и внед\-ре\-ния 
методов рационального и оптимального управ\-ле\-ния процессами расходования 
вре\-мен\-н$\acute{\mbox{ы}}$х, 
мате\-ри\-аль\-ных, трудовых и других ресурсов на всех стадиях жизненного цикла изделий (ЖЦИ) по 
критериям экономической целесообразности и эф\-фек\-тив\-ности.
  {\looseness=1

}
    
      В книге приведен краткий обзор причин возник\-новения и
      развития CALS-методологии как основы 
современных международных стандартов по созданию и функционированию глобальных 
ин\-фор\-ма\-ци\-он\-но-ком\-му\-ни\-ка\-ци\-он\-ных систем, ее ключевых возможностей и эффективности 
результатов ее использования. 
Авторы %\linebreak 
предлагают ряд научных обоснований для разработки 
единой теории проектирования и управления систем ИЛП для полноценного использования 
преимуществ %\linebreak
 суще\-ст\-ву\-ющей методологии, определяют \mbox{общую} структурную схему 
комплексной системы <<ИНП-СППО>> и необходимость разработки для ее описания 
гибридных стохастических моделей.
{%\looseness=1

}

%\columnbreak
      
      Книга состоит из пяти частей, где последовательно излагается материал по каждой из 
следующих тем: <<Интегрированная логистическая поддержка>>, <<Теория гибридных 
стохастических систем и компьютерная поддержка исследований и разработок>>, <<Основы 
математического моделирования, анализа и синтеза систем послепродажного обслуживания>>, 
<<Определение и анализ показателей экспортного потенциала ИНП при проектировании>>, 
<<Задачи управления поддержкой послепродажного обслуживания>>, а также 
<<Моделирование инвестиционных процессов ИЛП в условиях неравновесных финансовых 
рынков>>. 
   
      В конце каждой главы приведены выводы и даны вопросы и задания для 
самоконтроля. В~приложениях содержатся основные определения по программам работ по 
анализу ИЛП, логистическим базам данных и компьютерным решениям, эквивалентной статистической 
линеаризации нелинейных преобразований ИЛП, справочный материал, а также развернутые 
уравнения для вероятностных характеристик.


      \def\leftkol{РЕЦЕНЗИИ}

\def\rightkol{РЕЦЕНЗИИ} 

      
      Книга заинтересует широкий круг специалистов и может быть использована научными 
проектными организациями в сфере промышленного производства ИНП. Большое количество 
иллюстраций, примеров и вопросов, обращенных к читателю, позволяет использовать книгу 
также в качестве учебного пособия для студентов и аспирантов машиностроительных, 
транспортных и~других специальностей, а также для самостоятельного изучения. 
{%\looseness=-1

}

Книга 
представляет несомненный интерес для специалистов и студентов в области прикладной 
математики и информатики.
    

}

}
\end{multicols}

%\newpage

\def\stat{authorsrus}
{%\hrule\par
%\vskip 7pt % 7pt
\raggedleft\Large \bf%\baselineskip=3.2ex
О\,Б\ \ А\,В\,Т\,О\,Р\,А\,Х \vskip 17pt
    \hrule
    \par
\vskip 21pt plus 8pt minus 4pt }


\def\tit{\ }

\def\aut{\ }

\def\auf{\ }

\def\leftkol{\ } % ENGLISH ABSTRACTS}

\def\rightkol{ОБ АВТОРАХ} %ENGLISH ABSTRACTS}

\titele{\tit}{\aut}{\auf}{\leftkol}{\rightkol}
      
            \label{st\stat}



\vspace*{24pt}

\begin{multicols}{2}




\noindent
\textbf{Архипов Олег Петрович} (р.\ 1948)~---
кандидат технических наук, директор Орловского филиала Института проб\-лем информатики
Российской академии наук
%302025, г.Орел, Московское шоссе, д.137

\vspace*{3pt}

\noindent
\textbf{Бирюкова Татьяна Константиновна} (р.\ 1968)~---
кандидат фи\-зи\-ко-ма\-те\-ма\-ти\-че\-ских наук, старший научный сотрудник Института проб\-лем информатики
Российской академии наук

\vspace*{3pt}

\noindent 
\textbf{Бобков  Сергей Геннадьевич} (р.\ 1955)~---
доктор технических наук,  заведующий отделением На\-уч\-но-ис\-сле\-до\-ва\-тель\-ско\-го 
института системных исследований Российской академии наук
%117218, Москва, Нахимовский просп., 36, к.1 

\vspace*{3pt}

\noindent \textbf{Васильев Николай Семенович} (р.\ 1952)~--- доктор 
фи\-зи\-ко-ма\-те\-ма\-ти\-че\-ских наук, профессор, 
МГТУ им.\ Н.\,Э.~Баумана 
%, Москва 105005, 2-я Бауманская ул., д.~5,

\vspace*{3pt}

\noindent
\textbf{Гершкович Максим Михайлович} (р.\ 1968)~---
старший научный сотрудник Института проб\-лем информатики
Российской академии наук

\vspace*{3pt}

\noindent 
\textbf{Дьяченко Юрий Георгиевич} (р.\ 1958)~--- кандидат технических наук, 
старший научный сотрудник Института проб\-лем информатики
Российской академии наук

\vspace*{3pt}

\noindent 
\textbf{Ерошенко Александр Андреевич} (р.\ 1989)~--- аспирант кафедры 
математической статистики факультета вычисли\-тельной математики и кибернетики 
Московского государственного университета им.\ М.\,В.~Ломоносова
%119991, Москва ГСП-1, Ленинские горы, д.\ 1, стр. 52

\vspace*{3pt}
 
\noindent 
\textbf{Захаров Виктор Николаевич} (р.\ 1948)~--- 
доктор технических наук, доцент, ученый секретарь Института проб\-лем информатики
Российской академии наук

\vspace*{3pt}

\noindent
\textbf{Зейфман Александр Израилевич} (р.\ 1954)~---
доктор фи\-зи\-ко-ма\-те\-ма\-ти\-че\-ских наук, профессор, 
заведующий кафедрой Вологодского государственного университета; 
старший научный сотрудник Института проб\-лем информатики
Российской академии наук; главный научный сотрудник ИСЭРТ Российской академии наук

\vspace*{3pt}

\noindent
\textbf{Зыкин Сергей Владимирович} (р.\ 1959)~--- 
доктор технических наук, профессор, заведующий лабораторией Института математики 
им.\ С.\,Л.~Соболева Сибирского отделения Российской академии наук, Новосибирск 
%630090, пр.\ ак.\ Коптюга, 4 

\vspace*{4pt}

\noindent
\textbf{Киреев Владимир Иванович} (р.\ 1938)~---
доктор фи\-зи\-ко-ма\-те\-ма\-ти\-че\-ских наук, профессор Московского 
государственного горного университета
%Адрес: Россия, 119991, г. Москва, Ленинский проспект, д. 6

%\columnbreak

\vspace*{4pt}

\noindent
\textbf{Козеренко Елена Борисовна} (р.\ 1959)~---
кандидат филологических наук, заведующая лабораторией Института проб\-лем информатики
Российской академии наук

\vspace*{4pt}

\noindent
\textbf{Королев Виктор Юрьевич} (р.\ 1954)~--- доктор
фи\-зи\-ко-ма\-те\-ма\-ти\-че\-ских наук, профессор кафедры математической 
статистики факультета вычисли\-тельной математики и кибернетики 
Московского государственного университета; 
ведущий научный сотрудник Института проб\-лем информатики
Российской академии наук

\vspace*{4pt}

\noindent
\textbf{Коротышева Анна Владимировна} (р.\ 1988)~---
старший преподаватель Вологодского государственного университета

\vspace*{4pt}

\noindent 
\textbf{Кун Де Турк} (р.\ 1981)~--- научный сотрудник 
исследовательской группы SMACS факультета телекоммуникаций и обработки информации
Университета Гента, Бельгия
%В-9000 Гент, Бельгия

\vspace*{4pt}

\noindent
\textbf{Лупенцов Олег Сергеевич} (р.\ 1986)~---
аспирант Омского государственного института сервиса
%Омск 644043, ул.\ Певцова 13

\vspace*{4pt}

\noindent
\textbf{Лучко Олег Николаевич} (р.\ 1961)~---
кандидат педагогических наук, профессор, заведующий кафедрой 
Омского государственного института сервиса
%Омск 644043, ул.\ Певцова 13

\vspace*{4pt}

\noindent
\textbf{Малашенко Юрий Евгеньевич} (р.\ 1946)~---
доктор фи\-зи\-ко-ма\-те\-ма\-ти\-че\-ских наук, заведующий сектором 
Вычислительного центра им.\ А.\,А.~Дородницына Российской академии наук
%Адрес: 119333, Москва, ул. Вавилова, 40,

\vspace*{4pt}

\noindent
\textbf{Маньяков Юрий Анатольевич} (р.\ 1984)~---
кандидат технических наук, научный сотрудник Орловского филиала Института проб\-лем информатики
Российской академии наук
%302025, г.Орел, Московское шоссе, д.137

\vspace*{4pt}

\noindent
\textbf{Маренко Валентина Афанасьевна} (р.\ 1951)~---
кандидат технических наук, доцент, старший научный сотрудник 
Института математики им.\ С.\,Л.~Соболева Сибирского отделения Российской академии наук
%Новосибирск 630090, пр. ак. Коптюга, 4 

\vspace*{3pt}

\noindent 
\textbf{Морозов Евсей Викторович} (р.\ 1947)~--- доктор 
фи\-зи\-ко-ма\-те\-ма\-ти\-че\-ских, профессор, ведущий научный сотрудник 
Института прикладных математических исследований Карельского научного центра Российской
академии наук; 
%%185910 Россия, Республика Карелия, г.\ Петрозаводск, ул.\ Пушкинская, 11
профессор Петрозаводского государственного университета, Петрозаводск
%185910 Россия, Республика Карелия, г.\ Петрозаводск, пр.\ Ленина, 33

%\pagebreak

\vspace*{3pt}

\noindent
\textbf{Назарова Ирина Александровна} (р.\ 1966)~---
кандидат фи\-зи\-ко-ма\-те\-ма\-ти\-че\-ских наук, 
научный сотрудник Вычислительного центра им.\ А.\,А.~Дородницына Российской академии наук 
%Адрес: 119333, Москва, ул. Вавилова, 40

\vspace*{3pt}

\noindent
\textbf{Павлов Игорь Валерианович} (р.\ 1945)~--- 
доктор фи\-зи\-ко-ма\-те\-ма\-ти\-че\-ских наук, профессор МГТУ им.\ Н.\,Э.~Баумана 
%Москва 105005, 2-я Бауманская ул., д.~5 

%\pagebreak

\vspace*{3pt}

\noindent 
\textbf{Потахина Любовь Викторовна} (р.\ 1989)~--- аспирантка
Института прикладных математических исследований Карельского научного центра
Российской академии наук; 
%%185910 Россия, Республика Карелия, г.\ Петрозаводск, ул.\ Пушкинская, 11
инженер Петрозаводского государственного университета, Петрозаводск
%185910 Россия, Республика Карелия, г.\ Петрозаводск, пр.\ Ленина, 33

\vspace*{3pt}

\noindent 
\textbf{Рождественский Юрий Владимирович} (р.\ 1952)~--- 
кандидат технических наук, заведующий сектором Института проб\-лем информатики
Российской академии наук

\vspace*{3pt}

\noindent 
\textbf{Синицын Игорь Николаевич} (р.\ 1940)~--- доктор технических наук,
профессор, заслуженный деятель\linebreak\vspace*{-12pt}

\columnbreak

\noindent
 науки РФ, заведующий отделом Института проб\-лем информатики
Российской академии наук

\vspace*{7pt}


\noindent
\textbf{Сиротинин Денис Олегович} (р.\ 1984)~---
кандидат технических наук, научный сотрудник Орловского филиала Института проб\-лем информатики
Российской академии наук
%302025, г.Орел, Московское шоссе, д.137

\vspace*{7pt}

%\columnbreak

\noindent 
\textbf{Соколов  Игорь Анатольевич} (р.\ 1954)~--- академик (действительный член) Российской 
академии наук, доктор технических наук, директор Института проб\-лем информатики
Российской академии наук

\vspace*{7pt}

\noindent
\textbf{Степченков Юрий Афанасьевич} (р.\ 1951)~---
кандидат технических наук, заведующий отделом Института проб\-лем информатики
Российской академии наук

\vspace*{7pt}

\noindent
\textbf{Сурков Алексей Викторович} (р.\ 1978)~--- 
старший научный сотрудник На\-уч\-но-ис\-сле\-до\-ва\-тель\-ско\-го 
института системных исследований Российской академии наук
%117218, Москва, Нахимовский просп., 36, к.1 

\vspace*{7pt}

\noindent 
\textbf{Шестаков Олег Владимирович} (р.\ 1976)~--- доктор 
фи\-зи\-ко-ма\-те\-ма\-ти\-че\-ских, доцент кафедры математической статистики 
факультета вычисли\-тельной математики и кибернетики Московского 
государственного университета им.\ М.\,В.~Ломоносова; 
%119991, Москва ГСП-1, Ленинские горы, д.\ 1, стр. 52
старший научный сотрудник Института проб\-лем информатики
Российской академии наук
%, Москва 119333, ул. Вавилова, д.~44, корп.~2

\vspace*{7pt}

\noindent 
\textbf{Шоргин Сергей Яковлевич} (р.\ 1952.)~--- доктор
фи\-зи\-ко-ма\-те\-ма\-ти\-че\-ских наук, профессор, заместитель директора Института 
проб\-лем информатики Российской академии наук





%%%%%%%%%%%%%%%%%%%%%%%%%%%%%%%%%%%%%%%%%%%%%%%%%%%%%%%%%%%%%%%%%%%%%%%%%%%%%%%




%\def\rightkol{ОБ АВТОРАХ}
%\def\leftkol{ОБ АВТОРАХ}

 \label{end\stat}





%\def\leftfootline{\small{\textbf{\thepage}
%\hfill ИНФОРМАТИКА И ЕЁ ПРИМЕНЕНИЯ\ \ \ том~7\ \ \ выпуск~1\ \ \ 2013}
%}%
% \def\rightfootline{\small{ИНФОРМАТИКА И ЕЁ ПРИМЕНЕНИЯ\ \ \ том~7\ \ \ выпуск~1\ \ \ 2013
%\hfill \textbf{\thepage}}}


%\thispagestyle{myheadings}



\end{multicols}

\newpage

%\end{document}

%
\def\stat{rekl}
%\label{preobr}

%\def\tit{АКАДЕМИК ПУГАЧЁВ  ВЛАДИМИР СЕМЁНОВИЧ\\
%25.03.1911--25.03.1998}


%   \vspace*{-48pt}
%   \begin{center}\LARGE
%Академик Пугачёв  Владимир Семёнович\\ (25.03.1911--25.03.1998)
%   \end{center}

   %\vspace*{2.5mm}

   \begin{center}

{\prgsh\LARGE
ЮБИЛЕИ}

\end{center}
%\hrule

\vspace*{6pt}


   \vspace*{8mm}

   \thispagestyle{empty}


%\def\stat{emel}


\section*{К 70-летию заместителя директора ИПИ РАН,\\ члена редколлегии журнала
<<Информатика и её применения>>\\ доктора технических наук В.\,И.~Будзко}

\vspace*{18pt}




          \begin{multicols}{2}

%            \label{st\stat}

\begin{center}
\vspace*{1pt}
\mbox{%
\epsfxsize=78mm
\epsfbox{bud-1.eps}
}
\end{center}

\vspace*{12pt}

      14 августа 2014~г.\ исполнилось 70~лет за\-мес\-ти\-те\-лю директора ИПИ РАН по
научной работе доктору технических наук Владимиру Игоревичу Будзко.

      Владимир Игоревич Будзко родился в г.~Москве. Высшее образование получил на факультете
элект\-рон\-но-вы\-чис\-ли\-тель\-ных устройств в Московском
ин\-же\-нер\-но-фи\-зи\-че\-ском институте
(МИФИ), который он окончил в 1968~г., после чего был на\-прав\-лен для прохождения
службы в одну из войс\-ко\-вых частей, где прошел путь от инженера до первого заместителя
командира войсковой части.

      С приходом В.\,И.~Будзко в ИПИ РАН (2001~г.)\ в институте
сформировалось новое научное на\-прав\-ле\-ние теоретических исследований~--- <<Постро\-ение
ин\-фор\-ма\-ци\-он\-но-те\-ле\-ком\-му\-ни\-ка\-ци\-он\-ных\linebreak сис\-тем
высокой до\-ступ\-ности>>. В~рамках этого
направления выполнен широкий круг фундаментальных исследований по поиску подходов и
определению принципов построения средств обеспечения доступности, конфиденциальности
и целостности современных крупномасштабных
ин\-фор\-ма\-ци\-он\-но-те\-ле\-ком\-му\-ни\-ка\-ци\-он\-ных
сис\-тем (ИТС). Разработаны основные сис\-тем\-но-тех\-ни\-че\-ские принципы и базовые
архитектурные решения построения перспективных для условий России ИТС с
централизованной обработкой и хранением информации, сочетающих в себе свойства
высокой доступности, отказо- и катастрофоустойчивости, информационной защищенности.
Определены принципы, методы и математические основы рационального построения и
оптимизации средств восстановления функционирования центров обработки данных (ЦОД)
после возникновения отказов и катастроф, передачи и хранения данных, обеспечения
информационной безопасности при достижении минимальной совокупной стоимости
владения такими системами. Результаты нашли практическое воплощение при реализации
проектов в интересах ряда отечественных государственных и негосударственных
организаций, таких как Банк России (БР), Внешторгбанк, ОАО <<ГМК <<Норильский Никель>>,
<<Газпром>>, Минэкономразвития России, Правительство Москвы, а также ряд силовых
ведомств.

      Под руководством В.\,И.~Будзко начиная с 2001~г.\ выполнен комплекс
      на\-уч\-но-ис\-сле\-до\-ва\-тель\-ских и
      опыт\-но-кон\-ст\-рук\-тор\-ских работ (свыше 100~проектов),
направленных на развитие электронной информационной технологии БР.
Разработаны концепции развития ИТС БР сначала до 2008~г., а затем до 2013~г., которые
были приняты в качестве основы проведения технической политики. За реализацию проекта
<<Катастрофоустойчивая тер\-ри\-то\-ри\-аль\-но-рас\-пре\-де\-лен\-ная
      ин\-фор\-ма\-ци\-он\-но-те\-ле\-ком\-му\-ни\-ка\-ци\-он\-ная сис\-те\-ма централизованной
обработки банковской информации>> В.\,И.~Будзко удостоен Премии Правительства РФ в
области науки и техники за 2010~г.

      В.\,И.~Будзко возглавлял и возглавляет работы по ряду других прикладных проектов,
связанных с созданием, совершенствованием и развитием крупномасштабных ИТС.

      В.\,И.~Будзко~--- генерал-майор, доктор технических наук, член-кор\-рес\-пон\-дент
Академии криптографии РФ, известный ученый в области информатики и применения
информационных технологий при построении территориально распределенных ИТС
различного назначения. Является автором свыше 250~научных работ, опубликованных в
на\-уч\-но-тех\-ни\-че\-ских и специальных изданиях.

    \thispagestyle{empty}

      В.\,И.~Будзко уделяет большое внимание подготовке научных кадров. Под его
руководством защищено 6~диссертаций на соискание ученой степени кандидата
технических наук. Свыше 30~лет он читает лекции в ИКСИ Академии ФСБ, профессор
кафедры НИЯУ МИФИ. Является членом двух диссертационных советов, главным
редактором журнала <<Системы высокой доступности>> и членом редколлегии журнала
<<Информатика и её применения>>.

      \bigskip

      Редакционный совет и Редакционная коллегия журнала <<Информатика и её
применения>> сердечно поздравляют Владимира Игоревича Будзко с 70-ле\-ти\-ем и желают
крепкого здоровья и новых научных достижений.

\end{multicols}

\def\stat{cont}
{%\hrule\par
%\vskip 7pt % 7pt
\raggedleft\Large \bf%\baselineskip=3.2ex
А\,В\,Т\,О\,Р\,С\,К\,И\,Й\ \ У\,К\,А\,З\,А\,Т\,Е\,Л\,Ь\ \ З\,А\ \ 2\,0\,1\,0 г. \vskip 17pt
    \hrule
    \par
\vskip 21pt plus 6pt minus 3pt }

\label{st\stat}

\def\tit{\ }

\def\aut{\ }
\def\auf{\ }

\def\leftkol{\ } % ENGLISH ABSTRACTS}

\def\rightkol{\ } %АВТОРСКИЙ УКАЗАТЕЛЬ ЗА 2010 г.} %ENGLISH ABSTRACTS}

\titele{\tit}{\aut}{\auf}{\leftkol}{\rightkol}

\vspace*{-12pt}

{\tabcolsep=3pt
\begin{tabular}{p{388pt}rr}
&\textbf{Выпуск} & \textbf{Стр.}\\[6pt]
\hangindent=23pt\noindent\textbf{Арутюнян~А.\,Р.} Моделирование влияния деформаций отпечатков пальцев на 
точность\linebreak
\vspace*{-12pt}\\
\hspace*{23pt}дактилоскопической идентификации$\dotfill$&1&51\\
\hangindent=23pt\noindent\textbf{Архипов~О.\,П., Зыкова~З.\,П.} Интеграция гетерогенной информации о цветных 
пикселях\linebreak
\vspace*{-12pt}\\
\hspace*{23pt}и их цветовосприятии$\dotfill$&4&15\\
\hangindent=23pt\noindent\textbf{Баранов~С.\,И., Френкель~С.\,Л., Захаров~В.\,Н.} Полуформальная верификация 
цифрового устройства с конвейером, основанная на использовании алгоритмических машин\linebreak
\vspace*{-12pt}\\
\hspace*{23pt}состояния$\dotfill$&4&49\\
\textbf{Бекетова~И.\,В.} см.~Каратеев~С.\,Л.&&\\
\textbf{Белоусов~В.\,В.} см.~Синицын~И.\,Н.&&\\
\hangindent=23pt\noindent\textbf{Бенинг~В.\,Е., Королев~Р.\,А.} О предельном поведении мощностей критериев в 
случае\linebreak
\vspace*{-12pt}\\
\hspace*{23pt}распределения Лапласа$\dotfill$&2&63\\
\hangindent=23pt\noindent\textbf{Бенинг~В.\,Е., Сипина~А.\,В.} Асимптотическое разложение для мощности 
критерия,\linebreak
\vspace*{-12pt}\\
\hspace*{23pt}основанного на выборочной медиане, в случае распределения Лапласа$\dotfill$&1&18\\
\textbf{Бондаренко~А.\,В.} см.~Каратеев~С.\,Л.&&\\
\hangindent=23pt\noindent\textbf{Бородина~А.\,В., Морозов~Е.\,В.} Об оценивании асимптотики вероятности 
большого\linebreak
\vspace*{-12pt}\\
\hspace*{23pt}уклонения стационарной регенеративной очереди с одним прибором$\dotfill$&3&29\\
\hangindent=23pt\noindent\textbf{Бунтман~Н.\,В., Минель~Ж.-Л., Ле~Пезан~Д., Зацман~И.\,М.} Типология и 
компьютерное\linebreak
\vspace*{-12pt}\\
\hspace*{23pt}моделирование трудностей перевода$\dotfill$&3&77\\
\textbf{Визильтер~Ю.\,В.} см.~Каратеев~С.\,Л.&&\\
\hangindent=23pt\noindent\textbf{Гавриленко~С.\,В.} Оценки скорости сходимости распределений случайных сумм с 
безгранично делимыми индексами к нормальному закону$\dotfill$&4&81\\
\hangindent=23pt\noindent\textbf{Григорьева~М.\,Е., Шевцова~И.\,Г.} Уточнение неравенства 
Каца--Берри--Эссеена$\dotfill$&2&75\\
\hangindent=23pt\noindent\textbf{Грушо~А.\,А., Грушо~Н.\,А., Тимонина~Е.\,Е.} Поиск конфликтов в политиках 
безопасности: модель случайных графов$\dotfill$&3&38\\
\textbf{Грушо~Н.\,А.} см.~Грушо~А.\,А.&&\\
\hangindent=23pt\noindent\textbf{Гудков~В.\,Ю.} Математические модели изображения отпечатка пальца на основе 
описания линий$\dotfill$&1&58\\
\textbf{Гуртов~А.\,В.} см.~Лукьяненко~А.\,С.&&\\
\textbf{Желтов~С.\,Ю.} см.~Каратеев~С.\,Л.&&\\
\hangindent=23pt\noindent\textbf{Захаров~А.\,А., Серебряков~В.\,А.} Система управления электронной библиотекой 
LibMeta$\dotfill$&4&2\\
\textbf{Захаров~В.\,Н.} см.~Баранов~С.\,И.&&\\
\textbf{Захарова~Т.\,В.} см.~Матвеева~С.\,С.&&\\
\hangindent=23pt\noindent\textbf{Зацаринный~А.\,А., Чупраков~К.\,Г.} Некоторые аспекты выбора технологии для 
постро-\linebreak
\vspace*{-12pt}\\
\hspace*{23pt}ения систем отображения информации ситуационного центра$\dotfill$&3&59\\
\textbf{Зацман~И.\,М.} см.~Бунтман~Н.\,В.&&\\
\hangindent=23pt\noindent\textbf{Зейфман~А.\,И., Коротышева~А.\,В., Сатин~Я.\,А., Шоргин~С.\,Я.} Об 
устойчивости нестаци-\linebreak
\vspace*{-12pt}\\
\hspace*{23pt}онарных систем обслуживания с катастрофами$\dotfill$&3&9\\
\textbf{Зыкова~З.\,П.} см.~Архипов~О.\,П.&&\\
\hangindent=23pt\noindent\textbf{Илюшин~Г.\,Я., Соколов~И.\,А.} Организация управляемого доступа пользователей 
к\linebreak
\vspace*{-12pt}\\
\hspace*{23pt}разнородным ведомственным информационным ресурсам$\dotfill$&1&24\\
\hangindent=23pt\noindent\textbf{Кавагучи~Ю., Ульянов~В.\,В., Фуджикоши~Я.} Приближения для статистик, 
описывающих\linebreak
\vspace*{-12pt}\\
\hspace*{23pt}геометрические свойства данных большой размерности, с оценками 
ошибок$\dotfill$&1&12\\
\hangindent=23pt\noindent\textbf{Каратеев~С.\,Л., Бекетова~И.\,В., Ососков~М.\,В., Князь~В.\,А., 
Визильтер~Ю.\,В., Бондаренко~А.\,В., Желтов~С.\,Ю.} Автоматизированный контроль 
качества цифровых\linebreak
\vspace*{-12pt}\\
\hspace*{23pt}изображений для персональных документов$\dotfill$&1&65\\
\end{tabular}
}

\pagebreak

\def\leftkol{АВТОРСКИЙ УКАЗАТЕЛЬ ЗА 2010 г.} % ENGLISH ABSTRACTS}

\def\rightkol{АВТОРСКИЙ УКАЗАТЕЛЬ ЗА 2010 г.} %ENGLISH ABSTRACTS}

{\tabcolsep=3pt
\begin{tabular}{p{388pt}rr}
&\textbf{Выпуск} & \textbf{Стр.}\\[3pt]
\hangindent=23pt\noindent\textbf{Козеренко~Е.\,Б.} Лингвистические фильтры в статистических моделях машинного\linebreak
\vspace*{-12pt}\\
\hspace*{23pt}перевода$\dotfill$&2&83\\
\hangindent=23pt\noindent\textbf{Козеренко~Е.\,Б., Кузнецов~И.\,П.} Когнитивно-лингвистические представления в 
систе-\linebreak
\vspace*{-12pt}\\
\hspace*{23pt}мах обработки текстов$\dotfill$&3&69\\
\textbf{Князь~В.\,А.} см.~Каратеев~С.\,Л.&&\\
\hangindent=23pt\noindent\textbf{Колесников~А.\,В., Солдатов~С.\,А.} Алгоритм координации для гибридной 
интеллектуальной системы решения сложной задачи оперативно-производственного\linebreak
\vspace*{-12pt}\\
\hspace*{23pt}планирования$\dotfill$&4&61\\
\hangindent=23pt\noindent\textbf{Коновалов~М.\,Г.} О планировании потоков в системах вычислительных 
ресурсов$\dotfill$&2&3\\
\textbf{Конушин~А.\,С.} см.~Конушин~В.\,С.&&\\
\hangindent=23pt\noindent\textbf{Конушин~В.\,С., Кривовязь~Г.\,Р., Конушин~А.\,С.} Алгоритм распознавания людей 
в видео-\linebreak
\vspace*{-12pt}\\
\hspace*{23pt}последовательности по одежде$\dotfill$&1&74\\
\textbf{Корепанов~Э.\, Р.} см.~Синицын~И.\,Н.&&\\
\textbf{Королев~В.\,Ю.} см.~Соколов~И.\,А.&&\\
\textbf{Королев~Р.\,А.} см.~Бенинг~В.\,Е.&&\\
\textbf{Коротышева~А.\,В.} см.~Зейфман~А.\,И.&&\\
\hangindent=23pt\noindent\textbf{Кривенко~М.\,П.} Непараметрическое оценивание элементов байесовского 
клас\-си-\linebreak
\vspace*{-12pt}\\
\hspace*{23pt}фикатора$\dotfill$&2&13\\
\textbf{Кривовязь~Г.\,Р.} см.~Конушин~В.\,С.&&\\
\textbf{Крылов~А.\,С.} см.~Павельева~Е.\,А.&&\\
\hangindent=23pt\noindent\textbf{Крылов~В.\,А.} Моделирование и классификация многоканальных дистанционных\linebreak
\vspace*{-12pt}\\
\hspace*{23pt}изображений с использованием копул$\dotfill$&4&34\\
\hangindent=23pt\noindent\textbf{Крючин~О.\,В.} Разработка параллельных эвристических алгоритмов подбора 
весовых\linebreak
\vspace*{-12pt}\\
\hspace*{23pt}коэффициентов искусственной нейтронной сети$\dotfill$&2&53\\
\hangindent=23pt\noindent\textbf{Кудрявцев~А.\,А., Шоргин~С.\,Я.} Байесовские модели массового обслуживания и 
надеж-\linebreak
\vspace*{-12pt}\\
\hspace*{23pt}ности: характеристики среднего числа заявок в системе $M\vert M \vert 1\vert 
\infty$$\dotfill$&3&16\\
\hangindent=23pt\noindent\textbf{Кузнецов~А.\,А.} Связь между временными и структурно-топологическими 
характери-\linebreak
\vspace*{-12pt}\\
\hspace*{23pt}стиками диаграмм ритма сердца здоровых людей$\dotfill$&4&39\\
\textbf{Кузнецов~И.\,П.} см.~Козеренко~Е.\,Б.&&\\
\textbf{Ле~Пезан~Д.} см.~Бунтман~Н.\,В.&&\\
\hangindent=23pt\noindent\textbf{Лукьяненко~А.\,С., Морозов~Е.\,В., Гуртов~А.\,В.} Анализ сетевого протокола с общей 
функ-\linebreak
\vspace*{-12pt}\\
\hspace*{23pt}цией расширения окна передачи сообщения при конфликтах$\dotfill$&2&46\\
\hangindent=23pt\noindent\textbf{Лямин~О.\,О.} О предельном поведении мощностей критериев в случае обобщенного\linebreak
\vspace*{-12pt}\\
\hspace*{23pt}распределения Лапласа$\dotfill$&3&47\\
\hangindent=23pt\noindent\textbf{Маркин~А.\,В., Шестаков~О.\,В.} Асимптотики оценки риска при пороговой 
обработке\linebreak
\vspace*{-12pt}\\
\hspace*{23pt}вейвлет-вейглет коэффициентов в задаче томографии$\dotfill$&2&36\\
\hangindent=23pt\noindent\textbf{Матвеева~С.\,С., Захарова~Т.\,В.} Сети массового обслуживания с наименьшей 
длиной\linebreak
\vspace*{-12pt}\\
\hspace*{23pt}очереди$\dotfill$&3&22\\
\hangindent=23pt\noindent\textbf{Матюшенко~С.\,И.} Стационарные характеристики двухканальной системы 
обслужива-\linebreak
\vspace*{-12pt}\\
\hspace*{23pt}ния с переупорядочиванием заявок и распределениями фазового типа$\dotfill$&4&68\\
\textbf{Минель~Ж.-Л.} см.~Бунтман~Н.\,В.&&\\
\textbf{Морозов~Е.\,В.} см.~Бородина~А.\,В.&&\\
\textbf{Морозов~Е.\,В.} см.~Лукьяненко~А.\,С.&&\\
\textbf{Ососков~М.\,В.} см.~Каратеев~С.\,Л.&&\\
\hangindent=23pt\noindent\textbf{Павельева~Е.\,А., Крылов~А.\,С.} Поиск и анализ ключевых точек радужной 
оболочки\linebreak
\vspace*{-12pt}\\
\hspace*{23pt}глаза методом преобразования Эрмита$\dotfill$&1&79\\
\textbf{Печинкин~А.\,В.} см.~Френкель~С.\,Л.,&&\\
\hangindent=23pt\noindent\textbf{Протасов~В.\,И.} Составление субъективного портрета с использованием 
эволюционно-\linebreak
\vspace*{-12pt}\\
\hspace*{23pt}го морфинга и квалиметрия метода$\dotfill$&1&83\\
\hangindent=23pt\noindent\textbf{Рудаков~К.\,В., Торшин~И.\,Ю.} Вопросы разрешимости задачи распознавания 
вторичной\linebreak
\vspace*{-12pt}\\
\hspace*{23pt}структуры белка$\dotfill$&2&25\\
\textbf{Сатин~Я.\,А.} см.~Зейфман~А.\,И.&&\\
\hangindent=23pt\noindent\textbf{Сейфуль-Мулюков~Р.\,Б.} Нефть как носитель информации о своем 
происхождении,\linebreak
\vspace*{-12pt}\\
\hspace*{23pt}структуре и эволюции$\dotfill$&1&41\\
\end{tabular}
}

{\tabcolsep=3pt
\begin{tabular}{p{388pt}rr}
&\textbf{Выпуск} & \textbf{Стр.}\\[6pt]
\textbf{Семендяев~Н.\,Н.} см.~Синицын~И.\,Н.&&\\
\textbf{Серебряков~В.\,А.} см.~Захаров~А.\,А.&&\\
\textbf{Синицын~В.\,И.} см.~Синицын~И.\,Н.&&\\
\hangindent=23pt\noindent\textbf{Синицын~И.\,Н., Синицын~В.\,И., Корепанов~Э.\, Р., Белоусов~В.\,В., 
Семендяев~Н.\,Н.} Оперативное построение информационных моделей движения полюса 
Земли\linebreak
\vspace*{-12pt}\\
\hspace*{23pt}методами линейных и линеаризованных фильтров$\dotfill$&1&2\\
\textbf{Сипина~А.\,В.} см.~Бенинг~В.\,Е.&&\\
\hangindent=23pt\noindent\textbf{Соколов~И.\,А.} О работах заслуженного деятеля науки Российской Федерации 
И.\,Н.~Синицына в области информационных технологий и автоматизации (к 70-летию\linebreak
\vspace*{-12pt}\\
\hspace*{23pt}со дня рождения)$\dotfill$&3&84\\
\textbf{Соколов~И.\,А.} см.~Илюшин~Г.\,Я.&&\\
\hangindent=23pt\noindent\textbf{Соколов~И.\,А., Королев~В.\,Ю.} Предисловие$\dotfill$&2&2\\
\textbf{Солдатов~С.\,А.} см.~Колесников~А.\,В.&&\\
\hangindent=23pt\noindent\textbf{Степанов~С.\,Ю.} Использование координатного метода фрагментации 
коммутаторной\linebreak
\vspace*{-12pt}\\
\hspace*{23pt}нейронной сети для сокращения трафика$\dotfill$&2&57\\
\textbf{Тимонина~Е.\,Е.} см.~Грушо~А.\,А.&&\\
\textbf{Торшин~И.\,Ю.} см.~Рудаков~К.\,В.&&\\
\textbf{Ульянов~В.\,В.} см.~Кавагучи~Ю.&&\\
\textbf{Фазекаш~И.} см.~Чупрунов~А.\,Н.&&\\
\textbf{Френкель~С.\,Л.} см.~Баранов~С.\,И.&&\\
\hangindent=23pt\noindent\textbf{Френкель~С.\,Л., Печинкин~А.\,В.} Оценка времени самовосстановления в 
цифровых\linebreak
\vspace*{-12pt}\\
\hspace*{23pt}системах после сбоев, вызываемых переходными помехами$\dotfill$&3&2\\
\textbf{Фуджикоши~Я.} см.~Кавагучи~Ю.&&\\
\hangindent=23pt\noindent\textbf{Цискаридзе~А.\,К.} Математическая модель и метод восстановления позы человека 
по\linebreak
\vspace*{-12pt}\\
\hspace*{23pt}стереопаре силуэтных изображений$\dotfill$&4&27\\
\hangindent=23pt\noindent\textbf{Чупраков~К.\,Г.} К вопросу о размещении коллективных средств отображения в 
ситуа-\linebreak
\vspace*{-12pt}\\
\hspace*{23pt}ционном зале с заданными параметрами$\dotfill$&4&89\\
\textbf{Чупраков~К.\,Г.} см.~Зацаринный~А.\,А.&&\\
\hangindent=23pt\noindent\textbf{Чупрунов~А.\,Н., Фазекаш~И.} Законы повторного логарифма для числа 
безошибочных\linebreak
\vspace*{-12pt}\\
\hspace*{23pt}блоков при помехоустойчивом кодировании$\dotfill$&3&42\\
\textbf{Шевцова~И.\,Г.} см.~Григорьева~М.\,Е.&&\\
\hangindent=23pt\noindent\textbf{Шестаков~О.\,В.} Аппроксимация распределения оценки риска пороговой 
обработки вейвлет-коэффициентов нормальным распределением при использовании 
выбо-\linebreak
\vspace*{-12pt}\\
\hspace*{23pt}рочной дисперсии$\dotfill$&4&73\\
\textbf{Шестаков~О.\,В.} см.~Маркин~А.\,В.&&\\
\textbf{Шоргин~С.\,Я.} см.~Зейфман~А.\,И.&&\\
\textbf{Шоргин~С.\,Я.} см.~Кудрявцев~А.\,А.&&\\
\end{tabular}
}

%\thispagestyle{myheadings}
\def\leftfootline{\small{\textbf{\thepage}
\hfill ИНФОРМАТИКА И ЕЁ ПРИМЕНЕНИЯ\ \ \ том~4\ \ \ выпуск~4\ \ \ 2010}
}%
 \def\rightfootline{\small{ИНФОРМАТИКА И ЕЁ ПРИМЕНЕНИЯ\ \ \ том~4\ \ \ выпуск~4\ \ \ 2010
 \hfill \textbf{\thepage}}}
 \label{end\stat}


%Том 10 Выпуск 1-4 Год 2016

\def\stat{cont-e}
{%\hrule\par
%\vskip 7pt % 7pt
\raggedleft\Large \bf%\baselineskip=3.2ex
2\,0\,1\,6\ \ A\,U\,T\,H\,O\,R\ \ I\,N\,D\,E\,X \vskip 17pt
 \hrule
 \par
\vskip 21pt plus 6pt minus 3pt }

\label{st\stat}

\def\tit{\ }

\def\aut{\ }
\def\auf{\ }

\def\leftkol{\ } %2016 AUTHOR INDEX} % ENGLISH ABSTRACTS}

\def\rightkol{\ } %2016 AUTHOR INDEX} %ENGLISH ABSTRACTS}

\titele{\tit}{\aut}{\auf}{\leftkol}{\rightkol}

\def\leftfootline{\small{\textbf{\thepage}
\hfill INFORMATIKA I EE PRIMENENIYA~--- INFORMATICS AND APPLICATIONS\ \ \ 2016\
\ \ volume~10\ \ \ issue\ 4}
}%
 \def\rightfootline{\small{INFORMATIKA I EE PRIMENENIYA~--- INFORMATICS AND APPLICATIONS\ \ \ 2016\ \ \ volume~10\ \ \ issue\ 4
\hfill \textbf{\thepage}}}

\vspace*{-12pt}
\vspace*{-18pt}

{\tabcolsep=2.8pt
\begin{tabular}{p{382pt}cc}
&\textbf{Issue} & \textbf{Page}\\[6pt]
\Avtors{Agalarov~M.\,Ya.} see~Agalarov~Ya.\,M.&&\\
\Avtors{Agalarov~Ya.\,M., Agalarov~M.\,Ya., and
Shorgin~V.\,S.} About the optimal threshold of queue\linebreak
\\[-12pt]
\hspace*{23pt}length in a~particular problem of profit maximization
in the $M/G/1$ queuing system&2&70--79\\
\Avtors{Alexeyevsky~D.\,A.} BioNLP ontology extraction from 
a~restricted language corpus with\linebreak
\\[-12pt]
\hspace*{23pt}context-free grammars&1&119--128\\
\Avtors{Andreev~S.\,D.} see~Gaidamaka~Yu.\,V.&&\\
\Avtors{Andreev~S.\,D.} see~Ometov~A.\,Ya.&&\\
\Avtors{Arkhipov~O.\,P., Arkhipov~P.\,O., and Sidorkin~I.\,I.} The
option to create a~local coordinate\linebreak
\\[-12pt]
\hspace*{23pt}system for synchronization of selected images&3&91--97\\
\Avtors{Arkhipov~P.\,O.} see~Arkhipov~O.\,P.&&\\
\Avtors{Belousov~V.\,V.} see~Shnurkov~P.\,V.&&\\
\Avtors{Belousov~V.\,V.} see~Shnurkov~P.\,V.&&\\
\Avtors{Bening~V.\,E.} Calculation of~the~asymptotic deficiency
of~some statistical procedures based\linebreak
\\[-12pt]
\hspace*{23pt}on~samples with~random sizes&4&34--45\\
\Avtors{Borisov~A.\,V., Bosov~A.\,V., and Miller~G.\,B.} Modeling and
monitoring of VoIP connection&2&\hphantom{1}2--13\\
\Avtors{Bosov~A.\,V.} see~Borisov~A.\,V.&&\\
\Avtors{Briukhov~D.\,O.} see~Stupnikov~S.\,A.&&\\
\Avtors{Callaos~N.\,K.\ and Seyful-Mulyukov~R.\,B.} Complexity and
its information content&1&129--139\\
\Avtors{Chertok~A.\,V., Kadaner~A.\,I., Khazeeva~G.\,T., and
Sokolov~I.\,A.} Regime switching detection\linebreak
\\[-12pt]
\hspace*{23pt}for~the~Levy driven
Ornstein--Uhlenbeck process using CUSUM methods&4&46--56\\
\Avtors{Chichagov~V.\,V.} Asymptotic expansions of mean absolute
error of uniformly minimum variance unbiased and maximum likelihood
estimators on the one-parameter exponential\linebreak
\\[-12pt]
\hspace*{23pt}family model of lattice distributions&3&66--76\\
\Avtors{Danishevsky~V.\,I.} see~Kolesnikov A.\,V.&&\\
\Avtors{Fazliev~A.\,Z.} see~Kalinichenko~L.\,A.&&\\
\Avtors{Fedoseev~A.\,A.} What is behind the concept of ``knowledge in
small packages''&3&105--110\\
\Avtors{Gaidamaka~Yu.\,V., Andreev~S.\,D., Sopin~E.\,S.,
Samouylov~K.\,E., and Shorgin~S.\,Ya.} Interference analysis
of~the~device-to-device communications model with~regard to~a~signal\linebreak
\\[-12pt]
\hspace*{23pt}propagation environment&4&\hphantom{1}2--10\\
\Avtors{Gasilov~A.\,V.} see~Yakovlev~O.\,A.&&\\
\Avtors{Goncharov~A.\,V.\ and Strijov~V.\,V.} Metric time series
classification using weighted dynamic\linebreak
\\[-12pt]
\hspace*{23pt}warping relative to centroids of classes&2&36--47\\
\Avtors{Gordov~E.\,P.} see~Kalinichenko~L.\,A.&&\\
\Avtors{Gorshenin~A.\,K.} Concept of online service for stochastic
modeling of real processes&1&72--81\\
\Avtors{Gorshenin~A.\,K.} see~Shnurkov~P.\,V.&&\\
\Avtors{Gorshenin~A.\,K.} see~Shnurkov~P.\,V.&&\\
\Avtors{Grusho~A.\,A., Grusho~N.\,A., Zabezhailo~M.\,I., and
Timonina~E.\,E.} Integration of statistical and\linebreak
\\[-12pt]
\hspace*{23pt}deterministic methods for
analysis of information security&3&2--8\\
\Avtors{Grusho~A.\,A., Zabezhailo~M.\,I., and Zatsarinny~A.\,A.} On
the advanced procedure to reduce\linebreak
\\[-12pt]
\hspace*{23pt}calculation of Galois closures&4&\hphantom{1}96--104\\
\Avtors{Grusho~N.\,A.} see~Grusho~A.\,A.&&\\
\Avtors{Havanskov~V.\,A.} see~Minin~V.\,A.&&\\
\Avtors{Inkova~O.\,Yu.} see~Zatsman~I.\,M.&&\\
\Avtors{Isachenko~R.\,V.\ and Strijov~V.\,V.} Metric learning in
multiclass time series classification\linebreak
\\[-12pt]
\hspace*{23pt}problem&2&48--57\\
\end{tabular}
}
\pagebreak

\def\leftfootline{\small{\textbf{\thepage}
\hfill INFORMATIKA I EE PRIMENENIYA~--- INFORMATICS AND APPLICATIONS\ \ \ 2016\
\ \ volume~10\ \ \ issue\ 4}
}%
 \def\rightfootline{\small{INFORMATIKA I EE PRIMENENIYA~---
INFORMATICS AND APPLICATIONS\ \ \ 2016\ \ \ volume~10\ \ \ issue\ 4
\hfill \textbf{\thepage}}}

\def\leftkol{2016 AUTHOR INDEX} % ENGLISH ABSTRACTS}

\def\rightkol{2016 AUTHOR INDEX} %ENGLISH ABSTRACTS}


{\tabcolsep=2.83pt
\begin{tabular}{p{382pt}cc}
&\textbf{Issue} & \textbf{Page}\\[6pt]
\Avtors{Kadaner~A.\,I.} see~Chertok~A.\,V.&&\\[.255pt]
\Avtors{Kalinichenko~L.\,A., Volnova~A.\,A., Gordov~E.\,P.,
Kiselyova~N.\,N., Kovaleva~D.\,A., Malkov~O.\,Yu., Okladnikov~I.\,G.,
Podkolodnyy~N.\,L., Pozanenko~A.\,S., Ponomareva~N.\,V.,
Stupnikov~S.\,A.,} \textbf{and Fazliev~A.\,Z.} Data access challenges for data
intensive\linebreak
\\[-12pt]
\hspace*{23pt}research in Russia&1& 2--22\\[.255pt]
\Avtors{Karasikov~M.\,E.\ and Strijov~V.\,V.} Feature-based
time-series classification&4&121--131\\[.255pt]
\Avtors{Khazeeva~G.\,T.} see~Chertok~A.\,V.&&\\[.255pt]
\Avtors{Khokhlov~Yu.\,S.} Multivariate fractional Levy motion and its
applications&2&\hphantom{1}98--106\\[.255pt]
\Avtors{Kirikov~I.\,A., Kolesnikov~A.\,V., Listopad~S.\,V., and
Rumovskaya~S.\,B.} Fine-grained hybrid\linebreak
\\[-12pt]
\hspace*{23pt}intelligent systems. Part 2:
Bidirectional hybridization&1&\hphantom{1}96--105\\[.255pt]
\Avtors{Kirikov~I.\,A., Kolesnikov~A.\,V., Listopad~S.\,V., and
Rumovskaya~S.\,B.} ``Virtual council''~---\linebreak
\\[-12pt]
\hspace*{23pt}source environment
supporting complex diagnostic decision making&3&81--90\\[.255pt]
\Avtors{Kiselyova~N.\,N.} see~Kalinichenko~L.\,A.&&\\[.255pt]
\Avtors{Kolesnikov A.\,V., Listopad~S.\,V., Rumovskaya~S.\,B., and
Danishevsky~V.\,I.} Informal axiomatic\linebreak
\\[-12pt]
\hspace*{23pt}theory of~the~role visual models&4&114--120\\[.255pt]
\Avtors{Kolesnikov~A.\,V.} see~Kirikov~I.\,A.&&\\[.255pt]
\Avtors{Kolesnikov~A.\,V.} see~Kirikov~I.\,A.&&\\[.255pt]
\Avtors{Kolin~K.\,K.} Humanitarian aspects of information
security&3&111--121\\[.255pt]
\Avtors{Konovalov~M.\,G.\ and Razumchik~R.\,V.} Dispatching
to~two parallel nonobservable queues using\linebreak
\\[-12pt]
\hspace*{23pt}only static
information&4&57--67\\[.255pt]
\Avtors{Korchagin~A.\,Yu.} see~Korolev~V.\,Yu.&&\\[.255pt]
\Avtors{Korchagin~A.\,Yu.} see~Korolev~V.\,Yu.&&\\[.255pt]
\Avtors{Korepanov~E.\,R.} see~Sinitsyn~I.\,N.&&\\[.255pt]
\Avtors{Korepanov~E.\,R.} see~Sinitsyn~I.\,N.&&\\[.255pt]
\Avtors{Korolev~V.\,Yu., Korchagin~A.\,Yu., and Zeifman~A.\,I.} The
Poisson theorem for Bernoulli trials\linebreak
\\[-12pt]
\hspace*{23pt}with~a~random probability
of~success and~a~discrete analog of~the~Weibull distribution&4&11--20\\[.255pt]
\Avtors{Korolev~V.\,Yu., Zeifman~A.\,I., and Korchagin~A.\,Yu.}
Asymmetric Linnik distributions as~limit\linebreak
\\[-12pt]
\hspace*{23pt}laws for~random sums
of~independent random variables with~finite variances&4&21--33\\[.255pt]
\Avtors{Koucheryavy~E.\,A.} see~Ometov~A.\,Ya.&&\\[.255pt]
\Avtors{Kovaleva~D.\,A.} see~Kalinichenko~L.\,A.&&\\[.255pt]
\Avtors{Kovalyov~S.\,P.} Metaprogramming to increase
manufacturability of large-scale software-\linebreak
\\[-12pt]
\hspace*{23pt}intensive systems&1&56--66\\[.255pt]
\Avtors{Krivenko~M.\,P.} Significance tests of feature selection for
classification&3&32--40\\[.255pt]
\Avtors{Kruzhkov~M.\,G.} see~Zalizniak~Anna~A.&&\\[.255pt]
\Avtors{Kruzhkov~M.\,G.} see~Zatsman~I.\,M.&&\\[.255pt]
\Avtors{Kudryavtsev~A.\,A.} Bayesian queueing and reliability models:
\textit{A~priori} distributions with\linebreak
\\[-12pt]
\hspace*{23pt}compact support&1&67--71\\[.255pt]
\Avtors{Kudryavtsev~A.\,A.} Characteristics dependent on the balance
coefficient in Bayesian models\linebreak
\\[-12pt]
\hspace*{23pt}with compact support of \textit{a priori}
distributions&3&77--80\\[.255pt]
\Avtors{Kudryavtsev~A.\,A.\ and Palionnaia~S.\,I.} Bayesian recurrent
model of reliability growth:\linebreak
\\[-12pt]
\hspace*{23pt}Parabolic distribution of parameters&2&80--83\\[.255pt]
\Avtors{Kudryavtsev~A.\,A.\ and Titova~A.\,I.} Bayesian queuing
and~reliability models: Degenerate-\linebreak
\\[-12pt]
\hspace*{23pt}Weibull case&4&68--71\\[.255pt]
\Avtors{Leontyev~N.\,D.\ and Ushakov~V.\,G.} Analysis of a queueing
system with autoregressive arrivals\linebreak
\\[-12pt]
\hspace*{23pt}and nonpreemptive priority&3&15--22\\[.255pt]
\Avtors{Listopad~S.\,V.} see~Kirikov~I.\,A.&&\\[.255pt]
\Avtors{Listopad~S.\,V.} see~Kirikov~I.\,A.&&\\[.255pt]
\Avtors{Listopad~S.\,V.} see~Kolesnikov A.\,V.&&\\[.255pt]
\Avtors{Malkov~O.\,Yu.} see~Kalinichenko~L.\,A.&&\\[.255pt]
\Avtors{Markov~A.\,S., Monakhov~M.\,M., and
Ulyanov~V.\,V.} Generalized Cornish--Fisher expansions\linebreak
\\[-12pt]
\hspace*{23pt}for distributions of statistics based on samples
of random size&2&84--91\\[.255pt]
\Avtors{Melnikov~A.\,K.\ and Ronzhin~A.\,F.} Generalized statistical
method of~text analysis based\linebreak
\\[-12pt]
\hspace*{23pt}on~calculation of~probability distributions
of~statistical values&4&89--95\\
\end{tabular}
}
\pagebreak

\def\leftfootline{\small{\textbf{\thepage}
\hfill INFORMATIKA I EE PRIMENENIYA~--- INFORMATICS AND APPLICATIONS\ \ \ 2016\
\ \ volume~10\ \ \ issue\ 4}
}%
 \def\rightfootline{\small{INFORMATIKA I EE PRIMENENIYA~---
INFORMATICS AND APPLICATIONS\ \ \ 2016\ \ \ volume~10\ \ \ issue\ 4
\hfill \textbf{\thepage}}}

\def\leftkol{2016 AUTHOR INDEX} % ENGLISH ABSTRACTS}

\def\rightkol{2016 AUTHOR INDEX} %ENGLISH ABSTRACTS}


{\tabcolsep=3pt
\begin{tabular}{p{381pt}cc}
&\textbf{Issue} & \textbf{Page}\\[6pt]
\Avtors{Meykhanadzhyan~L.\,A.} Stationary characteristics of the finite
capacity queueing system with\linebreak
\\[-12pt]
\hspace*{23pt}inverse service order and generalized
probabilistic priority&2&123--131\\[.23pt]
\Avtors{Miller~G.\,B.} see~Borisov~A.\,V.&&\\[.23pt]
\Avtors{Minin~V.\,A., Zatsman~I.\,M., Havanskov~V.\,A., and
Shubnikov~S.\,K.} Intensity of citation of scientific publications in
inventions on information and computer technologies patented\linebreak
\\[-12pt]
\hspace*{23pt}in Russia by domestic and foreign applicants&2&107--122\\[.23pt]
\Avtors{Monakhov~M.\,M.} see~Markov~A.\,S.&&\\[.23pt]
\Avtors{Naumov~V.\,A.\ and Samouylov~K.\,E.} On relationship
between queuing systems with resources\linebreak
\\[-12pt]
\hspace*{23pt}and Erlang networks&3&\hphantom{1}9--14\\[.23pt]
\Avtors{Okladnikov~I.\,G.} see~Kalinichenko~L.\,A.&&\\[.23pt]
\Avtors{Ometov~A.\,Ya., Andreev~S.\,D., Turlikov~A.\,M., and
Koucheryavy~E.\,A.} Performance analysis of\linebreak
\\[-12pt]
\hspace*{23pt}a wireless data
aggregation system with contention for contemporary sensor
networks&3&23--31\\[.23pt]
\Avtors{Palionnaia~S.\,I.} see~Kudryavtsev~A.\,A.&&\\[.23pt]
\Avtors{Podkolodnyy~N.\,L.} see~Kalinichenko~L.\,A.&&\\[.23pt]
\Avtors{Ponomareva~N.\,V.} see~Kalinichenko~L.\,A.&&\\[.23pt]
\Avtors{Popkova~N.\,A.} see~Zatsman~I.\,M.&&\\[.23pt]
\Avtors{Pozanenko~A.\,S.} see~Kalinichenko~L.\,A.&&\\[.23pt]
\Avtors{Razumchik~R.\,V.} see~Konovalov~M.\,G.&&\\[.23pt]
\Avtors{Ronzhin~A.\,F.} see~Melnikov~A.\,K.&&\\[.23pt]
\Avtors{Rumovskaya~S.\,B.} see~Kirikov~I.\,A.&&\\[.23pt]
\Avtors{Rumovskaya~S.\,B.} see~Kirikov~I.\,A.&&\\[.23pt]
\Avtors{Rumovskaya~S.\,B.} see~Kolesnikov A.\,V.&&\\[.23pt]
\Avtors{Samouylov~K.\,E.} see~Gaidamaka~Yu.\,V.&&\\[.23pt]
\Avtors{Samouylov~K.\,E.} see~Naumov~V.\,A.&&\\[.23pt]
\Avtors{Serebryanskii~S.\,M.} see~Tyrsin~A.\,N.&&\\[.23pt]
\Avtors{Seyful-Mulyukov~R.\,B.} see~Callaos~N.\,K.&&\\[.23pt]
\Avtors{Shestakov~O.\,V.} Statistical properties of the denoising method
based on the stabilized hard\linebreak
\\[-12pt]
\hspace*{23pt}thresholding&2&65--69\\[.23pt]
\Avtors{Shestakov~O.\,V.} The strong law of large numbers for the risk
estimate in the problem of\linebreak
\\[-12pt]
\hspace*{23pt}tomographic image reconstruction from
projections with a correlated noise&3&41--45\\[.23pt]
\Avtors{Shestakov~O.\,V.} see~Zakharova~T.\,V.&&\\[.23pt]
\Avtors{Shnurkov~P.\,V., Gorshenin~A.\,K., and Belousov~V.\,V.}
Analytical solution of~the~optimal control\linebreak
\\[-12pt]
\hspace*{23pt}task of~a~semi-Markov
process with~finite set of~states&4&72--88\\[.23pt]
\Avtors{Shnurkov~P.\,V., Zasypko~V.\,V., Belousov~V.\,V., and
Gorshenin~A.\,K.} Development of the algorithm of numerical solution
of the optimal investment control problem\linebreak
\\[-12pt]
\hspace*{23pt}in the closed dynamical model of three-sector economy&1&82--95\\[.23pt]
\Avtors{Shorgin~S.\,Ya.} see~Gaidamaka~Yu.\,V.&&\\[.23pt]
\Avtors{Shorgin~V.\,S.} see~Agalarov~Ya.\,M.&&\\[.23pt]
\Avtors{Shubnikov~S.\,K.} see~Minin~V.\,A.&&\\[.23pt]
\Avtors{Sidorkin~I.\,I.} see~Arkhipov~O.\,P.&&\\[.23pt]
\Avtors{Sinitsyn~I.\,N.} Analytical modeling of processes in stochastic
systems with complex fractional\linebreak
\\[-12pt]
\hspace*{23pt}order Bessel nonlinearities&3&55--65\\[.23pt]
\Avtors{Sinitsyn~I.\,N.} Orthogonal supoptimal filters for nonlinear
stochastic systems on manifolds&1&34--44\\[.23pt]
\Avtors{Sinitsyn~I.\,N.\ and Korepanov~E.\,R.} Normal Pugachev
conditionally-optimal filters and extra-\linebreak
\\[-12pt]
\hspace*{23pt}polators for state linear stochastic systems&2&14--23\\[.23pt]
\Avtors{Sinitsyn~I.\,N.\ and Sinitsyn~V.\,I.} Analytical modeling of
distributions in stochastic systems on\linebreak
\\[-12pt]
\hspace*{23pt}manifolds based on ellipsoidal approximation&1&45--55\\[.23pt]
\Avtors{Sinitsyn~I.\,N., Sinitsyn~V.\,I., and
Korepanov~E.\,R.} Ellipsoidal suboptimal filters for nonlinear\linebreak
\\[-12pt]
\hspace*{23pt}stochastic systems on manifolds&2&24--35\\[.23pt]
\Avtors{Sinitsyn~V.\,I.} see~Sinitsyn~I.\,N.&&\\[.23pt]
\Avtors{Sinitsyn~V.\,I.} see~Sinitsyn~I.\,N.&&\\[.23pt]
\Avtors{Skvortsov~N.\,A.} see~Stupnikov~S.\,A.&&\\[.23pt]
\Avtors{Sokolov~I.\,A.} see~Chertok~A.\,V.&&\\
\end{tabular}
}
\pagebreak

\def\leftfootline{\small{\textbf{\thepage}
\hfill INFORMATIKA I EE PRIMENENIYA~--- INFORMATICS AND APPLICATIONS\ \ \ 2016\
\ \ volume~10\ \ \ issue\ 4}
}%
 \def\rightfootline{\small{INFORMATIKA I EE PRIMENENIYA~---
INFORMATICS AND APPLICATIONS\ \ \ 2016\ \ \ volume~10\ \ \ issue\ 4
\hfill \textbf{\thepage}}}

\def\leftkol{2016 AUTHOR INDEX} % ENGLISH ABSTRACTS}

\def\rightkol{2016 AUTHOR INDEX} %ENGLISH ABSTRACTS}


{\tabcolsep=3pt
\begin{tabular}{p{382pt}cc}
&\textbf{Issue} & \textbf{Page}\\[6pt]
\Avtors{Sopin~E.\,S.} see~Gaidamaka~Yu.\,V.&&\\
\Avtors{Strijov~V.\,V.} see~Goncharov~A.\,V.&&\\
\Avtors{Strijov~V.\,V.} see~Isachenko~R.\,V.&&\\
\Avtors{Strijov~V.\,V.} see~Karasikov~M.\,E.&&\\
\Avtors{Stupnikov~S.\,A., Briukhov~D.\,O., and Skvortsov~N.\,A.}
Co-lending systemic risk analysis over\linebreak
\\[-12pt]
\hspace*{23pt}heterogeneous data collections&1&23--33\\
\Avtors{Stupnikov~S.\,A.} see~Kalinichenko~L.\,A.&&\\
\Avtors{Suchkov~A.\,P.} see~Zatsarinny~A.\,A.&&\\
\Avtors{Timonina~E.\,E.} see~Grusho~A.\,A.&&\\
\Avtors{Titova~A.\,I.} see~Kudryavtsev~A.\,A.&&\\
\Avtors{Turlikov~A.\,M.} see~Ometov~A.\,Ya.&&\\
\Avtors{Tyrsin~A.\,N.\ and Serebryanskii~S.\,M.} Recognition of
dependences on the basis of inverse\linebreak
\\[-12pt]
\hspace*{23pt}mapping&2&58--64\\
\Avtors{Ulyanov~V.\,V.} see~Markov~A.\,S.&&\\
\Avtors{Ushakov~V.\,G.} Queueing system with working vacations and
hyperexponential input stream&2&92--97\\
\Avtors{Ushakov~V.\,G.} see~Leontyev~N.\,D.&&\\
\Avtors{Volnova~A.\,A.} see~Kalinichenko~L.\,A.&&\\
\Avtors{Yakovlev~O.\,A.\ and Gasilov~A.\,V.} Speeded-up stereo
matching using geodesic support weights&3&\hphantom{1}98--104\\
\Avtors{Zabezhailo~M.\,I.} see~Grusho~A.\,A.&&\\
\Avtors{Zabezhailo~M.\,I.} see~Grusho~A.\,A.&&\\
\Avtors{Zakharova~T.\,V.\ and Shestakov~O.\,V.} Precision analysis of
wavelet processing of aerodynamic\linebreak
\\[-12pt]
\hspace*{23pt}flow patterns&3&46--54\\
\Avtors{Zalizniak~Anna~A.\ and Kruzhkov~M.\,G.} Database
of~Russian impersonal verbal constructions&4&132--141\\
\Avtors{Zasypko~V.\,V.} see~Shnurkov~P.\,V.&&\\
\Avtors{Zatsarinny~A.\,A.\ and Suchkov~A.\,P.} Systems engineering
approaches to~the~establishment of\linebreak
\\[-12pt]
\hspace*{23pt}a~system for~decision support based
on~situational analysis&4&105--113\\
\Avtors{Zatsarinny~A.\,A.} see~Grusho~A.\,A.&&\\
\Avtors{Zatsman~I.\,M., Inkova~O.\,Yu., Kruzhkov~M.\,G., and
Popkova~N.\,A.} Representation of cross-\linebreak
\\[-12pt]
\hspace*{23pt}lingual knowledge about
connectors in supracorpora databases&1&106--118\\
\Avtors{Zatsman~I.\,M.} see~Minin~V.\,A.&&\\
\Avtors{Zeifman~A.\,I.} see~Korolev~V.\,Yu.&&\\
\Avtors{Zeifman~A.\,I.} see~Korolev~V.\,Yu.&&\\
\end{tabular}
}

%\thispagestyle{myheadings}
\def\leftfootline{\small{\textbf{\thepage}
\hfill INFORMATIKA I EE PRIMENENIYA~--- INFORMATICS AND APPLICATIONS\ \ \ 2016\
\ \ volume~10\ \ \ issue\ 4}
}%
 \def\rightfootline{\small{INFORMATIKA I EE PRIMENENIYA~---
INFORMATICS AND APPLICATIONS\ \ \ 2016\ \ \ volume~10\ \ \ issue\ 4
\hfill \textbf{\thepage}}}

 \label{end\stat}

\newpage

%\def\stat{rekl}
%\label{preobr}

%\def\tit{АКАДЕМИК ПУГАЧЁВ  ВЛАДИМИР СЕМЁНОВИЧ\\
%25.03.1911--25.03.1998}


%   \vspace*{-48pt}
%   \begin{center}\LARGE
%Академик Пугачёв  Владимир Семёнович\\ (25.03.1911--25.03.1998)
%   \end{center}
   
   %\vspace*{2.5mm}
   
   \begin{center}

{\prgsh\LARGE
ОБЪЯВЛЕНИЯ О КОНФЕРЕНЦИЯХ}

\end{center}
%\hrule

\vspace*{6pt}

   
   \vspace*{10mm}
   
   \thispagestyle{empty}

\noindent
\begin{tabular}{cc}
%\begin{center}
\multicolumn{1}{c}{\raisebox{-40pt}[0pt][0pt]{\mbox{%
\epsfxsize=33mm
\epsfbox{vspu.eps}
}}}
%\end{center}
&
\tabcolsep=0pt\begin{tabular}{c}
{\prg{\Large\textbf{XII Всероссийское совещание}}}\\[6pt]
{\prg{\Large\textbf{по проблемам управления}}}\\[12pt]
{\prg{\large 16--19 июня 2014~г.}}\\[6pt] 
{\prg{\large Институт проблем управления имени В.\,А.~Трапезникова РАН}}\\[6pt]
{\prg{\large Москва, Россия}}
\end{tabular}
\end{tabular}

\vspace*{60pt}

     
 { %\large    
 XII Всероссийское совещание по проблемам управления (ВСПУ XII), посвященное 75-летию 
Института проблем управления (ИПУ) имени В.\,А.~Трапезникова РАН, проводится 16--19~июня 
2014~г.\ 
в ИПУ РАН (г.~Москва, Россия). ВСПУ XII организуется ИПУ РАН при поддержке РФФИ, Отделения 
энергетики, машиностроения, механики и процессов управления Российской академии наук, 
Российского 
национального комитета по автоматическому управлению, Академии навигации и управ\-ле\-ния 
движением, 
Научного совета РАН по комплексным проблемам управления и автоматизации, Совета по 
мехатронике и робототехнике РАН. Официальный язык Совещания~--- русский.

\vspace*{24pt}
     
     \textbf{Направления работы}
     \begin{enumerate}[1.]
\item Теория систем управления
\item Управление подвижными объектами и навигация
\item Интеллектуальные системы управления
\item Управление в промышленности, транспортом и логистикой
\item Управление системами междисциплинарной природы
\item Средства измерения, вычислений и контроля в управлении
\item Системный анализ и принятие решений в задачах управления
\item Информационные технологии в управлении
\item Проблемы образования в области управления: современное содержание и технологии обучения
\end{enumerate}

\vspace*{24pt}

     Подробная информация о Совещании находится на сайте {\sf http://vspu2014.ipu.ru}. Срок 
окончательной подачи докладов через систему подачи докладов на сайте~--- \textbf{30~ноября} 
2013~г.
}

%\include{rekl-1}

%\end{document}

%   \vspace*{-48pt}

\begin{center}
\vspace*{6pt}
\mbox{%
\epsfxsize=53.502mm
\epsfbox{foto-1.eps}
}
\end{center}

\vspace*{6pt} %Академик


   \begin{center}
\fbox{\Large\textbf{Профессор Игорь Алексеевич Ушаков}}\\[12pt]
\textbf{\large 22.01.1935--27.02.2015}
   \end{center}


   %\vspace*{2.5mm}

   \vspace*{5mm}

   \thispagestyle{empty}

%\

%\vspace*{-12pt}


Редакционный совет и редакционная коллегия журнала <<Информатика и~её применения>> с~глубоким прискорбием извещают, что 27~февраля 2015~г.\ после тяжелой
и~продолжительной болезни скончался Игорь Алексеевич Ушаков~--- доктор технических наук, профессор, член редколлегии журнала <<Информатика и ее применения>>.

Игорь Алексеевич Ушаков окончил Московский авиационный институт, в~1963~г.\ защитил кандидатскую, а~в~1968~г.~--- докторскую диссертацию. С~1958 по 1989~гг.\ работал в~ряде научно-исследовательских организаций СССР, в~том числе руководил отделами в~НИИ АА и~ВЦ АН СССР; с 1969 по 1989 гг. преподавал в~МФТИ (был профессором, а~затем заведующим кафедрой) и~в~МЭИ. С~1989~г.~---- в~США: являлся профессором университета Дж.\ Вашингтона, университета Дж.\ Мэйсона и~Калифорнийского университета, сотрудником компаний MCI, Qualcomm и Hughes.

И.\,А.~Ушаков с момента основания журнала <<Надежность и~контроль качества>> был заместителем ответственного редактора, а~затем на протяжении многих лет членом редколлегии. В~2006~г.\ основал электронный международный журнал ``Reliability: Theory \& Application'', главным редактором которого оставался до конца жизни.

Учебниками и справочниками по теории надежности, написанными И.\,А.~Ушаковым, пользовались и~пользуются несколько поколений ученых и~специалистов в~разных странах мира.

Игорь Алексеевич всегда уделял огромное внимание работе с~молодежью; более~50 его учеников защитили докторские и~кандидатские диссертации.

И.\,А.~Ушаков вел активную научно-про\-све\-ти\-тель\-скую деятельность. В~частности, он был одним из организаторов и~руководителей Московского кабинета качества и~надежности при Политехническом музее (целью этого Кабинета было оказание консультаций работникам промышленных предприятий и~чтение курсов лекций для инженеров, занимающихся проблемой надежности). Находясь в~США, И.\,А.~Ушаков создал международный ин\-тер\-нет-фо\-рум им.\ Б.\,В.~Гнеденко, объединивший около~400~видных специалистов по приложениям теории вероятностей и~математической статистики, преимущественно в~об\-ласти теории надежности и~анализа риска, из десятков стран мира; коллективным членов этого Форума является и~наш журнал. Цели Форума~--- содействие контактам между специалистами из разных стран, организация обмена профессиональными 
новостями и~информацией (новые публикации, предстоящие события и~др.). Также необходимо отметить большое число на\-уч\-но-по\-пу\-ляр\-ных работ, опубликованных И.\,А.~Ушаковым.

И.\,А.~Ушаков обладал большим личным обаянием, имел широкий круг интересов. Все знавшие И.\,А.~Ушакова всегда будут помнить его как замечательного ученого и~прекрасного человека.

\bigskip

Редакционный совет и редакционная коллегия журнала <<Информатика и~её применения>> 
выражают глубокие соболезнования родным и близким покойного, всем, кто его знал и~работал с~ним.



%\end{document}

%\include{IPPM-25}

\def\stat{cont-rus}
{%\hrule\par
%\vskip 7pt % 7pt
\vspace*{-24pt}
\raggedleft\Large \bf%\baselineskip=3.2ex
Правила подготовки рукописей  для публикации в журнале
<<Информатика~и~её~применения>> \vskip 8pt
    \hrule
    \par
\vskip 14pt plus 6pt minus 3pt }

\label{st\stat}

\def\tit{\ }

\def\aut{\ }
\def\auf{\ }

\def\leftkol{\ }
% Правила подготовки рукописей  для публикации в журнале
%<<Информатика и её применения>>

\def\rightkol{\ }
%Правила подготовки рукописей  для публикации в журнале
%<<Информатика и её применения>>}


\titele{\tit}{\aut}{\auf}{\leftkol}{\rightkol}


\vspace*{-60pt}
{ %\small

Журнал <<Информатика и её применения>>
публикует теоретические, обзорные и дискуссионные статьи,
посвященные научным исследованиям и разработкам в области
информатики и ее приложений.

Журнал издается на русском языке. По специальному решению
редколлегии отдельные статьи могут печататься на английском языке.

Тематика журнала охватывает следующие направления:
\begin{itemize}
\item теоретические основы информатики;\\[-15pt]
      \item
математические методы исследования сложных систем и процессов;\\[-15pt]
           \item
информационные системы и сети;\\[-15pt]
                \item
информационные технологии;\\[-15pt]
                     \item
архитектура и программное обеспечение вычислительных комплексов и сетей.\\[-15pt]
\end{itemize}


\noindent
\begin{enumerate}[1.]
\item В журнале печатаются статьи, содержащие результаты, ранее не опубликованные и
не предназначенные к одновременной публикации в других изданиях.

%Публикация не должна нарушать закон об авторских правах.
Публикация предоставленной автором(ами) рукописи не должна нарушать 
положений глав~69, 70 раздела~VII части~IV Гражданского кодекса, 
которые определяют права на результаты интеллектуальной деятельности 
и~средства индивидуализации, в~том числе авторские права, в~РФ.

Ответственность за нарушение авторских прав, в~случае предъявления претензий к~редакции журнала,  
несут авторы статей.



Направляя рукопись в редакцию, авторы сохраняют свои права на данную
рукопись и при этом передают учредителям и редколлегии журнала неисключительные права на
издание статьи на русском языке 
(или на языке статьи, если он отличен от рус\-ско\-го) и~на перевод ее на английский
язык, а~также на
ее распространение в России и за рубежом. 
Каждый автор должен представить в~редакцию подписанный 
с~его стороны <<Лицензионный договор о~передаче неисключительных прав 
на использование произведения>>, текст которого размещен по адресу 
{\sf http://www.ipiran.ru/publications/licence.doc}. 
Этот договор может быть пред\-став\-лен в~бумажном (в~2-х экз.)\ 
или в~электронном виде (отсканированная копия заполненного и~подписанного документа).




Редколлегия вправе запросить у авторов экспертное заключение о возможности
пуб\-ли\-ка\-ции пред\-став\-лен\-ной статьи в открытой печати.\\[-13.5pt]

\item К статье прилагаются данные автора (авторов) (см.\ п.~8). При наличии нескольких
авторов указывается фамилия автора, ответственного за переписку с редакцией.\\[-13.5pt]

\item Редакция журнала осуществляет экспертизу присланных статей в соответствии с
принятой в журнале процедурой рецензирования.

Возвращение рукописи на доработку не означает ее принятия к печати.

Доработанный вариант с ответом на замечания рецензента необходимо прислать в
редакцию.\\[-13.5pt]

\item Решение редколлегии о публикации статьи или ее отклонении сообщается авторам.

Редколлегия может также направить авторам текст рецензии на их статью. Дискуссия по
поводу отклоненных статей не ведется.\\[-13.5pt]

%\pagebreak

\item Редактура статей высылается авторам для просмотра. Замечания к редактуре должны
быть присланы авторами в кратчайшие сроки.\\[-13.5pt]

\item Рукопись предоставляется в электронном виде в форматах MS WORD (.doc или
.docx) или \LaTeX\  (.tex), дополнительно~--- в формате .pdf, на дискете, лазерном диске
или электронной почтой. Предоставление бумажной рукописи необязательно.\\[-13.5pt]

\item При подготовке рукописи в MS Word рекомендуется использовать следующие
настройки.

Параметры страницы:
формат~--- А4; ориентация~--- книжная; поля (см): внутри~--- 2,5, снаружи~--- 1,5,
сверху~--- 2, снизу~--- 2, от края до нижнего колонтитула~--- 1,3.

Основной текст: стиль~--- <<Обычный>>, шрифт~--- Times New Roman, размер~---
14~пунк\-тов, абзацный отступ~--- 0,5~см, 1,5~интервала, выравнивание~--- по ширине.

\pagebreak

\def\leftkol{Правила подготовки рукописей  для публикации в журнале
<<Информатика и её применения>>}

\def\rightkol{Правила подготовки рукописей  для публикации в журнале
<<Информатика и её применения>>}



Рекомендуемый объем рукописи~--- не свыше 10~страниц указанного формата.
При превышении указанного объема редколлегия вправе потребовать от 
автора сокращения объема рукописи.


Сокращения слов, помимо стандартных, не допускаются. Допускается минимальное
количество аббревиатур.


Все страницы рукописи нумеруются.

Шаблоны оформления представлены в интернете:

\noindent
 {\sf
http://www.ipiran.ru/journal/template\_iiep\_ssi\_2024.zip}\\[-14pt]

\item Статья должна содержать следующую информацию на {\bfseries\textit{русском и
английском языках}}:\\[-16pt]

\begin{itemize}
\item название статьи;\\[-15pt]
\item Ф.И.О.\ авторов, на английском можно только имя и фамилию;\\[-15pt]
\item место работы, с указанием почтового адреса организации и электронного адреса каждого
автора;\\[-15pt]
\item сведения об авторах, в соответствии с форматом, образцы которого
представлены на страницах:



\def\leftfootline{\small{\textbf{\thepage}
\hfill ИНФОРМАТИКА И ЕЁ ПРИМЕНЕНИЯ\ \ \ том\ 18\ \ \ выпуск\ 3\ \ \ 2024}
}%
 \def\rightfootline{\small{ИНФОРМАТИКА И ЕЁ ПРИМЕНЕНИЯ\ \ \ том\ 18\ \ \ выпуск\ 3\ \ \ 2024
\hfill \textbf{\thepage}}}



{\sf http://www.ipiran.ru/journal/issues/2013\_07\_01/authors.asp} и

{\sf http://www.ipiran.ru/journal/issues/2013\_07\_01\_eng/authors.asp};
\item аннотация (не менее 100~слов на каждом из языков). Аннотация~--- это краткое
резюме работы, которое может публиковаться отдельно. Она является основным
источником информации в~ин\-фор\-ма\-ци\-он\-ных системах и базах данных. Английская
аннотация должна быть оригинальной, может не быть дословным переводом русского
текста и должна быть написана хорошим английским языком. В~аннотации не должно
быть ссылок на литературу и, по возможности, формул;\\[-15pt]
\item ключевые слова~--- желательно из принятых в мировой
на\-уч\-но-тех\-ни\-че\-ской литературе тематических тезаурусов. Предложения не
могут быть ключевыми словами;\\[-15pt]
\item источники финансирования работы (ссылки на гранты, проекты,
поддерживающие организации и~т.\,п.).
\end{itemize}



%\pagebreak

\item  Требования к спискам литературы.\\[-14pt]

Ссылки на литературу в тексте статьи нумеруются (в квадратных скобках) и
располагаются в каждом из списков литературы в порядке  первых упоминаний. Если источник имеет DOI и/или EDN,
то их необходимо указывать.

Списки литературы представляются в двух вариантах:\\[-14pt]


\noindent
\begin{enumerate}[(1)]
\item \textbf{Список литературы к русскоязычной части}. Русские и английские
работы~---  на языке и в алфавите оригинала;\\[-14.5pt]
\item  \textbf{References}. Русские работы и работы на других языках~--- в латинской
транслитерации с переводом на английский язык; английские работы и работы на других
языках~--- на языке оригинала.
\end{enumerate}

Необходимо для составления списка ``References'' пользоваться размещенной на сайте
{\sf http://www. translit.net/ru/bgn/} бесплатной программой транслитерации русского
 текста в~латиницу. %, при этом в~за\-клад\-ке <<варианты\ldots>> следует выбратьопцию BGN.

Список литературы ``References'' приводится полностью отдельным блоком, повторяя все
позиции из списка литературы к русскоязычной части, независимо от того, имеются или
нет в нем иностранные источники. Если в списке литературы к русскоязычной части есть
ссылки на иностранные публикации, набранные латиницей, они полностью повторяются в
списке ``References''.

Ниже приведены примеры ссылок на различные виды публикаций в списке ``References''.

\def\leftfootline{\small{\textbf{\thepage}
\hfill ИНФОРМАТИКА И ЕЁ ПРИМЕНЕНИЯ\ \ \ том\ 18\ \ \ выпуск\ 3\ \ \ 2024}
}%
 \def\rightfootline{\small{ИНФОРМАТИКА И ЕЁ ПРИМЕНЕНИЯ\ \ \ том\ 18\ \ \ выпуск\ 3\ \ \ 2024
\hfill \textbf{\thepage}}}

{\small

\noindent
\textbf{Описание статьи из журнала:}

\Aue{Zagurenko, A.\,G., V.\,A.~Korotovskikh, A.\,A.~Kolesnikov, A.\,V.~Timonov, and D.\,V.~Kardymon}. 2008.
Tekhniko-ekonomicheskaya optimizatsiya dizayna gidrorazryva plasta [Technical and
economic optimization of the design
of hydraulic fracturing]. \textit{Neftyanoe hozyaystvo} [\textit{Oil Industry}] 11:54--57.

\Aue{Zhang, Z., and D.~Zhu}. 2008. Experimental research on the localized
electrochemical micromachining. \textit{Russ. J.~Electrochem.}  44(8):926--930.
{\sf doi:10.1134/S1023193508080077}.

\noindent
\textbf{Описание статьи из электронного журнала:}

\Aue{Swaminathan, V., E.~Lepkoswka-White, and B.\,P.~Rao}. 1999. Browsers or buyers in cyberspace? An
investigation of electronic factors influencing electronic exchange. \textit{JCMC}
5(2). Available at: {\sf http://www.ascusc.org/jcmc/vol5/issue2/} (accessed April~28, 2011).

\def\leftkol{Правила подготовки рукописей  для публикации в журнале
<<Информатика и её применения>>}

\def\rightkol{Правила подготовки рукописей  для публикации в журнале
<<Информатика и её применения>>}


\noindent
\textbf{Описание статьи из продолжающегося издания (сборника трудов):}

\Aue{Astakhov, M.\,V., and T.\,V.~Tagantsev}. 2006. Eksperimental'noe
issledovanie prochnosti soedineniy ``stal'--kompozit'' [Experimental study of
the strength of joints ``steel--composite'']. \textit{Trudy MGTU
``Matematicheskoe modelirovanie slozhnykh tekh\-ni\-che\-skikh sistem''}
[\textit{Bauman MSTU ``Mathematical Modeling of Complex Technical
Systems'' Proceedings}]. 593:125--130.


\pagebreak



\noindent
\textbf{Описание материалов конференций:}

\Aue{Usmanov, T.\,S., A.\,A.~Gusmanov, I.\,Z.~Mullagalin, R.\,Ju.~Muhametshina, A.\,N.~Chervyakova, and
A.\,V.~Sveshnikov}. 2007. Osobennosti proektirovaniya razrabotki mestorozhdeniy
s primeneniem gidrorazryva
plasta [Features of the design of field development with the use of hydraulic fracturing].
\textit{Trudy 6-go
Mezhdu\-na\-rod\-no\-go Simpoziuma ``Novye resursosberegayushchie tekhnologii nedropol'zovaniya i povysheniya
neftegazootdachi''} [\textit{6th  Symposium (International) ``New Energy Saving Subsoil Technologies and
the Increasing of the Oil and Gas Impact'' Proceedings}]. Moscow. 267--272.



\def\leftfootline{\small{\textbf{\thepage}
\hfill ИНФОРМАТИКА И ЕЁ ПРИМЕНЕНИЯ\ \ \ том\ 18\ \ \ выпуск\ 3\ \ \ 2024}
}%
 \def\rightfootline{\small{ИНФОРМАТИКА И ЕЁ ПРИМЕНЕНИЯ\ \ \ том\ 18\ \ \ выпуск\ 3\ \ \ 2024
\hfill \textbf{\thepage}}}



\noindent
\textbf{Описание книги (монографии, сборники):}



Lindorf, L.\,S., and L.\,G.~Mamikoniants, eds. 1972.
\textit{Ekspluatatsiya turbogeneratorov s neposredstvennym
okhlazhdeniem} [\textit{Operation of turbine generators with direct cooling}].
Moscow: Energy Publs. 352~p.


\Aue{Latyshev, V.\,N.} 2009. \textit{Tribologiya rezaniya. Kn.~1: Friktsionnye protsessy
pri rezanii metallov}
[\textit{Tribology of cutting. Vol.~1: Frictional processes in metal cutting}]. Ivanovo: Ivanovskii
State Univ. 108~p.

\def\leftkol{Правила подготовки рукописей  для публикации в журнале
<<Информатика и её применения>>}

\def\rightkol{Правила подготовки рукописей  для публикации в журнале
<<Информатика и её применения>>}

\noindent
\textbf{Описание переводной книги}
(в списке литературы к русскоязычной части необходимо указать:~/ Пер.\ с англ.~---
после названия книги, а в конце ссылки указать оригинал книги в круглых скобках):
\begin{enumerate}[1.]
\item  В русскоязычной части:

\def\leftfootline{\small{\textbf{\thepage}
\hfill ИНФОРМАТИКА И ЕЁ ПРИМЕНЕНИЯ\ \ \ том\ 18\ \ \ выпуск\ 3\ \ \ 2024}
}%
 \def\rightfootline{\small{ИНФОРМАТИКА И ЕЁ ПРИМЕНЕНИЯ\ \ \ том\ 18\ \ \ выпуск\ 3\ \ \ 2024
\hfill \textbf{\thepage}}}

\Au{Тимошенко С.\,П., Янг Д.\,Х., Уивер~У.}
Колебания в инженерном деле~/ Пер.\ с англ.~--- М.: Машиностроение, 1985. 472~с.
(\Au{Timoshenko~S.\,P., Young~D.\,H., Weaver~W.}
Vibration problems in engineering.~--- 4th ed.~--- New York, NY, USA: Wiley, 1974. 521~p.)\\[-13.5pt]
\item  В англоязычной части:

\Aue{Timoshenko, S.\,P., D.\,H.~Young, and W.~Weaver}.
1974. \textit{Vibration problems in engineering}. 4th ed. New York: 
Wiley. 521~p.
\end{enumerate}

\vspace*{-3pt}


\noindent
\textbf{Описание неопубликованного документа:}


\Aue{Latypov, A.\,R., M.\,M.~Khasanov, and V.\,A.~Baikov}.
2004 (unpubl.). Geologiya i~dobycha (NGT GiD) [Geology and production (NGT GiD)]. Certificate on official registration of the computer program
No.\,2004611198. 

\noindent
\textbf{Описание интернет-ресурса:}


Pravila tsitirovaniya istochnikov [Rules for the citing of sources]. Available at: {\sf
http://www.scribd.com/doc/1034528/} (accessed February~7, 2011).

%\pagebreak

\noindent
\textbf{Описание диссертации или автореферата диссертации:}

\Aue{Semenov, V.\,I.}
2003. Matematicheskoe modelirovanie plazmy v sisteme kompaktnyy tor [Mathematical
modeling of the plasma in the compact torus].  Moscow.  D.Sc.\ Diss. 272~p.

\Aue{Kozhunova, O.\,S.} 2009. Tekhnologiya razrabotki semanticheskogo
slovarya informatsionnogo monitoringa [Technology of development of
semantic dictionary of information monitoring system].  Moscow: IPI RAN. PhD Thesis. 23~p.


\noindent
\textbf{Описание ГОСТа:}

GOST 8.586.5-2005. 2007. Metodika vypolneniya izmereniy. Izmerenie raskhoda i~kolichestva zhidkostey i~gazov
s~pomoshch'yu standartnykh suzhayushchikh ustroystv [Method of measurement.
Measurement of flow rate and volume of liquids and gases by means of orifice devices]. Moscow:
Standardinform  Publs. 10~p.

\noindent
\textbf{Описание патента:}

\Aue{Bolshakov, M.\,V., A.\,V.~Kulakov, A.\,N.~Lavrenov, and M.\,V.~Palkin}.
2006. Sposob orientirovaniya po krenu letatel'nogo
apparata s opti\-che\-skoy golovkoy
samonavedeniya [The way to orient on the roll of aircraft with optical homing head].
Patent RF No.\,2280590.
}

\item Присланные в редакцию материалы авторам не возвращаются.\\[-13.5pt]

\item При отправке файлов по электронной почте просим придерживаться следующих
правил:
\begin{itemize}
\item указывать в поле subject (тема) название журнала и фамилию автора;\\[-13.5pt]
\item указывать в тексте письма название статьи, авторов и~журнал, в~который направляется статья;\\[-13.5pt]
\item использовать attach (присоединение);\\[-13.5pt]
\item в состав электронной версии статьи должны входить: файл, содержащий текст
статьи, и файл(ы), содержащий(е) иллюстрации.\\[-13.5pt]
\end{itemize}

\item Журнал <<Информатика и её применения>> является некоммерческим изданием.
Плата за публикацию не взимается, гонорар авторам не выплачивается.
\end{enumerate}



\def\leftfootline{\small{\textbf{\thepage}
\hfill ИНФОРМАТИКА И ЕЁ ПРИМЕНЕНИЯ\ \ \ том\ 18\ \ \ выпуск\ 3\ \ \ 2024}
}%
 \def\rightfootline{\small{ИНФОРМАТИКА И ЕЁ ПРИМЕНЕНИЯ\ \ \ том\ 18\ \ \ выпуск\ 3\ \ \ 2024
\hfill \textbf{\thepage}}}


\vspace*{-1mm}

\begin{center}

\textbf{Адрес редакции журнала <<Информатика и её применения>>:} \\




Москва 119333, ул.~Вавилова, д.~44, корп.~2, ФИЦ ИУ РАН\\[-10pt]

\

Тел.: +7\,(499)\,135-86-92\ \ Факс:  +7\,(495)\,930-45-05\\[-10pt]

 \

e-mail:   {\sf iiep@frccsc.ru} (Стригина Светлана Николаевна)\\[-10pt]

\

{\sf http://www.ipiran.ru/journal/issues/}
\end{center}
}


\def\leftkol{Правила подготовки рукописей  для публикации в журнале
<<Информатика и её применения>>}

\def\rightkol{Правила подготовки рукописей  для публикации в журнале
<<Информатика и её применения>>}


\def\leftfootline{\small{\textbf{\thepage}
\hfill ИНФОРМАТИКА И ЕЁ ПРИМЕНЕНИЯ\ \ \ том\ 18\ \ \ выпуск\ 3\ \ \ 2024}
}%
 \def\rightfootline{\small{ИНФОРМАТИКА И ЕЁ ПРИМЕНЕНИЯ\ \ \ том\ 18\ \ \ выпуск\ 3\ \ \ 2024
\hfill \textbf{\thepage}}} 
\def\stat{podg-e}
{%\hrule\par
%\vskip 7pt % 7pt
\vspace*{-24pt}
\raggedleft\Large \bf%\baselineskip=3.2ex
Requirements for manuscripts submitted to Journal
``Informatics~and~Applications'' \vskip 8pt
    \hrule
    \par
\vskip 21pt plus 6pt minus 3pt }

\label{st\stat}

\def\tit{\ }

\def\aut{\ }
\def\auf{\ }

\def\leftkol{\ }

\def\rightkol{\ }
%Requirements for manuscripts submitted to Journal
%``Informatics~and~Applications''}

\titele{\tit}{\aut}{\auf}{\leftkol}{\rightkol}

\def\leftfootline{\small{\textbf{\thepage}
\hfill INFORMATIKA I EE PRIMENENIYA~--- INFORMATICS AND APPLICATIONS\ \ \ 2019\
\ \ volume~13\ \ \ issue\ 4}
}%
 \def\rightfootline{\small{INFORMATIKA I EE PRIMENENIYA~--- INFORMATICS AND APPLICATIONS\ \ \ 2019\ \ \ volume~13\ \ \ issue\ 4
\hfill \textbf{\thepage}}}

\vspace*{-60pt}

{\small

\noindent
Journal ``Informatics and Applications'' (Inform.\ Appl.)
publishes theoretical, review, and discussion
articles on the research and development in the
field of informatics and its applications.

The journal is published in Russian.
By a special decision of the editorial
board, some articles can be published in English.


The topics covered include the following areas:
\begin{itemize}
               \item
     theoretical fundamentals of informatics; \\[-14pt]
\item
mathematical methods for studying complex systems and processes; \\[-14pt]
\item
information systems and networks;\\[-14pt]
\item
information technologies; and \\[-14pt]
\item
architecture and software of computational complexes and networks. \\[-14pt]
\end{itemize}

\noindent
\begin{enumerate}[1.]
\item The Journal publishes original articles which have not been published before and are not
intended for simultaneous publication in other editions. An article submitted to the Journal must not violate the
Copyright law. Sending the manuscript to the Editorial Board, the authors retain all rights of the
owners of the manuscript and transfer the nonexclusive rights to publish the article in Russian
(or the language of the article, if not Russian) and its distribution in Russia and abroad to the
Founders and the Editorial Board. Authors should submit a letter to the Editorial Board in the
following form:

{\bfseries\textit{Agreement on the transfer of rights to publish:}}

``\textit{We, the undersigned authors of the manuscript ``\ldots'', pass to the
Founder and the Editorial Board of the Journal ``Informatics and Applications''
the nonexclusive right to publish the manuscript of the article in Russian (or
in English) in both print and electronic versions of the Journal. We affirm
that this publication does not violate the Copyright of other persons or
organizations.}

\textit{Author(s) signature(s): (name(s), address(es), date).}

This agreement should be submitted in paper form or in the form of a scanned copy (signed by
the authors).


%The Editorial Board has the right to request from the authors an official expert conclusion that
%the submitted article has no secret data prohibited for publication. \\[-13.5pt]
\item
A submitted article should be attached with \textbf{the data on the author(s)} (see item~8). If
there are several authors, the contact person should be indicated who is responsible for
correspondence with the Editorial Board and other authors about revisions and final approval
of the proofs.\\[-13.5pt]

\item The Editorial Board of the Journal examines the article according to the established
reviewing procedure. If the authors receive their article for correction after reviewing, it does not
mean that the article is approved for publication. The corrected article should be sent to the
Editorial Board for the subsequent review and approval.\\[-13.5pt]

\item The decision on the article publication or its rejection is communicated to the authors. The
Editorial Board may also send the reviews on the submitted articles to the authors. Any
discussion upon the rejected articles is not possible.\\[-13.5pt]

\item The edited articles will be sent to the authors for proofread. The comments of the authors
to the edited text of the article should be sent to the Editorial Board as soon as possible.\\[-13.5pt]

\item The manuscript of the article should be presented electronically in the MS WORD (.doc or
.docx) or \LaTeX\ (.tex) formats, and additionally in the .pdf format. All documents
 may be sent
by e-mail or provided on a CD or diskette. A~hard copy submission is not necessary.\\[-13.5pt]

\item The recommended typesetting instructions for manuscript.

Pages parameters: format A4, portrait orientation, document margins (cm): left~--- 2.5, right~---
1.5, above~--- 2.0, below~--- 2.0, footer 1.3.

Text: font~---Times New Roman, font size~--- 14, paragraph indent~--- 0.5, line spacing~--- 1.5,
justified alignment.

The recommended manuscript size: not more than 15~pages of the specified format.
If the specified size exceeded, the editorial board is entitled to require the author
to reduce the manuscript.

Use only standard abbreviations. Avoid  abbreviations in the title and
abstract. The full term for which an abbreviation stands should precede
its first use in the text unless it is a standard unit of measurement.

All pages of the manuscript should be numbered.

The templates for the manuscript typesetting are presented on site: {\sf
http://www.ipiran.ru/journal/template.doc}.\\[-13.5pt]


%\def\leftkol{Requirements for manuscripts submitted to Journal
%``Informatics~and~Applications''}

\item The articles should enclose data both in \textbf{Russian and English}:
\begin{itemize}
\item title;\\[-13.5pt]
\item author's name and surname;\\[-13.5pt]
\item affiliation~--- organization, its address with ZIP code, city, country, and
official e-mail address;\\[-13.5pt]
\item data on authors according to the format: (see site)

{\sf http://www.ipiran.ru/journal/issues/2013\_07\_01/authors.asp}  and

{\sf  http://www.ipiran.ru/journal/issues/2013\_07\_01\_eng/authors.asp};\\[-13.5pt]

\pagebreak

\def\leftfootline{\small{\textbf{\thepage}
\hfill INFORMATIKA I EE PRIMENENIYA~--- INFORMATICS AND APPLICATIONS\ \ \ 2019\
\ \ volume~13\ \ \ issue\ 4}
}%
 \def\rightfootline{\small{INFORMATIKA I EE PRIMENENIYA~--- INFORMATICS AND APPLICATIONS\ \ \ 2019\ \ \ volume~13\ \ \ issue\ 4
\hfill \textbf{\thepage}}}


%\def\leftkol{Requirements for manuscripts submitted to Journal
%``Informatics~and~Applications''}

%\def\rightkol{Requirements for manuscripts submitted to Journal
%``Informatics~and~Applications''}



\item abstract (not less than 100 words) both in Russian and in English. Abstract is a short
summary of the article that can be published separately. The abstract is the
main source of information on the article and it could be included in leading information
systems and data bases. The abstract in English has to be an original text and should
not be an exact translation of the Russian one. Good English is required.
In abstracts, avoid references and formulae;\\[-13.5pt]
\item indexing is performed on the basis of keywords. The use of keywords from the
internationally accepted thematic Thesauri is recommended.

%\def\leftkol{Requirements for manuscripts submitted to Journal
%``Informatics~and~Applications''}

%\def\rightkol{Requirements for manuscripts submitted to Journal
%``Informatics~and~Applications''}

Important! Keywords must not be sentences;
\item Acknowledgments.
\end{itemize}

\item References. Russian references have to be presented both in English translation and Latin
transliteration (refer {\sf http://www.translit.net/ru/bgn/}).

Please take into account the following examples of Russian references appearance:

\noindent
\textbf{Article in journal:}

\Aue{Zhang, Z., and D.~Zhu}. 2008. Experimental research on the localized electrochemical
micromachining.
\textit{Rus. J.~Electrochem.}  44(8):926--930. {\sf doi:10.1134/S1023193508080077}.


\noindent
\textbf{Journal article in electronic format:}

\Aue{Swaminathan, V., E.~Lepkoswka-White, and B.\,P.~Rao}. 1999. Browsers or buyers in
cyberspace? An
investigation of electronic factors influencing electronic exchange. \textit{JCMC}
5(2). Available at: {\sf http://www.ascusc.org/jcmc/vol5/issue2/} (accessed April~28, 2011).




\noindent
\textbf{Article from the continuing publication (collection of works, proceedings):}

\Aue{Astakhov, M.\,V., and T.\,V.~Tagantsev}. 2006. Eksperimental'noe
issledovanie prochnosti soedineniy ``stal'--kompozit'' [Experimental study of
the strength of joints ``steel--composite'']. \textit{Trudy MGTU
``Matematicheskoe modelirovanie slozhnykh tekh\-ni\-che\-skikh sistem''}
[\textit{Bauman MSTU ``Mathematical Modeling of Complex Technical
Systems'' Proceedings}]. 593:125--130.

\def\leftfootline{\small{\textbf{\thepage}
\hfill INFORMATIKA I EE PRIMENENIYA~--- INFORMATICS AND APPLICATIONS\ \ \ 2019\
\ \ volume~13\ \ \ issue\ 4}
}%
 \def\rightfootline{\small{INFORMATIKA I EE PRIMENENIYA~--- INFORMATICS AND APPLICATIONS\ \ \ 2019\ \ \ volume~13\ \ \ issue\ 4
\hfill \textbf{\thepage}}}

\def\leftkol{Requirements for manuscripts submitted to Journal
``Informatics~and~Applications''}

\def\rightkol{Requirements for manuscripts submitted to Journal
``Informatics~and~Applications''}

\noindent
\textbf{Conference proceedings:}

\Aue{Usmanov, T.\,S., A.\,A.~Gusmanov, I.\,Z.~Mullagalin, R.\,Ju.~Muhametshina,
A.\,N.~Chervyakova, and
A.\,V.~Sveshnikov}. 2007. Osobennosti proektirovaniya razrabotki mestorozhdeniy
s primeneniem gidrorazryva
plasta [Features of the design of field development with the use of hydraulic fracturing].
\textit{Trudy 6-go
Mezhdu\-na\-rod\-no\-go Simpoziuma ``Novye resursosberegayushchie tekhnologii
nedropol'zovaniya i povysheniya
neftegazootdachi''} [\textit{6th  Symposium (International) ``New Energy Saving Subsoil
Technologies and
the Increasing of the Oil and Gas Impact'' Proceedings}]. Moscow. 267--272.


\noindent
\textbf{Books and other monographs:}




Lindorf, L.\,S., and L.\,G.~Mamikoniants, eds. 1972.
\textit{Ekspluatatsiya turbogeneratorov s neposredstvennym
okhlazhdeniem} [\textit{Operation of turbine generators with direct cooling}].
Moscow: Energy Publs. 352~p.


%\Aue{Latyshev, V.\,N.} 2009. \textit{Tribologiya rezaniya. Kn.~1: Frikcionnye prosessy
%pri rezanii metallov}
%[\textit{Tribology of cutting. Vol.~1: Frictional processes in metal cutting}]. Ivanovo: Ivanovskii
%State Univ. 108~p.


%\noindent
%\textbf{Unpublished material:}

%\Aue{Latypov, A.\,R., M.\,M.~Khasanov, and V.\,A.~Baikov}.
%2004. Geology and production (NGT GiD). Certificate on official registration of the computer
%program
%No.\,2004611198. (In Russian, unpubl.)

%\noindent
%\textbf{Internet-source:}

%APA Style. 2011. Available at: {\sf http://www.apastyle.org/apa-style-help.aspx} (accessed
%February~5, 2011).

%Pravila citirovaniya istochnikov [Rules for the citing of sources]. Available at: {\sf
%http://www.scribd.com/doc/1034528/} (accessed February~7, 2011).


\noindent
\textbf{Dissertation and Thesis:}

%\Aue{Semenov, V.\,I.}
%2003. Matematicheskoe modelirovanie plazmy v sisteme kompaktnyy tor. [Mathematical
%modeling of the plasma in the compact torus]. D.Sc.\ Diss. Moscow. 272~p.

\Aue{Kozhunova, O.\,S.} 2009. Tekhnologiya razrabotki semanticheskogo
slovarya informatsionnogo monitoringa [Technology of development of
semantic dictionary of information monitoring system]. PhD Thesis. Moscow: IPI RAN. 23~p.


\noindent
\textbf{State standards and patents:}

GOST 8.586.5-2005. 2007. Metodika vypolneniya izmereniy. Izmerenie raskhoda i~kolichestva
zhidkostey i gazov 
s~pomoshch'yu standartnykh suzhayushchikh ustroystv [Method of measurement.
Measurement of flow rate and volume of liquids and gases by means of orifice devices]. M.:
Standardinform
Publs. 10~p.

%\noindent
%\textbf{Patent:}

\Aue{Bolshakov, M.\,V., A.\,V.~Kulakov, A.\,N.~Lavrenov, and M.\,V.~Palkin}.
2006. Sposob orientirovaniya po krenu letatel'nogo
apparata s opti\-che\-skoy golovkoy
samonavedeniya [The way to orient on the roll of aircraft with optical homing head].
Patent RF No.\,2280590.

References in Latin transcription are presented in the original language.

References in the text are numbered according to the order of their
first appearance; the number is
placed in square brackets. All items from the reference list should be
cited.\\[-13.5pt]

\item Manuscripts and additional materials are not returned to Authors by the Editorial Board.\\[-13.5pt]

\item Submissions of files by e-mail must include:\\[-13.5pt]
\begin{itemize}
\item   the journal title and author's name in the ``Subject'' field; \\[-13.5pt]
\item   an article and additional materials have to be attached using the ``attach'' function;\\[-13.5pt]
\item   an electronic version of the article should contain the file with the text and a separate file
with figures.\\[-13.5pt]
\end{itemize}

\item ``Informatics and Applications'' journal is not a profit publication. There are no
charges for the authors as well as there are no royalties.\\[-13.5pt]
\end{enumerate}

\def\leftfootline{\small{\textbf{\thepage}
\hfill INFORMATIKA I EE PRIMENENIYA~--- INFORMATICS AND APPLICATIONS\ \ \ 2019\
\ \ volume~13\ \ \ issue\ 4}
}%
 \def\rightfootline{\small{INFORMATIKA I EE PRIMENENIYA~--- INFORMATICS AND APPLICATIONS\ \ \ 2019\ \ \ volume~13\ \ \ issue\ 4
\hfill \textbf{\thepage}}}

\def\leftkol{Requirements for manuscripts submitted to Journal
``Informatics~and~Applications''}

\def\rightkol{Requirements for manuscripts submitted to Journal
``Informatics~and~Applications''}


%\vspace*{5mm}


\begin{center}
\textbf{Editorial Board address:} \\

%ABOUT AUTHORS



FRC CSC RAS, 44, block~2, Vavilov Str., Moscow 119333, Russia\\[-10pt]

\

Ph.: +7\,(499)\,135\,86\,92,\ \ Fax: +7\,(495)\,930\,45\,05\\[-10pt]

\

 e-mail: {\sf rust@ipiran.ru} (to Prof.\ Rustem Seyful-Mulyukov)\\[-10pt]

\

 {\sf http://www.ipiran.ru/english/journal.asp}
\end{center}
 }
%\thispagestyle{myheadings}

\def\leftkol{Requirements for manuscripts submitted to Journal
``Informatics~and~Applications''}

\def\rightkol{Requirements for manuscripts submitted to Journal
``Informatics~and~Applications''}

\def\leftfootline{\small{\textbf{\thepage}
\hfill INFORMATIKA I EE PRIMENENIYA~--- INFORMATICS AND APPLICATIONS\ \ \ 2019\
\ \ volume~13\ \ \ issue\ 4}
}%
 \def\rightfootline{\small{INFORMATIKA I EE PRIMENENIYA~--- INFORMATICS AND APPLICATIONS\ \ \ 2019\ \ \ volume~13\ \ \ issue\ 4
\hfill \textbf{\thepage}}}

 \label{end\stat}

\newpage


%\vspace*{-60pt} {\small
{\baselineskip=9.1pt
\section*{Правила подготовки рукописей статей для публикации в журнале
<<Информатика и её применения>>}

\thispagestyle{empty}

 Журнал <<Информатика и её применения>> публикует
теоретические, обзорные и дискуссионные статьи, посвященные научным
исследованиям и разработкам в области информатики и ее приложений. Журнал
издается на русском языке. По специальному решению редколлегии отдельные статьи,
в виде исключения, могут печататься на английском языке.
Тематика журнала охватывает следующие направления:
\begin{itemize}
\item теоретические основы информатики; %\\[-13.5pt]
\item математические методы исследования сложных систем и процессов; %\\[-13.5pt]
\item информационные системы и сети; %\\[-13.5pt]
\item информационные технологии; %\\[-13.5pt]
\item архитектура и программное
обеспечение вычислительных комплексов и сетей.
\end{itemize}
\begin{enumerate}
\item В журнале печатаются результаты, ранее не
опубликованные и не предназначенные к одновременной публикации в других
изданиях. Публикация не должна нарушать закон об авторских правах. Направляя
свою рукопись в редакцию, авторы автоматически передают учредителям и
редколлегии неисключительные права на издание данной статьи на русском языке и
на ее распространение в России и за рубежом. При этом за авторами сохраняются
все права как собственников данной рукописи. В связи с этим авторами должно
быть представлено в редакцию письмо в следующей форме:
Соглашение о передаче права на публикацию:

\textit{<<Мы, нижеподписавшиеся, авторы рукописи <<$\qquad\qquad$>>, передаем
учредителям и редколлегии журнала <<Информатика и её применения>>
неисключительное право опубликовать данную рукопись статьи на русском языке как
в печатной, так и в электронной версиях журнала. Мы подтверждаем, что данная
публикация не нарушает авторского права других лиц или организаций. Подписи
авторов: (ф.\,и.\,о., дата, адрес)>>.}

Указанное соглашение может быть представлено 
как в бумажном виде, так и в виде отсканированной копии (с подписями авторов).


Редколлегия вправе запросить у авторов экспертное заключение о возможности
опубликования представленной статьи в открытой печати. %\\[-13.5pt]
\item Статья
подписывается всеми авторами. На отдельном листе представляются данные автора
(или всех авторов): фамилия, полные имя и отчество, телефон, факс, e-mail,
почтовый адрес. Если работа выполнена несколькими авторами, указывается фамилия
одного из них, ответственного за переписку с редакцией. %\\[-13.5pt]
\item Редакция журнала
осуществляет самостоятельную экспертизу присланных статей. Возвращение рукописи
на доработку не означает, что статья уже принята к печати. Доработанный вариант
с ответом на замечания рецензента необходимо прислать в редакцию. %\\[-13.5pt]
\item Решение
редакционной коллегии о принятии статьи к печати или ее отклонении сообщается
авторам. Редколлегия не обязуется направлять рецензию авторам отклоненной
статьи. %\\[-13.5pt]
\item Корректура статей высылается авторам для просмотра. Редакция
просит авторов присылать свои замечания в кратчайшие сроки. %\\[-13.5pt]
\item При
подготовке рукописи в MS Word рекомендуется использовать следующие настройки.
Параметры страницы: формат~--- А4; ориентация~--- книжная; поля (см): внутри~---
2,5, снаружи~--- 1,5, сверху~--- 2, снизу~--- 2, от края до нижнего
колонтитула~--- 1,3. Основной текст: стиль~--- <<Обычный>>: шрифт Times New
Roman, размер 14~пунктов, абзацный отступ~--- 0,5~см, 1,5 интервала,
выравнивание~--- по ширине. Рекомендуемый объем рукописи~--- не свыше
25~страниц указанного формата. Ознакомиться с шаблонами, содержащими примеры
оформления, можно по адресу в Интернете:
\textsf{http://www.ipiran.ru/journal/template.doc}.
\item К рукописи, предоставляемой в 2-х
экземплярах, обязательно прилагается электронная версия статьи (как правило, в
форматах MS WORD (.doc) или \LaTeX\ (.tex), а также~--- дополнительно~--- в
формате .pdf) на дискете, лазерном диске или по электронной почте. Сокращения
слов, кроме стандартных, не применяются. Все страницы рукописи должны быть
пронумерованы. %\\[-13.5pt]
\item Статья должна содержать следующую информацию на русском и
английском языках: название, Ф.И.О. авторов, места работы авторов и их
электронные адреса, подробные сведения об авторах, оформленные в соответствии с форматом, 
определяемым файлами {\sf http://www.ipiran.ru/journal/issues/2011\_05\_01/authors.asp} и 
{\sf http://www.ipiran.ru/journal/issues/2011\_01\_eng/authors.asp},
аннотация (не более 100~слов), ключевые слова. Ссылки на
литературу в тексте статьи нумеруются (в квадратных скобках) и располагаются в
порядке их первого упоминания. В~списке литературы не должно быть позиций, на которые нет ссылки в тексте статьи.
Все фамилии авторов, заглавия статей, названия
книг, конференций и~т.\,п.\ даются на языке оригинала, если этот язык
использует кириллический или латинский алфавит. %\\[-13.5pt]
\item Присланные в редакцию материалы авторам не возвращаются.
\item При отправке файлов по электронной
почте просим придерживаться следующих правил:
\begin{itemize}
\item указывать в поле subject (тема) название журнала и фамилию автора; %\\[-13.5pt]
\item использовать attach (присоединение); %\\[-13.5pt]
\item в случае больших объемов информации возможно
использование общеизвестных архиваторов (ZIP, RAR); %\\[-13.5pt]
\item в состав электронной версии статьи должны входить: файл, содержащий текст статьи, и файл(ы),
содержащий(е) иллюстрации. %\\[-13.5pt]
\end{itemize}
\item Журнал <<Информатика и её применения>> является некоммерческим изданием. 
Плата за публикацию с авторов не взимается, гонорар авторам не выплачивается.
\end{enumerate}
\thispagestyle{empty}
\textbf{Адрес редакции:} Москва 119333,
ул.~Вавилова, д.~44, корп.~2, ИПИ РАН\\
\hphantom{\textbf{Адрес редакции:} }Тел.: +7 (499) 135-86-92\ \
Факс:  +7 (495) 930-45-05\ \  E-mail:   rust@ipiran.ru }
}

\end{document}


%\tableofcontents

%\end{document}





%\def\stat{cont}
{%\hrule\par
%\vskip 7pt % 7pt
\raggedleft\Large \bf%\baselineskip=3.2ex
А\,В\,Т\,О\,Р\,С\,К\,И\,Й\ \ У\,К\,А\,З\,А\,Т\,Е\,Л\,Ь\ \ З\,А\ \ 2\,0\,0\,7 г. \vskip 17pt
    \hrule
    \par
\vskip 21pt plus 6pt minus 3pt }

\label{st\stat}

\def\tit{\ }

\def\aut{\ }
\def\auf{\ }

\def\leftkol{\ } % ENGLISH ABSTRACTS}

\def\rightkol{\ } %ENGLISH ABSTRACTS}

\titele{\tit}{\aut}{\auf}{\leftkol}{\rightkol}


\contentsline {chapter}{\ }{Выпуск \quad Стр.} 
\contentsline {section}{\textbf{Батракова Д.\,А., Королев В.\,Ю., Шоргин С.\,Я.}\ \ Новый метод вероятностно-ста\-ти\-сти\-че\-ско\-го анализа информационных потоков в\nobreakspace {}телекоммуникационных сетях}{\qquad 1 \qquad 40} 
\contentsline {section}{\textbf{Борисов А.\,В.}\ \ Байесовское оценивание в системах наблюдения с\nobreakspace {}марковскими скачкообразными процессами: игровой подход}{\qquad 2 \qquad 65}
\contentsline {section}{\textbf{Босов А.\,В., Иванов А.\,В.}\ \ Программная инфраструктура информационного Web-пор\-тала}{\qquad 2 \qquad 50}
\contentsline {section}{\textbf{Захаров В.\,Н., Калиниченко Л.\,А., Соколов И.\,А., Ступников С.\,А.}\ \ Конструирование канонических информационных моделей для интегрированных информационных систем}{\qquad 2 \qquad 15}
\contentsline {section}{\textbf{Захаров В.\,Н., Козмидиади В.\,А.}\ \ Средства обеспечения отказоустойчивости при\-ло\-жений}{\qquad 1 \qquad 14} 
\contentsline {section}{\textbf{Иванов А.\,В.}\ \ см. Босов А.\,В.\hfill\hfill\hfill\hfill\hfill\hfill\hfill\hfill\hfill\hfill\hfill\hfill\hfill\hfill\hfill\hfill\hfill\hfill\hfill\hfill\hfill\hfill\hfill\hfill\hfill\hfill\hfill\hfill\hfill\hfill\hfill\hfill\hfill\hfill\hfill}{\ }
\contentsline {section}{\textbf{Ильин В.\,Д., Соколов И.\,А.}\ \ Символьная модель системы знаний информатики в\nobreakspace {}че\-ло\-ве\-ко-автоматной среде}{\qquad 1 \qquad 66} 
\contentsline {section}{\textbf{Калиниченко Л.\,А.}\ \ см. Захаров В.\,Н.\hfill\hfill\hfill\hfill\hfill\hfill\hfill\hfill\hfill\hfill\hfill\hfill\hfill\hfill\hfill\hfill\hfill\hfill\hfill\hfill\hfill\hfill\hfill\hfill\hfill\hfill\hfill\hfill\hfill\hfill\hfill\hfill\hfill\hfill\hfill}{\ }
\contentsline {section}{\textbf{Козеренко Е.\,Б.}\ \ Лингвистическое моделирование для систем машинного перевода и обработки знаний}{\qquad 1 \qquad 54} 
\contentsline {section}{\textbf{Козмидиади В.\,А.}\ \ см. Захаров В.\,Н.\hfill\hfill\hfill\hfill\hfill\hfill\hfill\hfill\hfill\hfill\hfill\hfill\hfill\hfill\hfill\hfill\hfill\hfill\hfill\hfill\hfill\hfill\hfill\hfill\hfill\hfill\hfill\hfill\hfill\hfill\hfill\hfill\hfill\hfill\hfill }{\ } 
\contentsline {section}{\textbf{Королев В.\,Ю.}\ \ см. Батракова Д.\,А.\hfill\hfill\hfill\hfill\hfill\hfill\hfill\hfill\hfill\hfill\hfill\hfill\hfill\hfill\hfill\hfill\hfill\hfill\hfill\hfill\hfill\hfill\hfill\hfill\hfill\hfill\hfill\hfill\hfill\hfill\hfill\hfill\hfill\hfill\hfill}{\ } 
\contentsline {section}{\textbf{Кудрявцев А.\,А., Шоргин С.\,Я.}\ \ Байесовский подход к\nobreakspace {}анализу систем массового обслуживания и\nobreakspace {}показателей надежности}{\qquad 2 \qquad 76}
\contentsline {section}{\textbf{Печинкин А.\,В., Соколов И.\,А., Чаплыгин В.\,В.}\ \ Многолинейная система массового обслуживания с конечным накопителем и ненадежными приборами}{\qquad 1 \qquad 27} 
\contentsline {section}{\textbf{Печинкин А.\,В., Соколов И.\,А., Чаплыгин В.\,В.}\ \ Стационарные характеристики многолинейной\nobreakspace {}системы массового обслуживания с\nobreakspace {}одновременными отказами приборов}{\qquad 2 \qquad 39}
\contentsline {section}{\textbf{Синицын И.\,Н.}\ \ Корреляционные методы построения аналитических информационных моделей флуктуаций полюса Земли по априорным данным}{\qquad 2 \qquad \hphantom{9}2}
\contentsline {section}{\textbf{Синицын И.\,Н.}\ \ Развитие теории фильтров Пугачева для оперативной обработки информации в стохастических системах}{{\qquad 1 \qquad \hphantom{9}3}} 
\contentsline {section}{\textbf{Соколов И.\,А.}\ \ см. Захаров В.\,Н.\hfill\hfill\hfill\hfill\hfill\hfill\hfill\hfill\hfill\hfill\hfill\hfill\hfill\hfill\hfill\hfill\hfill\hfill\hfill\hfill\hfill\hfill\hfill\hfill\hfill\hfill\hfill\hfill\hfill\hfill\hfill\hfill\hfill\hfill\hfill}{\ }
\contentsline {section}{\textbf{Соколов И.\,А.}\ \ см. Ильин В.\,Д.\hfill\hfill\hfill\hfill\hfill\hfill\hfill\hfill\hfill\hfill\hfill\hfill\hfill\hfill\hfill\hfill\hfill\hfill\hfill\hfill\hfill\hfill\hfill\hfill\hfill\hfill\hfill\hfill\hfill\hfill\hfill\hfill\hfill\hfill\hfill}{\ } 
\contentsline {section}{\textbf{Соколов И.\,А.}\ \ см. Печинкин А.\,В.\hfill\hfill\hfill\hfill\hfill\hfill\hfill\hfill\hfill\hfill\hfill\hfill\hfill\hfill\hfill\hfill\hfill\hfill\hfill\hfill\hfill\hfill\hfill\hfill\hfill\hfill\hfill\hfill\hfill\hfill\hfill\hfill\hfill\hfill\hfill}{\ } 
\contentsline {section}{\textbf{Соколов И.\,А.}\ \ см. Печинкин А.\,В.\hfill\hfill\hfill\hfill\hfill\hfill\hfill\hfill\hfill\hfill\hfill\hfill\hfill\hfill\hfill\hfill\hfill\hfill\hfill\hfill\hfill\hfill\hfill\hfill\hfill\hfill\hfill\hfill\hfill\hfill\hfill\hfill\hfill\hfill\hfill}{\ }
\contentsline {section}{\textbf{Ступников С.\,А.}\ \ см. Захаров В.\,Н.\hfill\hfill\hfill\hfill\hfill\hfill\hfill\hfill\hfill\hfill\hfill\hfill\hfill\hfill\hfill\hfill\hfill\hfill\hfill\hfill\hfill\hfill\hfill\hfill\hfill\hfill\hfill\hfill\hfill\hfill\hfill\hfill\hfill\hfill\hfill}{\ }
\contentsline {section}{\textbf{Чаплыгин В.\,В.}\ \ см. Печинкин А.\,В.\hfill\hfill\hfill\hfill\hfill\hfill\hfill\hfill\hfill\hfill\hfill\hfill\hfill\hfill\hfill\hfill\hfill\hfill\hfill\hfill\hfill\hfill\hfill\hfill\hfill\hfill\hfill\hfill\hfill\hfill\hfill\hfill\hfill\hfill\hfill}{\ } 
\contentsline {section}{\textbf{Чаплыгин В.\,В.}\ \ см. Печинкин А.\,В.\hfill\hfill\hfill\hfill\hfill\hfill\hfill\hfill\hfill\hfill\hfill\hfill\hfill\hfill\hfill\hfill\hfill\hfill\hfill\hfill\hfill\hfill\hfill\hfill\hfill\hfill\hfill\hfill\hfill\hfill\hfill\hfill\hfill\hfill\hfill}{\ }
\contentsline {section}{\textbf{Шоргин С.\,Я.}\ \ см. Батракова Д.\,А.\hfill\hfill\hfill\hfill\hfill\hfill\hfill\hfill\hfill\hfill\hfill\hfill\hfill\hfill\hfill\hfill\hfill\hfill\hfill\hfill\hfill\hfill\hfill\hfill\hfill\hfill\hfill\hfill\hfill\hfill\hfill\hfill\hfill\hfill\hfill}{\ } 
\contentsline {section}{\textbf{Шоргин С.\,Я.}\ \ см. Кудрявцев А.\,А.\hfill\hfill\hfill\hfill\hfill\hfill\hfill\hfill\hfill\hfill\hfill\hfill\hfill\hfill\hfill\hfill\hfill\hfill\hfill\hfill\hfill\hfill\hfill\hfill\hfill\hfill\hfill\hfill\hfill\hfill\hfill\hfill\hfill\hfill\hfill}{\ }
%\thispagestyle{myheadings}
\def\leftfootline{\small{\textbf{\thepage}
\hfill ИНФОРМАТИКА И ЕЁ ПРИМЕНЕНИЯ\ \ \ том~1\ \ \ выпуск~2\ \ \ 2007}
}%
 \def\rightfootline{\small{ИНФОРМАТИКА И ЕЁ ПРИМЕНЕНИЯ\ \ \ том~1\ \ \ выпуск~2\ \ \ 2007
 \hfill \textbf{\thepage}}}
 \label{end\stat}

%\def\stat{cont-e}
{%\hrule\par
%\vskip 7pt % 7pt
\raggedleft\Large \bf%\baselineskip=3.2ex
2\,0\,0\,7\ \ A\,U\,T\,H\,O\,R\ \ I\,N\,D\,E\,X \vskip 17pt
    \hrule
    \par
\vskip 21pt plus 6pt minus 3pt }

\label{st\stat}

\def\tit{\ }

\def\aut{\ }
\def\auf{\ }

\def\leftkol{\ } % ENGLISH ABSTRACTS}

\def\rightkol{\ } %ENGLISH ABSTRACTS}

\titele{\tit}{\aut}{\auf}{\leftkol}{\rightkol}


\contentsline {chapter}{\ }{Issue \quad Page} 
\contentsline {subsection}{\textbf{Batrakova D.\,A., Korolev V.\,Yu., Shorgin S.\,Ya.}\ \ A New Method for the Probabilistic and Statistical Analysis of Information Flows in Telecommunication Networks}{\qquad 1 \qquad 40} 
\contentsline {subsection}{\textbf{Borisov A.\,V.}\ \ Bayesian Estimation in\nobreakspace {}Observation Systems with\nobreakspace {}Markov Jump Processes: Game-Theoretic Approach}{\qquad 2 \qquad 65} 
\contentsline {subsection}{\textbf{Bosov A.\,V., Ivanov A.\,V.}\ \ Linguistic Simulation for Machine Translation and Knowledge Management Systems}{\qquad 2 \qquad 50} 
\contentsline {subsection}{\textbf{Chaplygin V.\,V.} see Pechinkin A.\,V.\hfill\hfill\hfill\hfill\hfill\hfill\hfill\hfill\hfill\hfill\hfill\hfill\hfill\hfill\hfill\hfill\hfill\hfill\hfill\hfill\hfill\hfill\hfill\hfill\hfill\hfill\hfill\hfill\hfill\hfill\hfill\hfill\hfill\hfill\hfill}{\ }
\contentsline {subsection}{\textbf{Chaplygin V.\,V.} see Pechinkin A.\,V.\hfill\hfill\hfill\hfill\hfill\hfill\hfill\hfill\hfill\hfill\hfill\hfill\hfill\hfill\hfill\hfill\hfill\hfill\hfill\hfill\hfill\hfill\hfill\hfill\hfill\hfill\hfill\hfill\hfill\hfill\hfill\hfill\hfill\hfill\hfill}{\ }
\contentsline {subsection}{\textbf{Ilyin V.\,D., Sokolov I.\,A.}\ \ The Symbol Model of Informatics Knowledge System in Human-Automaton Environment}{\qquad 1 \qquad 66} 
\contentsline {subsection}{\textbf{Ivanov A.\,V.} see Bosov A.\,V.\hfill\hfill\hfill\hfill\hfill\hfill\hfill\hfill\hfill\hfill\hfill\hfill\hfill\hfill\hfill\hfill\hfill\hfill\hfill\hfill\hfill\hfill\hfill\hfill\hfill\hfill\hfill\hfill\hfill\hfill\hfill\hfill\hfill\hfill\hfill}{\ }
\contentsline {subsection}{\textbf{Kalinichenko L.\,A.} see Zakharov V.\,N.\hfill\hfill\hfill\hfill\hfill\hfill\hfill\hfill\hfill\hfill\hfill\hfill\hfill\hfill\hfill\hfill\hfill\hfill\hfill\hfill\hfill\hfill\hfill\hfill\hfill\hfill\hfill\hfill\hfill\hfill\hfill\hfill\hfill\hfill\hfill}{\ }
\contentsline {subsection}{\textbf{Korolev V.\,Yu.} see Batrakova D.\,A.\hfill\hfill\hfill\hfill\hfill\hfill\hfill\hfill\hfill\hfill\hfill\hfill\hfill\hfill\hfill\hfill\hfill\hfill\hfill\hfill\hfill\hfill\hfill\hfill\hfill\hfill\hfill\hfill\hfill\hfill\hfill\hfill\hfill\hfill\hfill}{\ }
\contentsline {subsection}{\textbf{Kozerenko E.\,B.}\ \ Linguistic Simulation for Machine Translation and Knowledge Management Systems}{\qquad 1 \qquad 54} 
\contentsline {subsection}{\textbf{Kozmidiady V.\,A.} see Zakharov V.\,N.\hfill\hfill\hfill\hfill\hfill\hfill\hfill\hfill\hfill\hfill\hfill\hfill\hfill\hfill\hfill\hfill\hfill\hfill\hfill\hfill\hfill\hfill\hfill\hfill\hfill\hfill\hfill\hfill\hfill\hfill\hfill\hfill\hfill\hfill\hfill}{\ }
\contentsline {subsection}{\textbf{Kudryavtsev A.\,A., Shorgin S.\,Ya.}\ \ Bayesian Approach to Queueing Systems and Reliability Characteristics}{\qquad 2 \qquad 76} 
\contentsline {subsection}{\textbf{Pechinkin A.\,V., Sokolov I.\,A., Chaplygin V.\,V.}\ \ Multichannel Queuing System with Finite Buffer and Unreliable Servers}{\qquad 1 \qquad 27} 
\contentsline {subsection}{\textbf{Pechinkin A.\,V., Sokolov I.\,A., Chaplygin V.\,V.}\ \ Stationary Characteristics of a Multichannel Queueing System with\nobreakspace {}Simultaneous Refusals of Servers}{\qquad 2 \qquad 39} 
\contentsline {subsection}{\textbf{Shorgin S.\,Ya.} see Batrakova D.\,A.\hfill\hfill\hfill\hfill\hfill\hfill\hfill\hfill\hfill\hfill\hfill\hfill\hfill\hfill\hfill\hfill\hfill\hfill\hfill\hfill\hfill\hfill\hfill\hfill\hfill\hfill\hfill\hfill\hfill\hfill\hfill\hfill\hfill\hfill\hfill}{\ }
\contentsline {subsection}{\textbf{Shorgin S.\,Ya.} see Kudryavtsev A.\,A.\hfill\hfill\hfill\hfill\hfill\hfill\hfill\hfill\hfill\hfill\hfill\hfill\hfill\hfill\hfill\hfill\hfill\hfill\hfill\hfill\hfill\hfill\hfill\hfill\hfill\hfill\hfill\hfill\hfill\hfill\hfill\hfill\hfill\hfill\hfill}{\ }
\contentsline {subsection}{\textbf{Sinitsyn I.\,N.}\ \ Correlational Methods for Analytical Informational Models of the Earth Pole Fluctuations Design Based on a priori Data}{\qquad 2 \qquad \hphantom{9}2}
\contentsline {subsection}{\textbf{Sinitsyn I.\,N.}\ \ Development of Pugachev Filtering for Stochastic Systems}{\qquad 1 \qquad \hphantom{9}3}
\contentsline {subsection}{\textbf{Sokolov I.\,A.} see Ilyin V.\,D.\hfill\hfill\hfill\hfill\hfill\hfill\hfill\hfill\hfill\hfill\hfill\hfill\hfill\hfill\hfill\hfill\hfill\hfill\hfill\hfill\hfill\hfill\hfill\hfill\hfill\hfill\hfill\hfill\hfill\hfill\hfill\hfill\hfill\hfill\hfill}{\ }
\contentsline {subsection}{\textbf{Sokolov I.\,A.} see Pechinkin A.\,V.\hfill\hfill\hfill\hfill\hfill\hfill\hfill\hfill\hfill\hfill\hfill\hfill\hfill\hfill\hfill\hfill\hfill\hfill\hfill\hfill\hfill\hfill\hfill\hfill\hfill\hfill\hfill\hfill\hfill\hfill\hfill\hfill\hfill\hfill\hfill}{\ }
\contentsline {subsection}{\textbf{Sokolov I.\,A.} see Pechinkin A.\,V.\hfill\hfill\hfill\hfill\hfill\hfill\hfill\hfill\hfill\hfill\hfill\hfill\hfill\hfill\hfill\hfill\hfill\hfill\hfill\hfill\hfill\hfill\hfill\hfill\hfill\hfill\hfill\hfill\hfill\hfill\hfill\hfill\hfill\hfill\hfill}{\ }
\contentsline {subsection}{\textbf{Sokolov I.\,A.} see Zakharov V.\,N.\hfill\hfill\hfill\hfill\hfill\hfill\hfill\hfill\hfill\hfill\hfill\hfill\hfill\hfill\hfill\hfill\hfill\hfill\hfill\hfill\hfill\hfill\hfill\hfill\hfill\hfill\hfill\hfill\hfill\hfill\hfill\hfill\hfill\hfill\hfill}{\ }
\contentsline {subsection}{\textbf{Stupnikov S.\,A.} see Zakharov V.\,N.\hfill\hfill\hfill\hfill\hfill\hfill\hfill\hfill\hfill\hfill\hfill\hfill\hfill\hfill\hfill\hfill\hfill\hfill\hfill\hfill\hfill\hfill\hfill\hfill\hfill\hfill\hfill\hfill\hfill\hfill\hfill\hfill\hfill\hfill\hfill}{\ }
\contentsline {subsection}{\textbf{Zakharov V.\,N., Kalinichenko L.\,A., Sokolov I.\,A., Stupnikov S.\,A.}\ \ Development of Canonical Information Models for Integrated Information Systems}{\qquad 2 \qquad 15} 
\contentsline {subsection}{\textbf{Zakharov V.\,N., Kozmidiady V.\,A.}\ \ Means Providing Applications Fault Tolerance}{\qquad 1 \qquad 14} 
\def\leftfootline{\small{\textbf{\thepage}
\hfill ИНФОРМАТИКА И ЕЁ ПРИМЕНЕНИЯ\ \ \ том~1\ \ \ выпуск~2\ \ \ 2007}
}%
 \def\rightfootline{\small{ИНФОРМАТИКА И ЕЁ ПРИМЕНЕНИЯ\ \ \ том~1\ \ \ выпуск~2\ \ \ 2007
 \hfill \textbf{\thepage}}}
 \label{end\stat}


%\tableofcontents


\end{document}

\newcommand{\Ack}{\subsection*{\protect\large\bf Acknowledgments}}