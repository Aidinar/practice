\def\stat{stupnikov}

\def\tit{АНАЛИЗ СИСТЕМНОГО РИСКА СОВМЕСТНОГО КРЕДИТОВАНИЯ 
НАД~НЕОДНОРОДНЫМИ КОЛЛЕКЦИЯМИ ДАННЫХ$^*$}

\def\titkol{Анализ системного риска совместного кредитования над 
неоднородными коллекциями данных}

\def\aut{С.\,А.~Ступников$^1$, Д.\,О.~Брюхов$^2$, Н.\,А.~Скворцов$^3$}

\def\autkol{С.\,А.~Ступников, Д.\,О.~Брюхов, Н.\,А.~Скворцов}

\titel{\tit}{\aut}{\autkol}{\titkol}

{\renewcommand{\thefootnote}{\fnsymbol{footnote}} \footnotetext[1]
{Работа выполнена при поддержке РФФИ (проекты 13-07-00579, 14-07-00548 и~16-07-01028).}}



\renewcommand{\thefootnote}{\arabic{footnote}}
\footnotetext[1]{Институт проб\-лем информатики Федерального исследовательского центра 
<<Информатика и~управ\-ле\-ние>> Российской академии наук, sstupnikov@ipiran.ru}
\footnotetext[2]{Институт проб\-лем информатики Федерального исследовательского центра 
<<Информатика и~управ\-ле\-ние>> Российской академии наук,
brd@ipi.ac.ru}
\footnotetext[3]{Институт проб\-лем информатики Федерального исследовательского центра 
<<Информатика и~управ\-ле\-ние>> Российской академии наук,  nskv@ipi.ac.ru}

\vspace*{-8pt}

\Abst{Рассматривается подход к~решению задачи анализа системного 
риска совместного кредитования в~области финансового макромоделирования~--- 
одной из областей с~интенсивным использованием данных~--- над 
неоднородными коллекциями данных в~виртуально-материализованной среде 
интеграции. Виртуальная интеграция в~среде осуществляется с~использованием 
технологии предметных посредников. Материализованная интеграция 
реализуется с~использованием свободно распространяемой платформы 
распределенного хранения и~обработки данных Hadoop, а также системы Hive, 
предназначенной для организации реляционных хранилищ данных над Hadoop.} 

\KW{системный риск совместного кредитования; решение задач; интеграция 
данных; неоднородные коллекции данных}

\DOI{10.14357/19922264160102} %

\vspace*{-4pt}



\vskip 10pt plus 9pt minus 6pt

\thispagestyle{headings}

\begin{multicols}{2}

\label{st\stat}

\section{Введение}

\vspace*{-2pt}

      Рост объема и~разнообразия данных в~науке и~бизнесе в~последние годы 
ведет к~необходи\-мости следования \textit{Четвертой научной парадигме}, 
подчеркивающей роль данных в~исследованиях с~интенсивным использованием 
данных. Эта роль заключается в~том, что новые знания, как и~принятие 
решений, являются результатом анализа данных~[1]. В~различных областях, 
называемых \textit{областями с~интенсивным использованием данных}, 
происходит накопление массивных коллекций разнородных данных, 
представленных в~различных моделях данных.
      
      Спектр используемых моделей данных необычайно широк: он включает 
традиционные реляционные модели, объектные модели, основанные на 
многомерных массивах модели, графовые модели, модели ключ--зна\-че\-ние, 
документные модели, семантические модели (RDF~--- Research Description Framework, OWL~---
Web Ontology Language) и~др. Такое 
разнообразие моделей данных, увеличива-\linebreak ющееся со временем, приводит 
к~необходимости создания подходов к~интеграции моделей и~коллекций 
данных, представленных в~этих моделях, разработ\-ки подходов к~решению задач 
над неоднородными коллекциями. 

В~работе~[2] была предложена архитектура 
комбинированной вир\-ту\-аль\-но-ма\-те\-ри\-а\-ли\-зо\-ван\-ной среды интеграции 
неоднородных коллекций структурированных, слабоструктурированных 
и~неструктурированных данных (рис.~1). Среда поддерживает как 
виртуальную, так и~материализованную интеграцию коллекций данных, 
пред\-став\-лен\-ных как в~традиционных (реляционных), так и~в~нетрадиционных 
моделях данных. 

\begin{figure*}
 \vspace*{1pt}
 \begin{center}
 \mbox{%
 \epsfxsize=154.138mm
 \epsfbox{stu-1.eps}
 }
 \end{center}
 \vspace*{-9pt}
\Caption{Архитектура среды виртуально-материализованной интеграции}
\end{figure*}

      
      Виртуальная интеграция в~среде осуществляется с~использованием 
технологии предметных посредников~[3]. Посредники образуют 
промежуточный слой между пользователем (приложением) и~неоднородными 
информационными ресурсами; данные из ресурсов в~посреднике не 
материализуются. 
      
      При материализованной интеграции предполагается создание 
хранилища данных. В~хранилище загружаются подлежащие интеграции 
коллекции данных, при этом данные преобразуются из схемы коллекции 
в~общую схему хранилища. Материализованная интеграция реализуется 
с~использованием свободно распространяемой платформы распределенного 
хранения и~обработки данных Hadoop~[4], а~также системы Hive~[5], 
предназначенной для организации реляционных хранилищ данных над Hadoop.
      
      В настоящей статье рассматривается подход к~решению задач над 
неоднородными коллекциями данных в~вир\-ту\-аль\-но-ма\-те\-ри\-а\-ли\-зо\-ван\-ной среде 
интеграции. Подход иллюстрируется на примере задачи из области 
финансового макромоделирования~--- одной из областей с~интенсивным 
использованием данных.


      Статья структурирована следующим образом. В~разд.~2 рассмотрена 
постановка задачи анализа системного риска совместного кредитования. 
В~разд.~3 рассмотрены основные решения задачи в~среде интеграции: 
определение концептуальной схемы предметной области задачи 
и~декларативной спецификации задачи (подразд.~3.1 и~3.2), выбор релевантных 
предметной области ресурсов и~способов их интеграции (подразд.~3.3), 
определение схем ресурсов, подлежащих виртуальной интеграции, и~взглядов, 
связывающих их со схемой посредника (подразд.~3.5), определение схемы 
хранилища для материализации коллекций данных и~взглядов, связывающих ее 
со схемой посредника (подразд.~3.6), а~также архитектура среды решения 
задачи (подразд.~3.7).

\vspace*{-4pt}

\section{Интенсивное использование данных в~задаче анализа 
системного риска совместного кредитования}

\vspace*{-2pt}
      
      В данной статье в~качестве примера рас\-смат\-ри\-ва\-ется конкретная задача 
анализа системного риска совместного кредитования. Она состоит в~выявлении 
ведущих игроков на кредитном рынке, банкротство которых может вызвать 
системный финансовый кризис, оказать влияние на финансовое положе\-ние 
множества других игроков~[6, 7]. 
      
      Исходными данными для задачи являются документы (записи) 
о~совместных (синдицированных) кредитах. Записи содержат даты подписания 
и~погашения кредита, объем кредита, состав участников синдиката, 
предоставляющего кредит, и~другую информацию. На основании исходных 
данных формируется граф совместного кредитования. Вершинами графа 
являются организации. Две вершины соединяются ненаправленным ребром, 
если соответствующие организации предоставляют некоторый совместный 
кредит (возможно, вместе с~другими организациями). Так, если кредит 
предоставлен совместно пятью банками, ребрами соединяются всевозможные 
пары, образуемые этими пятью банками. Ребру приписывается вес 
в~зависимости от того, сколько совместных кредитов предоставлено парой 
организаций. Банкротство некоторого банка, очевидно, окажет прямое влияние 
на банки, связанные с~ним ребрами в~графе совместного кредитования, а также 
окажет опосредованное влияние на банки, связанные с~ним в~графе путями~[6].
      
      Определение того, банкротство каких банков имеет наибольший вклад 
      в~риск системного финансового кризиса, базируется на вычислении 
центральности вершин в~графе совместного кредитования. 
      
      Существуют различные варианты централь\-ности вершин в~графе. 
\textit{Степенная центральность} вершины вычисляется как сумма весов 
ребер, связанных с~вершиной. \textit{Центральность по посредничеству} 
(\textit{betweenness centrality})~[8] вычисляется как мера отношения числа 
кратчайших путей, проходящих через вершину, к~общему числу кратчайших 
путей в~графе. \textit{Центральность по собственному значению} 
(\textit{eigenvalue centrality})~[9] вычисляется как мера связи вершины 
с~другими вершинами с~высокой центральностью. Вершины с~наивысшими 
показателями центральности с~высокой вероятностью являются 
\textit{критическими хабами} для сети совместного кредитования.
      
      После того как в~графе совместного кредитования определены вершины 
с наибольшей центральностью, представитель финансового регулятора\linebreak может 
быть заинтересован в~дополнительной информации о соответствующих 
компаниях. К~такой информации относятся связи с~другими компаниями 
(владение, дочерняя компания), ключевые лица компании (директора), 
агрегированные финансовые данные и~т.\,д. 

\vspace*{-4pt}
      
\section{Решение задач в~виртуально-материализованной среде 
интеграции} 

\vspace*{-2pt}

      Решение задачи (класса задач) в~виртуально-ма\-те\-ри\-а\-ли\-зованной среде 
интеграции включает следующие этапы:
      \begin{itemize}
\item определение \textit{концептуальной схемы} предметной области 
задачи; такая схема становится спецификацией предметного посредника для 
виртуальной интеграции релевантных ресурсов;\\[-13pt]
\item \textit{описание задачи} в~виде декларативной программы над 
концептуальной схемой;\\[-13pt] 
\item определение \textit{ресурсов, релевантных предметной области} 
(содержащих данные, необходимые для решения задачи), определение 
способа их интеграции (виртуальная или материализованная);\\[-13pt]
\item создание \textit{отображений моделей данных информационных 
ресурсов}, подлежащих виртуальной интеграции, в~каноническую 
информационную модель предметных посредников;\\[-13pt]
\item создание \textit{адаптеров} информационных ресурсов, подлежащих 
виртуальной интеграции;\\[-13pt]
\item определение \textit{схем информационных ресурсов}, подлежащих 
виртуальной интеграции, и~\textit{отоб\-ра\-же\-ние} этих \textit{схем} 
в~каноническую модель (создание \textit{локальных} схем ресурсов);\\[-13pt]
\item определение \textit{взглядов} (представлений), связывающих элементы 
схемы посредника и~схем ресурсов для обеспечения возможности 
переписывания запросов при виртуальной интеграции ресурсов 
в~посреднике~[3];\\[-13pt]
\item определение \textit{схемы хранилища} для материализации коллекций 
данных и~ее отображение в~каноническую модель;\\[-13pt]
\item определение \textit{взглядов} (представлений), связывающих элементы 
схемы посредника и~схемы хранилища для обеспечения возможности 
переписывания запросов при виртуальной интеграции хранилища 
в~посреднике;\\[-13pt]
\item создание \textit{преобразований коллекций данных}, подлежащих 
материализованной интеграции, в~реляционную модель хранилища;\\[-13pt]
\item создание приложения, связывающего исполнительную среду 
предметных посредников~[3], адаптеры информационных ресурсов 
и~хранилище материализованных данных.
\end{itemize}

      Все эти этапы более подробно будут рассмотрены ниже на примере 
задачи анализа системного риска совместного кредитования.

\vspace*{-4pt}

\subsection{Концептуальная схема предметной области задачи}

\vspace*{-2pt}
      
      Определение концептуальной схемы производится с~использованием 
языка СИНТЕЗ~[10], используемого в~качестве канонической информационной 
модели предметных посредников. 
      
      Концептуальная схема предметной области задачи системного риска 
совместного кредитования представляется в~виде модуля (модуль является 
основной единицей спецификации канонической модели) 
\verb"ColendingSystemicRisk":
\begin{verbatim}
{ ColendingSystemicRisk; in: module;
class_specification: ...
function: ...
}
\end{verbatim}

      Модуль содержит секцию классов (моделирующих множества объектов 
предметной области) и~секцию функций. Секция классов включает, 
в~частности, классы \verb"companies", \verb"persons", \verb"colendings":
    {\small \begin{verbatim}
{ companies; in: class;
  instance_section: {
   id: string;
   names: {set; type_of_element: string;};
   ownedBy: {set; type_of_element: Company;};
   ownerOf: {set; type_of_element: Company;};
   keyPersons: {set; type_of_element: Person;};
}},
{ persons; in: class;
  instance_section: {
   id: string;
   names: {set; type_of_element: string;};
   keyPersonOf: {set; type_of_element: 
                 Company;};
}},
{ colendings; in: class;
  instance_section: {
    id: string;
    colender1: string;
    colender2: string;
    numberOfColendings: integer;
}};
\end{verbatim}
}

\vspace*{-4pt}

      Класс \verb"companies" отвечает компаниям, предо\-став\-ляющим кредиты 
(например, банкам). В~секции описания экземпляров класса 
(\verb"instance_section") определены атрибуты \verb"id" (уникальный 
идентификатор компании), \verb"names" (различные варианты названия 
компании), \verb"ownedBy" (множество компаний, владеющих долей данной 
компании), \verb"ownerOf" (множество компаний, долями которых владеет 
данная компания), \verb"keyPersons" (множество ключевых лиц компании~--- 
директоров, управляющих).
      
      Класс \verb"persons" отвечает лицам, принимающим участие 
в~управлении компаниями. Для экземпляров класса определены атрибуты 
\verb"id" (уникальный идентификатор персоны), names (различные варианты 
имени персоны), \verb"keyPersonOf" (множество компаний, в~которых персона 
занимает управ\-ля\-ющую должность).
      
      Класс \verb"colendings" отвечает отношению совместного кредитования 
между компаниями. Для экземпляров класса определены атрибуты \verb"id" 
(уникальный идентификатор отношения кредитования), \verb"colender1" 
и~\verb"colender2" (идентификаторы пары компаний, участвующих в~выдаче 
совместных кредитов), \verb"numberOfColendings" (количество совместно 
выданных кредитов). Таким образом, совокупность экземпляров класса 
\verb"colendings" задает граф совместного кредитования, вершинами которого 
являются компании. Атрибут \verb"numberOfColendings" задает вес ребра 
в~графе.
      
      Секция функций включает, в~частности, функцию 
\verb"isValidColending":

\vspace*{-3pt}

\noindent
     {\small \begin{verbatim}
{ isValidColending; in: function;
  params: {+rel/integer, +clnd1/integer,
           +clnd2/integer, -returns/boolean};
  predicative: {
  ex c/colendings.inst, cmp1/companies.inst, 
  cmp2/companies.inst(
   is_in(c, colending) & is_in(cmp1, companies) 
    & is_in(cmp2, companies) &
   c.id = rel & cmp1.id = clnd1 & 
    cmp2.id = clnd2 &
   (clnd1 = rel.colender1 & 
   clnd2 = rel.colender2 -> returns = true) &
   (clnd1 <> rel.colender1 | clnd2 <>
    rel.colender2 -> returns = false)  )
};};
\end{verbatim}
}
      
      Функция принимает на вход идентификатор экземпляра класса 
\verb"colendings (rel)", а также два идентификатора экземпляров класса 
\verb"companies" (\verb"clnd1" и~\verb"clnd2"). Предикативная спецификация функции 
задается формулой типизированной логики предикатов первого порядка~[10]. 
Функция возвращает значение \verb"true" в~том случае, если компании 
с~идентификаторами \verb"clnd1" и~\verb"clnd2" являются сокредиторами 
в~отношении совместного кредитования с~идентификатором \verb"rel".

\vspace*{-4pt}
      
\subsection{Описание задачи в~виде декларативной программы}

\vspace*{-2pt}

      Первая часть задачи заключается в~вычислении центральности компаний 
в графе совместного кредитования. Например, для вершинной центральности 
такое вычисление может быть описано в~виде одного Да\-та\-лог-по\-доб\-но\-го 
правила канонической модели~[10]:

\vspace*{-2pt}

\noindent
      {\small
      \begin{verbatim}
degreeCentrality(x/[companyId, 
 sumColendings]) :-
companies(comp/[companyId: id]) & 
 companies(neighbour/[neigbourId:id]) &
colendings(clnd/[colendId:id, 
 numberOfColendings]) &
isValidColending(colendId, companyId, 
 neigbourId) &
group_by(comp) &
sumColendings = sum(colend.numberOfColendings).
\end{verbatim}
}

\vspace*{-2pt}

      Предикат \verb"degreeCentrality" в~голове правила содержит атрибуты 
\verb"companyName" и~\verb"sumColendings", т.\,е.\ для каждой компании 
(вершины) возвращается суммарное число кредитов, выданных ей совмест-\linebreak но 
с~другими компаниями (сумма весов ребер). Перемен\-ная \verb"comp" 
объявляется пробегающей по экземплярам класса \verb"companies" 
с~использованием одноименного пре\-ди\-ка\-та-клас\-са. Атрибут~\verb"id" 
переименовывается в~\verb"companyId" (конструкция \verb"companyId: id") для 
предотвращения смешивания с~одноименными атрибутами других переменных. 
Аналогично объявляется переменная \verb"neighbour". Переменная \verb"clnd" 
пробегает по экземплярам класса \verb"colendings". Пре\-ди\-кат-функция 
\verb"isValidColending"\linebreak\vspace*{-12pt}

\begin{table*}[b]\small
\begin{center}
\Caption{Пример данных о совместном кредите}
\vspace*{2ex}

\begin{tabular}{|l|l|}
\hline
 Кредит&ВТБ, 2.2005\\
 Страна&Россия\\
 Объем&450\,000\,000 USD\\
 Ставка по кредиту&LIBOR\;+\;120.BP\\
 Дата подписания&Февраль 2005\\
 Период&36 месяцев\\
 Дата погашения&Февраль 2008\\
 Уполномоченные ведущие организаторы (MLAs)&ABN Amro, Citigroup и~ING\\
 \hline
 \end{tabular}
 \end{center}
 \end{table*}

\pagebreak

\noindent
 устанавливает связь между значениями \verb"comp", 
\verb"neighbour" и~\verb"colend" через идентификаторы: компания \verb"comp1" 
должна быть связана ребром \verb"colend" с~компанией c\verb"omp2". Операция 
\verb"group_by" производит группировку по имени компании, а функция 
\verb"sum" суммирует веса ребер в~соответствующей группе.
{\looseness=-1

}


      
      Вторая часть задачи заключается в~извлечении дополнительной 
информации о компаниях с~наибольшей центральностью. Пусть, например, 
наивысшей центральностью обладает компания \verb"ING Group". Запрос, 
выдающий идентификаторы всех компаний (\verb"owned"), в~которых 
у~\verb"ING" есть доля, представляется следующим декларативным правилом:

\vspace*{-2pt}

      {\small
      \begin{verbatim}
ownedByING([owned]) :-
 companies(x/[names]) & companies(y/[owned: id]) 
 & is_in('ING', x.names) & is_in(y, x.ownerOf).
\end{verbatim}
}

\vspace*{-2pt}

\noindent
      Здесь встроенный предикат \verb"is_in" означает принадлежность 
элемента (\verb"`ING'") множеству (\verb"x.names").
      
      Запрос, выдающий всех лиц (\verb"pers"), прини\-мающих участие 
в~управлении компанией \verb"ING" одновремен\-но с~управлением некоторой 
другой компанией (\verb"cmpn"), представляется следующим декларативным 
правилом:

\vspace*{-2pt}

      {\small
      \begin{verbatim}
overlappedPersonsOfING([pers, cmpn]) :-
 companies(x) & companies(y/[cmpn:iri]) &
  persons(z/[pers: iri]) &
 is_in('ING', x.names) &
  is_in(z, x.keyPersons) &
  is_in(z, y.keyPersons).
\end{verbatim}
}

\vspace*{-6pt}

\subsection{Релевантные ресурсы и~способы их~интеграции}

\vspace*{-2pt}

      Для решения задачи могут быть использованы, например, следующие 
ресурсы:
      \begin{itemize}
\item база данных на основе графовой системы управ\-ления базами данных (СУБД) Neo4j~[11], содержащая 
информацию о~совместном кредитовании компаний, пред\-став\-лен\-ную в~виде 
графа;\\[-15pt]
\item триплетная (RDF~[12]) база данных DBpedia, содержащая 
структурированную информацию, извлеченную из Википедии (в частности, 
информацию о компаниях и~персонах).
\end{itemize}
      
      Neo4j представляет собой популярную графовую СУБД, 
поддерживающую декларативный язык запросов Cypher~[11]. Естественным 
представляется использование такой СУБД для решения задач на графах и~ее 
виртуальная интеграция в~посреднике. База данных формируется на основе 
информации о~совместных кредитах, публикуемой на сайтах финансовых 
новостных агентств. Пример того, как может выглядеть информация о~кредите, 
приведен в~табл.~1.
      

      
      В этом случае в~графе совместного кредитования появятся вершины, 
соответствующие компаниям \verb"ABN Amro", \verb"Citigroup", \verb"ING", 
и~ребра, попарно соединяющие эти вершины.
      
      Дополнительную информацию о компаниях и~персонах можно извлечь 
из DBpedia. Эта информация представлена в~структурированном виде, удобном 
для материализации в~Hadoop и~преобразования к~реляционному виду.
      
      Доступ к~DBpedia осуществляется посредством запросов на языке 
SPARQL через точку доступа {\sf http://dbpedia.org/sparql}. Например, уникальные 
идентификаторы (URI~--- Uniform Resource Identifier) кредитных организаций могут быть извлечены при 
помощи следующего запроса:
      {\small
      \begin{verbatim}
SELECT DISTINCT ?bank
WHERE { {?bank rdf:type dbo:Bank} UNION 
 {?bank dbp:industry dbr:Bank} }
\end{verbatim}
}
      
      Уникальные идентификаторы (URI) персон могут быть извлечены при 
помощи следующего запроса:
      {\small
      \begin{verbatim}
SELECT DISTINCT ?person
WHERE { {?person rdf:type dbo:Person} UNION
 {?person rdf:type foaf:Person} }
\end{verbatim}
}

      Далее, конкретный RDF-до\-ку\-мент, описывающий банк или персону, 
можно извлечь из \mbox{DBpedia} в~необходимом формате с~использованием 
найденного URI. Так, для компании \textit{BNP Paribas} с~идентификатором 
{\sf http://dbpedia.org/resource/ BNP\_Paribas}, RDF-до\-ку\-мент с~описанием 
в~формате JSON~[13] доступен по ссылке {\sf 
http://dbpedia.org/ data/BNP\_Paribas.json} (приведена лишь часть документа):
      {\small
      \begin{verbatim}
{ "http://dbpedia.org/resource/ING_Group" : { 
  "http://dbpedia.org/ontology/industry" : [ 
   {"type": "uri", "value": "http://dbpedia.org/
    resource/Financial_services"},
   {"type" : "uri", "value" : "http://dbpedia.
    org/resource/Bank" } ],
  "http://dbpedia.org/property/assets" : 
   [ { "type" : "literal", 
      "value" : "1.169E12" , "datatype" : 
      "http://dbpedia.org/datatype/euro" } ] ,
  "http://dbpedia.org/property/keyPeople" : 
   [ { "type" : "literal", 
   "value": "Ralph Hamers Patrick Flynn Jeroen
    van der Veer", "lang": "en" }],
  "http://dbpedia.org/resource/Bank_Mendes_
    Gans" : {
   "http://dbpedia.org/ontology/parentCompany" : 
    [{ "type" : "uri", "value" : 
       "http://dbpedia.org/resource/
        ING_Group" } ] },
  "http://dbpedia.org/resource/
   Voya_Financial" : {
   "http://dbpedia.org/ontology/parentCompany" : 
   [ { "type" : "uri", "value" : "http://
        dbpedia.org/resource/ING_Group" } ] }
}
\end{verbatim}
}

      В документе содержится, в~частности, информация о ключевых лицах 
компании (\verb"keyPeople"), объеме имущества (\verb"assets"), дочерних 
компаниях (связь, обратная \verb"parentCompany") и~т.\,д.
      
      Таким образом, все необходимые дополнительные данные о~компаниях 
извлекаются из DBpedia и~загружаются в~Hadoop в~виде RDF-до\-ку\-ментов.

\subsection{Отображение моделей данных информационных ресурсов 
в~каноническую модель и~создание~адаптеров}
      
      Необходимым предусловием виртуальной ин\-теграции модельно 
однородного класса ин\-фор\-мационных ресурсов (представленных 
с~использованием одной модели данных) в~предметных\linebreak посредниках является 
\textit{унификация модели данных ресурсов}~--- ее отображение 
в~каноническую информационную модель (служащую общим языком в~среде 
разнообразных моделей ресурсов), сохраняющее информацию и~семантику 
операций языка манипулирования данными, а~также разработка адаптера для 
сопряжения ресурсов данного класса со средой исполнения предметных 
посредников. 
      
      Основные принципы отображения модели данных атрибутированных 
графов и~вопросы доказательства сохранения информации и~семантики\linebreak 
операций при этом отображении рассмотрены в~работе~[14]. Общие принципы 
построения адаптеров изложены в~работе~[15].
      
      Для реализации конкретного адаптера СУБД Neo4j необходима 
разработка трансформации запросов (программ) канонической модели 
в~запросы на языке Cypher, основывающейся на упомянутом отображении 
моделей. Вопросы построения такой трансформации будут вынесены 
в~отдельную \mbox{статью}. 
      
\subsection{Схема ресурса, подлежащего виртуальной интеграции, 
и~взгляды,~связывающие ее со~схемой~посредника}

      Схемой ресурса, подлежащего виртуальной интеграции~--- 
СУБД Neo4j, содержащей информацию о совместном кредитовании компаний, 
можно считать шаблоны операций языка Cypher, порожда\-ющие экземпляры 
базы данных. Например, операция вида
      \begin{verbatim}
merge (c:Organization{id: URI});
\end{verbatim}

\noindent
создает в~базе данных вершину с~меткой \verb"Organization" и~атрибутом 
\verb"id".

      Следующая операция:
      {\small
      \begin{verbatim}
match (c1: Organization{id: URI1}),
 (c2: Organization{id: URI2})
merge (n1)-[e:colends]-(n2)
on create set e.numberOfColendings = 1
on match set e.numberOfColendings = 
 e.numberOfColendings + 1;
\end{verbatim}
}

\noindent
создает в~базе данных ребро с~меткой \verb"colends" между двумя вершинами, 
помеченными как \verb"Organization", и~устанавливает значение атрибута 
\verb"numberOfColendings", равным~1 (либо увеличивает значение на~1, если 
ребро уже существует в~базе).

      Из этих операций можно заключить, что в~базе данных имеются 
вершины типа \verb"Organization" с~атрибутом \verb"id" и~их соединяют ребра 
типа \verb"colends" с~атрибутом \verb"numberOfColendings".
      
      Согласно принципам отображения модели данных атрибутированных 
графов в~каноническую модель~[14] схема базы данных, порождаемой 
приведенными выше операциями, представляется в~канониче\-ской модели 
сле\-ду\-ющим образом (приведено упрощенное подмножество соответст\-ву\-ющей 
спецификации):
      {\small\begin{verbatim}
{ PropertyGraph; in: module;
class_specification: 
{ vertices; in: class; 
  instance_section: {
   id: string;
}},
{ edges; in: class;
  instance_section: {
   id: string;
   startVertex: string;
   endVertex: string;
}};
function:
isValidEdge: { in: predicate; 
  params: {+edg/string, +stVtx/string,
   +endVtx/string, returns/boolean};
}; 
}
{ Colending; in: module; import: PropertyGraph;
{ Organization; in: class; superclass: vertices;
  instance_section: {
   id: string;
};},
{ colends; in: class; superclass: edges;
  instance_section: {
   numberOfColendings: integer;
};};
}
\end{verbatim}
}

     \noindent
      Здесь модуль \verb"PropertyGraph", содержащий классы \verb"vertices" 
(вершины) и~\verb"edges" (ребра), задает граф общего вида~[14], а~модуль 
\verb"Colending", содержащий классы \verb"Organization" и~\verb"colends", 
задает атрибуты вершин и~ребер конкретных типов. Вышеприведенная схема, 
представленная в~канонической модели, называется \textit{локальной схемой} 
ресурса (базы данных о совместном кредитовании компаний).
      
      Для обеспечения возможности переписывания запросов при 
виртуальной интеграции ресурсов в~посреднике необходимо определение 
взглядов, связывающих элементы схемы посредника и~схем ресурсов~[16, 17]. 
Взгляды представляются Да\-та\-лог-по\-доб\-ны\-ми декларативными правилами 
канонической модели. Для сопоставления вышеприведенных фрагментов схем 
достаточно трех следующих взглядов вида LAV (Local As View), 
определяющих, как элемент (класс или функция) локальной схемы выражается 
через элементы схемы посредника:
      {\small\begin{verbatim}
Colending.organization(x/[id]):-
 companies(x/[id]).

Colending.colends(x/[id, colender1, 
 colender2, numberOfColendings]):-
colendings(x/[id, colender1: startVertex,
           colender2: endVertex,
             numberOfColendings]).

Colending.isValidEdge(colend, comp1, comp2):-
isValidColending(colend, comp1, comp2).
\end{verbatim}
}

      Первый взгляд задает выражение класса \verb"organization" через класс 
\verb"companies", второй~--- класса \verb"colends" через класс \verb"colendings", 
третий~--- функции \verb"isValidEdge" через функцию \verb"isValidColending".
      
      Применение алгоритма переписывания запросов~[16] с~использованием 
вышеприведенных взглядов к~правилу вычисления вершинной централь\-ности 
(см.\ разд.~3.2) позволяет получить запрос в~терминах локальной схемы:

\vspace*{-2pt}

      \noindent
      {\small\begin{verbatim}
degreeCentrality(x/[companyId, 
 sumColendings]) :-
organization(comp/[companyId: id]) & 
 organization(neighbour/[neigbourId:id]) &
colends(clnd/[colendId:id, 
 numberOfColendings]) &
isValidEdge(colendId, companyId, neigbourId) &
group_by(comp) &
sumColendings = sum(colend.numberOfColendings).
\end{verbatim}
}

\vspace*{-2pt}

      Согласно принципам отображения языка правил канонической модели 
в~язык Cypher~[14] адап\-тер Neo4j должен преобразовывать такое правило 
в~следующий запрос на языке Cypher:

\vspace*{-2pt}

  \noindent
     {\small \begin{verbatim}
match(comp: Organization)-[clnd:
 colends]-(neighbour: Organization)
return comp.id as companyId, 
 sum(clnd.numberOfColendings) as sumColendings
\end{verbatim}
}

\vspace*{-4pt}

\subsection{Схема хранилища для~материализации коллекций 
данных~и~взгляды,~связывающие~ее со~схемой~посредника}

\vspace*{-2pt}
      
      Фрагмент реляционной схемы для представления информации об 
именах банков (отношение \verb"banks") и~персон (отношение \verb"persons"), 
отношении владения между компаниями (\verb"bankOwners"), ключевых лицах 
в~компаниях (отношение \verb"keyPersons") выглядит следующим образом:
{\small      \begin{verbatim}
banks(iri STRING, name STRING)
persons(iri STRING, name STRING)
bankOwners(owner STRING, owned STRING)
bankKeyPersons(person_iri STRING,
 bank_iri STRING)
\end{verbatim}
}


      Локальная схема в~канонической модели, соответствующая данной 
реляционной схеме, выглядит следующим образом:
%\vspace*{-2pt}
      \begin{verbatim}
{ BanksPersons; in: module;
class_specification:
{ banks; in: class;
  instance_section:{
    iri: string;
    name: string;
}},
{ persons; in: class;
  instance_section:{
    iri: string;
    name: string;
}},
\end{verbatim}



\noindent
\begin{verbatim}
{ bankOwners; in: class;
  instance_section:{
    owner: string;
    owned: string;
}},
{ keyPersons; in: class;
  instance_section:{
    person_iri: string;
    bank_iri: string;
}};
}
\end{verbatim}



      Схема представляется модулем \verb"BanksPersons", каждому 
отношению реляционной схемы соответствует одноименный класс модуля, 
каждому атрибуту отношения~--- одноименный атрибут типа экземпляров 
соответствующего класса.
      
      Вопросы построения преобразования RDF-кол\-лек\-ций компаний 
и~персон в~реляционное представление при помощи языка Jaql~[18] в~Hadoop, 
необходимого для загрузки данных в~реляционное хранилище Hive, будут 
рассмотрены в~отдельной статье.
      
      После того как данные загружены в~Hive (осуществлена 
материализованная интеграция), необходимо осуществить виртуальную 
интеграцию хранилища в~посреднике в~соответствии с~подходом, изложенным 
в~[2]. Для этого, как и~в случае с~интеграцией СУБД Neo4j, требуется 
установить соответствие между элементами схемы хранилища и~схемы 
посредника при помощи взглядов. LAV-взгля\-ды для приведенных фрагментов 
схем выглядят следующим образом:
{\small      \begin{verbatim}
BanksPersons.banks(b/[iri, name]) :- 
companies(c/[iri, names]) & is_in(name, names).

BanksPersons.people(p/[iri, name]) :- 
persons(p/[iri, names]) & is_in(name, names).

BanksPersons.bankOwners(bo/[owner, owned]) :- 
companies(x/[owner: iri]) & 
 companies(y/[owned: iri]) & 
 is_in(y, x.ownerOf).

BanksPersons.bankKeyPersons(kp/[person_iri,
 bank_iri]) :- 
persons(p/[person_iri: iri]) &
 companies(b/[bank_iri: iri]) & 
is_in(p, b.keyPersons).
\end{verbatim}
}

    \noindent
      Здесь встроенный предикат \verb"is_in(x, y)" обозначает принадлежность 
элемента \verb"x" множеству \verb"y". Первый взгляд выражает класс 
\verb"banks" схемы хранилища через класс \verb"companies" схемы посредника, 
\mbox{второй}~--- класс \verb"people" через класс \verb"persons", третий~--- класс 
bankOwners через класс \verb"companies", четвертый~--- класс 
\verb"bankKeyPersons" через классы \verb"persons" и~\verb"companies".
      
      Применение алгоритма переписывания запросов~[16] с~использованием 
вышеприведенных взглядов к~правилу вычисления лиц с~конфликтом интересов 
в компании \verb"ING" (см.\ разд.~3.2) позволяет получить запрос в~терминах 
локальной схемы:
      \begin{verbatim}
ownedByING([owned]) :-
 banks(x/[iri, name]) & name = 'ING' &
  bankOwners(y/[iri, owned]).
\end{verbatim}

   \noindent
      Адаптер Hive должен преобразовывать такое правило в~следующий 
запрос на языке HiveQL:
      {\small\begin{verbatim}
SELECT owned
FROM banks x JOIN bankOwners y ON x.iri = y.iri
WHERE x.name like "ING";
\end{verbatim}
}

\subsection{Архитектура среды решения задачи анализа системного риска 
совместного кредитования}
      
      Архитектура среды решения задачи представлена на рис.~2.


      Среда включает СУБД Neo4j; RDF-базу данных DBpedia; хранилище на 
основе Hive над Hadoop; а также приложение, связывающее среду исполнения 
предметных посредников и~адаптеры ресурсов.
      
      Для решения задачи анализа системного риска совместного 
кредитования в~СУБД Neo4j были загружены данные о~1500~совместных 
кредитах, выданных компаниям России и~стран ближнего зарубежья. Данные 
были получены с~веб-стра\-ниц сайта одного из финансовых новостных 
агентств. Полученный граф совместного кредитования включает 
около~400~вершин (компаний) и~4500~ребер.
      
      Также из базы данных DBpedia были извлечены RDF-до\-ку\-менты 
о~финансовых организациях и~персонах, принимающих участие в~управ\-ле\-нии 
такими организациями. RDF-до\-ку\-мен\-ты были преобразованы 
к~реляционному виду и~помещены в~хранилище Hive.
      
      Пример данных, полученных в~результате решения задачи 
в~комбинированной среде интеграции, приведен в~табл.~2. Были обнаружены 
четыре компании, нормализованная вершинная центральность которых 
превышает~0,5 (кандидаты в~критические хабы). На основании данных, 
извлеченных из DBpedia, были найдены компании, владельцами или 
совладельцами которых являются критические хабы (приведена лишь часть 
данных).
      


\vspace*{-4pt}

\section{Заключение}

\vspace*{-2pt}

      В статье рассмотрен подход к~решению задачи анализа системного 
риска совместного кредитования в~области финансового макромоделирования 
над неоднородными коллекциями данных\linebreak\vspace*{-12pt}

\pagebreak

\end{multicols}

\begin{figure} %fig2
\vspace*{1pt}
 \begin{center}
 \mbox{%
 \epsfxsize=148.304mm
 \epsfbox{stu-2.eps}
 }
 \end{center}
 \vspace*{-9pt}
\Caption{Архитектура среды решения задачи анализа системного риска совместного 
кредитования}
\end{figure}

\begin{table}\small
\begin{center}
\parbox{270pt}{\Caption{Пример результата решения анализа задачи риска совместного кредитования}
}

\vspace*{2ex}

\begin{tabular}{|l|c|l|}
\hline
\multicolumn{1}{|c|}{Название}&\tabcolsep=0pt\begin{tabular}{c}Нормализованная\\
вершинная\\ центральность\end{tabular}&\multicolumn{1}{c|}
{\tabcolsep=0pt\begin{tabular}{c}Владение другими\\ компаниями\end{tabular}}\\
\hline
ING Group&1&\tabcolsep=0pt\begin{tabular}{l} Bank Mendes Gans\\
Voya Financial\end{tabular}\\
\hline
UniCredit&0,608&\tabcolsep=0pt\begin{tabular}{l}Bank Austria\\
Pioneer Investments\end{tabular}\\
\hline
HSBC Bank&0,581&\tabcolsep=0pt\begin{tabular}{l}HSBC Bank Canada\\
HSBC Bank Australia\end{tabular}\\
\hline
BNP Paribas&0,504&\tabcolsep=0pt\begin{tabular}{l}SBI Life Insurance Company\\
Bank Insinger de Beaufort\end{tabular}\\
\hline
\end{tabular}
\end{center}
\vspace*{-6pt}
\end{table}


\begin{multicols}{2}

\noindent
 в~вир\-ту\-аль\-но-ма\-те\-риализованной 
среде интеграции. Основные этапы решения иллюстрируются на примере 
задачи анализа системного риска совместного кредитования. 

В~статье остались 
нераскрытыми следующие важные вопросы: создание трансформации графовой 
модели данных Neo4j в~каноническую информационную модель и~построение 
адап\-те\-ра Neo4j, а~также создание преобразования коллекции  
RDF-до\-ку\-мен\-тов из DBpedia в~реляционные данные, пригодные для 
загрузки в~хранилище Hive над Hadoop. Эти вопросы станут предметом 
отдельной статьи.

\vspace*{-6pt}


{\small\frenchspacing
 {%\baselineskip=10.8pt
 \addcontentsline{toc}{section}{References}
 \begin{thebibliography}{99}
 
 \vspace*{-2pt}
 
\bibitem{1-stu}
\Au{Hey T., Tansley S., Tolle K.} The fourth paradigm~--- data intensive scientific discovery. 
2009. {\sf http://goo.gl/edvr6W}.
\bibitem{2-stu}
\Au{Ступников С.\,А., Вовченко А.\,Е.} Комбинированная вир\-ту\-аль\-но-материализованная 
среда интеграции больших неоднородных коллекций данных~// Электронные библиотеки: 
перспективные методы и~технологии, электронные коллекции (RCDL 2014): Тр. 16-й 
Всеросс. науч. конф.~--- Дубна: ОИЯИ, 2014. С.~339--348.
\bibitem{3-stu}
\Au{Брюхов Д.\,О., Вовченко А.\,Е., Захаров~В.\,Н., Желенкова~О.\,П., Калиниченко~Л.\,А., 
Мартынов~Д.\,О., Скворцов~Н.\,А., Ступников~С.\,А.} Архитектура промежуточного слоя 
предметных посредников для решения\linebreak задач над множеством интегрируемых 
неоднородных распределенных информационных ресурсов в~гиб\-рид\-ной  
грид-ин\-фра\-струк\-ту\-ре виртуальных обсерваторий~// Информатика и~её применения, 
2008. Т.~2. Вып.~1. С.~2--34. 
\bibitem{4-stu}
Apache Hadoop project. 2014. {\sf http://hadoop.apache. org}.
\bibitem{5-stu}
\Au{Capriolo E., Wampler D., Rutherglen~J.} Programming Hive: Data warehouse and query 
language for Hadoop.~--- O'Reilly Media, 2012. 350~p.
\bibitem{6-stu}
\Au{Burdick D., Hern$\acute{\mbox{a}}$ndez M.\,A., Ho~H., Koutrika~G., Krish\-na\-murthy~R., 
Popa~L., Stanoi~I., Vaithyanathan~S., Das~S.\,R.} Extracting, linking and integrating data from 
public sources: A~financial case study~// IEEE Data Eng. Bull., 2011. Vol.~34. No.\,3.  
P.~60--67.
\bibitem{7-stu}
\Au{Burdick D., Evfimievski A., Krishnamurthy~R., Lewis~N., Popa~L., Rickards~S., Williams~P.} 
Financial analytics from public data~// SIGMOD/PODS'2014: Workshop on Data Science for 
Macro-Modeling Conference (International) Proceedings.~--- New York, NY, USA: ACM, 2014. 
P.~1--6.
\bibitem{8-stu}
\Au{Freeman L.\,C.} A~set of measures of centrality based on betweenness~// Sociometry, 1977. 
Vol.~40. No.\,1. P.~35--41.
\bibitem{9-stu}
\Au{Bonacich P.} Power and centrality: A~family of measures~// Am. J.~Sociol., 1987. 
Vol.~92. No.\,5. P.~1170--1182.
\bibitem{10-stu}
\Au{Kalinichenko L.\,A., Stupnikov S.\,A., Martynov~D.\,O.} \mbox{SYNTHESIS}: A~language for 
canonical information modeling and mediator definition for problem solving in heterogeneous 
information resource environments.~--- Moscow: IPI RAN, 2007. 171~p.
\bibitem{11-stu}
The Neo4j manual. 2014. {\sf http://goo.gl/cHiOGF}. 
\bibitem{12-stu}
RDF~1.1 concepts and abstract syntax~/
Eds. R.~Cyga\-niak, D.~Wood, M.~Lanthaler.~---  W3C Recommendation, February~25, 2014. {\sf 
http://www.w3.org/TR/2014/REC-rdf11-concepts-20140225}. 
\bibitem{13-stu}
Introducing JSON. 2014. {\sf http://www.json.org}.
\bibitem{14-stu}
\Au{Ступников С.\,А.} Отображение графовых моделей данных в~каноническую модель 
в~системах с~интенсивным использованием данных~// Системы высокой доступности, 
2014. Вып.~2. С.~13--31.
\bibitem{15-stu}
\Au{Вовченко А.\,Е.} Рассредоточенная реализация приложений в~среде предметных 
посредников: Дис.\ \ldots\ канд. техн. наук.~--- М.: ИПИ РАН, 2012. 216~с.
\bibitem{16-stu}
\Au{Kalinichenko L.\,A., Martynov~D.\,O., Stupnikov~S.\,A.} Query rewriting using views in a 
typed mediator environment~// Advances in databases and information systems~/
Eds.\ G.~Gottlob, A.~Benczur, J.~Demetrovics.~---
 Lecture notes in computer science ser.~---  
Berlin--Heidelberg: Springer-Verlag, 2004. Vol.~3255.  P.~37--53.
\bibitem{17-stu}
\Au{Briukhov D.\,O., Kalinichenko~L.\,A., Martynov~D.\,O.} Source registration and query 
rewriting applying LAV/GLAV techniques in a~typed subject mediator~// RCDL'2007:  9th 
Russian Conference on Digital Libraries Proceedings.~--- Pereslavl'-Zalesskiy: Pereslavl' 
University, 2007. P.~253--262.
\bibitem{18-stu}
\Au{Beyer K.\,S., Ercegovac V., Gemulla~R., Balmin~A., Eltabakh~M., Kanne~C.-Ch., Ozcan~F., 
Shekita~E.\,J.} Jaql: A~scripting language for large scale semistructured data analysis~//  VLDB 
Endowment Proceedings, 2011. Vol.~4. No.\,12. P.~1272--1283.
\end{thebibliography}

 }
 }

\end{multicols}

\vspace*{-12pt}

\hfill{\small\textit{Поступила в~редакцию 17.11.15}}

\vspace*{4pt}

%\newpage

%\vspace*{-24pt}

\hrule

\vspace*{2pt}

\hrule

\vspace*{-4pt}



\def\tit{CO-LENDING SYSTEMIC RISK ANALYSIS OVER~HETEROGENEOUS DATA COLLECTIONS}

\def\titkol{Co-lending systemic risk analysis over heterogeneous data collections}

\def\aut{S.\,A.~Stupnikov, D.\,O.~Briukhov, and N.\,A.~Skvortsov}

\def\autkol{S.\,A.~Stupnikov, D.\,O.~Briukhov, and N.\,A.~Skvortsov}

\titel{\tit}{\aut}{\autkol}{\titkol}

\vspace*{-11pt}


\noindent
Institute of Informatics Problems, Federal Research Center 
``Computer Science and Control'' of the Russian Academy of Sciences,
44-2 Vavilov Str., Moscow 119333, Russian Federation


\def\leftfootline{\small{\textbf{\thepage}
\hfill INFORMATIKA I EE PRIMENENIYA~--- INFORMATICS AND
APPLICATIONS\ \ \ 2016\ \ \ volume~10\ \ \ issue\ 1}
}%
 \def\rightfootline{\small{INFORMATIKA I EE PRIMENENIYA~---
INFORMATICS AND APPLICATIONS\ \ \ 2016\ \ \ volume~10\ \ \ issue\ 1
\hfill \textbf{\thepage}}}

\vspace*{2pt}

 \Abste{The paper considers an approach to solving the problem of co-lending systemic risk analysis 
over heterogeneous data collections in a combined virtual and materialized integration 
environment. The problem belongs to the data-intensive domain of financial macromodeling. 
Virtual integration is implemented using the subject mediation technology. Materialized 
integration is implemented using the Hadoop open-source software framework for distributed 
storage and processing of large datasets accompanied by the Hive system intended for relational 
warehousing over Hadoop.}


\KWE{co-lending systemic risk; problem solving; data integration; heterogeneous data collections}

\vspace*{-2pt}



\DOI{10.14357/19922264160102}

\vspace*{-12pt}

\Ack
\noindent
The work was supported by the Russian Foundation for Basic Research
(projects 13-07-00579, 14-07-00548, and 16-07-01028).



%\vspace*{3pt}

  \begin{multicols}{2}

\renewcommand{\bibname}{\protect\rmfamily References}
%\renewcommand{\bibname}{\large\protect\rm References}

{\small\frenchspacing
 {%\baselineskip=10.8pt
 \addcontentsline{toc}{section}{References}
 \begin{thebibliography}{99}
\bibitem{1-stu-1}
\Aue{Hey, T., S. Tansley, and K.~Tolle}. 2009. The fourth paradigm~--- data intensive scientific 
discovery. Available at: {\sf  http://goo.gl/edvr6W} (accessed November~17, 2015).
\bibitem{2-stu-1}
\Aue{Stupnikov, S.\,A., and A.\,E.~Vovchenko}. 2014. Kombinirovannaya  
virtual'no-materializovannaya sreda integratsii bol'shikh neodnorodnykh kollektsiy dannykh 
[Combined virtual and materialized environment for integration of large heterogeneous data 
collections]. \textit{Tr. 16-y Vseross. Konferentsii ``Elektronnye Biblioteki: Perspektivnye 
Metody i~Tekhnologii, Elektronnye Kollektsii''} [16th Russian Conference on Digital Libraries 
Proceedings, CEUR Workshop Proceedings]. 1297:339--348.
\bibitem{3-stu-1}
\Aue{Briukhov, D.\,O., A.\,E. Vovchenko, V.\,N.~Zakharov, O.\,P.~Zhelenkova, 
L.\,A.~Kalinichenko, D.\,O.~Martynov, N.\,A.~Skvortsov, and S.\,A.~Stupnikov}. 2008. 
Arhitektura promezhutochnogo sloya predmetnykh posrednikov dlya resheniya zadach nad 
mnozhestvom integri\-ru\-emykh neodnorodnykh raspredelennykh informatsionnykh resursov 
v~gibridnoy grid-infrastrukture virtual'nykh observatoriy [The middleware architecture of the 
subject mediators for problem solving over a~set of integrated heterogeneous distributed 
information resources in the hybrid grid-infrastucture of virtual observatories]. 
\textit{Informatika i~ee Primeneniya}~--- \textit{Inform. Appl.} 2(1):2--34.
\bibitem{4-stu-1}
Apache Hadoop project. 2015. Available at: {\sf http:// hadoop.apache.org} (accessed 
November~17, 2015).
\bibitem{5-stu-1}
\Aue{Capriolo, E., D. Wampler, and J.~Rutherglen}. 2012. \textit{Programming Hive: Data 
warehouse and query language for Hadoop}. O'Reilly Media. 350~p.
\bibitem{6-stu-1}
\Aue{Burdick, D., M.\,A.~Hern$\acute{\mbox{a}}$ndez, H.~Ho, G.~Koutrika, R.~Krishnamurthy, 
L.~Popa, I.~Stanoi, S.~Vaithyanathan, and S.\,R.~Das}. 2011. Extracting, linking and integrating 
data from public sources: A~financial case study. \textit{IEEE Data Eng. Bull.} 34(3):60--67.
\bibitem{7-stu-1}
\Aue{Burdick, D., A. Evfimievski, R.~Krishnamurthy, N.~Lewis, L.~Popa, S.~Rickards, and 
P.~Williams}. 2014. Financial analytics from public data. \textit{Workshop on Data Science for 
Macro-Modeling, SIGMOD/PODS'2014 Conference (International) Proceedings}. 1--6.
\bibitem{8-stu-1}
\Aue{Freeman, L.\,C.} 1977. A~set of measures of centrality based on betweenness. 
\textit{Sociometry} 40(1):35--41.
\bibitem{9-stu-1}
\Aue{Bonacich, P.} 1987. Power and centrality: A~family of measures. \textit{Am. 
J.~Sociol.} 92(5):1170--1182.
\bibitem{10-stu-1}
\Aue{Kalinichenko, L.\,A., S.\,A.~Stupnikov, and D.\,O.~Martynov}. 2007. \textit{SYNTHESIS: 
A~language for canonical information modeling and mediator definition for problem solving in 
heterogeneous information resource environments}. Moscow: IPI RAN. 171~p.
\bibitem{11-stu-1}
The Neo4j manual. 2015. Available at: {\sf http://goo.gl/ cHiOGF} (accessed November~17, 2015).
\bibitem{12-stu-1}
Cyganiak, R., D. Wood, and M.~Lanthaler, eds. 
February~25, 2014.
RDF~1.1 concepts and abstract syntax. 
W3C Recommendation.  Available at: {\sf  
http://www.\linebreak w3.org/TR/2014/REC-rdf11-concepts-20140225} (accessed November~17, 2015).
\bibitem{13-stu-1}
Introducing JSON. 2015. Available at: {\sf http://www. json.org/} (accessed November~17, 2015).
\bibitem{14-stu-1}
\Aue{Stupnikov, S.\,A.} 2014. Otobrazhenie grafovykh modeley dannykh v~kanonicheskuyu 
model' v sistemakh s intensivnym ispol'zovaniem dannykh [Mapping of graph data models into 
a~canonical model for the development of data intensive systems]. \textit{Sistemy Vysokoy 
Dostupnosti} [Systems of High Availability] 2:13--31.
\bibitem{15-stu-1}
\Aue{Vovchenko, A.\,E.} 2012. Rassredotochennaya realizatsiya prilozheniy v~srede predmetnykh 
posrednikov [Distributed implementation of the applications in the subject mediation 
environbment].  Moscow: IPI RAN. PhD Thesis. 216~p.
\bibitem{16-stu-1}
\Aue{Kalinichenko, L.\,A., D.\,O.~Martynov, and S.\,A.~Stupnikov.} 2004. 
Query rewriting using 
views in a typed mediator environment.  
\textit{Advances in databases and information systems}.
Eds.\ G.~Gottlob,  A.~Benczur, and J.~Demetrovics.
 Lecture notes in computer science ser.~---  
Berlin--Heidelberg: Springer-Verlag. 3255:37--53.

\bibitem{17-stu-1}
\Aue{Briukhov, D.\,O., L.\,A.~Kalinichenko, and D.\,O.~Martynov}. 2007. Source registration and 
query rewriting applying LAV/GLAV techniques in a~typed subject mediator. \textit{9th Russian 
Conference on Digital Libraries RCDL'2007 Proceedings}. Pereslavl'-Zalesskiy:  
Pereslavl University. 253--262.
\bibitem{18-stu-1}
\Aue{Beyer, K.\,S., V. Er\-ce\-govac, R.~Gemulla, A.~Balmin, M.~El\-ta\-bakh, C.-Ch.~Kanne, 
F.~Ozcan, and E.\,J.~Shekita}. 2011. Jaql: A~scripting language for large scale semistructured 
data analysis. \textit{VLDB Endowment Proceedings} 4(12):1272--1283.
\end{thebibliography}

 }
 }

\end{multicols}

\vspace*{-6pt}

\hfill{\small\textit{Received November 17, 2015}}

\vspace*{-12pt}
      
      \Contr
      
      
       \noindent
       \textbf{Stupnikov Sergey A.}\ (b.\ 1978)~--- Candidate of Science (PhD) in technology, 
senior scientist, Institute of Informatics Problems, Federal Research Center ``Computer Science and 
Control'' of the Russian Academy of Sciences, 44-2 Vavilov Str., Moscow 119333, Russian 
Federation; sstupnikov@ipi.ac.ru
       
       \vspace*{3pt}
       
       \noindent
       \textbf{Briukhov Dmitry O.}\ (b.\ 1971)~--- Candidate of Science (PhD) in technology, 
senior scientist, Institute of Informatics Problems, Federal Research Center ``Computer Science and 
Control'' of the Russian Academy of Sciences, 44-2 Vavilov Str., Moscow 119333, Russian Federation; 
brd@ipi.ac.ru 
       
       \vspace*{3pt}
       
       \noindent
       \textbf{Skvortsov Nikolay A.}\ (b.\ 1973)~---
       scientist, Institute of Informatics Problems, Federal Research Center ``Computer Science 
and Control'' of the Russian Academy of Sciences, 44-2 Vavilov Str., Moscow 119333, Russian 
Federation; nskv@ipi.ac.ru 
      
\label{end\stat}


\renewcommand{\bibname}{\protect\rm Литература}