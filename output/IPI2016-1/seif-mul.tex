\renewcommand{\figurename}{\protect\bf Figure}
\renewcommand{\tablename}{\protect\bf Table}

\def\stat{self-mul}


\def\tit{COMPLEXITY AND~ITS~INFORMATION CONTENT}

\def\titkol{Complexity and its information content}

\def\autkol{N.~Callaos and R.~Seyful-Mulyukov}

\def\aut{N.~Callaos$^1$ and R.~Seyful-Mulyukov$^2$}

\titel{\tit}{\aut}{\autkol}{\titkol}

%{\renewcommand{\thefootnote}{\fnsymbol{footnote}}
%\footnotetext[1] {This work was supported in part by the
%Russian Foundation for Basic Research (grants 15-07-03007 and 13-07-00223).}}

\renewcommand{\thefootnote}{\arabic{footnote}}
\footnotetext[1]{International Institute of Systemic, Cybernetics and Informatics,  
USA-Venezuela, 2206 Tillman Av., Winter Garden, FL 34787, USA}
\footnotetext[2]{Institute of Informatics Problems, Federal Research Center 
``Computer Science and Control'' of the Russian Academy of 
Sciences, 44-2~Vavilov Str.,  Moscow 119333, Russian Federation}


\vspace*{12pt}

\def\leftfootline{\small{\textbf{\thepage}
\hfill INFORMATIKA I EE PRIMENENIYA~--- INFORMATICS AND APPLICATIONS\ \ \ 2016\ \ \ volume~10\ \ \ issue\ 1}
}%
 \def\rightfootline{\small{INFORMATIKA I EE PRIMENENIYA~--- INFORMATICS AND APPLICATIONS\ \ \ 2016\ \ \ volume~10\ \ \ issue\ 1
\hfill \textbf{\thepage}}}


\Abste{The word `information' has been used in many senses and its related concepts have been 
defined in different ways. One of the senses in which the word is used relates to 
a~concept which is considered one of the main properties of matter. The definition of this 
conception of information supports the expression of concepts such as Complexity and  
Self-Organization. In this paper, Complexity and Self-Organization concepts are applied to 
systems at the macro- and microlevels. Their similarities and differences are analyzed and 
information content is considered. The regularities of Complexity and Self-Organization are 
applied to petroleum as a~complex natural thermodynamic system. Petroleum reflects all of the 
main and widely understood features of Complexity and Self-Organization but demonstrates 
additional properties which were not considered earlier. Complexity and Self-Organization can 
help to deepen our understanding of the origin of hydrocarbon molecules, their age, and behavior in 
the process of petroleum generation in general.}    

\KWE{complexity; complexity properties; complex system; self-organization; artificial 
complexity; natural complexity; petroleum origin; hydrocarbon molecule complexity; 
petroleum  information content}

\DOI{10.14357/19922264160112} 

\vspace*{9pt}


\vskip 12pt plus 9pt minus 6pt

      \thispagestyle{myheadings}

      \begin{multicols}{2}

                  \label{st\stat}

\section*{Introduction}
    
\noindent
Among the phenomena most disputed by scientists during the last three decades are Information  
and Complexity. Various scientists' understanding and description of the phenomena are rather 
different.  At present, there is no generally accepted definition. This is not unusual and is quite 
understandable because the significance of one's conception of Complexity and Information 
corresponds to the philosophy and world-view in a~given discipline. For example, in Cybernetics, 
Complexity refers to the structure and control of a~Complex System. In biology, Complexity is 
the basis from which the appearance and evolution of living species are studies. In mathematics, 
Complexity is not phenomenon but the rate of difficulty, confusion or entanglement of any 
discipline with problems solved by the abstract method of mathematical calculation. 
Mathematical theories of Complexity help our understanding of philosophical problems dealing 
with intelligence and cognition.

The cognition of Information as phenomena has about two-thousand-year history. Its initial 
meaning was simply data, language and knowledge that people transmitted to each other. It 
became clear only at the beginning of the XX~century that one of the several concepts of 
Information is related to fundamental properties of matter\footnote[3]{For a~detailed analysis regarding the different senses in the 
meaning of `information' and the different concepts and definitions related to this word, see~[1]; 
see~[2] for an extended version.}.

The present authors are not attempting to create a~new definition of Information or Complexity. The aim of this 
study is to analyze Complexity as phenomenon, the expression of which depends on the level of 
matter organization. It aims to show that self-organization is the main feature of Complexity. 
It is considered to be inherent in objects, especially at the microlevel. On that level, any characteristic of 
a~natural system is an expression of uncertainty, which is subordinate to informatics laws~\cite{3-sel}. 
The unit of measurement of uncertainty is information, in the sense being used in this article~\cite{4-sel, 5-sel}. 
{ %\looseness=1

}

%\vspace*{-12pt}

\section*{Complexity and Complex Systems}

\noindent
In a~general sense, Complexity as phenomena has two definitions supplementing each other. 
According to the first, Complexity is an aggregate of interdependent parts which comprise 
a~\textbf{Whole}. The properties and behavior of a~\textbf{Whole} are not to be found in any 
of its parts. The parts possess their own individual behavior and unique properties. They could 
be the results of biotic or abiotic phenomena, both of which were named 
agents by Heylighen~\cite{7-sel}. 
Usually Complexity as the \textbf{Whole} is associated with the notion of Complex Systems. 

 A second definition considers \textit{Complexity} to be a~state between order and chaos,
 or disorder. A~state 
of order could be structures, signals, data, images, and many other objects~\cite{6-sel}. 

In order to distinguish between the definitions of \textit{Complexity} and Complex System, 
Complexity considered as phenomena will be assigned \textit{italics} and Complex System 
considered as concrete (three-dimensional (3D)) manifestations of \textit{Complexity} assigned in 
ordinary letters for the remainder of this paper. 

In both definitions, \textit{Complexity} means the \textbf{Whole} has been organized by its parts 
from Chaos to Order. The key idea in both is organization, in the sense of a~transition from one 
state of matter to another. Transition is the process of development of the parts 
which make up the \textbf{Whole}. Development is the second property of \textit{Complexity}. 
Without development, there could be no organization and \textit{Complexity} could not be 
created. At the same time, development is one of the principal properties of matter. Any complex 
system tends to the state of minimal free energy and entropy. The change of state is the
third  property of \textit{Complexity}.

Leading from such assumptions and independently from different views on \textit{Complexity} 
as phenomena, one can see three main features of \textit{Complexity: state, development}, and 
\textit{properties}. Each feature is evaluated by its own independent measurements in units 
expressing time, mass, velocity, equilibrium, temperature, and some others. 

The properties and character of the manifestation of \textit{Complexity} depend upon the level 
of matter organization. Matter on the microlevel~--- atoms, molecules, crystals, and cells are
organized into \textit{Complexity} by one type of parts, while matter on the macrolevel~---
materials, substances, machine components and others, are organized into \textit{Complexity} by 
other kind of parts. 

\vspace*{-9pt}

\section*{Complex Systems on~the~Macrolevel}

\noindent
Various parts of any macrolevel Complex System can be observed, measured by 
instruments, investigated, and analyzed. Some are created consciously by humans or instinctively 
by living creatures (insects, marine and terrestrial animals and birds, etc.). Humans create 
Complex Systems consciously by means of their intelligence and work in order to construct 
something, ordering it according to what was conceived and planned in advance. This includes 
all human-made technical facilities, mechanisms, and hugely Complex Systems such as national 
and international rail way networks with rails, carriages, locomotives, and other parts. In 
these complex systems, parts of the Whole work together in an organized mechanism.

Complex Systems like termite mounds, beehives, cocoons, nests, beaver dams, coral reefs, 
etc.\ are instinctively made by living creatures. Their constructions are guided by instinct 
programmed at the genetic level. The basis of this instinct is the motivation or impulse to strive 
for survival in the environment and to reproduce a~complex system similar to itself. 

Complex Systems which are created artificially, consciously, and independently from the genetic 
level and brought into being by humans through manufacturing have common features as well as 
differences. Such forms contain the common attribute of utility or applicability. The technical 
nature of these Complex Systems are planned and realized by humans, according to their 
purposes. Aspects of these Complex Systems such as construction, operation, and development 
are planned in advance. 

The commands which regulate the sequence of actions and interactions 
of various parts can be made by software algorithms and/or manual operations. On the other
hand,  living creatures construct Complex Systems instinctively via conditioned and unconditioned 
reflexes.

All types of \textit{complexity} and complex systems created artificially (at the 
macrolevel) have 
a~common feature: they are not the result of self-organization which is an internal natural 
property at the microlevel. Artificial \textit{complexity} is manufactured by men or living 
creatures by means of ordering and relating macrolevel substances and components. Their 
process of manufacturing can be perceived as linear, open for observation, planned 
in advance consciously or instinctively in the case of nonhuman living creatures. 

These systems are constructed and developed with the support of physics, chemistry, 
geophysics,
and other general scientific laws to overcome Earth's gravity, air mass turbulence, entropy 
decrease, survival under temperature changes, as well as by creating microconditions in the 
environment, using magnetic and gravitation fields to navigate and resolve other problems, etc. 

Any Complex System is not \textit{Complexity} or a~\textbf{Whole} until the function or 
purpose for which it has been created has been fulfilled. When the purpose for creating it ceases 
to exist, then it can be considered a~`dead mechanism.' A Complex System which has lost its 
basic components, for example, an abandoned seashell or bird nest is not considered a~
\textbf{Whole} and, therefore, not considered as \textit{Complexity}. At such a~point in its 
existence, it is considered as chaos or disorder.  For example, discarded piles of plastic, wood, 
metal, sea shells or dead coral, and many other forms can be conceived as just chaos at the 
macrolevel. 

There are large formations of Complex Systems at the macrolevel in the human social sphere 
and complex natural phenomena like the matter and energy cycle within the 
Earth--ocean--atmosphere system. Complex natural processes of the organic matter cycle, complex trophic 
chains among fauna and flora communities, etc.\ are not considered in this 
study.

\section*{The Expression of \textit{Complexity} at~the~Microlevel}

\vspace*{3pt}

\noindent
The \textit{Complexity} of Natural Complex Systems is the result of Self-Organization of the 
parts which can be elementary particles, atoms, molecules, crystals, or cells. Natural 
\textit{Complexity} and corresponding Complex Systems are created without human intervention 
and without any effort by living creatures. Their constructions are subordinated
 to general laws of 
Mechanics and Physics and depend upon the mechanisms and modes by which their parts are 
organized on an atomic, molecular, or cellular levels. 

The main feature of a~microlevel natural 
Complex System is uncertainty. Uncertainly is determined by many properties of 
\textit{Complexity} and the self-organization process of each level. The Self-Organization of 
a~Complex System, i.\,e., to create order out of chaos, is the result of internal energy flow, nonlinear 
interactions among the parts, unpredictability, uncontrollability, and multilevel balancing~\cite{7-sel}. 

The simplest Complex System is the hydrogen atom. It consists of a~nucleon carrying
 one proton and one orbit with one rotating electron. According to Quantum Mechanics, the state, 
properties, and development of the parts, being elementary particles, are affected by quantum 
processes and possess their own impulses and momentum. In our case when one considers 
\textit{Complexity} of macrolevel objects, the laws of Newtonian Mechanics play the main 
role. When one considers the \textit{Complexity} of microlevel objects, both Newtonian and 
Quantum Mechanics Laws need to be taken into consideration. One of the main quantum 
characteristics is uncertainty which might be measured using Shannon's definition of 
information~\cite{8-sel, 9-sel}. which was made in the context of electronic communication systems. This 
definition happened to be {\bfseries\textit{mathematically isomorphic}} with the definition of 
thermodynamic entropy in statistical mechanics. But this does not imply that they have the 
same meaning. Several authors alerted with regards to this issue. 

Myron Tribus asked Shannon: ``What he had thought about when he had finally confirmed his 
famous measure.'' Shannon replied~\cite{10-sel}: 
``My greatest concern was what to call it. I thought of calling it `information,' but the word 
was overly used, so I decided to call it `uncertainty.' When I discussed it with John von 
Neumann, he had a~better idea. Von Neumann told me, `You should call it entropy, for two 
reasons. In the first place your uncertainty function has been used in statistical mechanics 
under that name, so it already has a~name. In the second place, and more important, no one 
knows what entropy really is, so in a~debate you will always have the advantage.'.''

\columnbreak

With regarding to this issue, Denbigh affirmed~\cite{11-sel} that 
``in my view Von Neumann did Science a~disservice! \ldots There are, of course, good 
mathematical reasons why information theory and statistical mechanics because require 
functions having the same formal structure. They have a~common origin in probability 
theory and they also need to satisfy certain common requirements such as additivity. Yet 
this formal similarity [mathematical isomorphism] does not imply the functions are 
necessarily signifies or represents the same concepts. The term `entropy' had already been 
given a~well established physical meaning in thermodynamics. And it remains to be seen 
under what conditions, if any, thermodynamic entropy and information are mutually 
inconvertible.''

Several authors (see, for example,~\cite{12-sel}) agree with Denbigh~\cite{11-sel} regarding the disservice Von Neumann made to 
Science. Since then, conceptual confusion has generated more confusions and some time implicit 
and unnoticed nonsense. This is why, several authors recommend using the respective units in 
order to avoid confusions. Schneider and Lewis~\cite{13-sel}, for example, affirm that ``if you are 
making computations from symbols, \textit{always} use the term uncertainty, with recommended 
units of bits per symbol. If you mean the entropy of a~physical system, then use the term entropy, 
which has units of joules per kelvin (energy per temperature).''

In other very detailed articles, Callaos and Callaos affirmed the following~\cite{1-sel, 2-sel}:
``The words selected by Shannon to refer to his mathematical definition and the 
identification of different concepts by the same mathematical definition has created 
semantic and conceptual confusions, and generated significant controversies. Thomas 
D.~Schneider (from the National Institutes of Health), for example, affirms, referring to the 
words used by Shannon in his communications theory, that 
``Information Is Not Entropy,  Information Is Not Uncertainty!'' 

Stonier affirmed that ``a result of Von Neumann advice, the communications engineers and 
information theorists all became the victims of a~bad joke: that the potential indeterminacy 
of a~message is the same thing as entropy. The confusion still reigns today \ldots Shannon's 
sleight of hand has been attacked by a~number of authorities,'' among of whom are the 
authors mentioned above. For example, Hubert P.~Yockey, physicist and information 
theorist, who worked at the University of California, Berkeley, and under Robert 
Oppenheimer on the Manhattan Project, after carefully examining this issue concluded that 
``$\ldots$there is, therefore, no relation between Maxwell--Boltzmann--Gibbs entropy of 
statistical mechanics and Shannon's entropy of communications systems.'' Should we then 
differentiate between Shannon's entropy and Maxwell--Boltzmann--Gibbs's entropy, i.\,e., 
between informational and thermodynamic entropies? These kinds of confusions and 
contradictions are even found among the same communications engineers and researchers 
working on the same kind of engineering problems at the same time. Norbert Wiener, for 
example, affirms explicitly that 
``the notion of the amount of information attaches itself 
very naturally to a~classical notion in statistical mechanics: that of entropy. Just as the 
amount of information in a~system is a~measure of its degree of organization, so the entropy 
of a~system is a~measure of its degree of disorganization$\ldots$ 
The amount of information, 
being the negative logarithm of a~quantity which we may consider as a~probability, is 
essentially a~negative entropy.'' So, does ``information equal entropy,'' as it has been 
identified in Shannon's Theory? Or is it essentially ``negative entropy'' as Wiener affirmed? 
We conceive information, at least from a~subjective perspective and, more generally, in a~
biological context, as Wiener's conceived it. It is almost common sense. Entropy is related 
to disorder and information to order. Entropy is related to disorganization and information 
is associated with organization. Entropy is related to uncertainty and information is 
associated with certainty, or a~decrease in the level of uncertainty. In 1956, about eight years 
after Shannon's and Wiener's opposite conceptions regarding the relationship between 
information and entropy, Leon Brillouin published his book titled ``Science and Information 
Theory,'' where he affirmed that ``we prove that information must be considered as a~
negative term in the entropy of a~system; in short information is negentropy$\ldots$ Entropy 
measures the lack of information.'' So, it seems that Brillouin takes Wiener's side, in 
conceiving information as negative entropy. 

We recommend keeping in mind what has been quoted above in order to avoid conceptual 
confusions which might generate implicit and unnoticed nonsense. Many more details regarding 
this potential conceptual confusion and how to avoid it can be found in~\cite{2-sel}.  To help keeping in 
mind the different concepts named with the same word we proposed, a~nominal distinction and 
we also suggested to flown other authors proposals regarding this issue. Accordingly we 
affirmed that
Peters~\cite{14-sel} proposed to use the term ``\textit{spread}'' or ``measure of spread of the 
probability distribution function'' to refer to what he called ``superior expression'' [the genre] 
of both kinds of entropies [species]. Peters affirms that ``only when this spread refers to the 
distribution function of microphysical state properties will be called entropy, when it refers 
to a~quantity or a~set of symbols not representing thermodynamical state properties it will be 
called information theoretical entropy, in short \textbf{intropy}.'' In order to ``generalize 
the notion of information,'' Peters affirms that 
``information is created by an act, by an event 
which reduces the \textit{a~priori} spread of any quantity. The numerical difference 
(\textit{a~priori} spread 
minus \textit{a~posteriori} spread) is called information. In the limit case, the spread is totally 
removed. 
This happens when an individual symbol is selected out of a~set of \textit{a~priori} 
possible symbols.'' Consequently, information is produced when an act reduces the
\textit{a~priori}
\textbf{intropy} or the information theoretical entropy. Reiterating what was 
said above, but 
using Peter's terms and conceptual perspective, one can say that in a~two-symbol, or  
two-state, system, with  maximum informational entropy, or \textbf{intropy} (i.\,e., where 
the two states are equiprobables, $p_1 = p_2 = 0.5$), has by definition 1 bit of information 
capacity. This means that its spread, or intropy, is 1~bit. The respective limit case (where 
the spread is completely removed) is when one of the two states is selected, i.\,e., when the 
intropy is reduced to zero. Consequently, 
\begin{multline*}
\mbox{\textit{A priori} intropy} - \mbox{\textit{a posteriori} intropy}\\
 = 1~\mbox{bit}- 0~\mbox{bit}  =  1~\mbox{bit}.
\end{multline*}

So, as it can be easily noticed, although the measure we are using for \textit{a~priori} intropy, 
\textit{a~posteriori} intropy, and information is the same, the concepts are different. Intropy (or 
information theoretical entropy) and information (\textit{a~priori} intropy\;$-$\;\textit{a~posteriori} 
intropy) are 
not the same concepts. The concept of ``information capacity'' is not the same as 
the ``information delivered'' by an action, or selection. The ``quantity
of information capacity'' is equal the ``quantity of information delivered'' just in the limit 
case where \textit{a~posteriori} intropy is equal to zero. In nonlimit cases (i.\,e.,
\textit{a~posteriori} intropy\;$>$\;0), both measures are not the same. We cannot deliver more information than the source 
capacity, but we can deliver less (or equal, in the limit case) quantity of information. 
Consequently,  the information measure of the source should be differentiated from the 
information measure of the delivered information, if one wants to avoid confusing 
ambiguities. 

As it was noted above, the uncertainty, which is one of the main quantum characteristics, could be 
measured by Shannon's definition of information or ``\textit{information entropy}'' if 
one can 
keep in mind the distinction referred to above in order to avoid the confusions. 
Broglie~\cite{15-sel} 
established that elementary particles behave as a~particle and as a~wave possessing by energy and 
impulse. The cinematic and dynamics of elementary particle movement in the quantum field was 
described by Heisenberg~\cite{4-sel}. Zeilinger proved that the \textbf{bit} (\textbf{b}inary 
un\textbf{it}), introduced by Shannon in 1948 as information unit, is, in fact, the information 
capability (\textit{a~priori} intropy)~\cite{5-sel}. Rashevsky~\cite{16-sel} and Trucco~\cite{17-sel} proved that the 3D 
volume of a~molecule has topological information that can be measured in \textbf{bits}. 

Natural Complex Systems at the molecular level are of two types. One type is associated to 
living organisms composed of very complex cells consisting of billions of molecules. Petroleum 
is another type composed of thousands of hydrocarbon molecules. According to the molecular 
organization of petroleum, it is considered to be in a~state between organic and inorganic.

On the other hand, Self-Organization of \textit{Complexity} and its information content can be 
understood if one considers two fundamental discoveries of science. First is the discovery of 
amino acids or macromolecules DNA and RNA in biology. The second is the discovery of the 
quantum behavior of elementary particles in all chemical elements including carbon and 
hydrogen.

DNA and RNA are the self-organizing molecules into a~macroform. The self-organized parts of the 
macroform molecule are genes. These genes form genetic code or the matrix carrying genetic 
information. The 10$^{14}$ cells of the most complex natural system~--- human beings~--- are 
reproduced according to the gene-matrix. Ashby  established the relationship between Variety, 
Organization, Complexity, and Information of microlevel objects~\cite{18-sel}. He showed their role in 
the appearance and development of biotic complex natural systems.

Self-Organization on the atomic level objects is generated by elementary particles (electrons and 
protons). Their quantum behavior, uncertainty, and corresponding information content have close 
connection to quantum mechanics principles.
 
\section*{Petroleum \textit{Complexity} is Identified by~Its~Composition and~Properties }

\noindent
Petroleum has three features showing its Self-Organization into a~Complex System.
\begin{enumerate}[1.]
  \item  The parts of petroleum are hydrocarbon molecules; which compose 95\% of its 
content. These parts are self-organized into three types of hydrocarbon molecules: ($i$)~saturated 
(paraffin's) molecules which are a~chain of carbon and hydrogen atoms; ($ii$)~unsaturated 
hydrocarbons (naphthenes) molecules which are the closed cycles; and ($iii$)~aromatic in which 
molecules are hexahedron in structure with a~carbon atom located in each of its corners. All these 
parts self-organize to form petroleum. 
  
   \item During the petroleum generation process, carbon and hydrogen atoms 
   are not drawn or 
extracted from any rocks, composed from geological media (matter) containing these elements. 
The initial quantity of carbon and hydrogen atoms in organic sediments (according to an organic 
model) or where such atoms were supplied by the Earth's mantle substance (according to 
inorganic model) remains constant during the entire process of petroleum generation. The three 
main types of hydrocarbon molecules mentioned above are contained in the final petroleum in 
their original respective quantities. This occurs despite the differences in the structure, molecular 
mass, geological media, and time of generation. This phenomenon shows that petroleum 
complexity is a~Self-Organizing process in which its parts are the hydrocarbon molecules. 
Self-Organization in that case is the transformation of one type of molecule into another. In the case 
of petroleum, this process is triggered by catalytic reactions between hydrocarbon molecules and 
the elements composing the surrounding rocks of geological media.
  
\item Hydrocarbon molecules are generated during different stages of the process in geological 
media where thermodynamic conditions are constantly changing. Therefore, the hydrocarbon 
molecules as the parts of the Complex System preserve their composition and structure. 
The fact of self-organization has 
been proved as the main mechanism of petroleum generation. 
Petroleum in fields located on The Arabian Peninsula, North Africa, Oklahoma, Western 
Siberia, and other areas are an aggregate of paraffin, naphthenic, and aromatic hydrocarbons 
molecules. There are direct indices of the presence of discrete and nonlinear mechanisms that 
account for the preservation of structural and chemical individuality in each type of 
hydrocarbon molecule. Seyful-Mulyukov showed this mechanism is the quantum matrix of the 
hydrocarbon molecule~\cite{19-sel}. Via analogical thinking, one can say that it is some sort of 
`\textit{hydrocarbon genetic code}' what guarantees the development of the given types of 
molecules during all stages of petroleum generation. The matrix is a~subatomic level 
phenomenon in which the elementary particles of atoms interact to generate hydrocarbon 
molecules.
\end{enumerate}

The generation of any hydrocarbon molecule depends upon the electron orbitals of carbon and 
hydrogen atoms hybridization forming a~molecule. New molecule generation results from 
a~change from one type of hybridization to another. This change is the reaction of a~Complex 
System to changes in the surrounding thermodynamic and geological media. Therefore, during 
creation of a~new type of molecules, all previously generated molecules maintain their spin or 
wave fields of elementary particles. The Quantum matrix provides for the preservation of each 
type of molecules appearing during the different stages of petroleum genesis. 

The petroleum quantum matrix is comparable to the macromolecule DNC and DNA in living 
species. In spite of the differences at the organization level, both fulfill the same function~--- 
\textit{Complex System Reproduction}. For example, the reproduction of a~human being, 
comprising some 10$^{14}$ cells, occurs as the DNA of the older organism is replicated within 
the new organism by use of the same quantity and quality of cells. The petroleum quantum 
matrix contains coded information that insures the parallel existence of the specific types of 
molecules within the system. DNC and DNA macromolecules are a~higher level of Organization 
compared to petroleum's quantum matrix.  

Natural Complex Systems such as petroleum demonstrate such features as \textit{Existence, 
Development, and Cognoscibility} besides various physical properties. They expose the 
regularities and mechanism of \textit{Complexity} development as well as \textit{Composition, 
Structure}, and \textit{Duration of Existence}. These features are the key elements to 
understand petroleum genesis. 

\textit{Existence} determines the behavior and properties of \textit{Complexity} as the 
\textbf{Whole} which is different from any of its parts. By existence, we mean such Complex 
System characteristics as \textit{Invariance, Synenergetics, Uniqueness, and Unpredictability}. 
Petroleum completely demonstrates these features. As the \textbf{Whole}, petroleum features 
invariance as it maintains its composition and properties in different geological structures, 
stratigraphic levels, and geophysical media. Petroleum is also synergetic because it exists as the 
\textbf{Whole} in which properties and characteristics are different from any of its parts. 
Petroleum is unique due to its hydrocarbon composition, structure, and molecular mass being 
inimitable and able to be generated and exist only in a~specific geological period of Earth's 
development. Petroleum generation requires specific thermodynamic, geological, geochemical 
conditions which appeared in the Earth's crust not long ago. The genesis process remains 
unpredictable as long as data on the trajectory of geophysical matter development at any moment 
of geological time cannot be established and expressed mathematically.    

Development is the totality of such dynamic petroleum features as \textit{Openness,  
Nonstationary, and Movement Permanency}. Petroleum openness is a~property that allows 
petroleum to exchange energy and information with the environment. This leads to changes in 
the hydrocarbon molecules structure and composition. This situation triggers a~change in the 
Complex System as a~\textbf{Whole}. Nonstationary is another property of petroleum which 
permits the ability to change structure and phase state depending on the surrounding geological 
media. If change leads to the loss of the main types of hydrocarbon molecules, the remainder is 
hydrocarbons but not petroleum as Complex System.  

Movement Permanency refers to constant changes in petroleum's internal structure and 
composition caused by changes in its entropy. Movement Permanency explains the impossibility 
of petroleum in maintaining its native composition and structure during millions of years. 
Entropy change refers to the difference between any initial and finite moment of petroleum 
development. That is why, Devonian, Carboniferous, Jurassic, or Cretaceous age petroleum could 
not exist now.

\textit{Cognoscibility} has a~direct relationship with petroleum origin, age, and 
\textit{Complexity}. Cognoscibility is the process of interaction between two systems: ($i$)~the 
\textit{subject} of cognition, i.\,e., the researcher in our case, and ($ii$)~the \textit{object} of 
cognition, which, in our case, is the petroleum. Researchers are studying the properties and 
composition of petroleum which has recently been extracted. Two opinions could be expressed 
on the age of that petroleum~--- either it was generated recently or it was generated hundreds of 
millions of years ago. 

The first opinion means that the researcher examined and fixed the data, facts and conditions 
characterizing the petroleum as a~Complex System, created under recently existing thermodynamic 
and geological conditions. The second opinion means that the petroleum is a~Complex System 
generated hundreds of millions of years ago. If so, that system maintained its composition and 
structure over hundreds of millions of years. Also, it means that the specific thermodynamic and 
geological conditions required for petroleum generation over those hundreds of millions of years 
were the same as they are today. This absolutely contradicts the postulates and laws of 
geochemistry and geology. Petroleum could not exist in its native state for hundreds of millions 
years.

\textit{Cognoscibility} is the interaction between the subject and objects of cognition within a~
determined time and space. Any natural complex system development can be comprehended if 
one takes into consideration that interaction. For example, a~paleontologist while modeling 
Cambrian fauna, namely, the trilobite (650~million years old) had to deal with a~simpler type 
(dummy) of natural stone that is not considered as a~Complex System but is rather a~chaotic mixture 
of minerals replicating the system by simple repetition of its form but not its content or natural 
\textit{Complexity}. Cognoscibility provides proof for the idea that any Complex System 
generated long ago and remaining unchanged for millennia could not exist. All 
\textit{Complexity} considered as a~Whole has a~life circle, i.\,e., creation, development, existence, 
and decay.
Creating and developing petroleum Complexity as a~phenomenon is characterized by the Laws of 
\textit{Simplicity}, \textit{Uncertainty}, and \textit{Requisite Variety}.

Simplicity is Natural or Universal Law. Nature generates any Complex System by selecting the 
simplest option among all existing options since it is the most efficient use of energy and 
resources. Petroleum \textit{Complexity} is the simplest aggregate of hydrocarbon molecules 
generated in a~given geochemical, geological, thermodynamic, and other circumstances 
relevant to specific stages in the development of the Earth. The Law of Simplicity provides 
researchers with an understanding of the age of any Complex System (petroleum, animals,  
etc.). This Law is the embodiment of the simplest optimal form of \textit{Complexity} possible 
in a~given, specific period of the Earth evolution.

Uncertainty is a~notion used in many sciences including Informatics, Mathematics, Philosophy, 
Cybernetics, Physics, and others. In these contexts, Informatics might be conceived 
as a~science 
of Uncertainty, i.\,e., science using mathematical and cybernetic methods to transform 
Uncertainty (potential information) into Information (actual or delivered information). 
Informatics, by means of applying a~range of technologies, allows the use, transmission, storage, 
retrieval, and many other ways of handling, managing, and utilizing Information.

Uncertainty is important for understanding the interactions between elementary particles of the 
carbon and hydrogen atoms during the process of generating hydrocarbon molecules. The key 
feature of that process is the conjugate variable states of the interacting objects subordinate to the 
Laws of Quantum Mechanics. Uncertainty is applied to express the passage of the complex 
system from one state to other. Uncertainty and Information share causal reciprocity insofar that 
a change in Uncertainty leads to change in Information, potential information might be 
transformed in delivered or actual information. 

As it is known, Requisite Variety Law was formulated first by Ashby~\cite{20-sel}.  The Law expresses 
the interconnection of two systems: ($i$)~the one which is controlling and ($ii$)~that which is under 
control.  According to this Law, the larger the variety of reactions that a~control system can 
generate (in its interaction with the controlled system), the larger the variety of perturbations it 
can compensate.  Metaphorically expressed, the Requisite Variety Law states that 
{\bfseries\textit{only the internal variety of a~system can `destroy' (deal with) its external 
variety}}. A~control system of two states, for example, cannot control a~system of three possible 
states or more. Consequently, the potential adaptability of a~system to the uncertainties of its 
environment depends on having more internal variety and the external ones. Regulative control 
prevents the transference of variety from the environment to the systems; so, the system can be 
less deviated from its `goal.' This can be conceived as the opposite to information transmission 
where the purpose is to maximize the conservation of variety. 

In general, if one has a~set~$D$ of disturbances (in the environment or the controlled system), 
a~set~$R$ of responses (of the system or of the regulative control), and a~set~$O$ 
of outcomes (of the 
system in its interaction with its environment or of the control systems 
interacting with the 
control system), then, Ashby affirmed~\cite[p.~47]{20-sel} that 
``if the varieties are measured logarithmically, this means that if the varieties of 
$D$, $R$, and actual outcomes are, respectively, $V_d$, $V_r$, and $V_o$, then the minimal value 
of~$V_o$ is $V_d -V_r$. If now~$V_d$ is given, $V_o$ minimum can be lessened only by 
a~corresponding increase in~$V_r$. This is the law of requisite variety. What it 
means is that restriction of the outcomes to the subset that is valued as Good 
demands a~certain variety in~$R$.''

The Requisite Variety Law might be applied to Complex Systems such as society, industrial and 
agricultural production systems, economic systems and nature, including petroleum as a~complex 
whole. The survival of the petroleum for 1--2~hundreds million years, in spite of all 
changes in successive geological environments with its respective high level of disturbances, 
shows that it has a~high internal variety which might be conceived as a~measure of its complexity, 
which increased though process in which simpler molecules formed part of larger, or more 
complex, ones as a~result of thermodynamic changes and catalytic reactions in the continuously 
changing environment.

\textit{Complexity} creation is not free-running, autonomous, and independent from 
environmental processes. Dependence on the state and properties of the environment demonstrates 
the creation of petroleum \textit{Complexity}. Petroleum's composition, structure, and 
development depend directly on the totality of specific geological media. If one considers that 
petroleum could be a~Complex System of one hundred or even two hundred million years of age, 
then one should also consider that the geological environment today must be the same as it was 
many millions of years ago. This, however, is implausible as everything is changing constantly, 
including the geological media and hydrocarbon molecules. This may account for the current 
failure to find petroleum which is not comparable to modern petroleum.

\vspace*{-6pt}

\section*{Petroleum Complex System and~Its~Information Content}

\noindent
Information is one of the main objective characteristics of the elementary particles in any atom 
including carbon, hydrogen, and hydrocarbon molecules~[3--5]. At the same time, Information 
volume (stored information or potential information) is one of the indices of a~Natural Complex 
Systems development process. Petroleum is a~perfect example for demonstrating this concept. 

Petroleum hydrocarbon molecules are 99\% composed of atoms of carbon (C), hydrogen (H), 
oxygen (O), nitrogen (N), and sulfur (S). Their information content in \textbf{bits} is H~--- 10,  
C~--- 109, N~--- 138, O~--- 149, and S~--- 317~\cite{3-sel}. Petroleum is generated from hydrocarbon 
molecules which combine in consecutive order as they change form. This process can be 
expressed by Information content in \textbf{bits} as molecules appear and more complex 
molecules are structured. 

Petroleum generation results in the appearance of carbon and hydrogen atoms in the normal 
nucleon-orbital configuration. Fomin showed that it occurs due to mantle plasma matter 
decompression~\cite{21-sel}. The bonding of these atoms is realized through several stages, the first 
being hydrocarbon protomolecule formation. The following stages are consecutive processes of 
gaseous and liquid hydrocarbon molecule generation accruing in deep lithosphere strata of the 
Earth. The main stage of petroleum hydrocarbon molecule generation is, as was briefly 
mentioned 
above, the transformation of simpler molecules into more complex molecules, a~process triggered 
by thermodynamic changes and catalytic reactions between hydrocarbon molecules and the 
crystal structure of surrounding geological media rocks. 

Following the Heylighen idea on the \textit{Complexity} development~\cite{7-sel}, petroleum generation 
completely corresponds to the process of \textit{Complexity}  formation. The process is 
accompanied with a~corresponding increase in Information content~\cite{22-sel}. Hydrogen and carbon 
atoms being formed in their normal configuration in upper mantle have the following bits. 
Information content H~--- \textbf{10~bit} and C~--- \textbf{109~bit}, 
respectively. Methane (СН$_4$) is 
the first hydrocarbon molecule generated with~\textbf{154~bit} of Information. Next, gases 
such as butane (С$_4$Н$_{10}$) form hydrocarbon molecules which are 
possessing~\textbf{547~bit} of Information.
 
The main types of hydrocarbon molecules including saturated (paraffin's), unsaturated 
(naphthenes), and aromatic molecules make up 90\% of petroleum composition. The total 
Information content of the pure hydrocarbon part of petroleum, which consists only of carbon 
and hydrogen atoms, is \textbf{4148~bit} in petroleum conventional molecule.

Petroleum completes its formation process and is accumulated in pools or within fields where 
finally all hydrocarbon molecules join. Petroleum's final composition consists of a~mixture of 
pure hydrocarbons (paraffin's, naphthenes, and aromatics), heteroatom molecules, and dash 
(sulfur, nitrogen, oxygen, and some others). Conventional molecules of light petroleum have an 
empirical formula of C$_{32}$H$_{66}$SN and a~corresponding Information content 
of~\textbf{16\,224~bit}.

In specific conditions at the depth interval of 50--200~m, petroleum loses its light components and 
transforms into bitumen. The empirical formula of conventional molecule of bitumen is 
C$_{45}$H$_{51}$O$_2$SN and an Information content of~\textbf{17\,789~bit}.

In a~general sense, petroleum genesis is a~process taking place in the upper lithosphere. From a~
chemical point of view, it is natural liquid substance formation made up of hydrocarbons 
molecules. From a~physical point of view, the formation of petroleum \textit{Complexity} is the 
natural ordered process of Self-Organization: elementary particle form atoms, atoms form 
hydrocarbon molecules, and these transform into three types which finally compose petroleum. 
The consecutive expansion of \textit{Complexity} is accompanied by a~corresponding increase in 
Information Content. 

\section*{Concluding Remarks}

\noindent
\textit{Complexity} and Complex Systems are considered as a~\textbf{Whole} composed of the 
parts (also referred to as objects or agents). The types, scales and modes of organization are 
different. On the macrolevel, \textit{Complexity} and Complex Systems are created artificially 
by human beings or other living species. They are not the result of Self-Organization and their 
\textit{Complexity} is created consciously or instinctively. Their main feature is utility because 
they are a~construct generated by a~determined aim or directed by instinct. Artificial complex 
systems that have been created by macrolevel objects are not the self-organizing systems. 

\textit{Complexity} and Natural Complex Systems of microlevel objects are created by Nature 
alone. Such microlevel objects are subdivided on the lowest atomic and highest molecular 
levels. The main feature of microlevel Complex Systems is \textit{Complexity} and 
Self-Organization. Its main feature is Uncertainty of the distribution and interaction of the parts in the 
context of a~\textbf{Whole}. That is why, one can use the concepts of information as defined by 
Shannon in expressing petroleum's information content \textit{Complexity} and Complex 
Systems at a~molecular level composed by the cells of living species or the petroleum 
hydrocarbon molecules, the latter according to a~scale of organization positioned between 
organic and inorganic.

Self-Organization of \textit{Complexity} and their Information Content on the atomic level are 
conditioned by the quantum behavior of the atom's elementary particles. It is the main factor of 
Complex System's parts reproduction and interaction. On the molecular level, \textit{Complexity} 
is determined by the properties, structure, and interaction of DNC and DNA macromolecules 
which contain the genes and the related biological information. Petroleum is a~Natural Complex 
Thermodynamic System which not only shows the main and well-known features of 
\textit{Complexity} but reveals some new features. Petroleum is a~Self-Organizing Natural 
Complex System structured by its parts (hydrocarbon molecules) via nonlinear discrete 
processes. The structuring and preservation of these parts in the Complex System is provided by 
quantum matrixes. Accordingly, quantum matrixes create three main types of hydrocarbon 
molecules which remain constant in petroleum in its native form. On the molecular level, 
hydrocarbons transformation into~500~molecular structures, which make up the petroleum, is 
realized by their catalytic reactions within the rocks of geological media. Petroleum demonstrates 
additional features of \textit{Complexity} including \textit{existence, development, 
cognoscibility, simplicity, uncertainty, and requisite variety}. Complexity and Natural Complex 
Systems at the microlevel possess Information Content corresponding to the quantity and the 
quality of atoms and molecules composing the parts of the Complex System.             
      
\Ack
\noindent
Originally, this article was written in Russian and English. Authors appreciate Scott Barbur for 
his efforts to integrate both and editing the text in English allowed a~more adequate 
understanding of the main idea of the article  to the Western scientific community.

\renewcommand{\bibname}{\protect\rmfamily References}


{\small\frenchspacing
{%\baselineskip=10.8pt
\begin{thebibliography}{99}

\bibitem{1-sel}
\Aue{Callaos, N., and B.~Callaos}. 2002. Toward a~systemic notion of information: Practical 
consequences. \textit{Informing Sci. J.} 5(1):1--11. 

\bibitem{2-sel}
\Aue{Callaos, B., and B.~Callaos}. 2011. Toward a~systemic notion of information: Practical 
consequences (extended version). Academia. 99~p. Available at: {\sf 
http://www. academia.edu/4434476/Toward\_a\_Systemic\_Notion\_of\_ Information\_Practical\_Consequences\_Extended\_Version\_} (accessed September~12, 2015). 

\bibitem{3-sel}
\Aue{Gurevich, I.\,M.} 2007. \textit{Zakony informatiki~--- 
osnova stroeniya i~poznaniya slozhnykh sistem} 
[Informatics laws~--- the basis of structure and knowledge of complex systems].
  Мoscow: TORUS PRESS. 399~p. 

\bibitem{4-sel}
\Aue{Heisenberg, W.} 1957. Quantum theory of fields and elementary particles. \textit{Rev. 
Mod. Phys.} 29(3):269--278.
\bibitem{5-sel}
\Aue{Zeilinger, A.\,A.} 1999. Foundation principle for Quantum Mechanics. \textit{Found. 
Phys.} 29(4):631--643.

\bibitem{7-sel}
\Aue{Heylighen, F.} 2008. Complexity and self-organization. \textit{Encyclopedia of library 
and information science}.  Eds.\ M.\,J.~Bates and M.\,N.~Maack. New York, NY: Taylor and 
Francis. 20~p. Available at: {\sf  http://pespmc1.vub.ac. be/papers/elis-complexity.pdf} (accessed 
September~19, 2015).

\bibitem{6-sel}
\Aue{Prigogine, I., and I.~Stengers}. 1984. \textit{Order out of Chaos:
Man's new dialogue with nature}. New York, NY: Bantam 
Books. 381~p.

\bibitem{8-sel}
\Aue{Shannon, C.\,E.} 1948. A~mathematical theory of communication. \textit{Bell Syst.
Tech.~J.} 27:379--423; 623--656.

\bibitem{9-sel}
\Aue{Shannon, C.\,E., and W.~Weaver}. 1963. \textit{The mathematical theory of 
communication}. Urbana and Chicago: University of Illinois Press. 125~p.

\bibitem{10-sel}
\Aue{Tribus, M., and C.~McIrvine}. 1971. Energy and information. \textit{Sci. Am.}  
225(3):179--188. 
%referenced by Arieh Ben-Naim, 2008, A Farewell to Entropy: Statistical Thermodynamics 
%Based on Information, New Jersey: World Scientific, p. xviii

\bibitem{11-sel}
\Aue{Denbigh, K.} 1981. How subjective is entropy? \textit{Chem. Brit.}  
17(4):168--185. 

\bibitem{12-sel}
\Aue{Ben-Naim, A.} 2008. \textit{A~farewell to entropy: Statistical thermodynamics based on 
information: $S=\log W$}. Hackensack, NJ: World Scientific. 409~p.

\bibitem{13-sel}
\Aue{Schneider, T., and K.~Lewis}. 2011.  
A~glossary for biological information theory and the Delila system. Available at: {\sf  
https://schneider.ncifcrf.gov/glossaryframes.html} (accessed September~27, 2015). 

\bibitem{14-sel}
\Aue{Peters, J.} 1975. Entropy and information: Conformities and controversies. \textit{Entropy 
and information in science and philosophy}. Eds. L.~Kub$\acute{\mbox{a}}$t and J.~Zeman. 
Amsterdam: Elsevier Scientific Publ. Co.  61--81.

\bibitem{15-sel}
\Aue{Broglie, L.} 1927. Wave mechanics and the atomic structure of matter and radiation. 
\textit{J.~Phys. Paris}  8(5):225--241.

\bibitem{16-sel}
\Aue{Rashevsky, N.} 1955. Life, information theory and topology. \textit{B.
Math. Biophys.} 17(3):229--235. 
\bibitem{17-sel}
\Aue{Trucco, E.} 1956. A~note on the information content of graphs. \textit{B. 
Math. Biophys.} 18(2):129--135. 
\bibitem{18-sel}
\Aue{Ashby, W.\,R.} 1956.  \textit{An introduction to 
cybernetics}. London: Chapman\,\&\,Hall. 283~p.

\bibitem{19-sel}
\Aue{Seyful-Mulyukov, R.\,B.} 2014. Quantum matrix of hydrocarbon molecules is the key 
element of petroleum genesis. \textit{14th Multidisciplinary Scientific Geoconference 
(International) and Expo SGEM Proceedings}.  Varna, Bulgaria. 1:759--765.

\bibitem{20-sel}
\Aue{Ashby, W.\,R.} 1958. Requisite variety and its implications for the control of complex 
systems. \textit{Cybernetica} 1-2:83--89. Available at:  
{\sf  http://pespmc1.vub.ac. be/books/ashbyreqvar.pdf} 
(accessed  October~11, 2015).  

\bibitem{21-sel}
\Aue{Fomin, Yu.\,M.} 2005. Verkhnyaya astenosfera~--- perekhodnaya zona 
mezhdu veshchestvom mantii i~li\-to\-sfe\-ry [The upper asthenosphere~--- 
the transition zone between the mantle and lithosphere]. 
\textit{Problemy evo\-lyu\-tsii} [Evolution problems]. Available at: 
{\sf http://www.\linebreak evolbiol.ru/fomin.htm} (accessed March~11, 2016).

\bibitem{22-sel}
\Aue{Seyful-Mulyukov, R., and M.\,K.~Hlava}. 2012. The nature of information as 
a~fundamental property of matter: A~case study using petroleum and hydrocarbon gases. 
\textit{3rd Multi-Conference (International) on Complexity, Informatics and Cybernetics Proceedings}. 
Orlando, FL. 173--178.
\end{thebibliography} }
 }

\end{multicols}

\vspace*{-6pt}

\hfill{\small\textit{Received October 15, 2015}}

\vspace*{-12pt}

\Contr

\noindent
\textbf{Callaos Nagib C.} (b.\ 1943)~--- PhD  in mathematics; President, 
International Institute of Informatics and Systemic, 14269 Lord Barclay Dr., 
Orlando, FL 32837, USA; n.c.callaos@callaos.com


\vspace*{3pt}

\noindent
\textbf{Seyful-Mulyukov Rustem B.} (b.\ 1928)~--- Doctor of Science in 
geology, professor, Head of Laboratory, Institute of Informatics Problems, 
Federal Research Center ``Computer Science and Control'' of the Russian 
Academy of Sciences, 44-2~Vavilov Str,  Moscow 119333, Russian Federation; 
rust@ipiran.ru


%\vspace*{8pt}

%\hrule

%\vspace*{2pt}

%\hrule

\newpage

\vspace*{-24pt}



\def\tit{СЛОЖНОСТЬ И ЕЕ ИНФОРМАЦИОННОЕ СОДЕРЖАНИЕ}

\def\aut{Н.\,К. Каллаос$^1$, Р.\,Б.~Сейфуль-Мулюков$^2$}


\def\titkol{Сложность и~ее информационное содержание}

\def\autkol{Н.\,К. Каллаос, Р.\,Б.~Сейфуль-Мулюков}

%{\renewcommand{\thefootnote}{\fnsymbol{footnote}}
%\footnotetext[1]{Работа проводится при финансовой поддержке Программы
%стратегического развития Петрозаводского государственного университета в~рамках
%на\-уч\-но-ис\-сле\-до\-ва\-тель\-ской деятельности.}}


\titel{\tit}{\aut}{\autkol}{\titkol}

\vspace*{-12pt}

\noindent
$^1$Международный институт информатики и~системных исследований, Орландо, Флорида, США,\linebreak
$\hphantom{^1}$n.c.callaos@callaos.com


\noindent
$^2$Институт проблем информатики Федерального исследовательского центра <<Информатика 
и~управление>>\linebreak
 $\hphantom{^1}$Российской академии наук, rust@ipiran.ru

\vspace*{6pt}

\def\leftfootline{\small{\textbf{\thepage}
\hfill ИНФОРМАТИКА И ЕЁ ПРИМЕНЕНИЯ\ \ \ том\ 10\ \ \ выпуск\ 1\ \ \ 2016}
}%
 \def\rightfootline{\small{ИНФОРМАТИКА И ЕЁ ПРИМЕНЕНИЯ\ \ \ том\ 10\ \ \ выпуск\ 1\ \ \ 2016
\hfill \textbf{\thepage}}}



     \Abst{Термин <<информация>> имеет разные толкования и~определения. Авторы статьи 
разделяют мнение исследователей, рассматривающих информацию как одно из свойств материи. 
Такое пред\-став\-ле\-ние информации использовано для анализа понятий <<сложность>> 
и~<<самоорганизация>>. Их связь с~информацией анализируется на примере закономерностей 
создания и~строения объектов субатомного, микро- и~макроуровня. Понятия <<сложность>>, 
<<самоорганизация>> и~<<информационное содержание>> применены к~нефти, на примере 
которой доказывается связь этих трех понятий. Анализ нефти как сложной термодинамической 
природной системы дал возможность расширить понятие <<сложность>> свойствами 
и~особенностями, которые ранее не рассматривались. Сложность и~самоорганизация помогают 
представить глубинное происхождение и~строение углеводородных молекул, возраст, а~также 
образование и~возраст нефти в~целом.}
     
     \KW{сложность; свойства сложности; сложные системы; самоорганизация; искусственная 
сложность; природная сложность; сложность углеводородных молекул; происхождение нефти; 
информационное содержание нефти}

\DOI{10.14357/19922264160112} 

\vspace*{18pt}


 \begin{multicols}{2}

\renewcommand{\bibname}{\protect\rmfamily Литература}
%\renewcommand{\bibname}{\large\protect\rm References}

{\small\frenchspacing
{%\baselineskip=10.8pt
\begin{thebibliography}{99}

\bibitem{1-sel-1}
\Au{Callaos N., Callaos B.} Toward a~systemic notion of information: Practical 
consequences~// Informing Sci.~J., 2002. Vol.~5. No.\,1. P.~1--11. 
\bibitem{2-sel-1}
\Au{Callaos N., Callaos B.} Toward a~systemic notion of information: Practical 
consequences (extended version).~--- Academia, 2011. 99~p. {\sf 
http://www.academia.edu/ 4434476/Toward\_a\_Systemic\_Notion\_of\_Information\_\linebreak 
Practical\_Consequences\_Extended\_Version}. 
\bibitem{3-sel-1}
\Au{Гуревич И.\,М.} Законы информатики~--- основа строения и~познания сложных 
систем.~--- М.: ТОРУС ПРЕСС, 2007. 399~с. 
\bibitem{4-sel-1}
\Au{Heisenberg W.} Quantum theory of fields and elementary particles~// Rev. 
Mod. Phys., 1957. Vol.~29. No.\,3. P.~269--278.
\bibitem{5-sel-1}
\Au{Zeilinger A.\,A.} Foundation principle for quantum mechanics~// Found. 
Phys., 1999. Vol.~29. No.\,4. P.~631--643.

\bibitem{7-sel-1}
\Au{Heylighen F.} Complexity and Self-Organization~// Encyclopedia of library and 
information sciences~/ Eds. M.\,J.~Bates, M.\,N.~Maack.~--- New York, NY, USA: Taylor 
and Francis, 2008. 20~p. {\sf  http://pespmc1.vub.ac.be/ papers/elis-complexity.pdf}.

\bibitem{6-sel-1}
\Au{Prigogine I., Stengers I.} Order out of Chaos:
Man's new dialogue with nature.~--- New York, NY, USA: Bantam Books, 
1984. 381~p.

\bibitem{8-sel-1}
\Au{Shannon C.\,E.} A~mathematical theory of communication~// 
Bell Syst. Tech.~J., 1948. Vol.~27. P.~379--423; 623--656.

\columnbreak 

\bibitem{9-sel-1}
\Au{Shannon C.\,E., Weaver W.} The mathematical rteory of communication.~--- Urbana and 
Chicago: University of Illinois Press, 1963. 125~p.
\bibitem{10-sel-1}
\Au{Tribus M., McIrvine C.} Energy and information~// Sci. Am., 1971. 
Vol.~225. No.\,3. P.~179--188.
\bibitem{11-sel-1}
\Au{Denbigh K.} How subjective is entropy?~// Chem. Brit., 1981. Vol.~17. No.\,4. 
P.~168--185.
\bibitem{12-sel-1}
\Au{Ben-Naim A.} A~farewell to entropy: Statistical thermodynamics based on 
information: $S=\log W$.~---  Hackensack, NJ, USA: World Scientific, 2008. 409~p.
\bibitem{13-sel-1}
\Au{Schneider T., Lewis K.} A~glossary for biological information theory and the Delila 
system. {\sf https:// schneider.ncifcrf.gov/glossaryframes.html}. 
\bibitem{14-sel-1}
\Au{Peters J.} Entropy and information: Conformities and controversies~// Entropy and 
information in science and philosophy~/ Eds.\ L.~Kub$\acute{\mbox{a}}$t, J.~Zeman.~--- 
Amsterdam: Elsevier Scientific Publ. Co., 1975. P.~61--81.
\bibitem{15-sel-1}
\Au{Broglie L.} Wave mechanics and the atomic structure of matter and radiation~// 
J.~Phys. Paris, 1927. Vol.~8. No.\,5. P.~225--241.
\bibitem{16-sel-1}
\Au{Rashevsky N.} Life, information theory and topology~// B.~Math. 
Biophys., 1955. Vol.~17. No.\,3. P.~229--235. 
\bibitem{17-sel-1}
\Au{Trucco E.} A~note on the information content of graphs~// B.~Math. 
Biophys., 1956. Vol.~18. No.\,2. P.~129--135. 
\bibitem{18-sel-1}
\Au{Ashby W.\,R.} 
An introduction to cybernetics.~--- London, U.K.: Chapman\,\&\,Hall, 
1956. 283~p.

\pagebreak

\bibitem{19-sel-1}
\Au{Seyful-Mulyukov~R.} Quantum matrix of hydrocarbon molecules is the key 
element of petroleum genesis~// 14th Multidisciplinary Scientific Geoconference 
(International) and Expo SGEM Proceedings.~--- Varna, Bulgaria, 2014. Vol.~1. P.~759--765.


\bibitem{20-sel-1}
\Au{Ashby W.\,R.} Requisite variety and its implications for the control of complex 
systems~// Cybernetica, 1958. Vol.~1. No.\,2. P.~83--89. {\sf 
http://pespmc1.vub.ac. be/books/ashbyreqvar.pdf}. 
\bibitem{21-sel-1}
\Au{Фомин Ю.\,М.} Верхняя астеносфера~--- переходная зона между веществом 
мантии и~литосферы~// Проблемы эволюции, 2005. {\sf http://www.evolbiol.ru/fomin.htm}.
\bibitem{22-sel-1}
\Au{Seyfoul-Mulyukov R., Hlava~M.\,M.\,K.} The nature of information as a~fundamental 
property of matter: A~case study using petroleum and hydrocarbon gases~// 
3rd Multi-Conference (International) on Complexity, Informatics 
and Cybernetics Proceedings.~--- Orlando, FL, USA, 2012. P.~173--178.

\end{thebibliography}
} }

\end{multicols}

 \label{end\stat}

 \vspace*{-3pt}

\hfill{\small\textit{Поступила в~редакцию  15.10.2015}}
%\renewcommand{\bibname}{\protect\rm Литература}
\renewcommand{\figurename}{\protect\bf Рис.}
\renewcommand{\tablename}{\protect\bf Таблица} 
 
 