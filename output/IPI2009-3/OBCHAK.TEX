\def\stat{abstr}
{%\hrule\par
%\vskip 7pt % 7pt
\raggedleft\Large \bf%\baselineskip=3.2ex
A\,B\,S\,T\,R\,A\,C\,T\,S \vskip 17pt
    \hrule
    \par
\vskip 21pt plus 6pt minus 3pt }


\def\tit{MULTICHANNEL QUEUEING SYSTEM WITH REFUSALS OF SERVERS GROUPS}

%1
\def\aut{A.~Pechinkin$^1$, I.~Sokolov$^2$, and V.~Chaplygin$^3$}

\def\auf{$^1$IPI RAN, apechinkin@ipiran.ru\\[1pt]
$^2$IPI RAN, isokolov@ipiran.ru\\[1pt]
$^3$IPI RAN, VasilyChaplygin@mail.ru}

\def\leftkol{\ } % ENGLISH ABSTRACTS}

\def\rightkol{\ } %ENGLISH ABSTRACTS}

\titele{\tit}{\aut}{\auf}{\leftkol}{\rightkol}


\noindent The multichannel queueing system SM/PH/n/r ($r\leq
\infty$) with unreliable servers and their group refusals is under
consideration. The refusals and the restorations of servers groups
occur with a constant intensity, the number of servers refusing
simultaneously is a stochastic value, and customers with the
interrupted servicing begin its servicing anew after server
restoration. The methods are offered to calculate the stationary
distribution of the number of the customers in the system under
different variants of the functioning of the system.

\label{st\stat}

 \KWN{multichannel queueing systems; unreliable servers; refusals and restorations of
servers groups
}

\vskip 14pt plus 6pt minus 3pt


\def\tit{ON THE LIMITING CHARACTERISTICS FOR $M(t)/M(t)/S$ QUEUE WITH~CATASTROPHES}

%2
\def\aut{A.\,I.~Zeifman$^1$,  Ya.\,A.~Satin$^2$,
A.\,V.~Korotysheva$^3$, and N.\,A.~Tereshina$^4$}
\def\auf{$^1$Vologda State Pedagogical University, a\_zeifman@mail.ru\\[1pt]
$^2$Vologda State Pedagogical University, yacovi@mail.ru\\[1pt]
$^3$Vologda State Pedagogical University, a\_korotysheva@mail.ru\\[1pt]
$^4$Vologda State Pedagogical University, nataliya\_tereshi@mail.ru}

\def\leftkol{\ } % ENGLISH ABSTRACTS}

\def\rightkol{\ } %ENGLISH ABSTRACTS}

\titele{\tit}{\aut}{\auf}{\leftkol}{\rightkol}

\noindent
An $M(t)/M(t)/S$ queue with catastrophes is considered and it is supposed that the catastrophes rates
depend on the length of the queue. Sufficient conditions for weak ergodicity of the respective
queue-length process and the respective bounds are obtained. Some examples are also considered.

%\label{st\stat}

\KWN{nonstationary Markovian queueing system; birth and death process with catastrophes;
weak ergodicity; bounds; limiting characteristics; approximation
}

%\pagebreak

% \thispagestyle{headings}

\vskip 14pt plus 6pt minus 3pt

%\vfil

%3
\def\tit{LARGE DEVIATION ASYMPTOTICS OF STATIONARY QUEUES}

\def\aut{E.\,V. Morozov}
\def\auf{Institute of Applied Mathematical Research, Karelian Research Center of the Russian Academy of Sciences, emorozov@krc.karelia.ru}

\titele{\tit}{\aut}{\auf}{\leftkol}{\rightkol}

\noindent
The paper is a survey of main asymptotic results playing an important role
in the quality of service estimation (QoS) of stationary systems. The
asymptotics of the probability that the workload/queue-size process with
heavy tail exceeds an increasing level is considered. Similar results for
the systems with Levy input process and light-tailed service time are given.
The proofs are based on the methods of large deviations theory and illustrated
in detail by the $M/M/1$ system. The asymptotics of the overflow probability within
regeneration cycle is considered, including the multiserver systems. An asymptotic
analysis of system with the long-range dependent input is discussed, with focus on
fractional Brownian process. The ties between the long-range dependence of a queue-size
process and the moment properties of the embedded process of the regenerations are discussed.


\KWN{stationary queue; large deviation probabilities; asymptotic analysis;
light-tailed distributions; fractional Brownian process; long-range dependent process; regeneration}
%\pagebreak


%\vfil
 \vskip 18pt plus 6pt minus 3pt
% \vskip 24pt plus 9pt minus 6pt

%4
\def\tit{TRAFFIC-LEVEL PROBABILITY MODEL FOR THE NETWORK CENTRIC SYSTEM
}

\def\aut{V.\,Y.~Borodakiy}
\def\auf{National Research Nuclear University ``MEPHI,'' vladbor@inbox.ru
}


%\def\leftkol{ENGLISH ABSTRACTS}

%\def\rightkol{ENGLISH ABSTRACTS}

\titele{\tit}{\aut}{\auf}{\leftkol}{\rightkol}

\noindent
A network centric system model with datacenters serving subscribers'
requests for a data transfer is considered. Data flows, so called elastic traffic, are characterized
by a class independent bandwidth requirement and have exponentially distributed size.
Flow request arriving is modeled by a Poisson process and data flows are served in
accordance with the processor sharing discipline. A single link probability
model analysis is presented and the exact expression of the flow blocking probability is derived.

\KWN{network centric system; elastic traffic; blocking probability; single link
}
%\pagebreak

%\vful

 \vskip 12pt plus 6pt minus 3pt

% \vskip 24pt plus 9pt minus 6pt
%\vskip 6pt plus 3pt minus 3pt
%\vspace*{12pt}

%5
\def\tit{AN APPROACH TO ACTUARIAL MODELING
WITH~QUASI-MONTE CARLO: SIMULATION~OF~RANDOM SUMS DEPENDING
ON~STOCHASTIC FACTORS}


\def\aut{G.~Temnov$^1$ and S.~Kucherenko$^2$}

\def\auf{$^1$Edgeworth Centre for Financial Mathematics, University
College Cork, Ireland, g.temnov@ucc.ie\\[1pt]
$^2$CPSE, Imperial College, London, UK, s.kucherenko@ic.ac.uk}


\def\leftkol{ENGLISH ABSTRACTS}

\def\rightkol{ENGLISH ABSTRACTS}

\titele{\tit}{\aut}{\auf}{\leftkol}{\rightkol}

\noindent
The problem of estimating the characteristics of a random
sum, when the number of summands is also random, is addressed. The
considered case includes an additional stochastic factor: although
the summed random variables come from a distribution of a known
form, the parameters of this distribution are stochastic and can
themselves be viewed as random variables (with known distributions).
The Quasi-Monte-Carlo techniques are used to handle this
problem and to analyze its efficiency relative to the regular
Monte-Carlo simulation methods. The typical area of the application
of the investigations is actuarial practice which often deals with
random sums of financial losses. Besides actuarial applications, the
proposed method may be useful in application to certain problems in
informatics, related to the aggregation of heavy-tailed data.


\KWN{actuarial modeling; quasi-Monte-Carlo simulation; random sums}

%\vskip 18pt plus 6pt minus 3pt

 \vskip 12pt plus 6pt minus 3pt

% \pagebreak

%6
\def\tit{ON STABILITY OF IMAGE RECONSTRUCTION IN THE PROBLEMS OF EMISSION TOMOGRAPHY
}

\def\aut{O.\,V.~Shestakov}
\def\auf{Department of Mathematical Statistics, Faculty of
Computational Mathematics and Cybernetics,\\ M.\,V.~Lomonosov Moscow State University, oshestakov@cs.msu.su}

\titele{\tit}{\aut}{\auf}{\leftkol}{\rightkol}

\noindent
This paper deals with the problem of reconstructing images from projections
in emission tomography settings. Within the frames of given mathematical model,
the closeness estimates are derived for reconstructed images when using finite number of projections.

\KWN{emission tomography; Radon transform; projections; closeness estimates}
%\pagebreak

 \vskip 12pt plus 6pt minus 3pt

%7
\def\tit{ON PROBABILISTIC ASPECTS OF ERROR CORRECTION CODES WHEN THE NUMBER OF ERRORS IS A RANDOM SET
}


\def\aut{A.\,N.~Chuprunov$^1$ and B.\,I.~Khamdeyev$^2$}
\def\auf{$^1$Research Institute of Mathematics and Mechanics, Kazan State University,
achuprunov@mail.ru\\[1pt]
$^2$Research Institute of Mathematics and Mechanics, Kazan State University,
Khamdeyevbi@mail.ru}

\titele{\tit}{\aut}{\auf}{\leftkol}{\rightkol}

\noindent
In the paper, $n$ messages each containing $N$ blocks are considered. Each
block is encoded with some antinoise coding method, which can
correct not more than $q$ mistakes. Here, it is assumed that the number of
mistakes lies in some random subset $N_i(\omega_1)$,
$\omega_1\in\Omega_1$ of integer numbers. The probability
${\bf P}(A)$ of the event $A$ is studied which means that all the mistakes
would be corrected. Probability ${\bf P}(A)$ is formulated in
terms of conditional probabilities.
 It is shown that as $n, N\to\infty$ so that
$\alpha=n/N\to \alpha_0<\infty$, at $q=1$, probabilities
${\bf P}(A)$ converge at almost all $\omega_1\in\Omega_1$. The
limit is obtained.


\KWN{generalized allocation scheme; convergence almost sure; Hamming code}
%\pagebreak

 \vskip 12pt plus 6pt minus 3pt

%8
\def\tit{ON THE DISTRIBUTION OF PARTICLE SIZE UNDER FRACTURING
}

\def\aut{V.\,Y.~Korolev}
\def\auf{Department of Mathematical Statistics, Faculty of Computational
Mathematics and Cybernetics,\\ M.\,V.~Lomonosov Moscow State University; Institute of
Informatics Problems, Russian Academy of Sciences}

\titele{\tit}{\aut}{\auf}{\leftkol}{\rightkol}

\noindent
A new model for the distribution of particle size under fracturing
with the account of nonconstant or random character of the intensity
of the flow of impacts. Within the framework of this model, a criterion
of the log-normality of the distribution under consideration is formulated
and the class of possible distributions of particle size under fracturing is
described. Along with many known models, this class contains scale mixtures of log-normal laws.


\KWN{log-normal distribution; mixtures of normal distributions; compound Cox process}
%\pagebreak

 \vskip 12pt plus 6pt minus 3pt

%9

\def\tit{SOME ESTIMATES FOR CHARACTERISTIC FUNCTIONS WITH AN APPLICATION TO~SHARPENING THE MISES INEQUALITY
}

\def\aut{I.\,G.~Shevtsova}
\def\auf{Faculty of Computational
Mathematics and Cybernetics, M.\,V.~Lomonosov Moscow State University,\linebreak ishevtsova@cs.msu.su}

\titele{\tit}{\aut}{\auf}{\leftkol}{\rightkol}

\noindent
New estimates are constructed for characteristic functions of distributions
with finite absolute moments of order $2+\delta$,\ \  $0\le\delta\leq 1$.
The Mises moment inequality for lattice distributions is also improved.

\KWN{Fourier transform; characteristic function; symmetrization;
convolution; lattice distribution; arithmetic distribution; span}
%\pagebreak



\vskip 12pt plus 6pt minus 3pt

%10
\def\tit{ON THE POWER OF THE TESTS IN THE CASE OF GENERALIZED LAPLACE DISTRIBUTION
}
\def\aut{V.\,E.~Bening$^1$ and O.\,O.~Lyamin$^2$}


\def\auf{$^1$Faculty of Computational Mathematics and Cybernetics,
M.\,V.~Lomonosov Moscow State University,\newline
\hphantom{$^1$}bening@yandex.ru\\[1pt]
$^2$Faculty of Computational Mathematics and Cybernetics,
M.\,V.~Lomonosov Moscow State University
}

%\def\leftkol{ENGLISH ABSTRACTS}

%\def\rightkol{ENGLISH ABSTRACTS}

\titele{\tit}{\aut}{\auf}{\leftkol}{\rightkol}

\noindent
In the paper, the formula for the normalized limit difference between
the power of asymptotically most powerful test and that of a locally most powerful
test in the case of the generalized Laplace distribution is obtained.


\KWN{generalized Laplace or
generalized double exponential distribution; power
function; deficiency; asymptotic expansion
}


 \label{end\stat}
 %\pagebreak