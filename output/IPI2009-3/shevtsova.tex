
%\newcommand{\eps}{\varepsilon}
%\renewcommand{\varkappa}{\varkappa}
%\renewcommand{\fii}{\varphi}
\renewcommand{\le}{\leqslant}
\renewcommand{\ge}{\geqslant}
\renewcommand{\d}{\delta}

\newcommand{\R}{\mathbb R}


\def\stat{shevtsova}


\def\tit{НЕКОТОРЫЕ ОЦЕНКИ ДЛЯ ХАРАКТЕРИСТИЧЕСКИХ ФУНКЦИЙ С ПРИМЕНЕНИЕМ
К~УТОЧНЕНИЮ  НЕРАВЕНСТВА МИЗЕСА$^*$}
\def\titkol{Некоторые оценки для характеристических функций с применением
к~уточнению  неравенства Мизеса}

\def\autkol{И.\,Г.~Шевцова}
\def\aut{И.\,Г.~Шевцова$^1$}

\titel{\tit}{\aut}{\autkol}{\titkol}

{\renewcommand{\thefootnote}{\fnsymbol{footnote}}\footnotetext[1]
{Работа выполнена при
поддержке Российского фонда фундаментальных исследований, проекты
08-01-00563, 08-01-00567,\linebreak  08-07-00152 и 09-07-12032-офи-м,
 а также Совета по грантам
президента Российской Федерации для поддержки молодых российских
ученых~-- кандидатов наук, проект МК-654.2008.1.}}

\renewcommand{\thefootnote}{\arabic{footnote}}
\footnotetext[1]{Факультет
вычислительной математики и кибернетики Московского государственного
университета им. М.~В.~Ломоносова, ishevtsova@cs.msu.su}

\Abst{Построены новые оценки для модулей
характеристических функций распределений, имеющих моменты порядка
$2+\d$, $0<\d\le1$. Уточнено моментное неравенство Мизеса для
решетчатых распределений.}

\KW{преобразование Фурье; характеристическая
функция; симметризация; свертка; решетчатое распределение;
арифметическое распределение; шаг распределения}


      \vskip 24pt plus 9pt minus 6pt

      \thispagestyle{headings}

      \begin{multicols}{2}

      \label{st\stat}

\section{Введение}
\vspace*{6pt}

При построении вероятностных моделей стоха\-стических ситуаций часто в
качестве аналити\-ческой модели распределения вероятностей
используется его асимптотическая аппроксимация, воз\-ни\-ка\-ющая в
соответствующей предельной теореме. Примером может служить
повсеместное применение нормальной аппроксимации, основанное на
центральной предельной теореме. Однако при этом столь же часто
обходится стороной вопрос о точ\-ности этой аппроксимации, играющий
важную роль при обосновании адекватности вероятностной модели.
Данная работа посвящена уточнению оценок для характеристических
функций~--- основного аппарата, используемого при построении
конкретных числовых оценок точности нормальной аппроксимации.

Пусть $F(x)$, $x\in\R$,~--- функция распределения с
характеристической функцией $f(t)$ такая, что

\vspace*{6pt}

\noindent
\begin{equation}
\left.
\begin{array}{rl}
\displaystyle\int\limits_{-\infty}^\infty x\,dF(x)&=0\,;\\
\sigma^2&\equiv\displaystyle\int\limits_{-\infty}^\infty x^2\,dF(x)>0\,;\\
\beta_{2+\d}&\equiv\displaystyle\int\limits_{-\infty}^\infty
|x|^{2+\delta}\,dF(x)<\infty
\end{array}
\right \}
\label{EDX1bet}
\end{equation}

\vspace*{6pt}

\noindent
для некоторого $0<\d\le1$.

В 1965~г.\ В.\,М.~Золотарёв~\cite{Zolotarev1965} доказал, что при
$\delta=1$ для $|f(t)|$ справедлива оценка

\noindent
\begin{equation}
\label{OcenkaZolotareva1965}
\ln|f(t)|\le -\fr{\sigma^2t^2}2+ \fr{\beta_3|t|^3}3\,,\quad
t\in\R\,.
\end{equation}
Два года спустя он уточнил~\cite{Zolotarev1967a} свой результат,
показав, что коэффициент при $\beta_3|t|^3$ можно снизить примерно
до $0.1983<1/3$:
\begin{equation}
\label{OcenkaZolotareva1967}
\ln|f(t)|\le -\fr{\sigma^2t^2}2+ 2\varkappa\beta_3|t|^3\,,\quad t\in\R\,,
\end{equation}
где
$$
\varkappa= \sup_{x>0}\fr{\left|\cos x-1+x^2/2\right|}{x^3}=0{,}099161\ldots
$$
(если быть совсем точными, в работе~\cite{Zolotarev1967a}
неравенство~\eqref{OcenkaZolotareva1967} упомянуто без
доказательства). Отношение коэффициентов при $|t|^3$
в~\eqref{OcenkaZolotareva1965} и~\eqref{OcenkaZolotareva1967}
составляет

\noindent
$$
\fr{1}{6\varkappa}=1{,}6807\ldots
$$

Шесть лет спустя Х.~Правитц~\cite{Prawitz1973} доказал неравенство

\noindent
\begin{equation}
\label{OcenkaPrawitza|f(t)|^2step}
|f(t)|^2\le1 -\sigma^2t^2+
2\varkappa\left(\beta_3+\sigma^3\right)|t|^3\,,\quad t\in\R\,,
\end{equation}
из которого вытекает еще одно уточнение оценки Золотарёва:

\noindent
\begin{equation}
\label{OcenkaPrawitzaLn|f(t)|step}
\ln|f(t)|\le -\fr{\sigma^2t^2}2+
\varkappa\left(\beta_3+\sigma^3\right)|t|^3\,,\quad t\in\R\,.
\end{equation}
Отношение коэффициентов при $|t|^3$ в~\eqref{OcenkaZolotareva1967}
и~\eqref{OcenkaPrawitzaLn|f(t)|step} в силу неравенства Ляпунова не
меньше единицы и может быть сколь угодно близко к двум, если
отношение Ляпунова $\beta_3/\sigma^3$ велико, а именно справедливы
следующие соотношения:

\noindent
$$
1=\inf_{F}\fr{2\beta_3}{\beta_3+\sigma^3}\le \sup\limits_{F}\fr{2\beta_3}{\beta_3+\sigma^3}=2\,,
$$
где инфимум и супремум берутся по всем распределениям $F$,
удовлетворяющим условиям~\eqref{EDX1bet} с $\d=1$.

Появление оценки~\eqref{OcenkaZolotareva1967} оказало значительное
влияние на результаты вычисления абсолютной константы $C$ в широко
известном неравенстве Берри--Эссеена~\cite{Berry1941, Esseen1942},
согласно которому для равномерного расстояния между $n$-кратной
норми\-рованной сверткой распределений~$F$, удов\-ле\-тво\-ря\-ющих
условиям~\eqref{EDX1bet} с $\d=1$, и стандартным\linebreak
 нормальным законом~$\Phi$ справедливо соотношение
$$
\sup_x\left|F^{*n}(x\sigma\sqrt n)-\Phi(x)\right|\le C
L_n^3\,;\quad L_n^3=\fr{\beta_3}{\sigma^3\sqrt{n}}\,.
$$
А именно использование более точной
оценки~\eqref{OcenkaZolotareva1967} позволило Золотарёву существенно
улучшить мажоранту константы~$C$ в этом неравенстве с
$1{,}301$~\cite{Zolotarev1966} до $0{,}82$~\cite{Zolotarev1967a, Zolotarev1967b},
при этом в работе~\cite{Zolotarev1967a} сам автор
отмечает: <<{\it В настоящее время удалось осуществить дальнейшее
снижение верхних оценок константы~$C$ за счет уточнения весьма
важной для выбранного метода леммы~$4$
работы}~$\cite{Zolotarev1965}$>> (суть упомянутой леммы~---
утверждение неравенства~\eqref{OcenkaZolotareva1965}).

В 1975~г.\ при решении аналогичной задачи об отыскании верхней
\textit{асимптотической} оценки константы~$C$, справедливой для
достаточно малых значений ляпуновской дроби~$L_n^3$,
Правитц~\cite{Prawitz1975} получил результат
\begin{equation}
\label{PrawitzAsymptBEconst}
C\le0{,}5152\,, \mbox{ если } L_n^3\le 0{,}0985\,.
\end{equation}
При этом существенную роль сыграла возможность использования
оценки~\eqref{OcenkaPrawitza|f(t)|^2step} и приводимой ниже
оценки~\eqref{OcenkaPrawitza|f(t)|^2cos}.

Неравенство Берри--Эссеена является наиболее популярным среди оценок
скорости сходимости в центральной предельной теореме и играет
значительную роль при решении конкретных задач, возникающих в теории
надежности, управлении запасами, страховой и финансовой математике и
многих других важных областях прикладной науки. Успех в применении
вероятностных моделей к решению этих задач во многом зависит от
точности используемых методов. Таким образом, актуальность задачи
несомненна.

В то же время при решении задач, связанных с большими рисками,
приходится отказываться от условия $\beta_3<\infty$ и каким-то
образом переносить полученные результаты на случай ${\d<1}$.

К одной из первых работ в этой области следует отнести диссертацию
В.~Тысиака, в которой фактически  была обобщена оценка
Правитца~\eqref{OcenkaPrawitza|f(t)|^2step} для $0<\delta<1$:
\begin{multline}
\label{OcenkaTysiaka1983|f(t)|^2}
|f(t)|^2\le1-\sigma^2t^2+
2\varkappa(\d)\left(\bet+\sigma^{2+\d}\right)|t|^{2+\d}\,,\\
t\in\R\,,
\end{multline}
где
$$
\varkappa(\d)=\sup_{x>0}\fr{\left|\cos
x-1+x^2/2\right|}{x^{2+\d}}\,,\quad  0<\d\le 1\,,
$$
хотя автор анонсирует и использует полученный результат лишь в форме
Золотарёва~\eqref{OcenkaZolotareva1967}:
$$
\ln |f(t)|\le-\fr{\sigma^2t^2}2+ 2\varkappa(\d)\bet|t|^{2+\d}\,,\quad
t\in\R\,.
$$
Применяя последний к неравенству сглаживания Золотарёва,
Тысиак~\cite{Tysiak1983} в 1983~г.\ получил верхние оценки
константы в аналоге неравенства Берри--Эссеена для распределений, не
имеющих третьего момента.
Доказательство неравенства~\eqref{OcenkaTysiaka1983|f(t)|^2} также
можно найти в монографии Ушакова~\cite{Ushakov1999}.

В упомянутой выше работе~\cite{Prawitz1973} Правитц получил еще одну
оценку, на которую существенно опирается доказательство
асимптотического результата~\eqref{PrawitzAsymptBEconst}:
\begin{align}
|f(t)|^2 & \le 1- 2 \left(\fr{\sigma^2}{\beta_3+\sigma^3}\right)^2
\left(1-\cos\fr{\left(\beta_3+ \sigma^3\right)t}{\sigma^2}
\right),\notag\\
&\label{OcenkaPrawitza|f(t)|^2cos} \\[-6pt]
\theta_0 &\le \left(\fr{\beta_3+\sigma^3}{\sigma^2}\right)|t| \le 2\pi\,,\notag
\end{align}
где $\theta_0=3{,}5995896\ldots$~--- единственный корень уравнения
$$
3(1-\cos\theta)-\theta\sin\theta-\fr{\theta^2}{2}=0\,,
$$
лежащий в интервале $(\pi,2\pi)$. Кроме того, в указанной работе при
каждом $t\in\R$ был найден минимум из трех оценок для $|f(t)|$,
устанавливаемых неравенствами~\eqref{OcenkaPrawitza|f(t)|^2step},
\eqref{OcenkaPrawitza|f(t)|^2cos} и $|f(t)|\le1$, и показано, что
наилучшая в этом смысле оценка имеет вид (для упрощения обозначений
здесь полагаем $\sigma^2=1$):
$$
|f(t)|\le
\begin{cases}
1-\fr{t^2}2+\varkappa(\d)\left(\beta_3+
 1\right)|t|^3\,,\hspace*{10pt}   |t|\le\fr{\theta_0}{\beta_3+1};\hspace*{-8pt} \\
1-\fr{1-\cos\left((\beta_3+1) t\right)}
 {(\beta_3+1)^2}\, , \\
\hspace*{88pt} \theta_0\le (\beta_3+1)|t|  \le 2\pi;\hspace*{-8pt} \\
    1, \hspace*{115pt} |t|\ge\fr{2\pi}{\beta_3+1}\,.\hspace*{-8pt}
\end{cases}
$$
Отсюда, кстати, вытекает  (см.\ теорему~5), что
максимальный шаг~$h$ решетчатых распределений
удов\-ле\-тво\-ря\-ет соотношению:
\begin{equation}
\label{PrawitzUtochnilMisesa}
\sigma^2h\le \beta_3+\sigma^3\,,
\end{equation}
в то время как широко известное неравенство Мизеса~\cite{Mises1939}
устанавливает лишь более грубую оценку
\begin{equation}
\label{MisesInequality}
\sigma^2h\le 2\beta_3\,.
\end{equation}
Как известно, неравенство Мизеса является в определенном смысле
точным: знак равенства в нем достигается на симметричном
распределении Бернулли, для которого $\beta_3/\sigma^3=1$,~--- и это
единственный случай, когда правые
части~\eqref{PrawitzUtochnilMisesa} и~\eqref{MisesInequality}
совпадают.
В остальных же случаях неравенство~\eqref{MisesInequality} строгое.

В данной работе обобщен результат
Правитца~\eqref{OcenkaPrawitza|f(t)|^2cos} на случай $0<\d<1$ (здесь
для упрощения обозначений снова полагаем $\sigma^2=1$):
\begin{align}
|f(t)|^2 & \le1- 2 \fr{1-\cos\left((\bet+1)^{1/\d} t\right)}
 {(\bet+1)^{2/\d}}\,;\notag\\[-6pt]
 &\label{OcenkaShevtsovoi|f(t)|^2cos}\\
\theta_0(\d) & \le (\bet+1)^{1/\d}|t|  \le2\pi\,,\notag
\end{align}
где $\theta_0(\d)$~--- единственный корень уравнения
$$
\fr{\d\theta^2}2+ \theta\sin \theta + (2+\d)(\cos \theta - 1)=0\,,
$$
лежащий в интервале $(\pi,\,2\pi)$. Также в данной работе показано,
что минимум из трех оценок для $|f(t)|$, устанавливаемых
неравенствами~\eqref{OcenkaTysiaka1983|f(t)|^2},
\eqref{OcenkaShevtsovoi|f(t)|^2cos} и $|f(t)|\le1$, имеет вид
$$
|f(t)|\le
\begin{cases}
 1-\fr{t^2}2+\varkappa(\d)\left(\beta_{2+\d}+
 1\right)|t|^{2+\d}\,, \\[6pt]
\hspace*{95pt} |t|\le\fr{\theta_0(\d)}{(\bet+1)^{1/\d}}\,;\hspace*{-8pt}  \\[9pt]
 1-\fr{1-\cos\left((\bet+1)^{1/\d} t\right)}
 {(\bet+1)^{2/\d}}\,,\\[9pt]
 \hspace*{52pt} \theta_0(\d)\le (\bet+1)^{1/\d}|t|  \le2\pi\,;\hspace*{-8pt}  \\[9pt]
    1\,, \hspace*{84pt} |t|\ge\fr{2\pi}{(\bet+1)^{1/\d}}\,.\hspace*{-8pt}
\end{cases}
$$

В качестве следствия полученной оценки для~$|f(t)|$ в данной работе
доказано более общее по сравнению с~\eqref{PrawitzUtochnilMisesa}
неравенство:
\begin{equation}
\label{MisesImproveGeneralize}
\fr{h}{\sigma}\le\left(\fr{\bet}{\sigma^{2+\d}}+
1\right)^{1/\d}\,,
\end{equation}
справедливое для любых решетчатых распределений, удовлетворяющих
условиям~\eqref{EDX1bet} с произвольным $0<\d\le1$. Также показано,
что неравенство~\eqref{MisesImproveGeneralize} не может быть
улучшено ни при каком $0<\d\le1$.

\section{Вспомогательные утверждения}

При построении верхних оценок для вещественных характеристических
функций успех во многом зависит от того, насколько точную мажоранту
удается построить для функции $\cos x$, $x\in\R$. Если при этом
предполагается существование у рас\-смат\-ри\-ва\-емо\-го распределения
абсолютного момента некоторого порядка $s>0$ (возможно, дробного), то
мажоранта для~$\cos x$ может включать степенные члены порядка не
выше~$s$. Особый интерес с точки зрения центральной предельной
теоремы пред\-став\-ля\-ет случай $s\ge 2$. Из разложения функции
$\cos x$ в ряд Маклорена вытекает, что ее поведение в окрест\-ности
нуля с точностью до бесконечно малых более высокого порядка
совпадает с поведением функции $(1-x^2/2)$. Поэтому при
конструировании верхней оценки, имеющей вид полинома, необходимо
включать в нее слагаемое вида $(1-ax^2)$ с $0\le a\le1/2$ (случай
$a<0$ приводит к слишком грубой оценке в окрестности нуля, и потому
интереса не представляет). С другой стороны, при больших~$x$
добиться выполнения неравенства $\cos x\le 1-ax^2$ с $a>0$ не
представляется возможным, поэтому в мажоранту необходимо включить
еще слагаемое вида~$x^s$ с неотрицательным коэффициентом.

Пусть теперь $s=2+\delta$, где $0<\delta\le1$. Тогда при каждом
$0\le a\le1/2$ найдется число $b=b(a)\ge 0$ такое, что для всех
$x\in\R$ выполняется неравенство
$$
\cos x\le 1-ax^2+b|x|^{2+\delta}\,.
$$
При этом минимально возможное значение~$b$, гарантирующее выполнение
указанного неравенства, имеет вид
$$
b(a)=\sup_{x\in\R}\fr{|\cos x-1+ax^2|}{|x|^{2+\delta}}\,,\quad 0\le
a\le \fr{1}{2}\,,
$$
другими словами, для любого $0\le a\le1/2$
$$
\inf_{x\in\R}(1-ax^2+b(a)|x|^{2+\delta}-\cos x)=0\,.
$$
Следующее утверждение дает более удобное параметрическое
представление пары ${(a,\,b(a))}$, ${0\le a\le 1/2}$, являющейся
решением описанной задачи.

\medskip

\noindent
\textbf{Лемма 1.}
\textit{Для произвольного $0<\d\le1$ и всех $x\in\R$ справедливо неравенство}
\begin{multline}
\label{OcenkaCos}
\cos x\le1-a(\d,\theta)x^2+b(\d,\theta)|x|^{2+\d}\,, \\
\theta_0(\d)\le\theta\le 2\pi\,,
\end{multline}
\textit{где}
\begin{align}
    a(\d,\theta)&=\fr{2+\d}{\d}\,\fr{1-\cos \theta}{\theta^2}-
    \fr{1}{\delta}\,\fr{\sin \theta}{\theta}\,;\label{CosA}\\
    b(\d,\theta)&=\fr{2}{\delta}\,\fr{1-\cos \theta}{\theta^{2+\d}}-
    \fr{1}{\d}\,\fr{\sin\theta}{\theta^{1+\d}}\,;\label{CosB}
\end{align}
\textit{$\theta_0(\d)$~--- единственный корень уравнения $a(\d,\theta)=1/2$,
лежащий в интервале $(\pi,\,2\pi)$.}
\medskip

Данное утверждение впервые было получено при $\d=1$ в работе
Правитца~\cite{Prawitz1973}. Для других~зна\-че\-ний~$\d$
лемма~1 была доказана лишь в три\-виаль\-ном случае
${\theta=\theta_0(\d)}$ (c~${a(\d,\theta)=1/2}$) в диссертации
Тысиака~\cite{Tysiak1983}, а также, в несколько ином виде, в
монографии Ушакова~\cite{Ushakov1999}. Для произвольных\linebreak
${0<\d<1}$ и ${\theta\in(\theta_0(\d),\,2\pi]}$ данное утверждение
пуб\-ли\-ку\-ет\-ся, по-видимому, впервые. Доказательству
леммы~1 будет предпослано несколько вспомогательных
утверждений.

\medskip

\noindent
\textbf{Лемма 2.}
\label{FiOdin}
\textit{Функция}
$$
\varphi(x)=\sin x -x\cos x\,,\quad 0<x\le2\pi\,,
$$
\textit{имеет единственный корень $x_0$, лежащий в интервале
$\left(\pi,\,3\pi/2\right)$, причем}
\begin{align*}
  \varphi(x)&>0\,, &\hspace*{-30pt}  0&<x<x_0\,; \\
  \varphi(x)&<0\,, &\hspace*{-30pt}  x_0&<x\le 2\pi\,.
\end{align*}

\medskip

Данное утверждение является очевидным следствием свойств функции~$\mathrm{tg}\,x$.

\medskip

\noindent
\textbf{Лемма 3.}
\label{FiDva}
\textit{Функция}
$$
\varphi(x)=\fr{\sin x}{x} -\fr{1+\cos x}{2}\,,\quad 0<x\le 2\pi\,,
$$
\textit{имеет единственный корень $x_1=\pi,$ причем}
\begin{align*}
  \varphi(x)&>0\,, &\hspace*{-30pt}0&<x<\pi\,; \\
  \varphi(x)&<0\,, &\hspace*{-30pt} \pi&<x\le2\pi\,.
\end{align*}

\medskip

\noindent
Д\,о\,к\,а\,з\,а\,т\,е\,л\,ь\,с\,т\,в\,о\,.\ Знак $\varphi(x)$ определяется знаком выражения
\begin{multline}
\label{FiDvaMest}
2\sin x-x(1+\cos x)= 4\sin\fr{x}{2} \cos\fr{x}{2}-{}\\
{}-2x\cos^2\fr{x}{2}=
4\cos\fr {x}{2}\left(\sin\fr{x}{2}-\fr{x}{2}\cos\fr{x}{2}\right)\,.
\end{multline}
Согласно лемме~2,
$$
\sin\fr{x}{2}-\fr{x}{2}\cos\fr{x}{2}>0\,,\quad 0<x\le 2\pi\,,
$$
поэтому знак~\eqref{FiDvaMest} определяется множителем $\cos(x/2)$,
который положителен при ${x\in(0,\pi)}$ и отрицателен при
$x\in(\pi,2\pi)$.

\medskip

\noindent
\textbf{Лемма 4.}\ \textit{Для произвольного $0\le p\le 1/2$ функция}
$$
\varphi_p(x)=\fr{\sin x}{x}-(p+(1-p)\cos x)\,,\quad 0<x\le2\pi\,,
$$
\textit{имеет единственный корень $x_p\in\left[\pi,3\pi/2\right)$, причем}
\begin{align*}
 \varphi_p\,(x)&>0\,, &\hspace*{-30pt}  0&<x<x_p\,; \\
  \varphi_p\,(x)&<0\,, &\hspace*{-30pt} x_p&<x\le2\pi\,.
\end{align*}

\medskip

\noindent
Д\,о\,к\,а\,з\,а\,т\,е\,л\,ь\,с\,т\,в\,о\,.\
Случаи $p=0$ и $p=1/2$ рассмотрены в леммах~2
и~3 соответственно, поэтому будем считать, что $0<p<1/2.$
Разобьем промежуток $(0,2\pi]$ на три части и рассмотрим поведение
функции~$\varphi_p(x)$ на каждом из них.

Для $x\in(0,\pi)$ в силу леммы~3 имеем
\begin{multline*}
p+(1-p)\cos x=\cos x+p(1-\cos x)\le{}\\
{}\le
\cos x+\fr {1}{2}\left( 1-\cos x\right )=
\fr{1+\cos x}{2}<\fr{\sin x}{x}\,;
\end{multline*}
следовательно, $\varphi_p(x)>0$ для $0<x<\pi.$ Заметим, что
$\varphi_p(\pi)=1-2p>0$, поэтому
$$
\varphi_p(x)>0\,,\quad 0<x\le\pi\,.
$$

Пусть $x_0$~--- единственный корень уравнения
$$
\sin x=x\cos x\,,\quad 0<x\le\pi\,,
$$
согласно лемме~2 лежащий в интервале ${(\pi,\, 3\pi/2)}$.
Тогда $\sin x/x<\cos x$ для ${x_0<x\le 2\pi}$ и
\begin{multline*}
p+(1-p)\cos x=\cos x+p(1-\cos x)\ge{}\\
{}\ge \cos x>\fr{\sin x}{x}\,;
\end{multline*}
следовательно, $\varphi_p(x)<0$ для $x_0<x\le2\pi.$ В точке~$x_0$
имеем
\begin{multline*}
\varphi_p(x_0)=\fr{\sin x_0}{x_0}-(p+(1-p)\cos x_0)={}\\
{}=  \cos x_0-(p+(1-p)\cos x_0)=p(\cos x_0-1)<0\,,
\end{multline*}
поскольку $x_0\in(\pi,\,3\pi/2)$, а значит,
$$
\varphi_p(x)<0\,,\quad x_0\le x\le 2\pi\,.
$$

На концах интервала $\pi< x< x_0$ непрерывная функция~$\varphi_p$
принимает значения разных знаков, поэтому хотя бы в одной точке
обращается в нуль. Единственность нуля вытекает из того, что функция~$\sin x/x$
монотонно убывает при $\pi< x< x_0$ (поскольку ее производная
$$
\left(\fr{\sin x}x\right)^\prime=\fr{x\cos x-\sin x}{x^2}
$$
в силу леммы~2 отрицательна при $x<x_0$), а функция
$p+(1-p)\cos x$ монотонно возрастает при $\pi< x< x_0$.

\medskip

\noindent
Д\,о\,к\,а\,з\,а\,т\,е\,л\,ь\,с\,т\,в\,о\ л\,е\,м\,м\,ы\,\ 1.\
В силу четности фигурирующих в неравенстве~\eqref{OcenkaCos} функций
можно считать, что $x>0$ (для $x=0$ неравенство~\eqref{OcenkaCos}
очевидно).

Рассмотрим сначала $0<x\le 2\pi.$ Введем функцию
$$
P_{a,b}(x)=1-ax^2+bx^{2+\d}-\cos x\,,\quad 0<x\le 2 \pi\,.
$$
Тогда неравенство~\eqref{OcenkaCos} для указанных $x$ равносильно уравнению
$$
\inf_{0< x\le2\pi} P_{a,b}(x)=0\,,
$$
решением которого является множество пар $(a(\theta),b(\theta))$,
${\theta\in(0,2\pi]}$, удовлетворяющих сле\-ду\-ющим условиям:
\begin{align*}
P_{a,b}(\theta)&=0\,;\\
\fr {d}{d\theta}P_{a,b}(\theta)&=0\,;\\
\fr {d^2}{d\theta^2}P_{a,b}(\theta)&>0\,;
\end{align*}
\vspace*{-6pt}

\noindent
$$
0 \le a \le\ 1/2\,,\quad b \ge0\,.
$$
Первые два уравнения равносильны системе
\begin{align}
a\theta^2&= 1-\cos\theta+b\theta^{2+\d}\,;
\label{CosANepolnoe}\\
b&=\fr{2a\theta-\sin\theta}{(2+\d)\theta^{1+\d}}\,,
\label{CosBNepolnoe}
\end{align}
решение которой задается формулами~\eqref{CosA} и~\eqref{CosB}.
Поскольку $\d$ фиксировано, для краткости будем опускать аргумент~$\d$
у функций $a=a(\d,\theta)$ и $b=b(\d,\theta)$, используя
обозначения~$a(\theta)$ и~$b(\theta)$.

Проверим условие неотрицательности~$b(\theta)$ и~$a(\theta)$.
Поскольку $\theta>0$, знак~$b(\theta)$ определяется знаком выражения
\begin{multline*}
\d\theta^{2+\d}b(\theta)=2(1-\cos \theta)-\theta\sin\theta={}\\
{}=
4\sin^2\fr {\theta}{2}-2\theta\sin\fr{\theta}{2}\cos\fr{\theta}{2}={}\\
{}=
4\sin\fr{\theta}{2}
\left(\sin\fr{\theta}{2}-\fr{\theta}{2}\cos\fr{\theta}{2}\right)\,,
\end{multline*}
которое, в свою очередь, принимает только неотрицательные значения,
так как ${\sin(\theta/2)\ge0}$ при всех ${0<\theta\le2\pi}$, а
выражение в скобках, согласно лемме~2, положительно для
указанных~$\theta$. Таким образом, ${b(\theta)\ge0}$ для всех
${0<\theta\le2\pi}$. Отсюда немедленно вытекает неотрицательность
$a(\theta)$, поскольку в силу~\eqref{CosANepolnoe} имеем
\begin{multline*}
\theta^2 a(\theta)=
1-\cos\theta+b(\theta)\theta^{2+\d}\ge{}\\
{}\ge 1-\cos\theta\ge0\,,\quad
0<\theta\le2\pi\,.
\end{multline*}

Условие $a(\theta)\le 1/2$ равносильно неравенству
\begin{multline*}
g(\theta)\equiv\d\theta^2(a(\theta)-1/2)={}\\
{}=
(2+\d)(1-\cos\theta)-\theta\sin\theta-\fr{\d\theta^2}{2} \le 0\,.
\end{multline*}
Заметим, что
$$
g(0)=0\,;\quad g(2\pi)=-2\pi^2\d<0\,;
$$
\begin{multline*}
g^\prime(\theta)=(2+\d)\sin\theta-\sin\theta-\theta\cos\theta-\d\theta={}\\
{}=(1+\d)\left[\fr{\sin
\theta}{\theta}-\left(\fr{\d}{1+\d}+\fr{1}{1+\d}\cos \theta\right)\right]\,.
\end{multline*}
Поскольку $0<\d/(1+\d)\le1/2$, в силу леммы~4 заключаем,
что на интервале $(0,2\pi)$ функция~$g^\prime(\theta)$ имеет единственный
нуль $\theta_1(\d)\in[\pi,3\pi/2)$, причем
\begin{align*}
  g^\prime(\theta)&>0\,, &\hspace*{-20pt} 0&<\theta<\theta_1(\d)\,; \\
  g^\prime(\theta)&<0\,, &\hspace*{-20pt} \theta_1(\d)&<\theta\le2\pi\,.
\end{align*}
Следовательно, $g(\theta)$ возрастает на $(0,\theta_1(\d))$ и
убывает на $(\theta_1(\d),2\pi)$, меняя знак в единственной точке
${\theta_0(\delta)>\theta_1(\delta)}$. Отсюда вытекает, что
${a(\d,\theta)\le1/2}$ для ${\theta\in[\theta_0(\d),\,2\pi]}$, где
$\theta_0(\d)$~--- единственный корень уравнения ${g(\theta)=0}$ на
интервале ${0<\theta\le2\pi}$, равносильного уравнению
$a(\d,\theta)=1/2$. При этом
$\theta_0(\delta)\in(\theta_1(\d),2\pi)\subset[\pi,2\pi)$ для всех
$\d\in(0,1]$.

Проверим положительность второй производной функции~$P_{a,b}(\theta)$
для задаваемых формулами~\eqref{CosA}
и~\eqref{CosB} значений ${a=a(\theta)}$, ${b=b(\theta)}$ на отрезке
$[\theta_0(\delta),\,2\pi]$. Рассмотрим функцию
\begin{multline*}
h(\theta)\equiv\left.
\theta^2\fr{d^2}{d\theta^2}\,P_{a,b}(\theta)\right\vert_{\substack{{a=a(\theta)}\\ {b=b(\theta)}}} ={}\\
 {}=
\theta^2(-2a(\theta)+(1+\d)(2+\d)\theta^{\d}b(\theta)+\cos \theta)={}\\
{}=2(2+\d)(1-\cos\theta)-
(3+\d)\theta\sin\theta+{}\\
{}+\theta^2\cos\theta\,,\quad
\theta_0(\delta)\le\theta\le 2\pi\,.
\end{multline*}
Покажем сначала, что $h(\theta)$ принимает положительные значения на
концах рассматриваемого отрезка. Для краткости будем опускать
аргумент~$\d$ у функции~$\theta_0(\delta)$. Имеем
\begin{gather*}
h(2\pi)=(2\pi)^2>0\,,\\
h(\theta_0)= \theta_0^2(-2a(\theta_0)+
(1+\d)(2+\d)\theta_0^{\d}b(\theta_0)+\cos\theta_0).
\end{gather*}
Учитывая условие $a(\theta_0)=1/2$ и
соотношение~\eqref{CosBNepolnoe}, запишем $h(\theta_0)$ в виде
\begin{multline*}
h(\theta_0)=
\theta_0^2\left(
-1+(1+\d)\fr{\theta_0-\sin\theta_0}{\theta_0}+
\cos\theta_0\right)={}\\
{}=
-(1+\d)\theta_0^2\left[\fr{\sin\theta_0}{\theta_0}-
\left(\fr{\d}{1+\d}+\fr{1}{1+\d}\cos\theta_0\right)\right]\equiv{}\\
{}\equiv -\theta_0^2
g^\prime(\theta_0)\,.
\end{multline*}
Как было замечено выше, $g'(\theta)<0$ при
${\theta_1(\delta)<\theta\le2\pi}$, а также
$\theta_0(\delta)>\theta_1(\delta)$ для всех $0<\delta\le 1$.
Следовательно, $h(\theta_0)>0$.

Рассмотрим поведение $h(\theta)$ внутри отрезка
$[\theta_0(\delta),\,2\pi]$. Вычисляя производную
\begin{equation*}
h^\prime(\theta)=
(1+\d)\sin\theta-(1+\d)\theta\cos\theta-\theta^2\sin\theta\,,
\end{equation*}
замечаем, что условие $h^\prime(\theta)=0$ равносильно уравнению
$$
\tg \theta=\fr{\theta}{1-\theta^2/(1+\d)}\,,
$$
которое на отрезке $[\theta_0(\delta),\,2\pi]\subset(\pi,2\pi]$
имеет единственный корень ${\theta_2(\d)\in(3\pi/2,\,2\pi)}$, причем
\begin{align*}
  h^\prime(\theta)&>0\,, & \hspace*{-30pt} \theta_0(\d)&\le\theta<\theta_2(\d)\,; \\
  h^\prime(\theta)&<0\,, & \hspace*{-30pt} \theta_2(\d)&<\theta\le2\pi\,.
\end{align*}
Таким образом, $h(\theta)$ возрастает при
$\theta_0(\d)\le\theta<$\linebreak $<\theta_2(\d)$, убывает при
${\theta_2(\d)<\theta\le2\pi}$ и принимает положительные значения на
концах отрезка ${\theta\in[\theta_0(\d),2\pi]}$; следовательно,
\begin{multline}
\label{OcenkaCos_h(theta)>0}
h(\theta)=2(2+\d)(1-\cos\theta)-
(3+\d)\theta\sin\theta+{}\\
{}+
\theta^2\cos\theta>0\,,\quad
\theta_0(\d)\le\theta\le 2\pi\,.
\end{multline}

Для завершения доказательства леммы~1 осталось
показать справедливость неравенства~\eqref{OcenkaCos} при $x>2\pi$.
В нижеследующей лемме~5 будет показано, что функция
$$
\varphi_x(\theta)=x^{\delta}b(\theta)-a(\theta)\,,\quad
\theta_0(\delta)\le\theta\le2\pi\,,
$$
монотонно убывает по~$\theta$ при $x>\theta$, поэтому для всех
$x\ge2\pi\ge\theta$
$$
\inf_{\theta_0(\delta)\le\theta\le2\pi}\varphi_x(\theta) =
\varphi_x(2\pi)=0\,,
$$
а следовательно,
\begin{multline*}
1-a(\theta)x^2+b(\theta)x^{2+\delta}={}\\
{}=
1+x^2\varphi_x(\theta)\ge 1\,,\quad \theta_0(\delta)\le\theta\le2\pi\,,
\end{multline*}
в то время как $\cos x\le1$ для всех $x\in\R$.
Таким образом, при $x>2\pi$ неравенство~\eqref{OcenkaCos} тривиально.

\medskip

\noindent
\textbf{Замечание 1.} Из построения функций~$a(\d,\theta)$ и~$b(\d,\theta)$ видно, что при
каждом ${\delta\in(0,1]}$ и ${\theta\in[\theta_0(\d),2\pi]}$
\begin{align*}
b(\d,\theta)&=\sup_{x>0}\fr{|\cos x-1+a(\d,\theta)x^2|}{x^{2+\d}}\,,\\
a(\d,\theta)&=\inf_{x>0}\fr{1-\cos x+b(\d,\theta)x^{2+\d}}{x^2}\,,
\end{align*}
причем супремум и инфимум достигаются в точке $x=\theta$. Кроме
того, из построения  $\theta_0=\theta_0(\d)$ как корня уравнения
$a(\d,\theta)=1/2$ на интервале $0<\theta\le 2\pi$ вытекает, что
\begin{align*}
a(\d,\theta_0(\d))&=\fr{1}{2}\,;\\
b(\d,\theta_0(\d))&=\sup_{x>0}\fr{\left|\cos
x-1+x^2/2\right|}{x^{2+\d}}\equiv \varkappa(\d)\,,
\end{align*}
причем супремум достигается в точке $x=\theta_0(\d)$.
Из~\eqref{CosBNepolnoe} также вытекает представление
$$
b(\d,\theta_0(\d))=
\fr{\theta_0(\d)-\sin\theta_0(\d)}{(2+\d)(\theta_0(\d))^{1+\d}}\,.
$$


\medskip

\noindent
\textbf{Лемма 5.} \textit{Для всех $0<\d\le1$ и $t>0$ функция}
\begin{equation*}
\varphi_{t}(\theta)= t^{\d}b(\d,\theta)-a(\d,\theta)\,,\quad
\theta_0(\d)\le\theta\le2\pi\,,
\end{equation*}
\textit{монотонно возрастает по $\theta$ при $t<\theta$ и монотонно убывает
по $\theta$ при $t>\theta$.}
\medskip

Д\,о\,к\,а\,з\,а\,т\,е\,л\,ь\,с\,т\,в\,о\,.\  \
Для $\theta_0(\delta)\le\theta\le2\pi$ введем функции
\begin{align*}
u(\theta)&\equiv -\fr{a(\d,\theta)}{\d}= \fr{\sin\theta}{\theta}-
(2+\d)\fr{1-\cos\theta}{\theta^2}\,;\\
v(\theta)&\equiv -\fr{\theta^{\d}b(\d,\theta)}{\d}=
\fr{\sin\theta}{\theta}-2\fr{1-\cos\theta}{\theta^2}\equiv{}\\
& \equiv u(\theta)+\d w(\theta)\,;\\
\ \ \ \ \ \ \ \ \ \ \ \ \ \ \ \ \ w(\theta)&=\fr{1-\cos\theta}{\theta^2}\,.
\end{align*}
Тогда можно записать
\begin{equation*}
\varphi_{t}(\theta)=
\d\bigg(u(\theta)-\left(\fr{t}{\theta}\right)^{\d}v(\theta)\bigg)\,,\quad
\theta_0(\d)\le\theta\le2\pi\,.
\end{equation*}
Найдем условия, при которых производная~$\varphi_{t}(\theta)$ по~$\theta$ меняет знак. Имеем
\begin{align*}
w^\prime(\theta)&=
\fr{\sin\theta}{\theta^2}-2\fr{1-\cos\theta}{\theta^3}=
\fr{1}{\theta}v(\theta)\,;\\
v^\prime(\theta)&= u^\prime(\theta)+\d w^\prime(\theta)=u^\prime(\theta)+\fr{\d}{\theta}v(\theta)\,;\\
u^\prime(\theta)&= \fr{1}{\theta^3}\left(2(2+\d)(1-\cos\theta)-{}\right.\\
&\hspace*{25mm}\left.{}-
(3+\d)\theta\sin\theta+\theta^2\cos\theta\right)\,.
\end{align*}
Заметим, что~$u'(\theta)$ с точностью до сомножителя~$\theta^{-3}$
совпадает с функцией~$h(\theta)$, введенной при доказательстве
леммы~1 (см.~\eqref{OcenkaCos_h(theta)>0}). Там же
было показано, что ${h(\theta)>0}$ при
${\theta_0(\d)\le\theta\le2\pi}$, поэтому
$$
u^\prime(\theta)>0\,,\quad\theta_0(\d)\le\theta\le2\pi\,.
$$
Учитывая найденные соотношения для производных, получаем
\begin{multline*}
\fr{d}{d\theta}\,\varphi_{t}(\theta)= \d u^\prime(\theta)+\fr{\d
t^{\d}}{\theta^{1+\d}}(\d v(\theta)-{}\\
{}-
\theta v'(\theta))= \d
u^\prime(\theta)\bigg(1-\left(\fr{t}{\theta}\right)^{\d}\bigg)\,.
\end{multline*}
Замечание о том, что знак производной~$\varphi_{t}(\theta)$
определяется знаком выражения в скобках, завершает доказательство
леммы.

\medskip

\noindent
\textbf{Лемма 6.}
\textit{Пусть $X$ и $Y$~--- независимые одинаково распределенные случайные
величины такие, что}
$$
\E X=0;\ \ \E X^2=\sigma^2;\ \ \E |X|^{2+\d}=\bet<\infty
$$
\textit{для некоторого $0<\delta\le1$. Тогда}
\begin{align*}
\E (X-Y)^2&=2\sigma^2\,;\\
\E |X-Y|^{2+\d}&\le2\left(\beta_{2+\d}+\sigma^{2+\d}\right)\,.
\end{align*}

\medskip

Д\,о\,к\,а\,з\,а\,т\,е\,л\,ь\,с\,т\,в\,о\,.\ Первое соотношение вытекает непосредственно из условий
леммы:
\begin{multline*}
\E (X-Y)^2=\E \left(X^2+2XY+Y^2\right)={}\\
{}= \E X^2+2\E X\E Y+E Y^2=2\sigma^2\,.
\end{multline*}
Рассмотрим второе соотношение. В силу неравенства
$|x-y|^{\d}\le|x|^{\d}+|y|^{\d}$, справедливого для всех
$x,y\in\mathbb R$ и $\d\in(0,1]$, имеем
\begin{multline*}
\E |X-Y|^{2+\d}= \E (X-Y)^2|X-Y|^{\d}\le{}\\
{}\le \E (X-Y)^2\left(|X|^{\d}+|Y|^{\d}\right).
\end{multline*}
В силу независимости $X$ и~$Y$,
$$
\E XY\left(|X|^{\d}+|Y|^{\d}\right)= \E |X|^{2+\d}\E Y+\E
|Y|^{2+\d}\E X=0\,,
$$
а следовательно,
\begin{multline*}
\E |X-Y|^{2+\d}\le \E (X^2+XY+Y^2)\left(|X|^{\d}+|Y|^{\d}\right)={}\\
{}= \E
(X^2+Y^2)\left(|X|^{\d}+|Y|^{\d}\right)={}\\
{}=
\E |X|^{2+\d}+\E Y^2\E |X|^{\d}+\E X^2\E |Y|^{\d}+\E |Y|^{2+\d}={}\\
{}=
2\left(\bet+\sigma^2\E |X|^{\d}\right).
\end{multline*}
Теперь для доказательства леммы осталось заметить, что
$$
\E |X|^{\d}\le\left(\E X^2\right)^{\d/2}=\sigma^{\d}
$$
в силу неравенства Ляпунова.

\smallskip

\section{Оценки для~характеристических функций}

\noindent
\textbf{Теорема 1.}
\label{Ocenkaf(t)Obw}
\textit{Пусть $f(t)$~--- характеристическая функция некоторого распределения
$F$, удовлетворяющего условиям~\eqref{EDX1bet}. Тогда для всех
$t\in\R$ справедливы оценки}
\begin{multline}
\label{Ocenka|f(t)|^2obwaja}
|f(t)|^2\le 1-2a(\d,\theta)\sigma^2t^2+{}\\
\!\!\!\!{}+ 2b(\d,\theta)\left(\bet+
\sigma^{2+\d}\right)|t|^{2+\d}\,,\ \theta_0(\d)\le\theta\le2\pi\,;\!\!
\end{multline}

\vspace*{-8pt}

\noindent
\begin{multline}
|f(t)|\le1 -2a(\d,\theta)\sigma^2t^2+{}\\
{}+2b(\d,\theta)\left(\bet+
\sigma^{2+\d}\right)|t|^{2+\d}\le{}\ \ \ \
\label{Ocenka|f(t)|obwaja}
\end{multline}

\vspace*{-12pt}

\noindent
\begin{multline}
\le \exp\left\{-2a(\d,\theta)\sigma^2t^2+{}\right.\\
\!\!\!\!\!\!\!\!\left.{}+2b(\d,\theta)\left(\bet+
\sigma^{2+\d}\right)|t|^{2+\d}\right\},\
\theta_0(\d)\le\theta\le2\pi \!\!\!
\label{OcenkaLn|f(t)|obwaja}
\end{multline}
с $a(\d,\theta)$, $b(\d,\theta)$ и~$\theta_0(\d)$, определенными в
формулировке леммы~1.

\medskip

Д\,о\,к\,а\,з\,а\,т\,е\,л\,ь\,с\,т\,в\,о\,.\
Пусть~$X_1$ и~$X_2$~--- независимые случайные величины с одинаковой
функцией распределения~$F$. Тогда из леммы~1
получаем оценку
\begin{multline*}
|f(t)|^2=\E \cos t(X_1-X_2)\le{}
\\
{}\le 1-2a(\d,\theta)t^2\E(X_1-X_2)^2+{}\\
{}+ 2b(\d,\theta)|t|^{2+\d}\E
|X_1-X_2|^{2+\d}\,,\\
t\in\R\,,\quad \theta_0(\d)\le\theta\le2\pi\,,
\end{multline*}
откуда в силу леммы~6 вытекает~\eqref{Ocenka|f(t)|^2obwaja}.

Для доказательства~\eqref{Ocenka|f(t)|obwaja} достаточно применить
элементарное неравенство
$$
1+2x\le(1+x)^2\,,\quad x\in\R\,,
$$
к правой части~\eqref{Ocenka|f(t)|^2obwaja} и заметить, что
$1+x\ge0$ всякий раз, когда $1+2x\ge0$.

Неравенство~\eqref{OcenkaLn|f(t)|obwaja} вытекает
из~\eqref{Ocenka|f(t)|obwaja} с учетом оценки ${1+x\le e^x}$,
справедливой для всех $x\in\R$.
\medskip

Выбирая в теореме~1 величину~$\theta$ равной~$\theta_0(\delta)$ и замечая, что для всех $0<\delta\le1$
\begin{align*}
a(\delta,\theta_0(\delta))&=\fr{1}{2}\,,\\
b(\d,\theta_0(\d))&=\sup_{x>0}\fr{\left|\cos
x-1+x^2/2\right|}{x^{2+\d}}\equiv\varkappa(\d)\,,
\end{align*}
получаем следующий результат, приведенный в работах
Тысиака~\cite{Tysiak1983} и Ушакова~\cite{Ushakov1999}.

\medskip

\noindent
\textbf{Теорема 2.}
\textit{Пусть $f(t)$ --- характеристическая функция некоторого распределения~$F$,
удовлетворяющего условиям~\eqref{EDX1bet}. Тогда для всех
$t\in\R$ справедливы оценки}

\noindent
\begin{align*}
|f(t)|^2&\le 1-\sigma^2t^2+2\varkappa(\d)\left(\beta_{2+\d}+
\sigma^{2+\d}\right)|t|^{2+\d}\,,\\
|f(t)|&\le 1-\fr{\sigma^2t^2}2+\varkappa(\d)\left(\beta_{2+\d}+
\sigma^{2+\d}\right)|t|^{2+\d}\le{}\\
&{}\le \exp\left\{-\fr{\sigma^2t^2}{2}+
\varkappa(\d)\left(\beta_{2+\d}+
\sigma^{2+\d}\right)|t|^{2+\d}\right\}\,.
\end{align*}

\medskip

Значения величины $\varkappa(\d)$ для некоторых $\d$ приведены в
табл.~1.

\bigskip

\begin{center} %tabl1
\noindent
{{\tablename~1}\ \ \small{Значения величины $\varkappa(\d)$ при некоторых $\d$}}
\end{center}
%\vspace*{2pt}

\begin{center}
\tabcolsep=7pt
\begin{tabular}{|c|c||c|c||c|c|}
\hline
$\d$ & $\varkappa(\d)$ & $\d $&$ \varkappa(\d)$ &$ \d$ & $\varkappa(\d)$ \\
\hline
0+   & 0,5\hphantom{999}    & 0,35 & 0,2702 & 0,70 & 0,1538 \\
%\hline
0,05 & 0,4564 & 0,40 & 0,2485 & 0,75 & 0,1425 \\
%\hline
0,10 & 0,4171 & 0,45 & 0,2288 & 0,80 & 0,1322 \\
%\hline
0,15 & 0,3816 & 0,50 & 0,2109 & 0,85 & 0,1228 \\
%\hline
0,20 & 0,3495 & 0,55 & 0,1946 & 0,90 & 0,1143 \\
%\hline
0,25 & 0,3204 & 0,60 & 0,1797 & 0,95 & 0,1064 \\
%\hline
0,30 & 0,2941 & 0,65 & 0,1662 & 1,00 & 0,0992 \\
\hline
\end{tabular}
\end{center}
\vspace*{6pt}


\bigskip
\addtocounter{table}{1}

Выбирая в теореме~1 величину~$\theta$
минимизирующей правую часть~\eqref{Ocenka|f(t)|^2obwaja}, получаем
следующее утверждение.

\medskip

\noindent
\textbf{Теорема 3.}
\textit{Предположим, что $f(t)$~--- характеристическая функция некоторого
распределения $F$, удовлетворяющего условиям~\eqref{EDX1bet} с
$\sigma^2=1$. Пусть $\theta_0(\d)$~--- единственный корень уравнения}
$$
\fr{\d\theta^2}2+ \theta\sin \theta + (2+\d)(\cos \theta - 1)=0\,,
$$
\textit{лежащий в интервале $(\pi,\,2\pi)$. Тогда для всех  $t$ таких, что}
$$
\theta_0(\d)\le (\bet+1)^{1/\d}|t|  \le2\pi\,,
$$
\textit{справедливы оценки}
\begin{equation}
\label{OcenkasCosin}
|f(t)|^2\le1- 2 \fr{1-\cos\left((\bet+1)^{1/\d} t\right)}
{(\beta_{2+\d}+1)^{2/\d}}\,;
\end{equation}
\begin{multline}
\label{OcenkaLgf(t)Cos}
|f(t)|\le 1-\fr{1-\cos\left((\bet+1)^{1/\d} t\right)}
{(\beta_{2+\d}+1)^{2/\d}}\le{}\\
{}\le
\exp\left\{-\fr{1-\cos\left((\bet+1)^{1/\d} t\right)}
{(\beta_{2+\d}+1)^{2/\d}}\right\}\,.
\end{multline}

\medskip

\noindent
\textbf{Замечание 2.}
Теорема~3 справедлива для любых $\sigma^2>0$, в этом
случае оценки становятся более громоздкими.
Например,~\eqref{OcenkasCosin} примет вид

\noindent
\begin{multline*}
|f(t)|^2\le1- 2
\left(\fr{\sigma^2}{\beta_{2+\d}+\sigma^{2+\d}}\right)^{2/\d}\times{}\\
{}\times
\left(1-\cos\left(\left(\fr{\beta_{2+\d}+
\sigma^{2+\d}}{\sigma^2}\right)^{1/\d} t\right)\right)\,;
\end{multline*}
\vspace*{-6pt}

\noindent
$$
\theta_0(\d)\le
|t|\left(\fr{\beta_{2+\d}+\sigma^{2+\d}}{\sigma^2}\right)^{1/{\d}}
 \le2\pi\,.
$$

\medskip

\noindent
Д\,о\,к\,а\,з\,а\,т\,е\,л\,ь\,с\,т\,в\,о\,\ т\,е\,о\,р\,е\,м\,ы\,\ 3\,.
Не ограничивая общности, будем считать, что $t>0$. Действительно,
характеристическая функция $|f(t)|^2$ и правая
часть~\eqref{OcenkasCosin} являются четными функциями, а при $t=0$
неравенство~\eqref{OcenkasCosin} очевидно. Обозначим
$$
\rho=\left(\beta_{2+\d}+1\right)^{1/\d}\,.
$$
Из теоремы~1 имеем
\begin{multline}
\label{|f(t)|^2<=a+b_bez_alpha}
|f(t)|^2\le 1-2at^2+2b
\left(\beta_{2+\d}+1\right)|t|^{2+\d}={}\\
{}=
1+2t^2\left((\rho t)^\d
b-a\right)\,,
\end{multline}
где $a=a(\d,\theta)$ и $b=b(\d,\theta)$,
$\theta_0(\d)\le\theta\le2\pi$, определены в формулировке
леммы~1. Введем функцию
$$
g_t(\theta)=(\rho t)^\d b(\d,\theta)-a(\d,\theta)\,,\quad
\theta_0(\d)\le\theta\le2\pi\,.
$$
Тогда последнее неравенство примет вид
$$
|f(t)|^2\le 1+2t^2g_t(\theta)\,,\quad \theta_0(\d)\le\theta\le2\pi\,.
$$
Найдем минимум правой части по~$\theta$. Из леммы~5
следует, что~$g_t(\theta)$ монотонно возрастает на
$\theta_0(\d)\le\theta\le2\pi$ при $\rho t<\theta$ и монотонно
убывает при $\rho t>\theta,$ поэтому
$$
\inf_{\theta_0(\d)\le\theta\le2\pi}g_t(\theta)=
\begin{cases}
g_t(\theta_*)\,,&\rho t<\theta\,;\\
g_t(\theta^*)\,,&\rho t>\theta\,,
\end{cases}
$$
где
$$
\theta_*=\max\{\theta_0(\d),\rho t\}\,, \quad \theta^*=\min\{2\pi,\rho
t\}\,.
$$
Если $\rho t\in[\theta_0(\d),2\pi]$, то
$$
\inf_{\theta_0(\d)\le\theta\le2\pi}g_t(\theta)= g_t(\rho t)= (\rho
t)^{\d}b(\rho t)-a(\rho t)\,,
$$
откуда с учетом соотношения
\begin{align*}
\theta^\d b(\d,\theta)-a(\d,\theta)&=
-\fr{1-\cos\theta}{\theta^2}\,,\\
\theta_0(\d)&\le\theta\le2\pi\,,
\end{align*}
вытекающего из определения функций $a(\d,\theta)$ и~$b(\d,\theta)$,
получаем
\begin{multline*}
|f(t)|^2\le1+2t^2\inf_{\theta_0(\d)\le\theta\le2\pi}g_t(\theta)={}\\
{}=
1-2t^2 \fr{1-\cos\theta}{\theta^2} \Big\vert_{\theta=\rho t}= 1-2 \fr{1-\cos\rho t}{\rho^2}\,.
\end{multline*}
Неравенства~\eqref{OcenkaLgf(t)Cos} доказываются
аналогично~\eqref{Ocenka|f(t)|obwaja} и~\eqref{OcenkaLn|f(t)|obwaja}.

\medskip

\noindent
\textbf{Замечание 3.}
Детальный анализ проведенного доказательства позволяет заключить,
что неравенства из теоремы~3 являются более точными в
своих интервалах, чем неравенства из теоремы~2. Другими
словами, минимум из трех оценок для~$|f(t)|$, уста\-нав\-ли\-ва\-емых
теоремами~2,~3 и неравенством $|f(t)|\le1$,
имеет следующий вид (для упрощения обозначений полагаем
$\sigma^2=1$):
$$
|f(t)|\le \exp\{-\psi_\d(t)\}\,,
$$
где
$$
\psi_\d(t)=
\begin{cases}
\fr{t^2}{2}-\varkappa(\d)\left(\beta_{2+\d}+
 1\right)|t|^{2+\d}\,, \\[6pt]
 \hspace*{96pt}  |t|\le\fr{\theta_0(\d)}{(\bet+1)^{1/\d}}\,;\hspace*{-8pt} \\[9pt]
\fr{1-\cos\left((\bet+1)^{1/\d}t\right)}{(\beta_{2+\d}+1)^{2/\d}}\,, \\[6pt]
\hspace*{52pt} \theta_0(\d)\le (\bet+1)^{1/\d}|t|  \le2\pi\,; \hspace*{-8pt}\\[9pt]
  0,  \hspace*{87pt}|t|\ge\fr{2\pi}{(\bet+1)^{1/\d}}\,.\hspace*{-8pt}
\end{cases}
$$

\medskip

Значения величины $\theta_0(\d)$ при некоторых $\d$ приведены в
табл.~2.


\bigskip

\begin{center}
\noindent
{{\tablename~2}\ \ \small{Значения величины $\theta_0(\d)$ при некоторых $\d$}}
\end{center}
%\vspace*{2ex}

\begin{center}
\tabcolsep=6.3pt
\begin{tabular}{|c|c||c|c||c|c|}
\hline
$\d$ &$ \theta_0(\d)$ &$ \d$ &$ \theta_0(\d)$ &$ \d$ &$ \theta_0(\d)$\\
\hline
0+   & 6,2831 & 0,35 & 5,3887 & 0,70 & 4,6374  \\
%\hline
0,05 & 6,1331 & 0,40 & 5,2778  & 0,75 & 4,5320  \\
%\hline
0,10 & 5,9941 & 0,45 & 5,1686 & 0,80 &  4,4263  \\
%\hline
0,15 & 5,8631 & 0,50 & 5,0609  & 0,85 &  4,3200 \\
%\hline
0,20 & 5,7384 & 0,55 & 4,9542  & 0,90 & 4,2131  \\
%\hline
0,25 & 5,6183 & 0,60 & 4,8483  & 0,95 &  4,1051 \\
%\hline
0,30 & 5,5021 & 0,65 &  4,7427  & 1,00 & 3,9959  \\
\hline
\end{tabular}
\end{center}
%\vspace*{-10pt}


\bigskip
\addtocounter{table}{1}




Как несложно убедиться,
$$
\lim_{\d\to0+}\theta_0(\d)=2\pi=6{,}2831\ldots
$$

\medskip

\noindent
\textbf{Замечание 4.}
Если есть необходимость использования оценок
вида~\eqref{OcenkasCosin} и \eqref{OcenkaLgf(t)Cos} для
$$|t|<\theta_0(\d)(\bet+1)^{-1/\d}\,,$$
 то можно заменить
неравенство~\eqref{|f(t)|^2<=a+b_bez_alpha} на более грубое
$$
|f(t)|^2\le 1-2at^2+2b(\bet+1)\alpha^\d|t|^{2+\delta}\,,\quad
\alpha\ge1\,,
$$
и получить оценки
$$
|f(t)|^2\le1- 2\fr{1-\cos\left((\bet+1)^{1/\d}\alpha
t\right)} {(\beta_{2+\d}+1)^{2/\d}\alpha^2}\,;
$$
\vspace*{-12pt}

\noindent
\begin{multline*}
|f(t)|\le 1-\fr{1-\cos\left((\bet+1)^{1/\d}\alpha t\right)}
{(\beta_{2+\d}+1)^{2/\d}\alpha^2}\le{}\\
{}\le
\exp\left\{-\fr{1-\cos\left((\bet+1)^{1/\d}\alpha t\right)}
{(\beta_{2+\d}+1)^{2/\d}\alpha^2}\right\}\,,
\end{multline*}
справедливые для
$$
\theta_0(\d)\le (\bet+1)^{1/\d}\alpha|t|  \le2\pi\,.
$$

%\smallskip

\section{Уточнение неравенства Мизеса}

Пусть случайная величина $X$ имеет решетчатое распределение с шагом
$h$. Обозначим
$$
\beta_s=\E|X-\E X|^s,\quad s>1\,,\quad \sigma^2=\beta_2\,.
$$
В 1939~г.\ Р.~фон~Мизес~\cite{Mises1939} получил следующий результат.

\medskip

\noindent
\textbf{Теорема 4.}
\textit{В сделанных выше предположениях для целых $s\ge2$ имеют место
неравенства}
$$
\beta_{s-1}\le \fr{2}{h}\,\beta_s\,.
$$
%\smallskip

Из этой теоремы при $s=3$ получаем
$$
\fr{h}{\sigma}\le 2 \fr{\beta_3}{\sigma^3}\,.
$$
Следующее утверждение, уточняющее неравенство Мизеса для $s=3$,
является простым следствием теоремы~3.

\noindent
\textbf{Теорема 5.}
\textit{В сделанных выше предположениях для ${s=2+\d}$ с $0{<\d\le1}$
справедлива оценка}
\begin{equation}
\label{ShevtsovaUtochnilaMisesa}
\fr {h}{\sigma}\le \left(\fr{\bet}{\sigma^{2+\d}}+1\right)^{1/\d}\,,
\end{equation}
\textit{причем}
\begin{equation}
\label{ExactnessOfShevtsova}
\sup\bigg(\fr {h}{\sigma}-
\left(\fr{\bet}{\sigma^{2+\d}}+1\right)^{1/\d}\bigg)=0\,,
\end{equation}
\textit{где супремум берется по всем распределениям, удовлетворяющим
условиям теоремы.}
\medskip

\noindent
Д\,о\,к\,а\,з\,а\,т\,е\,л\,ь\,с\,т\,в\,о\,.\
Без ограничения общности будем считать, что ${\E X=0}$ и
${\sigma^2=\E X^2=1}$. Пусть $f(t)$~--- характеристическая функция
случайной величины~$X$. Как известно, характеристическая функция~$f(t)$
задает решетчатое распределение тогда и только тогда, когда
существует такое $t_0\neq0$, что $|f(t_0)|=1$. При этом в качестве
шага распределения~$h$ можно взять

\noindent
$$
h=\fr{2\pi}{t_0}\,.
$$
С другой стороны, из замечания~3 к
теореме~3 вытекает, что $|f(t)|<1$ для всех $t$ таких, что
$$
|t|<\fr{2\pi}{(\bet+1)^{1/\d}}\,;
$$
следовательно,
$$
t_0\ge \fr{2\pi}{(\bet+1)^{1/\d}}\,,
$$
откуда с учетом соотношения $h=2\pi/t_0$
получаем~\eqref{ShevtsovaUtochnilaMisesa}.

Для доказательства второй части теоремы рассмотрим случайную
величину $X$, имеющую следующее распределение:
\begin{align*}
\p\left(X=\fr{h}{1+u}\right)&=\fr{u}{1+u}\,;\\
\p\left(X=-\fr{uh}{1+u}\right)&=\fr{1}{1+u}\,,\quad u,\,h>0\,.
\end{align*}
Шаг этого распределения равняется~$h$, $\E X=0$,
\begin{gather*}
\sigma^2=\E X^2= \fr{u h^2}{(1+u)^2}\,,\\
\bet=\E|X|^{2+\d} =\fr{u(1+u^{1+\d})h^{2+\d}}{(1+u)^{3+\d}}\,,\\
\fr{\bet}{\sigma^2 h^\d} + \left(\fr{\sigma}{h}\right)^\d=
\fr{1+u^{1+\d}}{(1+u)^{1+\d}}+\fr{u^{\d/2}}{(1+u)^\d}\equiv
g(u,\d)\,.
\end{gather*}
Тогда
\begin{multline*}
\inf \left(\fr{\bet}{\sigma^2 h^\d} +
\left(\fr{\sigma}{h}\right)^\d\right)\le\inf_{u>0}
g(u,\d)\le{}\\
{}\le
\lim_{u\to\infty} g(u,\d) =1\,,\quad 0<\d\le 1\,,
\end{multline*}
что равносильно~\eqref{ExactnessOfShevtsova}. Дополнительно заметим,
что при $\d=1$ инфимум функции $g(u,\d)$ также достигается в точке
$u=1$, что соответствует сим\-мет\-рич\-но\-му распределению Бернулли, для
которого $\beta_3/\sigma^3=1$.
\medskip

В заключение автор выражает благодарность В.\,Ю.~Королёву за идею
использования теоремы~3 в целях уточнения неравенства
Мизеса и за постоянное внимание к работе.


{\small\frenchspacing
{%\baselineskip=10.8pt
\addcontentsline{toc}{section}{Литература}
\begin{thebibliography}{99}

\bibitem{Zolotarev1965} %1
\Au{Золотарёв В.\,М.}
О близости распределений двух сумм независимых
случайных величин~// Теория вероятн. и ее примен., 1965. Т.~10.
Вып.~3. С.~519--526.


\bibitem{Zolotarev1967a} %2
\Au{Золотарёв В.\,М.}
Некоторые неравенства теории вероятностей и их
применение к уточнению теоремы А.~М.~Ляпунова~// ДАН СССР, 1967.
Т.~177. №~3. С.~501--504.

\bibitem{Prawitz1973} %3
\Au{Prawitz H.}
Ungleichungen f\"{u}r den absoluten Betrag einer
charakteristischen funktion~// Skand. Aktuarietidskr., 1973. No.\,1. P.~11--16.

\bibitem{Berry1941} %4
\Au{Berry A.\,C.}
The accuracy of the Gaussian approximation to the
sum of independent variates~// Trans. Amer. Math. Soc., 1941.
Vol.~49. P.~122--139.

\bibitem{Esseen1942} %5
\Au{Еsseen C.-G.}
On the Liapunoff limit of error in the theory of
probability~// Ark. Mat. Astron. Fys., 1942. Vol.~A28. No.\,9.
P.~1--19.

\bibitem{Zolotarev1966} %6
\Au{Золотарёв В.\,М.}
Абсолютная оценка остаточного члена в
центральной предельной теореме~// Теория вероятн. и ее примен.,
1966. Т.~11. Вып.~1. С.~108--119.

\bibitem{Zolotarev1967b} %7
\Au{Zolotarev V.\,M.}
A sharpening of the inequality of Berry--Esseen~// Wahrsch. verw. Geb., 1967. Bd.~8. S.~332--342.

\bibitem{Prawitz1975} %8
\Au{Prawitz~H.}
On the remainder in the central limit theorem~//
Scand. Actuarial J., 1975. No.\,3. P.~145--156.

\bibitem{Tysiak1983} %9
\Au{Tysiak W.}
Gleichm{\"a}$\beta$ige und
nicht-gleichm{\"a}$\beta$ige Berry--Esseen--Absch{\"a}tzungen.
Dissertation, Wuppertal, 1983.

\label{end\stat}

\bibitem{Ushakov1999} %10
\Au{Ushakov N.\,G.}
Selected topics in characteristic functions.~--- Utrecht: VSP, 1999.

\bibitem{Mises1939} %11
\Au{Von~Mises R.}
An inequaltiy for the moments of a discontinuous
distribution~// Skand. Aktuarietidskr., 1939. Vol.~22. No.\,1. P.~32--36.


 \end{thebibliography}
}
}
\end{multicols}