\def\stat{morozov}

\def\tit{АСИМПТОТИКИ ВЕРОЯТНОСТЕЙ БОЛЬШИХ УКЛОНЕНИЙ
СТАЦИОНАРНОЙ ОЧЕРЕДИ$^*$}
\def\titkol{Асимптотики вероятностей больших уклонений
стационарной очереди}

\def\autkol{Е.\,В.~Морозов}
\def\aut{Е.\,В.~Морозов$^1$}

\titel{\tit}{\aut}{\autkol}{\titkol}

{\renewcommand{\thefootnote}{\fnsymbol{footnote}}\footnotetext[1]
{Работа поддерживается РФФИ, грант 07-07-00088.}}

\renewcommand{\thefootnote}{\arabic{footnote}}
\footnotetext[1]{Институт прикладных
математических исследований КарНЦ РАН, emorozov@karelia.ru}

\Abst{Точные аналитические результаты доступны только
для сравнительно узкого класса сис\-тем обслуживания, и поэтому
асимптотические методы анализа оказываются полезным инструментом
исследования и оптимизации современных коммуникационных сетей со
сложной структурой потоков данных. Статья представляет собой обзор
основных асимптотических результатов, играющих важную роль в оценке
качества обслуживания (QoS) стационарных систем. Рассматриваются
асимптотики вероятности превышения процессом нагрузки/очереди
растущего уровня в случае, когда время обслуживания имеет тяжелый
хвост. Аналогичные результаты даны для систем с входным процессом
Леви, где время обслуживания имеет легкий хвост. Доказательства
 базируются на методах теории больших уклонений (ТБУ), которые
подробно иллюстрируются на примере системы $M/M/1$. Рассмотрена
асимптотика вероятности переполнения на цикле регенерации, в том
числе и для многоканальной системы. Приведен асимптотический анализ
систем, где входной процесс обладает долговременной зависимостью
(долгой памятью), причем основное внимание уделено фрактальному
броуновскому процессу. Обсуждаются связи между долговременной
зависимостью процесса очереди
и моментными свойствами вложенного процесса регенераций.}

\KW{стационарная очередь; вероятности больших
уклонений; асимптотический анализ; распределения с легким хвостом;
фрактальный броуновский процесс; процесс с долговременной
зависимостью; регенерация}

 \vskip 12pt plus 6pt minus 6pt

 \thispagestyle{headings}

 \begin{multicols}{2}

 \label{st\stat}


\section{Введение}

Статья содержит обзор основных результатов по асимптотике
вероятностей превышения растущего уровня стационарным процессом
нагрузки~$W$ (времени ожидания в очереди) и стационарной оче\-редью
$\nu$ в системах вида $GI/G/1$, где либо (1)~время обслуживания~$S$
имеет так называемый тяжелый хвост, либо (2)~$S$ имеет легкий хвост,
а также (3)~где входной процесс обладает долговременной
зависимостью.

Случаи~1 и~3 относительно недавно привлекли особый
интерес исследователей ввиду обнаружения тесных связей со свойствами
трафиков современных телекоммуникационных сетей~\cite {Will}. В~обоих
этих случаях асимптотики имеют, как правило, достаточно
простой вид, а в случае 2 требуется привлечение методов ТБУ и
параметры асимптотического представления не выражаются
непосредственно в терминах заданных распределений (в отличие от
случаев~1 и~3). Асимптотические результаты для случая~1 были
использованы для оценки качества статистического оценивания
вероятностей редких событий, получаемых в рамках регенеративной
модификации метода расщепления~\cite{BorMor}. Численное
моделирование дало хорошее согласие с этими (и известными
аналитическими) результатами и позволяет утверждать, что данный
метод дает надежные оценки (в том числе интервальные) для
вероятностей больших уклонений в регенеративных системах
обслуживания. В случае~2 асимптотический анализ связан с
вычислением так называемой {\it функции интенсивности} методами ТБУ
и является в типичных ситуациях весьма трудным. Возможность
расширения предложенного метода для прямого оценивания параметров
асимптотик в регенеративных системах является важным мотивом данной
работы и одной из целей дальнейшего исследования.

Основное внимание в статье уделено математическим аспектам
 асимптотического анализа, однако с трактовкой результатов в терминах конкретных
сетевых характеристик. Главная цель работы состоит в том, чтобы
представить наиболее важные асимптотические результаты в форме,
удобной для специалистов, работающих в области анализа сетей и
телекоммуникационных технологий. Это является актуальным, поскольку
многочисленные результаты, накопленные к настоящему времени в разных
источниках, зачастую представлены в несогласованном виде и в
трудной для практического использования форме.

% Статья организована следующим образом.
В разд.~2 рассмотрены асимптотики ве\-ро\-ят\-ности большого значения
процесса нагрузки/очереди в системе $GI/G/1$, когда время
обслуживания имеет тяжелый хвост. Раздел~3 посвящен системам, где
обслуживание имеет легкий хвост, а входной процесс~--- либо процесс
Леви, либо процесс вос\-ста\-нов\-ле\-ния. Подход на основе ТБУ подробно
ил\-люст\-ри\-руется на примере системы $M/M/1$. Здесь же приведена
асимптотика вероятности превышения очередью растущего уровня на
цикле регенерации, в том числе и для многоканальной стационарной
сис\-те\-мы $GI/G/m$. В разд.~4 рассматриваются сис\-темы, где входной
поток обладает долговременной зависимостью, в частности является
фрактальным броуновским процессом. В разд.~5 обсуждается связь
между долговременной зависимостью процесса очереди/нагрузки и
моментными свойствами вложенного процесса регенераций в связи с
возможностью применения регенеративного метода для оценивания
параметров таких процессов.

\section{Время обслуживания\newline с~тяжелым хвостом}

Рассмотрим случайную величину (с.в.)~$X$ с функцией распределения
(ф.р.) $F$ и обозначим через $\bar F=1-F$ ее {\it хвост}. Через
$F^{(n)}$ обозначим  $n$-кратную свертку ф.р.~$F$ с собой.
Будем писать $F(x)\sim G(x)$, если $ F(x)/G(x)\to 1 $ при $x\to\infty$.
Функция распределения~$F$ (и сама с.в.~$X$) имеет {\it тяжелый хвост}
(записывается $F\in\mathcal{H}$), если $\bar F(x)>0,\,x\ge 0$, и для
любого фиксированного $y>0$
\begin{equation*}
\lim_{x\to\infty}\p(X>x+y|X>x)=\lim_{x\to\infty}\fr{\bar
F(x+y)}{\bar F(x)}= 1\,.
\end{equation*}
Заметим, что если $F\in\mathcal{H}$, то производящая функция
моментов $\E e^{\varepsilon X}=\infty$ для любого $\varepsilon >0$.
Функция распределения~$F$ называется {\it субэкспоненциальной}
($F\in\mathcal{S} \subseteq \mathcal{H}$), если $\bar F(x)>0,\,x\ge 0$,
и для любого $n\ge 2$
\begin{equation}
\bar F^{(n)}(x)\sim n \bar F(x),\,\,x\to \infty\,.
\label{subexp}
\end{equation}
(На самом деле достаточно установить~({\ref{subexp}) для\linebreak $n= 2$.)
В действительности $e^{\varepsilon x}\bar F(x)\to\infty$, %\linebreak
$x\to\infty $, для любого $\varepsilon >0$, что поясняет название
клас\-са~$\mathcal{S}$~\cite{Sigman}.

Рассмотрим систему $GI/G/1$ с входным потоком восстановления с
независимыми, одинаково распределенными (н.о.р) интервалами
$\{\tau_n\}$ с интенсивностью $\lambda=1/\E\tau$ и с н.о.р.\
временами обслуживания~$\{S_n\}$ в предположении стационар\-ности,
$\rho:=\lambda\E S<1$. (Здесь и далее опущены индексы, обозначающие
типичный элемент последовательности н.о.р.с.в.) Последовательность
$\{W_n\}$ времен ожидания заявок в очереди является апериодической
марковской цепью, удовлетворяющей рекурсии Линдли $ W_{n+1} = (W_n +
X_n)^+ $, где $X_n=S_n - \tau_n,\,n \ge 1$. Поэтому существует
предел $W_n \Rightarrow W$ (по распределению), причем {\it
стационарная нагрузка}~$W$ распределена как максимум случайного
блуждания, $W=_{\mathrm{st}}\max_{n\ge 1} (X_1+\cdots+ X_n) $\linebreak
($=_{\mathrm{st}}$
означает равенство по распределению). Пусть с.в.\ $S_e$ имеет ф.р.~$F_e$
\begin{equation}
 F_e(x)= \fr{1}{\E S}
 \int\limits_{0}^{x}\bar {F} (y)\, dy\,.
 \label{tail}
\end{equation}
(Это ф.р.\ стационарного незавершенного времени восстановления в
процессе, порожденном с.в.~$S_n$.) Обозначим $G(x)=\p(W\le x)$.
Основной результат формулируется следующим образом~\cite{Asmus, Veraverbeke}: если
функция $F \in \mathcal S$, то
\begin{equation}
\bar G(x):= \p (W>x)\sim \fr{\rho}{1-\rho}\,\bar F_e(x)\,,\quad x\to \infty\,.
\label{5}
\end{equation}
Этот важный результат имеет простое качественное объяснение на
примере системы $M/G/1$, где (в очевидных обозначениях) плотность~$G'$
удовлетворяет {\it уравнению Такача}
\begin{multline}
G^{\,'}(x)=\lambda\int\limits_{0^-}^x \bar
F(x-y)\,dG(y)\,,\\
 x>0 \quad(G(0)=1-\rho>0)\,.
\label{4.32}
\end{multline}
Применяя преобразование Лапласа--Cтилтьеса к обеим частям
уравнения~(\ref{4.32}) и обозначая $g (\theta)=$\linebreak $=\E e^{-\theta W}$,
$f_e(\theta)=\E e^{-\theta S_e}$ $(\theta>0)$, получим
$g(\theta)=$\linebreak $= 1-\rho+\rho f_e(\theta)g(\theta)$, или
\begin{equation*}
g(\theta)=\fr{1-\rho}{1-\rho f_e(\theta)}=
\left( 1-\rho\right )\sum\limits_{n=0}^\infty (\rho
f_e(\theta))^n\,.
%\eqno (4.33)
\end{equation*}
Используя обратное преобразование и свойство свертки, получаем~(\ref{5}):
\begin{multline}
\bar G(x) = (1 - \rho) \sum\limits_{n=1}^{\infty}\rho^n \bar F
_e^{(n)}(x)\sim{}\\
{}\sim (1 - \rho) \rho \bar F_e
(x)\sum\limits_{n=1}^{\infty}\rho^{n-1}n={}\\
{}=\fr{\rho}{1-\rho}\,\bar F_e(x)\,,\quad x\to \infty\,.
\label{8}
\end{multline}
 В~(\ref{8}) первое равенство является {\it (обобщенной) формулой
Поллачека--Хинчина}, а эквивалентность (при $x \to\infty$) опирается
на~(\ref{subexp}). Таким образом, хвост ф.р.~$W$ можно выразить
через частичные суммы слага\-емых $\rho \bar F_e$, а затем через
единственное слага\-емое~$\bar F_e$. В действительности~$ \bar F_e$
является условным распределением {\it возрастающей строгой лестничной
высоты} в случайном блуждании c шагом~$S-\tau$ при условии
выполнения события $A=\{${\it достижение очередной лестничной
высоты}$\}$, причем $\p(A)=\rho$~\cite {Feller}. В целом~(\ref{8})
 отражает тот ключевой факт, что очень большое значение суммы с.в., имеющих
распределение с тяжелым хвостом, {\it вызвано очень большим
значением одного слагаемого}~\cite{Sigman, Asmus, AsKl, Greiner}.

Опираясь на~\cite {AKS}, рассмотрим стационарную очередь $\nu$
(общее число заявок) в системе $M/G/1$, где по свойству PASTA $\nu$ может рассматриваться как в
моменты прихода, так и в произвольный момент времени. (То же самое
верно и для~$W$.) В данном случае используется {\it обобщенный
закон Литтла}, согласно которому стационарная очередь~$\nu$
распределена как число приходов за время~$W$~\cite{Asmus}. (Это
далеко идущий аналог известного {\it закона Литтла} для средних,
$\lambda \E W=\E \nu$.) Если ф.р. времени обслуживания $F\in
\mathcal S$ и, кроме того,
\begin{multline}
\lim_{x \to \infty} \fr{\bar {F}(xe^{y/\sqrt{x}})}{\bar{F}(x)} = 1,\\
 \mbox{ локально равномерно по}\,\, y\ge 0\,,
\label{e24}
\end{multline}
то вероятность превышения уровня $k$ стационарной очередью~$\nu$
удовлетворяет асимптотическому соотношению~\cite{AKS}
\begin{equation}
P(\nu \ge k) \sim \fr{\rho}{1 - \rho}\,\bar
{F_e}\left(\fr{k}{\lambda}\right)\,,\quad k \to \infty\,.
\label{e25}
\end{equation}
Отметим, что~(\ref{5}) имеет место и для стационарного времени
пребывания заявки в системе $D=_{\mathrm{st}}W+S$, т.\,е.\ $\p (D>x)\sim \p(W>x)$.
(Это следствие того, что хвост~$\bar F_e$ {\it тяжелее},
чем~$\bar F$~\cite{Sigman}.) Соотношение~(\ref{e24}) справедливо
для распределения Парето, имеющего (стандартный) вид
 $\bar F(x) =x^{-\alpha}$, $\alpha>0$, $x\ge 1$, а также для
 распределения Вейбулла $\bar F(x)=\exp\{-x^\beta\}$, $x\ge 0$,
 с {\it тяжелым хвостом} $\beta\in(0,\,1/2)$. Однако~(\ref{e24})
 не верно, если $F\in\mathcal{H}\backslash \mathcal{S}$, например для
распределения Вейбулла с {\it умеренно тяжелым хвостом} $1/2\le
\beta<1$,~\cite {AKS}. Тем не менее, в этом случае

\noindent
\begin{multline*}
\!\!\!\!\p(\nu>k)\sim
\fr{1}{\beta}\left(\fr{k}{\lambda}\right)^{1-\beta}\!\exp
\left\{-\beta\left(\fr{k}{\lambda}\right)-(1-\beta)t^\beta\right\}\!,\\
k\to \infty\,,
\label{queue}
\end{multline*}
где $t=t(k)$ есть решение уравнения $\beta t^\beta+\lambda t= k$.
 (При $\beta=1/2$ явное решение этого квадратного уравнения приведено
в~\cite {AKS}.)

В~\cite{Asmus98} доказано, что в системе $GI/G/1$, где ф.р.\
времени обслуживания $F\in \mathcal{S}$, максимум
$W^*:=\max\limits_{0\le k<\Delta}(W_k)$
стационарной нагрузки на цик\-ле регенерации удовлетворяет асимптотическому соотношению
\begin{equation}
\p(W^*>x)\sim \bar F(x) \E \Delta\,,\quad x\to \infty\,,
\label{soren}
\end{equation}
где $\Delta$~--- длина цикла регенерации. (Регенеративные аргументы
показывают также, что~(\ref{soren}) влечет~(\ref{5}).) В~\cite{SorenJakob}
этот результат перенесен на более сложный процесс обслуживания. Заметим, что в~\cite{Foss Korsh} предложены несколько
иные условия для справедли\-вости~(\ref{e25}).

В заключение этого раздела сделаем несколько замечаний. Результаты
об асимптотике стационарного времени ожидания в системе $GI/G/2$
c субэкспоненциальным временем обслуживания, приведенные в работе~\cite {FK},
достаточно громоздки и (за единственным исключением) имеют вид неравенств.

 В {\it ациклической сети} для каждого узла $i$ можно естественным образом определить множество~$\Omega_i$}
всех ему {\it предшествующих узлов}. Тогда (в очевидных обозначениях) если время обслуживания~$S^{(i)}$
субэкспоненциально, то для стационарной нагрузки~$W^{(i)}$ имеет место асимптотика вида~(\ref{5})
$$
\p\left(W^{(i)}>x\right)\sim \fr{\rho_i}{1-\rho_i}\,\p\left(S_e^{(i)}>x\right)\,,
$$
если все времена обслуживания $S^{(j)}\in \Omega_i$ имеют более
легкие хвосты, чем~$S^{(i)}$~\cite {Sigman98}. (Заметим, что это интуитивно ожидаемый результат.)
 Укажем также работу~\cite{BF}, где исследуется асимптотика сетей с субэкспоненциальными временами обслуживания,
однако асимптотика сетей в данном обзоре не затрагивается.

В~\cite {retrial} показано, что если хвост ф.р.\ времени
обслуживания в системе типа $M/G/1$ с {\it повторными вызовами} (и
экспоненциальным временем ожидания на орбите) имеет вид $x^{-\alpha} L(x)$,
где $\alpha>1$, a функция~$L$ медленно меняется на
бесконечности, то хвост ф.р.\ стационарного числа заявок в системе
асимптотически эквивалентен хвосту ф.р.\ очереди в системе
$M/G/1$ и имеет вид $x^{-(\alpha-1)}L(x)$, $x\to \infty$, см.~(\ref{e25}).

\section{Время обслуживания\newline с~легким хвостом}

Опираясь в основном на работу~\cite {GW}, рас\-смот\-рим стационарную
систему $GI/G/1$, в которой время обслуживания~$S$ имеет легкий
хвост, т.\,е.\ (в отличие от $S\in\mathcal{S} $) {\it производящая
функция моментов}
\begin{equation}
 \E e^ {\theta S}<\infty
\label{10}
\end{equation}
в некоторой положительной окрестности пара\-мет\-ра $\theta =0$. Приведем
условия, при которых стационарная нагрузка $W$ удовлетворяет соотношению
\begin{equation}
x^{-1}\log \p (W>x)\to -\theta^*\,, \quad x\to \infty\,,
\label{41}
\end{equation}
где показатель $\theta^*>0$ является искомой величиной. Напомним,
что типичное приращение процесса нагрузки имеет вид $X=S - \tau$.
 Учитывая независимость~$\tau$ и~$S$, введем {\it логарифмическую производящую функцию моментов}
\begin{multline*}
\Lambda(\theta):=\log \E e^{\theta X}:=\Lambda_S(\theta)+\Lambda_\tau(-\theta)={}\\
{}= \log \E e^{\theta S}+\log \E e^{-\theta \tau}\,, \quad
\theta\ge 0\,.
\end{multline*}
 Условие стационарности $\rho=\E S/\E\tau<1$ влечет $\p(\tau>0)>0$, и поэтому
 $0<\E e^{-\theta \tau}< 1$ и $-\infty<$\linebreak $<\log\E e^{-\theta \tau}<0$. Так что
условие $-\infty<\Lambda(\theta)<\infty$ обеспечивается условием~(\ref{10}).
 Основным результатом является следующий. Параметр~$\theta^*$ в~(\ref{41}) оказывается единственным
положительным решением уравнения
\begin{equation*}
\Lambda(\theta)=0\,.
%\label{20}
\end{equation*}
(Параметр $\theta^*$ часто определяют как решение уравнения
$\E e^{\theta X}=1$~\cite {GW}.)

Определенная проблема при применении ТБУ (кроме вычисления
собственно параметра~$\theta^*$) состоит в том, что этот подход не
позволяет выяснить, является ли асимптотика хвоста $\p(W>x)$ в
точности экспоненциальной или содержит корректирующие множители, скажем, вида
\begin{equation*}
\p(W>x)=g(x)e^{-\theta^*x+f(x)}\,,
\end{equation*}
где функции~$g,f$ таковы, что $\log g(x)=o(x)$, $f(x)=$\linebreak $=o(x)$,
$x\to \infty$. Например, в~\cite {GW} упоминается
 система $M/G/1$ (из~\cite {Abate1}), где $\p(W>x)\sim $\linebreak $\sim cx^{-3/2}\exp\{-\theta^* x\}$.
 В этом случае (\ref{41})~имеет место, а ожидаемая сходимость $e^{ \theta^*x} \p (W>x)\to const>0$
 не верна. Это обстоятельство демонстрируется ниже на примере системы $M/M/1$, и его надо иметь в
виду при оценивании показателя~$\theta^*$.

\subsection{Асимптотика в~системе $M/M/1$}

В качестве иллюстрации применения методов ТБУ рассмотрим
асимптотику процессов нагрузки~$W$ и очереди~$\nu$ в
стационарной сис\-те\-ме $M/M/1$ с интенсивностью входного потока~$\lambda$ и интенсивностью обслуживания~$\mu$, полагая
$\rho:=\lambda/\mu<1$. Хорошо известно, что
\begin{equation}
\p(W\ge x)=\rho e^{-(\mu-\lambda)x}\,,\quad x\ge 0\,.
\label{23}
\end{equation}
Поэтому логарифмическая асимптотика~(\ref{41}) принимает вид
\begin{equation}
\lim_{t\to \infty}\fr{1}{x}\log \p(W\ge x):=-\theta^*=-(\mu-\lambda)\,,
\label{theta}
\end{equation}
разумеется, не улавливая постоянный множитель~$\rho$. Теперь
используем ТБУ, не опираясь на аналитический вид~(\ref{23}).
 Предполагая $\theta<\mu$, после несложных вычислений имеем
\begin{equation*}
\Lambda(\theta)=\log\E e^{\theta(S-\tau)} = \log
\left( \fr{\lambda\mu}{ (\lambda+\theta)(\mu - \theta)}\right)\,.
\end{equation*}
Решение уравнения $\Lambda(\theta) = 0$, равное $\theta^*= \mu -\lambda$,
согласуется с~(\ref{theta}). Такой результат можно получить, рассматривая
динамику процесса нагрузки в дискретном времени~\cite{BQ}:
\begin{equation*}
W(t+1)=(W(t)+a(t)-1)^+\,,\quad t=0,1,\ldots,
\end{equation*}
где~$a(t)$~--- нагрузка (суммарное время обслуживания заявок),
поступившая в систему в интервале $[t,t+1)$.
 Поскольку входной поток пуассоновский, то $a(t)=_{\mathrm{st}}a(0)$, $t\ge 0$,
 и, кроме того, $\E e^{\theta S}=\mu(\mu-\theta)^{-1}$. Поэтому
\begin{multline*}
\Lambda(\theta) = \log \E e^{\theta (a(0)-1)}=\log\E e^{\theta a(0)}-\theta ={}\\
{}=
\log \left\{ \sum\limits_{k\ge 0}\left(\E e^{\theta S}\right)^k
e^{-\lambda}\fr{\lambda^k}{k!}\right\}-\theta={}\\
{}= \log\left\{\exp \left\{
\fr{\mu\lambda}{\mu-\theta}-\lambda\right\} \right\}-\theta =
\fr{\mu\lambda}{\mu-\theta}-\lambda -\theta\,.
\end{multline*}
Уравнение $\Lambda(\theta)=0$ имеет единственное решение $\theta^*
=\mu-\lambda>0$, которое согласуется как с~(\ref{theta}), так и с
результатом для системы с постоянной скоростью обслуживания $C=1$
[19, с.~10--12].

Рассмотрим теперь вложенный (по моментам прихода) процесс очереди $\{\nu_n\}$, удовлетворяющий рекурсии
\begin{equation}
\nu_{n+1}=(\nu_n+1-D(n))^+\,,\quad n\ge 1\,,
\label{25}
\end{equation}
где $D(n)$~--- (потенциальное) число заявок, которое прибор мог бы
обслужить в течение $n$-го интервала входного потока. Очевидно, с.в.\
$\{D(n),\,n\ge 1\}$ являются н.о.р.\ и $\nu_n\Rightarrow \nu$, $n\to \infty$. Понятно, что~(\ref{25}) можно
трактовать как вариант рекурсии Линдли, где $n$-й шаг случайного блуждания равен $1-D(n)$.
Таким образом,
\begin{multline*}
\Lambda(\theta):= \log \E e ^{\theta(1-D(1))} ={}\\
{}=\theta+\log \left
\{\lambda\sum\limits_{k\ge 0}\int\limits_0^\infty e^{-(\mu+\lambda)x}
\fr{\left(\mu x e^{-\theta}\right)^k }{k!}\,dx\right\}={}\\
=\theta+\log \fr{\lambda}{\mu+\lambda-\mu e^{-\theta}}\,.
\end{multline*}
Уравнение $\Lambda(\theta)=0$ имеет единственное положительное
решение $\theta^*=-\log \rho>0$, согласующееся с известным
результатом $\p (\nu\ge k)=\rho^k$, $k\ge0$. Заметим, что здесь
асимптотика $k^{-1}\log\p(\nu>k)$ (при $k\to \infty$) точна в
отмеченном выше смысле. Используя дискретизацию времени, можно аналогично рассмотреть процесс очереди $\nu(t)$
в непрерывном времени
\begin{equation*}
 \nu(t+1)=(\nu(t)+a(t)-d(t))^+\,,\quad t=0,1,\ldots,
\end{equation*}
где $a(t)$~--- число приходов, а $d(t)$~--- потенциальное число уходов в интервале $[t,t+1)$. Очевидно,
$a(t)-d(t)=_{\mathrm{st}}a(0)-d(0),\,t\ge 0$, и
\begin{multline}
\Lambda(\theta)=\log \E e^{\theta (a(0)-d(0))} ={}\\
{}=\lambda \left(e^{\theta}-1\right)- \mu \left(1-e^{-\theta}\right)\,,
\label{2.5}
\end{multline}
что снова дает решение $\theta^*= -\log\rho$.

\subsection{Входной процесс Леви}

В данном разделе рассматривается система {\it типа } $M/G/1$,
где входной процесс имеет стационарные независимые приращения ({\it
процесс Леви}), включая пуассоновский поток.

Рассмотрим систему $M/G/1$ с постоянной скоростью обслуживания
$C>\lambda$, в которой стационарная нагрузка задается следующим образом~\cite {BQ}:
\begin{equation}
W=_{\mathrm{st}}\sup_{t\ge 0}(A(t)-Ct)\,,
\label{work}
\end{equation}
где $A(t)=\sum\limits_{i=0}^{t-1}a(i)$~-- суммарная нагрузка, поступившая в
интервале~$[0,t)$. Соотношение (\ref{work}) аналогично представлению
$W$ в сис\-те\-ме $GI/G/1$ как максимума случайного блуждания, однако
теперь независимые приращения на единичных интервалах $[t,\,t+1)$
имеют вид $a(t)-C.$ Решение $\theta^*$ уравнения
\begin{equation}
\log \E e ^{\theta a(0)}=\theta C
\label{28}
\end{equation}
имеет тот же вид $\theta^*=\mu-\lambda>0$, что и в сис\-те\-ме $M/M/1$.

В случае общего входного процесса Леви $A(t)$\linebreak $=\sum\limits_{i=0}^{t-1} a(i)$
предположим, что существует (конечный в некоторой окрестности
$\theta=0$) предел
\begin{equation*}
\lim_{t\to \infty}\frac{1}{t}\log \E e^{\theta
A(t)}:=\Lambda_A(\theta)=\log \E e^{\theta a(0)}\,.
%\label{50}
\end{equation*}
(То есть входной поток удовлетворяет {\it принципу больших
уклонений}.) Из~(\ref{28}) следует, что в данной системе параметр
$\theta^*$ определяется из условия
\begin{equation}
\theta^*=\sup (\theta>0: \Lambda_A(\theta)\le \theta C)\,.
\label{30}
\end{equation}
Приведем уточнение хвоста $\p(W>x)$ в предположении, что решение~$\theta^*$
уравнения $\Lambda_A(\theta)=$\linebreak $= \theta C$ принадлежит внутренности интервала,
где\linebreak $\Lambda_A(\theta)<\infty$~\cite{Kelly}.
(Тогда производная $\Lambda'(\theta^*)<\infty$.) Обозначая $\phi(\theta)=\Lambda_A(\theta)/\theta$, получим
\begin{equation*}
\p (W> x)\sim \fr{C-\phi(0)}{\theta^*\Lambda_A^\prime (\theta^*)}\,e^{-\theta^*x}\,,\quad
x\to \infty\,,
\end{equation*}
где $C-\phi(0)=C-\E a(0)>0$ ввиду стационарности.

Рассмотрим входной процесс $ A(t)=\lambda t +G(t)$ с
детерминированной компонентой~$\lambda t $ и гауссовской компонентой
 $G(t)= N(0,\sigma^2 t)$. (Как обычно, отрицательные значения~$A(t)$
 игнорируются~\cite{Kelly}.) Поскольку $ \E e^{\theta G(t)} = \exp
\left\{(1/2)\theta^2 \sigma^2t\right\}$, то для любого $t>0$
\begin{equation*}
\fr{1}{t} \log \E e ^{\theta A(t)}:=\Lambda_A(\theta)= \lambda
\theta + \fr{\theta^2 \sigma^2}{2}\,.
\end{equation*}
Поэтому показатель $\theta^* $ в~(\ref{41}) и (\ref{30})
 (напомним, что $C>\lambda$) равен
 $$
\theta^*=\fr{ 2(C-\lambda)}{\sigma^2}\,.
$$
Другой полезный пример дает входной {\it обобщенный пуассоновский процесс}~\cite {Feller}
$$
A(t)=\sum\limits_{n=0}^{N(t)}Y_n\,,
$$
где н.о.р.\ положительные с.в.\ $\{Y_n\}\, (Y_0=0) $ имеют
распределение~$F$, а~$N(\cdot)$~--- независимый пуассоновский
процесс с параметром $\lambda$. Обозначая $ \E e^{\theta Y_1} = b(\theta)$ и
используя равенство
$$ 1-b(\theta)=\int (1-e^{\theta x})\,dF(x),$$
легко получить
$$\E e^{\theta A(t)} = e^{-\lambda t(1-b(\theta))}.$$
 Поэтому
$$
\Lambda_A(\theta)=\fr {\log \E e^{A(t)}}{ t}= \lambda\int\limits_0^\infty
(e^{\theta x}-1)\,dF(x) \quad (t>0)\,.
$$
Возможность решения уравнения $\Lambda_A(\theta)=\theta C $ определяется видом ф.р.~$F$. Равенство
\begin{equation*}
\fr{\Lambda_A(\theta^*)}{\theta^*}=C
%\label{effect}
\end{equation*}
 определяет {\it эффективную пропускную способность } прибора (скорость)~$C$, если
 ее необходимо найти, например, чтобы гарантировать заданное качество обслуживания (QoS) в системе {\it с
конечным буфером}. Такая гарантия состоит в том, что (предельная)
доля потерь не должна превышать заранее заданной вероятности потери. С ростом размера буфера такая система сближается с системой
с неограниченным буфером, на чем и основана возможность применения
приведенной выше асимптотики~[19--21]. Актуальной
проблемой является построение достоверной оценки параметра $C$ в
случае, когда аналитическое выражение функции~$\Lambda_A$ получить
трудно (или невозможно). Эта проблема рас\-смат\-ри\-ва\-ет\-ся в работе~\cite{Irina07}.

\smallskip

\noindent
{\bf Замечание 1.} Асимптотический результат~(\ref{41}) верен (при дополнительных предположениях)
 и для более общего входного потока с эргодическими (зависимыми) приращениями~\cite {GW}.

\subsection{Многоканальная система}

Опираясь на~\cite{Sadowsky}, рассмотрим асимптотику вероятности
того, что стационарная очередь в $m$-ка\-наль\-ной системе $GI/G/m$
превысит высокий уровень {\it на цикле регенерации}. Рассмотрим
входной процесс восстановления с интервалами~$\tau$, и пусть
$S$~--- время обслуживания (в каждом из идентичных каналов).
Предположим, что
\begin{equation*}
\rho:=\fr{\E S}{\E \tau}<m\,, \quad \p(\tau>S)>0
%\label{condit}
\end{equation*}
и что логарифмическая функция моментов
$$\Lambda_S(\theta):= \log \E e^{\theta S}<\infty$$
 в некоторой окрестности $\theta=0$.
(То есть время обслуживания имеет легкий хвост.)
Пусть~$\nu_n$ означает число заявок в (стационарной) очереди в
момент прихода заявки~$n$. Обозначим через $\Delta$ длину цикла
регенерации системы, т.\,е.\ интервал между двумя последовательными
заявками, приходящими в полностью пустую систему. Пусть
$\gamma_c(k)=\p(\max\limits_{1\le l<
\Delta} \nu_l\ge k)$~--- вероятность превышения (стационарной)
очередью (целочисленного) уровня~$k$ на цикле регенерации.
 Обозначим $\Lambda(\theta)=\Lambda_\tau(-\theta m)+\Lambda_S(\theta)$. В~\cite{Sadowsky}
 показано, что существует единственное решение $\theta^*>0$ уравнения $\Lambda(\theta)=0$ и что
\begin{equation*}
\lim_{k\to \infty}\fr{1}{k}\log\gamma_c(k)=\Lambda_\tau(-\theta^* m)\,.
%\label{sad}
\end{equation*}
Например, в системе $M/M/1$ имеем (используя технику ветвящихся процессов~\cite {BorMor})
\begin{equation*}
\gamma_c (k)= \fr{\rho^{k-1}-\rho^{k}}{1-\rho^k } :=\rho^{k-1}
\phi(k)\,,\quad k\ge 1\,,
\end{equation*}
где функция $\phi(k)\to 1-\rho>0$ при $k\to \infty$. Поэтому $\log \gamma_c(k)=(k-1)\log \rho+o(k)$. Следовательно,
$\Lambda_\tau(-\theta^* )=\log \rho$,
 что совпадает с асимптотикой~(\ref{2.5}) превышения
растущего уровня стационарной очередью в системе $M/M/1$.

Рассмотрим стационарную $m$-канальную систему $M/M/m$ при
$\rho:=\lambda/(m\mu)< 1$. Решая уравнение $\Lambda(\theta)=0$,
получим $\theta^*=(\mu m -\lambda)/m$, и поэтому
$$%\begin{equation*}
\lim_{k\to \infty}\fr{1}{k}\log\gamma_c(k)=\log \E e^{-\theta^* m
\tau}=\log \rho\,.
%\label{itog}
$$%\end{equation*}
В данной системе стационарная вероятность
$$ \p(\nu\ge k)= D \rho^{k}$$
для $k\ge m$ (где константа~$D$ хорошо известна в явном
виде (см., например,~\cite{Gnedenko})). Поэтому снова получаем
совпадение логарифмических асимптотик вероятностей~$\gamma_c(k)$ и
 $ \p(\nu\ge k)$ при $k\to \infty$.

\smallskip
\noindent
{\bf Замечание~2.} В более общей ситуации как
нахождение аналитического решения~$\theta^*$ уравнения
$\Lambda(\theta)=0$, так и последующее вычисление функции
$\Lambda_\tau(-m\theta^* )$ является чрезвычайно сложной (или
невозможной) задачей. Это ставит вопрос об эффективных методах
 оценивания параметров асимптотического представления,
возможно через прямое оценивание малых вероятностей $\gamma_c(k)$.
Отметим, что именно при оценивании вероятности~$\gamma_c$
ускоренный метод регенеративного расщепления показал свою
эффективность~\cite{BorMor}.


\section{Входной процесс с~долговременной зависимостью}

Рассмотрим асимптотику вероятностей больших уклонений для
системы, где входной поток обладает {\it долговременной
зависимостью}. В этом обзоре анализ ограничивается наиболее
важным в современных сетевых исследованиях {\it фрактальным
входным процессом}, определяемым следующим образом:
\begin{equation}
A(t)=\lambda t+ B_H(t)\,,\quad t\ge 0\,,
\label{LongRD}
\end{equation}
где $A(t)$~--- суммарная нагрузка, поступившая в сис\-те\-му в интервале
$[0,\,t)$, константа $\lambda>0$, а $B_H(t)$~--- {\it фрактальный
броуновский процесс} (ФБП) с параметром {\it Херста} $H\in
(1/2,\,1)$~\cite{Norros}. (Заметим, что процесс~$B_H$ можно задать
на всей вещественной прямой.) Процесс~$B_H$ имеет непрерывные траектории, нормально
распределенные стационарные приращения, причем

\pagebreak

\noindent
\begin{gather*}
 B_H(0)=0\,,\quad \E B_H(t)= 0\,,\\
 \E \{B_H(t)\} ^2= \sigma^2 t ^{2H}\,,\quad
 t\ge 0\,.
\end{gather*}
Более того, процесс~$B_H$ обладает свойством {\it самоподобия}:
$B_H(tT)=_{\mathrm{st}}T^H B_H(t)$ для любого $T>0$. Далее положим
$\sigma^2=\E \{B_H(1)\}^2=1$ и отметим, что $B_{1/2}$ является
стандартным броуновским движением. Хорошо известно (и легко
подсчитать), что приращения ФБП на единичных интервалах
$B_H^*(n):=B_H(n+1)-B_H(n)$, $n\ge 0 $, образующие {\it фрактальный
броуновский шум}, имеют (авто)ковариационную функцию
\begin{multline*}
\!r(n):=\mathrm{cov}(B_H^*(0),B_H^*(n)) \sim H(2H-1) n^{-2(1-H)}\,, \\
 n\to \infty\,.
\end{multline*}
Поэтому
\begin{equation*}
\sum\limits_{n\ge 1}r(n)=\infty
%\label{LRD}
\end{equation*}
и по определению процесс~$B_H$ обладает {\it долго\-временной зависимостью}.
Используя дискретную\linebreak
 шкалу времени, положим далее
$t=n$. Для случая независимых приращений дисперсия входного процесса
растет линейно со временем: $\D A(n) =$\linebreak $= n \D a(1)$, в то время как
для процесса~(\ref{LongRD}) имеем $\D A(n) = n^{2H}$, $n\ge 1$.
Разложим (нормированную) логарифмическую производящую функцию
моментов входного потока $\Lambda_n(\theta)$ в ряд Тейлора
\begin{multline*}
\Lambda_n(\theta):=\fr{1}{n} \log \E e^{\theta A(n)}=\lambda+
n^{2H-1} \fr{\theta^2}{2}+o(\theta^2)\,, \\
\theta\to 0\,,
\end{multline*}
где учтено, что $\Lambda_n^{''}(0)=\D A(n)/n $. Поэтому
$\Lambda_n(\theta)\to \infty$, $n \to \infty$, и ТБУ более
не применима. Для решения возникшей проблемы используем другую
нормировку, полагая
\begin{equation*}
\hat \Lambda_n(\theta)= \fr{1}{a(n)} \log \E e^{a(n)\theta A(n)/n}\,,
\end{equation*}
где искомая неслучайная последовательность\linebreak
$a(n) \uparrow \infty$.
Для выбора этой последовательности используем разложение Тейлора
\begin{equation*}
\hat \Lambda_n(\theta)=\lambda+ \fr{a(n)}{n^2} \D\left\{
A(n)\right\}\fr{\theta^2}{2}+o\left(\theta^2\right)\,, \quad \theta\to 0\,.
% \label{3.38}
\end{equation*}
Поскольку $\D A(n)=n^{2H}$, то можно предположить, что (при малом~$\theta$)
требование сходимости~$\hat \Lambda_n(\theta)$ к
конечному пределу~$\hat \Lambda(\theta)$ эквивалентно выбору
последовательности $a(n) \sim n^{2(1-H)}.$ Действительно, такой
выбор приводит к следующему обобщению асимптотики хвоста распределения стационарной нагрузки:

\noindent
\begin{equation}
 \lim_{x \to \infty} \fr{1}{x^{2-2H}}\log \p(W > x) = -\theta^*\,,
 \label{norros}
\end{equation}
где
$$%\begin{align*}
\theta^* = \sup\{\theta > 0: \hat \Lambda(\theta)\le 0\}\,,
$$
а
$$\hat \Lambda (\theta)=\lim_{n\to \infty} \hat \Lambda_n (\theta)\,.
$$%\end{align*}
Важнейший вывод, следующий из этого анализа, состоит в том, что, в
отличие от классических сис\-тем, асимптотика имеет вид {\it
распределения Вейбулла с умеренно тяжелым хвостом} (см.\ разд.~2)
\begin{equation}
\p (W > x)\asymp e^{-\theta^* x^{2-2H}}\,,\quad H\in\left(\fr{1}{2},1\right)\,.
\label{duffy}
\end{equation}
(Здесь $\asymp$ означает логарифмическую эквивалентность.) Этот
результат верен и для более общего входного процесса, а в системе с
ФБП~(\ref{LongRD}) и скоростью обслуживания $C>\lambda$ параметр~$\theta^*$ в~(\ref{norros}) получен в явном виде:
\begin{equation}
 \theta^*=\fr{1}{2\sigma^2}\left(\fr{C-\lambda}{H}\right)^{2H}(1-H)^{-2(1-H)}\,.
 \label{ilkka}
\end{equation}
Этот результат (как нижняя граница) был найден в~\cite {Norros} с использованием соотношений
\begin{multline*}
\p(W>x)=\p\left(\max_{t\ge 0}(A(t)-Ct)> x\right)\ge {}\\
{}\ge
\max_{t\ge 0}\p(A(t)> x+Ct)={}\\
{}=\max_t\p\Bigl(N(0,1)>(x+t(C-\lambda))t^{-H}\Bigr)\,,
\end{multline*}
где $N(0,1)$~--- стандартная нормальная с.в., и учтен явный вид~(\ref{LongRD}) процесса~$A(t)$.
Этот результат был обобщен и уточнен (в форме~(\ref{duffy})) в работах~\cite {Duff-Connell, Duffy} (см.\
также~\cite {Kelly}).

Важность ФБП связана с тем, что этот процесс возникает, например,
как результат суммирования (и надлежащего нормирования) трафиков от
растущего числа так называемых ON/OFF-ис\-точн\-иков, где пе\-рио\-ды простоя (OFF) и/или пе\-рио\-ды
работы (ON) имеют распределения с
тяжелым хвос\-том, конечным средним и бесконечной дис\-пер\-си\-ей~[28--31].
Иными словами, ON/OFF-пе\-рио\-ды порождают {\it альтернирующий процесс восстановления} с бесконечной
дисперсией длины (обобщенного)\linebreak интервала. Такой процесс хорошо
описывает огромную вариативность современных трафиков, от кратких
электронных сообщений до видеоконференций~\cite {Will}. Для
пояснения этого фундаментального результата рассмотрим трафик,
по\-рож\-да\-емый $m$ независимыми (идентичными) ON/\linebreak OFF-источниками,
считая, что хвост по крайней мере одной из ф.р.\ ON- или OFF-периода имеет вид
\begin{equation}
 \bar F(x)\sim c x^{-\beta}L (x)\,,\quad 1<\beta<2\,,
 \label{3.5}
 \end{equation}
 где константа $c>0$, а функция $L$ медленно меняется на
 бесконечности~\cite {Taqqu, Resnick}. Условие~(\ref{3.5}) обеспечивает
 упомянутые выше моментные свойства. Для источника~$i$ определим индикатор
 $I_i(t)=1$, если момент~$t$ принадлежит ON-периоду ($I_i(t)=0$, иначе),
 и пусть $\gamma=\p(I_i(t)=1)$ (стационарная вероятность ON-периода).
 Рассмотрим процесс, равный суммарному ON-времени от всех источников
 в интервале $[0,\,Tt)$ (здесь $T>0$~--- масштабный множитель)
\begin{equation}
 W_m(Tt)=\int\limits_0^{Tt}\sum\limits_{i=1}^mI_i(u)\,du\,,\quad t\ge 0\,.
 \label{3.4}
\end{equation}
В работе~\cite {Taqqu} доказано, что если $m\to \infty$,
 а затем $T\to \infty$ (порядок взятия пределов важен), то
 имеет место следующая функциональная центральная предельная теорема
\begin{equation}
 \left\{ \fr
 {W_m(Tt)-Ttm\gamma}{T^H\sqrt{L_1(T)m}}\,,\quad t\ge 0\right\}\Rightarrow c_1 B_H\,,
 \label{3.7}
\end{equation}
 где параметр Херста $H\in (1/2,1)$ (предельного ФБП $B_H$),
 постоянная $c_1>0$ и (медленно меняющаяся на бесконечности) функция~$L_1$
 полностью определяются через заданные параметры ф.р.\ ON/OFF-пе\-рио\-дов.
 Таким образом, конечномерные рас\-пре\-де\-ления нормированного
 и центрированного\linebreak
 процесса~(\ref{3.4}) сходятся к соответствующим распределениям ФБП.
(Взятие пределов в обратном порядке приводит к процессу Леви.) В
работе~\cite {Kaj} для данной модели найдены условия согласованного роста параметров~$m$ и~$T$,
 при которых пределы можно переставлять, получая ФБП. Проведенный анализ мотивирует значительное внимание, уделенное
выше входному ФБП, а также введение тесно связанного с пределом~(\ref{3.7})
{\it фрактального броуновского трафика}, определяемого следующим образом~\cite {Norros, Norr95}:
\begin{equation*}
A(t) =mt +\sigma \sqrt { m}B_H(t)\,, \quad H\in (1/2,\,1)\,,\quad t\ge 0\,,
\end{equation*}
где $m>0$~--- интенсивность трафика, а постоянная $\sigma>0$ отражает его вариативность.

 Входной процесс с долговременной зависи\-мостью возникает также при суммировании
ON/OFF-потоков от $m$ независимых источников, где каж\-дый поток имеет
интенсивность $r>0$ в течение ON-периода, имеющего распределение с
тяжелым хвостом, а OFF-период, когда $r=0$, распределен
экспоненциально с параметром $\lambda$~[34--37].
В пределе при $m\to \infty$, $\lambda m\to \Lambda$ возникает входной {\it процесс типа} $M/G/\infty$,
который в каждый момент времени равен текущему числу ON-периодов
({\it активных сессий}). Этот входной процесс можно представлять как
стационарное число $\beta$ занятых каналов в системе $M/G/\infty$ с
интенсивностью пуассоновского потока~$\Lambda$ и {\it временем
обслуживания}, равным длине~$\tau$ ON-периода. В частности, в~\cite{Jelenkovic} показано,
что $\p(\beta=0)=\exp\{-\Lambda\E\tau\}$ (что
соответствует сис\-те\-ме $M/G/\infty$~[38, с.~258]) и
исследована асимптотика процесса нагрузки в сис\-теме, куда поступает
такой поток. (Обычно такой входной процесс рассматривается в рамках
{\it жидкостной } модели~\cite {Jelenkovic}.)

Много полезных асимптотических результатов содержится в
 диссертациях~\cite{Deng, Zwart} и недавней книге~\cite{Mandjes}, где большое
 внимание уделено анализу сис\-тем с входным процессом с
долговременной зависимостью. Дополнительные детали можно найти в~\cite {SS}.

\smallskip
\noindent
{\bf Замечание 3.} Параметр $\theta^*$ из~(\ref{ilkka}) возникает
также при исследовании асимптотики максимума стационарного
виртуального времени ожидания в системе с входным процессом вида~(\ref{LongRD}), см.~\cite{GZ}.

\section{Долговременная зависимость и~регенерация}

Теперь рассмотрим процесс обслуживания, который обладает свойством
долговременной зависимости, однако имеет положительно возвратный
вложенный процесс регенераций. (Последнее означает конечность
средней длины цикла регенерации.) Отметим, что собственно {\it
процесс вос\-ста\-нов\-ления обладает долговременной зависимостью тогда и
только тогда, когда длина интервала восстановления имеет
бесконечную дисперсию}~\cite {Daley}. (Этот эффект связан с так
называемым парадоксом времени ожидания~\cite {Feller}.) Поэтому
существование стационарных процессов очереди/нагрузки с долговременной зависимостью, и имеющих одновременно вложенный
процесс регенераций с {\it хорошими моментными свойствами}, на
первый взгляд представляется парадоксальным. В частности, такой
процесс допускает исследование и оценивание на основе регенеративного метода, что является основной мотивировкой включения
данного раздела.

Рассмотрим стационарную систему $GI/G/1$ в предположении, что время
обслуживания~$S$ обладает следующими моментными свойствами:
\begin{equation}
\E S ^3<\infty\,,\quad \E S ^4=\infty\,.
\label{1.2}
\end{equation}
 Обозначим длину цикла регенерации через
$$\Delta=_{\mathrm{st}}B+I,$$
где $B$~--- длина периода занятости, а $I$~--- длина интервала
простоя системы. В~\cite{Daley68} доказано, что при~(\ref{1.2})
стационарный процесс нагрузки $W=\{W_n\}$ обладает долговременной\linebreak
зависи\-мостью, т.\,е.\
\begin{equation}
\sum\limits _{n=0}^\infty \mathrm{cov}\,\left ( W_0,W_n\right )=\infty\,.
\label{lrd}
\end{equation}
Известно, что если $\E S ^3<\infty$, то также $\E B^3<\infty$, и
что $\E \tau^3<\infty$ влечет $\E I^3<\infty$, где $\tau$~--- интервал
 входного потока~\cite {Wolff}.
 Предположим, что $\E\tau^3<\infty$, тогда по неравенству
Минковского $[\E \Delta^3]^{1/3}\le [\E B^3]^{1/3}+[\E I^3]^{1/3}$, и
поэтому $\E \Delta^3<\infty$. Пусть $G(x)=\p(\Delta\le x)$, и
рассмотрим незавершенную в момент $t$ длину цикла регенерации $
\beta(t).$ Пусть c.в.~$\tau$ нерешетчатая, тогда длина цикла~$\Delta$ тоже нерешетчатая и существует (слабый) предел
$\beta(t)\Rightarrow \beta $, причем~\cite{Asmus} (см.\ также~(\ref{tail}))
\begin{equation*}
 \p (\beta\le x)=\fr{1}{\E \Delta} \int\limits_0^x (1-G(y))\,dy\,.
 \label{73}
\end{equation*}
Поэтому стационарное незавершенное время восстановления~$\beta$
имеет конечный второй момент
\vspace*{-4pt}

\noindent
\begin{multline*}
\E \beta^2 = \fr{1}{\E \Delta}\int\limits_0^\infty
x^2(1-G(x)\,dx={}\\[-5pt]
{}=\fr{1}{3\E \Delta}\int\limits_0^\infty x^3 G(dx)=
\fr{\E \Delta^3}{3\E \Delta}<\infty
\end{multline*}
и, в частности,
\begin{equation*}
\p( \beta> x) = o(x^{-2})\quad \mbox{ при}\quad x\to \infty\,.
%\label{tail-1}
\end{equation*}
Таким образом, моментные свойства~$\Delta$ и~$\beta$ допускают
статистическое исследование процесса нагрузки c долговременной зависимостью
регенеративным методом~\cite{GI}, (см.\ также~\cite{Morozov, MorBod}.
 Подчеркнем, что у самого процесса восстановления с долговременной зависимостью $\E\beta=\infty$, что
исключает применение регенеративного имитационного моделирования.
(Конечно, в рассматриваемой системе $\E\beta<\infty$.)
Дополнительные детали проведенного выше анализа можно найти в~\cite{Morozov, Morozov1}.

 Отметим близкий результат, касающийся асимптотики хвоста ф.р.\ периода занятости~$B$
 стационарной системы $M/G/1$, где $\rho:=\lambda \E S<1$ и
ф.р.~$F$ времени обслуживания~$S$ имеет регулярное изменение на
бесконечности, т.\,е.\ для любого $t>0$ и некоторого $\alpha>0$ верно
$\bar F(tx)\sim \bar F(x) t^{-\alpha}$, $x\to\infty$. Тогда~\cite{Teugels}
\begin{multline}
\bar G(x):=\p(B>x)\sim \fr{1}{1-\rho}\,\bar F((1-\rho)x)\,,\\
 x\to \infty\,.
\label{busy}
\end{multline}
%\columnbreak

\noindent
Например, пусть $S$ имеет распределение Парето с показателем $\alpha \in (3,4)$.
Поскольку хвост времени простоя $I$ легче, чем~$\bar G$, то он не играет роли в асимптотике~(\cite {Sigman},
Предложение~2.7), и соотношение~(\ref{busy}) дает
\begin{multline*}
\p(\Delta>x)\sim \bar G(x)\sim (1-\rho)^{-(\alpha+1)} x^{-\alpha} = {}\\
{}= o(x^{-3})\,,\quad x\to \infty\,.
\end{multline*}
Обсудим другие недавние результаты, каса\-ющие\-ся долговременной
зависимости регенеративных процессов обслуживания. Рассмотрим
стационарный процесс~$N(t)$, равный числу восстановлений в интервале
$[0,\,t)$ с интервалами~$\tau_n$, и пусть
$R_n=\tau_1+\cdots+\tau_n,\,R:=\{R_n,\,n\ge1\}$. Согласно~\cite {Daley-Vesilo}, процессы~$N$ и~$R$ обладают
долговременной зависимостью, если соответственно
\begin{equation}
\limsup_{t\to \infty}\fr{\D N(t)}{t}=\infty\,,\quad
\limsup_{n\to \infty}\fr{\D R_n}{n}=\infty\,.
\label{long}
\end{equation}
Достаточным условием для второго соотношения в~(\ref{long}) является
 (см.~(\ref{lrd}))
$$
\lim_{n\to \infty}\sum\limits_{i=1}^n \mathrm{cov} (\tau_0,\,\tau_i)=\infty\,.
$$
Также в~\cite{Daley-Vesilo} дается следующее определение параметра Херста~$H$ для
данных процессов:
\begin{align}
H_N&:=\inf \left\{h: \limsup_{t\to \infty}\fr {\D N(t)}{t^{2h}}<\infty\right\}\,,\notag\\[-6pt]
&\label{78}\\[-6pt]
H_R&:=\inf \left\{h: \limsup_{n\to \infty}\fr {\D R_n}{n^{2h}}<\infty\right\}\,.\notag
\end{align}
В~\cite {Daley-Vesilo} показано, в частности, что в стационарной
системе $GI/G/1$ при входном процессе восста\-нов\-ления~$R$ c
условиями $\E \tau^2<\infty$, $\E S^2<\infty$ выходной процесс~$\tilde R$
(последовательность интервалов между уходами)
также не обладает долговременной зависимостью. Кроме того, процесс~$\tilde R$ обладает долговременной
зависимостью тогда и только тогда, когда $H_R\in (1/2,\,1)$ в~(\ref{78}).
Более того, если в стационарной сис\-те\-ме $GI/M/1$
\begin{equation}
\p(\tau>x)=x^{-c} L(x)\,,\quad c\in (1,\,2)\,,\quad x>0\,,
\label{daley}
\end{equation}
где функция~$L$ медленно меняется на бесконечности, то выходной
процесс $\tilde N(t)$ (число уходов в $[0,\,t)$) обладает
долговременной зависимостью в смысле~(\ref{78}). Этот же результат
верен для процесса~$\tilde N(t)$
в сис\-те\-ме $M/G/1$, если хвост $ \p(S>x)$ удов\-ле\-тво\-ряет условию~(\ref{daley}) (вместо хвоста $\p(\tau>x)$).

Рассмотрим с.в.~$T>0$ с {\it максимальным конечным моментом}
$\gamma:=\sup(k>0: \E T^k<\infty)\in (1,\,2)$. Тогда процесс
восстановления
$$N(t)=\max (n:T_1+\cdots+T_n\le t)$$
(где $T_k=_{\mathrm{st}}T$) имеет параметр Херста
$H_N =$\linebreak $=\;(3-\gamma)/2\in (1/2,\,1)$~\cite {Daley}.

В работе~\cite {Carpio} c использованием техники случайных
блужданий и теории восстановления (по аналогии с основополагающей
работой~\cite {Daley68}) получено следующее обобщение последнего
результата. Если в сис\-те\-ме $GI/G/1$ время обслуживания имеет
максимальный конечный момент $\gamma\in (3,\,4)$, а его ф.р.\
регулярно изменяется на бесконечности, то стационарный процесс
нагрузки~$\{W_n\}$ обладает долговременной зависимостью и имеет
параметр Херста
\begin{multline}
H_W :=\inf \left\{h: \limsup_{n\to \infty}
\fr {\D\left(\sum\limits_{i=1}^n W_i\right)}{n^{2h}}<\infty\right\}={}\\
{}= \fr{5-\gamma}{2}\in \left(\fr{1}{2},\,1\right)\,.
\label{carpio}
\end{multline}
В случае же стационарной сис\-те\-мы $M/G/1$, если (авто)корреляции
процесса очереди~$\nu_n$ (в моменты ухода) удовлетворяют условию
$\mathrm{cor}(\nu_0,\,\nu_n)\sim c n^{-\alpha}$, $\alpha\in (0,\,1),$ то
этот процесс с долговременной зависимостью также имеет параметр
Херста, удовлетворяющий~(\ref{carpio})~\cite{Carpio}.

Приведенные результаты демонстрируют присутствие долговременной зависимости у процессов
обслуживания, у которых одновременно длина цик\-ла регенерации имеет
конечные моменты выше \mbox{3-го}, и (как отмечалось) открывают возможность
статистического исследования таких систем на основе регенеративного
метода. Подробное обсуждение данной проблемы предполагается про\-вес\-ти в отдельной работе.

\section{Заключение}

В работе приведен обзор асимптотик вероятностей больших уклонений
процессов нагрузки/очереди в стационарной сис\-те\-ме обслуживания.
Рассмотрены (1)~сис\-те\-мы, в которых время обслуживания имеет тяжелый
хвост; (2)~регенеративные сис\-те\-мы с входным процессом Леви и
временем обслуживания, имеющим легкий хвост; и (3)~сис\-те\-мы с входным
потоком с долговременной за\-ви\-си\-мостью. В качестве иллюстрации
применения ТБУ подробно рассмотрена сис\-те\-ма $M/M/1$.

Кроме того,
рассмотрены процессы с долговременной зави\-си\-мостью,
 допускающие применение регенеративного имитационного моделирования. За
исключением п.~3.3, рассматриваются\linebreak лишь одноканальные сис\-те\-мы
обслуживания.\linebreak Важным аспектом работы является указание возможных
областей применения методов оценивания параметров асимптотик,
учитывая известную сложность такого оценивания на основе ТБУ.
Например, в результате проведенного анализа естественной областью
применения регенеративного метода оказываются сис\-те\-мы вида~2, а
также регенеративные процессы с долговременной зависимостью,
рас\-смот\-рен\-ные в разд.~5. Представляется перспективным оценивание
параметров асимптотик в таких сис\-те\-мах с помощью недавно предложенного
 варианта метода расщепления~\cite {BorMor}. В
остальных случаях в основном параметры асимптотик выражаются через
заданные параметры, и такие модели могут быть использованы для
тестирования методов оценивания. С другой стороны, эти методы могут
быть использованы для проверки качества асимптотик.

 Обзор не мог вместить многих важных моделей. Например, за его пределами остались асимптотические
результаты для сис\-тем с разделением процессора~\cite{Zwart}, для
жидкостных моделей~\cite{BoxmaDumas1, Jelenkovic}, для
стохастических сетей~\cite{Sigman98, BF,Borovkov}. В частности,
работа~\cite{BoxmaDumas1} содержит великолепный обзор жидкостных
моделей с входным процессом с долговременной зависимостью, а в книге~\cite{Borovkov}
даны асимптотические результаты, касающиеся больших уклонений некоторых сетевых процессов с легкими хвостами (в том
числе в системах поллинга). Все эти темы требуют отдельного обсуждения.
\vspace*{-6pt}

{\small\frenchspacing
{%\baselineskip=10.8pt
\addcontentsline{toc}{section}{Литература}
\begin{thebibliography}{99}

\bibitem{Will}
\Au{Willinger~W., Taqqu~M., Sherman~R., Wilson~D.}
Self-similarity through high-variability: Statistical analysis
of Ethernet LAN traffic at the source level~// IEEE/ACM Transactions
on Networking. 1997. Vol.~5. No.\,1. P.~71--86.

\bibitem{BorMor}
\Au{Бородина А.\,В., Морозов~Е.\,В.}
Ускоренное регенеративное
моделирование вероятности перегрузки односерверной очереди~// ОПиПМ,
2007. Т.~14. Вып.~3. С.~385--397.

\bibitem{Sigman}
\Au{Sigman~K.}
Appendix: A primer on heavy-tailed distributions~//
Queueing Systems, 1999. Vol.~33. P.~261--275.

\bibitem{Veraverbeke}
\Au{Embrechts~P., Veraverbeke~N.}
Estimates for the probability of ruin with special emphasis on the possibility of
large claims~// Insurance: Mathemaics and Economics, 1982. Vol.~1. P.~55--72.

\bibitem{Asmus}
\Au{Asmussen~S.}
Applied probability and queues.~--- NY: Springer, 2003. 2nd ed.

\bibitem{Feller}
\Au{Феллер В.}
Введение в теорию вероятностей и ее приложения. Т.~2.~--- М.: Мир, 1984.

\bibitem{AsKl}
\Au{Asmussen S., Kl\" uppelberg~C.}
Stationary $M/G/1$ excursions in the presence of heavy tails~// J. Appl. Probab., 1997. Vol.~34. P.~208--212.

\bibitem{Greiner}
\Au{Greiner M., Jobmann~M., Kl\"{u}ppelberg~C.}
Telecommunication traffic, queueing models and subexponential distributions~//
Queueing Systems, 1999. Vol.~33. P.~125--152.

\bibitem{AKS}
\Au{Asmussen S., Kl\"uppelberg~C., Sigman~K.}
Sampling at subexponential times, with queueing applications~// Stochastic
Process. Appl., 1999. Vol.~79. P.~265--286.

\bibitem{Asmus98}
\Au{Asmussen S.}
Subexponential asymptotics for stochastic processes: Extremal
behavior, stationary distributions and first passage probabilities~//
 Ann. Appl. Probab., 1998. Vol.~8. P.~354--374.

\bibitem {SorenJakob}
\Au{Asmussen S., M{\hspace*{-3pt}\protect\ptb{\o}\hspace*{2pt}}ller~J.\,R.}
Tail asymptotics for $M/G/1$ type queueing processes with subexponential
increments~//
Queueing Systems, 1999. Vol.~33. P.~153--176.

\bibitem{Foss Korsh}
\Au{Foss S., Korshunov~D.}
Sampling at random time with a heavy-tailed distribution~// Markov Processes Relat. Fields,
2000. Vol.~6. P.~543--568.

\bibitem{FK}
\Au{Foss S., Korshunov~D.}
Heavy tails in multiserver queue~// Queueing Systems, 2006. Vol.~52. P.~31--48.

\bibitem{Sigman98}
\Au{Huang T., Sigman~K.}
Steady-state asymptotics for tandem, split-match and other feedforward queues with heavy tailed
service~// Queueing Systems, 1999. Vol.~33. P.~233--259.

\bibitem{BF}
\Au{Baccelli~F., Foss~S.}
Moments and tails in monotone-separable stochastic networks~//
 Ann. Appl. Probab., 2004. Vol.~14. P.~612--650.

\bibitem{retrial}
\Au{Shang W., Liu~L., Li~Q.-L.}
Tail asysmptotics for the queue length in an $M/G/1$ retrial queue~//
 Queueing Systems, 2006. Vol.~52. P.~193--198.

\bibitem {GW}
\Au{Glynn P.\,W., Whitt~W.}
Logarithmic asymptotics for steady-state tail probabilities in a single-server queue~// Adv.
Appl. Probab., 1994. P.~131--156.

\bibitem {Abate1}
\Au{Abate J., Choudhury~G.\,L., Whitt~W.}
Exponential approximations for tail probabilities in queues.
I:~Waiting times~// Operations Research, 1995. Vol.~43. P.~885--901.

\bibitem{BQ}
\Au{Ganesh A., O'Connell~N., Wischik~D.}
Big queues.~--- Berlin: Springer-Verlag, 2004.

\bibitem {Kelly}
\Au{Kelly F.\,P.}
Notes on effective bandwidths~/ In: Stochastic
networks, theory and applications~// F.~Kelly, S.~Zachary, I.~Ziedins,
eds.~--- Oxford: Clarendon Press, 1996. P.~141--168.

\bibitem{Chang}
\Au{Chang C.-S.}
Performance guarantees in communications networks.~--- London: Springer-Verlag, 2000.

\bibitem{Irina07}
\Au{Vorobieva I., Morozov~E., Pagano~M., Procissi~G.}
A new regenerative estimator for effective bandwidth prediction~//
AMICT'2007 Proceedings.~--- Pet\-ro\-za\-vodsk: Pet\-ro\-za\-vodsk University
Press, 2008. Vol.~9. P.~175--187.

\bibitem {Sadowsky}
\Au{Sadowsky~J.}
Large deviations theory and efficient simulation of excessive backlogs
 in a $GI/GI/m$ queue~//
IEEE Transactions on Automatic Control, 1991. Vol.~36. No.\,12. P.~1383--1394.

\bibitem{Gnedenko}
\Au{Гнеденко Б.\,В., Коваленко~И.\,Н.}
Введение в теорию массового обслуживания.~--- М.: Наука, 1987.

\bibitem{Norros}
\Au{Norros I.}
A storage model with self-similar input~//
 Queueing Systems, 1994. Vol.~16. P.~387--396.

\bibitem{Duff-Connell}
\Au{Duffield N.\,G., O'Connell~N.}
 Large deviations and overflow probability for the general
 single-server queue with applications~//
 Math. Proc. Cambridge Philos. Soc., 1995. Vol.~118. No.\,2. P.~363--374.

\bibitem {Duffy}
\Au{Duffy K., Lewis~T., Sullivan~W.\,G.}
Logarithmic asymptotics for the supremum of a stochastic process~// Annals of
Appl. Probab., 2003. Vol.~13. No.\,2. P.~430--445.

\bibitem{Leland}
\Au{Leland W., Taqqu~M., Willinger~W., Wilson~D.}
On the self-similar nature of ethernet traffic~//
IEEE/ACM Transactions on Networking, 1994. Vol.~2. No.\,1. P.~1--15.

\bibitem{Taqqu}
\Au{Taqqu M.\,S., Willinger~W., Sherman~R.}
Proof of a fundamental result in self-similar traffic modeling~// Computer
Communication Review, 1997. Vol.~27. P.~5--23.

\bibitem{Kaj}
\Au{Kaj~I.}
Convergence of scaled renewal processes to fractional
Brownian motion. Sci. report.~--- Dept. of Mathematcis, Uppsala
University, 1999. No.\,11. P.~1--28.

\bibitem{Samorodnitsky}
\Au{Samorodnitsky~G.}
Long range dependence, heavy tails and rare events.~--- Aarhus: Aarhus University, MaPhySto, 2002. P.~1--84.

\bibitem{Resnick}
\Au{Resnick S.}
Heavy tail modeling and teletraffic data~// The Annals of Statistics, 1997. Vol.~25. No.\,5. P.~1805--1869.

\bibitem{Norr95}
\Au{Norros~I.}
Studies on a model for connectionless traffic, based on fractional
 brownian motion~// Conference on Applied Probability in Engineering,
 Computer and Communications Sciences.~---
 Paris: INRIA/ ORSA/ TIMS/ SMAI, 1993. P.~1--13.

\bibitem{BoxmaDumas1}
\Au{Boxma O.\,J., Dumas~V.}
Fluid queues with long-tailed activity periods~//
 Computer Communications, 1998. Vol.~21. P.~1509--1529.

\bibitem{Jelenkovic}
\Au{Jelenkovi\'c P.\,R., Lazar~A.\,A.}
Subexponential asymptotics of a Markov-modulated random walk with queueing
 applications~// J.~Appl.
Probab., 1998. Vol.~35. P.~338--339.

\bibitem{RS}
\Au{Resnick S., Samorodnitsky~G.}
Activity periods of an infinite server queue and performance of certain
 heavy tailed fluid
queues~// Queueing Systems, 1999. Vol.~33. P.~43--71.

\bibitem{BoxmaDumas}
\Au{Dumas V., Simonian~A.}
Asymptotic bounds for the fluid queue fed by sub-exponential on/off
 sources~// Adv. Appl. Probab., 2000. Vol.~32. P.~244--255.

\bibitem{Borovkov1}
\Au{Боровков А.\,А.}
Вероятностные процессы в теории массового обслуживания. -- М.: Наука, 1972.

\bibitem{Deng}
\Au{Deng Q.} Queues with regular variation. Ph.D.\ Thesis.~--- Eindhoven:
Eindhoven University of Technology, 2001.

\bibitem{Zwart} %40
\Au{Zwart A.}
Queueing systems with heavy tails. Ph.D.\ Thesis.~--- Eindhoven: Eindhoven University of Technology, 2001.

\bibitem{Mandjes}%41
\Au{Mandjes M.}
 Large deviations of Gaussian queues.~--- Chichester: Wiley, 2007.

\bibitem{SS} %42
\Au{Morozov E.\,V.}
Self-similarity and long-range dependence in network traffic
modeling~// FDPW'99 ``Developments in Distributed
Systems and data Communications'' Proceedings, 1999. Vol.~2. P.~32--40.

\bibitem{GZ} %43
\Au{Zeevi A., Glynn~P.}
On the maximum workload of a queue fed by fractional Brownian motion~// Annals of Appl. Probab., 2000.
Vol.~10. No.\,4. P.~1084--1099.

\bibitem{Daley} %44
\Au{Daley D.\,J.}
The Hurst index of long-range dependent renewal process~// Annals of Probability, 1999. Vol.~27. P.~2035--2041.

\bibitem{Daley68}
\Au{Daley D.\,J.}
 The serial correlation coefficients of waiting times in a stationary single server queue~//
Austr. Math. Society, 1968. Vol.~8. P.~683--699.

\bibitem {Wolff} %46
\Au{Wolff R.\, W.}
Stochastic modeling and the theory of queues.~--- Prentice-Hall, 1989.

\bibitem{GI}
\Au{Glynn P., Iglehart~D.}
Conditions for the applicability of the regenerative method // Management Sci., 1993. Vol.~39. P.~1108--1111.

\bibitem{Morozov} %48
\Au{Morozov E.\,V.}
Weak regeneration in modeling of queuing processes~// Queueing Systems, 2000. Vol.~46. P.~293--313.

\bibitem{MorBod}
\Au{Морозов Е.\,В., Белый~А.\,В., Боденов~Д.\,В.}
Расширенная регенерация: применения к анализу сетевого трафика~// ОПиПМ, 2007. Т.~14. Вып.~6. С.~1022--1042.

\bibitem{Morozov1}
\Au{Morozov E.}
Communications systems: Rare event simulation and effective bandwidths.~--- Pamplona: Universidad Publica de Navarra, 2004.

\bibitem{Teugels}
\Au{De Meyer A., Teugels~J.\,L.\,T.}
On the asymptotic behaviour of the distribution of the busy period and service time in M/G/1~// J. Appl. Probab.,
1980. Vol.~17. P.~802--813.

\bibitem{Daley-Vesilo}
\Au{Daley D.\,J., Vesilo~R.}
Long range dependence of point processes, with queuing examples~// Stochastic Processes Appl., 1997. Vol.~70. P.~265--282.

\bibitem {Carpio}
\Au{Carpio K.\,J.\,E.}
Long-range dependence of stationary proc\-esses in single-server queues~// Queueing Systems, 2007. Vol.~55. P.~123--130.

\label{end\stat}

\bibitem{Borovkov}
\Au{Боровков~А.\,А.}
Эргодичность и устойчивость случайных процессов.~--- М.: Эдиториал УРСС, 1999.

%\bibitem{BorMor1}
%\Au{Бородина А.\,В., Морозов~Е.\,В.}
%Ускоренное состоятельное оценивание вероятности большой загрузки в
%системах $M/G/1$, $GI/G/1$~// Статистические методы оценивания и проверки
%гипотез.~--- Пермь: Пермский Университет, 2007. Вып.~68. С.~124--140.


 \end{thebibliography}
}
}
\end{multicols}