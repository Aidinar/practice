{ %\Large  
{ %\baselineskip=16.6pt

\vspace*{-48pt}
\begin{center}\LARGE
\textit{Предисловие}
\end{center}

%\vspace*{2.5mm}

\vspace*{15mm}

\thispagestyle{empty}

{ %\small
Вниманию читателей журнала <<Информатика и её применения>> предлагается очередной 
тематический выпуск <<Вероятностно-статистические методы и задачи информатики и информационных технологий>>. 
Первый тематический выпуск журнала по данному направлению вышел в 2008~году (т.~2, вып.~2). 

Статьи, собранные в данном журнале, посвящены разработке новых вероятностно-статистических методов, ориентированных 
на применение к решению конкретных задач информатики и информационных технологий, а также~--- в ряде случаев~--- 
и других прикладных задач. Проблематика, охватываемая публикуемыми работами, развивается в рамках научного 
сотрудничества между Институтом проблем информатики Российской академии наук (ИПИ РАН) и Факультетом вычислительной 
математики и кибернетики Московского государственного университета им.~М.\,В.~Ломоносова, в частности в ходе работ 
над совместными научными проектами. Многие из авторов статей, включенных в данный номер журнала, являются 
активными участниками традиционного международного семинара по проблемам устойчивости стохастических моделей, 
руководимого В.\,М.~Золотарёвым и В.\,Ю.~Королёвым; ежегодные сессии этого семинара проводятся под эгидой МГУ и ИПИ~РАН. 

Наряду с представителями ИПИ РАН и МГУ, в число авторов данного выпуска журнала входят ученые из Карельского научного 
центра РАН, Национального исследовательского ядерного университета <<МИФИ>>, Вологодского государственного 
педагогического университета, Казанского государственного университета, University College (Корк, Ирландия) 
и~Centre for Process Systems Engineering (Лондон, Великобритания).

Ряд статей журнала посвящен тематике, весьма активно разрабатываемой в течение многих лет специалистами ИПИ 
РАН и МГУ,~--- математическим методам моделирования информационно-те\-ле\-ком\-му\-ни\-ка\-ци\-он\-ных систем, основанных 
на теории массового обслуживания и теории телетрафика. 

В~статье А.\,В.~Печинкина, И.\,А.~Соколова и В.\,В.~Чаплыгина 
рассматривается важная для приложений задача анализа многолинейной системы массового обслуживания с ненадежными 
приборами. Предложены методы расчета стационарного распределения числа заявок в системе при различных вариантах 
функционирования системы. 

Статья А.\,И.~Зейфмана, Я.\,А.~Сатина, А.\,В.~Коротышевой и Н.\,А.~Терешиной 
посвящена изучению предельных характеристик сис\-те\-мы обслуживания с катастрофами в предположении, что 
интенсивности катастроф зависят от числа требований в системе. Получены достаточные условия слабой 
эргодичности процесса, описывающего число требований в системе, и соответствующие оценки. 

В~работе Е.\,В.~Морозова исследуются асимптотики вероятностей больших уклонений стационарной 
очереди в системах обслуживания для различных классов распределений времен обслуживания. 

В статье В.\,Ю.~Бородакия на основе теории телетрафика построена модель сетецентрической системы; 
проведен анализ модели отдельного звена такой системы, получены формулы для вычисления вероятности 
блокировки запроса из-за отсутствия достаточной для передачи блока данных ширины полосы пропускания.

Следующая группа статей выпуска посвящена разработке и применению стохастических методов для решения 
ряда прикладных задач. 

Статья Г.~Темнова (Ирландия) и С.~Кучеренко (Великобритания), публикуемая на 
английском языке, посвящена моделированию случайных сумм применительно к одной задаче актуарной математики. 

В~работе О.\,В.~Шестакова развиваются вероятностные методы анализа устойчивости реконструкции изображений 
в задачах эмиссионной томографии. 

Статья А.\,Н.~Чупрунова и Б.\,И.~Хамдеева посвящена асимптотическому 
анализу вероятности ис\-прав\-ле\-ния ошибок при помехоустойчивом кодировании.

Заключительная группа статей посвящена развитию перспективных теоретических вероятностно-статистических 
методов, которые могут найти широкое применение в различных задачах информатики и информационных технологий. 

В статье В.\,Ю.~Королёва рассматриваются и развиваются математические модели, описывающие распределение 
физических размеров частиц при дроблении. Подобные модели могут довольно успешно использоваться (и используются) 
при описании самых разных объектов, в том числе объемов сообщений в вычислительных или телекоммуникационных 
системах, капиталов фирм, размеров доходов и~др. 

Работа И.\,Г.~Шевцовой посвящена уточнению оценок для 
характеристических функций~--- основного аппарата, используемого при построении конкретных числовых 
оценок точности нормальной аппроксимации (такая аппроксимация весьма важна при математическом моделировании 
и анализе многих реальных систем, в том числе  информационных и телекоммуникационных). 

В~статье В.\,Е.~Бенинга и О.\,О.~Лямина рассмотрены вопросы проверки гипотез о параметрах 
обобщенного распределения Лапласа, которое находит широкое применение при математическом 
моделировании многих процессов в телекоммуникационных системах, экономике, финансовом деле, 
технике и других областях. Получена формула для предела отклонения мощности асимптотически 
оптимального критерия от мощности наилучшего критерия в случае обобщенного распределения Лапласа. 

Редакционная коллегия журнала выражает надежду, что данный тематический  выпуск будет интересен 
специалистам в области теории вероятностей и математической статистики и их применения к решению 
задач информатики и информационных технологий.

%\vspace*{2mm}
\vspace*{15mm}
\noindent
Заместитель главного редактора журнала <<Информатика и её применения>>,\\
директор ИПИ РАН, академик  \hfill
\textit{И.\,А.~Соколов}\\[-6pt]

\noindent
Редактор-составитель тематического выпуска,\\
профессор кафедры математической статистики\\
факультета вычислительной математики и кибернетики МГУ им.~М.\,В.~Ломоносова,\\
ведущий научный сотрудник ИПИ РАН,\\
доктор физико-математических наук\hfill
 \textit{В.\,Ю.~Королёв}


} }
}