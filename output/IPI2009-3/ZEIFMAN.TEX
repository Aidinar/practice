\def\stat{zeifman}

\def\tit{О ПРЕДЕЛЬНЫХ ХАРАКТЕРИСТИКАХ СИСТЕМЫ ОБСЛУЖИВАНИЯ $M(t)/M(t)/S$ С КАТАСТРОФАМИ}
\def\titkol{О предельных характеристиках системы обслуживания $M(t)/M(t)/S$ с катастрофами}

\def\autkol{А.\,И.~Зейфман, Я.\,А.~Сатин, А.\,В.~Коротышева, Н.\,А.~Терёшина}
\def\aut{А.\,И.~Зейфман$^1$, Я.\,А.~Сатин$^2$, А.\,В.~Коротышева$^3$, Н.\,А.~Терёшина$^4$}

\titel{\tit}{\aut}{\autkol}{\titkol}

%{\renewcommand{\thefootnote}{\fnsymbol{footnote}}\footnotetext[1]
%{Работа выполнена при финансовой поддержке РФФИ, грант 08-01-00567.}}

\renewcommand{\thefootnote}{\arabic{footnote}}
\footnotetext[1]{Вологодский государственный
педагогический университет,  Институт проблем информатики РАН и ВНКЦ
ЦЭМИ РАН,  a\_zeifman@mail.ru}
\footnotetext[2]{Вологодский государственный педагогический
университет, yacovi@mail.ru}
\footnotetext[3]{Вологодский государственный педагогический
университет,  a\_korotysheva@mail.ru}
\footnotetext[4]{Вологодский государственный педагогический
университет,  nataliya\_tereshi@mail.ru}


\Abst{Рассмотрена модель системы обслуживания $M(t)/M(t)/S$ с катастрофами в общем случае,
когда интенсивности катастроф зависят от числа требований в системе. Получены достаточные условия
слабой эргодичности процесса, описывающего число требований в системе, и соответствующие оценки.
Рассмотрено несколько примеров построения предельных характеристик системы.}

\vspace*{2pt}


\KW{нестационарные марковские системы обслуживания;
процесс рождения и гибели с катастрофами; слабая эргодичность; оценки;
предельные характеристики; аппроксимация}

      \vskip 28pt plus 9pt minus 6pt

      \thispagestyle{headings}

      \begin{multicols}{2}

      \label{st\stat}

\section{Введение}

Простейшие (стационарные) модели систем обслуживания с катастрофами начали
рассматриваться не очень давно (см., например,~[1--4]). В таких моделях предполагается, что если в системе обслуживания имеется ненулевое
число требований, то с ненулевой интенсивностью возможна катастрофа, т.\,е.\ потеря всех требований, с дальнейшим
продолжением функционирования системы обслуживания. Нестационарные марковские модели (процессы рождения и гибели)
с катастрофами изучались в  работах~\cite{z08, z09} для случая, когда интенсивности катастроф не зависят от числа
требований в сис\-те\-ме. В настоящей работе изучается модель сис\-те\-мы обслуживания
 $M(t)/M(t)/S$ с катастрофами в более общем случае, когда интенсивности катастроф
 зависят от числа требований в сис\-те\-ме обслуживания. При этом удается получить достаточно общие условия,
 гарантирующие наличие слабой эргодичности для процесса, описывающего число требований в такой сис\-те\-ме
 обслуживания, и получить оценки скорости сходимости, гарантирующие возможность приближенного построения
 предельных характеристик сис\-темы.

Обозначим через $X=X(t)$, $t\geq 0$, число требований в момент~$t$ для описываемой модели.
Тогда $X=X(t)$ является процессом рождения и гибели с катастрофами.  Интенсивности рождения,
гибели и катастрофы для процесса в случае, если  $X(t)= n$, есть  $\lambda_n(t) = \lambda(t)$,
$\mu_n(t) = \min(n,S)\mu(t)$ и $\xi_n(t)= \zeta_n \xi (t)$ соответственно.

Обозначим через $p_{ij}(s,t)$ вероятности перехода,
а через  $p_i(t)$~--- вероятности состояний для процесса $X=X(t)$:
\begin{align*}
p_{ij}(s,t)&=Pr\left\{ X(t)=j\left| X(s)=i\right. \right\}\,;\\
p_i(t)&=Pr\left\{ X(t) =i \right\}
\end{align*}

Тогда для описания процесса получаем прямую систему Колмогорова
\begin{equation}
%\begin{cases}
\left.
\begin{array}{l}
\displaystyle \fr{dp_0}{dt} = -\lambda_0 (t) p_0 +\mu_1(t)p_1 + \sum\limits_{k \ge 1}\xi_k(t)p_k\,;  \\
\displaystyle \fr{dp_k}{dt} = \lambda_{k-1} (t)p_{k-1} -\left(\lambda_{k} (t) +  \mu_{k}(t)+{}\right. \\
\hspace*{35pt}\left.{}+\xi_k(t)\right)p_k +\mu_{k+1}(t) p_{k+1}\,,\ \   k \ge 1\,.
%\end{cases}
\end{array}
\right \}
\label{zkts01}
\end{equation}

Пусть ${\bf p}(t)=\left(p_0(t),p_1(t),\dots\right)^T$, $t>0$,~---
вектор-столбец вероятностей состояний процесса, а\linebreak
$A(t) = \left\{a_{ij}(t),\: t\geq 0\right\}$~--- матрица системы~(\ref{zkts01}).

В работе будет предполагаться, что  интенсив\-ности поступления и обслуживания требова-\linebreak ний~$\lambda(t)$
и~$\mu(t)$ локально интегрируемы на $[0;\infty)$. Будем считать <<базисную>> интенсивность катастрофы
$\xi (t)$ локально интегрируемой на $[0;\infty)$, а коэффициенты состояний~--- ограниченными, т.\,е.\linebreak
$0 \leq \zeta_n \leq M $ при некотором $M < \infty$. Тогда систему~(\ref{zkts01}) можно рассматривать
как дифференциальное уравнение
\begin{equation*}
\fr{d{\bf p}}{dt}= A\left( t\right) {\bf p}\,, \quad t\ge 0\,,
\end{equation*}
\noindent в пространстве последовательностей~$l_1$ с ограниченной почти при всех $t \ge 0$ локально интегрируемой
оператор-функцией~ $A(t)$.
Следовательно, можно применять общий подход, предложенный впервые в заметке~\cite{gm} и развитый затем
в~\cite{z06, z08b}. Метод опирается на две основные составляющие: понятие и оценки, связанные с логарифмической
нормой операторной функции, и специальные преобразования редуцированной матрицы интенсивностей рассматриваемой
марковской цепи, и позволяет получать явные и точные оценки.

\section{Слабая эргодичность}

Напомним основные определения.
\medskip

Марковскую цепь $X(t)$ назовем \emph{слабо эргодичной},
если $\|{\bf p}^*(t)-{\bf p}^{**}(t)\| \to 0$ при $t \to \infty$ для любых начальных распределений
вероятностей состояний ${\bf p}^*(0), {\bf p}^{**}(0)$ (здесь и далее через $\|{\bf x}\|$ обозначена $l_1$-норма).

\medskip

Положим $E_k(t) = E\left\{X(t)\left|X(0)=k\right.\right\}$~--- математическое ожидание процесса в момент~$t$
при условии, что в нулевой момент времени он находится в состоянии~$k$, иногда будет встречаться также несколько
более общее выражение
$$
E_{\bf p}(t) =\sum_{k \ge 0} E\left\{X(t)\left|X(0)=k\right.\right\}p_k(0)\,.
$$

\medskip

Будем говорить, что марковская цепь~$X(t)$ имеет \textit{предельное среднее}  $\varphi (t)$, если
$$
 \lim_{t \to \infty }  \left(\varphi (t) - E_k(t)\right) = 0
$$
при любом  $k$.

\medskip

В настоящей работе будет изучаться слабая эргодичность и сопутствующие свойства числа требований
в рассматриваемой системе обслуживания для следующих важных ситуаций:
\begin{enumerate}[(1)]
\item интенсивности катастроф существенны при любой длине очереди;
\item интенсивность обслуживания требований достаточно велика;
\item достаточно велика интенсивность поступления требований, а
интенсивности катастроф существенны для ситуаций, когда длина очереди
(количество требований в системе) пропорционально некоторому натуральному \mbox{числу}.
\end{enumerate}


\medskip

\noindent
\textbf{Теорема 1.}\ \textit{Пусть}
\begin{equation}
\inf_n \zeta_n = \zeta > 0
\label{zkts02}
\end{equation}
\textit{и, кроме того,  найдется $\varepsilon > 0$ такое, что}
\begin{equation}
\int\limits_0^{\infty} \left(\zeta \xi(t)-\varepsilon\lambda(t)\right)\, dt = +\infty\,.
\label{zkts02a}
\end{equation}
\textit{Тогда процесс $X(t)$ слабо эргодичен и имеет предельное среднее.
Если при этом в качестве предельного режима и предельного среднего выбрать режим ${\bf \pi}(t)$ и
среднее~$\phi(t)$, соответствующие начальному условию $X(0) = 0$, то  при любом начальном условии вида
$X(0) = k$ справедливы следующие оценки:}
\begin{equation}
\|{\bf p}(t) - {\bf \pi}(t)\| \le 4\left(1+\varepsilon\right)^k \varepsilon^{-1} e^{-\int\limits_0^t
\left(\zeta\xi(\tau) - \varepsilon \lambda(\tau)\right)\, d\tau}
\label{zkts0201}
\end{equation}
\textit{и}
\begin{equation*}
\left|E_k(t) - E_0(t)\right| \le \fr{4\left(1+\varepsilon\right)^k}{\varepsilon \omega} e^{-\int\limits_0^t
\left(\zeta\xi(\tau) - \varepsilon \lambda(\tau)\right)\, d\tau}\,.
%\label{zkts0202}
\end{equation*}

\medskip

\noindent
Д\,о\,к\,а\,з\,а\,т\,е\,л\,ь\,с\,т\,в\,о\,.\
Полагая
$$
p_0(t) = 1 - \sum\limits_{i \ge 1} p_i(t)\,,
$$
получаем аналогично тому, как это проделано в~\cite{z09},
следующую систему:
\begin{equation}
\label{zkts03}
\fr{d{\bf z}(t)}{dt}=B(t){\bf z}(t)+{\bf f}(t)\,,
\end{equation}
в которой
\end{multicols}

\vspace*{6pt}

\hrule

\vspace*{6pt}

\noindent
$$
B(t) =
\begin{pmatrix}
 -(\lambda_0+\lambda_1+\mu_1+\xi_1) & \mu_2 -\lambda_0 & -\lambda_0 & -\lambda_0 & \cdots & \cdots\\
  \lambda_1 & -(\lambda_2+\mu_2+\xi_2) & \mu_3 & 0 &  0 &\cdots\\
 0 & \lambda_2 & -(\lambda_3+\mu_3+\xi_3) & \mu_4  & 0  & \cdots \\
\vdots & \vdots & \vdots & \vdots & \vdots & \ddots \\
\end{pmatrix}\,;
$$
$$
{\bf z}(t) = \left(p_1(t),p_2(t),\dots\right)^\mathrm{T}\,,\quad
{\bf f}(t) = \left(\lambda_0(t),0,0,\dots\right)^\mathrm{T}.
$$

\vspace*{3pt}

\hrule

\begin{multicols}{2}

\noindent
Обозначим оператор Коши линейной неоднородной системы~(\ref{zkts03}) через~ $V(t,z)$, введем в рассмотрение матрицу
\begin{equation*}
  D=
  \begin{pmatrix}
  d_0 & d_0 & d_0 & \cdots \\
  0   & d_1 & d_1 & \cdots \\
  0   & 0   & d_2 & \cdots \\
  \vdots & \vdots & \ddots & \ddots
\end{pmatrix}
\end{equation*}
\noindent и пространство последовательностей
$\ell_{1D}=$\linebreak
$=\;\left\{{\bf z} =(p_1,p_2,\ldots)\right\}$ таких, что $\|{\bf z}\|_{1D}=\|D{\bf
z}\|_1<\infty$, где  $d_i$~--- пока не выбранные положительные числа.

Теперь рассматриваемый процесс можно исследовать таким же образом, как и в~\cite{z09}.


\medskip

Положим
\begin{multline*}
\alpha_{k}\left( t\right) = \lambda _k\left( t\right) +\mu_{k+1}\left( t\right) +\xi_{k+1} \left( t\right) -{}\\
{}- \fr{d_{k+1}}{d_k}\,\lambda _{k+1}\left(
t\right) -\fr{d_{k-1}}{d_k}\,\mu _k\left( t\right), \quad k \ge 0,
\end{multline*}
и
\begin{equation*}
\alpha\left( t\right) = \inf_{k\geq 0} \alpha_{k}\left( t\right)\,.
\end{equation*}

Тогда получаем
\begin{multline}
\gamma \left(B(t)\right)_{1D} = \sup\limits_{i \ge 0}
\left(\fr{d_{i+1}}{d_i}\,
 \lambda_{i+1}(t) - \left(\lambda_i(t) +{}\right.\right.\\
\left.\left. {}+\mu_{i+1}(t)
+\xi_{i+1}(t)\right) +  \fr{d_{i-1}}{d_i}\, \mu_i(t) \right) = -
\alpha(t)\,.
\label{zkts04}
\end{multline}
Полагая теперь $d_{-1}=d_0=1$, $d_{k+1}=(1+\varepsilon)d_k$, $k \ge 0$,  получаем с учетом~ (\ref{zkts04})
следующую оценку:
\begin{multline}
\gamma \left(B(t)\right)_{1D} \le{}\\
{}\le  -\left(\zeta\xi(t) + \fr{\varepsilon}{1+\varepsilon}\left(S\mu(t)-
(1+\varepsilon)\lambda(t)\right)\right) \le{}\\
{}\le -\left(\zeta\xi(t)  - \varepsilon\lambda(t)\right)  = -\alpha_*(t)\,,
\label{zkts040}
\end{multline}
причем из~(\ref{zkts02}) вытекает, что
$$
\int\limits_0^{\infty} \alpha_*(t) \, dt = + \infty\,.
$$
Теперь с учетом проведенного ранее (см., например,~\cite{z06}) сравнения норм получаем слабую
эргодичность процесса и оценку
\begin{multline*}
\|{\bf p}^*(t)-{\bf p}^{**}(t)\| \le{}\\
{}\le4 e^{-\int\limits_0^t
\left(\zeta\xi(\tau) - \varepsilon \lambda(\tau)\right)\, d\tau} \|{\bf p}^*(0)-{\bf p}^{**}(0)\|_{1D}\,.
%\label{zkts04a}
\end{multline*}
Выбирая теперь ${\bf p}^*(0) = {\bf \pi}(0)={\bf e}_0$, ${\bf p}^{**}(0) = {\bf p}(0)=$\linebreak
$=\; {\bf e}_k$,
получаем неравенство~(\ref{zkts0201}).

Рассматривая величину $\omega = \inf\limits_{k \ge 1} d_k/k$ и убеждаясь, что
$\omega = \inf\limits_{k \ge 1} \left(1+\varepsilon\right)^k/k > 0 $,
получаем снова из сравнения норм существование предельного среднего и оценку
\begin{multline*}
|E_{\bf p^*}(t) - E_{\bf p^{**}}(t)| \le{}\\
{}\le \fr{4}{\omega}\, e^{-\int\limits_0^t
\left(\zeta\xi(\tau) - \varepsilon \lambda(\tau)\right)\, d\tau} \|{\bf p}^*(0)-{\bf p}^{**}(0)\|_{1D}\,.
%\label{zkts04b}
\end{multline*}
Для завершения доказательства достаточно выбрать снова ${\bf p}^*(0) = {\bf \pi}(0)= {\bf e}_0$,
${\bf p}^{**}(0) = {\bf p}(0)= {\bf e}_k$.

\bigskip
\noindent
\textbf{Замечание 1.}
Пусть все интенсивности (по\-ступ\-ления, обслуживания требований, катастроф)
\mbox{1-пе}\-ри\-о\-дичны. Тогда:
\begin{enumerate}[(1)]
\item
вместо условия~(\ref{zkts02a}) достаточно потребовать, чтобы $\int\limits_0^1\xi(t)\,dt>0$;
\item
можно выбрать <<особые>> предельные характеристики, а именно  существует 1-пе\-ри\-о\-ди\-ческий предельный режим
${\bf \pi}(t)$ и со\-от\-вет\-ствующее ему 1-периодическое предельное среднее $\phi(t)$;
\item эти предельные характеристики можно приближенно построить, пользуясь методикой,
разработанной в~\cite{z09, z06}, с использованием усеченного процесса и оценок скорости схо\-ди\-мости.
\end{enumerate}


\medskip

\noindent
\textbf{Теорема 2.}  \textit{Пусть при некотором $\varepsilon > 0$ выполняется условие}
\begin{equation}
\int\limits_0^{\infty} \left(S\mu(t)-\left(1+\varepsilon\right)\lambda(t)\right)\, dt = +\infty\,.
\label{zkts05}
\end{equation}
\textit{Тогда процесс $X(t)$ слабо эргодичен и имеет предельное среднее. Если при этом в качестве предельного режима и предельного среднего выбрать режим ${\bf \pi}(t)$ и среднее $\phi(t)$, соответствующие начальному условию $X(0) = 0$, то
при любом начальном условии вида $X(0) = k$ справедливы следующие оценки:}
\begin{multline*}
\|{\bf p}(t) - {\bf \pi}(t)\| \le{}\\
{}\le 4\left(1+\varepsilon\right)^k \varepsilon^{-1} e^{-\int\limits_0^t
\varepsilon(1+\varepsilon)^{-1}\left(S\mu(\tau)-(1+\varepsilon)\lambda(\tau)\right)\, d\tau} %\label{zkts05a}\\
\end{multline*}

\vspace*{-12pt}

\noindent
\textit{и}
\begin{multline*}
\left|E_k(t) - E_0(t)\right| \le{}\\
{}\le \fr{4\left(1+\varepsilon\right)^k}{\varepsilon \omega} e^{-\int\limits_0^t
\varepsilon \left( 1+\varepsilon \right)^{-1}\left(S\mu(\tau)-(1+\varepsilon)\lambda(\tau)\right)\, d\tau}\,.
%\label{zkts05b}
\end{multline*}


\medskip

\noindent
Д\,о\,к\,а\,з\,а\,т\,е\,л\,ь\,с\,т\,в\,о\ проводится так же, как в предыду\-щей теореме. При этом, строя то же вспомогательное пространство $\ell_{1D}$, вместо (\ref{zkts040}) получаем
неравенство
\begin{multline*}
\gamma \left(B(t)\right)_{1D} \le{}\\[3pt]
{}\le  -\left(\zeta\xi(t) + \fr{\varepsilon}{1+\varepsilon}\left(S\mu(t)-
(1+\varepsilon)\lambda(t)\right)\right)  \le{}\\[3pt]
{}\le -\fr{\varepsilon}{1+\varepsilon}\left(S\mu(t)-(1+\varepsilon)\lambda(t)\right) = -\alpha_*(t)\,,
%\label{zkts06}
\end{multline*}
и далее оценки получаются тем же образом.


\bigskip
\noindent
\textbf{Замечание 2.}\ Если все интенсивности (поступления, обслуживания требований, катастроф) 1-пе\-ри\-о\-дич\-ны, то:
\begin{enumerate}[(1)]
\item вместо условия~(\ref{zkts05}) достаточно потребовать, чтобы $\int\limits_0^1 \left(S\mu(t) - \lambda(t)\right) \,dt>0$;
\item
можно выбрать <<особые>> предельные харак\-теристики, а именно  существует 1-пе\-ри\-о\-ди\-че\-ский предельный режим~${\bf \pi}(t)$
и соответствующее ему 1-периодическое предельное\linebreak среднее~$\phi(t)$;
\item
эти предельные характеристики можно приближенно построить, пользуясь методикой,
разработанной в~\cite{z09, z06},
с использованием усеченного процесса и оценок скорости схо\-ди\-мости.
\end{enumerate}

\medskip

Рассмотрим, наконец, ситуацию, когда условия предыдущих теорем не выполняются.

\bigskip

\noindent
\textbf{Теорема 3.}\ \textit{Пусть при некотором  натуральном $N$ выполняется условие}
\begin{equation*}
\inf_n \zeta_{nN} = \zeta > 0
%\label{zkts10}
\end{equation*}
\textit{вместо}~(\ref{zkts02}). \textit{Пусть, кроме того,  найдется $\delta > 0$ такое, что}
\begin{equation*}
\int\limits_0^{\infty} g(t)\, dt = +\infty\,,
%\label{zkts11}
\end{equation*}
\textit{где}
\begin{multline*}
g(t) = \min \left(\zeta \xi(t)-\delta\lambda(t) - \delta S\mu(t),\right.\\
\left.
\lambda(t) - \left(1+\delta\right) S\mu(t)\right)\,.
%\label{zkts11a}
\end{multline*}
\textit{Тогда процесс $X(t)$ слабо эргодичен и имеет предельное среднее.}

\medskip

\noindent
Д\,о\,к\,а\,з\,а\,т\,е\,л\,ь\,с\,т\,в\,о\ проводится по той же схеме, что и в теореме~1.
При этом вспомогательная  последовательность $\{d_k\}$ выбирается следующим образом:
$d_{-1}=d_0=1$, $d_{k+1}=(1+\varepsilon)^{-1}d_k,$ если $k \neq iN-1$, $d_{k+1}=(1+\varepsilon)^N d_k$
при $k = iN-1$, где положительное число $\varepsilon < \delta$ таково, что $(1+\varepsilon)^N - 1 < \delta$.
Тогда получаем
\begin{multline*}
\alpha_{k}\left( t\right) \ge \lambda\left(t\right) + S\mu \left( t\right) + \zeta_{k+1} \xi(t) -{}\\[4pt]
{}- S\mu(t)(1+\varepsilon) - (1+\varepsilon)^{-1}
\lambda\left(t\right) \ge {}\\[4pt]
{}\ge \fr{\varepsilon}{1+\varepsilon}\left(\lambda (t)  - (1+\varepsilon)S\mu(t)\right)
\end{multline*}
\textit{при  $k \neq iN-1$ и}
\begin{equation*}
\alpha_{k}\left( t\right) \ge  \zeta \xi(t) - \varepsilon S\mu(t) - \left((1+\varepsilon)^{N} -1\right)
\lambda\left(t\right)
\end{equation*}
\textit{при}  $k = iN-1$.

Отсюда вытекает, что
\begin{equation*}
\gamma \left(B(t)\right)_{1D} \le - g(t)\,,
\end{equation*}
причем последовательность~$\{d_k\}$ построена так, что выполняется и условие
$\omega = \inf\limits_{k \ge 1} d_k/k > 0$. Отсюда и следует утверждение теоремы.
\bigskip

\noindent
\textbf{Следствие 1.}\
Если при выполнении условий теоремы~3 в качестве предельного режима и предельного среднего выбрать режим~${\bf \pi}(t)$
и среднее~$\phi(t)$, соответствующие начальному условию $X(0) = 0$, то  при любом начальном условии вида
$X(0) = k$ справедливы следующие оценки:
\begin{equation*}
\|{\bf p}(t) - {\bf \pi}(t)\| \le \frac{4r_k}{d} e^{-\int\limits_0^t
G(\tau)\, d\tau} \label{zkts21}
\end{equation*}
и
\begin{equation*}
\left|E_k(t) - E_0(t)\right| \le \fr{4r_k}{d \omega} e^{-\int\limits_0^t
G(\tau)\, d\tau}\,,
%\label{zkts22}
\end{equation*}
где $ d = \min d_k$, $r_k = \sum_{i \le k-1} d_i$, а
\begin{multline*}
\!\!\!\!G(t) =  \min \left(
\vphantom{\fr{\varepsilon}{1+\varepsilon}}
\zeta \xi(t) - \varepsilon S\mu(t) - \left(
\left(1+\varepsilon\right)^{N} -1\right)\lambda\left(t\right),\right.\\
\left.\fr{\varepsilon}{1+\varepsilon}\left(\lambda (t)  - \left(1+\varepsilon\right)
S\mu(t)\right)\right) \ge g(t)\,.
\end{multline*}

\bigskip
\noindent
\textbf{Замечание 3.}\
В случае 1-периодических интенсивностей за счет более сложного выбора вспомогательной последовательности~$\{ d_k\}$
условия теоремы можно существенно ослабить.


\section{Пример}

Рассмотрим систему обслуживания $M(t)/ M(t)/$ $100$ с катастрофами в случае периодических интенсивностей
в разных ситуациях, гарантиру\-ющих слабую эргодичность модели, и вычислим предельное среднее~$\phi(t)$, а также асимптотику величины
$J_0(t)$~--- вероятности того, что в момент~$t$ очередь пуста, т.\,е.\
в сис\-те\-ме обслуживания нет ни одного требования, и величины $J_{10}(t)= \Pr\left(X(t) \le 10\right)$.


При вычислениях используется следствие~4 из работы~\cite{z09}, которое с учетом конкретной модели
и зависимости интенсивностей катастроф от состояния выглядит следующим образом.


\medskip

\noindent
\textbf{Следствие 2.}\
Пусть выполнены условия теоремы~1, $X_n(t)$~--- соответствующий усеченный процесс,   а $X(0) = X_n (0) =0.$
Тогда при всех  $t \ge 0$ и любом $n$ справедливы оценки

\noindent
\begin{multline*}
\|{\bf \pi}(t) - {\bf p}_n (t)\|_{1} \le{}\\
 {}\le \limsup_{t \to \infty} \|{\bf \pi} (t)\|_{1D} \left(
4\fr{\left(1+\varepsilon\right)^k}{\varepsilon} e^{-\int\limits_0^t
\left(\zeta\xi(\tau) - \varepsilon \lambda(\tau)\right)\, d\tau} +{}\right.\\[-6pt]
\left.{}\vphantom{e^{-\int\limits_0^t
(\zeta\xi(\tau))}} + 6Lw_n^1 t\right)
%\label{add2202}
\end{multline*}
\vspace*{-6pt}

\noindent
и
\vspace*{-6pt}

\noindent
\begin{multline*}
\left|\phi(t) - E_{0,n}(t)\right| \le{}\\
\!{}\le \limsup_{t \to \infty} \|{\bf \pi} (t)\|_{1D}
\left(\fr{4\left(1+\varepsilon\right)^k}{\omega \varepsilon } e^{-\int\limits_0^t
\left(\zeta\xi(\tau) - \varepsilon \lambda(\tau)\right)\, d\tau} +{}\right.\\[-6pt]
\left.{}\vphantom{e^{-\int\limits_0^t
(\zeta\xi(\tau))}}+ 6Lw_n^2 t\right)\,,
%\label{add2203}
\end{multline*}
\vspace*{-6pt}

\noindent
где
\vspace*{-6pt}

\noindent
\begin{align*}
L &= \sup\limits_{[0,1]} \left(\lambda(t)+S\mu(t)+M\xi(t)\right)\,; \\
w_n^1 &=\sup\limits_{k\ge n} \fr{1}{d_k}\,;\\
 w_n^2 &= \sup\limits_{k\ge n} \fr{k}{d_k}\,;\\
E_{0,n} &=E_k(t) = E\left\{X_n(t)\left|X_n(0)=0\right.\right\}\,,
\end{align*}
а  ${\bf \pi} (t)$~--- существующий 1-пе\-ри\-о\-ди\-че\-ский предельный режим.

\medskip

\noindent
\textbf{Замечание 4.}
Аналогичные оценки (с изменением первого слагаемого в скобках) получаются при выполнении условий теорем~2 и~3.

\bigskip

\noindent
\begin{enumerate}[1.]
\item Выбираем вначале интенсивности следующим образом:
$\lambda(t) = 240+\cos 2\pi t $, $\mu(t) = 1 + \sin 4\pi t $, $\xi_n(t)=100 + \sin 4 \pi t$, $n \ge 1$.
Тогда при $\varepsilon =0{,}4$ выполнены условия теоремы~1 и для построения предельных 1-периодических
характеристик (с точностью до $10^{-5}$) достаточно выбрать размерность усеченного процесса $n=100$
и построить нужные характеристики для усеченного процесса с нулевым начальным условием на отрезке~$[8,9]$.
Соответствующие графики приведены на рис.~1.
\end{enumerate}

%\begin{figure}[b] %fig1
\vspace*{6pt}
\begin{center}
\vspace*{1pt}
\mbox{%
\epsfxsize=80.672mm
\hspace*{-.7pt}\epsfbox{zei-1t.eps}
}
\end{center}
%\vspace*{-16pt}
{{\figurename~1}\ \ \small{Случай~1:
(\textit{а})~приближенное значение предельного среднего;
(\textit{б})~приближенное значение предельной величины $J_{0}(t)= \Pr\left(X(t)  = 0\right)$;
(\textit{в})~приближенное значение предельной величины $J_{10}(t)= \Pr\left(X(t) \le 10\right)$
%\label{f1zat}
}}
%\end{figure}

\end{multicols}

\addtocounter{figure}{1}

\begin{figure}
\begin{center} %fig2+3
%\vspace*{12pt}
\mbox{%
\epsfxsize=164.219mm
\epsfbox{zei-2-3.eps}
}
\begin{minipage}[t]{80mm}
\vspace*{-12pt}
\Caption{Случай~2:
(\textit{а})~приближенное значение предельного среднего;
(\textit{б})~приближенное значение предельной величины $J_{0}(t)= \Pr\left(X(t)  = 0\right)$;
(\textit{в})~приближенное значение предельной величины $J_{10}(t)= \Pr\left(X(t)  \leq 0\right)$
\label{f2zat}}
\end{minipage}
\hfill
\begin{minipage}[t]{80mm}
\vspace*{-12pt}
\Caption{Случай~3:
(\textit{а})~приближенное значение предельного среднего;
(\textit{б})~приближенное значение предельной величины $J_{0}(t)= \Pr\left(X(t)  = 0\right)$;
(\textit{в})~приближенное значение предельной величины $J_{10}(t)= \Pr\left(X(t)  \leq 0\right)$
\label{f3zat}}
\end{minipage}
\end{center}
\vspace*{-12pt}
\end{figure}

\begin{multicols}{2}

\noindent
\begin{enumerate}[1.]
\addtocounter{enumi}{1}
\item
Выбирая теперь
%\vspace*{-5pt}
%\noindent
\begin{align*}
 \lambda(t) &= 240+\cos 2\pi t\, ;\\
 \mu(t) &= 10 + \sin 4\pi t \,;\\
 \xi_n(t)&=\fr{2 + \sin 4 \pi t}{n}\,,
 \end{align*}
получаем выполнение условий теоремы~2 при $\varepsilon =3{,}0$, при этом $n=65$, а
в качестве необходимого отрезка будет $[1.5,2.5]$ (рис.~\ref{f2zat}).
\item
Если, наконец, взять
\begin{align*}
\lambda(t)& = 240+\cos 2\pi t \,;\\
 \mu(t) &= 1 + \sin 4\pi t \,;\\
\xi_n(t) &=
\begin{cases}
0, & \mbox { если } \quad n \neq 3k\,, \\
153 + \sin 4 \pi t\,, & \mbox { если } \quad n = 3 k\,,\\
\end{cases}
\end{align*}
то выполнены условия теоремы~3, при этом  $\varepsilon =0{,}15$, $n=250$,
$t \in [19,20]$ (рис.~3).
\end{enumerate}


{\small\frenchspacing
{%\baselineskip=10.8pt
\addcontentsline{toc}{section}{Литература}
\begin{thebibliography}{9}

\bibitem{KK} %1
\Au{Krishna Kumar~B.,  Arivudainambi~D.}
Transient solution of an $M/M/1$ queue with catastrophes~//  Comput. Math. Appl., 2000. Vol.~40. P.~1233--1240.

\bibitem{Di}  %2
\Au{Di Crescenzo~A., Giorno~V., Nobile~A.\,G., Ricciardi~L.\,M.}
On the $M/M/1$ queue with catastrophes and its continuous approximation~//
 Queueing Syst., 2003. Vol.~43. P.~329--347.

\bibitem{DZ}  %3
\Au{Van Doorn E.\,A., Zeifman~A.}
Extinction probability in a birth--death process with killing~// J.~Appl. Probab., 2005. Vol.~42. P.~185--198.

\bibitem{Di08}  %4
\Au{Di Crescenzo~A., Giorno~V., Nobile~A.\,G., Ricciardi~L.\,M.}
A~note on birth--death processes with catastrophes~//
Statist. Probab. Lett., 2008. Vol.~78. P.~2248--2257.

\bibitem{z08}
\Au{Zeifman~A.,  Satin~Ya., Chegodaev~A., Bening~V., Shorgin~V.}
Some bounds for $M(t)/M(t)/S$ queue with catastrophes~//
SMCTools08 Proceedings.  Athens, Greece, 2008.

\bibitem{z09}
\Au{Зейфман А.\,И., Сатин~Я.\,А., Чегодаев~А.\,В.}
О нестационарных системах обслуживания с катастрофами~// Информатика и её применения, 2009. Т.~3. Вып.~1. С.~47--54.

\bibitem{gm}
\Au{Гнеденко Б.\,В., Макаров~И.\,П.}
Свойства решений задачи с потерями в случае периодических
интенсивностей~// Дифф. уравнения, 1971. Т.~7. С.~1696--1698.

\bibitem{z06}
\Au{Zeifman~A., Leorato~S., Orsingher~E., Satin~Ya., Shilova~G.}
Some universal limits for nonhomogeneous birth and death processes~//
Queueing Syst., 2006. Vol.~52. P.~139--151.

\label{end\stat}

\bibitem{z08b}
\Au{Зейфман~А.\,И., Бенинг~В.\,Е., Соколов~И.\,А.}
Марковские цепи и модели с непрерывным временем.~--- М.: ЭЛЕКС-КМ, 2008.
\end{thebibliography}
}
}
\end{multicols}