
%\newcommand{\p}{{\sf P}}
%\newcommand{\e}{{\sf E}\,}


%\newcommand{\eqd}{\stackrel{d}{=}}
\newcommand{\rR}{\mathbb{R}}

\def\stat{korolev}


\def\tit{О РАСПРЕДЕЛЕНИИ РАЗМЕРОВ ЧАСТИЦ ПРИ ДРОБЛЕНИИ$^*$}

\def\titkol{О распределении размеров частиц при дроблении}

\def\autkol{В.\,Ю.~Королёв}
\def\aut{В.\,Ю.~Королёв$^1$}

\titel{\tit}{\aut}{\autkol}{\titkol}

{\renewcommand{\thefootnote}{\fnsymbol{footnote}}\footnotetext[1]
{Работа выполнена при поддержке РФФИ, гранты
08-01-00345, 08-01-00363, 08-07-00152, 09-07-12032-офи-м.}}

\renewcommand{\thefootnote}{\arabic{footnote}}
\footnotetext[1]{Факультет
вычислительной математики и кибернетики Московского государственного
университета им. М.~В.~Ломоносова; Институт проблем информатики
РАН, vkorolev@comtv.ru}


\Abst{Предложена новая модель для
распределения размера дробящейся частицы, учи\-ты\-ва\-ющая
непостоянство или случайный характер интенсивности потока
соударений. В рамках этой модели сформулирован критерий
логнормальности указанного распределения и описан класс возможных
распределений размера частиц при дроблении. Наряду со многими
известными моделями этот класс содержит масштабные смеси
логнормальных законов.}

\KW{логнормальное распределение; смеси нормальных законов; обобщенный процесс Кокса}

      \vskip 18pt plus 9pt minus 6pt

      \thispagestyle{headings}

      \begin{multicols}{2}

      \label{st\stat}

\section{Введение}

В данной статье рассматриваются математические модели, описывающие
распределение физических размеров частиц при дроблении. Подобные
модели могут довольно успешно использоваться (и используются) при
описании самых разных объектов от величины капиталов фирм или
размера доходов до объемов сообщений в вычислительных или
телекоммуникационных системах.

В опубликованной в 1940~г.\ работе Н.\,К.~Разумовского указано много
случаев, в которых логарифмы размеров частиц (золотин в
золотоносных россыпях, частиц горных пород при их дроблении и~т.\,п.)
имеют примерно нормальное (гауссовское)\linebreak
распределение~\cite{1kk}. На
эту работу обратил внимание А.\,Н.~Колмогоров, который предложил
матема\-тическую модель процесса дробления частиц,\linebreak
аналитически
объясняющую возникновение лог\-нормального распределения размеров
частиц при\linebreak
 дроблении, а также содержания минералов в отдельных
пробах~\cite{2kk}. Результат Колмогорова справедлив при довольно сильных
предположениях.
%
В~част\-ности, в рамках модели Колмогорова для
лог\-нормальности распределения час\-тиц при дроб\-ле\-нии необходимо,
чтобы скорость дроб\-ле\-ния (уменьшения размеров час\-тиц) была
постоянной, т.\,е.\ не зависела от размеров дробящихся час\-тиц.
В~то же время практически очевидно, что c уменьшением размера
час\-ти\-цы интенсивность ее соударений с другими час\-ти\-ца\-ми или
деталями дробильного агрегата может изменяться, например\linebreak
уменьшаться в силу того, что вероятность столкновения с другими
частицами, очевидно, в определенном смысле пропорциональна
размерам частицы, и чем частица меньше, тем меньше и эта
вероятность. На это обстоятельство обратил внимание сам
Колмогоров, который в конце своей статьи~\cite{2kk}, упоминавшейся выше,
написал: <<Было бы интересно изучить математические схемы, в
которых скорость дробления частиц уменьшается (или увеличивается)
с уменьшением их размеров. Естественно рассмотреть при этом в
первую очередь случаи, в которых скорость дроб\-ле\-ния
пропорциональна той или иной степени размеров частицы. Если эта
степень отлична от нуля, то, по-видимому, логариф\-ми\-чески
нормальный закон будет уже неприменим>>. В данной работе сделана
попытка исследования этого предположения А.\,Н.~Колмогорова.

Предложена довольно естественная модель процесса дробления,
учи\-ты\-ва\-ющая зависимость интенсивности процесса дробления от
параметров распределения размера частицы (в частности, от его
среднего значения). В рамках этой модели оказывается возможным
сформулировать критерий (т.\,е.\ необходимые и достаточные условия)
логнормальности распределения размера частиц при дроб\-лении.

В некоторых работах приводятся эмпирические свидетельства того,
что в некоторых случаях лог\-нор\-маль\-ная модель для распределения
размера час\-тиц при дроблении неадекватна. Так, в 1941~г.\ Р.\,А.~Багнольд
в своей книге~\cite{3kk} заметил, что логарифм функции плотности
распределения для логарифма размера частицы в естественных запасах
песка больше похож на гиперболу, чем на параболу. Это означает,
что, согласно наблюдениям Багнольда, распределение размера не
логнормально, а скорее имеет экспоненциально уменьшающиеся хвосты.

Последнее обстоятельство побудило некоторых исследователей
обратить внимание на модели типа лог-несимметричного распределения\linebreak
Лап\-ла\-са, так называемого двойного (двустороннего)
Парето-логнормального распределения, а также
лог-гауссовского/\!/обратного гауссовского распределения. В
частности, в работах~\cite{4kk, 5kk} рассмотрен следующий механизм
формирования распределения размера дробящихся частиц.

Базовая посылка этих моделей заключается в том, что частица при
перемещении из одного места в другое может разделиться на
несколько меньших частиц вследствие соударения или другого
воздействия, что обусловливает случайность масс частиц после
разделения. Обозначим число разбиений (дроблений) изначально одной
частицы к моменту времени~$t$ за~$N(t)$. Первоначальный размер
частицы обозначим~$s_{0}$. Пусть $D_{i}$~--- часть (доля) частицы,
отделившаяся при $i$-м соударении. Тогда размер частицы в момент
времени~$t$ имеет вид
\begin{equation}
Z(t)=s_{0}\prod_{i=1}^{N(t)}(1-D_{i})\,.
\label{e1kk}
\end{equation}
Следовательно,
$$
S(t) \equiv
\ln Z(t)=\mu + \sum_{i=1}^{N(t)} X_{i}\,,
$$
где $\mu=\ln(s_{0})$ и $X_{i}=\ln(1-D_{i})$. Считается, что
$X_{i}$, $i\ge1$,~--- независимые и одинаково распределенные
случайные величины с математическими ожиданиями~$a$, дисперсиями
$\sigma^2\in(0,\infty)$ и, соответственно, вторыми моментами
$b_{2}=a^2+\sigma^2$. Предполагается, что процесс~$N(t)$~---
пуассоновский с некоторым параметром $\lambda>0$ и стохастически
независим от последовательности~$X_{i}$. Это означает, что число
соударений, вследствие которых частица разбивается, на
временн{$\acute{\mbox{о}}$}м участке длиной~$s$ имеет распределение Пуассона с
математическим ожиданием~$\lambda s$, а ч{$\acute{\mbox{и}}$}сла соударений на
непересекающихся временн{$\acute{\mbox{ы}}$}х интервалах являются независимыми
случайными величинами. В данных предположениях вследствие
центральной предельной теоремы при достаточно большом~$\lambda$
распределение случайной величины
$$
U_{\lambda}(t)=
\fr{1}{\sqrt{\lambda}}\left(\sum\limits_{i=1}^{N(t)}X_{i}-\lambda t
a\right )
$$
является приблизительно нормальным с нулевым математическим
ожиданием и дисперсией~$tb_{2}$. В~результате
\begin{equation}
{\sf P}\left(S(t)<x\right)\approx\Phi\left(\fr{x-\mu-\lambda t
a}{\sqrt{\lambda t b_{2}}}\right)
\label{e2kk}
\end{equation}
для больших $\lambda$, где $\Phi(x)$~--- стандартная нормальная
функция распределения, $x\in\mathbb{R}$. Не все час\-ти\-цы\linebreak
одновременно добираются до места назначения. Некоторые могут
поменять направление и застрять надолго, тогда как другие могут,
не задев других, \mbox{пройти} весь путь намного быстрее. Очевидно,
состояние частицы зависит от того, на каком расстоянии от
конечного пункта она находится. Это учитывается в моделях
Рида--Йоргенсена и Соренсена за счет случайности времени, в
которое наблюдается частица. По сути, Рид, Йоргенсен и Соренсен
ввели рандомизацию в модель, предложенную Колмогоровым в 1940~г.,
добавив в нее учет того обстоятельства, что частицы затрачивают
разное время на преодоление пути от одного источника до другого.

В модели Рида--Йоргенсена время считается показательно
распределенной случайной величиной, т.\,е.\ итоговое безусловное
распределение лога\-риф\-ма размера частицы представляет собой смесь\linebreak
рас\-пределения~(\ref{e2kk}) по параметру~$t$, имеющему показательное
распределение. Эта смесь как раз и является двусторонним
распределением Парето\linebreak
 (в~работе~\cite{4kk} к величине~$S(t)$
дополнительно прибавляется независимая от нее нормально
распределенная случайная величина, характеризующая <<случайный
выбор>> начального состояния частицы, так что итоговое
распределение оказывается так называемым двусторонним
Парето-логнормальным законом).

В модели Соренсена движение частицы считается броуновским со
сносом~$\nu$ и коэффициентом\linebreak диффузии~$\omega^{2}$. Это означает,
что дистанция, пройден\-ная за временной промежуток длины~$s$,
нормально распределена с математическим ожиданием~$\nu s$ и
дисперсией~$\omega^{2}s$, а расстояния, пройден\-ные за
непересекающиеся интервалы времени, являются независимыми
случайными величинами. Предполагается, что броуновское движение
частиц независимо от процессов дробления (разделения), т.\,е.\ от
пуассоновского процесса~$N(t)$ и случайных величин~$b_{i}$. Модель
одномерна, т.\,е.\ рассматривается только проекция частицы на
прямую. Если будет много движений в других направлениях, это
вызовет б{$\acute{\mbox{о}}$}льший износ частицы и может быть учтено за счет
увеличения интенсивности~$\lambda$ пуассоновского процесса
разделения частиц. Обозначим через~$\tau$ (случайное) время
прибытия определенной частицы к месту назначения. Для
рас\-смат\-ри\-ва\-емо\-го процесса броуновского движения, как известно,
распределение случайной величины~$\tau$ является обратным
гауссовским $IG(a/\omega,\nu/\omega)$, где $a$ означает расстояние
между точкой отправления и конечной точкой. Плотность обратного
гауссовского распределения~$IG(\delta,\gamma)$ равна
$$
\fr{\delta}{\sqrt{2\pi
x^{3}}}\exp\left\{-\fr{\gamma^{2}}{2x}\left(x-\fr{\delta}{\gamma^{2}}\right)\right\}\,,\
x>0\,.
$$
Логарифм размера частицы, прибывшей в пункт назначения в момент
$\tau=t$, очевидно, равен~$S(t)$. Следовательно, условное
распределение логарифма размера частицы, прибывшей в момент
$\tau=t$, приблизительно нормально:
$$
{\sf P}\big(S(t)<x\big)\approx\Phi\left(\fr{x-\mu-\beta\lambda
b_{2}t}{\sqrt{\lambda b_{2}t}}\right)\,,
$$
где
$$
\beta = \fr{a}{b_{2}}\,.
$$
Распределение случайной величины $\zeta=\lambda b_{2}\tau$
является обратным гауссовским:
$$
\zeta \sim IG\left( \fr{a\sqrt{\lambda
b_{2}}}{\omega},\,\fr{\nu}{\omega\sqrt{\lambda b_{2}}}\right)\,.
$$
Безусловное же распределение логарифма размера частицы, т.\,е.\
безусловное распределение случайной величины $S(\tau)=\ln Z(\tau)$,
является приблизительно смесью приведенных выше нормальных
распределений по распределению параметра~$\zeta$. Эта смесь по
определению является гауссовским$/\!/$обратным гауссовским
распределением $GIG(\alpha,\beta,\delta,\mu)$, которому
соответствует характеристическая функция
\begin{multline*}
f_{\alpha,\beta,\delta,\mu}(s)=\exp\left\{\delta\sqrt{\alpha^{2}-\beta^{2}}
\left [\vphantom{\sqrt{1+\fr{s^{2}-is2\beta}{\alpha^{2}-\beta^{2}}}}
1-{}\vphantom{\sqrt{\fr{S^2}{\beta^2}}}\right.\right.\\
\left.{}-\left.\sqrt{1+\fr{s^{2}-is2\beta}{\alpha^{2}-\beta^{2}}}\,\right ]+is\mu\right\}\,.
\end{multline*}
В рассматриваемом случае параметры $\alpha$, $\beta$ и~$\delta$
соответственно равны
\begin{align*}
\alpha &= \sqrt{\fr{a^{2}}{b_{2}^{2}}+\fr{\nu^{2}}{\omega^{2}\lambda
b_{2}}}\,;\\
\beta &= \fr{a}{b_{2}}\,;\\
\delta &= \fr{a}{\omega}\sqrt{\lambda b_{2}}\,.
\end{align*}
Распределения, логарифм функции плотности которых представляет
собой гиперболу (гипер\-болические распределения), введены
Барндорфф-Ниль\-се\-ном в 1977~г.~\cite{6kk}. В ряде работ гиперболические
распределения дали хорошие результаты при их подгонке к
эмпирическим распределениям реальных данных (см., например,~[7--9]).
В работе~\cite{6kk} фактически введен более общий класс распределений
вероятностей~--- класс так называемых обобщенных гиперболических
распределений, которые во многом подобны гиперболическим
распределениям. Одним из представителей этого класса является
гауссовское$/\!/$обратное гауссовское распределение, которое
демонстрирует еще лучшее согласие с эмпирическими распределениями
размера час\-тиц, нежели гиперболические распределения. Это
обусловлено тем, что у $GIG$-распределений хвосты являются более
тяжелыми по сравнению с обычным гиперболическим распределением,
что характерно для естественных запасов песка. Другое преимущество
$GIG$-распределений в том, что они имеют много благопрятных
математических свойств. Например, их моменты могут быть вычислены
явно.

Функция плотности распределения $GIG(\alpha,\beta,\delta,\mu)$
равна
$$
\fr{\alpha\delta}{\pi}\,e^{\delta\gamma} \cdot
\fr{K_{1}(\alpha\sqrt{\delta^{2}+(x-\mu)^{2}})}{\sqrt{\delta^{2}+(x-\mu)^{2}}}\cdot
e^{\beta(x-\mu)}\,,\  x\in\mathbb{R}\,.
$$
Здесь $K_{1}$~--- модифицированная функция Бесселя третьего
порядка. Возможные значения па\-ра\-мет\-ров: $\alpha>0$, $\delta>0$,
$\beta<|\alpha|$, $\mu\in\mathbb{R}$. Если случайная величина $X$
имеет $GIG(\alpha,\beta,\delta,\mu)$-распределение, то ее
математическое ожидание и дисперсия соответственно равны:
$$
{\sf E}X = \mu + \fr{\delta\beta}{\gamma}\,,\quad {\sf
D}X=\fr{\delta\alpha^{2}}{\gamma^{3}}\,,
$$
где $\gamma = \sqrt{\alpha^{2}-\beta^{2}}$. Коэффициент асимметрии
равен $3 \beta /(\alpha\sqrt{\delta\gamma})$, а коэффициент
эксцесса равен $3(1+4 \beta^{2} / \alpha^{2})/(\delta\gamma)$.

Модели Рида--Йоргенсена и Барндорфф-Ниль\-се\-на--Соренсена
демонстрируют хорошее согласие с экспериментальными данными,
отражающими размеры частиц в природных залежах или размеры
капиталов фирм. Однако в ее конструкции, тем не менее, не
учитывается отмеченное Колмогоровым обстоятельство: интенсивность
процесса дробления остается постоянной.

В следующем разделе данной работы будет предложена модель, в
которой интенсивность процесса дробления может быть переменной и
даже случайной.

\section{Модель процесса дробления частиц, основанная на обобщенных процессах Кокса}

\subsection{Базовая модель}

Базой для рассматриваемых конструкций служит структурная модель~\cite{1kk}.
Основное отличие предлагаемой модели от моделей,
рассмотренных выше, в том, что интенсивность соударений будет
считаться случайной.

Пусть в базовом представлении~(\ref{e2kk}) $X_1,X_2,\ldots$~--- одинаково
распределенные случайные величины, $s_0=1$ (так что $\mu=0$), а
$N(t)$~--- дважды стохастический пуассоновский процесс (процесс
Кокса), управляемый процессом $\Lambda(t)$, т.\,е.\
$$
N(t)=N_1\left(\Lambda(t)\right)\,,\quad t\ge 0\,,
$$
где $\Lambda(t)$~--- случайный процесс с неубывающими, непрерывными
справа траекториями, выходящими из нуля, $N_1(t)$~--- стандартный
пуассоновский процесс (однородный пуассоновский процесс с
единичной интенсивностью), причем процессы~$N_1(t)$ и~$\Lambda(t)$
независимы. Такая модель процесса, под\-счи\-ты\-ва\-юще\-го число
соударений (дроблений)\linebreak час\-ти\-цы к моменту~$t$, учитывает возможную
изменчивость и даже случайность интенсивности соударений
дробящейся частицы, которая на практике, например при дроблении
частиц грунта речного русла, может быть вызвана периодической
сменой времен года, погодой и другими причинами. Во многих
приложениях можно считать, что процесс~$\Lambda(t)$ имеет вид
$$
\Lambda(t)=\int\limits_{0}^{t}\lambda(\tau)\,d\tau\,,\quad t\ge0\,,
$$
где $\lambda(t)$~--- положительный случайный процесс с
интегрируемыми траекториями. При этом~$\lambda(t)$ можно
интерпретировать как мгновенную стохастическую интенсивность
процесса~$N(t)$. Поэтому иногда процесс~$\Lambda(t)$, управляющий
процессом Кокса~$N(t)$, называют {\it накопленной интенсивностью}
процесса~$N(t)$.

Предположим, что при каждом $t\geq0$ случайные величины
$N(t),X_1,X_2,\ldots$ независимы. Процесс
\begin{equation}
S(t)=\sum_{j=1}^{N(t)}X_j\,,\quad t\geq0\,,
\label{e3kk}
\end{equation}
назовем обобщенным процессом Кокса (при этом для определенности
будем считать, что $\sum\limits_{j=1}^{0} =0$). Процессы вида~(\ref{e3kk}) играют
чрезвычайно важную роль во многих прикладных задачах, будучи
естественными моделями неоднородных хаоти\-ческих потоков (см.,
например,~\cite{10kk, 11kk}). Достаточно сказать, что при $\Lambda(t)
\equiv \ell t$ с $\ell > 0$ процесс~$S(t)$ превращается в
классический обобщенный пуассоновский процесс, широко используемый
при моделировании многих явлений в физике, теории надежности,
финансовой и актуарной деятельности, биологии и~т.\,д., где
возникают потоки событий, абсолютно хао\-тич\-но, беспорядочно
распределенных во времени, или размещения точек, абсолютно
хао\-тич\-но, беспорядочно распределенных в пространстве. Большое
число разнообразных прикладных задач, приводящих к обобщенным
пуассоновским процессам, описано в книгах~\cite{10kk, 12kk}. Обобщенные
процессы Кокса играют важную роль при  моделировании характеристик
неоднородных хао\-ти\-ческих стохастических потоков случайных событий.
При этом особую важность при использовании обобщенных процессов
Кокса вида~(\ref{e3kk}), скажем в страховой математике при моделировании
неоднородных потоков страховых выплат или в теории управления
запасами при моделировании неоднородных потоков заявок на поставку
некоторого продукта, имеет тот случай, когда математическое
ожидание слагаемых~$X_j$ в сумме~(\ref{e3kk}) отлично от нуля (в частности,
в условиях приведенных примеров~--- положительно).
В~рассматриваемом в данной работе случае по смыслу задачи $0\le
D_i\le1$, так что $a={\sf E}X_1={\sf E}\ln(1-D_i)\le 0$. Для
опре\-де\-лен\-ности всюду далее считаем, что $a<0$. Обозначим ${\sf
D}X_1=\sigma^2$ и будем считать, что $0<\sigma^2<\infty$.

Символ $\Longrightarrow$ будет обозначать сходимость по
распределению, которая, как известно, в конечномерном случае
эквивалентна слабой сходимости.

Расстояние (метрика) Леви $\mathcal{L}(F_1,F_2)$ между функциями
распределения $F_1$ и $F_2$ определяется как
\begin{multline*}
\mathcal{L}(F_1,F_2)=\inf\{h>0:\ F_1(x-h)-h\le F_2(x)\le{}\\
{}\le
F_1(x+h)+h\ \mbox{\ \ \ для всех}\ x\in{\rR}\}\,.
\end{multline*}
Если $X_1$ и $X_2$~--- случайные величины с функциями распределения
$F_1$ и $F_2$ соответственно, то будем считать, что
$\mathcal{L}(X_1,X_2)=\mathcal{L}(F_1,F_2)$. Сходимость в метрике
Леви эквивалентна сходимости по распределению (см., например,~[13, гл.~2, разд.~9]).

\subsection{Предельные теоремы. Обоснование возможности аппроксимации
распределения логарифма размера частиц при дроблении сдвиговыми
смесями нормальных законов}

\noindent
\textbf{Теорема 1.} {\it Пусть $S(t)$~--- логарифм текущего размера
дробящейся частицы в момент времени $t$~--- пред\-став\-ля\-ет собой
обобщенный процесс Кокса, управляемый процессом накопленной
интенсивности~$\Lambda(t)$ и по\-рож\-ден\-ный последовательностью
$\{X_i\}_{i\ge1}$. Предположим, что ${\sf E}X_1=a<0$ и
$\Lambda(t)\longrightarrow\infty$ по вероятности при $t\to\infty$.
Пусть $D(t)>0$~--- некоторая функция такая, что $D(t)\to\infty$ при
$t\to\infty$. Тогда одномерные распределения неслучайно
центрированного и нормированного обобщенного процесса Кокса~$S(t)$ %\linebreak\vspace*{-12pt}
%\pagebreak
%
%\noindent
слабо сходятся при $t\to\infty$ к распределениям некоторой
случайной величины $Z$, т.\,е.
$$
\fr{S(t)-C(t)}{D(t)}\Longrightarrow Z\quad (t\to\infty)
$$
при некоторой вещественной функции~$C(t)$ тогда и только тогда,
когда
$$
\limsup_{t\to\infty}\fr{|C(t)|}{D^2(t)}\equiv k^2<\infty
$$
и существует случайная величина~$V$ такая, что
\begin{equation}
Z\eqd k\sqrt{\fr{a^2+\sigma^2}{|a|}}\cdot W+V\,,
\label{e4kk}
\end{equation}
где $W$~--- случайная величина со стандартным нормальным
распределением, независимая от случайной величины~$V$, и
\begin{equation*}
\mathcal{L}\left(\fr{a\Lambda(t)-C(t)}{D(t)},V(t)\right)\to 0\quad
 (t\to\infty)\,,
\end{equation*}
где распределение случайной величины~$V(t)$ определяется ее
характеристической функцией
\begin{multline}
{\sf E}\exp\{isV(t)\}=\exp\left\{-\fr{s^2(a^2+\sigma^2)}{2|a|}
\left[
\vphantom{\fr{|C(t)|}{D^2(t)}}
k^2-\vphantom{\fr{|C(t)|}{D^2(t)}}\right.\right.\\[6pt]
\left.\left.{}-\fr{|C(t)|}{D^2(t)}\right]\right\}{\sf E}\exp\{isV\}\,,\quad
s\in\rR\,.
\label{e5kk}
\end{multline}
}
\smallskip

\noindent
Д\,о\,к\,а\,з\,а\,т\,е\,л\,ь\,с\,т\,в\,о\ этого результата, основанное на общем утверждении о
сходимости распределений суперпозиций независимых случайных
процессов, доказанном в~\cite{18kk}, можно найти, например, в книге~\cite{10kk}
(см.\ теорему~9.2.1 там).

\smallskip

Из теоремы~1 вытекает следующий {\it критерий} логнормальности
распределения размера частицы при дроблении.

\smallskip

\noindent
\textbf{Следствие~1.} {\it В условиях теоремы~$1$ распределение
логарифма текущего размера дробящейся частицы~$S(t)$
асимптотически нормально с некоторой асимптотической дисперсией
$\delta^2>0$:
$$
{\sf P}\left(\fr{S(t)-C(t)}{D(t)}<x\right)\Longrightarrow
\Phi\left(\fr{x}{\delta}\right)\quad (t\to\infty)
$$
в том и только в том случае, когда
$$
\limsup_{t\to\infty}\fr{|C(t)|}{D^2(t)}\le\fr{|a|\delta^2}{a^2+\sigma^2}
$$
и
\begin{multline*}
\lim_{t\to\infty}\mathcal{L}\left(
\vphantom{\fr{\sqrt{|a|}D(t)x}{\sqrt{|a|\delta^2D^2(t)-(a^2+\sigma^2)
|C(t)|}}}
{\sf
P}\left(\fr{a\Lambda(t)-C(t)}{D(t)}<x\right)\,,\right.\\
\left.\Phi\left(\fr{\sqrt{|a|}D(t)x}{\sqrt{|a|\delta^2D^2(t)-(a^2+\sigma^2)
|C(t)|}}\right )\right)=0\,.
\end{multline*}
}
%\columnbreak

\noindent
Д\,о\,к\,а\,з\,а\,т\,е\,л\,ь\,с\,т\,в\,о\,.\ Это утверждение вытекает из теоремы~1 и теоремы
Леви--Крамера о разложимости нормального закона только на
нормальные компоненты, согласно которой любая случайная величина~$V$,
удовлетворяющая~(\ref{e4kk}), должна быть нормально распределенной с
нулевым средним и дисперсией $\delta^2-k^2(a^2+\sigma^2)/|a|$.
Отсюда вытекает, что любая случайная величина~$V(t)$,
удовлетворяющая~(\ref{e5kk}), неизбежно должна иметь нормальное
распределение с нулевым средним и дисперсией
$\delta^2-\left(a^2+\sigma^2\right)|C(t)|/\left( |a|D^2(t)\right)$.

\medskip

Другими словами, в условиях теоремы~1 распределение текущего
размера дробящейся частицы~$Z(t)$ асимптотически логнормально
тогда и\linebreak
только тогда, когда процесс накопленной интенсивности~$\Lambda(t)$
имеет асимптотически нормальные (в частности,
вырожденные) одномерные распределения.

Из теоремы~1 вытекает, что в рамках рассматриваемой модели
распределение логарифма дробящейся частицы разумно
аппроксимировать сдвиговыми смесями нормальных законов. При этом
вид смешивающего распределения полностью определяется
асимптотическим поведением накопленной интенсивности процесса
соударений. Другими словами, если $Z(t)=\exp\{S(t)\}$~--- размер
дробящейся частицы, то
$$
S(t)\approx X'+V'\,,
$$
где $X'$~--- случайная величина, имеющая нормальное распределение с
нулевым средним и некоторой дисперсией $\sigma_*^2$, а $V'$~---
независимая от~$X'$ случайная величина с, вообще говоря,
произвольным рас\-пределением, определяемым статистическими
за\-ко\-номерностями в поведении интенсивности столкновений. При этом
приближенное\linebreak
 равенство понимается как близость распределений левой
и правой частей. Иными словами,
$$
Z(t)\approx \xi \eta\,,
$$
где $\xi=e^{X'}$~--- случайная величина с логнормальным
распределением, a $\eta=e^{V'}$~--- независимая от~$\xi$
положительная случайная величина. Таким образом, в рамках
рассматриваемой модели распределение размера частиц при дроблении
является масштабной смесью логнормальных распределений.

Необходимо отметить, что в моделях Рида--Йор\-ген\-се\-на и Соренсена
чисто сдвиговые смеси нормальных законов не могут возникнуть в
качестве модели для распределения логарифма размера час\-тиц (и,
следовательно, чисто масштабные смеси
 логнормальных законов не
могут возникнуть в качестве модели для распределения самог{$\acute{\mbox{о}}$}
размера частиц).

На практике при определении распределения логарифма размера
дробящейся частицы разумно использовать дискретную аппроксимацию
для смешивающего закона, что позволит искать оценки параметров
смеси с помощью статистических процедур, традиционно используемых
для этих целей, например с использованием ЕМ-алгоритма или его
модификаций.

Если в теореме~1 задать конкретный вид функций~$C(t)$ и~$D(t)$, то
ситуация намного упростится. Предположим, что существуют такие
числа $\ell\in(0,\infty)$ и $s\in(0,\infty)$, что
$$
{\sf E}\Lambda(t)\equiv \ell t\,,\quad {\sf D}\Lambda(t)\equiv
s^2t\,,\quad t>0\,.
$$
Тогда, как несложно убедиться,
$$
{\sf E}S(t)=a\ell t\,,\quad {\sf
D}S(t)=\left[\ell(a^2+\sigma^2)+a^2s^2\right]t\,.
$$
Линейная зависимость ${\sf E}S(t)$ и~${\sf D}S(t)$ от~$t$
обуслов\-ле\-на тем, что в рамках указанного предположения о виде
моментов управляющего процесса согласно теореме~1 c $C(t)={\sf
E}S(t)$ и $D(t)=\sqrt{{\sf D}S(t)}$ только при совпадении
показателей степеней возможно существование собственного
невырожденного предельного распределения. С другой стороны, если
процесс~$\Lambda(t)$ имеет интегральное представление, то
равенство показателя степени~$t$ единице оказывается вполне
естественным.

Из теоремы~1 при этом непосредственно вытекает следующий
результат.

\medskip

\noindent
\textbf{Теорема 2.} {\it Одномерные распределения неслучайно
центрированных и нормированных обобщенных процессов Кокса слабо
сходятся к распределению некоторой случайной величины~$Z$ при
$t\to\infty$, т.\,е.
$$
\fr{S(t)-a\ell t}{\sqrt{t\left[\ell
(a^2+\sigma^2)+a^2s^2\right]}}\Longrightarrow Z\quad (t\to\infty)\,,
$$
тогда и только тогда, когда существует случайная величина~$V$
такая, что
$$
\fr{\Lambda(t)-\ell t}{s\sqrt{t}}\Longrightarrow V\quad
(t\to\infty)\,.
$$
При этом}
\begin{multline*}
{\sf P}(Z<x)={\sf
E}\Phi\left(x\sqrt{1+\fr{a^2s^2}{(a^2+\sigma^2)\ell}}-{}\right.\\
\left.{}-\fr{asV}{\sqrt{(\sigma^2+a^2)\ell}}
\vphantom{\sqrt{1+\fr{a^2s^2}{(a^2+\sigma^2)\ell}}}
\right)\,,\quad
x\in\mathbb{R}\,.
\end{multline*}

\medskip

Несложно видеть, что предельная случайная величина~$Z$ допускает
представление

\noindent
$$
Z\eqd \left[1+\fr{a^2s^2}{(a^2+\sigma^2)\ell}\right]^{-1/2}\!
X+\fr{as}{\sqrt{(a^2+\sigma^2)\ell+a^2s^2}}\cdot V,
$$
где $X$~--- случайная величина со стандартным нормальным
распределением, независимая от случайной величины~$V$.

Доказательство теоремы~2 совпадает с доказательством
соответствующего результата из книги~\cite{10kk} с точностью до
переобозначений.

Сравнивая предложенную модель с моделями Рида--Йоргенсена и
Барндорфф-Нильсена--Со\-рен\-се\-на, можно заметить, что теорема~1 (или\linebreak
теорема~2) играет ту же роль, какую в рассуждениях
Рида--Йоргенсена и Барндорфф-Нильсена--Соренсена играет
центральная предельная теорема для пуассоновских случайных сумм,
обосновывающая переход от~(\ref{e1kk}) к~(\ref{e2kk}). С использованием их логики,
приведенные выше рассуждения могут быть продолжены следующим
образом. При каждом фиксированном достаточно большом~$t$ из
теоремы~1 вытекает приближенное равенство
\begin{equation}
{\sf P}\big(S(t)<x\big)\approx{\sf
E}\Phi\left(\fr{x-C(t)}{D(t)k\sqrt{b_2}}-V\fr{\sqrt{|a|}}{k\sqrt{b_2}}\right)\,.
\label{e6kk}
\end{equation}
Теперь, например, привлекая соображения <<случайности времени
достижения>> из рассуждений Рида--Йоргенсена или Соренсена,
приведенных во введении, можно считать время~$t$, а вместе с ним и
функции~$C(t)$ и~$D(t)$ случайными, так что в итоге получается
совсем общая модель для распределения логарифма размера частицы
при дроблении, имеющая вид сдвиг-масштабной смеси нормальных
законов. В частности, обозначив
$$\zeta=C(t)\,,\quad \chi=V \fr{\sqrt{|a|}}{k\sqrt{b_2}}\,,
$$
получаем, что при $|C(t)|=k^2D^2(t)$
эта смесь принимает вид
$$
{\sf P}\left(S(t)<x\right)\approx{\sf
E}\Phi\left(\fr{x-\zeta}{\sqrt{b_2|\zeta|}}-\chi\right)\,.
$$
Теперь, наделяя величины~$\zeta$ и~$\chi$ теми или иными
распределениями, можно получить ту или иную конкретную модель для
распределения размера час\-ти\-цы при дроблении. В частности, пусть~$\zeta$
имеет показательное распределение. Тогда итоговая модель
распределения размера частицы имеет\linebreak
 вид масштабной смеси
лог-лапласовских распределений, т.\,е.\ получается обобщение
модели Рида--Йоргенсена (в рамках которой предполагается,
что~$\chi$ независима от $\zeta$ и имеет нормальное распределение).
Если же $\zeta$ имеет $IG$-распределение, то итоговая модель имеет
вид масштабной смеси лог-$GIG$-рас\-пре\-де\-ле\-ний, т.\,е.\ получается
обобщение модели Барн\-дорфф-Нильсена--Соренсена (в рамках которой
предполагается, что $\chi$ вы\-рож\-дена).
{\looseness=1

}

\subsection{Оценки точности асимптотических аппроксимаций распределения
логарифмов размеров частиц при~дроблении сдвиговыми смесями
нормальных законов}

Рассмотрим скорость сходимости в теореме~2, что позволит получить
оценку точности аппроксимации~(\ref{e6kk}) при специальном виде
центрирующих и нормирующих констант.

Прежде чем сформулировать основной результат данного раздела,
введем дополнительные обозначения:
\begin{align*}
\beta^3 &= \e|X_1|^3\,;\\
L_3 &= \fr{\beta^3}{(a^2+\sigma^2)^{3/2}}\,;\\
F_t(x) &= \p\left(\fr{S(t)-a\ell t}{\sqrt{t[\ell(a^2+\sigma^2)+a^2s^2]}}< x\right)\,;\\
\rho_t & =\sup_x \left|F_t(x) - {\sf E}\Phi\bigg(x\sqrt{1+\fr{a^2s^2}{(a^2+\sigma^2)\ell}}-{}\right.\\
&\left.\hspace*{100pt}-\fr{asV}{\sqrt{(\sigma^2+a^2)\ell}}\bigg)
\right|\,;\\
\Delta_t &= \sup_{v}\left|\,\p\left(\fr{\Lambda(t)-\ell
t}{s\sqrt{t}}<v\right)-\p \left(V <v\right)\right|\,.
\end{align*}

\smallskip

\noindent
\textbf{Теорема 3.} {\it Пусть $\beta^3<\infty$, $\e |V| < \infty$.
Тогда справедлива оценка
\begin{multline*}
\rho_t \leq \Delta_t
+\fr{1}{\sqrt{t}}\,\inf_{\epsilon\in(0,1)}\left\{\fr{C_0L_3}{\sqrt{(1-\epsilon)\ell}}+{}\right.\\
\left.{}+\fr{s}{\ell}\left(\fr{{\sf
E}|V|}{\epsilon}+Q(\epsilon)\right)\right\}\,,
\end{multline*}
где $C_0$~--- абсолютная постоянная в неравенстве Берри--Эссеена,}
$C_0\le 0{,}7005$,
$$
Q(\epsilon)=\max\left\{\fr{1}{\epsilon},\,\fr{\sqrt{1+\epsilon}}{\big(1+\sqrt{1-\epsilon}\big)\sqrt{2\pi
e(1-\epsilon)}}\right\}\,.
$$

\smallskip

\noindent
Д\,о\,к\,а\,з\,а\,т\,е\,л\,ь\,с\,т\,в\,о\ этой теоремы приведено в работе~\cite{14kk}.

\smallskip

Если дополнительно к условиям теоремы~3 предположить, что
семейство случайных величин
$$
\left\{\left|\fr{\Lambda(t)-\ell
t}{s\sqrt{t}}\right|\right\}_{t>0}
$$
равномерно интегрируемо, то с помощью неравенства Ляпунова можно
получить неравенство
\begin{multline}
{\sf E}|V|=\lim_{t\to\infty}{\sf E}\bigg|\fr{\Lambda(t)-\ell
t}{s\sqrt{t}}\bigg|\le{}\\
{}\le\lim_{n\to\infty}\sqrt{{\sf
D}\bigg(\fr{\Lambda(t)-\ell t}{s\sqrt{t}}\bigg)}=1\,.
\label{e7kk}
\end{multline}
При этом из теоремы~3 очевидным образом вытекает следующий
результат.

\smallskip

\noindent
\textbf{Следствие 2.} {\it Пусть в дополнение к условиям тео\-ре\-мы~$3$
при каждом $t>0$ выполнено~$(\ref{e7kk})$. Тогда справедлива оценка
$$
\rho_t \leq \Delta_t
+\fr{1}{\sqrt{t}}\,\inf_{\epsilon\in(0,1)}\left\{\fr{C_0L_3}{\sqrt{(1-\epsilon)\ell}}+
\fr{s}{\ell}\left(\fr{1}{\epsilon}+Q(\epsilon)\right)\right\},
$$
где} $C_0\le 0{,}7005$,
$$
Q(\epsilon)=\max\left\{\fr{1}{\epsilon},\,\fr{\sqrt{1+\epsilon}}{\left(1+\sqrt{1-\epsilon}\right)\sqrt{2\pi
e(1-\epsilon)}}\right\}\,.
$$

\medskip

\noindent
\textbf{Замечание 1.} Теорема~3 справедлива при {\it любых
фиксированных значениях параметров} $t$, $\ell$ и $s$. Поэтому ее
можно использовать для формализации возможной зависимости
интенсивности потока дроблений от текущего размера частицы.
Действительно, обозначим средний (ожидаемый) логарифм размера
частицы в момент времени~$t$ через~$m(t)$,
$$
m(t)={\sf E}S(t)=a\ell t\,.
$$
Тогда
$$
\ell=\ell(t)=\fr{m(t)}{at}\,.
$$
Соответственно, тогда и $s=s(t)$. Конкретный вид функций~$m(t)$ и~$s(t)$
зависит от конкретных физических, химических,
технологических и/или геологических условий. В таком случае, как
следует из теоремы~3, для возможности аппроксимации\linebreak\vspace*{-12pt}
\pagebreak


\noindent
 распределения
логарифма частицы при дроблении
 сдвиговыми смесями нормальных
законов достаточно, чтобы при $t\to\infty$:\\[-5pt]
\begin{enumerate}[1)]
\item $\Delta_t\longrightarrow 0$;\\[-6pt]
\item $|m(t)|\longrightarrow \infty$;\\[-6pt]
\item $t^{1/2}s(t)=o\big(|m(t)|\big)$.
\end{enumerate}

\medskip

\noindent
\textbf{Замечание 2.} Пусть, к примеру, процесс~$\Lambda(t)$ является
неоднородным пуассоновским с накопленной интенсивностью $\ell
t=s^2 t$. В таком случае распределение случайной величины~$N(t)$
при каждом $t>0$ является так называемым {\it
пуассон-пуас\-со\-нов\-ским} или ({\it инфекционным})
{\it распределением Неймана типа~А}. Класс таких распределений
введен Ю.~Нейманом в работе~\cite{15kk} в связи с некоторыми задачами из
области бактериологии и энтомологии. В этом случае величина~$\Delta_t$
также имеет порядок $O\big(t^{-1/2}\big)$, поскольку в
условиях примера справедлива оценка
$$
\Delta_t\le\fr{C_0}{\sqrt{\ell t}}=C_0\sqrt{\fr{a}{m(t)}}\,,
$$
причем случайная величина~$V$ имеет стандартное нормальное
распределение (см., например,~\cite{11kk}), а следовательно,
распределение случайной величины~$Z$ также является стандартным
нормальным. В~таком случае имеет место оценка
$$
\rho_t \leq
\sqrt{\fr{a}{m(t)}}\!\inf_{\epsilon\in(0,1)}\left\{C_0\left(1+
\fr{L_3}{\sqrt{1-\epsilon}}\right)+\fr{1}{\epsilon}+Q(\epsilon)\right\}.
$$
Тогда, как следует из сказанного в замечании~1, для
возможности аппроксимации распределения логарифма частицы при
дроблении сдвиговыми смесями нормальных законов достаточно, чтобы
$|m(t)|\longrightarrow \infty$ при $t\to\infty$.

\medskip

\noindent
\textbf{Замечание 3.} Другим примером управляющего процесса~$\Lambda(t)$,
при котором величина~$\Delta_t$ также имеет порядок
$O\big(t^{-1/2}\big)$, является случай, когда $\Lambda(t)=G(t)$,
где $G(t)$~--- гамма-процесс Леви, приращение которого на единичном
интервале имеет гамма-распределение с параметром масштаба
$\lambda=\ell/s^2$ и параметром формы $\alpha=\ell^2/s^2$. В таком
случае распределение случайной величины~$N(t)$ при
каждом $t>0$ является отрицательным биномиальным (см., например,~\cite{11kk}).
При этом случайная величина~$V$ имеет стандартное
нормальное распределение (см., например,~\cite{11kk}), а следовательно,
распределение случайной величины~$Z$ опять-таки является
стандартным нормальным.

\medskip

\noindent
\textbf{Замечание 4.} Если случайный процесс~$N(t)$ является
однородным пуассоновским (с ин\-тен\-сив\-ностью~$\ell$), то, очевидно,
$s=0$. При этом логично считать, что распределение случайной
величины~$V$ является вырожденным в нуле и $\Delta_t=0$. Поэтому в
таком случае оценка, приведенная в теореме~1, естественно
переходит в оценку скорости схо\-ди\-мости распределений пуассоновских
случайных сумм к нормальному закону, приведенную, например, в
работе~\cite{16kk} с константой~$C_0$, уточненной в работе~\cite{17kk}.

{\small\frenchspacing
{%\baselineskip=10.8pt
\addcontentsline{toc}{section}{Литература}
\begin{thebibliography}{99}

\bibitem{1kk}
\Au{Разумовский~Н.\,К.}
ДАН СССР, 1940. Т.~28. №\,8. С.~55--57. % [1] (Разумовский, 1940)

\bibitem{2kk}
\Au{Колмогоров А.\,Н.}
О логарифмически нормальном законе распределения размеров частиц при дроблении~//
ДАН СССР, 1941. Т.~31. С.~99--101. % [1] (Колмогоров, 1940)

\bibitem{3kk}
\Au{Bagnold~R.\,A.}
The physics of blown sand and desert dunes.~--- London: Methuen, 1941. % [3] (Bagnold, 1941)

\bibitem{4kk}
\Au{Reed~W.\,J., Jorgensen~M.}
The double Pareto-Lognormal distribution~--- a new
parametric model for size distribution~// Communications in
Statistics~--- Theory and Methods, 2004. Vol.~33. No.\,8. P.~1733--1753. % [4] (Reed and Jorgensen, 2003)

\bibitem{5kk}
\Au{S{\hspace*{-3pt}\protect\ptb{\o}\hspace*{2pt}}rensen M.}
On the size distribution of sand~// Dept. of Applied Mathematics
and Statistics, University of Copenhagen. Working paper. 2006. % [5] (S{\o}rensen, 2006)

\bibitem{6kk}
\Au{Barndorff-Nielsen~O.}
Exponentially decreasing distributions for the logarithm of particle size~//
Proc. Roy. Soc. L. A, 1977. Vol.~353. P.~401--419. % [6] (Barndorff-Nielsen, 1977)

\bibitem{7kk}
\Au{Vincent P.}
Differentiation of modern beach and coastal dune sands~--- a logistic regression approach
using the parameters of the hyperbolic function~// Sediment.\
Geology, 1986. Vol.~49. P.~167--176. %[7] (Vincent, 1986)

\bibitem{8kk}
\Au{McArthur D.\,S.}
Distinctions between grain-size distribution of accretion and encroachment deposits
in an inland dune~// Sediment. Geology, 1987. Vol.~54. P.~147--163. % [8](McArthur, 1987)

\bibitem{9kk}
\Au{Hartmann~D.}
Cross-shore selective sorting process and grain size distributional shape~//
Acta Mech. [Suppl.], 1991. Vol.~2. P.~49--63. % [9] (Hartmann, 1991)

\bibitem{10kk}
\Au{Bening V.\,E., Korolev~V.\,Yu.}
Generalized Poisson
models and their applications in insurance and finance.~--- Utrecht: VSP, 2002. %[10] (Bening and Korolev, 2002)

\bibitem{11kk}
\Au{Королёв В.\,Ю., Бенинг~В.\,Е., Шоргин~С.\,Я.}
Математические основы теории риска.~--- М.: Физматлит, 2007. %[11] (Королев, Бенинг и Шоргин, 2007)

\bibitem{12kk}
\Au{Gnedenko~B.\,V., Korolev~V.\,Yu.}
Random summation: Limit theorems and applications.~--- Boca Raton: CRC Press, 1996. %[12] (Gnedenko and Korolev, 1996)

\bibitem{13kk}
\Au{Гнеденко Б.\,В., Колмогоров~А.\,Н.}
Предельные распределения для сумм независимых случайных величин.~--- М.: ГИТТЛ, 1949. %[13] (Гнеденко и Колмогоров, 1949)


\bibitem{18kk} %14
\Au{Korolev~V.\,Yu.}
A general theorem on the limit behavior of superpositions of independent
random processes with applications to Cox processes~// J.\
Math. Sci., 1996. Vol.~81. No.~5. P.~2951--2956.

\bibitem{14kk}
\Au{Артюхов С.\,В., Королёв~В.\,Ю.}
Оценки скорости сходимости распределений обобщенных дважды стохастических пуассоновских
процессов с ненулевым средним к сдвиговым смесям нормальных
законов~// Обозрение промышленной и прикладной математики, 2008.
Т.~15. Вып.~6. С.~988--998. %[14] (Артюхов и Королев, 2008)

\bibitem{15kk} %16
\Au{Neyman~J.}
On a new class of ``contagious'' distributions, applicable in
entomology and bacteriology~// Ann. Math. Statist., 1939. Vol.~10. P.~35--57. %[15] (Neyman, 1939)

\bibitem{16kk} %17
\Au{Шевцова~И.\,Г.}
Уточнение структуры оценок скорости сходимости в центральной предельной теореме для сумм независимых
случайных величин. Дисс. на соискание ученой степени
кандидата физ.-мат. наук.~--- М.: Московский государственный
университет, 2006. %[16] (Шевцова, 2006)

\label{end\stat}

\bibitem{17kk} %18
\Au{Шевцова И.\,Г.}
Об абсолютной постоянной в неравенстве Берри--Эссеена~// Сборник статей молодых ученых факультета ВМиК
МГУ. Вып.~5.~--- М.: Изд-во факультета ВМиК МГУ, 2008. С.~101--110. %[17] (Шевцова, 2008)


 \end{thebibliography}
}
}
\end{multicols}