\def\stat{borodak}


\def\tit{ВЕРОЯТНОСТНАЯ МОДЕЛЬ ОБСЛУЖИВАНИЯ ТРАФИКА 
В~СИСТЕМЕ СЕТЕЦЕНТРИЧЕСКОГО ТИПА}

\def\titkol{Вероятностная модель обслуживания трафика 
в~системе сетецентрического типа}

\def\autkol{В.\,Ю.~Бородакий}
\def\aut{В.\,Ю.~Бородакий$^1$}

\titel{\tit}{\aut}{\autkol}{\titkol}

%{\renewcommand{\thefootnote}{\fnsymbol{footnote}}\footnotetext[1]
%{Работа выполнена при финансовой поддержке РФФИ,
%проекты 08-01-00567 и 08-07-00152.}}

\renewcommand{\thefootnote}{\arabic{footnote}}
\footnotetext[1]{Национальный исследовательский ядерный университет <<МИФИ>>, vladbor@inbox.ru}
  
  
  \Abst{В терминах теории телетрафика построена модель сетецентрической системы, в 
которой из центров обработки данных по запросам абонентов передается информация в виде 
блоков данных различной длины. Передача блоков описывается в терминах 
<<эластичного>> трафика, их длина распределена по экспоненциальному закону, а 
обслуживание осуществляется по дисциплине разделения процессора. Предполагается, что 
поток запросов абонентов является пуассоновским, а блок данных, независимо от его длины, 
характеризуется минимальным требованием к ширине полосы пропускания. Проведен 
анализ модели отдельного звена сети сетецентрической системы (СС), получено аналитическое 
выражение для вычисления вероятности блокировки запроса из-за отсутствия достаточной 
для передачи блока данных ширины полосы пропускания.}
  
  \KW{сетецентрическая система; эластичный трафик; вероятность блокировки; звено сети}

   \vskip 30pt plus 9pt minus 6pt

      \thispagestyle{headings}

      \begin{multicols}{2}

      \label{st\stat}
        
  \section{Введение}
  
  В настоящее время активно развиваются распределенные 
автоматизированные системы управ\-ле\-ния специального назначения, 
построенные на базе сетецентрической концепции~\cite{1bor}. От выбора 
структуры подобных систем и поддержки заданных показателей качества 
существенно зависит эффективность решения возложенных на них задач с 
точки зрения соответствия заданным требованиям по режиму работы и 
производительности. Известны системы~\cite{2bor}, где невыполнение этих 
требований не только снижает производительность, но и приводит к 
деградации системы вплоть до ее уничтожения. В~состав СС 
входят: абонентские узлы, обменивающиеся управляющими 
сигналами с объектами управления и обладающие потребностью 
информационного обмена с управляющими компонентами системы; центры 
обработки данных (ЦОД), при\-нимающие информацию от абонентских узлов и\linebreak 
по\-сылающие ответное воздействие; телекомму\-ни\-кационное обору\-до\-ва\-ние, 
обеспечивающее передачу информации между абонентскими узлами\linebreak и~ЦОД.
  
  Проектирование СС, как правило, осуществляется в автономном (off-line) 
режиме. В этом случае исходными данными для проектирования являются: 
расположение и конфигурация абонентских узлов, структура и топология сети 
передачи данных, пропускные способности звеньев сети, объемы 
информационных потребностей абонентских узлов. Одним из важнейших 
показателей качества функционирования СС является величина обслуженной 
нагрузки (объем переданной информации), поскольку в условиях ограниченной 
пропускной способности звеньев сети часть запросов с некоторой 
вероятностью может получить отказ в обслуживании из-за отсутствия 
достаточных ресурсов на маршрутах между абонентскими узлами и ЦОД.
  
  Для дальнейшего изложения понадобятся следующие обозначения:
\begin{description}
\item $\mathcal{V}_s$~--- множество абонентских узлов ($s$-або\-нен\-ты);
\item $\mathcal{V}_T$~--- множество ЦОД ($t$-цент\-ры);
\item $\mathcal{N}$~--- множество предоставляемых ЦОД типов данных ($\mathcal{N}:=\{1,\ldots ,N\}$);
\item $\theta_n$~--- длина блока данных $n$-ти\-па ($n$-бло\-ка);
\item $b$~--- требование к минимальному значению ширины полосы 
пропускания для передачи блока данных любого типа;
\item $\lambda_n (s,t)$~--- интенсивность потока запросов $s$-або\-нен\-та в 
$t$-центр на передачу $n$-бло\-ков;
\item $a_n(s,t):= \lambda_n(s,t)\theta_n$~--- нагрузка, создаваемая $n$-блоками 
между $s$-або\-нен\-том и $t$-цент\-ром (предложенная нагрузка);
\item $\mathcal{L} (s,t)$~--- множество маршрутов между $s$-або\-нен\-том и 
$t$-цент\-ром;
\item $l_i(s,t)$~--- $i$-маршрут между $s$-або\-нен\-том и $t$-цент\-ром;
\item ${\mathcal E}_l (s,t)$~--- множество звеньев $l_i(s,t)$-марш\-ру\-та.
\end{description}
\pagebreak

\end{multicols}

\begin{figure} %fig1
\vspace*{1pt}
\begin{center}
\mbox{%
\epsfxsize=154.841mm
\epsfbox{bor-1.eps}
}
\end{center}
\vspace*{-9pt}
\Caption{Схема модели сетецентрической системы
\label{f1bor}}
\end{figure}

\begin{multicols}{2}

Пример схемы модели CC показан на рис.~\ref{f1bor}, где 
проиллюстрированы введенные обозначения.

 
  Величина обслуженной нагрузки в СС определяется формулой
  $$
  \tilde{a} := \sum\limits_{s\in {\mathcal V}_s }\sum\limits_{t\in {\mathcal V}_T} 
\sum\limits_{i:l_i(s,t)\in {\mathcal L}(s,t)} \sum\limits_{n\in {\mathcal N}} 
a_{in}(s,t) (1- B_{in}(s,t))\,,
  $$
  где $a_{in}(s,t)$~--- предложенная $n$-блоками нагруз-\linebreak ка и $B_{in}(s,t)$~--- 
вероятность блокировки $n$-блока на $l_i(s,t)$-маршруте. Если считать 
известными ве\-личины предложенных нагрузок и множества маршрутов~${\mathcal 
L}(s,t)$, то задача оценки величины\linebreak обслуженной нагрузки $\tilde{a}$ сводится 
к вычислению вероятностей~$B_{in}(s,t)$. Эти вероятности в предположении, 
что блокировки на всех звеньях~$l_i(s,t)$-маршрута происходят независимо в 
совокупности, можно представить в виде

\noindent
\begin{multline*}
B_{in} (s,t)\approx 1- \prod_{e\in {\mathcal E}_i(s,t)} \left ( 1-B_{in}^e (s,t)\right)\,,\\
n\in {\mathcal N},\ i\mbox{:\ }l_i(s,t)\in {\mathcal L}(s,t),\ t\in{\mathcal V}_T,\ s\in {\mathcal V}_s\,,
\end{multline*}
где $B_{in}^e(s,t)$~--- вероятность блокировки запроса на передачу $n$-блока 
на $e$-звене  $l_i(s,t)$-маршрута.
  
  Предположение о независимости блокировок лежит в основе известного 
метода приближенного расчета вероятностей блокировок~--- метода 
просеянной нагрузки (reduced load approximation), изначально разработанного 
для сетей с коммутацией каналов~\cite{3bor} и развитого в~\cite{4bor} для 
сетей с многоадресными соединениями. Данный подход предполагает наличие 
точного метода для расчета вероятностей блокировок~$B_{in}^e(s,t)$ на 
отдельном звене сети. Применение метода просеянной нагрузки к анализу 
вероятностных характеристик СС исследовано в~\cite{5bor}, а в данной статье 
построена модель отдельного звена сети и проведен анализ ее вероятностных 
характеристик.
  
  \section{Модель звена сети сетецентрической системы}
  
  Предположим, что в модели сети СС (например, показанной на 
рис.~\ref{f1bor}), все звенья, кроме некоторого звена~$e^*$, имеют 
неограниченные ресурсы для обслуживания запросов абонентов, т.\,е.\ 
$C^e=\infty$ для $e\not= e^*$. Задача анализа блокировок в такой системе 
сводится к анализу сети, состоящей из одного звена~$e^*$, с одним 
ЦОД~$t^*$, который передает данные из множества ${\mathcal N} =\{1,\ldots ,N\}$. 
Для удобства записи далее индексы~$e^*$ и~$t^*$ опускаются.

\medskip

\begin{center} %fig2
\vspace*{12pt}
\mbox{%
\epsfxsize=74.604mm 
\epsfbox{bor-2.eps}
}
%\end{center}
\vspace*{4pt}
{{\figurename~2}\ \ \small{Модель звена сети СС}}
\end{center}
\vspace*{-6pt}


\bigskip
\addtocounter{figure}{1}

  
  Итак, рассмотрим модель звена сети ем\-костью $C$ условных единиц, 
схематично изображенную на рис.~2. По звену передаются блоки 
данных $N$~типов, запросы на передачу $n$-блока образуют пуассо\-новский 
поток интенсивности~$\lambda_n$, размер блока является случайной 
величиной, распределенной экспоненциально со средним~$\theta_n$, а 
величина предложенной $n$-блоками нагрузки вычисляется по формуле $a_n := 
\lambda_n\theta_n$. Блоки любого типа характеризуются минимальным 
требованием~$b$ к емкости звена, а их обслуживание производится по 
дисциплине разделения процессора (PS~--- Processor Sharing). С 
вероятностью~$B_n$ $n$-блок может получить отказ в обслуживании, если в 
момент его поступления свободны менее~$b$ из $C$~единиц емкости звена.
{\looseness=1

}


  Пусть $X_n(t)$~--- число передаваемых в момент~$t$ $n$-бло\-ков. Тогда 
составной случайный процесс (СП) $\{\mathbf{X}(t) = (X_1(t),\ldots , X_N(t)),\ 
t\geq 0\}$ описывает функционирование модели над пространством со\-сто\-яний
  $$
  {\mathcal X} := \left \{ \mathbf{x} :=(x_1,\ldots , x_N):\ b\sum\limits_{n\in {\mathcal N}} 
x_n\leq C\right \}\,.
  $$
  
  Схема обслуживания в рассматриваемой модели показана на рис.~\ref{f3bor}, 
где использовано понятие <<элас\-тич\-ный трафик>> 
(elastic traffic)~\cite{6bor, 7bor}. Такой режим известен также под названием 
передачи по принципу ``best effort'', суть которого заключается в том, что в 
процессе одновременной передачи по звену сети с ограниченной пропускной 
способностью нескольких потоков (блоков данных) все потоки получают 
одинаковую долю пропускной способности вне зависимости от своего объема. 
В сделанных выше предположениях СП $\mathbf{X}(t)$ является марковским 
и, как нетрудно убедиться, равновесное   распреде-\linebreak\vspace*{-12pt}
\columnbreak

  \end{multicols}

 \begin{figure}[b]  %fig3
\begin{center}
%\vspace*{3pt}
\mbox{%
\epsfxsize=157.777mm
\epsfbox{bor-3.eps}
}
\end{center}
\vspace*{-9pt}
\Caption{Схема обслуживания эластичного трафика
\label{f3bor}}
\end{figure}

  
\begin{multicols}{2}

\noindent
ление вероятностей его 
состояний $\pi (\mathbf{x})$ определяется из системы уравнений локального 
баланса
\begin{multline}
\pi (\mathbf{x} ) \fr{C}{\theta_n}\, x_n \left ( \sum\limits_{j\in {\mathcal N}}x_j\right )^{-1} 
=\pi (\mathbf{x} -\mathbf{e}_n)\lambda_n\,,\\
\mathbf{x}\in {\mathcal X}\,,\ x_n >0\,, \  n\in {\mathcal N}\,,
\label{e1bor}
\end{multline}
    где $\mathbf{e}_n$~--- вектор, $n$-я компонента которого равна~1, а 
остальные~--- 0. Заметим, что в сис\-те\-ме урав\-не\-ний~(\ref{e1bor}) величина $(\sum\limits_{j\in {\mathcal 
N}} x_j)^{-1} Cx_n/\theta_n$ соответствует интенсивности обслуживания 
  $n$-блока в состоянии $\mathbf{x}\in {\mathcal X}$, поскольку обслуживание 
производится по дисциплине~PS.


  Используя известное для систем массового обслуживания с 
дисциплиной PS решение~\cite{8bor}, получаем равновесное распределение состояний 
СП~$\mathbf{X}(t)$в мультипликативном виде:
  \begin{multline*}
  \pi (\mathbf{x} ) = G^{-1} ({\mathcal X}) \left ( \sum\limits_{n\in {\mathcal N}} x_n\right ) !
  \prod_{n\in {\mathcal N}} \fr{1}{x_n!} \left ( \fr{a_n}{C}\right )^{x_n}\\ %\!\!\!\!\,,\quad 
\mathbf{x}\in {\mathcal X}\,,
  \end{multline*} %$$
  где $G({\mathcal X}) =\pi^{-1}(0))$~--- нормировочная константа.
  
  Искомой характеристикой является вероятность $B_n := {\mathcal P}(\mathbf{x}\in 
{\mathcal B}_n)$ блокировки запроса на передачу $n$-блока, где 
$${\mathcal B}_n := 
  \left\{ \mathbf{x}\in {\mathcal X}:\ b\left ( \sum\limits_{j\in{\mathcal N}} x_j+1\right ) >C\right\}\,.
  $$
   В следующем разделе будет получено аналитическое выражения для вычисления 
вероятностей~$B_n$, $n\in {\mathcal N}$.

\vspace*{-2pt}
  
  \section{Анализ вероятностных характеристик модели}
  
  Представим пространство состояний~${\mathcal X}$ в виде 
  $$
  {\mathcal X} :=  \bigcup\limits_{c=0}^{C_b} {\mathcal X}(c)\,,
  $$ 
  где 
%  \vspace*{-1pt}
  
  \noindent
  $$
  C_b := \left\lfloor \fr{C}{b}\right\rfloor\,,\quad {\mathcal X}(c) :=  \left\{ 
\mathbf{x}\in {\mathcal X}:\ \sum\limits_{n\in {\mathcal N}} x_n=c\right\}\,,
$$ 
и обозначим 
$$
q(c) :=  {\mathcal P} (\mathbf{x} \in{\mathcal X}(c))\,,\quad c=0,\ldots ,C_b\,.
$$

Используя уравнения 
локального баланса~(\ref{e1bor}) и приняв $x_j>0$ для некоторого $j\in {\mathcal N}$, получаем

\noindent
  \begin{multline*}
  q(c) =\sum\limits_{\mathbf{x}\in {\mathcal X}(c)} \pi (\mathbf{x}) = {}\\
  {}=
\sum\limits_{\mathbf{x}\in {\mathcal X} (c)} \fr{\lambda_j\theta_j}{Cx_j}\left ( 
\sum\limits_{n\in {\mathcal N}} x_n\right )\pi \left ( \mathbf{x}-\mathbf{e}_j\right 
)\,, \\ c=1,\ldots ,C_b\,.
  \end{multline*}

Пусть также найдется $n\in{\mathcal N}$ такое, что $x_n >0$, и положим $j=n$. Тогда 
из~(\ref{e1bor}) следует, что
\begin{multline*}
q(c) = \sum\limits_{x\in{\mathcal X}(c)} \sum\limits_{n\in{\mathcal N}} \fr{a_n}{C}\,\pi \left (\mathbf{x} -\mathbf{e}_n\right )
={}\\[3pt]
{}=\sum\limits_{n\in {\mathcal N}} \fr{a_n}{C}\sum\limits_{\mathbf{x}\in {\mathcal X}(c)}\pi\left (
\mathbf{x}-\mathbf{e}\right ) ={}\\[3pt]
{}= q(c-1)\sum\limits_{n\in {\mathcal N}}\fr{a_n}{C}\,,\quad c=1,\ldots ,C_b\,.
\end{multline*}
     
  Обозначив $a := \sum\limits_{n\in {\mathcal N}} a_n$, окончательно получаем
  \begin{equation}
  q(c) = q(0) \left (\fr{a}{C}\right )^c\,, \quad c=1,\ldots ,C_b\,,
  \label{e2bor}
  \end{equation}
где
\begin{equation}
q(0) =  \left ( \sum\limits_{c=0}^{C_b} \left (\fr{a}{C}\right )^c\right )^{-1}\,.
\label{e3bor}
\end{equation}
  
  Таким образом, если $a\not= C$, тогда
  $$
  q(c) = \fr{a^c C^{C_b-c} (C-a)}{C^{C_b+1} -a^{C_b+1}}\,,\quad c=0,\ldots 
,C_b\,,
  $$
и вероятность блокировки запроса на передачу\linebreak $n$-бло\-ка данных имеет вид
\begin{equation}
B_n = q(C_b) =\fr{a^{C_b} (C-a)}{C^{C_b+1} -a^{C_b+1}} =: B\,,
\label{e4bor}
\end{equation}
а обслуженная нагрузка, создаваемая $n$-блоками, вычисляется по формуле
$$
\tilde{a}_n = a_n\left ( 1-\fr{a^{C_b} (C-a)}{C^{C_b+1} -a^{C_b+1}}\right 
)\,,\quad n\in {\mathcal N}\,.
$$
  
  Заметим, что вероятность блокировки блока данных не зависит от его длины, 
что объясняется одинаковым для блоков любого типа требованием~$b$ к 
емкости звена. В этих условиях формулу~(\ref{e4bor}) можно получить, 
проведя анализ системы массового обслуживания типа 
$M$$\vert$$G$$\vert$1$\vert$PS с нагрузочным параметром~$a$, емкостью 
прибора~$C$ и максимальным числом находящихся на обслуживании 
заявок~$C_b$. Распределение вероятностей состояний такой сис\-те\-мы имеет 
вид~(\ref{e2bor}),~(\ref{e3bor}).
  
  \section{Заключение}
  
  В статье проведен анализ модели отдельного звена сети СС 
  в терминах систем, обслуживающих потоки <<эластичного трафика>> 
с заданным требованием к минимальному значению выделяемой при его 
передаче ширины полосы пропускания. В отличие от известных работ в этой 
области~\cite{6bor, 7bor}, формулы для вычисления вероятностей блокировок 
получены в аналитическом виде, что дает преимущество при вычислениях 
вероятностей блокировок для сети в целом при использовании метода 
просеянной нагрузки или решения задач оптимизации па\-ра\-мет\-ров 
СС при  заданных ограничениях на показатели качества 
их функционирования. Эти вопросы, также как и численный анализ, не 
являются предметом изложения в данной статье, но частично исследованы и 
решены в~\cite{5bor}.

{\small\frenchspacing
{%\baselineskip=10.8pt
\addcontentsline{toc}{section}{Литература}
\begin{thebibliography}{99}    
  
  \bibitem{1bor}
  \Au{Alberts~D.\,S., Garstka~J.\,J., Stein~F.\,P.}
  Network centric warfare: Developing and leveraging information superiority~/ 
DoD C4ISR Cooperative Research Program publication series. 2nd ed. (revised), 
2000. 284~p.
  
  \bibitem{2bor}
  \Au{Котенко~И.\,В., Боговик~А.\,В., Ковалев~И.\,С., Загорулько~С.\,С., 
Масановец~В.\,В.}
  Теория управления в системах военного назначения~/ Под ред.\ 
И.\,В.~Котенко.~--- М.: Изд-во МО, 2001. 320~с.
  
  \bibitem{3bor}
  \Au{Ross K.\,W.}
  Multiservice loss models for broadband telecommunication networks.~--- 
Springer, 1995. 343~p.
  
  \bibitem{4bor}
  \Au{Наумов~В.\,А., Самуйлов~К.\,Е., Яркина~Н.\,В.}
  Теория телетрафика мультисервисных сетей.~--- М.: Изд-во РУДН, 2007. 
191~с.
  
  \bibitem{5bor}
  \Au{Бородакий~В.\,Ю.}
  К решению задачи размещения центров обработки 
данных в сетецентрической системе~// Вестник РУДН, cер. <<Математика, 
информатика, физика>>, 2009. №\,3.
  
  
  \bibitem{7bor} %6
  \Au{Iversen~V.\,B.}
  Teletraffic engineering: Handbook.~--- \mbox{ITU-D}, June 2006. 354~p. Адрес в 
Интернете: {\sf http:// www.com.dtu.dk/teletraffic/handbook/telenook.pdf}.

  \label{end\stat}

  \bibitem{6bor} %7
  \Au{Меликов~А.\,З., Пономаренко~Л.\,А., Паладюк~В.\,В.}
  Телетрафик: модели, методы, оптимизация.~--- Киев: ИПК 
<<Политехника>>, 2007. 285~с.
  
  \bibitem{8bor}
  \Au{Бочаров~П.\,П., Печинкин~А.\,В.}
  Теория массового обслуживания.~--- М.: Изд-во РУДН, 1995. 529~с.
 \end{thebibliography}
}
}
\end{multicols} 
 
 
 
 