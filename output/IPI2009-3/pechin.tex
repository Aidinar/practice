%\renewcommand{\ge}{\geqslant}
%\renewcommand{\le}{\leqslant}



\def\stat{pech}

\def\tit{МНОГОЛИНЕЙНАЯ СИСТЕМА МАССОВОГО ОБСЛУЖИВАНИЯ С~ГРУППОВЫМ ОТКАЗОМ ПРИБОРОВ$^*$}
\def\titkol{Многолинейная система массового обслуживания с групповым отказом приборов} 

\def\autkol{А.\,В.\ Печинкин, И.\,А.~Соколов, В.\,В.~Чаплыгин}
\def\aut{А.\,В.\ Печинкин$^1$, И.\,А.~Соколов$^2$, В.\,В.~Чаплыгин$^3$}

\titel{\tit}{\aut}{\autkol}{\titkol}

{\renewcommand{\thefootnote}{\fnsymbol{footnote}}\footnotetext[1]
{Работа выполнена при поддержке РФФИ 
(гранты 08-07-00152 и 09-07-12032-офи-м).}}

\renewcommand{\thefootnote}{\arabic{footnote}}
\footnotetext[1]{Институт проблем
информатики Российской академии наук, apechinkin@ipiran.ru}
\footnotetext[2]{Институт проблем
информатики Российской академии наук, isokolov@ipiran.ru}
\footnotetext[3]{Институт проблем
информатики Российской академии наук, VasilyChaplygin@mail.ru}

\vspace*{-12pt}


\Abst{Рассматривается многолинейная система массового обслуживания (СМО)
$SM/PH/n/r$ ($r\le\infty$) с ненадежными приборами, отказывающими группами.
Отказы и восстановления групп приборов происходят с постоянной интенсивностью,
число отказывающих одновременно приборов является случайной величиной,
а заявки с прерванным обслуживанием после восстановления прибора начинают
обслуживаться заново.
Предложены методы расчета стационарного распределения числа заявок в системе
при различных вариантах функционирования системы.}

\KW{многолинейные системы массового обслуживания;
ненадежные приборы; отказ и восстановление случайного числа приборов}

\vskip 12pt plus 9pt minus 6pt

      \thispagestyle{headings}

      \begin{multicols}{2}

      \label{st\stat}

\section{Введение}

Cистемам массового обслуживания с ненадежными приборами уже более полувека
уделяется значительное внимание.
Среди публикаций по этой тематике подавляющее число работ посвящено однолинейным
СМО~[1--48]
и, в частности, системе типа $M/G/1$ и различным ее обобщениям,
когда в системе находится один обслуживающий прибор,
поступающий поток заявок является пуассоновским
(в некоторых случаях марковским или даже групповым марковским),
время обслуживания распределено по произвольному закону, причем времена
работы прибора в исправном и неисправном состояниях могут иметь различные распределения,
а сами отказы могут оказывать самое разнообразное воздействие на нормальное функционирование системы.
Так, например, в~\cite{Mikadze-Khocholava-Khurodze_2003, Mikadze-Khocholava-Khurodze_2004, Mikadze-Khocholava_2005}
рассмотрена однолинейная СМО
в стационарном и нестационарном режимах с пуассоновским входящим потоком,
гиперэрланговским потоком отказов прибора и временами обслуживания и восстановления,
распределенными по произвольному закону.
Работа~\cite{Kotlyar_1995} посвящена исследованию однолинейной
СМО $M/G/1$ с пуассоновским входящим потоком,
интенсивность которого зависит от того, в каком (исправном или неисправном) состоянии находится обслуживающий прибор,
время между отказами прибора распределено по экспоненциальному закону,
а время обслуживания и время восстановления прибора независимы
и имеют произвольные функции распределения.

Весьма популярны модели, в которых вследствие ремонта сервера образуется очередь повторных заявок.
В частности, экспоненциальные\linebreak
системы типа $M/M/1$ с ненадежным прибором и повторными заявками рассмотрены 
в~\cite{Li-Zhao_2005, Sherman-Kharoufeh_2006, Falin_2008}.
Cистемы типа $M/G/1$ с ненадежным прибором и повторными заявками рассмотрены
в~\cite{Yang-Li_1994, Wang-Cao-Li_2001, Krishna-Kumar-Pavai-Vijayakumar_2002, Djellab_2002, Li-Wang_2006, Sztrik-Almasi-Roszik_2006, Wang_2008}.

Однолинейная система с групповым пуассоновским потоком, рекуррентным обслуживанием,
отказами обслуживающего прибора и повторными заявками рассмотрена в~\cite{Atencia-Bouza-Moreno_2008}.
В~\cite{Li-Ying-Zhao_2006} рассматривается однолинейная СМО
$BM\!AP/G/1$ с групповым марковским потоком, в которой время жизни (lifetime) сервера
распределено по экспоненциальному закону, а время восстановления~--- уже по произвольному закону.
Ненадежные однолинейные СМО с групповым марковским потоком,
рекуррентным и полумарковским обслуживанием исследовались соответственно в~\cite{Dudin-Kazimirsky-Klimenok_2004} 
и~\cite{Dudin_2002}.

В работах~\cite{Ibe-Trivedi_1990, Boxma-Weststrate-Yechiali_1993, Blanc-Mei_1994, Kofman-Yechiali_1996, Almasi_1999, Nakdimon-Yechiali_2001}
рассмотрены системы с ненадежными приборами и упорядоченным опросом нескольких очередей (поллингом),
которые получили широкое распространение в рамках развития теоретических основ
проектирования широкополосных беспроводных сетей.
Подробную библиографию по системам поллинга и, в част\-ности, по системам поллинга с ненадежными приборами,
можно найти в обзорной статье~\cite{Vishnevsky-Semenova_2006} и в монографии~\cite{Vishnevsky-Semenova_2007}.
%\pagebreak

Cистемы массового обслуживания с ненадежными приборами в дискретном времени
рас\-смот\-ре\-ны в~\cite{Dimitrov-Dokev_1981, Lee_1997, Moreno_2006, Atencia-Moreno_2006}.

В гораздо меньшей степени исследовались СМО с несколькими ненадежными приборами
и сети массового обслуживания с ненадежными серверами.
В~[51--54] рассмотрены
различные варианты экспоненциальных систем с ненадежными серверами.
В~\cite{Mikadze-Mikadze-Khocholava_2007} рассмотрена многолинейная СМО $M/G/n/c$ с ненадежными приборами
в стацио\-нарном и нестационарном режимах, заявки в которой имеют
ограниченное распределенное по экспоненциальному закону время ожидания в системе.
В~\cite{Chakravarthy_1987} исследована модель с двумя ненадежными экспоненциально отказывающими
и восстанавливающимися серверами, разделенными конечным буфером.
В~\cite{Tananko-Yudaeva_2007_1, Tananko-Yudaeva_2007_2}
рассмотрена сеть массового обслуживания с одним экспоненциальным источником,
образованная параллельными однолинейными экспоненциальными СМО.

Заметим, что кроме перечисленных работ исследования по СМО
с ненадежными приборами можно также найти в работах~[59--72].
%\cite{White-Christie_1958, Thiruvengadam_1963, Jaiswal-Thiruvengadam_1963, Avi-Itzhak-Naor_1963, Elsayed-Proctor_1979,
%Federgruen-Green_1986, Sztrik-Gal_1990, Artalejo_1994, Babitsky-Dudin-Klimenok_1996, Krishnamoorthy-Ushakumari_1999,
%Dimitrov-Chukova-Chakravarthy_2001, Gray-Wang-Scott_2003, Gray-Wang-Scott_2004, Martin-Mitrani_2008}.

В указанных выше публикациях многолинейные системы с полумарковским потоком заявок,
обслуживанием фазового типа на каждом приборе\linebreak и отказами приборов не рассматривались.
На\-сто\-ящая работа посвящена исследованию многолинейной системы $SM/PH/n/r$ с ненадежными приборами,
отказы которых могут происходить\linebreak группами.
В~[73--75] рассмотрены различные варианты функционирования
СМО $SM/PH/n/r$\ с независимыми и одновременными отказами приборов.
{\looseness=-1

}

Поскольку дальнейшие наши исследования будут опираться на базовые модели СМО $SM/MSP/n/r$
с конечным и бесконечным накопителем, рассмотренные в~\cite{PSC07_1, PC04}, приведем их краткое описание.
Модели отличаются только емкостью накопителя (конечная или бесконечная),
поэтому будем рассматривать одну модель, отмечая при необходимости различия.

Всюду в дальнейшем будем обозначать через $\vec1$ вектор-столбец из единиц,
размерность которого определяется либо из контекста, либо нижним индексом.

Полумарковский входящий поток заявок определяется полумарковским
процессом с конечным множеством состояний
$\{1,2,\ldots,I\}$, $1\le I<\infty$, поведение которого
описывается квадратной матрицей $A(x)$.
Элемент $A_{ij}(x)$, $i,j=\overline{1,I}$, этой матрицы
представляет собой вероятность того, что первый переход из
состояния $i$ полумарковский процесс совершит в состояние $j$ и
произойдет этот переход за время меньше $x$.
Предполагается, что среднее время $a$ между поступлениями заявок
в стационарном режиме функционирования системы удовлетворяет
условию
$$
a = \vec\pi_a \int\limits_0^\infty x\, dA(x)\, \vec1 < \infty\,,
$$
где $\vec\pi_a$~--- вектор-строка стационарных вероятностей цепи
Маркова с матрицей $A=A(\infty)$ переходных вероятностей
(вложенной цепи Маркова).
Более подробное описание полумарковского входящего потока, а
также некоторые естественные дополнительные предположения
относительно параметров, которые также будут предполагаться
выполненными, приведены в~\cite{PC04}.

Перейдем теперь к марковскому процессу обслуживания.
Если в системе имеется $k$,  $0\le k\;\le$\linebreak $\le\;n+r$, заявок (в этом
случае также будем говорить, что процесс обслуживания
находится на слое~$k$), то процесс обслуживания может находиться
на одной из $l_k$, $l_k<\infty$, фаз обслуживания, причем
интенсивность изменения фазы марковского процесса равна
соответствующему элементу матрицы
$\Lambda_k$, $k=\overline{0,n+r}$, если ни одна заявка не
обслужилась, и матрицы $N_k$, $k=\overline{1,n+r}$, если одна
из заявок обслужилась.
Предполагается, что $l_k=l$ при $k=\overline{n,n+r}$,
матрицы $\Lambda_k = \Lambda$ совпадают при $k=\overline{n,n+r}$,
матрицы $N_k = N$ совпадают при $k=\overline{n+1,n+r}$,
матрица $\Lambda+N$ является неразложимой, а матрица $N$~---
ненулевой.
Кроме того, если в момент поступления очередной заявки в системе
находится $k$, $k=\overline{0,n-1}$, заявок, то вероятность
изменения фазы обслуживания определяется элементом матрицы
$\Omega_k$, а если не меньше $n$ заявок, то процесс обслуживания
прос\-то переходит на следующий слой с сохранением фазы обслуживания.
Обозначим через $\mu = \vec \pi_s N \vec1$, где $\vec\pi_s$~---
вектор-строка стационарных вероятностей марковского процесса\linebreak с
инфинитезимальной матрицей $\Lambda+N$, стацио\-нарную
интенсивность обслуживания при бесконечной очереди.
В случае накопителя бесконечной %\linebreak
емкости предполагается, что $\rho<1$,
где $\rho = (a \mu)^{-1}$~--- нагрузка на систему.
Если же накопитель имеет конечную емкость $r$, то поступающая
заявка теряется тогда, когда застает в системе $n+r$ заявок
(т.\,е.\ процесс обслуживания на слое $n+r$), не изменяя при этом
фазы обслуживания.
Более подробное описание марковского процесса обслуживания,
а также способ формирования его инфинитезимальной матрицы по
заданному PH-распределению обслуживания заявки на каждом приборе
можно найти, например, в~\cite{PC04,PC03}.

\section{Описание системы}

Рассмотрим многолинейную СМО $SM/PH/n/r$\  ($r\le \infty$) с
полумарковским входящим потоком заявок, распределением фазового
типа времени обслуживания каждой заявки, накопителем конечной
или бесконечной емкости и ненадежными приборами.

Распределение фазового типа (PH-рас\-пре\-де\-ле\-ние) времени обслуживания
заявки с числом фаз обслуживания $J$, $1\le J <\infty$,
описывается квад\-рат\-ной матрицей $H$ порядка $J$ с элементами
$h_{ij}$,  $i,j=\overline{1,J}$, и вектор-строкой $\vec h$
размерности~$J$ с элементами $h_i$, $i=\overline{1,J}$.
Функцию распределения фазового типа
времени обслуживания заявки можно записать в виде
$$
H(x) = 1-\vec h\, e^{Hx} \vec1\,.
$$
Далее понадобится также вектор $\vec h^*= -H\vec1$, координатой
$h^*_i$ которого является интенсивность окончания обслуживания заявки
при фазе обслуживания $i$.
Более подробное описание распределения фазового типа и его свойств
можно найти в~\cite{BP95}, а применительно к распределению времени
обслуживания в системах с ненадежными приборами~--- в~[73--75].

Каждый из $n$ имеющихся в системе однотипных приборов может находиться
либо в исправном, либо в неисправном состоянии.
Состояние прибора будем считать исправным, если на приборе находится
заявка и прибор занят ее обслуживанием или если прибор свободен,
готов принять заявку и немедленно начать ее обслуживание.
Состояние прибора будем считать неисправным, если на приборе находится
заявка, но прибор ее не обслуживает, или если на приборе нет заявки,
но он не может немедленно начать обслуживание заявки, если таковая на него
поступает.
Если при отказе прибора (переходе прибора из исправного состояния в
неисправное) на нем находится заявка, то она остается на приборе до
момента восстановления (перехода прибора из неисправного состояния в
исправное) и затем обслуживается заново.

Будем называть прибор занятым, если на нем находится заявка, и
свободным~--- в противном случае.
Если в некоторый момент времени в систему поступает очередная заявка,
но ни один прибор не может принять ее на обслуживание, то эта заявка
попадает в накопитель, становясь в очередь на обслуживание (в случае
конечной емкости накопителя заявка становится в очередь при наличии
в накопителе свободных мест или теряется при их отсутствии).
Заявки из очереди на обслуживание выбираются в порядке их поступления
в накопитель.

В следующих разделах будут рассмотрены три модели СМО с ненадежными
приборами, которые отказывают и восстанавливаются группами случайного
размера.
Будет показано, как эти модели можно привести к базовым,
что позволит вычислять стационарные вероятности состояний с помощью
ранее разработанных алгоритмов.
Поскольку вся сложность такого подхода заключается в замене процесса
обслуживания СМО с ненадежными приборами марковским процессом
обслуживания базовых моделей, то ограничимся только
построением матриц $\Lambda_k$, $k=\overline{1,n-1}$,
$\Lambda$, $N_k$, $k=\overline{1,n}$, и $N$ (матрица $\Lambda_0$
имеет порядок~1 и состоит из одного элемента~--- 0).

В настоящей работе, так же как и в~[73--75],
вмес\-то линейной нумерации состояний процесса обслуживания будем
использовать мультииндексную нумерацию, при которой номер состояния
определяется мультииндексом $(i_1,\ldots,i_r)$ или объединением таких
мультииндексов, где каждому чис\-лу~$k$ заявок в системе соответствует
одно или несколько значений~$r$, а $i_1,\ldots,i_r=\overline{1,J}$
(напомним, что $J$~--- чис\-ло фаз PH-распределения времени
обслуживания заявки), причем в случае $k\ge n$ подмножества
мультииндексов совпадают.
Перейти от мультииндексной нумерации состояний процесса обслуживания
к линейной можно различными способами.
В частности, рациональный способ такого перехода (на примере СМО
$SM/PH/n/\infty$ с надежными приборами) использован в~\cite{PC04,PC03}.

\section{Отказы только занятых приборов}

Первая рассматриваемая здесь СМО с ненадежными приборами
характеризуется следующими особенностями:
\begin{enumerate}[(1)]
\item  eсли в некоторый момент времени среди $n$ приборов ровно $k$,
$k=\overline{1,n}$, приборов исправны и заняты обслуживанием заявок,
то за <<малое>> время $\Delta$ с вероятностью
$\alpha_i(k)\Delta+o(\Delta)$, $i=\overline{1,k}$, произойдет
отказ ровно $i$ приборов с заявками;
\item
eсли в некоторый момент времени среди $n$ приборов ровно $k$,
$k=\overline{1,n}$, приборов с заявками неисправны, то за <<малое>>
время $\Delta$ с вероятностью $\beta_i(k)\Delta+o(\Delta)$,
$i=\overline{1,k}$, произойдет восстановление ровно $i$ приборов с
заявками;
\item
 при отказе или восстановлении $i$ приборов из $k$ возможных
($1\le i\le k$) все варианты выбора $i$ конкретных приборов из
$k$ равновероятны (вероятность выбора каждого варианта
$q^{i}_{k}=1/C^i_k$, где $C^i_k$~--- число сочетаний из $k$
элементов по $i$);
\item
свободные приборы находятся только в исправном состоянии;
\item
после восстановления прибора заявка, на нем находящаяся,
обслуживается заново.
\end{enumerate}

Эту СМО можно привести к базовым моделям, определив марковский
процесс обслуживания следующим образом (напомним, что слой
$k$, где $k=\overline{0,n+r}$ при $r<\infty$ и $k\ge 0$ при
$r=\infty$, состоит из всех состояний с $k$ заявками в системе).

Множество ${\cal X}_k$ состояний слоя $k$ при $k=\overline{0,n-1}$
марковского процесса обслуживания имеет вид
$$
{\cal X}_k
=
\{(i_1,\ldots,i_k)\cup (i_1,\ldots,i_{k-1})\cup\ldots\cup(0)\}\,,
$$
где состояние
$(i_1,\ldots,i_m)$,  $i_1,\ldots,i_m=\overline{1,J}$,\linebreak
$m=\overline{1,k}$,
означает, что $m$ приборов обслуживают заявки на фазах
$i_1,\ldots,i_m$, а $(k-m)$ приборов с заявками неисправны и
восстанавливаются, состояние $(0)$~--- все $k$ занятых приборов
находятся в неисправном состоянии.

Множество ${\cal X}_k$ состояний слоя $k$ при $k\ge n$ марковского
процесса обслуживания имеет вид
$$
{\cal X}_k
=
\{(i_1,\ldots,i_n)\cup (i_1,\ldots,i_{n-1})\cup\ldots\cup(0)\}\,,
$$
где состояние
$(i_1,\ldots,i_m)$, $i_1,\ldots,i_m=\overline{1,J}$, $m\;=$\linebreak $=\;\overline{1,n}$,
означает, что $m$ приборов обслуживают заявки на фазах
$i_1,\ldots,i_m$, остальные $(n-m)$ приборов с заявками неисправны
и восстанавливаются и еще $(k-n)$ заявок находятся в накопителе.
Состояние~$(0)$ означает, что все приборы заняты и неисправны, а
$(k-n)$ заявок находятся в накопителе.

Далее для единообразия записи состояние~$(0)$ (т.\,е.\ состояние
$(i_1,\ldots,i_m)$ при $m=0$) при всех $k$ будем отождествлять
с состоянием~(\ ).

Найдем интенсивности возможных переходов внутри каждого слоя
(элементы мат\-риц $\Lambda_k$, $k\;=$\linebreak $=\;\overline{1,n-1}$, и $\Lambda$)
и между слоями (элементы мат\-риц $N_k$,\ \ $k=\overline{1,n}$, и $N$)
марковского процесса обслуживания базовой модели, соответствующей
рас\-смат\-ри\-ва\-емой СМО.
Но прежде договоримся, рас\-смат\-ри\-вая состояние марковского процесса,
записывать сначала фазы обслуживания выделяемых приборов,
указывая в виде верхнего индекса номер прибора в круглых скобках.
Например, состояние $\left (j_1^{(l_1)},j_2^{(l_2)},i_1,\ldots,i_{m-2}\right )$
с двумя выделенными приборами, стоящими на местах
$l_1$ и $l_2$, $l_1\ne l_2$
(для определенности пусть $l_1 < l_2$, причем $l_1$ и $l_2$
могут принимать любые значения от $1$ до $m$), означает,
что первый прибор находится на фазе $i_1,\,\ldots,$
$(l_1\!-\!1)$-й прибор находится на фазе $i_{l_1-1}$,
$l_1$-й прибор находится на фазе $j_1$,
$(l_1+1)$-й прибор находится на фазе $i_{l_1},\,\ldots,$
$(l_2-1)$-й прибор находится на фазе $i_{l_2-2}$,
$l_2$-й прибор находится на фазе $j_2$,
$(l_2+1)$-й прибор находится на фазе $i_{l_2-1},\,\ldots,$
$m$-й прибор находится на фазе $i_{m-2}$.

Рассмотрим вначале переходы внутри слоя, возникающие при
изменении фазы обслуживания:
\begin{itemize}
\item из состояния $\left ( i^{(l)},i_1,\ldots,i_{m-1}\right )$,
$m=\overline{1,k}$, $l\;=$\linebreak $=\;\overline{1,m}$,
слоя $k$, $k=\overline{1,n-1}$, возможен переход в состояние
$\left( j^{(l)},i_1,\ldots, i_{m-1}\right )$ того же слоя с интенсивностью $h_{ij}$
при изменении с $i$-й на $j$-ю, $j\ne i$, фазы обслуживания на $l$-м
приборе;
\item
аналогично возможен переход из состояния
$\left ( i^{(l)},i_1,\ldots,i_{m-1}\right )$, $m=\overline{1,n}$,
$l=\overline{1,m}$, слоя $k$,  $k\ge n$, в состояние
$\left ( j^{(l)},i_1,\ldots,i_{m-1}\right )$, $j\ne i$, того же слоя
с интенсивностью~$h_{ij}$.
\end{itemize}

Следующие переходы внутри слоя происходят при отказе занятых
приборов:
\begin{itemize}
\item из состояния $\left (j_1^{(l_1)},\ldots,j_i^{(l_i)},i_1,\ldots,i_{m-i}\right )$,
$i\;=$\linebreak $=\;\overline{1,m}$,\ \ $m=\overline{1,k}$,
$1\le l_1<\ldots<l_i\le m$, слоя~$k$, $k=\overline{1,n-1}$,
возможен переход в состояние $(i_1,\ldots,i_{m-i})$ того же слоя
с интенсивностью $q^{i}_{m}\alpha_i(m)$ при отказе $i$ приборов из $m$;
\item
из состояния $\left ( j_1^{(l_1)},\ldots,j_i^{(l_i)},i_1,\ldots,i_{m-i}\right )$,
$i\;=$\linebreak $=\;\overline{1,m}$,\ \ $m=\overline{1,n}$,
$1\le l_1<\ldots<l_i\le m$, слоя $k$, $k\ge n$,
возможен переход в состояние $(i_1,\ldots, i_{m-i})$ того же слоя с
интенсивностью $q^i_m\alpha_i(m)$.
\end{itemize}

Последний возможный тип переходов внутри слоя образуют
переходы при восстановлении занятых приборов, причем
предполагается, что восстановившимся приборам присваиваются
порядковые номера, следующие за номерами исправных до момента
восстановления приборов:
\begin{itemize}
\item из состояния $(i_1,\ldots, i_m)$, $m=\overline{0,k-1}$,
слоя~$k$, $k=\overline{1,n-1}$, возможен с интенсивностью
$h_{i_{m+1}}\cdots h_{i_{m+j}}\beta_j(k-m)$ переход в состояние
$(i_1,\ldots,i_{m},i_{m+1},\ldots,i_{m+j})$, $j=\overline{1,k-m}$,
того же слоя при восстановлении $j$ приборов из $(k-m)$ и
возобновлении обслуживания находящихся на них заявок с новых фаз
$i_{m+1},\,\ldots,$ $i_{m+j}$;
\item
%аналогично, из состояния $(0)$ слоя $k$,\ \ $k=\overline{1,n-1}$,
%возможен переход в состояние
%$(i_{1},\ldots,i_{j})$,\ \ $j=\overline{1,k}$,
%того же слоя при восстановлении $j$ приборов из $k$ с
%интенсивностью $h_{i_{1}}\cdots h_{i_{j}}\beta_j(k)$;
из состояния $(i_1,\ldots, i_m)$, $m=\overline{0,n-1}$,
слоя~$k$, $k\ge n$, возможен переход в состояние
$(i_1,\ldots,i_{m},i_{m+1},\ldots,i_{m+j})$, $j=\overline{1,n-m}$,
того же слоя с интенсивностью
$h_{i_{m+1}}\cdots h_{i_{m+j}}\beta_j(n\;-$\linebreak $-\;m)$.  %;
\end{itemize}
%аналогично, из состояния $(0)$ слоя $k$,\ \ $k\ge n$, возможен
%переход в состояние $(i_{1},\ldots,i_{j})$,\ \ $j=\overline{1,n}$,
%того же слоя с интенсивностью $h_{i_{1}}\cdots h_{i_{j}}\beta_j(n)$.

Найденные интенсивности позволяют сформировать матрицы
$\Lambda_k$, $k=\overline{1,n-1}$, и $\Lambda$ базовых моделей.

Теперь рассмотрим переходы из слоя $k$ в слой $(k-1)$, которые могут
происходить только при окончании обслуживания заявок на приборах,
и их интенсивности задают матрицы $N_k$, $k=\overline{1,n}$, и~$N$:
\begin{itemize}
\item из состояния $\left (i^{(l)},i_1,\ldots,i_{m-1}\right )$,
$m=\overline{1,k}$, $l\;=$\linebreak $=\;\overline{1,m}$,
слоя $k$, $k=\overline{1,n}$, возможен переход в состояние
$\left (i_1,\ldots, i_{m-1}\right )$             %(или $(0)$, если $m=1$)
слоя $(k-1)$ с интенсивностью $h_i^*$ при окончании обслуживания заявки на
$l$-м приборе;
\item
из состояния $\left (i^{(l)},i_1,\ldots,i_{m-1}\right )$,
$m=\overline{1,n}$, $l\;=$\linebreak $=\;\overline{1,m}$,
слоя $k$, $k\ge n+1$, возможен переход в состояние
$\left ( i_1,\ldots,i_{m-1},j\right)$            %(или $(0)$, если $m=1$)
слоя $(k-1)$ с
интенсивностью $h_j h_i^*$ при окончании обслуживания заявки на
$l$-м приборе и поступлении на него заявки из накопителя,
которая начинает обслуживаться на фазе $j$.
\end{itemize}

Наконец, определим ненулевые элементы мат\-риц
$\Omega_k$, $k=\overline{0,n-1}$,
связанных с поступлением заявок в систему:
\begin{itemize}
%из состояния $(0)$ слоя $k$,\ \ $k=\overline{0,n-1}$, при
%поступлении новой заявки на свободный прибор и начале ее
%обслуживания на фазе $i$ с вероятностью $h_i$ происходит
%переход на слой $k+1$ в состояние $(i)$;
\item из состояния $(i_1,\ldots,i_m)$, $m=\overline{0,k}$, слоя
$k$, $k=\overline{0,n-1}$, при поступлении новой заявки на
свободный прибор и начале ее обслуживания на фазе $j$ с
вероятностью $h_j$ происходит переход в состояние
$(i_1,\ldots,i_m,j)$ слоя $k+1$.
\end{itemize}

Перечислив все возможные переходы между слоями марковского процесса
обслуживания с указанием их интенсивностей и вероятностей  и
определив тем самым ненулевые элементы матриц
$\Lambda_k$, $k=\overline{1,n-1}$, $\Lambda$,\ \ $N_k$,
$k=\overline{1,n}$, $N$ и $\Omega_k$, $k=\overline{0,n-1}$,
можно воспользоваться полученными для базовых моделей в~\cite{PSC07_1, PC04}
формулами расчета стационарных характеристик, связанных с числом
заявок в системе.

\section{Отказы всех приборов: заявки могут поступать на свободные
неисправные приборы}

Вторая рассматриваемая здесь СМО с ненадежными приборами отличается
от разобранной в предыду\-щем разделе тем, что наряду с
занятыми могут отказывать также и свободные приборы.
А~именно выполнены следующие предположения:
\begin{enumerate}[(1)]
\item eсли в некоторый момент времени среди $n$ приборов ровно
$k$, $k=\overline{1,n}$, приборов исправны и заняты обслуживанием
заявок, то за <<малое>> время $\Delta$ с вероятностью
$\alpha_i(k)\Delta+o(\Delta)$, $i=\overline{1,k}$, произойдет отказ
ровно $i$ приборов с заявками;
\item
eсли в некоторый момент времени среди $n$ приборов
ровно $k$, $k=\overline{1,n}$, приборов с заявками неисправны,
то за <<малое>> время $\Delta$ с вероятностью
$\beta_i(k)\Delta+o(\Delta)$, $i=\overline{1,k}$,
произойдет восстановление ровно $i$ приборов с заявками;
\item
если в некоторый момент времени среди $n$ приборов
ровно $k$, $k=\overline{1,n}$, свободных приборов исправны,
то за <<малое>> время $\Delta$ с ве\-ро\-ят\-ностью
$\alpha^*_i(k)\Delta+o(\Delta)$, $i=\overline{1,k}$,
произойдет отказ ровно $i$ свободных приборов;
\item
если в некоторый момент времени среди $n$ приборов
ровно $k$, $k=\overline{1,n}$, свободных приборов неисправны,
то за <<малое>> время $\Delta$ с вероятностью
$\beta^*_i(k)\Delta+o(\Delta)$, $i=\overline{1,k}$,
произойдет восстановление ровно $i$ свободных неисправных приборов;
\item
при отказе или восстановлении $i$ приборов из $k$ возможных ($1\le i\le k$)
все варианты выбора $i$ конкретных приборов из $k$ равновероятны
(вероятность выбора каждого варианта $q^i_k=1/C^i_k$);
\item
заявка, заставшая в момент поступления все исправные приборы
занятыми, поступает на один из свободных неисправных приборов, а
если таких нет, то попадает в накопитель;
\item
после восстановления прибора заявка, на нем находящаяся,
обслуживается заново.
\end{enumerate}

Марковский процесс обслуживания, приводящий эту СМО к базовой
модели, мало отличается от аналогичного процесса из предыдущего
пункта.

В частности, множество ${\cal X}_k$ состояний слоя $k$ при
$k\ge n$ имеет такой же вид
$$
{\cal X}_k
=
\{(i_1,\ldots,i_k)\cup (i_1,\ldots,i_{k-1})\cup\ldots\cup(0)\}\,,
$$
как и раньше, с теми же самыми комментариями относительно самих
состояний.

Множество ${\cal X}_k$ состояний слоя $k$ при $k=\overline{0,n-1}$
имеет вид
$$
{\cal X}_k
=
\{(i_1,\ldots,i_k;s)\cup (i_1,\ldots,i_{k-1};s)\cup\ldots\cup(0;s)\}\,,
$$
где состояние
$(i_1,\ldots,i_m;s)$, $i_1,\ldots,i_m=\overline{1,J}$,
$m\;=$\linebreak $=\;\overline{1,k}$, $s=\overline{0,n-k}$,
означает, что $m$ приборов обслуживают заявки на фазах
$i_1,\ldots,i_m$, а $(k-m)$ приборов с заявками и $s$ свободных
приборов неисправны и восстанавливаются.
Состояние $(0;s)$, $s=\overline{0,n-k}$, соответствует тому
случаю, когда все занятые приборы и $s$ свободных приборов
находятся в неисправном состоянии.

Используя далее, как и в предыдущем разделе, обозначения~(\ ) и
$(\ ;s)$ наряду с обозначениями $(0)$ и $(0;s)$ и приняв
прежнее соглашение о том, что фазы выделенных приборов
помечаются верхним индексом с номером прибора в круглых скобках,
видим, что при $k\ge n$ возможные переходы внут\-ри каждого слоя,
возникающие при изменении фазы обслуживания, а также при отказе и
восстановлении занятых приборов, полностью совпадают с переходами
для модели предыдущего раздела.

Те же переходы при $k=\overline{1,n-1}$ определяются следующим
образом:
\begin{itemize}
\item из состояния $\left (i^{(l)},i_1,\ldots,i_{m-1};s\right )$,
$m=\overline{1,k}$, $l=\overline{1,m}$, $s=\overline{0,n-k}$,
слоя $k$, $k=\overline{1,n-1}$, возможен переход в состояние
$\left (j^{(l)},i_1,\ldots,i_{m-1};s\right )$ того же слоя с интенсивностью $h_{ij}$
при изменении с $i$-й на $j$-ю, $j\ne i$, фазы обслуживания на $l$-м
приборе;
\item
из состояния $\left (j_1^{(l_1)},\ldots,j_i^{(l_i)},i_1,\ldots,i_{m-i};s\right )$,
$i=$\linebreak $=\;\overline{1,m}$, $m=\overline{1,k}$,
$1\le l_1<\ldots<l_i\le m$, $s=\overline{0,n-k}$,
слоя $k$, $k=\overline{1,n-1}$, возможен переход в состояние
$(i_1,\ldots,i_{m-i};s)$ того же слоя с интенсивностью
$q^i_m\alpha_i(m)$ при отказе $i$ занятых приборов из $m$;
\item
из состояния
$(i_1,\ldots, i_m;s)$, $m=\overline{0,k-1}$, $s\;=$\linebreak $=\;\overline{0,n-k}$,
слоя $k$, $k=\overline{1,n-1}$,
возможен переход в состояние
$(i_1,\ldots,i_{m},i_{m+1},\ldots,i_{m+j};s)$, $j=\overline{1,k-m}$,
того же слоя с интенсивностью $h_{i_{m+1}}\cdots h_{i_{m+j}}\beta_j(k-m)$
при восстановлении $j$ занятых приборов из $(k-m)$.
\end{itemize}
%из состояния $(0;s)$,\ \ $s=\overline{0,n-k}$,
%слоя $k$,\ \ $k=\overline{1,n-1}$, возможен переход в состояние
%$(i_{1},\ldots,i_{j};s)$,\ \ $j=\overline{1,k}$,
%того же слоя с интенсивностью $h_{i_{1}}\cdots h_{i_{j}}\beta_j(k)$
%при восстановлении $j$ занятых приборов из $k$.

Однако теперь могут происходить дополнительные переходы внутри
слоя при отказе и вос\-ста\-нов\-ле\-нии свободных приборов:
\begin{itemize}
\item из состояния $(i_1,\ldots,i_{m};s)$, $m=\overline{0,k}$,
$s\;=$\linebreak $=\;\overline{0,n-k-1}$,
слоя $k$, $k=\overline{1,n-1}$, возможен переход в состояние
$(i_1,\ldots,i_{m};s+j)$, $j\;=$\linebreak $=\;\overline{1,n-k-s}$, того же
слоя с интенсивностью $\alpha^*_j(n-k-s)$ при отказе $j$ из $(n-k-s)$
исправных свободных приборов;
\item
из состояния $(i_1,\ldots,i_{m};s)$, $m=\overline{0,k}$,
$s\;=$\linebreak $=\;\overline{1,n-k}$,
слоя $k$, $k=\overline{1,n-1}$, возможен переход в состояние
$(i_1,\ldots,i_{m};s-j)$, $j=\overline{1,s}$, того же слоя
с интенсивностью $\beta^*_s(j)$ при вос\-ста\-нов\-ле\-нии $j$ из $s$
неисправных свободных приборов.     %;
\end{itemize}

Так же как и в предыдущем разделе, переходы из слоя $k$ в слой $(k-1)$
могут происходить только при окончании обслуживания заявки на
приборе. Если при $k\ge n+1$ переходы те же самые, что и
раньше, то при $k=\overline{1,n}$ они отличаются лишь в силу изменения
самого определения состояния:
\begin{itemize}
\item из состояния $(i^{(l)},i_1,\ldots,i_{m-1};s)$,
$m=\overline{1,k}$, $l=\overline{1,m}$, $s=\overline{1,n-k}$,
слоя $k$, $k=\overline{1,n}$, возможен переход в состояние
$(i_1,\ldots,i_{m-1};s)$                 %(или $(0;s)$, если $m=1$)
слоя $k-1$ с
интенсивностью $h_i^*$ при окончании обслуживания заявки на $l$-м
приборе.
\end{itemize}

Наконец, определим ненулевые элементы мат\-риц
$\Omega_k$, $k=\overline{0,n-1}$, определяющих вероятности
изменения состояний при поступлении заявок в систему:
\begin{itemize}
\item из состояния
$(i_1,\ldots,i_m;s)$, $m=\overline{0,k}$, $s\;=$\linebreak $=\;\overline{0,n-k-1}$,
слоя $k$, $k=\overline{0,n-1}$, возможен переход на слой $k+1$
в состояние $(i_1,\ldots,i_m,i;s)$ с вероятностью $h_{i}$
при поступлении новой заявки на исправный свободный прибор и начале ее
обслуживания на фазе $i$;
\item
%аналогично, из состояния $(0;s)$,\ \ $s=\overline{0,n-k-1}$,
%слоя $k$,\ \ $k=\overline{0,n-1}$, возможен переход на слой $k+1$
%в состояние $(i;s)$ с вероятностью $h_i$
%при поступлении новой заявки на исправный свободный прибор и начале ее
%обслуживания на фазе $i$;
из состояния
$(i_1,\ldots,i_m;n-k)$, $m=\overline{0,k}$,
слоя $k$, $k=\overline{0,n-1}$, возможен переход на слой $k+1$
в состояние $(i_1,\ldots,i_m;n-k-1)$ с вероятностью $1$ при
поступлении новой заявки на неисправный свободный прибор.   %;
\end{itemize}
%аналогично, из состояния $(0;n-k)$ слоя $k$,\ \ $k=\overline{0,n-1}$,
%возможен переход на слой $k+1$ в состояние $(0;n-k-1)$ с вероятностью
%$1$ при поступлении новой заявки на неисправный свободный прибор.

Теперь осталось воспользоваться формулами расчета стационарного
распределения вероятностей состояний, полученными для базовой
модели.

%\vfill
%\eject

\section{Отказы всех приборов: заявки могут поступать только
на свободные исправные приборы}

Последняя СМО, которая будет здесь рас\-смот\-ре\-на, отличается от СМО
предыдущего раздела лишь тем, что заявка может поступать только
на свободный исправный прибор; в противном случае она ожидает появления
таких приборов в общей очереди в накопителе.
Однако, как будет видно далее, множество состояний теперь будет
несколько другим.

Отметим, что в случае накопителя конечной емкости эту СМО нельзя
привести к предложенной ранее базовой модели.
Поэтому ограничимся здесь только случаем накопителя бесконечной
емкости.

Для приведения этой СМО к базовой модели определим марковский
процесс обслуживания следующим образом.

Множество состояний ${\cal X}_k$ слоя $k$ при $k=\overline{0,n-1}$
имеет вид
$$
{\cal X}_k
=
\{(0;i_1,\ldots,i_m;m_1)\cup (1;i_1,\ldots,i_m;m_2)\}\,,
$$
где:
\begin{itemize}
\item состояние $(0;i_1,\ldots,i_m;m_1)$, $m=\overline{0,k}$,
$i_1,\ldots,i_m=\overline{1,J}$, $m_1=\overline{0,n-k}$,
означает, что все $k$ заявок находятся на приборах,
причем $m$ из них обслуживаются на фазах $i_1,\ldots,i_m$,
а остальные $(k-m)$ находятся на неисправных приборах,
и $m_1$ свободных приборов неисправны;
\item
 состояние $(1;i_1,\ldots,i_m;m_2)$, $m=\overline{0,k-1}$,
$i_1,\ldots,i_m=\overline{1,J}$, $m_2=\overline{0,k-1-m}$,
означает, что все свободные приборы неисправны,
$m$ приборов обслуживают заявки на фазах $i_1,\ldots,i_m$,
$m_2$ приборов с заявками вос\-ста\-нав\-ли\-ва\-ют\-ся, и $k-m-m_2$\ \ 
$(k-m-m_2>0)$ заявок находятся в накопителе.
\end{itemize}

Подчеркнем, что смысл индексов $m_1$ и $m_2$ для этих двух типов
состояний различен.

Множество состояний ${\cal X}_k$ слоя $k$ при $k\ge n$
марковского процесса обслуживания имеет вид
$$
{\cal X}_k
=
\{ (i_1,\ldots, i_m; m_2)\}\,,
$$
где состояние $(i_1,\ldots, i_m; m_2)$, $m=\overline{0,n}$,
$i_1,\ldots,i_m\;=$\linebreak $=\;\overline{1,J}$, $m_2=\overline{0,n-m}$,
означает, что $m$ приборов обслуживают заявки на фазах
$i_1,\ldots, i_m$, $m_2$ приборов с заявками неисправны и
восстанавливаются,
все (если таковые имеются) $(n-m-m_2)$ свободных приборов неисправны
и еще $(k-m-m_2)$ заявок находятся в накопителе.

Как и прежде, для единообразия записи примем соглашение, что
состояния $(0;\,;m_1)$, $(1;\,;m_2)$ и $(\,;m_1)$ соответствуют
тому случаю, когда все занятые заявками приборы неисправны.

Рассмотрим возможные переходы марковского процесса обслуживания,
договорившись, как и ранее, верхним индексом в круглых скобках
обозначать фазы выделенных приборов.

Первый тип переходов~--- из слоя $k$ на тот же самый слой~$k$.

Обратимся сначала к переходам на слое $k$ при смене фазы обслуживания:
\begin{itemize}
\item из состояния $\left (0;i^{(l)},i_1,\ldots,i_{m-1};m_1\right )$,
$l=\overline{1,m}$,\linebreak $m=\overline{1,k}$, $m_1=\overline{0,n-k}$,
слоя $k$, $k\;=$\linebreak $=\;\overline{1,n-1}$, возможен переход в состояние
$\left (0;j^{(l)},i_1,\ldots,i_{m-1};m_1\right )$ того же слоя с интенсивностью
$h_{ij}$ при изменении с $i$-й на $j$-ю, $j\ne i$, фазы обслуживания
на $l$-м приборе;
\item
 из состояния $\left(1;i^{(l)},i_1,\ldots,i_{m-1};m_2\right)$,
$l=\overline{1,m}$, $m=\overline{1,k-1}$, $m_2=\overline{0,k-1-m}$,
слоя $k$, $k=\overline{1,n-1}$, возможен переход в состояние
$\left (1;j^{(l)},i_1,\ldots,i_{m-1};m_2\right )$ того же слоя с интенсивностью
$h_{ij}$ при изменении с $i$-й на $j$-ю, $j\ne i$, фазы обслуживания
на $l$-м приборе;
\item
 из состояния $\left (i^{(l)},i_1,\ldots,i_{m-1};m_2\right )$,
$l=\overline{1,m}$, $m=\overline{1,n}$, $m_2=\overline{0,n-m}$,
слоя $k$, $k\ge n$, возможен переход в состояние
$\left (j^{(l)},i_1,\ldots,i_{m-1};m_2\right )$ того же слоя с интенсивностью
$h_{ij}$ при изменении с $i$-й на $j$-ю, $j\ne i$, фазы обслуживания
на $l$-м приборе.
\end{itemize}

Теперь рассмотрим переходы при отказе занятых приборов:
\begin{itemize}
\item из состояния\,$\left (0;j_1^{(l_1)}\!,\ldots,j_i^{(l_i)},i_1,\ldots,i_{m-i};m_1\right )$,
$i=\overline{1,m}$, $1\le l_1<\ldots<l_i\le m$,
$m=\overline{1,k}$, $m_1=\overline{0,n-k}$,
слоя $k$, $k=\overline{1,n-1}$, возможен переход в состояние
$(0;i_1,\ldots,i_{m-i};m_1)$ того же слоя с интенсивностью
$q^i_m\alpha_i(m)$ при отказе $i$ занятых приборов из $m$;
\item
 из состояния\,$\left (1;j_1^{(l_1)}\!,\ldots,j_i^{(l_i)},i_1,\ldots,i_{m-i};m_2\right )$,
$i=\overline{1,m}$, $1\le l_1<\ldots<l_i\le m$, 
$m=\overline{1,k-1}$, $m_2=\overline{0,k-1-m}$,
слоя $k$, $k=\overline{1,n-1}$, возможен переход в состояние
$(1;i_1,\ldots,i_{m-i};m_2+i)$ того же слоя с интенсивностью
$q^i_m\alpha_i(m)$ при отказе $i$ занятых приборов из $m$;
\item
из состояния $\left (j_1^{(l_1)},\ldots,j_i^{(l_i)},i_1,\ldots,i_{m-i};m_2\right )$,
$i=\overline{1,m}$, $1\le l_1<\ldots<l_i\le m$,
$m=\overline{1,n}$, $m_2=\overline{0,n-m}$,
слоя $k$, $k\ge n$, возможен переход в состояние
$(i_1,\ldots,i_{m-i};m_2+i)$ того же слоя с интенсивностью
$q^i_m\alpha_i(m)$ при отказе $i$ занятых приборов из $m$.
\end{itemize}

Перейдем к переходам при восстановлении занятых приборов (предполагается,
что восстановившимся приборам присваиваются порядковые номера,
следующие за номерами исправных до момента восстановления приборов):
\begin{itemize}
\item из состояния $\left(0;i_1,\ldots,i_m;m_1\right)$, $m\;=$\linebreak $=\;\overline{0,k-1}$,
$m_1=\overline{0,n-k}$, слоя $k$, $k\;=$\linebreak $=\;\overline{1,n-1}$, возможен
переход в состояние $\left(0;i_1,\ldots,i_{m},i_{m+1},\ldots,i_{m+j};m_1\right)$,
$j\;=$\linebreak $=\;\overline{1,k-m}$,
того же слоя с интенсивностью $h_{i_{m+1}}\cdots h_{i_{m+j}}\beta_j(k-m)$
при восстановлении $j$ занятых приборов из $(k-m)$ и начале обслуживания
заново на фазах $i_{m+1},\,\ldots,$ $i_{m+j}$ находящихся на них заявок;
\item
 из состояния $(1;i_1,\ldots,i_m;m_2)$, $m\;=$\linebreak $=\;\overline{0,k-2}$,
$m_2=\overline{1,k-1-m}$, слоя~$k$, $k=\overline{1,n-1}$, возможен
переход в состояние $\left(1;i_1,\ldots,i_{m},i_{m+1},\ldots,i_{m+j};m_2\right)$,
$j\;=$\linebreak $=\;\overline{1,k-1-m}$, того же слоя с интенсивностью
$h_{i_{m+1}}\cdots h_{i_{m+j}}\beta_j(k-1-m)$ при восстановлении $j$
занятых приборов из $(k-1-m)$ и начале обслуживания заново на фазах
$i_{m+1},\,\ldots,$ $i_{m+j}$ находящихся на них заявок;
\item
 из состояния $\left(i_1,\ldots,i_m;m_2\right)$, $m\;=$\linebreak $=\;\overline{0,k-2}$,
$m_2=\overline{1,k-1-m}$, слоя~$k$, $k\ge n$, возможен переход в
состояние $\left(i_1,\ldots,i_{m},i_{m+1},\ldots,i_{m+j};m_2\right)$,
$j\;=$\linebreak $=\;\overline{1,k-1-m}$, того же слоя с интенсивностью
$h_{i_{m+1}}\cdots h_{i_{m+j}}\beta_j(k-1-m)$ при восстановлении $j$
занятых приборов из $(k-1-m)$ и начале обслуживания заново на фазах
$i_{m+1},\,\ldots,$ $i_{m+j}$ находящихся на них заявок.
\end{itemize}

Далее, обратимся к переходам при отказе свободных приборов:
\begin{itemize}
\item на слое $k$, $k=\overline{0,n-1}$, переходы возможны только
из состояний $(0;i_1,\ldots,i_m;m_1)$, $m=\overline{0,k}$,
$m_1=\overline{0,n-k-1}$, в состояния $(0;i_1,\ldots,i_m;m_1+j)$
того же слоя с интенсивностью $q^j_{n-k-m_1}\alpha^*_j(n-k-m_1)$
при отказе $j$ свободных приборов из $n-k-m_1$;
\item
на слое $k$, $k\ge n$, переходы из-за отказа свободных приборов
не происходят.
\end{itemize}

Наконец, рассмотрим переходы при вос\-ста\-нов\-ле\-нии свободных приборов:
\begin{itemize}
\item из состояния $(0;i_1,\ldots,i_m;m_1)$, $m=\overline{0,k}$,
$m_1\;=$\linebreak $=\;\overline{1,n-k}$, слоя $k$, $k=\overline{0,n-1}$, возможен
переход в состояние $(0;i_1,\ldots,i_{m};m_1-j)$,\ \ $j=\overline{1,m_1}$,
того же слоя с интенсивностью~$\beta^*_j(m_1)$ при восстановлении
$j$ свободных приборов из~$m_1$;
\item
из состояния $(1;i_1,\ldots,i_m;m_2)$, $m\;=$\linebreak $=\;\overline{0,k-1}$,
$m_2=\overline{1,k-1-m}$, слоя $k$, $k=\overline{1,n-1}$,
возможен переход в состояние
$(1;i_1,\ldots,i_{m},i_{m+1},\ldots,i_{m+j};m_2)$,
$j\;=$\linebreak $=\;\overline{1,k-1-m-m_2}$, того же слоя с интенсивностью
$h_{i_{m+1}}\cdots h_{i_{m+j}}\beta^*_j(n-m-m_2)$ при вос\-ста\-нов\-ле\-нии
$j$ свободных приборов из $(n-m-m_2)$ и начале обслуживания части
(а именно $j$) заявок из очереди на фазах $i_{m+1},\,\ldots,$ $i_{m+j}$;
\item
из состояния $(1;i_1,\ldots,i_m;m_2)$, $m=\overline{0,k-1}$,
$m_2=\overline{1,k-1-m}$, слоя $k$, $k=\overline{1,n-1}$,
возможен также переход в состояние
$(0;i_1,\ldots,i_{m},i_{m+1},\ldots,i_{k-m_2};n-m-m_2-j)$,
$j=\overline{k-m-m_2,n-m-m_2}$, того же слоя с интенсивностью
$h_{i_{m+1}}\cdots h_{i_{k-m_2}}\beta_j(n-m-m_2)$
при восстановлении $j$ свободных приборов из $(n-m-m_2)$
и начале обслуживания всех $(k-m-m_2)$ заявок из очереди на фазах
$i_{m+1},\,\ldots,$ $i_{k-m_2}$;
\item
 из состояния $(i_1,\ldots,i_m;m_2)$, $m\;=$\linebreak $=\;\overline{0,n-1}$,
$m_2=\overline{1,n-1-m}$, слоя~$k$, $k\ge n$,
возможен переход в состояние
$(i_1,\ldots,i_{m},i_{m+1},\ldots,i_{m+j};m_2)$, $j\;=$\linebreak $=\;\overline{1,n-m-m_2}$,
того же слоя с ин\-тен\-сив\-ностью
$h_{i_{m+1}}\cdots h_{i_{m+j}}\beta^*_j(n-m-m_2)$
при восстановлении $j$ свободных приборов из $(n-m-m_2)$
и начале обслуживания $j$ заявок из очереди на фазах
$i_{m+1},\,\ldots,$ $i_{m+j}$.
\end{itemize}

Второй тип переходов~--- из слоя $k$ на слой $(k-1)$~--- образуют
переходы при окончании обслуживания заявки на приборе:
\begin{itemize}
\item из состояния $\left(0;i^{(l)},i_1,\ldots,i_{m-1};m_1\right)$,
$l=\overline{1,m}$, $m=\overline{1,k}$,
$m_1=\overline{0,n-k}$, слоя $k$, $k=\overline{1,n-1}$,
возможен переход в состояние $\left(0;i_1,\ldots, i_{m-1};m_1\right)$
слоя $(k-1)$ с интенсивностью $h_i^*$ при окончании обслуживания
заявки на $l$-м приборе;
\item
 из состояния $\left(1;i^{(l)},i_1,\ldots,i_{m-1};m_2\right)$,
$l=\overline{1,m}$, $m=\overline{1,k-2}$,
$m_2=\overline{0,k-2-m}$, слоя~$k$, $k=\overline{1,n-1}$,
(в накопителе находится две или более заявок)
возможен переход в состояние $\left(1;i_1,\ldots,i_{m-1},i_m;m_2\right)$
слоя $(k-1)$ с интенсивностью $h_i^* h_{i_m}$ при окончании
обслуживания заявки на $l$-м приборе и поступлении на него
заявки из накопителя, которая немедленно начинает обслуживаться
на фазе $i_m$;
\item
 из состояния $\left(1;i^{(l)},i_1,\ldots,i_{m-1};k-1-m\right)$,
$l=\overline{1,m}$, $m=\overline{1,k-1}$,
слоя $k$, $k\;=$\linebreak $=\;\overline{1,n-1}$,
(в накопителе находится ровно одна заявка) возможен переход
в состояние $\left(0;i_1,\ldots,i_{m-1},i_m;n-k+1\right)$ слоя $(k-1)$
(все заявки находятся на приборах, а все свободные приборы
неисправны) с интенсивностью $h_i^* h_{i_m}$ при окончании
обслуживания заявки на $l$-м приборе и поступлении на него
единственной заявки из накопителя, которая немедленно начинает
обслуживаться на фазе~$i_m$;
\item
из состояния $\left(i^{(l)},i_1,\ldots,i_{m-1};n-1-m\right)$,\linebreak
$l=\overline{1,m}$, $m=\overline{1,n-1}$,
слоя $n$ (в накопителе находится ровно одна заявка) возможен
переход в состояние $(0,i_1,\ldots,i_{m-1},i_m;1)$ слоя $(n-1)$
(все заявки находятся на приборах, а единственный свободный
прибор неисправен) с ин\-тен\-сив\-ностью $h_i^*h_{i_m}$ при окончании
обслуживания заявки на $l$-м приборе и поступлении на него
единственной заявки из накопителя, которая немедленно начинает
обслуживаться на фазе~$i_m$;
\item
 из состояния $\left(i^{(l)},i_1,\ldots,i_{m-1};m_2\right)$,
$l=\overline{1,m}$, $m=\overline{1,n-2}$,
$m_2=\overline{0,n-2-m}$, слоя $n$ (в накопителе находится две
или более заявок) возможен переход в состояние
$\left(1,i_1,\ldots, i_{m-1},i_m;m_2\right)$ слоя $(n-1)$ с интенсивностью
$h_i^*h_{i_m}$ при окончании обслуживания заявки на $l$-м приборе
и поступлении на него заявки из накопителя, которая немедленно
начинает обслуживаться на фазе~$i_m$;
\item
из состояния $\left(i^{(l)},i_1,\ldots,i_{m-1};m_2\right)$,
$l=\overline{1,m}$, $m=\overline{1,n}$,
$m_2=\overline{0,n-m}$, слоя $k$, $k>n$, возможен переход
в состояние $\left(i_1,\ldots, i_{m-1},i_m;m_2\right)$ слоя $(k-1)$ с
интенсивностью $h_i^*h_{i_m}$ при окончании обслуживания заявки
на $l$-м приборе и поступлении на него заявки из накопителя,
которая немедленно начинает обслуживаться на фазе~$i_m$.
\end{itemize}

Осталось определить возможные переходы последнего типа, связанные
с поступлением заявки, и ненулевые элементы матриц
$\Omega_k$, $k=\overline{0,n-1}$:
\begin{itemize}
\item из состояния $\left(0;i_1,\ldots,i_m;m_1\right)$, $m=\overline{1,k}$,
$m_1=\overline{0,n-k-1}$,
слоя $k$, $k=\overline{0,n-2}$, возможен переход на слой $(k+1)$
в состояние $\left(0;i_1,\ldots,i_m,i_{m+1};m_1\right)$ с вероятностью
$h_{i_m+1}$ при поступлении новой заявки на свободный прибор и
начале ее обслуживания на фазе $i_{m+1}$;
\item
из состояния $\left(0;i_1,\ldots,i_m;0\right)$, $m=\overline{1,n-1}$,
слоя $(n-1)$ возможен переход на слой $n$ в состояние
$\left(i_1,\ldots,i_m,i_{m+1};n-m-1\right)$ с вероятностью $h_{i_m+1}$ при
поступлении новой заявки на единственный свободный (исправный)
прибор и начале ее обслуживания на фазе $i_{m+1}$;
\item
из состояния $\left(0;i_1,\ldots,i_m;n-k\right)$, $m=\overline{1,k}$,
слоя $k$, $k=\overline{0,n-2}$, возможен переход на слой
$k+1$ в состояние $\left(1;i_1,\ldots,i_m;n-k-m\right)$ с вероятностью
$h_{i_m+1}$ при поступлении новой заявки в накопитель;
\item
из состояния $\left(0;i_1,\ldots,i_m;1\right)$, $m=\overline{1,n-1}$,
слоя $(n-1)$ (единственный свободный прибор находится в неисправном
состоянии) с ве\-ро\-ят\-ностью~1 происходит переход на слой~$n$ в
со\-сто\-яние $\left(i_1,\ldots,i_m;n-m-1\right)$ при поступлении новой заявки в
накопитель;
\item
из состояния $\left(1;i_1,\ldots,i_m;m_2\right)$, $m=\overline{0,k-1}$,
$m_2=\overline{0,k-1-m}$,
слоя~$k$, $k=\overline{0,n-2}$, (в накопителе есть заявки, и
все свободные приборы неисправны) возможен переход на слой $(k+1)$
в состояние $\left(1;i_1,\ldots,i_m;m_2\right)$ при поступлении новой заявки
в накопитель;
\item
из состояния $\left(1;i_1,\ldots,i_m;m_2\right)$, $m=\overline{1,n-2}$,
$m_2=\overline{0,n-2-m}$, слоя $(n-1)$ возможен переход на слой $n$ в
состояние $\left(i_1,\ldots,i_m;m_2\right)$ при поступлении новой заявки в
накопитель.
\end{itemize}

Перечислив все возможные переходы между слоями марковского процесса
обслуживания с указанием их интенсивностей или вероятностей и определив
тем самым матрицы $\Lambda_k$, $k=\overline{1,n-1}$,
$\Lambda$, $N_k$, $k=\overline{1,n}$, $N$ и $\Omega_k$,
$k=\overline{0,n-1}$, можно теперь для расчета стационарного
распределения числа заявок в системе воспользоваться соотношениями,
полученными для базовой модели.

\section{Заключение}

Таким образом, в настоящей статье получены соотношения, позволяющие
с помощью разра\-ботанной ранее базовой модели вы\-чис\-лять стационарные
характеристики очереди для\linebreak многолинейных СМО с ненадежными приборами,
полумарковским
входящим потоком, PH-рас\-пре\-де\-ле\-ни\-ем времени об\-слу\-жи\-ва\-ния заявок,
накопителем конечной или бесконечной емкости, обслу\-жи\-ва\-ни\-ем заново
заявок после восстановления\linebreak приборов и тремя типами групповых
отказов и восстановлений приборов:
\begin{enumerate}[(1)]
\item отказывают только приборы, на которых находятся заявки;
\item отказывают все приборы, а заявки могут поступать как на исправные,
так и на неисправные свободные приборы;
\item
отказывают все приборы, но заявки могут поступать только на
исправные свободные приборы.
\end{enumerate}

К сожалению, задача нахождения математических соотношений для
расчета стационарных
распределений, связанных с временем пребывания заявки в рассматриваемых
системах, не может быть решена с помощью базовой модели и
требует отдельного изучения за рамками данной статьи.


{\small\frenchspacing
{\baselineskip=11.9pt
\addcontentsline{toc}{section}{Литература}
\begin{thebibliography}{99}    

\bibitem{Dimitrov-Petrov_1981} %1
\Au{Dimitrov~B., Petrov~P.}
The minimal blocking time by unreliable server and latent failures~//
14th European Meeting of Statisticians, Wroclaw, 1981. P.~126--127.

\bibitem{Dimitrov-Dokev_1981} %2
\Au{Dimitrov B., Dokev~Ch.}
The single server queue system with non-reliable server in discrete time.
Non-stationary characteristics~// Ann. of Univ. of Sofia Ser. Math., Sofia, 1981. Vol.~70. P.~175--190.

\bibitem{Cao-Cheng_1982} %3
\Au{Cao~J., Cheng~K.}
Analysis of $M/G/1$ queueing system with repairable service station~//
Acta Mathematicae Applicatae Sinica, 1982. Vol.~5. P.~113--127.

\bibitem{Nicola_1986} %4
\Au{Nicola V.\,F.}
A single-server queue with mixed types of interruptions~//
Acta Inform., 1986. Vol.~23. No.~4.\linebreak P.~465--486.

\bibitem{Kulkarni-Choi_1990} %5
\Au{Kulkarni V.\,G., Choi~B.\,D.}
Retrial queues with server subject to breakdowns and repairs~//
Queueing Syst., 1990. Vol.~7. No.\,2. P.~191--208.

\bibitem{Ibe-Trivedi_1990} %6
\Au{Ibe O.\,C., Trivedi~K.\,S.}
Two queues with alternating service and server breakdown~// Queueing
Syst., 1990. Vol.~7. No.~3. P.~253--268.

\bibitem{Dimitrov-Khalil_1990} %7
\Au{Dimitrov~B., Khalil Z.}
On a new characterization of the exponential distribution related to a queuing system with an unreliable server~//
J. Appl. Probab., 1990. Vol.~27. P.~221--226.

\bibitem{Dimitrov-Khalil_1993} %8
\Au{Dimitrov~B., Khalil~Z.}
Some characterizations of the exponential distribution based on the service
time properties of an unreliable server~//
Lect. Notes Math. Stability Problems for Stochastic Models, 1993. Vol.~1546. P.~17--25.

\bibitem{Boxma-Weststrate-Yechiali_1993} %9
\Au{Boxma O.\,J., Weststrate~J.\,A., Yechiali~U.}
A globally gated polling system with server
interruptions, and applications to the repairman problem~//
 Probab. Eng. Inform. Sci., 1993. Vol.~7. No.~2. P.~187--208.

\bibitem{Khalil-Dimitrov_1994} %10
\Au{Khalil Z., Dimitrov~B.}
The service time properties of an unreliable server characterize the exponential distribution~//
Adv. Appl. Probab., 1994. Vol.~26. P.~172--182.

\bibitem{Yang-Li_1994} %11
\Au{Yang T., Li~H.}
The $M/G/1$ retrial queue with the server subject to starting failures~//
Queueing Syst. 1994. Vol.~16. Nos.\,1--2. P.~83--96.

\bibitem{Blanc-Mei_1994} %12
\Au{Blanc J.\,P.\,C., van der~Mei~R.\,D.}
The power-series algorithm applied to polling systems with a Dormant server~//
The fundamental role of teletraffic in the evolution of telecommunication networks~/
Eds. J.~Labetoulle and J.\,W.~Roberts.~--- Amsterdam: Elsevier, 1994. P.~865--874.

\bibitem{Aissani_1995} %13
\Au{Aissani~A.}
A retrial queue with redundancy and unreliable server~// Queueing Syst., 1994. Vol.~17. No.\,3--4. P.~431--449.

\bibitem{Hsieh-Andersland_1995} %14
\Au{Hsieh Y., Andersland M.\,S.}
Repairable single-server systems with multiple breakdown modes~//
Microelectron. Reliab., 1995. Vol.~35. No.~2. P.~309--318.

\bibitem{Kotlyar_1995} %15
\Au{Kotlyar V.\,Yu.}
Queueing system with an absolutely unreliable server and a variable stream of customers~//
Cybern. Syst. Anal., 1995. Vol.~31. No.~2. P.~285--292.

\bibitem{Chukova-Dimitrov_1996} %16
\Au{Chukova S., Dimitrov B.}
Execution time on an unreliable server with latent breakdowns~//
Matrix-analytics methods in stochastic models~/
Eds. S.~Chakravarthy and A.~Alfa.~--- New York, Basel, Hong Kong: Marcel Dekker, Inc., 1996. P.~225--239.

\bibitem{Kofman-Yechiali_1996} %17
\Au{Kofman~D., Yechiali~U.}
Polling with stations breakdowns~// Perform. Eval., 1996.
Vol.~27--28. No.~4. P.~647--672.

\bibitem{Li-Shi-Chao_1997} %18
\Au{Li W., Shi~D., Chao~X.}
Reliability analysis of $M/G/1$ queueing systems with server breakdowns and vacations~//
J.\ Appl.\ Probab., 1997. Vol.~34. No.\,2. P.~546--555.

\bibitem{Tang_1997} %19
\Au{Tang Y.\,H.}
A single-server $M/G/1$ queueing system subject to breakdowns~--- some reliability and queueing problems~//
Microelectron. Reliab., 1997. Vol.~37. No.\,2. P.~315--321.

\bibitem{Lee_1997} %20
\Au{Lee~D.-S.}
Analysis of a single server queue with semi-Markovian service interruption~//
Queueing Syst., 1997. Vol.~27. No.~1--2. P.~153--178.

\bibitem{Aissani-Artalejo_1998} %21
\Au{Aissani A., Artalejo J.\,R.}
On the single server retrial queue subject to breakdowns~//
Queueing Syst., 1998. Vol.~30. No.\,3--4. P.~309--321.

\bibitem{Atencia-Bocharov-Puzikova_1999} %22
\Au{Атенсиа И.\,М., Бочаров~П.\,П., Пузикова~Д.\,А.}
Матрично-мультипликативное решение для однолинейной системы с
отключениями прибора, конечной очередью повторных заявок и распределениями фазового типа //
Автоматика и телемеханика, 1999. №\,9. С.~72--91.

\bibitem{Almasi_1999} %23
\Au{Almasi~B.}
A queuing model for a non-homogeneous polling system subject to breakdowns~//
Ann. Univ. Sci. Budapest, Sect. Comp., 1999. Vol.~18. P.~11--23.

\bibitem{Chakravarthy-Krishnamoorthy-Ushakumari_2001} %24
\Au{Chakravarthy S.\,R., Krishnamoorthy~A., Ushakumari~P.\,V.}
A $k$-out-of-$n$ reliability system with an unreliable server and phase type repairs and services: 
The ($N$, $T$) policy~//
J.\ Appl.\ Math. Stochastic Anal., 2001. Vol.~14. P.~361--380.

\bibitem{Wang-Cao-Li_2001} %25
\Au{Wang J., Cao J., Li~Q.}
Reliability analysis of the retrial queue with server breakdowns and repairs~//
Queueing Syst., 2001. Vol.~38. No.\,4. P.~363--380.


\bibitem{Krishna-Kumar-Arivudainambi-Vijayakumar_2002} %26
\Au{Krishna Kumar B., Arivudainambi~D., Vijayakumar~A.}
An $M/G/1/1$ queue with unreliable server and no waiting capacity~//
Inf. Manage. Sci., 2002. Vol.~13. P.~35--50.

\bibitem{Krishna-Kumar-Pavai-Vijayakumar_2002} %27
\Au{Krishna Kumar B., Pavai M.\,S.,  Vijayakumar~A.}
The $M/G/1$ retrial queue with feedback and
starting failures~// Appl. Math. Model., 2002. Vol.~26. P.~1057--1075.

\bibitem{Djellab_2002} %28
\Au{Djellab N.\,V.}
On the $M/G/1$ retrial queue subjected to breakdowns~//
RAIRO Oper.\ Res., 2002. Vol.~36.\linebreak P.~299--310.

\bibitem{Dudin_2002} %29
\Au{Дудин А.\,Н.}
Оптимальное гистерезисное управление ненадежной системой
$ВМAP/SM/1$ с двумя режимами работы~//
Автоматика и телемеханика, 2002. №\,10. С.~58--72.

\bibitem{Nakdimon-Yechiali_2001} %30
\Au{Nakdimon O., Yechiali~U.}
Polling systems with breakdowns and repairs~//
Eur. J. Oper. Res., 2003. Vol.~149. No.~3. P.~588--613.

\bibitem{Chakravarthy-Agarwal_2003} %31
\Au{Chakravarthy S. R., Agarwal A.}
Analysis of a machine repair problem with an unreliable server and phase type repairs and services~//
Nav.\ Res. Log., 2003. Vol.~50. P.~462--480.

\bibitem{Xueming-Li_2003} %32
\Au{Xueming Y., Li~W.}
Availability analysis of the queueing system GI/PH/1 with server breakdowns~//
J.\ Syst. Sci. Complexity, 2003. Vol.~16. No.~2. P.~177--183.

\bibitem{Mikadze-Khocholava-Khurodze_2003} %33
\Au{Хуродзе Р.\,А., Хочолава В.\,В., Микадзе~И.\,С.}
Об одной системе массового обслуживания с ненадежной обслуживающей системой~//
Проблемы прикладной механики, 2003. №\,3(12). C. 9--18.

\bibitem{Mikadze-Khocholava_2004} %34
\Au{Микадзе И.\,С., Хочолава~В.\,В.}
Об одной модели передачи информации по ненадежному каналу связи~//
Автоматика и телемеханика, 2004. №\,8. С.~85--90.


\bibitem{Mikadze-Khocholava-Khurodze_2004} %35
\Au{Микадзе И.\,С., Хочолава В.\,В., Хуродзе~Р.\,А.}
Виртуальное время ожидания в однолинейной СМО с ненадежным прибором~//
Автоматика и телемеханика, 2004. №\,12. С.~119--128.

\bibitem{Dudin-Kazimirsky-Klimenok_2004} %36
\Au{Dudin A.\,N., Kazimirsky~A.\,V., Klimenok~V.\,I.}
$BM\!AP/G/1$ system unreliable in an idle state~//
Bull. Kerala Math. Assoc., 2004. No.~2. P.~1--19.

\bibitem{Mikadze-Khocholava_2005} %37
\Au{Микадзе И.\,С., Хочолава~В.\,В.}
Исследование длины очереди в однолинейной СМО с ненадежным прибором~//
Автоматика и телемеханика, 2005. №\,1.\linebreak С.~72--81.

\bibitem{Li-Zhao_2005} %38
\Au{Li H., Zhao~Y.\,Q.}
A retrial queue with a constant retrial rate, server downs and impatient customers~//
Stoch. Models, 2005. Vol.~21. P.~531--550.

\bibitem{Sherman-Kharoufeh_2006} %39
\Au{Sherman N. P., Kharoufeh~J.\,P.}
An $M/M/1$ retrial queue with unreliable server~//
Oper. Res. Lett., 2006. Vol.~34. No.~6. P.~697--705.

\bibitem{Li-Wang_2006} %40
\Au{Li J., Wang~J.}
An $M/G/1$ retrial queue with second multi-optional service, feedback and unreliable server~//
Appl. Math. --- J. Chin. Univ., 2006. Vol.~21. No.~3. P.~252--262.

\bibitem{Moreno_2006} %41
\Au{Moreno P.}
A discrete-time retrial queue with unreliable server and general server lifetime~// J.\ 
Math. Sci., 2006. Vol.~132. No.\,5. P.~643--655.


\bibitem{Sztrik-Almasi-Roszik_2006} %42
\Au{Sztrik J., Almasi~B., Roszik~J.}
Heterogeneous finite-source retrial queues with server subject to breakdowns and repairs~//
J.\ Math. Sci., 2006. Vol.~132, No.~5. P.~677--685.

\bibitem{Atencia-Moreno_2006} %43
\Au{Atencia I., Moreno~ P.}
A discrete-time $Geo/G/1$ retrial queue with the server subject to starting failures~//
Ann. Oper. Res., 2006. Vol.~141. No.~1. P.~85--107.

\bibitem{Li-Ying-Zhao_2006} %44
\Au{Quan-Lin Li, Ying Yu, Yiqiang Zhao~Q.}
A $BM\!AP/G/1$ retrial queue with a server subject to breakdowns and repairs~//
Ann. Oper. Res., 2006. Vol.~141. No.~1. P.~233--270.

\bibitem{Kumar-Krishnamoorthy-Madheswari-Basha_2007} %45
\Au{Krishna Kumar B., Krishnamoorthy~A., Pavai Madheswari~S., Sadiq Basha~S.}
Transient analysis of a single server queue with catastrophes, failures and repairs~//
Queueing Syst., 2007. Vol.~56. No.~3--4. P.~133--141.

\bibitem{Falin_2008} %46
\Au{Falin G.\,I.}
The $M/M/1$ retrial queue with retrials due to server failures~//
Queueing Syst., 2008. Vol.~58. No.~3. P.~155--160.

\bibitem{Wang_2008} %47
\Au{Wang J.}
On the single server retrial queue with priority subscribers and server breakdowns~//
J.\ Syst. Sci. Complexity, 2008. Vol.~21. No.~2. P.~304--315.

\bibitem{Atencia-Bouza-Moreno_2008} %48
\Au{Atencia I., Bouza~G., Moreno~P.}
An $M\hphantom{}^{[X]}/G/1$ retrial queue with server breakdowns and constant rate of repeated attempts~//
Ann. Oper. Res., 2008. Vol.~157. No.~1. P.~225--243.

\bibitem{Vishnevsky-Semenova_2006} %49
\Au{Вишневский В.\,М., Семёнова~О.\,В.}
Математические методы исследования систем поллинга~//
Автоматика и телемеханика, 2006. №\,2. C. 3--56.

\bibitem{Vishnevsky-Semenova_2007} %50
\Au{Вишневский В.\,М., Семёнова~О.\,В.}
Системы поллинга: теория и применение в широкополосных беспроводных сетях.~---
М.: Техносфера, 2007.~--- 312~c.

\bibitem{Mytrany-Avi-Itzhak_1968} %51
\Au{Mytrany I.\,L., Avi-Itzhak~B.}
A many-server queue with service interruptions~//
Oper. Res., 1968. Vol.~16.\linebreak
 P.~628--638.

\bibitem{Kabak_1968} %52
\Au{Kabak I.\,V.}
Blocking and delays in $M\hphantom{}^{(n)}/M/c$ queueing systems~//
Oper.\ Res., 1968. Vol.~16. P.~830--840.

\bibitem{Neuts-Lucantoni_1979} %53
\Au{Neuts M.\,F., Lucantoni~D.\,M.}
A Markovian queue with $N$ servers subject to breakdowns and repairs~//
Mgmt. Sci., 1979. Vol.~25. P.~849--861.

\bibitem{Roszik-Sztrik_2007} %54
\Au{Roszik J., Sztrik~J.}
Performance analysis of finite-source retrial queues with nonreliable heterogenous servers~//
J.~Math. Sci., 2007. Vol.~146. No.~4. P.~6033--6038.

\bibitem{Mikadze-Mikadze-Khocholava_2007} %55
\Au{Микадзе З.\,И., Микадзе~И.\,С., Хочолава~В.\,В.}
Об одной многоканальной смешанной
системе массового обслуживания с ограниченным временем ожидания~//
Автоматика и телемеханика, 2007. №\,7. C. 44--51.

\bibitem{Chakravarthy_1987} %56
\Au{Chakravarthy S.\,R.}
Analysis of production line systems with two unreliable machines with
phase type processing times and a finite storage buffer~// Stoch. Models, 1987. Vol.~3. P.~369--391.


\bibitem{Tananko-Yudaeva_2007_1} %57
\Au{Тананко И.\,Е., Юдаева~Н.\,В.}
Исследование сети массового обслуживания с ненадежными системами и задержкой информации~//
Тезисы докл. VIII Всероссийского симпозиума по прикладной и промышленной математике.
Обозрение прикладной и промышленной математики, 2007. Т.~14. Вып.~6. С.~1137--1138.

\bibitem{Tananko-Yudaeva_2007_2} %58
\Au{Тананко И.\,Е., Юдаева~Н.\,В.}
Моделирование сети массового обслуживания с ненадежными системами и задержкой информации~//
Компьютерные науки и информационные технологии: Тез. докл. Междунар. науч. конф.,
посвященной памяти проф.\ А.\,М.~Богомолова.~--- Саратов: Изд-во Сарат. ун-та, 2007.\linebreak
С.~119--120.

%-------------------------------------------------------------------------------------------------------

\bibitem{White-Christie_1958} %59
\Au{White H., Christie~L.\,S.}
Queueing with preemptive priorities or with breakdowns~//
Oper.\ Res., 1958. Vol.~6. P.~79--95.

\bibitem{Thiruvengadam_1963} %60
\Au{Thiruvengadam~K.}
Queueing with breakdowns~// Oper. Res., 1963. Vol.~11. P.~62--71.

\bibitem{Avi-Itzhak-Naor_1963} %61
\Au{Avi-Itzhak~B., Naor~P.\,P.}
Some queuing problems with the service station subject to breakdown~//
Oper.\ Res., 1963. Vol.~11. P.~303--320.

\bibitem{Jaiswal-Thiruvengadam_1963} %62
\Au{Jaiswal N.\,K., Thiruvengadam~K.}
Simple machine interference with two types of failure~//
Oper. Res., 1963. Vol.~11. No.~4. P.~624--636.

\bibitem{Elsayed-Proctor_1979} %63
\Au{Elsayed E.\,A., Proctor~C.\,L.}
Two repair policies for a machine interference problem with two types of failures~//
Proceedings of the Annual Pittsburgh Conference ``Modeling and Simulation,'' 1979. P.~197.

\bibitem{Federgruen-Green_1986}
\Au{Federgruen A., Green~L.}
Queueing systems with service interruptions~//
Oper. Res., 1986. Vol.~34. P.~752--768.

\bibitem{Sztrik-Gal_1990}
\Au{Sztrik J., Gal~T.}
A recursive solution of a queueing model for a multi-terminal system subject to breakdowns~//
Perform. Eval., 1990. Vol.~11. No.~1. P.~1--7.

\bibitem{Artalejo_1994}
\Au{Artalejo J.\,R.}
New results in retrial queueing systems with breakdown of the servers~//
Stat. Neerl., 1994. Vol.~48. No.~1. P.~23--36.

\bibitem{Babitsky-Dudin-Klimenok_1996}
\Au{Бабицкий А.\,В., Дудин~А.\,Н., Клименок~В.\,И.}
К расчету характеристик ненадежной системы массового обслуживания с конечным источником~//
Автоматика и телемеханика. 1996. №\,1. С.\ 92--103.

\bibitem{Krishnamoorthy-Ushakumari_1999}
\Au{Krishnamoorthy~A., Ushakumari~P.\,V.}
Reliability of a $k$-out-of-$n$ system with repair and retrial of failed units~//
TOP, 1999. Vol.~7. No.~2. P.~293--304.

\bibitem{Dimitrov-Chukova-Chakravarthy_2001} %69
\Au{Dimitrov B., Chukova~S., Chakravarthy~S.}
A simple unreliable service model characterizes exponential distribution~//
Kuwait J.\ Sci., 2001. Vol.~28 No.~2. P.~203--212.

\bibitem{Gray-Wang-Scott_2003}
\Au{Gray W.\,J., Wang~P.\,P., Scott~M.}
A queueing model with service breakdowns and multiple stages of repair~//
J.\ Appl.\ Statistical Sci., 2003. Vol.~12. No.~1. P.~75--89.

\bibitem{Gray-Wang-Scott_2004}
\Au{Gray W.\,J., Wang~P.\,P., Scott~M.}
A queueing model with multiple types of server breakdowns~//
Qual.\ Technol. Quant. Manage., 2004. Vol.~1. No.~2. P.~245--255.

\bibitem{Martin-Mitrani_2008}
\Au{Martin S.\,P., Mitrani~I.}
Analysis of job transfer policies in systems with unreliable servers~//
Ann. Oper. Res., 2008. Vol.~162. No.~1. P.~127--141.

\bibitem{PSC06}
\Au{Печинкин А.\,В., Соколов~И.\,А., Чаплыгин~В.\,В.}
Многолинейные системы массового обслуживания с независимыми
отказами и восстановлениями приборов~//
Системы и средства информатики.
Cпец.\linebreak выпуск <<Математическое и алгоритмическое обеспечение
информационно-телекоммуникационных сис\-тем>>.~---
М.: ИПИ РАН, 2006. С.~99--123.

\bibitem{PSC07_1}
\Au{Печинкин А.\,В., Соколов~И.\,А., Чаплыгин~В.\,В.}
Многолинейная система массового обслуживания с конечным
накопителем и ненадежными приборами~//
Информатика и её применения, 2007. Т.~1. Вып.~1. С.~27--39.

\bibitem{PSC07_2}
\Au{Печинкин А.\,В., Соколов~И.\,А., Чаплыгин~В.\,В.}
Стационарные характеристики многолинейной системы массового
обслуживания с одновременными отказами приборов~//
Информатика и её применения, 2007. Т.~1. Вып.~2. С.~28--38.

\bibitem{PC04}
\Au{Печинкин~А.\,В., Чаплыгин~В.,В.}
Стационарные характеристики системы массового обслуживания $SM/MSP/n/r$~//
Автоматика и телемеханика, 2004. №\,9. С.~85--100.

\bibitem{PC03}
\Au{Печинкин А.\,В., Чаплыгин~В.\,В.}
Стационарные характеристики системы массового обслуживания $G/MSP/n/r$~//
Вестник РУДН. Сер. <<Прикладная математика и информатика>>,
2003. №\,1. С.~119--143.


\label{end\stat}


\bibitem{BP95}
\Au{Бочаров П.\,П., Печинкин~А.\,В.}
Теория массового обслуживания.~---
М.: Изд-во РУДН, 1995.
 \end{thebibliography}
}
}
\end{multicols}
 