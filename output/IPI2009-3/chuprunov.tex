
\def\stat{chuprunov}


\def\tit{О ВЕРОЯТНОСТИ ИСПРАВЛЕНИЯ ОШИБОК
ПРИ~ПОМЕХОУСТОЙЧИВОМ КОДИРОВАНИИ,\\
КОГДА ЧИСЛО ОШИБОК ПРИНАДЛЕЖИТ\\
НЕКОТОРОМУ КОНЕЧНОМУ МНОЖЕСТВУ}
\def\titkol{О вероятности исправления ошибок
при~помехоустойчивом кодировании} %, когда число ошибок принадлежит
%некоторому конечному множеству}

\def\autkol{А.\,Н.~Чупрунов, Б.\,И.~Хамдеев}
\def\aut{А.\,Н.~Чупрунов$^1$, Б.\,И.~Хамдеев$^2$}

\titel{\tit}{\aut}{\autkol}{\titkol}

%{\renewcommand{\thefootnote}{\fnsymbol{footnote}}\footnotetext[1]
%{К-654.2008.1.}}

\renewcommand{\thefootnote}{\arabic{footnote}}
\footnotetext[1]{Научно-исследовательский институт математики и механики
им.~Н.\,Г.~Чеботарева, achuprunov@mail.ru}
\footnotetext[2]{Научно-исследовательский институт математики и механики
им.\ Н.\,Г.~Чеботарева, khamdeyevbi@mail.ru}

\Abst{Рассматриваются $n$ сообщений, каждое из которых состоит
из $N$ блоков. Каждый блок кодируется помехоустойчивым кодом,
который может исправить не более $q$ ошибок. При этом
предполагается, что число ошибок в каждом сообщении принадлежит
некоторому конечному подмножеству множества натуральных чисел. В
работе изучается вероятность ${\bf P}(A)$ события $A$, состоящего
в том, что  все ошибки будут исправлены. Вероятность ${\bf P}(A)$
 формулируется в терминах условных вероятностей.
  Показано, что  при $n, N\to\infty$ так, что
$\alpha=n/N\to \alpha_0<\infty$,  при $q=1$ вероятности
${\bf P}(A)$ сходятся и найдено значение этого предела, ${\bf
P}(A)\to 1$,  при $q>1$.}

\KW{условная вероятность; обобщенная схема
размещения; код Хемминга}


 \vskip 18pt plus 9pt minus 6pt

      \thispagestyle{headings}

      \begin{multicols}{2}

      \label{st\stat}

\section{Введение}

Будем рассматривать код, который позволяет исправить не больше $q$~ошибок
типа замещения, т.\,е.\ сферически упакованный код. Частным
случаем такого кода является код Хемминга (см.\ о коде Хемминга,
например, в~\cite{1ch}), который позволяет исправлять единичные ошибки.
Работа посвящена изучению вероятности события~$A$, состоящего в том,
что в $n$~сообщениях, каждое из которых состоит из $N$~блоков и
каждый блок кодируется помехоустойчивым кодированием, все ошибки
будут исправлены.

Пусть случайная величина $\xi_{ij}$~--- количество ошибок в $j$-м
блоке $i$-го сообщения. Будем предполагать, что $\xi_{ij}$~---
независимые неотрицательные целочисленные случайные величины,
распределенные так же, как случайная величина~ $\xi$.
Кроме того, будем предполагать, что число ошибок в
 сообщении принадлежит некоторому (одному и тому же для всех
 сообщений) подмножеству~${\bf N'}$ множества натуральных
 чисел~${\bf  N}$. Тогда $A=\bigcap\limits_{i=1}^nA_i$, где событие
 \begin{multline*}
\!\!A_i=\{\xi_{i1}\le q,\,\xi_{i2}\le q ...\xi_{iN}\le q\,\, |\,\,
\xi_{i1}+\dots +\xi_{iN}= m\\
\mbox{для\,\,
некоторого}\,\,m\in{\bf N'} \}
\end{multline*}
состоит в том, что в $i$-м сообщении каждый блок имеет не более
$q$~ошибок при условии, что число ошибок в сообщении принадлежит
множеству~${\bf N'}$. Поэтому вероятность события~$A$ равна

\noindent
\begin{multline*}
{\bf P}(A)={}\\
{}=
\left({\bf P}\{\xi_{1}\le q,\xi_{2}\le q ...\xi_{N}\le
q\,\, |\,\, \xi_{1}+\dots +\xi_{N}= m\,\right.\\
\left.\mbox{\rm для\,
некоторого}\,\,m\in{\bf N'} \}\right)^n\,,
%\label{c1}
\end{multline*}
где случайные величины $\xi_1, \xi_2,\dots ,\xi_N$ независимы и
распределены так же, как случайная величина~$\xi$.

Будем предполагать, что распределение случайной величины~$\xi$ имеет
следующий вид, зависящий от параметра~$\theta$. Пусть существует
последовательность неотрицательных чисел  $b_0, b_1, \dots$ такая,
что радиус сходимости $R$~ряда
$$
B(\theta)=\sum_{k=0}^{\infty}\fr{b_k\theta^k}{k!}
$$
положителен. Тогда  случайная величина $\xi=\xi(\theta)$,
$0<\theta<R$, распределена по следующему закону:
$$
{\bf P}\{\xi=k\}=\fr{b_k\theta^k}{k!B(\theta)}\,,\quad k=0,1,2,\dots
$$
Везде далее будем предполагать, что  $b_0, b_1>0$.

Пусть $\alpha=n/N$. В работе  изучается асимптотическое
поведение вероятности~${\bf P}(A)$ при $n,N\to\infty$ так, что
$\alpha\to\alpha_0$, $\alpha_0<\infty$ и множество ${\bf N'}$
конечно. В первой части предполагается, что случайные величины
$\xi=\xi(\theta)$ и последовательность $(b_1/b_0)N\theta$
сходится к конечному пределу.  Показано, что для $q=1$ при этих
условиях вероятность~${\bf P}(A)$ сходится, и найдено значение ее
предела; ${\bf P}(A)\to 0$  для $q>1$,  и найдена скорость этой
сходимости.

 Частным случаем случайных величин
$\xi_i(\theta)$ являются пуассоновские случайные величины c
па\-ра\-мет\-ром $\lambda=\theta$ (случай: $b_0=b_1=\dots$). Во второй
части работы получены аналоги результатов первой части  для
случая, когда $\xi_i$~--- независимые одинаково распределенные
пуассоновские случайные величины. При этом предполагается, что
параметр $\lambda=\theta$~--- константа. При $q=1$ показано, что
${\bf P}(A)$ сходится, и найдено значение этого предела, которое,
как оказалось, зависит только от максимального элемента множества~${\bf N'}$
и~$\alpha_0$ и не зависит от $\lambda=\theta$;  ${\bf
P}(A)\to 1$ для $q>1$, и найдена скорость этой сходимости.

Если множество~${\bf N'}$ состоит из одного элемента ( ${\bf
N'}=\{m\}$), то событие~$A_i$ является событием обобщенной схемы
размещения: событие~$A_i$ состоит в том, что в обобщенной схеме
размещения $m$~частиц по $N$~ячейкам в каждой ячейке оказалось не
более $q$~частиц. Обобщенную схему размещения ввел В.\,Ф.~Колчин в~\cite{2ch}
(см.\ также о ней монографию В.\,Ф.~Колчина~\cite{3ch}). Случайные
величины~$\xi_i(\theta)$ были введены А.\,В.~Колчиным  в~\cite{4ch}.
В~работах В.\,Ф.~Колчина  и А.\,В.~Колчина~[4--6] получены предельные
теоремы для сумм случайных величин~$\xi_i(\theta)$. Теорема~2 этой
работы основана на уточнении пуассоновской предельной теоремы из~\cite{4ch}.
Результаты настоящей работы являются обобщением результатов
работы~\cite{7ch}, полученных для схемы размещения различимых частиц по
ячейкам, на обобщенную схему размещения.

\section{Асимптотическое поведения вероятности  ${\bf P}(A)$ в случае
$\xi=\xi(\theta)$}

Рассмотрим вероятность $B(m, N)={\bf P}\{\xi_j\le1$,\linebreak $1\le j\le N,
\xi_1+\xi_2+\xi_3+\dots +\xi_N=m\}$. Легко заметить, что
$$
B(m, N)=C_N^m\left(\fr{b_1\theta}{B\left(\theta\right)}\right)^m
\left(\fr{b_0}{B\left(\theta\right)}\right)^{N-m}\,.
$$

Будем обозначать через
$\pi_{\lambda}(m)=e^{-\lambda}(\lambda^m/m!)$, $m=0,1,\dots,
$ пуассоновские вероятности. Пусть $(b_1/b_0)N\theta =\lambda$. Тогда
\begin{multline}
B(m, N)=\pi_{\lambda}(m)\left(1+\fr{O(1)}{N}\right)\\
\mbox{\rm
для всех}\,\, m=0, 1, 2,\dots\,,
\label{c2}
\end{multline}
где $O(1)$ ограничена по~$N$ и~$\lambda>0$, принадлежащей
ограниченному множеству. Будем использовать неравенства:
\begin{equation}
C_N^k\fr{1}{N^k}\le\fr{1}{k!}\,\,\,{и}\,\,\,\fr{b_0}{B(\theta)}\le 1\,.
\label{c3}
\end{equation}

В работе А.\,Ф.~Колчина~\cite{4ch} (теорема~2) доказано, что если
$(b_1/b_0)N\theta\to\lambda$ при $N\to\infty$, то
\begin{equation}
{\bf P}\{\xi_1+\xi_2+\dots +\xi_N=m\}\to \pi_{\lambda}(m)
\label{c4}
\end{equation}
для всех $m=0, 1,\dots$ Сделаем следующее уточнение теоремы  А.\,Ф.~Колчина.

\medskip

\noindent
{\bf Лемма 1}. {\it  Пусть
$\theta=\theta(N)=b_0\lambda/(b_1N)=C_1/N$ и
$C_2=(b_2b_0\lambda^2)/(2(b_1)^2)$.
 Выберем $N_0\in {\bf N}$  такое, что
 $$
C_3=\sum\limits_{i=3}^{\infty}\fr{b_i}{i!}\left(\theta(N_0)\right)^{i-3}<\infty\,.
$$
Тогда при $N>N_0$ и $m\le N$ имеем:
\begin{equation}
{\bf P}\{\xi_1+\xi_2+\dots +\xi_N=m\}=B(m,N)
\label{c5}
\end{equation}
при $m=0, 1$;
\begin{multline}
{\bf P}\{\xi_1+\xi_2+\dots +\xi_N=m\}={}\\
{}= B(m,N)+ \fr{1}{N}\,\fr{b_2 (C_1)^2}{2B\left(\theta\right)}B(m-2, N-1)
\label{c6}
\end{multline}
при $m=2$;
\begin{multline}
{\bf P}\{\xi_1+\xi_2+\dots +\xi_N=m\}={}\\
\!\!\!{}=
B(m,N)+ \fr{1}{N}\fr{b_2(C_1)^2}{2B\left(\theta\right)}B(m-2, N-1) + \fr{C}{N^2},\!\!
\label{c7}
\end{multline}
где $0\le C\le C_4$ и $C_4=e^{\lambda}(b_3/(6b_0))C_1^3$,
 при $m=3$;
\begin{multline}
{\bf P}\{\xi_1+\xi_2+\dots +\xi_N=m\}={}\\
\!\!\!{}=
B(m,N)+ \fr{1}{N}\,\fr{b_2 (C_1)^2}{2B\left(\theta\right)}B(m-2, N-1) + \fr{C}{N^2}, \!\!
\label{c8}
\end{multline}
где $0\le C\le C_4$ и
$C_4=e^{\lambda}(b_3/(6b_0))C_1^3+$\linebreak
$+e^{(C_2/N)+\lambda}\left(C_3C_1^3+C_2^2/2\right)$, при $m > 3$.}

\medskip

\noindent
Д\,о\,к\,а\,з\,а\,т\,е\,л\,ь\,с\,т\,в\,о\,.\ Обозначим
$$
\nu_1=\sum\limits_{j=1}^N\xi_jI_{\{\xi_j\le
1\}},\,\,\,\nu_2(k)=\sum\limits_{j=1}^k\xi_jI_{\{\xi_j\ge 2\}}\,.
$$
При $k=N$  будем обозначать случайную величины~$\nu_2(k)$ через~$\nu_2$. Рассмотрим выражение
\begin{multline}
{\bf P}\{\xi_1+\xi_2+\dots +\xi_N=m\}={}\\
{}={\sum\limits_{i=0}^m{\bf
P}(\nu_1=m-i, \nu_2=i)}\,.
\label{c9}
\end{multline}

Заметим, что
\begin{align}
{\bf P}(\nu_1&=m, \nu_2=0)=B(m, N)\,;\notag\\[-6pt]
&\label{c10}\\[-6pt]
{\bf P}(\nu_1&=m-1, \nu_2=1)=0 \notag
\end{align}

\vspace*{-6pt}

\noindent
\begin{multline}
{\bf P}(\nu_1=m-2,\,\nu_2=2)={}\\
{}=
\sum\limits_{j=1}^N {\bf P}(\nu_1=m-2,\ \xi_j=2,\ \xi_i<2,\ i\ne j)={}\\
{}=
N{\bf P}\left(\sum_{j=1}^N \xi_iI_{\{\xi_i\le 1\}}=m-2,\right.\\
\left. \xi_1=2,\ \xi_i<2,\ 2\le i\le N \vphantom{\sum_{j=1}^N}
\right)={}\\
{}=
N {\bf P}(\xi_1=2){\bf P}\left(\sum_{j=2}^N\xi_iI_{\{\xi_i\le
1\}}=m-2,\ \xi_i<2,\right.\\
\left.   2\le i\le N \vphantom{\sum_{j=1}^N} \right)=
N\fr{b_2\theta^2}{2B\left(\theta\right)}\,B(m-2, N-1)\,, \label{c11}
\end{multline}
и, так как ${\bf P}(\xi=1)\le \lambda/N$, имеем
\begin{multline}
{\bf P}(\nu_1=m-3,\, \nu_2=3)={}\\[6pt]
{}=
C_N^{m-3}({\bf P}(\xi=1))^{m-3}{\bf P}(\nu_2(N-(m-3))=3)\le{}\\[3pt]
{}\le
C_N^{m-3}\fr{\lambda^{m-3}}{N^{m-3}}\left (N-(m-3)\right)\fr{b_3}{6B(\theta)}\theta^3\le{}\\
{}\le
 \fr{e^{\lambda}}{N^2}\,\fr{b_3}{6b_0}(C_1)^3\,.
\label{c12}
\end{multline}
Равенства~(\ref{c9}) и~(\ref{c10}) влекут~(\ref{c5}), равенства~(\ref{c9})--(\ref{c11}) влекут~(\ref{c6}),
равенства~(\ref{c9})--(\ref{c11}) и оценка~(\ref{c12}) влекут~(\ref{c7}).

Рассмотрим
\begin{multline*}
\sum\limits_{i=4}^m {\bf P}(\nu_1=m-i, \nu_2=i)={}\\
{}=\sum\limits_{i=4}^m C_{N}^{m-i}({\bf P}(\xi=1))^{m-i}{\bf P}(\nu_2(N-(m-i))=i)\,.\hspace*{-8pt}
%\label{c13}
\end{multline*}
Обозначим $\nu_3(k)=\sum\limits_{j=1}^k \xi_jI_{\{\xi_j\ge 3\}}$. Так как
 $$
 {\bf P}(\xi=2)\le \fr{C_2}{N^2}\,,
 $$

 \vspace*{-5pt}

 \noindent
  то
  \vspace*{-2pt}

  \noindent
\begin{multline*}
{\bf P}(\nu_2(N-(m-i))=i) ={}\\
{}=
\sum\limits_{k=0}^{i/2}
C^k_{N-(m-i)}({\bf P}(\xi=2))^k\times{}\\
{}\times
{\bf P}(\nu_3(N-(m-i)-k)=i-2k)\le{}
\end{multline*}
\noindent
\begin{multline}
{}\le \sum\limits_{k=0}^{[ i/2]}
C^k_{N-(m-i)}\left(\fr{C_2}{N^2}\right)^k \times{}\\
{}\times {\bf P}(\nu_3(N-(m-i)-k)=i-2k)\le{}\\
{}\le
\sum\limits_{k=0}^{[i/2]}
\fr{1}{k!}\left(\fr{C_2}{N}\right)^k\times{}\\
{}\times {\bf P}(\nu_3(N-(m-i)-k)=i-2k)\,.
 \label{c14}
\end{multline}

Заметим, что
\begin{equation*}
{\bf P}(\nu_3(k)=j)\le {\bf P}(\nu_3(k)\ge j)\le k{\bf P}(\xi\ge 3)\le
\fr{C_3}{N^2}
\end{equation*}
при $k\le N$  и $j\ge 3$. Поэтому имеем
\begin{multline}
{\bf P}(\nu_2(N-(m-i))=i)\le{}\\
{}\le \sum\limits_{k=0}^{(i-1)/2}\fr{\left(C_2/N\right)^k}{k!}\,
\fr{C_3C_1^3}{N^2} \le
\fr{C_3С_1^3}{N^2}\exp\left(\fr{C_2}{N}\right)\,,
\label{c15}
\end{multline}
если $i$ нечетно, и
\begin{multline}
{\bf P}(\nu_2(N-(m-i))=i)\le{}\\
{}\le
\sum\limits_{k=0}^{i/2-1}\fr{\left(C_2/N\right)^k}{k!}\,
\fr{C_3C_1^3}{N^2} +
\fr{\left(C_2/N\right)^{i/2}}{(i/2)!} \le{}\\
{}\le
\fr{C_3C_1^3}{N^2}\exp\left(\fr{C_2}{N}\right)+\fr{\left(C_2/N\right)^{i/2}}{(i/2)!}\,,
\label{c16}
\end{multline}
если $i$ четно. Из~(\ref{c14})--(\ref{c16}) вытекает
\begin{multline}
\sum\limits_{i=4}^m {\bf P}(\nu_1=m-i,
\nu_2=i)\le{}\\
{}\le
\fr{C_3C_1^3}{N^2}\,e^{C_2/N}\sum\limits_{i=4}^m
C_N^{m-i}\left(\fr{\lambda}{N}\right)^{m-i} +{}\\
{}+
\sum\limits_{4\le i\le m, i\in{\bf 2N}}
C_N^{m-i}\left(\fr{\lambda}{N}\right)^{m-i}\fr{\left(C_2/N\right)^{i/2}}
{(i/2)!}\le{}\\
{}\le
\fr{C_3C_1^3}{N^2}\,e^{\lambda+C_2/N}+\left(\fr{C_2}{N}\right)^2 \times{}\\
{}\times \sum\limits_{4\le
i\le m, i \in{\bf 2N}}
C_N^{m-i}\left(\fr{\lambda}{N}\right)^{m-i}\fr{\left(C_2/N\right)^{i/2}-2}{\left(i/2-2\right)!}\le{}\\
{} \le
\fr{C_3C_1^3}{N^2}\,e^{C_2/N+\lambda}+\fr{(C_2)^2}{2N^2}\,e^{C_2/N} \times{}\\
{}\times \sum\limits_{i=4}^m
C_N^{m-i}\left(\fr{\lambda}{N}\right)^{m-i}\le{}\\
{}\le
\fr{1}{N^2}\,e^{C_2/N+\lambda}\left(C_3C_1^3+\fr{C_2^2}{2}\right)\,,
\label{c17}
\end{multline}
где ${\bf 2N}$~--- множество четных чисел.

Тогда~(\ref{c9})--(\ref{c12}) и~(\ref{c17}) влекут~(\ref{c8}). Доказательство закончено.
\pagebreak

\medskip

\noindent
\textbf{Замечание 1.} Из леммы~1 следует, что
\begin{multline*}
{\bf P}(\xi_1+\xi_2\dots
+\xi_N=m)={}\\
{}=\pi_{\lambda}(m)\left(1+\fr{O(1)}{N}\right)
\,\,\,\,\mbox{\rm для всех}\,\,\,\,m=0, 1,2,\dots,
\end{multline*}
где $O(1)$ ограничена по~$N$ и~$\lambda>0$, принадлежащей
ограниченному множеству.  Таким образом, получено уточнение теоремы А.\,Ф.~Колчина.


\medskip

\noindent
{\bf Теорема 1.}   {\it Пусть $\lambda=(b_1/b_0)N\theta$. Предположим, что $q=1$, множество ${\bf N'}$ конечно и
 $\max\{m:m\in{\bf N'}\}\ge$\linebreak $\ge\; 2$.
  Тогда

\noindent
  \begin{multline}
{\bf P}(A)=\exp
\left(\vphantom{\fr{\sum\limits_{k\in{\bf
N'},k\ge 2}\pi_{\lambda}(k-2)}{\sum\limits_{k\in{\bf N'}}
\pi_{\lambda}(k)}}
-\fr{b_2\lambda^2}{2b_0}\left(\fr{b_0}{b_1}\right)^2\right.\times{}\\
{}\times \left.\left(\fr{\sum\limits_{k\in{\bf
N'},k\ge 2}\pi_{\lambda}(k-2)}{\sum\limits_{k\in{\bf N'}}
\pi_{\lambda}(k)}\right)\alpha\left(1+\fr{O(1)}{N}\right)\right)\,,
\label{c18}
\end{multline}
где $O(1)$ ограничена по~$N$ и~$\lambda>0$, принадлежащей
ограниченному множеству.}

\medskip
\noindent
Д\,о\,к\,а\,з\,а\,т\,е\,л\,ь\,с\,т\,в\,о\,.\  Используя~(\ref{c2}) и лемму~1, получаем
\end{multicols}

\hrule

\noindent
\begin{multline*}
{\bf P}(A)=({\bf P}(A_i))^n=\left(\fr{ {\bf P}\{\xi_j\le 1,
1\le j\le N, \xi_1+\xi_2+\xi_3+\dots +\xi_N=k \,\,\mbox{\rm для\,\,
некоторого}\,\, k\in {\bf N'}\}}{{\bf P}\{\xi_1+\xi_2+\xi_3+\dots
+\xi_N=k \,\,\mbox{\rm для\,\, некоторого}\,\,k\in{\bf
N'}\}}\right)^n={}\\
{}=
\left(\fr{\sum\limits_{k\in{\bf N'}} {\bf P}\{\xi_j\le 1, 1\le
j\le N, \xi_2+\xi_3+\dots +\xi_N= k\}}{\sum\limits_{k\in{\bf
N'}}{\bf P}\{\xi_1+\xi_2+\xi_3+\dots +\xi_N= k\}}\right)^n={}\\
{}=
\left({\sum\limits_{k\in{\bf N'}} B(k,N)}\left/
\left({\sum\limits_{k\in{\bf N'}} B(k,N)
+\fr{1}{N}\,\fr{b_2\lambda^2}{2B(\theta)}\left(\fr{b_0}{b_1}\right)^2
\left(\sum\limits_{k\in{\bf N'},k\ge 2}B(k-2, N-1) \right)
\left(1+\fr{O'(1)}{N}\right)}\right.\right)\right)^n={}\\
{}=
\left(1\left/ \left (
1 +\fr{1}{N}\,\fr{b_2\lambda^2}{2B(\theta)}\left(\fr{b_0}{b_1}\right)^2
\fr{\sum\limits_{k\in{\bf N'},k\ge 2}B(k-2, N-1)}
{\sum\limits_{k\in{\bf N'}} B(k,N)}\left(1+\fr{O'(1)}{N}
\right)\right )\right.\right)^{N\alpha}={}\\
{}=
\exp
\left(-\fr{b_2\lambda^2}{2b_0}\left(\fr{b_0}{b_1}\right)^2\left(\fr{\sum\limits_{k\in{\bf
N'},k\ge 2}\pi_{\lambda}(k-2)}{\sum\limits_{k\in{\bf N'}}
\pi_{\lambda}(k)}\right)\alpha\left(1+\fr{O(1)}{N}\right)\right)\,.
\end{multline*}

\noindent
Доказательство закончено.
\smallskip

\hrule

\medskip


\begin{multicols}{2}


%\medskip

\noindent
{\bf Следствие 1}. {\it Пусть $\lambda=(b_1/b_0)N\theta$.
Предположим, что $q=1$, множество ${\bf N'}$ конечно и
 $\max\{m:m\in{\bf N'}\}\ge 2$.
 Пусть $\lambda\to\lambda'$,
$\alpha\to\alpha_0$ при $n, N\to\infty$. Тогда}
\begin{multline*}
{\bf P}(A)\to{}\\
\!{}\to  \exp
\left(\!-\fr{b_2\lambda'^2}{2b_0}\left(\fr{b_0}{b_1}\right)^2\left(\fr{\sum\limits_{k\in{\bf
N'},k\ge 2}\pi_{\lambda'}(k-2)}{\sum\limits_{k\in{\bf N'}}
\pi_{\lambda'}(k)}\right)\!\alpha_0\right)\!.\hspace*{-8pt}
\end{multline*}



\noindent
Д\,о\,к\,а\,з\,а\,т\,е\,л\,ь\,с\,т\,в\,о\,.\ Условие  $\lambda\to\lambda'$,
$\alpha\to\alpha_0$ при $n, N\to\infty$, влечет ограниченность
семейств $\lambda=\lambda(N)$ и $\alpha=n/N$. Поэтому
применима теорема~1.

Пусть $F_{\lambda}(x)=\sum\limits_{i\le x, i\in{\bf N}}\pi_{\lambda}(i)$,
$x\in{\bf R}$,~---  функция распределения пуассоновской случайной
величины с параметром~$\lambda$.

\smallskip


\noindent
{\bf Следствие 2.}  {\it Пусть $\lambda=(b_1/b_0)N\theta$.
Предположим, что $q=1$, множество ${\bf N'}=\{1, 2,\dots ,m\} $, где
$2\le m<\infty$.
 Пусть $\lambda\to\lambda'$,
$\alpha\to\alpha_0$ при $n, N\to\infty$. Тогда}
\begin{equation*}
{\bf P}(A)\to \exp
\left(-\fr{b_2\lambda'^2}{2b_0}\left(\fr{b_0}{b_1}\right)^2\fr{F_{\lambda'}(m-2)}{F_{\lambda'}(m)}\,\alpha_0\right)\,.
\end{equation*}

\medskip

\noindent
Д\,о\,к\,а\,з\,а\,т\,е\,л\,ь\,с\,т\,в\,о\,.\
Так как множество ${\bf N'}=$\linebreak $=\;\{1, 2,\dots  ,m\}$ конечно, применимо следствие~1.

\medskip

Из следствия~1 вытекает

\noindent
{\bf Следствие 3.}   {\it  Пусть $\lambda=(b_1/b_0)N\theta$.
Предположим, что $q=1$, множество   ${\bf N'}=\{m\}$, где  $2\le
m<\infty$.  Пусть $\lambda\to\lambda'$, $\alpha\to\alpha_0$ при
$n, N\to\infty$. Тогда}
\begin{equation*}
{\bf P}(A)\to \exp
\left(-\fr{b_2}{2b_0}\left(\fr{b_0}{b_1}\right)^2\fr{m(m-1)}{2}\alpha_0\right)\,.
\end{equation*}

\medskip

Следующая теорема показывает, что случай $q>1$ существенно
отличается от случая $q=1$.

\medskip

\noindent
{\bf Теорема 2}.   {\it Пусть $\lambda=(b_1/b_0)N\theta$.
Предположим, что множество~${\bf N'}$ конечно и
 $m=\max\{i: i\in{\bf N'}\}> $\linebreak $>q>1$.
  Тогда
\vspace*{-2pt}

\noindent
\begin{equation}
1-{\bf P}(A)=
C\fr{\alpha}{N^{q-1}}\left(1+\fr{O(1)}{N}\right)\,,
 \label{c19}
\end{equation}
\vspace*{-4pt}

\noindent
где
\vspace*{-2pt}

\noindent
\begin{equation*}
C=\left(\fr{b_{q+1}}{(q+1)!b_0}\left(\fr{b_0\lambda}{/b_1}\right)^{q+1}\right)\left/ \sum\limits_{k\in
{\bf N'}}\pi_{\lambda}(k)\right.
\end{equation*}
и $O(1)$ ограничена по~$N$ и $\lambda>0$, принадлежащей ограниченному множеству.
В частности, если $\alpha=o\left(N^{q-1}\right)$ при $n, N\to\infty$, то
\begin{equation}
{\bf P}(A)\to 1\,\,\,\,\mbox{\rm при}\,\,\,\,N\to\infty
\label{c20}
\end{equation}
равномерно относительно $\lambda>0$, принадлежащей ограниченному множеству. }

\smallskip

\noindent
Д\,о\,к\,а\,з\,а\,т\,е\,л\,ь\,с\,т\,в\,о\,.\ Пусть событие
 \begin{multline*}
A_{ij}=\{\xi_{ij}\le q\,\,\, |\,\,\, \xi_{i1}+\dots +\xi_{iN}= m\\
\mbox{\rm для\,\, некоторого}\,\,m\in{\bf N'} \},
\end{multline*}
состоит в том, что в $j$-м блоке $i$-го сообщения имеется не больше
$q$~ошибок, $A_{ij}^c$~--- его дополнение.  Используя~(\ref{c3}), (\ref{c4}) и
замечание~1, получаем
\end{multicols}

\hrule

\noindent
\begin{multline*}
{\bf P}(A_{ij}^c)=\fr{{\bf P}\{\xi_1\ge q+1,\,\,\xi_2+\xi_3+\dots
+\xi_N=k \,\,\, \mbox{\rm для некоторого}\,\,\,\,k\in{\bf
N'}\}}{{\bf P}\{\xi_1+\xi_2+\xi_3+\dots +\xi_N=k \,\,\,\mbox{\rm для
некоторого}\,\,\,\,k\in{\bf N'}\}}\le{}\\[6pt]
{}\le
\fr{{\bf P}\{\xi \ge q+1\}}{\sum\limits_{k\in {\bf N'}}{\bf P}
\{\xi_1+\xi_2+\xi_3+\dots +\xi_N=k\}}\le{}\\[6pt]
{}\le \left(\fr{b_{q+1}}{(q+1)!b_0}\left(\fr{b_0\lambda}{b_1}\right)^{q+1}
 \fr{1}{N^{q+1}}\left(1+\fr{O'(1)}{N}\right)\right)\left/
\left(\left(\sum\limits_{k\in {\bf N'}}\pi_{\lambda}(k)\right)
\left(1+\fr{O^{\prime\prime}(1)}{N}\right)\right.\right) ={}\\[6pt]
{}=\fr{1}{N^{q+1}b_0}\left(\fr{b_{q+1}}{(q+1)!}\left(\fr{b_0\lambda}{b_1}\right)^{q+1}\left/
{\sum\limits_{k\in {\bf N'}}\pi_{\lambda}(k)}\right.\right)
\left(1+\fr{O(1)}{N}\right)=
\fr{C}{N^{q+1}}\left(1+\fr{O(1)}{N}\right)\,.
\end{multline*}
%\vspace*{-6pt}

\noindent
Поэтому
%\vspace*{-3pt}

\noindent
$$ %\begin{multline*}
1-{\bf P}(A)={\bf
P}\left(\bigcup\limits_{i=1}^n\bigcup\limits_{j=1}^{N}A_{ij}^c\right)\le
\sum\limits_{i=1}^n\sum\limits_{j=1}^{N}{\bf
P}\left(A_{ij}^c\right)\le 
 n N \fr{C}{N^{q+1}}(1+o(1))=
C\fr{\alpha}{N^{q-1}}\left(1+\fr{O(1)}{N}\right)\,.
$$%\end{multline*}
Итак, утверждение~(\ref{c19}) доказано.
\smallskip

\hrule

\begin{multicols}{2}

Так как  $C(\alpha/N^{q-1})(1+o(1))\to 0$ при $ N\to\infty$, (\ref{c19}) влечет~(\ref{c20}). Теорема доказана.

\medskip

Теперь рассмотрим событие~$A$ без условия $\xi_1 + \xi_1 + \dots +
\xi_N=k$ для некоторого~$k\in{\bf N'}$. Это соответствует случаю
${\bf N}={\bf N'}$. Для него получим выражение для вероятности~${\bf P}(A)$
такое, что аналоги~(\ref{c18}) и~(\ref{c19}) будут его следствиями.

\medskip

\noindent
{\bf Теорема~3}.  {\it  Пусть $\lambda=(b_1/b_0)N\theta$.
Предположим, что $q \ge 1$, множество ${\bf N}={\bf N'}$.
  Тогда

\noindent
\begin{multline}
{\bf P}(A)=\exp\left(-\fr{1}{N^{q-1}}\,\fr{b_{q+1}}{(q+1)!b_0}
\left(\fr{b_0}{b_1}\right)^{q+1}\times{}\right.\\
\left.{}\times \lambda^{q+1}\alpha\left(1+\fr{O(1)}{N}\right)\right)\,,
\label{c21}
\end{multline}
где $O(1)$ ограничена по~$N$ и~$\lambda>0$, принадлежащей
ограниченному множеству.}

\medskip

\noindent
Д\,о\,к\,а\,з\,а\,т\,е\,л\,ь\,с\,т\,в\,о\,.\ Имеем

\end{multicols}
\pagebreak

\hrule

\noindent
\begin{multline*}
{\bf P}(A)=\left({\bf P}\left\{\xi_i\le q, 1\le i\le
N\right\}\right)^n =
\left({\bf P}\left\{\xi_1\le
q\right\}\right)^{Nn}=\left(\fr{b_0}{B(\theta)} +
\fr{b_0}{B(\theta)}\theta + \cdots +
\fr{b_0}{B(\theta)}\theta^q\right)^{Nn} ={}
\\
%\begin{multline*}
{}=
\left(\left(b_0+\fr{b_1}{1!}\left(\fr{b_0}{b_1}\right)\fr{\lambda}{N}+\dots
+\fr{b_q}{q!}\left(\fr{b_0}{b_1}\right)^{q}\fr{\lambda^q}{N^q}\right)\left/
\left( b_0+\fr{b_1}{1!}\left(\fr{b_0}{b_1}\right)
\fr{\lambda}{N}+ \dots {}\right.\right.\right.\\[6pt]
\left.\left.\dots +\fr{b_q}{q!}\left(\fr{b_0}{b_1}\right)^{q}\fr{\lambda^q}{N^q}
+\fr{b_{q+1}}{(q+1)!}\left(\fr{b_0}{b_1}\right)^{q+1}\fr{\lambda^{q+1}}{N^{q+1}}+\dots\right)
\right)^{N^2\alpha}={}\\[6pt]
{}=
\left(1\left/ \left(1+\fr{b_{q+1}}{(q+1)!}\left(\fr{b_0}{b_1}\right)^{q+1}\fr{\lambda^{q+1}}{N^{q+1}}
\left({b_0}+\fr{b_1}{1!}\left(\fr{b_0}{b_1}\right)\fr{\lambda}{N}+\dots
+\fr{b_q}{q!}\left(\fr{b_0}{b_1}\right)^{q}\fr{\lambda^q}{N^q}\right)^{-1}\left(1+\fr{O(1)}{N}\right)\right)\right.
\right)^{N^2\alpha}={}\\[6pt]
{}=
\exp\left(-\fr{1}{N^{q-1}}\fr{b_{q+1}}{(q+1)!b_0}\left(\fr{b_0}{b_1}\right)^{q+1}\lambda^{q+1}\alpha\left(1+
\fr{O(1)}{N}\right)\right)\,.
\end{multline*}

\noindent
Доказательство закончено.
\smallskip

\hrule

\begin{multicols}{2}

\medskip

\noindent
{\bf Следствие 1}. {\it Пусть выполнены условия теоремы~3 и $q\ge 2$. Тогда
\begin{multline}
1-{\bf P}(A)\le
\fr{1}{N^{q-1}}\fr{b_{q+1}}{(q+1)!b_0}\left(\fr{b_0}{b_1}\right)^{q+1}\times{}\\
{}\times \lambda^{q+1}\alpha\left(1+\fr{O(1)}{N}\right)\,,
\label{c22}
\end{multline}
где $O(1)$ ограничена по~$N$ и~$\lambda>0$, принадлежащей ограниченному множеству.}

\medskip

\noindent
Д\,о\,к\,а\,з\,а\,т\,е\,л\,ь\,с\,т\,в\,о\,.\ Применяя к правой части~(\ref{c21}) элементарное
неравенство $1-e^{-x}< x$, $x>0$, получаем~(\ref{c22}).


\medskip

\noindent
{\bf Следствие 2}.  {\it Пусть выполнены условия теоремы~3 и $q= 1$. Тогда

\noindent
\begin{multline*}
{\bf
P}(A)={}\\
{}=\exp\left(-\fr{b_{q+1}}{(q+1)!b_0}\left(\fr{b_0}{b_1}\right)^{q+1}\lambda^{q+1}\alpha\left(1+
\fr{O(1)}{N}\right)\!\right),\hspace*{-8pt}
\end{multline*}
где $O(1)$ ограничена по~$N$ и~$\lambda>0$, принадлежащей ограниченному множеству.}

\medskip

Следующее предложение верно без каких-либо условий на распределение случайной величины~$\xi$.
\medskip

\noindent
{\bf Предложение 1}.  {\it Пусть $q\ge 1$. Предположим, что ${\bf
P}\{\xi_1+\xi_2+\xi_3+\dots +\xi_N=k \,\,\,\mbox{\rm для
некоторого}\,\,\,\,k\in{\bf N'}\}\ge\beta>0$  и ${\bf p}(\xi <
q)=\delta$. Тогда
$$
{\bf P}(A)\le \fr{\delta^{nN}}{\beta^n}\,.
$$
}

\smallskip

\noindent
Д\,о\,к\,а\,з\,а\,т\,е\,л\,ь\,с\,т\,в\,о\,.\ Имеем

\end{multicols}

\hrule
\smallskip

\noindent
\begin{equation*}
{\bf P}(A)=({\bf P}(A_{i})^n\le
\left(\fr{{\bf P}\{\xi_1\le
q,\,\,\xi_2\le q, \dots \xi_N\le q\}}{{\bf
P}\{\xi_1+\xi_2+\xi_3+\dots +\xi_N=k \,\,\,\mbox{\rm для
некоторого}\,\,\,\,k\in{\bf
N'}\}}\right)^n\le\fr{\delta^{nN}}{\beta^n}.
\end{equation*}

\noindent
Предложение доказано.
\smallskip

\hrule



\begin{multicols}{2}

\bigskip

\section{Асимптотическое поведение вероятности  ${\bf P}(A)$ в случае
пуассоновской случайной величины $\xi$}

 В этой части будем предполагать, что $\xi_i$~--- независимые случайные
 величины, распределенные по закону Пуассона с параметром~$\lambda$.

\medskip

\noindent
{\bf Теорема 4}. {\it Пусть $q=1$. Предположим, что ${\bf N'}$~---
конечное множество и $m=\sup\{i, i\in{\bf N'}\}\ge 2$.  Тогда
\begin{equation*}
{\bf P}(A)=
e^{(m(m-1)/2)\alpha\left(1+O(1)/N\right)}
\end{equation*}
и величина~$O(1)$ ограничена, если~$\lambda$ принадлежит
ограниченному множеству.}

\bigskip

\noindent
Д\,о\,к\,а\,з\,а\,т\,е\,л\,ь\,с\,т\,в\,о\,.\   Имеем

\end{multicols}
\pagebreak

\hrule

\noindent
\begin{multline*}
{\bf P}(A)=({\bf P}(A_i))^n=
\left(\fr{ \sum_{k\in{\bf N'}}{\bf P}\{\xi_j\le 1, 1\le j\le N,
\xi_1+\xi_2+\xi_3+\dots +\xi_N=k\}}{\sum_{k\in{\bf N'}}{\bf
P}\{\xi_1+\xi_2+\xi_3+\dots +\xi_N=k\}}\right)^n={}\\
{}= \left(\left({\sum\limits_{k\in{\bf N'}}
\fr{N(N-1)\dots (N-k+1)}{k!}\left(e^{-\lambda}\lambda\right)^k\left(e^{-\lambda}\right)^{N-k}}\right)\left/
\sum\limits_{k\in{\bf
N'}}e^{-N\lambda} \fr{(N\lambda)^k}{k!}\right.\right)^n={}\\
{}= \left(\left(\sum\limits_{k\in{\bf N'}}
\fr{N(N-1)\dots (N-k+1)}{k!}\right)\lambda^k\left/ \sum\limits_{k\in{\bf N'}}
\fr{(N\lambda)^k}{k!}\right.\right)^{N\alpha}={}\\
{}=\left(\left(N^m\lambda^m m!\sum\limits_{k\in{\bf N'}}
\fr{N(N-1)\dots (N-k+1)m!}{k!N^m}\lambda^{k-m}\right)\left/ \left(N^m\lambda^mm!\sum\limits_{k\in{\bf N'}}
\fr{N\lambda)^{k-m}m!}{k!}\right.\right)\right)^{N\alpha}={}\\
{}=
\left(\fr{N(N-1)\dots (N-m+1)m!}{m!N^m} +\sum\limits_{k\in{\bf
N'},k<m}
\fr{N(N-1)\dots (N-k+1)m!}{k!N^m}\lambda^{k-m}\right)^{N\alpha}\left/ \left(1+\vphantom{\sum\limits_{k\in{\bf N'},k<m}
\fr{N^{k-m}\lambda^{k-m}m!}{k!}}\right.\right.{}\\
\left.{}+
\sum\limits_{k\in{\bf N'},k<m}
\fr{N^{k-m}\lambda^{k-m}m!}{k!}\right)^{N\alpha}.
\end{multline*}


Пусть $m-1\notin {\bf N'}$. Так как
\begin{equation} %multline}
\fr{N(N-1)\dots (N-m+1)m!}{m!N^m} =
\left(1-\fr{1}{N}\right)\left(1-\fr{2}{N}\right)\dots\left(1-\fr{m-1}{N}\right)=
 1-\fr{m(m-1)}{2N} +\fr{O(1)}{N^2}\,,
\label{c23}
\end{equation} %multline}
то
\begin{equation} %multline}
{\bf P}(A)={\left(1- \fr{m(m-1)}{2N} +\fr{O(1)}{N^2}\right)^{N\alpha}}\left/{\left( 1
+\fr{O(1)}{N^2}\right)^{N\alpha}}\right.=
%{}=
\exp\left(-\fr{m(m-1)}{2}\alpha\left(1+\fr{O(1)}{N}\right)\right)\,.
\label{c24}
\end{equation} %multline}
Это соответствует теореме.

Пусть $m-1\in {\bf N'}$. Используя~(\ref{c23}), получаем


\noindent
\begin{multline}
{\bf P}(A)
=\left(\fr{N(N-1)\dots (N-m+1)m!}{m!N^m} +
\fr{N(N-1)\dots (N-m+2)m!}{(m-1)!N^m}\lambda^{-1}+{}
\vphantom{\sum\limits_{k\in{\bf N'},k<m}
\fr{N^{k-m}\lambda^{k-m}m!}{k!}}
\right.\\
\left.
+\!\sum_{k\in{\bf N'},k<m-1}\!\fr{N(N-1)\dots
(N-k+1)m!}{k!N^m}\lambda^{k-m}\right)^{N\alpha}\!\!\left/\left(\vphantom{\sum\limits_{k\in{\bf N'},k<m}
\fr{N^{k-m}\lambda^{k-m}m!}{k!}}
1+\fr{N^{-1}\lambda^{-1}m!}{(m-1)!}+\right.\!\sum\limits_{k\in{\bf N'},k<m-1}\!
\fr{N^{k-m}\lambda^{k-m}m!}{k!}\right)^{N\alpha}\!=\\
{}=
\left(\left(1-\fr{m(m-1)}{2N}+\fr{\lambda^{-1}m}{N}+\fr{O(1)}{N^2}\right)\left/ \left( 
1+\fr{\lambda^{-1}m}{N}+\fr{O(1)}{N^2}\right)\right. \right)^{N\alpha}={}\\
{}=
\exp\left(-\fr{m(m-1)}{2}\alpha\left(1+\fr{O(1)}{N}\right)\right)\,.
\label{c25}
\end{multline}

\hrule

\begin{multicols}{2}


Из~(\ref{c24}) и~(\ref{c25}) вытекает теорема~5. Доказательство закончено.

\smallskip

\noindent
{\bf Теорема 5}.  {\it Предположим, что ${\bf N'}$~--- конечное
множество и $m=\sup\{i, i\in{\bf N'}\}\ge 2$.  Пусть $m>q>1$. Тогда

\noindent
\begin{equation*}
1-{\bf P}(A)\le
\fr{C_m^{q+1}\alpha}{N^{q-1}}\left(1+\fr{O(1)}{N}\right),
\end{equation*}
где $O(1)$ ограничена по~$N$ и~$\lambda>0$, принадлежащей ограниченному множеству.}

\bigskip

\noindent
Д\,о\,к\,а\,з\,а\,т\,е\,л\,ь\,с\,т\,в\,о\,.\ Пусть $m>q$. Имеем

\end{multicols}

\hrule

\noindent
\begin{multline*}
{\bf P}(A^c_{ij})=
\fr{{\bf P}\{\xi_1\ge q+1,\,\,\xi_1+\xi_2+\dots +\xi_N=k\,\,\,
\mbox{\rm для некоторго}\,\,\,\,k\in{\bf N'}\}}{{\bf
P}\{\xi_1+\xi_2+\xi_3+\dots +\xi_N=k\,\,\,\mbox{\rm для
некоторого}\,\,\,\,k\in{\bf N'}\}}={}\\[6pt]
{}
=\fr{\sum\limits_{k\in {\bf N'}}{\bf P}\{\xi_1\ge q+1,
\xi_1+\xi_2+\xi_3+\dots +\xi_N=k\}}{\sum\limits_{k\in {\bf
N'}}{\bf P}\{\xi_1+\xi_2+\xi_3+\dots +\xi_N=k\}}=
\fr{\sum\limits_{k\in {\bf N'},k>q}\sum_{i=q+1}^k{\bf
P}\{\xi_1=i, \xi_2+\xi_3+\dots +\xi_N=k-i\}}{\sum\limits_{k\in
{\bf N'}}{\bf P}\{\xi_1+\xi_2+\xi_3+\dots +\xi_N=k\}}={}\\[6pt]
{}=\left(\sum\limits_{k\in {\bf N'},k>q}\sum_{i=q+1}^ke^{-\lambda}
\fr{\lambda^i}{i!}e^{-(N-1)\lambda}\fr{((N-1)\lambda)^{k-i}}{(k-i)!}\right)\left/
\left(\sum\limits_{k\in {\bf N'}}
e^{-N\lambda}\fr{(N\lambda)^k}{k!}\right)\right.\le{}\\[6pt]
{}\le
\left((N-1)^{m-(q+1)}\lambda^m\sum\limits_{k\in {\bf N'},k>q}\lambda^{k-m}\sum_{i=q+1}^k \fr{1}{i!}\,
\fr{(N-1)^{k-i-(m-(q+1))}}{(k-i)!}\right)\left/\left(
N^m\lambda^m\sum\limits_{k\in {\bf N'}}\fr{(N\lambda)^{k-m}}{k!}\right)\right.={}\\[6pt]
{}=
\fr{1}{N^{q+1}}\fr{m!}{(q+1)!(m-(q+1))!}\left(1+\fr{O(1)}{N}\right)=\fr{1}{N^{q+1}}C_m^{q+1}\left(1+\fr{O(1)}{N}\right)\,.
\end{multline*}

Следовательно,
$$ %\begin{multline*}
1-{\bf P}(A)={\bf P}\left(\cup_{i=1}^n\cup_{j=1}^{N}A_{ij}^c\right)\le
%{}\\{}\le
\sum\limits_{i=1}^n\sum\limits_{j=1}^{N}{\bf
P}\left(A_{ij}^c\right)\le n\cdot
N\cdot\fr{C_m^{q+1}}{N^{q+1}}\left(1+\fr{O(1)}{N}\right)=
%{}\\ {}=
\fr{\alpha}{N^{q-1}}C_m^{q+1}\left(1+\fr{O(1)}{N}\right)\,.
$$%\end{multline*}
Доказательство закончено.
\smallskip

\hrule

\begin{multicols}{2}

Заметим, что если ${\bf N'}=\{m\}$~--- одноэлементное множество, то
из доказательств  теорем~4 и~5 следует, что оценки, полученные в
них, не зависят от~$\lambda$. Это  естественно, так как событие~$A_i$
является событием схемы размещения $m$~различимых частиц по
$N$~ячейкам~\cite{3ch}. В случае одноэлементного множества~${\bf N'}$
теорема~4 и теорема~5 получены в~\cite{7ch}.

{\small\frenchspacing
{%\baselineskip=10.8pt
\addcontentsline{toc}{section}{Литература}
\begin{thebibliography}{9}

\bibitem{1ch}
\Au{Новиков Ф.\,А.}
Дискретная математика для программистов. 2-е изд.~--- СПб.: Питер, 2004.

\bibitem{2ch}
\Au{Колчин В.\,Ф.}
Один класс предельных теорем для условных распределений~// Литовск. матем. сб., 1968. Т.~8(1). С.~53--63.

\bibitem{3ch}
\Au{Колчин В.\,Ф.}
Случайные графы.~--- М.: Физматгиз, 2000.

\bibitem{4ch}
\Au{Колчин А.\,В.}
Предельные теоремы для обобщенной схемы размещения~// Дискрет. матем., 2003. Т.~15(4). С.~143--157.

\bibitem{5ch}
\Au{Колчин А.\,В., Колчин~В.\,Ф.}
О переходе распределений сумм независимых одинаково распределенных случайных величин с одной
решетки на другую в обобщенной схеме размещения~// Дискрет. матем., 2006. Т.~18(4). С.~113--127.

\bibitem{6ch}
\Au{Колчин А.\,В., Колчин~В.\,Ф.}
Переход с одной решетки на другую распределений сумм случайных величин, встречающихся в
обобщенной схеме размещения~// Дискрет. матем., 2007. Т.~19(3). С.~15--21.

\label{end\stat}

\bibitem{7ch}
\Au{Avkhadiev F.\,G., Chuprunov~A.\,N.}
The probability of a successful allocation of ball groups by boxes~// Lobachevskii J.
of Math., 2007. Vol.~25. P.~3--5.
 \end{thebibliography}
}
}
\end{multicols}