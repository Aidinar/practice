
%\newcommand{\R}{{\rm I\hspace{-0.7mm}R}} %  R - числовая прямая
\newcommand{\IT}{{\bf 1}}  % индикатор
%\newcommand{\h}{{\bf H}}

\def\stat{bening}


\def\tit{О МОЩНОСТИ КРИТЕРИЕВ В СЛУЧАЕ ОБОБЩЕННОГО РАСПРЕДЕЛЕНИЯ ЛАПЛАСА$^*$}

\def\titkol{О мощности критериев в случае обобщенного распределения Лапласа}

\def\autkol{В.\,Е.~Бенинг, О.\,О.~Лямин}
\def\aut{В.\,Е.~Бенинг$^1$, О.\,О.~Лямин$^2$}

\titel{\tit}{\aut}{\autkol}{\titkol}

{\renewcommand{\thefootnote}{\fnsymbol{footnote}}\footnotetext[1]
{Работа выполнена при финансовой поддержке РФФИ,
проекты 08-01-00567 и 08-07-00152.}}

\renewcommand{\thefootnote}{\arabic{footnote}}
\footnotetext[1]{Факультет
вычислительной математики и кибернетики Московского государственного
университета им. М.~В.~Ломоносова, bening@yandex.ru}
\footnotetext[2]{Факультет
вычислительной математики и кибернетики Московского государственного
университета им. М.~В.~Ломоносова}


\Abst{В работе на эвристическом уровне получена формула для
предела отклонения мощности асимптотически оптимального критерия
от мощности наилучшего критерия в случае обобщенного распределения
Лапласа. Это отклонение в силу нерегулярности этого распределения
имеет порядок~$n^{-1/2}$, в отличие от обычных регулярных
семейств, для которых этот порядок равен~$n^{-1}$.}

\KW{обобщенное распределение Лапласа; функция мощности; дефект;
асимптотическое разложение}

   \vskip 30pt plus 9pt minus 6pt

      \thispagestyle{headings}

      \begin{multicols}{2}

      \label{st\stat}
      
\section{Введение}

Хорошо известно, что распределение Лапласа находит широкое
применение при математическом моделировании многих процессов в
телекоммуникационных системах, в экономике, финансовом деле,
технике и других областях, например в задачах выделения полезного
сигнала на фоне помех (см., например, работы~[1--4]). 
Естественность возникновения этого распределения обоснована
в работе~\cite{9be}. Назовем обобщенным распределением Лапласа
распределение на действительной прямой с плотностью вида
\begin{multline}
p\left(x, \theta\right) = C \left(a,b\right)e^{-a(x -
\theta)^{2} -b\left|x - \theta\right|}\,,\\
a \ge 0\,, b> 0\,,  x\in {\sf R}^{1}\,, 
\label{e1.1}
\end{multline}
где $C\left(a,b\right)$~--- нормировочная константа такая,
что
$$
C\left(a,b\right)=
\begin{cases}
\fr{b}{2}\,, & a = 0\,; \\ 
\fr{\sqrt{a}}{\sqrt{\pi }
\exp\{b^{2} /4a\}{\mathrm{erfc}}\left(b/(2\sqrt{a})\right)}& a> 0 
\end{cases}
$$
и
$$
{\mathrm{erfc}}\left(x\right)=\fr{2}{\sqrt{\pi } }
\int\limits_{x}^{\infty }e^{-y^{2}}\,dy\,.
$$
Обычное распредление Лапласа получается при $a=0$. Обобщенное
распределение Лапласа может оказаться полезным в тех случаях,
когда необходим более тонкий контроль за поведением функции
плотности, чем может быть предоставлен однопараметрическими по
параметру формы нормальным и лапласовским распределениями. Так,
правильно подобрав параметры, можно получить распределение с
хвостами менее тяжелыми, чем у соответствующего распределения
Лапласа, при этом сохранив существенную особенность негладкости
функции плотности, которая отсутствует у нормального распределения.

Рассмотрим задачу проверки простой гипотезы
$$
{\sf H}_0:\theta = 0
$$
против последовательности близких сложных альтернатив вида
$$
{\sf H}_{n,1}:\theta = \fr{t}{\sqrt{n}}\,, \quad
 0 < t \le C\,,\  C > 0\,,
$$
на основе выборки $(X_1,\ldots,X_n)$~--- независисых одинаково
распределенных наблюдений, имеющих распределение с плотностью вида~(\ref{e1.1}). 
Для любого фиксированного $t \in (0,C]$ наилучший
критерий всегда существует согласно фундаментальной лемме
Неймана--Пирсона и основан на логарифме отношения правдоподобия~$\Lambda_n(t)$
\begin{align}
\Lambda_n(\theta) &=\sum\limits_{i=1}^{n}\left(l(X_i,\theta) - l(X_i,0)\right)\,; 
\label{e1.2}
\\
\Lambda_n(t) &\equiv\Lambda_n(t n^{-1/2})\,,\notag
\end{align}
где $l(x,\theta)=\log p(x,\theta)$, $\theta>0$.

Обозначим через $\beta_n^*(t)$ мощность такого критерия уровня
$\alpha\in (0,1)$. Заметим, что, поскольку~$t$ неизвестно, мы
не можем использовать статистику~$\Lambda_n(t)$ для построения
критерия проверки гипотезы~${\sf H}_0$ против альтернативы ${\sf
H}_{n,1}$. Однако~$\beta_n^*(t)$ дает верхнюю границу для мощности
любого критерия при проверке~${\sf H}_0$ против фиксированной альтернативы
$$
{\sf H}_{n,t}: \theta = \fr{t}{\sqrt{n}}\,,\quad  t > 0\,,
$$
и может служить стандартом при сравнении различных критериев.

В работе~\cite{1be} было показано, что в  случае регулярного семейства плотностей
справедливо соотношение
$$
\beta_n^*(t) \to \beta^*(t) = \Phi(t\sqrt{I} - u_\alpha)\,,
$$
где $\Phi(x)$~--- функция распределения стандартного нормального закона и
$\Phi(u_\alpha) = 1 - \alpha$, $\alpha \in (0, 1)$, $I$~--- фишеровская информация.
Там же было показано, что для проверки~${\sf H}_0$
против~${\sf H}_{n,1}$ существуют критерии, основанные
на статистиках, отличных от~$\Lambda_n(t)$, и имеющие
ту же предельную мощность~$\beta^*(t)$.
Такие критерии называются асимптотически наиболее мощными (АНМ).
Среди таких критериев существуют критерии, основанные на
статистиках, которые не зависят
от неизвестного параметра~$t$, и поэтому могут быть использованы
при проверке гипотезы ${\sf H}_0$ против сложной альтернативы ${\sf H}_{n,1}$.
При этом мощности таких критериев типичным образом  отличаются от~$\beta_n^*(t)$ на
величину порядка~$n^{-1}$~\cite{1be}.

В этой статье для распределения~(\ref{e1.1})  рассмотрим  АНМ критерий, основанный
на статистике вида
\begin{equation*}
T_{n} =\fr{1}{\sqrt{n} }\, \sum_{i=1}^{n}\left[2a X_{i} + b\, \mathrm{sign}\left(X_{i} \right)\right]\,. 
%\label{e1.3}
\end{equation*}
Обозначим через~$\beta_n(t)$ мощность этого критерия
уровня $\alpha\in(0,1)$,  через $I_{a,b}$~---
фишеровскую информацию распределения~(\ref{e1.1}) и пусть $\varphi(x)$~---
плотность стандартного нормального распределения.

Заметим, что семейство~(\ref{e1.1}) не является регулярным, поскольку у $p(x, \theta)$ 
не существует производной по~$\theta$ в точке $\theta = x$.
Покажем, что это отсутствие регулярности выражается в нарушении
естественного порядка~$n^{-1}$ разности $\beta_n^*(t) - \beta_n(t)$
и приводит к порядку~$n^{-1/2}$.

Далее всюду предполагается $a>0$. Случай   $a = 0$
подробно рассмотрен в работе~\cite{6be}. Цель настоящей статьи получить
результаты, аналогичные случаю распределения Лапласа (см.~\cite{6be}), для
случая обобщенного распределения Лапласа~(\ref{e1.1}).

\section{Асимптотическое поведение логарифма отношения
правдоподобия}

В этом разделе доказываются результаты, описывающие асимптотическое
($n\to\infty$) поведение логарифма отношения правдоподобия
$\Lambda_{n}\left(t\right)$ как при
гипотезе~ ${\sf H}_{0}$, так и при альтернативе~${\sf H}_{n,t}$.

\smallskip

\noindent
\textbf{Лемма 2.1.}{\it В случае распределения}~(\ref{e1.1}) {\it справедливы
следующие соотношения
\begin{align*}
I_{a,b} &=2\left(b C\left(a,b\right)+a\right)\,;\\
{\cal L}\left(\Lambda_{n} \left(t\right)|{\sf H}_{0} \right) &\to
{\cal N}\left(-\fr{1}{2} t^{2} I_{a,b},t^{2} I_{a,b}\right)\,;\\
{\cal L}\left(\Lambda_{n} \left(t\right)|{\sf H}_{n,t} \right) &\to
{\cal N}\left(\fr{1}{2} t^{2} I_{a,b} ,t^{2} I_{a,b}\right)\,,\quad n\to\infty\,.
\end{align*}
}

\smallskip

\noindent
Д\,о\,к\,а\,з\,а\,т\,е\,л\,ь\,с\,т\,в\,о\,.\
В силу свойств интеграла Лебега имеем
\begin{multline*}
I_{a,b} =\int\limits_{{\sf R}^{1}\backslash \{ 0\}}
p\left(x\right)\left(\fr{p'\left(x\right)}{p\left(x\right)}
\right) ^{2}\,dx={}\\
{}=2C\left(a,b\right)\int\limits_{0}^{\infty}
\left(2a x+b\right)^{2}e^{-ax^{2} -bx}\,dx={}\\
{}=
2\left(bC\left(a,b\right)+a\right)\,. 
%\label{e2.1}
\end{multline*}
Отсюда следует, что
\begin{equation}
C\left(a,b\right)=\fr{I_{a,b}-2a}{2b}\,. 
\label{e2.2}
\end{equation}
Рассмотрим логарифм отношения правдоподобия $\Lambda_{n}\left(\theta \right)$,
$\theta>0$ более подробно. Из определения~(\ref{e1.2}) следует, что
\begin{multline}
\Lambda_{n} \left(\theta \right)=
\sum_{i=1}^{n}\left(l\left(X_{i},\theta\right)-
l\left(X_{i},0\right)\right)={}\\
{}
=\sum_{i=1}^{n}\left[a \theta\left(2X_{i}
-\theta \right)+b \left(\left|X_{i} \right|-\left|X_{i}-\theta\right|\right)\right]
\equiv{}\\
{}\equiv \sum_{i=1}^{n}Z_{i}\left(\theta \right)\,, 
\label{e2.3}
\end{multline}
где
\begin{multline}
Z_{i} \left(\theta \right)=\\
=\!\begin{cases}
a\theta\left(2X_{i}-\theta \right)+b \theta\,,  & \!\!\!\!\!\!\!\!\!\!\!\!\!\!\!\!X_{i}>\theta\,;\\
a\theta \left(2X_{i}-\theta\right)+b\left(2X_{i}-\theta\right)={} &\\
{}= a\theta\left(2X_{i} -\theta\right)+2bX_{i}
{\IT}_{\left[0,\theta\right]}\left(X_{i}\right)+{} &\\
\ \ \ \ {}+2b\theta
{\IT}_{\left(\theta,\infty\right)}\left(X_{i}\right)- b\theta\,, &\!\!\!\!\!\!\!\!\!\! \!\!\!\!\!\!0\le X_{i}\le\theta\,;\\
a\theta\left(2X_{i}-\theta\right)-b\theta\,,  &\!\!\!\!\!\!\!\!\!\!\!\!\!\!\!\! X_{i}<0\,.
\end{cases}\!
\label{e2.4}
\end{multline}
Найдем характеристические функции случайной величины~$Z_{1}\left(\theta\right)$ 
при распределениях~${\p}_{0} $ и~${\p}_{\theta}$. Для распределения~${\p}_{0} $ имеем
\begin{multline}
f_{0} \left(s\right)\equiv
{\e}_{0}e^{isZ_{1}\left(\theta\right)}={}\\
{}= \int\limits_{-\infty}^{0}
e^{is\left(a\theta\left(2x-\theta\right)-b\theta\right)}
C\left(a,b\right)e^{-a x^{2}+bx}\,dx+{}\\
{}
+\int\limits_{0}^{\theta }e^{is\left(a\theta\left(2x
-\theta\right)+b\left(2x-\theta\right)\right)}
C\left(a,b\right)e^{-ax^{2}-bx}\,dx+{}\\
{}+
\int\limits_{\theta }^{\infty }e^{is\left(a\theta
\left(2x-\theta\right)+b\theta \right)}C\left(a,b\right)
e^{-ax^{2}-bx}\,dx={}\\
{}
=C\left(a,b\right)e^{-is\theta\left(a\theta+b\right)}
\int\limits_{-\infty }^{0}e^{-a x^{2}
+\left(2isa\theta+b\right)x}\,dx+{}\\
{}+C\left(a,b\right)
e^{-is\theta\left(a \theta+b\right)}\int\limits_{0}^{\theta }
e^{-a x^{2}+\left(2is\left(a\theta + b
\right)-b\right)x}\,dx+{}\\
{}
+C\left(a,b\right) e^{-is\theta\left(a\theta - b\right)}
\int\limits_{\theta }^{\infty } e^{-a x^{2}
+ \left(2isa\theta - b\right)x}\,dx\,.  
\label{e2.5}
\end{multline}
Далее нам понадобятся следующие соотношения:
\begin{multline}
\int\limits_{-\infty }^{0}e^{-a x^{2}+\left(\omega+b\right)x}\,dx={}\\
{}=
\fr{\sqrt{\pi }e^{\left(\omega+b\right)^{2}/(4a)}
{\mathrm{erfc}}\left((\omega+b)/(2\sqrt{a})\right)}{2\sqrt{a}}\,; 
\label{e2.6}
\end{multline}
\vspace*{-6pt}

\noindent
\begin{multline}
\int\limits_{0}^{\theta }e^{-a x^{2}+\left(\delta-b\right)x}\,dx ={}\\
{}=
\sqrt{\pi}e^{\left(\delta-b\right)^{2}/(4a)}
\left[{\mathrm{erfc}}\left(\fr{\left(\delta
-b\right)-2a\theta }{{\rm 2}\sqrt{a}}\right)-{}\right.\\
\left.{}-{\mathrm{erfc}}\left(\fr{\delta-b}{{\rm 2}\sqrt{a}}
\right)\right]\Bigg /(2\sqrt{a})\,, \label{e2.7}
\end{multline}
\vspace*{-6pt}

\noindent
\begin{multline}
\int\limits_{\theta }^{\infty }e^{-ax^{2}+\left(\sigma
-b\right)x}\,dx ={}\\
{}=
\fr{\sqrt{\pi}e^{\left(\sigma-
b\right)^{2}/(4a)}{\mathrm{erfc}}\left((2a\theta
-\left(\sigma-b\right))/(2\sqrt{a})
\right)}{2\sqrt{a}}\,, \label{e2.8}
\end{multline}
где $\omega$, $\delta$, $\sigma $~--- некоторые константы (в том числе комплексные), а также равенства

\noindent
\begin{equation}
\left.
\begin{array}{rl}
\mathrm{erf}{\rm c}'\left(x\right)&=-\fr{2e^{-x^{2}}}{\sqrt{\pi}}\,;\\
\mathrm{erf}{\rm c}''\left(x\right) &=\fr{4xe^{-x^{2}}}{\sqrt{\pi}}\,;\\
\mathrm{erf}{\rm c}'''\left(x\right)&=\fr{4xe^{-x^{2}}
\left(1-2x^{2}\right)}{\sqrt{\pi}}\,. 
\end{array}
\right \}
\label{e2.9}
\end{equation}
Из соотношений~(\ref{e2.5})--(\ref{e2.8}) и~(\ref{e2.9}) с $\omega=2isa\theta$,
$\delta=2is\left(a\theta + b\right)$,
$\sigma=2isa\theta$ получаем, что  в случае близких альтернатив
$\theta=\theta_{n} \to 0$ при каждом фиксированном
$s$ справедливы равенства
\begin{align*}
f_{0}\left(s\right)&=1+\fr{is\left(is-1\right)I_{a,b}}{2}
\,\theta_{n}^{2}+{\it o}\left(\theta_{n}^{2} \right)\,; %\label{e2.10}
\\[2pt]
{\e}_{0} Z_{1}\left(\theta_{n}\right) &=
-\fr{I_{a,b}}{2}\,\theta_{n}^{2}+{\it o} \left(\theta_{n}^{2} \right)\,; %\label{e2.11}
\\[2pt]
{\D}_{0}Z_{1}\left(\theta_{n} \right) &=
I_{a,b}\theta_{n}^{2}+{\it o}\left(\theta_{n}^{2} \right)\,. %\label{e2.12}
\end{align*}
Аналогично находя характеристическую функцию случайной величины~$Z_{1}\left(\theta \right)$ в
случае распределения~${\p}_{\theta}$, получим, что в случае близких
альтернатив $\theta =\theta_{n} \to 0$ при каждом фиксированном
$s$ справедливы равенства
\begin{align*}
f_{\theta_{n}}\left(s\right) &=1+\fr{is\left(is+1\right)
I_{a,b}}{2}\,\theta_{n}^{2}+{\it o}\left(\theta_{n}^{2}\right)\,; %\label{e2.13}
\\[2pt]
{\e}_{\theta_{n} }Z_{1}\left(\theta_{n}\right) &=
\fr{I_{a,b}}{2}\,\theta_{n}^{2}+{\it o}\left(\theta_{n}^{2}\right)\,; %\label{e2.14}
\\[2pt]
{\D}_{\theta_{n}}Z_{1}
\left(\theta_{n}\right) &=I_{a,b}\theta_{n}^{2}
+{\it o}\left(\theta_{n}^{2}\right)\,. %\label{e2.15}
\end{align*}
Теперь, если $\theta_{n}=t/\sqrt{n}$, то для любого
фиксированного $s$ имеем
\begin{multline*}
{\e}_{n,0}e^{is\Lambda_{n}\left(t\right)}=f_{0}^{n}
\left(s\right)=\left(1+\fr{is\left(is-1\right)t^{2}I_{a,b}}{2n}\;+\right.{}\\[2pt]
\!\!\!\!\!\!\left.{}+ {\it o}\left(\fr{1}{n} \right)\right)^{n}\to
e^{-s^{2} t^{2} I_{a,b}/2-ist^{2}I_{a,b}/2}\,,\  n\to \infty\,. 
%\label{e2.16}
\end{multline*}
Аналогично
\begin{multline*}
{\e}_{n,t/\sqrt{n} }e^{is\Lambda_{n}\left(t\right)}
=f_{t/\sqrt{n}}^{n}\left(s\right)\to{}\\[2pt]
{}\to
e^{-s^{2} t^{2}
I_{a,b}/2+ist^{2}I_{a,b}/2}\,,\quad n\to \infty\,.
%\label{e2.17}
\end{multline*}
Отсюда и из теоремы непрерывности (см., например,~[8, теорема~7.3.2] следует утверждение 
лем\-мы.~$\Box$

Из  этой леммы и работы~\cite{6be} (см.\ введение) непосредственно получаем


\smallskip

\noindent
\textbf{Следствие 2.1.}\ {\it В случае распределения}~(\ref{e1.1}) {\it справедливо соотношение
$$
\beta_n^*(t)\to\beta^*(t)=
\Phi\left(t\sqrt{I_{a,b}}-u_\alpha\right)\,, \quad n \to\infty\,.
$$
}


\section{Формула для предельного отклонения мощностей}

В этом разделе на эвристическом уровне будет показано, что
справедлива формула для предельного отклонения разностей мощностей
\begin{multline}
r(t)=\lim_{n \to \infty}{\sqrt n\left(\beta_n^*(t)-\beta_n(t)\right)}={}\\
{}=
\fr{(I_{a,b}-2a)bt^2}{3\sqrt{I_{a,b}}}\,
\varphi\left(t\sqrt{I_{a,b}}-u_{\alpha}\right)\,,
\label{e3.1}
\end{multline}
где~$\beta_n(t)$~--- функция мощности АНМ критерия, основанного на
знаковой статистике
\begin{equation}
T_{n} =\fr{1}{\sqrt{n}}\,\sum_{i=1}^{n}\left[2aX_{i}+b
\mathrm{sign}\left(X_{i} \right)\right]\,. \label{e3.2}
\end{equation}
Формула~(\ref{e3.1}) показывает, что отсутствие регулярности у
распределения~(\ref{e1.1}) приводит к нарушению естественного порядка
разности $\beta_n^*(t)-\beta_n(t)$,
равного~$n^{-1}$ (см.~\cite{1be}). Из формулы~(\ref{e3.1})  следует, что этот порядок равен~$n^{-1/2}$.

Получим сначала стохастическое разложение для~$\Lambda_n(t)$. Имеем
(см.~(\ref{e2.3}), (\ref{e2.4})),
\begin{multline*}
\Lambda_{n}\left(t\right)=
a\fr{t}{\sqrt{n}}\,\sum_{i=1}^{n}\left(
2X_{i}-\fr{t}{\sqrt{n}}\right)+{}\\
{}+2b\sum_{i=1}^{n}X_{i} {\IT}_{\left[0,t/\sqrt{n}\right]}
\left(X_{i}\right)+{}\\
{}+2b\fr{t}{\sqrt{n}}\,\sum_{i=1}^{n}{\IT}_{\left(t/\sqrt{n},\infty\right)}
\left(X_{i}\right)-b\fr{t}{\sqrt{n}}={}\\
{}
=2a\fr{t}{\sqrt{n}}\sum_{i=1}^{n}X_{i}-at^{2}+2b\sum_{i=1}^{n}X_{i} {\IT}_{\left[0,t/\sqrt{n}
\right]}\left(X_{i}\right)+{}\\
{}+2b\fr{t}{\sqrt{n}}\,\sum_{i=1}^{n}{\IT}_{\left(0,\infty \right)}
\left(X_{i}\right)-2b\fr{t}{\sqrt{n}}\,\sum_{i=1}^{n}{\IT}_{\left[0,t/\sqrt{n}
\right]}\left(X_{i}\right)-{}\\
{}
-b\fr{t}{\sqrt{n}}\sum_{i=1}^{n}{\IT}_{\left(0,\infty
\right)}\left(X_{i}\right)-b\fr{t}{\sqrt{n}}\sum_{i=1}^{n}{\IT}_{(-\infty ,0]}\left(X_{i}\right)={}\\
{}
=2a\fr{t}{\sqrt{n}}\sum_{i=1}^{n}X_{i}-at^{2}+{}\\
{}+2b\sum_{i=1}^{n}\left(X_{i}-\fr{t}{\sqrt{n}}
\right){\IT}_{\left[0,t/\sqrt{n}\right]}\left(X_{i}
\right)+{}\\
{}
+b\fr{t}{\sqrt{n}}\sum_{i=1}^{n}\mathrm{sign}\left(X_{i}
\right)+b\fr{t}{\sqrt{n}}\sum_{i=1}^{n}{\IT}_{\{ 0\}}
\left(X_{i} \right)\,.
\end{multline*}
Поскольку распределение~$X_i$ непрерывно, то
$$
{\p}_{n,\theta}\left(\sum_{i=1}^{n}{\IT}_{\{ 0\}}\left(X_{i}
\right)>0\right)=0\,,\quad \theta >0\,,
$$
и поэтому почти всюду справедливо представление

\noindent
\begin{multline}
\Lambda_{n} \left(t\right)=2a\fr{t}{\sqrt{n}}\,\sum_{i=1}^{n}X_{i}-at^{2}+{}\\
{}+
2b\sum_{i=1}^{n}\left(X_{i}
-\fr{t}{\sqrt{n}}\right){\IT}_{\left[0,t/\sqrt{n}
\right]}\left(X_{i}\right)+{}\\
{}+
b\fr{t}{\sqrt{n}}\sum_{i=1}^{n}{\mathrm{sign}}\left(X_{i}\right)={}\\
{}=\fr{t}{\sqrt{n}}\sum_{i=1}^{n}\left[2aX_{i}+b\,{\mathrm{sign}}
\left(X_{i}\right)\right]-at^{2}+{}\\
{}+2b\sum_{i=1}^{n}\left(X_{i}-\fr{t}{\sqrt{n}}
\right){\IT}_{\left[0,t/\sqrt{n}\right]}\left(X_{i}
\right). \label{e3.3}
\end{multline}
Рассмотрим следующее монотонное (не меняющее мощность критерия)
преобразование ($t > 0$) статистики  $T_n$ (см.~(\ref{e3.2}))
$$
S_{n} \left(t\right)=tT_{n} -\fr{1}{2}I_{a,b}t^{2}\,.
$$
Тогда
\begin{multline}
\Delta_{n}\left(t\right)\equiv S_{n}\left(t\right)-\Lambda_{n}
\left(t\right)=at^{2}-\fr{1}{2} I_{a,b}t^{2}-{}\\
{}
-2b\sum_{i=1}^{n}\left(X_{i}-\fr{t}{\sqrt{n}}
\right){\IT}_{\left[0,t/\sqrt{n}\right]}\left(X_{i}
\right)\,. \label{e3.4}
\end{multline}


\smallskip

\noindent
\textbf{Лемма 3.1.} {\it В случае распределения}~(\ref{e1.1}) {\it справедливы
следующие соотношения:
$$
{\cal L}\left(\sqrt[{4}]{n}\Delta_{n}\left(t\right)|{\sf H}_{0}
\right)\to {\cal N}\left(0,\fr{2}{3}b \left(I_{a,b} - 2a\right)t^{3} \right)\,;
$$
\begin{multline*}
{\cal L}\left(\left(\sqrt[{4}]{n}\Delta_{n}\left(t\right),\Lambda_{n}
\left(t\right)\right)|{\sf H}_{0}\right) \to{}\\
{}\to
 {\cal N}_{2}
\left(0, \,\fr{2}{3}\,b\left(I_{a,b} - 2a
\right)t^{3},\,0,\,-\fr{1}{2}\,I_{a,b}t^{2},\,I_{a,b}t^{2}\right)\,,
\end{multline*}
где ${\cal N}_2$~--- двумерный нормальный закон с соответствующими параметрами.}

\smallskip

\noindent
Д\,о\,к\,а\,з\,а\,т\,е\,л\,ь\,с\,т\,в\,о\,.\ Докажем первое соотношение методом
характеристических функций. С этой целью найдем характеристическую
функцию случайной величины
$$
\left(X_{1} - \theta \right){\IT}_{\left[0,\,\theta\right]}\left(X_{1}\right)\,,\quad \theta >0\,.
$$
Имеем
\begin{multline*}
g_{\theta}\left(s\right) \equiv {\e}_{0} 
e^{is\left(X_{1} - \theta \right){\IT}_{\left[0,\,\theta\right]} \left(X_{1}\right)} =
{\p}_{0} \left(X_{1} <0\right)+{}\\
{}+
 C\left(a,b\right) e^{-is\theta} \int\limits_{0}^{\theta }
e^{-ax^{2}+\left(is-b\right)x}\,dx +{} %\\
\end{multline*}
\begin{multline*}
{}+ {\p}_{0}\left(X_{1} > \theta\right) ={}\\
{}
=\fr{1}{2}+ C\left(a,b\right) e^{-is\theta}
\int\limits_{0}^{\theta} e^{-ax^{2}+\left(is-b\right)x}\,dx
+{}\\
{}+ 
C\left(a,b\right)\int\limits_{\theta}^{\infty} e^{-ax^{2}-bx}\,dx\,.
\end{multline*}
Используя соотношения~(\ref{e2.7}) с $\delta= is$, (\ref{e2.8}) с
$\sigma= 0$ и~(\ref{e2.9}), получим, что
если $\theta = \theta_{n} \to 0$, то при каж\-дом фиксированном~$s$
справедливо представление
\begin{multline}
g_{\theta_n}\left(s\right) = 1 - is \fr{C\left(a,b\right)}{2}\,
\theta_{n}^{2} +{}\\
{}+
 is\left(is + b\right)
\fr{C\left(a,b\right)}{6}\theta_{n}^{3}+{\it o}
\left(\theta_{n}^{3}\right)\,, \label{e3.6}
\end{multline}
поэтому с учетом~(\ref{e2.2}) имеем
\begin{multline}
\!\!\!\!\!\!{\e}_{n,0} e^{is\sqrt[{4}]{n}\Delta_{n}\left(t\right)}
= e^{is\sqrt[{4}]{n}t^{2}\left(a - I_{a,b}/2\right)}g_{t/\sqrt{n}}^{n}\left(-2b\sqrt[{4}]{n}
s\right)={}\\
\!{}
=\exp\!\left[ -is\sqrt[{4}]{n} t^{2}bC\!\left(a,b\right)
+ n\log\left(1 + is\fr{bC\left(a,b\right)t^{2}}
{n^{3/4}}+\right.\right.{}\hspace*{-0.785pt}\\
{}\left.\left.+ \fr{b^{2}\left(-4\sqrt{n} s^{2}-2\sqrt[{4}]{n}is\right)C\left(a,b\right)t^{3}}
{6n^{3/2}}+{\it o}\left(\fr{1}{n}\right)\right)\right]={}\\
{}
=\exp\left(-\fr{2b^{2}C\left(a,b\right)t^{3}
s^{2} }{3}+{\it o}\left(\fr{1}{\sqrt[{4}]{n}}\right)\right)\to{}\\[6pt]
{}\to
e^{-2b^{2}C\left(a,b\right)t^{3} s^{2} /3}\,. \label{e3.7}
\end{multline}
Отсюда следует  первое утверждение леммы.

Докажем второе утверждение леммы также методом характеристических функций,
хотя его можно доказать и с помощью двумерной центральной предельной
теоремы для схемы серий (см., например,~\cite{2be}, теорема~8.7.11). С этой
целью найдем двумерную характеристическую функцию  случайных величин
(см.~(\ref{e3.3}), (\ref{e3.4}))
$$
\left(\left(X_{1}-\theta \right){\IT}_{\left[0,\,\theta\right]}\left(X_{1}\right),Z_{1}
\left(\theta\right)\right)\,,\quad \theta>0\,;
$$
\begin{multline*}
g_{\theta}\left(u,v\right)\equiv{}\\
{}\equiv {\e}_{0} \exp\left(iu\left(X_{1}
-\theta\right){\IT}_{\left[0,\theta\right]}\left(X_{1}
\right)+ivZ_{1}\left(\theta\right)\right)={}\\
{}
= C\left(a,b\right) e^{-iv\theta\left(a\theta + b\right)} 
\int\limits_{-\infty }^{0} e^{-ax^{2} + \left(2iva\theta + b\right)x}\,dx +{}\\
{}+ 
C\left(a,b\right)e^{-iu\theta - iv\theta\left(a\theta + 
b\right)}\times{}\\
{}\times \int\limits_{0}^{\theta } e^{-ax^{2} + \left(iu + 2iv\left(a\theta + b\right) - b\right)x}\,
dx +{}\\
{}
+ C\left(a,b\right) e^{-iv\theta\left(a\theta 
- b\right)} \int\limits_{\theta }^{\infty } 
e^{-ax^{2} + \left(2iva\theta - b\right)x}\,dx\,.
\end{multline*}
Используя опять  формулы~(\ref{e2.6})--(\ref{e2.8})
с со\-от\-вет\-ст\-ву\-ющи\-ми значениями параметров, а также соотношение~(\ref{e2.9}),
получим, что  при $\theta=\theta_{n}\to 0$ и
фиксированных~$u$ и~$v$ имеем асимптотическое\linebreak пред\-став\-ле\-ние:
\begin{multline*}
g_{\theta_{n}}\left(u,v\right)=1+\left(
\vphantom{\fr{C}{2}}
aiv\left(iv-1\right)
+{}\right.\\
{}+\left.\fr{C\left(a,b\right)\left(-iu+ 2 b i v\left(
iv-1\right)\right)}{2}\right)\theta_{n}^{2}+{}\\
{}
+\fr{C\left(a,b\right)\left(ibu-u^{2}
+ 3b^{2}v\left(i + v\right)\right)}{6}\,\theta_{n}^{3} +{\it o}
\left(\theta_{n}^{3}\right)={}\\
{}
=1 + iv\left(iv - 1\right)\bigl(bC\left(a,b\right) + a\bigr)
\theta_{n}^{2} -{}\\
{}- \fr{iuC\left(a,b\right)}{2}\,
\theta_{n}^{2}+ \fr{ibuC\left(a,b\right)}{6}\,
\theta_{n}^{3} - \fr{u^{2} C\left(a,b\right)}{6}\,\theta_{n}^{3}-{}\\
{}- 
\fr{iv\left(iv - 1\right)b^{2}C\left(a,b\right)}{2}\,
\theta_{n}^{3} + {\it o}\left(\theta_{n}^{3} \right) ={}\\
{}= 
1 + iv\left(iv - 1\right)\fr{I_{a,b}}{2}\,
\theta_{n}^{2} - \fr{iuC\left(a,b\right)}{2}\,\theta_{n}^{2}+{}\\
{}+
 \fr{ibuC\left(a,b\right)}{6}\,
\theta_{n}^{3}- \fr{u^{2}C\left(a,b\right)}{6}\,\theta_{n}^{3} -{}\\
{}-
 \fr{iv\left(iv - 1\right)b^{2} C\left(a,b\right)}{2}\,
\theta_{n}^{3} + {\it o}\left(\theta_{n}^{3}\right)\,.
%\label{e3.9}
\end{multline*}
Теперь (см.~(\ref{e3.3}), (\ref{e3.4}))
\begin{multline*}
{\e}_{n,0} \exp\left(iu\sqrt[{4}]{n} \Delta_{n}
\left(t\right) + iv\Lambda_{n}
\left(t\right)\right) ={}\\
{}=
 e^{iu\sqrt[{4}]{n}t^{2}\left(a
-I_{a,b}/2\right)} g_{t/\sqrt{n}}^{n}\left(-2b
\sqrt[{4}]{n}u, v\right) ={}\\
{}
= \exp\left(\fr{I_{a,b} t^{2} }{2}\left(-iv - v^{2}\right) -
\fr{2b^{2}C\left(a,b\right)t^{3}u^{2}}{3}+{}\right.\\
{}+\left. {\it o}
\left(\fr{1}{\sqrt[{4}]{n}}\right)\right)\to
\exp \left(-iv\fr{I_{a,b}t^{2}}{2} - v^{2}\fr{I_{a,b}t^{2}}{2} -{}\right.\\
{}-\left.u^{2}
\fr{2b^{2}C\left(a,b\right)t^{3}}{3} \right)\,.
\end{multline*}
Теперь  из~(\ref{e2.2}) и  многомерной теоремы непрерывности (см.~\cite{2be},
теорема~7.6.2А) следует утверждение лем\-мы.~$\Box$

Из этой леммы следует, что случайные величины $\sqrt[4] n
\Delta_n(t)$ и~$\Lambda_n(t)$ асимптотически независимы, и поэтому
формула для~$r(t)$ приобретает вид (см.~(\ref{e3.1}) и работы~\cite{1be, 5be, 6be}):
\begin{multline}
r(t) \equiv \lim_{n \to \infty} {\sqrt
n\left(\beta_n^*(t)-\beta_n(t)\right)}={}\\
{}
=\fr{1}{2t\sqrt{I_{a,b}}}\varphi\left(t\sqrt{I_{a,b}}-u_{\alpha}\right)
\D(\Delta(t)|\Lambda(t)=c_t) ={}\\
{}=
\fr{1}{2t\sqrt{I_{a,b}}}\varphi\left(t\sqrt{I_{a,b}}-u_{\alpha}\right)
\D \Delta(t)={}\\
{}
= \fr{(I_{a,b} - 2a)bt^2}{3\sqrt{I_{a,b}}}
\varphi\left (t\sqrt{I_{a,b}} - u_{\alpha}\right )\,, 
\label{e3.10}
\end{multline}
где $\Lambda(t)$, $\Delta(t)$~--- независимые нормальные случайные
величины, распределенные  соответственно с параметрами
$(-(1/2)I_{a,b}t^2,\,I_{a,b}t^2)$, $(0,\,
(2/3)(I_{a,b}-$\linebreak $-\;2a)bt^3)$ и
$c_t=t\sqrt{I_{a,b}}u_{\alpha}-(1/2)t^2 I_{a,b}$.

Формула~(\ref{e3.10}) была получена с помощью общей теоремы~3.2.1
из работы~\cite{1be} (см.\ также~\cite{6be}, введение).
Формальное доказательство полученной формулы, состоящее в
проверке условий этой теоремы, не является целью этой статьи и будет приведено
в другой статье.
В этой работе будет доказано более слабое утверждение,
составляющее содержание следующей леммы.

\smallskip

\noindent
\textbf{Лемма 3.2.} {\it В случае распределения}~(\ref{e1.1}) {\it для
любого $0 \le \delta < 1/2$ справедливо соотношение
$$
n^{\delta}\left(\beta_n^*(t)-\beta_n(t)\right) \to  0\,,\ \ 
 n \to \infty\,,
  0 < t\le C\,,   C>0\,.
$$
}

\smallskip

\noindent
Д\,о\,к\,а\,з\,а\,т\,е\,л\,ь\,с\,т\,в\,о\,.
Для доказательства применим общую теорему из работы~\cite{3be}. Условия
этой теоремы в нашем случае сводятся к проверке следующих
соотношений:

существует константа $A>0$ такая, что для любого
$x_0\in {\sf R}^1$, любого $\gamma > 0$ и любого $t \in (0,\,C]$,
$C > 0$ справедливы равенства
\begin{align}
\!\!\!\sup_{x\le x_0} {\p}_{n,0}\left(x - n^{-\delta/2} \le \Lambda_n(t)  \le
x\right ) & = O(n^{-\delta/2})\,;\notag\\
\!\!\!\!\!\!{\e}_{n,0}|\Delta_n(t)| {\IT_{(\gamma n^{-\delta/2},A)}(|\Delta_n(t)|)}& = {\it o}(n^{-\delta})\,;\label{e3.12}\\
\!\!\!\!\!\!{\p}_{n,0}\left(\Delta_n(t) \ge  A\right) &= {\it  o}(n^{-\delta})\,;
\label{e3.13}\\
\!\!\!\!\!\!{\p}_{n,tn^{-1/2}}\left(\Delta_n(t) \le  -A\right) & = {\it o}(n^{-\delta})\,.\label{e3.14}
\end{align}
Найдем сначала выражения для ${\e}_{n,0}\Delta_n(t)$ 
и~${\e}_{n,0}\Delta_n^2(t)$. Из соотношения~(\ref{e3.4}) непосредственно получаем
\begin{multline}
{\e}_{n,0} \Delta_n(t)  =  - bC(a,b) t^2 -{}\\
{}- 2b
n {\e}_0 \left(X_1 - tn^{-1/2}\right){\IT_{[0,tn^{-1/2}]}(X_1)}\,;
\label{e3.15}
\end{multline}
\begin{multline}
{\e}_{n,0} \Delta_n^2(t) =  {\D}_{n,0}\Delta_n(t) +
({\e}_{n,0}\Delta_n(t))^2={}\\
{}
=4 {b}^2 n{\D}_0 (X_1 - tn^{-1/2}){\IT_{[0,tn^{-1/2}]}(X_1)}+{}\\
{}+
({\e}_{n,0} \Delta_n(t))^2\,. \label{e3.16}
\end{multline}
Найдем теперь  математическое ожидание и дис\-персию случайной
величины $(X_1-\theta){\IT_{[0,\theta]}(X_1)}$. Используя
выражение  для характеристической функции (см.~(\ref{e3.15}))
$g_{\theta}(s)$ и ее разложение  при\linebreak $\theta = \theta_n \to 0$
(см.~(\ref{e3.6})), после дифференцирования функции~$g_{\theta_n}(s)$ по~$s$ получим
\begin{multline}
{\e}_0 (X_1 - \theta_n){\IT_{[0,\theta_n]}(X_1)} =
\fr{g_{\theta_n}^{(1)}(0)}{i} ={}\\
{}= - \fr{C(a,b)}{2}\,\theta_n^2
+ \fr{bC(a,b)}{6}\,\theta_n^3 +O(\theta_n^4)\,; \label{e3.17}
\end{multline}
\begin{multline}
{\e}_0 (X_1 - \theta_n)^2{\IT_{[0,\,\theta_n]}(X_1)} =
- g_{\theta_n}^{(2)}(0) ={}\\
{}=
 \fr{C(a,b)}{3}\,\theta_n^3+
 O(\theta_n^4)\,. \label{e3.18}
\end{multline}
Полагая в выражениях~(\ref{e3.17}) и (\ref{e3.18})  $\theta_n = tn^{-1/2}$, из~(\ref{e3.15}) 
и~(\ref{e3.16}), получаем
\begin{align}
{\e}_{n,0} \Delta_n(t)& = -\fr{C(a,b)b^2 t^3}{3\sqrt{n}}+O(n^{-1})\,; \label{e3.19}\\
{\e}_{n,0} \Delta_n^2(t) &= \fr{4C(a,b)b^2 t^3}{3\sqrt{n}}+ O(n^{-1})\,. \label{e3.20}
\end{align}
Теперь соотношение~(\ref{e3.13}) следует из неравенства Чебышева, поскольку
при $0 \le \delta < 1/2$ из~(\ref{e3.20}) получаем, что
\begin{multline*}
{\p}_{n,0}\left ( \Delta_n(t)\ge  A\right )\le
\fr{{\e}_{n,0}\Delta_n^2(t)}{A^2} ={}\\
{}=
O(n^{-1/2}) ={\it o}(n^{-\delta})\,. % \label{e3.21}
\end{multline*}
Доказательство соотношения~(\ref{e3.14}) аналогично:
\begin{multline*}
\!\!\!\!\!{\p}_{n,tn^{-1/2}}\left(\Delta_n(t)\le -A\right)\le
{\p}_{n,tn^{-1/2}}(|\Delta_n(t)| \ge A) \le{}\\
{}
\le \fr{{\e}_{n,tn^{-1/2}} \Delta_n^2(t)}{A^2}= O(n^{-1/2}) =
{\it o}(n^{-\delta})\,. 
%\label{e3.22}
\end{multline*}
Докажем теперь соотношение~(\ref{e3.12}). Нам понадобится вспомогательная лемма.


\smallskip

\noindent
\textbf{Лемма 3.3.} {\it Для любого $x >  0$
существует константа  $C > 0$,  не зависящая от~$x$  и такая, что
справедливо следующее неравенство
$$
{\p}_{n,0}\left(\sqrt[{4}]{n} |\Delta_n(t)| \ge x\right) \le 
C e^{-x}\,, \quad C > 0\,.
$$
}

\vspace*{-8pt}

\noindent
Д\,о\,к\,а\,з\,а\,т\,е\,л\,ь\,с\,т\,в\,о\,. \ 
Учитывая неравенство Чебышева, рассматриваемую вероятность можно
представить в виде
\begin{multline*}
{\p}_{n,0}\left(\sqrt[{4}]{n} |\Delta_n(t)| \ge x  \right)=
{\p}_{n,0}\left(\sqrt[{4}]{n} \Delta_n(t)\ge x\right)+{}\\
{}
+{\p}_{n,0}\left(-\sqrt[{4}]{n} \Delta_n(t) \ge x\right) ={}\\
{}=
{\p}_{n,0}\left(\exp\{\sqrt[{4}]{n}\Delta_n(t)\} \ge \exp\{x\}
\right)+{}\\
+{\p}_{n,0}\left(\exp\{-\sqrt[{4}]{n}\Delta_n(t)\} \ge \exp\{x\}
\right )\le{}\\
\le e^{-x}\left({\e}_{n,0}\exp\{\sqrt[{4}]{n}\Delta_n(t)\}+{}\right.\\
{}+\left.
{\e}_{n,0}\exp\{-\sqrt[{4}]{n}\Delta_n(t)\}\right)\,.
\end{multline*}
Теперь, полагая в формуле~(\ref{e3.7}) $s = \pm i$,  получим
$$
{\e}_{n,0}\exp\{\pm\sqrt[{4}]{n}\Delta_n(t)\}=
\exp\left\{\fr{2b^2 C(a,b)t^3}{3}+{\it o}(1)\right\}\,.
$$
Отсюда следует утверждение лем\-мы.~$\Box$
\pagebreak

Проверим теперь условие~(\ref{e3.12})  леммы~3.2. Имеем
\begin{multline*}
{\e}_{n,0} |\Delta_n(t)| {\IT_{(\gamma
n^{-\delta/2},\,A)}\left(|\Delta_n(t)|\right)}\le{}\\
{}\le
A {\e}_{n,0}{\IT_{(\gamma n^{-\delta/2}, A)}}\left( |\Delta_n(t)|
\right)\le{}\\
{}
\le A {\p}_{n,0}\left(|\Delta_n(t)| \ge \gamma n^{-\delta/2}
\right)={}\\
{}=
 A{\p}_{n,0}\left(\sqrt[{4}]{n}|\Delta_n(t)| \ge
\gamma n^{1/4 - \delta/2}\right)\,.
\end{multline*}
Для оценки этой вероятности применим лемму~3.3, из которой следует,
что она не превосходит вели\-чины
$$
C\exp\{-\gamma n^{1/4 - \delta/2}\} = {\it o}(n^{-\delta})\,, \quad
C > 0\,, \ \ 0 \le \delta < \fr{1}{2}\,.
$$
Осталось доказать соотношение~(\ref{e3.13}). Для его доказательства
применим неравенство Берри--Эссеена (см.~[8, приложение~5, с.~449]. 
Из соотношений~(\ref{e3.3}), (\ref{e3.17}) и~(\ref{e3.18}) получаем
\begin{multline}
\mu_n \equiv {\e}_{n,0} \Lambda_n(t) = - at^2 + 2 b n
{\e}_0 \left(
\vphantom{\fr{t}{\sqrt{n}}}
X_1 -{}\right.\\
\!\left.{}-\fr{t}{\sqrt n}\right)
{\IT_{[0,\,t/\sqrt{n}]}(X_1)}  %= {}\\
=
- \fr{1}{2}I_{a,b}t^2 + O(n^{-1/2})\,;\!\!
\label{e3.25}
\end{multline}
\begin{multline}
\!\!\sigma_n^2 \equiv {\D}_{n,0}\Lambda_n(t) =\hspace*{-1.1pt} %{}\\
%{}=
n {\D}_0 \left[\fr{t}{\sqrt{n}}\left(2aX_1+ b\,
\mathrm{sign}\left(X_1\right)\right)+{}\right.\\
\left.{}+
2b\left(X_1 - t/\sqrt{n}\right)
{\IT_{[0,tn^{-1/2}]}(X_1)}
\vphantom{\fr{t}{\sqrt{n}}}
\right]={}\\
{}
= I_{a,b}t^2 + O(n^{-1/2})\,. 
\label{e3.26}
\end{multline}
Теперь в силу неравенства Бэрри--Эссеена и~(\ref{e3.25}), (\ref{e3.26}) имеем
\begin{multline}
{\p}_{n,0}\left(x - n^{-\delta/2} \le \Lambda_n(t) \le x\right)
\le{}\\
{}
\le \left |{\p}_{n,0}\left(\Lambda_n(t)\le  x\right) -
\Phi\left(\fr{x-\mu_n}{\sigma_n}\right)\right|+{}\\
{}
+ \Bigl|{\p}_{n,0}\Bigl(\Lambda_n(t) < x - n^{-\delta/2}\Bigr) -
\Phi\Bigl(\fr{x - n^{-\delta/2}-\mu_n}{\sigma_n}\Bigr)\Bigr|+{}\\
{}+
\Bigl|\Phi\Bigl(\fr{x - \mu_n}{\sigma_n}\Bigr)-
\Phi\Bigl(\fr{x-n^{-\delta/2}-\mu_n}{\sigma_n}\Bigr)\Bigr|\le{}\\
{}
\le \fr{C}{\sqrt{n}\sigma_n^{3}}{\e}_0\Bigl|
t\Bigl(2aX_1+ b\, \mathrm{sign}\left(X_1\right)\Bigr)+{}\\
{}+
2b\sqrt{n}\Bigl(X_1- \fr{t}{\sqrt{n}}\Bigr)
{\IT_{[0,t/\sqrt{n}]}}(X_1)+bC(a,b)\fr{t^2}{\sqrt{n}}
\Bigr|^3+{}\\
{}+ \fr{1}{\sqrt{2\pi}\sigma_nn^{\delta/2}}= 
O\left(n^{-1/2}\right)+{}\\
{}+
O\left(n^{-\delta/2}\right) =O\left(n^{-\delta/2}\right)\,,\quad
C > 0\,. 
\label{e3.27}
\end{multline}
Лемма доказана. \ \ $\Box$

\smallskip

Из леммы~3.3 непосредственно получаем следующее

\smallskip

\noindent
\textbf{Следствие 3.1.} {\it
Для любого $m > 0$ справедливо соотношение
$$
{\e}_{n,0} |\Delta_n(t)|^m = O(n^{-m/4})\,.
$$
}

\smallskip

\noindent
Д\,о\,к\,а\,з\,а\,т\,е\,л\,ь\,с\,т\,в\,о\,. \
Учитывая формулу интегрирования по частям, (см., например,~[11,  лемма~5.6.1, с.~178], имеем
$$
{\e}_{n,0} |\sqrt[{4}]{n}\Delta_n(t)|^m = \int\limits_0^\infty
{\p}_{n,0}\Bigl(|\sqrt[{4}]{n}\Delta_n(t)| \ge x^{1/m}\Bigr)\,dx\,.
$$
Теперь согласно лемме~3.3 последний интеграл не превосходит выражения
$$
C \int\limits_0^\infty \exp\Bigl\{-x^{1/m}\Bigr\}\,dx \le C_1
< \infty\,, \quad C > 0\,, \ \ C_1 > 0\,.
$$
Отсюда следует утверждение следствия. \ \ \ $\Box$


{\small\frenchspacing
{%\baselineskip=10.8pt
\addcontentsline{toc}{section}{Литература}
\begin{thebibliography}{99}    

\bibitem{13be} %1
\Au{Takeuchi~K.}
Asymptotic theory of statistical estimation.~---  Tokyo, 1974. (In Japanese.)

\bibitem{12be}  %2
\Au{Asrabadi~B.\,R.}
The exact confidence interval for 
the scale parameter and the MVUE of the Laplace distribution~// 
Communications in Statistics. Theory and Methods, 1985. Vol.~14, No.\,3. 
P.~713--733.

\bibitem{4be} %3
\Au{Бурнашев~М.\,В.} 
Асимптотические разложения для 
медианной оценки параметра~// Теория вероятности и её
применения, 1996. Т.~41. Вып.~4. С.~738--753.

\bibitem{8be}  %4
\Au{Kotz~S., Kozubowski~T.\,J., Podgorski~K.}
The Laplace distribution and generalizations: 
A revisit with applications to communications, economics, engineering, 
and finance.~--- Birkhauser, 2001.  349~p.

\bibitem{9be} %5
\Au{Бенинг~В.\,Е., Королёв~В.\,Ю.}
Некоторые статистические  задачи, связанные с распределением Лапласа~//
Информатика и её применения, 2008. Т.~2.  Вып.~2. С.~19--34.

\bibitem{1be}  %6
\Au{Bening~V.\,E.} 
Asymptotic theory of testing statistical hypotheses.~--- 
VSP, Utrecht, 2000. 277~p.

\bibitem{6be} %7
\Au{Королёв~Р.\,А., Тестова~А.\,В., Бенинг~В.\,Е.}
О мощ\-ности асимптотически оптимального критерия в случае 
распределения Лапласа~// Вестник Тверского государственного университета, 
сер. Прикладная математика, 2008. Вып.~8. №\,4(64). С.~5--23.

\bibitem{2be} %8
\Au{Боровков~А.\,А.} 
Теория вероятностей.~--- М.: УРСС, 2003.  470~с.

\bibitem{5be} %9
\Au{Королёв Р.\,А., Бенинг В.\,Е.} 
Асимптотические 
разложения для мощностей критериев в случае распределения Лапласа~//
Вестник Тверского государственного университета, сер. 
Прикладная математика, 2008. Вып.~3(10), №\,26(86). С.~97--107.

\bibitem{3be} %10
\Au{Bickel~P.\,J., Chibisov~D.\,M., Van Zwet~W.\,R.} 
On efficiency of first and second order~// 
Intern. Statist. Review, 1981. Vol.~49. P.~169--175.


\label{end\stat}

\bibitem{11be} 
\Au{Феллер~В.} 
Введение в теорию вероятностей и ее приложения. Т.~2.~---  М.: Мир, 1984. 751~с.

 \end{thebibliography}
}
}
\end{multicols}

%\bibitem{7be} 
%\Au{David~H.\,A., Nagaraja~H.\,N.}
%Order statistics. 3rd ed.~--- New Jersey: Wiley, 2003.  P.~458.

%\bibitem{10be} 
%\Au{Леман~Э.} 
%Проверка статистических гипотез.~---  М.: Наука, 1964. 498~с.