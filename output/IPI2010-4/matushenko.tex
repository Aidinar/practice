\def\stat{mat}

\def\tit{СТАЦИОНАРНЫЕ ХАРАКТЕРИСТИКИ ДВУХКАНАЛЬНОЙ СИСТЕМЫ ОБСЛУЖИВАНИЯ С~ПЕРЕУПОРЯДОЧИВАНИЕМ ЗАЯВОК 
И~РАСПРЕДЕЛЕНИЯМИ ФАЗОВОГО ТИПА}

\def\titkol{Стационарные характеристики двухканальной системы обслуживания с~переупорядочиванием заявок} 
%и~распределениями фазового типа}

\def\autkol{С.\,И.~Матюшенко}
\def\aut{С.\,И.~Матюшенко$^1$}

\titel{\tit}{\aut}{\autkol}{\titkol}

%{\renewcommand{\thefootnote}{\fnsymbol{footnote}}\footnotetext[1]
%{Исследование поддержано грантами РФФИ 08-07-00152 и 09-07-12032.
%Статья написана на основе материалов доклада, представленного на IV 
%Международном семинаре  <<Прикладные задачи теории вероятностей и математической статистики, 
%связанные с моделированием информационных систем>> (зимняя сессия, Аоста, Италия, январь--февраль 2010~г.).}}

\renewcommand{\thefootnote}{\arabic{footnote}}
\footnotetext[1]{Российский университет дружбы народов,  кафедра теории вероятностей и математической статистики, matushenko@list.ru}

%\vspace*{6pt}

\Abst{Рассматривается двухканальная система обслуживания
ограниченной емкости с распределениями фазового типа и
переупорядочиванием заявок. Получено выражение для преобразования
Лап\-ла\-са--Стилтьеса функции распределения задержки переупорядочивания
в рассматриваемой системе в стационарном режиме работы. Разработан
алгоритм для расчета факториальных моментов числа заявок в буфере
переупорядочивания.}

%\vspace*{2pt}


\KW{система  массового обслуживания; распределение
фазового типа; переупорядочивание заявок}

%\vspace*{6pt}

       \vskip 14pt plus 9pt minus 6pt

      \thispagestyle{headings}

      \begin{multicols}{2}

      \label{st\stat}

\section{Постановка задачи}

Рассмотрим двухканальную систему массового обслуживания (СМО) с общим накопителем ограниченной емкости~$r$, 
$r<\infty$, на которую поступает рекуррентный поток заявок с функцией распределения (ФР) фазового типа~$A(x)$,
\begin{equation*}
A(x)=1-\boldsymbol{\alpha}^T e^{\boldsymbol{\Lambda}x}\mathbf{1}\,, \quad
x\geq 0\,,\enskip \boldsymbol{\alpha}^T\mathbf{1}=1\,,
\end{equation*}
с неприводимым PH-представлением $(\boldsymbol{\alpha},\boldsymbol{\Lambda})$ порядка~$l$~\cite{1mat}.

Будем предполагать, что времена обслуживания на приборе~$j$ независимы  между собой и имеют общую 
ФР фазового типа
\begin{equation*}
B_j(x)=1-\boldsymbol{\beta}_j^T e^{\boldsymbol{M}_j x}\mathbf{1}\,, \quad
x\geq 0\,,\enskip \boldsymbol{\beta}_j^T\boldsymbol{1}=1
\end{equation*}
с неприводимым PH-представлением $(\boldsymbol{\beta}_j,\boldsymbol{M}_j)$ порядка $m_j$, $j=1,2$.

Далее, без ограничения общности примем, что интенсивность обслуживания на приборе~1 выше, чем на приборе~2. 
Заявка, поступающая в свободную СМО, направляется на первый прибор. При наличии очереди действует дисциплина FCFS.

Предположим, что всем заявкам при поступлении в систему присваивается порядковый номер. На выходе из СМО 
будем требовать сохранения порядка между заявками, установленного при входе в нее. Заявки, прошедшие 
обслуживание и нарушившие установленный порядок, будут накапливаться на выходе системы в  буфере 
переупорядочивания (БП).

В соответствии с обозначениями Кендалла рассматриваемую систему будем кодировать как $PH/PH/2/r/res$, 
где $res$~--- сокращение от английского $resequence$~--- переупорядочивание. Такая сис\-те\-ма уже 
рассматривалась автором в~\cite{2mat}. В~той работе функционирование системы описывалось однородным 
марковским процессом (ОМП) над пространством состояний без учета содержимого БП. В~итоге был разработан 
рекуррентный матричный алгоритм для расчета стационарных вероятностей состояний указанного процесса. 
Основная задача данной работы состоит в том, чтобы, опираясь на результаты~\cite{2mat}, 
получить показатели, характеризующие стационарное состояние БП. Перейдем к решению этой задачи.

\section{Построение математической модели}

Построим марковский процесс, описывающий функционирование рассматриваемой системы. 
Для этого введем понятие упорядоченности.  Будем считать, что система находится в 
упорядоченном состоянии (упорядочена), если на приборах~1 и~2 обслуживаются заявки с номерами~$N_1$ 
и~$N_2$, $N_1 < N_2$, в противном случае, т.\,е.\ при $N_1>N_2$, система не упорядочена. 
Система также упорядочена (не упорядочена) при наличии в ней одной заявки на приборе~1 (приборе~2).

Теперь рассматриваемую СМО  с учетом введенного понятия упорядоченности, а также 
с учетом вероятностной интерпретации PH-распределения (\cite{1mat}, с.~104) можно описать ОМП~$Y(t)$, 
$t\geq 0$, над пространством состояний

\noindent
\begin{align*}
\mathcal{Y}&=\bigcup\limits_{k=0}^{r+2}\mathcal{Y}_k\,,\enskip
\mathcal{Y}_k=\mathcal{Y}_{k1}\bigcup\mathcal{Y}_{k2}\,,\\
\mathcal{Y}_{ki}&=\bigcup_{n=0}^{\infty}\mathcal{Y}_{kin}\,,
k=\overline{1,r+2},\, i=1,2\,,
\end{align*}
где
\begin{align*}
\mathcal{Y}_0&=\{(s,0), s=\overline{1,l}\}\,;\\
\mathcal{Y}_{1in}&=\{(s,j_i,i,n), s=\overline{1,l}, j_i=\overline{1,m_i} \}\,, \\
&\hspace*{30mm}i=1,2\,, \enskip n\geq 0\,;\\
\mathcal{Y}_{kin}&=\{(k,s,j_1,j_2,i,n), s=\overline{1,l}, j_1=\overline{1,m_1},\\
&\hspace*{8mm} j_2=\overline{1,m_2}\}\,,\enskip
 k=\overline{2,r+2},\enskip i=1,2,\, n\geq 0\,.
\end{align*}
Здесь для некоторого момента времени~$t$: $Y(t)=$\linebreak $=(s,0)$, если в момент времени~$t$ система пус\-та, 
а процесс генерации заявки проходит фазу~$s$;  $Y(t)=(s,j_i,i,n)$, если в системе имеется одна 
заявка, обслуживаемая на первом приборе при $i=1$ либо на втором приборе при $i=2$, процесс 
обслуживания находится на фазе~$j_i$, а в БП содержится $n$~заявок; $Y(t)=(k,s,j_1,j_2,i,n)$, 
если в очереди и на приборах имеется $k$~заявок, процессы обслуживания заявок на приборах находятся 
на фазах~$j_1$ и~$j_2$ соответственно, причем система упорядочена, если $i=1$, либо не упорядочена, 
если $i=2$, а индексы~$s$ и~$n$ имеют прежний смысл.

В предположении, что интенсивности потока и обслуживания конечны, процесс~$Y(t)$ эргодичен и, 
следовательно, существуют вероятности
\begin{equation*}
p_y=\lim\limits_{t\rightarrow\infty}P\{Y(t)=y\}, y\in \mathcal{Y}\,,
\end{equation*}
совпадающие со стационарными.

Введем векторы
\begin{multline*}
\boldsymbol{p}_{1in}^{\mathrm{T}}=
(p_{11in},\ldots,p_{1m_iin},\ldots , p_{2m_iin},\ldots\\
\ldots , p_{l1in},\ldots, p_{lm_iin})\,,\quad  i=1,2,\enskip n\geq 0\,;
\end{multline*}

\vspace*{-14pt}

\noindent
\begin{multline*}
\boldsymbol{p}_{kin}^{\mathrm{T}}=(p_{ki11in},\ldots,p_{k11m_2in},\ldots,p_{k1m_1m_2in},
\ldots\\
\ldots,
p_{klm_1m_2in}),\quad k=\overline{2,r+2},\enskip i=1,2, \enskip n\geq 0\,.
\end{multline*}
Будем использовать обозначение $\boldsymbol{p}_{ki,\cdot}=\sum\limits_{n\geq 0} \boldsymbol{p}_{kin}$ и введем матрицы:
\begin{align*}
\boldsymbol{D}_{1i}&=\boldsymbol{\Lambda}_1\oplus \boldsymbol{M}_i\,,\quad i=1,2;\\
\boldsymbol{D}_{ki}&=\boldsymbol{\Lambda}_k\oplus \boldsymbol{M}_i\,,\quad k=\overline{2,r+1}\,,\enskip i=1,2\,;\\
\boldsymbol{D}_{r+2,i}&=\left(\boldsymbol{\Lambda}_{r+2}+ \boldsymbol{\lambda}_{r+2}\boldsymbol{\alpha}^{\mathrm{T}}\right)\oplus \boldsymbol{M}_i\,,\quad i=1,2\,;\\
\boldsymbol{E}_{1i}&=
\begin{cases}
\boldsymbol{\lambda}_1\boldsymbol{\alpha}^{\mathrm{T}}\otimes\boldsymbol{I}\otimes
\boldsymbol{\beta}_2^{\mathrm{T}}\,, & i=1\,,\\
\boldsymbol{\lambda}_1\boldsymbol{\alpha}^{\mathrm{T}}\otimes
\boldsymbol{\beta}_1^{\mathrm{T}}  \otimes\boldsymbol{I}\,, & i=2;
\end{cases}
\end{align*}
\begin{align*}
%\end{equation*}
%\begin{equation*}
\boldsymbol{E}_{ki}&=\boldsymbol{\lambda}_k\boldsymbol{\alpha}^{\mathrm{T}} \otimes\boldsymbol{I}\otimes\boldsymbol{I}\,,\quad
 k=\overline{2,r+2}\,, \enskip
 i=1,2\,;\\
\boldsymbol{G}_{2i}&=
\begin{cases}
\boldsymbol{I}\otimes \boldsymbol{I}\otimes
\boldsymbol{\mu}_2\,, & i=1\,,\\
\boldsymbol{I} \otimes \boldsymbol{\mu}_1 \otimes \boldsymbol{I}\,, & i=2\,;
\end{cases}
\\
%\begin{equation*}
\boldsymbol{G}_{ki}&=
\begin{cases}
\boldsymbol{I}\otimes \boldsymbol{I}\otimes
\boldsymbol{\mu}_2\boldsymbol{\beta}_2^{\mathrm{T}}\,, & i=1\,,\\
\boldsymbol{I} \otimes \boldsymbol{\mu}_1\boldsymbol{\beta}_1^{\mathrm{T}} \otimes \boldsymbol{I}\,, & i=2\,.
\end{cases}
\end{align*}
Здесь и далее $U\otimes V$~--- кронекерово произведение, а $U\oplus V$~--- кронекерова сумма матриц~$U$ и~$V$.

Стационарное распределение вероятностей $\{p_y, y\in\mathcal{Y}\}$ удовлетворяет следующей системе 
уравнений равновесия (СУР):
\begin{equation}
\label{eq:1}
\boldsymbol{0}^{\mathrm{T}}=\boldsymbol{p}_0^{\mathrm{T}} \boldsymbol{\Lambda}_0+
\boldsymbol{p}_{11,\cdot}^{\mathrm{T}}\left(
\boldsymbol{I}\otimes\boldsymbol{\mu}_1\right) + \boldsymbol{p}_{12,\cdot}^{\mathrm{T}}\left(\boldsymbol{I}\otimes \boldsymbol{\mu}_2\right)\,;
\end{equation}
\begin{multline}
\boldsymbol{0}^{\mathrm{T}}=u(1-n)u(2-i)\boldsymbol{p}_0^{\mathrm{T}} \left(\boldsymbol{\lambda}_0\boldsymbol{\alpha}^{\mathrm{T}} \otimes\boldsymbol{\beta}_1^{\mathrm{T}}\right)+\boldsymbol{p}_{1in}^{\mathrm{T}}\boldsymbol{D}_{1i}+{}\\
{}+\left[u(1-n)\boldsymbol{p}^{\mathrm{T}}_{2,3-i,\cdot}+ u(n)\boldsymbol{p}^{\mathrm{T}}_{2,i,n-1}\right]
\boldsymbol{G}_{2i}\,,\\
 i=1,2\,,\enskip n\geq 0\,;
\label{eq:2}
\end{multline}
\begin{multline}
\label{eq:3}
\boldsymbol{0}^{\mathrm{T}}=u(3-k)\boldsymbol{p}_{1in}^{\mathrm{T}}\boldsymbol{E}_{1i}+
u(k-2)\boldsymbol{p}_{k-1,in}^{\mathrm{T}}\boldsymbol{E}_{k-1,i}+{}\\
{}+\boldsymbol{p}_{kin}^{\mathrm{T}}\boldsymbol{D}_{ki}+
\left[u(1-n)\boldsymbol{p}_{k+1,3-i,\cdot}^{\mathrm{T}}+{}\right.\\
{}+\left. u(n)\boldsymbol{p}_{k+1,i,n-1,}^{\mathrm{T}}\right]\boldsymbol{G}_{k+1,i}\,,\\
k=\overline{2,r+1}\,,\ \  i=1,2\,,\ \ n\geq 0;
\end{multline}
\begin{multline}
\label{eq:4}
\boldsymbol{0}^{\mathrm{T}}=\boldsymbol{p}_{r+1,in}^{\mathrm{T}}\boldsymbol{E}_{r+1,i}+
\boldsymbol{p}_{r+2,in}^{\mathrm{T}}\boldsymbol{D}_{r+2,i}\,,\\ i=1,2\,,\enskip n\geq 0\,,
\end{multline}
с условием нормировки
\begin{equation}
\label{eq:5}
\boldsymbol{p}_0^{\mathrm{T}}\boldsymbol{1}+
\sum\limits_{k=1}^{r+2}\sum\limits_{i=1}^2\sum\limits_{n=0}^{\infty}
\boldsymbol{p}_{kin}^{\mathrm{T}}\boldsymbol{1}=1\,.
\end{equation}
Здесь и далее $u(x)=1$ при $x>0$ и $u(x)=0$ при $x\leq 0$.

Система уравнений~(\ref{eq:1})--(\ref{eq:5}) понадобится в дальнейшем для определения 
стационарных характеристик рассматриваемой СМО.

\section{Факториальные моменты числа заявок в буфере переупорядочивания}

Для определения факториальных моментов числа заявок в БП воспользуемся аппаратом производящих функций (ПФ). Положим
\begin{equation}
\left.
\begin{array}{rl}
 F_{sj_ii}(z)&=\sum\limits_{n\geq 0}p_{sj_iin}z^n,\\[6pt]
 F_{ksj_1j_2i}(z)&=\sum\limits_{n\geq 0}p_{ks j_1j_2in}z^n\,,\enskip s=\overline{1,l}\,,\\
 &\!\!\!\!\!\!\!\!\!\!\!\!\!\!\!\!\!\! j_i=\overline{1,m_i}\,,\  i=1,2\,,\  z\in\mathbb{C}\,,\ |z|\leq 1\,,
\end{array}
\right \}
\label{eq:6}
\end{equation}
и введем векторы $\boldsymbol{F}_{1i}(z)$ и $\boldsymbol{F}_{ki}(z)$, аналогичные по структуре 
векторам~$\boldsymbol{p}_{1i}$ и~$\boldsymbol{p}_{ki}$.


Умножая уравнения~(\ref{eq:2})--(\ref{eq:4}) для каждого фиксированного $n\geq 0$ справа на~$z^n$ 
и суммируя полученные равенства при каждом фиксированном $k=\overline{1,r+2}$, $i=1,2$,  по всем 
возможным значениям~$n$, с учетом~(\ref{eq:6}) приходим к следующей системе уравнений:
\begin{multline}
\label{eq:7}
\boldsymbol{0}^{\mathrm{T}}=u(1-n)u(2-i)\boldsymbol{p}_0^{\mathrm{T}} \left(\boldsymbol{\lambda}_0\boldsymbol{\alpha}^{\mathrm{T}} \otimes\boldsymbol{\beta}_1^{\mathrm{T}}\right)+
\boldsymbol{F}_{1i}^{\mathrm{T}}(z)\boldsymbol{D}_{1i}+{}\\
{}+\left[\boldsymbol{p}_{2,3-i,\cdot}^{\mathrm{T}}+z\boldsymbol{F}_{2i}^{\mathrm{T}}(z)\right] \boldsymbol{G}_{2i},\,\, i=1,2;
\end{multline}
\begin{multline}
\label{eq:8}
\boldsymbol{0}^{\mathrm{T}}=u(3-k)\boldsymbol{F}_{1i}^{\mathrm{T}}(z)\boldsymbol{E}_{1i}+
u(k-2)\boldsymbol{F}_{k-1,i}^{\mathrm{T}}(z)\boldsymbol{E}_{k-1,i}+{}\\
{}+\boldsymbol{F}_{ki}^{\mathrm{T}}(z)\boldsymbol{D}_{ki}
+\left[\boldsymbol{p}_{k+1,3-i,\cdot}^{\mathrm{T}}+z\boldsymbol{F}_{k+1,i}^{\mathrm{T}}(z) \right] \boldsymbol{G}_{k+1,i}\,,\\
k=\overline{2,r+1}\,,\enskip i=1,2\,;
\end{multline}
\begin{multline}
\label{eq:9}
\boldsymbol{0}^{\mathrm{T}}=\boldsymbol{F}_{r+1,i}^{\mathrm{T}}(z)\boldsymbol{E}_{r+1,i}+
\boldsymbol{F}_{r+2,i}^{\mathrm{T}}(z)\boldsymbol{D}_{r+2,i}\,,\\ i=1,2\,.
\end{multline}

Далее введем обозначения:
\begin{align}
\label{eq:10}
 \boldsymbol{v}_{ki\nu}&=\boldsymbol{F}_{ki}^{(\nu)}(1),\,\ 
k=\overline{1,r+2}\,, \ i=1,2\,,\ \nu\geq 0,\\
v_{\nu}&=\sum\limits_{k=1}^{r+2}\sum\limits_{i=1}^2 \boldsymbol{v}_{ki\nu}^{\mathrm{T}} \boldsymbol{1}\,, \enskip \nu\geq 1\,.
\label{dd}
\end{align}

Заметим, что $v_{\nu}$~--- факториальный момент  порядка~$\nu$, $\nu=1,2,\ldots$, числа заявок в БП, 
а $\boldsymbol{v}_{ki0}=\sum\limits_{n\geq 0}\boldsymbol{p}_{kin}$, $k=\overline{1,r+2}$, $i=1,2$. 
Причем система уравнений для определения $\boldsymbol{v}_{ki0}$ получается из~(\ref{eq:7})--(\ref{eq:9}) 
после подстановки в них $z=1$. Кроме того, эта система с учетом~(\ref{eq:1})
 и~(\ref{eq:5}) полностью совпадает с СУР из работы~\cite{2mat} и в силу единственности ее решения получаем
\begin{equation*}
%\label{eq:11}
\boldsymbol{v}_{ki0}=\boldsymbol{p}_{ki},\,\, k=\overline{1,r+2}, \,\, i=1,2.
\end{equation*}


Теперь получим систему уравнений для определения~$\boldsymbol{v}_{ki\nu}$, $k=\overline{1,r+2}$, $i=1,2$, $\nu\geq 1$. 
Для этого продифференцируем~(\ref{eq:7})--(\ref{eq:9}) $\nu$~раз по~$z$ и положим $z=1$. 
В~результате приходим к следующим уравнениям:
\begin{equation}
\!\!\boldsymbol{0}^{\mathrm{T}}=\boldsymbol{v}_{1i\nu}^{\mathrm{T}}\boldsymbol{D}_{1i}+
\left[\boldsymbol{v}_{2i\nu}^{\mathrm{T}}+\nu\boldsymbol{v}_{2i,\nu-1}^{\mathrm{T}}\right]
\boldsymbol{G}_{2i}\,,\enskip i=1,2\,;\!
\label{eq:12}
\end{equation}
\begin{multline}
\label{eq:13}
\boldsymbol{0}^{\mathrm{T}}=u(3-k)\boldsymbol{v}_{1i\nu}^{\mathrm{T}}\boldsymbol{E}_{1i}+
u(k-2)\boldsymbol{v}_{k-1,i\nu}^{\mathrm{T}}\boldsymbol{E}_{k-1,i}+{}\\
{}+\boldsymbol{v}_{ki\nu}^{\mathrm{T}}\boldsymbol{D}_{ki}
+\left[\boldsymbol{v}_{k+1,i\nu}^{\mathrm{T}}+\nu\boldsymbol{v}_{k+1,i,\nu-1}^{\mathrm{T}}\right]
\boldsymbol{G}_{k+1,i}\,,\\
 k=\overline{2,r+1}\,,\enskip i=1,2;
\end{multline}
\begin{multline}
\label{eq:14}
\boldsymbol{0}^{\mathrm{T}}=\boldsymbol{v}_{r+1,i\nu}^{\mathrm{T}}\boldsymbol{E}_{r+1,i}+
\boldsymbol{v}_{r+2,i\nu}\boldsymbol{D}_{r+2,i}\,,\\
i=1,2\,,\enskip \nu=1,2,\ldots
\end{multline}

Решение системы~(\ref{eq:12})--(\ref{eq:14}) получим с помощью блочного 
LU-разложения матрицы коэффициентов. Для этого введем векторы
\begin{align*}
\boldsymbol{v}_{i\nu}^{\mathrm{T}}&=(\boldsymbol{v}_{1i\nu}^{\mathrm{T}},\ldots,
\boldsymbol{v}_{r+2,i\nu}^{\mathrm{T}})\,,\enskip i=1,2\,,\enskip \nu\geq 1\,;
\\
\boldsymbol{d}_{ki\nu}^{\mathrm{T}}&=
\begin{cases}
& -\nu \boldsymbol{v}_{k+1,i,\nu-1}^{\mathrm{T}}\boldsymbol{G}_{k+1,i}\,,\enskip k=\overline{1,r+1}\,;\\
& \boldsymbol{0}^{\mathrm{T}},\,\, k=r+2\,,\enskip i=1,2,\enskip \nu\geq 1\,;
\end{cases}\\
\boldsymbol{d}_{i\nu}^{\mathrm{T}}&=(\boldsymbol{d}_{1i\nu}^{\mathrm{T}},\ldots,
\boldsymbol{d}_{r+2,i\nu}^{\mathrm{T}})\,,\enskip i=1,2\,,\enskip \nu\geq 1\,,
\end{align*}
и матрицы
{\small
\begin{multline*}
\boldsymbol{D}_{i}=
\begin{pmatrix}
\boldsymbol{D}_{1i} &\boldsymbol{E}_{1i} & \boldsymbol{0} &
\boldsymbol{0} & \ldots &\boldsymbol{0} &\boldsymbol{0} &\boldsymbol{0}\\
\boldsymbol{G}_{2i} & \boldsymbol{D}_{2i} & \boldsymbol{E}_{2i} &
\boldsymbol{0} & \ldots & \boldsymbol{0} &\boldsymbol{0} &\boldsymbol{0}\\
\boldsymbol{0} & \boldsymbol{G}_{3i} & \boldsymbol{D}_{3i} &
\boldsymbol{E}_{3i} & \ldots & \boldsymbol{0} & \boldsymbol{0} & \boldsymbol{0}\\
\vdots & \vdots & \vdots & \vdots & \ddots  & \vdots & \vdots & \vdots\\
\boldsymbol{0} & \boldsymbol{0} & \boldsymbol{0} & \boldsymbol{0} & \ldots &
\boldsymbol{G}_{r+1,i} & \boldsymbol{D}_{r+1,i} & \boldsymbol{E}_{r+1,i} \\
\boldsymbol{0} & \boldsymbol{0} & \boldsymbol{0} & \boldsymbol{0} & \ldots & \boldsymbol{0} & \boldsymbol{G}_{r+2,i} & \boldsymbol{D}_{r+2,i}
\end{pmatrix}\,,\\
i=1,2\,.
\end{multline*}}

С учетом введенных обозначений систему уравнений~(\ref{eq:12})--(\ref{eq:14}) можно записать в следующем виде:
\begin{equation}
\label{eq:15}
\boldsymbol{v}_{i\nu}\boldsymbol{D}_{i}=\boldsymbol{d}_{i\nu}^{\mathrm{T}}\,,\enskip i=1,2\,,\,\, \nu \geq 1\,.
\end{equation}

Матрица коэффициентов $\boldsymbol{D}_{i}$ для любого фиксированного $i=1,2$ неразложима 
и обладает  свойствами инфинитезимальной  матрицы с диагональным преобладанием. Следовательно, 
для решения системы~(\ref{eq:15}) можно воспользоваться выводами, полученными в~\cite{3mat}. 
Непосредственно из~\cite{3mat} вытекает  результат, который сформулируем в виде теоремы.

\medskip

\noindent
\textbf{Теорема 1.}\ \textit{Для каждого фиксированного $i=1,2$ и $\nu=1,2,\ldots$ решение системы}~(\ref{eq:15}) 
\textit{представимо в виде}
\begin{align*}
\boldsymbol{v}_{r+2,i\nu}^{\mathrm{T}}&=\boldsymbol{y}_{r+2,i\nu}^{\mathrm{T}}\,;\\
\boldsymbol{v}_{ki\nu}^{\mathrm{T}}&=\boldsymbol{y}_{ki\nu}^{\mathrm{T}}+
\boldsymbol{v}_{k+1,i\nu}^{\mathrm{T}}\boldsymbol{H}_{k+1,i}\,,\enskip k=\overline{r+1,1}\,,
\end{align*}
\textit{где}
\begin{align*}
\boldsymbol{y}_{1i\nu}^{\mathrm{T}}&=\boldsymbol{d}_{1i\nu}^{\mathrm{T}}\boldsymbol{S}_{1i}^{-1}\,;
\\
\boldsymbol{y}_{ki\nu}^{\mathrm{T}}&=\left[\boldsymbol{d}_{ki\nu}^{\mathrm{T}}-
\boldsymbol{y}_{k-1,i\nu}^{\mathrm{T}}\boldsymbol{E}_{k-1,i\nu}\right]
\boldsymbol{S}_{ki}^{-1}\,,\enskip k=\overline{2,r+2}\,;
\\
\boldsymbol{H}_{ki}&=-\boldsymbol{G}_{ki}\boldsymbol{S}_{k-1,i}^{-1}\,,\enskip k=\overline{2,r+2}\,,
\end{align*}
\textit{а невырожденные матрицы $\boldsymbol{S}_{ki}$, $k=\overline{1,r+2}$, задаются соотношениями:}
\begin{equation*}
\boldsymbol{S}_{k,i}=\boldsymbol{D}_{ki}+\boldsymbol{H}_{ki}
\boldsymbol{E}_{k-1,i}u(k-1)\,.
\end{equation*}


\smallskip

Итак, получен алгоритм, позволяющий для каж\-до\-го $\nu=1,2,\ldots$ вычислять факториальный момент~$v_{\nu}$ 
числа заявок в БП через моменты низшего порядка. При этом, начальный шаг алгоритма состоит в определении 
векторов~ $\boldsymbol{p}_{ki}$, для чего следует обратиться к результатам~\cite{2mat}.


\section{Преобразования Лапласа--Стилтьеса стационарной функции распределения задержки переупорядочивания}

Обозначим через~$\delta$ задержку переупорядочивания заявки в стационарном режиме работы СМО, а через~$f(s)$~--- 
преобразование Лапласа--Стилтьеса (ПЛС) ФР~$F_{\delta}(t)$ случайной величины (с.в.)~$\delta$. Далее, обозначим 
через~$f_j(s)$ ПЛС условной стационарной ФР задержки переупорядочивания при условии, что задержана заявка, 
обслуженная прибором~$j$, $j=1,2$.

Очевидно, что заявка, обслуженная прибором~$j$, $j=1,2$, не будет задержана для переупорядочивания, 
если после ее ухода система останется пус\-той либо если в момент $\tau -0$ окончания 
обслуживания этой заявки на приборе~$3-j$ будет\linebreak обслуживаться заявка, пришедшая в систему позже данной заявки. 
Если же в момент $\tau -0$ на приборе~$3-j$ будет обслуживаться заявка, пришедшая в систему  раньше данной, 
то данная заявка будет задержана до момента окончания обслуживания на приборе~$3-j$. Обозначим через 
$\pi_{D,j}^{-}(s,j)$ и $\pi_{D,j}^{-}(k,s,j_{3-j},i)$ стационарные вероятности макросостояний $(s,j)$, 
$(k,s,j_{3-j},i)$ в момент $\tau -0$ выхода заявки из прибора~$j$, $j=1,2$, $k=\overline{2,r+2}$, 
$s=\overline{1,l}$, $j_{3-j}=\overline{1,m_{3-j}}$, $i=1,2$, и введем векторы:
\begin{align*}
\boldsymbol{\pi}_{D,j}^{-\mathrm{T}}(1,j)&=\left(\pi_{D,j}^{-}(1,j),\ldots, \pi_{D,j}^{-}(l,j)\right)\,;
\\
\boldsymbol{\pi}_{D,j}^{-\mathrm{T}}(k,i)&=\left(\pi_{D,j}^{-}(k,1,1,i),\ldots \right.\\
&\!\!\left.\ldots,\pi_{D,j}^{-}(k,1,2,i),\ldots, \pi_{D,j}^{-}(k,l,m_{3-j},i)\right)\,,\\
&\hspace*{15mm}j=1,2\,,\enskip  i=1,2\,,\,\, k=\overline{2,r+2}\,.
\end{align*}

Тогда на основании вышеизложенного и с учетом~[1, c.~104] получаем
\begin{multline}
\label{eq:16}
f_1(s)=\sum\limits_{k=1}^{r+2}\boldsymbol{\pi}_{D,1}^{-\mathrm{T}}(k,1) \boldsymbol{1}+{}\\
{}
+\sum\limits_{k=2}^{r+2}\boldsymbol{\pi}_{D,1}^{-\mathrm{T}}(k,2)
\left({\bf{1}}\otimes\boldsymbol{I}\right)
\left(s\boldsymbol{I}-\boldsymbol{M}_2\right)^{-1}\boldsymbol{\mu}_2\,;
\end{multline}

\vspace*{-6pt}

\noindent
\begin{multline}
\label{eq:17}
f_2(s)=\sum\limits_{k=1}^{r+2}\boldsymbol{\pi}_{D,2}^{-\mathrm{T}}(k,2) \boldsymbol{1}+{}\\
{}+\sum\limits_{k=2}^{r+2}\boldsymbol{\pi}_{D,1}^{-\mathrm{T}}(k,1)
\left({\bf{1}}\otimes\boldsymbol{I}\right)
\left(s\boldsymbol{I}-\boldsymbol{M}_1\right)^{-1}\boldsymbol{\mu}_1\,.
\end{multline}

Для определения стационарных вероятностей $\pi_{D,j}^{-}(k,j_{3-j},i)$, $\pi_{D,j}^{-}(s,j)$, 
$j=1,2$, $k=\overline{2,r+2}$, $s=\overline{1,l}$, $j_{3-j}=\overline{1,m_{3-j}}$, $i=1,2$, 
воспользуемся результатами~\cite{4mat}, согласно которым
\begin{equation}
\label{eq:18}
\boldsymbol{\pi}_{D,j}^{-\mathrm{T}}(1,j)=\frac{1}{\lambda_{D}(j)}\boldsymbol{p}_{1j}^{\mathrm{T}}
\left(\boldsymbol{I}\otimes\boldsymbol{\mu}_j\right)\,,\,\, j=1,2\,;
\end{equation}
\begin{multline}
\label{eq:19}
\boldsymbol{\pi}_{D,1}^{-\mathrm{T}}(k,i)=\frac{1}{\lambda_{D}(1)}\boldsymbol{p}_{ki}^{\mathrm{T}}
\left(\boldsymbol{I}\otimes\boldsymbol{\mu}_1\otimes\boldsymbol{I}\right)\,,\\
k=\overline{2,r+2}\,,\enskip i=1,2\,;
\end{multline}
\begin{multline}
\label{eq:20}
\boldsymbol{\pi}_{D,2}^{-\mathrm{T}}(k,i)=\frac{1}{\lambda_{D}(2)}\boldsymbol{p}_{ki}^{\mathrm{T}}
\left(\boldsymbol{I}\otimes\boldsymbol{I}\otimes\boldsymbol{\mu}_2\right)\,,\\
 k=\overline{2,r+2}\,,\enskip i=1,2\,,
\end{multline}
где $\lambda_{D}(j)$~--- интенсивность выхода заявок, обслуженных прибором~$j$, определяемая выражением
\begin{multline*}
\lambda_{D}(j)=\boldsymbol{p}_{1j}^{\mathrm{T}}\left(1\otimes\boldsymbol{\mu}_j\right)+{}\\
{}+\sum\limits_{k=2}^{r+2}\boldsymbol{p}_{k,\cdot}^{\mathrm{T}}
\left[u(2-j)(\boldsymbol{1}\otimes\boldsymbol{\mu}_1\otimes\boldsymbol{1})+{}\right.\\
\left.{}+
u(j-1)(\boldsymbol{1}\otimes\boldsymbol{1}\otimes\boldsymbol{\mu}_2)\right]\,,\,\,
j=1,2\,.
\end{multline*}

Далее заметим, что вероятность выхода заявок из прибора~$j$ равна $\lambda_{D}(j)/\lambda_{D}$, $j=1,2$, 
где интенсивность выходящего из системы потока
\begin{equation*}
\lambda_D=\lambda_D(1)+\lambda_D(2)\,.
\end{equation*}

Таким образом, подытоживая рассуждения, получаем, что справедлива

\smallskip

\noindent
\textbf{Теорема 2.} \textit{Преобразование Лапласа--Стилтьеса ФР задержки переупорядочивания в СМО $PH/PH/2/r/res$ в стационарном режиме 
ее работы определяется выражением
\begin{equation*}
f(s)=\fr{1}{\lambda_D}\left[\lambda_D(1)f_1(s)+\lambda_D(2)f_2(s)\right]\,,
\end{equation*}
где $f_j(s)$, $j=1,2$, задаются формулами}~(\ref{eq:16}) \textit{и}~(\ref{eq:17}).

\smallskip

Обозначим через~$\delta_{\nu}$ начальный момент порядка~$\nu$, $\nu=1,2,\ldots$, 
задержки переупорядочивания заявки в исследуемой СМО. Тогда из теоремы~2 с 
учетом~(\ref{eq:18})--(\ref{eq:20}) и~[1, с.~104] получаем очевидное

\medskip

\noindent
\textbf{Следствие.} {\it{Начальный момент порядка~$\nu$ задержки переупорядочивания в 
СМО $PH/PH/2/r/res$ определяется выражением}}
\begin{multline}
\label{eq:21}
\delta_{\nu}=\fr{(-1)^{\nu}\nu!}{\lambda_D}\,
\sum\limits_{k=2}^{r+2}\left[\boldsymbol{p}_{k2}^{\mathrm{T}}
\left(\boldsymbol{1}\otimes\boldsymbol{\mu}_1\otimes\boldsymbol{M}_2^{-\nu} \boldsymbol{1}\right)+{}\right.\\
\left.{}+\boldsymbol{p}_{k1}^{\mathrm{T}}
\left(\boldsymbol{1}\otimes\boldsymbol{M}_1^{-\nu} \boldsymbol{1}\otimes\boldsymbol{\mu}_2\right)\right]\,.
\end{multline}



Теперь остановимся на связи средней задержки переупорядочивания со средним числом заявок в БП.

\bigskip

\noindent
{\bf{Теорема 3.}} {\it{Средняя величина задержки переупорядочивания~$\delta_1$ и среднее число заявок~$v_1$ 
в БП СМО $PH/PH/2/r/res$ связаны соотношением
\begin{equation}
\label{eq:22}
\delta_1=\fr{1}{\lambda_d}\,v_1\,.
\end{equation}
}}

\medskip

\noindent
Д\,о\,к\,а\,з\,а\,т\,е\,л\,ь\,с\,т\,в\,о\,.\ \ Умножим уравнения системы~(\ref{eq:12})--(\ref{eq:14}) 
при фиксированном $i=1,2$ и $\nu=1$ справа на матрицу~$\boldsymbol{Z}_i$:
\begin{equation*}
\boldsymbol{Z}_i=
\left.
\begin{cases}
\boldsymbol{1}\otimes \boldsymbol{I}\otimes \boldsymbol{1}\,,\enskip i=1\,,\\[6pt]
 \boldsymbol{1}\otimes \boldsymbol{1}\otimes \boldsymbol{I}\,,\enskip i=2\,.
\end{cases}\!\!\!\!\!\!
\right \}
\end{equation*}

В результате умножения получим следующую систему:
\begin{equation}
\label{eq:23}
\boldsymbol{0}^{\mathrm{T}}=\boldsymbol{v}_{1i1}^{\mathrm{T}}(-\boldsymbol{L}_{1i}+\boldsymbol{U}_i)+
(\boldsymbol{v}_{2i1}^{\mathrm{T}}+\boldsymbol{p}_{2i}^{\mathrm{T}})\boldsymbol{N}_i,\,\, i=1,2;
\end{equation}
\begin{multline}
\label{eq:24}
\boldsymbol{0}^{\mathrm{T}}=u(3-k)\boldsymbol{v}_{1i1}^{\mathrm{T}}\boldsymbol{L}_{1i}+
u(k-2)\boldsymbol{v}_{k-1,i1}^{\mathrm{T}}\boldsymbol{L}_{k-1,i}+{}\\
{}+\boldsymbol{v}_k^{\mathrm{T}}(-\boldsymbol{L}_{ki}+\boldsymbol{U}_i-\boldsymbol{N}_i)
+(\boldsymbol{v}_{k+1,i1}^{\mathrm{T}}+\boldsymbol{p}_{k+1,i}^{\mathrm{T}})\boldsymbol{N}_i\,,\\
i=1,2\,,\enskip k=\overline{2,r+1}\,;
\end{multline}
\begin{equation}
\label{eq:25}
\boldsymbol{0}^{\mathrm{T}}=\boldsymbol{v}_{r+1}^{\mathrm{T}}\boldsymbol{L}_{r+1,i}+
\boldsymbol{v}_{r+2}^{\mathrm{T}}(\boldsymbol{U}_i-\boldsymbol{N}_i)\,,\enskip i=1,2\,,
\end{equation}
где
\begin{align*}
\boldsymbol{L}_{ki}&=
\begin{cases}
\boldsymbol{\lambda}_k^{\mathrm{T}}\otimes\boldsymbol{I}\otimes\boldsymbol{1}\,, & i=1\,,\\[6pt]
\boldsymbol{\lambda}_k^{\mathrm{T}}\otimes\boldsymbol{1}\otimes\boldsymbol{I}\,, &i=2\,,\,\,
k=\overline{1,r+1}\,;
\end{cases}
\\
\boldsymbol{U}_i&=
\begin{cases}
\boldsymbol{1}\otimes\boldsymbol{M}_1\otimes\boldsymbol{1}\,, \enskip i=1\,,\\[6pt]
\boldsymbol{1}\otimes\boldsymbol{1}\otimes\boldsymbol{M}_2\,,\enskip i=2\,;
\end{cases}
\\
\boldsymbol{N}_i&=
\begin{cases}
\boldsymbol{1}\otimes\boldsymbol{I}\otimes\boldsymbol{\mu}_2\,,\enskip i=1\,,\\[6pt]
\boldsymbol{1}\otimes\boldsymbol{\mu}_1\otimes\boldsymbol{I}\,,\enskip i=2\,.
\end{cases}
\end{align*}

Последовательно суммируя уравнения системы~(\ref{eq:23})--(\ref{eq:25}), 
после несложных алгебраических преобразований приходим к следующим соотношениям:
\begin{equation}
\label{eq:26}
-\sum\limits_{k=1}^{r+2}\boldsymbol{v}_{ki1}^{\mathrm{T}}\boldsymbol{U}_i=
\sum\limits_{k=2}^{r+2}\boldsymbol{p}_{ki}^{\mathrm{T}}\boldsymbol{N}_i\,,\enskip i=1,2\,.
\end{equation}


Умножим~(\ref{eq:26}) при фиксированном $i=1,2$ справа на матрицу
\begin{equation*}
\boldsymbol{T}_i=
\begin{cases}
\boldsymbol{I}\otimes\boldsymbol{M}_1^{-1}\boldsymbol{1}\otimes\boldsymbol{I}\,, \enskip i=1\,\\[6pt]
\boldsymbol{I}\otimes\boldsymbol{I}\otimes\boldsymbol{M}_2^{-1}\boldsymbol{1}\,, \enskip i=2,
\end{cases}
\end{equation*}
и просуммируем по $i$ полученные выражения. В~результате, учитывая (\ref{eq:10}) и~(\ref{dd}), 
приходим к сле\-ду\-ющему равенству:
\begin{multline}
\label{eq:27}
v_1=-\sum\limits_{k=2}^{r+2}\left[\boldsymbol{p}_{k1}^{\mathrm{T}}
(\boldsymbol{1}\otimes\boldsymbol{M}_1^{-1}\boldsymbol{1} \otimes\boldsymbol{\mu}_2)+{}\right.\\
\left.{}+
\boldsymbol{p}_{k2}^{\mathrm{T}}
(\boldsymbol{1}\otimes\boldsymbol{\mu}_1 \otimes\boldsymbol{M}_2^{-1}\boldsymbol{1})\right]\,.
\end{multline}

Из сравнения~(\ref{eq:27}) и~(\ref{eq:21}) при $\nu=1$ очевидным образом вытекает~(\ref{eq:22}). 
Таким образом, теорема доказана.

%\medskip

Заметим, что~(\ref{eq:22}) является аналогом известной формулы Литтла и имеет вполне очевидную физическую интерпретацию.

{\small\frenchspacing
{%\baselineskip=10.8pt
\addcontentsline{toc}{section}{Литература}
\begin{thebibliography}{9}

\bibitem{1mat} 
\Au{Бочаров П.\,П., Печинкин А.\,В.} 
Теория массового обслуживания.~--- М.:  РУДН, 1995.

\bibitem{2mat} 
\Au{Матюшенко С.\,И.} 
Анализ двухканальной системы обслуживания ограниченной емкости с буфером переупорядочивания и с распределениями фазового типа~// 
Вестник РУДН: Прикладная математика. Информатика. Физика, 2010. №\,4. С.~84--88.

 \bibitem{3mat} 
 \Au{Наумов В.\,А.} 
 Численные методы анализа марковских систем. --- М.: УДН, 1985.
 
 \label{end\stat}

 \bibitem{4mat} 
 \Au{Наумов В.\,А.} 
 О предельных вероятностях полумарковского процесса~// Современные задачи в точных науках.~--- М.: УДН, 1975. С.~35--39.
 \end{thebibliography}
}
}


\end{multicols}