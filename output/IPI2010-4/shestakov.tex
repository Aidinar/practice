
\def\stat{shestakov}

\def\tit{АППРОКСИМАЦИЯ РАСПРЕДЕЛЕНИЯ ОЦЕНКИ РИСКА ПОРОГОВОЙ
ОБРАБОТКИ ВЕЙВЛЕТ-КОЭФФИЦИЕНТОВ НОРМАЛЬНЫМ
РАСПРЕДЕЛЕНИЕМ ПРИ~ИСПОЛЬЗОВАНИИ ВЫБОРОЧНОЙ ДИСПЕРСИИ$^*$}

\def\titkol{Аппроксимация распределения оценки риска пороговой
обработки вейвлет-коэффициентов} 
%нормальным распределением при~использовании выборочной дисперсии


\def\autkol{О.\,В.~Шестаков}
\def\aut{О.\,В.~Шестаков$^1$}

\titel{\tit}{\aut}{\autkol}{\titkol}

{\renewcommand{\thefootnote}{\fnsymbol{footnote}}\footnotetext[1]
{Работа выполнена при финансовой поддержке РФФИ (грант 08-01-00567).}}

\renewcommand{\thefootnote}{\arabic{footnote}}
\footnotetext[1]{Московский государственный университет им.\ М.\,В.~Ломоносова, 
кафедра математической статистики факультета вычислительной математики и кибернетики, oshestakov@cs.msu.su}

\vspace*{6pt}

\Abst{Исследуются асимптотические свойства оценки риска при
пороговой обработке коэффициентов вейвлет-разложения функции сигнала. 
Получены некоторые оценки скорости сходимости распределения оценки риска к нормальному закону.}

\vspace*{3pt}

\KW{вейвлеты; пороговая обработка; оценка риска; нормальное распределение; оценка скорости сходимости
}

\vspace*{12pt}

       \vskip 14pt plus 9pt minus 6pt

      \thispagestyle{headings}

      \begin{multicols}{2}

      \label{st\stat}


\section{Введение}

Популярность методов вейвлет-обработки сигналов и изображений значительно возросла за последние десятилетия. 
Объясняется это тем, что вейв\-лет-разложение представляет собой удобный математический аппарат, способный решать 
те задачи, в которых применение традиционного Фурье-ана\-ли\-за оказывается неэффективным. В настоящее время вейвлеты 
применяются в самых разнообразных областях, включая геофизику, оптику, физику плазмы, вычислительную томографию, 
компьютерную графику и~т.\,д. 

Основные задачи, для решения которых используются вейвлеты,~--- 
это сжатие сигналов/изображений и удаление шума. При этом чаще всего используется пороговая 
обработка вейвлет-коэффициентов, которая обнуляет коэффициенты, не превышающие заданного порога. 
Наличие шума и процедуры пороговой обработки неизбежно приводят к погрешностям в оце\-ни\-ва\-емом 
сигнале/изображении. Свойства оценки таких погрешностей (риска) исследовались во многих работах 
(см., например,~[1--7]). При определенных условиях оценка риска является асимптотически нормальной (см.~\cite{7she}). 

В~данной работе исследуется вопрос о скорости сходимости распределения оценки риска к нормальному закону в одномерном 
случае (т.\,е.\ при обработке одномерных сигналов).

\section{Свойства коэффициентов вейвлет-разложения}

При использовании вейвлет-разложения функция $f\in L^2(\mathbf{R})$, описывающая сигнал, 
представляется в виде ряда из сдвигов и растяжений некоторой вейвлет-функ\-ции~$\psi$:
\begin{equation}
f=\sum_{j,k\in Z}\langle f,\psi_{j,k}\rangle\psi_{j,k}\,,
\label{e1she}
\end{equation}
где $\psi_{j,k}(x)=2^{j/2}\psi(2^jx-k)$, а семейство $\{\psi_{j,k}\}_{j,k\in Z}$ образует ортонормированный 
базис в $L^2(\mathbf{R})$. Индекс~$j$ в~(\ref{e1she}) называется масштабом, а индекс~$k$~--- сдвигом. 
Функция~$\psi$ должна удовлетворять определенным требованиям~\cite{8she}, 
однако ее можно выбрать таким образом, чтобы она обладала некоторыми полезными свойствами, 
например была дифференцируемой нужное число раз и имела заданное число $M$ нулевых моментов (см.~\cite{8she}):
$$
\int\limits_{-\infty}^{\infty}x^k\psi(x)\,dx=0\,,\enskip k=0,\ldots,M-1\,.
$$

В дальнейшем будут рассматриваться функции сигнала $f\in L^2(\mathbf{R})$ на конечном отрезке $[a,b]$, 
равномерно регулярные по Липшицу с некоторым параметром $\gamma>0$, т.\,е.\ 
такие функции, для которых существует константа $L>0$ и полином $P_y$ степени 
$n=\lfloor\gamma\rfloor$ такой, что для любого $y\in[a,b]$ и любого $x\in\mathbf{R}$
$$
\left\vert f(x)-P_y(x)\right\vert\leqslant L\left\vert x-y\right\vert^\gamma\,.
$$

Для таких функций~$f$ известно~\cite{9she}, что если вейвлет-функция $M$ раз непрерывно 
дифференцируема ($M\geqslant\gamma$), имеет $M$ нулевых моментов и быстро убывает на бесконечности 
вместе со своими производными, т.\,е.\ для всех $0\leqslant k \leqslant M$ и любого $m\in N$ найдется константа~$C_m$, 
что при всех $x\in\mathbf{R}$
$$
\left\vert\psi^{(k)}(x)\right\vert\leqslant\fr{C_m}{1+\left\vert x\right\vert^m}\,,
$$
то найдется такая константа $A>0$, что
\begin{equation}
\langle f,\psi_{j,k}\rangle\leqslant\fr{A}{2^{j\left(\gamma+{1}/{2}\right)}}\,.
\label{e2she}
\end{equation}

На практике функции сигнала всегда заданы в дискретных отсчетах на конечном отрезке. 
Не ограничивая общности, будем считать, что это отрезок $[0,1]$ и функция~$f$ задана в 
точках ${i}/{N}$ ($i=1,\ldots N$, где $N=2^J$ для некоторого~$J$): $f_i=f\left({i}/{N}\right)$.
Дискретное вейвлет-преобразование представляет собой умножение вектора значений функции~$f$ 
(обозначим его через~$\overline{f}$) на ортогональную матрицу~$W$, определяемую вейвлет-функцией~$\psi$: 
$\overline{f}^{W}=W\overline{f}$ (см.~\cite{9she}). При этом если перейти к двойному индексу~$(j,k)$, 
как в непрерывном случае, то дискретные вейвлет-коэффициенты будут связаны с непрерывными следующим образом: 
$f^{W}_{j,k}\approx \sqrt{N}\langle f,\psi_{j,k}\rangle$ (см., например,~\cite{1she} или~\cite{10she}). 
Это приближение тем точнее, чем больше~$N$. Не будем обсуждать методы борьбы с краевыми эффектами, 
связанными с использованием вейвлет-разложения на конечном отрезке. Познакомиться с этими методами можно, 
например, в~\cite{11she}. В~дальнейшем для удобства будем нумеровать дискретные вейвлет-коэффициенты так же, 
как отсчеты функции~$f$, одним индексом~$i$ вместо двойного индекса~$(j,k)$.

В реальных наблюдениях всегда присутствует шум. Будем рассматривать следующую модель:
$$
Y_i=f_i+z_i\,,\quad i=1,\ldots,N\,,
$$
где $z_i$~--- независимые случайные величины,\linebreak име\-ющие нормальное распределение с нулевым средним и 
дисперсией~$\sigma^2$. Тогда в силу ортогональности матрицы~$W$ для дискретных вейвлет-коэффициентов 
примем следующую модель:
$$
Y^W_i=f^{W}_{i}+z^W_i\,,\quad i=1,\ldots,N\,,
$$
где $z^W_i$ так\-же независимы и нормально распределены с нулевым средним и дисперсией~$\sigma^2$, а $f^{W}_{i}$ равны 
соответствующим непрерывным вейвлет-ко\-эф\-фи\-ци\-ен\-там, умноженным на~$\sqrt{N}$.

\section{Пороговая обработка и оценка риска}

Смысл пороговой обработки вейвлет-ко\-эф\-фи\-ци\-ен\-тов заключается в удалении достаточно маленьких 
коэффициентов, которые считаются шумом. Будем использовать так называемую\linebreak мягкую пороговую обработку с порогом~$T$. 
К~каждому вейвлет-коэффи\-циен\-ту применяется функция $\rho_T(x)=sgn(x)\left(|x|-T\right)_{+}$, 
т.\,е.\ при такой пороговой обработке коэффициенты, которые по модулю меньше порога~$T$, обнуляются, 
а абсолютные величины остальных коэффициентов уменьшаются на величину порога.
Погрешность (или риск) мягкой пороговой обработки определяется следующим образом:
\begin{equation}
R_N(f)=\sum_{i=1}^{N}\mbox{E}\left(f^{W}_{i}-\rho_T(Y^W_i)\right)^2\,.
\label{e3she}
\end{equation}
В выражении~(\ref{e3she}) присутствуют неизвестные величины~$f^{W}_{i}$, поэтому вычислить значение~$R_N(f)$ нельзя. 
Однако его можно оценить. В каж\-дом слагаемом если $|Y^W_i|>T$, то вклад этого слагаемого в риск составляет 
$\sigma^2+T^2$, а если $|Y^W_i|\leqslant T$, то вклад составляет~$(f_i^W)^2$. 
Поскольку $\mbox{E}(Y_i^W)^2=\sigma^2+(f_i^W)^2$, величину $(f_i^W)^2$ можно оценить разностью $(Y_i^W)^2-\sigma^2$.

Таким образом, в качестве оценки риска можно использовать следующую величину:
\begin{equation}
\widetilde{R}_N(f)=\sum_{i=1}^{N}F[(Y_i^W)^2]\,,
\label{e4she}
\end{equation}
где 
$$
F[x]=(x-\sigma^2)\Ik_{|x|\leqslant T}+(\sigma^2+T^2)\mathbb{1}_{|x|>T}\,.
$$
Для так определенной оценки риска справедливо следующее утверждение (см.~\cite{9she}).

\medskip

\noindent
\textbf{Теорема 1.} $\mbox{E}\widetilde{R}_N(f)=R_N(f)$, \textit{т.\,е.\ $\widetilde{R}_N(f)$ является 
несмещенной оценкой для $R_N(f)$}.

\medskip

В работах~\cite{2she} и~\cite{3she} было предложено использовать порог $T=\sigma\sqrt{2\ln N}$. 
Было показано, что при таком пороге риск близок к минимальному (см.~\cite{2she}). 
Этот порог получил название <<универсальный>>. В~дальнейшем будем ориентироваться именно на такой вид порога.

Зачастую дисперсия~$\sigma^2$ не известна и ее также необходимо оценивать, при этом выражения~(\ref{e4she}) 
принимают вид
\begin{equation}
\widehat{R}_N(f)=\sum_{i=1}^{N}\widehat{F}[(Y_i^W)^2]\,,
\label{e5she}
\end{equation}
где
$$
\widehat{F}[x]=(x-\hat{\sigma}^2)\Ik_{|x|\leqslant\hat{T}}+(\hat{\sigma}^2+\hat{T}^2)\Ik_{|x|>\hat{T}}\,,
$$
а
$$
\hat{T}=\hat{\sigma}\sqrt{2\ln N}\,.
$$

Обычно дисперсия~$\sigma^2$ оценивается по выборке сигнала, однако ее можно оценить и по независимой выборке. 
Для этого следует произвести измерение пустого сигнала, тогда наблюдения будут представлять собой чистый шум, 
по которому и оценивается~$\sigma^2$. В~следующих разделах будут рассмотрены оба случая.

\section{Оценка скорости сходимости распределения оценки риска к~нормальному закону}

В работах~\cite{7she, 6she} исследуется поведение оценки риска~$\widehat{R}_N(f)$ при 
использовании различных оценок дисперсии~$\hat{\sigma}^2$. В~частности, показано, что при достаточно 
общих требованиях к~$\hat{\sigma}^2$ оценка риска является асимптотически нормальной. 
В~этом разделе будут получены некоторые оценки скорости сходимости распределения~$\widetilde{R}_N(f)$ 
к нормальному закону при использовании в качестве~$\hat{\sigma}^2$ выборочной дисперсии, которая строится 
по независимой выборке $(Y'_1,\ldots,Y'_N)$:
$$
\hat{\sigma}^2=\fr{1}{N-1}\sum\limits_{i=1}^{N}(Y'_i-\overline{Y'})^2\,,\ \mbox{где}\ 
\overline{Y'}=\fr{1}{N}\sum_{i=1}^{N}Y'_i\,.
$$

Далее для удобства будем обозначать~$Y_i^W$ через~$X_i$, а~$f_i^W$~--- через~$a_i$.

\medskip

\noindent

\textbf{Теорема 2.} \textit{Пусть $f\in L^2(\mathbf{R})$ задана на отрезке $[0,1]$ и является равномерно 
регулярной по Липшицу с параметром $\gamma=\fr{1}{2}+\alpha$ ($\alpha>0$) и пусть выборочная дисперсия~$\hat{\sigma}^2$ 
не зависит от наблюдений~$X_i$, тогда существует такая константа~$C_0$ (зависящая от~$\alpha$, $A$ и~$\sigma$), что
\begin{multline}
\sup\limits_{x\in\mathbf{R}}\left\vert\mbox{P}\left(\fr{\widehat{R}_N(f)-R_N(f)}{\sigma^2\sqrt{2N}}<x\right)-\Phi_2(x)\right\vert\leqslant{}\\
{}\leqslant\fr{C_0(\ln N)^{{1}/{2}}}{N^{{1}/{4}-{1}/(4(\alpha+1))}}\,,
\label{e6she}
\end{multline}
где $\Phi_2(x)$~--- функция распределения нормального закона с нулевым средним и дисперсией, равной~2.
}

\medskip

\noindent
Д\,о\,к\,а\,з\,а\,т\,е\,л\,ь\,с\,т\,в\,о\,.\ Пользуясь теоремой~1, запишем разность $\widehat{R}_N(f)-R_N(f)$ в виде
$$
\widehat{R}_N(f)-R_N(f)=S_N+V_N\,,
$$
где
\begin{multline*}
S_N=\sum_{i=1}^{N}\left(X_i^2\Ik_{|X_i|\leqslant \hat{T}}-\mbox{E}X_i^2\Ik_{|X_i|\leqslant T}\right)+{}\\
{}+2\sum_{i=1}^{N}\left(\hat{\sigma}^2\Ik_{|X_i|> \hat{T}}-\mbox{E}\sigma^2\Ik_{|X_i|> T}\right)+{}\\
{}+\sum_{i=1}^{N}\left(\hat{T}^2\Ik_{|X_i|> \hat{T}}-\mbox{E}T^2\Ik_{|X_i|> T}\right)\,,
\end{multline*}
а
$$V_N=N\left(\sigma^2-\hat{\sigma}^2\right)\,.
$$
Рассмотрим $S_N$. Разобьем это слагаемое на три суммы~$U_N$, $W_N$ и~$Z_N$:
\begin{align*}
U_N&=\sum_{i\in I_1}\left(X_i^2-\mbox{E}X_i^2\right)\,;\\
W_N&=-\sum_{i\in I_1}\left(X_i^2\Ik_{|X_i|> \hat{T}}-\mbox{E}X_i^2\Ik_{|X_i|>T}\right)+{}\\
&{}+2\sum_{i\in I_1}\left(\hat{\sigma}^2\Ik_{|X_i|> \hat{T}}-\mbox{E}\sigma^2\Ik_{|X_i|> T}\right)+{}\\
&{}+ \sum_{i\in I_1}\left(\hat{T}^2\Ik_{|X_i|> \hat{T}}-\mbox{E}T^2\Ik_{|X_i|> T}\right)\,;\\
Z_N&=\sum_{i\in I_2}\left(X_i^2\Ik_{|X_i|\leqslant \hat{T}}-\mbox{E}X_i^2\Ik_{|X_i|\leqslant T}\right)+{}\\
&{}+2\sum_{i\in I_2}\left(\hat{\sigma}^2\Ik_{|X_i|> \hat{T}}-\mbox{E}\sigma^2\Ik_{|X_i|> T}\right)+{}\\
&{}+\sum_{i\in I_2}\left(\hat{T}^2\Ik_{|X_i|> \hat{T}}-\mbox{E}T^2\Ik_{|X_i|> T}\right)\,,
\end{align*}
где $I_1$~--- множество тех~$i$, для которых в силу~(\ref{e2she}) выполнено $|a_i|\leqslant{A}/{N^\alpha}$, а 
$I_2$~--- множество остальных~$i$. Оценим сумму $W_N+Z_N$.

При произвольном $\varepsilon>0$, используя неравенство Чебышева, имеем
$$
\mbox{P}\left(\left\vert W_N+Z_N\right\vert>\varepsilon\right)\leqslant\fr{\mbox{E}\left\vert W_N\right\vert+
\mbox{E}\left\vert Z_N\right\vert}{\varepsilon}\,.
$$
Рассмотрим $\mbox{E}\left\vert Z_N\right\vert$.
\begin{multline*}
\mbox{E}\left\vert Z_N\right\vert\leqslant\sum_{i\in I_2}\mbox{E}
\left\vert X_i^2\Ik_{|X_i|\leqslant \hat{T}}-\mbox{E}X_i^2\Ik_{|X_i|\leqslant T}\right\vert+{}\\
{}+2\sum_{i\in I_2}\mbox{E}\left\vert\hat{\sigma}^2
\Ik_{|X_i|>\hat{T}}-\mbox{E}\sigma^2\Ik_{|X_i|>T}\right\vert+{}\\
{}+\sum_{i\in I_2}\mbox{E}\left\vert\hat{T}^2\Ik_{|X_i|>\hat{T}}-\mbox{E}T^2\Ik_{|X_i|> T}\right\vert\,.
\end{multline*}
Поскольку $f$ регулярна по Липшицу с $\gamma={1}/{2}+\alpha$, число слагаемых в каждой из этих трех сумм не 
превосходит $B_1 N^{{1}/(2(\alpha+1))}$, где $B_1$~--- некоторая константа, зависящая от~$\alpha$. 
Слагаемые в первой и третьей суммах не превосходят $\sigma^2\ln N$, а слагаемые во второй сумме не превосходят~$\sigma^2$. 
Следовательно, $\mbox{E}|Z_N|$ не превосходит $B_2N^{{1}/(2(\alpha+1))}\ln N$ для некоторой константы~$B_2$.

Оценим теперь $\mbox{E}|W_N|$.
\begin{multline}
\mbox{E}\left\vert W_N\right\vert\leqslant\sum_{i\in I_1}\mbox{E}
\left\vert X_i^2\Ik_{|X_i|> \hat{T}}-\mbox{E}X_i^2\Ik_{|X_i|> T}\right\vert+{}\\
{}+2\sum_{i\in I_1}\mbox{E}
\left\vert\hat{\sigma}^2\Ik_{|X_i|> \hat{T}}-\mbox{E}\sigma^2\Ik_{|X_i|> T}\right\vert+{}\\
{}+\sum_{i\in I_1}\mbox{E}
\left\vert\hat{T}^2\Ik_{|X_i|> \hat{T}}-\mbox{E}T^2\Ik_{|X_i|> T}\right\vert\,.
\label{e7she}
\end{multline}
Оценим первую сумму. Имеем
\begin{multline}
\mbox{E}\left\vert X_i^2\Ik_{|X_i|> \hat{T}}-\mbox{E}X_i^2\Ik_{|X_i|> T}\right\vert\leqslant{}\\
{}\leqslant
\mbox{E}\left\vert X_i^2\Ik_{|X_i|> \hat{T}}-X_i^2\Ik_{|X_i|> T}\right\vert+{}\\
{}+\mbox{E}\left\vert X_i^2\Ik_{|X_i|> T}-\mbox{E}X_i^2\Ik_{|X_i|> T}\right\vert\,.
\label{e8she}
\end{multline}
Рассмотрим первое слагаемое. В~силу незави\-си\-мости~$X_i$ и~$\hat{T}$ при достаточно больших~$N$ (таких, что $T-a_i>0$)
\begin{multline*}
\mbox{E}\left\vert X_i^2\Ik_{|X_i|> \hat{T}}-X_i^2\Ik_{|X_i|> T}\right\vert={}\\[1pt]
{}=
\mbox{E}X_i^2\left\vert\Ik_{\hat{T}\geqslant|X_i|> T}+\Ik_{T\geqslant|X_i|> \hat{T}}\right\vert\leqslant{}\\[1pt]
{}\leqslant \mbox{E}\hat{T}^2\Ik_{\hat{T}\geqslant|X_i|> T}+\mbox{E}T^2\Ik_{T\geqslant|X_i|> \hat{T}}\leqslant{}\\[1pt]
{}\leqslant\fr{1}{\sqrt{2\pi\sigma^2}}\left(
\mbox{E}\hat{T}^2 e^{-{(T-a_i)^2}/(2\sigma^2)}|\hat{T}-T|+{}\right.\\[1pt]
\left.{}+
\mbox{E}T^2|\hat{T}-T|\Ik_{\hat{T}\leqslant a_i}+{}\right.\\[1pt]
\left.{}+\mbox{E}T^2e^{-{(\hat{T}-a_i)^2}/(2\sigma^2)}
|\hat{T}-T|\Ik_{\hat{T}>a_i}\right)\,.
\end{multline*}
Обозначим $\beta_{i,N}={a_i}/(\sigma\sqrt{2\ln N}).$ Тогда начиная с некоторого~$N$
\begin{multline*}
\mbox{E}T^2e^{-{(\hat{T}-a_i)^2}/({2\sigma^2})}|\hat{T}-T|\Ik_{\hat{T}>a_i}={}\\[1pt]
{}=2\sqrt{2}\sigma^2(\ln N)^{{3}/{2}}\mbox{E}|\hat{\sigma}-\sigma|e^{-\ln N\left({\hat{\sigma}}/{\sigma}-\beta_{i,N}\right)^2}\Ik_{\hat{T}>a_i}={}\\[1pt]
{}=2\sqrt{2}\sigma^2(\ln N)^{{3}/{2}}\mbox{E}|\hat{\sigma}-\sigma|
e^{-\ln N\left({\hat{\sigma}}/{\sigma}-\beta_{i,N}\right)^2}\times{}\\[1pt]
{}\times
\left(\Ik_{\hat{\sigma}/{\sigma}\geqslant{1}/{\sqrt{2}}+\beta_{i,N}}+
\Ik_{\beta_{i,N}<{\hat{\sigma}}/{\sigma}< 1/\sqrt{2}+\beta_{i,N}}\right)\leqslant{}\\[1pt]
{}\leqslant2\sqrt{2}\sigma^2(\ln N)^{{3}/{2}}\left(e^{-\ln N/2}\mbox{E}|\hat{\sigma}-\sigma|+{}\right.\\[1pt]
\left.{}+
\mbox{E}|\hat{\sigma}-\sigma|\Ik_{\beta_{i,N}<{\hat{\sigma}}/\sigma<1/\sqrt{2}+\beta_{i,N}}\right)\leqslant{}
\end{multline*}
\begin{multline*}
{}\leqslant2\sqrt{2}\sigma(\ln N)^{3/2}\left(\fr{\sqrt{\mbox{D}\hat{\sigma}^2}}{N^{1/2}}+{}\right.\\
\left.{}+
\sqrt{\mbox{D}\hat{\sigma}^2\mbox{P}\left(\beta_{i,N}<
\fr{\hat{\sigma}}{\sigma}<\fr{1}{\sqrt{2}}+\beta_{i,N}\right)}\right)\,.
\end{multline*}
При переходе к последнему неравенству использовалось то, что

\noindent
$$
\mbox{E}|\hat{\sigma}-\sigma|^2=\mbox{E}\fr{(\hat{\sigma}^2-\sigma^2)^2}{(\hat{\sigma}+\sigma)^2}\leqslant
\mbox{E}\fr{(\hat{\sigma}^2-\sigma^2)^2}{\sigma^2}=\fr{\mbox{D}\hat{\sigma}^2}{\sigma^2}\,.
$$
Оценим вероятность, входящую в подкоренное выражение.

\noindent
\begin{multline*}
\mbox{P}\left(\beta_{i,N}<\frac{\hat{\sigma}}{\sigma}<\fr{1}{\sqrt{2}}+\beta_{i,N}\right)\leqslant{}\\
{}\leqslant
\mbox{P}\left(\fr{\hat{\sigma}}{\sigma}<\fr{1}{\sqrt{2}}+\beta_{i,N}\right)={}\\
{}=\mbox{P}\left(\fr{\hat{\sigma}^2}{\sigma^2}<\left(\fr{1}{\sqrt{2}}+\beta_{i,N}\right)^2\right)={}\\
{}=\mbox{P}\left(\sigma^2-\hat{\sigma}^2>\sigma^2\left(1-\left(\frac{1}{\sqrt{2}}+\beta_{i,N}\right)^2\right)\right)\leqslant{}\\
{}\leqslant\mbox{P}\left(\left\vert\sigma^2-\hat{\sigma}^2\right\vert>\sigma^2
\left(1-\left(\fr{1}{\sqrt{2}}+\beta_{i,N}\right)^2\right)\right)\leqslant{}\\
{}\leqslant\fr{\mbox{D}\hat{\sigma}^2}{\sigma^4\left(1-\left(\frac{1}{\sqrt{2}}+\beta_{i,N}\right)^2\right)^2}\,.
\end{multline*}
Последнее неравенство справедливо, поскольку $1-\left({1}/{\sqrt{2}}+\beta_{i,N}\right)^2>0$ начиная с некоторого~$N$.
В~результате, учитывая порядок дисперсии оценки~$\hat{\sigma}^2$~\cite{12she}, 
имеем, что для некоторой константы~$C_1$ справедливо
\begin{equation}
\mbox{E}T^2e^{-{(\hat{T}-a_i)^2}/(2\sigma^2)}|\hat{T}-T|\Ik_{\hat{T}>a_i}\leqslant\fr{C_1(\ln N)^{\frac{3}{2}}}{N}\,.
\label{e9she}
\end{equation}
Далее, начиная с некоторого~$N$
\begin{multline}
\mbox{E}T^2|\hat{T}-T|\Ik_{\hat{T}\leqslant a_i}={}\\[1pt]
{}=
2\sqrt{2}\sigma^3(\ln N)^{{3}/{2}}\mbox{E}\left\vert\fr{\hat{\sigma}}{\sigma}-1\right\vert
\Ik_{{\hat{\sigma}}/\sigma\leqslant\beta_{i,N}}\leqslant{}\\[1pt]
{}\leqslant 2\sqrt{2}\sigma^3(\ln N)^{{3}/{2}}\mbox{E}
\Ik_{{\hat{\sigma}}/\sigma\leqslant\beta_{i,N}}={}\\[1pt]
{}=2\sqrt{2}\sigma^3(\ln N)^{3/2}\mbox{P}\left(\sigma^2-\hat{\sigma}^2\geqslant\sigma^2(1-\beta^2_{i,N})\right)\leqslant {}\\[1pt]
{}\leqslant
2\sqrt{2}(\ln N)^{3/2}\fr{\mbox{D}\hat{\sigma}^2}{\sigma(1-\beta^2_{i,N})^2}\leqslant
\fr{C_2(\ln N)^{3/2}}{N}
\label{e10she}
\end{multline}
для некоторой константы~$C_2$.
\pagebreak

Наконец, для некоторых констант~$C_3$ и~$C_4$
\begin{multline}
\mbox{E}\hat{T}^2 e^{-{(T-a_i)^2}/(2\sigma^2)}|\hat{T}-T|\leqslant{}\\
{}\leqslant\fr{C_3(\ln N)^{3/2}}{N}\mbox{E}\hat{\sigma}^2|\hat{\sigma}-\sigma|\leqslant{}\\
{}\leqslant \fr{C_3(\ln N)^{3/2}}{N\sigma}\sqrt{\mbox{D}\hat{\sigma}^2\mbox{E}\hat{\sigma}^4}\leqslant
\fr{C_4(\ln N)^{3/2}}{N^{3/2}}\,.
\label{e11she}
\end{multline}
Рассмотрим теперь второе слагаемое в~(\ref{e8she}). Если в разности оба члена положительны, 
то модуль раз\-ности не превосходит максимума из модулей каж\-до\-го из них, поэтому
\begin{multline*}
\mbox{E}\left\vert X_i^2\Ik_{|X_i|> T}-\mbox{E}X_i^2\Ik_{|X_i|> T}\right\vert\leqslant\mbox{E}
\left\vert X_i^2\Ik_{|X_i|> T}\right\vert\leqslant{}\\
{}\leqslant\fr{2}{\sqrt{2\pi\sigma^2}}\int\limits_{T}^{\infty}u^2e^{-{(u-|a_i|)^2}/(2\sigma^2)}\,du={}\\
{}=\fr{2}{\sqrt{2\pi\sigma^2}}\int\limits_{T-|a_i|}^{\infty}(v+|a_i|)^2e^{-{v^2}/(2\sigma^2)}\,dv={}\\
{}=
\fr{2}{\sqrt{2\pi\sigma^2}}\int\limits_{T-|a_i|}^{\infty}(v^2+2v|a_i|+a_i^2)e^{-{v^2}/(2\sigma^2)}\,dv\,;
\end{multline*}

\vspace*{-12pt}

\noindent
\begin{multline*}
\int\limits_{T-|a_i|}^{\infty}v^2e^{-{v^2}/(2\sigma^2)}\,dv=
-\sigma^2ve^{-{v^2}/(2\sigma^2)}\vert^{\infty}_{T-|a_i|}+{}\\
{}+\;\sigma^2\!\!\int\limits_{T-|a_i|}^{\infty}\!\!
e^{-{v^2}/(2\sigma^2)}\,dv
\leqslant\fr{C_5(\ln N)^{1/2}}{N}+\fr{C_6}{N(\ln N)^{1/2}}
\end{multline*}
для некоторых констант~$C_5$ и~$C_6$. Далее, учитывая, что $|a_i|<A/N^{\alpha}$ для $i\in I_1$, имеем
\begin{align*}
\left\vert a_i\right\vert \int\limits_{T-|a_i|}^{\infty} ve^{-{v^2}/(2\sigma^2)}\,dv
&=\\
& \hspace*{-20mm}{}=\sigma^2|a_i|e^{-{(T-|a_i|)^2}/(2\sigma^2)}%\\&
\leqslant \fr{C_7}{N^{1+\alpha}}\,;\\
a_i^2\int\limits_{T-|a_i|}^{\infty}\!e^{-{v^2}/(2\sigma^2)}\,dv&\leqslant\fr{C_8}{N^{1+2\alpha}(\ln N)^{1/2}}\,.
\end{align*}
Следовательно, существует такая константа~$C_9$, что
\begin{equation}
\mbox{E}\left\vert X_i^2\Ik_{|X_i|> T}-\mbox{E}X_i^2\Ik_{|X_i|> T}\right\vert\leqslant
\fr{C_9(\ln N)^{1/2}}{N}\,.
\label{e12she}
\end{equation}
Объединяя~(\ref{e8she})--(\ref{e12she}), получаем, что существует такая константа~$C^*$, что первая сумма 
в~(\ref{e7she}) не превосходит $C^*(\ln N)^{3/2}$. Вторая и третья сумма в~(\ref{e7she}) 
оцениваются аналогично. Таким образом, существует такая константа~$C^{**}$, что
\begin{equation}
\mbox{E}|W_N|\leqslant C^{**}(\ln N)^{3/2}\,.
\label{e13she}
\end{equation}
Далее, для произвольного $\varepsilon>0$ справедливо
\begin{multline*}
\sup\limits_{x\in\mathbf{R}}\left\vert\mbox{P}\left(\fr{\widehat{R}_N(f)-R_N(f)}{\sigma^2\sqrt{2N}}<x\right)-
\Phi_2(x)\right\vert={}\\
{}=
\sup\limits_{x\in\mathbf{R}}\left\vert\mbox{P}\left(\fr{V_N+U_N+W_N+Z_N}{\sigma^2\sqrt{2N}}<x\right)-\Phi_2(x)\right\vert\leqslant{}\\
{}\leqslant\sup\limits_{x\in\mathbf{R}}\left\vert\mbox{P}\left(\fr{V_N+U_N}{\sigma^2\sqrt{2N}}<x\right)-\Phi_2(x)\right\vert+{}\\
{}+
\fr{\varepsilon}{2\sqrt{\pi}}+\mbox{P}\left(\left\vert W_N+Z_N\right\vert>\varepsilon\sigma^2\sqrt{2N}\right)\,.
\end{multline*}
Выберем $\varepsilon=(\ln N)^{1/2}N^{1/(4(\alpha+1))-1/4}$, тогда, учитывая оценку для~$\mbox{E}|Z_N|$ 
и~(\ref{e13she}), получаем, что для некоторой константы~$B_3$ справедливо
\begin{multline}
\sup\limits_{x\in\mathbf{R}}\left\vert\mbox{P}\left(\fr{\widehat{R}_N(f)-R_N(f)}{\sigma^2\sqrt{2N}}<x\right)-
-\Phi_2(x)\right\vert\leqslant{}\\
{}\leqslant \sup\limits_{x\in\mathbf{R}}\left\vert\mbox{P}\left(\fr{V_N+U_N}{\sigma^2\sqrt{2N}}<x\right)-\Phi_2(x)\right\vert+{}\\
{}+
\fr{B_3(\ln N)^{1/2}}{N^{1/4-1/(4(\alpha+1))}}\,.
\label{e14she}
\end{multline}
Так как $V_N$ и~$U_N$ независимы, имеем~\cite{13she}
\begin{multline}
\sup\limits_{x\in\mathbf{R}}\left\vert\mbox{P}\left(\fr{V_N+U_N}{\sigma^2\sqrt{2N}}<x\right)-\Phi_2(x)\right\vert\leqslant{}\\
{}\leqslant\sup\limits_{x\in\mathbf{R}}\left\vert\mbox{P}\left(\fr{V_N}{\sigma^2\sqrt{2N}}<x\right)-
\Phi(x)\right\vert+{}\\
{}+\sup\limits_{x\in\mathbf{R}}\left\vert\mbox{P}\left(\fr{U_N}{\sigma^2\sqrt{2N}}<x\right)-\Phi(x)\right\vert\,,
\label{e15she}
\end{multline}
где $\Phi(x)$~--- функция распределения стандартного нормального закона.
Величина $\hat{\sigma}^2(N-1)/\sigma^2$ имеет распределение~$\chi^2$ с $N-1$ степенью свободы (см.~\cite{12she}). 
Учитывая результаты работы~\cite{14she}, можно показать, что
\begin{equation}
\sup\limits_{x\in\mathbf{R}}\left\vert\mbox{P}\left(\fr{V_N}{\sigma^2\sqrt{2N}}<x\right)-\Phi(x)\right\vert
\leqslant\fr{B_4}{\sqrt{N}}
\label{e16she}
\end{equation}
для некоторой константы~$B_4$.

\pagebreak

Для второго слагаемого, учитывая, что $\mbox{D}X_i^2=$\linebreak $=2\sigma^4+4\sigma^2a_i^2$, имеем
\begin{multline*}
\sup\limits_{x\in\mathbf{R}}\left\vert\mbox{P}\left(\fr{U_N}{\sigma^2\sqrt{2N}}<x\right)-\Phi(x)\right\vert={}\\
{}=
\sup\limits_{x\in\mathbf{R}}\left\vert\mbox{P}\left(H_N\delta_N<x\right)-\Phi(x)\right\vert\,,
\end{multline*}
где
$$
H_N=\fr{U_N}{\sigma^2\sqrt{2N_1+\left(4/\sigma^{2}\right)\sum\limits_{i\in I_1}a_i^2}}\,,
$$
а
$$  
\delta_N=\fr{\sqrt{N_1+\left (2/\sigma^{2}\right)\sum\limits_{i\in I_1}a_i^2}}{\sqrt{N}}\,.
$$
Здесь  $N_1$~--- число индексов в~$I_1$ ($N\geqslant N_1\geqslant$\linebreak $\geqslant N-B_1 N^{1/(2(\alpha+1))}$).
Далее,
\begin{multline}
\sup\limits_{x\in\mathbf{R}}\left\vert\mbox{P}\left(H_N\delta_N<x\right)-\Phi(x)\right\vert\leqslant{}\\
{}\leqslant
\sup\limits_{x\in\mathbf{R}}\left\vert\mbox{P}\left(H_N<x\right)-\Phi(x)\right\vert+{}\\
{}+
\sup\limits_{x\in\mathbf{R}}\left\vert\Phi\left(\fr{x}{\delta_N}\right)-\Phi(x)\right\vert\leqslant{}\\
{}\leqslant\sup\limits_{x\in\mathbf{R}}\left\vert\mbox{P}\left(H_N<x\right)-\Phi(x)\right\vert+{}\\
{}+
\fr{1}{\sqrt{2\pi e}}\max\left(\fr{1}{\delta_N}-1,\delta_N-1\right)\,.
\label{e17she}
\end{multline}
Используя тот факт, что сумма $\sum\limits_{i\in I_1}a_i^2$ при $N\rightarrow\infty$ 
фактически представляет из себя остаток ряда из квадратов вейвлет-коэффициентов функции~$f$, 
умноженного на~$N$, а $f$ принадлежит $L^2(\mathbf{R})$ и регулярна по Липшицу с 
$\gamma=1/2+\alpha$, можно показать, что существует такая константа~$B_5$, что
\begin{equation}
\max\left(\fr{1}{\delta_N}-1,\delta_N-1\right)\leqslant\fr{B_5}{N^{1/2-1/(2(\alpha+1))}}\,.
\label{e18she}
\end{equation}
Далее, поскольку $\mbox{D}X_i^2=2\sigma^4+4\sigma^2a_i^2\geqslant2\sigma^4$ и величины 
$\mbox{E}|X_i^2-\mbox{E}X_i^2|^3$ ограничены некоторой константой~$B_6$ для $i\in I_1$, то в силу неравенства 
Берри--Эс\-се\-ена для сумм разнораспределенных независимых случайных величин~\cite{15she}, получаем

\noindent
\begin{equation}
\sup\limits_{x\in\mathbf{R}}\left\vert\mbox{P}\left(H_N<x\right)-\Phi(x)\right\vert\leqslant\fr{B_7}{\sqrt{N}}
\label{e19she}
\end{equation}
для некоторой константы~$B_7$.
Объединяя~(\ref{e14she})--(\ref{e19she}), получаем~(\ref{e6she}). Теорема доказана.
\smallskip

В~(\ref{e6she}) разность $\widehat{R}_N(f)-R_N(f)$ нормируется величиной, зависящей от~$\sigma^2$. 
Однако, поскольку в~(\ref{e5she}) в $\widehat{R}_N(f)$ вместо~$\sigma^2$ подставляется~$\hat{\sigma}^2$, 
естественнее подставить~$\hat{\sigma}^2$ и в эту нормировку. При этом из доказанной теоремы можно получить 
следующее следствие.

\smallskip

\noindent
\textbf{Следствие}. \textit{Если при выполнении условий теоремы~2 вместо~$\sigma^2$ в~(\ref{e6she}) 
подставить~$\hat{\sigma}^2$, то для константы~$C_0$ из теоремы~2 и некоторой константы~$B_0$ справедливо}

\noindent
\begin{multline}
\sup\limits_{x\in\mathbf{R}}\left\vert\mbox{P}\left(\fr{\widehat{R}_N(f)-R_N(f)}{\hat{\sigma}^2\sqrt{2N}}<x\right)-
\Phi_2(x)\right\vert\leqslant{}\\
{}\leqslant \fr{C_0(\ln N)^{1/2}}{N^{1/4-1/(4(\alpha+1))}}+\fr{B_0}{N^{1/4}}\,.
\label{e20she}
\end{multline}

\medskip

\noindent
Д\,о\,к\,а\,з\,а\,т\,е\,л\,ь\,с\,т\,в\,о\,.\ 
\begin{multline*}
\sup\limits_{x\in\mathbf{R}}
\left\vert\mbox{P}\left(\fr{\widehat{R}_N(f)-R_N(f)}{\hat{\sigma}^2\sqrt{2N}}<x\right)-\Phi_2(x)\right\vert={}\\
{}=\sup\limits_{x\in\mathbf{R}}
\left\vert\mbox{P}\left(\frac{\widehat{R}_N(f)-R_N(f)}{\sigma^{2}\sqrt{2N}}\frac{\sigma^{2}}{\hat{\sigma}^2}<x\right)-\Phi_2(x)\right\vert\,.
\end{multline*}
Для произвольного $0<\varepsilon<1$
\begin{multline}
\sup\limits_{x\in\mathbf{R}}\left\vert\mbox{P}\left(\fr{\widehat{R}_N(f)-R_N(f)}{\sigma^{2}\sqrt{2N}}\,
\fr{\sigma^{2}}{\hat{\sigma}^2}<x\right)-\Phi_2(x)\right\vert\leqslant{}\\
{}\leqslant\sup\limits_{x\in\mathbf{R}}\left\vert\mbox{P}\left(\fr{\widehat{R}_N(f)-R_N(f)}{\sigma^{2}\sqrt{2N}}<x\right)-\Phi_2(x)\right\vert+{}\\
{}+\mbox{P}\left(\left\vert\fr{\sigma^{2}}{\hat{\sigma}^2}-1\right\vert>\varepsilon\right)+
\fr{\varepsilon}{(1-\varepsilon)\sqrt{2\pi e}}\,.
\label{e21she}
\end{multline}
Далее,
\begin{multline*}
\mbox{P}\left(\left\vert\fr{\sigma^{2}}{\hat{\sigma}^2}-1\right\vert>\varepsilon\right)\leqslant
\fr{\mbox{E}\left\vert\hat{\sigma}^2-\sigma^{2}\right\vert(1+\varepsilon)}{\varepsilon\sigma^2}\leqslant{}\\
{}\leqslant
\fr{\sqrt{\mbox{D}\hat{\sigma}^2}(1+\varepsilon)}{\varepsilon\sigma^2}\,.
\end{multline*}
Выберем $\varepsilon=N^{-1/4}$. Тогда найдется такая константа~$B_0$, что
\begin{equation}
\mbox{P}\left(\left\vert\fr{\sigma^{2}}{\hat{\sigma}^2}-1\right\vert>\varepsilon\right)+
\fr{\varepsilon}{(1-\varepsilon)\sqrt{2\pi e}}\leqslant\fr{B_0}{N^{1/4}}\,.
\label{e22she}
\end{equation}
Объединяя~(\ref{e21she}), (\ref{e22she}) и~(\ref{e6she}), получаем~(\ref{e20she}).

\medskip

\noindent
\textbf{Замечание~1.} Хотя в утверждениях этого раздела требуется равномерная регулярность по 
Липшицу, когда дисперсия оценивается по независимой выборке, это требование можно ослабить, 
позволив функции быть разрывной в конечном числе точек, если потребовать, чтобы вейвлет-функция 
имела компактный носитель. При этом порядок оценок в теореме~2 и ее следствии не изменится.

\section{Оценивание дисперсии по~выборке сигнала}

Если дисперсия оценивается по выборке сигнала и функция~$f$ удовлетворяет требуемым условиям регулярности, то в 
силу~(\ref{e2she}) обычно ее оценивают по половине всех вейвлет-коэффициентов для $j=J-1$ 
(напомним, что $N=2^J$), так как эти коэффициенты фактически содержат только шум. 
Для доказательства утверждений этого пункта будем использовать две оценки дисперсии, 
каж\-дая из которых построена по половине вейвлет-коэффициентов из указанного множества, т.\,е.\
по четверти всех вейвлет-коэффициентов. Если перейти к одному индексу~$i$, то эти оценки будут 
выглядеть следующим образом:
\begin{equation}
\hat{\sigma}_1^2=\fr{1}{N/4-1}\sum_{i=N/2+1}^{3N/4}(X_i-\overline{X}_1)^2\,,
\label{e23she}
\end{equation}
где
$$ \overline{X}_1=\fr{4}{N}\sum_{i=N/2+1}^{3N/4}X_i\,;$$
\begin{equation}
\hat{\sigma}_2^2=\fr{1}{N/4-1}\sum_{i=3N/4+1}^{N}(X_i-\overline{X}_2)^2\,,
\label{e24she}
\end{equation}
где
$$\overline{X}_2=\fr{4}{N}\sum_{i=3N/4+1}^{N}X_i\,.
$$
При пороговой обработке для построения порога~$\hat{T}$ будем использовать~$\hat{\sigma}_1^2$ для тех наблюдений~$X_i$, 
которые не зависят от~$\hat{\sigma}_1^2$, и~$\hat{\sigma}_2^2$~--- для тех наблюдений~$X_i$, которые не зависят 
от~$\hat{\sigma}_2^2$. Для наблюдений, не зависящих ни от~$\hat{\sigma}_1^2$, ни от~$\hat{\sigma}_2^2$, 
будем использовать одну из этих оценок так, чтобы каждая из них использовалась одинаковое число раз. 
Таким образом, многие рассуждения, изложенные в теореме~2, останутся справедливыми.

Можно показать, что если~$f$ регулярна по Липшицу с параметром 
$\gamma=1/2+\alpha$, то
\begin{equation}
\left.
\begin{array}{rl}
\mbox{E}\hat{\sigma}_k^2&=\sigma^2+O\left(\fr{1}{N^{1+2\alpha}}\right)\,;\\
\mbox{D}\hat{\sigma}_k^2&=O\left(\fr{1}{N}\right)\,,\enskip k=1,2\,.
\end{array}
\right \}
\label{e25she}
\end{equation}

Справедлив аналог теоремы~2.

\medskip

\noindent
\textbf{Теорема 3.} \textit{Пусть $f\in L^2(\mathbf{R})$ задана на отрезке $[0,1]$ и является 
равномерно регулярной по Липшицу с параметром $\gamma=1/2+\alpha$ ($\alpha>0$) и пусть оценка~$\sigma^2$ 
задается соотношениями}~(\ref{e23she}) \textit{и}~(\ref{e24she}), 
\textit{тогда существует такая константа~$\tilde{C}_0$ (зависящая от~$\alpha$, $A$ и~$\sigma$), что
\begin{multline}
\sup\limits_{x\in\mathbf{R}}\left\vert\mbox{P}\left(\fr{\widehat{R}_N(f)-R_N(f)}{\sigma^2\sqrt{2N}}<x\right)-
\Phi(x)\right\vert\leqslant{}\\
{}\leqslant \fr{\tilde{C}_0(\ln N)^{1/2}}{N^{1/4-1/(4(\alpha+1))}}\,.
\label{e26she}
\end{multline}}
\textbf{Замечание~2.} В отличие от теоремы~2, предельный закон в~(\ref{e26she}) имеет дисперсию, равную~1.

\medskip

\noindent
Д\,о\,к\,а\,з\,а\,т\,е\,л\,ь\,с\,т\,в\,о\,.\ Так же, как в теореме~2, запишем разность $\widehat{R}_N(f)-R_N(f)$ в виде
$\widehat{R}_N(f)-R_N(f)=V_N+U_N+W_N+Z_N$. Рассмотрим сумму $V_N+U_N$:
\begin{multline*}
V_N+U_N=\sum_{i\in I_1}\left(X_i^2-\mbox{E}X_i^2\right)+{}\\
{}+N\left(\sigma^2-
\fr{1}{2}\left(\hat{\sigma_1^2}+\hat{\sigma_2^2}\right)\right)\,.
\end{multline*}
Заметим, что индексы всех слагаемых в суммах~(\ref{e23she}) и~(\ref{e24she}) содержатся в~$I_1$. 
Обозначим множество этих индексов через~$I'_1$. Таким образом, имеем
\begin{multline*}
V_N+U_N=\sum_{i\in I_1\backslash I'_1}\left(X_i^2-\mbox{E}X_i^2\right)-
\sum_{i\in I'_1}\left(X_i^2-\mbox{E}X_i^2\right)-{}\\
{}-2\sum_{i\in I'_1}a_i^2-\fr{8}{N-4}\sum_{i\in I'_1}X_i^2+\fr{N^2}{2(N-4)}\left(\overline{X}_1^2+\overline{X}_2^2\right)\,.
\end{multline*}
Пусть
$$
U'_N=\sum_{i\in I_1\backslash I'_1}\left(X_i^2-\mbox{E}X_i^2\right)-\sum_{i\in I'_1}\left(X_i^2-\mbox{E}X_i^2\right)\,;
$$
\begin{multline*}
\Delta_N=-2\sum_{i\in I'_1}a_i^2-\fr{8}{N-4}\sum_{i\in I'_1}X_i^2+{}\\
{}+\fr{N^2}{2(N-4)}(\overline{X}_1^2+\overline{X}_2^2)\,.
\end{multline*}
Так же, как в теореме~2, убеждаемся, что для некоторой константы~$\tilde{C}_1$
\begin{multline}
\sup\limits_{x\in\mathbf{R}}\left\vert\mbox{P}\left(\fr{U'_N}{\sigma^2\sqrt{2N}}<x\right)-\Phi(x)\right\vert\leqslant{}\\
{}\leqslant\fr{\tilde{C}_1}{\sqrt{N}}+\fr{\tilde{C}_2}{N^{1/2-1/(2(\alpha+1))}}\,.
\label{e27she}
\end{multline}
Далее, учитывая соотношения~(\ref{e25she}), можно оценить сумму 
$\mbox{E}|\Delta_N|+\mbox{E}|W_N|+\mbox{E}|Z_N|$ аналогично тому, как это было сделано в теореме~2 
для $\mbox{E}|W_N|+\mbox{E}|Z_N|$. Используя неравенство, аналогичное~(\ref{e14she}), 
и учитывая~(\ref{e27she}), получаем~(\ref{e26she}). Теорема доказана.

\medskip

 Из теоремы~3 можно сделать такое же следствие, как и из теоремы~2.
 
 \medskip
 
 \noindent
\textbf{Следствие}. \textit{Если при выполнении условий теоремы~3 вместо $\sigma^2$ в}~(\ref{e26she}) 
\textit{подставить $\hat{\sigma}^2=(\hat{\sigma}_1^2+\hat{\sigma}_2^2)/2$, то для константы~$\tilde{C}_0$ из теоремы}~3 
\textit{и некоторой константы~$\tilde{B}_0$ справедливо}
\begin{multline}
\sup\limits_{x\in\mathbf{R}}\left\vert\mbox{P}\left(\fr{\widehat{R}_N(f)-R_N(f)}{\hat{\sigma}^2\sqrt{2N}}<x\right)-
\Phi(x)\right\vert\leqslant{}\\
{}\leqslant\fr{\tilde{C}_0(\ln N)^{1/2}}{N^{1/4-1/(4(\alpha+1))}}+\fr{\tilde{B}_0}{N^{1/4}}\,.
\label{e28she}
\end{multline}
Доказательство неравенства~(\ref{e28she}) аналогично доказательству следствия из теоремы~2.

\medskip

\noindent
\textbf{Замечание~3.} Из доказательств приведенных утверждений следует, 
что оценки на самом деле имеют более сложную структуру. Можно заметить, что если уточнить структуру
оценок, то константа при главном члене не будет зависеть от~$\sigma^2$.
\smallskip

\noindent
\textbf{Замечание~4.} Метод разложения суммы на две компоненты, одна из которых ведет себя 
как сумма независимых случайных величин, а другая стремится к нулю по вероятности, как правило, 
приводит к оценкам, не являющимся оптимальными по порядку. Следовательно, можно ожидать, что 
порядок оценок в приведенных утверждениях может быть улучшен.


{\small\frenchspacing
{%\baselineskip=10.8pt
\addcontentsline{toc}{section}{Литература}
\begin{thebibliography}{99}

\bibitem{2she} %1
\Au{Donoho D., Johnstone I.\,M.} 
Ideal spatial adaptation via wavelet shrinkage~// Biometrika, 1994. Vol.~81. No.\,3. P.~425--455.

\bibitem{1she} %2
\Au{Donoho D., Johnstone I.\,M.} 
Adapting to unknown smoothness via wavelet shrinkage~// J. Amer. Stat. Assoc., 1995. Vol.~90. P.~1200--1224.


\bibitem{3she} %3
\Au{Donoho D.\,L., Johnstone I.\,M., Kerkyacharian G., Picard~D.} 
Wavelet shrinkage: Asymp\-to\-pia?~// J.~R.~Statist. Soc. Ser. B., 1995. Vol.~57. No.\,2. P.~301--369.

\bibitem{4she}
\Au{Marron J.\,S., Adak S., Johnstone I.\,M., Neumann~M.\,H., Patil~P.} 
Exact risk analysis of wavelet regression~// J.~Comput. Graph. Stat., 1998. Vol.~7. P.~278--309.

\bibitem{5she} %
\Au{Antoniadis A., Fan J.} 
Regularization of wavelet approximations~// J. Amer. Statist. Assoc., 2001. Vol.~96. No.\,455. P.~939--967.

\bibitem{7she}
\Au{Маркин А.\,В.} 
Предельное распределение оценки риска при пороговой обработке вейвлет-коэффициентов~// 
Информатика и её применения, 2009. Т.~3. №\,4. С.~57--63.

\bibitem{6she} %6
\Au{Маркин А.\,В., Шестаков О.\,В.} 
О состоятельности оценки риска при пороговой обработке вейвлет-ко\-эф\-фи\-ци\-ен\-тов~// 
Вестн. Моск. ун-та. Сер.~15. Вычисл. матем. и киберн., 2010. №\,1. C.~26--34.


\bibitem{8she}
\Au{Добеши И.} Десять лекций по вейвлетам.~--- Ижевск: НИЦ Регулярная и хаотическая динамика, 2001.

\bibitem{9she}
\Au{Mallat S.} 
A wavelet tour of signal processing.~--- Academic Press, 1999.

\bibitem{10she}
\Au{Abramovich F., Silverman B.\,W.} 
Wavelet decomposition approaches to statistical inverse problems~// Biometrika, 1998. Vol.~85. No.\,1. P.~115--129.

\bibitem{11she}
\Au{Boggess A., Narkowich F.} 
A first course in wavelets with Fourier analysis.~--- Prentice Hall, 2001.

\bibitem{12she}
\Au{Крамер Г.} Математические методы статистики.~--- М: Мир, 1975.

\bibitem{13she}
\Au{Senatov V.\,V.} Normal approximation: New results, methods, and problems.~--- VSP, 1998.

\bibitem{14she}
\Au{Ульянов В.\,В., Кристоф Г., Фуджикоши~Я.} 
О~приближениях преобразований хи-квадрат распределений в статистических приложениях~// Сиб. матем. журн., 2006. 
Т.~47. №\,6. С.~1401--1413.

 \label{end\stat}

\bibitem{15she}
\Au{Петров В.\,В.} Предельные теоремы для сумм независимых случайных величин.~--- М: Наука, 1987.
 \end{thebibliography}
}
}


\end{multicols}