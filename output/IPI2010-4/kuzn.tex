
\def\stat{kuzn}

\def\tit{СВЯЗЬ МЕЖДУ ВРЕМЕННЫМИ 
И~СТРУКТУРНО-ТОПОЛОГИЧЕСКИМИ ХАРАКТЕРИСТИКАМИ 
ДИАГРАММ РИТМА СЕРДЦА ЗДОРОВЫХ ЛЮДЕЙ}

\def\titkol{Связь между временными 
и~структурно-топологическими характеристиками 
диаграмм ритма сердца здоровых людей}

\def\autkol{А.\,А.~Кузнецов}
\def\aut{А.\,А.~Кузнецов$^1$}

\titel{\tit}{\aut}{\autkol}{\titkol}

%{\renewcommand{\thefootnote}{\fnsymbol{footnote}}\footnotetext[1]
%{Исследование поддержано грантами РФФИ 08-07-00152 и 09-07-12032.
%Статья написана на основе материалов доклада, представленного на IV 
%Международном семинаре  <<Прикладные задачи теории вероятностей и математической статистики, 
%связанные с моделированием информационных систем>> (зимняя сессия, Аоста, Италия, январь--февраль 2010~г.).}}

\renewcommand{\thefootnote}{\arabic{footnote}}
\footnotetext[1]{Владимирский государственный университет, artemi-k@mail.ru}

\vspace*{12pt}

\Abst{По данным 628~регистраций электрокардиограмм (ЭКГ) у~177~здоровых и больных людей проведен 
сравнительный анализ параметров реальной и виртуальной диаграмм ритма сердца для 
оценки влияния системы регуляции на ритм сердца. Между параметрами диаграмм и 
информационной энтропией в условиях дискретной сезонной адаптации определены 
функциональные связи. Предложены <<формулы функционального состояния организма>>, 
связывающие параметры макроструктуры диаграммы ритма сердца с параметрами ее 
ярусной микроструктуры. Обнаружено, что режим ритма сердца здорового человека вне 
зависимости от пола имеет цикл календарного года, в течение которого трижды дискретно 
меняется.}

\KW{диаграмма ритма сердца; функциональное состояние организма; ярусная структура; 
информационная энтропия; количество информации}

       \vskip 14pt plus 9pt minus 6pt

      \thispagestyle{headings}

      \begin{multicols}{2}

      \label{st\stat}
  
\section{Введение }

  При исследовании процессов, характеризуемых большим набором 
параметров, возникает вопрос\linebreak о <<цене>> каждого из них. Поначалу все 
па\-ра\-мет\-ры равноценны, поэтому обычно проводят многофакторный 
параметрический анализ, одним из\linebreak возможных инструментом которого служит 
дискриминантный анализ. По величине вероятности влияния на процесс 
анализируемых параметров, удовлетворяющих предложенной гипотезе, все они 
выстраиваются в вариационный ряд по статистической значимости, теряя при 
этом свою равноценность. С~одной стороны, это является хорошей подсказкой 
в выборе параметров для исследования процесса. С~другой стороны, 
статистическая независимость выделяемых параметров вызывает большие 
сомнения~[1--3]. При исследовании сис\-тем\-ных процессов в стороне остаются и 
редко реализуемые параметры, ответственные за функции управления, 
регуляции и контроля, и нелинейные связи между этими функциями, 
количественно характеризуемые указанными параметрами. Вероятно, 
дискриминантный анализ (ДА) может иметь успех на начальной стадии 
исследования реализаций результирующих сигналов <<искусственных 
сис\-тем\-ных процессов>>.
  
  Известны попытки применения технологии ДА для исследования 
естественных системных процессов, к которым относится ритм сердца 
человека, например, по набору параметров вариабельности ритма сердца 
(ВРС)~[4] или по морфологическим\linebreak параметрам ЭКГ~[5]. Результаты таких исследований являются техническими и не дают 
никаких предпосылок к пониманию универсальных и индивидуальных 
физиологических процессов и механизмов управления ими при организации 
результирующего системного процесса ритма сердца. Вероят\-но, исследование в 
указанном направлении следует начинать с определения <<критериев нормы>> 
и поиска нормальных физиологических закономерностей ритма сердца как 
сис\-тем\-но\-го процесса, характера и причин искажения или нарушения этих 
закономерностей в онтогенезе. 
  
  В~данной работе используется физическая модель исследования, 
включающая сравнительный качественный и количественный анализ поведения 
группы параметров ВРС реальной и виртуальной~[6] диаграммы ритма сердца 
(ДРС). Целью его применения является анализ статистических зависимостей 
общепринятых параметров при поиске общих закономерностей в динамике 
ритма. 

\vspace*{-12pt}
  
\section{Экспериментальная часть }
  
  Регистрация ЭКГ проводилась монитором\linebreak Холтера комплекса амбулаторной 
регистрации электрокардиосигнала <<\textit{AnnA Flash}~2000>>~[7] с 
использованием накожных электродов для электрокардиографии. При 
регистрации биопотенциалов применялись двухполюсные отведения по Небу: 
первый электрод располагался во втором межреберном положении у правого 
края грудины (соответствует~$V_5^2$), второй электрод располагался в области\linebreak 
верхушки сердца. Такое расположение электродов позволяет записать переднее 
грудное отведение\linebreak ($A$-\textit{anterior}), соответствующее стандартному 
отведению~II с максимальной амплитудой зубцов на ЭКГ. Данные каждой ЭКГ 
в лицензированной программе \textit{EScreen}~[8] конвертировались в 
ритмограммы в форме последовательности значений $R$--$R$\linebreak интервалов и 
далее посредством встроенной процедуры \textit{Heart rate variability} в 
программе \textit{EScreen} определялись выборочные значения всех параметров 
ВСР для каждой ритмограммы.
  
  Проведено 628~регистраций ЭКГ у 177 здоровых людей и больных~--- 
пациентов реанимационных и кардиореанимационных отделений. Серийные и 
групповые регистрации здоровых молодых людей выделены отдельно и 
представляют основной экспериментальный материал данной работы. 
  
  Серийные двадцатиминутные посуточные ре\-гист\-ра\-ции ЭКГ проводились 
тремя сезонными сериями в течение 5--7~недель каждая в одинаковых 
условиях покоя в одно время суток (двумя мониторами) для двух молодых 
людей 21~года: юноши~К.\ и девушки~Ш. Все серийные регистрации ЭКГ в 
количестве $N_{\mathrm{рег}} = 176$ проводились в домашних условиях при 
температуре 20--22~$^\circ$C в положении лежа на спине с периодом адаптации 
5--10~мин. Серия из 45~регистраций ЭКГ юноши~Р.\ (21~год) проводилась 
отдельно несколько раз в сутки в течение первых двух недель февраля в разных 
условиях покоя и движения. 
  
  Групповые двадцатиминутные регистрации ЭКГ проводились в течение 
9~недель (февраль--март 2008~г.) для группы из 32~молодых людей 19--
24~лет: 20 юношей и 12 девушек. Групповые регистрации проводились в 
лаборатории университета один раз в неделю в интервале времени 14:00--19:00 
в положении покоя: сидя, без адаптации к условиям регистрации. 

\section{Методика обработки и анализа данных}
  
  Индивидуальные особенности ритма сердца найти несложно даже при 
коротких записях ЭКГ~[9]. Сложнее найти общие закономерности ритма 
сердца одного человека в разные последовательные интервалы времени. Еще 
более сложно найти общие динамические закономерности ритма сердца разных 
людей, особенно если записи ЭКГ имеют разную длительность. Сравнение 
временных величин, характеризующих общую вариабельность ритма и 
вычисленных на основе записей различной длительности, является 
некорректным. Методы оценки общей вариабельности сердечного ритма и ее 
компонентов с коротким и длинным периодом не могут заменить друг 
друга~\cite{1ku, 3ku}. Поэтому при анализе ВСР возникают непреодолимые 
трудности при сопоставлении данных записей ЭКГ разной длины с нормой~[1] 
для фиксированной короткой или длинной записи. Более того, определение 
самой нормы функционального состояния организма посредством 
количественных показателей ВСР становится неоднозначным.
  
  Для решения этой проблемы предлагается перейти от анализа 
вариабельности ритма сердца по совокупности группы соответствующих 
показателей, представляющих по отдельности тот или иной информативный 
признак вариабельности, к функционально-параметрическому анализу их 
связей. При этом предлагается использовать <<во благо>> другой проблемный 
момент метода оценки ВСР: статистическую зависимость и дублирование 
информации разными параметрами ВСР~[1--3]. Тесная корреляционная связь 
между параметрами ВСР нивелирует их индивидуальные зависимости от длины 
записи на фазовой плоскости. При этом параметр ритма сердца, выбранный 
общим аргументом, не должен быть ограничен теми или иными 
характеристиками ритмограммы. В~качестве такого параметра была выбрана 
информационная энтропия ярусной диаграммы ритма 
  сердца~\cite{3ku, 6ku, 10ku}. 
    
    Существуют разные подходы к понятию <<энтропия>>, связанные с 
разными объектами и задачами исследования. К~наиболее известным 
относятся~[11--14] подходы: (1)~Клаузиуса, определяющий энтропию функцией 
состояния газовой системы при исследовании тепловых потоков; (2)~Бриллюэна, 
определяющий энтропию мерой <<деградации>>\linebreak энергии; 
(3)~Пригожина, определяющий энтропию мерой <<связанной 
энергии>>; (4)~Больцмана, определяющий энтропию мерой 
интенсивности молекулярного хаоса; (5)~Шеннона, связывающий\linebreak 
энтропию с количеством информации в информационном сообщении и 
определяющий ее как меру степени неопределенности состояния физической 
системы.
    
    Проблема оценки количества информации, содержащегося в сообщении, 
была решена в~1949~г.~\cite{14ku}. В~качестве единицы (бит) информации 
$I=-\log_2 p$ принимают количество информации в достоверном сообщении о 
событии, априорная вероятность~$p$ которого равна~1/2. 
  
  Известно~\cite{10ku, 15ku}, что количество информации~$I_X$, 
приобретаемое физической системой~$X$ при полном выяснении ее состояния, 
равно энтропии~$H(X)$ системы $I_X=H(X)$. Если непрерывную систему 
свести к~дискретной, установив предел\linebreak точности измерения (шаг 
дискретизации~$\Delta x$), это будет равносильно замене плавной кривой на 
графике функции плотности вероятности~$f(x)$ на ступенчатую~--- в форме 
гистограммы. При такой замене вероятности попадания~$p_i$ в 
соответствующие разряды определены в форме $p_i=f(x_i)\Delta x$. В~таком 
случае 
  \begin{equation}
  I_X =-\sum\limits_{i=1} p_i \log_2 p_i\,.
  \label{e1ku}
  \end{equation}
  
  При формировании диаграммы ритма сердца количество информации 
набирается дискретно: от систолы к систоле, поэтому принципиально 
невозможно использовать понятие скорости набора информации, являющееся 
основным параметром для технических устройств связи~\cite{15ku}. В~этом 
случае масштабной единицей становится переменный интервал времени 
события ($R$--$R$ интервал). Ритмограмма представляет собой номерной ряд 
последовательности $n$~таких событий. Постоянная\linebreak частота считывания 
монитором значений биопотенциалов при формировании ЭКГ задает 
постоянным шаг дискретизации~$\Delta x$ значений $R$--$R$ интервала на 
ДРС. Поэтому значения $R$--$R$\linebreak интервала на ДРС формируют <<ярусы 
микросостояний>>. Это позволяет трактовать ДРС как реализацию 
макросостояния системы ритма и применить к ней 
  струк\-тур\-но-то\-по\-ло\-ги\-че\-ский анализ неупорядоченности 
распределения значений $R$--$R$ интервала по микросостояниям с 
использованием информационной энтропии~\cite{6ku, 10ku}. 
  
  В качестве меры фрактальной размерности странных аттракторов в фазовом 
пространстве применяют информационную размерность~$D_I$. Мерой 
непредсказуемости в системе служит информационная энтропия~[10, 16--19]:
  \begin{equation}
  I(\varepsilon ) =-\sum\limits_{i=1} p_i\log_2 p_i\,.
  \label{e2ku}
  \end{equation}
  
  Перенося определение этой меры на ярусную ДРС, получим вертикальный 
размер~$\varepsilon$ ячейки\linebreak  покрытия, равный шагу дискретизации~$\Delta x$. 
В~таком представлении категории количества ин\-формации~(\ref{e1ku}) и 
информационной энтропии~(\ref{e2ku})\linebreak становятся тождественными. Чтобы не 
допускать путаницы и для конкретного объекта исследования (ярусной 
структуры ДРС) в обозначении информационной энтропии будем использовать 
символ~$I^*$. 
  
  Степень неопределенности состояния системы ритма может определяться и 
вероятностями ($p_i$) ее возможных состояний, и их количеством~\cite{15ku}, 
поэтому возникает возможность перехода от вероятностных категорий к 
макропараметрам ярусной ДРС. После несложных алгебраических 
преобразований формулы~(\ref{e1ku}) и~(\ref{e2ku}), примененные к ярусной 
структуре ДРС, можно представить в виде~\cite{6ku}: 
  \begin{equation}
  I^*=\fr{A}{n}\left[ \ln \Gamma +B\right ]\,,
  \label{e2-2ku}
  \end{equation}
где $A=1/\ln 2$~--- полиномиальный коэффициент, $\Gamma 
=N!/\prod\limits_{i=1}^n N_i!$, $N$~--- число дискретных значений~$R$--$R$ 
интервала в анализируемой выборке ритмограммы, $N_i$~--- число дискретных 
значений $R$--$R$ интервала на $i$-м ярусе ДРС.

  При принятой точности расчета (до двух значащих цифр) величиной~$B$ 
можно пренебречь уже при $n > 100$, так как величина абсолютной 
погрешности $\Delta I_X=AB_{\max}/n$ с ростом~$n$ асимптотически 
стремится к нулю~\cite{6ku}. С~учетом этого формула~(\ref{e2-2ku}) 
принимает окончательный расчетный вид: 
  \begin{equation}
  I^*=\fr{A}{n}\,\ln\Gamma\,.
  \label{e3ku}
  \end{equation}
  
  В формуле~(\ref{e3ku}) полиномиальный коэффициент~$\Gamma$ 
приобретает смысл термодинамической ве\-ро\-ят\-ности и определяет число 
микросостояний (ком\-бинаций), посредством которых реализуется\linebreak 
макросостояние системы~$X$. Величина~$I^*$, определенная с точностью до 
величины~$AB_{\max}/n$, определяет среднее количество информации, 
недостающее до полного описания одного отсчета. 
  
  С одной стороны, информационная энтропия~$I^*$ обладает основными 
свойствами физической энтропии~--- при фиксированных внешних условиях 
растет с ростом~$n$, принимая максимальное со\-вмес\-ти\-мое с внешними 
условиями значение. С~другой стороны, она определена отношением 
количества информации~$I_\Sigma$, недостающего для полного описания ДРС 
к объему выборки~$n$ и по смыс\-лу является средним количеством 
информации, недостающим для описания одного микроперехода на 
ДРС~\cite{6ku}. По сравнению с другими параметрами вариабельности 
сердечного ритма (ВСР), информационная энтропия~$I^*$ не теряет 
адекватного физического смыс\-ла для многомодального распределения и имеет 
постоянную, четко выраженную <<правую границу условного 
здоровья>>~\cite{6ku}. Очевидно, что разные заболевания могут дать один и 
тот же результат по величине~$I^*$ для ДРС. Это может означать, что разные 
стадии разных заболеваний подобны по результирующему параметру~$I^*$, 
т.\,е.\ по неупорядоченности ритма сердца обследуемых людей. В~таком случае 
параметр~$I^*$ может служить количественной оценкой общего 
функционального состояния человека (ФСО)~\cite{6ku}.
\end{multicols}

  \begin{figure} %fig1
  \vspace*{1pt}
\begin{center}
\mbox{%
\epsfxsize=162.85mm
\epsfbox{kuz-1.eps}
}
\end{center}
\vspace*{-6pt}
\Caption{Графики функциональных связей параметров ВСР и информационной 
энтропии для пяти серий ($N = 253$) сезонных регистраций ЭКГ здоровых 
молодых людей~(\textit{а})--(\textit{в}) и для 375~регистраций ЭКГ здоровых 
людей и пациентов отделений реанимации за 9~лет~(\textit{г}) 
  \label{f1ku}}
  \vspace*{6pt}
  \end{figure}
  
  \begin{multicols}{2}
  
  На рис.~\ref{f1ku},\,\textit{а}--\textit{г} в полулогарифмическом 
масштабе приведены точечные графики зависимости основных расчетных 
выборочных параметров ВСР от соответствующих значений информационной 
энтропии по всем $N$~ритмограммам. В~качестве основных параметров ВСР 
представлены: из временной области анализа ДРС~--- стандартное отклонение 
($\sigma$, мс), из частной области анализа~--- полная спектральная мощность 
(\textit{Total Power}, или~TP, мс$^2$), из набора производных показателей 
  Баевского~--- индекс напряжения (ИН), или стресс-ин\-декс (SI), 
характеризующий степень централизации управления ритмом~\cite{20ku}. 
  
  Графики на рис.~\ref{f1ku},\,\textit{а}--\textit{в} представлены 
услов\-но-се\-зон\-ны\-ми линиями одинакового наклона. Это указывает на 
наличие прочных функциональных связей между параметрами ВСР и 
информационной энтропией в условиях дискретной сезонной адаптации. 
В~зависимости от уровня ФСО в интервалах времени услов\-но-се\-зон\-но\-го 
исследования точки на графиках соответствующих параметров перемещаются 
вдоль линий функциональных кривых <<как по монорельсу>>. При смене 
сезона заполняются новые <<функциональные уровни>>, соответствующие 
обретению ритмом качественно новых стационарных режимов 
(рис.~\ref{f1ku},\,\textit{а}--\textit{г}). Данные юношей ($\times$,~$+$) и 
девушек ($\bullet$,~$\circ$) для каждого услов\-но-се\-зон\-но\-го интервала времени 
принадлежат соответствующему <<функциональному уровню>> 
(см.\ рис.~\ref{f1ku},\,\textit{а}--\textit{в}) без разделения по полу. На 
основании этого результата при одинаковом возрасте здоровых молодых 
людей, обследуемых и в группе, и серийно, оказалось возможным 
предположить, что ритм сердца меняет режим дискретно при изменении 
сезонных внешних условий. Механизм адаптации является стабилизирующим 
каждое новое качество ритма сердца.

  
  При изменении длины записи функциональные кривые лишь меняют свою 
длину изменением координат правой или левой границы. Величина $I^* = 
6$~бит характеризует режимы ритма здорового молодого человека на всех 
ступенях сезонной адаптации при двадцатиминутной записи ЭКГ. Обращает 
внимание, что определенная ранее нелинейная динамика выборочных 
коэффициента асимметрии и эксцесса распределений на ДРС при серийных 
регистрациях~\cite{6ku} не сказывается на форме графиков~$\sigma (I^*, N)$ в 
интервале времени любого услов\-но-се\-зон\-но\-го исследования.
  
  При серийных и групповых исследованиях обнаружен нелинейный характер 
связи стандартного отклонения~$\sigma$ и информационной энтропии~$I^*$ 
со сред\-не-вы\-бо\-роч\-ным значением $R$--$R$ интервала ($\langle X\rangle$) 
и моды (Mo) ритмограмм. Внешне форма функций $\langle X\rangle(I^*, N)$ и 
$\langle X\rangle(\sigma, N)$ на соответствующих графиках напоминает <<полет 
мухи под люстрой>>. При внутривыборочных исследованиях ритмограмм, 
проведенных с использованием <<метода скользящих средних>>~\cite{6ku}, 
оказалось, что на малые флуктуации среднего уровня ритма сердца 
ни~$\sigma$, ни~$I^*$ практически не откликаются. Если средний уровень 
ритма на некотором интервале времени ($\Delta n < 100$) постоянен, то и 
указанные параметры не меняются. Однако если постоянство $\langle 
X\rangle(n)$ во времени затягивается, то величины обоих параметров начинают 
медленно монотонно падать. На любое относительно резкое и длительное 
изменение $\langle X\rangle$ функция~$\sigma(n)$ откликается импульсно 
таким образом, что $\sigma_{\max}$ всегда приходится на точку перегиба 
графика~$\langle X\rangle(n)$. В~таком случае~$\sigma(n)$ определяется 
скоростью изменения функции $\langle X\rangle (n)$. При монотонном 
росте~$\langle X\rangle$ значения~$I^*$ и~$\sigma$ обретают тенденцию к 
росту за счет набора нерабочих (<<пустых>>) ярусов. При этом процентный 
состав рабочих ярусов падает, а неупорядоченность ДРС слабо растет. Таким 
образом, функция~$I^*(n)$ <<следит>> за средним уровнем~$\langle X\rangle 
(n)$ комплексно: за величиной, за длительностью его характерных 
динамических участков и за скоростью изменения. 
  

\section{Формулы функционального состояния организма}

    Подавляющее количество моделей и приемов исследования временных 
рядов относится к стационарным в широком смысле рядам, т.\,е.\ к рядам, для 
которых первые четыре момента не зависят от времени~\cite{21ku, 22ku}. 
В~общем случае даже в интервале одной двадцатиминутной регистрации ЭКГ 
не удается исключить тренд в дисперсии, асимметрии и эксцессе разностным 
дифференцированием рядов ДРС~\cite{6ku, 21ku}. Таким образом, реальные 
процессы ритма сердца не являются стационарными. Для того чтобы в первом 
приближении для задач краткосрочного прогноза считать их таковыми, 
необходимо создать определенные экспериментальные условия, а именно в 
состав анализа включать выборки только молодых здоровых людей, 
находящихся в состоянии стабильного психоэмоционального покоя. В~рамках 
такого приближения к стационарным рядам можно применить теорему Вольда 
о разложении~\cite{23ku}, согласно которой всякий стационарный процесс 
может быть единственным образом представлен в виде суммы двух не 
кор\-ре\-ли\-ру\-ющих между собой процессов: детерминированного (сингулярного) 
и случайного (регулярного белого шума). 

Если принять в качестве гипотезы утверждение, что ритм сердца является 
нестационарным процессом в той мере, в которой в него включены 
составляющие внешнего влияния механизмов управления и 
регуляции~\cite{2ku, 6ku}, то стационарным ритмическим процессом может 
характеризоваться состояние <<неуправляемого сердца>>. Обычно этот 
физиологический термин применяется для описания работы изолированного от 
организма сердца с перфузией. В~применении к описанию работы сердца, 
изолированного только от детерминистской информации внешнего влияния, 
жесткий ритм, обеспечиваемый функцией автоматии, дополняется на заданном 
уровне регулярным белым шумом, определяющим лишь присутствие и 
функциональную готовность разных механизмов управления и регуляции. 
Приближением к такому режиму ритма может быть ритм сердца молодого 
здорового человека, находящегося в условиях адаптации к условиям 
регистрации и в состоянии устойчивого психоэмоционального покоя. При этом 
распределение $R$--$R$ интервалов на ДРС приближается к нормальному 
распределению~[6, 24--28]. Крайней идеализацией такого режима ритма может 
служить <<виртуальный ритм>> с реализацией в форме цифрового ряда, 
полученного генерацией случайных чисел по нормальному закону и по 
заданным значениям стандартного отклонения и шага 
дискретизации~\cite{6ku,  28ku}. Для построения виртуальной диаграммы 
ритма сердца используется нормальный генератор случайный чисел в 
программе~Excel. Оказалось, что величина информационной 
энтропии~$I^*_{\mathrm{г}}$ виртуальной ДРС (ВДРС) 
при $n\rightarrow \infty$ монотонно приближается к значению энтропии 
$$
H(X) = \log_2\fr{(2\pi e)^{1/2}\sigma}{\Delta x}
$$ 
в форме математического ожидания для непрерывного 
множества случайных чисел, распределенных по нормальному 
закону~\cite{6ku, 15ku} (рис.~\ref{f2ku}). 
\begin{figure*} %fig2
  \vspace*{1pt}
\begin{center}
\mbox{%
\epsfxsize=164.59mm
\epsfbox{kuz-2.eps}
}
\end{center}
\vspace*{-6pt}
\begin{minipage}[t]{79.5mm}
\Caption{Графики зависимости $I_{\mathrm{г}}^*(\sigma, n)$ в сравнении с 
функцией~$H(X)$
\label{f2ku}}
%\end{figure*}
\end{minipage}
\hfill
\vspace*{-6pt}
\begin{minipage}[t]{79.5mm}
%  \begin{figure*} %fig3
%    \vspace*{1pt}
\Caption{Графики $N_{\mathrm{МЯП}}(n)$~(\textit{1}) 
и~$\langle\Delta_{\mathrm{ЯП}}\rangle (n)$~(\textit{2}) виртуальных диаграмм ритма 
сердца при заданных $\sigma = 40$~($\times$), 70~($\bullet$) и 
100~мс~($\circ$)
  \label{f3ku}}
  \end{minipage}
  \vspace*{9pt}
  \end{figure*}

  Очевидно, что величины энтропии~$H(X)$ и информационной 
энтропии~$I^*_{\mathrm{г}}$ сближаются при размерах $n$~цифровых рядов, 
стремящихся к бесконечности (см.\ рис.~\ref{f2ku}). Если величина выборки 
конечна, то результаты расчета по этим двум величинам расходятся. Они 
расходятся тем больше, чем меньше величина~$n$. Например, при $n = 1000$ 
отсчетов расхождение достигает 20\%. Такое расхождение объясняется тем, что 
связующим звеном между двумя формами записи энтропии является формула 
Стирлинга для случая $n\rightarrow \infty$~\cite{6ku, 29ku}. При уменьшении 
величины~$n$ расчетная погрешность этой формулы нарастает.
  
  Уравнения трендовых линий функциональных кривых (см.\ рис.~1,\,\textit{а},~\textit{г}) могут 
быть представлены в общем виде $3\sigma = 2^{I^*+i}$ при коэффициенте 
достоверности аппроксимации $R^2> 0{,}9$ (см.\ рис.~\ref{f1ku},\,\textit{а}) для 
ряда значений $i = 0$, 1, 2, 3. Эта формула, с одной стороны, иллюстрирует 
<<правило~$3\sigma$>> для ДРС в интерпретации, отличной от 
общепринятой~\cite{15ku, 23ku}: величина~$3\sigma$ представляет полное 
число комбинаций (число кодонов) при переходе от <<алфавита>> с двумя 
буквами к алфавиту с $k$~буквами ($k < 3\sigma$) и переменным размером 
кодона ($I^*+ i$). С~другой стороны, она может быть представлена в виде: $I^* 
= \log_23\sigma -i$. Такую форму записи автор назвал <<формулой ФСО>>, так 
как по величине~$i$ определяются и уровень ФСО здорового человека, и 
степень тяжести заболевания больного (см.\ далее). 
  
  При сопоставлении теоретической и экспериментальной форм записи 
энтропии~$H(X)$ и~$I^*$ величина~$\sigma$ оказывается одинаковой, 
поэтому в качестве переменных параметров могут служить~$\Delta x$ и~$i$. 
Если в формуле математического ожидания энтропии цифрового ряда 
случайных величин, распределенных по нормальному закону 
\begin{equation}
H(X) \approx  \log_2\left(\fr{4{,}13\sigma}{\Delta x}\right)\,,
\label{aa}
\end{equation}
принять шаг дискретизации исходной непрерывной 
функции $\Delta x = 1$~мс, то 
$$H(X) \approx \log_2(4{,}13\sigma)\,,
$$ 
где единица 
измерения $[\sigma ] = 1$~мс. При $\Delta x = 2$~мс получим 
$$
H(X) \approx  \log_2(4{,}13\sigma) - 1\,;$$
при $\Delta x = 4$~мс~--- 
$$H(X) \approx  \log_2(4{,}13\sigma)- 2$$ 
и~т.\,д. Точка на графике~$H(\sigma, \Delta x)$ 
сдвигается влево на единицу при изменении величины~$\Delta x$ кратно~2. 
Если $\Delta x = 0{,}5$~мс, точка на указанном графике (см.\ рис.~\ref{f2ku}) 
сместится на единицу вправо. При фиксированном значении~$\sigma$ 
график~$H(\Delta x)$ будет линейным в полулогарифмическом масштабе. 
  
  Величина шага дискретизации $\Delta x = 1$, определенная приборной 
частотой, не равна выборочному среднему расстоянию между ярусами, т.\,е.\ 
средней величине межъярусного промежутка $\langle 
\Delta_{\mathrm{ЯП}}\rangle$. Результаты анализа, проведенные для 
виртуальной ДРС, показали, что с ростом объема выборки~$n$ число 
межъярусных промежутков~$N_{\mathrm{МЯП}}$ на диаграмме монотонно 
растет, а величина $\langle \Delta_{\mathrm{ЯП}}\rangle \rightarrow \Delta x$ при 
$n\rightarrow\infty$ (рис.~\ref{f3ku}). 
  

  
  Если для виртуальных цифровых рядов формуле~(\ref{aa})
  поставить в адекватное соответствие форму 
записи 
\begin{equation}
I^*_{\mathrm{г}} \approx 
\log_2 \fr{4{,}13\sigma}{\langle\Delta_{\mathrm{ЯП}}\rangle}\,,
\label{aaa}
\end{equation} 
то при 
$n\rightarrow\infty$ они совпадут. Можно оценить, что $4{,}13\sigma /\langle 
\Delta_{\mathrm{ЯП}}\rangle \approx 3\sigma/\Delta x$ (см.\ рис.~\ref{f3ku}), и 
формула~(\ref{aaa}) описывает предельный режим ритма, 
информационная энтропия ДРС которого определена формулой 
$$I_{\max}^* = 
\log_2\fr{3\sigma}{\Delta x}
$$ (см.\ выше). Ясно, что всегда выполняется тройное 
неравенство: $H(X) > I_{\mathrm{г}}^* \geq I_{\max}^* \geq I^*$. Верхняя 
<<математическая граница>>, с~одной стороны, является идеализацией, а 
с~другой~--- эталоном для ритма сердца как маркер <<правой границы нормы 
условного здоровья>>. 
  
  Следуя той же логике, для реальных ДРС 
  $$
  I^* \approx \log_2\fr{4{,}13\sigma}{\langle \Delta_{\mathrm{ЯП}}\rangle }\,,
  $$ или с учетом экспериментальных 
данных (см.\ рис.~\ref{f1ku},\,\textit{а}) 
$$I^* \approx \log_2\fr{3\sigma}{2^i\Delta x}\,,
$$ 
где $\Delta x = 1$~мс. Эти формулы связывают параметры макроструктуры 
ДРС ($I^*$ и~$\sigma$) с параметрами ее ярусной микроструктуры ($\langle 
\Delta_{\mathrm{ЯП}}\rangle $ и~$i$). При $i = 0$ данные условно 
соответствуют весенним, при $i = 1$~--- зимним, при $i = 2$~--- осенним, при 
$i = 3$ и выше~--- болезни. Для людей одного возраста получается 
возможность организации шкалы ФСО по функции $\langle 
\Delta_{\mathrm{ЯП}}\rangle (N)$ или по величине~$i$ как по группе, так и по 
серии $N$~опытов. Предлагается величину~$i$ определять показателем ФСО: 
$i = 0$~--- норма, $i=1$~--- обратимое угнетение в рамках сезонной адаптации, 
$i = 2$~--- обратимое донозологическое состояние в рамках сезонной 
адаптации, $i = 3$~--- необратимое состояние вне рамок сезонной адаптации 
(патогенез). 


  
  Уравнением, связывающим среднюю величину микроперехода на ДРС и 
информационную энтропию ДРС, может служить $\langle 
\Delta_{\mathrm{ЯП}}\rangle  = - 2I^* + (13\div 17)$ (рис.~4). Для 
здорового молодого человека гради-\linebreak
%\noindent
\begin{center} %fig4
%\vspace*{6pt}
\mbox{%
\epsfxsize=70.456mm
\epsfbox{kuz-4.eps}
}
\end{center}
\vspace*{3pt}
%\begin{center}
{{\figurename~4}\ \ \small{Графики $\langle\Delta_{\mathrm{ЯП}}\rangle (I^*, N)$ по данным 
регистраций ЭКГ здоровых людей}}
%\end{center}
%\vspace*{3pt}

%\bigskip
\addtocounter{figure}{1}


\noindent
ент средней величины микроперехода 
направлен в сторону убывания~$I^*$ и определен величиной~--- $2$~мс/бит. 
При $\langle \Delta_{\mathrm{ЯП}}\rangle \rightarrow (\Delta x = 1)$ показатель 
ФСО $i = 0$ и величине~$I^*$ разрешено варьировать в пределах 6--8~бит; 
при $\langle \Delta_{\mathrm{ЯП}}\rangle \rightarrow 2$ ($i = 1$)~--- $I^* = 
5{,}5$--7,5~бит; при $\langle \Delta_{\mathrm{ЯП}}\rangle  \rightarrow 4$ 
($i=2$) $I^* = 4{,}5$--6,5~бит. Переход <<весна--осень>> соответствует 
дискретному четырехкратному увеличению величины~$\langle 
\Delta_{\mathrm{ЯП}}\rangle $ (см.\ рис.~4). 

  
\section{Относительная информационная энтропия}
  
  Оба энтропийных параметра: $I^*$ для ДРС и~$I_{\mathrm{г}}^*$ для 
виртуальной ДРС~--- зависят от~$n$~\cite{3ku, 6ku}. Это доставляет неудобства 
при сравнении цифровых рядов разной длины. Следовательно, необходим 
параметр, который бы сохранял информацию о неупорядоченности ярусной 
структуры ДРС и не зависел от~$n$. 
  
  При сравнении ДРС с ВДРС информационная энтропия~$I_{\mathrm{г}}^*$ 
имеет ту же функциональную погрешность по~$n$, что и~$I^*$ для реальной 
ДРС~\cite{6ku}. Поэтому их отношение $i_r = I^/I_{\mathrm{г}}^*$ становится 
свободным от функциональной погрешности, связанной с конечностью числа 
измерений. Оба энтропийных параметра~$I^*$ и~$I_{\mathrm{г}}^*$ зависят и 
от~$n$, и от~$\sigma$, но в первом из них содержится информация о 
функции~$\sigma(n)$, т.\,е.\ о детерминистской составляющей сигнала. 
Поэтому их отношение $I^*/I_{\mathrm{г}}^*$, оценивая это влияние, является 
мерой неупорядоченности ярусной структуры ДРС по отношению к 
максимально возможной неупорядоченности, ограниченной задаваемыми 
параметрами, сводя влияние объема выборки к пренебрежимо малому.
  
  Таким образом, относительную информационную энтропию 
($I^*/I_{\mathrm{г}}^*$) можно использовать как индикатор уровня 
ФСО по фактору регуляции ритма в 
смысле отклонения от нормального закона распределения. При этом можно 
исходить из утверждения, что отклонение гомеостатической функции от 
стационарного уровня вызвано внешним управляющим влиянием и в любом 
проявлении приводит к появлению и интенсификации работы механизмов 
регуляции. Механизмы адаптации тормозят процессы отклонения, а механизмы 
регуляции возвращают его в норму. 
  \begin{table*}\small
  \begin{center}
  \Caption{Показатели ФСО по всем регистрациям
  \label{t1ku}}
  \vspace*{2ex}
  
  \begin{tabular}{|l|c|c|c|c|c|l|}
  \hline
\multicolumn{1}{|c|}{\tabcolsep=0pt\begin{tabular}{c}Время\\ Серия или группа\end{tabular}}&
\tabcolsep=0pt\begin{tabular}{c} $N_{\mathrm{рег}}$\end{tabular}&
\tabcolsep=0pt\begin{tabular}{c}ЧСС,\\ уд./мин\end{tabular}&
\tabcolsep=0pt\begin{tabular}{c}Рабочие ярусы,\\ \%\end{tabular}&
\tabcolsep=0pt\begin{tabular}{c}$\langle  \Delta_{\mathrm{ЯП}}\rangle$,\\ мс\end{tabular}&
\tabcolsep=0pt\begin{tabular}{c}$I^*/I_{\mathrm{г}}^*$,\\ \%\end{tabular}&
\tabcolsep=0pt\begin{tabular}{c}Формула ФСО \\(см.\ рис.~\ref{f1ku},\,\textit{а}--\textit{г})\end{tabular}\\
\hline
Декабрь--январь 2008~г. &&&&&&\\
\tabcolsep=0pt\begin{tabular}{l}К.\ (21~г.) \\ Ш.\ (21~г.)\end{tabular}&
\tabcolsep=0pt\begin{tabular}{c}  34\\  48\end{tabular}&
\tabcolsep=0pt\begin{tabular}{c} $74 \pm 2$\\ $75 \pm 2$\end{tabular}&
\tabcolsep=0pt\begin{tabular}{c} $34{,}4 \pm 2{,}7$\\ $40{,}4 \pm 1{,}6$\end{tabular}&
\tabcolsep=0pt\begin{tabular}{c} $2{,}5 \pm 0,1$\\ $3{,}2 \pm 0,3$\end{tabular}&
\tabcolsep=0pt\begin{tabular}{c} $89{,}8 \pm 0{,}7$\\ $87{,}2 \pm 1,8$\end{tabular}&
\tabcolsep=0pt\begin{tabular}{l} $I^* = \log_23\sigma - 1$\\ $I^* = \log_23\sigma - 1$\end{tabular}\\
\hline
Январь--февраль 2008~г. &&&&&&\\
Р.\ (21~г.)&
45& $63 \pm 2$&  $36{,}7 \pm 1{,}4$&  $2{,}8 \pm 0{,}1$& $92{,}1 \pm 0{,}6$&  
$I^* = \log_23\sigma - 1$\\
\hline
Февраль--март  2008~г. &&&&&&\\
\tabcolsep=0pt\begin{tabular}{l}Гр.~32 (19--24~г.) \\Юноши\\ Девушки\end{tabular}&
 \tabcolsep=0pt\begin{tabular}{c}  32\\  20\\  12\end{tabular}&
\tabcolsep=0pt\begin{tabular}{c} $81 \pm 5$\\ $82 \pm 7$\\ $79 \pm 5$\end{tabular}&
\tabcolsep=0pt\begin{tabular}{c}
$72{,}6 \pm 2{,}6$\\ $72{,}3 \pm 3{,}3$\\ $73{,}2 \pm 4{,}2$\end{tabular}&
\tabcolsep=0pt\begin{tabular}{c} \ \\ $1{,}4 \pm 0{,}1$\\ $1{,}4 \pm 0{,}1$\end{tabular}&
\tabcolsep=0pt\begin{tabular}{c} $97{,}9 \pm 2$\\ $99{,}3 \pm 0{,}3$\\ $95{,}5 \pm 
4{,}5$\end{tabular}&
\tabcolsep=0pt\begin{tabular}{l} $I^* = \log_23\sigma$\\ $I^* = \log_23\sigma$\\ $I^* = 
\log_23\sigma$\end{tabular}\\
\hline
Октябрь--ноябрь 2008~г. &&&&&&\\
\tabcolsep=0pt\begin{tabular}{l}К.\ (22~г.)\\ Ш.\ (22~г.)\end{tabular}&
\tabcolsep=0pt\begin{tabular}{c} 34\\  33\end{tabular}&
\tabcolsep=0pt\begin{tabular}{c}
$68 \pm 3$\\ $81 \pm 2 $\end{tabular}&
\tabcolsep=0pt\begin{tabular}{c} $21{,}9 \pm 1{,}3$\\ $41{,}7 \pm 1{,}4$\end{tabular}&
\tabcolsep=0pt\begin{tabular}{c}
$4{,}6 \pm 0{,}2$\\ $2{,}4 \pm 0{,}1$\end{tabular}&
\tabcolsep=0pt\begin{tabular}{c} $79{,}0 \pm 1{,}1$\\ $88{,}5 \pm 0{,}4$\end{tabular}&
\tabcolsep=0pt\begin{tabular}{l}
$I^* = \log_23\sigma - 2$\\  $I^* = \log_23\sigma - 1$ \end{tabular}\\
\hline
Апрель--май 2009~г. &&&&&&\\
Ш.\ (22~г.)& 27& $79 \pm 2$& $73{,}2 \pm 2{,}6$& $1{,}4 \pm 0{,}1$& $99{,}4 \pm 0{,}2$& $I^* = 
\log_23\sigma$\\
\hline
1999--2009~гг. &&&&&&\\
Группа 330 (17--75~лет)&375&---&2--75&
\tabcolsep=0pt\begin{tabular}{c}
 (1,4)\\ (2,8)\\ (4,2)\\ (5,6)\end{tabular}&
20--99&
\tabcolsep=0pt\begin{tabular}{l}
$I^* = \log_23\sigma$\\ $I^* = \log_23\sigma - 1$\\ $I^* = \log_23\sigma - 2$\\ $I^* = \log_23\sigma- 3$\\ 
\ldots\end{tabular}\\
\hline
\multicolumn{7}{l}{\footnotesize(\ )~--- предполагаемые расчетные значения сертификации больных по 
ФСО.}
\end{tabular}
\end{center}
\vspace*{-12pt}
\end{table*}
Механизмы контроля удерживают его 
около нормы в разрешенных пределах флуктуаций. Например, для состояния 
<<весна>> механизмы контроля удерживают отношение~$I^*/I_{\mathrm{г}}^*$ 
около индивидуального значения, близкого к единице (или~100\%) и должны 
быть определены процессами непрерывного действия. По величине этого 
отношения и его динамике человек принимает решение: переходить к 
нозологическим процедурам или в этом нет необходимости. Иными словами, 
величина~$I^*/I_{\mathrm{г}}^*$ и ее динамика позволяют сопоставить 
субъективные оценки состояния человека с относительным уровнем 
хаотичности ритма сердца.
  
  В табл.~\ref{t1ku} представлены данные по сериям регистраций ЭКГ 
обследуемых~К., Ш., Р.\ и группы здоровых молодых людей (32~человека), а 
также данные массовых нерегулярных во времени ре\-гист\-ра\-ций по группе из 
330~человек. В~таблице приведены: число регистраций~$N_{\mathrm{рег}}$ и 
расчетные средние значения частоты сердечных сокращений (ЧСС), 
относительного количества рабочих ярусов, величины межъярусного 
промежутка и относительной информационной энтропии. Все расчеты 
проведены с уровнем значимости $\alpha = 0{,}05$. В~последнем столбце 
таблицы приведены формулы ФСО, представляющие зависимости~$I^*(\sigma, 
N)$ раздельно по ДРС выделенных в строках серий и групп ре\-гист\-раций. 
  


  
  Данные по группе из 330~человек разного воз\-рас\-та (140~здоровых людей, 
235~пациентов отделений реанимации за период 1999--2009~гг.) выделены в 
отдельную нижнюю строку таблицы. Соответствующие графики представлены 
ранее (см.\ рис.~\ref{f1ku},\,\textit{г}). В~сравнении с графиком на 
рис.~\ref{f1ku},\,\textit{а} и данными табл.~1 для них характерными являются 
следующие отличия: 

\noindent
  \begin{enumerate}[(1)]
  \item
   слабо выражены данные, определенные по формуле ФСО <<весна>>; 
  \item  отчетливо проявляются данные, соответст\-ву\-ющие формуле ФСО $I^* = 
\log_2 3\sigma - 3$; 
  \item заметно рассеяние данных, связанное в основном с большим 
расхождением возраста обследуемых.
  \end{enumerate}
  
  Визуальный анализ графиков (см.\ рис.~\ref{f1ku},\,\textit{а}--\textit{в})\linebreak 
и расчетных данных табл.~\ref{t1ku} приводит к следующим общим 
результатам: 
  \begin{enumerate}[1.]
  \item В~течение одного года режим ритма меняется скачкообразно трижды. 
  \item С~позиции~($I^*/I_{\mathrm{г}}^*)_{\min}$ условный год начинается в 
начале октября. Первый триместр, <<осень>> (октябрь--декабрь), и второй 
триместр, <<зима>>\linebreak (декабрь--февраль), характеризуются относительным 
постоянством индивидуальных величин~$I^*/I_{\mathrm{г}}^*$, но 
заканчиваются их скачко\-об\-разным ростом, характеризующим изменение\linebreak 
качества режима ритма и структуры ДРС смещением в сторону превалирования 
хаотической составляющей. Третий триместр, <<весна>> (март--сентябрь), 
характеризуется слабым рассеянием величины~$I^*/I_{\mathrm{г}}^*$ около 
единицы. В~ритме сердца превалирует хаотическая со\-став\-ля\-ющая со слабыми 
признаками проявления механизмов управления и регуляции. 
  \item  В~начале октября режим ритма сердца столь значительно и 
скачкообразно меняет качество, что переходный процесс претендует на 
категорию <<катастрофы>>~\cite{30ku}.
  \end{enumerate}
  
  Предварительные исследования показали, что вне зависимости от того, 
болеет человек или нет, с возрастом~$I^*$ падает и все дальше отходит от 
эталонного значения. Отношение~$I^*/I_{\mathrm{г}}^*$ с возрастом 
уменьшается, но не монотонно, а ступенчато, как бы задерживаясь на 
определенных функциональных уровнях, которые могут служить индикаторами 
биологического возраста человека. По колебанию величины этого отношения 
около индивидуального уровня соответствующего возраста можно определить 
уровень ФСО, поэтому данное отношение предлагается в качестве индикатора 
донозологической диагностики вне зависимости от возраста человека.
  
\section{Выводы}
  
  \noindent
  \begin{enumerate}[1.]
  \item Между параметрами ВСР и информационной энтропией ДРС в 
условиях дискретной сезонной адаптации существуют прочные 
функциональные связи. При смене сезона заполняются новые 
<<функциональные уровни>>, соответствующие обретению ритмом 
качественно новых стационарных режимов. 
  \item При изменении длины записи ЭКГ графики функциональных кривых 
сохраняются, меняя длину изменением координат правой или левой границы.
  \item Сравнительный анализ реальной ДРС и соответствующей ей 
виртуальной ДРС позволяет оценивать влияние системы регуляции на ритм 
сердца в форме отклонения распределения значений $R$--$R$ интервалов на 
ДРС от нормального закона.
  \item Всегда выполняется тройное неравенство: $H(X) > I_{\mathrm{г}}^* \geq 
I_{\max}^* \geq I^*$. Верхняя <<математическая граница>>, с одной стороны, 
является идеализацией, а с другой~--- эталоном для ритма сердца как маркер 
<<правой границы нормы условного здоровья>>. 
  \item Формулы ФСО адекватно связывают па\-ра\-мет\-ры макроструктуры ДРС 
($I^*$ и~$\sigma$) с па\-ра\-мет\-ра\-ми ее ярусной микроструктуры 
($\langle\Delta_{\mathrm{ЯП}}\rangle$, $\Delta x$ и~$i$).
  \item В формуле ФСО $I^* \approx \log_2[(3\sigma)/(2^i\Delta x)]$ 
величину~$i$ предлагается определять показателем ФСО: $ i= 0$~--- норма, $i = 
1$~--- обратимое угнетение в рамках сезонной адаптации, $i = 2$~--- обратимое 
донозологическое состояние в рамках сезонной адаптации, $i = 3$~--- 
необратимое состояние вне рамок сезонной адаптации (патогенез). 
  \item Уравнением, связывающим среднюю величину микроперехода и 
информационную энтропию ДРС, может служить $\langle 
\Delta_{\mathrm{ЯП}}\rangle  = - 2I^* + (13\div17)$. 
  \item  Относительную информационную энтропию ($I^*/I_{\mathrm{г}}^*$) 
можно использовать как индикатор уровня функционального состояния 
организма по фактору регуляции ритма. Режим ритма сердца здорового 
молодого человека вне зависимости от пола в течение календарного года 
трижды дискретно меняет свое качество: от $(I^*/I_{\mathrm{г}}^*)_{\min} \approx  
0{,}8$ в интервале октябрь--ноябрь к $(I^*/I_{\mathrm{г}}^*) \approx 0{,}9$ в 
интервале декабрь--февраль и до $I^*/I_{\mathrm{г}}^* \leq 1$ в интервале 
  март--сентябрь. Замыкает годовой цикл изменений режима ритма наиболее 
резкое изменение качества при возврате к значению 
$(I^*/I_{\mathrm{г}}^*)_{\min} \approx 0{,}8$ в начале октября.
  \item Обнаружено сильное и направленное влияние возраста человека на 
качество режима ритма сердца. Отношение~$I^*/I_{\mathrm{г}}^*$ с возрастом 
уменьшается, но не монотонно, а ступенчато, как бы задерживаясь на 
определенных функциональных уровнях, которые могут служить индикаторами 
биологического возраста человека.
  \end{enumerate}
  

{\small\frenchspacing
{%\baselineskip=10.8pt
\addcontentsline{toc}{section}{Литература}
\begin{thebibliography}{99}

\bibitem{1ku}
Heart rate variability. Standards of measurement, physiological interpretation, and 
clinical use. Task Force of The Europian Society of Cardiology and The North 
American Society of Pacing and Electrophysiology~// European Heart J., 1996. 
Vol.~17. P.~354--381. 

\bibitem{2ku}
\Au{Амиров Н.\,Б., Чухнин Е.\,В.}
Применение метода изуче\-ния вариабельности сердечного ритма при различных 
состояниях (Обзор литературы)~// Диагностика и лечение нарушений 
регуляции сер\-деч\-но-со\-су\-ди\-стой сис\-те\-мы.~--- М.: ГКГ МВД России, 
2008.~С. 63--75.

\bibitem{3ku}
\Au{Кузнецов А.\,А.}
Методы анализа и обработки электрокардиографических сигналов: Новые 
подходы к выделению информации.~--- Владимир: ВлГУ, 2008.~--- 140~с. 

\bibitem{4ku}
\Au{Малиновский Л.\,Г.}
Классификация объектов средствами дискриминантного анализа.~--- М.: Наука, 
1979.~--- 260~с. 

\bibitem{5ku}
\Au{Зозуля Е.\,П.}
Геометрический анализ нелинейных хаотических колебаний кардиоритма как 
новый метод для автоматического обнаружения фибрилляции предсердий~// 
Физика и радиоэлектроника в медицине и экологии. Кн.~1.~--- 
Владимир--Суздаль: ВлГУ, 2008. С.~172--175.

\bibitem{6ku}
\Au{Кузнецов А.\,А.}
Энтропия ритма сердца.~--- Владимир: ВлГУ, 2009.~--- 172~с. 

\bibitem{7ku}
\Au{Прилуцкий Д.\,А., Кузнецов А.\,А., Плеханов~А.\,А., Чепенко~В.\,В.}
Накопитель ЭКГ <<\textit{AnnA Flash}~2000>>~// Методы и средства 
измерений физических величин.~--- Н.~Новгород: НГТУ, 2006. С.~31.

\bibitem{8ku}
Medical Computer Systems, Zelenograd, Moscow. {\sf http://www.mks.ru}. 

\bibitem{9ku}
\Au{Кушаковский М.\,С., Журавлева Н.\,Б.}
Аритмии и блокады сердца (атлас электрокардиограмм).~--- Л.: Медицина, 
1981.~--- 340~с. 

\bibitem{10ku}
\Au{Мун Ф.}
Хаотические колебания: Вводный курс для научных сотрудников и инженеров~/
Пер. с англ. Ю.\,А.~Данилова и А.\,М.~Шукурова.~--- М.: Мир, 1990.~--- 312~с.

\bibitem{14ku} %11
\Au{Shannon C.\,E., Weaver~W.}
The mathematical theory of communication.~--- Urbana, IL: The University of 
Illinois Press, 1949.

\bibitem{13ku} %12
\Au{Матвеев А.\,Н.}
Молекулярная физика: Учеб. пособие для физ. спец. вузов.~--- М.: Высшая 
школа, 1987.~--- 360~с. 

\bibitem{12ku} %13
Биофизика: Учебник~/ Под ред. акад. П.\,Г.~Костюка.~--- Киев: Выща школа, 
1988.~--- 504~с.

\bibitem{11ku} %14
\Au{Блюменфельд Л.\,А.}
Информация, термодинамика и конструкция биологических систем~// СОЖ, 
1996. №\,7. С.~88--92. 

\bibitem{15ku}
\Au{Вентцель Е.\,С.}
Теория вероятностей: Учебник для вузов.~--- М.: Высшая школа, 1999.~--- 
576~с. 

\bibitem{19ku} %16
\Au{Shaw R.}
Strange attractors, chaotic behavior and information flow~// Z. Naturforsh., 1981. 
Vol.~A36. P.~80--112.

\bibitem{17ku}
\Au{Farmer J.\,D., Ott E., Yorke~J.\,A.}
The dimension of chaotic attractors~// Physica, 1983. Vol.~7D. P.~153--170. 

\bibitem{18ku}
\Au{Grassberger P., Proccacia~I.}
Characterization of strange attractors~// Phys. Rev. Lett., 1983. Vol.~50. 
P.~346--349.

\bibitem{16ku} %19
\Au{Кузнецов А.\,А.}
Энтропия, количество информации и информационная размерность 
$RR$-ин\-тер\-ва\-ло\-грам\-мы~// Биомедицинские технологии и 
радиоэлектроника, 2008. №\,6. С.~15--19.


\bibitem{20ku}
\Au{Баевский Р.\,М., Берсенева А.\,П.}
Введение в донозологическую диагностику.~--- М.: Слово, 2008.~--- 176~с. 

\bibitem{21ku}
\Au{Кобзарь А.\,И.}
Прикладная математическая статистика.~--- М.: Физматлит, 2006.~--- 816~с.

\bibitem{22ku}
\Au{Орлов Ю.\,Н., Осминин К.\,П.}
Методика определения оптимального объема выборки для прогнозирования 
нестационарного временного ряда~// Информационные технологии и 
вычислительные системы, 2008. №\,3. С.~3--13. 

\bibitem{23ku}
\Au{Королюк В.\,С., Портенко Н.\,И., Скороход~А.\,В., Турбин~А.\,Ф.} 
Справочник по теории вероятностей и математической статистике.~--- М.: 
Наука, 1985.~--- 640~с.

\bibitem{25ku} %24
\Au{Babloyantz A., Destexhe A.}
Is the normal heart a periodic oscillator~// Biol. Cybern., 1988. Vol.~58. P.~203.

\bibitem{24ku} %25
\Au{Pool R.}
Is it healthy to be chaotic?~// Science, 1989. Vol.~243. P.~604.


\bibitem{26ku}
\Au{Эйдукайтис А., Варонецкас Г., Жемайтите~Д.}
Применение теории хаоса для анализа сердечного ритма в различных стадиях 
сна у здоровых лиц~// Физиология человека, 2004. Т.~30. №\,5. С.~56--61.

\bibitem{27ku}
\Au{Кузнецов А.\,А.}
Проверка возможности применения функциональных уравнений для оценки 
состава и распределения ритма сердца~// Биомедицинские технологии и 
радиоэлектроника.~--- М.: Радиотехника, 2008. №\,3. С.~17--20.

\bibitem{28ku}
\Au{Кузнецов А.\,А.}
Структурно-топологические особенности диаграмм ритма сердца~// 
Инфокоммуникационные технологии, 2009. Т.~7. №\,3. С.~80--85. 

\bibitem{29ku}
Математическая энциклопедия~/ Гл. ред. И.\,М.~Виноградов. Т.~5.~--- М.: Сов. 
энциклопедия, 1984.~--- 1248~с.

 \label{end\stat}

\bibitem{30ku}
\Au{Арнольд В.\,И.}
Теория катастроф.~--- М.: Наука, 1990.~--- 128~с.
 \end{thebibliography}
}
}


\end{multicols}