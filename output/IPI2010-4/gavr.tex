%\newcommand{\ud}{\Delta_n}%{\rho(F_n,\Phi)} %uniform distance
%\newcommand{\infi}[1]{\inf\left\{#1\right\}}
%\newcommand{\supi}[1]{\sup\left\{#1\right\}}
%\DeclareMathOperator{\sign}{sign}
%\renewcommand{\ge}{\geqslant}
%\renewcommand{\le}{\leqslant}

%\input{macros}

\def\stat{gavr}

\def\tit{ОЦЕНКИ СКОРОСТИ СХОДИМОСТИ РАСПРЕДЕЛЕНИЙ СЛУЧАЙНЫХ СУММ С~БЕЗГРАНИЧНО ДЕЛИМЫМИ ИНДЕКСАМИ К~НОРМАЛЬНОМУ ЗАКОНУ$^*$}

\def\titkol{Оценки скорости сходимости распределений случайных сумм с~безгранично делимыми индексами}
% к~нормальному закону}

\def\autkol{С.\,В.~Гавриленко}
\def\aut{С.\,В.~Гавриленко$^1$}

\titel{\tit}{\aut}{\autkol}{\titkol}

{\renewcommand{\thefootnote}{\fnsymbol{footnote}}\footnotetext[1]
{Работа выполнена при поддержке Министерства образования и науки
(государственный контракт 16.740.11.0133 от 02.09.2010).}}

\renewcommand{\thefootnote}{\arabic{footnote}}
\footnotetext[1]{Московский государственный университет им.\ М.\,В.~Ломоносова, факультет 
вычислительной математики и кибернетики, gavrilenko.cmc@gmail.com}

\vspace*{6pt}

\Abst{Построены новые оценки скорости сходимости
распределений случайных сумм с безгранично делимыми индексами,
справедливые при существенно более широких условиях, нежели
известные. Показано, что новые оценки также могут быть заметно
более точными, нежели известные. В~качестве примера применения
этих результатов построены оценки точности нормальной
аппроксимации для распределений случайных сумм с индексами,
имеющими отрицательное биномиальное распределение.}

\vspace*{2pt}

\KW{случайная сумма; целочисленное безгранично
делимое распределение; обобщенное пуассоновское распределение;
отрицательное биномиальное распределение; нормальная аппроксимация}

\vspace*{12pt}

       \vskip 14pt plus 9pt minus 6pt

      \thispagestyle{headings}

      \begin{multicols}{2}

      \label{st\stat}




\section{Оценки скорости сходимости распределений случайных сумм с безгранично делимыми индексами}

Формально задача оценивания точности нормальной аппроксимации для
распределений случайных сумм с индексами, имеющими це\-ло\-чис\-ленные
безгранично делимые (т.\,е.\ обобщенные\linebreak пуассоновские)
распределения, имеет менее общий характер по сравнению с
рассмотренной, например, в работах~[1--3]
ситуацией, когда индекс~$M$ может иметь произвольное
распределение. Тем не менее данный частный случай представляет
существенный интерес с точки зрения практического применения
соответствующих моделей. Так, целочисленный случайный процесс с
независимыми приращениями представляет собой совокупность
случайных величин~$M_t,$ имеющих целочисленное безгранично делимое
(и, значит, обобщенное пуассоновское) распределение. Класс
целочисленных безгранично делимых распределений включает в себя
много широко применяемых представителей, в частности отрицательное
биномиальное распределение и~др. Задачи исследования случайных
сумм с индексами, являющимися значениями упомянутых процессов,
возникают при анализе систем массового обслуживания, в частности
вычислительных и телекоммуникационных систем, процессов риска в
страховой и финансовой математике и многих других областях.

Прежде всего введем некоторые обозначения, упрощающие запись
результатов данной статьи и их доказательств.

Предположим, что все рассматриваемые в данной статье случайные
величины заданы на одном и том же вероятностном пространстве~$(\Omega,\mathcal{A},\p)$. 
Сов\-па\-де\-ние распределений случайных
величин~$X$ и~$Y$ будем обозначать $X \stackrel{d}{=} Y$. Функцию
распределения и характеристическую функцию любой случайной
величины~$X$ обозначим~$F_X(x)$ $(x\in\mathbb{R})$ и~$f_X(t)$
$(t\in \mathbb{R})$ соответственно, а производящую функцию любой
неотрицательной целочисленной случайной величины~$M$ обозначим
$\psi_M(z)$ $(|z|\leqslant 1)$.

Пусть $M$~--- некоторая неотрицательная целочисленная случайная
величина, $X$ -- произвольная случайная величина. Случайную
величину, характеристическая функция которой равна
$\psi_M(f_X(t)),$\linebreak будем обозначать символом~$\{M,X\}$. Не\-слож\-но
убедиться (см., например,~\cite{1g}), что если %\linebreak 
случайная величина~$Z$ 
может быть представлена в \mbox{виде} $Z \stackrel{d}{=} \sum\limits_{j=1}^M
X_j$, где $X_1, X_2, \ldots$~--- одинаково распределенные случайные
величины, причем случайные величины $M, X_1, X_2, \ldots $ независимы, то
$Z\stackrel{d}{=}\{M,X\}$, где $X \stackrel{d}{=} X_j$ (для
определенности будем считать, что $\sum\limits_{j=1}^0 X_j = 0$).
Случайную величину~$\{M,X\}$ будем называть случайной суммой. При
этом случайная величина~$M$ будет называться индексом, а случайная
величина~$X$~--- случайным слагаемым.

Везде далее будем считать, что случайное слагаемое~$X$
удовлетворяет условиям
\begin{equation}
 \e X =0\,, \quad \D X =1\,.
 \label{Tu02}
\end{equation}

Предположим, что распределение случайной величины~$M$ является
безгранично делимым в классе распределений неотрицательных
целочисленных случайных величин, т.\,е.\ для любого натурального
числа~$n$ существует такая неотрицательная целочисленная случайная
величина~$M'_n$, что
$$
M \stackrel{d}{=} \{n,M'_n\}\,.
$$
Как известно (см., например,~\cite{3g}, гл.~XII, \S~3),
распределение является безгранично делимым в классе распределений
неотрицательных целочисленных случайных величин тогда и только
тогда, когда оно является обобщенным пуассоновским, т.\,е.\
соответствующая ему характеристическая функция имеет вид
$$
f_M(t) = \exp[\lambda(f_Y(t)-1)]
$$
для некоторого $\lambda>0$, где~$f_Y(t)$~--- характеристическая
функция некоторой неотрицательной це\-ло\-чис\-ленной случайной величины~$Y$. Другими сло\-вами,
\begin{equation}
 M \stackrel{d}{=} \{N_{\lambda},Y\}\,,
 \label{Tu03}
\end{equation}
где $N_{\lambda}$~--- пуассоновская случайная величина с параметром~$\lambda$. 
Сразу же обратим внимание на то, что для производящей
функции случайной величины~$M$ справедливо соотношение (см.~\cite{3g})
\begin{equation}
 \psi_M(s) = \exp \{\lambda(\psi_Y(s)-1)\}\,,
 \label{Tu04}
\end{equation}
где $\psi_M(s)$~--- это производящая функция случайной величины~$M$, 
а $\psi_Y(s)$~--- производящая функция случайной величины~$Y$,
той самой, что определена в~(\ref{Tu03}).

Будем считать, что
\begin{equation}
 \e |X|^3=\beta_3<\infty
 \label{Tu07}
\end{equation}
и, кроме того, существуют первые три момента случайной величины~$M$.

Отношением Ляпунова (или ляпуновской \mbox{дробью}) будем называть
величину
$$
L(X) = \fr{\e |X|^3}{(\e X^2)^{3/2}}\,.
$$

В данной статье рассмотрим оценку точности аппроксимации
распределения случайной величины $S_M \stackrel{d}{=}  \{M,X\}$
стандартным нормальным законом в ситуации, когда случайная
величина~$M$ имеет целочисленное безгранично делимое
распределение. Очевидно, что в такой ситуации
\begin{equation}
S_M \stackrel{d}{=} \{\{N_{\lambda},Y\},X\}\,.
\label{gp1}
\end{equation}
Центрированную и нормированную случайную сумму $S_M$ обозначим
$$
\tilde{S}_M = \fr{S_M - \e S_M}{\sqrt{\D S_M}}\,.
$$
Хорошо известно, что в сделанных предположениях
$$
\e S_M =\e X\cdot \e M\,,\quad \D S_M = \e M \cdot \D X + \D M
\cdot (\e X)^2\,.
$$
Учитывая~(\ref{Tu02}), замечаем, что в рассматриваемой ситуации
\begin{equation}
\label{Tu043} \e S_M =0\,,\quad \D S_M = \e M\,,
\end{equation}
а значит,
$$
\tilde{S}_M = \fr{S_M}{\sqrt{\e M}}\,.
$$
Обозначим
$$
\Delta = \sup\limits_x\left|F_{\tilde{S}_M}(x) - \Phi(x)\right|\,,
$$
где $\Phi(x)$~--- стандартная нормальная функция распределения.

Оценки точности нормальной аппроксимации для распределений
случайных сумм с произвольными индексами изучались во многих
работах, см., например,~[1--4] и списки литературы в
этих работах. Оценки из работы~\cite{6g} для случая $\e X = 0$
представляются близкими к окончательным. Однако при дополнительной
информации о распределении индекса эти оценки можно уточнить. 
В~част\-ности, в ситуации, когда случайная величина~$M$ имеет
распределение Пуассона, справедлива оценка, которую можно
сформулировать в виде следующей теоремы.

\medskip

\noindent
\textbf{Теорема 1.}
\textit{Если~$M$ имеет распределение Пуассона с параметром~$\lambda$, а
случайное слагаемое~$X$ имеет конечный третий абсолютный момент,
то справедливо неравенство}
\begin{equation}
\label{Tu05} \Delta \leqslant 0{,}3041\fr{L(X)}{\sqrt{\lambda}}\,.
\end{equation}


\medskip

Данная оценка с наилучшей на текущий момент константой~0,3041
получена в работе~\cite{2g}, где также описана предыстория вопроса.
В данной статье оценка~(\ref{Tu05}) будет обобщена на случай,
когда индекс~$M$ имеет целочисленное безгранично делимое
распределение. Аналогичная ситуация рассматривалась и в работе~\cite{5g}. 
Здесь будут построены оценки, альтернативные приведенным
в статье~\cite{5g} и справедливые при более широких условиях.

В статье~\cite{5g} доказана теорема, которая с учетом оценки
абсолютной константы, полученной в работе~\cite{2g}, имеет
следующий вид.

\medskip

\noindent
\textbf{Теорема 1.2.}
\textit{Пусть неотрицательная целочисленная случайная величина~$M$
является безгранично делимой и, кроме того, обладает тремя
конечными первыми моментами. Пусть случайное слагаемое~$X$ также
имеет первые три конечных момента. Тогда справедлива оценка}
\begin{multline}
\Delta = \sup\limits_{x}|\p(S_{M}<\sqrt{\D S_M}x + \e
S_M)-\Phi(x)| \leqslant{}\\
\leqslant \fr{0{,}3041}{\sqrt{\lambda}} \,
\left(\e[Y(Y-1)(Y-2)](\e|X|)^3 + {}\right.\\
\left.{}+3\e[Y(Y-1)]\e|X|\e {X}^2 +{}\right.\\
\left.{}+ \e Y
 \e|X|^3\right)\Big /
 [\e Y^2  (\e X)^2 +\e Y  \D X]^{3/2}\,,
\end{multline}
\textit{где~$\lambda$ и~$Y$~--- элементы представления {\rm (\ref{Tu03})}
случайной величины~$M$.}

\medskip

Обратим внимание, что в теореме требуется только конечность первых
трех моментов случайного слагаемого, при этом совсем не
обязательно выполнение требований нулевого математического
ожидания и единичной дисперсии. В~случае, когда условия~(\ref{Tu02}) соблюдены, имеет место


\smallskip


\noindent
\textbf{Следствие 1.1.}~\textit{Пусть неотрицательная целочисленная случайная величина~$M$
является безгранично делимой и, кроме того, обладает тремя
конечными первыми моментами. Пусть также выполнены условия}~(\ref{Tu02}) \textit{и}~(\ref{Tu07}). \textit{Тогда}
\begin{multline}
\Delta = \sup\limits_{x}|\p(S_{M}<\sqrt{\e M}x)-\Phi(x)| \leqslant
\\
\leqslant \fr{0{,}3041}{\sqrt{\lambda}} \,\left(
\e[Y(Y-1)(Y-2)](\e|X|)^3 +{}\right.\\
\left.{}+ 3\e[Y(Y-1)]\e|X| + \e Y
\e|X|^3\right) \Big /(\e Y)^{3/2}\,,
\label{Tu038}
\end{multline}
\textit{где $\lambda$ и~$Y$~--- элементы представления~{\rm (\ref{Tu03})}
случайной величины~$M$.}


\medskip

В конце данного раздела сравним полученную оценку с оценкой~(\ref{Tu038}), обсудим ее преимущества и недостатки.

Перейдем теперь непосредственно к получению оценки для случайных
сумм с индексом, имеющим целочисленное безгранично делимое (т.\,е.\ обобщенное пуассоновское) распределение. Для этого
понадобится несколько вспомогательных результатов.

\medskip

\noindent
\textbf{Лемма 1.1.} \textit{Пусть~$M$~--- целочисленная неотрицательная случайная величина,
имеющая обобщенное пуассоновское распределение. Тогда для
случайной величины $S_M=\{M,X\}$ справедливы представления
\begin{equation*}
S_M=\{M,X\}\stackrel{d}{=}\{\{N_{\lambda},Y\},X\}\stackrel{d}{=}\{N_{\lambda},U\}\,,
%\label{gp2}
\end{equation*}
где
\begin{equation}
U \stackrel{d}{=}\{Y,X\}\,.
\label{gp3}
\end{equation}
}

\vspace*{-6pt}

\noindent
Д\,о\,к\,а\,з\,а\,т\,е\,л\,ь\,с\,т\,в\,о\,.\ (См., например,~\cite{1g}.) Представление
$S_M\stackrel{d}{=}\{\{N_{\lambda},Y\},X\}$ (см.~(\ref{gp1}))
вытекает из определения случайной величины~$M$ с целочисленным
безгранично делимым распределением и упоминавшегося выше
результата из~\cite{4g} о представимости распределения каждой такой
случайной величины в виде обобщенного пуассоновского.
Представление $S_M\stackrel{d}{=}\{N_{\lambda},\{Y,X\}\}$ легко
получить из вида характеристических функций соответствующих
случайных сумм (см., например,~\cite{1g}, с.~83):
$$
f_{S_{M}}(t)=\psi_M(f_X(t))\,.
$$
Напомним, что здесь~$f_X(t)$~--- это характеристическая функция
случайного слагаемого~$X$. С~учетом представления~(\ref{Tu04})
можно записать
$$
\psi_M(f_X(t))=\exp\{\lambda(\psi_Y(f_X(t))-1)\}\,.
$$
Таким образом, $f_{S_{M}}(t)$ является характеристической функцией
пуассоновской случайной суммы $\{N_\lambda,U\}$, где~$N_\lambda$
имеет распределение Пуассона с параметром~$\lambda$, а случайная
величина~$U$ имеет характеристическую функцию $\psi_Y(f_X(t))$, т.\,е.\
 имеет место представление~(\ref{gp3}), где $Y$~--- случайная
величина, фигурирующая в представлении~(\ref{Tu03}) случайной
величины~$M$.

\medskip


Лемма~1.1 позволяет представить случайную сумму с безгранично
делимым индексом в виде пуассоновской случайной суммы
$$
S_M\stackrel{d}{=}\{N_\lambda,U\}\,,
$$
что, в свою очередь, позволяет использовать теорему~1.1 для
оценивания скорости сходимости распределения случайной величины~$S_M$ 
к нормальному закону. При этом согласно теореме~1.1
$$
\Delta\leqslant \fr{0{,}3041}{\sqrt{\lambda}}\,\fr{\e
|U|^3}{(\e U^2)^{3/2}}\,.
$$

Оценим $\e |U|^3$ и~$\e |U|^2$. С~учетом~(\ref{gp3}) по формуле
полной вероятности имеем
\begin{equation}
\! \e
|U|^3=\e\left|\sum_{k=0}^YX_k\right|^3=\sum\limits_{n=1}^\infty
\p(Y=n)\e\left|\sum\limits_{k=1}^n X_k\right|^3\!.\!\!
\label{Tu06}
\end{equation}
Далее потребуется еще одна лемма.
\pagebreak

%\medskip

\noindent
\textbf{Лемма 1.2.} \textit{Пусть $X_1,X_2,\ldots,X_n$~--- независимые случайные величины с
математическим ожиданием, равным нулю, и конечными моментами
порядка $u\geqslant2$. Тогда
$$
\e \left|\sum\limits_{i=1}^n X_i\right|^u\leqslant C_un^{u/2-1}\sum\limits_{k=1}^n \e
|X_k|^u\,,
$$ 
где
$$
C_u={\fr{u}{2}}\left(u-1\right)\max\left(1,2^{u-3}\right)\left(1+{\fr{2}{u}}K_{2m}^{(u-2)/2m}\right)\,,
$$
целое число~$m$ удовлетворяет условию} 
$$
2m\leqslant u<2m+2\,,\quad K_{2m}=\sum\limits_{j=1}^m \fr{j^{2m-1}}{(j-1)!}\,.
$$


\medskip

\noindent
Д\,о\,к\,а\,з\,а\,т\,е\,л\,ь\,с\,т\,в\,о\,.\ Доказательство можно найти в~\cite{9g} (также см.~\cite{10g}).

\medskip

Продолжим~(\ref{Tu06}). Применим лемму~1.2 к оцениванию
математического ожидания в правой час\-ти~(\ref{Tu06}). Заметим, что
$m=1$ для $u=3$, поэтому $K_{2m}=1$ и, следовательно, $C_3=5$.
Поэтому получаем
\begin{multline}
 \sum\limits_{n=1}^\infty \p(Y=n) \e
\left|\sum\limits_{k=1}^n X_k\right|^3 \leqslant{}\\
{}\leqslant  C_3 \e
|X|^3\sum\limits_{n=1}^\infty \p(Y=n)n^{3/2}=5 \beta_3\e Y^{3/2}\,.
\label{EU3/2}
\end{multline}
Далее,
\begin{multline}
(\e U^2)^{3/2}=\left[\D U+(\e U)^2\right]^{3/2}=\left[\e
Y\cdot\D X +{}\right.\\
\left.{}+(\e X)^2\cdot\D Y+(\e Y\cdot\e X)^2\right]^{3/2}=(\e
Y)^{3/2}\,.
\label{EU2}
\end{multline}
Таким образом, из~(\ref{EU3/2}) и~(\ref{EU2}) получаем, что
$$
\Delta \leqslant 5\cdot
0{,}3041\fr{\beta_3}{\sqrt{\lambda}}\,\fr{\e
Y^{3/2}}{(\e Y)^{3/2}}\,.
$$
Теперь все готово, чтобы сформулировать основной результат данного
раздела.

\medskip

\noindent
\textbf{Теорема 1.3.} \textit{Пусть целочисленная неотрицательная случайная величина~$M$
является безгранично делимой, причем случайная величина~$Y$ в
представлении}~(\ref{Tu03}) \textit{удов\-ле\-тво\-ря\-ет условию $\e
Y^{3/2}<\infty$. Предположим, что выполнены также условия}~(\ref{Tu02}) 
\textit{и}~(\ref{Tu07}). \textit{Тогда справедлива оценка}
\begin{multline}
 \Delta = \sup\limits_{x}|P(S_{M}<\sqrt{\e
M}\,x)-\Phi(x)| \leqslant{}\\
{}\leqslant
1{,}5205\fr{\beta_3}{\sqrt{\lambda}}\,\fr{\e
Y^{3/2}}{(\e Y)^{3/2}}\,.
\label{Tu036}
\end{multline}


%\medskip

Сравним полученную оценку с результатом работы~\cite{5g}.
Во-пер\-вых, сразу заметим, что теорема~1.3 имеет гораздо более
широкую область при\-ме\-ни\-мости, чем следствие~1.1, поскольку в ней
требуется существование только момента $\e Y^{3/2}$, тогда как для
справедливости следствия~1.1 требуется существование конечного
третьего момента случайной величины~$Y$.

Во-вто\-рых, рассмотрим вопрос о том, какая из оценок точнее в
ситуации, когда $\e Y^{3} < \infty$. Для того чтобы сравнить
оценки~(\ref{Tu036}) и~(\ref{Tu038}) в этой ситуации, очевидно,
достаточно сравнить величины
$$
A \equiv 5\e|X|^3 \e Y^{3/2}
$$
и
\begin{multline*}
B \equiv \e[Y(Y-1)(Y-2)](\e|X|)^3 +{}\\
{}+ 3\e[Y(Y-1)]\e|X| + \e Y
\e|X|^3\,.
\end{multline*}
Пусть $X$ имеет двухточечное распределение
$$
X = 
\begin{cases}
 1 & \text{с вероятностью } \fr{1}{2}\,,\\
-1 & \text{с вероятностью } \fr{1}{2}\,.
\end{cases}
$$
Тогда, как несложно видеть,
$$
\e X = 0\,,\quad \e|X|^k = 1
$$
для любого $k =1,2,\ldots$ Таким образом, указанное распределение
$X$ удовлетворяет требованиям~(\ref{Tu02}) и~(\ref{Tu07}).
Вычислим~$A$ и~$B$ для данного распределения. Имеем
\begin{align*}
A &= 5 \e Y^{3/2}\,,\\
B &= \e[Y(Y-1)(Y-2)]\ + 3\e[Y(Y-1)] + \e Y ={}\\
&\!\!\!\!\!{}= \e Y^3 - 3\e Y^2 + 2 \e Y + 3 \e Y^2 - 3 \e Y + \e Y = \e Y^3\,.
\end{align*}
Теперь в качестве распределения неотрицательной целочисленной
случайной величины~$Y$ также возьмем двухточечное распределение
$$
Y= \begin{cases}
{n} & \text{с вероятностью } \fr{1}{2}\,,\\
{0} & \text{с вероятностью } \fr{1}{2}\,,
\end{cases}
$$
где $n$~--- некоторое натуральное число. Легко видеть, что
$$
\e Y^k = \fr{n^k}{2}
$$
для любого $k =1,2,\ldots$ Значит,

\noindent
$$
\fr{B}{A}  = \fr{\e Y^3}{5 \e Y^{3/2}} = \fr{n^3}{5
n^{3/2}} = \fr{n^{3/2}}{5}\,.
$$
Отсюда следует, что если $n \ge 3 > 5^{2/3}$, то $B>A,$ т.\,е.\
оценка~(\ref{Tu036}) лучше оценки~(\ref{Tu038}). Если же $n \leqslant 2
< 5^{2/3}$ (т.\,е.\ $n=1$ или $n=2$), то оценка~(\ref{Tu038})
точнее, чем~(\ref{Tu036}).

Из приведенных примеров можно сделать вывод, что, вообще говоря,
оценки~(\ref{Tu036}) и~(\ref{Tu038}) не сравнимы. Но в ряде
случаев оценка~(\ref{Tu036}) является лучшей. Более того, область
применимости новой оценки намного шире. В~отличие от оценки из
работы~\cite{5g}, ей можно пользоваться в случаях, когда $\e
Y^{3/2} < \infty,$ но $\e Y^3 = \infty$.

\section{Оценки точности нормальной аппроксимации для распределений отрицательных биномиальных
случайных сумм}

В данном разделе рассмотрим случайные суммы с конкретным
целочисленным безгранично делимым индексом~--- случайные суммы с
индексом, имеющим отрицательное биномиальное распределение
(отрицательные биномиальные случайные суммы). Напомним, что
случайная величина~$M$ имеет отрицательное биномиальное
распределение с параметрами $r>0$ и $p \in (0,1),$ если
\begin{equation}
 \p(M=n) =
\fr{\Gamma(r+n)}{n!\Gamma(r)}p^r(1-p)^n\,,\quad  n=0,1,\ldots
\label{Tu041}
\end{equation}

Отрицательные биномиальные случайные суммы находят широкое
применение в качестве математических моделей в теории надежности,
тео\-рии страхового риска, финансовой математике, являясь наиболее
естественной моделью, например, для суммарного количества
затраченных на проект средств до получения первых $r$
положительных денежных потоков (прибылей) от него или для дохода
от кредитного портфеля до первых $r$ дефолтов по входящим в
портфель обязательствам.

Как известно, отрицательное биномиальное распределение может быть
представлено в виде обобщенного пуассоновского, т.\,е.\ является
целочисленным безгранично делимым. Этот факт позволяет
 использовать результат, полученный в предыдущем разделе, с целью
построения оценки точности нормальной аппроксимации для
распределений отрицательных биномиальных случайных сумм.

Итак, пусть $M$~--- случайная величина, имеющая отрицательное
биномиальное распределение с параметрами $r>0$ и $p \in (0,1)$.
Как и прежде, везде далее будем требовать выполнения условий~(\ref{Tu02}) 
и~(\ref{Tu07}). Далее потребуются некоторые вспомогательные результаты.

\medskip

\noindent
\textbf{Лемма 2.1.} \textit{Пусть $M$ -- случайная величина, имеющая отрицательное
биномиальное распределение}~(\ref{Tu041}). \textit{Тогда~$M$ является
обобщенной пуассоновской случайной величиной.}
\medskip

\noindent
Д\,о\,к\,а\,з\,а\,т\,е\,л\,ь\,с\,т\,в\,о\,.\ Этот хорошо известный результат приведен во многих
источниках (см., например,~\cite{3g}, гл.~XII, \S~2). Для удобства
по\-сле\-ду\-юще\-го изложения воспроизведем его доказательство еще раз.
Производящая функция случайной величины~$M$ имеет вид
$$
\psi_{M}(s)=\left(\fr{p}{1-(1-p)s}\right)^r\,.
$$
Запишем ее немного иначе:
\begin{multline*}
\psi_{M}(s)=\exp\left\{r\log\frac{p}{1-(1-p)s}\right\}={}\\
{}=\exp\left\{r\left(\log\fr{1}{1-(1-p)s}+\log
p\right)\right\}={}\\
{}=\exp\left\{r
\log\fr{1}{p}\left[\fr{1}{\log ({1/}{p})}\cdot\log\fr{1}{1-(1-p)s}-1\right]\right\}\,.
\end{multline*}
Обозначим
\begin{equation} 
\lambda= r \log\fr{1}{p}\,,
\quad
\psi(s)=\fr{r}{\lambda}\log\fr{1}{1-(1-p)s}\,.
\label{Tuu03}
\end{equation}
Таким образом,
$$
\psi_{M}(s)=\exp\left\{\lambda(\psi(s)-1)\right\}\,.
$$
Убедимся, что $\psi(s)$ является производящей функцией.
Действительно, при разложении в ряд Тейлора получаем
\begin{multline*}
\psi(s)=\fr{r}{\lambda}\,\log\left(\fr{1}{1-(1-p)s}\right)={}\\
{}=
\fr{1}{\log(1/p)}\sum\limits_{k=1}^\infty
\fr{[(1-p)s]^k}{k}\,.
\end{multline*}
Заметим, что, во-первых,~$\psi(s)$ не зависит от~$r$, а во-вторых,
$$
\fr{1}{\log(1/p)}\sum\limits_{k=1}^\infty\frac{(1-p)^k}{k}=1\,,
$$
откуда следует, что набор
$$
\fr{1}{\log(1/p)}\,\fr{(1-p)^k}{k}\,, \quad
k=1,2,\ldots
$$
задает дискретное вероятностное распределение (такое распределение
принято называть логарифмическим или распределением
логарифмического ряда~\cite{11g, 12g}). Поэтому
\begin{equation} 
\psi(s)=\psi_{Y}(s)=\e s^Y\,,
\label{Tu104}
\end{equation}
где
\begin{equation*}
\p(Y=k)=\fr{1}{\log({1}/{p})}\,\fr{(1-p)^k}{k}\,, \quad k=1,2,\ldots \,,
\end{equation*}
и
$$
M \stackrel{d}{=} \{N_{\lambda},Y\}\,,
$$
что и требовалось доказать.\hfill$\square$

\medskip

Лемма~2.1 показывает, что отрицательная бино\-миальная случайная
величина~$M$ является безгранично делимой, что дает право
воспользоваться\linebreak теоремой~1.3 для оценивания скорости схо\-ди\-мости
отрицательной биномиальной случайной суммы~$S_M$. Согласно теореме~1.3 имеем
\begin{multline*}
\Delta = \sup\limits_{x}|\p(S_{M}<\sqrt{\e M}x)-\Phi(x)| \leqslant{}\\
{}\leqslant
1{,}5205 \fr{\beta_3}{\sqrt{\lambda}}\,\fr{\e
Y^{3/2}}{(\e Y)^{3/2}}\,,
\end{multline*}
что, если вернуться к обозначениям~(\ref{Tuu03}), примет вид
\begin{equation}
 \Delta \leqslant 1{,}5205
K^*(p)\fr{\beta_3}{\sqrt{r}}\,, \label{Tu08}
\end{equation}
где
\begin{equation}
K^*(p)=\fr{\e Y^{3/2}}{(\e
Y)^{3/2}}\,\fr{1}{\sqrt{\log(1/p)}}\,.
\end{equation}
Как известно,
$$
\e M =  \fr{r(1-p)}{p}\,,
$$
что с учетом~(\ref{Tu043}) дает
$$
\e S_M =0\,,  \quad \D S_M = \fr{r(1-p)}{p}\,.
$$
Таким образом,
$$
\Delta  = \Delta_r \equiv \sup\limits_{x}\left|\p\left({
S}_{M}<x\sqrt{\fr{r(1-p)}{p}}\right)-\Phi(x)\right|\,.
$$

Зная распределение случайной величины~$Y$, описанное в~(\ref{Tu104}), 
приступим к оценке~$K^*(p)$ в~(\ref{Tu08}). Имеем
\begin{align*}
\e Y&=\fr{1}{\log(1/p)}\sum\limits_{k=1}^\infty k\fr{(1-p)^k}{k}={}\\
&\hspace*{3mm}{}=
\fr{1}{\log({1}/{p})}\sum\limits_{k=1}^\infty(1-p)^k=\fr{1-p}{p\log({1}/{p})}\,,
\end{align*}
\begin{align*}
\e Y^{3/2}&=\fr{1}{\log({1}/{p})}\sum\limits_{k=1}^\infty k^{3/2} \fr{(1-p)^k}{k}={}\\
&\hspace*{15mm}{}=
\fr{1}{\log({1}/{p})}\sum\limits_{k=1}^\infty \sqrt{k}
(1-p)^k\,.
\end{align*}
Рассмотрим функцию $g(x) = x^{1/2}(1-p)^x.$ Она возрастает на
отрезке $\left[0,\, (1/2)(\log (1/(1-p)))^{-1}\right]$ и
убывает при $x > (1/2)\left(\log(1/(1-p))\right)^{-1}$. Зная
поведение~$g(x)$, можно оценить $\e Y^{3/2}$ следующим образом.
Имеем

\noindent
\begin{multline*}
\e Y^{3/2} \leqslant
\fr{1}{\log({1}/{p})}\left[\int\limits_{0}^\infty
x^{{1}/{2}}(1-p)^x dx +{}\right.\\
\left.{}+
(1-p)^{(1/2)\left(\log(1/({1-p}))\right)^{-1}}
\fr{1}{\sqrt{2}}\left(\log\frac1{1-p}\right)^{-1/2}\right]\!.
%\label{Tu21}
\end{multline*}
Далее,

\noindent
\begin{multline*}
 \int\limits_{0}^\infty x^{1/2}(1-p)^x\, dx
={}\\
{}=
 \left(\log\fr{1}{1-p}\right)^{-1/2}
\left(\log\fr{1}{1-p}\right)^{-1} \int\limits_{0}^\infty
t^{{1}/{2}}e^{-t}\, dt ={}\\
{}=
\left(\log\fr{1}{1-p}\right)^{-3/2}\Gamma(3/2) =
\fr{\sqrt{\pi}}{2}\left(\log\fr{1}{1-p}\right)^{-3/2}\!\!\!,
\end{multline*}
где $\Gamma(a)$, $a>0$,~--- эйлерова гамма-функция,
$$
\Gamma(a)=\int\limits_{0}^{\infty}x^{a-1}e^{-x}\,dx\,.
$$
Таким образом,

\noindent
\begin{multline*}
 \e Y^{3/2} \leqslant
\fr{1}{\log({1}/{p})}\left(\log\fr{1}{1-p}\right)^{-3/2}\times{}\\
{}\times
\left[\fr{\sqrt{\pi}}{2}+\fr{1}{\sqrt{2}}(1-p)^{(1/2)\left(\log (1/(1-p))\right)^{-1}}
\log\fr{1}{1-p} \right]\,.
%\label{Tu31}
\end{multline*}
Так как

\noindent
$$
\log\fr{1}{1-p} \leqslant  p\left(1+\fr{p}{2(1-p)}\right)\,,
$$
то
\begin{multline}
 \e Y^{3/2} \leqslant
\fr{1}{2\log({1}/{p})}\left(\log\fr{1}{1-p}\right)^{-3/2}\times{}\\
{}\times
\left[\sqrt{\pi}+\fr{p}{\sqrt{2}}
\left(\fr{2-p}{1-p}\right)\right]\,.
\label{Tu32}
\end{multline}
Приведем еще одну оценку для $\e Y^{3/2}$. Напомним, что
$$
\e Y^{3/2}=\fr{1}{\log({1}/{p})}\sum\limits_{k=1}^\infty \sqrt{k}
(1-p)^k\,.
$$
Оценим сумму в правой части последнего соотношения с помощью
неравенства Коши--Бу\-ня\-ков\-ского:
\begin{multline*}
\sum\limits_{k=1}^\infty \sqrt{k} (1-p)^k =
\sum\limits_{k=1}^\infty \sqrt{k(1-p)^k} \sqrt{(1-p)^k} \leqslant{}\\
{}\leqslant
\left[\sum\limits_{k=1}^\infty k (1-p)^k \sum\limits_{k=1}^\infty
(1-p)^k\right]^{1/2}\,.
\end{multline*}
Осталось заметить, что
$$
\sum\limits_{k=1}^\infty (1-p)^k = \fr{1-p}{p}
$$
и
$$
\sum\limits_{k=1}^\infty k (1-p)^k = \fr{1-p}{p^2}\,.
$$
В итоге получаем, что
\begin{equation}
 \e Y^{3/2}\leqslant \fr{1-p}{p^{3/2}\log(1/p)}\,.
 \label{Tu33}
\end{equation}
Итак, получены две разные оценки для~$\e Y^{3/2}$. Подставив~(\ref{Tu32}) в выражение для~$K^*(p),$ получим
\begin{multline}
K^*(p) \leqslant  \fr{1}{\left(\log(1/p)\right)^{1/2}} \,
\fr{p^{3/2}}{(1-p)^{3/2}}\left(\log\fr{1}{p}\right)^{3/2} \times{}
\\
{}
\times\fr{1}{2\log({1}/{p})}
\left(\log\fr{1}{1-p}\right)^{-3/2}\times{}\\
{}\times \left[\sqrt{\pi}+\fr{p}{\sqrt{2}}\left(\fr{2-p}{1-p}\right)\right]\,.
\label{K^*(p)}
\end{multline}
Так как
$$
\log{\fr{1}{1-p}} > p\left(1+\fr{p}{2}\right)\,,
$$
то в продолжение~(\ref{K^*(p)}) получаем
\begin{multline}
K^*(p)\leqslant \fr{p^{3/2}}{2(1-p)^{3/2}} \,
\left(\log\fr{1}{1-p}\right)^{-3/2}\times{}\\
{}\times
\left[
\sqrt{\pi}+\fr{p}{\sqrt{2}}\left(\fr{2-p}{1-p}\right)\right]
\leqslant{}
\\
\!{}\leqslant
\fr{\sqrt{2}}{(1-p)^{3/2}(2+p)^{3/2}}
\left[\sqrt{\pi}+\fr{p}{\sqrt{2}}\left(\fr{2-p}{1-p}\right)\right]\,.\!
\label{Tu34}
\end{multline}
Подставив~(\ref{Tu33}) в то же самое выражение, имеем
\begin{multline}
K^*(p) \leqslant \fr{1}{\left(\log({1}/{p})\right)^{1/2}}
\, \fr{p^{3/2}}{(1-p)^{3/2}}\left(\log\fr{1}{p}\right)^{3/2}\times{}\\
{}\times
\fr{1}{\log({1}/{p})} \, \fr{1-p}{p^{3/2}} =
\fr{1}{\sqrt{1-p}}\,.
\label{Tu35}
\end{multline}
Очевидно, что при достаточно малых значениях~$p$ оценка~(\ref{Tu34}) лучше, 
так как $\sqrt{\pi}/2 < 1$. Однако при~$p$,
заметно отличающихся от нуля, лучшей будет оценка~(\ref{Tu35}).
Несложно проверить, что уже при $p > 0{,}15$ оценка~(\ref{Tu35})
выигрывает. Таким образом, можно записать, что
\begin{multline*}
K^*(p) \leqslant
\fr{1}{\sqrt{1-p}}\times{}\\
{}\times\min\left\{1;\,\fr{\sqrt{2}}{(1-p)(2+p)^{3/2}}
 \left[\sqrt{\pi}+\fr{p}{\sqrt{2}}\left(\fr{2-p}{1-p}\right)\right]\right\}.
\end{multline*}

Теперь, наконец, можно сформулировать теорему, являющуюся основным
результатом данного раздела.

\medskip

\noindent
\textbf{Теорема 2.1.} \textit{Пусть выполнены условия}~(\ref{Tu02}) \textit{и}~(\ref{Tu07}).
\textit{Тогда для распределения отрицательной биномиальной случайной суммы~$S_M$ с параметрами $r>0$ 
и $p \in (0,1)$ справедливо неравенство}
\begin{multline*}
\Delta_r=\sup_x
\left|\p\left(S_M<x\sqrt{\fr{r(1-p)}{p}}\right)-{}\right.\\
\left.{}-\Phi(x)  \vphantom{\sqrt{\fr{r(1-p)}{p}}}
\right|
\leqslant
K(p)\fr{\beta_3}{\sqrt{r}}\,,
%\label{Tu22}
\end{multline*}
где
\begin{multline*}
K(p)=\fr{1{,}5205}{\sqrt{1-p}}\times{}\\
{}\times\min\left\{1;\,\fr{\sqrt{2}}{(1-p)(2+p)^{3/2}}
\left[\sqrt{\pi}+\fr{p}{\sqrt{2}}\left(\fr{2-p}{1-p}\right)\right]\right\}.
\end{multline*}


%\medskip

Подводя итог, отметим, что при фиксированных значениях параметра~$p$ 
распределение отрицательных биномиальных случайных сумм,
находящих столь широкое практическое применение, можно приблизить
стандартным нормальным законом. При этом чем больше значение
параметра~$r$, тем точнее приближение. Более конкретно, точность
данной аппроксимации имеет порядок~$O\left(r^{-1/2}\right)$.


{\small\frenchspacing
{%\baselineskip=10.8pt
\addcontentsline{toc}{section}{Литература}
\begin{thebibliography}{99}

\bibitem{8g} %1
\Au{Englund G.} A remainder term estimate in a random-sum central limit theorem~// Теория
вероятностей и ее применения, 1983. Т.~28. Вып.~1. С.~143--149.
Письмо в редакцию // Теория вероятностей и ее применения, 1984. Т.~29. Вып.~1. С.~200--201.

\bibitem{6g} %2
\Au{Королев В.\,Ю.} 
О точности нормальной аппроксимации для распределений сумм случайного числа
 независимых случайных величин~// Теория вероятностей и ее
применения, 1988. Т.~33. Вып.~3. С.~577--581.

\bibitem{7g} %3
\Au{Круглов В.\,М., Королев В.\,Ю.} 
Предельные
теоремы для случайных сумм.~--- М.: МГУ, 1990.

\bibitem{1g} %4
\Au{Королев В.\,Ю., Бенинг В.\,Е., Шоргин С.\,Я.} 
Математические основы теории риска.~--- М.: Физматлит, 2007.

\bibitem{3g} %5
\Au{Феллер В.} Введение в теорию вероятностей и ее приложения. Т.~1.~--- М.: Мир,
1984.

\bibitem{2g} %6
\Au{Королев В.\,Ю., Шевцова И.\,Г.} Уточнение неравенства
Берри--Эссеена с приложениями к пуассоновским и смешанным
пуассоновским случайным суммам~// Обозрение прикладной и
промышленной математики, 2010. Т.~17. Вып.~1. С.~25--56.


\bibitem{5g} %7
\Au{Шоргин С.\,Я.} О точности нормальной аппроксимации распределений случайных сумм с
безгранично делимыми индексами~// Теория вероятностей и ее
применения, 1996. Т.~41. Вып.~4. С.~920--926.

\bibitem{4g} %8
\Au{Феллер В.} 
Введение в теорию вероятностей и ее приложения. Т.~2.~--- М.: Мир,
1984.


\bibitem{9g}
\Au{Dharmadhikari S.\,W., Jogdeo K.} Bounds on moments of certain random
variables~// Ann. Math. Statist., 1969. Vol.~40. No.\,4. P.~1506--1508.

\bibitem{10g}
\Au{Петров В.\,В.} Суммы независимых случайных
величин.~--- М.: Наука, 1972.

\bibitem{11g} 
\Au{Fisher R.\,A., Corbet A.\,S., Williams C.\,B.}  The
relation between the number of species and the number of
individuals~// J. Animal Ecology, 1943. Vol.~12. P.~42--58.

 \label{end\stat}

\bibitem{12g} \Au{Кендалл М.\, Стьюарт A.} Теория
распределений.~--- М.: Наука, 1966.
 \end{thebibliography}
}
}


\end{multicols}