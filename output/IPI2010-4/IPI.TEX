\documentclass[10pt]{book}
\usepackage[utf8]{inputenc}

\usepackage{latexsym,amssymb,amsfonts,amsmath,indentfirst,shapepar,%fleqn,%
picinpar,shadow,floatflt,enumerate,multicol,colortbl,ipi}

\usepackage{rotating}
\input{epsf}

%\nofiles

%\includeonly{avtor,avtor-eng}
%\includeonly{avtor-eng}
%\includeonly{pred}  %+
%\includeonly{podgot-2str}  %+



%\includeonly{serebr} %1+pdf
%\includeonly{arkhipov} %2+pdf
%\includeonly{tsiskar} %3pdf
%\includeonly{krylov}  %4+pdf
%\includeonly{kuzn}  %5+pdf
%\includeonly{frenkel}  %6+pdf
%\includeonly{koles}%7+pdf
%\includeonly{matushenko} %8+pdf
%\includeonly{shestakov}%9+pdf
%\includeonly{gavr}%10+pdf
%\includeonly{chuprakov}  %9 +pdf


%\includeonly{toc-rus, toc-en}
%\includeonly{toc-en}

%\includeonly{obchak}
%\includeonly{reshal}
%\includeonly{eng-index}
%\includeonly{cover3}

\usepackage{acad}
\usepackage{courier}
\usepackage{decor}
\usepackage{newton}
\usepackage{pragmatica}
\usepackage{zapfchan}
\usepackage{petrotex}
\usepackage{bm}                     % полужирные греческие буквы
\usepackage{upgreek}                % прямые греческие буквы
%\usepackage{verbatim}

\renewcommand{\bottomfraction}{0.99}
\renewcommand{\topfraction}{0.99}
\renewcommand{\textfraction}{0.01}

\setcounter{secnumdepth}{1} %здесь - 3 + chapter = 4

\arraycolsep=1.5pt

%\usepackage[pdftex]{graphicx}

%\usepackage{oz}

%NEW COMMANDS



%\renewcommand{\r}{{\rm I\hspace{-0.7mm}\rm R}}
\renewcommand{\r}{\mathbb{R}}
\newcommand{\I}{{\rm I\hspace{-0.7mm}I}}
\newcommand{\Ik}{\mbox{{\small \tt {1}}\hspace{-1.5mm}{\tt 1}}}
%\newcommand{\Ikl}{{\small \tt{1}}\hspace*{-0.4mm}\mathtt{1}}

%\mathrm{I}\hspace*{-0.7mm}\mathrm{R}

\newcommand{\il}[2]{\int\limits_{#1}^{#2}}%интеграл с пределами #1 и #2

%\newcommand{\il}[0]{\int\limits_{#1}^{#2}}%интеграл с пределами #1 и #2

\newcommand{\h}{{\bf H}}
\newcommand{\p}{{\sf P}}  % вероятность
\newcommand{\e}{{\sf E}}  % мат. ожидание
\newcommand{\D}{{\sf D}}  % дисперсия
\newcommand{\eps}{\varepsilon}
\newcommand{\vp}{\mathrm{v.p.}}
\newcommand{\F}{{\mathcal F}}
%\def\iint{\int\limits_{-\infty}^{\infty}}
\newcommand{\abs}[1]{\left|#1\right|}

\DeclareMathOperator{\sign}{sign}

%\newcommand{\gr}{{\geqslant}}

\newcommand{\g}{\mbox{\textit{g}}}

%\renewcommand{\la}{\lambda}
\newcommand{\si}{\sigma}
%\renewcommand{\a}{\alpha}

%\newcommand{\pto}{\stackrel{P}{\longrightarrow}} % сходимость по веpоятности

%\newcommand{\eqd}{\stackrel{d}{=}} % равенство по pаспpеделению

%\newcommand{\kp}{\kappa}
%\def\Q{{\cal Q}} \def\H{{\cal H}}
%\newcommand{\bet}{\beta_{2+\delta}}


%\newtheorem{definition}{Определение}
%\renewcommand{\thedefinition}{\arabic{definition}.}
%END NEW COMMANDS

%\renewcommand{\baselinestretch}{1.2}

%\pagestyle{myheadings}

\setlength{\textwidth}{167mm}      % 122mm
\setlength{\textheight}{658pt}
%\setlength{\textheight}{635.6pt}
\setlength{\columnsep}{4.5mm}

\setcounter{secnumdepth}{4}

%\addtolength{\headheight}{2pt}
%\addtolength{\headsep}{-2mm}

%\addtolength{\topmargin}{-20mm}  % for printing


\hoffset=-30mm  % From Yap
%\hoffset=-20mm  % From Acrobat

%\voffset=0mm % From Yap
%\voffset=-15mm   % From Acrobat

\addtolength{\evensidemargin}{-9.5mm} % for printing
\addtolength{\oddsidemargin}{9.5mm}  % for printing

%\renewcommand{\thefootnote}{\fnsymbol{footnote}}
%\renewcommand{\thefootnote}{\arabic{footnote}}
\renewcommand{\figurename}{\protect\bf Рис.}
\renewcommand{\tablename}{\protect\bf Таблица}

\newcommand{\Caption}[1]{\caption{\protect\small %\baselineskip=2.5ex
#1}}

\renewcommand{\thefigure}{\arabic{figure}}
\renewcommand{\thetable}{\arabic{table}}
\renewcommand{\theequation}{\arabic{equation}}
\renewcommand{\thesection}{\arabic{section}}

\renewcommand{\contentsname}{СОДЕРЖАНИЕ}
\newcommand{\fr}[2]{\displaystyle\frac{\displaystyle #1\mathstrut}{\displaystyle #2\mathstrut}}

%\renewcommand{\thefootnote}{\fnsymbol{footnote}}
%\newcommand{\g}{\mbox{\textit{g}}}

%\newcommand{\Caption}[1]{\caption{\protect\small\baselineskip=2ex #1}}
\newcounter{razdel}
\setcounter{razdel}{0}


\newcommand{\titel}[4]{%
\

\vspace*{5pt}

\ifodd\therazdel {\raggedright\noindent\Large\textrm\textbf
 \lineskip .75em
  \baselineskip=3.2ex #1 \par}
\vskip 1em {\noindent\large\textrm\textbf #2 \par}
\addcontentsline{toc}{subsection}{{\textrm\textbf #3}\protect\newline #1}
\def\rightheadline{\underline{\noindent\hbox to \textwidth{\hfill\small\textrm{#4}
%\hfill \large\bf\thepage
}}}
\def\leftheadline{\underline{\noindent\parbox{\textwidth}{
%\raggedleft\large\bf\thepage \hfill
\small\textit{#3}\hfill}}}
\def\leftfootline{\small{\textbf{\thepage}
\hfill ИНФОРМАТИКА И ЕЁ ПРИМЕНЕНИЯ\ \ \ том~4\ \ \ выпуск 4\ \ \ 2010}
}%
 \def\rightfootline{\small{ИНФОРМАТИКА И ЕЁ ПРИМЕНЕНИЯ\ \ \ том~4\ \ \ выпуск~4\ \ \ 2010
\hfill \textbf{\thepage}}} 
\vskip 2em \setcounter{figure}{0}
\setcounter{table}{0} 
\setcounter{equation}{0} 
\setcounter{section}{0}
\setcounter{subsection}{0} 
\setcounter{subsubsection}{0}
\setcounter{footnote}{0} 
\setcounter{razdel}{0}
%\end{flushleft}
\else {
 \raggedright\noindent\Large\textrm\textbf
 \lineskip .75em
\baselineskip=3.2ex #1 \par} \vskip 1em
%\begin{flushleft}
{\noindent\large\textrm\textbf #2 \par}
\addcontentsline{toc}{subsection}{{\textrm\textbf #3}\protect\newline #1}
\def\rightheadline{\underline{\noindent\hbox to \textwidth{\hfill\small\textrm{#4}
%\hfill \large\bf\thepage
}}}
\def\leftheadline{\underline{\noindent\parbox{\textwidth}{%\raggedleft\large\bf\thepage \hfill
\small\textit{#3}\hfill}}}
\def\leftfootline{\small{\textbf{\thepage}
\hfill ИНФОРМАТИКА И ЕЁ ПРИМЕНЕНИЯ\ \ \ том~4\ \ \ выпуск~4\ \ \ 2010}
}%
 \def\rightfootline{\small{ИНФОРМАТИКА И ЕЁ ПРИМЕНЕНИЯ\ \ \ том~4\ \ \ выпуск~4\ \ \ 2010
\hfill \textbf{\thepage}}} \vskip 2em \setcounter{figure}{0}
\setcounter{table}{0} \setcounter{equation}{0} \setcounter{section}{0}
\setcounter{subsection}{0} \setcounter{subsubsection}{0}
\setcounter{footnote}{0}
%\end{flushleft}
\fi}

\newcommand{\titelr}[2]{%
\

\vspace*{5pt}

\ifodd\therazdel {\raggedright\noindent\large\textrm\textbf
 \lineskip .75em
  \baselineskip=3.2ex #1 \par}
\vskip 1em {\noindent\normalsize\textrm\textbf #2 \par}
\else {
 \raggedright\noindent\large\textrm\textbf
 \lineskip .75em
\baselineskip=3.2ex #1 \par} \vskip 1em
%\begin{flushleft}
{\noindent\normalsize\textrm\textbf #2 \par}
\fi}

\newcommand{\titele}[5]{%
\

%\vspace*{5pt}

\ifodd\therazdel {\raggedright\noindent%\large
\textrm\textbf
 \lineskip .75em
%  \baselineskip=3.2ex
#1 \par}
\vskip .5em {\noindent\large\textrm\textbf #2 \par}
\vskip .5em
 {\noindent\textrm #3 \par}
\addcontentsline{toc}{subsection}{{\textrm\textbf #1}\protect\newline #2}
\def\rightheadline{\underline{\noindent\hbox to \textwidth{\hfill\small\textrm{#4}
%\hfill \large\bf\thepage
}}}
\def\leftheadline{\underline{\noindent\parbox{\textwidth}{
%\raggedleft\large\bf\thepage \hfill
\small\textrm{#5}\hfill}}}
\def\leftfootline{\small{\textbf{\thepage}
\hfill ИНФОРМАТИКА И ЕЁ ПРИМЕНЕНИЯ\ \ \ том~4\ \ \ выпуск~4\ \ \ 2010}
}%
 \def\rightfootline{\small{ИНФОРМАТИКА И ЕЁ ПРИМЕНЕНИЯ\ \ \ том~4\ \ \ выпуск~4\ \ \ 2010
\hfill \textbf{\thepage}}} \vskip 1em \setcounter{figure}{0}
\setcounter{table}{0} \setcounter{equation}{0} \setcounter{section}{0}
\setcounter{subsection}{0} \setcounter{subsubsection}{0}
\setcounter{footnote}{0} \setcounter{razdel}{0}
%\end{flushleft}
\else {
 \raggedright\noindent%\large
 \textrm\textbf
 \lineskip .75em
%\baselineskip=3.2ex
#1 \par} \vskip .5em
%\begin{flushleft}
{\noindent\large\textrm\textbf #2 \par} \vskip .5em
 {\noindent\textrm #3 \par}
\addcontentsline{toc}{subsection}{{\textrm\textbf #1}\protect\newline #2}
\def\rightheadline{\underline{\noindent\hbox to \textwidth{\hfill\small\textrm{#4}
%\hfill \large\bf\thepage
}}}
\def\leftheadline{\underline{\noindent\parbox{\textwidth}{%\raggedleft\large\bf\thepage \hfill
\small\textrm{#5}\hfill}}}
\def\leftfootline{\small{\textbf{\thepage}
\hfill ИНФОРМАТИКА И ЕЁ ПРИМЕНЕНИЯ\ \ \ том~4\ \ \ выпуск~4\ \ \ 2010}
}%
 \def\rightfootline{\small{ИНФОРМАТИКА И ЕЁ ПРИМЕНЕНИЯ\ \ \ том~4\ \ \ выпуск~4\ \ \ 2010
\hfill \textbf{\thepage}}} \vskip 1em \setcounter{figure}{0}
\setcounter{table}{0} \setcounter{equation}{0} \setcounter{section}{0}
\setcounter{subsection}{0} \setcounter{subsubsection}{0}
\setcounter{footnote}{0}
%\end{flushleft}
\fi}

\def\Abst#1{
\begin{center}\small\nwt
\parbox{150mm}{%\baselineskip=2.5ex
\textbf{Аннотация:}\ \
%\hspace*{\parindent}
#1}
\end{center}}
\def\Abste#1{
\begin{center}\small\nwt
\parbox{150mm}{%\baselineskip=2.5ex
\textbf{Abstract:}\ \
%\hspace*{\parindent}
#1}
\end{center}}

\def\KW#1{
\begin{center}\small\nwt
\parbox{150mm}{%\baselineskip=2.5ex
\textbf{Ключевые слова:}\ \ #1}
\end{center}}

\def\KWE#1{
\begin{center}\small\nwt
\parbox{150mm}{%\baselineskip=2.5ex
\textbf{Keywords:}\ \ #1}
\end{center}}


\def\KWN#1{
%\begin{center}
%\small
%\parbox{150mm}\end{center}
}

\renewcommand{\thesubsection}{\thesection.\arabic{subsection}\hspace*{-5pt}}
\renewcommand{\thesubsubsection}{\thesubsection\hspace*{5pt}.\arabic{subsubsection}\hspace*{-3pt}}

\begin{document}
\Rus

\nwt
%\ptb

%\renewcommand{\contentsname}{\protect\Large\bf Содержание}

\setcounter{tocdepth}{2}

%\tableofcontents

\renewcommand{\bibname}{\protect\rmfamily Литература}
  \def\Au#1{{\it #1}}

%\newcommand{\No}{№}
  \newcommand{\tg}{\,\mathrm{tg}\,}
    \newcommand{\ctg}{\,\mathrm{ctg}\,}
  \newcommand{\arctg}{\,\mathrm{arctg}\,}
  
\def\forallb{\mathop{\forall}}
\def\existsb{\mathop{\exists}}

\setcounter{page}{1}

\newpage
\addtocounter{razdel}{1}
%\def\razd{РЕГУЛИРУЕМЫЙ ЭЛЕКТРОПРИВОД ДЛЯ ЭЛЕКТРОЭНЕРГЕТИКИ}
%\newpage
%\def\stat{zakh}
\def\tit{СРЕДСТВА ОБЕСПЕЧЕНИЯ ОТКАЗОУСТОЙЧИВОСТИ ПРИЛОЖЕНИЙ}
\def\titkol{Средства обеспечения отказоустойчивости приложений}

\def\aut{В.\,Н.~Захаров$^1$, В.\,А.~Козмидиади$^2$}
\titel{\razd}{\tit}{\aut}{\titkol}


\Abst{Рассмотрены проблемы построения отказоустойчивых серверов, возникающие в связи с недетерминированностью поведения приложений. Предложена формальная модель, описывающая поведение приложения, основными объектами которой являются ресурсы и события. Предложены алгоритмы протоколирования работы приложения на резервном узле кластера, а также восстановления и продолжения его работы при отказе основного узла. При этом для клиентов сбой остается незаметным, за исключением некоторого увеличения времени обслуживания.}

\KW{сервер приложений $\bullet$ прозрачная отказоустойчивость $\diamond$
 процесс $\diamond$ ресурс $\diamond$ событие $\diamond$ контрольная точка
$\bullet$ детерминированность}

\vskip 12pt plus 6pt minus 3pt

\begin{multicols}{2}

\section*{ВВЕДЕНИЕ}

Средства вычислительной техники стали использоваться в областях,
требующих безотказной работы систем в течение многих лет (24 часа
в сутки, 365 дней в году).

\label{st\stat}

\footnotetext{$^1$ФГУП Центральный институт авиационного моторостроения
им. П.И. Баранова, Москва, Россия}
\footnotetext{$^2$ФГУП Центральный институт авиационного моторостроения
им. П.И. Баранова, Москва, Россия}

К таким областям относятся, например, центры хранения и обработки данных  в сетях (системы резервирования билетов, биллинговые,  банковские и т.д.), массированные распределенные вычисления (GRID-вычисления) и другие.

\thispagestyle{headings}

Обычно в подобных системах применяются частные решения, ориентированные в основном на обеспечение надежного хранения данных (например, файловые серверы, использующие для хранения RAID-контроллеры) и корректного их состояния при отказах (серверы баз данных с транзакционным выполнением запросов). Однако большинство приложений не гарантируют, что не произойдет потери части данных при отказе системы. Обычно предполагается, что клиентские средства должны повторять запросы после восстановления серверов, для того, чтобы данные не были потеряны, или что можно сделать возврат по времени на некоторое время назад и повторить работу с этого места. Однако далеко не все клиентские средства и условия применения приложений допускают это.

Отказоустойчивые системы для критически важных приложений, корректно решающие проблемы восстановления после сбоев,   предлагаемые ведущими производителями, как правило, дороги. Кроме того, они включают специфические серверные и клиентские приложения, не совместимые со стандартными приложениями, не обеспечивающими отказоустойчивость. Примером такого подхода к решению проблемы отказоустойчивости  хранения данных являются системы NetApp FAS компании Network Appliance, работающие на базе специализированной операционной системы Data ONTAP [1].

Построение отказоустойчивых систем, использующих серверы со стандартными приложениями, в свете вышесказанного, является актуальной проблемой, вызывающей значительный интерес. Рассмотрение методов достижения прозрачной отказоустойчивости таких систем и является предметом статьи.
\begin{figure*} %fig1
\vspace*{1pt}
\begin{center}
\mbox{%
\epsfxsize=1.6in
\epsfxsize=100mm
\epsfbox{BbR-1.eps}
}
\end{center}
\vspace*{-9pt}
\Caption{Базовый вариант трубы с разными выходными устройствами
(цилиндрическое, расширяющееся и сужающееся сопло)
\label{f1bab}}
\vspace*{-3pt}
\end{figure*}


\section{ОСНОВНЫЕ ПОНЯТИЯ И ПОДХОДЫ}

Под сервером в данной работе понимается вычислительный центр
(отдельный компьютер или кластер) в сети, предоставляющий клиентам
(пользователям, клиентским компьютерам) определенные услуги, разделяя
между ними свои ресурсы. Подобные серверы названы серверами приложений.
Широко распространенным примером сервера такого типа является файловый сервер, обеспечивающий удаленный коллективный доступ к файловой системе. Часто используются вычислительные серверы, предоставляющие клиентам возможность выполнять на них свои программы (например, в центрах коллективного пользования).


Обычно приложение представляет собой программу или группу программ, работающих в операционной среде, создаваемой операционной системой (в другой терминологии - один или несколько взаимодействующих процессов или потоков (threads)), которые реализуют функциональность сервера. Для построения отказоустойчивых серверов приложений широко используется кластерная технология. Следуя [2], кластером, названа разновидность параллельной или распределенной системы, которая:
\begin{itemize}
\item состоит из нескольких компьютеров (узлов кластера), связанных как минимум необходимыми коммуникационными каналами;
\item используется как единый, унифицированный компьютерный ресурс.
\end{itemize}

Прозрачная отказоустойчивость (Transparent Fault Tolerance, TFT) сервера приложений - это такое его поведение при возникновении аппаратных или программных отказов либо отказов в сети, при котором:
\begin{itemize}
\item отказ не вызывает потери или искажения данных, находящихся в базе данных сервера;
\item сервер продолжает нормально функционировать, несмотря на имевшие место отказы.
\end{itemize}

Клиенты сервера "не замечают" произошедших отказов. Единственным\footnote{допустимым
отклонением сервера от нормального поведения с точки зрения клиента является
некоторое увеличение времени обслуживания} (на несколько секунд или десятков секунд).

Обычно приложения, работающие на серверах приложений, не ориентированы на прозрачную отказоустойчивость. Они "заботятся" лишь о собственной целостности (например, состояния файловой системы или базы данных). Восстановление работоспособности сервера приводит к разрыву соединений с клиентами и потере их запросов. Это замечают клиенты - запросы следует повторять, на что клиентские приложения далеко не всегда рассчитаны. В данной работе предполагается, что приложения (прикладные программные средства), выполняемые на сервере, являются стандартными, то есть не имеют специальных средств, обеспечивающих отказоустойчивость.
\begin{figure*}[b] %fig1
\vspace*{1pt}
\begin{center}
\mbox{%
\epsfxsize=1.6in
\epsfxsize=100mm
\epsfbox{BbR-1.eps}
}
\end{center}
\vspace*{-9pt}
\Caption{Базовый вариант трубы с разными выходными устройствами
(цилиндрическое, расширяющееся и сужающееся сопло)
\label{f1bab}}
\vspace*{-3pt}
\end{figure*}

Серьезные исследования в области обеспечения отказоустойчивости серверов были развернуты после создания вычислительных серверов, предназначенных для решения задач, требующих больших вычислительных ресурсов. Решение этих задач выполняется на суперкомпьютерах, обеспечивающих массово-параллельные вычисления и представляющих собой кластеры из сотен и тысяч узлов (процессоров). Однако даже на этих "монстрах" решение может требовать десятков или сотен часов, и одиночный сбой, если не предприняты специальные меры, может привести к необходимости начинать работу сначала. Обычно решение вычислительной задачи в таких случаях осуществляется в модели относительно редко взаимодействующих между собой процессов, выполняемых на разных узлах кластера. Эти взаимодействия нужны для координации работы процессов, в частности, для обмена данными и промежуточными результатами. Взаимодействия опираются на специальный протокол, называемый MPI (Message-Passing Interface) и представляющий собой стандарт "de facto" [3].

Для преодоления последствий сбоя достаточно давно была разработана и широко применяется технология, опирающаяся на механизм контрольных точек (checkpoints) [4-6]. По этой технологии система должна иметь стабильную память, которая не меняется при отказах. Соответствующие программные средства периодически сохраняют информацию о состоянии процессов приложения в стабильной памяти. Все процессы также имеют доступ к устройству стабильной памяти.  В случае отказа или сбоя, записанная в стабильную память информация используется для повторения вычисления с момента, когда была записана эта информация, то есть выполняется откат назад по времени. Данные, сохранение которых позволяет выполнить откат, называются контрольной точкой. В качестве устройства стабильной памяти может использоваться дисковый том, энергонезависимая оперативная память, память другого узла или узлов кластера. В последнем случае узел, которому требуется сохранить информацию, пересылает ее через быстрый канал связи на другой узел. Стабильная память после отказа одного из узлов должна быть доступной узлу, на котором делается повтор.

Однако решение, опирающееся только на контрольные точки, не является прозрачным, поскольку не скрывает от клиентов факт отказа системы и требует от них выполнения определенных действий. Так как при работе процессы обмениваются сообщениями, возможны два варианта решения проблемы. Первый - все процессы выполняют записи контрольных точек одновременно, что затруднительно. Второй вариант, при несоблюдении синхронности, - возврат в каждом процессе к такому скоординированному набору контрольных точек, при котором невозможна противоречивая ситуация. Такая ситуация возникает, когда один процесс вернулся к контрольной точке, после которой он должен получить сообщение от другого процесса, а этот другой процесс вернулся к точке, которая следует за выдачей этого сообщения. Однако при повторе ожидаемое первым процессом сообщение не поступит. В этом случае  возможен эффект домино, в результате процессы оказываются отброшены как угодно далеко назад.

В этом состоит первая проблема, которую необходимо преодолеть.

Если нужно, чтобы последствия отказа узла не были видны клиенту,  это означает:
\begin{itemize}
\item клиент не должен терять и потом восстанавливать соединения с сервером;
\item клиент не должен повторять свои запросы;
\item клиент не должен повторно получать сообщения, которые он уже получил.
\end{itemize}

Вторая проблема, которую надо решать, связана с недетерминированностью поведения сервера приложений. Приведем пример.  Пусть имеется система продажи билетов на самолеты. Два клиента одновременно обратились к системе с запросом билета на один и тот же рейс. Клиентам безразлично, какие места им зарезервирует система. Система выполняет запросы клиентов параллельно, поэтому в какой-то момент между процессами, обрабатывающими эти запросы, может возникнуть конкуренция за ресурс - в данном случае, скажем, рейс. Один из процессов захватывает ресурс первым, резервирует место и освобождает ресурс. Потом второй процесс проделывает то же самое.

Порядок, в котором в этом примере процессы захватили ресурс, зависит от многих факторов и, в конечном счете, случаен. Однако  это не мешает правильному функционированию системы, поскольку клиентам важно одно - получить билеты, причем на разные места. Однако отсутствие детерминизма в поведении приложения приводит к тому, что при повторном выполнении могут быть получены другие результаты: например, клиенту уже сообщено, что ему зарезервировано место №5, а при повторе может получиться, что зарезервировано место №6. Система должна устранить это несоответствие и сделать его невидимым для клиента.
\begin{figure*} %fig1
\vspace*{1pt}
\begin{center}
\mbox{%
\epsfxsize=1.6in
\epsfxsize=100mm
\epsfbox{BbR-1.eps}
}
\end{center}
\vspace*{-9pt}
\Caption{Базовый вариант трубы с разными выходными устройствами
(цилиндрическое, расширяющееся и сужающееся сопло)
\label{f1bab}}
\vspace*{-3pt}
\end{figure*}

Недетерминированность поведения системы это следствие, по крайней мере, двух обстоятельств. Во-первых, это присущая системам с разделением времени неопределенность в порядке выполнения процессов. Во-вторых, это конкуренция процессов за общие ресурсы. Перечислим некоторые причины недетерминированного поведения приложений:
\begin{itemize}
\item синхронизация процессов с помощью семафоров или атомарных операций над операндами в общей памяти процессов;
\item зависимость от порядка получения клиентских запросов;
\item время, затраченное процессом на обработку полученного запроса;
\item генераторы случайных чисел;
\item системное управление процессами и потоками;
\item локальные таймеры;
\item доступ к реальному времени.
\end{itemize}

По различным  причинам время, которое тратится на выполнение вычислительной задачи с одними и теми же исходными данными, не является константой, то есть повторное выполнение может дать другое время. Процессы используют общие ресурсы, обращение к которым требует организации очередности выполнения (сериализации) - первым пришел, первым захватил. И, наконец,  результат работы процесса может зависеть от состояния ресурса, а это состояние может изменить другой процесс, ранее захвативший ресурс. Все это создает значительные трудности при попытках воспроизведения поведения процессов с сохраненной контрольной точки.

Прозрачная отказоустойчивость серверов приложений обычно осуществляется переносом приложения на другой узел кластера, идентичный первому по конфигурации аппаратных средств и операционной среды. Это делается методом, называемым snapshot/restore. На основном узле (оригинале)  периодически фиксируется состояние приложения на этом узле кластера (так называемый снимок или snapshot). После отказа оригинала на резервном узле (копии) делается восстановление (restore), то есть восстанавливается последнее зафиксированное состояние приложения. Операционная среда при этом приводится в состояние, которое соответствует моменту изготовления снимка. После этого узел-копия продолжает работу с зафиксированного места. Сравнение метода  snapshot/restore с другими подходами приведено в [7].

Ниже рассматриваются информационные  технологии, позволяющие решить ряд принципиальных вопросов, связанных с реализацией прозрачной отказоустойчивости серверов приложений. Ими являются:
\begin{itemize}
\item виртуализация операционной среды, в которой работает серверное приложение;
\item отказоустойчивая реализация протокола TCP;
\item создание контрольных точек состояния приложения и файловой системы, которые делаются внешним по отношению к приложению образом;
\item восстановление серверного приложения на основании контрольной точки.
\end{itemize}
\begin{figure*} %fig1
\vspace*{1pt}
\begin{center}
\mbox{%
\epsfxsize=1.6in
\epsfxsize=100mm
\epsfbox{BbR-1.eps}
}
\end{center}
\vspace*{-9pt}
\Caption{Базовый вариант трубы с разными выходными устройствами
(цилиндрическое, расширяющееся и сужающееся сопло)
\label{f1bab}}
\vspace*{-3pt}
\end{figure*}

\section{МОДЕЛЬ ОПИСАНИЯ ПОВЕДЕНИЯ ПРИЛОЖЕНИЯ}

Предлагаемый подход опирается на построение модели вычислений, связанной с использованием понятия времени в многопроцессных приложениях. Впервые подобные проблемы были изучены в классической работе Л. Лампорта [8].

Многопроцессными приложения называются потому, что в них параллельно работают несколько процессов. Процесс ведет себя детерминированно, пока в предписанном кодом порядке выполняет процессорные инструкции. Конечно, его работа может быть прервана практически в любой момент и процессор передан другому процессу или ядру. Поэтому абсолютное время, которое затрачивает процесс на выполнение определенной работы, не  константа, а случайная  величина. То же  относится к относительному времени, то есть времени, которое процесс занимал процессор,  поскольку одни и те же обращения к операционной среде могут вызвать работы разной длительности, а значит потребовать разное время на свое выполнение.

Кэшированность инструкций и данных, а также длина хэш-списков влияют на действительное время пребывания в операционной среде. Утрачивает смысл понятие одновременность действий, поскольку  нельзя установить, выполнили ли два разных процесса какие-либо действия одновременно или одно из них предшествовало другому. Таким образом, с процессом можно связать только его локальное время, которое линейно упорядочивает события,  происходившие в этом процессе.  Глобальное время, линейно упорядочивающее действия во всех процессах, отсутствует. Расстояние (в этом качестве используется время) между действиями оказывается случайной величиной.

Эти соображения важны, поскольку процессы в интересующих нас приложениях взаимодействуют и используют общие ресурсы. Для взаимодействия они используют средства синхронизации, предоставляемые операционной средой - например, наборы семафоров SVR4 (System V Release 4), POSIX-семафоры, бинарные семафоры и другие примитивы взаимного исключения (POSIX- mutual exclusion locks) и т.д. Подобные средства операционной среды, которые позволяют процессам синхронизировать свою деятельность друг с другом или сериализовать обращения к совместно используемым объектам,  будут ниже  называться ресурсами.

С каждым ресурсом связано свое локальное время, линейно упорядочивающее события в жизни ресурса. Например, в случае двоичных семафоров это создание семафора, а также его захват и освобождение процессом. Заметим, что событие - это не намерение процесса (например, захватить бинарный семафор), а сам факт захвата семафора процессом (т.е. успешное выполнение намерения). От изъявления намерения до его осуществления может многое произойти. Например, семафор, который хочет захватить рассматриваемый процесс, принадлежал другому процессу, потом тот процесс его освободил, но семафор был сначала передан операционной средой третьему процессу, который также на него претендовал, и т.д. Поведение рассматриваемого процесса в это время нас не интересует - он ресурсом еще не овладел, а только его захват определяет его дальнейшее поведение. По причинам,  изложенным выше, расстояние между двумя событиями - случайная величина. Однако, события замечательны тем, что они одновременно присутствуют и в локальном времени процесса, и в локальном времени ресурса. Поэтому все, что произошло в истории процесса или/и ресурса до этого события, предшествует ему. Далее  будет считаться, что истории и ресурсов и процессов состоят только из событий, причем между двумя последовательными событиями в жизни процесса последний ведет себя детерминированно.

Это означает, что на  поведении процесса сказывается только его предыдущая история, то есть состояние ресурсов, с которыми он взаимодействовал. Это свойство процессов ниже будет называться локальной детерминированностью. Этим свойством не обладают ресурсы, поскольку - следующее событие в истории ресурса не определяется однозначно по его предыдущей истории. Утверждение, касающееся детерминированного поведения процессов, неявно опирается на предположение,  что учтены все ресурсы, которые могут привести к  недетерминированности процессов.

Таким образом, описанное нами очень неформально время в многопроцессном комплексе представляет собой отношение частичного порядка, введенное на множестве событий. Зная полное состояние комплекса в некоторый момент времени,  нельзя однозначно определить, какое событие в истории ресурса наступит следующим. Можно говорить только о вероятности наступления того или иного события. Недетерминированность поведения есть следствие двух обстоятельств. Во-первых, это неопределенность времени, которое тратит процесс на переход от одного события к другому. Во-вторых, конкуренция процессов за общие ресурсы.

Выполнение приложения, на множестве событий которого введена частичная упорядоченность, можно описать направленным ациклическим графом выполнения. Вершинами этого графа являются события, с каждым  из которых связаны две входящие в него дуги. Одна дуга начинается в событии, которое непосредственно предшествует данному событию в истории процесса, другая - в истории ресурса.

Построение средств обеспечения прозрачной отказоустойчивости приложений опирается на следующее утверждение: для восстановления работы приложения после отказа достаточно располагать:
\begin{itemize}
\item контрольной точкой, которая отражает на некоторый момент времени состояния процессов и других ресурсов, образующих приложение;
\item графом выполнения приложения, который описывает работу приложения, начинающуюся с контрольной точки и заканчивающуюся отказом. Данные, которые нужны для построения графа выполнения, далее называются протоколом.
\end{itemize}
\begin{figure*} %fig1
\vspace*{1pt}
\begin{center}
\mbox{%
\epsfxsize=1.6in
\epsfxsize=100mm
\epsfbox{BbR-1.eps}
}
\end{center}
\vspace*{-9pt}
\Caption{Базовый вариант трубы с разными выходными устройствами
(цилиндрическое, расширяющееся и сужающееся сопло)
\label{f1bab}}
\vspace*{-3pt}
\end{figure*}

Вся эта информация должна находиться в стабильной памяти, не разрушающейся при отказе.

Ниже неформально описан алгоритм восстановления работы приложения после отказа, который опирается на наличие контрольной точки и графа выполнения. Будем считать, что в распоряжении имеются средства, позволяющие остановить процесс в тот момент, когда он намерен совершить некоторую операцию над ресурсом. Заметим, что событие в графе выполнения соответствует не изъявлению намерения, а его удовлетворению, то есть завершению выполнения операции.

Предварительно сделаем следующее:
\begin{itemize}
\item используя контрольную точку, приведем приложение в состояние, соответствующее этой контрольной точке;
\item в графе выполнения пометим все вершины (события) как "не наступившие". У некоторых вершин графа отсутствуют им непосредственно предшествующие; соответствующие события наступили сразу же после создания контрольной точки. Для каждой такой вершины включим в граф дополнительную вершину, ей предшествующую в истории процесса, и отметим эту дополнительную вершину как "наступившую";
\item разрешим процессам приложения выполняться.
\end{itemize}

Пусть некоторый процесс проявляет намерение выполнить операцию над каким-либо ресурсом. Отыщем для этого процесса в его истории последнее наступившее событие. Следующее в его истории событие - это то, которое соответствует требуемой операции. Посмотрим, наступило ли событие в истории ресурса, которое ему предшествует. Если нет, переведем процесс в состояния ожидания, отметив в предшествующем событии, что данный процесс ожидает его наступления. Если да, разрешим процессу выполняться, то есть выполнить операцию над ресурсом.

Пусть некоторый процесс объявляет, что он выполнил операцию над каким-либо ресурсом (это соответствует моменту протоколирования при оригинальном выполнении). Отыщем для этого процесса в его истории последнее наступившее событие и перейдем к следующему событию в его истории. Это опять то событие, которое мы рассматриваем. Отметим его как "наступившее". Если наступления этого события ожидал какой-нибудь процесс, выведем этот процесс из состояния ожидания. Наконец, разрешим процессу, выполнившему операцию, продолжаться дальше.

Когда выясняется, что наступили все события графа выполнения, повторное выполнение считается законченным.

Естественным следствием из сказанного является следующее утверждение: для того, чтобы размер протокола не рос неограниченно, нужно периодически создавать контрольные точки, очищая при этом протокол.

\section{ФОРМАЛЬНОЕ ОПИСАНИЕ МОДЕЛИ ПОВЕДЕНИЯ МНОГОПРОЦЕССНОГО ПРИЛОЖЕНИЯ}
\begin{figure*} %fig1
\vspace*{1pt}
\begin{center}
\mbox{%
\epsfxsize=1.6in
\epsfxsize=100mm
\epsfbox{BbR-1.eps}
}
\end{center}
\vspace*{-9pt}
\Caption{Базовый вариант трубы с разными выходными устройствами
(цилиндрическое, расширяющееся и сужающееся сопло)
\label{f1bab}}
\vspace*{-3pt}
\end{figure*}

Опишем формально поведение приложения, неформальное описание которого было приведено выше. Рассматриваются два типа объектов:
\begin{itemize}
\item ресурсы (r), например, наборы семафоров (POSIX- или SVR4-семафоры), бинарные семафоры (POSIX-mutex's), таймер реального времени, сокеты (sockets), то есть двусторонние виртуальные соединения с внешним миром;
\item процессы (p), например, процессы или потоки (threads) пользователя.
\end{itemize}

\end{multicols}

\label{end\stat}

%\def\stat{batr}

\def\tit{НОВЫЙ МЕТОД ВЕРОЯТНОСТНО-СТАТИСТИЧЕСКОГО\newline
АНАЛИЗА ИНФОРМАЦИОННЫХ ПОТОКОВ
В~ТЕЛЕКОММУНИКАЦИОННЫХ СЕТЯХ$^*$}
\def\titkol{Новый метод вероятностно-статистического
анализа информационных потоков
в~телекоммуникационных сетях}
\def\autkol{Д.\,А.~Батракова, В.\,Ю.~Королев, С.\,Я.~Шоргин}
\def\aut{Д.\,А.~Батракова$^1$, В.\,Ю.~Королев$^2$, С.\,Я.~Шоргин$^3$}

\titel{\tit}{\aut}{\autkol}{\titkol}

{\renewcommand{\thefootnote}{\fnsymbol{footnote}}\footnotetext[1]{Работа 
выполнена при поддержке РФФИ, проекты №№\,04-01-00671, 05-07-90103.} 
\renewcommand{\thefootnote}{\arabic{footnote}}}
 \footnotetext[1]{ИПИ РАН, 
daria.batrakova@gmail.com} \footnotetext[2]{Факультет вычислительной математики 
и кибернетики МГУ им.~М.\,В.~Ломоносова, ИПИ РАН, vkorolev@comtv.ru} 
\footnotetext[3]{ИПИ РАН, sshorgin@ipiran.ru}



\Abst{В данной работе предлагается метод исследования стохастической структуры
хаотических информационных потоков в сложных телекоммуникационных
сетях. Предлагаемый метод основан на стохастической модели
телекоммуникационной сети, в рамках которой она представляется в виде
суперпозиции некоторых простых последовательно-параллельных структур.
Эта модель естественно порождает смеси гамма-распределений для времени
выполнения (обработки) запроса сетью. Параметры получаемой смеси
гамма-распределений характеризуют стохастическую структуру
информационных потоков в сети. Для решения задачи статистического
оценивания параметров смесей экспоненциальных и гамма-распределений
(задачи разделения смесей) используется ЕМ-алгоритм. Чтобы проследить
изменение стохастической структуры информационных потоков во времени,
ЕМ-алгоритм применяется в режиме скользящего окна. Описывается
программный инструментарий для применения полученного решения к
реальным статистическим данным. Приводится интерпретация результатов.}

\KW{телекоммуникационные сети; информационные потоки;
разделение смесей  распределений;
метод скользящего окна;  программа для разделения смесей}

\vskip 24pt plus 9pt minus 6pt

\thispagestyle{headings}

\begin{multicols}{2}


\label{st\stat}

\section{Введение}

Развитие телекоммуникационных сетей, их усложнение поставило перед
инженерами важную прикладную задачу исследования характеристик
информационных потоков, возникающих в этих сетях. Здесь под
информационным потоком мы будем понимать упорядоченное движение
любого вида информации по сети.

Если на заре эры телекоммуникаций, в эпоху первых телефонных линий и
телеграфа эта проблема не была столь насущной, то со временем, при
постепенном охвате мирового пространства сетями возникла необходимость в
построении и исследовании адекватных моделей сетей и процессов,
происходящих в них.

\thispagestyle{headings}


Современные сети~--- это \textit{конвергентные} сети, т.е.\ совокупность крайне
разнородных как по топологии, так и по физической архитектуре сетей, которые
предлагают конечному пользователю самые разнообразные сервисы. Это~--- огромное
виртуальное и физическое пространство, состоящее из миллионов процессоров,
операционных платформ, линий передачи данных и стыковочных узлов.
%
Существует множество классификаций телекоммуникационных сетей по различным
признакам:
\begin{itemize}
\item масштабу (локальные сети~--- LAN, масштаба города~---
MAN, широкого масштаба~--- WAN);
\item топологии, или логической организации (<<звезда>>,
<<кольцо>>, <<шина>>);
\item физической организации (оптические сети, радио);
\item предлагаемым услугам (сотовые сети, для доступа в
Интернет);
\item назначению (военные, гражданские) и~др.
\end{itemize}


Конвергентная сеть входит во все эти классы, причем, как правило,
обладает всеми этими признаками. Поэтому построение модели для ее анализа
является и очень важной, и очень сложной задачей.

Существуют достаточно многочисленные математические методы, ориентированные на
моделирование и анализ телекоммуникационных сетей. В~большинстве своем они
основываются на теории массового обслуживания, то есть разделе теории
вероятностей, посвященном описанию функционирования сложных систем обслуживания
(в том чис\-ле телекоммуникационных сетей и систем) с помощью стохастических
процессов особого вида и анализу таких процессов. Указанная теория является
весьма развитой и широко применяется на практике. Тем не менее, ее применимость
ограничена~--- во-первых, все возрастающей сложностью структур и дисциплин
обслуживания в рас\-смат\-ри\-ва\-емых реальных сетях. Эта сложность во многих
случаях принципиально не может найти адекватного отображения в моделях
массового обслуживания, даже несмотря на постоянно растущую сложность самих
этих моделей. В результате даже модели, допускающие точный математический
анализ, дают возможность расчета всего лишь приближенных значений характеристик
реальных сетей, ибо предположения, принимаемые при построении моделей, во
многих случаях не соответствуют практике. Во-вторых, для описания
телекоммуникационной сети в виде сети массового обслуживания исследователь
должен располагать детальным описанием структуры сети, что далеко не всегда
имеет мес\-то на практике. В-третьих, разработано крайне мало моделей массового
обслуживания, в которых используется в качестве входной информация о
наблюдаемых (статистических) показателях функционирования сети; в то же время,
такая информация очень часто доступна исследователю, и ее использование при
анализе сети весьма целесообразно.

В данной работе предлагается в определенной степени альтернативный подход к
моделированию сложных телекоммуникационных сетей. Строится и исследуется
вероятностная модель сложной телекоммуникационной сети как суперпозиции
достаточно простых структур. При этом практически никакая априорная информация
о структуре исследуемой сети не используется~--- наоборот, в результате
исследования модели исследователь получает приближенное представление об этой
структуре. Характеристики типовых простых структур, составляющих в совокупности
модель сети, и сети в целом при этом подходе описываются
гам\-ма-рас\-пре\-де\-ле\-ни\-я\-ми. Ставится задача оценки параметров модели
на основе статистических данных о функционировании сети, а также предлагается
математическое решение этой задачи. В статье описан также созданный на основе
разработанных математических методов программный инструментарий и приведены
результаты расчетов для реального трафика. {\looseness=-1

}

\section{Математическая модель и~постановка задачи}

\subsection{Логическая модель сети}
 %1.1

Рассмотрим абстрактную \textit{конвергентную телекоммуникационную
сеть}. Это может быть как крупномасштабная транспортная сеть (WAN), сеть
отдельного оператора масштаба города (MAN) с различными сервисами, так и
локальная сеть (LAN).

Любой из этих случаев можно описать как ($p,\,q$)-\textit{сеть}.

\medskip
\textbf{Определение 1.} В теории графов и сетей под ($p,\,q)$-сетью понимается
набор вида $S =$\linebreak $=(G,\,V^\prime ,\,V^{\prime\prime})$, где $G$~---
граф, а $V^\prime$ и $V^{\prime\prime}$~--- выборки из множества $V(G)$ (вершин
графа) длины~$p$ и $q$ соответственно. При этом выборка $V^\prime$
($V^{\prime\prime}$) считается \textit{входной} (\textit{выходной}) выборкой, а
ее $i$-я вершина называется $i$-\textit{м} \textit{входным} (\textit{выходным})
\textit{полюсом} или, иначе, $i$-\textit{м} \textit{входом} (\textit{выходом})
сети~$S$. Вершины, не участвующие во входной и выходной выборках сети,
считаются ее внутренними вершинами~\cite{1bat}.

Сеть $S$ (рис.~\ref{f1bat}) имеет $p$ точек входа~--- точек соединения
с внешней средой (это могут быть точки стыковки разнородных сетей, сетей
различных операторов, физические подключения к интерфейсам
маршрутизаторов и~т.п.). Под \textit{внешней средой} будем понимать другие
сети, которые передают данные в сеть~$S$. Отдельные <<единицы>> данных
(кадры, сообщения, датаграммы, пакеты) поступают на входы сети~$S$,
обрабатываются и подаются на каждый из $q$ выходов, которые могут быть
соединены как с конечными пользователями, так и с другими сетями.
\begin{figure*} %fig1
\vspace*{1pt}
\begin{center}
\mbox{%
\epsfxsize=139.7mm \epsfbox{bat-1.eps}
%\epsfxsize=139.698mm
%\epsfbox{bek-3.eps}
}
\end{center}
\vspace*{-9pt} \Caption{Абстрактная телекоммуникационная сеть \label{f1bat}}
\end{figure*}

Следует отметить, что структура сложных телекоммуникационных сетей обладает
свойством некоторого самоподобия, т.е.\ на каком бы уровне сетевой архитектуры
мы ни рассматривали поведение информационных потоков, мы можем выделить
некоторые базовые структуры, субпотоки, суперпозицией которых мы можем получить
модель конкретной сети, какой бы уровень <<детализации>> сегментов сети мы ни
взяли. Так, например, физические подключения к интерфейсам оптического
кросс-коннекта в этом смысле подобны <<виртуальным>> подключениям к портам TCP
на сервере приложений.

Итак, независимо от уровня сетевой архитектуры мы можем
рассматривать некоторую величину, характеризующую количество каких-либо
ресурсов сети~$S$, занимаемых в процессе передачи и обработки данных.  Это
могут быть ресурсы, относящиеся как к <<объему>> (памяти сетевого
устройства, количеству занятых линий, размеру пакета), так и ко <<времени>>
(времени обслуживания заявки, времени простоя). Эта величина случайна, т.к.\
мы не можем абсолютно точно сказать для сложной телекоммуникационной
сети, какое сообщение на какой из входов поступит и какого размера оно будет.
Таким образом, случайный характер данной величины определяется
случайностью поведения внешней среды.

Пусть $R$~--- это описанная выше случайная величина, $R>0$. Далее, не
ограничивая общности, будем подразумевать под ней время, необходимое для
какой-либо операции сети (обработки <<единицы>> данных), предполагая, что
время обработки прямо зависит от объема сообщения.

\subsection{Вероятностная модель сети} %1.2.

Даже не зная реальной топологии сети, мы можем описать
функционирование некоторых ее участков как процесс выполнения операций
(задач сети) в последовательном  порядке (например, если доступен только
один канал данных) или как процесс одновременного выполнения субопераций
(когда доступно более одного пути выполнения). Это значит, что мы можем
представить функционирование сложной телекоммуникационной сети как
\textit{суперпозицию} таких <<последовательных>> и <<параллельных>>
блоков.

Для построения вероятностной модели распределения~$R$ используется
комбинация асимптотического подхода, основанного на предельных теоремах
теории вероятностей, и принципа максимальной неопределенности (энтропии).

Рассмотрим следующую модель. Предположим, что мы можем разделить
сеть~$S$ на несколько сегментов $S_i$. Пусть $T$~--- случайная величина,
время выполнения операции в отдельно взятом блоке $S_i$ (сегменте сети).

Если операции выполняются \textit{параллельно}, то время, необходимое
для их выполнения~--- это максимальное время, затрачиваемое на какую-либо
субоперацию:
$$
T = \underset{i}{\max}\, T_i\,,
$$
где $T_i$~--- случайные величины для со\-от\-вет\-ст\-ву\-ющих субопераций.
Модель такого выполнения пред\-став\-ле\-на на рис.~\ref{f2bat}.

\begin{figure*} %fig2
\vspace*{1pt}
\begin{center}
\mbox{%
\epsfxsize=117.271mm
\epsfbox{bat-2.eps}
}
\end{center}
\vspace*{-9pt}
\Caption{Параллельное выполнение
\label{f2bat}}
\end{figure*}

Известно, что предельное распределение экстремальных значений для
выборок ~--- это экспоненциальное распределение с плотностью~\cite{2bat}
$$
f(x) =
\begin{cases}
\lambda e^{-\lambda x}\,, & x>0\,,\\
0\,, & x\leq 0\,,
\end{cases}
$$
где $\lambda >0$~--- параметр масштаба.

 Учитывая также энтропийный подход, естественно будет считать
распределение $T$ экспоненциальным, т.к.\ экспоненциальное распределение
обладает наибольшей энтропией среди всех распределений с $x>0$.

Если же операции сети выполняются \textit{последовательно}, то величина
$T$~--- это сумма времен $T_i$, необходимых для выполнения каждой
субоперации:
$$
T = \sum\limits_i T_i\,,
$$
где $T_i$~--- случайные величины для со\-от\-вет\-ст\-ву\-ющих субопераций.
%
Такая модель представлена на рис.~\ref{f3bat}.

\begin{figure*} %fig3
\vspace*{1pt}
\begin{center}
\mbox{%
\epsfxsize=139.592mm
\epsfbox{bat-3.eps}
}
\end{center}
\vspace*{-9pt}
\Caption{Последовательное  выполнение
\label{f3bat}}
\end{figure*}

Это значит, что распределение общей длительности $T$ выполнения
блока~--- это свертка распределений <<элементарных>> времен $T_i$
(экспоненциально распределенных).

Известно, что результатом свертки экспоненциальных распределений
является гамма-распределение, определяемое плотностью
$$
\g_{\lambda , \alpha} (x) =
\begin{cases}
\fr{\lambda_0^{\alpha_0}}{\Gamma (\alpha_0 )}\,x^{\alpha_0-1}
e^{\lambda_0 x}\,, & x>0\,,\\
0,\, & x\leq 0\,,
\end{cases}
$$
где $\alpha >0$~--- параметр формы,  $\lambda >0$  параметр масштаба, а
$\Gamma (z)$~--- гамма-функция Эйлера:
$$
\Gamma (z) = \int\limits_0^\infty x^{z-1} e^{-x}\,dx\,.
$$

\begin{figure*} %fig4
\vspace*{1pt}
\begin{center}
\mbox{%
\epsfxsize=120.831mm
\epsfbox{bat-4.eps}
}
\end{center}
\vspace*{-9pt}
\Caption{Модель пути  обработки сообщения сетью~$S$
\label{f4bat}}
\end{figure*}

Известно~\cite{2bat}, что класс гамма-распределений замкнут над операцией
свертки, поэтому ре\-зуль\-ти\-ру\-ющее распределение случайной величины~$R$
будет также гамма-распределением
$$
\g_{\lambda , \alpha} (x) =
\begin{cases}
\fr{\lambda^{\alpha}}{\Gamma (\alpha )}\,x^{\alpha -1} e^{-\lambda x}\,, &
x>0\,,\\
0,\, & x\leq 0\,.
\end{cases}
$$

В силу случайного характера ввода данных в сеть~$S$ из внешней среды маршрут
передачи данных становится случайным, что представлено на рис.~\ref{f4bat}. Это
означает, что параметры ре\-зуль\-ти\-ру\-юще\-го распределения~$R$ тоже
случайны. Отсюда имеем следующую модель \textit{смеси
гам\-ма-рас\-пре\-де\-ле\-ний}, определяемой плотностью

\begin{equation} %1
p(x) = \iint \g_{\lambda , \alpha}(x)\,dH (\lambda ,\,\alpha )\,,
\end{equation}
где $H(\lambda , \alpha )$~--- смешивающая функция, функция распределения
входных параметров.

Поясним понятие \textit{смеси распределений}.

\medskip
\textbf{Определение~2.} Пусть имеется двух\-па\-ра\-мет\-ри\-че\-ское
семейство $n$-мерных плотностей  распределения
\begin{equation}
F = \{ f_\omega (x;\, \theta (\omega ))\}\,,
\end{equation}
где одномерный (целочисленный или непрерывный) параметр $\omega$ в
качестве нижнего индекса функции $f$ определяет специфику общего вида
каж\-до\-го компонента~--- распределения смеси, а в качестве аргумента при
многомерном, вообще говоря, параметре $\theta$ определяет зависимость
значений хотя бы части компонентов этого параметра от того, в каком именно
составляющем распределении $f_\omega$ он присутствует. Кроме того, пусть
$P = \{P(\omega )\}$~--- \textit{семейство смешивающих функций}
распределения.

Функция плотности распределения
\begin{equation}
f(x) = \int f_\omega (x;\,\theta(\omega ))\,dP (\omega )
\end{equation}
называется $P$-\textit{смесью} (или просто \textit{смесью})
\textit{распределений} семейства~$F$,  интеграл в~(3) понимается в смысле
Лебега--Стильтьеса~\cite{3bat}.

\medskip
\textbf{Определение 3.} Семейство смесей~(3) называется
\textit{идентифицируемым} (\textit{различимым}), если из равенства
$$
\int f_\omega (x;\,\theta(\omega ))\,dP (\omega ) =\int f_\omega
(x,\,\theta(\omega )) dP^*(\omega )
$$
следует, что $P(\omega ) \equiv P^*(\omega )$ для всех $P \in P(\omega
)$~\cite{3bat}.

\subsection{Постановка задачи} %1.3.

Перед нами встает задача \textit{разделения} такой смеси. Вообще говоря,
задача разделения смесей распределений со смешивающими функциями
общего вида является \textit{некорректно поставленной}, т.к.\ она допускает
существование нескольких решений. Поэтому будем искать решение в классе
\textit{конечных идентифицируемых смесей распределений}, где смешивающая
функция дискретна.

Для этого сузим данное выше определение и будем рассматривать в дальнейшем лишь 
случай конечного числа $k$ возможных значений па\-ра\-мет\-ра~$\omega$, что 
соответствует конечному числу скачков смешивающих функций $P(\omega )$.  
Величины этих скачков как раз и будут играть роль \textit{удельных весов} 
(\textit{априорных вероятностей}) $p_j$ компонентов смеси. Более того, в нашем 
случае мы постулируем также однотипность компонентов распределений $f_j$, т.е.\ 
принадлежность всех $f_j$ к одному общему па\-ра\-мет\-ри\-че\-ско\-му 
семейству $\{ f(X;\,\theta )\}$, где $\theta$~--- многомерный, вообще говоря, 
параметр. Так что~(3) в этом случае может быть записано в виде
\begin{equation} %4
p(x) = \sum\limits^k_{j=1} p_j f_j (x;\,\theta_j )\,.
\end{equation}

Переформулируем понятие идентифицируемости (различимости) смесей
специально применительно к такому виду смесей.

\medskip
\textbf{Определение 4.} \textit{Конечная смесь}~(3) называется
\textit{идентифицируемой} (\textit{различимой}), если из равенства
$$
\sum\limits_{j=1}^k p_j f_j (x;\,\theta_j ) = \sum\limits_{l=1}^{k^*} p_l^* f_l
(x;\,\theta_l^* )
$$
следует, что $k=k^*$ и для любого $j$ ($1\leq j \leq k$) найдется такое $l$ 
($1\leq l \leq k^*$), что $p_j = p_l^*$ и $f_j (x;\,\theta_j ) = f_l 
(x;\,\theta_l^* )$~\cite{3bat}.

Решить эту задачу в выборочном варианте~--- значит по выборке
классифицируемых наблюдений
$X_1,\ldots , X_n, $ извлеченной из генеральной совокупности, яв\-ля\-ющей\-ся смесью~(3)
генеральных совокупностей типа~(2) (при заданном общем виде составляющих
смесь функций $f_j (x;\,\theta_j )$), построить статистические оценки для числа
компонентов смеси~$k$, их удельных весов $p_j$ и, главное, для каждого из
компонентов %f_j (x;\,\theta_j )$ анализируемой смеси. Далее будет считать, что
функции $f_j$ однозначно определяются своими параметрами $\theta_j$: $f_j
=f(x;\,\theta_j)$.

Однако не следует ставить знак тождества между задачей разделения смеси
и задачей статистического оценивания параметров в модели~(4) по выборке $
X_1,\ldots , X_n$, поскольку задача разделения сохраняет смысл и
применительно к генеральным совокупностям, т.е.\ в теоретическом
варианте~\cite{3bat}.

Итак, для статистического анализа на основе реальных данных мы
аппроксимируем нашу общую модель~(1) следующей:
$$
p(x) \approx \hat{p}(x) = \sum\limits_{j=1}^k p_j \g_{\lambda_j , \alpha_j}
(x)\,,
$$
где $p_j$~--- дискретные смешивающие параметры, $\g_{\lambda_j , \alpha_j}
(x)$~--- плотности гамма-распределений.

Такая аппроксимация не только позволяет решить поставленную статистическую
задачу, но и полу\-чить наглядную визуализацию результатов. Существуют
достаточно эффективные методики разделения смесей распределений, среди них~---
семейство так называемых \textit{ЕМ-алгоритмов}
(\textit{Expectation-Maximization Algorithms}).

Полученные результаты могут быть достаточно просто интерпретированы. Число
компонентов смеси символизирует число типичных параллельных или
последовательных структур. Значения параметров составляющих смесь
гам\-ма-рас\-пре\-де\-ле\-ний показывают <<степень параллелизма>>
соответствующей структуры: чем ближе параметр формы к~1, тем выше эта
<<степень>>. И наоборот, чем дальше значение параметра формы от~1, тем больше
последовательных операций выполняется в соответствующем блоке.

Веса компонентов характеризуют примерную долю использования
ресурсов для сообщений, соответствующих каждому распределению входных
данных.

Итак, предложенный подход позволяет получить представление о
стохастической структуре телекоммуникационной сети.

\section{ЕМ-алгоритм разделения смесей распределений}

\subsection{Описание алгоритма} %2.1.

Определяемый ниже итерационный алгоритм решения поставленной в
предыдущем разделе задачи относится к процедурам, базирующимся на
\textit{методе максимального правдоподобия}.

Этот алгоритм позволяет находить максимум логарифмической функции
правдоподобия по параметрам $p_1,\,p_2,\ldots ,\,p_k$, $\theta_1 ,\,\theta_2,\ldots ,\,
\theta_k$ при фиксированном $k$ по выборке $X_1, \ldots , X_n$, т.е.\ решение
оптимизационной задачи вида

\begin{equation} \sum\limits_{i=1}^n \ln \left ( \sum\limits_{j=1}^k p_j f_j
(X_i;\,\theta_j )\right ) \rightarrow \underset{p_j,\,\theta_j}{\max}\,.
\end{equation}

Конкретные алгоритмы, построенные по этой схеме, часто называют
\textit{алгоритмами типа ЕМ}, поскольку в каждом из них можно выделить два
этапа, находящихся по отношению друг к другу в последовательности
итерационного взаимодействия: \textit{оценивание} (\textit{Estimation}) и
\textit{максимизация} (\textit{Maximization})~\cite{4bat}.

Введем в рассмотрение так называемые апостериорные вероятности
$\g_{ij}$ принадлежности наблюдения $X_i$ к $j$-му классу:
\begin{equation} %6
\g_{ij} = \fr{p_j f(X_i;\,\theta_j )}{\sum\limits_{l=1}^k p_l f(X_i;\,\theta_l 
)} \ (i=1,\ldots , n;\ j=1,\ldots ,k)\,.\!\!\end{equation} 
Очевидно, что для 
всех $i=1,\ldots ,n$ и $j=1,\ldots ,k$
$$
\g_{ij} \geq 0,\quad \sum_{j=1}^k \g_{ij} =1\,.
$$


Далее обозначим $\Theta = (p_1,\ldots p_k,\,\theta_1,\ldots ,\theta_k )$ и
представим анализируемую логарифмическую функцию правдоподобия
$$
\ln L(\Theta ) = \sum\limits_{i=1}^n \ln \left (\sum\limits_{j=1}^k p_j f_j
(X_i;\,\theta_j )\right )
$$
в виде
\begin{multline}
\ln L (\Theta ) = \sum\limits_{j=1}^k\sum\limits_{i=1}^n \g_{ij} \ln p_j+{}\\
{}+\sum\limits_{j=1}^k\sum\limits_{i=1}^n \g_{ij} f(X_i;\,\theta_j)-
\sum\limits_{j=1}^k\sum\limits_{i=1}^n \g_{ij} \ln \g_{ij}\,.
\end{multline}

Справедливость этого тождества легко проверяется с учетом
$$
\sum\limits_{j=1}^k \g_{ij} =1\,.
$$

Далее идея построения итерационного алгоритма вычисления оценок
$\hat{\Theta} = (\hat{p}_1,\ldots , \hat{p}_k,\
\hat{\theta}_1,\ldots , \hat{\theta}_k)$
для параметров $\Theta = (p_1,\ldots , p_k,\ \theta_1,\ldots , \theta_k)$ состоит в
следующем:
\begin{enumerate}[1.]
\item Выбирается некоторое \textit{начальное приближение}~$\hat{\Theta}^0$.
\item \textbf{E-step:} вычисляются по формулам~(6) начальные приближения
$\g_{ij}^0$ для апостериорных вероятностей $\g_{ij}$~--- \textit{этап
оценивания}.
\item \textbf{M-step:} затем, возвращаясь к~(7), при вычисленных значениях
$\g^0_{ij}$ следует определить значения $\hat{\Theta}^1$ из условия
максимизации отдельно каждого из первых двух слагаемых правой
части~(7), поскольку первое слагаемое
$$
\sum_{j=1}^k \sum_{i=1}^n \g_{ij} \ln p_j
$$
зависит только от параметров $p_j$, а второе слагаемое
$$
\sum_{j=1}^k \sum_{i=1}^n \g_{ij} f(X_i;\,\theta_j )
$$
зависит только от параметров $\theta_j$~--- \textit{этап максимизации}.
\item Проверяется \textit{условие останова}:
$$
\parallel \Theta^{(t)} - \Theta^{t-1}\parallel <\varepsilon\,,
$$
где $t$~--- номер итерации, а
$\parallel\bullet\parallel$~--- евклидова норма, для некоторого $\varepsilon
>0$.
\end{enumerate}

Очевидно, решение оптимизационной задачи
$$
\sum\limits_{j=1}^k\sum\limits_{i=1}^n \g_{ij}^{(t)}\ln p_j \rightarrow
\underset{p_j}{\max}
$$
дается выражением (с учетом $\sum_{j=1}^k p_j =1$):
$$
p_{ij}^{(t+1)} =\fr{1}{n}\,\sum\limits_{i=1}^n \g_{ij}^{(t)}\,,
$$
где $t$~--- номер итерации, $t = 0$, 1, 2,\,\ldots

Решение оптимизационной задачи
$$
\sum\limits_{j=1}^k \sum\limits_{i=1}^n \g_{ij}^{(t)} f(X_i;\,\theta_j )
\rightarrow \underset{\theta_j}{\max}
$$
получить намного проще решения задачи~(5): выражение для $\theta_j$
записывается с учетом знания конкретного вида функций
$f(X,\,\theta)$~\cite{3bat}.

\subsection{О сходимости алгоритма} %2.2.

В работе М.\,И.~Шлезингера~\cite{5bat}, где эта схема (позднее названная
ЕМ-схемой) впервые предложена, установлены и основные свойства
реа\-ли\-зу\-ющих ее алгоритмов. В частности, было доказано, что при достаточно
широких предположениях \textit{предельные точки} всякой последовательности,
порожденной итерациями ЕМ-алгоритма, являются стационарными точками
оптимизируемой логарифмической функции правдоподобия $\ln L(\Theta )$ и что
найдется неподвижная точка алгоритма, к которой будет сходиться каждая из таких
последовательностей. Если дополнительно потребовать положительной
определенности информационной мат\-ри\-цы Фишера для $\ln L(\Theta )$ при
истинных зна\-че\-ни\-ях па\-ра\-мет\-ра $\Theta$, то можно показать, что
асимптотически по $n\rightarrow\infty$ (т.е.\ при больших выборках) существует
единственное сходящееся (по веро\-ят\-но\-сти) решение $\hat{\Theta}(n)$
уравнений метода максимального правдоподобия и, кроме того, существует в
пространстве параметров $\Theta$ норма, в которой последовательность
$\Theta^{(t)}(n)$, порожденная ЕМ-ал\-го\-рит\-мом, сходится к $\hat{\Theta}
(n)$, если только начальная аппроксимация $\hat{\Theta}^0$ не была слишком
далека от~$\hat{\Theta} (n)$. {%\looseness=1

}

Таким образом, результаты исследования свойств ЕМ-алгоритмов метода
максимального правдоподобия разделения смеси и их практическое
использование показали, что они являются достаточно работоспособными (при
известном чис\-ле компонентов смеси) даже при большом чис\-ле $k$ компонентов и
при высоких размерностях анализируемого признака~$X$~\cite{3bat}.

\subsection{Уравнения для смеси экспоненциальных распределений}
%2.3.

Применим описанный выше алгоритм к разделению смеси
экспоненциальных распределений:
$$
p(x) = \sum\limits_{j=1}^k p_j \lambda_j e^{-\lambda_j x}\,.
$$
Получаем следующие итерационные уравнения:
\begin{align*}
\g_{ij}^{(t+1)} & = \fr{p_j^{(t)}\lambda_j^{(t)}e^{-
\lambda_j^{(t)}X_i}}{\sum\limits_{l=1}^k p_l^{(t)}\lambda_l^{(t)}
e^{-\lambda_l^{(t)}X_i}}\,,\\
p_j^{(t+1)} & = \fr{1}{n}\,\sum\limits_{i=1}^n \g_{ij}^{(t)}\,.
\end{align*}

Чтобы найти  оценки $\lambda_j$, подсчитаем первую производную функции
$$\sum_{j=1}^k\sum_{i=1}^n \g_{ij}^{(t)} \ln (\lambda_j e^{-\lambda_j X_i}):$$
\vspace*{-8pt}
\begin{multline*}
\left ( \sum\limits_{j=1}^k \sum\limits_{i=1}^n
\g_{ij}^{(t)}\ln \left ( \lambda_j
e^{-\lambda_j X_i} \right ) \right )^\prime \lambda_j =\\[-3pt]
{}= \left (
\sum\limits_{j=1}^k\sum\limits_{i=1}^n \g_{ij}^{(t)}\ln (\lambda_j -\lambda_j X_i )
\right )^\prime \lambda_j =\\[-3pt]
{}= \sum\limits_{i=1}^n \g_{ij}^{(t)}\left (
\fr{1}{\lambda_j} - X_i \right )\,.
\end{multline*}

Разрешая уравнение
$$
\sum\limits_{i=1}^n \g_{ij}^{(t)}\left ( \fr{1}{\lambda_j} -X_i\right ) =0
$$
относительно $\lambda_j$, получаем следующее итерационное уравнение:
$$
\lambda_j^{(t+1)} = \fr{\sum\limits_{i=1}^n
\g_{ij}^{(t)}}{\sum\limits_{i=1}^n \g_{ij}^{(t)} X_i}\,.
$$

\subsection{Уравнения для смеси гамма-распределений } %2.4.

Применим теперь ЕМ-алгоритм к смеси гам\-ма-рас\-пре\-де\-ле\-ний вида
$$
p(x) = \sum\limits_{j=1}^k p_j \fr{\alpha_j^{\alpha_j} x^{\alpha_j -
1}}{\lambda_j^{\alpha_j} \Gamma (\alpha_j )}\,e^{-(\alpha_j / \lambda_j)x}\,.
$$

Такая параметризация удобна для нахождения
оценок~$\alpha_j$~\cite{6bat}.

Аналогичным способом выписываются итерационные уравнения:
\begin{align*}
\g_{ij}^{(t+1)} & =   \fr{p_j^{(t)}\fr{(\alpha_j^{\alpha_j} )^{(t)}
x^{\alpha_j - 1}}{(\lambda_j^{\alpha_j} )^{(t)}\Gamma (\alpha_j)}\,
e^{-(\alpha_j /\gamma_j)^{(t)}x}}{\sum\limits_{l=1}^k
p_l^{(t)}\fr{(\alpha_l^{\alpha_l})^{(t)} x^{\alpha_l -
1}}{(\lambda_l^{\alpha_l})^{(t)}\Gamma (\alpha_l )}\,
e^{-(\alpha_l /\lambda_l)^{(t)} x}}\,,\\
p_j^{(t+1)} & = \fr{1}{n}\,\sum\limits_{i=1}^n \g_{ij}^{(t)}\,.
\end{align*}

Далее найдем оценки $\lambda_j$ для данного случая, приравнивая
производную
\begin{equation} %8
\sum\limits_{j=1}^k \sum\limits_{i=1}^n \g_{ij}^{(t)} \ln \left (
\fr{\alpha_j^{\alpha_j} x^{\alpha_j -1}}{\lambda_j^{\alpha_j}\Gamma
(\alpha_j)}\,e^{-(\alpha_j /\lambda_j) x}\right )
\end{equation}
по $\lambda_j$ к нулю и разрешая относительно~$\lambda_j$ уравнение:
$$
\sum\limits_{i=1}^n \g_{ij}^{(t+1)}\left ( \fr{\alpha_j^{(t)}}{\lambda_j^{(t)}}
- \fr{\alpha_j^{(t)}X_i}{\left ( \lambda_j^{(t)}\right )^2}\right ) =0 \,.
$$
Получаем
$$
\lambda_j^{(t+1)} = \fr{\sum\limits_{i=1}^n \g_{ij}^{(t)}
X_i}{\sum\limits_{i=1}^n \g_{ij}^{(t)}}\,.
$$

Для того чтобы получить итерационные уравнения для $\alpha_j$, найдем
первую производную~(8):
\begin{multline*}
\left ( \sum\limits_{j=1}^k\sum\limits_{i=1}^n \g_{ij}^{(t)}\ln \left (
\fr{\alpha_j^{\alpha_j} x^{\alpha_j -1}}{\lambda_j^{\alpha_j}\Gamma (\alpha_j
)}\,e^{-(\alpha_j /\lambda_j ) x} \right ) \right )^\prime \alpha_j ={}\\[-3pt]
{}=\left ( \sum\limits_{j=1}^k\sum\limits_{i=1}^n \g_{ij}^{(t)}\ln \left (
\fr{\alpha_j^{\alpha_j}}{\lambda_j^{\alpha_j}}\right ) - \ln \Gamma (\alpha_j )+{} \right.\\[-3pt]
{}+\left.
(\alpha_j -1 )\ln X_i - \fr{\alpha_j}{\lambda_j}\,X_i \right )^\prime \alpha_j =\\[-3pt]
{}=\sum\limits_{i=1}^n \g_{ij}^{(t)} \left ( \ln \alpha_j +1-\ln \lambda_j -
\fr{\Gamma^\prime (\alpha_j )}{\Gamma (\alpha_j)}\right.+\\[-3pt]
{}+\left. \ln X_i - \fr{X_i}{\lambda_j}\right )\,;
\end{multline*}
\begin{multline*}
\sum\limits_{i=1}^n \g_{ij}^{(t)} \left(  \ln \alpha_j +1 -\ln \lambda_j -{}\right. \\[-3pt]
\left. {}-\fr{\Gamma^\prime (\alpha_j )}{\Gamma (\alpha_j )}+\ln X_i 
-\fr{X_i}{\lambda_j} \right) =0\,;
\end{multline*}
\begin{multline}
\fr{\Gamma^\prime (\alpha_j )}{\Gamma (\alpha_j )} ={}\\[-3pt]
{}= \fr{\sum\limits_{i=1}^n \g_{ij}^{(t)} \left ( \ln \alpha_j +1-\ln\lambda_j 
+\ln X_i -\fr{X_i}{\lambda_j} \right )}{\sum\limits_{i=1}^n \g_{ij}^{(t)}}\,.
\end{multline}
%
Здесь $\Gamma^\prime (\alpha_j ) / \Gamma (\alpha_j )$~--- это
\textit{логарифмическая производная гамма-функции}. Для нее существует так
называемое \textit{разложение Абрамовитца}--\textit{Стигана}~\cite{4bat}:
$$
\fr{\Gamma^\prime (\alpha ) }{ \Gamma (\alpha )} = \mathrm{log}\,\alpha -
\fr{1}{2\alpha }-\fr{1}{12\alpha^2 }+\ldots
$$

Подставим первые три члена разложения в~(9) и разрешим это уравнение
относительно~$\alpha_j$:
$$
\alpha_{ij}^{(t+1)} = \fr{\sum\limits_{i=1}^n
\g_{ij}^{(t+1)}}{2\sum\limits_{i=1}^n \g_{ij}^{(t +1)}\left ( \fr{X_i}{\lambda_j^{(t)}} -
\ln \fr{X_i}{\lambda_j^{(t)}} -1\right )}\,.
$$
В итоге получаем итерационные уравнения для ~$\alpha_j$.

\section{Описание программного обеспечения (программа~ЕМ)}

\subsection{Назначение программы} %3.1.

Разработанная авторами статьи программа ЕМ предназначена для решения задачи
разделения смесей экспоненциальных и гамма-распределений, поставленной в
разд.~2, с использованием ЕМ-ал\-го\-рит\-ма и формул, описанных в разд.~3.

\subsection{Инструменты разработки} %3.2.

Для создания программы была использована среда разработки Microsoft
Visual Studio .NET 2005 и объектно-ориентированный язык C\#. Для
визуализации результатов была использована свободно распространяемая
графическая библиотека ZedGraph~\cite{7bat}.


\subsection{Возможности  программы} %3.3.

\noindent
\begin{itemize}
\item Загрузка выборочных данных из текстового файла
\item Оценивание по выборке параметров смеси экспоненциальных
распределений
\item Оценивание по выборке параметров смеси гамма-распределений
\item Отслеживание изменений параметров смесей распределений во
времени в режиме <<скользящего окна>>
\item Построение гистограммы по выборке
\end{itemize}

\subsection{Входные и выходные данные. Функционирование
программы} %3.4.

В качестве \textit{входных данных} программа ЕМ получает:
\begin{itemize}
\item выборочные данные из текстового файла;
\item число компонентов смеси;
\item размер <<скользящего окна>>;
\item размер класса гистограммы.
\end{itemize}

На \textit{выходе} мы получаем:
\begin{itemize}
\item точечные оценки параметров смеси экспоненциальных
распределений;
\item точечные оценки параметров смеси гамма-распределений;
\item графическое изображение результирующей смеси распределения;
\item графическое изображение компонентов каж\-дой смеси;
\item графическое изображение того, как меняются параметры смесей
распределений с течением времени в режиме <<скользящего окна>>;
\item гистограмма, построенная по выборке;
\item значение статистического теста.
\end{itemize}

Выборочные данные загружаются из текстового файла в память программы и подаются
на вход двум независимо работающим реализациям ЕМ-алгоритма~--- для
идентификации смеси экспоненциальных распределений и для идентификации смеси
гамма-распределений. Результатом их работы являются наборы значений оцениваемых
параметров модели, предложенной в разд.~2. Кроме того, результирующие
распределения визуализируются в виде графиков. В программе можно запустить
режим <<скользящего окна>>, который для всех подвыборок заданного
размера с помощью ЕМ-алгоритма оценивает параметры смесей распределений этих
подвыборок. Все действия программы документируются в окне информации.

\section{Описание тестовых расчетов}

С использованием разработанной программы были проведены тестовые
расчеты на выборочных данных реального сетевого трафика.

На вход программы EM были поданы выборки трафика:
\begin{enumerate}[I]
\item Между лабораторией Lawrence Berkeley (Berkeley, California) и
внешним миром размера примерно 7000~\cite{8bat}~--- \textit{выборка~1}.
\item
Сети радиодоступа ЗАО <<Синтерра>> размера примерно 1000~\cite{9bat}~---
 \textit{выборка~2}.
\end{enumerate}

\subsection{Выборка 1 ``Berkeley''} %5.1.

При числе компонентов смеси~5 и случайном начальном приближении
были получены результаты, представленные в табл.~\ref{t1bat}.


Данные результаты иллюстрирует рис.~\ref{f5bat}.

Гистограмма  на рис.~\ref{f6bat} показывает статистическую значимость
полученных результатов.

Данная выборка обладает той особенностью, что она собиралась в течение
достаточно длительного времени и в ней агрегирован самый разнородный
трафик. Поэтому в ней присутствует не только большое количество
<<коротких>> сообщений (что обычно для выборок из телетрафика), но и
некоторый массив сообщений средней длины, а также определенный
<<выброс>> больших сообщений. Это свидетельствует о \textit{пиковости}
телетрафика на довольно больших промежутках времени.

Как мы видим, ЕМ-алгоритм удачно справился с задачей идентификации
смеси.

\subsection{Выборка~2 ``Synterra''} %5.2.

Результаты применения ЕМ-алгоритма к выборке ``Synterra''
представлены в табл.~\ref{t2bat}.
\begin{table*}\small
\begin{minipage}[t]{76mm}
\begin{center}
\Caption{Результаты применения ЕМ-алго\-рит\-ма к выборке~1 ``Berkeley'' 
\label{t1bat}} \vspace*{2ex}

\tabcolsep=8.7pt
\begin{tabular}{|c|c|c|}
\hline
№&Начальное приближение&Результат\\
\hline
\multicolumn{3}{|c|}{$P$}\\
\hline
0&0,2&0,1896\\
1&0,2&0,1858\\
2&0,2&0,1830\\
3&0,2&0,2259\\
4&0,2&0,2154\\
\hline
\multicolumn{3}{|c|}{$\alpha$}\\
\hline
0&2,7028&10,9783\hphantom{9}\\
1&3,6273&5,8621 \\
2&5,7598&2,7092\\
3&0,2315&1,0235\\
4&0,9110&0,4772\\
\hline
\multicolumn{3}{|c|}{$\lambda$}\\
\hline
0&85,2066&137,1714  \\
1&23,9592&136,7349\\
2&63,8425&132,6482\\
3&\hphantom{9}1,8026&116,7317\\
4&98,3882&102,5278\\
\hline
\end{tabular}
\end{center}
\end{minipage}\hfill
\begin{minipage}[t]{76mm}
%\end{table*}
%\begin{table*}\small
\begin{center}
\Caption{Результаты применения ЕМ-алго\-рит\-ма к выборке~2 ``Synterra'' 
\label{t2bat}} \vspace*{2ex}

\tabcolsep=8.7pt
\begin{tabular}{|c|c|c|}
\hline
№&Начальное приближение&Результат\\
\hline
\multicolumn{3}{|c|}{$P$}\\
\hline
0&0,2&$0{,}3815\hphantom{{}\cdot 10^{-9}}$\\
1&0,2&$0{,}3594\hphantom{{}\cdot 10^{-9}}$\\
2&0,2&$0{,}2589\hphantom{{}\cdot 10^{-9}}$\\
3&0,2&$0{,}4401\cdot 10^{-9}$\\
4&0,2&$0{,}0\hphantom{{}\cdot 10^{-9}999}$\\
\hline
\multicolumn{3}{|c|}{$\alpha$}\\
\hline
0&6,0804&1,5833\\
1&3,1838&0,8554\\
2&1,4886&0,4557\\
3&4,6407&0,2278\\
4&3,7843&0,1139\\
\hline
\multicolumn{3}{|c|}{$\lambda$}\\
\hline
0&17,3387&15,8682\\
1&47,8294&16,9150\\
2&54,1984&19,2866\\
3&\hphantom{1}8,6254&19,2866\\
4&\hphantom{1}5,7252&19,2866\\
\hline
\end{tabular}
\end{center}
\end{minipage}
\end{table*}


Данные результаты иллюстрируют рис.~\ref{f7bat}.


Эти результаты также отражают действительную картину, как показано на
рис.~\ref{f8bat}.


Этот трафик был снят с базовой станции <<Лукойл-Юго-Запад>> сети
широкополосного радиодоступа ЗАО <<Синтерра>>. Сеть радиодоступа
является реализацией так называемой <<последней мили>>, переносящей два
разных вида трафика: данные (Ethernet пакеты) и голос (IP-телефония, VoIP).
Поэтому здесь присутствуют в качестве основной массы короткие, но
интенсивные сообщения (пакеты SIP и голосовые фреймы), а также длинные
сообщения, содержащие данные.

Как мы видим, программная реализация ЕМ-ал\-го\-рит\-ма успешно справилась с
задачей разделения смесей распределений для этих двух выборок, что делает
данную программу удобным инструментом построения стохастической картины
конкретной сети. По полученным данным, используя метод интерпретации,
предложенный в разд.~2, можно получить представление о количестве
последовательных и параллельных структур вероятностной модели сети.

\subsection{Режим <<скользящего окна>>} %5.3.

Результаты для выборки
``Berkeley'' в режиме <<скользящего окна>>  представлены
на рис.~\ref{f9bat}.


Данные графики показывают изменение параметров распределений подвыборок выборки 
``Berkeley''. Видно, что параметры распределений подвыборок не остаются 
неизменными во времени, наоборот, они имеют внешне случайный характер. На 
рис.~\ref{f9bat},\,\textit{в} видна даже своеобразная пульсация первой 
компоненты.
%
На основании расчетов можно сделать вывод о том, что пиковость трафика
обусловливается как формой, так и интенсивностью сообщений.

\section{Заключение}

В данной работе исследована вероятностная модель  информационных потоков,
возникающих в сложных телекоммуникационных конвергентных сетях, построенная с
помощью асимптотического и энтропийного подходов. Эта модель предполагает, что
функционирование сложной телекоммуникационной сети можно представить в виде
суперпозиции довольно простых стохастических структур~--- последовательных и
параллельных, которые по\-рож\-да\-ют смеси гамма-распределений для случайной
величины времени обработки и передачи сообщений в сети. Предложена простая
интерпретация параметров данной модели.
\begin{figure*} %fig5
\vspace*{1pt}
\begin{center}
\mbox{%
\epsfxsize=130mm %145.109mm 
\epsfbox{bat-5.eps} }
\end{center}
\vspace*{-13pt} \Caption{Компоненты смеси начального приближения~(\textit{а}) и 
результата~(\textit{б}) для выборки~1 ``Berkeley'' \label{f5bat}}
%\end{figure*}
%\begin{figure*} %fig6
\vspace*{12pt}
\begin{center}
\mbox{%
\epsfxsize=130mm %148.256mm 
\epsfbox{bat-7.eps} }
\end{center}
\vspace*{-13pt} \Caption{График смеси распределений~(\textit{1}) и гистограмма 
для выборки~1 ``Berkeley''~(\textit{2}) \label{f6bat}}
\end{figure*}



\begin{figure*} %fig7
\vspace*{1pt}
\begin{center}
\mbox{%
\epsfxsize=130mm %144.283mm 
\epsfbox{bat-8.eps} }
\end{center}
\vspace*{-16pt} \Caption{Компоненты смеси начального приближения~(\textit{а}) и 
результата~(\textit{б}) для выборки~2 ``Synterra'' \label{f7bat}}
%\end{figure*}
%\begin{figure*} %fig8
\vspace*{12pt}
\begin{center}
\mbox{%
\epsfxsize=130mm %148.256mm 
\epsfbox{bat-10.eps} }
\end{center}
\vspace*{-11pt} \Caption{График смеси распределений~(\textit{1}) и гистограмма
для выборки~2 ``Synterra''~(\textit{2}) \label{f8bat}}
\end{figure*}

\begin{figure*} %fig9
\vspace*{1pt}
\begin{center}
\mbox{%
\epsfxsize=119.041mm
\epsfbox{bat-11.eps} }
\end{center}
\vspace*{-9pt} \Caption{Изменение  смешивающих параметров~(\textit{а}), 
параметров формы~(\textit{б}) и параметров масштаба~(\textit{в}) во времени для 
выборки~1 ``Berkeley'' \label{f9bat}}
\end{figure*}

Для решения вытекающей из модели задачи предложен итерационный алгоритм,
базирующийся на методе максимального правдоподобия~--- ЕМ-ал\-го\-ритм, для
которого получены формулы для конкретного вида смесей~--- экспоненциальных и
гамма-распределений.
%
Кроме того, разработан программный инструментарий для оценки параметров 
предложенной модели на выборках из реальных трафиковых данных. Проведены 
исследования, которые подтвердили предположения вероятностной модели. 


Получение информации о стохастической структуре
телекоммуникационных сетей и наличие программных инструментов для
выявления более или менее стабильных структур позволит понять причины
возникновения неожиданных больших нагрузок, предотвратить такие нагрузки,
а также поможет в будущем в проектировании надежных, оптимальных по
стоимости и уровню сервиса телекоммуникационных сетей нового поколения.

%\vspace*{-15pt} 
{\small\frenchspacing
{%\baselineskip=10.8pt
\addcontentsline{toc}{section}{Литература}
\begin{thebibliography}{9}
\bibitem{1bat}
Teletraffic Engeneering Handbook. International Telecommunication Union, 
Geneva, 2005 {\sf http://www.itu.int}. \vspace*{5pt} 
\bibitem{2bat}
\Au{Севастьянов~Б.\,А.} Курс теории вероятностей и математической статистики. 
М., 2004. \vspace*{5pt} 
\bibitem{3bat}
\Au{Айвазян~C.\,А., Бухштабер~В.\,М., Енюков~И.\,С, Мешалкин~Л.\,Д.} Прикладная 
статистика. Классификация и снижение размерности~// Финансы и статистика. М., 
1989. \vspace*{5pt} 
\bibitem{4bat}
\Au{Bilmes~J.\,A.} A gentle tutorial of the EM algorithm and its application to 
parameter estimation for Gaussian mixture and hidden Markov models. Berkeley, 
CA, USA: International Computer Science Institute,  1998. \vspace*{5pt} 
\bibitem{5bat}
\Au{Шлезингер~М.\,И.} О самопроизвольном различении образов~// Шлезингер~М.\,И. 
Читающие. автоматы. Киев: Наукова думка, 1965. С.~38--45. \vspace*{5pt} 
\bibitem{6bat}
\Au{Hsiao~I.-T., Rangarajan~A., Gindi~G.}. Joint-MAP 
reconstruction/segmentation for transmission tomography using mixture-models as 
priors. Yale University, 1998. \vspace*{5pt} 
\bibitem{7bat}
{\sf http://zedgraph.org}. \vspace*{4pt} 
\bibitem{8bat}
{\sf http://ita.ee.lbl.gov/html/contrib/LBL-PKT.html}. \vspace*{5pt} 
\bibitem{9bat}
{\sf http://www.synterra.ru}.
\end{thebibliography}

} } \label{end\stat}
\end{multicols}


%\addtocounter{razdel}{1}
%\def\razd{НЕРЕГУЛИРУЕМЫЙ ЭЛЕКТРОПРИВОД ДЛЯ ЭЛЕКТРОЭНЕРГЕТИКИ}

\setcounter{page}{2}

%   { %\Large  
   { %\baselineskip=16.6pt
   
   \vspace*{-48pt}
   \begin{center}\LARGE
   \textit{Предисловие}
   \end{center}
   
   %\vspace*{2.5mm}
   
   \vspace*{25mm}
   
   \thispagestyle{empty}
   
   { %\small 

    
Вниманию читателей журнала <<Информатика и её применения>> предлагается 
очередной тематический выпуск <<Вероятностно-статистические методы и 
задачи информатики и информационных технологий>>. Предыдущие тематические 
выпуски журнала по данному направлению вышли в 2008~г.\ (т.~2, вып.~2), 
в 2009~г.\ (т.~3, вып.~3) и в 2010~г.\ (т.~4, вып.~2). 

Статьи, собранные в данном журнале, посвящены разработке новых вероятностно-статистических 
методов, ориентированных на применение к решению конкретных задач информатики и информационных 
технологий, а также~--- в ряде случаев~--- и других прикладных задач. Проблематика, охватываемая 
публикуемыми работами, развивается в рамках научного сотрудничества между Институтом проблем 
информатики Российской академии наук (ИПИ РАН) и Факультетом вычислительной математики и 
кибернетики Московского государственного университета им.\ М.\,В.~Ломоносова в ходе работ 
над совместными научными проектами (в том числе в рамках функционирования 
Научно-образовательного центра <<Вероятностно-статистические методы анализа рисков>>). 
Многие из авторов статей, включенных в данный номер журнала, являются активными участниками 
традиционного международного семинара по проблемам устойчивости стохастических моделей, 
руководимого В.\,М.~Золотаревым и В.\,Ю.~Королевым; регулярные сессии этого семинара 
проводятся под эгидой МГУ и ИПИ РАН (в 2011~г.\ указанный семинар проводится в октябре 
в Калининградской области РФ). 

Наряду с представителями ИПИ РАН и МГУ в число авторов данного выпуска журнала входят 
ученые из Научно-исследовательского института системных исследований РАН, Института 
проблем технологии микроэлектроники и особочистых материалов РАН, Института 
прикладных математических исследований Карельского НЦ РАН, Московского 
авиационного института, Вологодского государственного педагогического университета, 
НИИММ им.\ Н.\,Г.~Чеботарева, Казанского государственного университета, Дебреценского 
университета (Венгрия).

Несколько статей выпуска посвящено разработке и применению стохастических методов и 
информационных технологий для решения различных прикладных задач. В~работе В.\,Г.~Ушакова 
и О.\,В.~Шестакова рассмотрена задача определения вероятностных характеристик случайных 
функций по распределениям интегральных преобразований, возникающих в задачах эмиссионной 
томографии. В~статье Д.\,О.~Яковенко и М.\,А.~Целищева рассмотрены некоторые вопросы 
математической теории риска и предложен новый подход к диверсификации инвестиционных 
портфелей. Работа И.\,А.~Кудрявцевой и А.\,В.~Пантелеева посвящена построению и 
исследованию математической модели, описывающей динамику сильноионизованной плазмы. 
В~статье П.\,П.~Кольцова изучается качество работы ряда алгоритмов сегментации изображений. 
Статья А.\,Н.~Чупрунова и И.~Фазекаша посвящена вероятностному анализу числа без\-оши\-бочных 
блоков при помехоустойчивом кодировании; получены усиленные законы больших чисел для указанных 
величин.

В данном выпуске традиционно присутствует тематика, весьма активно разрабатываемая в течение 
многих лет специалистами ИПИ РАН и МГУ,~--- методы моделирования и управления для 
информационно-телекоммуникационных и вычислительных систем, в частности методы 
теории массового обслуживания. В~статье А.\,И.~Зейфмана с соавторами рассматриваются 
модели обслуживания, описываемые марковскими цепями с непрерывным временем в случае 
наличия катастроф. В~работе М.\,М.~Лери и И.\,А.~Чеплюковой рассматриваются случайные 
графы Интернет-типа, т.\,е.\ графы, степени вершин которых имеют степенные распределения; 
такие задачи находят применение при исследовании глобальных сетей передачи данных. 
Работа Р.\,В.~Разумчика посвящена исследованию систем массового обслуживания специального 
вида~--- с отрицательными заявками и хранением вытесненных заявок.

Ряд статей посвящен развитию перспективных теоретических 
вероятностно-статистических методов, которые находят широкое применение в различных 
задачах информатики и информационных технологий. В~работе В.\,Е.~Бенинга, А.\,К.~Горшенина 
и В.\,Ю.~Королева рассмотрена задача статистической проверки гипотез о числе компонент 
смеси вероятностных распределений, приводится конструкция асимптотически наиболее мощного 
критерия. Результаты этой работы найдут применение в ряде прикладных задач, использующих 
математическую модель смеси вероятностных распределений (в информатике, моделировании 
финансовых рынков, физике турбулентной плазмы и~т.\,д.). В~статье В.\,Ю.~Королева, 
И.\,Г.~Шевцовой и С.\,Я.~Шоргина строится новая, улучшенная оценка точности нормальной 
аппроксимации для пуассоновских случайных сумм; как известно, указанные случайные суммы 
широко используются в качестве моделей многих реальных объектов, в том числе в информатике, 
физике и других прикладных областях. Работа В.\,Г.~Ушакова и Н.\,Г.~Ушакова посвящена 
исследованию ядерной оценки плотности распределения; эти результаты могут применяться, 
в част\-ности, при анализе трафика в телекоммуникационных системах. Серьезные приложения 
в статистике могут получить результаты работы О.\,В.~Шестакова, в которой доказаны оценки 
скорости сходимости распределения выборочного абсолютного медианного отклонения к нормальному 
закону. 

\smallskip

Редакционная коллегия журнала выражает надежду, что данный тематический  выпуск 
будет интересен специалистам в области теории вероятностей и математической статистики 
и их применения к решению задач информатики и информационных технологий.
     
     %\vfill 
     \vspace*{20mm}
     \noindent
     Заместитель главного редактора журнала <<Информатика и её 
применения>>,\\
     директор ИПИ РАН, академик  \hfill
     \textit{И.\,А.~Соколов}\\
     
     \noindent
     Редактор-составитель тематического выпуска,\\
     профессор кафедры математической статистики факультета\\
      вычислительной математики и кибернетики МГУ им.\ М.\,В.~Ломоносова,\\
     ведущий научный сотрудник ИПИ РАН,\\ 
доктор физико-математических наук \hfill
      \textit{В.\,Ю.~Королев}
     
     } }
     }



%   { %\Large  
   { %\baselineskip=16.6pt
   
   \vspace*{-48pt}
   \begin{center}\LARGE
   \textit{Предисловие}
   \end{center}
   
   %\vspace*{2.5mm}
   
   \vspace*{25mm}
   
   \thispagestyle{empty}
   
   { %\small 

    
Вниманию читателей журнала <<Информатика и её применения>> предлагается 
очередной тематический выпуск <<Вероятностно-статистические методы и 
задачи информатики и информационных технологий>>. Предыдущие тематические 
выпуски журнала по данному направлению вышли в 2008~г.\ (т.~2, вып.~2), 
в 2009~г.\ (т.~3, вып.~3) и в 2010~г.\ (т.~4, вып.~2). 

Статьи, собранные в данном журнале, посвящены разработке новых вероятностно-статистических 
методов, ориентированных на применение к решению конкретных задач информатики и информационных 
технологий, а также~--- в ряде случаев~--- и других прикладных задач. Проблематика, охватываемая 
публикуемыми работами, развивается в рамках научного сотрудничества между Институтом проблем 
информатики Российской академии наук (ИПИ РАН) и Факультетом вычислительной математики и 
кибернетики Московского государственного университета им.\ М.\,В.~Ломоносова в ходе работ 
над совместными научными проектами (в том числе в рамках функционирования 
Научно-образовательного центра <<Вероятностно-статистические методы анализа рисков>>). 
Многие из авторов статей, включенных в данный номер журнала, являются активными участниками 
традиционного международного семинара по проблемам устойчивости стохастических моделей, 
руководимого В.\,М.~Золотаревым и В.\,Ю.~Королевым; регулярные сессии этого семинара 
проводятся под эгидой МГУ и ИПИ РАН (в 2011~г.\ указанный семинар проводится в октябре 
в Калининградской области РФ). 

Наряду с представителями ИПИ РАН и МГУ в число авторов данного выпуска журнала входят 
ученые из Научно-исследовательского института системных исследований РАН, Института 
проблем технологии микроэлектроники и особочистых материалов РАН, Института 
прикладных математических исследований Карельского НЦ РАН, Московского 
авиационного института, Вологодского государственного педагогического университета, 
НИИММ им.\ Н.\,Г.~Чеботарева, Казанского государственного университета, Дебреценского 
университета (Венгрия).

Несколько статей выпуска посвящено разработке и применению стохастических методов и 
информационных технологий для решения различных прикладных задач. В~работе В.\,Г.~Ушакова 
и О.\,В.~Шестакова рассмотрена задача определения вероятностных характеристик случайных 
функций по распределениям интегральных преобразований, возникающих в задачах эмиссионной 
томографии. В~статье Д.\,О.~Яковенко и М.\,А.~Целищева рассмотрены некоторые вопросы 
математической теории риска и предложен новый подход к диверсификации инвестиционных 
портфелей. Работа И.\,А.~Кудрявцевой и А.\,В.~Пантелеева посвящена построению и 
исследованию математической модели, описывающей динамику сильноионизованной плазмы. 
В~статье П.\,П.~Кольцова изучается качество работы ряда алгоритмов сегментации изображений. 
Статья А.\,Н.~Чупрунова и И.~Фазекаша посвящена вероятностному анализу числа без\-оши\-бочных 
блоков при помехоустойчивом кодировании; получены усиленные законы больших чисел для указанных 
величин.

В данном выпуске традиционно присутствует тематика, весьма активно разрабатываемая в течение 
многих лет специалистами ИПИ РАН и МГУ,~--- методы моделирования и управления для 
информационно-телекоммуникационных и вычислительных систем, в частности методы 
теории массового обслуживания. В~статье А.\,И.~Зейфмана с соавторами рассматриваются 
модели обслуживания, описываемые марковскими цепями с непрерывным временем в случае 
наличия катастроф. В~работе М.\,М.~Лери и И.\,А.~Чеплюковой рассматриваются случайные 
графы Интернет-типа, т.\,е.\ графы, степени вершин которых имеют степенные распределения; 
такие задачи находят применение при исследовании глобальных сетей передачи данных. 
Работа Р.\,В.~Разумчика посвящена исследованию систем массового обслуживания специального 
вида~--- с отрицательными заявками и хранением вытесненных заявок.

Ряд статей посвящен развитию перспективных теоретических 
вероятностно-статистических методов, которые находят широкое применение в различных 
задачах информатики и информационных технологий. В~работе В.\,Е.~Бенинга, А.\,К.~Горшенина 
и В.\,Ю.~Королева рассмотрена задача статистической проверки гипотез о числе компонент 
смеси вероятностных распределений, приводится конструкция асимптотически наиболее мощного 
критерия. Результаты этой работы найдут применение в ряде прикладных задач, использующих 
математическую модель смеси вероятностных распределений (в информатике, моделировании 
финансовых рынков, физике турбулентной плазмы и~т.\,д.). В~статье В.\,Ю.~Королева, 
И.\,Г.~Шевцовой и С.\,Я.~Шоргина строится новая, улучшенная оценка точности нормальной 
аппроксимации для пуассоновских случайных сумм; как известно, указанные случайные суммы 
широко используются в качестве моделей многих реальных объектов, в том числе в информатике, 
физике и других прикладных областях. Работа В.\,Г.~Ушакова и Н.\,Г.~Ушакова посвящена 
исследованию ядерной оценки плотности распределения; эти результаты могут применяться, 
в част\-ности, при анализе трафика в телекоммуникационных системах. Серьезные приложения 
в статистике могут получить результаты работы О.\,В.~Шестакова, в которой доказаны оценки 
скорости сходимости распределения выборочного абсолютного медианного отклонения к нормальному 
закону. 

\smallskip

Редакционная коллегия журнала выражает надежду, что данный тематический  выпуск 
будет интересен специалистам в области теории вероятностей и математической статистики 
и их применения к решению задач информатики и информационных технологий.
     
     %\vfill 
     \vspace*{20mm}
     \noindent
     Заместитель главного редактора журнала <<Информатика и её 
применения>>,\\
     директор ИПИ РАН, академик  \hfill
     \textit{И.\,А.~Соколов}\\
     
     \noindent
     Редактор-составитель тематического выпуска,\\
     профессор кафедры математической статистики факультета\\
      вычислительной математики и кибернетики МГУ им.\ М.\,В.~Ломоносова,\\
     ведущий научный сотрудник ИПИ РАН,\\ 
доктор физико-математических наук \hfill
      \textit{В.\,Ю.~Королев}
     
     } }
     }

\def\stat{serebr+ataeva}

\def\tit{ОНТОЛОГИЯ ЦИФРОВОЙ СЕМАНТИЧЕСКОЙ БИБЛИОТЕКИ LibMeta}

\def\titkol{Онтология цифровой семантической библиотеки LibMeta}

\def\aut{О.\,М.~Атаева$^1$, В.\,А.~Серебряков$^2$}

\def\autkol{О.\,М.~Атаева, В.\,А.~Серебряков}

\titel{\tit}{\aut}{\autkol}{\titkol}

\index{Атаева О.\,М.}
\index{Серебряков В.\,А.}
\index{Serebryakov V.\,A.}
\index{Ataeva O.\,M.}




%{\renewcommand{\thefootnote}{\fnsymbol{footnote}} \footnotetext[1]
%{Работа выполнена при финансовой поддержке РФФИ (проект 17-01-00816).}}


\renewcommand{\thefootnote}{\arabic{footnote}}
\footnotetext[1]{Вычислительный центр им.\ А.\,А.~Дородницына Федерального исследовательского 
центра <<Информатика и~управ\-ле\-ние>> Российской академии наук, 
\mbox{oli@ultimeta.ru}}
\footnotetext[2]{Вычислительный центр им.\ А.\,А.~Дородницына Федерального исследовательского 
центра <<Информатика и~управ\-ле\-ние>> Российской академии наук, \mbox{serebr@ultimeta.ru}}
%\vspace*{-6pt}



\Abst{При разработке цифровых библиотек особое внимание уделяют модели данных 
содержимого библиотеки. При этом контент цифровых библиотек может быть описан 
различными форматами и~представлен различными способами. Библиотека, определяемая 
с~по\-мощью системы LibMeta, рассматривается как хранилище структурированных 
разнообразных данных с~возможностью их интеграции с~другими источниками данных 
и~предполагает возможность специфицирования своего контента за счет описания 
предметной области. В~качестве средства формализации выступает онтология контента 
семантической библиотеки. Также вводятся основные понятия для описания задачи 
интеграции данных из источников Linked Open Data (LOD), понятия для определения 
произвольного тезауруса. Онтология построена таким образом, чтобы иметь возможность 
определения семантической библиотеки в~произвольной предметной области.}

\KW{семантические библиотеки; модель данных; онтологии; источники данных; поиск 
в~LOD}

  \DOI{10.14357/19922264180101} 
  
%\vspace*{9pt}


\vskip 10pt plus 9pt minus 6pt

\thispagestyle{headings}

\begin{multicols}{2}

\label{st\stat}

\section{Введение}

     В различных предметных областях модель данных содержимого 
цифровых семантических биб\-лио\-тек может существенно отличаться как по 
типам ресурсов, так и~по их структуре. При разработке таких биб\-лио\-тек особое 
внимание уделяют модели данных содержимого биб\-лио\-теки.
     
     Говоря о библиотеках, авторы прежде всего \mbox{имеют} в~виду 
разработанную информационную сис\-те\-му для создания семантических 
библиотек LibMeta~[1--3], с~по\-мощью которой создается и~описывается 
семантическая биб\-лио\-те\-ка некоторой предметной об\-ласти. 
     
     LibMeta представляет собой информационную сис\-те\-му, которая 
реализует функциональность, необходимую для работы с~контентом 
семантической биб\-лио\-те\-ки. LibMeta не является традиционной сис\-те\-мой 
управления электронными библиотеками (СУЭБ). 

Развитие современных 
технологий подталкивает к~переопределению как понятия биб\-лио\-те\-ки, так 
и~контента биб\-лио\-те\-ки, в~качестве которых не обязательно могут выступать 
традиционные описания печатных изданий, но и~любые другие типы 
циф\-ро\-вых объектов. При этом контент циф\-ро\-вых биб\-лио\-тек может быть 
описан различными форматами и~пред\-став\-лен различными способами. 
Биб\-лио\-те\-ка, реализуемая с~помощью LibMeta, рассматривается как 
хранилище структурированных разнообразных\ данных с~воз\-мож\-ностью их 
интеграции с~другими источниками данных и~предполагает возможность 
специфицирования своего контента путем описания предметной об\-ласти. 
     
     Определение предметной области задается тезаурусом~[4], который 
содержит основные термины этой предметной об\-ласти, связанные 
иерархическими и~горизонтальными связями между собой. Содержимое 
биб\-лио\-те\-ки задается типами ресурсов, описание которых задает,
в~свою очередь, множество 
допустимых объектов, возможно объединенных в~разнообразные коллекции, 
со\-став\-ля\-ющие вместе с~тезаурусом ее контент.
     
     Статья посвящена исследованию средств пред\-став\-ле\-ния знаний 
о~контенте семантической биб\-лио\-те\-ки. Эти средства необходимы для 
автоматизации описания ресурсов биб\-лио\-те\-ки конкретной предметной 
области и~воз\-мож\-ности их автоматизированной интеграции с~данными 
внеш\-них открытых источников. Необходимым условием для этого является 
структуризация и~формализация знаний в~об\-ласти описания контента 
семантической биб\-лио\-теки. 
     
     При реализации LibMeta авторы руководствовались набором основных 
задач, которые должна решать разрабатываемая система:
     \begin{enumerate}[(1)]
\item библиотека должна поддерживать возможность использования 
медийных объектов или ссылки на них при описании своих объектов, 
включая текст, аудио- и~видеофайлы или любую их комбинацию. Это 
требование отражается в~названии словом <<цифровая>>;\\[-10pt]
\item типы используемых ресурсов и~связи между ними должны быть 
описаны средствами сис\-те\-мы в~рамках определенных в~предыду\-щей работе 
понятий, составляющих семантическое описание ресурсов контента 
биб\-лио\-те\-ки. При этом, согласно принципам LOD, при описании ресурсов 
поддерживается использование классов и~свойств ранее используемых 
онтологий в~сообществе, поддерживающем LOD. Эта поддержка 
выражается либо в~непосредственном использовании готовых онтологий 
при описании ресурсов и~связей между ними, либо в~возможности ссылок 
на их элементы, используя связи на уровне описания ресурсов. Это 
требование отражается в~названии словом <<семантическая>>;\\[-10pt]
\item библиотека должна служить интеграционным узлом, предоставляя 
возможность связывания своих данных с~данными из разных источников, 
которые включены в~облако LOD. Должна также обеспечиваться 
возможность извлекать данные этой биб\-лио\-те\-ки в~машиночитаемом 
формате. Это требование отражается в~на\-зва\-нии словом <<открытая>>;\\[-10pt]
\item пользователи биб\-лио\-те\-ки должны иметь возможность организовывать 
свои коллекции по интересующему их научному на\-прав\-ле\-нию, добавляя 
новые термины в~предметный тезаурус, уточняя таким образом об\-ласть 
своих интересов. Пользователи должны также иметь возможность 
осуществлять поиск не только среди объектов в~рамках сис\-те\-мы, но и~по 
источникам данных, без необходимости использования 
специализированного языка для поисковых запросов. Это требование 
отражается в~названии словом <<персональная>>.
\end{enumerate}

     Основные требования, предъявляемые при этом к~контенту  
системы,~--- \textit{универсальность}, \textit{структурированность}, 
\textit{адаптируемость}~--- не противоречат этим свойствам и~обеспечивают 
поддержку настраиваемого хранилища метаданных для объектов 
и~расширяемый набор информационных ресурсов. \textit{Универсальность} 
обеспечивает описания типов ее ресурсов и~объектов независимо от 
предметной об\-ласти и~об\-ласти интересов пользователей. 
\textit{Структурированность} описания обеспечивает\linebreak\vspace*{-12pt}

\columnbreak

\noindent
 поддержку связей 
между различными типами ресурсов как внутри сис\-те\-мы, так и~вне нее, 
исходя из определений LOD. \textit{Адаптируемость} описания ресурсов 
обеспечивает возможность добавления новых свойств и~связей в~процессе 
развития сис\-те\-мы и~обеспечивает настройку пользовательских интерфейсов 
под эти изменения. 
{ %\looseness=1

}
     
     В качестве средства формализации выступает онтология~\cite{5-ser} 
контента семантической биб\-лио\-те\-ки. На основе этого описания можно 
выделить основные понятия описания задачи интеграции данных из 
открытых источников.
     
     В качестве открытых источников рас\-смат\-ри\-ва\-ют\-ся источники данных, 
включенные в~LOD~\cite{6-ser} и~соответствующие основным 
требованиям, предъявляемым к~таким источникам данных.
     
     В~качестве основных разделов освещаемой задачи рассмотрим 
определение тезауруса и~основные стандарты, выделим понятия, 
необходимые для описания контента семантической биб\-лио\-те\-ки 
в~произвольной предметной об\-ласти, определим основные понятия, 
необходимые для описания задачи интеграции данных из открытых 
источников и~выделим основные типы связей между этими понятиями.

\section{Тезаурус и~стандарты}

     Для описания какой-либо предметной области всегда используется 
определенный набор терминов, каждый из которых обозначает или 
описывает ка\-кую-ли\-бо концепцию из этой предметной об\-ласти. 
Совокупность терминов, описывающих предметную область с~указанием 
семантических отношений (связей) между ними, является тезаурусом. Такие 
отношения в~тезаурусе всегда указывают на наличие смысловой 
(семантической) связи между терминами.
     
     При этом модель тезауруса не должна быть ориентирована ни на одну 
из конкретных предметных областей и~быть достаточно гибкой для того, 
чтобы позволить всегда сохранять актуальность словаря и~удобство его 
использования для определения любой предметной об\-ласти.
     
     Тезаурус с~наличием связей различных типов позволяет реализовать 
гибкий настраиваемый поиск, результатом которого будет список объектов 
предметной об\-ласти, со\-от\-вет\-ст\-ву\-ющий выбранным терминам. 
     
     Рассматриваемая в~статье модель тезауруса соответствует стандарту 
ISO~2788-1986. Этот стандарт определяет тезаурус как набор терминов, 
связанных между собой со\-от\-вет\-ст\-ву\-ющи\-ми связями (отношениями). 

Термины могут иметь следующие атрибуты:
     \begin{itemize}
\item $\mathrm{SN}$~--- Scope Note. Комментарий к~термину. Например, представляет 
вербальное пояснение термина или правила его использования;
\item $\mathrm{TT}$~--- Top Term. Признак, выделяющий термины на самом верхнем 
уровне иерархии (термины наиболее общих понятий в~иерархии понятий).
\end{itemize}

     Связи между терминами могут быть сле\-ду\-ющими:
     \begin{itemize}
\item $\mathrm{USE}$~--- связывает термин с~наиболее предпочтительным термином 
для понятия. $A$~$\mathrm{USE}$~$B$ означает, что термин~$B$ является наиболее 
предпочтительным для понятия, обозначаемого термином~$A$;
\item $\mathrm{UF}$~--- Used For. Обращение связи $\mathrm{USE}$. Связывает наиболее 
подходящий термин с~синонимами и~квазисинонимами (менее 
подходящими терминами);
\item $\mathrm{BT}$~--- Broader Term. Связь термина с~термином более общего 
понятия. $A$~$\mathrm{BT}$~$B$ означает, что термин~$B$ обозначает более общее 
понятие по сравнению с~понятием, обозначаемым термином~$A$;
\item $\mathrm{BTG}$~--- Broader Term Generic. Вариант связи~$\mathrm{BT}$ в~случае, 
когда 
термин характеризует разно\-вид\-ность понятия, определяемого более 
общим термином. Например, <<попугаи>> и~<<птицы>>. Наличие связи 
$\mathrm{BTG}$ подразумевает наличие связи~$\mathrm{BT}$; 
\item $\mathrm{BTP}$~--- Broader Term Partitive. Вариант связи $\mathrm{BT}$ в~случае, когда 
термин характеризует часть понятия, определяемого более общим 
термином. Например, <<математика>> и~<<тео\-рия чисел>>. Наличие 
связи $\mathrm{BTP}$ подразумевает наличие связи~BT; 
\item $\mathrm{NT}$, $\mathrm{NTG}$ и~$\mathrm{NTP}$~--- Narrower Term, 
Narrower Term Generic и~Narrower 
Term Partitive~--- обращение связей $\mathrm{BT}$, $\mathrm{BTG}$ и~$\mathrm{BTP}$ 
со\-от\-вет\-ст\-венно; 
{\looseness=1

}
\item $\mathrm{RT}$~--- Related Term. Ассоциативная связь. Используется для 
семантически связанных между собою терминов, не находящихся при 
этом в~одной иерархии и~не являющихся синонимами или 
квазисинонимами. Эта связь проставляется в~тех случаях, когда 
пользователю тезауруса может быть полезно осуществлять поиск или 
индексацию не только по данному термину, но и~по связанному с~ним.
\end{itemize}

\section{Онтология}

     Исходя из вышесказанного, тезаурус~--- это \mbox{полный} 
сис\-те\-ма\-ти\-зи\-ро\-ван\-ный набор терминов о~ка\-кой-ли\-бо об\-ласти знаний 
и~больше относится к~лексике, используемой в~конкретной об\-ласти, тогда 
как онтология описывает ресурсы предметной об\-ласти и~их взаимосвязи. Для 
каждой предметной об\-ласти набор ресурсов может отличаться как по 
формату, так и~по набору самих ресурсов. Поэтому, задавая определение 
самой библиотеки, предлагается использовать для описания ресурсов, 
со\-став\-ля\-ющих контент конкретной предметной об\-ласти, понятия, общие для 
любой из них, т.\,е.\ набор понятий, формирующих описание контента 
биб\-лио\-те\-ки, должен быть настолько универсальным, чтобы мог 
адаптироваться под нужды конкретной об\-ласти. 

Так как одной из основных 
задач, решаемых в~рамках биб\-лио\-те\-ки, как было сказано выше, является 
интеграция данных из различных источников, такой подход поз\-во\-ля\-ет 
реализовать средства интеграции данных в~рамках биб\-лио\-те\-ки, 
адап\-ти\-ру\-емые под условия любой предметной об\-ласти без оглядки на ее 
специфику.
     
     Понятия, составляющие онтологию библиотеки LibMeta, условно 
делятся на предназначенные для:
     \begin{itemize}
\item описания контента предметной об\-ласти;
\item формирования тезауруса любой предметной об\-ласти;
\item описания тематических коллекций; 
\item описания задачи интеграции контента биб\-лио\-те\-ки с~данными 
источников из LOD.
\end{itemize}

     Между этими группами понятий определены семантически значимые 
связи.
     
     Рассмотрим далее основные формальные определения, необходимые 
для описания онтологии.
     
     \smallskip
     
     \noindent
\textbf{Определение~1.} \textit{Контент библиотеки} $C\hm=\langle \mathrm{IR}, A, 
\mathrm{IO}\rangle$ определяется типами ее информационных ресурсов, описанных 
связанными с~ними наборами атрибутов~$A$ и~набором входных данных, 
опре\-де\-ля\-ющих информационные объекты~$\mathrm{IO}$, которые являются 
непосредственно объектами, хранящимися в~биб\-лио\-теке.
\smallskip

\noindent
\textbf{Определение~2.} \textit{Тезаурус библиотеки} $\mathrm{TH}\hm=\langle T, 
R\rangle$ определяется терминами~$T$ и~связями~$R$ между ними. Набор 
терминов~$T$, со\-став\-ля\-ющих описание предметной об\-ласти, строго задан.

\smallskip

\noindent
\textbf{Определение~3.} \textit{Семантические метки} $M\hm=\left\{ 
m_i\right\}$ информационного объекта~--- это термины, которые не попали 
в~тезаурус, но являются необходимыми для специфицирования тематики 
информационного объекта. Семантические метки не связаны, в~отличие от 
терминов тезауруса, связями между собой или с~терминами тезауруса, но 
дают возможность дополнительного тематического разделения 
информационных объектов в~рамках предметной об\-ласти.

\smallskip

\noindent
\textbf{Определение~4.} \textit{Задача интеграции данных биб\-лио\-те\-ки} 
$\mathrm{IT}\hm = \langle \mathrm{DS}, R, A, M, D, D_S\rangle$ \textit{с~внешними 
источниками}~$\mathrm{DS}$ определяется типами ресурсов биб\-лио\-те\-ки и~набором 
их атрибутов~$A$, отображением~$M$ ресурсов~$R$ на схему источника 
данных~$S$ и~набором связей~$D_S$ с~данными из источника.

\smallskip

\noindent
\textbf{Определение~5.} \textit{Коллекция информационных объектов} 
$C\hm= \langle \mathrm{IO}, T, M, \mathrm{DS}\rangle$ представляет собой набор объектов, 
объединенных на основе совокупности признаков:
\begin{enumerate}[(1)]
\item по их термину тезауруса предметной об\-ласти; 
\item по семантическим меткам; 
\item по источнику данных, из которого поступили объекты.
\end{enumerate}

В коллекцию могут входить объекты различных типов ресурсов, заданных 
при описании контента биб\-лио\-те\-ки. При этом коллекции по каждому 
признаку могут формироваться автоматически и~будем называть их 
автоматическими коллекциями. В~случае, когда признаки определяет 
пользователь, будем называть такие коллекции просто \textit{коллекциями.}

\smallskip

\noindent
\textbf{Определение~6.} \textit{Семантически значимыми связями 
библиотеки} $P\hm=\left\{P_i\right\}$ назовем связи, определенные между 
контентом библиотеки, ее предметной об\-ластью (тезаурусом), 
семантическими метками и~объектами источника данных. Выделим 
сле\-ду\-ющие основные связи:
\begin{itemize}
\item $P_1(t, \mathrm{io})$~--- термин те\-за\-у\-ру\-са\,--\,ин\-фор\-ма\-ци\-он\-ный 
объект;
\item $P_2(\mathrm{io}, t)$~--- информационный объ\-ект\,--\,тер\-мин тезауруса;
\item $P_3(r, s)$~--- информационный ре\-сурс\,--\,класс объектов 
источника, где информационный ресурс~--- это общее определение для 
информационных объектов, хранящихся в~сис\-те\-ме; таким образом, 
фактически информационные объекты являются экземплярами 
информационных ресурсов;
\item $P_4(a, s_a)$~--- атрибут информационного ре\-сур\-са\,--\,свой\-ст\-во 
класса источника;
\item $P_5(\mathrm{io}, o_s)$~--- информационный объ\-ект\,--\,эк\-земп\-ляр класса из 
источника данных;
\item $P_6(m, \mathrm{io})$~--- семантическая мет\-ка\,--\,ин\-фор\-ма\-ци\-он\-ный 
объект;
\item $P_7(\mathrm{io}, m)$~--- информационный объ\-ект\,--\,се\-ман\-ти\-че\-ская 
метка.
\end{itemize}

На основе введенных явных связей можно определить связи, которые 
назовем \textit{неявными значимыми связями} (т.\,е.\ заданными по 
некоторым определенным заранее правилам) между семантическими 
метками и~терминами тезауруса и~объектами как самой биб\-лио\-те\-ки, так 
и~экземплярами связанных данных из источников: 
\begin{itemize}
\item $P_8(m, t) \leftarrow P_6(m, \mathrm{io}) \wedge P_2(\mathrm{io}, t)$ семантическая 
мет\-ка\,--\,ин\-фор\-ма\-ци\-он\-ный объект\,--\,тер\-мин тезауруса;
\item $P_9(t, m) \leftarrow P_1(t, \mathrm{io}) \wedge P_7(\mathrm{io}, m)$ термин  
те\-за\-у\-ру\-са\,--\,ин\-фор\-ма\-ци\-он\-ный  
объ\-ект\,--\,се\-ман\-ти\-че\-ская метка;
\item $P_{10}(m, o_s) \leftarrow P_6(m, \mathrm{io}) \wedge P_5(\mathrm{io}, o_s)$ семантическая  
мет\-ка\,--\,ин\-фор\-ма\-ци\-он\-ный объект\,--\,эк\-земп\-ляр класса из 
источника данных;
\item $P_{11}(t, o_s) \leftarrow P_1(t, \mathrm{io}) \wedge P_5(\mathrm{io}, o_s)$ термин  
те\-за\-у\-ру\-са\,--\,ин\-фор\-ма\-ци\-он\-ный объект~--- экземпляр класса из 
источника данных.
\end{itemize}

     Для представления онтологии LibMeta был выбран язык описания 
онтологий OWL (Web Ontology Language)\footnote{{\sf https://www.w3.org/TR/owl-ref}.}. Такая онтология 
со\-сто\-ит из классов, свойств классов и~индивидов. В~терминах 
OWL~$P_1$~\textit{инверсивно}~$P_2$, $P_6$~\textit{инверсивно}~$P_7$, 
$P_8$~\textit{инверсивно}~$P_9$ и~$P_{10}$ \textit{инверсивно}~$P_{11}$. При этом 
правила для неявных связей задаются с~по\-мощью правил SWRL
(Semantic Web Rule Language)\footnote{ {\sf 
https://www.w3.org/Submission/SWRL}.}. Правила SWRL как расширение 
OWL помогают описать 
абстрактный механизм оперирования объектами предметной области и~ее 
закономерности. Правила SWRL дают возможность выводить новые факты из 
существующих утверж\-де\-ний, что повышает эффективность описания 
предметной об\-ласти.
     
     В соответствии с~определениями были введены основные классы 
онтологии. Исходя из определения~1, вводятся классы:
     \begin{enumerate}[1.]
\item $\mathrm{IResource}$ (информационный ресурс биб\-лио\-те\-ки), который 
содержит общую информацию о~типе ресурса, название, 
$URI$ (Universal Resource Identifier)\footnote{{\sf https://tools.ietf.org/html/rfc3986}.} и~информацию об 
ис\-поль\-зу\-емом наборе атрибутов для описания структуры ресурса.
\item $\mathrm{IObject}$ (информационный объект библиотеки), который 
фактически пред\-став\-ля\-ет собой экземпляр некоторого ресурса и~по 
составу ат-\linebreak рибутов соответствует набору атрибутов связанного с~ним 
ресурса. Для описания со\-от\-вет\-ст\-ву\-ющих значений для информационного 
объекта имеется многозначное свойство $\mathrm{value}$, значениями 
которого являются экземпляры вспомогательного класса $\mathrm{AttributeValue}$, 
содержащие информацию о конкретном значении объекта и~соответствующем атрибуте.
\item $\mathrm{Attribute}$ (атрибут, элемент описания информационного ресурса), 
который имеет следующие свойства:
\begin{itemize}
\item[(а)] $\mathrm{name}$~--- название;
\item[(б)] $\mathrm{type}$~--- содержит информацию о типе значений 
этого атрибута и~может включать такие значения, как 
\textit{строка}, \textit{число}, \textit{дата}, \textit{тип ресурса} 
(т.\,е.\ значениями являются объекты некоторого выбранного типа 
ресурса);
\item[(в)] $\mathrm{view}$~--- указывает на об\-ласть 
применения ат-\linebreak рибута в~рамках системы. Может иметь зна-\linebreak чения 
\textit{поисковый} 
(участвует в~формировании поисковых форм), 
\textit{иден\-ти\-фи\-ци\-ру\-ющий} (является обязательным) 
и~\textit{описательный} (содержит дополнительную информацию 
об описываемом объекте).
\end{itemize}
\item $\mathrm{AttributeSet}$ (набор атрибутов, группирующий атрибуты, 
со\-от\-вет\-ст\-ву\-ющие одному пред\-став\-ле\-нию ресурса).
\end{enumerate}

\begin{figure*}[b] %fig1
\vspace*{1pt}
 \begin{center}
 \mbox{%
 \epsfxsize=105.266mm 
 \epsfbox{ata-1.eps}
 }
 \end{center}
\vspace*{-9pt}
\caption{Пример описания информационного ресурса в~терминах онтологии LibMeta}
\end{figure*}

Исходя из определения~2, согласно описанному ранее стандарту  
ISO~2788-1986 для тезаурусов, вводятся классы:
\begin{enumerate}[1.]
\setcounter{enumi}{4}
\item $\mathrm{Thesaurus}$ (тезаурус предметной области)~--- содержит в~себе общую 
информацию о~тезаурусе: название и~авторов (организации и~персоны). 
Наличие этой сущности позволяет загружать готовые тезаурусы, не 
смешивая их с~теми, что уже, быть может, есть в~сис\-теме.
\item $\mathrm{Concept}$~--- сущность, содержащая информацию о~понятиях 
тезауруса. Содержит следующие атрибуты:
\begin{itemize}
\item[(a)] $\mathrm{Name}$~--- название понятия. В~случае, если понятие 
не может иметь названия, пред\-став\-лен\-но\-го в~виде текста, 
используется ка\-кой-ли\-бо идентификатор;
\item[(б)] $\mathrm{RepresentationType}$~--- тип представления понятия. 
Понятие не всегда можно описать словами, иногда для этого 
гораздо больше подходит формула или изображение, поэтому 
необходимо иметь возможность добавления понятия в~любом виде;
\item[(в)] $\mathrm{Image}$~--- изображение;
\item[(г)] $\mathrm{Note}$~--- примечание.
\end{itemize}
\item $\mathrm{ConceptGroup}$~--- тематическое разделение понятий тезауруса. 
\item $\mathrm{HierarchicalRel}$~--- иерархические связи, определяющие 
древовидную структуру словаря. Содержит атрибуты, определяющие 
связи в~соответствии со стандартом ($\mathrm{BT}$, $\mathrm{BTG}$, $\mathrm{BTP}$).
\item $\mathrm{FamilyRel}$~--- горизонтальные связи. Они задают родственные 
отношения между понятиями и~позволяют находить публикации по 
похожим тематикам. Содержит также атрибуты, определяющие связи 
в~соответствии со стандартом ($\mathrm{NT}$, $\mathrm{NTG}$, $\mathrm{NTP}$).
\item $\mathrm{PrefferedTerm}$~--- дескрипторы понятия. Каж\-до\-му понятию 
соответствует единственный дескриптор на каждом языке. 
\item $\mathrm{NonPrefferedTerm}$~--- сюда включаются синонимы. Один 
дескриптор может иметь множество синонимов. В~этот класс объектов 
добавлен атрибут $\mathrm{Visibility}$~--- свойство, от\-ве\-ча\-ющее за видимость 
термина. Имеет два значения~--- $\mathrm{global}$ и~$\mathrm{private}$, 
глобальная и~приватная об\-ласти видимости соответственно. Этот атрибут 
введен для решения проблемы множественных терминологий~--- разные 
люди могут называть одни и~те же объекты по-раз\-но\-му (пусть даже эти 
названия будут похожи). Для того чтобы каж\-до\-му пользователю было 
комфортно работать в~сис\-те\-ме, ему дается возможность создавать свои 
термины, если таковых нет в~глобальной части тезауруса. Эти термины он 
может связывать с~другими терминами из глобальной части и~размечать 
ими свои публикации. Таким образом, если два пользователя создали 
в~своих локальных репозиториях удобные для них ключевые слова, 
разметили ими свои публикации и~связали эти ключевые слова с~одним 
и~тем же термином из глобального тезауруса, то они смогут находить 
и~получать публикации друг друга, пользуясь при этом своими 
терминологиями.
\item $\mathrm{Term}$~--- общий класс, объединяющий дескрипторы и~синонимы. 
Содержит набор свойств, который при необходимости позволяет 
произвольно расширять текстовые описания терминов и~определять связи 
с~информационными объектами системы.
\end{enumerate}

Исходя из определений~3 и~5, вводятся классы:
\begin{enumerate}[1.]
\setcounter{enumi}{12}
\item  $\mathrm{SemanticTag}$~--- класс семантических меток, который обладает 
следующими свойствами:
\begin{itemize}
\item[(а)] $\mathrm{title}$~--- краткое название семантической метки;
\item[(б)] $\mathrm{description}$~--- расширенное описание 
семантической метки.
\end{itemize}
\item $\mathrm{ICollection}$~--- класс коллекций, определенных человеком, 
который обладает следующими свойствами:
\begin{itemize}
\item[(а)] $\mathrm{name}$~--- название коллекции;
\item[(б)] $\mathrm{definition}$~--- описание коллекции;
\item[(в)] $\mathrm{resources}$~--- типы ресурсов, включаемых в~эту 
коллекцию.
\end{itemize}
\end{enumerate}

Исходя из определения 4 вводятся классы:
\begin{enumerate}[1.]
\setcounter{enumi}{14}
\item $\mathrm{DataSource}$ (источники данных LOD)~--- класс, который имеет 
следующие свойства:
\begin{itemize}
\item[(а)]  $\mathrm{name}$~--- название источника;
\item[(б)] $\mathrm{description}$~--- описание источника;
\item[(в)] $\mathrm{url}$~--- точка входа для извлечения данных;
\item[(г)] $\mathrm{resourceMapping}$~--- содержит информацию о~типах 
ресурсов, отображаемых на этот источник, и~со\-от\-вет\-ст\-ву\-ющие классы 
источника. Значениями являются экземпляры класса 
$\mathrm{ResourceMapping}$.
\end{itemize}
\item $\mathrm{ResourceMapping}$~--- класс, содержащий информацию об 
отображаемых на источник данных информационных ресурсах 
библиотеки:
\begin{itemize}
\item[(а)] $\mathrm{resource}$~--- тип ресурсов, отображаемых на этот 
источник;
\item[(б)] $\mathrm{class}$~--- ссылка на соответствующий класс 
источника данных;
\item[(в)] $\mathrm{attributeMappings}$~--- содержит экземпляры класса 
$\mathrm{AttributeMapping}$, содержащих информацию об отображении 
со\-от\-вет\-ст\-ву\-ющих ресурсу атрибутов.
\end{itemize}
\item $\mathrm{AttributeMapping}$~--- класс, содержащий информацию об 
отображаемых на источник данных атрибутах из набора атрибутов, 
со\-от\-вет\-ст\-ву\-юще\-го информационному ресурсу библиотеки:
\begin{itemize}
\item[(а)] $\mathrm{attribute}$~--- атрибут, отображаемый на этот 
источник;
\item[(б)] $\mathrm{property}$~--- ссылка на соответствующее свойство 
класса источника данных.
\end{itemize}
\end{enumerate}

     На рис.~1 и~2 приведены примеры описания конкретного 
информационного ресурса и~информационного объекта в~терминах этой 
онтологии согласно определению~1. 



\begin{figure*} %fig2
\vspace*{1pt}
 \begin{center}
 \mbox{%
 \epsfxsize=155.106mm 
 \epsfbox{ata-2.eps}
 }
 \end{center}
\vspace*{-9pt}
\Caption{Пример описания информационного объекта в~терминах онтологии LibMeta}
\vspace*{6pt}
\end{figure*}

\section{Использование онтологии контента библиотеки и~тезауруса
предметной области при~конструировании семантической библиотеки 
в~LibMeta}

     Для применения тезауруса конкретной предметной 
об\-ласти и~онтологии контента биб\-лио\-те\-ки необходимо придерживаться 
сле\-ду\-ющей последовательности их использования при конструировании 
семантической биб\-лио\-те\-ки в~рамках LibMeta:
     \begin{enumerate}[(1)]
\item на основе введенной модели задается набор информационных 
ресурсов, ис\-поль\-зу\-емых в~биб\-лио\-те\-ке. Для этого необходимо пред\-ста\-вить 
описания содержимого будущей биб\-лио\-те\-ки в~терминах предложенной 
модели. На базе классов, заданных для описания контента биб\-лио\-те\-ки, 
реализован модуль, в~котором задаются базовые свойства, атрибуты для 
ресурсов и~связи между ними;
\item осуществляется окончательная настройка структуры тезауруса. На 
базе определенных классов согласно определению тезауруса реализован 
модуль для его по\-стро\-ения, в~котором задаются используемые связи 
между терминами, расширяется при необходимости описание термина, 
определяются связи с~ресурсами сис\-темы; 
\item для выбора семантических меток можно использовать 
дополнительные словари по предметной области или оставить 
возможность их определения (доопределения) позднее; 
\item на основе заданных классов согласно определению задачи 
интеграции реализован модуль, в~рамках которого осуществляется 
подключение внешних источников данных. Это действие можно 
выполнить на любом этапе жизнедеятельности системы;
\item на основе заданных классов согласно определению коллекций 
реализован модуль, в~рамках которого осуществляются создание 
коллекций и~их наполнение; это можно выполнить также на 
любом этапе.
\end{enumerate}

     На основе выполненных действий происходит автоматическая 
адаптация пользовательских интер\-фей\-сов системы под заданные описания 
ресурсов, со\-став\-ля\-ющих содержимое биб\-лио\-те\-ки. Пользовательский 
интерфейс делится услов\-но на сле\-ду\-ющие категории:
     \begin{itemize}
\item интерфейсы поиска;\\[-13.5pt]
\item интерфейсы просмотра;\\[-13.5pt]
\item интерфейсы редактирования;\\[-13.5pt]
\item интерфейсы загрузки данных.
\end{itemize}

\vspace*{-8pt}

     \subsection*{Пример}
     
     \vspace*{-1pt}
     
     На основе предложенной модели была сконструирована 
библиотека для предметной об\-ласти обыкновенных дифференциальных 
урав\-нений (ОДУ). В~качестве тезауруса использован тезаурус ОДУ, 
разработанный коллективом специалистов в~этой области~\cite{7-ser}. 
     
     Объектами библиотеки рассматривались журнальные математические 
статьи. В~качестве примеров типов ресурсов, соответственно, 
рассматривались \textit{Авторы} и~\textit{Публикации}. Был определен набор 
атрибутов для каждого типа ресурсов в~рамках минимального набора свойств 
на основе Dublin Core\footnote{{\sf  http://dublincore.org.}} для публикаций 
и~FOAF (Friend of a~Friend)\footnote{{\sf http://xmlns.com/foaf/spec.}} для описания авторов. В~качестве 
примера источника данных рассматривались данные о~персонах из сис\-те\-мы 
MathNet\footnote{{\sf http://www.mathnet.ru}.}, которые были смоделированы в~виде 
источника, интегрированного в~LOD. Были определены отобра\-же\-ния 
атрибутов \textit{Авторов} на свойства персон из этого источника 
и~выявлены связи у~почти~50\% авторов рас\-смат\-ри\-ва\-емых пуб\-ли\-ка\-ций, при 
этом авторов было около~700.
     Средствами системы для каждой публикации на основе ее названия, 
аннотации и~ключевых слов были выявлены связи с~тезаурусом ОДУ. 
В~качестве семантических меток были использованы термины 
математической энциклопедии\footnote{{\sf 
https://www.encyclopediaofmath.org.}}~\cite{8-ser}, что позволило дополнительно 
выявить смежные предметные области и~произвести дополнительное 
тематическое разбиение пуб\-ли\-ка\-ций в~рамках предметной области. Такое 
связывание позволило выявить с~некоторой долей вероятности статьи, 
относящиеся к~предметной области ОДУ, и~организовать их в~коллекции на 
основе тезауруса и~выявленных семантических меток. Было использовано 
описание около~2000~пуб\-ли\-ка\-ций, из них около~30\% были отнесены 
к~об\-ласти ОДУ и~имели связи со смежными предметными областями, 
выявленными согласно семантическим меткам.

\vspace*{-10pt}
     
\section{Дальнейшее направление работ}

\vspace*{-3pt}

Работа с~полными текстами предоставленных статей пока находится 
в~активной стадии. Предполагается создание информационного образа 
статей для выделения мик\-ро\-те\-зау\-ру\-са на основе семантических меток 
и~терминов предметной области по каж\-дой статье с~дальнейшим 
определением возможностей расширения используемых тезаурусов или для 
создания облака ключевых понятий отдельных областей знания. 

Отдельной 
задачей является семантическая обработка формул из полных текстов 
и~определение их ключевых слов с~воз\-мож\-ностью дальнейшего поиска по 
формулам, а~также выделение отдельных направлений и~математических школ. При 
этом формулы рас\-смат\-ри\-ва\-ют\-ся как отдельный тип ресурсов сис\-темы.

\vspace*{-6pt}
     
{\small\frenchspacing
 {%\baselineskip=10.8pt
 \addcontentsline{toc}{section}{References}
 \begin{thebibliography}{9}
 
 \vspace*{-2pt}

\bibitem{2-ser} %1
\Au{Серебряков В.\,А., Атаева~О.\,М.} Персональная цифровая библиотека LibMeta как 
среда интеграции связанных открытых данных~// Электронные библиотеки: 
перспективные методы и~технологии, электронные коллекции: Тр. XVI Всеросс. науч. 
конф. RCDL'2014.~--- Дубна: ОИЯИ, 2014. С.~66--71.
\bibitem{1-ser} %2
\Au{Серебряков~В.\,А., Атаева О.\,М.} Основные понятия формальной 
модели семантических биб\-лио\-тек и~формализация процессов интеграции в~ней~// 
Программные продукты и~сис\-те\-мы, 2015. №\,4. С.~180--187.

\bibitem{3-ser}
\Au{Серебряков В.\,А., Атаева~О.\,М.} Информационная модель открытой персональной 
семантической библиотеки LibMeta~// Научный сервис в~сети Интернет: Тр. XVIII 
Всеросс. науч. конф.~--- М.: ИПМ им.\ М.\,В.~Келдыша, 2016. С.~304--313.
\bibitem{4-ser}
\Au{Нгуен М.\,Х., Аджиев~А.\,С.} Описание и~использование тезаурусов 
в~информационных системах, подходы и~реализация~// Электронные библиотеки, 2004. 
Т.~7. №\,1. С.~16--45.
\bibitem{5-ser}
\Au{Gruber T.\,R.} A~translation approach to portable ontologies~// Knowl. Acquis., 
1993. Vol.~5. No.\,2. P.~199--220.
\bibitem{6-ser}
\Au{Bizer C., Heath~T., Berners-Lee~T.} Linked data~--- the story so far~// Int. J.~Semantic 
Web Inf., 2009. Vol.~5. No.\,3. P.~1--22.
\bibitem{7-ser}
\Au{Моисеев~Е.\,И., Муромский~А.\,А., Тучкова~Н.\,П.} Тезаурус  
ин\-фор\-ма\-ци\-он\-но-по\-иско\-вый по предметной области <<обыкновенные 
дифференциальные уравнения>>.~--- М.: МАКС Пресс, 2005. 116~с.
\bibitem{8-ser}
Математическая энциклопедия: В~5~т.~/ Гл.\ ред. И.\,М.~Виноградов.~--- М.: Советская 
энциклопедия, 1977.
 \end{thebibliography}

 }
 }

\end{multicols}

\vspace*{-6pt}

\hfill{\small\textit{Поступила в~редакцию 03.05.17}}

%\vspace*{8pt}

\newpage

\vspace*{-28pt}

%\hrule

%\vspace*{2pt}

%\hrule

%\vspace*{8pt}


\def\tit{ONTOLOGY OF~THE~DIGITAL SEMANTIC LIBRARY LibMeta}

\def\titkol{Ontology of~the~digital semantic library LibMeta}

\def\aut{V.\,A.~Serebryakov and O.\,M.~Ataeva}

\def\autkol{V.\,A.~Serebryakov and O.\,M.~Ataeva}

\titel{\tit}{\aut}{\autkol}{\titkol}

\vspace*{-9pt}


\noindent
A.\,A.~Dorodnicyn Computing Center, Federal Research Center ``Computer Science and 
Control'' of the Russian Academy of Sciences,  40~Vavilov Str., Moscow 119333, Russian 
Federation 



\def\leftfootline{\small{\textbf{\thepage}
\hfill INFORMATIKA I EE PRIMENENIYA~--- INFORMATICS AND
APPLICATIONS\ \ \ 2018\ \ \ volume~12\ \ \ issue\ 1}
}%
 \def\rightfootline{\small{INFORMATIKA I EE PRIMENENIYA~---
INFORMATICS AND APPLICATIONS\ \ \ 2018\ \ \ volume~12\ \ \ issue\ 1
\hfill \textbf{\thepage}}}

\vspace*{3pt}



\Abste{During development of digital libraries, рarticular attention is paid to the 
library content data model. In this case, the content of digital libraries can be 
described in various formats and presented in various ways. The library defined by 
the LibMeta system is considered as a storehouse of structured diverse data with 
the possibility of their integration with other data sources and assumes the 
possibility of specifying its content by describing the subject area. The ontology of 
the semantic library content serves as a means of formalization. It also introduces 
the basic concepts for describing the task of data integration from sources of 
Linked Open Data (LOD), concepts for defining an arbitrary thesaurus. The 
ontology is constructed in such a~way that it is possible to determine the semantic 
library in an arbitrary domain.}

\KWE{semantic library; data model; ontology; data source; search in LOD}

  \DOI{10.14357/19922264180101} 

%\vspace*{-12pt}

%\Ack
%\noindent



%\vspace*{3pt}

  \begin{multicols}{2}

\renewcommand{\bibname}{\protect\rmfamily References}
%\renewcommand{\bibname}{\large\protect\rm References}

{\small\frenchspacing
 {%\baselineskip=10.8pt
 \addcontentsline{toc}{section}{References}
 \begin{thebibliography}{9} 

\bibitem{2-ser-1}
\Aue{Serebryakov, V.\,A., and O.\,M.~Ataeva.} 2014. Personal'naya tsifrovaya 
biblioteka LibMeta kak sreda integratsii svyazannykh otkrytykh dannykh [Personal 
Digital Library Libmeta as an integration environment of linked data]. \textit{Tr. XVI 
Vseross. nauch. konf. RCDL'2014}
[16th All-Russia Scientific Conference RCDL'2014 Proceedings]. Dubna: OIYI. 66--71.
\bibitem{1-ser-1}
\Aue{Serebryakov, V.\,A., and O.\,M.~Ataeva.} 2015. Osnovnye ponyatiya  
formal'noy modeli se\-man\-ti\-che\-skikh bib\-lio\-tek i~formalizatsiya protsessov 
integratsii v~ney [The basic concepts of a~formal model of semantic libraries 
and formalization of the integration processes in it]. \textit{Programmnye produkty 
i~sistemy} [Software Systems] 4:180--187.

\bibitem{3-ser-1}
\Aue{Serebryakov, V.\,A., and O.\,M.~Ataeva.} 2016. Informatsionnaya model' 
otkrytoy personal'noy semanticheskoy biblioteki LibMeta
[Information model of the open
personal semantic library LibMeta]. 
\textit{Nauchnyy servis v~seti Internet: Tr.\ XVIII Vseross. nauch. konf.}   
[Scietifical service in the Internet: 18th All-Russia Scientific Conference Proceedings]. 
Moscow: IPM.  304--313.
\bibitem{4-ser-1}
\Aue{Nguen, M.\,H., and A.\,S.~Adzhiev.} 2004. Opisanie i~ispol'zovanie tezaurusov 
v~informatsionnykh sistemakh, podkhody i~realizatsiya [Description and use of thesauri 
in information systems, approaches and implementation].\linebreak
 \textit{Elektronnye biblioteki} 
[Digital Library] 7(1):16--45.
\bibitem{5-ser-1}
\Aue{Gruber, T.\,R.} 1993. A~translation approach to portable ontologies. 
\textit{Knowl. Acquis.} 5(2):199--220.
\bibitem{6-ser-1}
\Aue{Bizer, C., T.~Heath, and T.~Berners-Lee.} 2009. Linked data~--- the story so far. 
\textit{Int. J.~Semantic Web Inf.} 5(3):1--22.
\bibitem{7-ser-1}
\Aue{Moiseev, E.\,I., A.\,A.~Muromskiy, and N.\,P.~Tuchkova.} 2005. \textit{Tezaurus 
informatsionno-poiskovyy po predmetnoy oblasti ``obyknovennye differentsial'nye 
uravneniya''} [Information search with thesaurus in application area of ordinary 
differential equations].  Moscow: MAKS Press. 116~p.
\bibitem{8-ser-1}
Vinogradov, I.\,M., ed. 1977.
\textit{Matematicheskaya enciklopediya: V~5~t.} [Mathematical encyclopedia: In 5~vols.]. 
Moscow: Sovetskaya Entsiklopediya.

\end{thebibliography}

 }
 }

\end{multicols}

\vspace*{-6pt}

\hfill{\small\textit{Received May 3, 2017}}

%\vspace*{-10pt}

\Contr

\noindent
\textbf{Ataeva Olga M.} (b.\ 1978)~--- junior scientist, A.\,A.~Dorodnicyn 
Computing Centre, Federal Research Center 
``Computer Science and Control'' of the Russian Academy of Sciences, 
40~Vavilov Str., Moscow 119333, Russian Federation; \mbox{oli@ultimeta.ru}

\vspace*{3pt}


\noindent
\textbf{Serebryakov Vladimir A.} (b.\ 1946)~--- Doctor of Science in physics 
and mathematics, professor, Head of Department, A.\,A.~Dorodnicyn Computing 
Centre, Federal Research Center ``Computer Science and Control'' of the Russian 
Academy of Sciences, 40~Vavilov Str., Moscow 119333, Russian Federation; 
\mbox{serebr@ultimeta.ru}


\label{end\stat}


\renewcommand{\bibname}{\protect\rm Литература}  %1
\def\stat{arkhipov}

\def\tit{ВАРИАНТ СОЗДАНИЯ ЛОКАЛЬНОЙ СИСТЕМЫ КООРДИНАТ 
ДЛЯ~СИНХРОНИЗАЦИИ ИЗОБРАЖЕНИЙ ВЫБРАННЫХ СНИМКОВ}

\def\titkol{Вариант создания локальной системы координат 
для~синхронизации изображений выбранных снимков}

\def\aut{О.\,П.~Архипов$^1$, П.\,О.~Архипов$^2$, И.\,И.~Сидоркин$^3$}

\def\autkol{О.\,П.~Архипов, П.\,О.~Архипов, И.\,И.~Сидоркин}

\titel{\tit}{\aut}{\autkol}{\titkol}

\index{Архипов О.\,П.}
\index{Архипов П.\,О.}
\index{Сидоркин И.\,И.}
\index{Arkhipov O.\,P.}
\index{Arkhipov P.\,O.}
\index{Sidorkin I.\,I.}


%{\renewcommand{\thefootnote}{\fnsymbol{footnote}} \footnotetext[1]
%{Работа выполнена при частичной поддержке РФФИ (проект 16-07-00272 А).}}


\renewcommand{\thefootnote}{\arabic{footnote}}
\footnotetext[1]{Орловский филиал Федерального исследовательского центра <<Информатика и~управление>> 
Российской академии наук, \mbox{arkhipov12@yandex.ru}}
\footnotetext[2]{Орловский филиал Федерального исследовательского центра <<Информатика и~управление>> 
Российской академии наук, \mbox{arpaul@mail.ru}}
\footnotetext[3]{Орловский филиал Федерального исследовательского центра <<Информатика и~управление>> 
Российской академии наук, \mbox{voronecburgsiti@mail.ru}}

\Abst{Рассмотрены проблемы сравнения пар изображений, имеющих 
искажения поворота и~сдвига сцен друг относительно друга. Разработан 
алгоритм создания локальной системы координат (ЛСК) для пар сравниваемых 
изображений.}

\KW{алгоритм; методика; локальная система координат; цветное 
изображение; синхронизация; пиксель; цветное пятно; фильтрация}

\DOI{10.14357/19922264160312} 


\vskip 12pt plus 9pt minus 6pt

\thispagestyle{headings}

\begin{multicols}{2}

\label{st\stat}

  \section{Введение}
  
  Важным этапом обработки кадров видеопотока является построение 
ЛСК для синхронизации обрабатываемых 
изображений. Под синхронизацией понимается процедура совмещения пары 
обрабатываемых кадров путем смещения одного изображения относительно 
другого для достижения совпадения одинаковых устойчивых робастных 
структур. В качестве общих робастных структур могут выступать границы 
объектов, имеющихся на полутоновых изображениях, и~центры одинаковых 
по площади цветных пятен соответствующих цветных изображений. 
В~случае необходимости сравнения пары кадров, полученных с~различных 
точек съемки либо с~отличным углом съемки, в~результате чего изображения 
оказались смещены относительно друг друга, синхронизация может стать 
единственно возможным решением для осуществления возможности 
машинного сравнения изображений. 

В~данной статье описывается процесс 
создания ЛСК для синхронизации пар 
обрабатываемых изоб\-ра\-же\-ний. Актуальность работы обусловлена 
необходимостью сравнения пар изоб\-ра\-же\-ний, которые были получены 
с~разных точек съемки, что привело к~искажениям поворота и~смещения. 

Целью данной работы является разработка варианта создания 
ЛСК для синхронизации пар изображений выбранных 
снимков. Основная идея работы состоит в~том, что для синхронизации двух 
изображений необходимо отыскать на этих изображениях робастные 
структуры, которые повторялись бы на каждом из этих изображений, а~затем 
выполнить создание ЛСК с~сохранением лишь 
общей части обрабатываемой пары изображений. Предполагается, что даже 
будучи смещенными друг относительно друга и/или повернутыми на 
произвольный угол, данные изображения, имеющие общую сов\-па\-да\-ющую 
часть, могут быть синхронизированы путем создания ЛСК
 и~преобразованием одного из изображений. Независимо от угла 
поворота и~смещения изображений, имеющих общую часть, робастные 
структуры данных изображений будут сов\-падать. 

\vspace*{-6pt}

  \section{Обзор аналогов}
  
  \vspace*{-2pt}
  
  Одними из наиболее распространенных методов определения 
геометрического рассогласования изображений являются корреляционные 
методы~[1, 2]. Данные методы позволяют рассчитать коэффициент 
корреляции для всех возможных вариантов смещения изображений друг 
относительно друга и~выбрать одно пиковое значение, которое будет 
соответствовать наибольшему совпадению двух сравниваемых изображений. 
Еще одним примером определения взаимного сдвига изображений являются 
статические методы, в~основе которых лежит процесс вычисления 
евклидовой меры взаимного рассогласования изображений~[3]. Однако 
данные методы являются весьма чувствительными к~шумам на 
изображениях, которые являются их неотъем-\linebreak\vspace*{-12pt}

\pagebreak

\noindent
лемой частью, и,~что более 
существенно, они не позволяют выполнить согласование изображений, 
имеющих искажение поворота. 
  
  Так как при съемке изображений нестационарной камерой получаемые 
изображения имеют именно искажения сдвига и~поворота, то пе\-ре\-чис\-лен\-ные 
выше методы не могут быть использованы для синхронизации таких 
изображений. В~данной статье предлагается метод, основанный на 
выявлении робастных характеристик, имеющих сходство на обоих 
обрабатываемых изображениях, который позволит выполнять 
синхронизацию изображений, подвергнутых искажениям сдвига и~поворота. 

\vspace*{-6pt}

  \section{Создание локальной системы координат 
для~синхронизации изображений выбранных снимков}

\vspace*{-2pt}
  
  Цветное изображение представляется в~виде двумерной 
последовательности пикселей вида
  \begin{multline*}
  \mathrm{Image}_i = \!\left\{\!
  \begin{matrix}
  p_{i,1,1}(\mathrm{R,G,B}), & p_{i,1,2}(\mathrm{R,G,B}), &\ldots\\ 
  \ldots  &\ldots  &\ldots\\
  p_{i,h,1}(\mathrm{R,G,B}), & p_{i,h,2}(\mathrm{R,G,B}), &\ldots\end{matrix}\right.\\ 
\hspace*{40mm}\left.\begin{matrix}\ldots, &p_{i,1,w}(\mathrm{R,G,B})\\
\ldots &\ldots\\
\ldots, &p_{i,h,w}(\mathrm{R,G,B})
  \end{matrix}\!
  \right\}\!\!,\hspace*{-0.7966pt}\\
   i\in [1, 2]\,,\enskip
  w\in [1, W_i]\,,\enskip h\in [1, H_i]\,,
%  \label{e1-ar}
  \end{multline*}
где Image$_i$~--- изображение снимка~$i$;
$p$~--- пиксели с~цветовыми координатами (R, G, B);
$W_i$ и~$H_i$~--- ширина и~высота изображения снимка~$i$ в~пикселях. 

  Для сравнения цветных изображений предлагается использовать цветные 
пятна изображений и~робастные характеристики этих изображений. Для 
этого необходимо выполнить процедуру получения полутоновых 
изображений для получения наборов робастных характеристик каждого из 
обра\-ба\-ты\-ва\-емых изображений вида

\noindent
  \begin{multline*}
  \mathrm{Im}_i ={}\\
  {}=Q\left(\varphi_{a,1}(\mathrm{Image}_i),   
\varphi_{a,2 }(\mathrm{Image}_i), 
\varphi_{a,3}(\mathrm{Image}_i)\right)\,,\\
  i\in [1, 2]\,,
%  \label{e2-ar}
  \end{multline*}
где Im$_i$~--- полутоновое изоб\-ра\-же\-ние снимка~$i$;
$Q$~--- функция объединения полутоновых преобразований;
$\varphi$~--- функция выполнения полутоновых преобразований~\cite{4-ar}.
  
  Перед выполнением сегментации цветных изоб\-ра\-же\-ний необходимо 
выполнить огрубление цветовых составляющих изображений до~256~цветов, 
что позволит получить более удобные для сегментации
изображения 
с~четким контрастированием цвето-\linebreak\vspace*{-12pt}

\columnbreak

\noindent вых пятен~\cite{5-ar}. Процедура 
аппроксимации изображений выполняется в~два этапа:
\begin{enumerate}[(1)]
\item  аппроксимация 
изображений до~4096~цветов вида

\noindent
  \begin{equation*}
  \mathrm{Img}_i=\left\{
  \Psi_{\mathrm{app}_{4096}} (\mathrm{Image}_i ,
\mathrm{Pal}_{4096})\right\}\,,\enskip i\in [1, 2]\,,
%  \label{3-ar}
  \end{equation*}
где Img$_i$~--- аппроксимированное до~4096~цветов изображение 
снимка~$i$;
$\Psi_{\mathrm{app}_{4096}}$~--- функция получения множества  
(R, G, B)-пик\-се\-лей в~результате аппроксимации к~4096~цветам;
$\mathrm{Pal}_{4096}$~--- палитра~4096~цветов;
\item 
аппроксимация изображений до~256~цветов вида

\noindent
\begin{equation*}
\mathrm{Img}_i=\left\{
\Psi_{\mathrm{app}_{256}}(\mathrm{Image}_i, 
\mathrm{Pal}_{256})\right\}\,,\enskip i\in [1, 2]\,,
%\label{e4-ar}
\end{equation*}
где $\Psi_{\mathrm{app}_{256}}$~--- функция получения множества  
(R, G, B)-пик\-се\-лей в~результате аппроксимации к~256~цветам;
Pal$_{256}$~--- палитра 256~цветов.
\end{enumerate}
  
  Полученные в~результате выполнения двух этапов аппроксимации 
изображения должны быть сегментированы с~целью формирования 
последовательности цветных пятен каждого изображения. 
Последовательность цветных пятен изображений можно представить в~виде: 
  \begin{multline*}
  \hspace*{-5pt}\Psi_{\mathrm{segm}_i}=\{\Psi_{i,j}\} = \left\{
  \begin{matrix}
  \psi_{i,1}(p_{i,1,1}),&\ldots,& \psi_{i,1}(p_{i,1,t}),\ldots\\
  \ldots&\ldots&\ldots\\
  \psi_{i,n}(p_{i,h,1}),&\ldots, &\psi_{i,n}(p_{i,h,t}),\ldots\end{matrix}\right.\\
\left.\begin{matrix}
\ldots, &\psi_{i,u}(p_{i,1,w-g}),&\ldots ,& \psi_{i,k}(p_{i,1,w})\\
  \ldots&\ldots&\ldots&\ldots\\
\ldots,& \psi_{i,j}(p_{i,h,w-d}),&\ldots ,& \psi_{i,j}(p_{i,h,w})
  \end{matrix} \right\}
%  \label{e5-ar}
  \end{multline*}
  
  \vspace*{-12pt}
  
  \noindent
  \begin{gather*}
  i\in [1,  2]\,,\enskip
  j\in [0,  J_j]\,,\enskip
  t\in [1, T_i]\,,\\
  d\in [1, D_i]\,,\enskip
  g\in [1, G_i]\,,\enskip u\in [1, U_i]\,,\\
  w\in [1, W_i]\,,\enskip h\in [1, H_i]\,,\enskip
  n\in [1, N_i]\,,\\
  N_i\leq J_i\,,\enskip U_i\leq J_i\,, \enskip T_i\leq J_i\,,\enskip
  D_i<W_i\,,
  \end{gather*}
где $\Psi_{\mathrm{segm}_i}$~--- множество сегментов изоб\-ра\-же\-ния 
снимка~$i$;
$\psi_{i,j}$~--- сегмент с~номером~$j$ цветного изоб\-ра\-же\-ния снимка~$i$;
$J_i$~--- максимальное количество цветных сегментов изоб\-ра\-же\-ния 
снимка~$i$.
  
  Полученное множество цветных пятен представляет собой 
последовательность из всех цветоразличимых сегментов обрабатываемых 
изоб\-ра\-же\-ний~[6--8]. Для осуществления синхронизации 
изображений путем определения и~сопоставления робастных структур пары 
обрабатываемых изображений необходимо выполнить фильтрацию 
полученного множества цветных пятен. Сравнительно маленькие и~большие 
по площади пятна на изоб\-ра\-же\-ни\-ях не позволяют выполнить синхронизацию 
изображений из-за того, что маленькие пятна могут с~большой вероятностью 
повторяться на паре обрабатываемых изоб\-ра\-же\-ний либо вовсе пропа-\linebreak\vspace*{-12pt}

\pagebreak

\noindent
дать, 
а~большие пятна могут оказаться на границах изоб\-ра\-же\-ний, что приведет 
к~невозможности точного определения их центров. Следовательно, 
необходимо выбрать для дальнейшего рассмотрения только цветные пятна 
средних размеров, имеющих граничные точки, полученные в~результате 
полутоновых преобразований.
  
  Средняя величина пятен определяется как суммарная величина значений 
площадей всех цветных пятен изображения, деленная на количество цветных 
пятен данного изображения. 
  
  Ввиду того что точное совпадение размеров пятен маловероятно, выбирать 
следует при фильт\-ра\-ции пятна, площадь которых будет принадлежать 
промежутку от $\mathrm{AveSize}/2$ до 3AveSize, где\linebreak AveSize~--- средний 
размер цветного пятна изоб\-ра\-же\-ния. Выбор данного интервала увеличивает 
количество цветных пятен, подлежащих рассмотрению, и~повышает 
вероятность успешной синхронизации изображений.
  
  Отфильтрованная последовательность цветных пятен обрабатываемых 
изображений может быть представлена в~виде:

\noindent
  \begin{multline*}
  \theta_{\mathrm{segm}_i}= \{\theta_{i,k}\}\subset \Psi_{\mathrm{segm}_i}:\ i\in 
[1, 2]\,,\\
  k\in [0,  K]\,,\enskip j\in [0,  J_i]\,,\enskip K_i\leq J_i\,,
%  \label{e6-ar}
  \end{multline*}
где $\theta_{\mathrm{segm}_i}$~--- множество цветных сегментов, оставшихся 
после фильтрации цветных сегментов изоб\-ра\-же\-ния снимка~$i$.

  После того как получена последовательность цветных пятен, 
удовлетворяющая всем условиям фильтрации: площадь и~наличие граничных 
точек, необходимо выполнить процедуру сравнения множеств~$\theta_{\mathrm{segm}_i}$ 
для первого и~второго изображений. Сравнение 
производится путем определения совпадения площадей цветных пятен 
и~взаимного удаления данных пятен от других на каждом изоб\-ра\-же\-нии. При 
этом, найдя цветное пятно $\theta_{1,b}:\ b\hm\leq K_1$, принадлежащее 
первому изображению, и~цветное пятно $\theta_{2,v}:\ v\hm\leq K_2$, 
площадь которого соответствует~$\theta_{1,b}$, необходимо выполнить 
проверку удаленности от всех цветных пятен на каждом изображении 
относительно данных пятен. Если $Y_{\mathrm{segm}}\hm=\emptyset$, то данная пара 
заносится в~множество как совпадающие друг с~другом сегменты. Если 
расстояния до большей части пятен, которые были занесены в~$Y_{\mathrm{segm}}$, 
совпадают, то необходимо добавить в~множество~$Y_{\mathrm{segm}}$ данную пару 
цветных пятен, иначе они будут признаны как ошибочно выбранные 
совпадающими друг с~другом. Получа\-емую последовательность  можно 
представить в~виде:
  \begin{multline*}
  Y_{\mathrm{segm}} =\{\tau_m\}\subset \theta_{\mathrm{segm}_i}:\
  i\in [1, 2]\,,\\
  m\in [0,M_i]\,,\enskip K_1\geq M_i\leq K_2\,,
%  \label{e7-ar}
  \end{multline*}
  
  \columnbreak
  
\noindent
где $Y_{\mathrm{segm}}$~--- множество цветных сегментов, выбранных в~качестве 
совпадающих на паре изображений выбранных снимков.

  Поскольку первый элемент множества~$Y_{\mathrm{segm}}$ не может быть 
проверен на удаленность от других цветных пятен, а второй и~последующий 
сравниваются только с~предыдущими элементами, то необходимо выполнить 
дополнительную проверку\linebreak уда\-лен\-ности каждого элемента 
множества~$Y_{\mathrm{segm}}$ для каждого изоб\-ра\-же\-ния, тем самым исключив 
возможные случайные ошибки. Результирующее множество сегментов, 
имеющих соответствие на паре обрабатываемых изоб\-ра\-же\-ний, может быть 
представлено в~виде:
  \begin{equation*}
  S_{\mathrm{segm}}= \{s_c\}\subset Y_{\mathrm{segm}}:\ c\in [0,C_i]\,,\ K_1\geq C_i\leq K_2\,,
%  \label{e8-ar}
  \end{equation*}
  
  \vspace*{-2pt}
  
  \noindent
где $S_{\mathrm{segm}}$~--- множество цветных сегментов, оставшихся после 
фильтрации цветных сегментов, выбранных в~качестве совпадающих на паре 
изображений выбранных снимков.

  Успешная синхронизация изображений возможна, если имеется три 
и~более сегментов, име\-ющих соответствие на паре обрабатываемых 
изоб\-ра\-же\-ний. При этом чем дальше данные цветовые\linebreak пятна будут 
располагаться друг от друга, тем выше точность синхронизации и~создания 
ЛСК. Если число соответствующих друг другу 
сегментов три и~более, то выполняется процедура создания 
ЛСК для обрабатываемой пары изображений путем 
определения угла поворота одного изображения относительно другого 
и~вычисления расстояний до краев общей области с~переносом пикселей 
каждого из изображений. Таким образом, при выполнении условия наличия 
минимум трех цветных пятен, которые совпадают на паре обрабатываемых 
изображений, за счет выполнения процедуры

\vspace*{2pt}

\noindent
  \begin{equation*}
  \mathrm{Im}\_s_i= \mathrm{Qs}_i(\mathrm{Image}, S_{\mathrm{segm}})\,,\enskip i\in [1, 
2]\,,
%  \label{e9-ar}
  \end{equation*}
  
  \vspace*{-2pt}
  
  \noindent
где Im\_s$_i$~--- изображения, преобразованные к~ЛСК;
Qs$_i$~--- процедура построения ЛСК для 
изоб\-ра\-же\-ния снимка~$i$;
получаем два новых изображения, которые будут иметь новые 
ЛСК, совпадающие на обоих изоб\-ра\-же\-ниях.
  
  На рис.~1 приведена схема алгоритма создания 
ЛСК для синхронизации изображений выбранных кадров.



  Процессы на рис.~1:
{1}~--- получение полутоновых представлений изоб\-ра\-же\-ний
Image$_1$ и~Image$_2$;
  {2}~--- аппроксимация изоб\-ра\-же\-ний Image$_1$ и~Image$_2$
  к~палитре~4096~цветов;
  {3}~--- аппроксимация изобра\-же\-ний Img$_1$ и~Img$_2$ 
к~палитре~256~цветов;
  {4}~---  сегментация изоб\-ра\-же\-ний Img$_1$ и~Img$_2$;
  {5}~--- фильт\-ра\-ция цветных сегментов $\Psi_\mathrm{segm_1}$ 
и~$\Psi_\mathrm{segm_2}$;\linebreak\vspace*{-12pt} 

  
  \pagebreak
  
  \end{multicols}
  
  \begin{figure} %fig1
\vspace*{1pt}
 \begin{center}  
\mbox{%
 \epsfxsize=140.988mm
 \epsfbox{arh-1.eps}
 }
\end{center} 
\vspace*{-9pt}
\Caption{Схема алгоритма создания ЛСК для синхронизации 
изображений выбранных кадров}
\end{figure}
  
  \begin{multicols}{2}
  
  \noindent
  {6}~--- определение соответствия цветных сегментов 
$\theta_{\mathrm{segm}_1}$ и~$\theta_{\mathrm{segm}_2}$ пары изоб\-ра\-же\-ний
Img$_1$ и~Img$_2$ вне зависимости от угла поворота и~смещения;
  {7}~--- фильт\-ра\-ция обнаруженных сегментов множества~$Y_{\mathrm{segm}}$;
  {8}~--- определение числа совпадающих пятен на паре сравниваемых 
изоб\-ра\-же\-ний и~возможности построения ЛСК;
  {9}~--- построение ЛСК;
  {10}~--- выдача ошибки построения ЛСК;
  {11}~--- завершение работы.
  
\vspace*{-6pt}

    \section{Результаты вычислительных экспериментов}
    
  Для тестирования предлагаемого варианта создания 
ЛСК для синхронизации изображений выбранных снимков была 
выбрана пара кадров, представленных на рис.~2. 
     
\begin{figure*} %fig2
\vspace*{1pt}
 \begin{center}  
\mbox{%
 \epsfxsize=156.221mm
 \epsfbox{arh-2.eps}
 }
\end{center} 
\vspace*{-9pt}
      \Caption{Изображения первого~(\textit{а}) и~второго~(\textit{а}) снимков}
      \end{figure*}
      
  После выполнения сегментации и~фильтрации сегментов оставшиеся 
сегменты на изображениях первого и~второго снимков выделены зеленым 
цветом. Для наглядности вручную были отмечены красными линиями 
и~стрелками части изображений, по которым наиболее отчетливо видно 
относительное смещение изображений (рис.~3).
  
\begin{figure*} %fig3
\vspace*{1pt}
 \begin{center}  
\mbox{%
 \epsfxsize=156.304mm
 \epsfbox{arh-3.eps}
 }
\end{center} 
\vspace*{-9pt}
  \Caption{Сегменты для синхронизации, визуализация относительного смещения 
изображений}
  \end{figure*}
  
  После выполнения вышеописанных процедур было 
обнаружено~6~совпадающих цветных сегментов, за счет которых была 
построена ЛСК и~преобразованы оба изображения к~виду, пригодному для 
сравнения. Полученные изображения пред\-став\-ле\-ны на
рис.~4,\,\textit{а} и~4,\,\textit{б}.
  
  \begin{figure*} %fig4
  \vspace*{1pt}
 \begin{center}  
\mbox{%
 \epsfxsize=162.337mm
 \epsfbox{arh-4.eps}
 }
\end{center} 
\vspace*{-9pt}
  \Caption{Преобразованные изображения первого~(\textit{а}) и~второго~(\textit{б}) снимков}
  \end{figure*}
  
  \vspace*{-12pt}
     
  \section{Заключение}
  
  В данной статье был предложен вариант создания 
ЛСК для синхронизации изображений выбранных снимков.
  %
  В результате его тестирования были получены положительные результаты 
для изображений, имеющих искажение сдвига и~поворота. 
  
{\small\frenchspacing
 {%\baselineskip=10.8pt
 \addcontentsline{toc}{section}{References}
 \begin{thebibliography}{9}


\bibitem{2-ar}
\Au{Прэтт У.} Цифровая обработка изображений~/ Пер. с~англ.~--- М.: Мир, 
1982. (Pratt~W.\,K. Digital image processing.~---Wiley-Interscience Publication, 
1978. 750~p.)
\bibitem{1-ar}
\Au{Форсайт Д., Понс Ж.} Компьютерное зрение. Современный подход.~--- 
М.: Вильямс, 2004. 928~с. 
\bibitem{3-ar}
\Au{Гонсалес Р., Вудс Р.} Цифровая обработка изображений~/ Пер. с~англ.~--- 
М.: Техносфера, 2005. 1070~с. (Gonzalez~R.\,C.,  Woods~R.\,E. Digital image 
processing.~--- Wiley-Interscience Publication, 2002. 793~p.)
\bibitem{4-ar}
\Au{Архипов О.\,П., Зыкова З.\,П.} Применение полутоновых представлений 
при анализе изменений цветных изоб\-ра\-же\-ний~// Информатика и~её 
применения, 2014. Т.~8. Вып.~3. С.~90--99.
\bibitem{5-ar}
\Au{Архипов О.\,П., Зыкова З.\,П.} Интеграция гетерогенной информации 
о~пикселях и~их цветовосприятии~// Информатика и~её применения, 2010. 
T.~4. Вып.~4. С.~14--25.
\bibitem{6-ar}
\Au{Архипов О.\,П., Зыкова З.\,П.} Функциональное описание 
индивидуального цветовосприятия~// Известия ОрелГТУ. Сер. 
Информационные системы и~технологии, 2010. №\,5. С.~5--12.
\bibitem{7-ar}
\Au{Архипов О.\,П., Зыкова З.\,П.} RGB-ха\-рак\-те\-ри\-за\-ция пространства 
цветовосприятия~// Системы и~средства информатики, 2010. Вып.~20. №\,1. 
С.~72--89.
\bibitem{8-ar}
\Au{Архипов О.\,П., Зыкова З.\,П.} Равноконтрастные градационные 
преобразования ступенчатых тоновых шкал~// Информационные системы 
и~технологии, 2011. №\,4. С.~39--46.
\end{thebibliography}

 }
 }

\end{multicols}

\vspace*{-6pt}

\hfill{\small\textit{Поступила в~редакцию 12.07.16}}

\vspace*{10pt}

%\newpage

%\vspace*{-24pt}

\hrule

\vspace*{2pt}

\hrule

%\vspace*{8pt}



\def\tit{THE OPTION TO CREATE A~LOCAL COORDINATE SYSTEM 
FOR~SYNCHRONIZATION OF~SELECTED IMAGES}

\def\titkol{The option to create a~local coordinate system 
for~synchronization of selected images}

\def\aut{O.\,P.~Arkhipov, P.\,O.~Arkhipov, and~I.\,I.~Sidorkin}

\def\autkol{O.\,P.~Arkhipov, P.\,O.~Arkhipov, and~I.\,I.~Sidorkin}

\titel{\tit}{\aut}{\autkol}{\titkol}

\vspace*{-9pt}

\noindent
Orel Branch of the 
Federal Research Center ``Computer Science and Control'' of the Russian Academy 
of Sciences, 137~Moskovskoe Sh., Orel 302025, Russian Federation


\def\leftfootline{\small{\textbf{\thepage}
\hfill INFORMATIKA I EE PRIMENENIYA~--- INFORMATICS AND
APPLICATIONS\ \ \ 2016\ \ \ volume~10\ \ \ issue\ 3}
}%
 \def\rightfootline{\small{INFORMATIKA I EE PRIMENENIYA~---
INFORMATICS AND APPLICATIONS\ \ \ 2016\ \ \ volume~10\ \ \ issue\ 3
\hfill \textbf{\thepage}}}

\vspace*{9pt}



\Abste{While comparing pairs of images, in most cases, the problem of 
misalignment of images arises in which one image is distortion of translation and 
rotation relative to another image. Such an image is quite difficult to compare in 
automatic mode. Existing methods of image pairs  synchronize a~large 
number of constraints due to which most of them are rarely used. The proposed 
option to create a~local coordinate system for synchronizing images is based on the 
analysis of color spots presented on the images. It is assumed that successful 
synchronization of two images on their total amount of colored spots is to be 
found that match on the data images. For comparison, it is suggested to use
colored spots and 
distances between spots. In order to successfully synchronize, one needs at 
least three colored spots, which would coincide in all modes of filtration. The 
experiments show acceptable results of synchronization.}

\KWE{algorithm; local coordinate system; color image; synchronization; pixel; 
colored spot; filtration}

\DOI{10.14357/19922264160312} 

\vspace*{9pt}

%\Ack
%\noindent



%\vspace*{6pt}

  \begin{multicols}{2}

\renewcommand{\bibname}{\protect\rmfamily References}
%\renewcommand{\bibname}{\large\protect\rm References}

{\small\frenchspacing
 {%\baselineskip=10.8pt
 \addcontentsline{toc}{section}{References}
 \begin{thebibliography}{9}

  
\bibitem{2-ar-1}
\Aue{Pratt, W.\,K.} 1978. \textit{Digital image processing}. Wiley-Interscience 
Publication. 750~p.
\bibitem{1-ar-1}
  \Aue{Forsyth, D.\,A., and J.~Ponce}. 2002. \textit{Computer vision: A~modern 
approach}. Prentice Hall Professional Technical Reference. 720~p.
\bibitem{3-ar-1}
\Aue{Gonzalez, R.\,C., and R.\,E.~Woods}. 2002. \textit{Digital image 
processing}. Wiley-Interscience Publication. 793~p.
  \bibitem{4-ar-1}
  \Aue{Arhipov, O.\,P., and Z.\,P.~Zykova}. 2014. Primenenie polutonovykh 
predstavleniy pri analize izmeneniy tsvetnykh izobrazheniy [The use of half-tone 
representations in the analysis of changes in color images]. \textit{Informatika i~ee 
Primeneniya~--- Inform. Appl.} 8(3):90--99.
  \bibitem{5-ar-1}
  \Aue{Arhipov, O.\,P., and Z.\,P.~Zykova}. 2010. Integratsiya geterogennoy 
informatsii o pikselyakh i~ikh tsvetovospriyatii [Integration of heterogeneous 
information about pixels and their color perception]. \textit{Informatika i~ee 
Primeneniya~--- Inform. Appl.} 4(4):14--25.
  \bibitem{6-ar-1}
  \Aue{Arhipov, O.\,P., and Z.\,P.~Zykova}. 2010. Funktsional'noe opisanie 
individual'nogo tsvetovospriyatiya [Characteristics of color perceptual space]. 
\textit{Izvestiya OrTGU. Ser. Informatsionnye sistemy i~tekhnologii} [Herald
of Oryol Technical State University. Ser. 
information systems and technologies] 5:5--12.

\pagebreak

  \bibitem{7-sr-1}
  \Aue{Arhipov, O.\,P., and Z.\,P.~Zykova}. 2010. RGB-kharakterizatsiya 
prostranstva tsvetovospriyatiya [RGB-characterization of color space]. 
\textit{Sistemy i~Sredstva Informatiki~--- Systems and Means of Informatics} 
1(20):\linebreak 72--89.
  \bibitem{8-ar-1}
  \Aue{Arhipov, O.\,P., and Z.\,P.~Zykova}. 2011. Ravnokontrastnye 
gradatsionnye preobrazovaniya stupenchatykh tonovykh shkal [Equal contrast 
graded transformation of step tinted scales]. \textit{Informatsionnye Sistemy 
i~Tekhnologii} [Information Systems and Technologies] 4:39--46.
\end{thebibliography}

 }
 }

\end{multicols}

\vspace*{-6pt}

\hfill{\small\textit{Received July 12, 2016}}

\vspace*{-3pt}


\Contr

\noindent
\textbf{Arkhipov Oleg P.}\ (b.\ 1948)~--- Candidate of Science (PhD) in technology, Director, Oryol Branch of  
Federal Research Center  ``Computer Science  and Control'' of the 
Russian Academy of Sciences, 137~Moskovskoe Sh., Oryol 
302025, Russian Federation; \mbox{arkhipov12@yandex.ru}

\vspace*{4pt}

\noindent
\textbf{Arkhipov Pavel O.}\ (b.\ 1979)~--- Candidate of Science (PhD) in technology, senior scientist, Oryol Branch 
of  Federal Research Center  ``Computer Science  and Control'' of the Russian 
Academy of Sciences, 137~Moskovskoe Sh., Oryol 
302025, Russian Federation; \mbox{arpaul@mail.ru}

\vspace*{4pt}

\noindent
\textbf{Sidorkin Ivan I.}\ (b.\ 1990)~--- engineer-researcher, Orel Branch of the 
Federal Research Center ``Computer Science and Control'' of the Russian Academy 
of Sciences, 137~Moskovskoe Sh., Orel 302025, Russian Federation; 
\mbox{voronecburgsiti@mail.ru}
  \label{end\stat}
  
  
  \renewcommand{\bibname}{\protect\rm Литература} %2
\def\stat{tsiskar}

\def\tit{МАТЕМАТИЧЕСКАЯ МОДЕЛЬ И~МЕТОД ВОССТАНОВЛЕНИЯ ПОЗЫ 
ЧЕЛОВЕКА ПО~СТЕРЕОПАРЕ СИЛУЭТНЫХ ИЗОБРАЖЕНИЙ$^*$}

\def\titkol{Математическая модель и~метод восстановления позы 
человека по~стереопаре силуэтных изображений}

\def\autkol{А.\,К.~Цискаридзе}
\def\aut{А.\,К.~Цискаридзе$^1$}

\titel{\tit}{\aut}{\autkol}{\titkol}

{\renewcommand{\thefootnote}{\fnsymbol{footnote}}\footnotetext[1]
{Работа выполнена при поддержке РФФИ, гранты №\,08-01-00670, №\,08-07-00270.}}

\renewcommand{\thefootnote}{\arabic{footnote}}
\footnotetext[1]{Московский физико-технический институт, AchikoTsi@gmail.com}

\vspace*{-6pt}

\Abst{Работа посвящена проблеме восстановления позы локально симметричного объекта по 
стереопаре силуэтов в случае отсутствия окклюзии. Рассматриваются две модели фигуры. В 
первом случае фигура описывается как объединение пространственных жирных линий и 
предлагается метод ее восстановления. Во втором случае фигура описывается в виде 
шарнирной модели и предлагается метод ее подгонки по стереопаре силуэтов. Методы 
основаны на построении непрерывных скелетов силуэтов.}

%\vspace*{4pt}

\KW{стереореконструкция; скелет; цилиндрические объекты; шарнирная модель; 
серединные оси}

%\vspace*{6pt}

       \vskip 14pt plus 9pt minus 6pt

      \thispagestyle{headings}

      \begin{multicols}{2}

      \label{st\stat}

\section{Введение}
     
     Задача восстановления формы пространственного объекта по нескольким 
двумерным изображениям хорошо известна и имеет множество приложений. 
В~частности, эта проблема возникает при распознавании позы и жестов 
человека в системах видеонаблюдения. Особенность рассматриваемой нами 
постановки этой задачи состоит в том, что двумерные изображения являются 
бинарными и представляют собой лишь силуэты пространственного объекта. 
Такая задача, в частности, возникает в системах видеонаблюдения, 
работающих в условиях плохой освещенности. В~этом случае камеры плохо 
передают текстурные особенности изображений и позволяют с достоверностью 
выявить лишь силуэты представленных на изображении объектов. Для 
распознавания позы и жестов требуется по этим силуэтам восстановить 
пространственную форму столь сложного и изменчивого объекта, как фигура 
человека.
     
     Невозможность анализа изображений на уровне текстур не позволяет 
применить для такой постановки задачи хорошо известные методы, 
основанные на автоматическом выявлении общих точек, присутствующих на 
обоих изображениях стереопары. Очевидно, что если на изображении 
пред\-став\-лен лишь силуэт объекта, то достоверно на нем можно выявить лишь 
точки границы этого объекта. Но на двух картинках в стереопаре изображений 
границы силуэтов полностью различаются, т.\,е.\ все точки границы одного 
силуэта отличаются от всех граничных точек другого силуэта. Поэтому 
выявление общих точек невозможно.
     
     В общей постановке задачи можно выделить два варианта, 
различающихся наличием так называемых окклюзий. Силуэтное изображение 
объекта называется изображением без окклюзии, если в каждую его точку 
проектируется не более двух точек поверхности объекта. В~соответствии с 
этим первый, более простой, вариант задачи представляют собой работу с 
изображениями без окклюзий. В~этом случае в силуэтах видны голова и все 
конечности. Второй вариант задачи~--- когда окклюзии имеют место. 
В~данной работе ограничимся рассмотрением изображений без окклюзий. Не 
будем также затрагивать вопрос выделения силуэтов на исходных картинах, 
считая что они уже получены с хорошей (1--2~пикселя) точностью.
     
     Подходы к решению этой задачи в полном объеме для сложных объектов 
в настоящее время только лишь формируются~[1--3]. Можно сослаться на работу~\cite{1ts}, 
в которой описан метод восстановления поверхности сложного 
пространственного объекта (фигура лошади) по серии силуэтных изображений, 
полученных в разных ракурсах. Для этого метода требуются изображения 
хорошего качества, а также большие ресурсы процессорного времени.
     
     Вместе с тем существует определенный класс объектов, чьи структурные 
особенности позволяют решить задачу восстановления пространственной 
формы и при отсутствии видимых общих точек.\linebreak Это объекты, состоящие из 
круговых цилиндров. Особенности такого пространственного объекта, как 
фигура человека, позволяют рассматривать\linebreak его приближенно как объединение 
нескольких\linebreak <<цилиндрических>> элементов, имеющих локальную осевую 
симметрию. Такие объекты некоторые\linebreak авторы называют обобщенными 
цилиндрами~\cite{2ts}. Встречается также термин <<трубчатые объекты>> 
(tubular)~\cite{3ts}. Если говорить более строго, под цилиндрическим элементом 
понимается пространственное тело, образованное семейством шаров, центры 
которых расположены на некоторой осевой пространственной кривой. 
В~работах~[7, 8] такие объекты называются пространственными жирными 
кривыми. Представляют интерес объекты, которые могут быть представлены в 
виде объединения небольшого числа пространственных жирных кривых. 
Человеческая фигура в некотором приближении также может быть составлена 
из жирных кривых по аналогии с тем, как дети лепят человечков из 
пластилина.
     
     Предлагаемый в данной работе подход к решению основывается на идее 
построения скелетов для стереопары силуэтных изображений. Скелет 
представляет собой совокупность серединных осей силуэта, определяемых как 
геометрическое место точек~--- центров вписанных в силуэт окружностей. 
Использование скелетов открывает несколько возможных путей для решения 
задачи. Рассмотрим два из них. 
     
     Первый путь состоит в построении пространственной циркулярной 
модели человеческой фигуры. Он основывается на идее конструирования 
стереопар <<невидимых>> общих точек обоих изоб\-ра\-же\-ний. Серединные оси 
силуэтов предлагается рассматривать как плоские проекции пространственных 
осевых линий жирных кривых, составляющих\linebreak фигуру человека. Такой подход 
позволяет свести задачу восстановления этих пространственных жирных линий 
к вычислению пространственных кривых по стереопарам их проекций. 
Результатом\linebreak решения задачи является циркулярная модель, представляющая 
собой объединение нескольких пространственных жирных линий. 
     
     Второй путь предполагает упрощенное описание фигуры человека в виде 
<<шарнирной>> модели заданной структуры. Шарнирная модель также 
состоит из пространственных жирных линий. Но эти жирные линии являются 
линейными сегментами постоянной заданной ширины. Поза человека ищется 
путем подбора некоторого преобразования шарнирной модели, при котором ее 
плоские проекции будут в наибольшей степени совпадать со скелетами 
стереопары силуэтов. 

\vspace*{-12pt}

\section{Циркулярная и шарнирная модели фигуры человека}

     Рассмотрим множество точек~$T$ в евклидовом пространстве~$R^3$, 
имеющее вид связного графа,\linebreak

\noindent
\begin{center} %fig1
\vspace*{-6pt}
\mbox{%
\epsfxsize=76.379mm
\epsfbox{cis-1.eps}
}
\end{center}
\vspace*{3pt}
%\begin{center}
{{\figurename~1}\ \ \small{Циркулярная модель фигуры человека: осевой граф~(\textit{а}) и огибающая 
поверхность~(\textit{б})}}
%\end{center}
\vspace*{3pt}

\bigskip
\addtocounter{figure}{1}

\noindent
описывающего человеческую фигуру 
(рис.~1). Граф имеет пять терминальных вершин и три вершины 
третьей степени, а его ребра являются непрерывными линиями. При этом ребра 
не имеют точек пересечения, не совпадающих с их концами.

   
     С каждой точкой $t\in T$ графа~$T$ связан некоторый шар~$c_t$ с 
центром в этой точке. Это семейство шаров $C=\{c_t,\,t\in T\}$ называется 
\textit{циркулярным графом}, для краткости~--- \textit{циркуляром}~\cite{5ts}. 
Граф~$T$ называется \textit{осевым графом} циркулярного графа.\linebreak 
Объединение $S=\bigcup\limits_{t\in T} c_t$ всех шаров семейства~$C$ как 
точечных множеств является циркулярной мо\-делью фигуры человека. 
Границей модели является огибающая поверхность семейства шаров~$C$. 
     
     Восстановление циркулярной модели сводится к нахождению 
составляющих циркулярный граф жирных линий, т.\,е.\ к описанию их осевых 
линий и функции ширины, задающей для каждой точки осевой линии радиус 
шара с центром в этой точке. При этом построение жирных линий должно 
осуществляться таким образом, чтобы проекции циркулярной модели хорошо 
согласовывались со стереопарой исходных изображений.
     
     Шарнирная модель описывает фигуру человека как объединение 
10~шарнирно закрепленных твердых тел. Каждый элемент этой конструкции 
представляет собой локально симметрический объект, образованный 
объединением шаров одинакового радиуса с центрами на прямолинейном 
отрезке (рис.~\ref{f2ts}). 
     
     С каждым элементом свяжем систему координат, ось~$X$ которой 
направлена вдоль его оси сим\-мет\-рии, а начало координат находится в точке 
крепления его с родительским телом. Для элемента известна его длина и 
ширина, а также точка крепления в системе координат родительского тела. По 
иерархии крепления элементы шарнирной модели образуют дерево с корневым 
элементом, соответствующим туловищу человека.
     
     \begin{figure*} %fig2
     \vspace*{1pt}
\begin{center}
\mbox{%
\epsfxsize=130.894mm
\epsfbox{cis-2.eps}
}
\end{center}
\vspace*{-6pt}
\Caption{Шарнирная модель фигуры человека: структура модели~(\textit{а}) и пространственные 
элементы модели~(\textit{б})
\label{f2ts}}
\end{figure*}
\begin{figure*}[b] %fig3
\vspace*{1pt}
\begin{center}
\mbox{%
\epsfxsize=152.966mm
\epsfbox{cis-3.eps}
}
\end{center}
\vspace*{-6pt}
\Caption{Стереопара исходных силуэтов~(\textit{а}) и стереопара скелетов~(\textit{б})
\label{f3ts}}
\end{figure*}
     
     Вращение каждой части шарнирной модели задается относительно 
родительской системы координат. Его можно параметризовать с помощью 
суперпозиции трех вращений $R_\theta \circ R\psi \circ R_\varphi$ относительно 
осей~$X, Y, Z$ своей системы координат (см.\ рис.~\ref{f2ts}). Так как структура 
скелета человека допускает не всякие вращения, для каждого тела введем 
ограничения на углы в виде параллелепипеда: $\varphi_{\min}\leq \varphi \leq 
\varphi_{\max}$, $\psi_{\min}\leq \psi\leq \psi_{\max}$, 
$\theta_{\min}\leq\theta\leq \theta_{\max}$.
     
     Под позой объекта будем понимать вектор значений динамических 
параметров модели. Каждой позе соответствует точка в пространстве из 
24~динамических параметров фигуры, а все множество поз описывается 
параллелепипедом~$\Theta$ в 24-мерном пространстве $\Theta\subset R^{24}$. 
Таким образом, задача состоит в том, чтобы по стереопаре бинарных 
изображений найти вектор динамических па\-ра\-мет\-ров шарнирной модели, 
аппроксимирующей форму пространственной фигуры.

\section{Восстановление позы в виде циркулярной модели }

\subsection{Структура метода} %3.1

     Предлагаемый подход к восстановлению формы человеческой фигуры в 
виде совокупности жирных линий иллюстрируется на рис.~3--5. 
\begin{figure*} %fig4
\vspace*{1pt}
\begin{center}
\mbox{%
\epsfxsize=88.443mm
\epsfbox{cis-4.eps}
}
\end{center}
\vspace*{-6pt}
\Caption{Предположение, что проекции осей объекта совпадают со скелетом силуэтов
\label{f4ts}}
\vspace*{6pt}
\end{figure*}


     Сначала для силуэтов, полученных на основе сегментации исходных 
изображений (см.\ рис.~\ref{f3ts}), строятся их скелеты в виде серединных 
осей~\cite{12ts}, с которыми связано множество вписанных в силуэты 
окружностей с центрами на серединных осях (см.\linebreak\vspace*{-12pt}
\pagebreak

\noindent
\begin{center} %fig5
\vspace*{-6pt}
\mbox{%
\epsfxsize=73.118mm
\epsfbox{cis-5.eps}
}
\end{center}
\vspace*{3pt}
%\begin{center}
{{\figurename~5}\ \ \small{Кривая, сопоставляющая стереопары точек}}
%\end{center}
\vspace*{3pt}

\bigskip
\addtocounter{figure}{1}


\noindent
 рис.~\ref{f3ts}). После этого из 
стереопары скелетов конструируется пространственный скелет объекта, а по 
семействам вписанных кругов вычисляются радиусы сфер с центрами на 
пространственном скелете. Таким образом, поза человека может быть 
представлена пространственным скелетом, либо огибающей поверхностью для 
семейства построенных сфер.
     
     В случае, когда нет окклюзии в силуэтах, в первом приближении можно 
считать, что направление взгляда на объект не сильно отклоняется от 
оптической оси камеры. Тогда можно предполагать, что на плоском 
изображении образ пространственной оси локально симметрического объекта 
совпадает со скелетом силуэта (см.\ рис.~\ref{f4ts}). Строго говоря, центральная 
проекция сферы на плоскость является эллипсом, но приближенно эти эллипсы 
будем считать окружностями. По двум плоским проекциям пространственной 
оси с помощью эпиполярной геометрии~[11] можно восстановить 
пространственную ось.
     


\subsection{Идентификация реперных точек на~скелетах} %3.2
     
     Рассмотрим стереопару силуэтов и их скелетов\linebreak (см.\  рис.~\ref{f3ts}). 
Объект <<фигура человека>> прибли\-женно представляется осесимметричными 
элементами. Тогда на плоских изображениях образ про-\linebreak странственной оси 
осесимметричного элемента\linebreak совпадает с соответствующими ветвями скелетов 
(см.\ рис.~\ref{f3ts}, кривые~$OA$ и $O^\prime A^\prime$). Отсюда можно 
сделать предположение, что множество стереопарных точек ветви~$OA$ скелета 
одного силуэта совпадает с ветвью~ $O^\prime A^\prime$ скелета другого 
силуэта, что позволяет построить кривую в пространстве. Задача состоит в том, 
чтобы наилучшим образом установить соответствие между точками этих 
скелетных ветвей. Если задать кривую~$OA$ как непрерывное 
отображение~$r_1$:~$[0,\,1] \rightarrow R^2$, а кривую~$O^\prime A^\prime$ 
как $r_2$:~$[0,\,1] \rightarrow R^2$, задача сведется к на\-хож\-де\-нию 
непрерывного монотонного отображения $w$:~$[0,\,1] \rightarrow [0,\,1]$, 
которое сопоставляет стереопары точек $r_1(t) \leftrightarrow r_2(w(t))$ и при 
этом минимизирует расхождение, заданное в виде функционала
     $$
     \min \int\limits_0^1 \rho (r_1(t),\,r_2(w(t)))\sqrt{1+w^\prime(t)^2}\,dt\,.
     $$
          Здесь $\rho(X,Y)$~--- функция штрафа~--- отражает,\linebreak насколько хорошо 
сопоставляются точки~$X$ и~$Y$.\linebreak Выбор этой функции осуществляется на 
основе следующего соображения. Для каждой точки изоб\-ра\-же\-ния существует 
луч в пространстве, который в нее проецируется. 
\begin{figure*} %fig6
\vspace*{1pt}
\begin{center}
\mbox{%
\epsfxsize=159.139mm
\epsfbox{cis-6.eps}
}
\end{center}
\vspace*{-6pt}
\Caption{Стереопара изображений и полученный пространственный объект
\label{f6ts}}
\end{figure*}
\begin{figure*}[b] %fig7
\vspace*{1pt}
\begin{center}
\mbox{%
\epsfxsize=164.469mm
\epsfbox{cis-7.eps}
}
\end{center}
\vspace*{-6pt}
\Caption{Стереопары и полученные пространственные объекты для <<Kungfu 
girl>>
\label{f7ts}}
\end{figure*}
При идеально правильном 
сопоставлении точек~$X$ и~$Y$ проходящие через них лучи должны 
пересекаться. Поэтому в качестве~$\rho$ можно взять расстояние между 
скрещивающимися лучами, которые проецируются в точки~$X$ и~$Y$.
     
     Полученная задача решается методом динамического программирования. 
Дискретизируя задачу сеткой $N\times N$, для нахождения точного решения 
получаем сложность $O(N^2d)$ (см.\ рис.~5). Здесь $d$~--- ограничение, 
задающее коридор для кривой. Построив пространственные оси, используя 
ширину скелетов, можно вычислить размеры шаров с центрами на этих осях. 
     

      
Пример визуализации модели фигуры, полученной по стереопаре изображений, 
представлен на рис.~\ref{f6ts}. Как видно из этого примера, визуализация 
является не вполне реалистичной, поскольку описание тела человека 
цилиндрами представляется весьма грубым. Однако для вычислений, 
связанных с распознаванием позы, такая точность представляется вполне 
приемлемой. 

\subsection{Вычислительные эксперименты} %3.3

     Эксперименты с восстановлением пространственной модели фигуры 
человека проводились с куклами размером 30~см. Это объясняется лишь\linebreak 
упрощением съемки в лабораторных условиях. На рис.~\ref{f6ts} показана 
стереопара исходных изоб\-ра\-же\-ний и полученный пространственный объект. 
Эксперименты также проводились с синтетическими данными <<Kungfu 
girl>>, предоставленными группой Graphics-Optics-Vision из института\linebreak 
     Max-Planck~\cite{6ts}. Эти данные представляют собой синтезированные 
пространственные модели виртуальных сцен и их проекции в виде двумерных 
изоб\-ра\-же\-ний размером $320\times 240$. Результаты реконструкции показаны 
на рис.~\ref{f7ts}.



     Эксперименты показывают, что модель локально симметричного объекта 
хорошо описывает фигуру человека в целом. Скорость работы на компьютере 
Intel Pentium~IV, Core~2 Duo, 2800~МГц составила более 5~кадров/с. 
Это дало возможность использовать предложенный метод в системах 
компьютерного зрения в реальном масштабе времени их работы. 
     

\section{Восстановление позы в виде шарнирной модели }

\subsection{Предлагаемый подход к решению задачи} %4.1
     
     Предположим, что определена шарнирная модель объекта вместе с 
длинами составляющих ее элементов. Задача подгонки шарнирной модели 
заключается в следующем:
     
     \textbf{Дано:} Шарнирная модель; стереопара силуэтов~$s_1$, $s_2$; 
проекционные матрицы камер.
     
     \textbf{Найти:} Вектор динамических параметров шарнирной модели 
$\vec{\theta}^*\in\Theta$, который имеет стереопару силуэтов, максимально 
совпадающую с наблю\-да\-емой стереопарой. 
     
     Другими словами, $\vec{\theta}^*\;=\;\mathrm{arg}\,\underset{\vec\theta \in 
\Theta}{\min}\,\mu (\vec\theta, S_1, S_2)$, где $\mu(\vec\theta, S_1, S_2)$~--- 
величина сходства позы с силуэтами. Такая постановка задачи возможна также 
для случая с окклюзиями. При этом основная сложность заключается в том, что 
любая целевая функция для таких сложных объектов, как фигура человека, 
является многоэкстремальной~\cite{9ts}. В~случае, когда нет окклюзии, 
удается выписать хорошую критериальную функцию. При этом на основе 
скелетов силуэтов делается предварительная сегментация силуэта. 

\subsection{Подгонка шарнирной модели под~стереопару наблюдаемых 
силуэтов для~задачи без~окклюзии } %4.2
     
     В данном подходе предполагается, что известна шарнирная модель 
объекта вместе с длинами отдельных ее элементов. Опишем функцию сходства 
шарнирной модели с наблюдаемыми силуэтами и метод ее минимизации. Для 
каждого силуэта $S_i$:~$i\in \{1,\,2\}$ построим его базовый скелет. Анализ 
геометрии скелетов позволяет выделить на них оси для каждой из 6~частей 
тела человека. На рис.~8 показана такая сегментация для одного из 
скелетов. Через $A_i$, $\gamma_i$, $i=1,\ldots , 5$, обозначены концевые 
вершины и сегменты соответственно. Для любой позы $\vec\vartheta\in\Theta$ 
шарнирной модели построим проекции ее осей на плоскость камеры и выделим 
на них соответствующие ветви. Обозначим их через $B_i$, $\lambda_i$,  $i = 1, 
\ldots , 5$ (см.\ рис.~8). Введем расхождение~$\rho$ между двумя 
кривыми~$\gamma$ и~$\lambda$ на плоскости как
     $$
     \rho(\gamma , \lambda ) =\int\limits_{\gamma} 
d(\gamma(t),\lambda)\,d\gamma+\int\limits_\gamma 
d(\lambda(t),\gamma)\,d\lambda\,.
     $$
Здесь $d(a,\lambda)=\underset{b\in\lambda}{\min} \vert a-b\vert^2$ задает 
расстояние от точки~$a$ до кривой~$\lambda$. Тогда функцию сходства 
$\mu(\theta , S_1, S_2)$ шарнирной модели с наблюдаемыми силуэтами введем 
как $\mu(\theta ,S_1, S_2) =\mu_1(\theta, S_1)+\mu_2(\theta ,S_2)$, где 
$\mu_1(\theta,S_1)=$\linebreak $=\sum\limits_{i=1}^5\left(\rho(\gamma_i,\lambda_i)+\alpha \vert 
A_i-B_i\vert^2\right)$ задает сходство по первой камере, а~$\mu_2$ 
определяется аналогичным образом для второй камеры. Второй член 
$\alpha\vert A_i-B_i\vert^2$ учитывает расхождение в концевых точках. 

\noindent
\begin{center} %fig8
\vspace*{12pt}
\mbox{%
\epsfxsize=79.684mm
\epsfbox{cis-8.eps}
}
\end{center}
\vspace*{3pt}
%\begin{center}
{{\figurename~8}\ \ \small{Сегментация скелета~(\textit{а}) и сегментация проекции осей шарнирной 
модели~(\textit{б})}}
%\end{center}
\vspace*{3pt}

\bigskip
\addtocounter{figure}{1}

     
     Полученная функция сходства обладает свойством унимодальности. Для 
простоты реализации использовалась квазиньютоновская схема LBFGS с 
численным градиентом. Градиент вычислялся разностной схемой по двум 
точкам. На рис.~\ref{f9ts} показан результат подгонки, спроецированный в 
разных ракурсах. Эксперименты показывают, что шарнирная модель хорошо 
описывает позу человека.

\begin{figure*} %fig9
\vspace*{1pt}
\begin{center}
\mbox{%
\epsfxsize=100.11mm
\epsfbox{cis-9.eps}
}
\end{center}
\vspace*{-6pt}
\Caption{Полученная шарнирная модель в разных ракурсах
\label{f9ts}}
\end{figure*}
      
\section{Заключение}

     В работе рассмотрена задача восстановления позы локально 
симметричного объекта на примере фигуры человека в случае отсутствия 
окклюзии в силуэтах. Рассматриваются две модели фигуры. В~первом случае 
фигура описывается как объединение пространственных жирных линий и 
предлагается метод ее восстановления. Во втором случае фигура описывается в 
виде шарнирной модели и предлагается метод ее подгонки по стереопаре 
силуэтов. В~обоих случаях эксперименты показали положительный результат. 
В~будущем планируется разработать методы подгонки шарнирной модели в 
случае окклюзии. Скелетное представление силуэта позволяет анализировать 
окклюзию, что может быть использовано для построения начального 
приближения в задаче подгонки.
     
     \bigskip
     Автор выражает благодарность своему научному руководителю 
профессору кафедры интеллектуальных систем Московского 
     физико-технического института Местецкому Леониду Моисеевичу за 
помощь в постановке задачи и внимание к работе.

{\small\frenchspacing
{%\baselineskip=10.8pt
\addcontentsline{toc}{section}{Литература}
\begin{thebibliography}{99}

\bibitem{10ts} %1
\Au{Agarwal A., Triggs B.}
Recovering 3D human pose from monocular images~// Pattern Analysis and 
Machine Intelligence, 2006.

\bibitem{7ts} %2
\Au{Tong M., Liu Yu., Huang T.\,S.}
3D human model and joint parameter estimation from monocular image~// Pattern 
Recognition Letters, 2007. Vol.~28. P.~797--805.

\bibitem{8ts} %3
\Au{Balan A.\,O., Sigal L., Black M.\,J., Davis J.\,E., Haussecker~H.\,W.}
Detailed human shape and pose from images~// IEEE Conference on Computer 
Vision and Pattern Recognition, 2007.
  
\bibitem{1ts} %4
\Au{Yezzi A.\,J., Soatto S.}
Structure from motion for scenes without features~// 2003 IEEE Computer Society 
Conference on Computer Vision and Pattern Recognition (CVPR'03) Proceedings. 
Vol.~1. P.~525--532.

\bibitem{2ts} %5
\Au{Senior A.} Real-time articulated human body tracking using silhouette 
information~// IEEE Workshop on Visual Surveillance/PETS Proceedings. France, 
2003.

\bibitem{3ts} %6
\Au{Cumani A. Guiducci A.}
Recovering the 3D structure of tubular objects from stereo silhouettes~// Pattern 
Recognition, 1997. Vol.~30. No.\,7. P.~1051--1059.

\bibitem{4ts} %7
\Au{Местецкий Л.\,М., Щетинин Д.\,В.}
Объемные примитивы Безье~// Графикон-2001: Тр.\ 11-й международной 
конф.~--- Нижний Новгород, 2001. С.~161--165.

\bibitem{11ts} %8
\Au{Mestetskiy L.}
Shape comparison of flexible objects~//  Conference (International) on Computer 
Vision Theory and Applications, 2007.

\bibitem{5ts} %9
\Au{Местецкий Л.\,М.}
Непрерывная морфология бинарных изображений: фигуры, скелеты, 
циркуляры.~--- М.: Физматлит, 2009.

\bibitem{12ts} %10
\Au{Местецкий Л.\,М.}
Непрерывный скелет бинарного раст\-ро\-во\-го изображения~// Графикон-98: Тр. 
международной конф.~--- М.: МГУ, 1998. С.~71--78. 

\bibitem{13ts} %11
\Au{Форсайт Д., Понс Ж.}
Компьютерное зрение.~--- М.: Вильямс, 2004.

\bibitem{6ts} %12
<<Kung-Fu Girl>>~--- a synthetic test sequence for multi-view reconstruction and 
rendering research. {\sf http://www.\linebreak mpi-inf.mpg.de/departments/irg3/kungfu}.


\label{end\stat}

\bibitem{9ts}
\Au{Sminchisescu C., Triggs B.}
Kinematic jump processes for monocular 3D human tracking~// 2003 IEEE 
Computer Society Conference on Computer Vision and Pattern Recognition 
(CVPR'03) Proceedings. Vol.~1. P.~69--76.

 \end{thebibliography}
}
}


\end{multicols} %3
\def\stat{krylov}

\def\tit{МОДЕЛИРОВАНИЕ И КЛАССИФИКАЦИЯ МНОГОКАНАЛЬНЫХ ДИСТАНЦИОННЫХ
ИЗОБРАЖЕНИЙ С~ИСПОЛЬЗОВАНИЕМ КОПУЛ$^*$}

\def\titkol{Моделирование и классификация многоканальных дистанционных
изображений с~использованием копул}

\def\autkol{В.\,А.~Крылов}
\def\aut{В.\,А.~Крылов$^1$}

\titel{\tit}{\aut}{\autkol}{\titkol}

{\renewcommand{\thefootnote}{\fnsymbol{footnote}}\footnotetext[1]
{Работа выполнена при поддержке ФЦП <<Научные и научно-педагогические кадры инновационной России>> 
на 2009--2013~гг.}}

\renewcommand{\thefootnote}{\arabic{footnote}}
\footnotetext[1]{Московский государственный университет им.\ М.\,В.~Ломоносова, факультет 
вычислительной математики и кибернетики, vkrylov@cs.msu.ru}

\Abst{Предложен метод моделирования многоканальных дистанционно полученных изображений (со спутника, самолета) с использованием копул.
    Суть предлагаемого подхода состоит в применении традиционных статистических моделей для моделирования вероятностных распределений отдельных каналов
    и построении совместного распределения для многоканального изображения при помощи копул.
    Рассмотрено также применение разработанного подхода в модели марковского случайного поля 
    (МСП) для байесовской классификации.
    Эксперименты с реальными изображениями, полученными радаром с синтезированной апертурой, демонстрируя результаты высокой точности, 
    указывают на
    преимущества предложенного подхода по сравнению с существующими методами.}

\KW{многоканальное изображение; копула; марковское случайное поле; байесовская классификация}

       \vskip 14pt plus 9pt minus 6pt

      \thispagestyle{headings}

      \begin{multicols}{2}

      \label{st\stat}


\section{Введение}


Статистический анализ изображений является на сегодняшний день активно развивающейся областью. 
Одним из важных и активно исследуемых источников изображений
служат спутники и радары, предоставляющие данные, широко исполь-\linebreak зуемые в задачах картографии, 
риск-менеджмен-\linebreak та (пожары, наводнения), эпидемиологии и~др.\linebreak
Современное оборудование позволяет получать многоканальные изображения, число каналов в 
которых зачастую доходит до сотен (мультиспектральные изображения).
Для обработки таких данных требуется адаптация существующих методов 
обработки отдельных каналов для работы с многоканальными изображениями. В~данной работе
рассматривается возможный статистический подход к решению подобной задачи многомерного моделирования.

На практике для решения задач требуется иметь аккуратную статистическую модель рас\-смат\-ри\-ва\-емых данных, 
в случае многоканальных изоб\-ра\-жений~--- многомерную модель. Большинство сущест\-ву\-ющих в литературе подходов 
строится в предположении, что вектор многоканальных данных имеет какое-то специфическое распределение,
например подчиняется многомерному нормальному распределению~\cite{Landgrebe} для мультиспектрального 
вектора или многомерному распределению Вишарта (Wishart)~\cite{2DNG} для комплексного мультиполяризованного 
микроволнового изображения. Недостаток подобных подходов состоит в большой потере точности при
повышении размерности из-за присутствия шума, погрешностей регистрации данных и аппроксимации. 
В~рамках статистического подхода адекватным способом решения данной
проблемы является использование более гибких моделей для построения многомерных распределений. 
Одной из таких моделей являются копулы~\cite{Nelsen}.

В работе также рассматривается одна из клас\-сиче\-ских проблем обработки дистанционно полученных изображений~--- 
задача классификации.\linebreak
Многомерные вероятностные распределения применяются совместно с моделью марковского случайного поля для 
байесовской классификации. Эксперименты с реальными изображениями, полученными радаром с синтезированной 
апертурой, демонстрируют преимущества модели с использованием копул.

Статья организована следующим образом. В~разд.~2 приводится изложение основ теории копул. В~разд.~3 
строится модель многомерных распределений с использованием
копул. В~разд.~4 излагается построение модели марковского случайного поля в задаче классификации. В~разд.~5 
приводятся эксперименты по применению разработанной модели к изображениям, полученным радаром с синтезированной 
апертурой. Раздел~6 содержит заключение.


\section{Копулы}

В этом разделе приводится краткий обзор тео\-рии двумерных копул. Все сформулированные ниже результаты 
могут быть обобщены на многомерный случай~\cite{Nelsen}.

Двумерная копула является вероятностным распределением на $[0, 1]^2$ таким, что маргинальные распределения 
распределены равномерно на~$[0,1]$.

\smallskip

\noindent
\textbf{Определение.} \textit{Двумерной копулой называется отображение $C: [0,1]^2 \to [0,1]$, такое что:}
\begin{enumerate}[1)]
\item \textit{оба маргинальных распределения являются равномерно распределенными с.в.\ на $[0,1]$;}
\item
$\forall u, v \in [0,1]$: $C(u,0) = C(0,v) = 0$ и $C(u,1)=$\linebreak $=u,\,C(1,v) = v$;
\item
$\forall u_1 \leq u_2$, $v_1 \leq v_2 \in [0,1]$: $C(u_2,v_2) - C(u_1,v_2)\;-$\linebreak $-\;C(u_2, v_1) + C(u_1, v_1) \geq 0$.
\end{enumerate}

\smallskip

Теоретическое обоснование правомерности использования копул в прикладных задачах обеспечивает 
теорема Склара~\cite{Nelsen}:

\smallskip

\noindent
\textbf{Теорема.} \textit{Пусть $X$, $Y$~--- произвольные случайные величины (с.в.)
с совместным распределением~$H(x,y)$ и маргинальными функциями
распределения~$F$ и~$G$. Тогда существует копула~$C$ такая, что
\begin{equation}
\label{Sklar}
H(x,y) = C\left(F(x), G(y)\right)
\end{equation}
$\forall x, y \in \mathbb R$. Если~$F$ и~$G$ непрерывны, то такая копула~$C$ единственна.}

\smallskip

Если $X$, $Y$ имеют плотности $f(x)$ и $g(y)$, то плотность совместного распределения~(\ref{Sklar}) можно представить в виде
\begin{equation}\label{Sklar_pdfs}
h(x,y) = f(x)g(y)\frac{\partial^2 C}{\partial x \partial y}(F(x),G(y)),
\end{equation}
где частная производная $\frac{\partial^2 C}{\partial x  \partial y}(x,y)$ задает плотность копулы $C(x,y)$.

На практике широкое применение получил класс архимедовых копул~\cite{Nelsen}.

\smallskip

\noindent
\textbf{Определение.} \textit{Архимедовой копулой называется копула $C$ вида}
$$
C(u_1, u_2) = \phi^{-1}(\phi(u_1) + \phi(u_2))\,,
$$
\textit{где функция $\phi(u)$, называемая \textit{функцией-ге\-не\-ра\-то\-ром}, удовлетворяет следующим требованиям:
(1)~$\phi(u)$ непрерывна на $[0,1]$;\quad (2)~$\phi(u)$ монотонно убывает, $\phi(1)=0$; (3)~$\phi(u)$ выпуклая.}

\smallskip

В рассматриваемых задачах моделирования и классификации
дистанционных изображений предлагается использовать следующие
копулы: пять архимедовых копул (Clayton, Ali-Mikhail-Haq, Gumbel,
Frank, A12)~\cite{Nelsen} и одну неархимедову копулу, содержащую
абсолютно непрерывную и сингулярную компоненту
(Marchal--Olkin)~\cite{Nelsen}. Такой набор копул обеспечивает
достаточно широкий выбор моделируемых структур
зависимости~\cite{Huard}, тем не менее для некоторых рассматриваемых
данных этот набор копул может требовать пополнения. Эксперименты с
более обширным набором копул проводились в~\cite{KrylovRR09}.
Информация о рассматриваемых копулах собрана в табл.~1.

\begin{table*}\small
\begin{center}
\Caption{Рассматриваемые копулы: Clayton,
Ali-Mikhail-Haq (AMH), Gumbel, Frank, A12 и Marchal--Olkin (Marchal)
\label{CopulasTheta}}
\vspace*{2ex}

\tabcolsep=10.5pt
\begin{tabular}{|l|l|l|l|}
\hline
Копула &  \multicolumn{1}{c|}{$C(u,v)$} & 
\multicolumn{1}{c|}{$\theta(\tau)$} & \multicolumn{1}{c|}{$\tau$-интервал}\\ 
\hline
Clayton     &
$(u^{-\theta}+v^{-\theta}-1)^{-1/\theta}$ &
$\theta = \fr{2\tau}{1-\tau}$ &
$\tau\in(0,1]$ \\
\hline
&&&\\[-10pt] 
AMH & $\fr{uv}{1-\theta(1-u)(1-v)}$ &
$\tau = \fr{3\theta-2}{3\theta} - \fr{2(1-\theta)^2}{3\theta^2}\ln \left(1-\theta\right)$ &
$\tau\in\left[-0{,}1817, \fr{1}{3}\right]$ \\ 
\hline
&&&\\[-10pt] 
Gumbel & $\exp\left(-\left[ \left(-\ln\left(u\right)\right)^{\theta} + (-\ln\left(v\right))^{\theta} 
\right]^{1/{\theta}}\right)$ & $\theta = \fr{1}{1-\tau}$ & $\tau\in[0,1]$
\\ 
\hline
Frank & $- \fr{1}{\theta}\log\left(1+\fr{(e^{-\theta u}-1)(e^{-\theta v}-1)}{e^{-\theta}-1}\right)$ &
$\tau = 1 - \fr{4}{\theta^2}\int\limits_0^{\theta}\fr{t}{e^{-t}-1}\,dt$ &
$\tau\in[-1,1]\setminus\{0\}$\\ 
\hline
A12      & $\left(1+\left[ (u^{-1}-1)^{\theta} + (v^{-1}-1)^{\theta} \right]^{1/{\theta}}\right)^{-1}$ &
$\theta = \fr{2}{3-3\tau}$ &
$\tau\in[\fr{1}{3},1]$ \\
 \hline
Marchal & $\min\left( u^{1-\theta}v, uv^{1-\theta}\right)$ &
$\theta = \fr{2\tau}{\tau+1}$ &
$\tau\in[0,1]$\\ 
\hline
\end{tabular}
\end{center}
\vspace*{-4pt}
\end{table*}


Простейшим инструментом для оценки копул является \textit{коэффициент ранговой корреляции Кендалла}~$\tau$ 
двух независимых реализаций $(Z_1, Z_2)$ и $(\hat{Z_1}, \hat{Z_2})$ с общим законом
распределения~$H(x,y)$: $\tau =  \mathbb P\{(Z_1 - \hat{Z_1})(Z_2 -
\hat{Z_2}) > 0\} -  \mathbb P\{(Z_1 - \hat{Z_1})(Z_2 - \hat{Z_2}) < 0\}$. 
На практике при наличии реализаций~$z_{1,l}$ и~$z_{2,l}$,
$l=1,\ldots,N$, эмпирической оценкой коэффициента Кендалла
является~\cite{Nelsen}:
\begin{equation*}
%\label{TauHat}
\hat{\tau} =  \fr{\sum_{l=1}^{N-1}\sum_{k=l+1}^{N}z_{1,lk}\:z_{2,lk}}{C_2^N}\,,
\end{equation*}
где
$$
z_{n,lk} = \begin{cases}
1\,, &\mbox{если}\  z_{n,l} \leq z_{n,k}\\
-1&\mbox{---~иначе}
\end{cases}\,,\enskip n=1,2\,.
$$

Интегрируя в определении~$\tau$ по $(\hat{Z_1}, \hat{Z_2})$, имеем
\begin{equation}
\left.
\begin{array}{rl}
\tau &= 4\displaystyle\int\limits_0^1\int\limits_0^1 C(u,v)\,dC(u,v)-1\,;\\[9pt]
\tau &= 1 + 4\displaystyle\int\limits_0^1\fr{\phi_A(t)}{\phi_A{'}(t)}\,dt\,,
\end{array}
\right \}
\label{TauC}
\end{equation}
где вверху получен общий вид зависимости между~$\tau$ и~$C$, а внизу~--- 
зависимость в частном случае архимедовых копул с
функ\-ци\-ей-ге\-не\-ра\-то\-ром~$\phi_A(t)$. Все рассматриваемые в работе
копулы однопараметрические, оценка их параметра~$\theta$ может быть
получена из~(\ref{TauC}) (см.\ табл.~1).

\section{Моделирование совместных распределений}


Рассматривается задача построения~$M$ условных распределений
изображения с $D$ каналами. В~центре внимания в данной работе
находится моделирование совместных распределений, поэтому
предполагается что маргинальные распределения
$p_{dm}(y_d|\omega_m)$, $d=1,\ldots,D$, $m=1,\ldots,M$, где
$\omega_m$~--- событие, состоящее в попадании наблюдения в класс с
номером~$m$, уже получены каким-либо методом (включая оценку
параметров). Таким образом, обобщая~(\ref{Sklar_pdfs}), имеем
совместную плотность вида
\begin{multline}
p_m( {\bf y}|\omega_m) = p_{1m}( y_1|\omega_m)\ldots\\
\ldots p_{Dm}( y_D|\omega_m)
\fr{\partial^D C_m^*}{\partial y_1 \ldots \partial y_D}(F_{1m}(y_1),\ldots\\
\ldots , F_{Dm}(y_D))\,.
\label{Model3}
\end{multline}

Следующим вопросом является выбор конкретной копулы для каждого класса. 
В~литературе\linebreak предложен целый ряд методов для решения этой проблемы. Так, для этого
применяются подходы, основанные на применении информационных критериев (в частноcти, 
Акаике)~\cite{Nelsen}. В~\cite{Kplots}
разработан метод для визуального анализа (K-plots) адекват\-ности копулы реальным данным. 
В~\cite{Huard} применяется функция правдоподобия.
В данной работе предлагается использовать критерий согласия~$\chi^2$ Пирсона (КСП) для 
нахождения наиболее подходящей копулы.

Итак, для выбора копулы~$C_m^*$ из~(\ref{Model3}) среди копул в
табл.~1 сначала отбрасываются копулы, эмпирическое $\hat{\tau}_m$
которых лежит вне соответствующего им $\tau$-интервала. Затем для
оставшихся копул оценивается значение параметра~$\theta$ и
выбирается копула, лучше всего согласующаяся с наблюдениями на
основе КСП. Нулевая гипотеза КСП состоит в согласии наблюдаемых
частот с частотами, предсказанными теоретической моделью. Статистика
КСП имеет следующий вид:
$$
X^2 = \sum\limits_{i=1}^N \fr{(O_i - E_i)^2}{E_i}\,,
$$
где $O_i$ и $E_i$~--- наблюдаемые и предсказанные частоты; $n$~--- число классов.
$P$-значение КСП определяется из сравнения наблюдаемого значения~$X^2$ с 
$\chi^2$-рас\-пре\-де\-ле\-ни\-ем: $X^2\sim\chi_{n-r-1}^2$, где $r$~--- количество ограничений числа степеней свободы
(количество параметров модели).

\section{Классификация многоканальных изображений}


На базе построенных условных распределений~(\ref{Model3}) самую примитивную классификацию 
можно получить, используя метод максимального\linebreak правдоподобия,
выбирая из $M$ классов для каж\-до\-го пикселя тот, который имеет максимальную вероятность.
Однако такого рода классификация слишком неоднородная, и для решения этой проб\-ле\-мы к правдоподобию добавляется
регуляризи\-рующее слагаемое, обеспечивающее б$\acute{\mbox{о}}$льшую однородность решения. 
Одним из таких способов\linebreak регуляризации является модель МСП~\cite{BesagMRF}. 
Концепция МСП получена обобщением марковского свойства на двумерный случай решетки. 
Классическая теория МСП и скрытых МСП приводится в~\cite{BesagMRF}.
Здесь изложение ограничится кратким введением в скрытые~МСП.

\begin{figure*}[b] %fig1
\vspace*{1pt}
\begin{center}
\mbox{%
\epsfxsize=163.235mm
\epsfbox{kry-1.eps}
}
\end{center}
\vspace*{-6pt}
\Caption{Исходное изображение~(\textit{a}), 
классификация c 2D моделью Накагами--Гамма~(\textit{б}) и классификация с моделированием копулами~(\textit{в}): 
вне карты классификации~--- белое, правильно классифицированная вода~(\textit{1}),
влажная почва~(\textit{2}), сухая почва~(\textit{3}), 
ошибочная классификация~--- черное
\label{f1kr}}
\end{figure*}

Задачу классификации относят к классу задач с неполными данными
$x=(y,z)$, где $y$~--- наблюдаемые данные (исходное изображение, поле~$Y$);
$z$~--- данные, подлежащие восстановлению (метки классов, поле~$Z$). 
В~решении задачи классификации используется модель скрытого
МСП. В этой модели неизвестные метки~$z_i$ предполагаются МСП, т.\,е.\
зависящими от меток \textit{только} соседних пикселей. Теорема
Хам\-мерс\-ли--Клиф\-фор\-да (Hammersley--Clifford)~\cite{BesagMRF}
предоставляет удобное представление для совместного распределения
с.~в., входящих в МСП, в виде распределения Гиббса
\begin{multline}
\label{Gibbs}
P_{G}(z) = W^{-1}\exp(-H(z))\\
\mbox{с энергией}\quad
H(z) = \sum\limits_{c \in C} V_c(z_c)\,,
\end{multline}
где $W$~--- нормирующая константа, $V_c(z_c)$~--- потенциалы;
$C$~--- система множеств (клик) на решетке $S$~\cite{BesagMRF}. В~данной
работе рассматриваются двухместные анизотропные потенциалы (т.\,е.\
попарно с каж\-дым из 8 соседних пикселей):
\begin{equation}
\label{Beta}
H(z|\beta) = 
\sum\limits_{c}V(z_c|\beta) = \!\!\sum\limits_{c=\{s, s'\}\in C} \left[ - \beta\,\delta_{z_s = z_{s'}}\right]\,,\!\!
\end{equation}
где $\delta_{z_s = z_{s'}} = 1,$ если $z_s = z_{s'}$, и 0~--- иначе.
В~(\ref{Beta}) $\beta$ играет роль веса, т.\,е.\ чем больше значение
$\beta$, тем выше вклад регуляризирующего слагаемого $H(z|\beta)$ в
суммарную энергию поля~$(Y,Z)$, что хорошо видно во втором уравнении
системы~(\ref{Energy}), приведенной ниже.

В свою очередь, наблюдения~$y_i$ предполагаются условно независимыми 
$P(y_i|y_{\Omega\setminus\{i\}},z_i) = P(y_i|z_i)$.
Таким образом, при наличии условных вероятностей $p_m( {\bf y}|\omega_m)$ распределение скрытого МСП $(Y,Z)$ 
на~$S$ имеет вид:
\begin{equation}
\label{Energy}
\left. 
\begin{array}{l}
P_{G}(z) = W^{-1}\exp(-U(\omega_m|{\bf y},\beta))\,;\\[9pt]
U(\omega_m|{\bf y},\beta) =\\[9pt]
= \sum\limits_{i\in S} \left[ - \text{ln} p_m( {\bf y}|\omega_m) - %\right.{}\\
%&\left.{}- 
\beta\,\sum\limits_{s:\{i,s\}\in C} \delta_{z_i = z_s}\right]\,,
\end{array}
\right \}
\end{equation}
где первое уравнение определяет энергию~$U(\cdot)$ аналогично~(\ref{Gibbs}), 
а второе уравнение задает~$U(\cdot)$ как сумму энергетических вкладов полей~$Y$ и~$Z$.

Для оценки параметра в~(\ref{Energy}) используется метод имитации отжига~\cite{MRFaccelerated}. 
Для нахождения точки минимума функции
$U(\omega_m|{\bf y},\beta)$ (и, соответственно, решения с максимальным правдоподобием~$P_{G}$)
необходимо решение задачи глобальной оптимизации, например тем же методом имитации отжига, 
однако в связи с высокой вычислительной сложностью для экспериментов
использовалась локальная оптимизация методом ICM~\cite{Besag}.

\vspace*{-12pt}

\section{Эксперименты}

Для проведения экспериментов используется амплитудное изображение,
полученное радаром с синтезированной апертурой системы TerraSAR-X
(\copyright Infoterra), с двумя каналами поляризации и разрешением 6~м на пиксель 
(рис.~\ref{f1kr},\,\textit{a}). Классификация проводится по трем
классам: вода, влажная и сухая почва. В~качестве способа оценки
маргинальных распределений классов применяется алгоритм с
использованием метода конечных смесей~\cite{KrylovSPIE09} на
обучающем изображении с заранее предоставленной классификацией.
Стоит отдельно отметить, что копулы представляются логичным
инструментом для обобщения~\cite{KrylovSPIE09} на многоканальный
случай, не требующим изменения самого метода. Для сравнения также
приводятся результаты классификации при моделировании совместного
распределения моделью двумерного (2D) На\-ка\-га\-ми--Гам\-ма
распределения~\cite{2DNG}.



Доля правильной классификации на рис.~\ref{f1kr} составляет 87,1\% для 2D
На\-ка\-га\-ми--Гам\-ма модели (рис.~\ref{f1kr},\,\textit{б}) и 94,8\% для предлагаемой модели с
использованием копул (рис.~\ref{f1kr},\,\textit{в}). При помощи КСП были выбраны
следующие копулы: Gumbel~--- для воды и Frank~--- для сухой и влажной
почвы. Такое повышение качества классификации возможно благодаря
более точному моделированию совместных распределений,
предоставляемому копулами. Дополнительные эксперименты и обсуждение
можно найти в~\cite{KrylovRR09}.

\section{Заключение}


В работе предлагается подход к моделированию распределений многоканальных изображений с использованием копул.
По сравнению с классическими методами оценки многомерных распределений конкретного вида, использование копул 
предос\-тав\-ля\-ет большую гибкость, позволяя получать более аккуратные модели.
Предложенная модель применяется для классификации дистанционно полученных изображений.
Эксперименты с изображениями, полученными радаром с синтезированной апертурой, подтверждают 
высокую описательную точность предложенного подхода.

\smallskip
Использованное в работе изображение системы TerraSAR-X распространяется на свободной 
основе (\copyright Infoterra, {\sf www.infoterra.de}).

\bigskip
Автор выражает благодарность своему научному руководителю доц.\
В.\,Ф.~Матвееву , а также проф.\ Г.~Мозеру, проф.\ С.~Серпико и проф.\
Д.~Зерубии за помощь в подготовке этой работы.


{\small\frenchspacing
{%\baselineskip=10.8pt
\addcontentsline{toc}{section}{Литература}
\begin{thebibliography}{99}

\bibitem{Landgrebe}
\Au{Landgrebe D.\,A.}
Signal theory methods in multispectral remote sensing.~--- N.-Y.: Wiley, 2003.

\bibitem{2DNG}
\Au{Lee J.-S., Hoppel K.\,W., Mango S.\,A., Miller A.\,R.}
Intensity and phase statistics of multilook polarimetric and
interferometric SAR imagery~// IEEE Trans. Geosci. Remote Sens.,
1994. Vol.~32. No.\,10. P.~1017--1028.

\bibitem{Nelsen}
\Au{Nelsen R.\,B.}
An introduction to copulas.~--- N.-Y.: Springer, 2007.

\bibitem{Huard}
\Au{Huard D., $\acute{\mbox{E}}$vin G., Favre A.-C.} Bayesian copula
selection~// Comput. Stat. Data Anal., 2006. Vol.~51. No.\,2.
P.~809--822.

\bibitem{KrylovRR09}
\Au{Krylov V., Zerubia J.}
High resolution SAR image classification.
INRIA, Research Report 7108, 2009.

\bibitem{Kplots}
\Au{Genest C., Boies J.-C.} Detecting dependence with Kendall
plots~// The American Statistician, 2003. Vol.~57. No.\,4.
P.~275--284.

\bibitem{BesagMRF}
\Au{Besag J.} Spatial interaction and the statistical analysis
of lattice systems~// J. Roy. Statistical Soc. B,
1974. Vol.~36. No.\,2. P.~192--236.

\bibitem{MRFaccelerated}
\Au{Yu Y., Cheng Q.} 
MRF parameter estimation by an accelerated method~// Pattern Recogn. Lett., 2003. Vol.~24.
No.\,9--10. P.~1251--1259.

\bibitem{Besag}
\Au{Besag J.} On the statistical analysis of dirty pictures~//
J.~Roy. Statistical Soc. B, 1986. Vol.~48. P.~259--302.

 \label{end\stat}

\bibitem{KrylovSPIE09}
\Au{Krylov V., Moser G., Serpico S., Zerubia J.}
Dictionary-based probability density function estimation for
high-resolution SAR data~// SPIE Proceedings.~--- San Jose, USA,
2009. Vol.~7246. P.~72460S-1--72460S-12.
 \end{thebibliography}
}
}


\end{multicols}  %4
\def\stat{kuzn}

\def\tit{ВЕРОЯТНОСТНО-СТАТИСТИЧЕСКАЯ ОЦЕНКА 
АДЕКВАТНОСТИ ИНФОРМАЦИОННЫХ ОБЪЕКТОВ}

\def\titkol{Вероятностно-статистическая оценка 
адекватности информационных объектов}

\def\autkol{Л.\,А.~Кузнецов}
\def\aut{Л.\,А.~Кузнецов$^1$}

\titel{\tit}{\aut}{\autkol}{\titkol}

%{\renewcommand{\thefootnote}{\fnsymbol{footnote}}\footnotetext[1]
%{Работа поддержана Российским фондом фундаментальных исследований
%(проекты 11-01-00515а и 11-07-00112а), а также Министерством
%образования и науки РФ в рамках ФЦП <<Научные и
%научно-педагогические кадры инновационной России на 2009--2013~годы>>.}}


\renewcommand{\thefootnote}{\arabic{footnote}}
\footnotetext[1]{Липецкий государственный технический университет, kuznetsov@stu.lipetsk.ru}


\Abst{Приведены математические основы и оригинальная методология разработки 
систем оценки семантической близости информационных объектов (ИО), представленных на 
естественном языке. Вводится ве\-ро\-ят\-но\-ст\-но-ста\-ти\-сти\-че\-ское представление 
сопоставляемых ИО. Используется теория информации для 
оценки уровня семантической близости ИО. Методология 
доведена до алгоритмов ее реализации в виде соответствующей автоматизированной 
системы. Представлены результаты практической проверки эффективности методологии. 
}

\vspace*{2pt}

\KW{информационные объекты; естественный язык; семантическая адекватность; 
вероятностная модель; теория информации}

\vspace*{6pt}

 \vskip 14pt plus 9pt minus 6pt

      \thispagestyle{headings}

      \begin{multicols}{2}
      
            \label{st\stat}

\section{Введение. Информация, знания, семантический анализ}
   
   Основным мотивом перехода от индустриальной к постиндустриальной 
модели развития в промышленно развитых странах, начавшегося в конце 
прошлого века, является стремительное увеличение скорости развития науки 
и знания. В~постиндустриальной модели развития, по мнению ведущих 
западных ученых в области социального развития и управления, 
принципиальным является изменение статуса и значения информации, науки 
и знания, которые становятся важнейшими факторами, определяющими 
эволюцию общества. 
   
   Питер Ф.~Друкер, признанный специалист в области организационного 
управления, пишет: <<Изменение значения знания, начавшееся 250~лет тому 
назад, преобразовало общество и экономику. Знание стало сегодня основным 
условием производства. Традиционные <<факторы производства>>~--- земля 
(природные ресурсы), рабочая сила и капитал~--- не исчезли, но приобрели 
второстепенное значение. Эти ресурсы можно получить, причем без особого 
труда, если есть необходимые знания>>~\cite{1-k}. 
   
   Информация и знания становятся главной движущей силой 
экономического развития и перехо-\linebreak дят из категории бесплатного 
общественного бла-\linebreak га в категорию товара. В~промышленно развитых\linebreak \mbox{странах} 
разработка и внедрение технологических инноваций~--- решающий фактор 
социального и экономического развития, залог экономической безопасности. 
В~США, по оценкам американских специалистов, прирост душевого 
национального дохода благодаря этому фактору составляет 90\%. 
   
   Беспрецедентный рост потока информации и знаний, скорости их 
передачи и возможностей доступа на первый план научных проблем 
выдвигает разработку технологий их автоматической обработки. Б$\acute{\mbox{о}}$льшая 
часть существующих и вновь формируемых знаний и информации 
представлена на естественном языке. 

Одной из актуальных, 
фундаментальных проблем в области обработки информации становится 
обеспечение возможности формального семантического сравнения, оценки 
семантической \mbox{бли\-зости} ИО, представленных 
на естественном языке. Разработка формализованных технологий оценки 
семантической близости ИО позволила бы перейти к практической 
реализации важных задач в сфере обработки информации, распространения 
знаний и образования. 
   
   В настоящее время в литературе задача семантического сравнения двух 
текстов, в основном, рас\-смат\-ри\-ва\-ется в контексте дубликатов в 
   веб-докумен\-тах и в системах автоматизированного перевода. При 
поиске дубликатов опираются на число слов, совпавших в двух текстах. 
Алгоритм сравнения на основе шинглов является наиболее простым и 
распространенным. Такой подход используется для нахождения копий 
текстов, полученных копированием и перестановкой слов, но он не позволяет 
оценить семантическую близость текстов.
   
   При более сложном анализе текстов учитывается структура входящих в 
них предложений. В~предложениях выделяются элементы (слова или группы 
слов) и сопоставляются определенные шаблоны для этих элементов. Данный 
подход описан в книге~\cite{2-k} и используется в ряде кандидатских 
диссертаций. 
   
   Однако задача оценки информационной бли\-зости двух текстов в 
обнаруженных автором работах не затрагивается. Используемые там 
концепции не ориентированы на ее решение и не могут быть использованы в 
качестве основы для ее решения.
   
   В данной статье предлагается оригинальная методология оценки 
информационной близости текстов на основании вероятностно-ста\-ти\-сти\-че\-ско\-го 
подхода и теории информации. Реализация методологии 
позволит перейти к практическому решению разнообразных задач, в которых 
требуется определять меру информационной адекватности\linebreak документов, 
представленных на естественном языке. В~част\-ности, разработанная 
концепция будет использована для синтеза автоматизированных сис\-тем 
оценки уровня знаний. 

\section{Формализация анализа текстов}

   Развитие информационных технологий, предо\-став\-ля\-ющих широкие 
возможности ав\-то\-ма\-ти\-зи\-рован\-но\-го анализа и обработки вербально 
пред\-став\-лен\-ной информации, существенно повысило интерес к разработке 
формальных методов исследования и сопоставления текстов. Современные 
компьютерные системы позволяют хранить и обрабатывать практически 
неограниченные объемы текстовой информации. Это стимулирует 
разработку формальных методов для поддержки выполнения постоянно 
расширяющихся и углубляющихся исследований информации, 
представленной на естественном языке. В~настоящее время интенсивно 
разрабатываются формальные методы, позволяющие автоматизировать 
решение задач в области морфологического, синтаксического и 
семантического анализа текстовой информации. 
   
   Из имеющихся публикаций следует, что методологии различных видов 
анализа базируются на сходных концепциях и достаточно близки по своему 
содержанию. Формализация морфологического анализа направлена на 
алгоритмическое пред\-став\-ле\-ние грамматики русского языка. Час\-ти речи 
русского языка определены, однозначно определены формы, в которых они 
могут быть, определены правила, следуя которым должно осуществляться 
изменение слов, принадлежащих к различным час\-тям речи при образовании 
соответствующих форм. Значительная часть правил изменения частей речи 
уже отражена в словарях. Все правила могут быть представлены в виде 
соответствующих процедур, функций, подпрограмм и~т.\,п. В~соответствии 
с имеющимися правилами может быть идентифицировано, какой частью 
речи является конкретное слово, в какой форме оно находится. Поэтому 
может быть написана программа, обеспечивающая автоматизированное 
выполнение морфологического анализа, так что на ее вход будет поступать 
слово предложения, а на выходе она выдаст результат анализа: какой частью 
речи является данное слово и в какой форме оно находится. 
   
   Формализация синтаксического анализа~--- задача также понятная: 
существует синтаксис русского языка, представляющий свод правил, следуя 
которым могут быть достаточно четко определены члены предложения. Раз 
правила существуют, то их можно представить в виде набора процедур, 
обеспечивающих выявление состояния (роли) каждого слова в предложении. 
Правила и процедуры могут быть более или менее сложными, но 
принципиально то, что правила имеются, а следовательно, и процедуры 
могут быть синтезированы по ним. На основании этих процедур может быть 
разработана система синтаксического анализа, которая, получая на свой вход 
предложение, выполнит его разбор и анализ и на выходе, как примерный 
ученик выдаст о каждом слове, входящем в предложение, информацию: 
каким его членом оно является. 
   
   Проблема семантического анализа текстов интенсивно исследуется в 
различных аспектах: разрабатываются правила и алгоритмы анализа 
предложений, выявления их структуры, установления соответствия между 
разноязычными текстами, поиска информации и~т.\,д. При этом, однако, 
формализация семантического анализа ка\-ко\-го-ли\-бо одноязычного текста 
или даже одного предложения представляется задачей весьма малопонятной. 
   
   Под формализацией обычно понимается однозначное математическое 
представление существующих правил, которые, возможно, в текстовом, 
вербальном виде содержат определение способа извлечения нужных 
сведений из начальных, исходно заданных данных. В~контексте 
формализации семантического анализа математическому оформлению 
должны подлежать правила, позволяющие извлечь из слова его смысл. Но, в 
отличие от морфологии и синтаксиса, не существует ка\-ких-ли\-бо 
формальных семантических правил, следуя которым можно было бы 
установить смысл, вложенный в предложение или в каждое отдельное его 
слово. Поэтому невозможно представить систему семантического анализа в 
виде упорядоченного набора правил или предписаний, которая (по аналогии 
с системами морфологического или синтаксического анализа) получала бы 
на входе предложение или слово, а на выходе выдавала бы его смысл. Ибо 
слово и есть его смысл. В~толковом словаре, конечно, разъясняется смысл 
отдельных слов, но, в конечном итоге, это разъяснение представляет 
сопоставление одному слову других слов, близких по смыслу, и следует из 
словаря, а не из каких-либо правил, которые можно было бы формализовать.
   
Следовательно, если морфологический и синтаксический анализ действительно 
представляют анализ в соответствии со смыслом этого слова, т.\,е.\ разбор 
предложения на составляющие его элементы и выяснение их роли и 
состояния, то семантический анализ может пониматься только в смысле 
сравнения и выяснения смысловой близости разных слов и текстов. Только 
при наличии эталона анализируемого предложения, смысл которого 
известен, опираясь на словари, в которых отражена семантическая близость 
отдельных слов и словосочетаний, может быть получен ответ, что 
анализируемое предложение находится в некотором соответствии с эталоном 
и, следовательно, имеет определенный смысл. Например, при переводе 
смысл на язы\-ке-ори\-ги\-на\-ле принимается за известный эталон. С~помощью 
словаря, в котором имеется соответствие между словами и 
словосочетаниями, находится соответствующее выражение на другом языке. 
Важно понимать, что соответствие при этом следует не из слов, а из словаря. 
   
Таким образом, представляется, что при сопоставлении одноязычных текстов 
более правильно говорить не об их семантическом анализе, а об 
уста\-нов\-ле\-нии уровня их информационной адекватности, об определении 
взаимного количества информации, общего для сравниваемых текстов, из 
общего объема информации, содержащегося в одном из них, принимаемом за 
эталонный текст. 
   
   Автоматизированная технология должна обеспечивать реализацию 
функций определения пересечения, общей части дубля и эталона. Для этого в 
автоматизированной технологии должны быть разработаны формально-математические 
инструменты для представления текстов дубля и эталона в 
виде, позволяющем оценить количество информации, содержащейся в них, 
определить долю информации в дубле, отражающую содержание эталона, и 
на этой основе сформировать оценку их семантической близости. 
   
\section{Ограниченность возможностей детерминированного 
подхода}
   
   Имеется принципиальное базовое отличие семантического анализа от 
морфологического и синтак\-си\-че\-ско\-го. Отмеченное кратко выше показывает, 
что объектом морфологического и синтаксического анализа является 
фиксированный, \mbox{полностью} однозначно определенный текстовый фрагмент. 
Определение роли и состояния оборотов или отдельных слов, составляющих 
предложения анализируемого фрагмента, производится по четко 
определенным, детерминированным правилам, которые могут быть 
представлены в виде более или менее сложных алгоритмов, формирующих 
морфологические или синтаксические характеристики предложений, 
оборотов и слов. Процесс и правила формирования морфологических и 
синтаксических характеристик и сами характеристики определенны и 
закономерны. Поэтому эти виды анализа закономерны или 
детерминированы.
   
   На первый взгляд, кажется заманчивой идея представить текст эталона и 
дубля в виде предложений, предложения в виде деревьев или иных 
детерминированных структур и затем сравнить структурированное таким 
образом представление эталона и дубля. Однако русский язык, а здесь 
подразумевается, что именно он используется для вербального 
представления информации, совершенно игнорирует какие-либо 
структурные ограничения по расположению членов предложения, по виду 
предложений, формированию фрагментов предложений из групп слов, 
изобилует бесконечным многообразием форм управления отдельными 
словами и группами слов. По этой причине можно ожидать, что в эталоне и 
дубле не окажется тождественно равных структурных единиц, а чис\-ло 
альтернатив, подлежащих сравнению, может быть бесконечным. В~такой 
ситуации становится принципиальной проблема определения альтернатив и 
логических правил их разрешения. Поэтому представляется, что 
детерминированный семантический анализ не вполне соответствует 
содержательному существу проб\-лемы.
   
   По мнению автора, семантический анализ как термин не вполне удачен. 
Речь может идти об установлении уровня информационного соответствия 
содержания одного, анализируемого текста, который здесь именуется 
дублем, содержанию другого, именуемого эталоном, текста. Представляется, 
что реализация сравнения, выявления уровня соответствия нескольких 
русскоязычных текстов использованием детерминированного 
структурирования и детерминированных правил оценки близости не 
представляется практически возможной. С~учетом интеллектуальной 
специфики одна и та же информация может случайным образом облекаться в 
различную текстовую оболочку. Основная проб\-ле\-ма оценки степени 
близости информационных объектов, представленных на естественном 
языке, следует из семантической многозначности слов и наличия синонимов. 
Эти обстоятельства приводят к неоднозначности лексического представления 
семантического содержания текстов. Проблемы неоднозначности текстовой 
информации известны и активно исследуются специалистами в области 
русского языка. Детерминированный подход сравнения текстов оказывается 
нацеленным фактически на формальное описание тонкостей образования 
синтаксических форм русского языка, многообразие которых представляется 
бесконечным. 
   
   Очевидно, что случай полного совпадения\linebreak текстов, используемый при 
формировании рег\-ла\-мента доступа к информации, здесь не рас\-смат\-ри\-ва\-ет\-ся. 
Должна присутствовать возможность выделения общности 
информационного содержания со\-по\-став\-ля\-емых информационных объектов 
из их случайным образом выбранной формы представления на естественном 
языке. Решение такой задачи может быть получено только при описании 
взаимосвязи семантического (информационного) содержания и лексического 
оформления обоих срав\-ни\-ва\-емых текстов с вероятностно-статистических 
позиций.

\section{Элементы теории информации}

   Более полувека существует в виде научной дисциплины теория 
информации. Ее основоположником является американский специалист в 
об\-ласти передачи информации в технических линиях связи Клод 
   Шен\-нон~\cite{3-k}. Значительный вклад в теорию информации, 
особенно в строгое доказательство ее основных принципов, внесли советские\linebreak 
ученые школы А.\,Н.~Колмогорова~\cite{4-k}. В~теории информа\-ции 
исследуются проблемы передачи и преобразования информации, при этом 
вводится количественная оценка информации. Применительно к проблеме 
сравнения близости ИО, которой посвящена 
данная статья, является важным, что в теории информации разработаны 
теоретические основы исследования бли\-зости сообщений, переданного 
передатчиком и принятого приемником на другом конце линии связи. При 
этом вводится количественная мера информации, которая позволяет 
осуществить сопоставление информационной емкости переданного и 
принятого сообщений и на этой основе оценить искажение (потери) 
информации в линии связи при ее пе\-ре\-даче. 
   
   Теория информации может быть использована для решения проблемы 
оценки близости ИО, представленных на естественном языке. Ничто не 
мешает вместо сообщений, принятого и переданного, рассматривать 
ИО, трактуя один из них~--- аналог переданного 
сообщения~--- как эталонный информационный объект (ЭИО), а другой~--- 
аналог принятого сообщения~--- как дубль ЭИО (ДИО). 
Как будет видно дальше, мера количества информации в 
одном объекте о другом симметрична, поэтому при количественной оценке 
их близости не важно, какой объект считать эталоном, а какой~--- дублем. 
Понятно, что сравнение ЭИО и ДИО на абсолютное их совпадение 
неприемлемо. В~теории информации исследуется количественная, а не 
содержательная сторона информации. В~связи с наличием случайных помех 
в системах формирования и линиях передачи информации сообщения, 
переданное и принятое, интерпретируются случайными величинами~$\xi$. 
Шенноном было предложено использовать энтропию\footnote{Энтропия (от 
греческого \textit{entropia}~--- превращение) введена в 1865~г.\ немецким физиком Р.~Клаузиусом как 
функция состояния термодинамической системы, изменение которой $dS$ в равновесном процессе равно 
отношению количества теплоты $dQ$, подведенного к системе или отведенного от нее, к 
термодинамической температуре системы~$T$: $dS=dQ/T$. Л.~Больцман, один из основателей 
статистической термодинамики, предложил использовать энтропию как меру вероятности пребывания 
системы в данном состоянии. Шеннон ввел энтропию в теорию информации в качестве меры количества 
информации, которое выражается через распределение вероятностей.} как вероятностную меру 
количества информации~\cite{3-k}.
   
   Энтропия исхода определяется в виде логарифма вероятности этого 
исхода:
   \begin{equation}
   H(\xi_i)=- \log p(\xi_i)\,,
   \label{e1-k}
   \end{equation}
а усредненная энтропия случайной величины~$\xi$ выражается через 
функцию распределения ее вероятностей в виде:
\begin{equation}
H_\xi =- \sum\limits_\xi p(\xi)\log p(\xi)\,,
\label{e2-k}
\end{equation}
где $\xi$~--- случайная величина; $p(\xi)\leq 1$~--- распределение ее 
вероятностей. 
   
   В рассматриваемом здесь случае анализа текстов случайными 
величинами могут быть слова или другие конструкции. Под количеством 
информации в теории информации понимается неопределенность, 
устраняемая в результате выяснения исхода, т.\,е.\ значения, принимаемого 
случайной величиной. 
   
   Простейший содержательный пример в контексте статьи может быть 
следующим. Слова: \textit{пример, образец, экспонат} в некотором контексте 
являются синонимами и могут использоваться для обозначения 
объекта~$\xi$ с вероятностями: $p(\xi=\;\mbox{\textit{пример}})\hm=0{,}2$; 
$p(\xi=\;\mbox{\textit{образец}})\hm=0{,}4$; 
$р(\xi=\;\mbox{\textit{экспонат}})\hm=0{,}6$. В~этом случае количество 
информации, получаемое при реализации конкретного исхода, допустим, при 
использовании слова экспонат, т.\,е.\ $\xi\hm=\;\mbox{\textit{экспонат}}$, будет 
в соответствии с~(\ref{e1-k}) равно $\log p(0{,}6)$, а усредненная 
неопределенность объекта~$\xi$ будет по~(\ref{e2-k}) равна: $H_\xi\hm=- 
p(0{,}2)\log p(0{,}2)\hm- p(0{,}4)\log p(0{,}4)\hm- p(0{,}6)\log p(0{,}6)$. 
В~теории информации наиболее часто используются логарифмы по 
основанию~2, в этом случае количество информации определяется в битах.
   
   В реальности чаще интерес представляет сравнение информационной 
емкости сообщений, оценка имеющейся в них совместной информации. Для 
этого по распределениям вероятностей сообщений определяется энтропия 
каждого из них (количество информации в каждом из них), а по совместному 
распределению вероятностей~--- совместная энтропия. По энтропиям 
оценивается количество взаимной информации. 
   
   При использовании теории информации для описания закономерностей 
передачи информации энтропия переданного сообщения определяет 
количество переданной информации, а энтропия принятого сообщения~--- 
количество принятой информации. Общая часть в переданном и принятом 
сообщениях определяет количество взаимной информации. Обозначив через 
$\xi$ переданное сообщение, а через~$\eta$~--- принятое, количество 
взаимной информации $I (\xi\eta)$, или количество информации, 
содержащееся в принятом сообщении~$\eta$ из переданного 
сообщения~$\xi$, можно определить, следуя теории информации~\cite{3-k}, 
в виде: 
   \begin{equation}
   I_{\xi\eta} =\int\limits_X\! \int\limits_Y p_{\xi\eta} (x,y)\log \fr{p_{\xi\eta} 
(x,y)}{p_\xi(x)p_\eta(y)}\,dxdy\,,
   \label{e3-k}
   \end{equation}
где $p_\xi(x)$~--- плотность распределения переданного сообщения;
   $p_\eta(y)$~--- плотность распределения принятого сообщения;
   $p_{\xi\eta}(x,y)$~--- плотность совместного распределения; 
   $X$ и $Y$~--- области определения~$x$ и~$y$ соответственно.
   
   В~(\ref{e3-k}) имеет место равноправное симметричное вхождение~$\xi$ 
и~$\eta$, поэтому взаимная информация симметрична относительно~$\xi$ 
и~$\eta$. Отсюда следует, что при количественной оценке взаимной 
информации не важно, какое сообщение выступает в роли переданного, а 
какое~--- в роли принятого. 
   
   Именно взаимная информация может использоваться в качестве меры 
подобия ИО. Нетрудно видеть, что содержательное существо теории 
информации, направленное на оценку потерь информации при ее передаче, 
адекватно содержательной сущности многих задач в области исследования 
семантической близости ИО на естественном языке. Применение теории 
информации для решения проблемы оценки близости ИО может 
основываться на замене сообщений исследуемыми ИО. 
Один объект (эталон) может интерпретироваться переданным 
сообщением, а второй (дубль)~--- принятым сообщением. Вследствие 
симметрии от перемены ролей количество взаимной информации не 
изменится. Формула~(\ref{e3-k}) отражает количество взаимной информации 
непрерывных сообщений, точнее сообщений, случайный характер которых 
определяется непрерывными функциями распределения вероятностей вида 
$p_\xi(x)$, $x\in X \hm= [x^\prime, x^{\prime\prime}]$, где $x^\prime$ и 
$x^{\prime\prime}$~--- предельные значения~$x$. 
   
   При анализе текстов в качестве случайных величин будут выступать 
синтаксические или морфологические компоненты, которые являются 
дискретными величинами. Их случайными лексическими значениями будут 
выступать слова или словосочетания, характеризуемые дискретными 
ве\-ро\-ят\-но\-стя\-ми, как это показано в приведенном выше кратком примере. 
В~примере под случайным объектом или компонентом~$\xi$ может 
пониматься подлежащее при использовании синтаксической структуризации 
или существительное при использовании морфологической структуризации. 
Компонент~$\xi$ в обоих случаях может принимать случайные значения 
\textit{пример}, \textit{образец}, \textit{экспонат} с вероятностями 
$p(\xi=\;\mbox{\textit{пример}})\hm=0{,}2$; 
$p(\xi=\;\mbox{\textit{образец}})\hm=0{,}4$; 
$p(\xi=\;\mbox{\textit{экспонат}})\hm=0{,}6$. Дискретные значения 
вероятностей могут суммироваться, и поэтому вмес\-то интегралов, 
присутствующих в~(\ref{e3-k}), будут исполь-\linebreak зоваться суммы по всем 
возможным значениям\linebreak случайных величин. Количество взаимной 
информации в дискретном случае будет определяться следующим образом:
   \begin{equation}
   I_{\xi\eta} =\sum\limits_{\xi\in X} \sum\limits_{\eta\in Y} p(\xi,\eta) \log 
\fr{p(\xi,\eta)}{p(\xi)p(\eta)}\,,
   \label{e4-k}
   \end{equation}
где $X = \{x_1, x_2, \ldots , x_n\}$, $Y \hm= \{y_1, y_2, \ldots , y_m\}$~--- 
множества значений случайных величин~$\xi$ и~$\eta$, 
   $p(\xi)$, $p(\eta)$ и $p(\xi,\eta)$~--- распределения их вероятностей. 
   
   Данная статья посвящена изложению общей концепции 
автоматизированной технологии оценки степени близости ИО, 
представленных на естественном языке. Поэтому здесь ограничимся этим 
кратким представлением основного существа теории информации и ее 
дальнейшее использование объясним <<на словах>>. Достаточно полные и 
строгие сведения по энтропии и взаимной информации интересующийся 
читатель сможет найти в оригинальной литературе, например~[3--5], а их 
применение в далекой от передачи информации области управления 
качеством и технологиями~--- в работах автора~\cite{6-k, 7-k} и~др.

\section{Понятие вероятностной модели}

   Структуризация текста с целью извлечения заключенного в нем смысла 
представляется не вполне определенной задачей. Неопределенность следует 
из сложности представления ее содержательного существа, откуда вытекают 
и проблемы с определением методов решения. При оценке степени подобия 
содержания двух ИО смысл каждого из них, вообще 
говоря, интереса не представляет, так как целью является не выяснение 
семантики, а оценка степени их содержательного подобия. Для этого 
необходимо оценить меру совпадения в текстах того, о чем (ком) идет речь, 
что, как, где, когда и~т.\,п.\ с ними происходит или они делают. 
   
   Синтез формального подхода к оценке бли\-зости ИО, представленных на 
естественном языке, тре\-бу\-ет формального пред\-став\-ления самих 
срав\-ни\-ва\-емых объектов, т.\,е.\ разработки модели представления ИО. 
Математические модели объектов\linebreak являются основой для разработки систем 
управления этими объектами, а также решения задач анализа объектов, 
исследования взаимосвязей между компонентами, образующими объект, 
синтеза суж\-дений о состоянии и эволюции объекта. Поэтому структура и 
содержание модели должны разрабатываться с учетом четкого представления 
целей, для достижения которых она будет использоваться. Модель должна 
адекватно отражать все наиболее важные для правильного решения 
поставленной задачи содержательные аспекты объекта и игнорировать те, 
которые, усложняя модель, не способствуют повышению качества решения. 
Модели одного и того же объекта, предназначенные для решения различных 
задач, могут значительно различаться глубиной учета отдельных деталей. 
   
   Применительно к проблеме формального представления 
ИО в задачах оценки их семантической близости 
модель должна обеспечивать возможность сопоставления эквивалентных 
компонентов объектов, отражающих содержание сопоставляемых ИО, и 
игнорировать стилистические тонкости, влия\-ющие на форму представления 
содержания, но не на его смысл. 
   
   Обычно первым шагом при построении модели является структуризация 
объекта, выделение его компонентов, которые в совокупности определяют 
рассматриваемый объект. 

При разработке структуры модели ИО можно было 
бы, следуя имеющимся в литературе примерам, исходить из структуры 
простых предложений, в виде совокупности которых тем или иным способом 
может быть пред\-став\-лен ИО. Простое предложение русская грамматика 
определяет центральной грамматической единицей. <<Это определяется тем, 
что простое предложение представляет собой элементарную 
предназначенную для передачи относительно законченной информации 
единицу\ldots>>~\cite[с.~405]{8-k}. Но далее следуют 154~параграфа, в 
которых излагаются типы и формы простых предложений. Их многообразие 
и присутствие неполной четкости деления по типам и формам делает 
нереальной задачу формального описания даже простых предложений, не 
говоря о более сложных типах предложений. Именно по этой причине 
детерминированный подход, опирающийся на представления русской 
грамматики, как отмечалось выше, представляется малопригодным для 
анализа семантической бли\-зости ИО. 
   
   Вследствие того, что целью разрабатываемой методологии является не 
анализ текстов с позиций грамматики русского языка, а сопоставление их 
семантического содержания, которое может\linebreak случайным образом облекаться в 
лексическую оболочку, к определению структуры модели пред\-став\-ля\-ет\-ся 
целесообразным подойти с вероятностно-ста\-ти\-сти\-че\-ских позиций. 
   
   В теории вероятностей~\cite{9-k} существует вероятностная модель, 
которая позволяет дать формальное, максимально полное описание 
   ве\-ро\-ят\-но\-ст\-но-ста\-ти\-сти\-че\-ско\-го объекта. Она определяется 
на множестве элементарных событий $\{\omega_1, \omega_2, \ldots , 
\omega_n\}$, которое образует пространство элементарных событий, или 
исходов $\Omega =\{\omega_1, \omega_2, \ldots , \omega_n\}$. Известны 
вероятности элементарных событий $p(\omega_i)$, $i=1, 2, \ldots , n$. На 
множестве элементарных событий задается алгебра $\aleph=(A_j \vert 
A_j\subseteq\Omega$) или, иначе, система случайных событий, составленных 
каким-ли\-бо определенным образом из элементарных событий $\omega_i 
\in\Omega$. Для каждого из случайных событий $A_j=\{\omega_i\in \Omega\}$, 
образующих алгебру, по вероятностям элементарных исходов $p(\omega_i)$, 
$\omega_i\in A_j$, определяется его вероятность $P(A_j)$. 
   
   Набор: множество элементарных событий $\Omega \hm=\{\omega_1, \ldots , 
\omega_n\}$, система случайных событий (ал\-геб\-ра) $\aleph=(A_j\vert  
A_j\subseteq\Omega$) и вероятности случайных событий $P(A_j)$~--- образует 
вероятностную модель случайного объекта. Она содержит всю информацию, 
которой может быть охарактеризован случайный объект. Формально 
вероятностная модель (или вероятностное пространство эксперимента с 
конечным пространством исходов~$\Omega$ и алгеброй событий~$\aleph$) 
может быть представлена в виде:
   \begin{equation}
   M_\Omega =\{ \Omega, \aleph, P(A)\}\,,
   \label{e5-k}
   \end{equation}
где $\Omega= \{\omega_1, \omega_2, \ldots , \omega_n\}$, $\aleph= (A_j\vert A_j 
\subseteq \Omega)$, $P(A) \hm= (P(A_j)\vert A_j\in \aleph)$.

\begin{table*}[b]\small
\begin{center}
\Caption{Представление ВСММ ИО (фрагмент)}
\vspace*{2ex}

\tabcolsep=4.5pt
\begin{tabular}{|c|c|c|c|c|c|c|c|c|}
\hline
\multicolumn{2}{|c|}{Существительные}&\multicolumn{2}{c|}{Прилагательные}&
\multicolumn{2}{c|}{Числительные}&\ldots&\multicolumn{2}{c|}{Глаголы}\\
\hline
Слова&Характеристики&Слова&Характеристики&Слова&Характеристики&\ldots&Слова&Характеристики\\
\hline
1 Дом&$p$(дом)&Серый&$p$(сер.)&Три&$p$ (три)&\ldots&Стоит&$P$(стоит)\\
2 Стол&$p$(стол)&Белый&$p$ (бел.)&Два&$p$ (два)&\ldots&Идет&$P$ (идет)\\
\ldots&\ldots&\ldots&\ldots&\ldots&\ldots&\ldots&\ldots&\ldots\\
\hline
\end{tabular}
\end{center}
\end{table*}
   
   Определить вероятностную модель конкретного случайного объекта 
значит определить все ее элементы~--- множество элементарных исходов, 
сис\-те\-му случайных событий и их вероятности~--- для этого конкретного 
объекта.

\section{Вероятностно-статистическая морфологическая модель 
информационного объекта}
   
   Информационный объект может быть пред\-став\-лен в виде вероятностной 
модели. В нем множество элементарных исходов $\Omega = \{\w_1, \w_2, 
\ldots , \w_n\}$ представляют слова $\w_i$, $i=1, 2, \ldots , n$, со\-став\-ля\-ющие 
текст ИО. Существуют системы структуризации лексического материала. 
Достаточно общими и пригодными для использования при разработке 
ве\-ро\-ят\-но\-ст\-но-ста\-ти\-сти\-че\-ской модели ИО являются синтаксическая и 
морфологическая структуризации русского языка. Морфологическая 
структуризация задается определением частей речи русского языка, которые 
разделяют язык на самые крупные грамматические классы слов~\cite{8-k}. 
Различают десять частей речи, среди которых шесть знаменательных: 
существительные, прилагательные, чис\-ли\-тель\-ные, 
   мес\-то\-име\-ния-су\-ще\-ст\-ви\-тель\-ные, наречия, глаголы и три 
служебные: предлоги, союзы, частицы. Десятой частью являются 
междометия. Части речи, к которым относятся отдельные слова, могут 
трактоваться случайными событиями~$A_j$, $j = 1, 2, \ldots , 10$. Каждое 
отдельное слово (реализация, элементарный исход) $\omega_i$, $i=1, 2, \ldots 
, n$, входит в текст с определенной вероятностью~$p(\omega_i)$. В~тексте 
роль вероятности играет относительная частота $p(\omega_ii)=n_i/n$, где 
$n_i$~--- число употреблений в ИО слова~$i$, $n$~--- общее количество слов 
в ИО. Относительная частота получается экспериментально и называется в 
теории вероятностей эмпирической вероятностью. По вероятностям 
$p(\omega_i)$ отдельных слов вычисляются вероятности событий~$A_j$~--- 
час\-тей речи. Вероятностно-статистическая модель ИО~(\ref{e5-k}), в которой 
алгебра (способ структуризации) слов определяется морфологией, может 
быть названа вероятностно-статистической морфологической моделью 
(ВСММ) ИО, которая может быть по аналогии с~(\ref{e5-k}) записана в виде: 
   \begin{equation}
   M_M =\{\Omega, \aleph_M, P(A)\}\,,
   \label{e6-k}
   \end{equation}
где индекс $M$ подчеркивает морфологический характер модели, который 
отражается через определение алгебры~$\aleph_M$.
   
   Конкретный ИО представляется в виде соответствующего 
   ве\-ро\-ят\-но\-ст\-но-ста\-ти\-сти\-че\-ско\-го морфологического образа 
ИО. Он синтезируется на основа\-нии модели~(\ref{e6-k}) введением 
конкретного множества элементарных исходов $\W_O= (\w_1, \w_2, \ldots$\linebreak $\ldots , 
\w_n)$ -- слов. Обозначение $\W_O$ подчеркивает, что это множество слов 
конкретного ИО. Множество структурируется в соответствии с введенной 
ал\-геб\-рой $\aleph_M=(A_1, A_2, \ldots , A_J)$, где $J$~--- число случайных 
событий (частей речи), используемых в образе, $J\leq  10$, т.\,е.\ некоторые 
части речи, например междометия, предлоги, могут не использоваться при 
формировании образа. В~результате пол\-ностью определяется 
ве\-ро\-ят\-но\-ст\-но-ста\-ти\-сти\-че\-ский морфологический образ (ВСМО) ИО ВСММ~(\ref{e5-k}) 
конкретного ИО, который может быть представлен в виде:
   \begin{equation}
   O_M=\{ \W_O,\aleph_M,P_O(A)\}\,,
   \label{e7-k}
   \end{equation}
где обозначения следуют из~(\ref{e5-k}), (\ref{e6-k}) и текста. 
   
   По множеству элементарных исходов $\W_O$ вычисляются 
количественные характеристики образа: вероятности $p(\w_i)$ и вероятности 
случайных событий $P(A_j)$. По вероятностям может быть в соответствии 
с~(\ref{e2-k}) определена энтропия $H_O$ образа ИО, характеризующая 
количество информации в об\-разе.
   
   Для пояснения, возможно, непривычного для исследований в области 
русского языка подхода и терминологии воспользуемся наглядной 
иллюстрацией. Вероятностно-ста\-ти\-сти\-че\-ская морфологическая
модель ИО может быть представлена в виде табл.~1.
   
   В шапке таблицы для сокращения размеров примера указаны в явном 
виде только 4~части речи из десяти, имеющихся в языке. В~модели будет 
использоваться таблица с полным набором частей речи. Шапка таблицы 
является атрибутом модели. Она отражает структуру ВСММ и является 
общей для представления образов всех ИО. Части речи, указанные в шапке, 
трактуются случайными величинами. В~вероятностном смысле шапка 
таблицы содержит все рассматриваемые при морфологическом подходе 
случайные события или полное поле событий. Смысл полного поля событий 
в данном контексте в том, что любое встреченное в тексте (в ИО) слово 
относится к одному из них (является ка\-кой-либо частью речи). 
   
   Все то, что находится в таблице под шапкой, отражает конкретный ИО, 
т.\,е.\ определяет ВСМО ИО~(\ref{e7-k}). Слова \textit{дом}, \textit{стол} 
являются в примере случайными значениями, которые приняла часть речи 
<<существительное>>, аналогично \textit{серый}, \textit{белый}~--- 
случайные значения <<прилагательного>>, \textit{стоит}, \textit{идет}~--- 
<<глагола>>. Кроме собственно значений, которые принимают части речи в 
ИО, в модели отражаются их случайные характеристики, например 
относительные частоты.
   
   Вероятностно-ста\-ти\-сти\-че\-ский морфологический
образ~(\ref{e7-k}) может быть сформирован для любого произвольного 
ИО, представленного на русском языке, да и не на 
русском тоже. При его формировании могут быть использованы имеющиеся 
достаточно эффективные инструменты морфологического анализа текстов, 
которые позволяют автоматизировать процедуры отнесения слов к частям 
речи. 
   
   При решении задачи оценки семантической близости 
ИО ВСМО синтезируется для обоих сравниваемых объектов. Для 
определенности один из них называется эталоном (ИОЭ), его 
морфологический образ обозначается $O_{\mathrm{МЭ}}$, а второй~--- дублем 
(ИОД), его образ~--- $O_{\mathrm{МД}}$. Ве\-ро\-ят\-но\-ст\-но-ста\-ти\-сти\-че\-ский 
морфологический образ содержит все слова ИО, 
структурированные по частям речи, ве\-ро\-ят\-ности отдельных слов и частей 
речи в ИО. Ве\-ро\-ят\-ности количественно характеризуют ВСМО ИОЭ и ВСМО 
ИОД. На их основе может быть осуществлена оценка количества 
информации в каждом из объектов и оценено количество взаимной 
информации~(\ref{e4-k}) в объектах. Выражение для оценки количества 
взаимной информации~(\ref{e4-k}) может быть приведено к виду:
   \begin{equation}
   I_{\mathrm{МЭД}} = H(O_{\mathrm{МЭ}}) + H(O_{\mathrm{МД}})- 
H(O_{\mathrm{МЭ}}, O_{\mathrm{МД}})\,,
   \label{e8-k}
   \end{equation}
где обозначения совпадают с введенными раньше. 
   
   Можно утверждать, что полное совпадение ВСМО объектов будет иметь 
место при полной идентичности ИОД и ИОЭ. Наличие отклонений ВСМО 
дубля от ВСМО эталона будет указывать на несовпадение содержания ИО, 
количественной оценкой которого является количество совместной 
информации. 
   
   Формирование ВСМО объектов может опираться на имеющиеся 
фундаментальные исследования в области морфологии русского языка и 
разработки\linebreak мощных инструментов морфологической структуризации. 
В~частности, выполнение морфологической структуризации в данном 
исследовании опирает\-ся на электронную версию словаря 
А.\,А.~Зализняка~\cite{10-k}, для практического использования которой 
разработаны оригинальные программные продукты. 

\section{Вероятностно-статистическая синтаксическая модель 
информационного объекта}
   
   С другой стороны, в русском языке классифицированы члены 
предложения. Определение членов предложения задает синтаксическую 
структуру русского языка. Важно подчеркнуть, что части речи обладают 
общностью синтаксических функций, так что эти два способа 
структуризации лексического состава русского языка взаимосвязаны и 
дополняют друг друга. Вероятностно-статистическая модель, синтезируемая 
на синтаксической основе, будет отличаться только системой случайных 
событий, образующих полное поле событий, т.\,е.\ шапкой таблицы, 
отражающей модель. 
   
   Между морфологической и синтаксической структуризацией имеется 
значительное отличие, сле\-ду\-ющее из того, что морфологическая 
структури\-за\-ция является фиксированной, так как имеется всего 10~час\-тей 
речи. Синтаксическая структуризация допускает введение более детальной 
структуры членов предложения. Она определяется разработчиком системы 
сравнения объектов и допускает определенный волюнтаризм в выборе 
системы случайных событий. Для возможности отражения семантических 
оттенков в систему случайных событий могут быть введены разнообразные 
синтаксические конструкции, связанные со спецификой содержания 
сравниваемых ИО. Понятно, что увеличение отражаемого в модели 
разнообразия конструкций, с одной стороны, будет способствовать 
повышению качества сравнения текстов, а с другой~--- усложнению модели. 
Однако если учесть табличное представление модели, стандартное для 
реляционных баз данных, то увеличение таблиц не приведет к 
принципиальным затруднениям при реализации систем оценки близости ИО. 
   
   Вероятностно-статистическая синтаксическая модель (ВССМ) ИО 
аналогична ВСММ ИО и может быть представлена в виде таблицы, подобной 
табл.~1. Шапка таблицы будет отражать принцип 
структурирования по случайным событиям, которые в ней связываются уже с 
членами предложения, т.\,е.\ с синтаксическими конструкциями языка. 
Алгебра вероятностной модели в этом случае будет определяться типами 
членов предложения, которые используются для представления ИО: 
$\aleph_C \hm= \{B_1, B_2, \ldots , B_L)$, где $B_1$~--- подлежащее; $B_2$~--- 
сказуемое; $B_3$~--- определение и~т.\,д.;
   $L$~--- общее число типов членов предложения, используемых в 
синтаксической модели и образующих в ней полное поле событий; 
   $B_l$, $l \hm= 1, 2, \ldots , L$, как и $A_j$, трактуются как случайные 
синтаксические события. Таким образом, ВССМ будет иметь вид: 
\begin{equation}
M_C = \{\Omega, \aleph_C,P(B)\}\,,
\label{e9-k}
\end{equation}
отличающийся от~(\ref{e6-k}) только алгеброй~$\aleph_C$.

Исследуемый ИО посредством какого-либо синтаксического анализатора 
разделяется на члены предложения. На основе этого разделения 
формируются случайные события и синтезируется вероятностно-статистический 
синтаксический образ (ВССО) ИО:
\begin{equation}
O_C=\{\W_O,\aleph_C,P_O(B)\}\,.
\label{e10-k}
\end{equation}
   
   Для оценки семантической близости ИО ВССО 
синтезируется для обоих сравниваемых объектов. На их основе может быть 
оценено количество взаимной информации в сравниваемых 
объектах~(\ref{e4-k}). Выражение для оценки количества взаимной 
информации получается из~(\ref{e8-k}) заменой морфологических образов 
син\-так\-си\-че\-скими:
   \begin{equation}
I_{\mathrm{СЭД}} = H(O_{\mathrm{СЭ}}) + H(O_{\mathrm{СД}})- 
H(O_{\mathrm{СЭ}}, O_{\mathrm{СД}})\,,
\label{e11-k}
\end{equation}
где обозначения совпадают с введенными раньше. 
   
   Синтаксический анализ текста представляет отдельную проблему, 
отличающуюся от рассматриваемой здесь. Поэтому для определения 
инструментов формирования синтаксических образов были 
проанализированы имеющиеся в литературе наработки в этом направлении и 
практически использовался <<Синтаксический анализатор Cognitive 
Dwarf 2.0>>~[11--13].

\vspace*{-0.9pt}

\section{Методология оценки семантической близости информационных объектов}
   
   В морфологической и в синтаксической структуре структурные 
компоненты несут достаточно определенную и близкую семантическую 
нагрузку. Поэтому могут быть установлены отношения эквивалентности 
между компонентами двух структур. Вследствие того, что эти структуры 
охватывают весь лексический состав и грамматический строй русского 
языка, они обеспечивают отражение семантического содержания текстов и, 
следовательно, являются достаточными для сопоставления этого 
семантического содержания. 
   
   Таким образом, ИО может быть представлен в виде 
морфологического\addtolength{\footnotesep}{1.351pt}\footnote{Минимальное количество грамматических терминов, 
используемых в статье, заимствовано из~\cite{8-k} с единственной целью: приблизить 
изложение терминологически к области русского языка, хотя содержание статьи, как 
представляется, достаточно далеко от вопросов собственно языка.}\addtolength{\footnotesep}{-1.351pt} и/или 
синтаксического образа. Формальное представление ИО в виде 
математических объектов~--- ве\-ро\-ят\-но\-ст\-но-ста\-ти\-сти\-че\-ских 
образов~--- позволяет использовать математический аппарат для получения 
количественных оценок их близости. Можно утверждать, что полное 
совпадение как ВСМО, так и ВССО со\-по\-став\-ля\-емых объектов будет 
соответствовать равенству представляемых ими ИО. 

Оценка степени близости ИО, представленных на естественном языке, может 
быть реализована на основе применения вероятностно-статистической 
морфологической и/или синтаксической модели. 
   
   Мерой степени близости служит энтропия и взаимная информация, 
количественные значения которых вычисляются по~(\ref{e2-k}), (\ref{e8-k}) 
и~(\ref{e11-k}). Выше отмечалось, что под количеством информации в теории
информации понимается количество неопределенности случайного объекта, 
которое исчезает при выяснении этой неопределенности. Неопределенность 
объекта характеризуется распределением его вероятностей. 
В~использованном выше примере объекта~$\xi$ с синонимами было задано 
распределение вероятностей: $p(\xi=\;\mbox{\textit{пример}}) \hm= 0{,}2$; 
$p(\xi\hm=\;\mbox{\textit{образец}}) \hm= 0{,}4$; $p(\xi\hm=\;\mbox{\textit{экспонат}}) 
\hm= 0{,}6$. Так что в этом случае энтропия отдельных исходов будет: 
$H(\xi=\;\mbox{\textit{пример}}) \hm=- \log 0{,}2$, 
$H(\xi=\;\mbox{\textit{образец}}) \hm=- \log 0{,}4$, а 
$H(\xi=\;\mbox{\textit{экспонат}}) \hm=- \log 0{,}6$, а\linebreak усредненная энтропия 
$H_\xi \hm=- 0{,}2 \log 0{,}2\hm- 0{,}4 \log 0{,}4\hm - 0{,}6 \log 0{,}6$. Таким образом, 
энтропия будет некоторым числом, зависящим от распределения 
вероятностей случайных событий, но не от их содержания. 
   
   Из выражений~(\ref{e8-k}) и (\ref{e11-k}) для взаимной информации 
можно видеть, что, во-пер\-вых, она тоже является числом, так как 
выражается через числа~--- значения соответствующих энтропий. 
   Во-вто\-рых, выражения~(\ref{e8-k}) и~(\ref{e11-k}) отражают смысл 
взаимной информации как меры неопределенности. 

Пусть используются 
синтаксические образы со\-по\-став\-ля\-емых объектов, а их взаимная информация 
оценивается выражением~(\ref{e11-k}). Рассмотрим два предельных случая. 
   
   В первом пусть ВССО эталона $O_{\mathrm{СЭ}}$ не имеет ничего общего 
с ИССО дубля~$O_{\mathrm{СД}}$. Отсутствие общего означает, что 
вероятность совместного распределения $P(O_{\mathrm{СЭ}},O_{\mathrm{СД}}) 
\hm= 0$. В~теории информации принято считать $0 \log 0=0$, поэтому 
$H(O_{\mathrm{СЭ}}, O_{\mathrm{СД}}) \hm= 0$
 и из~(\ref{e11-k}) следует, что 
количество совместной информации, содержащееся в двух со\-по\-став\-ля\-емых 
объектах, равно их общей неопределенности: $I_{\mathrm{СЭД}} \hm= 
H(O_{\mathrm{СЭ}}) \hm+ H(O_{\mathrm{СД}})$.
   
   Во втором случае пусть дубль полностью совпадает с эталоном, т.\,е.\ 
$O_{\mathrm{СЭ}}\hm=O_{\mathrm{СД}}$. Тогда совпадут энтропии 
$H(O_{\mathrm{СЭ}}) \hm= H(O_{\mathrm{СД}})$, более того, и совместная 
энтропия $H(O_{\mathrm{СЭ}}, O_{\mathrm{СД}})$ будет равна энтропии 
эталона или дубля. Так что количество совместной информации будет равно 
$I_{\mathrm{СЭД}} = H(O_{\mathrm{СЭ}})$, т.\,е.\ неопределенность дубля 
отсутствует, вся неопределенность связана только с неопределенностью 
эталона, только с количеством заключенной в нем информации. Отсюда 
можно заключить, что количество информации~(\ref{e11-k}) изменяется от 
значения\linebreak
$I_{\mathrm{СЭД}} \hm= H(O_{\mathrm{СЭ}})$ до значения 
$I_{\mathrm{СЭД}}\hm = H(O_{\mathrm{СЭ}}) \hm+ H(O_{\mathrm{СД}})$. При 
этом взаимная информация~---\linebreak
 величина положительная. Это следует из 
положительности энтропий: 
$
p(\xi) \leq 1$, $\log p(\xi)\hm\leq  0
$ и, следовательно, 
$$
H(\xi) =- \log p(\xi) \geq 0
$$ 
и факта 
$$
H(O+{\mathrm{СЭ}})  + H(O_{\mathrm{СД}})  \geq H(O_{\mathrm{СЭ}}, O_{\mathrm{СД}})\,.
$$ 

Разумеется, такой же результат может быть получен и для~(\ref{e8-k}). 
Вследствие свойств логарифмической функции количество информации 
изменяется монотонно в определенных выше пределах, что и позволяет 
использовать его в качестве меры семантической близости ИО. 
   
   Для практического применения абстрактные значения энтропии и 
взаимной информации необходимо проградуировать в некоторых понятных и 
связанных с содержательным существом задачи мерах оценки семантической 
близости ИО. 
%
Такая градуировка (тарирование) их значений может 
осуществляться разными способами и, в частности, обеспечивать реализацию 
функций адаптации сис\-те\-мы к различным задачам и типам ИО. Например, в 
простейшем случае может быть взят реальный эталонный ИО такого типа, 
для работы с которым предполагается использовать систему. 
%
Искажением 
эталонного ИО случайным образом и в заданных объемах может быть 
получена серия дублей с известной степенью семантического несовпадения. 
Для каждой пары <<эта\-лон--дубль>> находится значение взаимной 
информации, которое сопоставляется с известной степенью семантического 
несовпадения. На основании сопоставления определяется линейное 
преобразование перевода количества информации в удобную меру оценки 
степени семантического соответствия ИО.
   
   Методология реализуется в виде последовательности следующих этапов: 
   \begin{itemize}
\item выбор вида и формирование вероятностно-ста\-ти\-сти\-че\-ской модели 
конкретизацией ал\-геб\-ры (системы случайных событий) $\aleph_M$ или 
$\aleph_C$;
\item введение содержания (текстов) образов ИО;
\item формирование образа эталонного ИО в виде $O_{\mathrm{МЭ}}$ или 
$O_{\mathrm{СЭ}}$;
\item формирование образа второго ИО (дубля) в виде $O_{\mathrm{МД}}$ или 
$O_{\mathrm{СД}}$; 
\item определение энтропии эталонного ИО в виде $H(O_{\mathrm{МЭ}})$ или 
$H(O_{\mathrm{СЭ}})$;
\item определение энтропии второго ИО (дубля) в виде $H(O_{\mathrm{МД}})$ 
или $H(O_{\mathrm{СД}})$;
\item определение совместной энтропии эталонного ИО и дубля в виде 
$H(O_{\mathrm{МЭ}}, O_{\mathrm{МД}})$ или $H(O_{\mathrm{СЭ}}, 
O_{\mathrm{СД}})$; 
\item определение совместной информации в соответствии с~(\ref{e8-k}) 
или/и в соответствии с~(\ref{e11-k});
\item перевод совместной информации в выбранную меру информационной 
близости ИО.
\end{itemize}

   Разработанная методология оценки семантической близости ИО, 
кажущаяся, на первый взгляд, совершенно от семантики оторванной, имеет 
глубокую содержательную основу. Если следовать более общему 
представлению языка, чем детальное грамматическое, то множества 
элементарных исходов (слов), образующих в вероятностно-ста\-ти\-сти\-че\-ских 
образах случайные события (час\-ти речи в ВСМО и члены предложения в 
ВССО)\linebreak могут трактоваться как соответствующие обобщенные час\-ти речи и 
члены предложения, образующие эти образы. Такое представление позволяет 
выделить главное содержание в сравниваемых ИО.\linebreak Содержательным 
примером, подтверждающим реальность разработанного подхода, является 
достаточно час\-то встречающееся в реальности продуктивное общение людей 
на плохо знакомом им \mbox{языке}. Они не владеют склонениями, спряжениями, 
формами времени и другими элементами грамматики, но, зная две--три сотни 
слов, достаточно успешно общаются, вполне понимая друг друга. 
   
   Здесь излагается ядро методологии и не рассматриваются возможности 
привлечения дополнительных инструментов, повышающих адекватность 
оценки, таких как использование синонимов, введение весовых 
коэффициентов, детализация и комбинация событий и~т.\,п. 
   
   Заметим еще, что такой подход может быть использован и для оценки 
близости ИО, реализованных на других языках и с использованием иных 
алфавитов. Другие языки, например английский или немецкий, отличаются 
от русского в сторону уменьшения свободы в порядке слов и разнообразия 
способов управления, что упрощает задачи их структурирования и 
построения вероятностно-ста\-ти\-сти\-че\-ских моделей ИО, не требуя изменения 
методологии. 
   
   В ряде случаев ИО могут быть представлены с использованием не 
естественного языка, а, например, формального математического языка 
формул. В~этом случае изменяется входной алфавит и, возможно, принцип 
синтеза алгебры случайных событий. Но эти изменения не касаются 
представленной здесь собственно методологии оценки подобия ИО. 

\vspace*{-6pt}
   
\section{Оценка знаний}

\vspace*{-2pt}
   
   Одной из проблем, для решения которой предпринята данная разработка, 
является автоматизированный контроль знаний. Использование для этой 
цели системы тестов представляется автору неприемлемым по множеству 
причин. На кафедре АСУ Липецкого государственного технического 
университета разрабатывается <<Автоматизированная система поддержки 
образовательной программы обучения>> (АСПОП)~\cite{14-k}, одним из 
важнейших компонентов которой является подсистема автоматизированного 
контроля знаний. Концепция подсистемы базируется на изложенной 
методологии. 
   
   Практическая проверка в минимально возможном объеме 
работоспособности принципиальных положений концепции осуществлена 
проверкой знаний студентов. При реализации проверки студентам на экране 
демонстрировался эталонный ответ из АСПОП, который они воспроизводили 
на память и заносили в компьютер. По эталонным ответам из АСПОП 
формировались ВСМО $O_{\mathrm{МЭ}}$ и ВССО $O_{\mathrm{СЭ}}$. По 
ответам студентов формировались соответствующие ВСМО 
$O_{\mathrm{МД}}$ и ВССО $O_{\mathrm{СД}}$. По морфологическим образам 
$O_{\mathrm{МЭ}}$ и $O_{\mathrm{МД}}$ определялись энтропии 
$H(O_{\mathrm{МЭ}})$, $H(O_{\mathrm{МД}})$ и $H(O_{\mathrm{МЭ}}, 
O_{\mathrm{МД}})$, а по ним оценивалось количество взаимной информации. 
Также обрабатывались синтаксические образы эталонных ответов и их 
дублей~--- ответов студентов. 
   
   Распечатанные эталонные ответы и ответы студентов анонимно 
сопоставлялись группой преподавателей, которые выставляли оценки 
студентам по существующей методике по 100-балль\-ной шкале. Оценки 
преподавателей надлежащим образом усреднялись. По оценкам 
преподавателей и количествам взаимной информации, определенным 
автоматизированной системой, определялись параметры масштабного 
преобразования количества информации в принятые в университете 
   100-балль\-ные оценки. После введения коэффициентов масштабного 
преобразования система, как и преподаватели, выдавала 100-балль\-ные 
оценки. 
   
   Оценки, автоматически сформированные сис\-те\-мой, были сопоставлены с 
оценками, вы\-став\-лен\-ны\-ми преподавателями. В~итоге было получено, что 
среднее квадратическое отклонение оценок, вычисленных системой на 
основании сопоставлений вероятностно-статистических образов эталона и 
ответа по изложенной методологии, от оценок, выставленных 
преподавателями, по 100-балль\-ной шкале составило 10\%--15\%. 
Результаты со\-по\-став\-ле\-ния ВСМО и ВССО эталона и ответа оказались 
достаточно близкими. Отметим, что это была пробная проверка, 
предпринятая исключительно для обретения уверенности в практической 
эффективности оригинальной концепции.
   
   Углубление и детализация вероятностно-ста\-ти\-сти\-че\-ских моделей ИО на 
естественном языке, их исследование и применение представляют 
неограниченное, научно новое и практически полезное поле деятельности, в 
освоении которого автор может оказать посильную помощь. 

\vspace*{-6pt}
   
\section{Заключение}

\vspace*{-2pt}

   Разработана оригинальная методология оценки степени семантической 
близости инфор\-ма\-ционных объектов. Методология может служить 
   фор\-маль\-но-ма\-те\-ма\-ти\-че\-ской основой в сфере современных 
информационных технологий для решения разнообразных задач сравнения и 
оценки подобия информационных объектов, представленных на 
естественном языке. Методология вводит ве\-ро\-ят\-но\-ст\-но-ста\-ти\-сти\-че\-скую 
модель представления русскоязычного текста и определяет способы 
представления текстов в виде ве\-ро\-ят\-но\-ст\-но-ста\-ти\-сти\-че\-ских 
морфологических и синтаксических образов, которые позволяют оценить 
количественно и объем информации в информационных объектах, и степень 
их семантического совпадения. Экспериментальная прикидочная проверка 
показала эффективность применения методологии для разработки 
автоматизированных систем оценки знаний. Практическое применение 
методологии только в этой сфере может привести к принципиальным 
изменениям в сфере образования. 

\vspace*{-6pt}

{\small\frenchspacing
{%\baselineskip=10.8pt
\addcontentsline{toc}{section}{Литература}
\begin{thebibliography}{99}

\bibitem{1-k}
\Au{Друкер П.}
Посткапиталистическое общество. Новая постиндустриальная волна на Западе: 
Антология~/ Под ред. В.\,Л.~Иноземцева.~--- М.: Academia, 1990. 

\bibitem{2-k}
\Au{Мельчук И.\,А.}
Опыт теории лингвистических моделей <<Смысл\;$\leftrightarrow$\;Текст>>.~--- 
2-е изд.~--- М.: Школа <<Языки русской культуры>>, 1999.

\bibitem{3-k}
\Au{Шеннон К.}
Математическая теория связи. 1948~// Работы по теории информации и 
кибернетике~/ Пер. с англ. под ред. Р.\,Л.~Добрушина и О.\,Б.~Лупанова.~--- 
М.: ИЛ, 1963.

\bibitem{4-k}
\Au{Колмогоров А.\,Н.}
Теория информации и теория алгоритмов.~--- М.: Наука, 1987.

\bibitem{5-k}
\Au{Стратонович Р.\,Л.} Теория информации.~--- М.: Сов. радио, 1975.

\bibitem{6-k}
\Au{Кузнецов Л.\,А.}
Введение в САПР производства проката.~--- М.: Металлургия, 1991.

\bibitem{7-k}
\Au{Kuznetsov L.\,A.}
The entropy and information application to identify fuzzy sets~//  ICSC Symposium 
(International) on Fuzzy Logic Proceedings.~---  
Academic Press, 1995. P.~A109--A111.

\bibitem{8-k}
\Au{Белоусов В.\,Н., Ковтунова И.\,И., Кручинина~И.\,Н. и~др.}
Краткая русская грамматика~/ Под ред. Н.\,Ю.~Шведовой и 
В.\,В.~Лопатина~--- М.: Рус. яз., 1989.

\bibitem{9-k}
\Au{Гнеденко Б.\,В.}
Курс теории вероятностей: Учебник.~--- 9-е изд., испр.~--- М.: ЛКИ, 2007.

\bibitem{10-k}
\Au{Зализняк А.\,А.}
Грамматический словарь русского языка: Словоизменение.~--- 3-е изд., 
стер.~--- М.: Рус. яз.,1987. 880~с.
\bibitem{11-k}
Синтаксический анализатор Cognitive Dwarf 2.0. {\sf 
http://cs.isa.ru:10000/dwarf/d2/dw2.html}.

\bibitem{12-k}
\Au{Антонова А.\,А., Мисюрев~А.\,В.}
Реализация синтаксического разбора для русского и английского языков~// 
Системный анализ и информационные технологии (САИТ 2005): Мат-лы 
I~Междунар.\ конф.~--- Пе\-ре\-славль-За\-лес\-ский, 2005.~--- 
Переславль-Залесский, 2005. С.~245--249.

\bibitem{13-k}
\Au{Антонова А.\,А., Мисюрев А.\,В.}
Синтаксический анализатор для русского и английского языков~// Сб. трудов 
ИСА РАН~/ Под ред. В.\,Л.~Арлазарова и Н.\,Е.~Емельянова.~--- М.: УРСС, 
2007.

\label{end\stat}

\bibitem{14-k}
\Au{Кузнецов Л.\,А., Фарафонов А.\,С., Тищенко~А.\,Д., Капнин~А.\,В.}
Автоматизированная система поддержки образовательной программы 
обучения~// Качество. Инновации. Образование, 2010. №\,9. С.~12--20. 
 \end{thebibliography}
}
}


\end{multicols}         %5
%\newcommand{\eol}{\end{enumerate}\setlength{\itemsep}{-\parsep}}
%\newcommand{\ang}[1]{\langle{#1}\rangle}
%\newcommand{\infinity}{\infty}
%\newcommand{\mess}[1]{\mbox{\tt #1}}
%\newcommand{\var}[1]{\mbox{\it #1}}
%\newcommand{\order}[1]{\stackrel{#1}\fa}
%\newcommand{\orderr}[1]{\stackrel{#1}\Longrightarrow}
%\newcommand{\infrel}[1]{\stackrel{#1}\Longrightarrow}
%\newcommand{\prog}{\mbox{\tt Prog}}
%\newcommand{\comment}[1]{}
%\newcommand{\set}[1]{\{#1\}}
%\newcommand{\pair}[2]{\langle #1,#2 \rangle}
%\newcommand{\remove}[1]{}
%\renewcommand{\qed}{\hfill\rule{2mm}{2mm}}
%\newcommand{\bull}[1]{\begin{itemize}\item{#1}\end{itemize}}
%\newcommand{\marg}[1]{\marginpar{\small #1}}


\renewcommand{\figurename}{\protect\bf Figure}
\renewcommand{\tablename}{\protect\bf Table}

\def\stat{frenkel}


\def\tit{SEAMLESS ROUTE UPDATES IN SOFTWARE-DEFINED NETWORKING 
VIA QUALITY OF~SERVICE COMPLIANCE VERIFICATION}

\def\titkol{Seamless route updates in software-defined networking via 
quality of service compliance verification}

\def\autkol{S.\,L.~Frenkel and~D.~Khankin}

\def\aut{S.\,L.~Frenkel$^1$ and~D.~Khankin$^2$}

\titel{\tit}{\aut}{\autkol}{\titkol}

%{\renewcommand{\thefootnote}{\fnsymbol{footnote}}
%\footnotetext[1] {The 
%research of Yuri Kabanov was done under partial financial support of the grant 
%of RSF No.\,14-49-00079.}}

\renewcommand{\thefootnote}{\arabic{footnote}}
\footnotetext[1]{Institute of Informatics Problems, Federal Research 
Center ``Computer Science and Control'' of the Russian Academy of Sciences,
 44-2~Vavilov Str., Moscow 119333, Russian Federation, \mbox{fsergei51@gmail.com}}
\footnotetext[2]{Computer Science Department, Ben-Gurion University of the Negev, 
Beer-Sheva 84105, Israel, \mbox{danielkh@post.bgu.ac.il}}


\index{Frenkel S.\,L.}
\index{Khankin D.}
\index{Френкель С.}
\index{Ханкин Д.}

\def\leftfootline{\small{\textbf{\thepage}
\hfill INFORMATIKA I EE PRIMENENIYA~--- INFORMATICS AND
APPLICATIONS\ \ \ 2018\ \ \ volume~12\ \ \ issue\ 4}
}%
 \def\rightfootline{\small{INFORMATIKA I EE PRIMENENIYA~---
INFORMATICS AND APPLICATIONS\ \ \ 2018\ \ \ volume~12\ \ \ issue\ 4
\hfill \textbf{\thepage}}}

\vspace*{4pt}

\Abste{In software-defined networking (SDN), the control plane and the data 
plane are decoupled. This allows high flexibility by providing abstractions 
for network management applications and being directly programmable. 
However, reconfiguration and updates of a~network are sometimes inevitable due 
to topology changes, maintenance, or failures. In the scenario,  
a~current route~$C$ and a set of possible new routes~$\{N_i\}$, where one of the 
new routes is required to replace the current route, are given. There is a chance that 
a~new route $N_i$ is longer than a~different new route $N_j$, but $N_i$ is 
a~more reliable one and it will update faster or perform better after the update 
in terms of quality of service (QoS) demands. 
Taking into account the random nature of the network functioning, 
the present authors supplement the recently proposed algorithm by Delaet
\textit{et al}.\ for route updates with 
a~technique based on Markov chains (MCs). As such, an enhanced algorithm 
for complying QoS demands during route updates is proposed
in a~seamless fashion. First, 
an extension to the update algorithm of Delaet \textit{et al}.\ 
that describes the transmission of packets through a~chosen route and compares 
the update process for all possible alternative routes is suggested. Second, several 
methods for choosing a~combination of preferred subparts of new routes, resulting 
in an optimal, in the sense of QoS compliance, new route is provided.} 

\KWE{software-defined networking; Markov chains; quality of service}

\DOI{10.14357/19922264180408}


\vspace*{8pt}


\vskip 12pt plus 9pt minus 6pt

 \thispagestyle{myheadings}

 \begin{multicols}{2}

 \label{st\stat}

\section{Introduction}
\label{s:Intro}

\noindent
Software-defined networking is an emerging network paradigm, in which the 
control plane is decoupled from the data plane enabling centralized control 
logic. Such a~dynamic network may require frequent modifications and updates to 
the network topology and configuration. 
Also, the network topology is available to the centralized control entity, there, 
due to this flexibility, it is possible to perform offline optimized calculations.

Network functions virtualization (NFV) allows replacing traditional network 
devices with software that is running on commodity servers. This software 
implements the functionality that was previously provided by dedicated hardware. 
Network functions virtualization
 allows services to be composed of virtual network functions (VNF) hosted on 
different data centers. Software-defined networking, 
when applied to NFV, helps in addressing challenges 
of dynamic resource management and intelligent service 
orchestration~\cite{rao_sdn_2014}. Sometimes, traffic is often required to pass 
through and be processed by an ordered sequence of possibly remote 
VNFs~\cite{ghaznavi_service_2016}. For example, traffic may be required to pass 
through intrusion detection system, proxy, load balancer, or a~firewall. 
Such concatenation of services is called \textit{service function chaining} 
(SFC).

Consider, for example, two communicating parties in a~network featuring complex 
network topology (e.\,g., Small-world network), and the communication flow is 
passed over a~series of VNFs. It may be the case that the network operator is 
required to move the communicating flow to a~different path due to QoS 
requirements or other possible cost considerations. We are interested 
to model the anticipated expected number of steps until the update is complete 
given a~possible new route following the required QoS demands, e.\,g., 
delay, communication rounds, cost, etc. 

%Aforesaid dynamic networking requires frequent modifications and updates to the network. 
Let us consider a pair $(C, \{N_i\})$ where a~current route~$C$ from~$s$ to~$d$ 
is scheduled to be replaced by a new route from the set~$\{N_i\}$, each from~$s$ 
to~$d$ either. Let us model each route as an ordered list of network elements, such 
as VNFs (SFCs) or generally saying routers. Each new route~$N_i$ is constructed 
during the update process, and thus, certain delays may be introduced due to
 initial packet processing or due to possible losses. 
 %There, the eventual arrival of packets along the new route during the update process is critical for successful route update. Another possible example is when the routes are SFCs, and the requirement is to update a current chain to a new one, different service chains may exhibit different delays. 

The design goals must be achieved by constructing effective algorithms for 
efficient packet QoS routing in NFV/SDN computer network. Depending on the 
QoS metric, the lower (e.\,g., for reliability) or upper (e.\,g., for a~delay) 
constraints represent the desired bounds that the orchestration must meet. 
Since different configurations could meet these bounds, the designer should also 
optimize against a~specific metric by using these both ends of the extreme. 

Methods based on integer linear programming (ILP) were proposed in several works 
(see section~\ref{sec:related_work}). The difficulty of using tools based on ILP 
 in the operational work of an administrator is that in view of the possible 
 infeasibility of the resulting solution, it may take not a~few resources (time, efforts) 
 until acceptable QoS values can be ensured.

We consider the use of ``design via verification'' approach, suggesting a~method 
for complying QoS demands. The method is based on augmenting the update algorithm with
a~verification logic. Namely, we suggest the use of 
\textit{Probabilistic real-time Computation Tree Logic} 
(PCTL)~\cite{hansson_logic_1994} for expressing real-time and probability in systems. 
Using PCTL, we can express the probability for a~process to complete after 
a~certain number of steps along an execution path and verify the selected route 
for the update.


%Assume that packets are sent from a source node $s$ to a destination node $d$ along the current route. After the update process is finished, packets will be forwarded from $s$ to $d$ along the new route. 
Delaet \textit{et al.}\ proposed a~multicast-based scheme for seamlessly updating 
a~current route to a~new one~\cite{delaet_seamless_2015}. 
According to the multicast scheme, the controller instructs 
a~router to temporarily have two $(s,d)$ entries in the routing table. When 
a~router $r \neq d$ receives a~packet from~$s$ to~$d$, it sends the packet 
according to the forwarding instructions of all of its $(s,d)$ routing 
table entries. When two identical copies of a~packet that was multicasted 
over the current and new portion of a~route arrive, the controller can dismantle 
the current route, as the new route is ready. During the update process, packets 
should not be lost, no cycles should be formed, and communication should not 
be disrupted.

%Taking into account the random nature of the network functioning, we supplement the algorithm for route updates introduced by Delaet et al. in \cite{delaet_seamless_2015}, with a technique based on Markov chains. In our extension of the algorithm, we describe the transmission of packets through a chosen route and compare the update process for all the possible alternative routes that are candidates for replacement. 

Our contribution is a model for a successful route update, including its 
intermediate steps, as MC states, each with 
a~given probability. With our model, we are able to characterize the quality of 
an update by expected number of steps in the~MC. 
%We use Markov chains to characterize the quality of the update service, and represent the expected number of steps in the Markov chain as the quality of a successful update. While, the probability for an update event 

We suggest an enhanced update method for the network administrator to augment 
his decision regarding QoS demands in terms of various network parameters and 
possible failure of the update process. Moreover, in contrast to other works, 
we are able to provide a~version of an algorithm that can perform real-time QoS
 assessment during a~route update, for each subpart of a~route. At last, using 
 our method, it is possible that the active new route will be comprised of subparts 
 of different new routes, providing optimal route update service in regard of 
 required network QoS. 

%We assume that each new route is legal. 
%However, mixing subroutes belonging to different routes may result in inconsistent state or a cycle formed in the network. We use different 
%
%
%
%We model the update process as a service, namely as a VNF, and we use Markov chains to characterize the quality of the update service. Using the expected number of steps in the Markov chain representing the update, we abstract the quality of the update service. We calculate for each possible new (sub-)route the expected number of steps required to update an old (sub-)route successfully. Subsequently, the old route is updated to the new route which requires less number of steps with high probability. We supplement the seamless update algorithm proposed by the authors of \cite{delaet_seamless_2015} with the model in this work.

%The virtualized service implementing the update algorithm will provide a recommendation for an optimal choice of a route, based on the performed calculations. Fundamentally, we create a QoS VNF for seamlessly updating a route, regarding network parameters, and taking into consideration the complexity and possible failures of updating a route. In case there exist several alternatives for a route update, there is a chance that one of the possible new routes is much longer, however, a more reliable one, and as such will update faster. 
%
%
%One of the important requirements to modification process is that the update process should not form congestion in the network, nor result in time delays, and not lose any packets. 
%
%
%Additionally, we provide an enhanced version of an algorithm that can perform the quality of service assessment during the update process, for each subpart of the new route. 
%
%We propose a directed graph $G=(V,E)$, for representing the possible legal combinations of sub-routes. The set of common nodes to $(C, \{N_i\})$ subdivides the old route and each of the new routes to sub-routes. For two sub-routes represented by the nodes $u,v \in V$, the sub-route $v$ can be launched after $u$ if and only if there exists a directed edge $(u,v) \in E$. Otherwise, the launch of $v$ after $u$ is forbidden and can result in a cycle formed in the network.


%The results of this work helped to develop an operating strategy for a network administrator, supporting both, seamlessly updating a route, and providing QoS requirements. 

Extended abstract of this work appeared as a conference paper 
in~\cite{frenkel_predicting_2017} which presented preliminary results. 
In this work, we describe in detail the system settings and bring new results 
by providing two additional algorithms.
{\looseness=1

}

In the following section, we overview the related work. Next, we provide 
the required definitions and the system settings and describe the MC 
characterization of the network. Further, we describe different update setting, 
accordingly accompanying algorithms and data structures, used for QoS assessment 
during route updates.

\vspace*{-9pt}

\section{Related Work}
\label{sec:related_work}

\vspace*{-2pt}
%The design goals must be achieved by constructing effective algorithms for efficient packet QoS routing in NFV/SDN computer network. %These algorithms, which must enable an administrator to orchestrate the existing services exported by remote providers, were considered in \cite{martins_clickos_2014, zaalouk_orchsec:_2014}. Likewise, the functional behavior (e.g., services being deprecated by their providers), as well as changes in the non-functional behavior of the orchestrated services (e.g., an increased execution time) were also considered.

%Depending on the QoS metric, the lower (e.g., for reliability) or upper (e.g., for delay) constraints represent the desired bounds that the orchestration must meet. Since different configurations could meet these bounds, the designer must also optimize against a specific metric by using these both ends of extreme.

\noindent
Quality of service routing using multipath was proposed in~\cite{devi_approach_2015}. 
The routing algorithm, initially, eliminates all links that do not meet the 
bandwidth requirements. Then, it finds disjoint shortest paths based on 
the residual network graph in each iteration.

The work~\cite{egilmez_distributed_2012} proposed a~QoS optimized routing 
over multidomain OpenFlow networks managed by a~distributed control plane, 
where each controller performs optimal routing within its domain. 
The QoS routing problem was posed as a~constrained shortest path (CSP) problem, 
and the proposed solution computes a~near-optimal route, based on LARAC 
(Lagrange relaxation based aggregated cost)
algorithm~\cite{juttner_lagrange_2001}. The proposed algorithm is an approximation 
algorithm; it always gives a~suboptimal solution.

For traditional network architecture, a~routing strategy approach based on 
ILP was introduced in~\cite{yu_efficient_2013}.
 The main disadvantage of using ILP is that the problem is NP-hard. 
 Additionally, ILP cannot be applied to probabilistic values. 
 Using linear programming (not limited to integers) rounded to integer solutions 
 will not yield an optimal solution.
 

Route updates are extensively researched in SDN~\cite{foerster_survey_2016}, 
standing on the work by Reitblatt \textit{et al.}\ where requirements for SDN 
updates were examined. This work focused on per-packet consistency property, 
stating that packets have to be forwarded either using the initial configuration 
or the final configuration but never a~mixture of them, throughout the update 
process~\cite{reitblatt_consistent_2011}. The authors proposed 
a~2-phase commit technique which relies on packets tagging so that either of 
the rules is applied. However, such technique wastes critical network resources 
and complications are formed due to packet tagging~\cite{foerster_survey_2016}. 
Further, Delaet \textit{et al.}\ showed in~\cite{delaet_seamless_2015} 
that using a~careful multicast during route updates provides 
a~better working solution.

Hogan and Esposito propose in~\cite{hogan_stochastic_2017} the use of
 Bayesian networks for delay estimation as a~traffic engineering tool and model 
 the path selection problem using a~risk minimization technique. 
 However, the authors state that the accuracy of their model is limited by its 
 ability to correctly identify dependencies in the data. In our work, 
 we suggest a~general tool for probabilistic verification of any network parameter, 
 which does not depend on variance within the dataset.
 
 

In~\cite{mcgeer_safe_2012}, an update protocol proposed where packets are 
sent to the controller during updates; such approach adds 
a~significant cost to the control plane bandwidth~\cite{delaet_seamless_2015}. 
In~\cite{mcgeer_correct_2013}, an algorithm to find 
a~safe update sequence expressed as a~logic circuit has been proposed. 
However, the algorithm 
requires a~dedicated protocol which is not currently 
supported~\cite{foerster_survey_2016}. The authors 
of~\cite{katta_incremental_2013} propose to perform the 2-phase update 
scheme from~\cite{reitblatt_consistent_2011} incrementally, making the update longer. 
%For a thorough review of route updates, the reader is referred to \cite{foerster_survey_2016}.






Software-defined networking allows the involvement of the network administrator into the network 
management during route udpdates and, in particular, during packet transmission. 
Thus, it would be highly desirable to support the decision making process 
with the right tools. Our novelty is exactly such tool, for augmenting 
online decision making of the network administrator during network management 
in a~stochastic environment.
%In this work, we propose a technique to optimize the update process by selecting the preferred (sub-)route in order to reduce the update time. We use the expected number of steps for successfully completing the update as a QoS metric, and extend the algorithm by Delaet~et~al. with Discrete Time Markov Chains (DTMC) for finding (sub-)routes which are preferred in terms of QoS. % As such, we propose to use the route updates algorithm from \cite{delaet_seamless_2015} as a virtual service for network updates per QoS requirements.

%The interaction of software components have a greater weight in NFV context, which may lead to stochastic-like behavior 

%At present, certain routing algorithms (including $k$ Edge-Disjoint) are based on the shortest path (SP) problem solution \cite{wood_toward_2015}. However, the method proposed by Wood et al. is generic and valuable only in the case of request arrival, and do not consider certain additional important requirements, such as removal or priorities of requests. 

%Several approaches for efficient SP-based QoS routing have been recently proposed in \cite{buchbinder_improved_2006}, where the authors introduce and analyze a centralized algorithm for an online scheduling and routing of arbitrary sequence of communication requests. 

%Unsplittable (single-path) assignment for each request of QoS routing is probably competitive with the best possible splittable (multipath assignment).

The work by Delaet \textit{et al.}~[4] introduced the Make\&Activate-Before-Break 
approach for seamless
route update in SDN. The authors described in a~high-level the multicasting-based 
update, which we
employ in this work. Also, they introduced a~controller-based method for 
verifying the correctness
of a~new route before the traffic redirection. Dinitz \textit{et al.}~[16] 
extended the work~[4] to the general
case of several dependent (via shared links) routes pairs. The routes update 
problem was proved to
be NP-hard~\cite{17-aaa}. The authors of~[16] suggested the use of 
artificial intelligence (AI) methods for 
solving the problem. As a~basis for AI-based solutions, Dinitz 
\textit{et al.}\ proposed a dependence graph model describing the current
state of the problem instance at any replacement stage. 
In addition, route readiness verification similar
to that in~[4] was implemented in~[16] as a high-level network protocol.

In this work, we investigate a different problem; we consider the route updates 
problem from a~QoS
perspective and describe in high-level both the prediction and the update processes.

\vspace*{-9pt}

\section{Preliminaries and Definitions}

\vspace*{-2pt}

\noindent
The basic system settings are as follows. 
For a~(route) sequence~$X$, we denote by~$x_i$ the $i$th element in it.
In a~(directed) communication network, 
we are given a~route~$C$ from source~$s$ to destination~$d$. 
Additionally, we are given a~set of different new routes~$N_i$, each going from~$s$ 
to~$d$. We model each route as an ordered set of network nodes connected by network 
links. We assume that neither of the routes contains cycles. 
Each router in a~route matches a~packet from~$s$ to~$d$ 
and forwards the packet to the next router in order. After the update 
is complete, each router in the new route should forward the packets from~$s$ 
to~$d$ to the next router in order along the new route. 

In our work, we consider the route replacement problem as a~sequence of 
subroutes replacements.
The routes replacement subsystem was in great detail described by Dinitz 
\textit{et al.} in~\cite{dinitz_dependence_2017}. We borrow
from~[16] the relevant parts which we briefly describe here.

\smallskip

\noindent
\textbf{Definition~1.} We  define a~subset from $a\in X$ to $b\in X$ of an ordered
set~$X$, when $a$ precedes~$b$, as~a~subroute from~$a$ to~$b$, and denote such subroute by
$[a,b]$.

\smallskip

 

\textbf{Subroutes.} The current route~$C$ subdivides each new route 
to~$k$~common subroutes (a~subroute may consist of one router in the simplest case) 
and $k-1$ noncommon subroutes. 
For illustration, see Fig.~1.
In Fig.~1 and figures below, the current route is depicted
in a~light grey color full nodes, connected with
solid edges. The new route is depicted in white colored nodes, connected with
dashed edges. The common nodes are depicted as shaded. 
If there are several new
routes, the nodes of each route are filled with a~designating pattern. 
Additionally, for easier reading,
when it is possible, we denote subroutes of some route~$X$ as~$X^\prime$, $X^{\prime\prime}$, 
etc. In other cases, a~subroute~$j$
of a~new (current) route~$i$ is denoted as $N_j^i (C_i^j)$. 
Similarly, routers of some route~$X$ are denoted by~$r^\prime$,
$r^{\prime\prime}$, etc.

 { \begin{center}  %fig1
\vspace*{1pt}
 \mbox{%
 \epsfxsize=78.631mm 
 \epsfbox{fre-1.eps}
 }


\vspace*{3pt}


\noindent
{{\figurename~1}\ \ \small{Route $C$ with two possible new routes sharing a~link}}
\end{center}
}

\vspace*{6pt}






In the example in Fig.~1, 
noncommon new subroutes 
of route~$N_1$ are denoted by~$N^1_1=[s,r_2]$ and~$N^2_1=[r_2,d]$, while the noncommon new 
subroutes of~$N_2$ are denoted by~$N^1_2=[s,r_1]$, $N^2_2=[r_1,r_3]$, 
$N^3_2=[r_3,r_2]$, and~$N^4_2=[r_2,d]$. 

Note that in general, the order of common subroutes along~$C$ and along~$N$ 
can be different. See, for example, the common subroutes of~$C$ and~$N_2$ in 
%Figure \ref{fig:two_routes}.
Fig.~1.

\smallskip

\noindent
\textbf{Definition~2.} A~new noncommon subroute of~$N$ from router~$a$
to router~$b$ is legitimate for update only if~$a$ precedes~$b$ on the route~$C$.

\smallskip

Definition~2 guides us on which subroutes can be launched without creating routing cycles in the
network system. (See~[4] for details.)


When an update of a~subroute~$N^\prime$ from router~$r$ to~$r^\prime$ is finished, 
the update flow goes along~$C$ from~$s$ to~$r$, continues along~$N^\prime$ up to~$r^\prime$, 
and finishes along~$C$ from~$r^\prime$
 to~$d$. 
For illustration, see the result of launching~$N^4_2$ in Fig.~2.

 { \begin{center}  %fig2
\vspace*{-1pt}
 \mbox{%
 \epsfxsize=78.631mm 
 \epsfbox{fre-2.eps}
 }


\vspace*{3pt}


\noindent
{{\figurename~2}\ \ \small{$N^4_2$ was launched}}
\end{center}
}

\vspace*{4pt}


 

 Note that launching a~currently nonlegitimate new subroute, for example,~$N^3_2$ 
 in Fig.~1, is forbidden since it will form a~cycle 
 resulting in packets circulating and overwhelming the network. 

\textbf{Dynamics of the system.}
%\label{sec:dynamics} 
Dinitz \textit{et al.}\ performed a~detailed analysis on the dynamics of a~subroutes
system. After an update of a~subroute is complete, the set of current subroutes~$C$ 
and the set
of new subroutes~$N$ are recalculated. This may result in different system of subroutes. For example,
see Fig.~2 where after the launch of $N^4_2$ from the example in Fig.~1, 
the sets of subroutes are
recalculated. As a~result, we obtain different subroutes (for clarity, the previous labels are kept). See
also~[16] for details and extensive analysis.

\vspace*{-4pt}

\subsection{Markov chain characterization of~the~network~states}

\noindent
We characterize execution of some (sub)route in the network by 
a~packet delay time between the (sub)route's common sender and common destination 
routers as well the probability of a~packet drop. Let us for now define our 
network routing model (conceptual model) informally in the following terms. 
Delay of a~packet is obtained using a~physical delay and the total processing 
time in the router. We consider that transmission of packets in 
a~network can have a~random behavior, caused by the random character of both, 
the input, and possible loss of packets. There we are interested in 
a~probabilistic model, namely, a~Markov model. In order to fully characterize 
the network as an~MC, the internal state of each router 
(and, in particular, the buffer occupancies), as well as the characteristics
 of all flows, need to be expressed as states in the chain. 

However, such approach would result in an enormous and intractable number of states. 
Therefore, to simplify these computations, let us characterize the delay time as 
an abstract variable~$t$. This abstract variable can be interpreted in different ways, 
e.\,g., the current processing queue length and a~packet transmission rate of the link, 
or possibly a~fixed value, such as an interval between the beginning of 
a~packet transmission after being processed in some node and the end of processing 
at the next node. 

We describe the functioning of the network in the transmission of packets 
as transitions of a~discrete-time MC (DTMC). The state space corresponds to the set 
of nodes such that 
the transmission of a~packet from a~node that has finished processing the packet 
to the next node corresponds to the transition of the chain to the next state.


Discrete-time MC is defined as a~tuple $D\linebreak =(S, s_0, P)$. In the tuple, $S$ is 
the finite set of states, $s_0\in S$ is the initial
state, $P:S \times S \rightarrow [0, 1]$ is the transition probability matrix in 
which $\forall s\in S$, $\sum\nolimits_{s' \in S} P(s,s') = 1$. 
For any two states $s, s' \in S$, if $P(s,s')>0$, then~$s'$ is the successor of~$s$. 
For a~subset of states $T \subseteq S$, the probability of moving from a~state~$s$ 
to any state $t \in T$ in a~single step is denoted by $P(s, T)$ and is given by 
$P(s,T)=\sum\nolimits_{t \in T} P(s, t)$. 
%The row $P(s,:)$, in the transition matrix $P$, contains the probabilities of moving from $s$ to its successors, while the column $P(:, s)$ contains the probabilities of entering the state $s$ from any other state.

\vspace*{-6pt}

\subsection{Verification syntax}

\noindent
For implementation of our PCTL-based model, we use PRISM~--- 
probabilistic model checker~\cite{kwiatkowska_prism_2011}. There, we follow 
PRISM property specification language. Here, we briefly describe the essential 
syntax while more details can be found in~\cite{noauthor_prism_nodate}.

Given a property~$\Psi$, we say that~$\Psi$ is true with probability~$p$ 
and write that as
$P_p [ \Psi ]$. If the probability~$p$ is unknown, PRISM allows, for DTMC, 
writing properties queries of the form $P_{=?}[ \Psi ]$, meaning 
``what is the probability that~$\Psi$ is true?''. Additionally, it is possible 
to use a~time bound and write properties queries such as 
$P_{=?}[F^{\leq T} \Psi]$, meaning ``what is the probability that~$\Psi$ 
is true after less than~$T$~steps?''. At last, it is possible to compute 
properties such as expected time or expected number of steps. 
For example, $R_{=?}[F \Psi]$, meaning ``what is the expected number of 
steps until $\Psi$ is true?''. 
%\section{Model Settings}
%, and a subroute of route $X$ from router $a$ to router $b$ is specified by $[a,b]_X$

%When a new subroute of $N$ that is scheduled to update a current sub-route of $C_i$ is launched, the route $C$ is updated such that the updated sub-route is replaced by launched sub-route, and the new sub-route is now part of the current route $C$.

\setcounter{figure}{3}
\begin{figure*}[b] %fig4
\vspace*{-6pt}
 \begin{center}
 \mbox{%
 \epsfxsize=149.177mm 
 \epsfbox{fre-3.eps}
 }
 \end{center}
\vspace*{-9pt}

 \Caption{New routes~$N_1$~(\textit{a}) and $N_2$~(\textit{b}) and
 MC states for~$N_1$~(\textit{c}) 
and~$N_2$~(\textit{d})}
 \label{fig:routes_dtmc_example}
\end{figure*}



\vspace*{-6pt}

\section{Prediction of Preferred Update}
%\section{Prediction of Preferred Update}
\label{sec:dtmc}

\noindent
The states of a~DTMC describe the nodes in the new route and the transition 
probabilities in the chain represent the possible delay or 
a~packet loss in the routers along the new route. The
states are defined as 
$\{s_1, \ldots , s_n\}$ where~$n$ is the number
  of nodes in the new route. 
The network achieves the state~$s_i$ if a packet has reached the $i$th node. 
For example, in Fig.~3, the self-transition 
edge represents the probability for a~delay due to packet loss, rules installation 
at the router, or congestion on the router-controller link, while the 
forward transition edge represents the probability for 
a~successful transition to the next state. These probabilities can be estimated 
from network statistics (see, for example,~\cite{hogan_stochastic_2017}). 
The labels on edges are the probability values, when edge has no label
 means probability~1.
 
 The initial probability distribution of states is given by the vector~$P_0$ of size~$n$. 
We can determine the prob-\linebreak\vspace*{-12pt}
 
 %\linebreak\vspace*{-12pt}

{ \begin{center}  %fig3
\vspace*{-0.5pt}
  \mbox{%
 \epsfxsize=77.518mm 
 \epsfbox{fre-4.eps}
 }


\end{center}

\vspace*{-3pt}

\noindent
{{\figurename~3}\ \ \small{Probability as a~function of number of steps to update routes~$N_1$~(\textit{1})
 and~$N_2$~(\textit{2})}}
}

\vspace*{12pt}



\noindent
ability that a~particular route delays the update process 
by~$k$, that is, the number of steps required for a~successful update is given by 
$p(k)=P_0 P^k$. Using this characteristic, which is, in fact, the 
probability distribution of the number of steps $P(k < x)$, one can 
calculate various properties like average delay time for the new route, 
maximum or minimum number of steps to update, etc.
 
 Consider the example illustrated in Fig.~4. 
Figure~4\textit{a} illustrates the current route~$C$ and a candidate new route~$N_1$. 
Figure~4\textit{b} shows the same current route~$C$ with another candidate 
new route~$N_2$. 
Figures~4\textit{c} and~4\textit{d} 
show the MCs for new routes~$N_1$ and~$N_2$, accordingly, with given transition 
probabilities.

During the update process, packets are sent along the current and the new routes. 
Since the new route is\linebreak\vspace*{-9.5pt}

\columnbreak

\noindent
 not operational yet, packets can be delayed due to 
congestion on certain nodes or due to switch configurations. 
%
For example, if routing rules have not yet been installed in some switch, then an 
arriving packet is sent to the controller~\cite{onf_openflow_2015}. The controller 
then decides reactively on further actions whether to install an appropriate rule 
for the packet. Also, the controller may be busy with other work and not respond 
immediately. Those packet processing actions may delay the update process. 
In the case buffer becomes full, for example, if the network is being congested, 
packets may be dropped. There, the transition to the next state during the 
update process depends on the likelihood of a~delay or a~loss of a~packet in the 
current state. 

In the example, the number of steps required for launching~$N_2$ is smaller than 
the number of steps required for launching~$N_1$. However, due to a higher likelihood 
of delays along the route~$N_2$, it is possible that~$N_1$ is preferred having 
a~higher probability for a~successful update. The network administrator may ask 
which new route is recommended for the update process, considering the expected 
number of steps required for the update. 
%
That is, updating paths requires the operator to decide 
on the possible choice of a~subroute for the next step. 
One should consider the possibility of including a~decision tool augmenting the 
controller during route updates. 

There were many attempts to use the LP/ILP 
approach, as it was already mentioned above (see, e.\,g.,~\cite{juttner_lagrange_2001}), 
but they have encountered the same difficulties, especially when taking 
into account online implementation. We show that it is possible to describe 
the routing process as DTMC. Thus, taking into consideration~$O(n^3)$ worst case 
computation complexity, we consider using the ``design via verification'' 
mentioned above based on PCTL verification, similar to the one used in 
PRISM~\cite{kwiatkowska_prism_2011}.


We have calculated the probability for a~successful update as a~function of 
number of steps for routes~$N_1$ and~$N_2$ from the example in 
Fig.~\ref{fig:routes_dtmc_example}. See Fig.~3 
where this function is shown. Curve~\textit{1}
represents the plot for~$N_1$ and curve~\textit{2} represents
 the plot for~$N_2$. 

Observe that after~20~steps, both new routes will be launched with probability~1 
which can be written as 
$$
P_{1}\left[F^{>20}N_1\right]=P_{1}\left[F^{>20}N_2\right]=1\,.
$$
The expected number of steps required for~$N_1$ is smaller than the required for~$N_2$:
$$
R \left[F~N_1\right] < R \left[F~N_2\right]\,.
$$
However, the probability for successfully updating in less than~15~steps 
is higher for route~$N_2$ ($0.55 \pm 0.040$ for~$N_1$ and 
$0.717 \pm 0.036$ for~$N_2$, based on~99\% confidence level):
$P_{0.717 \pm 0.036}\left[F^{\leq 15} N_2 \right].$

\vspace*{-6pt}


\section{Route Updates per~Quality~of~Service}
\label{sec:updates_qos}

\vspace*{-2pt}

\noindent
In this section, we show algorithm that we propose for various settings. 
First, we show an enhancement for the sequential update algorithm 
from~\cite{delaet_seamless_2015}, which during the update process decides on 
preferred subroute from the set of possible subroutes as part of QoS requirements. 
In the multicast-based update, several methods were proposed 
in~\cite{delaet_seamless_2015} for eliminating duplicated packets. 
In the case the common destination router is not able to immediately eliminate 
duplicated packets, the algorithm begins the update from the end, 
ensuring a~correct update process~[4].



\begin{algorithm*} %alg1
 \setlength{\algowidth}{100mm}
 \setlength{\hsize}{\algowidth}
 \caption{Update per QoS Algorithm}
 \label{alg:update_per_qos}

%\hrule
%\vspace*{2pt}
%\centerline
%{\textbf{Algorithm~1:} Update per QoS Algorithm}\par

%\vspace*{2pt}

%\hrule
 \small
 
 %\Input
 {directed graph $G$} 
 
 \BlankLine
 \tcc{$A$ is a collection of nodes} $A \leftarrow$ choose nodes from $G$ with in-degree $0$ \\
 
 \Repeat {out-degree of node $N^t_i > 0$}
 {
 \ForEach{$v \in A$ \label{alg:inner_loop}}
 {
 calculate $R[F~v]$ \\
% calculate the expected QoS for this node as described in Section \ref{sec:updates_qos} \\
 }\label{alg:end_inner_loop}
 
% $N^t_i \leftarrow$ choose the node that maximizes QoS \label{alg:choose_qos}\\ 
 $N^t_i \leftarrow \argmax_{v} (R[F~v])$ \label{alg:choose_qos} \\
 launch $N^t_i$ \\
 update $C$ accordingly \\
 merge any new and common subroutes as described in section~3 \\ 
 $A \leftarrow$ choose nodes neighboring to $N^t_i$ \\ 
 }
 
 \BlankLine 
 
\end{algorithm*}





 
%The algorithm starts from any node with in-degree 0 since it means that such node has no precedence dependence. Updating is completed when the algorithm arrives to a node with out-degree zero, which would be the last subroute to launch.


After that, we show an algorithm that chooses the subroutes for update arbitrary, 
assuming that the common destination node will not leak duplicated packets. 
However, the packets sending rate along the new subroute need to be temporarily limited~[4].

At last, we present a supplementing algorithm that suggests which subroutes can 
be updated in parallel.

%The set of common nodes for each pair of routes subdivides the routes to sub-routes relatively to each other. 

\vspace*{12pt}

\subsection{Sequential update}

\noindent
Let us begin the update from the end, namely, from the last alternative 
subroute of any new route. Provably, this prevents the formation of 
cycles~\cite{delaet_seamless_2015}. In order to represent all possible choices 
of a~path from a current state of the update process to the end of the update process, 
we propose to use a directed graph which nodes are the new, legitimate for launching, 
subroutes of the network. The edges of the graph represent a~legal order of launching 
new subroutes. Each path in this graph from a~current node to the last node in 
the path represents a~legal combination of chosen subroutes. The update process is 
continued as long as there is a~possible node to transition to. 

Let us examine the two possible new routes~$N_1$ and~$N_2$ that can replace the 
current route~$C$ from the example depicted in Fig.~1. 
The new route~$N_1$ is composed of~$N^1_1$ and~$N^2_1$, while the new route~$N_2$ 
composed of~$N^1_2$, $N^2_2$, $N^3_2$, and~$N^4_2$. Starting from the end, the only 
new subroutes that are allowable to launch are~$N^2_1$ and~$N^4_2$. 
Assume that based on the DTMC calculations performed as described in section~4, 
the subroute~$N^4_2$ is chosen for update. After the update of the subroute is 
complete, the current route~$C$ is composed of not updated yet part of the old 
route and~$N^4_2$. See Fig.~2 where the change in~$C$ 
is depicted.

After the subroute~$N^4_2$ is launched, we arrive at a~smaller problem in which 
less subroutes are left to update. Due to dynamics of the system 
(see section~3), some new subroutes can merge into a~single new subroute.
See Fig.~2 where after~$N^4_2$ was launched, the 
new subroutes~$N^3_2$ and~$N^2_2$ are merged into a~single subroute. Now, one 
can launch either~$N^1_1$ or~$N^2_2$ merged with~$N^3_2$. Assume that we choose to 
launch~$N^1_1$, which launch
 finishes the update. The route~$C$ updated to~$N^1_1$ 
and~$N^4_2$. See Fig.~5 illustrating that.


Figure~6 shows the directed graph that represents 
the possible update sequences. Initially, the subroutes that %\linebreak\vspace*{-12pt}
 are legal 
for launch are~$N^2_1$ and~$N^4_2$. As such, these are
the only subroutes that
 have in-degree~0. Launching~$N^3_2$
 is forbidden; hence, there is no node in the 
 graph~$G$ that represents this subroute. After launching~$N^4_2$, we\linebreak\vspace*{-12pt}
 
 \setcounter{figure}{4}

{ \begin{center}  %fig5
\vspace*{12pt}
 \mbox{%
 \epsfxsize=78.631mm 
 \epsfbox{fre-5.eps}
 }


\vspace*{3pt}


\noindent
{{\figurename~5}\ \ \small{$N^1_1$ was launched}}
\end{center}
}

\vspace*{6pt}

{ \begin{center}  %fig6
\vspace*{1pt}
 \mbox{%
 \epsfxsize=36.428mm 
 \epsfbox{fre-6.eps}
 }


\end{center}


\noindent
{{\figurename~6}\ \ \small{Graph 
representation for possible update paths for routes update example from Fig.~1}}

}

%\vspace*{6pt}

\noindent
  can 
 proceed by launching~$N^1_1$ or~$N^2_2$. However, if~$N^2_1$ was launched first, 
 it would be forbidden to launch~$N^2_2$ since it shares a~common edge with~$N^2_1$. 
 This is reflected in the graph~$G$ by not having a~directed edge from the
  node~$N^2_1$ to the node~$N^2_2$. We finish the update process
 by arriving either 
 to~$N^1_1$ or to~$N^1_2$. Notably, these nodes have out-degree~0.

 Algorithm~1 updates subroutes according to calculated QoS for each new subroute, by
 choosing at each step the new subroute that maximizes QoS.


The algorithm starts by selecting the initial set of subroute nodes. 
These are nodes with in-degree~0. The algorithm continues traversing the graph up 
to arrival at a node with out-degree~0 which would be the last subroute to launch. 
The inner loop at lines~\ref{alg:inner_loop}--\ref{alg:end_inner_loop} 
calculates the QoS for each neighboring node. Afterward, at 
line~\ref{alg:choose_qos}, the algorithm chooses the node that maximizes QoS. 
Then launches this node and updates the route~$C$, accordingly (see 
Figs.~1--5 for illustration). 
Afterward, the algorithm selects the next neighboring nodes.

After execution of Algorithm~1, the resulting new route maximally complies QoS 
requirements.

%\vspace*{12pt}

\subsection{Arbitrary subroutes selection} 
%\label{sec:arbitrary}

%\vspace*{-12pt}

\noindent
In this subsection, we assume that immediate duplicate packets elimination is possible. 
It may be that some of the subroutes are not ready for an update yet. 
Thus, meanwhile, the administrator may want to proceed with the update process 
to other subroutes or see possible variations of the update. 
For such scenario, we provide an algorithm which can select a~subroute for 
update arbitrary and continue the update process from there. 
We create a~forest graph of all possible update combinations from which the 
desired update sequence can be chosen. 
{\looseness=1

}
 


Figure~7 shows all possible combinations from example 
in Fig.~1. Noticeable, as mentioned earlier, some\linebreak\vspace*{-12pt}

{ \begin{center}  %fig7
\vspace*{1pt}
  \mbox{%
 \epsfxsize=71.694mm 
 \epsfbox{fre-7.eps}
 }


\end{center}


\noindent
{{\figurename~7}\ \ \small{Forest graph representing execution combinations for example from 
 Fig.~1}}
}

\vspace*{12pt}


\noindent
 combinations 
exhibit fewer steps, though possible that its QoS compliance is worse than others.



Algorithm~2 starts by iterating over all roots of the forest graph and 
calculating QoS using Algorithm~1 each tree. Afterward, launch the update 
of the tree that maximizes QoS.

\begin{algorithm*} %alg2
\setlength{\algowidth}{100mm}
 \setlength{\hsize}{\algowidth}
 \caption{Arbitrary Selection Update}
 \label{alg:arbitrary_update}
 \small
 
% \Input
{directed graph $G$} 
 
 %\BlankLine
 
 $A_0 \leftarrow$ choose nodes from $G$ with in-degree $0$ \\
 $Q \leftarrow \{\}$ \\
 
 \BlankLine
 \tcc{iterate over all roots of trees in the forest $G$}
 \ForEach{$v_r \in A_0$}
 {
 $q \leftarrow$ get the expected QoS using Algorithm~1 for $v_r$ \\
 $Q \leftarrow Q \cup \{q \rightarrow \mathrm{root} \}$ \\
 }

 \BlankLine
 $q_{\max} \leftarrow \max_{\mathrm{QoS}}(Q)$ \\
 launch maximizing QoS update order in $\mathrm{root}=Q[q_{\max}]$ \\ 
 
 
\end{algorithm*}


%\columnbreak

\vspace*{12pt}





\subsection{Parallel update}

\noindent
In certain cases, it is possible to update in parallel several subroutes 
and, as such, decrease update time. However, launching subroutes in parallel 
is not always possible
 since subroute may share a~link and, thus, leads to congestion 
during the update process, close a~cycle, or lead to an inconsistent state of the 
system. In~\cite{delaet_seamless_2015}, it was shown that two new subroutes~$N'$ 
from~$a$ to~$b$ and~$N''$ from~$c$ to~$d$ can be launched in parallel only if~$c$ 
succeeds~$b$ or~$a$ succeeds~$d$.



%\begin{proposition}
% Let $N'$ from $a$ to $b$ and $N''$ from $c$ to $d$ be two legitimate new subroutes. $N'$ and $N''$ can be launched in parallel only if $c$ succeeds $b$ or $a$ succeeds $d$.
%%Two subroutes that are each legitimate can be launched in parallel only if they share at most one common subroute.
%\end{proposition}
%\begin{proof}
% \textbf{Direction}: $\Rightarrow$ Let $N'$ from router $a$ to $b$ and $N''$ from router $c$ to $d$, be two new legitimate sub-routes. The only way for them to share more than one common sub-route is if $b$ succeeds $c$ on $C$. In such case, launching $N'$ will eliminate the part of $C$ from $c$ to $b$ with no proper connection from $b$ to $c$, which leaves the system in an inconsistent state. The same occurs if $N''$ is launched. \\
% \textbf{Direction}: $\Leftarrow$ Let $N'$ from router $a$ to $b$ and $N''$ from router $c$ to $d$, be two new sub-routes, not necessary part of the same new route, such that $b$ precedes $c$ or $b=c$. If $a$ precedes $b$, than $N'$ is legal for launching independently of $N''$. Similarly, if $c$ precedes $b$, than $N''$ is legal for launching independently of $N'$. Thus, since $N'$ can be launched independently from $N''$, they can be launched in parallel. Symmetric considerations lead to same result in case $a$ succeeds $d$.
% 
%\noindent Generalization to more than two sub-routes is trivial.
%\end{proof}



\begin{algorithm*}[b] %[t] %alg3
\setlength{\algowidth}{100mm}
 \setlength{\hsize}{\algowidth}
 \caption{Parallel Update}
 \label{alg:parallel_update}
 \small
 
 %\Input
 {weighted graph $G_S$} 
 
 \BlankLine
 
 \While{there are still current subroutes to update}
 {
 $A \leftarrow$ find maximum-weight independent set in $G_S$ \\
 
 \BlankLine 
 \tcc{do in parallel} 
 \ForEach{$N^t_i \in A$} 
 { 
 launch $N^t_i$ \\
 }
 }
 
 \vspace*{6pt}
 
\end{algorithm*}

We create a supplementary graph~$G_S$, in which nodes are the new legitimate 
for launching subroutes, and edges represent restrictions on parallel 
launching of subroutes. See Fig.~8 for illustration, 
depicting subroutes from example in Fig.~1 and their parallel 
restrictions. For example, $N^4_2$ and~$N^1_2$ can be launched in parallel since 
there is no edge connecting them.

Clearly, any independent set of subroutes from the supplementary 
graph contains subroutes that can be launched in parallel. 
This can be further enhanced by setting QoS calculated values as weights 
on nodes of the graph and finding the subroutes that can be launched 
in parallel by finding a~maximum-weight independent set of the graph~$G_S$. 
Since~$G_S$ has few
 number of nodes (several tens), it is possible to find 
the
 maximum-weight independent set even by enumerating
 all possible independent 
sets~\cite{wu_review_2015} and comparing their total weights.
{\looseness=-1



{ \begin{center}  %fig8
\vspace*{12pt}
  \mbox{%
 \epsfxsize=36.666mm 
 \epsfbox{fre-8.eps}
 }


\end{center}


\noindent
{{\figurename~8}\ \ \small{Supplementary graph of the example in 
 Fig.~1, showing which subroutes cannot be run in parallel}
}}

%\vspace*{12pt}



} 



Important, the parallel method should not be launched on its own. 
For example, assume that at the first iteration of Algorithm~3, 
the independent sets of nodes are~$A_1$ and~$A_2$. Let us assume that~$A_1$ complies 
better to QoS demands than~$A_2$ and, thus, $A_1$ will be selected. 
Also, let us assume that~$B_1$ is the next independent set in the graph 
if~$A_1$ was selected and~$B_2$ if~$A_2$ was selected. 
Also, let us assume that~$B_1$ is
the next independent set in the graph if~$A_1$ was selected and~$B_2$ if~$A_2$ 
was selected.
It is possible that due to the dynamics of the system (see section~3), 
we could obtain overall higher QoS results if we initially launched the 
subroutes from the sets~$A_2$ and~$B_2$ afterwards than from the sets~$A_1$ and~$B_1$.
 

Therefore, the graph that we create in this section for parallelization constraints 
is a~supplementary graph which must be used in conjunction with the graphs from 
previous sections. Optimal results will be obtained when used in conjunction with 
the forest graph from subsection~5.2.

It is also important to note that, in the worst case, when there are 
no disjoint subroutes, the parallel method is reduced to the sequential 
method thought with a higher running time.

\vspace*{-12pt} 


\section{Implementation}

\noindent
We implemented the update algorithms from~\cite{delaet_seamless_2015} as 
services for our QoS verification module. The update algorithm itself 
was not modified. In other words, we treated the update itself as 
an atomic action. The route updates
 algorithms are implemented as 
applications interacting with the northbound interface of an SDN controller. 
We used POX~\cite{kaur_network_2014} as a~platform for controller development and 
Mininet~\cite{lantz_network_2010} for network topology emulation. 
Figure~9 depicts the schematic arrangement of the 
functional elements. 



We created networks with topology of random graph and small-world features. 
During each simulation trial, a~pair of common source and destination nodes $(s,d)$ 
were selected. A~path connecting~$s$ and~$d$ was selected as a~current route and 
a~set of~4~new routes connecting $(s,d)$, to replace the current route, were 
selected, possibly with shared links among themselves and the current route. 

We considered latency due to the formed congestion as QoS demands for the update, 
implemented by forming congestion on randomly selected subroutes. Route 
update was executed by the update algorithm from~\cite{delaet_seamless_2015} for 
each pair of current and new routes. Further, one of the enhanced versions 
was executed, updating to the
 preferred combination of subroutes, by identifying 
the congested subroutes (e.\,g., by estimating latency).

{ \begin{center}  %fig9
\vspace*{8pt}
  \mbox{%
 \epsfxsize=58.544mm 
 \epsfbox{fre-9.eps}
 }

\vspace*{3pt}


\noindent
{{\figurename~9}\ \ \small{Description of the system}
}
\end{center}}

%\vspace*{12pt}



%\vspace*{-45pt}

\section{Concluding Remarks}

\noindent
The study in this paper illustrates a~feasibility of modeling and 
designing the route update process via verification using DTMC. The goal was to 
strengthen the network administrator involvement in management and decision 
making during route update. In the present model, the network administrator is able 
to consider network parameters such as packet losses, delay, communication 
rounds, flow table updates, congestion, and other inherent unreliabilities of 
the network. 

We extended the updating algorithm with the ability to compute QoS as the 
MC characteristics, where the MC corresponds to the states 
of the update process. Using this MC computation ability, it is 
possible to predict the expected number of steps (delay time) required to 
complete the update process. These prediction results allow the administrator 
to make a~decision whether a~new route can satisfy the user requirements per QoS 
or a~more reliable route will be selected.

We provided sequential update algorithm and an arbitrary order algorithm 
when for the later, it is assumed that immediate duplicate packets elimination 
is possible. Further, we suggest a supplementary graph and algorithm for launching 
updates in parallel when it is possible.

This paper proposes a~conceptual approach. In future research, we will focus 
on optimization of predictions supplementing the network administrator with 
a~powerful tool which will be able to enhance the update process 
with fine grained analysis of the network.

\vspace*{-12pt}


\Ack
\noindent
The first author has partially been supported by the 
Russian Foundation for Basic Research under grants RFBR 18-07-00669 and 18-29-03100. 
The second author has partially been supported by the Rita Altura Trust Chair in
Computer Sciences; The Lynne and William Frankel Center for Computer
Science.

%\bigskip


The authors thank Prof.\ Shlomi Dolev 
for his valuable input and Prof.\ Yefim Dinitz for his comments.
 
\renewcommand{\bibname}{\protect\rmfamily References}

%\vspace*{-6pt}

\vspace*{-6pt}

{\small\frenchspacing
{\baselineskip=10.35pt
\begin{thebibliography}{99}



\bibitem{rao_sdn_2014}  %1
\Aue{Rao, S.\,K.} 2014. SDN and its use-cases~--- NV and NFV:
A~state-of-the-art survey. NEC Technologies India Ltd. 25~p.

\bibitem{ghaznavi_service_2016}  %2
\Aue{Ghaznavi, M., N.~Shahriar, R.~Ahmed, and R.~Boutaba}. 2016. 
Service function chaining simplified. {arXiv.org}. arXiv:1601.00751.

\bibitem{hansson_logic_1994}  %3
\Aue{Hansson, H., and B.~Jonsson}. 
1994. A~logic for reasoning about time and reliability. 
\textit{Form. Asp. Comput.} 6(5):512--535.

\bibitem{delaet_seamless_2015}  %4
\Aue{Delaet, S., S.~Dolev, D.~Khankin, S.~Tzur-David, and T.~Godinger}. 
2015. Seamless SDN route updates. \textit{IEEE 14th Symposium (International)
on Network Computing and Applications}. IEEE. 120--125.

\bibitem{frenkel_predicting_2017} 
\Aue{Frenkel, S., D.~Khankin, and A.~Kutsyy}. 
2017. Predicting and choosing alternatives of route updates per QoS VNF in SDN. 
\textit{IEEE 16th Symposium (International) on Network Computing and Applications}. 
IEEE. 1--6. 

\bibitem{devi_approach_2015} 
\Aue{Devi, G., and S.~Upadhyaya}. 2015. 
An approach to distributed multi-path QoS routing. 
\textit{Indian J.~Sci. Technol.} 8(20):1--14. 
doi: 10.17485/ijst/2015/v8i20/49253.

\bibitem{egilmez_distributed_2012} 
\Aue{Egilmez, H.\,E., S.~Civanlar, and A.\,M.~Tekalp}. 2012. 
A~distributed QoS routing architecture for scalable video streaming over multi-domain 
OpenFlow networks. \textit{19th IEEE Conference (International) on Image Processing}.
IEEE. 2237--2240.

\bibitem{juttner_lagrange_2001} 
\Aue{Juttner, A., B.~Szviatovski, I.~Mecs, and Z.~Rajko}. 2001. 
Lagrange relaxation based method
for the QoS routing problem. \textit{IEEE Conference on Computer Communications. 
20th Annual Joint Conference of the IEEE Computer and Communications Society
 Proceedings}. IEEE. 2:859--868.

\bibitem{yu_efficient_2013} %9
\Aue{Yu, Z., F.~Ma, J.~Liu, B.~Hu, and Z.~Zhang}. 2013. 
An efficient approximate algorithm for disjoint QoS routing.
\textit{Math. Probl. Eng.} 2013:489149. 9~p. 
doi: 10.1155/2013/489149.

\bibitem{foerster_survey_2016} 
\Aue{Foerster, K.-T., S.~Schmid, and S.~Vissicchio} 2016. 
A~survey of consistent network updates. \mbox{Arxiv.org}. \mbox{arXiv}:\linebreak 1609.02305.

\bibitem{reitblatt_consistent_2011} 
\Aue{Reitblatt, M., N.~Foster, J.~Rexford, and D.~Walker}. 
2011. Consistent updates for software-defined networks: Change you can believe in! 
\textit{10th ACM Workshop on Hot Topics in Networks Proceedings}.
New York, NY: ACM. Art.\ No.\,7. doi: 10.1145/2070562.2070569.

\bibitem{hogan_stochastic_2017} 
\Aue{Hogan, M., and F.~Esposito}. 
2017. Stochastic delay forecasts for edge traffic engineering via Bayesian networks. 
\textit{IEEE 16th Symposium (International) on Network Computing and Applications}. 
IEEE. 1--4.

\bibitem{mcgeer_safe_2012} %15
\Aue{McGeer, R.} 2012. A~safe, efficient Update Protocol for Openflow Networks. 
\textit{1st Workshop on Hot Topics in Software Defined Networks Proceedings}. 
New York, NY: ACM. 12:61--66.
\bibitem{mcgeer_correct_2013} 
\Aue{McGeer, R.} 2013. A~correct, zero-overhead protocol for network updates. 
\textit{2nd ACM SIGCOMM Workshop on Hot Topics in Software Defined Networking
Proceedings}. New York, NY: ACM. 13:161--162.
\bibitem{katta_incremental_2013} 
\Aue{Katta, N.\,P., J.~Rexford, and D.~Walker}. 
2013. Incremental consistent updates. \textit{2nd ACM SIGCOMM Workshop on Hot Topics 
in Software Defined Networking Proceedings}.
New York, NY: ACM. 13:49--54.

\bibitem{dinitz_dependence_2017}  %16
\Aue{Dinitz, Y., S.~Dolev, and D.~Khankin}. 
2017. Dependence graph and master switch for seamless dependent routes 
replacement in SDN. \textit{IEEE 16th Symposium 
(International) on Network Computing and Applications}. IEEE. 1--7.

\bibitem{17-aaa}
\Aue{Amiri, S.\,A., S.~Dudycz, S.~Schmid, and S.~Wiederrecht}.
2016. Congestion-free rerouting of flows
on DAGs. \mbox{ArXiv}.org. arXiv:1611.09296.
% [cs, math], Nov. 2016, arXiv: 1611.09296. [Online]. Available:
%http://arxiv.org/abs/1611.09296

\bibitem{kwiatkowska_prism_2011}  %17
\Aue{Kwiatkowska, M., G.~Norman, and D.~Parker}. 2011. 
PRISM~4.0: Verification of probabilistic real-time systems. 
\textit{Computer aided verification}.
Eds. G.~Gopalakrishnan and S.~Qadeer.
Lecture notes in computer science ser. Springer.
6806:585--591.

\bibitem{noauthor_prism_nodate}  %18
\Aue{Kwiatkowska, M., G.~Norman, and D.~Parker}. 2018. 
{PRISM manual}. Available at:
{\sf http://www.\linebreak prismmodelchecker.org/manual/}
(accessed December~10, 2018).

\bibitem{onf_openflow_2015} %19
{Open Networking Foundation}. 2015. 
OpenFlow Switch Specification Ver~1.5.1. 


\bibitem{wu_review_2015}  %20
\Aue{Wu, Q., and J.-K.~Hao}. 2015. 
A~review on algorithms for maximum clique problems. 
\textit{Eur. J.~Oper. Res.} 242(3):693--709.

\bibitem{kaur_network_2014}  %21
\Aue{Kaur, S., J.~Singh, and N.\,S.~Ghumman}. 2014. 
Network programmability using POX controller. 
\textit{Conference (International) on Communication, Computing and Systems}.
138.

\bibitem{lantz_network_2010}  %22
\Aue{Lantz, B., B.~Heller, and N.~McKeown}. 2010. 
A~network in a~laptop: Rapid prototyping for software-defined networks. 
\textit{9th ACM SIGCOMM Workshop on Hot Topics in Networks Proceedings}. 
New York, NY: ACM.  Art.\ No.\,19. doi: 10.1145/1868447.1868466.
\end{thebibliography} } }

\end{multicols}

\vspace*{-9pt}

\hfill{\small\textit{Received October 9, 2018}}

\vspace*{-22pt}

\Contr

\vspace*{-3pt}

\noindent
\textbf{Frenkel Sergey L.} (b.\ 1951)~--- 
Candidate of Science (PhD) in technology, associate professor, 
senior scientist, Institute of Informatics Problems, Federal Research Center 
``Computer Sciences and Control'' of the Russian Academy of Sciences, 
44-2~Vavilov Str., Moscow 119333, Russian Federation; \mbox{fsergei51@gmail.com}

\vspace*{1pt}

\noindent
\textbf{Khankin D.} (b.\ 1983)~--- MSc, doctorate student, Department of Computer 
Science, Ben-Gurion University of the Negev, Beer-Sheva 84105, Israel; 
\mbox{danielkh@post.bgu.ac.il}

\vspace*{4pt}

\hrule

\vspace*{2pt}

\hrule

\vspace*{-7pt}

%\newpage

%\vspace*{-28pt}

\def\tit{НЕПРЕРЫВНЫЕ ОБНОВЛЕНИЯ МАРШРУТА В~SDN С~ИСПОЛЬЗОВАНИЕМ ПРОВЕРКИ СООТВЕТСТВИЯ 
КАЧЕСТВУ~ОБСЛУЖИВАНИЯ$^*$\\[-7pt]}

\def\titkol{Непрерывные обновления маршрута в~SDN с~использованием проверки соответствия 
качеству обслуживания}

\def\aut{С.\,Л.~Френкель$^1$, Д.~Ханкин$^2$\\[-7pt]}

\def\autkol{С.\,Л.~Френкель, Д.~Ханкин}

{\renewcommand{\thefootnote}{\fnsymbol{footnote}} \footnotetext[1]
{Работа была частично поддержана РФФИ (гранты 18-07~00669 и~18-29-03100), 
а~также Rita Altura Trust Chair in
Computer Sciences; The Lynne and William Frankel Center for Computer
Science.}}



\titel{\tit}{\aut}{\autkol}{\titkol}

\vspace*{-22pt}

\noindent
$^1$Институт проблем информатики Федерального исследовательского центра 
<<Информатика и~управление>>\linebreak
$\hphantom{^1}$Российской академии наук
%, fsergei51@gmail.com 

\noindent
$^2$Университет им.\ Бен-Гуриона в Негеве, Беэр-Шева, Израиль
%, danielkh@post.bgu.ac.il 

\vspace*{1pt}

\def\leftfootline{\small{\textbf{\thepage}
\hfill ИНФОРМАТИКА И ЕЁ ПРИМЕНЕНИЯ\ \ \ том\ 12\ \ \ выпуск\ 4\ \ \ 2018}
}%
 \def\rightfootline{\small{ИНФОРМАТИКА И ЕЁ ПРИМЕНЕНИЯ\ \ \ том\ 12\ \ \ выпуск\ 4\ \ \ 2018
\hfill \textbf{\thepage}}}

\vspace*{-1pt}


 
\Abst{В программно-определяемой сети (SDN~--- software-defined networking) 
уровень управ\-ле\-ния 
и~уровень данных разделены. Это обеспечивает высокую гибкость эксплуатации, 
предоставляя абстракции для управления сетью приложений 
и~возможность непосредственного программирования маршрутов.
Однако из-за изменений топологии, процедуры обслуживания или происходящих 
сбоев иногда необходима реконфигурация и~обновление сети. 
В~предлагаемом сценарии рассматривается текущий маршрут~$C$
и~набор возможных новых маршрутов~~$\{N_i\}$, где для замены текущего 
маршрута требуется 
один из\linebreak\vspace*{-12pt}}

\Abstend{новых маршрутов. Существует вероятность того, что новый маршрут~$N_i$ 
окажется длиннее некоторого другого нового маршрута~$N_j$, но при этом~$N_i$ 
будет более надежным и~он будет обновляться быстрее или работать лучше 
после обновления с~точки зрения требований качества обслуживания (QoS~---
quality of service). Принимая 
во внимание случайный характер функционирования сети, авторы дополнили недавно 
предложенный алгоритм обновления маршрута Delaet с~соавт.\ методом оценки соблюдения 
требований QoS во время непрерывного обновления маршрута, основанным на 
использовании цепей Маркова. При этом, во-пер\-вых, предлагается расширить 
алгоритм передачи пакетов по выбранному маршруту, сравнивая процесс обновления 
для возможных альтернатив маршрута. Во-вто\-рых, предлагается несколько 
способов выбора комбинаций предпочтительных отрезков путей новых маршрутов, 
что приводит к оптимальному в~смысле соответствия QoS маршруту.}


\KW{программно-определяемые сети; цепи Маркова; качество обслуживания}

\DOI{10.14357/19922264180408}



%\vspace*{-3pt}


 \begin{multicols}{2}

\renewcommand{\bibname}{\protect\rmfamily Литература}
%\renewcommand{\bibname}{\large\protect\rm References}

{\small\frenchspacing
{\baselineskip=10.5pt
\begin{thebibliography}{99}
%\vspace*{-3pt}


\bibitem{2-fr-1}
\Au{Rao S.\,K.} SDN and its use-cases~--- NV and NFV: A~state-of-the-art survey.~--- 
NEC Technologies India Ltd., 2014. 25~p.
\bibitem{3-fr-1}
\Au{Ghaznavi M., Shahriar~N., Ahmed~R., Boutaba~R.} 
Service function chaining simplified~// Arxiv.org, 2016. \mbox{arXiv}:1601.00751cs.
\bibitem{4-fr-1}
\Au{Hansson H., Jonsson~B.} A~logic for reasoning about time and reliability~// 
Form. Asp. Comput., 1994. Vol.~6. No.\,5. P.~512--535.

\bibitem{1-fr-1} %4
\Au{Delaet S., Dolev~S., Khankin~D., Tzur-David~S., Godinger~T.}
Seamless SDN route updates~// IEEE 14th Symposium (International)
 on Network Computing and Applications.~--- IEEE, 2015. P.~120--125.
 
 
\bibitem{5-fr-1}
\Au{Frenkel S., Khankin D., Kutsyy~A.} Predicting and choosing alternatives 
of route updates per QoS VNF in SDN~// IEEE 16th Symposium (International)
on Network Computing and Applications.~--- IEEE, 2017. P.~1--6.
\bibitem{6-fr-1}
\Au{Devi G., Upadhyaya~S.} An approach to distributed multi-path QoS routing~// 
Indian J.~Sci. Technol., 2015. Vol.~8. Iss.~20. P.~1--14. 
doi: 10.17485/ijst/2015/v8i20/49253.
\bibitem{7-fr-1}
\Au{Egilmez H.\,E., Civanlar S., Tekalp~A.\,M.} 
A~distributed QoS routing architecture for scalable video streaming over multi-domain 
OpenFlow networks~// 19th IEEE Conference (International)
on Image Processing.~--- IEEE, 2012. P.~2237--2240.
\bibitem{8-fr-1}
\Au{Juttner A., Szviatovski B., Mecs~I., Rajko~Z.}
Lagrange relaxation based method for the QoS routing problem~// 
IEEE INFOCOM 2001 Conference on Computer Communications. 20th 
Annual Joint Conference of the IEEE Computer and Communications Society
Proceedings.~--- IEEE, 2001. Vol.~2. P.~859--868.
\bibitem{9-fr-1}
\Au{Yu Z., Ma F., Liu~J., Hu~B., Zhang~Z.}
An efficient approximate algorithm for disjoint QoS routing~// 
Math. Probl. Eng., 2013. Vol.~2013. Art.\ No.\,489149. 9~p. 
doi: 10.1155/2013/489149.
\bibitem{10-fr-1}
\Au{Foerster K.-T., Schmid S., Vissicchio~S.}
A~survey of consistent network updates~// Arxiv.org, 2016. arXiv:1609.02305.
\bibitem{11-fr-1}
\Au{Reitblatt M., Foster N., Rexford J., Walker~D.} 
Consistent updates for software-defined networks: Change you can believe in!~// 
10th ACM Workshop on Hot Topics in Networks Proceedings.~--- New York, NY, USA: ACM, 
2011. Art.\ No.\,7. doi: 10.1145/2070562.2070569.
\bibitem{12-fr-1}
\Au{Hogan M., Esposito F.} Stochastic delay forecasts for edge traffic engineering 
via Bayesian Networks~// IEEE 16th Symposium (International)
on Network Computing and Applications.~--- IEEE, 2017. P.~1--4.
\bibitem{13-fr-1}
\Au{McGeer R.} A~safe, efficient Update Protocol for Openflow Networks~// 
1st Workshop on Hot Topics in Software Defined Networks Proceedings.~--- 
New York, NY, USA: ACM, 2012. Vol.~12. P.~61--66.
\bibitem{14-fr-1}
\Au{McGeer R.} 2013. A~correct, zero-overhead protocol for network updates~// 
2nd Workshop on Hot Topics in Software Defined Networking Proceedings.~--- 
New York, NY, USA: ACM, 2013. Vol.~13. P.~161--162.
\bibitem{15-fr-1}
\Au{Katta N.\,P., Rexford J., Walker~D.} Incremental consistent updates~// 
2nd Workshop on Hot Topics in Software Defined Networking Proceedings.~--- 
New York, NY, USA: ACM, 2013. Vol.~13. P.~49--54.
\bibitem{16-fr-1}
\Au{Dinitz Y., Dolev S., Khankin~D.}
 Dependence graph and master switch for seamless dependent 
 routes replacement in SDN~// IEEE 16th Symposium 
 (International) on Network Computing and Applications.~--- IEEE, 2017. P.~1--7.
 \bibitem{17-aaa-1}
\Au{Amiri~S.\,A., Dudycz~S., Schmid~S., Wiederrecht~S}.
 Congestion-free rerouting of flows
on DAGs~// ArXiv.org, 2016. arXiv:1611.09296.
% [cs, math], Nov. 2016, arXiv: 1611.09296. [Online]. Available:
%http://arxiv.org/abs/1611.09296

\bibitem{17-fr-1}
\Au{Kwiatkowska M., Norman~G., Parker~D.}
 PRISM~4.0: Verification of probabilistic real-time systems~//
 Computer aided verification~/
 Eds. G.~Gopalakrishnan, S.~Qadeer.~---
Lecture notes in computer science ser.~--- Springer, 2011. 
 Vol.~6806. P.~585--591.
\bibitem{18-fr-1}
\Au{Kwiatkowska M., Norman G., Parker~D.}
 PRISM manual, 2018. 
{\sf http://www.prismmodelchecker.org/manual}.
\bibitem{19-fr-1}
Open Networking Foundation. OpenFlow Switch Specification Ver~1.5.1, 2015. 

\bibitem{21-fr-1}
\Au{Wu Q., Hao J.-K.} A~review on algorithms for maximum clique problems~// 
Eur. J.~Oper. Res., 2015. Vol.~242. No.\,3. P.~693--709.

\bibitem{20-fr-1}
\Au{Kaur S., Singh J., Ghumman~N.\,S.}
 Network programmability using POX controller~// Conference
 (International) on Communication, Computing and Systems, 2014. P.~138.
\bibitem{22-fr-1}
\Au{Lantz B., Heller B., McKeown~N.} 
A~network in a~laptop: Rapid prototyping for software-defined networks~// 
9th ACM SIGCOMM Workshop on Hot Topics in Networks Proceedings.~--- 
New York, NY, USA: ACM, 2010. Art.\ No.\,19. doi: 10.1145/1868447.1868466.
\end{thebibliography}
} }

\end{multicols}

 \label{end\stat}

 \vspace*{-9pt}

\hfill{\small\textit{Поступила в~редакцию 09.10.2018}}


%\renewcommand{\bibname}{\protect\rm Литература}
\renewcommand{\figurename}{\protect\bf Рис.}
\renewcommand{\tablename}{\protect\bf Таблица}  %6
\def\stat{koles}

\def\tit{АЛГОРИТМ КООРДИНАЦИИ ДЛЯ~ГИБРИДНОЙ ИНТЕЛЛЕКТУАЛЬНОЙ 
СИСТЕМЫ РЕШЕНИЯ СЛОЖНОЙ ЗАДАЧИ 
ОПЕРАТИВНО-ПРОИЗВОДСТВЕННОГО ПЛАНИРОВАНИЯ}

\def\titkol{Алгоритм координации для~ГиИС решения сложной задачи оперативно-производственного планирования} 



\def\autkol{А.\,В.~Колесников, С.\,А.~Солдатов}
\def\aut{А.\,В.~Колесников$^1$, С.\,А.~Солдатов$^2$}

\titel{\tit}{\aut}{\autkol}{\titkol}

%{\renewcommand{\thefootnote}{\fnsymbol{footnote}}\footnotetext[1]
%{Работа выполнена при финансовой поддержке РФФИ (грант 08-01-00567).}}

\renewcommand{\thefootnote}{\arabic{footnote}}
\footnotetext[1]{Калининградский филиал Института проблем информатики РАН, avkolesnikov@yandex.ru}
\footnotetext[2]{ООО <<Лайтон>>, Москва, ssa@west-automatica.com}


\Abst{Рассмотрена задача оперативно-производственного планирования на 
машиностроительном предприятии с заказным, мелкосерийным характером производства и 
описан подход к решению подобных задач на основе методологии функциональных 
гибридных интеллектуальных систем (ГиИС) с координацией.}

\KW{машиностроительное производство; координация; задача оперативного планирования; 
гибридная интеллектуальная система}

       \vskip 14pt plus 9pt minus 6pt

      \thispagestyle{headings}

      \begin{multicols}{2}

      \label{st\stat}

\section{Введение}

  Задачи оперативно-производственного планирования рассматриваются в 
работах многих отечественных и зарубежных ученых: С.\,Н.~Петракова, 
Г.\,Н.~Кальянова, Е.\,В.~Фрейдиной, В.\,П.~Заболотского, Ю.\,Е.~Звягинцева, 
Н.\,С.~Сачко, Д.~Тейлора и~др. Но, несмотря на имеющееся разнообразие 
научных методов и инженерного инструментария~[1--7], планирование 
производства по-прежнему остается плохо изученным объектом.
  
  Как показал анализ, выработанное в исследовании операций, искусственного 
интеллекта, теории принятия решения и системном анализе представление о 
задаче для индустриального общества устарело. В~известных методах и 
моделях отсутствует либо имеет ограниченную область применения важный 
для решения задачи опе\-ра\-тив\-но-про\-из\-вод\-ст\-вен\-но\-го планирования 
механизм взаимодействия (координации) подзадач в ходе решения сложной 
задачи~[8].
  
  Опыт применения функциональных интеллектуальных гибридных 
систем~[8] в планировании также показал нерелевантность применяемых 
представлений о задаче уровню сложности явлений и процессов в системах 
поддержки принятия решений (СППР).
  
  Настоящая работа развивает предложенное в~[8] двухуровневое 
представление сложных задач и предлагает новый подход к построению 
функциональных ГиИС, основанный на моделировании координации в ходе 
коллективного обсуждения решаемых проблем.
  
\section{Эволюция понятия <<координация>>}
  
  Исследование понятия <<координация>> для количественных, хорошо 
формализуемых задач исследования операций прослеживается в работах 
Дж.~Фар\-ка\-ша и Л.\,В.~Кантаровича~\cite{7kol}, Дж.~Данцига и 
П.~Вулфа~\cite{9kol}, Р.~Беллмана~\cite{10kol}, посвященных 
математическому программированию. При этом декомпозиция сложной задачи 
на составные части сводится к математическому приему поиска для исходной 
матрицы ее <<блочной>> структуры, вычислениям на частях-блоках и 
численным методам интеграции (объединения) частных решений в общее. 
В~работах М.~Месаровича, Д.~Мако и И.~Такахары предложен иной подход к 
координации в сложных формализованных задачах~\cite{11kol}. Сложная 
задача рассматривается как последовательность решаемых подзадач без 
структурно выделенного координирующего элемента. В~этом случае только 
завершение решения всех подзадач проясняет, получено ли решение общей 
задачи. Применение Ф.\,И.~Перегудовым, Ф.\,Л.~Тарасенко, Р.~Акоффом, 
Ф.~Эмери~\cite{12kol, 13kol} системного подхода к анализу решения сложных 
задач дало более совершенные теоретико-множественные представления. 
Сложные задачи обрели состав, структуру и эмерджентность. При этом 
возникли проблемы с количественной мерой сложности и исследованием 
  при\-чин\-но-след\-ст\-вен\-ных связей качественных (количественных) 
параметров задачи с ее интегративными свойствами.

\begin{figure*}[b] %fig1
\vspace*{1pt}
\begin{center}
\mbox{%
\epsfxsize=125.904mm
\epsfbox{kol-1.eps}
}
\end{center}
\vspace*{-6pt}
\Caption{Пример традиционного представления задачи в  системном подходе~(\textit{а}) и 
с учетом координации~(\textit{б}): $\pi_1^h, \ldots , \pi^h_3$~---  
за\-да\-чи-эле\-мен\-ты (подзадачи); 
$\pi^k$~--- задача-координатор; $R^{wq}\vert w$, $q=1,\ldots ,3$; $w\not=q$~--- отношения подзадач;  
$R^{kq}$~--- отношения координатора с подзадачами
  \label{f1kol}}
  \end{figure*}
  
  Фактическое положение вещей на примере\linebreak коллективного решения задачи 
оперативного планирования в СППР показывает обязательное\linebreak наличие 
координирующего элемента~--- лица, принимающего решения (ЛПР), 
диспетчера, в част\-ности выполняющего специфические, малоизученные 
функции, связанные с самоорганизацией в ходе коллективного обсуждения.
  
  В этой связи в рамках системного подхода предлагается новая модель 
сложной задачи: метод\linebreak моделирования решения сложных задач с координацией 
подзадач, а также архитектура ин\-фор\-ма\-ци\-он\-но-вычислительной системы, 
построенной по методологии функциональных гибридных интеллектуальных 
систем~\cite{8kol}.
  
\section{Самоорганизация в системах поддержки принятия решений}
  
  Системы поддержки принятия решений (система <<ЛПР--эксперты>>)~--- это 
выработанный жизнью способ решения сложных практических 
задач~\cite{8kol}. Применительно к планированию на машиностроительных 
предприятиях они названы планерками. Главная особенность таких 
  решений~--- самоорганизация в процессе коллективного обсуждения. Здесь 
многое зависит не только от экспертов и решения частных задач 
  (за\-дач-эле\-мен\-тов), но и от ЛПР, его знаний и опыта работы со сложными 
за\-да\-ча\-ми-сис\-те\-ма\-ми и управления ходом коллективного обсуждения.
  
  Обозначим задачу-систему~$\pi^u$, а за\-да\-чу-эле\-мент~--- $\pi^h$ 
(рис.~\ref{f1kol},\,\textit{а}). Тогда $\Pi^h =\{\pi^h_1, \ldots , \pi^h_N\}$~--- 
множество за\-дач-эле\-мен\-тов, входящих в~$\pi^u$;\linebreak 
$\dot{\Pi}^u=\{\dot{\pi}_1^u,\ldots , \dot{\pi}^u_M\}$~--- множество 
декомпозиций задачи~$\pi^u$; $R^{wq}\vert w$, $q=1,\ldots ,N$; $w\not=q$~--- 
отношения между задачами-элементами; $N$~--- здесь и далее мощность 
множества. Тогда модель задачи-сис\-те\-мы представим в следующем виде:
  \begin{equation}
\pi^u =\langle \Pi^h, \dot{\Pi}^u, R^{wq}\rangle\,.
\label{e1kol}
\end{equation}
  
  При решении задачи-системы за\-да\-чи-эле\-мен\-ты преимущественно 
отделены от внешней среды или ее состояние зафиксировано, т.\,е.\ 
выполняется требование о том, что связи внутри системы намного сильнее, чем 
с внешней средой.
  

  
  Модель~(\ref{e1kol}) имеет недостатки, и основной из них~--- нерелевантное 
отображение связей между элементами. Учитывать только связи~$R^{wq}$ 
недостаточно, а простое суммирование решений за\-дач-эле\-мен\-тов не дает 
решения за\-да\-чи-сис\-те\-мы. Исследования СППР показали, что в 
большинстве случаев эксперты не могут дать профессиональных решений в 
условиях, заданных им ЛПР изначально. А для изменения первоначальных 
условий в модели~(1) необходим существенный элемент~--- образ ЛПР, 
который выполнял бы функцию координатора, как <<перераспределителя>> 
ресурсов и переформулировал бы в зависимости от ситуации цели экспертов. 
Это позволило бы отобразить ситуацию, когда реальная СППР 
<<приспосабливается>> к различного рода обстоятельствам во внешней среде. 
Согласно синергетической парадигме это должно  происходить путем 
самоорганизации СППР.
  
  В~связи с вышесказанным в предлагаемом подходе задача рассматривается 
не только как отображение последовательности решений подзадач, но и как 
система с новым элементом~--- координатором~$\pi^k$. Его 
функция~--- мониторинг и управление процессом решения подзадач 
$\pi_1^h,\ldots , \pi_N^h$ экспертами в ходе коллективного обсуждения. 
Координатор связан отношениями $R^{kq}\vert q=1,\ldots ,N$ с каж\-дой задачей 
$\pi^h$ в системе~$\pi^u$, посредством которых собирает информацию о 
состоянии процесса решения экспертами за\-дач-эле\-мен\-тов и в конкретные 
моменты времени (планерки) выдает координирующие воздействия для 
изменения входного набора данных (ресурсов, целей). Тогда модель сложной 
задачи c координацией представим в следующем виде:
  \begin{equation}
  \pi^{uk} =\langle \Pi^h, \dot{\Pi}^u, \pi^k, R^{wq}, R^{kq}\rangle\,,
  \label{e2kol}
  \end{equation}
где $\pi^k$~--- координатор; $R^{kq}\vert q=1, \ldots , N$~--- отношения между 
координатором и за\-да\-ча\-ми-эле\-мен\-тами.
  
  Таким образом, можно дать следующее определение сложной практической 
задачи~--- это задача, включающая взаимодействующие 
  эле\-мен\-ты-под\-за\-да\-чи, между которыми происходит обмен данными 
(значениями переменных, синхроимпульсами и~т.\,п.), управляемый 
специальным элементом~--- координатором. Его присутствие отображает 
самоорганизацию СППР при решении за\-дач-сис\-тем.
  
  Сравнение (1) и~(2) показывает, что~(2) носит более общий характер и легко 
сводится к~(1). По сути, элемент-координатор может быть представлен 
<<координирующей задачей>> ($k$-за\-да\-чей), которая должна быть 
<<добавлена>> в декомпозицию $\dot{\pi}^u\in \dot{\Pi}^u$ сложной 
задачи~$\pi^u$, чтобы релевантно отображать в модели особенности 
коллективного решения задач планирования. Отметим, что 
  за\-да\-ча-ко\-ор\-ди\-на\-тор не дополняет имеющиеся подзадачи, а 
существует как подзадача сложной задачи, решение которой традиционно 
возлагалось на ЛПР.
  
  Можно также отметить, что с увеличением количества за\-дач-эле\-мен\-тов, 
актуальность координации их решения возрастает, так как комбинаторно растет 
число отношений между за\-да\-ча\-ми-эле\-мен\-тами.
  
\section{Алгоритм координации в~системах поддержки принятия~решений}
  
  Предлагаемый алгоритм отображает самоорганизацию в СППР по принципу 
<<как есть>> (лат.\ \textit{ad hoc}) и относится к алгоритмам, основанным на знаниях, 
где акцент смещается с использования формализованной математической 
схемы на извлечение профессиональных знаний и рассуждениям с их 
применением. Для разработки алгоритма такие знания были извлечены на 
примере СППР машиностроительного предприятия с мелкосерийным 
характером производства. Все знания пред\-став\-ле\-ны продукционными 
системами. База знаний ЛПР (начальника производственного центра) о 
  $k$-за\-да\-че содержит 24~правила. Базы знаний пяти экспертов 
(начальников отделов) содержат от~10 до 40~правил. В качестве оболочки ЭС 
выбрана программа Visual Rule Studio~\cite{14kol} и метод рассуждений в 
прямом направлении~\cite{8kol}.
  
  Приведенный ниже алгоритм имитирует последовательность заседаний 
СППР в относительном, модельном времени. При этом линии рассуждений 
экспертов координируются ЛПР. Полученные экспертами после каждой 
итерации (планерки) решения частных подзадач передаются ЛПР как исходные 
данные для задачи координации ($k$-за\-да\-чи). Лицо, принимающее решения, использует в процессе 
решения задачи координации данные, полученные после декомпозиции 
исходной сложной задачи. Решив задачу координации, ЛПР дает рекомендации 
каждому из экспертов. Эти рекомендации в совокупности с первоначально 
известными после декомпозиции  данными, служат исходными данными для 
решаемых экспертами подзадач на следующей итерации (планерке). В~случаях, 
когда не требуется вносить изменения в ход решения подзадачи экспертом, 
ЛПР выдает <<пустую команду>>.

\smallskip
  
\noindent
  \textbf{Дано:} СППР, состоящая из $N$ экспертов и ЛПР. Число планерок в 
плановом периоде~--- $k$. Установлено однозначное соответствие 
$$\psi_1:\ \ 
\mathrm{Out}\vert^{\pi_1^h}\cup\ldots \cup \mathrm{Out}\vert^{\pi_N^h}\rightarrow \mathrm{In}\vert^{\pi^k}
$$ 
между выходными параметрами $\mathrm{Out}\vert^{\pi_i^h}=$\linebreak $=\{\mathrm{Out}_1, \ldots , 
\mathrm{Out}_m\}\vert^{\pi_i^h}$ подзадач~$\pi_i^h$, $i=1, \ldots , N$, и входными 
параметрами $\mathrm{In}\vert^{\pi^k}=\{\mathrm{In}_1, \ldots , \mathrm{In}_n\}\vert^{\pi^k}$ задачи 
координации~$\pi^k$. Установлено однозначное соответствие
$$
\psi_2:\ \ 
\mathrm{Out}\vert^{\pi^k}\rightarrow \mathrm{In}\vert^{\pi_1^h}\cup\ldots\cup \mathrm{In}\vert^{\pi_N^h}
$$ 
между выходными параметрами $\mathrm{Out}\vert^{\pi^k}=$\linebreak $=\{\mathrm{Out}_1, \ldots 
,\mathrm{Out}_n\}\vert^{\pi^k}$ задачи координации~$\pi^k$ и входными параметрами 
$\mathrm{In}\vert^{\pi_i^h} =\{\mathrm{In}_1, \ldots ,\mathrm{In}_n\}\vert^{\pi_i^h}$ подзадач~$\pi_i^h$, $i=1, 
\ldots , N$. Соответствия~$\psi_1$ и~$\psi_2$ сюръективны и иньюктивны.

\smallskip
  
\noindent
  \textbf{Найти:} результаты решения (выходные пара\-мет\-ры) 
подзадач~$\pi_i^h$, $i=1, \ldots ,N$, экспертами с учетом координации их 
решения ЛПР.

\smallskip
  
\noindent  
\textbf{Обозначения:} БЗ$_i$, $i = 1, \ldots , N$,~--- базы знаний экспертов; 
БЗ$_{\mathrm{лпр}}$~--- база знаний ЛПР; БФ$_i$, $i = 1, \ldots , N$,~--- базы 
фактов экспертов; БФ$_{\mathrm{лпр}}$~--- база фактов ЛПР;  $j$~--- счетчик 
планерок; $i$~--- счетчик экспертов; $\mathrm{Run}(\mathrm{БФ,БЗ})$~--- процедура имитации\linebreak 
рассуждений ЛПР или экспертов, где БФ и БЗ~---\linebreak базы фактов и знаний 
соответственно; $\mathrm{Search}(\mathrm{БЗ})$~--- про\-це\-ду\-ра-про\-смотр правил из БЗ и 
сопоставление фактов с образцами с целью определения множества правил, 
которые могут быть активированы; Conf~--- процедура разрешения 
конфликтов в <<плане решения>> (Agenda) экспертной системы, когда 
возникает необходимость выбора между несколькими правилами из БЗ для 
продолжения рассуждений (метод разрешения конфликтов~--- <<первый в 
Agenda>>); Execute~--- процедура применения найден\-ных правила для 
продолжения рассуждений, т.\,е.\ выполнения действий указанных в правой 
части выбранных правил.

\smallskip
  
\noindent
  \textbf{Алгоритм}
  
  \noindent
  \begin{enumerate}[1)]
\item $j = 0$;
\item $i = 1$;
\item Заполнить БФ$_i$ с учетом $(\mathrm{In}_1, \ldots , \mathrm{In}_n)\vert^{\pi_i^h}$ и счетчика 
планерок~$j$;
\item Run(БФ$_i$,  БЗ$_i$):
\begin{itemize}
\item[1.1)] $\mathrm{Search}(\mathrm{БЗ}_i)$;
\item[1.2)] Conf;
\item[1.3)] Execute;
\item[1.4)] Если для продолжения рассуждений не выбрано ни одно из 
правил, то закончить имитацию рассуждений и перейти к п.~5. Иначе~---  
к~п.~4.1;
\end{itemize}
\item Присвоить $(\mathrm{Out}_1, \ldots , \mathrm{Out}_m)\vert^{\pi_i^h}$ значения из фактов~--- 
результатов рассуждений $i$-го эксперта;
\item Присвоить, используя $\psi_1$, соответствующим входным параметрам 
$(\mathrm{In}_1, \ldots , \mathrm{In}_n)\vert^{\pi^k}$ задачи координации~$\pi^k$ значения 
выходных па\-ра\-мет\-ров $(\mathrm{Out}_1, \ldots , \mathrm{Out}_m)\vert^{\pi_i^h}$ задачи~$\pi_i^h$;
\item Если $i < N + 1$, то $i = i + 1$ и перейти к~п.~3. Иначе перейти к~п.~8;
\item Заполнить БФ$_{\mathrm{лпр}}$ с учетом значений входных параметров 
$(\mathrm{In}_1, \ldots ,\mathrm{In}_n)\vert^{\pi^k}$ задачи координации~$\pi^k$ и счетчика 
планерок~$j$;
\item $\mathrm{Run}(\mathrm{БФ}_{\mathrm{лпр}}, \mathrm{БЗ}_{\mathrm{лпр}})$:
\begin{itemize}
\item[9.1)] $\mathrm{Search}(\mathrm{БЗ}_{\mathrm{лпр}})$;
\item[9.2)] Conf;
\item[9.3)] Execute;
\item[9.4)] Если для продолжения рассуждений не выбрано ни одно из 
правил, то закончить имитацию рассуждений и перейти к~п.~10. Иначе~--- 
к~п.~9.1;
\end{itemize}
\item Присвоить $(\mathrm{Out}_1, \ldots , \mathrm{Out}_n)\vert^{\pi^k}$ задачи 
координации~$\pi^k$ значения из фактов-результатов рассуждений ЛПР;
\item Присвоить, используя~$\psi_2$, соответствующим входным параметрам 
$(\mathrm{In}_1,  \ldots , \mathrm{In}_n)\vert^{\pi^k}$ подзадач $\pi_1, \ldots ,\pi_N$ значения 
выходных параметров задачи координации~$\pi^k$;
\item Если $j < k$, то $j = j + 1$ и перейти к~п.~2. Иначе перейти к~п.~13;
\item Вывести значения выходных параметров подзадач, полученные на 
$j$-й ~планерке:
\begin{itemize}
\item[13.1)] $i = 1$;
\item[13.2)] Печать $(\mathrm{Out}_1, \ldots ,\mathrm{Out}_m)\vert^{\pi_i^h}$;
\item[13.3)] Если $i < N + 1$, то $i = i + 1$ и перейти к~п.~13.2. Иначе 
перейти к~п.~14;
\end{itemize}
\item Конец.
\end{enumerate}
  
  Разработанный алгоритм дискретен и пред\-ставля\-ет координацию как 
последовательность прос\-тых шагов. Ему присуща определенность~---\linebreak каждое 
правило алгоритма детерминированное, четкое и не оставляет места для 
многозначности. Алгоритм приводит к решению сложной задачи за конечное 
число шагов (результативность), так как внешний цикл (имитация планерок) 
выполняется не более $k - 1$~раз. Вложенные циклы (имитация рассуждений 
каждого эксперта) и цикл вывода фактов-результатов выполняются не более 
$N$~раз. Отсутствие циклов в графе И/ИЛИ рассуждений экспертов и ЛПР 
гарантирует отсутствие зацикливания. Алгоритм координации разработан в 
общем виде (массовость), однако он не лишен недостатка <<хрупкости>> 
любой экспертной системы.
  
  Порядок опроса экспертов не важен, так как пока последний эксперт не 
сообщит результат решения своей подзадачи, ЛПР не может приступить к 
решению задачи координации.  В~случае, если в результате имитации 
рассуждений экспертов или ЛПР не могут быть выведены факты, необходимые 
для спецификации выходных параметров решаемых подзадач, то используются 
значения по умолчанию, извлеченные у экспертов и ЛПР при  разработке 
конкретной ин\-фор\-ма\-ци\-он\-но-вы\-чис\-ли\-тель\-ной сис\-те\-мы. 
Существенное отличие базы фактов ЛПР от баз фактов экспертов в том, что в 
ней\linebreak происходит интеграция результатов решения подзадач экспертами, 
основанная на при\-чин\-но-след\-ст\-вен\-ных связях, 
  час\-тич\-но-фор\-ма\-ли\-зо\-ван\-ных продукционными правилами и 
представленными\linebreak базой профессиональных знаний ЛПР, о качестве которых 
можно судить исключительно по результатам проведенной апробации. Для 
таких экспериментов была разработан программный продукт <<Гибридная 
система планирования>>.
  
\section{Модель гибридной интеллектуальной системы с~учетом 
координации}
  
  В качестве модели ГиИС как абстрактного автомата для решения сложной 
задачи опе\-ра\-тив\-но-про\-из\-вод\-ст\-вен\-но\-го планирования принята 
функциональная крупнозернистая ГиИС~$\alpha^{u}(t)$~\cite{8kol}. Ее  
расширение выполнено исходя из следующих посылок. В~процессе 
координации контролируются промежуточные состояния процесса решения 
подзадач. Под этими состояниями понимаются состояния функциональных 
элементов~$\alpha^h$, имитирующих решение подзадач~$\pi^h$, а также 
состояния технологических элементов~$\alpha^\tau$. На основании анализа 
этих состояний в ходе координации изменяются свойства 
<<вход>>~$\hat{x}_1^2$ одного или нескольких функциональных и 
технологических элементов~$\alpha^h$ и~$\alpha^\tau$. Для учета этого факта 
введем в модель крупнозернистой функциональной ГиИС $\alpha^{u}(t)$ 
следующую тройку: $\hat{x}_3^2(t) R^{22} \hat{x}_1^2(t+1)$. Иными словами, 
на основании состояния ГиИС~$\hat{x}_3^2(t)$ в момент времени~$t$ 
меняются исходные данные $\hat{x}_1^2(t+1)$ для ГиИС в момент времени 
$t+1$, т.\,е.\ для следующей итерации. Множество~$R^{22}$ устанавливает 
отношения между состоянием $\hat{x}_3^2(t)$ гибрида~$\alpha^u$ на данный 
момент модельного времени~$t$ и состоянием входов одного или нескольких 
функциональных и технологических элементов~$\alpha^h$ и~$\alpha^\tau$ на 
следующем шаге. Чтобы произвести необходимое изменение 
входов~$\hat{x}_1^2$ одного или нескольких функциональных и 
технологических элементов~$\alpha^h$ и~$\alpha^\tau$ в~$\alpha^u(t)$ введем 
тройку $\hat{x}_3^2(t) R^{23} X_1^3$, где $X^3=\{x_1^3, \ldots , x_6^3\}$~---\linebreak 
множество понятий, обозначающих ко\-ор\-ди\-ни\-ру\-ющие действия (интегральная 
координация, четкая координация, интервальная координация, лингвистическая 
координация, координация по времени, <<пустое действие>>), которое 
тождественно множеству координирующих действий, введенных 
в~\cite{15kol}. В~алгоритме координации эти действия (рекомендации 
экспертам) описаны в базе знаний ЛПР. Множество~$R^{23}$~--- это 
отношения между состоянием~$\hat{x}_3^2$ гибрида~$\alpha^u$ в момент 
модельного времени~$t$ и необходимыми координирующими 
действиями~$X^3$. Ниже приведена модифицированная схема ролевых 
концептуальных моделей~$\alpha^{uk}$ для спецификации крупнозернистой 
функциональной ГиИС с координацией
\begin{equation}
\alpha^{uk}(t)= \alpha^u(t)\circ \hat{x}_3^2(t) R^{22} \hat{x}_1^2(t+1)\circ \hat{x}_3^2(t) R^{23} X^3\,,
\label{e3kol}
\end{equation}
где $\circ$~--- знак конкатенации; 1, 2, 3 в качестве правого верхнего 
индекса~$X$ или~$x$~--- признак ресурса, свойства, действия соответственно; 
нижний правый индекс для $x$~--- порядковый номер класса понятий; верхний 
правый индекс для~$R$ обозначает между какими понятиями категориального 
ядра~\cite{8kol} установлены отношения (22~--- <<свой\-ст\-во--свой\-ст\-во>>, 
23~--- <<свой\-ст\-во--дей\-ст\-вие>>).
  
  Отношения $R^{22}$ и~$R^{23}$ не задаются заранее, а фиксируются в ходе 
функционирования ГиИС и поиска результата решения 
  за\-да\-чи-ко\-ор\-ди\-на\-то\-ра ($k$-за\-да\-чи). Поскольку в соответствии 
с~\cite{8kol} технологические элементы управляют порядком работы 
функциональных элементов и обменом информации между ними, 
целесообразно возложить решение задачи координации на технологический 
элемент.
  
  Рассмотрим пример ГиИС, состоящей из трех функциональных элементов 
$\alpha^h\vert_1^1$, $\alpha^h\vert_1^7$, $\alpha^h\vert_1^6$ и одного 
технологического элемента~$\alpha^\tau\vert_1^7$, где $w=1, \ldots , 7$ для 
$\alpha^h\vert_j^w$ и $\alpha^\tau\vert_j^w$ обозначает базовые классы методов 
функциональных ГиИС~\cite{8kol}, $j=1$~--- порядковый номер задачи в 
классе. На вход ГиИС подаются исходные данные, разделенные между 
функциональными элементами в соответствии с декомпозицией сложной 
задачи. На выходе имеем результаты работы функциональных элементов, 
агрегированные в общее решение задачи.

\begin{figure*} %fig2
\vspace*{1pt}
\begin{center}
\mbox{%
\epsfxsize=165.684mm
\epsfbox{kol-2.eps}
}
\end{center}
\vspace*{-3pt}
\Caption{Структурная схема ГиИС без координации~(\textit{а}) и с координацией~(\textit{б}):
\textit{1}~--- прямая и обратная координи\-ру\-ющие связи между технологическим элементом и
функциональными элементами;
\textit{2}~--- отношения порядка работы и обмена информацией между элементами;
\textit{3}~--- редукция сложной задачи (на входе) и интеграция результатов работы функциональных элементов;
\textit{4}~--- прямая и обратная связи между ГиИС и ЛПР при отсутствии координации внутри ГиИС;
\textit{5}~--- прямая и обратная связи между ГиИС и ЛПР при наличии координации внутри ГиИС
\label{f2kol}}
\vspace*{9pt}
\end{figure*}
  
  На рис.~\ref{f2kol},\,\textit{а} изображена структурная схема \mbox{ГиИС}, 
построенная для решения сложной задачи в соответствии с~$\alpha^u(t)$. Здесь 
моделируется только логически увязанная последовательность решения 
подзадач~$\pi^h$ из декомпозиции~$\dot{\pi}^u$ сложной задачи~$\pi^u$. Это 
соответствует модели сложной задачи на рис.~\ref{f1kol},\,\textit{а}. В~этом 
случае по каналу обратной связи ЛПР получает от компьютерной СППР 
решение сложной задачи. Если общее единое решение дает с точки зрения ЛПР 
ошибочный результат, то он на основании своих оценок по каналу прямой 
связи вносит изменения во множества входных данных и условий 
задачи~$\pi^u$. Далее ЛПР инициирует новый синтез ГиИС и повторное 
решение.
  
  На рис.~\ref{f2kol},\,\textit{б} изображена принципиально иная структурная 
схема ГиИС. Ее отличие от вышеприведенной в том, что технологический 
элемент~$\alpha^\tau\vert_1^7$ определяет не только порядок работы 
функциональных элементов и обмен информации между ними, но и в 
соответствии с~(\ref{e3kol}) по состоянию всех функциональных элементов 
итерационно корректирует для каждого из них входной набор данных и 
условий. Таким образом, часть функций ЛПР передается технологическому 
элементу, что отражено на рис.~\ref{f2kol},\,\textit{б} изменением размеров 
структурных блоков ЛПР и технологического элемента, а также толщины 
линий прямой и обратной связи в контуре управления.
  
  В представленной на рис.~\ref{f2kol},\,\textit{б} структурной схеме ЛПР по 
каналу обратной связи получает от компьютерной СППР результат решения 
сложной задачи. Если решение не устраивает ЛПР, например оно ведет к 
увеличению стоимости производимых изделий и~т.\,д., то он вмешивается в 
обсуждение и меняет условия координации, т.\,е.\ модель задачи-координатора. 
Далее ЛПР инициирует новый синтез ГиИС и повторное решение сложной 
задачи. В~итоге это позволяет моделировать процесс самоорганизации, о 
котором было сказано выше.
  
  В модели~(\ref{e3kol}) развито предположение, что включение в 
компьютерную модель СППР модели ЛПР приводит к возникновению 
синергетических эффектов~--- самоорганизации. При этом появляется 
возможность увязать результаты работы отдельных функциональных элементов 
СППР еще в процессе синтеза решения сложной задачи, а не после, как в 
известных моделях. Тем самым достигается большая релевантность 
компьютерной \mbox{СППР} реальному процессу коллективного обсуж\-де\-ния \mbox{проблем}.
  
\section{Заключение}
  
  Гибридные интеллектуальные системы с координацией элементов~--- новый 
шаг в синергетическом искусственном интеллекте, позволяющий полнее 
раскрыть и исследовать многообразие отношений в системе 
  <<ЛПР--экс\-пер\-ты>>. Это также результат и в моделировании категории 
<<время>> в гибридных системах на примере производственных планерок 
динамичных процессов опе\-ра\-тив\-но-про\-из\-вод\-ст\-вен\-но\-го 
планирования.
  
  По сравнению с известными алгоритмами 
  координации~[1, 8--10, 16, 17], предлагаемый 
алгоритм имеет следующие достоинства: итеративность позволяет имитировать 
обмен информацией в процессе решения сложной задачи, а применение 
профессиональных баз знаний делает модель релевантной сложной 
практической задаче.
  
  Апробация ГиИС, решающих $k$-задачу, на реальных данных показала 
положительный эффект от увеличения релевантности моделирования сложной 
задачи опе\-ра\-тив\-но-про\-из\-вод\-ст\-вен\-но\-го пла\-нирования, что 
улучшило качественные и ко\-личественные показатели машиностроительного\linebreak 
предприятия с мелкосерийным заказным характером производства. 
В~частности, ожидаемый экономический эффект от внедрения ГиИС на одном из 
предприятий в процентном отношении к ожидаемой прибыли предприятия за 
год составил~11\%.

{\small\frenchspacing
{%\baselineskip=10.8pt
\addcontentsline{toc}{section}{Литература}
\begin{thebibliography}{99}

\bibitem{7kol} %1
\Au{Канторович Л.\,В.}
Математические методы организации и планирования производства.~--- Л.: ЛГУ, 1959.



\bibitem{6kol} %2
\Au{Татевосов К.\,Г.}
Основы оперативно-про\-из\-вод\-ст\-вен\-но\-го планирования на машиностроительном 
предприятии.~--- Л.: Машиностроение, 1985.

\bibitem{5kol} %3
\Au{Заболотский В.\,П., Оводенко А.\,А., Степанов~А.\,Г.}
Математические модели в управлении: Учеб. пособие.~--- СПб.: \mbox{СПбГУАП}, 2001.


\bibitem{1kol} %4
\Au{Туровец О.\,Г., Родионов В.\,Б., Бухалков М.\,И.}
Организация производства и управление предприятием.~--- М.: ИНФРА-М, 2005.

\bibitem{2kol} %5
\Au{Кальянов Г.\,Н.}
Моделирование, анализ, реорганизация и автоматизация биз\-нес-про\-цес\-сов.~--- М.: 
Финансы и статистика, 2007.

\bibitem{4kol} %6
\Au{Сачко Н.\,С.}
Организация и оперативное управление машиностроительным производством.~--- Минск: 
Новое Знание, 2008.

\bibitem{3kol} %7
\Au{Тейлор Д., Рэйден Н.}
Почти интеллектуальные сис\-те\-мы. Как получить конкурентные преимущества путем 
автоматизации принятия скрытых решений.~--- СПб.: Сим\-вол-Плюс, 2009.


\bibitem{8kol}
\Au{Колесников А.\,В., Кириков И.\,А.}
Методология и технология решения сложных задач методами функциональных гибридных 
интеллектуальных систем.~--- М.: ИПИ РАН, 2007.

\bibitem{9kol}
\Au{Данциг Дж.}
Линейное программирование, его обобщения и применения.~--- М.: Прогресс, 1966.

\bibitem{10kol}
\Au{Беллман Р., Дрейфус~С.}
Прикладные задачи динамического программирования.~--- М.: Наука, 1965.

\bibitem{11kol}
\Au{Месарович М., Мако Д., Такахара И.}
Теория иерархических многоуровневых систем.~--- М.: Мир, 1973.


\bibitem{13kol} %12
\Au{Акофф Р., Эмери Ф.}
О целеустремленных системах.~--- М.: Советское радио, 1974.

\bibitem{12kol} %13
\Au{Перегудов Ф.\,И., Тарасенко Ф.\,Л.}
Введение в системный анализ.~--- М.: Высшая школа, 1989.

\bibitem{14kol}
Сайт компании Rule Machines Corporation. {\sf http://\linebreak www.rulemachines.com}.

\bibitem{15kol}
\Au{Колесников А.\,В., Солдатов С.\,А.}
Теоретические основы решения сложной задачи оперативно-про\-из\-вод\-ст\-вен\-но\-го 
планирования с учетом координации~// Вестник Российского государственного ун-та им.\ 
И.~Канта. Вып.~10. Сер. Физико-математические науки.~--- Калининград: РГУ им. 
И.~Канта, 2009. С.~82--98.

\bibitem{17kol} %16
\Au{Beat F., Schmid K., Stanoevska~S., Lei~Yu.}
Supporting distributed corporate planning through new coordination technologies, 1998. {\sf 
http://\linebreak www.alexandria.unisg.ch/Publikationen/9453}.



 \label{end\stat}
 
 \bibitem{16kol} %17
\Au{Geun-Sik Jo, Kang-Hee L., Hwi-Yoon L., Sang-Ho~H.}
Ramp activity expert system for scheduling and co-ordination at an airport~// Innovative 
Application of Artificial Intelligence '99, AAAI, July, 1999. P.~807--812. {\sf 
http://www.aaai.org/Papers/IAAI/1999/IAAI99-114.pdf}.


 \end{thebibliography}
}
}


\end{multicols}%7
\def\stat{mat}

\def\tit{СТАЦИОНАРНЫЕ ХАРАКТЕРИСТИКИ ДВУХКАНАЛЬНОЙ СИСТЕМЫ ОБСЛУЖИВАНИЯ С~ПЕРЕУПОРЯДОЧИВАНИЕМ ЗАЯВОК 
И~РАСПРЕДЕЛЕНИЯМИ ФАЗОВОГО ТИПА}

\def\titkol{Стационарные характеристики двухканальной системы обслуживания с~переупорядочиванием заявок} 
%и~распределениями фазового типа}

\def\autkol{С.\,И.~Матюшенко}
\def\aut{С.\,И.~Матюшенко$^1$}

\titel{\tit}{\aut}{\autkol}{\titkol}

%{\renewcommand{\thefootnote}{\fnsymbol{footnote}}\footnotetext[1]
%{Исследование поддержано грантами РФФИ 08-07-00152 и 09-07-12032.
%Статья написана на основе материалов доклада, представленного на IV 
%Международном семинаре  <<Прикладные задачи теории вероятностей и математической статистики, 
%связанные с моделированием информационных систем>> (зимняя сессия, Аоста, Италия, январь--февраль 2010~г.).}}

\renewcommand{\thefootnote}{\arabic{footnote}}
\footnotetext[1]{Российский университет дружбы народов,  кафедра теории вероятностей и математической статистики, matushenko@list.ru}

%\vspace*{6pt}

\Abst{Рассматривается двухканальная система обслуживания
ограниченной емкости с распределениями фазового типа и
переупорядочиванием заявок. Получено выражение для преобразования
Лап\-ла\-са--Стилтьеса функции распределения задержки переупорядочивания
в рассматриваемой системе в стационарном режиме работы. Разработан
алгоритм для расчета факториальных моментов числа заявок в буфере
переупорядочивания.}

%\vspace*{2pt}


\KW{система  массового обслуживания; распределение
фазового типа; переупорядочивание заявок}

%\vspace*{6pt}

       \vskip 14pt plus 9pt minus 6pt

      \thispagestyle{headings}

      \begin{multicols}{2}

      \label{st\stat}

\section{Постановка задачи}

Рассмотрим двухканальную систему массового обслуживания (СМО) с общим накопителем ограниченной емкости~$r$, 
$r<\infty$, на которую поступает рекуррентный поток заявок с функцией распределения (ФР) фазового типа~$A(x)$,
\begin{equation*}
A(x)=1-\boldsymbol{\alpha}^T e^{\boldsymbol{\Lambda}x}\mathbf{1}\,, \quad
x\geq 0\,,\enskip \boldsymbol{\alpha}^T\mathbf{1}=1\,,
\end{equation*}
с неприводимым PH-представлением $(\boldsymbol{\alpha},\boldsymbol{\Lambda})$ порядка~$l$~\cite{1mat}.

Будем предполагать, что времена обслуживания на приборе~$j$ независимы  между собой и имеют общую 
ФР фазового типа
\begin{equation*}
B_j(x)=1-\boldsymbol{\beta}_j^T e^{\boldsymbol{M}_j x}\mathbf{1}\,, \quad
x\geq 0\,,\enskip \boldsymbol{\beta}_j^T\boldsymbol{1}=1
\end{equation*}
с неприводимым PH-представлением $(\boldsymbol{\beta}_j,\boldsymbol{M}_j)$ порядка $m_j$, $j=1,2$.

Далее, без ограничения общности примем, что интенсивность обслуживания на приборе~1 выше, чем на приборе~2. 
Заявка, поступающая в свободную СМО, направляется на первый прибор. При наличии очереди действует дисциплина FCFS.

Предположим, что всем заявкам при поступлении в систему присваивается порядковый номер. На выходе из СМО 
будем требовать сохранения порядка между заявками, установленного при входе в нее. Заявки, прошедшие 
обслуживание и нарушившие установленный порядок, будут накапливаться на выходе системы в  буфере 
переупорядочивания (БП).

В соответствии с обозначениями Кендалла рассматриваемую систему будем кодировать как $PH/PH/2/r/res$, 
где $res$~--- сокращение от английского $resequence$~--- переупорядочивание. Такая сис\-те\-ма уже 
рассматривалась автором в~\cite{2mat}. В~той работе функционирование системы описывалось однородным 
марковским процессом (ОМП) над пространством состояний без учета содержимого БП. В~итоге был разработан 
рекуррентный матричный алгоритм для расчета стационарных вероятностей состояний указанного процесса. 
Основная задача данной работы состоит в том, чтобы, опираясь на результаты~\cite{2mat}, 
получить показатели, характеризующие стационарное состояние БП. Перейдем к решению этой задачи.

\section{Построение математической модели}

Построим марковский процесс, описывающий функционирование рассматриваемой системы. 
Для этого введем понятие упорядоченности.  Будем считать, что система находится в 
упорядоченном состоянии (упорядочена), если на приборах~1 и~2 обслуживаются заявки с номерами~$N_1$ 
и~$N_2$, $N_1 < N_2$, в противном случае, т.\,е.\ при $N_1>N_2$, система не упорядочена. 
Система также упорядочена (не упорядочена) при наличии в ней одной заявки на приборе~1 (приборе~2).

Теперь рассматриваемую СМО  с учетом введенного понятия упорядоченности, а также 
с учетом вероятностной интерпретации PH-распределения (\cite{1mat}, с.~104) можно описать ОМП~$Y(t)$, 
$t\geq 0$, над пространством состояний

\noindent
\begin{align*}
\mathcal{Y}&=\bigcup\limits_{k=0}^{r+2}\mathcal{Y}_k\,,\enskip
\mathcal{Y}_k=\mathcal{Y}_{k1}\bigcup\mathcal{Y}_{k2}\,,\\
\mathcal{Y}_{ki}&=\bigcup_{n=0}^{\infty}\mathcal{Y}_{kin}\,,
k=\overline{1,r+2},\, i=1,2\,,
\end{align*}
где
\begin{align*}
\mathcal{Y}_0&=\{(s,0), s=\overline{1,l}\}\,;\\
\mathcal{Y}_{1in}&=\{(s,j_i,i,n), s=\overline{1,l}, j_i=\overline{1,m_i} \}\,, \\
&\hspace*{30mm}i=1,2\,, \enskip n\geq 0\,;\\
\mathcal{Y}_{kin}&=\{(k,s,j_1,j_2,i,n), s=\overline{1,l}, j_1=\overline{1,m_1},\\
&\hspace*{8mm} j_2=\overline{1,m_2}\}\,,\enskip
 k=\overline{2,r+2},\enskip i=1,2,\, n\geq 0\,.
\end{align*}
Здесь для некоторого момента времени~$t$: $Y(t)=$\linebreak $=(s,0)$, если в момент времени~$t$ система пус\-та, 
а процесс генерации заявки проходит фазу~$s$;  $Y(t)=(s,j_i,i,n)$, если в системе имеется одна 
заявка, обслуживаемая на первом приборе при $i=1$ либо на втором приборе при $i=2$, процесс 
обслуживания находится на фазе~$j_i$, а в БП содержится $n$~заявок; $Y(t)=(k,s,j_1,j_2,i,n)$, 
если в очереди и на приборах имеется $k$~заявок, процессы обслуживания заявок на приборах находятся 
на фазах~$j_1$ и~$j_2$ соответственно, причем система упорядочена, если $i=1$, либо не упорядочена, 
если $i=2$, а индексы~$s$ и~$n$ имеют прежний смысл.

В предположении, что интенсивности потока и обслуживания конечны, процесс~$Y(t)$ эргодичен и, 
следовательно, существуют вероятности
\begin{equation*}
p_y=\lim\limits_{t\rightarrow\infty}P\{Y(t)=y\}, y\in \mathcal{Y}\,,
\end{equation*}
совпадающие со стационарными.

Введем векторы
\begin{multline*}
\boldsymbol{p}_{1in}^{\mathrm{T}}=
(p_{11in},\ldots,p_{1m_iin},\ldots , p_{2m_iin},\ldots\\
\ldots , p_{l1in},\ldots, p_{lm_iin})\,,\quad  i=1,2,\enskip n\geq 0\,;
\end{multline*}

\vspace*{-14pt}

\noindent
\begin{multline*}
\boldsymbol{p}_{kin}^{\mathrm{T}}=(p_{ki11in},\ldots,p_{k11m_2in},\ldots,p_{k1m_1m_2in},
\ldots\\
\ldots,
p_{klm_1m_2in}),\quad k=\overline{2,r+2},\enskip i=1,2, \enskip n\geq 0\,.
\end{multline*}
Будем использовать обозначение $\boldsymbol{p}_{ki,\cdot}=\sum\limits_{n\geq 0} \boldsymbol{p}_{kin}$ и введем матрицы:
\begin{align*}
\boldsymbol{D}_{1i}&=\boldsymbol{\Lambda}_1\oplus \boldsymbol{M}_i\,,\quad i=1,2;\\
\boldsymbol{D}_{ki}&=\boldsymbol{\Lambda}_k\oplus \boldsymbol{M}_i\,,\quad k=\overline{2,r+1}\,,\enskip i=1,2\,;\\
\boldsymbol{D}_{r+2,i}&=\left(\boldsymbol{\Lambda}_{r+2}+ \boldsymbol{\lambda}_{r+2}\boldsymbol{\alpha}^{\mathrm{T}}\right)\oplus \boldsymbol{M}_i\,,\quad i=1,2\,;\\
\boldsymbol{E}_{1i}&=
\begin{cases}
\boldsymbol{\lambda}_1\boldsymbol{\alpha}^{\mathrm{T}}\otimes\boldsymbol{I}\otimes
\boldsymbol{\beta}_2^{\mathrm{T}}\,, & i=1\,,\\
\boldsymbol{\lambda}_1\boldsymbol{\alpha}^{\mathrm{T}}\otimes
\boldsymbol{\beta}_1^{\mathrm{T}}  \otimes\boldsymbol{I}\,, & i=2;
\end{cases}
\end{align*}
\begin{align*}
%\end{equation*}
%\begin{equation*}
\boldsymbol{E}_{ki}&=\boldsymbol{\lambda}_k\boldsymbol{\alpha}^{\mathrm{T}} \otimes\boldsymbol{I}\otimes\boldsymbol{I}\,,\quad
 k=\overline{2,r+2}\,, \enskip
 i=1,2\,;\\
\boldsymbol{G}_{2i}&=
\begin{cases}
\boldsymbol{I}\otimes \boldsymbol{I}\otimes
\boldsymbol{\mu}_2\,, & i=1\,,\\
\boldsymbol{I} \otimes \boldsymbol{\mu}_1 \otimes \boldsymbol{I}\,, & i=2\,;
\end{cases}
\\
%\begin{equation*}
\boldsymbol{G}_{ki}&=
\begin{cases}
\boldsymbol{I}\otimes \boldsymbol{I}\otimes
\boldsymbol{\mu}_2\boldsymbol{\beta}_2^{\mathrm{T}}\,, & i=1\,,\\
\boldsymbol{I} \otimes \boldsymbol{\mu}_1\boldsymbol{\beta}_1^{\mathrm{T}} \otimes \boldsymbol{I}\,, & i=2\,.
\end{cases}
\end{align*}
Здесь и далее $U\otimes V$~--- кронекерово произведение, а $U\oplus V$~--- кронекерова сумма матриц~$U$ и~$V$.

Стационарное распределение вероятностей $\{p_y, y\in\mathcal{Y}\}$ удовлетворяет следующей системе 
уравнений равновесия (СУР):
\begin{equation}
\label{eq:1}
\boldsymbol{0}^{\mathrm{T}}=\boldsymbol{p}_0^{\mathrm{T}} \boldsymbol{\Lambda}_0+
\boldsymbol{p}_{11,\cdot}^{\mathrm{T}}\left(
\boldsymbol{I}\otimes\boldsymbol{\mu}_1\right) + \boldsymbol{p}_{12,\cdot}^{\mathrm{T}}\left(\boldsymbol{I}\otimes \boldsymbol{\mu}_2\right)\,;
\end{equation}
\begin{multline}
\boldsymbol{0}^{\mathrm{T}}=u(1-n)u(2-i)\boldsymbol{p}_0^{\mathrm{T}} \left(\boldsymbol{\lambda}_0\boldsymbol{\alpha}^{\mathrm{T}} \otimes\boldsymbol{\beta}_1^{\mathrm{T}}\right)+\boldsymbol{p}_{1in}^{\mathrm{T}}\boldsymbol{D}_{1i}+{}\\
{}+\left[u(1-n)\boldsymbol{p}^{\mathrm{T}}_{2,3-i,\cdot}+ u(n)\boldsymbol{p}^{\mathrm{T}}_{2,i,n-1}\right]
\boldsymbol{G}_{2i}\,,\\
 i=1,2\,,\enskip n\geq 0\,;
\label{eq:2}
\end{multline}
\begin{multline}
\label{eq:3}
\boldsymbol{0}^{\mathrm{T}}=u(3-k)\boldsymbol{p}_{1in}^{\mathrm{T}}\boldsymbol{E}_{1i}+
u(k-2)\boldsymbol{p}_{k-1,in}^{\mathrm{T}}\boldsymbol{E}_{k-1,i}+{}\\
{}+\boldsymbol{p}_{kin}^{\mathrm{T}}\boldsymbol{D}_{ki}+
\left[u(1-n)\boldsymbol{p}_{k+1,3-i,\cdot}^{\mathrm{T}}+{}\right.\\
{}+\left. u(n)\boldsymbol{p}_{k+1,i,n-1,}^{\mathrm{T}}\right]\boldsymbol{G}_{k+1,i}\,,\\
k=\overline{2,r+1}\,,\ \  i=1,2\,,\ \ n\geq 0;
\end{multline}
\begin{multline}
\label{eq:4}
\boldsymbol{0}^{\mathrm{T}}=\boldsymbol{p}_{r+1,in}^{\mathrm{T}}\boldsymbol{E}_{r+1,i}+
\boldsymbol{p}_{r+2,in}^{\mathrm{T}}\boldsymbol{D}_{r+2,i}\,,\\ i=1,2\,,\enskip n\geq 0\,,
\end{multline}
с условием нормировки
\begin{equation}
\label{eq:5}
\boldsymbol{p}_0^{\mathrm{T}}\boldsymbol{1}+
\sum\limits_{k=1}^{r+2}\sum\limits_{i=1}^2\sum\limits_{n=0}^{\infty}
\boldsymbol{p}_{kin}^{\mathrm{T}}\boldsymbol{1}=1\,.
\end{equation}
Здесь и далее $u(x)=1$ при $x>0$ и $u(x)=0$ при $x\leq 0$.

Система уравнений~(\ref{eq:1})--(\ref{eq:5}) понадобится в дальнейшем для определения 
стационарных характеристик рассматриваемой СМО.

\section{Факториальные моменты числа заявок в буфере переупорядочивания}

Для определения факториальных моментов числа заявок в БП воспользуемся аппаратом производящих функций (ПФ). Положим
\begin{equation}
\left.
\begin{array}{rl}
 F_{sj_ii}(z)&=\sum\limits_{n\geq 0}p_{sj_iin}z^n,\\[6pt]
 F_{ksj_1j_2i}(z)&=\sum\limits_{n\geq 0}p_{ks j_1j_2in}z^n\,,\enskip s=\overline{1,l}\,,\\
 &\!\!\!\!\!\!\!\!\!\!\!\!\!\!\!\!\!\! j_i=\overline{1,m_i}\,,\  i=1,2\,,\  z\in\mathbb{C}\,,\ |z|\leq 1\,,
\end{array}
\right \}
\label{eq:6}
\end{equation}
и введем векторы $\boldsymbol{F}_{1i}(z)$ и $\boldsymbol{F}_{ki}(z)$, аналогичные по структуре 
векторам~$\boldsymbol{p}_{1i}$ и~$\boldsymbol{p}_{ki}$.


Умножая уравнения~(\ref{eq:2})--(\ref{eq:4}) для каждого фиксированного $n\geq 0$ справа на~$z^n$ 
и суммируя полученные равенства при каждом фиксированном $k=\overline{1,r+2}$, $i=1,2$,  по всем 
возможным значениям~$n$, с учетом~(\ref{eq:6}) приходим к следующей системе уравнений:
\begin{multline}
\label{eq:7}
\boldsymbol{0}^{\mathrm{T}}=u(1-n)u(2-i)\boldsymbol{p}_0^{\mathrm{T}} \left(\boldsymbol{\lambda}_0\boldsymbol{\alpha}^{\mathrm{T}} \otimes\boldsymbol{\beta}_1^{\mathrm{T}}\right)+
\boldsymbol{F}_{1i}^{\mathrm{T}}(z)\boldsymbol{D}_{1i}+{}\\
{}+\left[\boldsymbol{p}_{2,3-i,\cdot}^{\mathrm{T}}+z\boldsymbol{F}_{2i}^{\mathrm{T}}(z)\right] \boldsymbol{G}_{2i},\,\, i=1,2;
\end{multline}
\begin{multline}
\label{eq:8}
\boldsymbol{0}^{\mathrm{T}}=u(3-k)\boldsymbol{F}_{1i}^{\mathrm{T}}(z)\boldsymbol{E}_{1i}+
u(k-2)\boldsymbol{F}_{k-1,i}^{\mathrm{T}}(z)\boldsymbol{E}_{k-1,i}+{}\\
{}+\boldsymbol{F}_{ki}^{\mathrm{T}}(z)\boldsymbol{D}_{ki}
+\left[\boldsymbol{p}_{k+1,3-i,\cdot}^{\mathrm{T}}+z\boldsymbol{F}_{k+1,i}^{\mathrm{T}}(z) \right] \boldsymbol{G}_{k+1,i}\,,\\
k=\overline{2,r+1}\,,\enskip i=1,2\,;
\end{multline}
\begin{multline}
\label{eq:9}
\boldsymbol{0}^{\mathrm{T}}=\boldsymbol{F}_{r+1,i}^{\mathrm{T}}(z)\boldsymbol{E}_{r+1,i}+
\boldsymbol{F}_{r+2,i}^{\mathrm{T}}(z)\boldsymbol{D}_{r+2,i}\,,\\ i=1,2\,.
\end{multline}

Далее введем обозначения:
\begin{align}
\label{eq:10}
 \boldsymbol{v}_{ki\nu}&=\boldsymbol{F}_{ki}^{(\nu)}(1),\,\ 
k=\overline{1,r+2}\,, \ i=1,2\,,\ \nu\geq 0,\\
v_{\nu}&=\sum\limits_{k=1}^{r+2}\sum\limits_{i=1}^2 \boldsymbol{v}_{ki\nu}^{\mathrm{T}} \boldsymbol{1}\,, \enskip \nu\geq 1\,.
\label{dd}
\end{align}

Заметим, что $v_{\nu}$~--- факториальный момент  порядка~$\nu$, $\nu=1,2,\ldots$, числа заявок в БП, 
а $\boldsymbol{v}_{ki0}=\sum\limits_{n\geq 0}\boldsymbol{p}_{kin}$, $k=\overline{1,r+2}$, $i=1,2$. 
Причем система уравнений для определения $\boldsymbol{v}_{ki0}$ получается из~(\ref{eq:7})--(\ref{eq:9}) 
после подстановки в них $z=1$. Кроме того, эта система с учетом~(\ref{eq:1})
 и~(\ref{eq:5}) полностью совпадает с СУР из работы~\cite{2mat} и в силу единственности ее решения получаем
\begin{equation*}
%\label{eq:11}
\boldsymbol{v}_{ki0}=\boldsymbol{p}_{ki},\,\, k=\overline{1,r+2}, \,\, i=1,2.
\end{equation*}


Теперь получим систему уравнений для определения~$\boldsymbol{v}_{ki\nu}$, $k=\overline{1,r+2}$, $i=1,2$, $\nu\geq 1$. 
Для этого продифференцируем~(\ref{eq:7})--(\ref{eq:9}) $\nu$~раз по~$z$ и положим $z=1$. 
В~результате приходим к следующим уравнениям:
\begin{equation}
\!\!\boldsymbol{0}^{\mathrm{T}}=\boldsymbol{v}_{1i\nu}^{\mathrm{T}}\boldsymbol{D}_{1i}+
\left[\boldsymbol{v}_{2i\nu}^{\mathrm{T}}+\nu\boldsymbol{v}_{2i,\nu-1}^{\mathrm{T}}\right]
\boldsymbol{G}_{2i}\,,\enskip i=1,2\,;\!
\label{eq:12}
\end{equation}
\begin{multline}
\label{eq:13}
\boldsymbol{0}^{\mathrm{T}}=u(3-k)\boldsymbol{v}_{1i\nu}^{\mathrm{T}}\boldsymbol{E}_{1i}+
u(k-2)\boldsymbol{v}_{k-1,i\nu}^{\mathrm{T}}\boldsymbol{E}_{k-1,i}+{}\\
{}+\boldsymbol{v}_{ki\nu}^{\mathrm{T}}\boldsymbol{D}_{ki}
+\left[\boldsymbol{v}_{k+1,i\nu}^{\mathrm{T}}+\nu\boldsymbol{v}_{k+1,i,\nu-1}^{\mathrm{T}}\right]
\boldsymbol{G}_{k+1,i}\,,\\
 k=\overline{2,r+1}\,,\enskip i=1,2;
\end{multline}
\begin{multline}
\label{eq:14}
\boldsymbol{0}^{\mathrm{T}}=\boldsymbol{v}_{r+1,i\nu}^{\mathrm{T}}\boldsymbol{E}_{r+1,i}+
\boldsymbol{v}_{r+2,i\nu}\boldsymbol{D}_{r+2,i}\,,\\
i=1,2\,,\enskip \nu=1,2,\ldots
\end{multline}

Решение системы~(\ref{eq:12})--(\ref{eq:14}) получим с помощью блочного 
LU-разложения матрицы коэффициентов. Для этого введем векторы
\begin{align*}
\boldsymbol{v}_{i\nu}^{\mathrm{T}}&=(\boldsymbol{v}_{1i\nu}^{\mathrm{T}},\ldots,
\boldsymbol{v}_{r+2,i\nu}^{\mathrm{T}})\,,\enskip i=1,2\,,\enskip \nu\geq 1\,;
\\
\boldsymbol{d}_{ki\nu}^{\mathrm{T}}&=
\begin{cases}
& -\nu \boldsymbol{v}_{k+1,i,\nu-1}^{\mathrm{T}}\boldsymbol{G}_{k+1,i}\,,\enskip k=\overline{1,r+1}\,;\\
& \boldsymbol{0}^{\mathrm{T}},\,\, k=r+2\,,\enskip i=1,2,\enskip \nu\geq 1\,;
\end{cases}\\
\boldsymbol{d}_{i\nu}^{\mathrm{T}}&=(\boldsymbol{d}_{1i\nu}^{\mathrm{T}},\ldots,
\boldsymbol{d}_{r+2,i\nu}^{\mathrm{T}})\,,\enskip i=1,2\,,\enskip \nu\geq 1\,,
\end{align*}
и матрицы
{\small
\begin{multline*}
\boldsymbol{D}_{i}=
\begin{pmatrix}
\boldsymbol{D}_{1i} &\boldsymbol{E}_{1i} & \boldsymbol{0} &
\boldsymbol{0} & \ldots &\boldsymbol{0} &\boldsymbol{0} &\boldsymbol{0}\\
\boldsymbol{G}_{2i} & \boldsymbol{D}_{2i} & \boldsymbol{E}_{2i} &
\boldsymbol{0} & \ldots & \boldsymbol{0} &\boldsymbol{0} &\boldsymbol{0}\\
\boldsymbol{0} & \boldsymbol{G}_{3i} & \boldsymbol{D}_{3i} &
\boldsymbol{E}_{3i} & \ldots & \boldsymbol{0} & \boldsymbol{0} & \boldsymbol{0}\\
\vdots & \vdots & \vdots & \vdots & \ddots  & \vdots & \vdots & \vdots\\
\boldsymbol{0} & \boldsymbol{0} & \boldsymbol{0} & \boldsymbol{0} & \ldots &
\boldsymbol{G}_{r+1,i} & \boldsymbol{D}_{r+1,i} & \boldsymbol{E}_{r+1,i} \\
\boldsymbol{0} & \boldsymbol{0} & \boldsymbol{0} & \boldsymbol{0} & \ldots & \boldsymbol{0} & \boldsymbol{G}_{r+2,i} & \boldsymbol{D}_{r+2,i}
\end{pmatrix}\,,\\
i=1,2\,.
\end{multline*}}

С учетом введенных обозначений систему уравнений~(\ref{eq:12})--(\ref{eq:14}) можно записать в следующем виде:
\begin{equation}
\label{eq:15}
\boldsymbol{v}_{i\nu}\boldsymbol{D}_{i}=\boldsymbol{d}_{i\nu}^{\mathrm{T}}\,,\enskip i=1,2\,,\,\, \nu \geq 1\,.
\end{equation}

Матрица коэффициентов $\boldsymbol{D}_{i}$ для любого фиксированного $i=1,2$ неразложима 
и обладает  свойствами инфинитезимальной  матрицы с диагональным преобладанием. Следовательно, 
для решения системы~(\ref{eq:15}) можно воспользоваться выводами, полученными в~\cite{3mat}. 
Непосредственно из~\cite{3mat} вытекает  результат, который сформулируем в виде теоремы.

\medskip

\noindent
\textbf{Теорема 1.}\ \textit{Для каждого фиксированного $i=1,2$ и $\nu=1,2,\ldots$ решение системы}~(\ref{eq:15}) 
\textit{представимо в виде}
\begin{align*}
\boldsymbol{v}_{r+2,i\nu}^{\mathrm{T}}&=\boldsymbol{y}_{r+2,i\nu}^{\mathrm{T}}\,;\\
\boldsymbol{v}_{ki\nu}^{\mathrm{T}}&=\boldsymbol{y}_{ki\nu}^{\mathrm{T}}+
\boldsymbol{v}_{k+1,i\nu}^{\mathrm{T}}\boldsymbol{H}_{k+1,i}\,,\enskip k=\overline{r+1,1}\,,
\end{align*}
\textit{где}
\begin{align*}
\boldsymbol{y}_{1i\nu}^{\mathrm{T}}&=\boldsymbol{d}_{1i\nu}^{\mathrm{T}}\boldsymbol{S}_{1i}^{-1}\,;
\\
\boldsymbol{y}_{ki\nu}^{\mathrm{T}}&=\left[\boldsymbol{d}_{ki\nu}^{\mathrm{T}}-
\boldsymbol{y}_{k-1,i\nu}^{\mathrm{T}}\boldsymbol{E}_{k-1,i\nu}\right]
\boldsymbol{S}_{ki}^{-1}\,,\enskip k=\overline{2,r+2}\,;
\\
\boldsymbol{H}_{ki}&=-\boldsymbol{G}_{ki}\boldsymbol{S}_{k-1,i}^{-1}\,,\enskip k=\overline{2,r+2}\,,
\end{align*}
\textit{а невырожденные матрицы $\boldsymbol{S}_{ki}$, $k=\overline{1,r+2}$, задаются соотношениями:}
\begin{equation*}
\boldsymbol{S}_{k,i}=\boldsymbol{D}_{ki}+\boldsymbol{H}_{ki}
\boldsymbol{E}_{k-1,i}u(k-1)\,.
\end{equation*}


\smallskip

Итак, получен алгоритм, позволяющий для каж\-до\-го $\nu=1,2,\ldots$ вычислять факториальный момент~$v_{\nu}$ 
числа заявок в БП через моменты низшего порядка. При этом, начальный шаг алгоритма состоит в определении 
векторов~ $\boldsymbol{p}_{ki}$, для чего следует обратиться к результатам~\cite{2mat}.


\section{Преобразования Лапласа--Стилтьеса стационарной функции распределения задержки переупорядочивания}

Обозначим через~$\delta$ задержку переупорядочивания заявки в стационарном режиме работы СМО, а через~$f(s)$~--- 
преобразование Лапласа--Стилтьеса (ПЛС) ФР~$F_{\delta}(t)$ случайной величины (с.в.)~$\delta$. Далее, обозначим 
через~$f_j(s)$ ПЛС условной стационарной ФР задержки переупорядочивания при условии, что задержана заявка, 
обслуженная прибором~$j$, $j=1,2$.

Очевидно, что заявка, обслуженная прибором~$j$, $j=1,2$, не будет задержана для переупорядочивания, 
если после ее ухода система останется пус\-той либо если в момент $\tau -0$ окончания 
обслуживания этой заявки на приборе~$3-j$ будет\linebreak обслуживаться заявка, пришедшая в систему позже данной заявки. 
Если же в момент $\tau -0$ на приборе~$3-j$ будет обслуживаться заявка, пришедшая в систему  раньше данной, 
то данная заявка будет задержана до момента окончания обслуживания на приборе~$3-j$. Обозначим через 
$\pi_{D,j}^{-}(s,j)$ и $\pi_{D,j}^{-}(k,s,j_{3-j},i)$ стационарные вероятности макросостояний $(s,j)$, 
$(k,s,j_{3-j},i)$ в момент $\tau -0$ выхода заявки из прибора~$j$, $j=1,2$, $k=\overline{2,r+2}$, 
$s=\overline{1,l}$, $j_{3-j}=\overline{1,m_{3-j}}$, $i=1,2$, и введем векторы:
\begin{align*}
\boldsymbol{\pi}_{D,j}^{-\mathrm{T}}(1,j)&=\left(\pi_{D,j}^{-}(1,j),\ldots, \pi_{D,j}^{-}(l,j)\right)\,;
\\
\boldsymbol{\pi}_{D,j}^{-\mathrm{T}}(k,i)&=\left(\pi_{D,j}^{-}(k,1,1,i),\ldots \right.\\
&\!\!\left.\ldots,\pi_{D,j}^{-}(k,1,2,i),\ldots, \pi_{D,j}^{-}(k,l,m_{3-j},i)\right)\,,\\
&\hspace*{15mm}j=1,2\,,\enskip  i=1,2\,,\,\, k=\overline{2,r+2}\,.
\end{align*}

Тогда на основании вышеизложенного и с учетом~[1, c.~104] получаем
\begin{multline}
\label{eq:16}
f_1(s)=\sum\limits_{k=1}^{r+2}\boldsymbol{\pi}_{D,1}^{-\mathrm{T}}(k,1) \boldsymbol{1}+{}\\
{}
+\sum\limits_{k=2}^{r+2}\boldsymbol{\pi}_{D,1}^{-\mathrm{T}}(k,2)
\left({\bf{1}}\otimes\boldsymbol{I}\right)
\left(s\boldsymbol{I}-\boldsymbol{M}_2\right)^{-1}\boldsymbol{\mu}_2\,;
\end{multline}

\vspace*{-6pt}

\noindent
\begin{multline}
\label{eq:17}
f_2(s)=\sum\limits_{k=1}^{r+2}\boldsymbol{\pi}_{D,2}^{-\mathrm{T}}(k,2) \boldsymbol{1}+{}\\
{}+\sum\limits_{k=2}^{r+2}\boldsymbol{\pi}_{D,1}^{-\mathrm{T}}(k,1)
\left({\bf{1}}\otimes\boldsymbol{I}\right)
\left(s\boldsymbol{I}-\boldsymbol{M}_1\right)^{-1}\boldsymbol{\mu}_1\,.
\end{multline}

Для определения стационарных вероятностей $\pi_{D,j}^{-}(k,j_{3-j},i)$, $\pi_{D,j}^{-}(s,j)$, 
$j=1,2$, $k=\overline{2,r+2}$, $s=\overline{1,l}$, $j_{3-j}=\overline{1,m_{3-j}}$, $i=1,2$, 
воспользуемся результатами~\cite{4mat}, согласно которым
\begin{equation}
\label{eq:18}
\boldsymbol{\pi}_{D,j}^{-\mathrm{T}}(1,j)=\frac{1}{\lambda_{D}(j)}\boldsymbol{p}_{1j}^{\mathrm{T}}
\left(\boldsymbol{I}\otimes\boldsymbol{\mu}_j\right)\,,\,\, j=1,2\,;
\end{equation}
\begin{multline}
\label{eq:19}
\boldsymbol{\pi}_{D,1}^{-\mathrm{T}}(k,i)=\frac{1}{\lambda_{D}(1)}\boldsymbol{p}_{ki}^{\mathrm{T}}
\left(\boldsymbol{I}\otimes\boldsymbol{\mu}_1\otimes\boldsymbol{I}\right)\,,\\
k=\overline{2,r+2}\,,\enskip i=1,2\,;
\end{multline}
\begin{multline}
\label{eq:20}
\boldsymbol{\pi}_{D,2}^{-\mathrm{T}}(k,i)=\frac{1}{\lambda_{D}(2)}\boldsymbol{p}_{ki}^{\mathrm{T}}
\left(\boldsymbol{I}\otimes\boldsymbol{I}\otimes\boldsymbol{\mu}_2\right)\,,\\
 k=\overline{2,r+2}\,,\enskip i=1,2\,,
\end{multline}
где $\lambda_{D}(j)$~--- интенсивность выхода заявок, обслуженных прибором~$j$, определяемая выражением
\begin{multline*}
\lambda_{D}(j)=\boldsymbol{p}_{1j}^{\mathrm{T}}\left(1\otimes\boldsymbol{\mu}_j\right)+{}\\
{}+\sum\limits_{k=2}^{r+2}\boldsymbol{p}_{k,\cdot}^{\mathrm{T}}
\left[u(2-j)(\boldsymbol{1}\otimes\boldsymbol{\mu}_1\otimes\boldsymbol{1})+{}\right.\\
\left.{}+
u(j-1)(\boldsymbol{1}\otimes\boldsymbol{1}\otimes\boldsymbol{\mu}_2)\right]\,,\,\,
j=1,2\,.
\end{multline*}

Далее заметим, что вероятность выхода заявок из прибора~$j$ равна $\lambda_{D}(j)/\lambda_{D}$, $j=1,2$, 
где интенсивность выходящего из системы потока
\begin{equation*}
\lambda_D=\lambda_D(1)+\lambda_D(2)\,.
\end{equation*}

Таким образом, подытоживая рассуждения, получаем, что справедлива

\smallskip

\noindent
\textbf{Теорема 2.} \textit{Преобразование Лапласа--Стилтьеса ФР задержки переупорядочивания в СМО $PH/PH/2/r/res$ в стационарном режиме 
ее работы определяется выражением
\begin{equation*}
f(s)=\fr{1}{\lambda_D}\left[\lambda_D(1)f_1(s)+\lambda_D(2)f_2(s)\right]\,,
\end{equation*}
где $f_j(s)$, $j=1,2$, задаются формулами}~(\ref{eq:16}) \textit{и}~(\ref{eq:17}).

\smallskip

Обозначим через~$\delta_{\nu}$ начальный момент порядка~$\nu$, $\nu=1,2,\ldots$, 
задержки переупорядочивания заявки в исследуемой СМО. Тогда из теоремы~2 с 
учетом~(\ref{eq:18})--(\ref{eq:20}) и~[1, с.~104] получаем очевидное

\medskip

\noindent
\textbf{Следствие.} {\it{Начальный момент порядка~$\nu$ задержки переупорядочивания в 
СМО $PH/PH/2/r/res$ определяется выражением}}
\begin{multline}
\label{eq:21}
\delta_{\nu}=\fr{(-1)^{\nu}\nu!}{\lambda_D}\,
\sum\limits_{k=2}^{r+2}\left[\boldsymbol{p}_{k2}^{\mathrm{T}}
\left(\boldsymbol{1}\otimes\boldsymbol{\mu}_1\otimes\boldsymbol{M}_2^{-\nu} \boldsymbol{1}\right)+{}\right.\\
\left.{}+\boldsymbol{p}_{k1}^{\mathrm{T}}
\left(\boldsymbol{1}\otimes\boldsymbol{M}_1^{-\nu} \boldsymbol{1}\otimes\boldsymbol{\mu}_2\right)\right]\,.
\end{multline}



Теперь остановимся на связи средней задержки переупорядочивания со средним числом заявок в БП.

\bigskip

\noindent
{\bf{Теорема 3.}} {\it{Средняя величина задержки переупорядочивания~$\delta_1$ и среднее число заявок~$v_1$ 
в БП СМО $PH/PH/2/r/res$ связаны соотношением
\begin{equation}
\label{eq:22}
\delta_1=\fr{1}{\lambda_d}\,v_1\,.
\end{equation}
}}

\medskip

\noindent
Д\,о\,к\,а\,з\,а\,т\,е\,л\,ь\,с\,т\,в\,о\,.\ \ Умножим уравнения системы~(\ref{eq:12})--(\ref{eq:14}) 
при фиксированном $i=1,2$ и $\nu=1$ справа на матрицу~$\boldsymbol{Z}_i$:
\begin{equation*}
\boldsymbol{Z}_i=
\left.
\begin{cases}
\boldsymbol{1}\otimes \boldsymbol{I}\otimes \boldsymbol{1}\,,\enskip i=1\,,\\[6pt]
 \boldsymbol{1}\otimes \boldsymbol{1}\otimes \boldsymbol{I}\,,\enskip i=2\,.
\end{cases}\!\!\!\!\!\!
\right \}
\end{equation*}

В результате умножения получим следующую систему:
\begin{equation}
\label{eq:23}
\boldsymbol{0}^{\mathrm{T}}=\boldsymbol{v}_{1i1}^{\mathrm{T}}(-\boldsymbol{L}_{1i}+\boldsymbol{U}_i)+
(\boldsymbol{v}_{2i1}^{\mathrm{T}}+\boldsymbol{p}_{2i}^{\mathrm{T}})\boldsymbol{N}_i,\,\, i=1,2;
\end{equation}
\begin{multline}
\label{eq:24}
\boldsymbol{0}^{\mathrm{T}}=u(3-k)\boldsymbol{v}_{1i1}^{\mathrm{T}}\boldsymbol{L}_{1i}+
u(k-2)\boldsymbol{v}_{k-1,i1}^{\mathrm{T}}\boldsymbol{L}_{k-1,i}+{}\\
{}+\boldsymbol{v}_k^{\mathrm{T}}(-\boldsymbol{L}_{ki}+\boldsymbol{U}_i-\boldsymbol{N}_i)
+(\boldsymbol{v}_{k+1,i1}^{\mathrm{T}}+\boldsymbol{p}_{k+1,i}^{\mathrm{T}})\boldsymbol{N}_i\,,\\
i=1,2\,,\enskip k=\overline{2,r+1}\,;
\end{multline}
\begin{equation}
\label{eq:25}
\boldsymbol{0}^{\mathrm{T}}=\boldsymbol{v}_{r+1}^{\mathrm{T}}\boldsymbol{L}_{r+1,i}+
\boldsymbol{v}_{r+2}^{\mathrm{T}}(\boldsymbol{U}_i-\boldsymbol{N}_i)\,,\enskip i=1,2\,,
\end{equation}
где
\begin{align*}
\boldsymbol{L}_{ki}&=
\begin{cases}
\boldsymbol{\lambda}_k^{\mathrm{T}}\otimes\boldsymbol{I}\otimes\boldsymbol{1}\,, & i=1\,,\\[6pt]
\boldsymbol{\lambda}_k^{\mathrm{T}}\otimes\boldsymbol{1}\otimes\boldsymbol{I}\,, &i=2\,,\,\,
k=\overline{1,r+1}\,;
\end{cases}
\\
\boldsymbol{U}_i&=
\begin{cases}
\boldsymbol{1}\otimes\boldsymbol{M}_1\otimes\boldsymbol{1}\,, \enskip i=1\,,\\[6pt]
\boldsymbol{1}\otimes\boldsymbol{1}\otimes\boldsymbol{M}_2\,,\enskip i=2\,;
\end{cases}
\\
\boldsymbol{N}_i&=
\begin{cases}
\boldsymbol{1}\otimes\boldsymbol{I}\otimes\boldsymbol{\mu}_2\,,\enskip i=1\,,\\[6pt]
\boldsymbol{1}\otimes\boldsymbol{\mu}_1\otimes\boldsymbol{I}\,,\enskip i=2\,.
\end{cases}
\end{align*}

Последовательно суммируя уравнения системы~(\ref{eq:23})--(\ref{eq:25}), 
после несложных алгебраических преобразований приходим к следующим соотношениям:
\begin{equation}
\label{eq:26}
-\sum\limits_{k=1}^{r+2}\boldsymbol{v}_{ki1}^{\mathrm{T}}\boldsymbol{U}_i=
\sum\limits_{k=2}^{r+2}\boldsymbol{p}_{ki}^{\mathrm{T}}\boldsymbol{N}_i\,,\enskip i=1,2\,.
\end{equation}


Умножим~(\ref{eq:26}) при фиксированном $i=1,2$ справа на матрицу
\begin{equation*}
\boldsymbol{T}_i=
\begin{cases}
\boldsymbol{I}\otimes\boldsymbol{M}_1^{-1}\boldsymbol{1}\otimes\boldsymbol{I}\,, \enskip i=1\,\\[6pt]
\boldsymbol{I}\otimes\boldsymbol{I}\otimes\boldsymbol{M}_2^{-1}\boldsymbol{1}\,, \enskip i=2,
\end{cases}
\end{equation*}
и просуммируем по $i$ полученные выражения. В~результате, учитывая (\ref{eq:10}) и~(\ref{dd}), 
приходим к сле\-ду\-ющему равенству:
\begin{multline}
\label{eq:27}
v_1=-\sum\limits_{k=2}^{r+2}\left[\boldsymbol{p}_{k1}^{\mathrm{T}}
(\boldsymbol{1}\otimes\boldsymbol{M}_1^{-1}\boldsymbol{1} \otimes\boldsymbol{\mu}_2)+{}\right.\\
\left.{}+
\boldsymbol{p}_{k2}^{\mathrm{T}}
(\boldsymbol{1}\otimes\boldsymbol{\mu}_1 \otimes\boldsymbol{M}_2^{-1}\boldsymbol{1})\right]\,.
\end{multline}

Из сравнения~(\ref{eq:27}) и~(\ref{eq:21}) при $\nu=1$ очевидным образом вытекает~(\ref{eq:22}). 
Таким образом, теорема доказана.

%\medskip

Заметим, что~(\ref{eq:22}) является аналогом известной формулы Литтла и имеет вполне очевидную физическую интерпретацию.

{\small\frenchspacing
{%\baselineskip=10.8pt
\addcontentsline{toc}{section}{Литература}
\begin{thebibliography}{9}

\bibitem{1mat} 
\Au{Бочаров П.\,П., Печинкин А.\,В.} 
Теория массового обслуживания.~--- М.:  РУДН, 1995.

\bibitem{2mat} 
\Au{Матюшенко С.\,И.} 
Анализ двухканальной системы обслуживания ограниченной емкости с буфером переупорядочивания и с распределениями фазового типа~// 
Вестник РУДН: Прикладная математика. Информатика. Физика, 2010. №\,4. С.~84--88.

 \bibitem{3mat} 
 \Au{Наумов В.\,А.} 
 Численные методы анализа марковских систем. --- М.: УДН, 1985.
 
 \label{end\stat}

 \bibitem{4mat} 
 \Au{Наумов В.\,А.} 
 О предельных вероятностях полумарковского процесса~// Современные задачи в точных науках.~--- М.: УДН, 1975. С.~35--39.
 \end{thebibliography}
}
}


\end{multicols}%8
\def\stat{shestakov+vor}

\def\tit{АСИМПТОТИЧЕСКАЯ НОРМАЛЬНОСТЬ И~СИЛЬНАЯ СОСТОЯТЕЛЬНОСТЬ ОЦЕНКИ РИСКА ПРИ~ИСПОЛЬЗОВАНИИ FDR-ПОРОГА В УСЛОВИЯХ СЛАБОЙ ЗАВИСИМОСТИ}

\def\titkol{Асимптотическая нормальность и~сильная состоятельность оценки риска при~использовании FDR-порога} % в~условиях слабой зависимости}

\def\aut{М.\,О.~Воронцов$^1$, О.\,В.~Шестаков$^2$}

\def\autkol{М.\,О.~Воронцов, О.\,В.~Шестаков}

\titel{\tit}{\aut}{\autkol}{\titkol}

\index{Воронцов М.\,О.}
\index{Шестаков О.\,В.}
\index{Vorontsov M.\,O.}
\index{Shestakov O.\,V.}


%{\renewcommand{\thefootnote}{\fnsymbol{footnote}} \footnotetext[1]
%{Работа 
%выполнена при поддержке Программы развития МГУ, проект №\,23-Ш03-03. При анализе 
%данных использовалась инфраструктура Центра коллективного пользования 
%<<Высокопроизводительные вычисления и~большие данные>> 
%(ЦКП <<Информатика>>) ФИЦ ИУ РАН (г.~Москва)}}


\renewcommand{\thefootnote}{\arabic{footnote}}
\footnotetext[1]{Московский государственный университет 
имени~М.\,В.~Ломоносова, факультет вычислительной математики и~кибернетики;  
Московский центр фундаментальной и~прикладной математики, \mbox{m.vtsov@mail.ru}}
\footnotetext[2]{Московский государственный университет 
имени М.\,В.~Ломоносова, факультет вычислительной математики и~кибернетики; 
Федеральный исследовательский центр <<Информатика и~управление>> Российской 
академии наук; Московский центр фундаментальной и~прикладной математики, 
\mbox{oshestakov@cs.msu.ru}}


\vspace*{-12pt}





\Abst{Рассматривается подход к~решению задачи удаления шума в~большом массиве 
разреженных данных, основанный на методе контроля средней доли ложных отклонений 
гипотез (False Discovery Rate, FDR). Данный подход эквивалентен процедурам 
пороговой обработки, обнуляющим компоненты массива, значения которых не 
превосходят некоторого заданного порога.  Наблюдения в~модели считаются слабо 
зависимыми. Для контроля степени зависимости используются ограничения на 
коэффициент сильного перемешивания и~максимальный коэффициент корреляции. 
В~качестве меры эффективности рассматриваемого подхода используется 
среднеквадратичный риск. Вычислить значение риска можно только на тестовых 
данных, поэтому в~работе рассматривается его статистическая оценка и~исследуются 
ее свойства. Показана асимптотическая нормальность и~сильная состоятельность 
оценки риска при использовании FDR-по\-ро\-га в~условиях слабой зависимости в~данных.}

\KW{пороговая обработка; множественная проверка гипотез; 
оценка риска}

\DOI{10.14357/19922264240309}{ZOQVTO}
  
%\vspace*{-6pt}


\vskip 10pt plus 9pt minus 6pt

\thispagestyle{headings}

\begin{multicols}{2}

\label{st\stat}



\section{Введение}

Во многих прикладных областях возникает задача обработки больших массивов 
зашумленных данных. Примерами служат задачи обработки изоб\-ра\-же\-ний с~высоким 
разрешением~\cite{FDRImage}, задачи множественной проверки гипотез, возникающие 
в~\mbox{исследованиях} в~об\-ласти генетики~\cite{MultipleTesting}, и~другие проб\-ле\-мы. 
В~связи с~этим рас\-смот\-рим модель
$$
x_i = \mu_i + z_i, \enskip i=\overline{1,n}\,,
$$
где $\mu_i\in\mathbb{R}$~--- <<полезные>> данные; $z_i \sim N(0,\sigma^2)$~--- 
шум. Задача заключается в~нахождении оценки неизвестного вектора $\mu \hm= 
(\mu_1,\ldots,\mu_n)$ как функции вектора $x \hm= (x_1,\ldots,x_n)$ и~может 
рассматриваться как задача множественной проверки гипотез о~равенстве нулю 
компонент вектора~$\mu$~\cite{AdaptingFDR}. При этом обычно предполагается, что 
вектор~$\mu$ имеет в~определенном смысле <<разреженную>> структуру, т.\,е.\ для 
<<полезных>> данных используется <<экономное>> представление.



В работе~\cite{AdaptingFDR} для решения рассматриваемой задачи в~условиях 
независимости компонент вектора~$x$ и~разреженности вектора~$\mu$ была 
предложена процедура построения оценки~$\hat{\mu}_F$ вектора~$\mu$, основанная 
на методе контроля средней доли ложных отклонений (FDR) 
гипотез при помощи алгоритма Бен\-жа\-ми\-ни--Хох\-бер\-га,
и~было проведено исследование асимптотики ее среднеквадратичного риска. 
В~работах~\cite{ZasShe17,Mathematics2020} была показана состоятельность 
и~асимптотическая нормальность оценки риска данной процедуры. Аналогичные 
результаты для других методов построения~$\hat{\mu}_F$ получены в~работах~\cite{Shestakov2021-1,Shestakov2021-2,Shestakov2022}.

В то же время в~определенных приложениях, например  при анализе полученных 
в~результате использования ДНК-мик\-ро\-чи\-пов данных~\cite{ResultsOnFDRUnderDependence}, исследовании геофизических процессов 
и~анализе помех\linebreak в~телекоммуникационных каналах, условие незави\-си\-мости компонент 
вектора $x$ может не выполняться. Ранее в~работах~\cite{VorontsovShestakov2023,Vorontsov2024} была \mbox{исследована} асимп\-то\-ти\-ка 
среднеквадратичного риска оценки~$\hat{\mu}_F$ \mbox{в~случае}, когда~$\mu$ принадлежит 
одному из классов разреженности
$$
l_0[\eta] = \left\{\mu\,:\, ||\mu||_0 \leq \eta n\right\}, \enskip \eta \in 
(0,1),
$$

\vspace*{-12pt}

\noindent
\begin{multline*}
m_p[\eta] \equiv{}\\
{}\equiv \left\{\mu \in \mathbb{R}^n : |\mu|_{(k)} \leq \eta n^{1/p} 
k^{-1/p},\ k=\overline{1,n}\right\}, \\
 p\in(0, 2),
\end{multline*}
а компоненты вектора~$x$ слабо зависимы~--- имеют достаточно быстро убывающий 
коэффициент сильного перемешивания~\cite{Bosq}

\noindent
\begin{multline*}
\alpha(k) = \sup\limits_{1\leq m\leq n}\alpha\left(\sigma(x_i, i\leq m), 
\sigma(x_i, i\geq m+k)\right), \\ 
k=\overline{1,n-1}\,,
\end{multline*}
где символом $\sigma(x_i, i\in I)$ обозначена сиг\-ма-ал\-геб\-ра, порожденная 
множеством случайных величин $\{x_i, i \hm\in I\}$, а~мера  $\alpha(\cdot, \cdot)$ 
близости двух сиг\-ма-ал\-гебр определяется как
$$
\alpha(\mathcal{B},\mathcal{C}) = \sup\limits_{B\in\mathcal{B}, 
C\in\mathcal{C}} \left|\p(BC)-\p(B)\p(C)\right|.
$$

В настоящей работе показана асимптотическая нормальность и~сильная 
состоятельность оценки риска при применении FDR-про\-це\-ду\-ры в~случае, когда 
компоненты вектора~$x$ слабо зависимы, а~$\mu$ принадлежит одному из классов 
раз\-ре\-жен\-ности: 
$l_0[\eta]$ или $m_p[\eta]$.


\section{Обработка вектора данных с~помощью FDR-процедуры}

Широким классом методов построения оценки~$\hat{\mu}$ стала пороговая обработка 
вектора~$x$ с~некоторым порогом~$T$. Различают жесткую пороговую обработку, при 
которой полагается
\begin{equation*}
\left(\hat{\mu}\right)_i  = p_H(x_i,T) \equiv
 \begin{cases}
   x_i, & |x_i| > T\,;\\
   0, & |x_i| \leq T\,,
 \end{cases}
\end{equation*}
и мягкую пороговую обработку, для которой
\begin{equation*}
(\hat{\mu})_i  = p_S(x_i,T) \equiv
 \begin{cases}
   x_i-T, & \hphantom{\vert\vert}x_i > T;\\
   x_i+T, & \hphantom{\vert\vert}x_i <- T;\\
   0, & |x_i| \leq T.
 \end{cases}
\end{equation*}
Среднеквадратичный риск подобных процедур определяется как
\begin{equation}
\label{riskDef}
R(T) = {\mathsf E} ||\hat{\mu}-\mu||^2 = \sum\limits_{i=1}^n {\mathsf E} \left((\hat{\mu})_i-
\mu_i\right)^2.
\end{equation}
Обозначим через~$T_m$ наилучшее значение порога:
$$
T_m : \, R(T_m) = \min\limits_{T} R(T).
$$

Предложенная в~\cite{AdaptingFDR} процедура заключается в~жесткой пороговой 
обработке компонент вектора~$x$ с~порогом $\hat{t}_F \hm= \hat{t}_F(x)$, и~ее 
результат~--- оценка $\hat{\mu}_F$ вектора~$\mu$ с~компонентами $(\hat{\mu}_F)_i  
\hm= p_H(x_i,\hat{t}_F)$, где
\begin{multline*}
\hat{t}_F = \sigma z\left(\fr{q \hat{k}_F}{2n}\right), \enskip
\hat{k}_F = \max 
\left\{k \, :\, |x|_{(k)} \geq t_k \right\}, \\
 t_k = \sigma z\left(\fr{q  k}{2n}\right);
\end{multline*}
$z(\alpha)$ --- квантиль уровня $(1\hm-\alpha)$ стандартного нормального 
распределения; $|x|_{(k)}$~--- $k$-й элемент вектора, получаемого в~результате 
упорядочения вектора~$|x|$ по невозрастанию:
$$
|x|_{(1)} \geq |x|_{(2)} \geq \cdots \geq |x|_{(n)};
$$
$q\in(0;1)$~--- управ\-ля\-ющий параметр FDR-ме\-то\-да.
Далее полагается, что $q\hm\equiv q_n$ зависит от~$n$. В~\cite{AdaptingFDR} 
показано, что эта процедура эквивалентна множественной проверке гипотез 
о~равенстве нулю компонент наблюдаемого вектора. Также показано, что с~помощью 
метода штрафных функций данную процедуру можно свести к~другим видам пороговой 
обработки, в~част\-ности к~мягкой пороговой обработке.

В работах~\cite{VorontsovShestakov2023, Vorontsov2024} была исследована 
асимптотика среднеквадратичного риска~$R(\hat{t}_F)$ описанной процедуры 
в~случае, когда компоненты вектора $x$ слабо зависимы, а $\mu$ принадлежит классу 
разреженности~$\Theta_n$, где~$\Theta_n$ есть~$l_0[\eta_n]$ или~$m_p[\eta_n]$. 
Было показано, что~$R(\hat{t}_F)$ асимптотически отличается от минимаксного 
риска
$\inf\nolimits_{\hat{\mu}\hm=\hat{\mu}(x)} \sup\nolimits_{\mu\in \Theta_n} {\mathsf E} 
||\hat{\mu}-\mu||^2$
на множитель не более чем логарифмического по\-рядка.

Отметим, что в~выражении для среднеквадратичного риска~(\ref{riskDef}) 
присутствуют неизвестные величины~$\mu_i$, а~потому вычислить~$R(T_m)$ и~$T_m$ 
не представляется возможным. На практике можно пользоваться, например, следующей 
оценкой среднеквадратичного риска~\cite{Mallat}:
$$
\hat{R}(T) = \sum\limits_{i=1}^n F[x_i, T],
$$
где  
\begin{multline*}
F[x_i, T] = {}\\[3pt]
{}=\!\begin{cases}
\left(x_i^2-\sigma^2\right) \Ik(|x_i|\leq T) + \sigma^2 \Ik\left(|x_i|>T\right) &\\[3pt]
&\hspace*{-53mm}\mbox{для\ жесткой\ пороговой\ обработки};\\[3pt]
\left(x_i^2-\sigma^2\right) \Ik\left(|x_i|\leq T\right) + (\sigma^2+T^2) 
\Ik \left(|x_i|>T\right) \hspace*{-11.21576pt}&\\[3pt]
&\hspace*{-51mm}\mbox{для\ мягкой\ пороговой\ обработки}.
\end{cases}\hspace*{-7.17859pt}
\end{multline*}


\noindent
\textbf{Замечание}.\ При пороговой обработке иногда также используется так 
называемый универсальный порог $T_U\hm = \sigma \sqrt{2\ln n}$, предложенный 
в~работе~\cite{spatialAdaptation}. Исследования в~\cite{AdaptingSURE, ExactRisk} 
показали, что порог~$T_U$ в~определенном смысле максимальный, и~рас\-смат\-ри\-вать 
пороги выше него не имеет смысла. Более того, нетрудно показать, что $t_k \hm< T_U$ 
для всех~$k$ и~всех достаточно больших~$n$, в~связи с~чем всюду далее полагаем, 
что порог~$\hat{t}_F$ выбирается на отрезке $[0; T_U]$.

\section{Вспомогательные утверждения}

Кроме коэффициента сильного перемешивания~$\alpha(\cdot)$ также понадобится 
следующее понятие~\cite{Bosq}.

\smallskip

\noindent
\textbf{Определение.} %\label{defRho}
Максимальным коэффициентом корреляции~$\rho(\cdot)$ компонент вектора~$x$ 
называется
\begin{multline*}
\rho (k) \equiv \rho_n (k) = {}\\
{}=\sup\limits_{1\leq m\leq n}\rho\left(\sigma(x_i, 
i\leq m), \sigma(x_i, i\geq m+k)\right), \\
 k=\overline{1,n-1}\,,
\end{multline*}
где мера $\rho(\cdot, \cdot)$ близости двух сиг\-ма-ал\-гебр определяется как
$$
\rho(\mathcal{B},\mathcal{C}) = \sup\limits_{\substack{\xi 
\in\mathcal{L}^2(\mathcal{B}) \\
 \eta \in\mathcal{L}^2(\mathcal{C})}} 
\left|\mathrm{corr}\,(\xi, \eta)\right|.
$$


Введем обозначения:
$$
T_1 = \sqrt{2\ln \eta_n^{-p}};  \,\gamma_n = \fr{1}{\ln\ln n}; \, \kappa_n 
= \fr{n \eta_n^p T_1^{-p}}{1 - q_n - \gamma_n}; 
$$
$$ 
\kappa_n^0 = \fr{[n \eta_n]}{1 - q_n - \gamma_n} ;\, \rho^\star (k) = 
\sup\limits_{n\geq k+1} \rho(k), k \in \mathbb{N} ;
$$
$$
t_{\kappa_n} = \sigma z\left(\fr{q_n \kappa_n }{2n}\right) , \,\, t_{\kappa_n^0} 
= \sigma z\left(\fr{q_n \kappa_n^0 }{2n}\right).
$$


Следующие два утверждения показывают, что случайный порог~$\hat{t}_F$ в~случае 
$\mu\hm\in m_p[\eta_n]$ (соответственно $\mu\hm\in l_0[\eta_n]$) с~большой 
вероятностью будет не меньше~$t_{\kappa_n}$ (соответственно~$ t_{\kappa_n^0}$). 
Их  доказательства приведены в~работах~\cite{VorontsovShestakov2023, Vorontsov2024}.

\smallskip

\noindent
%\begin{lem}\label{lem5}
\textbf{Лемма~1.}\ \textit{Пусть $n^{-\delta_1} \hm\leq \eta_n^p \hm\leq n^{-\delta_2}$, 
$0\hm<\delta_2\hm<\delta_1<1$, $\mathrm{lim\,inf} q_n \ln n \hm\geq C \hm> 0$, 
$m\hm\in[1;n/2]\cap\mathbb{N}$, а $\alpha(\cdot)$~--- коэффициент сильного 
перемешивания компонент вектора~$x$. Для некоторого $N\hm\in\mathbb{N}$ при $n \hm\geq 
N$ справедливо}
\begin{multline*}
\hspace*{-3pt}\sup\limits_{\mu\in m_p[\eta_n]} \p \left(\hat{k}_F \geq \kappa_n \right) \leq 
4 n \exp\left\{-\fr{m}{256n}  \kappa_n q_n \gamma_n^2    \right\}+{}\\
{}+ 22\left(1+\fr{8n}{\kappa_n q_n \gamma_n}\right)^{1/2} n m 
\alpha\left(\left[\fr{n}{2m}\right]\right).
\end{multline*}



\smallskip

\noindent
\textbf{Лемма 2.}\ 
%\label{lem1}
\textit{Пусть $\eta_n \hm\leq b\hm<1$, $m\in[1;n/2]\cap\mathbb{N}$, а~$\alpha(\cdot)$~--- 
коэффициент сильного перемешивания компонент вектора~$x$. Для некоторого 
$N\hm\in\mathbb{N}$ при $n \hm\geq N$ справедливо}
\begin{multline*}
\sup\limits_{\mu\in l_0[\eta_n]} \p \left(\hat{k}_F \geq \kappa_n^0 \right) 
\leq{}\\
{}\leq 4 n \exp\left\{-\fr{(1-b)m}{64n}\,  \kappa_n^0 q_n \gamma_n^2    
\right\}+{}\\
{}+ 22\left(1+\fr{4n}{(1-b)\kappa_n^0 q_n \gamma_n}\right)^{1/2} n m 
\alpha\left(\left[\fr{n}{2m}\right]\right).
\end{multline*}

Следующие два утверждения доказаны в~\cite{Bosq} и~представляют собой аналоги 
неравенств Хеффдинга и~Бернштейна для слабо зависимых случайных величин.


\smallskip

\noindent
\textbf{Лемма 3.}\
\textit{Пусть для набора действительных случайных величин $X_1, \ldots, X_n$ 
с~коэффициентом сильного перемешивания $\alpha(\cdot)$ выполняется ${\mathsf E} X_i \hm=0$, 
$|X_i|\hm\leq b$, $i\hm=\overline{1,n}$. Тогда для любого целого числа $m\hm\in[1; n/2]$ 
и~любого $\eps\hm>0$ справедливо}
\begin{multline*}
\p\left(\left|\sum\limits_{i=1}^n X_i\right| > n\eps \right) \leq 4 
\exp\left\{-\fr{\eps^2 m}{8 b^2}\right\}+ {}\\
{}+
22\left(1+\fr{4b}{\eps}\right)^{1/2} m\, 
\alpha\left(\left[\fr{n}{2m}\right]\right).
\end{multline*}


\smallskip

\noindent
\textbf{Лемма 4.}\
\textit{Пусть для набора действительных случайных величин $X_1, \ldots, X_k$ 
с~коэффициентом сильного перемешивания $\alpha(\cdot)$ выполняется ${\mathsf E} X_i \hm=0$, 
$|X_i|\hm\leq b$, $i\hm=\overline{1,k}$. Тогда для любого целого числа $m\hm\in[1; k/2]$ 
и~любого $\eps\hm>0$ справедливо}
\begin{multline*}
\p\left(\left|\sum\limits_{i=1}^k X_i\right| > \eps \right) \leq 4 
\exp\left\{-\fr{\eps^2 m}{8 v^2 k^2}\right\}+{}\\
{}+ 22\left(1+\fr{4bk}{\eps}\right)^{1/2} m\, 
\alpha\left(\left[\fr{k}{2m}\right]\right),
\end{multline*}
\textit{где $p = k/(2m)$}:
\begin{multline*}
v^2 =
 \fr{b \eps}{2k} + {}\\
 {}+\fr{2}{p^2} \,  \max\limits_{ j\in[0,\,2m-1]} 
{\mathsf E} \big( ([jp]+1-jp)X_{[jp]+1} + X_{[jp]+2}+{}\\
{}+ \cdots +  X_{[(j+1)p]} + ((j+1)p-[(j+1)p])X_{[(j+1)p+1]}\big)^2.
\end{multline*}

\noindent
\textbf{Замечание.}
Если существует такое число $S \hm> 0$, что сразу для всех $i\hm\in[1;k]$  выполняется 
${\mathsf E} X_i^2 \hm\leq S^2$, то в~качестве~$v^2$ можно взять
$$
v^2 = \fr{b \eps}{2k} + 8 S^2.
$$


Д\,о\,к\,а\,з\,а\,т\,е\,л\,ь\,с\,т\,в\,о\ \ сле\-ду\-юще\-го утверж\-де\-ния приведено в~работе~\cite{AdaptingFDR}.

\smallskip

\noindent
\textbf{Лемма 5.}\ 
\textit{Для $y\leq 0{,}01$ справедливы представления}
\begin{multline}
\label{lem1eq1}
z^2(y) = 2 \ln y^{-1} - \ln \ln y^{-1} - r_2(y), \\
 r_2(y) \in [1{,}8; 3];
\end{multline}

\noindent
\begin{equation}
\label{lem1eq2}
z(y) = \sqrt{2 \ln y^{-1}} - r_1(y), \, \, r_1(y) \in [0; 1{,}5].
\end{equation}


\section{Асимптотическая нормальность оценки риска при~применении FDR-процедуры в~условиях слабой зависимости}

Перейдем к~описанию достаточных условий для асимптотической нормальности оценки 
риска $\hat{R}(\hat{t}_F)$ в~случае $\mu \hm\in m_p[\eta_n]$.

\smallskip

\noindent
\textbf{Теорема~1.}\
\textit{Пусть $\mu \hm\in m_p[\eta_n],$ $\eta_n^p \hm\in[n^{-\delta_1}; n^{-\delta_2}],$ $1/2 \hm< 
\delta_2 \hm< \delta_1<1;$ имеются такие константы $c_1, c_2>0$, что для 
коэффициента сильного перемешивания $\alpha(\cdot)$ компонент вектора $x$ 
справедливо  $\alpha(k) \hm\leq c_1 k^{-1-(5/2)\delta_1/(1-\delta_1)-c_2},$ 
$k\hm=\overline{1,n-1};$ $q_n \hm< c_3 \hm< 1;$ $\mathrm{lim\,inf} q_n \ln n \hm= c_4 \hm> 0;$ и,~кроме того, 
для максимального коэффициента корреляции $\rho(\cdot)$ компонент вектора~$x$ 
справедливо}
$$
\sum\limits_{k = 1}^{\infty} \sup\limits_{n\geq k+1} \rho(k) \equiv 
\sum\limits_{k = 1}^{\infty}  \rho^\star (k) = c_5 < \infty. 
$$
\textit{Тогда при $n \to \infty$}
$$
\fr{\hat{R}(\hat{t}_F) - R(T_m)}{C_\rho \sqrt{2n}} \Rightarrow N(0, 1),
$$
\textit{где}
$$
C_\rho = \sigma^2\sqrt{1 +  \lim\limits_{n\to\infty} \fr{1}{n} \sum\limits_{j\neq i} \mathrm{corr}^2 (x_i, x_j)}.
$$

\noindent
Д\,о\,к\,а\,з\,а\,т\,е\,л\,ь\,с\,т\,в\,о\  \
 приводится для метода мягкой пороговой обработки; в~случае жесткой пороговой 
обработки доказательство аналогично. Обозначим
$$
U(T) = \hat{R}(T) -  \hat{R}(T_m) = \sum \limits_{i=1}^n H_i(T, T_m),
$$
где
$$
H_i(T, T_m) = F[x_i, T] - F[x_i, T_m].
$$
Имеем

\vspace*{-3pt}

\noindent
\begin{multline}
\label{D00}
\hat{R}(\hat{t}_F) - R(T_m) + \hat{R}(T_m) - \hat{R}(T_m) ={}\\
{}= \hat{R}(T_m) - 
R(T_m) + U(\hat{t}_F).
\end{multline}
Покажем, что
\begin{equation}
\label{D0}
\fr{\hat{R}(T_m) - R(T_m)}{C_\rho\sqrt{2n}} \Rightarrow N(0, 1).
\end{equation}


Повторяя рассуждения из~\cite{KuShe2016_1,KuShe2016_2,Jansen}, можно показать, 
что $T_m \hm\geq t_{\kappa_n}$. Учитывая также $T_m\hm \leq T_U$, имеем 
$$
C \sqrt{\ln n} \leq T_m \leq C^\prime \sqrt{\ln n}
$$ 
для некоторых положительных констант $C$ и~$C^\prime$.

\columnbreak

В случае мягкой пороговой обработки $\hat{R}(T_m)$ представляет собой 
несмещенную оценку~$R(T_m)$, а~при жесткой пороговой обработке и~выполнении 
условий теоремы смещение стремится к~нулю при делении на $\sqrt{n}$~\cite{Mallat}.

Для дисперсии числителя~(\ref{D0}) имеем:
\begin{multline*}
{\mathsf D} \left(\hat{R}(T_m) - R(T_m)\right) = \sum\limits_{i=1}^n {\mathsf D} F[x_i, T_m] + {}\\
{}+
\sum\limits_{i=1}^n\sum\limits_{\substack{j=1 \\  j\neq i}}^n \mathrm{cov}\left( F[x_i, T_m], F[x_j, 
T_m] \right).
\end{multline*}

Поскольку $\mu \in m_p[\eta_n]$,
\begin{equation}
\left.
\begin{array}{l}
 \displaystyle\sum\limits_{i: |\mu_i| > 1/T_1} {\mathsf D} F[x_i, T_m]  \leq{}\\
 \hspace*{15mm}{}\leq  4\left(\sigma^2 + T_m^2\right)^2 n \eta_n^p 
T_1^p = o(n);
\\[6pt]
\displaystyle \sum\limits_{\substack{{i,j: \max\{|\mu_i|, |\mu_j|\} > 1/T_1,}\\{j\neq i}}}  \hspace*{-12mm}\mathrm{cov}\,(F[x_i, 
T_m],F[x_j, T_m])  \leq{}\\
\hspace*{10mm}{}\leq 16\left(\sigma^2 + T_m^2\right)^2 n \eta_n^p T_1^p c_5 = o(n). 
\end{array}
\right\}    
\label{D2}
\end{equation}
Далее, учитывая что ${\mathsf D} x_i^2 \hm= 2\sigma^4 \hm+ 4\sigma^2 \mu_i^2$, нетрудно 
убедиться, что
\begin{multline}
\label{D3}
\sum\limits_{i: |\mu_i| \leq 1/T_1}\hspace*{-4mm} {\mathsf D} F[x_i, T_m] ={}\\
{}= \sum\limits_{i: |\mu_i| \leq 1/T_1} \hspace*{-4mm} {\mathsf D} 
x_i^2 + o(n) = 2\sigma^4 n + o(n).
\end{multline}


Введем обозначение 
$$
D_n = \left\{(i,j) : \max\left\{|\mu_i|, |\mu_j|\right\}  \leq \fr{1}{T_1}\,, \enskip j\hm\neq i\right\}.
$$
 Для суммы ковариаций аналогично~(\ref{D3}) получим
\begin{multline*}
\sum\limits_{(i,j)\in D_n} \hspace*{-2mm}\mathrm{cov}\left( F[x_i, T_m], F[x_j, T_m] \right) = {}\\
{}=
\sum\limits_{(i,j)\in D_n} \hspace*{-2mm}\mathrm{cov}\left( x_i^2, x_j^2 \right) + o(n).
\end{multline*}
Воспользуемся тождеством~\cite{Eroshenko}
$$
\mathrm{cov}\left (x_i^2, x_j^2\right) = 4 {\mathsf E} x_i {\mathsf E} x_j \mathrm{cov}\left(x_i, x_j\right) + 2 \mathrm{cov}^2 \left(x_i, x_j\right)
$$
для вектора $(x_i, x_j)$, имеющего двумерное нормальное распределение. Заметим, 
что
\begin{gather*}
 \sum\limits_{(i,j)\in D_n} 4 | {\mathsf E} x_i {\mathsf E} x_j \mathrm{cov}\left(x_i, x_j\right)| \leq 8 T_1^{-2} 
\sigma^2 n c_5 = o(n);
\\
\sum\limits_{(i,j)\in D_n} 2 \mathrm{cov}^2 (x_i, x_j)  = 2\sigma^4 \sum\limits_{(i,j)\in D_n} 
\mathrm{corr}^2 (x_i, x_j). 
\end{gather*}
Более того, поскольку  %< 4 \sigma^2 n c_5.$$
\begin{equation*}
\sum\limits_{\substack{{i,j: \max\{|\mu_i|, |\mu_j|\} > 1/T_1} \\ {j\neq i}}}
\hspace*{-10mm}\mathrm{corr}^2 (x_i, x_j)  
\leq  4 n \eta_n^p T_1^p c_5 =  o(n),
\end{equation*}
имеем
\begin{multline*}
\sum\limits_{(i,j)\in D_n} \mathrm{corr}^2 (x_i, x_j) ={}\\
{}= \sum\limits_{j\neq i} \mathrm{corr}^2 (x_i, x_j) 
+o(n)= c_6 n + o(n),
\end{multline*}
где
$$
c_6 = \lim\limits_{n\to\infty} \fr{1}{n} \sum\limits_{j\neq i} \mathrm{corr}^2 (x_i, x_j) 
\leq 2 c_5.
$$
Полагая $C_\rho \hm= \sigma^2\sqrt{1 + c_6}$, получим, наконец,
\begin{equation}
\label{D1}
{\mathsf D} \left(\hat{R}(T_m) - R(T_m)\right)  =  2 n C_\rho^2 + o(n).
\end{equation}
Заметим, что из~(\ref{D2}), (\ref{D3}) и~(\ref{D1}) следует, что
\begin{equation}
\label{D5}
\sup\limits_{n} \fr{\sum\nolimits_{i=1}^n {\mathsf D} F[x_i, T_m]}{V_n^2} < \infty\,,
\end{equation}
где 
$$
V_n^2 = {\mathsf D} \sum\limits_{i=1}^n \left(F[x_i, T_m] \hm- {\mathsf E} F[x_i, T_m]\right).
$$
Кроме того, поскольку $F[x_i, T_m]$ по модулю ограничены величиной $\sigma^2 \hm+ 
T_m^2$, выполнено условие Линдеберга: для любого $\eps\hm>0$ при $n \hm\to \infty$
\begin{multline}
\label{D6}
\!\!\!\fr{1}{V_n^2}\sum\limits_{i=1}^n {\mathsf E} \left( \!\left( F\left[x_i, T_m\right]\! -\! {\mathsf E} F\left[x_i, T_m\right]\right)^2 
\Ik \left(\vert F\left[x_i, T_m\right] -{}\right.\right.\hspace*{-2.69505pt}\\
\left.\left.{}- {\mathsf E} F\left[x_i, T_m\right]\vert >\eps V_n\right)\!
\vphantom{\left( F\left[x_i, T_m\right]\! -\! {\mathsf E} F\left[x_i, T_m\right]\right)^2}
\right) 
\to  0\,.
\end{multline}
Из~(\ref{D1})--(\ref{D6}), очевидного неравенства
$$ 
\lim\limits_{k\to\infty} \sup\limits_{n\geq k+1}\rho(k) \equiv 
\lim\limits_{k\to\infty} \rho^\star (k)  < 1
$$
 и~центральной предельной теоремы для сильно перемешанных случайных величин~\cite{Peligrad} следует~(\ref{D0}).

Перейдем к~доказательству того, что $U(\hat{t}_F) \, n^{-1/2} \overset{\, \p \, }{\to} 0$.
Всюду далее, не ограничивая общности, полагаем $\sigma=1$. 
Введем обозначения:

\noindent
\begin{align*}
S_1(T) &= \sum\limits_{i: |\mu_i| > 1/T_1} H_i(T, T_m); \\
S_2(T) &= \sum\limits_{i: |\mu_i| \leq 1/T_1} H_i(T, T_m); 
\\
N_1(a, b) &= \sum\limits_{i: |\mu_i| > 1/T_1} \Ik (a<|x_i|\leq b); \\ 
N_2(a, b) &= \sum\limits_{i: |\mu_i| \leq 1/T_1} \Ik (a<|x_i|\leq b);
\end{align*}

\noindent
\begin{align*}
Z_l(T) &= S_l(T) - {\mathsf E} S_l(T),\enskip l = 1,2\,; \\  
d_n &= \fr{T_U -  t_{\kappa_n}}{n};\\
T_j^{\prime} &= t_{\kappa_n}+j d_n,\enskip j = \overline{0,n-1}\,.
\end{align*} 

\vspace*{-3pt}

\noindent
Для произвольного $\eps>0$

\vspace*{-3pt}

\noindent
\begin{multline}
\p \left( \fr{|U(\hat{t}_F)|}{\sqrt{n}}> 4\eps \right) \leq 
\p\left(\hat{t}_F \leq t_{\kappa_n}\right) + {}\\
{}+\p \left(\fr{\sup\nolimits_{T\in 
[t_{\kappa_n}, T_U]} |U(T)|}{\sqrt{n}}>4\eps \right)\leq  {}\\
{}\leq \p\left(\hat{t}_F \leq t_{\kappa_n}\right) + \p\left(\fr{\sup\nolimits_{T\in 
[t_{\kappa_n}, T_U]} |{\mathsf E} U(T)|}{\sqrt{n}}>\eps\right)+{}\\
{}+ \p \left(\sup\limits_{T\in [t_{\kappa_n}, T_U]} |Z_1(T)| > 
\eps\sqrt{n}\right) +{}\\
{}+ \p \left(\sup\limits_{j \in [0, n-1]} |Z_2(T_j^{\prime})| > 
\eps\sqrt{n}\right) +{}\\
{}+ \p \left(\sup\limits_{\substack{j \in [0, n-1] \\
 T\in [T_j^{\prime},T_j^{\prime}+d_n]}} |Z_2(T)-Z_2(T_j^{\prime})| > \eps\sqrt{n}\right).
\label{M1}
\end{multline}
Заметим, что $\gamma_n\hm > \ln^{-1} n$, $\kappa_n\hm > n \eta_n^p \ln ^{-1} n \hm\geq 
n^{1-\delta_1} \ln ^{-1} n$ и~$q_n\hm > c_4 \ln ^{-1} n /2$ для всех достаточно 
больших~$n$.
Для первого слагаемого в~(\ref{M1}) по лемме~1 с~$m \hm= n^{\delta_1} \ln 
^7 n$ для  больших~$n$ имеем

\vspace*{-3pt}

\noindent
\begin{multline}
\label{M1next}
\p\left(\hat{t}_F \leq t_{\kappa_n}\right)  = \p \left(\hat{k}_F \geq \kappa_n 
\right) \leq 4 n e^{-\ln^2 n} + {}\\
{}+n^{1+(3/2)\,\delta_1} \ln^9 n \, 
\alpha\left(\left[\fr{n^{1-\delta_1}}{\ln^{7} n}\right]\right) = o(1)
\end{multline}
при $n\to\infty$. 
Для оценки второго слагаемого в~(\ref{M1}) заметим, что при $T \hm\in 
[t_{\kappa_n}, T_U]$ справедливо
\begin{equation}
\label{M2}
{\mathsf E} H_i(T, T_m) \leq T_U^2 + 1.
\end{equation}
Если же кроме $T \hm\in [t_{\kappa_n}, T_U]$ также выполнено $|\mu_i| \hm\leq T_1^{-1}$, то

\vspace*{-6pt}

\noindent
\begin{multline*}
|{\mathsf E} H_i (T, T_m)| \leq 2 T_U^2 \, \p \left(|x_i| > t_{\kappa_n}\right) \leq {}\\
{}\leq2 
T_U^2 \, \p \left(|x_i-\mu_i| > t_{\kappa_n}-T_1^{-1}\right) \leq{}\\
{}\leq 2 T_U^2  \exp\left\{ -\fr{1}{2} \left(t_{\kappa_n} - T_1^{-
1}\right)^2 \right\}  \leq{}\\
{}\leq
 4 (\ln n)  \exp\left\{ -\fr{1}{2} 
\left(z\left(\fr{q_n\kappa_n}{2n}\right)\right)^2 + t_{\kappa_n} T_1^{-
1}\right\},
\end{multline*}

\vspace*{-2pt}

\noindent
где использовано неравенство 

\noindent
$$
2(1-\Phi(x))\hm \leq \fr{e^{-x^2/2}}{x}
$$

\pagebreak


\noindent
 для $x\hm\geq 0$ 
($\Phi(x)$~--- функция распределения $N(0,1)$). Рас\-смот\-рим выражение 
в~экспоненте. Второе слагаемое не превышает $1\hm+o(1)$ при $n\hm\to\infty$, поскольку 
$t_{\kappa_n} \hm\leq T_1 (1+o(1))$ при $\sigma\hm=1$, что нетрудно получить из 
определения~$t_{\kappa_n}$, пред\-став\-ле\-ния~(\ref{lem1eq2}) и~ограничения на~$q_n$ 
из формулировки тео\-ре\-мы. Для первого слагаемого, используя пред\-став\-ле\-ние~(\ref{lem1eq1}) 
и~ограничения, наложенные на~$q_n$, при больших~$n$ получим
\begin{multline*}
-\fr{1}{2}\left(z\left(\fr{q_n \kappa_n}{2n}\right)\right)^2 \leq - \ln 
\fr{2n (1-q_n-\gamma_n)}{q_n n \eta_n^p T_1^{-p}} + {}\\
{}+\fr{1}{2} \ln 
\left((1+o(1)) \ln \eta_n^{-p}\right) + \fr{3}{2} \leq{}\\
{}\leq \ln \fr{c_3}{1-c_3} + \ln \eta_n^p + \ln T_1^{-p} + \ln T_1 + 
\fr{3}{2}+ o(1).
\end{multline*}
Из приведенных соотношений следует, что с~некоторой константой $c_7 = c_7(c_3, 
p, \delta_1, \delta_2, c_4)$
\begin{equation}\label{M3}
\sup\limits_{\substack{i: |\mu_i| \leq 1/T_1 \\ T\in [t_{\kappa_n}, T_U]}} |{\mathsf E} 
H_i (T, T_m)|  \leq c_7 (\ln n)^{(3-p)/2}\eta_n^p.
\end{equation}
Из (\ref{M2}) и~(\ref{M3}) с~учетом $\delta_2 \hm> 1/2$ следует
\begin{multline*}
\sup\limits_{T\in [t_{\kappa_n}, T_U]} |{\mathsf E} U(T)| \leq{}\\
{}\leq 
 n\eta_n^p T_1^p 
(T_U^2+1) + c_7 (\ln n)^{(3-p)/2} n \eta_n^p = o(\sqrt{n})
\end{multline*}
при $n\to\infty$, а следовательно, для любого $\eps\hm>0$ второе слагаемое в~(\ref{M1}) обращается в~ноль для всех достаточно больших~$n$.

Далее, поскольку при $T \hm\leq T_U$ и~$\sigma\hm=1$
$$
|H_i(T, T_m) - {\mathsf E} H_i(T, T_m)| \leq 2 (T_U^2 +2), \enskip i=\overline{1, n}\,,
$$
а число слагаемых в~$Z_1(T)$ не превосходит $n\eta_n^p T_1^p$, имеем
$$
\sup\limits_{T\in [t_{\kappa_n}, T_U]} |Z_1(T)|  \leq 2 n\eta_n^p T_1^p (T_U^2 
+2) = o(\sqrt{n})
$$
при $n\to\infty$, а следовательно, для любого $\eps\hm>0$ и~третье слагаемое в~(\ref{M1}) обращается в~ноль для всех достаточно больших~$n$.

Перейдем к~оценке четвертого слагаемого в~(\ref{M1}). Аналогично~(\ref{M3}) 
можно получить:
\begin{multline}
\label{M10}
\!\!\sup\limits_{\substack{i: |\mu_i| \leq 1/T_1 \\ T\in [t_{\kappa_n}, T_U]}} \!{\mathsf D} 
H_i (T, T_m)  \leq \!\sup\limits_{\substack{i: |\mu_i| \leq 1/T_1 \\ T\in 
[t_{\kappa_n}, T_U]}} \!{\mathsf E} \left(H_i (T, T_m)\right)^2  \leq{}\\
{}\leq 2 c_7 (\ln n)^{(5-p)/2} \eta_n^p.
\end{multline}
По лемме~4 с~$m \hm= \sqrt{n} (\ln n)^3$ и~$k \hm= n-[n\eta_n^p T_1^p]$ 
для четвертого слагаемого в~(\ref{M1}) имеем:

\noindent
\begin{multline}
\p \left(\sup\limits_{j \in [0, n-1]} |Z_2(T_j^\prime)| > \eps\sqrt{n}\right) 
\leq {}\\
{}\leq \sum\limits_{j \in [0, n-1]} \hspace*{-3mm}\p \left( |Z_2(T_j^\prime)| > \varepsilon\sqrt{n}\right)\leq{}\\
{}\leq 4 n \exp \left\{ - \fr{\eps^2 n^{3/2} (\ln n)^3}{n-[n\eta_n^p T_1^p]}\!\Bigg/\! \big( 8 (T_U^2+2)\eps\sqrt{n} +{}\right.\\
\left.{}+ 128 c_7 (\ln n)^{(5-p)/2} \eta_n^p  (n-
[n\eta_n^p T_1^p])\big) 
\vphantom{ \fr{\eps^2 n^{3/2} (\ln n)^3}{n-[n\eta_n^p T_1^p]}}
\right\} +{}\\
{}
+ 22 \left(1+\fr{8(T_U^2+2) (n-[n\eta_n^p T_1^p])}{\eps 
\sqrt{n}}\right)^{1/2}\times{}\\
{}\times n^{3/2} (\ln n)^3 \alpha\left(\left[\fr{n-[n\eta_n^p 
T_1^p]}{2 (\ln n)^3 \sqrt{n}}\right]\right).
\label{M5}
\end{multline}
Используя ограничения $n^{-\delta_1}\hm\leq \eta_n^p \leq n^{-\delta_2}$ 
и~$1/2\hm<\delta_2\hm<\delta_1\hm<1$, из~(\ref{M5}) получим для любого $\eps\hm>0$
$$
\p \left(\sup\limits_{j \in [0, n-1]} |Z_2(T_j^\prime)| > \eps\sqrt{n}\right) 
\to 0
$$
при $n \to \infty$.

Рассмотрим, наконец, пятое слагаемое в~(\ref{M1})). Заметим, что при $0\hm< a \hm< b$ 
справедливо
$$
|Z_2(b)-Z_2(a)| \leq 2 |N_2(a,b)-{\mathsf E} N_2(a,b)| + n (b^2-a^2).
$$
Полагая $a = T_j^\prime$, $b \hm= T \hm\in [T_j^\prime, T_j^\prime+d_n]$ для 
произвольного $j \hm\in [0, n-1]$ и~учитывая, что
$$
(T^2 - (T_j^\prime )^2) = (T - T_j^\prime)(T+ T_j^\prime ) \leq  2 d_n T_U < 2 
T_U^2 n^{-1}; 
$$

\vspace*{-12pt}

\noindent
\begin{multline*}
\p\left(T_j^\prime < |x_i| \leq T \right) \leq \p\left(T_j^\prime < |x_i| \leq 
T_j^\prime+d_n\right) <{}\\
{}< d_n < T_U n^{-1}, 
\end{multline*}
получим  оценку
$$
|Z_2(T)-Z_2(T_j^\prime)| \leq 2 N_2(T_j^\prime, T) +  3 T_U^2 .
$$
Далее, поскольку $N_2 (T_j^\prime, T) \hm\leq N_2 (T_j^\prime, T_j^\prime+d_n)$ и~${\mathsf E} N_2 (T_j^\prime, T_j^\prime+d_n) \hm< T_U^2$,
имеем
\begin{multline*}
\sup\limits_{T \in [T_j^\prime, T_j^\prime+d_n]} |Z_2(T)-Z_2(T_j^\prime)| \leq {}\\
{}\leq
2 \left|N_2 (T_j^\prime, T_j^\prime+d_n) - {\mathsf E} N_2 (T_j^\prime, 
T_j^\prime+d_n)\right| +  5 T_U^2 .
\end{multline*}
Аналогично~(\ref{M3}) показывается, что
\begin{multline}
\label{M11}
\sup\limits_{\substack{i : |\mu_i| \leq 1/T_1 \\ j \in [0, n-1]}} {\mathsf D} \Ik 
(T_j^\prime < |x_i| \leq T_j^\prime + d_n) <{}\\
{}< c_7 (\ln n)^{(1-p)/2} \eta_n^p.
\end{multline}
Пусть $n > N(\eps)$ настолько, что 
$$
\fr{\eps\sqrt{n} - 5 T_U^2}{2} > \fr{\eps \sqrt{n} }{4}\,.
$$
%
 Тогда для пятого слагаемого в~(\ref{M1}) по лемме~4 с~$m \hm= 
\sqrt{n} (\ln n)^2$ и~$k \hm= n\hm-[n\eta_n^p T_1^p]$ имеем
\begin{multline}
\p \left(\sup\limits_{\substack{j \in [0, n-1] \\ T\in 
[T_j^{\prime},T_j^{\prime}+d_n]}} |Z_2(T)-Z_2(T_j^{\prime})| > 
\eps\sqrt{n}\right) \leq{}\\
{}\leq  \sum\limits_{j \in [0, n-1]} \p \left(  \left|N_2 (T_j^\prime, 
T_j^\prime+d_n) -{}\right.\right.\\
\left.\left.{}- {\mathsf E} N_2 (T_j^\prime, T_j^\prime+d_n)\right| > \fr{\eps\sqrt{n}}{4} 
\right) \leq{}\\
{}\leq  4n \exp \left\{ -  \fr{\eps^2 n^{3/2} (\ln n)^2}{(n-[n\eta_n^p T_1^p])^{-1}}\Bigg/ 
\big( 16 \eps \sqrt{n} +{}\right.\\
\left.{}+ 64 c_7 (\ln n)^{(1-p)/2} \eta_n^p (n-[n\eta_n^p 
T_1^p]) \big) 
\vphantom{\fr{\eps^2 n^{3/2} (\ln n)^2}{(n-[n\eta_n^p T_1^p])^{-1}}}
\right\} +{}\\
{}+ 22 \left(1+\fr{16 (n-[n\eta_n^p T_1^p])}{\eps \sqrt{n}}\right)^{1/2}\times{}\\
{}\times 
n^{3/2} (\ln n)^2 \alpha\left(\left[\fr{n-[n\eta_n^p T_1^p]}{2 (\ln n)^2 
\sqrt{n}}\right]\right).
\label{M6}
\end{multline}
Используя ограничения $n^{-\delta_1}\hm\leq \eta_n^p\hm \leq n^{-\delta_2}$ 
и~$1/2\hm<\delta_2\hm<\delta_1<1$, из~(\ref{M6}) получим для любого $\eps\hm>0$
$$
\p \left(\sup\limits_{\substack{j \in [0, n-1] \\ T\in 
[T_j^{\prime},T_j^{\prime}+d_n]}} |Z_2(T)-Z_2(T_j^{\prime})| > 
\eps\sqrt{n}\right) \to 0
$$
при $n \to \infty$.

Таким образом, показано, что для любого $\eps>0$ все слагаемые в~(\ref{M1}) 
стремятся к~нулю при $n\to\infty$. Следовательно,
$$
\fr{|U(\hat{t}_F)|}{\sqrt{n}}  \overset{\, \p \, }{\to} 0 \,,
$$
что вместе с~(\ref{D0}) завершает доказательство тео\-ремы.~\hfill$\square$

\smallskip

Следующая теорема дает достаточные условия для асимптотической нормальности 
оценки риска $\hat{R}(\hat{t}_F)$ в~случае $\mu \hm\in l_0[\eta_n]$.

\smallskip

\noindent
\textbf{Теорема 2.}\ 
\textit{Пусть $\mu \hm\in l_0[\eta_n]$, $\eta_n\hm\in[n^{-\delta_1}, n^{-\delta_2}]$, $1/2\hm < 
\delta_2\hm < \delta_1\hm<1;$ имеются такие константы $c_1, c_2\hm>0$, что для 
коэффициента сильного перемешивания $\alpha(\cdot)$ компонент вектора~$x$ 
справедливо} 
\begin{gather*}
\alpha(k) \leq c_1 k^{-1-(5/2)\delta_1/(1\hm-\delta_1)\hm-c_2},\enskip 
k=\overline{1,n-1};\\
 q_n < c_3 < 1;\enskip \mathrm{lim\,inf} q_n \ln n = c_4 > 0;
\end{gather*}
\textit{для максимального коэффициента корреляции~$\rho(\cdot)$ компонент вектора~$x$ 
справедливо}
$$
\sum\limits_{k = 1}^{\infty} \sup\limits_{n\geq k+1} \rho(k) \equiv 
\sum\limits_{k = 1}^{\infty}  \rho^\star (k) = c_5 < \infty. 
$$
\textit{Тогда при $n \to \infty$}
$$
\fr{\hat{R}(\hat{t}_F) - R(T_m)}{C_\rho \sqrt{2n}} \Rightarrow N(0, 1),
$$
\textit{где}
$$
C_\rho = \sigma^2\sqrt{1 +   \lim\limits_{n\to\infty} \fr{1}{n} 
\sum\limits_{j\neq i} \mathrm{corr}^2 (x_i, x_j)}\,.
$$

\noindent
Д\,о\,к\,а\,з\,а\,т\,е\,л\,ь\,с\,т\,в\,о\  проводится аналогично доказательству теоремы~1. 
Переменная~$D_n$ теперь определяется как $D_n \hm= \{(i,j) : 
|\mu_i|\hm=|\mu_j|=0$, $j\hm\neq i\}$. Условия вида $|\mu_i|\hm<T_1^{-1}$ (вида 
$|\mu_i|\hm\geq T_1^{-1}$) заменяются условиями  $\mu_i\hm=0$ (соответственно 
$|\mu_i|\hm>0$).
Поскольку $\mu \hm\in l_0[\eta_n]$, количество~$i$ таких, что $|\mu_i|\hm>0$ 
(а~значит, и~число слагаемых в~$Z_1(T)$), не превышает~$[n \eta_n]$.

Для оценки первого слагаемого в~(\ref{M1}) используется лемма~2, 
в~которой можно взять, например, $b\hm=1/2$, а~для~$\kappa_n^0$ использовать оценку 
$\kappa_n^0 \hm> n \eta_n$. Формулы (\ref{M3}),  (\ref{M10}) и~(\ref{M11}) 
принимают вид соответственно
\begin{align*}
\sup\limits_{\substack{i: \mu_i =0 \\ T\in [t_{\kappa_n^0}, T_U]}} |{\mathsf E} H_i (T, 
T_m)| & \leq c_8 (\ln n)^{3/2} \eta_n ;
\\
\sup\limits_{\substack{i: \mu_i =0 \\ T\in [t_{\kappa_n^0}, T_U]}} {\mathsf D} H_i (T, 
T_m)  & \leq 2 c_8 (\ln n)^{5/2} \eta_n;
\\
\sup\limits_{\substack{i : \mu_i =0 \\ j \in [0, n-1]}} {\mathsf D} \Ik (T_j^\prime < 
|x_i| \leq T_j^\prime + d_n) &< c_8 (\ln n)^{1/2} \eta_n,
\end{align*}
где $c_8 = c_8(c_3,\delta_1, \delta_2, c_4)$. В~остальном доказательство 
аналогично.~\hfill$\square$

\section{Сильная состоятельность оценки риска при~применении FDR-процедуры 
в~условиях слабой зависимости}

Следующая теорема дает достаточные условия для сильной состоятельности оценки 
риска $\hat{R}(\hat{t}_F)$ в~случаях $\mu \hm\in m_p[\eta_n]$ и~$\mu \hm\in 
l_0[\eta_n]$.

\smallskip

\noindent
\textbf{Теорема 3.}
\textit{Пусть $\mu\hm \in m_p[\eta_n]$, $\eta_n^p\hm\in[n^{-\delta_1}, n^{-\delta_2}]$ либо 
$\mu \hm\in l_0[\eta_n]$, $\eta_n\hm\in[n^{-\delta_1}, n^{-\delta_2}]$; $0 \hm< \delta_2 
\hm< \delta_1<1$; имеются такие константы $c_1, c_2\hm>0$, что для коэффициента 
сильного перемешивания $\alpha(\cdot)$ компонент вектора~$x$ справедливо}  
$\alpha(k) \hm\leq c_1 k^{-2-(7/2)\delta_1/(1\hm-\delta_1)\hm-c_2}$, $k\hm=\overline{1,n-1}$; 
$q_n \hm< c_3 \hm< 1$; $\mathrm{lim\,inf} q_n \ln n \hm= c_4 \hm> 0$. \textit{Тогда при} $n \hm\to \infty$
$$
\fr{\hat{R}(\hat{t}_F) - R(T_m)}{n} \rightarrow 0 \, \, \,\textit{п.~в.}
$$


\noindent
Д\,о\,к\,а\,з\,а\,т\,е\,л\,ь\,с\,т\,в\,о\,.  Воспользуемся представлением~(\ref{D00}).

Покажем, что $(\hat{R}(T_m)-R(T_m))n^{-1}\hm \to 0$ п.~в.\ при $n\hm\to\infty$. 
При мягкой пороговой обработке ${\mathsf E} \hat{R}(T_m) \hm= R(T_m)$, а~при жесткой 
пороговой обработке
\begin{multline*}
\fr{\hat{R}(T_m)-R(T_m)}{n} = {}\\
{}=\fr{\hat{R}(T_m)-{\mathsf E} \hat{R}(T_m)}{n} 
+\fr{{\mathsf E}\hat{R}(T_m)-R(T_m)}{n}\,,
\end{multline*}
где второе слагаемое стремится к~нулю при $n\to\infty$ \cite{Mallat}. 
Следовательно, достаточно показать, что $(\hat{R}(T_m)\hm-{\mathsf E}\hat{R}(T_m))n^{-1} \hm\to 0$ п.~в.

Полагая в~лемме~3 $X_i \hm= F[x_i, T_m] \hm- {\mathsf E} F[x_i, T_m]$, $b \hm= 
2(\sigma^2\hm+T_m^2)$ и~$m \hm= n^{1/4}$ и~учитывая ограничения на $\alpha(\cdot)$ из 
условия, нетрудно убедиться, что для всех~$n$
$$
\p \left(\left| \fr{\hat{R}(T_m)-{\mathsf E} \hat{R}(T_m)}{n}\right| >\eps \right) 
\leq \fr{c_5}{n^{1+c_6}}\,, 
$$
где константы $c_5$, $c_6$ положительны. Отсюда
$$
\sum\limits_{n=1}^{\infty}\p \left(\left|\fr{\hat{R}(T_m)-{\mathsf E} 
\hat{R}(T_m)}{n}\right| >\eps \right) < \infty,
$$
и по теореме~1.3.4 из~\cite{Serfling2002} 
$$
\left(\hat{R}(T_m)-{\mathsf E}\hat{R}(T_m)\right)n^{-1} \to 0~\mbox{п.~в.}
$$



Покажем теперь, что  $U(\hat{t}_F) \, n^{-1}\hm \to 0$ п.~в. Доказательство 
проведено для $\mu \hm\in m_p[\eta_n]$, в~случае $\mu\hm \in l_0[\eta_n]$ 
доказательство аналогично.
Аналогично формуле~(\ref{M1}), для произвольного $\eps\hm>0$ в~терминах тео\-ре\-мы~1 имеем
\begin{multline*}
\p \left( \fr{|U(\hat{t}_F)|}{n}> 4\eps \right) \leq \p\left(\hat{t}_F 
\leq t_{\kappa_n}\right) +{}\\
{}+ \p\left(\fr{\sup\nolimits_{T\in [t_{\kappa_n}, T_U]} |{\mathsf E} 
U(T)|}{n}>\eps\right)+{}\\
{}+ \p \left(\sup\limits_{T\in [t_{\kappa_n}, T_U]} |Z_1(T)| > \eps n\right) +{}
\end{multline*}

\noindent
\begin{multline}
{}+ \p  \left(\sup\limits_{j \in [0, n-1]} |Z_2(T_j^{\prime})| > \eps n\right) +{}\\
{}+ \p \left(\sup\limits_{\substack{j \in [0, n-1] \\ T\in 
[T_j^{\prime},T_j^{\prime}+d_n]}} |Z_2(T)-Z_2(T_j^{\prime})| > \eps n\right).
\label{M1SC}
\end{multline}
Применяя рассуждения, аналогичные приведенным в~доказательстве теоремы~1, можно показать, что
$$
\sup\limits_{T\in [t_{\kappa_n}, T_U]} |{\mathsf E} U(T)| = o(n); \enskip
\sup\limits_{T\in [t_{\kappa_n}, T_U]} |Z_1(T)|  = o(n),
$$
откуда следует, что второе и~третье слагаемые в~(\ref{M1SC}) обращаются в~ноль 
для всех достаточно больших~$n$.

Для некоторых положительных констант  $c_7$ и~$c_8$ первое, четвертое и~пятое 
слагаемые  в~(\ref{M1SC}) не превышают $c_7 n^{-1-c_8}$ для всех достаточно 
боль\-ших~$n$, что можно показать с~помощью ограничения на $\alpha(\cdot)$ из 
условия и~рассуждений, аналогичных приведенным при выводе соответственно формул~(\ref{M1next}), (\ref{M5}) и~(\ref{M6}), с~тем отличием, что при применении 
леммы~4 полагается $m \hm= (\ln n)^3$.

Из доказанного следует, что
$$
\sum\limits_{n=1}^{\infty}\p \left( \fr{|U(\hat{t}_F)|}{n}> 4\eps \right) 
< \infty,
$$
и по теореме~1.3.4 из~\cite{Serfling2002} $U(\hat{t}_F) \, n^{-1} \to 0$ п.~в., 
что завершает доказательство теоремы.~\hfill$\square$



{\small\frenchspacing
 {\baselineskip=11.5pt
 %\addcontentsline{toc}{section}{References}
 \begin{thebibliography}{99}
\bibitem{FDRImage}
\Au{Krylov V.\,A., Moser~G., Serpico~S.\,B., Zerubia~J.}
False discovery rate approach to unsupervised image change detection~// IEEE 
T. Image Process., 2016. Vol.~25. No.\,10. P.~4704--4718. doi: 10.1109/TIP.2016.2593340.

\bibitem{MultipleTesting} %2
\Au{Menyhart~O., Weltz~B., Gyorffy~B.}
MultipleTesting.com: A~tool for life science researchers for multiple hypothesis 
testing correction~// PLoS One, 2021. Vol.~16. No.\,6. Art.~0245824. doi: 10.1371/journal.pone.0245824.

\bibitem{AdaptingFDR} %3
\Au{Abramovich~F., Benjamini~Y., Donoho~D., Johnstone~I.}
Adapting to unknown sparsity by controlling the false discovery rate~// Ann. Stat., 2006. Vol.~34. No.\,2. P.~584--653.
doi: 10.1214/009053606000000074.

\bibitem{ZasShe17} %4
\Au{Заспа~А.\,Ю., Шестаков~О.\,В.}
Состоятельность оценки риска при множественной проверке гипотез с~FDR-по\-ро\-гом~// 
Вестник ТвГУ. Сер. Прикладная математика, 2017. Вып.~1. С.~5--16.
doi: 10.26456/vtpmk119. EDN: YFYJXT.

\bibitem{Mathematics2020} %5
\Au{Palionnaya~S.\,I., Shestakov~O.\,V.}
Asymptotic properties of MSE estimate for the false discovery rate controlling 
procedures in multiple hypothesis testing // Mathematics, 2020. Vol.~8. No.~11. 
Art.~1913. 11~p. doi: 10.3390/ math8111913.

\bibitem{Shestakov2021-1} %6
\Au{Шестаков~О.\,В.}
Анализ несмещенной оценки среднеквадратичного риска метода блочной пороговой 
обработки~// Информатика и~её применения, 2021. Т.~15. Вып.~2. С.~30--35.
doi: 10.14357/19922264210205. EDN: DSQQAU.

\bibitem{Shestakov2021-2} %7
\Au{Шестаков~О.\,В.}
Пороговые функции в~методах подавления шума, основанных на вейв\-лет-раз\-ло\-же\-нии 
сигнала~// Информатика и~её применения, 2021. Т.~15. Вып.~3. С.~51--56.
doi: 10.14357/19922264210307. EDN: WSEAYG.

\bibitem{Shestakov2022} %8
\Au{Шестаков~О.\,В.}
Несмещенная оценка риска пороговой обработки с~двумя пороговыми значениями~// 
Информатика и~её применения, 2022. Т.~16. Вып.~4. С.~14--19.
doi: 10.14357/19922264220403. EDN: \mbox{DZBVLC}.

\bibitem{ResultsOnFDRUnderDependence} %9
\Au{Farcomeni~A.}
Some results on the control of the false discovery rate under dependence~// 
Scand. J. Stat., 2007. Vol.~34. No.\,2. P.~275--297.
doi: 10.1111/j.1467-9469.2006.00530.x.

\bibitem{VorontsovShestakov2023} %10
\Au{Воронцов~М.\,О., Шестаков~О.\,В.}
Среднеквадратичный риск FDR-про\-це\-ду\-ры в~условиях слабой за\-ви\-си\-мости~// 
Информатика и~её применения, 2023. Т.~17. Вып.~2. С.~34--40.
doi: 10.14357/19922264230205. EDN: AVJZDX.

\bibitem{Vorontsov2024} %11
\Au{Воронцов~М.\,О.}
Анализ среднеквадратичного риска при использовании методов множественной 
проверки гипотез для выбора параметров пороговой обработки в~условиях слабой 
зависимости~// Вестник Московского университета. Сер. 15: Вычислительная 
математика и~кибернетика, 2024. №\,2. С.~18--24.

\bibitem{Bosq} %12
\Au{Bosq~D.}
Nonparametric statistics for stochastic processes: Estimation and prediction.~--- 
Lecture notes in statistics ser.~--- New York, NY, USA: Springer, 1996. Vol.~110. 
188~p.

\bibitem{Mallat} %13
\Au{Mallat~S.}
A wavelet tour of signal processing.~--- New York, NY, USA: Academic Press, 1999. 
857~p.

\bibitem{spatialAdaptation} %14
\Au{Donoho~D., Johnstone~I.}
Ideal spatial adaptation via wavelet shrinkage~// Biometrika, 1994. Vol.~81. 
No.\,3. P.~425--455. doi: 10.1093/biomet/81.3.425.

\bibitem{AdaptingSURE} %15
\Au{Donoho D., Johnstone I.\,M.}
Adapting to unknown smoothness via wavelet shrinkage~// J.~Amer. Stat. Assoc., 
1995. Vol.~90. P.~1200--1224.

\bibitem{ExactRisk} %16
\Au{Marron J.\,S., Adak~S., Johnstone~I.\,M., Neumann~M.\,H., Patil~P.}
Exact risk analysis of wavelet regression~// J.~Comput. Graph. Stat., 1998. 
Vol.~7. P.~278--309. doi: 10.1080/ 10618600.1998.10474777.

\bibitem{Jansen} %17
\Au{Jansen~M.}
Noise reduction by wavelet thresholding.~-- Lecture notes in statistics ser.~--- 
New York, NY, USA: Springer, 2001. Vol.~161. 217~p.

\bibitem{KuShe2016_1} %18
\Au{Кудрявцев~А.\,А., Шестаков~О.\,В.}
Асимптотическое поведение порога, минимизирующего усредненную\linebreak вероятность ошибки 
вычисления вейв\-лет-ко\-эф\-фи\-ци\-ен\-тов~// Докл. Акад. наук, 2016. Т.~468. №\,5. 
С.~487--491.

\bibitem{KuShe2016_2} %19
\Au{Кудрявцев~А.\,А., Шестаков~О.\,В.}
Асимптотически оптимальная пороговая обработка вейв\-лет-ко\-эф\-фи\-ци\-ен\-тов в~моделях с~негауссовым распределением шума~// Докл. Акад. наук, 2016. Т.~471. №\,1. 
С.~11--15.



\bibitem{Eroshenko} %20
\Au{Ерошенко~А.\,А.}
Статистические свойства оценок сигналов и~изображений при пороговой обработке 
коэффициентов в~вейв\-лет-раз\-ло\-же\-ни\-ях: Дис.\ \ldots\ канд. физ.-мат. наук.~--- 
М.: МГУ, 2015. 82~с.

\bibitem{Peligrad} %21
\Au{Peligrad~M.}
On the asymptotic normality of sequences of weak dependent random variables~// 
J. Theor. Probab., 1996. Vol.~9. No.\,3. P.~703--715. doi: 10.1007/BF02214083.

\bibitem{Serfling2002} %22
\Au{Serfling~R.\,J.}
Approximation theorems of mathematical statistics.~--- New York, NY, USA: John Wiley \&~Sons, Inc., 2002. 371~p.

\end{thebibliography}

 }
 }

\end{multicols}

\vspace*{-6pt}

\hfill{\small\textit{Поступила в~редакцию 21.05.24}}

\vspace*{8pt}

%\pagebreak

%\newpage

%\vspace*{-28pt}

\hrule

\vspace*{2pt}

\hrule



\def\tit{ASYMPTOTIC NORMALITY AND STRONG CONSISTENCY\\ OF~RISK ESTIMATE WHEN USING THE~FDR THRESHOLD\\ UNDER WEAK DEPENDENCE CONDITION}


\def\titkol{Asymptotic normality and strong consistency of~risk estimate when using the~FDR threshold under weak dependence condition}


\def\aut{M.\,O.~Vorontsov$^{1,2}$ and~O.\,V.~Shestakov$^{1,2,3}$}

\def\autkol{M.\,O.~Vorontsov and~O.\,V.~Shestakov}

\titel{\tit}{\aut}{\autkol}{\titkol}

\vspace*{-13pt}


\noindent
$^{1}$Department of Mathematical Statistics, Faculty of Computational Mathematics and Cybernetics,
 M.\,V.~Lo\-mo-\linebreak
 $\hphantom{^1}$nosov Moscow State University, 1-52~Leninskie Gory, GSP-1, Moscow 119991, Russian Federation

\noindent
$^{2}$Moscow Center for Fundamental and Applied Mathematics, M.\,V.~Lomonosov Moscow State University,\linebreak
$\hphantom{^1}$1~Leninskie Gory, GSP-1, Moscow 119991, Russian Federation

\noindent
$^{3}$Federal Research Center ``Computer Science and Control'' of the Russian Academy of Sciences, 44-2~Vavilov\linebreak
$\hphantom{^1}$Str., Moscow 119333, Russian Federation


\def\leftfootline{\small{\textbf{\thepage}
\hfill INFORMATIKA I EE PRIMENENIYA~--- INFORMATICS AND
APPLICATIONS\ \ \ 2024\ \ \ volume~18\ \ \ issue\ 3}
}%
 \def\rightfootline{\small{INFORMATIKA I EE PRIMENENIYA~---
INFORMATICS AND APPLICATIONS\ \ \ 2024\ \ \ volume~18\ \ \ issue\ 3
\hfill \textbf{\thepage}}}

\vspace*{2pt}






\Abste{An approach to solving the problem of noise removal in a large array of sparse data is considered
 based on the method of controlling the average proportion of false hypothesis rejections (False Discovery Rate, FDR). 
 This approach is equivalent to threshold processing procedures that remove array components whose values do not exceed 
 some specified threshold. The observations in the model are considered weakly dependent. To control the\linebreak\vspace*{-12pt}}
 
 \Abstend{degree of dependence, 
 restrictions on the strong mixing coefficient and the maximum correlation coefficient are used. The mean-square risk is 
 used as a measure of the effectiveness of the considered approach. It is possible to calculate the risk value only on the test data;
  therefore, its statistical estimate is considered in the work and its properties are investigated. The asymptotic normality and
   strong consistency of the risk estimate are proved when using the FDR threshold under conditions of weak dependence in the data.}

\KWE{thresholding; multiple hypothesis testing; risk estimate}

\DOI{10.14357/19922264240309}{ZOQVTO}

%\vspace*{-12pt}


    
   %   \Ack

%\vspace*{-3pt}
%\noindent



  \begin{multicols}{2}

\renewcommand{\bibname}{\protect\rmfamily References}
%\renewcommand{\bibname}{\large\protect\rm References}

{\small\frenchspacing
 {\baselineskip=10.8pt
 \addcontentsline{toc}{section}{References}
 \begin{thebibliography}{99} 

%1
\bibitem{FDRImage-1}
\Aue{Krylov, V.\,A., G.~Moser, S.\,B.~Serpico, and J.~Zerubia.} 2016. 
False discovery rate approach to unsupervised image change detection. 
\textit{IEEE T. Image Process.} 25(10):4704--4718. doi: 10.1109/TIP.2016.2593340.

%2
\bibitem{MultipleTesting-1}
\Aue{Menyhart, O., B.~Weltz, and B.~Gyorffy.} 2021. 
MultipleTesting.com: A~tool for life science researchers for multiple hypothesis testing correction. 
\textit{PLoS One} 16(6):0245824. 
doi: 10.1371/journal.pone.0245824.

%3
\bibitem{AdaptingFDR-1}
\Aue{Abramovich, F., Y.~Benjamini, D.~Donoho, and I.\,M.~Johnstone.} 2006. 
Adapting to unknown sparsity by controlling the false discovery rate. 
\textit{Ann. Stat.} 34(2):584--653. 
doi: 10.1214/009053606000000074.


%4
\bibitem{ZasShe17-1}
\Aue{Zaspa, A.\,Yu., and O.\,V.~Shestakov.} 2017.
Sostoyatel'nost' otsenki riska pri mnozhestvennoy proverke gipotez s~FDR-porogom
 [Consistency of the risk estimate of the multiple hypothesis testing with the FDR threshold]. 
\textit{Vestnik TvGU. Ser.: Prikladnaya matematika} [Herald of Tver State University. Ser. Applied Mathematics] 1:5--16.
doi: 10.26456/vtpmk119. EDN: YFYJXT.

%5
\bibitem{Mathematics2020-1}
\Aue{Palionnaya, S.\,I., and O.\,V.~Shestakov.} 2020. 
Asymptotic properties of MSE estimate for the false discovery rate controlling procedures in multiple hypothesis testing. 
\textit{Mathematics} 8(11):1913. 11~p.
doi: 10.3390/math8111913.

%6
\bibitem{Shestakov2021-1-1}
\Aue{Shestakov, O.\,V.} 2021.
Analiz nesmeshchennoy otsenki srednekvadratichnogo riska metoda blochnoy po\-ro\-go\-voy obrabotki 
[Analysis of the unbiased mean-square risk estimate of the block thresholding method]. 
\textit{Informatika i~ee Primeneniya~--- Inform. Appl.} 15(2):30--35.
doi: 10.14357/19922264210205. EDN: DSQQAU.

%7
\bibitem{Shestakov2021-2-1}
\Aue{Shestakov, O.\,V.} 2021.
Porogovye funktsii v~metodakh podavleniya shuma, osnovannykh na veyvlet-razlozhenii signala 
[Thresholding functions in the noise suppression methods based on the wavelet expansion of the signal]. 
\textit{Informatika i~ee Primeneniya~--- Inform. Appl.} 15(3):51--56.
doi: 10.14357/19922264210307. EDN: WSEAYG.

%8
\bibitem{Shestakov2022-1}
\Aue{Shestakov, O.\,V.} 2022.
Nesmeshchennaya otsenka riska porogovoy obrabotki s dvumya porogovymi znacheniyami [Unbiased thresholding risk estimate with two threshold values]. 
\textit{Informatika i~ee Primeneniya~--- Inform. Appl.} 16(4):14--19.
doi: 10.14357/19922264220403. EDN: DZBVLC.

%9
\bibitem{ResultsOnFDRUnderDependence-1}
\Aue{Farcomeni, A.} 2007. Some results on the control of the false discovery rate under dependence. 
\textit{Scand. J. Stat.} 34(2):275--297. 
doi: 10.1111/j.1467-9469.2006.00530.x.

%10
\bibitem{VorontsovShestakov2023-1}
\Aue{Vorontsov, M.\,O., and O.\,V.~Shestakov.} 2023.
Sred\-ne\-kvad\-ra\-tich\-nyy risk FDR-protsedury v~usloviyakh slaboy za\-vi\-si\-mosti [Mean-square risk of the FDR procedure under weak dependence]. 
\textit{Informatika i~ee Primeneniya~--- Inform. Appl.} 17(2):34--40.
doi: 10.14357/19922264230205. EDN: AVJZDX.

%11
\bibitem{Vorontsov2024-1}
\Aue{Vorontsov, M.\,O.} 2024. 
RMS risk analysis when using multiple hypothesis testing select parameters of thresholding under conditions of weak dependence. 
\textit{Moscow University Computational Mathematics Cybernetics} 48:91--97. 
doi: 10.3103/S027864192470002X.

%12
\bibitem{Bosq-1}
\Aue{Bosq, D.} 1996. 
\textit{Nonparametric statistics for stochastic processes: Estimation and prediction}. 
Lecture notes in statistics ser. New York, NY: Springer Verlag. Vol.~110. 188~p.

%13
\bibitem{Mallat-1}
\Aue{Mallat, S.} 1999. 
\textit{A wavelet tour of signal processing}. New York, NY: Academic Press. 857~p.

%14
\bibitem{spatialAdaptation-1}
\Aue{Donoho, D., and I.\,M.~Johnstone.} 1994. 
Ideal spatial adaptation via wavelet shrinkage. 
\textit{Biometrika} 81(3):425--455. doi: 10.1093/biomet/81.3.425.

%15
\bibitem{AdaptingSURE-1}
\Aue{Donoho, D., and I.\,M.~Johnstone.} 1995. 
Adapting to unknown smoothness via wavelet shrinkage. 
\textit{J. Am. Stat. Assoc.} 90(432):1200--1224. doi: 10.1080/01621459. 1995.10476626.

%16
\bibitem{ExactRisk-1}
\Aue{Marron, J.\,S., S.~Adak, I.\,M.~Johnstone, M.\,H.~Neumann, and P.~Patil.} 1998. 
Exact risk analysis of wavelet regression. 
\textit{J.~Comput. Graph. Stat.} 7(3):278-309. doi: 10.1080/ 10618600.1998.10474777.

%17
\bibitem{Jansen-1}
\Aue{Jansen, M.} 2001. 
\textit{Noise reduction by wavelet thresholding}. Lecture notes in statistics ser. New York, NY: Springer Verlag. Vol.~161. 217~p.

%18
\bibitem{KuShe2016_1-1}
\Aue{Kudryavtsev, A.\,A., and O.\,V.~Shestakov.} 2016. 
Asymptotic behavior of the threshold minimizing the average probability of error in calculation of wavelet coefficients. 
\textit{Dokl. Math.} 93(3):295--299.
doi: 10.1134/S1064562416030212. EDN: WUMUEV. 

%19
\bibitem{KuShe2016_2-1}
\Aue{Kudryavtsev, A.\,A., and O.\,V.~Shestakov.} 2016. 
Asymptotically optimal wavelet thresholding in the models with non-Gaussian noise distributions. 
\textit{Dokl. Math.} 94(3):615--619.
doi: 10.1134/S1064562416060028. EDN: YUYVUP.




%20
\bibitem{Eroshenko-1}
\Aue{Eroshenko, A.\,A.} 2015. Statisticheskie svoystva otsenok signalov i~izobrazheniy pri porogovoy obrabotke ko\-ef\-fi\-tsi\-en\-tov 
v~veyvlet-razlozheniyakh 
[Statistical properties of signal and image estimates under thresholding of coefficients in wavelet decompositions]. Moscow: MSU. PhD Diss. 82~p.

%21
\bibitem{Peligrad-1}
\Aue{Peligrad, M.} 1996. 
On the asymptotic normality of sequences of weak dependent random variables. 
\textit{J. Theor. Probab.} 9(3):703--715. doi: 10.1007/BF02214083.

%22
\bibitem{Serfling2002-1}
\Aue{Serfling, R.\,J.} 2002. 
\textit{Approximation theorems of mathematical statistics}. New York, NY: John Wiley \&~Sons. 371~p.
\end{thebibliography}

 }
 }

\end{multicols}

\vspace*{-6pt}

\hfill{\small\textit{Received May 21, 2024}} 

%\vspace*{-18pt}

\Contr

\vspace*{-3pt}


\noindent
\textbf{Vorontsov Mikhail O.} (b.\ 1996)~--- PhD student, Department of Mathematical Statistics, 
Faculty of Computational Mathematics and Cybernetics, M.\,V.~Lomonosov Moscow State University, 1-52~Leninskie Gory, GSP-1, Moscow 119991, Russian Federation;  
mathematician, Moscow Center for Fundamental and Applied Mathematics, M.\,V.~Lomonosov Moscow State University, 1~Leninskie Gory, GSP-1, Moscow 119991, Russian Federation;
\mbox{m.vtsov@mail.ru}

\vspace*{6pt}

\noindent
\textbf{Shestakov Oleg V.} (b.\ 1976)~--- Doctor of Science in physics and mathematics, professor, Department of Mathematical Statistics,
 Faculty of Computational Mathematics and Cybernetics, M.\,V.~Lomonosov Moscow State University, 1-52~Leninskie Gory, GSP-1, Moscow 119991, Russian Federation; 
 senior scientist, Federal Research Center ``Computer Science and Control'' of the Russian Academy of Sciences, 44-2~Vavilov Str., Moscow 119333, 
 Russian Federation; leading scientist, Moscow Center for Fundamental and Applied Mathematics, M.\,V.~Lomonosov Moscow State University, 
 1~Leninskie Gory, GSP-1, Moscow 119991, Russian Federation; \mbox{oshestakov@cs.msu.su}


\label{end\stat}

\renewcommand{\bibname}{\protect\rm Литература} %9+
\newcommand{\R}{\mathbb R}
\newcommand{\dd}{3}
\renewcommand{\d}{1}
\newcommand{\betr}{\beta_{3}}
\newcommand{\Lo}{\fr{\betr}{\sqrt{n}}}
\newcommand{\Ll}{\fr{\betr+1}{\sqrt{n}}}
\newcommand{\pto}{\stackrel{P}{\longrightarrow}}

\def\stat{gavr}

\def\tit{УТОЧНЕНИЕ НЕРАВНОМЕРНОЙ ОЦЕНКИ СКОРОСТИ СХОДИМОСТИ
РАСПРЕДЕЛЕНИЙ ПУАССОНОВСКИХ СЛУЧАЙНЫХ СУММ К НОРМАЛЬНОМУ
ЗАКОНУ$^*$}

\def\titkol{Уточнение неравномерной оценки скорости сходимости
распределений пуассоновских случайных сумм % к нормальному закону
}

\def\autkol{С.\,В.~Гавриленко}
\def\aut{С.\,В.~Гавриленко$^1$}

\titel{\tit}{\aut}{\autkol}{\titkol}

{\renewcommand{\thefootnote}{\fnsymbol{footnote}}\footnotetext[1]
{Работа выполнена при поддержке Министерства
образования и науки (государственный контракт 16.740.11.0133 от
02.09.2010).}}

\renewcommand{\thefootnote}{\arabic{footnote}}
\footnotetext[1]{Московский государственный университет им.\ М.\,В.~Ломоносова, 
факультет вычислительной математики и кибернетики,
gavrilenko.cmc@gmail.com}


\Abst{Строятся неравномерные оценки скорости сходимости
в классической центральной предельной теореме с уточненной
структурой. С~помощью этих структурных уточнений показано, что
абсолютная константа в неравномерной оценке скорости сходимости в
центральной предельной теореме (ЦПТ) для пуассоновских случайных сумм
строго меньше, чем аналогичная константа в неравномерной оценке
скорости сходимости в классической ЦПТ,
и при условии существования третьих моментов слагаемых не
превосходит 22,7707. В~качестве следствия построены неравномерные
оценки скорости сходимости смешанных пуассоновских, в частности
отрицательных биномиальных случайных сумм.}

\KW{центральная предельная теорема; скорость
сходимости; неравномерная оценка; абсолютная константа;
пуассоновская случайная сумма; смешанное пуассоновское
распределение}

      \vskip 14pt plus 9pt minus 6pt

      \thispagestyle{headings}

      \begin{multicols}{2}
      
            \label{st\stat}

\section{Введение}

Пусть $X_1,X_2,\ldots$~--- последовательность независимых одинаково
распределенных случайных величин таких, что ${\sf E}X_1=0$, ${\sf
E}X_1^2=1$, ${\sf E}|X_1|^{\dd}=\betr<\infty$. Положим $F_n(x)={\sf
P}\big(X_1+\ldots+X_n<x \sqrt{n}\big)$. Пусть $\Phi(x)$~--- стандартная нормальная функция распределения, т.\,е.\
$$
\Phi(x) = \fr{1}{\sqrt{2\pi}}\int_{-\infty}^x e^{-t^2/2}\,dt\,.
$$

Известно, что при указанных условиях существуют абсолютные
положительные конечные константы $C_0$ и $C_1$ такие, что~[1, 2]
\begin{equation}
\sup_x|F_n (x)-\Phi(x)|\le C_0 \Lo
\label{e1gv}
\end{equation}
 и~[3, 4]:
\begin{multline}
\sup_x\left|F_n(x)-\Phi(x)\right|\le C_1\Ll={}\\
{}=
C_1\left(1+\fr{1}{\betr}\right)\Lo\label{e2gv}\,.
\end{multline}
 Для констант $C_0$ и $C_1$ известны следующие численные
оценки~[4, 5]:
\begin{equation*}
0{,}4097\approx\fr{\sqrt{10}+3}{6\sqrt{2\pi}}\le C_0\le
0{,}4784
%\label{e3gv}
\end{equation*}
и~[3, 4]:
\begin{equation*}
0{,}2659\approx\fr{2}{3\sqrt{2\pi}}\le C_1\le 0{,}3041\,.
%\label{e4gv}
\end{equation*}
При этом, поскольку всегда $\betr\ge1$, при больших значениях
$\betr$ оценка~(\ref{e2gv}) точнее, чем~(\ref{e1gv}), за счет меньших значений
абсолютных констант.

Оценка скорости сходимости $F_n(x)$ к $\Phi(x)$,
устанавливаемая неравенствами~(\ref{e1gv}) и~(\ref{e2gv}), \textit{равномерна} по~$x$.
Но поскольку и $F_n(x)$, и $\Phi(x)$~--- функции распределения, то
должно выполняться соотношение $|F_n(x)-\Phi(x)|\rightarrow 0$ при
$|x|\to0$. Это обстоятельство не учитывается в равномерных
оценках. Вместе с тем точность нормальной аппроксимации для
функции распределения сумм случайных величин именно при больших
значениях аргумента представляет особый интерес, например, при
вычислении рисков критически больших потерь. В~данной статье будут
рассмотрены неравномерные оценки скорости сходимости в центральной
предельной теореме.

По-видимому, исторически первая оценка такого рода была получена в
работе~\cite{Meshalkin}, где для $\delta=1$, то есть для случая
существования третьего момента слагаемых, было доказано
существование конечной положительной абсолютной постоянной~$A$
такой, что для любого $x\in\R$ справедливо неравенство
$$
(1+x^2)|F_n(x)-\Phi(x)|\le A\fr{\beta_3}{\sqrt{n}}\,.
$$

Этот результат был усилен в работе~\cite{Nagaev}, где было
показано, что существует такое положительное конечное число~$C$,
что
\begin{equation}
\sup_x\left(1+|x|^{\dd}\right)\left|F_n(x)-\Phi(x)\right|\le
C\Lo\,.\label{e5gv}
\end{equation}
При этом для рассматриваемых условий на моменты слагаемых порядок
оценки~(\ref{e5gv}) по $x$ неулучшаем без дополнительных предположений.

Что касается значения абсолютной константы~$C$ в~(\ref{e5gv}), то в работе~\cite{Mich81} 
было показано, что $C\le C_0+8(1+e)$, что с учетом
оценки $C_0\le 0{,}4784$, полученной в~\cite{KorolevBEs, KorSchev}, влечет оценку $C\le
30{,}2247$. Недавно эта оценка была уточнена в работе~\cite{Nefedova},
где было показано, что $C \le 25{,}7984$.

В данной работе с помощью модификации метода Л.~Падитца~\cite{Paditz89} будут построены альтернативные неравномерные
оценки скорости сходимости в центральной предельной теореме,
имеющие структуру, аналогичную неравенству~(\ref{e2gv}). Полученные оценки
затем будут использованы для уточнения абсолютной константы в
аналоге неравенства~(\ref{e5gv}) для пуассоновских случайных сумм.

Согласно результатам работы~\cite{Mich93}, в качестве абсолютной
константы в неравномерной оценке скорости сходимости в центральной
предельной тео\-ре\-ме для пуассоновских случайных сумм можно брать
абсолютную константу~$C$ из неравенства~(\ref{e5gv}). В~предлагаемой статье
с использованием упомянутых выше структурных уточнений неравенства~(\ref{e5gv})\linebreak 
будет показано, что на самом деле абсолютная константа в
неравномерной оценке ско\-рости схо\-ди\-мости в центральной предельной
теореме для пуассоновских случайных сумм строго меньше, чем\linebreak
аналогичная константа в неравенстве~(\ref{e5gv}), и не превосходит
22,7707. С~помощью этих результатов затем будут построены
неравномерные оценки скорости сходимости распределений смешанных
пуассоновских, в частности отрицательных биномиальных, случайных
сумм.

\section{Неравномерные оценки скорости сходимости в~центральной предельной теореме с~уточненной структурой}

Идея, лежащая в основе метода построения неравномерных оценок
точности нормальной аппроксимации для распределений сумм независимых
случайных величин, описанного в работе~\cite{Paditz89}, заключается
в подходящем разбиении вещественной прямой на зоны <<малых>>,
<<умеренных>> и <<больших>> значений~$x$. Традиционно используются
разбиения следующего вида:
\begin{enumerate}[$i$]
\item
<<малые>> значения $x$: $0\le x^2\le K^2$;
\item
<<умеренные>> значения $x$: $K^2\le x^2\le c_n(x;a,b)$;
\item
<<большие>> значения $x$: $c_n(x;a,b)\le x^2<\infty$,
\end{enumerate}
где $K>0$, $a>0$, $b>1$~--- вспомогательные свободные
параметры,
$$
c_n(x;a,b)=2b\left[\log|x|^{\dd}-\log\left(a\Lo\right)\right]
$$
(см., в частности,~\cite{Paditz89, Rychlik}).

\subsection{Случаи \boldmath{$i$} и \boldmath{$iii$}, то есть <<малые>> и~<<большие>> значения \boldmath{$x$}}

В случае~$i$, т.\,е.\ для $0\le |x|\le K$, в соответствии с
неравенством~(\ref{e2gv}) имеем
\begin{multline}
|x|^{\dd}\left|F_n(x)-\Phi(x)\right|\le
C_1K^{\dd}\Ll={}\\
{}=C_1K^{\dd}\Lo+C_1K^{\dd}\fr{1}{n^{\d/2}}\,. 
\label{e6gv}
\end{multline}


В случае же $iii$, т.\,е.\ для
$x^2\in[2b(\log|x|^{\dd}\hm-\log(a\beta_3/\sqrt{n})),\infty]$, в
соответствии с работами~\cite{Paditz81, Tysiak} справедлив следующий результат.
Обозначим
$$
P(a,b,K)=(2b)^{\dd}+a\exp\left\{\fr{(2b)^{\dd}}{a}-\fr{(b-1)K^2}{2b}\right\}\,.
$$

\smallskip

\noindent
\textbf{Лемма 2.1}. \textit{Предположим, что $x^2\ge c_n(x;a,b)\ge$\linebreak
$\ge K^2\ge (2\pi)^{-1}$. Тогда для любого $n\ge1$}
\begin{equation*}
|x|^{\dd}\left|F_n(x)-\Phi(x)\right|\le P(a,b,K)\Lo\,.
%\label{e7gv}
\end{equation*}

\smallskip

\noindent
Д\,о\,к\,а\,з\,а\,т\,е\,л\,ь\,с\,т\,в\,о\,\ см.\ в работах~\cite{Paditz81, Tysiak}.

\subsection{Случай \boldmath{$ii$}, то есть <<умеренные>> значения~\boldmath{$x$}}

Рассмотрение этого случая базируется на следующем фундаментальном
неравенстве (см.~\cite{Mich81} и~\cite{Tysiak}), в котором без
ограничения общности $x\ge0$:
\begin{multline}
\left|F_n(x)-\Phi(x)\right|\le n{\sf P}(|X_1|>y)
+
\left|
\vphantom{\fr{1}{2}}
f^n(h)-{}\right.\\
\left.{}-\exp\left\{\fr{1}{2}\,h^2n\right\}\right|
\exp\left\{-hx\sqrt{n}\right\}{\sf P}\left(S_n^*>x\sqrt{n}\right)+{}\\
{}+
2\exp\left\{\fr{1}{2}\,h^2n-hx\sqrt{n}\right\}\cdot\sup_{u\ge x}
\left|{\sf P}\left(S_n^*<u\sqrt{n}\right)-{}\right.\\
\left.{}-\Phi\left(u-h\sqrt{n}\right)\right|\,,
\label{e8gv}
\end{multline}
где $f(h)={\sf E}\exp\left\{h\overline{X}_1\right\}$,
$\overline{X}_1=X_1\I\{|X_1|<y\}$, $S_n^*=X_1^*+\ldots+X_n^*$~---
сумма независимых одинаково распределенных случайных величин с
общей функцией распределения
\begin{gather*}
{\sf P}(X_1^*<u)=\fr{1}{f(h)}\int\limits_{-\infty}^{u}e^{ht}\,d{\sf
P}\left(\overline{X}_1<t\right)\,,\\
y=\gamma x\sqrt{n}\,,\enskip h=\fr{(1-\gamma)x}{\sqrt{n}}\,,\enskip \gamma\in\left(0,\fr{1}{2}\right)\,.
\end{gather*}

Для начала сформулируем два утверждения, которые будут
неоднократно использоваться в дальнейшем. Во-первых, если $x^2\le
c_n(x;a,b)$, то (см.~$ii$)
\begin{equation}
\Lo\le\fr{|x|^{\dd}}{a}\,\exp\left\{-\fr{x^2}{2b}\right\}\,.\label{e9gv}
\end{equation}
Во-вторых, если $x^2\ge K^2$, то (см.~\cite{Paditz81})
\begin{equation}
x^r\exp\left\{-sx^2\right\}\le K^r\exp\left\{-sK^2\right\}\label{e10gv}
\end{equation}
при $x\ge\sqrt{r/(2s)}$ ($r>0$, $s>0$) или $r\le 0$.

Чтобы оценить выражение
$$
I_1=\left|f^n(h)-\exp\left\{\fr{1}{2}h^2n\right\}\right|
\exp\left\{-hx\sqrt{n}\right\}\,,
$$
воспользуемся результатом из~\cite{Tysiak}, согласно
которому
\begin{multline*}
I_1\le
\max\left\{
\vphantom{\left(1-\fr{\betr e^{hy}}{n^{\d/2}(\gamma
x)^{\dd}}\right)^{-1}}
\exp\left\{\fr{1}{2}\,h^2n-hx \sqrt{n}+
\fr{\betr e^{hy}}{n^{\d/2}(\gamma x)^{\dd}}\right\}\times\right.{}\\
\times \fr{\betr
e^{hy}}{n^{\d/2}(\gamma x)^{\dd}}\,, n\exp\left\{\fr{1}{2}h^2n-hx\sqrt{n}\right\}\times{}\\
\left.{}\times
\left(\fr{h^4}{4}+\fr{\betr
e^{hy}}{y^{\dd}}\right)\left(1-\fr{\betr e^{hy}}{n^{\d/2}(\gamma
x)^{\dd}}\right)^{-1}
\vphantom{}
\right\}.
\end{multline*}
Из~(\ref{e9gv}) вытекает, что
\begin{multline*}
1-\fr{\betr e^{hy}}{n^{\d/2}(\gamma x)^{\dd}}\ge{}\\
{}\ge 1-\fr{1}{a
\gamma^{\dd}}\exp\left\{\left(\gamma(1-\gamma)-\fr{1}{2b}\right)x^2\right\}\equiv
A_1(x)\,.
\end{multline*}
Здесь и далее символами $A(x)$, $A_1(x)$, $A_2(x)$,\ldots будут
обозначаться положительные функции аргумента $x$, а также
зависящие от параметров $a$, $b$, $\gamma$. Принимая во внимание
оценку
$$
n^{-1/2}\le\left(\Lo\right)^{1/3}
$$
и неравенство~(\ref{e9gv}), получим
\begin{multline*}
n\left(\fr{h^4}{4}+\fr{\betr e^{hy}}{y^{\dd}}\right)\le{}\\
\!{}\le
\fr{\betr}{n^{\d/2}|x|^{\dd}}\!\left[\fr{(1-\gamma)^4x^8}{4a^{1/3}}\,
\exp\!\left\{-\fr{x^2}{6b}\right\}+\fr{e^{\gamma(1-\gamma)x^2}}{\gamma^{\dd}}\right].
\end{multline*}
С учетом неравенства
$$
\exp\{1-A_1(x)\}\le\fr{1}{A_1(x)}\,,\enskip A_1(x)>0\,,
$$
также получаем
\begin{multline*}
\exp\left\{\fr{1}{2}\,h^2n-hx\sqrt{n}+ \fr{\betr
e^{hy}}{n^{1/2}(\gamma x)^{\dd}}\right\}\le{}\\
{}\le
\exp\left\{\fr{1}{2}\,h^2n-hx\sqrt{n}+1-A_1(x)\right\}\le{}\\
{}\le
\fr{1}{A_1(x)}\,\exp\left\{\fr{1}{2}h^2n-hx\sqrt{n}\right\}\,.
\end{multline*}
Наконец, если $\gamma(1-\gamma)-1/(2b)<0$, т.\,е.\
$b<$\linebreak $<[2\gamma(1-\gamma)]^{-1}$, то
\begin{equation}
I_1\le\fr{A_2(x)\betr}{n^{\d/2}A_1(K)|x|^{\dd}}\,,\label{e11gv}
\end{equation}
где $A_1(K)>0$ и
\begin{multline*}
A_2(x)=\fr{(1-\gamma)^4x^8}{4a^{1/3}}
\exp\left\{-\fr{x^2}{2}\left[\fr{1}{3b}+1-\gamma^2\right]\right\}+{}\\
{}+
\fr{1}{\gamma^{\dd}}\,\exp\left\{-\fr{x^2}{2}(1-\gamma)^2\right\}\,.
\end{multline*}
Для удобства дальнейших ссылок заметим, что с учетом~(\ref{e9gv})
$$
f(h)\ge 1-\fr{h\betr}{\gamma^{2}}\ge
1-\fr{x^2(1-\gamma)}{a\gamma^{2}}\,\exp\left\{-\fr{x^2}{2b}\right\}\equiv
A_3(x)
$$
(см.\ соотношение~(4.18) в \cite{Tysiak}).

Теперь оценим ${\sf E}\big(X_1^*\big)^2$. В~соответствии с~\cite{Tysiak} получаем, что если $K\ge\sqrt{2b}$ 
(см.~(\ref{e10gv}), то
\begin{equation}
{\sf E}\left(X_1^*\right)^2\le \fr{1}{A_3^{2}(K)}\left(h+\fr{\betr
e^{hy}}{y^{3}}\right)^2\!\le A_4(x)\Lo,\!\label{e12gv}
\end{equation}
где
\begin{multline*}
A_4(x)=\fr{1}{A_3^2(x)x^{\d}a^{1/3}}\,\exp\left\{-\fr{x^2}{6b}\right\}
\left[
\vphantom{\fr{1}{6}}
x^2(1-\gamma)+{}\right.\\
\left.{}+\fr{1}{\gamma^{2}a^{\d/3}}\exp\left\{\left(\gamma(1-\gamma)-
\fr{\d}{6b}\right)x^2\right\}
\right]^2\,.
\end{multline*}
Далее, согласно~\cite{Tysiak}

\noindent
\begin{multline*}
f^{-1}(h)\ge 2-f(h)\ge 1-\fr{h^2}{2}-\fr{\betr
e^{hy}}{y^{3}}\ge{}\\
{}\ge 1-A_5(x)\Lo\,,
\end{multline*}
где

\noindent
\begin{multline*}
A_5(x)=\fr{(1-\gamma)^2x^{3}}{2a^{1/3}}\,\exp\left\{-\fr{x^2}{6b}\right\}+{}\\
{}+
\fr{1}{\gamma^{3}a^{2/3}x}\,\exp\left\{\left(\gamma(1-\gamma)-
\fr{1}{3b}\right)x^2\right\}\,,
\end{multline*}
а также
$$
f(h){\sf E}\left(X_1^*\right)^2\ge 1-\betr\max\{\gamma^{-\d},\,h\}\,.
$$
Таким образом,
\begin{multline*}
B_n^2(h)\equiv {\sf D}S_n^*= n\left[{\sf
E}\left(X_1^*\right)^2-\left({\sf E}X_1^*\right)^2\right]\ge{}\\
{}\ge
n\left[f^{-1}(h)\left(1-\betr\max\{\gamma^{-\d},\,h\}\right)\right]-{}\\
{}- n\left({\sf
E}X_1^*\right)^2 \ge
n\left(1-\fr{A_5(x)\betr}{n^{\d/2}}\right)-{}\\
{}-\fr{n^{\d/2}\betr}{A_3(K)\gamma
x}\,\max\left\{1,\,\gamma(1-\gamma)x^2\right\}-{}\\
{}-A_4(x)n^{\d/2}\betr=n\left[1-A_6(x)\Lo\right]\,,
\end{multline*}
где
$$
A_6(x)=A_4(x)+A_5(x)+\fr{\max\{1,\,\gamma(1-\gamma)x^2\}}{A_3(K)\gamma
x}\,.
$$
С учетом~(\ref{e9gv}) имеем
\begin{multline}
B_n^2(h)\ge
n\left(1-\fr{x^{3}A_6(x)}{a}\exp\left\{-\fr{x^2}{2b}\right\}\right)\equiv{}\\
{}\equiv 
nA_7(x)\,;
\label{e13gv}
\end{multline}

\vspace*{-6pt}

\noindent
\begin{equation}
\fr{n-B_n^2(h)}{B_n^2(h)}\le\fr{A_6(x)}{A_7(x)}\Lo\,.\label{e14gv}
\end{equation}
Справедливы неравенства
\begin{multline*}
{\sf E}|X_1^*-{\sf E}X_1^*|^3\le{}\\
{}\le {\sf E}|X_1^*|^3+3{\sf
E}\left(X_1^*\right)^2|{\sf E}X_1^*|
+{\sf E}|X_1^*|\left({\sf E}X_1^*\right)^2\,;
\end{multline*}

\noindent
$$
{\sf
E}|X_1^*|^3\le\fr{e^{hy}\betr}{A_3(K)}\le\fr{1}{A_3(K)}\exp\{\gamma(1-\gamma)x^2\}\betr\,;
$$


\noindent
\begin{multline*}
3{\sf E}\left(X_1^*\right)^2\left|{\sf
E}X_1^*\right|\le{}\\
{}\le\fr{3\betr}{A_3^2(K)x}\left[\fr{(1-\gamma)x^3}{a^{1/3}}\,
\exp\left\{-\fr{x^2}{6b}\right\}+{}\right.\\
{}+
\fr{x+x^3\gamma(1-\gamma){\gamma^{2}a^{2/3}}}{\exp}\left\{\left(\gamma(1-\gamma)-
\fr{1}{3b}\right)x^2\right\}+{}\\
\left.{}+
\fr{x}{\gamma^{3}a}\exp\left\{\left(2\gamma(1-\gamma)-\fr{1}{2b}\right)x^2\right\}\right]
\equiv A_8(x)\betr\,;
\end{multline*}

\noindent
\begin{multline*}
{\sf E}|X_1^*|\left({\sf E}X_1^*\right)^2\le\sqrt{{\sf
E}\left(X_1^*\right)^2}\left({\sf E}X_1^*\right)^2\le{}\\
{}\le
\fr{\betr }{A_3^{5/2}(K)a^{2/3}}\,
\exp\left\{\left(\fr{5}{2}\gamma(1-\gamma)-\fr{3}{4b}\right)x^2\right\}\times{}\\
{}\times
\left[\fr{1}{\gamma
a^{1/3}}+\exp\left\{\left(\fr{1}{6b}-\gamma(1-\gamma)\right)x^2\right\}\right]^{1/2}\times{}\\
{}
\times\left[\fr{1}{\gamma^{2}a^{1/3}}+\right.\\
\left.{}+x^2(1-\gamma)\,\exp\left\{\left(\fr{1}{6b}-
\gamma(1-\gamma)\right)x^2\right\}\right]^2\equiv{}\\
{}\equiv A_9(x)\betr\,.
\end{multline*}
Таким образом,

\noindent
\begin{equation}
{\sf E}|X_1^*-{\sf E}X_1^*|^3\le A_{10}(x)\betr\,,\label{e15gv}
\end{equation}
где

\noindent
$$
A_{10}(x)=\fr{1}{A_3(K)}\,\exp\{\gamma(1-\gamma)x^2\}+A_8(x)+A_9(x)\,.
$$
Следовательно, в соответствии с неравенством~(\ref{e2gv}), учитывая~(\ref{e15gv}) и~(\ref{e13gv}), имеем

\noindent
\begin{multline}
\left|{\sf P}\left(\fr{S_n^*-{\sf E}S_n^*}{\sqrt{{\sf
D}S_n^*}}<u\right)-\Phi(u)\right|\le{}\\
{}\le 0{,}3041\fr{{\sf E}|X_1^*-{\sf
E}X_1^*|^3+1}{\sqrt{n}\big({\sf D}X_1^*\big)^{3/2}}\le{}\\
{}
\le 0{,}3041
\fr{A_{10}(x)\betr}{A_7^{3/2}(x)n^{\d/2}}+\fr{0{,}3041}{A_7^{3/2}(x)\sqrt{n}}\,.\label{e16gv}
\end{multline}
Далее в предположении, что

\noindent
\begin{multline*}
A(x)\equiv\fr{1}{2\gamma
a^{1/3}}\,\exp\left\{-\fr{x^2}{6b}\right\}+{}\\
{}+
\fr{1}{\gamma^4(1-\gamma)^2x^4a^{2/3}}\,\exp\left\{\!\left(\gamma(1-\gamma)-
\fr{1}{3b}\right)x^2\!\right\}\le\fr{1}{6}\hspace*{-2.1413pt}
\end{multline*}
(см.\ выражение для $g_{19}(x)$ в~\cite{Tysiak}), имеем
\begin{multline}
\left|{\sf
E}S_n^*-hB_n^*\right|\le\fr{\betr}{A_3(K)\gamma^{2}x^{2}}\left[
\vphantom{\fr{x^2}{3}}\exp\left\{\gamma(1-\gamma)x^2\right\}+{}\right.\\
\left.{}+
\fr{(1-\gamma)^2x^4}{a^{2/3}}\,\exp\left\{-\fr{x^2}{3b}\right\}\right]\equiv
A_{11}(x)\betr\,.\label{e17gv}
\end{multline}
Наряду с~(\ref{e14gv}) имеет место оценка
\begin{equation*}
B_n^2(h)-n\le
n\left[\fr{e^{hy}\betr}{f(h)y}+\fr{1}{f(h)}-1\right]\le{}\hspace*{8mm}
\end{equation*}
\pagebreak

\noindent
\begin{multline*}
{}\le 
\fr{n^{\d/2}\betr}{A_3(K)\gamma
x}\left[\exp\left\{\gamma(1-\gamma)x^2\right\}+{}\right.\\
\left.{}+\fr{(1-\gamma)x^2}{\gamma
a^{2/3}}\exp\left\{-\fr{x^2}{3b}\right\}\right]\equiv
A_{12}(x)n^{\d/2}\betr\,.
\end{multline*}
Таким образом,
\begin{multline}
\left|B_n^2(h)-n\right|\max\left\{\fr{1}{n},\,\fr{1}{B_n^2(h)}\right\}\le{}\\
{}\le \max\left\{A_{12}(x),\,\fr{A_6(x)}{A_7(x)}\right\}\Lo\,.\label{e18gv}
\end{multline}
Теперь можно приступить к оцениванию величин
(см.~(\ref{e8gv})):
\begin{align*}
I_2&\equiv {\sf P}\left(S_n^*>x\sqrt{n}\right) \  \mbox{и}\ \\
I_3&\equiv \sup_{u\ge x}\left|{\sf
P}\left(S_n^*<u\sqrt{n}\right)-\Phi\left(u-h\sqrt{n}\right)\right|
\end{align*}


Наряду с~(\ref{e13gv}) понадобится верхняя оценка для $B_n^2(h)$, которую
получим с учетом тождества $(1-z)^{-1}=z(1-z)^{-1}+1$:
\begin{multline*}
B_n^2(h)\le n{\sf E}\left(X_1^*\right)^2\le{}\\
{}\le
 n\left(1+\fr{e^{hy}\betr}{y}\right)\left(1-\fr{h\betr}{y^{2}}\right)^{-1}={}\\
{} =
n\left(1+\fr{e^{hy}\betr}{y}\right)\left[\fr{h\betr}{y^{2}}\left(1-\fr{h\betr}{y^{2}}\right)^{-1}+1\right]\le{}\\
{}\le
\fr{n}{A_3(K)}\left[\fr{h\betr}{y^{2}}+
\fr{he^{hy}\beta^2_{\dd}}{y^{3}}+A_3(K)\left(\!1+\fr{e^{hy}\betr}{y}\!\right)\right]
\equiv{}\hspace*{-0.7442pt}\\
{}\equiv n A_{13}(x)\,,
\end{multline*}
где
\begin{multline*}
A_{13}(x)=1+(\gamma x)^2\left(1-A_1(K)\right)+{}\\
{}
+\fr{\left(1-A_3(K)\right)x^2}{aA_3(K)}\left[a^{\d/3}\exp\left\{-\fr{x^2}{3b}\right\}+{}\right.\\
\left.{}+\fr{1}{\gamma}\,\exp\left\{\left(\gamma(1-\gamma)-\fr{1}{2b}\right)x^2\right\}\right]\,.
\end{multline*}
Найдем верхнюю оценку для ${\sf E}S_n^*-x\sqrt{n}$ (см.~(\ref{e17gv}):
\begin{multline*}
{\sf E}S_n^*-x\sqrt{n}={\sf
E}S_n^*-nh+nh-x\sqrt{n}\le{}\\
{}\le
\sqrt{n}\left(\fr{A_{11}(x)x^{\dd}}{a}\exp\left\{-\fr{x^2}{2b}\right\}-\gamma
x\right)\,,
\end{multline*}
т.\,е.\
\begin{multline}
\fr{x\sqrt{n}-{\sf E} S_n^*}{B_n(h)}\ge
\fr{1}{\sqrt{A_{13}(x)}}\left(\vphantom{\fr{x^3}{a}}\gamma x-{}\right.\\
\left.{}-\fr{x^{\dd}A_{11}(x)}{a}\exp\left\{-\fr{x^2}{2b}\right\}\right)\equiv
A_{14}(x)\,.\label{e19gv}
\end{multline}
Отсюда с учетом~(\ref{e1gv}), (\ref{e2gv}), (\ref{e9gv}) и (\ref{e16gv}) при
$A_{14}(x)\ge1$ получаем
\begin{multline}
I_2\le\left|{\sf
P}\left(S_n^*<x\sqrt{n}\right)-\Phi\left(\fr{x\sqrt{n}-{\sf
E}S_n^*}{\sqrt{{\sf
D}S_n^*}}\right)\right|+{}\\
{}+\Phi\left(-\fr{x\sqrt{n}-{\sf
E}S_n^*}{\sqrt{{\sf D}S_n^*}}\right)\le{}\\
{}\le 
0{,}3041\fr{A_{10}(x)x^{\dd}}{aA_7^{3/2}(x)}\exp\left\{-\fr{x^2}{2b}\right\}+{}\\
{}+
\fr{0{,}3041}{A_7^{3/2}(x)\sqrt{n}}+
\fr{1}{\sqrt{2\pi}A_{14}(x)}\exp\left\{-\fr{A_{14}^2(x)}{2}\right\}\equiv{}\\
{}
\equiv A_{15}(x)+\fr{0{,}3041}{A_7^{3/2}(x)\sqrt{n}}\,.\label{e20gv}
\end{multline}

Чтобы оценить $I_3$, воспользуемся соотношениями~(\ref{e13gv}),
(\ref{e16gv})--(\ref{e19gv}) и~(\ref{e2gv}) и получим
\begin{multline}
I_3=\sup_{u\ge x}\bigg|{\sf P}\left(\fr{S_n^*-{\sf
E}S_n^*}{B_n(h)}<\fr{u\sqrt{n}-{\sf
E}S_n^*}{B_n(h)}\right)-{}\\
{}-\Phi\left(\fr{u\sqrt{n}-{\sf
E}S_n^*}{B_n(h)}\right)+\Phi\left(\fr{u\sqrt{n}-{\sf
E}S_n^*}{B_n(h)}\right)-{}\\
{}-
\Phi\left(\fr{u\sqrt{n}-{\sf
E}S_n^*}{\sqrt{n}}\right)+\Phi\left(\fr{u\sqrt{n}-{\sf
E}S_n^*}{\sqrt{n}}\right)-{}\\
{}-\Phi\left(u-h\sqrt{n}\right)\bigg|\le{}\\
{}\le
\sup_{v\ge(x\sqrt{n}-{\sf E}S_n^*)/B_n(h)}\bigg[\bigg|{\sf
P}\bigg(\fr{S_n^*-{\sf
E}S_n^*}{B_n(h)}<v\bigg)-{}\\
{}-
\Phi(v)\bigg|+\bigg|\Phi(v)-\Phi\bigg(v\fr{B_n(h)}{\sqrt{n}}\bigg)
\bigg|\bigg]+{}\\
{}
+\sup_{u\ge x}\left|\Phi\left(u-{\sf
E}S_n^*/\sqrt{n}\right)-\Phi\left(u-h\sqrt{n}\right)\right|\le{}\\
{}\le
0{,}3041\fr{A_{10}(x)\betr}{A_7^{3/2}(x)n^{\d/2}}+\fr{0{,}3041}{A_7^{3/2}(x)\sqrt{n}}+{}\\
{}
+\fr{1}{\sqrt{8\pi}}\left|\fr{B_n^2(h)}{n}-1\right|\max\left\{1,\,\fr{n}{B_n^2(h)}\right\}\times{}\\
{}\times
\sup\left\{\vphantom{\fr{B_h(h)}{B_h(h)}}
|s|e^{-s^2/2}:\right.\\
\left.s\ge\fr{x\sqrt{n}-{\sf
E}S_n^*}{B_n(h)}\min\left\{1,\,\fr{B_n(h)}{\sqrt{n}}\right\}\right\}+{}\\
{}+
\fr{1}{\sqrt{2\pi}}\bigg|\fr{{\sf
E}S_n^*-nh}{\sqrt{n}}\bigg|
\sup\bigg\{e^{-s^2/2}:\\
s\ge\min\Big\{x-\fr{{\sf
E}S_n^*}{\sqrt{n}},\,x-h\sqrt{n}\Big\}\bigg\}\le{}\\
{}\le
A_{16}(x)\Lo+\fr{0{,}3041}{A_7^{3/2}(x)\sqrt{n}}\,,\label{e21gv}
\end{multline}
где
\begin{multline*}
A_{16}(x)=0{,}3041\fr{A_{10}(x)}{A_7^{3/2}(x)}+{}\\
{}+\fr{A_{14}(x)}{\sqrt{8\pi}}\max\left\{A_{12}(x),\,\fr{A_6(x)}{A_7(x)}\right\}
\exp\left\{-\fr{A_{14}^2(x)}{2}\right\}+{}\\
{}+
\fr{A_{11}(x)}{\sqrt{2\pi}}\exp\!\bigg\{\!-\fr{1}{2}\Big[\gamma
x-\fr{A_{11}(x)|x|^{\dd}}{a}\exp\left\{-\fr{x^2}{2b}\right\}\Big]^2\!\bigg\}.\hspace*{-1.467pt}
\end{multline*}

Соотношение~(\ref{e21gv}) справедливо в предположении
$A_{14}(x)\ge 1$, так что
$$
\gamma
x-\fr{A_{11}(x)|x|^{\dd}}{a}\,\exp\left\{-\fr{x^2}{2b}\right\}\ge 0\,.
$$
Первое слагаемое в правой части~(\ref{e8gv}) оценим с помощью неравенства
Маркова:
\begin{equation}
n{\sf P}(|X_1|>y)\le\fr{\betr}{\gamma^{\dd}
|x|^{\dd}n^{\d/2}}\,.\label{e22gv}
\end{equation}
В итоге из~(\ref{e8gv}) с учетом~(\ref{e11gv}), (\ref{e20gv})--(\ref{e22gv}) мы
получаем: для каждого~$x$ из рассматриваемого диапазона~ii
справедливо неравенство
\begin{multline}
\left|F_n(x)-\Phi(x)\right|\le\fr{\betr}{\gamma^{\dd}|x|^{\dd}n^{\d/2}}+I_1
I_2+{}\\
{}+2I_3\exp\left\{-(1-\gamma^2)\fr{x^2}{2}\right\}\le{}\\
{}\le 
Q_n(x;\,a,b,\gamma,K)\fr{\betr}{|x|^{\dd}n^{\d/2}}+{}\\
{}+
0{,}6082\fr{\exp\left\{-(1-\gamma^2)x^2/2\right\}}{A_7^{3/2}(x)\sqrt{n}}\,,\label{e23gv}
\end{multline}
где
\begin{multline*}
Q_n(x;\,a,b,\gamma,K)=\fr{1}{\gamma^{\dd}}+{}\\
{}+
\fr{A_2(x)}{A_1(K)}\left[1+A_{15}(x)+
\fr{0{,}3041}{A_7^{3/2}(x)\sqrt{n}}\right]+{}\\
{}+2A_{16}(x)|x|^{\dd}\exp\left\{-(1-\gamma^2)\fr{x^2}{2}\right\}
\end{multline*}
%$\big 
(отметим, что
$ %\begin{multline*}
\lim_{n\to\infty}Q_n(x;\,a,b,\gamma,K)={1}/{\gamma^{\dd}}+$\linebreak $+
{A_2(x)\left(1+A_{15}(x)\right)}/{A_1(K)}+2A_{16}(x)|x|^{\dd}\exp\{-(1-$\linebreak 
$-\gamma^2){x^2}/{2}\}$).
%\end{multline*}

Обозначим
$$
R(x;\,a,b,\gamma,K)=0{,}6082\fr{|x|^{\dd}\exp\left\{-(1-\gamma^2){x^2}/{2}\right\}}{A_7^{3/2}(x)}\,.
$$
Тогда справедливо следующее утверждение.

\columnbreak

\noindent
\textbf{Лемма 2.2} \textit{Предположим, что $K^2\le x^2\le
c_n(x;\,a,b)$,}

\noindent
\begin{multline*}
K^2\ge\fr{1}{2\pi}\,,\enskip  0<\gamma<\fr{1}{2}\,,\enskip  a>0\,,\\
1<b<\min\left\{\fr{1}{2\gamma(1-\gamma)},\,\fr{K^2}{2}\right\}
\end{multline*}
\textit{и}

\noindent
\begin{multline*}
A(x)\le\fr{1}{6}\,, \enskip A_1(K)>0\,,\enskip  A_3(K)>0\,,\\
 A_7(x)>0\,,\enskip 
A_{14}(x)\ge1\,.
\end{multline*}
\textit{Тогда для всех $n\ge1$}

\noindent
\begin{multline*}
|x|^{\dd}\left|F_n(x)-\Phi(x)\right|
\le
Q_n(x;\,a,b,\gamma,K)\Lo+{}\\
{}+R(x;\,a,b,\gamma,K)\frac{1}{\sqrt{n}}\,.
\end{multline*}

\smallskip

Приведем две мажоранты функции $Q_n(x;\,a,b,\gamma,K)$,
не зависящие от~$n$.

Во-первых, очевидно, что
\begin{multline}
Q_n(x;\,a,b,\gamma,K)\le Q_1(x;\,a,b,\gamma,K)={}\\
{}
=\fr{1}{\gamma^{\dd}}+\fr{A_2(x)}{A_1(K)}\left[1+A_{15}(x)+
\fr{0{,}3041}{A_7^{3/2}(x)}\right]+{}\\
{}+2A_{16}(x)|x|^{\dd}\exp\left\{-(1-\gamma^2)\fr{x^2}{2}\right\}\,.\label{e24gv}
\end{multline}


Во-вторых, с учетом того, что в сделанных предположениях о
моментах случайной величины~$X_1$ всегда $\betr\ge1$, из
соотношения~(\ref{e9gv}) вытекает неравенство
$$
\fr{1}{\sqrt{n}}\le\fr{|x|^{\dd}}{a}\exp\left\{-\fr{x^2}{2b}\right\}\,.
$$
Поэтому
\begin{multline}
Q_n(x;\,a,b,\gamma,K)\le Q'(x;\,a,b,\gamma,K)\equiv{}\\
{}\equiv
\fr{1}{\gamma^{\dd}}+\fr{A_2(x)}{A_1(K)}\bigg[1+{}\\
{}+A_{15}(x)+
\fr{0{,}3041}{A_7^{3/2}(x)}\bigg(\fr{|x|^{\dd}}{a}\exp\left\{-\fr{x^2}{2b}\right\}\bigg)\bigg]
+{}\\
{}+
2A_{16}(x)|x|^{\dd}\exp\left\{-(1-\gamma^2)\fr{x^2}{2}\right\}\,.\label{e25gv}
\end{multline}

Наконец, пытаясь ограничить $Q_n(x;\,a,b,\gamma,K)$, вместо
неравенства~(\ref{e2gv}) при оценивании величины~$I_2$ можно
воспользоваться неравенством~(\ref{e1gv}) с наилучшей известной на
сегодняшний день оценкой константы $C_0$: $C_0 \le 0{,}4784$~\cite{KorolevBEs}. 
Тогда из~(\ref{e1gv}), (\ref{e9gv}) и~(\ref{e16gv}) при $A_{14}(x)\ge1$
будет следовать оценка

\noindent
\begin{multline*}
I_2\le\bigg|{\sf
P}\big(S_n^*<x\sqrt{n}\big)-\Phi\bigg(\fr{x\sqrt{n}-{\sf
E}S_n^*}{\sqrt{{\sf
D}S_n^*}}\bigg)\bigg|+{}\\
{}+\Phi\bigg(\!-\fr{x\sqrt{n}-{\sf
E}S_n^*}{\sqrt{{\sf D}S_n^*}}\!\bigg)\le
0{,}4784\fr{A_{10}(x)x^{\dd}}{aA_7^{3/2}(x)}\exp\Big\{-\frac{x^2}{2b}\Big\}+{}\\
{}+
\fr{1}{\sqrt{2\pi}A_{14}(x)}\exp\left\{-\fr{A_{14}^2(x)}{2}\right\}\equiv
A'_{15}(x)\,.
\end{multline*}
При этом в~(\ref{e23gv}) и последующих выкладках и вычислениях вместо
$Q_n(x;\,a,b,\gamma,K)$ cоответственно следует использовать
величину

\noindent
\begin{multline}
Q''(x;\,a,b,\gamma,K)=\fr{1}{\gamma^{\dd}}+\fr{A_2(x)}{A_1(K)}\left[1+A'_{15}(x)\right]
+{}\\
{}+2A_{16}(x)|x|^{\dd}\exp\left\{-(1-\gamma^2)\fr{x^2}{2}\right\}\,.\label{e26gv}
\end{multline}

Положим

\noindent
\begin{multline*}
Q(x;\,a,b,\gamma,K)=\min\left\{Q_1(x;\,a,b,\gamma,K),\right.\\
\left.Q'(x;\,a,b,\gamma,K),\,Q''(x;\,a,b,\gamma,K)\right\}\,.
\end{multline*}

%\medskip

\noindent
\textbf{Следствие 2.1}. \textit{В условиях леммы~$2.2$ для всех $n\ge1$
справедливо неравенство}

\noindent
\begin{multline*}
|x|^{\dd}\left|F_n(x)-\Phi(x)\right|\le
Q(x;\,a,b,\gamma,K)\Lo+{}\\
{}+R(x;\,\d,a,b,\gamma,K)\fr{1}{\sqrt{n}}\,.
\end{multline*}

\subsection{Неравномерные оценки скорости сходимости в центральной предельной теореме}

Положим

\noindent
\begin{align*}
U_n&=\min_{a,b,\gamma,K}\max\Big\{C_1
K^{\dd}\,,\\
&\hspace*{-8mm}\max_{K\le|x|\le\sqrt{c_n(x;\,a,b)}}Q_n(x;\,a,b,\gamma,K),\,
P(a,b,K)\Big\}\,;\\
U&=\min_{a,b,\gamma,K}\max\Big\{C_1
K^{\dd}\,,\\
&\hspace*{-8mm}\max_{K\le|x|\le\sqrt{c_n(x;\,a,b)}}Q(x;\,a,b,\gamma,K),\,
P(a,b,K)\Big\}\,,
\end{align*}
где минимум берется по множеству значений вспомогательных
параметров, описанному в формулировке леммы~2. Значения параметров
$a$, $b$, $\gamma$, $K$, доставляющие вышеуказанные минимумы,
обозначим соответственно $a^{(n)}_0$, $b^{(n)}_0$,
$\gamma^{(n)}_0$, $K^{(n)}_0$ и $a_0$, $b_0$, $\gamma_0$, $K_0$.
Положим

\noindent
\begin{multline*}
R^{(n)}_0={}\hspace*{60mm}\\
=\!\!\!\max_{K^{(n)}_0\le|x|\le\sqrt{c_n(x;\,a^{(n)}_0,b^{(n)}_0)}}R(x;a^{(n)}_0,b^{(n)}_0,\gamma^{(n)}_0,K^{(n)}_0);\hspace*{-2.76pt}
\end{multline*}
\begin{gather*}
R_0=\max_{K_0\le|x|\le\sqrt{c_n(x;\,a_0,b_0)}}R(x;a_0,b_0,\gamma_0,K_0)\,;
\\
D_n=U_n+C_1,\enskip D=U+C_1\,;
\\
V_n=\max\left\{R^{(n)}_0,\,C_1(K^{(n)}_0)^{\dd}\right\}+C_1\,;
\\
V(n)=\max\left\{R_0,\,C_1 K_0^{\dd}\right\}+C_1.
\end{gather*}
Очевидно, что $V_n\le V_1$ и $V(n)\le V(1)\equiv V$.

Из соотношения~(\ref{e6gv}), лемм~2.1 и~2.2 вытекает следующее утверждение.

\medskip

\noindent
\textbf{Теорема 2.1.} \textit{Для всех $x\in\R$ и всех $n\ge1$
справедливы неравенства:}
\begin{align*}
\left(1+|x|^{\dd}\right)\left|F_n(x)-\Phi(x)\right|&\le
D_n\Lo+\fr{V_n}{n^{\d/2}}\,;
\\
\left(1+|x|^{\dd}\right)\left|F_n(x)-\Phi(x)\right|&\le
D\Lo+\fr{V(n)}{n^{\d/2}}\,.
\end{align*}

\medskip

\noindent
\textbf{Следствие 2.2} \textit{Для всех $x\in\R$ и всех $n\ge1$
справедливо неравенство}
$$
\left(1+|x|^{\dd}\right)\left|F_n(x)-\Phi(x)\right|\le
D\Lo+\fr{V}{n^{\d/2}}\,.
$$

\medskip

\noindent
\textbf{Замечание 2.1} На практике процедуру поиска оптимальных
значений $a$, $b$, $\gamma$, $K$ можно организовать следующим
образом. Поскольку выражение $C_1(1)K^{\dd}$ не зависит от $a$,
$b$, $\gamma$, сначала можно найти значение $K^*=K^*(a,b)$ из
условия
$$
C_1(K^*)^{\dd}=P(a,b,K^*)\,,
$$
затем найти пару $(a^*,b^*)$ из условия
$$
(a^*,b^*)=\mathrm{arg}\,\min_{a,b}P\left(a,b,K^*(a,b)\right)\,,
$$
а затем провести оптимизацию
\begin{align*}
&\hspace*{-5mm}\max_{
K^*(a^*,b^*)\le|x|\le\sqrt{c_n(x;\,a^*,b^*)}}Q_n(x;\,a^*,b^*,\gamma,\\
&\hspace*{35mm}K^*(a^*,b^*)) \longrightarrow\min_{\gamma}\,,
\\
&\hspace*{-5mm}\max_{
K^*(a^*,b^*)\le|x|\le\sqrt{c_n(x;\,a^*,b^*)}}Q(x;\,a^*,b^*,\gamma, \\
&\hspace*{35mm}K^*(a^*,b^*)) \longrightarrow\min_{\gamma}\,,
\\
&\hspace*{-5mm}\max_{
K^*(a^*,b^*)\le|x|\le\sqrt{c_n(x;\,a^*,b^*)}}\bar{Q}\left(x;\,a^*,b^*,\gamma,\right.\\
&\hspace*{35mm}\left.K^*(a^*,b^*)\right) \longrightarrow\min_{\gamma}\,.
\end{align*}
При этом вычисления показывают, что все три вышеперечисленных
максимума достигаются в точке $x = K^*(a^*,b^*).$

\smallskip

\noindent
\textbf{Замечание 2.2} Конкретные вычисления показали, что
значения функций $Q(x,a,b,\gamma,K)$ и $R(x,a,b,\gamma,K)$
оказываются строго меньшими, чем значения
$P(a,b,\gamma,K)$ и $C_1K^3,$ поэтому максимум в выражениях для
$U$ и $V(n)$ определяется лишь значениями функций
$P(a,b,\gamma,K)$ и $C_1K^3,$ а следовательно, $D$ и $V$
совпадают, причем оптимальное значение $D$ не превосходит~22,7707.
Таким образом, приведенная в следствии~2.2 оценка имеет структуру,
аналогичную оценке~(\ref{e2gv}), и справедливо следующее
утверждение.

\smallskip

\noindent
\textbf{Следствие 2.2\boldmath{$'$}} \textit{Для всех $x\in\R$ и всех $n\ge1$
справедливо неравенство}

\noindent
$$
(1+|x|^3)\left|F_n(x)-\Phi(x)\right|\leqslant
22{,}7707\fr{\beta_3+1}{n^{1/2}}\,.
$$

\smallskip

Следствие 2.2$'$ позволяет уточнить константу в неравномерной
оценке скорости сходимости в ЦПТ для пуассоновских случайных сумм,
чему посвящен следующий раздел.

\vspace*{-3pt}

\section{Неравномерная оценка скорости сходимости в~центральной предельной
теореме для~пуассоновских случайных сумм}

Пусть теперь $X_1, X_2, \ldots$~--- последовательность независимых
одинаково распределенных случайных величин таких, что
\begin{equation}
{\sf E}X_1\equiv\mu\,, \ \ {\sf D}X_1 \equiv \sigma^2>0\,,\ \ {\sf
E}|X_1|^{\dd} \equiv \betr<\infty\,. \label{e27gv}
\end{equation}
Пусть $N_\lambda$~--- случайная величина, имеющая распределение
Пуассона с параметром $\lambda>0.$ Предположим, что при каждом~$\lambda>0$ 
случайные величины $N_\lambda, X_1, X_2,\ldots$
независимы.

Рассмотрим пуассоновскую случайную сумму
$$
S_{\lambda} = X_1+\cdots+ X_{N_\lambda}\,.
$$
Для определенности полагаем, что  $S_\lambda = 0$ при $N_\lambda =
0.$ Несложно видеть, что в рас\-смат\-ри\-ва\-емых условиях на моменты
случайной величины~$X_1$ справедливы соотношения
$$
{\sf E}S_\lambda=\lambda\mu\,, \enskip {\sf
D}S_\lambda=\lambda(\mu^2+\sigma^2)\,.
$$
Функцию распределения стандартизованной пуассоновской случайной
суммы
$$
\widetilde{S}_\lambda
\equiv\fr{S_\lambda-\lambda\mu}{\sqrt{\lambda(\mu^2+\sigma^2)}}
$$
обозначим~$F_\lambda(x)$.

\columnbreak

Пуассоновские случайные суммы~$S_{\lambda}$ являются очень
популярными математическими моделями многих объектов и процессов в
самых разных областях, в том числе в страховании, где они
используются для описания суммы страховых требований, поступивших
в течение определенного периода времени, в теории управления
запасами, где они описывают суммарные заявки на продукт. При
анализе информационных систем также традиционным предположением
является пуассоновский характер потока заявок (клиентов,
требований, задач, сообщений), так что суммарные характеристики
заявок в информационных системах имеют вид пуассоновских случайных
сумм. Задаче изуче\-ния точ\-ности нормальной аппроксимации для
распределений пуассоновских случайных сумм~--- так называемых
обобщенных пуассоновских распределений~--- посвящена обширная
литература (см., например, библиографию в книгах~\cite{KorBenShorg} и~\cite{BenKor2002}).

В работе~\cite{Mich93} показано, что для любых $x\in\R$
и любых $n\ge1$ справедливо неравенство
\begin{multline}
\left(1+|x|^{3}\right)\left|F_{\lambda}(x)-\Phi(x)\right|\le{}\\
{}\le
C\cdot\fr{\betr}{\lambda^{\delta/2}(\mu^2+\sigma^2)^{1+\delta/2}}\,,\label{e28gv}
\end{multline}
где $C$~--- та же константа, что и в <<классической>> оценке~(\ref{e5gv}). 
В~данном разделе будет показано, что на самом деле неравенство~(\ref{e28gv})
справедливо с заменой~$C$ на~$D$ (см.\ следствие~2.2$'$). Для этого
понадобятся некоторые вспомогательные утверждения.

Обозначим
$$
\nu = \fr{\lambda}{n}\,.
$$

\medskip

\noindent
\textbf{Лемма 3.1.} \textit{Распределение пуассоновской случайной суммы
$S_\lambda$ совпадает с распределением суммы $n$ независимых
одинаково распределенных случайных величин, каким бы ни было
натуральное число $n\geq1:$}
$$
X_1+\cdots+X_{N_\lambda} \stackrel{d}{=} Y_{\nu,1}+\cdots+Y_{\nu,n}\,,
$$
\textit{где при каждом $n$ случайные величины $Y_{\nu,1},\ldots,
Y_{\nu,n}$ независимы и одинаково распределены. При этом если
случайная величина $X_1$ удовлетворяет условиям}~(\ref{e27gv}), \textit{то для
моментов случайной величины $Y_{\nu,1}$ имеют мес\-то соотношения:}
\begin{gather*}
{\sf E}Y_{\nu,1} = \mu\nu\,; \quad {\sf D}Y_{\nu,1} =
(\mu^2+\sigma^2)\nu\,;\\
{\sf E}|Y_{\nu,1} - \mu\nu|^{3}\leq \nu\beta_{3}(1+40\nu) \,, \quad
n\geq\lambda\,.
\end{gather*}

\medskip

Д\,о\,к\,а\,з\,а\,т\,е\,л\,ь\,с\,т\,в\,о\,\ см.\ в работе~\cite{Shev2007}.

\medskip

\noindent
\textbf{Следствие 3.1} \textit{Если выполнены условия $(27),$ то для
любого $n=1,2, \ldots$}
$$
\widetilde{S}_\lambda \stackrel{d}{=}
\fr{1}{\sqrt{n}}\sum_{k=1}^n Z_{\nu,k}\,,
$$
\textit{где при каждом $n$ случайные величины $Z_{\nu,1},\ldots,
Z_{\nu,n}$ независимы и одинаково распределены. Более того, эти
случайные величины имеют нулевое среднее и единичную дисперсию и
при всех $n\geq\lambda$}
\begin{equation}
{\sf E}|Z_{\nu,1}|^{3}\le
\fr{\beta_{3}(1+40\nu)}{(\mu^2+\sigma^2)^{3/2}}\left(\fr{n}{\lambda}\right)^{1/2}\,.
\label{e29gv}
\end{equation}

\medskip

Следующее утверждение представляет собой основной результат данной
статьи.

\medskip

\noindent
\textbf{Теорема 3.1} \textit{При условиях}~(\ref{e27gv}) \textit{для любого $\lambda>0$
справедливо неравенство}
$$
\sup_{x\in\R}(1+|x|^{3})\left|F_\lambda(x) - \Phi(x)\right|\leq
\fr{D\betr}{\lambda^{1/2}(\mu^2+\sigma^2)^{3/2}}\,,
$$
\textit{где константа~$D$ та же, что и в следствии~$2.2'$,
т.\,е.\ $D\le22.7707$.}

\medskip

\noindent
Д\,о\,к\,а\,з\,а\,т\,е\,л\,ь\,с\,т\,в\,о\,.\ Из леммы~3.1 и следствия~3.1
вытекает, что для любого целого $n\geq1$
$$
|F_\lambda(x) - \Phi(x)| = \left|{\sf
P}\left(\fr{1}{\sqrt{n}}\sum_{k=1}^n
Z_{\nu,k}<x\right)-\Phi(x)\right|\,.
$$
Следовательно, в силу следствия~2.2 для произвольного целого
$n\geq1$ при каждом фиксированном $x\in\R$ имеем
\begin{equation}
\left(1+|x|^{3}\right)\left|F_\lambda(x) - \Phi(x)\right| \leq D\fr{{\sf
E}|Z_{\nu,1}|^{3}}{n^{1/2}}+\fr{V}{n^{\d/2}}\,. \label{e30gv}
\end{equation}
Поскольку в~(\ref{e30gv}) $n$ произвольно, можно считать, что
$n\geq\lambda.$ Тогда, используя оценку~(\ref{e29gv}), в продолжение~(\ref{e30gv})
получаем неравенство
\begin{multline*}
\left(1+|x|^{3}\right)\left|F_\lambda(x) - \Phi(x)\right|\leq{}\\
{}\leq
\fr{D\betr}{\lambda^{1/2}(\mu^2+\sigma^2)^{3/2}}
\left(1+40\fr{\lambda}{n}\right)+\fr{V}{n^{\d/2}}\,.
\end{multline*}
Так как здесь $n\geq\lambda$ произвольно, устремляя
$n\rightarrow\infty,$ окончательно получаем
\begin{multline*}
\left(1+|x|^{3}\right)\left|F_\lambda(x) - \Phi(x)\right|\leq{}\\
{}\leq
\lim_{n\rightarrow\infty}\left[\fr{D\betr}{\lambda^{1/2}(\mu^2+\sigma^2)^{3/2}}
\left(1+40\fr{\lambda}{n}\right)+\fr{V}{n^{\d/2}}\right]={}\\
{}=
\fr{D\betr}{\lambda^{1/2}(\mu^2+\sigma^2)^{3/2}}\,,
\end{multline*}
что и требовалось доказать.

\section{Неравномерные оценки скорости сходимости в~предельных
теоремах для~смешанных пуассоновских случайных сумм}

Пусть $\Lambda_t$~--- положительная случайная величина, функция
распределения $G_t(x)={\sf P}(\Lambda_t<x)$ которой зависит от
некоторого параметра $t>0$. Под смешанным пуассоновским
распределением со структурным распределением~$G_t$ будем
подразумевать распределение случайной величины~$N(t)$, принимающей
целые неотрицательные значения с вероятностями

\noindent
$$
{\sf P}\left(N(t)=k\right)=\fr{1}{k!}\int\limits_{0}^{\infty}
e^{-\lambda}\lambda^kdG_t(\lambda),\enskip k=0,1,2,\ldots
$$
Известно несколько конкретных примеров смешанных пуассоновских
распределений, наиболее широко используемым среди которых, пожалуй,
является отрицательное биномиальное распределение (это распределение
было использовано в виде смешанного пуассоновского еще в работе~\cite{Greenwood1920} 
для\linebreak моделирования частоты несчастных
случаев на\linebreak производстве). Отрицательное биномиальное распределение
порождается структурным гам\-ма-рас\-пре\-де\-ле\-ни\-ем. Другими примерами
смешанных пуассоновских распределений являются распределение
Делапорте, порождаемое сдвинутым гам\-ма-струк\-тур\-ным распределением~\cite{Delaporte}, 
распределение Зихеля, порожденное
обратным нормальным структурным распределением~\cite{Holla}--\cite{Willmot}, обобщенное
распределение Варинга~\cite{Irwin, Seal}. Свойства смешанных пуассоновских
распределений подробно описаны в книгах~\cite{Grandell} и~\cite{BenKor2002}.

Пусть, как и ранее, $X_1,X_2,\ldots$~--- независимые одинаково
распределенные случайные величины. Предположим, что случайные
величины $N(t),X_1,X_2,\ldots$ независимы при каждом $t>0$. Положим

\noindent
$$
S(t)=X_1+ \ldots +X_{N(t)}
$$
(как и ранее, для определенности будем считать, что $S(t)=0$, если
$N(t)=0$). Случайную величину~$S(t)$ будем называть смешанной
пуассоновской случайной суммой, а ее распределение~--- обобщенным
смешанным пуассоновским.

Асимптотическое поведение смешанных пуассоновских случайных сумм~$S(t)$, когда $N(t)$ 
в определенном смысле неограниченно
возрастает, принципиально различно в зависимости от того, равно
нулю математическое ожидание~$\mu$ слагаемых или нет.

Сходимость по распределению и по вероятности будет обозначаться
символами $\Longrightarrow$ и~$\pto$ соответственно.

В этом разделе сосредоточимся на ситуации, когда ${\sf E}X_1=0$. 
В~таком случае предельными распределениями для стандартизованных
смешанных пуассоновских случайных сумм являются масштабные смеси
нормальных законов. Не ограничивая общности, будем считать, что
${\sf D}X_1=1$.

\smallskip

\noindent
\textbf{Теорема 4.1} \cite{Korolev96, BenKor2002}. \textit{Предположим, что
$\Lambda_t\pto\infty$ при $t\to\infty$. Тогда для положительной
неограниченно возрастающей функции~$d(t)$ существует функция
распределения $H(x)$ такая, что}
$$
{\sf P}\left(\fr{S(t)}{\sqrt{d(t)}}<x\right)\Longrightarrow H(x)\
\enskip (t\to\infty)\,,
$$
\textit{в том и только в том случае, когда существует функция
распределения~$G(x)$ такая, что при той же функции~$d(t)$}
\begin{equation}
G_t\left(xd(t)\right)\Longrightarrow G(x) \enskip (t\to\infty)\label{e31gv}
\end{equation}
\textit{и}
$$
H(x)=\int\limits_{0}^{\infty}\Phi\left(\frac{x}{\sqrt{y}}\right)dG(y)\,,\enskip
x\in\R\,.
$$

\smallskip

Легко видеть, что распределение смешанной пуассоновской случайной
суммы~$S(t)$ можно записать в виде
\begin{multline}
{\sf P}(S(t)<x)=\int_{0}^{\infty}{\sf
P}\left(\sum_{j=1}^{N_{\lambda}}X_j<x\right)dG_t(\lambda)\,,\\
x\in\R\,,
\label{e32gv}
\end{multline}
где $N_\lambda$~--- пуассоновская случайная величина с параметром
$\lambda>0$, такая что при каждом $\lambda>0$ случайные величины
$N_\lambda,X_1,X_2,\ldots$ независимы. (Следует заметить, что запись
приводимых соотношений в терминах случайных величин использована
лишь для удобства и наглядности. На самом деле речь идет о
соответствующих соотношениях для распределений, но такая форма
записи оказывается более громоздкой. Поэтому предположение о
существовании вероятностного пространства, на котором определены
упомянутые выше случайные величины с указанными свойствами, ни в
коей мере не ограничивает общности.) Пусть
\begin{equation}
{\sf E}X_1=0\,,\ \ {\sf E}X_1^2=1\,,\ \ \beta^3={\sf
E}|X_1|^3<\infty\label{e33gv}
\end{equation}
и $d(t)$, $t>0$,~--- некоторая положительная неограниченно
возрастающая функция. Равномерные оценки скорости сходимости в
теореме~4.1 приведены в работах~\cite{KorolevBEs, KorSchev, Gavrilenko} (см.\ также~\cite{KorBenShorg}). 
В~данном разделе будут приведены неравномерные
оценки скорости сходимости в теореме~4.1 и ее частных случаях.

Пусть $G(x)$~--- функция распределения такая, что $G(0)=0$. Если
выполнено условие~(\ref{e31gv}), то в соответствии с теоремой~4.1
обобщенные смешанные пуассоновские распределения случайной
величины~$S(t)$, нормированной квадратным корнем из функции~$d(t)$, 
сходятся к масштабным смесям нормальных законов, в которых
смешивающим распределением является~$G(x)$. Обозначим
\begin{gather*}
\Delta_t(x)=\left|{\sf P}\left(\fr{S(t)}{\sqrt{d(t)}}<x\right)-
\int\limits_{0}^{\infty}
\Phi\left(\fr{x}{\sqrt{\lambda}}\right)dG(\lambda)\right|\,,\\
\delta_t(x)=G_t\left(d(t)x\right)-G(x)\,.
\end{gather*}

\smallskip

\noindent
\textbf{Теорема 4.2.} \textit{Предположим, что выполнены условия}~(\ref{e28gv}).
\textit{Тогда при каждом $t>0$ при любом $x\in\R$ имеет место оценка}
\begin{multline}
\Delta_t(x)\le 22{,}7707\fr{\beta_3}{\sqrt{d(t)}}\,{\sf
E}\left\{\fr{\Lambda_t}{d(t)}\left[\left(\fr{\Lambda_t}{d(t)}\right)^{3/2}+{}\right.\right.\\
\left.\left.{}+|x|^3
\vphantom{\fr{\Lambda_t}{d(t)}}\right]^{-1}\right\}+
\int\limits_0^{\infty}|\delta_t(\lambda)|\,d_{\lambda}\Phi\left(\fr{x}{\sqrt{\lambda}}\right)\,.\label{e34gv}
\end{multline}

\smallskip

\noindent
Д\,о\,к\,а\,з\,а\,т\,е\,л\,ь\,с\,т\,в\,о\,.\ Идея доказательства аналогична идее
доказательства равномерной оценки в работе~\cite{Gavrilenko} и основана на использовании
представления~(\ref{e32gv}). Имеем
\begin{multline*}
\Delta_t(x)\le\int\limits_0^{\infty}\left|{\sf
P}\left(\fr{1}{\sqrt{\lambda d(t)}}\sum_{j=1}^{N_{\lambda
d(t)}}X_j<\fr{x}{\sqrt{\lambda}}\right)-{}\right.\\
\left.{}-\Phi\left(\fr{x}{\sqrt{\lambda}}\right)\right|\,dG_t\left(\lambda
d(t)\right)+
\left|\int\limits_0^{\infty}\Phi\left(\fr{x}{\sqrt{\lambda}}\right)\,d\delta_t(\lambda)\right|\equiv{}\\
{}\equiv
I_1(t;x)+I_2(t;x)\,.
\end{multline*}
Подынтегральное выражение в $I_1(t;x)$ оценим с помощью теоремы~3.1 и получим
\begin{multline}
I_1(t;x)\le{}\\
{}\le
D\beta_3\int\limits_0^{\infty}\fr{\lambda^{3/2}}{\sqrt{\lambda
d(t)}\left(\lambda^{3/2}+|x|^3\right)}\,dG_t\left(\lambda d(t)\right)={}\\
{}=
\fr{D\beta_3}{\sqrt{d(t)}}{\sf E}\left\{
\fr{\Lambda_t}{d(t)}\left[\left(\fr{\Lambda_t}{d(t)}\right)^{3/2}+|x|^3\right]^{-1}\right\}\,.\label{e35gv}
\end{multline}
Интегрируя по частям, получаем
\begin{equation}
I_2(t;x)\le\int\limits_0^{\infty}|\delta_t(\lambda)|\,d_{\lambda}\Phi\left(\fr{x}{\sqrt{\lambda}}\right)\,.\label{e36gv}
\end{equation}
Требуемое утверждение вытекает из~(\ref{e35gv}) и~(\ref{e36gv}). Тео\-ре\-ма доказана.

\smallskip

В качестве примера применения теоремы~4.2 рассмотрим случай, когда
при каждом $t>0$ случайная величина $\Lambda_t$ имеет
гам\-ма-рас\-пре\-де\-ле\-ние. Этот случай представляет особый интерес с
точки зрения его применения в финансовой мате-\linebreak матике для
асимптотического обоснования адекватности таких популярных моделей
эволюции\linebreak финансовых индексов, как дисперсионные гам\-ма-про\-цес\-сы Леви
(variance-gamma L$\acute{\mbox{e}}$vy processes, VG-processes)~\cite{Madan} или
двусторонние гам\-ма-про\-цес\-сы~\cite{Carr} (также см.~\cite{Korolev2010}).

Смешанное пуассоновское распределение со структурным
гам\-ма-рас\-пре\-де\-ле\-ни\-ем является не чем иным, как отрицательным
биномиальным распределением. Убедимся в этом. Плотность
гам\-ма-рас\-пре\-де\-ле\-ния с параметром формы $r>0$ и параметром масштаба
$\sigma>0$, как известно, имеет вид
$$
g_{r,\si}(x)=\fr{\si^r}{\Gamma(r)}e^{-\si x}x^{r-1}\,,\enskip x>0\,.
$$
Таким образом, смешанное пуассоновское распределение со
структурным гам\-ма-рас\-пре\-де\-ле\-ни\-ем имеет характеристическую функцию
\begin{multline*}
\psi(z)=\int\limits_{0}^{\infty}\exp\left\{y(e^{iz}-1)\right\}\fr{\si^r}{\Gamma(r)}
e^{-\si y}y^{r-1}\,dy={}\\
{}=
\fr{\si^r}{\Gamma(r)}\int\limits_{0}^{\infty}\exp\left\{-\si
y\left(1+ \fr{1-e^{iz}}{\si}\right)\right\}y^{r-1}\,dy={}\\
{}=
\left(1+\fr{1-e^{iz}}{\si}\right)^{-r}\,.
\end{multline*}
Вводя новую параметризацию
$$
\si=\fr{p}{1-p}\  \ \left(p=\fr{\si}{1+\si}\right)\,,\ \ 
p\in(0,1)\,,
$$
окончательно получаем
$$
\psi(z)=\left(\fr{p}{1-(1-p)e^{iz}}\right)^r\,,\enskip z\in\R\,,
$$
что совпадает с характеристической функцией отрицательного
биномиального распределения с параметрами $r>0$ и $p\in(0,1)$.

Функцию гамма-распределения с параметром масштаба $\si$ и
параметром формы $r$ обозначим $G_{r,\si}(x)$. Несложно убедиться,
что
\begin{equation}
G_{r,\si}(x)\equiv G_{r,1}(\si x)\,.\label{e37gv}
\end{equation}
Случайную величину с функцией распределения $G_{r,\si}(x)$
обозначим $U(r,\si)$. Хорошо известно, что
$$
{\sf E}U(r,\si)=\fr{r}{\si}\,.
$$
Зафиксируем параметр $r$ и в качестве структурной случайной
величины~$\Lambda_t$ возьмем $U(r,\si)$, предполагая, что
$t=\si^{-1}$:
$$
\Lambda_t=U(r,t^{-1})\,.
$$
В качестве функции~$d(t)$ возьмем $d(t)\equiv{\sf E}\Lambda_t=$\linebreak $={\sf
E}U(r,t^{-1})$. Очевидно, что ${\sf E}U(r,t^{-1})=rt$. Тогда с
учетом~(\ref{e37gv}) будем иметь
\begin{multline*}
G_t\left(xd(t)\right)={\sf P}\left(U(r,t^{-1})<xrt\right)={}\\
{}={\sf P}\left(U(r,1)<xr\right)={\sf P}\left(U(r,r)<x\right)=G_{r,r}(x)\,.
\end{multline*}
Заметим, что функция распределения в правой час\-ти последнего
соотношения не зависит от~$t$. Поэтому указанный выбор функции
$d(t)$ тривиальным образом гарантирует выполнение условия~(\ref{e31gv})
теоремы~4.1. Более того, в таком случае $\delta_t(x)=0$ для всех
$t>0$ и $x\in\R$.

При этом случайная величина~$N(t)$ имеет отрицательное
биномиальное распределение с па\-ра\-мет\-ра\-ми~$r$ и $p=(t+1)^{-1}$:
\begin{multline}
{\sf P}\left(N(t)=k\right)=C_{r+k-1}^{k}p^r(1-p)^k
={}\\
{}=C_{r+k-1}^{k}\fr{t^k}{(1+t)^{r+k}}\,,\enskip k=0,1,2,\ldots
\label{e38gv}
\end{multline}
Здесь для нецелых $r$ величина $C_{r+k-1}^{k}$ определяется как
$$
C_{r+k-1}^{k} = \fr{\Gamma(r+k)}{k!\Gamma(r)}\,.
$$

Итак, в рассматриваемом случае $G_t\big(xd(t)\big)\equiv$\linebreak $\equiv
G_{r,r}(x)$, $d(t)\equiv rt$ и второе слагаемое в правой час\-ти~(\ref{e34gv}) 
равно нулю. Вычислим первое слагаемое в правой части~(\ref{e34gv}) для
рассматриваемой ситуации в предположении, что
\begin{equation}
r>\fr{1}{2}\,.\label{e39gv}
\end{equation}
Имеем
\begin{multline}
\fr{1}{\sqrt{d(t)}}\,{\sf
E}\left\{\fr{\Lambda_t}{d(t)}\left[\left(\fr{\Lambda_t}{d(t)}\right)^{3/2}+|x|^3\right]^{-1}\right\}={}\\
{}=
\fr{1}{\sqrt{rt}}\int\limits_0^{\infty}\fr{y}{y^{3/2}+|x|^3}\,dG_{r,r}(x)={}\\
{}=
\fr{r^{r-1/2}}{\sqrt{t}\Gamma(r)}\int\limits_0^{\infty}\fr{y^re^{-ry}\,dy}{y^{3/2}+|x|^3}\,.
\label{e40gv}
\end{multline}

Приведем несколько оценок для правой час\-ти~(\ref{e40gv}). Во-пер\-вых,
несложно убедиться, что при $y>0$
$$
\sup_x\fr{1+|x|^3}{y^{3/2}+|x|^3}=\max\{1,\,y^{-3/2}\}\,.
$$
Поэтому
\begin{multline*}
J_r(x)\equiv\int\limits_0^{\infty}\fr{y^re^{-ry}dy}{y^{3/2}+|x|^3}\le{}\\
{}\le
\fr{1}{1+|x|^3}
\int\limits_0^{\infty}y^re^{-ry}\sup_x\left\{\fr{1+|x|^3}{y^{3/2}+|x|^3}\right\}\,dy\le{}\\
{}\le
\fr{1}{1+|x|^3}\left[\int\limits_0^1y^{r-3/2}e^{-ry}\,dy+\int\limits_1^{\infty}y^re^{-ry}\,dy\right]
={}\\
{}=
\fr{1}{1+|x|^3}\left[\fr{\gamma_r(r-{1}/{2})}{r^{r-1/2}}+\fr{\Gamma(r+1)-\gamma_r(r+1)}{r^{r+1}}\right]\,,
\end{multline*}
где $\gamma_x(\alpha)$~--- неполная гам\-ма-функ\-ция,
$$
\gamma_x(\alpha)=\int\limits_0^x e^{-z}z^{\alpha-1}\,dz\,,\ \ \
\alpha>0\,,\ x\ge0\,.
$$
Следовательно,
\begin{multline*}
\fr{r^{r-1/2}}{\sqrt{t}\Gamma(r)}J_r(x)\le\fr{1}{\sqrt{t}(1+|x|^3)}\left[
\fr{\gamma_r(r-{1}/{2})}{\Gamma(r)}+{}\right.\\
\left.{}+
\fr{1}{\sqrt{r}}-\fr{\gamma_r(r+1)}{\Gamma(r+1)\sqrt{r}}\right]\,.
\end{multline*}
Таким образом, справедливо

\smallskip

\noindent
\textbf{Следствие 4.1.} \textit{Пусть случайная величина $N(t)$ имеет
отрицательное биномиальное распределение}~(\ref{e38gv}), $t>0$.
\textit{Предположим, что выполнены условия}~(\ref{e33gv}) \textit{и}~(\ref{e39gv}). \textit{Тогда для
любых $t>0$ и $x\in\R$ справедлива оценка}

\noindent
\begin{multline*}
\left|{\sf P}(S(t)<x\sqrt{rt})-\int\limits_{0}^{\infty}
\Phi\left(\fr{x}{\sqrt{y}}\right)dG_{r,r}(y)\right|\le{}\\
{}\le \fr{22{,}7707\beta_3}{\sqrt{t}(1+|x|^3)}
K_1(r)\,,
\end{multline*}
\textit{где}
$$
K_1(r)=\fr{\gamma_r(r-{1}/{2})}{\Gamma(r)}+
\fr{1}{\sqrt{r}}-\fr{\gamma_r(r+1)}{\Gamma(r+1)\sqrt{r}}\,.
$$

\smallskip

С другой стороны, очевидно, что
$\gamma_x(\alpha)\le\Gamma(\alpha)$ при любых $x\ge0$ и
$\alpha>0$. Поэтому величину $K_1(r)$ можно оценить выражением,
содержащим только <<полные>> гам\-ма-функ\-ции:
\begin{equation}
K_1(r)\le \fr{\Gamma(r-{1}/{2})}{\Gamma(r)}+
\fr{1}{\sqrt{r}}\,.\label{e41gv}
\end{equation}

\smallskip

При $r=1$ случайная величина $\Lambda_t=U(1,t^{-1})$ имеет
показательное распределение с параметром $\sigma=1/t$.
Следовательно, как легко убедиться, смешанная пуассоновская
случайная величина~$N(t)$ с таким структурным распределением имеет
геометрическое распределение с параметром
$p=(t+1)^{-1}$. При этом предельное (при 
$t\rightarrow \infty$) распределение для стандартизованной
геометрической суммы~$S(t)$ является распределение Лапласа с
плотностью
$$
l(x)=\fr{1}{\sqrt{2}}\,e^{-\sqrt{2}\,|x|},\enskip x\in\R\,.
$$
Функцию распределения, соответствующую плотности $l(x)$, обозначим~$L(x)$:
$$
L(x)=\begin{cases}
\fr{1}{2}\,e^{\sqrt{2}x}\,, & \mbox{если}\ \ x<0\,,\\
1-\fr{1}{2}\,e^{-\sqrt{2}x}\,, & \mbox{если}\ \ x\ge0\,.
\end{cases}
$$
Из следствия~4.1 при этом получаем

\smallskip

\noindent
\textbf{Следствие 4.2.} \textit{Пусть случайная величина $N(t)$ имеет
геометрическое распределение с параметром $p=(1+t)^{-1}$, $t>0$.
Предположим, что выполнены условия}~(\ref{e33gv}). \textit{Тогда для любых $t>0$ и
$x\in\R$ справедлива оценка}
$$
\left|{\sf P}(S(t)<x\sqrt{t})-L(x)\right|\le
\fr{50{,}7652\beta_3}{\sqrt{t}\left(1+|x|^3\right)}\,.
$$

\smallskip

Заметим, что оценка, приведенная в следствии~4.2, точнее
равномерной оценки, приведенной в работах~\cite{KorolevBEs} и~\cite{KorSchev}, для
$|x|>4{,}5334$.

Заметим также, что использование оценки~(\ref{e41gv}) для $K_1(r)$
увеличивает абсолютную константу в следствии~4.2 до 63,1308~---
такова цена упрощения.

{\small\frenchspacing
{%\baselineskip=10.8pt
%\addcontentsline{toc}{section}{Литература}
\begin{thebibliography}{99}


\bibitem{Esseen42} 
\Au{Esseen C.\,G.} On the Liapunoff limit of error in
the theory of probability~// Ark. Mat. Astron. Fys., 1942. Vol.~A28.
No.\,9. P.~1--19.

\bibitem{Berry41} 
\Au{Berry A.\,C. } The accuracy of the Gaussian
approximation to the sum of the distributed random variables~// J.~Theor. Probab., 1994. Vol.~2. No.\,2. P.~211--224.

\bibitem{KorolevVOz} 
\Au{Королев В.\,Ю., Шевцова И.\,Г.} Уточнение верхней оценки
абсолютной постоянной в неравенстве Берри--Эссеена для смешанных
пуассоновских случайных сумм~// Докл. РАН,
2010. Т.~431. Вып.~1. С.~16--19.

\bibitem{KorolevBEs} 
\Au{Королев В.\,Ю., Шевцова И.\,Г.} Уточнение неравенства
Берри--Эссеена с приложениями к пуассоновским и смешанным
пуассоновским случайным суммам~// Обозрение прикладной и
промышленной математики, 2010. Т.~17. Вып.~1. С.~25--56.

\bibitem{Ess} \Au{Esseen C.\,G.} 
A moment inequality with an application to
the central limit theorem~// Skand. Aktuarrietidskr, 1956. Vol.~39.
P.~160--170.

\bibitem{Meshalkin} \Au{Мешалкин Л.\,Д., Рогозин Б.\,А.} Оценка расстояния между
функциями распределения по близости их характеристических функций
и ее применение к центральной предельной теореме~// Предельные
теоремы теории вероятностей.~--- Ташкент: АН УзССР, 1963. С.~40--55.

\bibitem{Nagaev} \Au{Нагаев С.\,В.} Некоторые предельные теоремы для больших
уклонений~// Теория вероятностей и ее применения, 1965. Т.~10.
Вып.~2. С.~231--254.

\bibitem{Mich81} \Au{Michel R.} On the constant in the nonuniform version of the
Berry--Esseen theorem~// Z.~Wahrsch. verw. Geb., 1981. Bd.~55. P.~109--117.

\bibitem{KorSchev} \Au{Korolev  V., Shevtsova~I.} An improvement of the
Berry--Esseen inequality with applications to Poisson and mixed
Poisson random sums~// Scandinavian Actuarial J., 2010.
{\sf http://www.informaworld.com/10.1080/03461238.\newline 2010.485370}.

\bibitem{Nefedova} \Au{Нефедова Ю.\,С., Шевцова И.\,Г.} О~точности нормальной
аппроксимации для распределений пуассоновских случайных сумм~//
Информатика и её применения, 2011. Т.~5. Вып.~1. С.~\pageref{nefedova1}--\pageref{end-nefedova}.

\bibitem{Paditz89} \Au{Paditz L.} On the analytical structure of the constant in
the nonuniform version of the Esseen inequality~// Statistics, 1989. Vol.~20. No.~3. P.~453--464.

\bibitem{Mich93} \Au{Michel R.} On Berry--Esseen results for the compound
Poisson distribution~// Insurance: Mathematics and Economics,
1993. Vol.~13. No.\,1. P.~35--37.

\bibitem{Rychlik} \Au{Rychlik Z.} Nonuniform central limit bounds and their
applications~// Теория вероятностей и ее применения, 1983. T.~28.
Вып.~3. С.~646--652.

\bibitem{Paditz81} \Au{Paditz L.} Einseitige Fehlerabsch$\ddot{\mbox{a}}$tzungen im zentralen
Grenzwertsatz~// Math. Operationsforsch. und Statist., ser.
Statist., 1981. Bd.~12. P.~587--604.

\bibitem{Tysiak} \Au{Tysiak W.} Gleichm$\ddot{\mbox{a}}${\ss}ige und
nicht-gleichm$\ddot{\mbox{a}}${\ss}ige Berry--Esseen-Absch$\ddot{\mbox{a}}$tzungen.
Dissertation.~--- Wuppertal, 1983.

\bibitem{KorBenShorg} \Au{Королев В.\,Ю., Бенинг В.\,Е., Шоргин С.\,Я.} 
Математические основы теории риска.~--- М.: Физматлит, 2007.

\bibitem{BenKor2002} \Au{Bening V., Korolev V.}  Generalized Poisson models
and their applications in insurance and finance.~--- Utrecht: VSP, 2002.

\bibitem{Shev2007} \Au{Шевцова И.\,Г.} О~точности нормальной аппроксимации для
распределений пуассоновских случайных сумм~// Обозрение
промышленной и прикладной математики, 2007. Т.~14. Вып.~1. С.~3--28.

\bibitem{Greenwood1920} \Au{Greenwood M., Yule G.\,U.} An inquiry into the nature of
frequency-distributions of multiple happenings, etc.~// J.~Roy.
Statist. Soc., 1920. Vol.~83. P.~255--279.

\bibitem{Delaporte} \Au{Delaporte P.} Un probl$\acute{\mbox{e}}$me de tarification de l'assurance
accidents d'automobile examin$\acute{\mbox{e}}$ par la statistique math$\acute{\mbox{e}}$matique~// 
Trans. 16th  Congress (International) of Actuaries.~--- Brussels,
1960. Vol.~2. P.~121--135.

\bibitem{Holla} \Au{Holla M. S.} On a Poisson-inverse Gaussian distribution~//
Metrika, 1967. Vol.~11. P.~115--121.

\bibitem{Sichel} \Au{Sichel H. S.} On a family of discrete distributions
particular suited to represent long tailed frequency data~// 
3rd Symposium on Mathematical Statistics Proceedings~/ Ed.\ N.\,F.~Laubscher.~---
Pretoria: CSIR, 1971. P.~51--97.

\bibitem{Willmot} \Au{Willmot G.\,E.} The Poisson-inverse Gaussian distribution
as an alternative to the negative binomial~// Scandinavian Actuar.~J., 1987. P.~113--127.

\bibitem{Irwin} \Au{Irwin J.\,O.} The generalized Waring distribution applied to
accident theory~// J.~Royal Statist. Soc., Ser. A, 1968. Vol.~130. P.~205--225.

\bibitem{Seal} \Au{Seal H.} Survival probabilities. The goal of risk
theory.~--- Chichester\,--\,New York\,--\,Brisbane\,--\,Toronto: Wiley, 1978.

\bibitem{Grandell} \Au{Grandell J.} Mixed Poisson processes.~--- London:
Chapman and Hall, 1997.

\bibitem{Korolev96} \Au{Korolev V.\,Yu.} A general theorem on the limit behavior of
superpositions of independent random processes with applications
to Cox processes~// J. Math. Sci., 1996. Vol.~81. No.\,5. P.~2951--2956.

\bibitem{Gavrilenko} \Au{Гавриленко С.\,В., Королев В.\,Ю.} Оценки скорости
сходимости смешанных пуассоновских случайных сумм~// 
Системы и средства информатики. Спец. вып. Математические модели в информационных технологиях.~--- М.: ИПИ РАН,
2006. С.~248--257.

\bibitem{Madan} \Au{Madan D.\,B.,  Seneta E.} The variance gamma ($V.G.$)
model for share market return~// J. Business, 1990. Vol.~63. P.~511--524.

\bibitem{Carr} \Au{Carr P.\,P., Madan D.\,B., Chang~E.\,C.} The variance gamma
process and option pricing~// Eur. Finance Rev., 1998. Vol.~2. P.~79--105.

 \label{end\stat}

\bibitem{Korolev2010} \Au{Королев В.\,Ю.} Ве\-ро\-ят\-ност\-но-ста\-ти\-сти\-че\-ские методы
декомпозиции волатильности хаотических процессов.~--- М.: МГУ, 2010.
 \end{thebibliography}
}
}


\end{multicols}  %10  +
\def\stat{chuprakov}

\def\tit{К ВОПРОСУ О РАЗМЕЩЕНИИ КОЛЛЕКТИВНЫХ СРЕДСТВ 
ОТОБРАЖЕНИЯ В СИТУАЦИОННОМ ЗАЛЕ С~ЗАДАННЫМИ 
ПАРАМЕТРАМИ}

\def\titkol{К вопросу о размещении коллективных средств 
отображения в ситуационном зале с~заданными 
параметрами}

\def\autkol{К.\,Г.~Чупраков}
\def\aut{К.\,Г.~Чупраков$^1$}

\titel{\tit}{\aut}{\autkol}{\titkol}

%{\renewcommand{\thefootnote}{\fnsymbol{footnote}}\footnotetext[1]
%{Исследование поддержано грантами РФФИ 08-07-00152 и 09-07-12032.
%Статья написана на основе материалов доклада, представленного на IV 
%Международном семинаре  <<Прикладные задачи теории вероятностей и математической статистики, 
%связанные с моделированием информационных систем>> (зимняя сессия, Аоста, Италия, январь--февраль 2010~г.).}}

\renewcommand{\thefootnote}{\arabic{footnote}}
\footnotetext[1]{Институт проблем информатики Российской академии наук, chkos@rambler.ru}

\vspace*{6pt}

\Abst{Установлены некоторые зависимости между основными параметрами 
ситуационного зала: его размерами, информативностью отображаемого контента, числом 
одновременных наблюдателей и размерами экрана. Основой для таких зависимостей 
стали рекомендации, закрепленные в ГОСТах, и простые геометрические соображения.}

\vspace*{2pt}

\KW{системы отображения информации; ситуационный зал; область наилучшего 
наблюдения; аналитические зависимости}

\vspace*{4pt}

       \vskip 14pt plus 9pt minus 6pt

      \thispagestyle{headings}

      \begin{multicols}{2}

      \label{st\stat}



\section{Введение}
    
    Использование ситуационных центров доказало их практическую 
значимость для решения задач управления крупными и сложными 
объектами, в числе которых государственные учреждения~[1], а также 
крупные корпорации и предприятия~[2]. Однако создание ситуационных 
центров выявило ряд проблем, возникающих и в момент их разработки, и в 
процессе эксплуатации~[1, 3--7].

Ситуационный зал является одним из приложений ситуационного центра для 
решения задач ситуационного управления с помощью двух основных 
методик: обсуждение возможных решений экспертной группой и доклад 
(презентация) некоторого материала с помощью средств отображения 
информации~\cite{8chu, 9chu}. В~ситуационном зале могут анализироваться 
и разрабатываться различные варианты решения стратегического характера. 
Поэтому создание ситуационного зала~--- это не просто оснащение комнаты 
презентационным оборудованием с обеспечением некоторого 
респектабельного облика помещения, но и создание максимального уровня 
комфорта, где присутствие даже самой незначительной детали должно быть 
обосновано на уровне технического задания.

Для создания ситуационного зала в первую очередь необходимо оценить 
целый ряд параметров: размеры помещения, максимальное число человек, 
участвующих в переговорах, необходимую производительность средств 
отображений информации с точки зрения статической информативности 
экрана, а также геометрические размеры экранов этих средств отображения.

В статье предложен подход к определению взаимосвязей между основными 
параметрами ситуационного зала и оценке относительного расположения 
рабочих мест сотрудников и коллективного экрана.    

\section{Общий подход. Термины и~определения}

Создание и оборудование ситуационного зала среди прочих требований 
должно опираться на существующие стандарты по эргономике, действующие 
на территории РФ. В табл.~1 перечислены основные термины и понятия, 
которые будут использованы в рамках статьи~[10--12].

\begin{table*}\small
\begin{center}
\Caption{Основные термины и понятия
}
\vspace*{2ex}

\tabcolsep=5.5pt
\begin{tabular}{|l|c|l|}
\hline
\multicolumn{1}{|c|}{Термин}&\tabcolsep=0pt\begin{tabular}{c}Обозна-\\ чение\end{tabular}&
\multicolumn{1}{c|}{Определение}\\
\hline
Активная часть экрана&&Часть экрана, ограниченная пикселами\\
\hline
\tabcolsep=0pt\begin{tabular}{l}Стягиваемый угол\\ (угловой размер)\end{tabular}&
$\psi$&\tabcolsep=0pt\begin{tabular}{l}Размер визуального объекта при данном конкретном 
расстоянии наблю-\\ дения:
$\psi=2\arctg(h/(2D))$, где $h$~--- высота объекта; $D$~--- расстояние\\ наблюдения\end{tabular}\\
\hline
Высота знака&$h$&Линейная высота знака, м\\
\hline
Ширина знака&$w$&Линейная ширина знака, м\\
\hline
Формат знака&&
\tabcolsep=0pt\begin{tabular}{l}Число пикселов по горизонтали и вертикали в матрице, используемой для\\ построения 
символа\end{tabular}\\
\hline
\tabcolsep=0pt\begin{tabular}{l}Проектное расстояние\\ наблюдения\end{tabular}&
$D$&\tabcolsep=0pt\begin{tabular}{l}Расстояние или диапазон расстояний между экраном и глазами 
наблюдате-\\ля, при котором изображение соответствует требованиям разборчивости и\\ удобочитаемости\end{tabular} \\
\hline
\tabcolsep=0pt\begin{tabular}{l}Рабочая площадь\\ помещения\end{tabular}&
&\tabcolsep=0pt\begin{tabular}{l}Множество положений в помещении, в которых сохраняется 
разборчивость,\\ удобочитаемость материала, отображенного экраном\end{tabular} \\
\hline
Угол обзора человека&$\gamma$&
\tabcolsep=0pt\begin{tabular}{l}Угол, стягивающий точки пространства, которые человек способен 
наблю-\\ дать без напряжения для глазных мышц и поворотов головы \end{tabular}\\
\hline
\end{tabular}
\end{center}
\end{table*}
\begin{figure*}[b] %fig1
%\vspace*{-24pt}
\begin{center}
\mbox{%
\epsfxsize=149.148mm
\epsfbox{chu-1.eps}
}
\end{center}
\vspace*{-6pt}
\Caption{Рекомендуемый угол обзора человека в зависимости от интенсивности 
наблюдения
\label{f1chu}}
\end{figure*}

     Опираясь на эти понятия и термины, можно сформулировать и 
проанализировать требования и зависимости, описанные в государственных 
стандартах.
     
     В качестве одного из основных будет использовано понятие 
\textbf{области наилучшего наблюдения}, под которой понимается область 
в трехмерном или двумерном пространстве, удовлетворяющая 
эргономическим требованиям ГОСТов.
    

\subsection{Угол обзора человека} %2.1

Согласно~\cite{13chu, 14chu}:
\begin{itemize}
\item очень часто используемые средства отображения информации, 
требующие точного и быст\-ро\-го считывания показаний, следует 
располагать так, чтобы в вертикальном сечении они были видны под 
углом $\pm 15^\circ$ от нормальной линии взгляда и в горизонтальном 
сечении~--- под углом $\pm 15^\circ$ от плоскости симметрии 
человеческого тела (рис.~1);
\item часто используемые средства отображения информации, 
требующие менее точного и быстрого считывания показаний,~--- 
$\pm30^\circ$ по вертикали и горизонтали;
\item редко используемые средства отображения информации~--- $\pm 
60^\circ$ по вертикали и горизонтали.
\end{itemize}



\subsection{Угловой размер знака} %2.2

     Стягиваемый угол определяется согласно~\cite{10chu, 15chu}. 
Рекомендуемые показатели для обеспечения разборчивости (для латинского алфавита):
минимальный $\psi_{\min} =16^\prime$; предпочтительный $\psi_{\mathrm{предп}}=$\linebreak $=20^\prime\mbox{--}22^\prime$.

     Из определения угла $\psi$ следует, что
     \begin{equation}
     \fr{h}{D}=2\tg\fr{\psi}{2}\approx \psi\,.
     \label{e1chu}
     \end{equation}
     
     Формула~(\ref{e1chu}) верна, так как угол~$\psi$, измеряемый здесь и 
далее в радианах, очень мал. Данное соотношение позволяет оценить 
минимальное и рекомендуемое отношение высоты знака и проектного 
расстояния:
минимальное $(h/D)_{\min}=\psi_{\min}=$\linebreak $= 4{,}7\cdot 10^{-3}$;
рекомендуемое $(h/D)_{\mathrm{рек}}=\psi_{\mathrm{рек}}=$\linebreak $= 5{,}8\mbox{--}6{,}4\cdot 10^{-3}$.

\subsection{Формат букв} %2.3

     Отношение линейных параметров ширины знака к его высоте должно 
соответствовать следующим значениям (для латинского алфавита)~[12]:
допустимый диапазон~--- от 0,5:1 до 1:1,  
предпочтительный диапазон~--- от 0,6:1 до 0,9:1.




\begin{table*}\small %tabl2
\begin{center}
\Caption{ Основные параметры, используемые в зависимостях
}
\vspace*{2ex}
\begin{tabular}{|l|c|l|}
\hline
\multicolumn{1}{|c|}{Термины}&Обозначение&\multicolumn{1}{c|}{Определение}\\
\hline
Информативность&$I$&Максимальное число знаков в одном кадре контента\\
\hline
Число рабочих мест&$N$&Число рабочих мест для одного коллективного экрана\\
\hline
\tabcolsep=0pt\begin{tabular}{l}Диаметр\\ помещения\end{tabular}&$P$&
\tabcolsep=0pt\begin{tabular}{l}Максимальное расстояние между двумя точками поме-\\щения, в 
том числе обусловленное его геометрией, м\end{tabular}\\
\hline
Ширина экрана&$W$&Геометрическая ширина активной части экрана, м\\
\hline
Высота экрана&$H$&Геометрическая высота активной части экрана, м\\
\hline
\end{tabular}
\end{center}
\vspace*{9pt}
\end{table*}



\section{Формирование зависимостей между основными 
параметрами ситуационного зала}

    На основании рекомендаций, обозначенных в п.~2, можно приступить 
к формированию взаимосвязей между основными параметрами системы 
<<помещение--экран--наблюдатели>>. Этими основными параметрами 
являются: информативность, число рабочих мест, диаметр помещения, 
ширина и высота экрана (табл.~2).


     Стоит отметить разницу между проектным\linebreak расстоянием 
наблюдения~$D$ и диаметром помещения~$P$. Первый параметр 
характеризуется свойствами системы <<экран--наблюдатели>>, а 
параметр~$P$~--- исключительно свойствами помещения.

\subsection{Область наилучшего наблюдения для~коллективного экрана} 
%3.1
    
    Пусть $\varphi$~--- угол направления обзора элемента экрана 
относительно нормали к поверхности экрана из центра элемента. Тогда 
размер элемента при наблюдении под углом~$\varphi$ уменьшится и может 
быть приближенно вычислен как
    \begin{equation}
    w(\varphi)=w\cos\varphi\,,
    \label{e2chu}
    \end{equation}
    где $w$~--- линейный размер элемента, а $w(\varphi)$~--- его линейный 
размер при наблюдении под углом~$\varphi$. Поэтому из сохранения 
углового размера символа следует соотношение:
    \begin{equation*}
    D(\varphi)=D\cos\varphi\,.
%    \label{e3chu}
    \end{equation*}

Следовательно, область наилучшего наблюдения элемента экрана (в плоском 
горизонтальном сечении)~--- круг, касающийся экрана в точке, 
соответствующей этому элементу и диаметром, равным~$D$~--- проектному 
расстоянию (рис.~2).




    Рассматривая экран как множество активных точек~\cite{16chu}, для 
каждой из которых строится область\linebreak\vspace*{-12pt}

\begin{center} %fig2
%\vspace*{6pt}
\mbox{%
\epsfxsize=43.302mm
\epsfbox{chu-2.eps}
}
\end{center}
\vspace*{4pt}
\begin{center}
{{\figurename~2}\ \ \small{Область наилучшего наблюдения для точечного элемента экрана}}
\end{center}
%\vspace*{9pt}

\medskip
\addtocounter{figure}{1}

\noindent
 наилучшего наблюдения, можно 
построить область наилучшего наблюдения для всего экрана~--- пересечение 
множества построенных кругов (рис.~3). Далее рассматривается случай, 
когда экран плоский и рабочие места располагаются в горизонтальной 
плоскости.



В случае, когда экран плоский, областью наилучшего наблюдения будет 
пересечение двух наиболее далеких друг от друга кругов.

     Другим существенным ограничением, которое необходимо наложить 
на область наилучшего наблюдения, являются пропорции наблюдаемых 
символов, которые, согласно данным п.~2.3, не могут меняться более чем в 2~раза, 
так как максимальное и минимальное допустимые значения пропорций 
отличаются друг от друга ровно в  2~раза (максимальная ширина равна 
одной высоте знака, а минимальная~--- ее половине). Значит, в силу 
формулы~(\ref{e2chu}) максимальный угол отклонения от нормали не 
должен превышать~60$^\circ$.
     
     Пусть $\alpha$~--- максимальный угол наблюдения для любого малого 
элемента поверхности экрана из некоторой точки пространства, 
расположенной со стороны активной поверхности экрана. Ясно, что этот 
максимум в горизонтальной плоскости будет достигаться возле правого или 
левого края экрана. Ввиду вышесказанного угол~$\alpha$ не должен 
превосходить~60$^\circ$.

\end{multicols}

\begin{figure} %fig3
\vspace*{1pt}
\begin{center}
\mbox{%
\epsfxsize=97.554mm
\epsfbox{chu-3.eps}
}
\end{center}
\vspace*{-6pt}
\Caption{Область наилучшего наблюдения экрана
\label{f3chu}}
\end{figure}

\begin{figure} %fig4
\vspace*{1pt}
\begin{center}
\mbox{%
\epsfxsize=97.554mm
\epsfbox{chu-4.eps}
}
\end{center}
\vspace*{-6pt}
\Caption{Геометрическая модель области наилучшего наблюдения
\label{f4chu}}
\vspace*{9pt}
\end{figure}

\begin{multicols}{2}

%\begin{center} %fig4
%%\vspace*{6pt}
%\mbox{%
%\epsfxsize=80mm
%\epsfbox{chu-4.eps}
%}
%\end{center}
%\vspace*{4pt}
%\begin{center}
%{{\figurename~4}\ \ \small{Область наилучшего наблюдения экрана}}
%\end{center}
%%\vspace*{9pt}

%\bigskip
%\addtocounter{figure}{1}
     
     Получается, что область наилучшего наблюдения заключена внутри 
угла пересечения двух лучей, проведенных из крайних точек экрана под 
углом~$\alpha$ к нормали. С~учетом всех ограничений область наилучшего 
наблюдения будет являться фигурой пересечения двух уже построенных 
областей (рис.~4).



    Точки~$A$, $B$, $C$ и~$E$ имеют следующие коорди\-наты:
    \begin{align*}
&A\left( 0;\,\fr{W\ctg\alpha}{2}\right)\,;
\end{align*}

\noindent
\begin{align*}
&B\left(\fr{D\sin 2\alpha-
W}{2};\,D\cos^2\alpha\right)\,;\\
& C\left(0;\,\fr{D+\sqrt{D^2- W^2}}{2}\right)\,; \\
&  E\left( \fr{-D\sin 2\alpha +W}{2};\,D\cos^2\alpha\right)\,.
\end{align*}

Площадь области $ABCE$ можно оценить снизу с помощью 
треугольников~$ABC$ и~$ACE$. Площадь каждого из них может быть 
посчитана по формуле

\noindent
\begin{multline*}
S_{\triangle}=\fr{1}{2}AC\cdot BH ={}\\
{}=\fr{1}{8}\left( D\sin 2\alpha -
W\right)\left(D-W\ctg\alpha+\sqrt{D^2-W^2}\right),
%\label{e4chu}
\end{multline*}
где $BH$~--- высота треугольника~$ABC$, опущенная из вершины~$B$ на 
сторону~$AC$. С~учетом того, что таких треугольника два, получаем оценку 
площади области наилучшего наблюдения:
\begin{multline}
S_{\mathrm{о.н.н.}}=\fr{D^2}{4}\left(\sin2\alpha-C\right)\times{}\\
{}\times\left(1-
C\ctg\alpha+\sqrt{1-C^2}\right) ={}\\
{}=
\fr{W^2}{4C^2}\left(\sin2\alpha-C\right)
\left(1-C\ctg\alpha+\sqrt{1-C^2}\right)\,,
\label{e5chu}
\end{multline}
где
$C=W/D$~--- константа системы, которая, как будет видно далее, зависит от 
информативности контента~$I$, отношения~$k$ высоты экрана к ширине, 
отношения~$p$ ширины знака к его высоте и угла~$\psi$, стягиваемого 
одним символом.

\subsection{Число рабочих мест в~области наилучшего наблюдения} %3.2

Пусть $D$ и~$W$ зафиксированы. Тогда определена область наилучшего 
наблюдения и можно оценить максимальное число человек~$N$, которые 
смогут находиться в области наилучшего наблюдения, используя один 
коллективный экран. Главным условием комфортной работы условимся 
считать отсутствие помех со стороны других пользователей на расстоянии 
вытянутой руки. Таким образом, каждому пользователю сопоставим круг 
радиусом, равным половине маховой сажени (маховая сажень $\approx 
1{,}8$~м). Наиболее плотной расстановкой этих кругов на плоскости будет 
расположение, когда каж\-дый из кругов касается других шести. При этом 
центры кругов образуют сетку из равносторонних треугольников со 
стороной, равной одной маховой сажени, или 1,8~м (рис.~5).


Для оценки числа точек равномерной треугольной сетки, которые могут 
попасть внутрь области
 наилучшего наблюдения, воспользуемся формулой
Пика, связывающей число узлов квадратной сетки
 с шагом~1, попавших 
внутрь и на границу многоугольника, с площадью этого 
многоугольника~\cite{17chu}:
\begin{equation*}
S=B+\fr{\Gamma}{2}-1\,,
%\label{e7chu}
\end{equation*}
где $S$~--- целочисленная площадь многоугольника, $B$~--- число узлов 
квадратной сетки с шагом~1, попавших внутрь многоугольника; 
$\Gamma$~--- число узлов этой сетки, которые попали на границу.
\begin{center} %fig5
%\vspace*{6pt}
\mbox{%
\epsfxsize=73.141mm
\epsfbox{chu-5.eps}
}
\end{center}
\vspace*{4pt}
\begin{center}
{{\figurename~5}\ \ \small{Сетка из центров кругов}}
\end{center}
%\vspace*{9pt}

%\bigskip
\addtocounter{figure}{1}



Треугольную сетку с шагом~1,8 можно рас\-смат\-ри\-вать как квадратную сетку 
с шагом~1, которая претерпела два преобразования:
\begin{itemize}
\item сжатие по направлению вектора, параллельного диагонали квадрата 
сетки с коэффициентом~$\sqrt{2}$;
\item растяжение по двум любым взаимно перпендикулярным осям с 
коэффициентом~1,8.
\end{itemize}

Поэтому для применения формулы Пика необходимо рассматривать не 
исходное значение\linebreak площади, а ее преобразование, обратное преобразованиям 
сетки. То есть в случае равномерной треугольной сетки с шагом~1 будет 
выполнено следующее соотношение:
\begin{equation}
\fr{\sqrt{2}}{1{,}8^2}\,S=B+\fr{\Gamma}{2}-1\,.
\label{e8chu}
\end{equation}
Число рабочих мест в области наилучшего наблюдения при этом составляет
\begin{equation}
N=B+\Gamma=\fr{\sqrt{2}}{1{,}8^2}\,S+1+\fr{\Gamma}{2}\,.
\label{e9chu}
\end{equation}
Таким образом, для получения оценки числа рабочих мест в области 
наилучшего наблюдения необходимо оценить максимальное 
значение~$\Gamma$. Важно понимать, что формула Пика описывает 
площадь многоугольника с вершинами в узлах сетки. Ясно, что в случае 
области наилучшего наблюдения такое обеспечить не всегда возможно. 
Поэтому формула~(\ref{e9chu}) представляет собой верхнюю оценку  
числа рабочих мест.

Из оценки периметра области наилучшего наблюдения получаем, что
\begin{multline*}
\Gamma\leq \fr{1}{1{,}8}\left(2\vert AB\vert 
+2\widehat{CB}\right)={}\\
{}=\fr{D}{1{,}8}\left(\fr{\sin2\alpha-C}{\sin\alpha}+2\alpha-
\arcsin C\right)\,.
%\label{e10chu}
\end{multline*}
Таким образом, число рабочих мест в области наилучшего наблюдения 
может быть оценено сверху следующей величиной:
\begin{multline}
N=B+\Gamma=0{,}44S+{}\\
{}+\fr{D}{3{,}6}\left( \fr{\sin2\alpha-C}{\sin\alpha}+2\alpha-
\arcsin C\right)+1\,.
\label{e11chu}
\end{multline}

\subsection{Оценка максимальной площади области наилучшего 
наблюдения и~максимального числа рабочих мест~в~этой области}

Пусть~$I$ определено. Тогда исходя из пространственных ограничений, 
характеризуемых диа\-мет\-ром~$P$, можно оценить максимальную площадь 
области наилучшего наблюдения при некоторых значениях проектного 
расстояния~$D$ и ширины экрана~$W$.

Ввиду того, что соотношение сторон экрана у большинства производителей 
на настоящий момент составляет от~16:9 до~16:12, можно исключить из 
рассмотрения параметр высоты экрана. Дальнейшие оценки получаются в 
результате подсчета с двух сторон площади активной части экрана
\begin{equation*}
S_{\mathrm{display}}=WH=Iwh\,.
%\label{e12chu}
\end{equation*}
Пусть
\begin{equation*}
H=kW\,; %\label{e13chu}
\quad w=ph\,, %\label{e14chu}
\end{equation*}
где $k\in [9/16;\,12/16]$, $p\in [0{,}6;\,0{,}9]$ (см.\ п.~2.3). %Значение~$p$ соответствует  табл.~2.
Тогда
\begin{equation}
kW^2=Iph^2\ \ \mbox{или}\ \ W=h\sqrt{\fr{Ip}{k}}\,.
\label{e15chu}
\end{equation}
Из формулы~(\ref{e1chu}) следует, что
\begin{equation}
h=D\psi\,.
\label{e16chu}
\end{equation}

Из формул~(\ref{e15chu}) и~(\ref{e16chu}) получается, что при 
фиксированных значениях информативности~$I$, отношения~$k$ высоты 
экрана к ширине, отношения $p$ ширины знака к его высоте и угла~$\psi$, 
стягиваемого одним символом, существует константа~$C$, 
удовлетворяющая следующим соотношениям:
\begin{equation}
C=\fr{W}{D}=\psi\sqrt{\fr{Ip}{k}}\,.
\label{e17chu}
\end{equation}
Из формулы~(\ref{e17chu}) следует, что
\begin{equation}
C_{\min} 
=\psi_{\min}\sqrt{\fr{p_{\min}}{k_{\max}}}\,\sqrt{I}=\fr{\sqrt{I}}{193}\,.
\label{e18chu}
\end{equation}
     
     Из-за ограниченности помещения проектное расстояние~$D$ не может 
превышать диаметра помещения~$P$, поэтому ввиду оценок~(\ref{e5chu}) 
и~(\ref{e18chu})
     \begin{multline}
     S_{\max} ={}\\
     {}=\fr{D^2}{4}\left(\sin2\alpha-C\right)\left(1-C\ctg\alpha+\sqrt{1-
C^2}\right)\leq{}\\
     {}\leq \fr{P^2}{2}\left(\sin2\alpha-\fr{\sqrt{I}}{193}\right)\times{}\\
     {}\times\left(1-
\fr{\ctg \alpha\cdot \sqrt{I}}{193}+\sqrt{1-\fr{I}{193^2}}\right)\,.
     \label{e19chu}
     \end{multline}
     
     Отметим, что информативность не должна превосходить 
$(193\sin2\alpha)^2$~знаков. В~противном случае область наилучшего 
наблюдения окажется пустым множеством.

На основании формулы~(\ref{e19chu}) можно оценить число рабочих мест, 
которые можно разместить в области наилучшего наблюдения с учетом 
возможностей помещения, ограниченных его диа\-мет\-ром~$P$. Согласно 
оценкам~(\ref{e11chu}) и~(\ref{e19chu})
\begin{multline}
N_{\max}=0{,}44S_{\max}+\fr{P}{3{,}6}\left(\fr{\sin2\alpha-\sqrt{I}/193}{\sin\alpha}+{}\right.\\
\left.{}+2\alpha-
\arcsin\fr{\sqrt{I}}{193}\right)+1\,.
\label{e20chu}
\end{multline}

\subsection{Оценка минимальной ширины активной поверхности 
экрана}

Чем меньше размер экрана, тем меньше его стои\-мость при прочих равных 
условиях. Поэтому размеры экрана, в том числе и его ширина, должны быть 
по возможности минимизированы. Пусть известна информативность 
контента~$I$, угол наблюдения~$\alpha$ и диаметр помещения~$P$. 
Рассмотрим два принципиально разных случая:
\begin{enumerate}[1.]
\item Число наблюдателей неизвестно, необходимо оценить размеры 
экрана (его ширину), позволяющие эффективно использовать 
пространство помещения.
\item Число наблюдателей~$N$ известно, но оно меньше, чем полученное 
по формуле~(\ref{e20chu}) для известных параметров~$P$, $I$ и~$\alpha$. 
Необходимо оценить минимальные размеры экрана (его ширину), 
позволяющие обеспечить расположение всех наблюдателей в области 
наилучшего наблюдения с учетом отсутствия взаимных помех.
\end{enumerate}

\smallskip

\noindent
\textbf{Случай 1.} Из соотношения~(\ref{e17chu}) следует, что
\begin{equation*}
W= PC\geq PC_{\min}=\fr{P\sqrt{I}}{193}\,.
%\label{e21chu}
\end{equation*}

\smallskip

\noindent
\textbf{Случай 2.} Оценим минимальную площадь, на которой могут 
разместиться $N$~наблюдателей. Для этого воспользуемся тем 
соображением, что минимальная площадь равна количеству внутренних 
рабочих мест (тех, которые не попали на границу), умноженному на площадь 
двух треугольников, ограниченных сеткой. Такое соображение следует из 
того, что каждой внутренней точке можно сопоставить 6~треугольников 
сетки, а каждому треугольнику сетки~--- 3~точки. Это значит, что каждой 
точке сопоставляется по 2~треугольника.

Из формул~(\ref{e8chu}) и~(\ref{e9chu}) следует, что
\begin{equation*}
B=\fr{2\sqrt{2}}{1{,}8^2}\,S+2-N\,.
%\label{e22chu}
\end{equation*}
С другой стороны, как было замечено,
\begin{equation*}
S=B\cdot 2S_{\triangle}\,.
%\label{e23chu}
\end{equation*}
Поэтому
\begin{equation*}
S_{\min} =(N-2)\fr{2\cdot 1{,}8^2\cdot1{,}4}{4\sqrt{2}\cdot1{,}4-
1{,}8^2}=1{,}94(N-2)\,.
%\label{e24chu}
\end{equation*}

Из формулы~(\ref{e5chu}) получаем:
\begin{multline*}
W_{\min} =2C_{\min}
\left(
{S_{\min}}\Big /\left(\vphantom{\sqrt{1-C^2_{\min}}}(
\sin2\alpha-C_{\min})\times{}\right.\right.\\
\left.\left.{}\times(1-
C_{\min}\ctg\alpha+\sqrt{1-C^2_{\min}})\right)\right)^{1/2}={}\\
{}=2
\left(
{1{,}94I(N-2)}\Big /\left(\vphantom{\sqrt{193^2-I}}
(193\sin2\alpha-\sqrt{I})\times{}\right.\right.\\
\left.\left.{}\times (193-
\sqrt{I}\ctg\alpha+\sqrt{193^2-I})\right)\right)^{1/2}\,.
%\label{e25chu}
\end{multline*}


\section{Заключение}

В статье сформулированы основные определения и термины, используемые 
при проектировании средств отображения информации, а также описаны 
рекомендации к ним, действующие в рамках государственных стандартов.

На основании этих требований и рекомендаций, а также на основании 
простых геометрических соображений получены оценки основных 
параметров для проектирования средства отображения и рабочих мест. 
В~качестве основной определяющей величины выступает информативность 
контента, которая формируется на основании задач ситуационного зала и 
ситуационных моделей отображения.
     
     В качестве основного использован параметр отношения ширины экрана 
к проектному расстоянию наблюдения. Часто этот параметр для удобства 
считается фиксированным, но в общем случае показано, что это отношение 
прямо пропорционально корню квадратному от величины информативности 
контента.
    
Если, помимо информативности контента, известен диаметр помещения, в 
котором располагается ситуационный зал, то можно определить 
максимальную площадь области наилучшего наблюдения и оценить число 
рабочих мест, которые могут быть расположены в области наилучшего 
наблюдения с помощью формулы Пика и аффинных преобразований.

В качестве обратной задачи, в которой известно количество человек, 
одновременно работающих с экраном, получена зависимость для 
определения минимально необходимой ширина экрана.

{\small\frenchspacing
{%\baselineskip=10.8pt
\addcontentsline{toc}{section}{Литература}
\begin{thebibliography}{99}


\bibitem{1chu}
\Au{Ильин Н.\,И.}
Основные направления развития ситуационных центров органов государственной 
власти~// ВКСС Connect! (Ведомственные корпоративные сети и системы), 2007. 
№\,6(45). С.~2--9.

\bibitem{2chu}
\Au{Лисица К.\,В.}
Опыт создания и применения Автоматизированной системы стратегического 
управления в ОАО <<Российские железные дороги>>~// Ситуационные центры: 
модели, технологии, опыт практической реализации: Мат-лы науч.-практич. 
конф.~--- М.: РАГС, 2007.

\bibitem{4chu} %3
\Au{Филиппович А.\,Ю.}
Ситуационная система~--- что это такое?~// PCWeek/RE, 2003. No.\,26.

\bibitem{6chu} %4
\Au{Зацаринный А.\,А., Ионенков Ю.\,С., Кондрашев~В.\,А.}
Об одном подходе к выбору системотехнических решений построения 
информационно-те\-ле\-ком\-му\-никационных систем~// Системы и средства информатики. 
Вып.~16.~--- М.: Наука, 2006. С.~65--72.


\bibitem{3chu} %5
\Au{Зацаринный А.\,А., Сучков А.\,В., Босов~А.\,В.}
Ситуа\-ционные центры в современных информационно-те\-ле\-ком\-му\-ни\-кационных 
системах специального\linebreak назначения~// ВКСС Connect! (Ведомственные корпоративные 
сети и системы), 2007. №\,5(44). С.~64--76.

\bibitem{7chu} %6
\Au{Зацаринный А.\,А., Ионенков Ю.\,С.}
Некоторые аспекты выбора технологии построения информационно-те\-ле\-ком\-муникационных сетей~// 
Системы и средства информатики. Вып.~17.~--- М.: 
Наука, 2007. С.~5--16.


\bibitem{5chu} %7
\Au{Зацаринный А.\,А.}
Тенденции развития ситуационных центров как компонентов информационно-те\-ле\-ком\-му\-ни\-ка\-ционных 
систем в условиях глобальной информатизации общества~// 
Докл. XXXV\linebreak междунар. конф. <<Информационные технологии в науке, 
образовании, телекоммуникации и бизнесе (IT\;+\;S\&E'08)>>. Ялта--Гурзуф, Украина. 
2008.


\bibitem{8chu}
Ситуационные центры (СЦ) и их история. {\sf http://\linebreak ta.interrussoft.com/s\_centre.html}.

\bibitem{9chu}
\Au{Зацаринный А.\,А.}
Организационные принципы сис\-тем\-но\-го подхода к разработке, проектированию и 
внедрению современных информационно-те\-ле\-ком\-му\-никационных сетей~// ВКСС 
Connect! (Ведомственные корпоративные сети и системы), 2007. №\,1(40). С.~60--67.

\bibitem{11chu} %10
ГОСТ 26387-84 Система <<Человек--машина>>. Термины и определения.~--- М.: 
Стандартинформ, 2006.

\bibitem{12chu} %11
ГОСТ 27833-88 Средства отображения информации. Термины и определения.~--- М.: 
Стандартинформ, 2005.

\bibitem{10chu} %12
ГОСТ Р~52324-2005 (ИСО 13406-2:2001) Эргономические требования к работе с 
визуальными дисплеями, основанными на плоских панелях.~--- М.: Стандартинформ, 
2005.

\bibitem{13chu}
ГОСТ 12.2.032-78 Система стандартов безопасности труда. Рабочее место при 
выполнении работ сидя. Общие эргономические требования.~--- М.: Изд-во 
стандартов, 2001.

\bibitem{14chu}
ГОСТ 12.2.033-78 Система стандартов безопасности труда. Рабочее место при 
выполнении работ стоя. Общие эргономические требования.~--- М.: Изд-во 
стандартов, 2001.

\bibitem{15chu}
ГОСТ Р ИСО 9241-3 Эргономические требования при выполнении офисных работ с 
использованием видеодисплейных терминалов.~--- М.: Изд-во стандартов, 2003.

\bibitem{16chu}
ГОСТ 21958-76 Зал и кабины операторов. Взаимное расположение рабочих мест.~--- 
М.: Изд-во стандартов, 1976.

\label{end\stat}

\bibitem{17chu}
\Au{Прасолов В.\,В.}
Задачи по планиметрии.~--- М.: \mbox{МЦНМО}, 2001.


 \end{thebibliography}
}
}


\end{multicols}  %11

%   { %\Large  
   { %\baselineskip=16.6pt
   
   \vspace*{-48pt}
   \begin{center}\LARGE
   \textit{Предисловие}
   \end{center}
   
   %\vspace*{2.5mm}
   
   \vspace*{25mm}
   
   \thispagestyle{empty}
   
   { %\small 

    
Вниманию читателей журнала <<Информатика и её применения>> предлагается 
очередной тематический выпуск <<Вероятностно-статистические методы и 
задачи информатики и информационных технологий>>. Предыдущие тематические 
выпуски журнала по данному направлению вышли в 2008~г.\ (т.~2, вып.~2), 
в 2009~г.\ (т.~3, вып.~3) и в 2010~г.\ (т.~4, вып.~2). 

Статьи, собранные в данном журнале, посвящены разработке новых вероятностно-статистических 
методов, ориентированных на применение к решению конкретных задач информатики и информационных 
технологий, а также~--- в ряде случаев~--- и других прикладных задач. Проблематика, охватываемая 
публикуемыми работами, развивается в рамках научного сотрудничества между Институтом проблем 
информатики Российской академии наук (ИПИ РАН) и Факультетом вычислительной математики и 
кибернетики Московского государственного университета им.\ М.\,В.~Ломоносова в ходе работ 
над совместными научными проектами (в том числе в рамках функционирования 
Научно-образовательного центра <<Вероятностно-статистические методы анализа рисков>>). 
Многие из авторов статей, включенных в данный номер журнала, являются активными участниками 
традиционного международного семинара по проблемам устойчивости стохастических моделей, 
руководимого В.\,М.~Золотаревым и В.\,Ю.~Королевым; регулярные сессии этого семинара 
проводятся под эгидой МГУ и ИПИ РАН (в 2011~г.\ указанный семинар проводится в октябре 
в Калининградской области РФ). 

Наряду с представителями ИПИ РАН и МГУ в число авторов данного выпуска журнала входят 
ученые из Научно-исследовательского института системных исследований РАН, Института 
проблем технологии микроэлектроники и особочистых материалов РАН, Института 
прикладных математических исследований Карельского НЦ РАН, Московского 
авиационного института, Вологодского государственного педагогического университета, 
НИИММ им.\ Н.\,Г.~Чеботарева, Казанского государственного университета, Дебреценского 
университета (Венгрия).

Несколько статей выпуска посвящено разработке и применению стохастических методов и 
информационных технологий для решения различных прикладных задач. В~работе В.\,Г.~Ушакова 
и О.\,В.~Шестакова рассмотрена задача определения вероятностных характеристик случайных 
функций по распределениям интегральных преобразований, возникающих в задачах эмиссионной 
томографии. В~статье Д.\,О.~Яковенко и М.\,А.~Целищева рассмотрены некоторые вопросы 
математической теории риска и предложен новый подход к диверсификации инвестиционных 
портфелей. Работа И.\,А.~Кудрявцевой и А.\,В.~Пантелеева посвящена построению и 
исследованию математической модели, описывающей динамику сильноионизованной плазмы. 
В~статье П.\,П.~Кольцова изучается качество работы ряда алгоритмов сегментации изображений. 
Статья А.\,Н.~Чупрунова и И.~Фазекаша посвящена вероятностному анализу числа без\-оши\-бочных 
блоков при помехоустойчивом кодировании; получены усиленные законы больших чисел для указанных 
величин.

В данном выпуске традиционно присутствует тематика, весьма активно разрабатываемая в течение 
многих лет специалистами ИПИ РАН и МГУ,~--- методы моделирования и управления для 
информационно-телекоммуникационных и вычислительных систем, в частности методы 
теории массового обслуживания. В~статье А.\,И.~Зейфмана с соавторами рассматриваются 
модели обслуживания, описываемые марковскими цепями с непрерывным временем в случае 
наличия катастроф. В~работе М.\,М.~Лери и И.\,А.~Чеплюковой рассматриваются случайные 
графы Интернет-типа, т.\,е.\ графы, степени вершин которых имеют степенные распределения; 
такие задачи находят применение при исследовании глобальных сетей передачи данных. 
Работа Р.\,В.~Разумчика посвящена исследованию систем массового обслуживания специального 
вида~--- с отрицательными заявками и хранением вытесненных заявок.

Ряд статей посвящен развитию перспективных теоретических 
вероятностно-статистических методов, которые находят широкое применение в различных 
задачах информатики и информационных технологий. В~работе В.\,Е.~Бенинга, А.\,К.~Горшенина 
и В.\,Ю.~Королева рассмотрена задача статистической проверки гипотез о числе компонент 
смеси вероятностных распределений, приводится конструкция асимптотически наиболее мощного 
критерия. Результаты этой работы найдут применение в ряде прикладных задач, использующих 
математическую модель смеси вероятностных распределений (в информатике, моделировании 
финансовых рынков, физике турбулентной плазмы и~т.\,д.). В~статье В.\,Ю.~Королева, 
И.\,Г.~Шевцовой и С.\,Я.~Шоргина строится новая, улучшенная оценка точности нормальной 
аппроксимации для пуассоновских случайных сумм; как известно, указанные случайные суммы 
широко используются в качестве моделей многих реальных объектов, в том числе в информатике, 
физике и других прикладных областях. Работа В.\,Г.~Ушакова и Н.\,Г.~Ушакова посвящена 
исследованию ядерной оценки плотности распределения; эти результаты могут применяться, 
в част\-ности, при анализе трафика в телекоммуникационных системах. Серьезные приложения 
в статистике могут получить результаты работы О.\,В.~Шестакова, в которой доказаны оценки 
скорости сходимости распределения выборочного абсолютного медианного отклонения к нормальному 
закону. 

\smallskip

Редакционная коллегия журнала выражает надежду, что данный тематический  выпуск 
будет интересен специалистам в области теории вероятностей и математической статистики 
и их применения к решению задач информатики и информационных технологий.
     
     %\vfill 
     \vspace*{20mm}
     \noindent
     Заместитель главного редактора журнала <<Информатика и её 
применения>>,\\
     директор ИПИ РАН, академик  \hfill
     \textit{И.\,А.~Соколов}\\
     
     \noindent
     Редактор-составитель тематического выпуска,\\
     профессор кафедры математической статистики факультета\\
      вычислительной математики и кибернетики МГУ им.\ М.\,В.~Ломоносова,\\
     ведущий научный сотрудник ИПИ РАН,\\ 
доктор физико-математических наук \hfill
      \textit{В.\,Ю.~Королев}
     
     } }
     }

%%%%%%%%%%%%%%%%%%%%%%%%%%%%%%%%%%%%%%%%%%%%%%%


                       
%\end{document}

%\def\stat{rez}
{%\hrule\par
%\vskip 7pt % 7pt
\raggedleft\Large \bf%\baselineskip=3.2ex
Р\,Е\,Ц\,Е\,Н\,З\,И\,И \vskip 17pt
    \hrule
    \par
\vskip 6pt plus 6pt minus 3pt }

%\thispagestyle{headings} %с верхним колонтитулом
%\thispagestyle{myheadings} %с нижним колонтитулом, но в верхнем РЕЦЕНЗИИ

\def\tit{НОВАЯ КНИГА И.\,Н.~СИНИЦЫНА, А.\,С.~ШАЛАМОВА <<ЛЕКЦИИ ПО ТЕОРИИ 
ИНТЕГРИРОВАННОЙ ЛОГИСТИЧЕСКОЙ ПОДДЕРЖКИ>> (М.: ТОРУС ПРЕСС, 2012. 624~с.)}

%1
\def\aut{Д.ф.-м.н., профессор С.\,Я.~Шоргин}

\def\auf{\ }

\def\leftkol{\ % РЕЦЕНЗИИ
}

\def\rightkol{ \ } 

%\def\leftkol{\ } % ENGLISH ABSTRACTS}

%\def\rightkol{\ } %ENGLISH ABSTRACTS}

%\def\leftkol{РЕЦЕНЗИИ}

%\def\rightkol{РЕЦЕНЗИИ}

\titele{\tit}{\aut}{\auf}{\leftkol}{\rightkol}
\vspace*{-18pt}


     \label{st\stat}

     \begin{multicols}{2}
     {\small
     {\baselineskip=10.1pt
     

      В книге представлено системное изложение теоретических основ одного из новейших 
направлений в \mbox{об\-ласти} экономики послепродажного обслуживания изделий наукоемкой 
продукции (ИНП) длительного пользования~--- интегрированной логистической поддержки
(ИЛП). 
{\looseness=1

}

Приведены также результаты новых работ, выполненных в Институте проблем информатики 
Российской академии наук в рамках научного направления <<Информационные технологии и 
анализ сложных сис\-тем>>.
 {%\looseness=1

}
     
      Излагаемые в книге научные подходы позво\-ляют карди\-наль\-но реформировать 
существующие системы производства и эксплуатации ИНП путем создания и внед\-ре\-ния 
методов рационального и оптимального управ\-ле\-ния процессами расходования 
вре\-мен\-н$\acute{\mbox{ы}}$х, 
мате\-ри\-аль\-ных, трудовых и других ресурсов на всех стадиях жизненного цикла изделий (ЖЦИ) по 
критериям экономической целесообразности и эф\-фек\-тив\-ности.
  {\looseness=1

}
    
      В книге приведен краткий обзор причин возник\-новения и
      развития CALS-методологии как основы 
современных международных стандартов по созданию и функционированию глобальных 
ин\-фор\-ма\-ци\-он\-но-ком\-му\-ни\-ка\-ци\-он\-ных систем, ее ключевых возможностей и эффективности 
результатов ее использования. 
Авторы %\linebreak 
предлагают ряд научных обоснований для разработки 
единой теории проектирования и управления систем ИЛП для полноценного использования 
преимуществ %\linebreak
 суще\-ст\-ву\-ющей методологии, определяют \mbox{общую} структурную схему 
комплексной системы <<ИНП-СППО>> и необходимость разработки для ее описания 
гибридных стохастических моделей.
{%\looseness=1

}

%\columnbreak
      
      Книга состоит из пяти частей, где последовательно излагается материал по каждой из 
следующих тем: <<Интегрированная логистическая поддержка>>, <<Теория гибридных 
стохастических систем и компьютерная поддержка исследований и разработок>>, <<Основы 
математического моделирования, анализа и синтеза систем послепродажного обслуживания>>, 
<<Определение и анализ показателей экспортного потенциала ИНП при проектировании>>, 
<<Задачи управления поддержкой послепродажного обслуживания>>, а также 
<<Моделирование инвестиционных процессов ИЛП в условиях неравновесных финансовых 
рынков>>. 
   
      В конце каждой главы приведены выводы и даны вопросы и задания для 
самоконтроля. В~приложениях содержатся основные определения по программам работ по 
анализу ИЛП, логистическим базам данных и компьютерным решениям, эквивалентной статистической 
линеаризации нелинейных преобразований ИЛП, справочный материал, а также развернутые 
уравнения для вероятностных характеристик.


      \def\leftkol{РЕЦЕНЗИИ}

\def\rightkol{РЕЦЕНЗИИ} 

      
      Книга заинтересует широкий круг специалистов и может быть использована научными 
проектными организациями в сфере промышленного производства ИНП. Большое количество 
иллюстраций, примеров и вопросов, обращенных к читателю, позволяет использовать книгу 
также в качестве учебного пособия для студентов и аспирантов машиностроительных, 
транспортных и~других специальностей, а также для самостоятельного изучения. 
{%\looseness=-1

}

Книга 
представляет несомненный интерес для специалистов и студентов в области прикладной 
математики и информатики.
    

}

}
\end{multicols}

%\newpage

\include{obchak}



\def\stat{authorsrus}
{%\hrule\par
%\vskip 7pt % 7pt
\raggedleft\Large \bf%\baselineskip=3.2ex
О\,Б\ \ А\,В\,Т\,О\,Р\,А\,Х \vskip 17pt
    \hrule
    \par
\vskip 21pt plus 8pt minus 4pt }


\def\tit{\ }

\def\aut{\ }

\def\auf{\ }

\def\leftkol{\ } % ENGLISH ABSTRACTS}

\def\rightkol{ОБ АВТОРАХ} %ENGLISH ABSTRACTS}

\titele{\tit}{\aut}{\auf}{\leftkol}{\rightkol}
      
            \label{st\stat}



\vspace*{24pt}

\begin{multicols}{2}




\noindent
\textbf{Архипов Олег Петрович} (р.\ 1948)~---
кандидат технических наук, директор Орловского филиала Института проб\-лем информатики
Российской академии наук
%302025, г.Орел, Московское шоссе, д.137

\vspace*{3pt}

\noindent
\textbf{Бирюкова Татьяна Константиновна} (р.\ 1968)~---
кандидат фи\-зи\-ко-ма\-те\-ма\-ти\-че\-ских наук, старший научный сотрудник Института проб\-лем информатики
Российской академии наук

\vspace*{3pt}

\noindent 
\textbf{Бобков  Сергей Геннадьевич} (р.\ 1955)~---
доктор технических наук,  заведующий отделением На\-уч\-но-ис\-сле\-до\-ва\-тель\-ско\-го 
института системных исследований Российской академии наук
%117218, Москва, Нахимовский просп., 36, к.1 

\vspace*{3pt}

\noindent \textbf{Васильев Николай Семенович} (р.\ 1952)~--- доктор 
фи\-зи\-ко-ма\-те\-ма\-ти\-че\-ских наук, профессор, 
МГТУ им.\ Н.\,Э.~Баумана 
%, Москва 105005, 2-я Бауманская ул., д.~5,

\vspace*{3pt}

\noindent
\textbf{Гершкович Максим Михайлович} (р.\ 1968)~---
старший научный сотрудник Института проб\-лем информатики
Российской академии наук

\vspace*{3pt}

\noindent 
\textbf{Дьяченко Юрий Георгиевич} (р.\ 1958)~--- кандидат технических наук, 
старший научный сотрудник Института проб\-лем информатики
Российской академии наук

\vspace*{3pt}

\noindent 
\textbf{Ерошенко Александр Андреевич} (р.\ 1989)~--- аспирант кафедры 
математической статистики факультета вычисли\-тельной математики и кибернетики 
Московского государственного университета им.\ М.\,В.~Ломоносова
%119991, Москва ГСП-1, Ленинские горы, д.\ 1, стр. 52

\vspace*{3pt}
 
\noindent 
\textbf{Захаров Виктор Николаевич} (р.\ 1948)~--- 
доктор технических наук, доцент, ученый секретарь Института проб\-лем информатики
Российской академии наук

\vspace*{3pt}

\noindent
\textbf{Зейфман Александр Израилевич} (р.\ 1954)~---
доктор фи\-зи\-ко-ма\-те\-ма\-ти\-че\-ских наук, профессор, 
заведующий кафедрой Вологодского государственного университета; 
старший научный сотрудник Института проб\-лем информатики
Российской академии наук; главный научный сотрудник ИСЭРТ Российской академии наук

\vspace*{3pt}

\noindent
\textbf{Зыкин Сергей Владимирович} (р.\ 1959)~--- 
доктор технических наук, профессор, заведующий лабораторией Института математики 
им.\ С.\,Л.~Соболева Сибирского отделения Российской академии наук, Новосибирск 
%630090, пр.\ ак.\ Коптюга, 4 

\vspace*{4pt}

\noindent
\textbf{Киреев Владимир Иванович} (р.\ 1938)~---
доктор фи\-зи\-ко-ма\-те\-ма\-ти\-че\-ских наук, профессор Московского 
государственного горного университета
%Адрес: Россия, 119991, г. Москва, Ленинский проспект, д. 6

%\columnbreak

\vspace*{4pt}

\noindent
\textbf{Козеренко Елена Борисовна} (р.\ 1959)~---
кандидат филологических наук, заведующая лабораторией Института проб\-лем информатики
Российской академии наук

\vspace*{4pt}

\noindent
\textbf{Королев Виктор Юрьевич} (р.\ 1954)~--- доктор
фи\-зи\-ко-ма\-те\-ма\-ти\-че\-ских наук, профессор кафедры математической 
статистики факультета вычисли\-тельной математики и кибернетики 
Московского государственного университета; 
ведущий научный сотрудник Института проб\-лем информатики
Российской академии наук

\vspace*{4pt}

\noindent
\textbf{Коротышева Анна Владимировна} (р.\ 1988)~---
старший преподаватель Вологодского государственного университета

\vspace*{4pt}

\noindent 
\textbf{Кун Де Турк} (р.\ 1981)~--- научный сотрудник 
исследовательской группы SMACS факультета телекоммуникаций и обработки информации
Университета Гента, Бельгия
%В-9000 Гент, Бельгия

\vspace*{4pt}

\noindent
\textbf{Лупенцов Олег Сергеевич} (р.\ 1986)~---
аспирант Омского государственного института сервиса
%Омск 644043, ул.\ Певцова 13

\vspace*{4pt}

\noindent
\textbf{Лучко Олег Николаевич} (р.\ 1961)~---
кандидат педагогических наук, профессор, заведующий кафедрой 
Омского государственного института сервиса
%Омск 644043, ул.\ Певцова 13

\vspace*{4pt}

\noindent
\textbf{Малашенко Юрий Евгеньевич} (р.\ 1946)~---
доктор фи\-зи\-ко-ма\-те\-ма\-ти\-че\-ских наук, заведующий сектором 
Вычислительного центра им.\ А.\,А.~Дородницына Российской академии наук
%Адрес: 119333, Москва, ул. Вавилова, 40,

\vspace*{4pt}

\noindent
\textbf{Маньяков Юрий Анатольевич} (р.\ 1984)~---
кандидат технических наук, научный сотрудник Орловского филиала Института проб\-лем информатики
Российской академии наук
%302025, г.Орел, Московское шоссе, д.137

\vspace*{4pt}

\noindent
\textbf{Маренко Валентина Афанасьевна} (р.\ 1951)~---
кандидат технических наук, доцент, старший научный сотрудник 
Института математики им.\ С.\,Л.~Соболева Сибирского отделения Российской академии наук
%Новосибирск 630090, пр. ак. Коптюга, 4 

\vspace*{3pt}

\noindent 
\textbf{Морозов Евсей Викторович} (р.\ 1947)~--- доктор 
фи\-зи\-ко-ма\-те\-ма\-ти\-че\-ских, профессор, ведущий научный сотрудник 
Института прикладных математических исследований Карельского научного центра Российской
академии наук; 
%%185910 Россия, Республика Карелия, г.\ Петрозаводск, ул.\ Пушкинская, 11
профессор Петрозаводского государственного университета, Петрозаводск
%185910 Россия, Республика Карелия, г.\ Петрозаводск, пр.\ Ленина, 33

%\pagebreak

\vspace*{3pt}

\noindent
\textbf{Назарова Ирина Александровна} (р.\ 1966)~---
кандидат фи\-зи\-ко-ма\-те\-ма\-ти\-че\-ских наук, 
научный сотрудник Вычислительного центра им.\ А.\,А.~Дородницына Российской академии наук 
%Адрес: 119333, Москва, ул. Вавилова, 40

\vspace*{3pt}

\noindent
\textbf{Павлов Игорь Валерианович} (р.\ 1945)~--- 
доктор фи\-зи\-ко-ма\-те\-ма\-ти\-че\-ских наук, профессор МГТУ им.\ Н.\,Э.~Баумана 
%Москва 105005, 2-я Бауманская ул., д.~5 

%\pagebreak

\vspace*{3pt}

\noindent 
\textbf{Потахина Любовь Викторовна} (р.\ 1989)~--- аспирантка
Института прикладных математических исследований Карельского научного центра
Российской академии наук; 
%%185910 Россия, Республика Карелия, г.\ Петрозаводск, ул.\ Пушкинская, 11
инженер Петрозаводского государственного университета, Петрозаводск
%185910 Россия, Республика Карелия, г.\ Петрозаводск, пр.\ Ленина, 33

\vspace*{3pt}

\noindent 
\textbf{Рождественский Юрий Владимирович} (р.\ 1952)~--- 
кандидат технических наук, заведующий сектором Института проб\-лем информатики
Российской академии наук

\vspace*{3pt}

\noindent 
\textbf{Синицын Игорь Николаевич} (р.\ 1940)~--- доктор технических наук,
профессор, заслуженный деятель\linebreak\vspace*{-12pt}

\columnbreak

\noindent
 науки РФ, заведующий отделом Института проб\-лем информатики
Российской академии наук

\vspace*{7pt}


\noindent
\textbf{Сиротинин Денис Олегович} (р.\ 1984)~---
кандидат технических наук, научный сотрудник Орловского филиала Института проб\-лем информатики
Российской академии наук
%302025, г.Орел, Московское шоссе, д.137

\vspace*{7pt}

%\columnbreak

\noindent 
\textbf{Соколов  Игорь Анатольевич} (р.\ 1954)~--- академик (действительный член) Российской 
академии наук, доктор технических наук, директор Института проб\-лем информатики
Российской академии наук

\vspace*{7pt}

\noindent
\textbf{Степченков Юрий Афанасьевич} (р.\ 1951)~---
кандидат технических наук, заведующий отделом Института проб\-лем информатики
Российской академии наук

\vspace*{7pt}

\noindent
\textbf{Сурков Алексей Викторович} (р.\ 1978)~--- 
старший научный сотрудник На\-уч\-но-ис\-сле\-до\-ва\-тель\-ско\-го 
института системных исследований Российской академии наук
%117218, Москва, Нахимовский просп., 36, к.1 

\vspace*{7pt}

\noindent 
\textbf{Шестаков Олег Владимирович} (р.\ 1976)~--- доктор 
фи\-зи\-ко-ма\-те\-ма\-ти\-че\-ских, доцент кафедры математической статистики 
факультета вычисли\-тельной математики и кибернетики Московского 
государственного университета им.\ М.\,В.~Ломоносова; 
%119991, Москва ГСП-1, Ленинские горы, д.\ 1, стр. 52
старший научный сотрудник Института проб\-лем информатики
Российской академии наук
%, Москва 119333, ул. Вавилова, д.~44, корп.~2

\vspace*{7pt}

\noindent 
\textbf{Шоргин Сергей Яковлевич} (р.\ 1952.)~--- доктор
фи\-зи\-ко-ма\-те\-ма\-ти\-че\-ских наук, профессор, заместитель директора Института 
проб\-лем информатики Российской академии наук





%%%%%%%%%%%%%%%%%%%%%%%%%%%%%%%%%%%%%%%%%%%%%%%%%%%%%%%%%%%%%%%%%%%%%%%%%%%%%%%




%\def\rightkol{ОБ АВТОРАХ}
%\def\leftkol{ОБ АВТОРАХ}

 \label{end\stat}





%\def\leftfootline{\small{\textbf{\thepage}
%\hfill ИНФОРМАТИКА И ЕЁ ПРИМЕНЕНИЯ\ \ \ том~7\ \ \ выпуск~1\ \ \ 2013}
%}%
% \def\rightfootline{\small{ИНФОРМАТИКА И ЕЁ ПРИМЕНЕНИЯ\ \ \ том~7\ \ \ выпуск~1\ \ \ 2013
%\hfill \textbf{\thepage}}}


%\thispagestyle{myheadings}



\end{multicols}

\newpage


%\vspace*{-48pt}
\begin{center}\LARGE
\textit{About Authors}
\end{center}

\thispagestyle{empty}
\def\tit{\ }

\def\aut{\ }

\def\auf{\ }


\def\leftkol{ABOUT AUTHORS}

\def\rightkol{ABOUT AUTHORS}

\vspace*{-18pt}

\titele{\tit}{\aut}{\auf}{\leftkol}{\rightkol}

%\vspace*{36pt}

\def\rightmark{{\noindent\hbox to \textwidth{\hfill\small ABOUT AUTHORS
%\hfill \large\bf\thepage
}}}
\def\leftmark{{\noindent\parbox{\textwidth}{
%\raggedleft\large\bf\thepage \hfill
\small\textrm{ABOUT AUTHORS}\hfill}}}


\def\leftfootline{\small{\textbf{\thepage}
\hfill ИНФОРМАТИКА И ЕЁ ПРИМЕНЕНИЯ\ \ \ том~6\ \ \ выпуск~2\ \ \ 2012}
}%
 \def\rightfootline{\small{ИНФОРМАТИКА И ЕЁ ПРИМЕНЕНИЯ\ \ \ том~6\ \ \ выпуск~2\ \ \ 2012
\hfill \textbf{\thepage}}}


\begin{multicols}{2}

\noindent
\textbf{Agalarov Yaver M.} (b.\ 1952)~--- Candidate of Science (PhD)
in technology, 
leading scientist, Institute of Informatics Problems, Russian Academy of Sciences

\vspace*{5pt}


  \noindent
\textbf{Bosov Alexey V.} (b.\ 1969)~--- Doctor of Science in technology, Head of
Laboratory, Institute of Informatics Problems, Russian Academy of Sciences

\vspace*{5pt}


\noindent
\textbf{Dulin Sergey K.} (b.\ 1950)~--- Doctor of Science in technology, 
professor, senior scientist, Institute of Informatics Problems, Russian Academy of Sciences

\vspace*{5pt}

\noindent
\textbf{Gorshenin Andrey K.}~--- (b.\ 1986)~--- Candidate of Science (PhD)
in physics and mathematics,
senior scientist, Institute of Informatics Problems, Russian Academy of Sciences

\vspace*{5pt}

\noindent
\textbf{Kalenov Nikolay E.}  (b.\ 1945)~--- Doctor of Science in technology,
professor, Director, Library for Natural Sciences,  Russian Academy of Sciences 

\vspace*{5pt}

\noindent
\textbf{Kalinichenko Leonid A.} (b.\ 1937)~--- Doctor of Science in physics and mathematics, 
professor, Honored scientist of RF, 
Head of Laboratory, Institute of Informatics Problems, Russian Academy of Sciences 

\vspace*{5pt}

\noindent
\textbf{Karpov Alexey A.} (b.\ 1978)~--- Candidate of Science (PhD) in technology, 
senior scientist, St.\ Petersburg Institute for
Informatics and Automation,  Russian Academy of Sciences

\vspace*{5pt}

\noindent
\textbf{Kuznetsov Igor P.} (b.\ 1938)~--- Doctor of Science in technology, 
professor, principal scientist, Institute of Informatics Problems, Russian Academy of Sciences

\vspace*{5pt}


\noindent
\textbf{Markova Natalia A.} (b.\ 1950)~--- Candidate of Science (PhD) in
physics and mathematics, leading scientist,  
Institute of Informatics Problems, Russian Academy of Sciences

\vspace*{5pt}

\noindent
\textbf{Nikolaev Andrey V.} (b.\ 1985)~--- Candidate of Science (PhD) in technology, 
senior lecturer, Tchaikovsky Technological Institute, Branch of the Izhevsk State Technical 
University

\vspace*{6pt}

\noindent
\textbf{Pavlov Igor V.} (b.\ 1945)~---  Doctor of Science in physics and mathematics,
professor, Bauman Moscow State Technical University

\vspace*{6pt}

%\columnbreak

\noindent
\textbf{Rozenberg Igor N.} (b.\ 1965)~--- Doctor of Science in technology, 
First Deputy Director General, Research \& Design Institute for Information 
Technology, Signalling and Telecommunications on Railway Transport (JSC NIIAS)

\vspace*{6pt}


\noindent
\textbf{Semenov Konstantin K.} (b.\ 1986)~--- MPhil, 
associate professor, St.\ Petersburg State Polytechnical University

\vspace*{6pt}

\noindent
\textbf{Sharnin Mikhail M.} (b.\ 1959)~--- Candidate of Science (PhD) 
in technology, senior scientist, Institute of Informatics Problems, Russian Academy of Sciences

\vspace*{6pt}

\noindent 
\textbf{Shestakov Oleg V.} (b.\ 1976)~--- Candidate of Science (PhD) in physics and mathematics,
associate professor, Department of Mathematical Statistics, Faculty of Computational Mathematics and Cybernetics,
M.\,V.~Lomonosov Moscow State University; senior scientist, Institute of Informatics Problems, 
Russian Academy of Sciences

\vspace*{6pt}

\noindent
\textbf{Stupnikov Sergey A.} (b.\ 1978)~--- Candidate of Science (PhD) in technology, 
senior scientist, Institute of Informatics Problems, Russian Academy of Sciences 

\vspace*{6pt}

\noindent
\textbf{Umansky Vladimir I.} (b.\ 1954)~--- Candidate of Science (PhD) in technology, 
Director General, ``IntechGeoTrans'' Closed Joint Stock Company

\vspace*{6pt}

\noindent
\textbf{Zhevnerchuk Dmitry V.} (b.\ 1978)~--- Candidate of Science (PhD) in technology, 
associate professor, Tchaikovsky Technological Institute, Branch of the Izhevsk State 
Technical University

%\vspace*{6pt}

\def\leftfootline{\small{\textbf{\thepage}
\hfill ИНФОРМАТИКА И ЕЁ ПРИМЕНЕНИЯ\ \ \ том~6\ \ \ выпуск~2\ \ \ 2012}
}%
 \def\rightfootline{\small{ИНФОРМАТИКА И ЕЁ ПРИМЕНЕНИЯ\ \ \ том~6\ \ \ выпуск~2\ \ \ 2012
\hfill \textbf{\thepage}}}



%\thispagestyle{myheadings}

\end{multicols}
\newpage

\def\stat{cont}
{%\hrule\par
%\vskip 7pt % 7pt
\raggedleft\Large \bf%\baselineskip=3.2ex
А\,В\,Т\,О\,Р\,С\,К\,И\,Й\ \ У\,К\,А\,З\,А\,Т\,Е\,Л\,Ь\ \ З\,А\ \ 2\,0\,1\,0 г. \vskip 17pt
    \hrule
    \par
\vskip 21pt plus 6pt minus 3pt }

\label{st\stat}

\def\tit{\ }

\def\aut{\ }
\def\auf{\ }

\def\leftkol{\ } % ENGLISH ABSTRACTS}

\def\rightkol{\ } %АВТОРСКИЙ УКАЗАТЕЛЬ ЗА 2010 г.} %ENGLISH ABSTRACTS}

\titele{\tit}{\aut}{\auf}{\leftkol}{\rightkol}

\vspace*{-12pt}

{\tabcolsep=3pt
\begin{tabular}{p{388pt}rr}
&\textbf{Выпуск} & \textbf{Стр.}\\[6pt]
\hangindent=23pt\noindent\textbf{Арутюнян~А.\,Р.} Моделирование влияния деформаций отпечатков пальцев на 
точность\linebreak
\vspace*{-12pt}\\
\hspace*{23pt}дактилоскопической идентификации$\dotfill$&1&51\\
\hangindent=23pt\noindent\textbf{Архипов~О.\,П., Зыкова~З.\,П.} Интеграция гетерогенной информации о цветных 
пикселях\linebreak
\vspace*{-12pt}\\
\hspace*{23pt}и их цветовосприятии$\dotfill$&4&15\\
\hangindent=23pt\noindent\textbf{Баранов~С.\,И., Френкель~С.\,Л., Захаров~В.\,Н.} Полуформальная верификация 
цифрового устройства с конвейером, основанная на использовании алгоритмических машин\linebreak
\vspace*{-12pt}\\
\hspace*{23pt}состояния$\dotfill$&4&49\\
\textbf{Бекетова~И.\,В.} см.~Каратеев~С.\,Л.&&\\
\textbf{Белоусов~В.\,В.} см.~Синицын~И.\,Н.&&\\
\hangindent=23pt\noindent\textbf{Бенинг~В.\,Е., Королев~Р.\,А.} О предельном поведении мощностей критериев в 
случае\linebreak
\vspace*{-12pt}\\
\hspace*{23pt}распределения Лапласа$\dotfill$&2&63\\
\hangindent=23pt\noindent\textbf{Бенинг~В.\,Е., Сипина~А.\,В.} Асимптотическое разложение для мощности 
критерия,\linebreak
\vspace*{-12pt}\\
\hspace*{23pt}основанного на выборочной медиане, в случае распределения Лапласа$\dotfill$&1&18\\
\textbf{Бондаренко~А.\,В.} см.~Каратеев~С.\,Л.&&\\
\hangindent=23pt\noindent\textbf{Бородина~А.\,В., Морозов~Е.\,В.} Об оценивании асимптотики вероятности 
большого\linebreak
\vspace*{-12pt}\\
\hspace*{23pt}уклонения стационарной регенеративной очереди с одним прибором$\dotfill$&3&29\\
\hangindent=23pt\noindent\textbf{Бунтман~Н.\,В., Минель~Ж.-Л., Ле~Пезан~Д., Зацман~И.\,М.} Типология и 
компьютерное\linebreak
\vspace*{-12pt}\\
\hspace*{23pt}моделирование трудностей перевода$\dotfill$&3&77\\
\textbf{Визильтер~Ю.\,В.} см.~Каратеев~С.\,Л.&&\\
\hangindent=23pt\noindent\textbf{Гавриленко~С.\,В.} Оценки скорости сходимости распределений случайных сумм с 
безгранично делимыми индексами к нормальному закону$\dotfill$&4&81\\
\hangindent=23pt\noindent\textbf{Григорьева~М.\,Е., Шевцова~И.\,Г.} Уточнение неравенства 
Каца--Берри--Эссеена$\dotfill$&2&75\\
\hangindent=23pt\noindent\textbf{Грушо~А.\,А., Грушо~Н.\,А., Тимонина~Е.\,Е.} Поиск конфликтов в политиках 
безопасности: модель случайных графов$\dotfill$&3&38\\
\textbf{Грушо~Н.\,А.} см.~Грушо~А.\,А.&&\\
\hangindent=23pt\noindent\textbf{Гудков~В.\,Ю.} Математические модели изображения отпечатка пальца на основе 
описания линий$\dotfill$&1&58\\
\textbf{Гуртов~А.\,В.} см.~Лукьяненко~А.\,С.&&\\
\textbf{Желтов~С.\,Ю.} см.~Каратеев~С.\,Л.&&\\
\hangindent=23pt\noindent\textbf{Захаров~А.\,А., Серебряков~В.\,А.} Система управления электронной библиотекой 
LibMeta$\dotfill$&4&2\\
\textbf{Захаров~В.\,Н.} см.~Баранов~С.\,И.&&\\
\textbf{Захарова~Т.\,В.} см.~Матвеева~С.\,С.&&\\
\hangindent=23pt\noindent\textbf{Зацаринный~А.\,А., Чупраков~К.\,Г.} Некоторые аспекты выбора технологии для 
постро-\linebreak
\vspace*{-12pt}\\
\hspace*{23pt}ения систем отображения информации ситуационного центра$\dotfill$&3&59\\
\textbf{Зацман~И.\,М.} см.~Бунтман~Н.\,В.&&\\
\hangindent=23pt\noindent\textbf{Зейфман~А.\,И., Коротышева~А.\,В., Сатин~Я.\,А., Шоргин~С.\,Я.} Об 
устойчивости нестаци-\linebreak
\vspace*{-12pt}\\
\hspace*{23pt}онарных систем обслуживания с катастрофами$\dotfill$&3&9\\
\textbf{Зыкова~З.\,П.} см.~Архипов~О.\,П.&&\\
\hangindent=23pt\noindent\textbf{Илюшин~Г.\,Я., Соколов~И.\,А.} Организация управляемого доступа пользователей 
к\linebreak
\vspace*{-12pt}\\
\hspace*{23pt}разнородным ведомственным информационным ресурсам$\dotfill$&1&24\\
\hangindent=23pt\noindent\textbf{Кавагучи~Ю., Ульянов~В.\,В., Фуджикоши~Я.} Приближения для статистик, 
описывающих\linebreak
\vspace*{-12pt}\\
\hspace*{23pt}геометрические свойства данных большой размерности, с оценками 
ошибок$\dotfill$&1&12\\
\hangindent=23pt\noindent\textbf{Каратеев~С.\,Л., Бекетова~И.\,В., Ососков~М.\,В., Князь~В.\,А., 
Визильтер~Ю.\,В., Бондаренко~А.\,В., Желтов~С.\,Ю.} Автоматизированный контроль 
качества цифровых\linebreak
\vspace*{-12pt}\\
\hspace*{23pt}изображений для персональных документов$\dotfill$&1&65\\
\end{tabular}
}

\pagebreak

\def\leftkol{АВТОРСКИЙ УКАЗАТЕЛЬ ЗА 2010 г.} % ENGLISH ABSTRACTS}

\def\rightkol{АВТОРСКИЙ УКАЗАТЕЛЬ ЗА 2010 г.} %ENGLISH ABSTRACTS}

{\tabcolsep=3pt
\begin{tabular}{p{388pt}rr}
&\textbf{Выпуск} & \textbf{Стр.}\\[3pt]
\hangindent=23pt\noindent\textbf{Козеренко~Е.\,Б.} Лингвистические фильтры в статистических моделях машинного\linebreak
\vspace*{-12pt}\\
\hspace*{23pt}перевода$\dotfill$&2&83\\
\hangindent=23pt\noindent\textbf{Козеренко~Е.\,Б., Кузнецов~И.\,П.} Когнитивно-лингвистические представления в 
систе-\linebreak
\vspace*{-12pt}\\
\hspace*{23pt}мах обработки текстов$\dotfill$&3&69\\
\textbf{Князь~В.\,А.} см.~Каратеев~С.\,Л.&&\\
\hangindent=23pt\noindent\textbf{Колесников~А.\,В., Солдатов~С.\,А.} Алгоритм координации для гибридной 
интеллектуальной системы решения сложной задачи оперативно-производственного\linebreak
\vspace*{-12pt}\\
\hspace*{23pt}планирования$\dotfill$&4&61\\
\hangindent=23pt\noindent\textbf{Коновалов~М.\,Г.} О планировании потоков в системах вычислительных 
ресурсов$\dotfill$&2&3\\
\textbf{Конушин~А.\,С.} см.~Конушин~В.\,С.&&\\
\hangindent=23pt\noindent\textbf{Конушин~В.\,С., Кривовязь~Г.\,Р., Конушин~А.\,С.} Алгоритм распознавания людей 
в видео-\linebreak
\vspace*{-12pt}\\
\hspace*{23pt}последовательности по одежде$\dotfill$&1&74\\
\textbf{Корепанов~Э.\, Р.} см.~Синицын~И.\,Н.&&\\
\textbf{Королев~В.\,Ю.} см.~Соколов~И.\,А.&&\\
\textbf{Королев~Р.\,А.} см.~Бенинг~В.\,Е.&&\\
\textbf{Коротышева~А.\,В.} см.~Зейфман~А.\,И.&&\\
\hangindent=23pt\noindent\textbf{Кривенко~М.\,П.} Непараметрическое оценивание элементов байесовского 
клас\-си-\linebreak
\vspace*{-12pt}\\
\hspace*{23pt}фикатора$\dotfill$&2&13\\
\textbf{Кривовязь~Г.\,Р.} см.~Конушин~В.\,С.&&\\
\textbf{Крылов~А.\,С.} см.~Павельева~Е.\,А.&&\\
\hangindent=23pt\noindent\textbf{Крылов~В.\,А.} Моделирование и классификация многоканальных дистанционных\linebreak
\vspace*{-12pt}\\
\hspace*{23pt}изображений с использованием копул$\dotfill$&4&34\\
\hangindent=23pt\noindent\textbf{Крючин~О.\,В.} Разработка параллельных эвристических алгоритмов подбора 
весовых\linebreak
\vspace*{-12pt}\\
\hspace*{23pt}коэффициентов искусственной нейтронной сети$\dotfill$&2&53\\
\hangindent=23pt\noindent\textbf{Кудрявцев~А.\,А., Шоргин~С.\,Я.} Байесовские модели массового обслуживания и 
надеж-\linebreak
\vspace*{-12pt}\\
\hspace*{23pt}ности: характеристики среднего числа заявок в системе $M\vert M \vert 1\vert 
\infty$$\dotfill$&3&16\\
\hangindent=23pt\noindent\textbf{Кузнецов~А.\,А.} Связь между временными и структурно-топологическими 
характери-\linebreak
\vspace*{-12pt}\\
\hspace*{23pt}стиками диаграмм ритма сердца здоровых людей$\dotfill$&4&39\\
\textbf{Кузнецов~И.\,П.} см.~Козеренко~Е.\,Б.&&\\
\textbf{Ле~Пезан~Д.} см.~Бунтман~Н.\,В.&&\\
\hangindent=23pt\noindent\textbf{Лукьяненко~А.\,С., Морозов~Е.\,В., Гуртов~А.\,В.} Анализ сетевого протокола с общей 
функ-\linebreak
\vspace*{-12pt}\\
\hspace*{23pt}цией расширения окна передачи сообщения при конфликтах$\dotfill$&2&46\\
\hangindent=23pt\noindent\textbf{Лямин~О.\,О.} О предельном поведении мощностей критериев в случае обобщенного\linebreak
\vspace*{-12pt}\\
\hspace*{23pt}распределения Лапласа$\dotfill$&3&47\\
\hangindent=23pt\noindent\textbf{Маркин~А.\,В., Шестаков~О.\,В.} Асимптотики оценки риска при пороговой 
обработке\linebreak
\vspace*{-12pt}\\
\hspace*{23pt}вейвлет-вейглет коэффициентов в задаче томографии$\dotfill$&2&36\\
\hangindent=23pt\noindent\textbf{Матвеева~С.\,С., Захарова~Т.\,В.} Сети массового обслуживания с наименьшей 
длиной\linebreak
\vspace*{-12pt}\\
\hspace*{23pt}очереди$\dotfill$&3&22\\
\hangindent=23pt\noindent\textbf{Матюшенко~С.\,И.} Стационарные характеристики двухканальной системы 
обслужива-\linebreak
\vspace*{-12pt}\\
\hspace*{23pt}ния с переупорядочиванием заявок и распределениями фазового типа$\dotfill$&4&68\\
\textbf{Минель~Ж.-Л.} см.~Бунтман~Н.\,В.&&\\
\textbf{Морозов~Е.\,В.} см.~Бородина~А.\,В.&&\\
\textbf{Морозов~Е.\,В.} см.~Лукьяненко~А.\,С.&&\\
\textbf{Ососков~М.\,В.} см.~Каратеев~С.\,Л.&&\\
\hangindent=23pt\noindent\textbf{Павельева~Е.\,А., Крылов~А.\,С.} Поиск и анализ ключевых точек радужной 
оболочки\linebreak
\vspace*{-12pt}\\
\hspace*{23pt}глаза методом преобразования Эрмита$\dotfill$&1&79\\
\textbf{Печинкин~А.\,В.} см.~Френкель~С.\,Л.,&&\\
\hangindent=23pt\noindent\textbf{Протасов~В.\,И.} Составление субъективного портрета с использованием 
эволюционно-\linebreak
\vspace*{-12pt}\\
\hspace*{23pt}го морфинга и квалиметрия метода$\dotfill$&1&83\\
\hangindent=23pt\noindent\textbf{Рудаков~К.\,В., Торшин~И.\,Ю.} Вопросы разрешимости задачи распознавания 
вторичной\linebreak
\vspace*{-12pt}\\
\hspace*{23pt}структуры белка$\dotfill$&2&25\\
\textbf{Сатин~Я.\,А.} см.~Зейфман~А.\,И.&&\\
\hangindent=23pt\noindent\textbf{Сейфуль-Мулюков~Р.\,Б.} Нефть как носитель информации о своем 
происхождении,\linebreak
\vspace*{-12pt}\\
\hspace*{23pt}структуре и эволюции$\dotfill$&1&41\\
\end{tabular}
}

{\tabcolsep=3pt
\begin{tabular}{p{388pt}rr}
&\textbf{Выпуск} & \textbf{Стр.}\\[6pt]
\textbf{Семендяев~Н.\,Н.} см.~Синицын~И.\,Н.&&\\
\textbf{Серебряков~В.\,А.} см.~Захаров~А.\,А.&&\\
\textbf{Синицын~В.\,И.} см.~Синицын~И.\,Н.&&\\
\hangindent=23pt\noindent\textbf{Синицын~И.\,Н., Синицын~В.\,И., Корепанов~Э.\, Р., Белоусов~В.\,В., 
Семендяев~Н.\,Н.} Оперативное построение информационных моделей движения полюса 
Земли\linebreak
\vspace*{-12pt}\\
\hspace*{23pt}методами линейных и линеаризованных фильтров$\dotfill$&1&2\\
\textbf{Сипина~А.\,В.} см.~Бенинг~В.\,Е.&&\\
\hangindent=23pt\noindent\textbf{Соколов~И.\,А.} О работах заслуженного деятеля науки Российской Федерации 
И.\,Н.~Синицына в области информационных технологий и автоматизации (к 70-летию\linebreak
\vspace*{-12pt}\\
\hspace*{23pt}со дня рождения)$\dotfill$&3&84\\
\textbf{Соколов~И.\,А.} см.~Илюшин~Г.\,Я.&&\\
\hangindent=23pt\noindent\textbf{Соколов~И.\,А., Королев~В.\,Ю.} Предисловие$\dotfill$&2&2\\
\textbf{Солдатов~С.\,А.} см.~Колесников~А.\,В.&&\\
\hangindent=23pt\noindent\textbf{Степанов~С.\,Ю.} Использование координатного метода фрагментации 
коммутаторной\linebreak
\vspace*{-12pt}\\
\hspace*{23pt}нейронной сети для сокращения трафика$\dotfill$&2&57\\
\textbf{Тимонина~Е.\,Е.} см.~Грушо~А.\,А.&&\\
\textbf{Торшин~И.\,Ю.} см.~Рудаков~К.\,В.&&\\
\textbf{Ульянов~В.\,В.} см.~Кавагучи~Ю.&&\\
\textbf{Фазекаш~И.} см.~Чупрунов~А.\,Н.&&\\
\textbf{Френкель~С.\,Л.} см.~Баранов~С.\,И.&&\\
\hangindent=23pt\noindent\textbf{Френкель~С.\,Л., Печинкин~А.\,В.} Оценка времени самовосстановления в 
цифровых\linebreak
\vspace*{-12pt}\\
\hspace*{23pt}системах после сбоев, вызываемых переходными помехами$\dotfill$&3&2\\
\textbf{Фуджикоши~Я.} см.~Кавагучи~Ю.&&\\
\hangindent=23pt\noindent\textbf{Цискаридзе~А.\,К.} Математическая модель и метод восстановления позы человека 
по\linebreak
\vspace*{-12pt}\\
\hspace*{23pt}стереопаре силуэтных изображений$\dotfill$&4&27\\
\hangindent=23pt\noindent\textbf{Чупраков~К.\,Г.} К вопросу о размещении коллективных средств отображения в 
ситуа-\linebreak
\vspace*{-12pt}\\
\hspace*{23pt}ционном зале с заданными параметрами$\dotfill$&4&89\\
\textbf{Чупраков~К.\,Г.} см.~Зацаринный~А.\,А.&&\\
\hangindent=23pt\noindent\textbf{Чупрунов~А.\,Н., Фазекаш~И.} Законы повторного логарифма для числа 
безошибочных\linebreak
\vspace*{-12pt}\\
\hspace*{23pt}блоков при помехоустойчивом кодировании$\dotfill$&3&42\\
\textbf{Шевцова~И.\,Г.} см.~Григорьева~М.\,Е.&&\\
\hangindent=23pt\noindent\textbf{Шестаков~О.\,В.} Аппроксимация распределения оценки риска пороговой 
обработки вейвлет-коэффициентов нормальным распределением при использовании 
выбо-\linebreak
\vspace*{-12pt}\\
\hspace*{23pt}рочной дисперсии$\dotfill$&4&73\\
\textbf{Шестаков~О.\,В.} см.~Маркин~А.\,В.&&\\
\textbf{Шоргин~С.\,Я.} см.~Зейфман~А.\,И.&&\\
\textbf{Шоргин~С.\,Я.} см.~Кудрявцев~А.\,А.&&\\
\end{tabular}
}

%\thispagestyle{myheadings}
\def\leftfootline{\small{\textbf{\thepage}
\hfill ИНФОРМАТИКА И ЕЁ ПРИМЕНЕНИЯ\ \ \ том~4\ \ \ выпуск~4\ \ \ 2010}
}%
 \def\rightfootline{\small{ИНФОРМАТИКА И ЕЁ ПРИМЕНЕНИЯ\ \ \ том~4\ \ \ выпуск~4\ \ \ 2010
 \hfill \textbf{\thepage}}}
 \label{end\stat}


%Том 10 Выпуск 1-4 Год 2016

\def\stat{cont-e}
{%\hrule\par
%\vskip 7pt % 7pt
\raggedleft\Large \bf%\baselineskip=3.2ex
2\,0\,1\,6\ \ A\,U\,T\,H\,O\,R\ \ I\,N\,D\,E\,X \vskip 17pt
 \hrule
 \par
\vskip 21pt plus 6pt minus 3pt }

\label{st\stat}

\def\tit{\ }

\def\aut{\ }
\def\auf{\ }

\def\leftkol{\ } %2016 AUTHOR INDEX} % ENGLISH ABSTRACTS}

\def\rightkol{\ } %2016 AUTHOR INDEX} %ENGLISH ABSTRACTS}

\titele{\tit}{\aut}{\auf}{\leftkol}{\rightkol}

\def\leftfootline{\small{\textbf{\thepage}
\hfill INFORMATIKA I EE PRIMENENIYA~--- INFORMATICS AND APPLICATIONS\ \ \ 2016\
\ \ volume~10\ \ \ issue\ 4}
}%
 \def\rightfootline{\small{INFORMATIKA I EE PRIMENENIYA~--- INFORMATICS AND APPLICATIONS\ \ \ 2016\ \ \ volume~10\ \ \ issue\ 4
\hfill \textbf{\thepage}}}

\vspace*{-12pt}
\vspace*{-18pt}

{\tabcolsep=2.8pt
\begin{tabular}{p{382pt}cc}
&\textbf{Issue} & \textbf{Page}\\[6pt]
\Avtors{Agalarov~M.\,Ya.} see~Agalarov~Ya.\,M.&&\\
\Avtors{Agalarov~Ya.\,M., Agalarov~M.\,Ya., and
Shorgin~V.\,S.} About the optimal threshold of queue\linebreak
\\[-12pt]
\hspace*{23pt}length in a~particular problem of profit maximization
in the $M/G/1$ queuing system&2&70--79\\
\Avtors{Alexeyevsky~D.\,A.} BioNLP ontology extraction from 
a~restricted language corpus with\linebreak
\\[-12pt]
\hspace*{23pt}context-free grammars&1&119--128\\
\Avtors{Andreev~S.\,D.} see~Gaidamaka~Yu.\,V.&&\\
\Avtors{Andreev~S.\,D.} see~Ometov~A.\,Ya.&&\\
\Avtors{Arkhipov~O.\,P., Arkhipov~P.\,O., and Sidorkin~I.\,I.} The
option to create a~local coordinate\linebreak
\\[-12pt]
\hspace*{23pt}system for synchronization of selected images&3&91--97\\
\Avtors{Arkhipov~P.\,O.} see~Arkhipov~O.\,P.&&\\
\Avtors{Belousov~V.\,V.} see~Shnurkov~P.\,V.&&\\
\Avtors{Belousov~V.\,V.} see~Shnurkov~P.\,V.&&\\
\Avtors{Bening~V.\,E.} Calculation of~the~asymptotic deficiency
of~some statistical procedures based\linebreak
\\[-12pt]
\hspace*{23pt}on~samples with~random sizes&4&34--45\\
\Avtors{Borisov~A.\,V., Bosov~A.\,V., and Miller~G.\,B.} Modeling and
monitoring of VoIP connection&2&\hphantom{1}2--13\\
\Avtors{Bosov~A.\,V.} see~Borisov~A.\,V.&&\\
\Avtors{Briukhov~D.\,O.} see~Stupnikov~S.\,A.&&\\
\Avtors{Callaos~N.\,K.\ and Seyful-Mulyukov~R.\,B.} Complexity and
its information content&1&129--139\\
\Avtors{Chertok~A.\,V., Kadaner~A.\,I., Khazeeva~G.\,T., and
Sokolov~I.\,A.} Regime switching detection\linebreak
\\[-12pt]
\hspace*{23pt}for~the~Levy driven
Ornstein--Uhlenbeck process using CUSUM methods&4&46--56\\
\Avtors{Chichagov~V.\,V.} Asymptotic expansions of mean absolute
error of uniformly minimum variance unbiased and maximum likelihood
estimators on the one-parameter exponential\linebreak
\\[-12pt]
\hspace*{23pt}family model of lattice distributions&3&66--76\\
\Avtors{Danishevsky~V.\,I.} see~Kolesnikov A.\,V.&&\\
\Avtors{Fazliev~A.\,Z.} see~Kalinichenko~L.\,A.&&\\
\Avtors{Fedoseev~A.\,A.} What is behind the concept of ``knowledge in
small packages''&3&105--110\\
\Avtors{Gaidamaka~Yu.\,V., Andreev~S.\,D., Sopin~E.\,S.,
Samouylov~K.\,E., and Shorgin~S.\,Ya.} Interference analysis
of~the~device-to-device communications model with~regard to~a~signal\linebreak
\\[-12pt]
\hspace*{23pt}propagation environment&4&\hphantom{1}2--10\\
\Avtors{Gasilov~A.\,V.} see~Yakovlev~O.\,A.&&\\
\Avtors{Goncharov~A.\,V.\ and Strijov~V.\,V.} Metric time series
classification using weighted dynamic\linebreak
\\[-12pt]
\hspace*{23pt}warping relative to centroids of classes&2&36--47\\
\Avtors{Gordov~E.\,P.} see~Kalinichenko~L.\,A.&&\\
\Avtors{Gorshenin~A.\,K.} Concept of online service for stochastic
modeling of real processes&1&72--81\\
\Avtors{Gorshenin~A.\,K.} see~Shnurkov~P.\,V.&&\\
\Avtors{Gorshenin~A.\,K.} see~Shnurkov~P.\,V.&&\\
\Avtors{Grusho~A.\,A., Grusho~N.\,A., Zabezhailo~M.\,I., and
Timonina~E.\,E.} Integration of statistical and\linebreak
\\[-12pt]
\hspace*{23pt}deterministic methods for
analysis of information security&3&2--8\\
\Avtors{Grusho~A.\,A., Zabezhailo~M.\,I., and Zatsarinny~A.\,A.} On
the advanced procedure to reduce\linebreak
\\[-12pt]
\hspace*{23pt}calculation of Galois closures&4&\hphantom{1}96--104\\
\Avtors{Grusho~N.\,A.} see~Grusho~A.\,A.&&\\
\Avtors{Havanskov~V.\,A.} see~Minin~V.\,A.&&\\
\Avtors{Inkova~O.\,Yu.} see~Zatsman~I.\,M.&&\\
\Avtors{Isachenko~R.\,V.\ and Strijov~V.\,V.} Metric learning in
multiclass time series classification\linebreak
\\[-12pt]
\hspace*{23pt}problem&2&48--57\\
\end{tabular}
}
\pagebreak

\def\leftfootline{\small{\textbf{\thepage}
\hfill INFORMATIKA I EE PRIMENENIYA~--- INFORMATICS AND APPLICATIONS\ \ \ 2016\
\ \ volume~10\ \ \ issue\ 4}
}%
 \def\rightfootline{\small{INFORMATIKA I EE PRIMENENIYA~---
INFORMATICS AND APPLICATIONS\ \ \ 2016\ \ \ volume~10\ \ \ issue\ 4
\hfill \textbf{\thepage}}}

\def\leftkol{2016 AUTHOR INDEX} % ENGLISH ABSTRACTS}

\def\rightkol{2016 AUTHOR INDEX} %ENGLISH ABSTRACTS}


{\tabcolsep=2.83pt
\begin{tabular}{p{382pt}cc}
&\textbf{Issue} & \textbf{Page}\\[6pt]
\Avtors{Kadaner~A.\,I.} see~Chertok~A.\,V.&&\\[.255pt]
\Avtors{Kalinichenko~L.\,A., Volnova~A.\,A., Gordov~E.\,P.,
Kiselyova~N.\,N., Kovaleva~D.\,A., Malkov~O.\,Yu., Okladnikov~I.\,G.,
Podkolodnyy~N.\,L., Pozanenko~A.\,S., Ponomareva~N.\,V.,
Stupnikov~S.\,A.,} \textbf{and Fazliev~A.\,Z.} Data access challenges for data
intensive\linebreak
\\[-12pt]
\hspace*{23pt}research in Russia&1& 2--22\\[.255pt]
\Avtors{Karasikov~M.\,E.\ and Strijov~V.\,V.} Feature-based
time-series classification&4&121--131\\[.255pt]
\Avtors{Khazeeva~G.\,T.} see~Chertok~A.\,V.&&\\[.255pt]
\Avtors{Khokhlov~Yu.\,S.} Multivariate fractional Levy motion and its
applications&2&\hphantom{1}98--106\\[.255pt]
\Avtors{Kirikov~I.\,A., Kolesnikov~A.\,V., Listopad~S.\,V., and
Rumovskaya~S.\,B.} Fine-grained hybrid\linebreak
\\[-12pt]
\hspace*{23pt}intelligent systems. Part 2:
Bidirectional hybridization&1&\hphantom{1}96--105\\[.255pt]
\Avtors{Kirikov~I.\,A., Kolesnikov~A.\,V., Listopad~S.\,V., and
Rumovskaya~S.\,B.} ``Virtual council''~---\linebreak
\\[-12pt]
\hspace*{23pt}source environment
supporting complex diagnostic decision making&3&81--90\\[.255pt]
\Avtors{Kiselyova~N.\,N.} see~Kalinichenko~L.\,A.&&\\[.255pt]
\Avtors{Kolesnikov A.\,V., Listopad~S.\,V., Rumovskaya~S.\,B., and
Danishevsky~V.\,I.} Informal axiomatic\linebreak
\\[-12pt]
\hspace*{23pt}theory of~the~role visual models&4&114--120\\[.255pt]
\Avtors{Kolesnikov~A.\,V.} see~Kirikov~I.\,A.&&\\[.255pt]
\Avtors{Kolesnikov~A.\,V.} see~Kirikov~I.\,A.&&\\[.255pt]
\Avtors{Kolin~K.\,K.} Humanitarian aspects of information
security&3&111--121\\[.255pt]
\Avtors{Konovalov~M.\,G.\ and Razumchik~R.\,V.} Dispatching
to~two parallel nonobservable queues using\linebreak
\\[-12pt]
\hspace*{23pt}only static
information&4&57--67\\[.255pt]
\Avtors{Korchagin~A.\,Yu.} see~Korolev~V.\,Yu.&&\\[.255pt]
\Avtors{Korchagin~A.\,Yu.} see~Korolev~V.\,Yu.&&\\[.255pt]
\Avtors{Korepanov~E.\,R.} see~Sinitsyn~I.\,N.&&\\[.255pt]
\Avtors{Korepanov~E.\,R.} see~Sinitsyn~I.\,N.&&\\[.255pt]
\Avtors{Korolev~V.\,Yu., Korchagin~A.\,Yu., and Zeifman~A.\,I.} The
Poisson theorem for Bernoulli trials\linebreak
\\[-12pt]
\hspace*{23pt}with~a~random probability
of~success and~a~discrete analog of~the~Weibull distribution&4&11--20\\[.255pt]
\Avtors{Korolev~V.\,Yu., Zeifman~A.\,I., and Korchagin~A.\,Yu.}
Asymmetric Linnik distributions as~limit\linebreak
\\[-12pt]
\hspace*{23pt}laws for~random sums
of~independent random variables with~finite variances&4&21--33\\[.255pt]
\Avtors{Koucheryavy~E.\,A.} see~Ometov~A.\,Ya.&&\\[.255pt]
\Avtors{Kovaleva~D.\,A.} see~Kalinichenko~L.\,A.&&\\[.255pt]
\Avtors{Kovalyov~S.\,P.} Metaprogramming to increase
manufacturability of large-scale software-\linebreak
\\[-12pt]
\hspace*{23pt}intensive systems&1&56--66\\[.255pt]
\Avtors{Krivenko~M.\,P.} Significance tests of feature selection for
classification&3&32--40\\[.255pt]
\Avtors{Kruzhkov~M.\,G.} see~Zalizniak~Anna~A.&&\\[.255pt]
\Avtors{Kruzhkov~M.\,G.} see~Zatsman~I.\,M.&&\\[.255pt]
\Avtors{Kudryavtsev~A.\,A.} Bayesian queueing and reliability models:
\textit{A~priori} distributions with\linebreak
\\[-12pt]
\hspace*{23pt}compact support&1&67--71\\[.255pt]
\Avtors{Kudryavtsev~A.\,A.} Characteristics dependent on the balance
coefficient in Bayesian models\linebreak
\\[-12pt]
\hspace*{23pt}with compact support of \textit{a priori}
distributions&3&77--80\\[.255pt]
\Avtors{Kudryavtsev~A.\,A.\ and Palionnaia~S.\,I.} Bayesian recurrent
model of reliability growth:\linebreak
\\[-12pt]
\hspace*{23pt}Parabolic distribution of parameters&2&80--83\\[.255pt]
\Avtors{Kudryavtsev~A.\,A.\ and Titova~A.\,I.} Bayesian queuing
and~reliability models: Degenerate-\linebreak
\\[-12pt]
\hspace*{23pt}Weibull case&4&68--71\\[.255pt]
\Avtors{Leontyev~N.\,D.\ and Ushakov~V.\,G.} Analysis of a queueing
system with autoregressive arrivals\linebreak
\\[-12pt]
\hspace*{23pt}and nonpreemptive priority&3&15--22\\[.255pt]
\Avtors{Listopad~S.\,V.} see~Kirikov~I.\,A.&&\\[.255pt]
\Avtors{Listopad~S.\,V.} see~Kirikov~I.\,A.&&\\[.255pt]
\Avtors{Listopad~S.\,V.} see~Kolesnikov A.\,V.&&\\[.255pt]
\Avtors{Malkov~O.\,Yu.} see~Kalinichenko~L.\,A.&&\\[.255pt]
\Avtors{Markov~A.\,S., Monakhov~M.\,M., and
Ulyanov~V.\,V.} Generalized Cornish--Fisher expansions\linebreak
\\[-12pt]
\hspace*{23pt}for distributions of statistics based on samples
of random size&2&84--91\\[.255pt]
\Avtors{Melnikov~A.\,K.\ and Ronzhin~A.\,F.} Generalized statistical
method of~text analysis based\linebreak
\\[-12pt]
\hspace*{23pt}on~calculation of~probability distributions
of~statistical values&4&89--95\\
\end{tabular}
}
\pagebreak

\def\leftfootline{\small{\textbf{\thepage}
\hfill INFORMATIKA I EE PRIMENENIYA~--- INFORMATICS AND APPLICATIONS\ \ \ 2016\
\ \ volume~10\ \ \ issue\ 4}
}%
 \def\rightfootline{\small{INFORMATIKA I EE PRIMENENIYA~---
INFORMATICS AND APPLICATIONS\ \ \ 2016\ \ \ volume~10\ \ \ issue\ 4
\hfill \textbf{\thepage}}}

\def\leftkol{2016 AUTHOR INDEX} % ENGLISH ABSTRACTS}

\def\rightkol{2016 AUTHOR INDEX} %ENGLISH ABSTRACTS}


{\tabcolsep=3pt
\begin{tabular}{p{381pt}cc}
&\textbf{Issue} & \textbf{Page}\\[6pt]
\Avtors{Meykhanadzhyan~L.\,A.} Stationary characteristics of the finite
capacity queueing system with\linebreak
\\[-12pt]
\hspace*{23pt}inverse service order and generalized
probabilistic priority&2&123--131\\[.23pt]
\Avtors{Miller~G.\,B.} see~Borisov~A.\,V.&&\\[.23pt]
\Avtors{Minin~V.\,A., Zatsman~I.\,M., Havanskov~V.\,A., and
Shubnikov~S.\,K.} Intensity of citation of scientific publications in
inventions on information and computer technologies patented\linebreak
\\[-12pt]
\hspace*{23pt}in Russia by domestic and foreign applicants&2&107--122\\[.23pt]
\Avtors{Monakhov~M.\,M.} see~Markov~A.\,S.&&\\[.23pt]
\Avtors{Naumov~V.\,A.\ and Samouylov~K.\,E.} On relationship
between queuing systems with resources\linebreak
\\[-12pt]
\hspace*{23pt}and Erlang networks&3&\hphantom{1}9--14\\[.23pt]
\Avtors{Okladnikov~I.\,G.} see~Kalinichenko~L.\,A.&&\\[.23pt]
\Avtors{Ometov~A.\,Ya., Andreev~S.\,D., Turlikov~A.\,M., and
Koucheryavy~E.\,A.} Performance analysis of\linebreak
\\[-12pt]
\hspace*{23pt}a wireless data
aggregation system with contention for contemporary sensor
networks&3&23--31\\[.23pt]
\Avtors{Palionnaia~S.\,I.} see~Kudryavtsev~A.\,A.&&\\[.23pt]
\Avtors{Podkolodnyy~N.\,L.} see~Kalinichenko~L.\,A.&&\\[.23pt]
\Avtors{Ponomareva~N.\,V.} see~Kalinichenko~L.\,A.&&\\[.23pt]
\Avtors{Popkova~N.\,A.} see~Zatsman~I.\,M.&&\\[.23pt]
\Avtors{Pozanenko~A.\,S.} see~Kalinichenko~L.\,A.&&\\[.23pt]
\Avtors{Razumchik~R.\,V.} see~Konovalov~M.\,G.&&\\[.23pt]
\Avtors{Ronzhin~A.\,F.} see~Melnikov~A.\,K.&&\\[.23pt]
\Avtors{Rumovskaya~S.\,B.} see~Kirikov~I.\,A.&&\\[.23pt]
\Avtors{Rumovskaya~S.\,B.} see~Kirikov~I.\,A.&&\\[.23pt]
\Avtors{Rumovskaya~S.\,B.} see~Kolesnikov A.\,V.&&\\[.23pt]
\Avtors{Samouylov~K.\,E.} see~Gaidamaka~Yu.\,V.&&\\[.23pt]
\Avtors{Samouylov~K.\,E.} see~Naumov~V.\,A.&&\\[.23pt]
\Avtors{Serebryanskii~S.\,M.} see~Tyrsin~A.\,N.&&\\[.23pt]
\Avtors{Seyful-Mulyukov~R.\,B.} see~Callaos~N.\,K.&&\\[.23pt]
\Avtors{Shestakov~O.\,V.} Statistical properties of the denoising method
based on the stabilized hard\linebreak
\\[-12pt]
\hspace*{23pt}thresholding&2&65--69\\[.23pt]
\Avtors{Shestakov~O.\,V.} The strong law of large numbers for the risk
estimate in the problem of\linebreak
\\[-12pt]
\hspace*{23pt}tomographic image reconstruction from
projections with a correlated noise&3&41--45\\[.23pt]
\Avtors{Shestakov~O.\,V.} see~Zakharova~T.\,V.&&\\[.23pt]
\Avtors{Shnurkov~P.\,V., Gorshenin~A.\,K., and Belousov~V.\,V.}
Analytical solution of~the~optimal control\linebreak
\\[-12pt]
\hspace*{23pt}task of~a~semi-Markov
process with~finite set of~states&4&72--88\\[.23pt]
\Avtors{Shnurkov~P.\,V., Zasypko~V.\,V., Belousov~V.\,V., and
Gorshenin~A.\,K.} Development of the algorithm of numerical solution
of the optimal investment control problem\linebreak
\\[-12pt]
\hspace*{23pt}in the closed dynamical model of three-sector economy&1&82--95\\[.23pt]
\Avtors{Shorgin~S.\,Ya.} see~Gaidamaka~Yu.\,V.&&\\[.23pt]
\Avtors{Shorgin~V.\,S.} see~Agalarov~Ya.\,M.&&\\[.23pt]
\Avtors{Shubnikov~S.\,K.} see~Minin~V.\,A.&&\\[.23pt]
\Avtors{Sidorkin~I.\,I.} see~Arkhipov~O.\,P.&&\\[.23pt]
\Avtors{Sinitsyn~I.\,N.} Analytical modeling of processes in stochastic
systems with complex fractional\linebreak
\\[-12pt]
\hspace*{23pt}order Bessel nonlinearities&3&55--65\\[.23pt]
\Avtors{Sinitsyn~I.\,N.} Orthogonal supoptimal filters for nonlinear
stochastic systems on manifolds&1&34--44\\[.23pt]
\Avtors{Sinitsyn~I.\,N.\ and Korepanov~E.\,R.} Normal Pugachev
conditionally-optimal filters and extra-\linebreak
\\[-12pt]
\hspace*{23pt}polators for state linear stochastic systems&2&14--23\\[.23pt]
\Avtors{Sinitsyn~I.\,N.\ and Sinitsyn~V.\,I.} Analytical modeling of
distributions in stochastic systems on\linebreak
\\[-12pt]
\hspace*{23pt}manifolds based on ellipsoidal approximation&1&45--55\\[.23pt]
\Avtors{Sinitsyn~I.\,N., Sinitsyn~V.\,I., and
Korepanov~E.\,R.} Ellipsoidal suboptimal filters for nonlinear\linebreak
\\[-12pt]
\hspace*{23pt}stochastic systems on manifolds&2&24--35\\[.23pt]
\Avtors{Sinitsyn~V.\,I.} see~Sinitsyn~I.\,N.&&\\[.23pt]
\Avtors{Sinitsyn~V.\,I.} see~Sinitsyn~I.\,N.&&\\[.23pt]
\Avtors{Skvortsov~N.\,A.} see~Stupnikov~S.\,A.&&\\[.23pt]
\Avtors{Sokolov~I.\,A.} see~Chertok~A.\,V.&&\\
\end{tabular}
}
\pagebreak

\def\leftfootline{\small{\textbf{\thepage}
\hfill INFORMATIKA I EE PRIMENENIYA~--- INFORMATICS AND APPLICATIONS\ \ \ 2016\
\ \ volume~10\ \ \ issue\ 4}
}%
 \def\rightfootline{\small{INFORMATIKA I EE PRIMENENIYA~---
INFORMATICS AND APPLICATIONS\ \ \ 2016\ \ \ volume~10\ \ \ issue\ 4
\hfill \textbf{\thepage}}}

\def\leftkol{2016 AUTHOR INDEX} % ENGLISH ABSTRACTS}

\def\rightkol{2016 AUTHOR INDEX} %ENGLISH ABSTRACTS}


{\tabcolsep=3pt
\begin{tabular}{p{382pt}cc}
&\textbf{Issue} & \textbf{Page}\\[6pt]
\Avtors{Sopin~E.\,S.} see~Gaidamaka~Yu.\,V.&&\\
\Avtors{Strijov~V.\,V.} see~Goncharov~A.\,V.&&\\
\Avtors{Strijov~V.\,V.} see~Isachenko~R.\,V.&&\\
\Avtors{Strijov~V.\,V.} see~Karasikov~M.\,E.&&\\
\Avtors{Stupnikov~S.\,A., Briukhov~D.\,O., and Skvortsov~N.\,A.}
Co-lending systemic risk analysis over\linebreak
\\[-12pt]
\hspace*{23pt}heterogeneous data collections&1&23--33\\
\Avtors{Stupnikov~S.\,A.} see~Kalinichenko~L.\,A.&&\\
\Avtors{Suchkov~A.\,P.} see~Zatsarinny~A.\,A.&&\\
\Avtors{Timonina~E.\,E.} see~Grusho~A.\,A.&&\\
\Avtors{Titova~A.\,I.} see~Kudryavtsev~A.\,A.&&\\
\Avtors{Turlikov~A.\,M.} see~Ometov~A.\,Ya.&&\\
\Avtors{Tyrsin~A.\,N.\ and Serebryanskii~S.\,M.} Recognition of
dependences on the basis of inverse\linebreak
\\[-12pt]
\hspace*{23pt}mapping&2&58--64\\
\Avtors{Ulyanov~V.\,V.} see~Markov~A.\,S.&&\\
\Avtors{Ushakov~V.\,G.} Queueing system with working vacations and
hyperexponential input stream&2&92--97\\
\Avtors{Ushakov~V.\,G.} see~Leontyev~N.\,D.&&\\
\Avtors{Volnova~A.\,A.} see~Kalinichenko~L.\,A.&&\\
\Avtors{Yakovlev~O.\,A.\ and Gasilov~A.\,V.} Speeded-up stereo
matching using geodesic support weights&3&\hphantom{1}98--104\\
\Avtors{Zabezhailo~M.\,I.} see~Grusho~A.\,A.&&\\
\Avtors{Zabezhailo~M.\,I.} see~Grusho~A.\,A.&&\\
\Avtors{Zakharova~T.\,V.\ and Shestakov~O.\,V.} Precision analysis of
wavelet processing of aerodynamic\linebreak
\\[-12pt]
\hspace*{23pt}flow patterns&3&46--54\\
\Avtors{Zalizniak~Anna~A.\ and Kruzhkov~M.\,G.} Database
of~Russian impersonal verbal constructions&4&132--141\\
\Avtors{Zasypko~V.\,V.} see~Shnurkov~P.\,V.&&\\
\Avtors{Zatsarinny~A.\,A.\ and Suchkov~A.\,P.} Systems engineering
approaches to~the~establishment of\linebreak
\\[-12pt]
\hspace*{23pt}a~system for~decision support based
on~situational analysis&4&105--113\\
\Avtors{Zatsarinny~A.\,A.} see~Grusho~A.\,A.&&\\
\Avtors{Zatsman~I.\,M., Inkova~O.\,Yu., Kruzhkov~M.\,G., and
Popkova~N.\,A.} Representation of cross-\linebreak
\\[-12pt]
\hspace*{23pt}lingual knowledge about
connectors in supracorpora databases&1&106--118\\
\Avtors{Zatsman~I.\,M.} see~Minin~V.\,A.&&\\
\Avtors{Zeifman~A.\,I.} see~Korolev~V.\,Yu.&&\\
\Avtors{Zeifman~A.\,I.} see~Korolev~V.\,Yu.&&\\
\end{tabular}
}

%\thispagestyle{myheadings}
\def\leftfootline{\small{\textbf{\thepage}
\hfill INFORMATIKA I EE PRIMENENIYA~--- INFORMATICS AND APPLICATIONS\ \ \ 2016\
\ \ volume~10\ \ \ issue\ 4}
}%
 \def\rightfootline{\small{INFORMATIKA I EE PRIMENENIYA~---
INFORMATICS AND APPLICATIONS\ \ \ 2016\ \ \ volume~10\ \ \ issue\ 4
\hfill \textbf{\thepage}}}

 \label{end\stat}

\newpage


\vspace*{-60pt} {\small
{\baselineskip=9.1pt
\section*{Правила подготовки рукописей статей для публикации в журнале
<<Информатика и её применения>>}

\thispagestyle{empty}

 Журнал <<Информатика и её применения>> публикует
теоретические, обзорные и дискуссионные статьи, посвященные научным
исследованиям и разработкам в области информатики и ее приложений. Журнал
издается на русском языке. По специальному решению редколлегии отдельные статьи,
в виде исключения, могут печататься на английском языке.
Тематика журнала охватывает следующие направления:
\begin{itemize}
\item теоретические основы информатики; %\\[-13.5pt]
\item математические методы исследования сложных систем и процессов; %\\[-13.5pt]
\item информационные системы и сети; %\\[-13.5pt]
\item информационные технологии; %\\[-13.5pt]
\item архитектура и программное
обеспечение вычислительных комплексов и сетей.
\end{itemize}
\begin{enumerate}
\item В журнале печатаются результаты, ранее не
опубликованные и не предназначенные к одновременной публикации в других
изданиях. Публикация не должна нарушать закон об авторских правах. Направляя
свою рукопись в редакцию, авторы автоматически передают учредителям и
редколлегии неисключительные права на издание данной статьи на русском языке и
на ее распространение в России и за рубежом. При этом за авторами сохраняются
все права как собственников данной рукописи. В связи с этим авторами должно
быть представлено в редакцию письмо в следующей форме:
Соглашение о передаче права на публикацию:

\textit{<<Мы, нижеподписавшиеся, авторы рукописи <<$\qquad\qquad$>>, передаем
учредителям и редколлегии журнала <<Информатика и её применения>>
неисключительное право опубликовать данную рукопись статьи на русском языке как
в печатной, так и в электронной версиях журнала. Мы подтверждаем, что данная
публикация не нарушает авторского права других лиц или организаций. Подписи
авторов: (ф.\,и.\,о., дата, адрес)>>.}

Указанное соглашение может быть представлено 
как в бумажном виде, так и в виде отсканированной копии (с подписями авторов).


Редколлегия вправе запросить у авторов экспертное заключение о возможности
опубликования представленной статьи в открытой печати. %\\[-13.5pt]
\item Статья
подписывается всеми авторами. На отдельном листе представляются данные автора
(или всех авторов): фамилия, полные имя и отчество, телефон, факс, e-mail,
почтовый адрес. Если работа выполнена несколькими авторами, указывается фамилия
одного из них, ответственного за переписку с редакцией. %\\[-13.5pt]
\item Редакция журнала
осуществляет самостоятельную экспертизу присланных статей. Возвращение рукописи
на доработку не означает, что статья уже принята к печати. Доработанный вариант
с ответом на замечания рецензента необходимо прислать в редакцию. %\\[-13.5pt]
\item Решение
редакционной коллегии о принятии статьи к печати или ее отклонении сообщается
авторам. Редколлегия не обязуется направлять рецензию авторам отклоненной
статьи. %\\[-13.5pt]
\item Корректура статей высылается авторам для просмотра. Редакция
просит авторов присылать свои замечания в кратчайшие сроки. %\\[-13.5pt]
\item При
подготовке рукописи в MS Word рекомендуется использовать следующие настройки.
Параметры страницы: формат~--- А4; ориентация~--- книжная; поля (см): внутри~---
2,5, снаружи~--- 1,5, сверху~--- 2, снизу~--- 2, от края до нижнего
колонтитула~--- 1,3. Основной текст: стиль~--- <<Обычный>>: шрифт Times New
Roman, размер 14~пунктов, абзацный отступ~--- 0,5~см, 1,5 интервала,
выравнивание~--- по ширине. Рекомендуемый объем рукописи~--- не свыше
25~страниц указанного формата. Ознакомиться с шаблонами, содержащими примеры
оформления, можно по адресу в Интернете:
\textsf{http://www.ipiran.ru/journal/template.doc}.
\item К рукописи, предоставляемой в 2-х
экземплярах, обязательно прилагается электронная версия статьи (как правило, в
форматах MS WORD (.doc) или \LaTeX\ (.tex), а также~--- дополнительно~--- в
формате .pdf) на дискете, лазерном диске или по электронной почте. Сокращения
слов, кроме стандартных, не применяются. Все страницы рукописи должны быть
пронумерованы. %\\[-13.5pt]
\item Статья должна содержать следующую информацию на русском и
английском языках: название, Ф.И.О. авторов, места работы авторов и их
электронные адреса, подробные сведения об авторах, оформленные в соответствии с форматом, 
определяемым файлами {\sf http://www.ipiran.ru/journal/issues/2011\_05\_01/authors.asp} и 
{\sf http://www.ipiran.ru/journal/issues/2011\_01\_eng/authors.asp},
аннотация (не более 100~слов), ключевые слова. Ссылки на
литературу в тексте статьи нумеруются (в квадратных скобках) и располагаются в
порядке их первого упоминания. В~списке литературы не должно быть позиций, на которые нет ссылки в тексте статьи.
Все фамилии авторов, заглавия статей, названия
книг, конференций и~т.\,п.\ даются на языке оригинала, если этот язык
использует кириллический или латинский алфавит. %\\[-13.5pt]
\item Присланные в редакцию материалы авторам не возвращаются.
\item При отправке файлов по электронной
почте просим придерживаться следующих правил:
\begin{itemize}
\item указывать в поле subject (тема) название журнала и фамилию автора; %\\[-13.5pt]
\item использовать attach (присоединение); %\\[-13.5pt]
\item в случае больших объемов информации возможно
использование общеизвестных архиваторов (ZIP, RAR); %\\[-13.5pt]
\item в состав электронной версии статьи должны входить: файл, содержащий текст статьи, и файл(ы),
содержащий(е) иллюстрации. %\\[-13.5pt]
\end{itemize}
\item Журнал <<Информатика и её применения>> является некоммерческим изданием. 
Плата за публикацию с авторов не взимается, гонорар авторам не выплачивается.
\end{enumerate}
\thispagestyle{empty}
\textbf{Адрес редакции:} Москва 119333,
ул.~Вавилова, д.~44, корп.~2, ИПИ РАН\\
\hphantom{\textbf{Адрес редакции:} }Тел.: +7 (499) 135-86-92\ \
Факс:  +7 (495) 930-45-05\ \  E-mail:   rust@ipiran.ru }
}

\end{document}


%\tableofcontents

%\end{document}





%\def\stat{cont}
{%\hrule\par
%\vskip 7pt % 7pt
\raggedleft\Large \bf%\baselineskip=3.2ex
А\,В\,Т\,О\,Р\,С\,К\,И\,Й\ \ У\,К\,А\,З\,А\,Т\,Е\,Л\,Ь\ \ З\,А\ \ 2\,0\,0\,7 г. \vskip 17pt
    \hrule
    \par
\vskip 21pt plus 6pt minus 3pt }

\label{st\stat}

\def\tit{\ }

\def\aut{\ }
\def\auf{\ }

\def\leftkol{\ } % ENGLISH ABSTRACTS}

\def\rightkol{\ } %ENGLISH ABSTRACTS}

\titele{\tit}{\aut}{\auf}{\leftkol}{\rightkol}


\contentsline {chapter}{\ }{Выпуск \quad Стр.} 
\contentsline {section}{\textbf{Батракова Д.\,А., Королев В.\,Ю., Шоргин С.\,Я.}\ \ Новый метод вероятностно-ста\-ти\-сти\-че\-ско\-го анализа информационных потоков в\nobreakspace {}телекоммуникационных сетях}{\qquad 1 \qquad 40} 
\contentsline {section}{\textbf{Борисов А.\,В.}\ \ Байесовское оценивание в системах наблюдения с\nobreakspace {}марковскими скачкообразными процессами: игровой подход}{\qquad 2 \qquad 65}
\contentsline {section}{\textbf{Босов А.\,В., Иванов А.\,В.}\ \ Программная инфраструктура информационного Web-пор\-тала}{\qquad 2 \qquad 50}
\contentsline {section}{\textbf{Захаров В.\,Н., Калиниченко Л.\,А., Соколов И.\,А., Ступников С.\,А.}\ \ Конструирование канонических информационных моделей для интегрированных информационных систем}{\qquad 2 \qquad 15}
\contentsline {section}{\textbf{Захаров В.\,Н., Козмидиади В.\,А.}\ \ Средства обеспечения отказоустойчивости при\-ло\-жений}{\qquad 1 \qquad 14} 
\contentsline {section}{\textbf{Иванов А.\,В.}\ \ см. Босов А.\,В.\hfill\hfill\hfill\hfill\hfill\hfill\hfill\hfill\hfill\hfill\hfill\hfill\hfill\hfill\hfill\hfill\hfill\hfill\hfill\hfill\hfill\hfill\hfill\hfill\hfill\hfill\hfill\hfill\hfill\hfill\hfill\hfill\hfill\hfill\hfill}{\ }
\contentsline {section}{\textbf{Ильин В.\,Д., Соколов И.\,А.}\ \ Символьная модель системы знаний информатики в\nobreakspace {}че\-ло\-ве\-ко-автоматной среде}{\qquad 1 \qquad 66} 
\contentsline {section}{\textbf{Калиниченко Л.\,А.}\ \ см. Захаров В.\,Н.\hfill\hfill\hfill\hfill\hfill\hfill\hfill\hfill\hfill\hfill\hfill\hfill\hfill\hfill\hfill\hfill\hfill\hfill\hfill\hfill\hfill\hfill\hfill\hfill\hfill\hfill\hfill\hfill\hfill\hfill\hfill\hfill\hfill\hfill\hfill}{\ }
\contentsline {section}{\textbf{Козеренко Е.\,Б.}\ \ Лингвистическое моделирование для систем машинного перевода и обработки знаний}{\qquad 1 \qquad 54} 
\contentsline {section}{\textbf{Козмидиади В.\,А.}\ \ см. Захаров В.\,Н.\hfill\hfill\hfill\hfill\hfill\hfill\hfill\hfill\hfill\hfill\hfill\hfill\hfill\hfill\hfill\hfill\hfill\hfill\hfill\hfill\hfill\hfill\hfill\hfill\hfill\hfill\hfill\hfill\hfill\hfill\hfill\hfill\hfill\hfill\hfill }{\ } 
\contentsline {section}{\textbf{Королев В.\,Ю.}\ \ см. Батракова Д.\,А.\hfill\hfill\hfill\hfill\hfill\hfill\hfill\hfill\hfill\hfill\hfill\hfill\hfill\hfill\hfill\hfill\hfill\hfill\hfill\hfill\hfill\hfill\hfill\hfill\hfill\hfill\hfill\hfill\hfill\hfill\hfill\hfill\hfill\hfill\hfill}{\ } 
\contentsline {section}{\textbf{Кудрявцев А.\,А., Шоргин С.\,Я.}\ \ Байесовский подход к\nobreakspace {}анализу систем массового обслуживания и\nobreakspace {}показателей надежности}{\qquad 2 \qquad 76}
\contentsline {section}{\textbf{Печинкин А.\,В., Соколов И.\,А., Чаплыгин В.\,В.}\ \ Многолинейная система массового обслуживания с конечным накопителем и ненадежными приборами}{\qquad 1 \qquad 27} 
\contentsline {section}{\textbf{Печинкин А.\,В., Соколов И.\,А., Чаплыгин В.\,В.}\ \ Стационарные характеристики многолинейной\nobreakspace {}системы массового обслуживания с\nobreakspace {}одновременными отказами приборов}{\qquad 2 \qquad 39}
\contentsline {section}{\textbf{Синицын И.\,Н.}\ \ Корреляционные методы построения аналитических информационных моделей флуктуаций полюса Земли по априорным данным}{\qquad 2 \qquad \hphantom{9}2}
\contentsline {section}{\textbf{Синицын И.\,Н.}\ \ Развитие теории фильтров Пугачева для оперативной обработки информации в стохастических системах}{{\qquad 1 \qquad \hphantom{9}3}} 
\contentsline {section}{\textbf{Соколов И.\,А.}\ \ см. Захаров В.\,Н.\hfill\hfill\hfill\hfill\hfill\hfill\hfill\hfill\hfill\hfill\hfill\hfill\hfill\hfill\hfill\hfill\hfill\hfill\hfill\hfill\hfill\hfill\hfill\hfill\hfill\hfill\hfill\hfill\hfill\hfill\hfill\hfill\hfill\hfill\hfill}{\ }
\contentsline {section}{\textbf{Соколов И.\,А.}\ \ см. Ильин В.\,Д.\hfill\hfill\hfill\hfill\hfill\hfill\hfill\hfill\hfill\hfill\hfill\hfill\hfill\hfill\hfill\hfill\hfill\hfill\hfill\hfill\hfill\hfill\hfill\hfill\hfill\hfill\hfill\hfill\hfill\hfill\hfill\hfill\hfill\hfill\hfill}{\ } 
\contentsline {section}{\textbf{Соколов И.\,А.}\ \ см. Печинкин А.\,В.\hfill\hfill\hfill\hfill\hfill\hfill\hfill\hfill\hfill\hfill\hfill\hfill\hfill\hfill\hfill\hfill\hfill\hfill\hfill\hfill\hfill\hfill\hfill\hfill\hfill\hfill\hfill\hfill\hfill\hfill\hfill\hfill\hfill\hfill\hfill}{\ } 
\contentsline {section}{\textbf{Соколов И.\,А.}\ \ см. Печинкин А.\,В.\hfill\hfill\hfill\hfill\hfill\hfill\hfill\hfill\hfill\hfill\hfill\hfill\hfill\hfill\hfill\hfill\hfill\hfill\hfill\hfill\hfill\hfill\hfill\hfill\hfill\hfill\hfill\hfill\hfill\hfill\hfill\hfill\hfill\hfill\hfill}{\ }
\contentsline {section}{\textbf{Ступников С.\,А.}\ \ см. Захаров В.\,Н.\hfill\hfill\hfill\hfill\hfill\hfill\hfill\hfill\hfill\hfill\hfill\hfill\hfill\hfill\hfill\hfill\hfill\hfill\hfill\hfill\hfill\hfill\hfill\hfill\hfill\hfill\hfill\hfill\hfill\hfill\hfill\hfill\hfill\hfill\hfill}{\ }
\contentsline {section}{\textbf{Чаплыгин В.\,В.}\ \ см. Печинкин А.\,В.\hfill\hfill\hfill\hfill\hfill\hfill\hfill\hfill\hfill\hfill\hfill\hfill\hfill\hfill\hfill\hfill\hfill\hfill\hfill\hfill\hfill\hfill\hfill\hfill\hfill\hfill\hfill\hfill\hfill\hfill\hfill\hfill\hfill\hfill\hfill}{\ } 
\contentsline {section}{\textbf{Чаплыгин В.\,В.}\ \ см. Печинкин А.\,В.\hfill\hfill\hfill\hfill\hfill\hfill\hfill\hfill\hfill\hfill\hfill\hfill\hfill\hfill\hfill\hfill\hfill\hfill\hfill\hfill\hfill\hfill\hfill\hfill\hfill\hfill\hfill\hfill\hfill\hfill\hfill\hfill\hfill\hfill\hfill}{\ }
\contentsline {section}{\textbf{Шоргин С.\,Я.}\ \ см. Батракова Д.\,А.\hfill\hfill\hfill\hfill\hfill\hfill\hfill\hfill\hfill\hfill\hfill\hfill\hfill\hfill\hfill\hfill\hfill\hfill\hfill\hfill\hfill\hfill\hfill\hfill\hfill\hfill\hfill\hfill\hfill\hfill\hfill\hfill\hfill\hfill\hfill}{\ } 
\contentsline {section}{\textbf{Шоргин С.\,Я.}\ \ см. Кудрявцев А.\,А.\hfill\hfill\hfill\hfill\hfill\hfill\hfill\hfill\hfill\hfill\hfill\hfill\hfill\hfill\hfill\hfill\hfill\hfill\hfill\hfill\hfill\hfill\hfill\hfill\hfill\hfill\hfill\hfill\hfill\hfill\hfill\hfill\hfill\hfill\hfill}{\ }
%\thispagestyle{myheadings}
\def\leftfootline{\small{\textbf{\thepage}
\hfill ИНФОРМАТИКА И ЕЁ ПРИМЕНЕНИЯ\ \ \ том~1\ \ \ выпуск~2\ \ \ 2007}
}%
 \def\rightfootline{\small{ИНФОРМАТИКА И ЕЁ ПРИМЕНЕНИЯ\ \ \ том~1\ \ \ выпуск~2\ \ \ 2007
 \hfill \textbf{\thepage}}}
 \label{end\stat}

%\def\stat{cont-e}
{%\hrule\par
%\vskip 7pt % 7pt
\raggedleft\Large \bf%\baselineskip=3.2ex
2\,0\,0\,7\ \ A\,U\,T\,H\,O\,R\ \ I\,N\,D\,E\,X \vskip 17pt
    \hrule
    \par
\vskip 21pt plus 6pt minus 3pt }

\label{st\stat}

\def\tit{\ }

\def\aut{\ }
\def\auf{\ }

\def\leftkol{\ } % ENGLISH ABSTRACTS}

\def\rightkol{\ } %ENGLISH ABSTRACTS}

\titele{\tit}{\aut}{\auf}{\leftkol}{\rightkol}


\contentsline {chapter}{\ }{Issue \quad Page} 
\contentsline {subsection}{\textbf{Batrakova D.\,A., Korolev V.\,Yu., Shorgin S.\,Ya.}\ \ A New Method for the Probabilistic and Statistical Analysis of Information Flows in Telecommunication Networks}{\qquad 1 \qquad 40} 
\contentsline {subsection}{\textbf{Borisov A.\,V.}\ \ Bayesian Estimation in\nobreakspace {}Observation Systems with\nobreakspace {}Markov Jump Processes: Game-Theoretic Approach}{\qquad 2 \qquad 65} 
\contentsline {subsection}{\textbf{Bosov A.\,V., Ivanov A.\,V.}\ \ Linguistic Simulation for Machine Translation and Knowledge Management Systems}{\qquad 2 \qquad 50} 
\contentsline {subsection}{\textbf{Chaplygin V.\,V.} see Pechinkin A.\,V.\hfill\hfill\hfill\hfill\hfill\hfill\hfill\hfill\hfill\hfill\hfill\hfill\hfill\hfill\hfill\hfill\hfill\hfill\hfill\hfill\hfill\hfill\hfill\hfill\hfill\hfill\hfill\hfill\hfill\hfill\hfill\hfill\hfill\hfill\hfill}{\ }
\contentsline {subsection}{\textbf{Chaplygin V.\,V.} see Pechinkin A.\,V.\hfill\hfill\hfill\hfill\hfill\hfill\hfill\hfill\hfill\hfill\hfill\hfill\hfill\hfill\hfill\hfill\hfill\hfill\hfill\hfill\hfill\hfill\hfill\hfill\hfill\hfill\hfill\hfill\hfill\hfill\hfill\hfill\hfill\hfill\hfill}{\ }
\contentsline {subsection}{\textbf{Ilyin V.\,D., Sokolov I.\,A.}\ \ The Symbol Model of Informatics Knowledge System in Human-Automaton Environment}{\qquad 1 \qquad 66} 
\contentsline {subsection}{\textbf{Ivanov A.\,V.} see Bosov A.\,V.\hfill\hfill\hfill\hfill\hfill\hfill\hfill\hfill\hfill\hfill\hfill\hfill\hfill\hfill\hfill\hfill\hfill\hfill\hfill\hfill\hfill\hfill\hfill\hfill\hfill\hfill\hfill\hfill\hfill\hfill\hfill\hfill\hfill\hfill\hfill}{\ }
\contentsline {subsection}{\textbf{Kalinichenko L.\,A.} see Zakharov V.\,N.\hfill\hfill\hfill\hfill\hfill\hfill\hfill\hfill\hfill\hfill\hfill\hfill\hfill\hfill\hfill\hfill\hfill\hfill\hfill\hfill\hfill\hfill\hfill\hfill\hfill\hfill\hfill\hfill\hfill\hfill\hfill\hfill\hfill\hfill\hfill}{\ }
\contentsline {subsection}{\textbf{Korolev V.\,Yu.} see Batrakova D.\,A.\hfill\hfill\hfill\hfill\hfill\hfill\hfill\hfill\hfill\hfill\hfill\hfill\hfill\hfill\hfill\hfill\hfill\hfill\hfill\hfill\hfill\hfill\hfill\hfill\hfill\hfill\hfill\hfill\hfill\hfill\hfill\hfill\hfill\hfill\hfill}{\ }
\contentsline {subsection}{\textbf{Kozerenko E.\,B.}\ \ Linguistic Simulation for Machine Translation and Knowledge Management Systems}{\qquad 1 \qquad 54} 
\contentsline {subsection}{\textbf{Kozmidiady V.\,A.} see Zakharov V.\,N.\hfill\hfill\hfill\hfill\hfill\hfill\hfill\hfill\hfill\hfill\hfill\hfill\hfill\hfill\hfill\hfill\hfill\hfill\hfill\hfill\hfill\hfill\hfill\hfill\hfill\hfill\hfill\hfill\hfill\hfill\hfill\hfill\hfill\hfill\hfill}{\ }
\contentsline {subsection}{\textbf{Kudryavtsev A.\,A., Shorgin S.\,Ya.}\ \ Bayesian Approach to Queueing Systems and Reliability Characteristics}{\qquad 2 \qquad 76} 
\contentsline {subsection}{\textbf{Pechinkin A.\,V., Sokolov I.\,A., Chaplygin V.\,V.}\ \ Multichannel Queuing System with Finite Buffer and Unreliable Servers}{\qquad 1 \qquad 27} 
\contentsline {subsection}{\textbf{Pechinkin A.\,V., Sokolov I.\,A., Chaplygin V.\,V.}\ \ Stationary Characteristics of a Multichannel Queueing System with\nobreakspace {}Simultaneous Refusals of Servers}{\qquad 2 \qquad 39} 
\contentsline {subsection}{\textbf{Shorgin S.\,Ya.} see Batrakova D.\,A.\hfill\hfill\hfill\hfill\hfill\hfill\hfill\hfill\hfill\hfill\hfill\hfill\hfill\hfill\hfill\hfill\hfill\hfill\hfill\hfill\hfill\hfill\hfill\hfill\hfill\hfill\hfill\hfill\hfill\hfill\hfill\hfill\hfill\hfill\hfill}{\ }
\contentsline {subsection}{\textbf{Shorgin S.\,Ya.} see Kudryavtsev A.\,A.\hfill\hfill\hfill\hfill\hfill\hfill\hfill\hfill\hfill\hfill\hfill\hfill\hfill\hfill\hfill\hfill\hfill\hfill\hfill\hfill\hfill\hfill\hfill\hfill\hfill\hfill\hfill\hfill\hfill\hfill\hfill\hfill\hfill\hfill\hfill}{\ }
\contentsline {subsection}{\textbf{Sinitsyn I.\,N.}\ \ Correlational Methods for Analytical Informational Models of the Earth Pole Fluctuations Design Based on a priori Data}{\qquad 2 \qquad \hphantom{9}2}
\contentsline {subsection}{\textbf{Sinitsyn I.\,N.}\ \ Development of Pugachev Filtering for Stochastic Systems}{\qquad 1 \qquad \hphantom{9}3}
\contentsline {subsection}{\textbf{Sokolov I.\,A.} see Ilyin V.\,D.\hfill\hfill\hfill\hfill\hfill\hfill\hfill\hfill\hfill\hfill\hfill\hfill\hfill\hfill\hfill\hfill\hfill\hfill\hfill\hfill\hfill\hfill\hfill\hfill\hfill\hfill\hfill\hfill\hfill\hfill\hfill\hfill\hfill\hfill\hfill}{\ }
\contentsline {subsection}{\textbf{Sokolov I.\,A.} see Pechinkin A.\,V.\hfill\hfill\hfill\hfill\hfill\hfill\hfill\hfill\hfill\hfill\hfill\hfill\hfill\hfill\hfill\hfill\hfill\hfill\hfill\hfill\hfill\hfill\hfill\hfill\hfill\hfill\hfill\hfill\hfill\hfill\hfill\hfill\hfill\hfill\hfill}{\ }
\contentsline {subsection}{\textbf{Sokolov I.\,A.} see Pechinkin A.\,V.\hfill\hfill\hfill\hfill\hfill\hfill\hfill\hfill\hfill\hfill\hfill\hfill\hfill\hfill\hfill\hfill\hfill\hfill\hfill\hfill\hfill\hfill\hfill\hfill\hfill\hfill\hfill\hfill\hfill\hfill\hfill\hfill\hfill\hfill\hfill}{\ }
\contentsline {subsection}{\textbf{Sokolov I.\,A.} see Zakharov V.\,N.\hfill\hfill\hfill\hfill\hfill\hfill\hfill\hfill\hfill\hfill\hfill\hfill\hfill\hfill\hfill\hfill\hfill\hfill\hfill\hfill\hfill\hfill\hfill\hfill\hfill\hfill\hfill\hfill\hfill\hfill\hfill\hfill\hfill\hfill\hfill}{\ }
\contentsline {subsection}{\textbf{Stupnikov S.\,A.} see Zakharov V.\,N.\hfill\hfill\hfill\hfill\hfill\hfill\hfill\hfill\hfill\hfill\hfill\hfill\hfill\hfill\hfill\hfill\hfill\hfill\hfill\hfill\hfill\hfill\hfill\hfill\hfill\hfill\hfill\hfill\hfill\hfill\hfill\hfill\hfill\hfill\hfill}{\ }
\contentsline {subsection}{\textbf{Zakharov V.\,N., Kalinichenko L.\,A., Sokolov I.\,A., Stupnikov S.\,A.}\ \ Development of Canonical Information Models for Integrated Information Systems}{\qquad 2 \qquad 15} 
\contentsline {subsection}{\textbf{Zakharov V.\,N., Kozmidiady V.\,A.}\ \ Means Providing Applications Fault Tolerance}{\qquad 1 \qquad 14} 
\def\leftfootline{\small{\textbf{\thepage}
\hfill ИНФОРМАТИКА И ЕЁ ПРИМЕНЕНИЯ\ \ \ том~1\ \ \ выпуск~2\ \ \ 2007}
}%
 \def\rightfootline{\small{ИНФОРМАТИКА И ЕЁ ПРИМЕНЕНИЯ\ \ \ том~1\ \ \ выпуск~2\ \ \ 2007
 \hfill \textbf{\thepage}}}
 \label{end\stat}


%\tableofcontents


\end{document}