\def\stat{arkh}

\def\tit{ИНТЕГРАЦИЯ ГЕТЕРОГЕННОЙ ИНФОРМАЦИИ О~ЦВЕТНЫХ ПИКСЕЛЯХ 
И~ИХ~ЦВЕТОВОСПРИЯТИИ}

\def\titkol{Интеграция гетерогенной информации о цветных пикселях 
и их цветовосприятии}

\def\autkol{О.\,П.~Архипов, З.\,П.~Зыкова}
\def\aut{О.\,П.~Архипов$^1$, З.\,П.~Зыкова$^1$}

\titel{\tit}{\aut}{\autkol}{\titkol}

%{\renewcommand{\thefootnote}{\fnsymbol{footnote}}\footnotetext[1]
%{Работа выполнена при поддержке ФЦП <<Научные и научно-педагогические кадры инновационной России>> 
%на 2009--2013~гг.}}

\renewcommand{\thefootnote}{\arabic{footnote}}
\footnotetext[1]{Орловский филиал Института проблем информатики РАН, ofran@orel.ru}
 
  
\Abst{Рассмотрена задача интеграции гетерогенной информации о цветных пикселях, их 
трансформации в пользовательской компьютерной системе и о стандартном и 
индивидуальном цветовосприятии пользователя в единую инфокоммуникационную среду. 
Решение задачи необходимо для создания специальных легко адаптируемых к компонентам 
системы программно-технических инструментов, обеспечивающих произвольным 
пользователям адекватное восприятие цветной информации, выводимой на периферийные 
устройства ПЭВМ. При решении задачи используются новые методы представления и 
анализа гетерогенной (количественной и качественной) информации о связанных цветовых 
пространствах, многокритериального выбора в части предсказания различия и неразличения 
цветной информации и классификации пикселей при структурировании цветовых 
пространств.}
  
  \KW{цветовое пространство; цветовое восприятие; пространство цветовосприятия; 
аномалия цветового зрения; частичная цветовая слепота; RGB-характеризация}

 \vskip 14pt plus 9pt minus 6pt

      \thispagestyle{headings}

      \begin{multicols}{2}

      \label{st\stat}

  
\section{Введение}
  
  Современные периферийные устройства делают доступной для любого 
пользователя ПЭВМ цветную информацию, зрительное восприятие которой 
является одним из основных средств коммуникации человека в окружающем 
мире.
  
  Известно, что видимое цветовое пространство шире цветового пространства 
монитора ($C_{\mathrm{м}}$), которое, в свою очередь, в общем, шире 
цветового пространства принтера ($C_{\mathrm{п}}$). Поскольку цветовое 
пространство наблюдателя является непрерывным,\linebreak а цветовые пространства 
периферийных устройств $C_{\mathrm{м}}$ и $C_{\mathrm{п}}$ дискретны, 
будет рассматриваться его\linebreak дискретный аналог (совокупность пикселей), 
обозначаемый в дальнейшем~$C_{\mathrm{н}}$. Это позволит рас\-смат\-ри\-вать 
цветовосприятие наблюдателя как цветопередачу~--- отображение цветовых 
пространств~$C_{\mathrm{м}}$ и~$C_{\mathrm{п}}$ в~$C_{\mathrm{н}}$. 
При этих условиях для описания цветовосприятия можно использовать 
функции цветопередачи, аргументами которых являются 
пиксели~$C_{\mathrm{м}}$ или~$C_{\mathrm{п}}$, а значениями~--- 
пиксели~$C_{\mathrm{н}}$.
  
  Цветовосприятие, или цветовое пространство наблюдателя, имеет 
индивидуальные особенности и отличается у разных людей. То же справедливо 
и по отношению к цветовым пространствам периферийных устройств и к 
функциям цветопередачи между связанными цветовыми пространствами. 
  
  Общеизвестно, что представление цветного изоб\-ра\-же\-ния, состоящего из 
пикселей RGB-ку\-ба~--- исходного цветового пространства 
($C_{\mathrm{и}}$)
  $$
  C_{\mathrm{и}} = \{\left(\mathrm{R, G, B}\right)\}\,,\quad 0\leq \mathrm{R, G, B}\leq255\,,
  $$
на двух различных мониторах (или принтерах), имеет заметные различия по 
цветовому решению. Например, отдельные фрагменты представления 
изоб\-ра\-же\-ния могут быть более (или менее) красными, синими и~т.\,д. 
  
  Технологии вывода на периферийные устройства, основанные на оцифровке 
и согласовании цветовых пространств, ориентируются на восприятие цвета 
<<средним стандартным колориметрическим наблюдателем>>~[1].  При 
цветовоспроизведении сохраняется корректное соотношения между цветами 
внутри каждого отдельного представления, а потому по структуре эти 
представления воспринимаются одинаково большинством наблюдателей 
(пользователей ПЭВМ). Это обеспечивают им адекватное восприятие цветной 
информации, включенной в оборот в инфокоммуникационных средах и 
выводимой на периферийные устройства компьютерных систем. 
  
  Если цветовосприятие наблюдателя близко к стандарту, то будем называть 
его стандартным наблюдателем. Наблюдателей, имеющих различные аномалии 
цветового зрения, вследствие чего их цветовосприятие отлично от стандарта, 
будем называть аномальными наблюдателями. В~рамках данной работы 
учитываются такие аномалии цветового зрения, как частичная цветовая 
слепота. 
  
  Наблюдатели с частичной цветовой слепотой не различают некоторые цвета, 
различаемые стандартными наблюдателями, и поэтому воспринимают цветную 
информацию в искаженном виде. Классификация цветовосприятия таких 
наблюдателей основана на том, какие именно цвета они не различают~[1, 2]:
  \begin{itemize}
\item дейтеранопия~--- это цветовосприятие, при котором не различаются 
зеленый и красный цвета с нормальной функцией спектральной световой 
эффективности; 
\item протанопия~--- это цветовосприятие, при котором не различаются 
зеленый и красный цвета с ненормально низкой функцией спектральной 
световой эффективности на длинноволновом участке спектра; 
\item тританопия~--- это цветовосприятие, при котором не различаются желтые 
и синие цвета. 
  \end{itemize}
  
  Медицинская диагностика упомянутых классических аномалий проводится с 
помощью специальных отпечатков, выполненных на полигра\-фи\-ческом 
оборудовании,~--- пороговых таблиц для\linebreak исследования цветового 
зрения~\cite{4ar, 3ar}. В~этой среде информация об аномалиях представлена в 
неформальном виде, поэтому не может быть использована при организации 
вывода на периферийные устройства ПЭВМ. 
  
  Из~\cite{5ar, 6ar} доступно цифровое описание некоторых типичных 
наблюдателей с аномалиями цветового зрения. Однако они противоречат друг 
другу и не позволяют учесть индивидуальные особенности цветовосприятия 
произвольного наблюдателя в конкретных условиях цветовоспроизведения.
  
  Чтобы учесть интересы произвольного наблюдателя хотя бы в жизненно 
важных областях применения (например, навигация, ориентирование и~т.\,д.), 
необходимо представить описание его цветовосприятия в виде некоторой 
функции~$F_0$
  $$
  y=F_0(x)\,,\quad x\in C_{\mathrm{и}}\,,\enskip y\in C_{\mathrm{н}}\,,
  $$
обладающей свойствами, соответствующими цветовосприятию наблюдателя. 
  
  Например, в случае стандартного наблюдателя каждому аргументу~$F_0$ 
соответствует только один пиксель его цветового пространства 
($C_{\mathrm{н}}^\prime$), т.\,е.\ если 
  \begin{gather*}
  x^\prime\not= x^{\prime\prime}\,, \enskip x^\prime,\ x^{\prime\prime}\in 
C_{\mathrm{и}}\,,\\
y^\prime=F_0(x^\prime)\,,\ 
y^{\prime\prime}=F_0(x^{\prime\prime})\,, \enskip y^\prime,\ y^{\prime\prime}\in 
C^\prime_{\mathrm{н}}\,,
  \end{gather*}
то $y^\prime\not= y^{\prime\prime}$.
  
  В случае аномального наблюдателя существует, по крайней мере, два 
различных аргумента~$F_0$
  $$
  x^\prime\not= x^{\prime\prime}\,,\quad x^\prime, \ x^{\prime\prime}\in 
C_{\mathrm{и}}\,,
  $$
которым соответствует один пиксель цветового пространства аномального 
наблюдателя ($C^{\prime\prime}_{\mathrm{н}}$)
$$ 
y^\prime=y^{\prime\prime}\,,\ y^\prime=F_0(x^\prime)\,,\ 
y^{\prime\prime}=F_0(x^{\prime\prime})\,,\enskip y^\prime,\; y^{\prime\prime}\in 
C^{\prime\prime}_{\mathrm{н}}\,.
$$
  
  Совокупности пикселей~$C_{\mathrm{и}}$, которым в цветовом 
пространстве наблюдателя соответствует только один цвет, называют зонами 
толерантности. Для удобства последующих рассуждений предположим, что 
зоны толерантности и их компоненты упорядочены в некоторые 
последовательности 
  \begin{align*}
  \{G_{0,i}\},\enskip & i=1, 2, \ldots , I_0\,,\enskip G_{0,i}\subset C_{\mathrm{и}}\,,\\
  G_{0,i}=\{x_{0,i,j}\}\,,\enskip & i=1, 2, \ldots , I_0\,,\\
\enskip  & 1\leq j\leq J_{0,i}\,,\  x_{0,i,j}\in C_{\mathrm{и}}\,.
  \end{align*} 
  
  Для стандартного наблюдателя при идеальном цветовоспроизведении 
существуют только вы\-рож\-ден\-ные (состоящие из одного компонента) зоны 
толерантности. Для аномального наблюдателя существуют не только 
вырожденные, но и невы\-рож\-ден\-ные (состоящие хотя бы из двух компонентов) 
зоны толерантности. 
  
  В соответствии с определением одна линия уровня функции~$F_0$ 
(геометрического места точек пространства аргументов, для которых значения 
функции одинаковы) совпадает с множеством пикселей одной зоны 
толерантности~$C_{\mathrm{н}}$. Заметим, что поскольку разным линиям 
уровня соответствуют разные значения функции, они не имеют общих точек, 
т.\,е.\ не могут пересекаться.
  
  Таким образом, введенная функция~$F_0$  описывает цветовосприятие, 
поскольку обладает его свойствами, но она не пригодна для его цифрового 
описания, поскольку не имеет цифровых значений. Для цифрового описания 
цветовосприятия предлагается использовать его RGB-ха\-рак\-те\-ри\-за\-цию 
с помощью RGB-функ\-ций, аргументами которых являются 
  RGB-пик\-се\-ли исходного пространства~$C_{\mathrm{и}}$, а 
значениями~--- пиксели из RGB-цве\-то\-вых пространств, характеризующих 
пространства периферийных устройств. 
  
  Для характеризации цветового пространства монитора будет использоваться 
исходное пространство~$C_{\mathrm{и}}$. Функция 
  RGB-ха\-рак\-те\-ри\-за\-ции цветовосприятия представления пикселей на 
мониторе имеет вид:
  $$
  y=F_1(x)\,,\quad x, y\in C_{\mathrm{и}}\,,
  $$
и обладает линиями уровня, совпадающими с зонами толерантности 
наблюдателя при восприятии им представлений на мониторе,
\begin{align*}
  \{G_{1,i}\},\enskip & i=1, 2, \ldots , I_1\,,\enskip G_{1,i}\subset C_{\mathrm{и}}\,,\\
  G_{1,i}=\{x_{1,i,j}\}\,,\enskip & i=1, 2, \ldots , I_1\,,\\
\enskip  & 1\leq j\leq J_{1,i}\,,\  x_{1,i,j}\in C_{\mathrm{и}}\,.
  \end{align*} 
  
  Следовательно, если представления двух разных пикселей 
  $$
  x^\prime\not=x^{\prime\prime}\,,\quad x^\prime, x^{\prime\prime}\in 
C_{\mathrm{и}}\,,
  $$
на мониторе воспринимаются наблюдателем как одинаковые, то значения 
функции~$F_1$ от них совпадают:
$$
y^\prime = y^{\prime\prime}\,,\ y^\prime=F_1(x^\prime)\,,\ 
y^{\prime\prime}=F_1(x^{\prime\prime})\,,\ y^\prime,y^{\prime\prime}\in 
C_{\mathrm{и}}\,.
$$
  
  Если представления двух разных пикселей 
  $$
  x^\prime\not=x^{\prime\prime}\,,\quad x^\prime, x^{\prime\prime}\in 
C_{\mathrm{и}}\,,
  $$
на мониторе воспринимаются наблюдателем как различные, то значения 
функции~$F_1$ от них различны:
$$
y^\prime\not= y^{\prime\prime}\,,\ y^\prime=F_1(x^\prime)\,,\ 
y^{\prime\prime}=F_1(x^{\prime\prime})\,,\enskip y^\prime, y^{\prime\prime}\in 
C_{\mathrm{и}}\,.
$$
  
  Для характеризации цветового пространства принтера будет использоваться 
цветовое пространство сканера ($C_{\mathrm{с}}$)~[7--9]. 
Функция RGB-ха\-рак\-те\-ри\-за\-ции цветовосприятия отпечатков пикселей:
  $$
  y=F_2(x)\,,\quad x\in C_{\mathrm{и}}\,, y\in C_{\mathrm{с}}\,,
  $$
должна обладать линиями уровня, совпадающими с зонами толерантности 
наблюдателя при восприятии им отпечатков,
\begin{align*}
  \{G_{2,i}\},\enskip & i=1, 2, \ldots , I_2\,,\enskip G_{2,i}\subset C_{\mathrm{и}}\,,\\
  G_{2,i}=\{x_{2,i,j}\}\,,\enskip & i=1, 2, \ldots , I_2\,,\\
\enskip  & 1\leq j\leq J_{2,i}\,,\  x_{2,i,j}\in C_{\mathrm{и}}\,.
\end{align*}
  
  Следовательно, если отпечатки двух разных пикселей 
  $$
  x^\prime \not= x^{\prime\prime}\,,\enskip x^\prime, x^{\prime\prime}\in 
C_{\mathrm{и}}\,,
  $$
воспринимаются наблюдателем как одинаковые, то значения функции~$F_2$ 
от них совпадают:
$$
y^\prime=y^{\prime\prime}\,,\ y^{\prime}=F_2(x^\prime)\,, \
y^{\prime\prime}=F_2(x^{\prime\prime})\,, \enskip y^\prime, y^{\prime\prime}\in 
C_{\mathrm{с}}\,.
$$
  
  Если отпечатки двух разных пикселей 
  $$
  x^\prime\not= x^{\prime\prime}\,,\enskip x^\prime, x^{\prime\prime}\in 
C_{\mathrm{и}}\,,
  $$
воспринимаются наблюдателем как различные, то значения функции~$F_2$ от 
них различны:
$$
y^\prime\not= y^{\prime\prime}\,,\ y^\prime=F_2(x^\prime)\,,\ 
y^{\prime\prime}=F_2(x^{\prime\prime})\,,\enskip y^\prime, y^{\prime\prime}\in 
C_{\mathrm{и}}\,.
$$
  
  Множество значений таких функций RGB-ха\-рак\-те\-ри\-за\-ции 
цветовосприятия произвольного\linebreak наблюдателя является подмножеством 
со\-от\-вет\-ст\-ву\-юще\-го RGB-про\-стран\-ст\-ва и может быть использовано в 
качестве цифрового описания~$C_{\mathrm{н}}$, а сама функция~--- для 
цифрового описания цветопередачи из~$C_{\mathrm{и}}$ в~$C_{\mathrm{н}}$. 
  
  Факт принадлежности произвольных пикселей одной или разным зонам 
толерантности тес\-ти\-ру\-емо\-го наблюдателя может быть установлен при\linebreak 
тес\-ти\-ро\-ва\-нии. Общепринятым подходом к тестированию цветовосприятия 
произвольного наблюдателя является подход, при котором наблюдателю 
последовательно предъявляются для визуализации и описания представления 
тестовых изображений на периферийном устройстве (мониторе, принтере). 
  
  Если наблюдатель адекватно воспринимает структуру тестов, то делается 
вывод о различении им цветных пикселей, использованных при создании 
изображения. В~этом случае можно говорить о том, что им соответствуют 
различные пиксели в цветовом пространстве тестируемого наблюдателя. 
  
  Если структура теста искажается при восприятии его наблюдателем, то, 
следовательно, он не различает соответствующие цветные пиксели и их можно 
считать представителями одной зоны толерантности.
  
  Из-за технологических особенностей вывода цветных изображений, 
проявляющихся на практике в нестабильности цветовоспроизведения, 
представления пикселей с разными координатами на периферийных 
устройствах могут иметь идентичные цвета. В~этом случае они являются 
представителями <<паразитных>> зон толерантности. Далее предполагается, 
что применяются такие периферийные устройства, для которых этот эффект 
проявляется только при выводе пикселей с достаточно близкими 
  RGB-ко\-ор\-ди\-на\-тами.
  
  В то же время из-за нестабильного цветовоспроизведения на периферийном 
устройстве (обычно\linebreak принтере) пиксель с одними и теми же 
  RGB-ко\-ор\-ди\-на\-та\-ми может иметь объективно различные по цвету 
представления. Далее предполагается, что рассматриваются периферийные 
устройства со стабильным цветовоспроизведением, при котором имеющиеся 
различия визуально незаметны для произвольных наблюдателей.
  
  Существование отличных от <<паразитных>> невырожденных зон 
толерантности обнаруживает наличие аномалий цветового зрения, а их 
реквизиты характеризуют основные свойства цветового пространства 
аномального наблюдателя и его отличие от цветового пространства 
стандартного наблюдателя. 
  
  Границы зон толерантности, возникающих из-за аномалий цветного зрения, 
размываются из-за паразитных зон толерантности и не могут быть определены 
точно. Чем выше стабильность вывода на периферийные устройства, тем 
меньше на практике может быть погрешность определения указанных границ.
  
  Предположим, что протестировано восприятие представлений на 
периферийном устройстве каж\-дой пары пикселей~$C_{\mathrm{и}}$. Для 
цифрового описания результатов тестирования достаточно использовать 
функцию от двух переменных (пикселей~$C_{\mathrm{и}}$), принимающую 
только два значения. Одно из значений используется, если при тестировании 
установлено, что цвета соответствующей пары пикселей наблюдатель 
различает, а второе~--- не различает. 
  
  Такую функцию после тестирования можно задать и хранить на диске в виде 
подходящей таблицы. При автоматической обработке из табличных данных 
можно, как это сделано, например, в~\cite{7ar}, извлечь описание зон 
толерантности. Как будет показано далее, на этой основе можно построить 
функцию RGB-ха\-рак\-те\-ри\-за\-ции цветовосприятия протестированного 
пользователя и использовать ее для создания наглядных графических 
иллюстраций и управления цветопередачей в компьютерной сис\-те\-ме в 
интересах протестированного наблюда-\linebreak теля.
  
  Однако понятно, что на практике такая процедура тестирования не может 
быть реализована. В~разумные временн$\acute{\mbox{ы}}$е сроки 
невозможно про\-тес\-ти\-ро\-вать восприятие представлений на мониторе каждой 
пары из более чем шестнадцати с половиной миллионов 
пикселей~$C_{\mathrm{и}}$, а на принтере препятствием являются не только 
чрезмерные временные затраты, но и соответствующий объем расходных 
материалов (бумаги, красителей).
  
  Выход состоит в проведении тестирования восприятия представления на 
мониторе ограниченного числа пикселей и приближенного вычисления 
функции RGB-ха\-рак\-те\-ри\-за\-ции на этой основе. Если затем построить 
функцию RGB-ха\-рак\-те\-ри\-за\-ции цветопередачи~$C_{\mathrm{и}}$ 
в~$C_{\mathrm{п}}$, то функция RGB-ха\-рак\-те\-ри\-за\-ции восприятия 
представления на мониторе может быть использована для аппроксимации 
функции RGB-ха\-рак\-те\-ри\-за\-ции восприятия отпечатков.

\section{Постановка задачи}

  Пусть имеются множества пикселей
  $$
  \{t\} \subset \{t^\prime\}\subset \{t^{\prime\prime}\}\subset C_{\mathrm{и}}\,,
  $$
а после тестирования восприятия наблюдателем представлений на мониторе 
пиксели из $\{t\}$ классифицированы по группам представителей различных 
зон толерантности
\begin{equation*}
\{g^\prime_{1,i}\}\,, \quad i=1, 2, \ldots , I^\prime_1\,,\ g^\prime_{1,i}\subset 
\{t\}\,,
\end{equation*}
\begin{gather*}
g^\prime_{1,i}=\{x_{1,i,j}\}\,,\quad i=1, 2, \ldots , I^\prime_1\,,\\
\hspace*{20mm} 1\leq j\leq 
j^\prime_{1,i}\,, \ x_{1,i,j}\in \{t\}\,,\\
\{t\} = \bigcup\limits_{i=1}^{I^\prime_1}  
g^\prime_{1,i}=\bigcup\limits_{i=1}^{I^\prime_1}\bigcup\limits_{j=1}^{j^\prime_{1,i}} 
x_{1,i,j}\,.
\end{gather*}
  
  Пусть далее каждая $i$-я группа дополнена пикселями из 
множества~$\{t^\prime\}$, не принадлежащими~$\{t\}$, если тестируемый 
наблюдатель не различает их по цвету с ее первым представителем~$x_{1,i,1}$. 
Полученные в результате группы представителей зон толерантности обозначим 
следующим образом:
  \begin{equation}
  \left.
  \begin{array}{c}
\{G^\prime_{1,i}\}\,, \quad i=1, 2, \ldots , I^\prime_1\,,\ G^\prime_{1,i}\subset 
\{t^\prime\}\,,\\[6pt]
G^\prime_{1,i}=\{x_{1,i,j}\}\,,\quad i=1, 2, \ldots , I^\prime_1\,,\\[6pt]
1\leq j\leq 
J^\prime_{1,i}\,, \ x_{1,i,j}\in \{t^\prime\}\,,\\[6pt]
\{t^\prime\} \supset \bigcup\limits_{i=1}^{I^\prime_1}  
G^\prime_{1,i}=\bigcup\limits_{i=1}^{I^\prime_1}\bigcup\limits_{j=1}^{J^\prime_{1,i}} 
x_{1,i,j}\,.
\end{array}
\right \}
\label{e1ar}
\end{equation}
  
  Требуется определить:
  \begin{itemize}
  \item 
   значения функции~$F_1$ от аргументов~$\{x_{1,i,j}\}$ 
из~(\ref{e1ar}) и построить на этой основе аппроксимацию~$\Psi_1$ функции 
$F_1$ на~$\{t^{\prime\prime}\}$
  $$
  \Psi_1(t^{\prime\prime}=y^{\prime\prime}\approx F_1(t^{\prime\prime}),\enskip 
\{y^{\prime\prime}\}\subset C_{\mathrm{и}}\,,
  $$
линии уровня которой на множестве~$\{t^\prime\}$, определяемые 
константами~$F_1(t)$, 
$$
\{G^{\prime\prime}_{1,i}\}\,,\enskip i = 1, 2, \ldots ,\  I^{\prime\prime}_1\,,\ 
G^{\prime\prime}_{1,i}\subset C_{\mathrm{и}}\,,
$$
удовлетворяют следующим соотношениям:
\begin{equation}
G^\prime_{1,i}\subset G^{\prime\prime}_{1,i}\,,\enskip  i = 1, 2, \ldots ,\ I^\prime_1 = 
I^{\prime\prime}_1\,;
\label{e2ar}
\end{equation}
  \item значения аппроксимации~$\Psi_3$ функции~$F_3$, 
являющейся функцией RGB-ха\-рак\-те\-ри\-за\-ции 
цветопередачи~$C_{\mathrm{и}}$ в~$C_{\mathrm{п}}$, от 
аргументов~$\{t^{\prime\prime}\}$ как значения цветовой 
  характеристики~[7--9] отпечатков пикселей
  $$
\Psi_3(t^{\prime\prime}) = s^{\prime\prime} \approx F_3(t^{\prime\prime})\,,\ 
\{s^{\prime\prime}\} \subset C_{\mathrm{с}}\,;
$$
  \item аппроксимацию~$\Psi_2$ функции~$F_2$ как 
суперпозицию функций~$\Psi_1$ и~$\Psi_3$ и вычислить ее значения от 
аргументов~$\{t^{\prime\prime}\}$ как значения функции~$\Psi_1$ от 
аргументов~$\{s^{\prime\prime}\}$:
  $$
  \Psi_2\left(t^{\prime\prime}\right) =\Psi_1\left(s^{\prime\prime}\right) 
=\Psi_1\left(\Psi_3\left(t^{\prime\prime}\right)\right)\,.
  $$
  \end{itemize}
  
\section{Аппроксимация функции RGB-ха\-рак\-те\-ри\-за\-ции 
восприятия представления пикселей на мониторе}
  
  Пусть при тестировании обнаружены группы представителей некоторых зон 
толерантности~(\ref{e1ar}). Тогда, выбирая один из пикселей каждой группы в 
качестве реперного пикселя, можно определять им значение функции 
  RGB-ха\-рак\-те\-ри\-за\-ции~$F_1$ для всех пикселей данной группы. 
  
  При выборе реперных пикселей в соответствии с~\cite{1ar} следует 
руководствоваться правилом: реперным является пиксель, который наиболее 
близок к серой шкале RGB-ку\-ба (прямой, проходящей через вершины куба 
(0,\,0,\,0) и (255,\,255,\,255)). 
  
  Обозначим~$\mu$ функцию, реализующую это правило:
  \begin{multline}
F_1(x_{1,i,j}) = \mu (G^\prime_{1,i}) = x_{1,i,j^{\prime\prime}}\,,\\
i = 1, 2, \ldots,\ I^\prime_1,\enskip 1 \leq j,\ j^{\prime\prime}\leq J^\prime_{1,i}\,,       
\label{e3ar}
\end{multline}
где значение индекса~$j^{\prime\prime}$ удовлетворяет соотношению: 
$$
\rho(x_{1,i,j^{\prime\prime}}, x^\prime_{1,i,j^{\prime\prime}})=
\underset{1\leq j\leq J^\prime_{1,i}}{\min} \rho \left( x_{1,i,j}, x^\prime_{1,i,j}\right)\,,
$$
а $x^\prime_{1,i,j}$~--- это ближайшая для~$x_{1,i,j}$ точка серой шкалы.
  
  Заметим, что если пиксель имеет RGB-ко\-ор\-ди\-на\-ты $(r,\,g,\,b)$, то 
ближайшая к нему точка серой шкалы лежит на пересечении серой шкалы и 
перпендикуляра, опущенного из точки~$(r,\,g,\,b)$ на серую шкалу, и имеет 
координаты
  $$
  \left( \fr{ r+g+b }{3}\,,\ \fr{r+g+b}{3}\,,\ \fr{r+g+b}{3}\right )\,.
  $$
  
  Если вершины RGB-ку\-ба входят в совокупность пикселей 
  $$
  \{x_{1,i,j}\}\,,\enskip i = 1, 2, \ldots ,\ I^\prime_1, 1 \leq j \leq J^\prime_{1,i}\,, 
  $$
то их можно использовать в качестве узлов интерполяции для построения 
аппроксимации~$\Psi_1$ функции~$F_1$ на~$\{t^{\prime\prime}\}$, поскольку 
в соответствии с~(\ref{e3ar}) в них будут определены значения 
функции~$F_1$.
  
  Действительно, в этом случае для определения значения~$\Psi_1$ в 
произвольной точке $x_0 = (r_0,\,g_0,\,b_0)$ из~$\{t^{\prime\prime}\}$, лежащей 
на некотором ребре куба, достаточно:
  \begin{itemize}
  \item на этом же ребре найти такие точки $x_1 =$\linebreak $= (r_1,\,g_1,\,b_1)$ и $x_2 = 
(r_2,\,g_2,\,b_2)$, в которых значения~$\Psi_1$ уже известны, а точка~$x_0$ 
является внутренней точкой отрезка $[x_1,\,x_2]$;
  \item значения $\Psi_1(x_0)$ определить, например, при линейной 
интерполяции по формуле
  \begin{multline*}
  \Psi_1(x_0) ={}\\
  {}=\fr{\Psi_1(x_1)/\rho(x_0, x_1)+{\Psi_1(x_2)}/\rho(x_0, x_2)}{
{1}/{\rho(x_0, x_1)}+{1}/\rho(x_0, x_2)}\,.
\end{multline*}
  \end{itemize}
  
  Для определения значения~$\Psi_1$ в произвольной точке $x_0 = 
(r_0,\,g_0,\,b_0)$ из~$\{t^{\prime\prime}\}$, лежащей внутри некоторой стороны 
куба, достаточно:
  \begin{itemize}
  \item на этой же стороне куба найти такие точки $x_1 =$\linebreak $= (r_1,\,g_1,\,b_1)$, 
$x_2=(r_2,\,g_2,\,b_2)$, $x_3=(r_3,\,g_3,\,b_3)$, $x_4=(r_4,\,g_4,\,b_4)$, которые 
лежат на пересечении соответствующих ребер куба и опущенных на них 
перпендикуляров из точки~$x_0$;
  \item вычислить значения~$\Psi_1(x_1)$, $\Psi_1(x_2)$, $\Psi_1(x_3)$, 
$\Psi_1(x_4)$ по ранее описанному алгоритму;
  \item определить значения~$\Psi_1(x_0)$, например, при интерполяции по 
формуле:
  \begin{multline*}
  \Psi_1(x_0)=
\left(\fr{\Psi_1(x_1)}{\rho(x_0, x_1)}+\fr{\Psi_1(x_2)}{ \rho(x_0, 
x_2)}+{}\right.\\
\left.{}+\fr{\Psi_1(x_3)}{ \rho(x_0, x_3)}+
\fr{\Psi_1(x_4)}{ \rho(x_0, 
x_4)}\right)\Bigg /  \left(\fr{1}{\rho(x_0,x_1)}+{}\right.\\
\left.{}+\fr{1}{\rho(x_0,x_2)}+\fr{1}{\rho(x_0,x_3)}+\fr{1}
{\rho(x_0,x_4)}\right)\,.
  \end{multline*}
  \end{itemize}
  
  Наконец, для определения значения~$\Psi_1$ в произвольной внутренней 
точке куба $x_0=(r_0,\,g_0,\,b_0)$ из~$\{t^{\prime\prime}\}$ достаточно:
  \begin{itemize}
  \item найти точки, которые лежат на пересечении сторон куба и опущенных 
на них перпендикуляров из точки~$x_0$;
  \item вычислить в них значения~$\Psi_1$ по ранее описанному алгоритму;
  \item определить значения~$\Psi_1(x_0)$, например, при интерполяции по 
формуле, аналогичной тем, которые были использованы ранее.
  \end{itemize}
  
  Очевидно, что вычисление приближенных значений функции 
  RGB-ха\-рак\-те\-ри\-за\-ции является трудоемкой задачей. Желательно 
один раз вы\-чис\-лить их и хранить в электронном виде (например, в табличной 
форме) для последующих многократных использований. Однако в 
  RGB-ку\-бе более шестнадцати с половиной миллионов цветов, что 
потребует использования таблиц соответствующего размера. 
  
  Практика показывает, что удобен компро\-мис\-сный вариант, когда часть 
значений функции цветопередачи на некотором 
подмножестве~$\{t^{\prime\prime}\}$ RGB-ку\-ба хранится, а часть~--- по 
мере необходимости вычисляется. 
  
  Для линий уровня построенной функции условие~(\ref{e2ar}), очевидно, 
выполняется, причем на практике нельзя добиться точного соответствия между 
линиями уровня у аппроксимации и линиями уровня у искомой функции. Это 
обусловлено возможностью появления <<паразитных>> зон толерантности 
  из-за погрешностей приближенных вычислений, возникающих при 
интерполяции. 

\section{Аппроксимация функции RGB-характеризации 
цветопередачи~\boldmath{$C_{\mathrm{и}}$}
 в~\boldmath{$C_{\mathrm{п}}$} 
}

  Процедура получения значений цветовой характеристики отпечатков 
пикселей
  $$
s^{\prime\prime} = \Psi_3(t^{\prime\prime})\,,\enskip \{s^{\prime\prime}\} \subset 
C_{\mathrm{c}}
$$
разработана авторами в более ранних работах (см., например,~[7--9]), поэтому 
приведем лишь краткое ее описание:
\begin{itemize}
  \item на основе пикселей~$\{t^{\prime\prime}\}$ создается специальное 
изображение~--- машиночитаемая зона;
  \item машиночитаемая зона печатается и сканируется по определенным 
правилам;
  \item из общего скана извлекается полезная часть, содержащая сканы меток 
позиционирования и сканы растровых точек исходных пикселей;
  \item при автоматической обработке полезной части скана определяется 
местоположение меток позиционирования, а затем на этой основе и 
мес\-то\-по\-ло\-же\-ние скана каждой растровой точки, и координаты составляющих 
его пикселей;
  \item при автоматической обработке координаты пикселей, составляющих 
скан каждой растровой точки, усредняются, а полученная величина 
принимается за значение цветовой характеристики отпечатка 
соответствующего пикселя.
  \end{itemize}
  
\section{Аппроксимация функции RGB-характеризации 
восприятия отпечатков пикселей}
  
  Пусть от аргументов $\{t^{\prime\prime}\}\subset C_{\mathrm{и}}$ известны 
значения функций
  \begin{align*}
  y^{\prime\prime} &= \Psi_1(t^{\prime\prime})\,,\enskip \{y^{\prime\prime}\} 
\subset C_{\mathrm{и}}\,,\\
s^{\prime\prime} & = \Psi_3(t^{\prime\prime})\,,\enskip \{s^{\prime\prime}\} \subset 
C_{\mathrm{c}}\,.
\end{align*}
  
  Обозначим $\{z^{\prime\prime}\} \subset C_{\mathrm{и}}$ последовательность 
RGB-пик\-се\-лей, имеющих те же RGB-ко\-ор\-ди\-на\-ты, что и 
  RGB-пик\-се\-ли $\{s^{\prime\prime}\} \subset C_{\mathrm{c}}$. Тогда имеем
  \begin{align*}
  v^{\prime\prime} & = \Psi_1(z^{\prime\prime})\,,\enskip \{v^{\prime\prime}\} 
\subset C_{\mathrm{и}},\ \{z^{\prime\prime}\} \subset C_{\mathrm{и}}\,,\\
s^{\prime\prime} &= \Psi_3(t^{\prime\prime})\,,\enskip \{s^{\prime\prime}\} \subset 
C_{\mathrm{c}}\,.
\end{align*}
  
  Обозначим $\{w^{\prime\prime}\} \subset C_{\mathrm{с}}$ 
последовательность RGB-пик\-се\-лей, имеющих те же
   RGB-ко\-ор\-ди\-на\-ты, что и RGB-пик\-се\-ли $\{v^{\prime\prime}\} 
\subset C_{\mathrm{и}}$. Определим значения функции~$\Psi_2$ от 
аргументов~$\{t^{\prime\prime}\}$:
  $$
w^{\prime\prime} = \Psi_2(t^{\prime\prime})\,,\enskip \{t^{\prime\prime}\} \subset 
C_{\mathrm{и}}\,,\ \{w^{\prime\prime}\} \subset C_{\mathrm{с}}\,.
$$
  
  Если найденные значения хранить на диске, то приближенные значения 
функции RGB-ха\-рак\-те\-ри\-за\-ции восприятия отпечатков пикселей от 
других значений аргументов, не принадлежащих\linebreak 
множеству~$\{t^{\prime\prime}\}$, можно вычислять по мере не\-об\-хо\-ди\-мости 
путем интерполяции аналогично тому, как это описано ранее. 

\section{Интегрированные данные о~цветовосприятии}
  
  В качестве~$\{t^{\prime\prime}\}$ рассмотрим, например, пиксели из 
множества~$M$
  \begin{gather*}
  M =\{(r_i,\,g_i,\,b_i)\}=\{(j\cdot17, k\cdot17, n\cdot17)\}\,,\\ j,k,n = 0, \ldots , 15\,,\\
  i=j+k\cdot16+n\cdot16\cdot16=0,1, \ldots , 4095\,. 
  \end{gather*}
  
  Предположим, что значения функций~$\Psi_1$ и~$\Psi_2$  на~$M$ 
определены и хранятся в электронном виде. Очевидно, что значения функций 
$\Psi_1$ и~$\Psi_2$ во всех других пикселях RGB-ку\-ба, которые, как можно 
заметить, принадлежат одному из меньших RGB-ку\-би\-ков размером 
$17\times 17\times 17$, могут быть вычислены при интерполяции по вершинам 
кубика, в которых значения функции известны, поскольку эти пиксели 
принадлежат множеству~$M$.
  
  Если из пикселей множества~$M$ и пикселей, являющихся значениями 
функций~$\Psi_1$ и~$\Psi_2$ построить изображения (обозначим их Img, 
$\mathrm{Img}^\prime$ и Img$^{\prime\prime}$ соответственно), то будет получена 
наглядная иллюстрация, показывающая различие и совпадение 
цветовосприятия стандартного и протестированного наблюдателя.

  \begin{figure*}[b] %fig1
  \vspace*{1pt}
\begin{center}
\mbox{%
\epsfxsize=76.729mm
\epsfbox{arh-1.eps}
}
\end{center}
\vspace*{-6pt}
  \Caption{Вид изображений: (\textit{а})~$M$; (\textit{б})~$M^\prime_1$; 
(\textit{в})~$M^\prime_2$; (\textit{г})~$M^\prime_3$; (\textit{д})~$M^\prime_4$; 
(\textit{е})~$M^\prime_5$; (\textit{ж})~$M^\prime_6$
   \label{f1ar}}
   \end{figure*}
  
  Сопоставление трех изображений (Img, Img$^\prime$ 
и~Img$^{\prime\prime}$) позволяет получить полное представление о 
цветовосприятии произвольным наблюдателем вывода на периферийные 
устройства его\linebreak компьютерной системы. К~тому же электронное описание 
цветовосприятия произвольного наблюдателя в виде изображений в 
  BMP-файлах более удобно (по сравнению с табличными данными в 
текстовых файлах) для интеграции гетерогенной информации о цветных 
пикселях и цветовосприятии и упрощает проведение визуального и 
программного анализа индивидуального цветовосприятия. 
  
  Пусть выводимая на периферийное устройство цветная информация 
представляет собой произвольное RGB-изоб\-ра\-же\-ние. Тогда можно 
модифицировать его, заменяя пиксели RGB-изоб\-ра\-же\-ния значениями в 
них функции~$\Psi_1$ или~$\Psi_2$, которые либо хранятся на диске, либо 
вычисляются, если исходный пиксель не принадлежит 
множеству~$\{t^{\prime\prime}\}$. Сравнивая между собой представления 
исходного изображения и его модификации, можно более детально оценить 
степень искажения информации при восприятии произвольным наблюдателем. 
Если из-за какой-либо аномалии цветового зрения искажения недопустимо 
велики, необходимо изменить подход к ее графическому представлению.
  
  Если модифицировать RGB-изоб\-ра\-же\-ние в соответствии с~\cite{10ar}, 
заменяя пиксели RGB-изоб\-ра\-же\-ния наиболее близкими к ним пикселями 
из области значений аппроксимирующей функции, то в результате будет 
получено RGB-изоб\-ра\-же\-ние, восприятие которого наблюдателем 
максимально близко к восприятию исходного изображения стандартным 
наблюдателем.
  
\section{Пример построения функции~\boldmath{$\Psi_1$}}
  
  Известно ПО~\cite{5ar, 6ar}, с помощью которого можно получить цифровое 
описание цветовых пространств таких аномальных наблюдателей, как\linebreak 
дейтеранопы, протанопы и тританопы в форме изоб\-ра\-жений~Img 
и~Img$^\prime$. 
  
  Использование файла с Img в качестве входных данных одной из 
программ~\cite{5ar, 6ar} позволяет получить на выходе 
изображение~Img$^\prime$, со\-от\-вет\-ст\-ву\-ющее одной из аномалий цветового 
зрения. Пиксели~Img$^\prime$ составляют цифровое описание цветового 
пространства аномального наблюдателя, а соотношение между ними 
характеризуют индивидуальные особенности его цветовосприятия.
  
  Обозначим:
  \begin{itemize}
  \item[\ ] $\varphi_1$~--- функцию RGB-ха\-рак\-те\-ри\-за\-ции 
дейтеранопа~\cite{5ar} и $M^\prime_1$: $M^\prime_1=\varphi_1(M)$; 
  \item[\ ] $\varphi_2$~--- функцию RGB-ха\-рак\-те\-ри\-за\-ции 
протанопа~\cite{5ar} и $M^\prime_2$: $M^\prime_2=\varphi_2(M)$; 
  \item[\ ] $\varphi_3$~--- функцию RGB-ха\-рак\-те\-ри\-за\-ции 
тританопа~\cite{5ar} и $M^\prime_3$: $M^\prime_3=\varphi_3(M)$; 
  \item[\ ] $\varphi_4$~--- функцию RGB- ха\-рак\-те\-ри\-за\-ции 
дейтеранопа~\cite{6ar} и $M^\prime_4$: $M^\prime_4=\varphi_4(M)$; 
  \item[\ ] $\varphi_5$~--- функцию RGB- ха\-рак\-те\-ри\-за\-ции 
протанопа~\cite{6ar} и~$M^\prime_5$: $M^\prime_5=\varphi_5(M)$; 
  \item[\ ] $\varphi_6$~--- функцию RGB- ха\-рак\-те\-ри\-за\-ции 
тританопа~\cite{6ar} и~$M^\prime_6$: $M^\prime_6=\varphi_6(M)$.
  \end{itemize}
  
  Изображение множеств~$M$ и~$M^\prime_i$, $i = 1, 2, \ldots , 6$, приведено 
на рис.~\ref{f1ar}. Заметим, что эти иллюстрации наглядно демонстрируют 
отсутствие единого подхода к оцифровке цветовых пространств аномальных 
наблюдателей в общем случае. Даже для указанных классических аномалий 
цветного зрения две указанные программы дают различные результаты при 
одних и тех же входных данных.
  
  Для реализации алгоритма построения функции~$\Psi_1$ было создано 
специальное ПО, функционирующее на базе одного компьютера типа PC IBM с 
оболочкой Windows~XP. Использовалась периферия: цветной лазерный 
принтер HP Color LaserJet 4700n и цветной сканер FUJITSU fi-60F.



  Практическое тестирование было заменено виртуальным. Рассмотрено шесть 
примеров в соответствии с~\cite{5ar, 6ar}. При этом неразличимость 
произвольных пикселей~$x^\prime$ и~$x^{\prime\prime}$ для $i$-го 
виртуального\linebreak
\begin{center} %fig2
\vspace*{6pt}
\mbox{%
\epsfxsize=41.275mm
\epsfbox{arh-2.eps}
}
\end{center}
\vspace*{4pt}
\begin{center}
{{\figurename~2}\ \ \small{Вид изображений: (\textit{а})~Image; (\textit{б})~$\mathrm{Image}\;\cap\{t\}$}}
\end{center}
%\vspace*{9pt}

\bigskip
\addtocounter{figure}{1}


\noindent
 наблюдателя определялась в соответствии с~\cite{11ar} на
основе 
исследования расстояния между пикселями~$\varphi_i(x^\prime)$ 
и~$\varphi_i(x^{\prime\prime})$, $i=1, 2, \ldots , 6$. Чтобы отличать значения 
функции~$\Psi_1$ для разных виртуальных наблюдателей, будем использовать 
второй индекс: значение~$\Psi_{1,i}$ функции~$\Psi_1$ соответствует $i$-му 
виртуальному наблюдателю.
  
  При расчетах использовались множества:
  
  \noindent
  \begin{gather*}
\{t\} = \{(r_i,\,g_i,\,b_i)\} = \{(j\cdot85, k\cdot85, n\cdot85)\}\,,\\
j,k,n = 0, 1, 2,  3\,,\\
i=j+k\cdot4+n\cdot4\cdot4=0,1, \ldots , 63\,;\\
\{t^\prime\} = \{t^{\prime\prime}\} = \{(r_i,\,g_i,\,b_i)\} = \{(j\cdot5, k\cdot5, 
n\cdot5)\}\,,\\
j,k,n = 0, \ldots , 51,\\
i=j+k\cdot52+n\cdot52\cdot52=0,1, \ldots , 140\,607\,. 
\end{gather*}
  
  Иллюстрации к полученным результатам построены на основе изображения 
$Image$, составленного из пикселей~$\{t^{\prime\prime}\}$, принадлежащих 
сторонам RGB-ку\-ба в соответствии с рис.~2. Результаты расчетов 
отображены на рис.~\ref{f3ar}--\ref{f9ar}.
  



\begin{figure*} %fig3
\vspace*{1pt}
\begin{center}
\mbox{%
\epsfxsize=123.825mm
\epsfbox{arh-3.eps}
}
\end{center}
\vspace*{-6pt}
\Caption{Вид изображений: (\textit{а})~линии уровня $\varphi_1$ и~$\Psi_{1,1}$ 
на Image; (\textit{б})~значения~$\Psi_{1,1}$ на линиях уровня; 
(\textit{в})~значения~$\varphi_1$ на линиях уровня; 
(\textit{г})~$\Psi_{1,1}(\mathrm{Image})$; (\textit{д})~$\varphi_1(\mathrm{Image})$
\label{f3ar}}
\end{figure*}

\begin{figure*} %fig4
\vspace*{1pt}
\begin{center}
\mbox{%
\epsfxsize=123.825mm
\epsfbox{arh-4.eps}
}
\end{center}
\vspace*{-6pt}
\Caption{Вид изображений: (\textit{а})~линии уровня~$\varphi_2$ и~$\Psi_{1,2}$ 
на Image; (\textit{б})~значения~$\Psi_{1,2}$ на линиях уровня; 
(\textit{в})~значения~$\varphi_2$ на линиях уровня; 
(\textit{г})~$\Psi_{1,2}(\mathrm{Image})$; (\textit{д})~$\varphi_2(\mathrm{Image})$
\label{f4ar}}
\end{figure*}

\begin{figure*} %fig5
\vspace*{1pt}
\begin{center}
\mbox{%
\epsfxsize=123.825mm
\epsfbox{arh-5.eps}
}
\end{center}
\vspace*{-6pt}
\Caption{Вид изображений: (\textit{а})~линии уровня~$\varphi_3$ и~$\Psi_{1,3}$ 
на Image; (\textit{б})~значения~$\Psi_{1,3}$ на линиях уровня; 
(\textit{в})~значения~$\varphi_3$ на линиях уровня; 
(\textit{г})~$\Psi_{1,3}(\mathrm{Image})$; (\textit{д})~$\varphi_3(\mathrm{Image})$
\label{f5ar}}
\end{figure*}

\begin{figure*} %fig6
\vspace*{1pt}
\begin{center}
\mbox{%
\epsfxsize=123.825mm
\epsfbox{arh-6.eps}
}
\end{center}
\vspace*{-6pt}
\Caption{Вид изображений: (\textit{а})~линии уровня~$\varphi_4$ и~$\Psi_{1,4}$ 
на Image; (\textit{б})~значения~$\Psi_{1,4}$ на линиях уровня; 
(\textit{в})~значения~$\varphi_4$ на линиях уровня; 
(\textit{г})~$\Psi_{1,4}(\mathrm{Image})$; (\textit{д})~$\varphi_4(\mathrm{Image})$
\label{f6ar}}
\end{figure*}

\begin{figure*} %fig7
\vspace*{1pt}
\begin{center}
\mbox{%
\epsfxsize=123.825mm
\epsfbox{arh-7.eps}
}
\end{center}
\vspace*{-6pt}
\Caption{Вид изображений: (\textit{а})~линии уровня~$\varphi_5$ и~$\Psi_{1,5}$ 
на Image; (\textit{б})~значения~$\Psi_{1,5}$ на линиях уровня; 
(\textit{в})~значения~$\varphi_5$ на линиях уровня; 
(\textit{г})~$\Psi_{1,5}(\mathrm{Image})$; (\textit{д})~$\varphi_5(\mathrm{Image})$
\label{f7ar}}
\end{figure*}

\begin{figure*} %fig8
\vspace*{1pt}
\begin{center}
\mbox{%
\epsfxsize=123.825mm
\epsfbox{arh-8.eps}
}
\end{center}
\vspace*{-6pt}
  \Caption{Вид изображений: (\textit{а})~линии уровня~$\varphi_6$ 
и~$\Psi_{1,6}$ на Image; (\textit{б})~значения~$\Psi_{1,6}$ на линиях уровня; 
(\textit{в})~значения~$\varphi_6$ на линиях уровня; 
(\textit{г})~$\Psi_{1,6}(\mathrm{Image})$; (\textit{д})~$\varphi_6(\mathrm{Image})$
\label{f8ar}}
\end{figure*}
  

\vspace*{-8pt}

\section{Заключение}

\vspace*{-4pt}
  
  Рассмотрена задача интеграции гетерогенной информации о цветных 
пикселях, их трансформации в пользовательской компьютерной системе и о 
стандартном и индивидуальном цветовосприятии пользователя в единую 
инфокоммуникационную среду. 
  
  Задача решена на основе RGB-ха\-рак\-те\-ри\-за\-ции цветовосприятия 
произвольным пользователем ПЭВМ вывода на периферийные устрой\-ства, что\linebreak 
позволяет оцифровать его цветовое пространство и цветовосприятия, 
представить оцифрованные\linebreak результаты в виде, удобном для организации\linebreak 
управ\-ле\-ния цветопередачей в пользовательской ком\-пьютерной системе в 
интересах наблюдателя, име\-ющего такую аномалию цветового зрения, как 
частичная цветовая слепота.
  
  Сравнение применяемого метода с известными подходами к характеризации 
цветовосприятия, например~\cite{5ar, 6ar}, показывает, что предлагаемая\linebreak 
модель позволяет устранить противоречие в характеризации 
цветовосприятия, %~\cite{5ar, 6ar}, 
поскольку обнаруживает различие в 
индивидуальных особенностях цветовосприятия, лежащих в их основе. 
Расхождение\linebreak между моделями~\cite{5ar, 6ar} обусловлено различными\linebreak 
гипотезами о цветовосприятии дейтеранопов, протанопов и тританопов, в 
частности к различию в определении зон толерантности одних и тех же 
типичных наблюдателей с аномалией цветового
 зрения.
 Это еще раз 
подчеркивает необходимость\linebreak тестирования  произвольного наблюдателя для точного
определения и дальнейшего корректного уче-\linebreak\vspace*{-12pt}

\pagebreak

\end{multicols}

\begin{figure} %fig9
\vspace*{1pt}
\begin{center}
\mbox{%
\epsfxsize=124.669mm
\epsfbox{arh-9.eps}
}
\end{center}
\vspace*{-6pt}
\Caption{Вид изображений: (\textit{а})~линии уровня~$\varphi_1$; 
(\textit{б})~линии уровня~$\varphi_4$; (\textit{в})~линии уровня~$\varphi_2$; 
(\textit{г})~линии уровня~$\varphi_5$; (\textit{д})~линии уровня~$\varphi_3$; 
(\textit{е})~линии уровня~$\varphi_6$
\label{f9ar}}
\end{figure}

\begin{multicols}{2}

\noindent
та индивидуальных 
особенностей его цветовосприятия.


  
  Результаты работы имеют важное практическое значение, поскольку на 
основе интеграции цифрового описания цветовосприятия произвольного 
наблюдателя с гетерогенной информацией:
  \begin{itemize}
  \item о цветовых пространствах периферийных устройств;
  \item о цветопередаче (отображении цветовых пространств друг на друга) в 
пользовательской компьютерной сис\-теме;
  \item о стандартном цветовосприятии
  \end{itemize}
они применимы для создания специальных легко адаптируемых к компонентам 
системы прог\-рам\-мно-тех\-ни\-че\-ских инструментов, обес\-пе\-чи\-ва\-ющих 
адекватное восприятие произвольными\linebreak пользователями вывода на 
периферийные устройства ПЭВМ. 

\vspace*{-6pt}

{\small\frenchspacing
{%\baselineskip=10.8pt
\addcontentsline{toc}{section}{Литература}
\begin{thebibliography}{99}


  \bibitem{1ar}
  \Au{Джадд Д., Вышецки Г.}
  Цвет в науке и технике.~--- М.: Мир, 1978.
  
  \bibitem{2ar}
  \Au{Шашлов Б.\,А.}
  Цвет и цветовоспроизведение.~--- М.: Мир, 1986.
  
    \bibitem{4ar}
  \Au{Юстова Е.\,Н., Алексеева К.\,А., Волоков В.\,В. и~др.}
  Пороговые таблицы для исследования цветового зрения: Методическое 
руководство.~--- М., 2000. 
  
  \bibitem{3ar}
  Пороговые таблицы для исследования цветового зрения: Паспорт. Комплект 
№\,4181.~--- М: Вида, 2008.
  
  
  \bibitem{5ar}
  \Au{Jenny B, Vaughn Kelso~N.}
  Color Oracle: Design for the color impaired. {\sf 
http://colororacle.cartography.ch}.
  
  \bibitem{6ar}
  Vischeck. {\sf http://www.vischeck.com}.
  
  \bibitem{7ar}
  \Au{Архипов О.\,П., Зыкова З.\,П.}
  Допечатное тестирование индивидуального зрительного восприятия~// 
Вестник компьютерных и информационных технологий, 2008. №\,12. С.~2--8.
  
  \bibitem{8ar}
  \Au{Архипов О.\,П., Бородина Л.\,Н., Зыков~Р.\,В. и~др.}
  Технология оцифровки цветовосприятия отпечатков.~--- М.: ИПИ РАН, 
2009. 
  
  \bibitem{9ar}
  \Au{Архипов О.\,П., Бородина Л.\,Н., Зыков~Р.\,В. и~др.}
  Инструментальная оценка цветов отпечатков.~--- М.: ИПИ РАН, 2009. 
  
  \bibitem{10ar}
  \Au{Соколов И.\,А., Архипов О.\,П., Захаров~В.\,Н., Зыкова~З.\,П., 
Архипов~П.\,О.}
  Способ компьютерного распознавания и визуального воспроизведения 
цветных изображений. Пат.\ 2005130683, приоритет 04.10.05.  Бюл.~№\,8 
от 20.03.07. 

 \label{end\stat}
  
  \bibitem{11ar}
  \Au{Городецкий В.\,И., Самойлов В.\,В.}
  Стеганография на основе цифровых изображений~// Информационные 
технологии и вычислительные системы, 2001. №\,2/3. С.~51--64.
 \end{thebibliography}
}
}


\end{multicols}  
  
  

  
  