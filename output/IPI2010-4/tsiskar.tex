\def\stat{tsiskar}

\def\tit{МАТЕМАТИЧЕСКАЯ МОДЕЛЬ И~МЕТОД ВОССТАНОВЛЕНИЯ ПОЗЫ 
ЧЕЛОВЕКА ПО~СТЕРЕОПАРЕ СИЛУЭТНЫХ ИЗОБРАЖЕНИЙ$^*$}

\def\titkol{Математическая модель и~метод восстановления позы 
человека по~стереопаре силуэтных изображений}

\def\autkol{А.\,К.~Цискаридзе}
\def\aut{А.\,К.~Цискаридзе$^1$}

\titel{\tit}{\aut}{\autkol}{\titkol}

{\renewcommand{\thefootnote}{\fnsymbol{footnote}}\footnotetext[1]
{Работа выполнена при поддержке РФФИ, гранты №\,08-01-00670, №\,08-07-00270.}}

\renewcommand{\thefootnote}{\arabic{footnote}}
\footnotetext[1]{Московский физико-технический институт, AchikoTsi@gmail.com}

\vspace*{-6pt}

\Abst{Работа посвящена проблеме восстановления позы локально симметричного объекта по 
стереопаре силуэтов в случае отсутствия окклюзии. Рассматриваются две модели фигуры. В 
первом случае фигура описывается как объединение пространственных жирных линий и 
предлагается метод ее восстановления. Во втором случае фигура описывается в виде 
шарнирной модели и предлагается метод ее подгонки по стереопаре силуэтов. Методы 
основаны на построении непрерывных скелетов силуэтов.}

%\vspace*{4pt}

\KW{стереореконструкция; скелет; цилиндрические объекты; шарнирная модель; 
серединные оси}

%\vspace*{6pt}

       \vskip 14pt plus 9pt minus 6pt

      \thispagestyle{headings}

      \begin{multicols}{2}

      \label{st\stat}

\section{Введение}
     
     Задача восстановления формы пространственного объекта по нескольким 
двумерным изображениям хорошо известна и имеет множество приложений. 
В~частности, эта проблема возникает при распознавании позы и жестов 
человека в системах видеонаблюдения. Особенность рассматриваемой нами 
постановки этой задачи состоит в том, что двумерные изображения являются 
бинарными и представляют собой лишь силуэты пространственного объекта. 
Такая задача, в частности, возникает в системах видеонаблюдения, 
работающих в условиях плохой освещенности. В~этом случае камеры плохо 
передают текстурные особенности изображений и позволяют с достоверностью 
выявить лишь силуэты представленных на изображении объектов. Для 
распознавания позы и жестов требуется по этим силуэтам восстановить 
пространственную форму столь сложного и изменчивого объекта, как фигура 
человека.
     
     Невозможность анализа изображений на уровне текстур не позволяет 
применить для такой постановки задачи хорошо известные методы, 
основанные на автоматическом выявлении общих точек, присутствующих на 
обоих изображениях стереопары. Очевидно, что если на изображении 
пред\-став\-лен лишь силуэт объекта, то достоверно на нем можно выявить лишь 
точки границы этого объекта. Но на двух картинках в стереопаре изображений 
границы силуэтов полностью различаются, т.\,е.\ все точки границы одного 
силуэта отличаются от всех граничных точек другого силуэта. Поэтому 
выявление общих точек невозможно.
     
     В общей постановке задачи можно выделить два варианта, 
различающихся наличием так называемых окклюзий. Силуэтное изображение 
объекта называется изображением без окклюзии, если в каждую его точку 
проектируется не более двух точек поверхности объекта. В~соответствии с 
этим первый, более простой, вариант задачи представляют собой работу с 
изображениями без окклюзий. В~этом случае в силуэтах видны голова и все 
конечности. Второй вариант задачи~--- когда окклюзии имеют место. 
В~данной работе ограничимся рассмотрением изображений без окклюзий. Не 
будем также затрагивать вопрос выделения силуэтов на исходных картинах, 
считая что они уже получены с хорошей (1--2~пикселя) точностью.
     
     Подходы к решению этой задачи в полном объеме для сложных объектов 
в настоящее время только лишь формируются~[1--3]. Можно сослаться на работу~\cite{1ts}, 
в которой описан метод восстановления поверхности сложного 
пространственного объекта (фигура лошади) по серии силуэтных изображений, 
полученных в разных ракурсах. Для этого метода требуются изображения 
хорошего качества, а также большие ресурсы процессорного времени.
     
     Вместе с тем существует определенный класс объектов, чьи структурные 
особенности позволяют решить задачу восстановления пространственной 
формы и при отсутствии видимых общих точек.\linebreak Это объекты, состоящие из 
круговых цилиндров. Особенности такого пространственного объекта, как 
фигура человека, позволяют рассматривать\linebreak его приближенно как объединение 
нескольких\linebreak <<цилиндрических>> элементов, имеющих локальную осевую 
симметрию. Такие объекты некоторые\linebreak авторы называют обобщенными 
цилиндрами~\cite{2ts}. Встречается также термин <<трубчатые объекты>> 
(tubular)~\cite{3ts}. Если говорить более строго, под цилиндрическим элементом 
понимается пространственное тело, образованное семейством шаров, центры 
которых расположены на некоторой осевой пространственной кривой. 
В~работах~[7, 8] такие объекты называются пространственными жирными 
кривыми. Представляют интерес объекты, которые могут быть представлены в 
виде объединения небольшого числа пространственных жирных кривых. 
Человеческая фигура в некотором приближении также может быть составлена 
из жирных кривых по аналогии с тем, как дети лепят человечков из 
пластилина.
     
     Предлагаемый в данной работе подход к решению основывается на идее 
построения скелетов для стереопары силуэтных изображений. Скелет 
представляет собой совокупность серединных осей силуэта, определяемых как 
геометрическое место точек~--- центров вписанных в силуэт окружностей. 
Использование скелетов открывает несколько возможных путей для решения 
задачи. Рассмотрим два из них. 
     
     Первый путь состоит в построении пространственной циркулярной 
модели человеческой фигуры. Он основывается на идее конструирования 
стереопар <<невидимых>> общих точек обоих изоб\-ра\-же\-ний. Серединные оси 
силуэтов предлагается рассматривать как плоские проекции пространственных 
осевых линий жирных кривых, составляющих\linebreak фигуру человека. Такой подход 
позволяет свести задачу восстановления этих пространственных жирных линий 
к вычислению пространственных кривых по стереопарам их проекций. 
Результатом\linebreak решения задачи является циркулярная модель, представляющая 
собой объединение нескольких пространственных жирных линий. 
     
     Второй путь предполагает упрощенное описание фигуры человека в виде 
<<шарнирной>> модели заданной структуры. Шарнирная модель также 
состоит из пространственных жирных линий. Но эти жирные линии являются 
линейными сегментами постоянной заданной ширины. Поза человека ищется 
путем подбора некоторого преобразования шарнирной модели, при котором ее 
плоские проекции будут в наибольшей степени совпадать со скелетами 
стереопары силуэтов. 

\vspace*{-12pt}

\section{Циркулярная и шарнирная модели фигуры человека}

     Рассмотрим множество точек~$T$ в евклидовом пространстве~$R^3$, 
имеющее вид связного графа,\linebreak

\noindent
\begin{center} %fig1
\vspace*{-6pt}
\mbox{%
\epsfxsize=76.379mm
\epsfbox{cis-1.eps}
}
\end{center}
\vspace*{3pt}
%\begin{center}
{{\figurename~1}\ \ \small{Циркулярная модель фигуры человека: осевой граф~(\textit{а}) и огибающая 
поверхность~(\textit{б})}}
%\end{center}
\vspace*{3pt}

\bigskip
\addtocounter{figure}{1}

\noindent
описывающего человеческую фигуру 
(рис.~1). Граф имеет пять терминальных вершин и три вершины 
третьей степени, а его ребра являются непрерывными линиями. При этом ребра 
не имеют точек пересечения, не совпадающих с их концами.

   
     С каждой точкой $t\in T$ графа~$T$ связан некоторый шар~$c_t$ с 
центром в этой точке. Это семейство шаров $C=\{c_t,\,t\in T\}$ называется 
\textit{циркулярным графом}, для краткости~--- \textit{циркуляром}~\cite{5ts}. 
Граф~$T$ называется \textit{осевым графом} циркулярного графа.\linebreak 
Объединение $S=\bigcup\limits_{t\in T} c_t$ всех шаров семейства~$C$ как 
точечных множеств является циркулярной мо\-делью фигуры человека. 
Границей модели является огибающая поверхность семейства шаров~$C$. 
     
     Восстановление циркулярной модели сводится к нахождению 
составляющих циркулярный граф жирных линий, т.\,е.\ к описанию их осевых 
линий и функции ширины, задающей для каждой точки осевой линии радиус 
шара с центром в этой точке. При этом построение жирных линий должно 
осуществляться таким образом, чтобы проекции циркулярной модели хорошо 
согласовывались со стереопарой исходных изображений.
     
     Шарнирная модель описывает фигуру человека как объединение 
10~шарнирно закрепленных твердых тел. Каждый элемент этой конструкции 
представляет собой локально симметрический объект, образованный 
объединением шаров одинакового радиуса с центрами на прямолинейном 
отрезке (рис.~\ref{f2ts}). 
     
     С каждым элементом свяжем систему координат, ось~$X$ которой 
направлена вдоль его оси сим\-мет\-рии, а начало координат находится в точке 
крепления его с родительским телом. Для элемента известна его длина и 
ширина, а также точка крепления в системе координат родительского тела. По 
иерархии крепления элементы шарнирной модели образуют дерево с корневым 
элементом, соответствующим туловищу человека.
     
     \begin{figure*} %fig2
     \vspace*{1pt}
\begin{center}
\mbox{%
\epsfxsize=130.894mm
\epsfbox{cis-2.eps}
}
\end{center}
\vspace*{-6pt}
\Caption{Шарнирная модель фигуры человека: структура модели~(\textit{а}) и пространственные 
элементы модели~(\textit{б})
\label{f2ts}}
\end{figure*}
\begin{figure*}[b] %fig3
\vspace*{1pt}
\begin{center}
\mbox{%
\epsfxsize=152.966mm
\epsfbox{cis-3.eps}
}
\end{center}
\vspace*{-6pt}
\Caption{Стереопара исходных силуэтов~(\textit{а}) и стереопара скелетов~(\textit{б})
\label{f3ts}}
\end{figure*}
     
     Вращение каждой части шарнирной модели задается относительно 
родительской системы координат. Его можно параметризовать с помощью 
суперпозиции трех вращений $R_\theta \circ R\psi \circ R_\varphi$ относительно 
осей~$X, Y, Z$ своей системы координат (см.\ рис.~\ref{f2ts}). Так как структура 
скелета человека допускает не всякие вращения, для каждого тела введем 
ограничения на углы в виде параллелепипеда: $\varphi_{\min}\leq \varphi \leq 
\varphi_{\max}$, $\psi_{\min}\leq \psi\leq \psi_{\max}$, 
$\theta_{\min}\leq\theta\leq \theta_{\max}$.
     
     Под позой объекта будем понимать вектор значений динамических 
параметров модели. Каждой позе соответствует точка в пространстве из 
24~динамических параметров фигуры, а все множество поз описывается 
параллелепипедом~$\Theta$ в 24-мерном пространстве $\Theta\subset R^{24}$. 
Таким образом, задача состоит в том, чтобы по стереопаре бинарных 
изображений найти вектор динамических па\-ра\-мет\-ров шарнирной модели, 
аппроксимирующей форму пространственной фигуры.

\section{Восстановление позы в виде циркулярной модели }

\subsection{Структура метода} %3.1

     Предлагаемый подход к восстановлению формы человеческой фигуры в 
виде совокупности жирных линий иллюстрируется на рис.~3--5. 
\begin{figure*} %fig4
\vspace*{1pt}
\begin{center}
\mbox{%
\epsfxsize=88.443mm
\epsfbox{cis-4.eps}
}
\end{center}
\vspace*{-6pt}
\Caption{Предположение, что проекции осей объекта совпадают со скелетом силуэтов
\label{f4ts}}
\vspace*{6pt}
\end{figure*}


     Сначала для силуэтов, полученных на основе сегментации исходных 
изображений (см.\ рис.~\ref{f3ts}), строятся их скелеты в виде серединных 
осей~\cite{12ts}, с которыми связано множество вписанных в силуэты 
окружностей с центрами на серединных осях (см.\linebreak\vspace*{-12pt}
\pagebreak

\noindent
\begin{center} %fig5
\vspace*{-6pt}
\mbox{%
\epsfxsize=73.118mm
\epsfbox{cis-5.eps}
}
\end{center}
\vspace*{3pt}
%\begin{center}
{{\figurename~5}\ \ \small{Кривая, сопоставляющая стереопары точек}}
%\end{center}
\vspace*{3pt}

\bigskip
\addtocounter{figure}{1}


\noindent
 рис.~\ref{f3ts}). После этого из 
стереопары скелетов конструируется пространственный скелет объекта, а по 
семействам вписанных кругов вычисляются радиусы сфер с центрами на 
пространственном скелете. Таким образом, поза человека может быть 
представлена пространственным скелетом, либо огибающей поверхностью для 
семейства построенных сфер.
     
     В случае, когда нет окклюзии в силуэтах, в первом приближении можно 
считать, что направление взгляда на объект не сильно отклоняется от 
оптической оси камеры. Тогда можно предполагать, что на плоском 
изображении образ пространственной оси локально симметрического объекта 
совпадает со скелетом силуэта (см.\ рис.~\ref{f4ts}). Строго говоря, центральная 
проекция сферы на плоскость является эллипсом, но приближенно эти эллипсы 
будем считать окружностями. По двум плоским проекциям пространственной 
оси с помощью эпиполярной геометрии~[11] можно восстановить 
пространственную ось.
     


\subsection{Идентификация реперных точек на~скелетах} %3.2
     
     Рассмотрим стереопару силуэтов и их скелетов\linebreak (см.\  рис.~\ref{f3ts}). 
Объект <<фигура человека>> прибли\-женно представляется осесимметричными 
элементами. Тогда на плоских изображениях образ про-\linebreak странственной оси 
осесимметричного элемента\linebreak совпадает с соответствующими ветвями скелетов 
(см.\ рис.~\ref{f3ts}, кривые~$OA$ и $O^\prime A^\prime$). Отсюда можно 
сделать предположение, что множество стереопарных точек ветви~$OA$ скелета 
одного силуэта совпадает с ветвью~ $O^\prime A^\prime$ скелета другого 
силуэта, что позволяет построить кривую в пространстве. Задача состоит в том, 
чтобы наилучшим образом установить соответствие между точками этих 
скелетных ветвей. Если задать кривую~$OA$ как непрерывное 
отображение~$r_1$:~$[0,\,1] \rightarrow R^2$, а кривую~$O^\prime A^\prime$ 
как $r_2$:~$[0,\,1] \rightarrow R^2$, задача сведется к на\-хож\-де\-нию 
непрерывного монотонного отображения $w$:~$[0,\,1] \rightarrow [0,\,1]$, 
которое сопоставляет стереопары точек $r_1(t) \leftrightarrow r_2(w(t))$ и при 
этом минимизирует расхождение, заданное в виде функционала
     $$
     \min \int\limits_0^1 \rho (r_1(t),\,r_2(w(t)))\sqrt{1+w^\prime(t)^2}\,dt\,.
     $$
          Здесь $\rho(X,Y)$~--- функция штрафа~--- отражает,\linebreak насколько хорошо 
сопоставляются точки~$X$ и~$Y$.\linebreak Выбор этой функции осуществляется на 
основе следующего соображения. Для каждой точки изоб\-ра\-же\-ния существует 
луч в пространстве, который в нее проецируется. 
\begin{figure*} %fig6
\vspace*{1pt}
\begin{center}
\mbox{%
\epsfxsize=159.139mm
\epsfbox{cis-6.eps}
}
\end{center}
\vspace*{-6pt}
\Caption{Стереопара изображений и полученный пространственный объект
\label{f6ts}}
\end{figure*}
\begin{figure*}[b] %fig7
\vspace*{1pt}
\begin{center}
\mbox{%
\epsfxsize=164.469mm
\epsfbox{cis-7.eps}
}
\end{center}
\vspace*{-6pt}
\Caption{Стереопары и полученные пространственные объекты для <<Kungfu 
girl>>
\label{f7ts}}
\end{figure*}
При идеально правильном 
сопоставлении точек~$X$ и~$Y$ проходящие через них лучи должны 
пересекаться. Поэтому в качестве~$\rho$ можно взять расстояние между 
скрещивающимися лучами, которые проецируются в точки~$X$ и~$Y$.
     
     Полученная задача решается методом динамического программирования. 
Дискретизируя задачу сеткой $N\times N$, для нахождения точного решения 
получаем сложность $O(N^2d)$ (см.\ рис.~5). Здесь $d$~--- ограничение, 
задающее коридор для кривой. Построив пространственные оси, используя 
ширину скелетов, можно вычислить размеры шаров с центрами на этих осях. 
     

      
Пример визуализации модели фигуры, полученной по стереопаре изображений, 
представлен на рис.~\ref{f6ts}. Как видно из этого примера, визуализация 
является не вполне реалистичной, поскольку описание тела человека 
цилиндрами представляется весьма грубым. Однако для вычислений, 
связанных с распознаванием позы, такая точность представляется вполне 
приемлемой. 

\subsection{Вычислительные эксперименты} %3.3

     Эксперименты с восстановлением пространственной модели фигуры 
человека проводились с куклами размером 30~см. Это объясняется лишь\linebreak 
упрощением съемки в лабораторных условиях. На рис.~\ref{f6ts} показана 
стереопара исходных изоб\-ра\-же\-ний и полученный пространственный объект. 
Эксперименты также проводились с синтетическими данными <<Kungfu 
girl>>, предоставленными группой Graphics-Optics-Vision из института\linebreak 
     Max-Planck~\cite{6ts}. Эти данные представляют собой синтезированные 
пространственные модели виртуальных сцен и их проекции в виде двумерных 
изоб\-ра\-же\-ний размером $320\times 240$. Результаты реконструкции показаны 
на рис.~\ref{f7ts}.



     Эксперименты показывают, что модель локально симметричного объекта 
хорошо описывает фигуру человека в целом. Скорость работы на компьютере 
Intel Pentium~IV, Core~2 Duo, 2800~МГц составила более 5~кадров/с. 
Это дало возможность использовать предложенный метод в системах 
компьютерного зрения в реальном масштабе времени их работы. 
     

\section{Восстановление позы в виде шарнирной модели }

\subsection{Предлагаемый подход к решению задачи} %4.1
     
     Предположим, что определена шарнирная модель объекта вместе с 
длинами составляющих ее элементов. Задача подгонки шарнирной модели 
заключается в следующем:
     
     \textbf{Дано:} Шарнирная модель; стереопара силуэтов~$s_1$, $s_2$; 
проекционные матрицы камер.
     
     \textbf{Найти:} Вектор динамических параметров шарнирной модели 
$\vec{\theta}^*\in\Theta$, который имеет стереопару силуэтов, максимально 
совпадающую с наблю\-да\-емой стереопарой. 
     
     Другими словами, $\vec{\theta}^*\;=\;\mathrm{arg}\,\underset{\vec\theta \in 
\Theta}{\min}\,\mu (\vec\theta, S_1, S_2)$, где $\mu(\vec\theta, S_1, S_2)$~--- 
величина сходства позы с силуэтами. Такая постановка задачи возможна также 
для случая с окклюзиями. При этом основная сложность заключается в том, что 
любая целевая функция для таких сложных объектов, как фигура человека, 
является многоэкстремальной~\cite{9ts}. В~случае, когда нет окклюзии, 
удается выписать хорошую критериальную функцию. При этом на основе 
скелетов силуэтов делается предварительная сегментация силуэта. 

\subsection{Подгонка шарнирной модели под~стереопару наблюдаемых 
силуэтов для~задачи без~окклюзии } %4.2
     
     В данном подходе предполагается, что известна шарнирная модель 
объекта вместе с длинами отдельных ее элементов. Опишем функцию сходства 
шарнирной модели с наблюдаемыми силуэтами и метод ее минимизации. Для 
каждого силуэта $S_i$:~$i\in \{1,\,2\}$ построим его базовый скелет. Анализ 
геометрии скелетов позволяет выделить на них оси для каждой из 6~частей 
тела человека. На рис.~8 показана такая сегментация для одного из 
скелетов. Через $A_i$, $\gamma_i$, $i=1,\ldots , 5$, обозначены концевые 
вершины и сегменты соответственно. Для любой позы $\vec\vartheta\in\Theta$ 
шарнирной модели построим проекции ее осей на плоскость камеры и выделим 
на них соответствующие ветви. Обозначим их через $B_i$, $\lambda_i$,  $i = 1, 
\ldots , 5$ (см.\ рис.~8). Введем расхождение~$\rho$ между двумя 
кривыми~$\gamma$ и~$\lambda$ на плоскости как
     $$
     \rho(\gamma , \lambda ) =\int\limits_{\gamma} 
d(\gamma(t),\lambda)\,d\gamma+\int\limits_\gamma 
d(\lambda(t),\gamma)\,d\lambda\,.
     $$
Здесь $d(a,\lambda)=\underset{b\in\lambda}{\min} \vert a-b\vert^2$ задает 
расстояние от точки~$a$ до кривой~$\lambda$. Тогда функцию сходства 
$\mu(\theta , S_1, S_2)$ шарнирной модели с наблюдаемыми силуэтами введем 
как $\mu(\theta ,S_1, S_2) =\mu_1(\theta, S_1)+\mu_2(\theta ,S_2)$, где 
$\mu_1(\theta,S_1)=$\linebreak $=\sum\limits_{i=1}^5\left(\rho(\gamma_i,\lambda_i)+\alpha \vert 
A_i-B_i\vert^2\right)$ задает сходство по первой камере, а~$\mu_2$ 
определяется аналогичным образом для второй камеры. Второй член 
$\alpha\vert A_i-B_i\vert^2$ учитывает расхождение в концевых точках. 

\noindent
\begin{center} %fig8
\vspace*{12pt}
\mbox{%
\epsfxsize=79.684mm
\epsfbox{cis-8.eps}
}
\end{center}
\vspace*{3pt}
%\begin{center}
{{\figurename~8}\ \ \small{Сегментация скелета~(\textit{а}) и сегментация проекции осей шарнирной 
модели~(\textit{б})}}
%\end{center}
\vspace*{3pt}

\bigskip
\addtocounter{figure}{1}

     
     Полученная функция сходства обладает свойством унимодальности. Для 
простоты реализации использовалась квазиньютоновская схема LBFGS с 
численным градиентом. Градиент вычислялся разностной схемой по двум 
точкам. На рис.~\ref{f9ts} показан результат подгонки, спроецированный в 
разных ракурсах. Эксперименты показывают, что шарнирная модель хорошо 
описывает позу человека.

\begin{figure*} %fig9
\vspace*{1pt}
\begin{center}
\mbox{%
\epsfxsize=100.11mm
\epsfbox{cis-9.eps}
}
\end{center}
\vspace*{-6pt}
\Caption{Полученная шарнирная модель в разных ракурсах
\label{f9ts}}
\end{figure*}
      
\section{Заключение}

     В работе рассмотрена задача восстановления позы локально 
симметричного объекта на примере фигуры человека в случае отсутствия 
окклюзии в силуэтах. Рассматриваются две модели фигуры. В~первом случае 
фигура описывается как объединение пространственных жирных линий и 
предлагается метод ее восстановления. Во втором случае фигура описывается в 
виде шарнирной модели и предлагается метод ее подгонки по стереопаре 
силуэтов. В~обоих случаях эксперименты показали положительный результат. 
В~будущем планируется разработать методы подгонки шарнирной модели в 
случае окклюзии. Скелетное представление силуэта позволяет анализировать 
окклюзию, что может быть использовано для построения начального 
приближения в задаче подгонки.
     
     \bigskip
     Автор выражает благодарность своему научному руководителю 
профессору кафедры интеллектуальных систем Московского 
     физико-технического института Местецкому Леониду Моисеевичу за 
помощь в постановке задачи и внимание к работе.

{\small\frenchspacing
{%\baselineskip=10.8pt
\addcontentsline{toc}{section}{Литература}
\begin{thebibliography}{99}

\bibitem{10ts} %1
\Au{Agarwal A., Triggs B.}
Recovering 3D human pose from monocular images~// Pattern Analysis and 
Machine Intelligence, 2006.

\bibitem{7ts} %2
\Au{Tong M., Liu Yu., Huang T.\,S.}
3D human model and joint parameter estimation from monocular image~// Pattern 
Recognition Letters, 2007. Vol.~28. P.~797--805.

\bibitem{8ts} %3
\Au{Balan A.\,O., Sigal L., Black M.\,J., Davis J.\,E., Haussecker~H.\,W.}
Detailed human shape and pose from images~// IEEE Conference on Computer 
Vision and Pattern Recognition, 2007.
  
\bibitem{1ts} %4
\Au{Yezzi A.\,J., Soatto S.}
Structure from motion for scenes without features~// 2003 IEEE Computer Society 
Conference on Computer Vision and Pattern Recognition (CVPR'03) Proceedings. 
Vol.~1. P.~525--532.

\bibitem{2ts} %5
\Au{Senior A.} Real-time articulated human body tracking using silhouette 
information~// IEEE Workshop on Visual Surveillance/PETS Proceedings. France, 
2003.

\bibitem{3ts} %6
\Au{Cumani A. Guiducci A.}
Recovering the 3D structure of tubular objects from stereo silhouettes~// Pattern 
Recognition, 1997. Vol.~30. No.\,7. P.~1051--1059.

\bibitem{4ts} %7
\Au{Местецкий Л.\,М., Щетинин Д.\,В.}
Объемные примитивы Безье~// Графикон-2001: Тр.\ 11-й международной 
конф.~--- Нижний Новгород, 2001. С.~161--165.

\bibitem{11ts} %8
\Au{Mestetskiy L.}
Shape comparison of flexible objects~//  Conference (International) on Computer 
Vision Theory and Applications, 2007.

\bibitem{5ts} %9
\Au{Местецкий Л.\,М.}
Непрерывная морфология бинарных изображений: фигуры, скелеты, 
циркуляры.~--- М.: Физматлит, 2009.

\bibitem{12ts} %10
\Au{Местецкий Л.\,М.}
Непрерывный скелет бинарного раст\-ро\-во\-го изображения~// Графикон-98: Тр. 
международной конф.~--- М.: МГУ, 1998. С.~71--78. 

\bibitem{13ts} %11
\Au{Форсайт Д., Понс Ж.}
Компьютерное зрение.~--- М.: Вильямс, 2004.

\bibitem{6ts} %12
<<Kung-Fu Girl>>~--- a synthetic test sequence for multi-view reconstruction and 
rendering research. {\sf http://www.\linebreak mpi-inf.mpg.de/departments/irg3/kungfu}.


\label{end\stat}

\bibitem{9ts}
\Au{Sminchisescu C., Triggs B.}
Kinematic jump processes for monocular 3D human tracking~// 2003 IEEE 
Computer Society Conference on Computer Vision and Pattern Recognition 
(CVPR'03) Proceedings. Vol.~1. P.~69--76.

 \end{thebibliography}
}
}


\end{multicols}