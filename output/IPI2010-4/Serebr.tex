
\def\stat{serebr}

\def\tit{СИСТЕМА УПРАВЛЕНИЯ ЭЛЕКТРОННОЙ БИБЛИОТЕКОЙ LIBMETA}

\def\titkol{Система управления электронной библиотекой LibMeta}

\def\autkol{А.\,А.~Захаров, В.\,А.~Серебряков}
\def\aut{А.\,А.~Захаров$^1$, В.\,А.~Серебряков$^2$}

\titel{\tit}{\aut}{\autkol}{\titkol}

%{\renewcommand{\thefootnote}{\fnsymbol{footnote}}\footnotetext[1]
%{Исследование поддержано грантами РФФИ 08-07-00152 и 09-07-12032.
%Статья написана на основе материалов доклада, представленного на IV 
%Международном семинаре  <<Прикладные задачи теории вероятностей и математической статистики, 
%связанные с моделированием информационных систем>> (зимняя сессия, Аоста, Италия, январь--февраль 2010~г.).}}

\renewcommand{\thefootnote}{\arabic{footnote}}
\footnotetext[1]{Вычислительный центр им.\ А.\,А.~Дородницына РАН, andreya@sufler.ru}
\footnotetext[2]{Вычислительный центр им.\ А.\,А.~Дородницына РАН, serebr@ccas.ru}

\vspace*{-6pt}

\Abst{В статье рассмотрены вопросы создания современных электронных 
библиотек (ЭБ). Перечислены основные требования к ЭБ, в частности требования 
по интеграции с внешними системами и стандартизации, а также предложены 
некоторые пути их удовлетворения. Рассмотрены основные мировые стандарты в 
области ЭБ. Представлена к рассмотрению система управления ЭБ (СУЭБ) 
LibMeta~--- универсальное средство для создания ЭБ.}

\vspace*{-2pt}

\KW{электронные библиотеки; Единое научное информационное пространство 
(ЕНИП)}

       \vskip 10pt plus 9pt minus 6pt

      \thispagestyle{headings}

      \begin{multicols}{2}

      \label{st\stat}


\section{Введение}
  
  В последние годы объемы информации в сети Интернет в связи с ее бурным развитием 
лавинообразно увеличиваются~[1]. Несмотря на все большее проникновение технологий 
Semantic Web~[2, 3], ощущается серьезная нехватка средств поиска и каталогизации 
информации, которые позволяли бы искать ее именно по семантике и связям, а не только по 
ключевым словам и полным текстам, как это делают универсальные поисковые системы. 
Одним из способов решения данной проблемы видится появление и все большее 
распространение различного рода ЭБ~[4, 5].
  
  Зачастую понятие ЭБ~[6] смешивают с уже давно существующими электронными 
каталогами~--- информационными системами, предназначенными для использования в 
классических библиотеках, содержащими только метаинформацию и служащими лишь 
средством поиска. Однако ЭБ~--- это совершенно другой класс систем: они хранят не только 
метаданные, но также и цифровые версии информационных ресурсов. Еще одним важным 
признаком ЭБ является то, что в качестве информационного ресурса может пониматься 
фактически все что угодно, например музейные предметы или архивные материалы. В~связи 
с этим можно говорить о сближении задач ЭБ, электронных архивов и цифровых музеев~--- 
все эти информационные системы фактически являются ЭБ с различной специализацией~[7].
  
  Несмотря на перспективы, в ходе развития ЭБ предстоит преодолеть немало препятствий. 
Самым большим из них, пожалуй, является недостаток в хороших стандартах в сфере ЭБ. 
Конечно же, существует много стандартов, так или иначе касающихся разных аспектов ЭБ, 
однако фактически нет архитектурных стандартов, которые описывали бы не только 
концептуальные части ЭБ, но и выдвигали бы требования к хранимой информации. 
Ситуация со стандартами на форматы данных немногим лучше: практически каждая 
создаваемая СУЭБ предлагает свой стандарт хранения и обмена данными, несовместимый с 
другими. Кроме того, в настоящее время во многих системах ЭБ стандарты не только не 
применяются вовсе, но и, что еще хуже, используются в ограниченном или измененном виде. 
Немаловажной проблемой видится и то, что уже существующие и вновь создаваемые ЭБ 
слабо интегрированы друг с другом: поиск и получение данных из нескольких ЭБ возможен 
только при непосредственном их посещении и повторении поискового запроса~[5].
  
  В~Вычислительном центре РАН на протяжении нескольких лет ведется исследование 
вопросов, связанных с ЭБ, и как результат этих исследований в данной статье представлена 
СУЭБ LibMeta. В~первой части статьи более подробно рассмотрены основные проблемы и 
требования к ЭБ, далее упоминаются наиболее полезные стандарты в сфере ЭБ и в третьей 
части описана СУЭБ LibMeta.

\vspace*{-12pt}
  
\section{Требования к электронной библиотеке}
  
  Современный мир предъявляет высокие требования к информационным системам: с точки 
зрения пользователей они должны быть удобны в использовании, просты в изучении; а с 
технической точки зрения должны быть тесно связаны с другими информационными 
системами и предоставлять стандартизованные службы. Все эти требования в полной мере 
относятся и к ЭБ.
  
  \subsection{Интеграция в другие информационные системы}
  
  Фактически любая современная инфор\-ма\-ци\-онная система, и в частности ЭБ, должна быть\linebreak 
интегрирована с другими информационными сис\-те\-ма\-ми. Это дает сразу несколько серьезных 
преимуществ:
  \begin{itemize}
\item отсутствие дублирования данных: исходные данные хранятся только в одной 
информационной системе, в других системах они используются по ссылкам того или иного 
вида либо реплицируются и автоматически обновляются при обновлении оригинала;\\[-14pt]
\item централизация сервисных служб, например служба аутентификации и 
авторизации пользователей: пользователи системы вводят свою идентификационную 
информацию и проходят аутентификацию лишь единожды для всей группы связанных 
информационных сис\-тем;\\[-14pt]
\item интеграция информационных ресурсов: ресурсы, даже хранящиеся в разных системах, 
представляются связанными друг с другом единой системой навигации.
\end{itemize}

  Для того чтобы снизить сложность разработки таких интегрированных информационных 
систем, имеет смысл строить их на одной общей архитектурной и технологической основе. 
Одним из таких архитектурных решений является <<Единое научное информационное 
пространство (ЕНИП) РАН>>. ЕНИП~--- это проект, реализуемый в последние несколько лет 
в рамках программы Президиума РАН <<Информатизация>>. Целью этой программы 
является создание и объединение информационных сис\-тем подразделений и научных 
институтов РАН для удовлетворения потребностей научных сотрудников как в части поиска 
качественной информации, так и в выставлении собственной информации в сети Интернет.
  
  Одними из наиболее важных составляющих ЕНИП являются так называемые схемы 
метаданных, которые представляют собой формализованные описания метаданных, 
циркулирующих в ЕНИП~[8]. Выдвинуты следующие требования к метаданным, 
описываемым схемами ЕНИП, которые должны:
  \begin{itemize}
\item  включать в себя основные типы информации, требующейся для 
поддержки работы научного сотрудника;\\[-14pt]
\item быть открытыми, т.\,е.\ обеспечивать доступ к соответствующей 
информации по этим описаниям;\\[-14pt]
\item быть расширяемыми, т.\,е.\ обеспечивать возможность детализации 
описаний;\\[-14pt]
\item обеспечивать возможности интеграции информации;\\[-14pt]
\item обеспечивать возможности уникальной идентификации информации;\\[-14pt]
\item обеспечивать возможности размещения и поиска информации в 
распределенной среде;\\[-14pt]
\item обеспечивать возможности интеропера\-бель\-ности с внешней средой.
\end{itemize}

  Кроме того, рекомендовано, чтобы схемы метаданных были ориентированы на 
семантический Веб (Semantic Web)~[2, 3].
  
  Схемы метаданных в ЕНИП являются не только обменными схемами разного уровня 
детализации, но и служат основой для построения конкретных информационных систем, 
входящих в пространство.

\vspace*{-6pt}

\subsection{Распределенность}
  
  Одним из наиболее бурно развивающихся направлений информационной индустрии 
последних лет стала разработка распределенных информационных систем. Причинами их 
быстрого роста стали достижения одновременно в нескольких областях, среди которых 
нужно особо отметить:
  \begin{itemize}
\item значительный рост пропускной способности каналов связи: скорость обмена по ним 
приближается к скоростям внутренних шин компьютеров;\\[-14pt]
\item рост производительности компьютеров как по скорости, так и по объемам памяти: и 
оперативной, и внешней;\\[-14pt]
\item широкое проникновение компьютеров и компьютерных технологий в повседневную 
дея\-тель\-ность как большинства организаций и учреждений, так и граждан;\\[-14pt]
\item развитие сети Интернет, обеспечивающей прос\-той и надежный доступ к огромному 
числу информационных ресурсов;\\[-14pt]
\item развитие самих информационных технологий; так, можно сказать, что 
программирование находится на четвертой фазе своего развития: 1)~<<классическое>> 
программирование (сначала в кодах, затем на ассемблере, и далее на языках высокого уровня) 
для больших ЭВМ; 2)~<<классическое>> программирование для персональных ЭВМ; 
3)~программирование с использование визуальных и CASE средств; 4)~<<сетевое>> 
программирование.
\end{itemize}

  Тенденция к увеличению доли распределенных систем не обошла и Российскую академию 
наук.
 Она имеет разветвленную структуру, которая объединяет большое число научно-ис\-сле\-до\-ва\-тель\-ских\linebreak\vspace*{-12pt}
\pagebreak

\noindent
учреждений и коллективов, расположенных на всей территории России и 
вовлеченных во все многообразие видов научной деятельности. Эти учреждения обладают 
уникальными научными информационными ресурсами. Среди них: опубликованные 
результаты научных исследований и экспериментов, библиографические и 
фактографические базы данных, сведения об ученых, их научной деятельности, 
публикациях, проектах и~т.\,п. Эти ресурсы представляют значительный интерес для 
сотрудников научных и административных учреждений, членов мирового научного 
сообщества, для представителей промышленности и предпринимателей, которые 
заинтересованы во внедрении результатов научных исследований.
  
  В связи с перечисленными факторами при разработке современной ЭБ повышенное 
внимание следует уделять созданию средств взаимодействия ЭБ друг с другом, а также 
средств по организации распределенных ЭБ, где данные, объем которых достаточно велик, 
чтобы их хранить в одной ЭБ, распределены между отдельными хранилищами и существуют 
специальные средства поиска и каталогизации, работающие <<над>> всеми данными та\-кой~ЭБ.

\vspace*{-6pt}
  
\subsection{Следование стандартам}
  
  В настоящее время существует огромное число стандартов на данные и метаданные. 
В~связи с этим возникает вопрос, в каком формате данные должны храниться в ЭБ и должны 
ли они в том же самом формате предоставляться конечным пользователям? Зачастую 
оказывается, что формат, удобный для хранения данных (пригодный для полнотекстового 
поиска, обеспечивающий целостность данных, легкость доступа к ним, содержащий 
минимальный набор метаинформации), не слишком удобен для пользователей, поскольку 
требует от них установки специального программного обеспечения, либо не приспособлен к 
передаче по сети, так как не обеспечивает необходимой степени сжатия. Таким образом, 
возникает необходимость поддержки как минимум двух форматов пред\-став\-ле\-ния данных 
для каждого вида ресурсов (тексты, аудио, видео, статические изображения). Также не 
следует забывать, что с течением времени существующие стандарты устаревают и 
возникают новые, что приводит к необходимости перехода на новые форматы.
  
  В связи со всем вышесказанным при разработке ЭБ, ориентированной на долговременное 
хранение, следует учесть необходимость поддержки различных форматов представления 
данных, а также преобразования между форматами, как в целях предо\-став\-ле\-ния 
пользователям ЭБ доступа к данным, так и в целях перехода на новый формат хранения 
данных в случае необходимости.

\vspace*{-6pt}

\subsection{Подготовка ресурсов}
  
  В большинстве случаев, когда ЭБ не предназначена для размещения так называемых 
<<изначально цифровых>> (born-digital) материалов, подготовка ресурсов к публикации 
является достаточно сложным и трудоемким процессом, вовлекающим многих участников. 
За простотой пользовательского интерфейса скрывается целая подсистема подготовки 
ресурсов, включающая в себя следующие службы:
  \begin{itemize}
\item оцифровки, создаваемые при библиотеках, музеях и других 
поставщиках данных там, где возможен непосредственный доступ к оцифровываемым 
материалам;
\item контроля качества оцифровки~--- единая служба контроля качества 
оциф\-ро\-вы\-ва\-емых данных, в которой задействованы специалисты по оцифровываемым 
предметам. Не\-смот\-ря на то что первичный контроль качества выполняется при самой 
оцифровке, иногда происходит утеря какой-либо части оциф\-ро\-вы\-ва\-емых данных либо 
оказывается, что с точки зрения специалиста по оциф\-ро\-вы\-ва\-емым данным оциф\-ров\-ка 
выполнена не в полном объеме;
\item подготовки метаданных. Иногда она может быть объединена со службой 
оцифровки, однако следует учитывать, что ввод метаданных должен осуществлять не 
специалист по оциф\-ров\-ке, а специалист в предметной области;
\item окончательной подготовки ресурсов и контроля качества, которая выполняет 
объединение данных и метаданных, следит за об\-нов\-ле\-ни\-ем данных и метаданных и 
осуществляет общий контроль качества подготавливаемых ресурсов.
\end{itemize}

  Перечисленные службы должны обладать своим инструментарием, обеспечивающим 
выполнение их задач и предоставляющим друг другу необходимые для работы данные. 
В~результате работы\linebreak подсистемы подготовки ресурсов получаются готовые к публикации в 
ЭБ ресурсы, которые по каналам связи автоматически представляются к пуб\-ли\-кации.

\vspace*{-6pt}

\begin{figure*} %fig1
\vspace*{1pt}
\begin{center}
\mbox{%
\epsfxsize=150.641mm
\epsfbox{zah-1.eps}
}
\end{center}
\vspace*{-12pt}
\Caption{Разделение понятий в DELOS
\label{f1ser}}
\vspace*{-6pt}
\end{figure*}

\section{Стандартизация в электронных библиотеках}
  
  Проблемой стандартов для ЭБ занимаются\linebreak многие сообщества и 
  организации~[9--13], но, несмот\-ря на значительные усилия и имеющиеся 
достижения, единства мнений добиться пока не удается. Необходимость стандартизации в 
этой об\-ласти ощущается, пожалуй, сильнее, чем в других видах информационно-поисковых 
систем, так как многие проекты ЭБ стараются объединить ресурсы нескольких 
существующих библиотечных и/или архивных систем.
  
  Стандарты, имеющие отношение к ЭБ, условно можно разделить на три большие группы: 
  \begin{itemize}
\item архитектурные стандарты, описывающие принципы и крупные компоненты, 
применяемые при построении ЭБ;\\[-14pt]
\item стандарты метаданных и их представления, описывающие наборы метаданных и, 
возможно, способы их представления при передаче и хранении;\\[-14pt]
\item стандарты информационного обмена, описывающие протоколы передачи метаданных 
и данных между системами в различных целях.
\end{itemize}

  Рассмотрим наиболее интересные и полезные, по мнению авторов, стандарты в сфере ЭБ.
  
\vspace*{-6pt}
  
\subsection{Архитектура электронных библиотек}
  
  Архитектурные стандарты описывают, как должна быть устроена ЭБ. Такие описания 
могут касаться как интерфейсов взаимодействия с другими системами и пользователями, так 
и состава и назначения компонентов системы.  Стандарты этой группы также пытаются дать 
четкое определение самому термину ЭБ и всем связанным с ним терминам.

\vspace*{-12pt}

  \subsection*{OAIS}
  
  Open Archival Information System~\cite{14ser}~--- это разрабатываемый и поддерживаемый 
организацией Consultative Committee for Space Data Systems~\cite{9ser} международный 
стандарт (ISO~14721:2003) на образцовую модель информационной системы открытого 
архива. В нем описывается подход к долговременному хранению данных, в частности 
освещается проблема устаревания форматов данных и физических носителей информации. 
Также описываются схемы взаимодействия участников архивной системы при различных 
сценариях работы. Кроме технической стороны работы открытого архива рассматривается 
также и работа ад\-ми\-ни\-ст\-ра\-тив\-но-управ\-лен\-че\-ских отделов при ЭБ, 
занимающихся планированием хранения информации, предоставлением доступа и другими 
задачами.

\vspace*{-12pt}
  
  \subsection*{DELOS}
  
  Группа DELOS~\cite{10ser} (существует под различными названиями с 1996~г.) является 
одной из наиболее известных и старых организаций, занимающихся стандартизацией в 
области ЭБ. Основным на\-прав\-ле\-ни\-ем ее деятельности является исследование и создание 
стандартов в области архитектуры ЭБ, их интеграции и отчуждения данных.
  
  Самый интересный результат работы DELOS~--- образцовая модель ЭБ Digital Library 
Reference Model (DLRM)~\cite{15ser}. В~этой модели большое внимание уделяется 
отделению цифровых объектов (содержимого ЭБ) от ЭБ и СУЭБ (рис.~\ref{f1ser}). 
Также вводятся различные классы пользователей ЭБ, решающих свои задачи в рамках ЭБ и 
которым ЭБ (и СУЭБ) должна предоставлять соответствующую функциональность, 
приводится весьма обширная концептуальная модель данной области с тщательными 
определениями важнейших представлений об архитектуре, ресурсах и функциональности 
ЭБ, в частности в UML-представлении.



\vspace*{-6pt}

\subsection{Представление метаданных}
  
  Стандарты метаданных играют в ЭБ очень важную роль: они описывают, в каком формате 
данные представляются для хранения и передачи из одной информационной системы в 
другую. Следует заметить, что не все стандарты метаданных в дополнение к самому 
описанию сущностей, их связей и атрибутов представляют описание контейнера для 
метаданных, однако большая часть стандартов подразумевает использование XML как 
одного из видов такого контейнера. 
  

\vspace*{-12pt}

  \subsection*{Dublin Core}
  
  Стандарт Dublin Core~\cite{11ser} состоит из двух час\-тей: минимальной Dublin Core 
Metadata Element Set, имеющей статус международного стандарта (ISO~15836:2009), и 
полной~--- DCMI Metadata Terms. Обе версии стандарта не привязаны ни к какой конкретной 
предметной области и могут описывать ресурсы любых видов. Минимальная версия 
содержит только 15~атрибутов, применяемых к любым сущностям, в которых можно задать 
название объекта, его описание, автора, административные метаданные и ссылку на 
источник. Полная версия содержит все атрибуты минимальной, а также наборы 
дополнительных сущностей, словарей и дополнительное множество атрибутов. Чаще всего 
используется именно минимальный набор, поскольку именно он позволяет максимально 
абстрагироваться от предметной области, но при этом, конечно же, проигрывая в деталях.
  
  Также существует рекомендация DCMI Abstract Model, содержащая сведения по 
расширению \mbox{DCMI} Metadata Terms для нужд различных предметных областей.
  
  Кроме описания схемы метаданных стандарты Dublin Core содержат описания 
контейнеров, в частности описан формат DC-Text для хранения в чисто текстовом формате, 
формат DC-XML для XML-представления и DC-RDF для представления в 
  RDF-подмножестве XML.
  
\vspace*{-12pt}

  \subsection*{CIDOC-CRM}
  
  С 1994~г.\ в составе International Council of Museums существует комитет Committee on 
Documentation of the International Council of Museums (CIDOC)~\cite{12ser}, который 
занимается стандартизацией в области музейных метаданных. Основной \mbox{целью} данного 
комитета является создание образцовой концептуальной модели (CRM) для описания 
сущностей и связей, используемых в документировании культурного наследия. В~1999~г.\ 
была\linebreak выпущена первая версия CIDOC-CRM~\cite{16ser}, а в 2006~г.\ модель получила 
статус международного стандарта (ISO~21127:2006).
  
  Модель CIDOC-CRM чрезвычайно подробно описывает предметную область, и поэтому 
на практике в полном объеме практически никогда не применяется. Тем не менее 
отображение метаданных ЭБ на эту модель представляется полезным, так как многие 
музейные системы заявляют свою со\-вмес\-ти\-мость с CIDOC-CRM.
  
\vspace*{-12pt}

  \subsection*{FRBR и FRBRoo}
  
  Изначально разрабатывавшийся The International Federation of Library Associations and 
Institutions стандарт Functional Requirements for Bibliographic Records (FRBR)~\cite{17ser} 
имеет примерно такую же роль в области описания публикаций, какую имеет CIDOC-CRM в 
описаниях предметов культурного наследия. С~2000~г.\ при содействии комитета CIDOC 
разрабатывается объект\-но-ори\-ен\-ти\-ро\-ван\-ный стандарт FRBRoo~\cite{18ser}, заимствующий 
часть сущностей и идей из CIDOC-CRM.
  
\vspace*{-12pt}

  \subsection*{PRISM}
  
  Стандарт Publishing Requirements for Industry Standard Metadata (PRISM)~\cite{13ser} 
разработан издательскими организациями, входящими в IDEAlliance (International Digital 
Enterprise Alliance), для обмена метаданными о публикациях. PRISM основан на DCMI, но в 
большей степени ориентирован на библиографические ресурсы. В~стандарте предлагается 
среда обмена и хранения данных и метаданных и ряд словарей значений этих элементов. 
В~ЕНИП используется схема контролируемых словарей PRISM и основной набор элементов 
PRISM.
  
\vspace*{-12pt}

  \subsection*{MARC}
  
  Стандарт MARC (MAchine-Readable Cataloging) по праву может считаться одним из 
самых старых стандартов в информационных технологиях. Он был разработан в начале 
  1960-х~гг.\ в библиотеке конгресса США. Основное предназначение стандарта~--- 
хранение библиографических записей для электронных каталогов. В настоящее время 
оригинальный MARC (USMARC) уже не используется, на смену ему пришли модификации 
MARC~21~\cite{19ser} (США и Канада) и национальные модификации (например, 
RUSMARC~\cite{20ser} в России). Также существует отображение MARC на XML, 
называемое MARCXML~\cite{21ser}. Хотя MARC основан на устаревших технологиях, он 
достаточно широко используется и многие современные информационные сис\-те\-мы заявляют 
свою с ним совместимость.

\vspace*{-6pt}

\subsection{Информационный обмен}
  
  Стандарты информационного обмена пред\-на\-зна\-че\-ны в первую очередь для открытых и 
рас\-пределен\-ных систем. Руководствуясь такими стандартами, можно получать доступ к 
данным и метадан\-ным ЭБ, ничего не зная об их устройстве,\linebreak\vspace*{-12pt}
\pagebreak

\noindent
 и, в свою очередь, предоставлять 
другим системам доступ к собственным данным и метаданным, не раскрывая деталей 
реализации и способов хранения.
  
\vspace*{-12pt}

  \subsection*{OAI-PMH}
  
  Стандарт Open Archives Initiative~--- Protocol for Metadata Harvesting~\cite{22ser} является 
признанным лидером среди стандартов распределенного поиска и репликации метаданных. 
Основу такой популярности обеспечило то, что он достаточно прост в реализации и может 
инкапсулировать метаданные в любом XML-формате. Каждая OAI-PMH со\-вмес\-ти\-мая 
система должна поддерживать метаданные в формате Dublin Core.
  
\vspace*{-12pt}

  \subsection*{OAI-ORE}
  
  Object Reuse and Exchange~\cite{23ser}~--- Semantic Web-ориентированный стандарт 
описания и обмена агрегированными ресурсами. Стандарт описывает способы объединения 
частей объекта (данных, метаданных) в один агрегированный ресурс, доступный по единому 
URL, а также способы обработки и предоставления частей объекта по такому URL.

\vspace*{-6pt}

\section{Система управления электронными библиотеками LibMeta}
  
  C 2007~г.\ в ВЦ РАН ведутся работы по созданию СУЭБ в рамках ЕНИП~\cite{24ser} под 
названием LibMeta, которая позволила бы библиотекам, архивам и музеям РАН иметь 
унифицированное решение, позволяющее публиковать полные тексты научных работ и 
разнообразные мультимедийные материалы, быть интегрированной в существующие 
информационные сис\-те\-мы РАН, а также соответствовать стандартам в области ЭБ.
  
  Портал ЭБ <<Научное наследие России>>~\cite{25ser} является первой установкой СУЭБ 
LibMeta, а также площадкой для обкатки технологических и архитектурных решений.


\vspace*{-6pt}

\subsection{Архитектура СУЭБ LibMeta}
  
  Поскольку предполагается использование \mbox{СУЭБ} LibMeta в научных институтах различной 
направленности, невозможно предоставить од\-ну-един\-ст\-вен\-ную схему метаданных, 
подходящую абсолютно под все задачи. Данная проблема может быть решена двумя 
способами: внесением избыточности в схему метаданных или предоставлением 
администратору системы возможности доопределять схему.
  
  Оба подхода не лишены своих недостатков. Так избыточная схема приводит к тому, что в 
каж\-дой конкретной установке ЭБ используется только часть схемы, как правило достаточно 
небольшая. Это приводит к нерациональному использованию ресурсов и уменьшению 
быстродействия. Кроме того, внесение такой избыточности требует глубоких исследований в 
предметной области, а также наличия общепринятых стандартов на подобные метаданные. 
  
  Основным недостатком второго подхода является то, что по метаданным, определенным 
администратором, поиск может осуществляться только с большими затратами ресурсов. 
Кроме того, сужа\-ют\-ся возможности по связыванию ресурсов друг с другом, так как 
администратор может определять только содержательные атрибуты.
  
  В СУЭБ LibMeta применяются оба подхода: для ресурсов типа <<\textbf{публикация}>> 
и <<\textbf{персона}>> применяется избыточная схема, разработанная на основе 
библиотечных стандартов, а для <<\textbf{музейных предметов}>> используется второй 
подход. Такое решение основано на том, что для музейных предметов практически 
невозможно создать единую (избыточную) схему, поскольку каждый музей обладает своей 
спецификой. Естественно, что для музейных предметов существует и фиксированная часть 
схемы, позволяющая связать их с другими ресурсами и включающая основные 
содержательные метаданные.

  \begin{figure*}[b] %fig2
  \vspace*{9pt}
\begin{center}
\mbox{%
\epsfxsize=100mm
\epsfbox{zah-2.eps}
}
\end{center}
\vspace*{-6pt}
  \Caption{Система <<Научное наследие России>>
  \label{f2ser}}
  \end{figure*}
  
  При разработке СУЭБ LibMeta были проанализированы мировые стандарты в области ЭБ, 
а также обобщен опыт их использования в других библиотечных системах. На основе этого 
анализа были выдвинуты предложения по использованию стандартов.
  
  В качестве архитектурных стандартов были использованы как DELOS~\cite{15ser}, так и 
OAIS~\cite{14ser}. Из DELOS взяты общие концепции и варианты использования ЭБ. Из 
OAIS почерпнуты сведения о процессах, происходящих внутри ЭБ, а также о ее 
взаимодействии с внешними системами.

  \begin{figure*}[b] %fig3
  \vspace*{6pt}
\begin{center}
\mbox{%
\epsfxsize=92.219mm
\epsfbox{zah-3.eps}
}
\end{center}
\vspace*{-6pt}
  \Caption{Профили метаданных в СУЭБ LibMeta
  \label{f3ser}}
  \end{figure*}
  
  Из стандартов метаданных были использованы Dublin Core (unqualified)~\cite{11ser}, 
CIDOC-CRM~\cite{16ser}, RUSMARC~\cite{20ser} и~др. На эти стандарты существует 
отображение схемы метаданных СУЭБ LibMeta и возможен информационный обмен по 
протоколам, поддерживающим инкапсуляцию данных в этих форматах. Перечисленные 
стандарты являются наиболее распространенными в мире ЭБ, и их использование позволяет 
достичь максимальной совместимости с существующими системами.
  
  В качестве основного стандарта информационного обмена выбран OAI-PMH~\cite{22ser}, 
поскольку он поддерживается большей частью библиотечных и архивных систем.
  
  Основой для обмена данными и семантической интероперабельности в ЕНИП служат 
технологии Semantic Web. Соответственно, оправданным представляется применение в 
ЕНИП существующих предложений по стандартизации наборов элементов метаданных для 
Semantic Web~\cite{2ser, 3ser, 8ser}. В технологиях Semantic Web широко используется язык 
RDF, а также его специализация для описания онтологий OWL. Логично было выбрать 
именно OWL как язык описания метаданных в СУЭБ LibMeta, а RDF~--- как язык обмена 
метаданными между системами.
  
  В качестве инфраструктурного решения для реализации СУЭБ LibMeta была выбрана 
прог\-рам\-мная платформа, разработанная ВЦ РАН,~--- система <<Научный 
институт>>~\cite{26ser}.
  
  <<Научный институт>> представляет собой типовой программный комплекс 
автоматизации информационной деятельности научного института\linebreak в составе Российской 
академии наук, поддержки научной деятельности его сотрудников, взаимодействия с 
другими информационными сис\-те\-ма\-ми в составе ЕНИП. <<Научный институт>>\linebreak изначально 
разрабатывается как модульная, расширяемая система, позволяющая гибко подбирать 
наиболее подходящий набор функциональных возможностей для каждой конкретной 
организации. Основу системы со\-став\-ля\-ет ядро (<<платформа>>)~--- программное решение, 
предназначенное для создания распределенных информационных систем, веб-порталов, 
интеграции данных, и набор функциональных модулей, предоставляющих специальные 
функции.
  Для СУЭБ LibMeta <<Научный институт>> является слоем абстракции от подсистемы 
хранения и контейнером приложений. Кроме того, в СУЭБ LibMeta используется 
библиографический профиль метаданных ЕНИП, предоставляющий большую часть схемы 
метаданных в части определения таких ресурсов, как публикации, персоны, проекты и 
организации. В состав системы <<Научный институт>> входит также большое число 
подключаемых модулей, которые могут быть легко включены в установки СУЭБ LibMeta, к 
примеру для организации форума. 
  
\vspace*{-6pt}

\subsection{Электронная библиотека <<Научное наследие России>>}
  
Электронная библиотека <<Научное наследие России>> (рис.~\ref{f2ser}) разрабатывается в рамках 
одноименной программы Президиума РАН с целью обеспечения сохранности и 
предоставления пуб\-лич\-но\-го доступа к научным трудам известных российских и зарубежных 
ученых и исследователей, работавших на территории России. Некоторые из подсистем 
ЭБ (системы хранения и представления электронных изданий 
конечным пользователям) создаются в рамках программы Президиума РАН\linebreak 
<<Информатизация>>. Общая координация и управ\-ле\-ние проектом осуществляется 
Межведомственным суперкомпьютерным центром (МСЦ) РАН.\linebreak Задачами подготовки 
электронных изданий и сопровождающей информации для размещения в хранилище данных 
электронной библиотеки занимаются ведущие библиотеки РАН, среди которых БАН, БЕН 
(Центральная библиотека и ее отделения), \mbox{ИНИОН}.

  
    
  Другой важной задачей этого проекта является интеграция существующих библиотечных 
ресурсов в ЕНИП РАН и обеспечение возможности централизованного доступа к ресурсам 
существующих хранилищ электронных изданий и метаданных об ученых и их научных 
трудах. Данная задача решается путем определения единой инфраструктуры распределенной 
системы, унификации форматов данных и протоколов взаимодействия компонентов 
системы, разработки единых регламентов подготовки и сопровождения электронных 
изданий.
  
\vspace*{-6pt}

  \subsection*{Профиль метаданных СУЭБ LibMeta}
  
  Одним из самых существенных недостатков многих схем метаданных ЭБ является то, что 
они рассматривают метаданные только в контексте описываемых ими данных и только как 
набор полей для поиска информации и индексирования ресурсов. В случае, когда ЭБ 
содержит один вид ресурсов, например только книги, такой подход, возможно, оправдан, 
однако современные ЭБ содержат разнообразные ресурсы, и в связи с этим такой подход 
неприемлем. Метаданные разных ресурсов должны содержать ссылки друг на друга, при 
этом оставаясь достаточно независимыми. Одним из наиболее удобных подходов к 
описанию такого рода метаданных является использование OWL-онтологий. Основной 
частью профиля метаданных в ЕНИП и <<Научном институте>> являются как раз такие 
онтологии~\cite{8ser}. Общая схема профилей метаданных, применяемых в СУЭБ LibMeta, а 
также основных сущностей в этих профилях приведена на рис.~\ref{f3ser}.
  
\vspace*{-6pt}
  
  \subsection*{Метаданные публикации}
  
  Базовый уровень \textbf{публикации} включает сле\-ду\-ющие свойства (здесь и далее 
курсивом выделены свойства, являющиеся ссылками на сущности LibMeta, по которым 
возможна навигация):
  \begin{itemize}
\item название;
\item альтернативный заголовок;
\item аннотация;
\item ключевые слова;
\item \textit{источник}~--- описание источника информации о данном ресурсе, например: наименование 
организации, ФИО и пр.;
\item авторские права;
\item веб-адрес;
\item полный текст;
\item язык;
\item дата издания;
\item идентификатор~--- указание идентификатора ресурса с помощью рекомендуемых стандартных 
систем идентификации;
\item \textit{авторы};
\item \textit{издатель};
\item \textit{редактор};
\item \textit{входит в состав}~--- данный ресурс является физически или логически частью указанного 
ресурса;
\item \textit{включает}~--- данный ресурс физически или логически включает указанный ресурс;
\item количество страниц;
\item реферат~--- реферат(ы) по данной публикации;
\item библиографическое описание~--- библио\-гра\-фическое описание публикации по ГОСТу\linebreak целиком, 
строкой. Может быть указано помимо отдель\-ных элементов описания, указыва\-емых полями 
<<название>>, <<номер тома/выпуска>> и~пр.;
\item \textit{полный код УДК};
\item \textit{рубрика ББК};
\item \textit{основной код УДК};
\item примечания.
\end{itemize}

  При использовании публикаций в на\-уч\-но-ис\-сле\-до\-ва\-тель\-ском процессе 
существует необходимость быстрого ознакомления с содержимым\linebreak публикации, и аннотации 
часто оказывается недостаточно. В~связи с этим в инструментарии СУЭБ LibMeta 
разработаны средства полуавтоматического выделения оглавления с обеспечением ссылок на 
соответствующие разделы документа, а также средства работы с библиографическими 
ссылками. 
  
  Приведем описание фрагмента профиля ЭБ, отражающего решение этих задач. 
Расширенная схема описания библиографической информации включает: 
  \begin{itemize}
\item список литературы (текстом)~--- список библиографических ссылок в текстовом виде, если не 
может быть разобран;
\item оглавление~--- оглавление данной публикации в виде отдельного файла, либо текстового или 
XML-фрагмента;
\item список литературы~--- список библиографических ссылок, указанных в тексте данной публикации, 
в виде списка структур <<Библиографическая ссылка>>:
\begin{itemize}
\item порядковый номер;
\item идентификатор ссылки~--- краткий идентификатор библиографической ссылки, например <<DC>> 
или <<12>>;
\item текст ссылки~--- исходный текст библиографической ссылки, желательно отформатированный как 
библиографическое описание по ГОСТу;
\item \textit{цитируемая публикация}; 
\end{itemize}
\item сведения об издании~--- сведения, относящиеся к изданию: в какой редакции, данные об оригинале 
для переводной литературы, место и год издания;
\item \textit{составитель};
\item \textit{коллективный автор публикаций};
\item \textit{переводчик публикаций};
\item \textit{редколлегия}.
\end{itemize}

\vspace*{-6pt}

  \subsection*{Метаданные автора}
  
  В приведенных описаниях элементов профиля ЭБ можно заметить использование 
элементов основного профиля ЕНИП: <<\textbf{персона}>>, <<\textbf{организационная 
единица}>>, <<\textbf{файл данных}>> и~других. Приведем состав наиболее часто 
используемого класса~--- <<\textbf{персоны}>>:
  \begin{itemize}
\item домашняя страница;\\[-14pt]
%\pagebreak
\item дата рождения;\\[-14pt]
\item адрес~--- полный почтовый адрес;\\[-14pt]
\item имя:\\[-14pt] 
\begin{itemize}
\item фамилия;\\[-14pt]
\item имя;\\[-14pt]
\item отчество;\\[-14pt]
\item значение~--- фамилия, имя отчество полностью, с дополнительными элементами (титулами и~пр.);\\[-14pt]
\end{itemize}
\item пол;\\[-14pt]
\item ученая степень:\\[-14pt]
\begin{itemize}
\item дата присуждения;\\[-14pt]
\item ученая степень;\\[-14pt]
\item специальность ВАК;\\[-14pt]
\end{itemize}
\item ученое звание:\\[-14pt]
\begin{itemize}
\item дата присуждения;\\[-14pt]
\item ученое звание;\\[-14pt]
\item \textit{присудившая организация};\\[-14pt]
\end{itemize}
\item дата смерти;\\[-14pt]
\item место рождения~--- место рождения данной личности (указывается в произвольной форме). Ввиду 
сложности поддержки исторической информации об ад\-ми\-ни\-стра\-тив\-но-тер\-ри\-то\-ри\-аль\-ном 
делении, классификатор регионов не используется для указания места рождения; \\[-14pt]
\item место смерти~--- место смерти данной исторической личности (указывается в произвольной 
форме, как и место рождения);\\[-14pt]
\item электронная почта;\\[-14pt]
\item телефон;\\[-14pt]
\item факс;\\[-14pt]
\item www-страница;\\[-14pt]
\item FTP-адрес.
\end{itemize}

\vspace*{-6pt}
  
  \subsection*{Метаданные предмета}
  
  Сближение задач ЭБ, архивов и музеев в представлении научного наследия выдвигает 
требование стандартизации метаданных физических музейных предметов и их 
мультимедийных (фото, видео, аудио) представлений. В связи с этим в СУЭБ LibMeta 
разработан дополнительный прикладной профиль, в котором для сущности 
<<\textbf{музейный предмет}>> определены следующие свойства и связи:
  \begin{itemize}
\item название;\\[-14pt]
\item альтернативный заголовок;\\[-14pt]
\item аннотация;\\[-14pt] 
\item ключевые слова;\\[-14pt]
\item источник;\\[-14pt]
\item \textit{держатель} (место хранения);\\[-14pt]
\item \textit{состав}~--- набор ссылок на другие предметы, из которых состоит описываемый;\\[-14pt]
\item \textit{автор описания};\\[-14pt]
\item состояние (сохранность)~--- состояние предмета по шкале определяемой его держателем;
\item количество предметов;\\[-14pt]
\item номер~--- идентифицирующие номера предмета:
\begin{itemize}
\item система нумерации;\\[-14pt]
\item номер;\\[-14pt]
\end{itemize}
\item \textit{автор сбора}~--- персона, впервые собравшая предмет;\\[-14pt]
\item дата сбора;\\[-14pt]
\item дата поступления~--- дата поступления предмета к хранителю;\\[-14pt]
\item география~--- место находки данного предмета (указывается в произвольной форме); ввиду %\linebreak\vspace*{-12pt}
%\columnbreak
%\noindent 
сложности поддержки исторической информации об ад\-ми\-ни\-ст\-ра\-тив\-но-тер\-ри\-то\-ри\-аль\-ном 
делении, классификатор регионов не использу\-ется;\\[-14pt]
\item размеры;\\[-14pt]
\item возраст~--- предполагаемая дата создания предмета:\\[-14pt]
\begin{itemize}
\item эра~--- описание эры с указанием в свободной форме ее начала и конца, например <<наша эра>> 
ведет отсчет от Рождества Христова;\\[-14pt]
\item время от начала~--- время от начала эры до наступления события. Формат зависит от эры, например 
для нашей эры форматом будет дата по григорианскому календарю;\\[-14pt]
\end{itemize}
\item способ поступления.
\end{itemize}

  В отличие от публикаций, описания музейных объектов могут сильно различаться в 
разных музеях, и здесь невозможно обеспечить всеобъемлющий набор необходимых 
свойств. В~связи с этим для данных объектов реализуется возможность определения 
дополнительных свойств в виде связей с двумя вспомогательными объектами: 
<<\textbf{дополнительные свойства}>> и <<\textbf{значения дополнительных 
свойств}>>. При этом в интерфейсе администратора системы предоставляется возможность 
определять дополнительные свойства предмета, а в интерфейсах ввода и вывода данных 
создаются представления соответствующих полей. Введенные значения дополнительных 
полей выдаются в полных сведениях о предмете, но поиск по ним не производится. В~целях 
унификации подобные наборы дополнительных свойств могут быть приданы не только 
музейным предметам, но и любым другим видам ресурсов.

\vspace*{-6pt}
  
  \subsection*{Мультимедийные представления}
  
  Для обеспечения хранения цифровых пред\-став\-ле\-ний ресурсов и абстрагирования от 
методов хранения данных в СУЭБ LibMeta разработан дополнительный прикладной профиль 
<<Расширенной поддержки хранения данных>>, в котором вводится ряд новых сущностей.
  
  Сущность <<\textbf{медиа-объект}>> предназначена для описания медиа-объекта как 
единого целого, состоящего из частей данных с различной функциональной нагрузкой. 
<<\textbf{Медиа-объект}>> включает в себя единственное свойство:
  \begin{itemize}
  \item[$\bullet$] части.
  \end{itemize}
  
  Сущность <<\textbf{часть медиа-объекта}>> позволяет в пределах одного целого 
  медиа-объекта, например публикации, иметь несколько частей с различной\linebreak\vspace*{-12pt}
%  \pagebreak
  
  \noindent
   функциональной 
нагрузкой, таких как содержание, образы фотографий, текстов или страниц в виде 
изображений, тексты в чисто текстовом (распознанном) формате, отформатированный текст 
пуб\-ли\-ка\-ции и~т.\,п. Свойствами <<\textbf{части медиа-объекта}>> являются:
  \begin{itemize}
  \item[$\bullet$] функциональный тип~--- показывает, какую функциональную нагрузку 
несет часть, например: <<содержание>>, <<страница книги>>;
  \item[$\bullet$] потоки данных в формате, соответствующем типу данных;
  \item[$\bullet$] порядок в медиа-объекте;
  \item[$\bullet$] название части.
\end{itemize}

  Сущность <<\textbf{единица хранения}>> представляет единый и неделимый поток 
двоичных данных. Позволяет абстрагироваться от конкретных методов хранения данных и 
собирать медиа-объекты, состоящие из частей, расположенных в разных местах и хранимых 
различными способами. Содержит следующие свойства:
  \begin{itemize}
\item тип данных~--- формат представления данных, хранимых в данной части, например <<Документ 
Microsoft Word>> или <<Изображение в формате JPEG>>; связывается с MIME-типами посредством 
словаря ЕНИП \textbf{IMT}~\cite{8ser};\\[-8pt]
\item тег~--- может содержать данные о кэшировании, преобразованном формате и~т.\,д.
\end{itemize}

  Введено три вида единиц хранения: ссылки на внешние источники, файлы в файловой 
системе и BLOB-записи в базе данных.
  
  Принцип использования представленного выше <<\textbf{медиа-объекта}>> в СУЭБ 
LibMeta несколько отличается от общепринятого в ЭБ. Для обеспечения цифровых 
представлений не только публикаций, но и музейных объектов, а также мультимедийных 
изображений коллекций, фотографий персон, коллективов, зданий организаций и~т.\,п.\ в 
сущность <<\textbf{ресурс}>>, являющуюся суперклассом для всех основных объектов 
онтологии, вводится свойство <<\textbf{ме\-диа-пред\-став\-ле\-ния}>>. Таким образом, одно или 
несколько мультимедийных представлений может сопровождать любой ресурс 
информационной веб-системы.

\vspace*{-6pt}
  
  \subsection*{Коллекции ресурсов}
  
  В базовых метаданных ЕНИП предусмотрена поддержка коллекций, однако требования 
ЭБ, в особенности с поддержкой хранения музейных предметов, не позволяют их 
полноценно использовать. В связи с этим базовый профиль ЕНИП дополнен коллекциями со 
следующими атрибутами:

\noindent
  \begin{itemize}
\item название; 
\item тип коллекции;
\item ключевые слова;
\item описание;\
\item \textit{администратор};
\item количество элементов в коллекции;
\item \textit{место хранения};
\item примечание;
\item \textit{элементы коллекции}.
\end{itemize}

  Коллекции такого рода позволяют хранить классические коллекции (архивные, музейные) 
и иметь любые вложенные наборы объектов (выставочные, выездные, по хранению и~пр.).

%\vspace*{-6pt}

\subsection{Интеграция СУЭБ LibMeta с другими электронными библиотеками}
  
  Интеграция СУЭБ LibMeta с другими информационными системами осуществляется 
несколькими путями: во-первых, это интеграция с ЕНИП, во-вторых, это интеграция с 
системами подготовки публикаций и, в-третьих, интеграция с универсальными агрегаторами.

\vspace*{-6pt}
  
  \subsection*{Интеграция с системами подготовки публикаций}
  
  Для обеспечения интеграции с системами подготовки публикаций в СУЭБ LibMeta 
существует компонент загрузки и обновления данных и метаданных из внешних источников. 
Данный компонент ставит в соответствие каждому загружаемому в библиотеку извне 
ресурсу <<состояние размещения>>, в зависимости от которого с ресурсом производятся 
действия по размещению и его периодическому обновлению. Когда ресурс уже размещен, 
также возможно его обновление по запросу от обнов\-ля\-ющей (размещающей) стороны.
  
  Описанный компонент настроен для работы с системой подготовки публикаций, 
применяемой в ЭБ <<Научое наследие России>>, однако ее архитектура рассчитана на 
простое добавление модулей для любых других систем подготовки публикаций. В~плане 
развития СУЭБ LibMeta стоит разработка подсистемы подготовки публикаций начального 
уровня, которая, естественно, будет работать с \mbox{СУЭБ} LibMeta.

%\vspace*{-6pt}
  
  \subsection*{Интеграция с универсальными агрегаторами}
  
  Для интеграции с универсальными агрегаторами в СУЭБ LibMeta полностью реализованы 
стандарты OAI-PMH~\cite{22ser} и Dublin Core~\cite{11ser}. Кроме того, для интеграции с 
музейными системами существует отображение метаданных системы на концептуальную 
модель CIDOC-CRM~\cite{16ser}. Для использования других протоколов и форматов обмена 
данными и метаданными, в особенности основанных на XML, не представляется сложной 
реализация модулей обмена данными.

\vspace*{-6pt}

\section{Заключение}
  
  В настоящее время идет становление ЭБ как в России, так и в мире. До сих пор не 
существует всеобъемлющего стандарта на ЭБ, либо реализации ЭБ, удовлетворяющей всем 
существующим мировым стандартам. В~данной статье рассмотрены основные требования, 
выдвигаемые к современным ЭБ. Также приведен обзор наиболее распространенных 
стандартов, на которые следует ориентироваться при разработке ЭБ, в частности в среде 
ЕНИП.
  
  Создаваемая в ВЦ РАН СУЭБ LibMeta наиболее полно удовлетворяет приведенным 
требованиям и позволяет научным институтам РАН, имеющим собственные библиотечные, 
архивные или музейные фонды, создавать свои ЭБ, интегрированные с ЕНИП, и легко 
выставлять данные ресурсы в Интернет как для научного, так и для широкого круга 
пользователей. На текущий момент работы по СУЭБ LibMeta практически завершены.

\vspace*{-6pt}

{\small\frenchspacing
{%\baselineskip=10.8pt
\addcontentsline{toc}{section}{Литература}
\begin{thebibliography}{99}

\bibitem{1ser}
\Au{Gantz J., Chute C., Manfrediz A., \textit{et al}.}
Доклад IDC при финансовой поддержке компанией EMC: Обновленный прогноз роста 
мирового объема информации до 2011~г.

\bibitem{2ser}
\Au{Berners-Lee T., Hendler J., Lassila O.}
The semantic Web~// Scientific Am., 2001. No.\,5. P.~34--43.

\bibitem{3ser}
\Au{Berners-Lee T., Shadbolt N., Hall W.}
The semantic Web revisited~// IEEE Intelligent Systems, 2006. No.\,6.

\bibitem{5ser}
\Au{Зацман И.\,М.}
Концептуальный поиск и качество информации.~--- М.: Наука, 2003.

\bibitem{4ser}
\Au{Галева И.\,С.}
Интернет как инструмент библиографического поиска.~--- М.: Профессия, 2007.

\bibitem{6ser}
\Au{Kahn R., Cerf V.}
The digital library project. Vol.~I: The world of knowbots (DRAFT): An open architecture for a 
digital library system and a plan for its development.~--- Reston, VA: Corporation for National 
Research Initiatives, 1988.

\bibitem{7ser}
\Au{Земсков А.\,И., Шрайберг Я.\,Л.}
Электронная информация и электронные ресурсы: публикации и документы, фонды и 
библиотеки.~--- М.: ФАИР, 2007.

\bibitem{8ser}
\Au{Бездушный А.\,Н., Бездушный А.\,А., Серебряков~В.\,А., Филиппов~В.\,И.}
Интеграция метаданных Единого научного информационного пространства РАН.~--- М.: 
ВЦ РАН, 2006.

\bibitem{9ser}
Consultative committee for space data systems. {\sf http://public.ccsds.org/default.aspx}.

\bibitem{10ser}
DELOS an Association for Digital Libraries. {\sf http://\linebreak  www.delos.info}.

\bibitem{11ser}
Dublin core. {\sf http://dublincore.org}.

\bibitem{12ser}
CIDOC CRM Home page. {\sf http://cidoc.ics.forth.gr/\linebreak index.html}.

\bibitem{13ser}
Publishing requirements for industry standard metadata. {\sf http://prismstandard.org}.

\bibitem{14ser}
CCSDS Secretariat, Program Integration Division (Code M-3). Reference model for an open 
archival information system (OAIS): Recommendation for space data system standards Blue 
book.~--- National Aeronautics and Space Administration, 2002.

\bibitem{15ser}
\Au{Candela L., Castelli D., Ferro N., \textit{et al}.}
The DELOS Digital Library Reference Model~--- foundations for digital libraries. 
Version~0.98.~--- GEIE ERCIM, 2008.

\bibitem{16ser}
\Au{Crofts N., Doerr M., Gill T., Stead S., Stiff M.}
Definition of the CIDOC Conceptual Reference Model, 2010.

\bibitem{17ser}
Functional requirements for bibliographic records. {\sf http://www.ifla.org/VII/s13/frbr/frbr.htm}.

\bibitem{18ser}
Functional requirements for bibliographic records object-oriented definition and mapping to 
FRBR\_ER. {\sf http://cidoc.ics.forth.gr/docs/frbr\_oo/frbr\_docs/\linebreak FRBR\_oo\_V0.9.pdf}.

\bibitem{19ser}
Network Development and MARC Standards Office MARC~21 concise formats.~--- Cataloging 
Distribution Service, Library of Congress, 2006.

\bibitem{20ser}
Система форматов RUSMARC. {\sf http://www.rba.ru/\linebreak rusmarc}.

\bibitem{21ser}
MARC 21 XML schema. {\sf http://www.loc.gov/\linebreak standards/marcxml}.

\bibitem{22ser}
Open archives initiative protocol for metadata harvesting. {\sf http://www.openarchives.org/pmh}.

\bibitem{23ser}
Open archives initiative object reuse and exchange. {\sf http://www.openarchives.org/ore}.

\bibitem{24ser}
\Au{Захаров А.\,А., Филиппов В.\,И.}
Поддержка цифровых библиотек и музейных объектов в среде ЕНИП~// Электронные 
библиотеки: перспективные методы и технологии, электронные коллекции: Тр.\ XI 
Всероссийской научной конф. RCDL'2009.~--- Петрозаводск: КарНЦ РАН, 2009.~--- 487~с.

\bibitem{25ser}
Портал ЭБ <<Научное наследие России>>. {\sf http://\linebreak e-heritage.ru}.

 \label{end\stat}

\bibitem{26ser}
\Au{Бездушный А.\,Н., Бездушный А.\,А., Нестеренко А.\,К. и~др.}
Информационная Web-система <<Научный институт на платформе ЕНИП>>.~--- М.: ВЦ 
РАН, 2007.
 \end{thebibliography}
}
}


\end{multicols}