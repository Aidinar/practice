\def\stat{koles}

\def\tit{АЛГОРИТМ КООРДИНАЦИИ ДЛЯ~ГИБРИДНОЙ ИНТЕЛЛЕКТУАЛЬНОЙ 
СИСТЕМЫ РЕШЕНИЯ СЛОЖНОЙ ЗАДАЧИ 
ОПЕРАТИВНО-ПРОИЗВОДСТВЕННОГО ПЛАНИРОВАНИЯ}

\def\titkol{Алгоритм координации для~ГиИС решения сложной задачи оперативно-производственного планирования} 



\def\autkol{А.\,В.~Колесников, С.\,А.~Солдатов}
\def\aut{А.\,В.~Колесников$^1$, С.\,А.~Солдатов$^2$}

\titel{\tit}{\aut}{\autkol}{\titkol}

%{\renewcommand{\thefootnote}{\fnsymbol{footnote}}\footnotetext[1]
%{Работа выполнена при финансовой поддержке РФФИ (грант 08-01-00567).}}

\renewcommand{\thefootnote}{\arabic{footnote}}
\footnotetext[1]{Калининградский филиал Института проблем информатики РАН, avkolesnikov@yandex.ru}
\footnotetext[2]{ООО <<Лайтон>>, Москва, ssa@west-automatica.com}


\Abst{Рассмотрена задача оперативно-производственного планирования на 
машиностроительном предприятии с заказным, мелкосерийным характером производства и 
описан подход к решению подобных задач на основе методологии функциональных 
гибридных интеллектуальных систем (ГиИС) с координацией.}

\KW{машиностроительное производство; координация; задача оперативного планирования; 
гибридная интеллектуальная система}

       \vskip 14pt plus 9pt minus 6pt

      \thispagestyle{headings}

      \begin{multicols}{2}

      \label{st\stat}

\section{Введение}

  Задачи оперативно-производственного планирования рассматриваются в 
работах многих отечественных и зарубежных ученых: С.\,Н.~Петракова, 
Г.\,Н.~Кальянова, Е.\,В.~Фрейдиной, В.\,П.~Заболотского, Ю.\,Е.~Звягинцева, 
Н.\,С.~Сачко, Д.~Тейлора и~др. Но, несмотря на имеющееся разнообразие 
научных методов и инженерного инструментария~[1--7], планирование 
производства по-прежнему остается плохо изученным объектом.
  
  Как показал анализ, выработанное в исследовании операций, искусственного 
интеллекта, теории принятия решения и системном анализе представление о 
задаче для индустриального общества устарело. В~известных методах и 
моделях отсутствует либо имеет ограниченную область применения важный 
для решения задачи опе\-ра\-тив\-но-про\-из\-вод\-ст\-вен\-но\-го планирования 
механизм взаимодействия (координации) подзадач в ходе решения сложной 
задачи~[8].
  
  Опыт применения функциональных интеллектуальных гибридных 
систем~[8] в планировании также показал нерелевантность применяемых 
представлений о задаче уровню сложности явлений и процессов в системах 
поддержки принятия решений (СППР).
  
  Настоящая работа развивает предложенное в~[8] двухуровневое 
представление сложных задач и предлагает новый подход к построению 
функциональных ГиИС, основанный на моделировании координации в ходе 
коллективного обсуждения решаемых проблем.
  
\section{Эволюция понятия <<координация>>}
  
  Исследование понятия <<координация>> для количественных, хорошо 
формализуемых задач исследования операций прослеживается в работах 
Дж.~Фар\-ка\-ша и Л.\,В.~Кантаровича~\cite{7kol}, Дж.~Данцига и 
П.~Вулфа~\cite{9kol}, Р.~Беллмана~\cite{10kol}, посвященных 
математическому программированию. При этом декомпозиция сложной задачи 
на составные части сводится к математическому приему поиска для исходной 
матрицы ее <<блочной>> структуры, вычислениям на частях-блоках и 
численным методам интеграции (объединения) частных решений в общее. 
В~работах М.~Месаровича, Д.~Мако и И.~Такахары предложен иной подход к 
координации в сложных формализованных задачах~\cite{11kol}. Сложная 
задача рассматривается как последовательность решаемых подзадач без 
структурно выделенного координирующего элемента. В~этом случае только 
завершение решения всех подзадач проясняет, получено ли решение общей 
задачи. Применение Ф.\,И.~Перегудовым, Ф.\,Л.~Тарасенко, Р.~Акоффом, 
Ф.~Эмери~\cite{12kol, 13kol} системного подхода к анализу решения сложных 
задач дало более совершенные теоретико-множественные представления. 
Сложные задачи обрели состав, структуру и эмерджентность. При этом 
возникли проблемы с количественной мерой сложности и исследованием 
  при\-чин\-но-след\-ст\-вен\-ных связей качественных (количественных) 
параметров задачи с ее интегративными свойствами.

\begin{figure*}[b] %fig1
\vspace*{1pt}
\begin{center}
\mbox{%
\epsfxsize=125.904mm
\epsfbox{kol-1.eps}
}
\end{center}
\vspace*{-6pt}
\Caption{Пример традиционного представления задачи в  системном подходе~(\textit{а}) и 
с учетом координации~(\textit{б}): $\pi_1^h, \ldots , \pi^h_3$~---  
за\-да\-чи-эле\-мен\-ты (подзадачи); 
$\pi^k$~--- задача-координатор; $R^{wq}\vert w$, $q=1,\ldots ,3$; $w\not=q$~--- отношения подзадач;  
$R^{kq}$~--- отношения координатора с подзадачами
  \label{f1kol}}
  \end{figure*}
  
  Фактическое положение вещей на примере\linebreak коллективного решения задачи 
оперативного планирования в СППР показывает обязательное\linebreak наличие 
координирующего элемента~--- лица, принимающего решения (ЛПР), 
диспетчера, в част\-ности выполняющего специфические, малоизученные 
функции, связанные с самоорганизацией в ходе коллективного обсуждения.
  
  В этой связи в рамках системного подхода предлагается новая модель 
сложной задачи: метод\linebreak моделирования решения сложных задач с координацией 
подзадач, а также архитектура ин\-фор\-ма\-ци\-он\-но-вычислительной системы, 
построенной по методологии функциональных гибридных интеллектуальных 
систем~\cite{8kol}.
  
\section{Самоорганизация в системах поддержки принятия решений}
  
  Системы поддержки принятия решений (система <<ЛПР--эксперты>>)~--- это 
выработанный жизнью способ решения сложных практических 
задач~\cite{8kol}. Применительно к планированию на машиностроительных 
предприятиях они названы планерками. Главная особенность таких 
  решений~--- самоорганизация в процессе коллективного обсуждения. Здесь 
многое зависит не только от экспертов и решения частных задач 
  (за\-дач-эле\-мен\-тов), но и от ЛПР, его знаний и опыта работы со сложными 
за\-да\-ча\-ми-сис\-те\-ма\-ми и управления ходом коллективного обсуждения.
  
  Обозначим задачу-систему~$\pi^u$, а за\-да\-чу-эле\-мент~--- $\pi^h$ 
(рис.~\ref{f1kol},\,\textit{а}). Тогда $\Pi^h =\{\pi^h_1, \ldots , \pi^h_N\}$~--- 
множество за\-дач-эле\-мен\-тов, входящих в~$\pi^u$;\linebreak 
$\dot{\Pi}^u=\{\dot{\pi}_1^u,\ldots , \dot{\pi}^u_M\}$~--- множество 
декомпозиций задачи~$\pi^u$; $R^{wq}\vert w$, $q=1,\ldots ,N$; $w\not=q$~--- 
отношения между задачами-элементами; $N$~--- здесь и далее мощность 
множества. Тогда модель задачи-сис\-те\-мы представим в следующем виде:
  \begin{equation}
\pi^u =\langle \Pi^h, \dot{\Pi}^u, R^{wq}\rangle\,.
\label{e1kol}
\end{equation}
  
  При решении задачи-системы за\-да\-чи-эле\-мен\-ты преимущественно 
отделены от внешней среды или ее состояние зафиксировано, т.\,е.\ 
выполняется требование о том, что связи внутри системы намного сильнее, чем 
с внешней средой.
  

  
  Модель~(\ref{e1kol}) имеет недостатки, и основной из них~--- нерелевантное 
отображение связей между элементами. Учитывать только связи~$R^{wq}$ 
недостаточно, а простое суммирование решений за\-дач-эле\-мен\-тов не дает 
решения за\-да\-чи-сис\-те\-мы. Исследования СППР показали, что в 
большинстве случаев эксперты не могут дать профессиональных решений в 
условиях, заданных им ЛПР изначально. А для изменения первоначальных 
условий в модели~(1) необходим существенный элемент~--- образ ЛПР, 
который выполнял бы функцию координатора, как <<перераспределителя>> 
ресурсов и переформулировал бы в зависимости от ситуации цели экспертов. 
Это позволило бы отобразить ситуацию, когда реальная СППР 
<<приспосабливается>> к различного рода обстоятельствам во внешней среде. 
Согласно синергетической парадигме это должно  происходить путем 
самоорганизации СППР.
  
  В~связи с вышесказанным в предлагаемом подходе задача рассматривается 
не только как отображение последовательности решений подзадач, но и как 
система с новым элементом~--- координатором~$\pi^k$. Его 
функция~--- мониторинг и управление процессом решения подзадач 
$\pi_1^h,\ldots , \pi_N^h$ экспертами в ходе коллективного обсуждения. 
Координатор связан отношениями $R^{kq}\vert q=1,\ldots ,N$ с каж\-дой задачей 
$\pi^h$ в системе~$\pi^u$, посредством которых собирает информацию о 
состоянии процесса решения экспертами за\-дач-эле\-мен\-тов и в конкретные 
моменты времени (планерки) выдает координирующие воздействия для 
изменения входного набора данных (ресурсов, целей). Тогда модель сложной 
задачи c координацией представим в следующем виде:
  \begin{equation}
  \pi^{uk} =\langle \Pi^h, \dot{\Pi}^u, \pi^k, R^{wq}, R^{kq}\rangle\,,
  \label{e2kol}
  \end{equation}
где $\pi^k$~--- координатор; $R^{kq}\vert q=1, \ldots , N$~--- отношения между 
координатором и за\-да\-ча\-ми-эле\-мен\-тами.
  
  Таким образом, можно дать следующее определение сложной практической 
задачи~--- это задача, включающая взаимодействующие 
  эле\-мен\-ты-под\-за\-да\-чи, между которыми происходит обмен данными 
(значениями переменных, синхроимпульсами и~т.\,п.), управляемый 
специальным элементом~--- координатором. Его присутствие отображает 
самоорганизацию СППР при решении за\-дач-сис\-тем.
  
  Сравнение (1) и~(2) показывает, что~(2) носит более общий характер и легко 
сводится к~(1). По сути, элемент-координатор может быть представлен 
<<координирующей задачей>> ($k$-за\-да\-чей), которая должна быть 
<<добавлена>> в декомпозицию $\dot{\pi}^u\in \dot{\Pi}^u$ сложной 
задачи~$\pi^u$, чтобы релевантно отображать в модели особенности 
коллективного решения задач планирования. Отметим, что 
  за\-да\-ча-ко\-ор\-ди\-на\-тор не дополняет имеющиеся подзадачи, а 
существует как подзадача сложной задачи, решение которой традиционно 
возлагалось на ЛПР.
  
  Можно также отметить, что с увеличением количества за\-дач-эле\-мен\-тов, 
актуальность координации их решения возрастает, так как комбинаторно растет 
число отношений между за\-да\-ча\-ми-эле\-мен\-тами.
  
\section{Алгоритм координации в~системах поддержки принятия~решений}
  
  Предлагаемый алгоритм отображает самоорганизацию в СППР по принципу 
<<как есть>> (лат.\ \textit{ad hoc}) и относится к алгоритмам, основанным на знаниях, 
где акцент смещается с использования формализованной математической 
схемы на извлечение профессиональных знаний и рассуждениям с их 
применением. Для разработки алгоритма такие знания были извлечены на 
примере СППР машиностроительного предприятия с мелкосерийным 
характером производства. Все знания пред\-став\-ле\-ны продукционными 
системами. База знаний ЛПР (начальника производственного центра) о 
  $k$-за\-да\-че содержит 24~правила. Базы знаний пяти экспертов 
(начальников отделов) содержат от~10 до 40~правил. В качестве оболочки ЭС 
выбрана программа Visual Rule Studio~\cite{14kol} и метод рассуждений в 
прямом направлении~\cite{8kol}.
  
  Приведенный ниже алгоритм имитирует последовательность заседаний 
СППР в относительном, модельном времени. При этом линии рассуждений 
экспертов координируются ЛПР. Полученные экспертами после каждой 
итерации (планерки) решения частных подзадач передаются ЛПР как исходные 
данные для задачи координации ($k$-за\-да\-чи). Лицо, принимающее решения, использует в процессе 
решения задачи координации данные, полученные после декомпозиции 
исходной сложной задачи. Решив задачу координации, ЛПР дает рекомендации 
каждому из экспертов. Эти рекомендации в совокупности с первоначально 
известными после декомпозиции  данными, служат исходными данными для 
решаемых экспертами подзадач на следующей итерации (планерке). В~случаях, 
когда не требуется вносить изменения в ход решения подзадачи экспертом, 
ЛПР выдает <<пустую команду>>.

\smallskip
  
\noindent
  \textbf{Дано:} СППР, состоящая из $N$ экспертов и ЛПР. Число планерок в 
плановом периоде~--- $k$. Установлено однозначное соответствие 
$$\psi_1:\ \ 
\mathrm{Out}\vert^{\pi_1^h}\cup\ldots \cup \mathrm{Out}\vert^{\pi_N^h}\rightarrow \mathrm{In}\vert^{\pi^k}
$$ 
между выходными параметрами $\mathrm{Out}\vert^{\pi_i^h}=$\linebreak $=\{\mathrm{Out}_1, \ldots , 
\mathrm{Out}_m\}\vert^{\pi_i^h}$ подзадач~$\pi_i^h$, $i=1, \ldots , N$, и входными 
параметрами $\mathrm{In}\vert^{\pi^k}=\{\mathrm{In}_1, \ldots , \mathrm{In}_n\}\vert^{\pi^k}$ задачи 
координации~$\pi^k$. Установлено однозначное соответствие
$$
\psi_2:\ \ 
\mathrm{Out}\vert^{\pi^k}\rightarrow \mathrm{In}\vert^{\pi_1^h}\cup\ldots\cup \mathrm{In}\vert^{\pi_N^h}
$$ 
между выходными параметрами $\mathrm{Out}\vert^{\pi^k}=$\linebreak $=\{\mathrm{Out}_1, \ldots 
,\mathrm{Out}_n\}\vert^{\pi^k}$ задачи координации~$\pi^k$ и входными параметрами 
$\mathrm{In}\vert^{\pi_i^h} =\{\mathrm{In}_1, \ldots ,\mathrm{In}_n\}\vert^{\pi_i^h}$ подзадач~$\pi_i^h$, $i=1, 
\ldots , N$. Соответствия~$\psi_1$ и~$\psi_2$ сюръективны и иньюктивны.

\smallskip
  
\noindent
  \textbf{Найти:} результаты решения (выходные пара\-мет\-ры) 
подзадач~$\pi_i^h$, $i=1, \ldots ,N$, экспертами с учетом координации их 
решения ЛПР.

\smallskip
  
\noindent  
\textbf{Обозначения:} БЗ$_i$, $i = 1, \ldots , N$,~--- базы знаний экспертов; 
БЗ$_{\mathrm{лпр}}$~--- база знаний ЛПР; БФ$_i$, $i = 1, \ldots , N$,~--- базы 
фактов экспертов; БФ$_{\mathrm{лпр}}$~--- база фактов ЛПР;  $j$~--- счетчик 
планерок; $i$~--- счетчик экспертов; $\mathrm{Run}(\mathrm{БФ,БЗ})$~--- процедура имитации\linebreak 
рассуждений ЛПР или экспертов, где БФ и БЗ~---\linebreak базы фактов и знаний 
соответственно; $\mathrm{Search}(\mathrm{БЗ})$~--- про\-це\-ду\-ра-про\-смотр правил из БЗ и 
сопоставление фактов с образцами с целью определения множества правил, 
которые могут быть активированы; Conf~--- процедура разрешения 
конфликтов в <<плане решения>> (Agenda) экспертной системы, когда 
возникает необходимость выбора между несколькими правилами из БЗ для 
продолжения рассуждений (метод разрешения конфликтов~--- <<первый в 
Agenda>>); Execute~--- процедура применения найден\-ных правила для 
продолжения рассуждений, т.\,е.\ выполнения действий указанных в правой 
части выбранных правил.

\smallskip
  
\noindent
  \textbf{Алгоритм}
  
  \noindent
  \begin{enumerate}[1)]
\item $j = 0$;
\item $i = 1$;
\item Заполнить БФ$_i$ с учетом $(\mathrm{In}_1, \ldots , \mathrm{In}_n)\vert^{\pi_i^h}$ и счетчика 
планерок~$j$;
\item Run(БФ$_i$,  БЗ$_i$):
\begin{itemize}
\item[1.1)] $\mathrm{Search}(\mathrm{БЗ}_i)$;
\item[1.2)] Conf;
\item[1.3)] Execute;
\item[1.4)] Если для продолжения рассуждений не выбрано ни одно из 
правил, то закончить имитацию рассуждений и перейти к п.~5. Иначе~---  
к~п.~4.1;
\end{itemize}
\item Присвоить $(\mathrm{Out}_1, \ldots , \mathrm{Out}_m)\vert^{\pi_i^h}$ значения из фактов~--- 
результатов рассуждений $i$-го эксперта;
\item Присвоить, используя $\psi_1$, соответствующим входным параметрам 
$(\mathrm{In}_1, \ldots , \mathrm{In}_n)\vert^{\pi^k}$ задачи координации~$\pi^k$ значения 
выходных па\-ра\-мет\-ров $(\mathrm{Out}_1, \ldots , \mathrm{Out}_m)\vert^{\pi_i^h}$ задачи~$\pi_i^h$;
\item Если $i < N + 1$, то $i = i + 1$ и перейти к~п.~3. Иначе перейти к~п.~8;
\item Заполнить БФ$_{\mathrm{лпр}}$ с учетом значений входных параметров 
$(\mathrm{In}_1, \ldots ,\mathrm{In}_n)\vert^{\pi^k}$ задачи координации~$\pi^k$ и счетчика 
планерок~$j$;
\item $\mathrm{Run}(\mathrm{БФ}_{\mathrm{лпр}}, \mathrm{БЗ}_{\mathrm{лпр}})$:
\begin{itemize}
\item[9.1)] $\mathrm{Search}(\mathrm{БЗ}_{\mathrm{лпр}})$;
\item[9.2)] Conf;
\item[9.3)] Execute;
\item[9.4)] Если для продолжения рассуждений не выбрано ни одно из 
правил, то закончить имитацию рассуждений и перейти к~п.~10. Иначе~--- 
к~п.~9.1;
\end{itemize}
\item Присвоить $(\mathrm{Out}_1, \ldots , \mathrm{Out}_n)\vert^{\pi^k}$ задачи 
координации~$\pi^k$ значения из фактов-результатов рассуждений ЛПР;
\item Присвоить, используя~$\psi_2$, соответствующим входным параметрам 
$(\mathrm{In}_1,  \ldots , \mathrm{In}_n)\vert^{\pi^k}$ подзадач $\pi_1, \ldots ,\pi_N$ значения 
выходных параметров задачи координации~$\pi^k$;
\item Если $j < k$, то $j = j + 1$ и перейти к~п.~2. Иначе перейти к~п.~13;
\item Вывести значения выходных параметров подзадач, полученные на 
$j$-й ~планерке:
\begin{itemize}
\item[13.1)] $i = 1$;
\item[13.2)] Печать $(\mathrm{Out}_1, \ldots ,\mathrm{Out}_m)\vert^{\pi_i^h}$;
\item[13.3)] Если $i < N + 1$, то $i = i + 1$ и перейти к~п.~13.2. Иначе 
перейти к~п.~14;
\end{itemize}
\item Конец.
\end{enumerate}
  
  Разработанный алгоритм дискретен и пред\-ставля\-ет координацию как 
последовательность прос\-тых шагов. Ему присуща определенность~---\linebreak каждое 
правило алгоритма детерминированное, четкое и не оставляет места для 
многозначности. Алгоритм приводит к решению сложной задачи за конечное 
число шагов (результативность), так как внешний цикл (имитация планерок) 
выполняется не более $k - 1$~раз. Вложенные циклы (имитация рассуждений 
каждого эксперта) и цикл вывода фактов-результатов выполняются не более 
$N$~раз. Отсутствие циклов в графе И/ИЛИ рассуждений экспертов и ЛПР 
гарантирует отсутствие зацикливания. Алгоритм координации разработан в 
общем виде (массовость), однако он не лишен недостатка <<хрупкости>> 
любой экспертной системы.
  
  Порядок опроса экспертов не важен, так как пока последний эксперт не 
сообщит результат решения своей подзадачи, ЛПР не может приступить к 
решению задачи координации.  В~случае, если в результате имитации 
рассуждений экспертов или ЛПР не могут быть выведены факты, необходимые 
для спецификации выходных параметров решаемых подзадач, то используются 
значения по умолчанию, извлеченные у экспертов и ЛПР при  разработке 
конкретной ин\-фор\-ма\-ци\-он\-но-вы\-чис\-ли\-тель\-ной сис\-те\-мы. 
Существенное отличие базы фактов ЛПР от баз фактов экспертов в том, что в 
ней\linebreak происходит интеграция результатов решения подзадач экспертами, 
основанная на при\-чин\-но-след\-ст\-вен\-ных связях, 
  час\-тич\-но-фор\-ма\-ли\-зо\-ван\-ных продукционными правилами и 
представленными\linebreak базой профессиональных знаний ЛПР, о качестве которых 
можно судить исключительно по результатам проведенной апробации. Для 
таких экспериментов была разработан программный продукт <<Гибридная 
система планирования>>.
  
\section{Модель гибридной интеллектуальной системы с~учетом 
координации}
  
  В качестве модели ГиИС как абстрактного автомата для решения сложной 
задачи опе\-ра\-тив\-но-про\-из\-вод\-ст\-вен\-но\-го планирования принята 
функциональная крупнозернистая ГиИС~$\alpha^{u}(t)$~\cite{8kol}. Ее  
расширение выполнено исходя из следующих посылок. В~процессе 
координации контролируются промежуточные состояния процесса решения 
подзадач. Под этими состояниями понимаются состояния функциональных 
элементов~$\alpha^h$, имитирующих решение подзадач~$\pi^h$, а также 
состояния технологических элементов~$\alpha^\tau$. На основании анализа 
этих состояний в ходе координации изменяются свойства 
<<вход>>~$\hat{x}_1^2$ одного или нескольких функциональных и 
технологических элементов~$\alpha^h$ и~$\alpha^\tau$. Для учета этого факта 
введем в модель крупнозернистой функциональной ГиИС $\alpha^{u}(t)$ 
следующую тройку: $\hat{x}_3^2(t) R^{22} \hat{x}_1^2(t+1)$. Иными словами, 
на основании состояния ГиИС~$\hat{x}_3^2(t)$ в момент времени~$t$ 
меняются исходные данные $\hat{x}_1^2(t+1)$ для ГиИС в момент времени 
$t+1$, т.\,е.\ для следующей итерации. Множество~$R^{22}$ устанавливает 
отношения между состоянием $\hat{x}_3^2(t)$ гибрида~$\alpha^u$ на данный 
момент модельного времени~$t$ и состоянием входов одного или нескольких 
функциональных и технологических элементов~$\alpha^h$ и~$\alpha^\tau$ на 
следующем шаге. Чтобы произвести необходимое изменение 
входов~$\hat{x}_1^2$ одного или нескольких функциональных и 
технологических элементов~$\alpha^h$ и~$\alpha^\tau$ в~$\alpha^u(t)$ введем 
тройку $\hat{x}_3^2(t) R^{23} X_1^3$, где $X^3=\{x_1^3, \ldots , x_6^3\}$~---\linebreak 
множество понятий, обозначающих ко\-ор\-ди\-ни\-ру\-ющие действия (интегральная 
координация, четкая координация, интервальная координация, лингвистическая 
координация, координация по времени, <<пустое действие>>), которое 
тождественно множеству координирующих действий, введенных 
в~\cite{15kol}. В~алгоритме координации эти действия (рекомендации 
экспертам) описаны в базе знаний ЛПР. Множество~$R^{23}$~--- это 
отношения между состоянием~$\hat{x}_3^2$ гибрида~$\alpha^u$ в момент 
модельного времени~$t$ и необходимыми координирующими 
действиями~$X^3$. Ниже приведена модифицированная схема ролевых 
концептуальных моделей~$\alpha^{uk}$ для спецификации крупнозернистой 
функциональной ГиИС с координацией
\begin{equation}
\alpha^{uk}(t)= \alpha^u(t)\circ \hat{x}_3^2(t) R^{22} \hat{x}_1^2(t+1)\circ \hat{x}_3^2(t) R^{23} X^3\,,
\label{e3kol}
\end{equation}
где $\circ$~--- знак конкатенации; 1, 2, 3 в качестве правого верхнего 
индекса~$X$ или~$x$~--- признак ресурса, свойства, действия соответственно; 
нижний правый индекс для $x$~--- порядковый номер класса понятий; верхний 
правый индекс для~$R$ обозначает между какими понятиями категориального 
ядра~\cite{8kol} установлены отношения (22~--- <<свой\-ст\-во--свой\-ст\-во>>, 
23~--- <<свой\-ст\-во--дей\-ст\-вие>>).
  
  Отношения $R^{22}$ и~$R^{23}$ не задаются заранее, а фиксируются в ходе 
функционирования ГиИС и поиска результата решения 
  за\-да\-чи-ко\-ор\-ди\-на\-то\-ра ($k$-за\-да\-чи). Поскольку в соответствии 
с~\cite{8kol} технологические элементы управляют порядком работы 
функциональных элементов и обменом информации между ними, 
целесообразно возложить решение задачи координации на технологический 
элемент.
  
  Рассмотрим пример ГиИС, состоящей из трех функциональных элементов 
$\alpha^h\vert_1^1$, $\alpha^h\vert_1^7$, $\alpha^h\vert_1^6$ и одного 
технологического элемента~$\alpha^\tau\vert_1^7$, где $w=1, \ldots , 7$ для 
$\alpha^h\vert_j^w$ и $\alpha^\tau\vert_j^w$ обозначает базовые классы методов 
функциональных ГиИС~\cite{8kol}, $j=1$~--- порядковый номер задачи в 
классе. На вход ГиИС подаются исходные данные, разделенные между 
функциональными элементами в соответствии с декомпозицией сложной 
задачи. На выходе имеем результаты работы функциональных элементов, 
агрегированные в общее решение задачи.

\begin{figure*} %fig2
\vspace*{1pt}
\begin{center}
\mbox{%
\epsfxsize=165.684mm
\epsfbox{kol-2.eps}
}
\end{center}
\vspace*{-3pt}
\Caption{Структурная схема ГиИС без координации~(\textit{а}) и с координацией~(\textit{б}):
\textit{1}~--- прямая и обратная координи\-ру\-ющие связи между технологическим элементом и
функциональными элементами;
\textit{2}~--- отношения порядка работы и обмена информацией между элементами;
\textit{3}~--- редукция сложной задачи (на входе) и интеграция результатов работы функциональных элементов;
\textit{4}~--- прямая и обратная связи между ГиИС и ЛПР при отсутствии координации внутри ГиИС;
\textit{5}~--- прямая и обратная связи между ГиИС и ЛПР при наличии координации внутри ГиИС
\label{f2kol}}
\vspace*{9pt}
\end{figure*}
  
  На рис.~\ref{f2kol},\,\textit{а} изображена структурная схема \mbox{ГиИС}, 
построенная для решения сложной задачи в соответствии с~$\alpha^u(t)$. Здесь 
моделируется только логически увязанная последовательность решения 
подзадач~$\pi^h$ из декомпозиции~$\dot{\pi}^u$ сложной задачи~$\pi^u$. Это 
соответствует модели сложной задачи на рис.~\ref{f1kol},\,\textit{а}. В~этом 
случае по каналу обратной связи ЛПР получает от компьютерной СППР 
решение сложной задачи. Если общее единое решение дает с точки зрения ЛПР 
ошибочный результат, то он на основании своих оценок по каналу прямой 
связи вносит изменения во множества входных данных и условий 
задачи~$\pi^u$. Далее ЛПР инициирует новый синтез ГиИС и повторное 
решение.
  
  На рис.~\ref{f2kol},\,\textit{б} изображена принципиально иная структурная 
схема ГиИС. Ее отличие от вышеприведенной в том, что технологический 
элемент~$\alpha^\tau\vert_1^7$ определяет не только порядок работы 
функциональных элементов и обмен информации между ними, но и в 
соответствии с~(\ref{e3kol}) по состоянию всех функциональных элементов 
итерационно корректирует для каждого из них входной набор данных и 
условий. Таким образом, часть функций ЛПР передается технологическому 
элементу, что отражено на рис.~\ref{f2kol},\,\textit{б} изменением размеров 
структурных блоков ЛПР и технологического элемента, а также толщины 
линий прямой и обратной связи в контуре управления.
  
  В представленной на рис.~\ref{f2kol},\,\textit{б} структурной схеме ЛПР по 
каналу обратной связи получает от компьютерной СППР результат решения 
сложной задачи. Если решение не устраивает ЛПР, например оно ведет к 
увеличению стоимости производимых изделий и~т.\,д., то он вмешивается в 
обсуждение и меняет условия координации, т.\,е.\ модель задачи-координатора. 
Далее ЛПР инициирует новый синтез ГиИС и повторное решение сложной 
задачи. В~итоге это позволяет моделировать процесс самоорганизации, о 
котором было сказано выше.
  
  В модели~(\ref{e3kol}) развито предположение, что включение в 
компьютерную модель СППР модели ЛПР приводит к возникновению 
синергетических эффектов~--- самоорганизации. При этом появляется 
возможность увязать результаты работы отдельных функциональных элементов 
СППР еще в процессе синтеза решения сложной задачи, а не после, как в 
известных моделях. Тем самым достигается большая релевантность 
компьютерной \mbox{СППР} реальному процессу коллективного обсуж\-де\-ния \mbox{проблем}.
  
\section{Заключение}
  
  Гибридные интеллектуальные системы с координацией элементов~--- новый 
шаг в синергетическом искусственном интеллекте, позволяющий полнее 
раскрыть и исследовать многообразие отношений в системе 
  <<ЛПР--экс\-пер\-ты>>. Это также результат и в моделировании категории 
<<время>> в гибридных системах на примере производственных планерок 
динамичных процессов опе\-ра\-тив\-но-про\-из\-вод\-ст\-вен\-но\-го 
планирования.
  
  По сравнению с известными алгоритмами 
  координации~[1, 8--10, 16, 17], предлагаемый 
алгоритм имеет следующие достоинства: итеративность позволяет имитировать 
обмен информацией в процессе решения сложной задачи, а применение 
профессиональных баз знаний делает модель релевантной сложной 
практической задаче.
  
  Апробация ГиИС, решающих $k$-задачу, на реальных данных показала 
положительный эффект от увеличения релевантности моделирования сложной 
задачи опе\-ра\-тив\-но-про\-из\-вод\-ст\-вен\-но\-го пла\-нирования, что 
улучшило качественные и ко\-личественные показатели машиностроительного\linebreak 
предприятия с мелкосерийным заказным характером производства. 
В~частности, ожидаемый экономический эффект от внедрения ГиИС на одном из 
предприятий в процентном отношении к ожидаемой прибыли предприятия за 
год составил~11\%.

{\small\frenchspacing
{%\baselineskip=10.8pt
\addcontentsline{toc}{section}{Литература}
\begin{thebibliography}{99}

\bibitem{7kol} %1
\Au{Канторович Л.\,В.}
Математические методы организации и планирования производства.~--- Л.: ЛГУ, 1959.



\bibitem{6kol} %2
\Au{Татевосов К.\,Г.}
Основы оперативно-про\-из\-вод\-ст\-вен\-но\-го планирования на машиностроительном 
предприятии.~--- Л.: Машиностроение, 1985.

\bibitem{5kol} %3
\Au{Заболотский В.\,П., Оводенко А.\,А., Степанов~А.\,Г.}
Математические модели в управлении: Учеб. пособие.~--- СПб.: \mbox{СПбГУАП}, 2001.


\bibitem{1kol} %4
\Au{Туровец О.\,Г., Родионов В.\,Б., Бухалков М.\,И.}
Организация производства и управление предприятием.~--- М.: ИНФРА-М, 2005.

\bibitem{2kol} %5
\Au{Кальянов Г.\,Н.}
Моделирование, анализ, реорганизация и автоматизация биз\-нес-про\-цес\-сов.~--- М.: 
Финансы и статистика, 2007.

\bibitem{4kol} %6
\Au{Сачко Н.\,С.}
Организация и оперативное управление машиностроительным производством.~--- Минск: 
Новое Знание, 2008.

\bibitem{3kol} %7
\Au{Тейлор Д., Рэйден Н.}
Почти интеллектуальные сис\-те\-мы. Как получить конкурентные преимущества путем 
автоматизации принятия скрытых решений.~--- СПб.: Сим\-вол-Плюс, 2009.


\bibitem{8kol}
\Au{Колесников А.\,В., Кириков И.\,А.}
Методология и технология решения сложных задач методами функциональных гибридных 
интеллектуальных систем.~--- М.: ИПИ РАН, 2007.

\bibitem{9kol}
\Au{Данциг Дж.}
Линейное программирование, его обобщения и применения.~--- М.: Прогресс, 1966.

\bibitem{10kol}
\Au{Беллман Р., Дрейфус~С.}
Прикладные задачи динамического программирования.~--- М.: Наука, 1965.

\bibitem{11kol}
\Au{Месарович М., Мако Д., Такахара И.}
Теория иерархических многоуровневых систем.~--- М.: Мир, 1973.


\bibitem{13kol} %12
\Au{Акофф Р., Эмери Ф.}
О целеустремленных системах.~--- М.: Советское радио, 1974.

\bibitem{12kol} %13
\Au{Перегудов Ф.\,И., Тарасенко Ф.\,Л.}
Введение в системный анализ.~--- М.: Высшая школа, 1989.

\bibitem{14kol}
Сайт компании Rule Machines Corporation. {\sf http://\linebreak www.rulemachines.com}.

\bibitem{15kol}
\Au{Колесников А.\,В., Солдатов С.\,А.}
Теоретические основы решения сложной задачи оперативно-про\-из\-вод\-ст\-вен\-но\-го 
планирования с учетом координации~// Вестник Российского государственного ун-та им.\ 
И.~Канта. Вып.~10. Сер. Физико-математические науки.~--- Калининград: РГУ им. 
И.~Канта, 2009. С.~82--98.

\bibitem{17kol} %16
\Au{Beat F., Schmid K., Stanoevska~S., Lei~Yu.}
Supporting distributed corporate planning through new coordination technologies, 1998. {\sf 
http://\linebreak www.alexandria.unisg.ch/Publikationen/9453}.



 \label{end\stat}
 
 \bibitem{16kol} %17
\Au{Geun-Sik Jo, Kang-Hee L., Hwi-Yoon L., Sang-Ho~H.}
Ramp activity expert system for scheduling and co-ordination at an airport~// Innovative 
Application of Artificial Intelligence '99, AAAI, July, 1999. P.~807--812. {\sf 
http://www.aaai.org/Papers/IAAI/1999/IAAI99-114.pdf}.


 \end{thebibliography}
}
}


\end{multicols}