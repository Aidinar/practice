\def\stat{chuprakov}

\def\tit{К ВОПРОСУ О РАЗМЕЩЕНИИ КОЛЛЕКТИВНЫХ СРЕДСТВ 
ОТОБРАЖЕНИЯ В СИТУАЦИОННОМ ЗАЛЕ С~ЗАДАННЫМИ 
ПАРАМЕТРАМИ}

\def\titkol{К вопросу о размещении коллективных средств 
отображения в ситуационном зале с~заданными 
параметрами}

\def\autkol{К.\,Г.~Чупраков}
\def\aut{К.\,Г.~Чупраков$^1$}

\titel{\tit}{\aut}{\autkol}{\titkol}

%{\renewcommand{\thefootnote}{\fnsymbol{footnote}}\footnotetext[1]
%{Исследование поддержано грантами РФФИ 08-07-00152 и 09-07-12032.
%Статья написана на основе материалов доклада, представленного на IV 
%Международном семинаре  <<Прикладные задачи теории вероятностей и математической статистики, 
%связанные с моделированием информационных систем>> (зимняя сессия, Аоста, Италия, январь--февраль 2010~г.).}}

\renewcommand{\thefootnote}{\arabic{footnote}}
\footnotetext[1]{Институт проблем информатики Российской академии наук, chkos@rambler.ru}

\vspace*{6pt}

\Abst{Установлены некоторые зависимости между основными параметрами 
ситуационного зала: его размерами, информативностью отображаемого контента, числом 
одновременных наблюдателей и размерами экрана. Основой для таких зависимостей 
стали рекомендации, закрепленные в ГОСТах, и простые геометрические соображения.}

\vspace*{2pt}

\KW{системы отображения информации; ситуационный зал; область наилучшего 
наблюдения; аналитические зависимости}

\vspace*{4pt}

       \vskip 14pt plus 9pt minus 6pt

      \thispagestyle{headings}

      \begin{multicols}{2}

      \label{st\stat}



\section{Введение}
    
    Использование ситуационных центров доказало их практическую 
значимость для решения задач управления крупными и сложными 
объектами, в числе которых государственные учреждения~[1], а также 
крупные корпорации и предприятия~[2]. Однако создание ситуационных 
центров выявило ряд проблем, возникающих и в момент их разработки, и в 
процессе эксплуатации~[1, 3--7].

Ситуационный зал является одним из приложений ситуационного центра для 
решения задач ситуационного управления с помощью двух основных 
методик: обсуждение возможных решений экспертной группой и доклад 
(презентация) некоторого материала с помощью средств отображения 
информации~\cite{8chu, 9chu}. В~ситуационном зале могут анализироваться 
и разрабатываться различные варианты решения стратегического характера. 
Поэтому создание ситуационного зала~--- это не просто оснащение комнаты 
презентационным оборудованием с обеспечением некоторого 
респектабельного облика помещения, но и создание максимального уровня 
комфорта, где присутствие даже самой незначительной детали должно быть 
обосновано на уровне технического задания.

Для создания ситуационного зала в первую очередь необходимо оценить 
целый ряд параметров: размеры помещения, максимальное число человек, 
участвующих в переговорах, необходимую производительность средств 
отображений информации с точки зрения статической информативности 
экрана, а также геометрические размеры экранов этих средств отображения.

В статье предложен подход к определению взаимосвязей между основными 
параметрами ситуационного зала и оценке относительного расположения 
рабочих мест сотрудников и коллективного экрана.    

\section{Общий подход. Термины и~определения}

Создание и оборудование ситуационного зала среди прочих требований 
должно опираться на существующие стандарты по эргономике, действующие 
на территории РФ. В табл.~1 перечислены основные термины и понятия, 
которые будут использованы в рамках статьи~[10--12].

\begin{table*}\small
\begin{center}
\Caption{Основные термины и понятия
}
\vspace*{2ex}

\tabcolsep=5.5pt
\begin{tabular}{|l|c|l|}
\hline
\multicolumn{1}{|c|}{Термин}&\tabcolsep=0pt\begin{tabular}{c}Обозна-\\ чение\end{tabular}&
\multicolumn{1}{c|}{Определение}\\
\hline
Активная часть экрана&&Часть экрана, ограниченная пикселами\\
\hline
\tabcolsep=0pt\begin{tabular}{l}Стягиваемый угол\\ (угловой размер)\end{tabular}&
$\psi$&\tabcolsep=0pt\begin{tabular}{l}Размер визуального объекта при данном конкретном 
расстоянии наблю-\\ дения:
$\psi=2\arctg(h/(2D))$, где $h$~--- высота объекта; $D$~--- расстояние\\ наблюдения\end{tabular}\\
\hline
Высота знака&$h$&Линейная высота знака, м\\
\hline
Ширина знака&$w$&Линейная ширина знака, м\\
\hline
Формат знака&&
\tabcolsep=0pt\begin{tabular}{l}Число пикселов по горизонтали и вертикали в матрице, используемой для\\ построения 
символа\end{tabular}\\
\hline
\tabcolsep=0pt\begin{tabular}{l}Проектное расстояние\\ наблюдения\end{tabular}&
$D$&\tabcolsep=0pt\begin{tabular}{l}Расстояние или диапазон расстояний между экраном и глазами 
наблюдате-\\ля, при котором изображение соответствует требованиям разборчивости и\\ удобочитаемости\end{tabular} \\
\hline
\tabcolsep=0pt\begin{tabular}{l}Рабочая площадь\\ помещения\end{tabular}&
&\tabcolsep=0pt\begin{tabular}{l}Множество положений в помещении, в которых сохраняется 
разборчивость,\\ удобочитаемость материала, отображенного экраном\end{tabular} \\
\hline
Угол обзора человека&$\gamma$&
\tabcolsep=0pt\begin{tabular}{l}Угол, стягивающий точки пространства, которые человек способен 
наблю-\\ дать без напряжения для глазных мышц и поворотов головы \end{tabular}\\
\hline
\end{tabular}
\end{center}
\end{table*}
\begin{figure*}[b] %fig1
%\vspace*{-24pt}
\begin{center}
\mbox{%
\epsfxsize=149.148mm
\epsfbox{chu-1.eps}
}
\end{center}
\vspace*{-6pt}
\Caption{Рекомендуемый угол обзора человека в зависимости от интенсивности 
наблюдения
\label{f1chu}}
\end{figure*}

     Опираясь на эти понятия и термины, можно сформулировать и 
проанализировать требования и зависимости, описанные в государственных 
стандартах.
     
     В качестве одного из основных будет использовано понятие 
\textbf{области наилучшего наблюдения}, под которой понимается область 
в трехмерном или двумерном пространстве, удовлетворяющая 
эргономическим требованиям ГОСТов.
    

\subsection{Угол обзора человека} %2.1

Согласно~\cite{13chu, 14chu}:
\begin{itemize}
\item очень часто используемые средства отображения информации, 
требующие точного и быст\-ро\-го считывания показаний, следует 
располагать так, чтобы в вертикальном сечении они были видны под 
углом $\pm 15^\circ$ от нормальной линии взгляда и в горизонтальном 
сечении~--- под углом $\pm 15^\circ$ от плоскости симметрии 
человеческого тела (рис.~1);
\item часто используемые средства отображения информации, 
требующие менее точного и быстрого считывания показаний,~--- 
$\pm30^\circ$ по вертикали и горизонтали;
\item редко используемые средства отображения информации~--- $\pm 
60^\circ$ по вертикали и горизонтали.
\end{itemize}



\subsection{Угловой размер знака} %2.2

     Стягиваемый угол определяется согласно~\cite{10chu, 15chu}. 
Рекомендуемые показатели для обеспечения разборчивости (для латинского алфавита):
минимальный $\psi_{\min} =16^\prime$; предпочтительный $\psi_{\mathrm{предп}}=$\linebreak $=20^\prime\mbox{--}22^\prime$.

     Из определения угла $\psi$ следует, что
     \begin{equation}
     \fr{h}{D}=2\tg\fr{\psi}{2}\approx \psi\,.
     \label{e1chu}
     \end{equation}
     
     Формула~(\ref{e1chu}) верна, так как угол~$\psi$, измеряемый здесь и 
далее в радианах, очень мал. Данное соотношение позволяет оценить 
минимальное и рекомендуемое отношение высоты знака и проектного 
расстояния:
минимальное $(h/D)_{\min}=\psi_{\min}=$\linebreak $= 4{,}7\cdot 10^{-3}$;
рекомендуемое $(h/D)_{\mathrm{рек}}=\psi_{\mathrm{рек}}=$\linebreak $= 5{,}8\mbox{--}6{,}4\cdot 10^{-3}$.

\subsection{Формат букв} %2.3

     Отношение линейных параметров ширины знака к его высоте должно 
соответствовать следующим значениям (для латинского алфавита)~[12]:
допустимый диапазон~--- от 0,5:1 до 1:1,  
предпочтительный диапазон~--- от 0,6:1 до 0,9:1.




\begin{table*}\small %tabl2
\begin{center}
\Caption{ Основные параметры, используемые в зависимостях
}
\vspace*{2ex}
\begin{tabular}{|l|c|l|}
\hline
\multicolumn{1}{|c|}{Термины}&Обозначение&\multicolumn{1}{c|}{Определение}\\
\hline
Информативность&$I$&Максимальное число знаков в одном кадре контента\\
\hline
Число рабочих мест&$N$&Число рабочих мест для одного коллективного экрана\\
\hline
\tabcolsep=0pt\begin{tabular}{l}Диаметр\\ помещения\end{tabular}&$P$&
\tabcolsep=0pt\begin{tabular}{l}Максимальное расстояние между двумя точками поме-\\щения, в 
том числе обусловленное его геометрией, м\end{tabular}\\
\hline
Ширина экрана&$W$&Геометрическая ширина активной части экрана, м\\
\hline
Высота экрана&$H$&Геометрическая высота активной части экрана, м\\
\hline
\end{tabular}
\end{center}
\vspace*{9pt}
\end{table*}



\section{Формирование зависимостей между основными 
параметрами ситуационного зала}

    На основании рекомендаций, обозначенных в п.~2, можно приступить 
к формированию взаимосвязей между основными параметрами системы 
<<помещение--экран--наблюдатели>>. Этими основными параметрами 
являются: информативность, число рабочих мест, диаметр помещения, 
ширина и высота экрана (табл.~2).


     Стоит отметить разницу между проектным\linebreak расстоянием 
наблюдения~$D$ и диаметром помещения~$P$. Первый параметр 
характеризуется свойствами системы <<экран--наблюдатели>>, а 
параметр~$P$~--- исключительно свойствами помещения.

\subsection{Область наилучшего наблюдения для~коллективного экрана} 
%3.1
    
    Пусть $\varphi$~--- угол направления обзора элемента экрана 
относительно нормали к поверхности экрана из центра элемента. Тогда 
размер элемента при наблюдении под углом~$\varphi$ уменьшится и может 
быть приближенно вычислен как
    \begin{equation}
    w(\varphi)=w\cos\varphi\,,
    \label{e2chu}
    \end{equation}
    где $w$~--- линейный размер элемента, а $w(\varphi)$~--- его линейный 
размер при наблюдении под углом~$\varphi$. Поэтому из сохранения 
углового размера символа следует соотношение:
    \begin{equation*}
    D(\varphi)=D\cos\varphi\,.
%    \label{e3chu}
    \end{equation*}

Следовательно, область наилучшего наблюдения элемента экрана (в плоском 
горизонтальном сечении)~--- круг, касающийся экрана в точке, 
соответствующей этому элементу и диаметром, равным~$D$~--- проектному 
расстоянию (рис.~2).




    Рассматривая экран как множество активных точек~\cite{16chu}, для 
каждой из которых строится область\linebreak\vspace*{-12pt}

\begin{center} %fig2
%\vspace*{6pt}
\mbox{%
\epsfxsize=43.302mm
\epsfbox{chu-2.eps}
}
\end{center}
\vspace*{4pt}
\begin{center}
{{\figurename~2}\ \ \small{Область наилучшего наблюдения для точечного элемента экрана}}
\end{center}
%\vspace*{9pt}

\medskip
\addtocounter{figure}{1}

\noindent
 наилучшего наблюдения, можно 
построить область наилучшего наблюдения для всего экрана~--- пересечение 
множества построенных кругов (рис.~3). Далее рассматривается случай, 
когда экран плоский и рабочие места располагаются в горизонтальной 
плоскости.



В случае, когда экран плоский, областью наилучшего наблюдения будет 
пересечение двух наиболее далеких друг от друга кругов.

     Другим существенным ограничением, которое необходимо наложить 
на область наилучшего наблюдения, являются пропорции наблюдаемых 
символов, которые, согласно данным п.~2.3, не могут меняться более чем в 2~раза, 
так как максимальное и минимальное допустимые значения пропорций 
отличаются друг от друга ровно в  2~раза (максимальная ширина равна 
одной высоте знака, а минимальная~--- ее половине). Значит, в силу 
формулы~(\ref{e2chu}) максимальный угол отклонения от нормали не 
должен превышать~60$^\circ$.
     
     Пусть $\alpha$~--- максимальный угол наблюдения для любого малого 
элемента поверхности экрана из некоторой точки пространства, 
расположенной со стороны активной поверхности экрана. Ясно, что этот 
максимум в горизонтальной плоскости будет достигаться возле правого или 
левого края экрана. Ввиду вышесказанного угол~$\alpha$ не должен 
превосходить~60$^\circ$.

\end{multicols}

\begin{figure} %fig3
\vspace*{1pt}
\begin{center}
\mbox{%
\epsfxsize=97.554mm
\epsfbox{chu-3.eps}
}
\end{center}
\vspace*{-6pt}
\Caption{Область наилучшего наблюдения экрана
\label{f3chu}}
\end{figure}

\begin{figure} %fig4
\vspace*{1pt}
\begin{center}
\mbox{%
\epsfxsize=97.554mm
\epsfbox{chu-4.eps}
}
\end{center}
\vspace*{-6pt}
\Caption{Геометрическая модель области наилучшего наблюдения
\label{f4chu}}
\vspace*{9pt}
\end{figure}

\begin{multicols}{2}

%\begin{center} %fig4
%%\vspace*{6pt}
%\mbox{%
%\epsfxsize=80mm
%\epsfbox{chu-4.eps}
%}
%\end{center}
%\vspace*{4pt}
%\begin{center}
%{{\figurename~4}\ \ \small{Область наилучшего наблюдения экрана}}
%\end{center}
%%\vspace*{9pt}

%\bigskip
%\addtocounter{figure}{1}
     
     Получается, что область наилучшего наблюдения заключена внутри 
угла пересечения двух лучей, проведенных из крайних точек экрана под 
углом~$\alpha$ к нормали. С~учетом всех ограничений область наилучшего 
наблюдения будет являться фигурой пересечения двух уже построенных 
областей (рис.~4).



    Точки~$A$, $B$, $C$ и~$E$ имеют следующие коорди\-наты:
    \begin{align*}
&A\left( 0;\,\fr{W\ctg\alpha}{2}\right)\,;
\end{align*}

\noindent
\begin{align*}
&B\left(\fr{D\sin 2\alpha-
W}{2};\,D\cos^2\alpha\right)\,;\\
& C\left(0;\,\fr{D+\sqrt{D^2- W^2}}{2}\right)\,; \\
&  E\left( \fr{-D\sin 2\alpha +W}{2};\,D\cos^2\alpha\right)\,.
\end{align*}

Площадь области $ABCE$ можно оценить снизу с помощью 
треугольников~$ABC$ и~$ACE$. Площадь каждого из них может быть 
посчитана по формуле

\noindent
\begin{multline*}
S_{\triangle}=\fr{1}{2}AC\cdot BH ={}\\
{}=\fr{1}{8}\left( D\sin 2\alpha -
W\right)\left(D-W\ctg\alpha+\sqrt{D^2-W^2}\right),
%\label{e4chu}
\end{multline*}
где $BH$~--- высота треугольника~$ABC$, опущенная из вершины~$B$ на 
сторону~$AC$. С~учетом того, что таких треугольника два, получаем оценку 
площади области наилучшего наблюдения:
\begin{multline}
S_{\mathrm{о.н.н.}}=\fr{D^2}{4}\left(\sin2\alpha-C\right)\times{}\\
{}\times\left(1-
C\ctg\alpha+\sqrt{1-C^2}\right) ={}\\
{}=
\fr{W^2}{4C^2}\left(\sin2\alpha-C\right)
\left(1-C\ctg\alpha+\sqrt{1-C^2}\right)\,,
\label{e5chu}
\end{multline}
где
$C=W/D$~--- константа системы, которая, как будет видно далее, зависит от 
информативности контента~$I$, отношения~$k$ высоты экрана к ширине, 
отношения~$p$ ширины знака к его высоте и угла~$\psi$, стягиваемого 
одним символом.

\subsection{Число рабочих мест в~области наилучшего наблюдения} %3.2

Пусть $D$ и~$W$ зафиксированы. Тогда определена область наилучшего 
наблюдения и можно оценить максимальное число человек~$N$, которые 
смогут находиться в области наилучшего наблюдения, используя один 
коллективный экран. Главным условием комфортной работы условимся 
считать отсутствие помех со стороны других пользователей на расстоянии 
вытянутой руки. Таким образом, каждому пользователю сопоставим круг 
радиусом, равным половине маховой сажени (маховая сажень $\approx 
1{,}8$~м). Наиболее плотной расстановкой этих кругов на плоскости будет 
расположение, когда каж\-дый из кругов касается других шести. При этом 
центры кругов образуют сетку из равносторонних треугольников со 
стороной, равной одной маховой сажени, или 1,8~м (рис.~5).


Для оценки числа точек равномерной треугольной сетки, которые могут 
попасть внутрь области
 наилучшего наблюдения, воспользуемся формулой
Пика, связывающей число узлов квадратной сетки
 с шагом~1, попавших 
внутрь и на границу многоугольника, с площадью этого 
многоугольника~\cite{17chu}:
\begin{equation*}
S=B+\fr{\Gamma}{2}-1\,,
%\label{e7chu}
\end{equation*}
где $S$~--- целочисленная площадь многоугольника, $B$~--- число узлов 
квадратной сетки с шагом~1, попавших внутрь многоугольника; 
$\Gamma$~--- число узлов этой сетки, которые попали на границу.
\begin{center} %fig5
%\vspace*{6pt}
\mbox{%
\epsfxsize=73.141mm
\epsfbox{chu-5.eps}
}
\end{center}
\vspace*{4pt}
\begin{center}
{{\figurename~5}\ \ \small{Сетка из центров кругов}}
\end{center}
%\vspace*{9pt}

%\bigskip
\addtocounter{figure}{1}



Треугольную сетку с шагом~1,8 можно рас\-смат\-ри\-вать как квадратную сетку 
с шагом~1, которая претерпела два преобразования:
\begin{itemize}
\item сжатие по направлению вектора, параллельного диагонали квадрата 
сетки с коэффициентом~$\sqrt{2}$;
\item растяжение по двум любым взаимно перпендикулярным осям с 
коэффициентом~1,8.
\end{itemize}

Поэтому для применения формулы Пика необходимо рассматривать не 
исходное значение\linebreak площади, а ее преобразование, обратное преобразованиям 
сетки. То есть в случае равномерной треугольной сетки с шагом~1 будет 
выполнено следующее соотношение:
\begin{equation}
\fr{\sqrt{2}}{1{,}8^2}\,S=B+\fr{\Gamma}{2}-1\,.
\label{e8chu}
\end{equation}
Число рабочих мест в области наилучшего наблюдения при этом составляет
\begin{equation}
N=B+\Gamma=\fr{\sqrt{2}}{1{,}8^2}\,S+1+\fr{\Gamma}{2}\,.
\label{e9chu}
\end{equation}
Таким образом, для получения оценки числа рабочих мест в области 
наилучшего наблюдения необходимо оценить максимальное 
значение~$\Gamma$. Важно понимать, что формула Пика описывает 
площадь многоугольника с вершинами в узлах сетки. Ясно, что в случае 
области наилучшего наблюдения такое обеспечить не всегда возможно. 
Поэтому формула~(\ref{e9chu}) представляет собой верхнюю оценку  
числа рабочих мест.

Из оценки периметра области наилучшего наблюдения получаем, что
\begin{multline*}
\Gamma\leq \fr{1}{1{,}8}\left(2\vert AB\vert 
+2\widehat{CB}\right)={}\\
{}=\fr{D}{1{,}8}\left(\fr{\sin2\alpha-C}{\sin\alpha}+2\alpha-
\arcsin C\right)\,.
%\label{e10chu}
\end{multline*}
Таким образом, число рабочих мест в области наилучшего наблюдения 
может быть оценено сверху следующей величиной:
\begin{multline}
N=B+\Gamma=0{,}44S+{}\\
{}+\fr{D}{3{,}6}\left( \fr{\sin2\alpha-C}{\sin\alpha}+2\alpha-
\arcsin C\right)+1\,.
\label{e11chu}
\end{multline}

\subsection{Оценка максимальной площади области наилучшего 
наблюдения и~максимального числа рабочих мест~в~этой области}

Пусть~$I$ определено. Тогда исходя из пространственных ограничений, 
характеризуемых диа\-мет\-ром~$P$, можно оценить максимальную площадь 
области наилучшего наблюдения при некоторых значениях проектного 
расстояния~$D$ и ширины экрана~$W$.

Ввиду того, что соотношение сторон экрана у большинства производителей 
на настоящий момент составляет от~16:9 до~16:12, можно исключить из 
рассмотрения параметр высоты экрана. Дальнейшие оценки получаются в 
результате подсчета с двух сторон площади активной части экрана
\begin{equation*}
S_{\mathrm{display}}=WH=Iwh\,.
%\label{e12chu}
\end{equation*}
Пусть
\begin{equation*}
H=kW\,; %\label{e13chu}
\quad w=ph\,, %\label{e14chu}
\end{equation*}
где $k\in [9/16;\,12/16]$, $p\in [0{,}6;\,0{,}9]$ (см.\ п.~2.3). %Значение~$p$ соответствует  табл.~2.
Тогда
\begin{equation}
kW^2=Iph^2\ \ \mbox{или}\ \ W=h\sqrt{\fr{Ip}{k}}\,.
\label{e15chu}
\end{equation}
Из формулы~(\ref{e1chu}) следует, что
\begin{equation}
h=D\psi\,.
\label{e16chu}
\end{equation}

Из формул~(\ref{e15chu}) и~(\ref{e16chu}) получается, что при 
фиксированных значениях информативности~$I$, отношения~$k$ высоты 
экрана к ширине, отношения $p$ ширины знака к его высоте и угла~$\psi$, 
стягиваемого одним символом, существует константа~$C$, 
удовлетворяющая следующим соотношениям:
\begin{equation}
C=\fr{W}{D}=\psi\sqrt{\fr{Ip}{k}}\,.
\label{e17chu}
\end{equation}
Из формулы~(\ref{e17chu}) следует, что
\begin{equation}
C_{\min} 
=\psi_{\min}\sqrt{\fr{p_{\min}}{k_{\max}}}\,\sqrt{I}=\fr{\sqrt{I}}{193}\,.
\label{e18chu}
\end{equation}
     
     Из-за ограниченности помещения проектное расстояние~$D$ не может 
превышать диаметра помещения~$P$, поэтому ввиду оценок~(\ref{e5chu}) 
и~(\ref{e18chu})
     \begin{multline}
     S_{\max} ={}\\
     {}=\fr{D^2}{4}\left(\sin2\alpha-C\right)\left(1-C\ctg\alpha+\sqrt{1-
C^2}\right)\leq{}\\
     {}\leq \fr{P^2}{2}\left(\sin2\alpha-\fr{\sqrt{I}}{193}\right)\times{}\\
     {}\times\left(1-
\fr{\ctg \alpha\cdot \sqrt{I}}{193}+\sqrt{1-\fr{I}{193^2}}\right)\,.
     \label{e19chu}
     \end{multline}
     
     Отметим, что информативность не должна превосходить 
$(193\sin2\alpha)^2$~знаков. В~противном случае область наилучшего 
наблюдения окажется пустым множеством.

На основании формулы~(\ref{e19chu}) можно оценить число рабочих мест, 
которые можно разместить в области наилучшего наблюдения с учетом 
возможностей помещения, ограниченных его диа\-мет\-ром~$P$. Согласно 
оценкам~(\ref{e11chu}) и~(\ref{e19chu})
\begin{multline}
N_{\max}=0{,}44S_{\max}+\fr{P}{3{,}6}\left(\fr{\sin2\alpha-\sqrt{I}/193}{\sin\alpha}+{}\right.\\
\left.{}+2\alpha-
\arcsin\fr{\sqrt{I}}{193}\right)+1\,.
\label{e20chu}
\end{multline}

\subsection{Оценка минимальной ширины активной поверхности 
экрана}

Чем меньше размер экрана, тем меньше его стои\-мость при прочих равных 
условиях. Поэтому размеры экрана, в том числе и его ширина, должны быть 
по возможности минимизированы. Пусть известна информативность 
контента~$I$, угол наблюдения~$\alpha$ и диаметр помещения~$P$. 
Рассмотрим два принципиально разных случая:
\begin{enumerate}[1.]
\item Число наблюдателей неизвестно, необходимо оценить размеры 
экрана (его ширину), позволяющие эффективно использовать 
пространство помещения.
\item Число наблюдателей~$N$ известно, но оно меньше, чем полученное 
по формуле~(\ref{e20chu}) для известных параметров~$P$, $I$ и~$\alpha$. 
Необходимо оценить минимальные размеры экрана (его ширину), 
позволяющие обеспечить расположение всех наблюдателей в области 
наилучшего наблюдения с учетом отсутствия взаимных помех.
\end{enumerate}

\smallskip

\noindent
\textbf{Случай 1.} Из соотношения~(\ref{e17chu}) следует, что
\begin{equation*}
W= PC\geq PC_{\min}=\fr{P\sqrt{I}}{193}\,.
%\label{e21chu}
\end{equation*}

\smallskip

\noindent
\textbf{Случай 2.} Оценим минимальную площадь, на которой могут 
разместиться $N$~наблюдателей. Для этого воспользуемся тем 
соображением, что минимальная площадь равна количеству внутренних 
рабочих мест (тех, которые не попали на границу), умноженному на площадь 
двух треугольников, ограниченных сеткой. Такое соображение следует из 
того, что каждой внутренней точке можно сопоставить 6~треугольников 
сетки, а каждому треугольнику сетки~--- 3~точки. Это значит, что каждой 
точке сопоставляется по 2~треугольника.

Из формул~(\ref{e8chu}) и~(\ref{e9chu}) следует, что
\begin{equation*}
B=\fr{2\sqrt{2}}{1{,}8^2}\,S+2-N\,.
%\label{e22chu}
\end{equation*}
С другой стороны, как было замечено,
\begin{equation*}
S=B\cdot 2S_{\triangle}\,.
%\label{e23chu}
\end{equation*}
Поэтому
\begin{equation*}
S_{\min} =(N-2)\fr{2\cdot 1{,}8^2\cdot1{,}4}{4\sqrt{2}\cdot1{,}4-
1{,}8^2}=1{,}94(N-2)\,.
%\label{e24chu}
\end{equation*}

Из формулы~(\ref{e5chu}) получаем:
\begin{multline*}
W_{\min} =2C_{\min}
\left(
{S_{\min}}\Big /\left(\vphantom{\sqrt{1-C^2_{\min}}}(
\sin2\alpha-C_{\min})\times{}\right.\right.\\
\left.\left.{}\times(1-
C_{\min}\ctg\alpha+\sqrt{1-C^2_{\min}})\right)\right)^{1/2}={}\\
{}=2
\left(
{1{,}94I(N-2)}\Big /\left(\vphantom{\sqrt{193^2-I}}
(193\sin2\alpha-\sqrt{I})\times{}\right.\right.\\
\left.\left.{}\times (193-
\sqrt{I}\ctg\alpha+\sqrt{193^2-I})\right)\right)^{1/2}\,.
%\label{e25chu}
\end{multline*}


\section{Заключение}

В статье сформулированы основные определения и термины, используемые 
при проектировании средств отображения информации, а также описаны 
рекомендации к ним, действующие в рамках государственных стандартов.

На основании этих требований и рекомендаций, а также на основании 
простых геометрических соображений получены оценки основных 
параметров для проектирования средства отображения и рабочих мест. 
В~качестве основной определяющей величины выступает информативность 
контента, которая формируется на основании задач ситуационного зала и 
ситуационных моделей отображения.
     
     В качестве основного использован параметр отношения ширины экрана 
к проектному расстоянию наблюдения. Часто этот параметр для удобства 
считается фиксированным, но в общем случае показано, что это отношение 
прямо пропорционально корню квадратному от величины информативности 
контента.
    
Если, помимо информативности контента, известен диаметр помещения, в 
котором располагается ситуационный зал, то можно определить 
максимальную площадь области наилучшего наблюдения и оценить число 
рабочих мест, которые могут быть расположены в области наилучшего 
наблюдения с помощью формулы Пика и аффинных преобразований.

В качестве обратной задачи, в которой известно количество человек, 
одновременно работающих с экраном, получена зависимость для 
определения минимально необходимой ширина экрана.

{\small\frenchspacing
{%\baselineskip=10.8pt
\addcontentsline{toc}{section}{Литература}
\begin{thebibliography}{99}


\bibitem{1chu}
\Au{Ильин Н.\,И.}
Основные направления развития ситуационных центров органов государственной 
власти~// ВКСС Connect! (Ведомственные корпоративные сети и системы), 2007. 
№\,6(45). С.~2--9.

\bibitem{2chu}
\Au{Лисица К.\,В.}
Опыт создания и применения Автоматизированной системы стратегического 
управления в ОАО <<Российские железные дороги>>~// Ситуационные центры: 
модели, технологии, опыт практической реализации: Мат-лы науч.-практич. 
конф.~--- М.: РАГС, 2007.

\bibitem{4chu} %3
\Au{Филиппович А.\,Ю.}
Ситуационная система~--- что это такое?~// PCWeek/RE, 2003. No.\,26.

\bibitem{6chu} %4
\Au{Зацаринный А.\,А., Ионенков Ю.\,С., Кондрашев~В.\,А.}
Об одном подходе к выбору системотехнических решений построения 
информационно-те\-ле\-ком\-му\-никационных систем~// Системы и средства информатики. 
Вып.~16.~--- М.: Наука, 2006. С.~65--72.


\bibitem{3chu} %5
\Au{Зацаринный А.\,А., Сучков А.\,В., Босов~А.\,В.}
Ситуа\-ционные центры в современных информационно-те\-ле\-ком\-му\-ни\-кационных 
системах специального\linebreak назначения~// ВКСС Connect! (Ведомственные корпоративные 
сети и системы), 2007. №\,5(44). С.~64--76.

\bibitem{7chu} %6
\Au{Зацаринный А.\,А., Ионенков Ю.\,С.}
Некоторые аспекты выбора технологии построения информационно-те\-ле\-ком\-муникационных сетей~// 
Системы и средства информатики. Вып.~17.~--- М.: 
Наука, 2007. С.~5--16.


\bibitem{5chu} %7
\Au{Зацаринный А.\,А.}
Тенденции развития ситуационных центров как компонентов информационно-те\-ле\-ком\-му\-ни\-ка\-ционных 
систем в условиях глобальной информатизации общества~// 
Докл. XXXV\linebreak междунар. конф. <<Информационные технологии в науке, 
образовании, телекоммуникации и бизнесе (IT\;+\;S\&E'08)>>. Ялта--Гурзуф, Украина. 
2008.


\bibitem{8chu}
Ситуационные центры (СЦ) и их история. {\sf http://\linebreak ta.interrussoft.com/s\_centre.html}.

\bibitem{9chu}
\Au{Зацаринный А.\,А.}
Организационные принципы сис\-тем\-но\-го подхода к разработке, проектированию и 
внедрению современных информационно-те\-ле\-ком\-му\-никационных сетей~// ВКСС 
Connect! (Ведомственные корпоративные сети и системы), 2007. №\,1(40). С.~60--67.

\bibitem{11chu} %10
ГОСТ 26387-84 Система <<Человек--машина>>. Термины и определения.~--- М.: 
Стандартинформ, 2006.

\bibitem{12chu} %11
ГОСТ 27833-88 Средства отображения информации. Термины и определения.~--- М.: 
Стандартинформ, 2005.

\bibitem{10chu} %12
ГОСТ Р~52324-2005 (ИСО 13406-2:2001) Эргономические требования к работе с 
визуальными дисплеями, основанными на плоских панелях.~--- М.: Стандартинформ, 
2005.

\bibitem{13chu}
ГОСТ 12.2.032-78 Система стандартов безопасности труда. Рабочее место при 
выполнении работ сидя. Общие эргономические требования.~--- М.: Изд-во 
стандартов, 2001.

\bibitem{14chu}
ГОСТ 12.2.033-78 Система стандартов безопасности труда. Рабочее место при 
выполнении работ стоя. Общие эргономические требования.~--- М.: Изд-во 
стандартов, 2001.

\bibitem{15chu}
ГОСТ Р ИСО 9241-3 Эргономические требования при выполнении офисных работ с 
использованием видеодисплейных терминалов.~--- М.: Изд-во стандартов, 2003.

\bibitem{16chu}
ГОСТ 21958-76 Зал и кабины операторов. Взаимное расположение рабочих мест.~--- 
М.: Изд-во стандартов, 1976.

\label{end\stat}

\bibitem{17chu}
\Au{Прасолов В.\,В.}
Задачи по планиметрии.~--- М.: \mbox{МЦНМО}, 2001.


 \end{thebibliography}
}
}


\end{multicols}