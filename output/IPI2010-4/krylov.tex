\def\stat{krylov}

\def\tit{МОДЕЛИРОВАНИЕ И КЛАССИФИКАЦИЯ МНОГОКАНАЛЬНЫХ ДИСТАНЦИОННЫХ
ИЗОБРАЖЕНИЙ С~ИСПОЛЬЗОВАНИЕМ КОПУЛ$^*$}

\def\titkol{Моделирование и классификация многоканальных дистанционных
изображений с~использованием копул}

\def\autkol{В.\,А.~Крылов}
\def\aut{В.\,А.~Крылов$^1$}

\titel{\tit}{\aut}{\autkol}{\titkol}

{\renewcommand{\thefootnote}{\fnsymbol{footnote}}\footnotetext[1]
{Работа выполнена при поддержке ФЦП <<Научные и научно-педагогические кадры инновационной России>> 
на 2009--2013~гг.}}

\renewcommand{\thefootnote}{\arabic{footnote}}
\footnotetext[1]{Московский государственный университет им.\ М.\,В.~Ломоносова, факультет 
вычислительной математики и кибернетики, vkrylov@cs.msu.ru}

\Abst{Предложен метод моделирования многоканальных дистанционно полученных изображений (со спутника, самолета) с использованием копул.
    Суть предлагаемого подхода состоит в применении традиционных статистических моделей для моделирования вероятностных распределений отдельных каналов
    и построении совместного распределения для многоканального изображения при помощи копул.
    Рассмотрено также применение разработанного подхода в модели марковского случайного поля 
    (МСП) для байесовской классификации.
    Эксперименты с реальными изображениями, полученными радаром с синтезированной апертурой, демонстрируя результаты высокой точности, 
    указывают на
    преимущества предложенного подхода по сравнению с существующими методами.}

\KW{многоканальное изображение; копула; марковское случайное поле; байесовская классификация}

       \vskip 14pt plus 9pt minus 6pt

      \thispagestyle{headings}

      \begin{multicols}{2}

      \label{st\stat}


\section{Введение}


Статистический анализ изображений является на сегодняшний день активно развивающейся областью. 
Одним из важных и активно исследуемых источников изображений
служат спутники и радары, предоставляющие данные, широко исполь-\linebreak зуемые в задачах картографии, 
риск-менеджмен-\linebreak та (пожары, наводнения), эпидемиологии и~др.\linebreak
Современное оборудование позволяет получать многоканальные изображения, число каналов в 
которых зачастую доходит до сотен (мультиспектральные изображения).
Для обработки таких данных требуется адаптация существующих методов 
обработки отдельных каналов для работы с многоканальными изображениями. В~данной работе
рассматривается возможный статистический подход к решению подобной задачи многомерного моделирования.

На практике для решения задач требуется иметь аккуратную статистическую модель рас\-смат\-ри\-ва\-емых данных, 
в случае многоканальных изоб\-ра\-жений~--- многомерную модель. Большинство сущест\-ву\-ющих в литературе подходов 
строится в предположении, что вектор многоканальных данных имеет какое-то специфическое распределение,
например подчиняется многомерному нормальному распределению~\cite{Landgrebe} для мультиспектрального 
вектора или многомерному распределению Вишарта (Wishart)~\cite{2DNG} для комплексного мультиполяризованного 
микроволнового изображения. Недостаток подобных подходов состоит в большой потере точности при
повышении размерности из-за присутствия шума, погрешностей регистрации данных и аппроксимации. 
В~рамках статистического подхода адекватным способом решения данной
проблемы является использование более гибких моделей для построения многомерных распределений. 
Одной из таких моделей являются копулы~\cite{Nelsen}.

В работе также рассматривается одна из клас\-сиче\-ских проблем обработки дистанционно полученных изображений~--- 
задача классификации.\linebreak
Многомерные вероятностные распределения применяются совместно с моделью марковского случайного поля для 
байесовской классификации. Эксперименты с реальными изображениями, полученными радаром с синтезированной 
апертурой, демонстрируют преимущества модели с использованием копул.

Статья организована следующим образом. В~разд.~2 приводится изложение основ теории копул. В~разд.~3 
строится модель многомерных распределений с использованием
копул. В~разд.~4 излагается построение модели марковского случайного поля в задаче классификации. В~разд.~5 
приводятся эксперименты по применению разработанной модели к изображениям, полученным радаром с синтезированной 
апертурой. Раздел~6 содержит заключение.


\section{Копулы}

В этом разделе приводится краткий обзор тео\-рии двумерных копул. Все сформулированные ниже результаты 
могут быть обобщены на многомерный случай~\cite{Nelsen}.

Двумерная копула является вероятностным распределением на $[0, 1]^2$ таким, что маргинальные распределения 
распределены равномерно на~$[0,1]$.

\smallskip

\noindent
\textbf{Определение.} \textit{Двумерной копулой называется отображение $C: [0,1]^2 \to [0,1]$, такое что:}
\begin{enumerate}[1)]
\item \textit{оба маргинальных распределения являются равномерно распределенными с.в.\ на $[0,1]$;}
\item
$\forall u, v \in [0,1]$: $C(u,0) = C(0,v) = 0$ и $C(u,1)=$\linebreak $=u,\,C(1,v) = v$;
\item
$\forall u_1 \leq u_2$, $v_1 \leq v_2 \in [0,1]$: $C(u_2,v_2) - C(u_1,v_2)\;-$\linebreak $-\;C(u_2, v_1) + C(u_1, v_1) \geq 0$.
\end{enumerate}

\smallskip

Теоретическое обоснование правомерности использования копул в прикладных задачах обеспечивает 
теорема Склара~\cite{Nelsen}:

\smallskip

\noindent
\textbf{Теорема.} \textit{Пусть $X$, $Y$~--- произвольные случайные величины (с.в.)
с совместным распределением~$H(x,y)$ и маргинальными функциями
распределения~$F$ и~$G$. Тогда существует копула~$C$ такая, что
\begin{equation}
\label{Sklar}
H(x,y) = C\left(F(x), G(y)\right)
\end{equation}
$\forall x, y \in \mathbb R$. Если~$F$ и~$G$ непрерывны, то такая копула~$C$ единственна.}

\smallskip

Если $X$, $Y$ имеют плотности $f(x)$ и $g(y)$, то плотность совместного распределения~(\ref{Sklar}) можно представить в виде
\begin{equation}\label{Sklar_pdfs}
h(x,y) = f(x)g(y)\frac{\partial^2 C}{\partial x \partial y}(F(x),G(y)),
\end{equation}
где частная производная $\frac{\partial^2 C}{\partial x  \partial y}(x,y)$ задает плотность копулы $C(x,y)$.

На практике широкое применение получил класс архимедовых копул~\cite{Nelsen}.

\smallskip

\noindent
\textbf{Определение.} \textit{Архимедовой копулой называется копула $C$ вида}
$$
C(u_1, u_2) = \phi^{-1}(\phi(u_1) + \phi(u_2))\,,
$$
\textit{где функция $\phi(u)$, называемая \textit{функцией-ге\-не\-ра\-то\-ром}, удовлетворяет следующим требованиям:
(1)~$\phi(u)$ непрерывна на $[0,1]$;\quad (2)~$\phi(u)$ монотонно убывает, $\phi(1)=0$; (3)~$\phi(u)$ выпуклая.}

\smallskip

В рассматриваемых задачах моделирования и классификации
дистанционных изображений предлагается использовать следующие
копулы: пять архимедовых копул (Clayton, Ali-Mikhail-Haq, Gumbel,
Frank, A12)~\cite{Nelsen} и одну неархимедову копулу, содержащую
абсолютно непрерывную и сингулярную компоненту
(Marchal--Olkin)~\cite{Nelsen}. Такой набор копул обеспечивает
достаточно широкий выбор моделируемых структур
зависимости~\cite{Huard}, тем не менее для некоторых рассматриваемых
данных этот набор копул может требовать пополнения. Эксперименты с
более обширным набором копул проводились в~\cite{KrylovRR09}.
Информация о рассматриваемых копулах собрана в табл.~1.

\begin{table*}\small
\begin{center}
\Caption{Рассматриваемые копулы: Clayton,
Ali-Mikhail-Haq (AMH), Gumbel, Frank, A12 и Marchal--Olkin (Marchal)
\label{CopulasTheta}}
\vspace*{2ex}

\tabcolsep=10.5pt
\begin{tabular}{|l|l|l|l|}
\hline
Копула &  \multicolumn{1}{c|}{$C(u,v)$} & 
\multicolumn{1}{c|}{$\theta(\tau)$} & \multicolumn{1}{c|}{$\tau$-интервал}\\ 
\hline
Clayton     &
$(u^{-\theta}+v^{-\theta}-1)^{-1/\theta}$ &
$\theta = \fr{2\tau}{1-\tau}$ &
$\tau\in(0,1]$ \\
\hline
&&&\\[-10pt] 
AMH & $\fr{uv}{1-\theta(1-u)(1-v)}$ &
$\tau = \fr{3\theta-2}{3\theta} - \fr{2(1-\theta)^2}{3\theta^2}\ln \left(1-\theta\right)$ &
$\tau\in\left[-0{,}1817, \fr{1}{3}\right]$ \\ 
\hline
&&&\\[-10pt] 
Gumbel & $\exp\left(-\left[ \left(-\ln\left(u\right)\right)^{\theta} + (-\ln\left(v\right))^{\theta} 
\right]^{1/{\theta}}\right)$ & $\theta = \fr{1}{1-\tau}$ & $\tau\in[0,1]$
\\ 
\hline
Frank & $- \fr{1}{\theta}\log\left(1+\fr{(e^{-\theta u}-1)(e^{-\theta v}-1)}{e^{-\theta}-1}\right)$ &
$\tau = 1 - \fr{4}{\theta^2}\int\limits_0^{\theta}\fr{t}{e^{-t}-1}\,dt$ &
$\tau\in[-1,1]\setminus\{0\}$\\ 
\hline
A12      & $\left(1+\left[ (u^{-1}-1)^{\theta} + (v^{-1}-1)^{\theta} \right]^{1/{\theta}}\right)^{-1}$ &
$\theta = \fr{2}{3-3\tau}$ &
$\tau\in[\fr{1}{3},1]$ \\
 \hline
Marchal & $\min\left( u^{1-\theta}v, uv^{1-\theta}\right)$ &
$\theta = \fr{2\tau}{\tau+1}$ &
$\tau\in[0,1]$\\ 
\hline
\end{tabular}
\end{center}
\vspace*{-4pt}
\end{table*}


Простейшим инструментом для оценки копул является \textit{коэффициент ранговой корреляции Кендалла}~$\tau$ 
двух независимых реализаций $(Z_1, Z_2)$ и $(\hat{Z_1}, \hat{Z_2})$ с общим законом
распределения~$H(x,y)$: $\tau =  \mathbb P\{(Z_1 - \hat{Z_1})(Z_2 -
\hat{Z_2}) > 0\} -  \mathbb P\{(Z_1 - \hat{Z_1})(Z_2 - \hat{Z_2}) < 0\}$. 
На практике при наличии реализаций~$z_{1,l}$ и~$z_{2,l}$,
$l=1,\ldots,N$, эмпирической оценкой коэффициента Кендалла
является~\cite{Nelsen}:
\begin{equation*}
%\label{TauHat}
\hat{\tau} =  \fr{\sum_{l=1}^{N-1}\sum_{k=l+1}^{N}z_{1,lk}\:z_{2,lk}}{C_2^N}\,,
\end{equation*}
где
$$
z_{n,lk} = \begin{cases}
1\,, &\mbox{если}\  z_{n,l} \leq z_{n,k}\\
-1&\mbox{---~иначе}
\end{cases}\,,\enskip n=1,2\,.
$$

Интегрируя в определении~$\tau$ по $(\hat{Z_1}, \hat{Z_2})$, имеем
\begin{equation}
\left.
\begin{array}{rl}
\tau &= 4\displaystyle\int\limits_0^1\int\limits_0^1 C(u,v)\,dC(u,v)-1\,;\\[9pt]
\tau &= 1 + 4\displaystyle\int\limits_0^1\fr{\phi_A(t)}{\phi_A{'}(t)}\,dt\,,
\end{array}
\right \}
\label{TauC}
\end{equation}
где вверху получен общий вид зависимости между~$\tau$ и~$C$, а внизу~--- 
зависимость в частном случае архимедовых копул с
функ\-ци\-ей-ге\-не\-ра\-то\-ром~$\phi_A(t)$. Все рассматриваемые в работе
копулы однопараметрические, оценка их параметра~$\theta$ может быть
получена из~(\ref{TauC}) (см.\ табл.~1).

\section{Моделирование совместных распределений}


Рассматривается задача построения~$M$ условных распределений
изображения с $D$ каналами. В~центре внимания в данной работе
находится моделирование совместных распределений, поэтому
предполагается что маргинальные распределения
$p_{dm}(y_d|\omega_m)$, $d=1,\ldots,D$, $m=1,\ldots,M$, где
$\omega_m$~--- событие, состоящее в попадании наблюдения в класс с
номером~$m$, уже получены каким-либо методом (включая оценку
параметров). Таким образом, обобщая~(\ref{Sklar_pdfs}), имеем
совместную плотность вида
\begin{multline}
p_m( {\bf y}|\omega_m) = p_{1m}( y_1|\omega_m)\ldots\\
\ldots p_{Dm}( y_D|\omega_m)
\fr{\partial^D C_m^*}{\partial y_1 \ldots \partial y_D}(F_{1m}(y_1),\ldots\\
\ldots , F_{Dm}(y_D))\,.
\label{Model3}
\end{multline}

Следующим вопросом является выбор конкретной копулы для каждого класса. 
В~литературе\linebreak предложен целый ряд методов для решения этой проблемы. Так, для этого
применяются подходы, основанные на применении информационных критериев (в частноcти, 
Акаике)~\cite{Nelsen}. В~\cite{Kplots}
разработан метод для визуального анализа (K-plots) адекват\-ности копулы реальным данным. 
В~\cite{Huard} применяется функция правдоподобия.
В данной работе предлагается использовать критерий согласия~$\chi^2$ Пирсона (КСП) для 
нахождения наиболее подходящей копулы.

Итак, для выбора копулы~$C_m^*$ из~(\ref{Model3}) среди копул в
табл.~1 сначала отбрасываются копулы, эмпирическое $\hat{\tau}_m$
которых лежит вне соответствующего им $\tau$-интервала. Затем для
оставшихся копул оценивается значение параметра~$\theta$ и
выбирается копула, лучше всего согласующаяся с наблюдениями на
основе КСП. Нулевая гипотеза КСП состоит в согласии наблюдаемых
частот с частотами, предсказанными теоретической моделью. Статистика
КСП имеет следующий вид:
$$
X^2 = \sum\limits_{i=1}^N \fr{(O_i - E_i)^2}{E_i}\,,
$$
где $O_i$ и $E_i$~--- наблюдаемые и предсказанные частоты; $n$~--- число классов.
$P$-значение КСП определяется из сравнения наблюдаемого значения~$X^2$ с 
$\chi^2$-рас\-пре\-де\-ле\-ни\-ем: $X^2\sim\chi_{n-r-1}^2$, где $r$~--- количество ограничений числа степеней свободы
(количество параметров модели).

\section{Классификация многоканальных изображений}


На базе построенных условных распределений~(\ref{Model3}) самую примитивную классификацию 
можно получить, используя метод максимального\linebreak правдоподобия,
выбирая из $M$ классов для каж\-до\-го пикселя тот, который имеет максимальную вероятность.
Однако такого рода классификация слишком неоднородная, и для решения этой проб\-ле\-мы к правдоподобию добавляется
регуляризи\-рующее слагаемое, обеспечивающее б$\acute{\mbox{о}}$льшую однородность решения. 
Одним из таких способов\linebreak регуляризации является модель МСП~\cite{BesagMRF}. 
Концепция МСП получена обобщением марковского свойства на двумерный случай решетки. 
Классическая теория МСП и скрытых МСП приводится в~\cite{BesagMRF}.
Здесь изложение ограничится кратким введением в скрытые~МСП.

\begin{figure*}[b] %fig1
\vspace*{1pt}
\begin{center}
\mbox{%
\epsfxsize=163.235mm
\epsfbox{kry-1.eps}
}
\end{center}
\vspace*{-6pt}
\Caption{Исходное изображение~(\textit{a}), 
классификация c 2D моделью Накагами--Гамма~(\textit{б}) и классификация с моделированием копулами~(\textit{в}): 
вне карты классификации~--- белое, правильно классифицированная вода~(\textit{1}),
влажная почва~(\textit{2}), сухая почва~(\textit{3}), 
ошибочная классификация~--- черное
\label{f1kr}}
\end{figure*}

Задачу классификации относят к классу задач с неполными данными
$x=(y,z)$, где $y$~--- наблюдаемые данные (исходное изображение, поле~$Y$);
$z$~--- данные, подлежащие восстановлению (метки классов, поле~$Z$). 
В~решении задачи классификации используется модель скрытого
МСП. В этой модели неизвестные метки~$z_i$ предполагаются МСП, т.\,е.\
зависящими от меток \textit{только} соседних пикселей. Теорема
Хам\-мерс\-ли--Клиф\-фор\-да (Hammersley--Clifford)~\cite{BesagMRF}
предоставляет удобное представление для совместного распределения
с.~в., входящих в МСП, в виде распределения Гиббса
\begin{multline}
\label{Gibbs}
P_{G}(z) = W^{-1}\exp(-H(z))\\
\mbox{с энергией}\quad
H(z) = \sum\limits_{c \in C} V_c(z_c)\,,
\end{multline}
где $W$~--- нормирующая константа, $V_c(z_c)$~--- потенциалы;
$C$~--- система множеств (клик) на решетке $S$~\cite{BesagMRF}. В~данной
работе рассматриваются двухместные анизотропные потенциалы (т.\,е.\
попарно с каж\-дым из 8 соседних пикселей):
\begin{equation}
\label{Beta}
H(z|\beta) = 
\sum\limits_{c}V(z_c|\beta) = \!\!\sum\limits_{c=\{s, s'\}\in C} \left[ - \beta\,\delta_{z_s = z_{s'}}\right]\,,\!\!
\end{equation}
где $\delta_{z_s = z_{s'}} = 1,$ если $z_s = z_{s'}$, и 0~--- иначе.
В~(\ref{Beta}) $\beta$ играет роль веса, т.\,е.\ чем больше значение
$\beta$, тем выше вклад регуляризирующего слагаемого $H(z|\beta)$ в
суммарную энергию поля~$(Y,Z)$, что хорошо видно во втором уравнении
системы~(\ref{Energy}), приведенной ниже.

В свою очередь, наблюдения~$y_i$ предполагаются условно независимыми 
$P(y_i|y_{\Omega\setminus\{i\}},z_i) = P(y_i|z_i)$.
Таким образом, при наличии условных вероятностей $p_m( {\bf y}|\omega_m)$ распределение скрытого МСП $(Y,Z)$ 
на~$S$ имеет вид:
\begin{equation}
\label{Energy}
\left. 
\begin{array}{l}
P_{G}(z) = W^{-1}\exp(-U(\omega_m|{\bf y},\beta))\,;\\[9pt]
U(\omega_m|{\bf y},\beta) =\\[9pt]
= \sum\limits_{i\in S} \left[ - \text{ln} p_m( {\bf y}|\omega_m) - %\right.{}\\
%&\left.{}- 
\beta\,\sum\limits_{s:\{i,s\}\in C} \delta_{z_i = z_s}\right]\,,
\end{array}
\right \}
\end{equation}
где первое уравнение определяет энергию~$U(\cdot)$ аналогично~(\ref{Gibbs}), 
а второе уравнение задает~$U(\cdot)$ как сумму энергетических вкладов полей~$Y$ и~$Z$.

Для оценки параметра в~(\ref{Energy}) используется метод имитации отжига~\cite{MRFaccelerated}. 
Для нахождения точки минимума функции
$U(\omega_m|{\bf y},\beta)$ (и, соответственно, решения с максимальным правдоподобием~$P_{G}$)
необходимо решение задачи глобальной оптимизации, например тем же методом имитации отжига, 
однако в связи с высокой вычислительной сложностью для экспериментов
использовалась локальная оптимизация методом ICM~\cite{Besag}.

\vspace*{-12pt}

\section{Эксперименты}

Для проведения экспериментов используется амплитудное изображение,
полученное радаром с синтезированной апертурой системы TerraSAR-X
(\copyright Infoterra), с двумя каналами поляризации и разрешением 6~м на пиксель 
(рис.~\ref{f1kr},\,\textit{a}). Классификация проводится по трем
классам: вода, влажная и сухая почва. В~качестве способа оценки
маргинальных распределений классов применяется алгоритм с
использованием метода конечных смесей~\cite{KrylovSPIE09} на
обучающем изображении с заранее предоставленной классификацией.
Стоит отдельно отметить, что копулы представляются логичным
инструментом для обобщения~\cite{KrylovSPIE09} на многоканальный
случай, не требующим изменения самого метода. Для сравнения также
приводятся результаты классификации при моделировании совместного
распределения моделью двумерного (2D) На\-ка\-га\-ми--Гам\-ма
распределения~\cite{2DNG}.



Доля правильной классификации на рис.~\ref{f1kr} составляет 87,1\% для 2D
На\-ка\-га\-ми--Гам\-ма модели (рис.~\ref{f1kr},\,\textit{б}) и 94,8\% для предлагаемой модели с
использованием копул (рис.~\ref{f1kr},\,\textit{в}). При помощи КСП были выбраны
следующие копулы: Gumbel~--- для воды и Frank~--- для сухой и влажной
почвы. Такое повышение качества классификации возможно благодаря
более точному моделированию совместных распределений,
предоставляемому копулами. Дополнительные эксперименты и обсуждение
можно найти в~\cite{KrylovRR09}.

\section{Заключение}


В работе предлагается подход к моделированию распределений многоканальных изображений с использованием копул.
По сравнению с классическими методами оценки многомерных распределений конкретного вида, использование копул 
предос\-тав\-ля\-ет большую гибкость, позволяя получать более аккуратные модели.
Предложенная модель применяется для классификации дистанционно полученных изображений.
Эксперименты с изображениями, полученными радаром с синтезированной апертурой, подтверждают 
высокую описательную точность предложенного подхода.

\smallskip
Использованное в работе изображение системы TerraSAR-X распространяется на свободной 
основе (\copyright Infoterra, {\sf www.infoterra.de}).

\bigskip
Автор выражает благодарность своему научному руководителю доц.\
В.\,Ф.~Матвееву , а также проф.\ Г.~Мозеру, проф.\ С.~Серпико и проф.\
Д.~Зерубии за помощь в подготовке этой работы.


{\small\frenchspacing
{%\baselineskip=10.8pt
\addcontentsline{toc}{section}{Литература}
\begin{thebibliography}{99}

\bibitem{Landgrebe}
\Au{Landgrebe D.\,A.}
Signal theory methods in multispectral remote sensing.~--- N.-Y.: Wiley, 2003.

\bibitem{2DNG}
\Au{Lee J.-S., Hoppel K.\,W., Mango S.\,A., Miller A.\,R.}
Intensity and phase statistics of multilook polarimetric and
interferometric SAR imagery~// IEEE Trans. Geosci. Remote Sens.,
1994. Vol.~32. No.\,10. P.~1017--1028.

\bibitem{Nelsen}
\Au{Nelsen R.\,B.}
An introduction to copulas.~--- N.-Y.: Springer, 2007.

\bibitem{Huard}
\Au{Huard D., $\acute{\mbox{E}}$vin G., Favre A.-C.} Bayesian copula
selection~// Comput. Stat. Data Anal., 2006. Vol.~51. No.\,2.
P.~809--822.

\bibitem{KrylovRR09}
\Au{Krylov V., Zerubia J.}
High resolution SAR image classification.
INRIA, Research Report 7108, 2009.

\bibitem{Kplots}
\Au{Genest C., Boies J.-C.} Detecting dependence with Kendall
plots~// The American Statistician, 2003. Vol.~57. No.\,4.
P.~275--284.

\bibitem{BesagMRF}
\Au{Besag J.} Spatial interaction and the statistical analysis
of lattice systems~// J. Roy. Statistical Soc. B,
1974. Vol.~36. No.\,2. P.~192--236.

\bibitem{MRFaccelerated}
\Au{Yu Y., Cheng Q.} 
MRF parameter estimation by an accelerated method~// Pattern Recogn. Lett., 2003. Vol.~24.
No.\,9--10. P.~1251--1259.

\bibitem{Besag}
\Au{Besag J.} On the statistical analysis of dirty pictures~//
J.~Roy. Statistical Soc. B, 1986. Vol.~48. P.~259--302.

 \label{end\stat}

\bibitem{KrylovSPIE09}
\Au{Krylov V., Moser G., Serpico S., Zerubia J.}
Dictionary-based probability density function estimation for
high-resolution SAR data~// SPIE Proceedings.~--- San Jose, USA,
2009. Vol.~7246. P.~72460S-1--72460S-12.
 \end{thebibliography}
}
}


\end{multicols}