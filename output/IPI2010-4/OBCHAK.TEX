\def\stat{abstr}
{%\hrule\par
%\vskip 7pt % 7pt
\raggedleft\Large \bf%\baselineskip=3.2ex
A\,B\,S\,T\,R\,A\,C\,T\,S \vskip 17pt
    \hrule
    \par
\vskip 21pt plus 6pt minus 3pt }


\def\tit{LibMeta~--- DIGITAL LIBRARY MANAGEMENT SYSTEM}

%1
\def\aut{A.\,A.~Zakharov$^1$ and V.\,A.~Serebryakov$^2$}

\def\auf{$^1$Dorodnicyn Computing Centre of RAS, andreya@sufler.ru\\[1pt]
$^2$Dorodnicyn Computing Centre of RAS, serebr@ccas.ru}

\def\leftkol{\ } % ENGLISH ABSTRACTS}

\def\rightkol{\ } %ENGLISH ABSTRACTS}

\titele{\tit}{\aut}{\auf}{\leftkol}{\rightkol}

%\vspace*{-2pt}

\noindent 
The problems of modern digital libraries (DLs) creation are addressed. 
General requirements on integration with external data sources and standartization are 
proposed among with some means to satisfy them. Leading world standards for DLs are considered. 
As a result of practical implementation of studies, DL management system LibMeta is presented as an
universal tool for DL creation.


\label{st\stat}

%\vspace*{-5pt}

\KWN{digital libraries; distributed information system
}


\vskip 14pt plus 6pt minus 3pt

%\vfil

%2
\def\tit{INTEGRATION OF~HETEROGENEOUS INFORMATION ABOUT~COLOR PIXELS AND~THEIR COLOR PERCEPTION}

\def\aut{O.\,P.~Arkhipov and Z.\,P.~Zykova}

\def\auf{IPI RAN, Orel Branch, ofran@orel.ru}


\def\leftkol{\ } % ENGLISH ABSTRACTS}

\def\rightkol{\ } %ENGLISH ABSTRACTS}

\titele{\tit}{\aut}{\auf}{\leftkol}{\rightkol}

%\vspace*{-2pt}

\noindent
The problem of integration of heterogeneous information about color pixels, 
their transformation in user computer system, standard and individual 
color perception of user in uniform information and communication environment 
has been considered. It is necessary to solve this problem in order to
create special easy-adaptable to system components program-technical tools, providing the adequate 
perception by arbitrary users of output on peripheral devices of PC. When solving the problem, 
the new representation methods and analysis of heterogeneous (quantitative and qualitative) 
information on bounded color spaces, multicriterion choice for prediction  of
difference in color information and pixels classification for structuring 
of color spaces are used.

%\vspace*{-5pt}

\KWN{color space; chromatic sensation; color perception space; anomaly of color vision; 
partial color blindness; RGB-characterization}
%\pagebreak

\vskip 14pt plus 6pt minus 3pt

%3
\def\tit{MATHEMATICAL MODEL AND~HUMAN POSE RECONSTRUCTION METHOD BASED~ON~STEREOMATE IMAGE SILHOUETTES}

\def\aut{A.~Tsiskaridze}
\def\auf{Moscow Institute of Physics and Technology, AchikoTsi@gmail.com}

\def\leftkol{\ } % ENGLISH ABSTRACTS}

\def\rightkol{\ } %ENGLISH ABSTRACTS}

%\def\leftkol{ENGLISH ABSTRACTS}

%\def\rightkol{ENGLISH ABSTRACTS}

\titele{\tit}{\aut}{\auf}{\leftkol}{\rightkol}

%\vspace*{-2pt}
\noindent
 A problem of locally symmetric object reconstruction 
based on a stereomate of silhouettes when occlusion is absent is discussed. Two models 
of an object are considered. In the first model, an object is described as a union of fat 
spatial curves and a method of restoration of such objects is presented. In the second case, 
an object is described as a joint model and a fitting method based on a stereomate 
of silhouettes is proposed. The methods are based on a construction of continuous skeletons. 
The methods have been compared for human body pose reconstruction. 

%\vspace*{-5pt}

\KWN{stereoreconstruction; skeleton; cylindrical objects; joint model; medial axes}
\pagebreak


%\vfil
%\vskip 14pt plus 6pt minus 3pt


%4
\def\tit{MODELING AND CLASSIFICATION OF~MULTICHANNEL REMOTELY SENSED IMAGES 
VIA~COPULAS}


\def\aut{V.\,A.~Krylov}
\def\auf{Department of Mathematical Statistics, Faculty of
Computational Mathematics and Cybernetics,\\  
M.\,V.~Lomonosov Moscow State University, vkrylov@cs.msu.ru}

\def\leftkol{\ } % ENGLISH ABSTRACTS}

\def\rightkol{\ } %ENGLISH ABSTRACTS}

\titele{\tit}{\aut}{\auf}{\leftkol}{\rightkol}

%\vspace*{-2pt}

\noindent
A novel approach to modeling of multichannel remotely 
sensed images is proposed.
This approach suggests to use the classical statistical probability 
distribution estimation methods for single channels
and then the construction of the joint probability distribution of a 
multichannel image via copulas.
An integration of the developed copula-based approach with a Markov 
random field model is proposed for supervised Bayesian 
image classification.
Experiments with real remotely sensed images captured by a synthetic 
aperture radar demonstrate high accuracy classification results proving
the efficiency of the developed approach as compared to state-of-the-art 
methods.

%\label{st\stat}

%\vspace*{-5pt}

\KWN{multichannel image; copula; Markov random field; Bayesian 
classification}

%\pagebreak

% \thispagestyle{headings}



% \vskip 24pt plus 9pt minus 6pt
\vskip 14pt plus 6pt minus 3pt

%5
\def\tit{COMMUNICATION BETWEEN TIME AND STRUCTURAL-TOPOLOGICAL CHARACTERISTICS OF~HEALTHY PEOPLE HEART RHYTHM DIAGRAMS}


\def\aut{А.\,A.~Kuznetsov}

\def\auf{Vladimir State University, artemi-k@mail.ru}


\def\leftkol{ENGLISH ABSTRACTS}

\def\rightkol{ENGLISH ABSTRACTS}

\titele{\tit}{\aut}{\auf}{\leftkol}{\rightkol}

%\vspace*{-2pt}

\noindent
According to 628~electrocardiogram registration at 177~healthy and ill people, the comparative analysis of 
parameters of real and virtual heart rhythm diagrams for a regulation system influence estimation 
on a heart rhythm is carried out. Between diagrams parameters and information entropy in conditions 
of discrete seasonal adaptation, functional communications are determined. ``Organism functional condition formulas'' 
connecting a heart rhythm diagram macrostructure parameters with its storey microstructures 
parameters are offered. It is revealed that the mode of a healthy person heart rhythm without 
sex dependence has a calendar year cycle, during which it three times discretely varies. 


%\vspace*{-6pt}

\KWN{heart rhythm diagram; organism functional condition; storey structure; 
information entropy; quantity of the information}

%\vskip 18pt plus 6pt minus 3pt

 \vskip 14pt plus 6pt minus 3pt

% \pagebreak


%6
\def\tit{SEMIFORMAL VERIFICATION FOR PIPELINED DIGITAL DESIGNS BASED ON ALGORITHMIC STATE MACHINES}

\def\aut{S.~Baranov$^1$, S.~Frenkel$^2$, and V.~Zakharov$^3$}

\def\auf{$^1$Holon Institute of Technology,Holon, Israel, samary@012.net.il\\[1pt]
$^2$IPI RAN, fsergei@mail.ru\\[1pt]
$^3$IPI RAN, VZakharov@ipiran.ru}

\titele{\tit}{\aut}{\auf}{\leftkol}{\rightkol}

%\vspace*{-6pt}

\noindent
The method and tools  to use Algorithmic State Machine (ASM) 
formalism for high-level specification of  complex digital designs and their 
Model-Checking-based verification are described. This methodology is based on the possibility of hierarchical 
description of the target digital designs at  algorithmic level of abstraction, and ability 
to  generate finite state machines (FSM) models of the systems from the ASM flowcharts. 
The software tool was developed for automatic generation of SMV (Symbolic Modes Verifier) 
codes  from the ASMs and corresponding FSMs. A way of this approach application to design 
verification is demonstrated for a pipelined microprocessor.

%\vspace*{-2pt}

\KWN{formal verification; Model Checking; finite state machines}
\pagebreak

%\vskip 14pt plus 6pt minus 3pt

%7
\def\tit{COORDINATION ALGORITHM  FOR HYBRID INTELLECTUAL SYSTEM SOLUTIONS OF~A~COMPLEX
PROBLEM  OF~OPERATIONAL INDUSTRIAL PLANNING}

\def\aut{A.\,V.~Kolesnikov$^1$ and S.\,A.~Soldatov$^2$}

\def\auf{$^1$Department of Computer Modeling and Information Systems,
``Immanuel Kant Russian State University;''\\
\hphantom{$^1$}Kaliningrad Branch of the IPI RAN, avkolesnikov@yandex.ru\\[1pt]
$^2$Limited Liability Company ``Lighton,'' Moscow, ssa@west-automatica.com% (в статье) или Kaliningrad State Technical University, Russia (в англ. информации)????
}


\titele{\tit}{\aut}{\auf}{\leftkol}{\rightkol}

\vspace*{-2pt}

\noindent
The problem of operational industrial planning at the machine-building enterprise with custom-made, 
small-scale character of manufacture is considered and the approach to the solution of similar problems 
on the basis of methodology of functional hybrid intellectual systems with coordination is described. 
The description of practical realization of hybrid intellectual system and the short analysis of the 
received results are given.

\vspace*{-5pt}

\KWN{machine-building enterprise; problem of operational planning; coordination; hybrid intellectual systems}
%\pagebreak

\vskip 8pt plus 6pt minus 3pt


%8
\def\tit{STATIONARY CHARACTERISTICS OF~THE~TWO-CHANNEL QUEUEING SYSTEM WITH~REORDERING CUSTOMERS 
AND~DISTRIBUTIONS OF~PHASE TYPE}

\def\aut{S.\,I.~Matyushenko}

\def\auf{Department of Probability Theory and Mathematical Statistics,\\
Peoples' Friendship University of Russia, matushenko@list.ru}

\titele{\tit}{\aut}{\auf}{\leftkol}{\rightkol}

\vspace*{-2pt}

\noindent
The two-channel finite-capacity queueing system with the distributions 
of phase type and reordering customers is considered. The Laplace--Stieltjes transform 
of the distribution function that characterizes the delay of reordering is obtained. The algorithm 
is developed to calculate the factorial moments of the number of customers being at the buffer of reordering.


\vspace*{-5pt}


\KWN{queueing system; distributions of phase type; reordering  customers}
%\pagebreak

 \vskip 8pt plus 6pt minus 3pt

%9

\def\tit{NORMAL APPROXIMATION FOR DISTRIBUTION OF~RISK ESTIMATE FOR~WAVELET COEFFICIENTS THRESHOLDING 
WHEN USING SAMPLE VARIANCE}

\def\aut{O.\,V.~Shestakov}
\def\auf{Department of Mathematical Statistics, Faculty of
Computational Mathematics and Cybernetics,\\  
M.\,V.~Lomonosov Moscow State University, oshestakov@cs.msu.su}

\titele{\tit}{\aut}{\auf}{\leftkol}{\rightkol}

\vspace*{-2pt}

\noindent
The asymptotic properties of risk estimate for thresholding 
wavelet coefficients of signal function are analyzed. Some estimates for 
rate of convergence to the normal law are obtained.

\vspace*{-5pt}

\KWN{wavelets; thresholding; risk estimate; normal distribution; rate of convergence}
%\pagebreak


\vskip 8pt plus 6pt minus 3pt

%10
\def\tit{ESTIMATES FOR CONVERGENCE RATE OF~DISTRIBUTIONS OF~RANDOM SUMS WITH~INFINITELY DIVISIBLE INDICES 
TO~THE~NORMAL DISTRIBUTION}

\def\aut{S.\,V.~Gavrilenko}


\def\auf{Department of Mathematical Statistics, Faculty of
Computational Mathematics and Cybernetics,\\  
M.\,V.~Lomonosov Moscow State University, gavrilenko.cmc@gmail.com}


%\def\leftkol{ENGLISH ABSTRACTS}

%\def\rightkol{ENGLISH ABSTRACTS}

\titele{\tit}{\aut}{\auf}{\leftkol}{\rightkol}

% \label{end\stat}

\vspace*{-2pt}

\noindent
New estimates for convergence rate of
distributions of random sums with infinitely divisible indices
were obtained. These estimates remain true under weaker conditions
than the known ones. As an example of this result, the article
contains estimates of the accuracy of the normal approximation for
the distribution functions of random sums with indices that have
negative binomial distribution.

\vspace*{-5pt}

\KWN{random sum; integer-valued infinitely divisible
distribution; generalized Poisson distribution; negative binomial
distribution; normal approximation}

\vskip 14pt plus 6pt minus 3pt



%11
\def\tit{ON COLLECTIVE DISPLAY FACILITIES PLACED IN~A~SITUATIONAL HALL WITH~PRESCRIBED PARAMETERS}

\def\aut{K.\,G.~Chuprakov}


\def\auf{IPI RAN, chkos@rambler.ru}


%\def\leftkol{ENGLISH ABSTRACTS}

%\def\rightkol{ENGLISH ABSTRACTS}

\titele{\tit}{\aut}{\auf}{\leftkol}{\rightkol}

 \label{end\stat}

\noindent
There are some dependences between main parameters of situational (or congress) hall. 
It is possible to link the diameter of hall (the largest distance between any two points of the hall), 
the informativity of demonstrated on screen content, the quantity of people working with a screen and 
also screen dimensions. The base for describing these dependences are the recommendations fixed in  the
Russian state standards and simple geometric considerations.

\KWN{visualization systems; situational hall; good observation area; algebraic dependences}

%\pagebreak

 