\def\stat{benderina}

\def\tit{УПРАВЛЕНИЕ УЧЕТНЫМИ ЗАПИСЯМИ И ПРАВАМИ ДОСТУПА ПОЛЬЗОВАТЕЛЕЙ 
В~ЦЕНТРАХ ОБРАБОТКИ ДАННЫХ ВЫСОКОЙ ДОСТУПНОСТИ}

\def\titkol{Управление учетными записями и правами доступа пользователей 
в~центрах обработки данных высокой доступности}

\def\autkol{М.\,В.~Бендерина, С.\,В.~Борохов, В.\,И.~Будзко и др.}
\def\aut{М.\,В.~Бендерина$^1$, С.\,В.~Борохов$^2$, В.\,И.~Будзко$^3$, П.\,В.~Степанов$^4$, 
А.\,П.~Сучков$^5$}

\titel{\tit}{\aut}{\autkol}{\titkol}

%{\renewcommand{\thefootnote}{\fnsymbol{footnote}}\footnotetext[1]
%{Работа выполнена при поддержке РФФИ (гранты 09-07-12098, 09-07-00212-а и
%09-07-00211-а) и Минобрнауки РФ (контракт №\,07.514.11.4001).}}


\renewcommand{\thefootnote}{\arabic{footnote}}
\footnotetext[1]{Институт проблем информатики Российской академии наук, mbenderina@ipiran.ru}
\footnotetext[2]{Институт проблем информатики Российской академии наук, sborokhov@ipiran.ru}
\footnotetext[3]{Институт проблем информатики Российской академии наук, vbudzko@ipiran.ru}
\footnotetext[4]{Институт проблем информатики Российской академии наук, pvstepanov@ipiran.ru}
\footnotetext[5]{Институт проблем информатики Российской академии наук, asuchkov@ipiran.ru}


\Abst{Изложены функционально-организационные схемы (ФОС) и принципы управ\-ле\-ния 
учетными записями и правами пользователей, разработанные для двух стратегий защиты 
информации, которые принимаются организацией или сообществом облачных вычислений. 
Определен порядок организации работ по созданию централизованной сис\-те\-мы управ\-ле\-ния 
учетными записями и правами пользователей в составе сис\-те\-мы обеспечения информационной 
безопасности (СОИБ) коллективных центров обработки данных (ЦОД) высокой доступности (ВД).}

\KW{информационная безопасность; высокая доступность; центр обработки данных}

 \vskip 14pt plus 9pt minus 6pt

      \thispagestyle{headings}

      \begin{multicols}{2}

            \label{st\stat}


\section{Введение}
       
       Консолидация обработки данных на коллективных вычислительных центрах, 
используемых многими пользователями,~--- современная мировая тенденция применения 
средств обработки и передачи данных. Консолидация обработки данных позволяет 
существенно снизить финансовые затраты на обеспечение выполнения приложений, в 
част\-ности за счет увеличения коэффициента полезного использования оборудования и 
уменьшения тре\-бу\-емо\-го ИТ-пер\-со\-на\-ла, а также существенно снизить количество 
нештатных ситуаций в сис\-те\-мо\-тех\-ни\-че\-ской инфраструктуре, которые отражаются на 
конечных пользователях. 

К~централизованной консолидированной обработке данных 
перешли крупные мировые компании и организации, создавая корпоративные хранилища 
данных с единой сис\-те\-мой сбора, накопления, верификации и хранения для всех 
приложений корпорации, что, в свою очередь, повысило достоверность, актуальность и 
точность получаемой на основе этих данных информации для поддержки принятия 
решений. 
       
       Многие небольшие компании не в состоянии самостоятельно создавать и 
поддерживать собственную необходимую для решения всех задач 
ин\-фор\-ма\-ци\-он\-но-технологическую инфраструктуру, обеспечи\-вающую требуемые вычислительные ресурсы 
и организационно-технические условия для осуществления безопасной технологии 
обработки данных. Им дешевле покупать (арендовать) необходимые услуги и ресурсы у 
ЦОД, который находится в промышленной эксплуатации и 
может их предоставить в требуемом объеме и при надлежащем качестве обслуживания.
       
       Динамическая масштабируемость современных вычислительных и 
телекоммуникационных средств позволяет наращивать ресурсы с ростом нагрузок со 
стороны прикладных сис\-тем (ПС) без остановки обработки, а наличие средств 
виртуализации~--- обеспечивать их эффективное использование. 

Реализация <<облачных 
вычислений>> также основана на использовании мощных ЦОД, 
которые позволяют хранить большие объемы данных и осуществлять исполнение тысяч 
приложений различных пользователей одновременно. 
       
       Яркий пример применения ЦОД~--- создание катастрофоустойчивой 
территориально распределенной информационно-телекоммуникационной сис\-те\-мы 
централизованной обработки банковской информа\-ции в Банке России~[1, 2]. 
       
       Современному ЦОД присущи следующие архитектурные особенности:
       \begin{itemize}
        \item  разделение одних и тех же вычислительных ресурсов множеством параллельно 
исполня\-емых~ПС;
       \item 
централизованное управление функционированием центра на уровне ресурсов, 
осуществляемое в интересах всех эксплуатируемых ПС;
       \item 
индивидуальное управление функционированием каждой ПС на прикладном уровне;
       \item 
удаленный доступ пользователей к своим ПС.
\end{itemize}

       Однако создание и использование ЦОД в ИТ-ин\-фра\-струк\-ту\-ре компании 
имеет свои особен\-ности, в том числе и в части обеспечения 
ИБ. Со\-став\-ная часть обеспечения ИБ~--- задача управления учетными 
записями и правами доступа пользователей, для решения которой разработаны, доступны 
и применяются разнообразные ап\-па\-рат\-но-про\-грам\-мные комплексы различных 
фирм-изготовителей. Наиболее распространенные из них обладают приблизительно 
равными возможностями и наборами функций. Выбор конкретного комплекса должен 
делаться на основе общей стратегии организации в области защиты информации, 
существующей модели угроз, используемых подходов, опыта внедрения и эксплуатации 
средств и решений по обеспечению ИБ и других факторов.
       
       Центру обработки данных присущ ряд особенностей, которые необходимо учитывать при 
создании  СОИБ ЦОД, а в данном случае~--- 
под\-сис\-те\-мы управ\-ле\-ния учетными записями и правами пользователей. Указанные 
особенности влияют прежде всего на распределение зон ответственности между 
различными структурными подразделениями организации, связанными с эксплуатацией 
ЦОД. Также требуется более четко определять иерархическую структуру и порядок 
взаимодействия между ее различными элементами.
       
       В настоящей статье предполагается рас\-смот\-реть особенности обеспечения ИБ 
ЦОД, ФОС сис\-тем управ\-ле\-ния 
учетными записями и правами пользователей, построенные на основе выбранной 
стратегии защиты информации, и основные проблемы, воз\-ни\-ка\-ющие при создании и/или 
вводе подобных сис\-тем в эксплуатацию.

\vspace*{-6pt}

\section{Описание центра обработки данных}

\vspace*{-2pt}

\subsection{Общее описание центра обработки данных}
       
       Центр обработки данных~--- это ор\-га\-ни\-за\-ци\-он\-но-тех\-ни\-че\-ский 
       комплекс, предназначенный для 
создания высокопроизводительной, отказоустойчивой информационной 
инфраструктуры~[3].
       
       С ростом централизации хранения и обработки информации возрастает значение 
адекватного проектирования и эксплуатации ЦОД, на котором сосредоточиваются 
основные информационные ресурсы и решаются задачи информационной поддержки 
деятельности подразделений одной или нескольких организаций и их отдельных 
сотрудников. Центр обработки данных становится существенно более критичным элементом сис\-те\-мы, чем 
при распределенной обработке. Увеличивается потенциальный ущерб, который может 
быть нанесен в результате прекращения функционирования или/и преодоления сис\-те\-мы 
безопасности ЦОД. 
       
       Выделяется особый класс сис\-тем~--- сис\-те\-мы ВД, 
требуемое время вос\-ста\-нов\-ле\-ния которых при любых причинах прерывания работы не 
должно превышать относительно небольшого значения (нескольких минут). 

Такие 
сис\-те\-мы должны \mbox{обладать} свойством ка\-та\-ст\-ро\-фо\-устой\-чи\-вости~--- спо\-соб\-ностью 
сохранять критически важные информационные и програм\-мные ресурсы и продолжать 
выполнение своих функ\-ций (возможно, с определенными ограничениями) в условиях 
деградации ФОС, вызванной массовым 
уничтожением элементов сис\-те\-мы, а также связей между ними в результате стихийных 
бедствий, техногенных аварий и катастроф, целенаправленного воздействия людей или 
групп людей (включая террористические акты)~[4--6]. 
       
       Как правило, ЦОД ВД состоят из идентичных площадок (вычислительных 
комплексов~-- ВК), расположенных в территориально разнесенных вычислительных 
центрах. Минимальное число таких площадок~---~2. Выделяют семь уровней 
катастрофоустойчивости ЦОД. Разделение на уровни происходит в зависимости от 
времени восстановления работоспособности ЦОД, объема предварительных мероприятий 
и других факторов, подробно изложенных в~[7].
       
       В настоящей статье рассматриваются проблемы обеспечения ИБ ЦОД ВД 
       с катастрофоустой\-чи\-востью не ниже пятого уровня.

\subsection{Особенности обеспечения информационной
безопасности центра обработки данных}

       Обеспечение ИБ любой автоматизированной сис\-те\-мы предполагает наличие 
опе\-ра\-тив\-но-тех\-ни\-че\-ско\-го управ\-ле\-ния комплексом мер и средств защиты и управления 
сис\-те\-мой безопасности~[7--11].
       
       Первое, как правило, связано непосредственно с управлением средствами 
обеспечения ИБ (штатными или дополнительными) и реализацией организационных мер. 
Второе подразумевает реализацию общего управления процессом обеспечения ИБ в 
организации, разработку и совершенствование нормативной и 
       нор\-ма\-тив\-но-ме\-то\-ди\-че\-ской базы организации, анализ текущего уровня 
ИБ организации.
       
       На практике оперативно-тех\-ни\-че\-ское управление реализуется силами 
структурного подразделения из состава эксплуатирующего персонала ЦОД. 
А~управление сис\-те\-мой безопасности является обязанностью отдельного подразделения в 
организации, ответственного за поддержание режима ИБ организации.
       
       Назначение ЦОД~--- предоставление ин\-фор\-маци\-он\-но-вы\-чис\-ли\-тель\-ных ресурсов 
и услуг для различных ПС. В~ЦОД могут функционировать несколько ПС, совместно 
использующих его вычислительные ресурсы. Данные этих ПС обрабатываются с 
использованием одних и тех же программных и технических средств. При этом сами 
прикладные сис\-те\-мы могут функционировать в интересах различных структурных 
подразделений организации\,--\,владельца ЦОД или/и его <<арендаторов>>. 

Эксплуатация 
и сопровождение этих сис\-тем также осуществляются различными структурными 
подразделениями организации, а в ряде случаев~--- с привлечением внешних организаций. 

Сам ЦОД по отношению к прикладным сис\-те\-мам~--- лишь отдельный элемент их 
сис\-те\-мо\-тех\-ни\-че\-ской инфраструктуры, причем не целиком, а в части объема 
предоставляемых им услуг и ресурсов.
       %
       Поэтому задача обеспечения ИБ ЦОД при консолидации обработки распадается 
на две подзадачи:
       \begin{enumerate}[(1)]
\item обеспечение ИБ ПС;
\item обеспечение ИБ ресурсов и услуг ЦОД.
\end{enumerate}

       Порядок обеспечения ИБ конкретной ПС на прикладном уровне определяется 
особенностями самой сис\-те\-мы. При этом необходимо обеспечивать взаимодействие и 
согласование порядка обеспечения ИБ ПС с порядком обеспечения ИБ\linebreak
 ЦОД.
       
       Обеспечение ИБ ресурсов ЦОД осуществляется в интересах всех ПС, 
функционирующих в ЦОД. Данная задача, как правило, решается отдельным структурным 
подразделением в составе службы эксплуатации ЦОД. При этом сама СОИБ может 
входить в состав ЦОД, а может быть самостоятельной сис\-те\-мой, смежной ЦОД. 
       
       Обеспечение взаимодействия сис\-тем безопасности ПС между собой и с сис\-те\-мой 
безопасности ЦОД является основной задачей управления информационной 
безопасностью в организации.
       
       Одной из задач, решаемых при обеспечении ИБ ЦОД, является управление 
учетными записями и правами пользователей в рамках СОИБ. Далее будут рассмотрены 
особенности ее решения.

\section{Функционально-организационная схема 
системы управления учетными~записями и~правами~пользователей}
       
       Системы управления учетными записями и правами пользователей могут 
создаваться на базе решений различных производителей программного обеспечения и 
технических средств, иметь различный масштаб и режим работы. Однако в основе 
ФОС, используемых при их создании, лежат три стратегии защиты информации:
       \begin{enumerate}[(1)]
\item оборонительная;
\item наступательная;
\item упреждающая.
\end{enumerate}

       Упреждающая стратегия предполагает тщательное исследование возможных 
угроз сис\-те\-ме и разработку мер по их нейтрализации еще на стадии проектирования и 
создания сис\-темы.
       
       Оборонительная стратегия используется тогда, когда не допускается 
вмешательство в процесс функционирования защищаемого объекта. В~этом случае 
обычно реализуются организационные меры защиты, направленные, прежде всего, на 
противодействие наиболее опасным угрозам.
       
       Наступательная стратегия занимает промежуточное положение. При ее 
реализации уже на начальной стадии создания сис\-те\-мы решаются вопросы 
обеспечения~ИБ. 
       
       Рассмотрим ФОС сис\-те\-мы 
       управ\-ле\-ния учетными записями и правами пользователей, построенные на базе 
наступательной и упреждающей стратегий. По мнению авторов статьи, данные\linebreak
 страте\-гии 
представляют наибольших интерес, так как позволяют разработчикам сис\-те\-мы 
безопас\-ности и службе безопас\-ности организации адекватно реальным условиям и 
требованиям выстраивать методы и средства защиты и формировать организацию и 
технологии этой сис\-те\-мы. Учитывая сказанное, далее будут рассмотрены:
       \begin{itemize}
\item ФОС сис\-те\-мы, построенная на основе 
наступательной стратегии;
\item ФОС сис\-те\-мы, построенная на основе упреж\-да\-ющей стратегии.
\end{itemize}

\begin{figure*}[b] %fig1
\vspace*{9pt}
 \begin{center}
 \mbox{%
 \epsfxsize=159.191mm
 \epsfbox{bud-1.eps}
 }
 \end{center}
 \vspace*{-9pt}
\Caption{Функционально-организационная схема
системы управления учетными записями и правами пользователей, основанная на 
наступательной стратегии ИБ}
\end{figure*}

       При реализации управления учетными записями и правами пользователей 
представленные подходы различаются, прежде всего, разделением обязанностей между 
администратором сис\-те\-мы и аудито\-ром. При использовании 
ФОС, основанной на наступательной стратегии, в обязанности 
администратора сис\-те\-мы входит поддержание сис\-те\-мы в работоспособном со\-сто\-я\-нии, 
выполнение всех задач администрирования в соответствии с уста\-нов\-лен\-ным в 
организации регламентом. В~час\-ти управ\-ле\-ния учетными записями и правами 
пользователей администратор имеет все полномочия по созданию, удалению и изменению 
учетной записи; он может предостав\-лять права доступа, изменять их, включать в группу 
и~т.\,п.
       
       Аудитор системы~--- роль, предназначенная для контроля функционирования 
сис\-те\-мы, работы эксплуатирующего персонала и функциональных пользователей, 
соответствия текущих настроек уста\-нов\-лен\-ной в организации политике ИБ. При 
управлении учетными записями и правами пользователей аудитор выполняет функции 
контроля соответствия набора прав и полномочий функциональных пользователей 
сис\-те\-мы и эксплуатирующего персонала выполняемым ими функциям (должностным 
обязанностям) и уста\-нов\-лен\-ной политике ИБ. В~случае выявления нарушений аудитор 
проводит расследование и выполняет другие действия в соответствии с уста\-нов\-лен\-ным 
регламентом.
       
       Функционально-организационная схема, основанная на упреж\-да\-ющей стратегии, предполагает, что аудитор 
выполняет все обязанности, приведенные для предыдущей ФОС, и дополнительно наделяется частью прав администратора 
сис\-те\-мы в соответствии с уста\-нов\-лен\-ной в организации политикой ИБ и другими 
нор\-ма\-тив\-но-ме\-то\-ди\-че\-ски\-ми документами. При рассмотрении подобной 
ФОС принято говорить уже не о 
роли аудитора сис\-те\-мы, а о роли администратора ИБ сис\-темы.
       
Далее каждый вид ФОС будет  рассмотрен применительно к консолидированной обработке в ЦОД.

\subsection{Функционально-организационная схема, 
построенная на~основе наступательной стратегии защиты информации}
       
       Применительно к консолидированной обработке в ЦОД ФОС сис\-те\-мы 
управления учетными записями и правами пользователей, построенная на основе 
наступательной стратегии защиты информации, имеет вид, представленный на рис.~1. 
       
       
       В системе обеспечения ИБ присутствуют следующие роли: администратор и 
аудитор сис\-те\-мы. Предполагается следующее разделение полномочий между 
выделенными ролями\footnote{Рассматриваемый набор ролей представляется наиболее 
интересным с точки зрения разделения полномочий эксплуатирующего персонала и ни в коем 
случае не является полным перечнем ролей эксплуатирующего персонала ЦОД.}.
       
       Аудитор СОИБ ЦОД в рамках всего комплекса в целом\footnote[2]{Здесь и далее под 
словосочетанием <<весь комплекс в целом>> будет пониматься совокупность ЦОД, СОИБ ЦОД и 
ПС, функционирующих на ЦОД.} осуществляет общее руководство процессом обеспечения 
ИБ, включая решение следующих задач:
       \begin{itemize}
\item разработку общей схемы управления ИБ, включая 
вопросы управления учетными записями и правами пользователей;
разработку нормативных и нор\-ма\-тив\-но-ме\-то\-ди\-че\-ских документов, определяющих 
процесс обеспечения ИБ всего комплекса;
\item координацию работ по ведению и настройке средств обеспечения ИБ, включая 
средства управления учетными записями и правами пользователей;
\item контроль и управление деятельностью аудиторов ЦОД и ПС;
\item  контроль поддержания заданного уровня ИБ комплекса в целом и разработку 
предложений по его совершенствованию;
\item интегрированный мониторинг ИБ всего комплекса в целом.
\end{itemize}

       В обязанности аудитора может входить:
       \begin{itemize}
\item контроль деятельности администратора СОИБ ЦОД;
\item контроль функционирования СОИБ ЦОД в час\-ти вопросов ИБ;
\item периодическая сверка и анализ данных регистрации событий ИБ, накапливаемых 
средствами обеспечения ИБ, а также ручными журналами учета.
\end{itemize}

       Администратор СОИБ в рамках всего комплекса в целом отвечает за его 
функционирование в целом, включая решение следующих задач:
       \begin{itemize}
\item разработку общей схемы обеспечения ИБ, включая основные принципы 
управления учетными записями и правами пользователей;
\item координацию работ по ведению средств обеспечения ИБ, в том числе средств 
управления учетными записями и правами пользователей;
\item периодический анализ настройки средств обеспечения ИБ и сверку этих данных 
с уста\-нов\-лен\-ны\-ми в организации правилами доступа к ресурсам;
\item управление и контроль деятельности администраторов ЦОД и ПС.
\end{itemize}

       Администратор системы также выполняет все функции по администрированию 
сис\-те\-мы, включая вопросы управления учетными записями и правами пользователей.
       
       В составе эксплуатирующего персонала ЦОД выделяются следующие роли.
       
       Аудитор ЦОД отвечает за поддержание режима ИБ ЦОД. В~его должностные 
обязанности \mbox{также} входит определение конкретной схемы конт\-ро\-ля доступа к ресурсам в 
соответствии с уста\-нов\-лен\-ны\-ми правилами, контроль заданных настроек\linebreak
 средств 
обеспечения ИБ, включая средства управ\-ле\-ния учетными записями и правами 
пользо\-вателей. Кроме того, он осуществляет контроль регистра\-ции действий 
пользователей и администраторов и оперативный мониторинг событий ИБ в зоне своей 
ответственности. Аудитор ЦОД осуществляет управление деятельностью аудиторов ПС. В~части задачи 
управ\-ле\-ния учетными записями и правами пользователей он отвечает за контроль процесса 
создания и изменения учетных записей, состава учетных записей администратором ЦОД. 
Аудитор ЦОД проводит анализ соответствия состава ролей и прав доступа каждой учетной записи 
выполняемым пользователем служебным обязанностям, а также анализ состава ролей и их 
полномочий на соответствие уста\-нов\-лен\-ной в организации схеме в зоне своей 
ответственности. 
       
       Администратор ЦОД отвечает за функционирование ЦОД. В~его должностные 
обязанности входят определение конкретной схемы разграничения доступа к 
защищаемым ресурсам в зоне своей ответственности, настройка и сопровождение средств 
обеспечения ИБ. При управлении учетными записями и правами пользователей в зоне 
своей ответственности администратор ЦОД выполняет следующие действия: 
       \begin{itemize}
\item создает, удаляет и изменяет учетные записи пользователей;\\[-14pt] 
\item назначает и изменяет права доступа пользователей к защищаемым ресурсам;\\[-14pt] 
\item создает, удаляет и изменяет роли в ЦОД в соответствии с уста\-нов\-лен\-ной схемой;\\[-14pt]
\item приписывает роль (группы ролей) учетной записи, изменяет состав ролей 
учетной записи.
\end{itemize}

       В ПС, функционирующих в ЦОД, присутствуют следующие роли.
       \pagebreak
       
       Аудитор ПС отвечает за поддержание режима ИБ ПС. В~рамках своих 
должностных обязанностей в зоне своей ответственности определяет конкретную схему 
контроля доступа к ресурсам, контролирует настройки средств обеспечения ИБ, включая 
средства управления учетными записями и правами пользователей, непрерывность 
регистрации действий пользователей и администратора ПС. Осуществляет оперативный 
мониторинг событий ИБ. В~части решения задачи управления учетными записями и 
правами пользователей в его обязанности входит: 
       \begin{itemize}
\item контроль процесса создания и изменения учетных записей, состава учетных 
записей администратором ПС; 
\item контроль соответствия состава ролей и прав доступа каждой учетной записи 
выполняемым пользователем служебным обязанностям; 
\item анализ состава ролей и их полномочий на соответствие уста\-нов\-лен\-ной в 
организации схеме.
\end{itemize}
       
       В обязанности администратора ПС входит поддержание функционирования ПС. 
Кроме того, в зоне своей ответственности он определяет конкретную схему разграничения 
доступа к ресурсам, устанавливает, сопровождает и настраивает прикладное программное 
обеспечение (ППО) в соответствии с уста\-нов\-лен\-ны\-ми в организации правилами. При 
управлении учетными записями и правами пользователей отвечает: 
       \begin{itemize}
\item за создание, удаление и изменение учетных записей пользователей; 
\item за назначение и изменение прав доступа пользователей в ПС; 
\item за создание, изменение и удаление ролей в ПС; 
\item за приписывание роли (группы ролей) учетной записи, изменение состава ролей 
учетной за\-писи.
\end{itemize}

\subsection{Функционально-организационная схема системы, основанная на~упреждающей стратегии 
защиты~информации}
       
       Как уже отмечалось ранее, основным различием между рассматриваемыми ФОС 
является разделение функций между администратором СОИБ и администратором ИБ. 
В~зависимости от политики ИБ организации и ее технических возможностей указанное 
разделение может быть реализовано на организационном уровне или на организационном 
и техническом уровнях.
       
       В случае консолидированной обработки на ЦОД ФОС сис\-те\-мы управ\-ле\-ния 
учетными записями и правами пользователей, основанная на упреждающей стратегии 
защиты информации, имеет вид, представленный на рис.~2.

\begin{figure*} %fig2
\vspace*{1pt}
 \begin{center}
 \mbox{%
 \epsfxsize=159.191mm
 \epsfbox{bud-2.eps}
 }
 \end{center}
 \vspace*{-9pt}
\Caption{Функционально-организационная схема системы управ\-ле\-ния 
учетными записями и правами пользователей, основанная на упреж\-да\-ющей стратегии}
%\vspace*{9pt}
\end{figure*}
       
       Рассматриваемая ФОС предполагает следующее разделение обязанностей между 
персоналом, участвующим в эксплуатации ЦОД.
       
       Для СОИБ ЦОД выделяются следующие роли эксплуатирующего персонала:
       \begin{itemize}
\item администратор СОИБ ЦОД;
\item администратор ИБ СОИБ ЦОД.
\end{itemize}

       Администратор СОИБ отвечает за поддержание работоспособности СОИБ ЦОД. 
Он осуществля-\linebreak ет контроль функционирования СОИБ ЦОД, а\linebreak также всего комплекса ЦОД 
в целом в части вопросов обеспечения ИБ. В~обязанности администратора СОИБ также 
входит разработка общей схемы обеспечения ИБ, включая основные принципы 
управления учетными записями и правами пользователей, координация работы по 
ведению средств обеспечения ИБ, в том числе средств управления учетными записями и 
правами пользователей, периодический анализ их настроек и сверка этих данных с 
уста\-нов\-лен\-ны\-ми в организации правилами доступа к ресурсам. Кроме того, он 
осуществляет управ\-ле\-ние и контроль деятельности администраторов ЦОД и ПС в зоне 
своей ответственности. При управлении учетными записями и правами пользователей 
администратор СОИБ
имеет ограниченные возможности по администрированию. В~зависи\-мости от политики 
ИБ организации он может иметь возможность создавать, удалять и изменять учетные записи 
пользователей СОИБ\footnote{В зависимости от масштаба ЦОД функции администратора 
СОИБ и администратора ЦОД могут разделяться между несколькими ролями. В~таком случае 
общие вопросы функционирования сис\-тем и обеспечения ИБ решаются администраторами ЦОД и 
СОИБ, а непосредственное администрирование осуществляется более мелкими ролями.}, однако 
может не иметь возможности назначать им права доступа, изменять имеющиеся или 
аннулировать права доступа пользователя.
       
       Администратор ИБ СОИБ ЦОД в рамках представленной схемы отвечает за 
поддержание режима ИБ СОИБ и всего ЦОД в целом. Он разрабатывает общую схему 
управления ИБ, включая вопросы управления учетными 
записями и правами пользователей, нормативные и нор\-ма\-тив\-но-ме\-то\-ди\-че\-ские документы, 
определяющие процесс обеспечения ИБ. Администратор ИБ
осуществляет общее руководство процессом 
обеспечения ИБ. В~рамках своих должностных обязанностей он координирует работы по 
ведению и настройке средств обеспечения ИБ, включая средства управления учетными 
записями и правами пользователей, периодически анализирует и сверяет данные 
регистрации событий ИБ, накапливаемые средствами обеспечения ИБ, а также ручными 
журналами учета. Он осуществляет интегрированный мониторинг событий ИБ на всех 
уровнях. Кроме того, в обязанности администратора ИБ СОИБ ЦОД входит контроль и 
управление деятельностью администраторов ИБ ЦОД и администратора СОИБ ЦОД в 
соответствии с уста\-нов\-лен\-ным в организации порядком. В~час\-ти решения задачи 
управления учетными записями и правами пользователей в рамках СОИБ ЦОД в 
обязанности администратора ИБ может входить назначение и/или изменение состава 
ролей внутри СОИБ ЦОД.
       
       Для ЦОД выделяются следующие роли:
       \begin{itemize}
\item администратор ЦОД;
\item администратор ИБ ЦОД;
\item администраторы ПС, функционирующих в ЦОД;
\item администраторы ИБ ПС, функционирующих в ЦОД.
\end{itemize}

       Администратор ЦОД отвечает за поддержание ЦОД в работоспособном 
состоянии. В~его обязанности также входит определение конкретной схемы 
разграничения доступа к защищаемым ресурсам, настройка и сопровождение средств 
обеспечения ИБ, включая средства управления учетными записями и правами 
пользователей, проведение необходимой настройки при уста\-нов\-ке нового прикладного и 
сис\-тем\-но\-го программного обеспечения 
в зоне своей ответственности. В~час\-ти решения задачи управ\-ле\-ния 
учетными записями в зависимости от политики ИБ организации администратор ЦОД 
может иметь полномочия на создание, удаление и изменение учетных записей 
пользователей в зоне своей ответственности. Кроме того, в его обязанности может 
входить создание новых ролей и/или изменение уже существующих. Однако ему может 
быть запрещено назначать и/или изменять права доступа пользователей к защищаемым 
ресурсам, приписывать роль (группы ролей) учетной записи, изменять состав ролей 
учетной записи.
       
       Администратор ИБ ЦОД отвечает за поддержание режима ИБ ЦОД. 
В~обязанности администратора ИБ ЦОД входят определение конкретной схемы контроля 
доступа к ресурсам в соответствии с уста\-нов\-лен\-ны\-ми правилами, контроль заданных 
настроек средств обеспечения ИБ, включая средства управления учетными записями и 
правами пользователей, в соответствии с уста\-нов\-лен\-ной схемой в зоне своей 
ответственности. Кроме того, администратор ИБ ЦОД осуществляет контроль 
регистрации действий пользователей и администраторов, оперативный мониторинг 
событий ИБ в зоне своей ответственности. В~час\-ти задачи управ\-ле\-ния учетными 
записями и правами пользователей он осуществляет контроль процесса создания и изменения 
администратором ЦОД учетных записей, состава учетных записей; анализ состава ролей и 
их полномочий на соответствие уста\-нов\-лен\-ной в организации схеме; контроль 
соответствия состава ролей и прав доступа каждой учетной записи выполняемым 
пользователем служебным обязанностям в зоне своей ответственности. 
Администратор ИБ ЦОД может также 
назначать и/или изменять права доступа пользователей, приписывать роли (группы ролей) 
учетной записи, изменять состав ролей учетной записи.
       
       Администратор ПС, функционирующей в ЦОД, отвечает за поддержание 
функционирования ПС. В~обязанности администратора входит определение конкретной 
схемы разграничения доступа к ресурсам в зоне своей ответственности, уста\-нов\-ка, 
настройка и сопровождение ППО в соответствии с уста\-нов\-лен\-ны\-ми 
правилами. В~час\-ти решения задачи управ\-ле\-ния учетными записями и правами 
пользователей на администратора ПС может быть возложено создание, удаление и 
изменение учетных записей пользователей; создание, изменение и удаление ролей в ПС в 
зоне своей ответственности. Однако может быть запрещено назначать и изменять права 
доступа пользователей в ПС, приписывать роль (группы ролей) учетной записи, изменять 
состав ролей учетной записи в~ПС.
       
       Администратор ИБ ПС, функционирующей в ЦОД, отвечает за поддержание 
режима ИБ дан-\linebreak ной сис\-те\-мы. Аналогично администратору ИБ ЦОД,
он в зоне своей 
ответственности определяет конкретную схему контроля доступа к ресурсам в 
соответствии с уста\-нов\-лен\-ны\-ми правилами, осуществляет контроль заданных настроек 
средств обеспечения ИБ, включая средства управления учетными записями и правами 
пользователей; контроль регистрации действий пользователей и администратора ПС; 
оперативный мониторинг событий ИБ. В~час\-ти задачи управ\-ле\-ния учетными 
записями и правами пользователей он осуществляет контроль процесса создания и изменения 
администратором ПС учетных записей, состава учетных записей ПС; анализ состава ролей 
и их полномочий на соответствие уста\-нов\-лен\-ной в организации схеме; контроль 
соответствия состава ролей и прав доступа каждой учетной записи выполняемым 
пользователем служебным обязанностям в зоне своей от\-вет\-ст\-вен\-ности. 
Администратор ИБ ПС может также 
назначать и/или изменять права доступа пользователей, приписывать роли (группы ролей) 
учетной записи, изменять состав ролей учетной за\-писи.
{%\looseness=1

}
       
       В зависимости от политики ИБ организации и режима работы ЦОД и ПС, 
функционирующих в ЦОД, также могут выделяться роли администратора нештатного 
режима и оператора (оператора~ИБ).
{%\looseness=1

}

Роль администратора нештатного режима 
пред\-по\-ла\-га\-ет наличие в сис\-те\-ме учетной записи, ис\-поль\-зу\-емой 
в случае перехода сис\-те\-мы 
в нештатный \mbox{режим} функционирования. Данная роль имеет максимальные права по 
доступу к ресурсам сис\-те\-мы. При штатном режиме функционирования сис\-те\-мы вход с 
использованием учетной записи данной роли в сис\-те\-му, как правило, запрещен. 

Оператор 
сис\-те\-мы имеет минимальные полномочия в соответствующей сис\-те\-ме. Данная роль 
предназначена для непрерывного наблюдения за работой сис\-те\-мы (ПС или самого ЦОД) 
и, в случае нарушения нормального функционирования, принятия оперативных действий 
в соответствии с уста\-нов\-лен\-ным в организации регламентом.
       {\looseness=1

}

       В зависимости от масштаба и сложности сис\-те\-мы (прикладной или самого ЦОД), 
а также от потребностей организации состав ролей может меняться. 

\section{Порядок организации работ по~созданию централизованной 
системы управления учетными записями и~правами пользователей}
       
       Для создания эффективной сис\-те\-мы централизованного управления учетными 
записями и правами пользователей в составе СОИБ ЦОД еще на стадии проектирования 
необходимо решение ряда вопросов. Далее будут приведены некоторые из них, способные 
оказать значительное влияние на создаваемую сис\-те\-му централизованного управ\-ле\-ния 
учетными записями и правами пользователей.
{\looseness=1

}

\subsection{Определение зон ответственности}

       Как уже отмечалось в разд.~2 настоящей статьи, задача обеспечения ИБ ЦОД 
включает в себя два направления:
обеспечение ИБ ПС, функционирующих в ЦОД, и
обеспечение ИБ ресурсов самого ЦОД.


       На практике задачи решаются различными подразделениями и на различных 
уровнях информационной инфраструктуры. Однако очевидно, что оба процесса не могут 
идти абсолютно независимо друг от друга.
       
       Как правило, СОИБ ЦОД функционирует на нижних уровнях информационной 
инфраструктуры ЦОД и обеспечивает поддержание заданного уровня ИБ ресурсов ЦОД, 
находящихся в совместном пользовании ПС. При этом сами прикладные сис\-те\-мы, 
погружаемые в среду ЦОД, представляются для СОИБ ЦОД некоторыми <<черными 
ящиками>>.
       
       Система ИБ ПС предназначена для обеспечения и поддержания требуемого 
уровня ИБ внут\-ри~ПС. 
Для эффективного функционирования каждой из сис\-тем в рамках 
консолидированной обработки на ЦОД необходимо для каждого подразделения 
организации, участвующего в эксплуатации ЦОД и/или СОИБ ЦОД, а также для 
подразделений, эксплуатирующих прикладные сис\-те\-мы, определить их зоны 
ответственности. Должны быть определены задачи, решаемые каждым подразделением, 
схема и порядок их взаимодействия при работе ЦОД в штатном и нештатном режимах
функционирования. Также дополнительно может быть определен приоритет 
подразделения (или ПС) при распределении вычислительных ресурсов ЦОД.
       {%\looseness=1

}

       В соответствии с уста\-нов\-лен\-ны\-ми зонами ответственности определяются 
необходимые каждому подразделению функции программных продуктов, 
предполагаемых к использованию в части управ\-ле\-ния учетными записями и правами 
пользова\-телей.
{%\looseness=1

}
       
       Другой немаловажный момент при определении зон ответственности~--- 
распределение обязанностей между ролями эксплуатирующего персонала в соответствии 
с уста\-нов\-лен\-ной в организации политикой ИБ. Для построения эффективной СОИБ ЦОД 
на начальных этапах ее создания необходимо определить (или разработать по 
не\-об\-хо\-ди\-мости) порядок разделения обязанностей экс\-плу\-а\-ти\-ру\-юще\-го персонала в части 
управления учетными записями и правами доступа. Как было показано в разд.~3 
настоящей статьи, существуют два принципиально различных подхода к решению данной 
за\-дачи.
{\looseness=1

}       
       После того как сделан выбор в пользу одного из вариантов, необходимо 
сформировать перечни ролей и функций для каждой роли. На основе полученных 
материалов проводится разработка технических решений и предложений по 
организационным мерам в части тех требований, которые невозможно реализовать 
техническими средствами в рамках предполагаемых к использованию решений.

\subsection{Формирование иерархической структуры}

       В большинстве продуктов, предназначенных для централизованного управления 
учетными записями и правами пользователей, организационное обеспечение СОИБ ЦОД 
представляется в\linebreak виде
определенной структуры. Данная структура\linebreak
представляет собой 
дерево, отображающее присутствующие в организации подразделения и их\linebreak взаимосвязь. 
Ввиду этого при проектировании сис\-те\-мы для каждого подразделения организации 
необходимо сформировать иерархическую структуру групп пользователей. Кроме того, 
для нормального функционирования сис\-те\-мы централизованного управления учетными 
записями и правами пользователей необходимо также определить регламенты 
(\textit{workflow}), в соответствии с которыми должны будут осуществляться действия, 
связанные с созданием, управлением и удалением учетных записей в сис\-те\-ме. Обычно 
данные регламенты содержат не только саму цепочку действий, направленных на решение 
определенной задачи (создание нового пользователя, предоставление прав доступа к 
ресурсу, снятие блокировки учетной записи и~т.\,д.), но и предполагают наличие 
вариантов действий сис\-те\-мы в случае возникновения различных проблем (таких как 
болезнь лица, подтверждающего/разрешающего предоставление прав, создание новой 
учетной записи, истечение периода реакции ответственного лица на заявку пользователя 
и~т.\,п.).
       
       Грамотное решение данного вопроса позволяет создать гибкую и практически 
автономную сис\-те\-му централизованного управления учетными записями и правами 
пользователей, учитывающую все особенности уста\-нов\-лен\-но\-го порядка.

\subsection{Формирование перечня разрешенных и~запрещенных к~использованию 
функций системы}

       Существующие решения по централизованному управлению учетными записями 
и правами пользователей, как правило, предоставляют пользователям широкий набор 
функций (включая функции\linebreak
самообслуживания, сброса и пе\-ре\-уста\-нов\-ки собственных 
паролей, просмотра списка доступных\linebreak
ресурсов с возможностью запроса необходимых 
полномочий, возможность удаленного администрирования). Наличие некоторых из этих 
функций у определенных подразделений и/или групп пользователей может противоречить 
существующей в организации политике ИБ или другим нормативным документам. Ввиду 
этого уже на стадии проектирования необходимо четко сформировать перечень 
разрешенных функций для подразделений и/или групп пользователей и перечень 
функций, которые должны быть запрещены (или заблокированы).
       
       Кроме того, хорошей практикой является определение минимального перечня 
прав и полномочий, которыми должны обладать вновь со\-зда\-ва\-емые учетные записи (или 
явное указание того, что они не имеют никаких прав), а также все имеющиеся в сис\-те\-ме 
роли (администратор, администратор ИБ, оператор и~т.\,п.).

\subsection{Определение порядка реализации организационных мер 
обеспечения информационной безопасности}
       
       Анализ требований, предъявляемых некоторыми организациями к сис\-те\-ме 
централизованного управления учетными записями и правами пользователей, показывает, 
что ряд требований не покрывается функциональными возможностями используемых 
программных продуктов. Очень важно уже на ранних стадиях создания сис\-те\-мы провести 
анализ соответствия между требованиями организации
в области обеспечения ИБ и 
функциями предполагаемых к использованию программных продуктов.\linebreak 
Необходимо  определить все механизмы реализации каждого требования, а также 
требования, ко\-торые  не реализуются средствами программного\linebreak
продукта и  предполагают использование 
дополнительных средств или реализации организационных мер. Для каждого требования 
(или группы требований), не реализуемого функциями предполагаемых к использованию 
программных продуктов, необходимо разработать организационные меры и порядок их 
реализации в подразделениях, имеющих доступ к ресурсам ЦОД. 

\subsection{Определение порядка использования технологических учетных записей}

       Существующие программные продукты осуществляют централизованное 
управление учетными записями и правами пользователей в подконтрольных сис\-те\-мах 
посредством технологических учетных записей. Внутри подконтрольной сис\-те\-мы такая 
учетная запись имеет максимальные права по доступу к ресурсам. При этом на практике 
защите технологических учетных записей уделяется мало внимания (или не уделяется 
вовсе).
       
       При создании сис\-те\-мы централизованного управления учетными записями и 
правами пользователей необходимо подробно изучить вопрос взаимодействия 
программных продуктов, предполагаемых к использованию, с подконтрольными им 
сис\-те\-ма\-ми (ЦОД, возможно ПС, функционирующими в ЦОД). В~результате изучения 
должен быть сформирован перечень необходимых для взаимодействия технологических 
учетных записей, их права в сис\-те\-мах и порядок использования, а также описание мер их 
защиты. Оптимальным вариантом во многих случаях считается запрет интерактивного 
входа в подконтрольную сис\-те\-му с использованием идентификаторов технологических 
учетных записей.

\subsection{Определение порядка внесения изменений в~учетные записи 
пользователей}
       
       При изучении функциональных возможностей программных продуктов для 
сис\-те\-мы централизованного управления учетными записями и правами пользователей 
необходимо также исследовать возможности отслеживания изменений в учетных записях 
пользователей в подконтрольной сис\-те\-ме средствами самих продуктов. Предлагаемые 
сегодня на рынке программные продукты можно условно разделить на два класса~--- 
обеспечивающие возможность отслеживания изменений, внесенных в учетные записи 
пользователей подконтрольной сис\-те\-мы средствами самой сис\-те\-мы, и не обеспечивающие 
такой возможности.
       
       В зависимости от того, какой программный продукт/продукты предполагается к 
использованию в сис\-те\-ме централизованного управления учетными записями и правами 
пользователей, должен решаться вопрос о разработке дополнительных организационных 
мер. В~частности, при использовании продуктов, не обеспечивающих отслеживание\linebreak 
изменений в учетных записях, представляется целесообразной разработка порядка 
внесения из\-менений в учетные записи пользователей только\linebreak
посредством программного 
продукта, предназначенного для централизованного управления учетными записями и 
правами пользователей.

\section{Заключение}
       
       Консолидация обработки данных на коллективных ЦОД, 
используемых многими разнообразными ПС и их пользователями, требует обеспечения 
необходимого уровня ИБ. Центр обработки данных, входящий в со\-став сис\-те\-мы 
ВД, обладает рядом особенностей, которые необходимо учитывать при 
создании СОИБ ЦОД и ее под\-сис\-те\-мы 
управления учетными записями и правами пользователей. В~статье изложены
ФОС организации  управ\-ле\-ния\linebreak
 учетными записями и правами пользователей, разработанные для 
двух стратегий защиты информации, которые принимаются организацией или 
сообщест\-вом облачных вычислений. Определен порядок организации работ по созданию 
централизованной сис\-те\-мы управ\-ле\-ния учетными записями и правами 
пользователей в составе СОИБ ЦОД.

{\small\frenchspacing
{%\baselineskip=10.8pt
\addcontentsline{toc}{section}{Литература}
\begin{thebibliography}{99}
\bibitem{1ben}
\Au{Будзко В.\,И., Сенаторов М.\,Ю., Михайлов~С.\,Ф., Курило~А.\,П., Соколов~И.\,А.}
Направления совершенствования и развития 
ин\-фор\-ма\-ци\-он\-но-те\-ле\-ком\-му\-ни\-ка\-ци\-он\-ной сис\-те\-мы Бан\-ка 
Рос\-сии~// Информационная безопасность России в условиях глобального информационного 
общества: Мат-лы 5-й Всеросс. конф.~--- М., 2003. С.~207--211.

\bibitem{2ben}
\Au{Беленков В.\,Г., Будзко В.\,И., Быстров~И.\,И., Козлов~А.\,Н., Кудряшов~А.\,А., 
Курило~А.\,П., Михайлов~С.\,Ф., Нагибин~С.\,Я., Сенаторов~М.\,Ю., Шмид~А.\,В.}
Ката\-строфоустойчивая территориально распределенная 
ин\-фор\-ма\-ци\-он\-но-те\-ле\-ком\-му\-ни\-ка\-ци\-он\-ная сис\-те\-ма централизованной 
обработки банковской информации~// Системы высокой доступности, 2011. Т.~7. №\,3. 
С.~6--47.

\bibitem{3ben}
\Au{Заенц Д.}
Введение в ЦОД (дата-центр). Опубликовано 30.08.2009. {\sf http://dcnt.ru/?p=325}.

\bibitem{4ben}
\Au{Будзко В.\,И., Синицын И.\,Н., Соколов~И.\,А.}
Построение ин\-фор\-ма\-ци\-он\-но-те\-ле\-ком\-му\-ни\-ка\-ци\-он\-ных сис\-тем высокой 
доступности~// Системы высокой доступности, 2005. Т.~1. №\,1. С.~6--14.

\bibitem{5ben}
\Au{Борохов С.\,В., Будзко В.\,И., Киселев~Э.\,В., Кейер~П.\,А.}
Функциональное структурирование и критерии оптимизации построения и 
функциони\-ро\-вания ин\-фор\-ма\-ци\-он\-но-те\-ле\-ком\-му\-ни\-ка\-ци\-он\-ных\linebreak
 сис\-тем 
высокой доступности~// Системы высокой доступности, 2005. Т.~1. №\,1. С.~15--25.

\bibitem{6ben}
\Au{Беленков В.\,Г., Борохов С.\,В., Будзко~В.\,И., Киселев~Э.\,В., Кейер~П.\,А.}
Экономические основы консолидации обработки в 
ин\-фор\-ма\-ци\-он\-но-те\-ле\-ком\-му\-ни\-ка\-ци\-он\-ных сис\-те\-мах высокой 
доступности~//\linebreak
Системы высокой доступности, 2005. Т.~1. №\,1. С.~26--37.

\bibitem{7ben}
\Au{Беленков~В.\,Г., Будзко~В.\,И., Кейер~П.\,А.}
Катастрофоустойчивые решения в 
ин\-фор\-ма\-ци\-он\-но-те\-ле\-ком\-му\-ни\-ка\-ци\-он\-ных сис\-те\-мах высокой 
доступности~// Сис\-те\-мы высокой доступности, 2005. Т.~1. №\,1. С.~57--69.

\bibitem{8ben}
\Au{Будзко В.\,И., Соловьев А.\,В.}
Вопросы защиты от угроз со стороны обслуживающего персонала в центрах обработки 
данных~// Вопросы защиты информации, 2003. №\,2(61). С.~33--39.

\bibitem{9ben}
\Au{Борохов С.\,В., Будзко В.\,И., Курило~А.\,П.}
ФОС и принципы построения сис\-те\-мы безопасности 
ин\-фор\-ма\-ци\-он\-но-те\-ле\-ком\-му\-ни\-ка\-ци\-он\-ных сис\-тем высокой 
доступности~// Системы высокой доступности, 2005. Т.~1. №\,1. С.~38--45.

\bibitem{10ben}
\Au{Борохов С.\,В., Будзко~В.\,И., Капырин~А.\,Ю.}
Опыт применения динамического контроля целостности в 
ин\-фор\-ма\-ци\-он\-но-те\-ле\-ком\-му\-ни\-ка\-ци\-он\-ных сис\-те\-мах высокой 
доступности~// Системы высокой доступности, 2006. Т.~2. №\,1. С.~46--50.

\label{end\stat}

\bibitem{11ben}
\Au{Борохов С.\,В., Будзко В.\,И.}
Информационная безопас\-ность при консолидированной обработке на\linebreak мейнфрейме~// 
Информационные технологии и математическое моделирование сис\-тем 2009--2010: Тр. 
Междунар. науч.-техн. конф.~--- М., 2010. С.~182--183.
 \end{thebibliography}
}
}


\end{multicols}