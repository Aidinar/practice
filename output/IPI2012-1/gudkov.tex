\def\stat{gudkov}

\def\tit{БЫСТРАЯ ОБРАБОТКА ИЗОБРАЖЕНИЙ ОТПЕЧАТКОВ ПАЛЬЦЕВ}

\def\titkol{Быстрая обработка изображений отпечатков пальцев}

\def\autkol{В.\,Ю.~Гудков, М.\,В.~Боков}
\def\aut{В.\,Ю.~Гудков$^1$, М.\,В.~Боков$^2$}

\titel{\tit}{\aut}{\autkol}{\titkol}

%{\renewcommand{\thefootnote}{\fnsymbol{footnote}}\footnotetext[1]
%{Работа выполнена при поддержке РФФИ (гранты 09-07-12098, 09-07-00212-а и 
%09-07-00211-а) и Минобрнауки РФ (контракт №\,07.514.11.4001).}}


\renewcommand{\thefootnote}{\arabic{footnote}}
\footnotetext[1]{Челябинский государственный университет, diana@sonda.ru}
\footnotetext[2]{Южно-Уральский государственный университет, guardian@mail.ru}


\Abst{Предложена последовательность методов распознавания частных 
признаков на изображении отпечатка пальца с жесткими ограничениями на время 
обработки. Частные признаки сохраняются в шаблоне изображения. По шаблонам 
выполняется идентификация изображений.}

\KW{отпечаток пальца; обработка изображений; матрица потоков; мат\-ри\-ца периодов; 
част\-ные признаки}

 \vskip 14pt plus 9pt minus 6pt

      \thispagestyle{headings}

      \begin{multicols}{2}
      
            \label{st\stat}

\section{Введение}
  
  Исследования в области биометрии начались более ста лет назад с 
разработки методов сравнения отпечатков пальцев. С~развитием 
вы\-чис\-ли\-тель\-ной техники появилась возможность учета лиц в электронных 
системах. Функционирование таких электронных систем, подобно 
деятельности экс\-пер\-та-кри\-ми\-на\-лис\-та, опирается на модель 
дактилоскопического изображения (ДИ) в виде частных признаков и 
отношений между ними~[1]. Среди электронных систем наиболее известны 
системы криминального и гражданского назначения. Если для первых сис\-тем 
основным показателем эффективности служит величина ошибки 
идентификации  подозреваемого лица, то для вторых наравне с величиной 
ошибки аутентификации пользователя не\linebreak\vspace*{-12pt}
\begin{center} %fig1
\vspace*{12pt}
\mbox{%
  \epsfxsize=75.856mm
 \epsfbox{gud-1.eps}
}
\end{center}
\begin{center}
\vspace*{3pt}
{{\figurename~1}\ \ \small{Изображение отпечатка пальца}}
\end{center}
%\vspace*{11pt}

%\smallskip
\addtocounter{figure}{1}


\noindent
 менее важна и производительность 
сис\-те\-мы~[2]. Это оказывает сильное влияние на выбор методов обработки 
ДИ, например в системах контроля и управления доступом к объекту.
  
  На рис.~1 на узоре левой петли выделены частные признаки в виде 
окончания и разветвления, распознавание которых простыми методами 
неэффективно~\cite{1-g, 3-g}. Поэтому быстрая обработка ДИ реализуется в 
виде последовательности специальных методов измерения, анализа и синтеза 
параметров изображения. Настройка и обучение этих методов минимизируют 
влияние дефектов изображения. Тем не менее жесткие ограничения по времени 
сужают класс ДИ, пригодных для быстрой обработки, преимущественно до 
изображений хорошего и среднего качества.


\section{Постановка задачи}

  Обычно при распознавании частных признаков выполняются этапы 
предварительной обработки и повышения качества ДИ. Для этого изображение 
представляется в прямоугольной области~$G$ мощностью $\vert G\vert 
\hm=x_0y_0$ в виде $F\hm=\{f(x,y)\hm\in 0, \ldots , 2^b-1\vert (x,y)\hm\in X\times Y\}$, где 
$b$~--- глубина изоб\-ра\-же\-ния; $X\hm=0, \ldots , x_0\hm-1$ и $Y\hm=0, \ldots 
, y_0\hm-1$. 
  
  Обработка изображения структурно пред\-став\-ля\-ет\-ся в виде 
пирамиды~$\mathfrak{R}$ взаимосвязанных иерархий~[3--5], в которых 
сегментация изображения производится для любого слоя произвольной 
иерархии. Например, $l$-й слой $k$-й иерархии $F_k^{(l)}$ разбивается на 
$x_h y_h$ квадратных сегментов $S_{hk}^{(l)}(x,y)$ с длиной стороны~$2^{h-
k}$ и вершинами $(x,y)\hm\in X_h\hm\times Y_h$, где $k\hm<h$ и $h$~--- номер иерархии; 
$X_h=0, \ldots , x_h-1$ и $Y_h\hm=0, \ldots , y_h-1$.
  
  Доступ к каждой точке сегмента $S_{hk}(x,y)$ по~\cite{3-g} записывается в 
координатах $(u,v)\in \overline{X}_{hk}\times \overline{Y}_{hk}$:

\noindent
  \begin{equation}
  \left.
  \begin{array}{rl}
  \overline{X}_{hk} &= \left\{ u+x2^{h-k}\vert x\in{}\right.\\[9pt]
  &\hspace*{7mm}\left.{}\in X_h \land u\in 0, \ldots , 2^{h-
k}-1\right\}\,;\\[9pt]
  \overline{Y}_{hk} &= \left\{ v+y2^{h-k}\vert y\in {}\right.\\[9pt]
  &\hspace*{7mm}\left.{}\in Y_h \land v\in 0, \ldots , 2^{h-
k}-1\right\}\,.
  \end{array}
  \right\}
  \label{e1-g}
  \end{equation}
  
  Доступ к центральной точке сегмента $S_{hk}(x,y)$ записывается в 
координатах $(u,v)\in \hat{X}_{hk}\times \hat{Y}_{hk}$:
  \begin{equation}
  \left.
  \begin{array}{rl}
  \hat{X}_{hk} &=\left\{ 2^{h-k-1}+x 2^{h-k}\vert x\in X_h\right\}\,;\\[9pt]
  \hat{Y}_{hk} &=\left\{ 2^{h-k-1}+y 2^{h-k}\vert y\in Y_h\right\}\,.
  \end{array}
  \right\}
  \label{e2-g}
  \end{equation}
  
  Размер области $h$-й иерархии: $x_h=\lceil x_0/2^h \rceil$ и $y_h=\lceil 
y_0/2^h\rceil$, где $\lceil a\rceil$~--- наименьшее целое число, превышающее 
вещественную величину~$a$.
  
  При иерархической сегментации сегменты слоя~$F_k$ отображаются на 
вершины сегментов слоя~$F_h$ пирамиды~$\mathfrak{R}$, где $k\hm<h$. 
Соответственно, вершины сегментов отображаются на сегменты, 
расположенные ближе к основанию пирамиды~\cite{3-g}. Размер сегмента 
заметно влияет на время и качество обработки. Далее положим 
$S_h(x,y)=S_{h0}(x,y)$ и вершины $S_h(x,y)\in F_h$.
  
  Слои пирамиды можно представить множеством действительных чисел, а 
исходное изображение~--- множеством неотрицательных действительных 
чисел~\cite{4-g, 5-g}. Это снимает необходимость утомительного 
целочисленного представления сигнала и упрощает выражения, однако 
дискретизация изображения (слоев пирамиды~$\mathfrak{R}$) в пространстве 
сохраняется.
  
  Для компактной математической формализации методов 
классификационного анализа (КА) широко применяется аппарат апертур. 
Ключевую роль при этом играют прямолинейные щелевые $A_h(x,y,\alpha,w)$ 
и $A_h^-(x,y,\alpha,w)$ и круговая $A_h(x,y,w)$ апертуры, представляемые 
множеством точек слоя данных $h$-й иерархии и связанными с ними углами в 
виде элементов упорядоченных троек $(u,v,\beta)$. Эти апертуры определяются 
по формулам:
  \begin{equation}
  \left.
  \begin{array}{rl}
  \!A_h(x,y,\alpha,w) &= \left\{ (u,v,\beta) 
={}\right.\\[6pt]
&\hspace*{-23mm}\left.{}=(x+]w\cos(\alpha)[,y+]w\sin(\alpha)[,\beta)\vert w\in Z_w\right\}\,;\\[6pt]
  \!A_h^- (x,y,\alpha,w) &=\left\{ 
(u,v,\beta)={}\right.\\[6pt]
&\hspace*{-23mm}\left.{}=(x+]w\cos(\alpha)[,y+]w\sin(\alpha)[,\beta)\vert w\in Z_w^-\right\}\,;
  \end{array}
  \right\}\!
  \label{e3-g}
  \end{equation}

\noindent
\begin{equation}
A_h(x,y,w) =\cupb\limits_{\alpha\in Z^*} A_h(x,y,\alpha,w)\,,
\label{e4-g}
\end{equation}
где $(x,y)\in X_h\times Y_h$~--- центр апертуры; $(u,v)\hm\in X_h\times Y_h$~--- 
точка апертуры; $w$~--- размер апертуры; $Z_w=1, \ldots , w$; $Z_w^- \hm = -
w, \ldots  , -1\cup 1, \ldots , w$; $\alpha$~--- угол на\-прав\-ле\-ния апертуры; $]a[$~--- 
ближайшая целая часть вещественного числа~$a$. Угол, определяющий 
направление из центра апертуры $(x,y)$ в точку $(u,v)$, находится в виде:
$$
\beta =\arctg \left( \fr{v-y}{u-x}\right)+\pi n\enskip \mbox{при}\ n\in 0, \ldots , 1\,.
$$
  
  Для задачи распознавания частных признаков этапы предварительной 
обработки и повышения качества ДИ должны удовлетворять требованию на 
ограничение по времени. При этом алгоритм должен обеспечивать приемлемое 
качество распознавания частных признаков, которое проверяется на тестовой 
базе ДИ. Список частных признаков формируется в виде:
  \begin{equation}
  L_m=\left\{ M_i=\left\{ x_i,y_i,\alpha_i,t_i\right\}\vert i\in 1, \ldots , n\right\}\,,
  \label{e5-g}
  \end{equation}
где $M_i$~---  частный признак, индексированный номером~$i$; $n=\vert 
L_m\vert$~---  мощность списка частных признаков; $(x_i,y_i)$~---  координаты 
частного признака~$M_i$; $\alpha_i\in 0, \ldots , 359$~---  направление~$M_i$ как 
угол; $t_i\in \{0,1\}$~--- тип~$M_i$ (окончание или разветвление).
  
  Компромиссным решением задачи, устра\-ня\-ющим противоречие 
  ка\-че\-ст\-во--ско\-рость, может служить реализация шести этапов обработки 
ДИ: (1)~построение матрицы потоков; (2)~сглаживание вдоль потоков; 
(3)~построение матрицы плотности линий; (4)~сегментация; (5)~бинаризация; 
(6)~скелетизация и распознавание частных признаков.

\section{Быстрая обработка}

  Большинство алгоритмов КА отпечатков пальцев нацелено на распознавание 
частных признаков~\cite{1-g}. Решение задачи быстрой обработки 
продемонстрируем на примере исходного изображения $F_0^{(0)}\hm= \left\{ 
f_0^{(0)}(x,y)\right\}$ (см.\ рис.~1).

\subsection{Интегральное изображение} %3.1
  
  Интегральное изображение~$I$ позволяет вы\-чис\-лить сумму элементов 
прямоугольной области изоб\-ра\-же\-ния с постоянным числом операций 
независимо от размера области. Для вычисления интегрального изображения 
используется следующая формула:
  \begin{multline}
  I(x,y)=f(x,y)+{}\\
  {}+I(x-1,y)+I(x,y-1)-I(x-1,y-1)\,.
  \label{e6-g}
  \end{multline}
  
  Результат вычисления интегрального изображения показан на рис.~2. 
Однажды вычислив интегральное изображение, можно найти сумму элементов 
любой прямоугольной области изображения с\linebreak\vspace*{-12pt}
\begin{center} %fig2
\vspace*{4pt}
\mbox{%
 \epsfxsize=70.657mm
 \epsfbox{gud-2.eps}
}
\end{center}
%\begin{center}
\vspace*{3pt}
{{\figurename~2}\ \ \small{Прямоугольная область изображения~(\textit{а}) и интегральное изображение~(\textit{б})}}
%\end{center}
\vspace*{8pt}

%\smallskip
\addtocounter{figure}{1}

\noindent
 левым верхним углом $(x_1, y_1)$ 
и правым нижним углом $(x_2, y_2)$ за постояное время, используя следующее 
уравнение:
  \begin{multline}
  \sum\limits_{x=x_1}^{x_2}\sum\limits_{y=y_1}^{y_2} f(x,y)=I(x_2,y_2)-I(x_1-
1,y_2)-{}\\
{}-I(x_2,y_1-1)+I(x_1-1,y_1-1)\,.
  \label{e7-g}
  \end{multline}



\subsection{Построение матрицы потоков} %3.2
  
  Это базовый этап обработки, от которого зависит точность распознавания 
частных признаков. Он состоит из двух последовательно выполняемых 
процедур обработки ДИ. Самый простой подход к вы\-чис\-ле\-нию матрицы 
потоков основан на вы\-чис\-ле\-нии градиента.
  
  \textit{Измерение матрицы потоков.} Суть метода заключается в разбиении 
изображения на сегменты с точками $(u,v)\in S_h(x,y)$ при $h\hm=3$ $(8\times 8)$ 
по~(\ref{e1-g}) и вычислении для вершины каждого сегмента величины угла 
$0^\circ\leq \delta_h^{(0)}(x,y)\hm<180^\circ$ как элемента матрицы 
потоков~$\Lambda_h^{(0)}$ по формуле:
  \begin{multline*}
  \Lambda_h^{(0)} =\left\{ \delta_h^{(0)}(x,y)\right\} ={}\\
  {}= \left\{ 
\fr{\pi}{2}+\fr{1}{2}\,\arctg \left( \fr{2J_{12}(x,y)}{J_{22}(x,y)-
J_{11}(x,y)}\right)\right\}\,,
%  \label{e8-g}
  \end{multline*}
где 

\noindent
\begin{align*}
J_{12}(x,y)&=\sum\limits_{(u,v)\in S_h(x,y)}\nabla_x \nabla_y\,;\\
J_{11}(x,y)&=\sum\limits_{(u,v)\in S_h(x,y)}\nabla_x\nabla_x\,;\\
J_{22}(x,y) &= 
\sum\limits_{(u,v)\in S_h(x,y)}\nabla_y\nabla_y\,.
\end{align*}
Компоненты градиента в 
отсчетах $(u,v)\in \overline{X}_{hk}\hm\times \overline{Y}_{hk}$ по~(\ref{e1-g}) 
основания сегмента $S_h(x,y)$ вычисляются в виде 
$\nabla_x=\mathbf{H}_x**f_0^{(0)}(u,v)$,  $\nabla_y\hm=\mathbf{H}_y ** 
f_0^{(0)}(u,v)$, где ядра двумерной свертки как оптимизированные по 
величине ошибки угла ориентации операторы Собела~\cite{5-g} находятся в 
виде:
$$
\mathbf{H}_x=\begin{bmatrix}
-3 & 0 & 3\\
-10 & 0 & 10\\
-3 & 0 & 3
\end{bmatrix}\,;\quad 
\mathbf{H}_y=\begin{bmatrix}
-3 & -10 & -3\\
0&0&0\\
3 &10&3
\end{bmatrix}\,.
$$
  
  Таким образом, предварительно необходимо вычислить интегральные 
изображения $IG_{xy}(x,y)$, $IG_{xx}(x,y)$ и $IG_{yy}(x,y)$ по~(\ref{e6-g}) 
для определения тензоров $J_{12}(x,y)$, $J_{11}(x,y)$ и $J_{22}(x,y)$ 
соответственно и расчета потока независимо от размера апертуры. Здесь
  \begin{multline*}
  IG_{xy}(x,y) =G_{xy}(x,y)+IG_{xy}(x-1,y)+{}\\
  {}+IG_{xy}(x,y-1)-IG_{xy}(x-1,y-1)\,,
  \end{multline*}
где $G_{xy}(x,y)=\nabla_x(x,y)\nabla_y(x,y)$.
  
  Интегральные изображения $IG_{xx}(x,y)$ и $IG_{yy}(x,y)$  вычисляются 
аналогично $IG_{xy}(x,y)$. Затем определяем $J_{12}(x,y)$, $J_{11}(x,y)$ и 
$J_{22}(x,y)$ по следующим формулам:
\begin{align*}
J_{12}(x,y) &=IG_{xy}(x_2,y_2)-IG_{xy}(x_1-1,y_2)-{}\\
&{}-IG_{xy}(x_2,y_1-1)+IG_{xy}(x_1-1,y_1-1)\,;\\
J_{11}(x,y) &=IG_{xx}(x_2,y_2)-IG_{xx}(x_1-1,y_2)-{}\\
&{}-IG_{xx}(x_2,y_1-1)+IG_{xx}(x_1-1,y_1-1)\,;\\ 
J_{22}(x,y) &=IG_{yy}(x_2,y_2)-IG_{yy}(x_1-1,y_2)-{}\\
&{}-IG_{yy}(x_2,y_1-1)+IG_{yy}(x_1-1,y_1-1)\,,
\end{align*}
 где точка $(x_1,y_1)$~--- левый верхний угол заданной прямоугольной области 
изображения, а точка $(x_2,y_2)$~--- правый нижний угол.
  
  Фактически элементы из~$\Lambda_h$ вычисляются сглаживанием в 
сегментах $\{S_h(x,y)\}$ компонент поточечного структурного тензорного 
оператора~\cite{5-g}, записываемого в виде:
  \begin{equation}
  J=\begin{bmatrix}
  J_{11}+J_{22}\\
  J_{22}-J_{11}\\
  2J_{12}
  \end{bmatrix}\,.
  \label{e9-g}
  \end{equation}
  
  \textit{Анализ и коррекция матрицы потоков}. В~иерархии $h\hm=3$ на 
основе~(\ref{e9-g}) для $(x,y)\in X_h\times Y_h$ рассчитывается когерентность 
потоков по фор\-муле:
  \begin{multline}
  M_h^{(0)} =  \left[ \mu_h^{(0)}(x,y) \right] ={}\\
  {}=\left[ 
\fr{\sqrt{(J_{22}(x,y)-
J_{11}(x,y))^2+4J^2_{12}(x,y)}}{J_{11}(x,y)+J_{22}(x,y)}\right]\,.
  \label{e10-g}
  \end{multline}



  Когерентность для идеальных линий равна единице, а для изотропной 
структуры~--- нулю~\cite{5-g}. На\linebreak\vspace*{-12pt}
\pagebreak

\end{multicols}

\begin{figure} %fig3
 \vspace*{1pt}
 \begin{center}
 \mbox{%
 \epsfxsize=161.898mm
 \epsfbox{gud-3.eps}
 }
 \end{center}
 \vspace*{-9pt}
\Caption{Результаты вычислений $\Lambda_h$ (справа) и
соответствующих им $M_h$ (слева):
(\textit{а})~$\Lambda_h^{(0)}$ и $M_h^{(0)}$;
(\textit{б})~$\Lambda_h^{(1)}$ и $M_h^{(1)}$;  
(\textit{в})~$\Lambda_h^{(2)}$ и $M_h^{(2)}$;
(\textit{г})~$\Lambda_h^{(3)}$ и $M_h^{(3)}$}
\end{figure}


\begin{multicols}{2}

\noindent
 основе ранее вычисленных интегральных 
изоб\-ра\-же\-ний находим $\Lambda_h^{(0)}$, $\Lambda_h^{(1)}$,  $\Lambda_h^{(2)}$, 
$\Lambda_h^{(3)}$ и $M_h^{(0)}$, $M_h^{(1)}$, $M_h^{(2)}$, $M_h^{(3)}$ при 
$h\hm=3$ и апертурах $A_h(x,y,8)$, $A_h(x,y,24)$, $A_h(x,y,40)$, $A_h(x,y,56)$ 
соответственно. Результаты вычислений представлены на рис.~3.
  
  Затем матрицы потоков $\Lambda_h^{(0)}$, $\Lambda_h^{(1)}$,  $\Lambda_h^{(2)}$, 
$\Lambda_h^{(3)}$ и матрицы когерентностей $M_h^{(0)}$, $M_h^{(1)}$, $M_h^{(2)}$, 
$M_h^{(3)}$ анализируют. Для этого рассчитывают матрицу производных 
$N_n^{(0)}$ и мат\-ри\-цу потоков $O_h^{(0)}$ в виде:
  \begin{align*}
  N_h^{(0)} &= \left [ v_h^{(0)}(x,y)\right] ={}\\
  &\hspace*{14mm}{}=\left[ \fr{1}{n}\sum\limits_{l=1}^{n-
1} \left( \mu_h^{(l)}(x,y)-\mu_h^{(l-1)}(x,y)\right)\right]\,;\\
  O_h^{(0)} &= \lfloor o_h^{(0)}(x,y)\rfloor =\lfloor \delta_h^{(c)}(x,y)\rfloor\,,
  \end{align*}
где $c=\arg \max\limits_l \left( \mu_h^{(l)}(x,y)\right)$; $n$~--- количество слоев 
(в реализации~4). Матрица~$N_h^{(0)}$ отображает изменение когерентности 
потоков, а $O_h^{(0)}$~--- лучшие потоки.
  
  На основе матриц $N_h^{(0)}$ и $O_h^{(0)}$ вычисляют матрицу потока 
$\Lambda_h^{(4)}$: 
\begin{equation}
 \Lambda_h^{(4)} =\left\{ \delta_h^{(4)} (x,y)\right\}\,.
 \label{e11-g}
 \end{equation}
 Здесь
 \begin{equation*}
  \delta_h^{(4)}(x,y)=
  \begin{cases}
\delta_h^{(3)}(x,y), &\mbox{если}\ v_h^{(0)}(x,y)>T\,;\\
  o_h^{(0)}(x,y) & \mbox{иначе}\,,
  \end{cases}
  \end{equation*}
где $T$~--- пороговое значение (в реализации~0.1).
  
  Элементы из $\Lambda_h^{(4)}$ корректируют по формулам:
  \begin{equation*}
  \Lambda_h^{(5)} = \left\{ \delta_h^{(5)}(x,y)\right\}= \left\{ \fr{1}{2}\arctg\left( 
\fr{\mathrm{Im}_h^{(0)}(x,y)}{\mathrm{Re}_h^{(0)}(x,y)}\right)\right\}\,;
\end{equation*}

\noindent
\begin{equation*}
\mathrm{Re}_h^{(0)}(x,y) = \!\!\!\!\sum\limits_{(u,v)\in A_h(x,y,1)}\!\!\!\!\mu_h^{(0)}(u,v)\cos\left( 
2\delta_h^{(4)}(u,v)\right)\,;
\end{equation*}

%\noindent

\begin{center} %fig4
\vspace*{12pt}
\mbox{%
 \epsfxsize=79mm
 \epsfbox{gud-4.eps}
}
\end{center}
\begin{center}
%\vspace*{3pt}
{{\figurename~4}\ \ \small{Результирующая матрица потоков}}
\end{center}
\vspace*{9pt}

%\smallskip
\addtocounter{figure}{1}


\noindent
\begin{equation*}
\mathrm{Im}_h^{(0)}(x,y) = \!\!\!\!\sum\limits_{(u,v)\in A_h(x,y,1)}\!\!\!\! \mu_h^{(0)}(u,v)\sin\left( 
2\delta_h^{(4)}(u,v)\right)\,,
  \end{equation*}
где $\mu_h^{(0)}(u,v)$ и $\delta_h^{(4)}(u,v)$~--- когерентность и поток 
по~(\ref{e10-g}) и~(\ref{e11-g}) в отсчете $(u,v)$ апертуры $A_h(x,y,1)$ 
по~(\ref{e4-g}). На рис.~4 показана матрица потоков~$\Lambda_h^{(5)}$.



\subsection{Сглаживание} %3.3

  Сглаживающий фильтр устраняет микроразрывы и микрозалипания линий. 
Перед сглаживанием вычисляют матрицу регулярности потока $IR_h^{(0)}$, 
$h\hm=3$, используя следующую формулу~\cite{1-g}:
  \begin{multline*}
  IR_h^{(0)} =\left[ ir_h^{(0)}(x,y)\right] ={}\\
  {}=\left[ 1-\fr{\left \Vert 
  \sum\limits_{m=-1}^1 \sum\limits_{n=-1}^1 d(x+n,y+m)\right\Vert}{\sum\limits_{m=-1}^1 
\sum\limits_{n=-1}^1 \parallel d(x+n,y+m)\parallel}\right]\,,
  \end{multline*}
где $d(x,y)=\lfloor \cos 2\delta_h^{(5)}(x,y),\;\sin2\delta_h^{(5)}(x,y)\rfloor$; 
$\parallel x\parallel$~--- норма~$L_2$.
  
  Затем в основании каждого сегмента $S_h(x,y)$ величины отсчетов 
сглаживают по формуле:
  \begin{equation*}
  F_0^{(1)} =\begin{cases}
  f_0^{(1)}(x,y)=\mathbf{H}_1*\Xi_0^{(\alpha)}(x,y)\,, &\\
  &\hspace*{-30pt}\mbox{если}\ 
ir_h^{(0)}(x,y)>t\,;\\
  f_0^{(1)}(x,y)=f_0^{(0)}(x,y)\,, & \hspace*{-11mm}\mbox{иначе}\,,
  \end{cases}
%  \label{e12-g}
  \end{equation*}
где $\mathbf{H}_1$~---  ядро одномерной свертки; $t$~--- некоторый порог (в 
реализации~0.3); набор $\Xi_0^{(\alpha)}(x,y)\hm=\left\{ 
\xi_0^{(\alpha)}(u,v)\right\}$ состоит из элементов, выбираемых из~$F_0^{(0)}$ 
прямолинейной щелевой апертурой по~(\ref{e3-g}) в виде
\begin{multline*}
\left\{ \xi_0^{(\alpha)}(u,v)\right\} ={}\\
{}=\left\{ f_0^{(0)}(u,v)\vert(u,v)\in A_0^-
(x,y,\alpha,w) \cup (x,y)\right\}\,;
\end{multline*}
$\alpha= \delta_h^{(5)}(x,y)\in \Lambda_h^{(5)}$~---  направление апертуры, 
одинаковое для всех отсчетов основания сегмента $S_h(x,y)$; $w$~---  размер 
апертуры. 

Перенумеруем упорядоченные отсчеты набора 
$\Xi_0^{(\alpha)}(x,y)\hm= \left\{ \xi_0^{(\alpha)}(u,v)\right\}$, 
сгенерированного щелевой {апертурой} по~(\ref{e3-g}), в виде $ k \mapsto 
(u_k,v_k)$, где $k\in 0, \ldots , N$; $N=2w+1$. Тогда ядро свертки~$\mathbf{H}_1$ 
рассчитывают в виде:
$$
\mathbf{H}_1=\exp\left ( -\fr{(w-k)^2}{2\sigma^2}\right)\,,
$$
где $\sigma$~---  среднеквадратичное отклонение, определяющее крутизну 
гауссианы~\cite{1-g} (2--4 в реализации); $k\equiv w$~---  отсчет центра окна, 
здесь равный размеру апертуры~$w$. Сглаживающий фильтр, по сути, является 
выделенным первым сомножителем разделимого фильтра Габора с  
числом отсчетов, меньшим в $\pi w/2$ раз, что позволяет повысить производительность 
обработки. Результат сглаживания представлен на рис.~5.

\begin{center} %fig5
\vspace*{12pt}
\mbox{%
  \epsfxsize=79mm
 \epsfbox{gud-5.eps}
}
\end{center}
\begin{center}
\vspace*{3pt}
{{\figurename~5}\ \ \small{Исходное~(\textit{а}) и сглаженное~(\textit{б}) изображения}}
\end{center}
%\vspace*{11pt}

%\smallskip
\addtocounter{figure}{1}


  

\subsection{Построение матрицы периодов} %3.4

  Это базовый этап обработки, влияющий на точность распознавания частных 
признаков. Он выполняется на той же иерархии $h\hm=3$ и состоит из двух 
последовательно выполняемых процедур обработки ДИ.
  
  \subsubsection*{Измерение матрицы локальных периодов линий}
  
  \noindent
  \textbf{Определение 1.} Под периодом линий понимается величина $t=w/n$, 
обратно пропорциональная среднему числу $n$~линий, умещающихся в 
окрестности размером~$w$ на прямой, проведенной перпендикулярно 
линиям~\cite{1-g}.
  
  Зададим отрезок $C(x,y)=A_0^- (x,y,\alpha,w)\cup(x,y)$, сгенерированный 
щелевой апертурой по~(\ref{e3-g}), и перенумеруем отсчеты $(u,v)\in C(x,y)$ в 
виде $k\hm\mapsto (u_k,v_k)$, где $k\hm\in(0,\ldots  , N$; $N\hm=2w\hm+1$. В~отрезке 
$C(x,y)$ с центром $k\in \{w\}$ в отсчете $(x,y)\in X\times Y$ собираются 
упорядоченные по~$k$ величины $f_0^{(1)}(k)$ яркости изображения. 
Ориентация щелевой апертуры определяется углом~$\alpha$, выбираемым 
перпендикулярно потоку по формуле
  $$
  \alpha =\fr{\pi}{2}+\delta_h^{(5)}(x,y)
  $$
при $(x,y)\in X_h\times Y_h$, а ее длина определяется окрестностью размером 
$w$ для отсчета $(x,y)\in X\times Y$ (16 в~реализации). Для отрезка $C(x,y)$, 
центрированного в отсчете $(x,y)\in \hat{X}_{h0}\times \hat{Y}_{h0}$ 
по~(\ref{e2-g}) на сегменте $S_h(x,y)$, введем автокорреляционную функцию в 
виде:
\begin{equation}
r(i) =\fr{1}{N}\sum\limits_{k=0}^{N-i-1} \overset{\frown}{f}_0^{(1)}(k) 
\overset{\frown}{f}_0^{(1)}(k+i)\,,
\label{e13-g}
\end{equation}
где $\overset{\frown}{f}_0^{(1)}(k)\hm= f_0^{(1)}(k)-\overline{f}$ и 
$\overline{f} \hm= (1/N) \sum\limits_{k=0}^{N-1} f_0^{(1)}(k)$;  $N$~--- чис\-ло 
отсчетов щелевой апертуры. Определим $\Delta r(i) \hm= r(i+1)\hm-
r(i)$. Тогда элементы матрицы локального периода линий $T_h^{(0)}\hm=\left\{ 
t_h^{(0)}(x,y)\right\}$ вычисляются по формуле:
\begin{multline}
t_h^{(0)} (x,y)={}\\
{}=\arg \min\limits_j \left\{ \left( \Delta r(0),\ldots , \Delta r(j)\right) 
\vert \Delta r(j-1)>0 \wedge{}\right.\\
\left.{}\wedge \Delta r(j)\leq 0\right\}\,.
\label{e14-g}
\end{multline}
  
  Фактически для каждого сегмента иерархии $h\hm=3$ выделяют его центр, 
через который проводят отрезок перпендикулярно потоку. На рис.~6 величины 
яркости изображения собираются в отсчетах забеленного отрезка. Для них 
по~(\ref{e14-g}) оценивается локальный период линий на основе 
автокорреляционной функции по~(\ref{e13-g}), график которой показан на 
рис.~6. Выбор иерархии $h\hm=3$ сокращает число оценок в 64~раза.
  
  Отметим, что формула~(\ref{e14-g}) определяет такой локальный период 
линий, который соответствует экстремуму отсчетов для положительных 
величин автокорреляционной функции во второй положительной полуволне. 
На рис.~6 период линий равен~9.
\begin{center} %fig6
\vspace*{3pt}
\mbox{%
 \epsfxsize=72.704mm
 \epsfbox{gud-6.eps}
}
\end{center}
\begin{center}
\vspace*{3pt}
{{\figurename~6}\ \ \small{Определение периода}}
\end{center}
\vspace*{9pt}

%\smallskip
\addtocounter{figure}{1}

\noindent
 Выбор экстремума <<центрирует>> маску 
фильт\-ра, применяемого для фильтрации ДИ. Однако предположение о том, что 
оценка периода линий $t_h^{(0)}(x,y)$ может быть смещена, оставляет 
пространство для маневрирования параметрами фильтрации. На ровном фоне 
изображения элементы мат\-ри\-цы $T_h^{(0)}$ нулевые.


  \subsubsection*{Анализ и коррекция матрицы периодов линий}
  
  Имея в виду то, что 
отношение значения в точке первого максимума $r(t_h^{(0)}(x,y))$ к 
постоянной составляющей автокорреляционной функции~$r(0)$ для функции 
синус равно~1, определим матрицу достоверности периода в виде
  \begin{equation}
  Z_h^{(0)} =\left\{ \zeta_h^{(0)}(x,y)\equiv \fr{r(t_h^{(0)}(x,y)}{r(0)}\right\}\,.
  \label{e15-g}
  \end{equation}
  
  Для ДИ с разрешением 500~dpi $4\leq t_h^{(0)}(x,y)\hm\leq 17$~\cite{1-g}. 
Это позволяет фильтровать ошибки распознавания локального периода, задавая 
$t_h^{(0)}(x,y)\hm=0$.
  
  Суть процедуры сводится к расчету матрицы периодов линий 
$T_h^{(1)}\hm=\left\{ t_h^{(1)}(x,y)\right\}$, которая в начальной итерации 
с номером $j\hm=0$ инициализируется: $T_h^{(1)}\hm=T_h^{(0)}$. Далее 
номер~$j$ итерации инкрементируется. В~первой итерации для 
$t_n^{(1)}(x,y)\not\in \{0\}$ период линий сглаживается по формуле:
  \begin{equation}
  t_{h,j}^{(1)}(x,y) =\fr{\sum\limits_R t_{h,j-
1}^{(1)}(u,v)\zeta_h^{(0)}(u,v)}{\sum\limits_R \zeta_h^{(0)}(u,v)}\,,
  \label{e16-g}
  \end{equation}
где условие суммирования $R=t_{h,j-1}^{(1)}(u,v)\hm>0$ элементов апертуры 
$(u,v)\in A_h(x,y,1)$ определяется  по~(\ref{e4-g}); $n=\sum\limits_R 1$~--- количество 
ненулевых элементов в апертуре. В~последующих итерациях для каждого 
отсчета с кодом пропуска $t_n^{(1)}(x,y)\in \{0\}$ и для смежных с ним 
ненулевых элементов количеством~$n$, если $n\hm>4$, период линий 
прогнозируется по~(\ref{e16-g}). Число итераций ограничивают 
величиной~2--3. Если ограничение снять, то б$\acute{\mbox{о}}$льшая часть элементов 
из~$T_h^{(1)}$ определится.
\pagebreak

\begin{center} %fig7
\vspace*{1pt}
\mbox{%
 \epsfxsize=79mm
 \epsfbox{gud-7.eps}
}
\end{center}
%\begin{center}
\vspace*{3pt}
{{\figurename~7}\ \ \small{Результирующая матрица периодов~(\textit{а}) и матрица их достоверностей~(\textit{б})}}
%\end{center}
\vspace*{14pt}

%\smallskip
\addtocounter{figure}{1}


  Таким образом, ошибки измерений фильтруются, периоды линий 
сглаживаются и в финале прогнозируются. Результат коррекции матрицы 
локальных периодов линий представлен на рис.~7. Нулевые значения периодов 
показаны черным цветом. Большие значения периодов окрашены светлее.



\subsection{Сегментация} %3.5
  
  Сегментация необходима для отделения информативных областей ДИ от 
неинформативных. Она выполняется на той же иерархии $h\hm=3$ и 
заключается в расчете матрицы меток $C_h^{(0)}\hm= \left\{ 
c_h^{(0)}(x,y)\right\}$ по формуле:
  \begin{multline*}
  c_h^{(0)} (x,y) ={}\\
  {}=
  \begin{cases}
  1\,, &\hspace*{-2mm} \mbox{если}\ k_1\zeta_h^{(1)}(x,y)+k_2\mu_h^{(1)}(x,y)>\kappa_0\,;\\
  0 & \hspace*{-2mm} \mbox{иначе}\,.
  \end{cases}
%  \label{e17-g}
  \end{multline*}
где $\zeta_h^{(1)}(x,y)$~--- величина достоверности периода по~(\ref{e15-g}), 
сглаженная с помощью маски размером $3\times 3$; $\mu_h^{(1)}(x,y)$~--- 
когерентность потоков по~(\ref{e10-g}), сглаженная с помощью маски 
размером $3\times 3$; $k_0$, $k_1$ и $k_2$~---  обучаемые коэффициенты.
  
  Фактически при выделении информативных областей опираются на два 
признака: корреляционную функцию и когерентность потоков. Эти признаки 
сами по себе комплексные. Их сочетание позволяет повысить точность 
сегментации.
  
  При сегментации могут образовываться островки <<разнородных>> 
областей. Их можно дополнительно классифицировать операциями 
морфологической обработки изображения~\cite{4-g, 5-g}. Однако из-за 
ограничения по времени это нежелательно.

\subsection{Бинаризация} %3.6
  
  Бинаризация опирается на ранее вычисленные данные, а именно: на матрицу 
периодов~$T_h^{(1)}$ и матрицу меток~$C_h^{(0)}$. Каждый элемент 
сегмента $S_h(x,y)$ изображения $f_0^{(1)}(x,y)$ с меткой $c_h^{(0)}(x,y)\hm\in 
\{1\}$ бинаризуют следующим образом. Если значение элемента 
$f_0^{(1)}(x,y)$ меньше среднего значения $m$ элементов в круговой апертуре 
$A_h(x,y,w)$, умноженного на некоторый коэффициент~$k$ (в реализации 
0.98), то элементу присваивается значение~1, иначе~--- 0, т.\,е. 
  $$
  f_0^{(2)} (x,y) =
  \begin{cases}
  1\,, &\ \mbox{если}\ f_0^{(1)}(x,y)<km\,;\\
  0 & \ \mbox{иначе}\,.
  \end{cases}
  $$


\begin{center} %fig8
\vspace*{9pt}
\mbox{%
 \epsfxsize=60mm
 \epsfbox{gud-8.eps}
}
\end{center}
\begin{center}
\vspace*{1pt}
{{\figurename~8}\ \ \small{Бинаризованное изображение}}
\end{center}
\vspace*{11pt}

%\smallskip
\addtocounter{figure}{1}

%\vspace*{-9pt}

  Среднее значение рассчитывается с помощью интегральных изображений по 
формулам~(\ref{e6-g}) и~(\ref{e7-g}), что позволяет заметно ускорить процесс 
бинаризации. Размер апертуры~$w$ для каждой точки изображения берется из 
матрицы периодов~$T_h^{(1)}$. Результат бинаризации представлен на рис.~8.

\subsection{Скелетизация и распознавание частных признаков} %3.7

\begin{figure*} %fig9
 \vspace*{1pt}
 \begin{center}
 \mbox{%
 \epsfxsize=150mm
 \epsfbox{gud-9.eps}
 }
 \end{center}
 \vspace*{-9pt}
\Caption{Исходное изображение~(\textit{а}), частные признаки~(\textit{б}) 
и скелет изображения~(\textit{в})}
\end{figure*}


  Линии бинарного ДИ утончаются до скелетных (рис.~9). Введем 
некоторые определения.
  
%  \medskip
  
  \noindent
  \textbf{Определение 2.} Скелетом линии называется прос\-тая цепь $\langle 
u,v\rangle$ с вершинами~$u$ и~$v$ в 8-смеж\-ности, проходящая вблизи 
геометрического центра линии, причем для каждой вершины $p_1\in\langle 
u,v\rangle$ существует ровно две смежные с ней вершины~$p_2$ и~$p_3$, при 
этом вершины~$p_2$ и~$p_3$ несмежные.
  
  \smallskip
  
  \noindent
  \textbf{Определение 3.} Окончанием называется такая вершина~$p_1$ 
скелета, что для вершины~$p_1$ существует ровно одна смежная с ней 
вершина~$p_2$.
  
  \smallskip
  
  \noindent
  \textbf{Определение 4.} Разветвлением называется такая вершина~$p_1$ 
скелета, что для вершины~$p_1$ существуют ровно три смежные с ней 
вершины~$p_2$, $p_3$ и~$p_4$, при этом любые две вершины из множества 
$\{p_2, p_3, p_4\}$ попарно несмежные.


  Скелетизация опирается на раскрашивание точек линий $f_0^{(2)}(x,y)\in 
\{0\}$ по правилам $P(\xi(x,y))$, определяемым в специальной табличной 
форме на основе идентификатора окрестности точки в виде
  $$
  \xi(x,y) =\sum\limits_{i\in I} f(i)\cdot 2^i\,,
  $$
где $f(i)$ принимает значение~1 для линии и~0 в противном случае; $i\in 
I=0, \ldots  ,7$~--- \mbox{номер} сектора апер\-ту\-ры $3\times 3$ по~(\ref{e4-g}). Величина 
$\xi(x,y)\hm\in 0,\ldots , 255$ определяет ячейку в табличной форме. 
Согласно~\cite{6-g}, итерационное применение правил из $P(\xi(x,y))$ 
позволяет вычислить скелет линий, показанный на рис.~9. С~вершин скелета 
как графа~\cite{7-g} считываются окончания и разветвления, располагающиеся 
в информативной области изоб\-ра\-же\-ния на достаточном расстоянии от ее 
границы, и помещаются в список~(\ref{e5-g}). Затем применяется структурная 
пост\-обра\-бот\-ка скелета~\cite{1-g}. Частные признаки и их направления 
показаны на рис.~9, причем окончания окрашены черным цветом, а 
разветвления~--- серым.

\vspace*{6pt}

\section{Заключение}
  
  В~статье предложена группа взаимосвязанных методов, обеспечивающая 
приемлемое качество распознавания частных признаков при жестких 
ограничениях на время обработки ДИ. К~ним относятся: измерение и 
коррекция матриц потоков, сглаживание изображения, измерение и коррекция 
матриц периодов линий, сегментация, бинаризация, скелетизация и считывание 
с вершин скелета частных признаков. Построение матриц потоков основано на 
тензорном анализе простых окрестностей, а матриц периодов линий~--- на 
автокорреляционной функции. Общее время обработки составляет 
приблизительно 100~мс (изображение $320 \times 480$~пикселей, процессор 
Intel Pentium~4 CPU 3.0~ГГц). 
  
  Обработка основана на операции свертки, что с учетом временн$\acute{\mbox{ы}}$х 
характеристик позволяет перенести ее на целевые платы TMS или 
процессоры DSP~\cite{8-g} и использовать встроенные в них операции 
свертки. Последнее позволяет реализовать простые портативные 
биометрические системы, удовлетворяющие достаточно жестким требованиям 
к производительности.
  
  Аналогично процедуре сглаживания в качестве фильтра может быть 
использован одномерный фильтр Габора. Это повысит качество обработки, но 
ухудшит производительность. Кроме того, полученное бинарное изображение 
можно сгладить, что обычно повышает качество обработки. Дальнейшее 
направление развития быстрой обработки видится в улучшении метода 
сегментации изображения, т.\,е.\ в более качественном определении 
информативных зон изображения.

{\small\frenchspacing
{%\baselineskip=10.8pt
\addcontentsline{toc}{section}{Литература}
\begin{thebibliography}{9}

\bibitem{1-g}
\Au{Maltoni~D., Maio~D., Jain~A.\,K., Prabhakar~S.} Handbook of fingerprint 
recognition.~--- New York: Springer-Verlag, 2009. 494~p.

\bibitem{2-g}
\Au{Bolle~R.\,M., Connel~J.\,Y., Pankanti~S., Ratha~N.\,K.}
Guide to biometrics.~--- New York: Springer-Verlag, 2004. 368~p.

\bibitem{3-g}
\Au{Гудков~В.\,Ю.} Методы первой обработки дактилоскопических 
изображений.~--- Миасс: Геотур, 2008. 127~с.

\bibitem{4-g}
\Au{Гонсалес~Р., Вудс~Р.} Цифровая обработка изображений~/ Пер. с англ.~--- 
М.: Техносфера, 2006. 1072~c.

\bibitem{5-g}
\Au{Яне~Б.}
Цифровая обработка изображений~/ Пер. с англ. А.\,М.~Измайловой.~--- М.: 
Техносфера, 2007. 584~с.

\bibitem{6-g}
\Au{Гудков~В.\,Ю., Коляда~А.\,А. , Чернявский~А.\,В.}
Новая технология формирования скелетов дактилоскопических изображений~// 
Методы, алгоритмы и программное обеспечение гибких информационных 
технологий для автоматизированных идентификационных систем.~--- Минск: 
БГУ, 1999. С.~71--82.

\bibitem{7-g}
\Au{Новиков~Ф.\,А.}
Дискретная математика для программистов: Учебник для вузов.~--- 3-е изд.~--- 
СПб.: Питер, 2008. 384~с.

\label{end\stat}

\bibitem{8-g}
\Au{Сергиенко~А.\,Б.}
Цифровая обработка сигналов.~--- СПб.: Питер, 2002. 608~с.
 \end{thebibliography}
}
}


\end{multicols}       