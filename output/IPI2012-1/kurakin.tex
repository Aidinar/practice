\def\stat{kurakin}

\def\tit{РАСПОЗНАВАНИЕ ЖЕСТОВ ЛАДОНИ В~РЕАЛЬНОМ ВРЕМЕНИ НА~ОСНОВЕ ПЛОСКИХ 
И~ПРОСТРАНСТВЕННЫХ СКЕЛЕТНЫХ МОДЕЛЕЙ$^*$}

\def\titkol{Распознавание жестов ладони в реальном времени на основе плоских и 
пространственных скелетных моделей}

\def\autkol{А.\,В.~Куракин}
\def\aut{А.\,В.~Куракин$^1$}

\titel{\tit}{\aut}{\autkol}{\titkol}

{\renewcommand{\thefootnote}{\fnsymbol{footnote}}\footnotetext[1]
{Работа выполнена при финансовой поддержке РФФИ, проекты №\,11-01-00783 и №\,11-07-00462.}}


\renewcommand{\thefootnote}{\arabic{footnote}}
\footnotetext[1]{Московский физико-технический институт (государственный университет), 
alekseyvk@yandex.ru}


\Abst{Рассмотрена задача распознавания жестов ладони и определения координат частей ладони в пространстве на основе анализа формы силуэта ладони.
Для ее решения разработан метод определения координат кончиков пальцев на бинарном изображении ладони посредством анализа его скелетного представления.
Предложен алгоритм определения координат пальцев и центра ладони в пространстве за счет анализа стереопары изображений силуэтов ладони.
Разработанные методы работают в реальном времени, что позволяет использовать их в прикладных системах распознавания жестов.}

\KW{непрерывный скелет; анализ формы; распознавание жестов; стереозрение}

 \vskip 14pt plus 9pt minus 6pt

      \thispagestyle{headings}

      \begin{multicols}{2}

            \label{st\stat}

\section{Введение}\label{sec:introduction}

О важности задачи распознавания жестов говорит огромное число работ в данной области.
Потенциальные приложения технологий распознавания жестов включают че\-ло\-ве\-ко-ма\-шин\-ное 
взаимодействие~\cite{BareHand3d2006},
приложения виртуальной реальности~\cite{GaVeDi}, распознавание языка 
жестов~\cite{GestureRecognitionVisapp2007} 
и др.~\cite{GestureRecognitionSurvey, VisionBasedReview2009}.

В данной работе рассматривается задача распознавания жестов ладони 
по видеопоследовательностям изображений, полученным с одной или двух веб-ка\-мер.
Для решения поставленной задачи в работе предлагается метод, работающий на 
основе анализа формы ладони.
Использование только формы, без текстурных признаков, позволяет применять 
метод даже на изображениях низкого качества, полученных с веб-камер.

Анализ формы выполняется на основе использования непрерывного скелета~---  
множества центров максимальных вписанных в силуэт исходной фигуры кругов.
Непрерывный скелет является весьма информативным дескриптором формы,
позволяет анализировать топологию объекта и измерять такие его признаки, 
как ширина объекта в произвольной точке скелета.
Понятие непрерывного скелета описано в разд.~\ref{sec:ContinuousSkeleton}.

В работе предлагается метод выделения кончиков 
пальцев, вводится понятие центра ладони и предлагается метод его вычисления.
Эти данные используются как признаки для классификации жес\-тов.
Рассматриваемый в разд.~\ref{seq:PalmShapeAnalysis} 
алгоритм позволяет найти двухмерные координаты кончиков 
пальцев и центра ладони на основе сегментации скелета силуэта ладони.
Предлагается простой классификатор жестов рук по вычисленным характеристикам.
Более того, алгоритм допускает обобщение на случай трех измерений за счет 
использования стереопары силуэтов ладони в качестве входных данных
(см.\ разд.~\ref{sec:HandTracking3D}).

Для апробации методов разработаны ап\-па\-рат\-но-про\-грам\-мные 
комплексы, состоящие из одной или двух веб-ка\-мер и программного 
обеспечения для распознавания жестов.
Они описаны в разд.~\ref{sec:Experiments}.

Рассмотренные методы работают в реальном времени и допускают 
применение в практических системах распознавания жестов.

\vspace*{-6pt}

\section{Обзор литературы}\label{sec:RelatedWork}

В литературе рассмотрено большое количество подходов к решению задачи распознавания жестов.
Эти подходы можно классифицировать по типу используемых входных данных и сенсоров для 
восприятия руки.
Во-пер\-вых, это методы, исполь\-зу\-ющие специализированные невизуальные сенсоры, такие 
как роботизированные перчатки~\cite{GaVeDi}.
Во-вто\-рых, методы, использующие визуальную информацию, 
но требующие от пользователя размещать на руке ка\-кие-ли\-бо маркеры, облегчающие 
трекинг и классификацию позы ладони~\cite{ColorGlove2009}.
В-третьих, методы, подобные рассмотренному в данной статье, 
которые работают исключительно с визуальной информацией и
не предъявляют специальных требований к оснащению пользователя 
дополнительным оборудованием~\cite{VisionBasedReview2009}.


Визуальные методы распознавания жестов можно разделить на три большие категории.
К~первой\linebreak\vspace*{-12pt}

\pagebreak

\noindent
 относятся методы, которые основаны на вос\-ста\-нов\-ле\-нии полной модели кисти 
с 27~степенями свободы по входному изображению~\cite{ArticulatedHandTracking3d}.
Теоретически это наиболее перспективные методы, так как 
они подразумевают полное оценивание позы и динамики руки.
Основными ограничениями подобных методов являются большая вы\-чис\-ли\-тель\-ная слож\-ность 
и ограниченная точность восстановления модели руки из-за наличия окклюзий,
что делает невозможным их применение на практике.

Ко второй категории относятся методы, которые вместо восстановления полной 
модели руки предлагают построение признакового описания входного изображения 
и дальнейшую классификацию жес\-тов именно по этому описанию.
В работах~\cite{BareHand3d2006, GestureRecognitionVisapp2007} в качестве 
такого описания используются макрохарактеристики силуэта ладони (размеры, положение, 
инварианты Ху~\cite{HuInvariants}).

К третьей категории относятся метрические методы распознавания жестов.
Подобные методы предполагают построение некоторой метрики на множестве 
входных изображений и выполнение классификации за счет сравнения входного 
изображения с набором эталонов.
Например, в~\cite{StableSkeletonization2010} предлагается метрика, 
характеризующая степень сходства скелетов силуэтов ладони, и выполняется 
классификация жестов с помощью метода ближайшего соседа.

Данная работа относится ко второй категории визуальных методов: по изображению 
генерируются признаки, на основе которых выполняется классификация жестов.
В качестве признаков используются координаты кончиков пальцев и центра ладони.

\section{Непрерывный скелет}\label{sec:ContinuousSkeleton}

Рассматриваемые в статье методы анализа формы базируются на понятии непрерывного скелета фигуры.
Определение непрерывного скелета вмес\-те со способом его получения кратко описано в этом разделе.
Более детальную информацию можно найти в книге~\cite{MestSkeletonBook}.

Для многоугольной фигуры~$F$ \textit{максимальным пус\-тым кругом} будем называть всякий круг~$B$,
полностью содержащийся внутри фигуры~$F$, такой что любой другой круг~$B'$, содержащийся 
внут\-ри фигуры~$F$, не содержит в себе~$B$, т.\,е.\ 
$\forall B' \subset F$,\linebreak $B' \not= B : B \not\subset B'$.

Используя понятие максимального пустого круга, определим скелет следующим образом:

\medskip

\noindent
\textbf{Определение 1.} %\label{def:Skeleton}
\textit{Скелетом многоугольной фигуры $F$ является множество центров ее 
максимальных пус\-тых кругов.}

\smallskip

На скелете определена \textit{радиальная функция} $R(x,y)$, которая 
ставит в соответствие каждой точке скелета $(x,y)$ значение радиуса 
максимального пустого круга с центром в этой точке.

Можно доказать~\cite{MestetskiySkeletonBezierCurves}, что скелет 
многоугольной фигуры состоит из объединения конечного числа отрезков и дуг парабол.
Таким образом, скелет многоугольной фигуры можно рассматривать как 
планарный граф, вершины которого~--- это точки соединения отрезков и дуг парабол, а ребра~--- 
это собственно отрезки и дуги парабол, составляющие скелет.
Степень любой вершины в таком графе будет рав\-на~1, 2 или~3.

Далее в статье будут использоваться свойства скелета и как графа, и как объединения кривых.

На практике при анализе формы зачастую приходится 
иметь дело не с многоугольниками, а с растровыми бинарными изображениями, на 
которых один из двух цветов обозначает принадлежность соответствующего пиксела объекту.
Соответственно, для построения непрерывного скелета необходимо в первую очередь построить 
многоугольную аппроксимацию исходной фигуры.
Далее для полученного многоугольника можно строить скелет.
Существующие эффективные алгоритмы позволяют выполнять построение скелета за время 
$O(N \log N)$,
где $N$~--- чис\-ло вершин в многоугольнике.
После построения скелета обычно выполняется его дополнительная обработка, 
называемая <<стрижкой>>, с целью удаления малозначимых и шумовых ветвей.
На рис.~\ref{fig:SkeletonConstruction} продемонстрирован процесс построения 
скелета на примере изображения ладони низкого разрешения.

\vspace*{-6pt}

\section{Анализ формы ладони и~распознавание жестов с~помощью скелета}\label{seq:PalmShapeAnalysis}

Распознавание жестов выполняется путем анализа бинарного изображения силуэта ладони.
Сначала для бинарного изображения ладони выполняется построение скелета и его 
регуляризация (<<стрижка>>).
Затем выполняется поиск координат кончиков пальцев и центра ладони на силуэте.
Далее выполняется распознавание жестов за счет анализа числа видимых пальцев и 
динамики их взаимного перемещения.

\begin{figure*} %fig1
\vspace*{1pt}
 \begin{center}
 \mbox{%
 \epsfxsize=145.854mm
 \epsfbox{kur-1.eps}
 }
 \end{center}
 \vspace*{-9pt}
\Caption{Процесс построения скелета: исходная бинарная картинка~(\textit{а}); 
многоугольная аппроксимация границы объекта~(\textit{б});
скелет многоугольника~(\textit{в}); скелет после стрижки~(\textit{г})
 \label{fig:SkeletonConstruction}}
\end{figure*}


\smallskip

\noindent
\textbf{Определение 2.} %\begin{Def}\label{def:HandCenter}
\textit{Центром ладони будем считать центр вписанного в силуэт ладони круга, имеющего максимальный радиус среди всех вписанных кругов.
}

\smallskip

Центр ладони определяется по скелету как точка, в которой радиальная 
функция принимает максимальное значение.
Алгоритм поиска пальцев описан ниже в подразд.~4.2,
а в подразд.~4.1 вводятся необходимые определения.

Состав рассматриваемых жестов и их назначение в разработанном программном комплексе 
(см.\ разд.~\ref{sec:Experiments}) следующие:
\begin{enumerate}[(1)]
\item виден один палец (см.\ рис.~4,\,\textit{б}). 
Палец и рука движутся как целое. Этот жест используется для перемещения 
курсора и перетаскивания объектов;
\item видны два пальца, и их кончики двигаются вдоль прямой (см.\ рис.~\ref{fig:DemoSoftware2D}). 
Этот жест используется для изменения размера объектов;
\item видны три пальца, и они двигаются так, чтобы длины 
сторон образованного ими треугольника менялись незначительно во времени 
(см.\ рис.~4,\,\textit{а}). 
Этот жест используется для вращения объектов;
\item большой и указательный пальцы образуют кольцо на небольшой фиксированный промежуток 
времени (см.\ рис.~2,\,\textit{а}).
    Этот жест используется для захвата объектов;
\item все пять пальцев видны и широко расставлены на 
небольшой фиксированный промежуток времени (см.\ рис.~\ref{fig:SkeletonConstruction}).
    Этот жест используется для освобождения объектов.
\end{enumerate}





Жесты с 1, 2, 3 и~5 пальцами легко распознаются путем подсчета числа видимых пальцев.
Для жестов из двух и трех пальцев начальное и конечное положение пальцев используется 
для определения количественных характеристик жеста,
таких как коэффициент масштабирования при изменении 
размера объекта и угол поворота при вращении.

Жест <<кольцо>> распознается путем поиска цик\-лов в скелетном графе.
Следует заметить, что только циклы, проходящие вблизи центра ладони, 
рас\-смат\-ри\-ва\-ют\-ся как подходящие (рис.~2),
потому что только такие циклы образованы кольцом между большим и указательным пальцами.
Такая\linebreak\vspace*{-12pt}
\begin{center} %fig2
\vspace*{1pt}
\mbox{%
 \epsfxsize=78.046mm
 \epsfbox{kur-2.eps}
}
\end{center}
%\begin{center}
\vspace*{1pt}
{{\figurename~2}\ \ \small{Циклы в скелете: цикл~(\textit{а}) соответствует жес\-ту-коль\-цу, 
  а цикл~(\textit{б}) жес\-том-коль\-цом не является
}}
%\end{center}
\vspace*{9pt}

\smallskip
\addtocounter{figure}{1}


\noindent
 классификация возможна благодаря тому, что известны координаты всех вершин скелета.
Более того, для обнаружения кольца нет необходимости анализировать контуры ладони, а 
достаточно лишь использования скелетного графа.

Жест <<кольцо>> и жест из пяти пальцев распознаются как динамические жесты,
т.\,е.\ измеряется время наблюдения жеста и он засчитывается, только если 
это время превышает заданный порог.
Динамическое распознавание жестов стало возможным благодаря высокой 
скорости обработки кадров.

\subsection{Ветвь скелета и~ее~свойства}\label{seq:GestureRecognition_SkeletonBranch}

Анализ формы ладони производится на основе анализа ветвей скелета.
Ветвь скелета~--- это прос\-то часть скелета, рассмотренная как непрерывная кривая.
Формальное определение дано ниже.

\medskip

\noindent
\textbf{Определение 3.} %\begin{Def}\label{def:SkeletonBranch}
\textit{Пусть $\vec s(\bullet): s(l) \hm= \{ x(l), y(l) \}, l \hm\in [0, L]$,~--- 
непрерывная ку\-соч\-но-глад\-кая кривая без самопересечений и $l$ является 
естественной па\-ра\-мет\-ри\-за\-ци\-ей кривой (т.\,е.\ длиной дуги кривой).
Пусть каждая точка кривой $\vec s(\bullet)$ является одновременно и точкой скелета.
В таком случае кривую $\vec s(\bullet)$, 
соединяющую точки скелета $r(0)$ и $r(L)$, будем называть} ветвью скелета.

\begin{figure*}[b] %fig3
\vspace*{-6pt}
 \begin{center}
 \mbox{%
 \epsfxsize=162.248mm
 \epsfbox{kur-3.eps}
 }
 \end{center}
 \vspace*{-12pt}
\Caption{Пример радиальной функции для пальца~(\textit{а}) и для ветви, не являющейся 
пальцем~(\textit{б})
\label{fig:FingerAndNonFingerR}}
\end{figure*}

Для каждой точки скелета с координатами $(x,y)$ известно значение радиальной 
функции $R(x,y)$, равное радиусу максимального пустого круга с центром в этой точке.
Дополнительно для произвольной ветви скелета $\vec s(\bullet)$ будем рассматривать 
\textit{радиальную функцию вдоль ветви} $R_{s}(l) \hm= R(\vec s(l))$, $l \hm\in [0, L]$.

Из-за наличия в скелете дуг парабол работа с радиальной функцией вдоль ветви 
сопряжена с вычислительными сложностями.
Но у скелетов реальных изображений дуги парабол очень короткие, имеют малую кривизну 
и приближенно могут быть рассмотрены как отрезки.
Таким образом, заменив все ду\-ги-па\-ра\-бо\-лы на отрезки, получим скелет, 
для которого радиальная функция вдоль любой ветви будет ку\-соч\-но-ли\-ней\-ной.
Будем называть ее \textit{аппроксимированной радиальной функцией вдоль ветви}, 
а вычислять описанным ниже способом.

Рассмотрим две вершины исходного скелета~$A$ и~$B$ и простой путь $P$ 
между ними в скелетном графе.
Путь~$P$ однозначно определяет ветвь скелета $\vec s(\bullet)$.

Обозначим вершины скелетного графа, входящие в путь~$P$ (в порядке прохождения пути), 
как $V_0 \hm= A$, $V_1, \dots, V_{n-1}$, $V_n \hm= B$.

Обозначим значения радиальной функции в этих вершинах как $R(V_i) \hm= R_i$.

Пусть $L_i$~--- длина пути между вершинами~$A$ и $V_i$ в предположении, 
что все дуги скелета являются отрезками, т.\,е.\ $L_i \hm= \sum\limits_{k=0}^{i-1} |V_{k}V_{k+1}|$.

В~данных обозначениях аппроксимированная радиальная функция $\tilde R_{\tilde s}(l)$ 
вдоль ветви $\vec s(\bullet)$ может быть вычислена следующим образом:
\begin{equation*}
\tilde R_{\tilde s}(l) =
\begin{cases}
R_i & \mbox{при } l = L_i;  \\
\alpha R_i + (1-\alpha)R_{i+1} &
% \begin{array}{l}
    \mbox{при } L_i < l < L_{i+1}\,,\\
&    \hspace*{5mm}\alpha = \fr{l - L_i}{L_{i+1} - L_{i}}\,.
 % \end{array}
\end{cases}
\end{equation*}


%\begin{equation}
%\tilde R_{\tilde s}(l) =
%\begin{cases}
%R_i &\ \mbox{при}\ l = L_i\,; \\
%\alpha R_i + (1-\alpha)R_{i+1} &\ \mbox{при}\ l ={}\\
%&\hspace*{-30mm}{}= \alpha L_i + (1-\alpha)L_{i+1},\
%        \alpha \in (0,1)\,.
%\end{cases}
%\end{equation}

\subsection{Распознавание пальцев}\label{sec:GestureRecognition_FingerDetection}

Поиск пальцев на силуэте ладони производится путем анализа ветвей скелета,
 а также радиальной функции вдоль них.

Анализ силуэтов ладоней разных людей по собранной базе примеров 
показал, что ветви, соответствующие пальцам, имеют набор сходных особенностей.

Во-первых, все ветви, соответствующие пальцам, оканчиваются в висячих вершинах скелета,
поэтому далее рассматриваем только ветви, соединяющие висячие вершины с вершинами степени~3.

Во-вторых, каждую ветвь, соответствующую пальцу, можно условно разбить на две части: 
собственно палец и пясть.
При этом для пальца радиальная функция колеблется в небольших пределах около значения, 
являющегося полушириной пальца, а для пясти она значительно растет.

График радиальной функции для пальца приведен на рис.~\ref{fig:FingerAndNonFingerR},\,\textit{а}.
Заметим, что для всех ветвей, соответствующих пальцам, радиальная функция выглядит сходным 
образом; отличия касаются только конкретного значения ширины пальца, а также положения 
точки, где заканчивается палец и начинается пясть.

В свою очередь, для ветвей, не являющихся пальцами, радиальная функция 
может иметь различный вид, но он всегда отличается от вида радиальной функции для пальца.
Один из примеров радиальной функции для ветви, не являющейся пальцем, приведен на 
рис.~\ref{fig:FingerAndNonFingerR},\,\textit{б}.



С учетом этих особенностей предлагается следующий эвристический 
алгоритм для распознавания пальцев:
\begin{enumerate}[1.]
\item Анализируем все ветви скелета, соединяющие висячие вершины скелета с ближайшими вершинами степени 3.
\item Для каждой такой ветви выполняем поиск точки~$C$, которая является наиболее вероятным местом сочленения пальца и пясти.
\item Когда определена точка~$C$, проверяем набор критериев на геометрические размеры (длину, толщину) ветви, чтобы отсечь те ветви, которые не являются пальцами.
\end{enumerate}

Будем использовать обозначения $V_i$, $R_i$ и $L_i$ из предыдущего подраздела
и введем дополнительно величину $D_i$ как дискретную производную~$R$ по~$L$.
Положим $D_0 \hm= 0$, $D_n \hm= + \infty$, а в остальных точках~$D_i$ будем вычислять по формуле:
\begin{equation*}
D_i = \fr{R_{i+1} - R_{i-1}}{L_{i+1} - L_{i - 1}}\,,\enskip \quad i = 1,
\ldots, n-1\,.
\end{equation*}

Поиск точки~$C$ выполняется из тех соображений, что в момент, 
когда заканчивается палец и начинается пясть,
происходит выполнение одного из следующих условий:
\begin{itemize}
\item $R$ увеличивается более чем в 2--2,5~раза по сравнению с началом пальца;
\item радиус начинает резко расти, т.\,е.\ частные производные~$D_i$ 
превосходят наперед заданный порог (использовалось значение порога, рав\-ное 0,4--0,6).
\end{itemize}

После того как для сегмента~$AB$ найдена точка~$C$ вероятного
сочленения пальца и ладони, выполняется вычисление длины сегментов~$AC$ и~$AB$ 
и толщины пальца (как радиуса максимальной вписанной
окружности в определенной точке скелета либо как среднего значения
радиуса вдоль $AC$). Сегмент~$AB$ классифицируется как палец, если
выполняются все следующие условия:
\begin{itemize}
\item $|AC| / |AB| \ge 0{,}35$;
\item толщина ветви $AC$ находится в заданных пределах;
\item длина $|AB|$ превышает величину порога (т.\,е.\ 
ветвь, классифицируемая как палец, должна быть достаточно длинной).
\end{itemize}

Конкретные значения параметров алгоритма получены эмпирическим путем 
по собранной базе изображений ладони.

На рис.~4 приведен пример результата 
работы алгоритма детектирования пальцев.
На нем изображены самые большие круги, вписанные в ладонь,
и маленькие круги, соответствующие кончикам пальцев и местам сочленения пальцев и ладони.


\section{Анализ стереопары ладоней и~слежение за~рукой в~пространстве}\label{sec:HandTracking3D}

Алгоритм анализа формы ладони, описанный в предыдущем разделе,
можно расширить для определения трехмерных координат руки и 
кончиков пальцев за счет использования стереопары силуэтов ладоней.

Рука человека может быть приближенно описана в виде циркулярной модели~--- 
пространственного графа, с каждой точкой которого связана сфера.
Если рука наблюдается без окклюзий, т.\,е.\ разные точки руки проецируются в 
разные точки ее изображения,
то скелет силуэта проекции руки приближенно совпадает с проекцией 
пространственного графа, порождающего руку~\cite{GeneralizedCylinders1995}.
Благодаря этому свойству можно выполнять восстановление модели руки по 
стереопаре силуэтов~\cite{Tsiskaridze2009, Tsiskaridze2010}.
Однако метод не будет работать при наличии окклюзий.

С другой стороны, для практических задач восстановление полной модели руки зачастую не нужно.
В ситуации, когда присутствуют окклюзии, можно выполнять частичное восстановление 
модели руки по ключевым точкам.
В качестве таких ключевых точек могут выступать, например, кончики пальцев и центр ладони.
Идея состоит в том, что, выполнив поиск этих точек на обоих изображениях стереопары 
силуэтов руки и найдя соответствие между точками, можно вычислить их координаты в пространстве.
Детально алгоритм определения пространственного положения кончиков пальцев и центра ладони 
описан ниже.

Пусть имеется пара откалиброванных камер~$P_1$ и $P_2$, наблюдающих искомую модель руки.
Обозначим через $S_1$ и $S_2$ силуэты искомой модели в проекции на плоскости камер~$P_1$ 
и~$P_2$.

Построим скелеты $M_1$ и $M_2$ для силуэтов $S_1$ и~$S_2$.

\setcounter{figure}{4}
\begin{figure*}[b] %fig5
 \vspace*{6pt}
 \begin{center}
 \mbox{%
 \epsfxsize=136.269mm
 \epsfbox{kur-5.eps}
 }
 \end{center}
 \vspace*{-9pt}
  \Caption{Иллюстрация работы комплекса для распознавания жестов в двух измерениях: изображен жест, используемый для масштабирования
  \label{fig:DemoSoftware2D}}
%\end{figure*}
%\begin{figure*} %fig6
 \vspace*{15pt}
 \begin{center}
 \mbox{%
 \epsfxsize=162.029mm
 \epsfbox{kur-6.eps}
 }
 \end{center}
 \vspace*{-9pt}
  \Caption{Иллюстрация работы комплекса для слежения за рукой в трехмерном пространстве:
  стереопара изображений, полученных с веб-ка\-мер~(\textit{а});
  модель руки, визуализированная в виртуальном трехмерном пространстве~(\textit{б})
  \label{fig:DemoSoftware3D}}
\end{figure*}

Для каждого из полученных скелетов выпол-\linebreak ним детектирование кончиков пальцев 
и центра ла\-дони, описанное в подразд.~4.2. 
Множества найден\-ных точек $\{A^1_i\hm=(x^1_i, y^1_i), i=1, \ldots , N_1\}$ и 
$\{A^2_i\hm=(x^2_i, y^2_i), i=1, \ldots , N_2\}$ для изображений,
полученных с первой и второй камер соответственно, являются ключевыми точками, 
используемыми далее для определения положения частей искомой модели в пространстве.


Для найденных ключевых точек выполним определение соответствия между ними. 
Алгоритм определения соответствия следующий:
\begin{itemize}
\item с помощью эпиполярной геометрии~\cite{MultiViewGeometry2004} выполняется 
сопоставление центров ладоней на паре кадров (если в кадре присутствует несколько ладоней);
\begin{center} %fig4
\vspace*{1pt}
\mbox{%
 \epsfxsize=70.232mm
 \epsfbox{kur-4.eps}
}
\end{center}
\begin{center}
%\vspace*{3pt}
{{\figurename~4}\ \ \small{Пример детектирования пальцев}}
\end{center}
\vspace*{3pt}

%\smallskip
%\addtocounter{figure}{1}

\item для каждой пары соответствующих проекций ладоней выполняется определение соответствия 
между кончиками пальцев этой ладони. При этом используется как эпиполярная геометрия, так и 
тот факт, что ориентация пальцев относительно центра ладони должна быть одинаковой на обоих 
кадрах.
\end{itemize}



В результате получается набор стереопар $\{A^1_{f(t)}, A^2_{g(t)}\}$, 
где $t \hm= 1, \ldots , N$, а $f(t)$ и $g(t)$~--- целочисленные функции, 
определяющие соответствие между точками.
При этом точки, для которых не нашлось пары на другом изображении, 
не участвуют в дальнейшем рассмотрении.

Для всех найденных стереопар выполним стереотриангуляцию~\cite{MultiViewGeometry2004} 
и вычислим трехмерные координаты их прообразов.
В результате получим пространственные координаты кончиков пальцев и центра ладони.

\section{Аппаратно-программный комплекс}\label{sec:Experiments}

Для экспериментов и демонстрации вышеописанных методов разработаны 
два ап\-па\-рат\-но-про\-грам\-мных комплекса.
Один из них выполняет распознавание жестов руки посредством анализа 
изображения, полученного с одной веб-ка\-ме\-ры.
Второй выполняет анализ изображений с двух веб-ка\-мер и определяет 
положение руки и кончиков пальцев в трехмерном пространстве.

Для двухмерного распознавания жестов используется следующий комплекс.
Обычная веб-ка\-ме\-ра (Logitech 9000) располагается над однородной темной по\-верх\-ностью.
Пользователь двигает рукой\linebreak
 перед поверхностью, и изображение снимается камерой.
С использованием точечных методов детектирования 
кожи~\cite{PixelBaseSkinSurveyGraphicon,SkinSegmentaionComparison} выделяется силуэт ладони.
Он анализируется алгоритмами из разд.~\ref{seq:PalmShapeAnalysis}, в результате чего 
осуществляется распознавание жес\-тов.
Обнаруженные жесты используются для перемещения, вращения и масштабирования 
объектов на экране компьютера.
На рис.~\ref{fig:DemoSoftware2D} приведены снимки экрана данной программы.



В аппаратно-программном комплексе для восстановления положения ладони 
и пальцев в трехмерном пространстве используется пара отка\-либ\-ро\-ван\-ных веб-ка\-мер.
В~остальном он повторяет двухмерный вариант.
На паре изображений, полученных с камер, выделяются силуэты руки и производится 
их обработка методом, описанным в разд.~\ref{sec:HandTracking3D}.
На рис.~\ref{fig:DemoSoftware3D} приведен пример его работы.


Эффективные алгоритмы построения и стрижки скелета позволяют использовать 
системы для слежения за рукой в реальном времени.
Например, для однопоточной реализации двухмерного алгоритмы распознавания жестов 
(включая сегментацию руки, построение и стрижку скелета, распознавание жестов и 
рисование результата) требуется около 22~мс на обработку одного кадра на компьютере 
2,4~ГГц Intel Core~2 Quad CPU.

\section{Заключение}\label{sec:Conclusion}

В работе рассмотрен метод распознавания жес\-тов ладони.
Предложен алгоритм определения положения кончиков пальцев 
и центра ладони с\linebreak помощью анализа скелета силуэта ладони.
Координаты кончиков пальцев и центра ладони могут быть 
найдены как на плоскости, при наличии одного изображения,
так и в пространстве, при наличии стереопары силуэтов.
Найденный набор координат используется как признаковое описание для классификации жестов.

Среди достоинств метода следует отметить:
\begin{itemize}
\item использование исключительно силуэта, что 
обеспечивает независимость от текстуры руки и освещения;
\item возможность применения недорогих веб-ка\-мер в качестве сенсоров;
\item возможность обработки видео в реальном времени.
\end{itemize}

Среди дальнейших направлений исследований можно отметить следующие:
\begin{enumerate}[(1)]
\item  усложнение набора распознаваемых жес\-тов.
В~част\-ности, мож\-но реализовать распознавание слож\-ных 
динамических жес\-тов с помощью скрытых марковских моделей;
\item расширение спектра возможных применений метода, 
например адаптация метода для рас-\linebreak\vspace*{-12pt}
\columnbreak

\noindent
познавания поз и жестов тела человека по силуэтам фигуры;
\item разработка методов сегментации руки на произвольном фоне.
\end{enumerate}

{\small\frenchspacing
{%\baselineskip=10.8pt
\addcontentsline{toc}{section}{Литература}
\begin{thebibliography}{99}

\bibitem{BareHand3d2006}
\Au{Dhawale~P., Masoodian~M., Rogers~B.} 
Bare-hand 3d gesture
  input to interactive systems~// CHINZ'06: 7th ACM SIGCHI
  New Zealand Chapter's  Conference (International) on Computer--Human Interaction:
  Design Centered HCI Proceedings.~---
New York, NY, USA: ACM, 2006.
P.~25--32.

\bibitem{GaVeDi}
\Au{Aguiar~R., Pereira~J.~M., Braz~J.} Gadevi~--- game
  development integrating tracking and visualization devices into virtools~//
  GRAPP 2009:  4th Conference (International) on Computer
  Graphics Theory and Applications Proceedings.~---
INSTICC Press, 2009. P.~313--321.

\bibitem{GestureRecognitionVisapp2007}
\Au{Burger~T., Urankar~A., Aran~O., Akarun~L., Caplier~A.}
  Cued speech hand shape recognition~--- belief functions as a
  formalism to fuse svms and expert systems~// VISAPP 2007: 
2nd Conference (International) on Computer Vision Theory and Applications
  Proceedings.~---
INSTICC Press, 2007.  Vol.~2. P.~5--12.

\bibitem{GestureRecognitionSurvey}
\Au{Mitra~S., Acharya~T.} Gesture recognition: A survey~//
  IEEE Trans. Syst. Man Cybernetics, Part~C, 2007. Vol.~37. No.\,3.
P.~311--324.

\bibitem{VisionBasedReview2009}
\Au{Garg~P., Aggarwal~N., Sofat~S.} Vision based hand gesture
  recognition~// World Academy Sci. Engng. Technol., 2009.
P.~972--977.

\bibitem{ColorGlove2009}
\Au{Wang~R.~Y., Popovi{\'c}~J.} Real-time hand-tracking with
  a color glove~// ACM Trans. Graphics, 2009. Vol.~28. No.\,3.

\bibitem{ArticulatedHandTracking3d}
\Au{Liu~T., Liang~W., Jia~Y.} 3d articulated hand tracking by
  nonparametric belief propagation on feasible configuration space~// VISAPP
  2008:  3rd  Conference (International) on Computer Vision
  Theory and Applications Proceedings.~---  INSTICC Press, 2008.  Vol.~2.
P.~508--513.

\bibitem{HuInvariants}
\Au{Hu~M.-K.} Visual pattern recognition by moment
  invariants~// IRE Trans. Information Theory, 1962. Vol.~8. No.\,2.
P.~179--187.

\bibitem{StableSkeletonization2010}
\Au{Beristain~A., Grana~M.} A stable skeletonization for
  tabletop gesture recognition~// Computational science and its
  applications~--- ICCSA 2010~/ Eds. D.~Taniar, O.~Gervasi, B.~Murgante,
  E.~Pardede, B.~Apduhan.~---
 Berlin/Heidelberg: Springer, 2010.   Lecture notes in computer science ser. Vol.~6016.
P.~610--621.

\bibitem{MestSkeletonBook}
\Au{Местецкий~Л.\,М.} Непрерывная морфология бинарных
  изображений: фигуры, скелеты, циркуляры.~--- М.: Физматлит, 2009.

\bibitem{MestetskiySkeletonBezierCurves}
\Au{Mestetskiy~L.} Skeleton representation based on compound
  Bezier curves~// VISAPP 2010:  5th Conference (International)
  on Computer Vision Theory and Applications Proceedings.~---
 INSTICC Press, 2010.  Vol.~1. 
 
 \bibitem{GeneralizedCylinders1995}
\Au{Pillow~N., Utcke~S., Zisserman~A.} Viewpoint-invariant
  representation of generalized cylinders using the symmetry set~//
  Conference on British Machine Vision\linebreak\vspace*{-12pt}
  \pagebreak
  
  \noindent
  Proceedings.~---
Surrey, UK: BMVA Press, 1994.  Vol.~2.  P.~539--548.

\bibitem{Tsiskaridze2009}
\Au{Mestetskiy~L., Tsiskaridze~A.} Spatial reconstruction of
  locally symmetric objects based on stereo mate images~// VISAPP 2009:
  4th Conference (International)  on Computer Vision Theory
  and Applications Proceedings.~---  INSTICC Press, 2009.  Vol.~1.  P.~443--448.

\bibitem{Tsiskaridze2010}
\Au{Цискаридзе~A.\,К.} Математическая модель и метод
  восстановления позы человека по стереопаре силуэтных изображений~//
  Информатика и её применения, 2010. Т.~4. Вып.~4. С.~26--32.

\bibitem{MultiViewGeometry2004}
\Au{Hartley~R.\,I., Zisserman~A.} Multiple view geometry in
computer vision.~--- 2nd~ed.~--- Cambridge University Press, 2004.

\bibitem{PixelBaseSkinSurveyGraphicon}
\Au{Vezhnevets~V., Sazonov~V., Andreeva~A.} A survey on
  pixel-based skin color detection techniques~//  GraphiCon Proceedings,
  2003. P.~85--92.
  
  \label{end\stat}

\bibitem{SkinSegmentaionComparison}
\Au{Phung~S.\,L., Bouzerdoum~A., Chai~D.} Skin segmentation
  using color pixel classification: Analysis and comparison~// IEEE
  Trans. Pattern Anal. Mach. Intell, 2005. Vol.~27.
P.~148--154.
 \end{thebibliography}
}
}


\end{multicols}