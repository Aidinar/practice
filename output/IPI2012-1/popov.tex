

\newcommand{\Iy}{\mathbb{I}}



%\renewcommand{\endproof}{\hfill$\Box$}


%\newcommand{\si}{{\rm Si}\:}
%\renewcommand{\Re}{{\rm Re}\:}
%\newcommand{\eqd}{\stackrel{d}{=}}

\def\stat{popov}

\def\tit{УТОЧНЕНИЕ НЕРАВНОМЕРНЫХ ОЦЕНОК СКОРОСТИ СХОДИМОСТИ 
В~ЦЕНТРАЛЬНОЙ ПРЕДЕЛЬНОЙ ТЕОРЕМЕ ПРИ~СУЩЕСТВОВАНИИ МОМЕНТОВ НЕ~ВЫШЕ
ВТОРОГО}

\def\titkol{Уточнение неравномерных оценок скорости сходимости в
центральной предельной теореме} % при существовании моментов не выше второго}

\def\autkol{С.\,В.~Попов}
\def\aut{С.\,В.~Попов$^1$}

\titel{\tit}{\aut}{\autkol}{\titkol}

%{\renewcommand{\thefootnote}{\fnsymbol{footnote}}\footnotetext[1]
%{Работа выполнена при поддержке РФФИ (гранты 09-07-12098, 09-07-00212-а и 
%09-07-00211-а) и Минобрнауки РФ (контракт №\,07.514.11.4001).}}


\renewcommand{\thefootnote}{\arabic{footnote}}
\footnotetext[1]{Факультет вычислительной
математики и кибернетики Московского государственного университета
им.\ М.\,В.~Ломоносова, popovserg@yandex.ru}


\Abst{В статье уточняются неравномерные оценки скорости
сходимости в центральной предельной теореме для сумм независимых
случайных величин, у которых существуют моменты не выше второго.}

\KW{центральная предельная теорема; оценка
скорости сходимости; абсолютные константы}

 \vskip 14pt plus 9pt minus 6pt

      \thispagestyle{headings}

      \begin{multicols}{2}
      
            \label{st\stat}


\section{Введение}

Многие исследователи отмечают, что при статистическом анализе тех
или иных характеристик трафика в информационных системах возникают
вероятностные распределения со столь тяжелыми хвостами, что можно
предполагать наличие моментов лишь второго порядка. Как известно, к
таким распределениям применима центральная предельная теорема,
однако скорость вытекающей из нее сходимости к нормальному закону в
таком случае может быть как угодно медленной и определяется
поведением хвостов распределений исходных величин. В~данной работе
строятся неравномерные оценки скорости сходимости в центральной
предельной теореме в терминах <<квадратичных хвос\-тов>>.


Пусть $X_1,X_2,\ldots$~--- независимые случайные величины с $\e
X_i=0$ и $\e X_i^2<\infty$. Для $n\in\N$ положим
$W_n=X_1+\cdots+X_n.$ Предположим, что $\D W_n=\e X_1^2+\cdots+\e
X_n^2=1$. Пусть $\Phi(x)$~-- стандартная нормальная функция
распределения,
$$
\Phi(x)=\fr{1}{\sqrt{2\pi}}\int\limits_{-\infty}^{x}e^{-z^2/2}\,dz\,,\
\ \ \ x\in\R\,.
$$
Для $x\in\R$ обозначим
$$
\Delta_x=|\p(W_n<x)-\Phi(x)|\,.
$$
В работе~\cite{ChSh2001} показано, что существует положительная
конечная постоянная $C$ такая, что для любого $x\in\R$ выполнено
неравенство:
\begin{multline}
\label{GeneralEq}
\Delta_x\leqslant C\sum\limits_{i=1}^n\left[\fr{\e X_i^2\Iy(|X_i|\geqslant
1+|x|)}{(1+|x|)^2}+{}\right.\\
\left.{}+\fr{\e |X_i|^3\Iy(|X_i|<
1+|x|)}{(1+|x|)^3}\right]\,.
\end{multline}

В некоторых работах предпринимались попытки оценить значение
константы~$C$ в неравенстве~\eqref{GeneralEq}. Прежде всего в связи с этой
задачей необходимо упомянуть недавние работы~\cite{TN2007, NT2007}. 
Из результатов последней из указанных статей вытекает
наилучшая (насколько известно автору) на сегодняшний день верхняя
оценка константы~$C\leqslant 76{,}17$.

Также необходимо отметить работу~\cite{KP2011_Neam}, в которой для
случая одинаково распределенных слагаемых показано, что
существуют положительные функции $C(x)$, $C_2(x)$ и $C_3(x)$
такие, что
\begin{multline*}
\Delta_x\leqslant \fr{C_2(x)}{(1+|x|)^2}\sum\limits_{i=1}^n\e
X_i^2\Iy(|X_i|\geqslant 1+|x|)+{}\\
{}+\fr{C_3(x)}{(1+|x|)^3}\sum\limits_{i=1}^n\e
|X_i|^3\Iy(|X_i|< 1+|x|)\,;
\end{multline*}

\vspace*{-12pt}

\noindent
\begin{multline*}
\Delta_x\leqslant C(x)\sum\limits_{i=1}^n\left[\fr{\e X_i^2\Iy(|X_i|\geqslant
1+|x|)}{(1+|x|)^2}+{}\right.\\
\left.{}+\fr{\e |X_i|^3\Iy(|X_i|<
1+|x|)}{(1+|x|)^3}\right]\,,
\end{multline*}
причем $C_2(x)\hm\leqslant 14{,}262$, $C_3(x)\hm\leqslant 41{,}229$, $C(x)\hm\leqslant 39{,}317$,
откуда вытекает, что в случае одинаково распределенных слагаемых
неравенство~\eqref{GeneralEq} справедливо с $C\hm\leqslant 39{,}317$.

В данной работе два последних приведенных нера\-вен\-ст\-ва рассматриваются
для необязательно одинаково распределенных слагаемых. Будет показано,
что в общем случае эти неравенства справедливы с константами
$C_2(x) \hm\leqslant 14{,}532$, $C_3(x)\hm\leqslant 49{,}468$
и $C(x)\hm\leqslant 47{,}648$.

\section{Основные результаты}

Без потери общности достаточно ограничиться
рассмотрением случая $x\geqslant 0$.

Для $x\geqslant 0$ и $n\in\N$ обозначим:
\begin{gather*}
Y_{i,x}=X_i \Iy (|X_i|<1+x)\,;\ \ \ \ S_x=\sum\limits_{i=1}^nY_{i,x}\,;
\\
\alpha_x=\sum\limits_{i=1}^n\e X_i^2\Iy (|X_i|\geqslant 1+x)\,;
\\
\beta_x=\sum\limits_{i=1}^n\e |X_i|^3\Iy (|X_i|<1+x)\,;
\\
\overline{Y}_{i,x}=\fr{Y_{i,x}-\e Y_{i,x}}{\sqrt{\D S_x}}\,;\ \ \
\ \overline{S}_x=\sum\limits_{i=1}^n\overline{Y}_{i,x}\,;
\\
\overline{\Delta}_x=\bigg |\p\bigg(\overline{S}_x\leqslant\fr{x-\e
S_x}{\sqrt{\D S_x}}\bigg)-\Phi\bigg(\fr{x-\e
S_x}{\sqrt{\D S_x}}\bigg)\bigg|.
\end{gather*}

\smallskip

\noindent
\textbf{Лемма~1.} %\label{delta_overline_lemma}
\textit{Если $\alpha_x\hm\leqslant A$ для некоторого $A\in(0,\,1/2)$, то
для любого $q\in [0,1]$ справедливы неравенства:}
\begin{multline} 
\label{delta_overline_lemma_tmp_1}
\overline{\Delta}_x\leqslant 0{,}5600\left[
 \fr{4 q(1+x)}{(1-2A)^{3/2}}\, 
 \fr{\alpha_x}{(1+x)^2}+{}\right.\\
\left. {}+\fr{(K+q-Kq)(1+x)^3}{(1-2A)^{3/2}}\,
 \fr{\beta_x}{(1+x)^3}\right]\,;
\end{multline}

\vspace*{-12pt}

\noindent
\begin{multline} 
\label{delta_overline_lemma_tmp_2}
\overline{\Delta}_x\leqslant 22{,}2460
\left[
 \fr{4  s(x,A)}{(1-2A)^{3/2} (1+x)^2}\,
 \fr{\alpha_x}{(1+x)^2}+{}\right.\\
\left. {}+\fr{s(x,A)}{(1-2A)^{3/2}}\, \fr{\beta_x}{(1+x)^3}
 \right]\,,
\end{multline}
\textit{где}
\begin{equation} 
\label{s(x,A)}
s(x,A)=\fr{(1+x)^3}{1+\left(x-{A}/(1+x)\right)^3}\,.
\end{equation}


\smallskip

\noindent
Д\,о\,к\,а\,з\,а\,т\,е\,л\,ь\,с\,т\,в\,о\,.\  В статье~\cite{KP2011_Neam} 
для одинаково распределенных слагаемых показаны неравенства:
\begin{multline} 
\label{delta_overline_lemma_tmp_3}
\sum\limits_{i=1}^n\e |\overline{Y}_{i,x}|^3 \leqslant {}\\
{}\leqslant
\fr{1}{(1-2A)^{3/2}}\left((K+q-Kq)\beta_x+\fr{4q\alpha_x}{1+x}\right)\,;
\end{multline}
\begin{equation} 
\label{delta_overline_lemma_tmp_5}
    x-\fr{A}{1+x}\leqslant
    \fr{x-\e S_x}{\sqrt{\D S_x}}\leqslant
    \fr{x+{A}/(1+x)}{\sqrt{1-2A}}\,.
\end{equation}
При этом отмечено (и в этом нетрудно убедиться), что эти
неравенства вместе с доказательствами остаются справедливыми и для
случая разнораспределенных слагаемых.

Поскольку $\overline{S}_x$ представляет собой сумму независимых
случайных величин $\overline{Y}_{i,x}$ с $\e
\overline{Y}_{i,x}=0$, причем $\D\overline{S}_{x}=1$, из
неравенства Бер\-ри--Эс\-се\-ена~\cite{Sh2010} с учетом~(\ref{delta_overline_lemma_tmp_3}) 
вытекает, что
\begin{multline*}
|\p(\overline{S}_x<z)-\Phi(z)|\leqslant
0{,}5600\sum\limits_{i=1}^n\e|\overline{Y}_{i,x}|^3\leqslant{}\\
{}\leqslant
\fr{0{,}5600}{(1-2A)^{3/2}}\left((K+q-Kq)\beta_x+\fr{4q\alpha_x}{1+x}\right)\,.
\end{multline*}
Последнее, очевидно, эквивалентно неравенству~(\ref{delta_overline_lemma_tmp_1}).

Применяя неравномерную оценку скорости сходимости в центральной
предельной теореме, приведенную в работе~\cite{GP2011}, с учетом
неравенств~(\ref{delta_overline_lemma_tmp_3}) 
и~(\ref{delta_overline_lemma_tmp_5}) получаем:
\begin{multline*}
\overline{\Delta}_x\leqslant\fr{22{,}2417}{1+\left((x-\e
S_x)/\sqrt{\D S_x}\right)^3}\sum\limits_{i=1}^n\e
|\overline{Y}_{i,x}|^3\leqslant{}\\
{}\leqslant\fr{22{,}2417} {1+\left(x-{A}/(1+x)\right)^3}
\sum\limits_{i=1}^n\e
|\overline{Y}_{i,x}|^3\leqslant{}\\
{}\leqslant\fr{22.2417s(x,A)}{(1-2A)^{3/2}(1+x)^3}\left(\beta_x+\fr{4\alpha_x}{1+x}\right)\,.
\end{multline*}
Последнее, очевидно, эквивалентно неравенству~(\ref{delta_overline_lemma_tmp_2}). Таким образом, лемма доказана.

\medskip


\noindent
\textbf{Лемма~2.} %\begin{lemma} \label{neamman_lemma_4}
\textit{Пусть $\alpha_x\leqslant A$ для
некоторого $A\in(0,\,1/2)$. Тогда}
\begin{multline*}
\Delta_x\equiv|\p(W_n<x)-\Phi(x)|\leqslant\overline\Delta_x+{}\\
{}+
\fr{\alpha_x}{(1+x)^2}\left(D(x,A)+1\right)\,,
\end{multline*}
\textit{где}
\begin{multline}
\label{D(x,A)}
D(x,A)=x(1+x)^2 e^{-x^2/2} \fr{B(A)e^A}{\sqrt{2\pi}}+ {}\\
{}+(1+x)
e^{-x^2/2} \fr{\left(1+AB(A)\right)e^A}{\sqrt{2\pi}}.
\end{multline}


\smallskip

\noindent
Д\,о\,к\,а\,з\,а\,т\,е\,л\,ь\,с\,т\,в\,о\,.\ См.~лемму~4 в~\cite{KP2011_Neam}.

\medskip

\noindent
\textbf{Терема~1.} \textit{%\begin{theorem} \label{main_theorem} 
Для любого $x\geqslant 0$ имеет место неравенство
$$
\Delta_x\leqslant
C_2(x)\fr{\alpha_x}{(1+x)^2}+C_3(x)\fr{\beta_x}{(1+x)^3},
$$
где $C_2(x)$ и $C_3(x)$~--- положительные ограниченные функции, для
которых справедлива каждая из следующих оценок:}

\pagebreak

\noindent\hspace*{2mm}$1^{\circ}$.
$
C_2(x)\hm\leqslant 2{,}0110(1+x)^2$;\ \,$C_3(x)\hm\leqslant 2{,}0110(1+x)^3.
$

\noindent
\hspace*{2mm}$2^{\circ}$. \textit{Если $\alpha_x\leqslant A$ для некоторого $A\in(0,\,1/2)$, то для любого $q\in [0,1]$}
$$
\begin{array}{ll}
(\mathrm{a})\quad & C_2(x)\leqslant {\displaystyle\fr{4\cdot 0{,}5600 q
(1+x)}{(1-2A)^{3/2}}+D(x,A)+1}\,;\\
&C_3(x)\leqslant
\fr{0{,}5600 (K+q-Kq)(1+x)^3}{(1-2A)^{3/2}}\,;\\
(\mathrm{б})\quad & C_2(x)\leqslant {\displaystyle\fr{4\cdot 22{,}2417
s(x,A)}{(1-2A)^{3/2} (1+x)^2}+D(x,A)+1}\,;\\ 
&C_3(x)\leqslant
\fr{22{,}2417 s(x,A)}{(1-2A)^{3/2}}\,, \end{array}
$$
\textit{где $s(x,A)$ и $D(x,A)$ определены соответственно в}~\eqref{s(x,A)}
\textit{и}~\eqref{D(x,A)}.

\noindent
\hspace*{2mm}$3^{\circ}$. \textit{Если $\alpha_x\geqslant A$ для некоторого $A\in(0,\,1/2)$, то}
$$
\begin{array}{ll}
(\mathrm{a})\quad & C_2(x)\leqslant\displaystyle \fr{0{,}541}{A} \left(1+x\right)^2\,;\
\ \ C_3(x)=0\,;\\
(\mathrm{б})\quad & C_2(x)\leqslant {\displaystyle
1+\fr{1}{x^4}+\fr{1}{x^4(1+x)^2}}+{}\\
&\hspace*{8mm}{}+\displaystyle{
\left[\fr{1}{x^4}+\fr{e^{-x^2/2}}{x\sqrt{2\pi}}\right]\fr{(1+x)^2}{A}};\\
 & C_3(x)\leqslant {\displaystyle\fr{(1+x)^4}{x^4}+\fr{(1+x)^2}{x^4}}\,.
\end{array}
$$


\smallskip

\noindent
Д\,о\,к\,а\,з\,а\,т\,е\,л\,ь\,с\,т\,в\,о\ пунктов $1^{\circ}$ и $3^{\circ}$ можно \mbox{найти} 
в~\cite{KP2011_Neam} (см.\ теорему~1).
Утверждение~$2^{\circ}$ следует непосредственно из лемм~1 и~2.

\smallskip

Теперь нетрудно понять, как следует использовать выводы теоремы~1 
в численных расчетах. Для каждого $x\hm>0$ и
$A\hm\in(0,\,1/2)$ следует определить наилучшую из оценок
$2^{\circ}\,(\mathrm{a})$ и $2^{\circ}\,(\mathrm{б})$ и наилучшую из оценок
$3^{\circ}\,(\mathrm{a})$ и $3^{\circ}\,(\mathrm{б})$. Далее при каждом~$x$
следует так выбирать параметр~$A$, чтобы оптимизировать худшую из
двух полученных оценок, которая и будет являться итоговой. Для
улучшения итоговой оценки в некоторых случаях следует затем
использовать оценку~$1^{\circ}$.

В задаче поиска наилучшей оценки имеется два критерия качества:
$C_2(x)$ и $C_3(x)$. Поэтому, как и в любой многокритериальной
задаче, необходимо уделить внимание вопросу сравнения двух оценок.
Сначала рассмотрим случай $q\hm=1$.

Нетрудно видеть, что при выполнении условия
\begin{equation} 
\label{est2_tmp_1}
\fr{22{,}2417}{1+\left(x-{A}/(1+x)\right)^3}>0{,}5600
\end{equation}
оценка $2^{\circ}\,(\mathrm{a})$ точнее оценки~$2^{\circ}\,(\mathrm{б})$ (для каждой
из функций $C_2(x)$ и~$C_3(x)$). В~противном случае следует
предпочесть оценку~$2^{\circ}\,(\mathrm{б})$. Неравенство~(\ref{est2_tmp_1}) можно преобразовать к виду:
$$
\Big(x-\fr{A}{1+x}\Big)^3<\fr{22{,}2417}{0{,}5600}-1=38{,}7173\,.
$$

Принимая во внимание область возможных значений~$A$, можно сделать
следующие выводы.

\medskip

\noindent
\textbf{Вывод~1.}
При $x\hm<3{,}3830$ следует использовать оценку $2^{\circ}\,(\mathrm{a})$. При
$x\hm>3{,}4943$ следует использовать оценку $2^{\circ}\,(\mathrm{б})$. 
В~остальных случаях следует рас\-смат\-ри\-вать обе оценки в зависимости
от значения~$A$.

\smallskip

Если выполнено условие
\begin{multline} 
\label{est2_tmp_2}
\fr{0{,}541}{A} \left(1+x\right)^2\leqslant 1+\fr{1}{x^4}+\fr{1}{x^4(1+x)^2}+{}\\
{}+
\left[\fr{1}{x^4}+\fr{e^{-x^2/2}}{x\sqrt{2\pi}}\right]\fr{(1+x)^2}{A}\,,
\end{multline}
то оценка $3^{\circ}\,(\mathrm{a})$, безусловно, точнее, чем~$3^{\circ}\,(\mathrm{б})$. 
Как показывают численные расчеты, это происходит
при $x\hm<1{,}2593$ при любом допустимом~$A$.

Неравенство, противоположное~(\ref{est2_tmp_2}), выполняется опять
же для всех допустимых значений~$A$ при $x\hm>1{,}3512$. В~таком случае
следует предпочесть оценку $3^{\circ}\,(\mathrm{б})$, несмотря на то что
оценка функции $C_3(x)$ становится грубее:
$$
\sup\limits_{x>1{,}2593}\fr{(1+x)^4}{x^4}+\fr{(1+x)^2}{x^4}\leqslant 12{,}39\,,
$$
а эта величина меньше, чем значение функции $C_3(x)$ при
применении оценок~$2^\circ$.

\medskip

\noindent
\textbf{Вывод 2.}
При $x\hm<1{,}2593$ следует использовать оценку $3^{\circ}\,(\mathrm{a})$. При
$x\hm>1{,}3512$ следует использовать оценку $3^{\circ}\,(\mathrm{б})$. 
В~остальных случаях следует рассматривать обе оценки в зависимости
от значения~$A$, но сравнение производить по значению функции~$C_2(x)$.

\smallskip

Вычисления в пакете MatLab позволяют сформулировать следующий результат.

\medskip

\noindent
\textbf{Теорема 2.} \textit{Для любого $x\geqslant 0$ имеет место неравенство
$$
\Delta_x\leqslant
C_2(x)\fr{\alpha_x}{(1+x)^2}+C_3(x)\fr{\beta_x}{(1+x)^3}\,,
$$
где $C_2(x)$ и $C_3(x)$~--- положительные ограниченные функции, для
которых справедливы оценки, приведенные в табл.~$1$.
}

\smallskip

Графики полученных оценок изображены на рис.~1.



\medskip

\noindent
\textbf{Следствие 1.} \textit{Для любого $x\geqslant0$ имеет место неравенство}
$$
\Delta_x\leqslant
14{,}532\fr{\alpha_x}{(1+x)^2}+49{,}468\fr{\beta_x}{(1+x)^3}\,.
$$

\end{multicols}

\begin{table}\small 
\begin{center}
\Caption{Верхние оценки функций $C_2(x)$ и $C_3(x)$}
\vspace*{2ex}

\begin{tabular}{||c|c|c||c|c|c||c|l|l||}
\hline
$x$ & $C_2\leqslant$ & $C_3\leqslant$ & $x$ & $C_2\leqslant$ & $C_3\leqslant$ & $x$ & 
\multicolumn{1}{c|}{$C_2\leqslant$} & \multicolumn{1}{c||}{$C_3\leqslant$}\\
\hline
$       0   $ & \hphantom{9}5{,}378   &   \hphantom{99}1{,}1308  & $ 3{,}3  \div   3{,}4 $ & 11{,}378\hphantom{99}  &   49{,}467  & $ 6{,}7  \div   6{,}8 $ & 3{,}2903  &   34{,}836  \\
$   0{,}0    \div   0{,}1 $ & \hphantom{99}5{,}9992  &   \hphantom{99}1{,}5163  & $ 3{,}4  \div   3{,}5 $ & 11{,}28\hphantom{999}   &   49{,}342  & $ 6{,}8  \div   6{,}9 $ & 3{,}2205  &   34{,}646  \\
$   0{,}1  \div   0{,}2 $ & \hphantom{99}6{,}6634  &   \hphantom{99}1{,}9843  & $ 3{,}5  \div   3{,}6 $ & 10{,}064\hphantom{99}  &   47{,}678  & $ 6{,}9  \div   7{,}0   $ & 3{,}1539  &   34{,}462  \\
$   0{,}2  \div   0{,}3 $ & \hphantom{99}7{,}3531  &   \hphantom{99}2{,}5522  & $ 3{,}6  \div   3{,}7 $ & 9{,}5333  &   46{,}915  & $ 7{,}0    \div   7{,}1 $ & 3{,}0902  &   34{,}284  \\
$   0{,}3  \div   0{,}4 $ & \hphantom{9}8{,}092   &   \hphantom{99}3{,}2345  & $ 3{,}7  \div   3{,}8 $ & 9{,}024\hphantom{9}   &   46{,}045  & $ 7{,}1  \div   7{,}2 $ & 3{,}0292  &   34{,}111  \\
$   0{,}4  \div   0{,}5 $ & \hphantom{99}8{,}8462  &   \hphantom{99}4{,}0295  & $ 3{,}8  \div   3{,}9 $ & 8{,}5783  &   45{,}367  & $ 7{,}2  \div   7{,}3 $ & 2{,}9709  &   33{,}944  \\
$   0{,}5  \div   0{,}6 $ & \hphantom{99}9{,}6318  &   \hphantom{99}4{,}9634  & $ 3{,}9  \div   4{,}0 $ & 8{,}1708  &   44{,}726  & $ 7{,}3  \div   7{,}4 $ & 2{,}915   &   33{,}781  \\
$   0{,}6  \div   0{,}7 $ & 10{,}422  &   \hphantom{9}6{,}026   & $ 4{,}0  \div   4{,}1 $ & 7{,}7954  &   44{,}12\hphantom{9}   & $ 7{,}4  \div   7{,}5 $ & 2{,}8615  &   33{,}623  \\
$   0{,}7  \div   0{,}8 $ & 11{,}233  &   \hphantom{99}7{,}2844  & $ 4{,}1  \div   4{,}2 $ & 7{,}4487  &   43{,}544  & $ 7{,}5  \div   7{,}6 $ & 2{,}8102  &   33{,}47   \\
$   0{,}8  \div   0{,}9 $ & 12{,}077  &   \hphantom{99}8{,}6961  & $ 4{,}2  \div   4{,}3 $ & 7{,}128\hphantom{9}   &   42{,}998  & $ 7{,}6  \div   7{,}7 $ & 2{,}7609  &   33{,}321  \\
$   0{,}9  \div   1{,}0   $ & 12{,}884  &   10{,}357  & $ 4{,}3  \div   4{,}4 $ & 6{,}8306  &   42{,}48\hphantom{9}   & $ 7{,}7  \div   7{,}8 $ & 2{,}7136  &   33{,}176  \\
$   1 {,}0   \div   1{,}1 $ & 13{,}741  &   12{,}257  & $ 4{,}4  \div   4{,}5 $ & 6{,}5543  &   41{,}986  & $ 7{,}8  \div   7{,}9 $ & 2{,}6682  &   33{,}035  \\
$   1{,}1  \div   1{,}2 $ & 14{,}531  &   14{,}304  & $ 4{,}5  \div   4{,}6 $ & 6{,}2971  &   41{,}517  & $ 7{,}9  \div   8{,}0   $ & 2{,}6246  &   32{,}898  \\
$   1{,}2  \div   1{,}3 $ & 14{,}531  &   14{,}601  & $ 4{,}6  \div   4{,}7 $ & 6{,}0572  &   41{,}069  & $ 8{,}0  \div   8{,}1 $ & 2{,}5827  &   32{,}765  \\
$   1{,}3  \div   1{,}4 $ & 14{,}479  &   15{,}062  & $ 4{,}7  \div   4{,}8 $ & 5{,}8332  &   40{,}641  & $ 8{,}1  \div   8{,}2 $ & 2{,}5423  &   32{,}635  \\
$   1{,}4  \div   1{,}5 $ & 13{,}61\hphantom{9}   &   15{,}698  & $ 4{,}8  \div   4{,}9 $ & 5{,}6236  &   40{,}233  & $ 8{,}2  \div   8{,}3 $ & 2{,}5035  &   32{,}509  \\
$   1{,}5  \div   1{,}6 $ & 12{,}947  &   16{,}497  & $ 4{,}9  \div   5{,}0  $ & 5{,}4273  &   39{,}843  & $ 8{,}3  \div   8{,}4 $ & 2{,}4661  &   32{,}386  \\
$   1{,}6  \div   1{,}7 $ & 12{,}416  &   17{,}422  & $ 5{,}0  \div   5{,}1 $ & 5{,}2329  &   39{,}345  & $ 8{,}4  \div   8{,}5 $ & 2{,}4301  &   32{,}266  \\
$   1{,}7  \div   1{,}8 $ & 11{,}99\hphantom{9}   &   18{,}51\hphantom{9}   & $ 5{,}1  \div   5{,}2 $ & 5{,}0572  &   38{,}989  & $ 8{,}5  \div   8{,}6 $ & 2{,}3954  &   32{,}149  \\
$   1{,}8  \div   1{,}9 $ & 11{,}648  &   19{,}69\hphantom{9}   & $ 5{,}2  \div   5{,}3 $ & 4{,}895\hphantom{9}   &   38{,}647  & $ 8{,}6  \div   8{,}7 $ & 2{,}3619  &   32{,}035  \\
$   1{,}9  \div   2{,}0   $ & 11{,}357  &   20{,}971  & $ 5{,}3  \div   5{,}4 $ & 4{,}7422  &   38{,}32\hphantom{9}   & $ 8{,}7  \div   8{,}8 $ & 2{,}3296  &   31{,}924  \\
$   2{,}0    \div   2{,}1 $ & 11{,}132  &   22{,}36\hphantom{9}   & $ 5{,}4  \div   5{,}5 $ & 4{,}5982  &   38{,}006  & $ 8{,}8  \div   8{,}9 $ & 2{,}2985  &   31{,}816  \\
$   2{,}1  \div   2{,}2 $ & 10{,}965  &   23{,}947  & $ 5{,}5  \div   5{,}6 $ & 4{,}4623  &   37{,}704  & $ 8{,}9  \div   9{,}0   $ & 2{,}2684  &   31{,}71   \\
$   2{,}2  \div   2{,}3 $ & 10{,}82\hphantom{9}   &   25{,}585  & $ 5{,}6  \div   5{,}7 $ & 4{,}3338  &   37{,}414  & $ 9{,}0    \div   9{,}1 $ & 2{,}2394  &   31{,}607  \\
$   2{,}3  \div   2{,}4 $ & 10{,}716  &   27{,}364  & $ 5{,}7  \div   5{,}8 $ & 4{,}2123  &   37{,}135  & $ 9{,}1  \div   9{,}2 $ & 2{,}2104  &   31{,}408  \\
$   2{,}4  \div   2{,}5 $ & 10{,}656  &   29{,}297  & $ 5{,}8  \div   5{,}9 $ & 4{,}2123  &   37{,}135  & $ 9{,}2  \div   9{,}3 $ & 2{,}182   &   31{,}31   \\
$   2{,}5  \div   2{,}6 $ & 10{,}642  &   31{,}399  & $ 5{,}9  \div   6{,}0   $ & 3{,}9883  &   36{,}607  & $ 9{,}3  \div   9{,}4 $ & 2{,}1546  &   31{,}214  \\
$   2{,}6  \div   2{,}7 $ & 10{,}692  &   33{,}583  & $ 6{,}0    \div   6{,}1 $ & 3{,}885\hphantom{9}   &   36{,}358  & $ 9{,}4  \div   9{,}5 $ & 2{,}1291  &   31{,}12   \\
$   2{,}7  \div   2{,}8 $ & 10{,}768  &   35{,}961  & $ 6{,}1  \div   6{,}2 $ & 3{,}7868  &   36{,}117  & $ 9{,}5  \div   9{,}6 $ & 2{,}1046  &   31{,}028  \\
$   2{,}8  \div   2{,}9 $ & 10{,}858  &   38{,}434  & $ 6{,}2  \div   6{,}3 $ & 3{,}6935  &   35{,}885  & $ 9{,}6  \div   9{,}7 $ & 2{,}0809  &   30{,}939  \\
$   2{,}9  \div   3{,}0   $ & 10{,}986  &   41{,}131  & $ 6{,}3  \div   6{,}4 $ & 3{,}6935  &   35{,}885  & $ 9{,}7  \div   9{,}8 $ & 2{,}058   &   30{,}851  \\
$   3{,}0  \div   3{,}1 $ & 11{,}115  &   43{,}846  & $ 6{,}4  \div   6{,}5 $ & 3{,}5205  &   35{,}444  & $ 9{,}8  \div   9{,}9 $ & 2{,}0358  &   30{,}766  \\
$   3{,}1  \div   3{,}2 $ & 11{,}26\hphantom{9}   &   46{,}862  & $ 6{,}5  \div   6{,}6 $ & 3{,}4401  &   35{,}234  & $ 9{,}9  \div   10  $ & 2{,}0143  &   30{,}682  \\
$   3{,}2  \div   3{,}3 $ & 11{,}378  &   49{,}467  & $ 6{,}6  \div   6{,}7 $ & 3{,}3634  &   35{,}032  & $      \geqslant   10  $ & 2{,}0143  &   30{,}682  \\
\hline
\end{tabular}
\end{center}
\vspace*{6pt}
\end{table}

\begin{multicols}{2}

\begin{center} %fig1
\vspace*{2pt}
\mbox{%
 \epsfxsize=63.453mm
 \epsfbox{pop-1.eps}
}
\end{center}
\begin{center}
\vspace*{6pt}
{{\figurename~1}\ \ \small{Графики функций $C(x)$~(\textit{1}), 
$C_2(x)$~(\textit{2}), $C_3(x)$~(\textit{3})}}
\end{center}
\vspace*{6pt}

%\smallskip
\addtocounter{figure}{1}


Как видно из формулировки теоремы~1, <<критическим>> параметром в
ней является значение $\alpha_x$, так что предыдущие результаты
получены с по\-мощью выбора параметра~$A$, оптимизирующего $C_2(x)$
при каждом~$x$. Если в качестве критерия оптимальности рассмотреть
$C(x)\hm=\max\left\{C_2(x),\, C_3(x)\right\}$, то параметр~$A$ следует
выбирать несколько иным способом. В~таком случае на промежутке
$x\hm<3{,}3830$ уже будет более целесообразным не фиксировать значение
$q\hm=1$, а проводить оптимизацию по параметру~$q$. Этот параметр
следует выбирать так, чтобы $C_2(x)\hm=C_3(x)$, если значение~$q$
лежит на отрезке $[0,\,1]$. Если указанное равенство $C_2(x)\hm=C_3(x)$
невозможно ни при каких допустимых значениях параметров, то надо,
как и ранее, полагать $q\hm=1$. Такой подход приводит с следующему
утверждению.

\smallskip

\noindent
\textbf{Теорема 3.} \textit{Для любого $x\geqslant 0$ имеет место неравенство}

\end{multicols}

\begin{table}\small % \label{res_table_2}
\begin{center}
\Caption{Верхние оценки функции $C(x)$}
\vspace*{2ex}

\begin{tabular}{||c|c||c|l||c|l||}
\hline
$x$ & $C\leqslant$ & $x$ & \multicolumn{1}{c||}{$C\leqslant$} & $x$ & 
\multicolumn{1}{c||}{$C\leqslant$}\\
\hline
$       0   $ & \hphantom{99}2{,}6454  & $ 3{,}3 \div    3{,}4 $ & 47{,}647  & $ 6{,}7 \div    6{,}8 $ & 33{,}552  \\
$   0{,}0 \div    0{,}1 $ & \hphantom{9}3{,}251   & $ 3{,}4 \div    3{,}5 $ & 47{,}495  & $ 6{,}8 \div    6{,}9 $ & 33{,}368  \\
$   0{,}1 \div    0{,}2 $ & \hphantom{9}3{,}918   & $ 3{,}5 \div    3{,}6 $ & 45{,}848  & $ 6{,}9 \div    7{,}0$ & 33{,}188  \\
$   0{,}2 \div    0{,}3 $ & \hphantom{99}4{,}6645  & $ 3{,}6 \div    3{,}7 $ & 45{,}097  & $ 7{,}0 \div    7{,}1 $ & 33{,}015  \\
$   0{,}3 \div    0{,}4 $ & \hphantom{9}5{,}482   & $ 3{,}7 \div    3{,}8 $ & 44{,}388  & $ 7{,}1 \div    7{,}2 $ & 32{,}847  \\
$   0{,}4 \div    0{,}5 $ & \hphantom{99}6{,}3743  & $ 3{,}8 \div    3{,}9 $ & 43{,}72   & $ 7{,}2 \div    7{,}3 $ & 32{,}684  \\
$   0{,}5 \div    0{,}6 $ & \hphantom{99}7{,}3317  & $ 3{,}9 \div    4{,}0 $ & 43{,}091  & $ 7{,}3 \div    7{,}4 $ & 32{,}526  \\
$   0{,}6 \div    0{,}7 $ & \hphantom{99}8{,}3775  & $ 4{,}0 \div    4{,}1 $ & 42{,}495  & $ 7{,}4 \div    7{,}5 $ & 32{,}373  \\
$   0{,}7 \div    0{,}8 $ & \hphantom{99}9{,}4707  & $ 4{,}1 \div    4{,}2 $ & 41{,}93   & $ 7{,}5 \div    7{,}6 $ & 32{,}224  \\
$   0{,}8 \div    0{,}9 $ & 10{,}649  & $ 4{,}2 \div    4{,}3 $ & 41{,}396  & $ 7{,}6 \div    7{,}7 $ & 32{,}08   \\
$   0{,}9 \div    1{,}0 $ & 11{,}863  & $ 4{,}3 \div    4{,}4 $ & 40{,}887  & $ 7{,}7 \div    7{,}8 $ & 31{,}939  \\
$   1{,}0 \div    1{,}1 $ & 13{,}154  & $ 4{,}4 \div    4{,}5 $ & 40{,}405  & $ 7{,}8 \div    7{,}9 $ & 31{,}803  \\
$   1{,}1 \div    1{,}2 $ & 14{,}449  & $ 4{,}5 \div    4{,}6 $ & 39{,}946  & $ 7{,}9 \div    8{,}0$ & 31{,}669  \\
$   1{,}2 \div    1{,}3 $ & 14{,}449  & $ 4{,}6 \div    4{,}7 $ & 39{,}509  & $ 8{,}0 \div    8{,}1 $ & 31{,}539  \\
$   1{,}3 \div    1{,}4 $ & 14{,}418  & $ 4{,}7 \div    4{,}8 $ & 39{,}091  & $ 8{,}1 \div    8{,}2 $ & 31{,}414  \\
$   1{,}4 \div    1{,}5 $ & 14{,}449  & $ 4{,}8 \div    4{,}9 $ & 38{,}693  & $ 8{,}2 \div    8{,}3 $ & 31{,}292  \\
$   1{,}5 \div    1{,}6 $ & 14{,}962  & $ 4{,}9 \div    5{,}0 $ & 38{,}313  & $ 8{,}3 \div    8{,}4 $ & 31{,}172  \\
$   1{,}6 \div    1{,}7 $ & 15{,}671  & $ 5{,}0 \div    5{,}1 $ & 37{,}948  & $ 8{,}4 \div    8{,}5 $ & 31{,}056  \\
$   1{,}7 \div    1{,}8 $ & 16{,}56\hphantom{9}   & $ 5{,}1 \div    5{,}2 $ & 37{,}6    & $ 8{,}5 \div    8{,}6 $ & 30{,}943  \\
$   1{,}8 \div    1{,}9 $ & 17{,}618  & $ 5{,}2 \div    5{,}3 $ & 37{,}267  & $ 8{,}6 \div    8{,}7 $ & 30{,}832  \\
$   1{,}9 \div    2{,}0 $ & 18{,}836  & $ 5{,}3 \div    5{,}4 $ & 36{,}948  & $ 8{,}7 \div    8{,}8 $ & 30{,}724  \\
$   2{,}0 \div    2{,}1 $ & 20{,}211  & $ 5{,}4 \div    5{,}5 $ & 36{,}64   & $ 8{,}8 \div    8{,}9 $ & 30{,}619  \\
$   2{,}1 \div    2{,}2 $ & 21{,}736  & $ 5{,}5 \div    5{,}6 $ & 36{,}346  & $ 8{,}9 \div    9{,}0$ & 30{,}516  \\
$   2{,}2 \div    2{,}3 $ & 23{,}41\hphantom{9}   & $ 5{,}6 \div    5{,}7 $ & 36{,}063  & $ 9{,0} \div    9{,}1 $ & 30{,}416  \\
$   2{,}3 \div    2{,}4 $ & 25{,}23\hphantom{9}   & $ 5{,}7 \div    5{,}8 $ & 35{,}791  & $ 9{,}1 \div    9{,}2 $ & 30{,}318  \\
$   2{,}4 \div    2{,}5 $ & 27{,}196  & $ 5{,}8 \div    5{,}9 $ & 35{,}791  & $ 9{,}2 \div    9{,}3 $ & 30{,}223  \\
$   2{,}5 \div    2{,}6 $ & 29{,}303  & $ 5{,}9 \div    6{,}0 $ & 35{,}276  & $ 9{,}3 \div    9{,}4 $ & 30{,}13   \\
$   2{,}6 \div    2{,}7 $ & 31{,}556  & $ 6{,}0 \div    6{,}1 $ & 35{,}033  & $ 9{,}4 \div    9{,}5 $ & 30{,}038  \\
$   2{,}7 \div    2{,}8 $ & 33{,}953  & $ 6{,}1 \div    6{,}2 $ & 34{,}798  & $ 9{,}5 \div    9{,}6 $ & 29{,}949  \\
$   2{,}8 \div    2{,}9 $ & 36{,}494  & $ 6{,}2 \div    6{,}3 $ & 34{,}572  & $ 9{,}6 \div    9{,}7 $ & 29{,}862  \\
$   2{,}9 \div    3{,}0 $ & 39{,}182  & $ 6{,}3 \div    6{,}4 $ & 34{,}572  & $ 9{,}7 \div    9{,}8 $ & 29{,}777  \\
$   3{,}0 \div    3{,}1 $ & 42{,}017  & $ 6{,}4 \div    6{,}5 $ & 34{,}144  & $ 9{,}8 \div    9{,}9 $ & 29{,}694  \\
$   3{,}1 \div    3{,}2 $ & 45{,}001  & $ 6{,}5 \div    6{,}6 $ & 33{,}94   & $ 9{,}9 \div    10  $ & 29{,}613  \\
$   3{,}2 \div    3{,}3 $ & 47{,}647  & $ 6{,}6 \div    6{,}7 $ & 33{,}742  & $     \geqslant   10  $ & 29{,}613  \\

\hline
\end{tabular}
\end{center}
\end{table}

\begin{multicols}{2}

\noindent
$$
\Delta_x\leqslant
C(x)\left[\fr{\alpha_x}{(1+x)^2}+\fr{\beta_x}{(1+x)^3}\right]\,,
$$

\noindent
\textit{где $C(x)$~--- положительная ограниченная функция, для которой
справедливы оценки, приведенные в табл.}~2.


\smallskip

График полученной оценки приведен на рис.~1.

\smallskip

\noindent
\textbf{Следствие 2.} \textit{Если слагаемые $X_1,\ldots,X_n$ одинаково распределены, то
неравенство~$(1)$ справедливо с $C\hm\leqslant 47{,}648$.
}


{\small\frenchspacing
{%\baselineskip=10.8pt
\addcontentsline{toc}{section}{Литература}
\begin{thebibliography}{9}


\bibitem{ChSh2001} %1
\Au{Chen L.\,H.\,Y., Shao Q.\,M.} A non-uniform Berry--Esseen bound
via Stein's method~// Prob. Theory Related Fields, 2001.
Vol.~120. P.~236--254.

\bibitem{TN2007} %2
\Au{Thongtha P., Neammanee K.} Refinement of the constants in the
non-uniform version of the Berry--Esseen theorem~// Thai J.~Math., 2007. Vol.~5. P.~1--13.

\bibitem{NT2007} %3
\Au{Neammanee K., Thongtha~P.} Improvement of the non-uniform
version of the Berry--Esseen inequality via Paditz--Shiganov
theorems~// J.~Inequalities Pure Appl.
Math., 2007. Vol.~8. Iss.~4. Art.~92.

\bibitem{KP2011_Neam} %4
\Au{Королев В.\,Ю., Попов С.\,В.} Уточнение оценок ско\-рости
сходимости в центральной предельной теореме при отсутствии моментов
порядков, больших второго~// Статистические методы оценивания и
проверки гипотез.~--- Пермь: ПермГУ, 2011. С.~32--45.

\bibitem{Sh2010} %5
\Au{Шевцова И.\,Г.} Уточнение оценок скорости схо\-ди\-мости в теореме
Ляпунова~// Докл. РАН, 2010. Т.~435. Вып.~1. С.~26--28.

\label{end\stat}
\bibitem{GP2011} %6
\Au{Григорьева М.\,Е., Попов С.\,В.} О неравномерных оценках
скорости сходимости в центральной предельной теореме~// Докл.
РАН, 2012 (в печати).

 \end{thebibliography}
}
}


\end{multicols}       