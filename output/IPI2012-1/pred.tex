\def\stat{pred}
{%\hrule\par
%\vskip 7pt % 7pt
\centering\Large \bf%\baselineskip=3.2ex
Т\,Е\,М\,А\,Т\,И\,Ч\,Е\,С\,К\,И\,Й\ \,Р\,А\,З\,Д\,Е\,Л \vskip 17pt
    \hrule
    \par
\vskip 21pt plus 6pt minus 3pt }      

\def\tit{\centerline{Обработка изображений и распознавание образов}}
\def\titkol{\ %Тематический раздел
}

\def\autkol{\ }
\def\aut{\ }

\titel{\tit}{\aut}{\autkol}{\titkol}

\def\leftkol{\ } % ENGLISH ABSTRACTS}

\def\rightkol{\ } %ENGLISH ABSTRACTS}

                 
      Настоящий раздел носит тематический характер и посвящен различным 
актуальным прикладным проблемам в области обработки изображений и распознавания 
образов. Материалы раздела содержат полные версии докладов, представленных на 
Международной конференции по компьютерной графике и зрению и Всероссийской 
конференции <<Математические методы распознавания образов>>. 
      
      Статья К.\,В.~Рудакова и И.\,Ю.~Торшина <<Анализ информативности мотивов на 
основе критерия разрешимости в задаче распознавания вторичной структуры белка>> 
посвящена актуальной проблеме развития методов и алгоритмов распознавания 
вторичной структуры белка по данным о первичной структуре. В~статье И.\,Н.~Белых, 
А.\,И.~Капустина, А.\,В.~Козлова, А.\,И.~Лохановой, Ю.\,Н.~Матвеева, 
Т.\,С.~Пеховского, К.\,К.~Симончика и А.\,К.~Шулипы <<Система идентификации 
дикторов по голосу для конкурса \textit{NIST SRE 2010}>> изложены результаты испытаний 
биометрической технологии текстонезависимой идентификации диктора <<Центра 
речевых технологий>> на международных испытаниях NIST SRE, где технология 
получила очень высокие оценки по точности распознавания. В~статье В.\,Ю.~Гудкова и 
М.\,В.~Бокова <<Быстрая обработка изображений отпечатков пальцев>> затронут вопрос 
ускорения обработки дактилоскопических изображений, что является критическим 
фактором для встраиваемых систем биометрической идентификации. В~статье 
Ю.\,В.~Визильтера, В.\,С.~Горбацевича, С.\,Л.~Каратеева, Н.\,А.~Костромова <<Обучение 
алгоритмов выделения кожи на цветных изображениях лиц>> представлено оригинальное 
решение задачи цветовой сегментации кожи человека на изображениях. Статья 
А.\,В.~Куракина <<Распознавание жестов ладони в реальном времени на основе плоских и 
пространственных скелетных моделей>> содержит описание экспериментального 
комплекса распознавания жестов ладони, реализующего разработанный автором метод 
определения координат кончиков пальцев на бинарном изображении ладони посредством 
анализа его скелетного представления. Статья Д.\,В.~Мурашова <<Комбинированный 
подход к локализации различий многомодальных изображений>> посвящена вопросу 
одновременного исследования муль\-ти\-спектральных изображений, получаемых при 
реставрации живописных произведений. Предложенные методы и алгоритмы позволяют 
существенно автоматизировать некоторые этапы реставрационных работ. В~статье 
О.\,С.~Ушмаева и В.\,В.~Кузнецова <<Алгоритмы защищенной биометрической 
верификации на основе бинарного представления топологии отпечатков пальцев>> 
представлены результаты исследования по совмещению криптографических конструкций 
и методов биометрической идентификации по отпечаткам пальцев.


\def\leftkol{\ } % ENGLISH ABSTRACTS}

\def\rightkol{\ } %ENGLISH ABSTRACTS}
      

 \label{end\stat}      
      