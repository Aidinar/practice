\def\stat{ushmaev}

\def\tit{АЛГОРИТМЫ ЗАЩИЩЕННОЙ БИОМЕТРИЧЕСКОЙ ВЕРИФИКАЦИИ НА~ОСНОВЕ 
БИНАРНОГО ПРЕДСТАВЛЕНИЯ~ТОПОЛОГИИ~ОТПЕЧАТКОВ~ПАЛЬЦЕВ$^*$}

\def\titkol{Алгоритмы защищенной биометрической верификации на~основе 
бинарного представления топологии отпечатков} % пальцев}

\def\autkol{О.\,С.~Ушмаев, В.\,В.~Кузнецов}
\def\aut{О.\,С.~Ушмаев$^1$, В.\,В.~Кузнецов$^2$}

\titel{\tit}{\aut}{\autkol}{\titkol}

{\renewcommand{\thefootnote}{\fnsymbol{footnote}}\footnotetext[1]
{Работы выполнена при поддержке программы <<Инфотекс Академия 2011>> и гранта Президента РФ 
МД.72-2011.9.}}


\renewcommand{\thefootnote}{\arabic{footnote}}
\footnotetext[1]{Институт проблем информатики Российской академии наук, oushmaev@ipiran.ru}
\footnotetext[2]{Институт проблем информатики Российской академии наук, k.v.net@rambler.ru}  

\Abst{Рассмотрена задача, относящаяся к проблеме совмещения биометрической верификации по 
отпечаткам пальцев и криптографических конструкций. Основой для такого совмещения является алгоритм 
извлечения достаточно длинной устойчивой бинарной строки из изображения отпечатка пальца. 
Предложен алгоритм извлечения бинарной строки из отпечатков пальцев на основе топологической 
связанности контрольных точек отпечатка. Каждая контрольная точка характеризуется ближайшими 
папиллярными линиями. При прослеживании папиллярной линии встречаются <<события>>: контрольные 
точки или их проекции. Эти события индексируются, что позволяет описывать окрестность любой точки 
бинарным вектором длиной 50--100~бит. Для извлечения более длинных векторов предложено два 
метода. Первый метод не требует взаимного выравнивания отпечатков, в то время как второй использует 
для выравнивания открытый хелпер. Таким образом, удается добиться построения векторов длиной 
384--756~бит. Полученные векторы содержат примерно 20\% ошибок, для исправления которых используется 
двухслойное кодирование (Боу\-за--Чоуд\-ху\-ри--Хок\-вин\-ге\-ма (БЧХ) и 
репликация). Эксперименты с использованием базы FVC2002 DB1 
показали, что возможно построение вектора с энтропией 20--40~бит с 90-про\-цент\-ной
вероятностью успешной  идентификации.}

\KW{защищенная биометрическая верификация; нечеткий экстрактор; отпечатки пальцев}

\vskip 14pt plus 9pt minus 6pt

      \thispagestyle{headings}

      \begin{multicols}{2}
      
            \label{st\stat}

\section{Введение}

      Одной из важных теоретических проблем защи\-ты информации является 
совмещение криптографических методов и биометрической верификации личности. При 
этом такое совмещение \mbox{имеет} два основных приложения: инкорпорирование механизмов 
верификации\- в криптографические конструк\-ции (биометрическая криптография) или 
защита процесса верификации (защищенная верификация). 

Первая работа по этой 
тематике была опубликована в 1991~г.~[1], и с тех пор данное направление активно 
развивается. На сегодняшний день в рамках этой задачи можно определить несколько 
универсальных подходов, которые могут с той или иной степенью успешности 
применяться ко многим биометрическим данным. 

Первый подход, называемый 
биохешированием~[2], заключается в том, что для защиты биометрических шаб\-ло\-нов 
используются специальные необратимые функции, переводящие шаблоны одного 
пользователя в идентичные образы (биохеши). При этом на общей совокупности 
пользователей функции близки к случайному шуму. Верификация личности производится 
путем сравнения био-\linebreak хешей. 

Второй распространенный подход~--- протоколы с нулевым 
разглашением (\textit{zero-knowledge protocols})~[3]. В~них используются свойства 
гомоморфных алгоритмов шифрования для кодирования\linebreak биометрических шаб\-ло\-нов и их 
последующего сравнения в зашифрованном виде. Эти протоколы применимы только для 
биометрических характеристик, представимых в виде бинарных векторов, которые можно 
сравнивать в метрике Хемминга, как, например, био\-мет\-рия лица или радужной оболочки 
глаза. 

Третий, наиболее распространенный, подход~--- <<нечеткие экстракторы>>~[4]. Он 
является более слабой версией первых двух подходов. При реализации нечетких 
экстракторов защита био\-мет\-ри\-че\-ских данных пользователя достигается за счет того, что 
био\-мет\-ри\-че\-ский шаб\-лон шифруется на этапе регистрации и все последующие операции с 
ним проводятся без промежуточной расшифровки. Отличительной особенностью метода 
является использование дополнительной открытой информации, на\-зы\-ва\-емой открытым 
хелпером (\textit{public helper}). Нечеткие экстракторы обычно строятся для 
био-\linebreak\vspace*{-12pt}
\begin{center} %fig1
\vspace*{1pt}
\mbox{%
 \epsfxsize=77.483mm
 \epsfbox{ush-1.eps}
}
%\end{center}
%\begin{center}

\vspace*{9pt}
{{\figurename~1}\ \ \small{Контрольные точки папиллярного рисунка}}
\end{center}

\vspace*{6pt}

%\smallskip
\addtocounter{figure}{1}


\noindent
мет\-ри\-че\-ских характеристик, естественным образом представимых в виде бинарной 
строки. К~таковым относятся радужная оболочка глаза, голос, лицо. Для био\-мет\-рии 
отпечатков пальцев этот подход используется редко. 
      
      Несмотря на существование универсальных методов защищенной биометрической 
верификации~[3, 4], в большинстве из опубликованных на данный момент эффективных 
решений используется специфика конкретных биометрических характеристик. 
В~настоящей статье будет рассматриваться биометрия отпечатков пальцев. 

Отпечатки 
пальцев чаще других биометрических характеристик используются для точной 
идентификации или верификации личности. Однако общепринятое представление 
шаблона отпечатка в виде неупорядоченного набора контрольных точек (разветвлений и 
окончаний папиллярных линий, рис.~1) является неудобным для задач защищенной 
верификации. Сдвиги и повороты изображения, исчезновение и мутации (из окончания в 
разветвление и обратно) контрольных точек~--- все это делает реализацию защищенной 
верификации с использованием универсальных методов крайне неэффективной.
      

      
      В литературе представлено несколько специализированных методов защищенной 
верификации по отпечаткам пальцев. Наиболее известен метод, называемый <<нечетким 
хранилищем>> (\textit{fuzzy vault})~[5]. Координаты контрольных точек рассматриваются 
как точки, лежащие на графике полинома. Его коэффициенты формируют секретный 
ключ, которым <<закрывается>> хранилище. В~хранилище помещается бинарный ключ, 
закодированный с использованием помехоустойчивого кодирования. На этапе 
верификации закрытый в хранилище ключ может быть восстановлен однозначно, если 
предъявленный набор минюций существенным образом перекрывается с исходным. 
В~част\-ности, ключ длиной 69~бит успешно восстанавливался с вероятностью ошибки 
первого рода в 30\%. Также следует отметить несколько работ~[6--12], описывающих 
построение нечетких экстракторов для отпечатков пальцев. 
      
      В настоящей работе рассматривается метод защищенной биометрической 
верификации, основанный на детальном исследовании топологии потока папиллярных 
линий. Основная идея, лежащая в основе метода, состоит в кодировании топологических 
отношений между минюциями (как, например, соседство двух минюций на одной линии) 
в виде бинарного характеристического вектора. Затем этот характеристический вектор 
используется для реализации нечеткого экстрактора.

\section{Схема защищенной идентификации}
      
      В настоящей работе предлагается использовать алгоритм защищенной 
биометрической верификации, основанный на статье~[13]. Алгоритм верификации можно 
разделить на две функции: регистрацию пользовательской биометрии и верификацию.
      
      Допустим, что имеется возможность извлекать из образцов отпечатков пальцев 
бинарные строки (называемые характеристическими векторами) достаточно большой 
длины (несколько сотен бит). Тогда на этапе регистрации пользователю выдается 
случайный ключ~$k$, который является случайным бинарным вектором. Затем этот 
вектор с использованием помехоустойчивого кодирования преобразуется в строку 
$K\hm=\mathrm{Encode}\left(k\right)$ такой же длины, как и характеристический вектор. Эта строка 
посредством исключающего ИЛИ складывается с характеристическим вектором 
$DH_{\mathrm{ref}}$, извлеченным из предъявленного при регистрации отпечатка \mbox{пальца}:
      \begin{equation*}
      T=DH_{\mathrm{ref}}\oplus K\,.
%      \label{e1usm}
      \end{equation*}
      
      Бинарный вектор~$T$, а также $H(k)$~--- хеш-функ\-ция ключа~$k$~--- 
сохраняются в качестве открытого шаблона (хелпера). Сразу следует отметить, что 
биометрические данные являются достаточно шумным объектом. Поэтому 
характеристические векторы отличаются в различных предъявлениях в достаточно 
большом числе битов.
      
      На этапе верификации восстанавливается ключ~$k$ с использованием открытого 
хелпера и предъявленного отпечатка пальца. В~соответствии с используемым алгоритмом 
строится со\-от\-вет\-ст\-ву\-ющий характеристический вектор $DH_{\mathrm{sam}}$ для предъявленного 
отпечатка пальца, который затем исключающим ИЛИ складывается со строкой~$T$ из 
открытого хелпера:
      \begin{equation}
      K^\prime= GH_{\mathrm{sam}}\oplus T=K\oplus \mathrm{err}\,,
      \label{e2-usm}
      \end{equation}
где еrr~--- ошибка, т.\,е.\ разница между $DH_{\mathrm{ref}}$ и~$DH_{\mathrm{sam}}$. Она будет 
относительно мала при предъявлении своего и велика при предъявлении чужого отпечатка 
пальца.
      
      Теперь можно попытаться восстановить ключевую информацию посредством 
декодирования:
      \begin{equation*}
      k^\prime= \mathrm{Decode}\left( K^\prime\right)\,.
%      \label{e3-usm}
      \end{equation*}
      
      Если хеш $H(k^\prime)$ декодированного ключа совпадает с вычисленным при 
регистрации хешем~$H(k)$, то делается вывод, что ключ восстановлен корректно 
(соответственно, личность верифицирована), иначе делается заключение о 
некорректности ключа~$k^\prime$ (соответственно, личность не верифицирована). 
      
      Для реализации такой схемы требуется, во-пер\-вых, синтезировать алгоритм 
извлечения достаточно длинного бинарного вектора из изображения отпечатка пальца 
таким образом, чтобы расстояние Хемминга для разных предъявлений одного отпечатка 
было статистически меньше расстояния Хемминга для разных отпечатков пальцев. 
      Во-вто\-рых, для конкретной реализации алгоритма извлечения бинарного вектора 
необходимо подобрать помехоустойчивые коды.
     
      Далее статья организована следующим образом. В~разд.~ 3 описывается топология 
отпечатка и метод ее кодирования. Раздел~4 содержит алгоритмы преобразования 
отпечатка пальца в бинарный вектор без использования дополнительной информации. 
В~разд.~5 предложен алгоритм извлечения бинарного вектора с опубликованием 
дополнительной информации. Эксплуатационные характеристики предложенных 
алгоритмов представлены в разд.~6.

\section{Построение топологических дескрипторов}

\subsection{Топологическая модель отпечатка пальца}
      
      Изображение отпечатка пальца представляет собой поток папиллярных линий. 
Ветвления и окончания папиллярных линий называются контрольными точками, или 
минюциями. Набор \mbox{минюций} с атрибутами (разветвление или окончание, на\-прав\-ле\-ние 
потока папиллярных линий) образует стандартизированный шаблон отпечатка пальца. 
Набор может иметь разные представления, такие как,
 например, множество или граф. 
В~данной работе рассматривается топологическое представление. Базовая идея, лежащая 
в основе использованной топологии, состоит в следующем. Описание каж\-дой минюции 
можно дополнить топологическими дескрипторами~--- бинарными векторами, 
опи\-сы\-ва\-ющи\-ми связь с другими минюциями через соседние папиллярные линии. 
      

      
      Формально это можно записать в виде отображения~$t$, переводящего 
изображение~$I$ в шаблон отпечатка~$b$:

\noindent
      \begin{align*}
      &t:\ I\rightarrow b\,; %\label{e4usm}
      \\
      &b=\left\{ x_i,y_i,D_i\right\}^m_{i=1}\,, %\label{e5usm}
      \end{align*}
где $m$~--- число минюций; $x_i$, $y_i$~--- их координаты; $D_i\in [0,\,1]^{n_d}$~--- 
топологический дескриптор длиной~$n_d$. Для того чтобы построить топологический 
дескриптор, введем понятия топологической связи и топологического события. Выберем 
произвольную контрольную точку. Из нее проведем отрезок перпендикулярно к потоку 
папиллярных линий. Точка пересечения этого отрезка с каждой из папиллярных линий 
делит их на два луча (рис.~2). Эти лучи называются \textit{топологическими связями}, а 
количество лучей, которые пересекает отрезок с каждой стороны,~--- \textit{глубиной 
прослеживания}.

\begin{center} %fig2
\vspace*{9pt}
\mbox{%
 \epsfxsize=76.744mm
 \epsfbox{ush-2.eps}
}
%\end{center}
%\begin{center}

\vspace*{9pt}
{{\figurename~2}\ \ \small{Построение топологического дескриптора}}
\end{center}

\vspace*{6pt}

%\smallskip
\addtocounter{figure}{1}

      
      Рассмотрим каждую из связей. Найдем на связи ближайшую минюцию или 
проекцию минюции на соседнюю папиллярную линию; назовем их 
<<\textit{топологическим событием}>>. В~патенте М.~Спэрроу~[14] различается восемь 
топологических событий, однако в работах В.\,Ю.~Гудкова~[15, 16] этот набор был 
расширен до четырнадцати событий, кодируемых четырьмя битами (рис.~3).

\begin{figure*} %fig3
 \vspace*{1pt}
 \begin{center}
 \mbox{%
 \epsfxsize=153.389mm
 \epsfbox{ush-3.eps}
 }
 \end{center}
 \vspace*{-9pt}
\Caption{Топологические события}
\vspace*{6pt}
\end{figure*}
      

      
      Топологические связи могут быть упорядочены однозначным образом (рис.~4). 
Если обозначить глу\-бин\-у прослеживания как~$d$, то тогда для каж\-дой минюции можно 
построить топологический дескриптор длиной $n_d\hm = 16d \hm+ 4$~бит. Например,\linebreak\vspace*{-12pt}
\begin{center} %fig4
\vspace*{1pt}
\mbox{%
 \epsfxsize=77.746mm
 \epsfbox{ush-4.eps}
}
%\end{center}
%\begin{center}

\vspace*{9pt}
{{\figurename~4}\ \ \small{Способы нумерации топологических связей}}
\end{center}

\vspace*{6pt}

%\smallskip
\addtocounter{figure}{1}


\noindent 
для глубины прослеживания $d \hm= 6$ получаем 100-бит\-ное описание минюции. 
В~среднем, отпечаток пальца содержит от 30 до 60~минюций~[17], т.\,е.\ 
топологические дескрипторы могут иметь суммарный объем до 6~Кбит. Таким образом, 
можно заключить, что рассмотренная выше топология позволяет извлекать достаточно 
много бинарной информации. 

\subsection{Статистические свойства дескрипторов}

      Перечислим желаемые свойства дескрипторов. Во-пер\-вых, на генеральной 
совокупности они должны иметь равновероятное распределение битов. Во-вто\-рых, если 
соответственные дескрипторы взяты из различных предъявлений одного и того же пальца 
(т.\,е.\ принадлежат одному человеку), то расстояние Хемминга между ними должно быть 
как можно ближе к~0. Если дескрипторы принадлежат разным людям, то требуется, чтобы 
их разница была близка к случайному шуму. Поскольку не исключается существование 
корреляций внутри дескрипторов, то важнейшим их параметром является энтропия. 
Статистические исследования настоящего раздела проводились на базах отпечатков 
пальцев FVC2002 DB1a~[18] и собственной базе отпечатков (полученных с 
использованием оптического сканера с разрешением 500~dpi).
      
      Для оценки вероятностей распределений битов анализировалось 
20\,000~топологических дескрипторов с глубиной прослеживания $d \hm= 4$. Результаты 
приведены на рис.~5. Как видно, единичные и нулевые биты имеют практически 
одинаковую вероятность, равную~0,5.


      Для исследования внутриклассовой вариации был проведен следующий 
эксперимент. Для двух  шаблонов <<своими>> дескрипторами называются те\linebreak\vspace*{-12pt}
\begin{center} %fig5
\vspace*{18pt}
\mbox{%
 \epsfxsize=65.974mm
 \epsfbox{ush-5.eps}
}
\end{center}
%\begin{center}
\vspace*{3pt}
{{\figurename~5}\ \ \small{Вероятности битов для каждой позиции в топологическом дескрипторе}}
%\end{center}

%\vspace*{6pt}

%\smallskip
\addtocounter{figure}{2}

\begin{figure*}[b] %fig7
 \vspace*{9pt}
 \begin{center}
 \mbox{%
 \epsfxsize=163.359mm
 \epsfbox{ush-7.eps}
 }
 \end{center}
 \vspace*{-9pt}
\Caption{Расстояния Хемминга для топологических дескрипторов соответственных~(\textit{1}) и 
несоответственных~(\textit{2}) минюций: (\textit{а})~$d\hm=1$; (\textit{б})~2; 
(\textit{в})~3; (\textit{г})~$d\hm=4$}
\end{figure*}


\noindent
\begin{center} %fig6
\vspace*{1pt}
\mbox{%
 \epsfxsize=78.512mm
 \epsfbox{ush-6.eps}
}
\end{center}
\begin{center}
\vspace*{9pt}
{{\figurename~6}\ \ \small{Пример соответствия контрольных точек}}
\end{center}

\vspace*{6pt}

%\smallskip
%\addtocounter{figure}{1}

\noindent
 дескрипторы, которые отвечают парам соответствующих друг другу минюций в двух 
предъявлениях одного отпечатка пальца. Иначе дескрипторы называются <<чужими>>. 

Примеры соответственных минюций приведены на рис.~6. 

На рис.~7 приведены 
гистограммы расстояний для базы FVC2002~DB1.


      Из представленных гистограмм видно, что для пар <<своих>> дескрипторов 
расстояние чаще всего оказывается ближе к нулю, а для <<чужих>> распределение 
достаточно близко соответствует случайному шуму. 

Подробный анализ распределения 
расстояний в начальных 20~битах дескриптора (соответст\-ву\-ющих глубине прослеживания 
$d\hm = 1$) показывает, что оно является биномиальным распределением с 18~степенями 
свободы, т.\,е.\ 20-бит\-ные дескрипторы практически случайны. Для больших глубин 
прослеживания ($d \hm> 2$) наблюдаются существенные корреляции и энтропия 
дескриптора начинает сильно отличаться от его длины. В~част\-ности, по результатам 
исследования FVC2002 DB1 распределение имеет 24~степени свободы для дескрипторов 
длиной 36~бит ($d\hm = 2$), 30~степеней~--- для 52~бит ($d\hm = 3$), 33~--- для 68 ($d 
\hm= 4$) и 35~--- для~84 ($d \hm= 5$). 

В~среднем ошибка наблюдается в 21\% битов для 
отпечатков в FVC2002 DB1 и 20\%~--- для собственной базы данных.
      

\section{Извлечение бинарного вектора без~вспомогательной 
информации}

      В рамках данной работы были разработаны два метода извлечения бинарного 
характеристического вектора. Первый из них вводит внутренний порядок на множестве 
дескрипторов и не требует публикации дополнительной информации. Второй вводит 
внешний порядок за счет публикации дополнительной информации в составе открытого 
хелпера. Формально эти методы можно представить в виде отображений~$f$, 
переводящих шаблон отпечатка пальца~$b$ в бинарный вектор $DH$ длиной~$n$. 
Рассмотрим построение первого алгоритма.
      
      В качестве исходных данных алгоритм использует множество топологических 
дескрипторов, найден\-ных на изображении отпечатка пальца. До\-пус\-тим, что в 
пространстве значений дескрипторов $[0,\,1]^{n_d}$ выбрано $2^l$~центров, 
обозначенных через $C\hm=\left\{ c_j\right\}^{2^l}_{j=1}$, $c_j\hm\in[0,\,1]^{n_d}$. Для 
каждого топологического дескриптора~$D_i$ выберем $k$ ближайших центров, 
используя в качестве метрики расстояние Хемминга. Таким образом, получим 
отображение неупорядоченного множества дескрипторов на множество центров:
      \begin{multline*}
      f_{1A}:\ \left\{ x_i,y_i,D_i\right\}^m_{i=1}\rightarrow \left\{ c_i^w\right\}\,,\quad 
i=1, \ldots ,m\,,\\ 
w=1,\ldots , k\,.
%      \label{e6-usm}
      \end{multline*}
      
      Теперь построим характеристический вектор $DH$ длиной $d_{DH}\hm = 2^l$. Для 
всех $j\hm\in [1\ldots 2^l]$ положим $j$-й бит вектора $DY$ равным~1 при условии, что 
$c_j\hm\in \left\{ c_i^w\right\}$ (т.\,е.\ хотя бы один дескриптор близок к $j$-му центру) 
и~0 в противном случае. В~векторе $DH$ будет приблизительно $mk$ ненулевых битов. 
Формально этот этап можно записать в виде отоб\-ра\-жения
      \begin{multline*}
      f_{1B}:\ \left\{ c_i^w\right\}\rightarrow DH\in [0,\,1]^{2^l}\,,\quad i=1, \ldots ,m\,,\\ 
w=1,\ldots , k\,.
%      \label{e7-usm}
      \end{multline*}
      
      Соответственно, искомое отображение $f_1$ будет являться 
композицией~$f_{1A}$ и~$f_{1B}$.
      
      Ключевой особенностью алгоритма является разбиение $[0,\,1]^{n_d}$ на области, 
характеризуемые соответствующим им множеством центров~$C$. Если~$C$ является 
разреженным множеством, т.\,е.\ расстояния Хемминга между элементами велики, 
результирующий алгоритм будет иметь высокий уровень FAR (false accept rate). 
Если $C$ плот\-но, то тогда вырастет~FRR (false reject rate). 
Одним из самых очевидных способов построения~$C$ является 
помехоустойчивое кодирование, а~именно: центры могут быть получены как результаты 
кодирования $l$~бит в~$n_d$, и в результате образуется отображение $i\hm\rightarrow c_i$. 
Соотношение между $c$ и плотностью центров~$e$, или корректирующая способность, 
может быть выведено из следующего неравенства:
      \begin{equation*}
      2^l\leq \fr{2^{n_d}}{\sum\limits_{i=0}^e C^e_{n_d}}\,.
%      \label{e8-usm}
      \end{equation*}
      %
      Приближенное решение этого неравенства находится из предела Шеннона:
      \begin{equation*}
      l\leq n_d\left( 1+\fr{e}{n_d}\log_2\left( \fr{e}{n_d}\right)\right)\,.
%      \label{e9-usm}
      \end{equation*}
      
      Для исправления ошибок в характеристическом векторе использовались 
      БЧХ-ко\-ды. Проведенное тестирование показало, что характеристические векторы, 
получаемые с помощью предложенного алгоритма, имеют низкую энтропию. 

\section{Извлечение бинарного вектора со~вспомогательной 
информацией}

      Альтернативным способом получения более длинного вектора из множества 
дескрипторов является конкатенация нескольких дескрипторов. Для этого необходимо 
ввести некоторый порядок на подмножестве минюций. С~этой целью на этапе 
регистрации из шаблона выбираются случайным образом $l$ минюций 
$\left\{p_i\right\}_{i=1,\ldots ,l}$, которые нумеруются, и их координаты сохраняются в 
составе открытого хелпера. Характеристический вектор получается конкатенацией 
дескрипторов и используется в схеме~(\ref{e2-usm}).
      
      На этапе верификации происходит поиск поворота~$R$, сдвига~$S$ и 
подмножества минюций $\left\{q_i\right\}_{i=1,\ldots ,l}$ предъявленного отпечатка, которые бы 
лучше всего соответствовали (в терминах среднеквадратичной ошибки) точкам, 
опубликованным в составе открытого хелпера:
      \begin{equation*}
      \sum\limits_{i=1}^l \left( Rq_i+S-p_i\right)^2\,.
%      \label{e10-usm}
      \end{equation*}
      
      Таким образом, устанавливается соответствие между $p_i$ и $q_i$, т.\,е.\ 
отыскиваются соответственные точки в двух предъявлениях. Для каждой из найден\-ных 
точек~$q_i$ построим ее топологический дескриптор~$D_i$. Они образуют 
упорядоченный набор и могут быть конкатенированы:
      \begin{equation*}
      DH=\overline{D_1, D_2, \ldots , D_l}\,,\enskip DH\in [0,\,1]^{ln_d}\,.
%      \label{e11-usm}
      \end{equation*}
      
      \begin{figure*} %fig8
 \vspace*{1pt}
 \begin{center}
 \mbox{%
 \epsfxsize=124.286mm
 \epsfbox{ush-8.eps}
 }
 \end{center}
 \vspace*{-9pt}
\Caption{Распределение ошибок в топологических векторах}
\end{figure*}
      
      Формально полученный алгоритм можно записать в виде отображения
      \begin{multline*}
      f_2:\ b\times \left\{ p_i\right\} \rightarrow DH\in [0,\,1]^{ln_d}\,,\enskip 
      2\leq l\leq m\,,\\
      i=1,  \ldots , l\,.
%      \label{e12-usm}
      \end{multline*}
      
      Обозначим длину характеристического вектора как $n_{DH}\hm = l n_d$. Тогда, к 
примеру, для глубины прослеживания $d \hm= 6$ и количества точек в хелпере $l \hm= 5$ 
алгоритм позволяет построить 500-бит\-ный характеристический вектор.
      

      
      Стоит заметить, что данный алгоритм чувствителен к <<выпадению>> 
      минюций~--- когда на предъявленном отпечатке пальца ключевая точка не была 
распознана правильно или ее предполагаемые координаты оказались за пределами 
информативной области отпечатка пальца. И~если бороться с первым достаточно 
нетривиально, то для второго есть простая эвристика~--- уменьшить вероятность 
включения граничных точек в $\left\{p_i\right\}_{i=1,\ldots ,l}$. Поэтому этап регистрации был 
доработан соответствующим образом. В~случае если минюция все же оказывается 
выпавшей, ее топологический дескриптор заменяется нулевым.
      
      Из формулы~(\ref{e2-usm}) очевидно следует, что ошибки в ключевой информации 
(из-за ассоциативности исключающего ИЛИ) есть в точности ошибки в векторе~$DH$, 
поэтому имеется возможность проанализировать их подробнее. Ошибки в нем можно 
разделить на две категории: блочные (связанные с <<выпадением>> минюций при 
построении $\left\{q_i\right\}$) и шумовые. Таким образом, в результирующей бинарной 
строке могут быть блоки с вероятностью ошибки 20\% (типичная шумовая ошибка 
дескрипторов, которые соответствуют верно сопоставленной паре минюций) и 50\% 
(рис.~8). Гистограммы распределения шумовых ошибок приведены на рис.~7.
      
      Для борьбы с подобными разнородными ошибками было применено двухслойное 
кодирование (рис.~9). На первом этапе ключ~$k$ длиной $\vert k\vert$ кодировался 
      БЧХ-ко\-дом в бинарную строку $K_{\mathrm{BCH}}$ длиной $n_{\mathrm{BCH}}$. Реализация БЧХ, 
использованная в эксперименте, позволяет восстанавливать до 25\% ошибок,\linebreak\vspace*{-12pt}

\begin{center} %fig9
\vspace*{1pt}
\mbox{%
 \epsfxsize=78.207mm
 \epsfbox{ush-9.eps}
}
\end{center}
\begin{center}
\vspace*{3pt}
{{\figurename~9}\ \ \small{Двухслойное кодирование}}
\end{center}

\vspace*{9pt}

%\smallskip
\addtocounter{figure}{1}

\noindent
 что само по 
себе недостаточно для полного восстановления ошибок с приемлемой вероятностью. 
Поэтому на втором этапе бинарная строка~$K$ получается путем $c$-крат\-ной 
репликации строки~$K_{\mathrm{BCH}}$:
      \begin{multline*}
      K=\underbrace{\overline{K_{\mathrm{BCH}},K_{\mathrm{BCH}},\ldots ,K_{\mathrm{BCH}}}}_c\,,\\
      K\in [0,\,1]^s\,,\enskip s=cn_{\mathrm{BCH}}\,.
      \end{multline*}


      Репликация помогает справиться с блочными ошибками. Очевидным требованием 
к параметрам кодирования является равенство длин~$K$ и~$DH$, т.\,е.\ необходимо 
выполнение равенства $s\hm = n_{DH}$. На этапе восстановления сначала выполняется 
восстановление~$K_{\mathrm{BCH}}$, а затем используется декодирование БЧХ-кодов.

\section{Эксперименты}

      Эксперименты по извлечению характеристического вектора без публикации 
дополнительных данных проводились только для FVC2002 DB1. Эксперимент c 
публикацией дополнительных данных проводился на обеих базах данных.
      
      Для каждого отпечатка проводился этап ре\-гист\-ра\-ции, после чего оставшиеся 
отпечатки с этого же пальца (по~2 для собственной базы данных и по~7 для FVC2002 DB1) 
использовались для верификации, что дало 5544 сравнений на TAR (true accept rate) 
для FVC2002 DB1 и 594 для собственной базы отпечатков. Для метода с 
публикацией дополнительной информации эксперимент проводился\linebreak\vspace*{-12pt}
\pagebreak

\noindent
{{\tablename~1}\ \ \small{Результаты эксперимента (собственная база данных)}}
%\end{center}

\vspace*{3pt}

\begin{center}
\tabcolsep=6.1pt
{\small 
\begin{tabular}{|c|c|c|c|c|c|c|c|}
      \hline
$\vert k\vert$&FAR&FRR&$n_{\mathrm{BCH}}$&$c$&$n_d$&$l$&$c$\\
\hline
      
15&0,03\%&12,09\%&127\hphantom{9}&\hphantom{9}3&64&6&381\\
22&0,00\%&18,35\%&127\hphantom{9}&\hphantom{9}3&64&6&381\\
\hline
15&0,03\%&11,11\%&127\hphantom{9}&\hphantom{9}3&80&5&381\\
22&0,00\%&14,08\%&127\hphantom{9}&\hphantom{9}3&80&5&381\\
29&0,00\%&16,97\%&127\hphantom{9}&\hphantom{9}3&80&5&381\\
\hline
16&0,07\%&\hphantom{9}4,67\%&63&\hphantom{9}7&64&7&441\\
18&0,02\%&\hphantom{9}5,38\%&63&\hphantom{9}7&64&7&441\\
24&0,00\%&11,54\%&63&\hphantom{9}7&64&7&441\\
30&0,00\%&14,29\%&63&\hphantom{9}7&64&7&441\\
36&0,00\%&17,97\%&63&\hphantom{9}7&64&7&441\\
\hline
16&0,02\%&\hphantom{9}3,25\%&63&\hphantom{9}7&76&6&441\\
18&0,02\%&\hphantom{9}5,42\%&63&\hphantom{9}7&76&6&441\\
24&0,02\%&12,64\%&63&\hphantom{9}7&76&6&441\\
30&0,00\%&15,16\%&63&\hphantom{9}7&76&6&441\\
\hline
16&0,07\%&\hphantom{9}3,13\%&31&15&64&8&465\\
21&0,02\%&\hphantom{9}5,88\%&31&15&64&8&465\\
26&0,01\%&10,38\%&31&15&64&8&465\\
\hline
16&0,11\%&\hphantom{9}2,89\%&31&15&80&6&465\\
21&0,05\%&\hphantom{9}5,78\%&31&15&80&6&465\\
26&0,02\%&11,19\%&31&15&80&6&465\\
\hline
11&0,87\%&\hphantom{9}1,32\%&15&32&64&8&480\\
\hline
15&0,00\%&\hphantom{9}4,91\%&127\hphantom{9}&\hphantom{9}5&80&8&635\\
22&0,00\%&\hphantom{9}7,59\%&127\hphantom{9}&\hphantom{9}5&80&8&635\\
29&0,00\%&\hphantom{9}9,82\%&127\hphantom{9}&\hphantom{9}5&80&8&635\\
36&0,00\%&17,41\%&127\hphantom{9}&\hphantom{9}5&80&8&635\\
\hline
16&0,07\%&\hphantom{9}1,34\%&31&21&84&8&651\\
21&0,04\%&\hphantom{9}2,23\%&31&21&84&8&651\\
26&0,02\%&\hphantom{9}5,80\%&31&21&84&8&651\\
\hline
16&0,02\%&\hphantom{9}1,34\%&63&11&88&8&693\\
18&0,02\%&\hphantom{9}2,23\%&63&11&88&8&693\\
24&0,00\%&\hphantom{9}4,46\%&63&11&88&8&693\\
30&0,00\%&\hphantom{9}\hphantom{9}6,70\%&63&11&88&8&693\\
36&0,00\%&\hphantom{9}7,14\%&63&11&88&8&693\\
39&0,00\%&\hphantom{9}8,93\%&63&11&88&8&693\\
45&0,00\%&11,61\%&63&11&88&8&693\\
\hline
\end{tabular}

}
\end{center}

\vspace*{3pt}

\begin{center}
\noindent
{{\tablename~2}\ \ \small{Результаты эксперимента (FVC2002 DB1)}}
%\end{center}

\vspace*{9pt}

{\small 
\tabcolsep=6.2pt
\begin{tabular}
{|c|c|c|c|c|c|c|c|}
\hline
$\vert k\vert$&FAR&FRR&$n_{\mathrm{BCH}}$&$c$&$n_d$&$l$&$c$\\
\hline
36&0,61\%&10,30\%&127&3&64&6&381\\
43&0,36\%&12,60\%&127&3&64&6&381\\
\hline
30&2,20\%&\hphantom{9}2,80\%&\hphantom{9}63&7&64&7&441\\
36&0,95\%&\hphantom{9}5,20\%&\hphantom{9}63&7&64&7&441\\
39&0,36\%&\hphantom{9}9,00\%&\hphantom{9}63&7&64&7&441\\
45&0,16\%&15,70\%&\hphantom{9}63&7&64&7&441\\
\hline
\end{tabular}

}
\end{center}
\vspace*{6pt}
%\end{table*}

%\end{table*}

\smallskip
\addtocounter{table}{1}






\noindent
 несколько раз, так 
как из-за случайности выбора точек имеется возможность провести несколько процедур 
регистрации для каждого отпечатка пальца. Для FVC2002 DB1 было выполнено два 
прохода, для собственной базы~--- десять. В~табл.~1 и~2 
и на рис.~10 и~11 приводятся усредненные 
результаты FRR и~FAR. 

\columnbreak
      

\noindent
\begin{center} %fig10
\vspace*{1pt}
\mbox{%
  \epsfxsize=72.074mm
 \epsfbox{ush-10.eps}
}
\end{center}
%\begin{center}

\vspace*{1pt}
\noindent
{{\figurename~10}\ \ \small{Кривая DET (detection error tradeoff) для алгоритма без публикации дополнительной информации,
длина БЧХ-кода 64~бита, $\vert k\vert \hm= 16$}}
%\end{center}

\vspace*{6pt}

\begin{center} %fig11
\vspace*{1pt}
\mbox{%
  \epsfxsize=73.158mm
 \epsfbox{ush-11.eps}
}
\end{center}
%\begin{center}

\vspace*{1pt}
\noindent
{{\figurename~11}\ \ \small{Кривая DET для алгоритма с публикацией дополнительной информации для различной 
глубины и количества точек (на подписях к точкам указаны энтропия ключа/длина 
характеристического вектора): \textit{1}~--- $d\hm=1$, $l\hm=10$;
\textit{2}~--- $d\hm=2$, $l\hm=8$;
\textit{3}~--- $d\hm=3$, $l\hm=6$;
\textit{4}~--- $d\hm=4$, $l\hm=6$;
\textit{5}~--- $d\hm=5$, $l\hm=8$}}
%\end{center}

\vspace*{9pt}

%\smallskip
%\addtocounter{figure}{1}

Как видно из графиков рис.~10 и~11, 
алгоритм с пуб\-ли\-ка\-ци\-ей дополнительной информации
 в открытом хелпере позволяет 
реализовывать защищенную верификацию по отпечаткам пальцев с приемлемыми 
ошибками первого и второго рода. 


\section{Заключение}
      В данной статье предложены алгоритмы защищенной биометрической 
верификации по отпечатку пальца. Для их реализации был предложен метод извлечения 
повторяемого бинарного вектора из изображения отпечатка пальца. Главные 
преимущества реализованного метода~--- устойчивое кодирование окрестности 
контрольных точек и, как следствие, приемлемые значения ошибок алгоритмов 
верификации ($\mathrm{TAR}\hm > 0{,}9$, $\mathrm{FAR}\hm\leq 10^{-3}$). Слабым местом алгоритма с 
частичной публикацией минюций является потенциальная утечка час\-ти 
пользовательского шаблона. Несмотря на то что идентифицировать личность 
пользователя по 5--8 случайным минюциям практически не\-воз\-можно, угроза 
безопасности пользовательского шаблона является недостатком алгоритма. 
Предложенный алгоритм без публикации информации \mbox{имеет} значительные ошибки 
первого и второго рода ($\mathrm{FRR}\hm\geq 0{,}3$ при $\mathrm{FAR}\hm =0{,}1\%$).
      
      В качестве направлений дальнейших исследований следует отметить 
совершенствование алгоритма с частичной публикацией минюций. В~част\-ности, 
перспективным подходом минимизации утечки представляется использование 
необратимого преобразования публикуемого набора минюций. Также можно увеличить 
энтропию топологических дескрипторов, если рассматривать не только топологические 
события, но и длину соответствующей топологической связи. 

{\small\frenchspacing
{%\baselineskip=10.8pt
\addcontentsline{toc}{section}{Литература}
\begin{thebibliography}{99}
      
\bibitem{1-usm}
\Au{Abraham D.\,G., Dolan G.\,M., Double~G.\,P., Stevens~J.\,V.}
Transaction security system~// IBM Syst.~J., 1991. Vol.~30. No.\,2. P.~206--229.

\bibitem{2-usm}
\Au{Ratha N., Connell J., Bolle~R.}
Enhancing security and privacy in biometrics-based authentication systems~// IBM Syst.~J., 
2001. Vol.~40. No.\,3. P.~614--634.

\bibitem{3-usm}
\Au{Bringer J., Chabanne H., Izabachene~M., Pointcheval~D., Tang~Q., Zimmer~S.}
An application of the goldwasser-micali cryptosystem to biometric authentication~// 
Conference (Australian) on Information Security and Privacy, ACISP 2007
Proceedings.~--- Berlin--Heidelberg: Springer,  2007. LNCS~4586. P.~96--106. 

\bibitem{4-usm}
\Au{Dodis Y., Ostrovsky R., Reyzin~L., Smith~A.}
Fuzzy extractors: How to generate strong keys from biometrics and other noisy data~// SIAM~J. 
Computing, 2008. Vol.~38. No.\,1. P.~97--139. 

\bibitem{5-usm}
\Au{Clancy T.\,C., Kiyavash N., Lin D.\,J.}
Secure smartcard-based fingerprint authentication~// ACM Workshop on Biometrics: Methods 
and Applications.~--- Berkeley, CA, Nov., 2003. P.~45--52.


\bibitem{8-usm} %6
\Au{Tong V.\,V.\,T., Sibert H., Lecoeur~J., Girault~M.}
Biometric fuzzy extractors made practical: A proposal based on fingercodes~// Advances in 
biometrics.~--- Berlin--Heidelberg: Springer, 2007. LNCS~4642. P.~604--613.

\bibitem{6-usm} %7
\Au{Arakala A., Jeffers J., Horadam~K.\,J.}
Fuzzy extractors for minutiae-based fingerprint authentication~// Advances in biometrics.~--- 
Berlin--Heidelberg: Springer, 2007. LNCS~4642. P.~760--769.

\bibitem{9-usm} %8
\Au{Draper S., Yedidia J., Khisti~A., Martinian~E., Vetro~A.}
Using distributed source coding to secure fingerprint biometrics~//  Conference (International)
on  Acoutics Speech Signal Processing Proceedings.~--- Honolulu, 2007. P.~129--132.

\bibitem{11-usm} %9
\Au{Farooq F., Bolle R.\,M., Jea~T.-Y., Ratha~N.\,K.}
Anonymous and revocable fingerprint recognition~// IEEE Conference on Computer Vision and 
Pattern Recognition, CVPR-07, 2007. P.~1--7. 

\bibitem{7-usm} %10
\Au{Barni M., Bianchi T., Catalano D., Raimondo~M., Labati~R., Failla~P., Fiore~D., 
Lazzeretti~R., Piuri~V., Scotti~F., Piva~A.}
Privacy-preserving fingercode authentication~// 12th ACM Workshop on Multimedia and 
Security Proceedings, 2010. P.~231--240.

\bibitem{10-usm} %11
\Au{Nagar A., Rane S.\,D., Vetro~A.}
Alignment and bit extraction for secure fingerprint biometrics~// Proc. SPIE, 2010. Vol.~7541; 
doi:10.1117/12.839130. 


\bibitem{12-usm}
\Au{Bringer J., Despiegel V.}
Binary feature vector fingerprint representation from minutiae vicinities~//  4th IEEE  
Conference (International) on Biometrics: Theory, Applications and Systems (BTAS-10) 
Proceedings, 2010. P.~1--6.

\bibitem{13-usm}
\Au{Hao F., Anderson R., Daugman~J.}
Combining cryptography with biometrics effectively: Technical Report UCAM-CL-TR-640.~--- 
Cambridge: University of Cambridge Computer Laboratory, 2005. 17~p.

\bibitem{14-usm}
\Au{Sparrow M.\,K.}
Vector based topological fingerprint matching. U.S.\ Patent 5631971. May 20, 1997.

\bibitem{16-usm} %15
\Au{Гудков В.\,Ю.}
Способ кодирования отпечатка папиллярного узора. Патент РФ №\,2321057 от 04.12.2006.

\bibitem{15-usm} %16
\Au{Гудков В.\,Ю.}
Способ генерирования набора параметров ключа доступа и система для аутентификации 
человека по отпечаткам пальцев: Патент РФ №\,2363048 от 11.10.2007.

\bibitem{17-usm}
\Au{Pankanti S., Prabhakar S., Jain~A.\,K.}
On the individuality of fingerprints~// IEEE Trans. PAMI, 2002. Vol.~24. No.\,8. P.~1010--
1025. 

\label{end\stat}

\bibitem{18-usm}
\Au{Maio D., Maltoni D., Cappelli~R., Wayman~J.\,L., Jain~A.\,K.}
FVC2002: Second fingerprint verification competition~// 16th Conference (International) on 
Pattern Recognition (ICPR2002) Proceedings.~--- Quebec City, 2002. Vol.~3. P.~811--814.
 \end{thebibliography}
}
}


\end{multicols}       

\newpage