\renewcommand{\figurename}{\protect\bf Figure}
\renewcommand{\tablename}{\protect\bf Table}
\renewcommand{\bibname}{\protect\rmfamily References}

\def\stat{kalin}

\def\tit{EXTENDING INFORMATION INTEGRATION TECHNOLOGIES 
FOR~PROBLEM SOLVING OVER~HETEROGENEOUS INFORMATION 
RESOURCES$^*$}

\def\titkol{Extending information integration technologies 
for~problem solving over~heterogeneous information 
resources}

\def\autkol{L.\,A.~Kalinichenko, S.\,A.~Stupnikov, and~V.\,N.~Zakharov}
\def\aut{L.\,A.~Kalinichenko$^1$, S.\,A.~Stupnikov$^2$, and~V.\,N.~Zakharov$^3$}

\titel{\tit}{\aut}{\autkol}{\titkol}

{\renewcommand{\thefootnote}{\fnsymbol{footnote}}\footnotetext[1]
{This work has been partially supported by the RFBR grants 10-07-00342-a and 
11-07-00402-a  and by the project~4.2 of the 
Program for basic research No.\,16 of the Presidium of RAS.}}


\renewcommand{\thefootnote}{\arabic{footnote}}
\footnotetext[1]{Institute of Informatics Problems, Russian Academy of Sciences, leonidk@synth.ipi.ac.ru}
\footnotetext[2]{Institute of Informatics Problems, Russian Academy of Sciences, ssa@ipi.ac.ru}
\footnotetext[3]{Institute of Informatics Problems, Russian Academy of Sciences, vzakharov@ipiran.ru} 


\Abste{This position paper is an attempt to match up the emerging 
challenges for problem solving over heterogeneous distributed 
information resources. State-of-the-art in subject mediation 
technology reached at IPI RAN is presented. The technology 
is aimed at filling the widening gap between the users 
(applications) and heterogeneous resources of data, knowledge, 
and services. Also, the paper affects the semantic-based information 
integration technologies challenges including investigation of 
application-driven approach for problem solving in the subject 
mediator environment, a provision for support of executable 
declarative specifications of the applications over the mediator, 
enhancement of presence of knowledge-based facilities at the mediator level, 
and mediation of databases with nontraditional data models motivated by the need 
of large data support.}

\KWE{subject mediation; heterogeneous information resources; scientific problem solving; information 
integration; application-driven approach; rule-based languages; nontraditional data models}

 \vskip 14pt plus 9pt minus 6pt

      \thispagestyle{headings}

      \begin{multicols}{2}

            \label{st\stat}

\section{Introduction}

\noindent
This position paper\footnote[4]{The paper has been prepared in connection with the first international call for Research 
Center proposals in frame of the MIT/SkTech initiative ({\sf http://web.mit.edu/SkTech/rc-call/}).} is an 
attempt to match up the emerging challenges for problem solving over heterogeneous information 
resources. Methods and tools for integration of data-oriented information systems within the Internet 
where heterogeneous databases are designed in different application domain contexts, modeled and 
implemented independently are well studied~[1]. The trend in data integration is in the direction of 
providing a unified data access and manipulation interface over a mediated schema. Each data resource in 
such middleware architecture is represented as a view over the mediated schema (the approach known as 
``view-based data manipulation'' in the settings known as LAV (Local As View)
and GLAV (Global/Local As View)~[2]). In such settings, the 
mediated schema is expressed in the context of the application domain applying the respective concept 
definitions. Semantic conflicts between the contexts of resources and of the application domain should be 
resolved. Such approach is called ``application-driven mediation'' in contrast to the ``resource-driven 
mediation'' intrinsic for the GAV (Global As View) setting.

The information integration technologies challenges touched in this paper include an emphasis on 
semantic issues requiring, in particular, a provision for support of executable declarative specification of 
the application over the mediator, enhancement of presence of knowledge-based facilities at the mediator 
level, compliance with the proliferation of new kinds of data sources in the data space, specifically with 
the need in the big data reflected in development of various nontraditional database systems and data 
models to be integrated.

\section{State-of-the-Art in~Subject Mediation Reached at~IPI RAN}

\subsection{Basic principles}

\noindent
Subject mediation technology is aimed to fill the widening gap between the users (applications) and 
heterogeneous distributed information resources of data, knowledge, and services and to provide 
methods and tools for problem solving over the multitude of such resources. Basic principles of 
subject mediation based organization of problem solving are the following~[3]:
\begin{itemize}
\item independence of definition of problem domain (the mediator definition) of the existing 
information resources;\\[-9pt]
\item definition of a mediator as a result of consolidated efforts of the respective scientific 
community;\\[-9pt] 
\item independence of user interfaces of the multiple information resources involved: the 
mediator users should know only the definition of a problem domain in the mediator 
(definition of concepts, structures, and behavior of the problem domain objects);\\[-9pt]
\item information about new resources can be published at any time independently of 
mediators acting at that time, such new resources (if relevant) can be integrated in the existing 
mediators without changing their (mediators) specifications;\\[-9pt]
\item GLAV-based setting for relevant information resources integration at the mediator;\\[-9pt] 
\item integrated access to the information resources in the process of problem solving; and\\[-9pt]
\item recursive structure of a mediator: each mediator is published as a new information 
resource that might be used while solving the problems belonging to the intersection of various 
subject domains.
\end{itemize}

\subsection{Canonical information model synthesis}

\noindent
To provide an integration of heterogeneous resources in the mediators, it is required to develop the 
\textit{canonical information model}~\cite{3kal} serving for adequate reflection of semantics of 
various information models used in the environment. The main principle of the \textit{canonical 
model synthesis} is the \textit{extensibility} of the canonical model kernel. A~kernel itself is fixed. 
For each specific information model~$M$ (called source model) of the environment, an extension of 
the kernel (target model) is defined so that this extension together with the kernel is refined 
by~$M$. Such refining transformation of models should be provably correct. The canonical model 
for the environment is synthesized as the union of extensions, constructed for all models~$M$ of 
the environment.

\subsection{Resources identification and~integration}

\noindent
Identification of resources relevant to the mediator specification is based on three models: 
\begin{enumerate}[($i$)]
\item metadata model (characterizing resource capabilities represented in external registries); 
\item ontological model (providing for definition of mediator concepts); and 
\item canonical model (providing 
for conceptual definition of structure and behavior of mediator objects). 
\end{enumerate}

Reasoning in canonical 
model is based on the semantics of the canonical model and facilities for proof of the refinement. 
Reasoning in the metadata model is a heuristic one based on nonfunctional requirements for the 
resources needed in application.

A process of integration of relevant information resources in a subject mediator (registration) 
follows GLAV that combines two approaches: LAV and GAV~\cite{4kal}. 
Such integration technique provides for stability of application problem 
specification during any modifications of specific information resources and of their actual 
presence as well as for scalability of mediators with regard to
the number of resources integrated.

\subsection{Subject mediation infrastructure}

\noindent
The subject mediation infrastructure is multilayered including resource layer, computation, and 
information resource environments (that can include \textit{grids} or \textit{clouds}), wrapper 
layer used for technical interface interoperability, semantic mediation middleware layer 
representing subject area semantics and defining unified mappings of various resource information 
models, application problem domain layer (Fig.~1). 


\subsection{Results obtained at~IPI RAN in~the~area of~semantic information 
integration and~mediation}

\noindent
\begin{enumerate}[1.]
\item A prototype of the subject mediation infrastructure has been developed~[5, 6]. The 
prototype is used for problem solving over multiple distributed information resources in 
astronomy problem domain. One of the possible environments for resource organization used is the 
AstroGrid environment.\\[-9pt]
\item Methods and tools for mapping and transformation of information models of 
heterogeneous resources intended for their \textit{unification} in mediation middleware have 
been developed. A~resource model is said to be unified if its refining mapping into the canonical 
information model has been constructed. As a canonical model kernel, a specific hybrid 
semistructured and object oriented data model (the SYNTHESIS language~\cite{2kal}) is used. 
Model mappings are verified using B-technology formalizing specification refinement. The 
Model Unifier prototype tool aimed at partial automation of heterogeneous information models 
unification has been implemented. First version is based on term-rewriting technology~[7, 8]. 
The second version as an Eclipse platform application based on model transformation 
languages is under implementation.
\end{enumerate}

\end{multicols}

\begin{figure}[h] %fig1
\vspace*{1pt}
 \begin{center}
 \mbox{%
 \epsfxsize=160.657mm
 \epsfbox{kal-1.eps}
 }
 \end{center}
 \vspace*{-9pt}
\Caption{Subject mediation infrastructure}
\vspace*{6pt}
\end{figure}


\begin{multicols}{2}

\begin{enumerate}[1.]
\addtocounter{enumi}{2}
\item Methods for information resources semantic interoperability support in a context of 
application problem domain have been investigated. Tools for identification of resources 
relevant to a problem on the basis of and ontological descriptions of problem domain as well as 
tools for registration of the relevant resources in the mediator have been implemented.\\[-9pt]
\item Methods and tools for rewriting of nonrecursive mediator programs into resource 
partial programs have been developed. The methods are oriented on applying of object 
schemas of resources and mediators and typed GLAV-views.\\[-9pt]
\item A method for optimizing planning of resource partial programs execution over 
distributed environment has been developed. The method takes into account capabilities of the 
resources, assigns places of operation's execution on the basis of estimative samples, applying 
an interface for interoperation of planner and wrappers which allows dynamic estimations of 
plans efficiency.\\[-10pt]
\item Methods for dispersed organization of problem solving in the mediation environment 
have been developed. An implementation of a problem in mediation environment may be 
dispersed among programming systems, mediators, GLAV-views, wrappers and resources. 
Methods and tools for representation, manipulation, and estimation of efficiency of dispersed 
organization are provided. Algorithms for construction of efficient dispersed organization have 
been implemented.\\[-10pt]
\item An original approach for binding of programming languages with declarative mediator 
rule language has been implemented. The approach combines static and dynamic binding 
overcoming impedance mismatch and allowing dynamic result types.
\end{enumerate}

%\pagebreak

Most of the methods developed so far are applicable not only for data integration, but for data 
exchange~\cite{9kal} as well.

\section{Directions of~Research and~Development}

\subsection{Investigation of application-driven approach for~scientific problem 
solving in~the~subject mediator environment}

\noindent
According to the application-driven approach, on the basis of a problem domain, an ontology 
(concepts and relationships among them) of the domain is created. After that, a conceptual schema 
(including data structures and methods required for problem solving) is created. Thus, a semantic 
specification of a problem independent of concrete resources is developed. 

Integration of resources relevant to a problem in a subject mediator requires semantic schema 
mapping methods and tools to be used for construction of mappings between mediator and 
resources schemas.

The authors got an experience of applying the application-driven approach to several problems, one of them 
being the \textit{problem of secondary standards search for photometric calibration of optical 
components of gamma-ray bursts} formulated by the Institute of Space Research of RAS and 
specified as text in a natural language. The idea of the problem is to find a set of standard stars 
(stars with well-known parameters) in some area around a gamma-ray burst. The problem was 
formalized and implemented applying the subject mediation. To do that, a glossary of the problem 
domain was manually extracted from the textual specification and astronomical literature. After 
that, an ontology required for problem solving was constructed. Data structures (abstract data 
types), methods, and functions (e.\,g., for cross-match, color index calculation, star variability 
checking) constituting problem domain schema were defined.

Resources relevant to the problem were identified in the Astrogrid and VizieR information grids. 
A~set of distributed resources includes SDSS, USNO B-1, 2MASS, GSC, and UCAC catalogs used for 
extraction of standards and VSX, ASAS, GCVS, and NSVS catalogs used for star variability checking. The 
resources were registered in the mediator and corresponding GLAV-views were obtained. The 
problem was formulated as a program consisted of a set of declarative rules over the mediator 
schema. 

The implemented mediator is a basis for an application monitoring in real time the e-mails 
informing about the gamma-ray bursts. The application extracts standards located in the area of a 
burst and e-mails them to subscribers~\cite{6kal}.

The issues requiring further investigations include:
\begin{itemize}
\item semantic identification of resources relevant to a mediator;
\item construction of semantic source to target schema mapping in the presence of constraints 
reflecting specificity of various data models; and
\item development of mediator program rewriting algorithms in the presence of source and 
mediator constraints over the classes of objects.
\end{itemize}

\subsection{Heterogeneous multidialect mediator infrastructure for~data, 
knowledge, and~services semantic integration}

\subsubsection{An approach for~the~infrastructure}

\noindent
Recently, the World Wide Web Consortium (W3C) adopted RIF (Rule Interchange Format, 
{\sf http://www. w3.org/TR/2010/NOTE-rif-overview-20100622/}) standard oriented on 
providing of interoperability of declarative programs represented in different languages and 
rule-based programming (inference) systems. The standard is oriented not only on Semantic Web, but 
also on a creation of the intellectual information systems as well as on a knowledge representation 
in different application areas. This standard motivated the following investigation: to find a solution 
of the complicated problem of integration of multilanguage knowledge representations and 
rule-based declarative programs, heterogeneous databases, and services on the basis of unified 
languages and multidialect mediation infrastructure. The methods and tools developed are aimed at 
scientific problems solving over heterogeneous distributed information resources. The methods and 
tools to be developed are intended for combining two paradigms of extensible unified languages 
construction. The first one is W3C RIF standard paradigm. The second one is a 
paradigm based on the GLAV approach built on the extensible canonical information model idea, 
applicable for database, service and process languages unification, and mediation.

The idea of the proposed approach consists in developing of a modular mediator infrastructure in 
which alongside with the modules representing data and services in the GLAV setting, the modules 
representing knowledge and declarative rule-based programs over various resources will be 
introduced. The infrastructure is planned to be based on the following principles:
\begin{itemize}
\item \textit{the multidialectal construction of the canonical model}. Mediators are represented 
as a functional composition of declarative specification of modules, each based on its own 
dialect with an appropriate semantics. Semantic of a conceptual definition in such setting 
becomes a multidialect one;
\item \textit{the mediator modules as peers}. Rule-based modules become the mediator 
components alongside with the GLAV-based modules. Interoperability of the modules is based 
on peer-to-peer (P2P) and W3C RIF techniques;
\item \textit{combination of integration and interoperability}. The information resource 
integration can be provided in the scope of an individual mediator module. The integration 
approaches in different modules can be different. The interoperability is provided between the 
modules supporting different dialects; and
\item \textit{rule-based specifications on different levels of the infrastructure}. Rule-based, 
inference providing modules are used for declarative programming over the mediators, to 
support various modules of a mediator, to support schema mapping for semantic integration of 
the information resources in the mediator, etc.
\end{itemize}

\subsubsection{Example}

\noindent
The idea of the multidialect mediation infrastructure is demonstrated on example (Fig.~2) of 
finding an optimal assignment of applicants among universities. The program calculating such 
assignment is defined in DLV (Answer Set Programming). The required information resources are 
integrated in a SYNTHESIS mediator. OntoBroker communicates with users and applying its 
ontologies, formulates the queries to the mediator and after collecting the required data, initiates a 
program in DLV. The assignment problem is formulated as follows. A~set of $n$ applicants is to be 
assigned among $m$ universities, where $q_i$ is the quota of the $i$th college. Applicants 
(universities) rank the universities (the applicants) in the order of their preference. The aim is to 
find optimal assignment from the quotas of the colleges and the two sets of orderings. 

An assignment of applicants to colleges is \textit{unstable} if there are two applicants~$\alpha$ 
and~$\beta$ who are assigned to colleges~$A$ and~$B$, respectively, although $\beta$ prefers~$A$ 
to~$B$ and $A$~prefers~$\beta$ to~$\alpha$; otherwise, an assignment is \textit{stable}.

A stable assignment is called \textit{optimal} if every applicant is at least as well off under it as 
under any other stable assignment. 

\subsubsection{Issues to~be investigated and~prototyped}

\noindent
\begin{enumerate}[1.]
\item Approaches for constructing the rule-based dialect mappings.
\item Methods for justification of semantic preservation by the mappings (e.\,g., as a preserving 
of entailment of initial formulae by the mapping justified by test case checking; reducing 
entailment to refinement; manual proof using structural induction over constructs of a dialect, 
etc.).
\item Investigation of approaches for modular representation of knowledge in the multidialect 
mediation environment.
\item Investigation of approaches for providing the interoperability of the mediator multidialect 
modules.
\item Infrastructure design and prototyping.
\item Real problems solving in a scientific subject domains chosen.
\item Expansion of the experience into the Semantic Web area.
\end{enumerate}

\subsection{Mediation of~databases with~nontraditional data models}

\subsubsection{Motivation for~the~proposed research and~development}

\noindent
The objective is to develop an approach providing for semantic integration of information 
resources represented in frame of the traditional as well as nontraditional data models aiming at 
problem solving over such integrated ensemble of resources. 

During last years, nonrelational data models are being intensively developed called here 
collectively as nontraditional data models. Classes of such data models include NoSQL
(not only Structured Query Language), graph data 
models, triple-based data models, ontological data models, ``scientific'' data models, etc.

\textit{The NoSQL data models} are oriented on the support of extra large volumes of data applying a 
``key-value'' technology for vertical storage. 
The examples of such systems include Dynamo, BigTable, 
HBase, Cassandra, MongoDB, CouchDB.

In the class of \textit{graph data models}, one can find such systems as Neo4j, InfiniteGraph, DEX, 
InfoGrid, HyperGraphDB, Trinity, and supporting flexible data structures.

\begin{figure*} %fig2
\vspace*{1pt}
 \begin{center}
 \mbox{%
 \epsfxsize=146.911mm
 \epsfbox{kal-2.eps}
 }
 \end{center}
 \vspace*{-9pt}
\Caption{Example of a problem specification in the multidialect mediation infrastructure}
\end{figure*}

\textit{The triple-based data model} (expressible in RDF (Resource Description Framework)
and RDFS (RDF Schema)) is used for representing information 
about the Web resources and is used together with the OWL profiles, logic inference terchniques, 
SPARQL query language. Systems belonging to this class include Virtuoso, OWLIM, 5Store, 
and Bigdata.

One of the OWL profiles (OWL QL) is oriented on support of \textit{ontological modeling} over 
relational databases. Recently, it was found an equivalent to OWL-inherent description logics 
mechanism expressible by data dependencies used together with declarative query languages (such 
as Datalog). 

For the support of ``\textit{scientific}'' data, several approaches are being developed (e.\,g., SciDB 
applying a multidimensional array data model). 

It is remarkable that for the most of these data models, the standards still do not exist. 
The most of these 
data models and systems are oriented on ``big data'' support applying massive parallel technique of 
the MapReduce kind.

Such diversity of nontraditional data models and systems makes urgent the investigation of a 
possibility of such data stores usage in frame of the subject mediation paradigm in the GLAV 
setting. Such investigation includes the methods of mapping and transformation into the canonical 
information mediation model of the nontraditional data models, techniques of interpretation of 
canonical model facilities in data manipulation languages of the nontraditional classes, as well as 
an efficiency of such interpretation.

\subsubsection{The results of~research to~be obtained}

\noindent
\begin{enumerate}[1.]
\item Development of information preserving methods of mapping and techniques for design of 
transformations of various classes of nontraditional data models into the canonical one, design 
of such mappings, and transformations for specific data models and of adequate extensions of 
the canonical data model.
\item Investigation of techniques for interpretation of canonical model 
data manipulation language (DML) (including query 
language) in the DMLs of different classes of nontraditional data models and approaches for 
their implementation.
\item Obtaining of architectural decisions on implementation of the massive parallel techniques 
on the level of mediators, evaluation of performance growth that can be reached.
\item Evaluation of suitability and efficiency of integration of nontraditional data models of 
different classes in the GLAV mediation infrastructure for various problem domains.
\end{enumerate}

\subsection{Storage of~very large volumes of~data}

\noindent
The objective is to develop a \textit{novel distributed parallel fault-tolerant file system} possessing 
the following capabilities:
\begin{itemize}
\item storage of data volumes of petabyte scale;
\item unlimited period of storage;
\item scalability;
\item efficient multiuser access support in different kinds of networks; and
\item usage of different storage types (e.\,g., hard disk drive and flash memory).
\end{itemize}

The experience of existing file systems vendors should be taken onto account:
\begin{itemize}
\item ReFS (Windows Server 8) by Microsoft;
\item VMFS by VMware;
\item Lustre;
\item ZFS by Sun Microsystems;
\item zFS (z/OS) by IBM; and
\item OneFS by Isilon.
\end{itemize}

\section{Concluding Remarks}

\noindent
The paper discusses the research and development directions required to match up the information 
integration technologies challenges including an emphasis on semantic issues requiring a provision for 
support of executable declarative specification of the application over the mediator, enhancement of 
presence of knowledge-based facilities at the mediator level, compliance with the proliferation of new 
kinds of data sources in the data space, specifically with the need in the big data reflected in development 
of various nontraditional database systems and data models to be integrated.

The paper has been prepared for the discussions in connection with the Information Session on 
SkTech International Research Centers Call for Proposals, February 9--10, 2012, 
at the Massachusetts 
Institute of Technology ({\sf http://web.mit.edu/sktech/news-events/pr-save-the-date.html}). 

%\bigskip

%\noindent
\section*{Acknowledgements}
The authors express deep gratitude to Dmitry Kovalev for the example formulation 
(paragraph~3.2.2).
 
 {\small\frenchspacing
{%\baselineskip=10.8pt
\addcontentsline{toc}{section}{Литература}
\begin{thebibliography}{9}

\bibitem{1kal}
\Au{Kalinichenko L.\,A., Briukhov~D.\,O., Martynov~D.\,O., Skvort\-sov~N.\,A., Stupnikov~S.\,A.}
Mediation framework for enterprise information system infrastructures~// 9th Conference 
(International) on Enterprise Information Systems ICEIS 2007 Proceedings:
Volume databases and information systems integration.~---
Funchal, 2007.  P.~246--251.

\bibitem{4kal} %2
\Au{Briukhov D.\,O., Kalinichenko L.\,A., Martynov~D.\,O.}
Source registration and query rewriting applying LAV/GLAV techniques in a typed subject 
mediator~// 9th Conference (Russian) on Digital Libraries RCDL'2007 Proceedings.~--- 
Pereslavl-Zalesskij: Pereslavl University, 2007. P.~253--262.

\bibitem{2kal} %3
\Au{Kalinichenko L.\,A., Stupnikov S.\,A., Martynov~D.\,O.}
SYNTHESIS: A language for canonical information modeling and mediator definition for problem 
solving in heterogeneous information resource environments.~--- M.: IPI RAN, 2007. 171~p.

\bibitem{3kal} %4
\Au{Kalinichenko L.\,A.}
Canonical model development techniques aimed at semantic interoperability in the heterogeneous 
world of information modeling~// Knowledge and model driven information systems engineering 
for networked organizations: CAiSE INTEROP Workshop Proceedings.~--- Riga: Riga Technical 
University, 2004.  P.~101--116.


\bibitem{5kal} %5
\Au{Briukhov D.\,O., Vovchenko A.\,E., Zakharov~V.\,N., Zhe\-len\-ko\-va~O.\,P., Kalinichenko~L.\,A., 
Martynov~D.\,O., Skvort\-sov~N.\,A., Stupnikov~S.\,A.}
The middleware architecture of the subject mediators for problem solving over a set of 
integrated 
heterogeneous distributed information resources in the hybrid grid-infrastucture of virtual 
observatories~// Informatics and Applications, 2008.  Vol.~2. Is.~1. P.~2--34.

\bibitem{6kal}
\Au{Vovchenko A.\,E., Kalinichenko L.\,A., Stupnikov~S.\,A.}
Mediation based semantic grid. distributed computing and grid-technologies in science and 
education~// 4th  Conference (International) Proceedings.~--- Dubna: JINR, 2010.  P.~309--318.

\bibitem{7kal}
\Au{Zakharov V.\,N., Kalinichenko L.\,A., Sokolov~I.\,A., Stup\-ni\-kov~S.\,A.}
Development of canonical information models for integrated information systems~// Informatics 
and Applications, 2007.  Vol.~1. Iss.~2. P.~15--38.

\bibitem{8kal}
\Au{Kalinichenko L.\,A., Stupnikov S.\,A.}
Constructing of mappings of heterogeneous information models into the canonical models of 
integrated information systems~// Advances in Databases and Information System:  
12th  Conference (East-European) Proceedings.~--- Pori: Tampere University of Technology, 2008.  P.~106--122.

\bibitem{9kal}
\Au{Fagin R., Kolaitis P., Miller~R., Popa~L.}
Data exchange: Semantics and query answering~// Theor. Computer Sci., 2005.  Vol.~336. 
No.\,1. P.~89--124.

 \end{thebibliography}
}
}


\end{multicols}

\vspace*{6pt}

\hrule

\vspace*{12pt}


\def\tit{РАЗВИТИЕ ТЕХНОЛОГИЙ ИНТЕГРАЦИИ ИНФОРМАЦИИ 
ДЛЯ~РЕШЕНИЯ ЗАДАЧ НАД~НЕОДНОРОДНЫМИ 
ИНФОРМАЦИОННЫМИ РЕСУРСАМИ}

\def\aut{Л.\,А.~Калиниченко$^1$, С.\,А.~Ступников$^2$, В.\,Н.~Захаров$^3$}

\titelr{\tit}{\aut}

\noindent
$^1$Институт проблем информатики Российской академии наук,
leonidk@synth.ipi.ac.ru\\
\noindent
$^2$Институт проблем информатики Российской академии наук,
ssa@ipi.ac.ru\\
\noindent
$^3$Институт проблем информатики Российской академии наук, vzakharov@ipiran.ru\\


\Abst{Рассмотрены актуальные проблемы в области решения задач над 
неоднородными распределенными информационными ресурсами. Изложены основные 
достижения технологии предметных посредников, предназначенной для заполнения 
увеличивающегося разрыва между пользователями (приложениями) и неоднородными ресурсами 
данных, знаний и сервисов. Рассмотрены также актуальные проблемы технологии семантической 
интеграции информации, включающие исследование движимого приложениями подхода к 
решению задач в среде предметных посредников; обеспечение поддержки исполняемых 
декларативных спецификаций приложений над посредниками; расширение применения 
ориентированных на знания средств на уровне посредников; применение технологии предметных 
посредников для баз данных, основанных на нетрадиционных моделях данных и мотивированных 
потребностями поддержки <<больших данных>>.}

\label{end\stat}

\KW{предметные посредники; неоднородные информационные ресурсы; решение 
научных задач; интеграция информации; движимый приложениями подход; языки правил; 
нетрадиционные модели данных}




\renewcommand{\figurename}{\protect\bf Рис.}
\renewcommand{\tablename}{\protect\bf Таблица}
\renewcommand{\bibname}{\protect\rmfamily Литература}