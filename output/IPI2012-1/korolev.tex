\def\stat{korolev}

\def\tit{СКОШЕННЫЕ РАСПРЕДЕЛЕНИЯ СТЬЮДЕНТА,
ДИСПЕРСИОННЫЕ ГАММА-РАСПРЕДЕЛЕНИЯ И~ИХ~ОБОБЩЕНИЯ КАК~АСИМПТОТИЧЕСКИЕ
АППРОКСИМАЦИИ$^*$}

\def\titkol{Скошенные распределения Стьюдента,
дисперсионные гамма-распределения и их обобщения} % как асимптотические аппроксимации}

\def\autkol{В.\,Ю.~Королев, И.\,А.~Соколов}
\def\aut{В.\,Ю.~Королев$^1$, И.\,А.~Соколов$^2$}

\titel{\tit}{\aut}{\autkol}{\titkol}

{\renewcommand{\thefootnote}{\fnsymbol{footnote}}\footnotetext[1]
{Работа выполнена при поддержке РФФИ (проекты
11-01-12026-офи-м, 11-07-00112 и 11-01-00515), а также Министерства
образования и науки РФ в рамках ФЦП <<Научные и
на\-уч\-но-пе\-да\-го\-ги\-че\-ские кадры инновационной России на 2009--2013~годы>>.}}


\renewcommand{\thefootnote}{\arabic{footnote}}
\footnotetext[1]{Московский государственный университет
им.\ М.\,В.~Ломоносова, факультет вычислительной математики и
кибернетики; Институт проблем информатики Российской академии наук,
vkorolev@comtv.ru}
\footnotetext[2]{Институт проблем
информатики Российской академии наук, isokolov@ipiran.ru}

\vspace*{-11pt}

\Abst{Показано, что cкошенные распределения Стьюдента и
(несимметричные) дисперсионные гам\-ма-рас\-пре\-де\-ле\-ния могут выступать
в качестве предельных в довольно простых предельных теоремах для
регулярных статистик, в частности в схеме случайного суммирования
случайных величин, и, следовательно, могут считаться
асимптотическими аппроксимациями для распределений многих
процессов, связанных с эволюцией сложных систем.}

\vspace*{-3pt}

\KW{скошенное распределение Стьюдента;
дисперсионное гам\-ма-рас\-пре\-де\-ле\-ние; предельная теорема; случайная
сумма; теорема переноса}

\vspace*{-2pt}

 \vskip 10pt plus 9pt minus 6pt

      \thispagestyle{headings}

      \begin{multicols}{2}
      
            \label{st\stat}

\section{Введение}

Скошенные распределения Стьюдента и дисперсионные
гам\-ма-рас\-пре\-де\-ле\-ния часто служат математическими моделями
статистических закономерностей, хорошо описывающими эффект наличия\linebreak
так называемых тяжелых или полутяжелых хвос-\linebreak тов. Такие модели очень
важны для адекватного описания статистических закономерностей
поведения различных характеристик сложных сис\-тем, эволюция которых
в значительной мере зависит от информационных потоков, к примеру
телекоммуникационных сетей или финансовых рынков. В~част\-ности, в
финансовой математике хорошо известны так называемые 
\textit{обобщенные гиперболические процессы} (GH-processes) и \textit{дисперсионные 
гам\-ма-про\-цес\-сы} (VG-processes).

Обобщенные гиперболические процессы~--- это процессы Леви (процессы
с независимыми ста\-ци\-о\-нар\-ны\-ми приращениями), одномерные
распределения которых имеют обобщенные гиперболические
распределения. 

Класс обобщенных гиперболических распределений был
описан О.\,Барн\-дорфф-Ниль\-се\-ном~\cite{BarndorffNielsen1977}.
Плотность обобщенного гиперболического распределения имеет вид:

\noindent
\begin{multline*}
 p_{\mathrm{GH}}(x;\lambda,\alpha,\beta,\delta,\mu) ={}\\
 {}=
\fr{\left(\alpha^2-\beta^2\right)^{\lambda/2}}
{\sqrt{2\pi}\alpha^{\lambda-1/2}\delta^{\lambda}
  K_{\lambda}(\delta\sqrt{\alpha^2-\beta^2})}
 \left(\delta^2+{}\right.\hspace*{2mm}
 \end{multline*}
 
 \noindent
 \begin{multline*}
\hspace*{-3.5mm}\left. {}+(x-\mu)^2\right)^{(\lambda-1/2)/2}
 K_{\lambda-{1}/{2}}\left(\alpha\sqrt{\delta^2+(x-\mu)^2}\right)
 \times{}\\
 {}\times \exp\left(\beta(x-\mu)\right)\,,\ \ x\in \mathbb{R}\,,
\end{multline*}
где $\mu\in\r$;
$$
\begin{array}{rccc}
  \delta \ge 0\,, & |\beta|<\alpha\,, & {\mbox{если}}\ &
\lambda>0\,;\\[6pt]
           \delta  > 0\,,  & |\beta|<\alpha\,, & {\mbox{если}}\ & \lambda=0\,;\\[6pt]
           \delta  > 0\,,  & |\beta|\le\alpha\,,&{\mbox{если}}\ &\lambda<0\,;
\end{array}
$$
$K_{\lambda}(x)$~--- функция Бесселя третьего рода порядка~$\lambda$. 
Обобщенное гиперболическое распределение имеет
<<полутяжелые>> хвосты в том смысле, что его плот\-ность
удовлетворяет асимптотическому соотношению
$$
p_{\mathrm{GH}}(x;\lambda,\alpha,\beta,\delta,\mu)\sim
|x|^{\lambda-1}\exp\{(\pm\alpha+\beta)x\}
$$ с точностью до
постоянного множителя при $x\hm\to\pm \infty$.

В работах~[2--7] установлено хорошее согласие\linebreak обобщенного
гиперболического распределения, \textit{гиперболического} ($\lambda\hm=1$)
и \textit{нормального обратного гауссовского} ($\lambda\hm=-1/2$)
распределений, входящих в\linebreak семейство обобщенных гиперболических
распределений вероятностей, с данными о ценах на датских и немецких
биржах, финансовыми индексами NYSE и DAX, обменными курсами валют и~т.\,д. 
С~точки зрения поведения <<хвостов>> эти распределения
занимают как бы промежуточное положение между устойчивыми
распределениями с индексом $\alpha\hm<2$ и нормальными (гауссовскими)
распределениями $\alpha\hm=2$: их <<хвосты>> убывают быстрее, чем у
устойчивых распределений ($\alpha\hm<2$), но медленнее нормальных.

Класс обобщенных гиперболических распределений весьма широк (см.,
например,~\cite{Korolev2011}). В~частности, заметим, что для
$\nu\hm>0$ при $\lambda\hm=-{\nu}/{2}$, $\alpha\hm=\beta\hm=\mu\hm=0$,
$\delta\hm=\sqrt{\nu}$ обобщенное гиперболическое распределение
совпадает с классическим распределением Стьюдента с~$\nu$
степенями свободы.

Известно несколько попыток описать несимметричные обобщения
распределения Стьюдента. Пожалуй, наиболее успешная из них~--- это
не\-симметричное (скошенное, skew) распределение Стьюдента,
описанное в работе~\cite{AasHaff2006} как некий\linebreak част\-ный 
 случай
обобщенного гиперболического распределения. 

В~статье~\cite{KimMcCulloch2007} 
распределению, предложенному в~\cite{AasHaff2006}, дано другое более удобно ин\-тер\-пре\-ти\-ру\-емое
определение как специальной сдвиг-мас\-штаб\-ной смеси нормальных
законов. Согласно~\cite{KimMcCulloch2007}, скошенным
распределением Стьюдента называется распределение с плот\-ностью
\begin{multline}
p_{\mathrm{SS}}(x;a,\sigma,\mu,\lambda)={}\\
\hspace*{-2mm}{}=\fr{1}{\sqrt{2\pi}\sigma}\int\limits_{0}^{\infty}\!
\exp\left\{-\fr{1}{2}\left(\fr{x-a u}{\sigma\sqrt{u}}\right)^2\right\}
\fr{h(u;\mu,\lambda)}{\sqrt{u}}\,du.\!\label{e1-kor}
\end{multline}
Здесь $a\in\r$; $\sigma\hm>0$; $\mu\hm>0$; $\lambda\hm>0$;
$h(x;\mu,\lambda)$~---плотность обратного гам\-ма-рас\-пре\-де\-ле\-ния, т.\,е.\ распределения
случайной величины $U^{-1}$, где $U$~--- случайная величина с
гам\-ма-рас\-пре\-де\-ле\-ни\-ем с параметром формы~$\mu$ и параметром
масштаба~$\lambda$:
\begin{equation}
h(x;\mu,\lambda)= \fr{\lambda^\mu}{\Gamma(\mu)}\,x^{-\mu-1}\exp
\left\{-\fr{\lambda}{x}\right\}\,.\label{e2-kor}
\end{equation}
 Напомним, что плот\-ность распределения
сам$\acute{\mbox{о}}$й случайной величины~$U$ имеет вид:
\begin{equation}
g(x;\mu,\lambda)=\fr{\lambda^{\mu}}{\Gamma(\mu)}x^{\mu-1}e^{-\lambda
x}\,,\enskip x\ge0\,.\label{e3-kor}
\end{equation}
Здесь и далее $\Gamma(\,\cdot\,)$~--- эйлерова гам\-ма-функ\-ция:
$$
\Gamma(z) = \int\limits_{0}^{\infty}e^{-y}y^{z-1}\,dy\,,\enskip z>0\,.
$$
Дисперсионные гамма-про\-цес\-сы, предложенные в 
работах~\cite{MadanSeneta1990, CarrMadanChang1998},~--- это процессы Леви,
одномерные распределения которых являются дисперсионными
гам\-ма-рас\-пре\-де\-ле\-ни\-ями. Плотность дисперсионного
гам\-ма-рас\-пре\-де\-ле\-ния имеет вид:
\begin{multline}
p_{\mathrm{VG}}(x;a,\sigma,\mu,\lambda)={}\\
\!\!{}=\fr{1}{\sqrt{2\pi}\sigma}
\int\limits_{0}^{\infty}\!\exp\left\{-\fr{1}{2}\left(\fr{x-au}{\sigma\sqrt{u}}\right)^2\right\}
\fr{g(u;\mu,\lambda)}{\sqrt{u}}\,du,\!\label{e4-kor}
\end{multline}
где $a\in\r$, $\sigma\hm>0$, $\mu\hm>0$, $\lambda\hm>0$, a
$g(x;\mu,\lambda)$~--- плот\-ность гам\-ма-рас\-пре\-де\-ле\-ния с параметрами
$\mu$ и~$\lambda$ (см.~(\ref{e3-kor})). Как отмечено в упомянутых работах,
подобные модели также демонстрируют высокую адекватность при
описании динамики цен финансовых активов.

Вместе с тем в прикладной теории вероятностей хорошо известен
принцип, согласно которому та или иная модель может считаться в
достаточной мере обоснованной только тогда, когда она является
\textit{асимптотической аппроксимацией}, т.\,е.\ когда существует
довольно простая предельная теорема, в которой рассматриваемая
модель выступает в качестве предельного распределения~\cite{GnedenkoKolmogorov1949}. 
В~книге~\cite{GnedenkoKorolev1996}
прослежена глубокая связь этого принципа с универсальным принципом
неубывания энтропии в замкнутых системах. Обе рассматриваемые в
данной статье модели имеют вид сдвиг-мас\-штаб\-ных смесей нормальных
законов. Как известно, нормальное распределение обладает
максимальной энтропией среди всех распределений, носителем которых
является вся числовая прямая и имеющих конечный второй момент.
Если бы моделируемая сложная система была информационно
изолирована от окружающей среды, то в соответствии с принципом
неубывания энтропии, который в теории вероятностей проявляется в
виде предельных теорем~\cite{GnedenkoKorolev1996}, наблюдаемые
статистические распределения ее характеристик были бы неотличимы
от нормального. Но поскольку любая математическая модель по своему
определению не может учесть все факторы, влияющие на состояние или
эволюцию моделируемой системы, то параметры этого нормального
закона изменяются в зависимости от состояния среды, внешней по
отношению к моделируемой сис\-те\-ме. Другими словами, эти параметры
являются случайными и изменяются под влиянием информационных
потоков между системой и внешней средой. Таким образом, во многих
ситуациях разумные модели статистических закономерностей изменения
параметров сложных сис\-тем должны иметь вид сдвиг-мас\-штаб\-ных смесей
нормальных законов, част\-ны\-ми случаями которых являются~(\ref{e1-kor}) и~(\ref{e4-kor}).

К сожалению, в первоисточниках упомянутые выше модели вводились
чисто умозрительно как распределения процесса броуновского
движения со случайным временем, в каждый момент име\-ющим
гам\-ма-рас\-пре\-де\-ле\-ние (дисперсионное гам\-ма-рас\-пре\-де\-ле\-ние) или
обратное гам\-ма-рас\-пре\-де\-ле\-ние (скошенное распределение Стьюдента).
<<Асимптотического>> обоснования этих моделей пока дано не было.

Частному случаю дисперсионных гам\-ма-рас\-пре\-де\-ле\-ний~---
несимметричному распределению Лапласа~--- и его практическому
применению посвящена работа~\cite{KotzKozubowskiPodgorski2001}
(см.\ так\-же~\cite{Korolev2011, KorolevBeningShorgin2011}.

В данной работе будет показано, что cкошенные распределения
Стьюдента и дисперсионные гам\-ма-рас\-пре\-де\-ле\-ния могут выступать в
качестве предельных в довольно простых предельных теоремах для
регулярных статистик, в частности в схеме случайного суммирования
случайных величин, и, следовательно, могут считаться
асимптотическими аппроксимациями для распределений многих
процессов, например, сходных с неоднородными случайными
блужданиями.

\section{Симметричный случай}

Сначала убедимся в том, что симметричные дисперсионные
гам\-ма-рас\-пре\-де\-ле\-ния и распределения Стьюдента можно считать
асимптотическими аппроксимациями в довольно простых предельных
схемах. Символ~$\Longrightarrow$ будет обозначать сходимость по
распределению. Рассмотрим традиционную для математической
статистики постановку задачи. Пусть\linebreak случайные величины
$N_1,N_2,\ldots ,X_1,X_2,\ldots$ определены на одном и том же
измеримом пространстве $(\Omega,{\cal A})$. Пусть на ${\cal A}$
задана вероятностная мера~${\sf P}$. Предположим, что при каждом
$n\hm\ge 1$\linebreak случайная величина~$N_n$ принимает только натуральные
значения и независима от последовательности $X_1,X_2,\ldots$ Пусть
$T_n\hm=T_n(X_1,\ldots ,X_n)$~--- некоторая статистика, т.\,е.\
измеримая функция от случайных величин $X_1,\ldots ,X_n$. Для
каждого $n\hm\ge1$ определим случайную величину $T_{N_n}$, положив
$T_{N_n}(\omega)\hm= T_{N_n(\omega)}\left(X_1(\omega),\ldots ,X_{N_n(\omega)}(\omega)\right)$
для каждого элементарного исхода $\omega\hm\in\Omega$.

Стандартную нормальную функцию распределения будем обозначать
$\Phi(x)$:
$$
\Phi(x)=\fr{1}{\sqrt{2\pi}}\int\limits_{-\infty}^{x}e^{-z^2/2}\,dz\,,\enskip
 x\in\r\,.
$$
Будем говорить, что статистика~$T_n$ асимптотически нормальна, если
существуют $\delta\hm>0$ и $t\hm\in\r$ такие, что
\begin{equation}
{\sf
P}\left(\delta\sqrt{n}\left (T_n-t\right)<x\right)\Longrightarrow\Phi(x)
\enskip (n\to\infty)\,.\label{e5-kor}
\end{equation}
Примеры асимптотически нормальных статистик хорошо известны.
Свойством асимптотической нормальности обладают, например,
выборочное среднее (при условии существования дисперсий),
центральные порядковые статистики или оценки максимального
правдоподобия (при достаточно общих условиях регулярности) и
многие другие статистики.

Наши дальнейшие рассуждения будут основаны на следующей лемме.

\smallskip

\noindent
\textbf{Лемма 1.} \textit{Пусть $\{d_n\}_{n\ge1}$~--- некоторая
неограниченно возрастающая последовательность положительных чисел.
Предположим, что $N_n\hm\to\infty$ по вероятности. Пусть статистика
$T_n$ асимптотически нормальна в смысле}~(\ref{e5-kor}). \textit{Для того чтобы
существовала такая функция распределения $F(x)$, что}
$$
{\sf P}\left(\delta\sqrt{d_n}\bigl(T_{N_n}-t\bigr)<x\right)
\Longrightarrow F(x)\ \ \ (n\to\infty),
$$
\textit{необходимо и достаточно, чтобы существовала функция распределения~$H(x)$, 
удовлетворяющая условиям:}
\begin{gather*}
H(0)=0\,,\enskip
F(x)=\int\limits_{0}^{\infty}\Phi\big(x\sqrt{y}\big)\,dH(y)\,,\enskip x\in\mathbb{R}\,;\\
{\sf P}(N_n<d_nx)\Longrightarrow H(x) \enskip
(n\to\infty).
\end{gather*}


\smallskip

\noindent
Д\,о\,к\,а\,з\,а\,т\,е\,л\,ь\,с\,т\,в\,о\,.\ Данная лемма по сути является
частным случаем теоремы~3 из~\cite{Korolev1995}, доказательство
которой, в свою очередь, основано на общих теоремах о сходимости
суперпозиций независимых случайных последовательностей~\cite{Korolev1994, Korolev1996}.

\smallskip

Пусть $F_{\mathrm{VG}}(x;a,\sigma,\mu,\lambda)$~--- функция распределения,
соответствующая плотности $p_{\mathrm{VG}}(x;a,\sigma,\mu,\lambda)$ (см.~(\ref{e4-kor})). 
Непосредственным следствием леммы~1 является следующее
утверждение.

\smallskip

\noindent
\textbf{Теорема 1.} \textit{Пусть $\{d_n\}_{n\ge1}$~--- некоторая
неограниченно возрастающая последовательность положительных чисел.
Предположим, что $N_n\hm\to\infty$ по вероятности. Пусть статистика
$T_n$ асимптотически нормальна в смысле}~(\ref{e5-kor}). \textit{Для того чтобы}
$$
{\sf P}\left(\delta\sqrt{d_n}\left(T_{N_n}-t\right)<x\right)
\Longrightarrow F_{\mathrm{VG}}(x;0,\delta,\mu,\lambda)
$$
\textit{при $n\to\infty$, необходимо и достаточно, чтобы}
\begin{equation}
{\sf P}(N_n<d_nx)\Longrightarrow H(x;\mu,\lambda)\quad
(n\to\infty)\,,\label{e6-kor}
\end{equation}
\textit{где $H(x;\mu,\lambda)$~--- функция обратного гам\-ма-рас\-пре\-де\-ле\-ния,
соответствующая плот\-ности}~(\ref{e2-kor}). 

\smallskip

Пусть $\gamma>0$, $P(x;\gamma)$~--- функция распределения
Стьюдента, соответствующая плот\-ности:
\begin{multline}
p(x;\gamma) =
\fr{\Gamma\left((\gamma+1)/2\right)}{\sqrt{\pi\gamma}\Gamma
(\gamma/2)}\left(1+\fr{x^2}{\gamma}\right)^{-(\gamma+1)/2}\,,\\
 -\infty<x<\infty\,.\label{e7-kor}
\end{multline}
Здесь $\gamma>0$~--- параметр, часто называемый \textit{чис\-лом
степеней свободы}. В~частности, при $\gamma\hm=1$ плот\-ность~(\ref{e7-kor}) имеет
вид:
$$
p(x;1)=\fr{1}{\pi(1+x^2)}\,,\enskip  -\infty<x<\infty\,,
$$
что соответствует распределению Коши. Несложно убедиться, что
$$
p(x;\gamma)=p_{\mathrm{SS}}\left(x;0,1,\fr{\gamma}{2},\fr{\gamma}{2}\right)\,.
$$
У распределения Стьюдента с параметром~$\gamma$ отсутствуют
моменты порядка $\delta\hm\ge\gamma$. Пусть $G(x;\mu,\lambda)$~---
функция гам\-ма-рас\-пре\-де\-ле\-ния, соответствующая плот\-ности
$g(x;\mu,\lambda)$ (см.~(\ref{e3-kor})).

С помощью леммы~1 в работе~\cite{BeningKorolev2004} доказано
следующее утверждение (см.\ так\-же~\cite{Korolev2011, KorolevBeningShorgin2011}).

\smallskip

\noindent
\textbf{Теорема 2.} \textit{Пусть $\gamma\hm>0$ произвольно и
$\{d_n\}_{n\ge1}$~--- некоторая неограниченно возрастающая
последовательность положительных чисел. Предположим, что
$N_n\hm\to\infty$ по вероятности. Пусть статистика~$T_n$
асимптотически нормальна в смысле}~(\ref{e5-kor}). \textit{Для того чтобы при}
$n\hm\to\infty$
$$
{\sf P}\left(\delta\sqrt{d_n}\left(T_{N_n}-t\right)<x\right) 
\Longrightarrow  P(x;\gamma)\,,
$$
\textit{необходимо и достаточно, чтобы}
\begin{equation}
{\sf P}(N_n<d_nx)\Longrightarrow G\left(x;\fr{\gamma}{2},\fr{\gamma}{2}\right)\quad
(n\to\infty)\,.\label{e8-kor}
\end{equation}

\smallskip

Также можно убедиться, что если условие асимптотической
нормальности~(\ref{e5-kor}), в котором используется <<растягивающая>>
нормировка, заменить аналогич\-ным условием со <<сжимающей>>
нормировкой, характерной для предельных теорем для сумм случайных
величин, то в соответствующих предельных теоремах условия~(\ref{e6-kor}) и~(\ref{e8-kor}) 
поменяются местами. Пусть в дополнение к приведенным выше
условиям случайные величины $X_1,X_2,\ldots$ одинаково
распределены, причем ${\sf E}X_1\hm=0$, $0\hm<{\sf
D}X_1\hm=\sigma^2<\infty$. Для $n\hm\ge1$ обозначим
$S_n\hm=X_1\hm+\cdots+X_n$. Два следующих утверждения являются частными
случаями общей теоремы о критериях сходимости случайных сумм~\cite{Korolev1994}.

\smallskip

\noindent
\textbf{Теорема 3.} \textit{Предположим, что $N_n\hm\to\infty$ по
вероятности. Для того чтобы}
$$
{\sf P}\left(\fr{S_{N_n}}{\sigma\sqrt{n}}<x\right)\Longrightarrow
F_{\mathrm{VG}}(x;0,\delta,\mu,\lambda)
$$
\textit{при $n\to\infty$, необходимо и достаточно, чтобы}
$$
{\sf P}(N_n<nx)\Longrightarrow G(x;\mu,\lambda)\quad  (n\to\infty)\,,
$$
\textit{где $G(x;\mu,\lambda)$~--- функция гам\-ма-рас\-пре\-де\-ле\-ния,
соответствующая плотности}~(\ref{e3-kor}). 

\smallskip

\noindent
\textbf{Теорема~4.} \textit{Предположим, что $N_n\hm\to\infty$ по
вероятности. Для того чтобы}
$$
{\sf P}\left(\fr{S_{N_n}}{\sigma\sqrt{n}}<x\right)\Longrightarrow
P(x;\gamma)
$$
\textit{при $n\to\infty$, необходимо и достаточно, чтобы}
\begin{equation}
{\sf P}(N_n<nx)\Longrightarrow H\left(x;\fr{\gamma}{2},\fr{\gamma}{2}\right)\quad
(n\to\infty)\,,\label{e9-kor}
\end{equation}
\textit{где $H(x;\mu,\lambda)$~--- функция обратного гамма-рас\-пре\-де\-ле\-ния,
соответствующая плотности}~(\ref{e2-kor}). 

\smallskip

Приведем пример ситуации, в которой выполнены условия~(\ref{e6-kor}) с
$\mu\hm=1$ и~(\ref{e9-kor}) c $\gamma\hm=2$. В~таком случае обратное
гамма-распределение становится обратным показательным
распределением~--- распределением случайной величины
$$
V=\fr{1}{U}\,,
$$
где случайная величина $U$ имеет стандартную показательную функцию
распределения $E(x)\hm=1\hm-e^{-x}$, $x\hm\ge0$. При этом

\noindent
\begin{multline*}
Q(x)\equiv {\sf P}(V<x)={\sf P}\left(\fr{1}{U}<x\right)={}\\
{}={\sf
P}\left(U>\fr{1}{x}\right)=e^{-1/x}\,,\enskip x\ge0\,.
\end{multline*}
Обратное показательное распределение~$Q(x)$ является частным
случаем распределения Фреше, хорошо известного в асимптотической
теории экстремальных порядковых статистик как предельное
распределение II~типа (см., например,~\cite{Gumbel1965}).

Приведем пример ситуации, в которой случайный объем выборки имеет
предельное распределение вида~$Q(x)$. Пусть $Y_1,Y_2,\ldots$~---
независимые одинаково распределенные случайные величины с одной и
той же непрерывной функцией распределения. Пусть $m$~---
произвольное натуральное число. Обозначим
$$
N(m)=\min\left\{n\ge1:\ \max\limits_{1\le j\le m}Y_j<\max\limits_{m+1\le k\le
m+n}Y_k\right\}\,.
$$
Случайная величина~$N(m)$ имеет смысл чис\-ла дополнительных
наблюдений, которые надо произвести, чтобы текущий (по~$m$
наблюдениям) максимум был перекрыт. Распределение случайной
величины~$N(m)$ было найдено С.~Уилксом, который в работе~\cite{Wilks1959} 
показал, что распределение величины~$N(m)$
является дискретным распределением Парето

\noindent
\begin{equation}
{\sf P}\left(N(m)\ge n\right)=\fr{m}{m+n}\,,\enskip  n\ge1\label{e10-kor}
\end{equation}
(см.\ так\-же~\cite[с.~85]{Nevzorov2000}).

\pagebreak

Пусть теперь $N^{(1)}(m),N^{(2)}(m),\ldots$~--- независимые
случайные величины с одним и тем же распределением~(\ref{e10-kor}). Целая
часть чис\-ла~$a$ будет обозначаться~$[a]$. Так как при любом
фиксированном $x\hm>0$
\begin{multline*}
1-\fr{m}{kx(1+(m-1)/(kx))}\le 1-\fr{m}{m-1+kx}\le{}\\
{}\le
1-\fr{m}{m+[kx]}\le  1-\fr{m}{m+kx}\le{}\\
{}\le
1-\fr{m}{kx(1+m/(kx))}\,,
\end{multline*}
то для любого $x>0$
\begin{multline*}
\lim_{k\to\infty}{\sf P}\left(\fr{1}{k}\,\max\limits_{1\le j\le k}N^{(j)}(m)<x\right)= {}\\
{}=
\lim_{k\to\infty}{\sf P}\left(\max_{1\le j\le
k}N^{(j)}(m)<kx\right)={}\\
{}=
\lim\limits_{k\to\infty}\left(1-\fr{m}{m+[kx]}\right)^k=
\lim\limits_{k\to\infty}\left(1-\fr{m}{kx}\right)^k={}\\
{}=e^{-m/x}\,.
\end{multline*}
Поэтому, если положить
$$
N_n=\max\limits_{1\le j\le n}N^{(j)}(m)
$$
при $m=1$, то теоремы~1 и~4 с $d_n\hm=n$ дают иллюстрацию того, как
вместо ожидаемого в соответствии с утверждениями классической
асимптотической статистики нормального распределения при замене
объема выборки случайной величиной в качестве предельного
распределения регулярных статистик возникают соответственно
распределение Лапласа или распределение Стьюдента. При этом
изменение значения параметра~$m$ влечет изменение параметра
масштаба (дисперсии) итогового распределения Лапласа в теореме~1.

\section{Несимметричный случай}

В этом разделе будут приведены просто формулируемые предельные
теоремы для сумм со случайным числом слагаемых, в которых в
качестве предель\-ных возникают несимметричные дисперсионные
гам\-ма-рас\-пре\-де\-ле\-ния или скошенные распределения Стьюдента.

Пусть $\{X_{n,j}\}_{j\ge1}$, $n=1,2,\ldots ,$~--- последовательность
серий независимых и одинаково в каждой серии распределенных
случайных величин, а~$N_n$, $n\hm=1,2,\ldots$,~--- положительные
целочисленные случайные величины такие, что при каждом~$n$
случайная величина~$N_n$ независима от по\-сле\-до\-ва\-тель\-ности
$\{X_{n,j}\}_{j\ge1}$. Для натуральных~$k$ обозначим
$$
S_{n,k}=X_{n,1}+\cdots+X_{n,k}\,.
$$
%Для определенности будем считать, что все функции распределения, 
%о~которых пойдет речь ниже, непрерывны справа. 
Все теоремы,
фор\-му\-ли\-ру\-емые ниже, по сути являются частными случаями следующего
утверждения, известного как \textit{теорема переноса}~\cite{GnedenkoFahim1969}.

\smallskip

\noindent
\textbf{Лемма~2.} \textit{Предположим, что существуют неограниченно
возрастающая последовательность натуральных чисел
$\{m_n\}_{n\ge1}$ и функции распределения~$H(x)$ и~$A(x)$ такие,
что}
\begin{align*}
{\sf P}(S_{n,m_n}<x)&\Longrightarrow H(x)\quad &  (n&\to\infty)\,;
\\
{\sf P}(N_n<m_nx)&\Longrightarrow A(x)\quad  &(n&\to\infty)\,.
\end{align*}
\textit{Тогда существует функция распределения $F(x)$ такая, что}
$$
{\sf P}(S_{n,N_n}<x)\Longrightarrow F(x)\quad  (n\to\infty)\,.
$$
\textit{При этом функция распределения~$F(x)$ соответствует
характеристической функции}
$$
f(t)=\int\limits_{0}^{\infty}h^u(t)\,dA(u),\ \ t\in\r,
$$
\textit{где $h(t)$~--- характеристическая функция, соответствующая функции
рас\-пре\-де\-ле\-ния}~$H(x)$.

\smallskip

Доказательство леммы~2 можно найти, например, в книге~\cite{KorolevBeningShorgin2011} 
(см.\ теорему~2.9.1 там).

Пусть $\Phi(x)$~--- стандартная нормальная функция распределения.

\smallskip

\noindent
\textbf{Следствие 1.} \textit{Если существуют числа $a\hm\in\r$,
$\sigma^2\hm>0$, $m_n\hm\ge1$ и функция распределения~$A(x)$ такие, что
при $n\to\infty$}
\begin{align}
{\sf P}\left(S_{n,m_n}<x\right)&\longrightarrow
\Phi\left(\fr{x-a}{\sigma}\right)\,;\label{e11-kor}
\\
{\sf P}(N_n<m_nx)&\Longrightarrow A(x)\,,\label{e12-kor}
\end{align}
\textit{то}
$$
{\sf
P}\left(S_{n,N_n}<x\right)\longrightarrow\int\limits_{0}^{\infty}\Phi
\left(\fr{x-au}{\sigma\sqrt{u}}\right)\,dG(u)\,.
$$


\smallskip

\noindent
\textbf{Теорема~5}. \textit{Предположим, что существуют числа $a \hm \in
\mathbb{R}$, $\sigma^{2}\hm \in (0, \infty )$, $\mu\hm\in(0,\infty)$,
$\lambda \hm\in (0, \infty )$ и последовательность натуральных чисел
$\left\{m_n \right\}_{n \geqslant 1}$ такие, что выполнены условия}~(\ref{e11-kor}) 
\textit{и}~(\ref{e12-kor}) \textit{с $A(x)\hm=G(x;\mu,\lambda)$. Тогда}
\begin{equation}
{\sf P}(S_{n, N_{n}}<x)\Longrightarrow
F_{\mathrm{VG}}(x;a,\sigma,\mu,\lambda) \enskip (n\rightarrow \infty)\,,\label{e13-kor}
\end{equation}
\textit{причем предельная случайная величина~$Z$ с дисперсионным
гам\-ма-рас\-пре\-де\-ле\-ни\-ем $F_{\mathrm{VG}}(x;a,\sigma,\mu,\lambda)$ может быть
представлена в виде разности независимых случайных величин, имеющих
гам\-ма-плот\-ности с одинаковыми параметрами формы и разными
масштабными параметрами.}

\medskip

\noindent
Д\,о\,к\,а\,з\,а\,т\,е\,л\,ь\,с\,т\,в\,о\,.\ По следствию~1 и определению~дисперсионного 
гам\-ма-рас\-пре\-де\-ле\-ния~(\ref{e4-kor}) из условий~(\ref{e11-kor}) и~(\ref{e12-kor})
вытекает~(\ref{e13-kor}). Остается убедиться, что предельная случайная
величина~$Z$ может быть представлена в виде разности независимых
случайных величин, имеющих гам\-ма-рас\-пре\-де\-ле\-ния с одинаковыми
параметрами формы и разными масштабными параметрами.

По лемме~2 функции распределения $F_{\mathrm{VG}}(x;a,\sigma,\mu,\lambda)$
случайной величины~$Z$ соответствует характеристическая функция
\begin{multline}
{\sf E}e^{itZ} ={}\\
{}= \int\limits_{0}^{\infty}\exp\left \{ z\left(ita -
\fr{\sigma^{2}t^2}{2}\right) \right \}\fr{\lambda^{\mu}}{\Gamma
(\mu)}e^{-\lambda z }z^{\mu-1}\,dz ={}\\[2pt]
{}=\fr{\lambda^{\mu}}{\Gamma (\mu)} \int\limits_{0}^{\infty}\exp
\left \{ z\left (ita   - \fr{\sigma^{2}t^2}{2} - \lambda\right) \right\} 
z^{\mu-1}\,dz ={}\\[2pt]
{}=\left (\fr{\lambda}{\lambda - ita  +
\sigma^{2}t^{2}/2}\right )^{\mu}\,,\enskip
t\in\r\,.\label{e14-kor}
\end{multline}
Введем перепараметризацию
\begin{equation}
\begin{cases} 
w-v = \displaystyle\fr{a }{\lambda}\,;\\[2pt]
v w = \displaystyle\fr{\sigma^{2}}{2\lambda}\,.\end{cases} 
\label{e15-kor}
\end{equation}
Из первого уравнения~(\ref{e15-kor}) получим
$$
w = v + \fr{a }{\lambda}\,.
$$
Из второго получим
$$
v\left(v + \fr{a }{\lambda}\right) = \fr{\sigma^{2}}{2\lambda}\,,
$$
или
$$
v^{2} + \fr{a }{\lambda}v - \fr{\sigma^{2}}{2\lambda} = 0\,.
$$
Система~(\ref{e15-kor}) имеет два решения относительно~$v$:
\begin{align*}
v_{1} &= -\fr {a}{2\lambda} + \fr{1}{2}\,
\sqrt{\fr{a^{2}}{\lambda^{2}}+\fr{2\sigma^{2}}{\lambda}}\,;
\\[2pt]
v_{2} &= -\fr{a} {2\lambda} - \fr{1}{2}\,
\sqrt{\fr{a^{2}}{\lambda^{2}}+\fr{2\sigma^{2}}{\lambda}}\,.
\end{align*}
Одно из них~--- $v_1$~--- положительно. При этом также положительно
значение
$$
w_{1} = v_{1} + \fr{a }{\lambda}\,.
$$
В дальнейшем будем использовать параметры
\begin{align*}
\lambda_1 &= \fr{1}{v_{1}}  = \left(\fr{1}{2}\,\sqrt{\fr{a^{2}}
{\lambda^{2}}+\fr{2\sigma^{2}}{\lambda}} -\fr{a}
{2\lambda}\right)^{-1}\,;
\\
\lambda_2 &= \fr{1}{w _{1}}  =
\left(\fr{1}{2}\,\sqrt{\fr{a^{2}}{\lambda^{2}}+\fr{2\sigma^{2}}{\lambda}} +\fr{a}
{2\lambda}\right)^{-1}\,.
\end{align*}
В этих обозначениях характеристическая функция~(\ref{e14-kor}) принимает вид:
\begin{multline*}
{\sf E}e^{itZ}=\left (\fr{\lambda}{\lambda - ita  +
\sigma^{2}t^{2}/2}\right )^{\mu} = {}\\
{}=\left(
\fr{\lambda_1\lambda_2}{\lambda_1\lambda_2 -(\lambda_2 -
\lambda_1)it +t^{2}/2}\right )^{\mu} =
{}\\
{}
=\left(
\fr{\lambda_1\lambda_2}{(\lambda_1-it)(\lambda_2+it)}\right)^{\mu} ={}\\
{}=
 \left (\fr{\lambda_1}{\lambda_1-it}\right )^{\mu} \left( 
\fr{\lambda_2}{\lambda_2 + it}\right )^{\mu} = {}\\
{}=
{\sf E}\exp\left\{it\left[U(\mu, \lambda_1) - U(\mu,
\lambda_2)\right]\right\}\,,
\end{multline*}
где $U(\mu,\lambda_i)$, $i\hm=1,2,$~--- независимые случайные
величины, имеющие соответственно гам\-ма-рас\-пре\-де\-ле\-ния с параметром
формы~$\mu$ и параметрами масштаба~$\lambda_i$, $i\hm=1,2$.

Таким образом, характеристическая функция~(\ref{e8-kor}) случайной величины~$Z$ 
является характеристической функцией разности независимых\linebreak
случайных величин, имеющих дисперсионное гам\-ма-рас\-пре\-де\-ле\-ние с
одинаковыми параметрами масштаба и различными параметрами формы.
Теорема доказана.

\smallskip

Теорема~5 обобщает теорему~12.7.3 из~\cite{KorolevBeningShorgin2011}, 
устанавливающую сходимость распределений случайных сумм к несимметричному распределению\linebreak
Лапласа, являющемуся дисперсионным гам\-ма-рас\-пре\-де\-ле\-ни\-ем с
параметром $\mu\hm=1$.

Из следствия~1 и определения скошенного распределения Стьюдента~(\ref{e1-kor}) 
вытекает следующий результат.


\smallskip

\noindent
\textbf{Теорема~6}. \textit{Предположим, что существуют числа $a  \hm\in
\mathbb{R}$, $\sigma^{2} \hm\in (0, \infty )$, $\mu\hm\in(0,\infty)$,
$\lambda \hm\in (0, \infty )$ и последовательность натуральных чисел
$\left\{m_n \right\}_{n \geqslant 1}$ такие, что выполнены условия}~(\ref{e11-kor}) 
\textit{и}~(\ref{e12-kor}) \textit{с $A(x)\hm=H(x;\mu,\lambda)$, где $H(x;\mu,\lambda)$~---
функция обратного гам\-ма-рас\-пре\-де\-ле\-ния с параметрами~$\mu$,
$\lambda$, соответствующая плот\-ности}~(\ref{e2-kor}). \textit{Тогда}
$$
{\sf P}(S_{n, N_{n}}<x)\Longrightarrow
P_{\mathrm{SS}}(x;a,\sigma,\mu,\lambda) \quad  (n\rightarrow \infty)\,,
$$
\textit{где $P_{\mathrm{SS}}(x;a,\sigma,\mu,\lambda)$~--- функция скошенного
распределения Стьюдента, соответствующая плот\-ности}
$p_{\mathrm{SS}}(x;a,\sigma,\mu,\lambda)$ (\textit{см}.~(\ref{e1-kor})).

\section{Заключение}

Гамма-распределение и обратное гамма-рас\-пре\-де\-ле\-ние являются
частными представителями класса обобщенных гам\-ма-рас\-пре\-де\-ле\-ний,
важная роль которых в моделировании и анализе стохастической
структуры информационных потоков описана в книге~\cite{KorolevShorgin2011}. 
Обобщенные гам\-ма-рас\-пре\-де\-ле\-ния
(ОГ-рас\-пре\-де\-ле\-ния) были впервые описаны как единое семейство в
1962~г.\ в работе~\cite{Stacy1962} в качестве семейства
вероятностных моделей, вклю\-ча\-юще\-го в себя одновременно
гам\-ма-рас\-пре\-де\-ле\-ния и распределения Вейбулла.

Обобщенным гамма-рас\-пре\-де\-ле\-ни\-ем называется распределение,
определяемое плот\-ностью вероятностей вида
\begin{multline}
f(x;\nu,\kappa,\delta)={}\\
\hspace*{-4.5mm}{}=
\begin{cases}{
\fr{|\nu|}{\delta\Gamma(\kappa)}\left(\fr{x}{\delta}\right)^{\kappa\nu-1}\!\!\exp
\left\{-\left(\fr{x}{\delta}\right)^{\nu}\right\}}\,,
& x\ge0\,;\\
0\,, & x<0\,,
\end{cases}\label{e16-kor}
\end{multline}
с параметрами $\nu\hm\in\mathbb{R},\,\kappa,\,\delta\hm\in{\mathbb R}^+$, 
отвечающими соответственно за \textit{степень, форму и
масштаб} (здесь $\Gamma(\kappa)$~--- эйлерова гамма-функ\-ция:
$
\Gamma(\kappa)\hm=\int\limits_{0}^{\infty}x^{\kappa-1}e^{-x}\,dx
$).

Семейство ОГ-рас\-пре\-де\-ле\-ний включает в
себя практически все наиболее популярные абсолютно непрерывные
распределения. В~част\-ности, семейство ОГ-рас\-пре\-де\-ле\-ний содержит
следующие распределения.
\begin{enumerate}[1.]
\item Гамма-распределение ($\nu\hm=1)$:
\begin{multline*}
f(x;\kappa,\theta)=\fr{1}{\Gamma(\kappa)}\theta^{\kappa}x^{\kappa-1}e^{-\theta x}\,,\\
x\ge0\,,\enskip \kappa>0\,,\enskip \theta>0\,.
\end{multline*}
\begin{enumerate}[{1}.1]
\item Показательное (экспоненциальное) распределение ($\nu\hm=1,\, \kappa\hm=1)$:
$$
f(x;\theta)=\theta e^{-\theta x}\,,\ \ \ x\ge0\,,\ \theta>0\,.
$$
\item Распределение Эрланга ($\nu=1,\, \kappa\in\mathbb{N})$:
\begin{multline*}
f(x;\kappa,\theta)=\fr{1}{\Gamma(\kappa)}\theta^{\kappa}x^{\kappa-1}e^{-\theta
x}\,,\\ 
x\ge0\,,\ \kappa>0\,,\ \theta>0\,.
\end{multline*}

\item Распределение хи-квад\-рат ($\nu\hm=1,\, \delta\hm=2)$:
\begin{multline*}
f(x;n)=\fr{1}{2\Gamma({n}/2)}\left(\fr{x}{2}\right)^{n/2-1}e^{-x/2}\,,\\
x\ge0\,,\ n\in\mathbb{N}\,.
\end{multline*}
\end{enumerate}
\item Распределение Накагами ($\nu\hm=2$):
\begin{multline*}
f(x;\mu,\lambda)=\fr{2(\lambda\mu)^{\mu}}{\Gamma(\mu)}x^{2\mu-1}e^{-\lambda\mu
x^2}\,,\\
 x\ge0\,, \ \mu>0\,,\ \lambda>0\,.
\end{multline*}
\begin{enumerate}[{2.}1]
\item Полунормальное распределение (распределение максимума
винеровского процесса на отрезке $[0,1]$)
($\nu\hm=2,\,\kappa\hm=1/2$):
\begin{equation*}
f(x;\delta)=\sqrt{\fr{2}{\pi\delta}}\exp\left\{-\fr{x^2}{2\delta^2}\right\},\
x\ge0, \ \delta>0.
\end{equation*}
\item Распределение Рэлея ($\nu\hm=2,\,\kappa\hm=1$):
\begin{equation*}
f(x;\delta)=\fr{x}{\delta^2}\exp\left\{-\fr{x^2}{2\delta^2}\right\}\,,\ \ 
x\ge0\,, \ \delta>0\,.
\end{equation*}

\item Хи-распределение ($\nu\hm=2,\,\delta\hm=\sqrt{2}$):
\begin{multline*}
\hspace*{-7pt}f(x;n)=\fr{1}{2^{n/2-1}\Gamma({n}/{2})}\,x^{n-1}\exp\left\{-\fr{x^2}{2}\right\}\,,\\
x\ge0\,, \ n\in\mathbb{N}\,.
\end{multline*}
\item Распределение Максвелла~--- распределение модуля скорости движения
молекул в разреженном газе ($\nu=\hm2,\,\kappa\hm=3/2$):
\begin{multline*}
f(x;\delta)=\sqrt{\fr{2}{\pi}}\,\fr{x^2}{\delta^3}\,\exp\left\{-\fr{x^2}{2\delta^2}\right\}\,,\\
x\ge0\,, \ \delta>0\,.
\end{multline*}
\end{enumerate}
\item Распределение Вей\-бул\-ла--Гне\-ден\-ко ($\kappa\hm=1$):
\begin{multline*}
f(x;\eta,\mu)=\fr{\eta x^{\eta-1}}{\mu^{\eta}}\,
\exp\left\{-\left(\fr{x}{\mu}\right)^{\eta}\right\}\,,\\
x\ge0\,, \ \eta>0\,,\ \mu>0\,.
\end{multline*}

\item Обратное гамма-распределение ($\nu\hm=-1$):
\begin{multline*}
f(x;\mu,\lambda)=\fr{1}{\mu\lambda\Gamma(\lambda)}\left(\fr{\mu\lambda}{x}\right)^{\lambda+1}
\exp\left\{-\fr{\mu\lambda}{x}\right\}\,,\\
 x\ge0\,, \ \lambda>0\,,\ \mu>0\,.
\end{multline*}
\begin{enumerate}[{4}.1]
\item Распределение Леви ($\nu\hm=-1,\, \kappa\hm=1/2$):
\begin{multline*}
f(x;\mu)=\sqrt{\fr{\mu}{2\pi}}\,\fr{1}{x^{3/2}}\exp\left\{-\fr{\mu}{2x}\right\}\,,\\
 x\ge0\,, \ \mu>0\,.
\end{multline*}
\end{enumerate}

\item Логнормальное распределение ($\kappa\hm\to\infty$):
\begin{multline*}
f(x;\mu,\delta)=\fr{1}{\delta x\sqrt{2\pi}}\exp\left\{-\fr{(\log x-\mu)^2}{2\delta^2}\right\}\,,\\
x\ge0\,, \ \mu\in\mathbb{R}\,,\ \delta>0\,.
\end{multline*}
\end{enumerate}

Широкая применимость ОГ-рас\-пре\-де\-ле\-ний обусловлена возможностью их
использования в качестве адекватных асимптотических аппроксимаций,
поскольку практически все они выступают в качестве предельных в
различных предельных теоремах теории вероятностей, а именно:
\begin{itemize}
\item показательное распределение выступает в качестве предельного
как в схеме максимума (минимума) (см., например,~\cite{Gumbel1965}), 
так и в схеме геометрического суммирования,
описывая распределение времени восстановления в прореженных
процессах восстановления, высту\-па\-ющих моделями потоков редких
событий (см., например,~\cite{Kalashnikov1997});
\item гамма-распределение является безгранично делимым и потому
выступает в качестве предельного для распределений сумм
независимых равномерно предельно малых случайных величин; при этом
распределение Эрланга возникает как\linebreak
допредельное распределение
суммы независимых экспоненциально распределенных случайных
величин, что в терминах случайной интенсивности может означать,
что если \mbox{случайная} интенсивность потока поступления запросов
имеет гам\-ма-рас\-пре\-де\-ле\-ние со значимым параметром формы, то при
обработке этих запросов в основном задействованы механизмы
последовательной обработки информации;
\item распределение Вей\-бул\-ла--Гне\-ден\-ко принадлежит к так
называемому первому типу предельных распределений экстремальных
порядковых статистик (минимума или максимума) (см.,\linebreak
например,~\cite{Gumbel1965}), 
что в терминах случайной интенсивности может
означать, что если случайная интенсивность потока поступления
запросов имеет распределение Вей\-бул\-ла--Гне\-ден\-ко со значимым
параметром степени, то при обработке этих запросов в основном
задействованы механизмы параллельной обработки информации;
\item полунормальное распределение (распределение модуля
стандартной нормальной случайной величины) возникает как
предельное для максимальных частичных сумм независимых случайных
величин (см., например,~\cite{KorolevSokolov2008});
\item распределение Леви принадлежит к классу устойчивых законов и
потому является предельным для сумм независимых одинаково
распределенных случайных величин; оно также является
распределением времени достижения стандартным винеровским
процессом (процессом броуновского движения) фиксированного уровня;
\item логнормальное распределение выступает в качестве предельного
для распределения размера частиц при дроблении (см., например,~\cite{Korolev2009}).
\end{itemize}

Эти свойства ОГ-рас\-пре\-де\-ле\-ний обосновывают, в част\-ности,
целесообразность моделирования с их помощью распределения
случайной интен\-сив\-ности потока запросов в информационных сис\-те\-мах.
Это семейство также широко используется в других прикладных
задачах в самых разных областях (см., например,~\cite{KorolevShorgin2011}).

Рассмотренные выше четырехпараметрические семейства скошенных
распределений Стьюдента и дисперсионных гамма-распределений
являются подклассами пятипараметрического семейства распределений
\begin{multline}
W(x;a,\sigma,\nu,\kappa,\delta)={}\\
{}=
\int\limits_{0}^{\infty}\Phi\left(\fr{x-au}{\sigma\sqrt{u}}\right)
f(u;\nu,\kappa,\delta)\,du\,,\label{e15-1-kor}
\end{multline}
где $f(u;\nu,\kappa,\delta)$~--- плотность ОГ-рас\-пре\-де\-ле\-ния~(\ref{e16-kor}).
Распределения вида~(\ref{e15-1-kor}) назовем \textit{обобщенными дисперсионными
гам\-ма-рас\-пре\-де\-ле\-ни\-ями}.

Задача поиска универсальной модели статистических закономерностей
во многих областях, в частности в финансовой математике или в
физике плазмы, подобна задаче отыскания <<философского камня>> в
алхимии и поэтому не имеет точного решения. Однако, основываясь на
вышепере\-чис\-ленных аналитических и асимптотических свойствах
представителей семейства ОГ-рас\-пре\-де\-ле\-ний и следствии~1 как
тео\-ре\-ти\-ко-ве\-ро\-ят\-но\-ст\-ной формализации принципа неубывания
не\-опре\-де\-лен\-ности в сложных сис\-те\-мах, можно утверждать, что
семейство обобщенных дисперсионных гам\-ма-рас\-пре\-де\-ле\-ний является
\textit{практически} универсальным для многих задач.

{\small\frenchspacing
{%\baselineskip=10.8pt
\addcontentsline{toc}{section}{Литература}
\begin{thebibliography}{99}


\bibitem{BarndorffNielsen1977} 
\Au{Barndorff-Nielsen O.\,E.} 
Exponentially decreasing distributions for
the logarithm of particle size~// Proc. R. Soc. A, 1977. Vol.~353.
P.~401--419.

\bibitem{EberleinKeller1995} 
\Au{Eberlein E., Keller~U.} 
Hyperbolic distributions in finance~// Bernoulli, 1995. Vol.~1. No.\,3.
P.~281--299.

\bibitem{Prause1997} 
\Au{Prause~K.} Modeling financial data using generalized hyperbolic
distri\-butions.~--- Freiburg: Universit$\ddot{\mbox{a}}$t Freiburg, Institut
f$\ddot{\mbox{u}}$r Mathematische Stochastic, 1997. Preprint No.\,48.

\bibitem{EberleinKellerPrause1998} 
\Au{Eberlein E., Keller~U., Prause~K.}
New insights into smile, mispricing and value at risk: The
hyperbolic model~// J.~Business, 1998. Vol.~71. P.~371--405.

\bibitem{BarndorffNielsen1998} 
\Au{Barndorff-Nielsen O.\,E.} Processes of normal inverse Gaussian type~//
Finance Stochastics, 1998. Vol.~2. P.~41--18.

\bibitem{EberleinPrause1998} 
\Au{Eberlein E., Prause~K.} 
The generalized hyperbolic model: Financial derivatives and risk
measures.~--- Freiburg: Universit$\ddot{\mbox{a}}$t Freiburg, Institut f$\ddot{\mbox{u}}$r
Mathematische Stochastic, 1998. Preprint No.\,56.

\bibitem{Eberlein1999} 
\Au{Eberlein E.} 
Application of generalized hyperbolic L$\grave{\mbox{e}}$vy
motions to finance.~--- Freiburg: Universit$\ddot{\mbox{a}}$t Freiburg, Institut
f$\ddot{\mbox{u}}$r Mathematische Stochastic, 1999. Preprint No.\,64.

\bibitem{Korolev2011} 
\Au{Королев В.\,Ю.} Ве\-ро\-ят\-но\-ст\-но-ста\-ти\-сти\-че\-ские методы
декомпозиции волатильности хаотических процессов.~--- М.: Изд-во
МГУ, 2011. 510~с.

\bibitem{AasHaff2006} 
\Au{Aas K., Haff I.\,H.} 
The generalized
hyperbolic skew Student's $t$-distribution~// J.~Financial
Econometrics, 2006. Vol.~4. No.\,2. P.~275--309.

\bibitem{KimMcCulloch2007} 
\Au{Kim Y., McCulloch  J.\,H.} The skew-student distribution with application to U.S.\ stock
market returns and the equity premium.~--- Columbus: Department of
Economics, Ohio State University, 2007. Preprint.

\bibitem{MadanSeneta1990} 
\Au{Madan D.\,B., Seneta~E.} The variance gamma (V.G.) model for
share market return~// J.~Business, 1990. Vol.~63.
P.~511--524.

\bibitem{CarrMadanChang1998} 
\Au{Carr P.\,P., Madan D.\,B., Chang~E.\,C.} 
The Variance Gamma process and option pricing~//
European Finance Rev., 1998. Vol.~2. P.~79--105.

\bibitem{GnedenkoKolmogorov1949} 
\Au{Гнеденко Б.\,В., Колмогоpов А.\,Н.} Пpедельные
pаспpеделения для сумм независимых случайных величин.~--- М.--Л.:
ГИТТЛ, 1949.

\bibitem{GnedenkoKorolev1996} 
\Au{Gnedenko B.\,V., Korolev V.\,Yu.} Random summation:
Limit theorems and applications.~--- Boca Raton: CRC Press, 1996.

\bibitem{KotzKozubowskiPodgorski2001} 
\Au{Kotz S., Kozubowski T.\,J., Podgorski~K.} The Laplace distribution and generalizations: A
revisit with applications to communications, economics,
engineering and finance.~--- Boston: Birkhauser, 2001.

\bibitem{KorolevBeningShorgin2011} 
\Au{Королев В.\,Ю., Бенинг~В.\,Е., Шоргин~С.\,Я.} 
Математические основы теории риска.~--- 2-е изд.,
перераб. и доп.~--- М.: Физматлит, 2011. 620~с.

\bibitem{Korolev1995} 
\Au{Королев В.\,Ю.} Сходимость случайных последовательностей с независимыми
случайными индексами. II~// Теория вероятностей и ее применения,
1995. Т.~40. Вып.~4. С.~907--910.

\bibitem{Korolev1994} 
\Au{Королев В.\,Ю.} Сходимость случайных
последовательностей с независимыми случайными индексами. I~// Теория
вероятностей и ее применения, 1994. Т.~39. Вып.~2. С.~313--333.

\bibitem{Korolev1996} 
\Au{Korolev V.\,Yu.} A general
theorem on the limit behavior of superpositions of independent
random processes with applications to Cox processes~// 
J.~Math. Sci., 1996. Vol.~81. No.\,5. P.~2951--2956.

\bibitem{BeningKorolev2004} 
\Au{Бенинг В.\,Е., Королев В.\,Ю.} Об использовании распределения Стьюдента в
задачах теории вероятностей и математической статистики~// Теория
вероятностей и ее применения, 2004. Т.~49. Вып.~3. С.~417--435.

\bibitem{Gumbel1965} 
\Au{Гумбель Э.} Статистика экстремальных
значений.~--- М.: Мир, 1965.

\bibitem{Wilks1959} 
\Au{Wilks S.\,S.} Recurrence of extreme observations~//
J.~Amer. Math. Soc., 1959. Vol.~1. No.\,1. P.~106--112.

\bibitem{Nevzorov2000} 
\Au{Невзоров В.\,Б.} Рекорды. Математическая теория.~--- М.: Фазис, 2000.

\bibitem{GnedenkoFahim1969}
\Au{Гнеденко Б.\,В., Фахим~Х.} Об~одной
теореме переноса~// Докл. АН СССР, 1969. Т.~187. №\,1. С.~15--17.

\bibitem{KorolevShorgin2011} 
\Au{Королев В.\,Ю., Шоргин С.\,Я.}
Математические методы анализа стохастической структуры
информационных потоков.~--- М.: ИПИ РАН, 2011. 130~с.

\bibitem{Stacy1962} 
\Au{Stacy  E.\,W.} A generalization of the gamma
distribution~// Annals Math. Statistics, 1962. Vol.~33. P.~1187--1192.

\bibitem{Kalashnikov1997} 
\Au{Kalashnikov V.\,V.} Geometric sums: Bounds for rare events with
applications.~--- Dordrecht: Kluwer Academic Publs., 1997.

\bibitem{KorolevSokolov2008} 
\Au{Королев В.\,Ю., Соколов И.\,А.} Математические модели
неоднородных потоков экстремальных событий.~--- М.: ТОРУС ПРЕСС, 2008.

\label{end\stat}

\bibitem{Korolev2009} 
\Au{Королев В.\,Ю.} О~распределении размеров частиц при
дроблении~// Информатика и её применения, 2009. Т.~3. Вып.~3. С.~60--68.

 \end{thebibliography}
}
}


\end{multicols}       