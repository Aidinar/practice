\def\ed{\mathop{1}}

\def\stat{sinits}

\def\tit{МАТЕМАТИЧЕСКОЕ ОБЕСПЕЧЕНИЕ ДЛЯ АНАЛИЗА
НЕЛИНЕЙНЫХ МНОГОКАНАЛЬНЫХ КРУГОВЫХ СТОХАСТИЧЕСКИХ
СИСТЕМ, ОСНОВАННОЕ НА~ПАРАМЕТРИЗАЦИИ РАСПРЕДЕЛЕНИЙ$^*$}

\def\titkol{Математическое обеспечение для анализа
нелинейных многоканальных круговых стохастических
систем} %, основанное на параметризации распределений}

\def\autkol{И.\,Н. Синицын}
\def\aut{И.\,Н. Синицын$^1$}

\titel{\tit}{\aut}{\autkol}{\titkol}

{\renewcommand{\thefootnote}{\fnsymbol{footnote}}\footnotetext[1]
{Работа выполнена при финансовой поддержке РФФИ
(проект №\,10-07-00021).}}


\renewcommand{\thefootnote}{\arabic{footnote}}
\footnotetext[1]{Институт проблем информатики Российской академии наук, sinitsin@dol.ru}

\vspace*{-4pt}

\Abst{Статья посвящена теории и математическому обеспечению для 
анализа одно- и многомерных распределений процессов в многоканальных 
нелинейных круговых стохастических системах (КСтС) 
на базе методов параметризации распределений. Рассматриваются 
круговые ортогональные разложения (КОР) плот\-ностей круговых случайных величин (КСВ) и 
процессов, стохастические уравнения многоканальных нелинейных КСтС, 
интегродифференциальные уравнения для одно- и многомерных плотностей, общий метод КОР, 
а также методы <<намотанной>> нормальной аппроксимации (МННА), начальных и центральных моментов. 
Описаны модули инструментального программного обеспечения <<CStS-ANALYSIS>> в среде MATLAB. 
Результаты проиллюстрированы примером.}

\vspace*{-3pt}

\KW{аналитическое моделирование;
коэффициенты кругового ортогонального разложения;
круговая случайная величина;
круговой стохастических процесс;
<<намотанная>> нормальная плотность;
нелинейная многоканальная стохастическая сис\-те\-ма;
одно- и многомерные плотности распределения;
ортогональное разложение плотности;
эталонная плотность;
MATLAB; <<CStS-ANALYSIS>>}

\vspace*{-3pt}

 \vskip 14pt plus 9pt minus 6pt

      \thispagestyle{headings}

      \begin{multicols}{2}
      
            \label{st\stat}


\section{Введение}

В~[1--4] рассмотрено математическое обеспечение для 
спектрально-корреляционного анализа процессов в нелинейных КСтС, 
основанное на методе круговой статистической 
линеаризации посредством <<намотанного>> (\textit{wrapped}) нормального распределения.

Для анализа одно- и многомерных распределений в нелинейных 
КСтС невысокой размерности прямые методы решения интегродифференциальных уравнений Пугачёва 
для характеристических функций приводят к алгоритмам, требующим применения только средств 
суперкомпьютерной техники~\cite{5s}.

Как известно [6--8], для анализа стохастических процессов в многомерных нелинейных 
КСтС в евкли\-довом пространстве широкое применение получили методы параметризации 
распределений\linebreak (нормальной аппроксимации, начальных и центральных моментов, квазимоментов, 
а также ортогональ\-ных разложений и их модификаций). В~настоящее время создано инструментальное 
программное обеспечение в среде MATLAB~[3, 7, 8]. Применительно к нелинейным 
КСтС это математическое обеспечение требует развития.

Статья включает девять разделов. В~разд.~2 и~3 рассматриваются 
ортогональные разложения плотностей КСВ и стохастических процессов. 
В~разд.~4 приводятся стохастические уравнения нелинейной многоканальной стохастической сис\-те\-мы, 
а также интегродифференциальные уравнения для одно- и многомерных плотностей. 
Общее математические обеспечение <<CStS-ANALYSIS>> описано в разд.~5. В~разд.~6 
рассмотрено математическое обеспечение для метода круговой <<намотанной>> нормальной 
аппроксимации. В~разд.~7 приведены уравнения методов круговых начальных и центральных моментов. 
В~разд.~8 дан иллюстративный пример. Заключение содержит основные выводы.

\vspace*{-6pt}

\section{Ортогональное разложение плотности круговой случайной величины}

В задачах стохастической информатики широкое распространение получили модели 
распределений КСВ, основанные на следующих основных подходах:
\begin{itemize}
\item <<наматывание>> (\textit{wrapping}) распределений линейных СВ~$X$ на круг единичного радиуса 
$\Theta =X(\mathrm{mod}\,2\pi)$;
\item
преобразования к полярным координатам двумерного линейного распределения 
(так называемые  \textit{offset distributions}) или использование стереографической проекции;
\item
характеризации (параметризации) кругового распределения круговыми моментами, семиинвариантами 
и другими характеристиками, имеющими важное предметное значение, на основе ортогонального 
разложения плот\-ности по некоторой биортонормальной сис\-те\-ме функций с весом, 
задаваемым некоторой эталонной плотностью распределения.
\end{itemize}

Рассмотрим ортогональное разложение плот\-ности $f_\theta (\theta)$ для КСВ $\Theta$, 
обладающей конечными круговыми моментами, по некоторой известной биортогональной сис\-те\-ме 
функций  $\{ p_\nu (\theta), q_\nu (\theta)\}$ с весом  $w_\theta  (\theta)$ таким, что
    $$
    \iin w_\theta(\theta) p_\nu (\theta) q_\nu(\theta)\, d\theta= \delta_{\nu\mu}=
    \begin{cases}
    0\ \mbox{при\ }  &\nu\ne\mu\,;\\
    1\ \hbox{при\ }  &\nu=\mu\,,
    \end{cases}
    $$
следующего вида [6--8]:
\begin{equation}
    f_\theta(\theta) = w_\theta(\theta) \sss_\nu c_\nu p_\nu(\theta)\,,
    \label{e1s}
    \end{equation}
где
\begin{multline}
c_\nu =\iin f_\theta(\theta) q_\nu(\theta)\, d\theta =
\left[ q_\nu \left(\fr{\prt}{i\prt \lambda}\right) g(\lambda)\right]_{\lambda=0} 
\\ (i=\sqrt{-1})\,.\label{e2s}
\end{multline}
Здесь величины  $c_\nu$ называются коэффициентами КОР; 
$g_\theta(\theta)$~--- характеристическая функция, соответствующая плот\-ности 
$f_\theta(\theta)$; $w_\theta(\theta)$~--- плотность эталонного распределения.

Если в~(1) потребовать совпадения круговых моментов первого и второго порядка плот\-ностей 
$f_\theta (\theta)$ и $w_\theta(\theta)$, то КОР примет следующий вид:
   \begin{equation}
f_\theta(\theta) = w_\theta(\theta) \left[ 1 +\sss_{\nu=3}^\infty c_\nu p_\nu(\theta)\right]\,.
\label{e3s}
\end{equation}

\medskip

\noindent
\textbf{Замечание~1.} 
Функции $p_\nu(\theta)$ и $q_\nu (\theta)$
необязательно должны быть полиномами. Они могут быть любыми
функциями, удовлетворяющими условию биортонормальности и условиям
существования интегралов~(2). Все сказанное о разложении~(3)
справедливо и в более общем случае. Однако, если функции
$q_\nu(\theta)$ не являются полиномами, то, несмотря на совпадение
моментов первого и второго порядка распределений $f_\theta(\theta)$
и $w_\theta(\theta)$, круговые коэффициенты $c_\nu$ не будут равны
нулю при  $\nu=1,2$, вследствие чего суммирование по  $\nu$ будет
начинаться с $\nu\hm=1$.

\medskip

\noindent
\textbf{Замечание~2.} Иногда применяют разложение по производным некоторой плот\-ности 
$w_\theta(\theta)$, име\-ющей производные и моменты всех порядков~[6--8]:
    $$
    f_\theta(\theta) =\sss_{\nu=0}^\infty c_\nu w_\theta^{(\nu)} (\theta)\,.
    $$
В этом случае $p_\nu (\theta) \hm= w_\theta^{(\nu)}(\theta)/w_\theta(\theta)$, 
а  функции  $q_\nu(\theta)$ представляют собой полиномы.

В общем случае в зависимости от того, какие величины приняты за
параметры конечного отрезка КОР, различают моментные КОР,
семиинвариантные КОР, квазимоментные КОР и~др. Поэтому, обозначая
эти параметры через $u$, будем записывать КОР~(3) в виде:
\begin{align*}
f_\theta(\theta;u)&= w_\theta(\theta;u) \left[ 1+ \sss_{\nu=3}^\infty c_\nu 
p_\nu(\theta)\right]\,;\\
c_\nu &=\left[ q_\nu \left( \fr{\prt}{i\prt \lambda}\right) 
g(\lambda)\right]_{\lambda=0}\,,
%\label{e4s}
\end{align*}
где $c_\nu\hm= c_\nu (u)$; $p_\nu (\theta) \hm=p_\nu (\theta,u)$; 
$q_\nu \hm= q_\nu (\theta, u)$.


\section{Ортогональные разложения одно- 
и~многомерных плотностей круговых стохастических процессов}

Для действительного КСтП одномерная плот\-ность для 
момента времени $t$ в силу~(2) и~(3) определяется следующим КОР:
\begin{equation}
f_1 (\theta_t; t,u) = w_1(\theta_t, t,u) \left[ 
1 + \sss_{\nu=3}^\infty c_{\nu t} p_\nu (\theta_t,u) \right]\,,\label{e5s}
\end{equation}
где
\begin{multline*}
c_{\nu t}=c_{\nu t} (t,u)= \iin f_1(\theta_t;t,u) q_\nu(\theta_t) = {}\\
{}=
\left[ q_\nu \left(\fr{\prt}{i \prt \lambda}\right) g_\theta (\lambda;t,u)\right]\,.
%\label{e6s}
\end{multline*}
Для действительных КСтП $n$-мер\-ные плот\-ности для моментов времени  $t_1\tr t_n$ 
определяются как совокупность согласованных КОР плот\-ностей КСВ  
$\theta_{t_1} \tr \theta_{t_n}$:
\begin{multline*}
 f_n (\theta_{t_1}\tr \theta_{t_n}; t_1\tr t_n,u) ={}\\
 {}=
 w_n (\theta_{t_1}\tr \theta_{t_n}; t_1\tr t_n,u)\times{}
 \end{multline*}
 
 \noindent
 \begin{multline}
{}\times \left[ 1+ \sss_{k=3}^\infty \sss_{\nu_1+\cdots+\nu_n =k} 
c_{\nu_1 \tr \nu_n} (t_1\tr t_n,u) \times{}\right.\\
\left.{}\times p_{\nu_1\tr \nu_n} (\theta_{t_1} \tr \theta_{t_n}; t_1\tr t_n,u) 
\vphantom{\sum\limits_{k=3}^\infty}\right]\,,
\label{e7s}
\end{multline}
где
\begin{multline*}
c_{\nu_1\tr \nu_n} (t_1\tr t_n,u) = {}\\
{}=\iin\cdots \iin f_n (\theta_{t_1}\tr \theta_{t_n}; t_1\tr t_n,u)\times{}\\
{}\times q_{\nu_1\tr \nu_n}(\theta_{t_1}\tr \theta_{t_n}; t_1\tr t_n,u)\times{}\\
{}\times d\theta_{t_1}\cdots  d\theta_{t_n}={}\\
{}= \biggl[ q_{\nu_1\tr \nu_n} \left( \fr{\prt}{ i\prt \lambda_1}\tr \fr{\prt}{i\prt \lambda_n}; 
t_1\tr t_n,u\right)\times{}\\
{}\times g_n (\lambda_1\tr \lambda_n; t_1\tr t_n,u)\biggr]_{\lambda_1 =\cdots = \lambda_n=0}\,.
%\label{e8s}
\end{multline*}
Здесь $\{ p_{\nu_1\tr \nu_n} (\theta_{t_1}\tr \theta_{t_n},u)$, 
$q_{\nu_1\tr \nu_n} (\theta_{t_1} , \cdots$\linebreak 
$\cdots , \theta_{t_n},u)\}$~--- согласованные биортонормальные 
сис\-те\-мы полиномов, соответствующие согласованным многомерным плотностям 
$w_n (\theta_{t_1}\tr \theta_{t_n},u)$, имеющим те же круговые моменты первого и второго порядка, 
что и КСтП $\Theta(t)\hm=\Theta_t$.

Ограничиваясь в~(\ref{e5s}) и~(\ref{e7s}) полиномами  не выше $N$-й степени, 
получим согласованное приближенное представление всех многомерных плотностей КСтП $\Theta_t$. 
Этим приближенным представлением можно практически пользоваться, если КСтП 
имеет конечные круговые моменты до $N$-го порядка включительно, независимо от того, 
существуют или не существуют его моменты высших порядков.

При рассмотрении круговых марковских СтП соответствующие КОР 
используют для двух плотностей: одномерной и переходной.

В рамках спектрально-корреляционной теории принимают $n\hm=1,2$ и ограничиваются 
рассмотрением круговых дисперсий и ковариационных функций~[2].

\section{Многоканальные нелинейные круговые стохастические системы}

Введем векторный КСтП $Y(t) \hm= Y_t$, составленный из КСтП 
$\Theta_j(t) \hm=\Theta_{jt}$ $(j\hm=1\tr r)$, $Y_t \hm=\lk \Theta_{1t}\ldots \Theta_{rt}\rk$, 
где $r_y \hm=r$~--- число каналов в КСтС. 
Пусть векторное нелинейное стохастическое дифференциальное уравнение (понимаемое в смысле Ито), 
описывающее эволюцию $Y_t$, имеет вид~[6--8]:
\begin{equation}
\dot Y_t =\varphi (Y_t, t) +\psi (Y_t, t) V_t\,;\quad Y(t_0) = Y_0\,,\label{e9s}
\end{equation}
где $\varphi (Y_t, t)$ и $\psi (Y_t, t)$~--- $(r_y \hm\times 1)$- и $(r_y\hm\times r_V)$-мер\-ные в 
общем случае  известные нелинейные функции; $V_t$~--- $(r_V \hm\times 1)$-мер\-ный 
круговой белый шум в строгом смысле, представляющий собой среднюю квадратическую 
производную по времени от кругового процесса с независимыми приращениями $W_t, V_t \hm= \dot W_t$.
Начальное значение $Y_0$ будем считать независимым от приращений~$W_t$.

Обозначим через  $\chi (\mu;t)$ логарифмическую производную по времени 
от одномерной характеристической функции $h_1(\mu, t)$ кругового белого шума~$V_t$,  
$\chi(\mu;t)\hm= (\prt/\prt t) h_1 (\mu ;t)$. Тогда при известных условиях~[6, 7] 
одно- и $n$-мер\-ные плотности будут определяться следующей системой интегродифференциальных 
уравнений:
\begin{multline*}
    \fr{\prt}{\prt t_n}\, f_n (y_{t_1} \tr y_{t_n}) ={}\\
    {}= \fr{1}{(2\pi)^{nr} }
    \iin \cdots\iin \left[ i\lambda_n^{\mathrm{T}} \varphi (\eta_n, t_n) +{}\right.\\
\left.    {}+\chi 
    (\psi (\eta_n, t_n)^{\mathrm{T}} \lambda_n ; t_n)\right] %\times{}\\
%{}\times 
\exp \lf i \sss_{k=1}^n \lambda_k^{\mathrm{T}} (\eta_k - y_{t_k}) \rf \times{}\\
{}\times
f_n (\eta_1\tr \eta_n; t_1\tr t_n)\times{}\\
{}\times d\eta_1 \cdots d\eta_n d \lambda_1\ldots d\lambda_n
%\label{e10s}
\end{multline*}
$(n=1,2,\ldots)$ при условиях:
\begin{multline*}
f_n (y_{t_1} \tr y_{t_{n-1}}; t_1\tr t_{n-1}, t_{n-1})={}\\
\!{}= 
f_{n-1} ( y_{t_1}\tr y_{t_{n-1}}; t_1\tr t_{n-1}) \delta (y_{t_n} - y_{t_{n-1}});\hspace*{-1.55534pt} %\label{e11s}
\end{multline*}

\vspace*{-12pt}

\noindent
\begin{equation*}
f_1 (y_t; t_0) = f_0 (y_t)\,. %\label{e12s}
\end{equation*}

\noindent
\textbf{Замечание~3.} Для дискретных КСтС соответствующие интегроразностные уравнения 
получаются путем замены оператора дифференцирования по времени на оператор сдвига по 
времени~[6--8].

\section{Математическое обеспечение, основанное на ортогональных разложениях}

При аппроксимации круговой одномерной плотности конечным отрезком КОР 
естественно за параметры $u$ принять: $m_t$~--- вектор круговых средних; 
$K_t$~--- круговую ковариационную мат\-ри\-цу (от которых зависит эталонная плот\-ность, 
а следовательно, и полиномы $\lf p_\nu (y_t), q_\nu (y_t)\rf$) и коэффициенты $c_{\kappa, t}$. 
Поэтому из~(\ref{e5s}) для отрезка КОР имеем:



\noindent
\begin{multline}
\tilde f_1 (y_t, u) ={}\\
{}= w_1 (y_t,u) \lk 1+ \sss_{l=3}^N \sss_{\left\vert\kappa\right\vert=l} 
c_{\kappa t} p_\kappa (y_t,u)\rk\,.\label{e13s}
\end{multline}
При этом, согласно~[6, 7], $m_t, K_t$ и $c_{\nu t}$ будут определяться 
следующей системой обыкновенных дифференциальных уравнений:
\begin{equation}
\left.
\begin{array}{rl}
\!\!\!\!\!\dot m_t &= \displaystyle
A_0 (m_t, K_t, t) + \sss_{l=3}^N \sss_{\left\vert\nu\right\vert=l} 
A_{1\nu}(m_t, K_t, t)  c_{\nu t};\\[9pt] 
\!\!\!\!\!m_0 &= m(t_0);
\end{array}\!\!
\right\}\!\!\!\label{e14s}
\end{equation}

\noindent
\begin{equation}
\left.
\begin{array}{rl}
\!\!\!\!\!\dot K_t &= B_0 (m_t, K_t, t) +\displaystyle \sss_{l=3}^N \sss_{\left\vert\nu\right\vert=l} 
B_{1\nu}(m_t, K_t, t)  c_{\nu t};\\[9pt]
\!\!\!\!\!K_0 &= K(t_0);
\end{array}\!\!
\right\}\!\!\!
\label{e15s}
\end{equation}

\noindent
\begin{equation}
\left.
\begin{array}{rl}
\!\!\!\!\dot c_{\kappa t} &= C_{\kappa 0}  (m_t, K_t, t)+ A_0 (m_t, K_t, t)^{\mathrm{T}} 
q_\kappa^m (\alpha ) +{}\\[9pt]
&\hspace*{2mm}{}+
\mathrm{tr}\, \lk B_0 (m_t, K_t, t) q_\kappa^K (\alpha)\rk+{}\\[8pt]
&\hspace*{5mm}{}+ \displaystyle \sss_{l=3}^N \sss_{\left\vert\nu\right\vert=l} 
\left\{ \vphantom{B_{1v}k_\kappa^K}
C_{\kappa\nu} (m_t, K_t, t)+{}\right.\\[9pt]
&\hspace*{7mm}\left.{}+A_{1\nu}(m_t, K_t, t)^{\mathrm{T}} q_\kappa^m (\alpha) + {}\right.\\[9pt]
&\hspace*{10mm}\left.{}+
\mathrm{tr}\, \lk B_{1\nu} (m_t, K_t, t)q_\kappa^K(\alpha)\rk\right\} c_{\nu t}\,;\\[9pt]
\!\!\!\!c_{\kappa 0}&= c_\kappa(t_0)\,.
\end{array}\!
\right\}\!\!
\label{e16s}
\end{equation}
Здесь введены следующие обозначения:
    \begin{equation}
    \left.
    \begin{array}{rl}
    A_0 (m_t, K_t, t)&= \displaystyle\iin \varphi(y_t, t) w_1 (y_t,u)\, dy_t\,;\\[9pt]
 A_{1\nu} (m_t, K_t, t) &=\displaystyle\iin \varphi (y_t, t) p_\nu (y_t,u)\, d y_t)\,;
 \end{array}
 \right\}
 \label{e16-1s}
 \end{equation}

\noindent
\begin{equation}
\left.
\begin{array}{rl}
B_0 (m_t, K_t, t)&= B_{01} (m_t, K_t, t)+{}\\[9pt]
&\hspace*{-15mm}{}+B_{01} (m_t, K_t, t)^{\mathrm{T}} +B_{02} (m_t, K_t, t)={}\\[9pt]
&\hspace*{-15mm}{}=\displaystyle\iin \left[ \varphi(y_t, t) (y_t - m_t)^{\mathrm{T}} +{}\right.\\
&\left.{}+(y_t - m_t) 
\varphi(y_t, t)^{\mathrm{T}} +{}\right.\\[9pt]
&\hspace*{-15mm}\left.{}+\psi(y_t, t) G_t \psi_(y_t, t)^{\mathrm{T}} \right]  w_1 (y_t,u)\, dy_t\,;\\[9pt]
B_{1\nu} (m_t, K_t, t)&=\displaystyle\iin \left[ 
\varphi(y_t, t) (y_t- m_t)^{\mathrm{T}} +{}\right.\\[9pt] 
&\hspace*{-23mm}\left.{}+(y_t-m_t) \varphi(y_t, t)^{\mathrm{T}} \psi (y_t, t) G_t 
\psi (y_t, t)^{\mathrm{T}} \right]\times{}\\[9pt]
&{}\times w_1 (y_t,u)\, dy_t\,;
\end{array}\!
\right\}\!
\label{e17s}
\end{equation}

\noindent
\begin{equation}
\left.
\begin{array}{rl}
C_{\kappa 0}(m_t, K_t, t) &={}\\[9pt]
&\hspace*{-8mm}{}=\displaystyle \iin \left\{ q_\kappa \left( \fr{\prt}{i\prt \lambda}\right) 
\left[ i\lambda^{\mathrm{T}} \varphi(y_t, t) +{}\right.\right.\\[9pt]
&\hspace*{-25mm}\left.\left.{}+\chi(\psi(y_t, t)^{\mathrm{T}} \lambda;t)\right]\exp 
\lk i\lambda^{\mathrm{T}} y_t\rk \vphantom{\fr{\partial}{\partial}}
\right\}_{\lambda=0}\times{}\\[9pt]
&\hspace*{17mm}{}\times w_1(y_t,u)\, dy_t\,;\\[9pt]
C_{\kappa \nu}(m_t, K_t, t) &={}\\[9pt]
&\hspace*{-8mm}{}=\displaystyle \iin \left\{ q_\kappa \left( \fr{\prt}{i\prt \lambda}\right) 
\left[ i\lambda^{\mathrm{T}} \varphi(y_t, t) +{}\right.\right.\\[9pt]
&\hspace*{-27mm}\left.\left.{}+\chi(\psi(y_t, t)^{\mathrm{T}} \lambda;t)\right]\exp 
\lk i\lambda^{\mathrm{T}} y_t\rk p_\nu (y_t,u)
\vphantom{\fr{\partial}{\partial}}
\right\}_{\lambda=0}\!\!\!\times{}\\[9pt]
&\hspace*{20mm}{}\times w_1(y_t,u)\, dy_t\,,
\end{array}\!\!
\right\}\!\!
\label{e17-1s}
\end{equation}
где $q_\kappa^m (y_t)$~--- мат\-ри\-ца-стол\-бец производных полинома  $q_\kappa (y_t)$ 
по компонентам вектора  $m_t$; $q_\kappa^K (y_t)$~--- квадратная мат\-ри\-ца 
производных полинома  $q_\kappa (y_t)$ по элементам мат\-ри\-цы  $K_t$; 
$G_t \hm= \lk G_{lj} (t)\rk$~--- мат\-ри\-ца интенсивностей векторного кругового белого шума~$V$, 
причем
    $$
    G_{lj} (t) = \lk \fr{\prt^2 \chi (\mu;t)}{\prt (i\mu_l) \prt (i\mu_j)}\rk_{\mu=0}\,;
    $$
$q_\kappa^m(\alpha)$  и $q_\kappa^K(\alpha)$~--- 
результат замены одночленов вида $y_{1t}^{\rho_1}\cdots y_{rt}^{\rho_r}$ 
соответствующими  начальными моментами $\alpha_{\rho_1 \tr \rho_r}$ которые, 
как известно, зависят от~$c_{\kappa t}$.

\medskip
\noindent
\textbf{Замечание~4.} Уравнения~(\ref{e16s}) нелинейны относительно  $c_{\kappa t}$. 
Если отказаться от требования совпадения первых  моментов и задать эти моменты для 
эталонной плотности априори, то для  $c_{\kappa t}$ получатся линейные уравнения. 
Эти уравнения проще чем~(\ref{e16s}), однако при этом придется взять большее~$N$ в~(\ref{e13s}).

\smallskip

Таким образом, при использовании полиномиальной биортонормальной сис\-те\-мы  
$\lf p_\nu (y_t,u), q_\mu (y_t,u)\rf$ в основе математического обеспечения 
кругового ортогонального~(\ref{e13s}) разложения одномерной плотности КСтП  
$Y_t$ в КСтС~(\ref{e9s}) лежат уравнения~(\ref{e14s})--(\ref{e16s}) при условиях~(\ref{e16-1s})--(\ref{e17-1s}).

Аналогично на основе результатов~[6, 7] составляются уравнения для $n$-мер\-ных 
плот\-но\-стей~(\ref{e7s}) $(n\hm\ge 2)$.

В состав математического обеспечения  <<CStS-ANALYSIS>> входят:
\begin{itemize}
\item
уравнения для различных типов систем;
\item
типовые системы биортонормальных сис\-тем функций для одно- и многомерных плотностей;
\item
ортогональные разложения плотностей одно- и многомерных распределений;
\item
обыкновенные дифференциальные (разностные) уравнения для параметров одно- и 
многомерных распределений с соответствующими начальными условиями;
\item
выражения для вычисления типовых функционалов, определяющих задачи вероятностного 
кругового анализа многоканальной КСтС;
\item
набор тестовых задач.
\end{itemize}

\section{Математическое обеспечение на~основе метода <<намотанной>> нормальной аппроксимации}

Обобщая результаты~[2] на случай, когда коэффициент при круговом белом шуме в 
уравнении~(\ref{e9s}) 
зависит от состояния, приведем основные уравнения кругового МННА:
\begin{align}
\dot m_t &=A_0 (m_t, K_t, t)\,,\enskip m_0 = m(t_0)\,;\label{e18s}\\
\dot K_t &=B_0 (m_t, K_t, t)\,,\enskip K_0 = K(t_0)\,;\label{e19s}
\end{align}

\noindent
\begin{equation}
\left.
\begin{array}{rl}
\fr{\prt K(t_1, t_2)}{\prt t_2} &=\\[9pt] 
&\hspace*{-15mm}{}=\fr{K(t_1, t_2)}{K(t_2)}\, B_{01}(m_{t_2}, K_{t_2}, t_2)\quad 
(t_1< t_2)\,;\\[9pt]
 K(t_1, t_1) &= K_{t_1}\,.
 \end{array}
 \right\}
 \label{e20s}
\end{equation}
Здесь  $A_0(m_t, K_t, t)$, $B_0(m_t, K_t, t)$ и $B_{01}(m_t, K_t, t)$  
определены в~(11)--(12) для 
<<намотанного>> нормального распределения.

В состав МННА включено математическое обеспечение~[2] для сис\-те\-мы~(\ref{e9s}) 
при $\psi (y_t, t) \hm= I_r$.

\section{Математическое обеспечение на~основе методов круговых начальных 
и~центральных моментов}

Из уравнений разд.~5 для круговых <<намотанных>> начальных моментов  
$\alpha_\rho \hm=\alpha_\rho (t)$ порядка  $\rho$, обобщая~[6, 7], имеем следующие уравнения:
\begin{equation}
\dot\alpha_\rho = A_{0,\rho} +\sss_{k=3}^N \sss_{\left\vert\nu\right\vert = 
k} A_{\nu,\rho} c_\nu (\alpha) \enskip ( c_\nu = q_\nu (\alpha))\,,\label{e21s}
\end{equation}
где
\begin{multline*}
A_{0,\rho} = \iin \biggl\{ \fr{\prt^{|\rho |}}{\prt ( i\la_1)^{\rho_1} \cdots 
\prt ( i \la_r)^{\rho_r}}
\left[ i\lambda^{\mathrm{T}} \varphi(y_t, t) +{}\right.\\ 
\left.{}+\chi (\psi(y_t, t)^{\mathrm{T}}\lambda;t)\right] \exp \lk i\lambda^{\mathrm{T}}
y_t\rk\biggr\}_{\lambda=0} 
w_1(y_t,u)\, dy_t\,;
\end{multline*}

\noindent
\begin{multline*}
A_{\nu,\rho} = \iin \biggl\{ \fr{\prt^{|\rho |}}{\prt ( i\lambda_1)^{\rho_1} \cdots 
\prt ( i \lambda_r)^{\rho_r}}
 \left[ i\lambda^{\mathrm{T}} \varphi(y_t, t) +{}\right.\\
\left. {}+ 
\chi (\psi(y_t, t)^{\mathrm{T}}\la;t)\right] \exp \lk i\lambda^{\mathrm{T}} 
y_t\rk\biggr\}_{\lambda=0} \times{}\\[4pt]
{}\times
p_\nu (y_t,u) w_1(y_t,u)\, dy_t %\label{e22s}
\end{multline*}


\vspace*{-8pt}

\noindent
\begin{equation*}
\left(\left\vert \rho\right\vert = \rho_1 +\cdots + \rho_r , 
\rho_1 \tr \rho_r = 0,1\tr N\right)\,.
\end{equation*}
Напомним, что $q_\nu (\alpha)$ представляет собой результат замены всех 
одночленов $y_{1t}^{\rho_1} \cdots y_{rt}^{\rho_r}$ в выражении полинома  
$q_\nu (\alpha)$ соответствующими моментами $\alpha_{\rho_1\tr \rho_r}$.

\medskip

\noindent
\textbf{Замечание~5.} При составлении уравнений~(\ref{e13s}) 
в конкретных задачах следует иметь в виду, что число $N_r^\rho$ 
моментов $\rho$-го порядка $r$-мер\-но\-го вектора определяется формулой:
    $$
    N_r^\rho = C_{r+\rho-1}^\rho = \fr{(r+\rho-1)!}{\rho! (r-1)!}\,,
    $$
а полное число моментов, не превосходящих $N$, для $r$-мер\-но\-го
вектора равно:
    $$
    P_r^N =\sss_{l=1}^N N_r^l = \fr{(N+r)!}{N! r!}-1\,.$$
    
    \medskip

\noindent
\textbf{Замечание~6.} Уравнения~(\ref{e21s}) линейны относительно  
$\alpha_\rho$ $(\left\vert \rho\right\vert = 3\tr N)$ и 
нелинейны относительно моментов первого и второго порядка, поскольку эталонная плот\-ность 
и полиномы  $p_\nu (y_t,u)$ и $q_\nu (y_t,u)$ зависят от моментов первого и второго порядка.

\smallskip

Обобщая [6, 7] для математических ожиданий $m_h$ $(h\hm=1\tr r)$ и центральных 
моментов $\mu_\rho$ порядка~$\rho$, представим уравнения в виде:
\begin{gather}
\dot m_h = A_{0, h} +\sss_{k=3}^\infty \sss_{|\nu |=k} A_{\nu r} c_\nu\,;\enskip 
c_\nu = q_\nu (\alpha)\,;\label{e23s}\\
\dot\mu_\rho = B_{0,\rho} - \sss_{h=1}^r \rho_h B_{0,h} \mu_{\rho- e_h} + {}\notag\\
{}+
\sss_{k=3}^N \sss_{|\nu |=k} \lk B_{\nu,\rho} - 
\sss_{h=1}^r \rho_h B_{\nu,h} \mu_{\rho - e_h}\rk c_\nu\notag \\
(\rho_1\tr \rho_r = 0,1,\tr N\,; \notag\\ 
\left\vert \rho\right\vert = 2\tr N)\,.\label{e24s}
\end{gather}
Здесь введены следующие обозначения:
\begin{equation*}
e_h = \lk 0\cdots 0 \ {\ed\limits_h}\  0\cdots 0\rk^{\mathrm{T}}\,;
\end{equation*}

\noindent
\begin{align*}
A_{0,h} &=\iin \varphi_h (y_t, t) w_1(y_t,u) \,d y_t\,;\\
A_{\nu,h} &=\iin \varphi_h (y_t, t)p_\nu (y_t,u) w_1(y_t,u)\, d y_t\,;\\
A_{0,\rho} &=\iin \biggl\{ \fr{\prt^{|\rho |}}{\prt (i\la_1)^{\rho_1} \cdots 
\prt ( i\lambda_r)^{\rho_r}} \left[ i\lambda^{\mathrm{T}} \varphi(y_t, t)+{}\right.\\
&\hspace*{15mm}\left.{}+
\chi(\psi (y_t, t)^{\mathrm{T}}\lambda;t)\right] \times{}\\
&{}\times \exp \lk i\lambda^{\mathrm{T}} (y_t-m)\rk \biggr\}_{\lambda=0}  w_1(y_t,u)\, d y_t\,;\\
B_{\nu,\rho} &=\iin \biggl\{ \fr{\prt^{|\rho |}}
{\prt (i\lambda_1)^{\rho_1} \cdots \prt ( i\lambda_r)^{\rho_r}} \left[ i\lambda^{\mathrm{T}} 
\varphi(y_t, t)+{}\right.\\
&\left.{}+\chi(\psi (y_t, t)^{\mathrm{T}}\lambda;t)\right ] 
\exp \lk i\lambda^{\mathrm{T}} (y_t-m)\rk \biggr\}_{\lambda=0}\times{}\\
&\hspace*{10mm}{}\times p_\nu (y_t,u)  
w_1(y_t,u)\, d y_t\,, %\label{e25s}
\end{align*}
а  $q_\nu (\alp)$ должны быть выражены через центральные моменты.

\medskip

\noindent
\textbf{Замечание~7.} Уравнения~(\ref{e23s}) и~(\ref{e24s}) 
всегда нелинейны из-за наличия слагаемых вида $\mu_{\rho-e_h} q_\nu (\alpha)$.

\medskip

\noindent
\textbf{Замечание~8.} В~практических задачах для 
полиномиальных функций  $\varphi (y_t, t) , \psi(y_t, t)$ и  $p_\nu (y_t), q_\nu(y_t)$ 
непосредственно используются символьные вы\-чис\-ле\-ния~[3].

%\vspace*{-12pt}

\section{Пример}

Для одномерной нелинейной круговой сис\-те\-мы
\begin{equation}
\dot Y +\varphi (Y)=V\,,\label{e26s}
\end{equation}
($\varphi (Y)$~--- скалярная нелинейная функция, $V$~--- круговой белый шум интенсивности~$G_t$), 
уравнения МННА~(\ref{e18s})--(\ref{e20s}) имеют вид:
\begin{gather*}
\dot m_t = - A_0 (m_t, D_t);\enskip \dot D_t =- 2 k_1 (m_t, D_t) + G_t\,;\\
\fr{\prt K(t_1, t_2)}{ \prt t_2} =- k_1 (m_t, D_t) K(t_1, t_2)\,,
\end{gather*}
где $A_0 (m_t, D_t)$ и $k_1 (m_t, D_t)$~--- 
коэффициенты статистической линеаризации нелинейной функции  $\varphi(Y)$ в~(\ref{e26s}) 
для <<намотанного>> нормального распределения с параметрами  $m_t$ и~$D_t$.

При $N=4$ уравнения~(\ref{e23s}) и~(\ref{e24s}) имеют вид:
\begin{multline*}
\dot m_t ={}\\
{}=  A_0 (m_t, D_t)+A_{13} (m_t, D_t)\mu_{3t} +A_{14} (m_t, D_t)\mu_{4t}\,;
\end{multline*}

\noindent
\begin{align*}
\dot D_t &={}\\[2pt]
&\hspace{-9pt}{}=B_0 (m_t, D_t)+B_{13} (m_t, D_t)\mu_{3t}+B_{14} (m_t, D_t) \mu_{4t}\,;\\[9pt]
\dot \mu_{3t} &={}\\[2pt]
&\hspace{-10pt}{}= C_{30}(m_t, D_t)+ C_{33}(m_t, D_t)\mu_{3t}+C_{34}(m_t, D_t) \mu_{4t}\,;\\[9pt]
\dot \mu_{4t} &={}\\[2pt]
&\hspace*{-16pt}{}= C_{40}(m_t, D_t)+ C_{41}(m_t, D_t)\mu_{3t}+C_{44}(m_t, D_t) \mu_{4t} -{}\\[2pt]
&\hspace*{36pt}{}- 4 A_{13} (m_t, D_t) \mu_{3t}^2 -A_{14} (m_t, D_t) \mu_{3t}\mu_{4t}\,.
\end{align*}
%
Для различных нелинейных функций $\varphi(y)$ эти урав\-не\-ния
позволяют оценить точность МННА.

\section{Заключение}

Разработана прикладная тео\-рия анализа одно- и многомерных распределений в 
нелинейных многоканальных круговых стохастических системах на основе ортогональных 
разложений плотностей.

Для кругового эталонного <<намотанного>> нормаль\-но\-го распределения 
описанное математическое обеспечение положено в основу инструментального 
программного обеспечения <<CStS-ANALYSIS>> в среде MATLAB.

В настоящее время в ИПИ РАН ведутся работы по созданию математического обеспечения для 
других эталонных распределений.

{\small\frenchspacing
{%\baselineskip=10.8pt
\addcontentsline{toc}{section}{Литература}
\begin{thebibliography}{9}
\bibitem{1s}
\Au{Синицын И.\,Н.}
Канонические разложения случайных функций и их применение в стохастических информационных 
технологиях научных исследований: Курс лекций~// Распознавание образов и анализ изображений: 
новые информационные технологии. РОАИ-10-2010: Мат-лы 1-й Междунар. конф.~--- СПб., 2010.

\bibitem{2s}
\Au{Синицын И.\,Н.}
Стохастические информационные технологии для исследования 
нелинейных круговых стохастических систем~// Информатика и её применения, 2011. Т.~5. Вып.~4. 
С.~2--5.

\bibitem{3s}
\Au{Синицын И.\,Н., Корепанов Э.\,Р., Белоусов В.\,В. и~др.}
Развитие компьютерной поддержки статистических научных исследований сис\-тем 
высокой точности и доступности~// Cистемы и средства информатики, 2011. Вып.~21. №\,1. С.~7--37.

\bibitem{4s}
\Au{Sinitsyn I.\,N., Belousov~V.\,V., Konashenkova~T.\,D.}
Software tools for circular stochastic systems analysis~// 29th
Seminar (International) on Stability Problems for Stochastic\linebreak\vspace*{-12pt}
\pagebreak

\noindent
 Models
and 5th Workshop <<Applied Problems in Theory of Probabilities and
Mathematical Statistics Related to Modeling of Information Systems>>
(APTP\;+\;MS'2011): Book on Abstracts.~--- M.: IPIRAS, 2011. P.~86--87.

\bibitem{5s}
\Au{Босов А.\,В., Будзко В.\,И., Захаров В.\,Н., Козмидиади~В.\,А., Корепанов~Э.\,Р., 
Синицын~И.\,Н., Шоргин~С.\,Я., Уш\-ма\-ев~О.\,С.}  
Информатика: состояние, проблемы, перспективы~/ Под. ред. И.\,А.~Соколова.~--- М.: ИПИ
РАН, 2009.

\bibitem{6s}
\Au{Пугачев В.\,С., Синицын И.\,Н. }
Стохастические дифференциальные системы. Анализ и фильтрация.~--- 2-е изд. доп.~--- М.: Наука, 1990.

\bibitem{7s}
\Au{Пугачев В.\,С., Синицын И.\,Н.}
Теория стохастических систем.~--- 2-е изд.~--- М.: Логос, 2004.

\label{end\stat}

\bibitem{8s}
\Au{Синицын И.\,Н.}
Канонические представления случайных функций и их применения 
в задачах компьютерной поддержки научных исследований.~--- М.: ТОРУС ПРЕСС, 2009.
 \end{thebibliography}
}
}


\end{multicols}       