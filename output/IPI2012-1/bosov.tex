\def\stat{bosov}

\def\tit{ЗАДАЧИ АНАЛИЗА И ОПТИМИЗАЦИИ ДЛЯ~МОДЕЛИ ПОЛЬЗОВАТЕЛЬСКОЙ 
АКТИВНОСТИ.\\ ЧАСТЬ~2. ОПТИМИЗАЦИЯ ВНУТРЕННИХ РЕСУРСОВ}

\def\titkol{Задачи анализа и оптимизации для~модели пользовательской 
активности. Ч.~2. Оптимизация внутренних ресурсов}

\def\autkol{А.\,В.~Босов}
\def\aut{А.\,В.~Босов$^1$}

\titel{\tit}{\aut}{\autkol}{\titkol}

%{\renewcommand{\thefootnote}{\fnsymbol{footnote}}\footnotetext[1]
%{Работа выполнена при поддержке РФФИ (гранты 09-07-12098, 09-07-00212-а и 
%09-07-00211-а) и Минобрнауки РФ (контракт №\,07.514.11.4001).}}


\renewcommand{\thefootnote}{\arabic{footnote}}
\footnotetext[1]{Институт проблем информатики Российской академии наук, AVBosov@ipiran.ru}
  

\Abst{Продолжено исследование математической модели описания активности 
пользователей, предложенной автором ранее. Сформулирована и решена задача 
оптимизации распределения <<внутренних>> ресурсов, используемых информационной 
системой, на основе квадратичного критерия качества. Предложены субоптимальные 
алгоритмы оптимизации.}

\KW{информационная система; стохастическая система наблюдения; квадратичный 
критерий}

\vskip 14pt plus 9pt minus 6pt

      \thispagestyle{headings}

      \begin{multicols}{2}
      
            \label{st\stat}

\section{Введение}
  
  В работе~\cite{1bos} предложена и исследована математическая модель описания 
пользовательской активности в некоторой информационной системе. Модель реализована в 
форме стохастической динамической системы наблюдения специального вида: 
ненаблюдаемое состояние~$x_t$ описывает число пользователей, формирующих запросы к 
узлу информационной системы, косвенные наблюдения~$y_t$ предполагаются линейными 
функциями состояния и задают число заданий, выполненных узлом в течение заданного 
промежутка времени. С~описанием характера показателя пользовательской активности 
связано понятие текущего режима. Предполагается, что выделено определенное число 
характерных для изучаемого процесса $x_t$ режимов, каждый из которых задается 
определенным диапазоном значений и изменяется при выходе~$x_t$ за границы текущего 
диапазона. Для соответствующей системы наблюдения решены задачи анализа, и на примере 
конкретного программного обеспечения~--- Информационного веб-пор\-та\-ла~\cite{2bos}~--- 
проиллюстрированы, во-первых, физический смысл параметров модели, во-вторых, 
работоспособность процедур оценивания.
  
  Традиционное применение такого рода моделей, в основном, заключается в разных формах 
анализа процессов в изучаемой системе, например с целью выявления <<узких>> мест, 
предельных распределений и~пр. Однако не менее перспективными представляются 
варианты применения моделей функционирования для оптимизации работы элементов 
информационной системы и, прежде всего, компонентов ее программного обеспечения. 
Примеры подобной оптимизации хорошо известны. Это и управление страничным файлом 
операционной системы, и оптимизация запросов реляционной системы управления базами 
данных, и алгоритмы диспетчеризации задач в многопроцессорных средах, и многие другие. 
Относительно этих примеров можно отметить, что существенного применения 
математического аппарата в таких задачах немного, более привычными оказываются слабо 
формализуемые эвристические подходы. В~качестве удачного примера обратного можно 
указать на класс задач, связанных с управляемым протоколом TCP (см., например,~[3--5]), 
однако в целом подобных приложений немного.
  
  Одной из причин такой ситуации, по-видимому, являются трудности не столько в 
моделировании конкретных процессов функционирования, сколько в корректной постановке 
задач оптимизации, включая выделение оптимизируемых характеристик, управляющих 
воздействий и критериев.
  
  В данной работе применительно к вычислительным ресурсам, используемым узлом 
информационной системы, удается преодолеть указанные трудности, в том чис\-ле благодаря 
наличию модели пользовательской активности. При этом в интересах подтверждения 
прикладной значимости ассоциация рассмотренной далее оптимизационной постановки с 
портальной технологией будет сохранена, хотя результат, очевидно, допускает и более 
общую трактовку.

\section{Используемые обозначения}
  
  Далее в работе будут использованы следующие обозначения:
  \begin{description}
  \item[\,] $\overset{\Delta}{=}$~--- равенство по определению;
\item[\,] $\mathbf{M}[x]$ и $\mathbf{M}[x\vert \mathfrak{J}]$~--- соответственно безусловное 
математическое ожидание случайной величины~$x$ и условное математическое 
ожидание~$x$ относительно $\sigma$-ал\-геб\-ры~$\mathfrak{J}$;
\item[\,] $x^{\mathrm{T}}$~--- операция транспонирования вектора (мат\-ри\-цы)~$x$;
\item[\,] $\mathrm{col}\,(x_1,\ldots ,x_n)\overset{\Delta}{=}(x_1,\ldots ,x_n)^{\mathrm{T}}$~--- век\-тор-стол\-бец с 
элементами $x_1,\ldots ,x_n$;
\item[\,] $\mathrm{row}\,(x_1,\ldots ,x_n) \overset{\Delta}{=}(x_1,\ldots ,x_n)$~--- век\-тор-стро\-ка с 
элементами  $x_1,\ldots ,x_n$;
\item[\,] $\mathfrak{J}_t^{y}\overset{\Delta}{=}\sigma\{y_\tau,\tau\hm\leq t\}$~--- 
$\sigma$-ал\-геб\-ра, порожденная наблюдениями~$y_\tau$, $\tau\hm\leq t$;
\item[\,] $\overline{\psi}_x(x,t,j)$, $j\hm=1, 2,\ldots ,$~--- условная плот\-ность вероятности 
$x_{t+j}$ относительно $\sigma$-ал\-геб\-ры~$\mathfrak{J}_t^y$;
\item[\,] $\hat{\psi}_x(x,t)$~--- условная плотность вероятности~$x_t$ относительно 
$\mathfrak{J}_t^y$.
\end{description}

\section{Модель распределения <<внутренних>> ресурсов портала}
  
  Любая программная система, и портал в частности, при реализации собственной 
функциональности задействует различные вычислительные ресурсы. Программы в процессе 
работы используют ресурсы операционной системы, технической платформы, 
телекоммуникационной инфраструктуры. Принципиально возможно сформировать полный 
перечень такого рода ресурсов и даже как-то их классифицировать. Однако конструктивно 
влиять на их выделение, как правило, оказывается невозможным. Объясняется это, в 
основном, тем, что ресурсы программами запрашиваются у внешних (обслуживающих) 
систем, управлять которыми программы обычно не могут. Таким образом, предлагать 
формулировки задач оптимизации расходования ресурсов следует, прежде всего, для 
  каких-либо <<внутренних>> ресурсов. К~последним надо относить те программные 
объекты, которые создаются (временно используются) программами при росте нагрузки и 
освобождаются при ее уменьшении.
  
  В работе Информационного веб-портала такой <<внутренний>> ресурс присутствует. Это 
так называемый пул запросов (подробнее см.~[6]). Для параллельного выполнения 
поступающих запросов в составе подсистемы интеграции и поиска портала выделен 
компонент исполнения запросов. Этот компонент поддерживает заданное (фиксированное) 
число очередей. Каждая пользовательская команда преобразуется в набор запросов, которые 
распределяются по этим очередям. Для каждой очереди заранее создана программная нить, в 
которой и исполняются последовательно запросы из очереди. Такое решение ограничивает, с 
одной стороны, чис\-ло одновременно работающих нитей, не позволяя системе 
<<зависнуть>>, с другой~--- при поступлении сложных команд, порождающих большое 
число запросов, позволяет им исполняться параллельно.
  
  Совокупность очередей с нитями, названная пулом запросов, и является тем самым 
<<внутренним>> ресурсом, возможностям оптимизации которого и посвящена данная 
статья. <<Внутренним>> пул назван потому, что его управление полностью зависит от 
программных приложений портала и никак не контролируется внешними обслуживающими 
сис\-те\-ма\-ми. Принципиальные технические проблемы\linebreak в обеспечении возможности 
динамического из\-ме\-нения размера пула отсутствуют, однако и программирование такой 
функциональности, и цена вы\-чис\-ли\-тель\-ных затрат на ее реализацию существенно высоки, 
что требует, соответственно, строго обосно\-ван\-но\-го подхода к данной проблеме.
  
  Требования к размеру пула определяет текущая пользовательская активность, 
исследование модели которой проведено в~[1]. Показатель пользовательской 
активности~$x_t$ за интервал наблюдения $(t-1;t]$  описывается разностным 
стохастическим уравнением
  \begin{multline}
  x_t=a\theta(x_{t-1})x_{t-1}+q\Theta(x_{t-1})+{}\\
  {}+b\Theta(x_{t-1})v_t\,,\enskip t=1,2,\ldots ,
  \label{e1bos}
  \end{multline}
в предположении, что область значений~$x_t$ разбита на непересекающиеся 
интервалы~$\Delta_k$:
\begin{gather*}
-\infty =a_1<a_2<\cdots <a_n<a_{n+1}=+\infty\,,\\
\Delta_k=(a_k,a_{k+1}],\ k=1,\ldots ,n-1,\ \Delta_n=(a_n,+\infty)
\end{gather*}
и текущий режим пользовательской активности задан индикаторной функцией~$\Theta(x)$:
\begin{gather}
\Theta(x) =\mathrm{col}\left( I_{\Delta_1}(x),\ldots ,I_{\Delta_n}(x)\right)\,;\notag\\
I_{\Delta_k} (x)= \begin{cases}
1\,, &\ \mbox{если}\ x\in \Delta_k\,;\\
0\,, &\ \mbox{если}\ x\not\in \Delta_k\,;
\end{cases}\label{e2bos}
\\
a=\mathrm{row}\left (a_1,\ldots , a_n\right)\,;\ q=\mathrm{row}\left (q_1,\ldots ,q_n\right)\,;\notag\\
b=\mathrm{row}\left (b_1,\ldots ,b_n\right)\,.\notag
\end{gather}
 
  В качестве наблюдений за~$x_t$, как и в~[1], будем использовать число команд~$y_t$, 
выполненных порталом за интервал наблюдения $(t-1;t]$:
  \begin{equation}
  y_t=c^y x_t+\sigma^y w_t^y\,.
  \label{e3bos}
  \end{equation}
  Здесь параметр $c^y$ определяет среднее число команд, направленных одним 
пользователем для выполне\-ния порталом; $w_t^y$~--- возмущение, характеризующее 
отклонение числа команд от среднего уровня; $\sigma^y$~--- среднее квадратическое 
отклонение этого возмущения (далее предполагается, что $\{w_t^y\}$~--- стандартный 
дискретный белый шум второго порядка).
  
  Кроме того, учтем, что на требуемый размер пула влияет не столько число выполняемых 
команд, сколько число сформированных из них запросов, поэтому в дополнение к 
наблюдениям~(\ref{e3bos}) будем рассматривать и число выполненных на интервале $(t-1;t]$  
запросов~$z_t$, предполагая его пропорциональным числу команд:
  \begin{equation}
  z_t=c^z y_t+\sigma^z w_t^z\,.
  \label{e4bos}
  \end{equation}
Здесь параметр $c^z$ определяет среднее число запросов к источникам, формируемых из 
одной команды; $w_t^z$~--- возмущение, характеризующее отклонение числа запросов от 
среднего уровня; $\sigma^z$~--- среднее квадратическое отклонение этого возмущения 
(далее предполагается, что $\{w_t^z\}$~--- стандартный дискретный белый шум второго 
порядка).
  
  Перейдем далее к формированию критерия оптимизации, т.\,е.\ к постановке задачи 
определения текущего размера пула запросов.
  
  Пусть известно среднее время~$T$ выполнения одного запроса. Для максимального 
удовлетворения потребностей пользователей текущий размер пула $u\hm=u_t$ следует 
положить равным $(l(t,t+1)/T) z_{t+1}$, где $l(t,t+1)$~--- длина интервала наблюдения 
$(t,t+1]$. Проблема, однако, состоит в том, что на момент~$t$ принятия решения об 
изменении размера пула точное значение $z_{t+1}$ еще не известно, поэтому можно было 
бы использовать некоторый прогноз~$\overline{z}_{t+1}$. Таким образом, первый 
компонент целевой функции в задаче управления размером пула можно представить в виде 
штрафа за ошибку прогнозирования
  \begin{equation}
  \mathbf{M}\left[ \left( z_{t+1}-\overline{z}_{t+1}\right)^2\right]\,.
  \label{e5bos}
  \end{equation}


  Целевая функция~(\ref{e5bos}) исходит из потребностей только одной 
(пользовательской) стороны, поскольку не включает штраф за выработанное управ\-ля\-ющее 
воздействие. Если пользоваться стратегией оптимизации вида 
\begin{equation}
u_t=\fr{l(t,t+1)}{T}\,\overline{z}_{t+1}\,,
\label{e6-1bos}
\end{equation}
то на каждом шаге наблюдений размер пула с 
большой вероятностью будет меняться, но изменения эти будут незначительны, а значит, и 
не скажутся принципиально на эффективности работы портала в целом, а только усложнят 
реализуемое управление. Кроме того, такое определение размера пула приведет к его 
потенциально бесконечному росту с увеличением текущего числа пользователей и в 
конечном итоге~--- к исчерпанию всех ресурсов обслуживающих систем. 
Усовершенствовать целевую функцию можно путем дополнения~(\ref{e5bos}) штрафными 
аддитивными слагаемыми, вначале для учета штрафа за изменение размера пула:
  \begin{equation}
  \mathbf{M}\left[\left( z_{t+1}-G_t u_t\right)^2+\left( u_t-u_{t-1}\right)^2\right]\,,
  \label{e6bos}
  \end{equation}
а затем ввести в~(\ref{e6bos}) <<плату>> и за общий размер пула:
\begin{equation}
\mathbf{M}\left[\left( z_{t+1}-G_t u_t\right)^2+\left( u_t-u_{t-1}\right)^2+u_t^2\right]\,.
\label{e7bos}
\end{equation}


  
  В~(\ref{e6bos}) и~(\ref{e7bos}) весовые коэффициенты~$G_t$ уместно задать, исходя из 
<<пользовательского>> варианта управления~(\ref{e6-1bos}), т.\,е.\ в 
виде $G_t\hm= T/l(t,t+1)$.
  
  Для придания (\ref{e7bos}) окончательного вида предположим, что задан горизонт 
оптимизации~$N$ (например, сутки или неделя) и дополним выбранные слагаемые 
весовыми коэффициентами. Окончательно получаем целевую функцию следующего вида:
  \begin{multline}
  \mathbf{J}(u_0,\ldots ,u_N)=\sum\limits_{t=0}^N \mathbf{M}\left[ S_t^{(1)}\left( z_{t+1}-
G_t u_t\right)^2+{}\right.\\
\left.{}+S_t^{(2)}\left( u_t-u_{t-1}\right)^2+S_t^{(3)} u_t^2\right]\,.
  \label{e8bos}
  \end{multline}



\section{Формальная постановка и~решение задачи управления размером пула}
  
  Всюду далее будем предполагать, что $\{v_t\}$ из~(\ref{e1bos})~--- стандартный 
дискретный белый шум в узком смысле, сечения которого имеют плот\-ность 
вероятности~$\varphi_v(\cdot)$; $x_0$~--- случайная величина, \mbox{имеющая} плот\-ность 
вероятности $\psi_0(\cdot)$; $\{w_t^y\}$ из~(\ref{e3bos})~--- стандартный дискретный белый 
шум в узком смыс\-ле, сечения которого имеют плотность ве\-ро\-ят\-ности~$\varphi_w^y(\cdot)$; 
$\{w_t^z\}$ из~(\ref{e4bos})~--- стандартный дискретный белый шум в узком смысле, 
сечения которого имеют плот\-ность вероятности $\varphi_w^z(\cdot)$; $\{v_t\}$, $x_0$, 
$\{w_t^y\}$, $\{w_t^z\}$ независимы в совокупности; $\mathbf{M}[v_t^2]\hm<\infty$; 
$\mathbf{M}[x_0^2]\hm<\infty$; $\mathbf{M}[(w_t^y)^2]\hm<\infty$; 
$\mathbf{M}(w_t^y)^2]\hm<\infty$; параметры $b_k$, $k\hm=1, \ldots ,n$, $\sigma^y$ и 
$\sigma^z$ неотрицательны.
  
  Будем предполагать также, что параметры целевого функционала~(\ref{e8bos}) 
$S_t^{(1)}$, $S_t^{(2)}$, $S_t^{(3)}$, $G_t$~--- известные неотрицательные функции~$t$; 
класс допустимых управлений~$U_t$ содержит все $\mathfrak{J}_t^y$-из\-ме\-ри\-мые функции 
$u_t:\ \mathbf{M}[u_t^2]\hm<\infty$. Таким образом, целью оптимизации является поиск 
закона изменения размера пула~$u_t^*$, удовлетворяющего потребности в ресурсах, 
описываемых выходом~$z_{t+1}$, с минимизацией затрат на управляющее воздействие, на 
изменение управляющего воздействия на текущем шаге и на всех последующих шагах 
вплоть до заданного горизонта~$N$:
  \begin{multline}
\mathrm{col} \left( u_0^*, \ldots ,u_N^*\right) ={}\\
  {}=\argmin\limits_{(u_0,\ldots , u_N)\in U_0 \cdots U_N} 
\mathbf{J}\left( u_0,\ldots ,u_N\right).
  \label{e9bos}
  \end{multline}
  
  \noindent
  \textbf{Теорема.}
  \textit{Пусть для целевого функционала}~(\ref{e8bos}) \textit{выполнено: 
$S_t^{(2)}\hm>0{,}1\hm\leq t\hm\leq N$, $S_0^{(2)}\hm=0$. Тогда решение $u_t^*$ задачи 
оптимизации}~(\ref{e9bos}) \textit{существует и определяется соотношениями:}
  \begin{equation}
  \hspace*{-2.7mm}\left.
\begin{array}{rl}
u_t^* &=\displaystyle\fr{1}{R(t)}\left ( S_t^{(2)} u_{t-1}^*+\!\!\!\sum\limits_{j=1}^{N-t+1}\!\! 
Q_j(t)\overline{z}_{t+j,t}\right);\\[9pt]
R(t) &=S_t^{(2)}+S_t^{(3)}+S_t^{(1)}G_t^2+{}\\[9pt]
&\hspace*{5mm}{}+S_{t+1}^{(2)}-
\fr{(S_{t+1}^{(2)})^2}{R(t+1)}\,,\quad  0\leq t<N\,;\\
R(N) &=S_N^{(2)}+S_N^{(3)}+S_N^{(1)}G_N^2;\\[9pt]
Q_j(t) &=\fr{S_{t+1}^{(2)}}{R(t+1)}\,Q_{j-1}(t+1);\\[9pt]
Q_1(t)&=S_t^{(1)}G_t\,,\quad  0\leq t<N\,,\\
&\hspace*{26mm} 1\leq j\leq N-t+1;\\[9pt]
Q_1(N) &=S_N^{(1)} G_N,
  \end{array}\!\!
  \right\}\!\!\!
  \label{e10bos}
  \end{equation}
\textit{где $\overline{z}_{t+j,t}$~--- оптимальные в среднем квадратическом прогнозы 
выхода $z_{t+j}$ по наблюдениям~$y_\tau$, $\tau\hm\leq t$.}

\smallskip

\noindent
  Д\,о\,к\,а\,з\,а\,т\,е\,л\,ь\,с\,т\,в\,о\,.\ Для решения задачи оптимизации~(\ref{e9bos}) 
воспользуемся методом динамического программирования~\cite{8bos, 7bos}. Обозначим 
через
  \begin{multline*}
  B(t) \overset{\Delta}{=}{}\\
{}\overset{\Delta}{=}   \min\limits_{(u_t,\ldots ,u_N)\in U_t\cdots 
U_N}\sum\limits_{\tau=t}^N \mathbf{M}\left[ S_\tau^{(1)}\left( z_{\tau+1}-G_\tau 
u_\tau\right)^2+{}\right.\\
\left.{}+ S_\tau^{(2)}\left (u_\tau-u_{\tau-1}\right)^2+S_\tau^{(3)}u_\tau^2\vert 
\mathfrak{J}_t^y\right]
  \end{multline*}
функцию Беллмана. При $t=N$ утверждение теоремы очевидно, так как  из выражения
\begin{multline*}
B(N) =\min\limits_{u_N\in U_N} \mathbf{M}\left[
S_N^{(1)}\left( z_{N+1}-G_Nu_N\right)^2+{}\right.\\
\left.\left.{}+S_N^{(2)}\left( u_N-u_{N-1}\right)^2+S_N^{(3)}u_N^2
\right\vert
\mathfrak{J}_N^y\right]
\end{multline*}
после очевидных преобразований с учетом $\mathfrak{J}_N^y$-из\-ме\-ри\-мости 
функций~$u_N$ и~$u_{N-1}$, а также обозначения $\overline{z}_{N+1,N}\hm = \mathbf{M} 
\left[ z_{N+1}\vert \mathfrak{J}_N^y\right]$ получаем:

\noindent
\begin{multline*}
u_N^* =\argmin\limits_{u_N\in U_N} \left( \left( S_N^{(2)}+S_N^{(3)} +S_N^{(1)}G_N^2\right) 
u_N^2- {}\right.\\
{}-2\left( S_N^{(2)} u_{N-1} +S_N^{(1)} G_t \overline{z}_{N+1,N}\right) u_N+{}\\
\left.{}+ S_N^{(2)} u^2_{N-1}+S_N^{(1)}\mathbf{M}\left[ z^2_{N+1}\vert 
\mathfrak{J}_N^y\right]\right)\,,
\end{multline*}
откуда с учетом независимости последних двух слагаемых от~$u_N$ и положительности 
коэффициента при $u_N^2$ получаем:
\begin{multline*}
u_N^*=\fr{S_N^{(2)} u_{N-1}+S_N^{(1)} G^2_N 
\overline{z}_{N+1,N}}{S_N^{(2)}+S_N^{(3)}+S_N^{(1)}G_N^2}={}\\
{}=\fr{1}{R(N)}\left(S_N^{(2)}
u_{N-1}+Q_1(N)\overline{z}_{N+1,N}\right)\,.
\end{multline*}

  Кроме того, получаем и выражение для функции Беллмана:
  \begin{multline*}
  B(N) =S_N^{(2)} u_{N-1}^2 - \fr{1}{R(N)} \left( S_N^{(2)} u_{N-1} +{}\right.\\
\left.  {}+Q_1(N) 
\overline{z}_{N+1,N}
\vphantom{S_N^{(2)}}\right)^2+\mathbf{M}\left[ A(N)\vert \mathfrak{J}_N^y\right]\,,
  \end{multline*}
где обозначено  $A(N)\hm= S_N^{(1)} z^2_{N+1}$.
  
  Предположим, что утверждение теоремы выполнено для $u^*_{t+1}$ и для функции 
Беллмана $B(t+1)$ имеет место следующее выражение:
  \begin{align*}
  B(t+1) &=S_{t+1}^{(2)} u_t^2-{}\\
 &\hspace*{-13mm} {}- \fr{1}{R(t+1)}\left( 
  S^{(2)}_{t+1} u_t 
+\sum\limits_{j=1}^{N-t} Q_j
  (t+1)\overline{z}_{t+j+1,t+1}\right)^2+{}\\
 & \hspace*{33mm}{}+ \mathbf{M}\left[ A(t+1)\vert 
\mathfrak{J}^y_{t+1}\right]\,;\\
  A(t+1) &= A(t+2) +S_{t+1}^{(1)}  z_{t+2}^2-{}\\
&  {}-
  \fr{1}{R(t+2)}\left(\sum\limits_{j=1}^{N-t-1} Q_j(t+2) \overline{z}_{t+j+2,t+2}\right)^2\,.
  \end{align*}
  Тогда уравнение Беллмана для~$B(t)$ имеет сле\-ду\-ющий вид:
  \begin{multline*}
  B(t) =\min\limits_{u_t\in U_t} \mathbf{M}\left[
  S_t^{(1)}\left( y_{t+1}^{(2)}-G_t u_t\right)^2+{}\right.\\
  {}+S_t^{(2)}\left( u_t-u_{t-1}\right)^2+S_t^{(3)} u_t^2+S_{t+1}^{(2)} u_t^2-{}\\
  {}-\fr{\left(S_{t+1}^{(2)}u_t+\sum\limits_{j=1}^{N-t} 
Q_j(t+1)\overline{z}_{t+j+1,t+1}\right)^2}{R(t+1)}+{}\\
\left.\left.  {}+\mathbf{M}[A(t+1)\vert \mathfrak{J}^y_{t+1}]
  \right\vert
  \mathfrak{J}_t^y
  \vphantom{S_t^{(1)}\left( y_{t+1}^{(2)}-G_t u_t\right)^2}\right].
  \end{multline*}

  Преобразовав это уравнение с учетом $\mathfrak{J}_t^y$-из\-ме\-ри\-мости функций 
$u_{t-1}$ и~$u_t$, запишем:
  \begin{multline*}
  B(t) =\min\limits_{u_t\in U_t} \left[
  \vphantom{\fr{\left(S^{(2)}_{t+1}\right)^2}{R(t+1)}}
  \left( \vphantom{  \fr{\left(S^{(2)}_{t+1}\right)^2}{R(t+1)}}
  S_t^{(1)} G_t+S_t^{(2)}+S_t^{(3)}+{}\right.\right.\\
\left.  {}+S^{(2)}_{t+1}-
  \fr{\left(S^{(2)}_{t+1}\right)^2}{R(t+1)}\right) u_t^2-{}\\
  {}-2\left( 
  \vphantom{\sum\limits_{j=1}^{N-t}}
  S_t^{(1)}G_t +S_t^{(2)}u_{t-1}+\fr{S_{t+1}^{(2)}}{R(t+1)}\times{}\right.\\
\left.  {}\times\sum\limits_{j=1}^{N-t}Q_j(t+1)\mathbf{M}[\overline{z}_{t+j+1,t+1}\vert \mathfrak{J}_t^y]\right) u_t+{}\\
  {}+ S_t^{(2)} u^2_{t-1}+\mathbf{M}\left[ S_t^{(2)} z_{t+1}^2-{}\right.\\
\left.  {}-\fr{1}{R(t+1)}\left( 
\sum\limits_{j=1}^{N-t}Q_j(t+1) \overline{z}_{t+j+1,t+1}\right)^2+{}\right.\\
\left.\left.{}+\mathbf{M}[A(t+1)\vert 
\mathfrak{J}^y_{t+1}]
  \vert
  \mathfrak{J}_t^y
\vphantom{S_t^{(2)} z_{t+1}^2}  \right]
  \vphantom{\fr{\left(S^{(2)}_{t+1}\right)^2}{R(t+1)}}
  \right]\,.
  \end{multline*}
  
  Применив в последнем выражении формулу полного математического ожидания и 
введенные в~(\ref{e10bos}) обозначения, получим:
  \begin{multline*}
  B(t)=\min\limits_{u_t\in U_t}\left[ 
  \vphantom{S_t^{(2)} u^2_{t-1}}
  R(t) u_t^2-{}\right.\\
 {}-2\left( %\vphantom{\sum\limits_{j=1}^{N-t+1}}
  S_t^{(2)}u_{t-1}+ \sum\limits_{j=1}^{N-t+1} 
Q_j(t)\overline{z}_{t+j,t}\right)u_t+{}\\
\left.  {}+S_t^{(2)} u^2_{t-1}+\mathbf{M}\left[A(t)\vert \mathfrak{J}_t^y\right]
  \right]\,.
  \end{multline*}

  Поскольку в полученном соотношении два последних слагаемых не зависят от~$u_t$, то в 
предположении положительности~$R(t)$ получается доказываемое 
соотношение~(\ref{e10bos}) для~$u_t^*$. Кроме того, подстановкой $u_t^*$ подтверждается 
справедливость индуктивного предположения относительно функции Беллмана.
  
  Для завершения доказательства остается показать, что $R(t)\hm>0$. Для этого достаточно 
показать, что $S_{t+1}^{(2)}-\left(S^{(2)}_{t+1}\right)^2\!\Big/R(t+1)\hm\geq0$. В~свою очередь, последнее 
неравенство верно, если $S_{t+2}^{(2)} -\left(S_{t+2}^{(2)}\right)^2\!\Big/R(t+2)\hm\geq 0$ и $R(t+1)\hm>0$. 
Выполнение, таким образом, индуктивного предположения вытекает из того, что 
$R(N)\hm>0$ и  $S_N^{(2)}-\left(S_N^{(2)}\right)^2\!\Big/R(N)\hm\geq0$. Теорема доказана.
  \medskip
  
  \noindent
  \textbf{Замечание.} В~полученном утверждении используются оптимальные в среднем 
квадратическом прогнозы $\overline{z}_{t+j,t}$ выхода~$z_{t+j}$ по наблюдениям~$y_\tau$, 
$\tau\hm\leq t$, $j\hm=1,2,\ldots$ В~теореме~2 работы~[1] получены аналогичные прогнозы 
для~$y_{t+j}$. Учитывая тривиальность уравнения~(\ref{e4bos}), нетрудно увидеть, что 
соответствующие соотношения для $\overline{z}_{t+j,t}$ имеют вид:
  \begin{align*}
  \overline{z}_{t+j,t} &=\\
  &\hspace*{-17pt}{}=\sum\limits_{k=1}^n \int\limits_{\Delta_k} c^y c^z (a_k\xi 
+q_k)\overline{\psi}_x (\xi, t,j)\,d\xi\,,\enskip j=2,3,\ldots\,;\\
  \overline{z}_{t+1,t} &=\sum\limits_{k=1}^n \int\limits_{\Delta_k} c^y c^z 
  (a_k\xi +q_k)\hat{\psi}_x (\xi, t)\,d\xi\,,
  \end{align*}
где прогнозирующие плотности вероятности определяются соотношениями:
\begin{align*}
\overline{\psi}_x (x,t,j)&=\sum\limits_{k=1}^n 
\fr{1}{b_k}\int\limits_{\Delta_k}\overline{\psi}_x(\xi,t,j-1){}\times\\[1pt]
&{}\times\varphi_v\left( \fr{x-a_k\xi- q_k}{b_k}\right)d\xi\,,\ j=2,3,\ldots\,;\\[3pt]
\overline{\psi}_x (x,t,1)&=
\sum\limits_{k=1}^n \fr{1}{b_k}\!\int\limits_{\Delta_k}\hat{\psi}_x(\xi,t)\varphi_v\left( \fr{x-
a_k\xi-q_k}{b_k}\right)d\xi;\\[3pt]
\hat{\psi}_x(x,t) &=\left(
\vphantom{\int\limits_{\Delta_k}}\varphi_{w^y}\left( \fr{y_t-c^yx}{\sigma^y}\right)
\sum\limits_{k=1}^n 
\fr{1}{b_k}\times{}\right.\\[1pt]
&\hspace*{-49.25456pt}\left.{}\times\! \int\limits_{\Delta_k}\! \hat{\psi}_x(\xi,t-1)\varphi_v
\left( \fr{x-a_k\xi-
q_k}{b_k}\right)d\xi\right)\!\!\Bigg /\!\!
\left(\sum\limits_{k=1}^n \fr{1}{b_k}\times{}\right.\\[1pt]
&{}\times\int\limits_{R^1}
\varphi_{w^y}\left( \fr{y_t-
c^yx}{\sigma^y}\right)\int\limits_{\Delta_k}\hat{\psi}_x(\xi,t-1)\times{}\\[1pt]
&\hspace*{10mm}\left.{}\times\varphi_v\left( \fr{x-a_k\xi-
q_k}{b_k}\right ) d\xi\,dx
\vphantom{\sum\limits_{k=1}^{n}}
\right)\,.
\end{align*}


\section{Субоптимальные управления}
  
  Основной проблемой применения оптимальной стратегии оптимизации размера 
пула~(\ref{e10bos}) является определение горизонта~$N$. 

Нетрудно видеть, что выбор 
больших значений~$N$ приводит к необхо\-ди\-мости вычисления большого числа прогнозов, 
особенно на первых шагах алгоритма. Расчет же прогнозов является весьма вычислительно 
затратным. Малые же значения~$N$ не обеспечат должного учета динамического характера 
задачи, модельная и априорная информация окажутся фактически невостребованными. Для 
преодоления данного противоречия предлагается использовать принцип локально-оп\-ти\-маль\-но\-го 
(адаптивного) управ\-ле\-ния~\cite{9bos}. Сохранив целевую функцию в 
виде~(\ref{e8bos}), определим в качестве субоптимального решения задачи ее минимизации 
функцию~$u_t^L$, доставляющую минимум функционалу
  \begin{multline*}
  \mathbf{J}_t(u_t) =\mathbf{M}\left[ S_t^{(1)}(z_{t+1}-G_t u_t)^2+{}\right.\\[2pt]
  {}+S_t^{(2)}(u_t-u_{t-
1})^2+S_t^{(3)}u_t^2+{}\\[2pt]
{}+S^{(1)}_{t+1}(z_{t+2}-G_{t+1} u_{t+1})^2+S_{t+1}^{(2)}(u_{t+1}-
u_t)^2+{}\\[2pt]
\left.{}+S^{(3)}_{t+1} u^2_{t+1}\right]\,.
  \end{multline*}

  Таким образом, для локально-оптимального решения рассматриваемой задачи 
оптимизации предлагается двухшаговый вариант целевой функции вида~(\ref{e8bos}), 
обновляемый на каждом следующем шаге алгоритма. 

В~целевую 
функцию~$\mathbf{J}_t(u_t)$, как легко видеть, включены штрафы за ошибку 
прогнозирования, за изменения размера пула и за собственно размер пула на текущем и 
следующем шаге.
  
  Отметим, что использование локально-оп\-ти\-маль\-ных критериев управления в качестве 
подходящей альтернативы интегральным критериям показало свою эффективность в 
традиционных задачах управ\-ле\-ния состоянием стохастической динамической 
системы~\cite{10bos, 11bos}.
  
  Требующееся выражение для функции $u_t^L\hm=\argmin\limits_{u_t\in U_t}
\mathbf{J}_t(u_t)$ получаем непосредственно как частный случай~(\ref{e10bos}):
  \begin{multline}
  u_t^L= \left(\vphantom{\fr{S^{(1)}_{t+1}S^{(2)}_{t+1}G_{t+1}}{S^{(2)}_{t+1}+S^{(3)}_{t+1}+S_t^{(1)}G^2_{t+1}}}
  S_t^{(2)} u^L_{t-1}+S_t^{(1)} G_t \overline{z}_{t+1,t}+ {}\right.\\[2pt]
\hspace*{-2mm}\left.  {}+
\fr{S^{(1)}_{t+1}S^{(2)}_{t+1}G_{t+1}}{S^{(2)}_{t+1}+S^{(3)}_{t+1}+S_t^{(1)}G^2_{t+1}}\,
\overline{z}_{t+2,t}   \right)\!\Bigg/\!
    \left(
\vphantom{\fr{\left(S^{(2)}_{t+1}\right)^2}{S^{(2)}_{t+1}+S^{(3)}_{t+1}+S^{(1)}_{t+1}G^2_{t+1}}}
  S_t^{(2)}+S_t^{(3)}+{}\right.\\[2pt]
 \hspace*{-2.5mm} \left.{}+S_t^{(1)}G_t^2+S^{(2)}_{t+1}-
  \fr{\left(S^{(2)}_{t+1}\right)^2}{S^{(2)}_{t+1}+S^{(3)}_{t+1}+S^{(1)}_{t+1}G^2_{t+1}}
  \vphantom{}
  \right).\!
  \label{e11bos}
  \end{multline}
  


  Наконец, самым простым вариантом возможного решения рассматриваемой задачи 
оптимизации является использование оптимальной програм\-мной стратегии~$u_t^P$. Такое 
<<усредненное>> решение можно записать в виде:
  \begin{equation}
  u_t^P=\mathbf{M}[u_t^*]\,.
  \label{e12bos}
  \end{equation}



\section{Результаты численных экспериментов}
  
  Для сравнения предложенных алгоритмов оптимизации использован модельный пример 
из~\cite{1bos}, незначительно измененный для учета дополнительного 
уравнения~(\ref{e4bos}) выхода (так, чтобы приведенные в~\cite{1bos} результаты 
прогнозирования сохранялись и здесь). Были заданы три интервала $\Delta_1\hm=(-\infty;3]$, 
$\Delta_2\hm=(3;7]$, $\Delta_3\hm=(7;+\infty)$ и следующие параметры 
уравнения~(\ref{e1bos}):
  
\noindent
  \begin{center}
  \begin{tabular}{|c|c|c|c|c|c|c|c|c|}
  \hline
$a_1$&$a_2$&$a_3$&$q_1$&$q_2$&$q_3$&$b_1$&$b_2$&$b_3$\\
\hline
0,3&0,4&0,7&1,4&3,0&3,0&0,9&1,5&2,5\\
\hline
\end{tabular}
\end{center}

\medskip


  Параметры наблюдений~(\ref{e3bos}): $c^y\hm=1{,}0$, $\sigma^y\hm=3{,}0$, параметры 
выхода~(\ref{e4bos}): $c^z\hm=3{,}0$, $\sigma^z\hm=1{,}0$. Распределения всех 
возмущений~--- стандартные гауссовские, распределение начального условия~$x_0$ также 
предполагалось гауссовским со средним и дис\-пер\-си\-ей, равными соответствующим моментам 
предельного распределения (подробнее см.~\cite{1bos}).
  
  Расчеты проводились для 10~шагов траектории: $t\hm=1, 2,\ldots ,10$, для вычисления 
значения целевой функции использовался пучок из 10\,000 траекторий. Параметры целевой 
функции~(\ref{e8bos}) выбраны следующим образом:
  \begin{gather*}
  S_t^{(2)}=0{,}1\,,\enskip 1\leq t\leq 10\,;\\
  S_t^{(1)}=S_t^{(3)}=0{,}1\,,\enskip 0\leq t\leq 10\,;\\
  G_0=0{,}0\,;\enskip G_1=G_2=G_3=0{,}1\,;\\
  G_4=G_5=G_6=1{,}0\,;\  G_7=G_8=G_9=G_{10}=4{,}0\,.
  \end{gather*}

  Для сравнения качества стратегий оптимизации~(\ref{e10bos})--(\ref{e12bos}) 
  вычислялось значение целевой функции в каждый момент~$t$, т.\,е.\ 
$\mathbf{J}(u_0,\ldots ,u_t)$.
  
  Результаты расчетов приведены на рис.~1 и~2. Рисунок~1 иллюстрирует характерные 
траектории компонентов системы наблюдения (1)--(4) и функций~$u_t^*$, $u_t^L$ и $u_t^P$.  
На рис.~2 приведены значения целевой функции: $\hat{J}_1$ (для~$u_t^*$), $\hat{J}_2$ (для 
$u_t^L$) и $\hat{J}$ (для $u_t^P$).
  

Из приведенных результатов видно, что значения целевой функции, достигаемые при 
использовании оптимального решения~$u_t^*$ и локально-оп\-ти\-маль\-но\-го~$u_t^L$, 
практически совпадают. 

Преимущество алгоритма, использующего стратегию $u_t^*$, в 
сравнении с $u_t^L$ в терминах целевой функции~(\ref{e8bos}) составляет не более 1,5\% к 
моменту достижения горизонта оптимизации. При этом очевидно, что вы\-чис\-ли\-тель\-ные 
трудности, возникающие в связи с необходимостью расчета большого числа прогнозов, 
весьма велики, и\linebreak\vspace*{-12pt}

\pagebreak

\end{multicols}

\begin{figure} %fig1
 \vspace*{1pt}
 \begin{center}
 \mbox{%
 \epsfxsize=157.746mm
 \epsfbox{bos-1.eps}
 }
 \end{center}
 \vspace*{-9pt}
\Caption{Характерные траектории: \textit{1}~--- $x_t$; \textit{2}~--- $y_t$;
\textit{3}~--- $z_t$; \textit{4}~--- $u_t^*$; \textit{5}~--- $u_t^L$;
\textit{6}~--- $u_t^P$}
\vspace*{9pt}
\end{figure}

\begin{multicols}{2}

\begin{center} %fig2
\vspace*{2pt}
\mbox{%
  \epsfxsize=72.509mm
 \epsfbox{bos-2.eps}
}
\end{center}
%\begin{center}
\vspace*{3pt}
{{\figurename~2}\ \ \small{Оценки целевых функций: \textit{1}~--- $\hat{J}_1$ 
(для~$u_t^*$); \textit{2}~--- $\hat{J}_2$ (для 
$u_t^L$); \textit{3}~--- $\hat{J}$ (для $u_t^P$)}}
%\end{center}
\vspace*{24pt}

%\smallskip
\addtocounter{figure}{1}

\noindent
можно считать, таким образом, что наиболее целесообразно использование 
ло\-каль\-но-оп\-ти\-маль\-но\-го решения. Также можно отметить, что
 программное 
(<<усредненное>>) решение хотя и проигрывает оптимальным алгоритмам, но сохраняет 
приемлемое качество. При этом следует указать на тенденцию роста преимущества 
оптимальных решений с удалением горизонта оптимизации, т.\,е.\ с ростом~$N$. Таким 
образом, практическое применение $u_t^P$ в этой задаче вряд ли целесообразно, так как на 
практике интерес представляют расчеты с большим~$N$.

\section{Заключение}
  
  В статье продолжено исследование модели пользовательской активности, предложенной 
в~\cite{1bos}. Стохастическая динамическая система наблюдения~(1)--(3), описывающая 
эволюцию числа пользователей, взаимодействующих с некоторой информационной 
системой, и косвенные наблюдения\linebreak за ними~--- чис\-ло поступивших для выполнения 
команд, расширено уравнением выхода~(\ref{e4bos}), описывающим число сформированных 
из поступивших\linebreak команд запросов. С~текущим значением выхода связана величина, 
описывающая объем вы\-де\-ля\-емых сис\-те\-мой вычислительных ресурсов. В~рас\-смот\-рен\-ной 
прикладной постановке это размер\linebreak пула, состоящего из множества очередей, 
поддерживаемых с целью параллельного выполнения формиру\-емых запросов. 
В~соответствии с предложенной терминологией такой ресурс назван <<внут\-рен\-ним>>, так 
как его обслуживание полностью контролируется самой программной системой.
  
  Основная решенная задача состоит в оптимизации в процессе работы программы размера 
пула на основе квадратичного критерия качества, позволяющего учесть пользовательские 
потребности, расходы на управление пулом и его размер. В~целом использованный подход 
вполне соответствует классической задаче динамического управления по квадратичному 
критерию качества, в том числе  и по использованному аппарату динамического 
программирования. Предложенная постановка, однако, имеет существенное отличие: 
управляющее воздействие не влияет на фазовый процесс (текущее число пользователей) и 
связано с последним только косвенно, через критерий оптимизации. Такой\linebreak результат вполне 
согласуется с физическим смыс\-лом рассматриваемой задачи~--- определить наилучший 
алгоритм расходования вычислительных\linebreak ресурсов на основании анализа состояния 
окружающей \mbox{среды}.
  
  Аналогичный подход в последующих работах предполагается реализовать и для 
<<внешних>> ресурсов, выделяемых программе обслуживающими системами.

{\small\frenchspacing
{%\baselineskip=10.8pt
\addcontentsline{toc}{section}{Литература}
\begin{thebibliography}{99}

\bibitem{1bos}
\Au{Босов А.\,В.} Задачи анализа и оптимизации для модели пользовательской активности. 
Часть~1. Анализ и прогнозирование~// Информатика и её применения, 2011. Вып.~4. Т.~5. 
С.~40--52.

\bibitem{2bos}
Информационный Веб-портал: Свидетельство об официальной регистрации программы для 
ЭВМ №\,2005612992 от 18.11.2005.

\bibitem{3bos}
\Au{Kelly F.\,P., Maulloo A., Tan D.}
Rate control in communication networks: Shadow prices, proportional fairness and stability~// J. 
Operational Research Soc., 1998. Vol.~49. P.~237--252.

\bibitem{4bos}
\Au{Low S.\,H., Paganini F., Doyle J.\,C.}
Internet congestion control~// IEEE Control Syst. Magazine, 2002. Vol.~22. No.\,1. P.~28--43.

\bibitem{5bos}
\Au{Миллер Б.\,М., Миллер Г.\,Б., Семенихин К.\,В.}
Методы синтеза оптимального управления марковским процессом с конечным множеством 
состояний при наличии ограничений~// Автоматика и телемеханика, 2011. №\,2. С.~111--130.

\bibitem{6bos}
\Au{Босов А.\,В.}
Моделирование и оптимизация процессов функционирования Информационного 
веб-пор\-та\-ла~// Программирование, 2009. №\,6. С.~53--66.


\bibitem{8bos}
\Au{Флеминг У., Ришел Р.} Оптимальное управление детерминированными и 
стохастическими системами.~--- М.: Мир, 1978.

\bibitem{7bos}
\Au{Бертсекас Д., Шрив С.}
Стохастическое оптимальное управление.~--- М.: Наука, 1985.


\bibitem{9bos}
\Au{Коган М.\,М., Неймарк Ю.\,И.}
Адаптивное локально-оп\-ти\-маль\-ное управление~// Автоматика и телемеханика, 1987. №\,8. 
С.~126--136.

\bibitem{10bos}
\Au{Босов А.\,В., Панков А.\,Р.}
Алгоритмы управления в системах с переключающимися каналами наблюдения~// Изв. РАН. 
Сер. Теория и системы управления, 1996. №\,2. С.~98--103.

\label{end\stat}

\bibitem{11bos}
\Au{Босов А.\,В., Панков А.\,Р.}
Алгоритмы управления для дискретных систем случайной структуры~// Автоматика и 
телемеханика, 1997. №\,10. С.~113--125.
 \end{thebibliography}
}
}


\end{multicols}       

  


  