\def\stat{ushakov}

\def\tit{О ВИРТУАЛЬНОМ ВРЕМЕНИ ОЖИДАНИЯ В~СИСТЕМЕ С~ОТНОСИТЕЛЬНЫМ ПРИОРИТЕТОМ 
И~ГИПЕРЭКПОНЕНЦИАЛЬНЫМ ВХОДЯЩИМ ПОТОКОМ$^*$}

\def\titkol{О виртуальном времени ожидания в~системе с~относительным приоритетом} 
%и~гиперэкпоненциальным входящим потоком}

\def\autkol{А.\,В.~Ушаков}
\def\aut{А.\,В.~Ушаков$^1$}

\titel{\tit}{\aut}{\autkol}{\titkol}

{\renewcommand{\thefootnote}{\fnsymbol{footnote}}\footnotetext[1]
{Работа выполнена при финансовой поддержке РФФИ (грант 11-07-00112а).}}


\renewcommand{\thefootnote}{\arabic{footnote}}
\footnotetext[1]{Институт проблем информатики Российской академии наук, grimgnau@rambler.ru}

\vspace*{6pt}

\Abst{Найдены преобразования Лапласа--Стилтьеса 
виртуальных времен ожидания в одноканальной
системе обслуживания с относительным приоритетом  и рекуррентным входящим 
потоком с гиперэкспоненциальным распределением
интервалов между поступлениями требований.}

\vspace*{2pt}


\KW{виртуальное время ожидания; относительный приоритет; гиперэкспоненциальный поток}

 \vskip 16pt plus 9pt minus 6pt

      \thispagestyle{headings}

      \begin{multicols}{2}
      
            \label{st\stat}


\section{Введение}

При проектировании и анализе функционирования
инфотелекоммуникационных сис\-тем в качест\-ве математических моделей
наиболее часто используются сис\-те\-мы и сети массового обслуживания. 
С~учетом сложности таких систем возникает необходимость разработки
математических методов анализа систем обслуживания, учитывающих
многие факторы: ненадежность каналов, нетерпеливость требований,
наличие требований различной важности и~т.\,д.

В работах~[1--4] разработаны методы анализа одноканальных сис\-тем массового обслуживания
с~приоритетами, не допускающими прерывания
уже начатого обслуживания (относительный приоритет, чередование приоритетов) 
и различными классами рекуррентных входящих потоков. Иссле\-довано
поведение одной из наиболее важных характеристик~--- вектора длин очередей из 
требований различных приоритетов. В~настоящей статье методы~[1--4] 
применены к анализу другой, не менее важной, характеристики~--- времени ожидания 
начала обслуживания.

\section{Описание системы}


Рассматривается одноканальная сис\-те\-ма обслуживания с $r$, $r\geqslant 1$, приоритетными 
классами требований. Длительности обслуживания~---
независимые в совокупности и не зависящие от входящего потока случайные величины с функцией 
распределения~$B_i(x)$ для требований $i$-го класса.

Входящий поток требований --- рекуррентный, определяемый плот\-ностью распределения 
интервалов между поступлениями требований вида
\begin{equation}
a(x)=
\begin{cases}
\sum\limits_{j=1}^{N}c_ja_j\exp\left(-a_jx\right)\,,&\  x\geqslant 0\,;\\
0\,,&\ x<0\,,
\end{cases}
\label{e1-u}
\end{equation}
где $a_i\ne a_j$ при $i\hm\ne j$, $c_j\hm>0$, $\sum\limits_{i=1}^{N}c_i\hm=1.$

Поступившее требование направляется в $i$-й приоритетный класс с вероятностью~$p_i$, 
$i=1,\ldots , r,$ независимо от остальных требований.
Рекуррентный входящий поток, задаваемый плотностью распределения~(\ref{e1-u}), 
эквивалентен следующему: интервалы времени между поступлениями
требований независимы в совокупности и показательно распределены 
со случайным параметром~$a$, принимающим значения~$a_i$ с вероятностями $c_i,\
i=1,\ldots,N,$ причем значение $a$ определяется непосредственно перед началом 
отсчета времени до следующего поступления и не меняется между двумя поступлениями. 
Событие $\{j(t)=j\}$ будет означать, что $a=a_j$ в момент времени $t.$

Будем предполагать, что требования из класса с меньшим номером имеют относительный 
приоритет перед требованиями из класса с б$\acute{\mbox{о}}$льшим номером.  Для 
требований одного класса будут рассмотрены две дисциплины обслуживания: прямой порядок 
обслуживания (дисциплина FIFO~--- first in, first out) и инверсионный порядок обслуживания (дисциплина LIFO~---
last in, first out). 
Пусть, кроме того, в начальный момент $t=0$ система свободна от требований.

\section{Основные обозначения и~определения}

Как уже было указано выше, функцию распределения времени обслуживания требований из $i$-го 
приоритетного класса будем обозначать $B_i(x)$. 
%
Пусть, далее, 
$b_i(x)$, $\beta_i(s)$ и $\beta_{ij}$~--- соответственно плотность распределения, 
преобразование Лапласа--Стилтьеса и $j$-й момент случайной величины с функцией 
распределения~$B_i(x).$

Введем следующие случайные процессы:
\begin{itemize}
\item
$w_i^{(0)}(t)$,  $i\hm=1,\ldots,N,$~--- виртуальное время ожидания в момент времени~$t$ для 
требований $i$-го приоритетного класса при дисциплине FIFO при условии, 
что после~$t$ требования 
в систему не поступают; $w_0^{(0)}(t)$~--- время с момента~$t$ до завершения обслуживания 
требования, находящегося в
этот момент на приборе (если в момент $t$ система свободна, то $w_0^{(0)}(t)=0$);
\item
$w_i^{(1)}(t)$ и $w_i^{(2)}(t)$, $i\hm=1,\ldots,N,$~--- виртуальные времена ожидания для 
требований $i$-го приоритетного класса в момент времени $t$  при
дисциплинах FIFO и LIFO соответственно.
\end{itemize}

Положим
\begin{align*}
W_{ij}^{(k)}(s,t)&=\int\limits_0^{\infty}e^{-sy}\,d_y\mathbf{P}(w_i^{(k)}(t)<y,j(t)=j)\,;
\\
\omega_{ij}^{(k)}(s,v)&=\int\limits_0^{\infty}e^{-vt}
W_{ij}^{(k)}(s,t)\,dt,\\\
&i=0,\ldots,r\,,\ j=1,\ldots,N\,,\ k=0,1,2\,;
\end{align*}
$P_{0j}(t)$~--- вероятность того, что в момент времени~$t$ система 
свободна и $j(t)=j$; $P_{mj}(t) \Delta\hm+o(\Delta)$~--- вероятность того, что в
интервале времени $[t,t+\Delta)$ началось обслуживание требования $m$-го приоритетного 
класса и $j(t)\hm=j$;
\begin{align*}
p_{0j}(v)&=\int\limits_0^{\infty}e^{-vt}P_{0j}(t)\,dt\,;\\
p_{mj}(v)&=\int\limits_0^{\infty}e^{-vt}P_{mj}(t)\,dt\,.
\end{align*}


\section{Вспомогательные результаты}

Функции $W_{ij}^{(0)}(s,t)$ удовлетворяют системе дифференциальных уравнений

\noindent
\begin{multline}
\label{e2-u}
\fr{\partial W_{ij}^{(0)}(s,t)}{\partial t}=(s-a_j) W_{ij}^{(0)}(s,t)-s P_{0j}(t)-{}\\
{}-
\sum\limits_{m=i+1}^r\left(1-\beta_m(s)\right) P_{mj}(t)+{}\\
{}+c_j \sum\limits_{k=1}^N a_k  W_{ik}^{(0)}(s,t) 
\left(\sum\limits_{m=1}^ip_m\beta_m(s)+\sum\limits_{m=i+1}^rp_m\right)\,,\\
i=0,\ldots,r\,,
\end{multline}
с начальным условием $W_{ij}^{(0)}(s,0)=c_j$.

Переходя в~(\ref{e2-u}) к преобразованиям Лапласа, получаем:
\begin{multline}
\label{e3-u}
\omega_{ij}^{(0)}(s,v)=\fr{c_j}{v-s+a_j}-\fr{s}{v-s+a_j} p_{0j}(v)-{}\\
{}-\sum\limits_{m=i+1}^r
\fr{1-\beta_m(s)}{v-s+a_j}\, p_{mj}(v)+\fr{c_j}{v-s+a_j}\times{}\\
\!\!{}\times
\sum\limits_{k=1}^N a_k \omega_{ik}^{(0)}(s,v)
\left(\sum\limits_{m=1}^ip_m\beta_m(s)+\sum\limits_{m=i+1}^rp_m\right).\!
\end{multline}
Умножая~(\ref{e3-u}) на $a_j$ и суммируя по $j$ от~1 до~$N$, находим:
\begin{multline}
\label{e4-u}
\left(1-\sum\limits_{j=1}^N\fr{c_ja_j}{v-s+a_j}\left(\sum\limits_{m=1}^ip_m\beta_m(s)+{}\right.\right.\\
\left.\left.{}+
\sum\limits_{m=i+1}^rp_m\right)\vphantom{\sum\limits_{j=1}^N}\right)
\sum\limits_{k=1}^N a_k\: \omega_{ik}^{(0)}(s,v)={}\\
{}= \sum\limits_{j=1}^N\fr{c_ja_j}{v-s+a_j}-s\sum\limits_{j=1}^N\fr{a_jp_{0j}(v)}{v-s+a_j}-{}\\
{}-
\sum\limits_{m=i+1}^r (1-\beta_m(s))\sum\limits_{j=1}^N\fr{a_jp_{mj}(v)}{v-s+a_j}\,.
\end{multline}

В первую очередь соотношения (\ref{e4-u}) используем для нахождения 
неизвестных функций~$p_{0j}(v)$ и~$p_{mj}(v)$.

Справедлива следующая лемма:

\medskip

\noindent
\textbf{Лемма 1.} \textit{При каждом $i=0,\ldots,r$ функциональное уравнение}
\begin{equation*}
%\label{e5-u}
\sum\limits_{j=1}^N\fr{c_ja_j}
{v-s+a_j}\left(\sum\limits_{m=1}^ip_m\beta_m(s)+\sum\limits_{m=i+1}^rp_m\right)=1
\end{equation*}
\textit{имеет $N$ решений $s=\alpha_{li}(v)$, $l=1,\ldots,N,$ аналитических в 
области~$\mathrm{Re}\,v>0$.}

\smallskip

Рассмотрим~(\ref{e4-u}) при $i=r$:
\begin{multline*}
\hspace*{-5mm}\left(\!1-\sum\limits_{j=1}^N\fr{c_ja_j}{v-s+a_j}
\sum\limits_{m=1}^rp_m\beta_m(s)\!\right)
\sum\limits_{k=1}^N a_k \omega_{ik}^{(0)}(s,v)={}\\
{}=
\sum\limits_{j=1}^N\fr{c_ja_j}{v-s+a_j}-s\sum\limits_{j=1}^N\fr{a_jp_{0j}(v)}{v-s+a_j}\,.
\end{multline*}
В силу леммы~1 левая часть последнего соотношения обращается 
в нуль при $s\hm=\alpha_{lr}(v)$, $l\hm=1,\ldots,N$. Отсюда
получаем систему линейных уравнений для определения функций $p_{0j}(v)$:
\begin{equation*}
\sum\limits_{j=1}^N\fr{a_j\:p_{0j}(v)}{v-\alpha_{lr}(v)+a_j}=
\alpha_{lr}^{-1}(v)\sum\limits_{j=1}^N\fr{c_ja_j}{v-\alpha_{lr}(v)+a_j},
\end{equation*}
из которой находим:
\begin{multline*}
%\label{e6-u}
a_k p_{0k}(v)=\sum\limits_{l=1}^N\sum\limits_{\nu=1}^N\fr{c_{\nu}a_{\nu}}
{\alpha_{lr}(v)(a_\nu+v-\alpha_{lr}(v))}\times{}\\
{}\times
\fr{\prod\limits_{j=1}^N((a_k+v-\alpha_{jr}(v))(a_j+v-\alpha_{lr}(v)))}
{(a_k+v-\alpha_{lr}(v)) \!\prod\limits_{n\ne l}(\alpha_{nr}(v)-\alpha_{lr}(v))
\prod\limits_{i\ne k}(a_k-a_i)},\\
k=1,\ldots,N\,.
\end{multline*}
Подставляя $s=\alpha_{li}(v)$ в~(\ref{e4-u}) последовательно при $i\hm=r-1,r-2,\ldots,1,0$,  
получаем системы линейных уравнений
\begin{equation*}
\sum\limits_{j=1}^N\fr{a_j p_{i+1j}(v)}{v-\alpha_{li}(v)+a_j}=f_{li}(v)\,,
\end{equation*}
где
\begin{multline*}
f_{li}(v)=(1-\beta_{i+1}(\alpha_{li}))^{-1}\left(\sum\limits_{j=1}^N\fr{c_ja_j}
{v-\alpha_{li}(v)+a_j}-{}\right.\\
{}-\alpha_{li}(v)\sum\limits_{j=1}^N\fr{a_jp_{0j}(v)}{v-\alpha_{li}(v)+a_j}-{}\\
\left.{}-\sum\limits_{m=i+2}^r
(1-\beta_m(\alpha_{li}(v)))\sum\limits_{j=1}^N\fr{a_jp_{mj}(v)}{v-\alpha_{li}(v)+a_j}\right)\,,
\end{multline*}
из которых находим систему рекуррентных соотношений для определения 
$p_{mj}(v)$, $m\hm=r,r-1,\ldots,1$:

\noindent
\begin{multline*}
%\label{e7-u}
a_kp_{i+1k}=\sum\limits_{l=1}^N f_{li}(v)\times{}\\
{}\times
\fr{\prod\limits_{j=1}^N((a_k+v-\alpha_{ji}(v))(a_j+v-\alpha_{li}(v)))}
{(a_k+v-\alpha_{li}(v))}\times{}
\end{multline*}

\noindent
\begin{multline*}
{}\times
\prod\limits_{n\ne l}(\alpha_{ni}(v)-\alpha_{li}(v))^{-1}
\prod\limits_{i\ne k}(a_k-a_i)^{-1}\,,\\
k=1,\ldots,N\,,\ i=r-1,\ldots,0\,.
\end{multline*}

После того как найдены все функции $p_{0j}(v)$ и $p_{mj}(v)$, $m\hm=1,\ldots,r$, 
$j\hm=1,\ldots,N,$ из~(\ref{e4-u}) находим
$\sum\limits_{k=1}^N a_k \omega_{ik}^{(0)}(s,v)$ и из~(\ref{e3-u}) 
$\omega_{ij}^{(0)}(s,v)$ для всех $i=0,1,\ldots,r$, $j\hm=1,\ldots,N$.

В дальнейшем понадобится совместная производящая функция чис\-ла требований всех 
приоритетных классов, поступивших в систему за заданное
время. Обозначим через $n(t)\hm=(n_1(t),\ldots,n_r(t))$ чис\-ло 
требований приоритетов $1,\ldots,r,$ поступивших в интервале $[0,t),$
\begin{multline*}
\!\!q_{\nu j}(z,t)=\mathbf{E}\left(z_1^{n_1(t)}\cdots z_r^{n_r(t)}I(j(t)=j)|j(0)=\nu\right)\,,\\ 
z=(z_1,\ldots,z_r)\,.
\end{multline*}
Функции $q_{\nu j}(z,t)$ удовлетворяют
следующей системе дифференциальных уравнений:
\begin{equation}
\label{e8-u}
\fr{\partial q_{\nu j}(z,t)}{\partial t}=-a_j q_{\nu j}(z,t)+c_j (p,z)
\sum\limits_{k=1}^N a_k q_{\nu k}(z,t)
\end{equation}
c начальным условием $q_{\nu j}(z,0)\hm=\delta_{\nu,j}$, 
где $\delta_{\nu,j}=1$ при $\nu\hm=j$ и $\delta_{\nu,j}\hm=0$ 
при $\nu\hm\ne j,$  $(p,z)\hm=\sum\limits_{i=1}^rp_i z_i$.

Решение~(\ref{e8-u}) имеет вид:
\begin{multline}
\label{e9-u}
q_{\nu j}(z,t)={}\\
{}=c_ja_{\nu}(p,z)\sum\limits_{k=1}^N\ 
\fr{\prod\limits_{i\ne\nu}(\mu_k(z)+a_i)}{\alpha_k(z)(\mu_k(z)+a_j)}\, e^{\mu_k(z)t}\,,
\end{multline}
где $\alpha_k(z)=\prod\limits_{i\ne k}(\mu_k(z)-\mu_i(z)),$ а $\mu_1(z),\ldots,\mu_N(z)$~--- 
корни многочлена
\begin{equation*}
\prod\limits_{i=1}^N\left(\mu\hm+a_i\right)\hm-(p,z) 
\sum\limits_{j=1}^N c_j a_j\prod\limits_{i\ne j}(\mu+a_i)\,.
\end{equation*}

И, наконец, понадобятся распределения различных промежутков
занятости исследуемой сис\-темы. Пусть $\Pi^{(i)}\left(n_1^{(0)},
\ldots,n_i^{(0)}\right)$~--- длительность\linebreak
 периода занятости
обслуживанием требований приоритетов $1,\ldots,i$, начавшегося с
$n_1^{(0)},\ldots,n_i^{(0)}$ требований этих приоритетов, т.\,е.\
случайного интервала времени, начинающегося с обслуживания одного из
$n_1^{(0)},\ldots,n_i^{(0)}$ требований и кончающегося в момент
первого после этого освобождения системы от требований приоритетов
$1,\ldots,i$. Обозначим
\begin{multline*}
\Pi_{j\nu}^{(i)}\left(n_1^{(0)},\ldots,n_i^{(0)},t\right)\Delta+o(\Delta)={}\\
{}=\mathbf{P}\left(\Pi^{(i)}\left(n_1^{(0)},
\ldots,n_i^{(0)}\right)\in{}\right.\\
\left.{}\in[t,t+\Delta),j(t)=j|j(0)=\nu\right)\,;
\end{multline*}

\vspace*{-12pt}

\noindent
\begin{multline*}
\pi_{j\nu}^{(i)}\left(n_1^{(0)},\ldots,n_i^{(0)},s\right)={}\\
{}=
\int\limits_0^{\infty}e^{-st}\Pi_{j\nu}^{(i)}\left(n_1^{(0)},\ldots,n_i^{(0)},t\right)\,dt\,;
\end{multline*}

\vspace*{-12pt}

\noindent
\begin{align*}
z_{ij}^{(k)}(s)&=\beta_j(\alpha_{ki}(s))\,;\\
\mu_{ki}^{*}(s)&=\mu_k\left(z_{i1}^{(k)}(s),\ldots,z_{ii}^{(k)}(s),1,\ldots,1\right)\,.
\end{align*}

Аналогично~\cite{2-u} показывается, что функции 
$\pi_{j\nu}^{(i)}\left(n_1^{(0)},\ldots,n_i^{(0)},s\right)$ удовлетворяют системе линейных
уравнений:
\begin{multline*}
\sum\limits_{j=1}^N\fr{a_j}{\mu_{ki}^{*}(s)+
a_j}\pi_{j\nu}^{(i)}\left(n_1^{(0)},\ldots,n_i^{(0)},s\right)={}\\
{}=
\fr{a_{\nu}}{\mu_{ki}^{*}(s)+a_{\nu}}\prod\limits_{l=1}^i (z_{il}^{(k)}(s))^{n_l^{(0)}}\,,
\enskip k=1,\ldots,N\,.
\end{multline*}
Отсюда вытекают следующие леммы:

\medskip

\noindent
\textbf{Лемма 2.} \textit{Функции  $\pi_{j\nu}^{(i)}\left(n_1^{(0)},\ldots,n_i^{(0)},s\right)$ определяются по формулам:}
\begin{multline*}
%\label{e10-u}
a_k\pi_{k\nu}^{(i)}\left(n_1^{(0)},\ldots,n_i^{(0)},s\right)={}\\
{}=
\sum\limits_{l=1}^N\fr{a_{\nu}}{\mu_{li}^{*}(s)+a_{\nu}}\,
\prod\limits_{m=1}^i (z_{im}^{(l)}(s))^{n_m^{(0)}}\times{} \\ 
{}\times\fr{\prod\limits_{j=1}^N((a_k+\mu_{ji}^{*}(s))(a_j+\mu_{ki}^{*}(s)))}
{(a_k+\mu_{li}^{*}(s))\prod\limits_{n\ne l}(\mu_{li}^{*}(s)-\mu_{ni}^{*}(s))\prod\limits_{q\ne k}(a_k-a_q)}\,.
\end{multline*}

\noindent
\textbf{Лемма 3.} \textit{Функции  $\pi_{\nu}^{(i)}\left(n_1^{(0)},\ldots,n_i^{(0)},s\right)
\hm=\sum\limits_{j=1}^N \pi_{j\nu}^{(i)}\left(n_1^{(0)},\ldots,n_i^{(0)},s\right)$ определяются по фор\-мулам:}

\noindent
\begin{multline}
\label{e11-u}
\pi_{\nu}^{(i)}\left(n_1^{(0)},\ldots,n_i^{(0)},s\right)=
\sum\limits_{k=1}^N\prod\limits_{m=1}^i (z_{im}^{(k)}(s))^{n_m^{(0)}}\times{}\\
{}\times
\prod_{l\ne \nu}\fr{\mu_{ki}^{*}(s)+a_l}{a_l}\prod_{j\ne k}\fr{\mu_{ji}^{*}(s)}{\mu_{ji}^{*}(s)-\mu_{ki}^{*}(s)}.
\end{multline}


\section{Основные результаты}

Теперь займемся изучением безусловных виртуальных времен ожидания 
при дисциплинах FIFO и LIFO. Во-пер\-вых, заметим, что виртуальное время
ожидания для некоторого приоритетного класса не зависит от дисциплины обслуживания, 
принятой для других классов.

Сначала рассмотрим дисциплину
FIFO. Очевидно, $W_{1j}^{(1)}(s,t)\hm=W_{1j}^{(0)}(s,t)$. При $i\hm\geqslant 1$ имеем
\begin{multline*}
W_{i+1j}^{(1)}(s,t)=
\sum\limits_{\nu=1}^N\sum\limits_{n_1=0}^{\infty}\ldots\\
\ldots\sum\limits_{n_i=0}^{\infty}\int\limits_0^{\infty}
e^{-sy}\mathbf{P}(\mbox{в интервале времени}\ [t,t+y)\\
 \mbox{поступило}\ n_k,\:k=1,\ldots,i,\ \mbox{требований}\\
 \mbox{приоритета}\ k, \ j(t+y)=\nu|j(t)=j)\times{}\\
 {}\times\pi_{\nu}^{(i)}\left(n_1,\ldots,n_i,s\right)d_y
 \mathbf{P}(w_{i+1}^{(0)}(t)<y,j(t)=j)\,.
\end{multline*}
Подставляя $\pi_{\nu}^{(i)}\left(n_1,\ldots,n_i,s\right)$ из (\ref{e11-u}), имеем:

\noindent
\begin{multline*}
W_{i+1j}^{(1)}(s,t)=\sum\limits_{\nu=1}^N\sum\limits_{n_1=0}^{\infty}\ldots{}\\
{}\ldots\sum\limits_{n_i=0}^{\infty}\int\limits_0^{\infty}
e^{-sy}\mathbf{P}(\mbox{в интервале времени}\ [t,t+y)\\
 \mbox{поступило}\ n_k,\:k=1,\ldots,i,\ \mbox{требований}\\
 \mbox{приоритета}\ k, \ j(t+y)=\nu|j(t)=j)\times{}\\
 {}\times\sum\limits_{k=1}^N\prod\limits_{m=1}^i (z_{im}^{(k)}(s))^{n_m}
\prod_{l\ne \nu}\fr{\mu_{ki}^{*}(s)+a_l}{a_l}\times\\
\times\prod_{p\ne k}\fr{\mu_{pi}^{*}(s)}{\mu_{pi}^{*}(s)-\mu_{ki}^{*}(s)}\,d_y
\mathbf{P}(w_{i+1}^{(0)}(t)<y,j(t)=j)\,.
\end{multline*}
Отсюда и из~(\ref{e9-u}) получаем:
\begin{multline*}
W_{i+1j}^{(1)}(s,t)=\sum\limits_{\nu=1}^N\int\limits_0^{\infty}
e^{-sy}\sum\limits_{k=1}^N
\prod_{l\ne \nu}\fr{\mu_{ki}^{*}(s)+a_l}{a_l}\times{}\\
{}\times\prod_{p\ne k}\fr{\mu_{pi}^{*}(s)}
{\mu_{pi}^{*}(s)-\mu_{ki}^{*}(s)}\,c_{\nu}a_j\times{}\\
\hspace*{-7.4pt}{}\times
\sum\limits_{c=1}^N\fr{\prod\limits_{q\ne j}(\mu_{ci}^{(k)}(s)+a_q)}
{(\mu_{ci}^{(k)}(s)+a_{\nu})
\alpha_c(z_{i1}^{(k)}(s),\ldots,z_{ii}^{(k)}(s),1,\ldots,1)}\times{}\\
{}\times \left(\sum\limits_{m=1}^i p_mz_{im}^{(k)}(s)+
\sum\limits_{m=i+1}^r p_m\right)\exp\left(\mu_{ci}^{(k)}(s)y\right)\,
d_y\times{}\\
{}\times \mathbf{P}(w_{i+1}^{(0)}(t)<y,j(t)=j)\,,
\end{multline*}
где $\mu_{ci}^{(k)}(s)=\mu_c(z_{i1}^{(k)}(s),\ldots,z_{ii}^{(k)}(s),1,\ldots,1).$
\pagebreak

Отсюда вытекает справедливость следующей тео\-ремы:

\bigskip

\noindent
\textbf{Теорема 1.} \textit{При дисциплине FIFO $W_{1j}^{(1)}(s,t)\hm=W_{1j}^{(0)}(s,t)$ 
и для $i\geqslant 1$}
\begin{multline*}
%\label{e12-u}
W_{i+1j}^{(1)}(s,t)={}\\[3pt]
{}=\sum\limits_{\nu=1}^N\sum\limits_{k=1}^N
\prod_{l\ne \nu}\fr{\mu_{ki}^{*}(s)+a_l}{a_l}
\prod_{p\ne k}\fr{\mu_{pi}^{*}(s)}{\mu_{pi}^{*}(s)-\mu_{ki}^{*}(s)}\,
c_{\nu}a_j\times{}\\[3pt]
{}\times
\sum\limits_{c=1}^N\fr{\prod\limits_{q\ne j}(\mu_{ci}^{(k)}(s)+a_q)}
{(\mu_{ci}^{(k)}(s)+a_{\nu})
\alpha_c(z_{i1}^{(k)}(s),\ldots,z_{ii}^{(k)}(s),1,\ldots,1)}\times{}\\[3pt]
{}\times \left(\sum\limits_{m=1}^i p_mz_{im}^{(k)}(s)+
\sum\limits_{m=i+1}^r p_m\right)\times{}\\[3pt]
{}\times W_{i+1j}^{(0)}\left(s-\mu_{ci}^{(k)}(s),t\right)\,.
\end{multline*}

При дисциплине LIFO связь между условными и безусловными виртуальными временами 
ожидания несколько другая:
\begin{multline*}
W_{ij}^{(2)}(s,t)=\sum\limits_{\nu=1}^N\sum\limits_{n_1=0}^{\infty}\ldots{}\\[3pt]
{}\ldots\sum\limits_{n_i=0}^{\infty}\int\limits_0^{\infty}
e^{-sy}\mathbf{P}(\mbox{в интервале времени}\ [t,t+y)\\[3pt]
 \mbox{поступило}\ n_k,\:k=1,\ldots,i,\ \mbox{требований}\\[3pt]
 \mbox{приоритета}\ k,  j(t+y)=\nu|j(t)=j)\times{}\\[3pt]
 {}\times\pi_{\nu}^{(i)}\left(n_1,\ldots,n_i,s\right)\,d_y
 \mathbf{P}(w_{i-1}^{(0)}(t)<y,j(t)=j)\,.
\end{multline*}

Отсюда вытекает

\medskip

\noindent
\textbf{Теорема 2.} \textit{При дисциплине LIFO при всех $i=1,\ldots,r$}
\begin{multline*}
%\label{e13-u}
W_{ij}^{(2)}(s,t)=\sum\limits_{\nu=1}^N\sum\limits_{k=1}^N
\prod_{l\ne \nu}\fr{\mu_{ki}^{*}(s)+a_l}{a_l}\times{}\\
{}\times
\prod_{p\ne k}\fr{\mu_{pi}^{*}(s)}{\mu_{pi}^{*}(s)-\mu_{ki}^{*}(s)}\,c_{\nu}a_j\times{}\\
{}\times
\sum\limits_{c=1}^N\fr{\prod\limits_{q\ne j}(\mu_{ci}^{(k)}(s)+a_q)}
{(\mu_{ci}^{(k)}(s)+a_{\nu})\alpha_c(z_{i1}^{(k)}(s),\ldots,z_{ii}^{(k)}(s),1,\ldots,1)}\times{}\\
{}\times 
\left(\sum\limits_{m=1}^i p_mz_{im}^{(k)}(s)+\sum\limits_{m=i+1}^r p_m\right)\times{}\\
{}\times W_{i-1j}^{(0)}
\left(s-\mu_{ci}^{(k)}(s),t\right)\,.
\end{multline*}

{\small\frenchspacing
{%\baselineskip=10.8pt
\addcontentsline{toc}{section}{Литература}
\begin{thebibliography}{9}

\bibitem{1-u}
\Au{Ушаков В.\,Г.} Система обслуживания с эрланговским входящим
потоком и относительным приоритетом~// Теория вероятности и ее примениния,
1977. Т.~22. С.~860--866.

\bibitem{2-u}
\Au{Матвеев В.\,Ф., Ушаков В.\,Г.} Системы массового обслуживания.~--- М.: МГУ, 1984.

\bibitem{3-u}
\Au{Ушаков В.\,Г.} Аналитические методы анализа системы массового
обслуживания $GI|G_r|1|\infty$ с относительным приоритетом~// Вестн.
Моск. ун-та. Сер. 15. Вычисл. матем. и киберн., 1993. №\,4.
С.~57--69.

\label{end\stat}

\bibitem{4-u}
\Au{Ушаков В.\,Г.}  О длине очереди в однолинейной системе
массового обслуживания с чередованием приоритетов~// Вестн. Моск.
ун-та. Сер.~15. Вычисл. матем. и киберн., 1994. №\,2. С.~29--36.
 \end{thebibliography}
}
}


\end{multicols}       