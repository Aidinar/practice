\def\stat{murashov}

\def\tit{КОМБИНИРОВАННЫЙ ПОДХОД К~ЛОКАЛИЗАЦИИ РАЗЛИЧИЙ 
МНОГОМОДАЛЬНЫХ ИЗОБРАЖЕНИЙ$^*$}

\def\titkol{Комбинированный подход к локализации различий 
многомодальных изображений}

\def\autkol{Д.\,М.~Мурашов}
\def\aut{Д.\,М.~Мурашов$^1$}

\titel{\tit}{\aut}{\autkol}{\titkol}

{\renewcommand{\thefootnote}{\fnsymbol{footnote}}\footnotetext[1]
{Работа выполнена при финансовой поддержке РФФИ, проект №\,09-07-00368.}}


\renewcommand{\thefootnote}{\arabic{footnote}}
\footnotetext[1]{Вычислительный центр им. А.\,А.~Дородницына Российской академии наук, d\_murashov@mail.ru}

 

\Abst{Предложен подход к решению задачи локализации различий изображений, 
полученных в разных спектральных диапазонах. Подход основан на применении 
специфических для конкретной прикладной задачи детекторов объектов на исследуемых 
изображениях и теоретико-информационных мер различия изображений. В качестве 
локальной меры различия изображений используется условная энтропия. Рассмотрено 
применение предложенного подхода к решению задачи локализации областей с нарушенным 
авторским лакокрасочным слоем на изображениях произведений живописи в видимом и 
ультрафиолетовом (УФ) диапазоне.}

\KW{многомодальные изображения; мера различия изображений; 
тео\-ре\-ти\-ко-ин\-фор\-ма\-ци\-он\-ная мера; условная энтропия; изображения произведений живописи}

 \vskip 14pt plus 9pt minus 6pt

      \thispagestyle{headings}

      \begin{multicols}{2}
      
            \label{st\stat}

\section{Введение}

  Рассматривается задача анализа изображений, полученных в различных спектральных 
диапазонах. 

В~ряде задач, например поиска дефектов на изображениях негативов, 
зафиксированных в технике тройной цветной фотографии начала XX~в.~[1, 2], или 
выявления нарушений авторской живописи на произведениях изобразительного искусства, 
необходимо найти объекты, видимые только на одном из анализируемых изображений. 

Многомодальные изображения широко применяются в музейной практике для целей 
реставрации и атрибуции. Важным аспектом исследований таких изображений является 
поиск невидимой для человеческого глаза, но важной для специалистов информации с 
использованием комбинирования изображений, зафиксированных в УФ, 
инфракрасном (ИК), рентгеновском и видимом спектральных диапазонах~[3]. Исследования с 
ИК излучением проявляют углеродсодержащие красители (наброски, сделанные 
углем, чернилами). Ультрафиолетовое излучение позволяет увидеть участки, ранее 
подвергавшиеся реставрации (рис.~1), подлаковые загрязнения и ряд других дефектов 
красочного слоя. Тяжелые металлы не пропускают рентгеновские лучи, что позволяет видеть 
красочные слои, выполненные, например, свинцовыми белилами~[4]. 

Для формирования 
задания на реставрационные работы необходимо выделить области с нарушенной авторской 
живописью на изображении в УФ диапазоне и обозначить контуры 
выделенных областей на цифровой фотографии в видимом\linebreak\vspace*{-12pt}
\begin{center} %fig1
\vspace*{2pt}
\mbox{%
  \epsfxsize=78.626mm
 \epsfbox{mur-1.eps}
}
\end{center}
%\begin{center}
\vspace*{2pt}
{{\figurename~1}\ \ \small{Изображения фрагментов портрета в видимом~(\textit{а}) и УФ~(\textit{б}) спектральных 
диапазонах. Картина хранится в Государственном историческом музее, г.~Москва}}
%\end{center}
\vspace*{9pt}

%\smallskip
\addtocounter{figure}{1}


\noindent
 спектральном диапазоне. Такая 
задача относится к классу задач выявления различий на изображениях.
  
  Для сравнения пары изображений необходимо иметь меру, которая позволила бы 
получить количественную характеристику различия сравниваемых изображений. 
Особенности многомодальных изоб\-ра\-же\-ний исключают возможность использования меры 
на основе попиксельной разности значений серого тона.
  
  Для достижения цели работы предлагается использовать следующий подход: 
(а)~вводится мера различия изображений; (б)~с помощью детекторов, 
ориентированных на определенные классы объектов, на одном из изображений, где 
проявляются искомые объекты, выделяются области интереса; (в)~по величине меры 
различия из найденных на этапе~(б) областей выбираются области, соответствующие 
решаемой задаче.
  
  Задача поиска различий на изображениях может быть сформулирована следующим 
образом. Предполагается, что имеются два зафиксированных в разных диапазонах спектра 
изображения сцены $U$ и $V$ размером $m\times n$ с $K$ и~$L$ градациями полутонов 
соответственно. Предполагается, что изображения совмещены. Пусть на~$U$ имеется 
$K$~объектов $O^U_k$,  $k \hm= 1, \ldots , K$, а на изображении~$V$ имеется $L$~объектов 
$O_l^V$, $l\hm = 1, \ldots , L$. Объекты~$O_k^U$ и~$O_l^V$ являются связными 
множествами пикселов, характеризуемых $k$-м и $l$-м уровнем полутонов соответственно. 
При наложении изображений $U$ и~$V$ $J$~объектов совпадают: $O_k^U\cap 
O_l^V\hm=O_k^U\hm =O_l^V$ для $J$~пар $k$ и~$l$, $J\leq K$, $J\leq L$. Требуется 
локализовать объекты $O_i^U$, видимые на изображении~$U$ и от\-сут\-ст\-ву\-ющие на~$V$ и 
удовлетворяющие условию $\mu(x,y)>t$, где $\mu(x,y)$~--- величина, характеризующая 
различия изображений~$U$ и~$V$ в точке $(x,y)\in O_i^U$, $t$~--- константа.

%\vspace*{-6pt}

\section{Современное состояние проблемы}

%\vspace*{-2pt}

  В литературе имеется ряд публикаций по методам поиска различий на сериях 
изображений при решении различных задач. 

В~работе~[5] представлен метод 
автоматизированного поиска скрытой информации по фотографии картины и ее 
рентгенограмме. Выявление невидимых глазу объектов на рентгеновском снимке 
осуществляется сравнением описаний пары изображений, построенных с помощью двух 
типов иерархических моделей: модели изображений и модели соответствия деталей 
изображений. На верхнем уровне изображения сегментируются на области, однородные по 
яркости. На нижнем уровне каждая из выделенных областей раскладывается на текстурную 
компоненту и компоненту среднего значения яркости. Анализируются перепады яр\-кости на 
границах однородных областей, текстурные признаки. Соответствия между деталями 
изображений описываются линейной регрессионной моделью для текстурной составляющей 
и результатами сравнения компонент среднего значения яркости. Метод обладает высокой 
вычислительной сложностью. Выполняется сегментация и анализ выделенных объектов и 
краев каждого из двух сравниваемых изображений. 

В~работе~\cite{2mur} при локализации 
дефектов на изоб\-ра\-жениях негативов, полученных в технике тройной цветной фотографии, 
применяется процедура на основе логических операций над бинарными изоб\-ра\-же\-ни\-я\-ми. При 
формировании масок дефектов отбрасываются те найденные детекторами об\-ласти, которые 
при наложении масок трех компонент негатива дают непустое пересечение. Такой метод 
требует значительных вычислительных затрат на больших изображениях (поскольку 
приходится обнаруживать объекты на трех компонентах негатива) и не обеспечивает 
надежного обнаружения искомых объектов. 

В~работе~\cite{6mur} представлена 
программная система для сравнения изображений видимых и скрытых слоев живописи. 
Система предназначена для визуального анализа комбинированных ИК-реф\-лек\-то\-грамм 
и цветных цифровых фотографий только в интерактивном режиме.



  В~работе~\cite{7mur} предложен метод оценивания визуальных различий изображений на 
основе модели зрительной системы человека. Мерой различия в точке с заданными 
координатами является вероятность обнаружения несовпадений двух изоб\-ра\-же\-ний в этой 
точке. 

На базе меры~\cite{7mur} в~\cite{8mur} вводятся модификации меры визуальных 
различий для последо\-ва\-тель\-ности многоспектральных изоб\-ра\-же\-ний. Постановка задачи 
оценивания визуальных различий изображений не в полной мере соответствует решаемой в 
данной работе задаче, где требуется выделить объекты, видимые только на одном из 
предъявляемых изображений.
  
  Ряд работ посвящен задаче сравнения изображений для оценивания визуальной 
различительной способности целей относительно фона~\cite{9mur}. Один из подходов 
основан на тео\-ре\-ти\-ко-ин\-фор\-ма\-ци\-он\-ных мерах, которые будут рассмотрены в 
следующем разделе.

\section{Теоретико-информационные меры различия изображений}
  
  В ряде работ для оценивания сходства и различия изображений применяются 
  тео\-ре\-ти\-ко-ин\-фор\-ма\-ци\-он\-ные подходы и методы. Особенностью 
  тео\-ре\-ти\-ко-информационных методов является то, что они не требуют 
  предобработки, сегментации 
изоб\-ра\-же\-ний и анализа выделенных компонент. Используются непосредственно значения 
уровней яркости в пикселах изображения.
  
  В работах~[10, 11] и~других для решения задачи совмещения изображений предложена 
  тео\-ре\-ти\-ко-ин\-фор\-ма\-ци\-он\-ная мера сходства модельного и преобразованного 
входного изображения в виде значения взаимной информации, вычисленного на этих 
изображениях. В~работе~[12] введена мера различия изображений в виде суммы их 
условных энтропий $H(X\vert Y)\hm+H(Y\vert X)$, где $X$ и~$Y$~--- случайные 
переменные, характеризующие значения яркости в пикселах изображений. Условная 
энтропия $H(X\vert Y)$ интерпретируется как средняя информация, которая требуется для 
того, чтобы определить~$X$, если известна~$Y$. В~работе~[13] для оценивания 
качественных показателей результата комбинирования последовательностей 
многоспектральных изображений использовалась мера стабильности комбинирования на 
основе взаимной информации.
  
  В работе~[9] для решения задачи измерения визуальной различительной способности 
целей относительно фона разработан метод оценивания информации о цели по двум 
изображениям: фона и цели на том же самом фоне. Показано, что мерой различия 
изображений является дивергенция Куль\-ба\-ка--Лейб\-ле\-ра, однако использование этой 
меры затруднено применительно к многомодальным изображениям, так как требуются 
сильные ограничения на распределения значений полутонов сравниваемых изображений. 
  
  Для использования теоретико-ин\-фор\-ма\-ци\-он\-но\-го подхода необходима 
вероятностная модель свя\-зи между изображениями. Пусть значения яр\-кости на 
сравниваемых изображениях в точке с координатами $(x,\,y)$ описываются дискретными 
случай\-ны\-ми переменными $U(x,\,y)$ и $V(x,\,y)$ со значениями~$u$ и~$v$, квантованными 
на конечное число уровней~$K$ и~$L$ соответственно. Поскольку изоб\-ра\-же\-ния~$U$ и~$V$ 
отображают одну и ту же сцену, то существует связь между переменными $U(x,\,y)$ и 
$V(x,\,y)$. Будет использоваться модель, аналогичная предложенной в~\cite{11mur}:
  \begin{equation}
  U\left(\mathrm{Tr}\left(x,\,y\right)\right) =F(V(x,\,y))+\eta(x,\,y)\,,
  \label{e1mur}
  \end{equation}
где Tr~--- преобразование координат (для совмещенных изображений 
$U\left(\mathrm{Tr}\left(x,\,y\right)\right)\hm=U(x,\,y))$; $F$~--- функция преобразования яркости, моделирующая связь 
между двумя изображениями объекта в двух спектральных диапазонах; $\eta(x,\,y)$~--- 
случайная переменная, моделирующая артефакты (например, нарушения лакокрасочного 
слоя, видимые при УФ освещении). Модель~(\ref{e1mur}) можно 
рассматривать как модель дискретной стохастической информационной системы с 
входом~$V$ и выходом~$U$. 
  
  В отличие от задачи совмещения изображений в решаемой задаче требуется мера, 
позволяющая выделить объекты на изображении~$U$, от\-сут\-ст\-ву\-ющие на~$V$. Это 
означает, что мера должна включать составляющие, обусловленные только объектами 
изображения~$U$, которых нет на~$V$, и не учитывать составляющие тех объектов на~$V$, 
которых нет на~$U$. Желательно, чтобы мера вычислялась на основе двумерных 
распределений, что позволит эффективнее использовать взаимосвязь анализируемой пары 
изображений.
  
  Предлагается характеризовать отличия изображения~$U$ от~$V$ значениями условных 
энтропий $H(U\vert V)$ и $H(V\vert U)$. В работе~\cite{14mur} условная энтропия 
для дискретной системы определяется следующим образом:
  \begin{align}
  H(U\vert V) &= -\sum\limits_{k=1}^K \sum\limits_{l=1}^L p(u_k,v_l)\log \left[ p(u_k\vert 
v_l)\right] ={}\notag\\
&{}=-\sum\limits_{k=1}^K \sum\limits_{l=1}^L p(u_k,v_l)\log 
\left[\fr{p(u_k,v_l)}{p(v_l)}\right]\,;\label{e2mur}\\
  H(V\vert U) &= -\sum\limits_{k=1}^K \sum\limits_{l=1}^L p(u_k,v_l)\log \left[ p(v_l\vert 
u_k)\right] ={}\notag\\
&{}=-\sum\limits_{k=1}^K \sum\limits_{l=1}^L p(u_k,v_l)\log 
\left[\fr{p(u_k,v_l)}{p(u_k)}\right]\,,\label{e3mur}
  \end{align}
где $p(u_k)$, $p(v_l)$ и $p(u_k,v_l)$~--- вероятности появления уровней $v_l$ и $u_k$ на 
входе и выходе системы и их совместная вероятность; $p(u_k\vert v_l)$ и 
$p(v_l\vert u_k)$~--- соответствующие 
условные вероятности.
  
  Условная энтропия неотрицательна и аддитивна. Дополнительные условия, при которых 
$H(U\vert V)$ и $H(V\vert U)$ выполняют функции меры различия, дают сформулированные 
далее утверждения. 

\medskip

\noindent
\textbf{Утверждение 1.} Условная энтропия $H(U\vert V)$ является мерой отличия 
изображения~$U$ от изображения~$V$ в тех случаях, когда
\begin{equation}
p(v_l)=p(u_k,v_l)\,,\enskip k=1, \ldots , K\,;\ l=1, \ldots ,L\,,
\label{e4mur}
\end{equation}
или
\begin{multline}
p(v_l)>p(u_k,v_l)\,;\quad p(u_k)=p(u_k,v_1)\,,\\
k=1, \ldots , K\,;\enskip l=1, \ldots ,L\,,
\label{e5mur}
\end{multline}
где $p(u_k)$, $p(v_l)$ и $p(u_k,v_l)$~--- вероятности появления уровней полутонов~$u_k$ 
и~$v_l$ на изображениях~$U$ и~$V$.

      \begin{figure*} %fig2
       \vspace*{1pt}
       \begin{minipage}[t]{80mm}
 \begin{center}
 \mbox{%
 \epsfxsize=78.099mm
 \epsfbox{mur-2.eps}
 }
 \end{center}
 \vspace*{-9pt}
      \Caption{Изображения, для которых  $H(U\vert V)\hm=0$: (\textit{а})~изоб\-ра\-же\-ние~$U$; 
(\textit{б})~изображение~$V$; (\textit{в})~совместная гистограмма изображений~$U$ и~$V$}
%      \end{figure*}
\end{minipage}
\hfill
%\begin{figure*} %fig3
       \vspace*{1pt}
       \begin{minipage}[t]{80mm}
 \begin{center}
 \mbox{%
 \epsfxsize=78.455mm
 \epsfbox{mur-3.eps}
 }
 \end{center}
 \vspace*{-9pt}
\Caption{Изображения $U$~(\textit{а}) и $V$~(\textit{б}), для которых $H(U\vert V)\hm>0$; 
(\textit{в})~совместная гистограмма изображений~$U$ и~$V$}
\end{minipage}
\vspace*{6pt}
\end{figure*}

\begin{figure*}[b] %fig4
       \vspace*{6pt}
 \begin{center}
 \mbox{%
 \epsfxsize=150.06mm
 \epsfbox{mur-5.eps}
 }
 \end{center}
 \vspace*{-9pt}
\Caption{Пример применения условной энтропии для выявления различий изображений: (\textit{а}) 
и~(\textit{б})~соответственно изоб\-ра\-же\-ния $U$ и~$V$; (\textit{в})~совместная гистограмма значений 
полутонов; (\textit{г}) и (\textit{д})~визуализированные локальные значения условных энтропий 
$H(U\vert V)$ и $H(V\vert U)$ соответственно}
\end{figure*}

  
  \smallskip
  
  Доказательство утверждения следует из выражения~(\ref{e2mur}). Если имеет место 
условие~(\ref{e4mur}), то 
  \begin{equation*}
  H(U\vert V) = -\sum\limits_{k=1}^K \sum\limits_{l=1}^L p(u_k,v_l)\log \left[ 
\fr{p(u_k,v_l)}{p(v_l)}\right] =0\,.
%  \label{e6mur}
  \end{equation*}
    Это означает, что на $U$ имеется $M$ объектов $O_k^U$, каждый из которых 
характеризуется $k$-м уровнем серого, $k\hm = 1, \ldots , M$, а на $V$ имеется $L$ объектов 
$O_l^V$ $l$-го уровня серого, $M\hm\leq K\hm\leq L$. Границы $M$ объектов на~$U$ при 
наложении совпадают с границами $M$~объектов на~$V$, т.\,е.\ $O_k^U\hm=O_l^V$, $k\hm 
=l\hm= 1, \ldots , M$. Имеются также объекты $O_l^V$, $O_k^U\cap 
O_l^V\hm=O_l^V\hm<O_k^U$ для некоторых $k\hm>M$, $l\hm>M$. Пример таких 
изображений показан на рис.~2,\,\textit{а} и~\textit{б}, совместная гистограмма 
изображений~$U$ и~$V$ показана на рис.~2,\,\textit{в}.
  

      
  Если выполняется~(\ref{e5mur}), то $H(U\vert V)\hm>0$, а величина $H(U\vert V)$ зависит 
от соотношения величин $p(u_k,v_l)$ и $p(v_l)$. Это означает, что на $U$ имеется 
$M$~объектов $O_k^U$, каждый из которых характеризуется $k$-м уровнем серого, $k\hm = 
1, \ldots , M$, а на $V$ имеется $L$ объектов $O_l^V$ $l$-го уровня серого, $M\hm\leq 
L\hm\leq K$. Границы $M$ объектов на~$U$ при наложении совпадают с границами $M$ 
объектов на~$V$, т.\,е.\ $O_k^U\hm=O_l^V$, $k\hm = l \hm= 1, \ldots , M$. Кроме того, 
имеются объекты $O_k^U$ такие, что $O_k^U\cap O_l^V\hm= O_k^U\hm<O_l^V$ для 
некоторых $k\hm>M$, $l\hm>M$. Этот случай проиллюстрирован на рис.~3.


  
  Условие $p(u_k) =p(u_k,v_l)$ в~(\ref{e5mur}) исключает со\-став\-ля\-ющие объектов $O_l^V$ 
на~$V$ в $H(U\vert V)$. Случай, когда условие $p(u_k)\hm=p(u_k,v_l)$ в~(\ref{e5mur}) не 
выполняется при $k\hm = 0$, показан на рис.~4.
  
  При решении практических задач в процессе анализа пар изображений возникает 
необходимость визуализации различий. В~качестве индикатора\linebreak
 различий целесообразно 
использовать условные энтропии $H(U\vert V)$ и $H(V\vert U)$, вычисляемые в некоторой 
окрестности каждого пиксела ана\-ли\-зи\-ру\-емых изображений с необходимым числом уровней 
квантования полутонов~\cite{15mur}. Размеры окрестности и число уровней квантования 
выбираются таким образом, чтобы: (а)~получить корректные оценки вероятностей; 
(б)~обеспечить выполнение условий~(\ref{e4mur}) и~(\ref{e5mur}); (в)~обеспечить 
приемлемую точность локализации различий.



  
  Если на изображениях имеются текстурные об\-ласти с большой дисперсией уровня серого 
тона, то не всегда удается выбором размеров окна обеспе-\linebreak\vspace*{-12pt}
\pagebreak

\end{multicols}

\begin{figure} %fig5
       \vspace*{1pt}
 \begin{center}
 \mbox{%
 \epsfxsize=161.362mm
 \epsfbox{mur-4.eps}
 }
 \end{center}
 \vspace*{-9pt}
\Caption{Локализация различий в синем и красном каналах цветного изображения: 
(\textit{а})~цветное изображение с внесенными в канал $B$ объектами в виде темных дисков; 
(\textit{б})~канал~$R$; (\textit{в})~канал~$B$; (\textit{г})~визуализированные локальные значения 
$H(U\vert V)$}
\vspace*{9pt}
%\end{figure*}
%\begin{table*}
\renewcommand{\figurename}{\protect\bf Таблица}
\setcounter{figure}{0}
{\small
\begin{center}
\Caption{Характеристики областей изображения на рис.~5}
\vspace*{2ex}

\begin{tabular}{|l|c|c|c|c|c|c|}
\hline
\multicolumn{1}{|c|}{\tabcolsep=0pt\begin{tabular}{c}\  \\ Фрагмент\end{tabular}}& 
\multicolumn{2}{c|}{Среднее значение серого} &
\multicolumn{2}{c|}{\tabcolsep=0pt\begin{tabular}{c}Среднеквадратическое\\ отклонение\end{tabular}}
&\multicolumn{2}{c|}{Энтропия}\\
\cline{2-7}
&Канал $B$&Канал $R$&Канал $B$&Канал $R$&Канал $B$&Канал $R$\\
\hline
Небо&198\hphantom{9}& 161 &14,70&11,08 &5,63&5,35\\
Лес&29&\hphantom{9}69&18,71&29,26&5,81&6.79\\
Поле&28&108 &28,98&39,54 &5,83&7,31\\
\hline
\end{tabular}
\end{center}}
\vspace*{9pt}
\end{figure}

\renewcommand{\figurename}{\protect\bf Рис.}
\renewcommand{\tablename}{\protect\bf Таблица}

\setcounter{figure}{5}
\setcounter{table}{1}


\begin{multicols}{2}

\noindent
чить выполнение 
условий~(\ref{e4mur}) и~(\ref{e5mur}). Так, на рис.~5 на фрагменте <<поле>> с помощью 
индикатора $H(U\vert V)$, где $U$ и $V$~--- синий и красный каналы цветного 
изображения, не удается локализовать объекты, внесенные в синий канал. Статистические 
характеристики фрагментов изображения, представленного на рис.~5, приведены в табл.~1. 
В~этом случае в качестве меры различия предлагается использовать условную энтропию 
$H(V\vert U)$. Условия, при которых $H(V\vert U)$ можно использовать в качестве меры 
различия, задаются утверждением~2. 




\medskip

\noindent
\textbf{Утверждение 2.} Условная энтропия $H(V\vert U)$ является мерой отличия 
изображения~$U$ от изображения~$V$ в тех случаях, когда выполняются условия:
\begin{equation}
p(u_k) =p(u_k,v_l)\,,\enskip k=1, \ldots , K\,;\ l=1, \ldots ,L\,,
\label{e6mur}
\end{equation}
или~(5) и
\begin{multline}
\exists u_0:\ p(u_0)>p(u_0,v_l)\,,\ p(u_0)={}\\
{}=\sum\limits_{l=1}^J p(u_0,v_l)\,,\ l=1, \ldots , 
J\,;\ J\leq L\,.
\label{e8mur}
\end{multline}


  
  Доказательство следует из выражения~(\ref{e3mur}). Если имеет место 
условие~(\ref{e6mur}), то $H(V\vert U)\hm=0$, что соответствует случаю, когда на~$U$ 
имеются только объекты, которые есть и на~$V$, границы объектов при наложении 
изображений совпадают ($O_k^U\hm=O_l^V$, $k\hm=l\hm=1, \ldots , K$).
  
  Если выполняются условия~(\ref{e5mur}) и~(\ref{e8mur}), то $H(V\vert U)\hm >0$. Это 
соответствует случаю, когда на~$U$ имеется объект~$O_0^U$ такой, что $O_0^U\cap 
O_l^V\hm\not= \varnothing\ \forall\,l$, $l\hm=1, \ldots ,K$. Условие $p(u_k)\hm=p(u_k,v_l)$ 
в~(\ref{e5mur}) исключает появление в $H(V\vert U)$ составляющей, вносимой объектами 
$O_k^U$ на~$U$.
  
  Пример пары изображений, для которых не применима мера $H(U\vert V)$, но 
выполняются условия~(\ref{e6mur}), (5) и~(\ref{e8mur}) и применима мера $H(V\vert U)$, показан 
на рис.~4,\,\textit{а} и~\textit{б}. Совместная гистограмма изображений показана на 
рис.~4,\,\textit{в}. Визуализация локальных значений $H(U\vert V)$ и $H(V\vert U)$ 
приведена на рис.~4,\,\textit{г} и~\textit{д}. Здесь оценки вероятностей получены по всем 
точкам изоб\-ра\-же\-ний, а энтропии вычислялись в каждой точке отдельно.




\pagebreak

\end{multicols}

\begin{figure} %fig6
       \vspace*{1pt}
 \begin{center}
 \mbox{%
 \epsfxsize=160.688mm
 \epsfbox{mur-6.eps}
 }
 \end{center}
 \vspace*{-11pt}
\Caption{Пример выявления артефактов в синем канале фрагмента цветной фотографии, показанной 
на рис.~5: (\textit{а})~фрагмент цветного изображения с артефактами в синем канале; 
(\textit{б})~синий канал фрагмента; (\textit{в}) и~(\textit{г})~визуализированные локальные значения 
условных энтропий $H(U\vert V)$ и $H(V\vert U)$ соответственно}
%\end{figure*}
%\begin{figure*} %fig7
       \vspace*{4pt}
 \begin{center}
 \mbox{%
 \epsfxsize=125.952mm
 \epsfbox{mur-7.eps}
 }
 \end{center}
 \vspace*{-11pt}
\Caption{Увеличенная область с артефактом изображения, показанного на рис.~6: (\textit{а})~цветное 
изображение; (\textit{б}) и~(\textit{в})~соответственно синий и красный канал фрагмента; 
(\textit{г})~совместная гистограмма синей и красной компоненты для выделенного фрагмента при 
256~уровнях квантования}
\end{figure}


\begin{multicols}{2}

  Пример выявления артефактов в виде дисков диаметром 8~пикселов в синем канале 
фрагмента цветной фотографии, показанной на рис.~5,\,\textit{а}, с помощью меры различия 
в виде условной энтропии
 $H(V\vert U)$ показан на рис.~6. 
 
 Вероятности оценивались в окне 
размером $11\times 11$~пикселов при 256~уровнях полутонов, значения условных энтропий 
вы\-чис\-ля\-лись в окне размером $5\times 5$. Увеличенная область, содержащая артефакт, ее 
цветовые компоненты и совместная гистограмма для 256~уровней полутонов показаны на 
рис.~7.


\section{Тестирование}

\vspace*{-15pt}

  Для проверки эффективности применения используемых теоретико-информационных мер 
различия для обнаружения артефактов на реальных изображениях проведен вычислительный 
эксперимент. 

Использовались цветные фотографии размером $988\times 1631$~пиксел (см.\ 
рис.~5) с однородными и текстурными областями (небо, лес, поле). Характеристики областей 
используемого изображения приведены в табл.~1. В~один из цветовых каналов
изоб-\linebreak\vspace*{-12pt}
\pagebreak

\noindent
ражения 
внедрялось 140~объектов в виде размытых дисков диаметром от~4 до 8~пикселов или 
сегментов кривых толщиной от~1 до 7~пикселов. 
  
  Для поиска внедренных объектов использовались меры $H(U\vert V)$ и $H(V\vert U)$. 
Оценивание вероятностей производилось в окнах размером от $7\times 7$ до $11\times 11$, а 
вычисление локальных значений условных энтропий~--- в окнах от $3\times 3$ до $5\times 5$ 
пикселов. Использовалось 8~уровней квантования для фрагмента <<небо>> и 256 для 
фрагментов <<лес>> и <<поле>>. Найдено 100\% внедренных объектов в виде дисков 
диаметром $d \hm= 8$~пикселов. Результаты обнаружения дисков диаметром $d\hm = 
4$~пиксела, внедренных в синий канал, представлены в табл.~2. Найдено 100\% внедренных 
линейных объектов на однородных фрагментах и 82\%--93\%~--- на фрагментах с текстурой. 
Наибольшую трудность для обнаружения представляют объекты малого размера на 
текстурных областях с сильными перепадами яркости (<<лес>>, <<поле>>).

\begin{center} %fig1
{{\tablename~2}\ \ \small{Результаты эксперимента при $d = 4$}}
\vspace*{9pt}

{\small 
\begin{tabular}{|l|c|c|c|}
\hline
\multicolumn{1}{|c|}{Фрагмент}&Число объектов&\multicolumn{2}{c|}{Найдено}\\
\hline
Небо&68&66&97,1\%\\
Лес&28&27&96,4\%\\
Поле&44&40&91,0\%\\
\hline
Всего&140\hphantom{9}&133\hphantom{9}&95,0\%\\
\hline
\end{tabular}

}

\end{center}
%\vspace*{11pt}

%\smallskip
\addtocounter{table}{1}


\noindent

      
\section{Применение теоретико-информационной меры различия в~задаче 
поиска записей на~изображениях произведений живописи в~различных 
спектральных диапазонах}
  
  Рассматривается задача выявления нарушений авторской живописи (записей) на 
произведениях изобразительного искусства (см.\ рис.~1): необходимо найти объекты, 
видимые только на одном из анализируемых изображений.
  
  Рассматриваются изображения в формате JPEG размером $1640\times 1950$~пикселов 
глубиной 8 или 24~бита, полученные с помощью фотокамеры с CCD-матрицей. 
Исследуемые изображения обладают рядом особенностей, оказывающих влияние на 
решение задачи. Во-пер\-вых, неравномерное освещение при съемке. Во-вто\-рых, объекты 
интереса~--- области реставрации и записи авторской живописи, области подлакового 
загрязнения~--- имеют различные яркостные профили и контрастность. В-третьих, размеры 
объектов могут составлять от нескольких десятков до нескольких сотен и тысяч пикселов. 
Форма объектов может быть самой разнообразной. В-чет\-вер\-тых, искомые области 
существенно неоднородны по яркости. Таким образом, области повреждений и 
вмешательства в авторский красочный слой могут быть разделены на классы по характеру 
проявления на изображениях и могут потребоваться разные методы для их локализации.
  
  В соответствии с предложенным во введении подходом далее будут представлены 
алгоритмы детекторов для нахождения областей, со\-от\-вет\-ст\-ву\-ющих по характеристикам 
решаемой задаче, и с помощью описанной выше меры различия отобраны только те объекты, 
которые видны на УФ-изоб\-ра\-же\-нии, но отсутствуют на фотографии в видимой части 
спектра.

\subsection{Детекторы областей интереса} %5.1
  
  При решении рассматриваемой задачи будут использованы модели изображений и 
детекторов, соответствующих проявлениям записей авторской живописи. 
  
  Пусть функции $U^k\hm=u^k(x,y)$, $(x,y)\in X$, $U^k:\ X\rightarrow Z^+$, $k\hm=1, \ldots 
,K$, определены в некоторой области $X\subset Z^2$ и описывают полутоновый рельеф на 
$K$ изображениях сцены, зафиксированных в разных диапазонах спектра при отсутствии 
дефектов или результатов вмешательства в авторский красочный слой. Все анализируемые 
изображения предварительно совмещены, и скомпенсирована неравномерность 
освещенности при съемке. Пусть функции  $\xi^k_{ij}\hm=v_{ij}^k(x,y)$, $\xi^k_{ij}:\ 
X\rightarrow Z$, $i\hm= 1, \ldots ,N_j$, $j\hm=1, \ldots ,J$, описывают рельеф дефектной 
области $D_{ij}^k\subset X$. Здесь $i$~--- номер дефекта, $j$~--- номер класса дефектов. 
Пусть отображение  $\varphi_j:\ X\times Z \rightarrow [0,\,1]$ описывает детектор дефектов 
класса~$j$:
  \begin{gather*}
  \varphi_j(u^k(x,y))=0\ \forall\ (x,y)\in X\,,\enskip (x,y)\not\in D_{ij}^k\,;\\
  \varphi_j\left(u^k(x,y)+\sum\limits_i \sum\limits_j \xi_{ij}^k\right) =1\ \forall\ (x,y)\in 
D_{ij}^k\,.
  \end{gather*}

  Для поиска областей интереса применяются два детектора. Первый предназначен для 
выделения крупных объектов и основан на операциях геодезической реконструкции 
полутоновых изображений~\cite{16mur} и понятии <<бассейна>> яркостного рельефа 
глубиной~$h$. Детектор включает операции выделения областей, в которых яркость 
пикселов меньше яркости внешних относительно бассейна пикселов на величину не более 
чем~$h$, и получения бинарной маски выделенных областей.

\begin{figure*}[b] %fig8
       \vspace*{1pt}
 \begin{center}
 \mbox{%
 \epsfxsize=142.786mm
 \epsfbox{mur-8.eps}
 }
 \end{center}
 \vspace*{-9pt}
\Caption{Изображения бинарных масок $M_1^U$~(\textit{а}) и $M_2^U$~(\textit{б}) областей 
интереса на УФ-изображении}
\end{figure*}

\smallskip

\noindent
\textbf{Детектор 1.} Бинарная маска формируется сле\-ду\-ющим образом:
\begin{equation}
M_1^U = T\left( U_{\mathrm{bas}}-U_{\mathrm{dom}}\right)\,,
\label{e11mur}
\end{equation}
где $T(\cdot)$~--- операция пороговой бинаризации, $U_{\mathrm{dom}}$~--- изображение найденных 
ярких областей (<<куполов>> яркостного рельефа) на~$U$:
\begin{equation}
U_{\mathrm{dom}} =U-R_U^\delta \left( U-g\right)\,.
\label{e12mur}
\end{equation}
Здесь $R_U^\delta\left(U-g\right)$~--- результат операции реконструкции геодезической 
дилатацией маски~$U$ из маркера $U-g$, где $g$~--- наибольшая относительная высота 
выделяемых <<куполов>>.
  
  <<Бассейны>> с относительной глубиной~$h$ на изоб\-ра\-же\-нии~$U$ в ультрафиолетовом 
спектральном диапазоне находятся как
  \begin{equation}
  U_{\mathrm{bas}} =R_U^\delta \left(U+h\right) -U\,,
  \label{e13mur}
  \end{equation}
где $ R_U^\delta \left(U+h\right)$~--- операция реконструкции геодезической эрозией маски U 
из изображения-маркера $U+h$, которое получено из изображения~$U$ увеличением 
яркости на~$h$. Операция поэлементного вычитания $U_{\mathrm{bas}}\hm-U_{\mathrm{dom}}$ 
в~(\ref{e11mur}) выполняется для повышения контрастности изображения~$U_{\mathrm{bas}}$ и 
повышения точности пороговой бинаризации. Бинарные маски областей интереса на 
УФ-изоб\-ра\-же\-нии, полученные с помощью детектора~1~(\ref{e11mur})--(\ref{e13mur}), 
показаны на рис.~8,\,\textit{а}.

  Второй детектор предназначен для выделения небольших фрагментов, отличающихся по 
уровню серого тона от окружающих областей, и основан на алгоритме локальной адаптивной 
пороговой бинаризации~\cite{17mur}.

\smallskip

\noindent
\textbf{Детектор 2.} Функция детектора строится сле\-ду\-ющим образом. Пусть 
$\overline{u}^k$~--- среднее значение функции $u^k(x,y)$ в некоторой области $W\subset X$, 
$u_m\hm< \overline{u}^k(x,y)\hm < u_M$ для $(x,y)\hm\in W$. Тогда функция детектора 
имеет вид:
\begin{equation}
\varphi(x,y) =\begin{cases}
1\,, & \ u(x,y)\geq u_M\,;\\
0\,, & \ u(x,y)<u_M\,.
\end{cases}
\label{e14mur}
\end{equation}
Здесь $u_M$ задается в виде $u_M\hm=\overline{u}^k(x,y)+q\sigma$, где $\sigma$~--- 
среднеквадратическое отклонение яркости, вы\-чис\-лен\-ное в скользящем окне~$W$; $q$~--- 
коэффициент. Изображение маски дефектов формируется в виде 
\begin{equation}
M_2^U(x,y)=\varphi(x,y)\,.
\label{e15mur}
\end{equation}
  
  Полученные с помощью детектора~2, заданного 
  выражениями~(\ref{e14mur}) и~(\ref{e15mur}), бинарные маски областей интереса на 
  УФ-изоб\-ра\-же\-нии показаны на рис.~8,\,\textit{б}.


  
  Однако не все выделенные объекты на бинарной маске соответствуют искомым объектам. 
Необходимо отобрать только те объекты, которые видны на УФ-изоб\-ра\-же\-нии и не 
видны на фотографии в видимом диапазоне. Для выявления указанных выше различий будут 
использоваться меры различия, описанные в предыдущих разделах. 
%\pagebreak

\end{multicols}

\begin{figure}[b] %fig9
       \vspace*{1pt}
 \begin{center}
 \mbox{%
 \epsfxsize=138mm %142.705mm
 \epsfbox{mur-9.eps}
 }
 \end{center}
 \vspace*{-11pt}
\Caption{Изображения, построенные по локальным значениям условных энтропий $H(U\vert 
V)$~(\textit{а}) и  $H(V\vert U)$~(\textit{б}), вычисленных для полутоновых версий изображений с 
рис.~1}
%\end{figure}
%\begin{figure} %fig10
       \vspace*{4pt}
 \begin{center}
 \mbox{%
 \epsfxsize=138mm %142.705mm
 \epsfbox{mur-10.eps}
 }
 \end{center}
 \vspace*{-11pt}
\Caption{Результирующая маска дефектов, видимых на УФ-изоб\-ра\-же\-нии,~(\textit{а}) и 
комбинация маски и цифровой фотографии картины~(\textit{б})}
\end{figure}


\begin{multicols}{2}

\subsection{Отбор объектов} %5.2
  
  В данном подразделе по значениям исполь\-зу\-емой меры различия с изображений, показанных 
на рис.~8, будут отобраны объекты, соответст\-ву\-ющие решаемой задаче.
  
  Изображения величин условных энтропий~(2) и~(3) для полутоновых изображений, 
полученных из изображений рис.~1, показаны на рис.~9. Вероятности оценивались в окне 
размером $11\times 11$ при 32~уровнях квантования, а значения $H(U\vert V)$ и $H(V\vert 
U)$ вычислялись в окрестности $3\times 3$. На изображении $H(U\vert V)$ видны контуры 
объектов, отсутствующих на изображении~$V$.

Для получения маркеров искомых объектов выполняется сле\-ду\-ющая операция:

  \pagebreak
  
  
\noindent
  $$
  M^{U\vert V}(x,y) =T\left( H\left( U\vert V\right)\bullet H(U)-H(V\vert U)\right)\,,
  $$
где $T(\cdot)$~--- операция пороговой бинаризации; $H(U)$~--- изображение энтропии 
выхода системы~(\ref{e1mur}); <<$\bullet$>>~--- операция поэлементного умножения 
изображений. Операции поэлементного умножения и вычитания применяются для 
повышения контраста и улучшения качества пороговой бинаризации.
  
  Тогда искомое изображение дефектов, проявляемых в УФ диапазоне, будет 
найдено с помощью морфологической реконструкции комбинации бинарных 
масок~(\ref{e11mur}) и~(\ref{e15mur}) (см.\ рис.~8):
  \begin{equation}
  M(U,V) =R^\delta_{M^U}\left(M^{U\vert V}\right)\,,
  \label{e16mur}
  \end{equation}
где $M(U)=M_1^U\vee M_2^U$, <<$\vee$>>~--- операция поэлементного логического 
<<ИЛИ>>. Изображения результирующей бинарной маски~(\ref{e16mur}) искомых объектов 
и маски, наложенной на изображение в видимом диапазоне, показаны на рис.~10.


\section{Выводы}

  Предложен комбинированный подход к локализации различий многомодальных 
изображений. Подход заключается в использовании детекторов, соответствующих 
специфике решаемой задачи, и выборе из множества найденных объектов подмножества, 
удовлетворяющего заданным значениям меры различия. Предложено использовать в 
качестве меры различий условную энтропию, вычисляемую по значениям уровней 
полутонов срав\-ни\-ва\-емых изображений. Сформулированы условия, при которых значения 
условной энтропии характеризуют локальные отличия изображений. Проведенный 
вычислительный эксперимент показал эффективность используемой теоретико-информационной меры. 

Разработанный подход использован для локализации записей 
авторского лакокрасочного слоя на изображениях произведений живописи. Объекты, 
соответствующие по величине яркости участкам вмешательства в авторскую живопись, 
находятся детекторами на УФ-изоб\-ра\-же\-нии на основе операций полутоновой 
морфологической реконструкции и пороговой бинаризации с определением порога по 
локальным характеристикам. На основе предложенной меры производится отбор найденных 
бинарных объектов. 

Предложенный подход позволил выполнять детектирование объектов 
только на одном из анализируемых изображений, в отличие от подходов, представленных 
в~[2, 5].

{\small\frenchspacing
{%\baselineskip=10.8pt
\addcontentsline{toc}{section}{Литература}
\begin{thebibliography}{99}

\bibitem{1mur}
\Au{Wagner J.}
Die additive Dreifarbenfotografie nach A.~Miethe~--- Untersuchung des Verfahrens und Wege zur 
Wiedergabe von Dreifarbendiapositiven, Diplomarbeit.~--- TU M$\ddot{\mbox{u}}$nchen, 2006.

\bibitem{2mur}
\Au{Minakhin V., Murashov D., Davidov~Yu., Dimentman~D.}
Compensation for local defects in an image created using a triple-color photo technique~// Pattern 
Recognition Image Analysis: Advances Math. Theory Applications, 2009. 
Vol.~19. No.\,1. P.~137--158.

\bibitem{3mur}
\Au{Kirsh A., Levenson R.\,S.}
Seeing through paintings: Physical examination in art historical studies.~--- Yale: Yale U. Press, 
2000.

\bibitem{4mur}
\Au{Иванова Е.\,Ю., Постернак О.\,П.}
Техника реставрации станковой масляной живописи.~--- М.: ИНДРИК, 2005.

\bibitem{5mur}
\Au{Heitz F., Maitre H., de~Couessin~C.} 
Event detection in multisource imaging: Application to fine arts painting analysis~// IEEE 
Trans. Acoustics Speech Signal Processing, 1990. Vol.~38. No.\,1. P.~695--704.

\bibitem{6mur}
\Au{Kammerer P., Hanbury A., Zolda~E.}
A~visualization tool for comparing paintings and their underdrawings~// Conference on Electronic 
Imaging and the Visual Arts (EVA 2004) Proceedings.~--- Florence, Italy, 2004. P.~148--153.

\bibitem{7mur}
\Au{Daly S.}
 The visible differences predictor: An algorithm for the assessment of image fidelity. Digital images 
and human vision.~--- Cambridge: MIT Press, 1993.

\bibitem{8mur}
\Au{Petrovic V., Xydeas C.} Evaluation of image fusion performance with visible differences~// 
ECCV'2004, LNCS, 2004. Vol.~3023. P.~380--391.

\bibitem{9mur}
\Au{Garcia J.\,A., Fdez-Valdivia J., Fdez-Vidal X.\,R., Rodriguez-Sanchez~R.}
Information theoretic measure for visual target distinctness~// IEEE Trans. Pattern 
Analysis Machine Intelligence, 2001. Vol.~23. No.\,4. P.~362--383.

\bibitem{10mur}
\Au{Viola P.}
Alignment by maximization of mutual information. Ph.D. Thesis.~--- Cambridge, MA: MIT, 1995.

\bibitem{11mur}
\Au{Escolano F., Suau P., Bonev~B.}
Information theory in computer vision and pattern recognition.~--- London: Springer-Verlag, 2009.

\bibitem{12mur}
\Au{Zhang J., Rangarajan A.}
Affine image registration using a new information metric~// IEEE Computer Vision and Pattern 
Recognition (CVPR), 2004. Vol.~1. P.~848--855.

\bibitem{13mur}
\Au{Rockinger O., Fechner T.}
Pixel-level image fusion: The case of image sequences~// SPIE Proceedings, 1998. Vol.~3374. 
P.~378--388.

\bibitem{14mur}
\Au{Gallager R.\,G.}
Information theory and reliable communication.~--- New York: J.\ Wiley Inc., 1968.

\bibitem{15mur}
\Au{Rajwade A., Banerjee A., Rangarajan~A.}
Continuous image representations avoid the histogram binning problem in mutual information 
based image registration~// IEEE Symposium (International) on Biomedical Imaging \mbox{(ISBI)} 
Proceedings, 2006. P.~840--843.

\bibitem{16mur}
\Au{Soille P.}
Morphological image analysis: Principles and applications.~--- Berlin: Springer-Verlag, 2004.

\label{end\stat}


\bibitem{17mur}
\Au{Niblack W.}
An introduction to digital image processing.~--- Englewood Cliffs, NJ: Prentice Hall, 1986.
 \end{thebibliography}
}
}


\end{multicols}       