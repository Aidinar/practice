   \vspace*{-36pt}

\begin{center}
\vspace*{6pt}
\mbox{%
\epsfxsize=79.5mm
\epsfbox{korov-tg.eps}
}
\end{center} 

\vspace*{12pt} %Академик


   \begin{center}
\fbox{\Large\textbf{Академик Сергей Константинович Коровин}}\\[12pt]
\textbf{\large 24.05.1945--7.12.2011}
   \end{center}
   
   %\vspace*{2.5mm}
   
   \vspace*{5mm}
   
   \thispagestyle{empty}

%\

%\vspace*{-12pt}


Редакционная коллегия журнала <<Информатика и её применения>> с глубоким 
прискорбием извещает, что 7 декабря~2011~года на 67-м году жизни скоропостижно 
скончался выдающийся российский ученый в области теории управления сложными 
динамическими системами, член редколлегии журнала <<Информатика и её применения>> 
академик КОРОВИН Сергей Константинович.

Коровин Сергей Константинович окончил факультет радиотехники и 
кибернетики Московского физико-технического института в 1969~г. 
С~1969~г.\ по 1975~г.\ работал в Институте проблем управления АН СССР. 
Здесь же без отрыва от производства учился в аспирантуре (1971--1974), 
защитил диссертацию на степень кандидата технических наук по теме 
<<Алгоритмы оптимизации на скользящих режимах>> (1975). С~1975 по 2011~гг.\ 
работал в Институте системного анализа Российской академии наук 
в должностях от ведущего инженера 
до главного научного сотрудника, заведующего лабораторией. В~1985~г.\ защитил 
диссертацию на степень доктора технических наук по теме <<Системы управления 
с автоматически регулируемыми связями>>, в 1990~г.\ ему присвоено ученое звание профессора.

С 1989~г.\ С.\,К.~Коровин работал в МГУ им.\ М.\,В.~Ломоносова, с 1996~г.\ 
являлся профессором кафедры нелинейных динамических систем и процессов 
управ\-ле\-ния факультета вычислительной математики и кибернетики. 

В 1994~г.\ избран членом-кор\-рес\-пон\-ден\-том РАН, в 2000~г.~--- действительным членом 
РАН (2000). Лауреат Государственной премии РФ (1994), премии Совета Министров СССР (1981), 
премии Правительства РФ (2009), премии РАН им.\ А.\,А.~Андронова (2000), Ломоносовской 
премии МГУ I~степени в области науки (2002). С.\,К.~Коровин~--- автор 260~научных работ, в том 
чис\-ле 15~книг, 50~авторских свидетельств. 

Сергей Константинович~Коровин являлся членом редколлегии журнала <<Информатика и её применения>> с 
момента основания журнала и принимал активное участие в формировании редакционной 
политики журнала. 