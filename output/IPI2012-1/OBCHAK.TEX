\def\stat{abstr}
{%\hrule\par
%\vskip 7pt % 7pt
\raggedleft\Large \bf%\baselineskip=3.2ex
A\,B\,S\,T\,R\,A\,C\,T\,S \vskip 17pt
    \hrule
    \par
\vskip 21pt plus 6pt minus 3pt }

\label{st\stat}

%\def\rightmark{\ }

%1
\def\tit{SKEW STUDENT DISTRIBUTIONS, VARIANCE-GAMMA
DISTRIBUTIONS AND THEIR GENERALIZATIONS AS ASYMPTOTIC
APPROXIMATIONS}

\def\aut{V.~Korolev$^1$ and I.~Sokolov$^2$}

\def\auf{$^1$Faculty of Computational Mathematics and Cybernetics, 
M.\,V.~Lomonosov Moscow State University; IPI RAN,\\
$\hphantom{^1}$vkorolev@comtv.ru\\[1pt]
$^2$IPI RAN, isokolov@ipiran.ru}

\def\leftkol{\ } % ENGLISH ABSTRACTS}
\def\rightkol{\ } %ENGLISH ABSTRACTS}

\titele{\tit}{\aut}{\auf}{\leftkol}{\rightkol}

%\vspace*{-2pt}

\noindent
The paper demonstrates that skew Student distributions and (skew)
variance-gamma distributions can appear as limit laws in rather
simple limit theorems for regular statistics, in particular, in
the scheme of random summation of random variables and, hence, can
be regarded as asymptotic approximations to the distributions of
many processes related to the evolution of complex systems.



%\vspace*{-5pt}

\KWN{skew Student distribution; variance-gamma distribution; limit theorem; 
random sum; transfer theorem}

%\thispagestyle{myheadings}


\vskip 12pt plus 6pt minus 3pt

%2
\def\tit{MATHEMATICAL SUPPORT FOR~NONLINEAR MULTICHANNEL CIRCULAR
STOCHASTIC SYSTEMS ANALYSIS BASED ON~DISTRIBUTION PARAMETRIZATION}

\def\aut{I.\,N.~Sinitsyn}

\def\auf{IPI RAN, sinitsin@dol.ru}


\def\leftkol{\ } % ENGLISH ABSTRACTS}

\def\rightkol{\ } %ENGLISH ABSTRACTS}

\titele{\tit}{\aut}{\auf}{\leftkol}{\rightkol}

%\vspace*{-2pt}

\noindent
Theory and mathematical support for nonlinear multichannel circular 
stochastic systems (CStS) on distribution parametrization  are given. 
The main topics are: circular orthogonal expansions (COE) for circular random 
variables and stochastic processes, stochastic equatioons for multichannel CStS, 
integrodifferential equations for one- and multidimensional densities, general 
COE method, methods of wrapped normal approximation, initial and central 
moments, software tools ``CStS-ANALYSIS.'' The results are illustrated by examples.

%\vspace*{-5pt}

\KWN{analytical modeling; circular random variable;  circular stochastic process;
coefficients of circular orthogonal expansion;
``CStS-ANALYSIS;'' MATLAB;
 nonlinear multichannel stochastic system;
one- and multidimensional densities; circular orthogonal expansion;
standard density; wrapped normal density}

\vskip 12pt plus 6pt minus 3pt


%3


\def\tit{ANALYSIS AND OPTIMIZATION PROBLEMS FOR~SOME USERS 
ACTIVITY MODEL. PART~2.~INTERNAL RESOURCES OPTIMIZATION}

\def\aut{A.\,V.~Bosov}

\def\auf{IPI RAN, AVBosov@ipiran.ru
}


\def\leftkol{\ } % ENGLISH ABSTRACTS}

\def\rightkol{\ } %ENGLISH ABSTRACTS}

\titele{\tit}{\aut}{\auf}{\leftkol}{\rightkol}

%\vspace*{-2pt}

\noindent
The paper continues investigation of the mathematical model describing the activity of users 
sugessted by the author earlier. 
The problem of optimizing the distribution of information system internal resources based 
on the quadratic criterion of quality is formulated and solved. 
Suboptimal optimization algorithms are presented.

%\vspace*{-5pt}

\KWN{information system; stochastic observation system; quadratic criterion}


%\pagebreak

% \vskip 12pt plus 6pt minus 3pt

\pagebreak

\def\leftkol{\ } % ENGLISH ABSTRACTS}
\def\rightkol{\ } %ENGLISH ABSTRACTS}

 %4
\def\tit{ON A VIRTUAL WAITING TIME IN THE QUEUEING SYSTEM WITH HEAD-OF-THE-LINE
PRIORITY AND HYPEREXPONENTIAL INPUT STREAM}

\def\aut{A.\,V.~Ushakov}

\def\auf{IPI RAN, grimgnau@rambler.ru}



\titele{\tit}{\aut}{\auf}{\leftkol}{\rightkol}


\noindent
The single server queue with hyperexponential input stream, 
head-of-the-line priority discipline and two service
disciplines, FIFO (first in, first out) 
and LIFO (last in, first out), for customers of one priority 
class is considered. The distributions of the virtual waiting time for each
priority class are obtained.

\vspace*{-2pt}

\KWN{virtual waiting time; head-of-the-line priority; hyperexponential input stream}

 \vskip 12pt plus 6pt minus 3pt

%5
\def\tit{A REFINEMENT OF NONUNIFORM ESTIMATES OF~THE~RATE OF~CONVERGENCE 
IN~THE~CENTRAL LIMIT THEOREM UNDER THE~EXISTENCE OF~MOMENTS OF~ORDER~NOT~HIGHER THAN THE~SECOND}

\def\aut{S.\,V.~Popov}

\def\auf{Faculty of Computational Mathematics and Cybernetics, 
M.\,V.~Lomonosov Moscow State University,\\
$\hphantom{^1}$popovserg@yandex.ru}

%\def\leftkol{ENGLISH ABSTRACTS}
%\def\rightkol{ENGLISH ABSTRACTS}

\titele{\tit}{\aut}{\auf}{\leftkol}{\rightkol}

%\vspace*{-2pt}

\def\leftkol{ENGLISH ABSTRACTS}

\def\rightkol{ENGLISH ABSTRACTS}

\noindent
Nonuniform estimates of the rate of convergence in the central limit theorem for sums of 
independent random variables with the moments of order not higher than the second are 
specified.

\vspace*{-2pt}

\KWN{central limit theorem; convergence rate estimate; absolute constants}


 \vskip 12pt plus 6pt minus 3pt

%5-1
\def\tit{COMPUTER SYSTEM OPTIMIZATION USING SIMULATION MODEL AND~ADAPTIVE ALGORITHMS}

\def\aut{M.\,G.~Konovalov}

\def\auf{IPI RAN, mkonovalov@ipiran.ru}

%\def\leftkol{ENGLISH ABSTRACTS}
%\def\rightkol{ENGLISH ABSTRACTS}

\titele{\tit}{\aut}{\auf}{\leftkol}{\rightkol}

%\vspace*{-2pt}

\def\leftkol{ENGLISH ABSTRACTS}

\def\rightkol{ENGLISH ABSTRACTS}

\noindent
A problem of effective jobs execution control in computer system is considered.
An approach based on the use of adaptive strategies in the simulation model is proposed.
On an example of the specific computer system, an original methodology of simulation model construction
is outlined. As adaptive strategies, the gradient algorithms are used, developed in the theory of
partially observed Markov decision-making process.
The results of computational experiment are presented.

\vspace*{-2pt}

\KWN{computer systems; simulation; adaptive algorithms}

 \vskip 12pt plus 6pt minus 3pt

%5-2
\def\tit{EXTRACTION OF~IMPLICIT INFORMATION FROM~THE~TEXTS IN~NATURAL LANGUAGE: PROBLEMS AND~METHODS}



\def\aut{I.\,P.~Kuznetsov$^1$ and N.\,V.~Somin$^2$}

\def\auf{$^1$IPI RAN, igor-kuz@mtu-net.ru\\[1pt]
$^2$IPI RAN, somin@post.ru}

%\def\leftkol{ENGLISH ABSTRACTS}
%\def\rightkol{ENGLISH ABSTRACTS}

\titele{\tit}{\aut}{\auf}{\leftkol}{\rightkol}

%\vspace*{-2pt}

\def\leftkol{ENGLISH ABSTRACTS}

\def\rightkol{ENGLISH ABSTRACTS}

\noindent
A semantic-oriented linguistic processor, which
provides deep analysis of text in natural language and forms 
knowledge structures, is considered.
A significant direction in development of such processors is connected with
the extraction from the texts the named entities, their properties, and links,
presented in implicit forms. The methods of such extraction at different levels
of text analysis (lexical-grammatical, syntactical-semantic and
conceptual)
are proposed.
 

\vspace*{-2pt}

\KWN{automatic text analysis; knowledge extraction; linguistic processor;
implicit information}

\pagebreak

%5-3
\def\tit{IDENTITY AND ACCESS MANAGEMENT OF~THE~USERS' RIGHTS IN~HIGH AVAILABLE DATA CENTER}

\def\aut{M.\,V.~Benderina$^1$, 
S.\,V.~Borokhov$^2$,  V.\,I.~Budzko$^3$, P.\,V.~Stepanov$^4$, 
and~A.\,P.~Suchkov$^5$} 


\def\auf{$^1$IPI RAN, mbenderina@ipiran.ru\\
$^2$IPI RAN, sborokhov@ipiran.ru\\[1pt]
$^3$IPI RAN, vbudzko@ipiran.ru\\[1pt]
$^4$IPI RAN, pvstepanov@ipiran.ru\\[1pt]
$^5$IPI RAN, asuchkov@ipiran.ru}

%\def\leftkol{ENGLISH ABSTRACTS}
%\def\rightkol{ENGLISH ABSTRACTS}

\titele{\tit}{\aut}{\auf}{\leftkol}{\rightkol}

%\vspace*{-2pt}

\def\leftkol{ENGLISH ABSTRACTS}

\def\rightkol{ENGLISH ABSTRACTS}

\noindent
The functional and organizational charts and principles of the identity and 
access management of the users' rights, developed for the two strategies to 
information protection, which are accepted by the organization or cloud 
computing community, are stated. The works organization procedure to create 
a centralized identity and access management system of the users' rights as 
a part of the information security maintenance system for high available 
collective data-processing centers is defined.

%\vspace*{-2pt}

\KWN{high availability; information security; data-processing center}


%5-4
\def\tit{EXTENDING INFORMATION INTEGRATION TECHNOLOGIES FOR~PROBLEM SOLVING 
OVER~HETEROGENEOUS INFORMATION RESOURCES}

\def\aut{L.\,A.~Kalinichenko$^1$, S.\,A.~Stupnikov$^2$, and V.\,N.~Zakharov$^3$} 


\def\auf{$^1$IPI RAN, leonidk@synth.ipi.ac.ru\\
$^2$IPI RAN, ssa@ipi.ac.ru\\[1pt]
$^3$IPI RAN, vzakharov@ipiran.ru}

%\def\leftkol{ENGLISH ABSTRACTS}
%\def\rightkol{ENGLISH ABSTRACTS}

\titele{\tit}{\aut}{\auf}{\leftkol}{\rightkol}

%\vspace*{-2pt}

\def\leftkol{ENGLISH ABSTRACTS}

\def\rightkol{ENGLISH ABSTRACTS}

\noindent
This position paper is an attempt to match up the emerging 
challenges for problem solving over heterogeneous distributed 
information resources. State-of-the-art in subject mediation 
technology reached at IPI RAN is presented. The technology 
is aimed at filling the widening gap between the users 
(applications) and heterogeneous resources of data, knowledge, 
and services. Also, the paper affects the semantic-based information 
integration technologies challenges including investigation of 
application-driven approach for problem solving in the subject 
mediator environment, a provision for support of executable 
declarative specifications of the applications over the mediator, 
enhancement of presence of knowledge-based facilities at the mediator level, 
and mediation of databases with nontraditional data models motivated by the need 
of large data support.

%\vspace*{-2pt}

\KWN{subject mediation; heterogeneous information resources; 
scientific problem solving; information integration; application-driven approach; 
rule-based languages; nontraditional data models}

 \vskip 12pt plus 6pt minus 3pt
 


%6
\def\tit{THE MOTIF INFORMATION ANALYSIS BASED ON~THE~SOLVABILITY CRITERION 
FOR~THE~PROTEIN SECONDARY STRUCTURE RECOGNITION}

\def\aut{K.\,V.~Rudakov and I.\,Yu.~Torshin$^2$}

\def\auf{$^1$Dorodnicyn Computing Center of the Russian Academy of Sciences;
Moscow Institute of Physics and Technology\\
$\hphantom{^1}$(State University), rudakov@ccas.ru\\[1pt]
$^2$Russian Center of the Trace Element Institute for UNESCO, tiy135@yahoo.com}


\def\leftkol{ENGLISH ABSTRACTS}

\def\rightkol{ENGLISH ABSTRACTS}

\titele{\tit}{\aut}{\auf}{\leftkol}{\rightkol}

%\vspace*{-2pt}

\noindent
The development of the formal description of recognition of protein 
secondary structure problem is presented. The key concepts of motif, 
informativity estimate of a motive, and the order of the motives, 
allowing to use the formalism for the analysis of actual precedent 
sets are introduced. The experiments on the solvability testing indicate a possibility 
of an efficient selection of the most informative motifs.
%\vspace*{-3pt}

\KWN{algebraic approach; bioinformatics; locality; solvability; feature value classification}
%\pagebreak

\vskip 12pt plus 6pt minus 3pt

%7
\def\tit{SPEAKER IDENTIFICATION SYSTEM FOR~THE~\textit{NIST~SRE~2010}}

\def\aut{I.\,N.~Belykh$^1$, A.\,I.~Kapustin$^2$, A.\,V.~Kozlov$^3$, A.\,I.~Lohanova$^4$, 
Yu.\,N.~Matveev$^5$, T.\,S.~Pekhovsky$^6$, K.\,K.~Simonchik$^7$, and~А.\,K.~Shulipa$^8$ 
}

\def\auf{$^1$Speech Technology Center, St.-Petersburg, belykh@speechpro.com\\[1pt]
$^2$Speech Technology Center, St.-Petersburg, kapustin@speechpro.com\\[1pt]
$^3$Speech Technology Center, St.-Petersburg, kozlov-a@speechpro.com\\[1pt]
$^4$Speech Technology Center, St.-Petersburg, lohanova@speechpro.com\\[1pt]
$^5$Speech Technology Center, St.-Petersburg, matveev@speechpro.com\\[1pt]
$^6$Speech Technology Center, St.-Petersburg, tim@speechpro.com\\[1pt]
$^7$Speech Technology Center, St.-Petersburg, simonchik@speechpro.com\\[1pt]
$^8$Speech Technology Center, St.-Petersburg, shulipa@speechpro.com}


\def\leftkol{ENGLISH ABSTRACTS}

\def\rightkol{ENGLISH ABSTRACTS}

\titele{\tit}{\aut}{\auf}{\leftkol}{\rightkol}

%\vspace*{-2pt}

\noindent
A description of a speaker identification system by voice 
is presented. This system was developed for submission on 
speaker recognition system evaluation at \textit{NIST~SRE~2010}.
 
%\vspace*{-2pt}

\KWN{biometry; speaker identification; voice recognition; pitch; formants; GMM; SVM; NIST}

%\pagebreak

\vskip 12pt plus 6pt minus 3pt

% \vskip 12pt plus 6pt minus 3pt

%8
\def\tit{FAST PROCESSING OF~FINGERPRINT IMAGES}

\def\aut{V.\,J.~Gudkov$^1$ and M.\,V.~Bokov$^2$}

\def\auf{$^1$Chelyabinsk State University, Department of Applied Mathematics, diana@sonda.ru\\[1pt]
$^2$South Ural State University, Department of Applied Mathematics, guardian@mail.ru}


\def\leftkol{ENGLISH ABSTRACTS}

\def\rightkol{ENGLISH ABSTRACTS}

\titele{\tit}{\aut}{\auf}{\leftkol}{\rightkol}

%\vspace*{-2pt}

\noindent 
A consistent method of identification of individual
features from fingerprint image with the severe restrictions is briefly described.
Individual features are stored in the image template. The templates are used for 
fingerprint identification.

%\vspace*{-2pt}

\KWN{fingerprint; image processing; flow matrix; period matrix; minutiae}

  \vskip 12pt plus 6pt minus 3pt
  
  %9
\def\tit{TEACHING OF SKIN EXTRACTION ALGORITHMS FOR~HUMAN FACE COLOR IMAGES}

\def\aut{Y.~Vizilter$^1$, V.~Gorbatcevich$^2$, S.~Karateev$^3$, and~N.~Kostromov$^4$}

\def\auf{$^1$State Research Institute of Aviation Systems (GosNIIAS), viz@gosniias.ru\\[1pt]
$^2$State Research Institute of Aviation Systems (GosNIIAS), gvs@gosniias.ru\\[1pt]
$^3$State Research Institute of Aviation Systems (GosNIIAS), goga@gosniias.ru\\[1pt]
$^4$State Research Institute of Aviation Systems (GosNIIAS)}

\def\leftkol{ENGLISH ABSTRACTS}

\def\rightkol{ENGLISH ABSTRACTS}

\titele{\tit}{\aut}{\auf}{\leftkol}{\rightkol}

%\vspace*{-2pt}

\noindent
Two methods for teaching of algorithms for the skin extraction in color images 
of human faces are proposed and discussed. The first method is based on self-organizing 
neural network called ``growing neural gas.'' 
The second one is based on morphological 
classification by minimal cutting of neighborhood graph for a training set in color space. 
The CIE Lab color space is applied for color description in both cases. The efficiency of 
both methods is demonstrated. The differences in selection results of the proposed methods are 
explored and demonstrated.


%\vspace*{-2pt}

\KWN{biometrics; human skin extraction; self-organizing neural networks; 
morphological classification; graph cut}


%\vskip 12pt plus 6pt minus 3pt
\pagebreak

%10
\def\tit{REAL-TIME HAND GESTURE RECOGNITION BY~PLANAR AND~SPATIAL SKELETAL MODELS}

\def\aut{A.\,V.~Kurakin}

\def\auf{Moscow Institute of Physics and Technology (State University), alekseyvk@yandex.ru}


\titele{\tit}{\aut}{\auf}{\leftkol}{\rightkol}

%\vspace*{-2pt}

\noindent
The hand gesture recognition problem is considered. Algorithm to detect planar
positions of fingertips by the image of hand silhouette is proposed. It operates with the
analysis of continuous skeleton of the hand shape for
silhouette segmentation. An extension of the algorithm to spatial case is
presented. It uses stereo pair of hand silhouettes to estimate spatial positions of hand
and fingertips. The developed methods work in real time, allowing their use in
applied hand gesture recognition systems.

%\vspace*{-2pt}

\KWN{continuous skeleton; shape analysis; gesture recognition; stereovision}

%\pagebreak

\vskip 12pt plus 6pt minus 3pt

%11
\def\tit{COMBINED APPROACH TO~LOCALIZATION OF~DIFFERENCES FOR~MULTIMODAL IMAGES}

\def\aut{D.\,M.~Murashov}

\def\auf{Dorodnicyn Computing Center, Russian Academy of Sciences, d\_murashov@mail.ru}


\def\leftkol{ENGLISH ABSTRACTS}
\def\rightkol{ENGLISH ABSTRACTS}

\titele{\tit}{\aut}{\auf}{\leftkol}{\rightkol}

%\vspace*{-2pt}
\noindent
An approach to the problem of localization of the differences in multimodal
images is suggested. The approach is based on specific object detectors and local
information-theoretical image difference measures implemented as conditional entropy of the
analyzed image pair. The proposed approach is applied to the problem of detecting repainting areas of
fine art paintings using the images acquired in visible and ultraviolet spectral ranges.

%\vspace*{-2pt}

\KWN{multimodal images; measure of image difference; information-theoretical measure;
conditional entropy; images of fine art paintings}


\vskip 12pt plus 6pt minus 3pt


%12
\def\tit{SECURED BIOMETRIC VERIFICATION BASED ON~FINGERPRINT TOPOLOGY BINARY REPRESENTATION}

\def\aut{O.\,S.~Ushmaev$^1$ and V.\,V.~Kuznetsov$^2$}

\def\auf{$^1$IPI RAN, oushmaev@ipiran.ru\\
$^2$IPI RAN, k.v.net@rambler.ru}


\def\leftkol{ENGLISH ABSTRACTS}

\def\rightkol{ENGLISH ABSTRACTS}

\titele{\tit}{\aut}{\auf}{\leftkol}{\rightkol}

%\vspace*{-2pt}
\noindent
The paper deals with combination of cryptographic constructions and 
fingerprint identification. A technique of repeatable binary 
string extraction from fingerprint images is suggested. The binary features 
from topological relations between minutiae points are extracted. For an arbitrary 
minutiae point, the neighboring ridges are traced until  the event is encountered: 
minutiae or projection of minutiae. Then, these events are encoded. Thus, 
50--100-bit descriptions for each minutiae point are obtained. In order to 
extract longer binary string, two techniques are suggested. The first one is the 
self-aligned technique, while the second one requires public helper. Thus, 
 384--756-bit binary string is extracted. The strings have approximately 20\% 
erroneous bits, which are corrected using two-layer Bose--Chaudhuri--Hocquenghem (BCH)
major voting codes. 
The experiments on FVC2002 DB1 dataset show that 20--40-bit error-free 
binary string can be reproduced from genuine fingerprint with 90~percent success rate.


 \label{end\stat}

%\vspace*{-5pt}

\KWN{secured biometric verification; fuzzy extractor; fingerprint}


%\pagebreak
\newpage