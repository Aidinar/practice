 
%\usepackage[T1]{fontenc}      %векторные шрифты, ещё нужно cm-super поставить


%\newtheorem{definition}{Определение}
%\newtheorem{theorem}{Теорема}
%\newtheorem{proposition}{Предложение}
%\renewcommand{\captionlabeldelim}{.}

\newcommand{\med}{\textrm{med}}
\newcommand{\sgn}{\textrm{sgn}}

%\renewcommand{\figurename}{\textbf{Рис.}}

\def\stat{markin}


\def\tit{ОТСЕВ ЭКТОПИЧЕСКИХ ИМПУЛЬСОВ ИЗ РИТМОГРАММЫ С~ИСПОЛЬЗОВАНИЕМ
РОБАСТНЫХ ОЦЕНОК}
\def\titkol{Отсев эктопических импульсов из ритмограммы с использованием
робастных оценок}
\def\autkol{А.\,В.~Маркин, О.\,В.~Шестаков}
\def\aut{А.\,В.~Маркин$^1$, О.\,В.~Шестаков$^2$}

\titel{\tit}{\aut}{\autkol}{\titkol}

%{\renewcommand{\thefootnote}{\fnsymbol{footnote}}\footnotetext[1]
%{Работа выполнена при
%поддержке РФФИ, гранты 08-01-00345, 08-01-00363,
%08-07-00152.}}

\renewcommand{\thefootnote}{\arabic{footnote}}
\footnotetext[1]{Московский государственный
университет им.~М.\,В.~Ломоносова, кафедра
математической статистики факультета ВМиК, artem.v.markin@mail.ru}
\footnotetext[2]{Московский
государственный университет им.~М.\,В.~Ломоносова,
кафедра
математической статистики факультета ВМиК, oshestakov@cs.msu.su}

\Abst{Предложен метод удаления эктопических
импульсов из ритмограммы на основе собственной математической
модели. Параметры модели оцениваются с использованием робастного
варианта линейной регрессии. Отсев эктопических импульсов
производится с помощью доверительных интервалов для разностей длин
$RR$-интервалов. Приведены результаты работы метода на реальных
данных.}

\KW{ритмограмма; робастные оценки; линейная регрессия; доверительные интервалы}

      \vskip 24pt plus 9pt minus 6pt

      \thispagestyle{headings}

      \begin{multicols}{2}

      \label{st\stat}

\section{Введение}

Ритмограмма $RR$-интервалов как мера вариабельности сердечного
ритма (ВСР) получила в\linebreak
 последние годы достаточно широкое
распространение в диагностике сердечно-сосудистых заболеваний
(ССЗ)~\cite{1ma, 2ma}. Благодаря развитию компью\-терной техники стал
возможным анализ десятков тысяч записей и выявление общих
закономер\-ностей. В частности, оказалось, что существует\linebreak
зависимость между ССЗ и спектральными характеристиками
ритмограммы. Пример электрокардиограммы (ЭКГ) и ритмограммы
приведен на рис.~\ref{pic_ecg_hrv}. График электрокардиограммы
построен для промежутка времени 81,25--86~с, ритмограмма же~---
для пятиминутного промежутка 10--310~с. Часть ритмограммы,
соответствующая промежутку 81,25--86~с, обведена пунктирным
прямоугольником.


В структуре ЭКГ выделяют несколько волн: $P$, $Q$, $R$, $S$ и $T$
(см.\ рис.~\ref{pic_ecg_hrv}). Эти волны отражают различные
состояния работы сердца. Под $RR$-ин\-тер\-ва\-ла\-ми подразумеваются
временные промежутки между пиками соседних $R$-волн.

Существует ряд рекомендаций относительно построения ритмограммы.
Они устанавливают минимальную продолжительность измерений, при
которой, во-первых, возможно вычислить спектр для определенного
диапазона частот и, во-вторых, спектральные характеристики будут
постоянны~\cite{1ma}. Однако в ритмограмме обычно присутствуют так
называемые эктопические импульсы~--- очень короткие или очень
длинные интервалы между биениями. Обычно сначала идет короткий
$RR$-интервал, а сразу после него~--- длинный. Эти импульсы имеют
иную природу, в отличие от нормальных (или, как их еще называют,
синусовых) интервалов~\cite{1ma, 3ma}. На рис.~\ref{pic_ecg_hrv} видно,
что в районе восемьдесят четвертой секунды структура кардиограммы
отклоняется от обычной, пик $R$-волны находится близко к
предыду\-ще\-му и далеко от следующего. На ритмограмме это отражается
в появлении двух скачков: сначала вниз, затем вверх. Это и есть
эктопические им\-пульсы.

Наличие всего нескольких эктопических импульсов оказывает
существенное влияние на спектр ритмограммы, поэтому обычно их
стараются удалить. Один из методов удаления эктопических импульсов
предложен в настоящей статье. Стоит отметить, что ритмограмма
относится к сигналам с неравномерным квантованием времени, так как
длина отсчета времени равна длине очередного $RR$-интервала. Это
необходимо учитывать при вычислении спектральных характеристик.

\section{Классический метод отсева}

Весьма популярный метод удаления эктопических импульсов состоит в
следующем~\cite{2ma}:
\begin{enumerate}[1.]
    \item Удалить все интервалы короче 300~мс и длиннее 2000~мс.
    \item Удалить все интервалы, каждый из которых отличается от предыдущего более чем на 200 мс.
    \item Удалить интервалы, отличающиеся по длине более чем на 20\% от среднего значения, где
    среднее вычисляется по пяти предыдущим синусовым интервалам.
\end{enumerate}

\begin{figure*} %fig1
%\centerline{\includegraphics[width=150mm,height=86.21mm]{ecg_hrv.eps}}
\vspace*{1pt}
\begin{center}
\mbox{%
\epsfxsize=159.156mm
\epsfbox{mar-1.eps}
}
\end{center}
\vspace*{-6pt}
\Caption{Электрокардиограмма (\textit{а}) и соответствующая ей ритмограмма~(\textit{б})
\label{pic_ecg_hrv}}
\vspace*{3pt}
\end{figure*}

Критические значения  вычислены по базе ЭКГ
Массачусетского технологического института (MIT-BIH Normal Sinus
Rhythm Database), насчитывающей десятки тысяч записей. Эти
значения являются некоторым усреднением, и поэтому вполне может
случиться так, что какие-то эктопические значения не будут отсеяны
и останутся в сигнале. При таком подходе фактически не учитывается
форма (структура) ритмограммы и ее индивидуальные особенности
(амплитуда колебаний). В~предлагаемом методе присутствует
адаптация под конкретный сигнал.


\section{Математические модели ритмограммы}%

Для сравнения методов обработки ритмограмм можно использовать как
реальные данные, так и математическую модель. Хотя модель всегда
является лишь приближением, она обычно обладает вполне
определенными параметрами, что позволяет оценивать методы
непосредственно по точности вычисления этих параметров.

\begin{figure*} %fig2
%\centerline{\includegraphics[width=150mm,height=86.21mm]{real_rhythm_clean.eps}}
\vspace*{1pt}
\begin{center}
\mbox{%
\epsfxsize=159.286mm
\epsfbox{mar-2.eps}
}
\end{center}
\vspace*{-9pt}
\Caption{Ритмограмма (\textit{а}) и график амплитуды частотных компонент~(\textit{б}). \label{pic_real_rhythm_clean}
Эктопические импульсы удалены вручную 
}
\end{figure*}

На практике для спектрального анализа  обычно используются
ритмограммы двух типов: короткие (5~мин) и длинные (24~ч).
Частотные характеристики 5-минутных ритмограмм можно считать
постоянными при выполнении ряда условий, тогда как измерения,
полученные за 24~ч, таким свойством не обладают~\cite{1ma}. При этом
длинные записи несут более полную спектральную информацию. Далее
будут рассматриваться только короткие ритмо-\linebreak граммы.

При работе с 5-минутными ритмограммами  выделяют два~диапазона
час\-тот: диапазон высоких час\-тот (ВЧ) (0,15--0,4~Гц) и диапазон
низких частот (НЧ) (0,04--0,15~Гц). Анализ данных показал~\cite{1ma, 2ma},
что в этих диапазонах часто можно четко выделить два биоритма:
волну Майера (Mayer wave) в НЧ области и дыхательную волну
(Respiratory Sinus Arrhythmia wave) в ВЧ области. Волна Майера
соответствует колебаниям кровяного давления, а дыхательная волна
отражает влияние дыхания на сердечный ритм.
На рис.~\ref{pic_real_rhythm_clean} представлена ритмограмма из
рис.~\ref{pic_ecg_hrv} с эктопическими импульсами, удаленными вручную,
и график амплитуды ее час\-тот\-ных компонент. Хорошо видны пики в НЧ
и ВЧ областях.

Как видим, на спектре имеются два подъема с четкими максимумами, по одному в 
НЧ и ВЧ диапазонах. Это наталкивает на мысль о пред\-став\-ле\-нии ритмограммы 
в виде суммы двух синусоид с круговыми частотами, соответствующими пикам спектра 
в НЧ и ВЧ областях. Таким образом, получаем представление:
\begin{multline}
\label{modelsin}
RR(t) = A_1\sin(\omega_1 t+\phi_1)+{}\\
{}+A_2\sin(\omega_2 t+\phi_2)+\epsilon(t)+C\,.
\end{multline}
Здесь
$RR(t)$~--- ритмограмма;
$\omega_1=2\pi f_1$ и $\omega_2\;=$\linebreak $=\;2\pi f_2$~--- круговые частоты, 
соответствующие час\-то\-там $f_1$ и $f_2$, где
    $0{,}04\leq f_1\leq 0{,}15$~Гц, $0{,}15 <f_2\;\leq$\linebreak $\leq\;0{,}4$~Гц;
 $A_1$ и $A_2$~--- амплитуды; $\phi_1$ и $\phi_2$~--- фазы синусоид;
 $\epsilon(t)$~--- шум, нормально распределенные случайные величины с нулевым средним и дисперсией $\sigma^2$;
 $C$~--- некоторая константа, выполняющая роль сдвига графика ритмограммы относительно нуля.

 Время меняется дискретно: 
 $$
 t \in \{t_1,t_2,\ldots,
t_N\,|\,0=t_1<t_2<\ldots<t_N \}\,.
$$ 
Следует отметить, что данная
модель является весьма грубой, так как форма реальной ритмограммы
несколько сложнее, чем сумма двух синусоид. Однако она вполне
достаточна, ведь стоит задача оценки амплитуды синусовой части
ритмограммы, а синусовая часть является довольно <<гладкой>> в
смысле малых изменений в соседних точках, в то время как
эктопические импульсы соответствуют скачкам на графике
ритмограммы.

\section{Построение доверительных интервалов}%

Идея предлагаемого метода состоит в сле\-ду\-ющем:
\begin{enumerate}[1.]
    \item Оценить параметры $\omega_1$, $\omega_2$, $\phi_1$, $\phi_2$, $A_1$ и $A_2$ по спектру с помощью робастного варианта регрессии.
    \item Оценить $\sigma^2$.\\[-14pt]
    \item Построить доверительные интервалы для разностей $RR(t_i)-RR(t_{i-k})$, 
    $i=1,2,\ldots,N$, $1\leq k\leq i$, c уровнем доверия не ниже $1-\gamma$. 
    В качестве $\gamma$ можно взять, например, $\gamma=0{,}05$.\\[-14pt]
    \item Отсеять импульсы, не попадающие в доверительные интервалы для разностей процесса $RR(t)$.\\[-14pt]
\end{enumerate}

Для разностей гармонической части 
$RR_h(t)\;\equiv$\linebreak $\equiv\;A_1\sin(\omega_1 t + \phi_1) + A_2\sin(\omega_2 t + \phi_2)$ будет выполнено:
\begin{multline}
\label{harmonicdiff}
   \left|RR_h(t_{i+k}) - RR_h(t_i)\right|\quad \leq {}\\
   \leq 4A_1\left|\sin\left(\fr{\omega_j (t_{i+k}-t_i)}{2}\right)\right| + {}\\
   {}+4A_2\left|\sin\left(\fr{\omega_j (t_{i+k}-t_i)}{2}\right)\right| \equiv h_i^k\,.
\end{multline}

При построении доверительных интервалов предполагается, что
дисперсия шума $\sigma^2$ известна. Тогда нормальная случайная
величина с нулевым средним и дисперсией $\sigma^2$ с вероятностью
$1-\gamma$ попадает в интервал
$\left(\sigma\Phi^{-1}\left(\gamma/2\right),
\sigma\Phi^{-1}\left(1-\gamma/2\right)\right)$, где
$\Phi(x)$ суть функция распределения стандартной нормальной
случайной величины. Причем длина этого интервала минимальна среди
всех интервалов $\mathcal{I}$ таких, что $P(\mathcal{I}) =
1-\gamma$ ($P$~--- распределение стандартной нормальной случайной
величины)~\cite{4ma}.

Доверительные интервалы для разностей $RR(t)$ будут выглядеть так:
\begin{multline}
\label{confint}
\left( - h_i^k + \sqrt{2}\cdot\sigma\Phi^{-1}\left(\fr{\gamma}{2}\right)\,,\right.\\
\left. h_i^k + \sqrt{2}\cdot\sigma\Phi^{-1}\left(1-\fr{\gamma}{2}\right) \right)\,.
\end{multline}

Однако $\sigma$ необходимо оценить. Для этого можно использовать разности 
$\Delta RR_i=RR(t_i)-RR(t_{i-1})$ функции $RR(t)$, исключив некоторым образом 
из этих разностей влияние полезного сигнала. Предлагается делать это в соответствии 
с моделью~(\ref{modelsin}): в качестве шума рассматривать значения
\begin{equation*}
%\label{purenoise}
r_i = \Delta RR_i - \left(RR_h(t_i)-RR_h(t_{i-1})\right)\,.
\end{equation*}
Коэффициенты $RR_h$ оцениваются методом робастной регрессии (см.\ разд.~5).

В качестве оценки для $\sqrt{2}\cdot\sigma$ используем нормированное
абсолютное медианное отклонение от медианы $\sigma^*$:
\begin{equation*}
%\label{definMAD}
\sigma^* = \fr{\med|r_i-\med
r_i|}{\Phi^{-1}(3/4)}\,.
\end{equation*}
Эта оценка является наиболее
робастной, ее пороговая точка (см.\ разд.~5) равна~0,5. Кроме того,~$\sigma^*$ 
будет состоятельной оценкой стандартного отклонения
$\sigma$ нормальной случайной величины~\cite{5ma}. Таким образом,
осталось вычислить $\omega_1$, $\omega_2$, $\phi_1$, $\phi_2$,
$A_1$ и $A_2$.

\section{Метод робастной регрессии \label{sect_robustregr} } %

Оценку параметров предлагается провести робастным вариантом
линейной регрессии. Под робастностью понимают нечувствительность к\linebreak
небольшим отклонениям от начальных предположений. Следует
отметить, что наличие больших ошибок в измерениях, составляющих
небольшую часть выборки, считается небольшим отклонением. 
В~рассматриваемом случае такими ошибочными измерениями являются
эктопические импульсы.

Классическое решение задачи линейной регрессии методом наименьших
квадратов строится следующим образом~\cite{5ma}. Необходимо оценить $p$
неизвестных параметров
$\bm{\uptheta}=(\theta_1,\theta_2,\ldots,\theta_p)^T$ по $N$
наблюдениям $\mathbf{y}=(y_1,y_2,\ldots,y_N)^T$, причем имеет
место связь:
\begin{equation*}
%\label{classlinregr}
\mathbf{y = X\bm{\uptheta}+u}\,,
\end{equation*}
где $x_{ij}$~--- некоторые
известные постоянные коэффициенты, $u_i$~--- независимые одинаково
распределенные случайные величины, $i=1,\ldots,N$, $j=1,\ldots,p$.

Эта задача сводится к решению задачи:
\begin{equation}
\label{classlinregrdiffmat}
\mathbf{X}^T\mathbf{X}\bm{\uptheta}=\mathbf{X}^T\mathbf{y}\,.
\end{equation}
Если матрица $\mathbf{X}$ имеет ранг $p$, то
решение системы~(\ref{classlinregrdiffmat}) будет выглядеть так:
\begin{equation*}
%\label{classlinregrsol}
\hat{\bm{\uptheta}} =
(\mathbf{X}^T\mathbf{X})^{-1}\mathbf{X}^T\mathbf{y}\,.
\end{equation*}

Формула~(\ref{classlinregrdiffmat}) не является робастной, так как
квадратичная функция слишком чувствительна к ошибкам, особенно
большим. Выход заключается в замене ее на некоторую менее быстро
растущую функцию остатков $\rho$ и решении следующей задачи:
\begin{equation}
\label{robustlinregr}
\sum_{i=1}^N\rho\left(y_i-\sum_{j=1}^p
x_{ij}\theta_j\right)\rightarrow\min\,.
\end{equation}
Ограничения
на $\rho$: выпуклость, немонотонность, наличие ограниченных
производных достаточно высоких порядков. В частности, $\psi=\rho'$
должна быть непрерывной ограниченной функцией~\cite{5ma}. В~этом случае
не появятся новые корни при переходе от задачи~(\ref{robustlinregr})
к системе
\begin{multline}
\label{robustlinregrdiff}
\sum_{i=1}^N\psi\left(y_i-\sum_{k=1}^p
x_{ik}\theta_k\right)x_{ij}=0\,,\\
j=1,\ldots,p\,,\quad \psi=\rho'\,.
\end{multline}
В качестве $\rho$ предлагается
использовать следующую функцию~\cite{6ma}:
\begin{equation*}
%\label{rhocustom}
\hat{\rho}(x)=
\begin{cases}
c\left(1-\cos\left(\fr{x}{c}\right)\right) & \text{при } |x|<c\pi\,,\\
2c & \text{при } |x|\geq c\pi\,.
\end{cases}
\end{equation*}
Ее производная равна
\begin{equation*}
%\label{psicustom}
\hat{\psi}(x)=\begin{cases}
\sin\left(\fr{x}{c}\right) & \text{при } |x|<c\pi\,,\\
0 & \text{при } |x|\geq c\pi\,.
\end{cases}
\end{equation*}
Вблизи нуля функция $\hat{\rho}$ имеет квадратичный порядок роста,
а начиная с некоторого значения~--- нулевой. В качестве $c$ берется
значение $\hat{c}=0{,}85$, которое было подобрано опытным путем.
Полезным свойством данной функции является то, что очень большие
по модулю значения аргумента не оказывают влияния на
сумму~(\ref{robustlinregrdiff}), так как значение функции $\hat{\psi}$ в
этих точках равно нулю. Это полностью согласуется с практикой,
которая говорит, что очень длинные ($\geq 2000$~мс) и очень
короткие интервалы ($\leq 300$~мс) однозначно являются
эктопическими. Именно такие интервалы дают большие абсолютные
значения разностей.

В робастном варианте регрессии возникает задача оценки масштаба,
так как функции $\rho$ и $\psi$ практически всегда имеют разный
порядок роста для разных значений аргумента. Соответственно,
параметр масштаба $s$ позволяет выполнить нормировку данных для
эффективного деления на большие и маленькие значения. Вместо
системы~(\ref{robustlinregrdiff}) нужно решать
\begin{multline}
\label{robustlinregrdiffscale}
\sum_{i=1}^N\psi\left(\frac{y_i-\sum_{k=1}^p
x_{ik}\theta_k}{s}\right)x_{ij}=0\,, \\
 j=1,\ldots,p\,.
\end{multline}
Значения $s$ и $\bm\uptheta$ находятся итеративным
путем, но так как функции $\sum\limits_j x_{ij}\theta_j$ являются
линейными относительно $\theta_j$, то решение находится за один
шаг. В~качестве начального значения параметра масштаба~$\tilde{s}$
можно взять абсолютное медианное отклонение от медианы (AMO)
\begin{equation*}
%\label{robustsigmaclassic}
\tilde{s} =
\med\left|RR_i-\med(RR_i)\right|\,,
\end{equation*}
которое обладает
хорошими теоретическими свойствами, в частности AMO имеет
пороговую точку $\epsilon^*=0{,}5$. На интуитивном уровне понятие
пороговой точки можно объяснить как максимальную долю паразитных
импульсов, с которыми может справиться оценка.

\medskip

\noindent
{\bf Определение 1}~[5]. %\label{def_treshpoint} 
 {\it
Асимптотическая пороговая точка $\epsilon^*$ оценки $T_n$ для
распределения $F_0$, принятого в модели, определяется следующим
образом:
\begin{equation*}
%\label{treshpoint}
\epsilon^* =
\epsilon^*(T_n,F_0)\equiv\sup\{\epsilon\,|\,b(\epsilon)<b(1)\}\,.
\end{equation*}
Здесь
\begin{equation*}
b(\epsilon)=\lim_{n\rightarrow\infty}\sup_{F\in\mathcal{P}_\epsilon}|M(F,T_n)|\,,
\end{equation*}
где $M(F,T_n)$~--- медиана распределения}
$\mathfrak{L}_F[T_n\;-$\linebreak $-\;T(F_0)]$.

\smallskip

В качестве
$\mathcal{P}_\epsilon$ может выступать \emph{окрестность}\linebreak \emph{ Леви}:
\begin{equation*}
%\label{Levineighborhood}
\mathcal{P}_\epsilon(F_0)=\{F\,|\,\forall
t\,F_0(t-\epsilon)-\epsilon\leq F(t)\leq
F_0(T+\epsilon)+\epsilon\} 
\end{equation*}
или \textit{окрестность
загрязнения}:
\begin{equation*}
%\label{dirtyneighborhood}
\mathcal{P}_\epsilon(F_0)=\{F\,|\,F=(1-\epsilon)F_0+\epsilon H\}\,,
\end{equation*}
где $H$~--- некоторое (загрязняющее) распределение.

Максимально возможное значение пороговой точки равно 0,5, и для
оценки $\tilde{s}$ она этого значения достигает. Однако если
известно, что доля эктопических выбросов в сигнале не превышает
$0<\alpha<0{,}5$, то в качестве оценки масштаба можно взять оценку с
пороговым значением не меньше~$\alpha$, которая лучше оценит
разброс полезных данных. В~качестве такой оценки предлагается
взять половину разности между квантилями порядка $\alpha$ и
$1-\alpha$:
\begin{equation*}
%\label{robustsigmastart}
\hat{s} =
\fr{\left(RR_{(1-\alpha)}-RR_{(\alpha)}\right)}{2}\,.
\end{equation*}

\begin{figure*} %fig3
%\centerline{\includegraphics[width=150mm,height=86.21mm]{model_function.eps}}
\vspace*{1pt}
\begin{center}
\mbox{%
\epsfxsize=157.775mm
\epsfbox{mar-3.eps}
}
\end{center}
\vspace*{-9pt}
\Caption{Модельная функция~(\textit{а}) и график амплитуды частотных
компонент~(\textit{б}) 
\label{pic_modelfunction}}
\end{figure*}

\medskip

\noindent
{\bf Предложение 1} %\label{prop_treshpointLest} 
[5]. 
\textit{Если
$\alpha>0$, то при условии, что мера $\mu$ не имеет ненулевой
массы в точках разрыва функции $F_0^{-1}$, функционал
$T(F)\equiv\int F^{-1}(s)\mu(ds)$ слабо непрерывен в $F_0$.
Пороговая точка $\epsilon^*$ удовлетворяет неравеству
$\epsilon^*\geq\alpha$. Если $\mu$~--- положительная мера, то
$\epsilon^*=\alpha$ и равенство $\alpha=0$ влечет за собой
разрывность} $T$. 

Стало быть, пороговое значение
для оценки $\hat{s}$ не меньше $\alpha$. Начальной оценкой для
$\bm{\uptheta}$ является решение задачи~(\ref{classlinregrdiffmat}),
которое обозначим $\bm{\uptheta}^0$.
Шаг метода заключается в вычислении винзоризованных остатков
$r_i$:
\begin{equation*}
%\label{winsoresid}
r_i =
\psi\left(\fr{y_i - \sum_{k=1}^p
x_{ik}{\theta}^0_k}{\hat{s}}\right)\hat{s}\,, \quad i=1,\ldots,N\,,
\end{equation*}
и решении относительно $\bm{\uptau}$ системы
\begin{equation}
\label{winsosol}
\mathbf{X}^T\mathbf{X}{\bm\uptau}=\mathbf{X}^T\mathbf{r}\,.
\end{equation}
Если решение~(\ref{winsosol}) обозначить
$\hat{\bm{\uptau}}$, то решение $\hat{\bm{\uptheta}}$
сис\-те\-мы~(\ref{robustlinregrdiffscale}) равно
\begin{equation}\label{robustthetasol}
\hat{\bm{\uptheta}} =
\bm{\uptheta}^0 + \hat{\bm{\uptau}}\,.
\end{equation}

В качестве факторов регрессии предлагается взять
ортогонализированные единичный и тригонометрические векторы. Под
единичным подразумевается вектор, полностью составленный из
единиц. Такой вид линейной регрессии называется компенсированным
дискретным преобразованием Фурье (КДПФ) (date-compensated discrete
Fourier transform)~\cite{7ma}. Дело в том, что неравномерное квантование
времени, а именно такое квантование имеет место при записи
ритмограммы, нарушает ортогональность и нормировку
тригонометрических векторов $\mathbf{X}_2$, $\mathbf{X}_3$ и
единичного вектора $\mathbf{X}_1$:

\noindent
\begin{align}
\mathbf{X}_1&=(1,1,\ldots,1)\,;\label{trialones}\\
\mathbf{X}_2&=(\cos(\omega t_1),\cos(\omega t_2),\ldots,\cos(\omega
t_N))\,; \label{trialcos}\\
\mathbf{X}_3&=(\sin(\omega t_1),\sin(\omega t_2),\ldots,\sin(\omega
t_N))\,.
\label{trialsin}
\end{align}
В результате амплитуда частотных компонент
оценивается неправильно, причем в отдельных случаях отличия от
реальных значений амплитуд весьма существенные~\cite{8ma}.


Эту проблему решает КДПФ. К трем векторам~(\ref{trialones})--(\ref{trialsin})
применяется процедура ортогонализации Грама--Шмидта:
\begin{align*}
\mathbf{x}_1 &= a_1 \mathbf{X}_1\,; \\
\mathbf{x}_2 &= a_2 \left[ \mathbf{X}_2 - \mathbf{x}_1(\mathbf{x}_1,\mathbf{X}_2) \right]\,; \\
\mathbf{x}_3 &= a_3 \left[ \mathbf{X}_3 -
\mathbf{x}_1(\mathbf{x}_1,\mathbf{X}_3) -
\mathbf{x}_2(\mathbf{x}_2,\mathbf{X}_3) \right]\,,
\end{align*}
где
круглыми скобками обозначено скалярное произведение. Коэффициенты
$a_1$, $a_2$ и $a_3$ находятся из условия ортонормированности
векторов $\mathbf{x}_1$, $\mathbf{x}_2$ и~$\mathbf{x}_3$. Здесь
частота $\omega$ фиксирована и $p=3$. Поэтому коэффициенты
регрессии выглядят так:
\begin{align*}
%\label{dcdftcoeff}
\theta_1 & = (\mathbf{z},\mathbf{x}_1) = a_1 (\mathbf{z},\mathbf{X}_1)\,; \\
\theta_2 &= (\mathbf{z},\mathbf{x}_2) = a_2 \left[ (\mathbf{z},\mathbf{X}_2)-a_1\theta_1(\mathbf{X}_1,\mathbf{X}_2) \right]\,; \\
\theta_3 &= (\mathbf{z},\mathbf{x}_3) = a_3 \left[ (\mathbf{z},\mathbf{X}_3)-a_1\theta_1(\mathbf{X}_1,\mathbf{X}_3)-{} \right.  \\
 &\ \ \ \ \ {}- \left.a_2\theta_2[(\mathbf{X}_2,\mathbf{X}_3)-a_1^2(\mathbf{X}_1,\mathbf{X}_2)(\mathbf{X}_1,\mathbf{X}_2)] \right]\,.
\end{align*}
Подставляя вместо вектора $\mathbf{z}$ вектор
наблюдений~$\mathbf{y}$ и вектор винзоризованных остатков
$\mathbf{r}$, получим, соответственно, векторы коэффициентов
$\bm{\uptheta}^0$ и $\hat{\bm{\uptau}}$. Используя~(\ref{robustthetasol}),
находим значение амплитуды для частоты
$\omega$:
\begin{equation*}
%\label{dcdftampl}
A(\omega) =
\sqrt{\fr{\hat{\theta}_2^2+\hat{\theta}_3^2}{N/2}}\,.
\end{equation*}
Остается посчитать для некоторого набора час\-тот~$\omega^l$, 
$0{,}04 \leq \omega^l/(2\pi) \leq 0{,}4$~Гц 
($l$~--- индекс, а не степень) значения соотвествующих
амплитуд и найти максимумы в НЧ и ВЧ:
\begin{align*}
A_1 &=
\max_{0{,}04\leq\omega^l\leq 0{,}15}\left\{A(\omega^l) \right\}\,;\\
%&\label{freqampmax}\\[-6pt]
A_2 &= \max_{0{,}15<\omega^l\leq 0{,}4}\left\{A(\omega^l)
\right\}\,.
\end{align*}
Соотвествующие им $\omega_1$, $\omega_2$,
$\phi_1$ и $\phi_2$ будут искомыми для модели~(\ref{modelsin}).
Приводимые ниже результаты были получены для набора частот с шагом
0,001~Гц.

  \begin{figure*} %fig4
%\centerline{\includegraphics[width=150mm,height=86.21mm]{real_rhythm_robust.eps}}
\vspace*{1pt}
\begin{center}
\mbox{%
\epsfxsize=157.775mm
\epsfbox{mar-4.eps}
}
\end{center}
\vspace*{-9pt}
\Caption{Исходная ритмограмма (\textit{а}) и робастная оценка амплитуды частотных
компонент~(\textit{б})
\label{pic_real_function}}
\end{figure*}


\section{Отсев эктопических импульсов}%

Теперь, имея оценки всех параметров модели, можно приступать
непосредственно к фильтрации эктопических выбросов. Фильтрация
будет производиться во временной области на основе~(\ref{harmonicdiff})
и~(\ref{confint}). Допустим, что
$\alpha=0{,}05$, т.\,е.\ доля несинусовых импульсов в сигнале не
превышает 5\%, что при пульсе 75~ударов в минуту означает не более
20~импульсов для 5-минутных записей. В качестве значения
доверительного уровня возьмем $1-\gamma$.

Первая  точка ритмограммы должна относиться к синусовому ритму,
это необходимо для правильной работы алгоритма. Предположим, что
до момента времени $t_{i-1}$ включительно ритмограмма
отфильтрована, т.\,е.\ последний из предшествующих синусовых
интервалов соответствует моменту времени $t_{i-1}$. Рассмотрим
момент времени $t_{i+k}$, $k$~--- целое число. Начинаем со значения
$k=0$.
\begin{description}
\item Если разность
$RR(t_{i+k})-RR(t_{i-1})$ попадает в\linebreak доверительный интервал,
построенный по формуле~(\ref{confint}), то точка $RR(t_{i+k})$
соответствует\linebreak модели~(\ref{modelsin}), она объявляется синусовой,
об\-нов\-ля\-ют\-ся значения $i$ и $k$: $i=i+k+1$, $k=0$. Выполняется
переход к моменту времени $t_i$ ($i$~--- обновленное).
\item Если
же разность в интервал не попала, то точка $RR(t_{i+k})$ удаляется
из ритмограммы как эктопический импульс. Далее $k$ увеличивается
на единицу, и рассматривается момент $t_{i+k}$ ($k$~---
обновленное).
\end{description}

Необходимо  отметить, что обычно $k\leq 3$, так что существенного
увеличения доверительного интервала и, как следствие, ухудшения
работы метода ожидать не стоит.

\begin{figure*} %fig5
%\centerline{\includegraphics[width=150mm,height=86.21mm]{robust_filter.eps}}
\vspace*{1pt}
\begin{center}
\mbox{%
\epsfxsize=156.767mm
\epsfbox{mar-5.eps}
}
\end{center}
\vspace*{-9pt}
\Caption{Исходная ритмограмма~(\textit{а}) и разности и доверительные интервалы
для разностей~(\textit{б})
\label{pic_robust_filter}}
\end{figure*}


\section{Результаты}%

Выясним, как робастная регрессия оценивает параметры модельной
функции. В качестве таковой использовалась функция
\begin{multline*}
%\label{modelfunction}
y(t) = 0{,}08\sin\left ( 0{,}23\cdot2\pi t + 1{,}0\right )+{}\\
{}+ 0{,}03\sin\left ( 0{,}1\cdot2\pi t + 2{,}0 \right) +
\epsilon(t) + 0{,}8\,,
\end{multline*}
где $\epsilon(t)\sim
\mathcal{N}(0,\,0{,}01^2)$, $t = 1,2,\ldots,300$. Коэффициенты выбраны
близкими к тем, что получаются при анализе реальных данных. К этой
функции добавлено 15~паразитных импульсов. Функция и амплитуда
частотных компонент, оцененная методом робастной регрессии,
приведены на рис.~\ref{pic_modelfunction}. Частоты оценены без
погрешности, пики находятся ровно в точках 0,1~Гц и 0,23~Гц,
погрешности оценок аплитуд равны соответственно 1,6\% и 0,3\%.

Теперь рассмотрим результаты оценки спектральных характеристик
реальных данных. На рис.~\ref{pic_real_function} приведена
ритмограмма с рис.~\ref{pic_ecg_hrv} и~\ref{pic_real_rhythm_clean},
но предварительного ручного удаления
эктопических импульсов, как на рис.~\ref{pic_real_rhythm_clean},
не проводилось. Оценка спектра
методом робастной регрессии дает точную оценку низкочастотного
пика, 0,079~Гц, погрешность оценки высокочастного пика равна
1,3\% (0,31~Гц~--- исходное значение, 0,306~Гц~--- оцененное).
Погрешность оценки амплитуд пиков со\-став\-ля\-ет 4,0\% и 11,6\%
соответственно.

Фильтрация ритмограммы проводилась путем построения доверительных
интервалов для разностей с использованием полученных значений
час\-тот и амплитуд. Доверительный уровень был\linebreak выбран равным 95\%,
т.\,е.\ $\gamma=0{,}05$. Результаты представлены на
рис.~\ref{pic_robust_filter}. Пунктирные линии обозначают границы
доверительных интервалов, а звездочками помечены те точки
ритмограммы, которые были удалены в результате работы метода.

Как  видим, все эктопические импульсы успешно удалены. Кроме того,
были удалены три точки (56,58, 114,7 и~201,2~с), которые
давали разности, немного выходящие за границы доверительного
интервала. Такого эффекта можно избежать, задав меньшее значение
$\gamma$, т.\,е.\ если построить интервалы с более высоким
доверительным уровнем.

\section{Заключение}%

В  статье предложен новый метод отсева эктопических импульсов
ритмограммы, основанный на собственной математической модели
сигнала и робастном подходе к оценке параметров этой модели.
Проведена демонстрация эффективной работы метода на реальных
данных.

{\small\frenchspacing
{%\baselineskip=10.8pt
\addcontentsline{toc}{section}{Литература}
\begin{thebibliography}{9}

\bibitem{1ma}
\Au{Malik~M.}
Heart rate variability~// European Heart
J., 1996. Vol.~17. No.\,3. P.~354--381.

\bibitem{2ma}
\Au{Azuaje~F., Clifford~G., McSharry~P.}
Advanced methods
and tools for ECG data analysis.~--- Artech House Publishers, 2005.
384~p.

\bibitem{3ma}
\Au{Clifford~G., McSharry~P., Tarassenko~L.}
Characterizing artefact in the normal human 24-hour RR time series
to aid identification and artificial replication of circadian
variations in human beat to beat heart rate using a simple
threshold~// Computers in Cardiology, 2002. P.~129--132.

\bibitem{4ma}
\Au{Ивченко~Г.\,И., Медведев~Ю.\,И.}
Математическая
статистика.~--- М.: Высш. шк., 1984. 248~с.

\bibitem{5ma}
\Au{Хьюбер~П.}
Робастность в статистике.~--- М.: Мир, 1984. 304~с.

\bibitem{6ma}
\Au{Andrews~D.}
A robust method for multiple linear
regression~// Technometrics, 1974. Vol.~16. No.\,4. P.~523--531.

\bibitem{7ma}
\Au{Ferraz-Mello~S.}
Estimation of periods from unequally
spaced observations~// The As\-tro\-no\-mi\-cal J., 1981.
Vol.~86. No.\,4. P.~619--624.

\label{end\stat}

\bibitem{8ma}
\Au{Foster~G.}
The cleanest Fourier spectrum~// The
Astronomical J., 1995. Vol.~109. No.\,4. P.~1889--1902.
\end{thebibliography}

}
}
\end{multicols}