
\newcommand{\Mla}{\widetilde{\lambda}}
\newcommand{\Mmu}{\widetilde{\mu}}
\newcommand{\Op}{\mbox {\bf O}}
\newcommand{\op}{\mbox {\bf o}}
\newcommand{\COp}{\mbox {\bf cov}}
\newcommand{\Np}{\mbox {\bf N}}
\newcommand{\Pp}{\mbox {\bf P}}
\newcommand{\Dp}{\mbox {\bf D}}
\newcommand{\Vp}{\mbox {\bf V}}
\newcommand{\Ep}{\mbox {\bf E}}
\newcommand{\Rp}{\mbox {\bf R}}


\newcommand{\intp}{\int\limits_{\bf R}}
\newcommand{\intl}{\int\limits}
\newcommand{\intd}{\int\limits_{-\infty}^\infty\int\limits_{-\infty}^\infty}

\newcommand{\Bt}{\mbox B_m(\theta)}
\newcommand{\Xp}{\mbox {\bf X}}
\newcommand{\xp}{\mbox {\bf x}}
\newcommand{\sv}{\mbox {\bf s}}
\newcommand{\dxp}{\mbox {\bf dx}}

\newcommand{\Th}{\Theta}
\newcommand{\bt}{\beta}
\newcommand{\ov}{\overline}
\newcommand{\dta}{\delta}
\newcommand{\ld}{\ldots}
\newcommand{\lm}{\lambda}
\newcommand{\vp}{\varepsilon}
\newcommand{\al}{\alpha}
\newcommand{\dt}{\delta}
\newcommand{\pp}{\partial}
\newcommand{\pat}{\fr{\partial}{\partial\theta}}
\newcommand{\patt}{\fr{\partial}{\partial\theta^\top}}
\newcommand{\Dt}{\Delta}
\newcommand{\ns}{\normalsize}


\newcommand{\ba}{\begin{eqnarray}}
\newcommand{\ea}{\end{eqnarray}}
\newcommand{\ban}{\begin{eqnarray*}}
\newcommand{\ean}{\end{eqnarray*}}
\def\stat{chich}


\def\tit{СТОХАСТИЧЕСКИЕ РАЗЛОЖЕНИЯ НЕСМЕЩЕННЫХ
ОЦЕНОК В~СЛУЧАЕ ОДНОПАРАМЕТРИЧЕСКОГО ЭКСПОНЕНЦИАЛЬНОГО СЕМЕЙСТВА}
\def\titkol{Стохастические разложения несмещенных
оценок в случае однопараметрического экспоненциального семейства}
\def\autkol{В.\,В.~Чичагов}
\def\aut{В.\,В.~Чичагов$^1$}

\titel{\tit}{\aut}{\autkol}{\titkol}

%{\renewcommand{\thefootnote}{\fnsymbol{footnote}}\footnotetext[1]
%{Работа выполнена при
%поддержке РФФИ, гранты 08-01-00345, 08-01-00363,
%08-07-00152.}}

\renewcommand{\thefootnote}{\arabic{footnote}}
\footnotetext[1]{Пермский государственный университет, chvv50@mail.ru}

\Abst{Получены асимптотические и
стохастические разложения для всех несмещенных оценок, которые
могут быть построены по повторной выборке, элементами которой
являются независимые случайные величины, имеющие одно и то же
распределение из однопараметрического экспоненциального семейства.
Проведено сравнение разложений несмещенных оценок и оценок
максимального правдоподобия. Найдено стохастическое разложение
несмещенной оценки дисперсии несмещенной оценки.}

\KW{несмещенная оценка; экспоненциальное
семейство; стохастическое разложение; несмещенная оценка
дисперсии}

      \vskip 24pt plus 9pt minus 6pt

      \thispagestyle{headings}

      \begin{multicols}{2}

      \label{st\stat}

\section{Введение}

  Точечные несмещенные оценки играют важную роль в современных
  научно-технических исследованиях. Техника получения несмещенных
  оценок хорошо разработана, решено значительное число
  теоретических и прикладных задач (см., например,~[1, 2]).
  Продолжают появляться новые работы  такой же направленности
  (например,~[3--7]). Однако известно значительно меньше работ, в которых
  бы изучалось предельное поведение несмещенных оценок при большом
  объеме выборки и осуществлялось сравнение несмещенных оценок с
  оценками другого вида, в первую очередь с оценками максимального правдоподобия.
  Возможной причиной сложившейся ситуации можно считать
  появление работ~[8, 9], в которых при весьма общих предположениях в
  случае экспоненциального семейства распределений была
  установлена асимптотическая эквивалентность несмещенных оценок и
  оценок максимального правдоподобия с точностью
  до слагаемого ${\bf O}_P\left( n^{-1}\right)$, где $n$~--- объем
  повторной выборки.
   В появившейся сравнительно недавно работе~[10] получено более
  общее решение этой проблемы с использованием счетной системы
  ассоциированных с экспоненциальным семейством ортогональных
  полиномов, которые были предложены в работе~[11].
   Другой подход, непосредственно ориентированный на изучение предельного
  поведения несмещенных оценок, предложен в работах автора~[12, 13].
  Он основывается на применении локальных предельных теорем
  и разложения Эджворта~[14--17] к исследованию плотности
  условного распределения, определяющей вид несмещенной оценки
  плотности~[18--22].
   В данной работе эта идея получила свое логическое завершение
  в случае однопараметрического экспоненциального семейства,
  которому принадлежит распределение наблюдаемой случайной величины.\linebreak
   Применение этого подхода, в отличие от упомянутых выше методов,
  позволяет получать для несмещенных оценок
  асимптотические разложения высокого порядка.
   Полученные в работах~[23--25] результаты, определяющие асимптотику условного
  распределения, хотя и близки
  в идейном плане, но оказались непригодными для решения поставленной
  проблемы, поскольку получены для случайных векторов, обе компоненты
  которых являются суммами случайных величин с большим числом слагаемых.
  В~рассматриваемом случае первая компонента состоит из конечного
  числа слагаемых.
    Еще одна задача, которая решается в
  данной работе, состоит в исследовании предельного поведения
  несмещенной оценки дисперсии несмещенной оценки.
   Проблемы, связанные с нахождением несмещенных оценок дисперсии и их
   применением, рас\-смат\-ри\-ва\-лись, например, в работах~[26--28].

  \section{Модель наблюдений и~основные предположения}

   Имеется $X_1,\ldots,X_n$~--- независимая повторная выборка,
  элементы которой имеют то же распределение, что и наблюдаемая
  случайная величина $\xi$, удовлетворяющая следующим предположениям.
{\addtolength{\leftmargini}{10pt}
  \begin{itemize}
  \item[$(A_1).$]
 Распределение вероятностей случайной величины $\xi$ принадлежит
 однопараметрическому экспоненциальному  семейству, определяемому
 выражением
\begin{multline}
f(x;\theta)=h(x) \exp\{\theta~T(x)+{}\\
{}+v(\theta)\}\,,
\quad x\in G\subset \Rp\,.
\label{e1c}
 \end{multline}
  Здесь $f(x;\theta)$~--- плотность распределения случайной величины $\xi$
 относительно меры $\mu(x)$, являющейся либо мерой Лебега, если $\xi$
 имеет абсолютно непрерывное распределение,\linebreak
 либо считающей мерой,
 когда $\xi$ имеет решетчатое распределение;
 $G$~--- носитель распределения, $\theta\in \Theta\subset\Rp$~--- неизвестный
 канонический параметр распределения,
 $h(x)$, $T(x)$, $v(\theta)$~--- известные борелевские функции.
  \item[\sffamily $(A_2).$]
 Если $\mu(x)$~--- мера Лебега, то для каждого $\theta\in \Theta$ существует
 $n_0\in {\bf N}$ такое, что случайная величина
  $$
Z_n=\fr{S_n-n a}{b\sqrt{n}}\,,
$$
где
\begin{align*}
S_n&=\sum^n_{i=1}T(X_i)\,;\\
a&=\Ep\left[ T(\xi)\right]\,;\\
b^2&=\Vp\left[ T(\xi)\right]\,,
\end{align*}
 имеет непрерывную ограниченную плотность $f_{Z_n}(x;\theta)$
 для $n\ge n_0$. Если $\mu(x)$~--- считающая мера, то носитель $G$
 не содержится ни в какой подрешетке решетки $\bf Z$.
  \item[\sffamily $(A_3).$]
 Носитель распределения $G$ не зависит от параметра $\theta$.
  \item[\sffamily $(A_4).$]
 Параметрическая область $\Theta$ содержит некоторый интервал,
 принадлежащий $\bf R$.
\end{itemize}
}

\section{Интегральное представление несмещенных оценок
и~несмещенно оцениваемых параметрических функций}

  Основной целью данной работы является исследование
  асимптотического поведения несмещенной оценки $h_n(S_n)$ заданной
  параметрической функции $g(\theta)$. Оценка  строится по
  независимой повторной выборке $X_1,\ldots,X_n$ объема $n$.

   Условие несмещенности для этой оценки может быть записано
  в виде уравнения
\begin{equation}
g(\theta)=\intp h_n(t) f_n(t;\theta) \mu(dt)\,,\quad n\ge l\,,
   \label{e2c}
\end{equation}
  где $l$~--- минимальный объем выборки, для которого выполняется~(\ref{e2c})
  и который является конечным числом, не зависящим от $n$;
  $f_n(t;\theta)$~--- плотность распределения статистики $S_n$, которая
  является достаточной для параметра $\theta$ в силу условий $(A_1)$ 
  и~$(A_3)$.
  Это означает, что любую параметрическую функцию
  $g(\theta)$, для которой существует несмещенная оценка, являющаяся
  функцией достаточной статистики $S_n$, можно представить в интегральной форме
  \begin{equation} 
g(\theta)=\intp h_l(t) f_l(t;\theta) \mu(dt).
      \label{e3c}
  \end{equation}

   Но в таком случае имеет место аналогичное представление и для
  несмещенной оценки параметрической функции~(\ref{e3c}):
\begin{equation} 
h_n(S_n)=\intp h_l(t) \widehat{f_l}(t|S_n) \mu(dt)\,,
\label{e4c}
\end{equation}
  где $\widehat{f_l}(t|S_n)$~--- несмещенная оценка плотности
  $f_l(t;\theta)$. Хорошо известно~[1, 18, 20], что несмещенная оценка
  плотности определяется выраже\-нием
\begin{equation} 
\widehat{f_l}(t|S_n)=\fr{f_l(t;\theta_0)
    f_{n-l}(S_n-t;\theta_0)} {f_n(S_n;\theta_0)}\,,
\label{e5c}
\end{equation}
 где $\theta_0$~--- произвольное значение $\theta$ из области
  $\Theta$ (будем полагать $\theta_0=\theta$).
  В силу условия полноты  $(A_4)$ экспоненциального семейства~(\ref{e1c})
  представление~(\ref{e4c}) единственно. Из 
формул~(\ref{e3c})--(\ref{e5c}) следует, что
  ошибка несмещенной оценки $h_n(S_n)$ определяется интегральным
  выражением
\begin{multline}  
h_n(S_n)-g(\theta)={}\\
{}=\intp h_l(t) f_l(t;\theta)
      \left[ \fr{f_{n-l}(S_n-t;\theta)} {f_n(S_n;\theta)}-
      1\right] \mu(dt)\,,
\label{e6c}
\end{multline}
 поэтому предельное при $n\rightarrow\infty$ поведение несмещенной оценки
  $h_n(S_n)$ определяется отношением
\begin{equation}  
U_1(S_n,t;\theta)=\fr{f_{n-l}(S_n-t;\theta)}{f_n(S_n;\theta)}\,.
\label{e7c}
\end{equation}
  Известно~[26], что $\widehat{\bf {V}}[h_n(S_n)]$~--- несмещенная оценка
  дисперсии несмещенной оценки ${\bf V}[h_n(S_n)]$ может быть найдена
  по формуле
  \begin{equation}
 \widehat{\bf {V}}[h_n(S_n)]=h_n^2(S_n)-\widehat{g^2}(\theta|S_n)\,,
\label{e8c}
\end{equation}
 где $\widehat{g^2}(\theta|S_n)$~--- несмещенная оценка параметрической функции
  $g^2(\theta)$. Воспользовавшись~(\ref{e8c}), нетрудно показать, что несмещенная оценка
  $\widehat{\bf {V}}[h_n(S_n)]$ может быть записана в интегральной форме:
  \begin{multline}
 \widehat{\bf {V}}[h_n(S_n)]=
     \intd \left[ \widehat{f_l}(t_1|S_n) \widehat{f_l}(t_2|S_n)-{}\right.\\
     {}-\left.
      \widehat{f_{l,l}}(t_1,t_2|S_n) \right]
      \prod_{j=1}^2 h_l(t_j) \mu(dt_j)={}\\
{}=
  \intd \left[ \fr{f_{n-l}(S_n-t_1;\theta)}{f_n(S_n;\theta)}
       \, \fr{f_{n-l}(S_n-t_2;\theta)}{f_n(S_n;\theta)}-\right.{}\\
       {}-\left.
       \fr{f_{n-2l}(S_n-t_1-t_2;\theta)}{f_n(S_n;\theta)}\right]\times{}\\
{}\times
   \prod_{j=1}^2 h_l(t_j) f_l(t_j;\theta) \mu(dt_j)\,, 
\label{e9c}
\end{multline}
  где $\widehat{f_{l,l}}(t_1,t_2|S_n)$~--- несмещенная оценка плотности $\prod_{j=1}^2 f_l(t_j;\theta)$:
  \begin{multline*}   
\widehat{f_{l,l}}(t_1,t_2|S_n)={}\\
{}=
       \fr{f_l(t_1;\theta) f_l(t_2;\theta)
       f_{n-2l}(S_n-t_1-t_2;\theta)} {f_n(S_n;\theta)}\,.
  \end{multline*}
  Поэтому предельное при $n\rightarrow\infty$ поведение несмещенной оценки
  $\widehat{\bf {V}}[h_n(S_n)]$ определяется выраже\-нием
  \begin{multline}
 U_2(S_n,t_1,t_2;\theta)={}\\
 {}=
       \fr{f_{n-l}(S_n-t_1;\theta)}{f_n(S_n;\theta)}
       \, \fr{f_{n-l}(S_n-t_2;\theta)}{f_n(S_n;\theta)}- {}\\
{}-
       \fr{f_{n-2l}(S_n-t_1-t_2;\theta)}{f_n(S_n;\theta)}\,.
      \label{e10c}
\end{multline}

\section{Стохастическое разложение несмещенной оценки плотности
  и~интегрального функционала}

  Сначала изучается асимптотическое поведение компоненты~(\ref{e7c}),
  определяющей ошибку несмещенной оценки~(\ref{e6c}). Затем
  определяется стохастическое разложение несмещенной оценки.
  В заключительной части раздела проводится сравнение
  стохастических разложений несмещенной оценки и оценки
  максимального правдоподобия.

\medskip

\noindent
  {\bf Теорема 1.} {\it Пусть выполнены предположения
  $(A_1)$--$(A_4)$, $\Delta_l(t)=(t-la)/b$.
  Тогда при $s=na\;+$\linebreak $+\;z b\sqrt{n}$, $n\rightarrow\infty$,
  равномерно по $z$ при $|z|\le z_0$ для любого фиксированного $z_0$
  справедливо разложение
  \begin{multline} 
\fr{f_{n-l}(s-t;\theta)}{f_n(s;\theta)}=
    1+\fr{z \Delta_l(t)}{n^{1/2}}-\fr{H_2(z)}{2n}
    \left[ \rho_3 \Delta_l(t)+{}\right.\\
    {}+\left. l-\Delta_l^2(t)\right]+{}\\
{}+\fr{1}{n^{3/2}} \sum_{j=0}^3 c_j(z) \Delta_l^j(t)+
    P_4\left(\Delta_l(t)\right)\cdot {\bf O}\left(n^{-2}\right)\,,\\
    \mbox{если}~~\Delta_l(t)={\bf o}\left(n^{1/2}\right)\,,
    \label{e11c}
\end{multline}
\begin{equation}
  \fr{f_{n-l}(s-t;\theta)}{f_n(s;\theta)}=
      {\bf O}\left(1\right)~~\mbox{при любом}~~t\in {\bf R}\,,
    \label{e12c}
\end{equation}
  где
\begin{align*}  
c_0(z)&=\fr{l\rho_3}{6}\left[2H_3(z)+3z\right]\,;\\
c_1(z)&=H_3(z)\left[\fr{\rho_3^2-l}{2}- \fr{\rho_4}{6}\right]+
             \fr{\rho_3^2 z}{2}\,;\\
 c_2(z)&=-\fr{\rho_3}{2}\left[ z+H_3(z)\right]\,;\\
 c_3(z)&=\fr{H_3(z)}{6}\,;
\end{align*}
  $H_2(z)=z^2-1;$ $H_3(z)=z^3-3z;$\
  $\rho_3$ и $\rho_4$~--- нормированные кумулянты (соответственно коэффициенты
  асимметрии и эксцесса) распределения случайной величины $\xi$;
  $P_4(x)$~--- некоторый полином степени~4.}

\smallskip
\noindent
Д\,о\,к\,а\,з\,а\,т\,е\,л\,ь\,с\,т\,в\,о. 
Сначала рассмотрим случай абсолютно
непрерывного распределения $\xi$. Воспользуемся тождеством
  \begin{equation} 
\fr{f_{n-l}(s-t;\theta)}{f_n(s;\theta)}=\sqrt{\fr{n}{n-l}}
     \cdot \fr{f_{Z_{n-l}}(u;\theta)}{f_{Z_n}(z;\theta)}\,,
\label{e13c}
\end{equation}
где 
$$
u=\sqrt{\fr{n}{n-l}} \left(z-\fr{\Delta_l(t)}{\sqrt{n}} \right)\,,
$$
а также асимптотическим разложением плотности
 $f_{Z_n}(z;\theta)$ с использованием приближения Эджворта,
 которое в случае принадлежности $f(x;\theta)$ экспоненциальному
  семейству (см., например, (А.3) в~[16]) имеет вид
  \begin{multline*}    
f_{Z_n}(z;\theta)=\phi(z)\left[1+
         \sum_{j=1}^3 \fr{q_{j}(z)}{n^{j/2}} \right]+
         {\bf O}\left(n^{-2}\right)\equiv {}\\
         {}\equiv w_{3,n}(z)+
         {\bf O}\left(n^{-2}\right)\,,
  \end{multline*}
  где
  \begin{align*}
  \phi(z)&=\fr{1}{\sqrt{2\pi}}\exp\left(-\fr{z^2}{2}\right)\,,\\
  q_{1}(z)&=\fr{\rho_3H_3(z)}{6}\,,\\
  q_{2}(z)&=\fr{\rho_4H_4(z)}{24}+\fr{\rho_3^2H_6(z)}{72}\,,
\end{align*}
  $q_{3}(z)$~--- некоторый полином степени~9, $\{H_j(z)\}$~--- полиномы
  Эрмита.

  Учитывая, что
  \begin{align*}  
\sqrt{\fr{n}{n-l}}&=1+\fr{l}{2n}+
      {\bf O}\left(n^{-2}\right)\,,\\
\fr{1}{\sqrt{n-l}}&=\fr{1}{\sqrt{n}}+\fr{l}{2n^{3/2}}+
       {\bf O}\left(n^{-2}\right)\,,\\
  u-z&=\fr{l z}{2n}-\left(1+\fr{l}{2n} \right)
      \fr{\Delta_l(t)}{\sqrt{n}}+ {}\\
&\ \ \ \ \ \ \ \ \ \ \ \ \ \ \ \ \ \       {}+{\bf O}\left(n^{-2}\right)\left [1+\Delta_l(t)\right]\,,
\end{align*}
  и используя свойство полиномов Эрмита
  $H_r(z) \phi(z)=(-1)^r \phi^{(r)}(z)$,
  осуществим разложение~(\ref{e7c}) при $S_n=s$ и
  $\Delta_l(t)={\bf o}\left(n^{1/2}\right)$ следующим образом:
\begin{multline}  
\sqrt{\fr{n}{n-l}}\cdot
      \fr{f_{Z_{n-l}}(u;\theta)}{f_{Z_n}(z;\theta)}=
     \sqrt{\fr{n}{n-l}}\times{}\\
     {}\times
      \fr{w_{3,n}(u)+
      \phi(u)\, l\rho_3/(12n^{3/2}) H_3(u)+
      {\bf O}\left(n^{-2}\right)}
      {w_{3,n}(z)+{\bf O}\left(n^{-2}\right)}={}\\
{}    =\left(1+\fr{l}{2n} \right) \left[1+
         \fr{1}{w_{3,n}(z)}\sum_{j=1}^3 w_{3,n}^{(j)}(z)
         \fr{(u-z)^j}{j!} +{}\right.\\
         {}+\left.
         \fr{l\rho_3 H_3(z)}{12n^{3/2}}+
         \fr{P_4\left(\Delta_l(t) \right)}{n^2}
         \vphantom{\sum_{j=1}^3 \fr{(u-z)^j}{j!}}\right ] +
         {\bf O}\left(n^{-2}\right)\,.
      \label{e14c}
  \end{multline}

  Путем элементарных преобразований получим выражения
  \begin{align*}   
(u-z)^2&=\fr{\Delta_l^2(t)}{n}-\fr{lz\Delta_l(t)}{n^{3/2}}+
       {\bf O}\left(n^{-2}\right)\left [1+\Delta_l^2(t)\right ]\,;\\
       (u-z)^3&=-\fr{\Delta_l^3(t)}{n^{3/2}}+
       {\bf O}\left(n^{-2}\right)\left [\Delta_l^3(t)\right ]\,;\\
  \fr{w_{3,n}^{(1)}(z)}{w_{3,n}(z)}&=\sum_{j=0}^2
       \fr{c_{1j}(z)}{n^{j/2}}+{\bf O}\left(n^{-3/2}\right)\,;\\
 \fr{w_{3,n}^{(2)}(z)}{w_{3,n}(z)}&=\sum_{j=0}^1
       \fr{c_{2j}(z)}{n^{j/2}}+{\bf O}\left(n^{-1}\right)\,;\\
      \fr{w_{3,n}^{(3)}(z)}{w_{3,n}(z)}&=H_3(z)+
       {\bf O}\left(n^{-1/2}\right)\,,
\end{align*}
  где
\begin{align*}   
c_{10}(z)&=z\,;\quad c_{11}(z)=-\fr{\rho_3H_2(z)}{2}\,;\\
c_{12}(z)&=-\fr{\rho_4H_3(z)}{6}+ \fr{\rho_3^2}{2}\left[z+ H_3(z)\right]\,;\\
c_{20}(z)&=H_2(z)\,;\quad c_{21}(z)=-\rho_3\left[z+ H_3(z)\right]\,.
\end{align*}
  Подставляя найденные выражения в~(\ref{e14c}), получим~(\ref{e11c}).

  В соответствии с локальной предельной теоремой, примененной к
  последовательности случайных величин $\{Z_{n-l}\}$,
  справедливость~(\ref{e12c}) следует из равномерной ограниченности
  плотности $f_{Z_{n-l}}(u;\theta)$ при достаточно больших $n$.

  Случай решетчатого распределения $\xi$ рас\-смат\-ри\-ва\-ет\-ся
  аналогично случаю абсолютно непрерывного распределения. Основное
  отличие этого случая от предыдущего состоит в другой форме
  записи левой части~(\ref{e13c}):
  $$ 
\fr{f_{n-l}(s-t;\theta)}{f_n(s;\theta)}=\sqrt{\fr{n}{n-l}}
     \cdot \fr{b\sqrt{n-l}\cdot f_{n-l}(s-t;\theta)}
     {b\sqrt{n}\cdot f_{n}(s;\theta)}\,,
  $$
  которая позволяет воспользоваться приближением Эджворта для
  решетчатого распределения (см., например, теорему~6.10 из~[17]).

  Теорема~1 доказана.

  Используя асимптотическое разложение~(\ref{e11c}), получим аналогичное
  разложение для значения ошибки несмещенной оценки~(\ref{e6c}).

  \medskip

\noindent
{\bf Теорема 2.} {\it Пусть выполнены предположения
  $(A_1)$--$(A_4)$. Пусть также справедливо условие:
  {\addtolength{\leftmargini}{10pt}
  \begin{itemize}
  \item[\sffamily $(A_5).$]
  Существует $0<\nu<0{,}5$ такое, что для $j=\overline{0,\,3}$
  при $n\rightarrow\infty$
  \begin{gather} 
\int\limits_{B_{n,\nu}}  h_l(t)\, \Delta_l^j(t)\,
     f_l(t;\theta)\, \mu(dt)={\bf O}\left(n^{-2}\right)\,,\notag\\
B_{n,\nu}=\left\{t\in{\bf R}:~~|\Delta_l(t)|>n^{\nu}\right\}\,;\notag\\
 D_j=\int\limits_{-\infty}^\infty
     h_l(t)\, \Delta_l^j(t)\, f_l(t;\theta)\,
\mu(dt)<\infty\,,\notag\\
j=\overline{0,\,4}\,.\label{e15c}
\end{gather}
   \end{itemize}}
   Тогда при $s=na+z\, b\sqrt{n}$, $n\rightarrow\infty$,
  равномерно по $z$ при $|z|\le z_0$ для любого фиксированного $z_0$
  справедливо разложение
  \begin{multline*} 
h_n(s)-g(\theta)=\fr{zD_1}{\sqrt{n}}-
     \fr{H_2(z)}{2n}\left[lD_0-D_2+\rho_3D_1\right]+{}\\
     {}+
     \fr{1}{n^{3/2}} \sum_{j=0}^3 c_j(z)D_j+
     {\bf O}\left(n^{-2}\right)\,.
  \end{multline*}
  }
\pagebreak

%  \medskip
  
\noindent
Д\,о\,к\,а\,з\,а\,т\,е\,л\,ь\,с\,т\,в\,о. 
Положим   
\begin{multline*}   
P_3(x)={}\\[2pt]
{}= \fr{zx}{\sqrt{n}}- \fr{H_2(z)}{2n} \left[\rho_3 x+l-x^2 \right]+
       \fr{1}{n^{3/2}}\sum_{j=0}^3 c_j(z)x^3\,.
\end{multline*}
 Используя теорему~1 и условие $(A_5),$ а также формулы~(\ref{e6c}),
  (\ref{e11c}) и~(\ref{e15c}), получим при
  $S_n=s,~n\rightarrow\infty$, соотношение
  \begin{multline*}   
\!\!h_n(s)-g(\theta)-\left\{\vphantom{\sum_{j=0}^3}
 \fr{zD_1}{\sqrt{n}}-
     \fr{H_2(z)}{2n}\left[lD_0-D_2+\rho_3D_1\right]+{}\right.\\
     {}+\left.
     \fr{1}{n^{3/2}} \sum_{j=0}^3 c_j(z)D_j\right\}={}\\
{}=\int\limits_{\overline{B}_{n,\nu}}
      h_l(t)\, P_4(\Delta_l(t)) f_l(t;\theta)\,
     \mu(dt)\cdot {\bf O}\left(n^{-2}\right)+{}\\
{}     \! +\!\int\limits_{B_{n,\nu}}
      h_l(t) \left[{\bf O}\left(1\right)-P_3(\Delta_l(t)) \right]
      f_l(t;\theta)\,\mu(dt)={}\\
      {}= {\bf O}\left(n^{-2}\right)\,,
\end{multline*}
  которое справедливо равномерно относительно $z$ при $|z|\le z_0$.

  Теорема~2 доказана.
\medskip

            Теперь сформулируем утверждение, которое позволяет обосновывать
  справедливость стохастического разложения, получаемого из
  асимптотического разложения методом подстановки.

\medskip


\noindent
{\bf Лемма 1.} {\it Пусть выполнены условия $(A_1)$--$(A_2),$
      а функция $G(z;n)$ такова, что при
      $n\rightarrow\infty$ равномерно относительно $z$ при $|z|\le z_0$
      для любого фиксированного $z_0$ справедливо асимптотическое разложение
  $$  
G(z;n)=\sum_{j=0}^r \fr{G_j(z)}{n^{j/2}}+
      {\bf O}\left(n^{-(r+1)/2}\right)\,,
  $$
 где $\left\{G_j(z)\,,\ j=\overline{0,r} \right\} $~--- некоторые
      полиномы относительно $z$. Тогда при
      $n\rightarrow\infty$ верно стохастическое разложение
  $$  
G(Z_n;n)=\sum_{j=0}^r \fr{G_j(Z_n)}{n^{j/2}}+
      {\bf O}_P\left(n^{-(r+1)/2}\right)\,,
  $$
где запись $U_n={\bf O}_P\left(n^{-(r+1)/2}\right)$ означает,
что выражение $n^{(r+1)/2}U_n$ ограничено по вероятности.
}

\medskip
Справедливость леммы~1 следует из центральной предельной
   теоремы для независимых одинаково распределенных случайных
   величин и неравенства
   %   \noindent
  \begin{multline*}
  {\bf P}\left(
       \left| G(Z_n;n)-\sum_{j=0}^r \fr{G_j(Z_n)}{n^{j/2}}\right|
       <\fr{k_\varepsilon}{n^{(r+1)/2}} \right)\ge
  {}\\
{}   \ge {\bf P} \left(\left|Z_n\right|\le z_0;~
        \left|{\bf O}\left(n^{-(r+1)/2}\right) \right|<
        \fr{k_\varepsilon}{n^{(r+1)/2}} \right)={}\\
        {}=
        {\bf P} \left(\left|Z_n\right|\le z_0\right)\,,
\end{multline*}
  которое верно при достаточно большом значении~$k_\varepsilon$.

  Из леммы~1 и теоремы~2 немедленно следует справедливость
  следующего утверждения.

\medskip

\noindent
{\bf Следствие 1.} {\it В условиях теоремы~2 для ошибки
  несмещенной оценки функции $g(\theta)$ верно стохастическое разложение
\begin{multline*} 
h_n(S_n)-g(\theta)={}\\
{}=\fr{Z_nD_1}{\sqrt{n}}-
     \fr{H_2(Z_n)}{2n}\left[lD_0-D_2+\rho_3D_1\right]+{}\\
{}+
     \fr{1}{n^{3/2}} \sum_{j=0}^3 c_j(Z_n)D_j+
     {\bf O}_P\left(n^{-2}\right)\,.
\end{multline*}
   }

   \noindent
   Вычисление интегралов $\left\{D_j,~j=\overline{0,\,3} \right\}$
   (см.~(\ref{e15c})) можно упростить с помощью следующего утверждения.

\medskip

\noindent
{\bf Лемма 2.} {\it Пусть выполнены предположения
  $(A_1)$--$(A_4)$. Предположим также, что справедливо условие:
{\addtolength{\leftmargini}{10pt}
  \begin{itemize}
  \item [\sffamily $(A_6).$] Производная $v'''(\theta)$ существует и непрерывна.
  \end{itemize}}
   Тогда интегралы $\left\{D_j\,,~j=\overline{0,\,3} \right\}$ и функция
  $g(\theta)$ связаны соотношениями
\begin{align} 
D_0&=g(\theta)\,,\notag\\
D_1&=\fr{g'(\theta)}{b}\,,\notag\\[-9pt]
&\label{e16c}\\[-9pt]
D_2&=l g(\theta)+\fr{g''(\theta)}{b^2}\,,\notag\\
D_3&=\fr{g'''(\theta)}{b^3}+l\rho_3 g(\theta)+
     3l\fr{g'(\theta)}{b}\,.\notag
\end{align}
  При этом $i(\theta)$~--- информация Фишера, содержащаяся в одном
  наблюдении $X_1$: 
  $$
  i(\theta)=b^2\,.
  $$
   }

%\smallskip

\noindent
Д\,о\,к\,а\,з\,а\,т\,е\,л\,ь\,с\,т\,в\,о. 
Известно (см., например, утверждение
  задачи~14 гл.~2~[29]), что
  \begin{align}
 a&={\bf E}\left[T(\xi) \right]=-v'(\theta)\,,\notag\\[-6pt]
 & \label{e17c}\\[-6pt]
      b^2&={\bf V}\left[T(\xi) \right]=-v''(\theta)\,.\notag
\end{align}
  Формулы~(\ref{e16c}) получим, последовательно вычисляя производные
  $g'(\theta)$, $g''(\theta)$ и $g'''(\theta)$ с помощью формул~(\ref{e15c}) 
и~(\ref{e17c}), исходя из сотношения
  $$ 
D_0=g(\theta)=\int\limits_{-\infty}^\infty
         h_l(t) f_l(t;\theta) \mu(dt)\,.
  $$

  Лемма~2 доказана.
\smallskip

  Используя лемму~2, запишем стохастическое разложение ошибки несмещенной
  оценки, приведенное в следствии~1, в другой форме,
  используя только изначально известные функции
  $g(\theta)$ и~$v(\theta)$.
  {\looseness=1
  
  }

\medskip

\noindent
{\bf Следствие 2.} {\it Пусть выполнены предположения
    $(A_1)$--$(A_6).$ Тогда при $ n\rightarrow\infty$ справедливо стохастическое
    разложение ошибки несмещенной оценки}
   \begin{multline*}
     h_n(S_n)-g(\theta)={}\\[3pt]
     {}=\fr{Z_n g'(\theta)}{b\sqrt{n}}-
     \fr{H_2(Z_n)}{2n}\left[\fr{\rho_3 g'(\theta)}{b}-
     \fr{g''(\theta)}{b^2}\right]+{}\\[3pt]
     {}+
     \fr{1}{n^{3/2}}
     \left\{ \fr{H_3(Z_n)g'''(\theta)}{6b^3}+{} \right.{}\\[3pt]
            \left.
           {}+\fr{g'(\theta)}{b}
           \left[\fr{\rho_3^2\left(H_3(Z_n)+Z_n \right)}{2}-
                       \fr{\rho_4H_3(Z_n)}{6}
                 \right] -{}\right.\\
\left.{}-           \fr{\rho_3\left(H_3(Z_n)+Z_n \right)g''(\theta)}{2b^2}
     \right\}+ {\bf O}_P\left(n^{-2}\right)\,.
\end{multline*}

  Теперь приведем стохастическое разложение оценки максимального правдоподобия
  $g(\widetilde{\theta})$ функции $g(\theta)$, которое
  может быть получено с помощью формулы Тейлора, лемм~1 и~2 путем
  элементарных преобразований.
\medskip

\noindent
{\bf Лемма 3.} {\it  Пусть выполнены предположения
    $(A_1)$--$(A_4)$ и $(A_6)$. Тогда при $ n\rightarrow\infty$
    справедливо стохастическое разложение ошибки оценки
    максимального правдоподобия
\begin{multline*}
    g(\widetilde{\theta})-g(\theta)=\fr{Z_ng'(\theta)}{b\sqrt{n}}-
     \fr{Z_n^2}{2n} \left[\fr{\rho_3g'(\theta)}{b}-
     \fr{g''(\theta)}{b^2} \right]+{}\\[3pt]
{}+\fr{Z_n^3}{6n^{3/2}}\left[
    \fr{\left(3\rho_3^2-\rho_4 \right)g'(\theta)}{b}-
    \fr{3\rho_3 g''(\theta)}{b^2}+\fr{g'''(\theta)}{b^3}
    \right]+{}\\[3pt]
    {}+{\bf O}_P\left(n^{-2}\right)\,,
\end{multline*}
    где $\widetilde{\theta}=S_n/n.$
   }

  Следующие два утверждения позволяют сравнить между собой
  несмещенную оценку и оценку максимального правдоподобия
  функции.

\medskip

\noindent
{\bf Следствие 3.} {\it Пусть выполнены предположения
    $(A_1)$--$(A_6)$. Тогда при $ n\rightarrow\infty$  для разности
    несмещенной оценки и оценки максимального правдоподобия верно
    стохастическое разложение}
  \begin{multline*}
  h_n(S_n)-g(\widetilde{\theta})=\fr{1}{2n}
      \left[\fr{\rho_3g'(\theta)}{b}-\fr{g''(\theta)}{b^2} \right]+{}
\\[2pt]
{}+\fr{Z_n}{n^{3/2}}\left\{ \left[\fr{\rho_4}{2}-\rho_3^2\right]
      \fr{g'(\theta)}{b}+\fr{\rho_3 g''(\theta)}{b^2}-
      \fr{g'''(\theta)}{2b^3} \right\}+{}\\
      {}+
      {\bf O}_P\left(n^{-2}\right)\,.
\end{multline*}

\medskip

\noindent
{\bf Следствие 4.} {\it Пусть выполнены предположения
    $(A_1)\mbox{--}(A_6)$. Тогда при $ n\rightarrow\infty$ }
   \begin{multline*} 
n\left[h_n(S_n)-g(\widetilde{\theta}) \right]
   \stackrel {P}{\longrightarrow} \fr{1}{2}
    \left[\fr{\rho_3g'(\theta)}{b}-\fr{g''(\theta)}{b^2}
    \right]+{}\\
    {}+{\bf O}_P\left(n^{-1/2}\right)\,.
   \end{multline*}

\smallskip

Утверждения следствий~3 и~4 уточняют результаты работы~[9],
   в которой была установлена асимптотическая эквивалентность
   несмещенной оценки и оценки максимального правдоподобия с точностью
   до слагаемого ${\bf O}_P\left( n^{-1}\right)$.

  \section{Стохастическое разложение несмещенной оценки дисперсии}

   В данном разделе получено стохастическое разложение несмещенной
   оценки дисперсии не\-смещенной оценки $\widehat{\bf {V}}[h_n(S_n)],$
   определяемой фор\-мулой~(\ref{e9c}). Оказалось, что главный член этого
   разложения совпадает с нижней границей неравенства Рао--Крамера
   для дисперсии несмещенной оценки ${\bf {V}}[h_n(S_n)]$. Еще
   одно стохастическое разложение, приводимое здесь же,
   определяет предельное поведение
   ошибки несмещенной оценки~(\ref{e6c}), нормированной с помощью
   несмещенной оценки дис\-персии. 

\medskip

\noindent
{\bf Теорема 3.} {\it  Пусть выполнены предположения
  $(A_1)\mbox{--}(A_6)$. Тогда при 
$s=na+z b\sqrt{n},~n\rightarrow\infty$,
равномерно по $z$ при $|z|\le z_0$ для любого фиксированного $z_0$
  справедливо асимптотическое разложение
  \begin{multline*} 
\widehat{\bf {V}}[h_n(s)]=\fr{D_1^2}{n}
     \left\{1+\fr{z}{\sqrt{n}}
     \left[ \fr{2\left(D_2-lD_0 \right)}{D_1}-\rho_3\right]
      \right\}+{}\\[3pt]
      {}+{\bf O}\left(n^{-2}\right)\,.
  \end{multline*}
   }

\smallskip


\noindent
Д\,о\,к\,а\,з\,а\,т\,е\,л\,ь\,с\,т\,в\,о. Учитывая, что
  $ \Delta_{2l}(t_1+t_2)=$\linebreak $=\;\Delta_l(t_1)+\Delta_l(t_2),$
  и используя~(\ref{e11c}), получим сначала
  асимптотическое разложение для~(\ref{e10c}) при $S_n=s$

  \pagebreak
  
\noindent
   \begin{multline*} 
U_2(s,t_1,t_2;\theta)=\prod_{i=1}^2 \Biggl\{
     1+\fr{z \Delta_l(t_i)}{n^{1/2}}-{}\\
     {}-\fr{H_2(z)}{2n}
    \left[ \rho_3
    \Delta_l(t_i)+l-\Delta_l^2(t_i)\right]+
{}\\
   {}  +\fr{1}{n^{3/2}} \sum_{j=0}^3 c_j(z) \Delta_l^j(t_i)+
      P_4\left(\Delta_l(t_i)\right)\cdot {\bf O}\left(n^{-2}\right)
    \Biggr\}-{}\\
    {}-
     \Biggl\{ 1+
    \fr{z\left[\Delta_l(t_1)+\Delta_l(t_2) \right]}{n^{1/2}}-
{}\\
    {} -\fr{H_2(z)}{2n}
       \left[ \vphantom{\fr{H_2(z)}{2n}}
        \rho_3 \left(
       \Delta_l(t_1)+\Delta_l(t_2) \right)+2l-{}\right.\\
\left.       {}-
       \left(\Delta_l(t_1)+\Delta_l(t_2) \right)^2 
       \vphantom{\fr{H_2(z)}{2n}}
       \right] \vphantom{\fr{H_2(z)}{2n}}+
    {}\\
    {} +\fr{1}{n^{3/2}} \sum_{j=0}^3 c_j^*(z)
       \left[\Delta_l(t_1)+\Delta_l(t_2) \right]^j+
       P_4\left(\Delta_l(t_1)+ {}\right.\\
\left.       {}+
\Delta_l(t_2)\right)\cdot
       {\bf O}\left(n^{-2}\right) \Biggr\}=
    {}\\
{} =U_2^*(s,t_1,t_2;\theta)+{\bf O}\left(n^{-2}\right)
       P_4^* \left(\Delta_l(t_1),\Delta_l(t_2)\right),
    \end{multline*}
    где
\begin{multline*}  
U_2^*(z,t_1,t_2;\theta)=\fr{\Delta_l(t_1)\Delta_l(t_2)}{n} +{}\\
{}+
        \fr{1}{n^{3/2}} \Biggl\{
        \sum_{j=0}^3 c_j(z)
        \left[\Delta_l^j(t_1)+\Delta_l^j(t_2)\right]-
{}\\
{}  -\sum_{j=0}^3 c_j^*(z)
        \left[
        \Delta_l(t_1)+\Delta_l(t_2)\right]^j-{}\\
        {}- \fr{zH_2(z)}{2}
         \left[\vphantom{\Delta^2_l}
          l\left(\Delta_l(t_1)+\Delta_l(t_2) \right)+\right.{}\\
{}        + 2\rho_3\Delta_l(t_1)\Delta_l(t_2)-\Delta_l^2(t_1)\Delta_l(t_2)-{}\\
\left. {}-
        \Delta_l(t_1)\Delta_l^2(t_2)\right] \Biggr\}\,;
   \end{multline*}
   
   \vspace*{-12pt}
   
   \noindent
   \begin{align*} 
c_0^*(z)&=2c_0(z)\,; & c_1^*(z)&=c_1(z)-\fr{l H_3(z)}{2}\,;\\
c_2^*(z)&=c_2(z)\,; &c_3^*(z)&=c_3(z)\,;
\end{align*}
   $P_4^*(x,y)$~--- некоторый многочлен степени~4.

   Воспользовавшись~(\ref{e15c}), преобразуем интеграл
   \begin{multline*} 
\iint\limits_{\bf R^2} U_2^*(z,t_1,t_2;\theta)
      \prod_{j=1}^2 h_l(t_j) f_l(t_j;\theta) \mu(dt_j)={}
\\
{} =\fr{D_1^2}{n}+\fr{1}{n^{3/2}}
      \Biggl\{ \sum_{j=0}^3 \left[c_j(z) 2D_0D_j \right]-
      2c_0(z)D_0^2-{}\\
      {}- \left[c_1(z)-\fr{lH_3(z)}{2} \right]
       2D_0D_1-\hspace*{40pt}
{}
      \end{multline*}
      
      
  \noindent
      \begin{multline}
{} -c_2(z) \left[2D_0D_2+2D_1^2 \right]-
      c_3(z) \left[2D_0D_3+6D_1D_2 \right]-
   {}\\
{} -\fr{zH_2(z)}{2} \left[2lD_0D_1+2\rho_3D_1^2-2D_1 D_2 \right]
      \Biggr\}={}\\
      {}=\fr{D_1^2}{n}\left\{1+
      \fr{z}{\sqrt{n}} \left[-\rho_3+
      \fr{2\left(D_2-lD_0 \right)}{D_1} \right] \right\}\,.
   \label{e18c}
\end{multline}
   Обозначим
   $B_2=\left\{\left(t_1,t_2\right)\in {\bf R^2}:~
    \left|\Delta_l(t_1)\right|<n^\nu\right.$, $\left.\left|\Delta_l(t_2)\right|<n^\nu 
    \vphantom{\bf R^2}\right\}$,
$\overline{B_2}={\bf R^2}\setminus B_2$.
    Используя~(\ref{e9c}), (\ref{e10c}), (\ref{e18c}) и 
условия $(A_5)$ и~$(A_6)$, оценим разность
   \begin{multline*}  
\!\!\!\widehat{\bf {V}}[h_n(s)]-\fr{D_1^2}{n}\left\{1+
      \fr{z}{\sqrt{n}} \left[-\rho_3+
      \fr{2\left(D_2-lD_0 \right)}{D_1} \right] \right\}=
   {}\hspace*{-0.5pt}\\
{} =\iint\limits_{B_2} \left[\vphantom{n^{-2}}
U_2^*(z,t_1,t_2;\theta)+
        P_4^* \left(\Delta_l(t_1),\Delta_l(t_2)\right)\right.\times{}\\
\left.{}        \times
        {\bf O}\left(n^{-2}\right)\right]
      \prod_{j=1}^2 h_l(t_j) f_l(t_j;\theta) \mu(dt_j)+
   {}\\
{}   +{\bf O}\left(1\right)\iint\limits_{\overline{B_2}}
        \prod_{j=1}^2 h_l(t_j) f_l(t_j;\theta) \mu(dt_j)-{}
        \\
        {}-
        \iint\limits_{\bf R^2}U_2^*(z,t_1,t_2;\theta)
        \prod_{j=1}^2 h_l(t_j) f_l(t_j;\theta) \mu(dt_j)={}\\
        {}=
        {\bf O}\left(n^{-2}\right)\,,
   \end{multline*}
   что и завершает доказательство теоремы~3.

\bigskip

\noindent
{\bf Следствие 5.} {\it Пусть выполнены предположения
  $(A_1)\mbox{--}(A_6)$. Тогда при $~n\rightarrow\infty$
  \begin{multline}
  \widehat{\bf {V}}[h_n(S_n)]=
      \fr{\left[g'(\theta) \right]^2}{nb^2}
      \vphantom{\fr{2g''(\theta)}{b g'(\theta)}}\left\{       \vphantom{\fr{2g''(\theta)}{b g'(\theta)}}
      1+{}\right.\\[3pt]
\left.      {}+
      \fr{Z_n}{\sqrt{n}} \left[\fr{2g''(\theta)}{b g'(\theta)}-\rho_3
       \right] \right\}+{\bf O}_P\left(n^{-2}\right)\,.
       \label{e19c}
\end{multline}
   }

\bigskip

\noindent
{\bf Примечание.} Как видно из~(\ref{e19c}), главный член разложения
   несмещенной оценки дисперсии несмещенной оценки
   $\widehat{\bf {V}}[h_n(S_n)]$ совпадает с нижней границей
   неравенства Рао--Крамера для дисперсии несмещенной оценки
   дисперсии ${\bf {V}}[h_n(S_n)]$, поскольку в силу леммы~2
   при сделанных предположениях информация Фишера, содержащаяся в
   одном наблюдении, равна $b^2$.

    Наконец, используя утверждения следствий~2 и~5, нетрудно
    получить результат, который можно применить к построению
    асимптотических доверительных интервалов, основываясь на
    несмещенных оценках. 

\medskip

\noindent
 {\bf Следствие 6.} {\it Пусть выполнены предположения
  $(A_1)\mbox{--}(A_6)$, $g'(\theta)\neq 0$. Тогда
  \begin{multline*}  
\mathrm{sign}\!\left[ g'(\theta) \right]
      \fr{h_n(S_n)-g(\theta)}{\sqrt{\widehat{\bf {V}}[h_n(S_n)]}}={}\\
      {}=
      Z_n+\fr{1}{2\sqrt{n}}
      \left[\rho_3 -\fr{g''(\theta)}{b g'(\theta)}
      \left(Z_n^2+1 \right) \right]+{\bf O}_P\left(n^{-1}\right)\\
            \mbox{при~}n\rightarrow\infty\,.
  \end{multline*}
   }


\vspace*{-12pt}

\section{Заключение}

   Найдены стохастические разложения, определяющие предельное
  поведение всех несмещенных оценок, которые могут быть построены
  на основе повторной выборки  из распределения, принадлежащего
  однопараметрическому экспоненциальному семейству.
   Реализованный в работе подход
  к построению стохастических разложений может быть перенесен на
  случай многопараметрического экспоненциального семейства.
   Полученные результаты могут быть применены к решению разнообразных
   асимптотических параметрических задач интервального оценивания и проверки
  статистических гипотез, основываясь на несмещенных оценках.
  
{\small\frenchspacing
{%\baselineskip=10.8pt
\addcontentsline{toc}{section}{Литература}
\begin{thebibliography}{99}    

\bibitem{1c}
\Au{Воинов В.\,Г., Никулин М.\,С.}
Несмещенные оценки и их применения.~---
 М.: Наука, 1989.
 %2
\bibitem{2c}
\Au{Voinov V.\,G., Nikulin M.\,S.}
Unbiased estimators and their
applications. Vol.~2: Мultivariate case.~---  Dordrecht, the Netherlands, 1996.
 %3
\bibitem{3c}
\Au{Pommeret~D.}
A construction of the UMVU estimator for simple
quadratic natural exponential families~// J. Multivariate Analysis, 2003.
Vol.~85. P.~217--233.
 %4
\bibitem{4c}
\Au{Blazquez F.\,L., Rubio D.\,G.}
Unbiased estimation in the
 multivariate natural exponential family with simple quadratic variance function~//
 J. Multivariate Analysis, 2003. Vol.~86. P.~1--13.
 %5
\bibitem{5c}
\Au{Aghili A.}
On minimum variance unbiased estimators of
exponential families~// Int. Math. J., 2004. Vol.~4. P.~383--387.
 %6
\bibitem{6c}
\Au{Wang B.}
Unbiased estimations for the exponential distribution
 based on step-stress accelerated life-testing data~// Applied Mathematics
 and Computation, 2006. Vol.~173. P.~1227--1237.
 %7
\bibitem{7c}
\Au{Liu~A., Wu C., Yu~K.\,F., Yuan~W.}
Completeness and unbiased
estimation of mean vector in the multivariate group sequential
case~// J. Multivariate Analysis, 2007. Vol.~98. P.~505--516.
 %8
\bibitem{8c}
\Au{Sharma D.}
Asymptotic equivalence for two estimators
for an exponential family~// Annals of Statistics, 1973. P.~973--980.
 %9
\bibitem{9c}
\Au{Portnoy S.}
Asymptotic efficiency of minimum
variance unbiased estimators~// Annals of Statistics, 1977. Vol.~5.
No.\,3. P.~522--529.
 %10
\bibitem{10c}
\Au{Lopez-Blazquez~F., Salamanca-Mino~B.}
Limit distributions of unbiased estimators
in natural exponential families~// Statistics, 2002. Vol.~36. No.\,4. P.~329--338.
 %11
\bibitem{11c}
\Au{Morris C.\,N.}
Natural exponential families with
quadratic variance functions: Statistical theory~//  Annals of
Statistics, 1983. Vol.~11. No.\,2. P.~515--529.

\bibitem{12c}
\Au{Чичагов В.\,В.}
Об асимптотической нормальности одного
класса несмещенных оценок в случае абсолютно непрерывных
распределений~/ Стат. методы оценивания и проверки гипотез.~---
 Пермь: Изд-во Пермского университета, 2000. С.~71--79. [Перевод:
\Au{Chichagov V.\,V.} Concerning asymptotic normality of a
class of unbiased estimators in the case of absolutely continuous
distributions~/ Statistical methods of estimation and testing of
hypotheses~// J. Math. Sci. (N.Y.), 2004. Vol.~119. No.\,3. P.~336--341.]

 %13
\bibitem{13c}
\Au{Чичагов В.\,В.}
Об асимптотическом поведении несмещенных оценок
вероятностей для решетчатых распределений, достаточной статистикой
которых является среднее~/ Стат. методы оценивания и проверки
гипотез.~--- Пермь: Изд-во Пермского университета, 2002. С.~106--120.

 %14
\bibitem{14c}
\Au{Петров В.\,В.}
Суммы независимых случайных величин.~--- М.: Наука, 1972.
 %15
\bibitem{15c}
\Au{Bhattacharya R.\,N., Ghosh~J.\,K.}
On the valitity of the
formal Edgeworth expansions~// Annals of Statistics, 1978. Vol.~6. P.~434--451.
 %16
\bibitem{16c}
\Au{Barndorff-Nielsen~O.\,E., Cox~D.\,R.}
Edgeworth and
Saddle-point approximations with statistical applications~// J. R.
Statist. Soc. B, 1979. Vol.~41. P.~279--312.
 %17
\bibitem{17c}
\Au{Барндорф-Нильсен~О., Кокс~Д.}
Асимптотические методы в
математической статистике.~--- М.: Мир, 1999. 

 %18
\bibitem{18c}
\Au{Lieberman G.\,J., Resnikoff~G.\,J.}
Sampling plans
for inspection by variables~// J. Amer. Statist. Assoc., 1955.
Vol.~50. P.~457--516.
 %19
\bibitem{19c}
\Au{Patil G.\,P., Wani~J.\,K.}
Minimum variance unbiased
estimation of the distribution function admitting a sufficient
statistics~// Ann. Inst. Statist. Math., 1966. Vol.~18. P.~39-47.
 %20
\bibitem{20c}
\Au{Лумельский Я.\,П., Сапожников~П.\,Н.}
Несмещенные оценки для
 плотностей распределений // Теория вероятностей и ее применение, 1969.
Vol.~XIV. №\,2. С.~372--380.
 %21
 \bibitem{21c}
\Au{Лумельский Я.\,П.}
Случайные блуждания, отвечающие
обобщенным урновым схемам~// ДАН СССР, 1973. Т.\,209. №\,6. С.~1281--1284.

\bibitem{22c}
\Au{Абусев Р.\,А., Лумельский~Я.\,П.}
Несмещенные оценки
и задачи классификации многомерных нормальных совокупностей~// Теор.
вероятностей и ее применение, 1980. Т.~25. №\,2. С.~381--389.


\bibitem{23c}
\Au{Michel R.}
Asymptotic expansions for conditional distributions~//
J. Multivariate Anal., 1979. P.~393--400.
 %25

\bibitem{24c}
\Au{Hipp C.}
Asymptotic expansions for conditional distributions:
The lattice case~// Probab. Math. Statist., 1984. Vol.~4. P.~207--219.

\bibitem{25c}
\Au{Skovgaard I.\,M.}
Saddlepoint expansions for conditional
distributions~// J. of Applied Probability, 1987. Vol.~24. No.\,4.
P.~875--887.

 %26
\bibitem{26c}
\Au{Лумельский Я.\,П.}
Несмещенные достаточные оценки
вероятностей в случае многомерного нормального закона~// Вестник
МГУ, 1968. Vol.~6. С.~14--17.  [Перевод: 
\Au{Lumelskii~Ya.\,P.}
Unbiased sufficient estimators for probabilities in the case of a
multivariate normal law~// Vestnik Moskov. Univ. Ser. Mat. Meh., 1968.
Vol.~23. No.\,6. P.~14--17; Selected translations in mathematical
statistics and probability. Vol.~13. Amer.\ Math.\ Soc.,
Providence, R.I., 1973. P.~251--256.]
 %27
\bibitem{27c}
\Au{Ившин В.\,В., Лумельский~Я.\,П.}
Статистические задачи оценивания в
модели <<нагрузка--прочность>>.~--- Пермь: Изд-во Пермского университета, 1995.
 %28
\bibitem{28c}
\Au{Лумельский Я.\,П., Фейгин~П.\,Д.}
Несмещенные оценки дисперсии в
параметрическом случае~/ Стат. методы оценивания и проверки
гипотез.~--- Пермь: Изд-во Пермского университета, 2002. С.~38--51. %26
 %29
\bibitem{29c}
\Au{Леман Э.}\label{end\stat}
Проверка статистических гипотез.~--- М.: Наука, 1979.
\end{thebibliography}

}
}
\end{multicols}