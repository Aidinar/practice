\def\stat{abstr}
{%\hrule\par
%\vskip 7pt % 7pt
\raggedleft\Large \bf%\baselineskip=3.2ex
A\,B\,S\,T\,R\,A\,C\,T\,S \vskip 17pt
    \hrule
    \par
\vskip 21pt plus 6pt minus 3pt }

\def\tit{NETWORK METHODS OF SEPARATION OF MIXTURES OF PROBABILITY
DISTRIBUTIONS AND~THEIR APPLICATION TO THE DECOMPOSITION OF VOLATILITY INDEXES}

%1
\def\aut{V.~Korolev$^1$, E.~Nepomnyashchiy$^2$, A.~Rybal'chenko$^3$, and A.~Vinogradova$^4$}

\def\auf{$^1$M.\,V.~Lomonosov Moscow State University; IPI RAN, vkorolev@comtv.ru\\[1pt]
$^2$45th Central Science Research Institute, Russian Ministry of Defence\\[1pt]
$^3$M.\,V.~Lomonosov Moscow State University,  alex-rybalchenko@yandex.ru\\[1pt]
$^4$M.\,V.~Lomonosov Moscow State University, a\_nuta@mail.ru
 }

\def\leftkol{\ } % ENGLISH ABSTRACTS}

\def\rightkol{\ } %ENGLISH ABSTRACTS}

\titele{\tit}{\aut}{\auf}{\leftkol}{\rightkol}


\noindent
Methods are proposed for the statistical separation of mixtures of
probability distributions based on the minimization of the discrepancy
between the theoretical and empirical distribution functions. Main
attention is paid to the minimization of sup- and L1-norms of the
discrepancy. It is demonstrated that these problems can be reduced
to problems of linear programming. The simplex-method is used for
their numerical realization. The proposed methods are applied to the
problem of decomposition of the volatility of financial indexes.
Examples of the decomposition of the volatility of AMEX, CAC 40,
NIKKEI, and NASDAQ indexes are presented.

\label{st\stat}

 \KWN{separation of mixtures of probability distributions;
problem of linear programming; simplex-method; volatility
}

\vskip 18pt plus 6pt minus 3pt


\def\tit{SOME STATISTICAL PROBLEMS RELATED TO THE LAPLACE DISTRIBUTION
}

%2
\def\aut{V.~Bening$^1$ and V.~Korolev$^2$}
\def\auf{$^1$M.\,V.~Lomonosov Moscow State University; IPI RAN, bening@yandex.ru\\[1pt]
$^2$M.\,V.~Lomonosov Moscow State University; IPI RAN, vkorolev@comtv.ru}

\def\leftkol{\ } % ENGLISH ABSTRACTS}

\def\rightkol{\ } %ENGLISH ABSTRACTS}

\titele{\tit}{\aut}{\auf}{\leftkol}{\rightkol}

\noindent
Approach proposed in~\Au{Bening~V., Korolev~V.}~// Rus. J. Probability Theory
and Its Applications, 2004. Vol.~49. No.\,3. P.~419--435 is developed. Some grounds are
given why it is natural to use the Laplace distribution in some problems
of probability theory and mathematical statistics. As a statistical
illustration, an application  of the Laplace distribution to problems
of the asymptotical testing of hypotheses is considered.

%\label{st\stat}

\KWN{Laplace distribution; testing of statistical hypotheses; power function;
asymptotically most powerful tests
}

%\pagebreak

% \thispagestyle{headings}

\vskip 18pt plus 6pt minus 3pt

%\vfil

%3
\def\tit{SOME BOUNDS FOR CLOSED TO ABSORBING MARKOV MODELS}

\def\aut{A.\,I.~Zeifman$^1$, A.\,V.~Chegodaev$^2$, and V.\,S.~Shorgin$^3$}
\def\auf{$^1$VSPU; IPI RAN; VSCC CEMI RAS, a\_zeifman@mail.ru\\[1pt]
$^2$VSPU, cheg\_al@mail.ru\\[1pt]
$^3$IPI RAN, vshorgin@ipiran.ru}

\titele{\tit}{\aut}{\auf}{\leftkol}{\rightkol}

\noindent Nonstationary continuous-time Markov models with
``almost absorbing'' zero state are considered. Such models describe some problems
of queueing theory. The properties and bounds for the limit characteristics
of such models are obtained. Simple continuous-time random walks as example of considered
models are also studied.

\KWN{queueing networks; continuous-time Markov chains;
ergodicity; birth and death process; simple random walk}
%\pagebreak


%\vfil
 \vskip 18pt plus 6pt minus 3pt
% \vskip 24pt plus 9pt minus 6pt

%4
\def\tit{THE OPTIMIZATION OF THE SPATIAL LOCATION OF SERVICE STATIONS}

\def\aut{T.~Zakharova}
\def\auf{M.\,V.~Lomonosov Moscow State Unisersity, lsa@cs.msu.su
}


%\def\leftkol{ENGLISH ABSTRACTS}

%\def\rightkol{ENGLISH ABSTRACTS}

\titele{\tit}{\aut}{\auf}{\leftkol}{\rightkol}

\noindent
In the $N$-dimensional Euclidean space, service stations are being
efficiently located according to the criterion of stationary average
waiting time.

\KWN{service stations location; optimality criterion; waiting time
}
%\pagebreak

%\vful

 \vskip 12pt plus 6pt minus 3pt

% \vskip 24pt plus 9pt minus 6pt
%\vskip 6pt plus 3pt minus 3pt
%\vspace*{12pt}

%5
\def\tit{ELIMINATION OF ECTOPIC BEATS FROM HEART TACHOGRAM
USING ROBUST ESTIMATES}

\def\aut{A.\,V.~Markin$^1$ and O.\,V.~Shestakov$^2$}

\def\auf{$^1$M.\,V.~Lomonosov Moscow State University, artem.v.markin@mail.ru\\[1pt]
$^2$M.\,V.~Lomonosov Moscow State University, oshestakov@cs.msu.su}

\def\leftkol{ENGLISH ABSTRACTS}

\def\rightkol{ENGLISH ABSTRACTS}

\titele{\tit}{\aut}{\auf}{\leftkol}{\rightkol}

\noindent
A method of removing ectopic beats from tachogram is introduced.
The method is based on authors' mathematical model. Parameters of the model
are estimated using robust
linear regression. Elimination of ectopic beats is made by the instrumentality
of confidential intervals for RR-interval differences.
Real data results are supplied and discussed.

\KWN{tachogram; robust estimates; linear regression; confidential
intervals}

%\vskip 18pt plus 6pt minus 3pt

 \vskip 12pt plus 6pt minus 3pt

% \pagebreak

%6
\def\tit{ON THE ASYMPTOTIC DISTRIBUTION OF THE MAXIMUM ORDER STATISTIC
IN A SAMPLE WITH RANDOM SIZE}

\def\aut{V.~Pagurova}

\def\auf{$^1$M.\,V.~Lomonosov Moscow State University, pagurova@yandex.ru
}

\titele{\tit}{\aut}{\auf}{\leftkol}{\rightkol}

\noindent
The asymptotic distribution of the normalized maximum is
investigated under the assumption that the random sample size
is representable as a sum of $n$ independent identically distributed
random variables. This paper generalizes the results of \Au{Pagurova~V.}~//
Statistical methods of estimating and testing of hypoteses.~--- Perm, 2005. P.~104--113
where the sample size was Poisson-distributed with a parameter~$n$. For a one-parameter family
of distributions depending on an unknown location parameter,
the rate of convergence of the distribution of the normalized
maximum to the limit law is investigated. The classes of distributions
with exponential and power-type tails are considered.



\KWN{randomly indexed maximum; one-parameter family of distributions;
convergence rate}

\vskip 12pt plus 6pt minus 3pt

%7
\def\tit{ESTIMATION OF DELAY DISTRIBUTION IN BIOLOGICAL DYNAMICAL
MODELS WITH A MODEL OF HIV INFECTION AS AN EXAMPLE
}
\def\aut{A.\,N.~Ushakova}

\def\auf{Norwegian University of Science and Technology, 
anastasi@math.ntnu.no}

%\def\leftkol{ENGLISH ABSTRACTS}

%\def\rightkol{ENGLISH ABSTRACTS}

\titele{\tit}{\aut}{\auf}{\leftkol}{\rightkol}

\noindent Two methods for estimating a delay distribution
in biological dynamical systems are presented. The model of HIV infection serves as an
example of such a system. The first method is based on parametric
approach and approximation of the delay density by a gamma-density.
The second method is nonparametric and is based on solution of a convolution equation
with selection of the regularization parameter either from the measurement
error estimated beforehand, or from the smoothness of the delay
distribution.

\KWN{dynamical systems with delay; distribution of delay; regularization
parameter}

\vskip 18pt plus 6pt minus 3pt

%8
\def\tit{EXISTENCE OF CONSISTENT TEST SEQUENCES AT THE
COMPLEX NULL HYPOTHESES IN~DISCRETE STATISTICAL PROBLEMS
}
\def\aut{A.~Grusho$^1$, E.~Timonina$^2$, and V.~Chentsov$^3$}

\def\auf{$^1$Russian State Humanitarian University; M.\,V.~Lomonosov Moscow State University, 
grusho@yandex.ru\\[1pt]
$^2$Russian State Humanitarian University, eltimon@yandex.ru\\[1pt]
$^3$IPI RAN, ipiran@ipiran.ru}

%\def\leftkol{ENGLISH ABSTRACTS}

%\def\rightkol{ENGLISH ABSTRACTS}

\titele{\tit}{\aut}{\auf}{\leftkol}{\rightkol}

\noindent
The problem of existence of a consistent test sequence is considered for
testing of complex hypothesis against complex alternatives in a sequence
of finite spaces. When the sequence of spaces is generated by Cartesian
product of a finite set and probability measures on these spaces are
consistent, it is possible to find sufficient conditions of existence
of consistent test sequence in terms of topological properties of the
certain sets. Under additional conditions it is possible to refuse the
requirement of domination of certain measures from a null hypothesis
and uniform limitation of density.


\KWN{consistent test sequence; complex hypothesis against complex alternatives;
finite spaces; probability measures; sufficient conditions}

\vskip 18pt plus 6pt minus 3pt

%9
\def\tit{STOCHASTIC EXPANSIONS OF UNBIASED ESTIMATORS FOR THE CASE\\
OF~ONE-PARAMETER EXPONENTIAL FAMILY
}
\def\aut{V.~Chichagov}

\def\auf{Perm State University, chvv50@mail.ru}

%\def\leftkol{ENGLISH ABSTRACTS}

%\def\rightkol{ENGLISH ABSTRACTS}

\titele{\tit}{\aut}{\auf}{\leftkol}{\rightkol}

\noindent
Asymptotic and stochastic expansions are obtained for all unbiased
estimators which can be constructed from a repeated sample of independent
identically distributed random variables having one and the same distribution
from a one-parameter exponential family. A comparison is conducted of
expansions of unbiased estimators with those of maximum likelihood
estimators. A stochastic expansion is obtained for an unbiased estimator
of the variance of an unbiased estimator.

\KWN{unbiased estimator; exponential family; stochastic expansion;
unbiased estimator of variance}


 \label{end\stat}
 %\pagebreak