{ %\Large  
{ %\baselineskip=16.6pt

\vspace*{-48pt}
\begin{center}\LARGE
\textit{Предисловие}
\end{center}

\thispagestyle{empty}

\hspace*{5mm}Статьи, собранные в данном тематическом выпуске журнала <<Информатика и её применения>>,
объединены тем, что все они посвящены разработке новых вероятностно-статистических методов и их
применению к решению конкретных задач информатики и информационных технологий. Проблематика,
охватываемая публикуемыми работами, развивается в рамках научного сотрудничества между Институтом
проблем информатики Российской академии наук и факультетом вычислительной математики и
кибернетики Московского государственного университета им. М.\,В.~Ломоносова, в частности в ходе работ
над совместными научными проектами. Многие из авторов статей, включенных в данный номер журнала,
являются активными участниками традиционного международного семинара по проблемам устойчивости
стохастических моделей, руководимого В.\,М.~Золотарёвым и В.\,Ю.~Королёвым; ежегодные сессии этого
семинара проводятся под эгидой МГУ и ИПИ РАН.

       В статье В.\,Ю.~Королёва, Е.\,В.~Непомнящего и А.\,В.~Виноградовой предлагается
принципиально новый метод исследования стохастической структуры хаотических процессов и приводятся
примеры его применения к анализу волатильности (степени изменчивости) финансовых индексов. Этот
метод также может быть использован для исследования статистической структуры других информационных
потоков.

       Статья В.\,Е.~Бенинга и В.\,Ю.~Королёва посвящена исследованию задач проверки статистических
гипотез о параметрах распределения Лапласа. Это распределение очень часто успешно используется в
качестве альтернативы нормальному закону в тех прикладных задачах, в которых статистические
закономерности поведения исследуемых характеристик демонстрируют склонность к большим
вероятностям существенных отклонений от среднего значения, нежели при нормальном распределении.

       Статья Т.\,В.~Захаровой посвящена исследованию оптимальных пространственных размещений
обслуживающих приборов в сложных системах обслуживания, в частности в сложных технических
(транспортных) и информационных системах.

       В статье А.\,И.~Зейфмана, А.\,В.~Чегодаева и В.\,С.~Шоргина методы теории дифференциальных
уравнений применяются к исследованию сложных марковских цепей, которые описывают поведение
сложных систем массового обслуживания.

       Статья О.\,В.~Шестакова и А.\,В.~Маркина посвящена исследованию методов статистического
анализа ритмограмм, что имеет важное значение при создании систем анализа кардиограмм в реальном
времени (в ``on-line'' режиме).

       В статье В.\,И.~Пагуровой исследуется точность аппроксимации распределения экстремальных
порядковых статистик в выборках случайного объема. Подобные задачи имеют первостепенное значение
при анализе рисков в неоднородных потоках экстремальных событий, в частности в хаотических
информационных потоках в сложных инфотелекоммуникационных системах.

       Статья А.\,Н.~Ушаковой посвящена математическому моделированию процессов распространения
инфекционных вирусных заболеваний, в частности ВИЧ-инфекции.

       В статье А.\,А.~Грушо, Е.\,Е.~Тимониной и В.\,М.~Ченцова решаются статистические задачи,
имеющие принципиальное значение в проблемах информационной безопасности, в частности при
исследовании возможности существования скрытых каналов передачи информации.

       Наконец, статья В.\,В.~Чичагова посвящена исследованию свойств несмещенных оценок
параметров распределений из экспоненциального семейства. В этой статье строятся асимптотические
разложения, позволяющие проводить сравнительное исследование точности несмещенных оценок
параметров разнообразных вероятностных моделей реальных процессов.

       Редакционная коллегия журнала выражает надежду, что данный специальный выпуск будет
интересен специалистам в области информатики, теории вероятностей и математической статистики.\\

\noindent
Заместитель главного редактора журнала <<Информатика и её применения>>,\\
директор ИПИ РАН, академик  \hfill
\textit{И.\,А.~Соколов}\\

\noindent
Редактор-составитель тематического выпуска,\\
профессор кафедры математической статистики\\
факультета вычислительной математики и кибернетики МГУ им.~М.\,В.~Ломоносова,\\
ведущий научный сотрудник ИПИ РАН,\\
доктор физико-математических наук\hfill
 \textit{В.\,Ю.~Королёв}


} }