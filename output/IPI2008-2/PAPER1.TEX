%\renewcommand{\r}{\rm {I\hspace{-0.7mm}R}}
%\setcounter{page}{3}

\def\stat{korol}


\def\tit{СЕТОЧНЫЕ МЕТОДЫ РАЗДЕЛЕНИЯ СМЕСЕЙ ВЕРОЯТНОСТНЫХ
РАСПРЕДЕЛЕНИЙ И ИХ ПРИМЕНЕНИЕ К ДЕКОМПОЗИЦИИ
ВОЛАТИЛЬНОСТИ ФИНАНСОВЫХ ИНДЕКСОВ$^*$}
\def\titkol{Сеточные методы разделения смесей вероятностных
распределений и их применение}
% к декомпозиции волатильности финансовых индексов}
\def\autkol{В.\,Ю.~Королёв,
Е.\,В.~Непомнящий, А.\,Г.~Рыбальченко, А.\,В.~Виноградова}
\def\aut{В.\,Ю.~Королёв$^1$,
Е.\,В.~Непомнящий$^2$, А.\,Г.~Рыбальченко$^3$, А.\,В.~Виноградова$^4$}

\titel{\tit}{\aut}{\autkol}{\titkol}

{\renewcommand{\thefootnote}{\fnsymbol{footnote}}\footnotetext[1]
{Работа выполнена при
поддержке РФФИ, гранты 08-01-00345, 08-01-00363,
08-07-00152.}}

\renewcommand{\thefootnote}{\arabic{footnote}}
\footnotetext[1]{Московский государственный
университет им.~М.\,В.~Ломоносова, факультет вычислительной
математики и кибернетики; Институт проблем
информатики Российской академии наук, vkorolev@comtv.ru}
\footnotetext[2]{45 ЦНИИ МО РФ}
\footnotetext[3]{Московский
государственный университет им.~М.\,В.~Ломоносова, факультет
вычислительной математики и кибернетики, alex-rybalchenko@yandex.ru}
\footnotetext[4]{Московский
государственный университет им.~М.\,В.~Ломоносова, факультет
вычислительной математики и кибернетики, a\_nuta@mail.ru}

\Abst{Предложены методы статистического
разделения смесей вероятностных распределений, основанные на
минимизации невязки между теоретической и эмпирической функциями
распределения. Основное внимание уделено минимизации $\sup$- и
$L_1$-норм невязки. Показано, что такие задачи могут быть сведены
к задачам линейного программирования. Для их численной реализации
используется симплекс-метод. Предложенные методы применены к
решению задачи декомпозиции волатильности финансовых индексов.
Приведены примеры декомпозиции волатильности индексов AMEX, CAC
40, NIKKEI, NASDAQ.}

\KW{разделение смесей вероятностных
распределений; задача линейного программирования; симплекс-метод;
волатильность}

      \vskip 24pt plus 9pt minus 6pt

      \thispagestyle{headings}

      \begin{multicols}{2}

      \label{st\stat}

\section{Введение}

Для численного решения задачи разделения конечных смесей
вероятностных распределений (т.\,е.\ задачи отыскания
статистических оценок весов компонент смеси и параметров компонент
смеси) при относительно большом числе компонент традиционно
применяется ЕМ-алгоритм. Если функция правдоподобия регулярна, то
этот метод, как правило, находит наиболее правдоподобные оценки
параметров смеси. Однако если функция правдоподобия нерегулярна,
имеет много локальных максимумов (возможно, к тому же
бесконечных), то ЕМ-алгоритм становится крайне неустойчивым. К
сожалению, последнее обстоятельство является серьезным
препятствием при интерпретации результатов применения ЕМ-алгоритма
к разделению конечных смесей нормальных законов. Именно такие
смеси повсеместно применяются при математическом моделировании
многих явлений в самых разных областях~--- от биологии до экономики
и от физики до финансового анализа.

В частности, как было экспериментально установлено, ЕМ-алгоритм
обладает сильной неустойчивостью по начальным данным. Например, в
случае четырехкомпонентной смеси нормальных\linebreak законов при объеме
выборки 200--300 наблюдений замена лишь одного наблюдения другим
может кардинально изменить итоговые оценки, полученные с помощью
ЕМ-алгоритма~\cite{1k}.

Поэтому крайне необходимо иметь альтернативные методы разделения
смесей, ориентированные не на максимизацию функции правдоподобия,
а на оптимизацию других разумных критериев качества получаемых
оценок. Наличие таких альтернативных достаточно эффективных
методов является принципиально важным с точки зрения возможности
адекватной практической интерпретации получаемых оценок. Именно
такие альтернативные методы и предлагаются в данной статье.

\section{Основная идея сеточных методов разделения
смесей вероятностных распределений}

Идея, которая лежит в основе метода разделения смесей,
описываемого в данной статье, очень близка к идее гармонического
анализа, когда периодическая функция раскладывается в ряд Фурье,
т.\,е.\ представляется взвешенной комбинацией (рядом), возможно,
бесконечно большого числа синусов и косинусов с различными (но
кратными) периодами (т.\,е.\ с различными параметрами масштаба~---
частотами). Возможность приближения исходной функции с помощью
такого разложения обоснована тем обстоятельством, что семейство
синусов и косинусов с указанными свойствами образует базис, т.\,е.\
полную систему (линейно независимых) функций в пространстве
(регулярных) периодических функций.

Аналогия будет видна более отчетливо, если рассмотреть семейство
чисто масштабных смесей нормальных законов с нулевым средним.
Поскольку при некоторых условиях регулярности каж\-дое распределение
вероятностей, сосредоточенное на неотрицательной полуоси, может
быть приближено решетчатым распределением с произвольно высокой
точностью (скажем, в метрике Леви, метризующей слабую сходимость),
можно\linebreak
заключить, что семейство нормальных распределений ${\bf
N}_{\mathrm{rat}}=\{\mathcal{N}(0,s):\,s$~--- рациональное$\}$
образует счетную полную систему функций в пространстве масштабных
смесей нормальных законов с нулевым средним. Это означает, что для любой
масштабной смеси нормальных законов и произвольно малого
$\epsilon>0$ существует конечная линейная комбинация распределений
из семейства ${\bf N}_{\mathrm{rat}}$ такая, что рас\-стояние
(скажем, расстояние Леви) между смесью и линейной комбинацией не
превышает $\epsilon>0$. Поскольку число слагаемых в такой линейной
комбинации конечно и все па\-ра\-мет\-ры масштаба\linebreak рациональны, найдется
такое (минимальное)\linebreak значение па\-ра\-мет\-ра масштаба, что параметры
масштаба всех членов рассматриваемой линейной комбинации будут
кратными этому минимальному значению па\-ра\-мет\-ра масштаба.

Конечные смеси нормальных законов находят самое широкое применение
как модели распределений многих величин, наблюдаемых на практике
(см., например,~\cite{2k}). Именно поэтому данная статья посвящена
применению сеточных методов разделения смесей к решению задачи
оценивания параметров конечных смесей нормальных законов. Всюду
далее стандартная нормальная функция распределения будет
обозначаться $\Phi(x)$.

Рассмотрим смесь функций распределения вида
\begin{equation}
F(x)=\sum_{i=1}^kp_i\Phi\Big(\fr{x-a_i}{\sigma_i}\Big)\,,\quad
x\in\r\,,
\label{e1k}
\end{equation}
где $k\ge 1$~--- целое. В классической задаче
разделения смесей параметрами, подлежащими статистическому
оцениванию, являются тройки $(p_i, a_i, \sigma_i)$,
$i=1,\ldots,k$, где $a_i\in\r$, $\sigma_i>0$, $p_i\ge 0$,
$p_1+\ldots+p_k=$\linebreak $=\;1$.

Предположим, что заранее известны числа $\underline a$, $\overline
a$ и~$\overline\sigma$ такие, что $\underline a\le a_i\le\overline
a$ и $\sigma_i\le\overline\sigma$ при всех $i=1,\ldots,k$. Другими
словами, известны диапазоны изменения неизвестных параметров $a_i$
и $\sigma_i$.

Идея, лежащая в основе рассматриваемого подхода, заключается в
замене {\it интервалов} $[\underline a, \overline a]$ и
$(0,\overline\sigma]$ возможных значений неизвестных параметров
масштаба $\sigma_i$ и сдвига $a_i$ {\it дискретными} множествами
известных точек. Эти точки могут быть выбраны, например, исходя из
следующих соображений.

Пусть $\varepsilon_a$ и $\varepsilon_{\sigma}$~--- положительные
числа, определяющие априорные требования к точности оценивания
параметров $a_i$ и $\sigma_i$:
\begin{equation}
\max_i|a_i-\widehat a_i|\le
\varepsilon_a\,,\quad \max_i|\sigma_i-\widehat\sigma_i|\le
\varepsilon_{\sigma}\,,
\label{e2k}
\end{equation}
где $\widehat a_i$ и
$\widehat\sigma_i$~--- искомые оценки параметров. Числа
$\varepsilon_a$ и $\varepsilon_{\sigma}$ также можно
интерпретировать как пороги различимости возможных значений
параметров: значения $a'$, $a''$ и $\sigma'$, $\sigma''$
соответственно считаются неразличимыми, если
\begin{equation}
|a'-a''|\le\varepsilon_a\,,\quad  |\sigma'-\sigma''|\le
\varepsilon_{\sigma}\,.
\label{e3k}
\end{equation}

Положим $k_{a}=[(\overline a-\underline a)/\varepsilon_a]+1$,
$k_{\sigma}=[\overline\sigma/\varepsilon_{\sigma}]+1$, где символ
$[z]$ обозначает целую часть числа $z$. Для $r=1,2,\ldots,k_{a}+1$
положим $\widetilde a_r=\underline a+(r-1)\varepsilon_a$.
Аналогично для $l=1,2,\ldots,k_{\sigma}$ положим $\widetilde
\sigma_l=l \varepsilon_{\sigma}$. Тогда точки с координатами
$(\widetilde a_r,\widetilde\sigma_l)$ образуют узлы конечной сети,
накрывающей прямоугольник $\{(a,\sigma):\,\underline a\le
a\le\overline a,\,0\le\sigma\le\overline\sigma\}$, представляющий
собой множество возможных значений параметров сдвига и масштаба
компонент смеси~(\ref{e1k}) (чтобы избежать возможной некорректности,
исключена возможность равенства параметра масштаба нулю). Число
узлов полученной сети равно $K=(k_{a}+1)k_{\sigma}$. Для удобства
записи и упрощения обозначений перенумеруем каким-либо образом
узлы указанной сети, вводя {\it единый} индекс $i$ для координат
$(\widetilde a_i,\widetilde \sigma_i)$ узла с номером $i$ после
перенумерации, $i=1,\ldots,K$.

Подход, рассматриваемый в данной статье, предложен в~\cite{1k}.
Базовая посылка этого подхода заключается в аппроксимации смеси~(1)
смесью с заведомо б$\acute{\mbox{о}}$льшим числом {\it известных}
компонент:
\begin{multline}
F(x)=\sum_{i=1}^kp_i\Phi\Big(\fr{x-a_i}{\sigma_i}\Big)\approx{}\\
{}\approx
\sum_{i=1}^K\widetilde p_i\Phi\Big(\fr{x-\widetilde
a_i}{\widetilde\sigma_i}\Big)\equiv \widetilde F(x)\,,\quad
x\in\r\,.
\label{e4k}
\end{multline}
Такое приближение практически допустимо,
поскольку в силу соотношений~(\ref{e2k}) и~(\ref{e3k}) для любой пары
$(a_r,\sigma_r)$ параметров компоненты смеси~(\ref{e1k}) обязательно
найдется практически не отличимая от нее пара $(\widetilde
a_i,\widetilde \sigma_i)$ параметров компоненты смеси $\widetilde
F(x)$. Веса же остальных компонент смеси $\widetilde F(x)$, для
параметров которых не найдется <<близкой>> пары параметров
$(a_r,\sigma_r)$ компоненты смеси~(\ref{e1k}), можно считать равными нулю.
Действительно, если бы в соотношении~(\ref{e4k}) вместо {\it
приближенного} было бы {\it точное} равенство, то в силу
идентифицируемости семейства конечных смесей нормальных законов и
в полном соответствии с определением идентифицируемости конечных
смесей с точностью до переиндексации были бы справедливы
равенства:
$$
k=K\,,\quad
 p_i=\widetilde p_i\,, \quad a_i=\widetilde a_i\,, \quad
 \sigma_i=\widetilde\sigma_i\,,\  i=1,\ldots,k\,.
$$

Заметим, что неизвестными параметрами смеси $\widetilde F(x)$
являются {\it только} веса $\widetilde p_1,\ldots,\widetilde p_K$.

Пусть $\emph{\textbf{x}} =(x_1,\ldots,x_n)$~--- (независимая)
выборка наблюдений, каждое из которых представляет собой
реализацию случайной величины с функцией распределения $F(x)$,
задаваемой соотношением~(\ref{e1k}). Пусть $(x_{(1)},\ldots,x_{(n)})$ и
$F_n(x)$~--- соответственно вариационный ряд и эмпирическая функция
распределения, построенные по выборке $\emph{\textbf{x}} $,
$$
F_n(x)=\fr {1}{n}\,\sum_{j=1}^n{\bf 1}(x_j<x)\,,\quad  x\in\r\,.
$$
При
этом очевидно, что
\begin{equation}
F_n(x_{(j)})=\fr{j}{n}\,,\quad j=1,\ldots,n\,.
\label{e5k}
\end{equation}
В силу теоремы Гливенко при больших $n$
равномерно по $x\in\r$ выполняется соотношение
\begin{equation}
F_n(x)\approx   F(x)\,.\label{e6k}
\end{equation}
Обозначим
$$
\Phi_{ij}=\Phi\Big(\fr{x_{(j)}-\widetilde
a_i}{\widetilde\sigma_i}\Big)\,,\quad
j=1,\ldots,n;\,i=1,\ldots,K\,.
$$
Отметим, что величины $\Phi_{ij}$
{\it известны}.

Из~(\ref{e4k}) и~(\ref{e5k}) вытекает, что при больших $n$
$$
\widetilde F(x)\approx F_n(x)\,,\quad  x\in\r\,,
$$
откуда с учетом~(\ref{e6k}) получается
приближенное соотношение
\begin{equation}
\widetilde F(x_{(j)})=\sum_{i=1}^K\widetilde p_i\Phi_{ij}\approx\fr{j}{n}\,,\quad
j=1,\ldots,n\,.
\label{e7k}
\end{equation}
Соотношение~(\ref{e7k}) можно использовать для
отыскания оценок параметров $\widetilde p_1,\ldots,\widetilde p_K$
с помощью метода наименьших квадратов и метода наименьших модулей,
рассматриваемых в последующих разделах. Реализация этих методов в
данном случае довольно проста, поскольку $\widetilde
F(x)$ зависит от параметров $\widetilde p_1,\ldots,\widetilde p_K$
{\it линейно}, так что мы не выходим за рамки линейной модели
регрессионного анализа.

\section{Разделение конечных смесей вероятностных распределений
с~фиксированными компонентами при помощи метода наименьших
квадратов}

Рассматриваемую задачу удобно записать в векторно-матричном виде.
Обозначим
\begin{gather*}
{\mathbf p}=
\begin{pmatrix}
\widetilde p_1\\[3pt]
\widetilde p_2\\[3pt]
\cdots\\[3pt]
\widetilde p_K
\end{pmatrix}\,;\quad
\mbox{\bf Ф}=
\begin{pmatrix}
\Phi_{11} & \Phi_{21} & \cdots & \Phi_{K1}\\[3pt]
\Phi_{12} & \Phi_{22} & \cdots & \Phi_{K2}\\[3pt]
\cdots    & \cdots    & \cdots & \cdots   \\[3pt]
\Phi_{1n} & \Phi_{2n} & \cdots & \Phi_{Kn}
\end{pmatrix}\,;\\
\mathbf{u}=
\begin{pmatrix}
\fr{1}{n}\\[4pt]
\fr{2}{n}\\[4pt]
\cdots\\[4pt]
\fr{n-1}{n}\\[4pt]
1
\end{pmatrix}\,.
\end{gather*}
Тогда соотношение~(\ref{e7k}) можно переписать в матричной форме:
$$
\mbox{\bf Ф}{\mathbf p}\approx\mathbf{u}
$$
или
$$
\mbox{\bf Ф}{\mathbf p}+{\mathbf d}=\mathbf{u}\,,
$$
где $d$~--- вектор-столбец невязок
$$
{\mathbf d}=\begin{pmatrix}
d_1\\[3pt]
d_2\\[3pt]
\cdots\\[3pt]
d_n
\end{pmatrix}
$$
 с компонентами
$$
d_j=\fr{j}{n}-\sum_{i=1}^K\widetilde
p_i\Phi_{ij}\,.
$$

Предположим, что $n\ge K$. В этом случае ранг случайной (в силу
случайности выборки $\emph{\textbf{x}} =(x_1,\ldots,x_n)$) матрицы
$\mbox{\bf Ф}$ с вероятностью единица равен $K$. Тогда из общей
теории метода наименьших квадратов для линейных моделей вытекает,
что решение
$$
\widehat{\,\mathbf p}^*=
\begin{pmatrix}
\hat p^*_1\\[3pt]
\hat p^*_2\\[3pt]
\cdots\\[3pt]
\hat p^*_K
\end{pmatrix}
$$
безусловной задачи наименьших квадратов
\begin{equation}
\arg\min_{{\mathbf p}}{\mathbf d}^{\top}{\mathbf d}=
\arg\min_{{\mathbf p}}\sum_{j=1}^n\left [\sum_{i=1}^K\widetilde
p_i\Phi_{ij}-\fr{j}{n}\right ]^2
\label{e8k}
\end{equation}
с вероятностью единица существует, единственно и имеет вид
\begin{equation}
\widehat{\mathbf p}^*=(\mbox{\bf Ф}^{\top}\mbox{\bf Ф})^{-1}\mbox{\bf
Ф}^{\top}\mathbf{u}\,.
\label{e9k}
\end{equation}
 При этом сумма компонент вектора
$\widehat{\mathbf p}^*$ отнюдь не обязана быть равной единице.
Обозначим
\begin{equation}
\gamma=1-\sum_{i=1}^K \hat p^*_i\,.\label{e10k}
\end{equation}
Пусть
${\mathbf a}$~--- $K$-мерный вектор-столбец, все компоненты
которого равны единице:
$$
{\mathbf a}^{\top}=(1,\,1,\,\ldots,1)\,.
$$
Тогда соотношение~(\ref{e10k}) можно записать в виде
$$
\gamma=1-{\mathbf a}^{\top}\widehat{\,\mathbf p}^*\,.
$$
Но в рассматриваемом случае
компоненты искомого вектора ${\mathbf p}$, будучи вероятностями в
дискретном распределении, связаны очевидным условием
\begin{equation}
\widetilde p_1+\ldots+\widetilde p_K=1\,.
\label{e11k}
\end{equation}
В терминах вектора
${\mathbf a}$ условие~(\ref{e11k}) запишется как
\begin{equation}
{\mathbf a}^{\top}{\mathbf p}=1\,.
\label{e12k}
\end{equation}
Хорошо известно, что если ранг
матрицы $\mbox{\bf Ф}$ равен~$K$, то решение $\widehat{\mathbf p}$
задачи~(\ref{e8k}) при условии~(\ref{e12k})
имеет вид
\begin{multline}
\widehat{\,\mathbf p}=\widehat{\,\mathbf p}^*+{}\\
{}+\left (\mbox{\bf Ф}^{\top}\mbox{\bf
Ф}\right )^{-1}{\mathbf a}\left [{\mathbf a}^{\top}\left (\mbox{\bf
Ф}^{\top}\mbox{\bf Ф}\right )^{-1}{\mathbf a}\right ]^{-1}\left (1-{\mathbf
a}^{\top}\widehat{\,\mathbf p}^*\right )
\label{e13k}
\end{multline}
(см., например, с.~85--86 в~\cite{3k}),
где вектор $\widehat{\,\mathbf p}^*$ определен в~(\ref{e9k}).
Несложно видеть, что выражение ${\mathbf a}^{\top}(\mbox{\bf
Ф}^{\top}\mbox{\bf Ф})^{-1}{\mathbf a}$ равно сумме всех элементов
матрицы $(\mbox{\bf Ф}^{\top}\mbox{\bf Ф})^{-1}$. Обозначим эту
сумму $S$. Тогда с учетом~(\ref{e10k}) соотношение~(\ref{e13k})
примет вид
\begin{equation}
\widehat{\,\mathbf p}=\widehat{\,\mathbf
p}^*+\fr{\gamma}{S}\left (\mbox{\bf Ф}^{\top}\mbox{\bf
Ф}\right )^{-1}{\mathbf{a}}\,.\label{e14k}
\end{equation}
Несложно видеть, что $i$-я
компонента $s_i$ вектор-столбца  $(\mbox{\bf Ф}^{\top}\mbox{\bf
Ф})^{-1}{\mathbf a}$ равна сумме всех элементов $i$-й строки
матрицы $(\mbox{\bf Ф}^{\top}\mbox{\bf Ф})^{-1}$. Таким образом,
из~(\ref{e14k}) окончательно получается, что
\begin{equation}
\widehat p_i=\hat p^*_i+\fr{\gamma s_i}{S}\,,\quad i=1,\ldots,K\,.
\label{e15k}
\end{equation}

Условие $n\ge K$, упрощающее вычисления, определяет выбор числа
$k_a+1$ возможных значений параметров сдвига и числа $k_{\sigma}$
возможных значений параметров масштаба компонент исходной смеси~(\ref{e1k}).
А именно: чтобы обеспечить существование единственного
решения $\widehat{\mathbf{p}}$ условной задачи наименьших
квадратов, параметры сетки $k_a$ и $k_{\sigma}$ должны быть
связаны с объемом выборки $n$ соотношением
$$
(k_a+1)k_{\sigma}\le n\,.
$$

Случай $n<K$ нуждается в более тщательном анализе, поскольку в
таком случае оценки наименьших квадратов определены неоднозначно.

Если оценки $\widehat p_i$ удовлетворяют неравенствам
\begin{equation}
0\le\widehat p_i\le 1
\label{e16k}
\end{equation}
для всех $i=1,\ldots,k$, то задача решена. Однако
может случиться так, что некоторые из последних неравенств не
реализуются на полученном векторе. В таком случае можно идти
несколькими путями.

Во-первых, в работе~\cite{4k} показано, что задачу наименьших квадратов
с ограничениями типа неравенств можно свести к задаче
последовательного решения нескольких задач наименьших квадратов
без ограничений. Вытекающий из этого результата алгоритм решения
рассматриваемой задачи таков.

Если для некоторых $i=1,\ldots,k$ соотношения~(\ref{e16k}) не выполнены,
то минимум функции
$$ SS({\mathbf p})=\sum_{j=1}^n\left [\sum_{i=1}^K\widetilde
p_i\Phi_{ij}-\fr{j}{n}\right ]^2
$$
на множестве
$$
\mathcal{P}=[0\,,1]^K\bigcap \left \{ {\mathbf p}:\, \widetilde
p_1+\ldots+\widetilde p_K=1\right \}
$$
достигается в одной из граничных
точек, в которых хотя бы для одного $i$ выполнено равенство
$p_i=0$ (напомним, что $\widehat p_1+\ldots+\widehat p_K=1$ по
построению). Пусть $I$~--- некоторое подмножество множества
$\{1,2,\ldots,K\}$. Как известно, всего таких подмножеств $2^K$.
Для каждого такого подмножества $I$ можно найти $\widehat{\mathbf
p}_I$~--- точку минимума функции
$$
SS_I({\mathbf
p_I})=\sum_{j=1}^n\left [\sum_{i\in I}\widetilde
p_i\Phi_{ij}-\fr{j}{n}\right ]^2
$$
при условии ${\mathbf
p}_I^{\top}{\mathbf p}_I=1$ по формулам, аналогичным~(\ref{e14k}) и~(\ref{e15k})
(здесь $\mathbf{p}_I$~--- это вектор $\mathbf{p}$, компоненты
которого с номерами, не попавшими в множество $I$, равны нулю).
Если при этом $\widehat{\mathbf p}_I\in {\mathcal P}$, то
вычисляется значение $SS_I(\widehat{\mathbf p}_I)$. Перебрав все
возможные подмножества $I$, можно найти решение исходной задачи~---
точку
$$
\widehat{\mathbf p}=\arg\min_{I}SS_I\left (\widehat{\mathbf
p}_I\right )\,.
$$
Однако при таком подходе требуется решить $2^K$ задач
наименьших квадратов с линейным ограничением, что при больших $K$
занимает чрезвычайно много времени.

Во-вторых, в работе~\cite{5k} показано, что задачу наименьших квадратов
с ограничениями типа неравенств можно свести к задаче
квадратичного\linebreak
программирования, и предложена итерационная
процедура~--- модифицированный симплекс-метод~--- для ее решения.
Эта процедура реализована в виде встроенного средства Optimization
Toolbox системы MATLAB и использована, в частности, в работе~\cite{6k}
для решения исходной задачи наименьших квад\-ра\-тов с ограничениями
типа неравенств и равенства. Однако при этом объем вычислений
трудно оценить, а статистические свойства оценок, получаемых при
таком подходе, чрезвычайно трудно исследовать.

Наконец, исходную задачу наименьших квадратов с ограничениями типа
неравенств и равенства можно заменить приближенной, накинув
конечную сетку и на множество возможных значений вектора
$(p_1,\ldots,p_k)$, сведя таким образом решение задачи к простому
перебору.

\section{Разделение конечных смесей вероятностных распределений
с~фиксированными компонентами при помощи метода наименьших
модулей}

\subsection{Решение в смысле минимума sup-нормы}

Соотношение~(\ref{e7k}) допускает различные конкретизации, связанные с
различными нормами невязки. Решение задачи по методу наименьших
квадратов, рассмотренное в предыдущем разделе, связано с
минимизацией обычной евклидовой нормы невязки.

В данном разделе будет рассмотрено решение, связанное с
минимизацией sup-нормы вектора невязок
$$
{\mathbf d}=
\begin{pmatrix}
d_1\\[3pt]
d_2\\[3pt]
\cdots\\[3pt]
d_n \end{pmatrix}\,,
$$
где, как и ранее,
$$
d_j=\fr{j}{n}-\sum_{i=1}^K\widetilde p_i\Phi_{ij}\,.
$$
Будем искать решение задачи
\begin{equation}
{\mathbf p}^*=\arg\min_{\mathbf
p}\max_{1\le j\le n}|d_j|\,.
\label{e17k}
\end{equation}
Как известно, задача~(\ref{e17k})
может быть сведена (см., например, с.~91 в~\cite{7k} и с.~137 в~\cite{8k}) к
задаче линейного программирования вида
\begin{equation}
f({\mathbf p},\theta)=\theta\longrightarrow\min_{({\mathbf
p},\theta)\in{\mathcal Q'}}\,;
\label{e18k}\\
\end{equation}
\begin{multline}
{\mathcal
Q'}=\bigg\{({\mathbf p},{\mathbf \theta})\in\r_+^{K+1}:\
\widetilde p_1+\ldots+\widetilde p_K\ge 1\,;\\
\widetilde p_1+\ldots+\widetilde p_K \le 1\,;
\theta\ge\sum_{i=1}^K
\widetilde p_i\Phi_{ij}-\fr{j}{n}\,;\\
\theta \ge-\sum_{i=1}^K
\widetilde p_i\Phi_{ij}+\fr{j}{n},\
j=1,\ldots,n\bigg\}\,. \label{e19k}
\end{multline}
Последняя задача, как известно,
решается с помощью стандартного симплекс-метода, заключающегося в
направленном переборе угловых точек множества ${\mathcal Q'}$
(см., например, гл.~1 в~\cite{7k}).

\subsection{Решение в смысле минимума L$_1$-нормы}

\markright{\underline{\it\centerline{Фиксированные компоненты:
метод минимума L$_1$-нормы}}}{}

В данном разделе будет рассмотрено решение, связанное с
минимизацией $L_1$-нормы вектора невязок ${\mathbf d}$. Другими
словами, будем искать решение задачи наименьших модулей
\begin{equation}
{\mathbf p}^{\star}=\arg\min_{\mathbf p}\sum_{j=1}^n|d_j|\,.
\label{e20k}
\end{equation}
Обозначим $\vec{\,\theta}=(\theta_1,\ldots,\theta_n)^{\top}$. Как
известно, задача~(\ref{e20k}) может быть сведена (см., например, с.~91
в~\cite{7k} и с.~137 в~\cite{8k}) к задаче линейного программирования вида
\begin{equation}
f({\mathbf p},\vec{\,\theta})=
\sum_{j=1}^n\theta_j\longrightarrow\min_{({\mathbf
p},{\vec{\,\theta}})\in{\mathcal Q''}}\,;
\label{e21k}
\end{equation}
\begin{multline}
{\mathcal
Q''}=\bigg\{({\mathbf p},{\vec{\,\theta}})\in\r_+^{K+n}:\
\widetilde p_1+\ldots+\widetilde p_K\ge1;\\
\widetilde p_1+\ldots+\widetilde p_K \le 1\,;
\theta_j\ge\sum_{i=1}^K
\widetilde p_i\Phi_{ij}-\fr{j}{n}\,;\\
\theta_j  \ge-\sum_{i=1}^K
\widetilde p_i\Phi_{ij}+\fr{j}{n},\
j=1,\ldots,n\bigg\}\,.\label{e22k}
\end{multline}
Последняя задача также решается с
помощью стандартного симплекс-метода, заключающегося в
направленном переборе угловых точек множества ${\mathcal Q''}$
(см., например, гл.~1 в~\cite{7k}).

Вычислительные реализации задач линейного программирования~(\ref{e18k}),
(\ref{e19k}) и~(\ref{e21k}), (\ref{e22k}) в популярных системах MATHCAD и MATLAB обладают
намного более высоким быстродействием, нежели вычислительные
реализации EM-алгоритма.

С содержательной точки зрения рассмотренные в данном разделе
задачи построения оценок вектора весов $\mathbf p$ вполне
равноправны с задачей, рассмотренной в предыдущем разделе. Более
того, как известно, оценки наименьших модулей более устойчивы по
отношению к наличию резко выделяющихся наблюдений, нежели оценки
наименьших квадратов. Поэтому оценки минимума sup-нормы и минимума
$L_1$-нормы в определенном смысле предпочтительнее оценок
наименьших квадратов.

Конечно, рассмотренные методы дают лишь приближенное решение
задачи разделения смесей. Однако при их реализации удается
избежать итерационных процедур типа ЕМ-алгоритма для поиска
решений экстремальных задач. Решение, полученное такими методами,
вполне можно использовать как начальное приближение для
ЕМ-алгоритма или его модификаций с целью последующего получения
более точного решения.

При этом, имея решение~(\ref{e15k}) задачи разделения смесей, начальное
приближение ЕМ-алгоритма можно выбирать следующим образом.
Упорядочим полученные оценки весов $\widehat p_1, \widehat
p_2,\ldots,\widehat p_K$ по убыванию и получим набор $\widehat
p_{i_1}, \widehat p_{i_2},\ldots,\widehat p_{i_K}$ такой, что
$\widehat p_{i_1}\ge\widehat p_{i_2}\ge\ldots\ge\widehat p_{i_K}$.
Выберем порог $\delta>0$ из тех соображений, что на практике, если
$p_{i_j}<\delta$, то будем считать, что компонента смеси,
соответствующая весу $p_{i_j}$, отсутствует. Положим
$$
k=\max\{j:\,p_{i_j}\ge\delta\}\,.
$$
Тогда в качестве начального
приближения для ЕМ-алгоритма возьмем параметры $a_{i_j}$ и
$\sigma^2_{i_j}$, соответствующие весам $\widehat p_{i_1},
\widehat p_{i_2},\ldots,\widehat p_{i_k}$. Особо следует отметить,
что при таком подходе число компонент смеси определяется
автоматически по заданному порогу пренебрежимости $\delta$.

Обратим внимание, что в рамках рассмотренного в данном разделе
подхода возможно классическое детерминистическое истолкование
точности в отношении приближения параметров $a_i$ и $\sigma_i$,
при котором точность характеризуется {\it одним} числом,\linebreak
задаваемым шагом дискретной сетки, на\-ки\-ды\-ва-\linebreak емой на множества
значений указанных парамет-\linebreak ров. В отношении же весовых параметров
$p_1,\ldots,p_k$, чтобы охарактеризовать точность приближения,
приходится пользоваться статистическим истолкованием, при котором
одного числа недостаточно, а нужно задавать еще и надежность
вывода (коэффициент доверия или уровень значимости).

\section{Декомпозиция волатильности с~помощью метода скользящего
разделения смесей}

Возможности описанных выше сеточных методов разделения конечных
смесей нормальных законов будут проиллюстрированы на примере
решения задачи декомпозиции волатильности (т.\,е.\ задачи
разложения волатильности на компоненты) некоторых финансовых
индексов.

Общая схема теоретического решения этой проб\-ле\-мы выглядит
следующим образом (см.~\cite{2k}).
\begin{enumerate}[($i$)]
\item Асимптотический подход, основанный на предельных
теоремах для обобщенных дважды стохастических пуассоновских
процессов как моделей неоднородных хаотических случайных
блужданий, естественно приводит к заключению о том, что
аппроксимации для распределений (логарифмов) приращений\linebreak
процессов
эволюции финансовых индексов на интервалах времени умеренной длины
следует искать в виде общих сдвиг/масштабных смесей нормальных
законов. Более \mbox{того}, при этом смешивающий закон определяется
накопленной (интегральной) интенсивностью потоков соответствующих
информативных событий (элементарных скачков,\linebreak <<тиков>>).
\item
Проблема статистической реконструкции распределений
приращений упомянутых процессов (или их логарифмов) сводится к
задаче статистического оценивания смешивающего распределения,
которое является параметром этой задачи.
\item
В самой общей постановке задача статистического
оценивания смешивающего распределения является некорректной, так
как общие сдвиг/масштабные смеси нормальных законов не являются
идентифицируемыми. Таким образом, в рамках общего принципа
регуляризации некорректных задач исходная проблема заменяется
задачей отыскания решения, наиболее близкого к истинному в классе
конечных дискретных сдвиг/масштабных смесей нормальных законов.
Эта <<редуцированная>> задача уже является корректной и имеет
единственное решение, так как семейство конечных дискретных
сдвиг/масштабных смесей нормальных законов идентифицируемо.
Поскольку сдвиг/масштабные смеси нормальных законов обладают
свойством устойчивости относительно смешивающего закона, эта
замена оправдана и регулярна. При этом, зная оценки устойчивости,
можно вычислить погрешности, образующиеся при замене исходной
задачи редуцированной. При упомянутой регуляризации происходит
автоматическое выделение типичных или более-менее устойчивых
структур в эволюции рассматриваемых сложных систем.
\item
Представление распределений (логарифмов) приращений
процессов эволюции фи\-нан\-со\-вых индексов в виде конечных
сдвиг/масштабных смесей нормальных законов естественно приводит к
многомерной интерпретации волатильности рассматрива\-емого процесса
и к возможности разложения волатильности на динамическую и
диффузионные компоненты. Действительно,\linebreak 
если функция распределения
логарифмического приращения $Z$ некоторого финансового индекса
имеет вид~(\ref{e1k}), то для нее справедливо пред\-став\-ле\-ние
\begin{multline*}
F(x)={\sf
P}(Z<x)=\sum_{j=1}^kp_j\Phi\Big(\fr{x-a_j}{\sigma_j}\Big)={}\\
{}={\sf
E}\Phi\Big(\fr{x-V}{U}\Big)\,,
\end{multline*}
где пара случайных величин $U,V$
имеет дискретное распределение
$$
{\sf P}((U,V)=(\sigma_j,a_j))=p_j\,,\quad  j=1,\ldots,k\,.
$$
Так что, как
продемонстрировано в книге~\cite{2k},
\begin{equation}
{\sf D}Z={\sf D}V+{\sf E}U^2\,,\label{e23k}
\end{equation}
причем
\begin{equation}
{\sf D}V=\sum_{j=1}^k(a_j-\overline
a)^2p_j\,,\quad {\sf E}U^2=\sum_{j=1}^kp_j\sigma_j^2\,,\label{e24k}
\end{equation}
где
$$
\overline a=\sum_{j=1}^ka_jp_j\,.
$$
Волатильность индекса
естественно отождествить с величиной ${\sf D}Z$ (или $\sqrt{{\sf
D}Z}$). При этом первое выражение в~(\ref{e24k}) зависит только от весов
$p_j$ и параметров положения (сдвига) $a_j$ компонент, и потому
характеризует ту часть волатильности, которая обусловлена наличием
локальных трендов, т.\,е.\ <<динамическую>> компоненту
волатильности, тогда как второе выражение в~(\ref{e24k}) зависит только от
весов $p_j$ и параметров масштаба (<<коэффициентов диффузии>>)
$\sigma_j$ компонент и потому характеризует <<чисто диффузионную>>
ее компоненту.

Если вспомнить традиционное одномерное представление о
волатильности как о стандартном отклонении приращения процесса, то
можно заметить, что разложение~(\ref{e23k}) уточняет это представление:
волатильность процесса представляет собой корень квадратный из
суммы двух компонент, первая из которых является характеристикой
разбросанности локальных трендов, а вторая характеризует диффузию
процесса. Если локальные тренды отсутствуют, то классическая
волатильность равна корню квадратному из взвешенной суммы
квадратов волатильностей\linebreak
 компонент, причем веса компонент
показывают важность соответствующей диффузионной компоненты.
\item
Статистические закономерности поведения
рассматриваемых процессов, формализованные в пункте~($i$),
изменяются во времени, результатом чего является отсутствие {\it
универсального} смешивающего закона. Таким образом, для изучения
динамики изменения статистических закономерностей в поведении
исследуемого хаотического процесса задача статистического
разделения конечных смесей нормальных законов должна быть
последовательно решена на интервалах времени, постоянно
сдвигающихся в направлении <<астрономического>> времени. Тем самым
параметры смесей (параметры сдвига (дрейфа), масштаба (диффузии),
а также соответствующие веса) оцениваются как функции времени. При
этом естественно возникают задачи, связанные как с выбором
подходящих методов оценивания параметров сдвига и масштаба, так и
с выбором оптимальных параметров вычислительных процедур,
реализующих эти методы: начального приближения, ширины скользящего
интервала (окна), правила остановки и~др.
\item
Наконец, для адекватной интерпретации результатов и
для идентификации фено\-ме\-нологически выделенных (статистически
оцененных) компонент, т.\,е.\ для адекватного\linebreak сопоставления
статистически оцененных компонент с реальными процессами или
явлениями, необходимо из многих возможных моделей выбрать наиболее
адекватную, например проверить, является выделенная динамическая
компонента волатильности статистически значимой или нет.
\end{enumerate}

\section{Применение сеточных методов разделения смесей для~декомпозиции 
волатильности~конкретных финансовых~индексов}

В качестве исходных данных использовались минутные логарифмические
приращения четырех биржевых индексов: AMEX, NASDAQ 100, NIKKEI 225
и CAC~40. В силу того что временн$\acute{\mbox{ы}}$е ряды логарифмических
приращений лишены явной трендовой составляющей и их среднее близко
к нулю, была взята нулевая сетка по параметрам $a_{i}$. При этом
количество узлов сетки по параметрам $\sigma_{i}$ взято равным
$K=50$. Для параметров $\sigma_{i}$, $i=\overline{1,\,50}$, было
выбрано максимальное граничное значение, составляющее три
стандартных отклонения по всему временн$\acute{\mbox{о}}$му ряду. Таким образом,
исходная задача оценки тройки параметров $(a,\sigma,p)$ свелась к
задаче подбора коэффициентов $\tilde p_{i}$, где
$i=\overline{1,\,50}$. При этом ширина скользящего окна $n$ взята
равной $300$ отсчетам (что соответствует пяти часам). Задача
оценивания весов последовательно решается для каждого положения
скользящего окна по выборке (отрезку исходного ряда),
соответствующей данному положению окна.

Оценки весов $p_i$ искались как решения двух задач линейного
программирования, к которым сводятся задачи минимизации
$\sup$-нормы и $L_1$-нор\-мы невязки между эмпирической и
теоретической функциями распределения (см.\ разд.~4).

Для решения использовалась стандартная процедура {\sf linprog} пакета
MATLAB~7.0.

Результаты представлены на графиках, приведенных на рис.~\ref{f1k}--\ref{f4k}.
На каждом графике горизонтальная ось~--- это ось времени: каждая
точка на горизонтальной оси соответствует конкретному значению
правого конца скользящего интервала времени, по которому (т.\,е.\
по наблюдениям, попавшим в который) вычисляются оценки весов.
Вертикальная ось~--- это ось значений параметров $\sigma_j$. Веса
компонент смеси, соответствующих конкретным значениям параметров
$\sigma_j$, показаны оттенками серого цвета. Чем прямоугольничек
темнее, тем вес больше.

Для сравнения на рисунках также представлены результаты решения
аналогичной задачи с по\-мощью ЕМ-алгоритма. Этот алгоритм
реализован программой ZHPlot, разработанной Ю.\,В.~Жуковым.

\subsection{Декомпозиция волатильности индекса AMEX}

Американская фондовая биржа (American Stock Exchange, AMEX)~---
одна из крупнейших региональных бирж США, расположена в Нью-Йорке.
Ведет свою историю с 1911~г., когда нью-йоркские уличные
торговцы акциями объединились в ассоциацию New York Curb Market
Association. В 1953~г.\ получила нынешнее название. После обвала
рынка в 1987~г.\ биржа ужесточила правила торгов, подняв уровень
биржевой маржи (инструмент гарантийного обеспечения при торгах) и
установив максимально допустимый уровень падения цен на акции,
после которого торги прекращаются. В начале 1990-х~гг.\ AMEX
первой в мире ввела систему электронных торгов с использованием
беспроводных терминалов. В 1998~г.\ биржа была куплена NASDAQ,
однако в 2004~г.\ участники AMEX выкупили площадку и остаются до
сегодняшнего дня акционерами. Основной используемый индекс~--- AMEX
Composite~--- обобщенный индекс, отражающий совокупную рыночную
стоимость всех компонентов рынка AMEX (обыкновенных акций,
депозитарных расписок и~пр.) по отношению к совокупной рыночной
стоимости этого рынка по состоянию на 29.12.1995.

В нашем исследовании рассматриваются ежеминутные данные с
04.02.2008 по 03.03.2008, т.\,е.\ длина анализируемого временного
ряда превышает 7000 значений (отсчетов). Результаты декомпозиции
волатильности индекса АМЕХ по минутным данным представлены на рис.~\ref{f1k}.

\begin{figure*} %fig1 %[p]
\vspace*{1pt}
\begin{center}
\mbox{%
\epsfxsize=104.223mm
\epsfbox{kor-1.eps}
}
\end{center}
\vspace*{-9pt}
\Caption{Декомпозиция волатильности индекса AMEX сеточным методом
с минимизацией $\sup$-нормы невязки между эмпирической и
теоретической функциями распределения~(\textit{а}), сеточным методом с
минимизацией $L_1$-нормы невязки между эмпирической и
теоретической функциями распределения~(\textit{б}) и ЕМ-алгоритмом~(\textit{в})
\label{f1k}}
\end{figure*}

{\bf Экономическая интерпретация поведения волатильности индекса
АМЕХ}. Данный индекс является низковолатильным. Наиболее весомые
компоненты не принимают значения больше~1,5. Первый заметный рост
волатильности наблюдается в среду 06.02, когда рынок акций США
продемонстрировал самое резкое почти за год снижение во вторник на
фоне данных, указавших на значительный спад активности в секторе
услуг США и предупреждения рейтингового агентства Standard \&
Poor's о том, что оно может понизить кредитные рейтинги банков.
Опасения по поводу возможности рецессии в США оказали давление на
акции различных секторов~--- от энергетического до
телекоммуникационного. Основной тон на фондовом рынке задавали
данные Института менеджеров по поставкам об активности в секторе
услуг в США за январь. Индекс активности снизился до минимального
с октября 2001~г.\ уровня, усилив беспокойство по поводу
рецессии. В дальнейшем ситуация была достаточно стабильной, но
11.02 и 12.02 волатильность стала опять расти, потому что
американские фондовые рынки в понедельник 11.02 выросли благодаря
акциям компаний высокотехнологичного сектора, которые помогли
отыграть потери, понесенные после открытия торгов. Акции банков
упали из-за заявлений министров финансов стран Большой семерки,
которые посчитали, что волнения на финансовых рынках ударят по
темпам роста мировой экономики. В итоговом коммюнике,
опубликованном по завершении встречи министров финансов G7,
говорится, что с момента предыдущей встречи в октябре 2007~г.\
перспективы экономического роста в ведущих индустриальных державах
ухудшились, хотя основы и остаются надежными и американская
экономика все еще может избежать рецессии. Ралли на рынке нефти
благоприятно сказалось на акциях нефтяных компаний. Цена на нефть
резко выросла после того, как президент Венесуэлы Уго Чавес
пригрозил прекратить поставки нефти в Соединенные Штаты.
Интересная картина наблюдается 14.02, когда одна из компонент
остается на уровне близком к~0, а вторая начинает расти. Ее можно
соотнести с отраслевой (промышленной) компонентой волатильности,
потому что 14.02 наблюдался рост американских фондовых
индексов, которые поднялись благодаря хорошим данным о розничных
продажах в Соединенных Штатах. Министерство торговли в своем
отчете в среду 13.02 сообщило, что объем розничных продаж в январе
вырос на 0,3\%, превзойдя прогнозы аналитиков благодаря
неожиданному увеличению продаж автомобилей и росту цен на бензин.
Экономисты ожидали, что розничные продажи в США снизятся на 0,3\%.
За прошедший год объем розничных продаж увеличился на 3,9\%. Рост
розничных продаж снизил опасения инвесторов по поводу возможного
сокращения потребительских расходов в США.


\subsection{Декомпозиция волатильности индекса CAC~40}
\begin{figure*} %fig2
\vspace*{1pt}
\begin{center}
\mbox{%
\epsfxsize=105.017mm
\epsfbox{kor-2.eps}
}
\end{center}
\vspace*{-9pt}
\Caption{Декомпозиция волатильности индекса CAC~40 сеточным
методом с минимизацией $\sup$-нормы невязки между эмпирической и
теоретической функциями распределения~(\textit{а}), сеточным методом с
минимизацией $L_1$-нормы невязки между эмпирической и
теоретической функциями распределения~(\textit{б}) и ЕМ-алгоритмом~(\textit{в})
\label{f2k}}
\end{figure*}


Индекс CAC 40 (Campagnie des Agents de Change~--- Ассоциация
французских фондовых брокеров)~--- один из основных французских
индексов, включающий в листинг сорок крупнейших французских
корпораций. Индекс САС~40 вычисляется как среднее взвешенное по
капитализации значение цен акций 40~крупнейших компаний Франции,
акции которых торгуются на бирже Euronext Paris. Начальное
значение индекса~-- 1000~пунктов~--- было установлено 31.12.1987.
Начиная с 01.12.2003 при подсчете
капитализации учитываются лишь акции, находящиеся в свободном
обращении. Индекс CAC~40 вычисляется каждые 30~с в рабочие
дни биржи с 9:00 до 17:30 по центрально-европейскому времени.

В данной статье анализируются ежеминутные данные индекса CAC~40 в
промежутке между 04.02.2008 и 15.02.2008, т.\,е.\ анализируется
более 4500~значений. Результаты декомпозиции волатильности индекса
CAC~40 по минутным данным представлены на рис.~\ref{f2k}.

{\bf Экономическая интерпретация поведения волатильности индекса
CAC~40}. Рассмотрим факторы, влиявшие на всплески волатильности.
Четвертого февраля 2008~г.\ наблюдался спад волатильности. Это происходило
на фоне трех сессий роста основных европейских индексов благодаря
новостям из банковского и добывающего секторов. Но уже 05.02
волатильность пошла резко вверх из-за того, что бумаги добывающих
компаний, поднимавшиеся в последнее время, опустились на фоне
противоречивой динамики цен на металлы. Акции европейских банков
подешевели вслед за бумагами американских финансовых компаний,
которые снизились на торгах в США днем ранее. При этом компонента
высокой волатильности теряет свой вес. Седьмого февраля наблюдается
очередной взрыв волатильности. Это происходит из-за того, что
большинство европейских фондовых индексов в этот четверг упало под
давлением неблагоприятных прогнозов крупных европейских и
американских компаний, которые стали еще одним признаком
замедления темпов роста мировой экономики. Французский CAC~40
закрылся со снижением на 92,63~пункта, или на 1,91\%, на отметке
4723,80. Перед выходными волатильность стабилизируется, на
французском рынке наблюдается снижение индекса, несмотря на то что
большинство европейских рынков вы\-рос\-ли благодаря акциям сырьевых
компаний, подорожавшим на фоне роста цен на нефть и металлы. Но
это связано с падением акций Societe Generale SA, второго
крупнейшего банка во Франции, которые снизились на 3,5\% на фоне
новых обстоятельств расследования одной из крупнейших афер в
банковском секторе. При этом компонент волатильности с меньшими
весами становится больше. Сле\-ду\-ющий взлет волатильности
наблюдается 12.02, когда на европейских рынках произошел самый
значительный рост за прошедшие две недели благодаря предложению
инвестора Уоррена Баффетта (Warren Buffett) перестраховать
муниципальные облигации на общую сумму 800~млрд долл. Как
отмечают аналитики, решение такого влиятельного инвестора взять на
себя риски по страхованию облигаций вселило оптимизм в игроков на
фондовом рынке. Предложение Баффетта, в случае его принятия
указанными компаниями, смогло бы разрядить ситуацию на рынке и
снять угрозу серьезных потерь в банковском секторе. Рост цен на
медь, цинк и другие металлы в Лондоне благоприятно сказался на
котировках акций горнодобывающих компаний. В~дальнейшем
наблюдается стабилизация ситуации и небольшой рост индекса. Но
15.02 происходит очередной спад в экономике из-за усилившихся
опасений того, что банки и дальше будут нести серьезные потери, а
в американской экономике начнется рецессия. Таким образом,
подобный вид графика (скачкообразный) четко отражает тенденции на
французском фондовом рынке~--- фазы резких взлетов и падений,
которые чередуется с фазами невысокой волатильности.


\subsection{Декомпозиция волатильности индекса NASDAQ}

Индекс NASDAQ~--- индекс внебиржевого рынка, публикуемый
Национальной ассоциацией дилеров по ценным бумагам (National
Association of Securities Dealers, NASD) и основанный на ее
котировках. Сводный индекс NASDAQ (NASD Automated Quotations) строится на основе
взвешенной рыночной стоимости акций эмитентов,\linebreak специализирующихся
в области высоких технологий. Это означает, что ценная бумага
каждой компании влияет на индекс пропорционально рыночной
стоимости компании. Тор\-гов\-ля на первом в мире электронном рынке
NASDAQ началась 08.02.1971.\linebreak А~поскольку на NASDAQ котируются
акции не только Hi-Tech компаний, то возникла целая сис\-те\-ма
индексов, каждый из которых отражает ситуацию в соответствующей
отрасли экономики. Существует 13~различных индексов, в
основе которых лежат котировки ценных бумаг, торгуемых в
электронной системе NASDAQ. Индекс NASDAQ~100 составляют
крупнейшие нефинансовые американские и иностранные компании,
входящие в листинг NASDAQ,~--- производители компьютерного
<<железа>> и <<софта>>, телекоммуникационные компании, розничные и
оптовые торговые фирмы, а также компании, связанные с
биотехнологиями. Первоначально главным критерием попадания в
верхние строчки индекса был вес компании, выраженный в долларах
США,~--- рыночная капитализация, но в декабре 1998~г.\ была
проделана хитрая <<перебалансировка>>, по которой рыночная
капитализация каждой компании уже умножалась на некий
ежеквартально пересматриваемый весовой коэффициент. Результатом
этого новшества, по мнению наздаковцев, должна была стать более
совершенная диверсификация. В итоге, например, Microsoft уступает
Cisco Systems, хотя по капитализации Microsoft в несколько раз
превосходит последнюю.

Рассматриваются ежеминутные данные в промежутке с 16.07.07 по
22.08.07, т.\,е.\ анализируется более 10\,000 данных. Результаты
декомпозиции волатильности индекса NASDAQ 100 по минутным данным
представлены на рис.~\ref{f3k}.

{\bf Экономическая интерпретация поведения волатильности индекса
NASDAQ}. В промежуток вре\-мени с 16.07 по 24.07 наблюдается
некоторая стабильность на рынке. Но во вторник 25.07 американские
фондовые рынки рухнули под давлением разочаровавших квартальных
отчетов и признаков распространения кризиса ипотечного
кредитования на клиентов с высоким рейтингом. Вторник стал самым
плохим днем для американских фондовых рынков за прошедшие четыре
месяца. Основным фактором, спровоцировавшим такое резкое падение
индексов, стал квартальный отчет компании Countrywide Financial
Corp.\ (CFC), из которого стало понятно, что кризис в секторе
ипотечного кредитования продолжает разрастаться, и в категорию
неплательщиков стали попадать и клиенты с высоким рейтингом. Это
спровоцировало рост волатильности. А последующий рост цен на нефть
вызвал небольшой рост фондовых рынков, что обеспечило спад
волатильности. Следующий значительный подъем волатильности
наблюдается с 07.07, что объясняется ростом фондовых рынков США
благодаря скупке инвесторами акций финансовых компаний,
снизившихся в конце прошлой недели на фоне рос\-та опасений
касательно стабильности кредитных рынков. На следующий день акции
электроэнергетических, финансовых и нефтедобывающих компаний
возглавили рост на американских фондовых рынках после
опубликования доклада Федеральной резервной системы. Вслед за
подъемом волатильности, как и обычно, наблюдается ее стабилизация,
так как продолжается рост рынка, но он не является значительным.
Очередной ее всплеск можно наблюдать в среду 15.08. Во вторник
американские фондовые индексы испытали значительное снижение на
фоне вновь появившихся опасений относительно слабости кредитной
сферы, пред\-по\-ла\-га\-емо\-го снижения потребительских расходов и новых
свидетельств кризиса ипотечного рынка. NASDAQ потерял~1,7\%.
\begin{figure*} %fig3
\vspace*{1pt}
\begin{center}
\mbox{%
\epsfxsize=105.017mm
\epsfbox{kor-3.eps}
}
\end{center}
\vspace*{-9pt}
\Caption{Декомпозиция волатильности индекса NASDAQ сеточным
методом с минимизацией $\sup$-нормы невязки между эмпирической и
теоретической функциями распределения~(\textit{а}), сеточным методом с
минимизацией $L_1$-нормы невязки между эмпирической и
теоретической функциями распределения~(\textit{б}) и ЕМ-алгоритмом~(\textit{в})
\label{f3k}}
\end{figure*}


\begin{figure*} %fig4
\vspace*{1pt}
\begin{center}
\mbox{%
\epsfxsize=104.223mm
\epsfbox{kor-4.eps}
}
\end{center}
\vspace*{-9pt}
\Caption{Декомпозиция волатильности индекса NIKKEI сеточным
методом с минимизацией $\sup$-нормы невязки между эмпирической и
теоретической функциями распределения~(\textit{а}), сеточным методом с
минимизацией $L_1$-нормы невязки между эмпирической и
теоретической функциями распределения~(\textit{б}) и ЕМ-алгоритмом~(\textit{в})
\label{f4k}}
\end{figure*}


\subsection{Декомпозиция волатильности индекса NIKKEI}

Индекс NIKKEI 225~--- наиболее широко используемый и изучаемый
показатель на Японском фондовом рынке. Индекс NIKKEI~225 впервые
был опубликован в 1950~г.\ Токийской фондовой биржей под
названием TSE Adjusted Stock Price Average. С~1970~г.\ индекс
вычисляется японской газетой Nihon Keizai Shimbun. Новое название
индекса произошло от сокращенного названия газеты~--- Nikkei. Он
вычисляется как среднее взвешенное значение цен акций 225~компаний,
наиболее активно торгуемых в первой секции Токийской
фондовой биржи. Эти акции выбраны из 450~наиболее ликвидных
компаний. Они разделены на шесть отраслевых секторов и отражают
различные тенденции Японского рынка. Список компаний, охваченных
индексом NIKKEI~225, пересматривается как минимум раз в год, в
октябре.

Рассматриваются ежеминутные данные с 15.01.2008 по 13.02.2008, т.\,е.\
анализируется более 5500 данных. Результаты декомпозиции
волатильности индекса NIKKEI по минутным данным представлены на
рис.~\ref{f4k}.

{\bf Экономическая интерпретация поведения волатильности индекса
NIKKEI}. В начале можно наблюдать резкий взлет обеих компонент
волатильности, что связано с ожиданием негативной финансовой
отчетности американских банков, которая может еще сильнее ухудшить
перспективы самой крупной экономики в мире. Индекс токийской биржи
NIKKEI к завершению торгов снизился на 0,98\% до 13.972,63 пункта,
опустившись ниже отметки 14.000 пунктов впервые с ноября 2005~г.
Опасения по поводу японской экономики усилились после того,
как глава Банка Японии Тосихико Фукуи предупредил, что темп
экономического роста может замедлиться на некоторое время,
несмотря на продолжающееся развитие экономики. Рынок ждал
публикации финансовых результатов американского банка Citigroup
Inc позднее во вторник. Новый генеральный директор Citi Викрам
Пандит, вероятно, сообщит о значительном сокращении дивидендов, об
увеличении капитала как минимум на 10~млрд долл., о списании
20~млрд долл., а также о сокращении более 20\,000 рабочих
мест, сообщила газета Wall Street Journal в понедельник.
Дальнейший спад мировой экономики со\-про\-вож\-дал\-ся стабильной, но
довольной высокой волатильностью, а на фоне сильнейшей регрессии
на японской фондовой бирже, связанной с опасением замедления двух
крупнейших в мире экономик~--- США и Японии, выделились две
противоположные компоненты, одна из которых пошла вверх, а другая~---
вниз. К 23.01.08 сильное падение японских фондовых индексов,
продолжавшееся два предыду\-щих дня, в среду прекратилось, и
основные индикаторы вышли, наконец, в положительную зону, что
привело к тому, что самые значимые компоненты волатильности
сошлись в одну. В последующие дни происходил значительный рост
индексов как в США и Европе, так и в Японии, что происходило на
фоне объявления мер помощи экономике США, неожиданного улучшения
на американском рынке труда, а также благодаря ожиданиям
очередного сокращения процентных ставок Федеральной резервной
системой с целью избежать обращения крупнейшей в мире экономики в
рецессию. Рынки акций Азии росли вслед за Уолл-стрит после того,
как президент США Джордж Буш и лидеры Конгресса в четверг приняли
решение о выделении 150~млрд долл.\ для предоставления
налоговых льгот с целью поддержать экономику, которой грозит
рецессия и спад ВВП два и более кварталов подряд. Акции
банковского сектора оправились после новости о мошенничестве
трейдера Societe Generale, из-за которого компания потеряла 7~млрд
долл. Индекс токийской биржи NIKKEI вырос на 4,10\% до
13.629,16 пункта, продолжив вос\-ста\-нов\-ле\-ние после резкого падения в
начале недели.

Если рассматривать результаты, получающиеся сеточным методом с
минимизацией $\sup$-нормы, то в последующие дни волатильность в
среднем находится на одном и том же уровне. Ее колебания
происходят внутри одного дня. Но сеточным методом с минимизацией
$L_1$-нормы можно выделить две компоненты, которые движутся в
унисон, при этом одна компонента~--- высоковолатильная, ее значения
превосходят уровень~5, а значения второй компоненты
волатильности флуктуируют вокруг уровня~2.


\section{Выводы}

Из результатов анализа волатильности индексов AMEX, CAC~40, NASDAQ
и NIKKEI, приведенных на рис.~\ref{f1k}--\ref{f4k}, можно сделать несколько
выводов.
\begin{enumerate}[1.]
\item Наличие нескольких выделенных компонент волатильности,
по-видимому, соответствует наличию различных уровней активности
разных секторов экономики в разные моменты времени.
\item
Обращает на себя внимание, что из 50 возможных значений
диффузионной компоненты волатильности значимыми (ненулевыми)
оказываются веса лишь двух--четырех компонент. Остальные
автоматически обнуляются.
\item
Результат применения ЕМ-алгоритма оказывается практически
аналогичен результату применения сеточного метода с минимизацией
$L_1$-нор\-мы.
\item
Минимизация $\sup$-нормы приводит к выделению меньшего числа
компонент волатильности, нежели минимизация $L_1$-нормы. Это,\linebreak
по-ви\-ди\-мо\-му, означает, что $L_1$-норма более чувствительна к
расхождениям между данными и моделью. В то же время минимизация
$L_1$-нормы дает более четкие <<портреты волатильности>>, менее
зашумленные по сравнению с теми, что получены минимизацией
$\sup$-нормы.
\end{enumerate}

\bigskip
Авторы считают своим приятным долгом выразить признательность Ф.\,П.~Васильеву
за полезное обсуждение вопросов, затронутых в данной\linebreak статье.

%Литература

{\small\frenchspacing
{%\baselineskip=10.8pt
\addcontentsline{toc}{section}{Литература}
\begin{thebibliography}{99}

\bibitem{1k}
\Au{Королёв В.\,Ю.}
ЕМ-алгоритм, его модификации и их
применение к задаче разделения смесей вероятностных распределений.
Теоретический обзор.~--- М.: Изд-во ИПИРАН, 2007.

\bibitem{2k}
\Au{Королёв В.\,Ю.}
Вероятностно-статистический анализ
хаотических процессов с помощью смешанных гауссовских моделей.
Декомпозиция волатильности финансовых индексов и турбулентной
плазмы.~--- М.: Изд-во ИПИРАН, 2007.

\bibitem{3k}
\Au{Себер Дж.}
Линейный регрессионный анализ.~--- М.: Мир, 1980.

\bibitem{4k}
\Au{Waterman M.\,S.}
A restricted least squares problem~//
Technometrics, 1974. Vol.~16. No.\,1. P.~135--136.

\bibitem{5k}
\Au{Judge~G.\,G., Takayama~T.}
Inequality restrictions in
regression analysis~// J.\ of American Statistical
Association, 1966. Vol.~61. No.\,1. P.~166--181.

\bibitem{6k}
\Au{Королёв В.\,Ю., Ломской~В.\,А., Пресняков~Н.\,Н., Рэй~М.}
Анализ компонент волатильности с помощью метода скользящего
разделения смесей~// Системы и средства информатики. Спец. вып.~---
М.: ИПИРАН, 2005. С.~180--206.

\bibitem{7k}
\Au{Васильев Ф.\,П., Иваницкий~А.\,Ю.}
Линейное программирование.~--- М.: Факториал Пресс, 2003.
\label{end\stat}

\bibitem{8k}
\Au{Ашманов С.\,А., Тимохов~А.\,В.}
Теория оптимизации в
задачах и упражнениях.~--- М.:  Наука, 1991.
\end{thebibliography}
}
}
\end{multicols}