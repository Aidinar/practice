\def\stat{grusho}


\def\tit{СУЩЕСТВОВАНИЕ СОСТОЯТЕЛЬНЫХ ПОСЛЕДОВАТЕЛЬНОСТЕЙ 
СТАТИСТИЧЕСКИХ\\ КРИТЕРИЕВ В ДИСКРЕТНЫХ 
СТАТИСТИЧЕСКИХ\\ ЗАДАЧАХ ПРИ СЛОЖНОЙ НУЛЕВОЙ 
ГИПОТЕЗЕ$^*$}

\def\titkol{Существование состоятельных последовательностей статистических критериев}
% В ДИСКРЕТНЫХ 
%СТАТИСТИЧЕСКИХ ЗАДАЧАХ ПРИ СЛОЖНОЙ НУЛЕВОЙ  ГИПОТЕЗЕ

\def\autkol{А.\,А.~Грушо, Е.\,Е.~Тимонина, В.\,М.~Ченцов}
\def\aut{А.\,А.~Грушо$^1$, Е.\,Е.~Тимонина$^2$, В.\,М.~Ченцов$^3$}

\titel{\tit}{\aut}{\autkol}{\titkol}

{\renewcommand{\thefootnote}{\fnsymbol{footnote}}\footnotetext[1]
{Работа выполнена при поддержке РФФИ, проект 07-01-00484.}}

\renewcommand{\thefootnote}{\arabic{footnote}}
\footnotetext[1]{Российский государственный гуманитарный университет, Институт информационных наук и технологий 
безопасности; Московский государственный университет им.~М.\,В.~Ломоносова, факультет вычислительной 
математики и кибернетики, grusho@yandex.ru}
\footnotetext[2]{Российский государственный гуманитарный университет, Институт информационных наук и технологий 
безопасности,\\ eltimon@yandex.ru}
\footnotetext[3]{Институт проблем информатики Российской академии наук, ipiran@ipiran.ru}
  

\Abst{В работе рассматривается задача существования состоятельной последовательности 
критериев при проверке сложной гипотезы против сложной альтернативы в 
последовательности конечных пространств. В тех случаях, когда последовательность 
пространств представляет собой декартово произведение конечного множества и 
вероятностные меры на этих пространствах согласованы, удается найти достаточные условия 
существования состоятельной последовательности критериев в терминах топологических 
свойств множеств, покрывающих носитель доминирующей меры для класса предельных мер 
из нулевой гипотезы. При дополнительных условиях удается отказаться от требования 
доминируемости класса предельных мер из нулевой гипотезы и равномерной 
ограниченности плотностей.}

\KW{состоятельная последовательность критериев; сложная гипотеза против сложной альтернативы;
конечные пространства; вероятностные меры; достаточные условия}

      \vskip 24pt plus 9pt minus 6pt

      \thispagestyle{headings}

      \begin{multicols}{2}

      \label{st\stat}

    По мере проникновения компьютерных и компьютеризированных систем 
во все сферы человеческой деятельности актуальность проблем компьютерной 
безопасности и защиты компьютерных систем становится все более 
актуальной. Очень важно понимать, что такое защищенность компьютерной 
системы и на чем должна быть основана уверенность, что требуемая 
защищенность имеется. Исторический опыт показывает, что формализацию как 
определения защищенности, так и обоснования защищенности следует искать 
математическими методами и в рамках математических моделей.
    
    Авторами ранее~\cite{5gr, 6gr} построены примеры доказуемо 
защищенных компьютерных систем. Ими разработаны методы контроля 
информационных потоков в компьютерных системах. Проведены исследования 
свойств скрытых от контроля информационных потоков в распределенных 
компьютерных системах. Доказана возможность скрытой передачи 
информации через такие сильные средства защиты, как межсетевые экраны и 
криптография. Эти результаты нашли практическое подтверждение. С 
помощью доказательства утверждений о несуществовании состоятельных 
последовательностей статистических критериев~\cite{1gr, 2gr} для выяв\-ления 
признаков искомых каналов удается дока\-зать <<невидимость>> некоторых 
информационных по\-токов. Обратные утверждения, напротив, свидетельствуют 
о том, что при определенных условиях <<невидимых>> потоков не существует. 
Такие обратные утверждения доказываются в данной\linebreak статье.
    

    Пусть задано конечное множество  $X =$\linebreak $=\;\{x_1,\ldots,x_m\}$. Через 
$X^\infty$ обозначим множество бесконечных последовательностей, где 
каж\-дый элемент последовательности принадлежит $X$. Пусть $\cal{A}$~--- 
$\sigma$-алгебра, порожденная цилиндрическими подмножествами $X^\infty$. 
Рассмотрим два семейства вероятностных мер на измеримом пространстве 
$\{X^\infty,\,\cal{A}\}$:
    $\{P_\lambda ,\ \lambda\in \Lambda\}$ и $\{P_\theta , \ \theta\in \Theta\}$.


    Обозначим через $P_{\lambda,n}$ и $P_{\theta,n}$ проекции введенных 
вероятностных мер на первые $n$ координат последовательностей из 
$X^\infty$. Для  каждого $n$ рассмотрим задачу проверки сложной 
статистической гипотезы $H_{0,n}:  \{P_{\lambda,n}, \ \lambda\in \Lambda\}$ 
против сложной альтернативы $H_{1,n}: \{P_{\theta,n},\ \theta\in\Theta\}$.  Для 
каждого $n$ критерий уровня значимости $\alpha$ описывается критическим 
множеством $S_n$, $P_{\lambda,n}(S_n)\leq \alpha$, $\lambda\in\Lambda$, и 
мощностью критерия $W_n(\theta ) =P_{\theta, n}(S_n)$.
    
    Последовательность статистических критериев с критическими 
множествами $S_n$ называется состоятельной~\cite{3gr}, если для каждого 
$\alpha \in (0;\,1]$ мощность критерия $W_n(\theta )\rightarrow 1$ для каждого 
$\theta\in\Theta$. При этом можно считать, что $\alpha\rightarrow 0$ при 
$n\rightarrow\infty$.
    
    Последовательность критериев с критическими множествами $S_n$ 
называется слабо состоятельной, если $P_{\lambda,n}(S_n)\rightarrow 0$ для 
каждого $\lambda\in\Lambda$  и $P_{\theta,n}(S_n)\rightarrow 1$ для каждого 
$\theta\in\Theta$.
    
    Ясно, что состоятельная последовательность критериев является слабо 
состоятельной, но не наоборот. Вопрос о существовании состоятельной 
последовательности критериев для простой нулевой гипотезы изучался 
в~\cite{1gr}. Вопрос о несуществовании состоятельной последовательности 
критериев исследовался в~\cite{6gr, 2gr}. В данной работе предпринята 
попытка обобщить некоторые результаты этих работ на случай сложной 
нулевой гипотезы. 
    
    Будем предполагать, что на $\{X^\infty,\,\cal{A}\}$ определена еще одна 
вероятностная мера $P_0$, которая доминирует семейство распределений 
$\{P_\lambda ,\ \lambda\in\Lambda\}$. Тогда для каждого $\lambda\in\Lambda$ 
существует плотность $p_\lambda$ такая, что $\forall B\in \cal{A}$
    $$
    P_\lambda (B) = \int\limits_B p_\lambda\,dP_0\,.
    $$
    
    Будем считать, что существует $C>0$ такое, что для любого $\lambda$ 
существует вариант плотности $p_\lambda$ такой, что $p_\lambda \leq C$. Отсюда 
следует, что если $A\in \cal{A}$ и $P_0 (A)=1$, то $P_\lambda (A)=1$. 
Действительно,
    $$
    P_\lambda (A) = \int\limits_A P_\lambda\,dP_0 = 
\int\limits_{X^\infty}p_\lambda\,dP_0 -\int\limits_{\overline{A}} p_\lambda\,dP_0 
=1\,,
    $$
так как
$$
\int\limits_{\overline{A}} p_\lambda\,dP_0\leq 
C\int\limits_{\overline{A}}dP_0=0\,.
$$

    Для каждого $\lambda\in\Lambda$, $D_n\subseteq X^n$
\begin{multline*}
    P_{\lambda,n} (D_n) = P_\lambda \left ( D_n\times X^\infty \right ) = {}\\
    {}=
\!\int\limits_{D_n\times X^\infty}\! p_\lambda\,dP_0\leq CP_0\left ( D_n\times 
X^\infty\right ) =
    C P_{0,n}(D_n)\,.\hspace*{-1.038pt}
    \end{multline*}
Будем также рассматривать дискретную топологию в $X$ и Тихоновское 
произведение в $X^\infty$~\cite{4gr}. 

    \medskip
    
        \noindent
    \textbf{Теорема 1.} \textit{Пусть существует замкнутое в Тихоновском 
произведении множество $A$ такое, что $P_0(A)=1$ и $\forall \theta\in\Theta$ 
существует $A(\theta ) \in \cal{A}$  такое, что $A\cap A(\theta )=\phi$ и 
$P_\theta (A(\theta)) =1$. Тогда существует состоятельная 
последовательность критериев для проверки} $H_{0,n}$ \textit{против} $H_{1,n}$.
    
    
    \smallskip
    
    \noindent
    Д\,о\,к\,а\,з\,а\,т\,е\,л\,ь\,с\,т\,в\,о. В рассматриваемых условиях по 
теореме~1 из работы~\cite{1gr} существует состоятельная последовательность 
критериев для проверки $H^\prime_{0,n}:P_{0,n}$ против $H_{1,n}$ с 
критическими множествами $S_n$ такая, что $P_{0,n}(S_n)\rightarrow 0$, $n\rightarrow 
\infty$, и $\forall \theta\in\Theta$ $P_{\theta,n}(S_n)\rightarrow 1$. Для каждого 
$\alpha\in (0;\,1]$ существует~$N$ такое, что для всех $n\geq N$ имеем 
$P_{0,n}(S_n)\leq \alpha$. Тогда для всех $\lambda\in \Lambda$
    $$
P_{\lambda,n} (S_n) \leq C\alpha\,.
$$
\noindent  
    Поэтому последовательность критериев с критическими множествами 
$S_n$, $n=1$, 2,\ \ldots\ также является состоятельной для проверки $H_{0,n}$ 
против $H_{1,n}$. Теорема доказана. 
    \medskip
    
    \noindent
    \textbf{Теорема 2.} \textit{Пусть существует открытое в Тихоновском 
произведении множество $A$ такое, что $P_0(A) =1$ и $\forall \theta\in\Theta$ 
существует $A(\theta ) \in \cal{A}$, что $A\cap A(\theta ) =\phi$ и $P_\theta 
(A(\theta )) =1$. Тогда существует состоятельная последовательность 
критериев для проверки $H_{0,n}$ против $H_{1,n}$.}
    
    \smallskip
    
    \noindent
    Д\,о\,к\,а\,з\,а\,т\,е\,л\,ь\,с\,т\,в\,о. По теореме~2 из работы~\cite{1gr} в условиях 
рассматриваемой теоремы существует состоятельная последовательность 
критериев для проверки простой гипотезы $H^\prime_{0,n}:P_{0,n}$ против 
сложной альтернативы $H_{1,n}$ с критическими множествами $S_n$. Далее,
так же как в теореме~1, доказывается, что построенная последовательность 
критериев будет  состоятельной для проверки $H_{0,n}$ против $H_{1,n}$. 
Теорема доказана.
    \smallskip
    
    Откажемся теперь от требования доминируемости мер $P_\lambda$, 
$\lambda\in\Lambda$. Однако предположим, что существует замкнутое в 
Тихоновском произведении множество $A$ такое, что для $\forall 
\lambda\in\Lambda$ существует $A_\lambda \in \cal{A}$  такое, что $A_\lambda 
\subseteq A$ и $P_\lambda (A_\lambda )=1$. Тогда можно доказать следующую 
теорему.
    
    \medskip
    
    \noindent
    \textbf{Теорема 3.} \textit{Если существует замкнутое множество $A_1$ 
такое, что $A\cap A_1=\phi$ и $\forall\theta\in\Theta$ $P_\theta (A_1)=1$, то 
существует состоятельная последовательность критериев для проверки 
$H_{0,n}$ против $H_{1,n}$.}
    
    \smallskip
    
    \noindent
    Д\,о\,к\,а\,з\,а\,т\,е\,л\,ь\,с\,т\,в\,о. Из замкнутости множеств $A$ и $A_1$ 
следует, что существуют~\cite{5gr} невозрастающие последовательности 
цилиндрических множеств $I_n$, $n = 1$,\ 2,\ldots, и $I^\prime_n$, $n = 1$, 
2,\  \ldots\ такие, что
\begin{gather*}
I_n = S_n\times X^\infty\,,\quad S_n\subseteq X^\infty\\
A=\bigcap\limits_{n=1}^\infty I^\prime_n\,,\quad A_1=\bigcap\limits_{n=1}^\infty 
I_n\,.
\end{gather*}

    По условию $A\cap A_1=\phi$. Тогда
    $$
    \bigcap\limits_{n=1}^\infty \left ( I^\prime_n \cap I_n\right )=\phi\,.
$$

    Любое пересечение цилиндрических множеств $I^\prime_n\cap 
I_n=I_n^{\prime\prime}$~--- цилиндрическое множество или\linebreak пустое множество. 
Последовательность $I_n^{\prime\prime}$, $n =$\linebreak $=\;1,$ 2,\ \ldots цилиндрических 
множеств монотонно не возрастает и $\bigcap\limits_{n=1}^\infty 
I_n^{\prime\prime}=\phi$. Топологическое пространство $X$~--- компакт. Тогда 
из компактности Тихоновского произведения $X^\infty$ следует~\cite{4gr}, что 
существует~$N$ такое, что $\bigcap\limits_{n=1}^N I_n^{\prime\prime}=\phi$. Из 
условия невозрастания $I_n^{\prime\prime}$ следует, что $I_N^{\prime\prime} 
=\phi$. 
    
    Рассмотрим последовательность статистических критериев с 
критическими множествами $S_n$, $n = 1$, 2,\ \ldots Из определения $A$ 
следует, что $A\subseteq I_n^\prime$. Тогда $A\cap I_n=\phi$, т.\,е.\ 
$I_N\subseteq X^\infty \backslash A$. Отсюда следует, что для каждого $\lambda 
\in \Lambda$ при $A_\lambda \in \cal{A}$  и $P_\lambda (A_\lambda )=1$
    \begin{multline*}
    0 = P_\lambda \left ( X^\infty \backslash  A_\lambda \right ) \geq
    P_\lambda \left ( X^\infty\backslash A\right ) \geq P_\lambda \left (I_N\right ) ={}\\
    {}=
    P_\lambda \left ( S_N\times X^\infty\right ) = P_{\lambda,N}(S_N)\,.
    \end{multline*}

    Из условия теоремы так же, как в теореме~3, работы~\cite{1gr}
    $$
    W_N(\theta ) = P_{\theta, N}(S_N) = P_\theta (I_N)\geq P_\theta \left ( A_1\right ) 
=1\,.
    $$
    
    Теорема доказана.
    \medskip
    
    Проверить замкнутость или открытость множеств в пространстве 
$X^\infty$ можно конструктивно, представив соответствующее множество в 
виде\linebreak монотонно невозрастающей или монотонно неубывающей 
последовательности цилиндрических\linebreak множеств. Сложнее проверяется свойство 
доминируемости семейства вероятностных мер и существование равномерно 
ограниченных плотностей. В~условиях теоремы~3 удается избежать такой 
проверки. 
    
    В работе~\cite{2gr} получено следующее достаточное условие 
несуществования состоятельной последовательности критериев в нашей 
модели при $\vert\Lambda\vert =1$. Если для каждого $A\in \cal{A}$  такого, 
что $P_\lambda (A)=1$, существует $\theta\in\Theta$ такое, что $P_\theta >0$, то 
состоятельной последовательности критериев не существует. 
    
    Отметим, что в случае $\vert\Lambda\vert =1$ это условие также является 
необходимым. В самом деле, если состоятельной последовательности не 
существует, но существует $A\in\cal{A}$  такое, что $P_\lambda(A)=1$, и для 
каж\-до\-го $\theta\in \Theta$ $P_\theta (A)=0$, то рассмотрим критерий с 
критическим множеством $X^\infty\backslash A$. Этот критерий удовлетворяет 
следующим условиям:
    $$
    P_\lambda (X^\infty \backslash A) =0
    $$
и для каждого $\theta \in \Theta$
$$
P_\theta (X^\infty \backslash A) =1\,.
$$
    
    Тогда последовательность критериев с одним и тем же критическим 
множеством $X^\infty \backslash A$ является состоятельной, что противоречит 
предположению. 
    
    Для произвольного $\Lambda$ справедлива следующая теорема.
    \medskip
    
    \noindent
    \textbf{Теорема 4.} \textit{Если существует $\lambda \in\Lambda$ такое, 
что для каждого множества $A\in\cal{A}$ $P_\lambda (A)=1$  и существует 
$\theta\in\Theta$ такое, что $P_\theta (A)>0$, то не существует состоятельной 
последовательности критериев проверки сложных гипотез 
$H_{0,n}:\{P_{\lambda,n},\ \lambda \in\Lambda\}$  против сложных 
альтернатив} $H_{1,n}:\{P_{\theta,n},\ \theta\in\Theta\}$.
    \smallskip
    
    \noindent
    Д\,о\,к\,а\,з\,а\,т\,е\,л\,ь\,с\,т\,в\,о. Из условий теоремы и приведенных 
выше достаточных условий следует, что не существует состоятельной 
последовательности критериев для проверки простых гипотез 
$H^\prime_{0,n}:P_{\lambda,n}$ против сложных альтернатив 
$H_{1,n}:\{P_{\theta,n},\ \theta\in\Theta\}$. Тогда, как это следует из 
определения, не существует слабо состоятельной последовательности 
критериев для проверки $H_{0,n}$ против $H_{1,n}$. Это, в свою очередь, 
означает, что не существует состоятельной последовательности критериев для 
проверки $H_{0,n}$ против $H_{1,n}$. Теорема доказана. 
    
    
{\small\frenchspacing
{%\baselineskip=10.8pt
\addcontentsline{toc}{section}{Литература}
\begin{thebibliography}{9}    

\bibitem{5gr} %1
\Au{Grusho~A., Kniazev~A., Timonina~E.}
Detection of illegal information flow~// Proceedings of 3rd International 
Workshop on Mathematical Methods, Models, and Architectures for Computer 
Network Security, MMM-ACNS 2005.~--- St.~Petersburg: Springer, 2005.  
LNCS~3685.  Р.~235--244.

\bibitem{6gr} %2
\Au{Grusho A., Grebnev~N., Timonina~E.}
Covert channel invisibility theorem~// Proceedings of 4th International 
Conference on Mathematical Methods, Models, and Architectures for Computer 
Network Security, MMM-ACNS 2007.~--- St.~Petersburg: Springer, 2007. Р.~187--196.

\bibitem{1gr} %3
\Au{Грушо~A.\,A., Тимонина~E.\,E.}
Некоторые связи между дискретными статистическими задачами и 
свойствами вероятностных мер на топологических пространствах~// 
Дискретная математика, 2006.  Т.~18. №\,4. С.~128--135.

\bibitem{2gr} %4
\Au{Грушо~А., Грушо~Н., Тимонина~E.}
Теоремы о несуществовании состоятельных последовательностей 
критериев в некоторых дискретных задачах~// Дискретная математика, 
2008  (в печати). Т.~20. №\,2.

\bibitem{3gr} %5
\Au{Леман~Е.}
Проверка статистических гипотез.~--- М.: Наука,  1964.
\label{end\stat}

\bibitem{4gr} %6
\Au{Прохоров Ю.\,В., Розанов Ю.\,А.}
Теория вероятностей.~--- М.: Наука, 1973.


\end{thebibliography}
}
}
\end{multicols}    
 
 
 
 