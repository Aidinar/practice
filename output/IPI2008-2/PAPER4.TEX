
\def\N{{\cal N}}



\newcommand{\suml}[2]{\sum\limits_{#1}^{#2}}   %сумма
\def\sumi{\sum\limits_{i=1}^n}   %сумма по i=1 до n
\newcommand{\maxl}[1]{\max\limits_{#1}}
\newcommand{\minl}[1]{\min\limits_{#1}}
\def\limn{\lim\limits_{n\to\infty}}
\newcommand{\liml}[1]{\lim\limits_{#1}}
\newcommand{\limi}[1]{\liminf\limits_{#1}}
\newcommand{\lims}[1]{\limsup\limits_{#1}}
\newcommand{\cupl}[2]{\bigcup\limits_{#1}^{#2}}


\newcommand{\rr}{{\mathbb {R}}^2}     % поле действительных чисел, плоскость
\newcommand{\naj}{{\mathbb{N}}}     % поле натуpальных чисел
%\newcommand{\rn}{\hbox{\aj R}^{\naj} }     % N-мерное пространство
\newcommand{\rn}{{\mathbb{R}}^N}     % N-мерное пространство


\def\P{\mathop{\bf P}}
\newcommand{\sqq}{\hbox{\vrule\vbox{\hrule\phantom{o}\hrule}\vrule}}

\def\stat{zah}


\def\tit{ОПТИМИЗАЦИЯ РАСПОЛОЖЕНИЯ СТАНЦИЙ ОБСЛУЖИВАНИЯ В ПРОСТРАНСТВЕ}
\def\titkol{Оптимизация расположения станций обслуживания в пространстве}
\def\autkol{Т.\,В.~Захарова}
\def\aut{Т.\,В.~Захарова$^1$}

\titel{\tit}{\aut}{\autkol}{\titkol}

%{\renewcommand{\thefootnote}{\fnsymbol{footnote}}\footnotetext[1]
%{Работа выполнена при
%поддержке РФФИ, гранты 08-01-00345, 08-01-00363,
%08-07-00152.}}

\renewcommand{\thefootnote}{\arabic{footnote}}
\footnotetext[1]{Московский государственный университет
им.\ М.\,В.~Ломоносова, факультет вычислительной математики и
кибернетики, lsa@cs.msu.su}

%\vspace*{24pt}

\Abst{В пространстве $\rn$ эффективным образом
располагаются станции обслуживания по критерию стационарного
среднего времени ожидания начала обслуживания.}

\KW{размещение станций обслуживания; критерий оптимальности; время
ожидания начала обслуживания}

      \vskip 24pt plus 9pt minus 6pt

      \thispagestyle{headings}

      \begin{multicols}{2}

      \label{st\stat}

\section{Введение}



В статье рассматривается неклассическая задача теории массового обслуживания. На обслуживание
поступают вызовы, распределенные в пространстве $\rn$. Каким образом разместить в этом
пространстве станции обслуживания?

Чаще всего рассматривается поток однородных требований, различающихся лишь моментами
поступления в систему. Спецификой изучаемого класса систем является довольно сложное фазовое
пространство системы, требующее учета информации о положении и числе обслуживающих приборов,
положении поступающих вызовов, их вероятностных характеристиках.

Такие модели систем массового обслуживания применяются для изучения
реальных систем, где обслуживание производится территориально расположенными объектами.
В связи с этим возникают определенные задачи оптимизации.

В данной статье решается задача оптимизации расположения станций обслуживания по критерию
стационарного среднего времени ожидания начала обслуживания для систем с дисциплиной
обслуживания FIFO (First In, First Out). Станции функционируют как независимые системы массового обслуживания типа
$M\vert G\vert 1$. Описываются свойства оптимальных размещений и приводятся алгоритмы построения
асимптотически оптимальных размещений.

\section{Постановка задачи}

В пространстве $\rn$ возникают требования в случайных точках $\xi_1,\xi_2,\dots,$ независимых и
одинаково распределенных с плотностью распределения $p$. Для обслуживания этих требований
имеется $n$ станций. Моменты поступления требований образуют пуассоновский поток с параметром
$\lambda$. Интенсивность входящего потока $\lambda$ изменяется с ростом чис\-ла станций $n$. В
случае, когда нужно подчеркнуть эту зависимость, параметр входящего потока будем обозначать
$\lambda(n)$.

\noindent
\textbf{Определение 1.} \textit{Размещением $n$ станций обслуживания в пространстве $\rn$ назовем
множество точек пространства $\{x_1,\ldots,x_n\}$, в которых они расположены.}

Обозначать размещение станций будем символом $x$, т.\,е.\ $x=\{x_1,\ldots,x_n\}$. Станцию
обслуживания и точку пространства, где она расположена, будем обозначать одним и тем же
символом.

\noindent
\textbf{Определение 2.} \textit{Зоной влияния станции $x_i$ назовем множество $C_i$ тех точек
пространства, для которых эта станция является ближайшей, т.\,е.}
$$
C_i=\{v\in\rn:|v-x_i|\leqslant|v-x_j|\,,\quad j=1,2,\ldots,n\}\,.
$$

Расстояние $|u-v|$ между точками $u$ и $v$ пространства $\rn$ задается чебышевской нормой:
$$ %\begin{align*}
\vert u- v\vert =\maxl{1\leqslant i\leqslant N}\left|u_i-v_i\right|\,,
$$
где $u=(u_1,\ldots,u_N);$ $v=(v_1,\ldots,v_N)$.
 %\end{align*}

Станции обслуживают заявки только из своих зон влияния. Обслуживание осуществляется прибором,
двигающимся только по прямой и с постоянной скоростью. При поступлении заявки прибор со
станции перемещается в точку вызова, заявка обслуживается прибором некоторое случайное время~$\eta$,
затем прибор возвращается обратно на станцию. Дисциплина обслуживания вызовов
следующая: если в момент поступления вызова прибор занят, то поступающий вызов ставится в
очередь. После освобождения прибора на обслуживание поступает первая заявка из очереди.

В системе $M\vert G\vert 1$ с дисциплиной обслуживания FIFO при загрузке системы меньше единицы
стационарное среднее время ожидания начала обслуживания определяется по формуле
$$
W=\fr{\lambda\beta_2}{2(1-\lambda\beta_1)}\,,
$$
где $\lambda$~--- интенсивность входящего потока; $\beta_1$, $\beta_2$~--- первый и второй
момент времени обслуживания~\cite{1za}. Этим определяется целесообразность введения критерия
оптимальности
$$
W(x)=\fr{1}{2n}\suml{i=1}{n}\fr{\lambda_i\beta_{i2}} {1-\lambda_i\beta_{i1}}
$$
при условии, что $\maxl{1\leqslant i\leqslant n} \lambda_i \beta_{i1}< 1 $, где
$\lambda_i$~--- интенсивность потока вызовов, поступающих на станцию $x_i$;
$\beta_{i1},\beta_{i2}$~--- соответственно первый и второй момент времени обслуживания на
$x_i$.

В статьях~\cite{2za, 3za} рассматривались аналогичные постановки задач, но с вызовами,
распределенными на плоскости, и эвклидовой нормой для определения расстояния.

\noindent
\textbf{Определение~3.} \textit{Размещение $x^*$ назовем оптимальным, если $W(x^*)\leqslant W(x)$ для
любого размещения $x$ такого, что $|x|=|x^*|$. Через $|x|$ здесь обозначено число элементов
размещения $x$.}

Введем еще ряд обозначений:

$|p|_m=\left(\int p^m(u)\,du\right)^{1/m}$\,;

$\e\eta=\beta_1$\,;

$\e\eta^2=\beta_2$\,;

$\{x\}$~--- последовательность размещений.

\section{Основные результаты}

\noindent
\textbf{Теоpема 1.} {\it Пусть плотность $p$ ограничена}, $p^{N/(N+1)}$ {\it интегрируема по
Лебегу}, $\e|\xi|^2<\infty$ {\it и} $\lambda(n)=o(n^{1/2})$. {\it Тогда для всякой
последовательности оптимальных размещений} $\{x^*\}:$
\begin{enumerate}[(1)]
%\begin{center}
%\begin{tabular}{lcl}
\item $\liml{n\rightarrow
\infty}n^{1/N}\left(\displaystyle\fr{n}{\lambda(n)}\, W(x^*)-0{,}5\beta_2\right)={}$

\noindent
\hfill${}=\displaystyle
\fr{N}{N+1}\,\beta_1 |p|_{N/(N+1)}\,;$\\
\item
$\liml{n\rightarrow\infty}\displaystyle\fr{\left|x_A^*\right|}{n}=|p|_{N/(N+1)}^{-N/(N+1)}\il{A}{}p^{N/(N+1)}(u)\,du\,,$
%\end{tabular}
%\end{center}
\end{enumerate}
{\it где $A$~--- произвольное измеримое по Лебегу множество},\
$x_A^*=x^*\cap A$.

\smallskip
Иные результаты в описанной модели получаются, если $\e \eta=0$, т.\,е.\ обслуживание заключается
лишь в перемещении прибора до вызова и обратно.

\smallskip

\noindent
\textbf{Теоpема 2.} {\it Если плотность $p$ ограничена}, $p^{1/N}$ {\it интегрируема по
Лебегу}, ${\e}|\xi|^2<\infty$ и $\lambda(n)= o(n^{1{,}5})$, {\it тогда для любой
последовательности оптимальных размещений} $\{x^*\}$:
%\begin{center}
{ %\tabcolsep=0pt
\begin{enumerate}[(1)]
\item
$\liml{n\rightarrow
\infty}\displaystyle\fr{n^{(N+2)/N}}{\lambda(n)}\, W(x^*)=\displaystyle\fr{1}{2}\,
\displaystyle\fr{N}{(N+2)} |p|_{1/N}\,;$
\item
$\liml{n\rightarrow\infty}\displaystyle\fr{\left|x_A^*\right|}{n}=|p|_{1/N}^{-1/N}\il{A}{}p^{1/N}(u)\,du$
\end{enumerate}}

\noindent
{\it для любого измеримого по Лебегу множества} $A$.

\section{Вспомогательные утверждения}

Все множества, рассматриваемые ниже, являются подмножествами $N$-мерного пространства $\rn$,
под интегралом понимается $N$-кратный интеграл Лебега.

\smallskip

\noindent
\textbf{Лемма 1.} \textit{Если $Q$~--- измеримое по Лебегу подмножество метрического пространства,
$S$~--- шар с центром в точке $v$ из того же пространства и меры Лебега множеств $Q$ и $S$ равны,
тогда}
$$
\il{Q}{} a(|u-v|)\,du \geqslant \il{S}{} a(|u-v|)\,du
$$
\textit{для любой неубывающей на $[0,\,\infty)$ действительной функции $a(u)$.}
\smallskip

\noindent
Д\,о\,к\,а\,з\,а\,т\,е\,л\,ь\,с\,т\,в\,о.~Пусть $R$~--- радиус шара~$S$. Под знаком интеграла будем
подразумевать $a(|u-v|)\,du$. Очевидны следующие соотношения
\begin{align*}
Q&=(Q\bigcap S)\bigcup (Q\backslash S)\,;\\
S&=(Q\bigcap S)\bigcup (S\backslash Q)
\Longrightarrow \mu(Q\backslash S)=\mu(S\backslash Q) \,;\\
\il{Q\backslash S}{} &\geqslant a(R) \mu(Q\backslash S) = a(R) \mu(S\backslash Q) \geqslant
\il{S\backslash Q}{}  \Longrightarrow\\
&\ \ \Longrightarrow
\il{Q}{}=\il{Q \bigcap S}{} + \il{Q\backslash S}{} \geqslant
\il{Q \bigcap S}{} +
\il{S\backslash Q}{} =
\il{S}{}\,. \qquad  %\sqq
\end{align*}
\smallskip

Следующая лемма является аналогом результатов Л.\,Ф.~Тота~\cite{4za}.

\smallskip

\noindent
\textbf{Лемма 2.} \textit{Пусть для $i=1,2,\ldots,n$ $S_i$ обозначает шар с центром в нуле,
$\sigma_n$~--- шар с центром в нуле и мерой $\sigma_n=(1/n) \sumi S_i$. Справедливо
следующее неравенство:}
\begin{equation}
\sumi \il{S_i}{} a(|u|)\,du \geqslant n \il{\sigma_n}{} a(|u|)\,du
\label{e1z}
\end{equation}
\textit{для любой неубывающей на $[0,\infty)$ действительной функции $a(u)$.}
\smallskip

\noindent
Д\,о\,к\,а\,з\,а\,т\,е\,л\,ь\,с\,т\,в\,о~проведем методом математической индукции. Для $n=1$ утверждение
леммы очевидно, так как неравенство обращается в равенство. Предположим, что $(1)$ верно для
$n=m$. Докажем его справедливость для $n=m+1$. Под знаком интеграла будем подразумевать
$a(|u|)\,du$, тогда
\begin{equation}
\suml{i=1}{m+1}\il{S_i}{} = \suml{i=1}{m} \il{S_i}{} + \il{S_{m+1}}{} \geqslant m
\il{\sigma_m}{} + \il{S_{m+1}}{} \,.
\label{e2z}
\end{equation}

Пусть для определенности $\sigma_m \subseteq \sigma_{m+1}$.
В случае $\sigma_{m+1} \subset
\sigma_m$ доказательство аналогичное.
\begin{multline*}
\sigma_{m+1}=\fr{1}{m+1} \left( \suml{i=1}{m}S_i + S_{m+1} \right) ={}\\
{}= \left(
\fr{1}{m}-\fr{1}{m(m+1)} \right) \left( \suml{i=1}{m}S_i + S_{m+1} \right)={}\\
{}=\sigma_m+\fr{1}{m}S_{m+1}-\fr{1}{m}\sigma_{m+1}={}\\
{}=
\sigma_m+\fr{1}{m}(S_{m+1}-\sigma_{m+1})
\Longrightarrow
%{}\\
%{}\Longrightarrow
S_{m+1} \geqslant \sigma_{m+1}\,,
\end{multline*}
так как по предположению $\sigma_{m+1}\geqslant \sigma_m$.

Поэтому
\begin{align*}
\il{\sigma_m}{}&=\il{\sigma_{m+1}}{}-\il{\sigma_{m+1}\backslash \sigma_m}{}\,; \\
%\qquad
\il{S_{m+1}}{}&=\il{\sigma_{m+1}}{}+\il{S_{m+1}\backslash \sigma_{m+1}}{}\,.
\end{align*}
Подставляя эти выражения в $(2)$, получим
\begin{multline*}
\suml{i=1}{m+1}\il{S_i}{}  \geqslant{}\\
{}\geqslant  m \int\limits_{\sigma_{m+1}} -\; m \il{\sigma_{m+1}\backslash
\sigma_m}{} +  \il{\sigma_{m+1}}{}  + \il{S_{m+1}\backslash \sigma_{m+1}}{} \geq{}
\\
{}\geq
 (m+1) \int\limits_{\sigma_{m+1}}  +\;  a\left ( R_{m+1}\right )(S_{m+1}- \sigma_{m+1})  - {}\\
 {}- m a\left ( R_{m+1}\right )(\sigma_{m+1} - \sigma_m) =
 (m+1) \il{\sigma_{m+1}}{} \,. \qquad  %\sqq
\end{multline*}

Предположим сначала, что $p$~--- простая, ограниченная функция и $p^{N/(N+1)}$
интегрируема по Лебегу.

\smallskip

\noindent
\textbf{Лемма 3.} {\it Если} ${\e} \eta > 0, \lambda(n)=o(n)$, {\it тогда для любой
последовательности оптимальных размещений} $\{x^*\}$
$$
\liml{n\rightarrow\infty}\frac{n}{\lambda(n)}W(x^*)-0,5\beta_2=0\,.
$$

Д\,о\,к\,а\,з\,а\,т\,е\,л\,ь\,с\,т\,в\,о.~Для всякого размещения $x=\{x_1,\ldots,x_n\}$ такого, что
$\maxl{1\leqslant i\leqslant n} \lambda_i \beta_{i1}< 1 $, выполняется неравенство
$$ %\begin{align*}
W(x)\geqslant \fr{1}{2n}\suml{i=1}{n}\lambda_i \beta_{i2}\,,
$$
где
$$
\beta_{i2}=
{\e}\left ( \fr{ (2|\xi-x_i|+\eta)^2}{\xi}\in C_i\right )\geqslant \e\eta^2=\beta_2\,;
$$
$C_i$~--- зона влияния станции $x_i$, а интенсивность потока поступающих на нее
требований есть $\lambda_i=\lambda(n) {\p} (C_i)$. С учетом этого получаем оценку снизу
для $W(x)$
$$
W(x)\geqslant 0{,}5\fr{\lambda(n)}{n}\,\beta_2\,.
$$

Пусть $K_i=\{u\in \rr: p(u)=p_i>0 \}$~--- измеримые по Лебегу непересекающиеся
множества. Множество и его меру Лебега $\mu$ иногда для краткости будем обозначать одной
и той же бук\-вой. Построим размещение $x^0$. Для этого каждое множество $K_i$ заменим
элементарным множеством $L_i$ таким, что $\mu(K_i \Delta L_i)<\varepsilon $, затем $L_i$
замещаем конгруэнтными $N$-мерными кубами, пе\-ре\-се\-ка\-ющи\-ми\-ся лишь по границе и объемом
$\sigma_i=K_i/m_i$, где
$$
m_i=m (1-\delta) \fr{p_i^{N/(N+1)}K_i}{\suml{j}{} p_j^{N/(N+1)}K_j}\,;
$$
$m$~---
некоторое натуральное число и $0<\delta<1$. Здесь множество и его мера Лебега обозначены одной
и той же буквой.

Если $\mu(\sigma_i \bigcap L_i)>0$, то в центр куба $\sigma_i$ помещается станция. Пусть
$n_i$~--- число таких станций. Построим последовательность вложенных рас\-ши\-ря\-ющих\-ся кубов
$K$ с центром в нуле, которые удовлетворяют условию  $\mu(K) = o(m)$.
$([m\delta]+1)$
станций разместим равномерно на множествах $(K_i\backslash L_i)\cap K$.

Итак, построено некоторое размещение $x\;=$\linebreak $=\;\{x_1,\ldots,x_n \}$, где $n$~--- общее число станций.

Оценим сверху $W(x^*)$. По определению $W(x^*)\leqslant W(x^0)$. Для $\{x^0\}$
%
%\noindent
\begin{equation*} %multline*}
n W(x^0)\leqslant \fr{1}{2} \suml{i}{}n_i
\lambda p_i\left (
\sigma_i^{(N+2)/N}\fr{N}{N+2}+{}\right. %\\
\end{equation*}

\noindent
\begin{multline*}
\left.{}+2 \beta_1 \fr{N}{N+1}\,\sigma_i^{(N+1)/N}+
\beta_2\sigma_i\right )\Bigg /\left (\vphantom{\Bigg /}
 1 - {}\right.\\
\left.{}-\lambda p_i\fr{N}{N+1}\,\sigma_i^{(N+1)/N}+\beta_1 \sigma_i
\vphantom{\Bigg /}\right )\,.
\end{multline*}
Знаменатель $1-\lambda p_i \left ( N/(N+1)\,\sigma_i^{(N+1)/N}+ \beta_1 \sigma_i\right )$
имеет порядок $1+o(1)$, так как $\lambda(n)=o(n)$ и
\begin{multline*}
p_i\sigma_i^{(N+1)/N}={}\\
{}=
\left(\suml{j}{}p_j^{N/(N+1)}K_j\right)^{(N+1)/N}n^{-(N+1)/N}+{}\\
{}+o(n^{-(N+1)/N})=O(n^{-(N+1)/N})\,;
\end{multline*}

\noindent
\begin{multline*}
p_i\sigma_i=p_i^{1/(N+1)}\left(\suml{j}{}p_j^{N/(N+1)}K_j\right)n^{-1}+{}\\
{}+o(n^{-1})=O(n^{-1})\,.
\end{multline*}
Таким образом,
$$
n W(x^0)\leqslant \fr{\lambda\beta_2}{2}(1+o(1)) \suml{i}{}p_iK_i\,.
$$
Устремляя $m$, а тем самым и $n$, к бесконечности и учитывая оценку снизу для $W$,
получаем утверждение леммы
$$
\liml{n\rightarrow\infty}\fr{n}{\lambda(n)}W(x^*)-0{,}5\beta_2=0\,.
$$

Введем некоторые обозначения.

Пусть $G$~--- некоторый компакт на носителе плотности $p$,
$D_n(x)=\maxl{1\leqslant i\leqslant n}\mathrm{diam}\,A_i$,
$A_i=\{u\;\in$\linebreak $\in\;A_i^{'} : p(u)>0 \}$, $A_i^{'}$~---
зона влияния $x_i$ на компакте~$G$.

\medskip
\noindent
\textbf{Лемма 4.} {\it Если для размещения $\{x\}$
$$
\liml{n\rightarrow\infty}\fr{n}{\lambda(n)}W(x)-0{,}5\beta_2=0\,,$$ тогда}
$$
\liml{n\rightarrow\infty} D_n(x)=0\,.
$$

\smallskip
\noindent
Д\,о\,к\,а\,з\,а\,т\,е\,л\,ь\,с\,т\,в\,о.~Предположим, что $\lims{n \to \infty~~} D_n(x)=d>0$. Тогда
выберем подпоследовательность размещений такую, что для некоторой станции $y_n$ из $n$-го
размещения этой подпоследовательности $\limn |u_n-v_n|=d$, где $u_n, v_n \in C_n$,
$C_n=\{u\in C_n' : p(u)>0 \}$, $C_n'$~--- зона влияния станции $y_n$ на $G$.

Ввиду компактности $G$ без ограничения общности можно считать, что $y_n \to y_0$,
$u_n \to u_0$, $v_n \to v_0 $.

Пусть $|u_0 - y_0|\geqslant d/2$. Это означает, что лишь конечное число размещений из
выбранной подпоследовательности имеет станции в $d/4$-окрестности точки $u_0$. Через $R$
обозначим $d/8$-окрестность точки $u_0$.

Поскольку $u_0$ является предельной точкой $C_0$ и $C_0$~--- подмножество носителя плотности,
то для некоторого $\varepsilon>0$
$$
\il{R}{}p(u)\,du > \varepsilon\,.
$$

Оценим снизу значение критерия $W$ на размещениях, не имеющих станций в $d/4$-окрестности
точки $u_0$:
\begin{multline*}
\beta_{i2}=\beta_2+\fr{4}{{\p} (C_i)}\il{C_i}{} |u-x_i|^2 p(u)\,du +{}\\
{}+ \fr{4 \beta_1}{{\p} (C_i)}\il{C_i}{} |u-x_i| p(u)\,du\,;
\end{multline*}

\noindent
$$
W(x)=\fr{1}{2n}\suml{i=1}{n}\fr{\lambda_i\beta_{i2}} {1-\lambda_i\beta_{i1}}
\geqslant \fr{\lambda(n)}{2n}\suml{i=1}{n} {\p} (C_i) \beta_{i2}\,.
$$
Учитывая, что на множестве $R$  $|u-x_i|> d/8$ для любого $i$ на выбранном размещении,
имеем
\begin{multline*}
\fr{2n}{\lambda(n)}W(x) \geqslant \beta_2 +
4 \il{R}{}\minl{1\leqslant i\leqslant
n}|u-x_i|^2 p(u)\,du +{}\\
{}+ 4 \beta_1 \il{R}{}\minl{1\leqslant i\leqslant n}|u-x_i| p(u)\,du
\geqslant
\\
\geqslant
 \beta_2+4\varepsilon\left (\fr{d}{8}\right )^2+4\beta_1\varepsilon\fr{d}{8} > 0\,.
\end{multline*}
Тем самым
$$
\liml{n\to\infty}\frac{n}{\lambda(n)}W(x)-0{,}5\beta_2 > 0\,,
$$
что противоречит условию теоремы.

Поэтому $d=0$.  \qquad  %\sqq


\medskip

\noindent
\textbf{Теоpема 3.} {\it Если $p^{N/(N+1)}$ интегрируема по Лебегу, тогда для всякой
последовательности оптимальных размещений} $\{x^*\}$
\begin{multline*}
\limi{n\rightarrow
\infty}n^{1/N}\left(\fr{n}{\lambda(n)}W(x^*)-0{,}5\beta_2\right)\geqslant{}\\
{}\geqslant
\fr{N}{N+1} \beta_1 |p|_{N/(N+1)}\,.
\end{multline*}

\smallskip
Д\,о\,к\,а\,з\,а\,т\,е\,л\,ь\,с\,т\,в\,о.~Пусть плотность $p$ определяется так же, как в лемме 3, $J$
-- конечное множество индексов. Тогда
$$
\rn=\cupl{i=1}{n}C_i\supset \cupl{j\in J}{}K_j\,.
$$
Следовательно,
\begin{multline*}
n\lambda^{-1}W(x)-0{,}5\beta_2\geqslant {}\\
{}\geqslant
\suml{j\in J}{}2p_j \left(\, \il{K_j}{}\minl{1\leqslant i\leqslant n}|u-x_i|^2\,du +{}\right.\\
\left.{}+\beta_1 \il{K_j}{}\minl{1\leqslant i\leqslant n}|u-x_i|\,du
\right)\,.
\end{multline*}

Так как $K_j$~--- ограниченное, измеримое по Лебегу множество, то для него найдется такое
элементарное множество $L_j$, что $\mu(K_j\backslash L_j)<\varepsilon K_j$, $0<\varepsilon <1$,
$\mu(L_j\backslash K_j)=0$.  Пусть
$$
L_j=\cupl{i\in I_j}{}P_i\,,
$$
$I_j$~--- конечное множество
индексов, $P_i$~--- непересекающиеся $N$-мерные параллелепипеды.

Число станций, попадающих в $P_i$ при оптимальном размещении $x^*$, обозначим через
$n_i$. Пусть $B_i$~--- параллелепипед объема $P_i(1-\varepsilon)$, гомотетичный $P_i$
относительно его центра симметрии. Обозначим через
$z^m=\{z_1,\dots,z_{n_i}\}$ оптимальное
размещение $n_i$ станций для $p \textbf{1}_{B_i}$ по критерию
${\e}\minl{1\leqslant l\leqslant n_i}|u-x_l|^m$ для~$m$ равного~1 или~2.

В силу леммы~4 при достаточно больших $n$ зоны влияния станций, находящихся в $P_i^c$, не
пересекаются с $B_i$, поэтому
$$
\il{B_i}{}\minl{1\leqslant l\leqslant n}|u-x_l|^m\,du \geqslant \il{B_i}{}\minl{1\leqslant l\leqslant n_i}|u-z_l|^m\,du\,.
$$
Обозначим через $A_l$ зону влияния станции $z_l$ на $B_i$.
Применяя лемму~2, имеем
\begin{multline*}
\il{B_i}{}\minl{1\leqslant l\leqslant n_i}|u-z_l|^m\,du
=\suml{l=1}{n_i}\il{A_l}{}|u-z_l|^m\,du \geqslant {}\\
{}\geqslant
n_i \il{\sigma_i}{}|u|^m\,du=C n_i \sigma_i^{(N+m)/N}\,,
\end{multline*}
где $\sigma_i$~--- $N$-мерный куб с центром в нуле и объемом $B_i/n_i$, $C=(1/2)(N/(N+1))$ для
$m=1$ и $C=(1/4) (N/(N+2))$ для $m=2$.

Следовательно,
\begin{multline*}
\il{K_j}{}\minl{1\leqslant i\leqslant n}|u-x_i|^m\,du \geqslant
\suml{i\in I_j}{}C n_i \sigma_i^{(N+m)/N}= {}\\
{}=
C\suml{i\in I_j}{}n_i^{-m/N}B_i^{(N+m)/2}\geqslant {}\\
{}\geqslant
Cn_j^{-m/N}K_j^{(N+m)/N}(1-\varepsilon)^{N+m}\,,
\end{multline*}
где $n_j$~--- число станций размещения $x_{K_j}^*$.

При выводе последнего соотношения было использовано неравенство Гельдера с параметрами
$(-N/m,\, N/(N+m))$.
Далее,
\begin{multline*}
n\lambda^{-1}W(x)-0{,}5\beta_2\geqslant\\
\geqslant
\suml{j\in J}{}p_j \left(\fr{1}{2}\,\fr{N}{N+2} n_j^{-2/N}K_j^{(N+2)/N}(1-\varepsilon) +{}\right.\\
{}+\left.
\fr{N}{N+1}\beta_1 n_j^{-1/N}K_j^{(N+1)/N}\right)(1-\varepsilon)^{N+1}\geqslant\\
\geqslant \fr{1}{2}\,\fr{N}{N+2}n^{-2/N}(1-\varepsilon)^{N+2}\times{}\\
{}\times \left(
\suml{j\in J}{}p_j^{N/(N+2)}K_j\right)^{(N+2)/N} +{}\\
{}
+\fr{N}{N+1}\beta_1 n^{-1/N}(1-\varepsilon)^{N+1}\times{}\\
{}\times \left( \suml{j\in
J}{}p_j^{N/(N+1)}K_j\right)^{(N+1)/N}\,.
\end{multline*}
Здесь дважды было применено неравенство Гельдера с параметрами
$(N/(N+2),\,-N/2)$ и
$(N/(N\;+{}$\linebreak ${}+\;1),\,-N)$.

Из интегрируемости $p^{N/(N+1)}$ следует неравенство
\begin{multline*}
\limi{n\rightarrow
\infty}n^{1/N}\left(\fr{n}{\lambda(n)}W(x^*)-0{,}5\beta_2\right)\geqslant{}\\
{}\geqslant
\fr{N}{N+1} \beta_1 |p|_{N/(N+1)}(1-\varepsilon)^{N+1}\,.
\end{multline*}

Так как $\varepsilon$ может быть выбрано произвольно малым, то
\begin{multline*}
\limi{n\rightarrow
\infty}n^{1/N}\left(\fr{n}{\lambda(n)}W(x^*)-0{,}5\beta_2\right)\geqslant{}\\
{}\geqslant
\fr{N}{N+1} \beta_1 |p|_{N/(N+1)}\,.
\end{multline*}

\section{Доказательство основных результатов}

Приведем только доказательство теоремы~1; доказательство теоремы~2 проводится аналогично.
Укажем способ построения таких размещений $\{x\}$, на которых неравенство теоремы 3
превращается в равенство. Такие размещения будем называть асимптотически оптимальными второго
порядка по критерию $W$.

Так как ${\e}|\xi|^2<\infty$, то можно выбрать последовательность вложенных расширяющихся
кубов $K$ с центром в нуле таких, что ${\e}|\xi|^2 \textbf{1}_{K^c}=o(m^{-1/N})$.

Так как $K_i$ -- измеримые по Лебегу множества, следовательно, для любого $\varepsilon > 0$
найдется такое элементарное множество $L_i$, что
$$
\mu(K_i \Delta L_i)<\fr{\varepsilon}{K}\ \ \mbox{для}\ \ \forall i \,.
$$

Затем каждое множество $L_i$ замещаем конгруэнтными $N$-мерными кубами, пересекающимися по
границе и объемом $\sigma_i=K_i/m_i$, где
$$
m_i=m(1-\delta) \fr{p_i^{N/(N+1)} K_i}{\suml{j=1}{r}p_i^{N/(N+1)} K_i}\,,\quad 0<\delta<1\,.
$$

Если $\mu(\sigma_i \bigcap L_i)>0$, то в центр $\sigma_i$ помещается станция. Пусть
$n_i$~--- число таких $\sigma_i$.  $([m\delta]+1)$ станций разместим равномерно на множествах
$(K_i\backslash L_i)\cap K$. Итак, построено некоторое размещение $x=\{x_1,\ldots,x_n
\}$, где $n$~--- общее число станций.

Тогда
\begin{multline*}
\fr{n}{\lambda} W(x)\leqslant \suml{i}{}n_i
\left (p_i\left ( 0{,}5 \sigma_i^{(N+2)/N}\fr{N}{N+2}+{}\right.\right.\\
{}+\left.\left.\beta_1 \fr{N}{N+1}\,\sigma_i^{(N+1)/N}\right )\right )\Bigg / \left (\vphantom{\Bigg /}
1 -{}\right.\\
\left.{}-\lambda p_i\left ( \fr{N}{N+1}\,
\sigma_i^{(N+1)/N}+\beta_1\sigma_i\right )
\vphantom{\Bigg /}\right )+{}\\
{}+\fr{1}{2}\suml{i}{}\fr{\beta_2 p_i K_i}{1+o(n^{-1/N})} + o(n^{-1/N})={}   \\
{}=(1+o(1))\suml{i}{}(0{,}5 \fr{N}{N+2}\,p_i^{(N-1)/(N+1)}K_i\times{}\\
{}\times \left(\suml{j}{}p_j^{N/(N+1)}K_j\right)(n(1-\delta))^{-2/N}+{}\\
{}+ \fr{N}{N+1}\,\beta_1 p_i^{N/(N+1)} K_i \left(\suml{j}{}p_j^{N/(N+1)} K_j\right)^{1/N}\times{}\\
{}\times
(n(1-\delta))^{-1/N} )+ 0{,}5\beta_2\left ( 1+o(n^{-1/N})\right )\,.
\end{multline*}

Совершая предельный переход, получаем
\begin{multline*}
\lims{n\rightarrow
\infty}n^{1/N}\left(\fr{n}{\lambda(n)}W(x)-0{,}5\beta_2\right)\leqslant{}\\
{}\leqslant
\fr{N}{N+1}\, \beta_1 |p|_{N/(N+1)}\,.
\end{multline*}

Из этого неравенства и теоремы~3 следует утверждение первого пункта теоремы~1 в случае,
когда плотность $p$ простая.

Пусть $p$~--- произвольная функция, удовлетворяющая условиям доказываемой теоремы. Введем
простые функции $\underline{p}_{~k}(u)$ по правилу:
$\underline{p}_{k}(u)=m/k$, если $m/k<p(u)\leqslant (m+1)/k$, $k\in \N$,
$m=0,1,\ldots$

В определении критерия $W$ заменим $p$ на $\underline{p}_{k}$ и полученное выражение
обозначим через $W_k$. Для простых функций было доказано, что
\begin{multline*}
n^{1/N}\left(\fr{n}{\lambda(n)}W_k(x^*)-0{,}5\beta_2\right)={}\\
{}=
\fr{N}{N+1} \beta_1 |\underline{p}_{~k}|_{N/(N+1)} + o(1)\,.
\end{multline*}

Применим теорему Лебега о предельном переходе под знаком интеграла к обеим частям равенства,
тогда
\begin{multline*}
n^{1/N}\left(\fr{n}{\lambda(n)}W(x)-0{,}5\beta_2\right) ={}\\
{}=
\fr{N}{N+1}\, \beta_1 |p|_{N/(N+1)}+o(1)\,.
\end{multline*}

Устремляя $n$ к бесконечности, получим
\begin{multline*}
\liml{n\rightarrow
\infty}n^{1/N}\left(\fr{n}{\lambda(n)}W(x)-0{,}5\beta_2\right) ={}\\
{}=
\fr{N}{N+1}\,\beta_1 |p|_{N/(N+1)}\,.
\end{multline*}

\smallskip
Метод доказательства второго пункта теоремы~1 для случая $N=2$
изложен в статье~\cite{3za}, и его
несложно обобщить на случай произвольного $N$.


{\small\frenchspacing
{%\baselineskip=10.8pt
\addcontentsline{toc}{section}{Литература}
\begin{thebibliography}{9}

\bibitem{1za}
\Au{Ивченко Г.\,И., Каштанов В.\,А., Коваленко~И.\,Н.}
Теория массового обслуживания.~--- М.: Высшая школа, 1982.

\bibitem{2za}
\Au{Захарова Т.\,В.}
Оптимизация расположения станций обслуживания на плоскости~//
Изв. АН СССР. Техн. киберн., 1987. №\,6. С.~83--91.
%\bigskip

\bibitem{3za}
\Au{Захарова Т.\,В.}
Размещения систем массового обслуживания, минимизирующие среднюю
длину очереди~// Информатика и её применения, 2008. Т.~2. Вып.~1. С.~35--38.

\label{end\stat}

\bibitem{4za}
\Au{Тот Ф.\,Л.}
Расположение на плоскости, на сфере и в пространстве. -- М.: ГИФМЛ, 1958.


\end{thebibliography}

}
}
\end{multicols}