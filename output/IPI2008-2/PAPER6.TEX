\def\stat{pagurova}


\def\tit{ОБ АСИМПТОТИЧЕСКОМ РАСПРЕДЕЛЕНИИ МАКСИМАЛЬНОЙ
ПОРЯДКОВОЙ  СТАТИСТИКИ В ВЫБОРКЕ СЛУЧАЙНОГО
ОБЪЕМА$^*$}
\def\titkol{Об асимптотическом распределении максимальной
порядковой  статистики в выборке случайного
объема}
\def\autkol{В.\,И.~Пагурова}
\def\aut{В.\,И.~Пагурова$^1$}

\titel{\tit}{\aut}{\autkol}{\titkol}

{\renewcommand{\thefootnote}{\fnsymbol{footnote}}\footnotetext[1]
{Работа выполнена при поддержке РФФИ, грант
08-01-00567.}}

\renewcommand{\thefootnote}{\arabic{footnote}}
\footnotetext[1]{Московский государственный университет
им.\ М.\,В.~Ломоносова, факультет вычислительной математики и
кибернетики, pagurova@yandex.ru}


\vspace*{24pt}

\Abst{Исследуется асимптотическое
распределение при $n \to \infty$ нормированного максимума в
предположении, что случайный объем выборки представим в виде суммы
$n$ независимых одинаково распределенных величин. Данная работа
является обобщением работы~\cite{1pag}, в которой объем выборки имеет
распределение Пуассона с параметром $n$. Для однопараметрического
семейства распределений, зависящего от неизвестного параметра
сдвига, исследуется скорость сходимости распределения
нормированного максимума к предельному закону. Рассматриваются
классы распределений с экспоненциальными и степенными хвостами.}

\KW{случайно индексированный максимум;
однопараметрическое семейство распределений; скорость сходимости}

      \vskip 36pt plus 9pt minus 6pt

      \thispagestyle{headings}

      \begin{multicols}{2}

      \label{st\stat}

\section{Введение}

Пусть случайные величины $X_1,\ldots,X_n$ независимы и одинаково
распределены (н.о.р.)\ с общей абсолютно непрерывной функцией
распределения (ф.р.) $F(x-\theta)$, $\theta$~--- неизвестный
параметр, $X_1^{(n)}\leq X_2^{(n)}\leq \ldots\leq X_n^{(n)}$~---
соответствующий вариационный ряд, $N$~--- неотрицательная
целочисленная случайная величина, не зависящая от $X_1,\ldots,X_n$,
причем $N$ имеет представление
\begin{equation}
N=\sum_{i=1}^{n}\xi_i\,,
\label{1}
\end{equation}
где $\xi_1,\ldots,\xi_n$~--- н.о.р. случайные величины,
\begin{equation}
{\bf E}\xi_1=\alpha>0\,, \quad 0<{\bf E}\xi_1^2<\infty\,.
\label{2}
\end{equation}

Исследуется влияние оценки неизвестного параметра $\theta$ на
асимптотическое поведение разности $X_N^{(N)}-\hat \theta_N $ при
$n \to\infty, \, \hat \theta_N$~--- состоятельная оценка параметра
$\theta$, построенная по наблюдениям $X_1,\ldots ,X_N$. Частный случай
данной задачи, когда $N$ имеет распределение Пуассона с параметром~$n$, 
был исследован в работе~\cite{1pag}. В качестве других примеров
распределения $N,$ удовлетворяющего условиям~(1) и~(2), можно
указать биномиальное распределение $b(n,\alpha), \, 0<\alpha<1$, и
отрицательное биномиальное распределение $\bar b(n,p)$,
$\alpha=(1-p)/p$.

Сначала рассмотрим случай известного па\-ра\-мет\-ра $\theta$. Без
нарушения общности положим $\theta=0$. Введем величину $y=y_n(t)$,
являющуюся решением уравнения
\begin{equation}
F(y)=1-\fr{t}{n}\,, \quad t>0\,,
\label{3}
\end{equation}
 тогда
 \begin{multline*}
M_n(t)={\bf P}\left\{X_N^{(N)}<y\right\}={\bf E}_N\left(1-\fr{t}{n}\right)^N={}\\
{}=\left({\bf E}
 \left(1-\fr{t}{n}\right)^{\xi_1}\right)^n\,.
\end{multline*}
Рассмотрим
 \begin{multline*}
Z_n=\left(1-\fr{t}{n}\right)^{\xi_1}=1-\fr{t\xi_1}{n}+{}\\
{}+
\fr{\xi_1(\xi_1-1)
 (1-\eta)^{\xi_1-2}}{2}\fr{t^2}{n^2}\,, \quad 0<\eta<\fr{t}{n}\,.
\end{multline*}
 Учитывая условия~(1) и~(2), получим при $n \to \infty$
 $$
{\bf E}Z_n=1-\fr{\alpha t}{n}+O\left(\fr{1}{n^2}\right)
$$
 равномерно в
 любом конечном интервале $t\in (0,\,t_0]$. Поэтому при $n \to
 \infty$
 $$
M_n(t)=\exp(-\alpha t)+O(n^{-1})\,.
$$

\section{Скорость сходимости}

Обозначим через (A) класс распределений $F(x)$, для которых при $x \to\infty$
\begin{align*}
1-F(x)&\sim ax^{\gamma}\exp(-bx^{\beta})\,; \\
F^{\prime}(x)=f(x)&\sim ab\beta
x^{\gamma+\beta-1}\exp(-bx^{\beta})\,;\\
|f^{\prime}(x)|&\sim
ab^2\beta^2x^{\gamma+2(\beta-1)}\exp(-bx^{\beta})\,, 
\end{align*}
$$
a,b,\beta>0, \quad -\infty<\gamma<\infty\,.
$$
Символом $(B)$
обозначим класс распределений $F(x)$, для которых при $x \to
\infty$
\begin{gather*}
1-F(x)\sim ax^{-\gamma}, \quad f(x)\sim a\gamma
x^{-\gamma-1}\,;\\
|f^{\prime}(x)|\sim
a\gamma(\gamma+1)x^{-\gamma-2}, \quad a,\gamma>0\,.
\end{gather*}
Для величины
$y=y_n(t),$ определяемой соотношением~(3), при $n \to \infty$ и
фиксированном $t>0$ для класса~(A) получаем
\begin{align*}
y&=O((\ln n)^{1/\beta})\,, \\ 
f(y)&=O(n^{-1}(\ln n)^{(\beta-1)/\beta})\,,
\\ 
f^{\prime}(y)&=O(n^{-1}(\ln n)^{2(\beta-1)/\beta})\,,
\end{align*}
для класса~(B) получаем
\begin{align*}
y&=O(n^{1/\gamma})\,, \\
f(y)&=O(n^{-1-1/\gamma})\,, \\ 
f^{\prime}(y)&=O(n^{-1-2/\gamma})\,.
\end{align*}
Распределения классов (А) и (В) играют важную роль при моделировании
экстремальных событий (см., например,~[2]).
Далее рассматривается величина
\begin{equation}
L_n(t)={\bf P}\{X_N^{(N)}-\hat \theta _N<y_n(t) \}\,,
\label{4}
\end{equation}
где $X_1,\ldots ,X_n$~--- н.о.р.\
случайные величины с общей ф.р. $F(x-\theta)$; $N$~---
неотрицательная целочисленная случайная величина, не зависящая от
$X_1,\ldots ,X_n$, удовлетворяющая соотношениям~(1) и~(2);
$\hat\theta_N$~--- оценка параметра $\theta$, построенная по
наблюдениям $X_1,\ldots ,X_N$;  $y_n(t)$ определяется
соотношением~(3).


\medskip
\noindent
{\bf Теорема~1.} \textit{Пусть $F(x)$ является непрерывной ф.р.,
распределение $F$ симметрично относительно нуля, $f(0)>0, \, \,
f^{\prime}(x)$ ограничена в окрестности нуля, $\hat
\theta_N=X_{[N/2]+1}^{(N)},$ тогда при $n \to \infty$ равномерно в
любом конечном интервале} $t\in (0,\,t_0]$
$$
L_n(t)=\exp(-\alpha t)+O(r_n)\,,
$$
\textit{где}
$$
r_n=\begin{cases}
\max(n^{-1},n^{-1}(\ln
n)^{2(\beta-1)/\beta}) &\! \mbox{\textit{для класса} $(A)$}, \\ 
n^{-1} &\!
\mbox{\textit{для класса} $(B)$}. \\ \end{cases}
$$

{Д\,о\,к\,а\,з\,а\,т\,е\,л\,ь\,с\,т\,в\,о} будем проводить по плану доказательства
теоремы~1 в~\cite{1pag}.
Заметим, что
$$
L_n(t)={\bf E}_NJ_N\,;
$$

\noindent
\begin{multline}
J_N={\bf P}\{X_N^{(N)}-X_{[N/2]+1}^{(N)}<y|N\}={}\\
{}=C_N\int\limits_{-\infty}^{\infty}F^{[N/2]}(u)(F(u+y)-{}\\
{}-F(u))^{N-[N/2]-1}
f(u)\,du=\\
{}=C_N\int\limits_{-\infty}^{\infty}H_N(u)f(u)\,du\,, \label{5}
\end{multline}
где
$$
%\begin{multline}
H_N(u)= F(u)\left ( 1-{}\right.\hspace*{24mm} %\\
$$
\begin{equation}
\left. -\;F(u)\right )^{N-2[N/2]}M^{[N/2]-1}(u)G^{N-[N/2]-1}(u)\,;\!\!\!\label{6}\\
\end{equation}


\noindent
\begin{align*}
C_N&=\fr{N!}{[N/2]!(N-[N/2]-1)!}\,; \\
 G(u)&=\fr{F(u+y)-F(u)}{1-F(u)}\,; \\
M(u)&=F(u)(1-F(u))\,.
\end{align*}
 Оценим
\begin{multline}
{\bf E}_NC_N\int\limits_{-\infty}^{-n^{-1/4}}H_N(u)f(u)\,du \leq {}\\
{}\leq {\bf E}_NC_N
 2^{-2\left[N/2\right]+2}\left(1-\fr{4f^2(0)}{\sqrt{n}}+{}\right.\\
\left.{} +o\left(\fr{1}{\sqrt{n}}\right)\right)^{\left[N/2\right]-1}\,,
 \label{7}
\end{multline}
так как $0\leq G(u) \leq 1$, а на интервале  
$ u\in$\linebreak $\in (-\infty, -n^{-1/4})$ имеем
\begin{multline*}
M(u)\leq
F(-n^{-1/4})(1-F(-n^{-1/4}))={}\\
{}=
\fr{1}{4}\left(1-\fr{4f^2(0)}{\sqrt{n}}+o\left(\fr{1}{\sqrt{n}}
\right)\right)\,.
\end{multline*}
Используя формулу Стирлинга, получим при $n\;\to$\linebreak $\to\;\infty$
\begin{equation}
C_n \left(\fr{\sqrt{2\pi}}{\sqrt{n}\cdot 2^n}\right)\longrightarrow 1\,.
\label{8}
\end{equation}
Так как $ N \stackrel{P}{\longrightarrow}\infty$ при
$n \to \infty$, отсюда получим
$$
C_N\left(\fr{\sqrt{2\pi}}{\sqrt{N}\cdot 2^N}\right)\stackrel{P}{\longrightarrow}1\,,
$$
а так  как  $N=\alpha n+o_p(n)$ при  $n \to \infty$, тогда правая
часть соотношения~(7) при $n \to \infty$ имеет порядок $o(r_n)$.
Аналогичным образом при $n \to \infty$
$$
{\bf E}_NC_N\int\limits_{n^{-1/4}}^{\infty}H_N(u)f(u)\,du=o(r_n)\,.
$$
Таким  образом,
\begin{equation}
{\bf E}_NJ_N={\bf E}_NC_N\int\limits_{-n^{-1/4}}^{n^{-1/4}}H_N(u)f(u)\,du+o(r_n)\,,
\label{9}
\end{equation}
$H_N(u)$ определяется соотношением~(6). Обозначим
\begin{equation}
k_{N}=N-\left[\fr{N}{2}\right]-1\,, \label{10}
\end{equation}  
тогда на
интервале $u\in (-n^{-1/4},n^{-1/4}) $ имеем
\begin{multline}
G^{k_{N}}(u)=G^{k_{N}}(0)+k_{N}G^{k_{N}-1}(0)G^{\prime}(u)u+{}\\
{}+(k_{N}(k_{N}-1)G^{k_{N}-2}(\xi) (G^{\prime}(\xi))^2+{}\\
{}+k_{N}G^{k_{N}-1}(\xi)G^{\prime \prime}(\xi))\fr{u^2}{2}\,, \quad
|\xi-u|<n^{-1/4}\,; \label{11}
\end{multline}
$$
G^{\prime}(u)=\fr{f(u+y)(1-F(u))-f(u)(1-F(u+y))}{(1-F(u))^2}\,;
$$
\begin{multline*}
G^{\prime \prime}(u)=((1-F(u))^2f^{\prime}(u+y)+{}\\
{}+2(1-F(u))f(u)f(u+y)-{}\\
{}-(1-F(u))(1-F(u+y))f^{\prime} (u)-{}\\
{}- 2f^2(u)(1-F(u+y)))(1-F(u))^{-3}\,.
\end{multline*} 
Учитывая соотношение~(3), получим при $n \to \infty$ 
\begin{equation}
A_n=\sup_{|u|\leq n^{-1/4}}|G^{\prime}(u)|=O(\max (f(y),n^{-1}))\,; 
\label{12}
\end{equation}
\begin{multline}
B_n=\sup_{|u|\leq n^{-1/4}}|G^{\prime \prime }(u)|={}\\
{}=
\begin{cases} 
O(\max(f^{\prime}(y),n^{-1})) & \mbox{для
класса $(A)$}\,, \\ 
O(n^{-1}) & \mbox {для класса $(B)$}\,, 
\end{cases}
\label{13}
\end{multline} 
на интервале $u\in (-n^{-1/4},n^{-1/4})$ имеем
\begin{align}
G(u)&=1-\fr{2t}{n}+o(n^{-1})\,,\notag\\
G(0)&=1-\fr{2t}{n}, \quad
G^{\prime}(0)=O(\max(f(y),n^{-1}))\,. \label{14}
\end{align}
Подставим представление~(11) в выражение~(5) и почленно проинтегрируем.
Получим 
\begin{equation}
{\bf E}_NJ_N={\bf E}_NG^{k_N}(0)+G^{\prime}(0){\bf E}_N
K_{1N}+{\bf E}_N K_{2N}\,, \label{15}
\end{equation} 
где 
\begin{align*}
K_{1N}&= k_NG^{k_N-1}(0)C_N \int\limits_{-\infty}^{\infty}F^{[N/2]}\left(1-{}\right.\\
&\ \ \ \ \ \ \ \ \ \left.{}-F(u)\right )^{N-[N/2]-1}uf(u)\,du\,;
\\
K_{2N}&\leq
\left ( A_n^2k_N(k_N-1)+{}\right.\\
&{}+\left.k_NB_n\right )C_N\int\limits_{-\infty}^{\infty}F^{[N/2]}(u)\left (1-{}\right.\\
&\left. {}- F(u)\right )^{N-[N/2]-1}u^2f(u)\,du\,;
\end{align*} 
$A_n$ и $B_n$ определяются соотношениями~(12) и~(13).

Покажем, что при $n \to \infty$
\begin{multline*}
D_n=k_nC_n\int\limits_{-\infty}^{\infty}F^{[n/2]}(u)\left ( 1-{}\right.\\
\left. {}-F(u)\right )^{n-[n/2]-1}
uf(u)\,du=O(1)\,.
\end{multline*}
Тогда, учитывая, что 
$$
\fr{N}{\alpha n}\stackrel{P}{\longrightarrow}1\,;\quad 
G^{k_n-1}(0)\longrightarrow \exp(-t) \quad (n \to \infty)\,,
$$ 
получим
$$
G^{k_N-1}(0)\stackrel{P}{\longrightarrow}\exp(-\alpha t), \quad
D_N=O_p(1)\,. 
$$ 
Имеем 
\begin{multline*}
D_n=k_nC_n\int\limits_{0}^{\infty}\left (F(u)\left ( 1-{}\right.\right.\\
\left.\left.{}-F(u)\right )\right )^{[n/2]-1}(2F(u)-1)uf(u)\,du={}\\
{}=\fr{k_nC_n}{n}\int\limits_{0}^{n^{-1/4}}(F(u)(1-F(u)))^{[n/2]}\,du+
o\left(\fr{1}{n}\right)={}\\
{}=\fr{k_nC_nC([n/2]!)^2}{n(n+1)!}{\bf
P}\left\{\fr{1}{2}\leq
\beta_{[n/2]+1,[n/2]+1}\leq{}\right. \\
\left.{}\leq
\fr{1}{2}+\fr{f(0)}{n^{1/4}}+o\left(\fr{1}{n^{1/4}}\right)\right \}+
o\left(\fr{1}{n}\right)=O(1)\,.
\end{multline*}
В последнем соотношении учтено поведение $f(u)$ в окрестности нуля,
соотношения~(8)--(10) и формула Стирлинга, $\beta_{k,n}$
означает случайную величину, имеющую бета-распределение с
параметрами $(k,n)$. Таким образом, при $n \to \infty$ 
$$
{\bf E}_NK_{1N}=O(1)\,.
$$ 
Аналогичным образом при $n \to \infty $ 
$$
{\bf E}_NK_{2N}=O(r_n)\,,
$$ 
$r_n$ определено в формулировке теоремы. С учетом соотношений~(14) и~(15) 
получим доказательство теоремы.

\bigskip
\noindent
{\bf Теорема~2.} \textit{Пусть $X_1,\ldots ,X_n$~--- н.о.р. случайные величины
с общей непрерывной ф.р. $F(x-\theta), \, \, N$ означает
неотрицательную целочисленную величину, не зависящую от
$X_1,\ldots ,X_n$ и удовле\-тво\-ря\-ющую условиям}~(1) и~(2), 
$F(0)=0$,  $f(0)>0$,  $f^{\prime}(x)$ \textit{ограничена в окрестности нуля, $L_n(t)$
определяется соотношением~(4), где $\hat \theta_N=X_1^{(N)}$,
тогда при $n \to \infty$ равномерно по $t$ в любом конечном
интервале} $t\in(0,\,t_0]$ 
$$
L_n(t)=\exp(-\alpha t)+O(r_n)\,, 
$$ 
\textit{где}
$$
r_n=\begin{cases}
\max(n^{-1},n^{-1}(\ln n)^{(\beta-1)/\beta}) & \mbox{\textit{для класса} $(A)$\,,} \\ 
n^{-1} & \mbox{\textit{для класса} $(B)$\,.} 
\end{cases}
$$


\noindent
{Д\,о\,к\,а\,з\,а\,т\,е\,л\,ь\,с\,т\,в\,о.} Как и прежде, 
\begin{equation}
L_n(t)={\bf E}_NJ_N\,, 
\label{16}
\end{equation} 
где 
\begin{multline}
J_N={\bf P}\{X_N^{(N)}-X_1^{(N)}<y|N\}={}\\
{}=N\int\limits_{0}^{\infty}(F(u+y)-F(u))^{N-1}f(u)\,du={}\\
{}=
N\int\limits_{0}^{\infty}G^{N-1}(u)(1-F(u))^{N-1}f(u)\,du={}\\
{}=\left(1-\fr{t}{n}\right)^{N-1}+
(N-1)\int\limits_{0}^{\infty}H_{1N}(u)\,du-{}\\
{}-(N-1)\int\limits_{0}^{\infty}H_{2N}(u)\,du\,.
\label{17}
\end{multline} 
Здесь
\begin{align*}
H_{1N}(u)&=(F(u+y)-{}\\
&\ \ \ \ \ \ \ \ \ \ \ \ \ \ \ \ {}- F(u))^{N-2}(1-F(u))f(u+y)\,;\\
H_{2N}(u)&=(F(u+y)-{}\\
&\ \ \ \ \ \ \ \ \ \ \ \ \ \ \ \ {}-F(u))^{N-2}(1-F(u+y))f(u)\,. 
\end{align*} 
Рассмотрим
\begin{multline*}
K_{1N}=(N-1)\int\limits_{n^{-1/2}}^{\infty}H_{1N}(u)\,du<{}\\
{}<(N-1)
(1-F(n^{-1/2}))\int\limits_{n^{-1/2}}^{\infty}
\left (\vphantom{\left. F(n^{-1/2})\right )}F(u+y)-{}\right.\\
{}-\left. F(n^{-1/2})\right )^{N-2}f(u+y)\,du={}\\
{}=(1-F(n^{-1/2}))^{N}-(1-F(n^{-1/2}))(F(n^{-1/2}+y)-{}\\
{}-F(n^{-1/2}))^{N-1}\,.
\end{multline*}
Так как 
$$
F(n^{-1/2})=f(0)n^{-1/2}+o(n^{-1/2})\,,\hspace*{30pt}
$$
\begin{multline*}
F(n^{-1/2}+y)=F(y)+f(\zeta)n^{-1/2}={}\\
{}=1-\fr{t}{n}+o\left(\fr{1}{n}\right)\,,
\quad |\zeta-y|<n^{-1/2}\,, 
\end{multline*} 
то при $n \to \infty$ получим 
$$
{\bf E}_NK_{1N}=o(r_n)
$$ 
равномерно по $t$ в любом конечном интервале
$t\in(0,\,t_0], \, r_n$ определено в формулировке теоремы.
Рассмотрим 
$$
K_{2N}=(N-1)\int\limits_{0}^{n^{-1/2}}H_{1N}(u)\,du\,.
$$ 
На интервале $u\in(0,\,n^{-1/2})$ имеем
\begin{gather*}
F(u+y)-F(u)=F(y)+(f(\eta)-f(\xi))u\,; \\ 
|\eta-y|<n^{-1/2}\,; \quad
 0<\xi<n^{-1/2}\,; \\ 
f(u+y)=O(f(y))\,.
\end{gather*} 
Тогда при достаточно
больших $n$ существует $C_1>0$ такое, что $f(\eta)-f(\xi)<-C_1$ и
\begin{multline*}
K_{2N}\leq (N-1)\int\limits_{0}^{n^{-1/2}}(F(y)-C_1u)^{N-2}\,du \,
O(f(y))={}\\
{}=O(f(y))\fr{(F(y))^{N-1}-(F(y)-C_1n^{-1/2})^{N-1}}{C_1}\,.
\end{multline*}
Отсюда следует, что при $n \to \infty$ 
$$
{\bf E}_NK_{2N}=O(f(y))\,.
$$ 
Далее,
\begin{gather*}
(N-1)\int\limits_{0}^{\infty}H_{2N}(u)\,du\leq(1-F(y))=\fr{t}{n}\,;\\
{\bf E}_N\left(1-\fr{t}{n}\right)^{N-1}=\exp(-\alpha
t)+O\left(\fr{1}{n}\right)
\end{gather*}
равномерно по $t$ в любом конечном
интервале $t\in(0,\,t_0].$ Учитывая соотношения~(16) и~(17), получим
доказательство теоремы.

\bigskip
\noindent
{\bf Теорема 2$^{\prime}$.} \textit{В условиях теоремы~2 пусть $N$ имеет
распределение Пуассона с параметром $\alpha n$. Тогда при  $n \to
\infty$ равномерно по $t$ в любом конечном интервале} $t\in(0,\,t_0]$
$$
L_n(t)=\exp(-\alpha t)+O(r_n)\,,
$$ 
\textit{где}
$$
r_n=\left\{\begin{array}{ll}n^{-1}(\ln n)^{(\beta-1)/\beta} &
\mbox{\textit{для класса} $(A)$\,,} \\ 
n^{-1-1/\gamma} & \mbox{\textit{для класса}
$(B)$\,.} \\ \end{array}\right. 
$$ 
Теорема $2^{\prime}$  была
доказана в работе~\cite{1pag}. Сравнение теорем~2 и $2^{\prime}$
показывает, что пуассоновское распределение обеспечивает  более
быструю сходимость к предельному закону, по крайней мере, для
класса~(В), чем в общем случае, рассматриваемом в теореме~2.

\medskip
\noindent
{\bf Пример.} Пусть 
$$
F(x)=1-\fr{1}{\sqrt{x+1}}\,, \quad x\geq
0\,,
$$ 
тогда при $n \to \infty $ равномерно в любом конечном
интервале $t\in(0,\,t_0]$ имеем 
$$
L_n(t)=\exp(-\alpha t)+O\left(\fr{1}{n^3}\right)\,,
$$ 
если $N$ имеет распределение Пуассона с па\-ра\-мет\-ром $\alpha n;$ 
$$
L_n(t)=\exp(-\alpha t)+O\left(\fr{1}{n}\right)\,,
$$ 
если $N$ имеет отрицательное биномиальное распределение с параметрами 
$(n,p), \,  \alpha=(1-p)/p$.

{\small\frenchspacing
{%\baselineskip=10.8pt
\addcontentsline{toc}{section}{Литература}
\begin{thebibliography}{9}

\bibitem{1pag}
\Au{Пагурова В.\,И.} 
Об асимптотическом распределении
случайно индексированного максимума~// В сб. Статистические методы
оценивания и проверки гипотез.~-- Пермь, 2005. С.~104--113.

\label{end\stat}

\bibitem{2pag}
\Au{Embrechts~P., Kluppelberg~K., Mikosch~T.} 
Modelling of
extremal events for finance and insurance.~--- Berlin--New York:
Springer, 1997.
\end{thebibliography}

} 
}
\end{multicols}