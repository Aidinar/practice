
\def\stat{zeif}


\def\tit{НЕКОТОРЫЕ ОЦЕНКИ ДЛЯ БЛИЗКИХ К ПОГЛОЩАЮЩИМ
МАРКОВСКИХ МОДЕЛЕЙ$^*$}
\def\titkol{Некоторые оценки для близких к поглощающим
марковских моделей}
\def\autkol{А.\,И.~Зейфман, А.\,В.~Чегодаев,  В.\,С.~Шоргин}
\def\aut{А.\,И.~Зейфман$^1$, А.\,В.~Чегодаев$^2$,  В.\,С.~Шоргин$^3$}

\titel{\tit}{\aut}{\autkol}{\titkol}

{\renewcommand{\thefootnote}{\fnsymbol{footnote}}\footnotetext[1]
{Работа выполнена при финансовой
поддержке РФФИ, грант~06-01-00111.}}

\renewcommand{\thefootnote}{\arabic{footnote}}
\footnotetext[1]{Вологодский государственный педагогический
университет; Институт проблем информатики Российской академии  наук;
ВНКЦ ЦЭМИ РАН, a\_zeifman@mail.ru}
\footnotetext[2]{Вологодский государственный педагогический
университет, cheg\_al@mail.ru}
\footnotetext[3]{Институт проблем информатики Российской академии  наук,
vshorgin@ipiran.ru}


\Abst{Изучаются, вообще говоря,
нестационарные системы, описываемые  счетными марковскими цепями с
непрерывным временем, нулевое состояние которых является <<почти
поглощающим>>. Такие системы возникают при описании некоторых
задач теории массового обслуживания. Исследуются предельные
характеристики таких моделей. В качестве примеров рассмотрены
модели, описываемые простыми нестационарными блужданиями.}

\KW{сети массового обслуживания; цепи Маркова с
непрерывным временем; эргодичность; процессы рождения и гибели;
простое случайное блуждание}

      \vskip 24pt plus 9pt minus 6pt

      \thispagestyle{headings}

      \begin{multicols}{2}

      \label{st\stat}

\section{Введение}

Марковские модели в задачах, связанных с сис\-те\-ма\-ми и сетями
массового обслуживания, исследуются и применяются очень давно
(см., например,~\cite{bha, bp}). При этом создание новых методов\linebreak
исследования марковских цепей  и расширение круга решаемых задач
вызвали появление целого ряда новых работ (см., например,~\cite{cn}
и приведенную там библиографию).

Более реалистические модели массового обслу\-живания, описываемые
нестационарными мар\-ковскими цепями, активно изучаются уже более
три\-дца\-ти лет, начиная с заметки~\cite{gm}. Построение предельного
режима и нахождение явных формул для вероятностей состояний таких
моделей, как правило, невозможно, поэтому, естественно, основной
интерес связан с вопросами аппроксимации характеристик таких систем
и получением количественных оценок (см., например,~[5--8]).

В данной статье-заметке исследуется класс марковских цепей с
непрерывным временем, для которых интенсивность выхода из нулевого
состояния в определенном смысле мала.  Такие цепи возникают при
изучении различных классов задач массового обслуживания. Метод
исследования, предложенный в~\cite{z95} (см.\ также~\cite{z08}),
основан на применении: (а)~логарифмической нормы оператора; (б)~некоторых
специальных перенормировок.

\medskip
Пусть $X(t)$,  $t\geq 0$,~--- нестационарная марковская цепь со
счетным прост\-ранст\-вом состояний $E=\{0,1,\ldots \}$. Обозначим
через $p_{ij}(s,t)\;=$\linebreak $=\;P(X(t)=j | X(s)=i)$, $i,j\in E,$ $0\leq s\leq
t$, вероятность перехода из состояния $i$ в состояние~$j$, а
$p_{i}(t)=P(X(t)=i)$, $i\in E$, $t\geq 0$,~--- вероятность
нахождения процесса в состоянии $i$. Пусть ${\bf
p}(t)=(p_{0}(t),p_{1}(t),\ldots )^{T}$~--- вектор вероятностей
состояний, $Q(t)=(q_{ij}(t))$,  $t\geq 0$,~--- соответствующая
матрица интенсивностей. Положим
$A(t)=(a_{ij}(t))=Q^{T}(t)=(q_{ij}(t))^{T}$. Для рассматриваемой в
настоящей статье ситуации  будет предполагаться, что
$|q_{00}(t)|=|a_{00}(t)|$ достаточно мал. Динамика такого процесса
при некоторых дополнительных предположениях описывается прямой
системой Колмогорова:
\begin{equation}
  \fr{d {\bf p}}{dt}=A(t){\bf p}\,,\quad {\bf p}={\bf p}(t)\,,\quad
t\geq 0\,. \label{01}
\end{equation}

Далее будем предполагать,что $A(t)$ локально интегрируема на
$[0,\,\infty)$ и ограничена почти при всех $t \ge 0 $ как
операторная функция  на $l_{1}$. Тогда систему~(\ref{01}) можно
рассматривать как дифференциальное уравнение с
ог\-ра\-ни\-чен\-ным оператором в пространстве $l_{1}$. При этом
оператор Коши $ U(t,s)$, $0{\leq s\leq t}$, уравнения~(\ref{01})
определяется матрицей $
U^{T}(t,s)=P(s,t)=\left(p_{ij}(s,t)\right)$. Далее в тексте, если
не указано противное,  будет использоваться  $l_{1}$-норма для
векторов и матриц $\| \bullet \|$, а именно $ \| x
\|=\sum\limits_{i}|x_{i}|$  и $ \| C \|=\sup\limits_j \sum_{i}|c_{ij}|$, где
$C=(c_{ij})$. Обозначим $\Omega$ множество всех стохастических
векторов: $\Omega= \{ {\bf x}=(x_{0},x_{1},\ldots )^{T} : {\bf x}\geq
0, \|{\bf x}\|=1 \}$.

Рассмотрим ${\bf p}(t)\in \Omega$. Исключим из системы уравнение
для нулевой координаты, полагая

$$
p_{0}(t)=1-\sum_{i\geq 1}
p_{i}(t)\,,
$$ тогда из~(\ref{01}) получается система
\begin{equation}
 \fr{d{\bf z}}{dt}=B(t){\bf z} + {\bf f} (t)\,, \label{02}
\end{equation}
где
\begin{align*}
B(t)&=\left(
\begin{array}{cccc}
a_{11}(t) - a_{10}(t) & a_{12}(t) - a_{10}(t)& \cdots \\
 a_{21}(t) - a_{20}(t) & a_{22}(t) - a_{20}(t) & \cdots
\\ \cdots & \cdots & \cdots
\\ \cdots & \cdots & \cdots
 \end{array}
\right)\,;\\
%\label{03}
{\bf z}&=(p_{1},p_{2}...)^{T}\,; \quad {\bf
f}=(a_{10},a_{20},\ldots )^{T}\,.
\end{align*}

\section{Эргодичность, общий случай}

Рассмотрим вспомогательную последовательность положительных чисел
 $\{d_i\}$ и будем вначале предполагать,  что
\begin{equation}
0 < m = \inf_i d_i; \quad M=\sup_{i,j}\frac{d_i}{d_j} < \infty\,.
\label{0800}
\end{equation}

Положим
\begin{equation*}
\alpha_i^*(t)=\sum_{j \ge 1}
\fr{d_j}{d_i}a_{ji} (t) %\label{n0003}
\end{equation*}
и
\begin{equation*}
\beta_* (t) = \inf_{ i \ge 1}\;-\;\alpha_i^*(t)\,.
%\label{n0004}
\end{equation*}
Пусть
\begin{equation}
|a_{00}(t)| \le \varepsilon \beta_* (t)\,,\quad t \ge 0\,,
\label{n00041}
\end{equation}
причем
\begin{equation}
\int\limits_0^{\infty} \beta_* (t)\,dt = +\infty \label{n00044}\,.
\end{equation}

\medskip

\noindent
{\bf Теорема 1. } {\it Пусть $\{d_i\}$~--- последовательность
положительных чисел такая, что~(\ref{0800})--(\ref{n00044})
выполнены. Пусть $\varepsilon$ достаточно мало.
Тогда $X(t)$ слабо эргодична, причем при всех   ${\bf p}(0)$
справедливы следующие оценки:
\begin{multline}
\sum_{i \ge 1} d_i \left|p_i(t) -
\pi_i(t)\right| \le{}\\
{}\le  e^{-\int\limits_0^t\left(1- M\varepsilon\right)
\beta_* (\tau)\, d\tau} \sum_{i \ge 1} d_i \left|p_i(0) -
\pi_i(0)\right| \le{}\\
{}\le
2M
e^{-\int\limits_0^t\left(1- M\varepsilon\right) \beta_* (\tau)\, d\tau}
\label{n00045}
\end{multline}
и
\begin{equation}
\liminf_{t \to \infty} p_0(t) \ge 1 -
 \fr{M\varepsilon}{m\left(1-M\varepsilon\right)}\,,
\label{n00048}
\end{equation}
где  ${\bf\pi}(t) = \left(\pi_0(t), \pi_1(t), \dots \right)^T$~--- предельное
$($квазистационарное$)$ распределение вероятностей цепи.}

\noindent
Д\,о\,к\,а\,з\,а\,т\,е\,л\,ь\,с\,т\,в\,о.~Рассмотрим вспомогательную диагональную матрицу
\begin{equation*}
 D=\mathrm{diag}\left (d_{1},d_{2},\ldots \right )
%\label{07}
\end{equation*}
и соответствующую обратную матрицу
\begin{equation*}
 D^{-1}=\mathrm{diag}\left ( d^{-1}_{1},d^{-1}_{2},\ldots \right )\,.
%\label{08}
\end{equation*}

Пусть $l_{1D}$~--- пространство последовательностей таких, что
$\|{\bf x}\|_{1D}=\sum\limits_{i=1}^{\infty}d_{i}|x_{i}|<\infty$.

Оценивая логарифмическую норму  $\gamma \left(B(t)\right)$ в
$l_{1D}$, получаем
\begin{multline*}
\gamma \left(B(t)\right) ={}\\
{}= \sup_{i \ge 1} \left(a_{ii} (t)- a_{i0} +
\sum_{j \neq i} \fr{d_j}{d_i}\left|a_{ji} (t)- a_{j0} (t)\right|
\right) \le{}\\
{}\le -\beta_* (t) + M\varepsilon \beta_* (t)\,. 
%\label{n00042}
\end{multline*}

Теперь, если  $V(t,s)$~--- оператор Коши уравнения~(\ref{02}),
получаем
\begin{equation*}
\left\|V(t,s)\right\| \le e^{\int\limits_s^t\gamma \left(B(\tau)\right)\,
d\tau} \le  e^{-\int\limits_s^t\left(1- M\varepsilon\right) \beta_*
(\tau)\, d\tau}\,. 
%\label{n00043}
\end{equation*}

Отсюда вытекает слабая эргодичность  $X(t)$ и неравенство~(\ref{n00045}).

Далее, при любом начальном условии в норме  $l_{1D}$ получаем

\vspace*{-1pt}

\noindent
\begin{multline*}
\left\|{\bf z}(t)\right\| =  \sum_{i \ge 1} d_i \left|p_i(t)\right|
\le \left\|V(t,0)\right\| \left\|{\bf z}(0)\right\| +{}\\[-1pt]
{}+ \int\limits_0^t
\left\|V(t,\tau)\right\| \left\|{\bf f}(\tau)\right\|\, d\tau \le{}\\[-1pt]
{}\le 
 e^{-\int\limits_0^t \left(1- M\varepsilon\right) \beta_* (\tau)\, d\tau}\left\|{\bf z}(0)\right\| +{}\\[-1pt]
 {}+
\int\limits_0^t e^{-\int\limits_\tau^t\left(1- M\varepsilon\right) \beta_* (s)\, d
s}  M \left|a_{00}(\tau)\right| \, d\tau  \le {}\\[-1pt]
{}\le  e^{-\int\limits_0^t  \left(1- M\varepsilon\right) \beta_* (\tau)\, d\tau}\left\|{\bf z}(0)\right\| +
 \fr{M\varepsilon}{1-M\varepsilon}\,,
%\label{n00046}
\end{multline*}
а значит,

\vspace*{-1pt}

\noindent
\begin{equation*}
\limsup_{t \to \infty} \left\|{\bf \pi}(t)\right\|_{1D} \le
 \frac{M\varepsilon}{1-M\varepsilon}\,,
%\label{n00047}
\end{equation*}
откуда следует~(\ref{n00048}).

\bigskip

Перейдем к рассмотрению ситуации, при которой вспомогательная
последовательность не ограничена сверху. Аналогично предыдущей
теореме  устанавливается следующее утверждение.

\medskip

\noindent
{\bf Теорема 2. } {\it Пусть $\{d_i\}$~--- неубывающая
последовательность положительных чисел такая, что выполнены~(\ref{n00041}), 
(\ref{n00044}),  а кроме того, при некотором~$N$ выполняется условие
$q_{0i}(t)=a_{i0}(t)=0$ при всех $i > N, \ t \ge 0$. Пусть
$\varepsilon$ достаточно мало. Тогда $X(t)$ слабо эргодична, причем
при всех ${\bf p}(0)$ справедливы следующие оценки:
\begin{multline*}
\sum_{i \ge 1} d_i \left|p_i(t) - \pi_i(t)\right|  \le{}\\
{}\le
e^{-\int\limits_0^t\left(1- d_N\varepsilon\right) \beta_* (\tau)\, d\tau}
\sum_{i \ge 1} d_i \left|p_i(0) - \pi_i(0)\right| \,, 
%\label{n00045a}
\end{multline*}
\begin{equation*}
\liminf_{t \to \infty} p_0(t) \ge 1 -
 \frac{d_N\varepsilon}{1-d_N\varepsilon}\,,
%\label{n00048a}
\end{equation*}
а кроме того,
\begin{equation*}
\limsup_{t \to \infty} E(t,{\bf p}) \le
 \fr{d_N\omega\varepsilon}{1-d_N\varepsilon }\,,
%\label{n00049}
\end{equation*}
где $\omega = \sup\limits_{k \ge 1} k/d_k$, а $E(t,{\bf
p})$~--- среднее (математическое ожидание) для $X(t)$ при начальном
распределении вероятностей состояний ${\bf p}(0) = {\bf p}$.}

\section{Эргодичность, процесс рождения и гибели}

Пусть теперь  $X(t)$~--- <<почти поглощающий>> процесс рождения и
гибели (ПРГ). В этом случае матрица интенсивностей
\begin{multline*} 
A\left( t\right) ={}\\
{}=\left( 
\begin{array}{ccccc}
-a_0\left( t\right) & b_1\left( t\right) & 0 & \cdots  \\
a_0\left( t\right) & -\left( a_1\left( t\right) +b_1\left( t\right)
\right)
& b_2\left( t\right) & \ddots &  \\
0 & a_1\left( t\right) & \ddots & \ddots  \\
\vdots & \ddots & \ddots & \ddots
\end{array}
\right)  
%\label{201}
\end{multline*}
является якобиевой (трехдиагональной).

Положим
\begin{multline*}
\alpha _k (t) =a_k(t)+b_{k+1}(t)-d_{k+2} d_{k+1}^{-1}
a_{k+1}(t)-{}\\
{}-d_{k}d_{k+1}^{-1} b_k(t)\,, \quad k \ge 1\,, %\label{204}
\end{multline*}
\begin{align*}
\alpha _0 (t) &=b_{1}(t)-d_{2} d_{1}^{-1} a_{1}(t)\\
%\label{205}
\intertext{и}
\beta (t) &= \inf_{i \ge 0} \alpha_i (t)\,. 
%\label{206}
\end{align*}

\vspace*{-2pt}


Рассмотрим теперь вместо~(\ref{n00041}) и~(\ref{n00044}) условия
\begin{equation}
|a_{00}(t)| = a_0(t) \le \varepsilon \beta (t)\,,\quad t \ge 0\,,
\label{210}
\end{equation}
 и
\begin{equation} 
\int\limits_0^{\infty} \beta (t)\, dt = +\infty
\label{207} 
\end{equation}
соответственно.

\medskip

\noindent
{\bf Теорема 3. } {\it Пусть $\{d_i\}$~--- неубывающая
последовательность положительных чисел такая, что $d_1=1$, 
и выполнены условия~(\ref{210}) и~(\ref{207}). Пусть
$\varepsilon$ достаточно мало. Тогда ПРГ $X(t)$ слабо эргодичен,
причем при всех ${\bf p}(0)$ справедливы следующие оценки:


\vspace*{-1pt}

\noindent
\begin{multline}
\sum_{i \ge 1} d_i \left|p_i(t) - \pi_i(t)\right|  \le{}\\[-2pt]
{}\le  4e^{-\int\limits_0^t
\beta (\tau)\, d\tau} \sum_{i \ge 1} g_i \left|p_i(0) -
\pi_i(0)\right| \,, 
\label{n00045b}
\end{multline}

\vspace*{-2pt}

\noindent
\begin{eqnarray}
\liminf_{t \to \infty} p_0(t) \ge 1 - 2\varepsilon\,, 
\label{n00048b}
\end{eqnarray}
а кроме того,

\noindent
\begin{eqnarray}
\limsup_{t \to \infty} E(t,{\bf p}) \le
 2\omega\varepsilon,
\label{n00049b}
\end{eqnarray}

\vspace*{-1pt}

\noindent
где $g_k = \sum\limits_{i=1}^{k} d_i $.}

\smallskip

\noindent
Д\,о\,к\,а\,з\,а\,т\,е\,л\,ь\,с\,т\,в\,о.~Рассмотрим матрицу 
\begin{equation*}
  {\bf T}=\left(
  \begin{array}{ccccc}
  d_1 & d_1 & d_1 & \cdots  \\
  0   & d_2 & d_2 & \cdots  \\
  0   & 0   & d_3 & \cdots  \\
  \vdots & \vdots & \ddots & \ddots
  \end{array}
  \right)\,,
%\label{202}
\end{equation*}
введем $1T$-норму как $\left\| {\bf x} \right\|_{1T} =
\left\| T{\bf x}\right\|$, соответствующее пространство
последовательностей $l_{1T}$ и вычислим логарифмическую норму
$\gamma \left(B(t)\right)_{1T}$. Тогда получаем
\begin{equation*}
\gamma \left(B(t)\right)_{1T} = \gamma \left(TB(t)T^{-1}\right)_{1}\,.
%\label{203}
\end{equation*}

Имеем при любом  $a_0(t)$:
\begin{multline*}
\gamma \left(B(t)\right)_{1T} =\\
{}=
\sup_{ k \ge 0} \left(-a_k(t)-b_{k+1}(t)+d_{k+2} d_{k+1}^{-1}
a_{k+1}(t)\right.+{}\\
{}+\left.d_{k}d_{k+1}^{-1} b_k(t)\right) \le -\beta (t)\,;
%\label{208} 
\end{multline*}
\begin{equation*}
\left\|V(t,s)\right\|_{1T} \le e^{-\int\limits_s^t \beta (\tau)\, d\tau}\,,
%\label{209}
\end{equation*}
\pagebreak

\noindent
откуда с учетом известной оценки~\cite{z06}
\begin{equation}
\left\|{\bf z}(t)\right\|_{1T} \ge \fr{1}{2}\,\sum_{i \ge 1}
d_i\left|p_i(t)\right| 
\label{212}
\end{equation}
получаем слабую эргодичность и неравенство~(\ref{n00045b}).

Далее, при любом начальном условии в норме  $l_{1T}$ получаем
\begin{multline*}
\!\!\!\!\!\left\|{\bf z}(t)\right\| \le \left\|V(t,0)\right\| \left\|{\bf
z}(0)\right\| + \int\limits_0^t
\left\|V(t,\tau)\right\| \left\|{\bf f}(\tau)\right\|\, d\tau \le {} \\
\le  e^{-\int\limits_0^t  \beta (\tau)\, d\tau}\left\|{\bf z}(0)\right\| +
\int\limits_0^t e^{-\int\limits_\tau^t \beta (s)\, d s}  \left|a_{00}(\tau)\right| \, d\tau  \le{} \\
{}\le
 e^{-\int\limits_0^t   \beta (\tau)\, d\tau}\left\|{\bf z}(0)\right\| +
\varepsilon\,,
% \label{212001}
\end{multline*}
а значит,
\begin{eqnarray*}
\limsup_{t \to \infty} \left\|{\bf \pi}(t)\right\|_{1T} \le
 \varepsilon\,,
\label{212002}
\end{eqnarray*}
откуда следует~(\ref{n00048b}). 

Для доказательства~(\ref{n00049b}) 
достаточно воспользоваться еще раз оценкой~(\ref{212}).

\bigskip

\noindent
{\bf Следствие.} При любом $k \ge 1$
\begin{equation*} 
\sum_{i \ge 1} d_i \left|p_{0i}(t) - p_{ki}(t)\right|
\le 2e^{-\int_0^t \beta (\tau)\, d\tau}  \sum_{i=1}^{k} d_i\,.
%\label{n000450b}
\end{equation*}

\bigskip

{\bf Замечание.} Тот же подход позволяет получать и нижние оценки
скорости сходимости (см.~\cite{z07, z95, z08}), а в случае
периодических интенсивностей строить очень важную характеристику
процесса~--- предельное среднее~\cite{z06, z08}.

\section{Нуль-эргодичность}

В двух предыдущих разделах интенсивность выхода из нулевого
состояния могла быть и нулевой. Здесь же рассмотрим другую возможную
ситуацию.

Положим 
\begin{align*} 
\alpha_i^*(t)&=\sum_{j \ge 0}
\fr{d_j}{d_i}a_{ji} (t) \\
%\label{n00031}
\intertext{и}
\beta^* (t) &= \inf_{ i \ge 0} -\alpha_i^*(t)\,.
%\label{n000410} 
\end{align*}

Пусть
\begin{equation}
\int\limits_0^{\infty} \beta^* (t)\, dt = +\infty \,.
\label{n000441}
\end{equation}

\medskip

\noindent
{\bf Теорема 4.} {\it  Пусть существует последовательность
$\{d_{i}\}$ поло\-жи\-тель\-ных чисел
 такая, что $d_0=1, \sup\limits_{i}d_{i}=d<\infty$ и~(\ref{n000441}) выполнено.  Тогда цепь $X(t)$ нуль-эргодична.

Более того, справедливы следующие оценки:\\

\noindent при всех $i, {\bf p}(0)$
\begin{equation}
p_i(t) \leq \fr{d}{{\partial}_i} e^{-\int\limits_0^t \beta^* (\tau)\,
d\tau}\,; 
\label{30001} 
\end{equation}

\noindent при всех $n, {\bf p}(0)$
\begin{equation}
\Pr \left(X(t) \le n \right) \leq \fr{d}{{\partial}_n}
e^{-\int\limits_0^t \beta^* (\tau)\, d\tau}\,; 
\label{30002} 
\end{equation}

\noindent при всех $n, k$
\begin{equation}
\Pr \left(X(t) \le n | X(0) = k \right) \leq
\fr{d_k}{{\partial}_n} e^{-\int\limits_0^t \beta^* (\tau)\, d\tau}\,,
\label{30003}
\end{equation}
где ${\partial}_n = \min\limits_{0 \le k \le n} d_k$.  }

\medskip

\noindent
Д\,о\,к\,а\,з\,а\,т\,е\,л\,ь\,с\,т\,в\,о.~Доказательство проводится почти так же, как в
теореме~1, только в качестве вспомогательного вводится пространство
$l_{1D^*}$  последовательностей таких, что   $\|{\bf
x}\|_{1D^*}=\sum\limits_{i=0}^{\infty}d_{i}|x_{i}|\;<$\linebreak $<\;\infty$, а затем
оценивается логарифмическая норма $\gamma \left(A(t)\right)$ в
$l_{1D^*}$. Из получаемого в итоге неравенства
\begin{equation*}
\left\|{\bf p}(t)\right\|_{1D^*} \le  e^{-\int_0^t  \beta^* (\tau)\,
d\tau}\left\|{\bf p}(0)\right\|_{1D^*} 
%\label{30004}
\end{equation*}
и сравнения норм и вытекают требуемые оценки.

\bigskip

\noindent
{\bf Следствие.} При выполнении условий теоремы~4 для любых ${\bf
p}= {\bf p}(0)$, $t \ge 0$ и любого $n$ справедлива следующая
оценка среднего:
\begin{equation*}
 E(t,{\bf p}) \ge
 \left(n+1\right)\left(1-\fr{d}{{\partial}_n }e^{-\int\limits_0^t  \beta^* (\tau)\,
d\tau} \right)\,. 
%\label{300005}
\end{equation*}

\bigskip
Более точная оценка очень просто получается в случае процесса
рождения и гибели.

Положим
\begin{equation*}
\phi (t) = \inf_{ i \ge 1} \left(a_i(t) - b_i(t)\right)\,.
%\label{30006}
\end{equation*}

Пусть
\begin{equation}
\int\limits_0^{\infty} \phi (t)\, dt = +\infty \,. 
\label{30007}
\end{equation}

\bigskip

\noindent
{\bf Теорема 5.} {\it  Пусть интенсивности ПРГ  $X(t)$ таковы, что~(\ref{30007}) 
выполнено и вдобавок $a_0(t) \ge \varepsilon \phi
(t)$ при некотором $\varepsilon \in (0,\,1)$. Тогда для любых ${\bf
p}= {\bf p}(0)$, $t \ge 0$  справедлива следующая оценка среднего:
\begin{equation}
 E(t,{\bf p}) \ge
 \varepsilon \int\limits_0^t   \phi (\tau)\,
d\tau  +  E(0,{\bf p})\,. 
\label{30008}
\end{equation}

}

Д\,о\,к\,а\,з\,а\,т\,е\,л\,ь\,с\,т\,в\,о получается непосредственно из оценки
\begin{equation*}
\fr{dE}{dt} \ge a_0(t)p_0 + \sum\limits_{i \ge 1} \left(a_i(t) - b_i(t)\right)
p_i \ge  \varepsilon \phi (t)\,. 
%\label{30009}
\end{equation*}

\section{Примеры}

\noindent
{\bf Пример 1. Простое случайное блуждание.} Пусть $X(t)$~---
процесс рождения и гибели с пространством состояний $E$ и
интенсивностями $a_{0}(t) =l(t)$ и $a_{i}(t)= a(t),\ b_{i}(t)=
b(t),\ i\geq 1$. Такой процесс служит для описания простых моделей
теории массового обслуживания (см.\ например,~\cite{t}). В этом
случае возможно применение всех описанных подходов.

 {\bf Эргодический случай.} Пусть существует $s>1$ такое,
что
\begin{equation}
 \int\limits_{0}^{\infty}(b(t)-sa(t)) \, dt=+\infty\,.
\label{e01}
\end{equation}
Положим $d_{k}=s^{k}$ для $k\geq1$. Тогда
\begin{align*}
\alpha _k (t) &=\left(1-s^{-1}\right)\left( b(t)-sa(t)\right)\,, \quad
k \ge 1\,, \\
\alpha _0 (t) &=  b(t)-sa(t) %\label{e03}
\\
\intertext{и}
\beta (t) &= \left(1-s^{-1}\right)\left( b(t)-sa(t)\right)\,,
%\label{e04}
\end{align*}
при этом условие (\ref{207}) выполнено.

Пусть при достаточно малом $\varepsilon$
\begin{equation}
l(t) \le \varepsilon \beta (t),\quad t \ge 0\,. 
\label{e06}
\end{equation}
Тогда ПРГ $X(t)$ слабо эргодичен, и при всех ${\bf p}(0)$ справедливы следующие оценки:
\begin{multline}
\sum_{k \ge 1} s^k \left|p_i(t) - \pi_i(t)\right|  \le {}\\
{}\le 4 e^{-\int\limits_0^t \beta (\tau)\, d\tau} \sum_{i \ge 1} g_i \left|p_i(0) -
\pi_i(0)\right| \,, 
\label{e07}
\end{multline}
\begin{align}
\liminf_{t \to \infty} p_0(t)& \ge 1 - 2\varepsilon\,, 
\label{e08}\\
\intertext{а кроме того,}
\limsup_{t \to \infty} E(t,{\bf p})& \le
 2\omega\varepsilon\,,
\label{e09}
\end{align}
где $g_i = \sum\limits_{k=1}^{i} s^k $, $\omega = \sup_{k \ge 1}
\frac{k}{s^k}$.

\bigskip

Отметим, что в случае {\it стационарного} блуждания условие
эргодичности~(\ref{e01}) означает, что $a < b$. Пусть при этом
вместо~(\ref{e06}) выполнено более прос\-тое неравенство $l \le
\varepsilon$. Тогда вместо~(\ref{e07})--(\ref{e09}) получаем при
всех ${\bf p}(0)$ следующие оценки:
\begin{align*}
\sum_{k \ge 1} s^k \left|p_i(t) - \pi_i\right| & \le 4 e^{- \beta t}
\sum_{i \ge 1} g_i \left|p_i(0) - \pi_i\right| \,; %\label{e071}
\\
\liminf_{t \to \infty} p_0(t) &\ge 1 - \fr{2\varepsilon}{\beta}\,,\\
%\label{e081}
\intertext{а кроме того,}
\limsup_{t \to \infty} E(t,{\bf p}) & \le \fr{
2\omega\varepsilon}{\beta}\,, %\label{e091}
\end{align*}
где $s= \sqrt{b/a}$, $\beta=\left(\sqrt{b} -
\sqrt{a}\right)^2$.

\bigskip

{\bf Нуль-эргодичность.} Пусть существует $s<1$ такое, что
\begin{equation}
\int\limits_{0}^{\infty}\left(sa(t)-b(t)\right)dt=+\infty\,, \label{e11}
\end{equation} 
и, кроме того, при некотором достаточно
малом $\varepsilon > 0$ 
\begin{equation} 
l(t) \ge \varepsilon
\left(sa(t)-b(t)\right)\,. 
\label{e12} 
\end{equation}
Тогда ПРГ нуль-эргодичен и при $d=1$, $d_k=s^k$, ${\partial}_n =
s^n$ и 
$\beta^* (t)= \varepsilon
\left(1-s\right)\left(sa(t)-b(t)\right)$ 
%\label{e13} 
%\end{equation}
выполняются оценки~(\ref{30001})--(\ref{30003}).

Кроме того, справедлива и оценка типа~(\ref{30008}), а именно
\begin{equation*}
 E(t,{\bf p}) \ge
 \varepsilon \int\limits_0^t   \left(sa(\tau)-b(\tau)\right)\,
d\tau  +  E(0,{\bf p})\,. 
%\label{e14}
\end{equation*}

В случае стационарного ПРГ  условие~(\ref{e11}) эквивалентно тому,
что $a > b$, а условие~(\ref{e12})~--- тому, что $l >0$. Выписав для
удобства это условие в том же виде~(\ref{e12}), получаем при
$s=\sqrt{b/a}$, $\beta^* = \varepsilon
\left(1-s\right)\left(sa -b\right)$ соответственно оценки:
\begin{align*}
p_n(t) &\leq s^{-n} e^{- \beta^* t}\,, %\label{e15}
\\
\Pr \left(X(t) \le n \right) &\leq s^{-n} e^{- \beta^* t}\,,
%\label{e16} 
\\
\Pr \left (X(t) \le n | X(0) = k \right)& \leq s^{k-n} e^{- \beta^* t}, %\label{e17}
\\
 E(t,{\bf p}) &\ge  \varepsilon  \left(sa -b\right)t
  +  E(0,{\bf p})\,. 
%\label{e18}
\end{align*}
          
\bigskip
          
\noindent
{\bf Пример 2. Простое случайное блуждание  с разными скачками.}
Рассмотрим систему массового обслуживания, в которой требования в
случае не\-пус\-той очереди поступают парами (с ин\-тен\-сив\-ностью
$a(t)$), а обслуживаются по одному (с ин\-тен\-сив\-ностью $b(t)$). При
этом в случае отсутствия требований в системе возможна
<<иммиграция>>, т.\,е.\ поступление пары  требований с
интенсивностью $l(t)$. В этом случае $X(t)$ уже не является
процессом рождения и гибели.

{\bf Эргодичность.} Пусть существует $s>1$ такое, что
\begin{equation}
 \int\limits_{0}^{\infty}(b(t)-s(s+1)a(t))\, dt=+\infty\,.
\label{e21}
\end{equation}

Положим $d_{k}=s^{k}$ для $k\geq1$. Тогда, как легко проверить,
\begin{equation*}
\beta_* (t) = \left(1-s^{-1}\right)\left(b(t)-s(s+1)a(t)\right)\,,
%\label{e22}
\end{equation*}
далее, $N = 2$ и при
$l(t) \le \varepsilon \beta_* (t) $
%\label{e23}
и достаточно малом~$\varepsilon$ выполнены условия теоремы~2. 
Следовательно, при всех ${\bf p}(0)$ справедливы следующие
оценки:
\begin{multline*}
\sum_{i \ge 1} s^i \left|p_i(t) - \pi_i(t)\right|  \le{}\\
{}\le 
e^{-\int\limits_0^t\left(1- s^2\varepsilon\right) \beta_* (\tau)\, d\tau}
\sum_{i \ge 1} s^i \left|p_i(0) - \pi_i(0)\right|\,, 
%\label{e24}
\end{multline*}

\noindent
\begin{align*}
\liminf_{t \to \infty} p_0(t)& \ge 1 -
 \fr{s^2\varepsilon}{1-s^2\varepsilon}\,,\\
%\label{e25}
\intertext{а кроме того,}
\limsup_{t \to \infty} E(t,{\bf p}) &\le
 \frac{s^2\omega\varepsilon}{1-s^2\varepsilon }\,,
%\label{e26}
\end{align*}
где $\omega = \sup\limits_{k \ge 1} k/s^k$.

\medskip

Отметим, что в случае стационарной ситуации (интенсивности не
зависят от времени) условие~(\ref{e21}) эквивалентно тому, что 
$2a <b$. Несложно получить оценки для этого  случая, выбрав 
 $s =\sqrt[3]{b/(2a)} > 1$.

\bigskip

{\bf Нуль-эргодичность.} Пусть существует $s<1$ такое, что
\begin{equation*}
 \int\limits_{0}^{\infty}\left(s(s+1)a(t) - b(t)\right)\, dt=+\infty
%\label{e27}
\end{equation*} 
и, кроме того, при некотором достаточно
малом $\varepsilon > 0$ 
\begin{equation*} 
l(t) \ge \varepsilon
\left(s(s+1)a(t) - b(t)\right)\,. 
%\label{e28} 
\end{equation*}
Тогда блуждание  нуль-эргодично, и при $d=1$, $d_k\;=$\linebreak $=\;s^k$,
${\partial}_n = s^n$ и
$\beta^* (t)= \varepsilon \left(1-s^2\right)
\left(s(s+1)a(t)\;-{}\right.$\linebreak $\left. -\;b(t)\right)$ 
%\label{e29}
выполняются оценки~(\ref{30001})--(\ref{30003}) и~(\ref{30008}).

             \medskip

Отметим, что в случае стационарной ситуации блуждание будет
нуль-эргодичным при $2a > b$ и $l > 0$.


{\small\frenchspacing
{%\baselineskip=10.8pt
\addcontentsline{toc}{section}{Литература}
\begin{thebibliography}{99}

\bibitem{bha} 
\Au{Баруча-Рид А.\,Т.} 
Элементы теории марковских  процессов и их приложения.~--- М.: Наука, 1969.
511~с.

\bibitem{bp} %2
\Au{Bocharov P.\,P., D'Apice~C., Pechinkin~A.\,V.,
Salerno~S.} 
Queueing theory.~--- Utrecht: VSP, 2004. 446~p.

\bibitem{cn} %3
\Au{Ching~Wai-Ki, Ng~M.~K.} 
Markov chains:
Models, algorithms and applications.~--- New York: Springer, 2006. 205~p.

\bibitem{gm}   %4
\Au{Гнеденко Б.\,В., Макаров~И.\,П.} 
Свойства решений
задачи с потерями в случае периодических интенсивностей~// Дифф.
уравнения, 1971. Т.~7. С.~1696--1698.

\bibitem{gz04}  %5
\Au{Granovsky B.\,L., Zeifman A.\,I.} 
Nonstationary
queues: Estimation of the rate of convergence~// Queueing Systems,
2004. Vol.~46. P.~363--388.

\bibitem{z06} %6
\Au{Zeifman~A., Leorato~S., Orsingher~E., Satin~Ya., Shilova~G.}
 Some universal limits for nonhomogeneous birth and death processes~//
Queueing Systems, 2006. Vol.~52. P.~139--151.

\bibitem{z07}  %7
\Au{Зейфман~А.\,И., Сатин Я.\,А.} 
Средние характеристики марковских систем обслуживания~// Автоматика и
телемеханика, 2007. №\,9. С.~122--133.

\bibitem{mar2} %8
\Au{Margolius~B.} 
Transient and periodic solution to the time-inhomogeneous quasi-birth 
death process~// Queueing Systems, 2007. Vol.~56. P.~183--194.

\bibitem{z95}  %9
\Au{Zeifman A.\,I.} 
Upper and lower bounds on the
rate of convergence for nonhomogeneous birth and death proc\-esses~// 
Stoch. Proc.  Appl., 1995. Vol.~59. P.~157--173.

\bibitem{z08} %10
\Au{Зейфман А.\,И., Бенинг В.\,Е., Соколов~И.\,А.}
Марковские цепи и модели с непрерывным временем.~--- М.: ЭЛЕКС-КМ,
2008. 168~с.

\bibitem{t} %11
\Au{Toyoizumi~H., Kobayashi~Y., Kaiwa~K., Shitozawa~J.}
 Stochastic features of computer viruses: Towards theoretical
analysis and simulation~// The 5th  St.\ Petersburg Workshop on
Simulation.~--- StPB.: SPBU, 2005. P.~695--702.

\end{thebibliography}
\label{end\stat}
} 
}
\end{multicols}